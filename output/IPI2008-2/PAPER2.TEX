

%файл макросов для Латеха.

%\font\aj=msbm10    % ажурный шрифт
\font\got=eufm10   % готический шрифт

%\renewcommand{\labelenumi}{\theenumi)} % метка в enumarate: n)
\newcommand{\pd}{\p(N_k=n)} % дискретная функций распределения
%\newcommand{\fr}[4]{\p(#1_1#2#3_1,\ldots,#1_{#4}#2#3_{#4})}
\newcommand{\eql}[1]{\openup1\jot
\tabskip=0pt plus1fil\halign to \textwidth{
\tabskip=0pt$\hfil\displaystyle##\hfil$\tabskip =0pt plus1fil&
{\rm##}\tabskip=0pt\crcr#1\crcr}} % многострочные формулы с номерами
%\renewcommand{\r}{{\rm I\hspace{-0.7mm}R}} % поле действительных чисел
%\newcommand{\r}{\hbox{\aj R}}     % поле действительных чисел
\newcommand{\nat}{{\mathbb{N}}}     % натуральные числа
\newcommand{\ce}{{\mathbb{C}}}     % константа Эйлера
\renewcommand{\a}{\alpha} % греческий
\newcommand{\be}{\beta}
%%\newcommand{\la}{\lambda} %..........
\newcommand{\La}{\Lambda} %..........
\renewcommand{\t}{\theta} %..........
\newcommand{\T}{\Theta}   %..........
\renewcommand{\d}{\delta} % алфавит
\newcommand{\de}{\Delta}  %..........
\newcommand{\G}{\Gamma}   %..........
\newcommand{\g}{\gamma}   % греч.
\renewcommand{\kp}{\mathop{\rm\ae{}}} % каппа
\newcommand{\ekp}{\mathop{\overline{\rm\ae}{}}}
\newcommand{\om}{\omega}  %..........


\newcommand{\ff}[2]{\left({#1\over#2}\right)} % дробь с круглыми скобками

\newcommand{\eqdf}{\stackrel{def}{=}} % равенство по определению
\newcommand{\cid}{\stackrel{d}{\longrightarrow}} % сходимость по pаспpеделению
\newcommand{\vct}[2]{{#1}_1+\ldots+{#1}_{#2}} % сумма #2 чисел #1
\newcommand{\vc}[2]{{#1}_1,\ldots,{#1}_{#2}}

\newcommand{\li}{\!\le\!}
\newcommand{\seq}[2]{\{#1_{#2}\}_{#2\ge1}} % последовательн. чисел #1 с инд. #2
\newcommand{\ex}[1]{\exp\left\{it#1\right\}}
\newcommand{\sqn}[2]{\{#1_{n,#2}\}_{#2\ge1}}
\newcommand{\n}{{\cal N}}  % каллиграфические буквы
\renewcommand{\k}{{\cal K}}  % --"--
\newcommand{\lev}{{\cal L}}  % --"--
\newcommand{\m}{{\cal M}}  % --"--
\renewcommand{\c}{{\sf C}} % --"--
\renewcommand{\b}{{\cal B}} % --"--
\newcommand{\h}{{\cal H}}  % --"--
\newcommand{\geo}{{\cal G}}  % --"--
\newcommand{\x}{{\cal X}}  % --"--
\renewcommand{\u}{{\cal U}} % --"--
\renewcommand{\I}{{\bf 1}}
\newcommand{\V}{\hbox{\sf V}}

\def\stat{bening}


\def\tit{НЕКОТОРЫЕ СТАТИСТИЧЕСКИЕ ЗАДАЧИ, СВЯЗАННЫЕ С~РАСПРЕДЕЛЕНИЕМ
ЛАПЛАСА$^*$}
\def\titkol{Некоторые статистические задачи, связанные с распределением
Лапласа}
\def\autkol{В.\,Е.~Бенинг, В.\,Ю.~Королёв}
\def\aut{В.\,Е.~Бенинг$^1$, В.\,Ю.~Королёв$^2$}

\titel{\tit}{\aut}{\autkol}{\titkol}

{\renewcommand{\thefootnote}{\fnsymbol{footnote}}\footnotetext[1]
{Работа выполнена при поддержке РФФИ, гранты
08-01-00567, 08-01-00563 и 08-07-00152.}}

\renewcommand{\thefootnote}{\arabic{footnote}}
\footnotetext[1]{Московский
государственный университет им.~М.\,В.~Ломоносова, факультет
вычислительной математики и кибернетики;
Институт проблем
информатики Российской академии наук, bening@yandex.ru}
\footnotetext[2]{Московский государственный
университет им.~М.\,В.~Ломоносова, факультет вычислительной
математики и кибернетики; Институт проблем
информатики Российской академии наук, vkorolev@comtv.ru}

%\vspace*{24pt}

\Abst{В работе развивается подход, предложенный в
статье~\cite{2be}. Обосновывается естественность возникновения
распределения Лапласа в задачах теории вероятностей и
математической статистики. В качестве статистической иллюстрации
рассмотрено  приложение распределения Лапласа к задачам
асимптотической проверки гипотез.}

\KW{распределение Лапласа; проверка
статистических гипотез; функция мощности; асимптотически наиболее
мощные критерии}

      \vskip 24pt plus 9pt minus 6pt

      \thispagestyle{headings}

      \begin{multicols}{2}

      \label{st\stat}

\section{Введение}

Классическое распределение Лапласа с нулевым средним и дисперсией
$\sigma^2$ было введено П.\,С.~Лапласом в 1774~г.~\cite{45be}. С тех
пор, наряду с нормальным, оно стало одной из наиболее активно
использу\-емых симметричных вероятностных моделей. Это распределение
задается характеристической функцией
$$
f(s)=\fr{2}{2+\sigma^2s^2}\,,\quad s\in{\r}^1\,,
$$
или плотностью
$$
\ell(x)=\fr{1}{\sigma\sqrt{2}}\exp\Big\{-\fr{\sqrt{2}|x|}{\sigma}\Big\}\,,\quad
\sigma>0\,,\ x\in{\r}^1\,.
$$
Популярность распределения
Лапласа как математической (вероятностной) модели обусловлена тем,
что его хвосты тяжелее, чем у нормального рас\-пределения (см.,
например, книги~[3--5], где описывается роль
распределения Лапласа в методах\linebreak робастного оценивания; работу~\cite{30be},
где обосновывается целесообразность использования
распределения Лапласа как модели распределения\linebreak
 погрешностей
измерений в энергетике; статью~\cite{37be}, посвященную применению
распределения Лапласа для моделирования ошибок в навигации; работу~\cite{48be},
в которой распределение Лапласа применяется в метеорологии;
статьи~\cite{21be, 28be}, посвященные применению распределения Лапласа в
управлении запасами и радиоэлектронике). Во многих работах описано
успешное применение распределения Лап\-ла\-са для моделирования
распределения приращений логарифмов финансовых
индексов~[11--13].
В работах~[14--17] распределение Лапласа
используется как модель логарифма доходов фирм и индивидуумов.
Многие работы посвящены применению распределения Лапласа для
моделирования распределения логарифма размера частиц
при дроб\-ле\-нии~[18--20].
Наконец, распределение Лапласа\linebreak применяется при
моделировании статистических закономерностей поведения некоторых
характе\-ристик атмосферной~\cite{24be} и плазменной~\cite{11be} турбулентности.
В~\cite{39be, 43be} можно найти дальнейшие ссылки на работы, в
которых описывается применение распределения Лапласа к решению
прикладных задач в самых разнообразных областях.

%\vspace*{-6pt}

\section{Распределение Лапласа как~симметризация}

%\vspace*{-6pt}

\subsection{Распределение Лапласа как~сверточная симметризация}

Пусть $X$ и $X'$~--- независимые случайные величины с одной и той
же функцией распределения $F(x)$. Характеристическую функцию,
соответствующую функции распределения $F(x)$, обозначим $f(s)$.
Тогда характеристическая функция случайной величины $X^{(s)}\equiv
X-X'$ имеет вид
\begin{multline*}
{\sf E}\exp\{is X^{(s)}\}={\sf E}\exp\{is( X- X')\}={}\\
{}=f(s) f(-s)=f(s) \overline{f(s)}=|f(s)|^2\,, \quad
s\in{\r}^1
\end{multline*}
\noindent
(черта сверху означает комплексное сопряжение).
\pagebreak

%\noindent
Последняя характеристическая функция вещественна, следовательно,
распределение, ей соответствующее, является симметричным в том
смысле, что для любого $x>0$
$$
{\sf P}( X^{(s)}<-x)={\sf P}( X^{(s)}>x)\,.
$$


Распределение, соответствующее характеристической функции
$|f(s)|^2$, называется {\it сверточной симметризацией}
распределения $F(x)$, а случайная величина $ X^{(s)}$,
соответственно, называется {\it сверточной симметризацией}
случайной величины $X$.

Рассмотрим случайную величину $X$ со стандартным показательным
распределением
\begin{equation}
{\sf P}( X<x)=(1-e^{-x})\I(x\ge0)\equiv E(x)\,.\label{2.1b}
\end{equation}
Здесь символом $\I(A)$ обозначается
индикаторная функция множества $A$. Как известно, функции
распределения $E(x)$ соответствует характеристическая функция
$$
f(s)=\fr{1}{1-is}\,,\quad s\in{\r}^1\,.
$$
Тогда в соответствии с приведенным выше определением сверточной
симметризации функции распределения $E(x)$ соответствует
характеристическая функция
$$
|f(s)|^2=\fr{1}{1-is} \fr{1}{1+is}=\fr{1}{1+s^2}\,,
$$
что совпадает с характеристической функцией распределения Лапласа с
$\sigma^2=2$. Таким образом, распределение Лапласа является
сверточной симметризацией экспоненциального (показательного)
распределения.

%\vspace*{-6pt}

\subsection{Распределение Лапласа как~рандомизационная симметризация}

Рассмотрим произвольную случайную величину $X$ и обозначим ее
функцию распределения $F(x)$. Назовем {\it рандомизационной
симметризацией} случайной величины $X$ случайную величину
$\widetilde X$ такую, что
$$
\widetilde X=
\begin{cases}
X &  \mbox{с\ вероятностью}\ \  0{,}5\,,\\
-X &  \mbox{с\ вероятностью}\ \  0{,}5\,.
\end{cases}
$$
Очевидно, что
\begin{multline*}
 {\sf P}(\widetilde X<x)=
\fr{1}{2}\,{\sf P}(X<x)+ \fr{1}{2}\,{\sf P}(-X<x)={}\\
{}=\fr{1}{2}\left [ F(x)+1-F(-x+0)\right ]={}\\
{}=\fr{1}{2}+
\fr{1}{2}\left [ F(x)-F(-x+0)\right ]\,.
\end{multline*}
Если при этом случайная величина
$X$ абсолютно непрерывна, то, обозначив плотности случайных
величин $X$ и $\widetilde X$ соответственно $p(x)$ и $q(x)$, из
последнего равенства получим соотношение
$$
q(x)=\fr{1}{2}\left [ p(x)+p(-x)\right ]\,.
$$
Более того, если
${\sf P}(X\ge0)=1$, то
$$
{\sf P}(\widetilde X<x)=
\begin{cases}
\fr{1}{2}\left [ 1+F(x)\right ]\,, & \mbox{если}\ \  x>0\,,\\
\fr{1}{2}\left [ 1-F(-x+0)\right ]\,, &  \mbox{если}\ \  x\le 0\,,
\end{cases}
$$
и
$$
q(x)=\fr{1}{2}p(|x|)\,.
$$

По аналогии функцию распределения $\widetilde F(x)\;=$\linebreak $=\;{\sf P}(\widetilde
X<x)$ будем называть {\it рандомизационной симметризацией} функции
распределения $F(x)$.

Рандомизационную симметризацию $\widetilde X$ случайной величины
$X$ удобно интерпретировать сле\-ду\-ющим образом. Пусть $Z$~---
независимая от $X$ случайная величина такая, что 
$$
{\sf
P}(Z=1)=1-{\sf P}(Z=-1)=\fr{1}{2}\,.
$$ 
Тогда 
$$ 
\widetilde X\ =\ X
Z\,. 
$$ 
Если $f(s)$~--- характеристическая функция случайной величины
$X$, то
\begin{multline*}
{\sf E}\exp\big\{is\widetilde X\big\}=
\fr{1}{2}f(s)+ \fr{1}{2}f(-s)= {}\\
{}=\fr{1}{2}\left[
{\sf E}\cos sX+i{\sf E}\sin sX+{\sf
E}\cos sX- i{\sf E}\sin sX\right ]={}\\
{}=
{\sf E}\cos sX={\bf Re}f(s)\,.
\end{multline*}
Рассмотрим теперь характеристическую функцию стандартного
экспоненциального распределения, для которой справедливо
представление
$$
f(s)=\fr{1}{1-is}=\fr{1+is}{(1-is)(1+is)}=\fr{1+is}{1+s^2}\,.
$$
Стало быть,
$$
{\bf Re}f(s)=\fr{1}{1+s^2}\,.
$$
Но последняя функция есть не что иное как характеристическая функция
распределения Лапласа с дисперсией $\sigma^2=2$. Поэтому, так как в
данном случае ${\bf Re}f(s)=|f(s)|^2$, рандомизационная
сим\-мет\-ри\-за\-ция показательного распределения совпадает с его
сверточной симметризацией, которая, как уже было показано,
представляет собой распределение Лапласа.

\section{Распределение Лапласа как~смесь}

В этом разделе будет показано, что распределение Лапласа допускает
несколько представлений в виде <<масштабной>> смеси некоторых
хорошо известных распределений вероятностей. Стандартную
нормальную функцию распределения будем обозначать $\Phi(x)$,
$$
\Phi(x)=\fr{1}{\sqrt{2\pi}}\int\limits_{-\infty}^{x}e^{-z^2/2}\,dz\,,\quad
 x\in{\r}^1\,.
$$

\medskip

\noindent
\textbf{Лемма 3.1}. {\it Распределение Лапласа является масштабной
смесью нормальных законов с нулевым средним при экспоненциальном
смешивающем распределении}:
$$
\Lambda(x)=\int\limits_{0}^{\infty}\Phi\bigg(\fr{x}{\sqrt{z}}\bigg)\,dE(z)\,,\quad
 x\in{\r}^1\,.
$$

\smallskip

\noindent
Д\,о\,к\,а\,з\,а\,т\,е\,л\,ь\,с\,т\,в\,о.~Имеем
\begin{multline*}
\int\limits_{0}^{\infty}\Phi\left(\fr{x}{\sqrt{z}}\right)\,dE(z)={}\\
{}=\int\limits_{0}^
{\infty}\left[ \fr{1}{2}+
\fr{1}{\sqrt{2\pi}}\int\limits_{0}^{x/\sqrt{z}}e^{-u^2/2}\,du\right ]
e^{-z}\,dz={}\\
{}=\fr{1}{2}+\fr{1}{\sqrt{2\pi}}\int\limits_{0}^{\infty}
\int\limits_{0}^{x/\sqrt{z}}\exp\left\{
-\fr{u^2}{2}-z\right \}\,du\,dz={}\\
{}=
\fr{1}{2}+\fr{1}{\sqrt{2\pi}}
\int\limits_{0}^{\infty}\int\limits_{0}^{x^2/u^2}e^{-z}\,dz\,
e^{-u^2/2}\,du={}\\
{}
=\fr{1}{2}+\fr{1}{\sqrt{2\pi}}\int\limits_{0}^{\infty}\left(
1-\exp\left \{-\fr{x^2}{u^2}\right \}\right )e^{-u^2/2}\,du={}\\
{}=
1-\fr{1}{\sqrt{2\pi}}\int\limits_{0}^{\infty}
\exp\left\{-\fr{u^2}{2}-\fr{x^2}{u^2}\right \}\,du={}\\
{}=
\begin{cases}
\fr{1}{2}\,e^{\sqrt{2}x}\,,&  x\le 0\,,\\
1-\fr{1}{2}\,e^{-\sqrt{2}x}\,, & x>0
\end{cases}
\end{multline*}
(см., например,~\cite{6be}, формула~3.335). Но в
правой час\-ти стоит не что иное как функция распределения Лапласа с
параметром $\sigma = 1$, которому соответствует плотность
$$
\ell(x)=\fr{1}{\sqrt{2}}\,\exp\{-\sqrt{2}|x|\}\,,\quad x\in{\r}^1\,.
$$

Лемма доказана.

\smallskip

Из леммы~3.1, безграничной делимости показательного закона и хорошо
известного утверж\-дения о том, что масштабные смеси нормальных\linebreak
законов безгранично делимы, если безгранично делимы смешивающие
распределения (см., например,~[26]), вытекает
безграничная делимость распределения Лапласа.

Пусть $W$~--- случайная величина, имеющая распределение Лапласа,
$X$ и $U$~--- случайные величины, имеющие соответственно
стандартное нормальное и стандартное показательное распределения.
Из леммы~3.1 вытекает следующая факторизация случайной величины $W$:
\begin{equation}
W\eqd X \sqrt{U}\,,\label{3.1b}
\end{equation}
где случайные величины $X$ и $U$ независимы.

Пусть $G(x)$~--- функция распределения максимума стандартного
винеровского процесса на единичном отрезке:
$$
G(x)=2\Phi\left (\max\{0,x\}\right )-1\,,\quad x\in{\r}^1\,.
$$ Несложно
видеть, что $G(x)={\sf P}(|X|<x)$, где $X$~--- случайная величина,
имеющая стандартное нормальное распределение.

Легко убедиться, что рандомизационная сим\-мет\-ри\-за\-ция функции
распределения $G(x)$ совпадает со стандартным нормальным
распределением:
\begin{equation}
\widetilde G(x)=\Phi(x)\,.\label{3.2b}
\end{equation}
Пусть $Z$~--- случайная величина, принимающая каж\-дое из значений
$1$ и $-1$ с вероятностью 1/2. Тогда в терминах случайных
величин $X$ и $Z$ соотношение~(\ref{3.2b}) принимает вид
\begin{equation}
X\eqd |X| Z\,,\label{3.3b}
\end{equation}
где случайные величины в правой части незави\-симы.

Далее, пусть $X$ и $U$~--- независимые случайные величины, имеющие
соответственно стандартное нормальное и стандартное показательное
распределения. Тогда для $x>0$
$$
{\sf P}\left (|X|\sqrt{U}<x\right )={\sf E}G\left (
\fr{x}{\sqrt{U}}\right )=
2{\sf E}\Phi\left (\fr{x}{\sqrt{U}}\right )-1={}
$$
$$
{}=2\left (1-\fr{1}{2}\,e^{-\sqrt{2}x}\right )-1=
1-e^{-\sqrt{2}x}={\sf P}(U<\sqrt{2}x)\,.
$$ %\end{multline*}
Отсюда вытекает следующая <<рекуррентная>> факторизация случайной
величины $U$ со стандартным показательным распределением:
\begin{equation}
U\eqd\sqrt{2U} |X|\,.\label{3.4b}
\end{equation}
Из соотношений~(\ref{3.1b})--(\ref{3.4b}) получается представление
\begin{equation}
W\eqd |X| Z \sqrt{U}\,.\label{3.5b}
\end{equation}
Применяя в представлении~(\ref{3.5b})
рекурсию~(\ref{3.4b}), окончательно получаем следующее утверждение.

\medskip

\noindent
{\bf Теорема 3.1}. {\it Справедливы представления
$$
W\eqd Z\sqrt{U} |X|
$$
и для любого натурального $k\ge2$
$$
W\eqd 2^{(2^{k-1}-1)/2^k} Z U^{1/2^k}
\prod_{m=1}^{k}|X_m|^{1/2^{m-1}}\,,
$$
где случайные величины
$X,X_1,X_2,\ldots$ имеют стандартное нормальное распределение,
причем все случайные величины в правых частях независимы}.

\smallskip

Устремляя $k$ к бесконечности, можно получить следующее утверждение.

\medskip

\noindent
{\bf Следствие~3.1}. {\it Справедливо представление
$$
W\eqd \sqrt{2}\cdot Z \prod_{m=1}^{\infty}|X_m|^{1/2^{m-1}}\,,
$$
где случайные величины $X_1,X_2,\ldots$ имеют стандартное нормальное
распределение, причем все случайные величины в правой части
независимы}.

\smallskip

Второе утверждение теоремы~3.1 и утверждение следствия~3.1 могут
быть записаны в эквивалентной форме: для любого натурального
$k\ge2$
\begin{multline*}
W\eqd Z \exp\bigg\{\fr{(2^{k-1}-1)\ln 2}{2^k}+{}\\
{}+
\fr{\ln U}{2^k}+\sum_{m=1}^{k}\fr{\ln|X_m|}{2^{m-1}}\bigg\}
\end{multline*}
и
$$
W\eqd Z \exp\bigg\{\fr{\ln
2}{2}+\sum_{m=1}^{\infty}\fr{\ln|X_m|}{2^{m-1}}\bigg\}\,.
$$

Наряду с представлением распределения Лапласа в виде масштабной
смеси нормальных законов, о котором говорится в лемме~3.1, из
представления~(\ref{3.5b}) вытекает еще одна возможность пред\-став\-ле\-ния
распределения Лапласа в виде масштабной смеси. А именно: при
$x\ge0$ справедливо очевидное соотношение
$$
{\sf P}(\sqrt{U}<x)={\sf P}(U<x^2)=E(x^2)=1-e^{-x^2}\,,
$$
откуда вытекает, что случайная величина $\sqrt{U}$ имеет распределение
Рэлея--Райса с плотностью
$$ r(x)=2xe^{-x^2}\,,\quad x\ge0\,.
$$
Но тогда распределение произведения $Z\sqrt{U}$, являющееся
рандомизационной симметризацией распределения Рэлея--Райса, имеет
плотность
$$
p(x)=|x|e^{-x^2}\,,\quad x\in{\r}^1\,.
$$
Этой плотности соответствует функция распределения
$$
P(x)=
\begin{cases}
1-\fr{1}{2}\,e^{-x^2}\,, &  \mbox{если}\ \ x>0\,,\\
\fr{1}{2}\,e^{-x^2}\,, &  \mbox{если}\ \ x\le 0\,.
\end{cases}
$$
Далее, очевидно, $|X|=\sqrt{X^2}$, причем
случайная величина $X^2$ имеет распределение хи-квадрат с одной
степенью свободы, поскольку $X$~--- случайная величина со
стандартным нормальным распределением. Как известно, плотность
распределения хи-квадрат с одной степенью свободы имеет вид
$$
g(x)=\fr{e^{-x/2}}{\sqrt{2\pi x}}\,,\quad x\ge0\,.
$$
Таким образом, из представления~(\ref{3.5b}) получаем следующее утверждение.

\medskip

\noindent
{\bf Следствие~3.2}. {\it Распределение Лапласа является
масштабной смесью симметризованного распределения Рэлея--Райса,
если смешивающее распределение является распределением хи-квадрат
с одной сте\-пенью свободы}:
$$
\Lambda(x)=\int\limits_{0}^{\infty}P\left(\fr{x}{\sqrt{z}}\right)
\fr{e^{-z/2}}{\sqrt{2\pi z}}\,dz\,,\quad x\in{\r}^1\,.
$$

\smallskip

Пусть $\xi,\xi_1,\xi_2,\ldots$~--- случайные величины, имеющие
распределение хи-квадрат с одной степенью свободы каждая, а $R$~---
случайная величина, име\-ющая распределение Рэлея--Райса. Тогда
следующая факторизация является переформулировкой следствия~3.1:
$$
W\eqd Z R \sqrt{\xi}\,.
$$

Более того, следующее утверждение, обоб\-ща\-ющее следствие~3.2,
является простой переформулировкой второго утверждения теоремы~3.1.

\medskip

\noindent
{\bf Следствие 3.3}. {\it Для любого натурального $k\ge2$
справедливо представление
$$
W\eqd 2^{(2^{k-1}-1)/2^k} Z
R^{1/2^{k-1}} \prod_{m=1}^{k}\xi_m^{1/2^m}\,,
$$
где все случайные величины в правой части не\-за\-ви\-симы}.

\smallskip

Устремляя в следствии~3.3 $k$ к бесконечности, замечаем, что имеет место
следующее утверждение (по сути являющееся переформулировкой
следствия~3.1).

\medskip

\noindent
{\bf Следствие~3.4}. {\it Справедливо представление
$$
W\eqd \sqrt{2}\cdot Z \prod_{m=1}^{\infty}\xi_m^{1/2^m}\,,
$$
где все случайные величины в правой части не\-за\-ви\-симы}.

\smallskip

Утверждения следствий~3.3 и~3.4 могут быть записаны в
эквивалентной форме: для любого натурального $k\ge 2$
$$
W\eqd Z \exp\left\{\fr{(2^{k-1\!}-\!1)\ln 2}{2^k}+ \fr{\ln
R}{2^{k-1}}+\sum_{m=1}^{k}\fr{\ln\xi_m}{2^m}\right \}
$$
и
$$
W\eqd Z \exp\left\{\fr{\ln
2}{2}+\sum_{m=1}^{\infty}\fr{\ln\xi_m}{2^m}\right\}\,.
$$

\noindent
{\bf Замечание~3.1.} Пусть $\zeta_{\gamma}$~--- случайная величина,
имеющая распределение Вейбулла с параметром~$\gamma$. Как
известно, плотность такого распределения имеет вид
$$
h(x)=\gamma x^{\gamma-1}e^{-x^{\gamma}}\mathbb{I}(x\ge0)\,.
$$
Несложно видеть, что $Y^{1/2^{k-1}}\eqd \zeta_{2^{k-1}}$ (при этом распределение
Рэлея является распределением Вейбулла с $\gamma=2$). Таким
образом, следствие~3 может быть переформулировано в терминах
распределения Вейбулла.


Теорема~3.1 указывает алгоритм построения других представлений
распределения Лапласа в виде масштабной смеси вероятностных
законов.

\section{Распределение Лапласа как~асимптотическая аппроксимация}

\subsection{Распределение Лапласа как~асимптотическая аппроксимация
в~схеме суммирования случайных величин}

Пусть $X_1,X_2,\ldots$~--- одинаково распределенные случайные
величины с конечным математическим ожиданием ${\sf E}X_1=a$ и
конечной дисперсией ${\sf D}X_1=\sigma^2>0$. Для произвольного
натурального $n$ обозначим
$$
S_n=X_1+\ldots+X_n\,.
$$
Пусть $\{N_p\}_{0<p<1}$~--- семейство натуральнозначных случайных
величин, причем при каждом $p\in(0,\,1)$ случайные величины
$N_p,X_1,X_2,\ldots$ независимы в совокупности. Предположим, что
случайная величина $N_p$ имеет геометрическое распределение с
параметром $p$:
\begin{equation}
{\sf P}(N_p=k)=p(1-p)^{k-1}\,,\quad k=1,2,\ldots\label{4.1b}
\end{equation}
Наша цель~--- описать асимптотическое
поведение случайных величин
$$
S_{N_p}=X_1+\ldots+X_{N_p}\quad (p\to 0)\,.
$$
Суммы $S_{N_p}$, называемые геометрическими случайными
суммами, играют важную роль в теории надежности (см., например,~[27]). 
Здесь будут приведены версии
<<центральной предельной теоремы>> для геометрических случайных сумм.

Хорошо известно, что
$$
{\sf E}S_{N_p}={\sf E}N_p{\sf E}X_1\,,\quad
{\sf D}S_{N_p}={\sf E}N_p{\sf D}X_1+{\sf D}N_p({\sf E}X_1)^2\,.
$$
Поэтому с учетом того, что
$$
{\sf E}N_p=\fr{1}{p}\,,\quad
{\sf D}N_p=\fr{1}{p^2}\,,
$$
получаем
$$
{\sf E}S_{N_p}=\fr{a}{p}\,,\quad
{\sf D}S_{N_p}=\fr{\sigma^2}{p}+\fr{a^2}{p^2}\,.
$$
Центральными предельными теоремами принято называть утверждения,
описывающие асимптотическое поведение распределений
стандартизованных сумм случайных величин с конечными дис\-пер\-сиями,
т.\,е.\ поведение распределений сумм\linebreak случайных величин,
центрированных своими математическими ожиданиями и нормированных
своими среднеквадратичными отклонениями.

Оказывается, что асимптотическое поведение распределений
стандартизованных геометрических случайных сумм $\widetilde
S_{N_p}$, во-первых, существенно отличается от асимптотического
поведения распределений стандартизованных сумм, описываемого
классической предельной теоремой, и, во-вторых, существенно зависит
от того, равно нулю или нет математическое ожидание $a$ отдельного
слагаемого.

В рассматриваемом случае
$$
\widetilde S_{N_p}\equiv\fr{S_{N_p}-{\sf E}S_{N_p}}{\sqrt{{\sf
D}S_{N_p}}}=\fr{S_{N_p}-a/p}{(1/p)\sqrt{p\sigma^2+a^2}}\,,
$$
и, если $a=0$, то
$$
\widetilde S_{N_p}=\fr{\sqrt{p}}{\sigma}S_{N_p}\,,
$$
а если $a\neq0$, то
\begin{equation}
\widetilde S_{N_p}=
\fr{a}{|a|}\,\fr{1}{\sqrt{1+p\sigma^2/a^2}}\left(
\fr{p}{a}S_{N_p}-1\right)\,.
\label{4.2b}
\end{equation}

Символом $\Lambda(x)$ будет обозначаться функция распределения
Лапласа с единичной дисперсией, соответствующая плотности
$$
\ell(x)=\fr{1}{\sqrt{2}}e^{-\sqrt{2}|x|}\,,\quad x\in{\r}^1\,.
$$

Стандартную показательную функцию распределения, как и ранее,
будем обозначать $E(x)$ (см.~(\ref{2.1b})).

\medskip

\noindent
{\bf Теорема~4.1}. ($i$) {\it Если $a=0$, то}
$$
\sup_x\big|{\sf P}\big(\widetilde
S_{N_p}<x\big)-\Lambda(x)\big|=o(1)\quad (p\to0)\,.
$$
($ii$) {\it Если $a\neq 0$, то}
$$
\sup_x\big|{\sf P}\big(\widetilde
S_{N_p}<x\big)-E(x\,\mathrm{sign}\,a+1)\big|=o(1)\quad
(p\to0)\,.
$$

\smallskip

\noindent
Д\,о\,к\,а\,з\,а\,т\,е\,л\,ь\,с\,т\,в\,о.~Пункт ($i$) представляет собой
центральную предельную теорему для гео\-мет\-ри\-че\-ских случайных сумм с
нулевым средним, доказательство которой можно найти, к примеру, в~\cite{14be},
гл.~6. В силу представления~(\ref{4.2b}) пункт~($ii$) является
следствием из теоремы Реньи~--- хорошо известной версии закона
больших чисел для геометрических случайных сумм, согласно которой
$$
\sup_x\Big|{\sf P}\Big(\fr{p}{a}S_{N_p}<x\Big)-E(x)\Big|=o(1)
$$
(см., например, [29] или~[30]).

\smallskip

Рассмотрим теперь случайную величину
$$
R_p=\sum_{j=1}^{N^{(1)}_p}X^{(1)}_j-\sum_{j=1}^{N^{(2)}_p}X^{(2)}_j\,,
$$
где $N^{(1)}_p$ и $N^{(2)}_p$~--- случайные величины с одинаковым
геометрическим распределением~(\ref{4.1b}), а случайные величины
$\{X^{(i)}_j\}$, $j\ge1$, $i=1,\,2$, имеют одинаковое распределение
с конечной дисперсией. Предположим, что случайные величины
$N^{(i)}_p$, $\{X^{(i)}_j\}$, $j\ge1$, $i=1,\,2$, независимы в
совокупности.

Величины вида $R_p$ являются математическими моделями значений
процесса риска со случайными премиями (см., например,~\cite{12be}, гл.~10)
и играют важную роль при исследовании процессов спекулятивной
деятельности~\cite{1be}. Очевидно, что ${\sf E}R_p=0$.

Из теоремы~4.1 при этом вытекает следующий результат.

\medskip

\noindent
{\bf Следствие~4.1}. {\it Если $a={\sf E}X^{(i)}_j\neq 0$, то}
$$
\sup_x\Big|{\sf P}\Big(\fr{R_p}{\sqrt{{\sf
D}R_p}}<x\Big)-\Lambda(x)\Big|=o(1)\quad (p\to0)\,.
$$

\smallskip

Этот результат является прямым следствием пункта~($ii$) теоремы~4.1,
поскольку распределение Лапласа является сверточной симметризацией
показательного распределения (см.\ параграф~2.1).

\subsection{Геометрическая устойчивость распределения Лапласа}

В предельных теоремах для геометрических случайных сумм
распределение Лапласа играет ту же роль, что нормальное
распределение. При этом аналогия отнюдь не ограничивается тем
свойством, которое описано в предыдущем параграфе и согласно
которому распределение Лапласа оказывается предельным для
геометрических сумм независимых одинаково распределенных случайных
величин с конечными дисперсиями.

В классической теории суммирования независимых случайных величин
{\it строго устойчивыми} называются те распределения, которые
обладают свойством сохранения типа при свертках. А именно: пусть
$X,X_1,X_2,\ldots$~--- случайные величины с одной и той же функцией
распределения $F(x)$, причем случайные величины $X_1,X_2,\ldots$
независимы. Распределение $F(x)$ называется {\it строго
устойчивым}, если для любого натурального $n$ существуют
постоянные $b_n$ такие, что
$$
X_1+\ldots+X_n\eqd b_n X\,.
$$
Хорошо известно, что в этом определении равенство возможно, только если
константы $b_n$ имеют специальный вид, а именно:
$$
b_n=n^{1/\alpha}\,,
$$
где $0<\alpha\le 2$. Нормальное распределение
является строго устойчивым с $\alpha=2$ и единственным строго
устойчивым законом с конечной дисперсией (см., например,~\cite{18be}).

В теории геометрического суммирования существует аналогичное
определение. А именно: пусть $X,X_1,X_2,\ldots$~--- случайные
величины с одной и той же функцией распределения $F(x)$ и пусть
$\{N_p\}_{0<p<1}$~--- семейство случайных величин, имеющих
геометрическое распределение~(\ref{4.1b}), причем при каждом $p\in(0,\,1)$
случайные величины $N_p,X_1,X_2,\ldots$ независимы. Распределение
$F(x)$ называется {\it геометрически строго устойчивым}, если
существует $\alpha>0$ такое, что для любого $p\in(0,\,1)$ выполнено
соотношение
$$
X_1+\ldots+X_{N_p}\eqd p^{-1/\alpha}X\,.
$$
Можно показать, что при этом $\alpha\le2$ (см., например,~\cite{8be}
или~\cite{34be}).

Несложно убедиться, что распределение Лапласа является
геометрически строго устойчивым с $\alpha=2$. Действительно, пусть
$F(x)$~--- функция распределения Лапласа с дисперсией $\sigma^2=2$.
Ей соответствует характеристическая функция
$$
f(s)=\fr{1}{1+s^2}\,,\quad s\in {\r}^1\,.
$$
Тогда
характеристическая функция нормированной геометрической случайной
суммы $\sqrt{p}\cdot S_{N_p}$ имеет вид
$$ %\begin{multline*}
{\sf E}\exp\{is\sqrt{p}\cdot S_{N_p}\}=
\sum_{k=1}^{\infty}p(1-p)^{k-1}f^k(s\sqrt{p})={}
$$
$$
{}=
\fr{p}{1-p}\sum_{k=1}^{\infty}\Big(\fr{1-p}{1+ps^2}\Big)^k= {}
$$
$$
{} =\fr{p}{1-p}\,\fr{1-p}{1+ps^2}\,\fr{1+ps^2}{1+ps^2-1+p}=
\fr{1}{1+s^2}=f(s)\,.
$$ %\end{multline*}

\vspace*{6pt}

\subsection{Распределение Лапласа как~асимптотическая аппроксимация
для~распределений регулярных статистик, построенных по выборкам
случайного объема}

\vspace*{3pt}

Рассмотpим традиционную для математической статистики постановку
задачи. Пусть случайные величины $N_1,N_2,\ldots,X_1,X_2,\ldots$
опpеделены на одном и том же измеpимом пpостpанстве $(\Omega,{\cal
A})$. Пусть на ${\cal A}$ задано семейство веpоятностных меp
$\{{\sf P}_{\theta},\, \theta\;\in{}$\linebreak $\in\;\Theta\}$. Пpедположим, что пpи
каждом $n\ge 1$ случайная величина $N_n$ пpинимает только
натуpальные значения и независима от последовательности
$X_1,X_2,\ldots$ относительно каждой из семейства меp $\{{\sf
P}_{\theta},\ \theta\in\Theta\}$. Пусть $T_n=T_n(X_1,\ldots,X_n)$~---
некотоpая статистика, т.\,е.\ измеримая функция от случайных
величин $X_1,\ldots,X_n$. Для каждого $n\ge1$ опpеделим случайную
величину $T_{N_n}$, положив
$T_{N_n}(\omega)= T_{N_n(\omega)}\left(X_1(\omega),\ldots,X_{N_n(\omega)}(\omega)\right)$
для каждого элементаpного исхода $\omega\in\Omega$. Будем
говоpить, что статистика $T_n$ асимптотически ноpмальна, если
существуют функции $\delta(\theta)$ и $t(\theta)$ такие, что пpи
каждом $\theta\in\Theta$
\begin{multline}
{\sf P}_{\theta}\left(\delta(\theta)\sqrt{n}\bigl(T_n- t(\theta)\bigr)<
x\right)\Longrightarrow\\
{}\Longrightarrow\Phi(x) \quad
(n\to\infty)\,.\label{4.3}
\end{multline}
Примеры асимптотически нормальных
статистик хорошо известны. Свойством асимптотической нормальности
обладают, например, выборочное среднее (при условии существования
дисперсий), центральные порядковые статистики или оценки
максимального правдоподобия (при достаточно общих условиях
регулярности) и многие другие статистики.

Дальнейшие рассуждения будут основаны на следующей лемме.

\medskip

\noindent
{\bf Лемма~4.1.} {\it Пусть $\{d_n\}_{n\ge1}$~--- некотоpая
не\-огpа\-ни\-чен\-но возpастающая последовательность положительных чисел.
Пpедположим, что $N_n\to\infty$ по вероятности пpи $n\to\infty$
относительно каждой вероятности из семейства $\{{\sf P}_{\theta},\
\theta\in\Theta\}$. Пусть статистика $T_n$ асимптотически
ноpмальна в смысле~$(9)$. Для того чтобы пpи каждом $\theta
\in\Theta$ существовала такая функция pаспpеделения $F(x,\theta)$,
что
\begin{multline*}
{\sf P}_{\theta}\left(\delta(\theta)\sqrt{d_n}
\left (T_{N_n}-t(\theta)\right )<x\right)
\Longrightarrow{}\\
{}\Longrightarrow  F(x,\theta)\quad (n\to\infty)\,,
\end{multline*}
необходимо и
достаточно, чтобы существовало семейство функций pас\-пpе\-де\-ле\-ния
${\cal H}=\{H(x,\theta):\ \theta\;\in$\linebreak $\in\;\Theta\}$, удовлетвоpяющее
условиям}
$$
H(x,\theta)=0\,,\quad x<0\,,\quad \theta\in\Theta\,;
$$

\vspace*{-12pt}

\noindent
\begin{multline}
F(x,\theta)=\int\limits_{0}^{\infty}\Phi\big(x\sqrt{y}\big)\,d_y\,H(y,\theta)\,,\\
x\in{\r}^1\,,\quad \theta\in\Theta\,;
\label{4.4b}
\end{multline}
$$
{\sf P}_{\theta}(N_n<d_nx)\Longrightarrow H(x,\theta)\quad
(n\to\infty)\,,\quad  \theta\in\Theta\,.
$$
\textit{Пpи этом если функции pаспpеделения
случайных величин $N_n$ не зависят от $\theta$, то не зависят от
$\theta$ и функции pаспpеделения $H(x,\theta)$, т.\,е.\ семейство
${\cal H}$ состоит из единственного элемента.}

\smallskip

\noindent
Д\,о\,к\,а\,з\,а\,т\,е\,л\,ь\,с\,т\,в\,о.$~$Данная лемма, по сути, лишь
пеpеобозначениями отличается от Теоpемы~3 из~\cite{10be}, доказательство
которой, в свою очередь, основано на общих теоремах о сходимости
суперпозиций независимых случайных последовательностей~\cite{9be, 42be}.

\smallskip

Смесь~(\ref{4.4b}) отличается от той, которая фигурирует в лемме~3.1,
тем, что смешивающий параметр стоит не в знаменателе аргумента
подынтегральной функции распределения, а в числителе. Поэтому из
сказанного в разд.~3 вытекает, что если $H(x,\theta)$ совпадает
с функцией обратного показательного распределения $Q(x)$, то
предельная функция $F(x,\theta)$ является функцией распределения
Лапласа.

Обратное показательное распределение~--- это распределение
случайной величины
$$
V=\fr{1}{U}\,,
$$
где случайная величина $U$
имеет стандартное показательное распределение $E(x)$. При этом
\begin{multline*}
Q(x)={\sf P}(V<x)={\sf P}\left(\fr{1}{U}<x\right)={}\\
{}={\sf P}\Big(U>\fr{1}{x}\Big)=e^{-1/x}\,,\quad x\ge0\,.
\end{multline*}
Обратное показательное распределение $Q(x)$ является частным случаем
распределения Фреше, хорошо известного в асимптотической теории
экстремальных порядковых статистик как предельное распределение
II~типа (см., например,~\cite{7be}).

Приведем пример ситуации, в которой случайный объем выборки имеет
предельное распределение вида $Q(x)$. Пусть $Y_1,Y_2,\ldots$~---
независимые одинаково распределенные случайные величины с одной и
той же непрерывной функцией распределения. Пусть $m$~---
произвольное натуральное число. Обозначим
$$
N(m)=\min\big\{n\ge1:\,\max_{1\le j\le m}Y_j<\max_{m+1\le k\le
m+n}Y_k\big\}\,.
$$
Случайная величина $N(m)$ имеет смысл количества
дополнительных наблюдений, которые надо произвести, чтобы текущий
(по $m$ наблюдениям) максимум был перекрыт. Распределение
случайной величины $N(m)$ было найдено С.~Уилксом, который в
работе~\cite{54be} показал, что распределение величины $N(m)$ является
дискретным распределением Парето:
\begin{equation}
{\sf P}\big(N(m)\ge
k\big)=\fr{m}{m+k}\,,\quad k\ge1
\label{4.4b'}
\end{equation}
(см.\ также~\cite{15be},
с.~85).

Пусть теперь $N^{(1)}(m),N^{(2)}(m),\ldots$~--- независимые
случайные величины с одним и тем же распределением~(\ref{4.4b'}). Целая
часть числа $a$ будет обозначаться $[a]$. Так как при любом
фиксированном $x>0$

\noindent
\begin{multline*}
1-\fr{m}{nx(1+(m-1)/(nx))}=
1-\fr{m}{m-1+nx}\le{}\\
{}\le  1-\fr{m}{m+[nx]}\le 1-m/(m+nx)={}\\
{}= 1-\fr{m}{nx(1+m/(nx))}\,,
\end{multline*}
то для любого $x>0$
\begin{multline*}
\lim_{n\to\infty}{\sf P}\Big(\fr{1}{n}\max_{1\le j\le
n}N^{(j)}(m)<x\Big)={}\\
{}= \lim_{n\to\infty}{\sf P}\big(\max_{1\le j\le
n}N^{(j)}(m)<nx\big)= {}\\
{}=\lim_{n\to\infty}\Big(1-\fr{m}{m+[nx]}\Big)^n={}\\
{}=\lim_{n\to\infty}\Big(1-
\fr{m}{nx}\Big)^n=e^{-m/x}\,.
\end{multline*}
Поэтому, если положить
$$
N_n=\max_{1\le j\le n}N^{(j)}(m)
$$
при $m=1$, лемма~4.1 с
$d_n=n$ дает иллюстрацию того, как вместо ожидаемого в
соответствии с утверждениями классической асимптотической
статистики нормального распределения при замене объема выборки
случайной величиной в качестве предельного распределения
регулярных статистик возникает распределение Лапласа. При этом
изменение значения параметра $m$ влечет изменение параметра
масштаба (дисперсии) итогового распределения Лапласа, точнее, если
\begin{equation}
\lim_{n\to\infty}{\sf P}\Big(\fr{N_n}{n}<x\Big)=
e^{-m/x}\,, \label{4.5b}
\end{equation}
то соответствующая плотность
распределения Лап\-ла\-са имеет вид
\begin{equation}
\ell(x)=\sqrt{\fr{m}{2}}\,e^{-\sqrt{2m}\:|x|}\,,\quad \sigma^2=\fr{1}{m}\,,
\quad x\in{\r}^1\,.
\label{4.6b}
\end{equation}

Примеры прикладных статистических задач, в которых объем выборки
существенно случаен, можно найти, например, в книгах~\cite{12be} и~\cite{3be}.

\section{Экстремальные энтропийные свойства распределения Лапласа}

Хорошо известно, что дифференциальная энт\-ро\-пия является
практически оптимальной характеристикой неопределенности
(не\-пред\-ска\-зу\-емости) вероятностных распределений. Если $Y$~---
абсолютно непрерывная случайная величина с плотностью $p(x)$, то
ее дифференциальная энтропия определяется как функционал
$$
H(Y)=-{\sf E}\ln p(Y)=-\int\limits_{-\infty}^{\infty}p(x)\ln
p(x)\,dx\,.
$$

Привлекательность распределения Лапласа в качестве вероятностной
модели при решении конкретных прикладных задач во многом
обусловливается его экстремальными энтропийными свойствами.
Согласно энтропийному (информационному) подходу построения
вероятностных математических моделей в условиях неопределенности
следует выбирать то модельное распределение, которое обладает
максимальной энтропией при заданном условиями задачи комплексе
ограничений. Выбор максимально неопределенной модели в
определенном смысле соответствует реализации минимаксного подхода.

Хорошо известны следующие экстремальные энтропийные свойства
распределения Лапласа.
\begin{enumerate}[1.]
\item  Пусть ${\cal F}_1$~--- класс всех абсолютно непрерывных
распределений, носителем которых является вся вещественная прямая,
с нулевым математическим ожиданием и абсолютным моментом первого
порядка, равным $1/\sqrt{2}$. Тогда, если, как и ранее, $W$~---
случайная величина с распределением Лапласа $\Lambda(x)$, 
$$
H(W)=\max\big\{H(Y):\: {\cal L}(Y)\in{\cal F}_1\big\}
$$
(см., например,~\cite{41be}).
\item
В разд.~3 было показано (см.\ лемму~3.1), что случайная
величина $W$ с распределением Лапласа допускает представление~(\ref{3.1b})
в виде произведения независимых случайных величин $X$ и
$\sqrt{U}$, где $X$~--- случайная величина со стандартным
нормальным распределением, а $U$~--- случайная величина со
стандартным показательным распределением. Хорошо известны
экстремальные энтропийные свойства нормального и показательного
распределений. Пусть ${\cal F}_2$~--- класс всех абсолютно
непрерывных распределений, носителем которых является вся
вещественная прямая, с нулевым математическим ожиданием и
единичной дисперсией; ${\cal F}_3$~--- класс всех абсолютно
непрерывных распределений, носителем которых является
неотрицательная полуось, с единичным математическим ожиданием.
Тогда
\begin{multline*}
H(W)=\max\left \{ H(Y):\: Y\eqd Y'\sqrt{Y''};\right. \\
\left.{\cal
L}(Y')\in{\cal F}_2,\,{\cal L}(Y'')\in{\cal F}_3\right \}
\end{multline*}
(см., например,~\cite{41be}). Этим свойством часто мотивируется выбор
распределения Лапласа в качестве распределения погрешностей
измерений, в которых точность (параметр масштаба) изменяется от
измерения к измерению случайным образом (см., в частности,~\cite{52be, 17be}).
Это свойство также позволяет построить методику определения
характерных временн$\acute{\mbox{ы}}$х масштабов в экспериментах с плазменной
турбулентностью~\cite{11be}.
\end{enumerate}

\section{Задача проверки гипотез: асимптотическая постановка}

Рассмотрим задачу проверки простой гипотезы в случае
однопараметрического семейства. Пусть имеются независимые наблюдения
${\bf X}_n\;=$\linebreak $=\;(X_1,\ldots,X_n)$, каждое из которых принимает
значения в произвольном измеримом пространстве $(\cal{X},\cal{A})$ и
имеет неизвестную с точностью до параметра $\theta$ плотность
$p(x,\theta)$ относительно некоторой $\sigma$-конечной меры
$\nu(\cdot)$ на $\cal{A}$. Предположим, что неизвестный параметр
$\theta$ принадлежит открытому множеству $\Theta\subset{\sf R}^1$,
содержащему ноль. Обозначим через ${\p}_{n,\theta}, {\e}_{n,\theta}$
соответственно распределение и математическое ожидание~${\bf X}_n$, 
а через ${\p}_{\theta}, {\e}_{\theta}$
соответственно распределение и математическое ожидание~$X_1$.

Пусть мы хотим проверить простую гипотезу
\begin{equation}
 {\sf H}_0:\theta=0
\label{6.1b}
\end{equation}
против сложной альтернативы $\theta\ne 0$. В
общем случае наилучшего (равномерно наиболее мощного) критерия не
существует, и поэтому рассмотрим асимптотический подход, при
котором $n\to\infty$. Рассмотрим сначала простую альтернативу
($\theta_1$ известно)
\begin{equation}
{\sf H}_1:\theta = \theta_1\ne0\,.
\label{6.2b}
\end{equation}
Заметим, что точка $\theta\:=\:0$ в гипотезе ${\sf H}_0$
может быть заменена на любую фиксированную точку
$\theta_0\in\Theta$. Этот случай сводится к предыдущему с помощью
рассмотрения семейства плотностей вида $p(x,\theta_0+\xi)$,
где $\xi$ принадлежит некоторой окрестности нуля. Согласно
фундаментальной лемме Неймана--Пирсона (см., например,~\cite{46be},
теорему~3.2.1) наилучший (наиболее мощный) критерий основан на
логарифме отношения правдоподобия
\begin{equation}
\Lambda_n(\theta) =
\sum_{i=1}^{n}(l(X_i,\theta)-l(X_i,0))\,,
\label{6.3b}
\end{equation}
где $l(x,\theta) = \ln p(x,\theta)$, и отвергает гипотезу ${\sf H}_0$
в случае, если
$$
\Lambda_n(\theta_1) > c_n\,,
$$
причем критическое значение $c_n$ выбирается из условия
\begin{equation}
{\p}_{n,0}(\Lambda_n(\theta_1)>c_n) = \alpha\,,
\label{6.4b}
\end{equation}
где $\alpha\in(0,\,1)$~--- фиксированный уровень значимости, и
для простоты предполагается непрерывность распределения
$\Lambda_n(\theta_1)$ при гипотезе ${\sf H}_0$, т.\,е.\ считается,
что
$$
{\p}_{n,0}(\Lambda_n(\theta_1)=c_n) = 0\,.
$$
Далее предположим также, что существуют все необходимые моменты
случайных величин $l(X_1,\theta)$ и все необходимые производные по
$\theta$ функций $l(x,\theta)$.

%\end{document}

Обозначим через
\begin{align}
\mu(\theta) &= {\e}_{\theta}(l(X_1,\theta)-l(X_1,0))\,,\label{6.5b}\\
\sigma^2(\theta)& = {\D}_{\theta}(l(X_1,\theta)-l(X_1,0))%\label{6.6b}
\notag
\end{align}
математическое ожидание и дисперсию случайных величин
$l(X_1,\theta)-l(X_1,0)$ при распределении ${\p}_{\theta}$.

Поскольку $\Lambda_n(\theta)$ есть сумма независимых одинаково
распределенных случайных величин, то согласно центральной предельной
теореме имеем при $n\to\infty$
\begin{equation}
{\cal L}\left( \fr{\Lambda_n(\theta_1)-n\mu(0)}{\sigma(0)\sqrt{n}}
\Bigl|{\sf H}_0\right )\to{\cal N}(0,\,1)\,, \label{6.7b}
\end{equation}
где ${\cal L}(Z\bigl|{\sf H}_i)$ означает распределение $Z$ при
гипотезе ${\sf H}_i,\,i = 0,\ 1$, а ${\cal N}(\mu,\sigma^2)$~---
нормальный закон с параметрами $\mu$ и $\sigma^2$. Из соотношения~(\ref{6.4b})
теперь следует (поскольку сходимость функций
распределения к нормальной функции распределения равномерна), что
\begin{equation}
c_n = \sqrt{n}\cdot \sigma(0)u_{\alpha}+n\mu(0)+{\it o}(1)\,,
\label{6.8b}
\end{equation}
где $u_{\alpha} = \Phi^{-1}(1-\alpha)$ и $\Phi(x)$~--- функция
распределения стандартного нормального закона.

Обозначим через $\beta_n^*(\theta_1)$ мощность наилучшего критерия
для проверки гипотезы ${\sf H}_0$ против альтернативы ${\sf H}_1$
(см.~(\ref{6.1b}), (\ref{6.2b})), основанного на $\Lambda_n(\theta_1)$, т.\,е.\
\begin{equation*}
\beta_n^*(\theta_1) = {\p}_{n,\theta_1}(\Lambda_n(\theta_1)>
c_n)\,.
%\label{6.9}
\end{equation*}
Покажем, что этот критерий состоятелен, т.\,е.\ справедлив
следующий хорошо известный результат (см., например,~\cite{16be},
теорему~3.3.1). Приведем здесь короткое доказательство этого факта,
основанное на неравенстве Иенсена.

\medskip
\noindent
{\bf Лемма~6.1.} {\it Если $\sigma^2(0)>0$, $\sigma^2(\theta_1)>0$,
то
$$
\beta_n^*(\theta_1)\to 1\quad (n \to \infty)\,.
$$
}

\medskip

\noindent
Д\,о\,к\,а\,з\,а\,т\,е\,л\,ь\,с\,т\,в\,о. Используя опять центральную предельную
теорему, получаем
\begin{equation*}
{\cal L}\left (\fr{\Lambda_n(\theta_1)-n\mu(\theta_1)}
{\sigma(\theta_1)\sqrt{n}}\Bigl|{\sf H}_1\right )\to {\cal
N}(0,\,1)\,,
%\label{6.10b}
\end{equation*} 
и поэтому с учетом~(\ref{6.8b}) для мощности
$\beta_n^*(\theta_1)$ имеем представление
\begin{multline}
\beta_n^*(\theta_1)= {\p}_{n,\theta_1}(\Lambda_n(\theta_1)>c_n) ={}\\
{}=
1-\Phi\biggl(\fr{c_n - n\mu(\theta_1)}
{\sigma(\theta_1)\sqrt{n}}\biggr)+{\it o}(1)= {}\\
{}=\Phi\left(\fr{\sqrt{n}(\mu(\theta_1)-\mu(0))-u_{\alpha}\sigma(0)}
{\sigma(\theta_1)}\right)+{\it o}(1)\,. \label{6.11b}
\end{multline}
Применив теперь неравенство Иенсена (см., например,~\cite{4be}, теорему~4.7.5)
к математическим ожиданиям $\mu(\theta_1)$, $\mu(0)$
(см.~(\ref{6.5b})), имеем
$$
\mu(0) ={\e}_0\ln\fr{p(X_1,\theta_1)}{p(X_1,0)}\le\ln
{\e}_0\fr{p(X_1,\theta_1)}{p(X_1,0)} = 0\,,
$$
\begin{multline*}
\mu(\theta_1)={\e}_{\theta_1}\ln\fr{p(X_1,\theta_1)}{p(X_1,0)}
= -{\e}_{\theta_1}\ln\fr{p(X_1,0)}{p(X_1,\theta_1)}\ge{}\\
{}\ge
-\ln {\e}_{\theta_1}\fr{p(X_1,0)}{p(X_1,\theta_1)} = 0\,.
\end{multline*}
Из этих неравенств следует, что
$$
\sqrt{n}(\mu(\theta_1)-\mu(0))\to +\infty\quad (n\to\infty)\,,
$$
и поэтому в силу формулы~(\ref{6.11b}) действительно
$$
\beta_n^*(\theta_1) \to 1\quad  (n \to \infty)\,.
$$
Факт стремления мощности $\beta_n^*(\theta_1)$ к единице (точнее, скорость
сходимости к единице) может быть использован для сравнения
различных состоятельных критериев (см., например,~\cite{16be} и обзор,
приведенный там). Однако здесь будет рассмотрен несколько иной,
асимптотический подход к сравнению различных критериев. Это так
называемый подход Питмэна (см.~\cite{50be}), согласно которому для
получения нетривиального предела мощности $\beta_n^*(\theta_1)$,
заключенного между $\alpha$ и $1$, рассматривают
последовательность альтернатив $\theta_1 = \theta_n$,
стремящуюся к нулю. Из центральной предельной
тео\-ре\-мы для схемы серий (см., например,~\cite{4be}, тео\-ре\-му~8.4.5)
следует, что в регулярном случае для выполнения этого должно быть
\begin{equation*}
\mu(\theta_n)-\mu(0)= {\cal O}\left (n^{-1/2}\right )\,,\quad
\theta_n = {\cal O}\left (n^{-1/2}\right )\,.
\end{equation*}
Поэтому будем рассматривать задачу проверки прос\-той гипотезы ${\sf H}_0$
(см.~(\ref{6.1b})) против последовательности сложных близких
альтернатив вида
\begin{equation*}
{\sf H}_{n,1}: \theta = \fr{t}{\sqrt{n}}\,,\quad 0<t\le C\,,\quad
C>0\,, %\label{6.14b}
\end{equation*}
где параметр $t$ неизвестен. Для любого
фиксированного $t\in (0,\,C]$ наилучший критерий для проверки
гипотезы ${\sf H}_0$ против простой альтернативы
\begin{equation*}
{\sf H}_{n,t}:\theta = \fr{t}{\sqrt{n}}
%\label{6.14}
\end{equation*}
 основан на логарифме отношения правдоподобия
\begin{equation}
\Lambda_n(t) \equiv \sum_{i=1}^{n}\left (
l(X_i,tn^{-1/2})-l(X_i,0)\right )\,.\label{6.15b}
\end{equation}
Обозначим через $\beta_n^*(t)$ мощность такого критерия уровня
$\alpha\in (0,\,1)$. Заметим, что поскольку $t$ неизвестно,
нельзя использовать статистику $\Lambda_n(t)$ для построения
критерия проверки гипотезы ${\sf H}_0$ против альтернативы ${\sf
H}_{n,1}$. Однако $\beta_n^*(t)$~--- так называемая огибающая
функция мощности~--- дает верхнюю границу для мощности любого
критерия при проверке гипотезы ${\sf H}_0$ против фиксированной
альтернативы ${\sf H}_{n,t}$, $t>0$, и может служить
стандартом при сравнении различных критериев.

Найдем предельное выражение для $\beta_n^*(t)$. При естественных
условиях регулярности формула Тейлора дает
\begin{multline}
l\left (X_i,tn^{-1/2}\right )-l(X_i,0) = {}\\
{}=
\fr{t}{\sqrt{n}}l^{(1)}(X_i)+\fr{1}{2}\,\fr{t^2}{n}l^{(2)}(X_i)+\cdots
\,,
\label{6.16b}
\end{multline}
где
$$
l^{(j)}(x) = \fr{\partial^j}{\partial\theta^j}l(x,\theta)\Bigr|_{\theta =
0}\,,\quad j = 1, 2, \ldots
$$
Поэтому из~(\ref{6.15b}) и~(\ref{6.16b}) получаем
стохастическое разложение для $\Lambda_n(t)$ в виде
\begin{equation}
\Lambda_n(t) = tL_n^{(1)} - \fr{1}{2}\,t^2I +
\fr{1}{2}\,\fr{t^2}{\sqrt{n}}L_n^{(2)} + \cdots\,, \label{6.17b}
\end{equation}
где
$$
L_n^{(j)} = \fr{1}{\sqrt{n}}\sum_{i=1}^{n}(l^{(j)}(X_i)-{\e}_{0}l^{(j)}(X_i))\,,\quad
 j = 1,\,2, \ldots \,,
$$
и $ I = {\e}_{0}(l^{(1)}(X_1))^2$~--- фишеровская информация. При этом в
выражении~(\ref{6.17b}) опущены неслучайный член
$(1/6)(t^3/\sqrt{n}){\e}_0l^{(3)}(X_1)$, члены более
высокого порядка малости, чем $n^{-1/2}$, и использован хорошо
известный факт, состоящий в том, что
\begin{equation*}
{\e}_0l^{(1)}(X_1) = 0\,,\quad
{\e}_0l^{(2)}(X_1) = -I\,.
%\label{6.18b}
\end{equation*}
Критерий, основанный на статистике $\Lambda_n(t)$, отвергает гипотезу
${\sf H}_0$ в пользу альтернативы ${\sf H}_{n,t}$, если
\begin{equation*}
\Lambda_n(t) > c_{n,t}\,,
%\label{6.19b}
\end{equation*}
где критическое значение $c_{n,t}$ выбирается из условия
\begin{equation*}
{\p}_{n,0}(\Lambda_n(t) > c_{n,t}) = \alpha\,.
%\label{6.20b}
\end{equation*}
Аналогично~(\ref{6.7b}) и (\ref{6.8b}) из~(\ref{6.17b}) следует, что
\begin{equation}
{\cal L}\left (\Lambda_n(t)|{\sf H}_0\right ) \to {\cal N}
\left (-\fr{1}{2}\,t^2I,t^2I\right )
\label{6.21b}
\end{equation}
и
\begin{equation}
c_{n,t}\to c_t =
 t\sqrt{I}\cdot u_{\alpha}  -\fr{1}{2}\,t^2I\,.
\label{6.22b}
\end{equation}
Найдем теперь предельное распределение $\Lambda_n(t)$ при альтернативе
${\sf H}_{n,t}$. Имеем (см.~(\ref{6.17b}))
\begin{multline*}
{\e}_{n,t/\sqrt{n}}\Lambda_n(t)=
t\sqrt{n}\cdot {\e}_{t/\sqrt{n}}l^{(1)}(X_1)- \fr{1}{2}\,t^2I+{}\\
{}
+\fr{1}{2}\,t^2\left (
{\e}_{t/\sqrt{n}}l^{(2)}(X_1)-{\e}_0l^{(2)}(X_1)\right ) +
{\cal O}\left (n^{-1/2}\right )={}\\
{}=\fr{t^2}{2}I + {\cal
O}\left (n^{-1/2}\right )\,,
%\label{6.23b}
\end{multline*}

\noindent
\begin{multline*}
{\D}_{n,t/\sqrt{n}}\Lambda_n(t) =
t^2{\D}_{n,t/\sqrt{n}}L_n^{(1)}+
{\cal O}\left (n^{-1/2}\right )={}\\
{}= t^2I+{\cal O}\left (n^{-1/2}\right )\,,
%\label{6.24b}
\end{multline*}
поэтому
\begin{equation}
{\cal L}(\Lambda_n(t)\:|\:{\sf H}_{n,t})\to {\cal N}
\left (\fr{1}{2}\,t^2I,t^2I\right )\,.\label{6.25b}
\end{equation}
Теперь с учетом~(\ref{6.22b}), (\ref{6.25b})  и равенства
$$
\beta_n^*(t) = {\p}_{n,t/\sqrt{n}}(\Lambda_n(t)>c_{n,t})
$$
имеем
\begin{equation}
\beta_n^*(t)\to \beta^*(t)= \Phi\left (
t\sqrt{I}-u_{\alpha}\right )\,.
\label{6.26b}
\end{equation}
Заметим,
что для проверки гипотезы ${\sf H}_0$ против альтернативы ${\sf H}_{n,1}$
существуют критерии, основанные на статистиках, отличных
от $\Lambda_n(t)$, и имеющие ту же предельную мощность
$\beta^*(t)$. Такие критерии называются асимптотически наиболее
мощными (АНМ), точнее локально АНМ, поскольку альтернатива
${\sf H}_{n,1}$ имеет локальный характер. Таковы, например, критерии,
основанные на статистиках $L_n^{(1)}$ и $\Lambda_n(t_0)$, где
$t_0 > 0$ фиксировано, оценках максимального правдоподобия и~т.\,п.
Заметим, что все эти статистики не зависят от неизвестного
параметра $t$, и поэтому могут быть использованы при проверке
гипотезы ${\sf H}_0$ против альтернативы ${\sf H}_{n,1}$.

Покажем, например, что критерий, основанный на статистике
$L_n^{(1)}$, является АНМ и имеет предельную мощность
$\beta^*(t)$. Имеем
\begin{align*}
L_n^{(1)} &= \fr{1}{\sqrt{n}}\sum_{i=1}^{n}\,l^{(1)}(X_i)\,, \\
{\e}_{n,0}L_n^{(1)}& = \sqrt{n}\cdot {\e}_0l^{(1)}(X_1) = 0\,, \\
{\D}_{n,0}L_n^{(1)}& = {\D}_0l^{(1)}(X_1) = I\,,
\end{align*}
поэтому, если
$$
{\p}_{n,0}(L_n^{(1)}>c_n^{(1)}) = \alpha \in (0,\,1)\,,
$$
для критического уровня $c_n^{(1)}$ (в силу
центральной предельной теоремы) имеем
\begin{equation}
c_n^{(1)} =
\sqrt{I}\cdot u_{\alpha}+{\it o}(1)\,.
\label{6.27b}
\end{equation}
Далее
\begin{multline*}
{\e}_{n,t/\sqrt{n}}L_n^{(1)}=
\sqrt{n}\cdot {\e}_{t/\sqrt{n}}l^{(1)}(X_1) = {}\\
{}= \sqrt{n}\cdot \int
l^{(1)}(x)p\left (x,tn^{-1/2}\right )\,d\nu(x)={}\\
{} = tI + {\cal O}\left (n^{-1/2}\right )\,,
\end{multline*}

\noindent
\begin{multline*}
{\D}_{n,t/\sqrt{n}}L_n^{(1)}= {\D}_{t/\sqrt{n}}l^{(1)}(X_1)={}\\
{}={\e}_0(l^{(1)}(X_1))^2+{\cal O}\left (n^{-1/2}\right )=
I + {\cal O}\left (n^{-1/2}\right )\,,
\end{multline*}
поэтому для мощности
$\beta_n(t)$ критерия, основанного на статистике $L_n^{(1)}$,  с
учетом~(\ref{6.27b}) имеем
\begin{multline*}
\beta_n(t)=
{\p}_{n,t/\sqrt{n}}\left ( L_n^{(1)}>c_n^{(1)}\right )={}\\
{}=
\Phi\left (\fr{tI-c_n^{(1)}}{\sqrt{I}}\right )+{\it o}(1)={}\\
{} = \Phi\left ( t\sqrt{I} - u_{\alpha}\right )+{\it o}(1)\to
\beta^*(t) ={}\\
{}= \Phi\left (t\sqrt{I} - u_{\alpha}\right )\,.
%\label{6.28b}
\end{multline*}
Соотношение~(\ref{6.26b}) создает
естественную основу для асимптотического сравнения различных АНМ
критериев, однако для различения критериев такого рода, т.\,е.\
удовлетворяющих соотношению
\begin{equation*}
\beta_n(t)\to \beta^*(t)\,,
%\label{6.29b}
\end{equation*}
где $\beta_n(t)$~---
мощность конкретного рассматриваемого критерия, нужны следующие
члены асимптотического разложения $\beta_n(t)$, т.\,е.\
представление типа
\begin{equation}
\beta_n(t) = \beta^*(t) + \fr{1}{\sqrt{n}}h_1(t) + \fr{1}{n}\,h_2(t)
+ \cdots\,. \label{6.30b}
\end{equation}
Асимптотическим разложениям в статистике
посвящены работы~\cite{26be} и~\cite{51be}.
При получении формул типа~(\ref{6.30b}) для
различных критериев было замечено, что при выполнении естественных
условий регулярности для АНМ критериев совпадают и члены $h_1(t)$,
различия наступают в членах порядка $n^{-1}$. Этим вопросам
посвящены работы~[47--50]. При этом величина
\begin{equation}
r(t) = \lim_{n\to\infty} n(\beta_n^*(t) - \beta_n(t))
\label{6.31b}
\end{equation}
допускает статистическую интерпретацию в терминах
необходимого числа наблюдений и позволяет находить асимптотический
дефект (см.~[49--51] и разд.~8, формулы~(\ref{8.11b}), (\ref{8.12b})).

Соотношение~(\ref{6.30b}) может быть понято следующим образом.
Предположим, что статистику $T_n$ АНМ критерия можно монотонным
преобразованием (не меняющим мощности критерия) преобразовать в
статистику $S_n(t)$ такую, что величина
\begin{equation}
\Delta_n(t) \equiv S_n(t) - \Lambda_n(t) \to 0\quad
(n\to\infty) \label{6.32b}
\end{equation}
по вероятности относительно распределений ${\p}_{n,0}$ и
${\p}_{n,t/\sqrt{n}}$. Тогда критерий, основанный на статистике
$S_n(t)$, имеет те же предельные распределения при гипотезах
${\sf H}_0$ и ${\sf H}_{n,t}$, что и критерий, основанный на
$\Lambda_n(t)$, и, следовательно, ту же предельную мощность
$\beta^*(t)$ (см.~(\ref{6.26b})). Например, в последнем примере
$$
T_n = L_n^{(1)}\,,
$$
тогда, полагая 
$$
S_n(t) =
tT_n - \fr{1}{2}\,t^2I\,,
$$
получим (см.~(\ref{6.17b}))
\begin{equation}
\Delta_n(t) = -\fr{1}{2}\,\fr{t^2}{\sqrt{n}}L_n^{(2)}+\cdots \to
0\,.
\label{6.33b}
\end{equation}
В том типичном случае, когда $\Delta_n(t)$, как
в~(\ref{6.32b}), имеет порядок $n^{-1/2}$, т.\,е.\ разность между
$S_n(t)$ и $\Lambda_n(t)$ имеет тот же порядок, можно ожидать, что
мощность $\beta_n(t)$ критерия, основанного на $S_n(t)$ (или на
$T_n$), отличается от $\beta_n^*(t)$ на величину порядка
$n^{-1/2}$. Однако было обнаружено, что для широкого класса АНМ
критериев это отличие имеет порядок $n^{-1}$ (см.~[49, 50]).
Первоначально выражения для $r(t)$ (см.~(\ref{6.31b})) строились с
помощью асимптотических разложений для $\beta_n^*(t)$ и
$\beta_n(t)$ (см.~\cite{26be, 51be}). Этот подход технически очень
трудоемкий и громоздкий. Однако в работах~\cite{19be, 25be} была получена
общая формула для величины $r(t)$ без построения асимптотических
разложений. Эта формула имеет наглядный вид в терминах условных
дисперсий. Для ее демонстрации обозначим через $\Lambda(t)$
нормальную случайную величину вида ${\cal N}(-(1/2)t^2I,t^2I)$.
Тогда в силу~(\ref{6.21b})
\begin{equation*}
{\cal L}(\Lambda_n(t)\:|\:{\sf H}_0)\to{\cal L}(\Lambda(t))\,.
%\label{6.34b}
\end{equation*}
Предположим, что при гипотезе ${\sf H}_0$ случайный вектор
%\begin{equation*}
$\left (\sqrt{n}\cdot \Delta_n(t),\:\Lambda_n(t)\right )
%\label{6.35b}
$ %\end{equation*}
имеет предельное распределение (типичным образом двумерное нормальное),
совпадающее с распределением вектора
%\begin{equation*}
$\left (\Delta(t),\,\Lambda(t)\right)$.
%\label{6.36b}
%\end{equation*}
Тогда можно показать~\cite{19be, 25be}, что
\begin{equation}
r(t) = \fr{1}{2t\sqrt{I}}\,\varphi\left (
u_{\alpha}-t\sqrt{I}\right )
{\D}(\Delta(t)\:|\:\Lambda(t) = c_t)\,,
\label{6.37b}
\end{equation}
где
$$
c_t= t\sqrt{I}\cdot u_{\alpha}-\fr{1}{2}\,t^2I
$$ 
и 
$$
\varphi(x)
=\Phi'(x)\,.
$$ 

Например, для критерия, основанного на статистике
$T_n =L_n^{(1)}$ в работе~[50] (формула~1.4.10) получено
выражение
\begin{multline*}
r(t) =\fr{t^3}{8\sqrt{I}}\,\varphi\left (u_{\alpha}-t\sqrt{I}\right )
({\D}_0l^{(2)}(X_1)-{}\\
{}-I^{-1}{\e}_0^2l^{(1)}(X_1)l^{(2)}(X_1))\,.
\end{multline*}
В работе~[50] рассмотрен общий случай в терминах общего
статистического эксперимента и приведена общая теорема~[50],
теорема~3.2.1), дающая достаточные условия для существования
предела
\begin{multline}
r = \lim_{n\to\infty}\tau_n^{-2}(\beta_n^* -
\beta_n) ={}\\
{}= \fr{1}{2}\,e^bp(b){\D}\left (\Delta\:|\:\Lambda = b\right )\,,
\label{6.38b}
\end{multline}
где
$b = \Phi_1^{-1}(1-\alpha)$,
$\Phi_1(x)$~--- функция распре\-де\-ления, предельная для логарифма
отношения\linebreak правдоподобия $\Lambda_n$ при гипотезе ${\sf H}_0$ (выше
было $\Phi_1(x)=\Phi((x+(1/2)t^2I)/(t\sqrt{I})))$,
$p(x) = \Phi_1'(x)$ и $\tau_n\to 0$~--- малый параметр (выше было
$\tau_n\;=$\linebreak $=\;n^{-1/2})$, $(\Delta,\,\Lambda)$~--- случайный
вектор, предельный для $(\tau_n^{-1}\Delta_n,\,\Lambda_n)$,\
$\Delta_n = S_n - \Lambda_n$,\  $S_n$~--- монотонное
преобразование статистики критерия $T_n$.

Далее  будет приведён пример, когда в формуле~(\ref{6.38b})
$\tau_n\ne n^{-1/2}$. С этой целью будет рассмотрено
распределение Лапласа с параметром сдвига и на эвристическом уровне
показано, что в этом случае $\tau_n = n^{-1/4}$.

\vspace*{-6pt}

\section{Асимптотически наиболее мощный критерий в случае
распределения Лапласа}

В разд.~4 было показано, как в
математической статистике естественно возникает распределение
Лапласа (см.\ формулы~(\ref{4.5b}) и~(\ref{4.6b})). В этом разделе рассмотрим
приложение распределения Лапласа к асимптотическим задачам
проверки гипотез, описанным в предыдущем разделе. Итак, рассмотрим
распределение Лапласа с неизвестным параметром сдвига, т.\,е.\
рассмотрим распределение с плот\-ностью
\begin{equation}
p(x,\theta) = \fr{1}{2}\,e^{-|x - \theta|}\,, \quad x,\,\theta\:\in{\r}^1\,.
\label{7.1b}
\end{equation}
Заметим, что это семейство не является
регулярным, поскольку у $p(x,\theta)$ не существует производной по
$\theta$ в точке $\theta=x$. Это отсутствие регулярности приводит
к нарушению естественного порядка $n^{-1}$ разности
${\beta}_n^*(t)-\beta_n(t)$ и приводит к порядку $n^{-1/2}$.
Факт нарушения обычных порядков при сравнении оценок в случае
распределения Лапласа был отмечен в работе~\cite{51be} (с.~34). При
этом там была ссылка на работу~[52], в которой на эвристическом
уровне построено асимптотическое разложение для оценок
максимального правдоподобия. Строгое доказательство для таких
оценок дано в работе~[53]. Рассмотрим здесь АНМ критерий,
основанный на знаковой статистике, и получим на эвристическом
уровне без привлечения асимптотических разложений формулу для
$r(t)$ с помощью общей теоремы~3.2.1 из работы~[50]. 
Формальное доказательство полученной формулы для $r(t)$ (см.~(\ref{8.10b})),
состоящее в проверке условий этой теоремы, будет опубликовано в
одной из наших следующих статей.

В работе~[54] была доказана следующая лемма
 об асимптотическом поведении логарифма отношения
правдоподобия $\Lambda_n(t) \equiv \Lambda_n(tn^{-1/2})$ (см.~(\ref{6.3b}) 
и~(\ref{6.15b})) в случае распределения Лапласа.

\vspace*{2pt}

\noindent
{\bf Лемма~7.1.} {\it В случае распределения Лапласа~(\ref{7.1b})
справедливы следующие соотношения: фишеровская информация равна~1,
т.\,е.\ $I=1$,
\begin{align*}
{\cal L}(\Lambda_n(t)\:|\:{\sf H}_0) & \to
{\cal N}\left (-\fr{t^2}{2}, t^2\right )\,,\\
{\cal L}(\Lambda_n(t)\:|\:{\sf H}_{n,t}) & \to {\cal N}\left(
\fr{t^2}{2}, t^2\right )\,,
\end{align*}
и поэтому}
$$
\beta_n^*(t) \to \beta^*(t)=\Phi(t-u_\alpha) \quad
(n \to \infty)\,.
$$ 
\pagebreak

\noindent
Эта лемма показывает, что отсутствие дифференцируемости по
$\theta$ функции $p(x, \theta)$ (см.~(\ref{7.1b})) в точке $\theta=x$
качественно не влияет на порядок альтернатив $\theta_n$ (равный
$n^{-1/2}$) и вид предельной мощности $\beta^*(t)$.

\section{Формула для предельного отклонения мощностей}

В этом разделе на эвристическом уровне будет показано, что справедлива
формула (см.~(\ref{6.37b}) и~(\ref{6.38b}))
\begin{equation}
r(t)=\lim_{n \to \infty}
{\sqrt {n}\,(\beta_n^*(t)-\beta_n(t))}=\fr{t^2}{3}\,\varphi(u_\alpha-t)\,,
\label{8.1b}
\end{equation}
где $\beta_n(t)$~--- функция мощности АНМ критерия,
основанного на знаковой статистике
\begin{equation}
T_n=\fr{1}{\sqrt
n}\sum \limits_{i=1}^n{\mathrm{sign}\left ( X_i\right )}\,.
\label{8.2b}
\end{equation}
Формула~(\ref{8.1b})
показывает, что отсутствие регулярности у распределения Лапласа
приводит к нарушению естественного порядка разности
$\beta_n^*(t)-\beta_n(t)$ (равного $n^{-1}$ (см.~(\ref{6.31b})). Из
формулы~(\ref{8.1b}) также следует, что этот порядок равен $n^{-1/2}$.

Получим сначала стохастическое разложение для $\Lambda_n(t)$.
Имеем:
\begin{multline*}
\Lambda_n(t)=t\sqrt n+2\sum\limits_{i=1}^{n}
(X_i - tn^{-1/2})\I_{[0,tn^{-1/2}]}(X_i)-{}\\
{}-\fr{2t}{\sqrt n}
\sum\limits_{i=1}^n \I_{(-\infty,0)}(X_i)=
\fr{t}{\sqrt n}\sum \limits_{i=1}^{n} \I_{[0,\,\infty)}(X_i)-{}\\
{}- 
\fr{t}{\sqrt n}\sum \limits_{i=1}^n \I_{(-\infty, 0)}(X_i)+{}\\
{}+
2\sum \limits_{i=1}^n \left (X_i-tn^{-1/2}\right )\I_{[0,\,tn^{-1/2}]}(X_i)={}\\
{}=\fr{t}{\sqrt n}\sum \limits_{i=1}^n \mathrm{sign}\left (X_i\right )+
\fr{t}{\sqrt n}\sum \limits_{i=1}^n \I_{\{0\}}(X_i)+{}\\
{}+
2\sum \limits_{i=1}^n (X_i-tn^{-1/2})\I_{[0,\,tn^{-1/2}]}(X_i)\,.
\end{multline*}
Поскольку распределение $X_i$ непрерывно, то
$$
\p_{n,\theta}\left (\sum \limits_{i=1}^n {\I_{\{0\}}}
(X_i)>0\right )=0\,, \quad  \theta>0\,,
$$
и поэтому почти всюду справедливо представление
\begin{multline*}
\Lambda_n(t)=\fr{t}{\sqrt n}
\sum \limits_{i=1}^n \mathrm{sign}\left(X_i\right )+{}\\
{}+
2\sum\limits_{i=1}^n\left ( X_i\:-\: tn^{-1/2}\right )
\I_{[0,\,tn^{-1/2}]}(X_i)\,.
%\label{8.3b}
\end{multline*}
Рассмотрим следующее монотонное преобразование
$(t>0)$ статистики $T_n$ (см.~(\ref{8.2b})):
$$
S_n(t) = tT_n-\fr{1}{2}\,t^2\,,
$$
тогда (см.~(\ref{6.32b}) и (\ref{6.33b}))
\begin{multline*}
\Delta_n(t)=S_n(t)-\Lambda_n(t)={}\\
{}=-\fr{1}{2}\,t^2 -
2 \sum\limits_{i=1}^n \left (X_i-\:tn^{-1/2}\right )
\I_{[0,\,tn^{-1/2}]}(X_i)\,.
%\label{8.4b}
\end{multline*}
Справедливо следующее утверждение (доказательство приведено в
работе~[54]).
\medskip

\noindent
{\bf Лемма~8.1.} {\it Для распределения Лапласа~(\ref{7.1b}) справедливы
следующие соотношения
$$
{\cal L}(\sqrt[4]n \cdot  \Delta_n(t)\:|\:{\sf H}_0) \to
{\cal N}\left (0,\, \fr{2t^3}{3}\right)\,,
$$

\noindent
\begin{multline*}
{\cal L}((\sqrt[4] n\cdot \Delta_n(t)\,,\Lambda_n(t))\:|\:{\sf H}_0) \to{}\\
{}\to 
{\cal N}_2 \left (0,\, \fr{2t^3}{3},\,0,\, -\fr{t^2}{2},\, t^2\right )\,,
\end{multline*}
где ${\cal N}_2$~--- двумерный нормальный закон с
соответствующими параметрами. }

Из этой леммы следует, что случайные величины $\sqrt[4] n \cdot \Delta_n(t)$ 
и $\Lambda_n(t)$ асимптотически независимы, и поэтому
формула для $r(t)$ (см.~(\ref{6.38b})) c $\tau_n=n^{-1/4}$ приобретает
вид (см.~(\ref{6.37b}) и~(\ref{8.1b}))
\begin{multline}
r(t)=\lim_{n \to \infty}{\sqrt n(\beta_n^*(t)-\beta_n(t))}= {}\\
=\fr{1}{2t}\,\varphi\left (
u_{\alpha}-t \right ) \D\left (
\Delta(t)\:|\:\Lambda(t)=c_t\right )={}\\
{}=\fr{1}{2t}\,\varphi\left (u_{\alpha}-t\right )
\D\Delta(t)= \fr{t^2}{3}\,\varphi\left (
u_{\alpha}-t\right )\,,
\label{8.10b}
\end{multline}
где $\Lambda(t)$ и $\Delta(t)$~--- независимые нормальные случайные
величины с параметрами соответственно $(-t^2/2,\,t^2)$ и
$(0,\,2t^3/3)$.

Найдем теперь асимптотическое представление для дефекта
(см.~[50, 51], с.~40) критерия, основанного на статистике $T_n$
(см.~(\ref{8.2b})). Напомним, что\linebreak дефект $d_n$ определяется как разность
$(k_n-n)$, где $k_n$~--- число наблюдений, необходимых критерию,
основанному на статистике $T_n$ для достижения той же мощности,
что и критерий, основанный на $\Lambda_n(t)$, при одинаковых
альтернативах $tn^{-1/2}$. Предполагая, что $d_n$~--- непрерывная
переменная, получаем равенство для ее определения
\begin{equation}
\beta_n^*(t)=\beta_{k_n}(t\sqrt{k_nn^{-1}})\,.
\label{8.11b}
\end{equation}
Из лемм~7.1 и 8.1 и формулы~(\ref{8.10b}) следует, что для мощностей
$\beta_n^*(t)$ и $\beta_n(t)$ справедливы представления
$$
\beta_n^*(t)=\Phi\left (
t-u_\alpha\right )+ \fr{1}{\sqrt n}\,\varphi\left (
t-u_{\alpha}\right ) h^*(t)+{\it o}(n^{-1/2})\,,
$$
$$
\beta_n(t)=\Phi\left (
t-u_\alpha\right )+ \fr{1}{\sqrt n}\,\varphi\left (
t-u_{\alpha}\right ) h(t)+{\it o}(n^{-1/2})\,,
$$
где $h^*(t)$ и $h(t)$~--- некоторые полиномы по $t$ и $u_{\alpha}$. Из
этих соотношений и равенства~(\ref{8.11b}) следует, что $k_n \to \infty$,
$k_n/n \to 1$ и
\begin{multline}
d_n=\sqrt{n}\,\fr{2(h^*(t)-h(t))}{t}+{\it o}(\sqrt n)= {}\\
{}=\sqrt{n}\, \fr{2r(t)}{t\varphi(u_{\alpha}-t)}+{\it o}(\sqrt n)= {}\\
{}= \sqrt {n}\, \fr{2t}{3}+{\it o}\left(\sqrt n\right)\,.
\label{8.12b}
\end{multline}
Таким образом, в отличие от регулярного случая, в
котором $d_n \to d<\infty$~[49, 50],
т.\,е.\ существует конечный асимптотический дефект,  в случае
распределения Лапласа дефект $d_n$ стремится к бесконечности со
скоростью $\sqrt n$.

В  работе~[54] доказано более
слабое утверждение (по сравнению с формулой~(\ref{8.10b})), составляющее
содержание следующей леммы.

\medskip

\noindent
{\bf Лемма~8.2.} {\it В случае распределения Лапласа~(\ref{7.1b})
для любого $0\le\delta<1/2$ справедливо
соотношение}
\begin{equation*}
n^{\delta}(\beta_n^*(t)-\beta_n(t))\to 0\  (n\to\infty),\ \ 
 0<t\le C, \ \  C>0.
\end{equation*}


{\small\frenchspacing
{%\baselineskip=10.8pt
\addcontentsline{toc}{section}{Литература}
\begin{thebibliography}{99}

\bibitem{2be} %1
\Au{Бенинг~В.\,Е., Королёв~В.\,Ю.}
Об использовании распределения Стьюдента
в задачах теории вероятностей и математической статистики~//
Теория
вероятностей и ее применения, 2004. Т.~49. Вып.~3. С.~419--435.

\bibitem{45be} %2
\Au{Laplace~P.\,S.}
M{\'e}moire sur la probabilit{\'e} des causes par les
{\'e}v{\'e}nemens~// M{\'e}moires de Math{\'e}matique et le
Physique, 1774. Vol.~6. P.~621-656. (English translation: Memoir
on the probability of the causes of events~// Statistical
Sciences, 1986. Vol.~1. No.\,3. P.~364--378.)

\bibitem{20be} %3
\Au{Andrews~D.\,F., Bickel~P.\,J., Hampel~F.\,R., Huber~P.\,J.,
Rogers~W.\,H., Tukey~J.\,W.}
Robust estimation of location.~--- Princeton, NJ:  Princeton
University Press, 1972.

\bibitem{35be} %4
Understanding
robust and exploratory data anaysis~/
Eds.\ D.\,C.~Hoaglin, F.~Mosteller, J.\,W.~Tukey.~---
 N.Y.: Wiley, 1983.
 
 \bibitem{52be} %5
\Au{Shevlyakov~G.\,L., Vilchevski~N.\,O.}
Robustness in data analysis:
Criteria and methods.~--- Utrecht: VSP,  2002.

\bibitem{30be} %6
\Au{Easterling~R.\,J.}
Exponential responses with double exponential
measurement error. A model for steam generator inspection~// In: 
Proceedings of the DOE Statistical Symposium, U.S.\ Department of
Energy, 1978. P.~90--110.

\bibitem{37be} %7
\Au{Hsu~D.\,A.}
Long-tailed distributions for position errors in navigation~//
Applied Statistics, 1979. Vol.~28. P.~62--72.

\bibitem{48be} %8
\Au{Okubo~T., Narita~N.}
On the distribution of extreme winds expected
in Japan~// In: National Bureau of Standards Special Publication
560-1, 1980. P.~12.

 \bibitem{21be} %9
\Au{Bagchi~U., Hayya~J.\,C., Ord~J.\,K.}
The Hermite distribution as a model
of demand during lead time for slow-moving items~// Decision
Sciences, 1983. Vol.~14. P.~447--466.

\bibitem{28be} %10
\Au{Dadi~M.\,I., Marks~R.\,J.}
Detector relative efficiencies in the presence
of Laplace noise~// IEEE Transactions in Aerospace
Electronic Systems, 1987. Vol.~23. P.~568--582.

\bibitem{29be} %11
\Au{Damsleth~E., El-Shaarawi~A.\,H.}
ARMA models with double-exponentially
distributed noise~// J.\ of The Royal Statistical
Society, 1989. Vol.~B51. No.\,1. P.~61--69.

\bibitem{47be} %12
\Au{Madan~D.\,B., Seneta~E.}
The variance gamma $($V.G.$)$ model for
share market return~// J. of Business, 1990. Vol.~63. P.~511--524.

\bibitem{44be} %13
\Au{Kozubowski~T.\,J., Podgorski~K.}
Asymmetric Laplace laws and modeling
financial data~// Math. Comput. Modelling, 2001. Vol.~34. P.~1003--1021.

\bibitem{32be} %14
\Au{Frech{\'e}t~M.}
Sur les formules de r{\'e}partition de
revenues~// Revue de l'Institute International de
Statistique, 1939. Vol.~7. No.\,1. P.~32--38.

\bibitem{33be} %15
\Au{Frech{\'e}t~M.}
Letter to the editor~// Econometrica,
1958. Vol.~26. P.~590--591.

\bibitem{38be} %16
\Au{Inoue~T.}
On income distribution: The welfare implication of the
general equilibrium model and the stochastic processes of income
distribution formation. PhD. Thesis. University of Minnesota,
1978.

\bibitem{49be} %17
\Au{Ord~J.\,K., Patil~G.\,P., Taillie~C.}
The choice of a distribution to
describe personal incomes~//
Statistical distributions in scientific work~/
Eds. C.~Taillie, G.\,P.~Patil, B.~Baldessari.~---  Dordrecht--Boston: Reidel, 1981. P.~193--202.
 
\bibitem{22be} %18
\Au{Bagnold~R.\,A.}
The size-grading of sand by wind~// Proc. Royal Soc.
London, 1937. Vol.~A163. P.~250--264.

\bibitem{23be} %19
\Au{Bagnold~R.\,A.}
The physics of blown sand and desert dunes.~---
London: Methuen, 1954.

\bibitem{31be} %20
\Au{Fieller~N.\,R.\,J., Gilbertson~D.\,D., Olbricht~W.}
Skew log Laplace
distributions to interpret particle size distribution data.
Manchester-Sheffield School of Probability and Statistics Research
Report No.\,235, 1984.

\bibitem{24be} %21
\Au{Barndorff-Nielsen~O.\,E.}
Models for
non-Gaussian variation, with applications to turbulence~//
Proc. Royal Soc. A, 1979. Vol.~353. P.~401--419.

\bibitem{11be} %22
\Au{Королёв~В.\,Ю.}
Вероятностно-статистический анализ
хаотических процессов с помощью смешанных Гауссовских моделей.
Декомпозиция волатильности финансовых индексов и турбулентной
плазмы.~--- М.: Изд-во ИПИРАН, 2007. 363~с.

\bibitem{39be} %23
\Au{Johnson~N.\,L., Kotz~S., Balakrishnan~N.}
Continuous univariate
distributions. Vol.~II. 2nd ed.~---  N.Y.: Wiley, 1995.

\bibitem{43be} %24
\Au{Kotz~S., Kozubowski~T.\,J., Podgorski~K.}
The Laplace distribution
and generalizations: A revisit with applications to
communications, economics, engineering and finance.~---
 Boston: Birkhauser, 2001.
 
 \bibitem{6be} %25
\Au{Гpадштейн~И.\,С., Рыжик~И.\,М.}
Таблицы интегpалов, сумм, pядов и
пpоизведений.~---  М.: Наука, 1971. 1108~с.

\bibitem{18be} %25+1
\Au{Феллер~В.}
Введение в теорию вероятностей и ее приложения. Т.~2.~---
М.: Мир, 1984.

\bibitem{34be} %26
\Au{Gnedenko~B.\,V., Korolev~V.\,Yu.}
Random summation: Limit
theorems and applications.~--- Boca Raton: CRC Press, 1996.
 
 \bibitem{14be} %27
\Au{Круглов~В.\,М., Королёв~В.\,Ю.}
Предельные теоремы для случайных
сумм.~---  М.: Изд-во Московского университета, 1990. 269~с.

\bibitem{40be} %28
\Au{Kalashnikov~V.\,V.}
Geometric sums: Bounds for rare events with
applications.~--- Dordrecht: Kluwer Academic Publishers, 1997.

\bibitem{12be} %29
\Au{Королёв~В.\,Ю., Бенинг~В.\,Е.,  Шоргин~С.\,Я.}
Математические основы
теории риска.~---  М.: Физматлит, 2007. 542~с.

\bibitem{1be} %30
\Au{Артюхов~С.\,В., Базюкина~О.\,А., Королёв~В.\,Ю., Кудрявцев~А.\,А.,
Шевцова~И.\,Г.}
Об оптимизации спекулятивной прибыли на примере
пункта обмена валют~// Актуарий, 2008. №\,1(2). С.~50--56.

\bibitem{8be} %31
\Au{Клебанов~Л.\,Б., Мания~Г.\,М., Меламед~И.\,А.}
Одна задача В.\,М.~Золотарёва
и аналоги безгранично делимых и устойчивых распределений в схеме
суммирования случайного числа случайных величин~// Теория
вероятностей и ее применения, 1984. Т.~29. Вып.~4. С.~791--794.

\bibitem{10be} %32
\Au{Коpолёв~В.\,Ю.}
Сходимость случайных последовательностей с независимыми
случайными индексами.~II~// Теоpия веpоятностей и ее
пpименения, 1995. Т.~40. Вып.~4. С.~907--910.

\bibitem{9be} %33
\Au{Коpолёв~В.\,Ю.}
Сходимость случайных последовательностей с независимыми
случайными индексами.~I~// Теоpия веpоятностей и ее
пpименения, 1994. Т.~39. Вып.~2. С.~313--333.

\bibitem{42be} %34
\Au{Korolev~V.\,Yu.}
A general theorem on the limit behavior of
superpositions of independent random processes with applications
to Cox processes~// J.\ of Mathematical Sciences, 1996.
Vol.~81. No.\,5. P.~2951--2956.

\bibitem{7be}  %35
\Au{Гумбель~Э.}
Статистика экстремальных значений.~--- М.: Мир, 1965.

\bibitem{54be} %36
\Au{Wilks~S.\,S.}
Recurrence of extreme observations~//
J.\ of American Mathematical Society, 1959. Vol.~1. No.\,1. P.~106--112.

\bibitem{15be} %37
\Au{Невзоров~В.\,Б.}
Рекорды. Математическая теория.
М.: Фазис, 2000.

\bibitem{3be} %38
\Au{Бенинг~В.\,Е., Королёв~В.\,Ю., Соколов~И.\,А.,  Шоргин~С.\,Я.}
Рандомизированные модели и методы теории надежности информационных
и технических систем.~--- М.: Торус Пресс, 2007. 248~с.

\bibitem{41be} %39
\Au{Kapur~J.\,N.}
Maximum-entropy models in science and
engineering.~--- N.Y.: Wiley, 1989.

\bibitem{17be} %40
\Au{Новицкий~П.\,В., Зогpаф~И.\,А.}
Оценка погpешностей результатов
измеpений.~---  Л.: Энеpгоатомиздат, 1991.

\bibitem{46be} %41
\Au{Lehmann~E.\,L., Romano~J.\,P.}
Testing statistical hypotheses. 3rd ed.~---
Springer, 2005. 784~p.

\bibitem{16be} %42
\Au{Никитин~Я.\,Ю.}
Асимптотическая эффективность
непараметрических критериев.~---  М.: Наука, 1995. 250~с.

\bibitem{4be} %43
\Au{Боровков~А.\,А.}
Теория вероятностей.~--- М.: УРСС, 2003. 470~с.

\bibitem{50be} %44
\Au{Pitman~E.\,J.\,G.}
Lecture notes on nonparametric statictical inference. Lectures given for the
University of North Carolina? Institute of Statistics, 1948.

\bibitem{26be} %45
\Au{Bickel~P.\,J.}
 Edgeworth expansions in nonparametric statistics~//
  Ann. of Statist., 1974. Vol.~2. No.\,1. P.~1--20.
  
  \bibitem{51be} %46
  \Au{Planzagl~J.}
  Asymptotic expansions in parametric statistical theory~// Developments in statistics~/ Ed.\
  P.\,R.~Krishnaiah.~--- N.Y.--London: Academic Press, 1989. Vol.~3. P.~1--97.
  
  \bibitem{27be}
\Au{Bickel~P.\,J., Chibisov~D.\,M., Van~Zwet~W.\,R.}
 On efficiency of first and second order~//
Intern. Statist. Review, 1981. Vol.~49. P.~169--175.

\bibitem{19be} %47
\Au{Чибисов~Д.\,М.}
Вычисление дефекта асимптотически
эффективных критериев~// Теория
вероятностей и ее применения, 1985. Т.~30. Вып.~2. С.~269--288.

\bibitem{25be} %48
\Au{Bening~V.\,E.}
Asymptotic theory of testing statistical
hypotheses.~---  Utrecht: VSP, 2000. 277~p.

\bibitem{36be} %49
\Au{Hodges~J.\,L., Lehmann~E.\,L.}
Deficiency~// Ann. Math. Statist., 1970. Vol.~41. No.\,5. P.~783--801.

\bibitem{53be} %50
\Au{Takeuchi~K.}
Asymptotic theory of statistical
estimation.~--- 1974, Tokyo (in Japanese).

\bibitem{5be} %51
\Au{Бурнашев~М.\,В.} %53
 Асимптотические разложения для медианной оценки параметра~//
Теория
вероятностей и ее применения, 1996. Т.~41. Вып.~4. С.~738--753.

\label{end\stat}

\bibitem{13be} %54
\Au{Королёв~Р.\,А., Тестова~А.\,В., Бенинг~В.\,Е.}
О мощности асмптотически оптимального
критерия в случае распределения Лапласа~//
Вестник Тверского  Государственного Университета.  Сер.\
прикладная математика, 2008.
№\,28. Вып.~1. С.~7--27.






\end{thebibliography}

} 
}
\end{multicols}