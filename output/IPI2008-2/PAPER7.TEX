\newtheorem{theorem}{\indent \sc Теорема}
\newtheorem{lemma}{\indent \sc Лемма}

\newcommand{\norm}[1]{\left\Vert#1\right\Vert}
\newcommand{\eps}{\varepsilon}
\newcommand{\R}{\mathbb R}
\newcommand{\E}{{\sf E}}
\renewcommand{\D}{{\sf D}}
\renewcommand{\le}{\leqslant}
\renewcommand{\ge}{\geqslant}
\renewcommand{\I}{{\bf1}}

\newcommand{\proof}{{\sc Доказательство}}

\def\stat{ushakova}


\def\tit{ОЦЕНИВАНИЕ РАСПРЕДЕЛЕНИЯ ЗАДЕРЖКИ В~БИОЛОГИЧЕСКИХ
ДИНАМИЧЕСКИХ СИСТЕМАХ НА~ПРИМЕРЕ~МОДЕЛИ, ОПИСЫВАЮЩЕЙ ВИЧ-ИНФЕКЦИЮ}
\def\titkol{Оценивание распределения задержки в биологических
динамических системах} % на примере модели, описывающей ВИЧ-инфекцию}
\def\autkol{А.\,Н.~Ушакова}
\def\aut{А.\,Н.~Ушакова$^1$}

\titel{\tit}{\aut}{\autkol}{\titkol}

%{\renewcommand{\thefootnote}{\fnsymbol{footnote}}\footnotetext[1]
%{Работа выполнена при поддержке РФФИ, грант
%08--01--00567.}}

\renewcommand{\thefootnote}{\arabic{footnote}}
\footnotetext[1]{Норвежский научно-технологический
университет (NTNU), Тронхейм, Норвегия, anastasi@math.ntnu.no}

\vspace*{-10pt}

\Abst{Рассмотрена проблема оценивания распределения задержки
в динамических сис\-те\-мах, описывающих, в частности, взаимодействие вируса иммуннодефицита
человека (ВИЧ-инфекция) с иммунной системой. Используются параметрический метод,
основанный на аппроксимации распределения задержки гамма-распределением, и
непараметрический, основанный на регуляризации с выбором параметра регуляризации
либо из предварительно оцененной погрешности измерений, либо
из предварительно оцененного уровня гладкости распределения задержки.}

\vspace*{-5pt}

\KW{системы с задержкой; распределение задержки; параметр регуляризации}

      \vskip 10pt plus 9pt minus 6pt

      \thispagestyle{headings}

      \begin{multicols}{2}

      \label{st\stat}


\section{Введение}
%\vspace*{-6pt}

Одной из важных задач в исследовании ВИЧ-ин\-фек\-ции  является
понимание явлений, происходящих в организме ВИЧ-инфицированного в
течение длительного асимптоматического периода, начинающегося
обычно через несколько недель (иногда месяцев) после заражения
вирусом. Этот период может продолжаться многие годы и неизбежно
заканчивается переходом в стадию СПИДа, когда иммунная система
становится не в состоянии противостоять внешним и внутренним
инфекциям. Для количественного описания указанных явлений
необходимо математическое моделирование инфекционного процесса и
противовирусного иммунитета. Модели, с одной стороны, должны
достаточно точно описывать реальную ситуацию, а с другой~--- быть
достаточно простыми, чтобы позволять оценивать содержащиеся в них
параметры. В настоящее время имеется несколько монографий (среди
которых отметим~\cite{1us}) и большое число статей, посвященных этой
проблеме.

Одна из наиболее известных простых моделей взаимодействия ВИЧ с
иммунной системой при использовании антиретровирусной терапии
(лекарств, которые подавляют вирус на различных стадиях развития,
в частности, делают вновь появившиеся вирионы неспособными к
инфекции) имеет вид 

\vspace*{-8pt}
\noindent
\begin{align} 
\fr{dI}{dt}&=kTV_I(t)-\delta I(t)\,;  \notag\\
\fr{dV_I}{dt}&=(1-\eta)pI(t)-c V_I(t)\,;\label{1us}\\ 
\fr{dV_{NI}}{dt}&=\eta pI(t)-cV_{NI}(t)\,,\notag
\end{align} 
                                                                                                                                                                                                             

\noindent
где $I(t)$~--- плотность зараженных клеток иммунной системы в крови пациента, а
$V_I$ и $V_{NI}$~--- концентрация соответственно способных и
неспособных к инфекции вирионов~\cite{2us}. Числа $k$, $T$, $\delta$,
$\eta$, $p$ и $c$ представляют собой коэффициенты, характеризующие
различные аспекты взаимодействия вируса с иммунной системой, а
также эффект воздействия лекарства. Однако, как было отмечено в~[3--5], 
необходимо учитывать задержку между моментом заражения
вирусом клетки и гибелью клетки с выходом в кровь группы
порожденных ею вирусов. Причем эта задержка может варьироваться в
пределах 1--2~дней~\cite{4us}, поэтому модели с фиксированной
(постоянной) задержкой не в полной мере адекватны. Наиболее
подходящей является модель, в которой задержка представляет собой
случайную величину.

С учетом этого, уравнение~(1) в модели заменяется уравнением
\begin{equation}
\fr{dI}{dt}=kT\int\limits_0^\infty V_I(t-t')f(t')\,dt'-\delta I(t)\,,
\label{2us}
\end{equation}
в котором функцию $f(t)$~--- плотность распределения задержки~---
необходимо оценить по наблюдениям.

\section{Оценивание в гамма-модели}

Поскольку для рассматриваемых далее в работе методов связь между
функциями $I(t)$, $dI(t)/dt$ и $V_I(t)$ не играет роли, вместо
уравнения~(2) будем рассматривать общее уравнение свертки 1-го
рода 


\noindent
\begin{equation}
\int\limits_0^\infty K(t-s)z(s)\,ds=u(t)\,,
\label{3us}
\end{equation} 
где $K(t)$ и
$u(t)$~--- наблюдаемые функции, а $z(t)$~--- оцениваемая функция
(в нашем случае~--- плотность распределения задержки). Функции
$K(t)$ и $u(t)$ предполагаются интегрируемыми, неотрицательными и
стремящимися к нулю при $t \to \pm \infty$.

В биологических системах распределения задержек часто хорошо
аппроксимируются гамма-рас\-пре\-де\-ле\-ниями~\cite{6us}. В настоящем разделе
будет предполагаться, что $z(t)$ является гамма-плот\-ностью, т.\,е.\
имеет вид
$$
z(t)=\fr{\beta^\alpha}{\Gamma(\alpha)}\,t^{\alpha-1}e^{-\beta t}\,,\quad 
t>0,\quad \alpha>0\,,\quad \beta>0\,,
$$ 
и, таким образом, задача
сводится к оценке параметров $\alpha$ и $\beta$. Функции $K(t)$ и
$u(t)$ наблюдаются в  моменты времени $t_1,\ldots ,t_n$ со случайными
ошибками.

В~\cite{4us} предлагается оценивать параметры по методу наименьших квадратов.
Однако в этом случае приходится минимизировать по $\alpha$ и $\beta$ выражение
\begin{multline*}
S(\alpha,\beta)={}\\
{}=\sum_{j=1}^n
\left(
\fr{\beta^\alpha}{\Gamma(\alpha)}
\int\limits_0^\infty K(t_j-s)s^{\alpha-1}e^{-\beta s}\,ds-u(t_j)
\right)^2\,,
\end{multline*}
что приводит к громоздким и сильно нелинейным уравнениям, очень чувствительным к
ошибкам измерений и замене непрерывной модели дискретной. Ниже приводится простой
в вычислительном плане метод, достаточно устойчивый к погрешностям измерений.
\begin{table*}\small
\begin{center}
\Caption{Среднеквадратическая ошибка исходных данных и оценок
\label{t1us}}
\vspace*{2ex}

\tabcolsep=13pt
\begin{tabular}{|c|c|c|c|c|c|}
\hline&&&&&\\[-9pt]
$\alpha$&$\beta$&$\fr{\sigma_\xi}{\max K(t)}$&$\fr{\sigma_\eta}{\max u(t)}$&
$\ \ \ \ \ \ \fr{\hat\sigma_\alpha}{\alpha}\ \  \ \ \ \ $&\ \ \ \ \ \  $\fr{ \hat\sigma_\beta}{\beta}$\ \ \  \ \ \ \\
&&&&&\\[-9pt]
\hline
%\hline
1&1&0,025&0,032&0,080&0,067\\
%\hline
1&1&0,050&0,064&0,171&0,143\\
%\hline
1&1&0,100&0,128&0,371&0,299\\
%\hline
2&1&0,025&0,040&0,033&0,031\\
%\hline
2&1&0,050&0,080&0,063&0,061\\
%\hline
2&1&0,100&0,155&0,131&0,125\\
%\hline
2&2&0,025&0,030&0,097&0,087\\
%\hline
2&2&0,050&0,060&0,225&0,202\\
%\hline
2&2&0,100&0,119&0,533&0,468\\
%\hline
3&1&0,025&0,045&0,009&0,009\\
%\hline
3&1&0,050&0,091&0,020&0,021\\
%\hline
3&1&0,100&0,182&0,042&0,046\\
%\hline
3&1&0,251&0,456&0,106&0,110\\
%\hline
3&2&0,025&0,032&0,085&0,078\\
%\hline
3&2&0,050&0,064&0,180&0,165\\
%\hline
3&2&0,100&0,128&0,396&0,355\\
\hline
\end{tabular}
\end{center}
\end{table*}

Пусть $\mu_z$ и $\sigma_z^2$~--- соответственно математическое
ожидание и дисперсия плотности распределения $z(t)$. Имеем
$$
\alpha=\fr{\mu_z^2}{\sigma_z^2}\,;\quad
\beta=\fr{\mu_z}{\sigma_z^2}\,.
$$ 
Положим
\begin{equation} %align}
\mu_K=\fr{\int\limits_0^\infty tK(t)\,dt}{\int\limits_0^\infty K(t)\,dt}\,;\quad %notag\\[-6pt]
%&\label{4us}\\[-6pt]
\mu_u=\fr{\int\limits_0^\infty tu(t)\,dt}{\int\limits_0^\infty
u(t)\,dt} %\notag
\end{equation} %align} 
и
\begin{align}
\sigma_K^2&=\fr{\int\limits_0^\infty(t-\mu_K)^2K(t)\,dt}{\int\limits_0^\infty
K(t)\,dt}\,;\notag\\[-6pt]
&\label{5us}\\[-6pt]
\sigma_u^2&=\fr{\int\limits_0^\infty(t-\mu_u)^2u(t)\,dt}{\int\limits_0^\infty
u(t)\,dt}\,.\notag
\end{align} 
Тогда 
$$
\mu_z=\mu_u-\mu_K\,;\quad 
\sigma_z^2=\sigma_u^2-\sigma_K^2
$$ 
и, следовательно,
\begin{equation} %align}
\alpha=\fr{(\mu_u-\mu_K)^2}{(\sigma_u^2-\sigma_K^2)}\,;\quad %\notag\\[-6pt]
%&\label{6us}\\[-6pt]
\beta=\fr{\mu_u-\mu_K}{(\sigma_u^2-\sigma_K^2)}\,.
\end{equation} %align}
Предположим, что точки $t_1,\ldots ,t_n$ представляют собой
равномерную сетку с шагом $h$, по\-кры\-ва\-ющую интервал, вне которого
функции $K(t)$ и $u(t)$ можно считать равными нулю. Параметры
$\mu_K$, $\mu_u$, $\sigma_K^2$ и $\sigma_u^2$ можно оценить,
заменяя интегралы в~(4) и~(5) соответствующими интегральными
суммами, т.\,е.\ посредством
\begin{equation*} %align*}
\hat\mu_K=\fr{\sum_{i=1}^nt_iK(t_i)}{\sum_{i=1}^nK(t_i)}\,;\quad
\hat\mu_u=\fr{\sum_{i=1}^nt_iu(t_i)}{\sum_{i=1}^nu(t_i)}
\end{equation*} %align*} 
и
\begin{align*}
\hat\sigma_K^2&=\fr{\sum_{i=1}^n(t_i-\hat\mu_K)^2K(t_i)}{\sum_{i=1}^nK(t_i)}\,;\\
\hat\sigma_u^2&=\fr{\sum_{i=1}^n(t_i-\hat\mu_u)^2u(t_i)}{\sum_{i=1}^nu(t_i)}\,.
\end{align*}
Заменяя в~(6) параметры функций $K(t)$ и $u(t)$ их оценками,
получим следующие оценки для $\alpha$ и $\beta$:
\begin{equation*} %align*}
\hat\alpha =\fr{(\hat\mu_u-\hat\mu_K)^2}{(\hat\sigma_u^2-\hat\sigma_K^2)}\,;\quad
\hat\beta =\fr{\hat\mu_u-\hat\mu_K}{(\hat\sigma_u^2-\hat\sigma_K^2)}\,.
\end{equation*} %align*}

Численный эксперимент показывает, что оценки $\hat\alpha$ и
$\hat\beta$ обладают довольно высокой точностью. В~табл.~\ref{t1us}
приведены полученные с помощью моделирования среднеквадратические
отклонения ошибок (вернее, среднеквадратические отклонения,
поделенные, соответственно, на максимальное значение наблюдаемой
функции либо на значение оцениваемой величины) исходных данных и
оценок.


Предполагается, что 
\begin{equation*} %align*}
K(t_j)=K_0(t_j)+\xi_j\,;\quad
u(t_j)=u_0(t_j)+\eta_j\,,
\end{equation*} %align*} 
где $K_0(t)$ и $u_0(t)$~--- точное ядро
и точная правая часть уравнения~(3), а $\xi_1, \ldots , \xi_n, \eta_1,
\ldots , \eta_n$~--- случайные ошибки. Случайные величины $\xi_1, \ldots ,
\xi_n$ так же, как и $\eta_1, \ldots , \eta_n$, являются независимыми
и одинаково распределенными; $\xi_i$ и $\eta_i$ имеют нормальные
распределения с нулевыми математическими ожиданиями и дисперсиями
$\sigma_\xi^2$ и $\sigma_\eta^2$ соответственно. Дисперсии оценок
$\hat\alpha$ и $\hat\beta$ обозначены $\sigma_\alpha^2$ и
$\sigma_\beta^2$. Ядро $K(t)$ является нормальной плотностью
распределения
$$
K(t)=\fr{1}{\sqrt{2\pi}}\,e^{-(t-4)^2/2}$$ (можно
считать, что это распределение сосредоточено на положительной
полупрямой).

Заметим, что в обратных задачах, подобных рассматриваемой, погрешность решения
обычно имеет порядок
$\sqrt\epsilon$, где
$\epsilon$~--- погрешность исходных данных. Для предлагаемых оценок результат
несколько лучше.

           
\section{Непараметрическое оценивание распределения задержки}

Использование параметрической модели удобно с точки зрения оценки
распределения задержки, однако может привести к потере важных
особенностей оцениваемой плотности. Например, все гамма-плот\-ности
являются одновершинными. В~то же время наличие двух или более
локальных максимумов, с одной стороны, вполне вероятно, а с другой
стороны, может отражать существенные особенности изучаемого
взаимодействия. В данном разделе будут кратко рассмотрены
возможности использования непараметрического подхода.

Поскольку основное уравнение~(3) является уравнением свертки 1-го рода, общая схема
реализации непараметрического подхода сводится к применению хорошо развитой техники
решения такого типа уравнений с использованием преобразований Фурье и регуляризации.
Однако наиболее важный этап решения~--- выбор параметра регуляризации~--- предлагается
производить с использованием специфики ситуации. В рас\-смат\-ри\-ва\-емом случае
параметрическая аппроксимация может привести к потере некоторых локальных
особенностей плотности, но позволяет оценить уровень ее гладкости или
погрешность измерений.

Обозначим преобразования Фурье (характеристические функции)
функций $K(t)$, $z(t)$ и $u(t)$ теми же буквами, но с аргументом
$\omega$. Регуляризованное решение имеет вид

\noindent
$$
z_\delta(t)=\fr{1}{2\pi}\int\limits_{-\infty}^\infty
\fr{K(-\omega)u(\omega)e^{-i\omega t}}{|K(\omega)|^2+\delta
M(\omega)}\,d\omega\,,
$$ 
где $M(\omega)$~--- четная неотрицательная
функция такая, что $M(0)\ge0$, $M(\omega)>0$ при $\omega\not=0$ и
$M(\omega)\ge c>0$ при достаточно больших $|\omega|$,
удов\-ле\-тво\-ря\-ющая некоторым условиям регулярности (подробнее см.~\cite{7us}), 
а $\delta$~--- параметр регуляризации (положительное число).
Можно, например, положить 

%\vspace*{-0.25pt}

\noindent
$$
M(\omega)=1-e^{-c\omega^2}\,,
$$ 
где параметр $c$ выбирается из условия, что функции $\exp(-c\omega^2)$
и $K(\omega)$ имеют, грубо говоря, приблизительно одинаковый
размах.

Если бы было известно отклонение $\gamma$ (в некоторой метрике
$\rho(\cdot,\cdot)$) наблюдаемой правой части уравнения~(3) $u(t)$
от точной $u_T(t)$: $\rho(u,u_T)=\gamma$, параметр регуляризации
можно было бы выбирать из условия 

%\vspace*{-0.25pt}

\noindent
$$
\rho(K\ast z_\delta,u)=\gamma\,.
$$ 
Поскольку это отклонение неизвестно, выбор
параметра регуляризации предлагается осуществлять следующим
образом. На первом этапе аппроксимируем $z(t)$ гамма-плотностью и
оцениваем ее параметры $\alpha$ и $\beta$. Для этого используем
метод, описанный в предыдущем разделе. Обозначим через $\hat z(t)$
полученную гамма-плотность. Положим 

\vspace*{2pt}

\noindent
$$
\hat u(t)=\int\limits_0^\infty K(t-s)\hat z(s)\,ds\,.
$$ 

\noindent
Ошибку (в $L^2$) измерения правой части
уравнения~(3) оценим величиной 

\vspace*{2pt}

\noindent
$$
\Delta=\int\limits_0^\infty(\hat u(t)-u(t))^2\,dt\,.
$$ 

\noindent
В качестве параметра регуляризации $\delta$
выбирается решение уравнения
$$
\fr{1}{2\pi}\int\limits_{-\infty}^\infty 
\fr{\delta^2M^2(\omega)|u(\omega)|^2}{(|K(\omega)|+\delta
M(\omega))^2}\,d\omega=\Delta\,,
$$  

\noindent
поскольку левая часть этого уравнения равна

\noindent
\begin{multline*}
\fr{1}{2\pi}\int\limits_{-\infty}^\infty
|K(\omega)z_\delta(\omega)-u(\omega)|^2\,d\omega={}\\
{}=
\int\limits_0^\infty(K\ast z(t)-u(t))^2\,dt\,.
\end{multline*}

Альтернативным путем использования па\-ра\-мет\-ри\-че\-ско\-го старта в непараметрическом
оценивании плотности распределения задержки является предварительное, на основе
параметрической модели, оценивание какого-либо функционала от плотности,
характеризующего уровень сглаженности,\linebreak например полную вариацию.
Параметр регуляризации в этом случае находится из условия равенства указанного
функционала у решения уравнения свертки и оцененного из параметрической
модели.

{\small\frenchspacing
{%\baselineskip=10.8pt
\addcontentsline{toc}{section}{Литература}
\begin{thebibliography}{9}

\bibitem{1us}
\Au{Nowak M.\,A., May~R.\,M.} 
Virus dynamics: Mathematical
principles of immunology and virology.~--- Oxford: Oxford University Press,
 2000.

\bibitem{2us}
\Au{Perelson A.\,S., Neumann A.\,U., Markowitz~M., Leonard~J.\,M., 
Ho~D.\,D.} 
HIV-1 dynamics in vivo: Virion clearance rate,
infected cell life-span, and viral generation time~// Science,
1996. Vol.~271. P.~1582--1586.

\bibitem{3us}
\Au{Herz~A.\,V.\,M., Bonhoeffer~S., Anderson~R.\,M., May~R.\,M.,
Nowak~M.\,A.} 
Viral dynamics in vivo: Limitations on estimates of
intracellular delay and virus decay~// Proc. Natl. Acad. Sci. USA,
1996. Vol.~93(14). P.~7247--7251.

\bibitem{4us}
\Au{Mittler J.\,E., Bernhard~S., Neumann~A.\,U., Perelson~A.\,S.}
Influence of delayed viral production on viral dynamics in HIV-1
infected patients~// Mathematical Biosciences, 1998. Vol.~152. P.~143--163.

\bibitem{5us}
\Au{Nelson P.\,W., Perelson~A.\,S.} 
Mathematical analysis of
delay differential equation models of HIV-1 infection~//
Mathematical Biosciences, 1998. Vol.~179. P.~73--94.

\bibitem{6us}
\Au{MacDonald~N.} 
Biological delay systems: Linear stability
theory.~--- Cambridge: Cambridge University, 1989.

\label{end\stat}

\bibitem{7us}
\Au{Тихонов А.\,Н., Арсенин~В.\,Я.} 
Методы решения некорректных
задач. 2-е изд.~--- М.: Наука, 2002.
\end{thebibliography}

} 
}
\end{multicols}