\def\stat{zeifman}

\def\tit{ОБ ОДНОЙ НЕСТАЦИОНАРНОЙ МОДЕЛИ ОБСЛУЖИВАНИЯ С~КАТАСТРОФАМИ И~ТЯЖЕЛЫМИ ХВОСТАМИ$^*$}

\def\titkol{Об одной нестационарной модели обслуживания с~катастрофами и~тяжелыми хвостами}

\def\aut{А.\,И.~Зейфман$^1$, Я.\,А.~Сатин$^2$,  И.\,А.~Ковалёв$^3$}

\def\autkol{А.\,И.~Зейфман, Я.\,А.~Сатин,  И.\,А.~Ковалёв}

\titel{\tit}{\aut}{\autkol}{\titkol}

\index{Зейфман А.\,И.}
\index{Сатин Я.\,А.}
\index{Ковалёв И.\,А.}
\index{Zeifman A.\,I.}
\index{Satin Ya.\,A.}
\index{Kovalev I.\,A.}

{\renewcommand{\thefootnote}{\fnsymbol{footnote}} \footnotetext[1]
{Исследование выполнено при финансовой поддержке Российского научного
фонда (проект 19-11-00020).}}


\renewcommand{\thefootnote}{\arabic{footnote}}
\footnotetext[1]{Вологодский государственный университет;
Институт проблем информатики Федерального исследовательского центра
<<Информатика и~управление>> Российской академии наук;  Вологодский
научный центр Российской академии наук, \mbox{a\_zeifman@mail.ru}}
\footnotetext[2]{Вологодский государственный университет, \mbox{yacovi@mail.ru}}
\footnotetext[3]{Вологодский государственный университет,
\mbox{kovalev.iv96@yandex.ru}}


%\vspace*{3pt}




\Abst{Рассматривается нестационарная система
массового обслуживания с~катастрофами,  одним сервером и~специальными групповыми 
поступлениями требований, причем интенсивности уве\-ли\-чи\-ва\-ющих\-ся групп требований могут 
убывать достаточно медленно. Рассмотрен процесс~$X(t)$, описывающий число требований в~такой системе,
 доказано существование предельного режима распределения вероятностей состояний и~предельного среднего 
 для~$X(t)$, получены оценки скорости сходимости к~предельному режиму и~предельному среднему. 
 Получены оценки аппроксимации с~помощью усечений конечными процессами. 
 В~качестве примера рассмотрена простая  модель нестационарной системы с~достаточно
  медленной скоростью убывания интенсивностей поступления групп требований, когда размер группы растет.}

\KW{нестационарная система массового обслуживания; счетные
марковские цепи; предельные характеристики; скорость сходимости; аппроксимации}

\DOI{10.14357/19922264210203}

\vspace*{18pt}


\vskip 10pt plus 9pt minus 6pt

\thispagestyle{headings}

\begin{multicols}{2}

\label{st\stat}


\section{Введение}

Первоначальное описание исходной модели и~первые исследования проведены в~\cite{marin2020}, 
соответствующая нестационарная ситуация была впервые изуче\-на в~\cite{zrsk2020}. 
Аналогичная  модель в~случае наличия катастрофических сбоев системы введена и~изуче\-на 
в~[3]. При этом число требований в~соответствующей системе обслуживания
 описывается процессом $X(t)$ (см.\ далее).
В~указанных работах изуча\-ет\-ся случай, когда интенсивности поступления групп 
требований экспоненциально убывают при увеличении размера группы.
В~настоящей работе исследуется ситуация <<тяжелых хвостов>>,  
когда эти интенсивности убывают со степенной ско\-ростью.
{\looseness=1

}

\section{Описание модели}

Подробное описание модели приведено в~[3], здесь же отметим, что 
рассматриваемый процесс~$X(t)$ является неоднородной марковской \mbox{цепью} с~непрерывным временем, 
счетным пространством состояний $\{ 0, 1, 2, \dots\}$, а~транспонированная матрица 
интенсивностей  $A(t)\hm=(a_{ij}(t))_{i,j=0}^{\infty}$  имеет вид:


\noindent
\begin{multline*}
A(t) =\left(
\begin{matrix}
-\lambda(t) & \mu(t) \!+\! \gamma(t) \\
\la(t)b_1 & - \left (\lambda(t)B_2 \!+\! \mu(t) + \gamma(t) \right )  \\
\la(t)b_2& \la(t)b_2  \\
\la(t)b_3 & \la(t)b_3  \\
\vdots& \vdots \\
\end{matrix}\right.\\[6pt]
\left.
\begin{matrix}
  \gamma(t) &  \gamma(t)&\ldots \\
    \mu(t)&0&\ldots\\
- \left ( \lambda(t)B_3 \!+\! \mu(t) \!+\! \gamma(t) \right )   &        \mu(t)&\ldots\\
 \la(t)b_3      &  - \left ( \lambda(t)B_4 \!+\! \mu(t) \!+\! \gamma(t) \right )&\ldots  \\
       \vdots &   \vdots &\ddots 
       \end{matrix}
     \right)\!.\hspace*{-8.6pt}
%\label{2.01}
\end{multline*}
Здесь $\lambda(t)$~--- интенсивность поступления группы требований; $\mu(t)$~--- 
интенсивность обслуживания одного требования; $\gamma(t)$~--- интенсивность катастрофы 
(одномоментной потери всех требований в~сис\-те\-ме). 

Далее, $\lambda(t)b_k$~--- 
интенсивность поступления группы требований такой, что общее их число в~сис\-те\-ме 
оказывается равным~$k$, т.\,е.\ если в~сис\-те\-ме уже есть $k-j$ требований, 
то это  интенсивность одновременного поступления группы из~$j$~требований. При этом 
$B_k\hm=\sum\nolimits_{j=k}^{\infty}b_j$ при всех $k \hm\ge 1$, а~$B_1\hm=1$. 
Все функции, определяющие интенсивности, предполагаются неотрицательными и~локально 
ин-\linebreak\vspace*{-12pt}

\pagebreak

\noindent
тегрируемыми на $[0,\infty)$.
Кроме того, разумеется, предполагается выполненным условие
\begin{equation}
\sum_{k \ge 1} B_k = \sum\limits_{k \ge 1} k b_k < \infty. 
\label{tail01}
\end{equation}


\section{Получение оценок для~процесса~$X(t)$}

Обозначив через $\mathbf{p}(t) \hm=\left(p_0(t), p_1(t),\dots\right)^{\mathrm{T}}$ 
вектор вероятностей состояний для процесса~$X(t)$, получаем прямую систему Колмогорова
$$
\fr{d}{dt}\,\mathbf{p}(t)=A(t)\mathbf{p}(t),
$$
которую в~рассматриваемой ситуации удобно преобразовать к~виду:
\begin{equation}
\fr{d  }{dt}\,{\bf p}(t)=A^*(t) {{\bf p} (t)}  +{\bf g}\left(t\right),  \ t \ge 0,
\label{s2-1}
\end{equation}
где ${\bf g}\left(t\right)\hm=\left(\gamma\left(t\right),0,0, \dots\right)^{\mathrm{T}}$, 
а~$A^*(t)$~--- матрица с~элементами~$a_{ij}^*(t)$,
\begin{equation*}
a_{ij}^*\left(t\right) =
\begin{cases}
a_{0j}\left(t\right) - \gamma\left(t\right) & \mbox { при }  i= 0\,;\\
a_{ij}\left(t\right) & \mbox { в~остальных случаях}.
\end{cases}
%\label{f2}
\end{equation*}


Далее будем предполагать, что найдется $\varepsilon \hm> 0$ такое, что
\begin{equation} 
\int\limits_0^\infty \left(\gamma(t)- \varepsilon \lambda(t)\right) dt = \infty. 
\label{cat01}
\end{equation}



Если, в~частности, интенсивности постоянны, то~(\ref{cat01}) 
выполнено при положительном~$\gamma$, а~если \mbox{1-пе}\-рио\-дич\-ны, то для выполнения~(\ref{cat01}) 
достаточно, чтобы $\int\nolimits_0^1 \gamma(t)\, dt \hm>0$.


Как показано в~[3], выполнение  условия~(\ref{cat01}) 
гарантирует слабую эргодичность~$X(t)$ в~равномерной операторной топологии и~оценку
\begin{multline}
\hspace*{-3mm}\left\|{{\bf p}}^{*}\left(t\right)-{{\bf p}}^{**}
\left(t\right)\right\|\le e^{-\int\nolimits_0^t  \gamma\left(u\right)\, du}\left\|{{\bf p}}^{*}
\left(0\right)-{{\bf p}}^{**}\left(0\right)\right\| \le{}\hspace*{-0.57915pt}\\
{}\le 2 e^{-\int\nolimits_0^t \gamma\left(u\right)\, du}, \ t \ge 0, 
\label{est01}
\end{multline}
справедливую при любых начальных условиях
${{\bf p}}^{*}\left(0\right)$ и~${{\bf p}}^{**}\left(0\right)$.


Однако, как и~в~предыдущих работах, интерес представляет не само наличие предельного режима, 
а~возможность его построения.

Для получения нужных свойств и~оценок потребуются некоторые вспомогательные <<взвешенные>> нормы.

Положим $d_0=1$, и~пусть $\{d_k\}$~--- неубывающая последовательность, $k\hm \ge 0$.
Рассмотрим диагональную мат\-ри\-цу
$\Lambda \hm= \mathrm{diag}\left(d_0, d_1, d_2, \dots  \right)$.

Тогда из~(\ref{s2-1}) получим уравнение:
\begin{equation}
\label{1.03}
\fr{d}{dt}\,\tilde{\mathbf{p}}(t)=\tilde{A}^*(t)\tilde{\mathbf{p}}(t) + {\tilde{\bf g}}\left(t\right),
\end{equation}
где $\tilde{\mathbf{p}}(t)\hm=\Lambda \mathbf{p}(t)$; $\tilde{A}(t)\hm=\Lambda A(t) \Lambda^{-1}$;
${\tilde{\bf g}}(t) \hm= \Lambda{\bf g}(t)$.

Далее будем оценивать логарифмическую норму оператора~$\tilde{A}(t)$.
Если обозначить через $-\tilde{\alpha}_k(t)$ сумму всех элементов $k$-го столбца 
матрицы~$\tilde{A}(t)$, то получим
\begin{align*}
\tilde{\alpha}_0(t) &\ge \gamma(t) -
\lambda(t)\sum\limits_{j=1}^{\infty}\left(\fr{d_j}{d_0}-1\right):=\beta(t);
\\
\tilde{\alpha}_k(t)& \ge \gamma(t) -\lambda(t)\sum\limits_{j=k+1}^{\infty}\left(
\fr{d_j}{d_k}-1\right) \ge \tilde{\alpha}_0(t)=\beta(t),\\
&\hspace*{61mm}k \ge 1\,.
\end{align*}

Из условия~(\ref{tail01}) вытекает, что для любого $\varepsilon \hm>0$ найдется натуральное~$N$ такое, что
\begin{equation*}
\sum\limits_{k \ge N} \left(k-1\right) b_k < \varepsilon. 
%\label{tail02}
\end{equation*}


Положим теперь $d_k\hm=1$, если $k\hm <N$, и~$d_k\hm=k$ при $k \hm\ge N$.


Тогда логарифмическая норма оператора~$\tilde{A}(t)$ равна
\begin{multline*}
-\beta^* (t)=  \sup\limits_i \left \{ \tilde{a}_{ii}(t) + \sum\limits_{j\neq     i}
\tilde{a}_{ji}(t)\right \} =  - \beta(t) \le{}\\
{}\le  -\left(\gamma(t) - \varepsilon \lambda(t)\right).
%\label{f19}
\end{multline*}
Следовательно, вместо~(\ref{est01}) получаем
\begin{multline*}
\hspace*{-0.48921pt}\left\|{\tilde{\mathbf{p}}}^{*}\left(t\right)-
{\tilde{\mathbf{p}}}^{**}\left(t\right)\right\|\le e^{-\int\nolimits_0^t \beta\left(u\right)\, du}
\left\|{\tilde{\mathbf{p}}}^{*}\left(0\right)-{\tilde{\mathbf{p}}}^{**}
\left(0\right)\right\|, \\
 t \ge 0\,. 
%\label{est02}
\end{multline*}

Далее, сравнивая соответствующие нормы и~математические ожидания, получаем такое утверждение.

\smallskip

\noindent
\textbf{Теорема~1.}\ \textit{Пусть выполнены условия}~(\ref{tail01}) 
\textit{и}~(\ref{cat01}). 
\textit{Тогда $X(t)$ слабо эргодичен, имеет предельное среднее и~справедливы 
следующие оценки ско\-рости схо\-ди\-мости}:
\begin{align*}
\left\|{\mathbf{p}}^{*}\left(t\right)-{\mathbf{p}}^{**}\left(
t\right)\right\|&\le e^{-\int\nolimits_0^t \beta\left(u\right)\, du}
\left\|{\tilde{\mathbf{p}}}^{*}\left(0\right)-{\tilde{\mathbf{p}}}^{**}
\left(0\right)\right\|, \\
 &\hspace*{41mm}t \ge 0\,; 
%\label{est03}
\\
|E(t,k) - E(t,0)| &\le kN e^{-\int\nolimits_0^t
    \beta\left(u\right)\, du}, \quad\hspace*{9mm}t \ge 0\,, 
   % \label{est04}
\end{align*}
\textit{где $E(t,j)$~--- математическое ожидание (среднее чис\-ло требований) для $X(t)$
 при условии, что $X(0)\hm=j$}.



\smallskip

Оценим теперь само предельное среднее. Дополнительно предположим выполнение условий
\begin{equation} 
e^{\int\nolimits_s^t \beta(u)\, du} \le Re^{-a(t-s)}, \enskip  \gamma(t) \le \theta, 
\label{cat02}
\end{equation}
для всех $0 \le s \hm\le t$ при некоторых положительных~$R$, $a$ и~$\theta$.


Обозначим через $\tilde{U}(t,s)$ оператор Коши уравнения~(\ref{1.03}). Тогда получим
\begin{equation*}
%\label{1.031}
\tilde{\mathbf{p}}(t)=\tilde{U}(t,0)\tilde{\mathbf{p}}(0) + 
\int\limits_0^t \tilde{U}(t,\tau){\tilde{\bf g}}\left(\tau\right) d\tau,
\end{equation*}
откуда
\begin{equation*}
%\label{1.032}
\limsup\limits_{t \to \infty} \tilde{\mathbf{p}}(t) \le  
\limsup\limits_{t \to \infty} \int\limits_0^t Re^{-a(t-\tau)} \theta \, d\tau \le \fr{R\theta}{a}\,.
\end{equation*}


Тогда имеем


\smallskip

\noindent
\textbf{Следствие~1.}
Пусть выполнены условия~(\ref{tail01}) и~(\ref{cat02}). Тогда при любом~$k$ справедлива следующая оценка:
\begin{equation*}
\limsup\limits_{t \to \infty} E(t,k) \le \fr{NR\theta}{a}\,.
% \label{est05}
\end{equation*}


\section{Аппроксимация усечениями}

В заключение рассмотрим вопрос о построении предельного режима и~предельного 
среднего с~помощью аппроксимации усеченными процессами.
Получение не зависящих от времени оценок при таких аппроксимациях описано в~4, 5].
%\cite{zeifman2014,zeifman-tpa2017-1}.

Аналогично этим работам, будем отождествлять конечные векторы и~счетные 
векторы с~теми же ненулевыми координатами.
Рассмотрим <<усеченную>> матрицу (для краткости зависимость от~$t$ не записываем)

\noindent
%{\footnotesize 
\begin{multline*}
A_K^* =\left( \begin{matrix}
-\lambda B_1^* - \gamma & \mu & 0 & 0 &  \dots \\
\la b_1 & - \left (\lambda B_2^* \!+ \mu + \gamma \right ) & \mu   &  0 & \dots\\
\dots & \dots & \dots & \dots & \dots \\
\la b_K & \la b_K & \dots & \dots & \dots\\
\end{matrix}\right.\\
%\label{2.01}
\left.
\begin{matrix}
\dots& 0 \\
\dots&  0 & \\
\dots&   \dots & \dots & \dots &\dots\\
 \dots&  \dots & \dots & - \left ( \lambda B_K^* + \mu + \gamma \right )
   \end{matrix}\right)\,,
   \end{multline*}

\noindent
где $B_k^*=\sum\nolimits_{j=k}^{K}b_j$.

Запишем аналогичную~(\ref{s2-1}) систему для усеченного процесса в~виде:
\begin{multline*}
\fr{d}{dt}\,{\bf p}_K(t)=A^*(t) {\bf p}_K (t)  +{\bf g}\left(t\right) +{}\\
{}+
\left(A^*_K(t)- A^*(t)\right) {\bf p}_K (t),  \enskip t \ge 0\,.
%\label{s2-11}
\end{multline*}
Тогда

\vspace*{-10pt}

\noindent
\begin{multline*}
%\label{1.032}
{\bf p}_K(t) ={U}(t,0){\mathbf{p}}(0) + \int\limits_0^t {U}(t,\tau){\bf g}\left(\tau\right)\, d\tau + {}\\
{}+\int\limits_0^t {U}(t,\tau)\left(A^*_K(\tau)- A^*(\tau)\right){\bf p}_K (\tau)\, d\tau = {}\\
{}={\mathbf{p}}(t) +  \int\limits_0^t {U}(t,\tau)\left(A^*_K(\tau)- A^*(\tau)\right){\bf p}_K (\tau)\, d\tau , 
\end{multline*}

\vspace*{-2pt}

\noindent
где $U(t,s)$~--- оператор Коши уравнения~(\ref{s2-1}).



Следовательно, в~любой норме справедлива оценка:

\vspace*{-4pt}

\noindent
\begin{multline}
\left\|
\mathbf{p}\left(t\right)-
\mathbf{p}_K\left(t\right)\right\|\le{}\\
{}\le 
\int\limits_0^t \|{U}(t,\tau)\| \|\left(A^*_K(\tau)- A^*(\tau)\right)
{\bf p}_K (\tau)\| \, d\tau . 
\label{est033}
\end{multline}

\vspace*{-4pt}

Рассмотрим норму $\|{\bf x}\|_{\Lambda}\hm=\|\Lambda{\bf x}\|$,
 тогда $\|\tilde{U}(t,s)\|\hm= \|U(t,s)\|_{\Lambda} \hm\le Re^{-a(t-s)}$.


Для оценки второго множителя под знаком интеграла в~(\ref{est033}) 
отметим, что в~левом верхнем квадрате матрицы $A^*_K\hm- A^*$ (где оба индекса не превосходят~$K$) 
ненулевыми являются только диагональные элементы, каждый из которых равен $-\lambda B_K$.
Значит,

\vspace*{-4pt}

\noindent 
\begin{multline*}
\left(A^*_K(\tau)- A^*(\tau)\right){{\bf p}_K (\tau)} = {}\\
{}=
- \lambda (\tau) B_K \left(p_0(\tau), \dots, p_K(\tau)\right)^{\mathrm{T}}.
\end{multline*}

\vspace*{-2pt}

\noindent
А~тогда, предполагая, что $\lambda(t) \hm\le \theta$ при всех~$t$, получаем 

\vspace*{-4pt}

\noindent
\begin{multline*}
\|\left(A^*_K(\tau)- A^*(\tau)\right){{\bf p}_K (\tau)}\|\le{}\\
{}\le  B_K\theta 
\sum\limits_{k \le K}d_kp_k(\tau) \le
B_K N \theta\,.
\end{multline*}


\vspace*{-2pt}

Тогда правая часть в~(\ref{est033}) в~$\Lambda$-нор\-ме 
не превосходит ${B_K N R\theta}/{a}$ и~получаем следующее утверждение.

\medskip

\noindent
\textbf{Теорема~2.} %\begin{theorem} 
\textit{Пусть выполнены условия}~(\ref{tail01}) \textit{и}~(\ref{cat02}). \textit{Тогда при $X(0)\hm=0$ 
справедливы следующие оценки}:

\vspace*{-4pt}

\noindent
\begin{align*}
\left\|{\mathbf{p}}\left(t\right)-{\mathbf{p}}_K\left(t\right)\right\|&\le 
\fr{B_K N R\theta}{a} \to 0  \mbox { при }  k \to \infty\,; 
%\label{est039}
\\[-1pt]
|E(t,0) - E_K(t,0)| &\le \frac{B_K N^2 R\theta}{a} 
 \to 0  \mbox { при }  k \to \infty. 
 %\label{est049}
\end{align*}





\section{Численный пример}

\vspace*{-2pt}

Рассмотрим описанную модель с~интенсивностями
$\gamma\hm=1$, %\enskip  
$\lambda(t)\hm=1\hm+\sin 2\pi t$ %\enskip 
и~$\mu(t)\hm=1\;+$\linebreak $+\;\cos 2\pi t$,
предполагая при этом, что  
%\pagebreak
%\noindent
$b_k\hm={4}/((k(k\;+$\linebreak $+\;1)(k\hm+2)))$,
%$$
т.\,е.\ убывание имеет степенной\linebreak  характер.

\pagebreak

\end{multicols}

\begin{figure*} %fig1-2
 \vspace*{1pt}
  \begin{center}
    \mbox{%
 \epsfxsize=160.453mm 
 \epsfbox{zei-1.eps}
 }
\end{center}

\vspace*{-18pt}
 \begin{minipage}[t]{80mm}
%\includegraphics[height=9cm]{marinaEx3n100p0.pdf}
\Caption{Поведение вероятности $p_0(t)$ для усеченного процесса, интервал  $[0,20]$:
(\textit{а})~100 состояний; (\textit{б})~200; (\textit{в})~300~состояний;
\textit{1}~--- $x(0)\hm=0$; \textit{2}~--- $x(0)=100$}
\end{minipage}
%\end{figure*}
\hfill
%\begin{figure*} %fig2
%\vspace*{1pt}
 % \begin{center}
 %   \mbox{%
 %\epsfxsize=159.209mm 
% \epsfbox{zei-2.eps}
 %}
%\end{center}
\vspace*{-18pt}
 \begin{minipage}[t]{80mm}
%\includegraphics[height=9cm]{marinaEx3n100Shirtp0.pdf}
\Caption{Предельная вероятность $p_0(t)$ для усеченного процесса, интервал  $[20,21]$:
(\textit{а})~100 состояний; (\textit{б})~200; (\textit{в})~300~состояний;
\textit{1}~--- $x(0)\hm=0$; \textit{2}~--- $x(0)=100$}
\end{minipage}
\vspace*{16pt}
\end{figure*}

%\vspace*{-18pt}

\begin{multicols}{2}

На рис.~1--4 показано поведение вероятности отсутствия требований в~системе~$p_0(t)$ и~среднего
 числа требований в~системе $E(t,k)$ для усеченных процессов с~числом состояний~100, 200 и~300. 

Можно  отметить, что погрешность вектора вероятностей состояний при усечениях, 
 соответствующих $K\hm=100$ и~$200$, получается $1{,}2\cdot10^{-2}$ и~$1{,}2\cdot10^{-3}$ 
 соответственно, а~для средних~--- $7{,}2\cdot10^{-2}$ и~$7{,}2\cdot 10^{-3}$ соответственно.
 
 %\columnbreak
 



{\small\frenchspacing
{ %\baselineskip=10.8pt
%\addcontentsline{toc}{section}{References}
\begin{thebibliography}{9}

\vspace*{-12pt}

\bibitem{marin2020}
\Au{Marin A.,  Rossi~S.} A~queueing model that works only on the biggest jobs~// 
16th European   Computer Performance Engineering Workshop Revised Selected Papers~/
Eds. M.~Gribaudo, M.~Iacono, T.~Phung-Duc, R.~Razumchik.~--- 
Lecture notes in computer science ser.~--- Springer, 2020. Vol.~12039. P.~118--132.

\bibitem{zrsk2020}
\Au{Zeifman A.\,I., Razumchik R.\,V., Satin Y.\,A.,  Kovalev I.\,A.} 
Ergodicity bounds for the Markovian queue with time-\linebreak\vspace*{-12pt}

\end{thebibliography}

}
}



\end{multicols}

\pagebreak

\begin{figure*} %fig3-4
\vspace*{1pt}
  \begin{center}
    \mbox{%
 \epsfxsize=157.626mm 
 \epsfbox{zei-3.eps}
 }
\end{center}
\vspace*{-21pt}
 \begin{minipage}[t]{80mm}
%\includegraphics[height=9cm]{marinaEx3n100Mean.pdf}
\Caption{Поведение среднего  $E(t,k)$ для усеченного процесса, интервал  $[0,20]$:
(\textit{а})~100 состояний; (\textit{б})~200; (\textit{в})~300~состояний;
\textit{1}~--- $x(0)\hm=0$; \textit{2}~--- $x(0)=100$}
\end{minipage}
%\end{figure*}
\hfill
%\begin{figure*} %fig4
%\vspace*{1pt}
 % \begin{center}
 %   \mbox{%
 %\epsfxsize=79mm 
% \epsfbox{zei-4.eps}
% }
%\end{center}
\vspace*{-21pt}
 \begin{minipage}[t]{80mm}
%\includegraphics[height=9cm]{marinaEx3n100ShirtMean.pdf}
\Caption{Предельное среднее  $E(t,k)$ для усеченного процесса, интервал  $[20,21]$:
(\textit{а})~100 состояний; (\textit{б})~200; (\textit{в})~300~состояний;
\textit{1}~--- $x(0)\hm=0$; \textit{2}~--- $x(0)=100$}
\end{minipage}
\vspace*{16pt}
%\vspace*{138pt}
\end{figure*}
    

\begin{multicols}{2}

{\small 
\begin{enumerate}[1.]
\setcounter{enumi}{2}
%\vspace*{150pt}

\item[{}]
varying
 transition intensities, 
batch arrivals and one queue skipping policy~// Appl. Math. Comput., 2021. Vol.~395. Art.~125846.
{ %\looseness=1

}

%\bibitem{Zeifmanmdpi2021}
\item
\Au{Zeifman A., Satin~Y., Kovalev~I., Razumchik~R.,  Korolev~V.} 
Facilitating numerical solutions of inhomogeneous continuous time Markov chains 
using ergodicity bounds obtained with logarithmic norm method~// Mathematics, 2021. Vol.~9. Iss.~1. 
Art.~42. 20~p.


\columnbreak

%\bibitem{zeifman2014}
\item
\Au{Zeifman A., Satin~Y., Korolev~V.,  Shorgin S.} 
On truncations for weakly ergodic inhomogeneous birth and death processes~// Int. J.~Appl. Math. 
Comp., 2014. Vol.~24. No.\,3. P.~503--518.\\[-14pt]

%\bibitem{zeifman-tpa2017}
\item
\Au{Зейфман А.\,И., Коротышева~А.\,В., Королев~В.\,Ю., Сатин~Я.\,А.}
Оценки погрешности аппроксимаций неоднородных марковских цепей с~непрерывным временем~//
 Теория вероятностей и~ее применения, 2016. Т.~61. №\,3. С.~563--569.
\end{enumerate}
}
 
 
 

\end{multicols}

\vspace*{-9pt}

\hfill{\small\textit{Поступила в~редакцию 07.03.2021}}

%\vspace*{8pt}

%\pagebreak

\newpage

\vspace*{-28pt}

%\hrule

%\vspace*{2pt}

%\hrule

%\vspace*{-2pt}

\def\tit{ON ONE NONSTATIONARY SERVICE MODEL WITH~CATASTROPHES AND~HEAVY TAILS}


\def\titkol{On one nonstationary service model with~catastrophes and~heavy tails}

\def\aut{A.\,I.~Zeifman$^{1,2,3}$, Ya.\,A.~Satin$^1$, and~I.\,A.~Kovalev$^1$}

\def\autkol{A.\,I.~Zeifman, Ya.\,A.~Satin, and~I.\,A.~Kovalev}


\titel{\tit}{\aut}{\autkol}{\titkol}

\vspace*{-11pt}


\noindent
$^1$Department of Applied Mathematics, Vologda State University, 15~Lenin Str., 
Vologda 160000, Russian Federation


\noindent
$^2$Institute of Informatics Problems, Federal Research Center ``Computer Science and Control'' 
of the Russian\linebreak
$\hphantom{^1}$Academy of Sciences, 44-2~Vavilov Str., Moscow 119133, Russian Federation

\noindent
$^3$Vologda Research Center of the Russian Academy of Sciences, 56A~Gorky Str., Vologda 160014, Russian\linebreak
$\hphantom{^1}$Federation
 
\def\leftfootline{\small{\textbf{\thepage}
\hfill INFORMATIKA I EE PRIMENENIYA~--- INFORMATICS AND
APPLICATIONS\ \ \ 2021\ \ \ volume~15\ \ \ issue\ 2}
}%
\def\rightfootline{\small{INFORMATIKA I EE PRIMENENIYA~---
INFORMATICS AND APPLICATIONS\ \ \ 2021\ \ \ volume~15\ \ \ issue\ 2
\hfill \textbf{\thepage}}}

\vspace*{3pt}



\Abste{The paper considers the nonstationary queuing system with catastrophes, 
one server, and special group arrivals of requests. The intensities of increasing
 groups of requests can decrease rather slowly. The process $X(t)$, which describes 
 the number of requirements in such system, is considered, the existence of a~limiting
  regime of the probability distribution of states and a~limiting average for $X(t)$ 
  is proved, and estimates of the rate of convergence to the limiting regime and
   the limiting average are obtained. Approximation estimates are obtained using truncations
    by finite processes. As an example, the authors consider a~simple model of a~nonstationary 
    system with a~rather slow rate of decrease in the arrival rates of customer groups when 
    the group size grows.}

\KWE{nonstationary queuing system; countable Markov chains; limiting characteristics; 
rate of convergence; approximation}

\DOI{10.14357/19922264210203}

\vspace*{-15pt}

 \Ack
\noindent
This work was financially supported by the Russian Science Foundation (grant No.\,19-11-00020).

%\vspace*{12pt}

  \begin{multicols}{2}

\renewcommand{\bibname}{\protect\rmfamily References}
%\renewcommand{\bibname}{\large\protect\rm References}

{\small\frenchspacing
 {%\baselineskip=10.8pt
 \addcontentsline{toc}{section}{References}
 \begin{thebibliography}{9}
\bibitem{1-zei-1}
\Aue{Marin, A., and S.~Rossi.} 2020. A~queueing model that works only on the biggest jobs. 
\textit{16th European Computer Performance Engineering Workshop Revised Selected Papers}. 
Eds. M.~Gribaudo, M.~Iacono, T.~Phung-Duc, and R.~Razumchik. 
Lecture notes in computer science ser. Springer. 12039:118--132.
\bibitem{2-zei-1}
\Aue{Zeifman, A.\,I., R.\,V.~Razumchik, Y.\,A.~Satin, and I.\,A.~Kovalev.}
 2021. Ergodicity bounds for the Markovian queue with time-varying transition intensities, 
 batch arrivals and one queue skipping policy.  \textit{Appl. Math. Comput.} 395:125846. 11~p.
\bibitem{3-zei-1}
\Aue{Zeifman, A., Y.~Satin, I.~Kovalev, R.~Razumchik, and V.~Korolev.}
 2021. Facilitating numerical solutions of inhomogeneous continuous time Markov 
 chains using ergodicity bounds obtained with logarithmic norm method. 
  \textit{Mathematics} 9(1):42. 20~p.
\bibitem{4-zei-1}
\Aue{Zeifman, A., Y.~Satin, V.~Korolev, and S.~Shorgin.}
 2014. On truncations for weakly ergodic inhomogeneous birth and death processes. 
  \textit{Int. J.~Appl. Math. Comp.} 24(3):503--518.
\bibitem{5-zei-1}
\Aue{Zeifman, A.\,I., A.\,V.~Korotysheva, V.\,Y.~Korolev, and Ya.\,A.~Satin.}
 2017. Truncation bounds for approximations of inhomogeneous continuous-time Markov chains. 
 \textit{Theor. Probab. Appl.} 61(3):513--520.
 
 \end{thebibliography}

 }
 }

\end{multicols}

\vspace*{-9pt}

  \hfill{\small\textit{Received March~7, 2021}}


%\pagebreak

\vspace*{-14pt}  

\Contr

\vspace*{-2pt}

\noindent
\textbf{Zeifman Alexander I.} (b.\ 1954)~--- Doctor of Science in physics and mathematics,
 professor, Head of Department, Vologda State University, 15~Lenin Str., 
 Vologda 160000, Russian Federation; senior scientist, Institute of Informatics Problems, 
 Federal Research Center ``Computer Science and Control'' 
 of the Russian Academy of Sciences, 44-2~Vavilov Str., Moscow 119133, Russian Federation; 
 principal scientist, Vologda Research Center of the Russian Academy of Sciences, 56A~Gorky Str., 
 Vologda 160014, Russian Federation; \mbox{a\_zeifman@mail.ru}

\vspace*{3pt}

\noindent
\textbf{Satin Yacov A.} (b.\ 1978)~---  
Candidate of Science (PhD) in physics and mathematics, associate professor, Department of Applied Mathematics, 
Vologda State University, 15~Lenin Str., Vologda 160000, Russian Federation; \mbox{yacovi@mail.ru}
 
\vspace*{3pt}

\noindent
\textbf{Kovalev Ivan A.} (b.\ 1996)~---  
PhD student, Department of Applied Mathematics, Vologda State University, 15~Lenin Str., Vologda 160000, 
Russian Federation;  \mbox{kovalev.iv96@yandex.ru}


\label{end\stat}

\renewcommand{\bibname}{\protect\rm Литература}