\def\stat{andrianova}

\def\tit{СТОХАСТИЧЕСКАЯ ДИНАМИКА САМООРГАНИЗУЮЩИХСЯ СОЦИАЛЬНЫХ СИСТЕМ 
С~ПАМЯТЬЮ\\ (ЭЛЕКТОРАЛЬНЫЕ ПРОЦЕССЫ)}

\def\titkol{Стохастическая динамика самоорганизующихся социальных систем 
с~памятью (электоральные процессы)}

\def\aut{А.\,С.~Сигов$^1$, Е.\,Г.~Андрианова$^2$, % Т.\,Ю.~Хватова$^3$, 
%Д.\,О.~Жуков$^4$, 
Л.\,А.~Истратов$^3$}

\def\autkol{А.\,С.~Сигов, Е.\,Г.~Андрианова, Л.\,А.~Истратов}
%Т.\,Ю.~Хватова и~др.}
%$^3$, Д.\,О.~Жуков$^4$}

\titel{\tit}{\aut}{\autkol}{\titkol}

\index{Сигов А.\,С.}
\index{Андрианова Е.\,Г.}
%\index{Хватова Т.\,Ю.} 
%\index{Жуков Д.\,О.}
\index{Истратов Л.\,А.}
\index{Sigov A.\,S.}
\index{Andrianova E.\,G.}
%\index{Khvatova T.\,Yu.}
%\index{Zhukov D.\,O.}
\index{Istratov L.\,A.}

%{\renewcommand{\thefootnote}{\fnsymbol{footnote}} \footnotetext[1]
%{Работа выполнена при финансовой поддержке РФФИ (проект 16-29-09458~офи\_м).}}


\renewcommand{\thefootnote}{\arabic{footnote}}
\footnotetext[1]{МИРЭА~--- Российский технологический университет (РТУ МИРЭА), sigov@mirea.ru}
\footnotetext[2]{МИРЭА~--- Российский технологический университет (РТУ МИРЭА), andrianova@mirea.ru}
%\footnotetext[3]{Emlyon Business School, Entrepreneurship \& Innovation Research Center, France, \mbox{khvatova@em-lyon.com}}
%\footnotetext[4]{МИРЭА~--- Российский технологический университет (РТУ МИРЭА), zhukovdm@yandex.ru}
\footnotetext[3]{МИРЭА~--- Российский технологический университет (РТУ МИРЭА), istratov@mirea.ru}


\vspace*{-2pt}
    

     \Abst{Обсуждаются вопросы применения методологии и~подходов теоретической 
информатики для анализа и~моделирования социальных групповых процессов. Обработка 
социологических данных электоральной кампании выборов президента США в~2016~г.\ 
позволила построить гистограммы плот\-ности вероятности амплитуд отклонений 
предпочтений избирателей в~зависимости от величины интервала времени их определения 
и~разработать модель описания стохастических социальных процессов с~учетом 
самоорганизации и~наличия памяти, учитывающую основные характеристики наблюдаемых 
процессов. При создании модели рассмотрены схемы вероятностей переходов между 
возможными состояниями социальной системы и~выведено нелинейное дифференциальное 
уравнение второго порядка. Сформулирована и~решена граничная задача для определения 
функции плотности вероятности амплитуды отклонений предпочтений избирателей 
в~зависимости от величины интервала времени ее определения. Дифференциальное 
уравнение модели содержит член, отвечающий за возможность самоорганизации, а~также 
учитывает наличие памяти. Возможность возникновения осцилляций определяется 
начальными условиями. Разработанную модель можно использовать для анализа 
электоральных кампаний и~принятия решений.}
      
     \KW{функция распределения амплитуд колебаний; стохастическая динамика; 
самоорганизация; наличие памяти; осцилляции плотности вероятности; электоральные 
процессы}

\DOI{10.14357/19922264210216}

\vspace*{2.5pt}


\vskip 10pt plus 9pt minus 6pt

\thispagestyle{headings}

\begin{multicols}{2}

\label{st\stat}
     
      
\section{Введение}

\vspace*{-3pt}

    Человеческий фактор в~социальных системах активно влияет на 
происходящие явления, внося неопределенность воздействием на протекающие 
процессы и~создавая возможности для самоорганизации систем. Возможность 
появления памяти о предыдущих состояниях системы и~оказанных на нее 
воздействиях приводит к~существенно нелинейной динамике процессов 
в~социальных системах. Применение методов и~средств теоретической 
информатики и~кибернетики для моделирования динамики 
самоорганизующихся социальных систем с~памятью позволяет получить 
качественно новые результаты для описания социальных сис\-тем. 

\vspace*{-9pt}
    
\section{Обзор исследований по~анализу и~моделированию 
социальных процессов}


\vspace*{-3pt}
 
    Социальные процессы характеризуются сложными механизмами 
протекания и~стохастичностью. Различные множественные состояния зависят 
от влияния друг на друга участников процесса~[1, 2]. Изначально 
теоретические подходы к~описанию социальных систем имели много общего 
с~кинетическим описанием физических систем~[3, 4]. Эти модели актуальны, 
однако наблюдения процессов в~социальных сетях показали ограничения 
кинетических моделей. В~част\-ности, кинетические модели фокусируются на 
появлении мгновенных глобальных каскадов, инициированных одиночными 
локальными возмущениями. 

Известны примеры, когда главную роль играют 
пороговые механизмы развития процессов в~социальных сис\-те\-мах. Состояния 
узлов могут зависеть от внеш\-них импульсов, например из средств массовой 
информации~[5], опре\-де\-ля\-ющих стохастическую со\-став\-ля\-ющую процессов. 

\begin{figure*}[b] %fig1
\vspace*{-6pt}
\begin{center}
 \mbox{%
 \epsfxsize=163mm 
\epsfbox{and-1.eps}
 }
 \end{center}
\vspace*{-9pt}
\Caption{Гистограммы плотности вероятности амплитуд отклонений предпочтений 
избирателей для различных интервалов времени их расчета: (\textit{а})~1~день; 
(\textit{б})~10~дней}
\end{figure*}

Более поздние модели описания процессов в~социальных сетях используют 
стохастические подходы, учитывающие зависимости состояний от  
вре\-мени~[6--8]. В~работе~\cite{6-sig} рассматривается модель\linebreak смешанного 
членства в~стохастически фор\-ми\-ру\-ющих\-ся группах, основанная на 
рассмотрении попарных измерений, таких как присутствие или отсутствие 
связей между парой объектов. Данная \mbox{модель} позволяет при определенных 
допущениях отследить динамику изменения численности членов 
в~формирующихся группах и~кластеризацию членов по группам. 
В~работах~\cite{7-sig, 8-sig} групповые социальные процессы рассматриваются 
с~позиций теории перколяции, что позволяет учитывать структуру сети. 
В~част\-ности, было исследовано влияние плотности сети на величину порога 
ее перколяции (проводимости сети в~целом) и~динамику его достижения. Для 
моделирования поведения участников социальных процессов используют 
теорию многоагентных систем. На основании некоторых правил переходов 
агенты принимают определенные состояния, образуют связанную по своим 
свойствам группу, могут сотрудничать, чтобы решить некую задачу или 
достигнуть определенной цели~\cite{7-sig}, а~поведение агентов может 
зависеть от динамически меняющихся условий~\cite{9-sig}. 
    
    В работе~[10] использована теория клеточных\linebreak автоматов для описания 
социальной системы, пове\-де\-ние которой зависит от свойств внешней среды 
и~структуры поведения. Поведение соци\-аль\-ной системы описывают четыре 
параметра: разнообразие; связность; взаимозависимость; адап\-ти\-ру\-емость. При 
увеличении взаимозависимости и~адап\-тив\-ности поведение сис\-те\-мы становится 
более упорядоченным и~целенаправленным. Считаем, что дальнейшая 
разработка моделей поведения участников социальных процессов является 
актуальной задачей. Рассмотрена электоральная кампания как стохастический 
динамический процесс перехода между возможными состояниями системы 
с~течением времени. Изменения состояния могут иметь не только случайный 
характер, но и~учитывать возможные процессы самоорганизации, наличие 
памяти и~осцилляции.
    
    Более подробное описание математических моделей динамики процессов 
в~сложных социальных и~экономических системах можно найти в~обзоре~[11] 
и~ряде других оригинальных работ~[12--22].



\section{Выбор данных для~анализа динамики социальных 
процессов и~получение гистограмм статистических распределений 
и~их моментов}

    Для создания и~проверки модели требуется значительный объем 
наблюдаемых данных. Большая статистически достоверная база данных по 
электоральным процессам доступна на ресурсе {\sf 
http:// www.realclearpolitics.com}. Анализ динамики изменения настроений 
избирателей в~ходе предвыборных кампаний и~прогнозирование на ее основе 
итогов представляет огромный интерес. Поэтому для анализа и~разработки 
модели выбраны электоральные процессы. Для обработки наблюдаемых 
данных (изменения процентов предпочтения избирателей в~США на 
протяжении~500~дней, с~1~июля 2015~г.\ по~7~ноября 2016~г.)\ 
и~определения функций плотности вероятности амплитуд колебаний 
предпочтений избирателей использован следующий алгоритм. 
    \begin{enumerate}[1.]
    \item Выбираем все значения исследуемых предпочтений избирателей за 
некоторый диапазон времени (сутки, неделя, месяц и~т.\,д.), вычисляем 
значения амплитуды изменения величины колебаний предпочтений 
избирателей за различные интервалы времени. 
\end{enumerate}

{ \begin{center}  %fig2
 \vspace*{-2pt}
   \mbox{%
 \epsfxsize=78.898mm 
\epsfbox{and-2.eps}
 }

\end{center}

\noindent
{{\figurename~2}\ \ \small{
Зависимости математического ожидания~(\textit{а}) 
и~дисперсии~(\textit{б}) амплитуд отклонений предпочтений избирателей
}}}

\vspace*{9pt}

\setcounter{figure}{2}

 

\begin{enumerate}[1.]
\setcounter{enumi}{1}
    \item Полученные для каждого из расчетных интервалов времени 
значения амплитуд сортируем в~порядке возрастания и~для каждого из 
интервалов строим гистограммы плотности распределения амплитуд (рис.~1).
    \item По полученным гистограммам для каждого из интервалов времени 
расчета амплитуд вы\-чис\-ля\-ем моменты распределений (среднее значение~--- 
математическое ожидание, дисперсия), проводим построение их зависимостей 
от величины интервала времени расчета амплитуды. 
    \end{enumerate}
    
    Визуальный анализ данных (см.\ рис.~1) показывает, что для небольших 
интервалов времени расчета амплитуд колебаний предпочтений избирателей 
(один день) гистограммы имеют большой центральный пик вблизи нулевого 
значения (вероятность около~0,60), а амплитуды с~большими значениями 
имеют маленькую вероятность. При увеличении интервала времени 
определения амплитуды колебаний предпочтений избирателей центральный 
пик уменьшается, ширина распределения увеличивается, появляются 
осцилляции.
    


    На рис.~2 приведены графики зависимостей математического ожидания 
и~дисперсии амплитуд отклонений предпочтений избирателей от величины 
интервала времени, для которого они были рассчитаны по наблюдаемым 
данным (см.\ рис.~1). Наблю\-да\-емые процессы носят сложный характер. 
    
   
     
\section{Модель стохастической динамики формирования состояний 
социальных систем с~учетом процессов самоорганизации 
и~наличия~памяти} 

\subsection{Вывод основного уравнения модели}
 
    Все множество возможных величин амплитуд отклонений предпочтений 
избирателей для любой величины интервала времени~$t$ 
обозначим~$\mathrm{X}$. Считаем, что интервал времени~$t$ состоит из 
малых частей~$\tau$. Тогда любое значение интервала времени~$t$ 
представим как $t_h\hm=h\tau$, где~$h$~--- номер шага ($h\hm=0, 1, 2,\ldots 
, N$). Величину амплитуды для выбранного интервала времени~$t$ 
обозначим~$x_h$ ($x_h\hm\in \mathrm{X}$). Анализ наблюдаемых величин 
амплитуд (см.\ рис.~1) показывает, что~$x_h$ могут иметь положительные 
и~отрицательные значения. Предположим, что значение амплитуды~$x_h$ при 
изменении дискретного времени~$h$ на единицу может увеличиваться на 
некоторую малую величину~$\varepsilon$ или уменьшаться на 
величину~$\xi$. Найдем вероятность ${\sf P}(x,h)$ того, что величина амплитуды 
отклонений предпочтений избирателей для некоторого интервала дискретного 
времени~$h$ окажется равна~$x$. Пусть ${\sf P}(x-\varepsilon,h-1)$~--- вероятность 
того, что для некоторого ($h\hm-1$) амплитуда имела величину ($x \hm -
\varepsilon$); ${\sf P}(x+\xi,h-1)$~--- вероятность того, что для некоторого ($h\hm-1$) 
амплитуда имеет величину ($x\hm+\xi$); ${\sf P}(x,h-1)$~--- вероятность того, что 
для некоторого ($h\hm-1$) амплитуда имеет величину~$x$. 
Вероятность~${\sf P}(x,h)$ того, что величина амплитуды отклонения предпочтений 
избирателей для интервала дискретного времени~$h$ окажется равной~$x$ 
(рис.~3), можно определить по формуле:
    \begin{multline*}
    {\sf P}(x, h)={}\\
  {}={\sf P}(x-\varepsilon, h-1)+{\sf P}(x+\xi, h-1)-{\sf P}(x, h-1).
    %\label{e1-sig}
    \end{multline*}



    Человеческий фактор, внося неопределенность воздействием на процессы и~создавая стохастичность, открывает возможности для самоорганизации 
и~определяет наличие памяти о предыдущих\linebreak\vspace*{-12pt}

{ \begin{center}  %fig3
 \vspace*{-2pt}
   \mbox{%
 \epsfxsize=79mm 
\epsfbox{and-3.eps}
 }

\end{center}

\noindent
{{\figurename~3}\ \ \small{
Схема переходов между величинами амплитуды при изменении~$h$ на~1
}}}

\vspace*{9pt}

\setcounter{figure}{3}


\noindent
 действиях. Для учета памяти 
определим вероятности ${\sf P}(x-\varepsilon, h)$, ${\sf P}(x+\xi, h)$ и~${\sf P}(x, h)$ через 
состояния на предыдущем, ($h\hm-1$)-м, шаге. Схемы соответствующих 
переходов можно изобразить аналогично схеме на рис.~3. Получаем для 
вероятности пе\-ре\-хода:
    \begin{multline*}
    {\sf P}(x, h+2) ={}\\
    {}=\left\{ {\sf P}(x-2\varepsilon,h)+{\sf P}(x-\varepsilon+\xi,h) -{\sf P}(x-
\varepsilon,h)\right\} +{}\\
{}+\left\{ {\sf P}(x+\xi-\varepsilon,h)+
    {\sf P}(x+\xi,h)-{\sf P}(x+\xi, h)\right\} -{}\\
    {}-{\sf P}(x-\varepsilon, h)-{\sf P}(x+\xi, h-1)+ {\sf P}(x,h)\,.
   % \label{e2-sig}
    \end{multline*}
    
    Далее, учитывая, что $t\hm=h\tau$, перейдем от~$h$ к~$t$, затем разложим в~ряд Тейлора:
    \begin{equation}
    \fr{d{\sf P}(x,t)}{dt} =a\fr{d^2{\sf P}(x,t)}{dx^2} -b\fr{d{\sf P}(x,t)}{dx}-
c\fr{d^2{\sf P}(x,t)}{dt^2}\,,
    \label{e3-sig}
    \end{equation}
где 
$$
a=\fr{\varepsilon^2-\varepsilon\xi+\xi^2}{\tau}\,;\enskip
b=\fr{\varepsilon-\xi}{\tau}\,;\enskip c=\tau\,.
$$
    
    Член уравнения вида $d{\sf P}(x,t)/dx$ описывает упорядоченный переход либо 
в~состояние, когда оно увеличивается ($\varepsilon\hm>\xi$), либо когда оно 
уменьшается ($\varepsilon\hm<\xi$); член уравнения вида $d^2{\sf P}(x,t)/dx^2$ 
описывает случайное изменение со\-сто\-яния (неопределенность изменения). 
Член уравнения вида $d{\sf P}(x,t)/dt$ определим как ско\-рость общего изменения 
со\-сто\-яния сис\-те\-мы с~течением времени; член уравнения вида $d^2{\sf P}(x,t)/dt^2$ 
описывает процесс, при котором состояния сами становятся источниками 
возникновения других со\-сто\-яний (\textit{самоорганизация} и~ускорение как 
упорядоченных ($d{\sf P}(x,t)/dx$), так и~случайных ($d^2{\sf P}(x,t)/dx^2$) переходов). 

\subsection{Формулировка и~решение граничной задачи для~нахождения 
функции распределения амплитуд отклонений предпочтений избирателей}
 
    Считая функцию ${\sf P}(x, t)$ непрерывной, пе\-рей\-дем от вероятности ${\sf P}(x, 
t)$~(\ref{e3-sig}) к~плотности вероятности $\rho(x, t)\hm=d{\sf P}(x, t)/dx$ 
и~сформулируем граничную задачу для нахождения за\-ви\-си\-мости \mbox{плотности} 
вероятности наблюдения различных величин амп\-ли\-туд отклонений 
предпочтений избирателей за произвольный интервал времени~$t$. Анализ 
статистических данных показывает, что вероятности наблюдения больших 
величин амплитуд отклонений предпочтений избирателей в~течение 
рассматриваемых интервалов времени ничтожно малы. Можно предположить, 
что функция плот\-ности ве\-ро\-ят\-ности быст\-ро убывает, и~задать граничные 
условия: 
    \begin{equation*}
    \rho(x,t)_{x=\infty} =0\,;\quad
    \rho(x,t)_{x=-\infty} =0\,.
    \end{equation*}
    
    Первое начальное условие зададим в~виде дель\-та-функ\-ции, исходя из 
того, что для интервала времени $t\hm = 0$ возможно только значение 
амплитуды $x_0\hm=0$: 
    $$
    \rho(x,t)\vert_{t=0} =\delta(x-0) =\begin{cases}
    \displaystyle \int \delta(x-0)\,dx=1\,, &\! x=0\,;\\
    0\,, &\! x\not= 0\,.
    \end{cases}
    $$
    
    Второе начальное условие $(\partial \rho(x,t)/\partial t)\vert_{t=0}$ задает 
скорость изменения плотности вероятности для любого значения амплитуды. 
Сложение множества различных типов поведения избирателей ведет к~тому, 
что некоторые амплитуды могут усиливаться, а некоторые ослабевать. 
В~конечном итоге это приводит к~периодичности для некоторых значений 
амплитуд, т.\,е.\ возникновению волн. Так как $\Delta t\hm\to \tau$ (по условиям 
протекания процессов), то второе начальное условие можно записать в~виде: 
    \begin{multline*}
    \left.\fr{\partial \rho(x,t)}{\partial t}\right\vert_{t=0} ={}\\
    {}=
    \lim\limits_{\Delta t \to\tau} 
    \left. \fr{\rho(x+\Delta x, t+\Delta t)-\rho(x,t)}{\Delta t}\right\vert_{t=0}={}\\
    {}= 
    \fr{\rho(x+\Delta x, 0+\tau)-\rho(x,0)}{\tau}=\fr{1}{\tau}\,\psi(x)\delta(x-y)\,,
    \end{multline*}
где $\psi(x)$~--- некоторая периодическая функция. Из решения граничной задачи 
для уравнения~(\ref{e3-sig}) получаем для функции плотности вероятности 
амплитуд отклонений предпочтений избирателей следующую зависимость: 

   \begin{figure*}[b] %fig4
\vspace*{1pt}
\begin{center}
 \mbox{%
 \epsfxsize=163mm 
\epsfbox{and-4.eps}
 }
 \end{center}
\vspace*{-9pt}
\Caption{Теоретические зависимости плотности вероятности величин амплитуд от времени 
их расчета ($\tau\hm=0{,}7$): (\textit{а})~$\varepsilon\hm=\xi\hm= 4{,}5$; 
(\textit{б})~$\varepsilon\hm=4{,}0$, $\xi\hm=2{,}5$; \textit{1}~--- 
$t\hm=1$; \textit{2}~--- 4; \textit{3}~--- 5; \textit{4}~--- $t\hm=15$}
\end{figure*}

\noindent
\begin{multline}
\rho(x,t)= \fr{\tau U(t-k) e^{(\varepsilon-\xi)x/(2(\varepsilon^2 -\varepsilon\xi+\xi^2))} 
e^{-t/(2\tau)}}{2\sqrt{\varepsilon^2-\varepsilon\xi+\xi^2}}\times{}\\
{}\times \left\{ 
\fr{1}{\tau}
\left\{ \fr{1}{2}+\psi(x)\right\}\sum\limits^\infty_{n=0} \fr{\left\{ \omega^2\left\{t^2-
k^2\right\} \right\}^n} {4^n(n!)^2}+{}\right.\\
\left.{}+ \fr{t}{t^2-k^2}\sum\limits^\infty_{n=0} 
\fr{2n\left\{\omega^2\left\{t^2-k^2\right\}\right\}^n}{4^n(n!)^2}\right\}\,,
\label{e4-sig}
\end{multline}
где 
\begin{align*}
\omega&= \sqrt{\fr{\varepsilon\xi}{4\tau^2(\varepsilon^2 -\varepsilon\xi+\xi^2)}}\,;\\
k&=\fr{\vert x\vert \tau}{\sqrt{\varepsilon^2-\varepsilon\xi +\xi^2}}\,;
\end{align*}
$\psi(x)$~--- периодические функции:
\begin{equation*}
\psi(x)= 
\begin{cases}
\cos \left\{ 2\pi\fr{x}{\lambda}\right\}\,;\\[6pt]
\sin \left\{ 2\pi\fr{x}{\lambda}\right\}\,;
\end{cases}
\end{equation*}
$U(t-k)$~--- функция Хэвисайда:
$$
U(t-k)=\begin{cases}
0\,, &\ \mbox{если }t<k\,;\\
    \fr{1}{2}\,, &\ \mbox{если } t=k\,;\\
    1\,, &\ \mbox{если } t>k\,.
    \end{cases}
    $$
    
 
    
    Величина~$\lambda$ в~периодической функции~$\psi(x)$ имеет смысл 
длины волны для процесса колебаний амплитуды. При отсутствии осцилляций 
$\psi(x)\hm=0$. Величина $k\hm= \vert x\vert \tau/ \sqrt{\varepsilon^2-
\varepsilon\xi+\xi^2}$ имеет смысл времени запаздывания распространения 
волнового процесса на время, равное~$k$ ($k\hm=\vert x\vert/V_0$, где 
$V_0\hm= \sqrt{\varepsilon^2-\varepsilon\xi+\xi^2}/\tau\hm= \lambda/\tau$~--- 
скорость распространения волнового процесса, $\lambda\hm= 
\sqrt{\varepsilon^2-\varepsilon\xi+\xi^2}$~--- длина волны). Для функции 
$\rho(x,t)$ выполняется условие нормировки 
$$
\int\limits_{-\infty}^{+\infty} 
\rho(x,t)\,dx=1\,. 
$$

\subsection{Анализ модели, учитывающей влияние на~амплитуду отклонений 
предпочтений избирателей, процессов самоорганизации и~наличия памяти}
 
    На рис.~4 представлены результаты моделирования зависимости 
плотности вероятности амплитуды отклонения от интервала времени ее расчета 
для различных наборов параметров~$\varepsilon$ и~$\xi$ с~использованием 
полученного уравнения~(\ref{e4-sig}). При увеличении длины волны 
$\lambda\hm= \sqrt{\varepsilon^2-\varepsilon\xi +\xi^2}$ число осцилляций на 
графиках плотности вероятности амплитуд отклонений предпочтений 
избирателей падает, а~при уменьшении~--- растет. С~увеличением интервала 
времени расчета амплитуд ширина и~число осцилляций увеличиваются, 
а~высота распределения уменьшается. При $\varepsilon\hm>\xi$ происходит 
смещение максимума плотности вероятности вправо, а~при 
$\varepsilon\hm<\xi$~--- влево. С~ростом интервала времени расчета амплитуд 
высота распределения уменьшается, а~ширина и~число осцилляций 
увеличиваются. Наблюдаются асимметрия распределения относительно линии 
максимума. Если $\psi(x,\lambda)\hm=0$, то осцилляции исчезают, а остальные 
характеристики поведения распределения сохраняют свою тенденцию. При 
выборе другого набора параметров~$\xi$, $\varepsilon$, $\tau$ и~$t$ поведение 
плотностей вероятности амплитуд остается прежним, но значения и~положения 
максимумов на графиках изменяются. Сравнение наблюдаемых гистограмм 
распределений (см.\ рис.~1) с~результатами теоретического моделирования (см.\ рис.~4) 
показывает хорошее соответствие разработанной модели наблюдаемым 
данным. 

\begin{figure*} %fig5
\vspace*{1pt}
\begin{center}
 \mbox{%
 \epsfxsize=163mm 
\epsfbox{and-5.eps}
 }
 \end{center}
\vspace*{-9pt}
\Caption{Теоретические зависимости математического ожидания~(\textit{а}) и~дисперсии 
величин амплитуд~(\textit{б}) от времени их расчета для наборов параметров ($\tau\hm=0{,}1$): 
\textit{1}~--- $\xi\hm=0{,}2$, $\varepsilon\hm=0{,}7$;
\textit{2}~--- $\xi\hm=0{,}7$, $\varepsilon\hm= 
0{,}2$}
\end{figure*}


На рис.~5 представлены зависимости математического ожидания 
и~дисперсии величины амплитуды в~зависимости от интервала времени ее 
расчета, полученные с~использованием разработанной модели (для различных 
наборов величин параметров~$\varepsilon$ и~$\xi$).
    
    Расчеты показывают, что при $\varepsilon\hm> \xi$ наблюдаются 
амплитуды роста и~их значения находятся в~положительной области на рис.~5 
(кривая~\textit{1}). При $\varepsilon<\xi$ наблюдаются амплитуды падения, их 
математическое ожидание находится в~отрицательной области (кривая~\textit{2}). 
Для наборов параметров с~инверсией 
величин~$\varepsilon$ и~$\xi$ ($\tau\hm=0{,}1$): $\xi\hm=0{,}7$, $\varepsilon\hm=0{,}2$ 
и~$\xi\hm=0{,}2$, $\varepsilon\hm=0{,}7$~--- поведение дисперсии имеет одинаковый характер, так как 
она является квадратичной величиной и~не имеет отрицательных значений.
    



    
    Среднее значение амплитуды отклонения (математическое 
ожидание)~$\mu(t)$ и~дисперсия~$\sigma^2(t)$ рассчитываются 
с~использованием выражения~(\ref{e4-sig}) следующим образом: 
    \begin{equation*}
    \mu(t)=\int\limits_{-\infty}^{+\infty} x\rho (x,t)\,dx\,;\enskip
\sigma^2(t)= \int\limits_{-\infty}^{+\infty} x^2\rho(x,t)\,dx\,.
\end{equation*}

\begin{figure*} %fig6
\vspace*{1pt}
\begin{center}
 \mbox{%
 \epsfxsize=163mm 
\epsfbox{and-6.eps}
 }
 \end{center}
\vspace*{-9pt}
\Caption{Теоретическое моделирование зависимостей математического ожидания 
и~дисперсии амплитуд отклонений предпочтений избирателей}
\end{figure*}


 Положение 
максимумов и~другие характеристики процессов зависят от выбора величин 
па\-ра\-мет\-ров модели: при увеличении значений параметров модели~$\varepsilon$ 
и~$\xi$ математическое ожидание амплитуд\linebreak уменьшается, а максимум 
сдвигается в~об\-ласть малых интервалов времени. Зависимости 
математического ожидания и~дисперсии наблюдаемых амплитуд отклонения 
предпочтений избирателей от \mbox{величины} интервала времени их расчета, 
полученные из социологических данных (см.\ рис.~2), отличаются по виду от 
результатов моделирования, представленных на рис.~5. Предполагая наличие 
нескольких процессов для выбора предпочтений избирателей по каждому из 
кандидатов с~разными весовыми коэффициентами $\alpha_1, \alpha_2,\ldots$ 
и~несколькими наборами параметров модели~$\varepsilon$ и~$\xi$, получим 
результаты теоретического моделирования (рис.~6), хорошо соответствующие 
наблюдаемым данным ($\tau\hm=0{,}1$). Для процесса~I (возрастание): $\xi_1\hm=0{,}15$; 
$\varepsilon_1\hm=0{,}45$; $\alpha_1\hm=0{,}65$; для процесса~II 
(возрастание): $\xi_2\hm=0{,}20$; $\varepsilon_2 = 0{,}75$;  
$\alpha_2\hm=0{,}15$; для процесса~III (возрастание): $\xi_3\hm= 0{,}25$; 
$\varepsilon_3\hm=0{,}95$; $\alpha_3\hm=0{,}10$; для 
процесса~IV (возрастание): $\xi_4\hm=0{,}30$; $\varepsilon_4\hm=1{,}70$; 
 $\alpha_4\hm=0{,}10$. Весовые коэффициенты каждого из 
процессов могут быть различными и~меняться от~0 до~1. Наличие нескольких 
процессов для динамики предпочтений одного и~того же кандидата может быть 
обусловлено различными группами избирателей с~различным типом поведения 
при выборе предпочтений, а величина весовых коэффициентов может зависеть 
от соотношения численности каждой из групп. 
    
    Сравнение наблюдаемых данных (см.\ рис.~2) и~результатов теоретического 
моделирования (см.\ рис.~6) показывает (с~учетом приближенности 
моделирования), что подбором величин параметров можно получить неплохое 
совпадение с~наблюдаемыми данными. Полученные результаты позволяют 
сделать общий вывод о том, что созданная модель в~целом хорошо описывает 
наблюдаемую динамику электоральных процессов и~ее можно использовать для 
прогнозирования. 
    
    Общий алгоритм прогнозирования: 
    \begin{enumerate}[(1)]
    \item на основе наблюдаемых за какой-то промежуток времени данных 
(например, первая половина избирательной кампании) строим гистограммы, 
описывающие зависимость амплитуд отклонений предпочтений избирателей от 
интервала времени их расчета. Далее находим зависимости математического 
ожидания и~дисперсии; 
    \item  на основе уравнения~(\ref{e4-sig}) и~наблюдаемых характеристик 
процессов (см.\ рис.~2) определяем величины параметров~$\xi$ и~$\varepsilon$. 
Используем полученные результаты для расчета величины предпочтений 
избирателей на момент окончания избирательной кампании. Если полученный 
результат не приводит к~победе, то необходимо повлиять на процесс 
и~изменить его параметры. Например, за счет средств массовой информации 
изменить величину параметра~$\varepsilon$ в~нужную сторону (увеличить свою 
или уменьшить у~соперника). 
    \end{enumerate}
    
    \vspace*{-11pt}

\section{Заключение и~выводы }

\vspace*{-3pt}

\noindent
\begin{enumerate}[1.]
    \item Проведен анализ динамики электоральной кампании по выборам 
президента США в~2016~г. На основании полученных данных построены 
гистограммы зависимостей амплитуд отклонений предпочтений избирателей от 
величины интервала времени их определения. На основе обработанных 
данных наблюдений создана модель стохастической динамики изменения 
предпочтения избирателей, учитывающая процессы самоорганизации, 
наличие памяти и~хорошо описывающая основные характеристики 
исследуемых процессов (появление осцилляций, изменение высоты и~ширины 
распределения при изменении интервала времени расчета амплитуд и~т.\,д.). 



    \item При создании модели стохастической динамики предпочтений 
избирателей рассмотрены схемы вероятностей переходов между ее 
возможными состояниями. Выведено нелинейное дифференциальное 
уравнение второго порядка, сформулирована и~решена граничная задача для 
определения функции плотности вероятности амплитуды отклонений 
предпочтений избирателей от величины интервала времени ее определения. 
Дифференциальное уравнение содержит член, отвечающий за возможность 
самоорганизации, а также учитывает наличие памяти. 
    \item Разработанную стохастическую модель динамики изменения 
величины предпочтений избирателей с~учетом процессов самоорганизации, 
наличия памяти и~осцилляций можно использовать для прогнозирования 
результатов электоральных кампаний и~принятия решений. 
    \end{enumerate}
   
{\small\frenchspacing
{%\baselineskip=10.8pt
%\addcontentsline{toc}{section}{References}
\begin{thebibliography}{99}
\bibitem{1-sig}
\Au{Easley D., Kleinberg~J.} Networks, crowds, and markets: Reasoning about a~highly connected 
world.~--- Cambridge: Cambridge University Press, 2010. 819~p. 
doi:  10.1017/ CBO9780511761942.
\bibitem{2-sig}
\Au{Karsai M., Iniguez~G., Kaski~K., Kertesz~J.} Complex contagion process in spreading of 
onlne innovation~// J.~R.~Soc. Interface, 2014. Vol.~11. Art. ID: 20140694. 8~p. doi: 
10.1098/rsif.2014.0694.

\bibitem{4-sig} %3
\Au{Gleeson J.\,P., Cahalane~D.\,J.} Seed size strongly affects cascades on random networks~// 
Phys. Rev.~E, 2007. Vol.~75. Art. ID: 056103. 4~p. doi: 10.1103/PhysRevE.75.0561037.

\bibitem{3-sig} %4
\Au{Barrat A., Barthelemy~M., Vespignani~A.} Dynamical processes on complex networks.~--- 
Cambridge: Cambridge University Press, 2012. 347~p. doi: 10.1017/ CBO9780511791383.

\bibitem{5-sig}
\Au{Kocsis G., Kun~F.} Competition of information channels in the spreading of innovations~// 
Phys. Rev.~E, 2011. Vol.~84. Art. ID: 026111. 7~p. doi: 10.1103/PhysRevE.84.026111.
\bibitem{6-sig}
\Au{Airoldi E.\,M., Blei~D.\,M., Fienberg~S.\,E., Xing~E.\,P.} Mixed membership stochastic 
blockmodels~// J.~Mach. Learn. Res., 2008. Vol.~9. P.~1981--2014.


\bibitem{8-sig} %7
\Au{Khvatova T., Block M., Zhukov~D., Lesko~S.} How to measure trust: The percolation model 
applied to intra-organisational knowledge sharing networks~// J.~Knowl. Manag., 2016. Vol.~20. 
Iss.~5. P.~918--935. doi: 10.1108/ JKM-11-2015-0464.

\bibitem{7-sig} %8
\Au{Khvatova T.\,Yu., Zaltsman~A.\,D., Zhukov~D.\,O.} Information processes in social networks: 
Percolation and stochastic dynamics~// CEUR Workshop Procee., 2017. Vol.~2064. 
P.~277--288.

\bibitem{9-sig}
\Au{Plikynas D., Raudys~A., Raudys~S.} Agent-based modelling of excitation propagation in social 
media groups~// J.~Experimental Theoretical Artificial Intelligence, 2015. Vol.~27. Iss.~4.  
P.~373--388. doi: 10.1080/ 0952813X.2014.954631.
\bibitem{10-sig}
\Au{Wang A., Wu~W., Chen~J.} Social network rumors spread model based on cellular automata~// 
10th Conference (International) on Mobile Ad-hoc and Sensor Networks Proceedings.~--- 
Piscataway, NJ, USA: IEEE, 2014. P.~236--242. doi: 10.1109/MSN.2014.39.
\bibitem{11-sig}
\Au{Андрианова Е.\,Г., Головин~С.\,А., Зыков~С.\,В., Лесько~С.\,А.,
Чукалина~Е.\,Р.}
Обзор современных моделей и~методов анализа временных рядов динамики процессов 
в~социальных, экономических и~социотехнических системах~//
Российский технологический ж., 2020.
  Т.~8. №\,4(36).  С.~7--45. doi: 
10.32362/2500-316X-2020-8-4-7-45.

\bibitem{16-sig} %12
\Au{Zhukov D.\,O., Lesko~S.\,A., Khvatova~T.\,Yu.} Percolation models of information distribution 
and blocking in social networks~// 5th Ashridge  Research Conference (International) 
Global Disruption and Organisational Innovation.~--- Berkhamsted, U.K., 2016. Art. ID: 
23423.

\bibitem{12-sig} %13
\Au{Zhukov D., Khvatova~T., Zaltsman~A.} Stochastic dynamics of influence expansion in social 
networks and managing users' transitions from one state to another~// 11th European Conference 
on Information Systems Management Proceedings.~--- Reading: Academic Conferences and 
Publishing International Ltd., 2017. P.~322--329. 

\bibitem{17-sig} %14
\Au{Zhukov D.\,O., Alyoshkin~A.\,S., Obukhova~A.\,G.} Modelling to be based on systems of 
differential kinetic equations to processes group selection voters during the electoral campaign of 
Trump--Clinton 2015--2016~// 7th Conference (International) on Information Communication and 
Management Proceedings.~--- New York, NY, USA: ACM, 2017. P.~88--94.
\bibitem{18-sig} %15
\Au{Sigov A.\,S., Zhukov~D.\,O., Khvatova~T.\,Yu., Andrianova~E.\,G.} Model of forecasting of 
information events on the basis of the solution of a boundary value problem for systems with 
memory and self-organization~// J.~Commun. Technol. El., 2018. Vol.~18. 
Iss.~2. P.~106--117.


\bibitem{14-sig} %16
\Au{Zhukov D., Khvatova~T., Istratov~L.} A~stochastic dynamics model for shaping stock indexes 
using self-organization processes, memory and oscillations~// European Conference on the Impact 
of Artificial Intelligence and Robotics Proceedings.~--- Oxford, U.K.: ACPIL, 2019. P.~390--401.
\bibitem{15-sig} %17
\Au{Zhukov D., Zaltsman~A., Khvatova~T.} Forecasting changes in states in social networks and 
sentiment security using the principles of percolation theory and stochastic dynamics~// IEEE  
Conference (International) ``Quality Management, Transport and Information Security, Information 
Technologies.''~--- Piscataway, NJ, USA: IEEE, 2019. P.~149--153. 


\bibitem{19-sig} %18
\Au{Smychkova A., Zhukov~D.} Complex of description models for analysis and control group 
behavior based on stochastic cellular automata with memory and systems of differential kinetic 
equations~// 1st  Conference (International) on Control Systems, Mathematical Modelling, 
Automation and Energy Efficiency Proceedings.~--- Lipetsk: Lipetsk State Technical 
University, 2019. P.~218--223.

\bibitem{13-sig} %19
\Au{Zhukov D., Khvatova~T., Millar~C., Zaltsman~A.} Modeling the stochastic dynamics of 
influence expansion and managing transitions between states in social networks~// Technol. 
Forecast. Soc., 2020. Vol.~158. Р.~1--15.

\bibitem{20-sig}
\Au{Жуков Д.\,О., Хватова~Т.\,Ю., Зальцман~А.\,Д.} Моделирование стохастической 
динамики изменения состояний узлов 
и~перколяционных переходов в~социальных сетях с~учетом самоорганизации 
и~наличия памяти~// Информатика и~её применения, 2021. Т.~15. Вып.~1. С.~102--110.
\bibitem{21-sig}
\Au{Zhukov D., Andrianova~E., Trifonova~O.} Stochastic diffusion model for analysis of dynamics 
and forecasting events in news feeds~// Symmetry, 2021. Vol.~13. Iss.~2. Art. No.\,257. 21~p. 
doi: 10.3390/sym13020257.
\bibitem{22-sig}
\Au{Zhukov D., Andrianova~E., Novikova~O.} Diffusion model for forecasting events in news 
feeds~// J.~Phys. Conf. Ser., 2021. Vol.~1727. Iss.~1. P.~21--32.
 \end{thebibliography}

}
}

\end{multicols}

\vspace*{-3pt}

\hfill{\small\textit{Поступила в~редакцию 15.09.2019}}

\vspace*{8pt}

%\pagebreak

%\newpage

%\vspace*{-28pt}

\hrule

\vspace*{2pt}

\hrule

%\vspace*{-2pt}

\def\tit{STOCHASTIC DYNAMICS OF~SELF-ORGANIZING SOCIAL SYSTEMS WITH~MEMORY (ELECTORAL 
PROCESSES)}


\def\titkol{Stochastic dynamics of~self-organizing social systems with~memory (electoral 
processes)}

\def\aut{A.\,S.~Sigov, E.\,G.~Andrianova, %T.\,Yu.~Khvatova$^2$, D.\,O.~Zhukov$^1$, 
and~L.\,A.~Istratov}

\def\autkol{A.\,S.~Sigov, E.\,G.~Andrianova, and~L.\,A.~Istratov}
%T.\,Yu.~Khvatova, et al.}
%D.\,O.~Zhukov$^1$

\titel{\tit}{\aut}{\autkol}{\titkol}

\vspace*{-11pt}


\noindent
Russian Technological University (MIREA), 78~Vernadskogo Ave., Moscow 119454, Russian Federation


 
\def\leftfootline{\small{\textbf{\thepage}
\hfill INFORMATIKA I EE PRIMENENIYA~--- INFORMATICS AND
APPLICATIONS\ \ \ 2021\ \ \ volume~15\ \ \ issue\ 2}
}%
\def\rightfootline{\small{INFORMATIKA I EE PRIMENENIYA~---
INFORMATICS AND APPLICATIONS\ \ \ 2021\ \ \ volume~15\ \ \ issue\ 2
\hfill \textbf{\thepage}}}

\vspace*{3pt}
      


\Abste{The paper discusses the use of the methods and approaches 
which are common for theoretical computer science as well as the use of 
its applications for analysis and modeling of social group processes. Based
 on the
 developed model for describing stochastic processes, taking into account 
 self-organization and the presence of
 memory, an analysis of the voter preference dynamics 
 during the 2016 U.S.\ presidential campaign was conducted.\linebreak\vspace*{-12pt}}
 
 \Abstend{The sociological data processing 
 allowed plotting the probability density histograms for the amplitudes of voter preference 
 deviation, depending on their determination interval, and developing a~model that well 
 describes the main characteristics of the observed processes (appearance of oscillations,
  changes in the height and width of the distribution depending on the changes in the amplitude 
  calculation interval, etc.). In the course of building the model, the probability schemes of
   transitions between the possible states of the social system (voter preferences) were considered and 
   a~second-order nonlinear differential equation was derived. In addition, 
   a~boundary problem to determine the probability density function of the amplitude of voter 
   preference deviation depending on its determination interval was formulated and solved.
    The model differential equation has a~term responsible for the self-organization possibility 
    and takes into account the presence of memory. The oscillation possibility depends on the 
    initial conditions.
 The developed model can be used for analyzing election campaigns and making relevant decisions.}

\KWE{oscillation amplitude distribution function; stochastic dynamics; self-organization; 
presence of memory; probability density oscillations; electoral processes}

\DOI{10.14357/19922264210216}

%\vspace*{-15pt}

% \Ack
%\noindent

\vspace*{6pt}

  \begin{multicols}{2}

\renewcommand{\bibname}{\protect\rmfamily References}
%\renewcommand{\bibname}{\large\protect\rm References}

{\small\frenchspacing
 {%\baselineskip=10.8pt
 \addcontentsline{toc}{section}{References}
 \begin{thebibliography}{99}


\bibitem{1-sig-1}
\Aue{Easley, D., and J.~Kleinberg.} 2010. \textit{Networks, crowds, and markets: Reasoning about 
a~highly connected  world}. Cambridge: Cambridge University Press. 819~p. 
doi:  10.1017/ CBO9780511761942.
\bibitem{2-sig-1}
\Aue{Karsai, M., G.~Iniguez, K.~Kaski, and J.~Kertesz.} 2014. Complex contagion process in spreading of online 
innovation. \textit{J.~R.~Soc. Interface} 11:20140694. 8~p. 
doi: 10.1098/ rsif.2014.0694.

\bibitem{4-sig-1} %3
\Aue{Gleeson, J.\,P., and D.\,J.~Cahalane.} 2007. Seed size strongly affects cascades on random networks. 
\textit{Phys. Rev.~E} 75:056103. 4~p.  doi: 10.1103/PhysRevE.75.0561037.

\bibitem{3-sig-1} %4
\Aue{Barrat, A., M.~Barthelemy, and A.~Vespignani.} 2012. \textit{Dynamical processes on complex networks}. 
Cambridge: Cambridge University Press. 347~p. doi: 10.1017/ CBO9780511791383.

\bibitem{5-sig-1}
\Aue{Kocsis, G., and F.~Kun.} 2011. Competition of information channels in the spreading of innovations. 
\textit{Phys. Rev.~E} 84:026111. 7~p. doi: 10.1103/PhysRevE.84.026111.
\bibitem{6-sig-1}
\Aue{Airoldi, E.\,M., D.\,M.~Blei, S.\,E.~Fienberg, and E.\,P.~Xing.} 2008. Mixed membership stochastic 
blockmodels. \textit{J.~Mach. Learn. Res.} 9:1981--2014.

\bibitem{8-sig-1} %7
\Aue{Khvatova, T., M.~Block, D.~Zhukov, and S.~Lesko.} 2016. How to measure trust: The percolation model 
applied to intraorganisational knowledge sharing networks. \textit{J.~Knowl. Manag.} 20(5):918--935.
doi: 10.1108/JKM-11-2015-0464.

\bibitem{7-sig-1} %8
\Aue{Khvatova, T.\,Yu., A.\,D.~Zaltcman, and D.\,O.~Zhukov.} 2017. Information processes in social networks: 
Percolation and stochastic dynamics. \textit{CEUR Workshop Procee}. 2064:277--288. 

\bibitem{9-sig-1}
\Aue{Plikynas, D., A.~Raudys, and S.~Raudys.} 2015. Agentbased modelling of excitation propagation in social media 
groups. \textit{J.~Exp. Theor. Artif. In.} 27(4):373--388. doi: 10.1080/0952813X.2014.954631.
\bibitem{10-sig-1}
\Aue{Wang, A., W.~Wu, and J.~Chen.} 2014. Social network rumors spread model based on cellular automata. 
\textit{10th Conference (International)  on Mobile Ad-Hoc and Sensor Networks Proceedings}. 
Piscatway, NJ: IEEE. 236--242. 
doi: 10.1109/MSN.2014.39.
\bibitem{11-sig-1}
\Aue{Andrianova, E.\,G., S.\,A.~Golovin, S.\,V.~Zykov, S.\,A.~Lesko, and E.\,R.~Chukalina.} 2020. 
Obzor sovremennykh modeley i~metodov analiza vremennykh ryadov dinamiki protsessov 
v~sotsial'nykh, ekonomicheskikh i~sotsiotekhnicheskikh sistemakh
[Review of 
modern models and methods of analysis of time series of dynamics of processes in social, economic and socio-technical 
systems]. \textit{Russ. Technological~J.} 8(4):7--45. doi: 
10.32362/2500-316X-2020-8-4-7-45.
\bibitem{16-sig-1} %12
\Aue{Zhukov, D.\,O., S.\,A.~Lesko, and T.\,Yu.~Khvatova.} 2016. 
Percolation models of information distribution and 
blocking in social networks. \textit{5th Ashridge  Research Conference (International)  Global Disruption and 
Organisational Innovation Proceedings}. Berkhamsted, U.K. 23423. 

\bibitem{12-sig-1} %13
\Aue{Zhukov, D., T.~Khvatova, and A.~Zaltsman.} 2017. Stochastic dynamics of influence expansion in social 
networks and managing users' transitions from one state to another. \textit{11th European Conference on Information 
Systems Management Proceedings}. Reading: Academic Publishing International Ltd. 322--329. 
\bibitem{17-sig-1} %14
\Aue{Zhukov, D.\,O., A.\,S.~Alyoshkin, and A.\,G.~Obukhova.} 2017. Modelling to be based on systems of 
differential kinetic equations to processes group selection voters during the electoral campaign
 of Trump--Clinton 2015--2016. \textit{7th  Conference (International) on Information Communication and Management Proceedings}. 
New York, NY: ACM.  88--94.
\bibitem{18-sig-1} %15
\Aue{Sigov, A.\,S., D.\,O.~Zhukov, T.\,Yu.~Khvatova, and E.\,G.~Andrianova}. 2018. Model of forecasting of 
information events on the basis of the solution of a boundary value problem for systems with memory and  
self-organization. \textit{J.~Commun. Technol. El.} 18(2):106--117.


\bibitem{14-sig-1} %16
\Aue{Zhukov, D., T.~Khvatova, and L.~Istratov.} 2019. A stochastic dynamics model for shaping stock indexes using 
self-organization processes, memory and oscillations. \textit{European Conference on the Impact of Artificial 
Intelligence and Robotics Proceedings}. Oxford, U.K.: ACPIL. 390--401.
\bibitem{15-sig-1} %17
\Aue{Zhukov, D., A.~Zaltsman, and T.~Khvatova.} 2019. Forecasting changes in states in social networks and 
sentiment security using the principles of percolation theory and stochastic dynamics. \textit{Conference (International) 
``Quality Management, Transport and Information Security, Information Technologies'' Proceedings}. 
Piscataway, NJ: IEEE. 149--153. 


\bibitem{19-sig-1} %18
\Aue{Smychkova, A., and D.~Zhukov.} 2019. Complex of description models for analysis and control group behavior 
based on stochastic cellular automata with memory and systems of differential kinetic equations. \textit{1st  Conference 
(International)  on Control Systems, Mathematical Modelling, Automation and Energy Efficiency Proceedings}. 
Lipetsk: Lipetsk State Technical 
University.  218--223.
 \bibitem{13-sig-1} %19
\Aue{Zhukov, D., T.~Khvatova, C.~Millar, and A.~Zaltsman.} 2020. Modeling the stochastic dynamics of influence 
expansion and managing transitions between states in social networks. \textit{Technol. Forecast. Soc.} 
158:1--15.

\bibitem{20-sig-1}
\Aue{Zhukov, D.\,O., T.\,Yu.~Khvatova, and A.\,D.~Zaltcman.} 2021. 
Modelirovanie stokhasticheskoy dinamiki izmereniya sostoyaniy uzlov i~perkolyatsionnykh
pe\-re\-kho\-dov v~so\-tsi\-al'\-nykh se\-tyakh s~uche\-tom sa\-mo\-or\-ga\-ni\-za\-tsii i~na\-li\-chiya pa\-mya\-ti
[Modeling of the stochastic dynamics of changes 
in node states and percolation transitions in social networks with self-organization and memory]. \textit{Informatika i~ee 
Primeneniya~--- Inform. Appl.} 15(1):102--110.
{\looseness=1

}

\bibitem{21-sig-1}
\Aue{Zhukov, D., E.~Andrianova, and O.~Trifonova.} 2021. Stochastic diffusion model for analysis of dynamics and 
forecasting events in news feeds. \textit{Symmetry} 13(2):257. 21~p. doi: 10.3390/sym13020257.
\bibitem{22-sig-1}
\Aue{Zhukov, D., E.~Andrianova, and O.~Novikova.} 2021. Diffusion model for forecasting events in news feeds. 
\textit{J.~Phys. Conf. Ser.} 1727(1):21--32.
\end{thebibliography}

 }
 }

\end{multicols}

\vspace*{-3pt}

  \hfill{\small\textit{Received September~15, 2019}}


%\pagebreak

\vspace*{-8pt}  


\Contr

\noindent
\textbf{Sigov Alexander S.} (b.\ 1945)~--- 
Doctor of Science in physics and mathematics, professor,
Academician of RAS, 
President of the Russian Technological University 
(MIREA),  78~Vernadskogo Ave., Moscow 119454, Russian Federation; 
\mbox{sigov@mirea.ru}

\vspace*{6pt}

\noindent
\textbf{Andrianova Elena G.} (b.\ 1963)~--- Candidate of Science (PhD) in technology, associate 
professor, Russian Technological University 
(MIREA),   78~Vernadskogo Ave., Moscow 119454, 
Russian Federation; andrianova@mirea.ru 

%\vspace*{6pt}

%\noindent
%\textbf{Khvatova Tatiana Yu.} (b.\ 1973)~--- Doctor of Science in economics, professor, Emlyon 
%Business School, 23~Avenue Guy de Collongue, Lyon-Ecully 69130, France;  
%\mbox{khvatova@em-lyon.com}

%\vspace*{6pt}

%\noindent
%\textbf{Zhukov Dmitry O.} (b.\ 1965)~--- Doctor of Science in technology, professor, Russian Technological University 
%(MIREA),   78~Vernadskogo Ave., Moscow 119454, Russian Federation; 
%\mbox{zhukovdm@yandex.ru}

\vspace*{6pt}

\noindent
\textbf{Istratov Leonid A.} (b.\ 1991)~--- student, Russian Technological University 
(MIREA),  
78~Vernadskogo Ave., Moscow 119454, Russian Federation; \mbox{istratov@mirea.ru}
      
\label{end\stat}

\renewcommand{\bibname}{\protect\rm Литература}