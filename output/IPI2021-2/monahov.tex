\def\stat{monahov}

\def\tit{РАЗЛОЖЕНИЯ ЧЕБЫШЁВА--ЭДЖВОРТА ДЛЯ~РАСПРЕДЕЛЕНИЙ ОБОБЩЕННЫХ СТАТИСТИК ТИПА 
ХОТЕЛЛИНГА, ПОСТРОЕННЫХ ПО~ВЫБОРКАМ СЛУЧАЙНОГО РАЗМЕРА$^*$}

\def\titkol{Разложения Чебышёва--Эджворта для~распределений обобщенных статистик типа 
Хотеллинга} %, построенных по~выборкам случайного размера}

\def\aut{М.\,М.~Монахов$^1$}

\def\autkol{М.\,М.~Монахов}

\titel{\tit}{\aut}{\autkol}{\titkol}

\index{Монахов М.\,М.}
\index{Monakhov M.\,M.}

{\renewcommand{\thefootnote}{\fnsymbol{footnote}} \footnotetext[1]
{Исследование выполнено в~соответствии с программой Московского центра фундаментальной 
и~прикладной математики.}}


\renewcommand{\thefootnote}{\arabic{footnote}}
\footnotetext[1]{Московский центр фундаментальной и~прикладной 
математики Московского государственного университета имени М.\,В.~Ломоносова, 
\mbox{mih\_monah@mail.ru.}}


%\vspace*{-12pt}



\Abst{Доказан аналог теоремы переноса для функций распределения статистики 
типа Хотеллинга, размер которой является случайной величиной, позволяющий 
оценить скорость сходимости разложения Че\-бы\-шё\-ва--Эдж\-вор\-та и~получить явный вид 
вышеупомянутого разложения для исходной статистики. На основании следствия 
к~доказанному аналогу теоремы переноса для случая, когда размер статистики имеет 
отрицательное биномиальное распределение (смещенное на 1), получен явный вид 
разложения Че\-бы\-шё\-ва--Эдж\-вор\-та второго порядка на базе предельного 
$F$-рас\-пре\-де\-ле\-ния. По построенному разложению Че\-бы\-шё\-ва--Эдж\-вор\-та для специального 
значения параметра случайного размера выборки построено разложение Кор\-ни\-ша--Фи\-ше\-ра 
второго порядка на базе квантилей $F$-рас\-пре\-де\-ле\-ния. Проведен 
вычислительный эксперимент и~построены графики, иллюстрирующие полученное 
разложение Че\-бы\-шё\-ва--Эдж\-ворта.}


\KW{обобщенные разложения Чебышёва--Эджворта; разложения Кор\-ни\-ша--Фи\-ше\-ра;
выборка случайного объема; $F$-рас\-пре\-де\-ле\-ние;  статистика типа Хотеллинга}

\DOI{10.14357/19922264210211}

%\vspace*{-3pt}


\vskip 10pt plus 9pt minus 6pt

\thispagestyle{headings}

\begin{multicols}{2}

\label{st\stat}


\section{Введение}
%\label{intro}

В анализе данных довольно часто возникает задача множественных сравнений. 
Например, различных возрастных, профессиональных, социальных слоев населения, 
или влияния различных доз препарата, методов диагностики и~т.\,д. Данную задачу 
помогает решить дисперсионный анализ, который применяется для исследования 
влияния одной или нескольких качественных переменных\linebreak (факторов) на одну 
зависимую количественную переменную. Дисперсионный  анализ широко применяется 
в~сфере производства, здравоохранения, рекла\-мы, продовольствия, обслуживания, его 
реализации представлены в~статистических пакетах для многих языков 
программирования. Сущность дисперсионного анализа заключается в~расчленении 
общей дисперсии изучаемого признака на отдельные компоненты, обусловленные 
влиянием конкретных факторов, и~проверке гипотез о~зна\-чи\-мости влияния этих 
факторов на исследуемый признак. Дополнительные проб\-ле\-мы возникают в~случае, 
когда объем наблюдения оказывается случайным~\cite{BenKorGal13}.

 В задачах многомерного однофакторного дисперсионного анализа рассматриваются~$q$ 
 выборок с фиксированным размером $n_1, \ldots, n_q$: $(X_{1 1},\ldots, X_{1 
n_1}), \ldots, (X_{q 1},\ldots, X_{q n_q})$, где $X_{i j}$~--- \mbox{$p$-мер}\-ное 
наблюдение, представимое в~виде:
$$
X_{i j} = \mu + \alpha_i + \epsilon_{ij}\,. 
$$
Здесь $\mu$ и~$\alpha_i$~--- неизвестные векторные параметры; $\epsilon_{ij}$~--- 
случайные ошибки, явля\-ющи\-еся независимыми одинаково распределенными случайными 
величинами с нормальным распределением~$N_p(0,B)$. При рассмотрении основной 
гипотезы од\-но\-род\-ности выборок
$$
H_0: \alpha_1 = \cdots = \alpha_q = 0
$$
определяются матрицы~$S_h$ и~$S_e$, отражающие межуровневые и~внутриуровневые 
различия соответственно для элементов выборок
\begin{align*}
S_h &= \sum\limits_{i=1}^{q}n_i(\bar{y}_i - \bar{y})(\bar{y}_i - \bar{y})'; \\
 S_e &= 
\sum\limits_{i=1}^{q} \sum\limits_{j=1}^{n_i}(y_{ij} - \bar{y}_i)(y_{ij} - \bar{y}_i)'
\end{align*}
с $n = n_1 + \cdots + n_q$ и~$$
\bar{y}_i = \fr{1}{n_i}\sum\limits_{j=1}^{n_i}y_{ij}; \quad \bar{y} = 
\fr{1}{n}\sum\limits_{i=1}^{q}\sum\limits_{j=1}^{n_i}y_{ij}.
$$
В предположении справедливости основной гипотезы~$H_0$ случайные матрицы~$S_h$ 
и~$S_e$ независимы и~имеют центральные распределения Уишарта $W_p(q, I_p)$ 
и~$W_p(n, I_p)$ соответственно. На базе мат\-риц~$S_h$ и~$S_e$ для проверки гипотезы~$H_0$ 
строятся статистики, одной из которых является статистика Лоу\-ли--Хо\-тел\-линга.

В работах~\cite{BenKorGal13,BenKorGal12} была доказана общая тео\-ре\-ма\linebreak переноса, 
позволяющая оценить ско\-рость сходимости разложения типа Че\-бы\-шё\-ва--Эдж\-вор\-та 
первого порядка для асимптотически нормальных статистик, построенных по выборкам 
случайного\linebreak объема, а~также получить явный вид данного разложения. В~качестве 
примера статистики в~этих работах рассматривается выборочное среднее, которое 
приближается нормальным распределением. В~работе~\cite{MMU16} получено 
разложение Кор\-ни\-ша--Фи\-ше\-ра первого порядка для разложения Че\-бы\-шё\-ва--Эдж\-вор\-та из 
работы~\cite{BenKorGal13}. В~работе~\cite{CMU} получены разложения 
Че\-бы\-шё\-ва--Эдж\-вор\-та и~Кор\-ни\-ша--Фи\-ше\-ра второго порядка для статистик типа выборочного 
среднего, построенных по выборкам случайного объема. Данная работа развивает 
результаты вышеперечисленных работ. Для статистики типа Хотеллинга случайного 
размера доказан аналог теоремы переноса и~построено асимптотическое разложение 
типа Че\-бы\-шё\-ва--Эдж\-вор\-та для функции распределения данной статистики.

Используем следующие обозначения: $\mathbf{R}$~--- множество вещественных чисел;
$\mathbf{N}:=\{1,2,\ldots\}$~--- положительные целые числа; $\mathbf{I}_{A}(x)$~--- индикаторная функция.

Определим статистику Лоу\-ли--Хо\-тел\-лин\-га (см., например,~\cite{UAF2016}):
\begin{equation}
\label{hotel}
T_n=T^2_0=n\,\mathrm{tr}\,S_h S_e^{-1}.
\end{equation}
Рассмотрим случай, когда параметр~$n$  не определен заранее, а является 
случайной величиной~$N_n$. В~этом случае размеры выборок $n_1, \ldots, n_q$ 
становятся независимыми одинаково распределенными случайными величинами 
$N_{n_1}, \ldots, N_{n_q}$, а случайная матрица~$S_e$ становится случайной 
матрицей~$S_{N_n}$, которая в~предположении справедливости основной гипотезы~$H_0$ 
имеет центральное распределение Уишарта $W_p(N_n, I_p)$. Обобщенная 
нормированная статистика Хотеллинга случайного размера запишется в~виде:
\begin{equation}
\label{r_hotel}
T_{N_n} = \widetilde{T}^2_0= g(n)\,\mathrm{tr}\,S_h S_{N_n}^{-1}.
\end{equation}

 В разд.~2 получен аналог теоремы переноса для обобщенной 
нормированной статистики Хотеллинга случайного размера, в~разд.~3 
построен аналог разложения Че\-бы\-шё\-ва--Эдж\-вор\-та для данной статистики, в~разд.~4 
получен явный вид разложения Кор\-ни\-ша--Фи\-ше\-ра для частного случая 
параметра размера данной статистики. В~разд.~5 приведены 
доказательства полученных тео\-рем.


\section{Аналог теоремы переноса для~статистики типа Хотеллинга}
%\label{tr_teo}

Запишем следующую теорему из работы~\cite[теорема 4.1]{FUS05}.

\smallskip

\noindent
\textbf{Теорема~1.}\
%\begin{theore}\label{teorUAF}
\textit{Пусть статистика~$T_n$ определена в~формуле~\eqref{hotel}, $ G_k(x)\hm=\mathrm{Pr}\left \{ \chi 
^2 < x \right\} $~--- функция распределения хи-квад\-рат с~$k$ степенями свободы. 
Существует вещественное число $C_1 \hm> 0$ такое, что для всех целых $n \hm\geq 1$}
\begin{multline}
\label{con1}
\sup\limits_x\left\vert 
\vphantom{\sum\limits_{j=0}^2}
\mathbb{P}\left(n\,\mathrm{tr}\,S_h S_e^{-1} \leq x \right) - G_k(x) - {}\right.\\
\left.{}-
\fr{k}{4n}\sum\limits_{j=0}^2 a_j G_{k+2j}(x)\right\vert \leq C_1 n^{-2},
\end{multline}
\textit{где $k=pq$}; $a_0\hm=q\hm-p\hm-1$; $a_1\hm=-2q$; $a_2\hm=q\hm+p\hm+1$.

\smallskip


Предположим, что функция распределения нормированного случайного размера выборки~$N_n$ 
удовлетворяет следующему условию.

\smallskip

\noindent
\textbf{Условие~1.}\ {Существуют константы $m \hm\in \mathbb{N}$, $\beta \hm> m/2$, $C_2 \hm> 
0$, функция распределения~$H(y)$ с~$H(0+)\hm = 0$, функции ограниченной вариации 
$h_i(y)$, $i\hm=1, \ldots, m$, последовательность $0\hm<g(n) \uparrow \infty$, $n 
\hm\rightarrow \infty$ такие, что для всех целых $n \hm\geq 1$}
\begin{multline}
\label{con2}
\sup\limits_{y \geq 0} \left\vert \mathbb{P}\left( \fr{N_n}{g(n)} \leq y\right) - H(y) - 
\sum\limits_{i=1}^{m}\fr{1}{n^{i/2}} h_i(y) \right\vert  \leq{}\\
{}\leq C_2 n^{- \beta},\enskip
 n \in \mathbb{N}\,.
\end{multline}

Сформулируем аналог теоремы переноса, позволяющий оценить распределение 
обобщенной нормированной статистики Хотеллинга случайного размера $g(n) 
\,\mathrm{tr}\, S_h S^{-1}_{N_n}$.

\smallskip

\noindent
\textbf{Теорема~2.}\
%\begin{theore}\label{transfer}
%
\textit{Пусть статистика~$T_{N_n}$ определена в~формуле~\eqref{r_hotel} и~для случайного 
размера выборки $N_n$ выполнено условие~$1$. Тогда существует константа $C_3\hm>0$ 
такая, что справедливо неравенство}
\begin{multline*}
%\label{eq10z}
\sup\limits_{x} \left\vert \mathbb{P}\left( g(n) \,\mathrm{tr}\,S_h S^{-1}_{N_n} \leq 
x\right)  - F_{n}(x) \right\vert \leq{}\\
{}\leq C_1\mathbb{E}N_n^{-2} + \fr{C_3 + C_2 
M_n}{n^{\beta}}\,,
\end{multline*}
\textit{где}
\begin{multline*}
%\label{gn}
F_{n}(x) = \int\limits^{\infty}_{1/g(n)} G_k(xy)\,dH(y) + {}\\
{}+
\sum\limits_{i=1}^{m}\fr{1}{n^{i/2}}\int\limits^{\infty}_{1/g(n)} G_k(xy)\,dh_i(y) 
+{} \\
{}+\fr{k}{4 g(n)} \int\limits_{1/g(n)}^\infty \sum\limits_{j=0}^{2} 
\fr{a_j}{y}\,G_{k+2j}(xy) \,dH(y)
+{}\\
{}+ \fr{k}{4 g(n)} \sum\limits_{i=1}^{m}\fr{1}{n^{i/2}} \int\limits_{1/g(n)}^\infty 
\fr{1}{y} \sum\limits_{j=0}^{2}a_j G_{k+2j}(xy)\,dh_i(y);
\end{multline*}

\vspace*{-12pt}

\noindent
\begin{multline*}
%\label{mn}
M_n=  \sup\limits_{x}    \int\limits_{1/g(n)}^{\infty}  \left\vert
\fr{\partial}{\partial y} 
\left( 
\vphantom{\sum\limits_{j=0}^{2}}
G_k\left(yx \right) + {}\right.\right.\\
\left.\left.{}+\fr{k}{4g(n)y}\sum\limits_{j=0}^{2}a_j G_{k+2j}\left(yx 
\right) \right)\right\vert dy\,.
\end{multline*}

 

\noindent
\textbf{Следствие~1.} 
%\begin{cor}\label{foll1}
В условиях теоремы~2 с дополнительными предположениями
\begin{gather*}
h_2(0) = 0\,; \enskip  H\left(\fr{1}{g(n)}\right)\leq c_0 n^{-\gamma}\,; \\
 h_2\left(\fr{1}{g(n)}\right)\leq c_1 n^{1-\gamma}; \\
\int\limits^{1/g(n)}_{0}\fr{1}{y}\,dH(y)\leq c_2 g(n) n^{-\gamma}, \\ 
\int\limits^{1/g(n)}_{0} \fr{1}{y} \,dh_2(y) \leq c_3 g(n) n^{1-\gamma}
\end{gather*}
для некоторого $\gamma > 1$
существует $C_3\hm=C_3(C_2,k,q)$ такая, что $ \forall\,n \hm\in \mathbb{N}$
\begin{multline*}
%\label{follow1}
\sup\limits_{x} \left| \mathbb{P}\left( g(n) \,\mathrm{tr}\,S_h S^{-1}_{N_n} \leq 
x\right)  - F_{2;n}(x) \right| \leq{}\\
{}\leq C_1\mathbb{E}N_n^{-2} + C_3 n^{-\min(\beta,\gamma)},
\end{multline*}
где
\begin{multline*}
F_{2;n}(x) = \int\limits^{\infty}_{0} G_k(xy)\,dH(y) +{}\\
{}+ \fr{k}{4g(n)} 
\int\limits_{0}^\infty \sum\limits_{j=0}^{2} \fr{a_j}{y} G_{k+2j}(xy)\,dH(y) +  {} 
\end{multline*}

\noindent
\begin{multline}
\label{gn2n}
{}+  \fr{1}{n}\int\limits^{\infty}_{0} G_k(xy) \,dh_2(y) + {}\\
{}+
\fr{k}{4ng(n)} \int\limits_{0}^\infty \sum\limits_{j=0}^{2}\fr{a_j}{y}\,G_{k+2j}(xy)\,dh_2(y)\,.
\end{multline}

\vspace*{-6pt}

\section{Разложение Чебышёва--Эджворта}
%\label{e_che}

Рассмотрим теперь пример применения тео\-ре\-мы~2. Пусть размер выборки~$N_n(r)$ 
имеет отрицательное биномиальное распределение (смещен на~1) 
с~вероятностью успеха~$1/n$ и~функцией вероятности

\vspace*{-6pt}

\noindent
\begin{multline}
\label{eq5}
\mathbb{P}(N_n(r)=j)= \fr{\Gamma(j+r-1)}{(j-1)!\Gamma(r)}\left(\fr{1}{n} \right)^{r}
\left( 1 - \fr{1}{n} \right)^{j-1}, \\
 r>0, \ \ j =1, 2, 
\ldots
\end{multline}

\vspace*{-2pt}

Теперь получим разложение Че\-бы\-шё\-ва--Эдж\-вор\-та для нормированной статистики типа 
Хотеллинга. Функция $F$-рас\-пре\-де\-ле\-ния~$F(x;a,b)$~--- это абсолютно 
непрерывная функция распределения вероятности, заданная плот\-ностью
\begin{multline*}
f(x;a,b) = {}\\
{}=\fr{1}{B(a/2,b/2)}\left(\fr{a}{b}\right)^{a/2} x^{a/2-1} 
\left(1+\fr{a}{b}x\right)^{-(a+b)/2},\\
 x>0\,.
\end{multline*}

\vspace*{-6pt}

\noindent
\textbf{Лемма~1.}
%\begin{lemm}\label{NM}  
\textit{Пусть $r\hm>1$, случайная величина~$N_n(r)$ определена 
формулой~\eqref{eq5}, тогда}
\begin{equation}
\label{NM2}
\mathbb{E}\left(N_n(r)\right)^{- 2} \leq C(r) 
\begin{cases} 
n^{- r}, &  1 < r < 2\,;\\ 
\ln(n)  n^{- 2}, &   r  = 2\,; \\
 n^{-2}, & r > 2\,.
 \end{cases}
\end{equation}
\textit{В случае если $r\hm=2$, скорость сходимости в~\eqref{NM2}, не может быть улучшена}.


\smallskip

Используя теорему~1 из работы~\cite{CMU}, получаем следующий результат.



\smallskip

\noindent
\textbf{Теорема~3.}\
%\begin{theore}\label{T3}
\textit{Пусть статистика $T_m$ определена формулой \eqref{hotel}.
Пусть также дискретная случайная величина $N_n\hm=N_n(r)$ с параметром $r \hm> 1$ 
имеет распределение, задаваемое}~\eqref{eq5}, \textit{и~независима от $W_p(q, I_p)$ 
и~$W_p(n, I_p)$.
Рассмотрим статистику $T_{N_n} \hm= g(n) \,\mathrm{tr} S_h S^{-1}_{N_n}$.  
Асимптотическое разложение для случайного объема $N_n(r)$ с~$r\hm>1$ из}~\cite[теорема~1]{CMU} 
\textit{справедливо с $g(n) = \mathbb{E}\left(N_n(r)\right) = r(n\hm-
1)\hm+1$. Тогда существует константа $C\hm=C(r)\hm>0$  такая, что для всех $n \in 
\mathbf{N}$}

\end{multicols}

\begin{figure*}[b] %fig1
  \vspace*{6pt}
  \begin{center}
    \mbox{%
 \epsfxsize=163mm 
 \epsfbox{mon-1.eps}
 }
\end{center}

%\vspace*{-3pt}

{\small Эмпирическая функция распределения $\mathbb{P}\left( g(n) 
\,\mathrm{tr}\,S_h S^{-1}_{N_n} \leq x\right)$~(\textit{1}), аппроксимация 
первого порядка $F\left({x}/{k};k,2r\right)$~(\textit{2}) 
и~аппроксимация второго порядка~$F_{2;n}(x)$~(\textit{3}) при
 $p\hm=1$, $n\hm=10$ и~$r\hm=3$:
 (\textit{а})~$q\hm=1$; (\textit{б})~3; (\textit{в})~5;
 (\textit{г})~$q\hm=9$}
 %\label{hotel_q1}
\end{figure*}

\begin{multicols}{2}

\noindent
\begin{multline}
\label{eq10q}
\sup\limits_{x} \left\vert \mathbb{P}\left( g(n) \,\textrm{tr} S_h 
S^{-1}_{N_n} \leq x\right) - F_{2; n}(x) \right\vert \leq{}\\
{}\leq  C  
\begin{cases} 
n^{- r}, &  1 < r < 2\,;\\ 
\ln(n) \, n^{- 2}, &   r  = 2\,; \\ 
n^{- 2}, & r > 2\,;
 \end{cases}
\end{multline}
\textit{где}
\begin{multline*}
%\label{g2f}
F_{2;n}(x) ={}\\
{}= F\left(\fr{x}{k};k,2r\right)
+ \fr{1}{n}\,\fr{(r-2)x}{2rk} \left(f\left(\fr{x}{k};k,2r\right) - {}\right.\\
\left.{}-
f\left( \fr{r-1}{rk}\,x;k,2r-2\right)\right) + 
\fr{k}{4(r(n-1)+1)}\times{}\\
{}\times \sum\limits_{j=0}^{2} a_j  F\left( \fr{r-1}{(k+2j)r} 
\,x;k+2j,2r-2\right) + {}\\
{}+ \fr{k}{4n(r(n-1)+1)} \times{}\\
{}\times 
\sum\limits_{j=0}^{2} a_j \! \left[ \fr{2-r}{2(r-1)} 
\,F\!\left( \fr{r-1}{r(k+2j)}\,x;k+2j,2r-2\!\right) +{} \right.\hspace*{-3.20157pt}
\end{multline*}

\noindent
\begin{multline*}
{}+\fr{r}{2(r-1)}\,F\left( \fr{r-2}{r(k+2j)} \,x;k+2j,2r-4\right)
-{}\\[2pt]
{}- \fr{(2-r)x}{2r(k+2j)} \,f\left( \fr{r-1}{r(k+2j)}\,x;k+2j,2r-2\right) - 
{}\\[2pt]
 {}- \fr{(r-2)x}{2(k+2j)(r-1)}\times{}\\[2pt]
\left.{}\times f\left( \fr{r-2}{r(k+2j)} 
\,x;k+2j,2r-4\right) \right]. 
%\end{alignedat}
\end{multline*}





На рисунке демонстрируется преимущество разложения 
Че\-бы\-шё\-ва--Эдж\-вор\-та второго порядка над разложением первого порядка в~приближении 
эмпирической функции распределения.



\section{Разложение Корниша--Фишера}
%\label{cor_fish}

Рассмотрим частный случай основного результата теоремы~3. Пусть в~\eqref{eq10q} 
$r\hm=3/2$. Обозначим $\mathbf{P}\left( g(n) \,\textrm{tr}\,S_h S^{-1}_{N_n} \leq x\right)
\hm = \bar{F}(x)$. Тогда

\noindent
\begin{equation}
\label{p_cheb}
\bar{F}(x) = F_{2; n}(x)|_{r=3/2} + \mathcal{O}(1/n^{3/2}), \enskip n \to \infty\,,
\end{equation}
где
\begin{multline*}
%\label{g2fr}
F_{2;n}(x) |_{r=3/2} = F\left(\fr{x}{k};k,3\right) - 
\fr{1}{n}\,\fr{x}{6k} \,f\left(\fr{x}{k};k,3\right) + {}\\
{} +\fr{1}{n}\,\fr{x}{6k} \,f\left( \fr{x}{3k};k,1\right) + {}\\
{}+\fr{k}{2(3n-1)} 
\sum\limits_{j=0}^{2} a_j  F\left( \fr{1}{(k+2j)3}\,x;k+2j,1\right).  
%\end{alignedat}
\end{multline*}

Домножим и~разделим третий и~четвертый член разложения на плот\-ность 
$f\left({x}/{k};k,3\right)$:
\begin{multline}
\label{cheb0}
F_{2;n}(x) |_{r=3/2} ={}\\
{}= F\left(\fr{x}{k};k,3\right) + \fr{1}{n}\,d_2(x) 
f\left(\fr{x}{k};k,3\right),
\end{multline}
где
\begin{multline*}
d_2(x) = - \fr{x}{6k} + \fr{x}{6k} \,\fr{f\left(  {x}/{(3k)};k,1\right)}
{f\left({x}/{k};k,3\right)} + {}\\
{}+
\fr{k}{2(3-1/n)} 
\sum\limits_{j=0}^{2} a_j  \fr{F\left( {x}/({(k+2j)3});k+2j,1\right)}{f\left({x}/{k};k,3\right)}\,.
\end{multline*}
Перепишем исходное выражение~\eqref{p_cheb}, используя~\eqref{cheb0}:
\begin{multline}
\label{cheb}
\bar{F}(x) = F\left(\fr{x}{k};k,3\right) + \fr{1}{n}\,d_2(x) 
f\left(\fr{x}{k};k,3\right) + {}\\
{}+\mathcal{O}\left(\fr{1}{n^{3/2}}\right), \quad
n \to \infty\,.
\end{multline}

Используя разложение Че\-бы\-шё\-ва--Эдж\-вор\-та~\eqref{cheb}~\cite[утверждение 2]{CMU}, 
вытекающее из более общих утверждений (см., например, работы~[7, гл.~5.6.1] и[8]), 
с~$a_1(x)\hm=0$, $a_2(x)\hm=d_2(x)$, 
$g(x)\hm=f({x}/{k};k,3)$, $G(x)\hm=F({x}/{k};k,3)$, получаем следующую тео\-ре\-му.

\smallskip

\noindent
\textbf{Теорема~4.}\
%\begin{theore}\label{cf_teor}
\textit{В условиях теоремы~$3$ пусть $x\hm=x_\alpha$, $u\hm=u_\alpha$~---
 $\alpha$-кван\-ти\-ли нормированной статистики $\mathbf{P}\left( g(n) \,\mathrm{tr}\,S_h S^{-1}_{N_n} 
 \hm\leq x\right)$ и~предельного $F$-рас\-пре\-де\-ле\-ния соответственно.
Тогда справедливо сле\-ду\-ющее асимптотическое разложение для}  $n \hm\to \infty$:
\begin{multline*}
%\label{eqCF1}
x=ku +  \fr{u}{6} \left(1-  \fr{f\left( u/3;k,1\right)}{f\left(u;k,3\right)} 
\right)n^{-1} - {}\\
{}=
\fr{k}{6} \sum\limits_{j=0}^{2} a_j  
\fr{F\left( {uk}/({(k+2j)3});k+2j,1\right)}{f\left(u;k,3\right)}\,n^{-1} + {}\\
{}+ \mathcal{O}(n^{- 3/2}).
\end{multline*}


\section{Доказательства}
%\label{proofs}



\noindent
Д\,о\,к\,а\,з\,а\,т\,е\,л\,ь\,с\,т\,в\,о\ \ тео\-ре\-мы~2.

 Используя формулу полной вероятности, получаем
\begin{multline}
\label{pr1_1}
\mathbb{P} \left( g(n) \,\mathrm{tr} \,S_h S^{-1}_{N_n} \leq x \right) = {}\\
{}=
\mathbb{E}\mathbb{P} \left( N_n \,\textrm{tr}\, S_h S^{-1}_{N_n} \leq 
\fr{N_n}{g(n)}\,x  \vert  N_n \right) = {} \\
{}= \sum\limits_{l=1}^{\infty} \mathbb{P} \left( l \,\mathrm{tr}\,S_h S^{-1}_{e} \leq 
\fr{l}{g(n)}\,x \right)  \mathbb{P}(N_n=l)\,.
 \end{multline}

Далее введем $F_n(x)$, подставив оценку~\eqref{con1} для статистики Хотеллинга 
из теоремы~1 без остаточного члена и~оценку~\eqref{con2} без 
остаточного члена для нормированной функции распределения размера выборки из 
условия~1 в~\eqref{pr1_1}:
\begin{multline}
\label{pr1_2}
F_n(x)={}\\
{}=\mathbb{P} \left( g(n) \,\mathrm{tr}\, S_h S^{-1}_{N_n} \leq x \right)
\stackrel{(\ref{con1})}{=}
 \mathbb{E}\left( 
 \vphantom{\sum\limits_{j=0}^{2}}
 G_k\left(\fr{N_n}{g(n)}\,x \right) + {}\right.\\
\left. {}+
\fr{k}{4N_n}\sum\limits_{j=0}^{2}a_j G_{k+2j}\left(\fr{N_n}{g(n)}\,x \right)\! \!\right) 
= \! \int\limits_{1/g(n)}^{\infty}\!\left( 
\vphantom{\sum\limits_{j=0}^{2}}
G_k\left(yx \right) + {}\right.\\
\left.{}+
\fr{k}{4g(n)y}\sum\limits_{j=0}^{2}a_j G_{k+2j}\left(yx \right) \right) 
d\mathbb{P}\left( \fr{N_n}{g(n)} < y \right)  \stackrel{(\ref{con2})}{=}{} \\
{}\stackrel{(\ref{con2})}{=} \int\limits_{1/g(n)}^{\infty}\left(
\vphantom{\sum\limits_{j=0}^{2}} 
G_k\left(yx \right) + 
\fr{k}{4g(n)y}\times{}\right.\\
\left.{}\times \sum\limits_{j=0}^{2}a_j G_{k+2j}\left(yx \right) 
\right) d\left( H(y) - \sum\limits_{i=1}^{m}\fr{1}{n^{i/2}}\, h_i(y) \right) 
={}\\
{}=\int\limits^{\infty}_{1/g(n)} G_k(xy)\,dH(y) + {}\\
{}+
\sum\limits_{i=1}^{m}\fr{1}{n^{i/2}}\int\limits^{\infty}_{1/g(n)} G_k(xy)\,dh_i(y) + {}
 \\
{}+ \fr{k}{4 g(n)} \int\limits_{1/g(n)}^\infty \sum\limits_{j=0}^{2} 
\fr{a_j}{y}\,G_{k+2j}(xy)\, dH(y)+\fr{k}{4 g(n)}\times{}\\
{}\times
 \sum\limits_{i=1}^{m}\fr{1}{n^{i/2}} \int\limits_{1/g(n)}^\infty 
\fr{1}{y} \sum\limits_{j=0}^{2}a_j G_{k+2j}(xy)\,dh_i(y)\,.
 \end{multline}

\noindent
Найдем теперь оценки для $\sup\nolimits_{x} \left| \mathbb{P}\left( g(n) \,\textrm{tr} \,S_h S^{-1}_{N_n}
\hm \leq\right.\right.$\linebreak $\left.\left.\leq x\right) \hm - F_{n}(x) \right|$. Введем обозначение
\begin{equation*}
\psi (n;y) =  \mathbb{P}\left( \fr{N_n}{g(n)} < y \right) - H(y) - 
\sum\limits_{i=1}^{m}\fr{1}{n^{i/2}}\, h_i(y)\,.
\end{equation*}

Согласно~\eqref{pr1_2},
\begin{multline*}
\sum\limits_{l=1}^{\infty} \left(
\vphantom{\sum\limits_{j=0}^{2}}
 G_k\left(\fr{l}{g(n)}\,x \right) +{}\right.\\
\left.{}+ 
\fr{k}{4l}\sum\limits_{j=0}^{2}a_j G_{k+2j}\left(\fr{l}{g(n)}x \right) \right)  
\mathbb{P}(N_n=l) ={} \\
{}= \int\limits_{1/g(n)}^{\infty}\left(
\vphantom{\sum\limits_{j=0}^{2}}
 G_k\left(yx \right) + {}\right.\\
\left.{}+
\fr{k}{4g(n)y}\sum\limits_{j=0}^{2}a_j G_{k+2j}\left(yx \right) \right) 
d\mathbb{P}\left( \fr{N_n}{g(n)} < y \right),
\end{multline*}
поэтому
\begin{equation*}
\sup\limits_{x} \left| \mathbb{P}\left(g(n) \,\mathrm{tr} \,S_h S^{-1}_{N_n} \leq 
x\right)  - F_{n}(x) \right| \leq I_{1n} + I_{n2},
\end{equation*}
где
\begin{align*}
I_{1n}&= \sup_{x} \left| \,  \int\limits_{1/g(n)}^{\infty}\left(
\vphantom{\sum\limits_{j=0}^{2}}
 G_k\left(yx \right) + {}\right.\right.\\
&\left.\left.{}+
\fr{k}{4g(n)y}\sum\limits_{j=0}^{2}a_j G_{k+2j}(yx) 
\right)  d\psi(n; y) 
\vphantom{\int\limits_{1/g(n)}^{\infty}}
\right|;\\
I_{2n}&= \sum\limits_{l=1}^{\infty} \sup\limits_{x} \left| 
\vphantom{\sum\limits_{j=0}^{2}}
 \mathbb{P} \left( l \,\mathrm{tr} \,
S_h S^{-1}_{l} \leq \fr{l}{g(n)}x \right) -\right.{}\\
&\left.{}- G_k\left(\fr{l}{g(n)}\,x \right) 
-  \fr{k}{4l}\sum\limits_{j=0}^{2}a_j G_{k+2j}\left(\fr{l}{g(n)}\,x \right) 
\right|\times{}\\
&\hspace*{52mm}{}\times \mathbb{P}(N_n=l).
\end{align*}

Используя формулу интегрирования по частям и~условие~1, получаем, что существует 
константа $C_3\hm>0$ такая, что
\begin{multline*}
I_{1n}\leq \fr{C_3}{n^{\beta}} + \sup\limits_{x}    \int\limits_{1/g(n)}^{\infty} \left| 
(\psi(n;y)\right| \left|\fr{\partial}{\partial y} \left(
\vphantom{\sum\limits_{j=0}^{2}}
 G_k\left(yx \right) + {}\right.\right.\\
\left.\left.{}+
\fr{k}{4g(n)y}\sum\limits_{j=0}^{2}a_j G_{k+2j}\left(yx \right) \right)\right| \,dy 
\leq \fr{C_3}{n^{\beta}} + \fr{C_2 M_n}{n^{\beta}}\,.
\end{multline*}

Используя результат теоремы~1, получаем
\begin{equation*}
I_{2n} \leq  \sum\limits_{l=1}^{\infty} \fr{C_1}{l^2}\,\mathbb{P}(N_n=l) = 
C_1\mathbb{E}N_n^{-2}. \enskip \square
\end{equation*}

%\vspace{3mm}
%\vskip12pt
\noindent
Д\,о\,к\,а\,з\,а\,т\,е\,л\,ь\,с\,т\,в\,о\ следствия~1 и~леммы~1 аналогичны доказательствам 
утверждения~1 и~леммы~1 из работы~\cite{CMU}.

 
\noindent
Д\,о\,к\,а\,з\,а\,т\,е\,л\,ь\,с\,т\,в\,о\ \ теоремы~3.

Обозначим
\begin{align*}
 J_1(x) &= \int\limits^{\infty}_{0} G_k(xy)dG_{r,r}(y)\,;\\
 J_2(x) &= {}\\
 &\hspace*{-9mm}{}= \!
\int\limits^{\infty}_{0} G_k(xy) \,d\left[  \fr{(y - 1)(2 - r) + 2 
Q_1\left(g(n)y\right)}{2 r}\,g_{r,r}(y) \right]\!;\hspace*{-0.60292pt}
 \\
 J_{3}(x) &=\int\limits_{0}^\infty \sum\limits_{j=0}^{2} \fr{a_j}{y}\, 
G_{k+2j}(xy)\,dG_{r,r}(y)\,;\\
 J_{4}(x) &= \int\limits_{0}^\infty \fr{1}{y} \sum\limits_{j=0}^{2}
 a_j \times{}\\
 &\hspace*{-9mm}{}\times G_{k+2j}(xy)\,d\left[  
\fr{(y - 1)(2 - r) + 2 Q_1(g(n)y)}{2 r}\,g_{r,r}(y) \right]\!.
 \end{align*}

Тогда общий вид функции~\eqref{gn2n} из следствия~1 запишется в~виде:
\begin{multline*}
%\label{gnst}
F_{2;n}(x) = J_1(x) + \fr{1}{n}\,J_2(x) +{}\\
{}+ \fr{k}{4g(n)} \,J_3(x) 
+ \fr{k}{4ng(n)}\,J_4(x).
\end{multline*}

Рассмотрим интеграл~$J_1(x)$:
\begin{multline*}
\fr{\partial}{\partial x}\, J_1(x) =  \int\limits^{\infty}_{0} y  g_k(xy)  
g_{r,r}(y)\,dy = {}\\
{}=
\fr{r^r\,x^{k/2-1}}{\Gamma(r)\,\Gamma(k/2)2^{k/2}}
\int\limits^{\infty}_{0} y^{r+k/2-1}e^{-(r+x/2)y}dy\,.
\end{multline*}


Используя формулу~2.3.3.1 из~\cite[с.~259]{Prudnikov}
\begin{equation*}
\int\limits^{\infty}_{0} x^{\alpha-1}e^{-px}\,dx=
\Gamma(\alpha)\,p^{-\alpha}, \enskip
\alpha,\,p > 0\,,
\end{equation*}
c $\alpha=r+k/2$ и~$p\hm=r\hm+x/2,$ получаем
\begin{equation*}
\fr{\partial}{\partial x}\,J_1(x) = \fr{r^r}{B(k/2,r)2^{k/2}}\,
\fr{x^{k/2-1}}{(x/2+r)^{k/2+r}}\,, \enskip x>0\,.
\end{equation*}
Тогда
\begin{multline}
\label{int1}
J_1(x) = \int\limits_{0}^{x} \fr{r^r}{B(k/2,r)2^{k/2}}\,
\fr{t^{k/2-1}dt}{(t/2+r)^{k/2+r}}  ={}\\
{}= \fr{1}{B(k/2,r)(2r)^{k/2}}  
\int\limits_{0}^{x}\fr{t^{k/2-1}dt}{(t/(2r)+1)^{k/2+r}} = {}\\
{}= \left\{  y = \fr{t}{k}, \enskip dt = k\,dy  \right\} = 
F\left(\fr{x}{k};k,2r\right).
\end{multline}


Обозначим
\begin{align*}
 J_{3_j}(x) &= \int\limits^{\infty}_{0} \fr{1}{y}\,G_{k+2j}(xy)\,dG_{r,r}(y), \
  j = 0,1,2; \\
   J_3(x)&= \sum\limits_{j=0}^{2} a_j  J_{3_j}(x).
 \end{align*}

По аналогии с $({\partial}/{\partial x})J_{1}$ для $({\partial}/{\partial x})J_{3_0}$ получаем
\begin{multline*}
\fr{\partial}{\partial x}\,J_{3_0}(x) = {}\\
{}=
\fr{r \Gamma(k/2+r-1)}{\Gamma(k/2) 
\Gamma(r) (2r)^{k/2}}\,x^{k/2-1}\left(1+\fr{x}{2r}\right)^{-(k/2+r-1)}\!.
\end{multline*}

Тогда
\begin{multline*}
J_{3_0}(x) =\int\limits_{0}^{x} \fr{r \Gamma(k/2+r-1)}{\Gamma(k/2) 
(r-1)\Gamma(r-1) (2r)^{k/2}}\times{}\\
{}\times t^{k/2-1}\left(1+\fr{t}{2r}\right)^{-(k/2+r-1)}dt ={}\\
{}= \left\{  y = \fr{r-1}{rk}\,t, \enskip dt = \fr{kr}{r-1}\,dy  \right\} = {}\\
{}=
\fr{r}{r-1} F\left( \fr{r-1}{rk} \,x;k,2r-2\right).
\end{multline*}
Аналогично получаем
\begin{align*}
J_{3_1}(x)&=\fr{r}{r-1} F\left( \fr{r-1}{r(k+2)} x;k+2,2r-2\right);  \\
J_{3_2}(x)&=\fr{r}{r-1} F\left( \fr{r-1}{r(k+4)}x;k+4,2r-2\right)
\end{align*}
и, таким образом,
\begin{multline}
J_3(x)={}\\
\!\!{}=\fr{r}{r-1}  \sum\limits_{j=0}^{2} a_j  F\left( \fr{r-1}{r(k+2j)} \,x;k+2j,2r-2\right).\!\!
\label{int3}
\end{multline}


Для вычисления~$J_2(x)$ используем интегрирование по частям:
\begin{multline}
\label{j2}
J_2(x) =- x\times{}\\
{}\times 
\int\limits^{\infty}_{0}\! g_k(xy)  \fr{(y - 1)(2 - r) + 2 
Q_1(g(n)y)}{2 r}g_{r,r}(y)\, dy = {}\\
{}= - \fr{x(2-r)}{2r}\, J_{2_1}(x) +  \fr{x(2-r)}{2r}\, J_{2_2}(x) -{}\\
{}- \fr{x}{r}\,J_{2_3}(x),
 \end{multline}
 где
 \begin{align*}
 J_{2_1}(x) &= \int\limits^{\infty}_{0}y g_k(xy)g_{r,r}(y)\, dy; \\ 
J_{2_2}(x) &= \int\limits^{\infty}_{0} g_k(xy)g_{r,r}(y) \,dy; \\
J_{2_3}(x) &= \int\limits^{\infty}_{0} g_k(xy)g_{r,r}(y)Q_1(g(n)y) \,dy.
 \end{align*}


Заметим, что 
$$
J_{2_1}(x) = \fr{\partial}{\partial x}\, J_1(x);\enskip 
J_{2_2}(x) = \fr{\partial}{\partial x}\, J_{3_0}(x).
$$

Рассмотрим третье сла\-га\-емое~$J_{2_3}(x)$:
\begin{multline*}
J_{2_3}(x) = \fr{r^r\,x^{k/2-1}}{\Gamma(r)\Gamma(k/2)2^{k/2}}\times{}\\
{}\times
\int\limits^{\infty}_{0} y^{r+k/2-2}
e^{-(r+x/2)y}Q_1(g(n)y)\,dy\,.
\end{multline*}

Применяя технику из доказательства теоремы~2 из работы~\cite{CMU}, для~$J^*_4(x)$ 
получаем:
$$
n^{-1}\, \left\vert J_{2_3}\right\vert \leq \fr{c(r,k)}{n^{r}} \sum\limits^{\infty}_{k=1} k^{-r}=
\fr{c_1(r,k)}{n^r}\,.
$$

Подставив выражения для~$J_{2_1}(x)$ и~$J_{2_2}(x)$ в~\eqref{j2}, получаем
\begin{multline}
\label{int2}
J_2(x) = \fr{(r-2)x}{2rk} \left(f\left(\fr{x}{k};k,2r\right) -{}\right.\\
\left.{}- f\left( 
\fr{r-1}{rk}\,x;k,2r-2\right)\right).
 \end{multline}


Обозначим
\begin{align*}
 J_{4_j}(x) &= \int\limits^{\infty}_{0} \fr{1}{y}\,G_{k+2j}(xy)\,dh_2(y)\,, \enskip j = 0,1,2\,; 
\\
J_4(x)&= \sum\limits_{j=0}^{2} a_j  J_{4_j}(x)
\end{align*}
и рассмотрим~$J_{4_0}$. Используя интегрирование по частям, получаем
\begin{multline}
\label{j40}
 \hspace*{-3pt}J_{4_0}(x) = -\!\int\limits^{\infty}_{0} \!\!\left( -\fr{1}{y^2}\,G_{k}(xy) + 
\fr{x}{y}\,g_{k}(xy)\!\right) h_2(y)\,dy ={}\\
{}= \int\limits^{\infty}_{0} \fr{1}{y^2}\,G_{k}(xy) h_2(y)\,dy - x \int\limits^{\infty}_{0} 
\fr{1}{y}\,g_{k}(xy) h_2(y)\,dy ={}\\
{}= J^{\prime}(x) - x J^{\prime\prime}(x).
\end{multline}

Заметим, что $J^{\prime\prime}(x) = ({\partial}/{\partial x}) J^{\prime}(x)$, поэтому 
достаточно рассмотреть $J^{\prime\prime}(x)$:
\begin{multline*}
 J^{\prime\prime}(x) = \int\limits^{\infty}_{0}\fr{1}{y}\,g_k(xy) \times{}\\
 {}\times \fr{(y - 1)(2 - 
r) + 2 Q_1(g(n)y)}{2 r}\,g_{r,r}(y)\,dy +{} \\
{}+ \fr{1}{r}  \int\limits^{\infty}_{0} \fr{1}{y} \,g_k(xy) g_{r,r}(y) 
Q_1(g(n) y)\,dy =  {}\\
{}=
\fr{2-r}{2r}\,J_1^{\prime\prime}(x) - \fr{2-r}{2r}\,J_2^{\prime\prime}(x) + 
\fr{1}{r}\,J_3^{\prime\prime}(x).
\end{multline*}

Заметим, что 
$$
J_1^{\prime\prime}(x) = \fr{1}{k}\,f\left( \fr{r-1}{rk}\, x;k,2r-2\right)
$$
и~оценка для~$J_3^{\prime\prime}(x)$ строится аналогично оценке для~$J_{2_3}$, но 
с~$\alpha \hm= r\hm+k/2\hm-2$. Рассмотрим~$J_2^{\prime\prime}(x)$:
\begin{multline*}
J_2^{\prime\prime}(x) ={}\\
{}= \fr{r^r\,x^{k/2-1}}{\Gamma(r)\Gamma(k/2)2^{k/2}}\int\limits^{\infty}_{0} y^{r+k/2-3}
e^{-(r+x/2)y}\,dy ={} \\
{}= \left\{  x = \fr{rk}{r-2}t \right\} = \fr{r}{(r-1)k}\,f\!\left( \fr{r-2}{rk}\,x;k,2r-4\right)\!,\hspace*{-3.6227pt}
\end{multline*}
откуда получаем
\begin{multline}
\label{j401}
 J^{\prime\prime}(x) =  \fr{2-r}{2rk} \,f\left( \fr{r-1}{rk}\,x;k,2r-2\right) +{}\\
 {}+ \fr{r-2}{2k(r-1)}\,f\left( \fr{r-2}{rk}\, x;k,2r-4\right);
\end{multline}

\vspace*{-12pt}

\noindent
\begin{multline}
\label{j402}
 J^{\prime}(x) =  \fr{2-r}{2(r-1)}\, F\left( \fr{r-1}{rk}\, x;k,2r-2\right) + {}\\
 {}+
\fr{r}{2(r-1)}\,F\left( \fr{r-2}{rk} \,x;k,2r-4\right).
\end{multline}
Подставляя~\eqref{j401} и~\eqref{j402} в~\eqref{j40}, получаем
\begin{multline*}
 J_{4_0}(x) =  \fr{2-r}{2(r-1)}\, F\left( \fr{r-1}{rk}\, x;k,2r-2\right) + {}\\
 {}+
\fr{r}{2(r-1)}\,F\left( \fr{r-2}{rk}\, x;k,2r-4\right) -{} \\
{}-   \fr{(2-r)x}{2rk}\, f\left( \fr{r-1}{rk}\, x;k,2r-2\right) -{}\\
{}- \fr{(r-2)x}{2k(r-1)}\,f\left( \fr{r-2}{rk}\, x;k,2r-4\right).
\end{multline*}

Аналогично получаем, что
\begin{multline*}
 J_{4_1}(x) =  \fr{2-r}{2(r-1)}\,F\left( \fr{r-1}{r(k+2)}\,x;k+2,2r-2\right) 
+ {}\\
{}+\fr{r}{2(r-1)}\,F\left( \fr{r-2}{r(k+2)} \,x;k+2,2r-4\right) -{}\\
{}-   \fr{(2-r)x}{2r(k+2)} \,f\left( \fr{r-1}{r(k+2)}\, x;k+2,2r-2\right)
 -{}\\
 {}- \fr{(r-2)x}{2(k+2)(r-1)}\,f\left( \fr{r-2}{r(k+2)}\,x;k+2,2r-4\right);
 \end{multline*}
 
 \vspace*{-12pt}
 
 \noindent
 \begin{multline*}
 J_{4_2}(x) =  \fr{2-r}{2(r-1)}\, F\left( \fr{r-1}{r(k+4)}\,x;k+4,2r-2\right) 
+ {}\\
{}+\fr{r}{2(r-1)}\,F\left( \fr{r-2}{r(k+4)}\, x;k+4,2r-4\right) -{}\\
{}-   \fr{(2-r)x}{2r(k+4)} \,f\left( \fr{r-1}{r(k+4)} \,x;k+4,2r-2\right)
 -{}\\
 {}- \fr{(r-2)x}{2(k+4)(r-1)}\,f\left( \fr{r-2}{r(k+4)} \,x;k+4,2r-4\right).
\end{multline*}
Таким образом,
\begin{multline}
 J_{4}(x) ={}\\
 {}= \sum\limits_{j=0}^{2} a_j  \!\left[ \fr{2-r}{2(r-1)} \,F\!\left( \fr{r-
1}{r(k+2j)}\,x;k+2j,2r-2\!\right)+{} \right.{}\hspace*{-0.42361pt}\\
{}+
\fr{r}{2(r-1)}\,F\left( \fr{r-2}{r(k+2j)} x;k+2j,2r-4\right) - {}\\
{}- \fr{(2-r)x}{2r(k+2j)} \,f\left( \fr{r-1}{r(k+2j)}\, x;k+2j,2r-2\right) - 
{}\\
{}-  \fr{(r-2)x}{2(k+2j)(r-1)}\times{}\\
\left.{}\times f\left( \fr{r-2}{r(k+2j)}\,x;k+2j,2r-
4\right) \right].
\label{int4}
\end{multline}

Объединяя $|1/g(n) -1/(r n)| \hm\leq \max\{2 , r\} (r\hm-1) (rn)^{-2} $,~\eqref{int1}, 
\eqref{int2}, \eqref{int3}, \eqref{int4} и~лемму~1, получаем~доказательство оценки \eqref{eq10q}.

\section{Заключение} %\label{concl}

Доказанный в~данной работе аналог теоремы переноса позволяет обобщить результаты 
работ~\cite{MMU16,CMU} для случая, когда статистика имеет распределения типа 
Хотеллинга случайного размера. Полученное в~работе асимптотическое разложение 
типа Че\-бы\-шё\-ва--Эдж\-вор\-та для функции распределения вышеупомянутой статистики 
позволяет построить разложения типа Кор\-ни\-ша--Фи\-ше\-ра для данной статистики.

\smallskip 

Автор выражает благодарность В.\,В.~Ульянову и~Г.~Кристофу за полезные обсуждения 
задачи.



{\small\frenchspacing
{%\baselineskip=10.8pt
%\addcontentsline{toc}{section}{References}
\begin{thebibliography}{9}

\bibitem{BenKorGal13}
\Au{Бенинг В.\,Е., Галиева~Н.\,К., Королев~В.\,Ю.} Асимптотические 
разложения для функций распределения статистик, построенных по выборкам 
случайного объема~// Информатика и~её применения, 2013. Т.~7. Вып.~2. С.~75--83.

\bibitem{BenKorGal12}
\Au{Бенинг В.\,Е., Галиева~Н.\,К., Королев~В.\,Ю.} Оценки скорости 
сходимости для функций распределения асимптотически нормальных статистик, 
основанных на выборках случайного объема~// Вестник Тверского гос. 
ун-та. Сер.: Прикладная математика, 2012. Т.~17. С.~53--65.

\bibitem{MMU16}
\Au{Марков А.\,С., Монахов~М.\,М., Ульянов~В.\,В.} Разложения типа Кор\-ни\-ша--Фи\-ше\-ра 
для распределений статистик, построенных по выборкам случайного размера~//  
Информатика и~её применения, 2016. Т.~10. Вып.~2. С.~84--91.

\bibitem{CMU}
\Au{Кристоф Г., Монахов~М.\,М., Ульянов~В.\,В.} Разложения Че\-бы\-ше\-ва--Эдж\-вор\-та
 и~Кор\-ни\-ша--Фи\-ше\-ра второго порядка для распределений статистик, 
построенных по выборкам случайного размера~// Записки научных семинаров ПОМИ, 
2017. Т.~466. С.~167--207.

\bibitem{UAF2016}
\Au{Ulyanov V.\,V., Aoshima~M., Fujikoshi~Y.} Non-asymptotic results for 
Cornish--Fisher expansions~// J.~Math. Sci., 2016. Vol.~218. 
No.\,3. P.~363--368.

\bibitem{FUS05} %6
\Au{Fujikoshi Y., Ulyanov~V.\,V., Shimizu~R.} $L_1$-norm error bounds for 
asymptotic expansions of multivariate scale mixtures and their applications to 
Hotelling's generalized~$T_0^2$~// J.~Multivariate Anal., 2005. 
Vol.~96. P.~1--19.



\bibitem{Fus10} %7
\Au{Fujikoshi Y., Ulyanov~V.\,V., Shimizu~R.} Multivariate statistics: High-dimensional and 
large-sample approximations.~--- Wiley ser. in probability and 
statistics. -- Hoboken, NJ, USA: Wiley, 2010. 568~p.

\bibitem{EncStat} %8
\Au{Ulyanov V.\,V.} Cornish--Fisher expansions~// International encyclopedia 
of statistical science~/ Ed. M.~Lovric.~--- Berlin: Springer, 2011. P.~312--315.

\bibitem{Prudnikov}
\Au{Прудников А.\,П., Брычков~Ю.\,А., Маричев~О.\,И.} Интегралы и~ряды.~--- М.: Наука, 1981. 
Т.~1. 259~с.
 \end{thebibliography}

}
}

\end{multicols}

\vspace*{-3pt}

\hfill{\small\textit{Поступила в~редакцию 22.06.2020}}

\vspace*{8pt}

%\pagebreak

%\newpage

%\vspace*{-28pt}

\hrule

\vspace*{2pt}

\hrule

%\vspace*{-2pt}

\def\tit{CHEBYSHEV--EDGEWORTH EXPANSIONS FOR~DISTRIBUTIONS OF~GENERALISED HOTELLING-TYPE STATISTICS 
BASED~ON~RANDOM SIZE SAMPLES}


\def\titkol{Chebyshev--Edgeworth expansions for~distributions of~generalised hotelling-type statistics based 
on~random size samples}

\def\aut{M.\,M.~Monakhov}

\def\autkol{M.\,M.~Monakhov}


\titel{\tit}{\aut}{\autkol}{\titkol}

\vspace*{-11pt}


\noindent
Moscow Center for Fundamental and Applied Mathematics, M.\,V.~Lomonosov Moscow State University, 
\mbox{1-52}~Leninskie Gory, GSP-1, Moscow 119991, Russian Federation

 
\def\leftfootline{\small{\textbf{\thepage}
\hfill INFORMATIKA I EE PRIMENENIYA~--- INFORMATICS AND
APPLICATIONS\ \ \ 2021\ \ \ volume~15\ \ \ issue\ 2}
}%
\def\rightfootline{\small{INFORMATIKA I EE PRIMENENIYA~---
INFORMATICS AND APPLICATIONS\ \ \ 2021\ \ \ volume~15\ \ \ issue\ 2
\hfill \textbf{\thepage}}}

\vspace*{3pt}



\Abste{The general transfer theorem for the distribution function of asymptotically normal
 statistics was generalized on the Hotelling-type statistics case and analog of general 
 transfer theorem for the distribution function of Hotelling-type statistics with 
 random size was proved. It allowed to obtain the Chebyshev--Edgeworth expansion 
 for initial Hotelling-type statistics. The explicit form of the Chebyshev--Edgeworth 
 expansion was obtained for the case when the random sample size distribution is the negative 
 binomial distribution shifted by 1. The limit distribution for this case was F-distribution. 
 The Cornish--Fisher expansion was obtained for the special case of parameter of 
 random sample size. The computational experiment was conducted and graphs were plotted 
 for Chebyshev--Edgeworth expansion illustration.}
 
\KWE{generalised Chebyshev--Edgeworth expansion; Cornish--Fisher expansion; sample with random size; 
F-disribution; Hotelling-type statstics}



\DOI{10.14357/19922264210211}

%\vspace*{-15pt}

 \Ack
\noindent
The research was conducted in accordance with the program of the Moscow Center for Fundamental 
and Applied Mathematics.

%\vspace*{12pt}

  \begin{multicols}{2}

\renewcommand{\bibname}{\protect\rmfamily References}
%\renewcommand{\bibname}{\large\protect\rm References}

{\small\frenchspacing
 {%\baselineskip=10.8pt
 \addcontentsline{toc}{section}{References}
 \begin{thebibliography}{9}
\bibitem{1-m}
\Aue{Bening, V.\,E., N.\,K.~Galieva, and V.\,Yu.~Korolev.}
 2013. Asimptoticheskie razlozheniya dlya funktsiy raspredeleniya statistik, 
 postroennykh po vyborkam sluchaynogo ob''ema [Asymptotic expansions for the distribution functions 
 of statistics constructed from samples with random sizes]. 
 \textit{Informatika i~ee Primeneniya~--- Inform. Appl.} 7(2):75--83.
\bibitem{2-m}
\Aue{Bening, V.\,E., N.\,K.~Galieva, and V.\,Yu.~Korolev.}
 2012. Otsenki skorosti skhodimosti dlya funktsiy raspredeleniya asimptoticheski normal'nykh 
 statistik, osnovannykh na vyborkakh sluchaynogo ob''ema [On rate of convergence in distribution 
 of asymptotically normal statistics based on samples of random size]. 
 \textit{Vestnik Tverskogo gos. un-ta. Ser.\ Prikladnaya matematika} 
 [Bull. of the Tverskoy State University. Ser.\ Appl. Math.] 17:53--65.
\bibitem{3-m}
\Aue{Markov, A.\,S., M.\,M.~Monakhov, and V.\,V.~Ulyanov.}
 2016. Razlozheniya tipa Kornisha--Fishera dlya raspredeleniy statistik, 
 postroennykh po vyborkam sluchaynogo razmera [Generalized Cornish--Fisher expansions 
 for distributions of statistics based on samples of random size]. 
  \textit{Informatika i~ee Primeneniya~--- Inform. Appl.} 10(2):84--91.
\bibitem{4-m}
\Aue{Christoph, G., M.\,M.~Monakhov, and V.\,V.~Ulyanov.}
 2017. Razlozheniya Chebysheva--Edzhvorta i~Kornisha--Fishera vtorogo poryadka dlya raspredeleniy 
 statistik, postroyennykh po vyborkam sluchaynogo razmera [Second order Chebyshev--Edgeworth and 
 Cornish--Fisher expansions for distributions of statistics constructed from samples with random sizes]. 
  \textit{Zapiski nauchnykh seminarov POMI} [POMI Notes of Scientific Seminars] 466:167--207.
  {\looseness=1
  
  }
\bibitem{5-m}
\Aue{Ulyanov, V.\,V., M.~Aoshima, and Y.~Fujikoshi.}
 2016. Non-asymptotic results for Cornish--Fisher expansions. 
 \textit{J.~Math. Sci.} 218(3):363--368.
\bibitem{6-m}
\Aue{Fujikoshi, Y., V.\,V.~Ulyanov, and R.~Shimizu.}
 2005. $L_1$-norm error bounds for asymptotic expansions of multivariate scale mixtures and their 
 applications to Hotelling's generalized $T_0^2$.  \textit{J.~Multivariate Anal.} 96:1--19.

\bibitem{8-m}
\Aue{Fujikoshi, Y., V.\,V.~Ulyanov, and R.~Shimizu.}
 2010.  \textit{Multivariate statistics: High-dimensional and large-sample approximations}. 
 Wiley ser. in probability and statistics.
 Hoboken, NJ: Wiley. 568~p.
 
 \bibitem{7-m}
\Aue{Ulyanov, V.\,V.}
 2011.  Cornish--Fisher expansions.  \textit{International encyclopedia of statistical science}. 
 Ed. M.~Lovric. Berlin: Springer. 312--315.
 
\bibitem{9-m}
\Aue{Prudnikov, A.\,P., Yu.\,A.~Brychkov, and O.\,I.~Marichev.}
 1992.  \textit{Integraly i~ryady} 
 [Integrals and series]. Moscow: Nauka. Vol.~1. 259~p.
 \end{thebibliography}

 }
 }

\end{multicols}

\vspace*{-3pt}

  \hfill{\small\textit{Received June~22, 2020}}


%\pagebreak

%\vspace*{-8pt}  

\Contrl

\noindent
\textbf{Monakhov Mikhail M.} (b.\ 1993)~--- 
laboratory assistant, Moscow Center for Fundamental and Applied Mathematics, 
M.\,V.~Lomonosov Moscow State University, 1-52~Leninskie Gory, GSP-1, Moscow 119991, Russian Federation; 
\mbox{mih\_monah@mail.ru}


\label{end\stat}

\renewcommand{\bibname}{\protect\rm Литература}