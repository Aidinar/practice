%\def\lt{\;\hbox{$<$}\discretionary{}{\hbox{$<$}}{}\;}
%\def\gt{\;\hbox{$>$}\discretionary{}{\hbox{$>$}}{}\;}
%\def\eq{\;\hbox{$=$}\discretionary{}{\hbox{$=$}}{}\;}
%\def\le{\;\hbox{$\leqslant$}\discretionary{}{\hbox{$\leqslant$}}{}\;}
%\let\leq=\leqslant
%\def\ge{\;\hbox{$\geqslant$}\discretionary{}{\hbox{$\geqslant$}}{}\;}
%\let\geq=\geqslant
%\def\equ{\;\hbox{$\equiv$}\discretionary{}{\hbox{$\equiv$}}{}\;}
%\def\pls{\;\hbox{$+$}\discretionary{}{\hbox{$+$}}{}\;}

\def\stat{suchkov}

\def\tit{АЛГОРИТМЫ СЖАТИЯ ДАННЫХ МАССИВОВ СИЛОВЫХ КРИВЫХ~I: КОДИРОВАНИЕ ОШИБОК ПРЕДСКАЗАНИЯ}

\def\titkol{Алгоритмы сжатия данных массивов силовых кривых~I: кодирование ошибок предсказания}

\def\aut{Д.\,В.\,Сушко$^1$}

\def\autkol{Д.\,В.\,Сушко}

\titel{\tit}{\aut}{\autkol}{\titkol}

\index{Сушко Д.\,В.}
\index{Sushko D.\,V.}

%{\renewcommand{\thefootnote}{\fnsymbol{footnote}} \footnotetext[1]
%{Работа выполнена при частичной поддержке РФФИ (проект 19-07-00187 А) 
%и~в~соответствии с~программой Московского центра фундаментальной и~прикладной математики.}}


\renewcommand{\thefootnote}{\arabic{footnote}}
\footnotetext[1]{Институт проблем информатики Федерального исследовательского центра 
<<Информатика и~управление>> Российской академии наук, \mbox{dsushko@ipiran.ru}}


\vspace*{-6pt}



\Abst{Рассмотрена задача обратимого (без потерь) сжатия данных массивов силовых кривых~--- 
трехмерных массивов, элементы которых суть 16-бит\-ные целые числа. 
Такие массивы являются результатом сканирования микрообъектов на атом\-но-си\-ло\-вом 
микроскопе (АСМ) в~режиме измерения силовых карт. Предложены алгоритмы обратимого сжатия массивов 
силовых кривых, основанные на универсальном арифметическом кодировании ошибок их предсказания. 
Применены два метода универсального кодирования. Первый основан на использовании статистической 
модели источника с~вычислимой последовательностью состояний и~предполагает разложение всей 
последовательности ошибок предсказания на две независимо кодируемые подпоследовательности. 
Второй предполагает выбор подходящего веса при построении используемых в~арифметическом кодировании 
кодовых вероятностей. Для предложенных алгоритмов на пяти тестовых массивах построены оценки 
скорости кодирования. Результаты показывают, что использование комбинации упомянутых выше
 методов универсального кодирования позволяет заметно уменьшить скорость кодирования. 
 Скорости кодирования тестовых массивов наиболее эффективным алгоритмом среди предложенных 
 практически применимых алгоритмов составили 3,9285, 3,5268, 3,5024, 4,2813 и~4,2246~бит/пик\-сель.}

\KW{атомно-силовой микроскоп; массив силовых кривых; обратимое сжатие; арифметическое кодирование; 
универсальное кодирование}

\DOI{10.14357/19922264210212}

\vspace*{4pt}


\vskip 10pt plus 9pt minus 6pt

\thispagestyle{headings}

\begin{multicols}{2}

\label{st\stat}

\section{Введение}

\vspace*{-4pt}

%\label{sec1}
В настоящее время широко распространенным методом исследования микрообъектов (например, клеток,
 вирусов, белков, нуклеиновых кислот и~др.\ в~микробиологии) стало их сканирование на 
АСМ в~режиме измерения силовых карт. Результатом таких 
 исследований являются массивы силовых кривых~-- трехмерные массивы данных большого объема 
 (в~типичном случае $\sim70$~МБ). Необходимость долгосрочного хранения и~передачи по каналам 
 связи таких данных делает задачу их сжатия весьма актуальной, а поскольку получение данных 
 связано с~проведением трудоемкого и~длительного эксперимента, потери при сжатии недопустимы, т.\,е.\
  сжатие должно быть обратимым (без потерь).

Разработка алгоритмов обратимого сжатия данных массивов силовых кривых представляет 
интерес с~теоретической точки зрения, так как распространенные алгоритмы обратимого 
сжатия \mbox{ориентированы} главным образом на сжатие данных одного из двух типов: текст 
или изображение, а~массивы силовых кривых, очевидным образом, не относятся ни к~одному из этих типов.

Исследование задачи обратимого сжатия данных массивов силовых кривых было начато в~работе~\cite{b1}, 
где были определены потенциальные возможности некоторых алгоритмов сжатия. 
В~качестве экспериментальных данных в~работе были\linebreak использованы массивы, полученные при 
сканировании мягких биологических образцов в~режиме измерения силовых карт на микроскопе MultiMode~V 
(Veeco, США). В~число рассмотренных\linebreak алгоритмов вошли стандартные алгоритмы (\mbox{DEFLATE}, JPEG~2000) 
и~предложенные в~\cite{b1} простые алгоритмы на основе арифметического кодирования. 
Были получены оценки скорости кодирования этих алгоритмов. Напомним, что 
\textit{ско\-ростью кодирования}~$R$ алгоритма называется отношение длины кодового слова~$L$ 
(в~битах), порождаемого алгоритмом для описания массива данных, к~полному числу элементов 
(пикселей)~$N$ этого массива; единица измерения скорости кодирования~-- бит/пик\-сель (бт/п). 
При этом коэффициент сжатия равен отношению длины элемента массива в~битах к~ско\-рости кодирования.

Цель настоящей работы~--- предложить и~исследовать более сложные алгоритмы обратимого сжатия, 
основанные на универсальном арифметическом кодировании ошибок предсказания. 
Скорость кодирования алгоритмов оценивается на тех же экспериментальных данных, что и~в~работе~\cite{b1}, 
что позволяет сравнивать получаемые результаты непосредственно. Вычисления проводятся 
программами, написанными на языке Python.
{\looseness=1

}

\section{Массивы силовых кривых}

%\label{sec2}
Кратко рассмотрим вопросы, связанные с~процессом сканирования и~структурой массивов 
силовых кривых. Более подробное изложение приведено в~\cite{b1}, детальное описание 
технологии \mbox{измерений} с~по\-мощью АСМ и~интерпретации со\-от\-вет\-ст\-ву\-ющих данных можно найти, 
например, в~\cite{b2}.

Принцип работы АСМ заключается в~сканировании поверхности образца атомарно острой иглой 
(зондом), которая является частью гибкого кронштейна (кантилевера), закрепленного на 
пьезоэлектрическом двигателе. Силы, действующие на зонд со стороны поверхности, 
вызывают изгиб кантилевера, что приводит к~перераспределению лазерного сигнала на 
фотодетекторе. Регистрируя величину этого сигнала и~зная жесткость кантилевера, 
можно определить силу взаимодействия зонда с~поверхностью.

Современные АСМ позволяют работать в~режиме измерения силовых кривых и~силовых карт. 
Силовая кривая представляет собой график зависимости силы взаимодействия зонда и~поверхности 
образца от расстояния между ними. При фиксированном положении зонда в~плоскости образца 
снимаются две кривые: кривая подвода (зонд приближается к~образцу) и~кривая отвода 
(зонд удаляется от образца). Пары силовых кривых снимаются для множества точек в~поле наблюдения. 
В~результате формируется карта силовых кривых~--- трехмерный массив данных.

Введем некоторые обозначения. Пусть $OXYZ$~--- трехмерная декартова система координат. 
Твердая подложка образца располагается в~горизонтальной плоскости~$OXY$, 
поле наблюдения представляет собой прямоугольник в~этой плоскости. Силовые кривые измеряются 
в~узлах $(x,y)\hm\equiv (x_i,y_j)$ равномерной прямоугольной решетки в~поле наблюдения, 
$i\hm=0,1,\ldots,I-1$, $j\hm=0,1,\ldots,J-1$. Перебор узлов решетки осуществляется 
в~следующем порядке: сначала по ширине (в~на\-прав\-ле\-нии~$OY$), затем по длине (в~на\-прав\-ле\-нии~$OX$). 
В~каждом узле $(x,y)$ решетки измеряется пара силовых кривых 
$F^\text{A}_{(x,y)}(z)$ (кривая подвода) и~$F^\text{R}_{(x,y)}(z)$ 
(кривая отвода), $z\hm\equiv z_k$, $k\hm=0,1,\ldots,K-1$. 
Шаг по вертикали равномерный, высота~$z$ отсчитывается от поверхности образца в~узле $(x,y)$.

Значения элементов силовых кривых $F^\text{A,R}_{(x,y)}(z)$ пропорциональны силе, 
действующей на зонд в~точке пространства, находящейся на расстоянии~$z$ 
от поверхности образца и~имеющей координаты $(x,y)$ в~поле наблюдения, в~процессе 
подвода зонда к~образцу и~отвода зонда от образца. Значения~$F$ записываются в~виде 16-бит\-ных 
целых чисел, т.\,е.\ целых чисел в~диапазоне $[-2^{15},2^{15}-1]$. Увеличение значения~$F$ 
отвечает увеличению отталкивания (уменьшению притяжения) между зондом и~поверхностью. 
При измерении кривой подвода $F^\text{A}_{(x,y)}(z)$ зонд может достичь поверхности 
образца до того, как будет зарегистрировано необходимое число ($K$) значений.
 В~таком случае осуществляется дополнение соответствующей строки до требуемой длины ($K$) 
 минимально возможным значением~$-2^{15}$, записываемым в~конец строки.

Для всего массива силовых кривых будем использовать обозначение 
\begin{multline*}
\mathbf{V}=\{V(i,j,k)\},\enskip i\hm= 0,1,\dots,I-1\,,\\
j=0,1,\dots,J-1\,,\enskip
 k=0,1,\dots,2K-1\,. 
 \end{multline*}
При этом
\begin{multline*}
V(i,j,k) ={}\\
{}= \begin{cases}
F^\text{A}_{(x_i,y_j)}(z_k)\,,        & k=0,1,\ldots,K-1\,; \\[6pt] 
F^\text{R}_{(x_i,y_j)}(z_{k-K})\,, & k=K,K+1,\ldots,2K-1\,.\hspace*{-3pt} 
\end{cases}
\end{multline*}

В качестве экспериментальных данных используются пять массивов силовых кривых (I--V), 
полученных при сканировании образцов, пред\-став-ляющих собой абсорбированные из раствора 
на\linebreak твердую подложку вирусы. 

Первый образец~(I)~--- это риновирус~2 на подложке из слюды, 
второй и~третий образцы (II и~III)~--- вирус мягкой мозаики ячменя на подложке из слюды, 
четвертый и~пятый образцы (IV и~V)~--- вирус табачной мозаики на подложке из стекла. 
Массивы силовых кривых имеют сле\-ду\-ющие размеры: $I\hm=J\hm=64$, $K\hm=4096$ (массив~I);\linebreak  
$I\hm=J\hm=128$, $K\hm=1024$ (массивы~II--V). Полное число элементов всех массивов равно~$2^{25}$. 
Каж-\linebreak дый элемент представляет собой 16-бит\-ное целое\linebreak число.
{ %\looseness=1

}


В качестве иллюстрации на рисунке представлены пары силовых кривых образца~II 
в~двух разных узлах поля наблюдения. Кривые подвода изображены черным цветом, 
кривые отвода~--- серым. На фрейме~(\textit{б}) 
представлена кривая подвода, при измерении которой зонд достиг поверхности образца до того, 
как было зарегистрировано необходимое число ($K\hm=1024$) значений, и~поэтому в~конец строки 
было записано нужное число минимальных значений~$(-2^{15})$.

\begin{figure*}
  \vspace*{1pt}
  \begin{center}
    \mbox{%
 \epsfxsize=161.251mm 
 \epsfbox{suc-1.eps}
 }


\vspace*{6pt}

{\small Силовые кривые образца~II}
\end{center}
\end{figure*} 


\section{Алгоритмы кодирования ошибок предсказания}
%\label{sec3}

Общепринятый метод решения задач обратимого сжатия цифровых данных заключается в~применении к~исходному 
массиву данных некоторых обратимых преобразований, обеспечивающих декорреляцию его 
отсчетов и/или уменьшение диапазона их значений, и~последующего арифметического кодирования~\cite{b3} 
полученного таким образом массива как последовательности независимых отсчетов.

При арифметическом кодировании конечной числовой последовательности $\mathbf{x}\hm=
\{x_n\}$, $n\hm= 0,1,\ldots,N-1$, принимающей значения в~априори известном диапазоне~$\mathfrak{A}$, 
множество \textit{условных кодовых распределений вероятностей} (или просто \textit{кодовых распределений})
$\{q_n(a)\hm= q_n(a\vert x_{n-1},\dots,x_0)$, $a\hm\in\mathfrak{A} \}$
используется для того, чтобы приписать последовательности~$\mathbf{x}$  кодовую 
вероятность~$Q(\mathbf{x})$ и~кодовое слово (результат сжатия) длины
\begin{multline}
\label{eq01}
L(\mathbf{x}) = \left\lceil -\log_2 \left(\fr{Q(\mathbf{x})}{2}\right)  \right\rceil ={}\\
{}=
\left\lceil -\log_2 \prod\limits_{n=0}^{N-1}q_n(x_n)  + 1 \right\rceil \leq{}\\
{}\leq
\sum\limits_{n=0}^{N-1} -\log_2 q_n(x_n) + 2\,.
\end{multline}
Здесь $\lceil\cdot\rceil$~--- результат округления вещественного чис\-ла до 
ближайшего целого вверх. При этом построение кодовых распределений, 
обес\-пе\-чи\-ва\-ющих получение возможно более коротких кодовых слов при 
неизвестной статистике,~--- задача универсального кодирования~\cite{b4}. Отметим, 
что вос\-ста\-нов\-ле\-ние исходной последовательности по кодовому слову осуществляется в~процессе 
декодирования без задержки. Это означает, что в~момент восстановления очередного значения~$x_n$ 
декодеру уже известны все предыдущие значения $\{x_0,\dots,x_{n-1}\}$ и~кодовые распределения 
могут быть построены декодером так же, как они были ранее построены кодером в~процессе кодирования. 
Это позволяет декодеру восстановить значение~$x_n$. 

В настоящей работе кодовые распределения строятся следующим образом:
\begin{multline}
\label{eq02}
q_n(a\vert x_{n-1},\dots,x_0) ={}\\
{}= \begin{cases}
 \fr{1+w\theta_n(a)}{a_{\max}(\mathbf{x})-a_{\min}(\mathbf{x})+1+wn} &
\mbox{ при } a\in{}\\
& \hspace*{-55pt}\in[a_{\min}(\mathbf{x}), a_{\max}(\mathbf{x})]\,; \\
0& \hspace*{-55pt}\mbox{в~противном случае}\,,
\end{cases}
\end{multline}
где $\theta_n(a)$~--- число элементов, принимающих значение~$a$, на
 начальном участке по\-сле\-до\-ва\-тель\-ности~$\mathbf{x}$ до $(n-1)$-го члена включительно; 
 $a_{\min}(\mathbf{x})$ и~$a_{\max}(\mathbf{x})$~--- нижняя и~верхняя границы диапазона 
 значений последовательности~$\mathbf{x}$; параметр $w\hm=1,2,\dots$~--- вес. В~частном случае 
 $w\hm=2$ формула~(\ref{eq02}) превращается в~соответствующую формулу работы~\cite{b1}. 
 Границы  $a_{\min}(\mathbf{x})$ и~$a_{\max}(\mathbf{x})$ могут быть вычислены
  кодером и~должны быть переданы декодеру помимо кодового слова, что, 
  вообще говоря, несколько увеличивает скорость кодирования. 
  Однако кодируемые в~работе массивы имеют~$2^{25}$~элементов, а для передачи значения 
  одной границы требуется $2^5$~бит; соответствующее увеличение скорости кодирования 
  составляет~$2^{-20}$~бт/п~--- величину, которой можно пренебречь.

В настоящей работе скорость кодирования оценивается по формуле
\begin{equation}
\label{eq03}
R(\mathbf{x}) \doteq \fr{1}{N}\,L(\mathbf{x}) =
\fr{1}{N} \sum\limits_{n=0}^{N-1}-\log_2 q_n(x_n)\,,
\end{equation}
вытекающей из (\ref{eq01}), если пренебречь бесконечно малым членом~$2N^{-1}$.

Величины $\theta(x)/N$, $x\hm\in\mathfrak{A}$, где $\theta(x)$~--- число элементов,
 принимающих значение~$x$, в~последовательности~$\mathbf{x}$, образуют \textit{частотное} (или 
 \textit{эмпирическое}) распределение вероятностей значений последовательности. Величина
\begin{equation}
\label{eq04}
H(\mathbf{x}) =  \sum\limits_{x\in\mathfrak{A}}
\fr{\theta(x)}{N} \left[ -\log_2 \fr{\theta(x)}{N}\right]
\end{equation}
(используется соглашение о том, что $0\cdot\log_20\hm=0$) называется 
\textit{квазиэнтропией} по\-сле\-до\-ва\-тель\-ности. Единица измерения квазиэнтропии~--- 
бт/п. Квазиэнтропия зависит только от самой последовательности и~представляет 
собой нижнюю границу скорости арифметического кодирования (см., например,~\cite{b4}). 
Разность $R\hm-H\hm\ge 0$~--- \textit{избыточность} арифметического кодирования. 
Величина избыточности характеризует качество решения одной из задач универсального кодирования~--- 
задачи построения кодовых распределений.

Арифметическое кодирование многомерного массива данных требует их одномерного 
упорядочения. В~работе принято так называемое \mbox{строчное} упорядочение, при котором 
трехмерный массив~$\mathbf{X}$ размерами $(I,J,K)$ превращается в~последовательность~$\mathbf{x}$   
размером $N\hm= I J K$ так, что $x(n)\hm=X(i,j,k)$, $n\hm=J K i \hm+ K j +k$.

В работе исследуются алгоритмы, ис\-поль\-зу\-ющие в~качестве основного декорреляционного 
преобразования переход к~ошибкам предсказания и~два предваряющих его преобразования, 
которые увеличивают неравномерность распределения и~уменьшают диапазон значений данных.

Первое преобразование заключается в~разделении массива силовых кривых~$\mathbf{V}$ 
на массивы кривых подвода~$\mathbf{V}^{\text A}$ и~отвода~$\mathbf{V}^{\text R}$:  
$\mathbf{V}\hm\to\{\mathbf{V}^{\text A},\mathbf{V}^{\text R}\}$,
\begin{multline}
V^{\text A}(i,j,k) = V(i,j,k), \\
V^{\text R}(i,j,k) = V(i,j,K+k), \\
k = 0,1,\ldots,K-1\,.
\label{eq05}
\end{multline}

Второе преобразование применяется к~полученному массиву кривых 
подвода~$\mathbf{V}^{\text A}$ и~заключается в~сужении диапазона
 значений этого массива, непомерно широкого из-за наличия значения~$-2^{15}$, 
 используемого для дополнения <<неполных>> строк:  $\mathbf{V}^{\text A}\hm\to  \bar{\mathbf{V}}^{\text A}$,
\begin{equation}
\label{eq06}
\bar{V}^{\text A} = \begin{cases}
 V^{\text A}_{\text d\min} - 1, &\ V^{\text A}=-2^{15}\,;\\
 V^{\text A}, &\ V^{\text A}>-2^{15}\,,
\end{cases}
\end{equation}
где $V^{\text A}_{\text d\min}$~--- динамический минимум значений массива~$\mathbf{V}^\text{A}$ 
(т.\,е.\ минимум значений без учета значения~$-2^{15}$). Чтобы обратить преобразование 
сужения диапазона~(\ref{eq06}), декодеру нужна информация о~том, встречалось 
ли значение~$-2^{15}$ в~исходном массиве кривых подвода. Для передачи декодеру этой известной 
кодеру информации требуется один бит, и~соответствующее увеличение скорости
 кодирования пренебрежимо мало.

Переход к~ошибкам предсказания $\mathbf{X}\to\mathbf{D}$ 
для данного трехмерного массива $\mathbf{X}\hm=\{X(i,j,k)\}$ размерами $(I,J,K)$ имеет вид:
\begin{multline}
D(i,j,k) ={}\\
\!\!{}=
\begin{cases}
X(0,0,0),                          & i=j=k=0; \\ 
X(i,0,0)-X(i-1,0,0), & i=1,\dots,I-1,\\
& j=k=0; \\
X(i,j,0)-X(i,j-1,0),  & i=1,\dots,I-1,\\
& \hspace*{-39pt}j=1,\dots,J-1,\ k=0; \\
X(i,j,k)-X(i,j,k-1),  & i=1,\dots,I-1,\\
& \hspace*{-85pt}j=1,\dots,J-1,\ k=1,\dots,K-1.
\end{cases}
\!\!\label{eq07}
\end{multline}
Преобразование применяется к~каждому из двух полученных в~результате предварительной 
обработки массивов: $\bar{\mathbf{V}}^\text{A}\to\mathbf{D}^\text{A}$; 
$\mathbf{V}^\text{R}\to\mathbf{D}^\text{R}$.

Все исследуемые в~работе алгоритмы используют преобразования (\ref{eq05})--(\ref{eq07}) и~различаются 
применяемыми методами универсального кодирования. Отметим, что использование этих 
преобразований обеспечило наименьшую скорость кодирования массивов силовых кривых 
среди рассмотренных в~\cite{b1} алгоритмов, основанных на кодировании ошибок предсказания.

Начнем с~рассмотренного в~\cite{b1} алгоритма A[1$\vert$0], который осуществляет 
независимое арифметическое кодирование массивов $\mathbf{D}^\text{A}$ и~$\mathbf{D}^\text{R}$ 
с~использованием построенных по формуле~(\ref{eq02}) с~весом $w\hm=2$ кодовых распределений. 
В~строке~1 табл.~\ref{tab1} приведена квазиэнтропия
$$
H[1]\doteq H(\{\mathbf{D}^\text{A}, \mathbf{D}^\text{R} \}) = 
\fr{1}{2}\left[H(\mathbf{D}^\text{A}) + H(\mathbf{D}^\text{A})\right]
$$
ошибок предсказания для массивов~I--V, а~в~строке~4~--- ско\-рость кодирования
$$
R[1\vert 0]\doteq R(\{\mathbf{D}^\text{A}, \mathbf{D}^\text{R} \}) = 
\fr{1}{2}\left[R(\mathbf{D}^\text{A}) + R(\mathbf{D}^\text{A})\right]
$$
алгоритма. Значения квазиэнтропии и~скорости кодирования в~табл.~1 
приводятся в~единицах бт/п с~точностью до четырех знаков после десятичной запятой. 
Избыточность кодирования алгоритма составляет 0,0009--0,0022~бт/п в~зависимости от массива.


\begin{table*}\small %[t]
\begin{center}
\Caption{Квазиэнтропия и~скорость кодирования}
\label{tab1}
\vspace{2ex}

\begin{tabular}{|c|l|c|c|c|c|c|}
\hline
№   & \multicolumn{1}{c|}{Величина}  & I          & II         & III        &   IV      & V \\
\hline
1   & $H$[1]      & 3,9496    & 3,6922    & 3,6945    & 4,3091    & 4,2454  \\
2   & $H$[2,opt]                    & 3,9256  & 3,5261  & 3,5015    & 4,2806    & 4,2239 \\
3   & $H$[2,fix]                            & 3,9281    & 3,5261    & 3,5015    & 4,2806    & 4,2239 \\
\hline
4   & $R[1\vert 0]$  & 3,9504    & 3,6944    & 3,6959    & 4,3103    & 4,2463 \\
5   & $R[1\vert \mathrm{opt}]$                    & 3,9498    & 3,6927    & 3,6951    & 4,3096    & 4,2458 \\
6   & $R[1\vert \mathrm{fix}]$                            & 3,9498    & 3,6928    & 3,6951    & 4,3096    & 4,2458 \\
7   & $R[2,\mathrm{fix}\vert 0] $                     & 3,9294    & 3,5288    & 3,5038    & 4,2823    & 4,2256 \\
8   & $R[2,\mathrm{fix}\vert \mathrm{opt}]$                & 3,9285    & 3,5268 & 3,5024   & 4,2813    & 4,2246 \\
9   & $R[2,\mathrm{fix}\vert \mathrm{fix}]$                    & 3,9285    & 3,5268 & 3,5024   & 4,2813    & 4,2246 \\
\hline
\end{tabular}
\end{center}
\end{table*}

Рассмотрим метод сжатия, основанный на использовании статистической модели 
\textit{источника с~вычислимой последовательностью состояний} (см., например,~\cite{b4}). 
Модель предполагает, что элементы последовательности~$\mathbf{x}$ 
(в~данном случае одномерно упорядоченные ошибки предсказания $\mathbf{D}^\text{A}$ 
и~$\mathbf{D}^\text{R}$)\linebreak один за другим <<порождаются>> некоторым источником данных и~распределение 
значений очередного элемента~$x_n$ зависит только от текущего со\-сто\-яния источника, 
которое, в~свою очередь,\linebreak \mbox{определяется} значением предыдущего элемента~$x_{n-1}$  
последовательности. Назовем элемент последовательности, <<порожденный>> источником в~некотором 
состоянии, элементом этого состояния. Элементы каждого состояния будем ко\-ди\-ро\-вать/де\-ко\-ди\-ро\-вать независимо. 
Это возможно, поскольку в~момент обработки данного элемента значение предыдущего элемента известно не только 
кодеру, но и~декодеру (декодирование осуществляется без задержки). Эффективность кодирования 
зависит от выбора множества состояний.

Определим способ построения состояний в~стиле~\cite{b5}. Выберем некоторое натуральное число~$t$~--- 
порог. Отнесем к~нулевому состоянию те элементы~$x_n$, для которых $n\hm\geq 1$ и~$|x_{n-1}|\hm<t$, 
прочие элементы отнесем к~первому состоянию. Теперь последовательность~$\mathbf{x}$  
разложена на две подпоследовательности элементов нулевого и~первого состояний $\mathbf{x}_0$ 
и~$\mathbf{x}_1$. Квазиэнтропия и~скорость кодирования пары подпоследовательностей равны
\begin{multline}
H(\{ \mathbf{x}_0, \mathbf{x}_1 \}) = 
\fr{N_0}{N}H(\mathbf{x}_0) + \fr{N_1}{N}H(\mathbf{x}_1); \\
R(\{ \mathbf{x}_0, \mathbf{x}_1 \}) = 
\fr{N_0}{N}R(\mathbf{x}_0) + \fr{N_1}{N}R(\mathbf{x}_1), 
\label{eq08}
\end{multline}
где $N_0$ и~$N_1$~--- число элементов нулевого и~первого состояний, а квазиэнтропия
 и~ско\-рость кодирования каждой отдельной подпоследовательности $\mathbf{x}_0$ 
и~$\mathbf{x}_1$ даются формулами~(\ref{eq04}) и~(\ref{eq03}).

Результирующие квазиэнтропия и~скорость кодирования~(\ref{eq08}) зависят 
от выбора порога~$t$, определяющего состояния. При любом значении порога квазиэнтропия~(\ref{eq08}) 
не превышает квазиэнтропии всей последовательности~$\mathbf{x}$~(\ref{eq04}) 
(см., например,~\cite{b4}). Естественная оптимизационная задача~--- 
нахождение порога, при котором квазиэнтропия~(\ref{eq08}) принимает минимальное значение,~--- 
может быть решена для конкретных данных численно путем перебора.

Применим описанный метод разложения данных на два состояния к~ошибкам 
предсказания   $\mathbf{D}^\text{A}$ и~$\mathbf{D}^\text{R}$. 
Оптимальные пороги в~обоих случаях принимают одинаковые значения, равные~3, 4, 4, 3, 4 
для массивов I,\dots,V. Значения оптимальной квазиэнтропии
$H[2,\text{opt}] \doteq H(\{ 
\{{\mathbf{D}^\text{A} }_0,{\mathbf{D}^\text{A} }_1,\},
\{{\mathbf{D}^\text{R} }_0,{\mathbf{D}^\text{R} }_1,\}
\})$
для массивов~I--V представлены в~строке~2 табл.~\ref{tab1}. 
Среднее по массивам уменьшение по сравнению с~квазиэнтропией~$H$[1] составляет 0,0866~бт/п, 
минимальное~--- 0,0215~бт/п (массив~V), максимальное~--- 0,1930~бт/п (массив~III).

Нахождение оптимальных порогов для данного массива силовых кривых требует значительного 
времени счета и~не может быть реализовано на этапе кодирования в~режиме реального времени. 
Поэтому для построения состояний в~алгоритме, предназначенном для практического применения, 
следует использовать общие фиксированные значения порогов для всей совокупности подлежащих 
сжатию массивов. В~строке~3 табл.~\ref{tab1} представлены значения квазиэнтропии $H$[2,fix], 
отвечающие построенным с~порогами $t\hm=4$ двум состояниям ошибок предсказания $\mathbf{D}^\text{A}$ 
и~$\mathbf{D}^\text{R}$, для массивов~I--V. Увеличение квазиэнтропии по сравнению 
с~оптимальным значением $H$[2,opt] составляет 0,0025~бт/п для массива~I, пренебрежимо мало 
для массива~IV\linebreak и~равно нулю для остальных массивов. Таким об\-разом, использование 
общих фиксированных поро\-гов для построения состояний не приводит к~значительному 
увеличению квазиэнтропии по сравнению с~оптимальными значениями.

Скорости кодирования
$R[2,\text{fix}|0] \hm\doteq  R(\{ 
\{{\mathbf{D}^\text{A} }_0,{\mathbf{D}^\text{A} }_1,\},
\{{\mathbf{D}^\text{R} }_0,{\mathbf{D}^\text{R} }_1,\}
\})$
массивов~I--V алгоритмом A[2,fix$\vert$0], который независимо кодирует элементы состояний ошибок 
предсказания~$\mathbf{D}^\text{A}$ и~$\mathbf{D}^\text{R}$, построенных с~фиксированными порогами, 
принимающими значение $t\hm=4$, и~использует формулу~(\ref{eq02}) с~весом $w\hm=2$ для 
построения кодовых распределений, представлены в~строке~7 табл.~\ref{tab1}.\linebreak  
Для всех массивов имеет место снижение скорости кодирования по сравнению с~алгоритмом~A[1$\vert$0], 
которое лишь немного меньше, чем уменьшение \mbox{квазиэнтропии} $H$[2,fix] по 
сравнению с~квазиэнтропией~$H$[1]. Среднее по массивам снижение составляет~0,0855~бт/п, минимальное~--- 
0,0207~бт/п (массив~V), максимальное~--- 0,1921~бт/п (массив~III). Избыточность 
кодирования алгоритма составляет 0,0012--0,0026~бт/п и~в~1,2--1,8~раза больше избыточности алгоритма~A[1$\vert$0].

Рассмотрим метод уменьшения избыточности кодирования, 
основанный на выборе веса~$w$ в~формуле~(\ref{eq02}) для кодовых распределений. 
Отметим, что снижение скорости кодирования при таком подходе заведомо не превысит 
избыточности кодирования с~принятым по умолчанию значением веса $w\hm=2$. Оптимизационная 
задача нахождения веса~$w$, при котором для конкретной последовательности 
данных минимальна скорость кодирования~(\ref{eq03}), может быть решена численно 
путем перебора. В~колонках табл.~2, обозначенных I,\dots,V, представлены значения 
оптимальных весов~$w$ для ошибок предсказания $\mathbf{D}^\text{A}$ 
и~$\mathbf{D}^\text{R}$ каждого из массивов~I--V и~элементов состояний 
${\mathbf{D}^\text{A}}_0$, ${\mathbf{D}^\text{A}}_1$,
${\mathbf{D}^\text{R}}_0$ и~${\mathbf{D}^\text{R}}_1$
этих ошибок; состояния построены с~фиксированными порогами,  принимающими значение $t\hm=4$ (см.\ выше).


Обозначим через A[1$\vert$opt] алгоритм независимого кодирования ошибок предсказания~$\mathbf{D}^\text{A}$
и~$\mathbf{D}^\text{R}$\linebreak с~использованием оптимальных весов для по\-строения 
 кодовых распределений. Скорости кодирования $R$[1$\vert$opt] массивов~I--V этим алгоритмом \mbox{приведены} 
 в~строке~5 табл.~\ref{tab1}. Избыточность кодирования по сравнению с~алгоритмом~A[1$\vert$0] уменьшается 
 в~2,1--4,4~раза и~составляет теперь 0,0003--0,0006~бт/п в~зависимости от массива. 
 \mbox{Соответствующее} снижение скорости кодирования составляет 0,0005--0,0017~бт/п.

Вычисление оптимальных весов требует значительного времени. Поэтому 
в~предназначенном для применения на практике алгоритме следует использовать 
общие фиксированные значения весов. Такие значения для совокупности массивов~I--V 
представлены в~колонках табл.~2, обозначенных~\mbox{I--V}.

Обозначим через A[1$\vert$fix] алгоритм независимого кодирования ошибок 
предсказания $\mathbf{D}^\text{A}$ и~$\mathbf{D}^\text{R}$ с~использованием 
фиксированных весов для построения кодовых распределений. Скорости кодирования $R$[1$\vert$fix] 
массивов~I--V этим алгоритмом, приведенные в~строке~6 табл.~\ref{tab1}, 
практически не\linebreak\vspace*{-12pt}

%\pagebreak

% tabl2
%\begin{table*}

\begin{center}

{{\tablename~2}\ \ \small{
Оптимальные и~фиксированные веса
}}

%\label{tab2}
\vspace{2ex}

{\small \tabcolsep=7.5pt
\begin{tabular}{|l|ccccc|c|}
\hline
\multicolumn{1}{|c|}{Массив}& I     & II    
& III   & IV    & V     & I--V  \\
\hline
&&&&&&\\[-9pt]
\hspace*{3mm}$\mathbf{D}^\text{A}$       &30 &22 &14 &11 &13 &12 \\
\hspace*{3mm}${\mathbf{D}^\text{A}}_0$     &37 &13 &15 &15 &17&15\\
\hspace*{3mm}${\mathbf{D}^\text{A}}_1$     &22 &34 &32 &13 &16 &20\\  
\hline
&&&&&&\\[-9pt]
\hspace*{3mm}$\mathbf{D}^\text{R}$      &59 &285    &81 &81 &42 &45  \\
\hspace*{3mm}${\mathbf{D}^\text{R}}_0$    &29 &\hphantom{1}54 &22 &22 &46 &30  \\
\hspace*{3mm}${\mathbf{D}^\text{R}}_1$    &68 &332    &96 &80 &46 &60  \\
\hline
\end{tabular}
}
\end{center}

%\end{table*}

\vspace*{9pt}


\noindent
 отличаются от скоростей кодирования $R$[1$\vert$opt] алгоритма A[1$\vert$opt] 
с~оптимальными весами.

Обозначим через A[2,fix$\vert$opt] алгоритм, который независимо кодирует элементы состояний
${\mathbf{D}^\text{A}}_0$, ${\mathbf{D}^\text{A}}_1$,
${\mathbf{D}^\text{R}}_0$ и~${\mathbf{D}^\text{R}}_1$
ошибок предсказания, построенных с~фиксированными порогами,  
принимающими значение $t\hm=4$, и~использует оптимальные веса для построения 
кодовых распределений. Скорости кодирования $R$[2,fix$\vert$opt] массивов~I--V 
этим алгоритмом приведены в~строке~8 табл.~\ref{tab1}. Избыточность кодирования по 
сравнению с~алгоритмом A[2,fix$\vert$0] уменьшается в~2,5--3,8~раза до 0,0004--0,0008~бт/п 
в~зависимости от массива. Снижение скорости кодирования составляет 0,0009--0,0020~бт/п.

Обозначим через A[2,fix$\vert$fix] алгоритм, аналогичный A[2,fix$\vert$opt], 
но использующий фиксированные веса для построения кодовых распределений. 
Скорости кодирования $R$[2,fix$\vert$fix] \mbox{массивов~I--V} этим алгоритмом,
 приведенные в~строке~9 табл.~\ref{tab1}, 
практически не отличаются от скоростей кодирования $R$[2,fix$\vert$opt] алгоритма~A[2,fix$\vert$opt], 
использующего оптимальные веса.




\section{Заключение}
%\label{sec4}

Рассмотрен ряд алгоритмов обратимого сжатия массивов силовых кривых, 
основанных на универсальном арифметическом кодировании ошибок предсказания, и~получены 
оценки скорости кодирования таких алгоритмов. Показано, что разложение ошибок 
предсказания на независимо кодируемые вычислимые состояния и~выбор подходящих 
весов в~формуле для кодовых распределений позволяют уменьшить скорость кодирования.

Применимым на практике алгоритмом, обеспечивающим наименьшую скорость кодирования, 
оказался алгоритм A[2,fix$\vert$fix]. Скорости кодирования этим алгоритмом массивов~I--V даны в~строке~9 
табл.~\ref{tab1}. Алгоритм A[2,fix$\vert$fix] дает заметно меньшую скорость
 кодирования по сравнению с~алгоритм обратимого сжатия стандарта JPEG~2000~\cite{b1}. 
 Для массивов I,\dots,V выигрыш составляет 0,1737, 0,1967, 0,1591, 0,1404, 0,1117~бт/п со\-от\-вет\-ст\-венно.
 {\looseness=1
 
 }

Интересно применить использованные методы универсального кодирования (разделение на
 вычислимые состояния, выбор весов в~формуле~(\ref{eq02})) в~случае 
 арифметического кодирования компонент вейв\-лет-пре\-об\-ра\-зо\-ва\-ния~\cite{b1}. 
 Это должно стать предметом сле\-ду\-ющей работы.

{\small\frenchspacing
{%\baselineskip=10.8pt
%\addcontentsline{toc}{section}{References}
\begin{thebibliography}{9}
\bibitem{b1}
\Au{Стефанович А.\,И., Сушко~Д.\,В.} О сжатии данных массивов силовых кривых~// 
Информационные процессы, 2020. Т.~20. №\,3. С.~284--296.
\bibitem{b2}
\Au{Butt H.-J., Cappella~B., Kappl~M.} 
Force measurements with the atomic force microscope: Technique, interpretation and applications~// 
Surf. Sci. Rep., 2005. Vol.~59. P.~1--152. doi: 10.1016/j.surfrep.2005.08.003.
\bibitem{b3}
\Au{Witten~I.\,H., Neal~R.,M., Cleary~J.\,G.} Arithmetic coding for data compression~// 
Commun.  ACM, 1987. Vol.~30. No.\,6. P.~520--540. doi: 10.1145/214762.214771.
\bibitem{b4}
{\it Штарьков Ю.\,М.} Универсальное кодирование. Теория и~алгоритмы.~--- М.: Физматлит, 2013. 288~с.
\bibitem{b5}
{\it Сушко Д.\,В., Штарьков~Ю.\,М.} О сжатии томографических данных~// 
Информационные процессы, 2008. Т.~8. №\,4. С.~240--255.

 \end{thebibliography}

}
}

\end{multicols}

\vspace*{-3pt}

\hfill{\small\textit{Поступила в~редакцию 30.12.2020}}

\vspace*{8pt}

%\pagebreak

%\newpage

%\vspace*{-28pt}

\hrule

\vspace*{2pt}

\hrule

%\vspace*{-2pt}

\def\tit{COMPRESSION ALGORITHMS FOR FORCE VOLUME DATA~I:
CODING OF PREDICTION ERRORS}


\def\titkol{Compression algorithms for force volume data~I: Coding of prediction errors}

\def\aut{D.\,V.~Sushko}

\def\autkol{D.\,V.~Sushko}


\titel{\tit}{\aut}{\autkol}{\titkol}

\vspace*{-11pt}




\noindent
Institute of Informatics Problems, Federal Research Center ``Computer Science and Control''
 of the Russian Academy of Sciences, 44-2~Vavilov Str., Moscow 119333, Russian Federation

 
\def\leftfootline{\small{\textbf{\thepage}
\hfill INFORMATIKA I EE PRIMENENIYA~--- INFORMATICS AND
APPLICATIONS\ \ \ 2021\ \ \ volume~15\ \ \ issue\ 2}
}%
\def\rightfootline{\small{INFORMATIKA I EE PRIMENENIYA~---
INFORMATICS AND APPLICATIONS\ \ \ 2021\ \ \ volume~15\ \ \ issue\ 2
\hfill \textbf{\thepage}}}

\vspace*{3pt}


\Abste{The author considers the problem of reversible (lossless) 
compression of force volume data which are the three-dimensional arrays with 16-bit 
integer elements. Such arrays are the result of atomic force microscopy scanning 
of microobjects in the force mapping mode. The author proposes reversible compression 
algorithms of force volume data based on the universal arithmetic coding of their 
prediction errors. The author uses two methods of universal coding. The first method 
based on the statistical model of the source with the calculable sequence of states 
implies the decomposition of an error prediction sequence into two subsequences 
which are coded independently. The second method implies a choice of the appropriate 
weight while constructing the code probabilities used in arithmetic coding. The 
author constructs bit rate estimations for the proposed algorithms for five test arrays. 
The results show that combination of the universal coding methods mentioned above 
makes significant reduction of the bit rate. The bit rates of the most efficient 
algorithm among proposed practically applicable algorithms for the test arrays 
are~3.9285, 3.5268, 3.5024, 4.2813, and 4.2246 bit/pixel.}

\KWE{atomic force microscope; force volume data; reversible compression; arithmetic coding; universal coding}

\DOI{10.14357/19922264210212}

%\vspace*{-15pt}

 %\Ack
%\noindent

%\vspace*{12pt}

  \begin{multicols}{2}

\renewcommand{\bibname}{\protect\rmfamily References}
%\renewcommand{\bibname}{\large\protect\rm References}

{\small\frenchspacing
 {%\baselineskip=10.8pt
 \addcontentsline{toc}{section}{References}
 \begin{thebibliography}{9}
\bibitem{1-ss}
\Aue{Stefanovich, A.\,I., and D.\,V.~Sushko.}
 2020. O~szhatii dannykh massivov silovykh krivykh [On data compression of force volumes]. 
 \textit{Informatsionnye protsessy} [Information Processes] 20(3):284--296. 
\bibitem{2-ss}
\Aue{Butt, H.-J., B.~Cappella, and M.~Kappl.}
 2005. Force measurements with the atomic force microscope: Technique, interpretation and 
 applications. \textit{Surf. Sci. Rep.} 59:1--152.
 doi: 10.1016/j.surfrep.2005.08.003.
\bibitem{3-ss}
\Aue{Witten, I.\,H., R.\,M.~Neal, and J.\,G.~Cleary.}
 1987. Arithmetic coding for data compression. \textit{Commun. ACM} 30(6):520--540. 
 doi: 10.1145/214762.214771.
\bibitem{4-ss}
\Aue{Shtar'kov, Yu.\,M.} 2013. \textit{Universal'noe kodirovanie. Teoriya i~algoritmy} [Universal coding.
 Theory and algorithms]. Мoscow: Fizmatlit. 288 p. 
\bibitem{5-ss}
\Aue{Sushko, D.\,V., and Yu.\,M.~Shtar'kov.} 2008. 
O~szhatii tomograficheskikh dannykh [On tomography data compression]. 
\textit{Informatsionnye Protsessy} [Information Processes] 8(4):240--255. 
 \end{thebibliography}

 }
 }

\end{multicols}

\vspace*{-4pt}

  \hfill{\small\textit{Received December~30, 2020}}


%\pagebreak

\vspace*{-12pt}  

\Contrl

\noindent
\textbf{Sushko Dmitry V.} (b.\ 1962)~--- 
Candidate of Science (PhD) in physics and mathematics, senior scientist, 
Institute of Informatics Problems, Federal Research Center ``Computer 
Science and Control'' of the Russian Academy of Sciences, 44-2~Vavilov Str., 
Moscow 119333, Russian Federation; \mbox{dsushko@ipiran.ru}

\label{end\stat}

\renewcommand{\bibname}{\protect\rm Литература}