\def\stat{gonch+zatsman}

\def\tit{ПРИНЦИПЫ СТРУКТУРИЗАЦИИ СТАТЕЙ\\ В~ЭЛЕКТРОННЫХ СЛОВАРЯХ$^*$}

\def\titkol{Принципы структуризации статей в~электронных словарях}

\def\aut{А.\,А.~Гончаров$^1$, И.\,М.~Зацман$^2$}

\def\autkol{А.\,А.~Гончаров, И.\,М.~Зацман}

\titel{\tit}{\aut}{\autkol}{\titkol}

\index{Гончаров А.\,А.}
\index{Зацман И.\,М.}
\index{Goncharov A.\,A.}
\index{Zatsman I.\,M.}

{\renewcommand{\thefootnote}{\fnsymbol{footnote}} \footnotetext[1]
{Работа выполнена в~Институте проблем информатики ФИЦ ИУ РАН при поддержке РФФИ (проект  
20-012-00166).}}


\renewcommand{\thefootnote}{\arabic{footnote}}
\footnotetext[1]{Институт проблем информатики Федерального исследовательского центра <<Информатика 
и~управление>> Российской академии наук, \mbox{a.gonch48@gmail.com}}
\footnotetext[2]{Институт проблем информатики Федерального исследовательского центра <<Информатика 
и~управление>> Российской академии наук, \mbox{izatsman@yandex.ru}}

\vspace*{-12pt}


  
  \Abst{Рассмотрены две задачи, возникающие при переводе бумажных словарей 
в~электронную форму представления: (1)~структуризация унаследованных 
и~существующих в~бумажной форме словарных статей, обеспечивающая расширение 
функциональных возможностей электронного словаря по сравнению с~бумажным; 
(2)~замена традиционных способов шрифтового выделения структурных элементов 
словарной статьи на способы, обеспечивающие их программную адресацию в~базе 
данных. Показано, что структуру словарных статей, используемую в~традиционной 
лексикографии, необходимо детализировать и~одновременно с~этим категоризировать 
часть структурных элементов для расширения функциональных возможностей 
электронного словаря. Описан подход к~формированию классификационной системы, 
интегрированной в~электронный словарь, и~последующей рубрикации структурных 
элементов словарных статей на ее основе. Предлагаемые решения позволяют значительно 
расширить функционал электронного словаря по сравнению с~его бумажным аналогом 
и~преодолеть ограничения традиционной лексикографии, обусловленные бумажной 
формой представления.}
  
  \KW{принципы структуризации; электронный словарь; электронная лексикография; 
классификационная система}

\DOI{10.14357/19922264210213}

%\vspace*{-3pt}


\vskip 10pt plus 9pt minus 6pt

\thispagestyle{headings}

\begin{multicols}{2}

\label{st\stat}
  
\section{Введение}

  Еще в~2000~г.\ французский лингвист Б.~Серкилини (B.~Cerquiglini) 
в~ходе своего выступления на Седьмой международной конференции 
<<Journ$\acute{\mbox{e}}$e des dictionnaires>>, тема которой звучала как 
<<От бумажных словарей к~словарям электронным>>, выделил в~развитии 
электронной лексикографии три этапа. Первый из них~--- создание 
бумажных словарей с~использованием компьютера; второй~--- перевод 
существующих бумажных словарей в~электронный\linebreak формат; третий (на тот 
момент только начинавшийся)~--- изначальная разработка словарей 
в~электронном формате с~пользовательскими функциями,\linebreak реализация 
которых невозможна при издании словарей на бумажном носителе~[1, 
с.~188], например использование потоковых объектов мультимедиа.
  
  В том же 2000~г. Р.~Вешлер и~К.~Питтс сравнили электронные 
и~бумажные словари применительно к~обучению английскому языку как 
иностранному. Вывод, который они сделали, оказался неутешительным: 
<<электронные словари по-преж\-не\-му остаются по своей сути бумажными 
словарями, записанными на электронный носитель>>~[2]. В~2012~г., по 
мнению С.~Гранже, это утверждение~--- не\-смот\-ря на улучшение  
ситуации~--- все еще оставалось актуальным для значительного числа 
электронных словарей~[3, с.~2].
  
  Сегодня также приходится признавать, что работа по созданию 
электронных словарей, которые принципиально расширяли бы функционал 
словарей бумажных и~были бы удобны для пользователя и~лексикографа, не 
теряет актуальности. Более того, при создании электронного словаря на 
основе бумажного необходимо учитывать отличия как в~структуре их 
словарных статей, так и~в способах выделения структурных элементов. 
Например, в~бумажных словарях для этой цели может использоваться 
шрифтовое выделение (курсив, полужирный шрифт, заглавные буквы), 
символы шрифта Wingdings (\raisebox{-1pt}[0pt][0pt]{\mbox{%
   \epsfxsize=2.5mm 
  \epsfbox{listi.eps}
   }}, \raisebox{-1pt}[0pt][0pt]{\mbox{%
   \epsfxsize=2.5mm 
  \epsfbox{flag.eps}
   }}, \raisebox{-1pt}[0pt][0pt]{\mbox{%
   \epsfxsize=3.5mm 
  \epsfbox{kniga.eps}
   }} и~т.\,д.), нумерация арабскими и/или римскими цифрами и~спецсимволы 
(точка с~запятой, пробелы, скобки, слеши, кавычки и~т.\,п.)~[4].

  \begin{table*}\small
  \begin{center}
  \begin{tabular}{|c|c|c|p{70mm}|}
  \multicolumn{4}{c}{Пример распределения содержания словарной статьи из~\cite{5-gz} 
по зонам}\\
  \multicolumn{4}{c}{\ }\\[-6pt]
  \hline
\multicolumn{3}{|c|}{Заглавное слово (=\;лемма) и~его 
варианты}&\textbf{k$\ddot{\mbox{o}}$nnen}\\
\hline
\multicolumn{3}{|c|}{Зона грамматической информации о лемме в~целом}&\textit{vmod} 
(\textit{perf} hat k$\ddot{\mbox{o}}$nnen, \textit{в~неполных предложениях, где пропущен 
инфинитив полнозначного глагола} hat gekonnt)\\
\hline
\hspace*{-2pt}{\raisebox{-75pt}{\rotatebox{90}{Зона значения}}}
&
\hspace*{-2pt}{\raisebox{-67pt}{\rotatebox{90}{Значение 1}}}&
\tabcolsep=0pt\begin{tabular}{c}Толкование леммы\\ в~данном значении\end{tabular}&
\textit{для  выражения потенциальной возможности}\\[-65pt]
\cline{3-4}
&&\raisebox{-6pt}[0pt][0pt]{\tabcolsep=0pt\begin{tabular}{c}Варианты перевода леммы\\ 
в~данном значении\end{tabular}}&
мочь, иметь возможность; можно; 
\textit{под отрицанием} \mbox{нельзя}\\
\cline{3-4}
&&\raisebox{-6pt}[0pt][0pt]{\tabcolsep=0pt\begin{tabular}{c}Примеры употребления\\ 
леммы в~данном значении\end{tabular}}&
ich habe heute frei und kann dich 
besuchen я~сегодня свободен и~могу к~тебе зайти; [$\ldots$]\\
\cline{3-4}
&&\raisebox{-16pt}[0pt][0pt]{\tabcolsep=0pt\begin{tabular}{c}Устойчивые конструкции\\
 с~использованием леммы\\ 
в~данном значении$^*$\end{tabular}}&$\ldots$, 
\textbf{ich kann dir sagen!} (\textit{только в~постпозиции и~с~прямым порядком слов}) 
$\ldots$, просто фантастика!; das war eine Schl$\ddot{\mbox{a}}$gerei, ich kann dir sagen ну 
и~драка была, я~тебе скажу!; [$\ldots$]\\
\cline{2-4}
&$\ldots$&$\ldots$&$\ldots$\\
\hline
\multicolumn{3}{|c|}{Зона идиоматики}&[$\ldots$] \textbf{(gut) k$\ddot{\mbox{o}}$nnen} 
(\textit{mit jmdm.}) \textit{разг.}\ быть в~(дружеских) отношениях (\textit{с~кем-л.}); die 
beiden k$\ddot{\mbox{o}}$nnen einfach nicht miteinander отношения у~них просто не 
складываются; [$\ldots$]\\
\hline
\multicolumn{4}{p{150mm}}{\footnotesize \hspace*{2mm}$^*$В~более ранних работах, 
в~том числе~\cite{9-gz, 10-gz}, эта зона носила название <<Грамматическая фразеология 
с~использованием леммы в~данном значении>>.}
\end{tabular}
\end{center}
\vspace*{-5pt}
\end{table*}
  
  Следовательно, когда при подготовке электронного словаря используются 
наследуемые лексикографические ресурсы, возникают две задачи:\linebreak
%\begin{enumerate}[(1)]
%\item 
(1)~более 
детальная (по сравнению с~бумажным словарем) структуризация словарных 
статей, обес\-пе\-чи\-ва\-ющая расширение его функциональных возможностей; 
%\item 
(2)~замена традиционных способов\linebreak выделения структурных элементов 
(=\;по\-лей) разметкой тегами и/или использование баз данных в~процессе его 
подготовки. 
%\end{enumerate}
Лишь после их решения появляется возможность существенного 
расширения функциональных возможностей электронного словаря по 
сравнению с~бумажным.
  
Цель статьи состоит в~описании принципов решения первой задачи на примере двуязычного 
(не\-мец\-ко-рус\-ско\-го) словаря, создаваемого группой лексикографов под руководством Д.\,О.~Доб\-ро\-вольского~[5], 
с~применением надкорпусной базы данных (НБД) немецких модальных глаголов, созданной в~ФИЦ ИУ РАН 
(см.\ подробнее~[6]). Решение этой задачи с~применением НБД необходимо, в~частности, и~для того, чтобы 
обеспечить возможность фиксации ретроспективы изменений, вносимых в~словарные статьи 
лексикографами (см.\ об этом~[7, 8]\footnote{В этих работах механизм фиксации изменений, вносимых 
в~словарные статьи, рассматривается на примере лишь одной зоны статьи~--- зоны значения, приводимой, 
кроме того, в~сокращенном виде. Однако при условии, что словарные статьи были структурированы, 
применение НБД позволяет фиксировать ретроспективу изменений, вносимых в~любую из зон статьи.}). 
Частично рассматривается решение и~второй задачи, которая более детально описана  
в~работе~\cite{4-gz}\footnote{В~этой работе задача решается на примере немецко-русского 
фразеологического словаря.}.


\vspace*{-6pt}
  
\section{Структуризация словарных статей}

\vspace*{-3pt}

  В работе~[9, с.~92] в~форме таблицы была пред\-став\-ле\-на структура статьи, 
используемая в~бумажном словаре~\cite{5-gz}\footnote{Более подробное описание 
см.\ в~\cite[с.~40--44]{10-gz}; хотя там говорится о структуре статьи нового большого 
не\-мец\-ко-рус\-ско\-го словаря, она во многом совпадает со структурой статьи словаря~\cite{5-gz}. 
Другие зоны, 
которые также используются при составлении словарей, перечисляются в~\cite{11-gz}.}. 
Каждая строка таблицы соответствовала зоне словарной статьи с~точки 
зрения традици\-он\-ной лексикографии, а~столбцы показывали уровень 
вложенности зон. 

В~таб\-ли\-це проиллюстрировано, каким образом 
содержание словарной статьи распределено по зонам. Для примера взята 
статья на один из модальных глаголов немецкого языка~--- глагол 
\textit{k$\ddot{\mbox{o}}$nnen}. В~целях экономии места в~таблице 
приведено только первое из девяти его значений, 
а~также сокращено чис\-ло примеров употребления, устойчивых конструкций и~идиом
(пропуски отмечены как 
[$\ldots$]).
  



  Хотя на первый взгляд может показаться, что\linebreak такого распределения 
содержания статьи по зонам~--- структурным элементам верхнего уровня~--- 
достаточно для работы со словарем в~\mbox{электронном} формате, почти каждая 
зона с~содержательной точки зрения может быть разделена на поля~--- 
структурные элементы нижних уровней, что даст возможность расширить 
спектр областей поиска в~электронном словаре.
  
  Так, <<зона грамматической информации о лемме в~целом>> объединяет 
как минимум два поля: (1)~<<\textit{vmod}>> (`модальный глагол')~--- 
информация о том, к~какой части речи принадлежит\linebreak лемма; (2)~<<\textit{perf} hat 
k$\ddot{\mbox{o}}$nnen, \textit{в~неполных предложениях, где пропущен 
инфинитив полнозначного глагола} hat gekonnt>>~--- информация об 
особенностях образования грамматических форм леммы. Более того, внут\-ри 
второго поля можно выделить элемент <<\textit{perf}>> (`перфект'), 
ука\-зы\-ва\-ющий, формы какого грамматического времени глагола образуются 
не по общему правилу. 
  
  Зона вариантов перевода леммы в~данном значении в~примере из таблицы 
распадается на 3~поля, в~бумажном оформлении разделенные точкой 
с~запятой. Таким образом сгруппированы наиболее близкие по значению 
варианты перевода: (1)~<<мочь, иметь возможность>>; (2)~<<можно>>; 
(3)~<<\textit{под отрицанием} нельзя>>. Кроме того, третье поле содержит 
комментарий, в~данном случае~--- объяснение условий, при которых следует 
использовать этот вариант перевода (<<\textit{под отрицанием}>>).
  
  Зона примеров употребления леммы в~данном значении, во-пер\-вых, 
состоит из отдельных примеров (в~бумажном оформлении также 
разделенных точкой с~запятой), в~каждом из которых, во-вторых, можно 
выделить оригинальный текст примера и~его перевод (отделенные друг от 
друга пробелом).
  
  Зона <<Устойчивые конструкции с~использованием леммы в~данном 
значении>> имеет сходную структуру, которая, однако, включает и~другие 
поля. Полужирным шрифтом выделена сама устойчивая конструкция, для 
которой в~скобках курсивом могут указываться (1)~синтаксические 
валентности\footnote{Синтаксической валентностью называется <<способность слова 
вступать в~синтаксические связи с~другими элементами>>~\cite[с.~79--80]{12-gz}. В~нижней 
строке таблицы для идиомы <<(gut) k$\ddot{\mbox{o}}$nnen>> указана валентность <<\textit{mit 
jmdm.}>>~--- буквально `с~кем-ли\-бо'. Валентность <<\textit{с~кем-л.}>> указана и~для перевода 
данной идиомы на русский язык.} и~(2)~комментарии особенностей употребления. 
Более того, для некоторых устойчивых конструкций приводятся примеры их 
употребления (оригинал и~перевод).
  
  Зона идиоматики с~точки зрения структуры почти полностью повторяет 
зону <<Устойчивые конструкции$\ldots$>>, однако идиомы могут 
со\-про\-вож\-дать\-ся указаниями на стилистические особенности их 
употребления~--- стилистическими пометами (см.\ <<\textit{разг.}>> 
в~нижней строке таблицы).
  
  Из сказанного выше можно сделать вывод, что для создания электронного 
словаря, функционал которого был бы шире функционала соответствующего 
бумажного словаря, требуется сделать пригодными для адресации 
и~программной обработки не только традиционно выделяемые зоны статьи, но и~вложенные в~них поля. 
Так, если представить пример из таблицы в~виде  
XML-де\-ре\-ва\footnote{Идея использования XML-разметки словарных статей не нова и~ранее 
была описана в~контексте обмена словарными ресурсами~[13, 14, с.~291--329].} с~учетом 
предлагаемого уровня детализации, получим следующий результат 
структуризации словарной статьи\footnote{Разметка выполнена для статьи из таблицы 
с~использованием следующих тегов: $\langle${\sf entry}$\rangle$~--- словарная статья; 
$\langle${\sf hdw}$\rangle$~--- лемма; $\langle${\sf grinf}$\rangle$~--- 
грамматическая информация; $\langle${\sf pos}$\rangle$~--- информация о том, к~какой части 
речи принадлежит лемма; $\langle${\sf grforms}$\rangle$~--- описание особенностей 
образования грамматических форм; $\langle${\sf form}$\rangle$~--- грамматическая форма; 
$\langle${\sf usn}$\rangle$~--- стилистические пометы; $\langle${\sf 
idmfld}$\rangle$~--- зона идиоматики; $\langle${\sf idm}$\rangle$~--- идиома; 
$\langle${\sf mnfld}$\rangle$~--- зона значения; $\langle${\sf mn}$\rangle$~--- 
значение; $\langle${\sf mndesc}$\rangle$~--- описание значения; $\langle${\sf 
interp}$\rangle$~--- толкование значения; $\langle${\sf orig}$\rangle$~--- текст 
оригинала; $\langle${\sf trnsl}$\rangle$~--- текст перевода; $\langle${\sf 
tgroup}$\rangle$~--- группа вариантов перевода; $\langle${\sf phrasfld}$\rangle$~---  
зона <<Устойчивые конструкции$\ldots$>>;
   $\langle${\sf phras}$\rangle$~--- устойчивая конструкция; $\langle${\sf 
comm}$\rangle$~--- комментарий; $\langle${\sf exfld}$\rangle$~--- зона примеров; 
$\langle${\sf ex}$\rangle$~--- пример; $\langle${\sf val}$\rangle$~--- зона 
синтаксических валентностей.}:

%\vspace*{2pt}

\noindent
  {\small 
  \begin{tabular}{l}
  $\langle${\sf entry}$\rangle$\\
  $\langle${\sf 
hdw}$\rangle$\textbf{k$\ddot{\mbox{o}}$nnen}$\langle$/{\sf hdw}$\rangle$
  \\
  $\langle${\sf grinf}$\rangle$\\
  \hspace*{3mm}$\langle${\sf pos}$\rangle$\textit{vmod}$\langle$/{\sf pos}$\rangle$\\
  \hspace*{3mm}$\langle${\sf grforms}$\rangle$\\
  \hspace*{6mm}$\langle${\sf 
form}$\rangle$\textit{perf}$\langle$/{\sf form}$\rangle$\\
  \hspace*{8mm}hat k$\ddot{\mbox{o}}$nnen, \textit{в неполных 
предложениях, где}\\ 
\hspace*{8mm}\textit{пропущен инфинитив полнозначного глагола}\\
\hspace*{8mm}hat  gekonnt\\
  \hspace*{3mm}$\langle$/{\sf grforms}$\rangle$\\
  $\langle$/{\sf grinf}$\rangle$\\
  $\langle${\sf mnfld}$\rangle$\\
  \hspace*{3mm}$\langle${\sf mn}$\rangle$\\
  \hspace*{6mm}$\langle${\sf mndesc}$\rangle$\\
  \hspace*{10mm}$\langle${\sf 
interp}$\rangle$\textit{для выражения 
потенциальной}\\
\hspace*{10mm}\textit{возможности}$\langle$/{\sf interp}$\rangle$\\
  \hspace*{8mm}$\langle${\sf trnsl}$\rangle$\\
  \hspace*{10mm}$\langle${\sf 
tgroup}$\rangle$мочь, иметь 
возможность$\langle$/{\sf tgroup}$\rangle$\\
  \hspace*{10mm}$\langle${\sf 
tgroup}$\rangle$можно$\langle$/{\sf tgroup}$\rangle$\\
  \hspace*{10mm}$\langle${\sf tgroup}$\rangle$\\
  \hspace*{12mm}$\langle${\sf comm}$\rangle$\textit{под отрицанием}$\langle$/{\sf comm}$\rangle$ 
нельзя\\
  \hspace*{10mm}$\langle$/{\sf 
tgroup}$\rangle$\\
  \hspace*{8mm}$\langle$/{\sf trnsl}$\rangle$\\
  \hspace*{6mm}$\langle$/{\sf mndesc}$\rangle$\\
  \hspace*{6mm}$\langle${\sf exfld}$\rangle$\\
  \hspace*{8mm}$\langle${\sf ex}$\rangle$\\
  \hspace*{12mm}$\langle${\sf orig}$\rangle$ich habe heute frei und kann dich\\
  \hspace*{12mm}besuchen$\langle$/{\sf orig}$\rangle$\\
  \hspace*{12mm}$\langle${\sf  trnsl}$\rangle$я сегодня свободен и~могу к~тебе\\
  \hspace*{12mm}зайти$\langle$/{\sf trnsl}$\rangle$\\
  \hspace*{8mm}$\langle$/{\sf ex}$\rangle$\\
  \hspace*{8mm}$\ldots$\\
  \hspace*{6mm}$\langle$/{\sf exfld}$\rangle$\\
  \hspace*{6mm}$\langle${\sf phrasfld}$\rangle$\\
  \hspace*{8mm}$\langle${\sf phras}$\rangle$\\
  \hspace*{10mm}$\langle${\sf orig}$\rangle$\\
  \hspace*{10mm}$\ldots$, \textbf{ich kann dir sagen!}\\ 
  \hspace*{12mm}$\langle${\sf comm}$\rangle$\textit{только в~постпозиции и~с~прямым}\\
  \hspace*{12mm}\textit{порядком  слов}$\langle$/{\sf comm}$\rangle$\\
  \hspace*{10mm}$\langle$/{\sf orig}$\rangle$\\
  \hspace*{10mm}$\langle${\sf trnsl}$\rangle$$\ldots$, просто фантастика!$\langle$/{\sf trnsl}$\rangle$\\
  \hspace*{10mm}$\langle${\sf exfld}$\rangle$$\ldots$$\langle$/{\sf exfld}$\rangle$\\
  \hspace*{8mm}$\langle$/{\sf phras}$\rangle$\\
  \hspace*{8mm}$\ldots$

  \end{tabular}
  
  }
  
\vspace*{0.5pt}
 
  \pagebreak
  
\noindent
   {\small 
  \begin{tabular}{l}
     \hspace*{6mm}$\langle$/{\sf phrasfld}$\rangle$\\
           \hspace*{3mm}$\langle$/{\sf mn}$\rangle$\\
  \hspace*{3mm}$\ldots$\\
  $\langle$/{\sf mnfld}$\rangle$\\
  $\langle${\sf idmfld}$\rangle$\\
  \hspace*{3mm}$\langle${\sf idm}$\rangle$\\
  \hspace*{6mm}$\langle${\sf orig}$\rangle$\textbf{(gut) 
k$\ddot{\mbox{o}}$nnen}\\ 
  \hspace*{8mm}$\langle${\sf 
val}$\rangle$\textit{mit jmdm.}$\langle$/{\sf val}$\rangle$\\
  \hspace*{8mm}$\langle${\sf usn}$\rangle$\textit{разг.}$\langle$/{\sf usn}$\rangle$\\
  \hspace*{6mm}$\langle$/{\sf orig}$\rangle$\\
  \hspace*{6mm}$\langle${\sf trnsl}$\rangle$\\
  \hspace*{8mm}быть в~(дружеских) отношениях\\
  \hspace*{8mm}$\langle${\sf 
val}$\rangle$\textit{с~кем-л.}$\langle$/{\sf val}$\rangle$\\
  \hspace*{6mm}$\langle$/{\sf trnsl}$\rangle$\\
  \hspace*{6mm}$\langle${\sf 
exfld}$\rangle$$\ldots$$\langle$/{\sf exfld}$\rangle$\\
  \hspace*{3mm}$\langle$/{\sf idm}$\rangle$\\
  \hspace*{3mm}$\ldots$\\
  $\langle$/{\sf idmfld}$\rangle$\\
  $\langle$/{\sf entry}$\rangle$
  \end{tabular}
  }
  
  \vspace*{2pt}
  
  Такая структура является более детальной, чем представленное в~таблице 
традиционное деление на зоны, что и~обеспечивает существенное 
расширение функционала электронного словаря. Наиболее очевидная новая 
возможность~--- поиск словарных статей по тексту любого из структурно 
выделенных полей.
{\looseness=-1

}
  
  Другие возможные объекты поиска: все статьи, где описываются единицы, 
которые можно перевести на русский словом <<можно>>; все статьи, 
которые включают интересующий пользователя или лексикографа 
структурный элемент (например, зону <<Устойчивые конструкции$\ldots$>> 
или зону идиоматики) и~т.\,д. Это может быть ценным для лексикографии 
(создание словарей разных типов), для обучения иностранному языку (отбор 
материала по значениям грамматических признаков), а также для решения 
переводческих задач.

%\vspace*{-9pt}
  
\section{Формирование классификационной системы}

%\vspace*{-3pt}

  Хотя одно только выделение новых структурных элементов 
статьи расширяет функционал электронного словаря по сравнению 
с~бумажным, выполнение категоризации этих элементов способно 
обеспечить решение еще более широкого круга задач. Категоризация 
структурного элемента (=\;по\-ля) словарной статьи~--- это отнесение его 
к~некоторому классу или группе согласно некоторому признаку, например: 
(1)~<<часть речи>> (значения признака: $n$~--- существительное, $v$~--- 
глагол и~т.\,д.); (2)~<<постоянные грамматические характеристики>> 
(отметим, что наборы значений этого признака отличаются в~зависимости от 
значения признака <<часть речи>>, см.\ об этом также~[13, с.~116]: так, для 
существительного это грамматический род~--- мужской ($m$), средний ($n$) 
или женский ($f$); для глагола~--- переходность ($vt$) или непереходность 
($vi$) и~т.\,п.); (3)~<<стилистические особенности упо\-треб\-ле\-ния>> (значения 
признака: \textit{разг.}, \textit{груб.} и~т.\,д.) и~др.
{\looseness=-1

}
  
  Значения признаков могут быть объединены в~фасеты, которые, в~свою 
очередь, объединяются в~фасетную классификацию (с~по\-мощью которой 
выполняется рубрикация всей словарной статьи и/или ее структурных 
элементов).
  
  Создание и~использование такой классификации даст возможность искать 
по значению признака, например, словарные статьи, где лемма пред\-став\-ля\-ет 
собой: слово разговорного стиля; \mbox{существительное} среднего рода; 
прилагательное, име\-ющее особенности образования форм сравнительной 
степени, и~т.\,д. Важно отметить, что могут отбираться статьи, где, например, 
к~разговорному стилю относится не лемма, а~только идиома с~этой леммой 
(см.\ идиому с~пометой <<\textit{разг.}>> в~нижней строке таблицы).
  
  Существует также возможность добавления новых классификационных 
признаков. Одним из таких признаков, имеющих особую ценность для 
пользователей, может стать признак <<семантика леммы>>. Для его 
использования следует, во-пер\-вых, выбрать готовую (или создать новую) 
классификационную систему, которая будет использоваться в~электронном 
словаре, и,~во-вто\-рых, добавить поле для значения признака <<семантика 
леммы>>, ука\-зы\-ва\-юще\-го на принадлежность леммы к~некоторому семантическому классу\footnote{Применительно 
к~фразеологическому 
словарю об этом говорилось в~[15].}. Таким образом, двуязычный словарь 
приобретает отдельные свойства словаря идеографического или 
тезауруса~[16,~17].
{\looseness=-1

}
  
  Элементы реализации идеи включения в~словарь семантической 
классификационной системы можно найти во фран\-цуз\-ско-рус\-ской 
лексикографии~--- это так называемые <<комплексные словарные статьи>> 
в~[18]. В~отличие от традиционных статей двуязычного словаря 
комплексные статьи решают специализированные задачи: в~них могут 
разъясняться трудности перевода применительно к~описываемой паре языков 
(в~случае словаря~[18] это пара <<фран\-цуз\-ский--рус\-ский>>), 
рассматриваться способы выражения семантических категорий (например, 
<<цель>>, <<причина>>). Также в~словарь могут включаться семантические 
группы слов (названия и~перевод месяцев, дней недели, стран света и~т.\,п.). 
Однако в~рамках бумажного словаря лексикографы сталкиваются 
с~б$\acute{\mbox{о}}$льшими сложностями при реализации этой идеи, чем при создании словаря 
электронного.
  
  Включение в~электронный словарь фасетной классификации 
с~возможностью отбирать словарные статьи по значениям разных признаков и~их сочетаниям 
дает пользователю широкие возможности поиска и~существенно увеличивает 
спектр решаемых лексикографических задач\footnote[1]{Возможность отбирать 
словарные статьи в~зависимости от семантики описываемых в~них единиц позволит проверить, 
описаны ли эти единицы аналогичным образом и не пропущена ли ка\-кая-ли\-бо из единиц. 
Такая ситуация была детально рассмотрена В.\,А.~Успенским на 
примере включения в~толковые словари русского языка названий букв 
русского алфавита в~работе~[19, с.~605--609].}.

\vspace*{-6pt}

\section{Заключение}

\vspace*{-3pt}

  Существующие средства информатики дают возможность значительно 
расширить функционал электронных словарей по сравнению с~\mbox{бумажными} 
словарями, записанными на электронный носитель. Однако для 
использования накопленных лексикографических ресурсов требуется 
выполнить структуризацию наследуемых словарных статей, 
обеспечивающую последующее наполнение баз данных наследуемыми 
лексикографическими ресурсами, формирование электронных словарей 
и~выполнение в~них лек\-си\-ко-грам\-ма\-ти\-че\-ских видов поиска.
  
  Для создания лексикографических баз знаний с~развитыми возможностями 
семантического поиска необходимо предварительно сформировать и~потом 
использовать лингвистическую фасетную классификацию, объединяющую 
грамматические, функ\-ци\-о\-наль\-но-сти\-ли\-сти\-че\-ские и~семантические 
признаки с~их простановкой как в~словарных стать\-ях, так и~в~их 
структурных элементах. В~настоящее время проблема создания 
лексикографических баз знаний с~подобными возможностями находится на 
начальной стадии решения.


\vspace*{-6pt}


{\small\frenchspacing
{\baselineskip=10.7pt
%\addcontentsline{toc}{section}{References}
\begin{thebibliography}{99}

\vspace*{-2pt}

\bibitem{1-gz}
\Au{Pruvost J.} Des dictionnaires papier aux dictionnaires $\acute{\mbox{e}}$lectroniques: VIIe 
Journ$\acute{\mbox{e}}$e des dictionnaires (22~mars 2000): Rapport de colloque~// Int. 
J.~Lexicogr., 2000. Vol.~13. Iss.~3. P.~187--193. doi: 10.1093/ijl/13.3.187.
\bibitem{2-gz}
\Au{Weschler R., Pitts Chr.} An experiment using electronic dictionaries with EFL students. {\sf 
http://iteslj.org/ Articles/Weschler-ElectroDict.html}.
\bibitem{3-gz}
Electronic lexicography~/
Eds.\ S.~Granger, M.~Paquot.~--- Oxford University Press, 2012. 517~p.
\bibitem{4-gz}
\Au{Вакуленко В.\,В., Зацман~И.\,М.} Наследуемые лексикографические ресурсы базы 
данных фразеологического словаря~// Системы и~средства информатики, 2021. Т.~31. №\,2. 
С.~129--138.
\bibitem{5-gz}
Немецко-рус\-ский словарь актуальной лексики~/
Под ред. Д.\,О.~Добровольского.~--- М.: Лексрус, 2021 (в~пе\-чати).
\bibitem{6-gz}
\Au{Добровольский Д.\,О., Зализняк Анна~А.} Немецкие конструкции с~модальными 
глаголами и~их русские соответствия: проект надкорпусной базы данных~//\linebreak 
Компьютерная лингвистика и~интеллектуальные технологии: По мат-лам Междунар. 
конф. <<Диалог>>.~--- М.: РГГУ, 2018. Вып.~17(24). С.~172--184.
\bibitem{7-gz}
\Au{Гончаров А.\,А., Зацман~И.\,М., Кружков~М.\,Г.} Эволюция классификаций 
в~надкорпусных базах данных~// Информатика и~её применения, 2020. Т.~14. Вып.~4. 
С.~108--116.
\bibitem{8-gz}
\Au{Гончаров А.\,А., Зацман~И.\,М., Кружков~М.\,Г.} Пред\-став\-ле\-ние новых 
лексикографических знаний в~динамических классификационных системах~// 
Информатика и~её применения, 2021. Т.~15. Вып.~1. С.~82--89.
\bibitem{9-gz}
\Au{Гончаров А.\,А., Зацман~И.\,М., Кружков~М.\,Г.} Темпоральные данные 
в~лексикографических базах знаний~// Информатика и~её применения, 2019. Т.~13. 
Вып.~4. С.~90--96.
\bibitem{10-gz}
\Au{Добровольский Д.\,О.} Беседы о немецком слове.~--- М.: Языки славянской культуры, 
2013. 744~с.
\bibitem{11-gz}
\Au{Lehmann Chr.} Lexicography. Microstructure: Structure of a~lexical entry. {\sf 
https://www.\linebreak christianlehmann.eu/ling/ling\_meth/ling\_description/\linebreak lexicography/index.html}.
\bibitem{12-gz}
Языкознание: Большой энциклопедический словарь~/ Гл.\ ред. В.\,Н.~Ярцева.~---  2-е 
изд.~--- М.: Большая Российская энциклопедия, 1998. 685~с.
\bibitem{13-gz}
\Au{Ide N., Kilgarriff~A., Romary~L.} A~formal model of dictionary structure and content~// 9th 
EURALEX Congress (International) Proceedings.~--- Stuttgart: Institut f$\ddot{\mbox{u}}$r 
Maschinelle Sprachverarbeitung, 2000. P.~113--126.
\bibitem{14-gz}
TEI P5: Guidelines for Electronic Text Encoding and Interchange. Version~4.2.1.~--- TEI 
Consortium, 2021. {\sf https://tei-c.org/release/doc/tei-p5-doc/ en/Guidelines.pdf}.
\bibitem{15-gz}
\Au{Вакуленко В.\,В., Гончаров~А.\,А., Дурново~А.\,А., Зацман~И.\,М.} Задачи базы данных 
фразеологического словаря и~стадии ее проектирования~// Системы и~средства 
информатики, 2020. Т.~30. №\,2. С.~113--123.
\bibitem{16-gz}
WordNet: An electronic lexical database~/ Ed. Chr.~Fellbaum.~--- Cambridge, MA, USA: MIT Press, 
1998. 423~p.


\bibitem{17-gz}
\Au{Лукашевич Н.\,В.} Тезаурусы в~задачах информационного поиска.~--- М.: Изд-во 
Московского ун-та, 2011. 512~с.
\bibitem{18-gz}
\Au{Гак В.\,Г., Триомф~Ж.} Фран\-цуз\-ско-рус\-ский словарь активного типа.~--- М.: 
Русский язык, 1991. 1056~с.
\bibitem{19-gz}
\Au{Успенский В.\,А.}  
Невт$\acute{\mbox{о}}$н--Ньют$\acute{\mbox{о}}$н--Нь$\acute{\mbox{ю}}$тон, или 
Сколько сторон имеет языковой знак?~//  
Сб. к~60-ле\-тию Андрея Анатольевича Зализняка <<Русистика. Славистика. Индоевропеистика>>.~--- 
М.: Индрик, 1996. С.~598--659.
 \end{thebibliography}

}
}

\end{multicols}

\vspace*{-7pt}

\hfill{\small\textit{Поступила в~редакцию 14.04.2021}}

%\vspace*{8pt}

%\pagebreak

\newpage

\vspace*{-28pt}

%\hrule

%\vspace*{2pt}

%\hrule

%\vspace*{-2pt}

\def\tit{STRUCTURING PRINCIPLES OF~ELECTRONIC DICTIONARY'S ENTRIES}


\def\titkol{Structuring principles of~electronic dictionary's entries}

\def\aut{A.\,A.~Goncharov and~I.\,M.~Zatsman}

\def\autkol{A.\,A.~Goncharov and~I.\,M.~Zatsman}


\titel{\tit}{\aut}{\autkol}{\titkol}

\vspace*{-11pt}




\noindent
Institute of Informatics Problems, Federal Research Center ``Computer Science and Control''
 of the Russian Academy of Sciences, 44-2~Vavilov Str., Moscow 119333, Russian Federation

 
\def\leftfootline{\small{\textbf{\thepage}
\hfill INFORMATIKA I EE PRIMENENIYA~--- INFORMATICS AND
APPLICATIONS\ \ \ 2021\ \ \ volume~15\ \ \ issue\ 2}
}%
\def\rightfootline{\small{INFORMATIKA I EE PRIMENENIYA~---
INFORMATICS AND APPLICATIONS\ \ \ 2021\ \ \ volume~15\ \ \ issue\ 2
\hfill \textbf{\thepage}}}

\vspace*{3pt}



\Abste{Two tasks that arise when converting paper dictionaries into an electronic 
form are considered. In the first place, the authors suggest structuring inherited 
dictionary entries which provides the enrichment of the electronic dictionary's 
functionality, and in the second place, replacing the decorative design of the structural 
elements of dictionary entries with tagging that provide their addressing in 
databases. It is shown that the structure of dictionary entries used in traditional 
lexicography should be detailed. Simultaneously, it is necessary to categorize some 
of the structural elements to enrich the electronic dictionary's functionality. An 
approach to creating a~classification system integrated into an electronic 
dictionary and classifying dictionary entries' structural items is described. The 
proposed solutions allow to significantly enrich the electronic dictionary's 
functionality compared to its paper version and overcome traditional lexicography 
limitations related to the paper form of dictionary representation.}

\KWE{structuring principles; electronic dictionary; electronic lexicography; 
classification system}



\DOI{10.14357/19922264210213}

\vspace*{-15pt}

 \Ack
\noindent
The study was conducted at the Institute of Informatics Problems of the Federal 
Research Center ``Computer Science and Control'' of the Russian Academy of 
Sciences with financial support from the Russian Foundation for Basic Research (grant 
No.\,20-012-00166).

%\vspace*{12pt}

  \begin{multicols}{2}

\renewcommand{\bibname}{\protect\rmfamily References}
%\renewcommand{\bibname}{\large\protect\rm References}

{\small\frenchspacing
 {%\baselineskip=10.8pt
 \addcontentsline{toc}{section}{References}
 \begin{thebibliography}{99}
\bibitem{1-gz-1}
\Aue{Pruvost, J.} 2000. Des dictionnaires papier aux dictionnaires 
$\acute{\mbox{e}}$lectroniques. VIIe Journ$\acute{\mbox{e}}$e des dictionnaires 
(22~mars 2000). Rapport de colloque. \textit{Int. J.~Lexicogr.}  
13(3):187--193. doi: 10.1093/ijl/13.3.187.
\bibitem{2-gz-1}
\Aue{Weschler, R., and Chr.~Pitts.}  An experiment using electronic dictionaries 
with EFL students. Available at: {\sf  
http://iteslj.org/Articles/Weschler-ElectroDict.html} (accessed May~17, 2021).
\bibitem{3-gz-1}
Granger, S., and M.~Paquot, eds. 2012. \textit{Electronic lexicography}. Oxford 
University Press. 517~p.
\bibitem{4-gz-1}
\Aue{Vakulenko, V.\,V., and I.\,M.~Zatsman.} 2021. Nasleduemye 
leksikograficheskie resursy bazy dannykh frazeologicheskogo slovarya 
[Inheritable lexicographic resources of the phraseological dictionary database]. 
\textit{Sistemy i~Sredstva Informatiki~--- Systems and Means of Informatics}  
31(2):129--138.
\bibitem{5-gz-1}
Dobrovol'skiy, D.\,O., ed. 2021 (in press). \textit{Nemetsko-russkiy slovar' 
aktual'noy leksiki} [German--Russian dictionary of actual vocabulary]. Moscow: 
Leksrus.
\bibitem{6-gz-1}
\Aue{Dobrovol'skiy, D.\,O., and A.\,A.~Zaliznyak.} 2018. Ne\-me\-tskie konstruktsii 
s~modal'nymi glagolami i~ikh russkie sootvetstviya: proekt nadkorpusnoy bazy 
dannykh [German constructions with modal verbs and their Russian correlates: 
A~supracorpora database project]. \textit{Komp'yuternaya lingvistika 
i~intellektual'nye tekhnologii: po mat-lam Mezhdunar. konf. ``Dialog''} 
[Computational Linguistics and Intellectual Technologies. Papers from the Annual 
Conference (International) ``Dialogue'']. Moscow. 17(24):172--184.
\bibitem{7-gz-1}
\Aue{Goncharov, A.\,A., I.\,M.~Zatsman, and M.\,G.~Kruzhkov.} 2020. 
Evolyutsiya klassifikatsiy v~nadkorpusnykh ba\-zakh dannykh [Evolution of 
classifications in supracorpora databases]. \textit{Informatika i~ee  
Primeneniya~--- Inform. Appl.} 14(4):108--116.
\bibitem{8-gz-1}
\Aue{Goncharov, A.\,A., I.\,M.~Zatsman, and M.\,G.~Kruzhkov.} 2021. 
Predstavlenie novykh leksikograficheskikh znaniy v~dinamicheskikh 
klassifikatsionnykh sistemakh [Representation of new lexicographical knowledge 
in dynamic classification systems]. \textit{Informatika i~ee Primeneniya~--- 
Inform. Appl.} 15(1):82--89.
\bibitem{9-gz-1}
\Aue{Goncharov, A.\,A., I.\,M.~Zatsman, and M.\,G.~Kruzhkov.} 2019. 
Temporal'nye dannye v~leksikograficheskikh bazakh znaniy [Temporal data in 
lexicographic databases]. \textit{Informatika i~ee Primeneniya~--- Inform. Appl.} 
13(4):90--96.
{\looseness=1

}
\bibitem{10-gz-1}
\Aue{Dobrovol'skiy, D.\,O.} 2013. \textit{Besedy o~nemetskom slove} [Studies on 
German lexis]. Moscow: Yazyki slavyanskoy kul'tury. 744~p
\bibitem{11-gz-1}
\Aue{Lehmann, Chr.} Lexicography. Microstructure: Structure of a lexical entry. 
Available at: {\sf 
https://www.\linebreak christianlehmann.eu/ling/ling\_meth/ling\_description/\linebreak
lexicography/index.html} (accessed May~17, 2021).
\bibitem{12-gz-1}
Yartseva, V.\,N., ed. 1998. \textit{Yazykoznanie: Bol'shoy entsiklopedicheskiy 
slovar'} [Linguistics. Great encyclopedic dictionary]. 2nd ed. Moscow: Bol'shaya 
Rossiyskaya entsiklopediya. 685~p.
\bibitem{13-gz-1}
\Aue{Ide, N., A.~Kilgarriff, and L.~Romary.} 2000. A~formal model of dictionary 
structure and content. \textit{9th EURALEX
Congress (International) 
Proceedings}. Stuttgart: Institut f$\ddot{\mbox{u}}$r Maschinelle 
Sprachverarbeitung. 113--126.
\bibitem{14-gz-1}
TEI P5: Guidelines for electronic text encoding and interchange. Version~4.2.1. 
Available at: {\sf https://tei-c. org/release/doc/tei-p5-doc/en/Guidelines.pdf} 
(accessed May~17, 2021).
\bibitem{15-gz-1}
\Aue{Vakulenko, V.\,V., A.\,A.~Goncharov, A.\,A.~Durnovo, and 
I.\,M.~Zatsman.} 2020. Zadachi bazy dannykh fra\-ze\-o\-lo\-gi\-che\-sko\-go slovarya 
i~stadii ee proektirovaniya [Tasks of the phraseological dictionary database and 
stages of its design]. \textit{Sistemy i~Sredstva Informatiki~--- Systems and Means 
of Informatics} 30(2):113--123.
\bibitem{16-gz-1}
\Aue{Fellbaum, Ch.} 1998. \textit{WordNet: An electronic lexical database}. 
Cambridge, MA: MIT Press. 423~p.

\columnbreak

\bibitem{17-gz-1}
\Aue{Loukachevitch, N.\,V.}  2011. \textit{Tezaurusy v~zadachakh 
informatsionnogo poiska} [Thesauri in information retrieval tasks]. Moscow:  
Izd-vo Moskovskogo un-ta. 512~p.
\bibitem{18-gz-1}
\Aue{Gak, V.\,G., and Zh.~Triomf.} 1991. \textit{Frantsuzsko-russkiy slovar' 
aktivnogo tipa} [French--Russian dictionary of the active type]. Moscow: Russkiy 
yazyk. 1056~p.
\bibitem{19-gz-1}
\Aue{Uspenskiy, V.\,A.} 1996.  
Nevt$\acute{\mbox{o}}$n--N'yut$\acute{\mbox{o}}$n--N'y$\acute{\mbox{u}}$ton, ili Skol'ko 
storon imeet yazykovoy znak?  
[Nevt$\acute{\mbox{o}}$n--N'yut$\acute{\mbox{o}}$n--N'y$\acute{\mbox{u}}$ton, or How many sides 
does a~linguistic sign have?]. \textit{Sbornik k~60-letiyu Andreya Anatol'evicha\linebreak 
Zaliznyaka ``Rusistika. Slavistika. Indoevropeistika''} [A~collection of writings in 
honour of the 60th birthday of Andrey A.~Zaliznyak ``Russian studies. 
Slavic studies. Indo-European studies'']. Moscow: Indrik. 598--659.
{\looseness=1

}
\end{thebibliography}

 }
 }

\end{multicols}

\vspace*{-3pt}

  \hfill{\small\textit{Received April~14, 2021}}


%\pagebreak

%\vspace*{-8pt}  

\Contr

\noindent
\textbf{Goncharov Alexander A.} (b.\ 1994)~---
junior scientist, Institute of Informatics Problems, Federal Research Center 
``Computer Science and Control'' of the Russian Academy of Sciences,  
44-2~Vavilov Str., Moscow 119333, Russian Federation; 
\mbox{a.gonch48@gmail.com}

\vspace*{3pt}

\noindent
\textbf{Zatsman Igor M.} (b.\ 1952)~--- Doctor of Science in technology, Head of 
Department, Institute of Informatics Problems, Federal Research Center 
``Computer Science and Control'' of the Russian Academy of Sciences,  
44-2~Vavilov Str., Moscow 119333, Russian Federation; 
\mbox{izatsman@yandex.ru}

\label{end\stat}

\renewcommand{\bibname}{\protect\rm Литература}