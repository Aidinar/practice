\def\stat{krivenko}

\def\tit{МЯГКИЕ ВЫЧИСЛЕНИЯ В~ЗАДАЧАХ\\ МЕДИЦИНСКОЙ ДИАГНОСТИКИ}

\def\titkol{Мягкие вычисления в~задачах медицинской диагностики}

\def\aut{М.\,П.~Кривенко$^1$}

\def\autkol{М.\,П.~Кривенко}

\titel{\tit}{\aut}{\autkol}{\titkol}

\index{Кривенко М.\,П.}
\index{Krivenko M.\,P.}

%{\renewcommand{\thefootnote}{\fnsymbol{footnote}} \footnotetext[1]
%{Работа выполнена при финансовой поддержке РФФИ (проект 16-29-09458~офи\_м).}}


\renewcommand{\thefootnote}{\arabic{footnote}}
\footnotetext[1]{Институт проблем информатики Федерального исследовательского центра 
<<Информатика и~управление>> Российской академии наук, 
\mbox{mkrivenko@ipiran.ru}}


%\vspace*{-8pt}




\Abst{В последние годы возрастает значение информатики для интерпретации и~анализа 
данных с~использованием вычислительных методов, в~частности так называемых 
<<мягких>> вычислений (Soft Computing, SC). Рассматриваются возможности 
применения SC для решения проблем, связанных с~медициной, и~в~особенности в~задачах 
поддержки принятия решений. При этом демонстрируется, что не следует искусственно 
задействовать новации, тем более что ценой небольших усилий можно обратиться 
к~классическим подходам, методологически строгим и~приводящим к~гарантированным 
результатам. Несомненный интерес к~методологиям SC в~различных дисциплинах 
(генетика, физиология, радиология, кардиология, неврология и~т.\,д.)\ показывает, что их 
изучение крайне плодотворно, и~ожидается, что будущие исследования в~медицине будут 
использовать соответствующие методы в~большей степени, чем сегодня, и~для решения 
более сложных задач.}

\KW{медицина; мягкие вычисления; референсные значения; байесовский подход}

\DOI{10.14357/19922264210208}

\vspace*{3pt}


\vskip 10pt plus 9pt minus 6pt

\thispagestyle{headings}

\begin{multicols}{2}

\label{st\stat}

\section{Введение}

     <<Мягкие>> вычисления~--- не новый термин, он 
часто применяется в~компьютерных науках и~информационных технологиях. 
Инструментарий\linebreak <<мягких>> вычислений использует технику нечетких систем 
(нечеткие множества, нечеткая логика, <<грубые>> множества), искусственные 
нейронные сети, генетические алгоритмы и~эволюционное моделирование, 
в~том чис\-ле иммунные алгоритмы, алгоритмы роевого интеллекта. Приведенное 
описание состава <<мягких>> вычислений не является единственным, для 
этого достаточно сравнить аннотации для SC на сайтах трех издательств: 
Elsevier, Springer и~Wiley.
     
     У каждой компоненты SC есть свои достоинства. В~сочетании они 
представляют собой не просто набор инструментов, а скорее партнерство, 
в~котором каждый предлагает свою методологию для решения общей 
проблемы. Главное, что выделяется в~настоящее время при описании сути  
SC,~--- это единство отдельных подходов (методологий), которые работают 
синергетически и~предоставляют в~той или иной форме гибкие возможности 
оценки неоднозначных ситуаций в~реальной жизни.
%
 Цель применения SC 
состоит в~том, чтобы учесть допуски, неточности, неопределенности данных, 
приблизительность аргументации и~правдоподобия при построении вывода для 
получения гибких, но надежных недорогих решений.
     
     В последние годы наблюдается рост биоинформатики и~медицинской 
информатики с~использованием вычислительных методов для интерпретации 
и~анализа данных. Если ограничиться базовыми
 методологиями нечеткой логики 
(Fuzzy Logic, FL),
нейронных сетей (Neural Networks, NN) и~применения 
генетического алгоритма (Genetic Algorithm,
\begin{table*}\small %tabl1
%\vspace*{-12pt}
\begin{center}
\Caption{Число упомянутых в~Medline работ по группам методов за два последних 
десятилетия}
\vspace*{2ex}

\tabcolsep=9pt
\begin{tabular}{|c|l|c|c|}
\hline
Группа методов&\multicolumn{1}{c|}{Применяемые методы}&2001--2010~гг.&2011--2020~гг.\\
\hline
 &Только FL&206&332\\
Одиночные&Только NN&4\,633\hphantom{9\,}&11\,431\hphantom{9\,9}\\
&Только GA&677&970\\
\hline
\raisebox{-18pt}[0pt][0pt]{Смешанные}&Только FL и~NN&\hphantom{99}2&\hphantom{99}8\\
&Только FL и~GA&\hphantom{99}2&\hphantom{99}1\\
&Только NN и~GA&\hphantom{9}66&119\\
&FL и~NN и~GA&\hphantom{99}1&\hphantom{99}1\\
\hline
Общая&Всевозможные комбинации FL, NN, GA&5\,587\hphantom{9\,}&12\,862\hphantom{9\,9}\\
\hline
\end{tabular}
\end{center}
\end{table*}
  GA), то поиск в~базе данных 
Medline по названиям работ за два последних десятилетия даст результаты, 
представленные в~табл.~1. Из нее видно: число публикаций во второй декаде 
увеличилось на~130\%, что действительно свидетельствует о возрастании 
в~медицине внимания к~SC; доля смешанных методов по отношению к~базовым 
(чистым) подходам не изменилась и~составляет~1\%, т.\,е.\ о~всеохватывающем 
синергетическом эффекте SC пока говорить не приходится; бросающееся 
в~глаза внимание к~генетическим алгоритмам демонстрирует проявление 
близости гуманитарной природы медицины и~сути эволюционных алгоритмов.
     

     Цель данной статьи состоит в~том, чтобы продемонстрировать 
возможности применения SC для решения проблем, связанных с~медициной, 
и~в~особенности в~задачах поддержки принятия решений. Несомненный интерес 
к~изучению методологий SC в~различных дисциплинах (генетика, физиология, 
радиология, кардиология, неврология и~т.\,д.)\ показывает, что их освоение 
очень плодотворно, и~ожидается, что будущие исследования в~медицине будут 
использовать соответствующие методы в~большей степени, чем сегодня, и~для 
решения более сложных задач. При этом не следует искусственно 
задействовать новации, тем более что ценой небольших усилий можно 
обратиться к~классическим подходам, методологически строгим и~приводящим к~гарантированным результатам.
     
\section{Области применения}

     В медицине можно выделить четыре области: фундаментальная, 
диагностическая, клиническая и~хирургическая.
     
     \textbf{Фундаментальная медицина} характеризуется наличием 
множества явлений с~крайне сложным взаимодействием отдельных элементов, 
с~одной стороны, и~небогатым опытом моделирования подобных явлений, 
с~другой. По этой причине она подходит для всех методологий SC. В~первую 
очередь речь идет о~биохимических, генетических, физиологических 
и~фармакологических отраслях. Не следует забывать о~междисциплинарных 
направлениях типа науки о~здоровье.
     
     Результаты исследования источников по использованию методологий  
FL--NN приводят к~таким отраслям, как цитология, физиология, генетика 
и~биостатистика. Ярким проявлением \mbox{заинтересованности} фундаментальной 
науки в~SC служит обработка изображений~[1].
     
     Другой интересный пример~--- применение FL--NN в~генетике. 
В~частности, комбинированный метод для прогнозирования раковой ткани по 
данным экспрессии генов~[2]~--- нейрокомпьютинг, основанный на  
знаниях,~--- использовался для создания нечетких правил, которые указывают 
на гены, тесно связанные с~определенными типами рака.
     
     Категория FL--GA представлена несколькими областями исследований, 
причем в~первую очередь речь идет о генетике. В~исследовании экспрессии 
генов~[3] предпочтение было отдано FL-конт\-рол\-ле\-ру, настроенному 
с~помощью GA. Был предложен интерпретируемый классификатор с~точной 
и~компактной базой нечетких правил для анализа данных на микрочипах.
     
     Категория NN--GA стала наиболее предпочтительной для дисциплин 
фундаментальных наук: биохимии, биостатистики, генетики, гистологии, 
патологии, фармакологии и~физиологии.
     
     Обращает внимание пример использования указанной методологии 
в~фармакологии при изучении взаимодействия лекарственных средств 
с~живыми\linebreak организмами. В~[4] исследована модель для прогнозирования 
проницаемости структурно различных препаратов в~зависимости от выбранных 
\mbox{молекулярных} дескрипторов с~использованием искусственных нейронных 
сетей, а генетический алгоритм используется для выбора подмножества\linebreak 
дескрипторов, которые наилучшим образом описывают степень проникновения 
препарата.
     
     Не всегда привлечение отдельных методов SC оказывается ожидаемым: 
авторы~[5] использовали методологию GA в~качестве алгоритма обучения при 
сравнении производительности NN, они отметили, что GA оказывается 
неэффективным при тонкой настройке локального поиска.
     
     \textbf{Диагностическая медицина} в~основном включает в~себя 
радиологические и~клинические лабораторные исследования.
     
     Преимущественно SC применяются в~интервенционной радиологии. При 
этом превалируют FL--NN приложения.
       
     Сегментация цифровых изображений~--- один из наиболее важных 
этапов их анализа. Изображения всегда содержат значительный уровень шума 
(вызванного действиями оператора, оборудованием и~окружающей средой), что 
может привести к~серь\-ез\-ным неточностям при сегментации. Авторы~[6] 
эффективно использовали метод FL--NN в~своих исследованиях для решения 
задач маг\-нит\-но-ре\-зо\-нанс\-ной томографии.
     
     Клиническая медицина стала наиболее популярной и~подходящей 
областью для применения методологии SC. Сравнение предпочтений в~этой 
области указывает на первенство FL--NN и~NN--GA, при этом на первых ролях 
такие отрасли, как кардиология, неврология, терапия критических состояний, 
анестезиология, физическая медицина и~реабилитационная медицина.
     
     В области анестезии есть ряд убедительных примеров использования 
адаптивных систем для контроля артериального давления, обезболивания, 
паралича, потери сознания и~септического шока. Примером последних 
публикаций на эту тему может служить~[7], где исследуется применимость 
адаптивной ней\-ро-не\-чет\-кой стратегии в~управлении анестезией с~обратной 
связью. Показывается, что построенный контроллер адаптивен и~устойчив 
к~проблеме вариабельности между пациентами и~по отношению к~каждому из 
них, а также эффективен в~условиях наличия шума измерительных устройств.
     
     Большая часть работы по использованию адап\-тив\-ных систем в~области 
кардиологии была направлена на кардиостимуляторы. Здесь активно 
используется модель ANFIS (Adaptive Neuro-Fuzzy Inference System), 
исследование~[8] которой продемонстрировало, что ее показатели точности 
выше, чем у автономной модели NN.
     
     По сравнению с~FL--NN методология NN--GA оказывается менее 
предпочтительной в~клинической науке. Несмотря на это, имеют место 
некоторые успешные применения в~нефрологии, пульмонологии, неврологии, 
психиатрии, физической и~реабилитационной медицине.
     
     В~[9] представлен способ классификации результатов аускультации 
легких: GA использован с~целью поиска оптимальной структуры и~па\-ра\-мет\-ров 
обучения NN для лучшего прогнозирования динамики шумов.
     
     \textbf{Хирургия.} В~[10] указывается, что результаты поиска в~базе 
данных Medline показали, что хирургическая наука не обращается к~тем или 
иным элементам SC. Причина проста: существует общее мнение о хирургии, 
что она в~основном связана с~навыками хирурга. А~работы, освещающие 
<<околохирургические>> проблемы, относят обычно к~другим отраслям 
медицины. Примером может служить~[11], где предлагаются решения на 
основе SC для прогнозирования послеоперационной выживаемости при раке 
легкого. Встречаются работы по использованию SC при описании объектов 
хирургической практики: в~исследованиях~[12] применяется подход, 
основанный на эволюционной технике вкупе с~искусственной нейронной сетью 
для нахождения решений нелинейных обыкновенных дифференциальных 
уравнений второго порядка, используемых в~качестве модели роговицы глаза.
     
     В последнее время безразличие хирургии к~SC уходит в~прошлое. 
Современные операции чаще всего проводятся с~использованием 
роботизированных хирургических инструментов и~другого обору\-до\-ва\-ния, 
поэтому все более актуальными\linebreak становятся работы типа определения  
инстру\-мен\-таль\-но-тка\-не\-во\-го нажима в~роботизированной 
лапароскопической хирургии с~использованием нейро-эволюционных нечетких 
сис\-тем~[13].
     
     Другие примеры применения SC в~различных отраслях медицины можно 
найти в~[10], а именно: критическая медицина (управление искусственной\linebreak 
вентиляцией легких, контроль развития септического шока во время 
пребывания пациентов в~отделении интенсивной терапии, обнаружение 
нормальной и~искаженной плетизмограммы);\linebreak \mbox{неврология} (распознавание 
стадии сна на основе\linebreak знаний и~обработки биосигналов); физическая\linebreak 
и~реабилитационная медицина (контроль ин\-тен\-сив\-ности физической нагрузки 
спортсмена, диагностика и~контроль тремора); дерматология\linebreak 
(дифференциальная диагностика эри\-те\-ма\-тоз\-но-сква\-моз\-ных очагов); 
эндокринология (моделирование динамики сложных метаболических сис\-тем 
применительно к~внут\-ри\-кле\-точ\-ной кинетике тиамина); онкология (поддержка 
принятия решений по идентификации подтипа клеток острого лимфобластного 
лейкоза у~детей с~использованием данных по экспрессии генов; 
прогнозирование рака молочной железы и~простаты); гастроэнтерологии 
(дифференциальная диагностика при сложных же\-лу\-доч\-но-ки\-шеч\-ных 
расстройствах, таких как диспепсический синд\-ром или хронический 
панкреатит, когда симп\-то\-мы у~пациента могут лежать на пересечении группы 
расстройств).
     
\section{Поддержка принятия решений на~основе референсных 
значений}

     Поддержка принятия решений при диагностировании заболеваний по 
результатам клинических исследований может рассматриваться как задача 
классификации набора показателей, входящих в~анализы отдельного пациента, 
где класс~--- это определенный синдром. При этом результат классификации 
должен представлять собой совокупность диагнозов, упорядоченную 
в~направлении от наиболее к~наименее правдоподобным. При нечетком 
задании исходных данных совершенно естественно обратиться к~возможностям 
SC, тем более что для этого имеются соответствующие наработки в~виде 
монографий (например,~[14]) и~отдельных публикаций.

\begin{table*}\small %tabl2
\begin{center}
\Caption{Характеристики синдрома <<первичный гиперпаратиреоз>>}
\vspace*{2ex}

\begin{tabular}{|c|l|l|c|}
\hline
№&\multicolumn{1}{c|}{$f_i$}&
\tabcolsep=0pt\begin{tabular}{c}Критическая\\ область\\ показателя $S_i$\end{tabular}&
\tabcolsep=0pt\begin{tabular}{c}Вес \\ показателя $\nu_i$\end{tabular}\\ 
\hline 
1&Кальций в~крови&\hspace*{5mm}$> 2{,}55$&8\\ 
2&Паратгормон в~крови&\hspace*{5mm}$> 55$&7\\ 
$\ldots$&$\ldots$&\hspace*{5mm}$\ldots$&$\ldots$\\ 
13\hphantom{9}&Отношение клиренс кальция\,/\,клиренс креатинина&\hspace*{5mm}$> 
0{,}02$&8\\ 
\hline
&Сумма весов&&87\hphantom{9}\\ 
\hline 
\end{tabular} 
\end{center} 
\end{table*}
     
     В случае когда показатели суть категориальные и/или числовые, 
нечеткие рассуждения можно перевести на формальный язык математической 
статистики и~теории принятия решений. Это позволит опереться на строгую 
методологическую основу, формулировать и~решать возникающие задачи, ясно 
отдавать себе отчет в~том, что и~как получено в~результате.
     
     Далее будем считать, что при построении классификатора заданы 
следующие характеристики: час\-то\-та встречаемости ситуаций типа 
<<определенный диапазон значений по\-ка\-за\-те\-ля\,--\,кон\-крет\-ный синдром>>, 
причем такие данные имеются далеко не по всем комбинациям  
<<по\-ка\-за\-тель--синд\-ром>>; частота встречаемости в~генеральной 
совокупности данных тех или иных синдромов.
     
     Показатели в~зависимости от сложности их задания разбиваются на 
следующие группы (примеры в~скобках приведены для стандартного 
клинического анализа крови без указания единиц измерения): бинарные 
(например, наличие или отсутствие теста по определению общих 
триглицеридов); однопороговые (например, при определении АСТ норма 
составляет диапазон от~0 до~40); многопороговые (например, при определении 
ЛПНП-бета оптимальные значения суть 0,3--2,4, близкие к~оптимальным~--- 
от~2,5 до~3,1, погранично высокие~--- от~3,2 до~3,9, высокие~--- от~4,0 
до~4,8, очень высокие~--- более~4,8).
     
     Для конкретности далее рассмотрим задачу, которая возникла в~ходе 
разработки информа\-ци\-он\-но-ана\-ли\-ти\-че\-ской автоматизированной системы 
<<Мегалит>>~[15].
     
     Распространенным на практике алгоритмом классификации набора 
показателей служит метод голосования, когда для показателей определены 
области нор\-маль\-ных/ано\-маль\-ных референсных значений, а вес синдрома 
определяется через долю случаев, когда значение показателя отклонилось от 
нормы.
     
     Пусть показатель, входящий в~состав анализа, обозначен как~$f_i$, 
$i\hm=1,\ldots , d$, а некоторый синдром~--- $D_j$, $j\hm=1,\ldots , M$. 
Конкретно речь шла о~33~показателях (в частности, кальций в~крови, 
паратгормон в~крови и~т.\,п.), $d\hm=33$, и~6~синдромах (в~част\-ности, 
первичный гиперпаратиреоз, вторичный гиперпаратиреоз и~т.\,п.), $M\hm=6$. 
Представление об исходных данных для одного из синдромов можно получить 
из табл.~2 (используется нотация специалистов предметной области, для 
сокращения записи опущены единицы измерения). Вес показателя, 
приведенный в~последнем столбце этой таблицы, взят из опыта 
диагностирования и~фактически означает относительную частоту 
встречаемости. Заметим, что в~матрице <<по\-ка\-за\-тель--синд\-ром>> всего 
198~элементов, при этом 129~комбинаций оказались незаданными.
     

     
     Таким образом, алгоритм классификации методом голосования, 
обрабатывающий данные для комбинаций~$f_i$, $D_j$ и~фор\-ми\-ру\-ющий 
в~результате перечень весов диагнозов, принимает вид: 
для $j\hm=1,\ldots ,M$  
подсчитать веса диагнозов по формуле
     $$
     W_j= \fr{\sum\nolimits^d_{i=1} \mathbf{1}_{S_i}(f_i) 
\nu_i}{\sum\nolimits^d_{i=1} \nu_i}\,,
     $$
      где $\mathbf{1}_A (x)$~--- индикатор множества~$A$, 
после чего следует отдать предпочтение диагнозу с~наибольшим значением 
веса. Данный алгоритм является эвристическим и~по виду соответствует 
результату обучения NN.

\begin{table*}\small%$tabl3
\begin{center}
\Caption{Представление многопорогового показателя}
\vspace*{2ex}

\begin{tabular}{|c|c|c|c|c|c|c|c|}
\hline

$j$&$S_2$&$\nu_2$&\multicolumn{5}{c|}{Вероятности для пяти категорий }\\
\hline
1&$>55$&7&0,3/2&0,3/2&0,7/3&0,7/3&0,7/3\\
2&\hphantom{9}$>200$&9&0,1/3&0,1/3&0,1/3&0,9/2&0,9/2\\
3&\hphantom{9}$>500$&9&0,1/4&0,1/4&0,1/4&0,1/4&0,9/1\\
4&$>10$&5&0,5/1&0,5/4&0,5/4&0,5/4&0,5/4\\
5&$>55$&8&0,2/2&0,2/2&0,8/3&0,8/3&0,8/3\\
6&N/A&N/A&0,2&0,2&0,2&0,2&0,2\\
\hline
\end{tabular}
\end{center}
\end{table*}

     Использование референсных интервалов для принятия решений создает 
предпосылки для привлечения нечетких множеств, а фактическое отсутствие 
обоснованных правил постановки диагноза по совокупности показателей~--- 
для мобилизации методов SC, в~частности нейронных сетей. В~последнем 
случае естественней опереться на обучающую выборку, которой в~описываемой 
ситуации нет.
  %   
     Поэтому была сделана попытка продвинуться по пути формирования 
вероятностной модели так, чтобы можно было моделировать исходные данные, 
а также ставить и~решать задачу построения решающих правил. Ее основу 
составило представление отдельного показателя в~виде категориальной 
переменной с~вероятностями появления отдельных значений.
     
     Примером однопорогового показателя служит <<кальций в~крови>>. 
Соответствующая переменная принимает два значения: категория <<1>> 
(значения $f_1\hm<2{,}55$); категория~<<2>> (значения $f_1\hm\geq 2{,}55$). 
Вероятности появления категорий зависят от класса~--- синдрома: так, для 
$j\hm=1$ имеем $S_1\hm= (>2{,}55)$, $\nu_1\hm=8$ и~вероятности категорий 
$(0{,}2; 0{,}8)$, а~для $j\hm=2$ имеем $S_1\hm= (<2{,}55)$, $\nu_1\hm=8$ 
и~вероятности категорий $(0{,}8; 0{,}2)$. При описании случаев, когда 
значения показателей не заданы, все исходы принимались равновероятными. 
Подобных однопороговых показателей среди исходных оказалось всего~23.
     
     В качестве примера многопорогового показателя рассмотрим 
<<паратгормон в~крови>>, для которого большинству 
синдромов~соответствуют свои пороговые значения (см.\ столбец~$S_2$ 
табл.~3), поэтому число категорий возрастает до пяти. Соответствующие 
вероятности для категорий приведены в~последних столбцах табл.~3. Запись 
вероятности в~виде, например, 0,7/3 наглядно демонстрирует использование 
столбца весов~$\nu_2$. Для обозначения ситуации, когда значение показателя не 
задано, используется аббревиатура N/A. Подобных многопороговых 
показателей набралось~8.
     

     
     Примером бинарных показателей может служить <<вид мочевой 
инфекции>>, сводящийся к~фиксации наличия или отсутствия мочеви\-но\-раз\-ла\-га\-ющей 
флоры. В~этом случае вводятся две \mbox{категории}: если появление 
показателя не задано, то вероятности категорий $(0{,}5; 0{,}5)$, если же 
мо\-че\-ви\-но\-раз\-ла\-га\-ющая флора встречается, вероятности категорий 
$(1{,}0;0{,}0)$. Подобные двоичные показатели встретились~2~раза.
     
     Перечисленные группы показателей исчерпывают все имеющиеся на 
данный момент варианты. Таким образом, результаты анализов описываются 
случайным вектором, каждая координата которого имеет дискретное 
распределение, в~принципе, с~разным числом исходов. Теперь появляется 
возможность строить оптимальные правила классификации, в~частности на 
байесовских прин\-ципах.
{\looseness=1

}
     
     \textbf{Сравнительный анализ классификаторов.} Построенная 
вероятностная модель данных позволяет провести сравнительный анализ 
алгоритмов классификации как методом голосования, так и~байесовским. 
В~качестве критерия использовалась вероятность правильной 
классификации~$p_{\mathrm{CC}}$. Для некоторого классификатора, заданного 
с~помощью вероятностных характеристик, можно построить матрицу 
результатов классификации~$\mathbf{C}$, где $c_{ij}$~--- вероятность 
отнесения классифицируемого элемента из $i$-го класса к~$j$-му. Тогда 
$p_{\mathrm{CC}}\hm= \mathrm{tr}\,(\mathbf{C})$.
     
     Для реализации байесовского классификатора необходимо определиться с~его параметрами: в~качестве функции потерь рассматривалась единичная, 
объединение отдельных показателей осуществлялось в~предположении их 
независимости; априорные вероятности классов~$\pi_{j}$ принимались либо 
равными (случай отсутствия информации о~вероятностях появления 
синдромов), либо <<реальными>>, полученными на основе анализа 
прецедентов; в~последнем случае речь шла о значениях превалентности 
$\boldsymbol{\pi}\hm= (0{,}33; 0{,}49; 0{,}05; 0{,}02; 0{,}08; 0{,}03)$.
     
     Дискретный характер данных создает иллюзию, что матрицу   можно 
посчитать точно, но из-за проб\-лем многомерности реализовать это практически 
не представляется возможным: общее число различных значений 
классифицируемых элементов со\-став\-ля\-ет 733\,835\,427\,840 ($\sim10^{12}$). 
Поэтому был применен метод моделирования, в~основе которого\linebreak лежит 
генератор из смеси дискретных распределений с~вероятностями появления 
элементов смеси~$\boldsymbol{\pi}$.
     
     Результаты сравнительного анализа двух классификаторов при числе 
экспериментов, равном~$10^4$, сведены в~табл.~4. Большое число 
экспериментов позволяет уверенно говорить о явных преимуществах 
байесовского классификатора.
     
\begin{table*}\small %tabl4
\begin{center}
\parbox{312pt}{\Caption{Результаты сравнительного анализа классификаторов на основе оценок~$p_{\mathrm{CC}}$}
}

\vspace*{2ex}

\begin{tabular}{|l|c|c|}
\hline
\multicolumn{1}{|c|}{Способ классификации}&
\tabcolsep=0pt\begin{tabular}{c}Равновероятные\\ синдромы\end{tabular}&
\tabcolsep=0pt\begin{tabular}{c}<<Реальные>>\\ вероятности\\ синдромов\end{tabular}\\
\hline
Байесовский классификатор&74,3\%&82,4\%\\
Классификация методом голосования&62,6\%&59,7\%\\
\hline
\end{tabular}
\end{center}
\end{table*}
     
     
     Как кажется на первый взгляд, условия задачи поддержки принятия 
решений при постановке диагноза предопределяют использование SC. Но 
дополнительные усилия по формированию вероятностной модели принесли 
свои плоды: методологически выверенное построение модели помогло 
формализовать постановку задачи анализа; использование байесовского 
классификатора обеспечило учет априорной информации 
в~б$\acute{\mbox{о}}$льшем объеме (например, вероятности появления 
синдромов), гарантировало наименьшие потери, а~в~случае единичной 
функции потерь и~наименьшие значения вероятности ошибки; предложенная 
вероятностная модель данных позволила на основе имеющейся информации 
проводить исследования методов и~алгоритмов анализа данных 
и~систематизировать алгоритмы принятия решений, строить схемы 
последующего обучения процедур обработки.
     
     Понятно, что все это удается сделать для оказавшейся приемлемой 
сложности реальной задачи. Если это не так, то фактор многомерности данных 
(многочисленность показателей, синдромов, отдельных значений показателей) 
просто сделает актуальной постановку задач 
о~снижении размерности, приведет к~необходимости развития соответствующих методов.

\vspace*{-8pt}

\section{Заключение}

\vspace*{-2pt}

     Анализ применения SC проведен по основным медицинским 
приложениям. На основании этого можно предварительно прогнозировать 
дальнейшее развитие этой технологии в~медицине, получить представление 
о~возникающих типовых задачах анализа данных. Из представленного 
описания различных приложений можно увидеть, что методы SC применяются 
в~широком спектре областей медицины для визуализации, диагностики, 
прогнозирования и~контроля протекающих процессов.
     
     Методологии SC, которые имитируют человеческий стиль мышления 
     и~принятия решений при решении сложных проблем, может преодолеть 
недостатки традиционных систем поддержки принятия медицинских решений, 
основанных на статистических моделях и~традиционных методах 
искусственного интеллекта. Как и~все другие приближенные методы,  
SC-ме\-то\-ды имеют относительные преимущества и~недостатки.
     
     Примеры постановки рассмотренных задач с~использованием 
<<мягких>> вычислений свидетельствуют о возможности применения 
соответствующих методов, а не о необходимости прибегать к~ним. 
Единственным исключением служит ситуация, когда исходная проблема 
формулируется на языке лингвистических переменных. Но и~здесь не очевидно, 
что надо обращаться к~нечеткому выводу, нейронным сетям, генетическим 
алгоритмам и~т.\,п., а~не пытаться воспользоваться методологически 
выверенным математическим подходом, дающим четкое представление 
о~возникающих ограничениях и~гарантирующим свойства построенных 
решений.
     
     Надо признать, что при использовании <<мягких>> вычислений 
возникают множества скрытых параметров, задача выбора значений которых не 
решается, а подменяется общими рекомендациями о влиянии этих параметров 
на итоговое качество принимаемых решений.
     
     Крайне затрудняет освоение полученных результатов то, что источники 
по <<мягким>> вы\-чис\-лениям подчас содержат, казалось бы, мелкие 
не\-точности, но порождающие большие сомнения.\linebreak Публи\-ка\-ции по <<мягким>> 
вычислениям подчас \mbox{носят} рекламный характер, они хороши для знакомства 
с~новыми идеями в~области создания и~применения информационных 
технологий, но оказываются бесполезными, а~иногда и~вредными,\linebreak с~точки 
зрения специалистов, которым необходимо решать актуальные наукоемкие 
задачи практики.
     
     Все перечисленные недостатки не снижают интерес к~описанному 
подходу, а скорее подогревают. Причина этого в~первую очередь в~том, что 
далеко не все задачи практики удается пока формально полностью поставить 
и~тем более решить. На данный момент самым важным является улучшение 
понимания сильных и~слабых сторон идей и~методов SC, использование их 
лучших возможностей.

\vspace*{-8pt}

{\small\frenchspacing
{%\baselineskip=10.8pt
%\addcontentsline{toc}{section}{References}
\begin{thebibliography}{99}

\vspace*{-2pt}

\bibitem{1-kr}
\Au{Zanaty E.\,A., Ghoniemy~S.} Medical image segmentation techniques: An overview~// Int. 
J.~Informatics Medical Data Processing, 2016. Vol.~1. No.\,1. P.~16--37.
\bibitem{2-kr}
\Au{Catto J.\,W.\,F., Linkens~D.\,A., Abbod~M.\,F., Chen~M., Burton~J.\,L., Feeley~K.\,M., Hamdy~F.\,C.} 
Artificial intelligence in predicting bladder cancer outcome: A~comparison of neuro-fuzzy modeling 
and artificial neural networks~// Clin. Cancer Res., 2003. Vol.~9. No.\,11. P.~4172--4177.
\bibitem{3-kr}
\Au{Ho S.\,Y., Hsieh C.\,H., Chen~H.\,M., Huang~H.\,L.} Interpretable gene expression classifier with 
an accurate and compact fuzzy rule base for microarray data analysis~// Biosystems, 2006. Vol.~85. 
P.~165--176.
\bibitem{4-kr}
\Au{Agatonovic-Kustrin S., Evans~A., Alany~R.\,G.} Prediction of corneal permeability using 
artificial neural networks~// Pharmazie, 2003. Vol.~58. No.\,10. P.~725--729.
\bibitem{5-kr}
\Au{Ghaffari A., Abdollahi~H., Khoshayand M.\,R., Bozchaloi~S., Dadgar~A., Rafiee-Tehrani~M.} 
Performance comparison of neural network training algorithms in modeling of bimodal drug 
delivery~// Int. J.~Pharm., 2006. No.\,327. P.~126--138.
\bibitem{6-kr}
\Au{Shen S., Sandham~W., Granat~M., Sterr~A.} MRI fuzzy segmentation of brain tissue using 
neighborhood attraction with neural-network optimization~// IEEE T.~Inf. 
Technol.~ B., 2005. Vol.~9. No.\,3. P.~459--467.
\bibitem{7-kr}
\Au{Li R., Wu1 Q., Liu~J., Wu~Q., Li~C., Zhao~Q.} Monitoring depth of anesthesia based on hybrid 
features and recurrent neural network~// Front. Neurosci.~--- Switz., 2020. Vol.~14. Art.~26.
\bibitem{8-kr}
\Au{Ubeyli E.\,D., Guler~I.} Adaptive neuro-fuzzy inference systems for analysis of internal carotid 
arterial Doppler signals~// Comput. Biol. Med., 2005. Vol.~35. No.\,8. P.~687--702.
\bibitem{9-kr}
\Au{Guler I., Polat H., Ergun~U.} Combining neural network and genetic algorithm for prediction 
of lung sounds~// J.~Med. Syst., 2005. Vol.~29. No.\,3. P.~217--231.
\bibitem{10-kr}
\Au{Yardimci A.} Soft computing in medicine~// Appl. Soft Comput., 2009. Vol.~9. P.~1029--1043.
\bibitem{11-kr}
\Au{Iraji M.\,S.} Prediction of post-operative survival expectancy in thoracic lung cancer surgery 
with soft computing~// J.~Appl. Biomed., 2017. Vol.~15. Iss.~2. P.~151--159.
\bibitem{12-kr}
\Au{Waseem W., Sulaiman M., Alhindi~A., Alhakami~H.} Soft computing approach based on 
fractional order DPSO algorithm designed to solve the corneal model for eye surgery~// IEEE 
Access, 2020. Vol.~8. P.~61576--61592.
\bibitem{13-kr}
\Au{Mozaffari A., Behzadipour~S., Kohani~M.} Identifying the tool-tissue force in robotic 
laparoscopic surgery using neuro-evolutionary fuzzy systems and a synchronous self-learning hyper 
level supervisor~// Appl. Soft Comput., 2014. Vol.~14. Part~A. P.~12--30.
\bibitem{14-kr}
\Au{Chang M.} Artificial intelligence for drug development, precision medicine, and healthcare.~--- 
Boca Raton, FL, USA: Chapman \& Hall/CRC, 2020. 355~p.
\bibitem{15-kr}
\Au{Голованов С.\,А., Кривенко~М.\,П., Савченко~П.\,А., Сивков~А.\,В., Сучков~А.\,П.} 
Ин\-фор\-ма\-ци\-он\-но-ана\-ли\-ти\-че\-ская автоматизированная система <<Мегалит>> в~оптимизации 
диагностики и~лечения мочекаменной\linebreak болезни~// Информатика и~её применения, 2013. Т.~7. 
Вып.~4. С.~82--93.
 \end{thebibliography}

}
}

\end{multicols}

\vspace*{-6pt}

\hfill{\small\textit{Поступила в~редакцию 01.12.2020}}

\vspace*{8pt}

%\pagebreak

%\newpage

%\vspace*{-28pt}

\hrule

\vspace*{2pt}

\hrule

%\vspace*{-2pt}

\def\tit{SOFT COMPUTING IN~PROBLEMS OF~MEDICAL DIAGNOSTICS}


\def\titkol{Soft computing in~problems of~medical diagnostics}

\def\aut{M.\,P.~Krivenko}

\def\autkol{M.\,P.~Krivenko}

\titel{\tit}{\aut}{\autkol}{\titkol}

\vspace*{-11pt}


\noindent
Institute of Informatics Problems, Federal Research Center ``Computer Science and Control'' of the Russian 
Academy of Sciences, 44-2~Vavilov Str., Moscow 119333, Russian Federation
 
\def\leftfootline{\small{\textbf{\thepage}
\hfill INFORMATIKA I EE PRIMENENIYA~--- INFORMATICS AND
APPLICATIONS\ \ \ 2021\ \ \ volume~15\ \ \ issue\ 2}
}%
\def\rightfootline{\small{INFORMATIKA I EE PRIMENENIYA~---
INFORMATICS AND APPLICATIONS\ \ \ 2021\ \ \ volume~15\ \ \ issue\ 2
\hfill \textbf{\thepage}}}

\vspace*{3pt}


\Abste{In recent years, the importance of informatics has increased for the interpretation and analysis of data 
using computational methods, in particular, the so-called ``soft'' computing (Soft Computing~--- SC). The 
article discusses the possibilities of using SC for solving problems related to medicine and, especially, 
problems of decision support. At the same time, it is demonstrated that one should not artificially use 
innovations, especially since, at the cost of little effort, one can turn to classical approaches that are 
methodologically rigorous and lead to guaranteed results. The undoubted interest in the study of SC 
methodologies in various disciplines (genetics, physiology, radiology, cardiology, neurology, etc.)\ 
demonstrates that their study is extremely fruitful and it is expected that future research in medicine will use 
the corresponding methods to a greater extent than today and for more complex tasks.}

\KWE{medicine; soft computing; reference values; Bayesian approach}

\DOI{10.14357/19922264210208}

%\vspace*{-15pt}

% \Ack
%\noindent


\vspace*{6pt}

  \begin{multicols}{2}

\renewcommand{\bibname}{\protect\rmfamily References}
%\renewcommand{\bibname}{\large\protect\rm References}

{\small\frenchspacing
 {%\baselineskip=10.8pt
 \addcontentsline{toc}{section}{References}
 \begin{thebibliography}{99}
\bibitem{1-kr-1}
\Aue{Zanaty, E.\,A., and S.~Ghoniemy.} 2016. Medical image segmentation techniques: An overview. 
\textit{Int. J.~Informatics Medical Data Processing} 1(1):16--37.
\bibitem{2-kr-1}
\Aue{Catto, J.\,W.\,F., D.\,A.~Linkens, M.\,F.~Abbod, M.~Chen, J.\,L.~Burton, K.\,M.~Feeley, and F.\,C.~Hamdy.}
 2003. 
Artificial intelligence in predicting bladder cancer outcome: A~comparison of neuro-fuzzy modeling and 
artificial neural networks. \textit{Clin. Cancer Res.} 9(11):4172--4177.
\bibitem{3-kr-1}
\Aue{Ho, S.\,Y., C.\,H. Hsieh, H.\,M.~Chen, and H.\,L.~Huang.} 2006. Interpretable gene expression classifier with 
an accurate and compact fuzzy rule base for microarray data analysis. 
\textit{Biosystems} 85:165--176.
\bibitem{4-kr-1}
\Aue{Agatonovic-Kustrin, S., A.~Evans, and R.\,G.~Alany.}
2003. Prediction of corneal permeability using artificial 
neural networks. \textit{Pharmazie} 58(10):725--729.
\bibitem{5-kr-1}
\Aue{Ghaffari, A., H.~Abdollahi, M.\,R.~Khoshayand, S.~Bozchaloi., A.~Dadgar, and M.~Rafiee-Tehrani.} 2006. 
Performance comparison of neural network training algorithms in modeling of bimodal drug delivery. 
\textit{Int. J.~Pharm.} 327:126--138.
\bibitem{6-kr-1}
\Aue{Shen, S., W.~Sandham, M.~Grana, and A.~Sterr.} 2005. MRI fuzzy segmentation of brain tissue using 
neighborhood attraction with neural-network optimization. \textit{IEEE T.~Inf. Technol.~B.} 9(3):459--467.
\bibitem{7-kr-1}
\Aue{Li, R., Q.~Wu1, J.~Liu, Q.~Wu, C.~Li, and Q.~Zhao.} 2020. Monitoring depth of anesthesia
 based on hybrid 
features and recurrent neural network. \textit{Front. Neurosci.~---Switz.} 14:26. 11~p.
\bibitem{8-kr-1}
\Aue{Ubeyli, E.\,D., and I.~Guler.} 2005. Adaptive neuro-fuzzy inference systems for analysis of internal carotid 
arterial Doppler signals. \textit{Comput. Biol. Med.} 35(8):687--702.
\bibitem{9-kr-1}
\Aue{Guler, I., H.~Polat~H., and U.~Ergun.} 2005. Combining neural network and genetic algorithm for prediction 
of lung sounds. \textit{J. Med. Syst.} 29(3):217--231.
\bibitem{10-kr-1}
\Aue{Yardimci, A.} 2009. Soft computing in medicine. \textit{Appl. Soft Comput.} 9:1029--1043.
\bibitem{11-kr-1}
\Aue{Iraji, M.\,S.} 2017. Prediction of post-operative survival expectancy in thoracic lung cancer surgery with 
soft computing. \textit{J.~Appl. Biomed.} 15(2):151--159.
\bibitem{12-kr-1}
\Aue{Waseem, W., M.~Sulaiman, A.~Alhindi, and H.~Alhakami.} 2020. Soft computing approach based on 
fractional order DPSO algorithm designed to solve the corneal model for eye surgery. 
\textit{IEEE Access} 8:61576--61592.
\bibitem{13-kr-1}
\Aue{Mozaffari, A., S.~Behzadipour, and M.~Kohani.} 2014. Identifying the tool-tissue force in robotic 
laparoscopic surgery using neuro-evolutionary fuzzy systems and a synchronous self-learning hyper level 
supervisor. \textit{Appl. Soft Comput.} 14(A):12--30.
\bibitem{14-kr-1}
\Aue{Chang, M.} 2020. \textit{Artificial intelligence for drug development, precision medicine, and healthcare}. Boca 
Raton, FL, USA: Chapman \& Hall/CRC. 355~p.
\bibitem{15-kr-1}
\Aue{Golovanov, S.\,A., M.\,P.~Krivenko, P.\,A.~Savchenko, A.\,V.~Sivkov, and A.\,P.~Suchkov.} 2013. 
Informatsionno-analiticheskaya avtomatizirovannaya sistema ``Megalit'' v~optimizatsii diagnostiki 
i~lecheniya mochekamennoy bolezni [The information-analytical computer system ``Megalith'' in optimization 
of the diagnosis and treatment of urolithiasis]. \textit{Informatika i~ee Primeneniya~--- Inform. Appl.}
 7(4):82--93.
 
 \end{thebibliography}

 }
 }

\end{multicols}

\vspace*{-3pt}

  \hfill{\small\textit{Received December~1, 2020}}


%\pagebreak

%\vspace*{-8pt}  



\Contrl

\noindent
\textbf{Krivenko Michail P.} (b.\ 1946)~--- Doctor of Science in technology, professor, leading scientist, 
Institute of Informatics Problems, Federal Research Center ``Computer Science and Control'' of the Russian 
Academy of Sciences, 44-2~Vavilov Str., Moscow 119333, Russian Federation; 
\mbox{mkrivenko@ipiran.ru}

\label{end\stat}

\renewcommand{\bibname}{\protect\rm Литература}