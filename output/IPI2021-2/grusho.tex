\def\stat{grusho}

\def\tit{ИНТЕЛЛЕКТУАЛЬНЫЙ АНАЛИЗ ПОПОЛНЯЕМЫХ КОЛЛЕКЦИЙ BIG DATA В~РЕЖИМЕ 
ПРОЦЕССНО-РЕАЛЬНОГО ВРЕМЕНИ$^*$}

\def\titkol{Интеллектуальный анализ пополняемых коллекций Big Data в~режиме 
процессно-реального времени}

\def\aut{А.\,А.~Грушо$^1$, М.\,И.~Забежайло$^2$, Д.\,В.~Смирнов$^3$, 
Е.\,Е.~Тимонина$^4$}

\def\autkol{А.\,А.~Грушо, М.\,И.~Забежайло, Д.\,В.~Смирнов, 
Е.\,Е.~Тимонина}

\titel{\tit}{\aut}{\autkol}{\titkol}

\index{Грушо А.\,А.}
\index{Забежайло М.\,И.}
\index{Смирнов Д.\,В.}
\index{Тимонина Е.\,Е.}
\index{Grusho A.\,A.}
\index{Zabezhailo M.\,I.}
\index{Smirnov D.\,V.}
\index{Timonina E.\,E.}

{\renewcommand{\thefootnote}{\fnsymbol{footnote}} \footnotetext[1]
{Работа частично поддержана РФФИ (проект 18-29-03081).}}

\renewcommand{\thefootnote}{\arabic{footnote}}
\footnotetext[1]{Институт проблем информатики Федерального исследовательского центра <<Информатика и~управление>> 
Российской академии наук, \mbox{grusho@yandex.ru}}
\footnotetext[2]{Вычислительный центр Федерального исследовательского центра <<Информатика и~управление>> 
Российской академии наук, \mbox{m.zabezhailo@yandex.ru}}
\footnotetext[3]{ПАО Сбербанк России, dvlsmirnov@sberbank.ru}
\footnotetext[4]{Институт проблем информатики Федерального исследовательского центра <<Информатика 
и~управление>> Российской академии наук, \mbox{eltimon@yandex.ru}}


\vspace*{-8pt}


  
  
  \Abst{Обсуждается задача поиска релевантных заданной цели данных в~постоянно 
пополняемых новой информацией коллекциях Big Data в~условиях жестких ограничений на 
допустимое время (так называемое про\-цес\-сно-ре\-аль\-ное время) анализа данных (АД)
и~поддержки принятия решений (ППР). В~основе развиваемого подхода~--- использование 
современных методов искусственного интеллекта, в~частности представления знаний 
и~формализации рассуждений в~сис\-те\-мах интеллектуального АД (ИАД). 
Рассматривается ряд критически значимых для результативности такого ИАД барьеров, 
в~том числе обусловленных доказуемой трудноразрешимостью возникающих здесь 
комбинаторных задач, особенностями представления знаний и~управ\-ле\-ния перебором 
вариантов, а~также некоторыми аспектами управления полнотой и~точностью порождаемых 
результатов. Представлена схема формализации развиваемой процедурной конструкции 
ИАД. Обсуждаемый подход сопровождается иллюстрациями его реализации в~рамках 
системы идентификации признаков вредоносной инсайдерской активности в~крупном 
отечественном коммерческом банке.}
   
  \KW{Big Data; процессно-реальное время; интеллектуальный анализ данных; 
информационная безопасность; поиск инсайдеров}

\DOI{10.14357/19922264210206}

\vspace*{-2pt}


\vskip 10pt plus 9pt minus 6pt

\thispagestyle{headings}

\begin{multicols}{2}

\label{st\stat}
   

  \section{Введение}
   
  Потребность в~разработке проблемно-ори\-ен\-ти\-ро\-ван\-ных средств 
<<навигации>> в~Big Data очевидным образом ассоциируется с~по\-треб\-ностью 
в~разработке эффективных поисковых технологий,\linebreak которые позволяли бы 
результативно обрабатывать большие объемы не только собственно исходных 
<<сырых>> данных, но и~формируемой на их основе\linebreak аналитики~--- витрин 
данных, графической информации, dashboard'ов и~др. Критичными здесь 
оказываются потребности в~обработке больших объемов постоянно 
изменяющейся, пополняемой\linebreak новыми сведениями информации в~условиях 
жестких временн$\acute{\mbox{ы}}$х ограничений (так называемого  
про\-цес\-сно-ре\-аль\-но\-го времени) АД и~ППР.
  
  При решении поисковых задач такого типа сегодня широко используется ряд 
коммерческих корпоративных поисковых систем, в~частности Algolia~[1], IBM 
Watson Discovery~[2], Yext~[3], Swiftype~[4], SearchUnify~[5]. Популярны 
корпоративные поисковые системы с~открытым исходным кодом, например 
ElasticSearch~[6], Solr~[7], Sphinx~[8]. Среди отечественных коммерческих 
корпоративных поисковых систем, по-видимому, наиболее известны Спутник 
(Ростелеком)~[9] и~1C~[10].
  
  Говоря о критически значимых характеристиках корпоративных поисковых 
систем (так называемых \textit{поисковиков}~--- см.~[1--10]), следует в~первую 
очередь обратить внимание:
  \begin{itemize}
\item на скорость индексирования первичных данных, т.\,е.\ быстроту 
переработки поисковиком входных <<сырых>> данных для занесения в~свой 
внутренний поисковый аппарат~--- системы поисковых индексов, 
классификаторы и~т.\,п. Обычно этот параметр оценивается в~мегабайтах 
чистого <<сырого>> входного текста в~секунду;
\item на скорость переиндексации~--- реконструкции поискового 
инструментария, т.\,е.\ обновления индексов или создания новых с~приходом 
новой входной информации. При этом может поддерживаться как 
инкрементальное индексирование, так и~полная перестройка 
(переформирование) индекса, могут использоваться и~дополнительные 
индексы, в~том числе так называемый дель\-та-ин\-декс, в~который 
включается только новая информация;
\item на поддерживаемые API (application programming interface). Поисковое ядро необходимо связывать 
с~приложениями, которые могут иметь библиотеки, работающие с~API 
поисковика;
\item на взаимосвязь размеров базы и~скорости поиска, так как некоторые 
поисковики попросту перестают отвечать на запросы при индексах, 
содержащих более 50~млн записей;
\item на поддерживаемые типы входных документов, т.\,е.\ возможность 
индексации различных типов источников~--- сис\-тем управ\-ле\-ния базами данных, файловых хранилищ 
и~т. п. 
\end{itemize}

  К~сожалению, на текущий момент рынок не\linebreak предлагает надежных 
коммерческих систем поддержки поиска и~идентификации признаков 
вредоносной инсайдерской активности, способных\linebreak обеспечить результативное 
применение в~крупных отечественных финансовых структурах. Именно\linebreak по 
этим причинам це\-ле\-на\-прав\-лен\-но\-го обсуж-\linebreak дения заслуживает разработка 
методики вы\-яв\-ле\-ния признаков инсайдерской актив\-ности и~со\-зда\-ние 
компьютерной сис\-те\-мы поддержки профильной\linebreak дея\-тель\-ности оперативных 
работников служб безопас\-ности.
  
  \section{Профиль угроз}
  
  Стандартный подход к решению задачи идентификации признаков 
вредоносной инсайдерской активности в~текущих данных мониторинга 
функционирования объекта защиты базируется на выделении в~наблюдаемых 
данных отслеживаемого поведения пользователей защищаемой системы таких 
фактических действий, которые могут быть классифицированы как 
потенциально опасное, способное привести к вредоносным последствиям 
поведение. 
  
  Разработка методики и~инструментария идентификации признаков 
инсайдерской активности основана на формировании актуальной модели угроз. 
Модель угроз формализуется в~виде профиля угроз (ПУ), представляющего 
собой постоянно поддерживаемый в~актуальном состоянии перечень так 
называемых типовых сценариев (ТС). Исходные ТС рождаются из информации, 
взятой из опыта\linebreak оперативных сотрудников, вовлеченных в~расследования 
конкретных случаев мошенничества (признаки инсайдерской активности). 
Опыт оперативных сотрудников сначала фиксируется \mbox{в~виде} текстового 
описания, которое далее преобразуется в~машиночитаемый формализованный 
вид. При этом может быть задействовано промежуточное представление знаний 
о~каждом из ТС в~виде фрейма. Для описания данных в~слотах подобных 
фреймов предусмотрены иерархии типов данных: от булевых значений 
признаков:  <<да>>/<<нет>>~--- до графов параметров и~отношений между 
такими параметрами с~пометками на вершинах и~ребрах, а~также текстовых 
комментариев, например в~виде Binary Large Objects (BLOB). Кроме 
обобщения опыта в~формировании ТС используется ИАД и~машинное обучение (МО).
  
  Простейший вариант представления знаний в~ТС ПУ~--- использование 
булевых значений <<да>>/<<нет>>, позволяющих описать каждый такой фрейм 
в~виде множества характеризующих именно его признаков. В~свою очередь, 
множество всех используемых при описании текущего ПУ признаков 
определяет битовый вектор, соответствующими\linebreak единицами которого 
кодируется каждый из соответствующих ТС в~ПУ. Обработка ма\-ши\-но\-чи\-та\-емо\-го 
описания фреймов, представленных в~виде битовых векторов, дает 
возможность получить существенный выигрыш в~производительности при 
анализе текущих данных, так как позволяет организовать сравнение текущей 
ситуации с~описаниями ТС средствами одной вычислительной макрооперации.
  
  Множество {ТС} представляет систему фильтров, позволяющих сократить 
объем данных для дальнейшего анализа и,~по возможности, не пропустить 
информацию о признаках действий инсайдеров. 
  
  \section{Организация быстрой фильтрации на~основе 
множества типовых сценариев} 
  
  Критически важную роль в~обеспечении результативности обсуждаемого 
подхода к выделению из исходных <<сырых>> Big Data информации, 
содержащей признаки вредоносной инсайдерской активности, играют две 
группы характеристик:
  \begin{enumerate}[(1)]
\item \textit{чувствительность} критериев релевантности ПУ, т.\,е.\ 
возможности ошибок первого и~второго рода и, как следствие, возможности 
неполноты идентификации признаков инсайдерской активности и~ложных 
срабатываний;
\item \textit{размерность} текущего описания ПУ, определяющая параметры 
перебора всех совпадений его фрагментов с~элементами описания текущего 
множества отслеживаемых данных.
\end{enumerate}

  Если перебирать совпадения описаний каждого ТС с~текущим множеством 
отслеживаемых данных по каждому пользователю, то задача приобретет 
экспоненциальную сложность. Однако даже в~этой ситуации перебор можно 
сократить за счет отказа от повторных проверок общих для нескольких ТС 
фрагментов:
  \begin{itemize}
\item выделения всех общих для всех ТС частей; 
\item упорядочения их по взаимной вложимости, т.\,е.\ формирования 
частично упорядоченного множества таких фрагментов в~виде диаграммы 
сходств;
\item организации выделения общих с~множеством отслеживаемых данных 
фрагментов, начиная с~нижнего <<этажа>> этой диаграммы с~наиболее 
часто встречаемых в~ТС общих фрагментов, затем двигаясь <<вверх>> 
к~наименее часто встречающимся общим фрагментам и~далее~--- 
к~собственно полным описаниям ак\-ту\-аль\-ных~ТС.
\end{itemize}

   В общем случае приходится иметь дело с~экспоненциально быстро 
растущим (с~линейным ростом размеров описания текущих ТС) числом 
элементов в~такой диаграмме. Однако можно показать, что верхняя и~нижняя 
границы такой диаграммы могут быть построены полиномиально быстро, 
а~анализ большинства реальных множеств от\-сле\-жи\-ва\-емых данных требует 
проверки лишь нескольких из всех цепей частичного порядка в~таких 
диаграммах. Каждая из таких цепей, ведущая от одного из элементов нижней 
границы диаграммы к одному из ее верхних элементов, имеет длину, 
ограниченную полиномом от размеров текущего описания имеющихся ТС. Это 
легко объясняется содержательными соображениями, т.\,е.\ характеристиками 
биз\-нес-ак\-тив\-ности пользователей в~реальных ситуациях. Дополнительно 
<<навигация>> <<снизу вверх>>\linebreak по цепям частичного порядка открывает 
возможности проактивной ориентации офицеров безопас\-ности 
в~<<подсвеченных>> ситуациях, де\-мон\-ст\-ри\-ру\-ющих релевантность текущей 
отслеживаемой ситуа\-ции фрагментам описания ка\-ко\-го-ли\-бо из известных 
ТС. Таким образом, этот метод поз\-во\-ля\-ет целенаправленно фокусировать 
внимание и~ресурсы на потенциально опасных направлениях развития каждой 
конкретной отслеживаемой ситуации в~динамике ее изменений.
   
  \section{Формирование типовых сценариев с~помощью интеллектуального анализа данных и~машинного обучения}
    
    Первоначально ТС строятся на основе опыта оперативных работников, т.\,е.\
     ТС формируются как прецеденты, содержащие признаки инсайдеров (они 
помечаются меткой~<<+>>), и~как прецеденты, которые заведомо не связаны 
с~деятельностью инсайдеров (помечаются меткой <<$-$>>). Остальные 
прецеденты помечаются меткой~<<0>> как неопределенные. Описания 
прецедентов формализуются следующим образом.
    
  Исходные (<<сырые>>) данные описываются множеством наблюдаемых 
значений параметров $x_1, x_2,\ldots , x_n$. В~условиях открытости это 
множество может меняться. Каждый параметр~$x_i$, $i\hm=1,\ldots , n$, имеет 
домен своих значений (характеристик): 

\vspace*{-7pt}

\noindent
  \begin{align*}
  A_1&=\left\{ a_{11}, a_{12}, \ldots , a_{1m_1}\right\}\,;\\
  A_2&=\left\{ a_{21}, a_{22}, \ldots , a_{2m_2}\right\}\,;\\
  &\cdots\\
  A_n&=\left\{ a_{n1}, a_{n2}, \ldots , a_{nm_n}\right\}\,.\\
  \end{align*}
  
  \vspace*{-12pt}
  
  Одним из элементов каждого из этих множеств является <<пустой>> символ, 
который означает, что соответствующий параметр не участвует в~изу\-ча\-емом 
объекте. 
  
  \smallskip
  
  \noindent
  \textbf{Определение~1.}\ Объектом~$o$ называется произвольный элемент 
множества $A_1\times A_2\times \cdots \times A_n$. 
  
  \smallskip
  
  Формализация опыта оперативных работников выражается в~формировании 
классов:

\vspace*{-2pt}

\noindent
  \begin{align*}
  O^+ &=\left\{ o_1^+, o_2^+, \ldots , o_s^+\right\}\subseteq A_1\times A_2\times 
\cdots \times A_n={}\\
&\hspace*{55mm}{}=\prod\limits^n_{i=1} A_i\,;\\[-3pt]
  O^- &= \left\{ o_1^-, o_2^-, \ldots , o_r^-\right\} \subseteq \prod\limits^n_{i=1} 
A_i\,;\\
  O^0 &= \left\{ o_1^0, o_2^0, \ldots , o_v^0\right\} \subseteq \prod\limits^n_{i=1} 
A_i\,.
  \end{align*}
  
  \vspace*{-4pt}
  
  Определим множество ТС на обученных прецедентах как множество 
векторов $O^+\cup O^-$.
  
  Если описывать ТС с~помощью длинных двоичных векторов, то можно 
использовать возможности вычислительной техники для быстрого 
и~качественного отбора прецедентов в~соответствии\linebreak с~построенной 
классификацией. Множество $O^+\cup O^-$ объявляется релевантным на 
данном этапе и~обозначается~$R$. 
  
  Определим функцию $f_R$: $A_1\times A_2\times\cdots\times A_n\hm\to 
\{0,1\}$, где 

\noindent
  $$
  f_R=\begin{cases}
  1\,, &\ o\in R\,;\\
  0\,, &\ o\notin R\,.
  \end{cases}
  $$
  
  Обозначим через $A\hm\subseteq \prod\nolimits^n_{i=1} A_i$ текущее 
множество <<сырых>> данных. 
  
  
 
  
  Задача расширения множества ТС состоит в~доопределении функции~$f_R$ 
(где это возможно) на данные $A\backslash (O^+\cup O^-)$. Доопределение 
значений функции~$f_R$ на исходно неопределенных элементах множества 
$A\backslash (O^+\cup O^-)$ может быть выполнено по традиционной для МО  
(\textit{ин\-тер\-по\-ля\-ци\-он\-но-экстра\-по\-ля\-ци\-он\-ной}) схеме.
  
  \textbf{I~этап.} На элементах множества $O^+\cup O^-$ 
строятся \textit{зависимости} того или иного класса таким образом, что все 
кортежи из множества~$O^+$ <<лежат>> на ка\-ких-ли\-бо из таких 
зависимостей, а все кортежи из множества~$O^-$ не <<лежат>> ни на одной из 
них. При этом на каждом элементе множества~$O^+$ хотя бы одна из таких 
зависимостей выполняется, а~на каж\-дом из элементов множества~$O^-$ не 
выполняется ни одна из таких зависимостей.
  
  \textbf{II~этап.} Построенные на первом этапе зависимости 
экстраполируются, где это возможно, на исходно недоопределенные кортежи из 
множества $A\backslash (O^+\cup O^-)$. Все такие случаи классифицируются 
как  $f_R\hm=1$, а~во всех оставшихся случаях  $f_R\hm=0$.
  
  Процесс экстраполяции отношения~$R$ с~име\-ющих\-ся примеров 
и~контрпримеров на <<сырые>> данные~--- процедура восстановления 
значений функции~$f_R$ на оставшихся кортежах из $A\backslash (O^+\hm\cup 
O^-)$. Она основана на анализе сходства описаний прецедентов, уточняемого 
как бинарная алгебраическая операция~\cite{11-gr}. При этом 
  \begin{itemize}
\item[(а)]
задействовано описание прецедентов в~виде кортежа значений 
соответствующих па\-ра\-мет\-ров $x_1, x_2,\ldots , x_n$; 
\item[(б)]
на $x_1, x_2,\ldots , x_n$ определяется (покомпонентными сравнениями 
значений параметров) бинарная операция сходства~$\otimes$, 
удовле\-тво\-ря\-ющая стандартным условиям~\cite{11-gr}: $\forall\,o_1,o_2,o_3$ 
выполнено
\begin{enumerate}[(1)]
\item $o\otimes o\hm= o$; 
\item $o_1\otimes o_2\hm= o_2\otimes o_1$; 
\item $o_1\otimes o_2\otimes o_3\hm= (o_1\otimes o_2)\otimes o_3\hm= o_1\otimes 
(o_2\otimes o_3)$.
\end{enumerate}
 При этом значения параметра~$x_i$ у~срав\-ни\-ва\-емых 
объектов~$o_1$ и~$o_2$ считаются сходными, если $x_i\hm\in A_i^* \hm 
\subseteq A_i$, где~$A_i^*$~--- заранее заданная окрестность значений 
параметра~$x_i$. Для признаков, принимающих лишь булевы значения~0 
или~1, это будет операция  пересечения множеств~$\cap$;
\item[(в)]
отношение сходства прецедентов определяется по непустому результату 
вычисления операции сходства представляющих эти прецеденты кортежей 
значений признаков. Два прецедента сходны, если результат применения 
операции~$\otimes$ к их описаниям не является пустым объектом;
\item[(г)]
для каждого прецедента~$o$ из множества~$A$ класс сходства T$(o)$ всех 
сходных с~ним прецедентов из~$A$ формируется объединением в~T$(o)$ 
всех элементов множества~$A$, сходных с~$o$;
\item[(д)]
сформированные классы сходства после дополнительного анализа 
эмпирических закономерностей распадаются на две части, построенные на 
позитивных примерах и~построенные на контрпримерах; 
\item[(е)]
для каждого из всех имеющихся позитивных, т.\,е.\ построенных на 
примерах, классов сходства T$(o)$ выделяются подклассы T$_V(o)$, каждый 
из которых порождается одним из со\-от\-вет\-ст\-ву\-ющих (участвующих 
в~порождении этого T$(o)$) сходств~$V$ примеров из T$(o)$, т.\,е.\ $V$~--- 
множество значений вектора о, принадлежащих всем векторам из множества 
T$_V(o)$. Из множества T$_V(o)$ удаляются все такие элементы, в~которые 
вкладывается хотя бы один из контрпримеров из множества~$O^-$. Это 
условие проверяется по всем~$V$, участвующим в~порождении Т($o$);
\item[(ж)]
экстраполяция час\-тич\-но-опре\-де\-лен\-но\-го отношения~$R$ на все 
элементы множества $A\backslash (O^+\hm \cup O^-)$ реализуется с~помощью 
проверки вло\-жимости каждого элемента $A\backslash (O^+\hm\cup O^-)$ 
в~какой-либо подкласс T$_V(o)$ позитивных классов сходства T$(o)$. 
В~случае такого попадания $f_R\hm=1$ на данном прецеденте, в~противном 
случае считается, что $f_R\hm=0$. Таким образом, расширяется множество 
ТС с~признаками вредоносной активности инсайдера. Аналогично из 
множества элементов, для которых $f_R\hm=0$, можно выделять ТС 
с~отрицательными признаками наличия вредоносной инсайдерской 
активности.
\end{itemize}
  
  \section{Примеры профилей угроз и~диаграмма сходств типовых~сценариев}
   
  Рассмотрим диаграммы D$_{\mathrm{ТС}}$ сходств ТС на примере 
представления знаний в~описывающих ТС фреймах в~виде булевых векторов 
признаков. Множество~$V$ в~построенных положительных классах T$_V(o)$ 
назовем актуальными признаками ТС, принадлежащим T$_V(o)$.
  
  Тогда каждый ТС может быть представлен как кортеж из~0 и~1, в~котором 
единицы соответствуют актуальным для данного ТС признакам.
  
  \smallskip
  
  \noindent
  \textbf{Пример.} Пусть ТС$_{\mathrm{АТ}}$~--- сценарий, текстовое описание 
которого фиксирует опасность одновременного доступа пользователя 
в~аналитические (А) и~транзакционные (Т) приложения. Тогда актуальные 
признаки сценария ТС$_{\mathrm{АТ}}$:
  \begin{itemize}
  \item[\,] $P_1$~--- доступ в~аналитический блок;\\[-9pt] 
  \item[\,] $P_2$~--- доступ в~транзакционный блок;\\[-9pt]
  \item[\,] $P_3$~-- идентификация имеющего эти доступы пользователя 
с~учетом штатного профиля его доступов. 
  \end{itemize}
  
  Таким образом, ТС$_{\mathrm{АТ}}$ сопоставлено множество признаков $\{ 
P_1, P_2, P_3\}$. 
  
  В построенном языке представления знаний актуальный ПУ представляет 
собой набор множеств актуальных для соответствующего ТС признаков, 
которые можно представить в~виде набора векторов вида:
  \begin{align*}
  \alpha_{\mathrm{TC}_1} &=\langle \alpha_{\mathrm{TC}_1,1},  
\alpha_{\mathrm{TC}_1,2}, \ldots \alpha_{\mathrm{TC}_1,n}\rangle\,;\\
  \alpha_{\mathrm{TC}_2} &=\langle \alpha_{\mathrm{TC}_2,1},  
\alpha_{\mathrm{TC}_2,2}, \ldots \alpha_{\mathrm{TC}_2,n}\rangle\,;\\
  &\ldots\\
  \alpha_{\mathrm{TC}_m} &=\langle \alpha_{\mathrm{TC}_m,1},  
\alpha_{\mathrm{TC}_m,2}, \ldots \alpha_{\mathrm{TC}_m,n}\rangle\,,
  \end{align*}
где $\mathrm{ПУ}=\{\mathrm{TC}_1, \mathrm{TC}_2, \ldots , \mathrm{TC}_m\}$.
  
  Вычисление сходств между векторами $\alpha_{\mathrm{TC}_i}$ 
реализуется с~помощью бинарной операции~$\otimes$ покомпонентного 
сравнения булевых векторов:
  $$
  \alpha_{\mathrm{TC}_i,1}\otimes \alpha_{\mathrm{TC}_j,1} =1\Leftrightarrow 
\alpha_{\mathrm{TC}_i,1}= \alpha_{\mathrm{TC}_j,1}=1\,.
  $$
  
  В остальных случаях результат сравнения равен~0.
  
  Таким образом может быть сформирована диаграмма $D_{\mathrm{ТС}}$ 
всех непустых сходств описаний ТС, имеющихся на текущий момент. При 
этом:
  \begin{itemize}
\item нижний <<этаж>> этой диаграммы формируется сходствами 
(множествами общих признаков) максимальных по числу элементов 
возможных подмножеств ТС из всего текущего ПУ; 
\item верхний <<этаж>> этой диаграммы формируется минимальными по 
числу образующих их ТС и~максимальными по числу общих признаков 
сходствами ТС.
\end{itemize}

  Проверка <<сырых>> данных на возможное отношение к известным 
признакам инсайдерской активности, представленным описаниями ТС, 
начинается с~нижнего <<этажа>> диаграммы $D_{\mathrm{ТС}}$ и~далее 
осуществляется не сравнением <<всех>> со <<всеми>>, а лишь просмотром 
релевантных цепей частичного порядка (по взаимной вложимости сходств) 
<<снизу вверх>> (к~<<верхнему>> этажу диаграммы $D_{\mathrm{ТС}}$, 
а~далее~--- к полным описаниям соответствующих ТС). 
  
  \section{Оценки сложности вычислений при формировании 
$D_{\mathrm{ТС}}$}
   
  Пусть заданы два множества: $A\hm=\{ a_1, a_2, \ldots, a_n\}$~--- множество 
признаков, используемых при формировании ТС из ПУ, и~текущее множество 
$\mathrm{ПУ}\hm = \{ \mathrm{TC}_1, \mathrm{TC}_2,\ldots , 
\mathrm{TC}_m\}\hm\subseteq 2^A\backslash \phi$ описаний ТС в~виде 
подмножеств признаков из~$A$. В~\cite{12-gr, 13-gr} показано, что справедливы 
следующие утверждения.
  
  \smallskip
  
  \noindent
  \textbf{Утверждение~1.} Задача о вычислении числа элементов диаграммы 
D$_{\mathrm{ТС}}$ принадлежит классу \#\,PC~--- так называемых 
перечислительно полных комбинаторных проблем~[14--16].
  
  Таким образом, в~общем случае размер диаграммы $D_{\mathrm{ТС}}$ 
растет \textit{экспоненциально} быстро с~линейным ростом размеров 
множеств~$A$ и~ПУ.
  
  \smallskip
  
  \noindent
  \textbf{Утверждение~2.} Верхняя и~нижняя границы диаграммы 
$D_{\mathrm{ТС}}$ (множества соответственно максимальных 
и~минимальных по взаимному вложению ее элементов) могут быть 
сформированы алгоритмом полиномиальной от размеров множеств~$A$ и~ПУ 
вычислительной сложности.
  
  Таким образом, порождая для всех текущих D$_{\mathrm{ТС}}$ эти границы 
и~далее в~каждом конкретном случае направленным образом достраивая 
релевантные цепочки частичного порядка в~D$_{\mathrm{ТС}}$, можно 
управ\-лять перебором вариантов при поиске признаков вредоносной 
инсайдерской активности. При этом за счет оптимизации перебора и,~если 
необходимо, за счет подключения дополнительных вы\-чис\-ли\-тель\-ных ресурсов 
обеспечивается достижение ограничений про\-цес\-сно-ре\-аль\-но\-го времени. 
  
  Интерактивные сервисы, обеспечивающие сотрудникам службы 
безопасности оперативный доступ к деталям описаний диаграммы 
D$_{\mathrm{ТС}}$, вместе\linebreak с~возможностями проактивного отслеживания 
потенци\-аль\-но опас\-ных на\-прав\-ле\-ний развития те\-кущей ситуации оказываются 
дополнительным фактором повышения пол\-но\-ты и~точ\-ности идентификации 
признаков вредоносной инсайдерской активности. При этом оперативный 
доступ к~результатам аналитической обработки ранее выявленных признаков 
дает дополнительные инструменты управ\-ле\-ния качеством идентификации 
и~противодействия потенциально опасной активности пользователей.

\vspace*{-9pt}
  
  \section{Заключение}
  
  \vspace*{-3pt}
  
  Исследована задача поиска релевантных для дальнейшего анализа данных 
в~Big Data, постоянно пополняемых новой информацией. В~условиях жестких 
ограничений на время АД и~ППР, т.\,е.\  
про\-цес\-сно-ре\-аль\-но\-го времени, необходимо использовать средства ИАД. 
При этом существенные объемы анализируемых данных (эффект Big) и~их 
постоянное пополнение новыми элементами (эф-\linebreak\vspace*{-12pt}

\pagebreak

\noindent
фект Open)~--- не 
единственные критически значимые факторы такого анализа. 
  
  В исследованных ситуациях достаточно часто встречались доказуемо 
трудноразрешимые комбинаторные проблемы, попадающие в~те или иные 
известные классы вычислительной сложности. Для их решения использовались 
специальные проб\-лем\-но-ори\-ен\-ти\-ро\-ван\-ные частные решения. 
В~связи с~этим возникает необходимость предварительного выделения данных, 
релевантных целям поиска.
  
  Как следствие, возникает дополнительная проб\-ле\-ма <<балансировки>> 
детальности представления знаний и~управления объемами необходимых для\linebreak 
их обработки вычислений. Кроме того, эффективность исследований зависит от 
разработки на\-дежных и~быстрых программных сервисов для обеспечения 
эффективного интерактивного режима взаимодействия  
экс\-пер\-та-ана\-ли\-ти\-ка и~компьютерной системы ИАД. Соответствующая 
система АД и~ППР должна 
предоставлять гибкие возможности направленной обработки знаний эксперта 
в~том или ином формализованном виде. В~част\-ности, сис\-те\-ма должна 
позволять детализировать пред\-став\-ле\-ние знаний о~тех или иных аспектах 
анализируемой проб\-ле\-мы, <<сужая>> об\-ласть исследования до конкретного 
<<сектора>> и~сохраняя при этом <<согласованность>> с~требованиями 
режима про\-цес\-сно-ре\-аль\-но\-го времени.
  
  Следует отдельно подчеркнуть, что ответственность за результаты АД и~ППР, 
выполненного с~помощью компьютерной системы ИАД, ложится все-та\-ки на 
экс\-пер\-та-ана\-ли\-ти\-ка. Именно по этой\linebreak причине сервисы интерактивного 
взаимодействия эксперта и~системы ИАД, возможности реализации тех или 
иных эвристик в~процессе поиска, сервисы управления перебором вариантов 
и~т.\,п.\  оказываются не только полезными <<инструментами>> повышения 
полноты и~точности поиска результатов, но и~помогают эксперту сохранять 
понимание способа порождения и~неформальную объяснимость получаемых 
заключений и~рекомендаций.
  
  Экспериментальное подтверждение ра\-бо\-то\-способности и~результативности 
предлагаемой методики, а~также ее процедурной реализации при решении 
задачи идентификации признаков вредоносной инсайдерской активности было 
получено в~крупном отечественном коммерческом банке.
  
{\small\frenchspacing
{%\baselineskip=10.8pt
%\addcontentsline{toc}{section}{References}
\begin{thebibliography}{99}
\bibitem{1-gr}
The flexible AI-powered Search \& Discovery platform. {\sf https://www.algolia.com}.
\bibitem{2-gr}
IBM Watson Discovery. {\sf https://www.ibm.com/cloud/\linebreak watson-discovery}.
\bibitem{3-gr}
Power your website with the world's best search. {\sf https:// www.yext.com}.
\bibitem{4-gr}
A~powerful search experience for your website~--- without the learning curve. 
{\sf https://swiftype.com}.
\bibitem{5-gr} 
SearchUnify wins two silver Stevies~--- one in collaboration with Bluebeam~--- in 2021 
Stevie Awards for Sales \& Customer Service. {\sf https://www.searchunify.com}.
\bibitem{6-gr}
ELASTIC: Search more, spend less. {\sf https://www.elastic.\linebreak co}.
\bibitem{7-gr}
Solr is the popular, blazing-fast, open source enterprise search platform built on Apache 
Lucene$^{\mathrm{TM}}$. {\sf https:// lucene.apache.org/solr}.
\bibitem{8-gr} 
Introduction to search with SPHINX. {\sf http://\linebreak sphinxsearch.com}.
\bibitem{9-gr}
Корпоративный поиск <<Спутник>>. {\sf https://www. sputnik.ru/searchbox}. 
\bibitem{10-gr}
Архитектура платформы 1С-Предприятие: глобальный поиск. 
{\sf https://v8.1c.ru/platforma/globalnyy-poisk}. 
\bibitem{11-gr}
\Au{Кон П.\,М.} Универсальная алгебра~/ Пер. с~англ.~--- М.: Мир, 1968. 359~с.
(\Au{Cohn~P.\,M.} {Universal algebra}.~--- New York, NY, USA: Harper and Row, 1965. 333~p.)

\bibitem{13-gr} %12
\Au{Забежайло М.\,И.} О~некоторых оценках сложности вычислений в~ДСМ-рас\-суж\-де\-ни\-ях~// 
Искусственный интеллект и~принятие решений, 2015. Часть~I: №\,1. С.~3--17; Часть~II: №\,2. 
С.~3--17.

\bibitem{12-gr} %13
\Au{Грушо А.\,А., Забежайло~М.\,И., Зацаринный~А.\,А., Тимонина~Е.\,Е.} О~некоторых 
возможностях управления ресурсами при организации проактивного противодействия 
компьютерным атакам~// Информатика и~её применения, 2018. Т.~12. Вып.~1. С.~62--70.

\bibitem{14-gr}
\Au{Simon J.} On the difference between one and many~// Automata, languages and 
programming~/ Eds. A.~Salomaa, M.~Steinby.~---
 Lecture notes in computer science ser.~--- Springer, 1977. Vol.~52. P.~480--491. 
\bibitem{15-gr}
\Au{Valiant L.\,G.} The complexity of enumeration and reliability problems~// SIAM 
J.~Comput., 1979. Vol.~8. Iss.~1. P.~410--421.
\bibitem{16-gr}
\Au{Valiant L.\,G.} The complexity of computing the permanent~// Theor. Comput. Sci., 
1979. Vol.~8. P.~189--201.
  \end{thebibliography}

}
}

\end{multicols}

\vspace*{-3pt}

\hfill{\small\textit{Поступила в~редакцию 04.04.2021}}

%\vspace*{8pt}

%\pagebreak

\newpage

\vspace*{-28pt}

%\hrule

%\vspace*{2pt}

%\hrule

%\vspace*{-2pt}

\def\tit{INTELLIGENT ANALYSIS OF~BIG DATA EXTENDIBLE COLLECTIONS UNDER~THE~LIMITS OF~PROCESS-REAL TIME}

\def\titkol{Intelligent analysis of~Big Data extendible collections under the 
limits of~process-real time}

\def\aut{A.\,A.~Grusho$^1$, M.\,I.~Zabezhailo$^2$, D.\,V.~Smirnov$^3$, and~E.\,E.~Timonina$^1$}

\def\autkol{A.\,A.~Grusho, M.\,I.~Zabezhailo, D.\,V.~Smirnov, and~E.\,E.~Timonina}

\titel{\tit}{\aut}{\autkol}{\titkol}

\vspace*{-9pt}


\noindent
$^1$Institute of Informatics Problems, Federal Research Center ``Computer Science and Control'' 
of the Russian\linebreak
$\hphantom{^1}$Academy of Sciences, 44-2~Vavilov Str., Moscow 119133, Russian Federation

\noindent
$^2$A.\,A.~Dorodnicyn Computing Center, Federal Research Center ``Computer Science and 
Control'' of the Russian\linebreak
$\hphantom{^1}$Academy of Sciences, 40~Vavilov Str., Moscow 119333, Russian Federation

\noindent
$^3$Sberbank of Russia, 19~Vavilov Str., Moscow 117999, Russian Federation

 
\def\leftfootline{\small{\textbf{\thepage}
\hfill INFORMATIKA I EE PRIMENENIYA~--- INFORMATICS AND
APPLICATIONS\ \ \ 2021\ \ \ volume~15\ \ \ issue\ 2}
}%
\def\rightfootline{\small{INFORMATIKA I EE PRIMENENIYA~---
INFORMATICS AND APPLICATIONS\ \ \ 2021\ \ \ volume~15\ \ \ issue\ 2
\hfill \textbf{\thepage}}}

\vspace*{6pt} 



\Abste{The problem how to extract relevant to the fixed goal data from regularly extended by new 
information collections of Big Data not braking given limits for data analysis and decision making 
(being in agreement with so-called process-real time restrictions) is discussed. The proposed 
approach is based on implementation of modern artificial intelligence 
techniques including knowledge representation 
and reasoning formalization for so-called Intelligent Data Analysis (IDA) computer systems. Some 
critical barriers preventing efficient application of this type IDA (e.\,g., computational complexity of 
some related to IDA combinatorial problems, including provable getting some of them in 
well-known classes of computationally hard problems, some characteristic features of knowledge 
representation and search iteration enumeration control, optimization of accuracy, and completeness 
of search results) are analyzed. A~formalized description for the designed IDA set of procedures is 
presented. The discussed approach is illustrated by examples of its implementation in a corporate 
computer system of malicious insider activities identification and counteraction operating in a large 
Russian commercial bank.}

\KWE{Big Data; process-real time; intelligent data analysis; information security; insider 
malicious activities}



\DOI{10.14357/19922264210206}

%\vspace*{-15pt}

\Ack
\noindent
The paper was partially supported by the Russian Foundation for Basic Research (project  
18-29-03081).

\vspace*{12pt}

  \begin{multicols}{2}

\renewcommand{\bibname}{\protect\rmfamily References}
%\renewcommand{\bibname}{\large\protect\rm References}

{\small\frenchspacing
 {%\baselineskip=10.8pt
 \addcontentsline{toc}{section}{References}
 \begin{thebibliography}{99}
\bibitem{1-gr-1}
The flexible AI-powered Search \& Discovery platform. Available at: 
{\sf https://www.algolia.com} (accessed May~12, 2021).
\bibitem{2-gr-1}
IBM Watson Discovery. Available at: {\sf https://www.ibm.\linebreak com/cloud/watson-discovery}
(accessed May~12, 2021).
\bibitem{3-gr-1}
Power your website with the world's best search. Available at: {\sf https://www.yext.com} 
(accessed May~12, 2021).
\bibitem{4-gr-1}
A~powerful search experience for your website~--- without the learning curve. Available at: 
{\sf https://swiftype.com} (accessed May~12, 2021).
\bibitem{5-gr-1}
SearchUnify wins two silver Stevies~--- one in collaboration with Bluebeam~--- in 2021 Stevie 
Awards for Sales \& Customer Service. Available at: {\sf https://www.\linebreak searchunify.com} (accessed May~12, 2021).
\bibitem{6-gr-1}
ELASTIC: Search more, spend less. Available at: {\sf https:// www.elastic.co} (accessed May~12, 
2021).
\bibitem{7-gr-1}
Solr is the popular, blazing-fast, open source enterprise search platform built on Apache 
Lucene$^{\mathrm{TM}}$. Available at: {\sf https://lucene.apache.org/solr/} (accessed May~12, 2021).
\bibitem{8-gr-1}
Introduction to Search with SPHINX. Available at: {\sf http://sphinxsearch.com} (accessed May~12, 2021). 
\bibitem{9-gr-1}
Korporativnyy poisk ``Sputnik'' [Corporate Search ``Sputnik'']. Available at: 
{\sf https://www.sputnik.ru/searchbox} (accessed May~12, 2021).
\bibitem{10-gr-1}
Arkhitektura platformy 1S-Predpriyatie: global'nyy poisk [1C-Enterprise Platform 
Architecture: Global search]. Available at: {\sf https://v8.1c.ru/platforma/globalnyy-poisk/} (accessed 
May~12, 2021). 
\bibitem{11-gr-1}
\Aue{Cohn, P.\,M.} 1965. \textit{Universal algebra}. New York, NY: \mbox{Harper} and Row. 333~p.

\bibitem{13-gr-1}
\Aue{Zabezhailo, M.\,I.} 2015. O~nekotorykh otsenkakh slozhnosti vichisleniy 
v~DSM-rassuzhdeniyakh [To the\linebreak computational complexity of hypotheses generation in JSM-method]. 
\textit{Iskusstvennyy intellect i~prinyatie resheniy} [\mbox{Artificial} Intelligence and Decision Making]. Part~I. 
1:\mbox{3--17}; Part~II. 2:3--17.
\bibitem{12-gr-1}
\Aue{Grusho, A.\,A., M.\,I.~Zabezhailo, A.\,A.~Zatsarinny, and E.\,E.~Timonina.} 2018. O~nekotorykh 
voz\-mozh\-no\-styakh\linebreak uprav\-le\-niya resursami pri organizatsii proaktivnogo\linebreak pro\-ti\-vo\-deystviya 
komp'yuternym atakam [On some possibilities of resource management for organizing active 
counteraction to computer attacks]. \textit{Informatika i~ee Pri\-me\-ne\-niya~--- Inform. Appl.} 12(1):62--70.

\pagebreak


\bibitem{14-gr-1}
\Aue{Simon, J.} 1977. On the difference between one and many. \textit{Automata, 
languages and programming}. Eds. A.~Salomaa and M.~Steinby. Lecture notes in computer science 
ser. Berlin--Heidelberg: Springer. 52:480--491.
\bibitem{15-gr-1}
\Aue{Valiant, L.\,G.} 1979. The complexity of enumeration and reliability problems. \textit{SIAM 
J.~Comput.} 8:410--421.

\vspace*{-2pt}

\bibitem{16-gr-1}
\Aue{Valiant, L.\,G.} 1979. The complexity of computing the permanent. \textit{Theor. Comput. Sci.} 
8:189--201.

 \end{thebibliography}

 }
 }

\end{multicols}

\vspace*{-3pt}

  \hfill{\small\textit{Received April~4, 2021}}


%\pagebreak

%\vspace*{-8pt}     
   

\Contr

\noindent
\textbf{Grusho Alexander A.} (b.\ 1946)~--- Doctor of Science in physics and mathematics, 
professor, principal scientist, Institute of Informatics Problems, Federal Research Center 
``Computer Science and Control'' of the Russian Academy of Sciences, 44-2~Vavilov Str., 
Moscow 119133, Russian Federation; \mbox{grusho@yandex.ru}

\vspace*{3pt}

\noindent
\textbf{Zabezhailo Michael I.} (b. 1956)~--- Doctor of Science in physics and mathematics, principal 
scientist, A.\,A.~Dorodnicyn Computing Center, Federal Research Center 
``Computer Science and 
Control'' of the Russian Academy of Sciences, 40~Vavilov Str., Moscow 119333, Russian Federation; 
m.zabezhailo@yandex.ru

\vspace*{3pt}

\noindent
\textbf{Smirnov Dmitry V.} (b.\ 1984)~--- business partner for IT security department, Sberbank of 
Russia, 19~Vavilov Str., Moscow 117999, Russian Federation; \mbox{dvlsmirnov@sberbank.ru}

\vspace*{3pt}

\noindent
\textbf{Timonina Elena E.} (b.\ 1952)~--- Doctor of Science in technology, professor, leading 
scientist, Institute of Informatics Problems, Federal Research Center ``Computer Science and 
Control'' of the Russian Academy of Sciences, 44-2~Vavilov Str., Moscow 119133, Russian 
Federation; \mbox{eltimon@yandex.ru}


\label{end\stat}

\renewcommand{\bibname}{\protect\rm Литература}