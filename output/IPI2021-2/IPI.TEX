\documentclass[10pt]{book}
\usepackage[utf8]{inputenc}

\usepackage{latexsym,amssymb,amsfonts,amsmath,amsxtra,dsfont,
indentfirst,shapepar,%fleqn,%
picinpar,shadow,floatflt,enumerate,multicol,colortbl,moreverb,cite,ipi}

\usepackage{rotating}
\usepackage{mathrsfs}
\usepackage[noend]{algorithmic}
\usepackage{ulem}
\usepackage{graphicx}
%\usepackage{algorithm2e}
\usepackage[linesnumbered,boxed,ruled]{algorithm2e}
%\usepackage{xypic}
\usepackage{oldgerm}
\usepackage{epic}
\usepackage{eepic}

\SetAlgorithmName{Algorithm}{алгоритм}{Список алгоритмов}

%из Дюковой

\newcommand{\algKeyword}[1]{{\bf #1}}
\newcommand{\Proc}[1]{\text{\tt #1}}
\def\CALL{\algKeyword{call}~}

\newenvironment{AlgProcedure}[1]
{
\small
\medskip
%    \hrule
\medskip
\algKeyword{PROCEDURE} #1
\begin{algorithmic}[1]}
{\end{algorithmic}
%    \hrule
\bigskip
}

\def\CALL{\algKeyword{call}~}

%конец для Дюковой

%\RequirePackage[ruled]{algorithm}


\input{epsf}

%\nofiles

%\includeonly{avtor}                          %pdf+авт
%\includeonly{podgot-rus-site,podgot-eng-site}  
%\includeonly{podgot-rus,podgot-eng}  
%\includeonly{ipi-ind} 
%\includeonly{index14}
%\includeonly{toc-rus, toc-en}
%\includeonly{toc-rus}
%\includeonly{toc-en} 


%\includeonly{bosov}                          %1pdf+авт+новая верстка
%\includeonly{borisov}                        %2pdf+авт+
%\includeonly{zeifman}                        %3pdf+авт+
%\includeonly{ushakov}                        %4pdf+авт+ 
%\includeonly{shestakov}                      %5pdf+авт+
%\includeonly{grusho}                         %6pdf+авт+
%\includeonly{zaharov-goncharenko}            %7pdf+авт+
%\includeonly{krivenko}                       %8pdf+авт+
%\includeonly{bazilevskii}                    %9pdf+авт+
%\includeonly{kirikov}                        %10pdf+авт+
%\includeonly{monahov}                        %11pdf+авт+
%\includeonly{sushko}                         %12pdf+авт+
%\includeonly{gonch+zats}                     %13pdf+авт+
%\includeonly{gonch-inkova}                   %14pdf+авт+
%\includeonly{nuriev-egorova}                 %15pdf+авт+
%\includeonly{andrianova}                     %16pdf+авт+  




%\includeonly{rekl}




\usepackage{acad}
%\usepackage{courier}
\usepackage{decor}
\usepackage{newton}
\usepackage{pragmatica}
\usepackage{zapfchan}
\usepackage{petrotex}
\usepackage{bm}                     % полужирные греческие буквы
\usepackage{upgreek}                % прямые греческие буквы \upalpha
\usepackage{eufrak}
\usepackage{verbatim}

\renewcommand{\bottomfraction}{0.99}
\renewcommand{\topfraction}{0.99}
\renewcommand{\textfraction}{0.01}

\setcounter{secnumdepth}{1} %здесь - 3 + chapter = 4

\arraycolsep=1.5pt

%\usepackage[pdftex]{graphicx}

%\usepackage{oz}

%NEW COMMANDS


\renewcommand*{\hm}[1]{#1\nobreak\discretionary{}%
            {\hbox{$\mathsurround=0pt #1$}}{}} %% Дублирует знаки операций
                               %при переносе в формуле (перед знаком, который
                               %надо продублировать ставится команда \hm)

%\newcommand{\endproof}{\hfill$\Box$}
\renewcommand{\r}{\mathbb{R}}
%\newcommand{\I}{{\rm I\hspace{-0.7mm}I}}
%\newcommand{\Ikl}{{\tt{1}}\hspace*{-1.44mm}\mathtt{1}}
\newcommand{\Ik}{\mbox{{\small \tt {1}}\hspace{-1.3mm}{\tt 1}}}
\newcommand{\argmin}{\mathop{\mathrm{arg}\,\mathrm{min}}}
\newcommand{\argmax}{\mathop{\mathrm{arg}\,\mathrm{max}}}
%\newcommand{\capr}{\mathop{\cap\,}}
%\newcommand{\cupr}{\mathop{\cup\,}}
%\def\argmin{\mathop{arg\,min}}

\def\vrp{\varphi}
\def\prt{\partial}
\def\mm{{\sf M}}
\def\modnop#1{\mathop{#1}\limits_{n}}
\def\eam{\mathbin{{\mathop{=}\limits^{\mathrm{def}}}}}
\def\dey#1#2{#1 (#2)}
\def\deyc#1#2{#1 \cdot  #2}
\def\ra#1{\;\mathop{\to}\limits^{#1}\;}
\def\raz#1{\;\mathop{\longrightarrow}\limits^{\!\!\!#1}\;}
\def\ral#1{\;\mathop{\longrightarrow}\limits^{#1}\;}

\newcommand{\Nor}{\mathcal{N}}
\newcommand{\T}{\mathbb{T}}
\newcommand{\Z}{\mathbb{Z}}



\newcommand{\il}[2]{\int\limits_{#1}^{#2}}%интеграл с пределами #1 и #2

\def\sm2{\mathop {\sum\limits^{n^\Theta}\sum\limits^{n^\Theta}}}
\def\sss{\sum\limits}
\def\tr{,\,\ldots\,,\,}
\def\rk{\right]}
\def\lk{\left[}
\def\rf{\right\}}
\def\lf{\left\{}
\def\lv{\,\left\vert}
\def\rv{\right\vert\,}
\def\iii{\int\limits}
\def\iin{\int\limits_{-\infty}^\infty}
\def\rrv{\right\vert}


\def\ee{{\cal E}}
\def\ww{{\cal W}}
\def\yy{{\cal Y}}
\def\vv{{\cal V}}

\newcommand{\R}{\mathbb R}
\newcommand{\E}{\mathbb E}
\newcommand{\N}{\mathbb N}

\renewcommand{\P}{\mathbb{P}}

\newcommand{\h}{{\bf H}}
\newcommand{\p}{{\sf P}}  % вероятность

\newcommand{\e}{{\sf E}}  % мат. ожидание
\newcommand{\D}{{\sf D}}  % дисперсия
\newcommand{\eps}{\varepsilon}
\newcommand{\vp}{{\mathbf p}}
\newcommand{\vz}{{\mathbf z}}
\newcommand{\vx}{{\mathbf x}}
\newcommand{\vf}{{\mathbf f}}
\newcommand{\F}{{\mathcal F}}
\def\ap{{\mathrm{ЭР}}}
\newcommand{\ud}{\Delta_n} %uniform ditance
\newcommand{\nud}{\Delta_n(x)}
%\renewcommand{\Re}{\mathrm{Re}\,}

\newcommand{\abs}[1]{\left\vert#1\right\vert}

\newcommand{\norm}[1]{\left\Vert#1\right\Vert}
\def\da{(\Delta_t,A)}

\newcommand{\corr}{\mathrm{corr}}

\newcommand{\cov}{\mathrm{cov}}
\newcommand{\Expect}{\mathbb{E}}

\def\w{\omega}
\def\W{\Omega}

\def\inh{\int\limits_{nh}^{(n+1)h}}

\def\sumin{\sum_{i=1}^N}


\def\bxt{(Y,t)}
\def\xt{(y,t)}

\def\ovth{{\fr{\tau-nh}{h}}}
\def\ov{\overline}
\def\tm{\tilde m}
\def\tl{\tilde\lambda}
\def\tB{\widetilde B}
\def\tb{\tilde b}
\def\ld{\ldots}
\def\cd{\cdots}


\DeclareMathOperator{\sign}{sign}

%\newcommand{\gr}{{\geqslant}}


\newcommand{\g}{\mbox{\textit{g}}}

\renewcommand{\la}{\lambda}
\newcommand{\si}{\sigma}
\newcommand{\alp}{\alpha}

\newcommand{\pto}{\stackrel{P}{\longrightarrow}} % сходимость по веpоятности

\newcommand{\eqd}{\stackrel{\mathrm{d}}{=}} % равенство по pаспpеделению
\newcommand{\eqdelta}{\stackrel{\triangle}{=}} % равенство по pаспpеделению

\def\be#1{\begin{equation}\label{#1}}
\def\ee{\end{equation}}
\def\re#1{(\ref{#1})}

\def\bn{\begin{enumerate}}
\def\en{\end{enumerate}}
\def\bi{\begin{itemize}}
\def\ei{\end{itemize}}
%\def\i{\item}

%\newcommand{\kp}{\kappa}
%\def\Q{{\cal Q}} \def\H{{\cal H}}
%\newcommand{\bet}{\beta_{2+\delta}}


%\newtheorem{definition}{Определение}
%\renewcommand{\thedefinition}{\arabic{definition}.}
%END NEW COMMANDS

%\renewcommand{\baselinestretch}{1.2}

%\pagestyle{myheadings}

\setlength{\textwidth}{167mm}      % 122mm
\setlength{\textheight}{658pt}
%\setlength{\textheight}{635.6pt}
\setlength{\columnsep}{4.5mm}

\setcounter{secnumdepth}{4}

%\addtolength{\headheight}{2pt}
%\addtolength{\headsep}{-2mm}

\addtolength{\topmargin}{-7mm}  % for printing


%\hoffset=-30mm  % From Yap
\hoffset=-23mm  % From Acrobat

%\voffset=0mm % From Yap
\voffset=-5mm   % From Acrobat

%\addtolength{\evensidemargin}{-2.5mm} % for printing
%\addtolength{\oddsidemargin}{2.5mm}  % for printing

\addtolength{\evensidemargin}{-12mm} % for printing
\addtolength{\oddsidemargin}{8mm}  % for printing

%\renewcommand{\thefootnote}{\fnsymbol{footnote}}
%\renewcommand{\thefootnote}{\arabic{footnote}}
\renewcommand{\figurename}{\protect\bf Рис.}
\renewcommand{\tablename}{\protect\bf Таблица}

\newcommand{\Caption}[1]{\caption{\protect\small %\baselineskip=2.5ex
#1}}

\renewcommand{\thefigure}{\arabic{figure}}
\renewcommand{\thetable}{\arabic{table}}
\renewcommand{\theequation}{\arabic{equation}}
\renewcommand{\thesection}{\arabic{section}}

\renewcommand{\contentsname}{СОДЕРЖАНИЕ}
\newcommand{\fr}[2]{\displaystyle\frac{\displaystyle #1\mathstrut}{\displaystyle #2\mathstrut}}

%\renewcommand{\thefootnote}{\fnsymbol{footnote}}
%\newcommand{\g}{\mbox{\textit{g}}}

%\newcommand{\Caption}[1]{\caption{\protect\small\baselineskip=2ex #1}}
\newcounter{razdel}
\setcounter{razdel}{0}

\def\god{2021}
\def\tom{15}
\def\vyp{2}


\newcommand{\titel}[4]{%
\

\vspace*{5pt}

\ifodd\therazdel {\raggedright\noindent\Large\textrm\textbf
 \lineskip .75em
  \baselineskip=3.2ex #1 \par}
\vskip 1em {\noindent\large\textrm\textbf #2 \par}
\addcontentsline{toc}{subsection}{{\textrm\textbf #1}\protect\newline #2}
\def\rightheadline{\underline{\noindent\hbox to \textwidth{\hfill\small\textrm{#4}
%\hfill \large\bf\thepage
}}}
\def\leftheadline{\underline{\noindent\parbox{\textwidth}{
%\raggedleft\large\bf\thepage \hfill
\small\textit{#3}\hfill}}}
\def\leftfootline{\small{\textbf{\thepage}
\hfill ИНФОРМАТИКА И ЕЁ ПРИМЕНЕНИЯ\ \ \ том~\tom\ \ \ выпуск~\vyp\ \ \ \god}
}%
 \def\rightfootline{\small{ИНФОРМАТИКА И ЕЁ ПРИМЕНЕНИЯ\ \ \ том~\tom\ \ \ выпуск~\vyp\ \ \ \god
\hfill \textbf{\thepage}}}
\vskip 2em \setcounter{figure}{0}
\setcounter{table}{0}
\setcounter{equation}{0}
\setcounter{section}{0}
\setcounter{subsection}{0}
\setcounter{subsubsection}{0}
\setcounter{footnote}{0}
\setcounter{razdel}{0}
%\end{flushleft}
\else {
 \raggedright\noindent\Large\textrm\textbf
 \lineskip .75em
\baselineskip=3.2ex #1 \par} \vskip 1em
%\begin{flushleft}
{\noindent\large\textrm\textbf #2 \par}
\addcontentsline{toc}{subsection}{{\textrm\textbf #1}\protect\newline #2}
\def\rightheadline{\underline{\noindent\hbox to \textwidth{\hfill\small\textrm{#4}
%\hfill \large\bf\thepage
}}}
\def\leftheadline{\underline{\noindent\parbox{\textwidth}{%\raggedleft\large\bf\thepage \hfill
\small\textit{#3}\hfill}}}
\def\leftfootline{\small{\textbf{\thepage}
\hfill ИНФОРМАТИКА И ЕЁ ПРИМЕНЕНИЯ\ \ \ том~\tom\ \ \ выпуск~\vyp\ \ \ \god}
}%
 \def\rightfootline{\small{ИНФОРМАТИКА И ЕЁ ПРИМЕНЕНИЯ\ \ \ том~15\ \ \ выпуск~\vyp\ \ \ 2021
\hfill \textbf{\thepage}}} \vskip 2em \setcounter{figure}{0}
\setcounter{table}{0} \setcounter{equation}{0} \setcounter{section}{0}
\setcounter{subsection}{0} \setcounter{subsubsection}{0}
\setcounter{footnote}{0}
%\end{flushleft}
\fi}

\newcommand{\titelr}[2]{%
\

\vspace*{5pt}

\ifodd\therazdel {\raggedright\noindent%\Large\textrm\textbf
 \lineskip .75em
  \baselineskip=3.2ex #1 \par}
\vskip 1em {\noindent\normalsize\textrm\textbf #2 \par}
\else {
 \raggedright\noindent\Large\textrm\textbf
 \lineskip .75em
\baselineskip=3.2ex #1 \par} \vskip 1em
%\begin{flushleft}
{\noindent\large\textrm\textbf #2 \par
%\noindent\normalsize\textrm\textbf #2 \par
} \fi}

\newcommand{\titele}[5]{%
\

%\vspace*{5pt}

\ifodd\therazdel {\raggedright\noindent\large
\textrm\textbf
 \lineskip .75em
%  \baselineskip=3.2ex
#1 \par}
\vskip .5em {\noindent\large\textrm\textbf #2 \par}
\vskip .5em
 {\noindent\textrm #3 \par}
\addcontentsline{toc}{subsection}{{\textrm\textbf #1}\protect\newline #2}
\def\rightheadline{\underline{\noindent\hbox to \textwidth{\hfill\small\textrm{#4}
%\hfill \large\bf\thepage
}}}
\def\leftheadline{\underline{\noindent\parbox{\textwidth}{
%\raggedleft\large\bf\thepage \hfill
\small\textrm{#5}\hfill}}}
\def\leftfootline{\small{\textbf{\thepage}
\hfill ИНФОРМАТИКА И ЕЁ ПРИМЕНЕНИЯ\ \ \ том~15\ \ \ выпуск~2\ \ \ 2021}
}%
 \def\rightfootline{\small{ИНФОРМАТИКА И ЕЁ ПРИМЕНЕНИЯ\ \ \ том~15\ \ \ выпуск~2\ \ \ 2021
\hfill \textbf{\thepage}}} \vskip 1em \setcounter{figure}{0}
\setcounter{table}{0} \setcounter{equation}{0} \setcounter{section}{0}
\setcounter{subsection}{0} \setcounter{subsubsection}{0}
\setcounter{footnote}{0} \setcounter{razdel}{0}
%\end{flushleft}
\else {
 \raggedright\noindent\large
 \textrm\textbf
 \lineskip .75em
%\baselineskip=3.2ex
#1 \par} \vskip .5em
%\begin{flushleft}
{\noindent\large\textrm\textbf #2 \par} \vskip .5em
 {\noindent\textrm #3 \par}
\addcontentsline{toc}{subsection}{{\textrm\textbf #1}\protect\newline #2}
\def\rightheadline{\underline{\noindent\hbox to \textwidth{\hfill\small\textrm{#4}
%\hfill \large\bf\thepage
}}}
\def\leftheadline{\underline{\noindent\parbox{\textwidth}{%\raggedleft\large\bf\thepage \hfill
\small\textrm{#5}\hfill}}}
\def\leftfootline{\small{\textbf{\thepage}
\hfill ИНФОРМАТИКА И ЕЁ ПРИМЕНЕНИЯ\ \ \ том~15\ \ \ выпуск~2\ \ \ 2021}
}%
 \def\rightfootline{\small{ИНФОРМАТИКА И ЕЁ ПРИМЕНЕНИЯ\ \ \ том~15\ \ \ выпуск~2\ \ \ 2021
\hfill \textbf{\thepage}}} \vskip 1em \setcounter{figure}{0}
\setcounter{table}{0} \setcounter{equation}{0} \setcounter{section}{0}
\setcounter{subsection}{0} \setcounter{subsubsection}{0}
\setcounter{footnote}{0}
%\end{flushleft}
\fi}

\def\Abst#1{
\begin{center}\small\nwt
\parbox{150mm}{%\baselineskip=2.5ex
\textbf{Аннотация:}\ \
%\hspace*{\parindent}
#1}
\end{center}}
\def\Abste#1{
\begin{center}\small\nwt
\parbox{150mm}{%\baselineskip=2.5ex
\textbf{Abstract:}\ \
%\hspace*{\parindent}
#1}
\end{center}}

\def\DOI#1{
\begin{center}\small\nwt
\parbox{150mm}{%\baselineskip=2.5ex
\textbf{DOI:}\ \
%\hspace*{\parindent}
#1}
\end{center}}

\def\Abstend#1{
\begin{center}\small\nwt
\parbox{150mm}{%\baselineskip=2.5ex
%\hspace*{\parindent}
#1}
\end{center}}


\def\KW#1{
\begin{center}\small\nwt
\parbox{150mm}{%\baselineskip=2.5ex
\textbf{Ключевые слова:}\ \ #1}
\end{center}}

\def\KWE#1{
\begin{center}\small\nwt
\parbox{150mm}{%\baselineskip=2.5ex
\textbf{Keywords:}\ \ #1}
\end{center}}


\def\KWN#1{
%\begin{center}
%\small
%\parbox{150mm}\end{center}
}

\newcommand{\Avtors}[1]{%\smallskip
%\vspace*{.5pt}
\hangindent=23pt\noindent
%\nwt
{\bfseries#1}\
}


\renewcommand{\thesubsection}{\thesection.\arabic{subsection}\hspace*{-5pt}}
\renewcommand{\thesubsubsection}{\thesubsection\hspace*{5pt}.\arabic{subsubsection}\hspace*{-3pt}}

\newcommand{\Ack}{\section*{\protect\rmfamily Acknowledgments}\noindent}
\newcommand{\Contr}{\section*{\protect\rmfamily Contributors}\noindent}
\newcommand{\Contrl}{\section*{\protect\rmfamily Contributor}\noindent}

\makeindex


\begin{document}
\Rus

\nwt
%\ptb


%\renewcommand{\contentsname}{\protect\Large\bf Содержание}

\setcounter{tocdepth}{2}

%\tableofcontents

\renewcommand{\bibname}{\protect\rmfamily Литература}
  \def\Au#1{{\it #1}}
    \def\Aue#1{{#1}}

%\newcommand{\No}{№}
  \newcommand{\tg}{\,\mathrm{tg}\,}
    \newcommand{\ctg}{\,\mathrm{ctg}\,}
  \newcommand{\arctg}{\,\mathrm{arctg}\,}

\def\forallb{\mathop{\forall}}
\def\cupb{\mathop{\cup}}
\def\existsb{\mathop{\exists}}


\newpage
\addtocounter{razdel}{1}
%\def\razd{РЕГУЛИРУЕМЫЙ ЭЛЕКТРОПРИВОД ДЛЯ ЭЛЕКТРОЭНЕРГЕТИКИ}


\setcounter{page}{3}

%   { %\Large  
   { %\baselineskip=16.6pt
   
   \vspace*{-48pt}
   \begin{center}\LARGE
   \textit{Предисловие}
   \end{center}
   
   %\vspace*{2.5mm}
   
   \vspace*{25mm}
   
   \thispagestyle{empty}
   
   { %\small 

    
Вниманию читателей журнала <<Информатика и её применения>> предлагается 
очередной тематический выпуск <<Вероятностно-статистические методы и 
задачи информатики и информационных технологий>>. Предыдущие тематические 
выпуски журнала по данному направлению вышли в 2008~г.\ (т.~2, вып.~2), 
в 2009~г.\ (т.~3, вып.~3) и в 2010~г.\ (т.~4, вып.~2). 

Статьи, собранные в данном журнале, посвящены разработке новых вероятностно-статистических 
методов, ориентированных на применение к решению конкретных задач информатики и информационных 
технологий, а также~--- в ряде случаев~--- и других прикладных задач. Проблематика, охватываемая 
публикуемыми работами, развивается в рамках научного сотрудничества между Институтом проблем 
информатики Российской академии наук (ИПИ РАН) и Факультетом вычислительной математики и 
кибернетики Московского государственного университета им.\ М.\,В.~Ломоносова в ходе работ 
над совместными научными проектами (в том числе в рамках функционирования 
Научно-образовательного центра <<Вероятностно-статистические методы анализа рисков>>). 
Многие из авторов статей, включенных в данный номер журнала, являются активными участниками 
традиционного международного семинара по проблемам устойчивости стохастических моделей, 
руководимого В.\,М.~Золотаревым и В.\,Ю.~Королевым; регулярные сессии этого семинара 
проводятся под эгидой МГУ и ИПИ РАН (в 2011~г.\ указанный семинар проводится в октябре 
в Калининградской области РФ). 

Наряду с представителями ИПИ РАН и МГУ в число авторов данного выпуска журнала входят 
ученые из Научно-исследовательского института системных исследований РАН, Института 
проблем технологии микроэлектроники и особочистых материалов РАН, Института 
прикладных математических исследований Карельского НЦ РАН, Московского 
авиационного института, Вологодского государственного педагогического университета, 
НИИММ им.\ Н.\,Г.~Чеботарева, Казанского государственного университета, Дебреценского 
университета (Венгрия).

Несколько статей выпуска посвящено разработке и применению стохастических методов и 
информационных технологий для решения различных прикладных задач. В~работе В.\,Г.~Ушакова 
и О.\,В.~Шестакова рассмотрена задача определения вероятностных характеристик случайных 
функций по распределениям интегральных преобразований, возникающих в задачах эмиссионной 
томографии. В~статье Д.\,О.~Яковенко и М.\,А.~Целищева рассмотрены некоторые вопросы 
математической теории риска и предложен новый подход к диверсификации инвестиционных 
портфелей. Работа И.\,А.~Кудрявцевой и А.\,В.~Пантелеева посвящена построению и 
исследованию математической модели, описывающей динамику сильноионизованной плазмы. 
В~статье П.\,П.~Кольцова изучается качество работы ряда алгоритмов сегментации изображений. 
Статья А.\,Н.~Чупрунова и И.~Фазекаша посвящена вероятностному анализу числа без\-оши\-бочных 
блоков при помехоустойчивом кодировании; получены усиленные законы больших чисел для указанных 
величин.

В данном выпуске традиционно присутствует тематика, весьма активно разрабатываемая в течение 
многих лет специалистами ИПИ РАН и МГУ,~--- методы моделирования и управления для 
информационно-телекоммуникационных и вычислительных систем, в частности методы 
теории массового обслуживания. В~статье А.\,И.~Зейфмана с соавторами рассматриваются 
модели обслуживания, описываемые марковскими цепями с непрерывным временем в случае 
наличия катастроф. В~работе М.\,М.~Лери и И.\,А.~Чеплюковой рассматриваются случайные 
графы Интернет-типа, т.\,е.\ графы, степени вершин которых имеют степенные распределения; 
такие задачи находят применение при исследовании глобальных сетей передачи данных. 
Работа Р.\,В.~Разумчика посвящена исследованию систем массового обслуживания специального 
вида~--- с отрицательными заявками и хранением вытесненных заявок.

Ряд статей посвящен развитию перспективных теоретических 
вероятностно-статистических методов, которые находят широкое применение в различных 
задачах информатики и информационных технологий. В~работе В.\,Е.~Бенинга, А.\,К.~Горшенина 
и В.\,Ю.~Королева рассмотрена задача статистической проверки гипотез о числе компонент 
смеси вероятностных распределений, приводится конструкция асимптотически наиболее мощного 
критерия. Результаты этой работы найдут применение в ряде прикладных задач, использующих 
математическую модель смеси вероятностных распределений (в информатике, моделировании 
финансовых рынков, физике турбулентной плазмы и~т.\,д.). В~статье В.\,Ю.~Королева, 
И.\,Г.~Шевцовой и С.\,Я.~Шоргина строится новая, улучшенная оценка точности нормальной 
аппроксимации для пуассоновских случайных сумм; как известно, указанные случайные суммы 
широко используются в качестве моделей многих реальных объектов, в том числе в информатике, 
физике и других прикладных областях. Работа В.\,Г.~Ушакова и Н.\,Г.~Ушакова посвящена 
исследованию ядерной оценки плотности распределения; эти результаты могут применяться, 
в част\-ности, при анализе трафика в телекоммуникационных системах. Серьезные приложения 
в статистике могут получить результаты работы О.\,В.~Шестакова, в которой доказаны оценки 
скорости сходимости распределения выборочного абсолютного медианного отклонения к нормальному 
закону. 

\smallskip

Редакционная коллегия журнала выражает надежду, что данный тематический  выпуск 
будет интересен специалистам в области теории вероятностей и математической статистики 
и их применения к решению задач информатики и информационных технологий.
     
     %\vfill 
     \vspace*{20mm}
     \noindent
     Заместитель главного редактора журнала <<Информатика и её 
применения>>,\\
     директор ИПИ РАН, академик  \hfill
     \textit{И.\,А.~Соколов}\\
     
     \noindent
     Редактор-составитель тематического выпуска,\\
     профессор кафедры математической статистики факультета\\
      вычислительной математики и кибернетики МГУ им.\ М.\,В.~Ломоносова,\\
     ведущий научный сотрудник ИПИ РАН,\\ 
доктор физико-математических наук \hfill
      \textit{В.\,Ю.~Королев}
     
     } }
     }


   
\def\stat{bosov+stef}

\def\tit{УПРАВЛЕНИЕ ВЫХОДОМ СТОХАСТИЧЕСКОЙ ДИФФЕРЕНЦИАЛЬНОЙ СИСТЕМЫ 
ПО~КВАДРАТИЧНОМУ КРИТЕРИЮ. I.~ОПТИМАЛЬНОЕ РЕШЕНИЕ МЕТОДОМ 
ДИНАМИЧЕСКОГО ПРОГРАММИРОВАНИЯ$^*$}

\def\titkol{Управление выходом стохастической дифференциальной системы 
по~квадратичному критерию. I}
%.~Оптимальное решение методом 
%динамического программирования}

\def\aut{А.\,В.~Босов$^1$, А.\,И.~Стефанович$^2$}

\def\autkol{А.\,В.~Босов, А.\,И.~Стефанович}

\titel{\tit}{\aut}{\autkol}{\titkol}

\index{Босов А.\,В.}
\index{Стефанович А.\,И.}
\index{Bosov A.\,V.}
\index{Stefanovich A.\,I.}




{\renewcommand{\thefootnote}{\fnsymbol{footnote}} \footnotetext[1]
{Работа выполнена при частичной поддержке РФФИ (проект 16-07-00677).}}


\renewcommand{\thefootnote}{\arabic{footnote}}
\footnotetext[1]{Институт проблем информатики Федерального исследовательского центра <<Информатика 
и~управление>> Российской академии наук, \mbox{AVBosov@ipiran.ru}}
\footnotetext[2]{Институт проблем информатики Федерального исследовательского центра <<Информатика 
и~управление>> Российской академии наук, \mbox{AStefanovich@frccsc.ru}}

%\vspace*{8pt}



  
  \Abst{Решается задача оптимального управления для диффузионного процесса 
Ито и~линейного управ\-ля\-емо\-го выхода. Рассматриваемая постановка близка 
к~классической ли\-ней\-но-квад\-ра\-тич\-ной гауссовской задаче управления 
(linear-quadratic Gaussian (LQG) control). Отличия состоят в~том, что состояние описывается нелинейным 
дифференциальным уравнение Ито $dy_t\hm= A_t(y_t) \,dt\hm+ \Sigma_t(y_t)\,dv_t$ 
и~не зависит от управ\-ле\-ния~$u_t$, оптимизации подлежит управ\-ля\-емый 
линейный выход $dz_t\hm= a_t y_t\,dt\hm+ b_t z_t \,dt\hm+ c_t u_t \,dt\hm+ \sigma_t\, 
dw_t$. Дополнительные обобщения внесены в~квад\-ра\-тич\-ный критерий качества 
с~целью воз\-мож\-ности постановки таких задач, как отслеживание выходом 
состояния или управ\-ле\-ни\-ем~--- линейной комбинации состояния и~выхода. Для 
решения используется метод динамического программирования. Функцию 
Беллмана позволяет найти предположение о~ее структуре вида $V_t(y,z)\hm= 
\alpha_t z^2\hm+ \beta_t(y)z \hm+\gamma_t(y)$. Решение дают три 
дифференциальных уравнения для коэффициентов~$\alpha_t$, $\beta_t(y)$ 
и~$\gamma_t(y)$. Эти уравнения со\-став\-ля\-ют оптимальное решение 
рас\-смат\-ри\-ва\-емой задачи.}
  
  \KW{стохастическое дифференциальное уравнение; оптимальное управ\-ле\-ние; 
динамическое программирование; функция Беллмана; уравнение Риккати; 
линейные уравнения параболического типа}

\DOI{10.14357/19922264180314}
  
%\vspace*{4pt}


\vskip 10pt plus 9pt minus 6pt

\thispagestyle{headings}

\begin{multicols}{2}

\label{st\stat}

\section{Введение}

     Ключевые результаты в~области оптимизации стохастических 
динамических систем, со\-став\-ля\-ющие классическую теорию управления, 
получены более~40~лет назад (такова работа~[1] в~отношении задачи 
управ\-ле\-ния ли\-ней\-но-гаус\-сов\-ски\-ми стохастическими сис\-те\-ма\-ми по 
квад\-ра\-тич\-но\-му критерию). К~классической тео\-рии следует относить 
линейные модели стохастических сис\-тем и~квадратичный критерий качества. 
Это исходный базис, на котором основано множество успешно 
исследованных и~решенных задач стохастического управ\-ле\-ния 
и~оптимизации. 

Дальнейшее развитие~--- это новые модели и~критерии, но 
прежде всего это новые методы: от тео\-рии линейных регуляторов, метода 
динамического программирования и~принципа максимума к~адаптивному 
и~минимаксному подходу, импульсному управ\-ле\-нию и~т.\,д. Множество 
инноваций как в~час\-ти моделей, так и~в~час\-ти математического аппарата, 
имевших мес\-то в~по\-сле\-ду\-ющие годы, существенно обогатили тео\-рию 
управ\-ле\-ния. Но и~до настоящего времени линейные модели и~квадратичный 
критерий, несмотря на всю справедливую критику в~отношении их 
аде\-кват\-ности и~гиб\-кости, сохраняют исследовательский интерес и~находят 
современные области приложения.
     
     Не претендуя на сколь\-ко-ни\-будь полное обосно\-ва\-ние последнего 
тезиса, приведем несколько примеров, показавшихся наиболее ин\-те\-рес\-ными. 

Так, в~[2] решается ли\-ней\-но-квад\-ра\-тич\-ная за\-да\-ча в~игровой 
постановке с~запаздыванием. В~близ\-кой по модели работе~[3] задача 
управ\-ле\-ния ставится в~терминах $H_\infty$-ро\-баст\-ности. Точнее \mbox{называть} 
эту тематику $H_2/H_\infty$-управ\-ле\-ни\-ем, и~работ по этой теме очень 
много. Аккуратности ради следует уточнить, что под линейными 
понимаются модели с~мультипликативными по состоянию воз\-му\-ще\-ниями. 

Совсем другой класс моделей, особо популярных в~по\-след\-ние годы, 
составляют скачкообразные процессы. Например, линейные уравнения 
в~сочетании с~пуассоновскими скачками в~[4] используются в~моделях, 
описывающих различные показатели функционирования сетевых протоколов 
передачи данных транспортного уровня. Телекоммуникации представляют 
в~последние годы самый популярный прикладной материал для 
исследований, работ по этой проб\-ле\-ма\-ти\-ке множество, математические 
техники привлекаются самые разные и~самые современные, но и~линейным 
моделям место находится. Еще один любопытный пример исследования 
скачкообразного процесса и~оптимизации на основе квад\-ра\-тич\-но\-го критерия 
можно найти в~[5] применительно к~задаче инвестирования на финансовом 
рынке. Наконец, упомянем еще работу~[6], подводящую итог исследований 
в~отношении классической детерминированной  
ли\-ней\-но-квад\-ра\-тич\-ной задачи с~использованием техники матричных 
неравенств.
     
     В данной работе также эксплуатируются привлекательные свойства 
линейных моделей и~квад\-ра\-тич\-но\-го критерия, причем в~стохастической 
постановке. На\-прав\-ле\-ни\-ем для обобщения \mbox{выбрана} модель динамики 
сис\-те\-мы: основные усилия на\-прав\-ле\-ны на то, чтобы сделать ее нелинейной. 
Кроме того, пред\-став\-лен\-ная постановка может рас\-смат\-ри\-вать\-ся и~как 
обобщение ранее решенной задачи в~дискретном времени~[7, 8] на время 
непрерывное. В~упомянутых работах помимо собственно модельной 
постановки важна еще и~привлекаемая прикладная об\-ласть~--- 
функционирование сложных программных сис\-тем. Результатов, 
ориентированных непосредственно на такие приложения, к~настоящему 
времени пренебрежимо мало, поэтому~[7, 8]~--- это еще и~прикладное 
обоснование рас\-смат\-ри\-ва\-емой далее задачи.
     
     Оптимизируемая динамическая сис\-те\-ма описывается двумя 
уравнениями. Состояние задается нелинейным стохастическим 
дифференциальным уравнением Ито, не содержащим управ\-ля\-емой 
переменной. Возмущение здесь описывается стандартным винеровским 
процессом, накладываются простые условия существования 
и~един\-ст\-вен\-ности решения. Поскольку состояние не управ\-ля\-ет\-ся, то уместно 
его интерпретировать как слож\-ное внешнее возмущение. Вторая 
переменная~--- управ\-ля\-емый выход~--- задается линейным стохастическим 
дифференциальным уравнением. Цель оптимизации выхода формируется 
квадратичным критерием общего вида. Формальная постановка задачи 
приведена в~сле\-ду\-ющем разделе.
     
     Для решения задачи используется метод динамического 
программирования, решается уравнение Беллмана~[9]. Соответственно, 
в~результате получаются аналитические выражения и~для оптимального 
управ\-ле\-ния, и~для значения функционала качества. Технически 
традиционный, стандартный подход к~задаче обременен, пожалуй, 
единственной проблемой~--- поиском верного пред\-став\-ле\-ния структуры 
функции Беллмана. Справиться с~этой проблемой в~большей степени удается 
за счет результата, полученного при решении дискретного по времени 
аналога рассматриваемой постановки~\cite{8-bos}. Конечные соотношения 
для оптимального решения, как и~во всех подобных задачах, включая 
классическую ли\-ней\-но-квад\-ра\-тич\-ную, содержат решения 
определенных дифференциальных уравнений (обыкновенных и~в~частных 
производных). Вывод этих уравнений и~со\-став\-ля\-ет содержание первой час\-ти 
данной работы. Во второй части будет обсуждаться их приближенное 
чис\-лен\-ное решение и~компьютерные эксперименты.
     
     Кратко обозначим основные положения, при\-вле\-ка\-емые далее 
к~решению задачи, следуя в~основном обозначениям 
и~терминологии~\cite{9-bos}, а~именно: будем рассматривать задачу 
оптимального управления в~стохастической динамической сис\-те\-ме по полной 
информации, применяя метод динамического программирования. В~качестве 
целевого функционала, опре\-де\-ля\-юще\-го качество управ\-ле\-ния $U_0^T\hm= \{ 
u_t,\ 0\leq t\leq T\}$, выступает
     \begin{equation}
     J\left(U_0^T\right)={\sf E}\left\{ \int\limits_0^T L_t \left(x_t, u_t\right)\,dt+ 
l\left(x_T\right)\right\}\,.
     \label{e1-bos}
     \end{equation}
Здесь ${\sf E}\{\cdot\}$~--- оператор математического ожидания; $x_t$~--- 
случайный процесс, описываемый стохастическим дифференциальным 
уравнением Ито
     \begin{equation}
     dx_t=m_t\left( x_t, u_t\right) dt+ \sigma_t\left( x_t\right)dW_t\,,\enskip 
x_0=X\,,
     \label{e2-bos}
     \end{equation}
где $W_t$~--- стандартный винеровский процесс подходящей раз\-мер\-ности; 
$X$~--- случайный вектор.

     $U_0^T$ будем выбирать из класса допустимых неупреждающих (по 
отношению к~$W_t$) управлений~\cite{9-bos}. Соответственно, 
относительно функций сноса и~диффузии~$m_t$ и~$\sigma_t$  
в~(\ref{e2-bos}) будем предполагать выполненными ка\-кие-ли\-бо условия 
существования сильного решения для заданного до\-пус\-ти\-мо\-го управ\-ле\-ния. 
Например, для управ\-ле\-ния с~обратной связью $u_t\hm= u_t(x_t)$ будем 
считать, что $m_t(x,u_t(x))$ и~$\sigma_t(x)$ удовлетворяют условию 
линейного рос\-та и~локальному условию Липшица по~$x$ равномерно 
по~$t$ (т.\,е.\ условиям Ито).
     
     Для поиска оптимального управления, минимизирующего $J(U_0^T)$, 
рас\-смат\-ри\-ва\-ет\-ся функция Беллмана
     \begin{equation}
     V_t(x)=\left.\mathop{\mathrm{inf}}\limits_{U_t^T} {\sf E} \left\{ \int\limits_t^T 
L_t \left( x_t, u_t\right)\,dt+l\left( x_T\right) \right\vert \mathcal{F}_t^x\right\}\,,
     \label{e3-bos}
     \end{equation}
где $\mathcal{F}_t^x$~--- $\sigma$-ал\-геб\-ра, по\-рож\-ден\-ная~$x_\tau$, 
$0\hm\leq \tau\hm\leq t$, ${\sf E}\{\cdot\vert \mathcal{F}\}$~--- оператор условного 
математического ожидания относительно~$\mathcal{F}$. Соответственно, 
в~качестве достаточного условия оп\-ти\-маль\-ности воспользуемся уравнением 
динамического программирования
\begin{multline}
\fr{\partial V_t(x)}{\partial t} +\fr{1}{2}\sum\limits^n_{i,j=1} \sigma^2_{t_{ij}}
\fr{\partial^2 V_t(x)}{\partial x_i \partial x_j}+{}\\
{}+\min\limits_u\left[  
\sum\limits^n_{i=1} m_{t_i} \fr{\partial V_t(x)}{\partial x_i} + L_t(x,u)\right] 
=0\,,\\
V_T(x)=l(x)\,,
\label{e4-bos}
\end{multline}
где $m_{t_i}$~--- $i$-й элемент век\-тор-функ\-ции~$m_t(x,u)$; 
$\sigma^2_{t_{ij}} \hm= \sum\nolimits^m_{k=1} 
\sigma_{t_{ik}}\sigma_{t_{ki}}$, $\sigma_{t_{ij}}$~--- $i$-й по строке, $j$-й 
по столб\-цу элемент мат\-рич\-ной функции~$\sigma_t(x)$; $n$ и~$m$~--- 
размерности~$x_t$ и~$W_t$ соответственно.

     Традиционно в~рамках применения метода динамического 
программирования будем предполагать, что функции~$L_t$, $l$, $m_t$ 
и~$\sigma_t$ обеспечивают существование хотя бы одного решения 
уравнения~(\ref{e4-bos}), а~следовательно, и~оптимального 
управления~$u_t^*$, $0\hm\leq t\hm\leq T$, до\-став\-ля\-юще\-го минимум 
целевому функционалу~(\ref{e1-bos}). Задача оптимизации далее получается 
путем указания конкретных выражений для~$L_t$, $l$, $m_t$ и~$\sigma_t$.

\section{Постановка задачи управления выходом}

     Рассматриваемые далее случайные функции будут предполагаться 
скалярными. Такое упрощение позволит разгрузить выкладки и~итоговые 
выражения от не самых существенных деталей.
     
     Рассмотрим стохастическую дифференциальную сис\-те\-му, со\-сто\-яние 
которой представляет диффузи\-он\-ный процесс~$y_t$, описываемый 
нелинейным стохастическим дифференциальным уравнением Ито
     \begin{equation}
     dy_t=A_t\left( y_t\right) dt +\Sigma_t \left( y_t\right) dv_t\,,\enskip 
y_0=Y\,,
     \label{e5-bos}
     \end{equation}
где $v_t$~--- стандартный (одномерный) винеровский процесс; $Y$~--- 
случайная величина с~конечным вторым моментом; функции~$A_t$ 
и~$\Sigma_t$ удовлетворяют условиям Ито:
\begin{equation*}
\left\vert A_t(y)\right\vert +\left\vert \Sigma_t(y)\right\vert \leq C(1+\vert y\vert )\ 
\mbox{для\ всех } 0\leq t\leq T\,;
\end{equation*}

\vspace*{-12pt}

\noindent
\begin{multline*}
\hspace*{-2.10051pt}\left\vert A_t\left(y_1\right) -A_t \left( y_2\right) \right\vert +\left\vert 
\Sigma_t\left( y_1\right) -\Sigma_t \left(y_2\right)\right\vert \leq
C\left\vert y_1-y_2\right\vert\\
 \mbox{для\ всех\ } 0\leq t\leq T\ \mbox{и } 
y_1,y_2\in \mathbb{R}^1\,,
\end{multline*}
обеспечивающим существование единственного сильного (потраекторного) 
решения уравнения.
     
     Будем считать, что~$y_t$ описывает состояние некоторой 
динамической системы. Соответственно, поведение этой сис\-те\-мы опишем 
выходом, линейно связанным с~со\-сто\-янием:
     \begin{equation}
     dz_t=a_t y_t \,dt+ b_t z_t \,dt+ c_t u_t \,dt+\sigma_t \,dw_t\,,\enskip
     z_0=Z\,.
     \label{e6-bos}
     \end{equation}
Здесь $w_t$~--- не зависящий от~$v_t$, $Y$ и~$Z$ стандартный (одномерный) 
винеровский процесс; $Z$~--- случайная величина с~конечным вторым 
моментом; $u_t$~--- допустимое неупреждающее управ\-ле\-ние, качество 
которого определяется целевым функционалом следующего вида:
\begin{multline}
\!\hspace*{-3.98538pt}J\left( U_0^T\right) ={\sf E}\left\{ \int\limits_0^T \!\left( S_t\left( s_ty_t-g_t z_t -h_t 
u_t\right)^2 +G_t z_t^2+{}\right.\right.\\
\left.\left.{}+ H_t u_t^2
\vphantom{S_t\left( s_ty_t-g_t z_t -h_t 
u_t\right)^2}
\right) dt+S_T\left( s_T y_T -g_T 
z_T\right)^2+G_T z_T^2
\vphantom{\int\limits_0^T}\right\}\,,
\label{e7-bos}
\end{multline}
где $S_t$, $G_t$ и~$H_t$~--- неотрицательные функции\linebreak
$0\hm\leq t\hm\leq T$. 
Такой критерий отражает физический смысл задачи распределения ресурсов 
со\-глас\-но аналогичной~(\ref{e5-bos})--(\ref{e7-bos}) задаче для дис\-крет\-но\-го 
времени, рас\-смот\-рен\-ной в~\cite{7-bos}. В~част\-ности,  
функци\-онал~(\ref{e7-bos}) поз\-во\-ля\-ет ставить задачи отслеживания
 выходом 
со\-сто\-яния сис\-те\-мы, используя сла\-га\-емое $(y_t\hm- z_t)^2$, или 
управлением~--- линейной комбинации со\-сто\-яния и~выхода, сла\-га\-емое типа\linebreak 
$(y_t\hm+ z_t\hm- u_t)^2$. Поскольку задача формулируется 
в~предположении наличия пол\-ной информации о~со\-сто\-янии~$y_t$ 
и~выходе~$z_t$ (соответствующую $\sigma$-ал\-геб\-ру 
обозначим~$\mathcal{F}_t^{y,z}$), то допустимое управ\-ле\-ние ищется 
в~классе~$\mathcal{F}_t^{y,z}$-из\-ме\-ри\-мых неупреждающих функций 
(и,~как будет показано далее, оказывается управ\-ле\-ни\-ем с~обратной связью).

     Функции~$a_t$, $b_t$, $c_t$ и~$\sigma_t$ будем предполагать 
ограниченными: $\vert a_t\vert \hm+ \vert b_t\vert \hm+\vert c_t\vert \hm+ \vert 
\sigma_t \vert \hm\leq C$ для всех $0\hm\leq t\hm\leq T$, процесс  
управления~--- допустимым не\-упреж\-да\-ющим~\cite{9-bos}, обеспечивая, 
таким образом, существование сильного решения урав\-не\-ния~(\ref{e6-bos}) 
для любого допустимого управ\-ления.
     
     Задачу составляет поиск~$u_t^*$~--- допустимого управ\-ле\-ния, 
доставляющего минимум квад\-ра\-тич\-но\-му функционалу~$J(U_0^T)$.
      
     Поставленная задача очевидным образом формулируется в~терминах 
введенных выше в~(\ref{e1-bos})--(\ref{e3-bos}) обозначений, а~именно: 
     требуется обозначить
     \begin{gather*}
      x_t=\begin{pmatrix}
     y_t\\ z_t\end{pmatrix};\quad  m_t(x_t, u_t)=\begin{pmatrix}
     A_t(y_t)\\ a_t y_t +b_t z_t +c_t u_t\end{pmatrix};\\
     \sigma_t(x_t)= \begin{pmatrix}
     \Sigma_t(y_t)& 0\\
     0& \sigma_t\end{pmatrix};\quad W_t=\begin{pmatrix}
     v_t \\ w_t\end{pmatrix}
     %     \label{e8-bos}
     \end{gather*}
для записи уравнения со\-сто\-яния типа~(\ref{e2-bos}) и
\begin{align*}
L_t(x,u)&= L_t(y,z,u) ={}\\
&\hspace*{3mm}{}=S_t\left( s_t y-g_t z -h_t u\right)^2 +G_t z^2 +H_t  u^2\,;\\
l(x)&= l(y,z) =S_T \left( S_T y-g_T z\right)^2 +G_T z^2
%\label{e9-bos}
\end{align*}
для записи целевого функционала в~виде~(\ref{e1-bos}).

     Функция Беллмана~(\ref{e3-bos}) принимает вид 
     $V_t(x)\hm= V_t(y,z)$. Для записи со\-от\-вет\-ст\-ву\-юще\-го~(\ref{e4-bos}) 
уравнения Беллмана для~$V_t(y,z)$ заметим, что
     $$
     \left( \sigma^2_{t_{ij}}\right)_{i,j=1,2}= \begin{pmatrix}
     \Sigma_t^2(y) & 0\\
     0 & \sigma_t^2\end{pmatrix}\,.
     $$
     
     С~учетом перечисленных обозначений урав\-не\-ние динамического 
программирования~(\ref{e4-bos}) принимает вид:
     \begin{multline}
     \fr{\partial V_t(y,z)}{\partial t} +\fr{1}{2}\left( \Sigma_t^2(y) \fr{\partial^2 
V_t(y,z)} {\partial y^2}+\sigma_t^2\fr{\partial^2 V_t(y,z)} {\partial 
z^2}\right)+{}\\
    {}+\min\limits_u\! \left[ A_t(y) \fr{\partial V_t(y,z)}{\partial y}+\left( a_t 
y+b_t z+c_t u\right) \fr{\partial V_t(y,z)}{\partial z} +{}\right.\hspace*{-3pt}\\
\left.{}+ S_t\left( s_t y-g_t z-h_t 
u\right)^2+G_t z^2+H_t u^2
     \vphantom{\fr{\partial V_t(y,z)}{\partial y}}\right] =0\,,\\
     V_T(y,z)=S_T\left( s_T y-g_T z\right)^2+G_T z^2\,.
     \label{e10-bos}
     \end{multline}
     Это и~есть то самое уравнение, которое требуется решить: 
существование решения данного урав\-не\-ния суть достаточное условие 
оптимальности; оптимальное управ\-ле\-ние при этом~--- точ\-ка минимума 
со\-от\-вет\-ст\-ву\-юще\-го сла\-га\-емого.
     
\section{Динамическое программирование и~оптимальное 
управление}

     В рассматриваемой постановке линейность\linebreak выхода и~квадратичность 
критерия дают те же преимущества, что и~в~классической  
ли\-ней\-но-квад\-ра\-тич\-ной задаче управ\-ле\-ния~\cite{1-bos}, а~именно: 
позволяют сразу определить вид оптимального управ\-ле\-ния и~фактические 
условия его существования. Действительно, со\-хра\-няя в~(\ref{e10-bos}) под 
знаком $\min\nolimits_u$ только члены, зависящие от~$u$, получаем
     \begin{multline*}
     \fr{\partial V_t(y,z)}{\partial t} +\fr{1}{2}\left( \Sigma_t^2(y) \fr{\partial^2 
V_t(y,z)} {\partial y^2}+\sigma_t^2\fr{\partial^2 V_t(y,z)} {\partial 
z^2}\right)+{}\\
     {}+A_t(y)\fr{\partial V_t(y,z)}{\partial y}+\left( a_t y+b_t z\right) 
\fr{\partial V_t(y,z)}{\partial z}+{}\\
{}+S_t\left( s_t y-g_t z\right)^2 +G_t z^2+{}
\end{multline*}

\noindent
\begin{multline*}
     {}+\min\limits_u \left[ \left( c_t \fr{\partial V_t(y,z)}{\partial z}-2S_t \left( 
s_t y-g_t z\right) h_t\right)u +{}\right.\\
\left.{}+\left( S_t h_t^2+H_t\right) u^2
\vphantom{\fr{\partial V_t(y,z)}{\partial z}}
\right]=0\,,
     %\label{e11-bos}
     \end{multline*}
откуда в~предположении $S_t h_t^2\hm+ H_t\hm>0$ следует, что существует 
оптимальное управ\-ле\-ние, которое определяется равенством
\begin{multline}
u_t^* = u_t^*(y,z)=-\fr{1}{2}\left( S_t h_t^2 +H_t\right)^{-1} \left( c_t 
\fr{\partial V_t(y,z)}{\partial z}-{}\right.\\
\left.{}-2S_t\left( s_t y-g_t z\right) h_t
\vphantom{\fr{\partial V_t(y,z)}{\partial z}}
\right)
\label{e12-bos}
\end{multline}
и доставляет минимум соответствующему сла\-га\-емо\-му в~урав\-не\-нии Беллмана, 
равный
$-\left( S_t h_t^2\hm+\right.$\linebreak
$\left.{}+H_t\right)^{-1} \left( c_t 
{\partial V_t(y,z)}/{\partial 
z}\hm-2S_t\left( s_t y \hm-g_t z\right) h_t \right)^2/4.
$ 
     
     Отметим, что, как и~в~классической ли\-ней\-но-квад\-ра\-тич\-ной 
задаче, управ\-ле\-ние из класса до\-пус\-ти\-мых не\-упреж\-да\-ющих получилось 
управ\-ле\-ни\-ем с~обратной связью.
     
     Таким образом, функция Беллмана описывается сле\-ду\-ющим 
дифференциальным уравнением:
     \begin{multline}
     \fr{\partial V_t(y,z)}{\partial t} +\fr{1}{2}\left( \Sigma_t^2(y) \fr{\partial^2 
V_t(y,z)} {\partial y^2}+\sigma_t^2\fr{\partial^2 V_t(y,z)} {\partial 
z^2}\right)+{}\\
     {}+ A_t(y) \fr{\partial V_t(y,z)}{\partial y}+\left( a_t y+b_t z\right) 
\fr{\partial V_t(y,z)}{\partial z}+{}\\
{}+ S_t \left( s_t y- g_t z\right)^2 +G_t z^2-
 \fr{1}{4}\left( S_t h_t^2+H_t\right)^{-1}\times{}\\
 {}\times \left( c_t \fr{\partial V_t(y,z)} 
{\partial z}-2S_t\left( s_t y -g_t z\right) h_t \right)^2=0\,.
     \label{e13-bos}
     \end{multline}
     
     Возводя в~квадрат по\-след\-нее сла\-га\-емое в~(\ref{e13-bos}), перепишем 
его в~виде:
     \begin{multline}
     \fr{\partial V_t(y,z)}{\partial t} +\fr{1}{2}\left( \Sigma_t^2(y) \fr{\partial^2 
V_t(y,z)} {\partial y^2}+\sigma_t^2\fr{\partial^2 V_t(y,z)} {\partial 
z^2}\!\right)+{}\\
{}+A_t(y) \fr{\partial V_t(y,z)}{\partial y}
+ \left( 
\vphantom{\left( S_t h_t^2 +H_t\right)^{-1}}
a_t y+b_t z+{}\right.\\
\left.{}+\left( S_t h_t^2 +H_t\right)^{-1}
 c_t S_t \left( s_t y-g_t z\right) h_t
\right) 
     \fr{\partial V_t(y,z)}{\partial z}+{}\\
     {}+\left( S_t-\left( S_t h_t^2 +H_t\right)^{-1} S_t^2 h_t^2\right)\left( s_t y -
g_t z\right)^2+{}\\
     \!\!{}+
     G_t z^2 -\fr{1}{4}\left( S_t h_t^2+H_t\right)^{-1}\! c_t^2
     \left(\fr{\partial V_t(y,z)}{\partial z}\right)^{\!2}=0\,.\!\!
     \label{e14-bos}
     \end{multline}
     
     Рассматривая полученное уравнение, заметим, что его решение может 
быть пред\-став\-ле\-но в~виде:
   \begin{equation}
     V_t(y,z)= \alpha_t z^2+\beta_t(y) z +\gamma_t(y)\,,
     \label{e15-bos}
     \end{equation}
т.\,е.\ будем искать решение при дополнительном предположении 
о~квад\-ра\-тич\-ности функции Белл\-ма\-на по переменной~$z$, и~сведем, таким 
образом, поиск оптимального решения к~уравнениям относительно функций 
$\alpha_t$, $\beta_t(y)$ и~$\gamma_t(y)$. Отметим сразу, что явный вид 
функции~$\gamma_t(y)$ для реализации оптимального управ\-ле\-ния не 
требуется, однако далее будет предложен вариант вы\-чис\-ле\-ния и~этой 
функции, что пред\-став\-ля\-ет\-ся небесполезным, поскольку позволит выполнять 
расчет минимума целевого функционала. Источником для 
предложения~(\ref{e15-bos}) является уже упоминавшаяся аналогичная 
задача для случая дис\-крет\-но\-го времени~\cite{7-bos, 8-bos}. В~той задаче 
выражение для функции Беллмана получается формально без 
дополнительных усилий. При этом форма~(\ref{e15-bos}) обнаруживается 
как свойство оптимального решения. В~рассматриваемом случае 
непрерывного времени~(\ref{e15-bos}) постулируется, а~пра\-виль\-ность 
постулата под\-тверж\-да\-ет\-ся далее ре\-зуль\-ти\-ру\-ющи\-ми уравнениями 
для~$\alpha_t$, $\beta_t(y)$ и~$\gamma_t(y)$ Кроме того, данное 
предположение пред\-став\-ля\-ет\-ся вы\-те\-ка\-ющим из линейной структуры задачи 
в~отношении переменной~$z$, в~част\-ности, тем фактом, что такой вид 
функции Беллмана обеспечивает линейность оптимального 
управ\-ле\-ния~(\ref{e12-bos}) по~$z$.

     Граничное условие при выбранном предположении~(\ref{e15-bos}) 
принимает вид:

\noindent
     \begin{multline*}
     V_T(y,z)= S_T\left( s_T y- g_T z\right)^2+G_T z^2 ={}\\[-0.5pt]
     {}=\alpha_T z^2 
+\beta_T(y) z +\gamma_T(y)\,,
    \end{multline*}
т.\,е.

\noindent
\begin{align*}
\alpha_T&= S_T g_T^2 +G_T\,;\\[-0.5pt]
\beta_T(y)&=-2S_T s_T g_T y\,;\\[-0.5pt]
\gamma_T(y)&=S_T s_T^2 y^2\,.
%\label{e16-bos}
\end{align*}
          При этом само оптимальное управ\-ле\-ние, определенное 
выражением~(\ref{e12-bos}), оказывается управ\-ле\-ни\-ем с~обратной связью 
по~$y_t$ и~$z_t$:

\noindent
     \begin{multline}
     u_t^*=u_t^*(y,z) ={}\\[-0.5pt]
     {}=
     -\fr{1}{2}\left( S_t h_t^2 +H_t\right)^{-1}
     \left( c_t \left( 2\alpha_t z +\beta_t(y)\right) +{}\right.\\[-0.5pt]
    \left. {}+2S_t\left( s_t y-g_t z\right) 
h_t\right)\,.
     \label{e17-bos}
     \end{multline}
          Подставляем $V_t(y,z)\hm= \alpha_t z^2 \hm+ \beta_t(y) 
z\hm+\gamma_t(y)$ в~(\ref{e14-bos}):

\noindent
     \begin{multline*}
     \fr{\partial \alpha_t}{\partial t}\, z^2 +
     \fr{\partial \beta_t(y)}{\partial t}\,z +
     \fr{\partial \gamma_t(y)}{\partial t}+{}\\[-0.5pt]
     {}+\fr{1}{2}\left( \Sigma_t^2(y) \left( 
\fr{\partial^2\beta_t(y)}{\partial y^2}\,z +\fr{\partial^2 \gamma_t(y)}{\partial 
y^2}\right) +2\sigma_t^2\alpha_t\right)+{}\\[-0.5pt]
 {}+A_t(y)\left(\fr{\partial \beta_t(y)}{\partial y}\,z + \fr{\partial 
\gamma_t(y)}{\partial y}\right) +{}\\[-0.5pt]
\hspace*{-0.22987pt}{}+\left( a_t y+b_t z+\left( S_t h_t^2 +H_t\right)^{-1} c_t S_t \left( s_t y-
g_t z\right) h_t\right)\times{}
\end{multline*}

\noindent
\begin{multline*}
         {}\times \left( 2\alpha_t z+\beta_t(y)\right)+{}\\
     {}+\left( S_t-\left( S_t h_t^2 +H_t\right)^{-1} S_t^2 h_t^2\right)\left( s_t y-
g_t z\right)^2+{}\\
     {}+ G_t z^2 -\fr{1}{4}\left( S_t h_t^2 +H_t\right)^{-1} c_t^2 \left( 
2\alpha_t z+\beta_t(y)\right)^2=0\,.
     \end{multline*}
          Далее выделяем слагаемые при~$z^2$, $z$ и~$z^0$
          
          \noindent
     \begin{multline*}
     \fr{\partial \alpha_t}{\partial t}\, z^2 +\fr{\partial \beta_t(y)}{\partial t}\,z +
     \fr{\partial \gamma_t(y)}{\partial 
t}+\fr{1}{2}\,\Sigma_t^2(y)\fr{\partial^2\beta_t(y)}{\partial y^2}\,z+ {}\\
{}+
\fr{1}{2}\,\Sigma_t^2(y)\fr{\partial^2\gamma_t(y)}{\partial 
y^2}+\sigma_t^2\alpha_t+A_t(y)\fr{\partial \beta_t(y)}{\partial y}\,z +{}\\
{}+A_t(y) \fr{\partial 
\gamma_t(y)}{\partial y}+{}\\
{}+ 2\alpha_t \left( b_t -\left( S_t h_t^2+H_t\right)^{-1} c_t 
S_t h_t g_t \right)z^2+{}\\
     {}+
     \left( 2\alpha_t\left( \alpha_t+\left( S_t h_t^2+H_t\right)^{-1} c_t S_t h_t 
s_t\right)y +{}\right.\\
\left.{}+\beta_t(y) \left( b_t-\left( S_t h_t^2+H_t\right)^{-1} c_t S_t h_t 
g_t\right) \right) z+{}\\
     {}+\beta_t(y)\left( a_t +\left( S_t h_t^2+H_t\right)^{-1} c_t S_t h_t s_t\right) 
y+{}\\
{}+ \left( S_t -\left( S_t h_t^2+H_t\right)^{-1} S_t^2 h_t^2\right) g_t^2 z^2-{}\\
     {}- 2\left( S_t -\left( S_t h_t^2+H_t\right)^{-1} S_t^2 h_t^2\right) s_t g_t yz 
+{}\\
{}+
     \left( S_t-\left( S_t h_t^2+H_t\right)^{-1} S_t^2 h_t^2\right) s_t^2 y^2+{}\\
     {}+G_t z^2 -\left( S_t h_t^2 +H_t\right)^{-1} c_t^2 \alpha_t^2 z^2 -{}\\
     {}-\left( 
S_t h_t^2+H_t\right)^{-1} c_t^2 \alpha_t \beta_t(y) z-{}\\
{}-
\fr{1}{4}\left( S_t h_t^2+H_t\right)^{-1}  c_t^2 \beta_t^2(y)=0\,,
     \end{multline*}
группируем их и~получаем сле\-ду\-ющие уравнения:
\begin{itemize}
\item  для~$\alpha_t$:

\noindent
\begin{multline}
\fr{\partial\alpha_t}{\partial t}+2\alpha_t\left( b_t-\left( S_t h_t^2+H_t\right)^{-1} c_t 
S_t h_t g_t\right)+{}\\
{}+ \left( S_t- \left( S_t h_t^2+H_t\right)^{-1} S_t^2 h_t^2\right) 
g_t^2+G_t-{}\\
\hspace*{-8mm}{}-\left( S_t h_t^2+H_t\right)^{-1} c_t^2 \alpha_t^2 =0\,,\enskip \alpha_T=S_T 
g_t^2+G_T\,;\!\!
\label{e18-bos}
\end{multline}
\item для $\beta_t$:

\noindent
\begin{multline}
\fr{\partial\beta_t(y)}{\partial 
t}+\fr{1}{2}\,\Sigma_t^2(y)\fr{\partial^2\beta_t(y)}{\partial y^2} 
+A_t(y)\fr{\partial \beta_t(y)}{\partial y}+{}\\
{}+ 2\alpha_t\left( a_t +\left( S_t h_t^2+H_t\right)^{-1} c_t S_t h_t s_t\right) y+{}\\
{}+
\beta_t(y)\left( b_t -\left( S_t h_t^2 +H_t\right)^{-1} c_t S_t h_t g_t\right)-{}\\
{}-2\left( S_t-\left( S_t h_t^2+H_t\right)^{-1} S_t^2 h_t^2\right) s_t g_t y-{}
\\
{}-
\left( S_t h_t^2+H_t\right)^{-1} c_t^2 \alpha_t \beta_t(y)=0\,,\\
\beta_T(y)=-2S_T s_T g_T y\,;
\label{e19-bos}
\end{multline}
\item  для $\gamma_t$:
\begin{multline}
\hspace*{-0.8pt}\fr{\partial \gamma_t(y)}{\partial t}+\fr{1}{2}\,\Sigma_t^2(y)
\fr{\partial^2 \gamma_t(y)}{\partial y^2} +\sigma_t^2 \alpha_t +A_t(y)
\fr{\partial \gamma_t(y)}{\partial y}+{}\\
{}+ \beta_t(y)\left( a_t +\left( S_t h_t^2+H_t\right)^{-1} c_t S_t h_t s_t\right) y+{}\\
{}+
\left( S_t-\left( S_t h_t^2+H_t\right)^{-1} S_t^2 h_t^2\right)  s_t^2 y^2-{}\\
{}-\fr{1}{4}\left( S_t h_t^2+H_t\right)^{-1} c_t^2 \beta_t^2(y) =0\,,\\
\gamma_T(y)=S_T s_T^2 y^2\,.
\label{e20-bos}
\end{multline}
\end{itemize}
     
     Уравнение~(\ref{e18-bos}), легко заметить, является уравнением 
Риккати, которое в~силу сформулированного выше условия   
имеет единственное неотрицательное решение для всех $0\hm\leq t\hm\leq T$. 
Этот факт требует дополнительного комментария. Для получения 
уравнения~(\ref{e18-bos}) рас\-смот\-рим исходную задачу при дополнительных 
условиях $a_t\hm=0$ и~$s_t\hm=0$ для всех $0\hm\leq t\hm\leq T$. Нетрудно 
видеть, что эти условия рассматриваемую по\-ста\-нов\-ку сводят фактически 
к~классической ли\-ней\-но-квад\-ра\-тич\-ной задаче. Имеющуюся 
в~рассматриваемой формулировке чуть более общую форму целевой 
функции (принципиального значения это обобщение, конечно, не имеет) 
сведем к~классической еще одним предположением: $S_t\hm=0$ для всех 
$0\hm\leq t\hm\leq T$. Теперь уравнение для~$\alpha_t$ принимает хорошо 
известный вид:
     \begin{equation}
     \fr{\partial \alpha_t}{\partial t}+2\alpha_t b_t +G_t- H_t^{-1} c_t^2 
\alpha_t^2=0\,,\enskip \alpha_T=G_T\,.
     \label{e21-bos}
     \end{equation}

     В таком случае, как известно~\cite{10-bos}, существует единственное 
оптимальное управление~--- линейное с~обратной связью по выходу~$z_t$, 
с~коэффициентом усиления, опи\-сы\-ва\-емым уравнением  
Риккати~(\ref{e21-bos}). Именно этот результат дают  
уравнения~(\ref{e18-bos})--(\ref{e20-bos}) и~описываемая ими функция 
Беллмана~(\ref{e15-bos}), так как из $a_t\hm=0$ и~$s_t\hm=0$ немедленно 
следует, что $\beta_t(y)\hm=0$, откуда, в~свою очередь, с~учетом 
не\-за\-ви\-си\-мости решения от~$y_t$ следует, что $\gamma_t(y)\hm=\gamma_t$, 
т.\,е.\ не зависит от~$y$ и~задается уравнением: 
     $$
     \fr{\partial \gamma_t(y)}{\partial t} +\sigma^2_t \alpha_t=0\,,\enskip 
\gamma_T=0\,.
     $$ 
     Оптимальное управ\-ле\-ние при этом 
     $$
     u_t^*= -H_t^{-1} c_t \alpha_t z_t\,,
     $$
      т.\,е.\ все полностью совпадает с~известным классическим решением.
     
     С уравнениями~(\ref{e19-bos}) и~(\ref{e20-bos}) ситуация, естественно, 
обстоит сложнее. Это линейные уравнения второго порядка параболического 
типа, поскольку\linebreak
 $\Sigma_t^2(y)\hm>0$. Фактически отсутствуют 
конструктивные условия, гарантирующие существование их\linebreak
 решений 
(требовать, чтобы все фигурирующие в~уравнениях коэффициенты были 
представлены аналитическими функциями на всем пространстве значений, 
вряд ли целесообразно), поэтому далее будем предполагать, что данные 
уравнения имеют на рас\-смат\-ри\-ва\-емом интервале $0\hm\leq t\hm\leq T$ хотя 
бы одно ограниченное решение и~именно эти условия будем рас\-смат\-ри\-вать 
как достаточные условия существования оптимального решения 
рассматриваемой задачи.
     
     Таким образом, доказана следующая тео\-рема.
     
     \smallskip
     
     \noindent
     \textbf{Теорема.}\ \textit{Пусть для диффузионного 
процесса}~(\ref{e5-bos}) \textit{выполнены условия Ито, для 
     процесса}~(\ref{e6-bos})~--- \textit{ограничены коэффициенты, 
уравнения}~(\ref{e18-bos})--(\ref{e20-bos}) \textit{имеют ограниченные 
решения для $0\hm\leq t\hm\leq T$. Тогда минимум  
функционалу}~(\ref{e7-bos}) \textit{доставляет оптимальное 
управ\-ле\-ние}~(\ref{e17-bos}), \textit{где} $y\hm= y_t$; $z\hm=z_t$.
     
\section{Заключение}

     Рассмотренная задача оптимизации в~целом близка и~по модели, и~по 
критерию к~классической ли\-ней\-но-квад\-ра\-тич\-ной постановке. 
Принципиальным отличием является нелинейная модель для описания 
со\-сто\-яния динамической сис\-те\-мы, в~которой отсутствует управ\-ля\-ющее 
воздействие.\linebreak
 Такую модель наряду с~традиционной интер\-пре\-тацией  
<<со\-сто\-яние--вы\-ход>> мож\-но понимать как\linebreak модель неконтролируемого 
слож\-но\-го внешнего воздействия. Небольшое дополнительное отличие дает 
предложенная форма квад\-ра\-тич\-но\-го критерия, поз\-во\-ля\-ющая, в~част\-ности, 
ставить такие задачи, как отслеживание выходом или управ\-ле\-ни\-ем со\-сто\-яния 
сис\-те\-мы или ее выхода.
     
     Поскольку обсуждать возможности точного решения уравнений, 
определяющих оптимальное управ\-ле\-ние, не имеет смыс\-ла, наиболее 
актуальной далее является задача их приближенного чис\-лен\-но\-го решения 
и~анализа воз\-мож\-ности практической реализации. Этому посвящена вторая 
часть данной работы, пла\-ни\-ру\-емая к~выходу в~ближайшее время.

{\small\frenchspacing
 {%\baselineskip=10.8pt
 \addcontentsline{toc}{section}{References}
 \begin{thebibliography}{99}
\bibitem{1-bos}
\Au{Athans M.} Editorial on the LQG problem~// IEEE~T. Automat. Contr., 1971. Vol.~16. 
No.\,6. P.~528--552. doi: 10.1109/TAC.1971.1099845.
\bibitem{2-bos}
\Au{Wu Z.} Forward-backward stochastic differential equations, linear quadratic stochastic 
optimal control and nonzero sum differential games~// J.~Syst. Sci. Complex., 2005. Vol.~18. 
No.\,2. P.~179--192.
\bibitem{3-bos}
\Au{Chen B.\,S., Zhang~W.} Stochastic H2/H1 control with state-dependent noise~// IEEE 
T.~Automat. Contr., 2004. Vol.~49. No.\,1. P.~45--56. doi: 10.1109/TAC.2003.821400.
\bibitem{4-bos}
\Au{Bohacek S.} A~stochastic model of TCP and fair video transmission~// IEEE 
INFOCOM, 2003. Vol.~2. P.~1134--1144. doi: 10.1109/INFCOM.2003.1208950.
\bibitem{5-bos}
\Au{Домбровский В.\,В., Объедко~Т.\,Ю.} Управление с~прогнозированием системами 
с~марковскими скачками при ограничениях и~применение к~оптимизации 
инвестиционного портфеля~// Автомат. телемех., 2011. №\,5. С.~96--112. doi: 
10.1134/S0005117911050079.
\bibitem{6-bos}
\Au{Баландин Д.\,В., Коган~М.\,М.} Оптимальное линейно-квад\-ра\-тич\-ное управление: от 
матричных уравнений к~линейным матричным неравенствам~// Автомат. телемех., 2011. 
№\,11. С.~60--69. doi: 10.1134/ S0005117911110038.
\bibitem{7-bos}
\Au{Босов А.\,В.} Обобщенная задача распределения ресурсов программной системы~// 
Информатика и~её применения, 2014. Т.~8. Вып.~2. С.~39--47. doi: 
10.14357/19922264140204.
\bibitem{8-bos}
\Au{Босов А.\,В.} Управление линейным выходом дискретной стохастической системы по 
квадратичному критерию~// Изв. РАН. Теория и~системы управления, 2016. №\,3.  
С.~19--35. doi: 10.1134/S1064230716030060.
\bibitem{9-bos}
\Au{Флеминг У., Ришел~Р.} Оптимальное управление детерминированными 
и~стохастическими системами~/ Пер. с~англ.~--- М.: Мир, 1978. 316~с. 
(\Au{Fleming~W.\,H., Rishel~R.\,W.} Deterministic and stochastic optimal control.~--- New 
York, NY, USA: Springer-Verlag, 1975. 222~p.)
\bibitem{10-bos}
\Au{Девис М.\,Х.\,А.} Линейное оценивание и~стохастическое управление~/ Пер. с~англ.~--- 
М.: Наука, 1984. 206~с. (\Au{Davis~M.\,H.\,A.} Linear estimation and stochastic control.~--- 
London: Chapman and Hall, 1977. 224~p.)

 \end{thebibliography}

 }
 }

\end{multicols}

\vspace*{-6pt}

\hfill{\small\textit{Поступила в~редакцию 30.03.18}}

\vspace*{4pt}

%\newpage

%\vspace*{-24pt}

\hrule

\vspace*{2pt}

\hrule

\vspace*{-2pt}


\def\tit{STOCHASTIC DIFFERENTIAL SYSTEM OUTPUT CONTROL 
BY~THE~QUADRATIC CRITERION.~I.~DYNAMIC\\ PROGRAMMING 
OPTIMAL SOLUTION}


\def\titkol{Stochastic differential system output control 
by~the~quadratic criterion. I.~Dynamic programming 
optimal solution}

\def\aut{A.\,V.~Bosov and~A.\,I.~Stefanovich}

\def\autkol{A.\,V.~Bosov and~A.\,I.~Stefanovich}

\titel{\tit}{\aut}{\autkol}{\titkol}

\vspace*{-11pt}


\noindent
Institute of Informatics Problems, Federal Research Center ``Computer Science 
and Control'' of the Russian Academy of Sciences, 44-2~Vavilov Str., Moscow 
119333, Russian Federation


\def\leftfootline{\small{\textbf{\thepage}
\hfill INFORMATIKA I EE PRIMENENIYA~--- INFORMATICS AND
APPLICATIONS\ \ \ 2018\ \ \ volume~12\ \ \ issue\ 3}
}%
 \def\rightfootline{\small{INFORMATIKA I EE PRIMENENIYA~---
INFORMATICS AND APPLICATIONS\ \ \ 2018\ \ \ volume~12\ \ \ issue\ 3
\hfill \textbf{\thepage}}}

\vspace*{3pt}



\Abste{The problem of optimal control for the Ito diffusion 
process and a~controlled linear output is solved. The considered 
statement is close to the classical linear-quadratic Gaussian 
control  (LQG control) problem. Differences consist in the fact 
that the state is described by the nonlinear differential Ito equation  $dy_y = A_t(y_t) 
\,dt+\Sigma_t(y_t)\,dv_t$ and does not depend on the control~$u_t$, 
optimization subject is controlled linear output 
 $dz_t=a_ty_t\,dt +b_tz_t\,dt +c_t u_t\,dt +\sigma_t \,dw_t$. 
Additional generalizations are included in the quadratic 
quality criterion for the purpose of statement such problems 
as state tracking by output or a linear combination of state 
and output tracking by control. The method of dynamic programming 
is used for the solution. 
The assumption about Bellman function in the form  $V_t(y,z)= \alpha_t 
z^2+\beta_t(y) z+\gamma_t(y)$ allows one to find it. 
Three differential equations for the coefficients $\alpha_t$,  $\beta_t(y)$,
and $\gamma_t(y)$ give the solution. 
These equations constitute the optimal solution of the problem under consideration.}

\KWE{stochastic differential equation; optimal control; dynamic programming; 
Bellman function; Riccati equation; linear differential equations of parabolic type}


\DOI{10.14357/19922264180314}

\vspace*{-12pt}

\Ack
\noindent
This work was partially supported by the Russian Science Foundation (grant  
16-07-00677).



%\vspace*{6pt}

  \begin{multicols}{2}

\renewcommand{\bibname}{\protect\rmfamily References}
%\renewcommand{\bibname}{\large\protect\rm References}

{\small\frenchspacing
 {%\baselineskip=10.8pt
 \addcontentsline{toc}{section}{References}
 \begin{thebibliography}{99}
\bibitem{1-bos-1}
\Aue{Athans, M.} 1971. Editorial on the LQG problem. \textit{IEEE~T. 
Automat. Contr.} 16(6):528--552. doi: 10.1109/ TAC.1971.1099845.
\bibitem{2-bos-1}
\Aue{Wu, Z.} 2005. Forward-backward stochastic differential equations, linear 
quadratic stochastic optimal control and\linebreak\vspace*{-12pt}

\columnbreak

\noindent
 nonzero sum differential games. 
\textit{J.~Syst. Sci. Complex.} 18(2):179--192.
\bibitem{3-bos-1}
\Aue{Chen, B.\,S. and W.~Zhang.} 2004. Stochastic H2/H1 control with  
state-dependent noise. \textit{IEEE~T. Automat. Contr.} 49(1):45--56.
doi: 10.1109/TAC.2003.821400.
\bibitem{4-bos-1}
\Aue{Bohacek, S.} 2003. A~stochastic model of TCP and fair video 
transmission. \textit{IEEE INFOCOM}. 2:1134--1144.
doi: 10.1109/INFCOM.2003.1208950.
\bibitem{5-bos-1}
\Aue{Dombrovskii, V.\,V., and T.\,Yu.~Ob''edko.} 2011. Predictive control of 
systems with Markovian jumps under constraints and its application to the 
investment portfolio optimization. \textit{Automat. Rem. Contr.}  
72(5):989--1003.
\bibitem{6-bos-1}
\Aue{Balandin, D.\,V., and M.\,M.~Kogan.} 2011. Optimal linear-quadratic 
control: From matrix equations to linear matrix inequalities. \textit{Automat. 
Rem. Contr.} 72(11):2276--2284.
\bibitem{7-bos-1}
\Aue{Bosov, A.\,V.} 2014. Obobshchennaya zadacha raspredeleniya resursov 
programmnoy sistemy [The generalized problem of software system resources 
distribution]. \textit{Informatika i~ee Primeneniya~--- Inform. Appl.}  
8(2):39--47. doi: 
10.14357/19922264140204.
\bibitem{8-bos-1}
\Aue{Bosov, A.\,V.} 2016. Discrete stochastic system linear output control 
with respect to a quadratic criterion. \textit{J.~Comput. Syst. Sc. 
Int.} 55(3):349--364.
\bibitem{9-bos-1}
\Aue{Fleming, W.\,H., and R.\,W.~Rishel.} 1975. \textit{Deterministic and 
stochastic optimal control.} New York, NY: Springer-Verlag. 222~p.
\bibitem{10-bos-1}
\Aue{Davis, M.\,H.\,A.} 1977. \textit{Linear estimation and stochastic 
control.} London: Chapman and Hall. 224~p.
\end{thebibliography}

 }
 }

\end{multicols}

\vspace*{-6pt}

\hfill{\small\textit{Received March 30, 2018}}

%\pagebreak

%\vspace*{-18pt}
     
     \Contr
     
       \noindent
       \textbf{Bosov Alexey V.} (b.\ 1969)~--- Doctor of Science in technology, 
principal scientist, Institute of Informatics Problems, Federal Research 
Center ``Computer Science and Control'' of the Russian Academy of Sciences, 
44-2~Vavilov Str., Moscow 119333, Russian Federation; 
\mbox{AVBosov@ipiran.ru}
       
       \vspace*{3pt}
       
       \noindent
       \textbf{Stefanovich Alexey I.} (b.\ 1983)~--- principal specialist, 
Institute of Informatics Problems, Federal Research Center ``Computer Science 
and Control'' of the Russian Academy of Sciences, 44-2~Vavilov Str., Moscow 
119333, Russian Federation; \mbox{AStefanovich@frccsc.ru}
\label{end\stat}

\renewcommand{\bibname}{\protect\rm Литература}       

        %1
%\newcommand {\ff}{{\mathcal F}}
\newcommand {\ebd}{\triangleq}
\newcommand{\me}[2]{\mathbf{E}_{ #1 }\left\{ \mathop{#2} \right\} }



\def\stat{borisov}

\def\tit{ФИЛЬТРАЦИЯ СОСТОЯНИЙ МАРКОВСКИХ СКАЧКООБРАЗНЫХ ПРОЦЕССОВ 
ПО~ДИСКРЕТИЗОВАННЫМ НАБЛЮДЕНИЯМ$^*$}

\def\titkol{Фильтрация состояний марковских скачкообразных процессов 
по~дискретизованным наблюдениям}

\def\aut{А.\,В.~Борисов$^1$}

\def\autkol{А.\,В.~Борисов}

\titel{\tit}{\aut}{\autkol}{\titkol}

\index{Борисов А.\,В.}
\index{Borisov A.\,A.}




{\renewcommand{\thefootnote}{\fnsymbol{footnote}} \footnotetext[1]
{Работа выполнена при частичной поддержке РФФИ (проект 16-07-00677).}}


\renewcommand{\thefootnote}{\arabic{footnote}}
\footnotetext[1]{Институт проблем информатики Федерального исследовательского центра <<Информатика 
и~управление>> Российской академии наук,
\mbox{aborisov@frccsc.ru}}

%\vspace*{8pt}



\Abst{Статья посвящена решению задачи оптимальной 
фильтрации состояний однородного марковского скачкообразного процесса (МСП). 
Наблюдения представляют собой приращения случайных процессов~--- интегральных 
преобразований состояний, зашумленные винеровскими процессами, интенсивность 
которых также зависит от оцениваемого состояния. Оптимальная оценка в~моменты 
получения нового наблюдения вычисляется как функция предыдущей оценки и~новых 
наблюдений, а~между моментами наблюдений~--- простейшим прогнозом в~силу системы 
уравнений Колмогорова. Рекуррентная формула пересчета ресурсозатратна, так как 
содержит  интегралы~--- мас\-штаб\-но-сдви\-го\-вые смеси многомерных гауссиан, 
где в~качестве смешивающих выступают распределения времени пребывания 
состояния в~каждом из возможных значений. Предложены более простые аппроксимации, 
основанные на предположении об ограниченности числа скачков состояния за время между 
наблюдениями. Получены универсальные локальная и~глобальная характеристики точности 
аппроксимаций, зависящие от па\-ра\-мет\-ров оцениваемого процесса, величины 
временн$\acute{\mbox{о}}$го шага  между наблюдениями и~максимального числа учитываемых скачков.}

\KW{марковский скачкообразный процесс; оптимальная фильтрация; мультипликативные 
шумы в~наблюдениях; стохастическое дифференциальное уравнение; численная аппроксимация}

\DOI{10.14357/19922264180316}
  
%\vspace*{4pt}


\vskip 10pt plus 9pt minus 6pt

\thispagestyle{headings}

\begin{multicols}{2}

\label{st\stat}



 \section{Введение}
 
 Фильтр Вонэма~\cite{Won_65}~--- один из редких удачных случаев, когда 
 оценка оптимальной фильтрации состо\-яния стохастической системы наблюдения 
 выражается в~виде решения некоторой замк\-ну\-той\linebreak конечномерной сис\-те\-мы 
 стохастических дифференциальных уравнений. 
 
 Алгоритм данного фильт\-ра 
 позволяет вычислить оценку фильт\-ра\-ции со\-сто\-яния \textit{марковского скачкообразного 
 процесса} с~\mbox{конечным} множеством состояний по наблюдениям в~присутствии 
 аддитивных винеровских шумов. Теоретически оптимальная оценка со\-сто\-яния~--- 
 его условное распределение в~текущий момент времени~--- 
 обладает очевидными свойствами неотрицательности и~нормировки. 
 При чис\-лен\-ной реализации данного фильтра классическим методом 
 Эй\-ле\-ра--Ма\-ру\-ямы~\cite{KP_92} данные свойства могут не сохраняться и~процедура 
 вы\-чис\-ле\-ния становится неустойчивой.  В~связи с~этим обстоятельством разрабатывались 
 другие алгоритмы чис\-лен\-но\-го решения уравнения фильтра Вонэма, обладающие 
 требуемыми свойствами устойчивости (см.~\cite{YZL_04, PR_10} и~библиографию в~них). 
 В~час\-ти этих работ доказана лишь слабая сходимость пред\-ла\-га\-емых аппроксимационных 
 схем к~оценке фильт\-ра Вонэма, в~то время как ка\-кая-ли\-бо 
 характеризация точ\-ности этих приближений отсутствует.
 
 В~\cite{B_18} было представлено распространение фильт\-ра Вонэма на случай 
 наблюдений с~мультипликативными шумами. При этом уравнение обобщенного 
 фильт\-ра содержит в~правой части квадратическую характеристику шумов в~наблюдениях. 
 Данный процесс на практике никогда не наблюдается непосредственно, а~является лишь 
 некоторым нелинейным интегральным преобразованием наблюдений. Очевидно, что 
 имеющиеся в~настоящий момент времени алгоритмы приближенного вычисления оценки 
 фильтрации Вонэма для данной системы не подходят. 
 
 Целью предлагаемой работы является ис\-поль\-зование результатов оптимальной 
 фильтрации со\-стояний сис\-тем с~дискретным временем для аппроксимации решения 
 аналогичной задачи для\linebreak стохастических дифференциальных сис\-тем. 
 
 Статья организована следующим образом. Раздел~2 содержит формальную постановку 
 задачи фильт\-ра\-ции со\-сто\-яний однородного МСП с~конечным множеством со\-сто\-яний 
 по наблюдениям, полученным путем временн$\acute{\mbox{о}}$й дискретизации процессов с~непрерывным 
 временем~--- интегральных преобразований со\-сто\-яния сис\-те\-мы в~присутствии 
 мультипликативных винеровских шумов.\linebreak
  В~разд.~3 пред\-став\-ле\-но решение поставленной 
 задачи фильт\-ра\-ции: пересчет оценок со\-сто\-яний в~момент получения новых 
 дискретизованных наблюдений выполняется в~соответствии с~некоторыми\linebreak 
 рекуррентными интегральными соотношениями, в~то время как между 
 моментами наблюдений оценка корректируется в~соответствии с~прогнозом в~силу 
 сис\-те\-мы уравнений Колмогорова. Вы\-чис\-ли\-тель\-ная слож\-ность 
 упомянутых выше интегральных\linebreak 
 соотношений связана с~тем, что в~расчет принимается воз\-мож\-ность того, что между 
 моментами наблюдений оцениваемое со\-сто\-яние может совершить произвольное чис\-ло 
 скачков. В~разд.~4 пред\-став\-лен более простой алгоритм приближенного вы\-чис\-ле\-ния 
 оценки фильт\-ра\-ции, основанный на ограничении возможного числа учитываемых скачков 
 МСП. Доказана тео\-ре\-ма, опре\-де\-ля\-ющая как\linebreak
  локальную (одношаговую), так и~глобальную 
 (многошаговую) характеристики точ\-ности предложенного при\-бли\-же\-ния~--- 
 $\ell_1$-нор\-мы ошибки аппроксимации. Полученные характеристики являются\linebreak 
 универсальными, т.\,е.\ не асимптотическими по шагу дискретизации, и~зависят от характеристик 
 самого МСП, %\linebreak
  шага временн$\acute{\mbox{о}}$й дискретизации и~чис\-ла
  скачков со\-сто\-яния, учи\-ты\-ва\-емых 
 на шаге. Об\-суж\-де\-ние результатов и~заключительные комментарии пред\-став\-ле\-ны 
 в~разд.~5.
 
 \section{Постановка задачи фильтрации}
 
 На полном вероятностном пространстве с~фильт\-ра\-цией 
 $(\Omega,\mathcal{F},\mathcal{P},\{\mathcal{F}_{t}\}_{t \geqslant 0})$ рассматривается система наблюдений
\begin{equation}
 \left.
 \begin{array}{rl}
 \displaystyle X_t &=X_0 +  \displaystyle
 \int\limits_0^t \Lambda^{\top}X_{s}\,ds + \mu_s\,;  \\[6pt]
 \displaystyle Y_k &=  \displaystyle\int\limits_{t_{k-1}}^{t_k}fX_s\,ds+
 \int\limits_{t_{k-1}}^{t_k} 
 \sum\limits_{n=1}^NX_s^ng_n \,dW_s, \\[6pt]
 &\hspace*{10mm}\{t_k\}_{k \geqslant 0}: \; 0 = t_0 < t_1 < t_2\cdots,
 \end{array}
 \right\}
 \label{eq:obsys_1}
 \end{equation}
 где
  \begin{itemize}
  \item
  $X_t \ebd \mathrm{col}\left(X_t^1,\ldots,X_t^N\right) \hm\in \mathbb{S}^N$~--- 
  ненаблюда\-емое состояние системы, являющееся однородным МСП с~конечным 
  множеством состояний $ \mathbb{S}^N \ebd$\linebreak $\ebd \{e_1,\ldots,e_N\}$ ($\mathbb{S}^N$~--- 
  множество единичных векторов евклидова пространства~$\mathbb{R}^N$), 
  матрицей интенсивностей переходов~$\Lambda$ и~начальным распределением~$\pi$;
  \item
  $\mu_t \ebd \mathrm{col}\left(
  \mu_t^1,\ldots,\mu_t^N\right)\hm\in \mathbb{R}^N$~--- 
  ${\mathcal{F}}_t$-со\-гла\-со\-ван\-ный мартингал;
  \item
  $\{Y_k\}_{k \in \mathbb{N}}:\;  Y_k \ebd \mathrm{col}\left(Y_k^1,\ldots,Y_k^M\right) 
  \hm\in \mathbb{R}^M$~--- последовательность дискретизованных наблюдений, 
  доступных в~известные неслучайные  моменты времени~$\{t_k\}_{k \in \mathbb{N}}$,
в~которых $W_t \ebd$\linebreak $\ebd \mathrm{col}\left(W_t^1,\ldots,W_t^M\right) \hm\in \mathbb{R}^M$
 является ${\mathcal{F}}_t$-со\-гла\-со\-ван\-ным стандартным винеровским процессом, 
 определяющим шумы в~наблюдениях,\linebreak  $f$~--- $(M \times N)$-мер\-ная 
 мат\-ри\-ца плана наблюдений, а~набор мат\-риц~$\{g_n\}_{n=\overline{1,N}}$ 
 характеризует интенсивности шумов в~зависимости от текущего состояния~$X_t$.
  \end{itemize}
  
  Введем также в~рассмотрение неубывающие семейства $\sigma$-ал\-гебр 
  $\mathcal{O}_k \ebd \sigma\{ Y_{\ell}: \; 1 \hm\leqslant \ell \hm\leqslant k\}$ 
  и~$\mathcal{O}_t \ebd  \mathcal{O}_{k(t)}$, где 
  $k(t) \ebd \sum\nolimits_{j \in \mathbb{N}}\mathbf{I}(t-t_{j})$; 
  $\mathcal{O}_0 \ebd \{\varnothing,\; \Omega\}$.
  
   \textit{Задача оптимальной фильтрации состояния~$X$ по наблюдениям~$Y$} 
   заключается в~нахождении \textit{условного математического ожидания} (УМО)
  \begin{equation*}
  \widehat{X}_t \ebd {\sf E}\left\{X_t|\mathcal{O}_{t} \right\}\,.
 % \label{eq:fest_1}
  \end{equation*}
  
  Относительно системы~(\ref{eq:obsys_1})  сделаны следующие предположения:
   \begin{itemize}
 \item[(а)]
 ${\mathcal{F}}_t \equiv {\mathcal{F}}_{t}^X \bigvee 
 {\mathcal{F}}_{t}^W $ для любого $t \hm\geqslant 0$;
 \item[(б)]
 шумы в~наблюдениях равномерно невырожденные, т.\,е.\
  $g_ng_n^{\top} \hm\geqslant \alpha I \hm> 0$ для всех $n\hm=\overline{1,N}$ 
  и~некоторого $\alpha\hm>0$.
% \item
 % Верно неравенство
  %\begin{equation}
  %\min_{1\leqslant k \leqslant N}|\lambda_{kk}| > 0.
  %\label{eq:ineq_0}
  % \end{equation}
 %\item
 %Для любого $t \geqslant 0$ все компоненты вектора $p_t \ebd \me{}{X_t}$ строго %положительны. 
 \end{itemize} 

 \section{Уравнения оптимального фильтра} 
 
 Для получения уравнений оптимального фильт\-ра воспользуемся подходом, 
 применяемым для решения аналогичной задачи в~стохастических сис\-те\-мах 
 наблюдения с~дискретным временем~\cite{BSh_85}. 
 Воспользу\-ем\-ся методом математической индукции. 
 
 При $r=0$ 
 \begin{equation}
 \widehat{X}_{t_0}={\sf E}\{X_0|\mathcal{O}_0\}={\sf E}\{X_0\}=\pi\,.
 \label{eq:in_cond}
 \end{equation} 
 
 Пусть для некоторого $ r \hm\geqslant 0$ известна оценка оптимальной 
 фильтрации~$\widehat{X}_{t_r} \hm= {\sf E}{X_{t_r} |\mathcal{O}_r}$. 
 Определим оценку оптимальной фильтрации~$\widehat{X}_{t} $ для $t\hm \in (t_r,t_{r+1}]$. 
 
 Для произвольного момента $t \hm\in (t_r,t_{r+1})$ в~силу мартингального 
 разложения МСП~$X_t$ и~свойств УМО верна следующая цепочка равенств:
 \begin{multline*}
 \widehat{X}_{t} = {\sf E}\left\{X_t | \mathcal{O}_r\right\}={}\\
 {}=
 {\sf E}\left\{{\cal P}^{\top}(t_r,t)X_{t_r}+
 \int\limits_{t_r}^t{\cal P}^{\top}(t_r,s)\,dM_s\big\vert \mathcal{O}_r\right\} = {}
\end{multline*}

\noindent
   \begin{multline}
 \hspace*{-11.66pt}{}=\mathcal{P}^{\top}(t_r,t)\widehat{X}_{t_r} + {\sf E}\hspace*{-2pt}
 \left\{{\sf E}\hspace*{-2pt}\left\{\int\limits_{t_r}^t\hspace*{-2pt}\mathcal{P}^{\top}(t_r,s)\,dM_s |
 {\mathcal{F}}_{t_r}\right\}\!\big\vert 
 \mathcal{O}_r\!\right\} ={}\hspace*{-4.24124pt}\\
 {}=
  \mathcal{P}^{\top}(t_r,t)\widehat{X}_{t_r}\,,
 \label{eq:bw_obs}
 \end{multline}
 где $\mathcal{P}(s,t)$ $(s \hm\leqslant t)$~--- матрица переходной ве\-ро\-ят\-ности МСП 
 на промежутке $[s,t]$, являющаяся решением сис\-те\-мы дифференциальных 
 уравнений Колмогорова
 \begin{equation*}
 \mathcal{P}'_t(s,t) = \mathcal{P}(s,t) \Lambda, \enskip t > s, \enskip \mathcal{P}(s,s) = I.
 \end{equation*}
 В случае однородного МСП $\mathcal{P}(s,t) \hm= e^{(t-s)\Lambda}$.
 
 Далее необходимо определить совместное распределение $(X_{t_{r+1}},Y_{r+1})$ 
 относительно~$ \mathcal{O}_r$. Из модели наблюдений следует, что 
 распределение~$Y_{r+1}$ относительно 
 $\sigma$-ал\-геб\-ры~$\mathcal{F}^X_{t_{r+1}} \vee \mathcal{O}_r$~---
 гауссовское с~параметрами 
 \begin{align*}
{\sf E}\left\{Y_{r+1}|{\mathcal{F}}^X_{t_{r+1}}\right\}& = f \tau_{r+1}\,; \\[6pt]
 \mathrm{cov} \left(Y_{r+1},Y_{r+1}|{\mathcal{F}}^X_{t_{r+1}}\right) &= 
 \displaystyle\sum\limits_{n=1}^N \tau_{r+1}^n g_ng_n^{\top}\,,
% \label{eq:occup_1}
 \end{align*}
 где $\tau_{r+1} \hm= \tau_{r+1}(X(\omega))=
 \mathrm{col}\left(\tau_{r+1}^1,\ldots,\tau_{r+1}^N\right) \ebd$\linebreak
 $\ebd 
 \int\nolimits_{t_r}^{t_{r+1}}X_s\,ds$~--- случайный вектор, $n$-я 
 компонента которого равна времени пребывания процесса~$X$ в~со\-сто\-янии~$e_n$ 
 на  интервале времени $[t_r, t_{r+1}]$. 
 Обозначим через $\mathcal{D}_{r+1} \ebd \{u=\mathrm{col}\,(u^1,\ldots,u^N):\; 
 u_m \hm\geqslant 0,\; \sum\nolimits_{m=1}^Mu_m\hm= t_{r+1}-t_r\}$ $(M-1)$-мер\-ный 
 симплекс в~пространстве~$\mathbb{R}^M$, являющийся носителем распределения 
 вектора~$\tau_{r+1}$. Пусть $\rho^{k,\ell}_{r+1}(du)$~--- 
 распределение вектора $\tau_{r+1} X_{t_{r+1}}^{\ell}$ при условии $X_{t_r}\hm=e_k$, 
 т.\,е.\ 
 для любого $\mathcal{A} \hm\in \mathcal{B}(\mathbb{R}^M)$ верно тождество:
\begin{multline*}
 \mathbf{P}\left\{\omega: \; X_{t_{r+1}}(\omega)=e_{\ell},\right.\\
 \left. 
 \tau_{r+1}(X(\omega)) \in \mathcal{A}\;|\;X_{t_r}=e_k\right\} \equiv
   \rho^{k,\ell}_{r+1}(\mathcal{A})\,.
\end{multline*}
 
Обозначим через
\begin{multline*}
 \mathcal{N}(y,m,K) \ebd (2\pi)^{-M/2} \mathrm{ det}^{-1/2} K\times{}\\
 {}\times\exp
 \left\{ -\fr{1}{2}\left(y-m\right)^{\top}K^{-1}(y-m)\right\}
\end{multline*}
 $M$-мер\-ную плот\-ность гауссовского распределения с~математическим 
 ожиданием~$m$ и~ковариационной матрицей~$K$.
 
 Из марковского свойства  $\{X_{t_{r}},Y_{r})\}_{r \geqslant 0}$ 
 относительно~${\mathcal{F}}_{t_{r}}$~\cite{ZhSh_95} и~теоремы Фубини следует, что 
 для любого  множества $\mathcal{A} \hm\in \mathcal{B}(\mathbb{R}^M)$ 
 верна следующая цепочка равенств:
 \begin{multline*}
 {\sf E}\left\{X_{t_{r+1}}\mathbf{I}_{\mathcal{A}}
 \left(Y_{r+1}\right)\big|\mathcal{O}_r\right\}={}\\
 {}=
{\sf E}\left\{{\sf E}\left\{X_{t_{r+1}}\mathbf{I}_{\mathcal{A}}
\left(Y_{r+1}\right)\big|
\mathcal{F}^X_{t_{r+1}} \vee \mathcal{O}_r\right\}
 \big|\mathcal{O}_r\right\} = {}
\end{multline*}

\noindent
\begin{multline*}
 %{}=
% {\sf E}\left\{{\sf E}\left\{X_{t_{r+1}}\mathbf{I}_{\mathcal{A}}
% \left(Y_{r+1}\right)\vert X_{t_r}\right\}
% \vert\mathcal{O}_r\right\} = {}\\
% {}=
%{\sf E}\left\{\sum\limits_{k=1}^N {\sf E}\left\{X_{t_{r+1}}\mathbf{I}_{\mathcal{A}}
%\left(Y_{r+1}\right)  \big| X_{t_r}=e_k\right\}X_{t_r}^k
% \big|\mathcal{O}_r\right\} = {}\\ 
% {}=
% \sum\limits_{k=1}^N{\sf E}
% \left\{X_{t_{r+1}}\mathbf{I}_{\mathcal{A}}\left(Y_{r+1}\right)\bigl| X_{t_r}=e_k\right\} 
% \widehat{X}_{t_r}^k ={}\\
% {}=\!
% \sum\limits_{k=1}^N{\sf E}
% \left\{{\sf E}\left\{X_{t_{r+1}}\mathbf{I}_{\mathcal{A}}
% \left(Y_{r+1}\right)\!\bigl| {\mathcal{F}}_{t_{r+1}}\right\}\!\bigl| 
% X_{t_r}\!=e_k\right\} \widehat{X}_{t_r}^k ={}\\
% {}=
% \sum\limits_{k=1}^N {\sf E}\left\{
% \vphantom{\int\limits_A\left(\sum\limits_{p=1}^N\right)}
% X_{t_{r+1}} \times{}\right.\\
% {}\times\int\limits_{\mathcal{A}}  
% \mathcal{N}\left(y,f \tau_{r+1}(X),\sum\limits_{p=1}^N \tau_{r+1}^p(X) g_pg_p^{\top}\right)dy
% \Biggl| X_{t_r}={}\\
%\left. {}=e_k
% \vphantom{\int\limits_A\left(\sum\limits_{p=1}^N\right)}
%\right\} \widehat{X}_{t_r}^k = 
% \sum\limits_{k=1}^N \int\limits_{\mathcal{A}}{\sf E}\left\{ 
% \vphantom{\sum\limits_{p=1}^N}
% X_{t_{r+1}} \times{}\right.\\
% {}\times\mathcal{N}\left(y,f \tau_{r+1}(X),\sum\limits_{p=1}^N \tau_{r+1}^p(X) 
% g_p g_p^{\top}\right)
% \Biggl| X_{t_r}={}\\
%\left. {}=e_k
%\vphantom{\sum\limits^N_{p=1}}
%\right\} \widehat{X}_{t_r}^k\, dy
 %={}\\
 {}=
 \sum\limits_{\ell=1}^N e_{\ell} \int\limits_{\mathcal{A}} 
 \left[ \sum\limits_{k=1}^N 
 \int\limits_{\mathcal{D}_{r+1}} 
 \mathcal{N}\left(y,f u,\sum_{p=1}^N u^p g_pg_p^{\top}\right)\times{}\right.\\
\left. {}\times
 \rho^{k,\ell}_{r+1}(du)\widehat{X}_{t_r}^k
 \vphantom{\int\limits_A\sum\limits_{p=1}^N}
 \right] 
 dy,
 \end{multline*}
 из чего следует, что интегранд в~квадратных скобках в~последнем выражении 
 определяет искомое совместное распределение $(X_{t_{r+1}},Y_{r+1})$ 
 относительно~$ \mathcal{O}_r$. Оценка~$\widehat{X}_{t_{r+1}}$ покомпонентно 
 определяется~\cite{BSh_85} с~помощью обобщенного варианта формулы Байеса:
 \begin{multline}
 \widehat{X}_{t_{r+1}}^j = {}\\
 \hspace*{-1mm}{}=
 \fr{\int\nolimits_{\mathcal{D}_{r+1}}\hspace*{-6mm} 
 \mathcal{N}\left(Y_{r+1},f u,\sum\nolimits_{p=1}^N \hspace*{-2mm}
 u^p g_pg_p^{\top}\!\right)\hspace*{-1mm}
 \sum\nolimits_{k=1}^N \hspace*{-2mm}
 \widehat{X}_{t_r}^k
 \rho^{k,j}_{r+1}(du)
 }
 { \int\nolimits_{\mathcal{D}_{r+1}} \hspace*{-6mm}
 \mathcal{N}\left(Y_{r+1},f v,\sum\nolimits_{q=1}^N \hspace*{-2mm}
 v^q g_qg_q^{\top}\!\right)\hspace*{-1mm}
 \sum\nolimits_{i,\ell=1}^N \hspace*{-2mm}
 \widehat{X}_{t_r}^i
 \rho^{i,\ell}_{r+1}(dv)
  },  \\ 
  j = \overline{1,N}\,.
 \label{eq:filt_1}
 \end{multline}
 Таким образом, доказана следующая
 
 %\smallskip
 
 \noindent
 \textbf{Лемма~1.}
\textit{Если для системы наблюдения}~(\ref{eq:obsys_1}) 
\textit{верны условия~(а) и~(б), то оценка~$\widehat{X}_t$ оптимальной фильтрации 
определяется формулой}~(\ref{eq:in_cond}) 
\textit{при $t\hm=0$, рекуррентным соотношением}~(\ref{eq:filt_1})~---
\textit{в~моменты~$t_{r+1}$ получения наблюдений~$Y_{r+1}$ 
и~формулой}~(\ref{eq:bw_obs})~--- 
\textit{в~промежутках времени между моментами получения наблюдений}.


\smallskip
 

 
 Несмотря на компактную запись~(\ref{eq:filt_1}), их прямая численная реализация 
 ресурсозатратна. Во-пер\-вых, в~(\ref{eq:filt_1}) требуется вычислять 
 распределения мас\-штаб\-но-сдви\-го\-вых смесей многомерных нормальных 
 распределений, что является трудоемкой\linebreak процедурой. Во-вто\-рых, 
 распределения~$\rho^{k,j}_{r+1}$ вре-\linebreak мени пребывания представляют собой 
 сумму\linebreak бесконечного ряда, слагаемые которого вычис\-ляются с~помощью 
 некоторой рекуррентной про\-це\-дуры~\cite{S_00}. В-третьих, 
 распределения~$\rho^{k,j}_{r+1}$ не являются абсолютно непрерывными 
 относительно меры Ле\-бега.
 { %\looseness=1
 
 }
 
 Следующий раздел посвящен численной аппроксимации~(\ref{eq:filt_1}) и~исследованию 
 ее точностных характеристик.
 
 \section{Приближенное вычисление оценки фильтрации}
 
 Без ограничения общности будем считать, что сетка~$\{t_r\}_{r \geqslant 0}$ 
 является равномерной с~шагом~$\Delta$, т.\,е.\ $t_r \hm= r \Delta$ 
 и~$\mathcal{D}_r \hm\equiv \mathcal{D}$.
 Обозначим через~$N_{r+1}$ об-\linebreak\vspace*{-12pt}
 
 \pagebreak
 
 \noindent
 щее число скачков процесса~$X_t$, имевших место 
 на промежутке $(t_r,t_{r+1}]$. Тогда из формулы полной вероятности следует, 
 что~(\ref{eq:filt_1}) представима в~виде:
 \begin{multline}
 \widehat{X}_{t_{r+1}}^j =  \left(
 \int\limits_{\mathcal{D}} 
 \mathcal{N}\left(Y_{r+1},f u,\sum\limits_{p=1}^N u^p g_pg_p^{\top}\right)\times{}\right.\\
\left. {}\times
 \sum\limits_{h=0}^{\infty}\sum\limits_{k=1}^N \widehat{X}_{t_r}^k
 \rho^{k,j,h}_{r+1}(du)
 \right)\Bigg/ \\
 \left(
 \vphantom{\sum\limits_{m=0}^{\infty}
 \sum\limits_{i,\ell=1}^N \widehat{X}_{t_r}^i
 \rho^{i,\ell,m}_{r+1}(dv)}
 \int\limits_{\mathcal{D}} 
 \mathcal{N}\left(Y_{r+1},f v,\sum\limits_{q=1}^N v^q g_qg_q^{\top}\right)\times{}\right.\\
\left.{}\times \sum\limits_{m=0}^{\infty}
 \sum\limits_{i,\ell=1}^N \widehat{X}_{t_r}^i
 \rho^{i,\ell,m}_{r+1}(dv)
 \right)
  \,, \enskip j = \overline{1,N}\,,
  \label{eq:filt_1_1}
 \end{multline}
 где 
 $ \rho^{k,j,h}_{r+1}(du)$~--- распределение вектора 
 $\tau_{r+1}X_{t_{r+1}}^{j}\mathbf{I}_{\{h\}}(N_{r+1})$ при 
 условии $X_{t_r}\hm=e_k$, т.\,е.\ 
 для любого $\mathcal{A} \hm\in \mathcal{B}(\mathbb{R}^M)$ верно тождество
\begin{multline*}
 \mathbf{P}\left\{\omega: \; X_{t_{r+1}}(\omega)=e_{j}, \; N_{r+1} = h,\right.\\ 
\left. \tau_{r+1}(X(\omega)) \in \mathcal{A}\;|\;X_{t_r}=e_k\right\} \equiv
  \rho^{k,j,h}_{r+1}(\mathcal{A}).
\end{multline*}
В качестве аппроксимации оценок можно использовать  
 $\overline{X}_{t_{r+1}}^n \ebd 
 \mathrm{col}\,(\overline{X}_{t_{r+1}}^{n,1},\ldots,\overline{X}_{t_{r+1}}^{n,N})$, 
 полученные из~(\ref{eq:filt_1_1}) путем урезания сумм ряда в~числителе и~знаменателе:
 
 \noindent
 \begin{multline}
 \overline{X}_{t_{r+1}}^{n,j} = 
 \left(
 \int\limits_{\mathcal{D}} 
 \mathcal{N}\left(Y_{r+1},f u,\sum\limits_{p=1}^N u^p g_pg_p^{\top}\right)\times{}\right.\\[-1pt]
\left.{}\times \sum\limits_{h=0}^{n}\sum\limits_{k=1}^N \overline{X}_{t_r}^k
 \rho^{k,j,h}_{r+1}(du)
 \right)\Bigg/ \\[-1pt]
 \left(
 \int\limits_{\mathcal{D}} 
 \mathcal{N}\left(Y_{r+1},f v,\sum\limits_{q=1}^N v^q g_qg_q^{\top}\right)\times{}\right.\\[-1pt]
\left. {}\times
 \sum\limits_{m=0}^{n}
 \sum\limits_{i,\ell=1}^N \overline{X}_{t_r}^i
 \rho^{i,\ell,m}_{r+1}(dv)
  \right)\,, \enskip
   j = \overline{1,N}.
  \label{eq:filt_2}
 \end{multline}
 Ниже по формуле полной вероятности получены интегралы из~(\ref{eq:filt_2}) для 
 $h\hm=0,1,2$:
 
\vspace*{-3pt}

 \noindent
  \begin{multline*}
 \int\limits_{\mathcal{D}}  \mathcal{N}
 \left(Y_{r+1},f u,\sum\limits_{p=1}^N u^p g_pg_p^{\top}\right) 
 \rho^{k,j,0}_{r+1}(du) = {}\\[-1pt]
 {}=
 \delta_{kj}\mathcal{N}\left(Y_{r+1},\Delta f^j,\Delta g_jg_j^{\top}\right)
 e^{\lambda_{jj}\Delta};
 %\label{eq:h0}
\\[-1pt]
 \int\limits_{\mathcal{D}}  \mathcal{N}\left(
 Y_{r+1},f u,\sum\limits_{p=1}^N u^p g_pg_p^{\top}\right) 
 \rho^{k,j,1}_{r+1}(du) ={} 
 \end{multline*}
 
 \noindent
 \begin{multline}
 \hspace*{-6.7pt}{}=\left(1-\delta_{kj}\right)\lambda_{kj}e^{\lambda_{jj}\Delta}
\! \int\limits_0^{\Delta}\!
 e^{(\lambda_{kk}-\lambda_{jj})u^k}
 \mathcal{N}\left(Y_{r+1},u^kf^k +{}\right.\hspace*{-0.28818pt}\\[-1pt]
\hspace*{-3mm}\left. {}+ \left(\Delta - u^k\right)f^j, u^k g_kg_k^{\top}+
 \left(\Delta-u^k\right)g_jg_j^{\top}\right)\,du^k;
 \label{eq:h1}
 \end{multline}
 
 \vspace*{-12pt}
 
 \noindent
 \begin{multline}
 \int\limits_D \mathcal{N}\left( 
Y_{r+1},f u,\sum\limits_{p=1}^N u^p g_pg_p^{\top}\right)du ={}\\[-1pt]
{}=
\sum\limits_{\substack{{\ell:\ell \neq k,}\\ {\ell \neq j}}}
 \lambda_{k\ell}\lambda_{\ell j} e^{\lambda_{jj}\Delta}\times {}\\[-1pt] 
 {}\times
 \int\limits_0^{\Delta} \int\limits_0^{\Delta-u^k} \!
e^{(\lambda_{kk}-\lambda_{\ell\ell})u^k+(\lambda_{\ell\ell}-
 \lambda_{jj})u^{\ell}}\times{} \\[-1pt] 
{}  \times
 \mathcal{N}\left(Y_{r+1},u^k f^k+u^{\ell}f^{\ell}+\left(
 \Delta-u^k-u^{\ell} \right)f^j,\right.\\[-1pt]
 \hspace*{-1mm}\left.
 u^k g_kg_k^{\top}+u^{\ell}g_{\ell}g_{\ell}^{\top}+\left(
 \Delta-u^k-u^{\ell} \right)
 g_jg_j^{\top}
 \right) du^{\ell}du^{k}, \!\!
  \label{eq:h2}
 \end{multline} 
 
\vspace*{-2pt}
 
 \noindent
  где  $\delta_{ij}$~--- символ Кронекера. Интегралы для $h\hm>2$ также могут 
  быть получены в~явном виде, однако их сложность резко возрастает.
 

   Так как система~(\ref{eq:obsys_1}) является автономной, то в~качестве локальной 
   характеристики бли\-зости~$\{\overline{X}_{t_r}\}$ 
   к~$\{\widehat{X}_{t_r}\}$ может быть выбрана величина
   
\noindent
 \begin{multline*}
 \overline{\sigma}(\pi) \ebd {\sf E}\left\{
 \|\widehat{X}_{t_{1}}(\pi, Y_{1}) - \overline{X}_{t_{1}}
 \left(\pi,Y_{1}\right)\|_{1}\right\} = {}\\
 {}=
 \sum\limits_{j=1}^N{\sf E}
 \left\{\left\vert \widehat{X}^j_{t_{1}}\left(\pi, Y_{1}\right) - \overline{X}^{n,j}_{t_{1}}
 \left(\pi,Y_{1}\right)\right\vert\right\}.
 %\label{eq:prec_1}
 \end{multline*}
 При этом начальное распределение $\pi \hm\in \mathcal{D}_1 \ebd $\linebreak $\ebd
 \{\mathrm{col}\,(\pi^1,\ldots,\pi^N):\;\pi^j > 0$, 
 $\sum\nolimits_{j=1}^N\pi^j\hm=1\}$ является начальным условием применения 
 одного шага рекурсии~(\ref{eq:filt_1}) или~(\ref{eq:filt_2}) для вычисления 
 оценки~$\widehat{X}_{t_{1}}$
   или~$\overline{X}_{t_{1}}$ соответственно. Фактически, 
 характеристика~$\overline{\sigma}(\pi)$ определяет, насколько сильно 
 рекурсивные схемы~(\ref{eq:filt_1}) и~(\ref{eq:filt_2}) разойдутся за 
 один шаг, стартуя из общей точки~$\pi$.
 
 Рекуррентные схемы~(\ref{eq:filt_1}) и~(\ref{eq:filt_2}), примененные~$r$~раз, 
 позволяют вычислить оценки~$\widehat{X}_{t_r}$ и~$\overline{X}_{t_r}$ 
 в~точке~$t_r$. В~качестве характеристики точности глобальной аппроксимации в~этом 
 случае естественно рассмотреть величину
 
 \vspace*{-2pt}
 
 \noindent
 \begin{equation*}
 \overline{\Sigma}_{t_r}(\pi) \ebd {\sf E}
 \left\{\|\widehat{X}_{t_{r}} - \overline{X}_{t_{r}}\|_{1}\right\} = 
 \!\sum\limits_{j=1}^N\!{\sf E}
 \left\{\left\vert \widehat{X}^j_{t_{r}} - 
 \overline{X}^{n,j}_{t_{r}}\right\vert \right\}.
% \label{eq:prec_2}
 \end{equation*}
 
 Следующее утверждение определяет оценки локальной и~глобальной 
 точности схемы аппроксимации~(\ref{eq:filt_2}).
 
 %\smallskip
 
 \noindent
 \textbf{Теорема~1.}\
\textit{Выполняются неравенства} 

%\vspace*{-2pt}

\noindent
 \begin{equation}
 \sup_{\pi \in \mathcal{D}_1} \overline{\sigma}(\pi) 
 \leqslant 2 \fr{(\overline{\lambda}\Delta)^{n+1}}{(n+1)!}\,;
 \label{eq:prec_loc}
\end{equation}

\noindent
\begin{align}
  \sup\limits_{\pi \in \mathcal{D}_1} \overline{\Sigma}_{t_r}(\pi)
   &\leqslant 2r \fr{(\overline{\lambda}\Delta)^{n+1}}{(n+1)!} +{}\notag\\[-0.5pt]
   &\hspace*{-20mm}{}+
  r(r-1)\left(
  \fr{(\overline{\lambda}\Delta)^{n+1}}{(n+1)!}
  \right)^2
  \left(
  1-\fr{(\overline{\lambda}\Delta)^{n+1}}{(n+1)!}
  \right)^{r-2},
 \label{eq:prec_glob}
 \end{align}
 
 \vspace*{-2pt}
 
 \noindent
 \textit{где} $\overline{\lambda} \ebd \max_{1 \leqslant j \leqslant N}|\lambda_{jj}|$.


%\smallskip

 Доказательство теоремы~1 приведено в~приложении.
 
 Данное утверждение представляет полезные оценки точности. Во-пер\-вых, 
 они являются равномерными по начальному распределению $\pi \hm\in \mathcal{D}_1$. 
 Во-вто\-рых, оценки носят универсальный, а~не асимптотический характер. Это 
 существенно в~практических задачах оценивания по дискретизованным 
 наблюдениям с~физическими или алгоритмическими ограничениями на шаг 
 по времени. Например, в~случае наблюдаемого процесса восстановления в~силу 
 центральной предельной теоремы для процессов восстановления~\cite{B_80} его
  приращения можно рассматривать как гауссовские случайные величины. 
  Однако данная аппроксимация обладает удовлетворительной точностью 
  только в~случае, когда шаг дискретизации по времени достаточно большой. 
 %
 В-третьих, неравенство~(\ref{eq:prec_glob}) позволяет получить порядок 
 аппроксимации при $\Delta \hm\to 0$. Зафиксируем момент времени $t\hm=T$ и~рассмотрим 
 характеристику $\sup\nolimits_{\pi \in \mathcal{D}_1} 
 \overline{\Sigma}_{T}(\pi)$ при $r\hm={T}/{\Delta}$ и~$\Delta \hm\to 0$. 
 Как только~$\Delta$ становится настолько мало, что 
 $\max\left({(\overline{\lambda}\Delta)^{n+1}}/{(n+1)!}, 
 \Delta ({T\lambda^{n+1}}/{(n+1)!})\right)\hm< 1$, из~(\ref{eq:prec_glob}) 
 следует неравенство
  %\begin{equation}
  $\sup\nolimits_{\pi \in \mathcal{D}_1} \overline{\Sigma}_{T}(\pi) 
  \hm\leqslant  ({3\overline{\lambda}^{n+1}}/{(n+1)!}) T\Delta^n.$
 %\label{eq:prec_asympt}
 %\end{equation}
 Это значит, что с~ростом времени~$T$ 
 ошибка аппроксимации копится пропорционально~$T$ и~при этом порядок точности 
 по~$\Delta$ равен~$n$.
 
 %\vspace*{-7pt}
 
  \section{Заключение}
  
  \vspace*{-4pt}
 
  В работе решена задача оценивания состояния однородного МСП по 
  дискретизованным наблюдениям. Получено аналитическое решение и~его 
  чис\-лен\-ные аппроксимации. Локальные и~глобальные показатели точ\-ности этих 
  приближений в~статье так\-же пред\-став\-ле\-ны. Примечательно, что  част\-ный случай 
  аппроксимаций~(\ref{eq:filt_2}) при $n\hm=0$ и~$\Lambda\hm=0$ был ранее 
  пред\-став\-лен в~\cite{B_17_1,B_17_2} для решения задачи байесовской классификации 
  случайного вектора по непрерывным наблюдениям с~мультипликативными шумами. 
 % 
Алгоритм оптимальной фильт\-ра\-ции и~его субоптимальные версии могут 
рас\-смат\-ри\-вать\-ся в~качестве основы чис\-лен\-ной реализации обобщения фильт\-ра 
Вонэма для сис\-тем с~мультипликативными шумами в~наблюдениях. 
Однако для их непосредственного использования необходимо решить 
следующие проб\-ле\-мы. Во-пер\-вых, в~(\ref{eq:h1}) и~(\ref{eq:h2}) присутствуют
 многомерные интегралы. Следует выяснить, какую результирующую погрешность 
 будут вносить ошибки их вы\-чис\-ле\-ния. Во-вто\-рых, представляется интересным 
 определить характеристики точ\-ности оптимальной фильт\-ра\-ции по дискретизованным 
 наблюдениям по отношению к~оптимальной фильт\-ра\-ции по непрерывным наблюдениям: 
 каков порядок точ\-ности по шагу временной дискретизации~$\Delta$? Для случая 
 вы\-чис\-ле\-ния классического фильт\-ра Вонэма с~по\-мощью алгоритма Эй\-ле\-ра--Ма\-ру\-ямы 
 подобный результат известен: порядок глобальной ошибки равен~${1}/{2}$. 
 Перечисленные задачи являются предметом дальнейших исследований.
 
 
  \vspace*{-10pt}
 
{\small
\subsection*{\raggedleft Приложение} 

\vspace*{-2pt}


\noindent
Д\,о\,к\,а\,з\,а\,т\,е\,л\,ь\,с\,т\,в\,о\ \ теоремы~1.\ \ Введем следующие 
обозначения для случайных величин и~мат\-риц, составленных из них:
\begin{align*}
\xi^{ji}(\ell)&\ebd 
\sum\limits_{h=0}^n \int\limits_{\mathcal{D}} 
 \mathcal{N}\left(Y_{\ell},f u,\sum\limits_{p=1}^N u^p g_pg_p^{\top}\right)
 \rho^{j,i,h}_{1}(du)\,; \\
  \theta^{ji}(\ell)&\ebd 
\sum\limits_{h=n+1}^{\infty} \int\limits_{\mathcal{D}} 
 \mathcal{N}\left(Y_{\ell},f u,\sum\limits_{p=1}^N u^p g_pg_p^{\top}\right)
 \rho^{j,i,h}_{1}(du)\,;
\\
 \xi(\ell)&\ebd \|\xi^{ji}(\ell)\|_{j,i=\overline{1,N}}\,,\quad 
 \Xi(r) \ebd \xi(r) \xi(r-1)\cdots \xi(1)\,;
 \\
 \theta(\ell)&\ebd \|\theta^{ji}(\ell)\|_{j,i=\overline{1,N}}\,, \quad 
 \Theta(r) \ebd \theta(r) \theta(r-1)\cdots \theta(1)\,.
%\label{eq:not_1}
\end{align*}
 
 Рекуррентные формулы~(\ref{eq:filt_1}) и~(\ref{eq:filt_2}) можно записать в~явной 
 форме
 
 
\noindent
\begin{align*}
 \widehat{X}_{t_r}& = \left( \mathbf{1}\left(\Xi(r) + 
 \Theta(r)\right)\pi\right)^{-1} \left(\Xi(r) + \Theta(r)\right)\pi\,;
\\
 \overline{X}_{t_r} &= \left( \mathbf{1}\Xi(r)\pi\right)^{-1} \Xi(r) \pi,
\end{align*}

\vspace*{-2pt}

\noindent
где $\mathbf{1} \ebd (1,\ldots,1)$~--- век\-тор-стро\-ка 
подходящей раз\-мер\-ности, составленная из единиц.

%Далее для краткости записи зависимость от~$r$ в~обозначениях~$\Xi(r)$ 
%и~$\Theta(r)$ будет опущена. 
Верна следующая цепочка неравенств:

 \vspace*{-3pt}

\noindent
\begin{multline}
\overline{\Sigma}_{t_r}(\pi)=%
%\me{}{\left\| 
%\widehat{X}_{t_r}(\pi, Y_1,\ldots,Y_r) - \overline{X}_{t_r}(\pi, Y_1,\ldots,Y_r)
%\right\|_1} =\\=
{\sf E}\left\{\left\| 
\fr{1}{\mathbf{1}\left(\Xi(r) + \Theta(r)\right)\pi} \left(\Xi(r) +{}\right.\right.\right.\\[-1pt]
\left.\left.\left.{}+ \Theta(r)\right)\pi
- \fr{1}{\mathbf{1}\Xi(r)\pi}\,\Xi(r) \pi
\right\|_1\right\} ={} \\[-1pt]
{}=
{\sf E}\left\{\fr{1}{\mathbf{1}\left(\Xi(r) + \Theta(r)\right)\pi \mathbf{1}\Xi(r)\pi}
\left\|
 \mathbf{1}\Xi(r) \pi \Theta(r)\pi -{}\right.\right.\\[-1pt]
\left.\left. {}- \mathbf{1}\Theta(r)\pi \Xi(r) \pi
 \right\|_1
 \vphantom{\fr{1}{\mathbf{1}\left(\Xi(r) + \Theta(r)\right)\pi \mathbf{1}\Xi(r)\pi}}
\right\} \leqslant {}\\[-1pt]
{}\leqslant 
{\sf E}\left\{\fr{1}{\mathbf{1}\left(\Xi(r) + \Theta(r)\right)\pi \mathbf{1}\Xi(r)\pi}
\left(
\mathbf{1}\Xi(r)\pi \| \Theta(r)\pi \|_1 +{}\right.\right.\\[-1pt]
\left.\left.{}+ \mathbf{1}\Theta(r)\pi 
\|
\Xi(r) \pi
\|_1
\right)
 \vphantom{\fr{1}{\mathbf{1}\left(\Xi(r) + \Theta(r)\right)\pi \mathbf{1}\Xi(r)\pi}}
\right\} ={}\\[-1pt]
{}=
2\,{\sf E}\left\{\fr{1}{\mathbf{1}\left(\Xi(r) + \Theta(r)\right)\pi}\mathbf{1}\Theta(r)\pi 
\right\}.
\label{eq:ineq_1}
\end{multline}

 
 \noindent
 Рассмотрим случайные события $a_{\ell} \ebd \{\omega \in \Omega: 
 N_{\ell}(\omega) \hm\leqslant n\}$, $\ell \hm= \overline{1,r}$, и~$A_r \ebd \{
 \omega\hm \in \Omega: \max_{1 \leqslant {\ell} \leqslant r}N_{\ell}(\omega) 
 \hm\leqslant n
 \}\hm=\prod\nolimits_{\ell=1}^r a_{\ell}$ и~оценку 
 $
 \widetilde{X}_{t_r}(\pi, Y_1,\ldots,Y_r)\ebd$\linebreak $\ebd
 {\sf E}\left\{X_{t_r}(\omega)\mathbf{I}_{A_r}(\omega)|\mathcal{O}_r\right\}.
 $
 Используя введенные выше обозначе\-ния и~абстрактный вариант формулы Байеса, 
 получаем, что
 
 \noindent
\begin{align}
\widetilde{X}_{t_r}& = \fr{1}{{\mathbf{1}\left(\Xi(r) + 
 \Theta(r)\right)\pi}}\,\Xi(r)\pi\,;\notag
 \\
\widehat{X}_{t_r} - \widetilde{X}_{t_r} &=
{\sf E}\left\{X_{t_r}(\omega)\mathbf{I}_{\overline{A}_r}(\omega)|\mathcal{O}_r\right\} ={}\notag\\[-1pt]
&\hspace*{17mm}{}= 
\fr{1}{\mathbf{1}\left(\Xi(r) + \Theta(r)\right)\pi}\Theta(r)\pi\,. 
\label{eq:eq_2}
 \end{align}
 Из (\ref{eq:ineq_1}) и~(\ref{eq:eq_2}) для $r\hm=1$ следует, что
 
 \vspace*{-4pt}
 
 \noindent
 \begin{multline}
 \overline{\sigma}(\pi) \leqslant 2\,{\sf E}
 \left\{\|{\sf E}\left\{X_{t_1}(\omega)\mathbf{I}_{\overline{a}_1}(\omega)|\mathcal{O}_1
 \right\}\|_1
 \right\} ={}\\[-1.5pt]
 {}=
 2\,{\sf E}\left\{\sum\limits_{n=1}^N {\sf E}
 \left\{X^n_{t_1}(\omega)\mathbf{I}_{\overline{a}_1}
 (\omega)|\mathcal{O}_1\right\}\right\} ={} \\[-2pt] 
 {}=
  2\,{\sf E}\left\{{\sf E}\left\{\mathbf{I}_{\overline{a}_1}(\omega)|\mathcal{O}_1
  \right\}\right\} =
   2 \mathbf{P}\left\{\overline{a}_1(\omega)\right\}.
\label{eq:ineq_3}
\end{multline}

 \vspace*{-2pt}
 
 \noindent
 Процесс $N^X_t$ общего числа скачков состояния~$X_t$ является считающим, и~его
  квадратическая характеристика равна 
  
\vspace*{-2pt}
  
  \noindent
 $$
 \langle N^X, N^X\rangle_t = - \int\limits_0^t \sum\limits_{n=1}^N \lambda_{nn} X_s^n\,ds\,,
 $$
 поэтому искомая вероятность ограничена сверху:
 $$ 
 \mathbf{P}\left\{\overline{a}_1(\omega)\right\} \leqslant 
 e^{-\overline{\lambda}\Delta}\sum\limits_{k=n+1}^{\infty} 
 \fr{(\overline{\lambda}\Delta)^{k}}{k!} <
 \fr{(\overline{\lambda}\Delta)^{n+1}}{(n+1)!}.
 $$
 
  \vspace*{-2pt}
  
  \noindent
 Из последнего неравенства и~(\ref{eq:ineq_3}) следует, что  для любого 
 начального распределения~$\pi$ выполняется неравенство $\overline{\sigma}(\pi)  
 \hm< 2({(\overline{\lambda}\Delta)^{n+1}}/{(n+1)!})$, т.\,е.\ 
 локальная оценка~(\ref{eq:prec_loc}) верна.
 
 С помощью марковского свойства пары $(X_t, N^X_t)$ и~последнего 
 неравенства можно оценить сверху вероятность 
 $\mathbf{P}\left\{\overline{A}_r(\omega)\right\}$:
 
  \vspace*{-2pt}
 
 \noindent
 \begin{multline*}
 \mathbf{P}\left\{\overline{A}_r(\omega)\right\} \leqslant 1 - \left(
 1- \fr{(\overline{\lambda}\Delta)^{n+1}}{(n+1)!}
 \right)^r \leqslant r \fr{(\overline{\lambda}\Delta)^{n+1}}{(n+1)!} + {}\\[-1pt]
 {}+\left|
 \sum\limits_{k=2}^r C_r^k \left(-\fr{(\overline{\lambda}\Delta)^{n+1}}{(n+1)!}
 \right)^k
 \right| \leqslant
 r \fr{(\overline{\lambda}\Delta)^{n+1}}{(n+1)!} +{}\\[-1pt]
 {}+\fr{r(r-1)}{2}
 \left(
 \fr{(\overline{\lambda}\Delta)^{n+1}}{(n+1)!}
 \right)^2
 \left(
 1-\fr{(\overline{\lambda}\Delta)^{n+1}}{(n+1)!}
 \right)^{r-2},
 \end{multline*} 
 из чего следует истинность глобальной оценки~(\ref{eq:prec_glob}).
Теорема~1 доказана.

}

%\vspace*{-12pt}

{\small\frenchspacing
 {%\baselineskip=10.8pt
 \addcontentsline{toc}{section}{References}
 \begin{thebibliography}{99}

\bibitem{Won_65}
\Au{Wonham W.} 
Some applications of stochastic differential equations to optimal
  nonlinear filtering~//
SIAM~J.~Control, 1965. Vol.~2. P.~347--369. 

\bibitem{KP_92}
\Au{Kloeden P., Platen E.} Numerical solution of stochastic
differential equations.~--- Berlin: Springer, 1992.~636~p.

\bibitem{YZL_04}
\Au{Yin G., Zhang Q., Liu Y.} 
Discrete-time approximation of Wonham filters~//
J.~Control Theory Applications, 2004. Iss.~2. P.~1--10.

\bibitem{PR_10}
\Au{Platen E., Rendek R.}
Quasi-exact approximation of hidden Markov chain filters~//
Communicat.~Stoch.~Analys., 2010. Vol.~4. Iss.~1. P.~129--142.

\bibitem{B_18}
\Au{Борисов А.} Фильтрация Вонэма по наблюдениям с~мультипликативными шумами~// 
Автоматика и~телемеханика, 2018.
№~1. C.~52--65. 
 
  \bibitem{BSh_85} %6
\Au{Бертсекас Д., Шрив С.} Стохастическое оптимальное управление. 
Случай дискретного времени~/ Пер. с~англ.~--- М.: Наука, 1985.~280~c.
(\Au{Betsekas~D.\,P., Shreve~S.\,E.} Stochastic optimal control:
The discrete-time case.~--- Orlando, FL, USA:
Academic Press Inc., 1978. 323~p.)

  \bibitem{ZhSh_95} %7
\Au{Жакод Ж., Ширяев А.} Предельные теоремы для случайных процессов,~I.~/
Пер. с~англ.~--- 
М.: Физматлит, 1995.~544~c.
(\Au{Jacod~J., Shiryaev~A.} Limit theorems for stochastic processes.~---
Berlin: Springer, 2003. 664~p.)

\bibitem{S_00}
\Au{Sericola B.} Occupation times in Markov processes~//
Commun. Stat. Stochastic Models, 2000. Vol.~16. Iss.~5. P.~479--510. 

  \bibitem{B_80}
\Au{Боровков А.} Асимптотические методы в~тео\-рии массового обслуживания.~--- 
М.: Физматлит, 1995.~384~c.

  \bibitem{B_17_1}
\Au{Борисов А.} Классификация по непрерывным наблюдениям с~мультипликативными шумами.~I. 
Формулы байесовской оценки~// Информатика и~её применения, 2017. Т.~11. Вып.~1. C.~11--19.
doi: 10.14357/19922264170102.

  \bibitem{B_17_2}
\Au{Борисов А.} Классификация по непрерывным наблюдениям с~мультипликативными 
шумами.~II. Алгоритм численной реализации оценки~// Информатика и~её 
применения, 2017. Т.~11. Вып.~2. C.~33--41.
doi: 10.14357/19922264170204.

 \end{thebibliography}

 }
 }

\end{multicols}

\vspace*{-4pt}

\hfill{\small\textit{Поступила в~редакцию 10.07.18}}

\vspace*{6pt}

%\pagebreak

%\newpage

%\vspace*{-28pt}

\hrule

\vspace*{2pt}

\hrule

%\vspace*{-2pt}

\def\tit{FILTERING OF~MARKOV JUMP PROCESSES\\ BY~DISCRETIZED OBSERVATIONS}

\def\titkol{Filtering of Markov jump processes by discretized observations}

\def\aut{A.\,V.~Borisov}

\def\autkol{A.\,V.~Borisov}

\titel{\tit}{\aut}{\autkol}{\titkol}

\vspace*{-11pt}


\noindent
Institute of Informatics Problems, Federal Research Center ``Computer Science 
and Control'' of the Russian Academy of Sciences, 44-2~Vavilov Str., Moscow 
119333, Russian Federation


\def\leftfootline{\small{\textbf{\thepage}
\hfill INFORMATIKA I EE PRIMENENIYA~--- INFORMATICS AND
APPLICATIONS\ \ \ 2018\ \ \ volume~12\ \ \ issue\ 3}
}%
 \def\rightfootline{\small{INFORMATIKA I EE PRIMENENIYA~---
INFORMATICS AND APPLICATIONS\ \ \ 2018\ \ \ volume~12\ \ \ issue\ 3
\hfill \textbf{\thepage}}}

\vspace*{6pt}



\Abste{The article is devoted to a~solution of the optimal filtering problem 
of a~homogenous Markov
jump process state. The available observations represent 
time increments of the integral transformations of the Markov\linebreak\vspace*{-12pt}}

\Abstend{state corrupted by 
Wiener processes. The noise intensity is also state-dependent. At the instant of 
the consecutive
observation obtaining, the optimal estimate is calculated recursively 
as a~function of previous estimate and the new observation, meanwhile between 
observations the filtering estimate is a simple forecast by virtue of the Kolmogorov 
differential system. The recursion is rather expensive because of  need to calculate 
the integrals, which are the location-scale mixtures of Gaussians. The mixing 
distributions represent the occupation of the state in each of possible values 
during the mid-observation intervals. The paper contains numerically cheaper 
approximations, based on the restriction of the state transitions number between 
the observations. Both the local and global characteristics of approximation 
accuracy are obtained as functions of the dynamics parameters, mid-observation 
interval length, and upper bound of transitions number.}

\KWE{Markov jump process; optimal filtering; multiplicative observation noises; 
stochastic differential equation; numerical approximation}




\DOI{10.14357/19922264180316}

%\vspace*{-14pt}

\Ack
\noindent
The work was supported in part by the Russian Foundation
for Basic Research (Project No.\,16-07-00677).



%\vspace*{6pt}

  \begin{multicols}{2}

\renewcommand{\bibname}{\protect\rmfamily References}
%\renewcommand{\bibname}{\large\protect\rm References}

{\small\frenchspacing
 {%\baselineskip=10.8pt
 \addcontentsline{toc}{section}{References}
 \begin{thebibliography}{99}
\bibitem{Won_65-1}
\Aue{Wonham, W.} 1965.
Some applications of stochastic differential equations to optimal
  nonlinear filtering.
\textit{SIAM~J.~Control} 2:347--369. 

\bibitem{KP_92-1}
\Aue{Kloeden,~P., and E.~Platen.} 1992. \textit{Numerical solution of stochastic
differential equations.} Berlin: Springer. 636~p.

\bibitem{YZL_04-1}
\Aue{Yin,~G., Q.~Zhang, and Y.~Liu.} 2004.
Discrete-time approximation of Wonham filters.
\textit{J.~Control Theory Applications} 2:1--10.

\bibitem{PR_10-1}
\Aue{Platen, E., and R.~Rendek.} 2010.
Quasi-exact approximation of hidden Markov chain filters.
\textit{Communicat. Stoch. Analys.} 4(1):129--142.

\bibitem{B_18-1}
\Aue{Borisov, A.} 2018. Wonham filtering by observations
with multiplicative noises. \textit{Automat.~Rem.~Contr.} 79(1):39--50.  
doi: 10.1134/ S0005117918010046.
 
  \bibitem{BSh_85-1}
\Aue{Bertsekas, D., and S.~Shreve.} 1996.
\textit{Stochastic optimal control: The discrete-time case}.
Nashua, NH: Athena Scientific. 330~p.
  
  \bibitem{ZhSh_95-1}
  \Aue{Jacod,~J., and A.~Shiryaev.} 2003.
\textit{Limit theorems for stochastic processes.}
Berlin: Springer. 664~p.

\bibitem{S_00-1}
\Aue{Sericola, B.}
2000. Occupation times in Markov processes.
\textit{Commun. Stat.} 16(5):479--510. 

  \bibitem{B_80-1}
\Aue{Borovkov, A.} 1984.
 \textit{Asymptotic methods in queueing theory}. 
 Hoboken, NJ: Wiley-Blackwell.~304~p.

  \bibitem{B_17_1-1}
  \Aue{Borisov, A.} 2017. 
  Klassifikatsiya po ne\-pre\-ryv\-nym nablyu\-de\-miyam s~mul'tiplikativnymi shumami. I. 
  Formuly bayesov\-skoy otsenki [Classification by continuous-time observations
in multiplicative noise. I.~Formulae for Bayesian 
estimate]. \textit{Informatika i~ee Primeneniya~--- Inform.~Appl.}
11(1):11--19. doi: 10.14357/19922264170102.

  \bibitem{B_17_2-1}
\Aue{Borisov, A.} 2017. Klassifikatsiya po nepreryvnym nablyudemiyam 
s~mul'tiplikativnymi summami. II.~Formuly bayesovskoy otsenki 
[Classification by continuous-time observations
in multiplicative noise. II.~Numerical algorithm].
\textit{Informatika i~ee Primeneniya~--- Inform.~Appl.}
11(2):33--41. doi: 10.14357/19922264170204.

\end{thebibliography}

 }
 }

\end{multicols}

\vspace*{-6pt}

\hfill{\small\textit{Received July 10, 2018}}

%\pagebreak

%\vspace*{-18pt}

\Contrl

\noindent
\textbf{Borisov Andrey V.} (b.\ 1965)~--- 
Doctor of Science in physics and mathematics, principal scientist, Institute of
Informatics Problems, Federal Research Center ``Computer Science and Control''
 of the Russian Academy of
Sciences, 44-2 Vavilov Str., Moscow 119333, Russian Federation; 
\mbox{aborisov@frccsc.ru}
\label{end\stat}

\renewcommand{\bibname}{\protect\rm Литература}        %2
%\newcommand{\A}{{\mathbf A}}
%\newcommand{\B}{{\mathbf B}}
%\newcommand{\la}{{\lambda}}
%\newcommand{\be}{\begin{equation}}
%\newcommand{\ee}{\end{equation}}
%\newcommand{\ber}{\begin{eqnarray}}
%\newcommand{\eer}{\end{eqnarray}}

%\newcommand{\nin}{\noindent}
%\newcommand{\non}{\nonumber}
%\newcommand{\half}{\frac{1}{2}}
%\newcommand{\quarter}{\frac{1}{4}}

\def\stat{zeifman}

\def\tit{ОБ ОДНОМ КЛАССЕ МАРКОВСКИХ СИСТЕМ ОБСЛУЖИВАНИЯ$^*$}

\def\titkol{Об одном классе марковских систем обслуживания}

\def\autkol{Я.\,А.~Сатин, А.\,И.~Зейфман, А.\,В.~Коротышева, С.\,Я.~Шоргин}
\def\aut{Я.\,А.~Сатин$^1$, А.\,И.~Зейфман$^2$, А.\,В.~Коротышева$^3$, С.\,Я.~Шоргин$^4$}

\titel{\tit}{\aut}{\autkol}{\titkol}

{\renewcommand{\thefootnote}{\fnsymbol{footnote}}\footnotetext[1]
{Исследование поддержано РФФИ, гранты 11-07-00112-а и 11-01-12026-офи-м.}}


\renewcommand{\thefootnote}{\arabic{footnote}}
\footnotetext[1]{Вологодский государственный педагогический
университет, yacovi@mail.ru}
\footnotetext[2]{Вологодский государственный педагогический университет;  
Институт проблем информатики Российской академии наук; 
Институт социально-экономического развития территорий Российской академии наук,  a\_zeifman@mail.ru}
\footnotetext[3]{Вологодский государственный педагогический
университет,  a\_korotysheva@mail.ru}
\footnotetext[4]{Институт проблем информатики Российской академии наук, SShorgin@ipiran.ru}


\Abst{Рассматриваются модели обслуживания, описываемые конечными марковскими 
цепями с непрерывным временем. При этом предполагается,  что интенсивности 
поступления и обслуживания требований не зависят от числа требований в сис\-те\-ме. 
Получены оценки скорости сходимости и устойчивости различных характеристик таких сис\-тем.}

\KW{нестационарные марковские системы
обслуживания; скорость сходимости; устойчивость; оценки}

 \vskip 14pt plus 9pt minus 6pt

      \thispagestyle{headings}

      \begin{multicols}{2}
      
            \label{st\stat}

\section{Введение}

Классы систем массового обслуживания, описываемых процессами
рождения и гибели (стационарными и нестационарными, с катастрофами)
изучались начиная с 1970-х~гг.\ многими авторами
(см., например,~[1--6]). С~помощью методов,
разработанных одним из авторов настоящей \mbox{статьи}\linebreak (подробное изложение
этих методов приведено в~[7--9]), для таких сис\-тем
удалось получить точные оценки скорости сходимости и устойчивости.

Оказывается, этот же подход можно применить и к существенно более 
общему классу систем обслуживания.

Рассмотрим систему массового обслуживания, число требований в которой 
описывается нестационарной марковской цепью с непрерывным временем и 
конечным пространством состояний, причем требования могут поступать и 
обслуживаться группами.

Пусть $X=X(t)$, $t\geq 0$,~--- число требований в системе обслуживания ($0 \hm\le X(t) \hm\le r$).

Обозначим через 
\begin{gather*}
p_{ij}(s,t)=\mathrm{Pr}\left\{ X(t)=j\left| X(s)=i\right.
\right\}\,,\\
i,j \ge 0\,,\ 0\leq s\leq t\,,
\end{gather*}
переходные вероятности
процесса $X\hm=X(t)$, а через  $p_i(t)\hm=\mathrm{Pr}\left\{ X(t) \hm=i \right\}$~---
его вероятности состояний.

Будем предполагать, что интенсивности поступления и обслуживания $k$ требований в 
момент~$t$ в сис\-те\-ме об\-слу\-жи\-ва\-ния ($\lambda_{k}(t)$ и  $\mu_{k}(t)$ соответственно)  
не зависят от числа требований, находящихся в системе в момент~$t$, являются локально 
интегрируемыми на $[0,\infty)$ функциями времени~$t$ и, кроме того, 
$\lambda_{k+1}(t) \hm\le \lambda_{k}(t)$ и  $\mu_{k+1}(t) \hm\le \mu_{k}(t)$ при всех~$k$ 
и почти при всех $t \hm\ge 0$.

Тогда для описания вероятностной динамики процесса получаем прямую систему Колмогорова в виде
\begin{equation} 
\fr{d\vp}{dt}=A(t)\vp(t)\,,
\label{ur_1}
\end{equation}
 где
 {\footnotesize
\begin{multline*}
A(t)={}\\
{}=
\begin{pmatrix}
a_{00}(t) & \mu_1(t)  & \mu_2(t)   & \mu_3(t)  & \mu_4(t) & \cdots & \mu_r(t) \\
\la_1(t)   & a_{11}(t)  & \mu_1(t)  & \mu_2(t)   & \mu_3(t)  & \cdots & \mu_{r-1}(t) \\
\la_2(t)  & \la_1(t)    & a_{22}(t)& \mu_1(t)  & \mu2(t)    &  \cdots & \mu_{r-2}(t) \\
\cdots&\cdots&\cdots&\cdots&\cdots&\cdots&\cdots \\
\la_r(t) & \la_{r-1}(t) & \la_{r-2}(t) & \cdots & \la_2(t)  & \la_1 (t)   &  a_{rr}(t)
\end{pmatrix}\,,
\end{multline*}}
причем  
$$
a_{ii}(t)=-\sum\limits_{k=1}^{i}\mu_k(t) - \sum\limits_{k=1}^{r-i} \la_{r-k}(t)\,.
$$

Далее будем обозначать через $\|\bullet\|$  $l_1$-нор\-му, т.\,е.\ 
$\|{\vx}\|\hm=\sum|x_i|$, а $\|B\| \hm= \max\limits_j \sum\limits_i |b_{ij}|$, 
если $B \hm= (b_{ij})_{i,j=0}^{r}$.
%
Тогда, в частности, имеем 
$$
\|A(t)\| \le 2\sum\limits_{k=1}^{r}(\la_{k}(t)+ \mu_k(t))
$$ 
при  всех $t \hm\ge 0$.

Через 
$$
E(t,k) = E\left\{X(t)\left|X(0)\hm=k\right.\right\}
$$ 
будем далее обозначать математическое ожидание процесса (среднее число требований) в момент~$t$ 
при условии, что в нулевой момент времени он находится в состоянии~$k$, 
а через $E_{\bf p}(t)$ обозначим математическое ожидание процесса в момент~$t$ 
при начальном распределении вероятностей состояний $\mathbf{p}(0) \hm= \mathbf{p}$.

\section{Оценки скорости сходимости}

Рассмотрим вспомогательную последовательность положительных чисел $\{d_i\}$, $i\hm=1, \dots,r$.

Положим
\begin{equation*}
d=\min\limits_{1 \le i \le r} d_i\,; \enskip 
G=\sum\limits_{i=1}^r d_i\,; \enskip W=\min\limits_k \fr{d_k}{k}\,.
%\label{2.01}
\end{equation*}

Рассмотрим величины
\begin{multline*}
\alpha_i(t)= -a_{ii}(t)+\la_{r-i+1}(t)-\sum\limits_{k=1}^{i-1}(\mu_{i-k}(t)-{}\\
{}-
\mu_i(t))\fr{d_k}{d_i}-\sum\limits_{k=1}^{r-i}(\la_k(t)-\la_{i+r-1}(t))\fr{d_{k+i}}{d_i}\,,
%\label{2.02}
\end{multline*}

\noindent
\begin{equation*}
\alpha(t)=\min\limits_{1 \le i \le r}\alpha_i(t)\,.
%\label{2.03}
\end{equation*}

\smallskip

\noindent
\textbf{Теорема~1.} \textit{Пусть существует последовательность положительных 
чисел  $\{d_j\}$ такая, что}
\begin{equation}
\int\limits_0^{\infty} \alpha(t)\, dt = + \infty\,.
\label{2.031}
\end{equation}
\textit{Тогда $X(t)$ слабо эргодичен, при
любых начальных условиях} $\mathbf{p}^*(s)$, $\mathbf{p}^{**}(s)$ 
\textit{и любых $s$, $t$, $0\le s\le t$, справедлива оценка
\begin{equation} 
\label{2.04}
\|\vp^*(t)-\vp^{**}(t)\| \le \fr{8G}{d}\,e^{-\int\limits_s^t {\alpha(u)\,du}}\,.
\end{equation}
Кроме того,  $X(t)$ имеет предельное среднее $\phi(t)$ и при любых~$k$ и $t \hm\ge 0$ справедливо неравенство}:
\begin{equation}
\label{2.05}
|E(t,k)-\phi(t)|\le \fr{4G}{W}\,e^{-\int\limits_0^t {\alpha(u)\,du}}\,.
\end{equation}


\smallskip


\noindent
Д\,о\,к\,а\,з\,а\,т\,е\,л\,ь\,с\,т\,в\,о\,.\

Пользуясь предложенным в предыдущих работах способом, 
выразим 
$$
p_0=1-\sum\limits_{1\le i \le r}{p_i}\,.
$$

Тогда получим неоднородное уравнение:
\begin{equation} 
\label{ur_per}
\fr{d\vz}{dt}= B(t)\vz(t)+\vf(t)\,, 
%\label{2.06}
\end{equation}
\noindent
где $\vf(t)=\left(\la_1, \la_2,\cdots,\la_r \right)^{\mathrm{T}}$;

\end{multicols}


\hrule

\vspace*{6pt}

\begin{equation*}
B = \left(
\begin{array}{cccccccc}
a_{11}- \la_1   & \mu_1 - \la_1   & \mu_2 - \la_1   & \mu_3 -\la_1   & \cdots& \cdots & \mu_{r-1}- \la_1  \\
\la_1 -\la_2    & a_{22} -\la_2  & \mu_1-\la_2   & \mu_2 -\la_2     & \cdots&  \cdots & \mu_{r-2} -\la_2 \\
\la_2 -\la_3    & \la_1 -\la_3   & a_{33} -\la_3  & \mu_1-\la_2   & \cdots&  \cdots & \mu_{r-3} -\la_3 \\
\cdots&\cdots&\cdots&\cdots&\cdots&\cdots&\cdots \\
\la_{r-1} -\la_r  &\la_{r-2} -\la_r & \cdots & \cdots & \la_2 -\la_r   & \la_1 -\la_r     &  a_{rr} -\la_r
\end{array}
\right)\,.
%\label{2.07}
\end{equation*}

Рассмотрим треугольную матрицу
\begin{equation*}
D=\begin{pmatrix}
d_1   & d_1 & d_1 & \cdots & d_1 \\
0   & d_2  & d_2  &   \cdots & d_2 \\
\cdots&\cdots&\cdots&\cdots&\cdots \\
0  & 0 & \cdots & 0 &  d_r
\end{pmatrix}
%\label{2.08}
\end{equation*}
и соответствующую норму $\|{\bf z}\|_{D}\hm=\|D {\bf z}\|_1$.

Тогда имеем:
\begin{equation*}
 D BD^{-1}=\left(
\begin{array}{ccccccc}
a_{11}-\la_r  &  (\mu_1-\mu_2) \fr{d_1}{d_2}  & (\mu_2-\mu_3)\fr{d_1}{d_3}  & \cdots &  (\mu_{r-1}-\mu_r)\fr{d_1}{d_r} \\
(\la_1-\la_r) \fr{d_2}{d_1} &  a_{22}-\la_{r-1}  &(\mu_1-\mu_3)\fr{d_2}{d_3}  & \cdots &  (\mu_{r-2}-\mu_r)\fr{d_2}{d_r} \\
(\la_2-\la_r) \fr{d_3}{d_1} &  (\la_1-\la_{r-1})\fr{d_3}{d_2}   &a_{33}-\la_{r-2}   & \cdots &  (\mu_{r-3}-\mu_r)
\fr{d_3}{d_r}  \\
\cdots&\cdots&\cdots&\cdots&\cdots \\
(\la_{r-1} -\la_r) \fr{d_r}{d_1} & (\la_{r-2} -\la_{r-1}) \fr{d_r}{d_2}  & (\la_{r-3} -\la_{r-2}) \fr{d_r}{d_3}  & \cdots & a_{rr}-\la_1 \\
\end{array}
\right)\,.
%\label{2.09}
\end{equation*}


\begin{multicols}{2}


Далее, оценивая логарифмическую норму оператора~$B(t)$ (см., например, 
подробное рассмотрение в~[8--10]), получаем
\begin{multline*}
\gamma \left(B(t)\right)_{1D} = \gamma \left(DB(t)D^{-1}\right)_{1}={}\\
{}=
\max \left(\vphantom{\sum\limits_{k=1}^{i-1}}
a_{ii}(t) - \la_{r-i+1}(t) + \sum\limits_{k=1}^{i-1}\left(\mu_{i-k}(t)-{}\right.\right.\\
\left.\left.{}-\mu_i(t)\right)
\fr{d_k}{d_i} +
\sum\limits_{k=1}^{r-i}(\la_k(t)-\la_{i+r-1}(t))\fr{d_{k+i}}{d_i}\right) ={}\\
{}=
 - \min \alpha_i(t) = - \alpha(t)\,.
% \label{2.10}
\end{multline*}
Тогда\\[-7.9pt]
\begin{equation*}
\|\vz^*(t)-\vz^{**}(t)\|_{1D}\le  e^{-\int\limits_s^t {\alpha(u)du}}\|\vz^*(s)-\vz^{**}(s)\|_{1D}
%\label{2.11}
\end{equation*}
для всех $0 \le s \le t$ и любых начальных условий $\vz^*(s)$, $\vz^{**}(s)$.

Теперь, учитывая оценки для сравнения норм (см., например,~\cite{z08b}), получаем:
\begin{multline*}
\|\vp^*(t)-\vp^{**}(t)\| \le 2\|\vz^*(t)-\vz^{**}(t)\| \le{}\\
{}\le  \fr{4}{d}\|\vz^*(t)-\vz^{**}(t)\|_{1D}\le{} \\
{} \le \fr{4}{d}\,e^{-\int\limits_s^t {\alpha(u)\,du}}\|\vz^*(s)-\vz^{**}(s)\|_{1D} 
\le{}\\
{}\le
 \fr{4G}{d}\,e^{-\int\limits_s^t {\alpha(u)\,du}}\|\vz^*(s)-\vz^{**}(s)\| \le{} \\
{} \le  \fr{4G}{d}\,e^{-\int\limits_s^t {\alpha(u)\,du}}\|\vp^*(s)-\vp^{**}(s)\| \le 
\fr{8G}{d}\,e^{-\int\limits_s^t {\alpha(u)\,du}} 
%\label{2.11-a}
\end{multline*}
для любых начальных условий ${\bf p^*}(s)$, ${\bf p^{**}}(s)$ и любых $s,t$, $0\hm\le s\hm\le t$.

Из слабой эргодичности процесса с конечным пространством состояний 
вытекает существование предельного среднего, начальные условия для которого можно 
в общем случае выбрать произвольно.
Для оценки средних воспользуемся неравенством, приведенным в параграфе~2.3 из~\cite{z08b}:
\begin{multline*}
\|{\bf z}\|_{1D} = d_0 \left|\sum\limits_{i=1}^{\infty} p_i \right|
+ d_1 \left|\sum\limits_{i=2}^{\infty} p_i \right| + \dots \ge{}\\
{}\ge 
 W \sum\limits_{k \ge 1} k \left|\sum\limits_{i \ge k} p_i\right| \ge \fr{W}{2}
\sum\limits_{k \ge 1} k \left|p_k\right|\,.  
%\label{2.12}
\end{multline*}
Получаем теперь
\begin{multline*}
|E(t,k)-\phi(t)|\le \fr{2}{W}\,\|\vp^*(t)-\vp^{**}(t)\|_{1D}\le {} \\
{}\le\fr{2}{W}\,e^{-\int\limits_0^t {\alpha(u)\,du}}\|{\bf e}_k -
\vp^{**}(0)\|_{1D} \le \frac{4G}{W}e^{-\int\limits_0^t
{\alpha(u)\,du}}\,,
%\label{2.13}
\end{multline*}
что и требовалось доказать.
\columnbreak

%\smallskip

\noindent
\textbf{Замечание~1.} {Положим в условиях теоремы~1 
$$
\beta(t)=\max\limits_{1 \le i \le r}\alpha_i(t)\,.
$$ 
Тогда, пользуясь внедиагональной неотрицательностью матрицы $DB(t)D^{-1}$ 
с помощью методики, описанной в~\cite{z08b, z95b}, получаем справедливость неравенства

\noindent
\begin{equation*} 
%\label{2.14}
\|\vp^*(t)-\vp^{**}(t)\| \ge \fr{d}{8G}\,e^{-\int\limits_s^t {\beta(u)\,du}}
\end{equation*}
при любых $s$, $t$, $0\le s\le t$ и уже не при любых начальных условиях~${\bf p^*}(s)$, 
${\bf p^{**}}(s)$, а таких, что  $D\left({\bf p^*}(s) \hm-{\bf p^{**}}(s)\right) \hm\ge 0.$ 
Следовательно, оценки тео\-ре\-мы~1 будут заведомо иметь точный по времени порядок, если удастся 
выбрать вспомогательную последовательность $\{d_i\}$ так, что $\alpha(t)\hm=\beta(t)$, т.\,е.\ 
все $\alpha_i(t)$ одинаковы (не зависят от индекса~$i$)}.



\smallskip

Введем теперь в рассмотрение величины

\vspace*{-1pt}

\noindent
\begin{multline*}
\zeta_i(t)= -a_{ii}(t)+\la_{r-i+1}(t)+{}\\
{}+\sum\limits_{k=1}^{i-1}\left(\mu_{i-k}(t)-
\mu_i(t)\right) \fr{d_k}{d_i}+{}\\
{}+\sum\limits_{k=1}^{r-i}\left(\la_k(t)-\la_{i+r-1}(t)\right)\fr{d_{k+i}}{d_i}\,;
%\label{2.0211}
\end{multline*}
\begin{equation*}
\chi(t)=\max\limits_{1 \le i \le r}\zeta_i(t)\,.
%\label{2.0311}
\end{equation*}

\noindent
\textbf{Замечание 2.} {В условиях теоремы~1 при любых начальных условиях 
${\bf p^*}(s)$, ${\bf p^{**}}(s)$ и любых $s,t$,  $0\le s\le t$, 
справедлива следующая двухсторонняя оценка скорости сходимости:

\vspace*{-1pt}

\noindent
\begin{multline*} 
%\label{2.041}
\!\!\!\fr{d}{4G}\,e^{-\int\limits_s^t {\chi(u)\,du}}\|\vp^*(s)-\vp^{**}(s)\| \le
 \|\vp^*(t)-\vp^{**}(t)\| \le {}\\
 {}\le\fr{4G}{d}\,e^{-\int\limits_s^t {\alpha(u)\,du}}\|\vp^*(s)-\vp^{**}(s)\|.
\end{multline*}
Таким образом, можно оценить и сверху и снизу время  вхождения 
сис\-те\-мы обслуживания в предельный режим. Более подробно о получении 
нижних оценок см., например, в~\cite{z95b, gz05}.}

\smallskip

Рассмотрим два частных случая теоремы.

\smallskip

\noindent
\textbf{Следствие 1}. \textit{Пусть при выполнении остальных условий теоремы~1 
вместо}~(\ref{2.031}) \textit{выполняется условие $\alpha(t) \hm\ge \alpha \hm> 0$ 
почти при всех $t \hm\ge 0$. Тогда вместо}~(\ref{2.04}) \textit{и}~(\ref{2.05}) 
\textit{справедливы оценки}:

\vspace*{-1pt}

\noindent
\begin{align*} 
%\label{2.15}
\|\vp^*(t)-\vp^{**}(t)\| &\le \fr{8G}{d}\,e^{-\alpha \left(t-s\right)}\,;
\\
%\label{2.16}
|E(t,k)-\phi(t)|&\le \fr{4G}{W}\,e^{- \alpha t}\,.
\end{align*}

\pagebreak

%\smallskip

Положим 
\begin{gather*}
M_0=\max\limits_{|t-s|\le 1}\int\limits_s^t \alpha(u)\,du;\\
\alpha^* = \int\limits_0^1 \alpha(t)\, dt\,; \quad
M=e^{M_0+\alpha^*}\,.
\end{gather*}
С учетом неравенства 
$$
e^{-\int\limits_s^t {\alpha(u)\,du}} \hm\le M e^{-\alpha^* (t-s)}
$$ 
получаем следующее утверждение.

\smallskip

\noindent
\textbf{Следствие~2.} \textit{Пусть все $\lambda_k(t)$ и $\mu_k(t)$ 1-пе\-ри\-одич\-ны,  
а при выполнении остальных условий теоремы~1 вместо}~(\ref{2.031}) 
\textit{выполняется условие  $\alpha^* \hm> 0$.  Тогда предельный режим (скажем, $\vp^*(t)$) 
и соответствующее ему предельное среднее $\phi^*(t)$ можно выбрать 
1-пе\-ри\-оди\-че\-ски\-ми, а вместо}~(\ref{2.04}) \textit{и}~(\ref{2.05}) 
\textit{справедливы оценки}:
\begin{equation*} 
%\label{2.17}
\|\vp(t) - \vp^*(t)\| \le \fr{8GM}{d}\,e^{-\alpha^*t}
\end{equation*}
\textit{и, кроме того,}
\begin{equation*}
|E(t,k)-\phi^*(t)|\le \fr{4GM}{W}\,e^{-\alpha^*t}
%\label{2.18}
\end{equation*}
\textit{при любом $k$ и $t \ge 0$}.



\section{Устойчивость}

Рассмотрим также <<возмущенный>> процесс обслуживания $\bar{X}\hm=\bar{X}(t)$, $t\hm\geq 0$, 
в котором интенсивности поступления и обслуживания требований также не зависят от чис\-ла 
требований в системе, обозначая его соответствующие характеристики теми же буквами с 
чертой сверху. Для прос\-то\-ты записи оценок будем предполагать, что возмущения 
<<равномерно малы>>, т.\,е.\ выполняется неравенство $\| A(t)-\bar{A}(t)\| \hm\le \varepsilon$. 
Первые результаты для нестационарных цепей с непрерывным временем получены в~\cite{z85}, 
а детальное рассмотрение для более общего случая неравномерных оценок можно без труда 
провести так же, как это сделано в~\cite{z98, ae}. Для получения требуемых равномерных 
оценок устойчивости необходима экспоненциальная эргодичность соответствующего процесса, 
т.\,е.\ существование положительных констант $N$, $a$ таких, что  для правой части~(\ref{2.04}) 
справедливо неравенство:
\begin{equation}
e^{-\int\limits_s^t {\alpha(u)\,du}} \le Ne^{-a\left(t-s\right)}\,.
\label{3.01}
\end{equation}
Оценка~(\ref{3.01}) заведомо имеет место, в частности, если выполнены условия одного из следствий 
предыду\-ще\-го параграфа.

\smallskip

\noindent
\textbf{Теорема~2.}
\textit{Пусть выполнены условия теоремы~1 и}~(\ref{3.01}). \textit{Тогда при
 любых начальных условиях ${\bf p}(s)$ и ${\bar{\bf p}}(s)$ для процессов~$X(t)$ 
 и $\bar{X}(t)$ соответственно справедливы следующие оценки устойчивости:}
\begin{align*} 
%\label{3.02}
\limsup_{t \to \infty}  \|{\bf p}(t)- \bar{\bf p}(t)\| &\le
\fr{\varepsilon(1+\ln(4GN/d))}{a}\,;
\\
% \label{3.03}
\limsup\limits_{t \to \infty}   |E_{\bf p}(t)- \bar{E}_{\bar{\bf p}(t)}|&\le 
\fr{r \varepsilon(1+\ln(4GN/d))}{a}\,.
\end{align*}


\smallskip

\noindent
Д\,о\,к\,а\,з\,а\,т\,е\,л\,ь\,с\,т\,в\,о\ основано на подходе, 
введенном для стационарных процессов в~\cite{mit03} и описанном для нестационарной 
ситуации в~\cite{z11}.
Если  при любых начальных условиях для исходного процесса справедлива оценка
\begin{equation*} 
%\label{3.04}
\|\vp(t) - \vp^*(t)\| \le ce^{-b\left(t-s\right)}\,,
\end{equation*}
то, полагая
\begin{multline*}
\beta (t, s)=\sup\limits_{ \| {\bf v} \| =1, \sum {v_i}=0}
{\|V(t,s){\bf v}(t,s)\|} ={}\\
{}= \fr{1}{2} \max_{i,j} \sum\limits_k {|p_{ik}(t,
s)-p_{jk}(t, s)|}\,, 
\end{multline*}
где $V(t, s)$~--- матрица Коши
уравнения~(\ref{ur_1}), получаем в итоге следующее неравенство:
\begin{equation*}
\|{\bf p}(t)-\bar{\bf p}(t)\| \le{}
\begin{cases}
\|{\bf p}(s)-{\bf \bar{p}}(s)\|+ (t-s)\varepsilon \,, &\\
&\hspace*{-35mm} 0<t< b^{-1} \ln \left(\fr{c}{2}\right)\,; \\
b^{-1}\left(\ln \fr{c}{2} +1-\fr{c}{2}\,e^{-b(t-s)}\right)\varepsilon +{}&\\
{}+
\fr{c}{2}\,e^{-b(t-s)} \|{\bf p}(s)-{\bf \bar{p}}(s)\|\,, &\\
&\hspace*{-30mm}t\ge b^{-1}\ln \left(\fr{c}{2}\right)
\end{cases}
%\label{3.05}
\end{equation*}
для любых начальных условий ${\bf p}(s)$ и $\bar{\bf p}(s)$.
Из неравенств~(\ref{2.04}) и~(\ref{3.01}) вытекает, что $b=a$, $c={8GN}/{d}$.  
Устремив $t \hm\to \infty$ и взяв $s\hm=0$, получаем требуемые оценки.


\smallskip

\noindent
\textbf{Замечание~3.} 
В полученную оценку устойчивости для математического ожидания процесса 
в качестве множителя входит размерность~$r$, поэтому иногда лучший результат 
удается получить при помощи другого подхода, описанного в работе~\cite{z11}.

\smallskip

Положим 
$$
S=\max\limits_{{1 \le i, j \le r}} \fr{d_i}{d_j}\,,
$$ 
и пусть числа $K, L$ таковы, что 

\noindent
$$
d_1\la_1(t) + (d_1+d_2)\la_2(t) + \dots + 
\left(\sum\limits_{1 \le i \le r}d_i\right) \la_r(t) \le K\,,
$$ 
а 

\noindent
\begin{multline*}
d_1(\la_1(t)-\bar{\la}_1(t)) + (d_1+d_2)(\la_2(t)-\bar{\la}_2(t)) + \dots\\
\dots + 
\left(\sum\limits_{1 \le i \le r}d_i\right) (\la_r(t)-\bar{\la}_r(t)) \le 
L\varepsilon
\end{multline*} 
почти при всех $t \ge 0.$

\smallskip

\noindent
\textbf{Теорема~3.}
\textit{Пусть  выполнены условия теоремы~2 и, кроме того, при всех~$k$ 
и почти всех $t \hm\ge 0$ $\la_k(t) \hm< \infty$. Тогда при любых начальных условиях 
${\bf p}(s)$ и ${\bar{\bf p}}(s)$ для процессов $X(t)$ и $\bar{X}(t)$ 
соответственно справедливо неравенство}

\noindent
\begin{equation*}
\limsup\limits_{t \to \infty}   |E_{\bf p}(t)- \bar{E}_{\bar{\bf p}(t)}|\le 
\fr{ N\varepsilon\left(L a+ 2KNS\right)}{W a \left(a-2\varepsilon S\right)}\,.
\end{equation*}


\smallskip

\noindent
Д\,о\,к\,а\,з\,а\,т\,е\,л\,ь\,с\,т\,в\,о.\
 Перепишем исходную систему~(\ref{ur_per}) для невозмущенного процесса в следующем виде:
 \noindent
 
\begin{equation*}
\fr{d\vp}{dt}=\bar{B}(t)\vp(t) + {\bf f}(t)+\left(B(t)-\bar{B}(t)\right)\vp(t)\,.
%\label{eq112-n}
\end{equation*}
Тогда

\noindent
\begin{multline*}
\vp(t)=\bar{U}(t,0)\vp(0)+\int\limits_0^t \bar{U}(t,\tau){\bf{f}}(\tau) \, d\tau+{}\\
{}+\int\limits_0^t \bar{U}(t,\tau) \left(B(\tau)-\bar{B}(\tau)\right)\vp(\tau)\, d\tau\,;
\end{multline*}

\vspace*{-9pt}

\begin{equation*}
\hspace*{-15mm}\bar{\vp}(t)=\bar{U}(t,0)\bar{\vp}(0)+\int\limits_0^t \bar{U}(t,\tau){\bf{f}}(\tau) \, d\tau,
\end{equation*}
где $U(t,s)$~--- матрица Коши для уравнения~(\ref{ur_per}).
В любой норме при одинаковых начальных условиях получаем следующую оценку:
%\noindent
\begin{multline}
 \label{3000}
\!\!\!\!\!\!\left\|\vp(t)-\bar{\vp}(t)\right\|\le \!\!\int\limits_0^t \!\!\|\bar{U}(t,\tau)\|
\left(\| B(\tau)-\bar{B}(\tau)\| \|\vp(\tau)\| +\right.\\
\left.{}+ \| \vf(\tau)-\bar{\vf}(\tau)\|\right)\,d\tau\,.\!
\end{multline}
Имеем почти при всех $t \ge 0$:
\begin{equation*}
\|B(t)-\bar{B}(t)\|_{1D}=\|D(B(t)-\bar{B}(t))D^{-1}\| \le 2S\varepsilon\,;
%\label{3002}
\end{equation*}
%
%\vspace*{-14pt}
%
%\noindent
\begin{multline*}
\|{\bf f}(t)\|_{1D} \le d_1\la_1(t) + (d_1+d_2)\la_2(t) + \dots + {}\\
{}+
\left(\sum\limits_{1 \le i \le r}d_i\right) \la_r(t) \le K\,, 
\quad \|\vf(\tau)-\bar{\vf}(\tau)\|_{1D} \le L\varepsilon\,.
%\label{3002-a}
\end{multline*}
А тогда
\begin{multline*}
\gamma(\bar{B}(t))_{1D} \le \gamma(DB(t)D^{-1})+\|B(t)-\bar{B}(t)\|_{1D} \le  {}\\
{}\le -
\alpha(t)+2S \varepsilon \,.
% \label{3003}
\end{multline*}

Оценим теперь
\begin{multline*} 
%\label{8402}
\!\|{\bf p}(t)\|_{1D} \le
\|U(t){\bf p}(0) \|_{1D} +
 \int\limits_0^t \!\!\| U(t,\tau){\bf f}(\tau)\, d\tau \|_{1D} \le {}\\
 {}\le
 N e^{-a t} \| \vp(0)\|_{1D}  + \fr{K N}{a}.
\end{multline*}

 Тогда с учетом~(\ref{3000}) получаем:
\begin{multline*} 
%\label{3004}
\left\|\vp(t)-\bar{\vp}(t)\right\|_{1D}\le N\int\limits_0^t e^{-(a - 2\varepsilon S)(t-\tau)}\times{}\\
{}\times
\left(2S\varepsilon (N e^{-a \tau} \| \vp(0)\|_{1D}  + \fr{K N}{a}) +  L\varepsilon \right)\, d\tau  \le {} \\
{}\le  o(1)+\fr{ N\varepsilon(L+{2KNS}/{a})}{a-2\varepsilon S}\,. 
\end{multline*}

\vspace*{-9pt}

\section{Примеры}

\noindent
\textbf{Пример 1.}

Рассмотрим исходный процесс обслуживания с интенсивностями 
$\la_1(t)\hm=\la_2(t)\hm=\la_3(t)\hm=\la(t) \hm= 3\hm+\sin{2\pi t}$, 
$\mu_1(t)\hm=\mu_2(t)\hm=  \mu(t) \hm= 2\hm+\cos{2\pi t}$, 
$\la_4(t)=\ldots=\la_r(t)\hm=\mu_3(t)=\ldots=\mu_r(t)\hm=0$. Выберем последовательность  
$d_k\hm=h^k$, где $0{,}82 \hm< h \hm<1$. Тогда имеем
$$
d=h^r\,; \quad G \le \fr{h}{1-h}\,; \quad W=\fr{h^r}{r}\,.
$$

Будем предполагать, что возмущенный процесс имеет такую же структуру 
мат\-ри\-цы интенсивностей, причем $|\la(t)\hm-\bar{\la}(t)| \hm\le \varepsilon$ 
и  $|\mu(t)\hm-\bar{\mu}(t)| \hm\le \varepsilon$ почти при всех $t \hm\ge 0$. 
Отметим кстати, что при этом $\| A(t)\hm-\bar{A}(t)\| \hm\le 10 \varepsilon$ почти при 
всех $t \hm\ge 0$. Рассмотрим дальнейшие оценки:
$$
S=\fr{1}{h^2}\,; \ K=4 \left(3h+2h^2+h^3\right)\,; \ L=3h+2h^2+h^3\,;
$$
$$
\alpha(t) \ge \la(t)\left(3 - h - h^2 -h^3\right)-\mu(t)\left(\fr{1}{h^2}+\fr{1}{h}-2\right)\,;
$$
$$
\alpha^*= 3\left(3 - h - h^2 -h^3\right)-2\left(\fr{1}{h^2}+\fr{1}{h}-2\right)\,;
$$


\noindent
\begin{multline*}
M_0 \le \int\limits_0^1 |\alpha(t)|\, dt \le 4\left(3 - h - h^2 -h^3\right)+{}\\
{}+
3\left(\fr{1}{h^2}+\fr{1}{h}-2\right)\,;
\end{multline*}

\vspace*{-9pt}

\noindent
$$
M=e^{\alpha^*+M_0}\,.
$$

Если, например, взять 
$h\hm=0{,}9$, то $\alpha^*\hm=0{,}992$, $M_0\hm=3{,}281$, $M\hm=71{,}737$.

Тогда получаем следующие оценки.

По следствию~2
\begin{align*}
 \|{\bf p}(t)- {\bf p^{*}}(t)\| &\le \fr{8Me^{-\alpha^*t}}{h^{r-1}(1-h)}\,;\\
|E_{\bf p}(t)-\phi^*(t)| &\le  \fr{4Mre^{-\alpha^*t}}{h^{r-1}(1-h)}\,.
\end{align*}

По теореме~2 ($N=M$, $a=\alpha^*$) с использованием оценок следствия~2
\begin{align*}
\limsup\limits_{t \to \infty} \|{\bf p}(t)- \bar{\bf p}(t)\| &\le{} \notag\\
&\hspace*{-15mm}{}\le \fr{\varepsilon(1+\ln({4M}/({h^{r-1}(1-h)})))}{\alpha^*}\,;\\
\limsup\limits_{t \to \infty}   |E_{\bf p}(t)- \bar{E}_{\bar{\bf p}(t)}| &\le \notag\\
&\hspace*{-15mm}{}\le\fr{r\varepsilon(1+\ln(4M/(h^{r-1}(1-h))))}{\alpha^*}\,.
\end{align*}

По теореме~3 с использованием оценок следствия~2
\begin{multline*}
\limsup\limits_{t \to \infty}   |E_{\bf p}(t)- \bar{E}_{\bar{\bf p}(t)}| \le {}\\
{}\le
\fr{rM\varepsilon(3h+2h^2+h^3)(\alpha^* h^2+8M)}{h^r\alpha^*(\alpha^* h^2-2\varepsilon)}\,.
\end{multline*}

\noindent

\textbf{Пример 2.}

Рассмотрим процесс с интенсивностями 
$\la_1(t)\hm=\la_2(t)\hm=\ldots=\la_r(t) \hm= \la(t) \hm= 3\hm+\sin{2\pi t}$; 
$\mu_1(t)\hm=\mu_2(t)\hm= \mu(t) \hm= 2+\cos{2\pi t}$;
$\mu_3(t)=\ldots=\mu_r(t)=0$.

Будем предполагать, что возмущенный процесс имеет такую же структуру 
мат\-ри\-цы интен\-сив\-ностей, причем $|\la(t)-\bar{\la}(t)| \hm\le \varepsilon$ и  
$|\mu(t)-\bar{\mu}(t)| \hm\le \varepsilon$ почти при всех $t \hm\ge 0$. 
При этом будем иметь $\| A(t)\hm-\bar{A}(t)\| \hm\le 2r \varepsilon$ почти при всех $t \hm\ge 0$.

Выберем последовательность $d_k\hm=1$. Тогда  
\begin{gather*}
d=1\,; \enskip G=r\,; \enskip W=\fr{1}{r}\,; \enskip S=1\,; \\
K=\fr{4r(1+r)}{2}\,; \quad L=\fr{r(1+r)}{2}\,;
\\
\alpha(t)=\la(t)\,; \ \alpha=2\,; \ \alpha^*=3\,; M_0 \le 4\,; \ M \le  e^{7}\,.
\end{gather*}

И получаем следующие оценки.

\columnbreak

По следствию~1
\begin{align*}
 \|{\bf p^*}(t)- {\bf p^{**}}(t)\| &\le 8re^{-2t}\,;\\
|E_{\bf p}(t)- \phi(t)|&\le  4r^2 e^{-2t}\,.
\end{align*}

По следствию~2
\begin{align*}
\|{\bf p}(t)- {\bf p^{*}}(t)\| &\le 8re^{7-3t}\,;
\\[6pt]
|E_{\bf p}(t)- \phi^*(t)| &\le 4r^2 e^{7-3t}\,.
\end{align*}

По теореме~2 ($N=1$, $a=\alpha$) с учетом оценок следствия~1
\begin{align*}
\limsup\limits_{t \to \infty} \|{\bf p}(t)- \bar{\bf p}(t)\| &\le 
\fr{\varepsilon(1+\ln{4r})}{2}\,;
\\[6pt]
\limsup\limits_{t \to \infty}   |E_{\bf p}(t)- \bar{E}_{\bar{\bf p}(t)}|
&\le \fr{r\varepsilon(1+\ln{4r})}{2}\,.
\end{align*}

По теореме~2 ($N=M$, $a=\alpha^*$) с учетом оценок следствия~2
\begin{align*}
\limsup\limits_{t \to \infty} \|{\bf p}(t)- \bar{\bf p}(t)\| &\le 
\fr{\varepsilon(8+\ln{4r})}{3}\,;
\\
\limsup\limits_{t \to \infty}   \left|E_{\bf p}(t)- \bar{E}_{\bar{\bf p}(t)}\right| &\le 
\fr{r\varepsilon(8 + \ln{4r})}{3}\,.
\end{align*}

По теореме~3 с учетом оценок следствия~1
\begin{equation*}
\limsup\limits_{t \to \infty}   \left|E_{\bf p}(t)- \bar{E}_{\bar{\bf p}(t)}\right| \le 
\fr{5 \varepsilon r^2 (1+r)}{4(1- \varepsilon)}\,.
\end{equation*}

По теореме~3 с учетом оценок следствия~2
\begin{equation*}
\limsup\limits_{t \to \infty}   \left|E_{\bf p}(t)- \bar{E}_{\bar{\bf p}(t)}\right| \le 
\fr{\varepsilon e^{7} r^2 (1+r) (3+8e^{7})}{6(3-2\varepsilon)}\,.
\end{equation*}

{\small\frenchspacing
{%\baselineskip=10.8pt
\addcontentsline{toc}{section}{Литература}
\begin{thebibliography}{99}

 \bibitem{b} %1
\Au{Баруча-Рид~А.\,Т.} Элементы теории марковских процессов и их
приложения.~--- М.: Наука, 1969.

\bibitem{gm}  %2
\Au{Гнеденко~Б.\,В., Макаров~И.\,П.} Свойства решений задачи с потерями
в случае периодических интенсивностей~// Дифф. уравнения, 1971.
Вып.~7. С.~1696--1698.

\bibitem{g1}   %3
\Au{Gnedenko~D.\,B.} On a generalization of Erlang formulae~// 
Zastosow. Mat., 1971. Vol.~12. P.~239--242.

\bibitem{S}  %4
\Au{Саати~Т.\,Л.} Элементы теории массового обслуживания
 и ее приложения.~--- М.: Сов. радио, 1971.

\bibitem{g}  %5
\Au{Gnedenko~B., Soloviev~A.} On the conditions of the
existence of final probabilities for a Markov process~// Math.
Operations. Stat., 1973. P.~379--390.

\bibitem{gk} %6
\Au{Гнеденко~Б.\,В., Коваленко~И.\,Н.} Введение в теорию массового
обслуживания.~--- М.: Наука, 1987.
\pagebreak

\bibitem{gz00}   %7
\Au{Granovsky~B.\,L., Zeifman~A.\,I.}  The N-limit of spectral gap of 
a class of birth-death Markov chains~//
 Appl. Stoch. Models Business Ind., 2000. Vol.~16. P.~235--248.

\bibitem{z08b}  %8
\Au{Зейфман~А.\,И., Бенинг~В.\,Е., Соколов~И.\,А.} 
Марковские цепи и модели с непрерывным временем.~--- М.: Элекс-КМ, 2008.

\bibitem{dzp} %9
\Au{Van Doorn~E.\,A., Zeifman~A.\,I., Panfilova~T.\,L.}  
Bounds and asymptotics for the rate of convergence of birth-death processes~//  
Th. Prob. Appl., 2010. Vol.~54. P.~97--113.

\bibitem{z95b}   %10
\Au{Zeifman~A.\,I.} Upper and lower bounds on the rate of
convergence for nonhomogeneous birth and death processes~//  Stoch.
Proc. Appl., 1995. Vol.~59. P.~157--173.

\bibitem{gz05}  %11
\Au{Granovsky~B.\,L., Zeifman~A.\,I.} On the lower bound of the spectrum
 of some mean-field models~// Theory Prob. Appl., 2005. Vol.~49. P.~148--155.
 
\bibitem{z85}  %12
\Au{Zeifman~A.\,I.} Stability for contionuous-time
nonhomogeneous Markov chains~// Lect. Notes Math.,  1985. Vol.~1155.
P.~401--414.

\bibitem{z98} %13
\Au{Zeifman~A.} Stability of birth and death processes~// 
J.~Math. Sci., 1998. Vol.~91. P.~3023--3031.

\bibitem{ae} %14
\Au{Андреев~Д., Елесин~М., Кузнецов~А., Крылов~Е., Зейфман~А.}
Эргодичность и устойчивость нестационарных систем обслуживания~//
Теория вероятностей и математическая статистика, 2003. Т.~68.
С.~1--11.

\bibitem{mit03} %15
\Au{Mitrophanov~A.\,Yu.} Stability and exponential convergence of continuous-time 
Markov chains~//  J. Appl. Prob., 2003. Vol.~40. P.~970--979.

\label{end\stat} 

\bibitem{z11} %16
\Au{Зейфман~А.\,И., Коротышева~А.\,В., Панфилова~Т.\,Л., Шоргин~С.\,Я.} 
Оценки устойчивости  для некоторых систем обслуживания с катастрофами~//  
Информатика и её применения, 2011. Т.~5. Вып.~3. С.~27--33.
 \end{thebibliography}
}
}


\end{multicols}           %3 
\def\stat{kondranin+ushakov}

\def\tit{СИСТЕМА ОБСЛУЖИВАНИЯ С~ОТНОСИТЕЛЬНЫМ ПРИОРИТЕТОМ  И~ПРОФИЛАКТИКАМИ ПРИБОРА$^*$}

\def\titkol{Система обслуживания с~относительным приоритетом  и~профилактиками прибора}

\def\aut{Е.\,С.~Кондранин$^1$,  В.\,Г.~Ушаков$^2$}

\def\autkol{Е.\,С.~Кондранин,  В.\,Г.~Ушаков}

\titel{\tit}{\aut}{\autkol}{\titkol}

\index{Кондранин Е.\,С.}
\index{Ушаков В.\,Г.}
\index{Kondranin E.\,S.}
\index{Ushakov V.\,G.}




{\renewcommand{\thefootnote}{\fnsymbol{footnote}} \footnotetext[1]
{Работа выполнена при финансовой поддержке РФФИ (проект 18-07-00678).}}


\renewcommand{\thefootnote}{\arabic{footnote}}
\footnotetext[1]{Факультет вычислительной математики и~кибернетики Московского государственного 
университета им.\ М.\,В.~Ломоносова, \mbox{ekondranin@yandex.ru}}
\footnotetext[2]{Факультет вычислительной математики и~кибернетики
Московского государственного университета им.\ М.\,В.~Ломоносова;
Институт проб\-лем информатики Федерального исследовательского
центра <<Информатика и~управ\-ле\-ние>> Российской академии наук,
\mbox{vgushakov@mail.ru}}

\vspace*{-10pt}




\Abst{Изучена одноканальная система
массового обслуживания с~двумя типами требований, бесконечным
числом мест для ожидания, гиперэкспоненциальным входящим потоком 
и~профилактиками обслуживающего прибора при освобождении системы.
Тип  требования определяется случайно с~заданными вероятностями 
в~момент его поступления в~систему обслуживания. Требования первого
типа имеют относительный приоритет перед требованиями второго
типа. Найдено нестационарное совместное распределение числа
требований каждого типа в~системе. Профилактики прибора
заключаются в~том, что в~момент освобождения системы от требований
прибор на случайное время с~заданным распределением становится
недоступным для обслуживания. Если за время профилактики поступает
хотя бы одно требование, то начинается нормальное функционирование
системы. Если требования не поступают, то прибор отправляется на
новую профилактику. Такие системы хорошо описывают
функционирование большого числа реальных вычислительных и~информационных систем.}

\KW{гиперэкспоненциальный поток; профилактики
обслуживающего прибора; одноканальная система; относительный
приоритет; длина очереди}

\DOI{10.14357/19922264180405}
  
%\vspace*{4pt}


\vskip 10pt plus 9pt minus 6pt

\thispagestyle{headings}

\begin{multicols}{2}

\label{st\stat}

\section{Введение}

В классической системе массового обслуживания ожидание требований
в очереди связано только с~занятостью обслуживающего прибора. В~то
же время в~реальных системах сам  прибор может пребывать как 
в~активном, так и~в~неактивном состоянии. Такое неактивное
состояние прибора (в~литературе на английском языке используется
термин vacation, а~на русском~--- профилактика или прогулка) может
быть связано со многими причинами. В~част\-ности, сис\-те\-мы
обслуживания с~профилактиками прибора хорошо описывают
функционирование  реальных вычислительных и~информационных систем,
в которых наряду с~основными требованиями имеются второстепенные.
Второстепенные требования всегда присутствуют в~сис\-те\-ме, а~их
обслуживание может проводиться только тогда, когда нет основных,
т.\,е.\ в~фоновом режиме.

С точки зрения самого процесса профилактики прибора существует
несколько ее разновидностей. Во-пер\-вых, могут быть разными
правила, задающие условия начала профилактики: прибор может брать
перерыв только при  полном исчерпании требований в~очереди
(exhaustive service) либо при наличии определенного их числа
(nonexhaustive service). Во-вто\-рых, могут быть разными правила
возвращения прибора в~работу. С~этой точки зрения различают случаи
однократного (single vacation) и~многократного (multiple vacation)
перерыва в~работе. В~первом случае ушедший на профилактику прибор
после ее окончания находится в~рабочем состоянии независимо от
наличия требований в~системе. Во втором случае прибор, не
обнаружив новых требований в~очереди, уходит на новую
профилактику.


В работах~[1--4] можно найти обзор известных результатов, большое
число постановок задач, описание различных приложений и~обширную
библиографию по анализу систем с~профилактиками обслуживающего
прибора.


В настоящей работе исследуется совместное распределение длин
очередей в~нестационарном режиме в~однолинейной системе 
с~ожиданием, гиперэкспоненциальным входящим потоком, двумя типами
требований и~относительным приоритетом. Аналогичная неприоритетная
система обслуживания исследована в~[5].

\vspace*{-6pt}

\section{Описание модели}

Рассматривается однолинейная система массового обслуживания 
с~двумя приоритетными классами требований. Входящий поток~---
гиперэкспоненциальный с~функцией распределения интервалов между
поступлениями требований вида:
\begin{multline*}
A(t)=\sum\limits_{i=1}^kc_i\left(1-e^{-a_it}\right),\enskip t>0,\enskip
a_i>0,\enskip c_i>0,\\
a_i\ne a_j\,,\enskip i\ne j\,,\enskip  \sum\limits_{i=1}^k c_i=1\,.
\end{multline*}

Каждое поступившее требование направляется в~первый класс 
с~вероятностью~$p$ и~во второй класс с~вероятностью $1\hm-p$
независимо от остальных требований. Требования первого класса
обладают относительным приоритетом перед требованиями второго
класса. Длительности обслуживания требований $i$-го приоритетного
класса~--- независимые в~совокупности и~не зависящие от входящего
потока случайные величины с~функцией распределения~$B_i(x)$,
$i\hm=1,2.$
 Если в~некоторый момент времени система освободилась от требований, 
 то обслуживающий прибор
 отправляется на профилактику, которая длится случайное время с~функцией 
 распределения~$C(x).$
 Не ограничивая общности, будем считать, что $B_i(x)\hm<1$
 и~$C(x)\hm<1$  для любого~$x$ 
 и~существуют плотности
 распределения~$b_i(x)$ и~$c(x).$
  Обозначим:
$$
 \beta_i(s)=\int\limits_0^{\infty}e^{-sx}b_i(x)\,dx\,;\enskip 
  \gamma(s)=\int\limits_0^{\infty}e^{-sx}c(x)\,dx\,.
$$
Пока прибор находится на профилактике, он не доступен для
обслуживания. Если за время профилактики поступают требования,
после ее завершения начинается их обслуживание. Если ни одно
требование не поступает, то прибор отправляется на новую
профилактику. Длительности различных профилактик являются
независимыми случайными величинами 
и~не зависят от входящего потока и~времен обслуживания.

\section{Вспомогательные результаты}

  Рассмотрим многочлен по $\mu$ степени $k$ вида:
\begin{multline}
\label{1}
\prod\limits_{i=1}^k\left(\mu+a_i\right)-{}\\
{}-
\left(pz_1+(1-p)z_2\right)\sum\limits_{j=1}^kc_ja_j\prod\limits_{i\ne
j}\left(\mu+a_i\right)\,.
\end{multline}
Занумеруем его корни $\mu_1(z_1,z_2),\ldots,\mu_k(z_1,z_2)$ таким образом,
чтобы они были непрерывными функциями и~$\mu_1(1,1)\hm=0.$ Тогда
$\mathrm{Re}\, \mu_j\left(z_1,z_2\right)\hm<0$, $|z_1|\hm<1$, 
$|z_2|\hm<1,$ $\mu_i(z_1,z_2)\hm\ne \mu_j(z_1,z_2),$ $ i\hm\ne j$,
$j\hm=1,\ldots,k.$ Обозначим:
$$
\alpha_m(z_1,z_2)=\prod\limits_{j\ne m}\left(\mu_m\left(z_1,z_2\right)-
\mu_j\left(z_1,z_2\right)\right)\,.
$$
Справедливы следующие леммы.

\smallskip

\noindent
\textbf{Лемма~1.}\
\textit{Для любого $l=1,\ldots,\:k$ система уравнений}
$$
z_j=\beta_j(s-\mu_l(z_1,z_2)),\ \ j=1,2,
$$
\textit{имеет единственное решение $z_i=z_{il}(s)$ такое, 
что $|z_{il}(s)|\hm<1$ при $l\hm=2,\ldots, k,$ $\mathrm{Re}\, s\hm\geqslant 0,$ 
а~$z_{i1}(0)\hm=1$, $|z_{i1}(s)|\hm<1$ при} $\mathrm{Re}\, s\hm> 0$, $i\hm=1,2.$

\smallskip

\noindent
\textbf{Лемма~2.}\
\textit{При каждом $l\hm=1,\ldots,k$ уравнение}
$$
z_1=\beta_1\left(s-\mu_l(z_1,z_2)\right)
$$
\textit{имеет единственное решение $z_1\hm=z_{1l}(z_2,s),$ 
аналитическое в~области $\mathrm{Re}\, s\hm>0$, $|z_2|\hm<1.$
}

\smallskip

Положим
$$
\lambda_l(s)=\mu_l\left(z_{1l}(s),z_{2l}(s)\right)\,.
$$




\section{Распределение длины очереди}

  Гиперэкспоненциальный поток можно рас\-смат\-ри\-вать как
пуассоновский поток со случайной интен\-сив\-ностью~$a,$ которая
принимает $k$ различных значений $a_1,\ldots,a_k$  с~вероятностями
$c_1,\ldots,c_k.$ Текущее значение~$a$ разыгрывается в~момент
поступления требования и~не меняется между двумя соседними
поступлениями. Введем случайный процесс~$j(t)$ такой, что если
$a\hm=a_j$ в~момент времени $t,$ то $j(t)\hm=j.$

Целью работы является нахождение распределения случайного процесса
$\left(L_1(t),L_2(t)\right),$ где $L_i(t)$~--- число требований из
$i$-го приоритетного класса, находящихся в~системе в~момент
времени~$t.$

При сделанных предположениях относительно параметров изучаемой
системы обслуживания\linebreak процесс $\left(L_1(t),L_2(t)\right)$ не
является, вообще говоря, марковским. Пусть $i(t)=i$, $i\hm=1,2,$ если
в~момент времени~$t$ обслуживается требование из $i$-го
приоритетного класса, и~$i(t)\hm=0,$ если в~момент времени~$t$ прибор
находится на профилактике. Случайный процесс~$x(t)$ определим
следующим образом. Если $i(t)\hm\ne 0,$ то $x(t)$ есть
время, прошедшее с~начала обслуживания требования, находящегося на
приборе, до момента~$t.$ Если $i(t)\hm=0,$ то $x(t)$ есть время,
прошедшее с~начала профилактики прибора до момента~$t.$ Случайный
процесс $\left(L_1(t),L_2(t),i(t),j(t),x(t)\right)$ является
однородным марковским процессом. Положим
\begin{multline*}
P_{ij}(n_1,n_2,x,t)=\fr{\partial}{\partial x}
\mathbf{P}\left(L_1(t)=n_1,L_2(t)=n_2,\right.\\
\left. i(t)=i,j(t)=j,x(t)<x
\vphantom{L_1}\right)\,,\enskip 
 x\geqslant 0,\\ 
 j=1,\ldots,k,\enskip i=0,1,2;
\end{multline*}
\begin{gather*}
\eta_i(x)=\fr{b_i(x)}{1-B_i(x)},\ i=1,2;\enskip 
\eta_0(x)=\fr{c(x)}{1-C(x)}\,;\\
\delta_{i,j}=\begin{cases}
1,&\ i=j;\\ 
0,&\ i\ne j\,.
\end{cases}
\end{gather*}
Функции $P_{ij}(n_1,n_2,x,t)$  удовлетворяют при $x\hm>0$
системам дифференциальных уравнений:
\begin{multline}
\label{3}
\fr{\partial P_{ij}(n_1,n_2,x,t)}{\partial t}+\fr{\partial
P_{ij}(n_1,n_2,x,t)}{\partial
x}={}\\
{}=-(a_j+\eta_i(x))P_{ij}(n_1,n_2,x,t)+ {}\\
{}+
c_j\sum\limits_{l=1}^ka_l\left(p\:P_{il}(n_1-1,n_2,x,t)+{}\right.\\
\left.{}+
(1-p)P_{il}(n_1,n_2-1,x,t)\right)
\end{multline}
и краевым условиям при $x\hm=0$:
\begin{multline}
\label{5}
P_{0j}(n_1,n_2,0,t)=0,\ n_1+n_2>0;\\
P_{0j}(0,0,0,t)=\int\limits_0^{\infty}P_{0j}(0,0,x,t)\eta_0(x)\,dx+{}\\
 {}+\int\limits_0^{\infty}P_{1j}(1,0,x,t)\eta_1(x)dx+{}\\
 {}+
\int\limits_0^{\infty}P_{2j}(0,1,x,t)\eta_2(x)\,dx\,;
\end{multline}

\vspace*{-12pt}

\noindent
\begin{multline}
\label{6}
P_{1j}(n_1,n_2,0,t)+P_{2j}(n_1,n_2,0,t)={}\\
{}=\int\limits_0^{\infty}P_{1j}(n_1+1,n_2,x,t)\eta_1(x)\,dx+{}\\
{}+
\int\limits_0^{\infty}P_{2j}(n_1,n_2+1,x,t)\eta_2(x)\,dx+{}\\
{}+\int\limits_0^{\infty}P_{0j}(n_1,n_2,0,t)\eta_0(x)\,dx\,.
\end{multline}

Будем предполагать, что в~начальный момент времени $t\hm=0$ система
свободна от требований, а~с~начала профилактики прибора прошло
случайное время с~заданным распределением с~плотностью $d(x).$
Таким образом,
\begin{align*}
P_{ij}\left(n_1,n_2,x,0\right)&=0,\ i=1,2;
\\
P_{0j}\left(n_1,n_2,x,0\right)&=c_jd(x)\delta_{n_1+n_2,0},\ \
j=1,\ldots,k\,.
\end{align*}
Положим
\begin{multline*}
p_{ij}\left(z_1,z_2,x,s\right)={}\\
{}=\sum\limits_{n_1=0}^{\infty}
\sum\limits_{n_2=0}^{\infty}z_1^{n_1}z_2^{n_2}\!
\int\limits_0^{\infty}e^{-st}P_{ij}(n_1,n_2,x,t)\,dt\,;
\end{multline*}
$$
  \psi(s)=\int\limits_0^{\infty}e^{-sx}\,dx
  \int\limits_0^{\infty}\fr{c(u+x)d(u)}{1-C(u)}\,du\,.
$$
Тогда, учитывая начальные условия,  из \eqref{3}
получаем:
\begin{multline}
\label{7} 
\fr{\partial p_{ij}(z_1,z_2,x,s)}{\partial x}={}\\
{}=-\left(s+a_j+\eta_i(x)\right)p_{ij}
\left(z_1,z_2,x,s\right)+{}\\
{}+c_j\left(pz_1+(1-p)z_2\right)
\sum\limits_{l=1}^ka_lp_{il}\left(z_1,z_2,x,s\right),\\ 
i=1,2;
\end{multline}

\vspace*{-12pt}

\noindent
\begin{multline}
\label{8} 
\fr{\partial p_{0j}(z_1,z_2,x,s)}{\partial x}={}\\
{}=-\left(s+a_j+\eta_0(x)\right)p_{0j}\left(z_1,z_2,x,s\right)+{}\\
{}+c_j\left(pz_1+(1-p)z_2\right)\sum\limits_{l=1}^ka_lp_{0l}\left(z_1,z_2,x,s\right)+{}\\
{}+ c_jd(x).
\end{multline}
Решения \eqref{7} и~\eqref{8} имеют вид:
\begin{multline}
\label{9}
p_{ij}\left(z_1,z_2,x,s\right)=\left(1-B_i(x)\right)c_j\times{}\\
{}\times \sum\limits_{m=1}^k\fr{\gamma_i^{(m)}(z_1,z_2,s)}{\mu_m(z_1,z_2)+a_j}\,
e^{-(s-\mu_m(z_1,z_2))x}\,,\\
 i=1,2\,,
\end{multline}
\vspace*{-12pt}

\noindent
\begin{multline}
\label{10}
p_{0j}\left(z_1,z_2,x,s\right)={}\\
{}=\left(1-C(x)\right)
c_j\!\!\sum\limits_{m=1}^k\!\! e^{-(s-\mu_m(z_1,z_2))x}\!
\!\left(\!
\vphantom{\int\limits_{l=1}^k}
\delta^{(m)}\left(z_1,z_2,s\right)+{}\right.\\
%\left.
{}+\alpha_m^{-1}\left(z_1,z_2\right)
\prod\limits_{l=1}^k
\left(\mu_m\left(z_1,z_2\right)+a_l\right)\times{}\\
\left.{}\times \int\limits_0^x\!
e^{(s-\mu_m(z_1,z_2))u}
\fr{d(u)}{1-C(u)}\,du
\right)
\!\Bigg/ \!\left(\mu_m\left(z_1,z_2\right)+{}\right.\\
\left.{}+a_j\right)\,,
\end{multline}
где функции $\gamma_i^{(m)}(z_1,z_2,s)$  и~$\delta^{(m)}(z_1,z_2,s)$ являются
произвольными функциями указанных переменных и~определяются из
краевых условий. Из~\eqref{5} и~\eqref{6} получаем:
\begin{multline}
\label{11}
p_{1j}\left(z_1,z_2,0,s\right)+p_{2j}\left(z_1,z_2,0,s\right)={}\\
{}=z_1^{-1}\int\limits_0^{\infty}p_{1j}\left(z_1,z_2,x,s\right)\eta_1(x)\,dx+{}
\\
+z_2^{-1}\int\limits_0^{\infty}p_{2j}\left(z_1,z_2,x,s\right)\eta_2(x)\,dx+{}\\
{}+
\int\limits_0^{\infty}p_{0j}\left(z_1,z_2,x,s\right)\eta_0(x)\,dx
-p_{0j}\left(z_1,z_2,0,s\right)\,.
\end{multline}
Заметим, что $p_{0j}(z_1,z_2,0,s)$ не зависит от $z_1$ и~$z_2,$ т.\,е.\
$p_{0j}(z_1,z_2,0,s)\hm=q_j(s).$ 
Подставляя~\eqref{9} и~\eqref{10} в~\eqref{11}, получаем:
\begin{multline}
\label{12}
\gamma_1^{(m)}\left(z_1,z_2,s\right)\left(1-z_1^{-1}\beta_1(s-\mu_m(z_1,z_2))\right)+{}\\
{}+
\gamma_2^{(m)}(z_1,z_2,s)\left(1-z_2^{-1}\beta_2(s-\mu_m(z_1,z_2))\right)={}\\
{} =
\delta^{(m)}\left(z_1,z_2,s\right)\left(\gamma\left(s-\mu_m\left(z_1,z_2\right)\right)-1\right)+{}\\
{}+
\alpha_m^{-1}\left(z_1,z_2\right)\prod\limits_{l=1}^k
\left(\mu_m\left(z_1,z_2\right)+a_l\right)\psi\left(s-\mu_m(z_1,z_2)\right),\\
j=1,\ldots,k.
\end{multline}
В силу леммы~1 левая часть~\eqref{12} обращается в~0 при
$z_1\hm=z_{1m}(s)$ и~$z_2\hm=z_{2m}(s)$, $m\hm=1,\ldots,k.$ Следовательно,
\begin{multline}
\label{13}
\delta^{(m)}\left(z_{1m}(s),z_{2m}(s),s\right)={}\\
{}=\fr{\psi(s-\lambda_m(s))}{\alpha_m(z_{1m}(s),z_{2m}(s))
(1-\gamma(s-\lambda_m(s)))}\times{}\\
{}\times \prod\limits_{l=1}^k\left(\lambda_m(s)+a_l\right).
\end{multline}
Из \eqref{10} следует, что
$$
q_j(s)=c_j\sum\limits_{m=1}^k\fr{\delta^{(m)}(z_1,z_2,s)}{\mu_m(z_1,z_2)+a_j},\
j=1,\ldots,k .
$$
Решая эту систему уравнений относительно
$\delta^{(m)}(z_1,z_2,s),$ получаем:
\begin{multline}
\label{n1}
\delta^{(m)}(z_1,z_2,s)=\left(pz_1+(1-p)z_2\right)\times{}\\
{}\times
\fr{\prod\nolimits_{j=1}^k(\mu_m(z_1,z_2)+a_j)}
{\alpha_m(z_1,z_2)}\sum\limits_{l=1}^k\frac{a_lq_l(s)}{\mu_m(z_1,z_2)+a_l}.
\end{multline}
Подставляя в~\eqref{n1} $z_1\hm=z_{1m}(s)$ и~$z_2\hm=z_{2m}(s),$ имеем:
\begin{multline}
\label{14}
\delta^{(m)}\left(z_{1m}(s),z_{1m}(s),s\right)={}\\
{}=
\left(pz_{1m}(s)+(1-p)z_{2m}(s)\right)\times{}\\
{}\times
\fr{\prod\nolimits_{j=1}^k
(\lambda_m(s)+a_j)}{\alpha_m(z_{1m}(s),z_{1m}(s))}
\sum\limits_{l=1}^k\fr{a_lq_l(s)}{\lambda_m(s)+a_l}\,.
\end{multline}
Сравнивая два представления~\eqref{13} в~\eqref{14} для
$\delta^{(m)}(z_m(s),s),$ получаем систему уравнений для~$q_l(s)$:
\begin{multline*}
\sum\limits_{l=1}^k\fr{a_lq_l(s)}{\lambda_m(s)+a_l}={}\\
{}=\fr{\psi(s-\lambda_m(s))}{(pz_{1m}(s)+(1-p)z_{2m}(s))
(1-\gamma(s-\lambda_m(s)))},\\
m=1,\ldots,k\,,
\end{multline*}
из которой находим
\begin{multline}
\hspace*{-3pt}q_l(s)=c_l\prod\limits_{j=1}^k
\left(\lambda_l(s)+a_j\right) 
\sum\limits_{m=1}^k
%\fr
\psi(s-\lambda_m(s))\!\Bigg/ \!
\Bigg(\left(1-{}\right.\\
\left.
{}-\gamma\left(s-\lambda_m(s)\right)\right)(\lambda_m(s)+a_l)\times{}\\
{}\times \prod\limits_{n\ne m}(\lambda_m(s)-\lambda_n(s))\!\Bigg).
\label{15}
\end{multline}
Подставляя \eqref{15} в~\eqref{n1} и~учитывая~\eqref{1}, получаем:
\begin{multline*}
\delta^{(m)}(z_1,z_2,s)=\fr{(pz_1+(1-p)z_2)}{\alpha_m(z_1,z_2)}\times
\\
\times\sum\limits_{j=1}^k
\fr{\psi(s-\lambda_j(s))\prod\nolimits_{l=1}^k(\lambda_j(s)+a_l)}
{(pz_{1j}(s)+(1-p)z_{2j}(s))(1-\gamma(s-\lambda_j(s)))}\times{}\\
{}\times\prod\limits_{\nu\ne j}
\fr{\mu_m(z_1,z_2)-\lambda_{\nu}(s)}{\lambda_j(s)-\lambda_{\nu}(s)}\,.
\end{multline*}
Положим
$$
\lambda_m(z_2,s)=\mu_m\left(z_{1m}(z_2,s),z_2\right),\enskip m=1,\ldots,k\,.
$$
Подставляя в~\eqref{12} $z_1\hm=z_{1m}(z_2,s)$, имеем:
\begin{multline}
\label{1q}
\gamma_2^{(m)}\left(z_{1m}(z_2,s),z_2,s\right)={}\\
{}=\fr{\delta^{(m)}(z_{1m}(z_2,s),z_2,s)(\gamma_m(s-\lambda_m(z_2,s))-1)}
{1-z_2^{-1}\beta_2(s-\lambda_m(z_2,s))}+{}
\\
{}+\alpha_m^{-1}(z_{1m}(z_2,s),z_2)\psi(s-\lambda_m(z_2,s))
\prod\limits_{l=1}^k\left(\lambda_m(z_2,s)+{}\right.\\
\left.{}+a_l\right)\!\Bigg/\!
\left(
1-z_2^{-1}\beta_2(s-\lambda_m(z_2,s))\right).
\end{multline}
Далее, из~\eqref{9} следует:
$$
p_{2j}(z_1,z_2,0,s)=c_j\sum\limits_{m=1}^k
\fr{\gamma_2^{(m)}(z_1,z_2,s)}{\mu_m(z_1,z_2)+a_j}\,.
$$
Отсюда
\begin{multline}
\label{2q}
\gamma_2^{(m)}(z_1,z_2,s)=\fr{pz_1+(1-p)z_2}{\alpha_m(z_1,z_2)}\times{}\\
{}\times
\prod\limits_{j=1}^k(\mu_m(z_1,z_2)+a_j)
\sum\limits_{l=1}^k\fr{a_lp_{2l}(z_1,z_2,0,s)}{\mu_m(z_1,z_2)+a_l}\,.
\end{multline}
Так как $p_{2j}(z_1,z_2,0,s)$ не зависит от $z_1$, то
\begin{multline}
\label{3q}
p_{2j}\left(z_1,z_2,0,s\right)={}\\
{}=c_j
\sum\limits_{m=1}^k\fr{\gamma_2^{(m)}\left(z_{1m}(z_2,s),z_2,s\right)}{\lambda_m(z_2,s)+a_j}\,.
\end{multline}
Таким образом, соотношения~\eqref{1q}--\eqref{3q} полностью
определяют $\gamma_2^{(m)}(z_1,z_2,s)$ при любых $z_1$ и~$z_2$.
Теперь из~\eqref{12} можно найти $\gamma_2^{(m)}(z_1,z_2,s)$.

Все функции, необходимые для вычисления $p_{ij}(z_1,z_2,x,s)$,
$i\hm=0,1,2$, $j\hm=1,\ldots,k,$ найде-\linebreak\vspace*{-12pt}

\columnbreak

\noindent
ны. Искомая производящая функция
процесса $(L_1(t),L_2(t))$ равна:

\noindent
\begin{multline*}
\int\limits_0^{\infty}e^{-st}\mathbf{E}
z_1^{L_1(t)} z_2^{L_2(t)}\,dt={}\\
{}=
\sum\limits_{i=0}^2\sum\limits_{j=1}^k\int\limits_0^{\infty}p_{ij}
\left(z_1,z_2,x,s\right)\,dx\,.
\end{multline*}

\vspace*{-18pt}

{\small\frenchspacing
 {%\baselineskip=10.8pt
 \addcontentsline{toc}{section}{References}
 \begin{thebibliography}{9}
\bibitem{1-u}
\Au{Doshi B.\,T.} Queueing systems with vacations~--- a~survey~// 
Queueing Syst., 1986. Vol.~1.  P.~29--66.
\bibitem{2-u}
\Au{Takagi H.} Time-dependent analysis of $M\vert G\vert 1$ vacation models 
with exhaustive service~// Queueing Syst.,
1990. Vol.~6.  P.~369--390.
\bibitem{3-u}
\Au{Li J., Tian N., Zhang~Z.\,G. , Luh~H.\,P.} 
Analysis of the $M\vert G\vert 1$ queue with exponentially working vacations~--- 
a~matrix analytic approach~// Queueing Syst., 2009. Vol.~61.
P.~139--166.
\bibitem{4-u}
\Au{Bouman N., Borst S.\,C., Boxma~O.\,J., Leeuwaarden~J.\,S.\,H.} 
Queues with random back-offs~// Queueing Syst.,
2014. Vol.~77. P.~33--74.
\bibitem{5-u}
\Au{Ушаков~В.\,Г.} Система обслуживания с~гиперэкспоненциальным входящим потоком 
и~профилактиками прибора~// Информатика и~её применения, 2016. Т.~10. 
Вып.~2. С.~93--98.
 \end{thebibliography}

 }
 }

\end{multicols}

\vspace*{-9pt}

\hfill{\small\textit{Поступила в~редакцию 11.05.18}}

\vspace*{6pt}

%\pagebreak

%\newpage

%\vspace*{-28pt}

\hrule

\vspace*{2pt}

\hrule

%\vspace*{-2pt}

\def\tit{A~HEAD OF~THE~LINE PRIORITY QUEUE\\ WITH~WORKING VACATIONS}

\def\titkol{A head of the line priority queue with working vacations}

\def\aut{E.\,S.~Kondranin$^1$ and~V.\,G.~Ushakov$^{1,2}$}

\def\autkol{E.\,S.~Kondranin and~V.\,G.~Ushakov}

\titel{\tit}{\aut}{\autkol}{\titkol}

\vspace*{-11pt}


\noindent
$^1$Department of 
Mathematical Statistics, Faculty of Computational Mathematics and Cybernetics, 
M.\,V.~Lo\-mo-\linebreak
$\hphantom{^1}$no\-sov Moscow State University, 1-52~Leninskiye Gory, 
Moscow 119991, GSP-1, Russian Federation

\noindent
$^2$Institute of Informatics Problems, Federal Research Center 
``Computer Science and Control'' of the Russian\linebreak
$\hphantom{^1}$Academy of Sciences,  44-2~Vavilov Str., Moscow 119333, Russian Federation

\def\leftfootline{\small{\textbf{\thepage}
\hfill INFORMATIKA I EE PRIMENENIYA~--- INFORMATICS AND
APPLICATIONS\ \ \ 2018\ \ \ volume~12\ \ \ issue\ 4}
}%
 \def\rightfootline{\small{INFORMATIKA I EE PRIMENENIYA~---
INFORMATICS AND APPLICATIONS\ \ \ 2018\ \ \ volume~12\ \ \ issue\ 4
\hfill \textbf{\thepage}}}

\vspace*{3pt}



\Abste{The authors analyze the single-server queueing system with 
two types of customers, head of the line priority, hyperexponential 
input stream, and working vacations. The authors obtain the Laplace 
transform (with respect to an arbitrary point in time) of the joint 
distribution of server state, queue size, and elapsed time in that state. 
The authors restrict themselves to a~system with exhaustive service (the 
queue must be empty when the server starts a vacation) and multiple vacations. 
The queueing systems with vacations have been well studied because of their 
applications in modeling computer networks, communication, and manufacturing 
systems. For example, in many digital systems, the processor is multiplexed 
among a~number of jobs and, hence, is not available all the time to handle one job type. 
Besides such an application, theoretical interest in vacation models 
has been aroused with respect to their relationship with polling models.}

\KWE{hyperexponential input stream; working vacations; single server; 
head of the line priority; queue length}



\DOI{10.14357/19922264180405}

\vspace*{-14pt}

\Ack
\noindent
This work was supported by the Russian Foundation for Basic Research 
(project 18-07-00678).


%\vspace*{6pt}

  \begin{multicols}{2}

\renewcommand{\bibname}{\protect\rmfamily References}
%\renewcommand{\bibname}{\large\protect\rm References}

{\small\frenchspacing
 {%\baselineskip=10.8pt
 \addcontentsline{toc}{section}{References}
 \begin{thebibliography}{9}
\bibitem{1-u-1}
\Aue{Doshi, B.\,T.} 1986. Queueing systems with vacations~--- a~survey. 
\textit{Queueing Syst.} 1:29--66.
\bibitem{2-u-1}
\Aue{Takagi, H.} 1990. Time-dependent analysis of $M\vert G\vert M\vert 1$ 
vacation models with exhaustive service. \textit{Queueing Syst.} 6:369--390.
\bibitem{3-u-1}
\Aue{Li, J., N. Tian, Z.\,G.~Zhang,  and H.\,P.~Luh.} 2009. Analysis of the 
$M\vert G\vert 1$ queue with exponentially working vacations~--- 
a~matrix analytic approach. \textit{Queueing Syst.} 61:139--166.
{\looseness=1

}
\bibitem{4-u-1}
\Aue{Bouman, N., S.\,C.~Borst, O.\,J.~Boxma, and J.\,S.\,H.~Leeuwaarden.} 
2014. Queues with random back-offs. \textit{Queueing Syst.} 77:33--74.
\bibitem{5-u-1}
\Aue{Ushakov, V.\,G.} 2016. Sistema obsluzhivaniya s~gipereksponentsialnym 
vkhodyashchim potokom i~profilaktikami\linebreak pribora [Queueing system with working 
vacations and hyperexponential input stream]. 
\textit{Informatika i~ee Primeneniya~--- Inform. Appl.} 10(2):93--98.
\end{thebibliography}

 }
 }

\end{multicols}

\vspace*{-6pt}

\hfill{\small\textit{Received May 11, 2018}}

%\pagebreak

%\vspace*{-18pt}

\Contr

\noindent
\textbf{Kondranin Egor S.} (b.\ 1995)~---  MSc student, Department of 
Mathematical Statistics, Faculty of Computational Mathematics and Cybernetics, 
M.\,V.~Lomonosov Moscow State University, 1-52~Leninskiye Gory, 
Moscow 119991, GSP-1, Russian Federation; \mbox{ekondranin@yandex.ru}

\vspace*{6pt}

\noindent
\textbf{Ushakov Vladimir G.} (b.\ 1952)~--- 
Doctor of Science in physics and mathematics, professor, Department of Mathematical 
Statistics, Faculty of Computational Mathematics and Cybernetics, 
M.\,V.~Lomonosov Moscow State University, 1-52~Leninskiye Gory, Moscow 119991, 
GSP-1, Russian Federation; 
senior scientist, Institute of Informatics Problems, Federal Research Center 
``Computer Science and Control'' of the Russian Academy of Sciences, 
44-2~Vavilov Str., Moscow 119333, Russian Federation; \mbox{vgushakov@mail.ru}
\label{end\stat}

\renewcommand{\bibname}{\protect\rm Литература}         %4 
\def\stat{shestakov+vor}

\def\tit{АСИМПТОТИЧЕСКАЯ НОРМАЛЬНОСТЬ И~СИЛЬНАЯ СОСТОЯТЕЛЬНОСТЬ ОЦЕНКИ РИСКА ПРИ~ИСПОЛЬЗОВАНИИ FDR-ПОРОГА В УСЛОВИЯХ СЛАБОЙ ЗАВИСИМОСТИ}

\def\titkol{Асимптотическая нормальность и~сильная состоятельность оценки риска при~использовании FDR-порога} % в~условиях слабой зависимости}

\def\aut{М.\,О.~Воронцов$^1$, О.\,В.~Шестаков$^2$}

\def\autkol{М.\,О.~Воронцов, О.\,В.~Шестаков}

\titel{\tit}{\aut}{\autkol}{\titkol}

\index{Воронцов М.\,О.}
\index{Шестаков О.\,В.}
\index{Vorontsov M.\,O.}
\index{Shestakov O.\,V.}


%{\renewcommand{\thefootnote}{\fnsymbol{footnote}} \footnotetext[1]
%{Работа 
%выполнена при поддержке Программы развития МГУ, проект №\,23-Ш03-03. При анализе 
%данных использовалась инфраструктура Центра коллективного пользования 
%<<Высокопроизводительные вычисления и~большие данные>> 
%(ЦКП <<Информатика>>) ФИЦ ИУ РАН (г.~Москва)}}


\renewcommand{\thefootnote}{\arabic{footnote}}
\footnotetext[1]{Московский государственный университет 
имени~М.\,В.~Ломоносова, факультет вычислительной математики и~кибернетики;  
Московский центр фундаментальной и~прикладной математики, \mbox{m.vtsov@mail.ru}}
\footnotetext[2]{Московский государственный университет 
имени М.\,В.~Ломоносова, факультет вычислительной математики и~кибернетики; 
Федеральный исследовательский центр <<Информатика и~управление>> Российской 
академии наук; Московский центр фундаментальной и~прикладной математики, 
\mbox{oshestakov@cs.msu.ru}}


\vspace*{-12pt}





\Abst{Рассматривается подход к~решению задачи удаления шума в~большом массиве 
разреженных данных, основанный на методе контроля средней доли ложных отклонений 
гипотез (False Discovery Rate, FDR). Данный подход эквивалентен процедурам 
пороговой обработки, обнуляющим компоненты массива, значения которых не 
превосходят некоторого заданного порога.  Наблюдения в~модели считаются слабо 
зависимыми. Для контроля степени зависимости используются ограничения на 
коэффициент сильного перемешивания и~максимальный коэффициент корреляции. 
В~качестве меры эффективности рассматриваемого подхода используется 
среднеквадратичный риск. Вычислить значение риска можно только на тестовых 
данных, поэтому в~работе рассматривается его статистическая оценка и~исследуются 
ее свойства. Показана асимптотическая нормальность и~сильная состоятельность 
оценки риска при использовании FDR-по\-ро\-га в~условиях слабой зависимости в~данных.}

\KW{пороговая обработка; множественная проверка гипотез; 
оценка риска}

\DOI{10.14357/19922264240309}{ZOQVTO}
  
%\vspace*{-6pt}


\vskip 10pt plus 9pt minus 6pt

\thispagestyle{headings}

\begin{multicols}{2}

\label{st\stat}



\section{Введение}

Во многих прикладных областях возникает задача обработки больших массивов 
зашумленных данных. Примерами служат задачи обработки изоб\-ра\-же\-ний с~высоким 
разрешением~\cite{FDRImage}, задачи множественной проверки гипотез, возникающие 
в~\mbox{исследованиях} в~об\-ласти генетики~\cite{MultipleTesting}, и~другие проб\-ле\-мы. 
В~связи с~этим рас\-смот\-рим модель
$$
x_i = \mu_i + z_i, \enskip i=\overline{1,n}\,,
$$
где $\mu_i\in\mathbb{R}$~--- <<полезные>> данные; $z_i \sim N(0,\sigma^2)$~--- 
шум. Задача заключается в~нахождении оценки неизвестного вектора $\mu \hm= 
(\mu_1,\ldots,\mu_n)$ как функции вектора $x \hm= (x_1,\ldots,x_n)$ и~может 
рассматриваться как задача множественной проверки гипотез о~равенстве нулю 
компонент вектора~$\mu$~\cite{AdaptingFDR}. При этом обычно предполагается, что 
вектор~$\mu$ имеет в~определенном смысле <<разреженную>> структуру, т.\,е.\ для 
<<полезных>> данных используется <<экономное>> представление.



В работе~\cite{AdaptingFDR} для решения рассматриваемой задачи в~условиях 
независимости компонент вектора~$x$ и~разреженности вектора~$\mu$ была 
предложена процедура построения оценки~$\hat{\mu}_F$ вектора~$\mu$, основанная 
на методе контроля средней доли ложных отклонений (FDR) 
гипотез при помощи алгоритма Бен\-жа\-ми\-ни--Хох\-бер\-га,
и~было проведено исследование асимптотики ее среднеквадратичного риска. 
В~работах~\cite{ZasShe17,Mathematics2020} была показана состоятельность 
и~асимптотическая нормальность оценки риска данной процедуры. Аналогичные 
результаты для других методов построения~$\hat{\mu}_F$ получены в~работах~\cite{Shestakov2021-1,Shestakov2021-2,Shestakov2022}.

В то же время в~определенных приложениях, например  при анализе полученных 
в~результате использования ДНК-мик\-ро\-чи\-пов данных~\cite{ResultsOnFDRUnderDependence}, исследовании геофизических процессов 
и~анализе помех\linebreak в~телекоммуникационных каналах, условие незави\-си\-мости компонент 
вектора $x$ может не выполняться. Ранее в~работах~\cite{VorontsovShestakov2023,Vorontsov2024} была \mbox{исследована} асимп\-то\-ти\-ка 
среднеквадратичного риска оценки~$\hat{\mu}_F$ \mbox{в~случае}, когда~$\mu$ принадлежит 
одному из классов разреженности
$$
l_0[\eta] = \left\{\mu\,:\, ||\mu||_0 \leq \eta n\right\}, \enskip \eta \in 
(0,1),
$$

\vspace*{-12pt}

\noindent
\begin{multline*}
m_p[\eta] \equiv{}\\
{}\equiv \left\{\mu \in \mathbb{R}^n : |\mu|_{(k)} \leq \eta n^{1/p} 
k^{-1/p},\ k=\overline{1,n}\right\}, \\
 p\in(0, 2),
\end{multline*}
а компоненты вектора~$x$ слабо зависимы~--- имеют достаточно быстро убывающий 
коэффициент сильного перемешивания~\cite{Bosq}

\noindent
\begin{multline*}
\alpha(k) = \sup\limits_{1\leq m\leq n}\alpha\left(\sigma(x_i, i\leq m), 
\sigma(x_i, i\geq m+k)\right), \\ 
k=\overline{1,n-1}\,,
\end{multline*}
где символом $\sigma(x_i, i\in I)$ обозначена сиг\-ма-ал\-геб\-ра, порожденная 
множеством случайных величин $\{x_i, i \hm\in I\}$, а~мера  $\alpha(\cdot, \cdot)$ 
близости двух сиг\-ма-ал\-гебр определяется как
$$
\alpha(\mathcal{B},\mathcal{C}) = \sup\limits_{B\in\mathcal{B}, 
C\in\mathcal{C}} \left|\p(BC)-\p(B)\p(C)\right|.
$$

В настоящей работе показана асимптотическая нормальность и~сильная 
состоятельность оценки риска при применении FDR-про\-це\-ду\-ры в~случае, когда 
компоненты вектора~$x$ слабо зависимы, а~$\mu$ принадлежит одному из классов 
раз\-ре\-жен\-ности: 
$l_0[\eta]$ или $m_p[\eta]$.


\section{Обработка вектора данных с~помощью FDR-процедуры}

Широким классом методов построения оценки~$\hat{\mu}$ стала пороговая обработка 
вектора~$x$ с~некоторым порогом~$T$. Различают жесткую пороговую обработку, при 
которой полагается
\begin{equation*}
\left(\hat{\mu}\right)_i  = p_H(x_i,T) \equiv
 \begin{cases}
   x_i, & |x_i| > T\,;\\
   0, & |x_i| \leq T\,,
 \end{cases}
\end{equation*}
и мягкую пороговую обработку, для которой
\begin{equation*}
(\hat{\mu})_i  = p_S(x_i,T) \equiv
 \begin{cases}
   x_i-T, & \hphantom{\vert\vert}x_i > T;\\
   x_i+T, & \hphantom{\vert\vert}x_i <- T;\\
   0, & |x_i| \leq T.
 \end{cases}
\end{equation*}
Среднеквадратичный риск подобных процедур определяется как
\begin{equation}
\label{riskDef}
R(T) = {\mathsf E} ||\hat{\mu}-\mu||^2 = \sum\limits_{i=1}^n {\mathsf E} \left((\hat{\mu})_i-
\mu_i\right)^2.
\end{equation}
Обозначим через~$T_m$ наилучшее значение порога:
$$
T_m : \, R(T_m) = \min\limits_{T} R(T).
$$

Предложенная в~\cite{AdaptingFDR} процедура заключается в~жесткой пороговой 
обработке компонент вектора~$x$ с~порогом $\hat{t}_F \hm= \hat{t}_F(x)$, и~ее 
результат~--- оценка $\hat{\mu}_F$ вектора~$\mu$ с~компонентами $(\hat{\mu}_F)_i  
\hm= p_H(x_i,\hat{t}_F)$, где
\begin{multline*}
\hat{t}_F = \sigma z\left(\fr{q \hat{k}_F}{2n}\right), \enskip
\hat{k}_F = \max 
\left\{k \, :\, |x|_{(k)} \geq t_k \right\}, \\
 t_k = \sigma z\left(\fr{q  k}{2n}\right);
\end{multline*}
$z(\alpha)$ --- квантиль уровня $(1\hm-\alpha)$ стандартного нормального 
распределения; $|x|_{(k)}$~--- $k$-й элемент вектора, получаемого в~результате 
упорядочения вектора~$|x|$ по невозрастанию:
$$
|x|_{(1)} \geq |x|_{(2)} \geq \cdots \geq |x|_{(n)};
$$
$q\in(0;1)$~--- управ\-ля\-ющий параметр FDR-ме\-то\-да.
Далее полагается, что $q\hm\equiv q_n$ зависит от~$n$. В~\cite{AdaptingFDR} 
показано, что эта процедура эквивалентна множественной проверке гипотез 
о~равенстве нулю компонент наблюдаемого вектора. Также показано, что с~помощью 
метода штрафных функций данную процедуру можно свести к~другим видам пороговой 
обработки, в~част\-ности к~мягкой пороговой обработке.

В работах~\cite{VorontsovShestakov2023, Vorontsov2024} была исследована 
асимптотика среднеквадратичного риска~$R(\hat{t}_F)$ описанной процедуры 
в~случае, когда компоненты вектора $x$ слабо зависимы, а $\mu$ принадлежит классу 
разреженности~$\Theta_n$, где~$\Theta_n$ есть~$l_0[\eta_n]$ или~$m_p[\eta_n]$. 
Было показано, что~$R(\hat{t}_F)$ асимптотически отличается от минимаксного 
риска
$\inf\nolimits_{\hat{\mu}\hm=\hat{\mu}(x)} \sup\nolimits_{\mu\in \Theta_n} {\mathsf E} 
||\hat{\mu}-\mu||^2$
на множитель не более чем логарифмического по\-рядка.

Отметим, что в~выражении для среднеквадратичного риска~(\ref{riskDef}) 
присутствуют неизвестные величины~$\mu_i$, а~потому вычислить~$R(T_m)$ и~$T_m$ 
не представляется возможным. На практике можно пользоваться, например, следующей 
оценкой среднеквадратичного риска~\cite{Mallat}:
$$
\hat{R}(T) = \sum\limits_{i=1}^n F[x_i, T],
$$
где  
\begin{multline*}
F[x_i, T] = {}\\[3pt]
{}=\!\begin{cases}
\left(x_i^2-\sigma^2\right) \Ik(|x_i|\leq T) + \sigma^2 \Ik\left(|x_i|>T\right) &\\[3pt]
&\hspace*{-53mm}\mbox{для\ жесткой\ пороговой\ обработки};\\[3pt]
\left(x_i^2-\sigma^2\right) \Ik\left(|x_i|\leq T\right) + (\sigma^2+T^2) 
\Ik \left(|x_i|>T\right) \hspace*{-11.21576pt}&\\[3pt]
&\hspace*{-51mm}\mbox{для\ мягкой\ пороговой\ обработки}.
\end{cases}\hspace*{-7.17859pt}
\end{multline*}


\noindent
\textbf{Замечание}.\ При пороговой обработке иногда также используется так 
называемый универсальный порог $T_U\hm = \sigma \sqrt{2\ln n}$, предложенный 
в~работе~\cite{spatialAdaptation}. Исследования в~\cite{AdaptingSURE, ExactRisk} 
показали, что порог~$T_U$ в~определенном смысле максимальный, и~рас\-смат\-ри\-вать 
пороги выше него не имеет смысла. Более того, нетрудно показать, что $t_k \hm< T_U$ 
для всех~$k$ и~всех достаточно больших~$n$, в~связи с~чем всюду далее полагаем, 
что порог~$\hat{t}_F$ выбирается на отрезке $[0; T_U]$.

\section{Вспомогательные утверждения}

Кроме коэффициента сильного перемешивания~$\alpha(\cdot)$ также понадобится 
следующее понятие~\cite{Bosq}.

\smallskip

\noindent
\textbf{Определение.} %\label{defRho}
Максимальным коэффициентом корреляции~$\rho(\cdot)$ компонент вектора~$x$ 
называется
\begin{multline*}
\rho (k) \equiv \rho_n (k) = {}\\
{}=\sup\limits_{1\leq m\leq n}\rho\left(\sigma(x_i, 
i\leq m), \sigma(x_i, i\geq m+k)\right), \\
 k=\overline{1,n-1}\,,
\end{multline*}
где мера $\rho(\cdot, \cdot)$ близости двух сиг\-ма-ал\-гебр определяется как
$$
\rho(\mathcal{B},\mathcal{C}) = \sup\limits_{\substack{\xi 
\in\mathcal{L}^2(\mathcal{B}) \\
 \eta \in\mathcal{L}^2(\mathcal{C})}} 
\left|\mathrm{corr}\,(\xi, \eta)\right|.
$$


Введем обозначения:
$$
T_1 = \sqrt{2\ln \eta_n^{-p}};  \,\gamma_n = \fr{1}{\ln\ln n}; \, \kappa_n 
= \fr{n \eta_n^p T_1^{-p}}{1 - q_n - \gamma_n}; 
$$
$$ 
\kappa_n^0 = \fr{[n \eta_n]}{1 - q_n - \gamma_n} ;\, \rho^\star (k) = 
\sup\limits_{n\geq k+1} \rho(k), k \in \mathbb{N} ;
$$
$$
t_{\kappa_n} = \sigma z\left(\fr{q_n \kappa_n }{2n}\right) , \,\, t_{\kappa_n^0} 
= \sigma z\left(\fr{q_n \kappa_n^0 }{2n}\right).
$$


Следующие два утверждения показывают, что случайный порог~$\hat{t}_F$ в~случае 
$\mu\hm\in m_p[\eta_n]$ (соответственно $\mu\hm\in l_0[\eta_n]$) с~большой 
вероятностью будет не меньше~$t_{\kappa_n}$ (соответственно~$ t_{\kappa_n^0}$). 
Их  доказательства приведены в~работах~\cite{VorontsovShestakov2023, Vorontsov2024}.

\smallskip

\noindent
%\begin{lem}\label{lem5}
\textbf{Лемма~1.}\ \textit{Пусть $n^{-\delta_1} \hm\leq \eta_n^p \hm\leq n^{-\delta_2}$, 
$0\hm<\delta_2\hm<\delta_1<1$, $\mathrm{lim\,inf} q_n \ln n \hm\geq C \hm> 0$, 
$m\hm\in[1;n/2]\cap\mathbb{N}$, а $\alpha(\cdot)$~--- коэффициент сильного 
перемешивания компонент вектора~$x$. Для некоторого $N\hm\in\mathbb{N}$ при $n \hm\geq 
N$ справедливо}
\begin{multline*}
\hspace*{-3pt}\sup\limits_{\mu\in m_p[\eta_n]} \p \left(\hat{k}_F \geq \kappa_n \right) \leq 
4 n \exp\left\{-\fr{m}{256n}  \kappa_n q_n \gamma_n^2    \right\}+{}\\
{}+ 22\left(1+\fr{8n}{\kappa_n q_n \gamma_n}\right)^{1/2} n m 
\alpha\left(\left[\fr{n}{2m}\right]\right).
\end{multline*}



\smallskip

\noindent
\textbf{Лемма 2.}\ 
%\label{lem1}
\textit{Пусть $\eta_n \hm\leq b\hm<1$, $m\in[1;n/2]\cap\mathbb{N}$, а~$\alpha(\cdot)$~--- 
коэффициент сильного перемешивания компонент вектора~$x$. Для некоторого 
$N\hm\in\mathbb{N}$ при $n \hm\geq N$ справедливо}
\begin{multline*}
\sup\limits_{\mu\in l_0[\eta_n]} \p \left(\hat{k}_F \geq \kappa_n^0 \right) 
\leq{}\\
{}\leq 4 n \exp\left\{-\fr{(1-b)m}{64n}\,  \kappa_n^0 q_n \gamma_n^2    
\right\}+{}\\
{}+ 22\left(1+\fr{4n}{(1-b)\kappa_n^0 q_n \gamma_n}\right)^{1/2} n m 
\alpha\left(\left[\fr{n}{2m}\right]\right).
\end{multline*}

Следующие два утверждения доказаны в~\cite{Bosq} и~представляют собой аналоги 
неравенств Хеффдинга и~Бернштейна для слабо зависимых случайных величин.


\smallskip

\noindent
\textbf{Лемма 3.}\
\textit{Пусть для набора действительных случайных величин $X_1, \ldots, X_n$ 
с~коэффициентом сильного перемешивания $\alpha(\cdot)$ выполняется ${\mathsf E} X_i \hm=0$, 
$|X_i|\hm\leq b$, $i\hm=\overline{1,n}$. Тогда для любого целого числа $m\hm\in[1; n/2]$ 
и~любого $\eps\hm>0$ справедливо}
\begin{multline*}
\p\left(\left|\sum\limits_{i=1}^n X_i\right| > n\eps \right) \leq 4 
\exp\left\{-\fr{\eps^2 m}{8 b^2}\right\}+ {}\\
{}+
22\left(1+\fr{4b}{\eps}\right)^{1/2} m\, 
\alpha\left(\left[\fr{n}{2m}\right]\right).
\end{multline*}


\smallskip

\noindent
\textbf{Лемма 4.}\
\textit{Пусть для набора действительных случайных величин $X_1, \ldots, X_k$ 
с~коэффициентом сильного перемешивания $\alpha(\cdot)$ выполняется ${\mathsf E} X_i \hm=0$, 
$|X_i|\hm\leq b$, $i\hm=\overline{1,k}$. Тогда для любого целого числа $m\hm\in[1; k/2]$ 
и~любого $\eps\hm>0$ справедливо}
\begin{multline*}
\p\left(\left|\sum\limits_{i=1}^k X_i\right| > \eps \right) \leq 4 
\exp\left\{-\fr{\eps^2 m}{8 v^2 k^2}\right\}+{}\\
{}+ 22\left(1+\fr{4bk}{\eps}\right)^{1/2} m\, 
\alpha\left(\left[\fr{k}{2m}\right]\right),
\end{multline*}
\textit{где $p = k/(2m)$}:
\begin{multline*}
v^2 =
 \fr{b \eps}{2k} + {}\\
 {}+\fr{2}{p^2} \,  \max\limits_{ j\in[0,\,2m-1]} 
{\mathsf E} \big( ([jp]+1-jp)X_{[jp]+1} + X_{[jp]+2}+{}\\
{}+ \cdots +  X_{[(j+1)p]} + ((j+1)p-[(j+1)p])X_{[(j+1)p+1]}\big)^2.
\end{multline*}

\noindent
\textbf{Замечание.}
Если существует такое число $S \hm> 0$, что сразу для всех $i\hm\in[1;k]$  выполняется 
${\mathsf E} X_i^2 \hm\leq S^2$, то в~качестве~$v^2$ можно взять
$$
v^2 = \fr{b \eps}{2k} + 8 S^2.
$$


Д\,о\,к\,а\,з\,а\,т\,е\,л\,ь\,с\,т\,в\,о\ \ сле\-ду\-юще\-го утверж\-де\-ния приведено в~работе~\cite{AdaptingFDR}.

\smallskip

\noindent
\textbf{Лемма 5.}\ 
\textit{Для $y\leq 0{,}01$ справедливы представления}
\begin{multline}
\label{lem1eq1}
z^2(y) = 2 \ln y^{-1} - \ln \ln y^{-1} - r_2(y), \\
 r_2(y) \in [1{,}8; 3];
\end{multline}

\noindent
\begin{equation}
\label{lem1eq2}
z(y) = \sqrt{2 \ln y^{-1}} - r_1(y), \, \, r_1(y) \in [0; 1{,}5].
\end{equation}


\section{Асимптотическая нормальность оценки риска при~применении FDR-процедуры в~условиях слабой зависимости}

Перейдем к~описанию достаточных условий для асимптотической нормальности оценки 
риска $\hat{R}(\hat{t}_F)$ в~случае $\mu \hm\in m_p[\eta_n]$.

\smallskip

\noindent
\textbf{Теорема~1.}\
\textit{Пусть $\mu \hm\in m_p[\eta_n],$ $\eta_n^p \hm\in[n^{-\delta_1}; n^{-\delta_2}],$ $1/2 \hm< 
\delta_2 \hm< \delta_1<1;$ имеются такие константы $c_1, c_2>0$, что для 
коэффициента сильного перемешивания $\alpha(\cdot)$ компонент вектора $x$ 
справедливо  $\alpha(k) \hm\leq c_1 k^{-1-(5/2)\delta_1/(1-\delta_1)-c_2},$ 
$k\hm=\overline{1,n-1};$ $q_n \hm< c_3 \hm< 1;$ $\mathrm{lim\,inf} q_n \ln n \hm= c_4 \hm> 0;$ и,~кроме того, 
для максимального коэффициента корреляции $\rho(\cdot)$ компонент вектора~$x$ 
справедливо}
$$
\sum\limits_{k = 1}^{\infty} \sup\limits_{n\geq k+1} \rho(k) \equiv 
\sum\limits_{k = 1}^{\infty}  \rho^\star (k) = c_5 < \infty. 
$$
\textit{Тогда при $n \to \infty$}
$$
\fr{\hat{R}(\hat{t}_F) - R(T_m)}{C_\rho \sqrt{2n}} \Rightarrow N(0, 1),
$$
\textit{где}
$$
C_\rho = \sigma^2\sqrt{1 +  \lim\limits_{n\to\infty} \fr{1}{n} \sum\limits_{j\neq i} \mathrm{corr}^2 (x_i, x_j)}.
$$

\noindent
Д\,о\,к\,а\,з\,а\,т\,е\,л\,ь\,с\,т\,в\,о\  \
 приводится для метода мягкой пороговой обработки; в~случае жесткой пороговой 
обработки доказательство аналогично. Обозначим
$$
U(T) = \hat{R}(T) -  \hat{R}(T_m) = \sum \limits_{i=1}^n H_i(T, T_m),
$$
где
$$
H_i(T, T_m) = F[x_i, T] - F[x_i, T_m].
$$
Имеем

\vspace*{-3pt}

\noindent
\begin{multline}
\label{D00}
\hat{R}(\hat{t}_F) - R(T_m) + \hat{R}(T_m) - \hat{R}(T_m) ={}\\
{}= \hat{R}(T_m) - 
R(T_m) + U(\hat{t}_F).
\end{multline}
Покажем, что
\begin{equation}
\label{D0}
\fr{\hat{R}(T_m) - R(T_m)}{C_\rho\sqrt{2n}} \Rightarrow N(0, 1).
\end{equation}


Повторяя рассуждения из~\cite{KuShe2016_1,KuShe2016_2,Jansen}, можно показать, 
что $T_m \hm\geq t_{\kappa_n}$. Учитывая также $T_m\hm \leq T_U$, имеем 
$$
C \sqrt{\ln n} \leq T_m \leq C^\prime \sqrt{\ln n}
$$ 
для некоторых положительных констант $C$ и~$C^\prime$.

\columnbreak

В случае мягкой пороговой обработки $\hat{R}(T_m)$ представляет собой 
несмещенную оценку~$R(T_m)$, а~при жесткой пороговой обработке и~выполнении 
условий теоремы смещение стремится к~нулю при делении на $\sqrt{n}$~\cite{Mallat}.

Для дисперсии числителя~(\ref{D0}) имеем:
\begin{multline*}
{\mathsf D} \left(\hat{R}(T_m) - R(T_m)\right) = \sum\limits_{i=1}^n {\mathsf D} F[x_i, T_m] + {}\\
{}+
\sum\limits_{i=1}^n\sum\limits_{\substack{j=1 \\  j\neq i}}^n \mathrm{cov}\left( F[x_i, T_m], F[x_j, 
T_m] \right).
\end{multline*}

Поскольку $\mu \in m_p[\eta_n]$,
\begin{equation}
\left.
\begin{array}{l}
 \displaystyle\sum\limits_{i: |\mu_i| > 1/T_1} {\mathsf D} F[x_i, T_m]  \leq{}\\
 \hspace*{15mm}{}\leq  4\left(\sigma^2 + T_m^2\right)^2 n \eta_n^p 
T_1^p = o(n);
\\[6pt]
\displaystyle \sum\limits_{\substack{{i,j: \max\{|\mu_i|, |\mu_j|\} > 1/T_1,}\\{j\neq i}}}  \hspace*{-12mm}\mathrm{cov}\,(F[x_i, 
T_m],F[x_j, T_m])  \leq{}\\
\hspace*{10mm}{}\leq 16\left(\sigma^2 + T_m^2\right)^2 n \eta_n^p T_1^p c_5 = o(n). 
\end{array}
\right\}    
\label{D2}
\end{equation}
Далее, учитывая что ${\mathsf D} x_i^2 \hm= 2\sigma^4 \hm+ 4\sigma^2 \mu_i^2$, нетрудно 
убедиться, что
\begin{multline}
\label{D3}
\sum\limits_{i: |\mu_i| \leq 1/T_1}\hspace*{-4mm} {\mathsf D} F[x_i, T_m] ={}\\
{}= \sum\limits_{i: |\mu_i| \leq 1/T_1} \hspace*{-4mm} {\mathsf D} 
x_i^2 + o(n) = 2\sigma^4 n + o(n).
\end{multline}


Введем обозначение 
$$
D_n = \left\{(i,j) : \max\left\{|\mu_i|, |\mu_j|\right\}  \leq \fr{1}{T_1}\,, \enskip j\hm\neq i\right\}.
$$
 Для суммы ковариаций аналогично~(\ref{D3}) получим
\begin{multline*}
\sum\limits_{(i,j)\in D_n} \hspace*{-2mm}\mathrm{cov}\left( F[x_i, T_m], F[x_j, T_m] \right) = {}\\
{}=
\sum\limits_{(i,j)\in D_n} \hspace*{-2mm}\mathrm{cov}\left( x_i^2, x_j^2 \right) + o(n).
\end{multline*}
Воспользуемся тождеством~\cite{Eroshenko}
$$
\mathrm{cov}\left (x_i^2, x_j^2\right) = 4 {\mathsf E} x_i {\mathsf E} x_j \mathrm{cov}\left(x_i, x_j\right) + 2 \mathrm{cov}^2 \left(x_i, x_j\right)
$$
для вектора $(x_i, x_j)$, имеющего двумерное нормальное распределение. Заметим, 
что
\begin{gather*}
 \sum\limits_{(i,j)\in D_n} 4 | {\mathsf E} x_i {\mathsf E} x_j \mathrm{cov}\left(x_i, x_j\right)| \leq 8 T_1^{-2} 
\sigma^2 n c_5 = o(n);
\\
\sum\limits_{(i,j)\in D_n} 2 \mathrm{cov}^2 (x_i, x_j)  = 2\sigma^4 \sum\limits_{(i,j)\in D_n} 
\mathrm{corr}^2 (x_i, x_j). 
\end{gather*}
Более того, поскольку  %< 4 \sigma^2 n c_5.$$
\begin{equation*}
\sum\limits_{\substack{{i,j: \max\{|\mu_i|, |\mu_j|\} > 1/T_1} \\ {j\neq i}}}
\hspace*{-10mm}\mathrm{corr}^2 (x_i, x_j)  
\leq  4 n \eta_n^p T_1^p c_5 =  o(n),
\end{equation*}
имеем
\begin{multline*}
\sum\limits_{(i,j)\in D_n} \mathrm{corr}^2 (x_i, x_j) ={}\\
{}= \sum\limits_{j\neq i} \mathrm{corr}^2 (x_i, x_j) 
+o(n)= c_6 n + o(n),
\end{multline*}
где
$$
c_6 = \lim\limits_{n\to\infty} \fr{1}{n} \sum\limits_{j\neq i} \mathrm{corr}^2 (x_i, x_j) 
\leq 2 c_5.
$$
Полагая $C_\rho \hm= \sigma^2\sqrt{1 + c_6}$, получим, наконец,
\begin{equation}
\label{D1}
{\mathsf D} \left(\hat{R}(T_m) - R(T_m)\right)  =  2 n C_\rho^2 + o(n).
\end{equation}
Заметим, что из~(\ref{D2}), (\ref{D3}) и~(\ref{D1}) следует, что
\begin{equation}
\label{D5}
\sup\limits_{n} \fr{\sum\nolimits_{i=1}^n {\mathsf D} F[x_i, T_m]}{V_n^2} < \infty\,,
\end{equation}
где 
$$
V_n^2 = {\mathsf D} \sum\limits_{i=1}^n \left(F[x_i, T_m] \hm- {\mathsf E} F[x_i, T_m]\right).
$$
Кроме того, поскольку $F[x_i, T_m]$ по модулю ограничены величиной $\sigma^2 \hm+ 
T_m^2$, выполнено условие Линдеберга: для любого $\eps\hm>0$ при $n \hm\to \infty$
\begin{multline}
\label{D6}
\!\!\!\fr{1}{V_n^2}\sum\limits_{i=1}^n {\mathsf E} \left( \!\left( F\left[x_i, T_m\right]\! -\! {\mathsf E} F\left[x_i, T_m\right]\right)^2 
\Ik \left(\vert F\left[x_i, T_m\right] -{}\right.\right.\hspace*{-2.69505pt}\\
\left.\left.{}- {\mathsf E} F\left[x_i, T_m\right]\vert >\eps V_n\right)\!
\vphantom{\left( F\left[x_i, T_m\right]\! -\! {\mathsf E} F\left[x_i, T_m\right]\right)^2}
\right) 
\to  0\,.
\end{multline}
Из~(\ref{D1})--(\ref{D6}), очевидного неравенства
$$ 
\lim\limits_{k\to\infty} \sup\limits_{n\geq k+1}\rho(k) \equiv 
\lim\limits_{k\to\infty} \rho^\star (k)  < 1
$$
 и~центральной предельной теоремы для сильно перемешанных случайных величин~\cite{Peligrad} следует~(\ref{D0}).

Перейдем к~доказательству того, что $U(\hat{t}_F) \, n^{-1/2} \overset{\, \p \, }{\to} 0$.
Всюду далее, не ограничивая общности, полагаем $\sigma=1$. 
Введем обозначения:

\noindent
\begin{align*}
S_1(T) &= \sum\limits_{i: |\mu_i| > 1/T_1} H_i(T, T_m); \\
S_2(T) &= \sum\limits_{i: |\mu_i| \leq 1/T_1} H_i(T, T_m); 
\\
N_1(a, b) &= \sum\limits_{i: |\mu_i| > 1/T_1} \Ik (a<|x_i|\leq b); \\ 
N_2(a, b) &= \sum\limits_{i: |\mu_i| \leq 1/T_1} \Ik (a<|x_i|\leq b);
\end{align*}

\noindent
\begin{align*}
Z_l(T) &= S_l(T) - {\mathsf E} S_l(T),\enskip l = 1,2\,; \\  
d_n &= \fr{T_U -  t_{\kappa_n}}{n};\\
T_j^{\prime} &= t_{\kappa_n}+j d_n,\enskip j = \overline{0,n-1}\,.
\end{align*} 

\vspace*{-3pt}

\noindent
Для произвольного $\eps>0$

\vspace*{-3pt}

\noindent
\begin{multline}
\p \left( \fr{|U(\hat{t}_F)|}{\sqrt{n}}> 4\eps \right) \leq 
\p\left(\hat{t}_F \leq t_{\kappa_n}\right) + {}\\
{}+\p \left(\fr{\sup\nolimits_{T\in 
[t_{\kappa_n}, T_U]} |U(T)|}{\sqrt{n}}>4\eps \right)\leq  {}\\
{}\leq \p\left(\hat{t}_F \leq t_{\kappa_n}\right) + \p\left(\fr{\sup\nolimits_{T\in 
[t_{\kappa_n}, T_U]} |{\mathsf E} U(T)|}{\sqrt{n}}>\eps\right)+{}\\
{}+ \p \left(\sup\limits_{T\in [t_{\kappa_n}, T_U]} |Z_1(T)| > 
\eps\sqrt{n}\right) +{}\\
{}+ \p \left(\sup\limits_{j \in [0, n-1]} |Z_2(T_j^{\prime})| > 
\eps\sqrt{n}\right) +{}\\
{}+ \p \left(\sup\limits_{\substack{j \in [0, n-1] \\
 T\in [T_j^{\prime},T_j^{\prime}+d_n]}} |Z_2(T)-Z_2(T_j^{\prime})| > \eps\sqrt{n}\right).
\label{M1}
\end{multline}
Заметим, что $\gamma_n\hm > \ln^{-1} n$, $\kappa_n\hm > n \eta_n^p \ln ^{-1} n \hm\geq 
n^{1-\delta_1} \ln ^{-1} n$ и~$q_n\hm > c_4 \ln ^{-1} n /2$ для всех достаточно 
больших~$n$.
Для первого слагаемого в~(\ref{M1}) по лемме~1 с~$m \hm= n^{\delta_1} \ln 
^7 n$ для  больших~$n$ имеем

\vspace*{-3pt}

\noindent
\begin{multline}
\label{M1next}
\p\left(\hat{t}_F \leq t_{\kappa_n}\right)  = \p \left(\hat{k}_F \geq \kappa_n 
\right) \leq 4 n e^{-\ln^2 n} + {}\\
{}+n^{1+(3/2)\,\delta_1} \ln^9 n \, 
\alpha\left(\left[\fr{n^{1-\delta_1}}{\ln^{7} n}\right]\right) = o(1)
\end{multline}
при $n\to\infty$. 
Для оценки второго слагаемого в~(\ref{M1}) заметим, что при $T \hm\in 
[t_{\kappa_n}, T_U]$ справедливо
\begin{equation}
\label{M2}
{\mathsf E} H_i(T, T_m) \leq T_U^2 + 1.
\end{equation}
Если же кроме $T \hm\in [t_{\kappa_n}, T_U]$ также выполнено $|\mu_i| \hm\leq T_1^{-1}$, то

\vspace*{-6pt}

\noindent
\begin{multline*}
|{\mathsf E} H_i (T, T_m)| \leq 2 T_U^2 \, \p \left(|x_i| > t_{\kappa_n}\right) \leq {}\\
{}\leq2 
T_U^2 \, \p \left(|x_i-\mu_i| > t_{\kappa_n}-T_1^{-1}\right) \leq{}\\
{}\leq 2 T_U^2  \exp\left\{ -\fr{1}{2} \left(t_{\kappa_n} - T_1^{-
1}\right)^2 \right\}  \leq{}\\
{}\leq
 4 (\ln n)  \exp\left\{ -\fr{1}{2} 
\left(z\left(\fr{q_n\kappa_n}{2n}\right)\right)^2 + t_{\kappa_n} T_1^{-
1}\right\},
\end{multline*}

\vspace*{-2pt}

\noindent
где использовано неравенство 

\noindent
$$
2(1-\Phi(x))\hm \leq \fr{e^{-x^2/2}}{x}
$$

\pagebreak


\noindent
 для $x\hm\geq 0$ 
($\Phi(x)$~--- функция распределения $N(0,1)$). Рас\-смот\-рим выражение 
в~экспоненте. Второе слагаемое не превышает $1\hm+o(1)$ при $n\hm\to\infty$, поскольку 
$t_{\kappa_n} \hm\leq T_1 (1+o(1))$ при $\sigma\hm=1$, что нетрудно получить из 
определения~$t_{\kappa_n}$, пред\-став\-ле\-ния~(\ref{lem1eq2}) и~ограничения на~$q_n$ 
из формулировки тео\-ре\-мы. Для первого слагаемого, используя пред\-став\-ле\-ние~(\ref{lem1eq1}) 
и~ограничения, наложенные на~$q_n$, при больших~$n$ получим
\begin{multline*}
-\fr{1}{2}\left(z\left(\fr{q_n \kappa_n}{2n}\right)\right)^2 \leq - \ln 
\fr{2n (1-q_n-\gamma_n)}{q_n n \eta_n^p T_1^{-p}} + {}\\
{}+\fr{1}{2} \ln 
\left((1+o(1)) \ln \eta_n^{-p}\right) + \fr{3}{2} \leq{}\\
{}\leq \ln \fr{c_3}{1-c_3} + \ln \eta_n^p + \ln T_1^{-p} + \ln T_1 + 
\fr{3}{2}+ o(1).
\end{multline*}
Из приведенных соотношений следует, что с~некоторой константой $c_7 = c_7(c_3, 
p, \delta_1, \delta_2, c_4)$
\begin{equation}\label{M3}
\sup\limits_{\substack{i: |\mu_i| \leq 1/T_1 \\ T\in [t_{\kappa_n}, T_U]}} |{\mathsf E} 
H_i (T, T_m)|  \leq c_7 (\ln n)^{(3-p)/2}\eta_n^p.
\end{equation}
Из (\ref{M2}) и~(\ref{M3}) с~учетом $\delta_2 \hm> 1/2$ следует
\begin{multline*}
\sup\limits_{T\in [t_{\kappa_n}, T_U]} |{\mathsf E} U(T)| \leq{}\\
{}\leq 
 n\eta_n^p T_1^p 
(T_U^2+1) + c_7 (\ln n)^{(3-p)/2} n \eta_n^p = o(\sqrt{n})
\end{multline*}
при $n\to\infty$, а следовательно, для любого $\eps\hm>0$ второе слагаемое в~(\ref{M1}) обращается в~ноль для всех достаточно больших~$n$.

Далее, поскольку при $T \hm\leq T_U$ и~$\sigma\hm=1$
$$
|H_i(T, T_m) - {\mathsf E} H_i(T, T_m)| \leq 2 (T_U^2 +2), \enskip i=\overline{1, n}\,,
$$
а число слагаемых в~$Z_1(T)$ не превосходит $n\eta_n^p T_1^p$, имеем
$$
\sup\limits_{T\in [t_{\kappa_n}, T_U]} |Z_1(T)|  \leq 2 n\eta_n^p T_1^p (T_U^2 
+2) = o(\sqrt{n})
$$
при $n\to\infty$, а следовательно, для любого $\eps\hm>0$ и~третье слагаемое в~(\ref{M1}) обращается в~ноль для всех достаточно больших~$n$.

Перейдем к~оценке четвертого слагаемого в~(\ref{M1}). Аналогично~(\ref{M3}) 
можно получить:
\begin{multline}
\label{M10}
\!\!\sup\limits_{\substack{i: |\mu_i| \leq 1/T_1 \\ T\in [t_{\kappa_n}, T_U]}} \!{\mathsf D} 
H_i (T, T_m)  \leq \!\sup\limits_{\substack{i: |\mu_i| \leq 1/T_1 \\ T\in 
[t_{\kappa_n}, T_U]}} \!{\mathsf E} \left(H_i (T, T_m)\right)^2  \leq{}\\
{}\leq 2 c_7 (\ln n)^{(5-p)/2} \eta_n^p.
\end{multline}
По лемме~4 с~$m \hm= \sqrt{n} (\ln n)^3$ и~$k \hm= n-[n\eta_n^p T_1^p]$ 
для четвертого слагаемого в~(\ref{M1}) имеем:

\noindent
\begin{multline}
\p \left(\sup\limits_{j \in [0, n-1]} |Z_2(T_j^\prime)| > \eps\sqrt{n}\right) 
\leq {}\\
{}\leq \sum\limits_{j \in [0, n-1]} \hspace*{-3mm}\p \left( |Z_2(T_j^\prime)| > \varepsilon\sqrt{n}\right)\leq{}\\
{}\leq 4 n \exp \left\{ - \fr{\eps^2 n^{3/2} (\ln n)^3}{n-[n\eta_n^p T_1^p]}\!\Bigg/\! \big( 8 (T_U^2+2)\eps\sqrt{n} +{}\right.\\
\left.{}+ 128 c_7 (\ln n)^{(5-p)/2} \eta_n^p  (n-
[n\eta_n^p T_1^p])\big) 
\vphantom{ \fr{\eps^2 n^{3/2} (\ln n)^3}{n-[n\eta_n^p T_1^p]}}
\right\} +{}\\
{}
+ 22 \left(1+\fr{8(T_U^2+2) (n-[n\eta_n^p T_1^p])}{\eps 
\sqrt{n}}\right)^{1/2}\times{}\\
{}\times n^{3/2} (\ln n)^3 \alpha\left(\left[\fr{n-[n\eta_n^p 
T_1^p]}{2 (\ln n)^3 \sqrt{n}}\right]\right).
\label{M5}
\end{multline}
Используя ограничения $n^{-\delta_1}\hm\leq \eta_n^p \leq n^{-\delta_2}$ 
и~$1/2\hm<\delta_2\hm<\delta_1\hm<1$, из~(\ref{M5}) получим для любого $\eps\hm>0$
$$
\p \left(\sup\limits_{j \in [0, n-1]} |Z_2(T_j^\prime)| > \eps\sqrt{n}\right) 
\to 0
$$
при $n \to \infty$.

Рассмотрим, наконец, пятое слагаемое в~(\ref{M1})). Заметим, что при $0\hm< a \hm< b$ 
справедливо
$$
|Z_2(b)-Z_2(a)| \leq 2 |N_2(a,b)-{\mathsf E} N_2(a,b)| + n (b^2-a^2).
$$
Полагая $a = T_j^\prime$, $b \hm= T \hm\in [T_j^\prime, T_j^\prime+d_n]$ для 
произвольного $j \hm\in [0, n-1]$ и~учитывая, что
$$
(T^2 - (T_j^\prime )^2) = (T - T_j^\prime)(T+ T_j^\prime ) \leq  2 d_n T_U < 2 
T_U^2 n^{-1}; 
$$

\vspace*{-12pt}

\noindent
\begin{multline*}
\p\left(T_j^\prime < |x_i| \leq T \right) \leq \p\left(T_j^\prime < |x_i| \leq 
T_j^\prime+d_n\right) <{}\\
{}< d_n < T_U n^{-1}, 
\end{multline*}
получим  оценку
$$
|Z_2(T)-Z_2(T_j^\prime)| \leq 2 N_2(T_j^\prime, T) +  3 T_U^2 .
$$
Далее, поскольку $N_2 (T_j^\prime, T) \hm\leq N_2 (T_j^\prime, T_j^\prime+d_n)$ и~${\mathsf E} N_2 (T_j^\prime, T_j^\prime+d_n) \hm< T_U^2$,
имеем
\begin{multline*}
\sup\limits_{T \in [T_j^\prime, T_j^\prime+d_n]} |Z_2(T)-Z_2(T_j^\prime)| \leq {}\\
{}\leq
2 \left|N_2 (T_j^\prime, T_j^\prime+d_n) - {\mathsf E} N_2 (T_j^\prime, 
T_j^\prime+d_n)\right| +  5 T_U^2 .
\end{multline*}
Аналогично~(\ref{M3}) показывается, что
\begin{multline}
\label{M11}
\sup\limits_{\substack{i : |\mu_i| \leq 1/T_1 \\ j \in [0, n-1]}} {\mathsf D} \Ik 
(T_j^\prime < |x_i| \leq T_j^\prime + d_n) <{}\\
{}< c_7 (\ln n)^{(1-p)/2} \eta_n^p.
\end{multline}
Пусть $n > N(\eps)$ настолько, что 
$$
\fr{\eps\sqrt{n} - 5 T_U^2}{2} > \fr{\eps \sqrt{n} }{4}\,.
$$
%
 Тогда для пятого слагаемого в~(\ref{M1}) по лемме~4 с~$m \hm= 
\sqrt{n} (\ln n)^2$ и~$k \hm= n\hm-[n\eta_n^p T_1^p]$ имеем
\begin{multline}
\p \left(\sup\limits_{\substack{j \in [0, n-1] \\ T\in 
[T_j^{\prime},T_j^{\prime}+d_n]}} |Z_2(T)-Z_2(T_j^{\prime})| > 
\eps\sqrt{n}\right) \leq{}\\
{}\leq  \sum\limits_{j \in [0, n-1]} \p \left(  \left|N_2 (T_j^\prime, 
T_j^\prime+d_n) -{}\right.\right.\\
\left.\left.{}- {\mathsf E} N_2 (T_j^\prime, T_j^\prime+d_n)\right| > \fr{\eps\sqrt{n}}{4} 
\right) \leq{}\\
{}\leq  4n \exp \left\{ -  \fr{\eps^2 n^{3/2} (\ln n)^2}{(n-[n\eta_n^p T_1^p])^{-1}}\Bigg/ 
\big( 16 \eps \sqrt{n} +{}\right.\\
\left.{}+ 64 c_7 (\ln n)^{(1-p)/2} \eta_n^p (n-[n\eta_n^p 
T_1^p]) \big) 
\vphantom{\fr{\eps^2 n^{3/2} (\ln n)^2}{(n-[n\eta_n^p T_1^p])^{-1}}}
\right\} +{}\\
{}+ 22 \left(1+\fr{16 (n-[n\eta_n^p T_1^p])}{\eps \sqrt{n}}\right)^{1/2}\times{}\\
{}\times 
n^{3/2} (\ln n)^2 \alpha\left(\left[\fr{n-[n\eta_n^p T_1^p]}{2 (\ln n)^2 
\sqrt{n}}\right]\right).
\label{M6}
\end{multline}
Используя ограничения $n^{-\delta_1}\hm\leq \eta_n^p\hm \leq n^{-\delta_2}$ 
и~$1/2\hm<\delta_2\hm<\delta_1<1$, из~(\ref{M6}) получим для любого $\eps\hm>0$
$$
\p \left(\sup\limits_{\substack{j \in [0, n-1] \\ T\in 
[T_j^{\prime},T_j^{\prime}+d_n]}} |Z_2(T)-Z_2(T_j^{\prime})| > 
\eps\sqrt{n}\right) \to 0
$$
при $n \to \infty$.

Таким образом, показано, что для любого $\eps>0$ все слагаемые в~(\ref{M1}) 
стремятся к~нулю при $n\to\infty$. Следовательно,
$$
\fr{|U(\hat{t}_F)|}{\sqrt{n}}  \overset{\, \p \, }{\to} 0 \,,
$$
что вместе с~(\ref{D0}) завершает доказательство тео\-ремы.~\hfill$\square$

\smallskip

Следующая теорема дает достаточные условия для асимптотической нормальности 
оценки риска $\hat{R}(\hat{t}_F)$ в~случае $\mu \hm\in l_0[\eta_n]$.

\smallskip

\noindent
\textbf{Теорема 2.}\ 
\textit{Пусть $\mu \hm\in l_0[\eta_n]$, $\eta_n\hm\in[n^{-\delta_1}, n^{-\delta_2}]$, $1/2\hm < 
\delta_2\hm < \delta_1\hm<1;$ имеются такие константы $c_1, c_2\hm>0$, что для 
коэффициента сильного перемешивания $\alpha(\cdot)$ компонент вектора~$x$ 
справедливо} 
\begin{gather*}
\alpha(k) \leq c_1 k^{-1-(5/2)\delta_1/(1\hm-\delta_1)\hm-c_2},\enskip 
k=\overline{1,n-1};\\
 q_n < c_3 < 1;\enskip \mathrm{lim\,inf} q_n \ln n = c_4 > 0;
\end{gather*}
\textit{для максимального коэффициента корреляции~$\rho(\cdot)$ компонент вектора~$x$ 
справедливо}
$$
\sum\limits_{k = 1}^{\infty} \sup\limits_{n\geq k+1} \rho(k) \equiv 
\sum\limits_{k = 1}^{\infty}  \rho^\star (k) = c_5 < \infty. 
$$
\textit{Тогда при $n \to \infty$}
$$
\fr{\hat{R}(\hat{t}_F) - R(T_m)}{C_\rho \sqrt{2n}} \Rightarrow N(0, 1),
$$
\textit{где}
$$
C_\rho = \sigma^2\sqrt{1 +   \lim\limits_{n\to\infty} \fr{1}{n} 
\sum\limits_{j\neq i} \mathrm{corr}^2 (x_i, x_j)}\,.
$$

\noindent
Д\,о\,к\,а\,з\,а\,т\,е\,л\,ь\,с\,т\,в\,о\  проводится аналогично доказательству теоремы~1. 
Переменная~$D_n$ теперь определяется как $D_n \hm= \{(i,j) : 
|\mu_i|\hm=|\mu_j|=0$, $j\hm\neq i\}$. Условия вида $|\mu_i|\hm<T_1^{-1}$ (вида 
$|\mu_i|\hm\geq T_1^{-1}$) заменяются условиями  $\mu_i\hm=0$ (соответственно 
$|\mu_i|\hm>0$).
Поскольку $\mu \hm\in l_0[\eta_n]$, количество~$i$ таких, что $|\mu_i|\hm>0$ 
(а~значит, и~число слагаемых в~$Z_1(T)$), не превышает~$[n \eta_n]$.

Для оценки первого слагаемого в~(\ref{M1}) используется лемма~2, 
в~которой можно взять, например, $b\hm=1/2$, а~для~$\kappa_n^0$ использовать оценку 
$\kappa_n^0 \hm> n \eta_n$. Формулы (\ref{M3}),  (\ref{M10}) и~(\ref{M11}) 
принимают вид соответственно
\begin{align*}
\sup\limits_{\substack{i: \mu_i =0 \\ T\in [t_{\kappa_n^0}, T_U]}} |{\mathsf E} H_i (T, 
T_m)| & \leq c_8 (\ln n)^{3/2} \eta_n ;
\\
\sup\limits_{\substack{i: \mu_i =0 \\ T\in [t_{\kappa_n^0}, T_U]}} {\mathsf D} H_i (T, 
T_m)  & \leq 2 c_8 (\ln n)^{5/2} \eta_n;
\\
\sup\limits_{\substack{i : \mu_i =0 \\ j \in [0, n-1]}} {\mathsf D} \Ik (T_j^\prime < 
|x_i| \leq T_j^\prime + d_n) &< c_8 (\ln n)^{1/2} \eta_n,
\end{align*}
где $c_8 = c_8(c_3,\delta_1, \delta_2, c_4)$. В~остальном доказательство 
аналогично.~\hfill$\square$

\section{Сильная состоятельность оценки риска при~применении FDR-процедуры 
в~условиях слабой зависимости}

Следующая теорема дает достаточные условия для сильной состоятельности оценки 
риска $\hat{R}(\hat{t}_F)$ в~случаях $\mu \hm\in m_p[\eta_n]$ и~$\mu \hm\in 
l_0[\eta_n]$.

\smallskip

\noindent
\textbf{Теорема 3.}
\textit{Пусть $\mu\hm \in m_p[\eta_n]$, $\eta_n^p\hm\in[n^{-\delta_1}, n^{-\delta_2}]$ либо 
$\mu \hm\in l_0[\eta_n]$, $\eta_n\hm\in[n^{-\delta_1}, n^{-\delta_2}]$; $0 \hm< \delta_2 
\hm< \delta_1<1$; имеются такие константы $c_1, c_2\hm>0$, что для коэффициента 
сильного перемешивания $\alpha(\cdot)$ компонент вектора~$x$ справедливо}  
$\alpha(k) \hm\leq c_1 k^{-2-(7/2)\delta_1/(1\hm-\delta_1)\hm-c_2}$, $k\hm=\overline{1,n-1}$; 
$q_n \hm< c_3 \hm< 1$; $\mathrm{lim\,inf} q_n \ln n \hm= c_4 \hm> 0$. \textit{Тогда при} $n \hm\to \infty$
$$
\fr{\hat{R}(\hat{t}_F) - R(T_m)}{n} \rightarrow 0 \, \, \,\textit{п.~в.}
$$


\noindent
Д\,о\,к\,а\,з\,а\,т\,е\,л\,ь\,с\,т\,в\,о\,.  Воспользуемся представлением~(\ref{D00}).

Покажем, что $(\hat{R}(T_m)-R(T_m))n^{-1}\hm \to 0$ п.~в.\ при $n\hm\to\infty$. 
При мягкой пороговой обработке ${\mathsf E} \hat{R}(T_m) \hm= R(T_m)$, а~при жесткой 
пороговой обработке
\begin{multline*}
\fr{\hat{R}(T_m)-R(T_m)}{n} = {}\\
{}=\fr{\hat{R}(T_m)-{\mathsf E} \hat{R}(T_m)}{n} 
+\fr{{\mathsf E}\hat{R}(T_m)-R(T_m)}{n}\,,
\end{multline*}
где второе слагаемое стремится к~нулю при $n\to\infty$ \cite{Mallat}. 
Следовательно, достаточно показать, что $(\hat{R}(T_m)\hm-{\mathsf E}\hat{R}(T_m))n^{-1} \hm\to 0$ п.~в.

Полагая в~лемме~3 $X_i \hm= F[x_i, T_m] \hm- {\mathsf E} F[x_i, T_m]$, $b \hm= 
2(\sigma^2\hm+T_m^2)$ и~$m \hm= n^{1/4}$ и~учитывая ограничения на $\alpha(\cdot)$ из 
условия, нетрудно убедиться, что для всех~$n$
$$
\p \left(\left| \fr{\hat{R}(T_m)-{\mathsf E} \hat{R}(T_m)}{n}\right| >\eps \right) 
\leq \fr{c_5}{n^{1+c_6}}\,, 
$$
где константы $c_5$, $c_6$ положительны. Отсюда
$$
\sum\limits_{n=1}^{\infty}\p \left(\left|\fr{\hat{R}(T_m)-{\mathsf E} 
\hat{R}(T_m)}{n}\right| >\eps \right) < \infty,
$$
и по теореме~1.3.4 из~\cite{Serfling2002} 
$$
\left(\hat{R}(T_m)-{\mathsf E}\hat{R}(T_m)\right)n^{-1} \to 0~\mbox{п.~в.}
$$



Покажем теперь, что  $U(\hat{t}_F) \, n^{-1}\hm \to 0$ п.~в. Доказательство 
проведено для $\mu \hm\in m_p[\eta_n]$, в~случае $\mu\hm \in l_0[\eta_n]$ 
доказательство аналогично.
Аналогично формуле~(\ref{M1}), для произвольного $\eps\hm>0$ в~терминах тео\-ре\-мы~1 имеем
\begin{multline*}
\p \left( \fr{|U(\hat{t}_F)|}{n}> 4\eps \right) \leq \p\left(\hat{t}_F 
\leq t_{\kappa_n}\right) +{}\\
{}+ \p\left(\fr{\sup\nolimits_{T\in [t_{\kappa_n}, T_U]} |{\mathsf E} 
U(T)|}{n}>\eps\right)+{}\\
{}+ \p \left(\sup\limits_{T\in [t_{\kappa_n}, T_U]} |Z_1(T)| > \eps n\right) +{}
\end{multline*}

\noindent
\begin{multline}
{}+ \p  \left(\sup\limits_{j \in [0, n-1]} |Z_2(T_j^{\prime})| > \eps n\right) +{}\\
{}+ \p \left(\sup\limits_{\substack{j \in [0, n-1] \\ T\in 
[T_j^{\prime},T_j^{\prime}+d_n]}} |Z_2(T)-Z_2(T_j^{\prime})| > \eps n\right).
\label{M1SC}
\end{multline}
Применяя рассуждения, аналогичные приведенным в~доказательстве теоремы~1, можно показать, что
$$
\sup\limits_{T\in [t_{\kappa_n}, T_U]} |{\mathsf E} U(T)| = o(n); \enskip
\sup\limits_{T\in [t_{\kappa_n}, T_U]} |Z_1(T)|  = o(n),
$$
откуда следует, что второе и~третье слагаемые в~(\ref{M1SC}) обращаются в~ноль 
для всех достаточно больших~$n$.

Для некоторых положительных констант  $c_7$ и~$c_8$ первое, четвертое и~пятое 
слагаемые  в~(\ref{M1SC}) не превышают $c_7 n^{-1-c_8}$ для всех достаточно 
боль\-ших~$n$, что можно показать с~помощью ограничения на $\alpha(\cdot)$ из 
условия и~рассуждений, аналогичных приведенным при выводе соответственно формул~(\ref{M1next}), (\ref{M5}) и~(\ref{M6}), с~тем отличием, что при применении 
леммы~4 полагается $m \hm= (\ln n)^3$.

Из доказанного следует, что
$$
\sum\limits_{n=1}^{\infty}\p \left( \fr{|U(\hat{t}_F)|}{n}> 4\eps \right) 
< \infty,
$$
и по теореме~1.3.4 из~\cite{Serfling2002} $U(\hat{t}_F) \, n^{-1} \to 0$ п.~в., 
что завершает доказательство теоремы.~\hfill$\square$



{\small\frenchspacing
 {\baselineskip=11.5pt
 %\addcontentsline{toc}{section}{References}
 \begin{thebibliography}{99}
\bibitem{FDRImage}
\Au{Krylov V.\,A., Moser~G., Serpico~S.\,B., Zerubia~J.}
False discovery rate approach to unsupervised image change detection~// IEEE 
T. Image Process., 2016. Vol.~25. No.\,10. P.~4704--4718. doi: 10.1109/TIP.2016.2593340.

\bibitem{MultipleTesting} %2
\Au{Menyhart~O., Weltz~B., Gyorffy~B.}
MultipleTesting.com: A~tool for life science researchers for multiple hypothesis 
testing correction~// PLoS One, 2021. Vol.~16. No.\,6. Art.~0245824. doi: 10.1371/journal.pone.0245824.

\bibitem{AdaptingFDR} %3
\Au{Abramovich~F., Benjamini~Y., Donoho~D., Johnstone~I.}
Adapting to unknown sparsity by controlling the false discovery rate~// Ann. Stat., 2006. Vol.~34. No.\,2. P.~584--653.
doi: 10.1214/009053606000000074.

\bibitem{ZasShe17} %4
\Au{Заспа~А.\,Ю., Шестаков~О.\,В.}
Состоятельность оценки риска при множественной проверке гипотез с~FDR-по\-ро\-гом~// 
Вестник ТвГУ. Сер. Прикладная математика, 2017. Вып.~1. С.~5--16.
doi: 10.26456/vtpmk119. EDN: YFYJXT.

\bibitem{Mathematics2020} %5
\Au{Palionnaya~S.\,I., Shestakov~O.\,V.}
Asymptotic properties of MSE estimate for the false discovery rate controlling 
procedures in multiple hypothesis testing // Mathematics, 2020. Vol.~8. No.~11. 
Art.~1913. 11~p. doi: 10.3390/ math8111913.

\bibitem{Shestakov2021-1} %6
\Au{Шестаков~О.\,В.}
Анализ несмещенной оценки среднеквадратичного риска метода блочной пороговой 
обработки~// Информатика и~её применения, 2021. Т.~15. Вып.~2. С.~30--35.
doi: 10.14357/19922264210205. EDN: DSQQAU.

\bibitem{Shestakov2021-2} %7
\Au{Шестаков~О.\,В.}
Пороговые функции в~методах подавления шума, основанных на вейв\-лет-раз\-ло\-же\-нии 
сигнала~// Информатика и~её применения, 2021. Т.~15. Вып.~3. С.~51--56.
doi: 10.14357/19922264210307. EDN: WSEAYG.

\bibitem{Shestakov2022} %8
\Au{Шестаков~О.\,В.}
Несмещенная оценка риска пороговой обработки с~двумя пороговыми значениями~// 
Информатика и~её применения, 2022. Т.~16. Вып.~4. С.~14--19.
doi: 10.14357/19922264220403. EDN: \mbox{DZBVLC}.

\bibitem{ResultsOnFDRUnderDependence} %9
\Au{Farcomeni~A.}
Some results on the control of the false discovery rate under dependence~// 
Scand. J. Stat., 2007. Vol.~34. No.\,2. P.~275--297.
doi: 10.1111/j.1467-9469.2006.00530.x.

\bibitem{VorontsovShestakov2023} %10
\Au{Воронцов~М.\,О., Шестаков~О.\,В.}
Среднеквадратичный риск FDR-про\-це\-ду\-ры в~условиях слабой за\-ви\-си\-мости~// 
Информатика и~её применения, 2023. Т.~17. Вып.~2. С.~34--40.
doi: 10.14357/19922264230205. EDN: AVJZDX.

\bibitem{Vorontsov2024} %11
\Au{Воронцов~М.\,О.}
Анализ среднеквадратичного риска при использовании методов множественной 
проверки гипотез для выбора параметров пороговой обработки в~условиях слабой 
зависимости~// Вестник Московского университета. Сер. 15: Вычислительная 
математика и~кибернетика, 2024. №\,2. С.~18--24.

\bibitem{Bosq} %12
\Au{Bosq~D.}
Nonparametric statistics for stochastic processes: Estimation and prediction.~--- 
Lecture notes in statistics ser.~--- New York, NY, USA: Springer, 1996. Vol.~110. 
188~p.

\bibitem{Mallat} %13
\Au{Mallat~S.}
A wavelet tour of signal processing.~--- New York, NY, USA: Academic Press, 1999. 
857~p.

\bibitem{spatialAdaptation} %14
\Au{Donoho~D., Johnstone~I.}
Ideal spatial adaptation via wavelet shrinkage~// Biometrika, 1994. Vol.~81. 
No.\,3. P.~425--455. doi: 10.1093/biomet/81.3.425.

\bibitem{AdaptingSURE} %15
\Au{Donoho D., Johnstone I.\,M.}
Adapting to unknown smoothness via wavelet shrinkage~// J.~Amer. Stat. Assoc., 
1995. Vol.~90. P.~1200--1224.

\bibitem{ExactRisk} %16
\Au{Marron J.\,S., Adak~S., Johnstone~I.\,M., Neumann~M.\,H., Patil~P.}
Exact risk analysis of wavelet regression~// J.~Comput. Graph. Stat., 1998. 
Vol.~7. P.~278--309. doi: 10.1080/ 10618600.1998.10474777.

\bibitem{Jansen} %17
\Au{Jansen~M.}
Noise reduction by wavelet thresholding.~-- Lecture notes in statistics ser.~--- 
New York, NY, USA: Springer, 2001. Vol.~161. 217~p.

\bibitem{KuShe2016_1} %18
\Au{Кудрявцев~А.\,А., Шестаков~О.\,В.}
Асимптотическое поведение порога, минимизирующего усредненную\linebreak вероятность ошибки 
вычисления вейв\-лет-ко\-эф\-фи\-ци\-ен\-тов~// Докл. Акад. наук, 2016. Т.~468. №\,5. 
С.~487--491.

\bibitem{KuShe2016_2} %19
\Au{Кудрявцев~А.\,А., Шестаков~О.\,В.}
Асимптотически оптимальная пороговая обработка вейв\-лет-ко\-эф\-фи\-ци\-ен\-тов в~моделях с~негауссовым распределением шума~// Докл. Акад. наук, 2016. Т.~471. №\,1. 
С.~11--15.



\bibitem{Eroshenko} %20
\Au{Ерошенко~А.\,А.}
Статистические свойства оценок сигналов и~изображений при пороговой обработке 
коэффициентов в~вейв\-лет-раз\-ло\-же\-ни\-ях: Дис.\ \ldots\ канд. физ.-мат. наук.~--- 
М.: МГУ, 2015. 82~с.

\bibitem{Peligrad} %21
\Au{Peligrad~M.}
On the asymptotic normality of sequences of weak dependent random variables~// 
J. Theor. Probab., 1996. Vol.~9. No.\,3. P.~703--715. doi: 10.1007/BF02214083.

\bibitem{Serfling2002} %22
\Au{Serfling~R.\,J.}
Approximation theorems of mathematical statistics.~--- New York, NY, USA: John Wiley \&~Sons, Inc., 2002. 371~p.

\end{thebibliography}

 }
 }

\end{multicols}

\vspace*{-6pt}

\hfill{\small\textit{Поступила в~редакцию 21.05.24}}

\vspace*{8pt}

%\pagebreak

%\newpage

%\vspace*{-28pt}

\hrule

\vspace*{2pt}

\hrule



\def\tit{ASYMPTOTIC NORMALITY AND STRONG CONSISTENCY\\ OF~RISK ESTIMATE WHEN USING THE~FDR THRESHOLD\\ UNDER WEAK DEPENDENCE CONDITION}


\def\titkol{Asymptotic normality and strong consistency of~risk estimate when using the~FDR threshold under weak dependence condition}


\def\aut{M.\,O.~Vorontsov$^{1,2}$ and~O.\,V.~Shestakov$^{1,2,3}$}

\def\autkol{M.\,O.~Vorontsov and~O.\,V.~Shestakov}

\titel{\tit}{\aut}{\autkol}{\titkol}

\vspace*{-13pt}


\noindent
$^{1}$Department of Mathematical Statistics, Faculty of Computational Mathematics and Cybernetics,
 M.\,V.~Lo\-mo-\linebreak
 $\hphantom{^1}$nosov Moscow State University, 1-52~Leninskie Gory, GSP-1, Moscow 119991, Russian Federation

\noindent
$^{2}$Moscow Center for Fundamental and Applied Mathematics, M.\,V.~Lomonosov Moscow State University,\linebreak
$\hphantom{^1}$1~Leninskie Gory, GSP-1, Moscow 119991, Russian Federation

\noindent
$^{3}$Federal Research Center ``Computer Science and Control'' of the Russian Academy of Sciences, 44-2~Vavilov\linebreak
$\hphantom{^1}$Str., Moscow 119333, Russian Federation


\def\leftfootline{\small{\textbf{\thepage}
\hfill INFORMATIKA I EE PRIMENENIYA~--- INFORMATICS AND
APPLICATIONS\ \ \ 2024\ \ \ volume~18\ \ \ issue\ 3}
}%
 \def\rightfootline{\small{INFORMATIKA I EE PRIMENENIYA~---
INFORMATICS AND APPLICATIONS\ \ \ 2024\ \ \ volume~18\ \ \ issue\ 3
\hfill \textbf{\thepage}}}

\vspace*{2pt}






\Abste{An approach to solving the problem of noise removal in a large array of sparse data is considered
 based on the method of controlling the average proportion of false hypothesis rejections (False Discovery Rate, FDR). 
 This approach is equivalent to threshold processing procedures that remove array components whose values do not exceed 
 some specified threshold. The observations in the model are considered weakly dependent. To control the\linebreak\vspace*{-12pt}}
 
 \Abstend{degree of dependence, 
 restrictions on the strong mixing coefficient and the maximum correlation coefficient are used. The mean-square risk is 
 used as a measure of the effectiveness of the considered approach. It is possible to calculate the risk value only on the test data;
  therefore, its statistical estimate is considered in the work and its properties are investigated. The asymptotic normality and
   strong consistency of the risk estimate are proved when using the FDR threshold under conditions of weak dependence in the data.}

\KWE{thresholding; multiple hypothesis testing; risk estimate}

\DOI{10.14357/19922264240309}{ZOQVTO}

%\vspace*{-12pt}


    
   %   \Ack

%\vspace*{-3pt}
%\noindent



  \begin{multicols}{2}

\renewcommand{\bibname}{\protect\rmfamily References}
%\renewcommand{\bibname}{\large\protect\rm References}

{\small\frenchspacing
 {\baselineskip=10.8pt
 \addcontentsline{toc}{section}{References}
 \begin{thebibliography}{99} 

%1
\bibitem{FDRImage-1}
\Aue{Krylov, V.\,A., G.~Moser, S.\,B.~Serpico, and J.~Zerubia.} 2016. 
False discovery rate approach to unsupervised image change detection. 
\textit{IEEE T. Image Process.} 25(10):4704--4718. doi: 10.1109/TIP.2016.2593340.

%2
\bibitem{MultipleTesting-1}
\Aue{Menyhart, O., B.~Weltz, and B.~Gyorffy.} 2021. 
MultipleTesting.com: A~tool for life science researchers for multiple hypothesis testing correction. 
\textit{PLoS One} 16(6):0245824. 
doi: 10.1371/journal.pone.0245824.

%3
\bibitem{AdaptingFDR-1}
\Aue{Abramovich, F., Y.~Benjamini, D.~Donoho, and I.\,M.~Johnstone.} 2006. 
Adapting to unknown sparsity by controlling the false discovery rate. 
\textit{Ann. Stat.} 34(2):584--653. 
doi: 10.1214/009053606000000074.


%4
\bibitem{ZasShe17-1}
\Aue{Zaspa, A.\,Yu., and O.\,V.~Shestakov.} 2017.
Sostoyatel'nost' otsenki riska pri mnozhestvennoy proverke gipotez s~FDR-porogom
 [Consistency of the risk estimate of the multiple hypothesis testing with the FDR threshold]. 
\textit{Vestnik TvGU. Ser.: Prikladnaya matematika} [Herald of Tver State University. Ser. Applied Mathematics] 1:5--16.
doi: 10.26456/vtpmk119. EDN: YFYJXT.

%5
\bibitem{Mathematics2020-1}
\Aue{Palionnaya, S.\,I., and O.\,V.~Shestakov.} 2020. 
Asymptotic properties of MSE estimate for the false discovery rate controlling procedures in multiple hypothesis testing. 
\textit{Mathematics} 8(11):1913. 11~p.
doi: 10.3390/math8111913.

%6
\bibitem{Shestakov2021-1-1}
\Aue{Shestakov, O.\,V.} 2021.
Analiz nesmeshchennoy otsenki srednekvadratichnogo riska metoda blochnoy po\-ro\-go\-voy obrabotki 
[Analysis of the unbiased mean-square risk estimate of the block thresholding method]. 
\textit{Informatika i~ee Primeneniya~--- Inform. Appl.} 15(2):30--35.
doi: 10.14357/19922264210205. EDN: DSQQAU.

%7
\bibitem{Shestakov2021-2-1}
\Aue{Shestakov, O.\,V.} 2021.
Porogovye funktsii v~metodakh podavleniya shuma, osnovannykh na veyvlet-razlozhenii signala 
[Thresholding functions in the noise suppression methods based on the wavelet expansion of the signal]. 
\textit{Informatika i~ee Primeneniya~--- Inform. Appl.} 15(3):51--56.
doi: 10.14357/19922264210307. EDN: WSEAYG.

%8
\bibitem{Shestakov2022-1}
\Aue{Shestakov, O.\,V.} 2022.
Nesmeshchennaya otsenka riska porogovoy obrabotki s dvumya porogovymi znacheniyami [Unbiased thresholding risk estimate with two threshold values]. 
\textit{Informatika i~ee Primeneniya~--- Inform. Appl.} 16(4):14--19.
doi: 10.14357/19922264220403. EDN: DZBVLC.

%9
\bibitem{ResultsOnFDRUnderDependence-1}
\Aue{Farcomeni, A.} 2007. Some results on the control of the false discovery rate under dependence. 
\textit{Scand. J. Stat.} 34(2):275--297. 
doi: 10.1111/j.1467-9469.2006.00530.x.

%10
\bibitem{VorontsovShestakov2023-1}
\Aue{Vorontsov, M.\,O., and O.\,V.~Shestakov.} 2023.
Sred\-ne\-kvad\-ra\-tich\-nyy risk FDR-protsedury v~usloviyakh slaboy za\-vi\-si\-mosti [Mean-square risk of the FDR procedure under weak dependence]. 
\textit{Informatika i~ee Primeneniya~--- Inform. Appl.} 17(2):34--40.
doi: 10.14357/19922264230205. EDN: AVJZDX.

%11
\bibitem{Vorontsov2024-1}
\Aue{Vorontsov, M.\,O.} 2024. 
RMS risk analysis when using multiple hypothesis testing select parameters of thresholding under conditions of weak dependence. 
\textit{Moscow University Computational Mathematics Cybernetics} 48:91--97. 
doi: 10.3103/S027864192470002X.

%12
\bibitem{Bosq-1}
\Aue{Bosq, D.} 1996. 
\textit{Nonparametric statistics for stochastic processes: Estimation and prediction}. 
Lecture notes in statistics ser. New York, NY: Springer Verlag. Vol.~110. 188~p.

%13
\bibitem{Mallat-1}
\Aue{Mallat, S.} 1999. 
\textit{A wavelet tour of signal processing}. New York, NY: Academic Press. 857~p.

%14
\bibitem{spatialAdaptation-1}
\Aue{Donoho, D., and I.\,M.~Johnstone.} 1994. 
Ideal spatial adaptation via wavelet shrinkage. 
\textit{Biometrika} 81(3):425--455. doi: 10.1093/biomet/81.3.425.

%15
\bibitem{AdaptingSURE-1}
\Aue{Donoho, D., and I.\,M.~Johnstone.} 1995. 
Adapting to unknown smoothness via wavelet shrinkage. 
\textit{J. Am. Stat. Assoc.} 90(432):1200--1224. doi: 10.1080/01621459. 1995.10476626.

%16
\bibitem{ExactRisk-1}
\Aue{Marron, J.\,S., S.~Adak, I.\,M.~Johnstone, M.\,H.~Neumann, and P.~Patil.} 1998. 
Exact risk analysis of wavelet regression. 
\textit{J.~Comput. Graph. Stat.} 7(3):278-309. doi: 10.1080/ 10618600.1998.10474777.

%17
\bibitem{Jansen-1}
\Aue{Jansen, M.} 2001. 
\textit{Noise reduction by wavelet thresholding}. Lecture notes in statistics ser. New York, NY: Springer Verlag. Vol.~161. 217~p.

%18
\bibitem{KuShe2016_1-1}
\Aue{Kudryavtsev, A.\,A., and O.\,V.~Shestakov.} 2016. 
Asymptotic behavior of the threshold minimizing the average probability of error in calculation of wavelet coefficients. 
\textit{Dokl. Math.} 93(3):295--299.
doi: 10.1134/S1064562416030212. EDN: WUMUEV. 

%19
\bibitem{KuShe2016_2-1}
\Aue{Kudryavtsev, A.\,A., and O.\,V.~Shestakov.} 2016. 
Asymptotically optimal wavelet thresholding in the models with non-Gaussian noise distributions. 
\textit{Dokl. Math.} 94(3):615--619.
doi: 10.1134/S1064562416060028. EDN: YUYVUP.




%20
\bibitem{Eroshenko-1}
\Aue{Eroshenko, A.\,A.} 2015. Statisticheskie svoystva otsenok signalov i~izobrazheniy pri porogovoy obrabotke ko\-ef\-fi\-tsi\-en\-tov 
v~veyvlet-razlozheniyakh 
[Statistical properties of signal and image estimates under thresholding of coefficients in wavelet decompositions]. Moscow: MSU. PhD Diss. 82~p.

%21
\bibitem{Peligrad-1}
\Aue{Peligrad, M.} 1996. 
On the asymptotic normality of sequences of weak dependent random variables. 
\textit{J. Theor. Probab.} 9(3):703--715. doi: 10.1007/BF02214083.

%22
\bibitem{Serfling2002-1}
\Aue{Serfling, R.\,J.} 2002. 
\textit{Approximation theorems of mathematical statistics}. New York, NY: John Wiley \&~Sons. 371~p.
\end{thebibliography}

 }
 }

\end{multicols}

\vspace*{-6pt}

\hfill{\small\textit{Received May 21, 2024}} 

%\vspace*{-18pt}

\Contr

\vspace*{-3pt}


\noindent
\textbf{Vorontsov Mikhail O.} (b.\ 1996)~--- PhD student, Department of Mathematical Statistics, 
Faculty of Computational Mathematics and Cybernetics, M.\,V.~Lomonosov Moscow State University, 1-52~Leninskie Gory, GSP-1, Moscow 119991, Russian Federation;  
mathematician, Moscow Center for Fundamental and Applied Mathematics, M.\,V.~Lomonosov Moscow State University, 1~Leninskie Gory, GSP-1, Moscow 119991, Russian Federation;
\mbox{m.vtsov@mail.ru}

\vspace*{6pt}

\noindent
\textbf{Shestakov Oleg V.} (b.\ 1976)~--- Doctor of Science in physics and mathematics, professor, Department of Mathematical Statistics,
 Faculty of Computational Mathematics and Cybernetics, M.\,V.~Lomonosov Moscow State University, 1-52~Leninskie Gory, GSP-1, Moscow 119991, Russian Federation; 
 senior scientist, Federal Research Center ``Computer Science and Control'' of the Russian Academy of Sciences, 44-2~Vavilov Str., Moscow 119333, 
 Russian Federation; leading scientist, Moscow Center for Fundamental and Applied Mathematics, M.\,V.~Lomonosov Moscow State University, 
 1~Leninskie Gory, GSP-1, Moscow 119991, Russian Federation; \mbox{oshestakov@cs.msu.su}


\label{end\stat}

\renewcommand{\bibname}{\protect\rm Литература}   %5
\def\stat{grusho}

\def\tit{АРХИТЕКТУРНЫЕ РЕШЕНИЯ В~ЗАДАЧЕ ВЫЯВЛЕНИЯ МОШЕННИЧЕСТВА ПРИ~АНАЛИЗЕ 
ИНФОРМАЦИОННЫХ ПОТОКОВ В~ЦИФРОВОЙ ЭКОНОМИКЕ$^*$}

\def\titkol{Архитектурные решения в~задаче выявления мошенничества при~анализе 
информационных потоков в
%~цифровой 
экономике}

\def\aut{А.\,А.~Грушо$^1$, М.\,И.~Забежайло$^2$, Н.\,А.~Грушо$^3$, 
Е.\,Е.~Тимонина$^4$}

\def\autkol{А.\,А.~Грушо, М.\,И.~Забежайло, Н.\,А.~Грушо, 
Е.\,Е.~Тимонина}

\titel{\tit}{\aut}{\autkol}{\titkol}

\index{Грушо А.\,А.}
\index{Забежайло М.\,И.}
\index{Грушо Н.\,А.}
\index{Тимонина Е.\,Е.}
\index{Grusho A.\,A.}
\index{Zabezhailo M.\,I.}
\index{Grusho N.\,A.}
\index{Timonina E.\,E.}


{\renewcommand{\thefootnote}{\fnsymbol{footnote}} \footnotetext[1]
{Работа частично поддержана РФФИ (проекты 18-29-03081 и~18-07-00274).}}


\renewcommand{\thefootnote}{\arabic{footnote}}
\footnotetext[1]{Институт проблем информатики Федерального исследовательского центра <<Информатика и~управление>> 
Российской академии наук, grusho@yandex.ru}
\footnotetext[2]{Институт проблем информатики Федерального исследовательского центра <<Информатика и~управление>> 
Российской академии наук, m.zabezhailo@yandex.ru}
\footnotetext[3]{Институт проблем информатики Федерального исследовательского центра <<Информатика и~управление>> 
Российской академии наук, info@itake.ru}
\footnotetext[4]{Институт проблем информатики Федерального исследовательского центра <<Информатика и~управление>> 
Российской академии наук, eltimon@yandex.ru}

\vspace*{-12pt}
   

 
  
  \Abst{Cформулирован подход к~исследованию некоторых видов мошенничества в~цифровой 
экономике с~использованием причинно-следственных связей. Во всех видах рассматриваемых 
мошенничеств должно наблюдаться несоответствие между целями финансовых транзакций 
и~реальной стоимостью достижения этих целей. Данные о транзакциях можно собирать, 
наблюдая информационные потоки, в~которых отражаются эти транзакции. Архитектура сбора 
данных и~их анализа может быть организована с~помощью распределенных реестров 
с~централизованным консенсусом, что позволяет создать аналог электронной бухгалтерской 
книги, фиксирующей финансово-экономическую деятельность субъектов цифровой экономики в~регионе. 
  Рассматриваемые методы выявления мошенничества основаны на противоречиях 
между действиями, описанными в~транзакциях, и~информацией, содержащейся в~планах, 
стандартах, прецедентах и~др. Рассмотрен метод, основанный на некоторой упрощенной схеме 
реализации абстрактного проекта. Для выявления противоречий необходимо проводить анализ 
от следствия к~причине, т.\,е.\ искать аномалии в~информации, описывающей порождение 
наблюдаемых следствий. 
  Показано, как в~реализации проекта можно выделять простые <<необходимые условия>> 
нарушения при\-чин\-но-след\-ст\-вен\-ных связей, т.\,е.\ множество <<необходимых условий>>, 
нарушение которых свидетельствует о наличии мошенничества. Это множество <<необходимых 
условий>> можно назвать метаданными для контроля проекта на выявление мошенничества.} 
 
 
  \KW{цифровая экономика; информационные потоки; при\-чин\-но-след\-ст\-вен\-ные связи; 
выявление мошеннических схем} 

\DOI{10.14357/19922264190204}
  
\vspace*{-4pt}


\vskip 10pt plus 9pt minus 6pt

\thispagestyle{headings}

\begin{multicols}{2}

\label{st\stat}

\section{Введение}

\vspace*{3pt}

  В работе сформулирован подход к~исследованию некоторых видов 
мошенничества в~цифровой экономике с~использованием  
при\-чин\-но-след\-ст\-вен\-ных связей. Рассматриваются три вида мошенничества, 
а именно:
  \begin{enumerate}[(1)]
\item отмыв денег; 
\item обман при выполнении договорных обязательств при реализации 
технических проектов (строительные проекты и~др.); 
\item незаконный вывод денег. 
\end{enumerate}

  Названные виды мошенничества могут быть сведены к~решению одного типа 
задач. Для отмывания денег источник должен заключать фиктивные контракты, 
в~соответствии с~которыми будут переводиться средства за заведомо ненужную 
работу и~материалы. 
  
  Мошенничество, связанное с~невыполнением договорных обязательств, связано 
со снижением качества услуг, качества и~количества закупаемых 
материалов, выполнением работ с~ненадлежащим качеством. 
  
  Вывод денег связан с~переводом средств фир\-мам-од\-но\-днев\-кам, которые 
заведомо не могут выполнить обязательства по контрактам, за которые им 
переводятся средства. 
  
  Таким образом, во всех трех видах рассматриваемых мошенничеств должно 
наблюдаться несоответствие между целями финансовых транзакций и~реальной 
стоимостью достижения этих целей. Данные о транзакциях можно собирать, 
наблюдая информационные потоки, в~которых отражаются эти транзакции. 
  
  Однако для наблюдения таких информационных потоков необходимо создавать 
архитектуру\linebreak телекоммуникационной системы, позволяющей перехватывать 
и~собирать данные о всех транзакциях. Например, такая архитектура может быть 
организована с~помощью распределенных реестров с~централизованным 
консенсусом, т.\,е.\ все информационные потоки, сформированные в~цифровой 
экономике и~несущие информацию о транзакциях, проходят через некоторый 
центральный узел, запоминающий их в~форме распределенного реестра. Такие 
реестры могут дублироваться в~аналогичных центрах различных регионов, что 
позволяет создать аналог электронной бухгалтерской книги, фиксирующей 
фи\-нан\-со\-во-эко\-но\-ми\-че\-скую деятельность субъектов цифровой экономики. Такой 
подход предложено реализовать на базе системы ситуационных центров, что 
отражено в~работах~[1, 2].
  
  Собранная из информационных потоков информация о~транзакциях, т.\,е.\ 
о~контрактах, договорах, платежах, отчетах, закупленных материалах, 
характеристиках исполнителей работ и~др., собирается в~базе данных в~указанном 
центре. Согласно теории интеллектуальных сис\-тем~[3], эту базу данных можно 
называть базой фактов (БФ). Базу фактов можно представить как бинарную мат\-ри\-цу, 
строки которой описывают характеристики, входящие в~транзакции, а столбцы 
нумеруются характеристиками. Строки матрицы будем называть 
\textit{объектами}~[4, 5]. 
  
  Рассматриваемые в~работе методы выявления мошенничества будут основаны 
на противоречиях между действиями, описанными в~транзакциях, и~информацией, 
содержащейся в~планах, стандартах, прецедентах и~др. Для нахождения 
противоречий в~архитектуре центра предусмотрена другая база данных~--- база 
знаний (БЗ)~\cite{3-gr, 6-gr}, которая устроена так же, как БФ. 
  
  Информация в~БЗ собирается на основе положительного опыта или расчетов. 
Используя БЗ, можно выводить факты нарушения при\-чин\-но-след\-ст\-вен\-ных 
связей. Нарушения при\-чин\-но-след\-ст\-вен\-ных связей будем называть 
\textit{аномалиями}. 
  
  Для упрощения дальнейшее изложение будет вестись в~рамках поиска 
противоречий при выполнении некоторого абстрактного проекта. Выявление 
аномалий будет происходить на основе фактов из БФ с~помощью знаний из БЗ 
методами искусственного интеллекта и~интеллектуального анализа 
данных~\cite{6-gr}. 

\vspace*{-10pt}
  
  \section{Модели}
  
  \vspace*{-3pt}
  
  Наиболее сложная из рассмотренных выше задач~--- выявление противоречий, 
т.\,е.\ использование БЗ для получения новых знаний и~выявление аномалий из 
полученных фактов. 
  
  Все способы выявления противоречий основаны на определении 
  причинно-следственных связей. При этом противоречия в~параметрах транзакций по 
отношению к~требуемым в~БЗ составляют сущность аномалий. 
  
   Далее будет рассмотрен метод, основанный на некоторой упрощенной схеме 
реализации абстрактного проекта. 
  
  Каждый проект имеет цель: например, цель представляет собой построение 
некоторой системы. Воспользуемся структурным подходом, который позволяет 
строить проект на основе разбиения системы на подсистемы и~определения 
взаимодействий подсистем~\cite{7-gr}. При этом каждая подсистема также 
представима структурной моделью. 
  
  Как сама система, так и~каждая ее подсистема имеют свой функционал 
и~спецификацию, па\-ра\-мет\-ры настройки и~домены параметров настройки. Кроме 
этих характеристик существует множество характеристик, связанных 
с~<<жизненным циклом>> создания системы. Сюда входят работы, ресурсы, 
сроки выполнения работ по созданию подсистем и~самой системы, стоимости 
компонентов и~материалов, стоимости работ, схемы поставок, договорные 
обязательства и~др. Все характеристики связаны между собой, поэтому можно 
говорить о стоимости и~времени изготовления структурных компонентов системы. 
  
  Одной из важнейших характеристик является смета (система смет для 
подсистем). Смета сопоставляет каждому компоненту системы стоимость его 
изготовления и~настройки. 
  
  Схема построения системы может быть пред\-став\-ле\-на диаграммой, 
изображенной на рис.~1. 

{ \begin{center}  %fig1
 \vspace*{9pt}
   \mbox{%
 \epsfxsize=79mm 
 \epsfbox{gru-1.eps}
 }


\vspace*{9pt}


\noindent
{{\figurename~1}\ \ \small{Диаграмма достижения цели}}
\end{center}
}

\vspace*{9pt}

\addtocounter{figure}{1}
  
  


  Представленная на рис.~1 диаграмма позволяет описать основные классы 
возможных противоречий при достижении цели. Противоречия возникают, когда 
данные БФ не соответствуют требуемым характеристикам. 
  
  
  \section{Потенциальные классы аномалий при~достижении цели}
  
  Выделим четыре потенциальных класса противоречий, которые показывают, 
каким образом нужно искать эти противоречия.
  
 
  Противоречие цели и~проекта (рис.~2) возникает при отсутствии обоснования 
или в~случае логического противоречия между возможностями проектируемого 
функционала и~целью системы. Отметим, что в~проект входят сроки, перечень 
работ, материалы, настройки, которые описываются соответствующими 
параметрами и~допустимыми значениями этих параметров. Проект формируется 
на основе БЗ и~расчетов, исходя из информации, полученной по аналогии 
с~другими проектами и~решениями, которые считаются апробированными. 
  
  Отметим, что цель порождает проект и~в этом смысле является причиной 
проекта. Однако для анализа противоречий необходимо двигаться по штриховой 
стрелке диаграммы (см.\ рис.~2) от проекта к~цели. В~самом деле, любой компонент 
проекта направлен на теоретическое достижение цели. Цель~--- сложный объект, 
поэтому в~проекте могут возникнуть характеристики, противоречащие хотя бы 
некоторым характеристикам цели. Это делает проект противоречивым, но вывод 
об этом может быть сделан только на уровне описания цели. 
  

  Противоречия между проектом и~его реализацией, исключая настройки 
(рис.~3), могут возникать, например, при закупке исполнителем материалов более 
низкого качества по более низким ценам, при попытках достижения требуемых 
сроков работы за счет снижения качества выполнения работ, за счет нахождения 
<<объективных>> причин для увеличения сроков работы и,~следовательно, 
увеличения цены реализации проекта. 


  Для выявления указанных противоречий необходимо двигаться по диаграмме 
(см.\ рис.~3) в~обратную сторону в~соответствии со~штриховыми стрелками. 
Действительно, выявить противоречия между характеристиками закупленных 
материалов и~требуемыми по проекту можно только при обращении к~проекту 
и~его спецификациям. Манипуляции со сроками работы также можно выявить 
только при обращении к~соответствующим расчетам в~проекте. Задержки в~сроках 
работы, связанные с~поставками материалов, можно определить только на 
предыдущем этапе диаграммы (см.\ рис.~3) в~описании проекта. 


  


  Противоречия между реализацией проекта и~его настройкой (рис.~4) возникает, 
когда не удается добиться требуемых значений параметров функционала, не 
удается обеспечить необходимый уровень\linebreak\vspace*{-12pt}

{ \begin{center}  %fig2
 \vspace*{-6pt}
   \mbox{%
 \epsfxsize=16mm 
 \epsfbox{gru-2.eps}
 }


\vspace*{6pt}


\noindent
{{\figurename~2}\ \ \small{Противоречия цели и~проекта}}
\end{center}
}

%\vspace*{9pt}

\addtocounter{figure}{1}

{ \begin{center}  %fig3
 \vspace*{6pt}
    \mbox{%
 \epsfxsize=79mm 
 \epsfbox{gru-3.eps}
 }


\end{center}

\vspace*{-2pt}


\noindent
{{\figurename~3}\ \ \small{Противоречия проекта и~его реализации (без настройки)}}
}

\vspace*{6pt}

\addtocounter{figure}{1}

{ \begin{center}  %fig4
 \vspace*{1pt}
   \mbox{%
 \epsfxsize=54.5mm 
 \epsfbox{gru-4.eps}
 }


\end{center}


\noindent
{{\figurename~4}\ \ \small{Противоречия реализации проекта и~его на\-стройки}}
}

%\vspace*{9pt}

\addtocounter{figure}{1}

{ \begin{center}  %fig5
 \vspace*{5pt}
    \mbox{%
 \epsfxsize=79mm 
 \epsfbox{gru-5.eps}
 }


\end{center}



\noindent
{{\figurename~5}\ \ \small{Противоречия цели и~достигнутой реализации проекта}}
}

\vspace*{6pt}

\addtocounter{figure}{1}

\noindent
 качества реализации проекта. Для 
определения противоречия в~настройках надо опять же двигаться по диаграмме 
(см.\ рис.~4) в~обратную сторону по штриховым стрелкам, так как для выявления 
характеристик результатов работы, которые не дают возможности реализации 
определенного функционала, необходимо иметь информацию о результатах этой 
работы. 


  



  Противоречие между целью и~достигнутой реализацией проекта (рис.~5) 
возникает, когда реализованная система не позволяет достичь цели. В~этом случае 
опять противоречие нужно искать, двигаясь от цели к~реальному достигнутому 
функционалу по штриховой стрелке (см.\ рис.~5).
  
  Суммируя положения, изложенные в~данном разделе, приходим к~выводу, что 
для выявления противоречий необходимо проводить анализ от следствия 
к~причине, т.\,е.\ искать аномалии в~информации, описывающей порождение 
наблюдаемых следствий. 
  
  
  \section{Связь противоречий и~причин}
  
  Прежде чем построить связь между причинами и~противоречиями, кратко 
опишем простейшую модель связи этих понятий. Причины и~противоречия будут 
сформулированы для представления компонентов системы как объектов, 
обладающих наборами известных характеристик~\cite{4-gr, 5-gr}. 
  
  Пусть $U\hm=\{\alpha, \beta, \ldots\}$~--- совокупность характеристик 
(пространство характеристик). Согласно~\cite{4-gr} \textit{объектом}~$O$ 
называется любое подмножество характеристик $O\hm\subseteq U$. Рассмотрим 
последовательность объектов, возможно в~различных пространствах 
характеристик. 
  
  \smallskip
  
  \noindent
  \textbf{Определение~1.}\ Объект~$P$ с~числом характеристик, большим или 
равным~2, является \textit{причиной} объекта (\textit{свойства})~$B$ в~цепочке 
наблюдаемых объектов тогда и~только тогда, когда выполнены следующие 
условия:
  \begin{enumerate}[(1)]
\item для каждого объекта~$C$, если $P\hm\subseteq C$, то $C\mapsto B$, где 
$C\mapsto B$ означает, что объект~$B$ присутствует в~объекте, следующем за 
объектом~$C$;
\item объект~$P$ является минимальным объектом, удовлетворяющим 
условию~1, а~именно: $\forall \alpha\hm\in P$ объект~$P\backslash \{\alpha\}$ 
не является причиной, т.\,е.\ $\exists C:\ \alpha\not\in C$, $P\backslash 
\{\alpha\}\hm\subseteq C$ и~$C\not\mapsto B$, где $C\not\mapsto B$ означает, 
что~$B$ не может содержаться в~объекте, следующем за объектом~$C$. 
\end{enumerate}

  Приведенное определение причины является упрощением причин, 
возникающих в~реальном мире. Например, реальные причины могут возникать\linebreak 
как совокупность характеристик из разных пространств. Одно следствие может 
порождаться разными причинами или возникать из внешних\linebreak и~ненаблюдаемых 
характеристик. Однако пред\-став\-лен\-ная далее формализация позволяет доступно 
изложить при\-чин\-но-след\-ст\-вен\-ные истоки противоречий, которые 
инициируют в~дальнейшем глубокое исследование рассматриваемых процессов.
  
  Будем считать, что для любого интересующего нас свойства~$B$ существует 
причина. Тогда справедлива следующая теорема.
  
  \smallskip
  
  \noindent
  \textbf{Теорема~1.}\ \textit{Для любого свойства~$B$ существует 
единственная причина}. 
  
  \smallskip
  
  \noindent
  Д\,о\,к\,а\,з\,а\,т\,е\,л\,ь\,с\,т\,в\,о\,.\ \ Доказательство будем вести от противного, 
т.\,е.\ предположим, что существуют две причины свойства~$B$: $P$ 
и~$P^\prime$, $P\hm\not= P^\prime$. Тогда существует $\alpha\hm\in U$, которое 
удовлетворяет одному из двух условий:
  \begin{itemize}
\item[(а)] $\alpha\in P$, $\alpha\notin P^\prime$;
\item[(б)] $\alpha\notin P$, $\alpha \in P^\prime$.
\end{itemize}

  Пусть выполняется условие~(б). Тогда $P^\prime\backslash \{\alpha\}$ не 
является причиной по условию~2 определения~1, т.\,е.\ $\exists C$ такое, что 
$\alpha\notin C$, $P^\prime\backslash \{\alpha\}\hm\subseteq C$ и~$C\not\mapsto B$. 
Но если~$B$ произошло и~$P$ его причина, то $C\mapsto B$, что противоречит 
предположению. Теорема~1 доказана.
  
  \smallskip
  
  \noindent
  \textbf{Лемма.} \textit{Если $P$~--- причина появления свойства~$B$, то 
объект~$B$ определяет существование свойства~$P$ в~объекте, 
предшествующем~$B$. }
  
  \smallskip
  
  \noindent
  Д\,о\,к\,а\,з\,а\,т\,е\,л\,ь\,с\,т\,в\,о\,.\ \ Из предположения, что у~каж\-до\-го 
свойства~$B$ есть причина, и~условия, что~$P$ является причиной~$B$, следует, 
что при появлении в~данных свойства~$B$ объект~$C$, предшествующий 
появлению~$B$, содержит как часть объект~$P$. Это следует из теоремы~1 
и~определения причины. 
  
  Докажем принцип <<необходимого условия>>, который, несмотря на простоту 
доказательства, будет играть в~дальнейшем существенную роль.
  
  \smallskip
  
  \noindent
  \textbf{Теорема~2.} \textit{Если~$P$~--- причина появления свойства~$B$ 
и~$A\hm\subseteq P$, то объект~$B$ определяет наличие свойства~$A$ 
в~объекте, предшествующем~$B$}. 
  
  \smallskip
  
  \noindent
  Д\,о\,к\,а\,з\,а\,т\,е\,л\,ь\,с\,т\,в\,о\,.\ \ Пусть в~данных имеется объект~$B$ 
и~$P\mapsto B$, тогда в~силу существования и~единственности причины~$B$ 
в~данных должен существовать объект~$C$, предшествующий~$B$ 
и~содержащий причину~$P$. Поскольку $A\hm\subseteq P$ и~$B$ содержит 
причину~$P$, то $B\mapsto A$. С~учетом леммы теорема~2 доказана.
  
  \smallskip
  
  Пусть даны пространства $U_1, U_2,\ldots$ и~имеется последовательность 
данных (процесс выполнения этапов проекта в~соответствии с~рис.~1) $A, B, 
\ldots$, где каждый объект является подмножеством некоторого 
пространства~$U_i$, $i\hm=1,\ldots$ Тогда в~объекте~$A$ присутствует 
причина~$P$ появления интересующего нас свойства~$C$ в~объекте~$B$. Пусть 
$P\hm\subseteq A$, тогда по теореме~2 $\forall \alpha\hm\in P$:  
$C\mapsto \{\alpha\}$, т.\,е.\ из появления~$C$ следует появление 
характеристики~$\alpha$ в~предшествующем объекте. Это необходимое условие 
того, что~$C$ удовлетворяет причинно-следственным связям развития процесса 
выполнения проекта. Если для~$C$ нет характеристики~$\alpha$, которую можно 
отнести к~причине~$C$, то можно считать, что нарушена  
при\-чин\-но-след\-ст\-вен\-ная связь и~$C$~--- аномальный объект. 
  
  \smallskip
  
  \noindent
  \textbf{Пример.} Если объект~$C$ состоит в~получении суммы~$a$ 
фирмой~$K$, то согласно теореме~2 в~пред\-шест\-ву\-ющем объекте должна 
существовать причина перевода суммы~$a$ на фирму~$K$. Если эта причина 
в~проекте отсутствует, то это можно считать признаком мошеннической схемы. 
Все проекты по предположению собираются из <<кубиков>>, содержащихся в~БЗ. 
Тогда можно сравнить цену объекта~$C$, породившего получение суммы~$a$, 
и~сумму, присутствующую в~смете проекта. Если разница велика, то это либо 
ошибка проекта, либо признак мошеннической схемы.
  
  \section{Поиск противоречий на~основе~принципа <<необходимых~условий>>}
   
  Как было показано в~разд.~3, нахождение противоречий соответствуют 
движению от следствия к~причине. Для каждого объекта в~наблюдаемых данных 
выявление причин его появления является трудоемкой задачей. Кроме того, при 
реализации контроля соблюдения при\-чин\-но-след\-ст\-вен\-ных связей на 
большом множестве участников экономической деятельности задача анализа 
причин становится трудоемкой. Поэтому процедуру контроля необходимо разбить 
на два этапа, где первый этап состоит в~анализе простых <<необходимых 
условий>> проявления мошенничества, когда используется хотя бы одна 
известная характеристика причины. Второй этап (в~режиме офлайн) состоит 
в~выявлении причин, позволяющих провести анализ источников мошеннических 
схем. 
  
  Один из подходов к~выбору <<необходимых условий>> состоит в~построении 
множества подцелей исходной цели проекта (структурный метод построения 
проекта~\cite{7-gr}). Каждая подцель описывается диаграммой на рис.~1, 
и~реализации подцелей должны образовывать полный функционал цели. Это 
является необходимым, но не достаточным условием достижения цели, так как 
при таком подходе отсутствует компонент согласования всех подцелей в~единую 
систему. Однако такой подход значительно упрощает анализ выполнения проекта 
на предмет поиска мошенничества. Если признаки мошенничества будут 
обнаружены в~реализации хотя бы одной из подцелей, то это значит, что 
мошенничество присутствует в~реализации всего проекта. 
  
  Аналогично в~реализации каждого этапа в~любой из подцелей можно выделять 
простые <<необходимые условия>> нарушения при\-чин\-но-след\-ст\-венн\-ых 
связей. 
  
  Таким образом, получается множество <<необходимых условий>>, нарушение 
которых свидетельствует о наличии мошенничества. Это множество 
<<необходимых условий>> можно назвать метаданными~[8, 9] для контроля 
проекта на выявление мошенничества. 
  
  
  \section{Заключение }
  
  В поиске противоречий необходимо от транзакций, соответствующих 
следствиям при\-чин\-но-след\-ст\-вен\-ных связей, переходить к~анализу причин 
наблюдаемых следствий. Это сложная задача, которая связана с~описанием причин 
определенных свойств. 
  
  В работе представлена модель, позволяющая строить множество необходимых 
условий соответствия наблюдаемого следствия вызвавшей его причине. Этот 
подход делает поиск противоречий вполне вычислимой задачей, но не гарантирует 
успех. 
  
  {\small\frenchspacing
 {%\baselineskip=10.8pt
 \addcontentsline{toc}{section}{References}
 \begin{thebibliography}{9}
\bibitem{1-gr}
\Au{Грушо А.\,А., Зацаринный~А.\,А., Тимонина~Е.\,Е.} Блокчейны цифровой экономики на базе 
системы ситуационных центров и~централизованного консенсуса~// Радиолокация, навигация, 
связь: Мат-лы XXV Междунар. научн.-технич. конф.~---
Воронеж: Издательский дом ВГУ, 2019. Т.~6. С.~183--191. 
\bibitem{2-gr}
\Au{Grusho A., Zatsarinny~A., Timonina~E.} A~system approach to information security in 
distributed ledgers on the situational centers platform.~---
Lecture notes in computer science ser.~--- Springer, 2019 
(in press).
\bibitem{3-gr}
\Au{Финн В.\,К.} Искусственный интеллект: Методология, применения, философия.~--- М.: 
Красанд, 2011. 448~с.

\bibitem{5-gr} %4
\Au{Аншаков~О.\,М., Фабрикантова~Е.\,Ф.} ДСМ-ме\-тод автоматического порождения 
гипотез: Логические и~эпистемологические основания.~--- М.: Либроком, 2009. 432~с.

\bibitem{4-gr} %5
\Au{Poelmans J., Elzinga~P., Viaene~S., Dedene~G.} Formal concept analysis in knowledge 
discovery: A~survey~// Conceptual structures: From information to intelligence~/ Eds.\ M.~Croitoru, 
S.~Ferr$\acute{\mbox{e}}$, and D.~Lukose.~--- Lecture notes in computer science 
ser.~--- Berlin--Heidelberg: Springer, 2010. Vol.~6208.  P.~139--153.

\bibitem{6-gr}
\Au{Панкратова~Е.\,С., Финн~В.\,К.} Автоматическое по\-рож\-де\-ние гипотез в~интеллектуальных 
системах.~--- М.: Либроком, 2009. 528~с. 
\bibitem{7-gr}
\Au{Денисов А.\,А., Колесников~Д.\,Н.} Теория больших систем управления.~--- Л.: Энергоиздат, 1982. 488~с.

\bibitem{9-gr}
\Au{Грушо А.\,А., Грушо Н.\,А., Забежайло~М.\,И., Смирнов~Д.\,В., Тимонина~Е.\,Е.} 
Параметризация в~прикладных задачах поиска эмпирических причин~// Информатика и~её 
применения, 2018. Т.~12. Вып.~3. С.~62--66.

\bibitem{8-gr}
\Au{Грушо А.\,А., Грушо Н.\,А., Левыкин~М.\,В., Тимонина~Е.\,Е.} Методы идентификации 
захвата хоста в~распределенной ин\-фор\-ма\-ци\-он\-но-вы\-чис\-ли\-тель\-ной сис\-те\-ме, 
защищенной с~помощью метаданных~// Информатика и~её применения, 2018. Т.~12. Вып.~4. 
С.~41--45.

 \end{thebibliography}

 }
 }

\end{multicols}

\vspace*{-3pt}

\hfill{\small\textit{Поступила в~редакцию 03.04.19}}

%\vspace*{8pt}

%\pagebreak

\newpage

\vspace*{-28pt}

%\hrule

%\vspace*{2pt}

%\hrule

%\vspace*{-2pt}

\def\tit{ARCHITECTURAL DECISIONS IN~THE~PROBLEM 
OF~IDENTIFICATION OF~FRAUD IN~THE~ANALYSIS 
OF~INFORMATION FLOWS IN~DIGITAL ECONOMY\\[-5pt]}


\def\titkol{Architectural decisions in~the~problem 
of~identification of~fraud in~the~analysis 
of~information flows in~digital economy}

\def\aut{A.\,A.~Grusho, M.\,I.~Zabezhailo, N.\,A.~Grusho, and~E.\,E.~Timonina}

\def\autkol{A.\,A.~Grusho, M.\,I.~Zabezhailo, N.\,A.~Grusho, and~E.\,E.~Timonina}

\titel{\tit}{\aut}{\autkol}{\titkol}

\vspace*{-13pt}


 \noindent
   Institute of Informatics Problems, Federal Research Center ``Computer Sciences and 
Control'' of the Russian Academy of Sciences; 44-2~Vavilov Str., Moscow 119133, 
Russian Federation

\def\leftfootline{\small{\textbf{\thepage}
\hfill INFORMATIKA I EE PRIMENENIYA~--- INFORMATICS AND
APPLICATIONS\ \ \ 2019\ \ \ volume~13\ \ \ issue\ 2}
}%
 \def\rightfootline{\small{INFORMATIKA I EE PRIMENENIYA~---
INFORMATICS AND APPLICATIONS\ \ \ 2019\ \ \ volume~13\ \ \ issue\ 2
\hfill \textbf{\thepage}}}

\vspace*{3pt}


   
     
   \Abste{An approach to a~research of some types of fraud in digital economy with the usage of relationships of 
cause and effect is formulated. In all types of the considered frauds, the discrepancy between the 
purposes of financial transactions and actual cost of achievement of these purposes
has to be observed. Data on 
transactions can be collected by observing information flows in which these transactions are reflected. 
The architecture of data collection and their analysis can be organized by means of the distributed 
ledgers with the centralized consensus that allows creating an analog of the electronic account book 
fixing financial and economic activity of subjects of digital economy in the region. 
   The methods of fraud identification considered are based on the contradictions 
between actions described in transactions and information, which is contained in plans, standards, 
precedents, etc. 
   The method based on a~simplified scheme of implementation of the abstract project is considered. 
For identification of contradictions, it is necessary to carry out the analysis from the effect to the cause, 
i.\,e., to look for anomalies in information describing the generation of the observed effects. 
   It is shown how in implementation of the project it is possible to allocate simple ``necessary 
conditions'' of violation of cause and effect relationships, i.\,e., a~set of ``necessary conditions'' 
violation of which demonstrates fraud existence. It is possible to call this set of "necessary conditions" 
by metadata for control of the project for fraud identification.} 
   
   \KWE{digital economy; information flows; relationships of reason and effect; detection of 
fraudulent schemes}
   
  

 \DOI{10.14357/19922264190204}

\vspace*{-20pt}

 \Ack
   \noindent
   The work was partially supported by the Russian Foundation for Basic Research (projects  
18-29-03081 and 18-07-00274).



%\vspace*{6pt}

  \begin{multicols}{2}

\renewcommand{\bibname}{\protect\rmfamily References}
%\renewcommand{\bibname}{\large\protect\rm References}

{\small\frenchspacing
 {\baselineskip=10.5pt
 \addcontentsline{toc}{section}{References}
 \begin{thebibliography}{9}
\bibitem{1-gr-1}
\Aue{Grusho, A.\,A., A.\,A.~Zatsarinny, and E.\,E.~Timonina.} 2019. Blokcheyny tsifrovoy ekonomiki 
na baze sistemy situatsionnykh tsentrov i~tsentralizovannogo konsensusa [Blockchains of digital 
economy on the basis of the system of the situational centres and the centralized consensus]. 
\textit{25th Scientific and Technical Conference (International) ``Radar-Location, Navigation, 
Communication'' Proceedings}. Voronezh: VSU Publs. 6:183--191.
\bibitem{2-gr-1}
\Aue{Grusho, A., A.~Zatsarinny, and E.~Timonina.} 2019 (in press). 
A~system approach to information security 
in distributed ledgers on the situational centers platform. 
Lecture notes in computer science ser. Springer.
\bibitem{3-gr-1}
\Aue{Finn, V.\,K.} 2011. \textit{Iskusstvennyy intellekt: Metodologiya, primeneniya, filosofiya} 
[Artificial intelligence: Methodology, applications, philosophy]. Moscow: KRASAND. 448~p.

\bibitem{5-gr-1}
\Aue{Anshakov, O.\,M., and E.\,F.~Fabrikantova}. 2009. \textit{DSM-metod avtomaticheskogo porozhdeniya gipotez: Logicheskie 
i~epistemologicheskie osnovaniya} [JSM-method of automatic hypothesis generation: Logical and 
epistemological]. Moscow: KD LIBROKOM. 432~p.
\bibitem{4-gr-1} %5
\Aue{Poelmans, J., P.~Elzinga, S.~Viaene, and G.~Dedene.} 2010. Formal concept analysis in 
knowledge discovery: A~survey. \textit{Conceptual structures: From information to intelligence}. 
Eds.\ M.~Croitoru, S.~Ferr$\acute{\mbox{e}}$, and D.~Lukose. Lecture notes in 
computer science ser. Berlin--Heidelberg: Springer. 6208:139--153.

\bibitem{6-gr-1}
\Aue{Pankratov, E.\,S., and V.\,K.~Finn}. 
2009. \textit{Avtomaticheskoe porozhdenie gipotez v~intellektual'nykh 
sistemakh} [Automatic hypotheses generation in intelligent systems]. Moscow: KD 
\mbox{LIBROKOM}.  528~p. 
\bibitem{7-gr-1}
\Aue{Denisov, A.\,A., and D.\,N.~Kolesnikov.} 1982. \textit{Teoriya bol'shikh 
sistem upravleniya} [Theory of big control systems]. Leningrad: Energoizdat. 488~p.

\bibitem{9-gr-1}
\Aue{Grusho, A.\,A., N.\,A.~Grusho, M.\,I.~Zabezhailo, D.\,V.~Smirnov, and 
E.\,E.~Timonina.} 2018. 
Parametrizatsiya v~prikladnykh zadachakh poiska empiricheskikh prichin 
[Parametrization in applied 
problems of search of the empirical reasons]. 
\textit{Informatika i~ee Primeneniya~--- 
Inform. Appl.} 12(3):62--66.

\bibitem{8-gr-1}
\Aue{Grusho, A.\,A., N.\,A.~Grusho, M.\,V.~Levykin, and E.\,E.~Timonina.} 2018. Metody 
identifikatsii zakhvata khosta v~raspredelennoy informatsionno-vychislitel'noy sisteme, 
zashchishchennoy s~pomoshch'yu metadannykh [Methods of identification of host capture 
in the  distributed information system which is protected on the base of meta data].
\textit{Informatika i~ee 
Primeneniya~--- Inform. Appl.} 12(4):41--45.
{ %\looseness=1

}

\end{thebibliography}

 }
 }

\end{multicols}

\vspace*{-12pt}

\hfill{\small\textit{Received April 3, 2019}}

%\pagebreak

%\vspace*{-18pt}

\Contr

\noindent
\textbf{Grusho Alexander A.} (b.\ 1946)~--- Doctor of Science in physics and 
mathematics, professor, principal scientist, Institute of Informatics Problems, 
Federal Research Center ``Computer Sciences and Control'' of the Russian 
Academy of Sciences; 44-2~Vavilov Str., Moscow 119133, Russian Federation; 
\mbox{grusho@yandex.ru} 

\vspace*{3pt}

\noindent
\textbf{Zabezhailo Michael I.} (b.\ 1956)~--- Doctor of Science in physics and 
mathematics, principal scientist, Institute of Informatics Problems, Federal Research 
Center ``Computer Sciences and Control'' of the Russian Academy of Sciences;  
44-2~Vavilov Str., Moscow 119133, Russian Federation; 
\mbox{m.zabezhailo@yandex.ru} 

\vspace*{3pt}


\noindent
\textbf{Grusho Nikolai A.} (b.\ 1982)~--- Candidate of Science (PhD) in physics 
and mathematics, senior scientist, Institute of Informatics Problems, Federal 
Research Center ``Computer Sciences and Control'' of the Russian Academy of 
Sciences; 44-2~Vavilov Str., Moscow 119133, Russian Federation; 
\mbox{info@itake.ru} 

\vspace*{3pt}


\noindent
\textbf{Timonina Elena E.} (b.\ 1952)~--- Doctor of Science in technology, 
professor, leading scientist, Institute of Informatics Problems, Federal Research 
Center ``Computer Sciences and Control'' of the Russian Academy of Sciences;  
44-2~Vavilov Str., Moscow 119133, Russian Federation; 
\mbox{eltimon@yandex.ru} 

\label{end\stat}

\renewcommand{\bibname}{\protect\rm Литература}    %6


\def\stat{zah-gonch}

\def\tit{НЕКОТОРЫЕ СВОЙСТВА СМЕСЕЙ НОРМАЛЬНЫХ
РАСПРЕДЕЛЕНИЙ И~ИХ~ПРИЛОЖЕНИЯ К ЗАДАЧАМ МАГНИТОЭНЦЕФАЛОГРАФИИ$^*$}

\def\titkol{Некоторые свойства смесей нормальных
распределений и~их~приложения к~задачам магнитоэнцефалографии}

\def\aut{М.\,Б.~Гончаренко$^1$, Т.\,В.~Захарова$^2$}

\def\autkol{М.\,Б.~Гончаренко, Т.\,В.~Захарова}

\titel{\tit}{\aut}{\autkol}{\titkol}

\index{Гончаренко М.\,Б.}
\index{Захарова Т.\,В.}
\index{Goncharenko M.\,B.}
\index{Zakharova T.\,V.}

{\renewcommand{\thefootnote}{\fnsymbol{footnote}} \footnotetext[1]
{Работа выполнена при частичной поддержке РФФИ (проект 19-07-00352) 
и~в~соответствии с~программой Московского центра фундаментальной и~прикладной математики.}}


\renewcommand{\thefootnote}{\arabic{footnote}}
\footnotetext[1]{АО Интел,  goncharenko.mir@yandex.ru}
\footnotetext[2]{Московский государственный университет имени М.\,В.~Ломоносова,
    факультет вычислительной математики и~кибернетики; 
    Институт проб\-лем информатики Федерального исследовательского центра <<Информатика и~управ\-ле\-ние>> 
    Российской академии наук, \mbox{tvzaharova@mail.ru}}


\vspace*{-12pt}



\Abst{Рассматриваются различные свойства общих смесей вероятностных распределений. 
Особое внимание уделено случаю, когда смешиваемое распределение является нормальным. 
Установлены сходства в~поведении нормальных смесей и~нормальных распределений при трансформациях. 
Рассмотрено приложение к~задачам исследования головного мозга методом магнитоэнцефалографии (МЭГ). 
Определены условия, при которых применима оценка Эйткена (обобщенного метода наименьших квадратов) 
для поиска источников нейрофизиологической активности в~случае, когда распределение шума 
является нормальной смесью общего вида.}

\KW{смеси распределений; смеси нормальных распределений; смеси распределений Стьюдента; 
смеси логнормальных распределений; смеси гам\-ма-рас\-пре\-де\-ле\-ний; магнитоэнцефалография; 
МЭГ; обратная задача МЭГ; оценка Эйткена}

\DOI{10.14357/19922264210207}

\vspace*{-2pt}


\vskip 10pt plus 9pt minus 6pt

\thispagestyle{headings}

\begin{multicols}{2}

\label{st\stat}

\section{Введение}

\vspace*{-1pt}

Смеси вероятностных распределений общего вида (compound probability distribution) 
возникают в~широком классе математических моделей, где параметры вероятностных распределений 
сами являются случайными величинами. Например, такая ситуация естественным образом возникает в~процедуре 
байесовского вывода при подсчете апостериорного распределения. Другое распространенное приложение смесей~--- 
моделирование распределений с~тяжелыми хвостами. Оно оказывается полезным для описания данных
 эксперимента с~более высокой наблюдаемой дисперсией, чем предполагала оригинальная модель. 
 Стоит отметить, что распределение случайных сумм также имеет вид смеси, а~важный частный случай~--- 
 конечной смеси распределений~--- широко используется при обработке неоднородных данных и,~в~част\-ности, 
 в~задачах классификации наблюдений.


Данная статья посвящена исследованию свойств смесей нормальных законов. Повышенное 
внимание к~нормальному распределению вызвано его широкой распространенностью в~прикладных моделях 
анализа данных. Подробнее о~нормальных смесях и~их различных применениях можно 
прочесть в~книгах~\cite{mac, tit}. Знание рассмотренных в~статье свойств поможет понять, 
как изменяются свойства модели при замене предположения о~нормальности распределения ка\-ко\-го-ли\-бо 
параметра (например, аддитивного шума) на смесь нормальных распределений. 
В~рамках данной работы были обобщены результаты, полученные для конечных нормальных 
смесей в~статье~\cite{simple_mix}, на случай нормальных смесей общего вида. 
Отдельно рассматривается применение нормальных смесей в~обратной задаче нейровизуализации 
(исследования распределения источников активности внутри головного мозга) методом МЭГ.

\vspace*{-3pt}

\section{Базовые понятия}

Чтобы исследовать свойства смесей нормальных распределений, сначала надо ввести 
строгое определение смеси вероятностных распределений. По этой теме имеется обширная 
литература (см.\ например~\cite{teicher_60, teicher_63}), но в~более современном виде 
понятие смеси дается в~книге В.\,Ю.~Королева~\cite{korolev}, которое и~будет процитировано ниже.

Рассмотрим функцию $F (x, \textbf{y})$, определенную на множестве $\mathbb{R} \times \mathbb{Y}.$

Пусть $\mathbb{Y}$~--- это некоторое подмножество $m$-мер\-но\-го 
евклидова пространства, $\mathbb{Y}\hm\subseteq\mathbb{R}^m$ при некотором $m \hm\geqslant 1$, 
причем множество~$\mathbb{Y}$ снабжено борелевской $\sigma$-ал\-геб\-рой $\mathcal{B}$. 
Более того, при каждом фиксированном~$\textbf{y}$ функция $F (x, \mathbf{y})$ 
является функцией распределения по~$x$, а~при каждом фиксированном~$x$ функция~$F (x, \mathbf{y})$ 
измерима по $\mathbf{y}$, т.\,е.\ для любых $x \hm\in \mathbb{R}$ и~$c \hm\in \mathbb{R}$ 
выполнено условие $\{\mathbf{y} : F (x, \mathbf{y}) \hm< c\} \hm\in \mathcal{B}$. Пусть 
${\sf Q}$~--- вероятностная мера, определенная на измеримом пространстве~$(\mathbb{Y}, \mathcal{B})$.

Функция распределения
$$
        H(x) = \int\limits_ \mathbb{Y} F (x, \mathbf{y}) {\sf Q} (d\mathbf{y}), \enskip  x \in \mathbb{R},
$$
{\bfseries\textit{называется смесью функции распределения $F (x, \textbf{y})$ по~$\textbf{y}$
 относительно $\sf Q$}}. Распределение $F (x, \mathbf{y})$ называется смешиваемым, в~то время как мера~$\sf Q$ 
 задает смешивающее распределение.

Введем $m$-мерную случайную величину~$\mathbf{Y}$: $\mathbf{Y}(y) \hm\equiv y$, $y \hm\in \mathbb{Y}$,  
определенную на вероятностном пространстве $(\mathbb{Y}, \mathcal{B},{\sf Q})$. Тогда
функция распределения $H(x)$ может быть записана в~виде:
$$
        H(x) = {\sf E} F (x, \mathbf{Y}), \quad  x \in \mathbb{R}.
$$

Если $f(x, \mathbf{y})$~--- плотность распределения, соответствующая функции 
распределения $F (x, \mathbf{y})$,
$$
f(x, \mathbf{y}) = \fr{d}{dx} \, F (x,\mathbf{y}),
$$
то смеси $H(x)$ соответствует плотность
$$
        h(x) = {\sf E} f(x, \mathbf{Y}) = \int\limits_ \mathbb{Y} f(x, \textbf{y}) 
        {\sf Q} (d\mathbf{y}), \quad  x \in \mathbb{R}.
$$

Далее будет рассмотрен важный частный случай вероятностных смесей: 
так называемая сдвиг/мас\-штаб\-ная смесь. Введем определение согласно~\cite{korolev}.

Пусть в~определении, сформулированном выше, $m \hm= 2$. Предположим, что вектор~$\mathbf{y}$ имеет вид:
$$
\mathbf{y} = (u, v),
$$
где $u > 0$ и~$v \hm\in \mathbb{R}$, так что функция распределения $F (x,\textbf{y})$ 
допускает представление
$$
        F (x,\textbf{y}) = F \left(\fr{x-v}{u}\right), \quad  x \in \mathbb{R}.
$$

Тогда $\mathbb{Y}$~--- это положительная полуплоскость, т.\,е.\ $\mathbb{Y = R^+ \times R}$, 
и~функция распределения
$$
        H(x) = \int\limits_ \mathbb{Y} F\left(\fr{x-v}{u}\right) {\sf Q} (du, dv), \quad  x \in \mathbb{R},
$$
{\bfseries\textit{называется сдвиг/масштабной смесью функции распределения~$F$ относительно меры~$Q$}} 
с~параметром масштаба~$u$ и~параметром сдвига (положения)~$v$.

Если функция распределения~$F$ имеет плотность~$f$, то смеси~$H(x)$ соответствует плотность
$$
        h(x) = \int\limits_ \mathbb{Y} \fr{1}{u} f\left(\fr{x-v}{u}\right) {\sf Q} (du, dv), \quad  x \in \mathbb{R}.
$$

\vspace*{-12pt}


\section{Основные результаты}

\subsection{Свойства нормальных смесей}

\noindent
\textbf{Определение~1.}
Распределение случайной величины~$\xi$ является масштабной смесью нормальных 
распределений, если плотность~$\mathit{p}_\xi(x)$ представима в~виде:
$$
\mathit{p}_\xi(x) = \il 0 \infty \fr{1}{\sigma} \, \varphi\left(\fr{x}{\sigma}\right) {\sf Q} (d\sigma), 
\quad x \in \mathbb{R},
$$
где $\varphi(x)$~--- плотность стандартного нормального распределения.

\smallskip

Далее исследуем свойства этих смесей и~покажем, какие из свойств 
нормального распределения остаются справедливыми, а~какие~--- нет.


Для полноты изложения приведем доказательство известного утверждения.

\smallskip

\noindent
\textbf{Утверждение~1.}
\textit{Если плотность случайной величины~$\xi$ является масштабной смесью нормальных распределений, 
случайная величина~$X^2$ имеет распределение~$\chi^2(n)$~--- хи-квад\-рат с~$n$~степенями свободы, 
$X^2$ и~$\xi$ независимы, то случайная величина $t \hm= {\xi}/{\sqrt{X^2/n}}$ 
имеет плотность распределения, являющуюся смесью распределений Стьюдента с~$n$~степенями свободы}.


\smallskip

\noindent
Д\,о\,к\,а\,з\,а\,т\,е\,л\,ь\,с\,т\,в\,о\,.\ \
Рассмотрим плот\-ность~$\xi$, она имеет вид:
$$
\mathit{p}_\xi(x) =  \il 0 \infty \fr{1}{\sigma} \, \varphi\left(\fr{x}{\sigma}\right) {\sf Q} (d\sigma),
$$
где $\varphi(x)$~--- плотность стандартного нормального распределения.

Введем вспомогательную величину $\eta \hm= \sqrt{{X^2}/{n}}$ с~плот\-ностью
$$
\mathit{p}_\eta(y) = \fr{2ny({1}/{2})^{{n}/{2}}(ny^2)^{n/{2}-1}}
{\Gamma\left({n}/{2}\right)e^{{ny^2}/{2}}},\mbox{ где } y\ge0.$$

Тогда плотность частного ${\xi}/{\eta}$ имеет вид:
\begin{multline*}
\mathit{p}_{{\xi}/{\eta}}(z) = \mathit{p}_t(z) = {}\\
{}=
\int\limits_0^\infty y p_\eta(y) \, \left(\il 0 \infty \fr{1}{\sigma} \, 
\varphi\left(\fr{z\, y}{\sigma}\right) {\sf Q} (d\sigma) \right) {d}y={}
\\
{}= \int\limits_0^\infty \fr{ n^{n/2} \, y^n \, e^{-{ny^2}/{2}} } {2^{{n}/{2}-1} 
 \Gamma\left({n}/{2}\right) } 
  \left(\il 0 \infty \fr{1}{\sigma}\, \varphi\left(\fr{z\, y}{\sigma}\right) {\sf Q} 
  (d\sigma) \right) {d}y.
\end{multline*}
В силу теоремы Фубини изменив порядок интегрирования,  проведем следующие преобразования:
\begin{multline*}
\mathit{p}_t(z) = \int\limits_0^\infty \fr{1}{\sqrt{2\pi} \, \sigma} \,
\fr{n^{{n}/{2}}}{2^{{n}/{2}-1}  \Gamma\left({n}/{2}\right)}\times{}\\
{}\times 
\il 0 \infty y^n  e^{-({1}/{2}) \left({z^2}/{\sigma^2}+n\right) y^2} {d}y \, {\sf Q} (d\sigma) = {}\\
{}=
\int\limits_0^\infty \fr{1}{\sqrt{2\pi} \, \sigma} \,
\fr{n^{{n}/{2}}}{2^{{n}/{2}-1} \Gamma\left({n}/{2}\right)} \times{}\\
{}\times
\il 0 \infty (y^2)^{({n+1})/{2}-1}  e^{-({1}/{2}) \left({z^2}/{\sigma^2}+n\right) y^2} \,
{d}\left(\fr{y^2}{2}\right)  {\sf Q} (d\sigma) = {}\\
{}= \int\limits_0^\infty \fr{1}{\sqrt{2\pi} \, \sigma} \,
\fr{n^{{n}/{2}}}{2^{{n}/{2}-1}  \Gamma\left({n}/{2}\right)} \, 
\fr{1}{2^{({1-n})/{2}}} \times{}\\
{}\times \left(\fr{z^2}{\sigma^2}+n\right)^{-({n+1})/{2}} 
 \Gamma\left(\fr{n+1}{2}\right) {\sf Q} (d\sigma).
\end{multline*}


После упрощения подынтегрального выражения плотность~$\mathit{p}_t(z)$ примет вид:
\begin{multline*}
\mathit{p}_t(z) = \int\limits_0^\infty \fr{1}{\sqrt{\pi} \, \sigma} \,
\fr{\Gamma\left(({n+1})/{2}\right)} {\Gamma\left({n}/{2}\right)} \, 
 \fr{1}{\sqrt{n}}\times{}\\
 {}\times \left(\fr{n}{{z^2}/{\sigma^2}+n}\right)^{({n+1})/{2}}  {\sf Q} (d\sigma) ={}  \\
{}= \int\limits_0^\infty \fr{1}{\sigma} \, \fr{1}{\sqrt{\pi \, n}} \,
\fr{\Gamma\left(({n+1})/{2}\right)} {\Gamma\left({n}/{2}\right)}\times{}\\
{}\times
 \left(\fr{1}{1+{z^2}/({\sigma^2 n}})\right)^{({n+1})/{2}}  {\sf Q} (d\sigma).
\end{multline*}

Используя обозначение $\mathit{s}_n(x)$ для плотности распределения Стьюдента с~$n$~степенями свободы
$$
\mathit{s}_n(x) = \fr{\Gamma\left(({n+1})/{2}\right)}
{\Gamma({n}/{2})\sqrt{\pi n}}\left(1 + \fr{x^2}{n}\right)^{-({n+1})/{2}}, \enskip x \in \mathbb{R},
$$
получим следующее выражение для плотности~$\mathit{p}_t$:
$$
\mathit{p}_t(x) = \il 0 \infty \fr{1}{\sigma} \, s_n\left(\fr{x}{\sigma}\right) {\sf Q} (d\sigma).
$$


Таким образом, плотность~$\mathit{p}_t$ является масштабной смесью распределений Стьюдента 
относительно~${\sf Q}$.

Утверждение доказано.~\hfill$\square$


\smallskip

\noindent
\textbf{Теорема~1.}
\textit{Если случайная величина~$\xi$ имеет смешанное нормальное распределение 
относительно вероятностной меры~$Q$, то случайная величина 
$\eta\hm=\exp(\xi)$ имеет смешанное логнормальное распределение относительно~$Q$. 
И~наоборот, если~$\eta$ имеет смешанное логнормальное распределение, то $\xi\hm=\ln 
\eta$ имеет смешанное нормальное распределение относительно одной и~той же меры~$Q$}.


\smallskip

\noindent
Д\,о\,к\,а\,з\,а\,т\,е\,л\,ь\,с\,т\,в\,о\,.\ \
Докажем сначала первое утверждение теоремы.

Получим выражение для плотности~$\eta$ в~явном виде. По условию теоремы
$$
{\sf P} (\xi \le x) = \il 0 \infty  \il {-\infty} \infty \Phi \left(\fr{ x-a}{\sigma} \right) {\sf Q}
 (d\sigma, da),
$$
поэтому
\begin{multline*}
\mathit{F}_\eta(x) = {\sf P} (\eta \le x)) = {\sf P} (\exp(\xi) \le x)) = {}\\
{}={\sf P} (\xi \le \ln x))
=\il 0 \infty  \il {-\infty} \infty \Phi \left(\fr{\ln x-a}{\sigma} \right) {\sf Q} (d\sigma, da),
\end{multline*}
где $\Phi(x)$~--- функция распределения стандартного нормального закона, $x\hm>0$.

Следовательно,
$$
\mathit{p}_\eta(x) = \il 0 \infty  \il {-\infty} \infty \fr{1}{x \sigma} \,
\varphi\left(\fr{\ln x-a}{\sigma} \right) {\sf Q} (d\sigma, da).
$$

Таким образом, плотность $\mathit{p}(x)$ смешиваемого распределения имеет вид
$$
\mathit{p}(x) = \fr{1}{x\sigma\sqrt{2\pi}}\,e^{-{(\ln x - a)^2}/({2\sigma^2})}, \enskip x>0.
$$
А значит, $\eta$ имеет смешанное логнормальное распределение.

Для доказательства обратного утверждения теоремы воспользуемся следующей известной леммой.

\noindent
\textbf{Лемма~1.}
\textit{Если функция $y\hm=g(x)$ возрастает и~дифференцируема, случайная величина~$\xi$ имеет 
плотность $p_\xi$, тогда плотность~$p_\eta$ случайной 
величины $\eta\hm=g(\xi)$ определяется формулой}:
$$
p_\eta(y)=p_\xi(g^{-1}(y))\, \fr{1}{g'(g^{-1}(y))}\,.
$$



Итак, пусть плотность случайной величины~$\xi$ является смесью логнормальных распределений относительно~$Q$.
Тогда плотность случайной величины $\eta\hm=\ln\xi$, с~учетом леммы, равна

\noindent
\begin{multline*}
p_\eta(y)=p_\xi(\exp(y)) \fr{1}{1/\exp(y)}=
\exp(y)\times{}\\
{}\times \il 0 \infty  \il {-\infty} \infty \fr{1}{\exp(y)\, \sigma} \,
\varphi \left(\fr{\ln \exp(y)-a}{\sigma} \right) {\sf Q} (d\sigma, da)={}\\
\\
{}=\il 0 \infty  \il {-\infty} \infty \fr{1}{\sigma} \,
\varphi \left(\fr{y-a}{\sigma} \right) 
{\sf Q} (d\sigma, da).
\end{multline*}

Таким образом, плотность случайной величины~$\eta$ является смесью 
нормальных распределений относительно той же смешивающей меры~$Q$.


Теорема доказана.~\hfill$\square$

\smallskip

\noindent
\textbf{Замечание~1.}\ 
Связь между нормальным и~логнормальным распределениями сохраняется и~для соответствующих 
смесей нормальных и~логнормальных распределений. Данная теорема обобщает результаты, 
полученные авторами в~статье~\cite{simple_mix}.


\smallskip

\noindent
\textbf{Теорема~2.}\
\textit{Если плотность случайной величины~$\xi$ является масштабной смесью нормальных распределений,
то случайная величина~$\xi^2$ будет распределена с~плотностью, являющейся масштабной смесью 
гам\-ма-рас\-пре\-де\-ле\-ний, т.\,е.}
\begin{multline*}
{p}_{\xi^2}(x) =  \il 0 \infty  \fr{1}{\sigma \, \sqrt{x}} \, 
\varphi\left(\fr{\sqrt{x}}{\sigma}\right) {\sf Q} (d\sigma)= {}\\
{}=\il 0 \infty  \fr{1}{\sigma  \sqrt{2 \pi x }} \,   e^{-{x}/({2 \, \sigma^2})} {\sf Q} (d\sigma).
\end{multline*}


\smallskip

\noindent
Д\,о\,к\,а\,з\,а\,т\,е\,л\,ь\,с\,т\,в\,о\,.\ \
Выпишем функцию распределения для~$\xi^2$ и~проведем 
необходимые преобразования с~использованием теоремы Фубини:
\begin{multline*}
\mathrm{F}_{\xi^2}(x) = {\sf P} (\xi^2 \leq x) =  {\sf P} \left( -\sqrt{x} \leq \xi \leq \sqrt{x}\right) = {}\\
{}=
\int\limits_{-\sqrt{x}}^{\sqrt{x}} \il 0 \infty \fr{1}{\sigma}\, 
\varphi\left(\fr{x}{\sigma}\right) {\sf Q} (d\sigma)\, {d}x ={} \\
{}= \il 0 \infty \int\limits_{-\sqrt{x}}^{\sqrt{x}} \fr{1}{\sigma} \,
\varphi\left(\fr{x}{\sigma}\right)  {d}x {\sf Q} (d\sigma) = {}\\
{}=
\il 0 \infty \left(\Phi\left(\fr{\sqrt{x}}{\sigma}\right) - \Phi\left(\fr{-\sqrt{x}}{\sigma}\right)\right) 
{\sf Q} (d\sigma) = {}\\
{}= \il 0 \infty \left( 2 \Phi\left(\fr{\sqrt{x}}{\sigma}\right) - 1\right) {\sf Q} (d\sigma) ={}\\
{}=
 2 \il 0 \infty \Phi\left(\fr{\sqrt{x}}{\sigma}\right) {\sf Q} (d\sigma) - 1.
\end{multline*}

Далее для нахождения плотности распределения~$\xi^2$ продифференцируем функцию распределения, полученную выше:
\begin{multline*}
{p}_{\xi^2}(x) = \fr{{d}}{{d}x}\,\mathrm{F}_{\xi^2}(x) = 
\il 0 \infty \fr{1}{\sigma \sqrt{x}} \,\varphi\left(\fr{\sqrt{x}}{\sigma}\right)
 {\sf Q} (d\sigma) ={}\\
 {}=
  \il 0 \infty \fr{1}{\sigma \sqrt{2\pi \, x}}\, e^{-{x}/({2\sigma^2})} {\sf Q} (d\sigma).
\end{multline*}

Таким образом, плотность распределения случайной величины~$\xi^2$ является масштабной 
смесью гам\-ма-рас\-пре\-де\-ле\-ний.


Теорема доказана.~\hfill$\square$

\smallskip

\noindent
\textbf{Замечание 2.} Эти три теоремы наглядно демонстрируют схожесть в~поведении 
нормальных смесей и~нормального распределения. Аналоги данных теорем для дискретных 
смесей можно найти в~\cite{simple_mix}. Но смеси нормальных распределений обладают 
особенностями поведения, отличающими их от нормального распределения.
В~част\-ности, широко известный факт об эквивалентности свойств некоррелированности 
и~независимости для компонент многомерного нормального распределения~\cite{borovkov}
уже не выполняется для конечной нормальной смеси, как показано в~\cite{simple_mix}.

\subsection{Обратная задача магнитоэнцефалографии}

В статье \cite{simple_mix} рассматривалось приложение конечных нормальных смесей для 
моделирования \mbox{шума} измерений активности головного мозга методом МЭГ. 
Магнитоэнцефалография~--- неинвазивная технология нейровизуализации, позволяющая исследовать 
электромагнитную активность человеческого мозга путем измерения магнитного поля непосредственно 
вблизи поверхности головы испытуемого (подробнее о~МЭГ см.~\cite{hama_main, our}).


С помощью МЭГ можно исследовать различные аспекты функционирования головного мозга 
с~высоким временн$\acute{\mbox{ы}}$м разрешением, сопоставимым со скоростью передачи 
нервного импульса\linebreak (это качественно отличает МЭГ от другого популярного метода функциональной 
нейровизуализации~--- функциональной маг\-нит\-но-ре\-зо\-нанс\-ной\linebreak томографии, фМРТ). 
Запись МЭГ представляет собой многоканальный сигнал, регистрируемый массивом сенсоров внут\-ри 
специального шлема, под который помещается голова испытуемого. Отдельный интерес представляет 
локализация (указание точных координат и~интенсивностей) источников активности 
внут\-ри головного мозга испытуемого. Для ее установления необходимо решить обратную задачу МЭГ.


Рассмотрим обратную задачу МЭГ:
\begin{equation}
\label{MEG_inv}
Y = L\Theta + \mathcal{E},
\end{equation}
где $Y \in \mathbb{R}^n$~--- измеряемые данные; $L\hm \in \mathbb{R}^{n \times k}$~--- оператор 
Био--Са\-ва\-ра--Лап\-ла\-са; $\Theta\hm \in \mathbb{R}^k$~--- 
неизвестные амплитуды источников; $\mathcal{E} \hm\in \mathbb{R}^n$~--- шум; $k$~--- 
количество источников активности; $n$~--- чис\-ло МЭГ-сен\-со\-ров, $k \hm\ge n$.
Классический подход к~решению подобных задач предполагает минимизацию нормы ошибки:
$$
||\mathcal{E}||^2 = ||Y - L\Theta||^2 \rightarrow \min\limits_\Theta.
$$
О других подходах можно прочесть в~статье~\cite{bayes_inverse}. 
Физические и~математические свойства модели обратной задачи МЭГ рассматриваются 
в~\cite{MEG_source_loc}.

В теории линейной регрессии доказано, что у~задачи~\eqref{MEG_inv} существует решение 
наименьших квадратов~\cite{app_regr} при выполнении следующих условий:
\begin{itemize}
    \item ${\sf E}\mathcal{E} = 0$~--- математическое ожидание шума равно нулю;
    \item $\Sigma > 0$~--- матрица ковариации ошибок положительно определена;
    \item $\mathrm{rank}\,L = n$, т.\,е.~$L$~--- мат\-ри\-ца полного строкового ранга.
\end{itemize}

Решение $\hat{\Theta}$, полученное методом взвешенных наименьших квадратов при 
выполнении обозначенных выше условий, называют оценкой Эйткена~\cite{app_regr}:
\begin{equation}
\label{Aitken_WLS}
\hat{\Theta} = \left(L^{\top}\Sigma^{-1}L\right)^{-1}L^{\top}\Sigma^{-1}Y\,.
\end{equation}

Оценка~\eqref{Aitken_WLS} является несмещенной, состоятельной и~оптимальной в~классе всех 
линейных оценок~\cite{app_regr}.


Анализ реальных записей <<пустой комнаты>> (собственного шума сенсоров и~окружающей среды, 
без испытуемого) показал, что зачастую распределение шума имеет сложную структуру с~такими 
особенностями, как тяжелые хвосты, мультимодальность, несимметричность. Таким образом, 
для более адекватного описания реальных данных требуется более сложная модель шума.


В данном разделе будет рассматриваться модель шума, имеющего распределение в~виде 
многомерной нормальной смеси общего вида с~плотностью
\begin{equation}
\label{eq:general_gaussian_mixture}
h(\vec{x}) = \int \limits_{\mathbb{Y}} f_y\left(\vec{x}; \vec{\mu_y}, \Sigma_y\right) {\sf Q}({d}y),
\end{equation}
где $f_y(\vec{x}; \vec{\mu_y}, \Sigma_y)$~--- плотность $k$-мер\-но\-го 
нормального распределения с~вектором средних~$\vec{\mu_y}$ и~ковариационной матрицей~$\Sigma_y$  
(для упрощения выкладок будем считать, что $\Sigma_y$ невырождена $\forall y \hm\in \mathbb{Y}$), 
$\vec{x}\hm \in \mathbb{R}^k$. Далее будем использовать сокращенную запись~$f_y(\vec{x})$.


В статье~\cite{simple_mix} была доказана теорема о~том, что матрица ковариации 
конечной нормальной смеси положительно определена (в~случае если в~смеси есть компоненты с~положительно 
определенными ковариационными матрицами).


Для доказательства соответствующей теоремы в~случае общей нормальной смеси сначала докажем 
следующую лемму.

\smallskip

\noindent
\textbf{Лемма~2.}
\textit{Матрица ковариации нормальной смеси}~\eqref{eq:general_gaussian_mixture} \textit{имеет вид}:
$$
\Sigma = {\sf E}_y \Sigma_y + {\sf E}_y \left(\vec{\mu_y} - {\sf E}_y\vec{\mu_y}\right)\left(\vec{\mu_y} - 
{\sf E}_y\vec{\mu_y}\right)^\top,
$$
\textit{где $\vec{\mu_y}$ и~$\Sigma_y$~--- 
вектор средних и~ковариационная мат\-ри\-ца смешиваемого нормального распределения}.


\smallskip

\noindent
Д\,о\,к\,а\,з\,а\,т\,е\,л\,ь\,с\,т\,в\,о\,.\ \
Пусть случайная величина $\vec{X}$ имеет смешанное $k$-мер\-ное нормальное 
распределение с~плотностью~\eqref{eq:general_gaussian_mixture}. 
Тогда ее матрица ковариации по определению равна 
$$
\Sigma = 
{\sf E}\vec{X}\vec{X}^\top - {\sf E}\vec{X}{\sf E}\vec{X}^\top\,.
$$
 Рассмотрим каждое из слагаемых подробнее:
\begin{multline}
{\sf E}\vec{X}\vec{X}^\top =
\int\limits_{\mathbb{R}^k} \vec{x}\vec{x}^\top 
\int\limits_\mathbb{Y} f_y(\vec{x}) {\sf Q}\,(dy) \,{d}\vec{x} = {}\\
{}=
\!\!\!\int\limits_\mathbb{Y} \int\limits_{\mathbb{R}^k} \!\vec{x}\vec{x}^\top f_y(\vec{x})\,d\vec{x} 
{\sf Q}(dy) = 
\!\!\!\int\limits_\mathbb{Y}\!\! \left( \Sigma_y + \vec{\mu_y}\vec{\mu_y}^\top \right)
\! {\sf Q}(dy) = {} \\
 {} = 
{\sf E}_y \left( \Sigma_y + \vec{\mu_y}\vec{\mu_y}^\top \right);
\label{eq:first_summand}
\end{multline}

\vspace*{-12pt}

\noindent
\begin{multline}
\label{eq:second_summand}
{\sf E}\vec{X} =\int\limits_{\mathbb{R}^k} \vec{x} 
\int\limits_\mathbb{Y} f_y(\vec{x}) {\sf Q}(dy)\, d\vec{x} ={}\\
{}=
 \int\limits_\mathbb{Y} \int\limits_{\mathbb{R}^k} \vec{x} f_y(\vec{x}) 
 \,{d}\vec{x} {\sf Q}(dy) =
  \int\limits_\mathbb{Y} \vec{\mu_y} {\sf Q}(dy) = 
{\sf E}_y \vec{\mu_y}.
\end{multline}
    
Объединяя результаты~\eqref{eq:first_summand} и~\eqref{eq:second_summand} и~перегруппировывая 
слагаемые, получим итоговое выражение для ковариационной матрицы в~виде:
\begin{equation}
\label{eq:covar_matrix}
\Sigma = {\sf E}_y \Sigma_y + {\sf E}_y
 (\vec{\mu_y} - {\sf E}_y\vec{\mu_y})(\vec{\mu_y} - {\sf E}_y\vec{\mu_y})^\top.
\end{equation}
Лемма доказана.~\hfill$\square$


\smallskip

\noindent
\textbf{Теорема~3.}
\textit{Ковариационная матрица нормальной смеси}~\eqref{eq:general_gaussian_mixture} 
\textit{положительно определена}.


\smallskip

\noindent
Д\,о\,к\,а\,з\,а\,т\,е\,л\,ь\,с\,т\,в\,о\,.\ \
Рассмотрим случайную величину $\vec{X}$, имеющую смешанное нормальное распределение 
с~плот\-ностью~$\eqref{eq:general_gaussian_mixture}$. Ее ковариационная матрица имеет 
вид~\eqref{eq:covar_matrix}. По определению матрица~$\Sigma$ является положительно определенной, 
если выполнено
$$
u^\top\Sigma u > 0,\quad \forall\  u \in \mathbb{R}^k.
$$

Распишем более подробно с~учетом предыдущей леммы
\begin{multline*}
u^\top\Sigma u =ХЪ\\
ХЪ= u^\top{\sf E}_y \Sigma_y u +  
u^\top {\sf E}_y (\vec{\mu_y} - {\sf E}_y\vec{\mu_y})(\vec{\mu_y} - 
{\sf E}_y\vec{\mu_y})^\top u ={} \\ 
{}= {\sf E}_y u^\top \Sigma_y u +  u^\top {\sf E}_y (\vec{\mu_y} - {\sf E}_y\vec{\mu_y})(\vec{\mu_y} - 
{\sf E}_y\vec{\mu_y})^\top u = {}\\ 
{}= {\sf E}_y \underbrace{u^\top \Sigma_y u}_{> 0} + 
 \underbrace{u^\top {\sf E}_y (\vec{\mu_y} - {\sf E}_y\vec{\mu_y})(\vec{\mu_y} - 
 {\sf E}_y\vec{\mu_y})^\top u}_{\geqslant 0}.
\end{multline*}
Первое слагаемое строго положительно из-за положительной определенности матриц 
$\Sigma_y \forall y \hm \in \mathbb{Y}$. 
Второе слагаемое есть ковариационная мат\-ри\-ца~$\vec{\mu_y}$; следовательно, 
она неотрицательно определена. В~итоге получим, что
$$
u^\top\Sigma u > 0.
$$

Теорема доказана.~\hfill$\square$


\noindent
\textbf{Замечание~3.}\ Теорема остается справедливой, даже если матрицы~$\Sigma_y$ 
вырождены при некотором $y\hm \in A \hm\subset \mathbb{Y}$. Это следует из того, что
\begin{multline*}
u^\top{\sf E}_y \Sigma_y u = u^\top\int\limits_\mathbb{Y} \Sigma_y {\sf Q}(dy) u 
= {}\\
{}=u^\top\int\limits_{\mathbb{Y} \setminus A} \Sigma_y {\sf Q}(dy) u + 
u^\top\int\limits_A \Sigma_y {\sf Q}(dy) u \geqslant {}\\ 
{}\geqslant
u^\top\int\limits_{\mathbb{Y} \setminus A} \Sigma_y {\sf Q}(dy) u > 0,\ \forall
 u \in \mathbb{R}^k.
\end{multline*}

Также из доказательства видно, что для справедливости теоремы достаточно, 
чтобы у~смешиваемого распределения была положительно определенная матрица 
ковариации, а~непосредственный вид смешиваемого распределения значения не имеет.


Таким образом, при использовании модели шума в~виде нормальной смеси общего вида 
оценка интенсивностей источников с~помощью обобщенного метода наименьших 
квадратов остается справедливой. Стоит отметить, что решение обратной задачи 
таким методом пользуется большой популярностью в~прикладных нейрофизиологических исследованиях.

%\vspace*{-10pt}

\section{Заключение}

В~статье представлены базовые понятия непрерывных смесей вероятностных
 распределений и~подробно рассмотрен частный случай нормальных смесей общего вида, 
 определены законы распределения случайных величин, являющихся функциональным 
 преобразованием случайных величин с~плотностью в~виде нормальной смеси общего \mbox{вида}.

Смеси распределений общего вида возникают в~множестве прикладных задач, а~также 
они используются как средство представления не-нормальных распределений. Для важного частного 
случая, где смешиваемое распределение является нормальным, были рассмотрены распределения 
трансформации смеси и~установлены сходства в~поведении нормальных смесей и~нормального 
распределения. Также было доказано, что обобщенный метод наименьших квадратов поиска 
псевдообратного оператора остается применимым и~в~случае шума, имеющего распределение в~виде 
смеси общего вида. Этот результат говорит о~применимости широко распространенных методов решения 
обратной задачи МЭГ и~в~случае не-нор\-маль\-но\-го шума, который может быть представлен в~виде 
нормальной смеси. Такая ситуация часто встречается при обработке данных реальных экспериментов.


\vspace*{-12pt}

{\small\frenchspacing
{%\baselineskip=10.8pt
%\addcontentsline{toc}{section}{References}
\begin{thebibliography}{99}

\vspace*{-4pt}

\bibitem{tit}        
\Au{Titterington D.\,M., Smith~A.\,F.\,M., Makov~U.\,E.} 
Statistical analysis of finite mixture distributions.~--- New York, NY, USA: Wiley, 1985. 243~p.

\bibitem{mac}       
\Au{McLachlan G.\,J., Peel D.} Finite mixture models.~--- New York, NY, USA: Wiley \& Sons, 2000. 419~p.

\bibitem{simple_mix}     %3
\Au{Гончаренко М.\,Б., Захарова~Т.\,В.} 
Особенности поведения конечных смесей нормальных распределений~// 
Вестник Московского университета. Сер.~15: Вы\-чис\-ли\-тель\-ная математика и~кибернетика, 2018.  
№\,3. С.~30--36.
\bibitem{teicher_60}     
\Au{Teicher H.} On the mixture of distributions~// Ann. Math. Stat., 1960. Vol.~31. No.\,1. P.~55--73.
\bibitem{teicher_63}     
\Au{Teicher H.} Identifiability of finite mixtures~// 
Ann. Math. Stat., 1963. Vol.~34. No.\,4. P.~1265--1269.
\bibitem{korolev}       
\Au{Королев В.\,Ю.} EM-ал\-го\-ритм, его модификации и~их применение к~задаче 
разделения смесей вероятностных распределений: Теоретический обзор.~--- М.: ИПИ РАН, 2007. 94~с.
\bibitem{borovkov}      
\Au{Боровков А.\,А.} Математическая статистика.~--- 
4-е изд.~--- М.: Лань, 2010. 704~с.
\bibitem{hama_main}     
\Au{Hamalainen M., Hari~R., Ilmoniemi~R.\,J., Knuutila~J., Lounasmaa~O.\,V.} 
Magnetoencephalography~--- theory, instrumentation, and applications to noninvasive studies of 
the working human brain~// Rev. Mod. Phys., 1993. Vol.~65. No.\,2. P.~413--497.
\bibitem{our}          
\Au{Захарова Т.\,В., Никифоров~С.\,Ю., Гончаренко~М.\,Б., Драницына~М.\,А., Климов~Г.\,А., 
Хазиахметов~М.\,Ш., Чаянов~Н.\,В.} 
Методы обработки сигналов для лока-\linebreak\vspace*{-12pt}

\pagebreak

\noindent
лизации невосполнимых областей головного мозга~// 
Системы и~средства информатики, 2012. Т.~22. №\,2. С.~157--175.
\bibitem{bayes_inverse} 
\Au{Гончаренко~М.\,Б., Захарова~Т.\,В.} 
Вероятностный подход к~решению обратной задачи МЭГ~// Системы и~средства информатики, 2018. Т.~28. 
№\,1. С.~35--52.
\bibitem{MEG_source_loc} 
\Au{Zakharova T.\,V., Karpov~P.\,I., Bugaevskii~V.\,M.} 
Localization of the activity source in the inverse problem of magnetoencephalography~// 
Comput. Math. Model., 2017. Vol.~28. No.\,2. P.~148--157.
\bibitem{app_regr} 
\Au{Дрейпер Н.\,Р., Смит~Г.} Прикладной регрессионный анализ~/
Пер. с~англ.~--- 3-е изд.~--- М.: Диалектика, 2016. 912~с.
(\Au{Draper~N., Smith~H.} {Applied regression analysis}.~--- 3rd ed.~---
 New York, NY, USA: John Wiley \& Sons, Inc., 1998. 736~p.)
 \end{thebibliography}

}
}

\end{multicols}

\vspace*{-3pt}

\hfill{\small\textit{Поступила в~редакцию 27.09.2020}}

\vspace*{8pt}

%\pagebreak

%\newpage

%\vspace*{-28pt}

\hrule

\vspace*{2pt}

\hrule

%\vspace*{-2pt}

\def\tit{SOME PROPERTIES OF GAUSSIAN MIXTURES AND~APPLICATIONS TO~MAGNETOENCEPHALOGRAPHY PROBLEMS}


\def\titkol{Some properties of gaussian mixtures and~applications to~magnetoencephalography problems}

\def\aut{M.\,B.~Goncharenko$^1$ and T.\,V.~Zakharova$^{2,3}$}

\def\autkol{M.\,B.~Goncharenko and T.\,V.~Zakharova}


\titel{\tit}{\aut}{\autkol}{\titkol}

\vspace*{-10pt}


\noindent
$^1$INTEL A/O, 17-4 Krylatskaya Str., Moscow 121614, Russian Federation

\noindent
$^2$Department of Mathematical Statistics, Faculty of Computational Mathematics and Cybernetics,
M.\,V.~Lomo-\linebreak 
$\hphantom{^1}$nosov Moscow State University, 1-52~Leninskie Gory, GSP-1, Moscow 119991, Russian
Federation

\noindent
$^3$Institute of Informatics Problems, Federal Research Center ``Computer Science and Control'' 
of the Russian\linebreak
$\hphantom{^1}$Academy of Sciences, 44-2~Vavilov Str., Moscow 119333, Russian Federation

 
\def\leftfootline{\small{\textbf{\thepage}
\hfill INFORMATIKA I EE PRIMENENIYA~--- INFORMATICS AND
APPLICATIONS\ \ \ 2021\ \ \ volume~15\ \ \ issue\ 2}
}%
\def\rightfootline{\small{INFORMATIKA I EE PRIMENENIYA~---
INFORMATICS AND APPLICATIONS\ \ \ 2021\ \ \ volume~15\ \ \ issue\ 2
\hfill \textbf{\thepage}}}

\vspace*{5pt}




\Abste{The article is dedicated to research of various properties of compound probability 
distributions (mixture distributions). Special attention is paid to the case when the mixed 
distribution is Gaussian. The authors establish the similarities in the behavior of Gaussian 
mixtures and Gaussian distributions during transformations. The authors study applications 
to magnetoencephalographic brain research. The authors determine the conditions under which 
the Aitken estimator (generalized least squares) is applicable for localization of sources 
of neurophysiologic activity in the case of noise having compound Gaussian distribution.}

\KWE{compound distributions; compound Gaussian distribution; compound Student 
distribution; compound lognormal distribution; compound gamma distributions; 
magnetoencephalography; MEG; inverse MEG problem; Aitken's estimator}



\DOI{10.14357/19922264210207}

\vspace*{-3pt}

 \Ack
\noindent
The work was partly supported by the Russian Foundation for Basic Research (project 19-07-00352). 
The research was conducted in accordance with the program of the Moscow Center for 
Fundamental and Applied Mathematics.

\vspace*{6pt}

  \begin{multicols}{2}

\renewcommand{\bibname}{\protect\rmfamily References}
%\renewcommand{\bibname}{\large\protect\rm References}

{\small\frenchspacing
 {%\baselineskip=10.8pt
 \addcontentsline{toc}{section}{References}
 \begin{thebibliography}{99}

\bibitem{2-zg}
\Aue{Titterington, D.\,M., A.\,F.~M. Smith, and U.\,E.~Makov.}
 1985. \textit{Statistical analysis of finite mixture distributions}. New York, NY: Wiley. 243~p.
 
 \bibitem{1-zg}
\Aue{McLachlan, G., and D.~Peel.} 2000. 
\textit{Finite mixture models}. New York, NY: Wiley \& Sons. 419~p.

\bibitem{3-zg}
\Aue{Goncharenko, M.\,B., and T.\,V.~Zakharova.}
 2018. Oso\-ben\-nosti povedeniya konechnykh smesey normal'nykh raspredeleniy 
 [Features of behavior of finite mixtures of normal distributions].  
 \textit{Vestnik Moskovskogo Universiteta. Ser.~15: Vychislitel'naya matematika i~kibernetika} 
 [Bull. Moscow State University. Ser.~15: Comput. Math., Cybern.] 3:30--36.
\bibitem{4-zg}
\Aue{Teicher, H.} 1960. On the mixture of distributions.
 \textit{Ann. Math. Stat.} 31(1):55--73.
\bibitem{5-zg}
\Aue{Teicher, H.} 1963. Identifiability of finite mixtures. 
 \textit{Ann. Math. Stat.} 34(4):1265--1269.
\bibitem{6-zg}
\Aue{Korolev, V.\,Yu.} 2007.  \textit{EM-algoritm, ego modifikatsii i~ikh primenenie 
k~zadache razdeleniya smesey veroyatnostnykh raspredeleniy. Teoreticheskiy obzor} 
[EM algorithm modifications and their application to the separation of mixtures of probability 
distributions. Theoretical review]. Moscow: IPI RAN. 94~p.
\bibitem{7-zg}
\Aue{Borovkov, A.\,A.}
 2010.  \textit{Matematicheskaya statistika} [Mathematical statistics]. 4th ed. Moscow: Lan. 704~p.
\bibitem{8-zg}
\Aue{Hamalainen, M., R.~Hari, R.\,J.~Ilmoniemi, J.~Knuutila, and O.\,V.~Lounasmaa.}
 1993. Magnetoencephalography~--- theory, instrumentation, and applications to noninvasive studies 
 of the working human brain.  \textit{Rev. Mod. Phys.} 65(2):413--497. 
\bibitem{9-zg}
\Aue{Zakharova, T.\,V., S.\,Yu.~Nikiforov, M.\,B.~Goncharenko, M.\,A.~Dranitsyna, G.\,A.~Klimov, 
M.\,Sh.~Khaziakhmetov, and N.\,V.~Chayanov.}
 2012. Metody obrabotki signalov dlya lokalizatsii nevospolnimykh oblastey golovnogo mozga 
 [Signal processing methods for localization of nonrenewable brain regions]. 
  \textit{Sistemy i~Sredstva Informatiki~--- Systems and Means of Informatics} 22(2):157--175.
\bibitem{10-zg}
\Aue{Goncharenko, M.\,B., and T.\,V.~Zakharova.}
 2018. Ve\-ro\-yat\-nost\-nyy podkhod k~resheniyu obratnoy za\-da\-chi mag\-ni\-to\-en\-tse\-lo\-gra\-fii 
 [Probabilistic approach to solving the magnetoencephalography inverse problem].
  \textit{Sistemy i~Sredstva Informatiki~--- Systems and Means of Informatics} 28(1):35--52.
\bibitem{11-zg}
\Aue{Zakharova, T.\,V., P.\,I.~Karpov, and V.\,M.~Bugaevskii.}
 2017. Localization of the activity source in the inverse problem of magnetoencephalography. 
  \textit{Comput. Math. Model.} 28(2):148--157.
\bibitem{12-zg}
\Aue{Draper, N., and H.~Smith.} 1998. 
 \textit{Applied regression analysis}. 3rd ed. New York, NY: John Wiley \& Sons, Inc. 736~p.
 \end{thebibliography}

 }
 }

\end{multicols}

\vspace*{-3pt}

  \hfill{\small\textit{Received September~27, 2020}}


%\pagebreak

%\vspace*{-8pt}  

\Contr

\noindent
\textbf{Goncharenko Miroslav B.} (b.\ 1991)~--- 
software development engineer for graphics, INTEL A/O, 17-4~Krylatskaya Str., 
Moscow 121614, Russian Federation; \mbox{goncharenko.mir@yandex.ru}

\vspace*{3pt}

\noindent
\textbf{Zakharova Tatiana V.} (b.\ 1962)~--- 
Candidate of Science (PhD) in physics and mathematics, associate professor, Department
 of Mathematical Statistics, Faculty of Computational Mathematics and Cybernetics, M.\,V.~Lomonosov 
 Moscow State University, 1-52~Leninskie Gory, GSP-1, Moscow 119991, Russian Federation; 
 senior scientist, Institute of Informatics Problems, Federal Research Center 
 ``Computer Science and Control'' of the Russian Academy of Sciences, 44-2~Vavilov Str., Moscow 119333, 
 Russian Federation; \mbox{lsa@cs.msu.ru}
\label{end\stat}

\renewcommand{\bibname}{\protect\rm Литература} %7
\def\stat{krivenko}

\def\tit{МНОГОМЕРНЫЙ РЕФЕРЕНСНЫЙ РЕГИОН\\ ВЫСОКОЙ ПЛОТНОСТИ}

\def\titkol{Многомерный референсный регион высокой плотности}

\def\aut{М.\,П.~Кривенко$^1$}

\def\autkol{М.\,П.~Кривенко}

\titel{\tit}{\aut}{\autkol}{\titkol}

\index{Кривенко М.\,П.}
\index{Krivenko M.\,P.}


%{\renewcommand{\thefootnote}{\fnsymbol{footnote}} \footnotetext[1]
%{Работа выполнена при финансовой поддержке РФФИ (проекты 16-07-00677 
%и~15-37-20611-мол\_а\_вед).}}


\renewcommand{\thefootnote}{\arabic{footnote}}
\footnotetext[1]{Институт проблем информатики Федерального исследовательского центра <<Информатика и~управление>> Российской академии наук,
\mbox{mkrivenko@ipiran.ru}}

\vspace*{4pt}



\Abst{Рассматриваются принципы построения многомерных референсных регионов
(MRR~--- multivariate reference region). 
Предложен оригинальный метод построения региона на основе областей с~высокой 
плотностью точек и~аппроксимации распределения данных с~помощью смеси нормальных 
распределений. Для оценки порога для плотности распределения используется  
бут\-стреп-ме\-тод. В~качестве эксперимента рассмотрена задача построения 
и~использования эталонной области для прогнозирования типа мочевого камня. Обработка 
реальных данных продемонстрировала преимущества предлагаемых решений.}

\KW{многомерный референсный регион; область высокой плотности; бут\-стреп-ме\-тод; 
смесь многомерных нормальных распределений}

\vspace*{6pt}

\DOI{10.14357/19922264170207} 


\vskip 10pt plus 9pt minus 6pt

\thispagestyle{headings}

\begin{multicols}{2}

\label{st\stat}

\section{Введение}

     Многомерный референсный регион 
был предложен в~литературе по клинической химии в~начале 1970-х~гг.\ как 
альтернатива одномерным референсным интервалам~[1]. Там излагались 
преимущества предлагаемых множественных тестов, хоть и~имеющих 
упрощенный вид, но снижающих (по отношению к~одномерным вариантам) 
число ложных положительных результатов. Появление MRR оказалось 
особенно привлекательным для интерпретации результатов наборов 
медицинских тестов. Тем не менее возникали трудности в~построении 
и~использовании процедур многомерного анализа (см., например,~[2]), 
связанные, в~частности, с~быстрым увеличением числа параметров, которые 
должны быть оценены. Немногие лаборатории использовали MRR в~своей 
практике, причем в~экспериментальном режиме, и,~как следствие, на 
сегодняшний день имеется относительно малое количество соответствующих 
публикаций. 

\vspace*{-6pt}

\section{Многомерный референсный регион на основе расстояния Махалонобиса}

\vspace*{-2pt}

     Одномерный референсный интервал, полученный статистическим путем, 
использует центральную часть значений анализируемого показателя, обычно 
соответствующую~95\% некоторой популяции~--- совокупности особей 
определенного вида (например, здоровой части населения определенного пола 
из некоторого диапазона возрастов). Одномерные референсные интервалы 
применялись в~течение многих лет в~качестве стандартного приема 
интерпретации лабораторных данных. Они легко формируются, хранятся, 
извлекаются и~передаются в~лабораторных информационных системах, просты 
в~понимании, хорошо воспринимаются медицинским сообществом в~ходе 
длительного использования. Тем не менее одномерные референсные интервалы 
при классификации данных могут дать большое число ложно аномальных 
результатов. Этот далеко не единственный недостаток однофакторного 
референсного интервала может быть полностью или частично устранен 
с~помощью MRR.
     
     Простейшим и~весьма распространенным способом построения MRR 
является использование прямого произведения отдельных референсных 
интервалов в~предположении, что они статистически независимы. Пусть 
$(1\hm-\alpha)$~--- вероятность попадания в~MRR, а~$p_0$~--- вероятность 
попадания в~референсный интервал для любого из~$d$~признаков, тогда 
$p_0\hm= \sqrt[d]{1-\alpha}$. С~ростом размерности~$d$ значения~$p_0$ 
быстро приближаются к~1, что фактически лишает смысла применение MRR.
     
     Как и~в одномерном случае, отправной точкой для построения MRR 
может стать нормальное распределение. Идеи центрального расположения 
референсного региона и~заданной вероятности попадания в~него приводят для 
$d$-мер\-но\-го нормального распределения, имеющего плотность 
распределения
     \begin{multline*}
     \varphi(y,\mu,\Sigma) ={}\\
     {}=(2\pi)^{-d/2}\vert\Sigma\vert^{-1/2}\exp \left( -\fr{\left(y-
\mu\right)^{\mathrm{T}} \Sigma^{-1}(y-\mu)}{2}\right),
   \end{multline*}
где величина $(y-\mu)^{\mathrm{T}} \Sigma^{-1} (y-\mu)$ есть квадрат так 
называемого расстояния Махаланобиса между~$y$ и~$\mu$, к~использованию 
многомерного эллипсоида
\begin{multline*}
(2\pi)^{-d/2}\vert\Sigma\vert^{-1/2}\exp \left( -\fr{\left(y-\mu\right)^{\mathrm{T}}
\Sigma^{-1} 
(y-\mu)}{2}\right) ={}\\
{}=const
\end{multline*}
или, что то же самое, 
$$ 
(y-\mu)^{\mathrm{T}} \Sigma^{-1}(y-\mu)=const\,.
$$
Его называют эллипсоидом равной плотности распределения (или просто 
эллипсоидом равной вероятности). 
     
     Если задаться вероятностью $(1\hm-\alpha)$ попадания в~эллипсоид 
равной вероятности вида $(y\hm-\mu)^{\mathrm{T}}\Sigma^{-1} (y\hm-\mu)\hm= 
\rho$, то параметр~$\rho$ удовлетворяет уравнению $\mathrm{Pr}\left\{ 
\chi_d^2\leq \rho\right\} \hm=1\hm-\alpha$.
     
     Использование эллипсоида в~качестве MRR будет оправдано только 
тогда, когда исходное распределение данных есть многомерное нормаль-\linebreak ное. 
Поэтому становятся актуальными критерии\linebreak подгонки, а~также использование 
процедур норма\-ли\-зации распределения данных в~многомерном\linebreak случае.
 Если 
с~помощью тестов выявляется, что распределение не является нормальным, то 
Международная федерация клинической химии и~лабораторной медицины 
рекомендует, согласно~[3], использовать двухступенчатую процедуру 
нормализации. Следует обратить внимание, что многошаговость здесь 
относится не к~многомерности, а касается лишь покоординатного 
преобразования распределения данных к~нормальному.
     
     Первые же попытки применения MRR на основе расстояния 
Махалонобиса (фактически это означает принятие модели нормального 
распределения референсных значений) выявили ряд недостатков (более 
подробно смотри в~\cite[разд.~6.2]{4-kri}):
     \begin{itemize}
\item проявление <<проклятий>> размерности при механическом 
увеличении~$d$, в~особенности если игнорируется этап анализа состава 
признаков~[1, 5, 6];
\item из-за небольших объемов обучающей выборки невысокая устойчивость 
при применении, в~частности чувствительность к~увеличению неточностей 
измерений после того, как регион был установлен~\cite{5-kri, 7-kri}. 
\item предположение о нормальном распределении и~попытки <<подправить>> 
действительность с~помощью преобразований реальных данных для их 
нормализации при увеличении размерности данных становятся все более 
шаткими~\cite{5-kri};
\item представление и~интерпретация выводов на основе MRR трудно 
понимаемы не только для специалистов в~предметной области~[8].
\end{itemize}

\vspace*{-9pt}

\section{Многомерный референсный регион высокой плотности}

\vspace*{-2pt}

     Заметим, что в~случае нормального распределения референсных значений 
для точек внут\-ри построенного эллипсоида значения плотности\linebreak распределения 
больше, чем на границе, а~вне~--- меньше. Это замечание позволяет 
предложить другой подход к~построению MRR.
     
     \smallskip
     
     \noindent
     \textbf{Определение.}\ Eсли плотность распределения референсных 
значений есть $f(y)$, то MRR есть область $A_t\hm= \left\{ y\in 
\mathcal{R}^d\vert f(y)\hm\geq t\right\}$ для некоторого порогового 
значения~$t$. 
     
     \smallskip
     
     Для нормального распределения это уже упомянутый эллипсоид равной 
вероятности. Если задается вероятность $(1\hm-\alpha)$ попадания в~$A_t$, то 
пороговое значение~$t$ есть решение уравнения $\int\nolimits_{A_t} 
f(u)\,du\hm=1\hm-\alpha$, получить которое аналитически в~случае 
произвольной плотности распределения вряд ли возможно. Здесь присутствуют 
две проблемы: вычисление многомерного интеграла и~зависимость области 
интегрирования от неизвестного значения. Для решения их предлагается 
привлечь метод моделирования.
     
     Сгенерируем выборку из $f(y)$, которую обозначим как $Y^f\hm= \left\{ 
y_1^f, \ldots, y_m^f\right\}$. Для оценки $\int\nolimits_{A_t} f(u)\,du$ 
используем отношение:

\noindent
\begin{multline*}
     \fr{\left\vert \left\{ y_i^f\vert y_i^f\in A_t\right\}\right\vert }{m} =
      \fr{\left\vert\left\{ y_i^f\vert 
f\left(y_i^f\right) \geq t\right\}\right\vert }{m} ={}\\
{}= 1-\fr{\left\vert \left\{ y_i^f\vert f(y_i^f)<t\right\}\right\vert }{m}=1-
F_m(t)\,,
     \end{multline*}
где $F_m(t)$~--- эмпирическая функция распределения случайной 
величины~$f(y)$, т.\,е.\ случайной величины, являющейся результатом 
преобразования с~помощью функции~$f(\cdot)$ случайной величины, име\-ющей 
плотность распределения~$f(u)$.

     Таким образом, искомая оценка~$t^*$ должна удовле\-тво\-рять уравнению 
$F_m(t^*)\hm=\alpha$ и~может быть получена как непараметрическая оценка 
квантиля\linebreak\vspace*{-12pt}

\pagebreak

\noindent
 порядка~$\alpha$ из распределения $F_m(\cdot)$. Если обозначить 
$f_i\hm= f(y_i^f)$, то~$t^*$ есть~$f_{(r)}$, где
     $$
     r= \begin{cases}
     m\alpha, &\ m\alpha~\mbox{---~целое}\,;\\
     \lfloor m\alpha+1\rfloor\,, & m\alpha~\mbox{--- не целое}\,.
     \end{cases}
     $$
     Заметим, что для такой оценки можно указать доверительный интервал.
     
     Для построения MRR необходимо знать распределение данных. При 
реализации принципа точек высокой плотности в~первую очередь следует 
обратиться к~параметрическим моделям, в~част\-ности к~смеси нормальных 
распределений, име\-ющей плотность распределения
     $$
     f(u) =\sum\limits_{j=1}^k p_j \varphi\left (u,\mu_j, \Sigma_j\right)\,.
     $$
Если $\hat{f}(u)$~--- оценка смеси, то~$t^*$ строится сле\-ду\-ющим образом:
\begin{itemize}
\item генерируется выборка $\left\{ y_1^f,\ldots , y_m^f\right\}$ из $\hat{f}(u)$ и~
для каждого ее $i$-го элемента подсчитывается значение $\hat{f}\left( 
y_i^f\right)$;
\item в~качестве~$t^*$ берется непараметрическая оценка квантиля 
порядка~$\alpha$ (в случае необходимости дополнительно находится 
непараметрическая оценка доверительного интервала для~$t^*$, что 
может характеризовать правильность выбранного объема для 
генерируемой выборки).
\end{itemize}

     Пусть для $f(u)$ имеется~$A_t$, а также получена $\hat{f}(u)$ 
и~соответствующий MRR вида~$\hat{A}_t$. Качество аппроксимации~$A_t$ 
с~по\-мощью~$\hat{A}_t$ можно оценить с~по\-мощью вероятности совпадения 
этих областей, т.\,е. 
     $$
     P_c= \int\limits_{\{ u\in A_t\}\cup \{u\in \hat{A}_t\}} \hspace*{-6mm}
f(u)\,du+\int\limits_{\{u\not\in A_t\} \cup\{ u\not\in \hat{A}_t\}}\hspace*{-6mm} f(u)\,du\,.
     $$
     
     Для оценки  $P_c$ можно использовать величину
     \begin{multline*}
     \hat{P}_c= \fr{\left\vert \left\{ 
     y_i^f\vert y_i^f \in \left\{\left\{ y_i^f\in A_t\right\}\cup \left\{y_i^f\in 
\hat{A}_t\right\}\right\}\right\}\right\vert}{m}+{}\\
{}+\fr{\left\vert \left\{ y_i^f\vert y_i^f \in \left\{\left\{ y_i^f\not\in A_t\right\}\cup 
\left\{ y_i^f\not\in \hat{A}_t\right\}\right\}\right\}\right\vert}{m}\,.
     \end{multline*}
     
     Использование MRR высокой плотности для диагностирования сводится 
к~реализации так называемого слабого критерия значимости для наблюденного 
значения~$x$: нулевая гипотеза заключается в~том, что $x\hm\in A_t$, 
статистика критерия есть $\hat{f}(x)$ и~решение о~принадлежности 
критической об\-ласти~$A_t$ принимается при больших значениях~$\hat{f}(x)$.
     
     Для медицинской практики важна возможность использования 
референсного региона при интерпретации результатов обследования 
некоторого пациента с~вектором признаков~$x$. В~подобных случаях 
сложившейся практикой для слабых критериев значимости является 
использование критического уровня~$\alpha_{\mathrm{cr}}$ (более распространенным 
в~медицине является употребление термина $p$-зна\-че\-ние)  $\alpha_{\mathrm{cr}}\hm= 
\mathrm{Pr}\left\{ \hat{f}(y)\hm\leq \hat{f}(x)\right\}$, где $y$~--- случайная 
величина, имеющая плотность распределения~$\hat{f}(u)$, а $\hat{f}(x)$~--- 
значение плотности распределения~$\hat{f}(u)$ в~точке~$x$. Эта 
характеристика дает представление о~том, насколько сильно данное 
наблюденное значение~$x$ противоречит гипотезе (или подкрепляет ее) 
о~принадлежности данных MRR. При выбранном же заранее уровне 
значимости с~помощью~$\alpha_{\mathrm{cr}}$ сразу же можно принять конкретное 
решение. 

\vspace*{-9pt}

\section{Эксперименты}

\vspace*{-2pt}

     Для демонстрации возможностей MRR использовались данные по 
прогнозу химического состава мочевых камней по метаболическим 
показателям мочи и~сыворотки крови, а также антропологическим 
характеристикам пациентов~[9]. В качестве исходной классификации камней 
рассматривалась следующая: чисто оксалатные (далее обозначены как O), чисто 
уратные (U), чисто фосфатные (P), смесь только оксалатных и~уратных (OU), 
смесь только оксалатных и~фосфатных (OP), смесь только уратных 
и~фосфатных (UP), все остальные. Данная классификация была построена 
в~[10] на основе доминирующих частот встречаемости основных компонентов. 
В~качестве референсных значений рассматривались наборы метаболических 
и~антропологических показателей (их всего было~14), соответствующих 
определенному классу камней.

\begin{table*}\small
\begin{center}


\begin{tabular}{|c|c|c|c|c|c|c|}
\multicolumn{7}{c}{Качество классификации с~помощью MRR}\\
\multicolumn{7}{c}{\ }\\[-6pt]
\hline
\multicolumn{1}{|c|}{\raisebox{-6pt}[0pt][0pt]{\tabcolsep=0pt\begin{tabular}{c}Тип\\ камня\end{tabular}}}&
\multicolumn{1}{c|}{\raisebox{-6pt}[0pt][0pt]{$N$}}&$(1-\alpha)$, 
&\multicolumn{2}{c|}{MRR(5)}&\multicolumn{2}{c|}{MRR(1)}\\
\cline{4-7}
&&&&&&\\[-9pt]
&&\%&$(1-\hat{\alpha})$, \%&$\hat{\beta}$, \%&$(1-\hat{\alpha})$, \%&$\hat{\beta}$, \%\\
\hline
\multicolumn{1}{|c|}{\raisebox{-18pt}[0pt][0pt]{O}}&
\multicolumn{1}{c|}{\raisebox{-18pt}[0pt][0pt]{82}}
&95&100\hphantom{9}&71&90&24\\
&&85&96&78&89&36\\
&&75&91&85&77&44\\
&&65&76&88&74&50\\
\hline
\multicolumn{1}{|c|}{\raisebox{-18pt}[0pt][0pt]{U}}&
\multicolumn{1}{c|}{\raisebox{-18pt}[0pt][0pt]{76}}&95&100\hphantom{9}&75&91&24\\
&&85&99&85&80&35\\
&&75&82&89&74&48\\
&&65&71&91&68&56\\
\hline
\multicolumn{1}{|c|}{\raisebox{-18pt}[0pt][0pt]{P}}&
\multicolumn{1}{c|}{\raisebox{-18pt}[0pt][0pt]{83}}&95&100\hphantom{9}&66&87&25\\
&&85&94&78&86&33\\
&&75&86&82&82&41\\
&&65&77&87&75&47\\
\hline
\end{tabular}
\end{center}
\end{table*}
     
     
     Для каждого из основных классов O, U, P, OU, OP и~UP перед построением 
MRR проводилась селекция признаков и~принималось то значение размерности 
признакового пространства~$d$ и~соответствующий набор показателей, 
которые позволяли прогнозировать состав камней без потери качества 
(методика описана в~\cite{9-kri} и~привела к~значению $d\hm=9$). В~качестве 
модели данных в~первую очередь рассматривалась смесь многомерных 
нормальных распределений из пяти элементов (подбор числа элементов смеси 
проводился с~по\-мощью AIC~--- Akaike information criterion), для соответствующего региона было принято 
обозначение MRR(5). Для сравнения также использовалась модель 
нормального распределения, которой соответствовал MRR(1). Полученные 
результаты приводятся час\-тич\-но в~таблице, где $N$~--- объем 
классифицируемых данных; $\hat{\alpha}$~--- оценка для~$\alpha$; 
$\hat{\beta}$~--- оценка мощности критерия при определении типа камня на 
основании MRR.


     Одной из базовых характеристик является вероятность попадания в~MRR 
$(1\hm-\alpha)$ и~ее оценка $(1\hm-\hat{\alpha})$. Сравнение соответствующих 
столбцов с~учетом значений~$N$ и~ориентировочных значений разброса 
(стандартные отклонения на основе биномиального распределения) не 
позволило выявить явных отклонений. Необходимо, правда, отметить, что во 
всех проанализированных случаях для MRR(5) оказалось, что $1\hm-
\hat{\alpha}\hm\geq 1\hm-\alpha$.
     
     Назначение MRR, заключающееся в~сжатом представлении референсных 
значений, в~многомерном случае практически не проявляется. Для задания 
MRR(5) необходимо указать следующие величины: $1\hm-\alpha$, $t$, 
$p_1,\ldots, p_{k-1}$, $\mu_1, \Sigma_1,\ldots , \mu_k,\Sigma_k$, общее 
количество которых равно  $[2\hm+ (k\hm-1)\hm+ k(d\hm+ d(d\hm+1)/2)]$ 
и,~в~частности, в~рассматриваемых экспериментах~--- 276. Для MRR(1) это 
значение меньше и~равно~56. При этом для обрабатываемой обучающей 
выборки в~зависимости от класса камней речь идет о~порядка~10$^2$ векторах 
данных (см.\ столбец со значениями~$N$), что приблизительно 
дает~10$^3$~скалярных величин.
     
     Другое назначение MRR состоит в~его использовании для 
диагностирования (классификации). В~этой связи в~первую очередь 
проводился сравнительный анализ MRR(1) (фактически это означает, что 
построение региона осуществляется на основе расстояния Махаланобиса) 
и~MRR(5) (модель смеси нормальных распределений и~предложенный 
в~данной работе метод оценивания па\-ра\-мет\-ров региона). Показателем 
информативности метода построения многомерного региона выступала 
мощность соответствующего слабого критерия значимости, а~именно: 
вероятность не попасть в~MRR при условии, что данные берутся из дополнения 
к~классу, для которого построена MRR. Сравнение соответствующих столбцов 
говорит о~явном преимуществе двух предложенных моментов: усложнение 
модели данных путем перехода от нормального распределения к~смеси 
нормальных распределений и~построение региона высокой плотности.
     
     Использование критического уровня можно продемонстрировать  
с~по\-мощью зависимости результатов сравнения двух классов от того, какой 
класс взять за основу. Введем для возможных значений $p$-ве\-ли\-чи\-ны три 
интервала: $(-\infty, 1\%)$, $[1\%, 5\%)$, $[5\%, 100\%)$ с~соответствующей 
интерпретацией положения наблюденного набора показателей для пациента 
относительно построенного MRR: уверенное непопадание, неуверенное 
попадание, уверенное попадание. Если MRR построить для оксалатных камней, 
то результаты для анализа пациентов с~фосфатными камнями дадут следующий 
вектор относительных частот попадания $p$-ве\-ли\-чин в~указанные 
интервалы: $(60\%, 18\%, 22\%)$. Если же MRR строить для фосфатных 
камней, то получим $(71\%, 5\%, 24\%)$. Таким образом, для классификации 
указанных камней при приблизительно одинаковых частотах попадания в~MRR 
(22\% или~24\%) уверенный отказ от референсного региона происходит чаще, 
если принять за базовый MRR регион для фосфатных камней. Построение 
шкалы, подобной рассмотренной, является прерогативой специалистов 
в~предметной области, в~данной работе она использовалась только для 
иллюстрации. 

\vspace*{-6pt}

\section{Заключение}

\vspace*{-2pt}

     На настоящий момент имеется относительно мало примеров применения 
MRR в~клинической практике. Тому есть несколько причин. Математическое 
обеспечение, необходимое для получения и~применения MRR, не отвечает 
возможностям большинства клинических лабораторий. Лаборатории слабо 
оснащены программными средствами\linebreak для реализации достаточно сложного 
математического аппарата многомерного анализа, а~еще важнее, что 
отсутствуют методики, инструкции по\linebreak использованию соответствующих 
средств. Лишь немногие клинические применения демонстрируют 
преимущества MRR, хотя свидетельств неудачных попыток больше.
     
     Несмотря на сложности внедрения мно\-го\-мерно\-го анализа референсных 
значений, можно сформулировать некоторые рекомендации по иссле\-до\-ва\-нию 
и~разработке MRR. Во-пер\-вых, эффективная размерность в~MRR должна 
быть как можно меньше, чтобы избежать затенения диагностически полезной 
информации тестами, со\-зда\-ющи\-ми шум. Низкая размерность также должна 
уменьшить неблагоприятные последствия увеличения неточности результатов 
в~связи с~ростом числа анализируемых показателей. Во-вто\-рых, показатели 
(тес\-ты), включенные в~MRR, должны быть физиологически релевантными 
исследуемому кругу расстройств, чтобы максимизировать информацию, 
полученную от MRR. В-треть\-их, чтобы учесть эффекты долгосрочной 
лабораторной из\-мен\-чи\-вости, данные, используемые для получения MRR, 
долж\-ны быть собраны и~проанализированы в~течение достаточно большого 
периода времени (от нескольких недель до нескольких месяцев).  
В-чет\-вер\-тых, представление результатов лабораторных исследований 
следует осуществлять в~графическом виде, чтобы помочь врачам лучше понять 
MRR. Различные подходы к~уменьшению размерности помогут выполнить это 
требование.
     
     Необходима дальнейшая разработка пояснительных инструментов, 
способных воспринять результаты анализа MRR. При этом дополнительно 
необходима информация о~том, какие именно тес\-ты являются важнейшими 
факторами нарушения нормы. Надо признать, что соответствующий 
математический аппарат еще предстоит разработать. Решение перечисленных 
вопросов играет важную роль для обеспечения постоянного клинического 
применения MRR. 

\vspace*{-6pt}
     
{\small\frenchspacing
 {%\baselineskip=10.8pt
 \addcontentsline{toc}{section}{References}
 \begin{thebibliography}{99}
 
 \vspace*{-2pt}
 
\bibitem{1-kri}
\Au{Boyd J.\,C.} Reference regions of two or more dimensions~// Clin. Chem. Lab. 
Med., 2004. Vol.~42. No.\,7. P.~739--746.
\bibitem{2-kri}
\Au{Winkel P.} Patterns and clusters~--- multivariate approach for interpreting 
clinical chemistry results~// Clin. Chem., 1973. Vol.~19. No.\,12. P.~1329--1333.
\bibitem{3-kri}
IFCC. Expert panel on theory of reference values. Approved recommendation on the 
theory of reference values. Part~5. Statistical treatment of collected reference values. 
Determination of reference limits~// J.~Clin. Chem. Clin. Biochem., 1987. Vol.~25. 
No.\,9. P.~645--656.
\bibitem{4-kri}
\Au{Кривенко М.\,П.} Статистические методы представления и~предварительной 
обработки референсных значений.~--- М.: ФИЦ ИУ РАН, 2016. 160~с.
\bibitem{5-kri}
\Au{Boyd J.\,C., Lacher~D.\,A.} The multivariate reference range: An alternative 
interpretation of multi-test profiles~// Clin. Chem., 1982. Vol.~28. No.\,2.  
P.~259--265.
\bibitem{6-kri}
\Au{Albert A., Harris~E.\,K.} Multivariate interpretation of clinical laboratory  
data.~--- New York, NY, USA: CRC Press, 1987. 328~p.
\bibitem{7-kri}
\Au{Linnet K.} Influence of sampling variation and analytical errors on the 
performance of the multivariate reference region~// Meth. Inf. Med., 1988. Vol.~27. 
No.\,1. P.~37--42.
\bibitem{8-kri}
\Au{Durbridge T.\,C.} Clinical acceptance of a multi-test reference region for 
biochemical-panel results~// Clin. Chem., 1983. Vol.~29. No.\,10. P.~1724--1726.
\bibitem{9-kri}
\Au{Кривенко М.\,П.} Критерии значимости отбора признаков классификации~// 
Информатика и~её применения, 2016. Т.~10. Вып.~3. С.~32--40.
\bibitem{10-kri}
\Au{Кривенко М.\,П., Голованов~С.\,А., Сивков~А.\,В.} Анализ однородности 
данных о химическом составе камней при уролитиазе~// Информатика и~её 
применения, 2013. Т.~7. Вып.~4. С.~94--104.
 \end{thebibliography}

 }
 }

\end{multicols}

\vspace*{-10pt}

\hfill{\small\textit{Поступила в~редакцию 5.12.16}}

\vspace*{4pt}

%\newpage

%\vspace*{-24pt}

\hrule

\vspace*{2pt}

\hrule

\vspace*{-3pt}


\def\tit{HIGH-DENSITY MULTIVARIATE REFERENCE REGION\\[-5pt]}

\def\titkol{High-density multivariate reference region}

\def\aut{M.\,P.~Krivenko\\[-7pt]}

\def\autkol{M.\,P.~Krivenko}

\titel{\tit}{\aut}{\autkol}{\titkol}

\vspace*{-16pt}


\noindent
Institute of Informatics Problems, Federal Research Center 
``Computer Science and Control'' of the Russian
Academy of Sciences,  44-2~Vavilov Str., Moscow 119333, Russian Federation



\def\leftfootline{\small{\textbf{\thepage}
\hfill INFORMATIKA I EE PRIMENENIYA~--- INFORMATICS AND
APPLICATIONS\ \ \ 2017\ \ \ volume~11\ \ \ issue\ 2}
}%
 \def\rightfootline{\small{INFORMATIKA I EE PRIMENENIYA~---
INFORMATICS AND APPLICATIONS\ \ \ 2017\ \ \ volume~11\ \ \ issue\ 2
\hfill \textbf{\thepage}}}

\vspace*{2pt}




\Abste{The paper considers the principles of construction of multivariate 
reference regions. An original method of construction of 
a~region on the basis of areas of high density of points and approximation 
of data distribution with a~mixture of normal distributions is suggested. 
To estimate the threshold for the probability density, the bootstrap method is used. 
As an experiment, the paper considers the problem of description and use of 
the reference region for predicting the type of urinary stones. 
Real data treatment demonstrated the benefits of the proposed solutions.}

\KWE{multivariate reference region; high-density region; bootstrap method; 
multivariate normal mixture}

\DOI{10.14357/19922264170207} 

%\vspace*{-18pt}

%\Ack
%\noindent



%\vspace*{3pt}

  \begin{multicols}{2}

\renewcommand{\bibname}{\protect\rmfamily References}
%\renewcommand{\bibname}{\large\protect\rm References}

{\small\frenchspacing
 {%\baselineskip=10.8pt
 \addcontentsline{toc}{section}{References}
 \begin{thebibliography}{99}
\bibitem{1-kri-1}
\Aue{Boyd, J.\,C.} 2004. Reference regions of two or more dimensions. \textit{Clin. 
Chem. Lab. Med.} 42(7):739--746.

\bibitem{2-kri-1}
\Aue{Winkel, P.} 1973. Patterns and clusters~--- multivariate approach for interpreting 
clinical chemistry results. \textit{Clin. Chem.} 19(12):1329--1333.
\bibitem{3-kri-1}
IFCC. 1987. Expert panel on theory of reference values. Approved recommendation on the 
theory of reference values. Part~5. Statistical treatment of collected reference values. 
Determination of reference limits. \textit{J.~Clin. Chem. Clin. Biochem.} 
25(9):645--656.
\bibitem{4-kri-1}
\Aue{Krivenko, M.\,P.} 2016. \textit{Statisticheskie metody predstavleniya 
i~predvaritel'noy obrabotki referensnykh znacheniy}
[Statistical methods for representation and preliminary processing of
reference values]. Moscow: FRC CSC RAS. 160~p.

\bibitem{5-kri-1}
\Aue{Boyd, J.\,C., and D.\,A.~Lacher.} 1982. The multivariate reference range: An 
alternative interpretation of multi-test profiles. \textit{Clin. Chem.}  
28(2):259--265.
\bibitem{6-kri-1}
\Aue{Albert, A., and E.\,K.~Harris.} 1987. \textit{Multivariate interpretation of 
clinical laboratory data}. New York, NY: CRC Press. 328~p.
\bibitem{7-kri-1}
\Aue{Linnet, K.} 1988. Influence of sampling variation and analytical errors on the 
performance of the multivariate reference region. \textit{Meth. Inf. Med.}  
27(1):37--42.
\bibitem{8-kri-1}
\Aue{Durbridge, T.\,C.} 1983. Clinical acceptance of a multi-test reference region 
for biochemical-panel results. \textit{Clin. Chem.} 29(10):1724--1726.
\bibitem{9-kri-1}
\Aue{Krivenko, M.\,P.} 2016. Kriterii znachimosti otbora priznakov klassifikatsii
[Significance tests of feature selection for~classification]. \textit{Informatika i~ee 
Primeneniya~--- Inform. Appl.} 10(3):32--40.
\bibitem{10-kri-1}
\Aue{Krivenko, M.\,P., S.\,A.~Golovanov, and A.\,V.~Sivkov}. 2013. Analiz 
odnorodnosti dannykh o~khimicheskom sostave kamney pri urolitiaze
[Analysis of data homogeneity of~the~chemical compositions 
of~stones in~case of~urolithiasis]. \textit{Informatika i~ee Primeneniya~---
Inform Appl.} 7(4):94--104.
\end{thebibliography}

 }
 }

\end{multicols}

\vspace*{-3pt}

\hfill{\small\textit{Received December 5, 2016}}


\Contrl

\noindent
\textbf{Krivenko Michail P.} (b.\ 1946)~--- Doctor of Science in technology, 
professor, leading scientist, Institute of Informatics Problems, Federal Research 
Center ``Computer Science and Control'' of the Russian Academy of Sciences, 
\mbox{44-2}~Vavilov Str., Moscow 119333, Russian Federation; \mbox{mkrivenko@ipiran.ru}

\label{end\stat}


\renewcommand{\bibname}{\protect\rm Литература}  %8
\def\stat{bazilevsky}

\def\tit{МЕТОД ВЫПРЯМЛЕНИЯ\\ ИСКАЖЕННЫХ ИЗ-ЗА~МУЛЬТИКОЛЛИНЕАРНОСТИ\\ 
КОЭФФИЦИЕНТОВ В~РЕГРЕССИОННЫХ МОДЕЛЯХ}

\def\titkol{Метод выпрямления искаженных из-за~мультиколлинеарности 
коэффициентов в~регрессионных моделях}

\def\aut{М.\,П.~Базилевский$^1$}

\def\autkol{М.\,П.~Базилевский}

\titel{\tit}{\aut}{\autkol}{\titkol}

\index{Базилевский М.\,П.}
\index{Bazilevskiy M.\,P.}

%{\renewcommand{\thefootnote}{\fnsymbol{footnote}} \footnotetext[1]
%{Работа выполнена при финансовой поддержке РФФИ (проект 16-29-09458~офи\_м).}}


\renewcommand{\thefootnote}{\arabic{footnote}}
\footnotetext[1]{Иркутский государственный университет путей сообщения, кафедра математики, 
\mbox{mik2178@yandex.ru}}


\vspace*{-12pt}



  \Abst{При построении регрессионной модели из-за сильной мультиколлинеарности 
объясняющих переменных происходит искажение ее коэффициентов, в~частности их знаков, 
что негативно сказывается на интерпретационных качествах такой регрессии. Статья посвящена 
разработке метода выпрямления искаженных из-за мультиколлинеарности коэффициентов. 
В~основе этого метода лежит свойство, которым обладают ранее предложенные автором 
модели полносвязной линейной регрессии. Исследована нелинейная система, по которой 
осуществляется оценивание полносвязных регрессий. Показано, что решение этой системы 
может быть получено численно с~помощью метода простых итераций. Предложен способ 
выбора неизвестных лямбда-параметров в~полносвязной регрессии. Установлено, что 
в~многофакторных полносвязных моделях при сильной корреляции всех факторов знаки 
коэффициентов при переменных во вторичном уравнении совпадают с~соответствующими 
знаками коэффициентов корреляции. Для выпрямления искаженных коэффициентов на основе 
проведенного исследования разработан алгоритм <<Selection~B>>. Разработанный метод 
выпрямления успешно продемонстрирован на примере моделирования валового внутреннего продукта (ВВП) России.}
  
  \KW{регрессионный анализ; модель полносвязной линейной регрессии; 
мультиколлинеарность; интерпретация; численный метод; ВВП России}

\DOI{10.14357/19922264210209}

%\vspace*{-3pt}


\vskip 10pt plus 9pt minus 6pt

\thispagestyle{headings}

\begin{multicols}{2}

\label{st\stat}
  
\section{Введение}

  При оценивании неизвестных параметров регрессионных моделей, например 
с~по\-мощью метода наименьших квадратов (МНК), на практике часто 
приходится сталкиваться с~проблемой мультиколлинеарности~[1, 2]. Это 
негативное явление возникает из-за наличия сильной корреляции между двумя 
или более независимыми переменными. Мультиколлинеарность факторов 
приводит к~искажению коэффициентов в~уравнении регрессии. В~частности, их 
знаки могут противоречить теоретическим предпосылкам решаемой задачи. 
Поэтому построенная при мультиколлинеарности регрессионная модель остается 
годной в~лучшем случае только для прогнозирования, но никак не для 
интерпретации и~принятия ка\-ких-ли\-бо правильных управленческих решений.
  
  Проблема мультиколлинеарности на сегодня еще окончательно не решена. 
Существуют лишь несколько основных подходов к~ее устранению~[3,~4].
{\looseness=-1

}
  
  Во-первых, это метод исключения~\cite{4-baz}. Он заключается в~том, что на 
основе матрицы парных коэффициентов корреляции определяются пары сильно 
коррелированных объясняющих переменных. Затем из каждой пары исключается 
тот фактор, который слабее коррелирует с~зависимой переменной. После чего по 
оставшимся факторам оценивается регрессионная модель. Недостаток данного 
подхода состоит в~том, что в~полученном уравнении из-за исключения нельзя 
изучать совместное влияние всех исходных объясняющих переменных на 
объясняемую.
  
  Во-вторых, метод главных компонент~[5]. С~помощью этого способа 
происходит формирование\linebreak новых и~не коррелирующих между собой 
переменных~--- главных компонент, являющихся линейными комбинациями 
старых переменных. К~сожалению, в~этом случае возникает проблема 
с~\mbox{интерпретацией} главных компонент.
  
  В-третьих, ридж-регрессия~[6]. В~этом случае в~формулу для  
МНК-оце\-ни\-ва\-ния регрессии до\-бав\-ля\-ет\-ся так называемый коэффициент 
регуляризации, который решает проблему мультиколлинеарности. Однако нет 
четких правил для выбора этого коэффициента. И~нет гарантии, что в~полученной 
модели коэффициенты будут удовлетворять содержательному смыслу задачи.
  
  Как справедливо отмечено в~работе~[7], все эти методы ориентированы на 
устранение только вычислительных проб\-лем. Но проблема, связанная 
с~построением интерпретируемых при мультиколлинеарности регрессионных 
моделей, остается нерешенной.
  
  Целью данной работы ставилась разработка метода выпрямления искаженных 
из-за мультиколлинеарности коэффициентов линейных регрессионных моделей. 
Основой для этого метода послужило замеченное автором свойство 
двухфакторных полносвязных регрессий~[8, 9], состоящее в~том, что в~их 
вторичных уравнениях знаки коэффициентов при объясняющих переменных 
совпадают с~соответствующими знаками коэффициентов корреляции. Это же 
свойство может быть справедливо и~для многофакторных моделей полносвязной 
линейной регрессии, впервые предложенных в~работе~[10].
  
\section{Многофакторная модель полносвязной линейной регрессии 
без~выходной переменной}

  Пусть $x_{ij}$, $i\hm=\overline{1,n}$, $j\hm=\overline{1,m}$,~--- наблюдаемые 
значения~$m$~входных переменных $x_1, x_2, \ldots , x_m$. Предположим, что 
существуют их <<истинные>> значения $x^*_{i1}, x^*_{i2},\ldots , x^*_{im}$, 
$i\hm=\overline{1,n}$, связанные с~наблюдаемыми значениями соотношениями
  \begin{equation}
  x_{ij}= x^*_{ij}+\varepsilon_i^{(x_j)}\,,\enskip i=\overline{1,n}\,,\ 
j=\overline{1,m}\,.
  \label{e1-baz}
  \end{equation}
  
  Предположим, что между <<истинными>> переменными $x_1^*, x_2^*, \ldots , 
x_m^*$ имеют место функциональные зависимости
  \begin{equation}
  x_j^*=a_j+b_jx_m^*\,,\enskip j=\overline{1,m-1}\,,
  \label{e2-baz}
  \end{equation}
где $a_j$ и~$b_j$, $j\hm=\overline{1,m-1}$,~--- неизвестные параметры.

  Совокупность уравнений~(\ref{e1-baz}) и~(\ref{e2-baz}) называется 
многофакторной моделью полносвязной линейной регрессии без выходной 
переменной~\cite{10-baz}.
  %
  Для ее оценивания применим взвешенный метод наименьших полных 
квадратов (ВМНПК):
  \begin{multline}
  S=\lambda_1\sum\limits^n_{i=1} \left( x_{i1}-a_1-b_1x^*_{im}\right)^2 
+{}\\
{}+
\lambda_2\sum\limits^n_{i=1} \left( x_{i2}-a_2-b_2x^*_{im}\right)^2+\cdots {}\\
  {}\cdots\ +\lambda_{m-1}\sum\limits^n_{i=1}\left( x_{i,m-1}-a_{m-1}-b_{m-1} 
x^*_{im}\right)^2+{}\\
{}+\sum\limits^n_{i=1} \left( x_{im} -x^*_{im}\right)^2\to \min\,,
  \label{e3-baz}
  \end{multline}
где $\lambda_1, \lambda_2,\ldots , \lambda_{m-1}$~--- положительные весовые 
коэффициенты (лямбда-параметры).
  
  В работе~\cite{10-baz} показано, что если лямб\-да-па\-ра\-мет\-ры известны, то 
решение задачи~(\ref{e3-baz}) осуществляется по следующему алгоритму.
  \begin{enumerate}[1.]
\item Находятся оценки $\tilde{b}_1, \tilde{b}_2,\ldots , \tilde{b}_{m-1}$ 
параметров $b_1, b_2, \ldots, b_{m-1}$. Для этого численно решается нелинейная 
система вида:
\begin{multline} 
b_p\left( D_{x_m}+\sum\limits_{j=1}^{m-1} \lambda^2_j b_j^2 
D_{x_j} +{}\right.\\
{}+2\sum\limits^{m-2}_{j_1=1} \sum\limits^{m-1}_{j_2=j_1+1} 
\lambda_{j_1} \lambda_{j_2} b_{j_1}b_{j_2} 
K_{x_{j_1}x_{j_2}}+{}\\
\left.{}+2\sum\limits_{j=1}^{m-1} \lambda_j b_j 
K_{x_jx_m}\right)= \left( 1+\sum\limits_{j=1}^{m-1} \lambda_j b_j^2\right)\times{}\\
\hspace*{-5mm}{}\times \left( 
\sum\limits_{j=1}^{m-1} \lambda_j b_j K_{x_jx_p}+K_{x_mx_p}\right),\enskip 
  p=\overline{1,m-1}\,.\!\!
  \label{e4-baz}
  \end{multline}
\item Определяются оценки $\tilde{a}_1, \tilde{a}_2, \ldots , \tilde{a}_{m-1}$ 
па\-ра\-мет\-ров $a_1, a_2, \ldots , a_{m-1}$ по формулам:
\begin{equation}
\tilde{a}_j= \overline{x}_j -\tilde{b}_j \overline{x}_m\,,\enskip j\hm=\overline{1,m-
1}\,.
\label{e5-baz}
\end{equation}
\item Вычисляются оценки <<истинных>> значений переменной~$x_m$ по 
формулам:
\begin{multline}
\tilde{x}_{im}^* =\left( 1+\sum\limits^{m-1}_{j=1} \lambda_j \tilde{b}_j^2\right)^{-
1} \left( -\sum\limits_{j=1}^{m-1} \lambda_j \tilde{a}_j \tilde{b}_j 
+{}\right.\\
\left.{}+\sum\limits_{j=1}^{m-1} \lambda_j \tilde{b}_j x_{ij} +x_{im}\right),\enskip
i=\overline{1,n}\,.
\label{e6-baz}
\end{multline}
  \end{enumerate}
  
  Очевидно, что если абсолютные значения парных коэффициентов корреляции 
переменных $x_1, x_2,\ldots , x_m$ равны~1, то при оценивании полносвязной 
регрессии по критерию~(\ref{e3-baz}) все остатки будут равны~0, а ее 
оценки~$\tilde{b}_i$, $i\hm=\overline{1,m-1}$, будут совпадать  
с~МНК-оцен\-ка\-ми соответствующих парных регрессий. А~знаки этих оценок 
согласуются со знаками соответствующих коэффициентов 
корреляции~$r_{x_ix_m}$, $i\hm=\overline{1,m-1}$, т.\,е.\ справедливы условия 
$\tilde{b}_i r_{x_ix_m}\hm>0$, $i\hm=\overline{1,m-1}$. Значит, они будут справедливы и~при сильной 
корреляции факторов.

\section{Численный метод решения нелинейной  
системы~(\ref{e4-baz})}

  Систему~(\ref{e4-baz}) нетрудно привести к~виду:
  \begin{equation}
  H_p b_p^2 +B_p b_p +C_p=0\,,\enskip p=\overline{1,m-1}\,,
  \label{e7-baz}
  \end{equation}
  где 
 \begin{align*}
  H_p&=\lambda_p\!\left( K_{x_mx_p}+\sum\limits_{j\in \{1,\ldots, m-1\}\backslash \{p\}} 
\!\!\!\!\!\!\!\!\!\!\lambda_j b_j K_{x_jx_p}\right)\,;
 \\
  B_p&=D_{x_m}+\sum\limits_{ j\in \{1,\ldots, m-1\}\backslash\{p\}} \!\!\!\!\!\!\!\!\!\!\lambda_j^2 b_j^2 
D_{x_j} + {}\\
  &\hspace*{-20pt}{}+ 2\!\!\!\!\!\!\sum\limits_{ j_1\in \{1,\ldots, m-2\}\backslash \{p\}} \sum\limits_{ j_2\in 
\{j_1+1,\ldots, m-1\}\backslash \{p\}}\!\!\!\!\!\!\!\!\!\!\!\!\!\!\!
 \lambda_{j_1} \lambda_{j_2} b_{j_1} b_{j_2} 
K_{x_{j_1} x_{j_2}}+{}\\
  &{}+2\!\!\! \sum\limits_{ j\in \{1,\ldots, m-1\}\backslash \{p\}} \!\!\!\!\!\!\!\!\!\!\lambda_jb_jK_{x_jx_m}-
\lambda_pD_{x_p} -\lambda_p D_{x_p} \times{}\\
&\hspace*{35mm}{}\times \sum\limits_{ j\in \{1,\ldots, m-1\}\backslash 
\{p\}} \!\!\!\!\!\!\!\!\!\!\lambda_j b_j^2\,;
  \\
  C_p&=-\left( 1+\sum\limits_{ j\in \{1,\ldots, m-1\}\backslash \{p\}}  \!\!\!\!\!\!\!\!\!\!
\lambda_jb_j^2\right)\times{}\\
&\hspace*{15mm}{}\times \left( K_{x_mx_p} +\sum\limits_{ j\in \{1,\ldots, m-1\}\backslash 
\{p\}} \!\!\!\!\!\!\!\!\!\!\lambda_j b_j K_{x_j x_p}\right)\,.
  \end{align*}
    Тогда систему~(\ref{e7-baz}) можно представить в~виде:
  \begin{equation}
  H_p\left( b_p-b^*_{p,1}\right) \left( b_p-b^*_{p,2}\right)=0\,,\enskip 
p=\overline{1,m-1}\,,
  \label{e8-baz}
  \end{equation}
где $b_{p,1}^*\hm= (-B_p-\sqrt{\mathrm{Disc}_p})/(2H_p)$ и~$b^*_{p,2}\hm= 
(-B_p\hm+\sqrt{\mathrm{Disc}_p})/(2H_p)$~--- корни $p$-го квадратного трехчлена 
системы~(\ref{e7-baz}); $\mathrm{Disc}_p\hm= B_p^2\hm- 4H_p C_p$~---  
дискриминанты $p$-го квадратного трехчлена системы~(\ref{e7-baz}), которые, 
как видно, всегда положительны.

  Понятно, что система~(\ref{e8-baz}) равносильна совокупности $2^{m-1}$ 
систем
  \begin{multline}
  \left\{
  \begin{array}{l}
  b_1=b^*_{1,1};\\[3pt]
  b_2=b^*_{2,1};\\[3pt]
   \ldots \\[3pt]
  b_{m-1}=b^*_{m-1,1}\,;
  \end{array}\right.
  \enskip 
  \left\{
  \begin{array}{l}
  b_1=b^*_{1,2};\\[3pt]
  b_2=b^*_{2,1};\\[3pt]
   \ldots \\[3pt]
  b_{m-1}=b^*_{m-1,1}\,;
  \end{array}
  \right.\quad \cdots
\\
\cdots \quad \left\{
  \begin{array}{l}
  b_1=b^*_{1,2};\\[3pt]
  b_2=b^*_{2,2};\\[3pt]
   \ldots \\[3pt]
  b_{m-1}=b^*_{m-1,2}\,.
  \end{array}\right.
  \label{e9-baz}
  \end{multline}
  
  Покажем, что решение задачи~(\ref{e3-baz}) удовлетворяет только системе
  \begin{equation}
  \left\{ 
  \begin{array}{l}
  b_1=b^*_{1,2}\,;\\[3pt]
  b_2=b^*_{2,2}\,;\\[3pt]
  \ldots\\[3pt]
  b_{m-1} =b^*_{m-1,2}\,.
  \end{array}
  \right.
  \label{e10-baz}
  \end{equation}
  
  Вторые частные производные функции~(\ref{e3-baz}) имеют вид:
  
  \noindent
  \begin{multline}
  \fr{\partial^2 S}{\partial b_p^2} =2\lambda_p n\left(  1+\sum\limits_{j=1}^{m-1} 
\lambda_j b_j^2\right)^{-2} \left[
\vphantom{\left( 1+\sum\limits_{j=1}^{m-1} \lambda_j b_j^2\right)}
 2H_p b_p +B_p-{}\right.\\
  \left.{}- 2\fr{b_p}{n}\left( 1+\sum\limits_{j=1}^{m-1} \lambda_j b_j^2\right) 
\fr{\partial S}{\partial b_p}\right]\,,\enskip p=\overline{1,m-1}\,.
  \label{e11-baz}
  \end{multline}
  
  Для того чтобы функция~(\ref{e3-baz}) имела минимум в~некоторой точке, 
матрица Гессе, составленная из частных производных второго порядка, должна 
быть положительно определенной. По критерию Сильвестра для положительно 
определенной мат\-ри\-цы Гессе все ее элементы на главной  
диагонали~(\ref{e11-baz}) должны быть положительными. А~из~$2^{m-1}$ 
систем~(\ref{e9-baz}) это условие выполняется только для случая~(\ref{e10-baz}). 
Поэтому для нахождения оценок полносвязной регрессии вместо 
системы~(\ref{e4-baz}) достаточно решить систему~(\ref{e10-baz}). Если 
$m\hm\geq 3$, то для этого можно воспользоваться методом простых итераций. 
При $m\hm=3$ можно также применить метод подстановки.
  
  Как уже отмечалось, до решения системы~(\ref{e4-baz}) необходимо задать 
значения лямб\-да-па\-ра\-мет\-ров. По мнению автора, рациональным будет 
выбор таких значений этих параметров, при которых суммарное 
аппроксимационное качество полносвязной регрессии~(\ref{e1-baz}),  
(\ref{e2-baz}) будет наилучшим. Для этого введем аддитивный коэффициент 
детерминации
  \begin{equation*}
  R^2_{\mathrm{add}}  =\sum\limits^m_{j=1} R^2_{x_j}\,,
  %\label{e12-baz}
  \end{equation*}
где $R^2_{x_j}$~--- коэффициент детерминации для переменной~$x_j$ 
полносвязной регрессии~(\ref{e1-baz}),~(\ref{e2-baz}).

  Сформулируем следующую оптимизационную задачу:
  \begin{equation*}
  \sum\limits^m_{j=1} R^2_{x_j}\to \max\,,
 % \label{e13-baz}
  \end{equation*}
которая, по определению $R^2_{x_j}$, равносильна задаче
\begin{multline}
\fr{\sum\nolimits^n_{i=1}\left(\varepsilon_i^{(x_1)}\right)^2}{D_{x_1}} +
\fr{\sum\nolimits^n_{i=1}\left(\varepsilon_i^{(x_2)}\right)^2}{D_{x_2}}+\cdots{} \\{}\cdots +
\fr{\sum\nolimits^n_{i=1}\left(\varepsilon_i^{(x_m)}\right)^2}{D_{x_m}}\to  \min\,.
\label{e14-baz}
\end{multline}
    А~задача~(\ref{e14-baz}) эквивалентна задаче~(\ref{e3-baz}) при 
$\lambda_1\hm= D_{x_m}/D_{x_1}, \lambda_2= D_{x_m}/D_{x_2}, \ldots, 
\lambda_m\hm=D_{x_m}/D_{x_{m-1}}$. Таким образом, для полученных 
значений ламб\-да-па\-ра\-мет\-ров аппроксимационное качество многофакторной 
полносвязной регрессии~(\ref{e1-baz}), (\ref{e2-baz}) будет наилучшим.

\section{Многофакторная модель полносвязной линейной 
регрессии с~выходной переменной и~алгоритм <<Straight~B>>}

  Дополним набор входных переменных $x_1, x_2, \ldots$\linebreak $\ldots , x_m$ выходной 
переменной~$y$, которая сильно коррелирует с~ними. Свяжем оцененные 
<<истинные>> значения, например, переменной~$\tilde{x}^*_m$ со значениями 
переменной~$y$ моделью парной линейной регрессии:
  \begin{equation}
  y_i= c_0+c_1\tilde{x}_{im}^* +\varepsilon_i\,,\enskip i=\overline{1,n}\,,
  \label{e15-baz}
  \end{equation}
где $c_0$ и~$c_1$~--- неизвестные параметры, которые находятся с~помощью 
обычного МНК.

  Совокупность уравнений (\ref{e1-baz}), (\ref{e2-baz}), (\ref{e15-baz}) называется 
многофакторной моделью полносвязной линейной регрессии с~выходной 
переменной~$y$~\cite{10-baz}. Если параметры $\lambda_1, \lambda_2,\ldots , 
\lambda_{m-1}$ известны, то ее оценки находятся в~два этапа.
  \begin{enumerate}[1.]
\item С помощью МНПК оценивается полносвязная регрессия без выходной 
переменной~(\ref{e1-baz}), (\ref{e2-baz}).
\item С~по\-мощью МНК оценивается модель парной линейной  
регрессии~(\ref{e15-baz}).
\end{enumerate}

  Пусть оцененная модель (\ref{e1-baz}), (\ref{e2-baz}) , (\ref{e15-baz}) имеет вид:
  \begin{align}
  \tilde{y} &= \tilde{c}_0 +\tilde{c}_1 \tilde{x}_m^*\,;\label{e16-baz}\\
  \tilde{x}_j^* &= \tilde{a}_j +\tilde{b}_j \tilde{x}_m^*\,,\enskip j=\overline{1,m-
1}\,; \notag%\label{e17-baz}
\\
  \tilde{x}_m^*&=A_0 +\sum\limits^m_{j=1} A_j x_j\,,
  \label{e18-baz}
  \end{align}
где
\begin{align*}
A_0&=-\left( 1+\sum\limits^{m-1}_{j=1} \lambda_j \tilde{b}^2_j \right)^{-1} 
\sum\limits_{j=1}^{m-1} \lambda_j \tilde{a}_j \tilde{b}_j\,;\\
A_j&=\lambda_j \tilde{b}_j \left( 1+ \sum\limits^{m-1}_{j=1} \lambda_j 
\tilde{b}^2_j\right)^{-1}\,,\enskip j=\overline{1,m-1}\,;\\
A_m&=\left( 1+\sum\limits^{m-1}_{j=1} \lambda_j \tilde{b}_j^2\right)^{-1}\,.
\end{align*}
  Используя~(\ref{e18-baz}), перепишем уравнение~(\ref{e16-baz}) в~виде:
  \begin{equation}
  \tilde{y}=\theta_0 +\sum\limits^m_{j=1} \theta_j x_{ij}\,,
  \label{e19-baz}
  \end{equation}
где $\theta_0 =\tilde{c}_0 \hm+ \tilde{c}_1A_0$; $\theta_j\hm= \tilde{c}_1A_j$, 
$j\hm=\overline{1,m}$.
  
  Уравнение~(\ref{e19-baz}) называется вторичным уравнением многофакторной 
модели полносвязной линейной регрессии~\cite{10-baz}.
  
  Как уже отмечалось, при сильной корреляции входных переменных $x_1, x_2, 
\ldots , x_m$ коэффициенты уравнения~(\ref{e18-baz}) удовлетворяют условиям
  \begin{equation}
  A_j r_{x_j x_m}>0\,,\enskip j=\overline{1,m-1}\,;\enskip A_m>0\,,
  \label{e20-baz}
  \end{equation}
а при сильной корреляции~$y$ с~этими переменными угловой коэффициент 
уравнения~(\ref{e16-baz})~--- условию
\begin{equation}
\tilde{c}_1 r_{yx_m}>0\,.
\label{e21-baz}
\end{equation}
  
  Перемножив неравенства~(\ref{e20-baz}) на~(\ref{e21-baz}), получим 
$\tilde{c}_1 A_j r_{x_jx_m} r_{yx_m}\hm>0$, $ j\hm=\overline{1,m-1}$, 
и~$\tilde{c}_1 A_m r_{yx_m}\hm>0$.\linebreak Отсюда, учитывая, что знаки 
произведений~$r_{x_jx_m}r_{yx_m}$ совпадают со знаками~$r_{yx_j}$,  
$j\hm=\overline{1,m-1}$, следует, что
  \begin{equation*}
  \theta_j r_{yx_j}>0\,,\enskip    j=\overline{1,m}\,,
  %\label{e22-baz}
  \end{equation*}
т.\,е.\ знаки коэффициентов при объясняющих переменных во вторичном 
уравнении~(\ref{e19-baz}) совпадают с~соответствующими знаками 
коэффициентов корреляции~$r_{yx_j}$, $j\hm=\overline{1,m}$.
  
  На основе проведенных автором исследований разработан следующий 
алгоритм <<Straight~B>>, реализующий метод выпрямления искаженных 
коэффициентов (МВИК) для многофакторных регрессионных моделей.
  \begin{enumerate}[1.]
\item При $\lambda_1=D_{x_m}/D_{x_1}, \lambda_2\hm= D_{x_m}/D_{x_2}, 
\ldots, \lambda_{m-1}\hm= D_{x_m}/D_{x_{m-1}}$ из системы~(\ref{e10-baz}) 
численно находятся оценки $\tilde{b}_1, \tilde{b}_2, \ldots , \tilde{b}_{m-1}$, 
затем по формулам~(\ref{e5-baz})~--- коэффициенты $\tilde{a}_1, \tilde{a}_2, 
\ldots , \tilde{a}_{m-1}$ и, наконец, по формулам~(\ref{e6-baz})~--- оцененные 
<<истинные>> значения переменной~$\tilde{x}_m^*$.
\item С помощью МНК оценивается модель~(\ref{e15-baz}).
\item Путем подстановки~(\ref{e18-baz}) в~равенство~(\ref{e16-baz}) определяется 
искомое уравнение регрессии.
\end{enumerate}

\section{Моделирование валового внутреннего продукта России}

  Для демонстрации МВИК решалась задача моделирования ВВП России. Для 
этого были использованы статистические данные с~сайта Федеральной службы 
государственной статистики ({\sf https://rosstat.gov.ru}) за период с~2000 
по~2018~гг.\ по следующим семи переменным: $y$~--- ВВП России (млрд руб.); $x_1$~--- среднемесячная заработная плата 
в~России (руб.); $x_2$~--- численность безработных в~России (тыс.\ чел.); 
$x_3$~--- потребление электроэнергии в~России (млн кВт$\cdot$ч); $x_4$~--- 
продукция сельского хозяйства России (млрд руб.); $x_5$~--- грузооборот 
железнодорожного транспорта России (млрд т$\cdot$км); $x_6$~--- оборот 
розничной торговли (млн руб.).
  
  Матрица парных коэффициентов корреляции для этих переменных 
представлена в~таблице.
  
  \begin{table*}\small
  \begin{center}
  \begin{tabular}{|c|c|c|c|c|c|c|c|}
  \multicolumn{8}{c}{Матрица парных коэффициентов корреляции}\\
  \multicolumn{8}{c}{\ }\\[-6pt]
  \hline
&$y$&$x_1$&$x_2$&$x_3$&$x_4$&$x_5$&$x_6$\\
\hline
$y$&1&0,997&$-$0,843&0,953&0,985&0,942&0,997\\
$x_1$&0,997&1&$-$0,827&0,95\hphantom{9}&0,988&0,94\hphantom{9}&0,998\\
$x_2$&$-$0,843\hphantom{$-$}&$-$0,827\hphantom{$-$}&1&$-$0,888\hphantom{$-$}&$-$0,805\hphantom{$-$}
&$-$0,922\hphantom{$-$}&$-$0,831\hphantom{$-$}\\
$x_3$&0,953&0,95\hphantom{9}&$-$0,888&1&0,92\hphantom{9}&0,981&0,954\\
$x_4$&0,985&0,988&$-$0,805&0,92\hphantom{9}&1&0,917&0,987\\
$x_5$&0,942&0,94\hphantom{9}&$-$0,922&0,981&0,917&1&0,937\\
$x_6$&0,997&0,998&$-$0,831&0,954&0,987&0,937&1\\
\hline
\end{tabular}
\end{center}
\end{table*}
       
  
  Как видно из таблицы, все объясняющие переменные тесным образом 
коррелируют между собой. Следовательно, в~оцененной по исходным данным 
модели множественной линейной регрессии будет присутствовать эффект 
муль\-ти\-кол\-ли\-не\-ар\-ности. С~помощью МНК была построена следующая 
регрессионная модель:
  \begin{multline}
  \tilde{y}= 24236{,}7+1{,}399x_1-2{,}136x_2-0{,}0067x_3-{}\\
  {}- 0{,}225x_4-3{,}073x_5 +0{,}0013x_6\,.
  \label{e23-baz}
  \end{multline}
  
  Коэффициент детерминации модели~(\ref{e23-baz}) $R^2\hm= 0{,}996$. Как 
и~оказалось, из-за мультиколлинеарности знаки коэффициентов при 
переменных~$x_3$, $x_4$ и~$x_5$ не согласуются с~экономическим смыслом 
задачи. Так, по регрессии~(\ref{e23-baz}) можно сделать абсурдный вывод о~том, 
что для повышения ВВП России требуется снижать объемы производства 
продукции сельского хозяйства.
  
  Для МВИК на основе алгоритма <<Selection~B>> на языке программирования 
hansl эконометрического пакета Gretl была написана соответствующая программа. 
С~помощью этой программы было получено вторичное уравнение полносвязной 
регрессии:
  \begin{multline}
  \tilde{y}=-54754{,}02+0{,}409x_1-4{,}765x_2+0{,}0693x_3+{}\\
  {}+ 3{,}475x_4 +15{,}73x_5+0{,}00054x_6\,.
  \label{e24-baz}
  \end{multline}
  
  Коэффициент детерминации модели~(\ref{e24-baz}) $R^2\hm=0{,}971$. Как 
видно, теперь знаки абсолютно всех коэффициентов согласуются 
с~экономическим смыслом задачи. При этом по отношению  
к~модели~(\ref{e23-baz}) качество регрессии~(\ref{e24-baz}) снизилось 
незначительно, поэтому ее можно использовать не только для интерпретации, но и~для прогнозирования.
  
{\small\frenchspacing
{%\baselineskip=10.8pt
%\addcontentsline{toc}{section}{References}
\begin{thebibliography}{99}
\bibitem{1-baz}
\Au{Gunst R.\,F., Webster~J.\,T.} Regression analysis and problems of multicollinearity~// 
Commun. Stat. Theory, 1975. Vol.~4. P.~277--292.
\bibitem{2-baz}
\Au{Tamura R., Kobayashi~K., Takano~Y., Miyashiro~R., Nakata~K., Matsui~T.} Best subset 
selection for eliminating multicollinearity~// J.~Oper. Res. Soc.  Jpn., 2017. Vol.~60. 
No.\,3. P.~321--336.

\bibitem{4-baz} %3
\Au{Ферстер Э., Ренц Б.} Методы корреляционного и~регрессионного анализа~/
Пер. с~нем.~--- М.: Финансы 
и~статистика, 1983. 304~с.
(\Au{F$\ddot{\mbox{o}}$rster E., R$\ddot{\mbox{o}}$nz B.} Methoden der korrelation und regressionsanalyse.~--- 
Berlin: Verlag Die Wirtschaft, 1979. 369~p.)

\bibitem{3-baz} %4
\Au{Chatterjee S., Hadi A.\,S.} Regression analysis by example.~--- 5th ed.~--- Hoboken, NJ, USA: 
Wiley, 2012. 424~p. 

\bibitem{5-baz}
\Au{Jolliffe I.\,T.} Principal component analysis.~--- New York, NY, USA: Springer-Verlag, 2002. 
488~p.
\bibitem{6-baz}
\Au{Hoerl A.\,E., Kennard~R.\,W.} Ridge regression: Biased estimation for nonorthogonal 
problems~// Technometrics, 1970. Vol.~12. P.~55--67.
\bibitem{7-baz}
\Au{Мокшина С.\,И., Шуршикова~Г.\,В., Щекунских~С.\,С.} Метод построения содержательно 
интерпретируемых регрессионных моделей в~условиях мультиколлинеарности~// Современная 
экономика: проблемы и~решения, 2017. №\,5(89). С.~81--94.
\bibitem{8-baz}
\Au{Базилевский М.\,П.} Синтез модели парной линейной регрессии и~простейшей  
EIV-мо\-де\-ли~// Моделирование, оптимизация и~информационные технологии, 2019. Т.~7. 
№\,1(24). С.~170--182.
\bibitem{9-baz}
\Au{Базилевский М.\,П.} Исследование двухфакторной модели полносвязной линейной 
регрессии~// Моделирование, оптимизация и~информационные технологии, 2019. Т.~7. 
№\,2(25). С.~80--96.
\bibitem{10-baz}
\Au{Базилевский М.\,П.} Многофакторные модели полносвязной линейной регрессии без 
ограничений на соотношения дисперсий ошибок переменных~// Информатика и~её применения, 
2020. Т.~14. Вып.~2. С.~92--97.
 \end{thebibliography}

}
}

\end{multicols}

\vspace*{-3pt}

\hfill{\small\textit{Поступила в~редакцию 21.09.2020}}

%\vspace*{8pt}

%\pagebreak

\newpage

\vspace*{-28pt}

%\hrule

%\vspace*{2pt}

%\hrule

%\vspace*{-2pt}

\def\tit{METHOD OF STRAIGHTENING DISTORTED DUE~TO~MULTICOLLINEARITY 
COEFFICIENTS\\ IN~REGRESSION MODELS}


\def\titkol{Method of straightening distorted due~to~multicollinearity 
coefficients in~regression models}

\def\aut{M.\,P.~Bazilevskiy}

\def\autkol{M.\,P.~Bazilevskiy}

\titel{\tit}{\aut}{\autkol}{\titkol}

\vspace*{-11pt}


\noindent
Department of Mathematics, Irkutsk State Transport University, 15~Chernyshevskogo 
Str., Irkutsk 664074, Russian Federation
 
\def\leftfootline{\small{\textbf{\thepage}
\hfill INFORMATIKA I EE PRIMENENIYA~--- INFORMATICS AND
APPLICATIONS\ \ \ 2021\ \ \ volume~15\ \ \ issue\ 2}
}%
\def\rightfootline{\small{INFORMATIKA I EE PRIMENENIYA~---
INFORMATICS AND APPLICATIONS\ \ \ 2021\ \ \ volume~15\ \ \ issue\ 2
\hfill \textbf{\thepage}}}

\vspace*{3pt}



\Abste{When constructing regression models, due to the strong multicollinearity of the 
explanatory variables, its coefficients are distorted, in particular, their signs, which 
negatively affects the interpretational qualities of such regression. This article is devoted 
to the development of a~method of straightening coefficients distorted due to 
multicollinearity. This method is based on the property of the fully connected linear 
regression models proposed by the author. A~nonlinear system, which is used to 
estimate fully connected regressions, is investigated. It is shown that the solution of this 
system can be obtained numerically using the method of simple iterations. A method for 
choosing unknown lambda-parameters in fully connected regression is proposed. It was 
found that in multivariate fully connected models with a~strong correlation of all 
factors, the signs of the coefficients for the variables in the secondary equation coincide 
with the corresponding signs of the correlation coefficients. To straighten the distorted 
coefficients on the basis of this research, the ``Selection~B'' algorithm was developed. 
The developed method of straightening has been successfully demonstrated by the 
example of modeling Russia's gross domestic product (GDP).}

\KWE{regression analysis; fully connected linear regression model; multicollinearity; 
interpretation; numerical method; GDP of Russia}


\DOI{10.14357/19922264210209}

%\vspace*{-15pt}

% \Ack
%\noindent


%\vspace*{12pt}

  \begin{multicols}{2}

\renewcommand{\bibname}{\protect\rmfamily References}
%\renewcommand{\bibname}{\large\protect\rm References}

{\small\frenchspacing
 {%\baselineskip=10.8pt
 \addcontentsline{toc}{section}{References}
 \begin{thebibliography}{99}
\bibitem{1-baz-1}
\Aue{Gunst, R.\,F., and J.\,T.~Webster.} 1975. Regression analysis and problems of 
multicollinearity. \textit{Commun. Stat. Theory}  
4:277--292.
\bibitem{2-baz-1}
\Aue{Tamura, R., K.~Kobayashi, Y.~Takano, R.~Miyashiro, K.~Nakata, and 
T.~Matsui.} 2017. Best subset selection for eliminating multicollinearity. 
\textit{J.~Oper. Res. Soc. Jpn.} 60(3):321--336.

\bibitem{4-baz-1} %3
\Aue{F$\ddot{\mbox{o}}$rster, E., and B.~R$\ddot{\mbox{o}}$nz.} 1983. 
\textit{Methoden der korrelation und regressionsanalyse}. Berlin: Verlag Die Wirtschaft, 1979. 369~p.
\bibitem{3-baz-1} %4
\Aue{Chatterjee, S., and A.\,S.~Hadi.} 2012. \textit{Regression analysis by example}. 
5th  ed. Hoboken, NJ: Wiley. 424~p.
\bibitem{5-baz-1}
\Aue{Jolliffe, I.\,T.} 2002. \textit{Principal component analysis}. New York, NY: 
Springer-Verlag. 488~p.
\bibitem{6-baz-1}
\Aue{Hoerl, A.\,E., and R.\,W.~Kennard.} 1970. Ridge regression: Biased estimation for 
nonorthogonal problems. \textit{Technometrics} 12:55--67.
\bibitem{7-baz-1}
\Aue{Mokshina, S.\,I., G.\,V.~Shurshikova, and S.\,S.~Shchekunskikh.} 2017. Metod 
postroeniya soderzhatel'no interpretiruemykh regressionnykh modeley v~usloviyakh 
mul'tikollinearnosti [The construction method of meaningful interpreted regression 
models in conditions of multicollinearity]. \textit{Sovremennaya ekonomika: problemy 
i~resheniya} [Modern Economics: Problems and Solutions] 89(5):81--94.
\bibitem{8-baz-1}
\Aue{Bazilevskiy, M.\,P.} 2019. Sintez modeli parnoy lineynoy regressii i~prosteyshey 
EIV-modeli [Synthesis of linear regression model and EIV-model]. 
\textit{Modelirovanie, optimizatsiya i~informatsionnye tekhnologii} [Modeling, 
Optimization and Information Technology] 7(1):170--182.
\bibitem{9-baz-1}
\Aue{Bazilevskiy, M.\,P.} 2019. Issledovanie dvukhfaktornoy modeli polnosvyaznoy 
lineynoy regressii [Investigation of a two-factor fully connected linear regression 
model]. \textit{Modelirovanie, optimizatsiya i~informatsionnye tekhnologii} [Modeling, 
Optimization and Information Technology] 7(2):80--96.
\bibitem{10-baz-1}
\Aue{Bazilevskiy, M.\,P.} 2020. Mnogofaktornye modeli polnosvyaznoy lineynoy 
regressii bez ogranicheniy na sootnosheniya dispersiy oshibok peremennykh 
[Multifactor fully connected linear regression models without constraints to the ratios of 
variables errors variances]. \textit{Informatika i~ee Primeneniya~--- Inform. Appl.} 
14(2):92--97.
\end{thebibliography}

 }
 }

\end{multicols}

\vspace*{-3pt}

  \hfill{\small\textit{Received September~21, 2020}}


%\pagebreak

%\vspace*{-8pt}  

\Contrl

\noindent
\textbf{Bazilevskiy Mikhail P.} (b.\ 1987)~--- Candidate of Science (PhD) in 
technology, associate professor, Irkutsk State Transport University, 
15~Chernyshevkogo Str., Irkutsk 664074, Russian Federation; 
\mbox{mik2178@yandex.ru}

\label{end\stat}

\renewcommand{\bibname}{\protect\rm Литература} %9
\def\stat{kirikov}

\def\tit{<<ВИРТУАЛЬНЫЙ КОНСИЛИУМ>>~--- ИНСТРУМЕНТАЛЬНАЯ 
СРЕДА ПОДДЕРЖКИ ПРИНЯТИЯ 
  СЛОЖНЫХ ДИАГНОСТИЧЕСКИХ РЕШЕНИЙ$^*$}

\def\titkol{<<Виртуальный консилиум>>~--- инструментальная 
среда поддержки принятия сложных диагностических решений}

\def\aut{И.\,А.~Кириков$^1$, А.\,В.~Колесников$^2$, С.\,В.~Листопад$^3$, 
С.\,Б.~Румовская$^4$}

\def\autkol{И.\,А.~Кириков, А.\,В.~Колесников, С.\,В.~Листопад, 
С.\,Б.~Румовская}

\titel{\tit}{\aut}{\autkol}{\titkol}

\index{Кириков И.\,А.}
\index{Колесников А.\,В.}
\index{Листопад С.\,В.} 
\index{Румовская С.\,Б.}
\index{Kirikov I.\,А.}
\index{Kolesnikov А.\,V.}
\index{Listopad S.\,V.}
\index{Rumovskaya S.\,B.}


{\renewcommand{\thefootnote}{\fnsymbol{footnote}} \footnotetext[1]
{Работа выполнена при частичной поддержке РФФИ (проект 16-07-00272 А).}}


\renewcommand{\thefootnote}{\arabic{footnote}}
\footnotetext[1]{Калининградский филиал Федерального исследовательского центра <<Информатика и~управление>> 
Российской академии наук, \mbox{baltbipiran@mail.ru}}
\footnotetext[2]{Балтийский Федеральный университет
имени  И.~Канта, Калининградский филиал Федерального 
исследовательского центра <<Информатика и~управление>> Российской академии наук, 
\mbox{avkolesnikov@yandex.ru}}
\footnotetext[3]{Калининградский филиал Федерального исследовательского центра <<Информатика и~управление>> 
Российской академии наук, \mbox{ser-list-post@yandex.ru}}
\footnotetext[4]{Калининградский филиал Федерального исследовательского центра <<Информатика 
и~управление>> Российской академии наук, \mbox{sophiyabr@gmail.com}}
 
 \vspace*{-3pt}
 
  \Abst{Рассматривается проблема принятия индивидуального решения при диагностике 
пациентов в~ам\-бу\-ла\-тор\-но-по\-ли\-кли\-ни\-че\-ских учреждениях на примере 
диагностики артериальной гипертензии (АГ). Предлагается повысить качество принятия 
индивидуального решения за счет консультаций с~системой поддержки принятия  
решения~--- <<Виртуальным консилиумом>>, моделирующим коллективный интеллект 
врачей стационара многопрофильного больничного учреждения. Приведены результаты 
проектирования и~экспериментального исследования лабораторного прототипа 
<<Виртуального консилиума>>.}

  \KW{система поддержки принятия решения; виртуальный консилиум; функциональная 
гибридная интеллектуальная система}

\DOI{10.14357/19922264160311} 


\vskip 10pt plus 9pt minus 6pt

\thispagestyle{headings}

\begin{multicols}{2}

\label{st\stat}
  

\section{Введение}

  Степень исследования, понимания и~качества диагностики проблемных сред и~их 
окружения отражена в~научной картине мира, онтологи\-зи\-ру\-ющей его представления 
и~делающей рассуждения и~целенаправленную деятельность <<зависимыми>> от них. 
В~искусственном интеллекте понятию <<картина мира>> соответствует понятие <<модель 
внешнего мира>> М.\,Г.~Га\-азе-Рап\-по\-пор\-та и~Д.\,А.~Поспелова~[1]. 
  
  Новая картина мира складывается из многочисленных теорий и~взглядов: <<ноосфера>>, 
<<разумный мир>> (В.\,И.~Вернадский, Н.\,Н.~Моисеев, А.\,В.~Поздняков); <<мир 
диалектики>>~--- мир диалога разных логик (Е.\,Л.~Доценко); социальная парадигма 
искусственного интеллекта (<<The society of mind>>) М.~Минского;  
сис\-тем\-но-ор\-га\-ни\-за\-ци\-он\-ный подход в~искусственном интеллекте 
В.\,Б.~Тарасова; теория иерархических многоуровневых систем М.~Месаровича, Д.~Мако 
и~И.~Такахары и~др.~--- и~укладывается в~семь постулатов~[2]: (1)~признание 
гетерогенности мира и~любого объекта, разнообразия жизни; (2)~неопределенность границ 
объектов и~связь <<всего со всем>>; (3)~относительность любой иерархии и~горизонтальные 
связи; (4)~дополнительность и~сотрудничество; (5)~полицентризм; (6)~относительность 
знания; (7)~соответствие управления сложности объекта. 
  
  Сложная задача диагностики АГ (СЗДАГ)~---
  за\-да\-ча-сис\-те\-ма, вклю\-ча\-ющая диагностические и~технологические подзадачи, 
повышающие эффективность обработки симптоматической информации о пациенте. 
Разнообразие подзадач СЗДАГ с~различными характеристическими свойствами требует 
разнообразия соответствующих методов принятия решений, системного анализа, 
искусственного интеллекта и~инженерии знаний. 
  
  Анализ результатов влияния новой картины мира на ментальную составляющую 
врачебной практики и~медицинской информатики~[3] показал, что, несмотря на стремление 
биомедицины к~гетерогенности восприятия организма человека и~процесса его диагностики 
в~рамках семипостулатной картины мира, человек по-преж\-не\-му остается 
<<расчлененным>> объектом познания, что сформировало <<узких>> специалистов, 
поглощенных решением частных задач. Новый тип ученого <<праг\-ма\-ти\-ка-фак\-то\-ло\-га>> 
утратил системное мышление, перестал задумываться над тем, что делается <<вокруг>> 
и~какое значение могут иметь добытые им факты для понимания работы организма в~целом. 
В~этой связи\linebreak\vspace*{-12pt}

\pagebreak

\noindent
 очевидна необходимость перехода от методов <<конкурентной>> диагностики 
к системному мышлению и~методам гетерогенной диагностики.
  
  В~[3--5] представлены результаты системного анализа СЗДАГ, следуя 
  проблемно-структурной (ПС) методологии, этапы~1--5~[6]: идентификация, редукция сложной задачи, 
спецификация диагностических подзадач, выбор методов их решения, а~также проверка 
неоднородности сложной задачи диагностики. Работы~[3--5] подтвердили релевантность 
применения междисциплинарных инструментариев для решения 
СЗДАГ, мо\-де\-ли\-ру\-ющих разнообразие информации, 
сотрудничество, дополнительность и~относительность знаний, сочетающих методы 
и~методики системного анализа диагностической проблемы с~динамическим синтезом 
метода ее решения и~имитацией работы искусственного гетерогенного коллектива~--- 
<<виртуального консилиума>>.
  
  Разнообразие~--- признак, проявление гетерогенности. Следствие закона необходимого 
разнообразия У.\,Р.~Эшби констатирует, что управ\-ле\-ние обеспечивается, если разнообразие 
средств управ\-ля\-юще\-го не меньше разнообразия управ\-ля\-емой им ситуации. Для отображения 
в информатике ситуативного разнообразия в~естественных гетерогенных системах в~[6] 
введены модели <<гетерогенная, неоднородная задача>> и~<<гомогенная, однородная 
задача>>, а~сам закон трактуется так: только разнообразная, скоординированная клиническая 
деятельность, элементы которой в~комбинации решают одну задачу, сделает результат 
диагностики качественно лучше в~обществе с~новой научной картиной мира. Специфике 
такой работы соответствует коллективный труд экспертов в~малых группах за круглым 
столом~--- консилиумы, совещания, естественные гетерогенные системы для решения 
сложных задач~\cite{3-kir}, где на первый план выходят знания и~опыт лица, принимающего 
решения (ЛПР), и~экспертов.
  
  \begin{figure*} %fig1
\vspace*{1pt}
 \begin{center}  
\mbox{%
 \epsfxsize=147.497mm
 \epsfbox{kir-1.eps}
 }
\end{center} 
%\vspace*{-9pt}
%\Caption{Концептуальная модель процесса диагностики артериальной гипертензии: в~многопрофильном 
%стационарном больничном учреждении~(\textit{а}); в~амбулаторно-поликлиническом~(\textit{б})}
  \end{figure*}

  \addtocounter{figure}{1}
  
  Настоящая работа~--- продолжение работ~[3--5,\linebreak 7] и~имеет целью представить: (1)~результаты 
исследования процесса диагностики АГ  
в~ле\-чеб\-но-про\-фи\-лак\-ти\-че\-ских больничных учреждениях (ЛПУ) широкого 
профиля~--- предлагается повысить эффективность и~качество индивидуальных 
диагностических решений в~ЛПУ широкого профиля ам\-бу\-ла\-тор\-но-по\-ли\-кли\-ни\-че\-ско\-го 
характера (рис.~1,\,\textit{а}) за счет внедрения информационной технологии 
<<Виртуальный консилиум>>, моделирующей коллективное обсуждение; 
(2)~архитектуру <<Виртуального консилиума>> и~результаты лабораторных экспериментов с~
его интегрированными моделями (первые результаты лабораторных экспериментов 
приведены в~[7]).

\section{Диагностика артериальной гипертензии в~многопрофильном 
стационарном больничном учреждении и~в~амбулаторно-поликлиническом 
учреждении}

\vspace*{-9pt}


  В~[8, 9] представлены результаты исследования процесса диагностики 
АГ в~Калининградской клинической областной больнице (КОКБ) 
(см.\ рис.~1,\,\textit{б}) и~ее Диагностическом центре (см.\ рис.~1,\,\textit{а}). 

Для формирования 
полного дифференциального диагноза АГ коллективом врачей во главе с~лечащим врачом, 
ЛПР-кар\-дио\-ло\-гом, в~стационаре привлекаются до тринадцати вра\-чей-экс\-пер\-тов~--- носителей 
знаний из различных разделов медицины: невролог, нефролог, сосудистый хирург, уролог, 
психолог, педиатр, аку\-шер-ги\-не\-ко\-лог, онколог, окулист, врачи функциональной 
диагностики, эндокринолог, терапевт, кардиолог. 

Для исследований выбраны шесть 
специалистов (см.\ рис.~1,\,\textit{б}), решающих двенадцать функциональных подзадач 
(рис.~\ref{f2-kir}), возникающих в~90\%~случаев диагностики АГ, 
каждый из которых формирует промежуточные заключения о~состоянии объекта 
диагностики в~своей области медицинских зна\-ний. 
{\looseness=1

}

Полученные исходные данные об объекте 
диагностики разнородны (содержатся в~медицинской карте): количественные,  
ви\-зу\-аль\-но-графиче\-ские параметры (детерминированные переменные),\linebreak 
лингвистические четкие и~нечеткие переменные. Лицо, при\-ни\-ма\-ющее решение, изучает в~медицинской карте 
симптомы и~частные диагностические мнения вра\-чей-экс\-пер\-тов, множество которых 
подбирает сам, и~ставит заключительный диагноз. Вра\-чам-экс\-пер\-там доступны симптомы 
и~мнения других врачей-экспертов из медицинской карты.
\mbox{Лицо}, при\-ни\-ма\-ющее решение, и~вра\-чи-экс\-пер\-ты 
обследуют пациента и~формируют диагностические заключения согласно нормативным 
документам, например~[10]. В~ЛПУ широкого профиля (см.\ рис.~1,\,\textit{а}) ЛПР~--- это врач 
общей практики или терапевт (иногда кардиолог, но зачастую без опыта работы, к~которому 
направляет терапевт сразу же при выявлении повышенного артериального давления), это 
врач <<праг\-ма\-тик-фак\-то\-лог>>~\cite{9-kir}, объединяющий в~себе роли вра\-ча-ЛПР  
и~вра\-чей-экс\-пер\-тов узкой специализации.

\end{multicols}

\begin{figure} %fig2
\vspace*{1pt}
 \begin{center}  
\mbox{%
 \epsfxsize=163.044mm
 \epsfbox{kir-2.eps}
 }
\end{center} 
\vspace*{-9pt}
\Caption{Архитектура ВКДАГ }
\label{f2-kir}
\vspace*{3pt}
\end{figure}

\begin{multicols}{2}
  

  Исследования диагностического процесса на материалах Диагностического центра КОКБ 
по модели на рис.~1,\,\textit{а} показали, что~70\%~пациентов с~АГ 
амбулаторно-поликлинического учреждения не знают о своем заболевании, в~то время как в~стационарных 
медицинских учреждениях (см.\ рис.~1,\,\textit{б}) практически в~100\%~случаев имеет место 
как адекватное проведение, так и~отображение в~медицинских картах симптоматических 
данных обследования с~подтверждением диагноза  
ла\-бо\-ра\-тор\-но-ин\-ст\-ру\-мен\-таль\-ны\-ми методами исследования. 
  
  В этой связи предлагается повысить эффективность и~качество индивидуальных 
диагностических решений в~ЛПУ широкого профиля амбула\-тор\-но-по\-ли\-кли\-ни\-че\-ско\-го 
характера (см.\ рис.~1,\,\textit{а}) за счет внед\-ре\-ния информационной технологии 
<<Виртуальный консилиум>> (см.\ рис.~\ref{f2-kir}), моделирующей коллективное обсуждение, 
обладающего синергией, опытом и~знаниями в~решении подзадач диагностики 
АГ в~стационаре (см.\ рис.~1,\,\textit{б}). 


  

  
\section{Инструментальная среда <<Виртуальный консилиум для~диагностики 
артериальной гипертензии>>}

\vspace*{-18pt}

  Инструментальная среда <<Виртуальный консилиум>>, архитектура которой 
представлена на рис.~\ref{f2-kir}, а~структура в~\cite{7-kir}, ограничена пациентами 
стар\-ше~18~лет, без особых состояний, нет распознавания снимков, не предусматривается 
назначение лечения и~не диагностируется ряд симптоматических артериальных гипертензий. 

Архитектура <<Виртуального консилиума для диагностики артериальной гипертензии>> 
(ВКДАГ) включает межмодульные интерфейсы~$\zeta^u$ для модулей, реализованных 
посредством различных методологий гибридных интеллектуальных сис\-тем (\mbox{ГиИС}) 
(генетические алгоритмы ($g$), нечеткие 
сис-\linebreak\vspace*{-12pt}

\pagebreak

\end{multicols}

\begin{table*}\small
%\vspace*{-12pt}
\begin{center}
\Caption{Описание блоков архитектуры ВКДАГ}
\vspace*{2ex}

\begin{tabular}{|p{30mm}|p{40mm}|p{39mm}|p{39mm}|}
\hline
\multicolumn{1}{|c|}{Наименование блока}&\multicolumn{1}{c|}{Функции}&\multicolumn{1}{c|}{Вход}&\multicolumn{1}{c|} 
{Выход}\\
\hline
Технологический модуль $i$-й&
Организация эффективной обработки данных и~знаний, выбирается для 
включения в~функциональную \mbox{ГиИС}~--- построение информативного набора 
признаков для диагностики&Популяция 
индивидуумов, накладывающихся как маска на $i$-й функциональный модуль&
Наилучшая особь с~оптимальным набором признаков~--- накладывается как 
маска на $i$-й функциональный модуль\\
\hline
Функциональный модуль $i$-й&Классификация состояния здоровья пациента в~рамках 
\mbox{$i$-й} диагностической 
подзадачи, выбирается для включения в~функциональную \mbox{ГиИС} &
Подмножество $i$-е симптомов с~интерфейса 
пользователя&Частное $i$-е заключение о~со\-сто\-янии здоровья пациента\\
\hline
Функциональный модуль {HCCCC}, моделирующий ЛПР&
Формирование заключительного диагноза 
АГ (всегда в~составе <<Виртуального консилиума>>)&Подмножество симптомов 
с~интерфейса пользователя, множество выходов функциональных модулей&
Заключительный диагноз АГ \\
\hline
Функциональный модуль {ИНСРЭКГ}&Классификация патологического состояния пациента по его 
электрокардиограмме&\multicolumn{2}{p{60mm}|}{Рассмотрены подробно в~\cite{4-kir}}\\
\cline{1-2}
Функциональный модуль {ИНССМАД}&Прогноз нормальных зна\-чений суточного мониторирования 
артериального давле\-ния и~вычисление отклонения &\multicolumn{2}{c|}{\ }\\
\hline
Интерфейс модификации структуры {ВКДАГ}&Исключение из диагностики модулей, решающих не 
интересующие пользователя подзадачи &
Выбранные пользователем подзадачи диагностики &
Функциональная ГиИС, 
синтезированная посредством алгоритма из~\cite{4-kir}\\
\hline
Интерфейс пользователя <<Диагноз>>&Визуализация результатов диагностики и~корректировка их 
пользователем &Заключительный диагноз от функционального модуля НСССС&Отчет, содержащий 
множество симптомов и~диагноз\\
\hline
Интерфейс пользователя &Ввод информации о~со\-сто\-янии здоровья пациента &
Множество значений 
показателей состояния здоровья пациента&
Показатели состояния здоровья пациента, распределенные по 
функциональным модулям \\
\hline
Модификация интерфейса пользователя&Деактивация элементов на интерфейсе пользователя для ввода 
значений показателей состояния здоровья&Множество выходов технологических модулей&Частично 
деактивированный интерфейс пользователя \\
\hline
\end{tabular}
\end{center}
\end{table*}

\begin{multicols}{2}

\noindent 
те\-мы ($f$), искусственные нейронные сети ($n$)).
 В~библиотеке модулей диагностики 
и~препро\-цессии хранятся заранее инициализированные\linebreak в~программной среде 
функциональные и~технологические модели. 
По умолчанию все модули включены 
в~структуру <<Виртуального консилиума>>, их описание пред\-став\-ле\-но в~табл.~1. %\\[-15pt]
%
      <<Виртуальный консилиум>> (см.\ рис.~\ref{f2-kir}) запускает интерфейс пользователя, 
ЛПР-вра\-ча~--- <<{Интерфейс модификации структуры ВКДАГ}>>, посредством 
которого включаются функциональные 
 и~технологические модули в~работу сис\-те\-мы: модуль 
<<Анализ СМАД>>, модуль <<Распознавание ЭКГ>>, модули технологических подзадач из 
группы <<Построение информативного набора признаков\linebreak (симптомов) при диагностике 
заболеваний>> и~модули подзадач из группы <<Диагностика критериев оценки 
сер\-деч\-но-со\-су\-ди\-сто\-го риска и~вторичной АГ у~пациента>> ({ДАГ}$_1$, \ldots , {ДАГ}$_9$): 
диагностики\linebreak поражений ор\-га\-нов-ми\-ше\-ней, факторов риска, цереброваскулярных 
болезней, метаболического синд\-ро\-ма и~сахарного диабета, заболеваний периферических 
артерий, ишемической болезни сердца,\linebreak эндокринной АГ, паренхиматозной нефропатии 
и~реноваскулярной АГ соответственно. Все выбранные $i$-е технологические модули 
запускаются, решают соответствующую подзадачу и~передают информацию на блок 
<<{Модификация интерфейса пользователя}>>. Он деактивирует показатели 
со\-сто\-яния здоровья на <<{Интерфейсе пользователя для\linebreak ввода значений показателей 
состояния здоровья пациента}>> и~корректирует работу $i$-го функционального модуля 
подзадач {ДАГ}$_1$, \ldots\linebreak \ldots , {ДАГ}$_9$. Далее активируется откорректированный 
интерфейс, вводятся симптомы, которые передаются функциональным нечетким модулям, 
решающим подзадачи {ДАГ}$_1$, \ldots , {ДАГ}$_9$\linebreak (моделируют принятие 
решения экспертами, врачами смежных специальностей~--- кардиологом как экспертом, 
неврологом, нефрологом, терапевтом, эндокринологом, урологом). Последние в~свою 
очередь передают информацию о~патологиях, выявленных ими у~пациента, 
функциональному модулю {НСССС} (моделирует принятие решения ЛПР~---  
вра\-чом-кар\-дио\-ло\-гом), решающему подзадачу <<Оценка степени и~стадии 
артериальной гипертензии, степени риска сер\-дечно-сосу\-ди\-стых заболеваний>>. 

В~библиотеке ВКДАГ есть еще два функциональных модуля (см.\ табл.~1), вклю\-ча\-ющих\-ся 
в~работу консилиума посредством <<{Интерфейса модификации структуры 
ВКДАГ}>>: 
      \begin{enumerate}[(1)]
      \item {ИНСРЭКГ}, передающий информацию на модули диагностики поражений 
ор\-га\-нов-ми\-ше\-ней (на рис.~\ref{f2-kir}~--- это {НСДАГ}$_1$), цереброваскулярных 
болезней ({НСДАГ}$_3$) и~ишемической болезни сердца ({НСДАГ}$_6$); 
      \item {ИНССМАД}, формирующий информацию о~нормальных значениях 
суточного артериального давления на функциональный модуль {НСССС}.
      \end{enumerate}
      
\section{Экспериментальное лабораторное исследование программной 
реализации прототипа инструментальной среды <<Виртуальный консилиум>>}
  
  Экспериментальное лабораторное исследование программной реализации 
исследовательского прототипа функциональной гибридной интеллектуальной системы 
ВКДАГ для поддержки принятия сложных диагностических решений необходимо для 
подтверждения его релевантности~[3--5, 7] реальной ситуации диагностики АГ. В~[4] 
пред\-став\-ле\-на информация по особенностям функциональных и~технологических моделей 
гетерогенного модельного поля ВКДАГ, а~в~[7]~--- информация по их инициализации 
в~среде MATLAB-Simulink, результаты исследований качества работы каждой модели 
гетерогенного модельного поля <<Виртуального консилиума>> автономно, а~также 
подтверждена их релевантность работе экспертов~--- врачей узкой специализации, что 
предотвращает распространение ошибок работы автономных моделей на работу 
интегрированной модели. 

В~настоящей работе приведены результаты исследования качества 
интегрированных моделей, синтезированных <<Виртуальным консилиумом>>\linebreak 
и~моделирующих дополнительность и~сотрудничество, которые имитируют коллективные 
рас\-суж\-де\-ния специалистов при постановке диагноза. 

В~табл.~2 представлены критерии 
и~результаты тес\-ти\-ро\-ва\-ния интегрированных моделей <<Виртуального консилиума>> 
с~различными комбинациями знаний врачей, классифицирующих патологическое состояние 
пациента. Порядок работы моделей гетерогенного модельного поля \mbox{ВКДАГ}: запускаются 
модели первой очереди~--- модели технологических элементов {ГАППС}$_{1\mbox{--}9}$, 
корректирующие множества входных переменных моделей {НСДАГ}$_{1\mbox{--}9}$ 
и~{НСССС}; обработка информации передается функциональным элементам: модели 
второй очереди <<отправляют>> информацию на модели третьей, пятой, шес\-той и~седьмой 
очередей~--- \mbox{ИНСРЭКГ} (модель, решающая задачу распознавания электрокардиограммы (ЭКГ)), 
{ИНССМАД} (формирует оптимальные множества показателей суточного давления), 
{НСДАГ}$_9$, {НСДАГ}$_2$ и~{НСДАГ}$_6$; третья\linebreak очередь содержит 
модели НСДАГ$_4$ и~НСДАГ$_5$, передающие выходную информацию на вход моделей четвертой 
и~седьмой очередей; четвертая очередь содержит модель {НСДАГ}$_8$, пе\-ре\-да\-ющую 
информацию  модели пятой очереди {НСДАГ}$_1$, которая в~свою очередь передает 
информацию\linebreak {НСДАГ}$_3$ (шес\-тая очередь); от {НСДАГ}$_3$ передается 
информация {НСДАГ}$_7$ (седьмая очередь); последней запускается модель 
{НСССС}, формирующая заключительный диагноз, на вход которой передается 
выходная информация функциональных моделей вто\-рой--седь\-мой очередей.
  
  Таким образом: (1)~без знаний кардиолога, или нефролога, или эндокринолога 
сред\-не\-квад\-ратическая ошибка наибольшая~--- 0,697; 0,448 и~0,211 соответственно, 
и~объясняется это тем, что кардиолог играет ключевую роль в~обработке ин\-формации, 
поступающей от других врачей\linebreak\vspace*{-12pt}


\pagebreak

\end{multicols}

\begin{table}\small
\begin{center}
\Caption{Параметры и~результаты тестирования интегрированных моделей }
\vspace*{2ex}

\begin{tabular}{|p{66mm}|p{88mm}|}
\hline
\multicolumn{1}{|c|}{\tabcolsep=0pt\begin{tabular}{c}Наименование параметров\\ 
и результатов тестирования\end{tabular}}&
\multicolumn{1}{c|}{Значения параметров и~результатов 
тестирования}\\
\hline
Объем тестовой выборки ВКДАГ, интегрирующего знания всех шести врачей&800 наблюдений~--- 500 с~
диагнозами эссенциальной АГ и~300 с~диагнозами вторичной АГ\\
\hline
Объем тестовой выборки ВКДАГ, интегрирующего знания менее шести врачей&400 наблюдений~--- 200 с~
диагнозами эссенциальной АГ и~200 с~диагнозами вторичной АГ\\
\hline
Источник формирования тестовой вы\-борки&Архив медицинских карт пациентов 1-го кардиологического 
отделения КОКБ\\
\hline
Элемент тестирующей последова\-тель\-ности&
Содержит множество нечетких лингвистических переменных и~вектор образа электрокардиограммы (может отсутствовать)\\
\hline
Эталонный диагноз&Результаты деятельности лечащего вра\-ча-кар\-дио\-ло\-га, подводящего общий итог~--- 
дифференциальный диагноз АГ\\
\hline
Критерии тестирования&Среднеквадратическая ошибка $f$ классификации состояния здоровья пациента~[7]\\
\hline
$f$(шесть врачей)&0,0837\\
\hline
$f$(без кардиолога)&0,697\\
\hline
$f$(без нефролога)&0,448 (в остальных 55,2\% случаях диагноз не вызовет доверия)\\
\hline
$f$(без терапевта)&0,151\\
\hline
$f$(без невролога)&0,149\\
\hline
$f$(без эндокринолога)&0,211 (в остальных 78,9\% случаях диагноз не вызовет доверия)\\
\hline
$f$(без сосудистого хирурга)&0,0798\\
\hline
$f$(без знаний терапевта, невролога, неф\-ро\-ло\-га, эндокринолога, сосудистого хирурга)&0,711\\
\hline
$f$(без знаний терапевта, невролога, эндокринолога, сосудистого хирурга)&0,485\\
\hline
$f$(без знаний невролога, эндокринолога, сосудистого хирурга)&0,334\\
\hline
$f$(без знаний невролога, сосудистого хи\-рурга)&0,167\\
\hline
\end{tabular}
\end{center}
\end{table}

\begin{multicols}{2}


\noindent
 и~от ла\-бораторных исследований, и~в~постановке
заключительного диагноза, а~нефролог и~эндокринолог~--- в~исключении вторичной 
АГ; (2)~знания врача~--- сосудистого хирурга не влияют на 
результаты работы <<Виртуального консилиума>>, и~объясняется это тем, что знания 
сосудистого хирурга, касающиеся диагностики АГ, составляют только~20\% базы знаний 
нечеткой системы, распознающей заболевания периферических артерий (ассоциативные 
клинические состояния), встречающихся не более чем у~10\% населения~\cite{11-kir}, 
и~в~тес\-то\-вую выборку не попала ни одна карта с~данными заболеваниями; (3)~чем больше 
численный состав <<Виртуального консилиума>>, тем с~меньшей среднеквадратической 
ошибкой он классифицирует состояние здоровья пациента; (4)~<<Виртуальный консилиум>> 
в~со\-ста\-ве шести врачей диагностирует АГ со среднеквадратической 
ошибкой постановки диагноза $f = 0{,}0837$, т.\,е.\ дает диагноз, верный в~84\% слу\-чаях. 
{\looseness=1

}
  
  Поскольку <<Виртуальный консилиум>> разра\-ботан на основе всероссийских~\cite{9-kir} 
и~между\-народных рекомендаций по диагностике АГ и~со\-пут\-ст\-ву\-ющих заболеваний, 
которых должен придерживать\-ся каж\-дый врач в~своей практике, при переносе \mbox{ВКДАГ} 
в~другое больничное учреж\-де\-ние необходимо пред\-оста\-вить врачам данного учреждения 
протоколы подтверждения диагностических правил всех баз знаний экспериментальными 
данными из архива КОКБ для ознакомления 
и~внесения при необходимости коррективов в~связи с~возможными особенностями их 
контингента пациентов, а~также возможных требований по устранению ограничений 
системы со стороны персонала нового больничного учреждения. Значительной 
корректировки баз знаний не потребуется.
  
  Таким образом, лабораторные эксперименты с~прототипом <<Виртуального 
консилиума>> дали обнадеживающие результаты. 

Верное решение получено в~84\% 
случаев. В~ам\-бу\-ла\-тор\-но-кли\-ни\-че\-ских учреждениях диагноз не 
выявляется у~70\% пациентов в~основном по причине инертности врачей, недостатка опыта 
врачей узкой специализации и~нехватки кадров в~ЛПУ
широкого профиля, что по результатам экспериментов может быть устранено с~по\-мощью 
применения \mbox{ВКДАГ} во время приема пациентов с~подозрением на АГ.

\section{Заключение}

  Лабораторно подтверждена эффективность предлагаемого подхода для проектирования 
диагностических систем как гетерогенных искусственных диагностических систем со 
свойствами дополнительности, сотрудничества и~относительности\linebreak
 знаний, синтезирующих 
интегрированные методы и~модели, разнообразие которых устраняет разнообразие 
диагностической информации об организме человека~--- <<Виртуальных консилиумов>>,\linebreak 
моделиру\-ющих работу коллектива врачей в~многопрофильном стационарном больничном 
учреждении (на примере КОКБ) и~внедрение 
которых повыша\-ет эффективность и~качество индивидуальных диагностических решений 
в~ам\-бу\-ла\-тор\-но-по\-ли\-кли\-ни\-че\-ском учреждении широкого профиля (на примере 
Диагностического центра КОКБ), где заключение о состоянии больного из-за проблемы 
с~кадрами узкой специализации принимает чаще всего один специалист~--- терапевт или 
врач общей практики, иногда кардиолог, но без опыта работы.

{\small\frenchspacing
 {%\baselineskip=10.8pt
 \addcontentsline{toc}{section}{References}
 \begin{thebibliography}{99}
\bibitem{1-kir}
\Au{Гаазе-Раппопорт М.\,Г., Поспелов~Д.\,А.} От амебы до робота: модели поведения.~--- 
М.: Наука, 1987. 288~с.
\bibitem{2-kir}
\Au{Колесников А.\,В., Кириков~И.\,А., Листопад~С.\,В. %Румовская~С.\,Б. 
и~др.} Решение 
сложных задач коммивояжера методами функциональных гибридных интеллектуальных 
сис\-тем.~--- М.: ИПИ РАН, 2011. 295~с.
\bibitem{3-kir}
\Au{Кириков И.\,А., Колесников~А.\,В., Румовская~С.\,Б.} Исследование сложной задачи 
диагностики артериальной гипертензии в~методологии искусственных гетерогенных  
сис\-тем~// Системы и~средства информатики, 2013. Т.~23. №\,2. С.~81--99. doi: 
10.14357/08696527130208.
\bibitem{4-kir}
\Au{Кириков И.\,А., Колесников~А.\,В., Румовская~С.\,Б.} Функциональная гибридная 
интеллектуальная система для поддержки принятия решений при диагностике артериальной 
гипертензии~// Системы и~средства информатики, 2014. Т.~24. №\,1. С.~153--179. doi: 
10.14357/08696527140110.
\bibitem{5-kir}
\Au{Колесников А.\,В., Румовская~С.\,Б., Листопад~С.\,В., Кириков~И.\,А.} Системный 
анализ в~решении сложных диагностических задач~// Системный анализ и~информационные 
технологии (САИТ-2015): Тр. VI~Междунар. конф.~--- М.: 
ИСА РАН, 2015. Т.~1. С.~157--167.
\bibitem{6-kir}
\Au{Колесников А.\,В., Кириков~И.\,А.} Методология и~технология решения сложных задач 
методами функциональных гибридных интеллектуальных систем.~--- М.: ИПИ РАН, 2007. 
387~с.
\bibitem{7-kir}
\Au{Кириков И.\,А., Колесников~А.\,В., Румовская~С.\,Б.} Исследование лабораторного 
прототипа искусственной гетерогенной системы для диагностики артериальной 
гипертензии~// Системы и~средства информатики, 2014. Т.~24. №\,3. С.~131--143. doi: 
10.14357/08696527140309.
\bibitem{8-kir}
\Au{Румовская С.\,Б.} Методы и~средства информатики для диагностики 
артериальной гипертензии в~ле\-чеб\-но-про\-фи\-лак\-ти\-че\-ских учреждениях 
широкого профиля~// Задачи современной информатики (ЗСИ-2015): Тр. 2-й 
молодежной научной конф.~--- М.: ФИЦ ИУ РАН, 2015. 
С.~168--174.
\bibitem{9-kir}
\Au{Кириков~И.\,А., Румовская~С.\,Б.} Гетерогенная диагностика артериальной 
гипертензии~// Информатика, управление и~системный анализ (ИУСА-2016): Тр. 
4-й Всеросс. научной конф. молодых ученых с~международным участием.~--- 
Тверь: ТвГТУ, 2016. Т.~1. С.~180--188.
\bibitem{10-kir}
Комитет экспертов ВНОК. Диагностика и~лечение артериальной гипертензии. 
Российские рекомендации~// Системные гипертензии, 2010. Вып.~3. С.~5--26.
\bibitem{11-kir}
\Au{Галимзянов Ф.\,В.} Заболевания периферических артерий (клиника, 
диагностика, лечение)~// Международный журнал экспериментального образования, 
2014. Вып.~8. С.~113--114. 

\end{thebibliography}

 }
 }

\end{multicols}

\vspace*{-6pt}

\hfill{\small\textit{Поступила в~редакцию 18.06.16}}

\vspace*{8pt}

%\newpage

%\vspace*{-24pt}

\hrule

\vspace*{2pt}

\hrule

%\vspace*{8pt}



\def\tit{``VIRTUAL COUNCIL''~--- SOURCE ENVIRONMENT SUPPORTING 
COMPLEX DIAGNOSTIC DECISION MAKING}

\def\titkol{``Virtual council''~--- source environment supporting 
complex diagnostic decision making}

\def\aut{I.\,А.~Kirikov$^1$, А.\,V.~Kolesnikov$^{1,2}$, S.\,V.~Listopad$^1$, and 
S.\,B.~Rumovskaya$^1$}

\def\autkol{I.\,А.~Kirikov, А.\,V.~Kolesnikov, S.\,V.~Listopad, and 
S.\,B.~Rumovskaya}

\titel{\tit}{\aut}{\autkol}{\titkol}

\vspace*{-9pt}

\noindent
$^1$Kaliningrad Branch of the Federal Research Center ``Computer Science and 
Control'' of the Russian Academy\linebreak
$\hphantom{^1}$of Sciences, 5~Gostinaya Str., Kaliningrad 236000, 
Russian Federation
   
   \noindent
   $^2$Immanuel Kant Baltic Federal University, 14~Nevskogo Str., Kaliningrad 236041, 
Russian Federation


\def\leftfootline{\small{\textbf{\thepage}
\hfill INFORMATIKA I EE PRIMENENIYA~--- INFORMATICS AND
APPLICATIONS\ \ \ 2016\ \ \ volume~10\ \ \ issue\ 3}
}%
 \def\rightfootline{\small{INFORMATIKA I EE PRIMENENIYA~---
INFORMATICS AND APPLICATIONS\ \ \ 2016\ \ \ volume~10\ \ \ issue\ 3
\hfill \textbf{\thepage}}}

\vspace*{3pt}
  
    
  
\Abste{The paper considers the problem of individual decision making during 
diagnostics of 
patients in outpatient clinics by the example of arterial 
hypertension diagnostics. It is proposed to 
raise the quality of individual decision\linebreak\vspace*{-12pt}}

\Abstend{making by means of consultations with the ``Virtual council'' 
decision support system, which models the work of physician councils in inpatient multifield 
clinics. The results of development and experimental research of the 
laboratory prototype of ``Virtual council'' are presented.}

\KWE{decision support system; virtual council; functional hybrid intellectual system}

\DOI{10.14357/19922264160311} 

\vspace*{-9pt}

\Ack
\noindent
The work was performed with partial support of the Russian
Foundation for Basic Research (grant No.\,16-07-00272~А).


%\vspace*{3pt}

  \begin{multicols}{2}

\renewcommand{\bibname}{\protect\rmfamily References}
%\renewcommand{\bibname}{\large\protect\rm References}

{\small\frenchspacing
 {%\baselineskip=10.8pt
 \addcontentsline{toc}{section}{References}
 \begin{thebibliography}{99}
\bibitem{1-kir-1}
\Aue{Gaaze-Rappoport, M.\,G., and D.\,A.~Pospelov}. 1987. \textit{Ot ameby do robota: Modeli 
povedeniya} [From ameba to robotic mashine: Behavior model] Moscow: Nauka. 288~p.
\bibitem{2-kir-1}
\Aue{Kolesnikov,~A.\,V., I.\,A.~Kirikov, S.\,V.~Listopad, \textit{et al.}}. 2011. \textit{Reshenie 
slozhnykh zadach kommivoyazhera metodami funktsional'nykh gibridnykh intellektual'nykh 
sistem} [Solving of the complex traveling salesman problem by means of functional hybrid 
intellectual systems]. Moscow: IPI RAN. 295~p.
\bibitem{3-kir-1}
\Aue{Kirikov, I.\,A., A.\,V.~Kolesnikov, and S.\,B.~Rumovskaya}.\linebreak
 2013. Issledovanie slozhnoy 
zadachi diagnostiki arterial'noy gipertenzii v~metodologii iskusstvennykh geterogennykh sistem 
[Research of the complex problem at\linebreak diagnosing of the arterial hypertension within the 
methodology of artificial heterogeneous systems]. \textit{Sistemy i~Sredstva Informatiki~--- 
Systems and Means of Informatics} 23(2):81--99. doi: 10.14357/08696527130208.
\bibitem{4-kir-1}
\Aue{Kirikov, I.\,A., A.\,V.~Kolesnikov, and S.\,B.~Rumovskaya}.\linebreak
 2014. Funktsional'naya 
gibridnaya intellektual'naya sistema dlya podderzhki prinyatiya resheniya pri diagnostike 
arterial'noy gipertenzii [Functional hybrid intelligent decision support system for diagnosing of the 
\mbox{arterial} hypertension]. \textit{Sistemy i~Sredstva Informatiki~--- Systems and Means of Informatics} 
24(1):153--179. doi: 10.14357/08696527140110. 
\bibitem{5-kir-1}
\Aue{Kolesnikov, A.\,V., I.\,A.~Kirikov, S.\,V.~Listopad, and S.\,B.~Rumovskaya}. 2015. 
Sistemnyy analiz v~reshenii slozhnykh diagnosticheskikh zadach [Systems analysis for solving 
complex diagnostic tasks]. \textit{Tr. 6-y Mezhdunar. konf. ``Sistemnyy analiz i~informatsionnye 
tekhnologii''} [6th Conference (International) ``Systems Analysis and Information Technology'' 
Proceedings]. Moscow.  1:157--167.
\bibitem{6-kir-1}
\Au{Kolesnikov, A.\,V., and I.\,A.~Kirikov}. 2007. \textit{Metodologiya i~tekhnologiya resheniya 
slozhnykh zadach metodami funk\-tsi\-o\-nal'\-nykh gibridnykh intellektual'nykh sistem} [Methodology 
and technology for solving of complex problems using the methodology of functional hybrid 
artificial systems]. Moscow: IPI RAN. 387~p.
\bibitem{7-kir-1}
\Aue{Kirikov, I.\,A., A.\,V.~Kolesnikov, and S.\,B.~Rumovskaya}. 2014. Issledovanie 
laboratornogo prototipa iskusstvennoy geterogennoy sistemy dlya diagnostiki arterial'noy 
gipertenzii [Research of the laboratory prototype of the artificial heterogeneous system for 
diagnosing of the arterial hypertension]. \textit{Sistemy i~Sredstva informatiki~--- Systems and 
Means of Informatics} 24(3):131--143. doi: 10.14357/08696527140309.
\bibitem{8-kir-1}
\Au{Rumovskaya, S.\,B.} 2015. Metody i~sredstva informatiki dlya diagnostiki 
arterial'noy gipertenzii v~lechebno-profilakticheskikh uchrezhdeniyakh shirokogo profilya 
[Methods and tools of informatics for diagnostics of arterial hypertension in multiskilled 
medical preventive institution]. \textit{Tr. 2-y molodezhnoy nauchnoy konf. ``Zadachi 
sovremennoy informatiki''} [2nd Youth Conference ``Tasks of Modern Informatics'' 
Proceedings]. Moscow: FRC ``Computer Science and Control'' RAS. 168--174.
\bibitem{9-kir-1}
\Aue{Kirikov, I.\,A., and S.\,B.~Rumovskaya}. 2016. Geterogennaya diagnostika arterial'noy 
gipertenzii [Heterogeneous diagnostics of arterial hypertension]. \textit{Tr. 4-y Vseross. 
nauchnoy konf. molodykh uchenykh s~mezhdunarodnym uchastiem ``Informatika, 
upravlenie i~sistemnyy analiz''} [4th Youth Conference (International) ``Informatics, Control 
and Systems Analysis'' Proceedings]. Tver: Tver State Technical University. 1:180--188.
\bibitem{10-kir-1}
Komitet ekspertov VNOK [Committee of experts of All-Russia Scientific Society of Сardiologists]. 
2010. Diagnostika i~lechenie arterial'noy gipertenzii. Rossiyskie 
rekomendatsii [Diagnosing and treatment of arterial 
hypertension. Russian recommenation]. 
\textit{Sistemnye gipertenzii} [Systemic Hypertension] 3:5--26. 
\bibitem{11-kir-1}
\Aue{Galimzyanov, F.\,V.} 2014. Zabolevaniya perifericheskikh arteriy (Klinika, 
diagnostika, lechenie) [Peripheral vascular disease (Clinic, diagnostics, treatment]. 
\textit{Mezhdunarodnyy zhurnal eksperimental'nogo obrazovaniya} [Int. J.~Research 
Education] 8:113--114. 
   \end{thebibliography}

 }
 }

\end{multicols}

\vspace*{-9pt}

\hfill{\small\textit{Received June 18, 2016}}

\vspace*{-3pt}
    
  
  \Contr
  
  \noindent
  \textbf{Kirikov Igor A.}\ (b.\ 1955)~---
  Candidate of  Sciences (PhD) in technology; director, Kaliningrad Branch of the 
  Federal Research Center ``Computer Science and Control'' of the Russian Academy 
  of Sciences, 5~Gostinaya Str., Kaliningrad 236000,  Russian Federation; 
baltbipiran@mail.ru
  
  \pagebreak
%  \vspace*{3pt}
  
  \noindent
  \textbf{Kolesnikov Alexander V.}\ (b.\ 1948)~---
  Doctor of Sciences in technology; professor, 
Department of Telecommunications, 
 Immanuel Kant Baltic Federal University, 14~Nevskogo Str., Kaliningrad 236041, Russian Federation; senior scientist, Kaliningrad Branch of 
  the Federal Research Center ``Computer Science and Control'' of the Russian 
  Academy of Sciences, 5~Gostinaya Str., Kaliningrad 236000,  Russian Federation; 
  avkolesnikov@yandex.ru
  
  \vspace*{4pt}
  
  \noindent
  \textbf{Listopad Sergey V.}\ (b.\ 1984)~---
  Candidate of  Sciences (PhD) in technology; scientist, Kaliningrad Branch of the 
  Federal Research Center ``Computer Science and Control'' of the Russian Academy 
  of Sciences, 5~Gostinaya Str., Kaliningrad 236000,  Russian Federation;   
ser-list-post@yandex.ru
  
  \vspace*{4pt}
  
  \noindent
  \textbf{Rumovskaya Sophiya B.}\ (b.\ 1985)~--- programmer~I, Kaliningrad Branch 
  of the Federal Research Center ``Computer Science and Control'' of the Russian 
  Academy of Sciences, 5~Gostinaya Str., Kaliningrad 236000,  Russian Federation; 
  sophiyabr@gmail.com
  \label{end\stat}
  
  
  \renewcommand{\bibname}{\protect\rm Литература}  %10
\def\stat{monahov}

\def\tit{РАЗЛОЖЕНИЯ ЧЕБЫШЁВА--ЭДЖВОРТА ДЛЯ~РАСПРЕДЕЛЕНИЙ ОБОБЩЕННЫХ СТАТИСТИК ТИПА 
ХОТЕЛЛИНГА, ПОСТРОЕННЫХ ПО~ВЫБОРКАМ СЛУЧАЙНОГО РАЗМЕРА$^*$}

\def\titkol{Разложения Чебышёва--Эджворта для~распределений обобщенных статистик типа 
Хотеллинга} %, построенных по~выборкам случайного размера}

\def\aut{М.\,М.~Монахов$^1$}

\def\autkol{М.\,М.~Монахов}

\titel{\tit}{\aut}{\autkol}{\titkol}

\index{Монахов М.\,М.}
\index{Monakhov M.\,M.}

{\renewcommand{\thefootnote}{\fnsymbol{footnote}} \footnotetext[1]
{Исследование выполнено в~соответствии с программой Московского центра фундаментальной 
и~прикладной математики.}}


\renewcommand{\thefootnote}{\arabic{footnote}}
\footnotetext[1]{Московский центр фундаментальной и~прикладной 
математики Московского государственного университета имени М.\,В.~Ломоносова, 
\mbox{mih\_monah@mail.ru.}}


%\vspace*{-12pt}



\Abst{Доказан аналог теоремы переноса для функций распределения статистики 
типа Хотеллинга, размер которой является случайной величиной, позволяющий 
оценить скорость сходимости разложения Че\-бы\-шё\-ва--Эдж\-вор\-та и~получить явный вид 
вышеупомянутого разложения для исходной статистики. На основании следствия 
к~доказанному аналогу теоремы переноса для случая, когда размер статистики имеет 
отрицательное биномиальное распределение (смещенное на 1), получен явный вид 
разложения Че\-бы\-шё\-ва--Эдж\-вор\-та второго порядка на базе предельного 
$F$-рас\-пре\-де\-ле\-ния. По построенному разложению Че\-бы\-шё\-ва--Эдж\-вор\-та для специального 
значения параметра случайного размера выборки построено разложение Кор\-ни\-ша--Фи\-ше\-ра 
второго порядка на базе квантилей $F$-рас\-пре\-де\-ле\-ния. Проведен 
вычислительный эксперимент и~построены графики, иллюстрирующие полученное 
разложение Че\-бы\-шё\-ва--Эдж\-ворта.}


\KW{обобщенные разложения Чебышёва--Эджворта; разложения Кор\-ни\-ша--Фи\-ше\-ра;
выборка случайного объема; $F$-рас\-пре\-де\-ле\-ние;  статистика типа Хотеллинга}

\DOI{10.14357/19922264210211}

%\vspace*{-3pt}


\vskip 10pt plus 9pt minus 6pt

\thispagestyle{headings}

\begin{multicols}{2}

\label{st\stat}


\section{Введение}
%\label{intro}

В анализе данных довольно часто возникает задача множественных сравнений. 
Например, различных возрастных, профессиональных, социальных слоев населения, 
или влияния различных доз препарата, методов диагностики и~т.\,д. Данную задачу 
помогает решить дисперсионный анализ, который применяется для исследования 
влияния одной или нескольких качественных переменных\linebreak (факторов) на одну 
зависимую количественную переменную. Дисперсионный  анализ широко применяется 
в~сфере производства, здравоохранения, рекла\-мы, продовольствия, обслуживания, его 
реализации представлены в~статистических пакетах для многих языков 
программирования. Сущность дисперсионного анализа заключается в~расчленении 
общей дисперсии изучаемого признака на отдельные компоненты, обусловленные 
влиянием конкретных факторов, и~проверке гипотез о~зна\-чи\-мости влияния этих 
факторов на исследуемый признак. Дополнительные проб\-ле\-мы возникают в~случае, 
когда объем наблюдения оказывается случайным~\cite{BenKorGal13}.

 В задачах многомерного однофакторного дисперсионного анализа рассматриваются~$q$ 
 выборок с фиксированным размером $n_1, \ldots, n_q$: $(X_{1 1},\ldots, X_{1 
n_1}), \ldots, (X_{q 1},\ldots, X_{q n_q})$, где $X_{i j}$~--- \mbox{$p$-мер}\-ное 
наблюдение, представимое в~виде:
$$
X_{i j} = \mu + \alpha_i + \epsilon_{ij}\,. 
$$
Здесь $\mu$ и~$\alpha_i$~--- неизвестные векторные параметры; $\epsilon_{ij}$~--- 
случайные ошибки, явля\-ющи\-еся независимыми одинаково распределенными случайными 
величинами с нормальным распределением~$N_p(0,B)$. При рассмотрении основной 
гипотезы од\-но\-род\-ности выборок
$$
H_0: \alpha_1 = \cdots = \alpha_q = 0
$$
определяются матрицы~$S_h$ и~$S_e$, отражающие межуровневые и~внутриуровневые 
различия соответственно для элементов выборок
\begin{align*}
S_h &= \sum\limits_{i=1}^{q}n_i(\bar{y}_i - \bar{y})(\bar{y}_i - \bar{y})'; \\
 S_e &= 
\sum\limits_{i=1}^{q} \sum\limits_{j=1}^{n_i}(y_{ij} - \bar{y}_i)(y_{ij} - \bar{y}_i)'
\end{align*}
с $n = n_1 + \cdots + n_q$ и~$$
\bar{y}_i = \fr{1}{n_i}\sum\limits_{j=1}^{n_i}y_{ij}; \quad \bar{y} = 
\fr{1}{n}\sum\limits_{i=1}^{q}\sum\limits_{j=1}^{n_i}y_{ij}.
$$
В предположении справедливости основной гипотезы~$H_0$ случайные матрицы~$S_h$ 
и~$S_e$ независимы и~имеют центральные распределения Уишарта $W_p(q, I_p)$ 
и~$W_p(n, I_p)$ соответственно. На базе мат\-риц~$S_h$ и~$S_e$ для проверки гипотезы~$H_0$ 
строятся статистики, одной из которых является статистика Лоу\-ли--Хо\-тел\-линга.

В работах~\cite{BenKorGal13,BenKorGal12} была доказана общая тео\-ре\-ма\linebreak переноса, 
позволяющая оценить ско\-рость сходимости разложения типа Че\-бы\-шё\-ва--Эдж\-вор\-та 
первого порядка для асимптотически нормальных статистик, построенных по выборкам 
случайного\linebreak объема, а~также получить явный вид данного разложения. В~качестве 
примера статистики в~этих работах рассматривается выборочное среднее, которое 
приближается нормальным распределением. В~работе~\cite{MMU16} получено 
разложение Кор\-ни\-ша--Фи\-ше\-ра первого порядка для разложения Че\-бы\-шё\-ва--Эдж\-вор\-та из 
работы~\cite{BenKorGal13}. В~работе~\cite{CMU} получены разложения 
Че\-бы\-шё\-ва--Эдж\-вор\-та и~Кор\-ни\-ша--Фи\-ше\-ра второго порядка для статистик типа выборочного 
среднего, построенных по выборкам случайного объема. Данная работа развивает 
результаты вышеперечисленных работ. Для статистики типа Хотеллинга случайного 
размера доказан аналог теоремы переноса и~построено асимптотическое разложение 
типа Че\-бы\-шё\-ва--Эдж\-вор\-та для функции распределения данной статистики.

Используем следующие обозначения: $\mathbf{R}$~--- множество вещественных чисел;
$\mathbf{N}:=\{1,2,\ldots\}$~--- положительные целые числа; $\mathbf{I}_{A}(x)$~--- индикаторная функция.

Определим статистику Лоу\-ли--Хо\-тел\-лин\-га (см., например,~\cite{UAF2016}):
\begin{equation}
\label{hotel}
T_n=T^2_0=n\,\mathrm{tr}\,S_h S_e^{-1}.
\end{equation}
Рассмотрим случай, когда параметр~$n$  не определен заранее, а является 
случайной величиной~$N_n$. В~этом случае размеры выборок $n_1, \ldots, n_q$ 
становятся независимыми одинаково распределенными случайными величинами 
$N_{n_1}, \ldots, N_{n_q}$, а случайная матрица~$S_e$ становится случайной 
матрицей~$S_{N_n}$, которая в~предположении справедливости основной гипотезы~$H_0$ 
имеет центральное распределение Уишарта $W_p(N_n, I_p)$. Обобщенная 
нормированная статистика Хотеллинга случайного размера запишется в~виде:
\begin{equation}
\label{r_hotel}
T_{N_n} = \widetilde{T}^2_0= g(n)\,\mathrm{tr}\,S_h S_{N_n}^{-1}.
\end{equation}

 В разд.~2 получен аналог теоремы переноса для обобщенной 
нормированной статистики Хотеллинга случайного размера, в~разд.~3 
построен аналог разложения Че\-бы\-шё\-ва--Эдж\-вор\-та для данной статистики, в~разд.~4 
получен явный вид разложения Кор\-ни\-ша--Фи\-ше\-ра для частного случая 
параметра размера данной статистики. В~разд.~5 приведены 
доказательства полученных тео\-рем.


\section{Аналог теоремы переноса для~статистики типа Хотеллинга}
%\label{tr_teo}

Запишем следующую теорему из работы~\cite[теорема 4.1]{FUS05}.

\smallskip

\noindent
\textbf{Теорема~1.}\
%\begin{theore}\label{teorUAF}
\textit{Пусть статистика~$T_n$ определена в~формуле~\eqref{hotel}, $ G_k(x)\hm=\mathrm{Pr}\left \{ \chi 
^2 < x \right\} $~--- функция распределения хи-квад\-рат с~$k$ степенями свободы. 
Существует вещественное число $C_1 \hm> 0$ такое, что для всех целых $n \hm\geq 1$}
\begin{multline}
\label{con1}
\sup\limits_x\left\vert 
\vphantom{\sum\limits_{j=0}^2}
\mathbb{P}\left(n\,\mathrm{tr}\,S_h S_e^{-1} \leq x \right) - G_k(x) - {}\right.\\
\left.{}-
\fr{k}{4n}\sum\limits_{j=0}^2 a_j G_{k+2j}(x)\right\vert \leq C_1 n^{-2},
\end{multline}
\textit{где $k=pq$}; $a_0\hm=q\hm-p\hm-1$; $a_1\hm=-2q$; $a_2\hm=q\hm+p\hm+1$.

\smallskip


Предположим, что функция распределения нормированного случайного размера выборки~$N_n$ 
удовлетворяет следующему условию.

\smallskip

\noindent
\textbf{Условие~1.}\ {Существуют константы $m \hm\in \mathbb{N}$, $\beta \hm> m/2$, $C_2 \hm> 
0$, функция распределения~$H(y)$ с~$H(0+)\hm = 0$, функции ограниченной вариации 
$h_i(y)$, $i\hm=1, \ldots, m$, последовательность $0\hm<g(n) \uparrow \infty$, $n 
\hm\rightarrow \infty$ такие, что для всех целых $n \hm\geq 1$}
\begin{multline}
\label{con2}
\sup\limits_{y \geq 0} \left\vert \mathbb{P}\left( \fr{N_n}{g(n)} \leq y\right) - H(y) - 
\sum\limits_{i=1}^{m}\fr{1}{n^{i/2}} h_i(y) \right\vert  \leq{}\\
{}\leq C_2 n^{- \beta},\enskip
 n \in \mathbb{N}\,.
\end{multline}

Сформулируем аналог теоремы переноса, позволяющий оценить распределение 
обобщенной нормированной статистики Хотеллинга случайного размера $g(n) 
\,\mathrm{tr}\, S_h S^{-1}_{N_n}$.

\smallskip

\noindent
\textbf{Теорема~2.}\
%\begin{theore}\label{transfer}
%
\textit{Пусть статистика~$T_{N_n}$ определена в~формуле~\eqref{r_hotel} и~для случайного 
размера выборки $N_n$ выполнено условие~$1$. Тогда существует константа $C_3\hm>0$ 
такая, что справедливо неравенство}
\begin{multline*}
%\label{eq10z}
\sup\limits_{x} \left\vert \mathbb{P}\left( g(n) \,\mathrm{tr}\,S_h S^{-1}_{N_n} \leq 
x\right)  - F_{n}(x) \right\vert \leq{}\\
{}\leq C_1\mathbb{E}N_n^{-2} + \fr{C_3 + C_2 
M_n}{n^{\beta}}\,,
\end{multline*}
\textit{где}
\begin{multline*}
%\label{gn}
F_{n}(x) = \int\limits^{\infty}_{1/g(n)} G_k(xy)\,dH(y) + {}\\
{}+
\sum\limits_{i=1}^{m}\fr{1}{n^{i/2}}\int\limits^{\infty}_{1/g(n)} G_k(xy)\,dh_i(y) 
+{} \\
{}+\fr{k}{4 g(n)} \int\limits_{1/g(n)}^\infty \sum\limits_{j=0}^{2} 
\fr{a_j}{y}\,G_{k+2j}(xy) \,dH(y)
+{}\\
{}+ \fr{k}{4 g(n)} \sum\limits_{i=1}^{m}\fr{1}{n^{i/2}} \int\limits_{1/g(n)}^\infty 
\fr{1}{y} \sum\limits_{j=0}^{2}a_j G_{k+2j}(xy)\,dh_i(y);
\end{multline*}

\vspace*{-12pt}

\noindent
\begin{multline*}
%\label{mn}
M_n=  \sup\limits_{x}    \int\limits_{1/g(n)}^{\infty}  \left\vert
\fr{\partial}{\partial y} 
\left( 
\vphantom{\sum\limits_{j=0}^{2}}
G_k\left(yx \right) + {}\right.\right.\\
\left.\left.{}+\fr{k}{4g(n)y}\sum\limits_{j=0}^{2}a_j G_{k+2j}\left(yx 
\right) \right)\right\vert dy\,.
\end{multline*}

 

\noindent
\textbf{Следствие~1.} 
%\begin{cor}\label{foll1}
В условиях теоремы~2 с дополнительными предположениями
\begin{gather*}
h_2(0) = 0\,; \enskip  H\left(\fr{1}{g(n)}\right)\leq c_0 n^{-\gamma}\,; \\
 h_2\left(\fr{1}{g(n)}\right)\leq c_1 n^{1-\gamma}; \\
\int\limits^{1/g(n)}_{0}\fr{1}{y}\,dH(y)\leq c_2 g(n) n^{-\gamma}, \\ 
\int\limits^{1/g(n)}_{0} \fr{1}{y} \,dh_2(y) \leq c_3 g(n) n^{1-\gamma}
\end{gather*}
для некоторого $\gamma > 1$
существует $C_3\hm=C_3(C_2,k,q)$ такая, что $ \forall\,n \hm\in \mathbb{N}$
\begin{multline*}
%\label{follow1}
\sup\limits_{x} \left| \mathbb{P}\left( g(n) \,\mathrm{tr}\,S_h S^{-1}_{N_n} \leq 
x\right)  - F_{2;n}(x) \right| \leq{}\\
{}\leq C_1\mathbb{E}N_n^{-2} + C_3 n^{-\min(\beta,\gamma)},
\end{multline*}
где
\begin{multline*}
F_{2;n}(x) = \int\limits^{\infty}_{0} G_k(xy)\,dH(y) +{}\\
{}+ \fr{k}{4g(n)} 
\int\limits_{0}^\infty \sum\limits_{j=0}^{2} \fr{a_j}{y} G_{k+2j}(xy)\,dH(y) +  {} 
\end{multline*}

\noindent
\begin{multline}
\label{gn2n}
{}+  \fr{1}{n}\int\limits^{\infty}_{0} G_k(xy) \,dh_2(y) + {}\\
{}+
\fr{k}{4ng(n)} \int\limits_{0}^\infty \sum\limits_{j=0}^{2}\fr{a_j}{y}\,G_{k+2j}(xy)\,dh_2(y)\,.
\end{multline}

\vspace*{-6pt}

\section{Разложение Чебышёва--Эджворта}
%\label{e_che}

Рассмотрим теперь пример применения тео\-ре\-мы~2. Пусть размер выборки~$N_n(r)$ 
имеет отрицательное биномиальное распределение (смещен на~1) 
с~вероятностью успеха~$1/n$ и~функцией вероятности

\vspace*{-6pt}

\noindent
\begin{multline}
\label{eq5}
\mathbb{P}(N_n(r)=j)= \fr{\Gamma(j+r-1)}{(j-1)!\Gamma(r)}\left(\fr{1}{n} \right)^{r}
\left( 1 - \fr{1}{n} \right)^{j-1}, \\
 r>0, \ \ j =1, 2, 
\ldots
\end{multline}

\vspace*{-2pt}

Теперь получим разложение Че\-бы\-шё\-ва--Эдж\-вор\-та для нормированной статистики типа 
Хотеллинга. Функция $F$-рас\-пре\-де\-ле\-ния~$F(x;a,b)$~--- это абсолютно 
непрерывная функция распределения вероятности, заданная плот\-ностью
\begin{multline*}
f(x;a,b) = {}\\
{}=\fr{1}{B(a/2,b/2)}\left(\fr{a}{b}\right)^{a/2} x^{a/2-1} 
\left(1+\fr{a}{b}x\right)^{-(a+b)/2},\\
 x>0\,.
\end{multline*}

\vspace*{-6pt}

\noindent
\textbf{Лемма~1.}
%\begin{lemm}\label{NM}  
\textit{Пусть $r\hm>1$, случайная величина~$N_n(r)$ определена 
формулой~\eqref{eq5}, тогда}
\begin{equation}
\label{NM2}
\mathbb{E}\left(N_n(r)\right)^{- 2} \leq C(r) 
\begin{cases} 
n^{- r}, &  1 < r < 2\,;\\ 
\ln(n)  n^{- 2}, &   r  = 2\,; \\
 n^{-2}, & r > 2\,.
 \end{cases}
\end{equation}
\textit{В случае если $r\hm=2$, скорость сходимости в~\eqref{NM2}, не может быть улучшена}.


\smallskip

Используя теорему~1 из работы~\cite{CMU}, получаем следующий результат.



\smallskip

\noindent
\textbf{Теорема~3.}\
%\begin{theore}\label{T3}
\textit{Пусть статистика $T_m$ определена формулой \eqref{hotel}.
Пусть также дискретная случайная величина $N_n\hm=N_n(r)$ с параметром $r \hm> 1$ 
имеет распределение, задаваемое}~\eqref{eq5}, \textit{и~независима от $W_p(q, I_p)$ 
и~$W_p(n, I_p)$.
Рассмотрим статистику $T_{N_n} \hm= g(n) \,\mathrm{tr} S_h S^{-1}_{N_n}$.  
Асимптотическое разложение для случайного объема $N_n(r)$ с~$r\hm>1$ из}~\cite[теорема~1]{CMU} 
\textit{справедливо с $g(n) = \mathbb{E}\left(N_n(r)\right) = r(n\hm-
1)\hm+1$. Тогда существует константа $C\hm=C(r)\hm>0$  такая, что для всех $n \in 
\mathbf{N}$}

\end{multicols}

\begin{figure*}[b] %fig1
  \vspace*{6pt}
  \begin{center}
    \mbox{%
 \epsfxsize=163mm 
 \epsfbox{mon-1.eps}
 }
\end{center}

%\vspace*{-3pt}

{\small Эмпирическая функция распределения $\mathbb{P}\left( g(n) 
\,\mathrm{tr}\,S_h S^{-1}_{N_n} \leq x\right)$~(\textit{1}), аппроксимация 
первого порядка $F\left({x}/{k};k,2r\right)$~(\textit{2}) 
и~аппроксимация второго порядка~$F_{2;n}(x)$~(\textit{3}) при
 $p\hm=1$, $n\hm=10$ и~$r\hm=3$:
 (\textit{а})~$q\hm=1$; (\textit{б})~3; (\textit{в})~5;
 (\textit{г})~$q\hm=9$}
 %\label{hotel_q1}
\end{figure*}

\begin{multicols}{2}

\noindent
\begin{multline}
\label{eq10q}
\sup\limits_{x} \left\vert \mathbb{P}\left( g(n) \,\textrm{tr} S_h 
S^{-1}_{N_n} \leq x\right) - F_{2; n}(x) \right\vert \leq{}\\
{}\leq  C  
\begin{cases} 
n^{- r}, &  1 < r < 2\,;\\ 
\ln(n) \, n^{- 2}, &   r  = 2\,; \\ 
n^{- 2}, & r > 2\,;
 \end{cases}
\end{multline}
\textit{где}
\begin{multline*}
%\label{g2f}
F_{2;n}(x) ={}\\
{}= F\left(\fr{x}{k};k,2r\right)
+ \fr{1}{n}\,\fr{(r-2)x}{2rk} \left(f\left(\fr{x}{k};k,2r\right) - {}\right.\\
\left.{}-
f\left( \fr{r-1}{rk}\,x;k,2r-2\right)\right) + 
\fr{k}{4(r(n-1)+1)}\times{}\\
{}\times \sum\limits_{j=0}^{2} a_j  F\left( \fr{r-1}{(k+2j)r} 
\,x;k+2j,2r-2\right) + {}\\
{}+ \fr{k}{4n(r(n-1)+1)} \times{}\\
{}\times 
\sum\limits_{j=0}^{2} a_j \! \left[ \fr{2-r}{2(r-1)} 
\,F\!\left( \fr{r-1}{r(k+2j)}\,x;k+2j,2r-2\!\right) +{} \right.\hspace*{-3.20157pt}
\end{multline*}

\noindent
\begin{multline*}
{}+\fr{r}{2(r-1)}\,F\left( \fr{r-2}{r(k+2j)} \,x;k+2j,2r-4\right)
-{}\\[2pt]
{}- \fr{(2-r)x}{2r(k+2j)} \,f\left( \fr{r-1}{r(k+2j)}\,x;k+2j,2r-2\right) - 
{}\\[2pt]
 {}- \fr{(r-2)x}{2(k+2j)(r-1)}\times{}\\[2pt]
\left.{}\times f\left( \fr{r-2}{r(k+2j)} 
\,x;k+2j,2r-4\right) \right]. 
%\end{alignedat}
\end{multline*}





На рисунке демонстрируется преимущество разложения 
Че\-бы\-шё\-ва--Эдж\-вор\-та второго порядка над разложением первого порядка в~приближении 
эмпирической функции распределения.



\section{Разложение Корниша--Фишера}
%\label{cor_fish}

Рассмотрим частный случай основного результата теоремы~3. Пусть в~\eqref{eq10q} 
$r\hm=3/2$. Обозначим $\mathbf{P}\left( g(n) \,\textrm{tr}\,S_h S^{-1}_{N_n} \leq x\right)
\hm = \bar{F}(x)$. Тогда

\noindent
\begin{equation}
\label{p_cheb}
\bar{F}(x) = F_{2; n}(x)|_{r=3/2} + \mathcal{O}(1/n^{3/2}), \enskip n \to \infty\,,
\end{equation}
где
\begin{multline*}
%\label{g2fr}
F_{2;n}(x) |_{r=3/2} = F\left(\fr{x}{k};k,3\right) - 
\fr{1}{n}\,\fr{x}{6k} \,f\left(\fr{x}{k};k,3\right) + {}\\
{} +\fr{1}{n}\,\fr{x}{6k} \,f\left( \fr{x}{3k};k,1\right) + {}\\
{}+\fr{k}{2(3n-1)} 
\sum\limits_{j=0}^{2} a_j  F\left( \fr{1}{(k+2j)3}\,x;k+2j,1\right).  
%\end{alignedat}
\end{multline*}

Домножим и~разделим третий и~четвертый член разложения на плот\-ность 
$f\left({x}/{k};k,3\right)$:
\begin{multline}
\label{cheb0}
F_{2;n}(x) |_{r=3/2} ={}\\
{}= F\left(\fr{x}{k};k,3\right) + \fr{1}{n}\,d_2(x) 
f\left(\fr{x}{k};k,3\right),
\end{multline}
где
\begin{multline*}
d_2(x) = - \fr{x}{6k} + \fr{x}{6k} \,\fr{f\left(  {x}/{(3k)};k,1\right)}
{f\left({x}/{k};k,3\right)} + {}\\
{}+
\fr{k}{2(3-1/n)} 
\sum\limits_{j=0}^{2} a_j  \fr{F\left( {x}/({(k+2j)3});k+2j,1\right)}{f\left({x}/{k};k,3\right)}\,.
\end{multline*}
Перепишем исходное выражение~\eqref{p_cheb}, используя~\eqref{cheb0}:
\begin{multline}
\label{cheb}
\bar{F}(x) = F\left(\fr{x}{k};k,3\right) + \fr{1}{n}\,d_2(x) 
f\left(\fr{x}{k};k,3\right) + {}\\
{}+\mathcal{O}\left(\fr{1}{n^{3/2}}\right), \quad
n \to \infty\,.
\end{multline}

Используя разложение Че\-бы\-шё\-ва--Эдж\-вор\-та~\eqref{cheb}~\cite[утверждение 2]{CMU}, 
вытекающее из более общих утверждений (см., например, работы~[7, гл.~5.6.1] и[8]), 
с~$a_1(x)\hm=0$, $a_2(x)\hm=d_2(x)$, 
$g(x)\hm=f({x}/{k};k,3)$, $G(x)\hm=F({x}/{k};k,3)$, получаем следующую тео\-ре\-му.

\smallskip

\noindent
\textbf{Теорема~4.}\
%\begin{theore}\label{cf_teor}
\textit{В условиях теоремы~$3$ пусть $x\hm=x_\alpha$, $u\hm=u_\alpha$~---
 $\alpha$-кван\-ти\-ли нормированной статистики $\mathbf{P}\left( g(n) \,\mathrm{tr}\,S_h S^{-1}_{N_n} 
 \hm\leq x\right)$ и~предельного $F$-рас\-пре\-де\-ле\-ния соответственно.
Тогда справедливо сле\-ду\-ющее асимптотическое разложение для}  $n \hm\to \infty$:
\begin{multline*}
%\label{eqCF1}
x=ku +  \fr{u}{6} \left(1-  \fr{f\left( u/3;k,1\right)}{f\left(u;k,3\right)} 
\right)n^{-1} - {}\\
{}=
\fr{k}{6} \sum\limits_{j=0}^{2} a_j  
\fr{F\left( {uk}/({(k+2j)3});k+2j,1\right)}{f\left(u;k,3\right)}\,n^{-1} + {}\\
{}+ \mathcal{O}(n^{- 3/2}).
\end{multline*}


\section{Доказательства}
%\label{proofs}



\noindent
Д\,о\,к\,а\,з\,а\,т\,е\,л\,ь\,с\,т\,в\,о\ \ тео\-ре\-мы~2.

 Используя формулу полной вероятности, получаем
\begin{multline}
\label{pr1_1}
\mathbb{P} \left( g(n) \,\mathrm{tr} \,S_h S^{-1}_{N_n} \leq x \right) = {}\\
{}=
\mathbb{E}\mathbb{P} \left( N_n \,\textrm{tr}\, S_h S^{-1}_{N_n} \leq 
\fr{N_n}{g(n)}\,x  \vert  N_n \right) = {} \\
{}= \sum\limits_{l=1}^{\infty} \mathbb{P} \left( l \,\mathrm{tr}\,S_h S^{-1}_{e} \leq 
\fr{l}{g(n)}\,x \right)  \mathbb{P}(N_n=l)\,.
 \end{multline}

Далее введем $F_n(x)$, подставив оценку~\eqref{con1} для статистики Хотеллинга 
из теоремы~1 без остаточного члена и~оценку~\eqref{con2} без 
остаточного члена для нормированной функции распределения размера выборки из 
условия~1 в~\eqref{pr1_1}:
\begin{multline}
\label{pr1_2}
F_n(x)={}\\
{}=\mathbb{P} \left( g(n) \,\mathrm{tr}\, S_h S^{-1}_{N_n} \leq x \right)
\stackrel{(\ref{con1})}{=}
 \mathbb{E}\left( 
 \vphantom{\sum\limits_{j=0}^{2}}
 G_k\left(\fr{N_n}{g(n)}\,x \right) + {}\right.\\
\left. {}+
\fr{k}{4N_n}\sum\limits_{j=0}^{2}a_j G_{k+2j}\left(\fr{N_n}{g(n)}\,x \right)\! \!\right) 
= \! \int\limits_{1/g(n)}^{\infty}\!\left( 
\vphantom{\sum\limits_{j=0}^{2}}
G_k\left(yx \right) + {}\right.\\
\left.{}+
\fr{k}{4g(n)y}\sum\limits_{j=0}^{2}a_j G_{k+2j}\left(yx \right) \right) 
d\mathbb{P}\left( \fr{N_n}{g(n)} < y \right)  \stackrel{(\ref{con2})}{=}{} \\
{}\stackrel{(\ref{con2})}{=} \int\limits_{1/g(n)}^{\infty}\left(
\vphantom{\sum\limits_{j=0}^{2}} 
G_k\left(yx \right) + 
\fr{k}{4g(n)y}\times{}\right.\\
\left.{}\times \sum\limits_{j=0}^{2}a_j G_{k+2j}\left(yx \right) 
\right) d\left( H(y) - \sum\limits_{i=1}^{m}\fr{1}{n^{i/2}}\, h_i(y) \right) 
={}\\
{}=\int\limits^{\infty}_{1/g(n)} G_k(xy)\,dH(y) + {}\\
{}+
\sum\limits_{i=1}^{m}\fr{1}{n^{i/2}}\int\limits^{\infty}_{1/g(n)} G_k(xy)\,dh_i(y) + {}
 \\
{}+ \fr{k}{4 g(n)} \int\limits_{1/g(n)}^\infty \sum\limits_{j=0}^{2} 
\fr{a_j}{y}\,G_{k+2j}(xy)\, dH(y)+\fr{k}{4 g(n)}\times{}\\
{}\times
 \sum\limits_{i=1}^{m}\fr{1}{n^{i/2}} \int\limits_{1/g(n)}^\infty 
\fr{1}{y} \sum\limits_{j=0}^{2}a_j G_{k+2j}(xy)\,dh_i(y)\,.
 \end{multline}

\noindent
Найдем теперь оценки для $\sup\nolimits_{x} \left| \mathbb{P}\left( g(n) \,\textrm{tr} \,S_h S^{-1}_{N_n}
\hm \leq\right.\right.$\linebreak $\left.\left.\leq x\right) \hm - F_{n}(x) \right|$. Введем обозначение
\begin{equation*}
\psi (n;y) =  \mathbb{P}\left( \fr{N_n}{g(n)} < y \right) - H(y) - 
\sum\limits_{i=1}^{m}\fr{1}{n^{i/2}}\, h_i(y)\,.
\end{equation*}

Согласно~\eqref{pr1_2},
\begin{multline*}
\sum\limits_{l=1}^{\infty} \left(
\vphantom{\sum\limits_{j=0}^{2}}
 G_k\left(\fr{l}{g(n)}\,x \right) +{}\right.\\
\left.{}+ 
\fr{k}{4l}\sum\limits_{j=0}^{2}a_j G_{k+2j}\left(\fr{l}{g(n)}x \right) \right)  
\mathbb{P}(N_n=l) ={} \\
{}= \int\limits_{1/g(n)}^{\infty}\left(
\vphantom{\sum\limits_{j=0}^{2}}
 G_k\left(yx \right) + {}\right.\\
\left.{}+
\fr{k}{4g(n)y}\sum\limits_{j=0}^{2}a_j G_{k+2j}\left(yx \right) \right) 
d\mathbb{P}\left( \fr{N_n}{g(n)} < y \right),
\end{multline*}
поэтому
\begin{equation*}
\sup\limits_{x} \left| \mathbb{P}\left(g(n) \,\mathrm{tr} \,S_h S^{-1}_{N_n} \leq 
x\right)  - F_{n}(x) \right| \leq I_{1n} + I_{n2},
\end{equation*}
где
\begin{align*}
I_{1n}&= \sup_{x} \left| \,  \int\limits_{1/g(n)}^{\infty}\left(
\vphantom{\sum\limits_{j=0}^{2}}
 G_k\left(yx \right) + {}\right.\right.\\
&\left.\left.{}+
\fr{k}{4g(n)y}\sum\limits_{j=0}^{2}a_j G_{k+2j}(yx) 
\right)  d\psi(n; y) 
\vphantom{\int\limits_{1/g(n)}^{\infty}}
\right|;\\
I_{2n}&= \sum\limits_{l=1}^{\infty} \sup\limits_{x} \left| 
\vphantom{\sum\limits_{j=0}^{2}}
 \mathbb{P} \left( l \,\mathrm{tr} \,
S_h S^{-1}_{l} \leq \fr{l}{g(n)}x \right) -\right.{}\\
&\left.{}- G_k\left(\fr{l}{g(n)}\,x \right) 
-  \fr{k}{4l}\sum\limits_{j=0}^{2}a_j G_{k+2j}\left(\fr{l}{g(n)}\,x \right) 
\right|\times{}\\
&\hspace*{52mm}{}\times \mathbb{P}(N_n=l).
\end{align*}

Используя формулу интегрирования по частям и~условие~1, получаем, что существует 
константа $C_3\hm>0$ такая, что
\begin{multline*}
I_{1n}\leq \fr{C_3}{n^{\beta}} + \sup\limits_{x}    \int\limits_{1/g(n)}^{\infty} \left| 
(\psi(n;y)\right| \left|\fr{\partial}{\partial y} \left(
\vphantom{\sum\limits_{j=0}^{2}}
 G_k\left(yx \right) + {}\right.\right.\\
\left.\left.{}+
\fr{k}{4g(n)y}\sum\limits_{j=0}^{2}a_j G_{k+2j}\left(yx \right) \right)\right| \,dy 
\leq \fr{C_3}{n^{\beta}} + \fr{C_2 M_n}{n^{\beta}}\,.
\end{multline*}

Используя результат теоремы~1, получаем
\begin{equation*}
I_{2n} \leq  \sum\limits_{l=1}^{\infty} \fr{C_1}{l^2}\,\mathbb{P}(N_n=l) = 
C_1\mathbb{E}N_n^{-2}. \enskip \square
\end{equation*}

%\vspace{3mm}
%\vskip12pt
\noindent
Д\,о\,к\,а\,з\,а\,т\,е\,л\,ь\,с\,т\,в\,о\ следствия~1 и~леммы~1 аналогичны доказательствам 
утверждения~1 и~леммы~1 из работы~\cite{CMU}.

 
\noindent
Д\,о\,к\,а\,з\,а\,т\,е\,л\,ь\,с\,т\,в\,о\ \ теоремы~3.

Обозначим
\begin{align*}
 J_1(x) &= \int\limits^{\infty}_{0} G_k(xy)dG_{r,r}(y)\,;\\
 J_2(x) &= {}\\
 &\hspace*{-9mm}{}= \!
\int\limits^{\infty}_{0} G_k(xy) \,d\left[  \fr{(y - 1)(2 - r) + 2 
Q_1\left(g(n)y\right)}{2 r}\,g_{r,r}(y) \right]\!;\hspace*{-0.60292pt}
 \\
 J_{3}(x) &=\int\limits_{0}^\infty \sum\limits_{j=0}^{2} \fr{a_j}{y}\, 
G_{k+2j}(xy)\,dG_{r,r}(y)\,;\\
 J_{4}(x) &= \int\limits_{0}^\infty \fr{1}{y} \sum\limits_{j=0}^{2}
 a_j \times{}\\
 &\hspace*{-9mm}{}\times G_{k+2j}(xy)\,d\left[  
\fr{(y - 1)(2 - r) + 2 Q_1(g(n)y)}{2 r}\,g_{r,r}(y) \right]\!.
 \end{align*}

Тогда общий вид функции~\eqref{gn2n} из следствия~1 запишется в~виде:
\begin{multline*}
%\label{gnst}
F_{2;n}(x) = J_1(x) + \fr{1}{n}\,J_2(x) +{}\\
{}+ \fr{k}{4g(n)} \,J_3(x) 
+ \fr{k}{4ng(n)}\,J_4(x).
\end{multline*}

Рассмотрим интеграл~$J_1(x)$:
\begin{multline*}
\fr{\partial}{\partial x}\, J_1(x) =  \int\limits^{\infty}_{0} y  g_k(xy)  
g_{r,r}(y)\,dy = {}\\
{}=
\fr{r^r\,x^{k/2-1}}{\Gamma(r)\,\Gamma(k/2)2^{k/2}}
\int\limits^{\infty}_{0} y^{r+k/2-1}e^{-(r+x/2)y}dy\,.
\end{multline*}


Используя формулу~2.3.3.1 из~\cite[с.~259]{Prudnikov}
\begin{equation*}
\int\limits^{\infty}_{0} x^{\alpha-1}e^{-px}\,dx=
\Gamma(\alpha)\,p^{-\alpha}, \enskip
\alpha,\,p > 0\,,
\end{equation*}
c $\alpha=r+k/2$ и~$p\hm=r\hm+x/2,$ получаем
\begin{equation*}
\fr{\partial}{\partial x}\,J_1(x) = \fr{r^r}{B(k/2,r)2^{k/2}}\,
\fr{x^{k/2-1}}{(x/2+r)^{k/2+r}}\,, \enskip x>0\,.
\end{equation*}
Тогда
\begin{multline}
\label{int1}
J_1(x) = \int\limits_{0}^{x} \fr{r^r}{B(k/2,r)2^{k/2}}\,
\fr{t^{k/2-1}dt}{(t/2+r)^{k/2+r}}  ={}\\
{}= \fr{1}{B(k/2,r)(2r)^{k/2}}  
\int\limits_{0}^{x}\fr{t^{k/2-1}dt}{(t/(2r)+1)^{k/2+r}} = {}\\
{}= \left\{  y = \fr{t}{k}, \enskip dt = k\,dy  \right\} = 
F\left(\fr{x}{k};k,2r\right).
\end{multline}


Обозначим
\begin{align*}
 J_{3_j}(x) &= \int\limits^{\infty}_{0} \fr{1}{y}\,G_{k+2j}(xy)\,dG_{r,r}(y), \
  j = 0,1,2; \\
   J_3(x)&= \sum\limits_{j=0}^{2} a_j  J_{3_j}(x).
 \end{align*}

По аналогии с $({\partial}/{\partial x})J_{1}$ для $({\partial}/{\partial x})J_{3_0}$ получаем
\begin{multline*}
\fr{\partial}{\partial x}\,J_{3_0}(x) = {}\\
{}=
\fr{r \Gamma(k/2+r-1)}{\Gamma(k/2) 
\Gamma(r) (2r)^{k/2}}\,x^{k/2-1}\left(1+\fr{x}{2r}\right)^{-(k/2+r-1)}\!.
\end{multline*}

Тогда
\begin{multline*}
J_{3_0}(x) =\int\limits_{0}^{x} \fr{r \Gamma(k/2+r-1)}{\Gamma(k/2) 
(r-1)\Gamma(r-1) (2r)^{k/2}}\times{}\\
{}\times t^{k/2-1}\left(1+\fr{t}{2r}\right)^{-(k/2+r-1)}dt ={}\\
{}= \left\{  y = \fr{r-1}{rk}\,t, \enskip dt = \fr{kr}{r-1}\,dy  \right\} = {}\\
{}=
\fr{r}{r-1} F\left( \fr{r-1}{rk} \,x;k,2r-2\right).
\end{multline*}
Аналогично получаем
\begin{align*}
J_{3_1}(x)&=\fr{r}{r-1} F\left( \fr{r-1}{r(k+2)} x;k+2,2r-2\right);  \\
J_{3_2}(x)&=\fr{r}{r-1} F\left( \fr{r-1}{r(k+4)}x;k+4,2r-2\right)
\end{align*}
и, таким образом,
\begin{multline}
J_3(x)={}\\
\!\!{}=\fr{r}{r-1}  \sum\limits_{j=0}^{2} a_j  F\left( \fr{r-1}{r(k+2j)} \,x;k+2j,2r-2\right).\!\!
\label{int3}
\end{multline}


Для вычисления~$J_2(x)$ используем интегрирование по частям:
\begin{multline}
\label{j2}
J_2(x) =- x\times{}\\
{}\times 
\int\limits^{\infty}_{0}\! g_k(xy)  \fr{(y - 1)(2 - r) + 2 
Q_1(g(n)y)}{2 r}g_{r,r}(y)\, dy = {}\\
{}= - \fr{x(2-r)}{2r}\, J_{2_1}(x) +  \fr{x(2-r)}{2r}\, J_{2_2}(x) -{}\\
{}- \fr{x}{r}\,J_{2_3}(x),
 \end{multline}
 где
 \begin{align*}
 J_{2_1}(x) &= \int\limits^{\infty}_{0}y g_k(xy)g_{r,r}(y)\, dy; \\ 
J_{2_2}(x) &= \int\limits^{\infty}_{0} g_k(xy)g_{r,r}(y) \,dy; \\
J_{2_3}(x) &= \int\limits^{\infty}_{0} g_k(xy)g_{r,r}(y)Q_1(g(n)y) \,dy.
 \end{align*}


Заметим, что 
$$
J_{2_1}(x) = \fr{\partial}{\partial x}\, J_1(x);\enskip 
J_{2_2}(x) = \fr{\partial}{\partial x}\, J_{3_0}(x).
$$

Рассмотрим третье сла\-га\-емое~$J_{2_3}(x)$:
\begin{multline*}
J_{2_3}(x) = \fr{r^r\,x^{k/2-1}}{\Gamma(r)\Gamma(k/2)2^{k/2}}\times{}\\
{}\times
\int\limits^{\infty}_{0} y^{r+k/2-2}
e^{-(r+x/2)y}Q_1(g(n)y)\,dy\,.
\end{multline*}

Применяя технику из доказательства теоремы~2 из работы~\cite{CMU}, для~$J^*_4(x)$ 
получаем:
$$
n^{-1}\, \left\vert J_{2_3}\right\vert \leq \fr{c(r,k)}{n^{r}} \sum\limits^{\infty}_{k=1} k^{-r}=
\fr{c_1(r,k)}{n^r}\,.
$$

Подставив выражения для~$J_{2_1}(x)$ и~$J_{2_2}(x)$ в~\eqref{j2}, получаем
\begin{multline}
\label{int2}
J_2(x) = \fr{(r-2)x}{2rk} \left(f\left(\fr{x}{k};k,2r\right) -{}\right.\\
\left.{}- f\left( 
\fr{r-1}{rk}\,x;k,2r-2\right)\right).
 \end{multline}


Обозначим
\begin{align*}
 J_{4_j}(x) &= \int\limits^{\infty}_{0} \fr{1}{y}\,G_{k+2j}(xy)\,dh_2(y)\,, \enskip j = 0,1,2\,; 
\\
J_4(x)&= \sum\limits_{j=0}^{2} a_j  J_{4_j}(x)
\end{align*}
и рассмотрим~$J_{4_0}$. Используя интегрирование по частям, получаем
\begin{multline}
\label{j40}
 \hspace*{-3pt}J_{4_0}(x) = -\!\int\limits^{\infty}_{0} \!\!\left( -\fr{1}{y^2}\,G_{k}(xy) + 
\fr{x}{y}\,g_{k}(xy)\!\right) h_2(y)\,dy ={}\\
{}= \int\limits^{\infty}_{0} \fr{1}{y^2}\,G_{k}(xy) h_2(y)\,dy - x \int\limits^{\infty}_{0} 
\fr{1}{y}\,g_{k}(xy) h_2(y)\,dy ={}\\
{}= J^{\prime}(x) - x J^{\prime\prime}(x).
\end{multline}

Заметим, что $J^{\prime\prime}(x) = ({\partial}/{\partial x}) J^{\prime}(x)$, поэтому 
достаточно рассмотреть $J^{\prime\prime}(x)$:
\begin{multline*}
 J^{\prime\prime}(x) = \int\limits^{\infty}_{0}\fr{1}{y}\,g_k(xy) \times{}\\
 {}\times \fr{(y - 1)(2 - 
r) + 2 Q_1(g(n)y)}{2 r}\,g_{r,r}(y)\,dy +{} \\
{}+ \fr{1}{r}  \int\limits^{\infty}_{0} \fr{1}{y} \,g_k(xy) g_{r,r}(y) 
Q_1(g(n) y)\,dy =  {}\\
{}=
\fr{2-r}{2r}\,J_1^{\prime\prime}(x) - \fr{2-r}{2r}\,J_2^{\prime\prime}(x) + 
\fr{1}{r}\,J_3^{\prime\prime}(x).
\end{multline*}

Заметим, что 
$$
J_1^{\prime\prime}(x) = \fr{1}{k}\,f\left( \fr{r-1}{rk}\, x;k,2r-2\right)
$$
и~оценка для~$J_3^{\prime\prime}(x)$ строится аналогично оценке для~$J_{2_3}$, но 
с~$\alpha \hm= r\hm+k/2\hm-2$. Рассмотрим~$J_2^{\prime\prime}(x)$:
\begin{multline*}
J_2^{\prime\prime}(x) ={}\\
{}= \fr{r^r\,x^{k/2-1}}{\Gamma(r)\Gamma(k/2)2^{k/2}}\int\limits^{\infty}_{0} y^{r+k/2-3}
e^{-(r+x/2)y}\,dy ={} \\
{}= \left\{  x = \fr{rk}{r-2}t \right\} = \fr{r}{(r-1)k}\,f\!\left( \fr{r-2}{rk}\,x;k,2r-4\right)\!,\hspace*{-3.6227pt}
\end{multline*}
откуда получаем
\begin{multline}
\label{j401}
 J^{\prime\prime}(x) =  \fr{2-r}{2rk} \,f\left( \fr{r-1}{rk}\,x;k,2r-2\right) +{}\\
 {}+ \fr{r-2}{2k(r-1)}\,f\left( \fr{r-2}{rk}\, x;k,2r-4\right);
\end{multline}

\vspace*{-12pt}

\noindent
\begin{multline}
\label{j402}
 J^{\prime}(x) =  \fr{2-r}{2(r-1)}\, F\left( \fr{r-1}{rk}\, x;k,2r-2\right) + {}\\
 {}+
\fr{r}{2(r-1)}\,F\left( \fr{r-2}{rk} \,x;k,2r-4\right).
\end{multline}
Подставляя~\eqref{j401} и~\eqref{j402} в~\eqref{j40}, получаем
\begin{multline*}
 J_{4_0}(x) =  \fr{2-r}{2(r-1)}\, F\left( \fr{r-1}{rk}\, x;k,2r-2\right) + {}\\
 {}+
\fr{r}{2(r-1)}\,F\left( \fr{r-2}{rk}\, x;k,2r-4\right) -{} \\
{}-   \fr{(2-r)x}{2rk}\, f\left( \fr{r-1}{rk}\, x;k,2r-2\right) -{}\\
{}- \fr{(r-2)x}{2k(r-1)}\,f\left( \fr{r-2}{rk}\, x;k,2r-4\right).
\end{multline*}

Аналогично получаем, что
\begin{multline*}
 J_{4_1}(x) =  \fr{2-r}{2(r-1)}\,F\left( \fr{r-1}{r(k+2)}\,x;k+2,2r-2\right) 
+ {}\\
{}+\fr{r}{2(r-1)}\,F\left( \fr{r-2}{r(k+2)} \,x;k+2,2r-4\right) -{}\\
{}-   \fr{(2-r)x}{2r(k+2)} \,f\left( \fr{r-1}{r(k+2)}\, x;k+2,2r-2\right)
 -{}\\
 {}- \fr{(r-2)x}{2(k+2)(r-1)}\,f\left( \fr{r-2}{r(k+2)}\,x;k+2,2r-4\right);
 \end{multline*}
 
 \vspace*{-12pt}
 
 \noindent
 \begin{multline*}
 J_{4_2}(x) =  \fr{2-r}{2(r-1)}\, F\left( \fr{r-1}{r(k+4)}\,x;k+4,2r-2\right) 
+ {}\\
{}+\fr{r}{2(r-1)}\,F\left( \fr{r-2}{r(k+4)}\, x;k+4,2r-4\right) -{}\\
{}-   \fr{(2-r)x}{2r(k+4)} \,f\left( \fr{r-1}{r(k+4)} \,x;k+4,2r-2\right)
 -{}\\
 {}- \fr{(r-2)x}{2(k+4)(r-1)}\,f\left( \fr{r-2}{r(k+4)} \,x;k+4,2r-4\right).
\end{multline*}
Таким образом,
\begin{multline}
 J_{4}(x) ={}\\
 {}= \sum\limits_{j=0}^{2} a_j  \!\left[ \fr{2-r}{2(r-1)} \,F\!\left( \fr{r-
1}{r(k+2j)}\,x;k+2j,2r-2\!\right)+{} \right.{}\hspace*{-0.42361pt}\\
{}+
\fr{r}{2(r-1)}\,F\left( \fr{r-2}{r(k+2j)} x;k+2j,2r-4\right) - {}\\
{}- \fr{(2-r)x}{2r(k+2j)} \,f\left( \fr{r-1}{r(k+2j)}\, x;k+2j,2r-2\right) - 
{}\\
{}-  \fr{(r-2)x}{2(k+2j)(r-1)}\times{}\\
\left.{}\times f\left( \fr{r-2}{r(k+2j)}\,x;k+2j,2r-
4\right) \right].
\label{int4}
\end{multline}

Объединяя $|1/g(n) -1/(r n)| \hm\leq \max\{2 , r\} (r\hm-1) (rn)^{-2} $,~\eqref{int1}, 
\eqref{int2}, \eqref{int3}, \eqref{int4} и~лемму~1, получаем~доказательство оценки \eqref{eq10q}.

\section{Заключение} %\label{concl}

Доказанный в~данной работе аналог теоремы переноса позволяет обобщить результаты 
работ~\cite{MMU16,CMU} для случая, когда статистика имеет распределения типа 
Хотеллинга случайного размера. Полученное в~работе асимптотическое разложение 
типа Че\-бы\-шё\-ва--Эдж\-вор\-та для функции распределения вышеупомянутой статистики 
позволяет построить разложения типа Кор\-ни\-ша--Фи\-ше\-ра для данной статистики.

\smallskip 

Автор выражает благодарность В.\,В.~Ульянову и~Г.~Кристофу за полезные обсуждения 
задачи.



{\small\frenchspacing
{%\baselineskip=10.8pt
%\addcontentsline{toc}{section}{References}
\begin{thebibliography}{9}

\bibitem{BenKorGal13}
\Au{Бенинг В.\,Е., Галиева~Н.\,К., Королев~В.\,Ю.} Асимптотические 
разложения для функций распределения статистик, построенных по выборкам 
случайного объема~// Информатика и~её применения, 2013. Т.~7. Вып.~2. С.~75--83.

\bibitem{BenKorGal12}
\Au{Бенинг В.\,Е., Галиева~Н.\,К., Королев~В.\,Ю.} Оценки скорости 
сходимости для функций распределения асимптотически нормальных статистик, 
основанных на выборках случайного объема~// Вестник Тверского гос. 
ун-та. Сер.: Прикладная математика, 2012. Т.~17. С.~53--65.

\bibitem{MMU16}
\Au{Марков А.\,С., Монахов~М.\,М., Ульянов~В.\,В.} Разложения типа Кор\-ни\-ша--Фи\-ше\-ра 
для распределений статистик, построенных по выборкам случайного размера~//  
Информатика и~её применения, 2016. Т.~10. Вып.~2. С.~84--91.

\bibitem{CMU}
\Au{Кристоф Г., Монахов~М.\,М., Ульянов~В.\,В.} Разложения Че\-бы\-ше\-ва--Эдж\-вор\-та
 и~Кор\-ни\-ша--Фи\-ше\-ра второго порядка для распределений статистик, 
построенных по выборкам случайного размера~// Записки научных семинаров ПОМИ, 
2017. Т.~466. С.~167--207.

\bibitem{UAF2016}
\Au{Ulyanov V.\,V., Aoshima~M., Fujikoshi~Y.} Non-asymptotic results for 
Cornish--Fisher expansions~// J.~Math. Sci., 2016. Vol.~218. 
No.\,3. P.~363--368.

\bibitem{FUS05} %6
\Au{Fujikoshi Y., Ulyanov~V.\,V., Shimizu~R.} $L_1$-norm error bounds for 
asymptotic expansions of multivariate scale mixtures and their applications to 
Hotelling's generalized~$T_0^2$~// J.~Multivariate Anal., 2005. 
Vol.~96. P.~1--19.



\bibitem{Fus10} %7
\Au{Fujikoshi Y., Ulyanov~V.\,V., Shimizu~R.} Multivariate statistics: High-dimensional and 
large-sample approximations.~--- Wiley ser. in probability and 
statistics. -- Hoboken, NJ, USA: Wiley, 2010. 568~p.

\bibitem{EncStat} %8
\Au{Ulyanov V.\,V.} Cornish--Fisher expansions~// International encyclopedia 
of statistical science~/ Ed. M.~Lovric.~--- Berlin: Springer, 2011. P.~312--315.

\bibitem{Prudnikov}
\Au{Прудников А.\,П., Брычков~Ю.\,А., Маричев~О.\,И.} Интегралы и~ряды.~--- М.: Наука, 1981. 
Т.~1. 259~с.
 \end{thebibliography}

}
}

\end{multicols}

\vspace*{-3pt}

\hfill{\small\textit{Поступила в~редакцию 22.06.2020}}

\vspace*{8pt}

%\pagebreak

%\newpage

%\vspace*{-28pt}

\hrule

\vspace*{2pt}

\hrule

%\vspace*{-2pt}

\def\tit{CHEBYSHEV--EDGEWORTH EXPANSIONS FOR~DISTRIBUTIONS OF~GENERALISED HOTELLING-TYPE STATISTICS 
BASED~ON~RANDOM SIZE SAMPLES}


\def\titkol{Chebyshev--Edgeworth expansions for~distributions of~generalised hotelling-type statistics based 
on~random size samples}

\def\aut{M.\,M.~Monakhov}

\def\autkol{M.\,M.~Monakhov}


\titel{\tit}{\aut}{\autkol}{\titkol}

\vspace*{-11pt}


\noindent
Moscow Center for Fundamental and Applied Mathematics, M.\,V.~Lomonosov Moscow State University, 
\mbox{1-52}~Leninskie Gory, GSP-1, Moscow 119991, Russian Federation

 
\def\leftfootline{\small{\textbf{\thepage}
\hfill INFORMATIKA I EE PRIMENENIYA~--- INFORMATICS AND
APPLICATIONS\ \ \ 2021\ \ \ volume~15\ \ \ issue\ 2}
}%
\def\rightfootline{\small{INFORMATIKA I EE PRIMENENIYA~---
INFORMATICS AND APPLICATIONS\ \ \ 2021\ \ \ volume~15\ \ \ issue\ 2
\hfill \textbf{\thepage}}}

\vspace*{3pt}



\Abste{The general transfer theorem for the distribution function of asymptotically normal
 statistics was generalized on the Hotelling-type statistics case and analog of general 
 transfer theorem for the distribution function of Hotelling-type statistics with 
 random size was proved. It allowed to obtain the Chebyshev--Edgeworth expansion 
 for initial Hotelling-type statistics. The explicit form of the Chebyshev--Edgeworth 
 expansion was obtained for the case when the random sample size distribution is the negative 
 binomial distribution shifted by 1. The limit distribution for this case was F-distribution. 
 The Cornish--Fisher expansion was obtained for the special case of parameter of 
 random sample size. The computational experiment was conducted and graphs were plotted 
 for Chebyshev--Edgeworth expansion illustration.}
 
\KWE{generalised Chebyshev--Edgeworth expansion; Cornish--Fisher expansion; sample with random size; 
F-disribution; Hotelling-type statstics}



\DOI{10.14357/19922264210211}

%\vspace*{-15pt}

 \Ack
\noindent
The research was conducted in accordance with the program of the Moscow Center for Fundamental 
and Applied Mathematics.

%\vspace*{12pt}

  \begin{multicols}{2}

\renewcommand{\bibname}{\protect\rmfamily References}
%\renewcommand{\bibname}{\large\protect\rm References}

{\small\frenchspacing
 {%\baselineskip=10.8pt
 \addcontentsline{toc}{section}{References}
 \begin{thebibliography}{9}
\bibitem{1-m}
\Aue{Bening, V.\,E., N.\,K.~Galieva, and V.\,Yu.~Korolev.}
 2013. Asimptoticheskie razlozheniya dlya funktsiy raspredeleniya statistik, 
 postroennykh po vyborkam sluchaynogo ob''ema [Asymptotic expansions for the distribution functions 
 of statistics constructed from samples with random sizes]. 
 \textit{Informatika i~ee Primeneniya~--- Inform. Appl.} 7(2):75--83.
\bibitem{2-m}
\Aue{Bening, V.\,E., N.\,K.~Galieva, and V.\,Yu.~Korolev.}
 2012. Otsenki skorosti skhodimosti dlya funktsiy raspredeleniya asimptoticheski normal'nykh 
 statistik, osnovannykh na vyborkakh sluchaynogo ob''ema [On rate of convergence in distribution 
 of asymptotically normal statistics based on samples of random size]. 
 \textit{Vestnik Tverskogo gos. un-ta. Ser.\ Prikladnaya matematika} 
 [Bull. of the Tverskoy State University. Ser.\ Appl. Math.] 17:53--65.
\bibitem{3-m}
\Aue{Markov, A.\,S., M.\,M.~Monakhov, and V.\,V.~Ulyanov.}
 2016. Razlozheniya tipa Kornisha--Fishera dlya raspredeleniy statistik, 
 postroennykh po vyborkam sluchaynogo razmera [Generalized Cornish--Fisher expansions 
 for distributions of statistics based on samples of random size]. 
  \textit{Informatika i~ee Primeneniya~--- Inform. Appl.} 10(2):84--91.
\bibitem{4-m}
\Aue{Christoph, G., M.\,M.~Monakhov, and V.\,V.~Ulyanov.}
 2017. Razlozheniya Chebysheva--Edzhvorta i~Kornisha--Fishera vtorogo poryadka dlya raspredeleniy 
 statistik, postroyennykh po vyborkam sluchaynogo razmera [Second order Chebyshev--Edgeworth and 
 Cornish--Fisher expansions for distributions of statistics constructed from samples with random sizes]. 
  \textit{Zapiski nauchnykh seminarov POMI} [POMI Notes of Scientific Seminars] 466:167--207.
  {\looseness=1
  
  }
\bibitem{5-m}
\Aue{Ulyanov, V.\,V., M.~Aoshima, and Y.~Fujikoshi.}
 2016. Non-asymptotic results for Cornish--Fisher expansions. 
 \textit{J.~Math. Sci.} 218(3):363--368.
\bibitem{6-m}
\Aue{Fujikoshi, Y., V.\,V.~Ulyanov, and R.~Shimizu.}
 2005. $L_1$-norm error bounds for asymptotic expansions of multivariate scale mixtures and their 
 applications to Hotelling's generalized $T_0^2$.  \textit{J.~Multivariate Anal.} 96:1--19.

\bibitem{8-m}
\Aue{Fujikoshi, Y., V.\,V.~Ulyanov, and R.~Shimizu.}
 2010.  \textit{Multivariate statistics: High-dimensional and large-sample approximations}. 
 Wiley ser. in probability and statistics.
 Hoboken, NJ: Wiley. 568~p.
 
 \bibitem{7-m}
\Aue{Ulyanov, V.\,V.}
 2011.  Cornish--Fisher expansions.  \textit{International encyclopedia of statistical science}. 
 Ed. M.~Lovric. Berlin: Springer. 312--315.
 
\bibitem{9-m}
\Aue{Prudnikov, A.\,P., Yu.\,A.~Brychkov, and O.\,I.~Marichev.}
 1992.  \textit{Integraly i~ryady} 
 [Integrals and series]. Moscow: Nauka. Vol.~1. 259~p.
 \end{thebibliography}

 }
 }

\end{multicols}

\vspace*{-3pt}

  \hfill{\small\textit{Received June~22, 2020}}


%\pagebreak

%\vspace*{-8pt}  

\Contrl

\noindent
\textbf{Monakhov Mikhail M.} (b.\ 1993)~--- 
laboratory assistant, Moscow Center for Fundamental and Applied Mathematics, 
M.\,V.~Lomonosov Moscow State University, 1-52~Leninskie Gory, GSP-1, Moscow 119991, Russian Federation; 
\mbox{mih\_monah@mail.ru}


\label{end\stat}

\renewcommand{\bibname}{\protect\rm Литература}  %11
%\def\lt{\;\hbox{$<$}\discretionary{}{\hbox{$<$}}{}\;}
%\def\gt{\;\hbox{$>$}\discretionary{}{\hbox{$>$}}{}\;}
%\def\eq{\;\hbox{$=$}\discretionary{}{\hbox{$=$}}{}\;}
%\def\le{\;\hbox{$\leqslant$}\discretionary{}{\hbox{$\leqslant$}}{}\;}
%\let\leq=\leqslant
%\def\ge{\;\hbox{$\geqslant$}\discretionary{}{\hbox{$\geqslant$}}{}\;}
%\let\geq=\geqslant
%\def\equ{\;\hbox{$\equiv$}\discretionary{}{\hbox{$\equiv$}}{}\;}
%\def\pls{\;\hbox{$+$}\discretionary{}{\hbox{$+$}}{}\;}

\def\stat{suchkov}

\def\tit{АЛГОРИТМЫ СЖАТИЯ ДАННЫХ МАССИВОВ СИЛОВЫХ КРИВЫХ~I: КОДИРОВАНИЕ ОШИБОК ПРЕДСКАЗАНИЯ}

\def\titkol{Алгоритмы сжатия данных массивов силовых кривых~I: кодирование ошибок предсказания}

\def\aut{Д.\,В.\,Сушко$^1$}

\def\autkol{Д.\,В.\,Сушко}

\titel{\tit}{\aut}{\autkol}{\titkol}

\index{Сушко Д.\,В.}
\index{Sushko D.\,V.}

%{\renewcommand{\thefootnote}{\fnsymbol{footnote}} \footnotetext[1]
%{Работа выполнена при частичной поддержке РФФИ (проект 19-07-00187 А) 
%и~в~соответствии с~программой Московского центра фундаментальной и~прикладной математики.}}


\renewcommand{\thefootnote}{\arabic{footnote}}
\footnotetext[1]{Институт проблем информатики Федерального исследовательского центра 
<<Информатика и~управление>> Российской академии наук, \mbox{dsushko@ipiran.ru}}


\vspace*{-6pt}



\Abst{Рассмотрена задача обратимого (без потерь) сжатия данных массивов силовых кривых~--- 
трехмерных массивов, элементы которых суть 16-бит\-ные целые числа. 
Такие массивы являются результатом сканирования микрообъектов на атом\-но-си\-ло\-вом 
микроскопе (АСМ) в~режиме измерения силовых карт. Предложены алгоритмы обратимого сжатия массивов 
силовых кривых, основанные на универсальном арифметическом кодировании ошибок их предсказания. 
Применены два метода универсального кодирования. Первый основан на использовании статистической 
модели источника с~вычислимой последовательностью состояний и~предполагает разложение всей 
последовательности ошибок предсказания на две независимо кодируемые подпоследовательности. 
Второй предполагает выбор подходящего веса при построении используемых в~арифметическом кодировании 
кодовых вероятностей. Для предложенных алгоритмов на пяти тестовых массивах построены оценки 
скорости кодирования. Результаты показывают, что использование комбинации упомянутых выше
 методов универсального кодирования позволяет заметно уменьшить скорость кодирования. 
 Скорости кодирования тестовых массивов наиболее эффективным алгоритмом среди предложенных 
 практически применимых алгоритмов составили 3,9285, 3,5268, 3,5024, 4,2813 и~4,2246~бит/пик\-сель.}

\KW{атомно-силовой микроскоп; массив силовых кривых; обратимое сжатие; арифметическое кодирование; 
универсальное кодирование}

\DOI{10.14357/19922264210212}

\vspace*{4pt}


\vskip 10pt plus 9pt minus 6pt

\thispagestyle{headings}

\begin{multicols}{2}

\label{st\stat}

\section{Введение}

\vspace*{-4pt}

%\label{sec1}
В настоящее время широко распространенным методом исследования микрообъектов (например, клеток,
 вирусов, белков, нуклеиновых кислот и~др.\ в~микробиологии) стало их сканирование на 
АСМ в~режиме измерения силовых карт. Результатом таких 
 исследований являются массивы силовых кривых~-- трехмерные массивы данных большого объема 
 (в~типичном случае $\sim70$~МБ). Необходимость долгосрочного хранения и~передачи по каналам 
 связи таких данных делает задачу их сжатия весьма актуальной, а поскольку получение данных 
 связано с~проведением трудоемкого и~длительного эксперимента, потери при сжатии недопустимы, т.\,е.\
  сжатие должно быть обратимым (без потерь).

Разработка алгоритмов обратимого сжатия данных массивов силовых кривых представляет 
интерес с~теоретической точки зрения, так как распространенные алгоритмы обратимого 
сжатия \mbox{ориентированы} главным образом на сжатие данных одного из двух типов: текст 
или изображение, а~массивы силовых кривых, очевидным образом, не относятся ни к~одному из этих типов.

Исследование задачи обратимого сжатия данных массивов силовых кривых было начато в~работе~\cite{b1}, 
где были определены потенциальные возможности некоторых алгоритмов сжатия. 
В~качестве экспериментальных данных в~работе были\linebreak использованы массивы, полученные при 
сканировании мягких биологических образцов в~режиме измерения силовых карт на микроскопе MultiMode~V 
(Veeco, США). В~число рассмотренных\linebreak алгоритмов вошли стандартные алгоритмы (\mbox{DEFLATE}, JPEG~2000) 
и~предложенные в~\cite{b1} простые алгоритмы на основе арифметического кодирования. 
Были получены оценки скорости кодирования этих алгоритмов. Напомним, что 
\textit{ско\-ростью кодирования}~$R$ алгоритма называется отношение длины кодового слова~$L$ 
(в~битах), порождаемого алгоритмом для описания массива данных, к~полному числу элементов 
(пикселей)~$N$ этого массива; единица измерения скорости кодирования~-- бит/пик\-сель (бт/п). 
При этом коэффициент сжатия равен отношению длины элемента массива в~битах к~ско\-рости кодирования.

Цель настоящей работы~--- предложить и~исследовать более сложные алгоритмы обратимого сжатия, 
основанные на универсальном арифметическом кодировании ошибок предсказания. 
Скорость кодирования алгоритмов оценивается на тех же экспериментальных данных, что и~в~работе~\cite{b1}, 
что позволяет сравнивать получаемые результаты непосредственно. Вычисления проводятся 
программами, написанными на языке Python.
{\looseness=1

}

\section{Массивы силовых кривых}

%\label{sec2}
Кратко рассмотрим вопросы, связанные с~процессом сканирования и~структурой массивов 
силовых кривых. Более подробное изложение приведено в~\cite{b1}, детальное описание 
технологии \mbox{измерений} с~по\-мощью АСМ и~интерпретации со\-от\-вет\-ст\-ву\-ющих данных можно найти, 
например, в~\cite{b2}.

Принцип работы АСМ заключается в~сканировании поверхности образца атомарно острой иглой 
(зондом), которая является частью гибкого кронштейна (кантилевера), закрепленного на 
пьезоэлектрическом двигателе. Силы, действующие на зонд со стороны поверхности, 
вызывают изгиб кантилевера, что приводит к~перераспределению лазерного сигнала на 
фотодетекторе. Регистрируя величину этого сигнала и~зная жесткость кантилевера, 
можно определить силу взаимодействия зонда с~поверхностью.

Современные АСМ позволяют работать в~режиме измерения силовых кривых и~силовых карт. 
Силовая кривая представляет собой график зависимости силы взаимодействия зонда и~поверхности 
образца от расстояния между ними. При фиксированном положении зонда в~плоскости образца 
снимаются две кривые: кривая подвода (зонд приближается к~образцу) и~кривая отвода 
(зонд удаляется от образца). Пары силовых кривых снимаются для множества точек в~поле наблюдения. 
В~результате формируется карта силовых кривых~--- трехмерный массив данных.

Введем некоторые обозначения. Пусть $OXYZ$~--- трехмерная декартова система координат. 
Твердая подложка образца располагается в~горизонтальной плоскости~$OXY$, 
поле наблюдения представляет собой прямоугольник в~этой плоскости. Силовые кривые измеряются 
в~узлах $(x,y)\hm\equiv (x_i,y_j)$ равномерной прямоугольной решетки в~поле наблюдения, 
$i\hm=0,1,\ldots,I-1$, $j\hm=0,1,\ldots,J-1$. Перебор узлов решетки осуществляется 
в~следующем порядке: сначала по ширине (в~на\-прав\-ле\-нии~$OY$), затем по длине (в~на\-прав\-ле\-нии~$OX$). 
В~каждом узле $(x,y)$ решетки измеряется пара силовых кривых 
$F^\text{A}_{(x,y)}(z)$ (кривая подвода) и~$F^\text{R}_{(x,y)}(z)$ 
(кривая отвода), $z\hm\equiv z_k$, $k\hm=0,1,\ldots,K-1$. 
Шаг по вертикали равномерный, высота~$z$ отсчитывается от поверхности образца в~узле $(x,y)$.

Значения элементов силовых кривых $F^\text{A,R}_{(x,y)}(z)$ пропорциональны силе, 
действующей на зонд в~точке пространства, находящейся на расстоянии~$z$ 
от поверхности образца и~имеющей координаты $(x,y)$ в~поле наблюдения, в~процессе 
подвода зонда к~образцу и~отвода зонда от образца. Значения~$F$ записываются в~виде 16-бит\-ных 
целых чисел, т.\,е.\ целых чисел в~диапазоне $[-2^{15},2^{15}-1]$. Увеличение значения~$F$ 
отвечает увеличению отталкивания (уменьшению притяжения) между зондом и~поверхностью. 
При измерении кривой подвода $F^\text{A}_{(x,y)}(z)$ зонд может достичь поверхности 
образца до того, как будет зарегистрировано необходимое число ($K$) значений.
 В~таком случае осуществляется дополнение соответствующей строки до требуемой длины ($K$) 
 минимально возможным значением~$-2^{15}$, записываемым в~конец строки.

Для всего массива силовых кривых будем использовать обозначение 
\begin{multline*}
\mathbf{V}=\{V(i,j,k)\},\enskip i\hm= 0,1,\dots,I-1\,,\\
j=0,1,\dots,J-1\,,\enskip
 k=0,1,\dots,2K-1\,. 
 \end{multline*}
При этом
\begin{multline*}
V(i,j,k) ={}\\
{}= \begin{cases}
F^\text{A}_{(x_i,y_j)}(z_k)\,,        & k=0,1,\ldots,K-1\,; \\[6pt] 
F^\text{R}_{(x_i,y_j)}(z_{k-K})\,, & k=K,K+1,\ldots,2K-1\,.\hspace*{-3pt} 
\end{cases}
\end{multline*}

В качестве экспериментальных данных используются пять массивов силовых кривых (I--V), 
полученных при сканировании образцов, пред\-став-ляющих собой абсорбированные из раствора 
на\linebreak твердую подложку вирусы. 

Первый образец~(I)~--- это риновирус~2 на подложке из слюды, 
второй и~третий образцы (II и~III)~--- вирус мягкой мозаики ячменя на подложке из слюды, 
четвертый и~пятый образцы (IV и~V)~--- вирус табачной мозаики на подложке из стекла. 
Массивы силовых кривых имеют сле\-ду\-ющие размеры: $I\hm=J\hm=64$, $K\hm=4096$ (массив~I);\linebreak  
$I\hm=J\hm=128$, $K\hm=1024$ (массивы~II--V). Полное число элементов всех массивов равно~$2^{25}$. 
Каж-\linebreak дый элемент представляет собой 16-бит\-ное целое\linebreak число.
{ %\looseness=1

}


В качестве иллюстрации на рисунке представлены пары силовых кривых образца~II 
в~двух разных узлах поля наблюдения. Кривые подвода изображены черным цветом, 
кривые отвода~--- серым. На фрейме~(\textit{б}) 
представлена кривая подвода, при измерении которой зонд достиг поверхности образца до того, 
как было зарегистрировано необходимое число ($K\hm=1024$) значений, и~поэтому в~конец строки 
было записано нужное число минимальных значений~$(-2^{15})$.

\begin{figure*}
  \vspace*{1pt}
  \begin{center}
    \mbox{%
 \epsfxsize=161.251mm 
 \epsfbox{suc-1.eps}
 }


\vspace*{6pt}

{\small Силовые кривые образца~II}
\end{center}
\end{figure*} 


\section{Алгоритмы кодирования ошибок предсказания}
%\label{sec3}

Общепринятый метод решения задач обратимого сжатия цифровых данных заключается в~применении к~исходному 
массиву данных некоторых обратимых преобразований, обеспечивающих декорреляцию его 
отсчетов и/или уменьшение диапазона их значений, и~последующего арифметического кодирования~\cite{b3} 
полученного таким образом массива как последовательности независимых отсчетов.

При арифметическом кодировании конечной числовой последовательности $\mathbf{x}\hm=
\{x_n\}$, $n\hm= 0,1,\ldots,N-1$, принимающей значения в~априори известном диапазоне~$\mathfrak{A}$, 
множество \textit{условных кодовых распределений вероятностей} (или просто \textit{кодовых распределений})
$\{q_n(a)\hm= q_n(a\vert x_{n-1},\dots,x_0)$, $a\hm\in\mathfrak{A} \}$
используется для того, чтобы приписать последовательности~$\mathbf{x}$  кодовую 
вероятность~$Q(\mathbf{x})$ и~кодовое слово (результат сжатия) длины
\begin{multline}
\label{eq01}
L(\mathbf{x}) = \left\lceil -\log_2 \left(\fr{Q(\mathbf{x})}{2}\right)  \right\rceil ={}\\
{}=
\left\lceil -\log_2 \prod\limits_{n=0}^{N-1}q_n(x_n)  + 1 \right\rceil \leq{}\\
{}\leq
\sum\limits_{n=0}^{N-1} -\log_2 q_n(x_n) + 2\,.
\end{multline}
Здесь $\lceil\cdot\rceil$~--- результат округления вещественного чис\-ла до 
ближайшего целого вверх. При этом построение кодовых распределений, 
обес\-пе\-чи\-ва\-ющих получение возможно более коротких кодовых слов при 
неизвестной статистике,~--- задача универсального кодирования~\cite{b4}. Отметим, 
что вос\-ста\-нов\-ле\-ние исходной последовательности по кодовому слову осуществляется в~процессе 
декодирования без задержки. Это означает, что в~момент восстановления очередного значения~$x_n$ 
декодеру уже известны все предыдущие значения $\{x_0,\dots,x_{n-1}\}$ и~кодовые распределения 
могут быть построены декодером так же, как они были ранее построены кодером в~процессе кодирования. 
Это позволяет декодеру восстановить значение~$x_n$. 

В настоящей работе кодовые распределения строятся следующим образом:
\begin{multline}
\label{eq02}
q_n(a\vert x_{n-1},\dots,x_0) ={}\\
{}= \begin{cases}
 \fr{1+w\theta_n(a)}{a_{\max}(\mathbf{x})-a_{\min}(\mathbf{x})+1+wn} &
\mbox{ при } a\in{}\\
& \hspace*{-55pt}\in[a_{\min}(\mathbf{x}), a_{\max}(\mathbf{x})]\,; \\
0& \hspace*{-55pt}\mbox{в~противном случае}\,,
\end{cases}
\end{multline}
где $\theta_n(a)$~--- число элементов, принимающих значение~$a$, на
 начальном участке по\-сле\-до\-ва\-тель\-ности~$\mathbf{x}$ до $(n-1)$-го члена включительно; 
 $a_{\min}(\mathbf{x})$ и~$a_{\max}(\mathbf{x})$~--- нижняя и~верхняя границы диапазона 
 значений последовательности~$\mathbf{x}$; параметр $w\hm=1,2,\dots$~--- вес. В~частном случае 
 $w\hm=2$ формула~(\ref{eq02}) превращается в~соответствующую формулу работы~\cite{b1}. 
 Границы  $a_{\min}(\mathbf{x})$ и~$a_{\max}(\mathbf{x})$ могут быть вычислены
  кодером и~должны быть переданы декодеру помимо кодового слова, что, 
  вообще говоря, несколько увеличивает скорость кодирования. 
  Однако кодируемые в~работе массивы имеют~$2^{25}$~элементов, а для передачи значения 
  одной границы требуется $2^5$~бит; соответствующее увеличение скорости кодирования 
  составляет~$2^{-20}$~бт/п~--- величину, которой можно пренебречь.

В настоящей работе скорость кодирования оценивается по формуле
\begin{equation}
\label{eq03}
R(\mathbf{x}) \doteq \fr{1}{N}\,L(\mathbf{x}) =
\fr{1}{N} \sum\limits_{n=0}^{N-1}-\log_2 q_n(x_n)\,,
\end{equation}
вытекающей из (\ref{eq01}), если пренебречь бесконечно малым членом~$2N^{-1}$.

Величины $\theta(x)/N$, $x\hm\in\mathfrak{A}$, где $\theta(x)$~--- число элементов,
 принимающих значение~$x$, в~последовательности~$\mathbf{x}$, образуют \textit{частотное} (или 
 \textit{эмпирическое}) распределение вероятностей значений последовательности. Величина
\begin{equation}
\label{eq04}
H(\mathbf{x}) =  \sum\limits_{x\in\mathfrak{A}}
\fr{\theta(x)}{N} \left[ -\log_2 \fr{\theta(x)}{N}\right]
\end{equation}
(используется соглашение о том, что $0\cdot\log_20\hm=0$) называется 
\textit{квазиэнтропией} по\-сле\-до\-ва\-тель\-ности. Единица измерения квазиэнтропии~--- 
бт/п. Квазиэнтропия зависит только от самой последовательности и~представляет 
собой нижнюю границу скорости арифметического кодирования (см., например,~\cite{b4}). 
Разность $R\hm-H\hm\ge 0$~--- \textit{избыточность} арифметического кодирования. 
Величина избыточности характеризует качество решения одной из задач универсального кодирования~--- 
задачи построения кодовых распределений.

Арифметическое кодирование многомерного массива данных требует их одномерного 
упорядочения. В~работе принято так называемое \mbox{строчное} упорядочение, при котором 
трехмерный массив~$\mathbf{X}$ размерами $(I,J,K)$ превращается в~последовательность~$\mathbf{x}$   
размером $N\hm= I J K$ так, что $x(n)\hm=X(i,j,k)$, $n\hm=J K i \hm+ K j +k$.

В работе исследуются алгоритмы, ис\-поль\-зу\-ющие в~качестве основного декорреляционного 
преобразования переход к~ошибкам предсказания и~два предваряющих его преобразования, 
которые увеличивают неравномерность распределения и~уменьшают диапазон значений данных.

Первое преобразование заключается в~разделении массива силовых кривых~$\mathbf{V}$ 
на массивы кривых подвода~$\mathbf{V}^{\text A}$ и~отвода~$\mathbf{V}^{\text R}$:  
$\mathbf{V}\hm\to\{\mathbf{V}^{\text A},\mathbf{V}^{\text R}\}$,
\begin{multline}
V^{\text A}(i,j,k) = V(i,j,k), \\
V^{\text R}(i,j,k) = V(i,j,K+k), \\
k = 0,1,\ldots,K-1\,.
\label{eq05}
\end{multline}

Второе преобразование применяется к~полученному массиву кривых 
подвода~$\mathbf{V}^{\text A}$ и~заключается в~сужении диапазона
 значений этого массива, непомерно широкого из-за наличия значения~$-2^{15}$, 
 используемого для дополнения <<неполных>> строк:  $\mathbf{V}^{\text A}\hm\to  \bar{\mathbf{V}}^{\text A}$,
\begin{equation}
\label{eq06}
\bar{V}^{\text A} = \begin{cases}
 V^{\text A}_{\text d\min} - 1, &\ V^{\text A}=-2^{15}\,;\\
 V^{\text A}, &\ V^{\text A}>-2^{15}\,,
\end{cases}
\end{equation}
где $V^{\text A}_{\text d\min}$~--- динамический минимум значений массива~$\mathbf{V}^\text{A}$ 
(т.\,е.\ минимум значений без учета значения~$-2^{15}$). Чтобы обратить преобразование 
сужения диапазона~(\ref{eq06}), декодеру нужна информация о~том, встречалось 
ли значение~$-2^{15}$ в~исходном массиве кривых подвода. Для передачи декодеру этой известной 
кодеру информации требуется один бит, и~соответствующее увеличение скорости
 кодирования пренебрежимо мало.

Переход к~ошибкам предсказания $\mathbf{X}\to\mathbf{D}$ 
для данного трехмерного массива $\mathbf{X}\hm=\{X(i,j,k)\}$ размерами $(I,J,K)$ имеет вид:
\begin{multline}
D(i,j,k) ={}\\
\!\!{}=
\begin{cases}
X(0,0,0),                          & i=j=k=0; \\ 
X(i,0,0)-X(i-1,0,0), & i=1,\dots,I-1,\\
& j=k=0; \\
X(i,j,0)-X(i,j-1,0),  & i=1,\dots,I-1,\\
& \hspace*{-39pt}j=1,\dots,J-1,\ k=0; \\
X(i,j,k)-X(i,j,k-1),  & i=1,\dots,I-1,\\
& \hspace*{-85pt}j=1,\dots,J-1,\ k=1,\dots,K-1.
\end{cases}
\!\!\label{eq07}
\end{multline}
Преобразование применяется к~каждому из двух полученных в~результате предварительной 
обработки массивов: $\bar{\mathbf{V}}^\text{A}\to\mathbf{D}^\text{A}$; 
$\mathbf{V}^\text{R}\to\mathbf{D}^\text{R}$.

Все исследуемые в~работе алгоритмы используют преобразования (\ref{eq05})--(\ref{eq07}) и~различаются 
применяемыми методами универсального кодирования. Отметим, что использование этих 
преобразований обеспечило наименьшую скорость кодирования массивов силовых кривых 
среди рассмотренных в~\cite{b1} алгоритмов, основанных на кодировании ошибок предсказания.

Начнем с~рассмотренного в~\cite{b1} алгоритма A[1$\vert$0], который осуществляет 
независимое арифметическое кодирование массивов $\mathbf{D}^\text{A}$ и~$\mathbf{D}^\text{R}$ 
с~использованием построенных по формуле~(\ref{eq02}) с~весом $w\hm=2$ кодовых распределений. 
В~строке~1 табл.~\ref{tab1} приведена квазиэнтропия
$$
H[1]\doteq H(\{\mathbf{D}^\text{A}, \mathbf{D}^\text{R} \}) = 
\fr{1}{2}\left[H(\mathbf{D}^\text{A}) + H(\mathbf{D}^\text{A})\right]
$$
ошибок предсказания для массивов~I--V, а~в~строке~4~--- ско\-рость кодирования
$$
R[1\vert 0]\doteq R(\{\mathbf{D}^\text{A}, \mathbf{D}^\text{R} \}) = 
\fr{1}{2}\left[R(\mathbf{D}^\text{A}) + R(\mathbf{D}^\text{A})\right]
$$
алгоритма. Значения квазиэнтропии и~скорости кодирования в~табл.~1 
приводятся в~единицах бт/п с~точностью до четырех знаков после десятичной запятой. 
Избыточность кодирования алгоритма составляет 0,0009--0,0022~бт/п в~зависимости от массива.


\begin{table*}\small %[t]
\begin{center}
\Caption{Квазиэнтропия и~скорость кодирования}
\label{tab1}
\vspace{2ex}

\begin{tabular}{|c|l|c|c|c|c|c|}
\hline
№   & \multicolumn{1}{c|}{Величина}  & I          & II         & III        &   IV      & V \\
\hline
1   & $H$[1]      & 3,9496    & 3,6922    & 3,6945    & 4,3091    & 4,2454  \\
2   & $H$[2,opt]                    & 3,9256  & 3,5261  & 3,5015    & 4,2806    & 4,2239 \\
3   & $H$[2,fix]                            & 3,9281    & 3,5261    & 3,5015    & 4,2806    & 4,2239 \\
\hline
4   & $R[1\vert 0]$  & 3,9504    & 3,6944    & 3,6959    & 4,3103    & 4,2463 \\
5   & $R[1\vert \mathrm{opt}]$                    & 3,9498    & 3,6927    & 3,6951    & 4,3096    & 4,2458 \\
6   & $R[1\vert \mathrm{fix}]$                            & 3,9498    & 3,6928    & 3,6951    & 4,3096    & 4,2458 \\
7   & $R[2,\mathrm{fix}\vert 0] $                     & 3,9294    & 3,5288    & 3,5038    & 4,2823    & 4,2256 \\
8   & $R[2,\mathrm{fix}\vert \mathrm{opt}]$                & 3,9285    & 3,5268 & 3,5024   & 4,2813    & 4,2246 \\
9   & $R[2,\mathrm{fix}\vert \mathrm{fix}]$                    & 3,9285    & 3,5268 & 3,5024   & 4,2813    & 4,2246 \\
\hline
\end{tabular}
\end{center}
\end{table*}

Рассмотрим метод сжатия, основанный на использовании статистической модели 
\textit{источника с~вычислимой последовательностью состояний} (см., например,~\cite{b4}). 
Модель предполагает, что элементы последовательности~$\mathbf{x}$ 
(в~данном случае одномерно упорядоченные ошибки предсказания $\mathbf{D}^\text{A}$ 
и~$\mathbf{D}^\text{R}$)\linebreak один за другим <<порождаются>> некоторым источником данных и~распределение 
значений очередного элемента~$x_n$ зависит только от текущего со\-сто\-яния источника, 
которое, в~свою очередь,\linebreak \mbox{определяется} значением предыдущего элемента~$x_{n-1}$  
последовательности. Назовем элемент последовательности, <<порожденный>> источником в~некотором 
состоянии, элементом этого состояния. Элементы каждого состояния будем ко\-ди\-ро\-вать/де\-ко\-ди\-ро\-вать независимо. 
Это возможно, поскольку в~момент обработки данного элемента значение предыдущего элемента известно не только 
кодеру, но и~декодеру (декодирование осуществляется без задержки). Эффективность кодирования 
зависит от выбора множества состояний.

Определим способ построения состояний в~стиле~\cite{b5}. Выберем некоторое натуральное число~$t$~--- 
порог. Отнесем к~нулевому состоянию те элементы~$x_n$, для которых $n\hm\geq 1$ и~$|x_{n-1}|\hm<t$, 
прочие элементы отнесем к~первому состоянию. Теперь последовательность~$\mathbf{x}$  
разложена на две подпоследовательности элементов нулевого и~первого состояний $\mathbf{x}_0$ 
и~$\mathbf{x}_1$. Квазиэнтропия и~скорость кодирования пары подпоследовательностей равны
\begin{multline}
H(\{ \mathbf{x}_0, \mathbf{x}_1 \}) = 
\fr{N_0}{N}H(\mathbf{x}_0) + \fr{N_1}{N}H(\mathbf{x}_1); \\
R(\{ \mathbf{x}_0, \mathbf{x}_1 \}) = 
\fr{N_0}{N}R(\mathbf{x}_0) + \fr{N_1}{N}R(\mathbf{x}_1), 
\label{eq08}
\end{multline}
где $N_0$ и~$N_1$~--- число элементов нулевого и~первого состояний, а квазиэнтропия
 и~ско\-рость кодирования каждой отдельной подпоследовательности $\mathbf{x}_0$ 
и~$\mathbf{x}_1$ даются формулами~(\ref{eq04}) и~(\ref{eq03}).

Результирующие квазиэнтропия и~скорость кодирования~(\ref{eq08}) зависят 
от выбора порога~$t$, определяющего состояния. При любом значении порога квазиэнтропия~(\ref{eq08}) 
не превышает квазиэнтропии всей последовательности~$\mathbf{x}$~(\ref{eq04}) 
(см., например,~\cite{b4}). Естественная оптимизационная задача~--- 
нахождение порога, при котором квазиэнтропия~(\ref{eq08}) принимает минимальное значение,~--- 
может быть решена для конкретных данных численно путем перебора.

Применим описанный метод разложения данных на два состояния к~ошибкам 
предсказания   $\mathbf{D}^\text{A}$ и~$\mathbf{D}^\text{R}$. 
Оптимальные пороги в~обоих случаях принимают одинаковые значения, равные~3, 4, 4, 3, 4 
для массивов I,\dots,V. Значения оптимальной квазиэнтропии
$H[2,\text{opt}] \doteq H(\{ 
\{{\mathbf{D}^\text{A} }_0,{\mathbf{D}^\text{A} }_1,\},
\{{\mathbf{D}^\text{R} }_0,{\mathbf{D}^\text{R} }_1,\}
\})$
для массивов~I--V представлены в~строке~2 табл.~\ref{tab1}. 
Среднее по массивам уменьшение по сравнению с~квазиэнтропией~$H$[1] составляет 0,0866~бт/п, 
минимальное~--- 0,0215~бт/п (массив~V), максимальное~--- 0,1930~бт/п (массив~III).

Нахождение оптимальных порогов для данного массива силовых кривых требует значительного 
времени счета и~не может быть реализовано на этапе кодирования в~режиме реального времени. 
Поэтому для построения состояний в~алгоритме, предназначенном для практического применения, 
следует использовать общие фиксированные значения порогов для всей совокупности подлежащих 
сжатию массивов. В~строке~3 табл.~\ref{tab1} представлены значения квазиэнтропии $H$[2,fix], 
отвечающие построенным с~порогами $t\hm=4$ двум состояниям ошибок предсказания $\mathbf{D}^\text{A}$ 
и~$\mathbf{D}^\text{R}$, для массивов~I--V. Увеличение квазиэнтропии по сравнению 
с~оптимальным значением $H$[2,opt] составляет 0,0025~бт/п для массива~I, пренебрежимо мало 
для массива~IV\linebreak и~равно нулю для остальных массивов. Таким об\-разом, использование 
общих фиксированных поро\-гов для построения состояний не приводит к~значительному 
увеличению квазиэнтропии по сравнению с~оптимальными значениями.

Скорости кодирования
$R[2,\text{fix}|0] \hm\doteq  R(\{ 
\{{\mathbf{D}^\text{A} }_0,{\mathbf{D}^\text{A} }_1,\},
\{{\mathbf{D}^\text{R} }_0,{\mathbf{D}^\text{R} }_1,\}
\})$
массивов~I--V алгоритмом A[2,fix$\vert$0], который независимо кодирует элементы состояний ошибок 
предсказания~$\mathbf{D}^\text{A}$ и~$\mathbf{D}^\text{R}$, построенных с~фиксированными порогами, 
принимающими значение $t\hm=4$, и~использует формулу~(\ref{eq02}) с~весом $w\hm=2$ для 
построения кодовых распределений, представлены в~строке~7 табл.~\ref{tab1}.\linebreak  
Для всех массивов имеет место снижение скорости кодирования по сравнению с~алгоритмом~A[1$\vert$0], 
которое лишь немного меньше, чем уменьшение \mbox{квазиэнтропии} $H$[2,fix] по 
сравнению с~квазиэнтропией~$H$[1]. Среднее по массивам снижение составляет~0,0855~бт/п, минимальное~--- 
0,0207~бт/п (массив~V), максимальное~--- 0,1921~бт/п (массив~III). Избыточность 
кодирования алгоритма составляет 0,0012--0,0026~бт/п и~в~1,2--1,8~раза больше избыточности алгоритма~A[1$\vert$0].

Рассмотрим метод уменьшения избыточности кодирования, 
основанный на выборе веса~$w$ в~формуле~(\ref{eq02}) для кодовых распределений. 
Отметим, что снижение скорости кодирования при таком подходе заведомо не превысит 
избыточности кодирования с~принятым по умолчанию значением веса $w\hm=2$. Оптимизационная 
задача нахождения веса~$w$, при котором для конкретной последовательности 
данных минимальна скорость кодирования~(\ref{eq03}), может быть решена численно 
путем перебора. В~колонках табл.~2, обозначенных I,\dots,V, представлены значения 
оптимальных весов~$w$ для ошибок предсказания $\mathbf{D}^\text{A}$ 
и~$\mathbf{D}^\text{R}$ каждого из массивов~I--V и~элементов состояний 
${\mathbf{D}^\text{A}}_0$, ${\mathbf{D}^\text{A}}_1$,
${\mathbf{D}^\text{R}}_0$ и~${\mathbf{D}^\text{R}}_1$
этих ошибок; состояния построены с~фиксированными порогами,  принимающими значение $t\hm=4$ (см.\ выше).


Обозначим через A[1$\vert$opt] алгоритм независимого кодирования ошибок предсказания~$\mathbf{D}^\text{A}$
и~$\mathbf{D}^\text{R}$\linebreak с~использованием оптимальных весов для по\-строения 
 кодовых распределений. Скорости кодирования $R$[1$\vert$opt] массивов~I--V этим алгоритмом \mbox{приведены} 
 в~строке~5 табл.~\ref{tab1}. Избыточность кодирования по сравнению с~алгоритмом~A[1$\vert$0] уменьшается 
 в~2,1--4,4~раза и~составляет теперь 0,0003--0,0006~бт/п в~зависимости от массива. 
 \mbox{Соответствующее} снижение скорости кодирования составляет 0,0005--0,0017~бт/п.

Вычисление оптимальных весов требует значительного времени. Поэтому 
в~предназначенном для применения на практике алгоритме следует использовать 
общие фиксированные значения весов. Такие значения для совокупности массивов~I--V 
представлены в~колонках табл.~2, обозначенных~\mbox{I--V}.

Обозначим через A[1$\vert$fix] алгоритм независимого кодирования ошибок 
предсказания $\mathbf{D}^\text{A}$ и~$\mathbf{D}^\text{R}$ с~использованием 
фиксированных весов для построения кодовых распределений. Скорости кодирования $R$[1$\vert$fix] 
массивов~I--V этим алгоритмом, приведенные в~строке~6 табл.~\ref{tab1}, 
практически не\linebreak\vspace*{-12pt}

%\pagebreak

% tabl2
%\begin{table*}

\begin{center}

{{\tablename~2}\ \ \small{
Оптимальные и~фиксированные веса
}}

%\label{tab2}
\vspace{2ex}

{\small \tabcolsep=7.5pt
\begin{tabular}{|l|ccccc|c|}
\hline
\multicolumn{1}{|c|}{Массив}& I     & II    
& III   & IV    & V     & I--V  \\
\hline
&&&&&&\\[-9pt]
\hspace*{3mm}$\mathbf{D}^\text{A}$       &30 &22 &14 &11 &13 &12 \\
\hspace*{3mm}${\mathbf{D}^\text{A}}_0$     &37 &13 &15 &15 &17&15\\
\hspace*{3mm}${\mathbf{D}^\text{A}}_1$     &22 &34 &32 &13 &16 &20\\  
\hline
&&&&&&\\[-9pt]
\hspace*{3mm}$\mathbf{D}^\text{R}$      &59 &285    &81 &81 &42 &45  \\
\hspace*{3mm}${\mathbf{D}^\text{R}}_0$    &29 &\hphantom{1}54 &22 &22 &46 &30  \\
\hspace*{3mm}${\mathbf{D}^\text{R}}_1$    &68 &332    &96 &80 &46 &60  \\
\hline
\end{tabular}
}
\end{center}

%\end{table*}

\vspace*{9pt}


\noindent
 отличаются от скоростей кодирования $R$[1$\vert$opt] алгоритма A[1$\vert$opt] 
с~оптимальными весами.

Обозначим через A[2,fix$\vert$opt] алгоритм, который независимо кодирует элементы состояний
${\mathbf{D}^\text{A}}_0$, ${\mathbf{D}^\text{A}}_1$,
${\mathbf{D}^\text{R}}_0$ и~${\mathbf{D}^\text{R}}_1$
ошибок предсказания, построенных с~фиксированными порогами,  
принимающими значение $t\hm=4$, и~использует оптимальные веса для построения 
кодовых распределений. Скорости кодирования $R$[2,fix$\vert$opt] массивов~I--V 
этим алгоритмом приведены в~строке~8 табл.~\ref{tab1}. Избыточность кодирования по 
сравнению с~алгоритмом A[2,fix$\vert$0] уменьшается в~2,5--3,8~раза до 0,0004--0,0008~бт/п 
в~зависимости от массива. Снижение скорости кодирования составляет 0,0009--0,0020~бт/п.

Обозначим через A[2,fix$\vert$fix] алгоритм, аналогичный A[2,fix$\vert$opt], 
но использующий фиксированные веса для построения кодовых распределений. 
Скорости кодирования $R$[2,fix$\vert$fix] \mbox{массивов~I--V} этим алгоритмом,
 приведенные в~строке~9 табл.~\ref{tab1}, 
практически не отличаются от скоростей кодирования $R$[2,fix$\vert$opt] алгоритма~A[2,fix$\vert$opt], 
использующего оптимальные веса.




\section{Заключение}
%\label{sec4}

Рассмотрен ряд алгоритмов обратимого сжатия массивов силовых кривых, 
основанных на универсальном арифметическом кодировании ошибок предсказания, и~получены 
оценки скорости кодирования таких алгоритмов. Показано, что разложение ошибок 
предсказания на независимо кодируемые вычислимые состояния и~выбор подходящих 
весов в~формуле для кодовых распределений позволяют уменьшить скорость кодирования.

Применимым на практике алгоритмом, обеспечивающим наименьшую скорость кодирования, 
оказался алгоритм A[2,fix$\vert$fix]. Скорости кодирования этим алгоритмом массивов~I--V даны в~строке~9 
табл.~\ref{tab1}. Алгоритм A[2,fix$\vert$fix] дает заметно меньшую скорость
 кодирования по сравнению с~алгоритм обратимого сжатия стандарта JPEG~2000~\cite{b1}. 
 Для массивов I,\dots,V выигрыш составляет 0,1737, 0,1967, 0,1591, 0,1404, 0,1117~бт/п со\-от\-вет\-ст\-венно.
 {\looseness=1
 
 }

Интересно применить использованные методы универсального кодирования (разделение на
 вычислимые состояния, выбор весов в~формуле~(\ref{eq02})) в~случае 
 арифметического кодирования компонент вейв\-лет-пре\-об\-ра\-зо\-ва\-ния~\cite{b1}. 
 Это должно стать предметом сле\-ду\-ющей работы.

{\small\frenchspacing
{%\baselineskip=10.8pt
%\addcontentsline{toc}{section}{References}
\begin{thebibliography}{9}
\bibitem{b1}
\Au{Стефанович А.\,И., Сушко~Д.\,В.} О сжатии данных массивов силовых кривых~// 
Информационные процессы, 2020. Т.~20. №\,3. С.~284--296.
\bibitem{b2}
\Au{Butt H.-J., Cappella~B., Kappl~M.} 
Force measurements with the atomic force microscope: Technique, interpretation and applications~// 
Surf. Sci. Rep., 2005. Vol.~59. P.~1--152. doi: 10.1016/j.surfrep.2005.08.003.
\bibitem{b3}
\Au{Witten~I.\,H., Neal~R.,M., Cleary~J.\,G.} Arithmetic coding for data compression~// 
Commun.  ACM, 1987. Vol.~30. No.\,6. P.~520--540. doi: 10.1145/214762.214771.
\bibitem{b4}
{\it Штарьков Ю.\,М.} Универсальное кодирование. Теория и~алгоритмы.~--- М.: Физматлит, 2013. 288~с.
\bibitem{b5}
{\it Сушко Д.\,В., Штарьков~Ю.\,М.} О сжатии томографических данных~// 
Информационные процессы, 2008. Т.~8. №\,4. С.~240--255.

 \end{thebibliography}

}
}

\end{multicols}

\vspace*{-3pt}

\hfill{\small\textit{Поступила в~редакцию 30.12.2020}}

\vspace*{8pt}

%\pagebreak

%\newpage

%\vspace*{-28pt}

\hrule

\vspace*{2pt}

\hrule

%\vspace*{-2pt}

\def\tit{COMPRESSION ALGORITHMS FOR FORCE VOLUME DATA~I:
CODING OF PREDICTION ERRORS}


\def\titkol{Compression algorithms for force volume data~I: Coding of prediction errors}

\def\aut{D.\,V.~Sushko}

\def\autkol{D.\,V.~Sushko}


\titel{\tit}{\aut}{\autkol}{\titkol}

\vspace*{-11pt}




\noindent
Institute of Informatics Problems, Federal Research Center ``Computer Science and Control''
 of the Russian Academy of Sciences, 44-2~Vavilov Str., Moscow 119333, Russian Federation

 
\def\leftfootline{\small{\textbf{\thepage}
\hfill INFORMATIKA I EE PRIMENENIYA~--- INFORMATICS AND
APPLICATIONS\ \ \ 2021\ \ \ volume~15\ \ \ issue\ 2}
}%
\def\rightfootline{\small{INFORMATIKA I EE PRIMENENIYA~---
INFORMATICS AND APPLICATIONS\ \ \ 2021\ \ \ volume~15\ \ \ issue\ 2
\hfill \textbf{\thepage}}}

\vspace*{3pt}


\Abste{The author considers the problem of reversible (lossless) 
compression of force volume data which are the three-dimensional arrays with 16-bit 
integer elements. Such arrays are the result of atomic force microscopy scanning 
of microobjects in the force mapping mode. The author proposes reversible compression 
algorithms of force volume data based on the universal arithmetic coding of their 
prediction errors. The author uses two methods of universal coding. The first method 
based on the statistical model of the source with the calculable sequence of states 
implies the decomposition of an error prediction sequence into two subsequences 
which are coded independently. The second method implies a choice of the appropriate 
weight while constructing the code probabilities used in arithmetic coding. The 
author constructs bit rate estimations for the proposed algorithms for five test arrays. 
The results show that combination of the universal coding methods mentioned above 
makes significant reduction of the bit rate. The bit rates of the most efficient 
algorithm among proposed practically applicable algorithms for the test arrays 
are~3.9285, 3.5268, 3.5024, 4.2813, and 4.2246 bit/pixel.}

\KWE{atomic force microscope; force volume data; reversible compression; arithmetic coding; universal coding}

\DOI{10.14357/19922264210212}

%\vspace*{-15pt}

 %\Ack
%\noindent

%\vspace*{12pt}

  \begin{multicols}{2}

\renewcommand{\bibname}{\protect\rmfamily References}
%\renewcommand{\bibname}{\large\protect\rm References}

{\small\frenchspacing
 {%\baselineskip=10.8pt
 \addcontentsline{toc}{section}{References}
 \begin{thebibliography}{9}
\bibitem{1-ss}
\Aue{Stefanovich, A.\,I., and D.\,V.~Sushko.}
 2020. O~szhatii dannykh massivov silovykh krivykh [On data compression of force volumes]. 
 \textit{Informatsionnye protsessy} [Information Processes] 20(3):284--296. 
\bibitem{2-ss}
\Aue{Butt, H.-J., B.~Cappella, and M.~Kappl.}
 2005. Force measurements with the atomic force microscope: Technique, interpretation and 
 applications. \textit{Surf. Sci. Rep.} 59:1--152.
 doi: 10.1016/j.surfrep.2005.08.003.
\bibitem{3-ss}
\Aue{Witten, I.\,H., R.\,M.~Neal, and J.\,G.~Cleary.}
 1987. Arithmetic coding for data compression. \textit{Commun. ACM} 30(6):520--540. 
 doi: 10.1145/214762.214771.
\bibitem{4-ss}
\Aue{Shtar'kov, Yu.\,M.} 2013. \textit{Universal'noe kodirovanie. Teoriya i~algoritmy} [Universal coding.
 Theory and algorithms]. Мoscow: Fizmatlit. 288 p. 
\bibitem{5-ss}
\Aue{Sushko, D.\,V., and Yu.\,M.~Shtar'kov.} 2008. 
O~szhatii tomograficheskikh dannykh [On tomography data compression]. 
\textit{Informatsionnye Protsessy} [Information Processes] 8(4):240--255. 
 \end{thebibliography}

 }
 }

\end{multicols}

\vspace*{-4pt}

  \hfill{\small\textit{Received December~30, 2020}}


%\pagebreak

\vspace*{-12pt}  

\Contrl

\noindent
\textbf{Sushko Dmitry V.} (b.\ 1962)~--- 
Candidate of Science (PhD) in physics and mathematics, senior scientist, 
Institute of Informatics Problems, Federal Research Center ``Computer 
Science and Control'' of the Russian Academy of Sciences, 44-2~Vavilov Str., 
Moscow 119333, Russian Federation; \mbox{dsushko@ipiran.ru}

\label{end\stat}

\renewcommand{\bibname}{\protect\rm Литература}   %12
\def\stat{gonch+zatsman}

\def\tit{ПРИНЦИПЫ СТРУКТУРИЗАЦИИ СТАТЕЙ\\ В~ЭЛЕКТРОННЫХ СЛОВАРЯХ$^*$}

\def\titkol{Принципы структуризации статей в~электронных словарях}

\def\aut{А.\,А.~Гончаров$^1$, И.\,М.~Зацман$^2$}

\def\autkol{А.\,А.~Гончаров, И.\,М.~Зацман}

\titel{\tit}{\aut}{\autkol}{\titkol}

\index{Гончаров А.\,А.}
\index{Зацман И.\,М.}
\index{Goncharov A.\,A.}
\index{Zatsman I.\,M.}

{\renewcommand{\thefootnote}{\fnsymbol{footnote}} \footnotetext[1]
{Работа выполнена в~Институте проблем информатики ФИЦ ИУ РАН при поддержке РФФИ (проект  
20-012-00166).}}


\renewcommand{\thefootnote}{\arabic{footnote}}
\footnotetext[1]{Институт проблем информатики Федерального исследовательского центра <<Информатика 
и~управление>> Российской академии наук, \mbox{a.gonch48@gmail.com}}
\footnotetext[2]{Институт проблем информатики Федерального исследовательского центра <<Информатика 
и~управление>> Российской академии наук, \mbox{izatsman@yandex.ru}}

\vspace*{-12pt}


  
  \Abst{Рассмотрены две задачи, возникающие при переводе бумажных словарей 
в~электронную форму представления: (1)~структуризация унаследованных 
и~существующих в~бумажной форме словарных статей, обеспечивающая расширение 
функциональных возможностей электронного словаря по сравнению с~бумажным; 
(2)~замена традиционных способов шрифтового выделения структурных элементов 
словарной статьи на способы, обеспечивающие их программную адресацию в~базе 
данных. Показано, что структуру словарных статей, используемую в~традиционной 
лексикографии, необходимо детализировать и~одновременно с~этим категоризировать 
часть структурных элементов для расширения функциональных возможностей 
электронного словаря. Описан подход к~формированию классификационной системы, 
интегрированной в~электронный словарь, и~последующей рубрикации структурных 
элементов словарных статей на ее основе. Предлагаемые решения позволяют значительно 
расширить функционал электронного словаря по сравнению с~его бумажным аналогом 
и~преодолеть ограничения традиционной лексикографии, обусловленные бумажной 
формой представления.}
  
  \KW{принципы структуризации; электронный словарь; электронная лексикография; 
классификационная система}

\DOI{10.14357/19922264210213}

%\vspace*{-3pt}


\vskip 10pt plus 9pt minus 6pt

\thispagestyle{headings}

\begin{multicols}{2}

\label{st\stat}
  
\section{Введение}

  Еще в~2000~г.\ французский лингвист Б.~Серкилини (B.~Cerquiglini) 
в~ходе своего выступления на Седьмой международной конференции 
<<Journ$\acute{\mbox{e}}$e des dictionnaires>>, тема которой звучала как 
<<От бумажных словарей к~словарям электронным>>, выделил в~развитии 
электронной лексикографии три этапа. Первый из них~--- создание 
бумажных словарей с~использованием компьютера; второй~--- перевод 
существующих бумажных словарей в~электронный\linebreak формат; третий (на тот 
момент только начинавшийся)~--- изначальная разработка словарей 
в~электронном формате с~пользовательскими функциями,\linebreak реализация 
которых невозможна при издании словарей на бумажном носителе~[1, 
с.~188], например использование потоковых объектов мультимедиа.
  
  В том же 2000~г. Р.~Вешлер и~К.~Питтс сравнили электронные 
и~бумажные словари применительно к~обучению английскому языку как 
иностранному. Вывод, который они сделали, оказался неутешительным: 
<<электронные словари по-преж\-не\-му остаются по своей сути бумажными 
словарями, записанными на электронный носитель>>~[2]. В~2012~г., по 
мнению С.~Гранже, это утверждение~--- не\-смот\-ря на улучшение  
ситуации~--- все еще оставалось актуальным для значительного числа 
электронных словарей~[3, с.~2].
  
  Сегодня также приходится признавать, что работа по созданию 
электронных словарей, которые принципиально расширяли бы функционал 
словарей бумажных и~были бы удобны для пользователя и~лексикографа, не 
теряет актуальности. Более того, при создании электронного словаря на 
основе бумажного необходимо учитывать отличия как в~структуре их 
словарных статей, так и~в способах выделения структурных элементов. 
Например, в~бумажных словарях для этой цели может использоваться 
шрифтовое выделение (курсив, полужирный шрифт, заглавные буквы), 
символы шрифта Wingdings (\raisebox{-1pt}[0pt][0pt]{\mbox{%
   \epsfxsize=2.5mm 
  \epsfbox{listi.eps}
   }}, \raisebox{-1pt}[0pt][0pt]{\mbox{%
   \epsfxsize=2.5mm 
  \epsfbox{flag.eps}
   }}, \raisebox{-1pt}[0pt][0pt]{\mbox{%
   \epsfxsize=3.5mm 
  \epsfbox{kniga.eps}
   }} и~т.\,д.), нумерация арабскими и/или римскими цифрами и~спецсимволы 
(точка с~запятой, пробелы, скобки, слеши, кавычки и~т.\,п.)~[4].

  \begin{table*}\small
  \begin{center}
  \begin{tabular}{|c|c|c|p{70mm}|}
  \multicolumn{4}{c}{Пример распределения содержания словарной статьи из~\cite{5-gz} 
по зонам}\\
  \multicolumn{4}{c}{\ }\\[-6pt]
  \hline
\multicolumn{3}{|c|}{Заглавное слово (=\;лемма) и~его 
варианты}&\textbf{k$\ddot{\mbox{o}}$nnen}\\
\hline
\multicolumn{3}{|c|}{Зона грамматической информации о лемме в~целом}&\textit{vmod} 
(\textit{perf} hat k$\ddot{\mbox{o}}$nnen, \textit{в~неполных предложениях, где пропущен 
инфинитив полнозначного глагола} hat gekonnt)\\
\hline
\hspace*{-2pt}{\raisebox{-75pt}{\rotatebox{90}{Зона значения}}}
&
\hspace*{-2pt}{\raisebox{-67pt}{\rotatebox{90}{Значение 1}}}&
\tabcolsep=0pt\begin{tabular}{c}Толкование леммы\\ в~данном значении\end{tabular}&
\textit{для  выражения потенциальной возможности}\\[-65pt]
\cline{3-4}
&&\raisebox{-6pt}[0pt][0pt]{\tabcolsep=0pt\begin{tabular}{c}Варианты перевода леммы\\ 
в~данном значении\end{tabular}}&
мочь, иметь возможность; можно; 
\textit{под отрицанием} \mbox{нельзя}\\
\cline{3-4}
&&\raisebox{-6pt}[0pt][0pt]{\tabcolsep=0pt\begin{tabular}{c}Примеры употребления\\ 
леммы в~данном значении\end{tabular}}&
ich habe heute frei und kann dich 
besuchen я~сегодня свободен и~могу к~тебе зайти; [$\ldots$]\\
\cline{3-4}
&&\raisebox{-16pt}[0pt][0pt]{\tabcolsep=0pt\begin{tabular}{c}Устойчивые конструкции\\
 с~использованием леммы\\ 
в~данном значении$^*$\end{tabular}}&$\ldots$, 
\textbf{ich kann dir sagen!} (\textit{только в~постпозиции и~с~прямым порядком слов}) 
$\ldots$, просто фантастика!; das war eine Schl$\ddot{\mbox{a}}$gerei, ich kann dir sagen ну 
и~драка была, я~тебе скажу!; [$\ldots$]\\
\cline{2-4}
&$\ldots$&$\ldots$&$\ldots$\\
\hline
\multicolumn{3}{|c|}{Зона идиоматики}&[$\ldots$] \textbf{(gut) k$\ddot{\mbox{o}}$nnen} 
(\textit{mit jmdm.}) \textit{разг.}\ быть в~(дружеских) отношениях (\textit{с~кем-л.}); die 
beiden k$\ddot{\mbox{o}}$nnen einfach nicht miteinander отношения у~них просто не 
складываются; [$\ldots$]\\
\hline
\multicolumn{4}{p{150mm}}{\footnotesize \hspace*{2mm}$^*$В~более ранних работах, 
в~том числе~\cite{9-gz, 10-gz}, эта зона носила название <<Грамматическая фразеология 
с~использованием леммы в~данном значении>>.}
\end{tabular}
\end{center}
\vspace*{-5pt}
\end{table*}
  
  Следовательно, когда при подготовке электронного словаря используются 
наследуемые лексикографические ресурсы, возникают две задачи:\linebreak
%\begin{enumerate}[(1)]
%\item 
(1)~более 
детальная (по сравнению с~бумажным словарем) структуризация словарных 
статей, обес\-пе\-чи\-ва\-ющая расширение его функциональных возможностей; 
%\item 
(2)~замена традиционных способов\linebreak выделения структурных элементов 
(=\;по\-лей) разметкой тегами и/или использование баз данных в~процессе его 
подготовки. 
%\end{enumerate}
Лишь после их решения появляется возможность существенного 
расширения функциональных возможностей электронного словаря по 
сравнению с~бумажным.
  
Цель статьи состоит в~описании принципов решения первой задачи на примере двуязычного 
(не\-мец\-ко-рус\-ско\-го) словаря, создаваемого группой лексикографов под руководством Д.\,О.~Доб\-ро\-вольского~[5], 
с~применением надкорпусной базы данных (НБД) немецких модальных глаголов, созданной в~ФИЦ ИУ РАН 
(см.\ подробнее~[6]). Решение этой задачи с~применением НБД необходимо, в~частности, и~для того, чтобы 
обеспечить возможность фиксации ретроспективы изменений, вносимых в~словарные статьи 
лексикографами (см.\ об этом~[7, 8]\footnote{В этих работах механизм фиксации изменений, вносимых 
в~словарные статьи, рассматривается на примере лишь одной зоны статьи~--- зоны значения, приводимой, 
кроме того, в~сокращенном виде. Однако при условии, что словарные статьи были структурированы, 
применение НБД позволяет фиксировать ретроспективу изменений, вносимых в~любую из зон статьи.}). 
Частично рассматривается решение и~второй задачи, которая более детально описана  
в~работе~\cite{4-gz}\footnote{В~этой работе задача решается на примере немецко-русского 
фразеологического словаря.}.


\vspace*{-6pt}
  
\section{Структуризация словарных статей}

\vspace*{-3pt}

  В работе~[9, с.~92] в~форме таблицы была пред\-став\-ле\-на структура статьи, 
используемая в~бумажном словаре~\cite{5-gz}\footnote{Более подробное описание 
см.\ в~\cite[с.~40--44]{10-gz}; хотя там говорится о структуре статьи нового большого 
не\-мец\-ко-рус\-ско\-го словаря, она во многом совпадает со структурой статьи словаря~\cite{5-gz}. 
Другие зоны, 
которые также используются при составлении словарей, перечисляются в~\cite{11-gz}.}. 
Каждая строка таблицы соответствовала зоне словарной статьи с~точки 
зрения традици\-он\-ной лексикографии, а~столбцы показывали уровень 
вложенности зон. 

В~таб\-ли\-це проиллюстрировано, каким образом 
содержание словарной статьи распределено по зонам. Для примера взята 
статья на один из модальных глаголов немецкого языка~--- глагол 
\textit{k$\ddot{\mbox{o}}$nnen}. В~целях экономии места в~таблице 
приведено только первое из девяти его значений, 
а~также сокращено чис\-ло примеров употребления, устойчивых конструкций и~идиом
(пропуски отмечены как 
[$\ldots$]).
  



  Хотя на первый взгляд может показаться, что\linebreak такого распределения 
содержания статьи по зонам~--- структурным элементам верхнего уровня~--- 
достаточно для работы со словарем в~\mbox{электронном} формате, почти каждая 
зона с~содержательной точки зрения может быть разделена на поля~--- 
структурные элементы нижних уровней, что даст возможность расширить 
спектр областей поиска в~электронном словаре.
  
  Так, <<зона грамматической информации о лемме в~целом>> объединяет 
как минимум два поля: (1)~<<\textit{vmod}>> (`модальный глагол')~--- 
информация о том, к~какой части речи принадлежит\linebreak лемма; (2)~<<\textit{perf} hat 
k$\ddot{\mbox{o}}$nnen, \textit{в~неполных предложениях, где пропущен 
инфинитив полнозначного глагола} hat gekonnt>>~--- информация об 
особенностях образования грамматических форм леммы. Более того, внут\-ри 
второго поля можно выделить элемент <<\textit{perf}>> (`перфект'), 
ука\-зы\-ва\-ющий, формы какого грамматического времени глагола образуются 
не по общему правилу. 
  
  Зона вариантов перевода леммы в~данном значении в~примере из таблицы 
распадается на 3~поля, в~бумажном оформлении разделенные точкой 
с~запятой. Таким образом сгруппированы наиболее близкие по значению 
варианты перевода: (1)~<<мочь, иметь возможность>>; (2)~<<можно>>; 
(3)~<<\textit{под отрицанием} нельзя>>. Кроме того, третье поле содержит 
комментарий, в~данном случае~--- объяснение условий, при которых следует 
использовать этот вариант перевода (<<\textit{под отрицанием}>>).
  
  Зона примеров употребления леммы в~данном значении, во-пер\-вых, 
состоит из отдельных примеров (в~бумажном оформлении также 
разделенных точкой с~запятой), в~каждом из которых, во-вторых, можно 
выделить оригинальный текст примера и~его перевод (отделенные друг от 
друга пробелом).
  
  Зона <<Устойчивые конструкции с~использованием леммы в~данном 
значении>> имеет сходную структуру, которая, однако, включает и~другие 
поля. Полужирным шрифтом выделена сама устойчивая конструкция, для 
которой в~скобках курсивом могут указываться (1)~синтаксические 
валентности\footnote{Синтаксической валентностью называется <<способность слова 
вступать в~синтаксические связи с~другими элементами>>~\cite[с.~79--80]{12-gz}. В~нижней 
строке таблицы для идиомы <<(gut) k$\ddot{\mbox{o}}$nnen>> указана валентность <<\textit{mit 
jmdm.}>>~--- буквально `с~кем-ли\-бо'. Валентность <<\textit{с~кем-л.}>> указана и~для перевода 
данной идиомы на русский язык.} и~(2)~комментарии особенностей употребления. 
Более того, для некоторых устойчивых конструкций приводятся примеры их 
употребления (оригинал и~перевод).
  
  Зона идиоматики с~точки зрения структуры почти полностью повторяет 
зону <<Устойчивые конструкции$\ldots$>>, однако идиомы могут 
со\-про\-вож\-дать\-ся указаниями на стилистические особенности их 
употребления~--- стилистическими пометами (см.\ <<\textit{разг.}>> 
в~нижней строке таблицы).
  
  Из сказанного выше можно сделать вывод, что для создания электронного 
словаря, функционал которого был бы шире функционала соответствующего 
бумажного словаря, требуется сделать пригодными для адресации 
и~программной обработки не только традиционно выделяемые зоны статьи, но и~вложенные в~них поля. 
Так, если представить пример из таблицы в~виде  
XML-де\-ре\-ва\footnote{Идея использования XML-разметки словарных статей не нова и~ранее 
была описана в~контексте обмена словарными ресурсами~[13, 14, с.~291--329].} с~учетом 
предлагаемого уровня детализации, получим следующий результат 
структуризации словарной статьи\footnote{Разметка выполнена для статьи из таблицы 
с~использованием следующих тегов: $\langle${\sf entry}$\rangle$~--- словарная статья; 
$\langle${\sf hdw}$\rangle$~--- лемма; $\langle${\sf grinf}$\rangle$~--- 
грамматическая информация; $\langle${\sf pos}$\rangle$~--- информация о том, к~какой части 
речи принадлежит лемма; $\langle${\sf grforms}$\rangle$~--- описание особенностей 
образования грамматических форм; $\langle${\sf form}$\rangle$~--- грамматическая форма; 
$\langle${\sf usn}$\rangle$~--- стилистические пометы; $\langle${\sf 
idmfld}$\rangle$~--- зона идиоматики; $\langle${\sf idm}$\rangle$~--- идиома; 
$\langle${\sf mnfld}$\rangle$~--- зона значения; $\langle${\sf mn}$\rangle$~--- 
значение; $\langle${\sf mndesc}$\rangle$~--- описание значения; $\langle${\sf 
interp}$\rangle$~--- толкование значения; $\langle${\sf orig}$\rangle$~--- текст 
оригинала; $\langle${\sf trnsl}$\rangle$~--- текст перевода; $\langle${\sf 
tgroup}$\rangle$~--- группа вариантов перевода; $\langle${\sf phrasfld}$\rangle$~---  
зона <<Устойчивые конструкции$\ldots$>>;
   $\langle${\sf phras}$\rangle$~--- устойчивая конструкция; $\langle${\sf 
comm}$\rangle$~--- комментарий; $\langle${\sf exfld}$\rangle$~--- зона примеров; 
$\langle${\sf ex}$\rangle$~--- пример; $\langle${\sf val}$\rangle$~--- зона 
синтаксических валентностей.}:

%\vspace*{2pt}

\noindent
  {\small 
  \begin{tabular}{l}
  $\langle${\sf entry}$\rangle$\\
  $\langle${\sf 
hdw}$\rangle$\textbf{k$\ddot{\mbox{o}}$nnen}$\langle$/{\sf hdw}$\rangle$
  \\
  $\langle${\sf grinf}$\rangle$\\
  \hspace*{3mm}$\langle${\sf pos}$\rangle$\textit{vmod}$\langle$/{\sf pos}$\rangle$\\
  \hspace*{3mm}$\langle${\sf grforms}$\rangle$\\
  \hspace*{6mm}$\langle${\sf 
form}$\rangle$\textit{perf}$\langle$/{\sf form}$\rangle$\\
  \hspace*{8mm}hat k$\ddot{\mbox{o}}$nnen, \textit{в неполных 
предложениях, где}\\ 
\hspace*{8mm}\textit{пропущен инфинитив полнозначного глагола}\\
\hspace*{8mm}hat  gekonnt\\
  \hspace*{3mm}$\langle$/{\sf grforms}$\rangle$\\
  $\langle$/{\sf grinf}$\rangle$\\
  $\langle${\sf mnfld}$\rangle$\\
  \hspace*{3mm}$\langle${\sf mn}$\rangle$\\
  \hspace*{6mm}$\langle${\sf mndesc}$\rangle$\\
  \hspace*{10mm}$\langle${\sf 
interp}$\rangle$\textit{для выражения 
потенциальной}\\
\hspace*{10mm}\textit{возможности}$\langle$/{\sf interp}$\rangle$\\
  \hspace*{8mm}$\langle${\sf trnsl}$\rangle$\\
  \hspace*{10mm}$\langle${\sf 
tgroup}$\rangle$мочь, иметь 
возможность$\langle$/{\sf tgroup}$\rangle$\\
  \hspace*{10mm}$\langle${\sf 
tgroup}$\rangle$можно$\langle$/{\sf tgroup}$\rangle$\\
  \hspace*{10mm}$\langle${\sf tgroup}$\rangle$\\
  \hspace*{12mm}$\langle${\sf comm}$\rangle$\textit{под отрицанием}$\langle$/{\sf comm}$\rangle$ 
нельзя\\
  \hspace*{10mm}$\langle$/{\sf 
tgroup}$\rangle$\\
  \hspace*{8mm}$\langle$/{\sf trnsl}$\rangle$\\
  \hspace*{6mm}$\langle$/{\sf mndesc}$\rangle$\\
  \hspace*{6mm}$\langle${\sf exfld}$\rangle$\\
  \hspace*{8mm}$\langle${\sf ex}$\rangle$\\
  \hspace*{12mm}$\langle${\sf orig}$\rangle$ich habe heute frei und kann dich\\
  \hspace*{12mm}besuchen$\langle$/{\sf orig}$\rangle$\\
  \hspace*{12mm}$\langle${\sf  trnsl}$\rangle$я сегодня свободен и~могу к~тебе\\
  \hspace*{12mm}зайти$\langle$/{\sf trnsl}$\rangle$\\
  \hspace*{8mm}$\langle$/{\sf ex}$\rangle$\\
  \hspace*{8mm}$\ldots$\\
  \hspace*{6mm}$\langle$/{\sf exfld}$\rangle$\\
  \hspace*{6mm}$\langle${\sf phrasfld}$\rangle$\\
  \hspace*{8mm}$\langle${\sf phras}$\rangle$\\
  \hspace*{10mm}$\langle${\sf orig}$\rangle$\\
  \hspace*{10mm}$\ldots$, \textbf{ich kann dir sagen!}\\ 
  \hspace*{12mm}$\langle${\sf comm}$\rangle$\textit{только в~постпозиции и~с~прямым}\\
  \hspace*{12mm}\textit{порядком  слов}$\langle$/{\sf comm}$\rangle$\\
  \hspace*{10mm}$\langle$/{\sf orig}$\rangle$\\
  \hspace*{10mm}$\langle${\sf trnsl}$\rangle$$\ldots$, просто фантастика!$\langle$/{\sf trnsl}$\rangle$\\
  \hspace*{10mm}$\langle${\sf exfld}$\rangle$$\ldots$$\langle$/{\sf exfld}$\rangle$\\
  \hspace*{8mm}$\langle$/{\sf phras}$\rangle$\\
  \hspace*{8mm}$\ldots$

  \end{tabular}
  
  }
  
\vspace*{0.5pt}
 
  \pagebreak
  
\noindent
   {\small 
  \begin{tabular}{l}
     \hspace*{6mm}$\langle$/{\sf phrasfld}$\rangle$\\
           \hspace*{3mm}$\langle$/{\sf mn}$\rangle$\\
  \hspace*{3mm}$\ldots$\\
  $\langle$/{\sf mnfld}$\rangle$\\
  $\langle${\sf idmfld}$\rangle$\\
  \hspace*{3mm}$\langle${\sf idm}$\rangle$\\
  \hspace*{6mm}$\langle${\sf orig}$\rangle$\textbf{(gut) 
k$\ddot{\mbox{o}}$nnen}\\ 
  \hspace*{8mm}$\langle${\sf 
val}$\rangle$\textit{mit jmdm.}$\langle$/{\sf val}$\rangle$\\
  \hspace*{8mm}$\langle${\sf usn}$\rangle$\textit{разг.}$\langle$/{\sf usn}$\rangle$\\
  \hspace*{6mm}$\langle$/{\sf orig}$\rangle$\\
  \hspace*{6mm}$\langle${\sf trnsl}$\rangle$\\
  \hspace*{8mm}быть в~(дружеских) отношениях\\
  \hspace*{8mm}$\langle${\sf 
val}$\rangle$\textit{с~кем-л.}$\langle$/{\sf val}$\rangle$\\
  \hspace*{6mm}$\langle$/{\sf trnsl}$\rangle$\\
  \hspace*{6mm}$\langle${\sf 
exfld}$\rangle$$\ldots$$\langle$/{\sf exfld}$\rangle$\\
  \hspace*{3mm}$\langle$/{\sf idm}$\rangle$\\
  \hspace*{3mm}$\ldots$\\
  $\langle$/{\sf idmfld}$\rangle$\\
  $\langle$/{\sf entry}$\rangle$
  \end{tabular}
  }
  
  \vspace*{2pt}
  
  Такая структура является более детальной, чем представленное в~таблице 
традиционное деление на зоны, что и~обеспечивает существенное 
расширение функционала электронного словаря. Наиболее очевидная новая 
возможность~--- поиск словарных статей по тексту любого из структурно 
выделенных полей.
{\looseness=-1

}
  
  Другие возможные объекты поиска: все статьи, где описываются единицы, 
которые можно перевести на русский словом <<можно>>; все статьи, 
которые включают интересующий пользователя или лексикографа 
структурный элемент (например, зону <<Устойчивые конструкции$\ldots$>> 
или зону идиоматики) и~т.\,д. Это может быть ценным для лексикографии 
(создание словарей разных типов), для обучения иностранному языку (отбор 
материала по значениям грамматических признаков), а также для решения 
переводческих задач.

%\vspace*{-9pt}
  
\section{Формирование классификационной системы}

%\vspace*{-3pt}

  Хотя одно только выделение новых структурных элементов 
статьи расширяет функционал электронного словаря по сравнению 
с~бумажным, выполнение категоризации этих элементов способно 
обеспечить решение еще более широкого круга задач. Категоризация 
структурного элемента (=\;по\-ля) словарной статьи~--- это отнесение его 
к~некоторому классу или группе согласно некоторому признаку, например: 
(1)~<<часть речи>> (значения признака: $n$~--- существительное, $v$~--- 
глагол и~т.\,д.); (2)~<<постоянные грамматические характеристики>> 
(отметим, что наборы значений этого признака отличаются в~зависимости от 
значения признака <<часть речи>>, см.\ об этом также~[13, с.~116]: так, для 
существительного это грамматический род~--- мужской ($m$), средний ($n$) 
или женский ($f$); для глагола~--- переходность ($vt$) или непереходность 
($vi$) и~т.\,п.); (3)~<<стилистические особенности упо\-треб\-ле\-ния>> (значения 
признака: \textit{разг.}, \textit{груб.} и~т.\,д.) и~др.
{\looseness=-1

}
  
  Значения признаков могут быть объединены в~фасеты, которые, в~свою 
очередь, объединяются в~фасетную классификацию (с~по\-мощью которой 
выполняется рубрикация всей словарной статьи и/или ее структурных 
элементов).
  
  Создание и~использование такой классификации даст возможность искать 
по значению признака, например, словарные статьи, где лемма пред\-став\-ля\-ет 
собой: слово разговорного стиля; \mbox{существительное} среднего рода; 
прилагательное, име\-ющее особенности образования форм сравнительной 
степени, и~т.\,д. Важно отметить, что могут отбираться статьи, где, например, 
к~разговорному стилю относится не лемма, а~только идиома с~этой леммой 
(см.\ идиому с~пометой <<\textit{разг.}>> в~нижней строке таблицы).
  
  Существует также возможность добавления новых классификационных 
признаков. Одним из таких признаков, имеющих особую ценность для 
пользователей, может стать признак <<семантика леммы>>. Для его 
использования следует, во-пер\-вых, выбрать готовую (или создать новую) 
классификационную систему, которая будет использоваться в~электронном 
словаре, и,~во-вто\-рых, добавить поле для значения признака <<семантика 
леммы>>, ука\-зы\-ва\-юще\-го на принадлежность леммы к~некоторому семантическому классу\footnote{Применительно 
к~фразеологическому 
словарю об этом говорилось в~[15].}. Таким образом, двуязычный словарь 
приобретает отдельные свойства словаря идеографического или 
тезауруса~[16,~17].
{\looseness=-1

}
  
  Элементы реализации идеи включения в~словарь семантической 
классификационной системы можно найти во фран\-цуз\-ско-рус\-ской 
лексикографии~--- это так называемые <<комплексные словарные статьи>> 
в~[18]. В~отличие от традиционных статей двуязычного словаря 
комплексные статьи решают специализированные задачи: в~них могут 
разъясняться трудности перевода применительно к~описываемой паре языков 
(в~случае словаря~[18] это пара <<фран\-цуз\-ский--рус\-ский>>), 
рассматриваться способы выражения семантических категорий (например, 
<<цель>>, <<причина>>). Также в~словарь могут включаться семантические 
группы слов (названия и~перевод месяцев, дней недели, стран света и~т.\,п.). 
Однако в~рамках бумажного словаря лексикографы сталкиваются 
с~б$\acute{\mbox{о}}$льшими сложностями при реализации этой идеи, чем при создании словаря 
электронного.
  
  Включение в~электронный словарь фасетной классификации 
с~возможностью отбирать словарные статьи по значениям разных признаков и~их сочетаниям 
дает пользователю широкие возможности поиска и~существенно увеличивает 
спектр решаемых лексикографических задач\footnote[1]{Возможность отбирать 
словарные статьи в~зависимости от семантики описываемых в~них единиц позволит проверить, 
описаны ли эти единицы аналогичным образом и не пропущена ли ка\-кая-ли\-бо из единиц. 
Такая ситуация была детально рассмотрена В.\,А.~Успенским на 
примере включения в~толковые словари русского языка названий букв 
русского алфавита в~работе~[19, с.~605--609].}.

\vspace*{-6pt}

\section{Заключение}

\vspace*{-3pt}

  Существующие средства информатики дают возможность значительно 
расширить функционал электронных словарей по сравнению с~\mbox{бумажными} 
словарями, записанными на электронный носитель. Однако для 
использования накопленных лексикографических ресурсов требуется 
выполнить структуризацию наследуемых словарных статей, 
обеспечивающую последующее наполнение баз данных наследуемыми 
лексикографическими ресурсами, формирование электронных словарей 
и~выполнение в~них лек\-си\-ко-грам\-ма\-ти\-че\-ских видов поиска.
  
  Для создания лексикографических баз знаний с~развитыми возможностями 
семантического поиска необходимо предварительно сформировать и~потом 
использовать лингвистическую фасетную классификацию, объединяющую 
грамматические, функ\-ци\-о\-наль\-но-сти\-ли\-сти\-че\-ские и~семантические 
признаки с~их простановкой как в~словарных стать\-ях, так и~в~их 
структурных элементах. В~настоящее время проблема создания 
лексикографических баз знаний с~подобными возможностями находится на 
начальной стадии решения.


\vspace*{-6pt}


{\small\frenchspacing
{\baselineskip=10.7pt
%\addcontentsline{toc}{section}{References}
\begin{thebibliography}{99}

\vspace*{-2pt}

\bibitem{1-gz}
\Au{Pruvost J.} Des dictionnaires papier aux dictionnaires $\acute{\mbox{e}}$lectroniques: VIIe 
Journ$\acute{\mbox{e}}$e des dictionnaires (22~mars 2000): Rapport de colloque~// Int. 
J.~Lexicogr., 2000. Vol.~13. Iss.~3. P.~187--193. doi: 10.1093/ijl/13.3.187.
\bibitem{2-gz}
\Au{Weschler R., Pitts Chr.} An experiment using electronic dictionaries with EFL students. {\sf 
http://iteslj.org/ Articles/Weschler-ElectroDict.html}.
\bibitem{3-gz}
Electronic lexicography~/
Eds.\ S.~Granger, M.~Paquot.~--- Oxford University Press, 2012. 517~p.
\bibitem{4-gz}
\Au{Вакуленко В.\,В., Зацман~И.\,М.} Наследуемые лексикографические ресурсы базы 
данных фразеологического словаря~// Системы и~средства информатики, 2021. Т.~31. №\,2. 
С.~129--138.
\bibitem{5-gz}
Немецко-рус\-ский словарь актуальной лексики~/
Под ред. Д.\,О.~Добровольского.~--- М.: Лексрус, 2021 (в~пе\-чати).
\bibitem{6-gz}
\Au{Добровольский Д.\,О., Зализняк Анна~А.} Немецкие конструкции с~модальными 
глаголами и~их русские соответствия: проект надкорпусной базы данных~//\linebreak 
Компьютерная лингвистика и~интеллектуальные технологии: По мат-лам Междунар. 
конф. <<Диалог>>.~--- М.: РГГУ, 2018. Вып.~17(24). С.~172--184.
\bibitem{7-gz}
\Au{Гончаров А.\,А., Зацман~И.\,М., Кружков~М.\,Г.} Эволюция классификаций 
в~надкорпусных базах данных~// Информатика и~её применения, 2020. Т.~14. Вып.~4. 
С.~108--116.
\bibitem{8-gz}
\Au{Гончаров А.\,А., Зацман~И.\,М., Кружков~М.\,Г.} Пред\-став\-ле\-ние новых 
лексикографических знаний в~динамических классификационных системах~// 
Информатика и~её применения, 2021. Т.~15. Вып.~1. С.~82--89.
\bibitem{9-gz}
\Au{Гончаров А.\,А., Зацман~И.\,М., Кружков~М.\,Г.} Темпоральные данные 
в~лексикографических базах знаний~// Информатика и~её применения, 2019. Т.~13. 
Вып.~4. С.~90--96.
\bibitem{10-gz}
\Au{Добровольский Д.\,О.} Беседы о немецком слове.~--- М.: Языки славянской культуры, 
2013. 744~с.
\bibitem{11-gz}
\Au{Lehmann Chr.} Lexicography. Microstructure: Structure of a~lexical entry. {\sf 
https://www.\linebreak christianlehmann.eu/ling/ling\_meth/ling\_description/\linebreak lexicography/index.html}.
\bibitem{12-gz}
Языкознание: Большой энциклопедический словарь~/ Гл.\ ред. В.\,Н.~Ярцева.~---  2-е 
изд.~--- М.: Большая Российская энциклопедия, 1998. 685~с.
\bibitem{13-gz}
\Au{Ide N., Kilgarriff~A., Romary~L.} A~formal model of dictionary structure and content~// 9th 
EURALEX Congress (International) Proceedings.~--- Stuttgart: Institut f$\ddot{\mbox{u}}$r 
Maschinelle Sprachverarbeitung, 2000. P.~113--126.
\bibitem{14-gz}
TEI P5: Guidelines for Electronic Text Encoding and Interchange. Version~4.2.1.~--- TEI 
Consortium, 2021. {\sf https://tei-c.org/release/doc/tei-p5-doc/ en/Guidelines.pdf}.
\bibitem{15-gz}
\Au{Вакуленко В.\,В., Гончаров~А.\,А., Дурново~А.\,А., Зацман~И.\,М.} Задачи базы данных 
фразеологического словаря и~стадии ее проектирования~// Системы и~средства 
информатики, 2020. Т.~30. №\,2. С.~113--123.
\bibitem{16-gz}
WordNet: An electronic lexical database~/ Ed. Chr.~Fellbaum.~--- Cambridge, MA, USA: MIT Press, 
1998. 423~p.


\bibitem{17-gz}
\Au{Лукашевич Н.\,В.} Тезаурусы в~задачах информационного поиска.~--- М.: Изд-во 
Московского ун-та, 2011. 512~с.
\bibitem{18-gz}
\Au{Гак В.\,Г., Триомф~Ж.} Фран\-цуз\-ско-рус\-ский словарь активного типа.~--- М.: 
Русский язык, 1991. 1056~с.
\bibitem{19-gz}
\Au{Успенский В.\,А.}  
Невт$\acute{\mbox{о}}$н--Ньют$\acute{\mbox{о}}$н--Нь$\acute{\mbox{ю}}$тон, или 
Сколько сторон имеет языковой знак?~//  
Сб. к~60-ле\-тию Андрея Анатольевича Зализняка <<Русистика. Славистика. Индоевропеистика>>.~--- 
М.: Индрик, 1996. С.~598--659.
 \end{thebibliography}

}
}

\end{multicols}

\vspace*{-7pt}

\hfill{\small\textit{Поступила в~редакцию 14.04.2021}}

%\vspace*{8pt}

%\pagebreak

\newpage

\vspace*{-28pt}

%\hrule

%\vspace*{2pt}

%\hrule

%\vspace*{-2pt}

\def\tit{STRUCTURING PRINCIPLES OF~ELECTRONIC DICTIONARY'S ENTRIES}


\def\titkol{Structuring principles of~electronic dictionary's entries}

\def\aut{A.\,A.~Goncharov and~I.\,M.~Zatsman}

\def\autkol{A.\,A.~Goncharov and~I.\,M.~Zatsman}


\titel{\tit}{\aut}{\autkol}{\titkol}

\vspace*{-11pt}




\noindent
Institute of Informatics Problems, Federal Research Center ``Computer Science and Control''
 of the Russian Academy of Sciences, 44-2~Vavilov Str., Moscow 119333, Russian Federation

 
\def\leftfootline{\small{\textbf{\thepage}
\hfill INFORMATIKA I EE PRIMENENIYA~--- INFORMATICS AND
APPLICATIONS\ \ \ 2021\ \ \ volume~15\ \ \ issue\ 2}
}%
\def\rightfootline{\small{INFORMATIKA I EE PRIMENENIYA~---
INFORMATICS AND APPLICATIONS\ \ \ 2021\ \ \ volume~15\ \ \ issue\ 2
\hfill \textbf{\thepage}}}

\vspace*{3pt}



\Abste{Two tasks that arise when converting paper dictionaries into an electronic 
form are considered. In the first place, the authors suggest structuring inherited 
dictionary entries which provides the enrichment of the electronic dictionary's 
functionality, and in the second place, replacing the decorative design of the structural 
elements of dictionary entries with tagging that provide their addressing in 
databases. It is shown that the structure of dictionary entries used in traditional 
lexicography should be detailed. Simultaneously, it is necessary to categorize some 
of the structural elements to enrich the electronic dictionary's functionality. An 
approach to creating a~classification system integrated into an electronic 
dictionary and classifying dictionary entries' structural items is described. The 
proposed solutions allow to significantly enrich the electronic dictionary's 
functionality compared to its paper version and overcome traditional lexicography 
limitations related to the paper form of dictionary representation.}

\KWE{structuring principles; electronic dictionary; electronic lexicography; 
classification system}



\DOI{10.14357/19922264210213}

\vspace*{-15pt}

 \Ack
\noindent
The study was conducted at the Institute of Informatics Problems of the Federal 
Research Center ``Computer Science and Control'' of the Russian Academy of 
Sciences with financial support from the Russian Foundation for Basic Research (grant 
No.\,20-012-00166).

%\vspace*{12pt}

  \begin{multicols}{2}

\renewcommand{\bibname}{\protect\rmfamily References}
%\renewcommand{\bibname}{\large\protect\rm References}

{\small\frenchspacing
 {%\baselineskip=10.8pt
 \addcontentsline{toc}{section}{References}
 \begin{thebibliography}{99}
\bibitem{1-gz-1}
\Aue{Pruvost, J.} 2000. Des dictionnaires papier aux dictionnaires 
$\acute{\mbox{e}}$lectroniques. VIIe Journ$\acute{\mbox{e}}$e des dictionnaires 
(22~mars 2000). Rapport de colloque. \textit{Int. J.~Lexicogr.}  
13(3):187--193. doi: 10.1093/ijl/13.3.187.
\bibitem{2-gz-1}
\Aue{Weschler, R., and Chr.~Pitts.}  An experiment using electronic dictionaries 
with EFL students. Available at: {\sf  
http://iteslj.org/Articles/Weschler-ElectroDict.html} (accessed May~17, 2021).
\bibitem{3-gz-1}
Granger, S., and M.~Paquot, eds. 2012. \textit{Electronic lexicography}. Oxford 
University Press. 517~p.
\bibitem{4-gz-1}
\Aue{Vakulenko, V.\,V., and I.\,M.~Zatsman.} 2021. Nasleduemye 
leksikograficheskie resursy bazy dannykh frazeologicheskogo slovarya 
[Inheritable lexicographic resources of the phraseological dictionary database]. 
\textit{Sistemy i~Sredstva Informatiki~--- Systems and Means of Informatics}  
31(2):129--138.
\bibitem{5-gz-1}
Dobrovol'skiy, D.\,O., ed. 2021 (in press). \textit{Nemetsko-russkiy slovar' 
aktual'noy leksiki} [German--Russian dictionary of actual vocabulary]. Moscow: 
Leksrus.
\bibitem{6-gz-1}
\Aue{Dobrovol'skiy, D.\,O., and A.\,A.~Zaliznyak.} 2018. Ne\-me\-tskie konstruktsii 
s~modal'nymi glagolami i~ikh russkie sootvetstviya: proekt nadkorpusnoy bazy 
dannykh [German constructions with modal verbs and their Russian correlates: 
A~supracorpora database project]. \textit{Komp'yuternaya lingvistika 
i~intellektual'nye tekhnologii: po mat-lam Mezhdunar. konf. ``Dialog''} 
[Computational Linguistics and Intellectual Technologies. Papers from the Annual 
Conference (International) ``Dialogue'']. Moscow. 17(24):172--184.
\bibitem{7-gz-1}
\Aue{Goncharov, A.\,A., I.\,M.~Zatsman, and M.\,G.~Kruzhkov.} 2020. 
Evolyutsiya klassifikatsiy v~nadkorpusnykh ba\-zakh dannykh [Evolution of 
classifications in supracorpora databases]. \textit{Informatika i~ee  
Primeneniya~--- Inform. Appl.} 14(4):108--116.
\bibitem{8-gz-1}
\Aue{Goncharov, A.\,A., I.\,M.~Zatsman, and M.\,G.~Kruzhkov.} 2021. 
Predstavlenie novykh leksikograficheskikh znaniy v~dinamicheskikh 
klassifikatsionnykh sistemakh [Representation of new lexicographical knowledge 
in dynamic classification systems]. \textit{Informatika i~ee Primeneniya~--- 
Inform. Appl.} 15(1):82--89.
\bibitem{9-gz-1}
\Aue{Goncharov, A.\,A., I.\,M.~Zatsman, and M.\,G.~Kruzhkov.} 2019. 
Temporal'nye dannye v~leksikograficheskikh bazakh znaniy [Temporal data in 
lexicographic databases]. \textit{Informatika i~ee Primeneniya~--- Inform. Appl.} 
13(4):90--96.
{\looseness=1

}
\bibitem{10-gz-1}
\Aue{Dobrovol'skiy, D.\,O.} 2013. \textit{Besedy o~nemetskom slove} [Studies on 
German lexis]. Moscow: Yazyki slavyanskoy kul'tury. 744~p
\bibitem{11-gz-1}
\Aue{Lehmann, Chr.} Lexicography. Microstructure: Structure of a lexical entry. 
Available at: {\sf 
https://www.\linebreak christianlehmann.eu/ling/ling\_meth/ling\_description/\linebreak
lexicography/index.html} (accessed May~17, 2021).
\bibitem{12-gz-1}
Yartseva, V.\,N., ed. 1998. \textit{Yazykoznanie: Bol'shoy entsiklopedicheskiy 
slovar'} [Linguistics. Great encyclopedic dictionary]. 2nd ed. Moscow: Bol'shaya 
Rossiyskaya entsiklopediya. 685~p.
\bibitem{13-gz-1}
\Aue{Ide, N., A.~Kilgarriff, and L.~Romary.} 2000. A~formal model of dictionary 
structure and content. \textit{9th EURALEX
Congress (International) 
Proceedings}. Stuttgart: Institut f$\ddot{\mbox{u}}$r Maschinelle 
Sprachverarbeitung. 113--126.
\bibitem{14-gz-1}
TEI P5: Guidelines for electronic text encoding and interchange. Version~4.2.1. 
Available at: {\sf https://tei-c. org/release/doc/tei-p5-doc/en/Guidelines.pdf} 
(accessed May~17, 2021).
\bibitem{15-gz-1}
\Aue{Vakulenko, V.\,V., A.\,A.~Goncharov, A.\,A.~Durnovo, and 
I.\,M.~Zatsman.} 2020. Zadachi bazy dannykh fra\-ze\-o\-lo\-gi\-che\-sko\-go slovarya 
i~stadii ee proektirovaniya [Tasks of the phraseological dictionary database and 
stages of its design]. \textit{Sistemy i~Sredstva Informatiki~--- Systems and Means 
of Informatics} 30(2):113--123.
\bibitem{16-gz-1}
\Aue{Fellbaum, Ch.} 1998. \textit{WordNet: An electronic lexical database}. 
Cambridge, MA: MIT Press. 423~p.

\columnbreak

\bibitem{17-gz-1}
\Aue{Loukachevitch, N.\,V.}  2011. \textit{Tezaurusy v~zadachakh 
informatsionnogo poiska} [Thesauri in information retrieval tasks]. Moscow:  
Izd-vo Moskovskogo un-ta. 512~p.
\bibitem{18-gz-1}
\Aue{Gak, V.\,G., and Zh.~Triomf.} 1991. \textit{Frantsuzsko-russkiy slovar' 
aktivnogo tipa} [French--Russian dictionary of the active type]. Moscow: Russkiy 
yazyk. 1056~p.
\bibitem{19-gz-1}
\Aue{Uspenskiy, V.\,A.} 1996.  
Nevt$\acute{\mbox{o}}$n--N'yut$\acute{\mbox{o}}$n--N'y$\acute{\mbox{u}}$ton, ili Skol'ko 
storon imeet yazykovoy znak?  
[Nevt$\acute{\mbox{o}}$n--N'yut$\acute{\mbox{o}}$n--N'y$\acute{\mbox{u}}$ton, or How many sides 
does a~linguistic sign have?]. \textit{Sbornik k~60-letiyu Andreya Anatol'evicha\linebreak 
Zaliznyaka ``Rusistika. Slavistika. Indoevropeistika''} [A~collection of writings in 
honour of the 60th birthday of Andrey A.~Zaliznyak ``Russian studies. 
Slavic studies. Indo-European studies'']. Moscow: Indrik. 598--659.
{\looseness=1

}
\end{thebibliography}

 }
 }

\end{multicols}

\vspace*{-3pt}

  \hfill{\small\textit{Received April~14, 2021}}


%\pagebreak

%\vspace*{-8pt}  

\Contr

\noindent
\textbf{Goncharov Alexander A.} (b.\ 1994)~---
junior scientist, Institute of Informatics Problems, Federal Research Center 
``Computer Science and Control'' of the Russian Academy of Sciences,  
44-2~Vavilov Str., Moscow 119333, Russian Federation; 
\mbox{a.gonch48@gmail.com}

\vspace*{3pt}

\noindent
\textbf{Zatsman Igor M.} (b.\ 1952)~--- Doctor of Science in technology, Head of 
Department, Institute of Informatics Problems, Federal Research Center 
``Computer Science and Control'' of the Russian Academy of Sciences,  
44-2~Vavilov Str., Moscow 119333, Russian Federation; 
\mbox{izatsman@yandex.ru}

\label{end\stat}

\renewcommand{\bibname}{\protect\rm Литература} %13


\def\stat{gonch+inkova}

\def\tit{ИЗВЛЕЧЕНИЕ ЗНАНИЙ О~СРЕДСТВАХ ВЫРАЖЕНИЯ ЛОГИКО-СЕМАНТИЧЕСКИХ 
ОТНОШЕНИЙ\\ ПРИ~ПОМОЩИ НАДКОРПУСНОЙ БАЗЫ ДАННЫХ}

\def\titkol{Извлечение знаний о~средствах выражения логико-семантических 
отношений при~помощи НБД} %надкорпусной базы данных}

\def\aut{А.\,А.~Гончаров$^1$, О.\,Ю.~Инькова$^2$}

\def\autkol{А.\,А.~Гончаров, О.\,Ю.~Инькова}

\titel{\tit}{\aut}{\autkol}{\titkol}

\index{Гончаров А.\,А.}
\index{Инькова О.\,Ю.} 
\index{Goncharov A.\,A.}
\index{Inkova O.\,Yu.}

%{\renewcommand{\thefootnote}{\fnsymbol{footnote}} \footnotetext[1]
%{Работа выполнена в~Институте проблем информатики ФИЦ ИУ РАН при поддержке РФФИ (проект  
%20-012-00166).}}


\renewcommand{\thefootnote}{\arabic{footnote}}
\footnotetext[1]{Институт проблем информатики Федерального исследовательского центра <<Информатика 
и~управление>> Российской академии наук, \mbox{a.gonch48@gmail.com}}
\footnotetext[2]{Институт проблем информатики Федерального исследовательского центра <<Информатика 
и~управление>> Российской академии наук, \mbox{olyainkova@yandex.ru}}

\vspace*{-9pt}



\Abst{Цель статьи~--- показать продуктивность использования параллельных текстов и~их 
аннотирования в~надкорпусной базе данных (НБД) коннекторов для извлечения знаний об 
альтернативных средствах выражения ло\-ги\-ко-семантических отношений (ЛСО). На примере 
наиболее известных дискурсивно аннотированных корпусов~--- Penn Discourse Treebank (PDTB), 
Prague Dependency Treebank (PDT) и~Rhetorical Structure Theory  Discourse Treebank (RST-DT)~--- авторы 
показывают, что в~существующих исследованиях нет консенсуса относительно того, какие 
языковые средства относить к~классу коннекторов (прототипических показателей  
ЛСО), а~какие~--- к~альтернативным средствам. 
В~исследовании продемонстрировано, что применение сопоставительного метода 
и~использование возможностей НБД коннекторов позволяет не только извлекать новое 
знание о~средствах выражения ЛСО в~изучаемых языках, но и~создавать тезаурусы таких 
средств, в~том числе альтернативных коннекторам. Кроме того, 
информация, хранящаяся в~НБД, дает возможность получать новые знания о том, какие 
ЛСО могут быть выражены неспециализированными средствами и~какова частотность  
использования этих средств для каждого ЛСО в~каждом из изучаемых языков.}

\KW{надкорпусная база данных; логико-семантические отношения; коннекторы; 
извлечение новых знаний; параллельные тексты}

\DOI{10.14357/19922264210214}

\vspace*{4pt}


\vskip 10pt plus 9pt minus 6pt

\thispagestyle{headings}

\begin{multicols}{2}

\label{st\stat}

\section{Вводные замечания}

\vspace*{-6pt}

Логико-семантические, или, шире, дискурсивные, отношения, 
обеспечивающие связность текста на естественном языке, привлекают 
внимание лингвистов и~специалистов по информатике уже не один десяток 
лет: первые исследования начали появляться в~1970-х~гг.\ (например, работы 
Дж.~Хоббса~[1, 2]). Однако многие вопросы до сих пор остаются 
дискуссионными: это, в~первую очередь, и~само понятие <<дискурсивное 
отношение>>, и~понятие <<коннектор>> (единицы этого класса считаются 
прототипическими эксплицитными показателями таких отношений). Нет 
консенсуса и~относительно того, можно ли создать исчерпывающий список 
коннекторов для исследуемого языка. Тем не менее важность списков 
коннекторов для разработки дискурсивных парсеров и, шире, средств 
автоматической обработки текста и~автоматического извлечения информации 
из текста подчеркивается в~ряде работ (см.,\ например,~[3, с.~55]). В~[4], где 
описываются результаты разработки дискурсивного парсера для русского 
языка, отмечается, что наличие показателя 
ЛСО служит наиболее надежным признаком для определения 
того, каким именно отношением связаны фрагменты текста.

В то же время в~[5] показано, что в~зависимости от типа текста и~вида ЛСО 
коннекторы используются лишь в~30\%--40\% случаев. Остальные случаи 
представляют собой либо имплицитные ЛСО (подробнее об этом понятии 
см., например,~[6]), либо ЛСО, показателем которого являются языковые 
средства, отличные от коннекторов. Следовательно, качество результатов 
автоматической обработки текстов на естественном языке непосредственно 
зависит от уровня наших знаний не только о коннекторах, но и~об этих, 
альтернативных, средствах выражения ЛСО. Цель статьи состоит в~том, 
чтобы показать продуктивность использования параллельных текстов 
и~поисковых возможностей НБД коннекторов, разработанной в~ИПИ ФИЦ 
ИУ РАН (подробнее см.~\cite{7-in, 8-in, 9-in}), для извлечения знаний об 
альтернативных средствах выражения ЛСО.

\vspace*{-8pt}

\section{Существующие подходы}

\vspace*{-1pt}

Отправной точкой для активных исследований средств выражения ЛСО, 
альтернативных коннекторам, стала статья~[10], отражающая подход 
создателей Пенсильванского дискурсивно аннотированного корпуса (PDTB)~[11] к~этому вопросу. При аннотировании 
корпуса выяснилось, что если два фрагмента текста связаны ка\-ки\-м-ли\-бо 
ЛСО (дискурсивным отношением в~терминах PDTB), то это отношение 
может: ($i$)~выражаться коннектором\footnote{К~коннекторам относятся 
сочинительные и~подчинительные союзы, а~также некоторые другие языковые единицы, 
за которыми грамматиками английского языка традиционно признается связующая 
функция.}; ($ii$)~не выражаться коннектором, причем какой-либо коннектор 
можно добавить (=\;импли\-цит\-ные ЛСО); ($iii$)~не выражаться коннектором, 
причем никакой коннектор не может быть добавлен из-за возникающей 
в~этом случае семантической из-\linebreak быточности. Авторы пришли к~выводу, что 
такая\linebreak избыточность вызвана наличием альтернативных коннекторам 
лексических средств выражения ЛСО~--- <<альтернативных 
лексикализаций>> (alternative lexicalizations, AltLex).

Если в~PDTB в~разряд коннекторов попадает ограниченный круг языковых 
единиц, то разработчики Пражского корпуса синтаксических зависимостей 
 (PDT) трактуют понятие коннектор более 
широко, включая в~этот класс большинство лексических средств, которые так 
или иначе могут выражать ЛСО~\cite{3-in, 12-in}. Коннекторы при этом 
разделяются на <<первичные>> (primary) и~<<вторичные>> (secondary), 
довольно разнообразные по своей морфологической природе. Не включаются в~чис\-ло коннекторов лишь так называемые <<неуниверсальные>>, или 
<<свободные связующие сочетания>> (non-\linebreak universal\,/\,free connecting 
phrases), образующие третий класс средств выражения ЛСО. Ср.~(1)--(4) 
из~\cite[с.~51, 68]{3-in}:
{\small
\begin{enumerate}[(1)]
\item Fred didn't stop joking. \textbf{As a result}, his friends enjoyed hilarity throughout the 
evening.

`Фред не переставал шутить. \textbf{В~результате} его друзья смеялись весь 
вечер'.\footnote[2]{Здесь и~далее в~отсутствие других указаний перевод авторов статьи.}

\item  I had all the necessary qualifications. \textbf{Despite this}, I~didn't get the job.

`Я~удовлетворял всем квалификационным требованиям. \textbf{Несмотря на это}, на 
работу меня не приняли'.

\item 
Fred didn't stop joking. This \textbf{caused} hilarity among his friends for the whole evening.

`Фред не переставал шутить. Это \textbf{вызывало} смех его друзей весь вечер'.

\item 
 Fred has pneumonia. \textbf{Because of this illness}, he will be absent from his work for 
two weeks.

`У~Фреда пневмония. \textbf{Вследствие этой болезни} его не будет на работе две 
недели'.
\end{enumerate}
}

Так, в~(1) причинно-следственные ЛСО выражены союзом \textit{as 
a~result}~--- <<первичным коннектором>>. В~(2) и~(3) используются 
<<вторичные коннекторы>>: в~(2) это сочетание предлога \textit{despite} 
с~анафорическим выражением \textit{this}, отсылающим к~ситуации 
\textit{I~had all the necessary qualifications}, которое выражает уступительные 
ЛСО; в~(3) <<вторичным коннектором>> считается глагол \textit{caused}, 
вы\-ра\-жа\-ющий причинные ЛСО. Наконец, в~(4) \textit{because of this illness} 
рассматривается как <<неуниверсальное>>, или <<свободное связующее 
сочетание>>, поскольку оно непосредственно связано с~предыдущим 
контекстом (в~котором упомянута пневмония Фреда), в~отличие от 
\textit{despite this} в~(3), имеющего более общее значение. 

<<Альтернативным лексикализациям>> в~подходе PDTB соответствуют 
второй и~третий классы единиц в~PDT~\cite[с.~54]{3-in}. Такого же широкого 
подхода к~<<коннекторам>> придерживаются \mbox{разработчики} русского 
дискурсивно аннотированного корпуса~\cite{13-in}.

Подчеркнем, что во всех описанных случаях во внимание принимаются лишь 
лексические средства выражения ЛСО. В~\cite[с.~62]{3-in} даже особо 
отмечается, что <<коннекторами>> не считаются синтаксические 
и~морфологические средства, например относительные придаточные или 
деепричастия, которые в~ряде языков способны выражать ЛСО (см.\ ниже). 
Подчеркивая важность создания лексиконов связующих средств, 
разработчики Пражского корпуса не дают, тем не менее, списка 
<<вторичных коннекторов>>.

В последней версии PDTB~(3.0) появился класс показателей ЛСО AltLexC 
(где <<С>> означает \textit{Construction}), включающий лексико-син\-так\-си\-че\-ские 
средства выражения ЛСО~\cite[с.~9--10, 76]{14-in}. Однако 
если для <<первичных коннекторов>> приводится список языковых единиц, 
то ни для <<альтернативных лексикализаций>>, ни для нового класса 
AltLexC списков не дается. В~\cite[с.~75--76]{14-in} приводятся лишь ЛСО, 
которые выражают те единицы этих классов, что зафиксированы в~последней 
версии корпуса PDTB.

В рамках Теории риторической структуры (Rhetorical Structure Theory, RST) 
принципы классификации языковых единиц, способных выражать 
риторические (в~данной терминологии) отношения, даны  
в~работе~\cite[с.~8, 9]{15-in}, которая служит пособием по аннотированию 
показателей риторических отношений в~корпусе RST Discourse Treebank 
(RST-DT). Поскольку объем понятия <<риторическое отношение>> шире, 
чем ЛСО, так как включает не только отношения связности, которые могут 
быть выражены коннектором, то и~набор показателей этих отношений шире. 
К~<<первичным>> коннекторам добавляются показатели самой 
разнообразной природы: лексические, морфологические (временн$\acute{\mbox{ы}}$е формы), 
семантические (синонимия, антонимия и~др.), синтаксические (различные 
виды придаточных и~др.), графические (знаки препинания и~др.) и~т.\,д.; 
причем эти средства могут также комбинироваться друг 
с~другом\footnote[1]{Заметим, что в~более ранних версиях корпусов, размеченных 
в~соответствии с~RST, во всех таких случаях риторическое отношение считалось 
имплицитным. Согласно версии, представленной в~\cite{15-in}, оно оказывается 
эксплицитным, а в~тех случаях, когда никакой из потенциальных показателей отношения 
не может быть идентифицирован, проставляется метка \textit{unsure}.}. Однако 
в~пособии по аннотированию приводятся лишь типы показателей 
риторических отношений, а~не их список.
{\looseness=1

}

\section{Альтернативные средства~выражения логико-семантических отношений в~надкорпусной
базе данных коннекторов}

Если во всех упомянутых выше работах средства выражения ЛСО изучаются 
на одноязычном материале, то НБД коннекторов позволяет проводить 
исследования на материале параллельных текстов, используя методы 
сопоставительной лингвистики. Аннотирование употреблений коннекторов 
в~параллельных текстах позволило заметить, что в~некоторых случаях 
использованный в~оригинале коннектор переведен не коннектором, а~другим 
языковым средством (или наоборот, в~оригинале коннектор отсутствует, но 
появляется в~переводе). С~точки зрения сопоставительного подхода такие 
случаи представляют собой примеры <<дивергентного 
перевода>>\footnote[2]{Термин <<дивергентный перевод>> заимствован из 
работы~\cite{16-in}, посвященной использованию многоязычных корпусов 
в~контрастивных исследованиях; он был уточнен для исследования коннекторов 
в~\cite{17-in}.}.

Для обозначения языковых единиц, не являющихся коннекторами, но 
способными выражать ЛСО, предлагается использовать термин 
<<альтернативные коннекторам средства выражения ЛСО>>. В НБД такие 
средства делятся на ($i$)~лексические; ($ii$)~грамматические 
и~($iii$)~пунктуационные.

\subsection{Лексические средства}

В примере~(5) коннектор \textit{то есть}, выражающий ЛСО 
переформулирования, двумя переводчиками передан альтернативными 
коннекторам лексическими средствами~--- `я~имею в~виду' и~`я~хочу 
сказать' соответственно.
{\small 
\begin{enumerate}[(1)]
\setcounter{enumi}{4}
\item 
 Моя теща, \textbf{то есть} мать жены моей, тоже ничего не видит. [Н.\,В.~Гоголь. Нос 
(1832--1833)]

`Ma belle-m$\grave{\mbox{e}}$re, \textbf{j'entends} la m$\grave{\mbox{e}}$re de ma 
femme, a, elle aussi, la vue faible.' [Tr.\,H.~Mongault (1938)]

`Ma belle-m$\grave{\mbox{e}}$re, \textbf{je veux dire}, la m$\grave{\mbox{e}}$re de ma 
femme, elle non plus, elle n'y voit rien du tout.' [Tr.\,A.~Markowitz (2007)]
\end{enumerate}
}

\subsection{Грамматические средства}

В примере~(6) для передачи коннектора \textit{потому что}, выражающего 
ЛСО причины, также два переводчика используют форму причастия 
настоящего времени, которая во французском языке способна выражать это 
отношение.
{\small
\begin{enumerate}[(1)]
\setcounter{enumi}{5}
\item В~заключение прибавлял, что он <<был бы счастлив, если б удалось ему на себе 
оправдать свое убеждение, но что достичь этого он не надеется, \textbf{потому что} это 
очень трудно>>. [И.\,А.~Гончаров. Обломов (1848--1859)]

`Et il concluait en ajoutant qu'il serait tout $\grave{\mbox{a}}$~fait heureux s'il parvenait 
$\grave{\mbox{a}}$~justifier ses id$\acute{\mbox{e}}$es par son comportement, mais qu'il 
n'esp$\acute{\mbox{e}}$rait pas y~parvenir, cette ad$\acute{\mbox{e}}$quation 
$\acute{\mbox{\textbf{e}}}$\textbf{tant} fort difficile $\grave{\mbox{a}}$~atteindre.' 
[Tr.\,A.~Adamov (1959)]

`En guise de conclusion il ajoutait <<qu'il serait heureux s'il pouvait justifier ses convictions 
par sa propre vie, mais qu'il ne l'esp$\acute{\mbox{e}}$rait pas, cet objectif 
$\acute{\mbox{\textbf{e}}}$\textbf{tant} trop difficile $\grave{\mbox{a}}$~atteindre>>.' 
[Tr.\,L.~Jurgenson (1988)]
\end{enumerate}
}

\subsection{Пунктуационные средства}

В~(7) уже упоминавшийся выше коннектор \textit{потому что} передан 
в~двух переводах двоеточием.
{\small
\begin{enumerate}[(1)]
\setcounter{enumi}{6}
\item
$\ldots$но коллежский асессор Ковалев не мог слышать запаха, \textbf{потому что} 
закрылся платком и~потому что самый нос его находился бог знает в~каких местах. 
[Н.\,В.~Гоголь. Нос (1832--1833)]

`$\ldots$Mais l'assesseur de coll$\grave{\mbox{e}}$ge Kovaliov ne pouvait pas s'en rendre 
compte\textbf{:} il avait cach$\acute{\mbox{e}}$ son visage sous un mouchoir, et d'ailleurs son nez 
se trouvait en cet instant Dieu sait o$\grave{\mbox{u}}$.' [Tr.\,B.~de Schloezer (1925)]

`$\ldots$Cependant le major Kovaliov ne s'en trouvait point incommod$\acute{\mbox{e}}$\textbf{:} 
il tenait son mouchoir sur son visage, et d'ailleurs son nez se promenait$\ldots$ Dieu sait 
o$\grave{\mbox{u}}$.' [Tr.\,H.~Mongault (1938)]
\end{enumerate}
}

Тот факт, что в~примерах~(5)--(7) несколько переводчиков выбирают 
альтернативные средства выражения ЛСО, свидетельствует о~том, что эти 
средства выражают ЛСО на регулярной основе, а~не являются единичными 
переводческими ре\-ше\-ни\-ями.
{\looseness=1

}

\begin{table*}[b]\small %tabl1
\begin{center}
\Caption{Виды ПС для коннекторов русского языка в~НБД (направление перевода 
рус\-ский--фран\-цуз\-ский)}
\vspace*{2ex}

\begin{tabular}{|c|c|c|c|c|}
\hline
Всего&Конгруэнтное ПС&Дивергентное ПС&Конгруэнтно-дивергентное 
ПС&\tabcolsep=0pt\begin{tabular}{c}Эксплицитная\\ языковая\\ единица\\ отсутствует\end{tabular}\\
\hline
11\,175 
(100\%)&8\,948 
(80,07\%)&942 
(8,43\%)&167 
(1,5\%)&1\,118 
(10\%)\\
\hline
\end{tabular}
\end{center}
%\end{table*}
%\begin{table*}[b]\small %tabl2
\vspace*{8pt}
\begin{center}
\parbox{358pt}{\Caption{Наиболее употребительные дивергентные ПС, зафиксированные в~НБД 
(на\-прав\-ле\-ние перевода рус\-ский--фран\-цуз\-ский)}
}

\vspace*{2ex}

\begin{tabular}{|c|l|c|}
\hline
№&\multicolumn{1}{c|}{Средство выражения ЛСО 
в~переводе}&\tabcolsep=0pt\begin{tabular}{c}Число ПС\\ (с коннектором\\ 
в~оригинале)\end{tabular}\\
\hline
1&Придаточное определительное предложение &47\\
2&Форма деепричастия настоящего времени&35\\
3&Конструкция с~местоименным повтором&20\\
4&Форма деепричастия настоящего времени в~сочетании с~\textit{tout}&20\\
5&Форма причастия настоящего времени&17\\
$\ldots$&$\ldots$&$\ldots$\\
46\hphantom{9}&\textit{Il n'y a que$\ldots$\ qui}&\hphantom{9}1\\
\hline
&&246\hphantom{9}\\
\hline
\end{tabular}
\end{center}
\end{table*}




Промежуточное положение между коннекторами и~альтернативными им 
средствами выражения ЛСО занимают языковые единицы, пред\-став\-ля\-ющие 
собой сочетание коннектора и~лексического и/или грамматического средства. 
Так, в~(8) при переводе коннектора \textit{так как} на французский язык\linebreak 
использовано сочетание коннектора \textit{puisque} и~формы причастия 
настоящего времени. Для обо\-значения таких сочетаний используется термин 
<<комбинированные средства выражения ЛСО>>, а~переводное 
соответствие считается кон\-гру\-энт\-но-ди\-вер\-гент\-ным.
{\small
\begin{enumerate}[(1)]
\setcounter{enumi}{7}
\item Предприятия, расположенные в~городе и~севернее города, не выполнили своих 
обязательств перед государством, \textbf{так как} находятся в~районе военных 
действий. [В.\,С.~Гроссман. Жизнь и~судьба (1960)]

`Les entreprises situ$\acute{\mbox{e}}$es dans la ville, ou un peu au nord, n'avaient pu 
remplir leurs obligations envers l'$\acute{\mbox{E}}$tat, \textbf{puisque se trouvant} en 
pleine zone d'op$\acute{\mbox{e}}$rations militaires.' [Tr.\,A.~Berelowitch (1980)]
\end{enumerate}
}

Похожая группа появляется в~последней версии PDTB (3.0)~--- <<AltLex 
Relations Linked with Explicits>>~\cite[с.~80]{14-in}. Она, однако, не 
аналогична классу комбинированных средств выражения ЛСО, так как 
включает единицы, относимые нами к~коннекторам, такие как \textit{and in 
general, but in general}.

В табл.~1 приводятся данные (по состоянию на 02.03.2021) о числе 
зафиксированных в~НБД коннекторов переводных соответствий (далее~--- 
ПС), где в~русском языке (языке оригинала) исследуемая языковая единица 
является коннектором. Для сравнения: в~Пражском корпусе на <<вторичные 
коннекторы>>, при довольно широкой трактовке этого термина, 
приходится~5\%~\cite[с.~456]{12-in}, а~в~PDTB~3.0 на AltLex и~AltLexC~--- 
в~сумме чуть более~3\%~\cite[с.~5]{14-in}.
{\looseness=1

}


По данным НБД, в~дивергентных ПС коннектор чаще всего передается 
лексическими средствами, а~реже всего~--- знаками препинания. На 
грамматические средства, которые совсем не учитываются в~Пражском 
корпусе и~лишь недавно стали аннотироваться в~PDTB, приходится 
более~26\% альтернативных средств. Этим, видимо, можно объяснить более 
высокую долю дивергентных соответствий в~НБД. В~табл.~2 приводятся 
наиболее употребительные грамматические средства выражения ЛСО. На 
данный момент они не разделены на более мелкие подклассы, но фасетная 
классификация, используемая в~НБД (см.~\cite{18-in}), позволяет решить эту 
задачу.
{\looseness=-1

}


В табл.~3 сравниваются, с~одной стороны, данные о том, какие ЛСО могут 
передаваться с~использованием альтернативных средств при переводе 
с~русского языка на французский, и,~с~другой стороны, данные о том, какие 
ЛСО могут выражаться альтернативными средствами в~англоязычном 
корпусе PDTB~3.0. В~работе~\cite{14-in} отсутствуют данные об общем 
числе примеров для каждого ЛСО, поэтому в~табл.~3 приводятся только 
абсолютные цифры. В~четвертом столбце табл.~3 указано общее число 
примеров для AltLex и~AltLexC, так как обе группы соответствуют 
альтернативным средствам выражения ЛСО в~НБД. Данные из табл.~3 
показывают, что использование параллельных текстов позволяет извлечь 
знания об альтернативных средствах выражения большего числа ЛСО, чем 
анализ одноязычного материала.

\begin{table*}\small %tabl3
\begin{center}
\Caption{Альтернативные коннекторам средства выражения в~НБД и~в~PDTB}
\vspace*{2ex}

\tabcolsep=2.5pt
\begin{tabular}{|c|c|c|c|}
\hline
\textbf{Отношение в~НБД}&\tabcolsep=0pt\begin{tabular}{c}\textbf{Дивер-}\\ 
\textbf{гентных}\\ \textbf{ПС}\end{tabular}&\textbf{Отношение в~PDTB}&
\textbf{AltLex}\\
\hline
\tabcolsep=0pt\begin{tabular}{c}Исключение\\  Исключение из 
рассмотрения\end{tabular}&\tabcolsep=0pt\begin{tabular}{c}131\hphantom{9}\\ 27 
\end{tabular}&Exception.Arg2-as-excpt&\hphantom{9}3\\
\hline
<<Вопреки ожидаемому>>&47&\multicolumn{1}{|c|}{\raisebox{-36pt}[0pt][0pt]{Contrast}}&
\multicolumn{1}{|c|}{\raisebox{-36pt}[0pt][0pt]{41}}\\
Сопоставительные&35&&\\
<<Вопреки ожидаемому>> иллокутивные&19&&\\
Возместительное противопоставление&12&&\\
Противительно-уступительные иллокутивные&\hphantom{9}4&&\\
Противительно-уступительные&\hphantom{9}5&&\\
Контраст&\hphantom{9}2&&\\
Противопоставление&\hphantom{9}2&&\\
\hline
\tabcolsep=0pt\begin{tabular}{c}Аддитивные иллокутивные\\ Аддитивные пропозициональные\\
Соединительные\end{tabular}&\tabcolsep=0pt\begin{tabular}{c}45\\ 16\\ 28\end{tabular}&Conjunction&139\hphantom{9}\\
\hline
Замещение&73&\tabcolsep=0pt\begin{tabular}{c}Substitution.Arg1-as-subst;\\ 
Substitution.Arg2-as-subst\end{tabular}&29\\
\hline
Переформулирование&68&Equivalence&10\\
\hline
Спецификация&60&\tabcolsep=0pt\begin{tabular}{c}Instantiation.Arg2-as-instance;\\ 
Level-of-detail.Arg2-as-detail\end{tabular}&106\hphantom{9}\\
\hline
Временные & 53 & Asynchronous.Precedence;&\\ 
\cline{1-2}
Временные 
метаязыковые& \hphantom{9}4&\tabcolsep=0pt\begin{tabular}{c} 
Asynchronous.Succession;\\ Synchronous\end{tabular}&
\tabcolsep=0pt\begin{tabular}{c} 160\hphantom{9}\\ \ \end{tabular}\\
\hline
Пропозициональное сопутствование&42&---&\hphantom{9}0\\
\hline
Уступительные&42&\tabcolsep=0pt\begin{tabular}{c}Concession.Arg1-as-denier; 
\\Concession.Arg2-as-denier\end{tabular}&39\\
\hline
\tabcolsep=0pt\begin{tabular}{c}Условные\\ Метаязыковые 
условные\end{tabular}&\tabcolsep=0pt\begin{tabular}{c}36\\ \hphantom{9}1 
\end{tabular}&\tabcolsep=0pt\begin{tabular}{c}Condition.Arg1-as-cond; \\Condition.Arg2-as-cond\end{tabular}&74\\
\hline
Коррекция&32&---&\hphantom{9}0\\
\hline
Пропозициональная причина&32&Cause.Reason&281\hphantom{9}\\
\hline
Сравнительные&31&Similarity&63\\
\hline
Неединственности&25&---&\hphantom{9}0\\
\hline
Аналогия&18&---&\hphantom{9}0\\
\hline
Иллокутивная причина&16&Cause+Belief.Reason+Belief&\hphantom{9}6\\
\hline
Иллокутивное сопутствование&16&---&\hphantom{9}0\\
\hline
\tabcolsep=0pt\begin{tabular}{c}Пропозициональная альтернатива\\ Гипотетическая альтернатива\end{tabular}
&\tabcolsep=0pt\begin{tabular}{c}15\\ \hphantom{9}1 \end{tabular}&Disjunction&\hphantom{9}0\\
\hline
\tabcolsep=0pt\begin{tabular}{c}Экстенсиональная генерализация\\ Интенсиональная генерализация\end{tabular}
&\tabcolsep=0pt\begin{tabular}{c} \hphantom{9}7\\ \hphantom{9}5 \end{tabular}&Instantiation.Arg1-as-instance&\hphantom{9}1\\
\hline
Тождество&11&---&\hphantom{9}0\\
\hline
Несоответствие&\hphantom{9}6&---&\hphantom{9}0\\
\hline
Обобщающее переформулирование&\hphantom{9}6&Level-of-detail.Arg1-as-detail&16\\
\hline
\tabcolsep=0pt\begin{tabular}{c}Отрицательная альтернатива\\ Оговорка\end{tabular}&
\tabcolsep=0pt\begin{tabular}{c}\hphantom{9}4\\ \hphantom{9}2\end{tabular}&
%\tabcolsep=0pt\begin{tabular}{l}
Negative-condition.Arg2-as-negCond%\end{tabular}
&\hphantom{9}2\\
\hline
Следствие&\hphantom{9}4&\tabcolsep=0pt\begin{tabular}{c}Cause.Result;  Cause.negResult;\\
Cause+Belief.Result+Belief\end{tabular}&663\hphantom{9}\\
\hline
Уступительные иллокутивные&\hphantom{9}4&
%\tabcolsep=0pt\begin{tabular}{l}
Concession+SpeechAct.Arg2-as-denier+SpeechAct%\end{tabular}
&\hphantom{9}1\\
\hline
Отрицание тождества&\hphantom{9}1&---&\hphantom{9}0\\
\hline
Цель&\hphantom{9}0&\tabcolsep=0pt\begin{tabular}{c}Purpose.Arg1-as-goal; \\Purpose.Arg2-as-goal\end{tabular}&35\\
\hline
---&\hphantom{9}0&\tabcolsep=0pt\begin{tabular}{c}Manner.Arg1-as-manner;\\ Manner.Arg2-as-manner\end{tabular}&\hphantom{9}3\\
\hline
&988\hphantom{9}&&1672\hphantom{99}\\
\hline
\end{tabular}
\end{center}
\end{table*}

\section{Заключительные замечания}

Таким образом, исследование ЛСО с~использованием параллельных текстов 
и~аннотирование их показателей в~НБД позволяет, во-пер\-вых, извлекать 
новое знание о~средствах выражения ЛСО (в~том числе альтернативных 
коннекторам) и~создавать тезаурусы таких средств в~изучаемых языках;  
во-вто\-рых, на основе информации, хранящейся в~НБД, получать новые 
знания о~том, какие ЛСО могут быть выражены неспециализированными 
средствами и~какова частность их использования для каждого ЛСО в~каждом 
из изучаемых языков. Все это способно улучшить работу дискурсивных 
парсеров за счет пополнения спектра признаков, на основании которых 
принимаются решения о наличии того или иного ЛСО между фрагментами 
текста.

{\small\frenchspacing
{%\baselineskip=10.8pt
%\addcontentsline{toc}{section}{References}
\begin{thebibliography}{99}
\bibitem{1-in}
\Au{Hobbs J.\,R.} A~computational approach to discourse analyses.~--- 
New York, NY, USA: Department of Computer Science, City College, City University of New 
York, 1976.  Research Report 76-2.
\bibitem{2-in}
\Au{Hobbs J.\,R.} Why is discourse coherent?~--- Menlo Park, CA, 
USA: SRI International, 1978.  SRI Technical Note~176.
\bibitem{3-in}
\Au{Danlos L., Rysov$\acute{\mbox{a}}$~K., Rysov$\acute{\mbox{a}}$~M., Stede~M.} Primary 
and secondary discourse connectives: Definitions and lexicons~// Dialogue Discourse, 2018. 
Vol.~9. No.\,1. P.~50--78.
\bibitem{4-in}
\Au{Chistova~E.\,V., Shelmanov~A.\,O., Kobozeva~M.\,V., Pisarevskaya~D.\,B., Smirnov~I.\,V., 
Toldova~S.\,Yu.} Classification models for RST discourse parsing of texts in Russian~// 
Компьютерная лингвистика и~интеллектуальные технологии: По мат-лам ежегодной 
Междунар. конф. <<Диалог>>.~--- М.: РГГУ, 2019. Вып.~18(25). С.~163--176.
\bibitem{5-in}
\Au{Taboada M.} Discourse markers as signals (or not) of rhetorical relations~// J.~Pragmatics, 
2006. Vol.~38. No.\,4. P.~567--592.
\bibitem{6-in}
\Au{Гончаров А.\,А., Инькова~О.\,Ю.} Имплицитные логико-се\-ман\-ти\-че\-ские отношения 
и~метод их поиска в~параллельных текстах~// Компьютерная лингвистика 
и~интеллектуальные технологии: По мат-лам ежегодной Междунар. конф.  
<<Диалог>>.~--- М.: РГГУ, 2020. Вып.~19(26). С.~310--320.
\bibitem{7-in}
\Au{Зацман И.\,М., Инькова~О.\,Ю., Кружков~М.\,Г., Попкова~Н.\,А.} Представление 
кроссязыковых знаний о~коннекторах в~надкорпусных базах данных~// Информатика и~её 
применения, 2016. Т.~10. Вып.~1. С.~106--118.
\bibitem{8-in}
\Au{Зацман И., Кружков~М., Лощилова~Е.} Методы и~средства информатики для 
описания структуры неоднословных коннекторов~// Структура коннекторов и~методы ее 
описания~/ Под ред. О.\,Ю.~Иньковой.~--- М.: ТОРУС ПРЕСС, 2019. С.~205--230.
\bibitem{9-in}
Семантика коннекторов: количественные методы описания~/
Под ред.\ О.~Иньковой.~--- Bern/Berlin: Peter Lang, 2021. 276~с.
\bibitem{10-in}
\Au{Prasad R., Joshi~A., Webber~B.} Realization of discourse relations by other means: 
Alternative lexicalizations~// 23rd Conference (International) on Computational 
Linguistics Proceedings.~--- 
Beijing, China, 2010. P.~1023--1031. {\sf https://www.aclweb.org/anthology/C10-2118.pdf}.
\bibitem{11-in}
Penn Discourse Treebank Project (PDTB). {\sf https://www.\linebreak seas.upenn.edu/$\sim$pdtb}.
\bibitem{12-in}
\Au{Rysov$\acute{\mbox{a}}$~M., Rysov$\acute{\mbox{a}}$~K.} The centre and periphery of 
discourse connectives~// 28th Pacific Asia Conference on Language, Information and 
Computing Proceedings.~--- Phuket: Department of Linguistics, Chulalongkorn University, 
2014. P.~452--459. {\sf https://www.aclweb.org/ anthology/Y14-1052.pdf}.
\bibitem{13-in}
Ru-RSTreebank. Русскоязычный дискурсивный корпус. {\sf https://rstreebank.ru}.
\bibitem{14-in}
\Au{Webber B., Prasad~R., Lee~A., Joshi~A.} The Penn Discourse Treebank~3.0: Annotation 
Manual, 2019. {\sf https://\linebreak catalog.ldc.upenn.edu/docs/LDC2019T05/PDTB3-Annotation-Manual.pdf}.
\bibitem{15-in}
\Au{Das D., Taboada~M.} RST Signalling Corpus: Annotation Manual, 2014. {\sf 
https://www.sfu.ca/$\sim$mtaboada/docs/\linebreak publications/RST\_Signalling\_Corpus\_Annotation\_\linebreak Manual.pdf}.
\bibitem{16-in}
\Au{Johansson S.} Seeing through multilingual corpora: On the use of corpora in contrastive 
studies.~--- Amsterdam/Philadelphia: John Benjamins, 2007. 377~p.
\bibitem{17-in}
\Au{Инькова О.\,Ю.} Аннотирование параллельных текстов: понятие <<дивергентный 
перевод>>~// Компьютерная лингвистика и~интеллектуальные технологии: По мат-лам 
ежегодной Междунар. конф. <<Диалог>>.~--- М.: РГГУ, 2019. Вып.~18(25). С.~227--238.
\bibitem{18-in}
\Au{Зацман И.\,М., Инькова~О.\,Ю., Нуриев~В.\,А.} По\-стро\-ение классификационных схем: 
методы и~технологии экспертного формирования~// На\-уч\-но-тех\-ни\-че\-ская 
информация. Сер.~2: Информационные процессы и~сис\-те\-мы, 2017. №\,1. С.~8--22.
 \end{thebibliography}

}
}

\end{multicols}

\vspace*{-3pt}

\hfill{\small\textit{Поступила в~редакцию 06.04.2021}}

%\vspace*{8pt}

%\pagebreak

\newpage

\vspace*{-28pt}

%\hrule

%\vspace*{2pt}

%\hrule

%\vspace*{-2pt}

\def\tit{EXTRACTING KNOWLEDGE ABOUT~MEANS\\ OF~EXPRESSION 
 OF~LOGICAL-SEMANTIC RELATIONS\\ FROM~THE~SUPRACORPORA DATABASE}


\def\titkol{Extracting knowledge about~means of~expression 
 of~logical-semantic relations from~the~supracorpora database}

\def\aut{A.\,A.~Goncharov and~O.\,Yu.~Inkova}

\def\autkol{A.\,A.~Goncharov and~O.\,Yu.~Inkova}


\titel{\tit}{\aut}{\autkol}{\titkol}

\vspace*{-11pt}




\noindent
Institute of Informatics Problems, Federal Research Center ``Computer Science and Control''
 of the Russian Academy of Sciences, 44-2~Vavilov Str., Moscow 119333, Russian Federation

 
\def\leftfootline{\small{\textbf{\thepage}
\hfill INFORMATIKA I EE PRIMENENIYA~--- INFORMATICS AND
APPLICATIONS\ \ \ 2021\ \ \ volume~15\ \ \ issue\ 2}
}%
\def\rightfootline{\small{INFORMATIKA I EE PRIMENENIYA~---
INFORMATICS AND APPLICATIONS\ \ \ 2021\ \ \ volume~15\ \ \ issue\ 2
\hfill \textbf{\thepage}}}

\vspace*{3pt}




\Abste{The goal of this paper is to demonstrate how parallel texts annotated with a supracorpora 
database (SCDB) can be efficiently used to extract knowledge about alternative means of 
expression of logical-semantic relations (LSR). The authors review the most prominent 
discursively annotated corpora (Penn Discourse Treebank, Prague Dependency Treebank, 
and Rhetorical Structure Theory Discourse Treebank) to support the observation that 
there is no consensus among the researchers as to which linguistic means are to be considered 
connectives (i.\,e., prototypical markers of LSR) and which means are 
deemed ``alternative.'' The research shows that application of the comparative method while 
leveraging the capabilities of the SCDB of connectives makes it possible not only to extract new 
knowledge about LSR markers but also to create thesauri of various means of LSR expression in 
the languages involved, including the alternative ones. In addition, the SCDB data makes it 
possible to generate new knowledge on correlations between specific LSRs and unconventional 
means of LSR expression and calculate frequencies of utilization of these means for the studied 
languages.}

\KWE{supracorpora database; logical-semantic relations; connectives; knowledge generation; 
parallel texts}

\DOI{10.14357/19922264210214}

%\vspace*{-15pt}

% \Ack
%\noindent


\vspace*{3pt}

  \begin{multicols}{2}

\renewcommand{\bibname}{\protect\rmfamily References}
%\renewcommand{\bibname}{\large\protect\rm References}

{\small\frenchspacing
 {%\baselineskip=10.8pt
 \addcontentsline{toc}{section}{References}
 \begin{thebibliography}{99}
 
 \vspace*{-3pt}
 
\bibitem{1-in-1}
\Aue{Hobbs, J.\,R.} 1976. A~computational approach to discourse analyses.  New York, NY: Department of Computer Science, City College, City University of New 
York.  Research Report  
76-2.
\bibitem{2-in-1}
\Aue{Hobbs, J.\,R.} 1978. Why is discourse coherent?  Menlo Park, 
CA: SRI International. SRI Technical Note~176.
\bibitem{3-in-1}
\Aue{Danlos, L., K.~Rysov$\acute{\mbox{a}}$, M.~Rysov$\acute{\mbox{a}}$, and 
M.~Stede.} 2018. Primary and secondary discourse connectives: Definitions and lexicons. 
\textit{Dialogue Discourse} 9(1):50--78.
\bibitem{4-in-1}
\Aue{Chistova, E.\,V., A.\,O.~Shelmanov, M.\,V.~Kobozeva, D.\,B.~Pisarevskaya, 
I.\,V.~Smirnov, and S.\,Yu.~Toldova.} 2019. Classification models for RST discourse parsing of 
texts in Russian. \textit{Komp'yuternaya lingvistika i~intellektual'nye tekhnologii: po mat-lam 
Mezhdunar. konf. ``Dialog''} [Computational Linguistics and Intellectual Technologies: Papers 
from the Annual Conference (International) ``Dialogue'']. Moscow: RSHI. 18(25):163--176.
\bibitem{5-in-1}
\Aue{Taboada, M.} 2006. Discourse markers as signals (or not) of rhetorical relations. 
\textit{J.~Pragmatics} 38(4):567--592.
\bibitem{6-in-1}
\Aue{Goncharov, A.\,A., and O.\,Yu.~Inkova.} 2020. Implitsitnye logiko-semanticheskie 
otnosheniya i~metod ikh poiska v~parallel'nykh tekstakh [Implicit logical-semantic relations and 
a~method of their identification in parallel texts]. \textit{Komp'yuternaya lingvistika 
i~intellektual'nye tekhnologii: po mat-lam Mezhdunar. konf. ``Dialog''} [Computational 
Linguistics and Intellectual Technologies: Papers from the Annual Conference (International) 
``Dialogue'']. Moscow: RSHI. 19(26):310--320.
\bibitem{7-in-1}
\Aue{Zatsman, I.\,M., O.\,Yu.~Inkova, M.\,G.~Kruzhkov, and N.\,A.~Popkova.} 2016. 
Predstavlenie kross-yazykovykh znaniy o~konnektorakh v~nadkorpusnykh bazakh dannykh 
[Representation of cross-lingual knowledge about connectors in suprocorpora databases]. 
\textit{Informatika i~ee Primeneniya~--- Inform. Appl.} 10(1):106--118.
\bibitem{8-in-1}
\Aue{Zatsman, I., M.~Kruzhkov, and E.~Loshchilova.} 2019. Metody i~sredstva informatiki 
dlya opisaniya struktury neodnoslovnykh konnektorov [Methods and means of informatics for 
multiword connectives structure description]. \textit{Struktura konnektorov i~metody ee 
opisaniya} [Connectives structure and methods of its description]. Ed. O.\,Yu.~Inkova. Moscow: 
TORUS PRESS. 205--230.
\bibitem{9-in-1}
Inkova, O., ed. 2021. \textit{Semantika konnektorov: kolichestvennye metody opisaniya} 
[Semantics of connectives: Quantitative methods of analysis]. Bern/Berlin: Peter Lang. 276~p.
\bibitem{10-in-1}
\Aue{Prasad, R., A.~Joshi, and B.~Webber.} 2010. Realization of discourse relations by other 
means: Alternative lexicalizations. \textit{23rd Conference (International) on Computational 
Linguistics Proceedings}. Beijing, China. 1023--1031.
Available at: {\sf https://www.aclweb.org/anthology/C10-2118.pdf}
(accessed June~15, 2021).
\bibitem{11-in-1}
Penn Discourse Treebank Project. Available at: {\sf https:// www.seas.upenn.edu/$\sim$pdtb/} 
(accessed May~19, 2021).
\bibitem{12-in-1}
\Aue{Rysov$\acute{\mbox{a}}$, M., and K.~Rysov$\acute{\mbox{a}}$.} 2014. The centre and 
periphery of discourse connectives. \textit{28th Pacific Asia Conference on Language, 
Information and Computing Proceedings}. Phuket: Department of Linguistics, Chulalongkorn 
University. 452--459. 
Available at: {\sf https://www.aclweb.org/\linebreak anthology/Y14-1052.pdf}
(accessed June~15, 2021).
\bibitem{13-in-1}
Ru-RSTreebank: Russkoyazychnyy diskursivnyy korpus [Ru-RSTreebank: Russian discourse 
corpus]. Available at: {\sf https://rstreebank.ru/} (accessed May~19, 2021)
\bibitem{14-in-1}
\Aue{Webber, B., R.~Prasad, A.~Lee, and A.~Joshi.} 2019. The Penn Discourse Treebank~3.0: 
Annotation manual. Available at: {\sf  
https://catalog.ldc.upenn.edu/docs/\linebreak LDC2019T05/PDTB3-Annotation-Manual.pdf} (accessed 
May~19, 2021)
\bibitem{15-in-1}
\Aue{Das, D., and M.~Taboada.} 2014. RST signalling corpus: Annotation manual. Available 
at:  {\sf https://www.\linebreak sfu.ca/$\sim$mtaboada/docs/publications/RST\_Signalling\_\linebreak Corpus\_Annotation\_Manual.pdf} 
(accessed May~19, 2021)
\bibitem{16-in-1}
\Aue{Johansson, S.} 2007. \textit{Seeing through multilingual corpora: On the use of corpora in 
contrastive studies}. Amsterdam/Philadelphia: John Benjamins. 377~p.
\bibitem{17-in-1}
\Aue{Inkova, O.\,Yu.} 2019. Annotirovanie parallel'nykh tekstov: ponyatie ``divergentnyy 
perevod'' [Annotation of parallel texts: The concept of divergent translation]. 
\textit{Komp'yuternaya lingvistika i~intellektual'nye tekhnologii: po mat-lam Mezhdunar. konf. 
``Dialog''} [Computational Linguistics and Intellectual Technologies: Papers from the Annual 
Conference (International) ``Dialogue'']. Moscow: RSHI. 18(25):227--238.
\bibitem{18-in-1}
\Aue{Zatsman, I., O.~Inkova, and V.~Nuriev.} 2017. The construction of classification schemes: 
Methods and technologies of expert formation. \textit{Autom. Doc. Math. 
Linguist.} 51(1):27--41.
\end{thebibliography}

 }
 }

\end{multicols}

\vspace*{-3pt}

  \hfill{\small\textit{Received April~6, 2021}}


%\pagebreak

%\vspace*{-8pt}  

\Contr

\noindent
\textbf{Goncharov Alexander A.} (b.\ 1994)~--- junior scientist, Institute of Informatics 
Problems, Federal Research Center``Computer Science and Control'' of the Russian Academy of 
Sciences, 44-2~Vavilov Str., Moscow 119333, Russian Federation; 
\mbox{a.gonch48@gmail.com}

\vspace*{6pt}

\noindent
\textbf{Inkova Olga Yu.} (b.\ 1965)~--- Doctor of Science  in philology, senior scientist, 
Institute of Informatics Problems, Federal Research Center ``Computer Science and Control'' of 
the Russian Academy of Sciences, 44-2~Vavilov Str., Moscow 119333, Russian Federation; 
\mbox{olyainkova@yandex.ru}

\label{end\stat}

\renewcommand{\bibname}{\protect\rm Литература} %14
\def\stat{nuriev-egorova}

\def\tit{МЕТОДЫ ОЦЕНКИ КАЧЕСТВА МАШИННОГО ПЕРЕВОДА: СОВРЕМЕННОЕ 
СОСТОЯНИЕ}

\def\titkol{Методы оценки качества машинного перевода: современное 
состояние}

\def\aut{В.\,А.~Нуриев$^1$, А.\,Ю.~Егорова$^2$}

\def\autkol{В.\,А.~Нуриев, А.\,Ю.~Егорова}

\titel{\tit}{\aut}{\autkol}{\titkol}

\index{Нуриев В.\,А.}
\index{Егорова А.\,Ю.} 
\index{Nuriev V.\,A.}
\index{Egorova A.\,Yu.}

%{\renewcommand{\thefootnote}{\fnsymbol{footnote}} \footnotetext[1]
%{Работа выполнена в~Институте проблем информатики Федерального исследовательского центра
%<<Информатика и~управление>> Российской академии наук.}}


\renewcommand{\thefootnote}{\arabic{footnote}}
\footnotetext[1]{Институт проблем информатики Федерального исследовательского центра <<Информатика 
и~управление>> Российской академии наук, \mbox{nurieff.v@gmail.com}}
\footnotetext[2]{Институт проблем информатики Федерального исследовательского центра <<Информатика 
и~управление>> Российской академии наук, \mbox{ann.shurova@gmail.com}}

\vspace*{-10pt}
  
    
  
  \Abst{Представлен обзор современных методов оценки качества машинного 
перевода (МП). В~основе этих методов лежат два подхода~--- автоматический и~экспертный. 
Автоматическая оценка построена на сопоставлении с~референтным  
(про\-фес\-сио\-наль\-ным/эта\-лон\-ным) переводом (РП). Экспертная (с~привлечением 
че\-ло\-ве\-ка-экс\-пер\-та) учитывает в~первую очередь функциональность: качество перевода 
оценивается в~праг\-ма\-ти\-ко-функ\-цио\-наль\-ном аспекте, т.\,е.\ принимается во 
внимание, насколько полученный перевод справляется со своими задачами. В~первой 
части статьи рассматривается ряд метрик, исполь\-зу\-емых для автоматической оценки 
МП, отмечаются их недостатки и~описываются новые направления в~их 
разработке. Вторая часть статьи сфокусирована на экспертной оценке 
МП. Здесь приведены несколько основных способов такой оценки: оценивание 
в~соответствии с~критериями точности и~естественности, ранжирование переводов, 
прямое оценивание, оценка с~учетом коэффициента редактирования перевода человеком, 
аннотирование перевода с~применением типологии ошибок.}
  
  \KW{машинный перевод; качество перевода; оценка качества машинного перевода; 
автоматические метрики; прямое оценивание; типология ошибок машинного перевода}

\DOI{10.14357/19922264210215}

\vspace*{-6pt}


\vskip 10pt plus 9pt minus 6pt

\thispagestyle{headings}

\begin{multicols}{2}

\label{st\stat}
  
  \section{Введение}
  
  \vspace*{-3pt}
  
  Проблема и, следовательно, необходимость оценки качества переводного 
текста возникает на регулярной основе, причем не только 
в~профессиональном сообществе, но и~в жизни обычного\linebreak человека. Во 
многом это связано с~тем, что МП стал неотъемлемой 
частью повседневной ре\-аль\-ности. Происходит реструктуризация рынка 
переводческих услуг и,~в~част\-ности, МП, а~бюджет индустрии постоянно 
наращивает объемы (см.\ об этом~[1, с.~260--263]). Ежедневно с~по\-мощью 
пуб\-лич\-но доступного веб-сер\-ви\-са нейронного МП Google.Translate 
обрабатывается около 143~млрд слов в~100~языковых парах~[2]. Человек 
\mbox{использует} МП для решения задач широкого профиля: для получения 
информации, связанной с~конкретной и~требующей незамедлительных 
действий проблемой (перевод технического со\-про\-вож\-де\-ния, инструкции 
к~лекарствам и~т.\,д.); для покупок на зарубежных сайтах; в~целях 
оптимизации \mbox{профессиональной} деятельности переводчика с~по\-мощью 
внедрения в~его рабочий цикл этапа, предполагающего по\-сле\-ду\-ющую 
редактуру автоматически сгенерированного текста. Не все указанные задачи 
предполагают обязательный высокий уровень качества МП. Для 
осуществления ряда из них достаточно общего понимания содержания даже 
при наличии несущественных ошибок, на\-ру\-ша\-ющих правила целевого языка. 
Для выполнения других задач требуется высокое качество полученного 
автоматическим способом перевода, что указывает на необходимость 
постоянно оценивать динамику этого качества, изучать и~совершенствовать 
методы его оценки. Этим обусловлено повышенное внимание научного 
сообщества к~данной проб\-ле\-ме: за последние четыре года были 
опубликованы несколько авторитетных монографий, 
фо\-ку\-си\-ру\-ющих\-ся на оценке качества МП~[3--5].
  
  Целью статьи, таким образом, ставится обзор современных тенденций 
в~разработке методов оценки качества МП. В~основе этих методов лежат 
два подхода~--- автоматический и~экспертный. Автоматическая оценка 
построена на сопоставлении с~референтным
(про\-фес\-сио\-наль\-ным/эта\-лон\-ным) переводом (\textit{англ}.\ reference translation). Экспертная 
(с~привлечением че\-ло\-ве\-ка-экс\-пер\-та) оценка учитывает в~первую 
очередь функциональность: качество перевода оценивается тем выше, чем 
успешнее он справляется со своими задачами.

\vspace*{-9pt}
  
  \section{Автоматическая оценка качества машинного перевода}
  
  Как правило, автоматическая оценка измеряет уровень соответствия МП 
одному или нескольким РП. Чтобы определить уровень соответствия МП 
и~РП, применяются критерии точности (доля правильно переведенного) 
и~полноты (доля переведенных слов, совпадающих с~профессиональным 
переводом). Ниже представлены некоторые метрики автоматической оценки 
качества МП.
  
  Метрика, к~которой обращаются чаще всего,~--- это разработанная в~IBM 
метрика BLEU (Bilingual Evaluation Understudy)~\cite{6-nur}. Она вошла 
в~золотой стандарт автоматической оценки качества МП и~нередко 
применяется в~качестве эталонной. Сопоставление МП и~РП проводится 
путем вы\-чис\-ле\-ния $n$-грам\-мной точ\-ности (максимальная длина  
$n$-грам\-мно\-го блока слов равна~4). Чтобы избежать искажения в~оценке, 
за слишком короткий перевод назначается штраф (brevity penalty~--- BP),  
$n$-грам\-мная точ\-ность при этом представляет собой <<отношение 
последовательностей из~$n$ слов, совпадающих в~МП и~РП, к~общему числу 
последовательностей из~$n$~слов в~МП>>~\cite[с.~111--112]{7-nur}. 
<<Оценка$\ldots$ вычисляется как произведение среднего геометрического 
из полученных модифицированных коэффициентов и~штрафного 
коэффициента>>~\cite[с.~86]{8-nur}. Полученное значение BLEU изменяется 
в~пределах от~0 до~1. При процентном представлении значение изменяется 
в~промежутке от~0\% до~100\%. Вычисляется метрика по следующей 
формуле:
  \begin{multline*}
  \mathrm{BLEU} =\mathrm{BP}\exp \left( \sum\limits^n_{n=1} w_i \log 
p_i\right) \!,\\
 \mathrm{BP} =\min\left( 1,\fr{c}{r}\right),
  \end{multline*}
  <<где $w_i$~--- положительные веса для каждого используемого параметра  
$n$-грамм$\ldots$ $n$~--- максимальная длина $n$-грамм, $i$~--- длина 
блока в~пределах $n$-грам\-мы, $p_i$~--- модифицированная точность  
$n$-грамм, $c$~--- длина полученного машинного перевода, $r$~--- длина 
наилучшего совпадающего эталонного текста>>~\cite[с.~86]{8-nur}. Мет\-ри\-ка BLEU 
использует статистические инструменты, не принимая во внимание 
лингвистические знания.
  
  Другая наиболее востребованная метрика~--- \mbox{METEOR} (Metric for Evaluation 
of Translation with Explicit Ordering)~--- предусмат\-ри\-ва\-ет интеграцию языковых 
знаний. Так, наряду с~$n$-грам\-мны\-ми совпадениями в~МП и~РП она 
учитывает изменения в~словоформах, синонимические ряды  
и~т.\,д.~\cite{9-nur}. Поэтому для обеспечения ее функционирования 
необходимо привлечение баз данных, содержащих лингвистическую 
информацию, нужна морфологическая разметка и~вычислительно затратное 
пословное выравнивание. Иначе говоря, требуется сложная и~тонкая 
настройка, куда вовлечено гораздо больше параметров, чем в~BLEU.
  
  Еще одной получившей широкое распространение метрикой 
автоматической оценки качества МП стала TER (Translation Error Rate). Она 
исходит из расчета ис\-прав\-ле\-ний/транс\-фор\-ма\-ций, необходимых для 
приведения МП к~эталонному образцу, и~вычисляется по следующей 
формуле:
  $$
  \mathrm{TER} =\fr{\mathrm{Число\ редактирований}}{\mathrm{Средняя\ 
длина\ эталонных\ переводов}}\,.
  $$
    При этом пунктуационные знаки принимаются за отдельные слова, 
а~трансформациями считаются не только удаление, вставка и~замена, но 
и~перестановка~--- в~отличие, например, от метрики WER (Word Error Rate), 
которая эту последнюю трансформацию не учитывает (подробнее о~TER 
и~WER см.\ в~\cite{8-nur, 10-nur}).
  
  Наряду с~рассмотренными метриками автоматической оценки качества МП 
также имеются: PER (Position-Independent Word Error Rate), chrF  
(Character F-measure), NIST (название образовано от US National Institute of 
Standards and Technology) и~др.\ (о~них  
см.~\cite{7-nur, 8-nur, 11-nur, 12-nur}).
  
  Целесообразность использования метрик автоматической оценки качества 
МП постоянно ставится под вопрос. Действительно, может ли метрика 
с~простейшим алгоритмом вычисления типа BLEU (как и~другие метрики) 
адекватно отражать отличия между МП и~РП? Основные критические 
замечания, высказываемые в~этой связи, заключаются в~следующем.
  \begin{enumerate}[1.]
  \item При вычислении не принимается во внимание, что слова несут на 
себе разную функциональную нагрузку и~имеют неодинаковую 
релевантность для формирования предложения.
  \item Сравнение РП и~МП носит локальный характер и~проводится на 
уровне $n$-грам\-мно\-го соответствия, при этом упускается из виду 
грамматическая связность в~рамках всего предложения, что искажает 
результаты в~пользу систем МП, которые лучше переводят отдельные 
словарные блоки, но не всегда способны грамматически правильно оформить 
целое предложение.
  \item Вычисляемые значения не информативны: неизвестно, как 
интерпретировать значение BLEU, равное, например, 30,7\%, так как при 
вычислении задействовано множество факторов~--- число РП, языковая пара, 
терминологическое наполнение текста, схема токенизации, используемая для 
вычленения слов в~РП и~МП.
  \item Ненадежность алгоритма оценки. Так, недавние эксперименты 
показали, что BLEU оценивает выполненные человеком переводы на том же 
уровне, что и~машинные, хотя последние имеют гораздо худшее качество. 
В~ходе этих экспериментов с~помощью BLEU выполнялось сравнение между 
несколькими РП, а~также между РП и~МП.
  \end{enumerate}
  
  Эти недостатки необходимо учитывать в~разработке новых метрик 
автоматической оценки МП, как необходимо учитывать и~то, что 
в~идеальном случае оценка, получаемая с~помощью такой мет\-ри\-ки, должна 
демонстрировать явную корреляцию с~оценкой че\-ло\-ве\-ка-экс\-пер\-та. 
Обычно эту корреляцию рассчитывают с~помощью коэффициента 
корреляции Пирсона~\cite[с.~61]{10-nur}. Его значение варьируется от~0 
до~1, и~чем оно выше в~указанном диапазоне, тем лучше метрика.
  
  Автоматические метрики получили всеобщее признание в~качестве 
эффективного способа для оценки продуктивности систем статистического 
МП, однако они не совсем приспособлены, чтобы сравнивать 
производительность систем МП разного типа между собой, и~в этом 
отношении разработки средств автоматической оценки МП пока не достигли 
сколь-нибудь значимых результатов. Вместе с~тем такие разработки 
интенсивно ведутся, и~автоматические метрики постоянно 
совершенствуются.
  
  Так, все большее распространение получает подход, учитывающий при 
сопоставлении МП и~РП морфологические, синтаксические и~семантические 
параметры. Это, например, MEANT, где сопоставляются синтаксические 
древовидные структуры и~принимаются во внимание такие свойства, как 
семантические роли~\cite{13-nur}. Или RIBES (Rank-based Intuitive Bilingual 
Evaluation Score)~--- метрика, специально разработанная для языковых пар 
типа япон\-ский--анг\-лий\-ский, где коренным образом различается 
синтаксическое устройство.
  
  Имеются попытки применять машинное обуче\-ние~--- обучать метрики на 
данных, полученных по результатам оценки че\-ло\-ве\-ком-экс\-пер\-том (см., 
например, BEER (BEtter Evaluation as Ranking)~[14, 15] или \mbox{BLEURT} 
(Bilingual Evaluation Understudy with Representations from Transformers)~--- 
одну из самых новых мет\-рик, которая использует нейросетевую языковую 
модель \mbox{BEURT} и~обуча\-ет\-ся на рейтинговых данных~[16]).
  
  Наряду с~увеличением степени корреляции между оценкой 
  человека-эксперта и~автоматической оценкой МП разработчики также стремятся 
обеспечить большую информативность метрик (см.\ выше замечание 
о~непрозрачности значения BLEU) и~снижение вычислительной 
трудоемкости.
  
  Важным сейчас становится создание информационных ресурсов, 
содержащих лингвистические знания разной направленности, 
предназначенные для обучения современных автоматизированных метрик. 
Примером таких ресурсов служат надкорпусные базы данных, 
разрабатываемые в~отделе~54 ФИЦ ИУ РАН~[17].
  
  Подробнее о новейших разработках в~области автоматизированных 
средств оценки качества МП см.~\cite[с.~59--64]{10-nur}.
  
  \section{Экспертная оценка качества машинного перевода}
  
  Говоря об экспертной оценке качества МП с~привлечением специалистов 
(лингвистов, переводчиков), можно выделить несколько основных способов: 
оценивание в~соответствии с~критериями точности и~естественности, 
ранжирование переводов, прямое оценивание, оценка с~учетом коэффициента 
редактирования перевода человеком, аннотирование перевода с~применением 
типологии ошибок.
  
  Понятие <<правильности>> перевода недоопределено и, следовательно, 
плохо применимо. Вот почему для оценки качества МП руководствуются 
критериями\footnote{Другие возможные критерии оценки описаны в~\cite{18-nur}.} 
точности (adequacy) и~естественности (fluency), используя в~опросе 
экспертов 5-балль\-ную шкалу Ликерта~[19]. Такой подход имеет свои 
недостатки: эксперты не всегда последовательны в~своем выборе из-за 
неоднозначности определений в~шкале оценки, к~тому же одни специалисты 
более снисходительны при назначении оценок, чем другие.
  
  Чтобы избежать этих трудностей, при оценке двух и~более систем МП 
применяется ранжирование переводов относительно друг друга. Для 
измерения меры согласия между экспертами используют коэффициенты 
каппа Коэна~\cite[с.~48]{10-nur}, каппа Флейса~\cite{20-nur}.
  
  Так, в~работе~[21] ранжирование проводилось для оценки качества 
переводов, реализованных посредством системы статистического фразового 
МП (СФМП) и~системы нейронного МП
(НМП). В~ходе эксперимента каждому эксперту были представлены 
триплеты, состоящие из предложения на исходном языке и~двух его 
переводов (полученных с~помощью СФМП и~НМП). Экспертам предлагалось 
оценить триплет и~приписать его к~одному из трех классов, показывающих 
соотношение качества сравниваемых переводов: 
$$
\mathrm{СФМП} = 
\mathrm{НМП}\,;\  \mathrm{СФМП} <  \mathrm{НМП};\  \mathrm{СФМП}> 
\mathrm{НМП}\,.
$$
 Полученные экспертные оценки были сопоставлены 
с~результатами автоматических метрик (BLEU, TER, Character F-measure).
  
  В эксперименте, описанном в~[22], помимо ранжирования еще 
задействованы постредактирование переводов, экспертная аннотация ошибок 
в~МП, а~также оценка точности/естественности. Подобно~[21], для 
установления корреляции между экспертной и~автоматической оценкой 
используются автоматические метрики.
  
  Одной из последних разработок в~области оценки качества МП является 
прямое оценивание (direct assessment)~\cite[с.~49--50]{10-nur}. Оно 
предполагает оценку одного предложения единовременно (в~отличие от 
ранжирования переводов) с~применением 100-балль\-ной шкалы, которая 
имеет вид немаркированной прямой с~бегунком. Для экспертов характерны 
неодинаковые ожидания в~отношении качества МП: одни склонны его 
оценивать выше, а другие, наоборот, ниже, что может объясняться 
имеющимися предубеждениями о низком качестве МП. Кроме того, разными 
экспертами 5-балль\-ная шкала используется неравномерно~--- некоторые 
никогда не ставят самый низкий и~самый высокий баллы. 100-балль\-ная 
шкала представляет собой более гибкий оценочный инструмент. Она дает 
возможность измерить ожидания в~отношении качества МП у~каждого 
эксперта с~помощью среднего балла всех его оценок, выявляя 
задействованный интервал шкалы, который отражается в~дисперсии оценок. 
Оценки разных экспертов нормируются согласно формулам в~\cite[с.~49--50]{10-nur}. 
Переводы, поступающие эксперту для обработки, генерируются в~разных 
системах МП и~выбираются случайным образом. После нормирования 
оценок, полученных от каждого из экспертов, вычисляется средний балл для 
переводов отдельно взятой системы~МП.

  
  Прямое оценивание было использовано в~ходе краудсорсинговой кампании 
по оценке качества МП, организованной ACL (Association for Computational 
Linguistics) в~2018~г.\ в~рамках Конференции по компьютерной лингвистике 
(Workshop on Machine Translation, WMT).
  
  Оценивать качество перевода можно и~с~точки зрения усилий по его 
постредактированию. Так, при оценке МП с~учетом 
HTER\footnote{Аббревиатура совпадает с~названием автоматической метрики HTER  
(Human-targeted Translation Error Rate) (подробнее см.~\cite[с.~25]{18-nur}).} (Human 
Translation Edit Rate~--- коэффициент редактирования перевода 
человеком)~\cite[с.~51--52]{10-nur} эксперты получают подборку переводов, 
выполненных разными системами МП, которые им предлагается 
отредактировать. Затем для каждой системы МП проводится сопоставление 
перевода с~его отредактированной версией и~подсчитывается число 
изменений, сделанных экспертом.
  
  Качество МП может оцениваться и~в процессе аннотирования перевода 
с~применением типологии ошибок. Обзор классификаций представлен 
в~работе~[23].
  
  Одной из наиболее известных является типология DQF/MQM (Dynamic 
Quality Framework~--- динамическая модель оценки качества; 
Multidimensional Quality Metrics~--- многомерные метрики\linebreak \mbox{качества}), 
разработанная в~TAUS\footnote{Translation Automation User Society~--- Пользовательское 
сообщество по автоматизации перевода.} и~DFKI\footnote{Deutsches Forschungszentrum 
f$\ddot{\mbox{u}}$r K$\ddot{\mbox{u}}$nstliche Intelligenz~--- Немецкий центр исследований 
искусственного интеллекта.} в~2014~г.~[24]. Типология имеет 4~уровня: наиболее 
специфицированные типы ошибок относятся к~четвертому уровню; при этом 
при оценке перевода можно выбирать степень спецификации, т.\,е.\ 
использовать от одного до четырех уровней в~зависимости от задачи. Также 
в~типологии учитываются четыре степени критичности ошибок. Подробнее 
о~MQM-метриках см.\ в~работе~\cite{7-nur}.
  
  Типология DQF/MQM получила широкое распространение. Так, в~[25] она 
применяется для проведения количественного анализа работы разных систем 
МП. При этом классификация претерпевает ряд изменений, обусловленных 
необходимостью учитывать особенности славянских языков (в данном 
случае хорватского).
  
  Следует отметить, что типологии ошибок могут быть специфицированы 
в~зависимости от цели исследования. Так, по мнению авторов статьи~[26], 
категория <<Терминология>> в~классификации MQM не отражает нюансы, 
которые могут возникать при ошибочном переводе терминов. В~работе 
предпринята попытка провести анализ ошибок в~переводе терминов, 
уточнить их классификацию и~сопоставить на этой основе работу систем 
СФМП и~НМП.
  
  Представленная в~[26] типология ошибок включает в~себя 5~классов: 
\begin{enumerate}[(1)]
\item <<Ошибка в~словопорядке>> (Reorder error);\\[-10pt] 
\item <<Ошибка 
в~формообразовании>> (Inflectional error);\\[-10pt] 
\item <<Ошибка в~части термина>> 
(Partial error); \\[-10pt]
\item <<Лексическая ошибка>> (Incorrect lexical selection); \\[-10pt]
\item <<Пропуск термина>> (Term drop).
\end{enumerate}
 Оставшиеся виды ошибок образуют 
6-й класс, который подразделяется на три подкласса: 
\begin{itemize}
\item <<Копирование 
исходного термина>> (Source term copied); \\[-10pt]
\item <<Ошибка, вызванная 
затруднением при снятии многозначности слова на целевом языке>> 
(Disambiguation issue in target); 
\item <<Другие ошибки>> (Other error).
\end{itemize}

 Переводы 
исходных терминов, в~которых не было допущено ошибок, объединены 
в~отдельный класс <<Правильного перевода>> (Correct translation). В~нем 
авторы исследования выделяют еще~7~подклассов, которые 
демонстрируют разнообразие моделей перевода и~отображают степень 
соответствия переводного эквивалента исходному термину.
  
  Еще одним примером типологии ошибок может послужить 
классификация, представленная в~[27]. Она подробно описана в~работе~[28]. 
Эта типология имеет~5~укрупненных классов ошибок, которые, в~свою 
очередь, делятся на подклассы.
  
  В работе~[29] проводится количественный анализ ошибок 
мультимодальных систем НМП, способных обрабатывать изображения. За 
основу взята классификация ошибок~\cite{27-nur} с~некоторыми 
уточнениями. Изменения в~ней обусловлены интересом авторов 
исследования к~тому, как мультимодальные системы НМП переводят 
<<визуальные>> термины (visual terms)~--- термины, обозначающие понятия, 
прямое соответствие которым можно найти на предъявляемом изображении, 
причем задействованы только укрупненные классы типологии 
ошибок~\cite{27-nur} без уточнения их дальнейшего иерархического 
устройства.
  
  С учетом всех преобразований модифицированная классификация~[29] 
включает в~себя следующие классы ошибок: 
\begin{enumerate}[(1)]
\item <<Пропущенные слова>> 
(Missing words); 
\item <<Неправильные слова>> (Incorrect words), куда входят 
подклассы
\begin{itemize}
\item  <<Неправильный перевод>> (Mistranslation);
\item ~<<Неправильная 
форма, лишние слова, стилистическая ошибка>> (Incorrect form, extra words 
or style);
\end{itemize} 
\item <<Другие ошибки>> (Other), куда входят подклассы 
\begin{itemize}
\item <<Словопорядок>> (Word order); 
\item <<Неизвестные слова>> (Unknown words); 
\item <<Пунктуационная ошибка>> (Punctuation).
\end{itemize}
\end{enumerate}
  
  Также был добавлен новый, 4-й класс, получивший название 
<<Визуальная категория>> (Visual category). В~него входят 
4~подкласса: 
\begin{itemize}
\item <<Правильный перевод>> (Correct); 
\item <<Неправильный перевод>> 
(Mistranslation); 
\item <<Неправильный, но интересный перевод>> (Incorrect but 
interesting); 
\item <<Новый термин>> (Novel).
\end{itemize}

  
  Разнообразие способов экспертной оценки МП свидетельствует 
о~неослабевающем интересе профессионального сообщества к~этой области, 
а также говорит о том, что даже с~учетом существенно меньшей стоимости 
и~большей быстроты автоматической оценке не доверяют полностью 
и~стремятся проверить ее с~помощью мнения компетентного человека.
  
  \section{Заключение}
  
  В статье представлен обзор современных подходов к~оценке качества 
МП. Выделены два основных направления: 
автоматизированная оценка и~оценивание с~привлечением че\-ло\-ве\-ка-экс\-пер\-та. 
С~изменением парадигмы МП и~внед\-ре\-ни\-ем нейросетей в~архитектуру 
автоматических переводчиков изменяются и~разработки в~области оценки 
качества МП. Это затрагивает в~первую очередь автоматические метрики, 
используемые для оценивания переводов: для обеспечения их работы 
пытаются применять машинное обучение. В~качестве тренировочных 
привлекаются данные, полученные по результатам оценки человеком. 
Нововведения имеются и~в~об\-ласти экспертной оценки качества МП. Одна 
из последних разработок здесь~--- прямое оценивание. Востребованным 
остается аннотирование МП с~применением типологии ошибок. Оно стало 
одним из самых продуктивных способов оценивания, поскольку позволяет 
гибко типологизировать ошибки в~соответствии с~целым рядом параметров, 
которые легко варьировать в~зависимости от конкретных характеристик 
текста, поступающего на вход системы МП.
  
{\small\frenchspacing
{%\baselineskip=10.8pt
%\addcontentsline{toc}{section}{References}
\begin{thebibliography}{99}
\bibitem{1-nur}
\Au{Larsonneur C.} Neural machine translation: From commodity to commons?~//
 When  translation goes digital: Case studies and critical reflections~/ Eds. R.~Desjardins, 
C.~Larsonneur, Ph.~Lacour.~--- Cham, Switzerland: Palgrave Macmillan, 2021. P.~257--280.
\bibitem{2-nur}
\Au{Davenport C.} Google Translate processes 143~billion words every day~// Android Police, 
2018. {\sf https://\linebreak www.androidpolice.com/2018/10/09/google-translate- processes-143-billion-words-every-day}.
\bibitem{3-nur}
Translation quality assessment: From principles to practice~/ Eds. J.~Moorkens, 
Sh.~Castilho, F.~Gaspari, S.~Doherty.~--- Machine translation: Technologies and applications ser.~---
Cham, Switzerland: Springer International Publishing,  2018. 
Vol.~1. 292~p.
\bibitem{4-nur}
\Au{Specia L., Scarton~C., Paetzold~G.\,H.} Quality estimation for machine translation.~--- Synthesis lectures on
human language technologies ser.~--- London: Morgan \& Claypool, 2018. 162~p.
\bibitem{5-nur}
\Au{Bittner H.} Evaluating the evaluator: A~novel perspective on translation quality 
assessment.~--- New York, NY, USA: Routledge, 2020. 282~p.
\bibitem{6-nur}
\Au{Papineni K., Roukos~S., Ward~T., Zhu~W.\,J.} BLEU: A~method for automatic evaluation of 
machine translation~// 40th Annual Meeting on Association for Computational Linguistics 
Proceedings.~--- Philadelphia, PA, USA: Association for Computational Linguistics, 2002. 
P.~311--318.
\bibitem{7-nur}
\Au{Рычихин А.\,К.} О~методах оценки качества машинного перевода~// Системы 
и~средства информатики, 2019. Т.~29. №\,4. С.~106--118.
\bibitem{8-nur}
\Au{Козина А.\,В., Черепков~Е.\,А., Белов~Ю.\,С.} Автоматические метрики оценки качества 
машинного перевода~// Системный администратор, 2019. №\,11. С.~84--87.
\bibitem{9-nur}
\Au{Banerjee S., Lavie~A.} METEOR: An automatic metric for MT evaluation with improved 
correlation with human judgments~// Workshop on Intrinsic and Extrinsic 
Evaluation Measures for MT and/or Summarization at the 43rd Annual Meeting of the 
Association of Computational Linguistics Proceedings.~---
 Ann Arbor, MI, USA: Association of 
Computational Linguistics, 2005. P.~65--72.
\bibitem{10-nur}
\Au{Koehn Ph.} Neural machine translation.~--- New York, NY, USA: Cambridge University 
Press, 2020. 394~p.
\bibitem{11-nur}
\Au{Popovi$\acute{\mbox{c}}$~M.} chrF: Character $n$-gram F-score for automatic MT evaluation~//  
10th Workshop on Statistical Machine Translation Proceedings.~--- Lisboa, Portugal: Association for 
Computational Linguistics, 2015. P.~392--395.
\bibitem{12-nur}
\Au{Popovi$\acute{\mbox{c}}$~M.} chrF deconstructed: $\beta$ parameters and \mbox{$n$-gram} weights~// 
1st Conference on Machine Translation Proceedings.~--- Berlin, Germany: Association for Computational 
Linguistics, 2016. Vol.~2. P.~499--504.
\bibitem{13-nur}
\Au{Chi-kiu Lo.} MEANT 2.0: Accurate semantic MT evaluation for any output language // 
Conference on Machine Translation Proceedings.~---
 Copenhagen, Denmark: Association for Computational Linguistics, 2017. Vol.~2. P.~589--597.
\bibitem{14-nur}
\Au{Stanojevi$\acute{\mbox{c}}$ M., Sima'an~K.} BEER: BEtter evaluation as ranking~//  
9th Workshop on Statistical Machine Translation Proceedings.~--- Baltimore, MD, USA: Association for 
Computational Linguistics, 2014. P.~414--419.
\bibitem{15-nur}
\Au{Stanojevi$\acute{\mbox{c}}$ M., Sima'an~K.} Evaluating MT systems with BEER~// Prague Bulletin  
Mathematical Linguistics, 2015. No.\,104. P.~17--26.
\bibitem{16-nur}
\Au{Sellam T., Das~D., Parikh~A.\,P.} BLEURT: Learning robust metrics for text generation~// 
arXiv.org, 9 Apr 2020. arXiv:2004.04696 [cs.CL].
\bibitem{17-nur}
\Au{Инькова О.\,Ю.} Надкорпусная база данных как инструмент изучения формальной 
вариативности коннекторов~// Компьютерная лингвистика и~интеллектуальные 
технологии: По мат-лам ежегодной \mbox{Международ.} конф. <<Диалог>>.~---
 М.: РГГУ, 2018. Вып.~17(24). С.~240--253.
\bibitem{18-nur}
\Au{Castilho Sh., Doherty~S, Gaspari~F., Moorkens~J.} Approaches to human and machine 
translation quality assessment~// Translation quality assessment: From principles to practice~/ 
Eds. J.~Moorkens, Sh.~Castilho, F.~Gaspari, S.~Doherty.~--- Cham, Switzerland: Springer, 2018.  P.~9--38.
\bibitem{19-nur}
\Au{Likert R.} A~technique for the measurement of attitudes~// Arch. Psychol., 1932. 
Vol.~140. P.~1--55
\bibitem{20-nur}
\Au{Fleiss J.\,L.} Measuring nominal scale agreement among many raters~// Psychol. Bull., 1971. 
Vol.~76. No.\,5. P.~378--382.
\bibitem{21-nur}
\Au{Shterionov D., Superbo~R., Nagle~P., \textit{et al.}} Human versus automatic quality evaluation of 
NMT and PBSMT~// Machine Translation, 2018. Vol.~32. P.~217--235.
\bibitem{22-nur}
\Au{Castilho S., Moorkens~J., Gaspari~F., \textit{et al.}} Evaluating MT for massive open online 
courses. A~multifaceted comparison between PBSMT and NMT systems~// Machine Translation, 
2018. Vol.~32. P.~255--278.
\bibitem{23-nur}
\Au{Popovic M.} Error classification and analysis for machine translation quality assessment~// 
Translation quality assessment: From principles to practice~/ Eds. J.~Moorkens, Sh.~Castilho, 
F.~Gaspari, S.~Doherty.~--- Cham, Switzerland: Springer, 2018. P.~129--158.
\bibitem{24-nur}
\Au{Lommel A.} Metrics for translation quality assessment: A~case for standardizing error 
typologies~// Translation quality assessment: From principles to practice~/ Eds. J.~Moorkens, 
Sh.~Castilho, F.~Gaspari, S.~Doherty.~--- Cham, Switzerland: Springer, 2018. P.~109--127.
\bibitem{25-nur}
\Au{\mbox{Klubi\!{\!\ptb{\v{c}}}ka} F., Toral~A., S$\acute{\mbox{a}}$nchez-Cartagena~V.\,M.}
 Quantitative fine-grained human 
evaluation of machine translation systems: A~case study on English to Croatian~// Machine 
Translation, 2018. Vol.~32. P.~195--215.
\bibitem{26-nur}
\Au{Haque R., Hasanuzzaman~M., Way~A.} Analysing terminology translation errors in 
statistical and neural machine translation~// Machine Translation, 2020. Vol.~34. P.~149--195.
\bibitem{27-nur}
\Au{Vilar D., Xu~J., D'Haro~L., Ney~H.} Error analysis of statistical machine translation output~// 
5th Conference (International) on Language Resources and Evaluation Proceedings.~--- Genoa, 
Italy: European Language Resources Association, 2006. P.~697--702.
%{\sf https://www.researchgate.net/publication/307174612\_Error\_Analysis\_of\_Machine\_Translation\_Output}.
\bibitem{28-nur}
\Au{Гончаров А.\,А., Бунтман~Н.\,В., Нуриев~В.\,А.} Ошибки в~машинном переводе: 
проблемы классификации~// Системы и~средства информатики, 2019. Т.~29. №\,3. С.~92--103.
\bibitem{29-nur}
\Au{Calixto I., Liu~Q.} An error analysis for image-based multi-modal neural machine 
translation~// Machine Translation, 2019. Vol.~33. P.~155--177.
  \end{thebibliography}

}
}

\end{multicols}

\vspace*{-7pt}

\hfill{\small\textit{Поступила в~редакцию 14.04.2021}}

%\vspace*{8pt}

%\pagebreak

\newpage

\vspace*{-28pt}

%\hrule

%\vspace*{2pt}

%\hrule

%\vspace*{-2pt}

\def\tit{METHODS OF QUALITY ESTIMATION FOR~MACHINE TRANSLATION: STATE-OF-THE-ART}


\def\titkol{Methods of quality estimation for~machine translation: State-of-the-art}

\def\aut{V.\,A.~Nuriev and~A.\,Yu.~Egorova}

\def\autkol{V.\,A.~Nuriev and~A.\,Yu.~Egorova}


\titel{\tit}{\aut}{\autkol}{\titkol}

\vspace*{-11pt}




\noindent
Institute of Informatics Problems, Federal Research Center ``Computer Science and Control''
 of the Russian Academy of Sciences, 44-2~Vavilov Str., Moscow 119333, Russian Federation

 
\def\leftfootline{\small{\textbf{\thepage}
\hfill INFORMATIKA I EE PRIMENENIYA~--- INFORMATICS AND
APPLICATIONS\ \ \ 2021\ \ \ volume~15\ \ \ issue\ 2}
}%
\def\rightfootline{\small{INFORMATIKA I EE PRIMENENIYA~---
INFORMATICS AND APPLICATIONS\ \ \ 2021\ \ \ volume~15\ \ \ issue\ 2
\hfill \textbf{\thepage}}}

\vspace*{3pt}  
   


\Abste{The paper reviews the state-of-the-art methods of quality estimation for machine translation. 
These methods are grounded in two general approaches: automatic and manual. The automatic 
assessment builds on the data from comparison of the machine translation system output against the 
human-generated reference translation. The manual (human) evaluation primarily takes into account 
pragmatic and functional aspects: the translation quality is assessed bearing in mind how well the system 
output is suited to fulfill the translation tasks. The first part presents some automatic metrics for 
evaluation of machine translation quality. Also, it speaks about both  shortcomings of such metrics and 
new trends in their development. The other part of the paper is focused on human evaluation of machine 
translation. It describes: ($i$)~evaluation of adequacy and fluency; 
($ii$)~ranking of translations; ($iii$)~direct 
assessment; ($i\nu$)~computation of the human translation edit rate, and 
($\nu$)~translation annotation involving an 
error typology.}

\KWE{machine translation; translation quality; evaluation of machine translation quality; automatic 
metrics; direct assessment; typology of machine translation errors}

\DOI{10.14357/19922264210215}

%\vspace*{-15pt}

% \Ack
%\noindent
%%The work was performed at the Institute of Informatics Problems, Federal Research Center ``Computer Science and Control''
% of the Russian Academy of Sciences.


%\vspace*{12pt}

  \begin{multicols}{2}

\renewcommand{\bibname}{\protect\rmfamily References}
%\renewcommand{\bibname}{\large\protect\rm References}

{\small\frenchspacing
 {%\baselineskip=10.8pt
 \addcontentsline{toc}{section}{References}
 \begin{thebibliography}{99}
\bibitem{1-nur-1}
\Aue{Larsonneur, C.} 2021. Neural machine translation: From commodity to commons? \textit{When 
translation goes digital: Case studies and critical reflections.} 
Eds. R.~Desjardins, C.~Larsonneur, and P.~Lacour. Cham: Palgrave Macmillan. 
257--280.
\bibitem{2-nur-1}
\Aue{Davenport, C.} 2018.
Google Translate processes 143~billion words every day. \textit{Android Police}. Available at: 
{\sf https://www.androidpolice.com/2018/10/09/google-translate-processes-143-billion-words-every-day/} 
(accessed May~5, 2021).
\bibitem{3-nur-1}
Moorkens, J., S.~Castilho, F.~Gaspari, and S.~Doherty, eds. 2018. \textit{Translation quality assessment:
From principles to practice}. 
Machine translation: Technologies and applications ser. Cham: Springer International Publishing. Vol.~1. 
299~p.
\bibitem{4-nur-1}
\Aue{Specia, L., C.~Scarton, and G.\,H.~Paetzold.} 2018. \textit{Quality estimation for machine translation}. 
Synthesis lectures on
human language technologies ser.
London: Morgan \& Claypool Publs. 162~p.
\bibitem{5-nur-1}
\Aue{Bittner, H.} 2020. \textit{Evaluating the evaluator: A~novel perspective on translation quality assessment}. 
New York, NY: Routledge. 282~p.
\bibitem{6-nur-1}
\Aue{Papineni, K., S.~Roukos, T.~Ward, and W.\,J.~Zhu.} 2002. BLEU: A~method for automatic 
evaluation of machine translation. \textit{40th Annual Meeting on Association for Computational Linguistics 
Proceedings}. Philadelphia, PA: Association for Computational Linguistics. 311--318.
\bibitem{7-nur-1}
\Aue{Rychikhin, A.\,K.} 2019. O~metodakh otsenki kachestva mashinnogo perevoda [On methods of 
machine translation quality assessment]. \textit{Sistemy i~Sredstva Informatiki~--- Systems and Means of 
Informatics} 29(4):106--118.
\bibitem{8-nur-1}
\Aue{Kozina, A.\,V., E.\,A.~Cherepkov, and Yu.\,S.~Belov.} 2019. Avtomaticheskie metriki otsenki 
kachestva mashinnogo perevoda [Automatic metrics for machine translation evaluation]. \textit{Sistemnyy 
administrator} [System Administrator] 11:84--87.
\bibitem{9-nur-1}
\Aue{Banerjee, S., and A.~Lavie.} 2005. METEOR: An automatic metric for MT evaluation with 
improved correlation with human judgments. \textit{Workshop on Intrinsic and Extrinsic 
Evaluation Measures for MT and/or Summarization at the 
43rd Annual Meeting of the Association of Computational 
Linguistics Proceedings}. Ann Arbor, MI: Association of Computational Linguistics. 65--72.
\bibitem{10-nur-1}
\Aue{Koehn, Ph.} 2020. \textit{Neural machine translation}. New York, NY: Cambridge University 
Press. 394~p.
\bibitem{11-nur-1}
\Aue{Popovi$\acute{\mbox{c}}$, M.} 2015. chrF: Character $n$-gram F-score for automatic MT evaluation. 
\textit{10th Workshop 
on Statistical Machine Translation Proceedings}. Lisboa, Portugal: Association for Computational 
Linguistics. 392--395.
\bibitem{12-nur-1}
\Aue{Popovi$\acute{\mbox{c}}$, M.} 2016. chrF deconstructed: $\beta$ parameters and $n$-gram weights. 
\textit{1st Conference on 
Machine Translation Proceedings}. Berlin, Germany: Association for Computational Linguistics. 2:499--504.
\bibitem{13-nur-1}
\Aue{Chi-kiu, Lo.} 2017. MEANT~2.0: Accurate semantic MT evaluation for any output language. 
\textit{Conference on Machine Translation Proceedings}. Copenhagen, Denmark: Association for Computational 
Linguistics. 2:589--597.
\bibitem{14-nur-1}
\Aue{Stanojevi$\acute{\mbox{c}}$, M., and K.~Sima'an.} 2014. BEER: BEtter evaluation as ranking. 
\textit{9th Workshop on 
Statistical Machine Translation Proceedings}. Baltimore, MD: Association for Computational 
Linguistics. 414--419.
\bibitem{15-nur-1}
\Aue{Stanojevi$\acute{\mbox{c}}$, M., and K.~Sima'an.} 2015. Evaluating MT systems with BEER. 
\textit{Prague Bulletin Mathematical Linguistics} 104:17--26.
\bibitem{16-nur-1}
\Aue{Sellam, T., D.~Das, and A.\,P.~Parikh.} 2020. BLEURT: Learning robust metrics for text 
generation. Available at: {\sf https://arxiv.org/pdf/2004.04696.pdf} (accessed May~5, 2021).
\bibitem{17-nur-1}
\Aue{Inkova, O.\,Yu.} 2018. Nadkorpusnaya baza dannykh kak instrument formal'noy variativnosti 
konnektorov [Supracorpora database as an instrument of the study of the formal variability of 
connectives]. \textit{Komp'yuternaya ling\-vi\-sti\-ka i~intellektual'nye tekhnologii: po mat-lam ezhegodnoy 
Mezhdunar. konf. ``Dialog''} [Computer Linguistic and Intellectual Technologies: Conference 
(International) ``Dialog'' Proceedings]. Moscow. 17(24):240--253.
\bibitem{18-nur-1}
\Aue{Castilho, Sh., S.~Doherty, F.~Gaspari, and J.~Moorkens}.
 2018. Approaches to human and machine 
translation quality assessment. \textit{Translation quality assessment: From principles to practice}. Eds. 
J.~Moorkens, Sh.~Castilho, F.~Gaspari, and S.~Doherty. Cham: Springer. 9--38.
\bibitem{19-nur-1}
\Aue{Likert, R.} 1932. A~technique for the measurement of attitudes. \textit{Arch. Psychol.} 140:1--55.
\bibitem{20-nur-1}
\Aue{Fleiss, J.\,L.} 1971. Measuring nominal scale agreement among many raters. \textit{Psychol. Bull.} 
76(5):378--382.
\bibitem{21-nur-1}
\Aue{Shterionov, D., R.~Superbo, P.~Nagle, \textit{et al.}}
 2018. Human versus automatic quality evaluation of 
NMT and PBSMT. \textit{Machine Translation} 32:217--235.
\bibitem{22-nur-1}
\Aue{Castilho, S., J.~Moorkens, F.~Gaspari, \textit{et al.}} 2018. Evaluating MT for massive open online 
courses. A~multifaceted comparison between PBSMT and NMT systems. \textit{Machine Translation} 32:255--278.
\bibitem{23-nur-1}
\Aue{Popovic, M.} 2018. Error classification and analysis for machine translation quality assessment. 
\textit{Translation quality assessment: From principles to practice}. 
Eds. J.~Moorkens, Sh.~Castilho, F.~Gaspari, 
and S.~Doherty. Cham: Springer. 129--158.
\bibitem{24-nur-1}
\Aue{Lommel,~A.} 2018. Metrics for translation quality assessment: A~case for standardising error 
typologies. \textit{Translation quality assessment: From principles to practice}. Eds. J.~Moorkens, 
Sh.~Castilho, 
F.~Gaspari, and S.~Doherty. Cham: Springer. 109--127.
\bibitem{25-nur-1}
\Aue{\mbox{Klubi\!{\!\ptb{\v{c}}}ka}, F., A.~Toral, and V.\,M.~S$\acute{\mbox{a}}$nchez-Cartagena.}
 2018. Quantitative fine-grained human 
evaluation of machine translation systems: A~case study on English to Croatian. 
\textit{Machine Translation} 32:195--215.
\bibitem{26-nur-1}
\Aue{Haque, R., M.~Hasanuzzaman, and A.~Way.} 2020. Analysing terminology translation errors in 
statistical and neural machine translation. \textit{Machine Translation} 34:149--195.
\bibitem{27-nur-1}
\Aue{Vilar, D., J.~Xu, L.~D'Haro, and H.~Ney.} 2006. Error analysis of statistical machine translation 
output. \textit{5th Conference (International) on Language Resources and Evaluation Proceedings}. 697--702. 
\bibitem{28-nur-1}
\Aue{Goncharov, A.\,A., N.\,V.~Buntman, and V.\,A.~Nuriev.} 2019. Oshibki v~mashinnom perevode: 
problemy klassifikatsii [Machine translation errors: Problems of classification]. 
\textit{Sistemy i~Sredstva  Informatiki~--- Systems and Means of Informatics} 29(3):92--103.
\bibitem{29-nur-1}
\Aue{Calixto, I., and Q.~Liu.} 2019. An error analysis for image-based multi-modal neural machine 
translation. \textit{Machine Translation} 33:155--177.
\end{thebibliography}

 }
 }

\end{multicols}

\vspace*{-3pt}

  \hfill{\small\textit{Received April~14, 2021}}


%\pagebreak

%\vspace*{-8pt}  

\Contr

\noindent
\textbf{Nuriev Vitaly A.} (b.\ 1980)~--- Candidate of Science (PhD) in philology, leading scientist, 
Institute of Informatics Problems, Federal Research Center ``Computer Science and Control'' of the 
Russian Academy of Sciences, 44-2~Vavilov Str., Moscow 119333, Russian Federation; 
\mbox{nurieff.v@gmail.com}

\vspace*{3pt}

\noindent
\textbf{Egorova Anna Yu.} (b.\ 1991)~--- junior scientist, Institute of Informatics Problems, Federal 
Research Center ``Computer Science and Control'' of the Russian Academy of Sciences, 44-2~Vavilov 
Str., Moscow 119333, Russian Federation; \mbox{ann.shurova@gmail.com}
   
\label{end\stat}

\renewcommand{\bibname}{\protect\rm Литература}  %15     
\def\stat{andrianova}

\def\tit{КОНТЕКСТНЫЙ ПОИСК НА ФОТОНАХ С~ИСПОЛЬЗОВАНИЕМ~ТЕСТОВ~БЕЛЛА$^*$}

\def\titkol{Контекстный поиск на фотонах с~использованием тестов Белла}

\def\aut{\fbox{С.\,Н.~Андрианов}$^1$, Н.\,С.~Андрианова$^2$, Ф.\,М.~Аблаев$^3$, 
Ю.\,Ю.~Кочнева$^4$}

\def\autkol{С.\,Н.~Андрианов, Н.\,С.~Андрианова, Ф.\,М.~Аблаев, 
Ю.\,Ю.~Кочнева}

\titel{\tit}{\aut}{\autkol}{\titkol}

\index{Андрианов С.\,Н.}
\index{Андрианова Н.\,С.}
\index{Аблаев Ф.\,М.} 
\index{Кочнева Ю.\,Ю.}
\index{Andrianov S.\,N.}
\index{Andrianova N.\,S.}
\index{Ablaev F.\,M.}
\index{Kochneva Yu.\,Yu.}


{\renewcommand{\thefootnote}{\fnsymbol{footnote}} \footnotetext[1]
{Работа Ф.\,М.~Аблаева выполнена за счет средств субсидии, выделенной Казанскому 
(Приволжскому) федеральному университету для выполнения государственного задания 
в~сфере научной деятельности, проект №\,0671-2020-0065.}}


\renewcommand{\thefootnote}{\arabic{footnote}}
\footnotetext[1]{Институт прикладных исследований, Академия наук Республики Татарстан}
\footnotetext[2]{Казанский (Приволжский) федеральный университет, \mbox{natalia\_an83@mail.ru}}
\footnotetext[3]{Казанский (Приволжский) федеральный университет, \mbox{fablayev@gmail.com}}
\footnotetext[4]{Институт прикладных исследований, Академия наук Республики Татарстан,
\mbox{instpianrt@gmail.com}}

\vspace*{-6pt}




     
     \Abst{Рассмотрена возможность конкретной физической реализации контекстного 
поиска на квантовых состояниях с использованием тестов Белла, который рассматривался 
ранее лишь как абстрактная математическая процедура. Для этого предложено использовать 
контекстную кодировку слов в документах на поляризационных фотонных кубитах. 
Получены конкретизированные аналитические выражения для определения на основе тестов 
Белла параметра контекстного поиска по паре слов, которые могут быть связанными или нет 
в зависимости от значения этого параметра. Наибольшей связанности отвечает состояние 
квантовой перепутанности волновых функций документов по паре выбранных слов, 
которому соответствует определенное значение параметра контекстного поиска. 
Предложенные способы реализации семантического контекстного поиска необходимы для 
определения нелокальной контекстности, которая часто требуется при автоматизированном 
поиске и машинном переводе. При этом второе слово в паре поисковых слов поясняет смысл 
первого через их семантическую связь.}
     
     \KW{гипоним; гипероним; изотопия; родовидовая связь; фотонный кубит; 
перепутанные состояния; белловский тест; голографический процессор}

\DOI{10.14357/19922264220103}
  
\vspace*{-4pt}


\vskip 10pt plus 9pt minus 6pt

\thispagestyle{headings}

\begin{multicols}{2}

\label{st\stat}
     
    В~рамках познавательной дея\-тель\-ности человека информация 
структурируется с точ\-ки зрения ее значения и организации, постепенно 
пре\-вра\-ща\-ясь в знание. Фор\-ма\-ли\-зу\-емые на естественном языке знания 
систематизируются в рамках лексической сис\-те\-мы языка в толковых, 
семантических, идеографических словарях, а~так\-же в~циф\-ро\-вых сетях, 
вклю\-чая квантовые.
    
    Формализация знаний с по\-мощью концептуальной схемы может 
осуществляться в~виде лексических онтологий, например WordNet. Этот 
ресурс является одним из способов пред\-став\-ле\-ния знаний на основе 
установления семантических отношений меж\-ду понятиями с по\-мощью 
составления синонимических рядов (синсетов) со\-от\-вет\-ст\-ву\-ющей час\-ти речи, 
которые, в свою очередь, так\-же связаны меж\-ду собой разнообразными 
семантическими отношениями (гиперонимия, меронимия, антонимия 
и~т.\,д.)~\cite{4-an}.
    
    От логической модели построения баз данных отличается подход, 
представленный методом HAL (Hyperspace Analogue to Language). При 
построении этого языка считается, что любой языковой знак находится 
в~контексте, а также учитывается линейный характер текста и бли\-зость слов 
друг другу в его линейной развертке. Речь идет о построении базы данных 
связей слов с учетом их непосредственного словесного окружения~\cite{5-an}. 
Авторы этой работы вводят так называемое семантическое пространство, 
в~котором смысл словосочетаний отображается при помощи специального 
языка HAL.
    
    В языке HAL слова распределяются не так, как в~словаре обычного языка--- 
последовательно, например просто в алфавитном порядке, а~в~виде\linebreak  
мат\-ри\-цы (таблицы), где слова распределены по вертикали в~самом левом 
столб\-це по алфавиту, как и~в~обычном словаре. Обычный словарь в~этом 
смыс\-ле является аналогом векторного пространства. В~языке HAL в~строках 
каждое слово размещается в пар\-ных сочетаниях с~другими словами, что должно 
отражать ка\-кую-то связь этого отдельного слова с~остальными словами. Будучи 
расположенными в мат\-ри\-це, слова в~языке HAL имеют как бы тензорный 
характер, т.\,е.\ относятся к некоему гиперпространству. При этом слова 
в~мат\-ри\-це располагаются тем ближе, чем они дальше друг от друга  
в~рас\-смат\-ри\-ва\-емом текс\-те. Аналогия с языком в методе HAL заключается 
в~том, что каж\-дой точке гиперпространства можно сопоставить то или иное 
словосочетание языка рас\-смат\-ри\-ва\-емо\-го текста.
    
    Таким путем эмпирический подход языка HAL позволяет выявить 
взаимосвязь слов. Но эта связь может быть и чисто формальной, следствием 
случайных совпадений. Если такая связь является изотопической, то она имеет 
смысловой характер. 
    
    Авторы работы~\cite{6-an} такую связь между парой слов в том или ином 
текс\-те предложили искать целенаправленно, используя квантовые алгоритмы 
при записи слов в текс\-те при помощи языка HAL. При этом квантовые 
алгоритмы реализовывались чис\-то математически с использованием известных 
в~квантовой механике абстрактных формул. В~интерпретации языка HAL 
в~работе~\cite{6-an} квантовое векторное со\-сто\-яние документа определяется 
как $\vert \Psi\rangle \hm= \sum\nolimits_i^N \vert w_{n_i}\rangle$, где $\vert 
w_{ni}\rangle$~--- $i$-е собственное со\-сто\-яние оператора некоторой 
величины~$n$, со\-от\-вет\-ст\-ву\-ющее слову~$i$, т.\,е.\ это вектор, яв\-ля\-ющий\-ся 
суммой векторов отдельных слов.
    
    Рассмотрим возможность физической реализации такого подхода 
с~использованием конкретных фотонных квантовых со\-сто\-яний $\vert 
w_{v_i}\rangle\hm=\sum\nolimits_j^M \vert w_{k_{ij}}\rangle$ как квантовых 
со\-сто\-яний слова~$i$, которые характеризуются час\-то\-той фотона~$v_i$ и его 
вол\-но\-вым вектором~$k_{ij}$. Можно представить эти со\-сто\-яния как сумму 
проекций нормированного со\-сто\-яния фотона $\vert u_{v_{ij}}\rangle \hm= 
a^{(i)}_{\vec{k}_j} \vert u_{\vec{k}_j}\rangle$  на базовые со\-сто\-яния 
на\-прав\-ле\-ний его волнового век\-то\-ра, лежащих в той или иной плос\-кости: 
$$
\vert  u_{v_i}\rangle = \sum\limits_j^M a^{(i)}_{\vec{k}_j} \vert 
u_{\vec{k}_j}\rangle.
$$
    
    Двум словам~$A$ и~$B$ из текста можно сопоставить векторные 
состояния $\vert u_{v_A}\rangle$ и~$\vert u_{v_B}\rangle$  и~об\-щую плос\-кость, 
проходящую через эти векторы. Тогда векторные со\-сто\-яния документа в базисе 
этих слов мож\-но определить как векторную сумму проекций состояния 
документа на со\-сто\-яния этих слов в данной плос\-кости с последующим 
поворотом на~90$^\circ$ относительно ортогональных к векторам слов осей 
и~проекции на оси в плос\-кости, ортогональной векторам слов:
    \begin{equation}
    \left.
    \begin{array}{rl}
    \vert \Psi_A\rangle &= \fr{1}{\sqrt{2}}\left( \alpha_{\sigma_+}\vert 
u_{v_A,\sigma_+} \rangle +\alpha_{\sigma_-}\vert u_{v_A,\sigma_-}\rangle\right);\\[6pt]
%    \label{e1-an}
    \vert \Psi_B\rangle &= \fr{1}{\sqrt{2}}\left( \beta_{\sigma_+}\vert 
u_{v_B,\sigma_+} \rangle +\beta_{\sigma_-}\vert u_{v_B,\sigma_-}\rangle\right).
\end{array}
\right\}
    \label{e2-an}
    \end{equation}
    
    Векторы~(\ref{e2-an})  являются по своей фор\-ме 
поляризационными фотонными кубитами.
    
    Коэффициенты в выражениях~(\ref{e2-an}) мож\-но записать 
как
    \begin{equation}
    \left.
    \begin{array}{rl}
    \alpha_{\sigma_+} &= \fr{\langle u_{v_{A,\sigma_+}}\vert \Psi\rangle} 
{\sqrt{\langle u_{v_{A,\sigma_+}}\vert \Psi\rangle^2}+\langle u_{v_{A,\sigma_-
}}\vert \Psi\rangle^2}\,; %\label{e3-an}
\\[6pt]
    \alpha_{\sigma_-} &= \fr{\langle u_{v_{A,\sigma_-}}\vert \Psi\rangle} 
{\sqrt{\langle u_{v_{A,\sigma_+}}\vert \Psi\rangle^2}+\langle u_{v_{A,\sigma_-
}}\vert \Psi\rangle^2}\,;
\end{array}
\right\}
\label{e4-an}
\end{equation}
\begin{equation}
\left.
\begin{array}{rl}
    \beta_{\sigma_+} &= \fr{\langle u_{v_B,\sigma_+}\vert \Psi\rangle} 
{\sqrt{\langle u_{v_B,\sigma_+}\vert \Psi\rangle^2}+\langle u_{v_B,\sigma_-}\vert 
\Psi\rangle^2}\,;\\[6pt]
%    \label{e5-an}\\
    \beta_{\sigma_-} &= \fr{\langle u_{v_B,\sigma_-}\vert \Psi\rangle} 
{\sqrt{\langle u_{v_B,\sigma_+}\vert \Psi\rangle^2}+\langle u_{v_B,\sigma_-}\vert 
\Psi\rangle^2}\,.
\end{array}
\right\}
    \label{e6-an}
    \end{equation}
    
    Состояния вида~(\ref{e2-an}) позволяют связать каж\-дое 
слово с тем или иным квантовым битом (кубитом) информации. Определим 
теперь операторы запроса контекстного поиска на фотонах с учетом того, что 
поляризации фотонов можно ассоциировать с их спиновыми состояниями. 
Оператор прямого значения слова соответствует оператору $z$-про\-ек\-ции 
спина: 
    \begin{align*}
    \hat{A}\vert \Psi_A\rangle &= \hat{S}_{Az} \vert \Psi_A\rangle 
={}\notag\\
&{}=\fr{1}{\sqrt{2}}\left( \alpha_{\sigma_+} \vert u_{v_A,\sigma_+}\rangle -
\alpha_{\sigma_-} \vert u_{v_A,\sigma_-}\rangle\right);
   % \label{e7-an}
   \\
    \hat{B}\vert \Psi_B\rangle &= \hat{S}_{Bz} \vert \Psi_B\rangle 
={}\notag\\
&{}=\fr{1}{\sqrt{2}}\left( \beta_{\sigma_+} \vert u_{v_B,\sigma_+}\rangle -
\beta_{\sigma_-} \vert u_{v_B,\sigma_-}\rangle\right).
  %  \label{e8-an}
    \end{align*}
    
    Оператор запроса противоположного значения можно определить как
    \begin{multline*}
    \hat{A}_x\vert \Psi_A\rangle = \hat{S}_{Ax} \vert \Psi_A\rangle 
={}\\
{}= \fr{1}{\sqrt{2}}\left( \alpha_{\sigma_-}\vert u_{v_A,\sigma_+}\rangle 
+\alpha_{\sigma_+}\vert u_{v_B,\sigma_-}\rangle\right).
  %  \label{e9-an}
    \end{multline*}
    
    Белловский параметр поиска можно записать через матричные элементы 
операторов запроса по известной из работы~\cite{6-an} формуле:
    \begin{multline*}
    S_{\mathrm{query}}=\left\vert \left \langle \hat{A}\hat{B}_+\right\rangle_\Psi 
+\left\langle \hat{A}_x\hat{B}_+\right\rangle_\Psi\right\vert +{}\\
{}+
    \left\vert \left\langle \hat{A}\hat{B}_-\right\rangle_\Psi -\left\langle 
\hat{A}_x\hat{B}_-\right\rangle_\Psi\right\vert\,,
    %\label{e10-an}
    \end{multline*}
где $\hat{B}_+= -(\hat{B}+\hat{B}_x)$; $\hat{B}_-\hm= \hat{B}\hm- \hat{B}_x$.
    
    Простое вычисление дает
    \begin{multline}
    S_{\mathrm{query}}= \fr{1}{2} \left\{ \left\vert \left( \alpha^2_{\sigma_+} 
+2\alpha_{\sigma_+} \alpha_{\sigma_-}-\alpha^2_{\sigma_-}\right)\right\vert 
+{}\right.\\
    \left.{}+
    \left\vert \left( \alpha^2_{\sigma_+} -2\alpha_{\sigma_+}\alpha_{\sigma_-} - 
\alpha^2_{\sigma_-}\right)\right\vert \right\} 
    \left\vert \left(\beta^2_{\sigma_+} +{}\right.\right.\\
   \left.\left. {}+ 2\beta_{\sigma_+} \beta_{\sigma_-} -
\beta^2_{\sigma_-}\right)\right\vert\,.
    \label{e11-an}
    \end{multline}
                
    
    Вычислив коэффициенты $\alpha_{\sigma_+}$, $\alpha_{\sigma_-}$, 
$\beta_{\sigma_+}$ и~$\beta_{\sigma_-}$ по  
формулам~(\ref{e4-an}) и~(\ref{e6-an}), можно установить значения слов по 
матричным элементам этих операторов. Также можно вычислить белловский 
параметр~$S_{\mathrm{query}}$, величина которого определяет степень пе\-ре\-пу\-тан\-ности 
состояний документа по словам~$A$ и~$B$. Перепутанность состояний 
означает, что со\-сто\-яния связаны между собой путем взаимодействия через 
ка\-кие-ли\-бо другие со\-сто\-яния. Поэтому таким путем мож\-но установить наличие 
смысловой связи между выбранными словами в~этом до\-ку\-менте. 
    
    В работе~\cite{6-an} такие вычисления проведены на обычном 
компьютере. Но можно и по\-стро\-ить автономное вы\-чис\-ли\-тель\-ное устройство, 
работающее, например, на час\-ти\-цах света~--- фотонах. Такое устройство будет 
обладать повышенным быст\-ро\-дей\-ст\-ви\-ем как за счет предельно высокой 
ско\-рости и безынер\-ци\-он\-ности фотонов, так и за счет кван\-то\-вой па\-рал\-лель\-ности 
ис\-поль\-зу\-емых алгоритмов.
    
    В этом устройстве можно определить коэффициенты в состояниях 
документа, вычисляя скалярные произведения  
в~формулах~(\ref{e4-an}) и~(\ref{e6-an}) при помощи классического оптического 
процессора. Особенно удоб\-но использовать голографические 
процессоры~\cite{7-an, 8-an}. Эти процессоры позволяют записывать результат 
скалярного произведения векторов при помощи интерференции фотонов, 
а~затем получать результат с использованием счи\-ты\-ва\-юще\-го поля. После 
определения значения коэффициентов можно вы\-чис\-лить значение па\-ра\-мет\-ра~$S_{\mathrm{query}}$ 
по формуле~(\ref{e11-an}). Так\-же мож\-но методами квантовой 
информатики сгенерировать состояния двух фотонов, со\-от\-вет\-ст\-ву\-ющих 
словам~$A$ и~$B$ и провести измерение па\-ра\-мет\-ра $S_{\mathrm{query}}$ по 
стандартным схемам работ~\cite{9-an, 10-an, 11-an}. 
    
    Итак, в данной работе путем использования фотонных со\-сто\-яний найден 
конкретный способ реализации квантового алгоритма контекстного поиска, 
позволяющий искать изотопию, т.\,е.\ общий\linebreak семантический признак, 
свя\-зы\-ва\-ющий понятия. При поисковом запросе установление связи (выявление 
изотопии) меж\-ду понятиями может осуществляться в текстах различного 
характера (текс\-ты, относящиеся к одной терминологической об\-ласти; текс\-ты 
раз\-ных тематических областей). Подход работы~\cite{6-an} позволяет 
определить, относится ли текст к определенному вопросу, путем введения при 
поиске пары слов. Так, если искать информацию о~политическом скандале  
<<Иран--конт\-рас>>, то, понимая, что в это время Рейган был президентом 
Соединенных Штатов, причастных к скандалу, можно ввести при запросе пару 
слов Рей\-ган--Иран. Если параметр поиска покажет перепутанность со\-сто\-яний, 
соответствующих этим словам, то это будет означать, что рас\-смат\-ри\-ва\-емый 
текст соответствует теме запроса, т.\,е.\ задача найти нуж\-ный текст решена.
    
    Характер текстов, а также характер запроса пользователя (например, 
определение значения тер\-ми\-на-нео\-ло\-гиз\-ма посредством сравнения его 
с~термином, относящимся к той же терминологии, или поиск двух явно не 
связанных друг с~другом понятий) влияет и~на определение изотопии 
(семантической связи) между этими понятиями: гипонимия, гиперонимия 
(родовидовые отношения), метафора, антонимия. Таким образом, можно 
установить семантический признак, свя\-зы\-ва\-ющий понятия. Полученные 
данные могут быть использованы для создания как толковых словарей, так 
и~специализированных словарей (тезаурусов) в~той или иной об\-ласти 
в~зависимости от характера использованных текс\-тов. Они могут применяться 
в~сис\-те\-мах поиска~\cite{12-an, 13-an} и~сис\-те\-мах автоматизированного 
перевода~\cite{14-an, 15-an}.
    
{\small\frenchspacing
 {%\baselineskip=10.8pt
 %\addcontentsline{toc}{section}{References}
 \begin{thebibliography}{99}
%\bibitem{1-an}
%\Au{Greimas A.\,J.} S$\acute{\mbox{e}}$mantique structurale. Recherche de 
%m$\acute{\mbox{e}}$thode.~--- Paris: Larousse, 1966. 262~p.
%\bibitem{2-an}
%\Au{Rastier F.} Le d$\acute{\mbox{e}}$veloppement du concept d'isotopie~// Actes 
%S$\acute{\mbox{e}}$miotiques Documents, 1981. Vol.~3. No.\,29. 48~p.
%\bibitem{3-an}
%\Au{Величковский Б.\,М.} Когнитивная наука: Основы психологии познания: в 2~т.~--- М.: 
%Смысл; Академия, 2006. Т.~2. 432~с.
\bibitem{4-an}
\Au{Усталов Д.} Семантические сети и обработка естественного языка~// Открытые системы. 
СУБД, 2017. №\,2. С.~46--47. {\sf https://www.osp.ru/os/2017/ 02/13052229}.
\bibitem{5-an}
\Au{Lund K., Burgess~C.} Producing high-dimensional semantic spaces from lexical  
co-occurrence~// Behav. Res. Meth. Ins.~C., 1996. Vol.~28.  
P.~203--208.
\bibitem{6-an}
\Au{Barros J., Toffano~Z., Meguebli~Y., Doan B.-L.} Contextual query using bell tests~//  
Quantum interaction~~/
Eds. H.~Atmanspacher, E.~Haven, K.~Kitto, D.~Raine.~--- Lecture notes in computer 
science ser.~--- Springer, 2013. Vol.~8369. P.~110--121.
\bibitem{7-an} %4
\Au{Yariv A.} Phase conjugate optics and real-time holography~// IEEE J.~Quantum Elect., 
1978. Vol.~QE-14. No.\,9. P.~650--660.
\bibitem{8-an}
\Au{Dolev S., Fandina~N., Rosen~J.} Holographic parallel processor for calculating Kronecker 
product~// Nat. Comput., 2015. Vol.~14. P.~433--436.
\bibitem{9-an}
\Au{Clauser J.\,F., Horne~M.\,A., Shimony~A., Holt~R.\,A.} Proposed experiment to test local 
hidden-variable theories~// Phys. Rev. Lett., 1969. Vol.~23. No.\,15. P.~880--884.
\bibitem{10-an}
\Au{Freedman S.\,J., Clauser~J.\,F.} Experimental test of local hidden-variable theories~// Phys. 
Rev. Lett., 1972. Vol.~28. No.\,14. P.~938--941.
\bibitem{11-an}
\Au{Tanji H., Simon~J., Ghosh~S., Vuletic~V.} Simplified measurement of the Bell parameter 
within quantum mechanics~// arXiv.org, 2008. arXiv:0801.4549 [quant-ph].
\bibitem{12-an}
\Au{Beltran L., Geriente~S.} Quantum entanglement in corpuses of documents~// Found. 
Sci., 2019. Vol.~24. P.~227--246.
\bibitem{13-an}
\Au{Бессмертный И.\,А., Васильев~А.\,В., Королева~Ю.\,А., Платонов~А.\,В., 
Полещук~Е.\,А.} Методы квантового формализма в информационном поиске и обработке 
текстов на естественных языках~// Изв. вузов. Приборостроение, 2019. Т.~62. №\,8.  
С.~702--709.
\bibitem{14-an}
\Au{Wang C., Seneff~S.} High-quality speech-to-speech translation for computer-aided language 
learning~// ACM Transactions Speech Language Processing, 2006. Vol.~3. No.\,2. P.~1--21.
\bibitem{15-an}
\Au{Jia Ye., Weiss R.\,J., Biadsy~F., Macherey~W., Johnson~M., Chen~Z., Wu~Y.}  
Direct speech-to-speech translation with a sequence-to-sequence model~// arXiv.org, 2019. 
arXiv:\linebreak 1904.06037v2 [cs.CL].
\end{thebibliography}

 }
 }

\end{multicols}

\vspace*{-9pt}

\hfill{\small\textit{Поступила в~редакцию 12.03.20}}

\vspace*{6pt}

%\pagebreak

%\newpage

%\vspace*{-28pt}

\hrule

\vspace*{2pt}

\hrule

\vspace*{-2pt}

\def\tit{CONTEXT QUERY ON~PHOTONS WITH THE~USE OF~BELL TESTS}


\def\titkol{Context query on~photons with~the~use of~Bell tests}


\def\aut{\fbox{S.\,N.~Andrianov}$^1$, N.\,S.~Andrianova$^2$, F.\,M.~Ablaev$^2$, and~Yu.\,Yu.~Kochneva$^1$}

\def\autkol{S.\,N.~Andrianov, N.\,S.~Andrianova, F.\,M.~Ablaev, and~Yu.\,Yu.~Kochneva}

\titel{\tit}{\aut}{\autkol}{\titkol}

\vspace*{-11pt}


\noindent
$^1$Institute of Applied Research, Tatarstan Academy of Sciences, 36~Levobulachnaya Str., Kazan 420011, 
Russian\linebreak
$\hphantom{^1}$Federation

\noindent
$^2$Kazan Federal University, 18~Kremlyovskaya Str., Kazan 420008, Russian Federation

\def\leftfootline{\small{\textbf{\thepage}
\hfill INFORMATIKA I EE PRIMENENIYA~--- INFORMATICS AND
APPLICATIONS\ \ \ 2022\ \ \ volume~16\ \ \ issue\ 1}
}%
 \def\rightfootline{\small{INFORMATIKA I EE PRIMENENIYA~---
INFORMATICS AND APPLICATIONS\ \ \ 2022\ \ \ volume~16\ \ \ issue\ 1
\hfill \textbf{\thepage}}}

\vspace*{1pt} 




\Abste{The possibilities for physical realization of contextual query on photons 
in an optical processor using Bell tests are considered. To solve this problem, 
context coding of words in documents on quantum states of single photons using the 
well-known method of hyperspace analog language is proposed. Analytical expressions 
for determination of parameters of contextual query by a~pair of words that can be 
bound or not bound depending on the value of this parameter were obtained. 
Most connected is quantum entangled state of document wave functions chosen by 
a~pair of words that corresponds to a~certain value of the contextual query parameter. 
The suggested methods of realization of semantic contextual query are necessary 
for determination of nonlocal context that is demanded for acquiring better 
understanding during automated search and machine translation. 
The second word in the pair of query words clarifies the meaning of the 
first word through their semantic connection.}

\KWE{hyponym; hyperonym; isotopy; genus-species relations; 
photonic qubit; entangled state; Bell test; holographic processor}

\DOI{10.14357/19922264220103}

\vspace*{-24pt}

\Ack

\vspace*{-6pt}

\noindent
The research of F.\,M.~Ablaev was funded by the subsidy allocated to Kazan 
Federal University for the state assignment in the sphere of scientific activities, 
project No.\,0671-2020-0065.



%\vspace*{-6pt}

  \begin{multicols}{2}

\renewcommand{\bibname}{\protect\rmfamily References}
%\renewcommand{\bibname}{\large\protect\rm References}

{\small\frenchspacing
 {%\baselineskip=10.8pt
 \addcontentsline{toc}{section}{References}
 \begin{thebibliography}{99}
 
 \vspace*{-1pt}
 
%\bibitem{1-an-1}
%\Aue{Greimas, A.\,J.} 1966. \textit{S$\acute{\mbox{e}}$mantique structurale. Recherche de 
%m$\acute{\mbox{e}}$thode}. Paris: Larousse. 262~p.
%\bibitem{2-an-1}
%\Aue{Rastier, F.} 1981. Le d$\acute{\mbox{e}}$veloppement du concept d'isotopie. \textit{Actes 
%S$\acute{\mbox{e}}$miotiques Documents} 3(29):1--48.
%\bibitem{3-an-1}
%\Aue{Velichkovskii, B.\,М.} 2006. \textit{Kognitivnaya nauka: Osnovy psikhologii poznaniya} [Cognitive 
%science: Basics of knowledge psychology]. Moscow: Smysl; Akademiya. Vol.~2. 432~p.
\bibitem{4-an-1}
\Aue{Ustalov, D.} 2017. Semanticheskie seti i~obrabotka estestvennogo yazyka [Semantic nets and natural 
language processing].
\textit{Otkrytye sistemy. SUBD} [Open Systems. DBMS] 2:46--47.
 Available at: {\sf https://www.osp.ru/os/ 2017/02/13052229/} (accessed December~22, 
2021).
\bibitem{5-an-1}
\Aue{Lund, K., and C.~Burgess.} 1996. Producing high-dimensional semantic spaces from lexical  
co-occurrence. \textit{Behav. Res. Meth. Ins.~C.} 28:203--208.
\bibitem{6-an-1}
\Aue{Barros, J., Z.~Toffano, Y.~Meguebli, and B.-L.~Doan.} 2013. Contextual query using Bell tests. 
\textit{Quantum interaction}. Eds.\ H.~Atmanspacher, E.~Haven, K.~Kitto, and D.~Raine.
 Lecture notes in computer science ser. 
Springer. 8369:110--121.
\bibitem{7-an-1}
\Aue{Yariv, A.} 1978. Phase conjugate optics and real-time holography. \textit{IEEE J.~Quantum 
Elect.} QE-14(9):650--660.
\bibitem{8-an-1}
\Aue{Dolev, S., N.~Fandina, and J.~Rosen.} 2015. Holographic parallel processor for calculating Kronecker 
product. \textit{Nat. Comput.} 14:433--436.
\bibitem{9-an-1}
\Aue{Clauser, J.\,F., M.\,A.~Horne, A.~Shimony, and R.\,A.~Holt.} 1969. Proposed experiment to test local 
hidden-variable theories. \textit{Phys. Rev. Lett.} 23(15):880--884.
\bibitem{10-an-1}
\Aue{Freedman, S.\,J., and J.\,F.~Clauser.} 1972. Experimental test of local hidden-variable theories. 
\textit{Phys. Rev. Lett.} 28(14):938--941.
\bibitem{11-an-1}
\Aue{Tanji, H., J.~Simon, S.~Ghosh, and V.~Vuletic.} 2008. Simplified measurement of the Bell parameter 
within quantum mechanics. \textit{arXiv.org}. Available at: {\sf https://arxiv.org/abs/0801.4549} (accessed 
December~22, 2021).
\bibitem{12-an-1}
\Aue{Beltran, L., and S.~Geriente.} 2019. Quantum entanglement in corpuses of documents. 
\textit{Found. Sci.} 24:227--246.
\bibitem{13-an-1}
\Aue{Bessmertnyi, I.\,А., А.\,V.~Vasiliev, Yu.\,А.~Koroleva, А.\,V.~Platonov, and Е.\,А.~Poleschuk.} 
2019. Metody kvantovogo formalizma v~informatsionnom poiske i~obrabotke tekstov na estestvennykh 
yazykakh [Quantum formalism methods in information retrieval and processing of texts on natural languages]. 
\textit{Izvestiya vysshikh uchebnykh zavedeniy. Priborostroenie} [J.~Instrument Engineering]  
62(8):702--709.
\bibitem{14-an-1}
\Aue{Wang, C., and S.~Seneff.} 2006. High-quality speech-to-speech translation for computer-aided 
language learning. \textit{ACM Transactions Speech Language Processing} 3(2):1--21.
\bibitem{15-an-1}
\Aue{Jia, Ye., R.\,J.~Weiss, F.~Biadsy, W.~Macherey, M.~Johnson, Z.~Chen, and Y.~Wu.} 2019. Direct 
speech-to-speech translation with a sequence-to-sequence model. \textit{arXiv.org}. Available at: {\sf 
https://arxiv.org/abs/1904.06037} (accessed December~22, 2021).
\end{thebibliography}

 }
 }

\end{multicols}

\vspace*{-12pt}

\hfill{\small\textit{Received March 12, 2020}}

\pagebreak

%\vspace*{-18pt}

\Contr

\noindent
\textbf{Andrianov Sergey N.} (1959--2020)~--- Doctor of Science in physics and mathematics, principal 
scientist, Institute of Applied Research, Tatarstan Academy of Sciences, 36~Levobulachnaya Str., Kazan 
420011, Russian Federation

\vspace*{3pt}

\noindent
\textbf{Andrianova Nataliya S.} (b.\ 1983)~--- Candidate of Science (PhD) in philology, associate professor, 
Department of Theory and Practice of Teaching Foreign Languages, Institute of Philology and Intercultural 
Communication, Kazan Federal University, 18~Kremlyovskaya Str., Kazan 420008, Russian Federation; 
\mbox{natalia\_an83@mail.ru} 

\vspace*{3pt}

\noindent
\textbf{Ablaev Farid M.} (b.\ 1953)~--- Doctor of Science in physics and mathematics, professor, Head of 
Department of Theoretical Cybernetics, Institute of Computational Mathematics and Information 
Technologies, Kazan Federal University, 18~Kremlyovskaya Str., Kazan 420008, Russian Federation; 
\mbox{fablayev@gmail.com}

\vspace*{3pt}

\noindent
\textbf{Kochneva Yulia Yu.} (b.\ 1985)~--- scientist, Institute of Applied Research, Tatarstan Academy of 
Sciences, 36~Levobulachnaya Str., Kazan 420011, Russian Federation; \mbox{instpianrt@gmail.com}




\label{end\stat}

\renewcommand{\bibname}{\protect\rm Литература}   %16



%%%%%%%%%%%%%%%%%%%%%%%%%%%%%%%%%%%%%%%%

%\def\stat{rez}
{%\hrule\par
%\vskip 7pt % 7pt
\raggedleft\Large \bf%\baselineskip=3.2ex
Р\,Е\,Ц\,Е\,Н\,З\,И\,И \vskip 17pt
    \hrule
    \par
\vskip 6pt plus 6pt minus 3pt }

%\thispagestyle{headings} %с верхним колонтитулом
%\thispagestyle{myheadings} %с нижним колонтитулом, но в верхнем РЕЦЕНЗИИ

\def\tit{НОВАЯ КНИГА И.\,Н.~СИНИЦЫНА, А.\,С.~ШАЛАМОВА <<ЛЕКЦИИ ПО ТЕОРИИ 
ИНТЕГРИРОВАННОЙ ЛОГИСТИЧЕСКОЙ ПОДДЕРЖКИ>> (М.: ТОРУС ПРЕСС, 2012. 624~с.)}

%1
\def\aut{Д.ф.-м.н., профессор С.\,Я.~Шоргин}

\def\auf{\ }

\def\leftkol{\ % РЕЦЕНЗИИ
}

\def\rightkol{ \ } 

%\def\leftkol{\ } % ENGLISH ABSTRACTS}

%\def\rightkol{\ } %ENGLISH ABSTRACTS}

%\def\leftkol{РЕЦЕНЗИИ}

%\def\rightkol{РЕЦЕНЗИИ}

\titele{\tit}{\aut}{\auf}{\leftkol}{\rightkol}
\vspace*{-18pt}


     \label{st\stat}

     \begin{multicols}{2}
     {\small
     {\baselineskip=10.1pt
     

      В книге представлено системное изложение теоретических основ одного из новейших 
направлений в \mbox{об\-ласти} экономики послепродажного обслуживания изделий наукоемкой 
продукции (ИНП) длительного пользования~--- интегрированной логистической поддержки
(ИЛП). 
{\looseness=1

}

Приведены также результаты новых работ, выполненных в Институте проблем информатики 
Российской академии наук в рамках научного направления <<Информационные технологии и 
анализ сложных сис\-тем>>.
 {%\looseness=1

}
     
      Излагаемые в книге научные подходы позво\-ляют карди\-наль\-но реформировать 
существующие системы производства и эксплуатации ИНП путем создания и внед\-ре\-ния 
методов рационального и оптимального управ\-ле\-ния процессами расходования 
вре\-мен\-н$\acute{\mbox{ы}}$х, 
мате\-ри\-аль\-ных, трудовых и других ресурсов на всех стадиях жизненного цикла изделий (ЖЦИ) по 
критериям экономической целесообразности и эф\-фек\-тив\-ности.
  {\looseness=1

}
    
      В книге приведен краткий обзор причин возник\-новения и
      развития CALS-методологии как основы 
современных международных стандартов по созданию и функционированию глобальных 
ин\-фор\-ма\-ци\-он\-но-ком\-му\-ни\-ка\-ци\-он\-ных систем, ее ключевых возможностей и эффективности 
результатов ее использования. 
Авторы %\linebreak 
предлагают ряд научных обоснований для разработки 
единой теории проектирования и управления систем ИЛП для полноценного использования 
преимуществ %\linebreak
 суще\-ст\-ву\-ющей методологии, определяют \mbox{общую} структурную схему 
комплексной системы <<ИНП-СППО>> и необходимость разработки для ее описания 
гибридных стохастических моделей.
{%\looseness=1

}

%\columnbreak
      
      Книга состоит из пяти частей, где последовательно излагается материал по каждой из 
следующих тем: <<Интегрированная логистическая поддержка>>, <<Теория гибридных 
стохастических систем и компьютерная поддержка исследований и разработок>>, <<Основы 
математического моделирования, анализа и синтеза систем послепродажного обслуживания>>, 
<<Определение и анализ показателей экспортного потенциала ИНП при проектировании>>, 
<<Задачи управления поддержкой послепродажного обслуживания>>, а также 
<<Моделирование инвестиционных процессов ИЛП в условиях неравновесных финансовых 
рынков>>. 
   
      В конце каждой главы приведены выводы и даны вопросы и задания для 
самоконтроля. В~приложениях содержатся основные определения по программам работ по 
анализу ИЛП, логистическим базам данных и компьютерным решениям, эквивалентной статистической 
линеаризации нелинейных преобразований ИЛП, справочный материал, а также развернутые 
уравнения для вероятностных характеристик.


      \def\leftkol{РЕЦЕНЗИИ}

\def\rightkol{РЕЦЕНЗИИ} 

      
      Книга заинтересует широкий круг специалистов и может быть использована научными 
проектными организациями в сфере промышленного производства ИНП. Большое количество 
иллюстраций, примеров и вопросов, обращенных к читателю, позволяет использовать книгу 
также в качестве учебного пособия для студентов и аспирантов машиностроительных, 
транспортных и~других специальностей, а также для самостоятельного изучения. 
{%\looseness=-1

}

Книга 
представляет несомненный интерес для специалистов и студентов в области прикладной 
математики и информатики.
    

}

}
\end{multicols}

%\newpage

\def\stat{authorsrus}
{%\hrule\par
%\vskip 7pt % 7pt
\raggedleft\Large \bf%\baselineskip=3.2ex
О\,Б\ \ А\,В\,Т\,О\,Р\,А\,Х \vskip 17pt
    \hrule
    \par
\vskip 21pt plus 8pt minus 4pt }


\def\tit{\ }

\def\aut{\ }

\def\auf{\ }

\def\leftkol{\ } % ENGLISH ABSTRACTS}

\def\rightkol{ОБ АВТОРАХ} %ENGLISH ABSTRACTS}

\titele{\tit}{\aut}{\auf}{\leftkol}{\rightkol}
      
            \label{st\stat}



\vspace*{24pt}

\begin{multicols}{2}




\noindent
\textbf{Архипов Олег Петрович} (р.\ 1948)~---
кандидат технических наук, директор Орловского филиала Института проб\-лем информатики
Российской академии наук
%302025, г.Орел, Московское шоссе, д.137

\vspace*{3pt}

\noindent
\textbf{Бирюкова Татьяна Константиновна} (р.\ 1968)~---
кандидат фи\-зи\-ко-ма\-те\-ма\-ти\-че\-ских наук, старший научный сотрудник Института проб\-лем информатики
Российской академии наук

\vspace*{3pt}

\noindent 
\textbf{Бобков  Сергей Геннадьевич} (р.\ 1955)~---
доктор технических наук,  заведующий отделением На\-уч\-но-ис\-сле\-до\-ва\-тель\-ско\-го 
института системных исследований Российской академии наук
%117218, Москва, Нахимовский просп., 36, к.1 

\vspace*{3pt}

\noindent \textbf{Васильев Николай Семенович} (р.\ 1952)~--- доктор 
фи\-зи\-ко-ма\-те\-ма\-ти\-че\-ских наук, профессор, 
МГТУ им.\ Н.\,Э.~Баумана 
%, Москва 105005, 2-я Бауманская ул., д.~5,

\vspace*{3pt}

\noindent
\textbf{Гершкович Максим Михайлович} (р.\ 1968)~---
старший научный сотрудник Института проб\-лем информатики
Российской академии наук

\vspace*{3pt}

\noindent 
\textbf{Дьяченко Юрий Георгиевич} (р.\ 1958)~--- кандидат технических наук, 
старший научный сотрудник Института проб\-лем информатики
Российской академии наук

\vspace*{3pt}

\noindent 
\textbf{Ерошенко Александр Андреевич} (р.\ 1989)~--- аспирант кафедры 
математической статистики факультета вычисли\-тельной математики и кибернетики 
Московского государственного университета им.\ М.\,В.~Ломоносова
%119991, Москва ГСП-1, Ленинские горы, д.\ 1, стр. 52

\vspace*{3pt}
 
\noindent 
\textbf{Захаров Виктор Николаевич} (р.\ 1948)~--- 
доктор технических наук, доцент, ученый секретарь Института проб\-лем информатики
Российской академии наук

\vspace*{3pt}

\noindent
\textbf{Зейфман Александр Израилевич} (р.\ 1954)~---
доктор фи\-зи\-ко-ма\-те\-ма\-ти\-че\-ских наук, профессор, 
заведующий кафедрой Вологодского государственного университета; 
старший научный сотрудник Института проб\-лем информатики
Российской академии наук; главный научный сотрудник ИСЭРТ Российской академии наук

\vspace*{3pt}

\noindent
\textbf{Зыкин Сергей Владимирович} (р.\ 1959)~--- 
доктор технических наук, профессор, заведующий лабораторией Института математики 
им.\ С.\,Л.~Соболева Сибирского отделения Российской академии наук, Новосибирск 
%630090, пр.\ ак.\ Коптюга, 4 

\vspace*{4pt}

\noindent
\textbf{Киреев Владимир Иванович} (р.\ 1938)~---
доктор фи\-зи\-ко-ма\-те\-ма\-ти\-че\-ских наук, профессор Московского 
государственного горного университета
%Адрес: Россия, 119991, г. Москва, Ленинский проспект, д. 6

%\columnbreak

\vspace*{4pt}

\noindent
\textbf{Козеренко Елена Борисовна} (р.\ 1959)~---
кандидат филологических наук, заведующая лабораторией Института проб\-лем информатики
Российской академии наук

\vspace*{4pt}

\noindent
\textbf{Королев Виктор Юрьевич} (р.\ 1954)~--- доктор
фи\-зи\-ко-ма\-те\-ма\-ти\-че\-ских наук, профессор кафедры математической 
статистики факультета вычисли\-тельной математики и кибернетики 
Московского государственного университета; 
ведущий научный сотрудник Института проб\-лем информатики
Российской академии наук

\vspace*{4pt}

\noindent
\textbf{Коротышева Анна Владимировна} (р.\ 1988)~---
старший преподаватель Вологодского государственного университета

\vspace*{4pt}

\noindent 
\textbf{Кун Де Турк} (р.\ 1981)~--- научный сотрудник 
исследовательской группы SMACS факультета телекоммуникаций и обработки информации
Университета Гента, Бельгия
%В-9000 Гент, Бельгия

\vspace*{4pt}

\noindent
\textbf{Лупенцов Олег Сергеевич} (р.\ 1986)~---
аспирант Омского государственного института сервиса
%Омск 644043, ул.\ Певцова 13

\vspace*{4pt}

\noindent
\textbf{Лучко Олег Николаевич} (р.\ 1961)~---
кандидат педагогических наук, профессор, заведующий кафедрой 
Омского государственного института сервиса
%Омск 644043, ул.\ Певцова 13

\vspace*{4pt}

\noindent
\textbf{Малашенко Юрий Евгеньевич} (р.\ 1946)~---
доктор фи\-зи\-ко-ма\-те\-ма\-ти\-че\-ских наук, заведующий сектором 
Вычислительного центра им.\ А.\,А.~Дородницына Российской академии наук
%Адрес: 119333, Москва, ул. Вавилова, 40,

\vspace*{4pt}

\noindent
\textbf{Маньяков Юрий Анатольевич} (р.\ 1984)~---
кандидат технических наук, научный сотрудник Орловского филиала Института проб\-лем информатики
Российской академии наук
%302025, г.Орел, Московское шоссе, д.137

\vspace*{4pt}

\noindent
\textbf{Маренко Валентина Афанасьевна} (р.\ 1951)~---
кандидат технических наук, доцент, старший научный сотрудник 
Института математики им.\ С.\,Л.~Соболева Сибирского отделения Российской академии наук
%Новосибирск 630090, пр. ак. Коптюга, 4 

\vspace*{3pt}

\noindent 
\textbf{Морозов Евсей Викторович} (р.\ 1947)~--- доктор 
фи\-зи\-ко-ма\-те\-ма\-ти\-че\-ских, профессор, ведущий научный сотрудник 
Института прикладных математических исследований Карельского научного центра Российской
академии наук; 
%%185910 Россия, Республика Карелия, г.\ Петрозаводск, ул.\ Пушкинская, 11
профессор Петрозаводского государственного университета, Петрозаводск
%185910 Россия, Республика Карелия, г.\ Петрозаводск, пр.\ Ленина, 33

%\pagebreak

\vspace*{3pt}

\noindent
\textbf{Назарова Ирина Александровна} (р.\ 1966)~---
кандидат фи\-зи\-ко-ма\-те\-ма\-ти\-че\-ских наук, 
научный сотрудник Вычислительного центра им.\ А.\,А.~Дородницына Российской академии наук 
%Адрес: 119333, Москва, ул. Вавилова, 40

\vspace*{3pt}

\noindent
\textbf{Павлов Игорь Валерианович} (р.\ 1945)~--- 
доктор фи\-зи\-ко-ма\-те\-ма\-ти\-че\-ских наук, профессор МГТУ им.\ Н.\,Э.~Баумана 
%Москва 105005, 2-я Бауманская ул., д.~5 

%\pagebreak

\vspace*{3pt}

\noindent 
\textbf{Потахина Любовь Викторовна} (р.\ 1989)~--- аспирантка
Института прикладных математических исследований Карельского научного центра
Российской академии наук; 
%%185910 Россия, Республика Карелия, г.\ Петрозаводск, ул.\ Пушкинская, 11
инженер Петрозаводского государственного университета, Петрозаводск
%185910 Россия, Республика Карелия, г.\ Петрозаводск, пр.\ Ленина, 33

\vspace*{3pt}

\noindent 
\textbf{Рождественский Юрий Владимирович} (р.\ 1952)~--- 
кандидат технических наук, заведующий сектором Института проб\-лем информатики
Российской академии наук

\vspace*{3pt}

\noindent 
\textbf{Синицын Игорь Николаевич} (р.\ 1940)~--- доктор технических наук,
профессор, заслуженный деятель\linebreak\vspace*{-12pt}

\columnbreak

\noindent
 науки РФ, заведующий отделом Института проб\-лем информатики
Российской академии наук

\vspace*{7pt}


\noindent
\textbf{Сиротинин Денис Олегович} (р.\ 1984)~---
кандидат технических наук, научный сотрудник Орловского филиала Института проб\-лем информатики
Российской академии наук
%302025, г.Орел, Московское шоссе, д.137

\vspace*{7pt}

%\columnbreak

\noindent 
\textbf{Соколов  Игорь Анатольевич} (р.\ 1954)~--- академик (действительный член) Российской 
академии наук, доктор технических наук, директор Института проб\-лем информатики
Российской академии наук

\vspace*{7pt}

\noindent
\textbf{Степченков Юрий Афанасьевич} (р.\ 1951)~---
кандидат технических наук, заведующий отделом Института проб\-лем информатики
Российской академии наук

\vspace*{7pt}

\noindent
\textbf{Сурков Алексей Викторович} (р.\ 1978)~--- 
старший научный сотрудник На\-уч\-но-ис\-сле\-до\-ва\-тель\-ско\-го 
института системных исследований Российской академии наук
%117218, Москва, Нахимовский просп., 36, к.1 

\vspace*{7pt}

\noindent 
\textbf{Шестаков Олег Владимирович} (р.\ 1976)~--- доктор 
фи\-зи\-ко-ма\-те\-ма\-ти\-че\-ских, доцент кафедры математической статистики 
факультета вычисли\-тельной математики и кибернетики Московского 
государственного университета им.\ М.\,В.~Ломоносова; 
%119991, Москва ГСП-1, Ленинские горы, д.\ 1, стр. 52
старший научный сотрудник Института проб\-лем информатики
Российской академии наук
%, Москва 119333, ул. Вавилова, д.~44, корп.~2

\vspace*{7pt}

\noindent 
\textbf{Шоргин Сергей Яковлевич} (р.\ 1952.)~--- доктор
фи\-зи\-ко-ма\-те\-ма\-ти\-че\-ских наук, профессор, заместитель директора Института 
проб\-лем информатики Российской академии наук





%%%%%%%%%%%%%%%%%%%%%%%%%%%%%%%%%%%%%%%%%%%%%%%%%%%%%%%%%%%%%%%%%%%%%%%%%%%%%%%




%\def\rightkol{ОБ АВТОРАХ}
%\def\leftkol{ОБ АВТОРАХ}

 \label{end\stat}





%\def\leftfootline{\small{\textbf{\thepage}
%\hfill ИНФОРМАТИКА И ЕЁ ПРИМЕНЕНИЯ\ \ \ том~7\ \ \ выпуск~1\ \ \ 2013}
%}%
% \def\rightfootline{\small{ИНФОРМАТИКА И ЕЁ ПРИМЕНЕНИЯ\ \ \ том~7\ \ \ выпуск~1\ \ \ 2013
%\hfill \textbf{\thepage}}}


%\thispagestyle{myheadings}



\end{multicols}

\newpage  

%\def\stat{cont}
{%\hrule\par
%\vskip 7pt % 7pt
\raggedleft\Large \bf%\baselineskip=3.2ex
А\,В\,Т\,О\,Р\,С\,К\,И\,Й\ \ У\,К\,А\,З\,А\,Т\,Е\,Л\,Ь\ \ З\,А\ \ 2\,0\,0\,7 г. \vskip 17pt
    \hrule
    \par
\vskip 21pt plus 6pt minus 3pt }

\label{st\stat}

\def\tit{\ }

\def\aut{\ }
\def\auf{\ }

\def\leftkol{\ } % ENGLISH ABSTRACTS}

\def\rightkol{\ } %ENGLISH ABSTRACTS}

\titele{\tit}{\aut}{\auf}{\leftkol}{\rightkol}


\contentsline {chapter}{\ }{Выпуск \quad Стр.} 
\contentsline {section}{\textbf{Батракова Д.\,А., Королев В.\,Ю., Шоргин С.\,Я.}\ \ Новый метод вероятностно-ста\-ти\-сти\-че\-ско\-го анализа информационных потоков в\nobreakspace {}телекоммуникационных сетях}{\qquad 1 \qquad 40} 
\contentsline {section}{\textbf{Борисов А.\,В.}\ \ Байесовское оценивание в системах наблюдения с\nobreakspace {}марковскими скачкообразными процессами: игровой подход}{\qquad 2 \qquad 65}
\contentsline {section}{\textbf{Босов А.\,В., Иванов А.\,В.}\ \ Программная инфраструктура информационного Web-пор\-тала}{\qquad 2 \qquad 50}
\contentsline {section}{\textbf{Захаров В.\,Н., Калиниченко Л.\,А., Соколов И.\,А., Ступников С.\,А.}\ \ Конструирование канонических информационных моделей для интегрированных информационных систем}{\qquad 2 \qquad 15}
\contentsline {section}{\textbf{Захаров В.\,Н., Козмидиади В.\,А.}\ \ Средства обеспечения отказоустойчивости при\-ло\-жений}{\qquad 1 \qquad 14} 
\contentsline {section}{\textbf{Иванов А.\,В.}\ \ см. Босов А.\,В.\hfill\hfill\hfill\hfill\hfill\hfill\hfill\hfill\hfill\hfill\hfill\hfill\hfill\hfill\hfill\hfill\hfill\hfill\hfill\hfill\hfill\hfill\hfill\hfill\hfill\hfill\hfill\hfill\hfill\hfill\hfill\hfill\hfill\hfill\hfill}{\ }
\contentsline {section}{\textbf{Ильин В.\,Д., Соколов И.\,А.}\ \ Символьная модель системы знаний информатики в\nobreakspace {}че\-ло\-ве\-ко-автоматной среде}{\qquad 1 \qquad 66} 
\contentsline {section}{\textbf{Калиниченко Л.\,А.}\ \ см. Захаров В.\,Н.\hfill\hfill\hfill\hfill\hfill\hfill\hfill\hfill\hfill\hfill\hfill\hfill\hfill\hfill\hfill\hfill\hfill\hfill\hfill\hfill\hfill\hfill\hfill\hfill\hfill\hfill\hfill\hfill\hfill\hfill\hfill\hfill\hfill\hfill\hfill}{\ }
\contentsline {section}{\textbf{Козеренко Е.\,Б.}\ \ Лингвистическое моделирование для систем машинного перевода и обработки знаний}{\qquad 1 \qquad 54} 
\contentsline {section}{\textbf{Козмидиади В.\,А.}\ \ см. Захаров В.\,Н.\hfill\hfill\hfill\hfill\hfill\hfill\hfill\hfill\hfill\hfill\hfill\hfill\hfill\hfill\hfill\hfill\hfill\hfill\hfill\hfill\hfill\hfill\hfill\hfill\hfill\hfill\hfill\hfill\hfill\hfill\hfill\hfill\hfill\hfill\hfill }{\ } 
\contentsline {section}{\textbf{Королев В.\,Ю.}\ \ см. Батракова Д.\,А.\hfill\hfill\hfill\hfill\hfill\hfill\hfill\hfill\hfill\hfill\hfill\hfill\hfill\hfill\hfill\hfill\hfill\hfill\hfill\hfill\hfill\hfill\hfill\hfill\hfill\hfill\hfill\hfill\hfill\hfill\hfill\hfill\hfill\hfill\hfill}{\ } 
\contentsline {section}{\textbf{Кудрявцев А.\,А., Шоргин С.\,Я.}\ \ Байесовский подход к\nobreakspace {}анализу систем массового обслуживания и\nobreakspace {}показателей надежности}{\qquad 2 \qquad 76}
\contentsline {section}{\textbf{Печинкин А.\,В., Соколов И.\,А., Чаплыгин В.\,В.}\ \ Многолинейная система массового обслуживания с конечным накопителем и ненадежными приборами}{\qquad 1 \qquad 27} 
\contentsline {section}{\textbf{Печинкин А.\,В., Соколов И.\,А., Чаплыгин В.\,В.}\ \ Стационарные характеристики многолинейной\nobreakspace {}системы массового обслуживания с\nobreakspace {}одновременными отказами приборов}{\qquad 2 \qquad 39}
\contentsline {section}{\textbf{Синицын И.\,Н.}\ \ Корреляционные методы построения аналитических информационных моделей флуктуаций полюса Земли по априорным данным}{\qquad 2 \qquad \hphantom{9}2}
\contentsline {section}{\textbf{Синицын И.\,Н.}\ \ Развитие теории фильтров Пугачева для оперативной обработки информации в стохастических системах}{{\qquad 1 \qquad \hphantom{9}3}} 
\contentsline {section}{\textbf{Соколов И.\,А.}\ \ см. Захаров В.\,Н.\hfill\hfill\hfill\hfill\hfill\hfill\hfill\hfill\hfill\hfill\hfill\hfill\hfill\hfill\hfill\hfill\hfill\hfill\hfill\hfill\hfill\hfill\hfill\hfill\hfill\hfill\hfill\hfill\hfill\hfill\hfill\hfill\hfill\hfill\hfill}{\ }
\contentsline {section}{\textbf{Соколов И.\,А.}\ \ см. Ильин В.\,Д.\hfill\hfill\hfill\hfill\hfill\hfill\hfill\hfill\hfill\hfill\hfill\hfill\hfill\hfill\hfill\hfill\hfill\hfill\hfill\hfill\hfill\hfill\hfill\hfill\hfill\hfill\hfill\hfill\hfill\hfill\hfill\hfill\hfill\hfill\hfill}{\ } 
\contentsline {section}{\textbf{Соколов И.\,А.}\ \ см. Печинкин А.\,В.\hfill\hfill\hfill\hfill\hfill\hfill\hfill\hfill\hfill\hfill\hfill\hfill\hfill\hfill\hfill\hfill\hfill\hfill\hfill\hfill\hfill\hfill\hfill\hfill\hfill\hfill\hfill\hfill\hfill\hfill\hfill\hfill\hfill\hfill\hfill}{\ } 
\contentsline {section}{\textbf{Соколов И.\,А.}\ \ см. Печинкин А.\,В.\hfill\hfill\hfill\hfill\hfill\hfill\hfill\hfill\hfill\hfill\hfill\hfill\hfill\hfill\hfill\hfill\hfill\hfill\hfill\hfill\hfill\hfill\hfill\hfill\hfill\hfill\hfill\hfill\hfill\hfill\hfill\hfill\hfill\hfill\hfill}{\ }
\contentsline {section}{\textbf{Ступников С.\,А.}\ \ см. Захаров В.\,Н.\hfill\hfill\hfill\hfill\hfill\hfill\hfill\hfill\hfill\hfill\hfill\hfill\hfill\hfill\hfill\hfill\hfill\hfill\hfill\hfill\hfill\hfill\hfill\hfill\hfill\hfill\hfill\hfill\hfill\hfill\hfill\hfill\hfill\hfill\hfill}{\ }
\contentsline {section}{\textbf{Чаплыгин В.\,В.}\ \ см. Печинкин А.\,В.\hfill\hfill\hfill\hfill\hfill\hfill\hfill\hfill\hfill\hfill\hfill\hfill\hfill\hfill\hfill\hfill\hfill\hfill\hfill\hfill\hfill\hfill\hfill\hfill\hfill\hfill\hfill\hfill\hfill\hfill\hfill\hfill\hfill\hfill\hfill}{\ } 
\contentsline {section}{\textbf{Чаплыгин В.\,В.}\ \ см. Печинкин А.\,В.\hfill\hfill\hfill\hfill\hfill\hfill\hfill\hfill\hfill\hfill\hfill\hfill\hfill\hfill\hfill\hfill\hfill\hfill\hfill\hfill\hfill\hfill\hfill\hfill\hfill\hfill\hfill\hfill\hfill\hfill\hfill\hfill\hfill\hfill\hfill}{\ }
\contentsline {section}{\textbf{Шоргин С.\,Я.}\ \ см. Батракова Д.\,А.\hfill\hfill\hfill\hfill\hfill\hfill\hfill\hfill\hfill\hfill\hfill\hfill\hfill\hfill\hfill\hfill\hfill\hfill\hfill\hfill\hfill\hfill\hfill\hfill\hfill\hfill\hfill\hfill\hfill\hfill\hfill\hfill\hfill\hfill\hfill}{\ } 
\contentsline {section}{\textbf{Шоргин С.\,Я.}\ \ см. Кудрявцев А.\,А.\hfill\hfill\hfill\hfill\hfill\hfill\hfill\hfill\hfill\hfill\hfill\hfill\hfill\hfill\hfill\hfill\hfill\hfill\hfill\hfill\hfill\hfill\hfill\hfill\hfill\hfill\hfill\hfill\hfill\hfill\hfill\hfill\hfill\hfill\hfill}{\ }
%\thispagestyle{myheadings}
\def\leftfootline{\small{\textbf{\thepage}
\hfill ИНФОРМАТИКА И ЕЁ ПРИМЕНЕНИЯ\ \ \ том~1\ \ \ выпуск~2\ \ \ 2007}
}%
 \def\rightfootline{\small{ИНФОРМАТИКА И ЕЁ ПРИМЕНЕНИЯ\ \ \ том~1\ \ \ выпуск~2\ \ \ 2007
 \hfill \textbf{\thepage}}}
 \label{end\stat} 
                     
%\def\stat{cont-e}
{%\hrule\par
%\vskip 7pt % 7pt
\raggedleft\Large \bf%\baselineskip=3.2ex
2\,0\,0\,7\ \ A\,U\,T\,H\,O\,R\ \ I\,N\,D\,E\,X \vskip 17pt
    \hrule
    \par
\vskip 21pt plus 6pt minus 3pt }

\label{st\stat}

\def\tit{\ }

\def\aut{\ }
\def\auf{\ }

\def\leftkol{\ } % ENGLISH ABSTRACTS}

\def\rightkol{\ } %ENGLISH ABSTRACTS}

\titele{\tit}{\aut}{\auf}{\leftkol}{\rightkol}


\contentsline {chapter}{\ }{Issue \quad Page} 
\contentsline {subsection}{\textbf{Batrakova D.\,A., Korolev V.\,Yu., Shorgin S.\,Ya.}\ \ A New Method for the Probabilistic and Statistical Analysis of Information Flows in Telecommunication Networks}{\qquad 1 \qquad 40} 
\contentsline {subsection}{\textbf{Borisov A.\,V.}\ \ Bayesian Estimation in\nobreakspace {}Observation Systems with\nobreakspace {}Markov Jump Processes: Game-Theoretic Approach}{\qquad 2 \qquad 65} 
\contentsline {subsection}{\textbf{Bosov A.\,V., Ivanov A.\,V.}\ \ Linguistic Simulation for Machine Translation and Knowledge Management Systems}{\qquad 2 \qquad 50} 
\contentsline {subsection}{\textbf{Chaplygin V.\,V.} see Pechinkin A.\,V.\hfill\hfill\hfill\hfill\hfill\hfill\hfill\hfill\hfill\hfill\hfill\hfill\hfill\hfill\hfill\hfill\hfill\hfill\hfill\hfill\hfill\hfill\hfill\hfill\hfill\hfill\hfill\hfill\hfill\hfill\hfill\hfill\hfill\hfill\hfill}{\ }
\contentsline {subsection}{\textbf{Chaplygin V.\,V.} see Pechinkin A.\,V.\hfill\hfill\hfill\hfill\hfill\hfill\hfill\hfill\hfill\hfill\hfill\hfill\hfill\hfill\hfill\hfill\hfill\hfill\hfill\hfill\hfill\hfill\hfill\hfill\hfill\hfill\hfill\hfill\hfill\hfill\hfill\hfill\hfill\hfill\hfill}{\ }
\contentsline {subsection}{\textbf{Ilyin V.\,D., Sokolov I.\,A.}\ \ The Symbol Model of Informatics Knowledge System in Human-Automaton Environment}{\qquad 1 \qquad 66} 
\contentsline {subsection}{\textbf{Ivanov A.\,V.} see Bosov A.\,V.\hfill\hfill\hfill\hfill\hfill\hfill\hfill\hfill\hfill\hfill\hfill\hfill\hfill\hfill\hfill\hfill\hfill\hfill\hfill\hfill\hfill\hfill\hfill\hfill\hfill\hfill\hfill\hfill\hfill\hfill\hfill\hfill\hfill\hfill\hfill}{\ }
\contentsline {subsection}{\textbf{Kalinichenko L.\,A.} see Zakharov V.\,N.\hfill\hfill\hfill\hfill\hfill\hfill\hfill\hfill\hfill\hfill\hfill\hfill\hfill\hfill\hfill\hfill\hfill\hfill\hfill\hfill\hfill\hfill\hfill\hfill\hfill\hfill\hfill\hfill\hfill\hfill\hfill\hfill\hfill\hfill\hfill}{\ }
\contentsline {subsection}{\textbf{Korolev V.\,Yu.} see Batrakova D.\,A.\hfill\hfill\hfill\hfill\hfill\hfill\hfill\hfill\hfill\hfill\hfill\hfill\hfill\hfill\hfill\hfill\hfill\hfill\hfill\hfill\hfill\hfill\hfill\hfill\hfill\hfill\hfill\hfill\hfill\hfill\hfill\hfill\hfill\hfill\hfill}{\ }
\contentsline {subsection}{\textbf{Kozerenko E.\,B.}\ \ Linguistic Simulation for Machine Translation and Knowledge Management Systems}{\qquad 1 \qquad 54} 
\contentsline {subsection}{\textbf{Kozmidiady V.\,A.} see Zakharov V.\,N.\hfill\hfill\hfill\hfill\hfill\hfill\hfill\hfill\hfill\hfill\hfill\hfill\hfill\hfill\hfill\hfill\hfill\hfill\hfill\hfill\hfill\hfill\hfill\hfill\hfill\hfill\hfill\hfill\hfill\hfill\hfill\hfill\hfill\hfill\hfill}{\ }
\contentsline {subsection}{\textbf{Kudryavtsev A.\,A., Shorgin S.\,Ya.}\ \ Bayesian Approach to Queueing Systems and Reliability Characteristics}{\qquad 2 \qquad 76} 
\contentsline {subsection}{\textbf{Pechinkin A.\,V., Sokolov I.\,A., Chaplygin V.\,V.}\ \ Multichannel Queuing System with Finite Buffer and Unreliable Servers}{\qquad 1 \qquad 27} 
\contentsline {subsection}{\textbf{Pechinkin A.\,V., Sokolov I.\,A., Chaplygin V.\,V.}\ \ Stationary Characteristics of a Multichannel Queueing System with\nobreakspace {}Simultaneous Refusals of Servers}{\qquad 2 \qquad 39} 
\contentsline {subsection}{\textbf{Shorgin S.\,Ya.} see Batrakova D.\,A.\hfill\hfill\hfill\hfill\hfill\hfill\hfill\hfill\hfill\hfill\hfill\hfill\hfill\hfill\hfill\hfill\hfill\hfill\hfill\hfill\hfill\hfill\hfill\hfill\hfill\hfill\hfill\hfill\hfill\hfill\hfill\hfill\hfill\hfill\hfill}{\ }
\contentsline {subsection}{\textbf{Shorgin S.\,Ya.} see Kudryavtsev A.\,A.\hfill\hfill\hfill\hfill\hfill\hfill\hfill\hfill\hfill\hfill\hfill\hfill\hfill\hfill\hfill\hfill\hfill\hfill\hfill\hfill\hfill\hfill\hfill\hfill\hfill\hfill\hfill\hfill\hfill\hfill\hfill\hfill\hfill\hfill\hfill}{\ }
\contentsline {subsection}{\textbf{Sinitsyn I.\,N.}\ \ Correlational Methods for Analytical Informational Models of the Earth Pole Fluctuations Design Based on a priori Data}{\qquad 2 \qquad \hphantom{9}2}
\contentsline {subsection}{\textbf{Sinitsyn I.\,N.}\ \ Development of Pugachev Filtering for Stochastic Systems}{\qquad 1 \qquad \hphantom{9}3}
\contentsline {subsection}{\textbf{Sokolov I.\,A.} see Ilyin V.\,D.\hfill\hfill\hfill\hfill\hfill\hfill\hfill\hfill\hfill\hfill\hfill\hfill\hfill\hfill\hfill\hfill\hfill\hfill\hfill\hfill\hfill\hfill\hfill\hfill\hfill\hfill\hfill\hfill\hfill\hfill\hfill\hfill\hfill\hfill\hfill}{\ }
\contentsline {subsection}{\textbf{Sokolov I.\,A.} see Pechinkin A.\,V.\hfill\hfill\hfill\hfill\hfill\hfill\hfill\hfill\hfill\hfill\hfill\hfill\hfill\hfill\hfill\hfill\hfill\hfill\hfill\hfill\hfill\hfill\hfill\hfill\hfill\hfill\hfill\hfill\hfill\hfill\hfill\hfill\hfill\hfill\hfill}{\ }
\contentsline {subsection}{\textbf{Sokolov I.\,A.} see Pechinkin A.\,V.\hfill\hfill\hfill\hfill\hfill\hfill\hfill\hfill\hfill\hfill\hfill\hfill\hfill\hfill\hfill\hfill\hfill\hfill\hfill\hfill\hfill\hfill\hfill\hfill\hfill\hfill\hfill\hfill\hfill\hfill\hfill\hfill\hfill\hfill\hfill}{\ }
\contentsline {subsection}{\textbf{Sokolov I.\,A.} see Zakharov V.\,N.\hfill\hfill\hfill\hfill\hfill\hfill\hfill\hfill\hfill\hfill\hfill\hfill\hfill\hfill\hfill\hfill\hfill\hfill\hfill\hfill\hfill\hfill\hfill\hfill\hfill\hfill\hfill\hfill\hfill\hfill\hfill\hfill\hfill\hfill\hfill}{\ }
\contentsline {subsection}{\textbf{Stupnikov S.\,A.} see Zakharov V.\,N.\hfill\hfill\hfill\hfill\hfill\hfill\hfill\hfill\hfill\hfill\hfill\hfill\hfill\hfill\hfill\hfill\hfill\hfill\hfill\hfill\hfill\hfill\hfill\hfill\hfill\hfill\hfill\hfill\hfill\hfill\hfill\hfill\hfill\hfill\hfill}{\ }
\contentsline {subsection}{\textbf{Zakharov V.\,N., Kalinichenko L.\,A., Sokolov I.\,A., Stupnikov S.\,A.}\ \ Development of Canonical Information Models for Integrated Information Systems}{\qquad 2 \qquad 15} 
\contentsline {subsection}{\textbf{Zakharov V.\,N., Kozmidiady V.\,A.}\ \ Means Providing Applications Fault Tolerance}{\qquad 1 \qquad 14} 
\def\leftfootline{\small{\textbf{\thepage}
\hfill ИНФОРМАТИКА И ЕЁ ПРИМЕНЕНИЯ\ \ \ том~1\ \ \ выпуск~2\ \ \ 2007}
}%
 \def\rightfootline{\small{ИНФОРМАТИКА И ЕЁ ПРИМЕНЕНИЯ\ \ \ том~1\ \ \ выпуск~2\ \ \ 2007
 \hfill \textbf{\thepage}}}
 \label{end\stat} 


%\end{document}

%
\def\stat{rekl}
%\label{preobr}

%\def\tit{АКАДЕМИК ПУГАЧЁВ  ВЛАДИМИР СЕМЁНОВИЧ\\
%25.03.1911--25.03.1998}


%   \vspace*{-48pt}
%   \begin{center}\LARGE
%Академик Пугачёв  Владимир Семёнович\\ (25.03.1911--25.03.1998)
%   \end{center}

   %\vspace*{2.5mm}

   \begin{center}

{\prgsh\LARGE
ЮБИЛЕИ}

\end{center}
%\hrule

\vspace*{6pt}


   \vspace*{8mm}

   \thispagestyle{empty}


%\def\stat{emel}


\section*{К 70-летию заместителя директора ИПИ РАН,\\ члена редколлегии журнала
<<Информатика и её применения>>\\ доктора технических наук В.\,И.~Будзко}

\vspace*{18pt}




          \begin{multicols}{2}

%            \label{st\stat}

\begin{center}
\vspace*{1pt}
\mbox{%
\epsfxsize=78mm
\epsfbox{bud-1.eps}
}
\end{center}

\vspace*{12pt}

      14 августа 2014~г.\ исполнилось 70~лет за\-мес\-ти\-те\-лю директора ИПИ РАН по
научной работе доктору технических наук Владимиру Игоревичу Будзко.

      Владимир Игоревич Будзко родился в г.~Москве. Высшее образование получил на факультете
элект\-рон\-но-вы\-чис\-ли\-тель\-ных устройств в Московском
ин\-же\-нер\-но-фи\-зи\-че\-ском институте
(МИФИ), который он окончил в 1968~г., после чего был на\-прав\-лен для прохождения
службы в одну из войс\-ко\-вых частей, где прошел путь от инженера до первого заместителя
командира войсковой части.

      С приходом В.\,И.~Будзко в ИПИ РАН (2001~г.)\ в институте
сформировалось новое научное на\-прав\-ле\-ние теоретических исследований~--- <<Постро\-ение
ин\-фор\-ма\-ци\-он\-но-те\-ле\-ком\-му\-ни\-ка\-ци\-он\-ных\linebreak сис\-тем
высокой до\-ступ\-ности>>. В~рамках этого
направления выполнен широкий круг фундаментальных исследований по поиску подходов и
определению принципов построения средств обеспечения доступности, конфиденциальности
и целостности современных крупномасштабных
ин\-фор\-ма\-ци\-он\-но-те\-ле\-ком\-му\-ни\-ка\-ци\-он\-ных
сис\-тем (ИТС). Разработаны основные сис\-тем\-но-тех\-ни\-че\-ские принципы и базовые
архитектурные решения построения перспективных для условий России ИТС с
централизованной обработкой и хранением информации, сочетающих в себе свойства
высокой доступности, отказо- и катастрофоустойчивости, информационной защищенности.
Определены принципы, методы и математические основы рационального построения и
оптимизации средств восстановления функционирования центров обработки данных (ЦОД)
после возникновения отказов и катастроф, передачи и хранения данных, обеспечения
информационной безопасности при достижении минимальной совокупной стоимости
владения такими системами. Результаты нашли практическое воплощение при реализации
проектов в интересах ряда отечественных государственных и негосударственных
организаций, таких как Банк России (БР), Внешторгбанк, ОАО <<ГМК <<Норильский Никель>>,
<<Газпром>>, Минэкономразвития России, Правительство Москвы, а также ряд силовых
ведомств.

      Под руководством В.\,И.~Будзко начиная с 2001~г.\ выполнен комплекс
      на\-уч\-но-ис\-сле\-до\-ва\-тель\-ских и
      опыт\-но-кон\-ст\-рук\-тор\-ских работ (свыше 100~проектов),
направленных на развитие электронной информационной технологии БР.
Разработаны концепции развития ИТС БР сначала до 2008~г., а затем до 2013~г., которые
были приняты в качестве основы проведения технической политики. За реализацию проекта
<<Катастрофоустойчивая тер\-ри\-то\-ри\-аль\-но-рас\-пре\-де\-лен\-ная
      ин\-фор\-ма\-ци\-он\-но-те\-ле\-ком\-му\-ни\-ка\-ци\-он\-ная сис\-те\-ма централизованной
обработки банковской информации>> В.\,И.~Будзко удостоен Премии Правительства РФ в
области науки и техники за 2010~г.

      В.\,И.~Будзко возглавлял и возглавляет работы по ряду других прикладных проектов,
связанных с созданием, совершенствованием и развитием крупномасштабных ИТС.

      В.\,И.~Будзко~--- генерал-майор, доктор технических наук, член-кор\-рес\-пон\-дент
Академии криптографии РФ, известный ученый в области информатики и применения
информационных технологий при построении территориально распределенных ИТС
различного назначения. Является автором свыше 250~научных работ, опубликованных в
на\-уч\-но-тех\-ни\-че\-ских и специальных изданиях.

    \thispagestyle{empty}

      В.\,И.~Будзко уделяет большое внимание подготовке научных кадров. Под его
руководством защищено 6~диссертаций на соискание ученой степени кандидата
технических наук. Свыше 30~лет он читает лекции в ИКСИ Академии ФСБ, профессор
кафедры НИЯУ МИФИ. Является членом двух диссертационных советов, главным
редактором журнала <<Системы высокой доступности>> и членом редколлегии журнала
<<Информатика и её применения>>.

      \bigskip

      Редакционный совет и Редакционная коллегия журнала <<Информатика и её
применения>> сердечно поздравляют Владимира Игоревича Будзко с 70-ле\-ти\-ем и желают
крепкого здоровья и новых научных достижений.

\end{multicols}

%%Информатика и её применения
%Том 14 Выпуск 1-4 Год 2020

\def\stat{cont}
{%\hrule\par
%\vskip 7pt % 7pt
\raggedleft\Large \bf%\baselineskip=3.2ex
А\,В\,Т\,О\,Р\,С\,К\,И\,Й\ \ У\,К\,А\,З\,А\,Т\,Е\,Л\,Ь\ \ З\,А\ \ 2\,0\,2\,0 г. \vskip 17pt
 \hrule
 \par
\vskip 21pt plus 6pt minus 3pt }

\label{st\stat}

\def\tit{\ }

\def\aut{\ }
\def\auf{\ }

\def\leftkol{\ } % ENGLISH ABSTRACTS}

\def\rightkol{\ } %АВТОРСКИЙ УКАЗАТЕЛЬ ЗА 2020 г.} %ENGLISH ABSTRACTS}

\titele{\tit}{\aut}{\auf}{\leftkol}{\rightkol}
\addcontentsline{toc}{subsection}{\textrm\textbf Авторский указатель за 2020 г.}

\vspace*{-24pt}

\noindent
{\tabcolsep=3pt
\begin{tabular}{p{397pt}cc}
&\textbf{Вып.} & \textbf{Стр.}\\[6pt]
\Avtors{Абгарян~К.\,К., Гаврилов~Е.\,С.} Интеграционная платформа для многомасштабного моде-\linebreak
\\[-12pt]
\hspace*{23pt}лирования нейроморфных систем&2&104--110\\
\Avtors{Абгарян~К.\,К., Колбин~И.\,С.} Применение многомасштабного подхода и методов анализа\linebreak
\\[-12pt]
\hspace*{23pt}данных для моделирования теплопроводности в слоистых структурах&4&91--99\\
\Avtors{Агаларов~Я.\,М.} Оптимизация емкости основного накопителя в системе массового\linebreak
\\[-12pt]
\hspace*{23pt}обслуживания типа $G/M/1/K$ с дополнительным накопителем&2&72--79\\
\Avtors{Агасандян~Г.\,А.} Вычислительные аспекты применения CC-VaR на совокупности рынков&3&62--70\\
\Avtors{Агеев~К.\,А., Сопин~Э.\,С., Яркина~Н.\,В., Самуйлов~К.\,Е., Шоргин~С.\,Я.} Анализ механизмов\linebreak
\\[-12pt]
\hspace*{23pt}нарезки сети с учетом гарантий для различных типов трафика&3&\hphantom{1}94--100\\
\Avtors{Адамова~К.\,А.} см.\ Шнурков~П.\,В.&&\\
\Avtors{Базилевский~М.\,П.} Многофакторные модели полносвязной линейной регрессии без\linebreak
\\[-12pt]
\hspace*{23pt}ограничений на соотношения дисперсий ошибок переменных&2&92--97\\
\Avtors{Бахтеев~О.\,Ю.} см.\ Грабовой~А.\,В.&&\\
\Avtors{Беленков~В.\,Г.} см.\ Будзко~В.\,И.&&\\
\Avtors{Бетелин~В.\,Б., Кушниренко~А.\,Г., Леонов~А.\,Г.} Основные понятия программирования\linebreak
\\[-12pt]
\hspace*{23pt}в изложении для дошкольников&3&55--61\\
\Avtors{Бетелин~В.\,Б., Кушниренко~А.\,Г., Семенов~А.\,Л., Сопрунов~С.\,Ф.} О цифровой грамотности\linebreak
\\[-12pt]
\hspace*{23pt}и средах ее формирования&4&100--107\\
\Avtors{Борисов~А.\,В.} Численные схемы фильтрации марковских скачкообразных процессов по\linebreak
\\[-12pt]
\hspace*{23pt}дискретизованным наблюдениям II: случай аддитивных шумов&1&17--23\\
\Avtors{Борисов~А.\,В.} Численные схемы фильтрации марковских скачкообразных процессов по\linebreak
\\[-12pt]
\hspace*{23pt}дискретизованным наблюдениям III: случай мультипликативных шумов&2&10--18\\
\Avtors{Босов~А.\,В.} Управление выходом стохастической дифференциальной системы по квад-\linebreak
\\[-12pt]
\hspace*{23pt}ратичному критерию. V. Случай неполной информации о состоянии&2&19--25\\
\Avtors{Босов~А.\,В., Мартюшова~Я.\,Г., Наумов~А.\,В., Сапунова~А.\,П.} Байесовский подход к~по\-стро\-ению индивидуальной траектории пользователя в~системе дистанционного\linebreak
\\[-12pt]
\hspace*{23pt}обучения&3&86--93\\
\Avtors{Босов~А.\,В., Стефанович~А.\,И.} Управление выходом стохастической дифференциальной\linebreak
\\[-12pt]
\hspace*{23pt}системы по квадратичному критерию. IV. Альтернативное численное решение&1&24--30\\
\Avtors{Брюхов~Д.\,О., Ступников~С.\,А., Ковалёв~Д.\,Ю., Шанин~И.\,А.} Нейрофизиология как\linebreak
\\[-12pt]
\hspace*{23pt}предметная область для решения задач с интенсивным использованием данных&1&40--47\\
\Avtors{Будзко~В.\,И., Ядринцев~В.\,В., Соченков~И.\,В., Королёв~В.\,И., Беленков~В.\,Г.} Об одном подходе
 к формированию в условиях высокой неопределенности марке-\linebreak
\\[-12pt]
\hspace*{23pt}ров конфиденциальности в системах интенсивного использования данных&4&69--76\\
\Avtors{Вайсер~К.\,О.} см.\ Потанин~М.\,С.&&\\
\Avtors{Вохминцев~А.\,В., Мельников~А.\,В., Пачганов~C.\,А.} Метод навигации и составления карты в трехмерном пространстве на основе комбинированного решения вариационной\linebreak
\\[-12pt]
\hspace*{23pt}подзадачи точка--точка ICP для аффинных преобразований&1&101--112\\
\Avtors{Гаврилов~Е.\,С.} см.\ Абгарян~К.\,К.&&\\
\Avtors{Гайдамака~Ю.\,В.} см.\  Москалева~Ф.\,А.&&\\
\Avtors{Голембиовский~Д.\,Ю.} см.\ Данилишин~А.\,Р.&&\\
\Avtors{Голембиовский~Д.\,Ю.} см.\ Данилишин~А.\,Р.&&\\
\Avtors{Гончаров~А.\,А., Зацман~И.\,М., Кружков~М.\,Г.} Эволюция классификаций в надкорпусных\linebreak
\\[-12pt]
\hspace*{23pt}базах данных&4&108--116\\
\Avtors{Гончаров~А.\,В., Стрижов~В.\,В.} Выравнивание декартовых произведений упорядоченных\linebreak
\\[-12pt]
\hspace*{23pt}множеств&1&31--39\\
\end{tabular}
}

\pagebreak

\def\leftkol{АВТОРСКИЙ УКАЗАТЕЛЬ ЗА 2020 г.} % ENGLISH ABSTRACTS}

\def\rightkol{АВТОРСКИЙ УКАЗАТЕЛЬ ЗА 2020 г.} %ENGLISH ABSTRACTS}

%\thispagestyle{myheadings}
\def\leftfootline{\small{\textbf{\thepage}
\hfill ИНФОРМАТИКА И ЕЁ ПРИМЕНЕНИЯ\ \ \ том~14\ \ \ выпуск~4\ \ \ 2020}
}%
 \def\rightfootline{\small{ИНФОРМАТИКА И ЕЁ ПРИМЕНЕНИЯ\ \ \ том~14\ \ \ выпуск~4\ \ \ 2020
 \hfill \textbf{\thepage}}}


\noindent
{\tabcolsep=3pt
\begin{tabular}{p{394pt}cc}
&\textbf{Вып.} & \textbf{Стр.}\\[3pt]
\Avtors{Горшенин~А.\,К., Королев~В.\,Ю.} Аппроксимация распределений размеров частиц лунного\linebreak
\\[-12pt]
\hspace*{23pt}реголита на основе метода статистической симуляции выборок&2&50--57\\
\Avtors{Горшенин~А.\,К., Королев~В.\,Ю., Щербинина~А.\,А.} Статистическое оценивание распределений случайных коэффициентов стохастического дифференциального уравнения\linebreak
\\[-12pt]
\hspace*{23pt}Ланжевена&3&\hphantom{1}3--12\\
\Avtors{Горшенин~А.\,К., Кузьмин~В.\,Ю.} Анализ конфигураций LSTM-сетей для построения\linebreak
\\[-12pt]
\hspace*{23pt}среднесрочных векторных прогнозов&1&10--16\\
\Avtors{Грабовой~А.\,В., Бахтеев~О.\,Ю., Стрижов~В.\,В.} Введение отношения порядка на множестве\linebreak
\\[-12pt]
\hspace*{23pt}параметров аппроксимирующих моделей&2&58--65\\
\Avtors{Грушо~А.\,А., Забежайло~М.\,И., Смирнов~Д.\,В., Тимонина~Е.\,Е.} О вероятностных оценках\linebreak
\\[-12pt]
\hspace*{23pt}достоверности эмпирических выводов&4&3--8\\
\Avtors{Грушо~А.\,А., Забежайло~М.\,И., Смирнов~Д.\,В., Тимонина~Е.\,Е., Шоргин~С.\,Я.} Методы\linebreak
\\[-12pt]
\hspace*{23pt}математической статистики в задаче поиска инсайдера&3&71--75\\
\Avtors{Грушо~А.\,А., Забежайло~М.\,И., Тимонина~Е.\,Е.} О каузальной репрезентативности обуча-\linebreak
\\[-12pt]
\hspace*{23pt}ющих выборок прецедентов в задачах диагностического типа&1&80--86\\
\Avtors{Грушо~А.\,А., Тимонина~Е.\,Е., Грушо~Н.\,А., Терехина~И.\,Ю.} Выявление аномалий с по-\linebreak
\\[-12pt]
\hspace*{23pt}мощью метаданных&3&76--80\\
\Avtors{Грушо~А.\,А.} см.\ Грушо~Н.\,А.&&\\
\Avtors{Грушо~Н.\,А., Грушо~А.\,А., Забежайло~М.\,И., Тимонина~Е.\,Е.} Методы нахождения причин\linebreak
\\[-12pt]
\hspace*{23pt}сбоев в информационных технологиях  с помощью метаданных&2&33--39\\
\Avtors{Грушо~Н.\,А.} см.\ Грушо~А.\,А.&&\\
\Avtors{Данилишин~А.\,Р., Голембиовский~Д.\,Ю.} Оценка стоимости опционов на основе моделей\linebreak
\\[-12pt]
\hspace*{23pt}ARIMA--GARCH с ошибками, распределенными по закону $S_u$ Джонсона&4&83--90\\
\Avtors{Данилишин~А.\,Р., Голембиовский~Д.\,Ю.} Риск-нейтральная динамика для модели ARIMA-\linebreak
\\[-12pt]
\hspace*{23pt}GARCH с ошибками, распределенными по закону $S_U$ Джонсона&1&48--55\\
\Avtors{Диментов~А.\,В.} см.\ Краснов~Ф.\,В.&&\\
\Avtors{Донской~В.\,И.} Извлечение оптимизационных моделей из данных&3&109--118\\
\Avtors{Дубнов~Ю.\,А.} см.\ Попков~Ю.\,С.&&\\
\Avtors{Дулин~С.\,К., Дулина~Н.\,Г., Ермаков~П.\,В.} Информационный синтез документов&1&128--135\\
\Avtors{Дулина~Н.\,Г.} см.\ Дулин~С.\,К.&&\\
\Avtors{Дьяченко~Ю.\,Г.} см.\ Соколов~И.\,А.&&\\
\Avtors{Ермаков~П.\,В.} см.\ Дулин~С.\,К.&&\\
\Avtors{Ефросинин~Д.\,В.} см.\ Харин~П.\,А.&&\\
\Avtors{Жолобов~В.\,А.} см.\ Потанин~М.\,С.&&\\
\Avtors{Забежайло~М.\,И.} см.\ Грушо~А.\,А.&&\\
\Avtors{Забежайло~М.\,И.} см.\ Грушо~А.\,А.&&\\
\Avtors{Забежайло~М.\,И.} см.\ Грушо~А.\,А.&&\\
\Avtors{Забежайло~М.\,И.} см.\ Грушо~Н.\,А.&&\\
\Avtors{Захаров В. Н.} см.\ Френкель С. Л.&&\\
\Avtors{Зацман~И.\,М.} Проблемно-ориентированная верификация полноты темпоральных\linebreak
\\[-12pt]
\hspace*{23pt}онтологий и заполнение понятийных лакун&3&119--128\\
\Avtors{Зацман~И.\,М.} см.\ Гончаров~А.\,А.&&\\
\Avtors{Зацман~И.\,М.} см.\ Нуриев~В.\,А.&&\\
\Avtors{Зейфман~А.\,И.} см.\ Сатин~Я.\,А.&&\\
\Avtors{Кириков~И.\,А.} см.\ Румовская~С.\,Б.&&\\
\Avtors{Кирилюк~И.\,Л., Сенько~О.\,В.} Выбор моделей оптимальной сложности методами Монте-Карло (на примере моделей производственных функций регионов Российской\linebreak
\\[-12pt]
\hspace*{23pt}Федерации)&2&111--118\\
\Avtors{Ковалёв~Д.\,Ю.} см.\ Брюхов~Д.\,О.&&\\
\Avtors{Козеренко~Е.\,Б., Михеев~М.\,Ю., Сомин~Н.\,В., Эрлих~Л.\,И., Кузнецов~К.\,И.} Аналити\-че\-ская
текс\-тология в системах интеллектуальной обработки неструктурированных\linebreak
\\[-12pt]
\hspace*{23pt}данных&1&113--120\\
\Avtors{Колбин~И.\,С.} см.\ Абгарян~К.\,К.&&\\
\end{tabular}
}

\pagebreak

\def\leftkol{АВТОРСКИЙ УКАЗАТЕЛЬ ЗА 2020 г.} % ENGLISH ABSTRACTS}

\def\rightkol{АВТОРСКИЙ УКАЗАТЕЛЬ ЗА 2020 г.} %ENGLISH ABSTRACTS}

%\thispagestyle{myheadings}
\def\leftfootline{\small{\textbf{\thepage}
\hfill ИНФОРМАТИКА И ЕЁ ПРИМЕНЕНИЯ\ \ \ том~14\ \ \ выпуск~4\ \ \ 2020}
}%
 \def\rightfootline{\small{ИНФОРМАТИКА И ЕЁ ПРИМЕНЕНИЯ\ \ \ том~14\ \ \ выпуск~4\ \ \ 2020
 \hfill \textbf{\thepage}}}


\noindent
{\tabcolsep=3pt
\begin{tabular}{p{394pt}cc}
&\textbf{Вып.} & \textbf{Стр.}\\[3pt]
\Avtors{Королев~В.\,Ю.} О распределении отношения суммы элементов выборки, превосходящих\linebreak
\\[-12pt]
\hspace*{23pt}некоторый порог, к сумме всех элементов выборки.~I&3&35--43\\
\Avtors{Королев~В.\,Ю.} О распределении отношения суммы элементов выборки, превосходящих\linebreak
\\[-12pt]
\hspace*{23pt}некоторый порог, к сумме всех элементов выборки.~II&4&33--36\\
\Avtors{Королев~В.\,Ю.} см.\ Горшенин~А.\,К&&\\
\Avtors{Королев~В.\,Ю.} см.\ Горшенин~А.\,К.&&\\
\Avtors{Королёв~В.\,И.} см.\ Будзко~В.\,И.&&\\
\Avtors{Костина~А.\,А., Мирин~А.\,Ю., Молдовян~Д.\,Н., Фахрутдинов~Р.\,Ш.} Метод задания конечных некоммутативных ассоциативных алгебр произвольной четной размерности\linebreak
\\[-12pt]
\hspace*{23pt}для построения постквантовых криптосхем&1&\hphantom{1}94--100\\
\Avtors{Кочеткова~И.\,А.} см.\ Харин~П.\,А.&&\\
\Avtors{Краснов~Ф.\,В., Диментов~А.\,В., Шварцман~М.\,Е.} Использование тематических моделей\linebreak
\\[-12pt]
\hspace*{23pt}для парного сравнения  коллекций научных статей&3&129--135\\
\Avtors{Кривенко~М.\,П.} Последовательный анализ серий данных на основе многомерных ре-\linebreak
\\[-12pt]
\hspace*{23pt}фе\-рен\-с\-ных регионов&2&86--91\\
\Avtors{Кружков~М.\,Г.} см.\ Гончаров~А.\,А.&&\\
\Avtors{Кудрявцев~А.\,А., Шестаков~О.\,В.} Метод логарифмических моментов для оценивания\linebreak
\\[-12pt]
\hspace*{23pt}параметров гамма-экспоненциального распределения&3&49--54\\
\Avtors{Кузнецов~К.\,И.} см.\ Козеренко~Е.\,Б.&&\\
\Avtors{Кузьмин~В.\,Ю.} см.\ Горшенин~А.\,К.&&\\
\Avtors{Кушниренко~А.\,Г.} см.\ Бетелин~В.\,Б.&&\\
\Avtors{Кушниренко~А.\,Г.} см.\ Бетелин~В.\,Б.&&\\
\Avtors{Леонов~А.\,Г.} см.\ Бетелин~В.\,Б.&&\\
\Avtors{Макеева~Е.\,Д.} см.\ Харин~П.\,А.&&\\
\Avtors{Малашенко~Ю.\,Е., Назарова~И.\,А.} Аппроксимация множества достижимых потоков\linebreak
\\[-12pt]
\hspace*{23pt}многопользовательской сети&3&81--85\\
\Avtors{Мартюшова~Я.\,Г.} см.\ Босов~А.\,В.&&\\
\Avtors{Матюшенко~С.\,И., Разумчик~Р.\,В.} Стационарные характеристики системы Geo$/G/1/\infty $\linebreak
\\[-12pt]
\hspace*{23pt}с неординарным входящим потоком, управляющим размером очереди&4&25--32\\
\Avtors{Мейханаджян~Л.\,А., Разумчик~Р.\,В.} Стационарные характеристики системы $M/G/2/\infty$ с одним частным случаем дисциплины инверсионного порядка обслуживания\linebreak
\\[-12pt]
\hspace*{23pt}с обобщенным  вероятностным приоритетом&2&66--71\\
\Avtors{Мельников~А.\,В.} см.\ Вохминцев~А.\,В.&&\\
\Avtors{Мельников~С.\,Ю., Самуйлов~К.\,Е.} Статистические свойства двоичных неавтономных\linebreak
\\[-12pt]
\hspace*{23pt}регистров сдвига  с внутренним суммированием&2&80--85\\
\Avtors{Милованова~Т.\,А., Разумчик~Р.\,В.} Однолинейная система массового обслуживания с инверсионным порядком обслуживания с вероятностным приоритетом, групповым\linebreak
\\[-12pt]
\hspace*{23pt}пуассоновским потоком и фоновыми заявками&3&26--34\\
\Avtors{Мирин~А.\,Ю.} см.\ Костина~А.\,А.&&\\
\Avtors{Михеев~М.\,Ю.} см.\ Козеренко~Е.\,Б.&&\\
\Avtors{Молдовян~Д.\,Н.} см.\ Костина~А.\,А.&&\\
\Avtors{Москалева~Ф.\,А., Гайдамака~Ю.\,В., Шоргин~В.\,С.} Влияние параметров изоляции на\linebreak
\\[-12pt]
\hspace*{23pt}разделение ресурсов при нарезке сети&4&\hphantom{1}9--16\\
\Avtors{Назарова~И.\,А.} см.\ Малашенко~Ю.\,Е.&&\\
\Avtors{Наумов~А.\,В.} см.\ Босов~А.\,В.&&\\
\Avtors{Наумов~В.\,А., Самуйлов~К.\,Е.} О марковских и рациональных потоках случайных со-\linebreak
\\[-12pt]
\hspace*{23pt}бытий.~I&3&13--19\\
\Avtors{Наумов~В.\,А., Самуйлов~К.\,Е.} О марковских и рациональных потоках случайных со-\linebreak
\\[-12pt]
\hspace*{23pt}бытий.~II&4&37--46\\
\Avtors{Новиков~Д.\,А.} см.\ Шнурков~П.\,В.&&\\
\Avtors{Нуриев~В.\,А., Зацман~И.\,М.} Редуцирование спектра моделей перевода в надкорпусных\linebreak
\\[-12pt]
\hspace*{23pt}базах данных&2&119--126\\
\Avtors{Пачганов~C.\,А.} см.\ Вохминцев~А.\,В.&&\\
\end{tabular}
}

\pagebreak

\def\leftkol{АВТОРСКИЙ УКАЗАТЕЛЬ ЗА 2020 г.} % ENGLISH ABSTRACTS}

\def\rightkol{АВТОРСКИЙ УКАЗАТЕЛЬ ЗА 2020 г.} %ENGLISH ABSTRACTS}

%\thispagestyle{myheadings}
\def\leftfootline{\small{\textbf{\thepage}
\hfill ИНФОРМАТИКА И ЕЁ ПРИМЕНЕНИЯ\ \ \ том~14\ \ \ выпуск~4\ \ \ 2020}
}%
 \def\rightfootline{\small{ИНФОРМАТИКА И ЕЁ ПРИМЕНЕНИЯ\ \ \ том~14\ \ \ выпуск~4\ \ \ 2020
 \hfill \textbf{\thepage}}}


\noindent
{\tabcolsep=3pt
\begin{tabular}{p{394pt}cc}
&\textbf{Вып.} & \textbf{Стр.}\\[3pt]
\Avtors{Попков~А.\,Ю.} см.\ Попков~Ю.\,С.&&\\
\Avtors{Попков~Ю.\,С., Попков~А.\,Ю., Дубнов~Ю.\,А.} Методы детерминированных и рандомизи-\linebreak
\\[-12pt]
\hspace*{23pt}рованных энтропийных проекций для редукции размерности матрицы данных&4&47--54\\
\Avtors{Попов~Г.\,А., Симаворян~С.\,Ж., Симонян~А.\,Р., Улитина~Е.\,И.} Моделирование процесса мониторинга систем информационной безопасности на основе систем массового\linebreak
\\[-12pt]
\hspace*{23pt}обслуживания&1&71--79\\
\Avtors{Попов~М.\,В., Посыпкин~М.\,А.} Аппроксимация множества решений систем нелинейных\linebreak
\\[-12pt]
\hspace*{23pt}неравенств с использованием графических ускорителей&3&20--25\\
\Avtors{Посыпкин~М.\,А.} см.\ Попов~М.\,В.&&\\
\Avtors{Потанин~М.\,С., Вайсер~К.\,О., Жолобов~В.\,А., Стрижов~В.\,В.} Оптимизация структуры\linebreak
\\[-12pt]
\hspace*{23pt}сетей глубокого обучения&4&55--62\\
\Avtors{Разумчик~Р.\,В.} см.\ Матюшенко~С.\,И.&&\\
\Avtors{Разумчик~Р.\,В.} см.\ Мейханаджян~Л.\,А.&&\\
\Avtors{Разумчик~Р.\,В.} см.\ Милованова~Т.\,А.&&\\
\Avtors{Рождественский~Ю.\,В.} см.\ Соколов~И.\,А.&&\\
\Avtors{Румовская~С.\,Б., Кириков~И.\,А.} Метод визуального представления конфликтов в гибрид-\linebreak
\\[-12pt]
\hspace*{23pt}ных интеллектуальных многоагентных системах&4&77--82\\
\Avtors{Самуйлов~К.\,Е.} см.\ Агеев~К.\,А.&&\\
\Avtors{Самуйлов~К.\,Е.} см.\ Мельников~С.\,Ю.&&\\
\Avtors{Самуйлов~К.\,Е.} см.\ Наумов~В.\,А.&&\\
\Avtors{Самуйлов~К.\,Е.} см.\ Наумов~В.\,А.&&\\
\Avtors{Сапунова~А.\,П.} см.\ Босов~А.\,В.&&\\
\Avtors{Сатин~Я.\,А., Зейфман~А.\,И., Шилова~Г.\,Н.} О подходах к построению предельных режимов\linebreak
\\[-12pt]
\hspace*{23pt}для некоторых моделей массового обслуживания&2&3--9\\
\Avtors{Севастьянов~Л.\,А., Щетинин~Е.\,Ю.} О методах повышения точности многоклассовой\linebreak
\\[-12pt]
\hspace*{23pt}классификации на несбалансированных данных&1&63--70\\
\Avtors{Семенов~А.\,Л.} см.\ Бетелин~В.\,Б.&&\\
\Avtors{Сенько~О.\,В.} см.\ Кирилюк~И.\,Л.&&\\
\Avtors{Серебрянский~С.\,М., Тырсин~А.\,Н.} Повышение точности решения обратных задач за\linebreak
\\[-12pt]
\hspace*{23pt}счет уточнения граничных условий&1&56--62\\
\Avtors{Симаворян~С.\,Ж.} см.\ Попов~Г.\,А.&&\\
\Avtors{Симонян~А.\,Р.} см.\ Попов~Г.\,А.&&\\
\Avtors{Смирнов~Д.\,В.} см.\ Грушо~А.\,А.&&\\
\Avtors{Смирнов~Д.\,В.} см.\ Грушо~А.\,А.&&\\
\Avtors{Соколов~И.\,А., Степченков~Ю.\,А., Дьяченко~Ю.\,Г., Рождественский~Ю.\,В.} Повышение\linebreak
\\[-12pt]
\hspace*{23pt}сбоеустойчивости самосинхронных схем&4&63--68\\
\Avtors{Сомин~Н.\,В.} см.\ Козеренко~Е.\,Б.&&\\
\Avtors{Сопин~Э.\,С.} см.\ Агеев~К.\,А.&&\\
\Avtors{Сопрунов~С.\,Ф.} см.\ Бетелин~В.\,Б.&&\\
\Avtors{Соченков~И.\,В.} см.\ Будзко~В.\,И.&&\\
\Avtors{Степченков~Ю.\,А.} см.\ Соколов~И.\,А.&&\\
\Avtors{Стефанович~А.\,И.} см.\ Босов~А.\,В.&&\\
\Avtors{Стрижов~В.\,В.} см.\ Гончаров~А.\,В.&&\\
\Avtors{Стрижов~В.\,В.} см.\ Грабовой~А.\,В.&&\\
\Avtors{Стрижов~В.\,В.} см.\ Потанин~М.\,С.&&\\
\Avtors{Ступников~С.\,А.} см.\ Брюхов~Д.\,О.&&\\
\Avtors{Терехина~И.\,Ю.} см.\ Грушо~А.\,А.&&\\
\Avtors{Тимонина~Е.\,Е.} см.\  Грушо~А.\,А.&&\\
\Avtors{Тимонина~Е.\,Е.} см.\ Грушо~А.\,А.&&\\
\Avtors{Тимонина~Е.\,Е.} см.\ Грушо~А.\,А.&&\\
\Avtors{Тимонина~Е.\,Е.} см.\ Грушо~А.\,А.&&\\
\Avtors{Тимонина~Е.\,Е.} см.\ Грушо~Н.\,А.&&\\
\Avtors{Тырсин~А.\,Н.} см.\ Серебрянский~С.\,М.&&\\
\Avtors{Улитина~Е.\,И.} см.\ Попов~Г.\,А.&&\\
\end{tabular}
}

\pagebreak

\def\leftkol{АВТОРСКИЙ УКАЗАТЕЛЬ ЗА 2020 г.} % ENGLISH ABSTRACTS}

\def\rightkol{АВТОРСКИЙ УКАЗАТЕЛЬ ЗА 2020 г.} %ENGLISH ABSTRACTS}

%\thispagestyle{myheadings}
\def\leftfootline{\small{\textbf{\thepage}
\hfill ИНФОРМАТИКА И ЕЁ ПРИМЕНЕНИЯ\ \ \ том~14\ \ \ выпуск~4\ \ \ 2020}
}%
 \def\rightfootline{\small{ИНФОРМАТИКА И ЕЁ ПРИМЕНЕНИЯ\ \ \ том~14\ \ \ выпуск~4\ \ \ 2020
 \hfill \textbf{\thepage}}}


\noindent
{\tabcolsep=3pt
\begin{tabular}{p{394pt}cc}
&\textbf{Вып.} & \textbf{Стр.}\\[3pt]
\Avtors{Фахрутдинов~Р.\,Ш.} см.\ Костина~А.\,А.&&\\
\Avtors{Френкель С. Л., Захаров В. Н.} Совместная оценка предсказуемости данных и качества\linebreak
\\[-12pt]
\hspace*{23pt}предикторов&2&40--49\\
\Avtors{Харин~П.\,А., Макеева~Е.\,Д., Кочеткова~И.\,А., Ефросинин~Д.\,В., Шоргин~С.\,Я.} 
Система массового обслуживания с орбитами для анализа совместного обслуживания трафика 
с малыми задержками URLLC и~широкополосного доступа eMBB в~беспроводных\linebreak
\\[-12pt]
\hspace*{23pt}сетях пятого поколения&4&17--24\\
\Avtors{Хусаинов~А.\,А.} Производительность ограниченного конвейера&1&87--93\\
\Avtors{Шанин~И.\,А.} см.\ Брюхов~Д.\,О.&&\\
\Avtors{Шварцман~М.\,Е.} см.\ Краснов~Ф.\,В.&&\\
\Avtors{Шестаков~О.\,В.} Асимптотика оценки среднеквадратичного риска в задаче обращения\linebreak
\\[-12pt]
\hspace*{23pt}преобразования Радона по проекциям, регистрируемым на случайной сетке&2&26--32\\
\Avtors{Шестаков~О.\,В.} Асимптотическая регулярность вейвлет-методов обращения линейных однородных операторов по наблюдениям, регистрируемым в случайные моменты\linebreak
\\[-12pt]
\hspace*{23pt}времени&1&3--9\\
\Avtors{Шестаков~О.\,В.} О статистических свойствах оценки риска в задаче обращения преобра-\linebreak
\\[-12pt]
\hspace*{23pt}зования Радона при случайном объеме проекционных данных&3&44--48\\
\Avtors{Шестаков~О.\,В.} см.\ Кудрявцев~А.\,А.&&\\
\Avtors{Шилова~Г.\,Н.} см.\ Сатин~Я.\,А.&&\\
\Avtors{Шихиев~Ф.\,Ш.} см.\ Шихиев~Ш.\,Б.&&\\
\Avtors{Шихиев~Ш.\,Б., Шихиев~Ф.\,Ш.} Инкапсуляция семантических представлений в элементы\linebreak
\\[-12pt]
\hspace*{23pt}грамматики&1&121--127\\
\Avtors{Шнурков~П.\,В., Адамова~К.\,А.} Решение задачи безусловного экстремума для дробно-\linebreak
\\[-12pt]
\hspace*{23pt}линейного интегрального функционала, зависящего от параметра&2&\hphantom{1}98--103\\
\Avtors{Шнурков~П.\,В., Новиков~Д.\,А.} О концепции стохастической модели с управлением в~моменты выхода процесса на границу заданного подмножества множества\linebreak
\\[-12pt]
\hspace*{23pt}состояний&3&101--108\\
\Avtors{Шоргин~В.\,С.} см.\ Москалева~Ф.\,А.&&\\
\Avtors{Шоргин~С.\,Я.} см.\ Агеев~К.\,А.&&\\
\Avtors{Шоргин~С.\,Я.} см.\ Грушо~А.\,А.&&\\
\Avtors{Шоргин~С.\,Я.} см.\ Харин~П.\,А.&&\\
\Avtors{Щербинина~А.\,А.} см.\ Горшенин~А.\,К.&&\\
\Avtors{Щетинин~Е.\,Ю.} см.\ Севастьянов~Л.\,А.&&\\
\Avtors{Эрлих~Л.\,И.} см.\ Козеренко~Е.\,Б.&&\\
\Avtors{Ядринцев~В.\,В.} см.\ Будзко~В.\,И.&&\\
\Avtors{Яркина~Н.\,В.} см.\ Агеев~К.\,А.&&\\
\end{tabular}
}

%\thispagestyle{myheadings}
\def\leftfootline{\small{\textbf{\thepage}
\hfill ИНФОРМАТИКА И ЕЁ ПРИМЕНЕНИЯ\ \ \ том~14\ \ \ выпуск~4\ \ \ 2020}
}%
 \def\rightfootline{\small{ИНФОРМАТИКА И ЕЁ ПРИМЕНЕНИЯ\ \ \ том~14\ \ \ выпуск~4\ \ \ 2020
 \hfill \textbf{\thepage}}}

 \label{end\stat}

\newpage

\def\stat{cont-e}
{%\hrule\par
%\vskip 7pt % 7pt
\raggedleft\Large \bf%\baselineskip=3.2ex
2\,0\,2\,0\ \ A\,U\,T\,H\,O\,R\ \ I\,N\,D\,E\,X \vskip 17pt
 \hrule
 \par
\vskip 21pt plus 6pt minus 3pt }

\label{st\stat}

\def\tit{\ }

\def\aut{\ }
\def\auf{\ }

\def\leftkol{\ } %2020 AUTHOR INDEX} % ENGLISH ABSTRACTS}

\def\rightkol{\ } %2020 AUTHOR INDEX} %ENGLISH ABSTRACTS}

\titele{\tit}{\aut}{\auf}{\leftkol}{\rightkol}
\addcontentsline{toc}{subsection}{\textrm\textbf 2020 Author Index}

\def\leftfootline{\small{\textbf{\thepage}
\hfill INFORMATIKA I EE PRIMENENIYA~--- INFORMATICS AND APPLICATIONS\ \ \ 2020\
\ \ volume~14\ \ \ issue\ 4}
}%
 \def\rightfootline{\small{INFORMATIKA I EE PRIMENENIYA~--- INFORMATICS AND APPLICATIONS\ \ \ 2020\ \ \ volume~14\ \ \ issue\ 4
\hfill \textbf{\thepage}}}

\vspace*{-24pt}

\noindent
{\tabcolsep=3pt
\begin{tabular}{p{395.89pt}cc}
&\textbf{Issue} & \textbf{Page}\\[6pt]
\Avtors{Abgaryan~K.\,K. and Gavrilov~E.\,S.} Integration platform for multiscale modeling of neuromorphic\linebreak
\\[-12pt]
\hspace*{23pt}systems&2&104--110\\
\Avtors{Abgaryan~K.\,K. and Kolbin~I.\,S.} Application of multiscale approach and data sciences for\linebreak
\\[-12pt]
\hspace*{23pt}modeling thermal conductivity in layered structures&4&91--99\\
\Avtors{Adamova~K.\,A.} see Shnurkov~~P.\,V.&&\\
\Avtors{Agalarov~Ya.\,M.} Optimization of the capacity of the main storage in $G/M/1/K$ queueing system\linebreak
\\[-12pt]
\hspace*{23pt}with an additional storage device&2&72--79\\
\Avtors{Agasandyan~G.\,A.} Computational aspects of optimization on CC-VaR in a complex of markets&3&62--70\\
\Avtors{Ageev~K.\,A., Sopin~E.\,S., Yarkina~N.\,V., Samouylov~K.\,E., and Shorgin~S.\,Ya.} Analysis of the\linebreak
\\[-12pt]
\hspace*{23pt}network slicing mechanisms with guaranteed allocated resources for various traffic types&3&\hphantom{1}94--100\\
\Avtors{Bakhteev~O.\,Yu.} see Grabovoy~A.\,V.&&\\
\Avtors{Bazilevskiy~M.\,P.} Multifactor fully connected linear regression models without constraints to the\linebreak
\\[-12pt]
\hspace*{23pt}ratios of variables errors variances&2&92--97\\
\Avtors{Belenkov~V.\,G.} see Budzko~V.\,I.&&\\
\Avtors{Betelin~V.\,B., Kushnirenko~A.\,G., and Leonov~A.\,G.} Basic concepts of programming expounded\linebreak
\\[-12pt]
\hspace*{23pt}for preschoolers&3&55--61\\
\Avtors{Betelin~V.\,B., Kushnirenko~A.\,G., Semenov~A.\,L., and Soprunov~S.\,F.} About digital literacy and\linebreak
\\[-12pt]
\hspace*{23pt}environments for its development&4&100--107\\
\Avtors{Borisov~A.\,V.} Numerical schemes of Markov jump process filtering given discretized observa-\linebreak
\\[-12pt]
\hspace*{23pt}tions~II: Additive noise case&1&17--23\\
\Avtors{Borisov~A.\,V.} Numerical schemes of Markov jump process filtering given discretized observa-\linebreak
\\[-12pt]
\hspace*{23pt}tions III: Multiplicative noises case&2&10--18\\
\Avtors{Bosov~A.\,V.} Stochastic differential system output control by the quadratic criterion. V. Case of\linebreak
\\[-12pt]
\hspace*{23pt}incomplete state information&2&19--28\\
\Avtors{Bosov~A.\,V., Martyushova~Ya.\,G., Naumov~A.\,V., and Sapunova~A.\,P.} Bayesian approach to the\linebreak
\\[-12pt]
\hspace*{23pt}construction of an individual user trajectory in the system of distance learning&3&86--93\\
\Avtors{Bosov~A.\,V. and Stefanovich~A.\,I.} Stochastic differential system output control by the quadratic\linebreak
\\[-12pt]
\hspace*{23pt}criterion. IV. Alternative numerical decision&1&24--30\\
\Avtors{Briukhov~D.\,O., Stupnikov~S.\,A., Kovalev~D.\,Yu., and Shanin~I.\,A.} Neurophysiology as a subject\linebreak
\\[-12pt]
\hspace*{23pt}domain for~data intensive problem solving&1&40--47\\
\Avtors{Budzko~V.\,I., Yadrintsev~V.\,V., Sochenkov~I.\,V., Korolev~V.\,I., and Belenkov~V.\,G.} Extraction of confidentiality markers from texts under conditions of high uncertainty in systems with\linebreak
\\[-12pt]
\hspace*{23pt}data intensive usage&4&69--76\\
\Avtors{Danilishin~A.\,R. and Golembiovsky~D.\,Yu.} Estimating the fair value of options based on\linebreak
\\[-12pt]
\hspace*{23pt}ARIMA--GARCH models with errors distributed according to the Johnson's $S_u$ law&4&83--90\\
\Avtors{Danilishin~A.\,R. and Golembiovsky~D.\,Yu.} Risk-neutral dynamics for the ARIMA-GARCH\linebreak
\\[-12pt]
\hspace*{23pt}random process with errors distributed according to the Johnson's $S_u$ law&1&48--55\\
\Avtors{Diachenko~Yu.\,G.} see Sokolov~I.\,A.&&\\
\Avtors{Dimentov~A.\,V.} see Krasnov~F.\,V.&&\\
\Avtors{Donskoy~V.\,I.} Optimization models extraction from data&3&109--118\\
\Avtors{Dubnov~Y.\,A.} see Popkov~Y.\,S.&&\\
\Avtors{Dulin~S.\,K., Dulina~N.\,G., and Ermakov~P.\,V.} Information fusion of documents&1&128--135\\
\Avtors{Dulina~N.\,G.} see Dulin~S.\,K.&&\\
\Avtors{Efrosinin~D.\,V.} see Kharin~P.\,A.&&\\
\Avtors{Ehrlich~L.\,I.} see Kozerenko~E.\,B.&&\\
\Avtors{Ermakov~P.\,V.} see Dulin~S.\,K.&&\\
\end{tabular}
}
\pagebreak

\def\leftfootline{\small{\textbf{\thepage}
\hfill INFORMATIKA I EE PRIMENENIYA~--- INFORMATICS AND APPLICATIONS\ \ \ 2020\
\ \ volume~14\ \ \ issue\ 4}
}%
 \def\rightfootline{\small{INFORMATIKA I EE PRIMENENIYA~---
INFORMATICS AND APPLICATIONS\ \ \ 2020\ \ \ volume~14\ \ \ issue\ 4
\hfill \textbf{\thepage}}}

\def\leftkol{2020 AUTHOR INDEX} % ENGLISH ABSTRACTS}

\def\rightkol{2020 AUTHOR INDEX} %ENGLISH ABSTRACTS}


\noindent
{\tabcolsep=3pt
\begin{tabular}{p{395.48108pt}cc}
&\textbf{Issue} & \textbf{Page}\\[6pt]
\Avtors{Fahrutdinov~R.\,Sh.} see Kostina~A.\,A.&&\\
\Avtors{Frenkel~S.\,L. and Zakharov~V.\,N.} Joint assessment of data predictability and quality pre-\linebreak
\\[-12pt]
\hspace*{23pt}dictors&2&40--49\\
\Avtors{Gaidamaka~Yu.\,V.} see Moskaleva~F.\,A.&&\\
\Avtors{Gavrilov~E.\,S.} see Abgaryan~K.\,K.&&\\
\Avtors{Golembiovsky~D.\,Yu.} see Danilishin~A.\,R.&&\\
\Avtors{Golembiovsky~D.\,Yu.} see Danilishin~A.\,R.&&\\
\Avtors{Goncharov~A.\,V. and Strijov~V.\,V.} Alignment of ordered set Cartesian product&1&31--39\\
\Avtors{Goncharov~A.\,A., Zatsman~I.\,M., and Kruzhkov~M.\,G.} Evolution of classifications in supracorpora\linebreak
\\[-12pt]
\hspace*{23pt}databases&4&108--116\\
\Avtors{Gorshenin~A.\,K. and Korolev~V.\,Yu.} Approximation of particle size distributions of lunar regolith\linebreak
\\[-12pt]
\hspace*{23pt}based on the resampling&2&50--57\\
\Avtors{Gorshenin~A.\,K., Korolev~V.\,Yu., and Shcherbinina~A.\,A.} Statistical estimation of distributions\linebreak
\\[-12pt]
\hspace*{23pt}of random coefficients in the Langevin stochastic differential equation&3&\hphantom{1}3--12\\
\Avtors{Gorshenin~A.\,K. and Kuzmin~V.\,Yu.} Analysis of configurations of LSTM networks for medium-\linebreak
\\[-12pt]
\hspace*{23pt}term vector forecasting&1&10--16\\
\Avtors{Grabovoy~A.\,V., Bakhteev~O.\,Yu., and Strijov~V.\,V.} Ordering the set of neural network parameters&2&58--65\\
\Avtors{Grusho~A.\,A., Timonina~E.\,E., Grusho~N.\,A., and Teryokhina~I.\,Yu.} Identifying anomalies using\linebreak
\\[-12pt]
\hspace*{23pt}metadata&3&76--80\\
\Avtors{Grusho~A.\,A., Zabezhailo~M.\,I., Smirnov~D.\,V., and Timonina~E.\,E.} On probabilistic estimates of\linebreak
\\[-12pt]
\hspace*{23pt}the validity of empirical conclusions&4&3--8\\
\Avtors{Grusho~A.\,A., Zabezhailo~M.\,I., and Timonina~E.\,E.} On causal representativeness of training\linebreak
\\[-12pt]
\hspace*{23pt}samples of precedents in diagnostic type tasks&1&80--86\\
\Avtors{Grusho~A.\,A.} see Grusho~N.\,A.&&\\
\Avtors{Grusho~N.\,A., Grusho~A.\,A., Zabezhailo~M.\,I., and Timonina~E.\,E.} Methods of finding the causes\linebreak
\\[-12pt]
\hspace*{23pt}of information technology failures by means of metadata&2&33--39\\
\Avtors{Grusho~N.\,A., Zabezhailo~M.\,I., Smirnov~D.\,V., Timonina~E.\,E., and Shorgin~S.\,Ya.} Mathematical\linebreak
\\[-12pt]
\hspace*{23pt}statistics in the task of identifying hostile insiders&3&71--75\\
\Avtors{Grusho~N.\,A.} see Grusho~A.\,A.&&\\
\Avtors{Kharin~P.\,A., Makeeva~E.\,D., Kochetkova~I.\,A., Efrosinin~D.\,V., and Shorgin~S.\,Ya.} Retrial\linebreak
\\[-12pt]
\hspace*{23pt}queuing model for analyzing joint URLLC and eMBB transmission in 5G networks&4&17--24\\
\Avtors{Khusainov~A.\,A.} Performance of the bounded pipeline&1&87--93\\
\Avtors{Kirikov~I.\,A.} see Rumovskaya~S.\,B.&&\\
\Avtors{Kirilyuk~I.\,L. and Sen'ko~O.\,V.} Selection of optimal complexity models by methods of nonparametric statistics (on the example of production function model of regions of the Russian\linebreak
\\[-12pt]
\hspace*{23pt}Federation)&2&111--118\\
\Avtors{Kochetkova~I.\,A.} see Kharin~P.\,A.&&\\
\Avtors{Kolbin~I.\,S.} see Abgaryan~K.\,K.&&\\
\Avtors{Korolev~V.\,I.} see Budzko~V.\,I.&&\\
\Avtors{Korolev~V.\,Yu.} On the distribution of the ratio of the sum of sample elements exceeding\linebreak
\\[-12pt]
\hspace*{23pt}a threshold to the total sum of sample elements.~I&3&35--43\\
\Avtors{Korolev~V.\,Yu.} On the distribution of the ratio of the sum of sample elements exceeding\linebreak
\\[-12pt]
\hspace*{23pt}a threshold to the total sum of sample elements.~II&4&33--36\\
\Avtors{Korolev~V.\,Yu.} see Gorshenin~A.\,K.&&\\
\Avtors{Korolev~V.\,Yu.} see Gorshenin~A.\,K.&&\\
\Avtors{Kostina~A.\,A., Mirin~A.\,Yu., Moldovyan~D.\,N., and Fahrutdinov~R.\,Sh.} Method for defining finite noncommutative associative algebras of arbitrary even dimension for development of the\linebreak
\\[-12pt]
\hspace*{23pt}postquantum cryptoschemes&1&\hphantom{1}94--100\\
\Avtors{Kovalev~D.\,Yu.} see Briukhov~D.\,O.&&\\
\Avtors{Kozerenko~E.\,B., Mikheev~M.\,Y., Somin~N.\,V., Ehrlich~L.\,I., and Kuznetsov~K.\,I.} Analytical\linebreak
\\[-12pt]
\hspace*{23pt}textology in intelligent processing systems for unstructured data&1&113--120\\
\Avtors{Krasnov~F.\,V., Dimentov~A.\,V., and Shvartsman~M.\,E.} Using topic models for pairwise comparison\linebreak
\\[-12pt]
\hspace*{23pt}of collections of scientific papers&3&129--135\\
\end{tabular}
}
\pagebreak

\def\leftfootline{\small{\textbf{\thepage}
\hfill INFORMATIKA I EE PRIMENENIYA~--- INFORMATICS AND APPLICATIONS\ \ \ 2020\
\ \ volume~14\ \ \ issue\ 4}
}%
 \def\rightfootline{\small{INFORMATIKA I EE PRIMENENIYA~---
INFORMATICS AND APPLICATIONS\ \ \ 2020\ \ \ volume~14\ \ \ issue\ 4
\hfill \textbf{\thepage}}}

\def\leftkol{2020 AUTHOR INDEX} % ENGLISH ABSTRACTS}

\def\rightkol{2020 AUTHOR INDEX} %ENGLISH ABSTRACTS}


\noindent
{\tabcolsep=3pt
\begin{tabular}{p{395.48108pt}cc}
&\textbf{Issue} & \textbf{Page}\\[6pt]
\Avtors{Krivenko~M.\,P.} Sequential analysis of serial measurements based on multivariate reference\linebreak
\\[-12pt]
\hspace*{23pt}regions&2&86--91\\
\Avtors{Kruzhkov~M.\,G.} see Goncharov~A.\,A.&&\\
\Avtors{Kudryavtsev~A.\,A. and Shestakov~O.\,V.} Method of logarithmic moments for estimating the\linebreak
\\[-12pt]
\hspace*{23pt}gamma-exponential distribution parameters&3&49--54\\
\Avtors{Kushnirenko~A.\,G.} see Betelin~V.\,B.&&\\
\Avtors{Kushnirenko~A.\,G.} see Betelin~V.\,B.&&\\
\Avtors{Kuzmin~V.\,Yu.} see Gorshenin~A.\,K.&&\\
\Avtors{Kuznetsov~K.\,I.} see Kozerenko~E.\,B.&&\\
\Avtors{Leonov~A.\,G.} see Betelin~V.\,B.&&\\
\Avtors{Makeeva~E.\,D.} see Kharin~P.\,A.&&\\
\Avtors{Malashenko~Yu.\,E. and Nazarova~I.\,A.} Approximation of the multiuser network feasible\linebreak
\\[-12pt]
\hspace*{23pt}flows set&3&81--85\\
\Avtors{Martyushova~Ya.\,G.} see Bosov~A.\,V.&&\\
\Avtors{Matyushenko~S.\,I. and Razumchik~R.\,V.} Stationary characteristics of discrete-time Geo$/G/1/\infty$\linebreak
\\[-12pt]
\hspace*{23pt}queue with batch arrivals and one queue skipping policy&4&25--32\\
\Avtors{Melnikov~A.\,V.} see Vokhmintcev~A.\,V.&&\\
\Avtors{Melnikov~S.\,Yu. and Samouylov~K.\,E.} Statistical properties of binary nonautonomous shift\linebreak
\\[-12pt]
\hspace*{23pt}registers with internal xor&2&80--85\\
\Avtors{Meykhanadzhyan~L.\,A. and Razumchik~R.\,V.} Stationary characteristics of $M/G/2/\infty$ queue\linebreak
\\[-12pt]
\hspace*{23pt}with identical servers, LIFO service, and resampling policy&2&66--71\\
\Avtors{Mikheev~M.\,Y.} see Kozerenko~E.\,B.&&\\
\Avtors{Milovanova~T.\,A. and Razumchik~R.\,V.} A single-server queueing system with LIFO service,\linebreak
\\[-12pt]
\hspace*{23pt}probabilistic priority, batch Poisson arrivals, and background customers&3&26--34\\
\Avtors{Mirin~A.\,Yu.} see Kostina~A.\,A.&&\\
\Avtors{Moldovyan~D.\,N.} see Kostina~A.\,A.&&\\
\Avtors{Moskaleva~F.\,A., Gaidamaka~Yu.\,V., and Shorgin~V.\,S.} Impact of the isolation parameters on\linebreak
\\[-12pt]
\hspace*{23pt}resource allocation in the network slicing model&4&\hphantom{1}9--16\\
\Avtors{Naumov~A.\,V.} see Bosov~A.\,V.&&\\
\Avtors{Naumov~V.\,A. and Samouylov~К.\,Е.} On Markovian and rational arrival processes.~I&3&13--19\\
\Avtors{Naumov~V.\,A. and Samouylov~K.\,E.} On Markovian and rational arrival processes.~II&4&37--46\\
\Avtors{Nazarova~I.\,A.} see Malashenko~Yu.\,E.&&\\
\Avtors{Novikov~D.\,A.} see Shnurkov~P.\,V.&&\\
\Avtors{Nuriev~V.\,A. and Zatsman~I.\,M.} Reducing the spectrum of translation models in supracorpora\linebreak
\\[-12pt]
\hspace*{23pt}databases&2&119--126\\
\Avtors{Pachganov~S.\,A.} see Vokhmintcev~A.\,V.&&\\
\Avtors{Popkov~A.\,Y.} see Popkov~Y.\,S.&&\\
\Avtors{Popkov~Y.\,S., Popkov~A.\,Y., and Dubnov~Y.\,A.} Deterministic and randomized methods of entropy\linebreak
\\[-12pt]
\hspace*{23pt}projection for dimensionality reduction problems&4&47--54\\
\Avtors{Popov~G.\,A., Simavoryan~S.\,Zh., Simonyan~A.\,R., and Ulitina~E.\,I.} Modeling of monitoring of\linebreak
\\[-12pt]
\hspace*{23pt}information security process on the basis of queuing systems&1&71--79\\
\Avtors{Popov~M.\,V. and Posypkin~M.\,A.} Approximation of the set of solutions of systems of nonlinear\linebreak
\\[-12pt]
\hspace*{23pt}inequalities using graphic accelerators&3&20--25\\
\Avtors{Posypkin~M.\,A.} see Popov~M.\,V.&&\\
\Avtors{Potanin~M.\,S., Vayser~K.\,O., Zholobov~V.\,A., and Strijov~V.\,V.} Deep learning neural network\linebreak
\\[-12pt]
\hspace*{23pt}structure optimization&4&55--62\\
\Avtors{Razumchik~R.\,V.} see Matyushenko~S.\,I.&&\\
\Avtors{Razumchik~R.\,V.} see Meykhanadzhyan~L.\,A.&&\\
\Avtors{Razumchik~R.\,V.} see Milovanova~T.\,A.&&\\
\Avtors{Rogdestvenski~Yu.\,V.} see Sokolov~I.\,A.&&\\
\Avtors{Rumovskaya~S.\,B. and Kirikov~I.\,A.} Conflict visual representation method in hybrid intelligent\linebreak
\\[-12pt]
\hspace*{23pt}multiagent systems&4&77--82\\
\Avtors{Samouylov~K.\,E.} see Ageev~K.\,A.&&\\
\end{tabular}
}
\pagebreak

\def\leftfootline{\small{\textbf{\thepage}
\hfill INFORMATIKA I EE PRIMENENIYA~--- INFORMATICS AND APPLICATIONS\ \ \ 2020\
\ \ volume~14\ \ \ issue\ 4}
}%
 \def\rightfootline{\small{INFORMATIKA I EE PRIMENENIYA~---
INFORMATICS AND APPLICATIONS\ \ \ 2020\ \ \ volume~14\ \ \ issue\ 4
\hfill \textbf{\thepage}}}

\def\leftkol{2020 AUTHOR INDEX} % ENGLISH ABSTRACTS}

\def\rightkol{2020 AUTHOR INDEX} %ENGLISH ABSTRACTS}


\noindent
{\tabcolsep=3pt
\begin{tabular}{p{395.48108pt}cc}
&\textbf{Issue} & \textbf{Page}\\[6pt]
\Avtors{Samouylov~K.\,E.} see Melnikov~S.\,Yu.&&\\
\Avtors{Samouylov~K.\,E.} see Naumov~V.\,A.&&\\
\Avtors{Samouylov~K.\,Е.} see Naumov~V.\,A.&&\\
\Avtors{Sapunova~A.\,P.} see Bosov~A.\,V.&&\\
\Avtors{Satin~Ya.\,A., Zeifman~A.\,I., and Shilova~G.\,N.} On approaches to constructing limiting regimes\linebreak
\\[-12pt]
\hspace*{23pt}for some queuing models&2&3--9\\
\Avtors{Semenov~A.\,L.} see Betelin~V.\,B.&&\\
\Avtors{Sen'ko~O.\,V.} see Kirilyuk~I.\,L.&&\\
\Avtors{Serebryanskii~S.\,M. and Tyrsin~A.\,N.} Improvement of the accuracy of solution of tasks for the\linebreak
\\[-12pt]
\hspace*{23pt}account of the construction of boundary conditions&1&56--62\\
\Avtors{Sevastianov~L.\,A. and Shchetinin~E.\,Yu.} On methods for improving the accuracy of multiclass\linebreak
\\[-12pt]
\hspace*{23pt}classification on imbalanced data&1&63--70\\
\Avtors{Shanin~I.\,A.} see Briukhov~D.\,O.&&\\
\Avtors{Shcherbinina~A.\,A.} see Gorshenin~A.\,K.&&\\
\Avtors{Shchetinin~E.\,Yu.} see Sevastianov~L.\,A.&&\\
\Avtors{Shestakov~O.\,V.} Asymptotic regularity of the wavelet methods of inverting linear homogeneous\linebreak
\\[-12pt]
\hspace*{23pt}operators from observations recorded at random times&1&3--9\\
\Avtors{Shestakov~O.\,V.} Asymptotics of the mean-square risk estimate in the problem of inverting the\linebreak
\\[-12pt]
\hspace*{23pt}Radon transform from projections registered on a random grid&2&29--32\\
\Avtors{Shestakov~O.\,V.} On the statistical properties of risk estimate in the problem of inverting the\linebreak
\\[-12pt]
\hspace*{23pt}Radon transform with a random volume of projection data&3&44--48\\
\Avtors{Shestakov~O.\,V.} see Kudryavtsev~A.\,A.&&\\
\Avtors{Shihiev~F.\,Sh.} see Shihiev~Sh.\,B.&&\\
\Avtors{Shihiev~Sh.\,B. and Shihiev~F.\,Sh.} Incapsulation of semantic representations into elements of\linebreak
\\[-12pt]
\hspace*{23pt}a grammar&1&121--127\\
\Avtors{Shilova~G.\,N.} see Satin~Ya.\,A.&&\\
\Avtors{Shnurkov~~P.\,V. and Adamova~K.\,A.} Solution of the unconditional extremal problem for a~linear-\linebreak
\\[-12pt]
\hspace*{23pt}fractional integral functional dependent on the parameter&2&\hphantom{1}98--103\\
\Avtors{Shnurkov~P.\,V. and Novikov~D.\,A.} On the concept of a stochastic model with control at the\linebreak
\\[-12pt]
\hspace*{23pt}moments of the process at the border of a presented subset of multiple states&3&101--108\\
\Avtors{Shorgin~S.\,Ya.} see Ageev~K.\,A.&&\\
\Avtors{Shorgin~S.\,Ya.} see Grusho~N.\,A.&&\\
\Avtors{Shorgin~S.\,Ya.} see Kharin~P.\,A.&&\\
\Avtors{Shorgin~V.\,S.} see Moskaleva~F.\,A.&&\\
\Avtors{Shvartsman~M.\,E.} see Krasnov~F.\,V.&&\\
\Avtors{Simavoryan~S.\,Zh.} see Popov~G.\,A.&&\\
\Avtors{Simonyan~A.\,R.} see Popov~G.\,A.&&\\
\Avtors{Smirnov~D.\,V.} see Grusho~A.\,A.&&\\
\Avtors{Smirnov~D.\,V.} see Grusho~N.\,A.&&\\
\Avtors{Sochenkov~I.\,V.} see Budzko~V.\,I.&&\\
\Avtors{Sokolov~I.\,A., Stepchenkov~Yu.\,A., Diachenko~Yu.\,G., and Rogdestvenski~Yu.\,V.} Improvement of\linebreak
\\[-12pt]
\hspace*{23pt}self-timed circuit soft error tolerance&4&63--68\\
\Avtors{Somin~N.\,V.} see Kozerenko~E.\,B.&&\\
\Avtors{Sopin~E.\,S.} see Ageev~K.\,A.&&\\
\Avtors{Soprunov~S.\,F.} see Betelin~V.\,B.&&\\
\Avtors{Stefanovich~A.\,I.} see Bosov~A.\,V.&&\\
\Avtors{Stepchenkov~Yu.\,A.} see Sokolov~I.\,A.&&\\
\Avtors{Strijov~V.\,V.} see Goncharov~A.\,V.&&\\
\Avtors{Strijov~V.\,V.} see Grabovoy~A.\,V.&&\\
\Avtors{Strijov~V.\,V.} see Potanin~M.\,S.&&\\
\Avtors{Stupnikov~S.\,A.} see Briukhov~D.\,O.&&\\
\Avtors{Teryokhina~I.\,Yu.} see Grusho~A.\,A.&&\\
\Avtors{Timonina~E.\,E.} see Grusho~A.\,A.&&\\
\end{tabular}
}
\pagebreak

\def\leftfootline{\small{\textbf{\thepage}
\hfill INFORMATIKA I EE PRIMENENIYA~--- INFORMATICS AND APPLICATIONS\ \ \ 2020\
\ \ volume~14\ \ \ issue\ 4}
}%
 \def\rightfootline{\small{INFORMATIKA I EE PRIMENENIYA~---
INFORMATICS AND APPLICATIONS\ \ \ 2020\ \ \ volume~14\ \ \ issue\ 4
\hfill \textbf{\thepage}}}

\def\leftkol{2020 AUTHOR INDEX} % ENGLISH ABSTRACTS}

\def\rightkol{2020 AUTHOR INDEX} %ENGLISH ABSTRACTS}


\noindent
{\tabcolsep=3pt
\begin{tabular}{p{395.48108pt}cc}
&\textbf{Issue} & \textbf{Page}\\[6pt]
\Avtors{Timonina~E.\,E.} see Grusho~A.\,A.&&\\
\Avtors{Timonina~E.\,E.} see Grusho~A.\,A.&&\\
\Avtors{Timonina~E.\,E.} see Grusho~N.\,A.&&\\
\Avtors{Timonina~E.\,E.} see Grusho~N.\,A.&&\\
\Avtors{Tyrsin~A.\,N.} see Serebryanskii~S.\,M.&&\\
\Avtors{Ulitina~E.\,I.} see Popov~G.\,A.&&\\
\Avtors{Vayser~K.\,O.} see Potanin~M.\,S.&&\\
\Avtors{Vokhmintcev~A.\,V., Melnikov~A.\,V., and Pachganov~S.\,A.} Simultaneous localization and mapping method in  three-dimensional space based on the combined solution of the  point--point\linebreak
\\[-12pt]
\hspace*{23pt}variation problem ICP for an affine transformation&1&101--112\\
\Avtors{Yadrintsev~V.\,V.} see Budzko~V.\,I.&&\\
\Avtors{Yarkina~N.\,V.} see Ageev~K.\,A.&&\\
\Avtors{Zabezhailo~M.\,I.} see Grusho~A.\,A.&&\\
\Avtors{Zabezhailo~M.\,I.} see Grusho~A.\,A.&&\\
\Avtors{Zabezhailo~M.\,I.} see Grusho~N.\,A.&&\\
\Avtors{Zabezhailo~M.\,I.} see Grusho~N.\,A.&&\\
\Avtors{Zakharov~V.\,N.} see Frenkel~S.\,L.&&\\
\Avtors{Zatsman~I.\,M.} Problem-oriented verifying the completeness  of~temporal ontologies and\linebreak
\\[-12pt]
\hspace*{23pt}filling~conceptual lacunas&3&119--128\\
\Avtors{Zatsman~I.\,M.} see Goncharov~A.\,A.&&\\
\Avtors{Zatsman~I.\,M.} see Nuriev~V.\,A.&&\\
\Avtors{Zeifman~A.\,I.} see Satin~Ya.\,A.&&\\
\Avtors{Zholobov~V.\,A.} see Potanin~M.\,S.&&\\
\end{tabular}
}

%\thispagestyle{myheadings}
\def\leftfootline{\small{\textbf{\thepage}
\hfill INFORMATIKA I EE PRIMENENIYA~--- INFORMATICS AND APPLICATIONS\ \ \ 2020\
\ \ volume~14\ \ \ issue\ 4}
}%
 \def\rightfootline{\small{INFORMATIKA I EE PRIMENENIYA~---
INFORMATICS AND APPLICATIONS\ \ \ 2020\ \ \ volume~14\ \ \ issue\ 4
\hfill \textbf{\thepage}}}

 \label{end\stat}

\newpage


%\linebreak
%\\[-12pt]
%\hspace*{23pt}

%   \vspace*{-48pt}

\begin{center}
\vspace*{6pt}
\mbox{%
%\epsfxsize=50mm %56.519mm  
%\epsfbox{smu-1.eps} 

\epsfxsize=50mm %46.402 mm
\epsfbox{nec-rb.eps}
}
%\end{center}

\vspace*{9pt} %Академик


%   \begin{center}
\fbox{\large\textbf{Рустем Бадриевич Сейфуль-Мулюков}}\\[6pt]
\textbf{\large 1928--2020}
   \end{center}


   %\vspace*{2.5mm}

   \vspace*{5mm}

   \thispagestyle{empty}

%\

%\vspace*{-12pt}

  
      Редакция журнала <<Информатика и~её применения>> с глубоким 
      прискорбием сообщают, что 17~марта 2020~г.\ на 93-м~году жизни 
      скончался заведующий редакцией журнала, главный научный сотрудник Федерального исследовательского центра <<Информатика и~управление>> Российской академии наук
      Рустем Бадриевич Сейфуль-Мулюков.
           
     Всю свою жизнь Рустем Бадриевич посвятил служению науке. Закончив в~1956~г.\ аспирантуру Московского ордена Трудового Красного знамени Нефтяного института им.\ академика
     И.\,М.~Губкина, он прошел путь от заведующего отделом Института геологии зарубежных стран Министерства геологии СССР до заместителя директора ВИНИТИ
     АН СССР, доктора гео\-ло\-го-ми\-не\-ра\-ло\-ги\-че\-ских наук, профессора.
     
     С марта 2002~г.\ Рустем Бадриевич успешно применял свои знания и~организационный талант в ИПИ
     РАН (в~дальнейшем~--- ФИЦ ИУ РАН), в~котором руководил лабораторией и~отделом, занимающимися вопросами технологий информационной технической деятельности. 
Р.\,Б.~Сейфуль-Мулюков, являясь автором значительного количества научных трудов и~монографий по геологии, информационным технологиям и~теоретической информатике, осуществлял организацию издания монографий ИПИ РАН и~ФИЦ ИУ РАН, библиографий научных сотрудников Центра.
     
     Р.\,Б.~Сейфуль-Мулюков являлся заведующим редакцией журналов <<Информатика и~её применения>> и~<<Системы и~средства информатики>>, членом редколлегии журнала <<Системы и~средства информатики>>. Он вложил огромный вклад в становление и~развитие этих журналов, организацию их регистрации, функционирования, редактуры и~издания. Включение этих журналов в ряд отечественных и~зарубежных информационных баз и~систем цитирования во многом является его личной заслугой.
     
     На всех занимаемых должностях Рустем Бадриевич отличался высоким профессионализмом, преданностью делу и~вниманием к коллегам.
     
     \thispagestyle{empty}
     
     Рустема Бадриевича отличали доброта, отзывчивость, неиссякаемый
      оптимизм, простота и~сердечность.
     
     Коллеги Рустема Бадриевича запомнят его как многогранного в~своих увлечениях человека, живописца,
     эрудита и~энциклопедиста, интересующегося историей, литературой и~искусством.
     
     Выражаем глубокое
     соболезнование семье, родственникам, друзьям и~коллегам по работе в~связи с~тяжелой невосполнимой утратой.
     Светлый образ Рустема Бадриевича навсегда сохранится в~нашей памяти.
     

      

%\def\stat{cont}
{%\hrule\par
%\vskip 7pt % 7pt
\raggedleft\Large \bf%\baselineskip=3.2ex
А\,В\,Т\,О\,Р\,С\,К\,И\,Й\ \ У\,К\,А\,З\,А\,Т\,Е\,Л\,Ь\ \ З\,А\ \ 2\,0\,1\,0 г. \vskip 17pt
    \hrule
    \par
\vskip 21pt plus 6pt minus 3pt }

\label{st\stat}

\def\tit{\ }

\def\aut{\ }
\def\auf{\ }

\def\leftkol{\ } % ENGLISH ABSTRACTS}

\def\rightkol{\ } %АВТОРСКИЙ УКАЗАТЕЛЬ ЗА 2010 г.} %ENGLISH ABSTRACTS}

\titele{\tit}{\aut}{\auf}{\leftkol}{\rightkol}

\vspace*{-12pt}

{\tabcolsep=3pt
\begin{tabular}{p{388pt}rr}
&\textbf{Выпуск} & \textbf{Стр.}\\[6pt]
\hangindent=23pt\noindent\textbf{Арутюнян~А.\,Р.} Моделирование влияния деформаций отпечатков пальцев на 
точность\linebreak
\vspace*{-12pt}\\
\hspace*{23pt}дактилоскопической идентификации$\dotfill$&1&51\\
\hangindent=23pt\noindent\textbf{Архипов~О.\,П., Зыкова~З.\,П.} Интеграция гетерогенной информации о цветных 
пикселях\linebreak
\vspace*{-12pt}\\
\hspace*{23pt}и их цветовосприятии$\dotfill$&4&15\\
\hangindent=23pt\noindent\textbf{Баранов~С.\,И., Френкель~С.\,Л., Захаров~В.\,Н.} Полуформальная верификация 
цифрового устройства с конвейером, основанная на использовании алгоритмических машин\linebreak
\vspace*{-12pt}\\
\hspace*{23pt}состояния$\dotfill$&4&49\\
\textbf{Бекетова~И.\,В.} см.~Каратеев~С.\,Л.&&\\
\textbf{Белоусов~В.\,В.} см.~Синицын~И.\,Н.&&\\
\hangindent=23pt\noindent\textbf{Бенинг~В.\,Е., Королев~Р.\,А.} О предельном поведении мощностей критериев в 
случае\linebreak
\vspace*{-12pt}\\
\hspace*{23pt}распределения Лапласа$\dotfill$&2&63\\
\hangindent=23pt\noindent\textbf{Бенинг~В.\,Е., Сипина~А.\,В.} Асимптотическое разложение для мощности 
критерия,\linebreak
\vspace*{-12pt}\\
\hspace*{23pt}основанного на выборочной медиане, в случае распределения Лапласа$\dotfill$&1&18\\
\textbf{Бондаренко~А.\,В.} см.~Каратеев~С.\,Л.&&\\
\hangindent=23pt\noindent\textbf{Бородина~А.\,В., Морозов~Е.\,В.} Об оценивании асимптотики вероятности 
большого\linebreak
\vspace*{-12pt}\\
\hspace*{23pt}уклонения стационарной регенеративной очереди с одним прибором$\dotfill$&3&29\\
\hangindent=23pt\noindent\textbf{Бунтман~Н.\,В., Минель~Ж.-Л., Ле~Пезан~Д., Зацман~И.\,М.} Типология и 
компьютерное\linebreak
\vspace*{-12pt}\\
\hspace*{23pt}моделирование трудностей перевода$\dotfill$&3&77\\
\textbf{Визильтер~Ю.\,В.} см.~Каратеев~С.\,Л.&&\\
\hangindent=23pt\noindent\textbf{Гавриленко~С.\,В.} Оценки скорости сходимости распределений случайных сумм с 
безгранично делимыми индексами к нормальному закону$\dotfill$&4&81\\
\hangindent=23pt\noindent\textbf{Григорьева~М.\,Е., Шевцова~И.\,Г.} Уточнение неравенства 
Каца--Берри--Эссеена$\dotfill$&2&75\\
\hangindent=23pt\noindent\textbf{Грушо~А.\,А., Грушо~Н.\,А., Тимонина~Е.\,Е.} Поиск конфликтов в политиках 
безопасности: модель случайных графов$\dotfill$&3&38\\
\textbf{Грушо~Н.\,А.} см.~Грушо~А.\,А.&&\\
\hangindent=23pt\noindent\textbf{Гудков~В.\,Ю.} Математические модели изображения отпечатка пальца на основе 
описания линий$\dotfill$&1&58\\
\textbf{Гуртов~А.\,В.} см.~Лукьяненко~А.\,С.&&\\
\textbf{Желтов~С.\,Ю.} см.~Каратеев~С.\,Л.&&\\
\hangindent=23pt\noindent\textbf{Захаров~А.\,А., Серебряков~В.\,А.} Система управления электронной библиотекой 
LibMeta$\dotfill$&4&2\\
\textbf{Захаров~В.\,Н.} см.~Баранов~С.\,И.&&\\
\textbf{Захарова~Т.\,В.} см.~Матвеева~С.\,С.&&\\
\hangindent=23pt\noindent\textbf{Зацаринный~А.\,А., Чупраков~К.\,Г.} Некоторые аспекты выбора технологии для 
постро-\linebreak
\vspace*{-12pt}\\
\hspace*{23pt}ения систем отображения информации ситуационного центра$\dotfill$&3&59\\
\textbf{Зацман~И.\,М.} см.~Бунтман~Н.\,В.&&\\
\hangindent=23pt\noindent\textbf{Зейфман~А.\,И., Коротышева~А.\,В., Сатин~Я.\,А., Шоргин~С.\,Я.} Об 
устойчивости нестаци-\linebreak
\vspace*{-12pt}\\
\hspace*{23pt}онарных систем обслуживания с катастрофами$\dotfill$&3&9\\
\textbf{Зыкова~З.\,П.} см.~Архипов~О.\,П.&&\\
\hangindent=23pt\noindent\textbf{Илюшин~Г.\,Я., Соколов~И.\,А.} Организация управляемого доступа пользователей 
к\linebreak
\vspace*{-12pt}\\
\hspace*{23pt}разнородным ведомственным информационным ресурсам$\dotfill$&1&24\\
\hangindent=23pt\noindent\textbf{Кавагучи~Ю., Ульянов~В.\,В., Фуджикоши~Я.} Приближения для статистик, 
описывающих\linebreak
\vspace*{-12pt}\\
\hspace*{23pt}геометрические свойства данных большой размерности, с оценками 
ошибок$\dotfill$&1&12\\
\hangindent=23pt\noindent\textbf{Каратеев~С.\,Л., Бекетова~И.\,В., Ососков~М.\,В., Князь~В.\,А., 
Визильтер~Ю.\,В., Бондаренко~А.\,В., Желтов~С.\,Ю.} Автоматизированный контроль 
качества цифровых\linebreak
\vspace*{-12pt}\\
\hspace*{23pt}изображений для персональных документов$\dotfill$&1&65\\
\end{tabular}
}

\pagebreak

\def\leftkol{АВТОРСКИЙ УКАЗАТЕЛЬ ЗА 2010 г.} % ENGLISH ABSTRACTS}

\def\rightkol{АВТОРСКИЙ УКАЗАТЕЛЬ ЗА 2010 г.} %ENGLISH ABSTRACTS}

{\tabcolsep=3pt
\begin{tabular}{p{388pt}rr}
&\textbf{Выпуск} & \textbf{Стр.}\\[3pt]
\hangindent=23pt\noindent\textbf{Козеренко~Е.\,Б.} Лингвистические фильтры в статистических моделях машинного\linebreak
\vspace*{-12pt}\\
\hspace*{23pt}перевода$\dotfill$&2&83\\
\hangindent=23pt\noindent\textbf{Козеренко~Е.\,Б., Кузнецов~И.\,П.} Когнитивно-лингвистические представления в 
систе-\linebreak
\vspace*{-12pt}\\
\hspace*{23pt}мах обработки текстов$\dotfill$&3&69\\
\textbf{Князь~В.\,А.} см.~Каратеев~С.\,Л.&&\\
\hangindent=23pt\noindent\textbf{Колесников~А.\,В., Солдатов~С.\,А.} Алгоритм координации для гибридной 
интеллектуальной системы решения сложной задачи оперативно-производственного\linebreak
\vspace*{-12pt}\\
\hspace*{23pt}планирования$\dotfill$&4&61\\
\hangindent=23pt\noindent\textbf{Коновалов~М.\,Г.} О планировании потоков в системах вычислительных 
ресурсов$\dotfill$&2&3\\
\textbf{Конушин~А.\,С.} см.~Конушин~В.\,С.&&\\
\hangindent=23pt\noindent\textbf{Конушин~В.\,С., Кривовязь~Г.\,Р., Конушин~А.\,С.} Алгоритм распознавания людей 
в видео-\linebreak
\vspace*{-12pt}\\
\hspace*{23pt}последовательности по одежде$\dotfill$&1&74\\
\textbf{Корепанов~Э.\, Р.} см.~Синицын~И.\,Н.&&\\
\textbf{Королев~В.\,Ю.} см.~Соколов~И.\,А.&&\\
\textbf{Королев~Р.\,А.} см.~Бенинг~В.\,Е.&&\\
\textbf{Коротышева~А.\,В.} см.~Зейфман~А.\,И.&&\\
\hangindent=23pt\noindent\textbf{Кривенко~М.\,П.} Непараметрическое оценивание элементов байесовского 
клас\-си-\linebreak
\vspace*{-12pt}\\
\hspace*{23pt}фикатора$\dotfill$&2&13\\
\textbf{Кривовязь~Г.\,Р.} см.~Конушин~В.\,С.&&\\
\textbf{Крылов~А.\,С.} см.~Павельева~Е.\,А.&&\\
\hangindent=23pt\noindent\textbf{Крылов~В.\,А.} Моделирование и классификация многоканальных дистанционных\linebreak
\vspace*{-12pt}\\
\hspace*{23pt}изображений с использованием копул$\dotfill$&4&34\\
\hangindent=23pt\noindent\textbf{Крючин~О.\,В.} Разработка параллельных эвристических алгоритмов подбора 
весовых\linebreak
\vspace*{-12pt}\\
\hspace*{23pt}коэффициентов искусственной нейтронной сети$\dotfill$&2&53\\
\hangindent=23pt\noindent\textbf{Кудрявцев~А.\,А., Шоргин~С.\,Я.} Байесовские модели массового обслуживания и 
надеж-\linebreak
\vspace*{-12pt}\\
\hspace*{23pt}ности: характеристики среднего числа заявок в системе $M\vert M \vert 1\vert 
\infty$$\dotfill$&3&16\\
\hangindent=23pt\noindent\textbf{Кузнецов~А.\,А.} Связь между временными и структурно-топологическими 
характери-\linebreak
\vspace*{-12pt}\\
\hspace*{23pt}стиками диаграмм ритма сердца здоровых людей$\dotfill$&4&39\\
\textbf{Кузнецов~И.\,П.} см.~Козеренко~Е.\,Б.&&\\
\textbf{Ле~Пезан~Д.} см.~Бунтман~Н.\,В.&&\\
\hangindent=23pt\noindent\textbf{Лукьяненко~А.\,С., Морозов~Е.\,В., Гуртов~А.\,В.} Анализ сетевого протокола с общей 
функ-\linebreak
\vspace*{-12pt}\\
\hspace*{23pt}цией расширения окна передачи сообщения при конфликтах$\dotfill$&2&46\\
\hangindent=23pt\noindent\textbf{Лямин~О.\,О.} О предельном поведении мощностей критериев в случае обобщенного\linebreak
\vspace*{-12pt}\\
\hspace*{23pt}распределения Лапласа$\dotfill$&3&47\\
\hangindent=23pt\noindent\textbf{Маркин~А.\,В., Шестаков~О.\,В.} Асимптотики оценки риска при пороговой 
обработке\linebreak
\vspace*{-12pt}\\
\hspace*{23pt}вейвлет-вейглет коэффициентов в задаче томографии$\dotfill$&2&36\\
\hangindent=23pt\noindent\textbf{Матвеева~С.\,С., Захарова~Т.\,В.} Сети массового обслуживания с наименьшей 
длиной\linebreak
\vspace*{-12pt}\\
\hspace*{23pt}очереди$\dotfill$&3&22\\
\hangindent=23pt\noindent\textbf{Матюшенко~С.\,И.} Стационарные характеристики двухканальной системы 
обслужива-\linebreak
\vspace*{-12pt}\\
\hspace*{23pt}ния с переупорядочиванием заявок и распределениями фазового типа$\dotfill$&4&68\\
\textbf{Минель~Ж.-Л.} см.~Бунтман~Н.\,В.&&\\
\textbf{Морозов~Е.\,В.} см.~Бородина~А.\,В.&&\\
\textbf{Морозов~Е.\,В.} см.~Лукьяненко~А.\,С.&&\\
\textbf{Ососков~М.\,В.} см.~Каратеев~С.\,Л.&&\\
\hangindent=23pt\noindent\textbf{Павельева~Е.\,А., Крылов~А.\,С.} Поиск и анализ ключевых точек радужной 
оболочки\linebreak
\vspace*{-12pt}\\
\hspace*{23pt}глаза методом преобразования Эрмита$\dotfill$&1&79\\
\textbf{Печинкин~А.\,В.} см.~Френкель~С.\,Л.,&&\\
\hangindent=23pt\noindent\textbf{Протасов~В.\,И.} Составление субъективного портрета с использованием 
эволюционно-\linebreak
\vspace*{-12pt}\\
\hspace*{23pt}го морфинга и квалиметрия метода$\dotfill$&1&83\\
\hangindent=23pt\noindent\textbf{Рудаков~К.\,В., Торшин~И.\,Ю.} Вопросы разрешимости задачи распознавания 
вторичной\linebreak
\vspace*{-12pt}\\
\hspace*{23pt}структуры белка$\dotfill$&2&25\\
\textbf{Сатин~Я.\,А.} см.~Зейфман~А.\,И.&&\\
\hangindent=23pt\noindent\textbf{Сейфуль-Мулюков~Р.\,Б.} Нефть как носитель информации о своем 
происхождении,\linebreak
\vspace*{-12pt}\\
\hspace*{23pt}структуре и эволюции$\dotfill$&1&41\\
\end{tabular}
}

{\tabcolsep=3pt
\begin{tabular}{p{388pt}rr}
&\textbf{Выпуск} & \textbf{Стр.}\\[6pt]
\textbf{Семендяев~Н.\,Н.} см.~Синицын~И.\,Н.&&\\
\textbf{Серебряков~В.\,А.} см.~Захаров~А.\,А.&&\\
\textbf{Синицын~В.\,И.} см.~Синицын~И.\,Н.&&\\
\hangindent=23pt\noindent\textbf{Синицын~И.\,Н., Синицын~В.\,И., Корепанов~Э.\, Р., Белоусов~В.\,В., 
Семендяев~Н.\,Н.} Оперативное построение информационных моделей движения полюса 
Земли\linebreak
\vspace*{-12pt}\\
\hspace*{23pt}методами линейных и линеаризованных фильтров$\dotfill$&1&2\\
\textbf{Сипина~А.\,В.} см.~Бенинг~В.\,Е.&&\\
\hangindent=23pt\noindent\textbf{Соколов~И.\,А.} О работах заслуженного деятеля науки Российской Федерации 
И.\,Н.~Синицына в области информационных технологий и автоматизации (к 70-летию\linebreak
\vspace*{-12pt}\\
\hspace*{23pt}со дня рождения)$\dotfill$&3&84\\
\textbf{Соколов~И.\,А.} см.~Илюшин~Г.\,Я.&&\\
\hangindent=23pt\noindent\textbf{Соколов~И.\,А., Королев~В.\,Ю.} Предисловие$\dotfill$&2&2\\
\textbf{Солдатов~С.\,А.} см.~Колесников~А.\,В.&&\\
\hangindent=23pt\noindent\textbf{Степанов~С.\,Ю.} Использование координатного метода фрагментации 
коммутаторной\linebreak
\vspace*{-12pt}\\
\hspace*{23pt}нейронной сети для сокращения трафика$\dotfill$&2&57\\
\textbf{Тимонина~Е.\,Е.} см.~Грушо~А.\,А.&&\\
\textbf{Торшин~И.\,Ю.} см.~Рудаков~К.\,В.&&\\
\textbf{Ульянов~В.\,В.} см.~Кавагучи~Ю.&&\\
\textbf{Фазекаш~И.} см.~Чупрунов~А.\,Н.&&\\
\textbf{Френкель~С.\,Л.} см.~Баранов~С.\,И.&&\\
\hangindent=23pt\noindent\textbf{Френкель~С.\,Л., Печинкин~А.\,В.} Оценка времени самовосстановления в 
цифровых\linebreak
\vspace*{-12pt}\\
\hspace*{23pt}системах после сбоев, вызываемых переходными помехами$\dotfill$&3&2\\
\textbf{Фуджикоши~Я.} см.~Кавагучи~Ю.&&\\
\hangindent=23pt\noindent\textbf{Цискаридзе~А.\,К.} Математическая модель и метод восстановления позы человека 
по\linebreak
\vspace*{-12pt}\\
\hspace*{23pt}стереопаре силуэтных изображений$\dotfill$&4&27\\
\hangindent=23pt\noindent\textbf{Чупраков~К.\,Г.} К вопросу о размещении коллективных средств отображения в 
ситуа-\linebreak
\vspace*{-12pt}\\
\hspace*{23pt}ционном зале с заданными параметрами$\dotfill$&4&89\\
\textbf{Чупраков~К.\,Г.} см.~Зацаринный~А.\,А.&&\\
\hangindent=23pt\noindent\textbf{Чупрунов~А.\,Н., Фазекаш~И.} Законы повторного логарифма для числа 
безошибочных\linebreak
\vspace*{-12pt}\\
\hspace*{23pt}блоков при помехоустойчивом кодировании$\dotfill$&3&42\\
\textbf{Шевцова~И.\,Г.} см.~Григорьева~М.\,Е.&&\\
\hangindent=23pt\noindent\textbf{Шестаков~О.\,В.} Аппроксимация распределения оценки риска пороговой 
обработки вейвлет-коэффициентов нормальным распределением при использовании 
выбо-\linebreak
\vspace*{-12pt}\\
\hspace*{23pt}рочной дисперсии$\dotfill$&4&73\\
\textbf{Шестаков~О.\,В.} см.~Маркин~А.\,В.&&\\
\textbf{Шоргин~С.\,Я.} см.~Зейфман~А.\,И.&&\\
\textbf{Шоргин~С.\,Я.} см.~Кудрявцев~А.\,А.&&\\
\end{tabular}
}

%\thispagestyle{myheadings}
\def\leftfootline{\small{\textbf{\thepage}
\hfill ИНФОРМАТИКА И ЕЁ ПРИМЕНЕНИЯ\ \ \ том~4\ \ \ выпуск~4\ \ \ 2010}
}%
 \def\rightfootline{\small{ИНФОРМАТИКА И ЕЁ ПРИМЕНЕНИЯ\ \ \ том~4\ \ \ выпуск~4\ \ \ 2010
 \hfill \textbf{\thepage}}}
 \label{end\stat}
%
%Том 10 Выпуск 1-4 Год 2016

\def\stat{cont-e}
{%\hrule\par
%\vskip 7pt % 7pt
\raggedleft\Large \bf%\baselineskip=3.2ex
2\,0\,1\,6\ \ A\,U\,T\,H\,O\,R\ \ I\,N\,D\,E\,X \vskip 17pt
 \hrule
 \par
\vskip 21pt plus 6pt minus 3pt }

\label{st\stat}

\def\tit{\ }

\def\aut{\ }
\def\auf{\ }

\def\leftkol{\ } %2016 AUTHOR INDEX} % ENGLISH ABSTRACTS}

\def\rightkol{\ } %2016 AUTHOR INDEX} %ENGLISH ABSTRACTS}

\titele{\tit}{\aut}{\auf}{\leftkol}{\rightkol}

\def\leftfootline{\small{\textbf{\thepage}
\hfill INFORMATIKA I EE PRIMENENIYA~--- INFORMATICS AND APPLICATIONS\ \ \ 2016\
\ \ volume~10\ \ \ issue\ 4}
}%
 \def\rightfootline{\small{INFORMATIKA I EE PRIMENENIYA~--- INFORMATICS AND APPLICATIONS\ \ \ 2016\ \ \ volume~10\ \ \ issue\ 4
\hfill \textbf{\thepage}}}

\vspace*{-12pt}
\vspace*{-18pt}

{\tabcolsep=2.8pt
\begin{tabular}{p{382pt}cc}
&\textbf{Issue} & \textbf{Page}\\[6pt]
\Avtors{Agalarov~M.\,Ya.} see~Agalarov~Ya.\,M.&&\\
\Avtors{Agalarov~Ya.\,M., Agalarov~M.\,Ya., and
Shorgin~V.\,S.} About the optimal threshold of queue\linebreak
\\[-12pt]
\hspace*{23pt}length in a~particular problem of profit maximization
in the $M/G/1$ queuing system&2&70--79\\
\Avtors{Alexeyevsky~D.\,A.} BioNLP ontology extraction from 
a~restricted language corpus with\linebreak
\\[-12pt]
\hspace*{23pt}context-free grammars&1&119--128\\
\Avtors{Andreev~S.\,D.} see~Gaidamaka~Yu.\,V.&&\\
\Avtors{Andreev~S.\,D.} see~Ometov~A.\,Ya.&&\\
\Avtors{Arkhipov~O.\,P., Arkhipov~P.\,O., and Sidorkin~I.\,I.} The
option to create a~local coordinate\linebreak
\\[-12pt]
\hspace*{23pt}system for synchronization of selected images&3&91--97\\
\Avtors{Arkhipov~P.\,O.} see~Arkhipov~O.\,P.&&\\
\Avtors{Belousov~V.\,V.} see~Shnurkov~P.\,V.&&\\
\Avtors{Belousov~V.\,V.} see~Shnurkov~P.\,V.&&\\
\Avtors{Bening~V.\,E.} Calculation of~the~asymptotic deficiency
of~some statistical procedures based\linebreak
\\[-12pt]
\hspace*{23pt}on~samples with~random sizes&4&34--45\\
\Avtors{Borisov~A.\,V., Bosov~A.\,V., and Miller~G.\,B.} Modeling and
monitoring of VoIP connection&2&\hphantom{1}2--13\\
\Avtors{Bosov~A.\,V.} see~Borisov~A.\,V.&&\\
\Avtors{Briukhov~D.\,O.} see~Stupnikov~S.\,A.&&\\
\Avtors{Callaos~N.\,K.\ and Seyful-Mulyukov~R.\,B.} Complexity and
its information content&1&129--139\\
\Avtors{Chertok~A.\,V., Kadaner~A.\,I., Khazeeva~G.\,T., and
Sokolov~I.\,A.} Regime switching detection\linebreak
\\[-12pt]
\hspace*{23pt}for~the~Levy driven
Ornstein--Uhlenbeck process using CUSUM methods&4&46--56\\
\Avtors{Chichagov~V.\,V.} Asymptotic expansions of mean absolute
error of uniformly minimum variance unbiased and maximum likelihood
estimators on the one-parameter exponential\linebreak
\\[-12pt]
\hspace*{23pt}family model of lattice distributions&3&66--76\\
\Avtors{Danishevsky~V.\,I.} see~Kolesnikov A.\,V.&&\\
\Avtors{Fazliev~A.\,Z.} see~Kalinichenko~L.\,A.&&\\
\Avtors{Fedoseev~A.\,A.} What is behind the concept of ``knowledge in
small packages''&3&105--110\\
\Avtors{Gaidamaka~Yu.\,V., Andreev~S.\,D., Sopin~E.\,S.,
Samouylov~K.\,E., and Shorgin~S.\,Ya.} Interference analysis
of~the~device-to-device communications model with~regard to~a~signal\linebreak
\\[-12pt]
\hspace*{23pt}propagation environment&4&\hphantom{1}2--10\\
\Avtors{Gasilov~A.\,V.} see~Yakovlev~O.\,A.&&\\
\Avtors{Goncharov~A.\,V.\ and Strijov~V.\,V.} Metric time series
classification using weighted dynamic\linebreak
\\[-12pt]
\hspace*{23pt}warping relative to centroids of classes&2&36--47\\
\Avtors{Gordov~E.\,P.} see~Kalinichenko~L.\,A.&&\\
\Avtors{Gorshenin~A.\,K.} Concept of online service for stochastic
modeling of real processes&1&72--81\\
\Avtors{Gorshenin~A.\,K.} see~Shnurkov~P.\,V.&&\\
\Avtors{Gorshenin~A.\,K.} see~Shnurkov~P.\,V.&&\\
\Avtors{Grusho~A.\,A., Grusho~N.\,A., Zabezhailo~M.\,I., and
Timonina~E.\,E.} Integration of statistical and\linebreak
\\[-12pt]
\hspace*{23pt}deterministic methods for
analysis of information security&3&2--8\\
\Avtors{Grusho~A.\,A., Zabezhailo~M.\,I., and Zatsarinny~A.\,A.} On
the advanced procedure to reduce\linebreak
\\[-12pt]
\hspace*{23pt}calculation of Galois closures&4&\hphantom{1}96--104\\
\Avtors{Grusho~N.\,A.} see~Grusho~A.\,A.&&\\
\Avtors{Havanskov~V.\,A.} see~Minin~V.\,A.&&\\
\Avtors{Inkova~O.\,Yu.} see~Zatsman~I.\,M.&&\\
\Avtors{Isachenko~R.\,V.\ and Strijov~V.\,V.} Metric learning in
multiclass time series classification\linebreak
\\[-12pt]
\hspace*{23pt}problem&2&48--57\\
\end{tabular}
}
\pagebreak

\def\leftfootline{\small{\textbf{\thepage}
\hfill INFORMATIKA I EE PRIMENENIYA~--- INFORMATICS AND APPLICATIONS\ \ \ 2016\
\ \ volume~10\ \ \ issue\ 4}
}%
 \def\rightfootline{\small{INFORMATIKA I EE PRIMENENIYA~---
INFORMATICS AND APPLICATIONS\ \ \ 2016\ \ \ volume~10\ \ \ issue\ 4
\hfill \textbf{\thepage}}}

\def\leftkol{2016 AUTHOR INDEX} % ENGLISH ABSTRACTS}

\def\rightkol{2016 AUTHOR INDEX} %ENGLISH ABSTRACTS}


{\tabcolsep=2.83pt
\begin{tabular}{p{382pt}cc}
&\textbf{Issue} & \textbf{Page}\\[6pt]
\Avtors{Kadaner~A.\,I.} see~Chertok~A.\,V.&&\\[.255pt]
\Avtors{Kalinichenko~L.\,A., Volnova~A.\,A., Gordov~E.\,P.,
Kiselyova~N.\,N., Kovaleva~D.\,A., Malkov~O.\,Yu., Okladnikov~I.\,G.,
Podkolodnyy~N.\,L., Pozanenko~A.\,S., Ponomareva~N.\,V.,
Stupnikov~S.\,A.,} \textbf{and Fazliev~A.\,Z.} Data access challenges for data
intensive\linebreak
\\[-12pt]
\hspace*{23pt}research in Russia&1& 2--22\\[.255pt]
\Avtors{Karasikov~M.\,E.\ and Strijov~V.\,V.} Feature-based
time-series classification&4&121--131\\[.255pt]
\Avtors{Khazeeva~G.\,T.} see~Chertok~A.\,V.&&\\[.255pt]
\Avtors{Khokhlov~Yu.\,S.} Multivariate fractional Levy motion and its
applications&2&\hphantom{1}98--106\\[.255pt]
\Avtors{Kirikov~I.\,A., Kolesnikov~A.\,V., Listopad~S.\,V., and
Rumovskaya~S.\,B.} Fine-grained hybrid\linebreak
\\[-12pt]
\hspace*{23pt}intelligent systems. Part 2:
Bidirectional hybridization&1&\hphantom{1}96--105\\[.255pt]
\Avtors{Kirikov~I.\,A., Kolesnikov~A.\,V., Listopad~S.\,V., and
Rumovskaya~S.\,B.} ``Virtual council''~---\linebreak
\\[-12pt]
\hspace*{23pt}source environment
supporting complex diagnostic decision making&3&81--90\\[.255pt]
\Avtors{Kiselyova~N.\,N.} see~Kalinichenko~L.\,A.&&\\[.255pt]
\Avtors{Kolesnikov A.\,V., Listopad~S.\,V., Rumovskaya~S.\,B., and
Danishevsky~V.\,I.} Informal axiomatic\linebreak
\\[-12pt]
\hspace*{23pt}theory of~the~role visual models&4&114--120\\[.255pt]
\Avtors{Kolesnikov~A.\,V.} see~Kirikov~I.\,A.&&\\[.255pt]
\Avtors{Kolesnikov~A.\,V.} see~Kirikov~I.\,A.&&\\[.255pt]
\Avtors{Kolin~K.\,K.} Humanitarian aspects of information
security&3&111--121\\[.255pt]
\Avtors{Konovalov~M.\,G.\ and Razumchik~R.\,V.} Dispatching
to~two parallel nonobservable queues using\linebreak
\\[-12pt]
\hspace*{23pt}only static
information&4&57--67\\[.255pt]
\Avtors{Korchagin~A.\,Yu.} see~Korolev~V.\,Yu.&&\\[.255pt]
\Avtors{Korchagin~A.\,Yu.} see~Korolev~V.\,Yu.&&\\[.255pt]
\Avtors{Korepanov~E.\,R.} see~Sinitsyn~I.\,N.&&\\[.255pt]
\Avtors{Korepanov~E.\,R.} see~Sinitsyn~I.\,N.&&\\[.255pt]
\Avtors{Korolev~V.\,Yu., Korchagin~A.\,Yu., and Zeifman~A.\,I.} The
Poisson theorem for Bernoulli trials\linebreak
\\[-12pt]
\hspace*{23pt}with~a~random probability
of~success and~a~discrete analog of~the~Weibull distribution&4&11--20\\[.255pt]
\Avtors{Korolev~V.\,Yu., Zeifman~A.\,I., and Korchagin~A.\,Yu.}
Asymmetric Linnik distributions as~limit\linebreak
\\[-12pt]
\hspace*{23pt}laws for~random sums
of~independent random variables with~finite variances&4&21--33\\[.255pt]
\Avtors{Koucheryavy~E.\,A.} see~Ometov~A.\,Ya.&&\\[.255pt]
\Avtors{Kovaleva~D.\,A.} see~Kalinichenko~L.\,A.&&\\[.255pt]
\Avtors{Kovalyov~S.\,P.} Metaprogramming to increase
manufacturability of large-scale software-\linebreak
\\[-12pt]
\hspace*{23pt}intensive systems&1&56--66\\[.255pt]
\Avtors{Krivenko~M.\,P.} Significance tests of feature selection for
classification&3&32--40\\[.255pt]
\Avtors{Kruzhkov~M.\,G.} see~Zalizniak~Anna~A.&&\\[.255pt]
\Avtors{Kruzhkov~M.\,G.} see~Zatsman~I.\,M.&&\\[.255pt]
\Avtors{Kudryavtsev~A.\,A.} Bayesian queueing and reliability models:
\textit{A~priori} distributions with\linebreak
\\[-12pt]
\hspace*{23pt}compact support&1&67--71\\[.255pt]
\Avtors{Kudryavtsev~A.\,A.} Characteristics dependent on the balance
coefficient in Bayesian models\linebreak
\\[-12pt]
\hspace*{23pt}with compact support of \textit{a priori}
distributions&3&77--80\\[.255pt]
\Avtors{Kudryavtsev~A.\,A.\ and Palionnaia~S.\,I.} Bayesian recurrent
model of reliability growth:\linebreak
\\[-12pt]
\hspace*{23pt}Parabolic distribution of parameters&2&80--83\\[.255pt]
\Avtors{Kudryavtsev~A.\,A.\ and Titova~A.\,I.} Bayesian queuing
and~reliability models: Degenerate-\linebreak
\\[-12pt]
\hspace*{23pt}Weibull case&4&68--71\\[.255pt]
\Avtors{Leontyev~N.\,D.\ and Ushakov~V.\,G.} Analysis of a queueing
system with autoregressive arrivals\linebreak
\\[-12pt]
\hspace*{23pt}and nonpreemptive priority&3&15--22\\[.255pt]
\Avtors{Listopad~S.\,V.} see~Kirikov~I.\,A.&&\\[.255pt]
\Avtors{Listopad~S.\,V.} see~Kirikov~I.\,A.&&\\[.255pt]
\Avtors{Listopad~S.\,V.} see~Kolesnikov A.\,V.&&\\[.255pt]
\Avtors{Malkov~O.\,Yu.} see~Kalinichenko~L.\,A.&&\\[.255pt]
\Avtors{Markov~A.\,S., Monakhov~M.\,M., and
Ulyanov~V.\,V.} Generalized Cornish--Fisher expansions\linebreak
\\[-12pt]
\hspace*{23pt}for distributions of statistics based on samples
of random size&2&84--91\\[.255pt]
\Avtors{Melnikov~A.\,K.\ and Ronzhin~A.\,F.} Generalized statistical
method of~text analysis based\linebreak
\\[-12pt]
\hspace*{23pt}on~calculation of~probability distributions
of~statistical values&4&89--95\\
\end{tabular}
}
\pagebreak

\def\leftfootline{\small{\textbf{\thepage}
\hfill INFORMATIKA I EE PRIMENENIYA~--- INFORMATICS AND APPLICATIONS\ \ \ 2016\
\ \ volume~10\ \ \ issue\ 4}
}%
 \def\rightfootline{\small{INFORMATIKA I EE PRIMENENIYA~---
INFORMATICS AND APPLICATIONS\ \ \ 2016\ \ \ volume~10\ \ \ issue\ 4
\hfill \textbf{\thepage}}}

\def\leftkol{2016 AUTHOR INDEX} % ENGLISH ABSTRACTS}

\def\rightkol{2016 AUTHOR INDEX} %ENGLISH ABSTRACTS}


{\tabcolsep=3pt
\begin{tabular}{p{381pt}cc}
&\textbf{Issue} & \textbf{Page}\\[6pt]
\Avtors{Meykhanadzhyan~L.\,A.} Stationary characteristics of the finite
capacity queueing system with\linebreak
\\[-12pt]
\hspace*{23pt}inverse service order and generalized
probabilistic priority&2&123--131\\[.23pt]
\Avtors{Miller~G.\,B.} see~Borisov~A.\,V.&&\\[.23pt]
\Avtors{Minin~V.\,A., Zatsman~I.\,M., Havanskov~V.\,A., and
Shubnikov~S.\,K.} Intensity of citation of scientific publications in
inventions on information and computer technologies patented\linebreak
\\[-12pt]
\hspace*{23pt}in Russia by domestic and foreign applicants&2&107--122\\[.23pt]
\Avtors{Monakhov~M.\,M.} see~Markov~A.\,S.&&\\[.23pt]
\Avtors{Naumov~V.\,A.\ and Samouylov~K.\,E.} On relationship
between queuing systems with resources\linebreak
\\[-12pt]
\hspace*{23pt}and Erlang networks&3&\hphantom{1}9--14\\[.23pt]
\Avtors{Okladnikov~I.\,G.} see~Kalinichenko~L.\,A.&&\\[.23pt]
\Avtors{Ometov~A.\,Ya., Andreev~S.\,D., Turlikov~A.\,M., and
Koucheryavy~E.\,A.} Performance analysis of\linebreak
\\[-12pt]
\hspace*{23pt}a wireless data
aggregation system with contention for contemporary sensor
networks&3&23--31\\[.23pt]
\Avtors{Palionnaia~S.\,I.} see~Kudryavtsev~A.\,A.&&\\[.23pt]
\Avtors{Podkolodnyy~N.\,L.} see~Kalinichenko~L.\,A.&&\\[.23pt]
\Avtors{Ponomareva~N.\,V.} see~Kalinichenko~L.\,A.&&\\[.23pt]
\Avtors{Popkova~N.\,A.} see~Zatsman~I.\,M.&&\\[.23pt]
\Avtors{Pozanenko~A.\,S.} see~Kalinichenko~L.\,A.&&\\[.23pt]
\Avtors{Razumchik~R.\,V.} see~Konovalov~M.\,G.&&\\[.23pt]
\Avtors{Ronzhin~A.\,F.} see~Melnikov~A.\,K.&&\\[.23pt]
\Avtors{Rumovskaya~S.\,B.} see~Kirikov~I.\,A.&&\\[.23pt]
\Avtors{Rumovskaya~S.\,B.} see~Kirikov~I.\,A.&&\\[.23pt]
\Avtors{Rumovskaya~S.\,B.} see~Kolesnikov A.\,V.&&\\[.23pt]
\Avtors{Samouylov~K.\,E.} see~Gaidamaka~Yu.\,V.&&\\[.23pt]
\Avtors{Samouylov~K.\,E.} see~Naumov~V.\,A.&&\\[.23pt]
\Avtors{Serebryanskii~S.\,M.} see~Tyrsin~A.\,N.&&\\[.23pt]
\Avtors{Seyful-Mulyukov~R.\,B.} see~Callaos~N.\,K.&&\\[.23pt]
\Avtors{Shestakov~O.\,V.} Statistical properties of the denoising method
based on the stabilized hard\linebreak
\\[-12pt]
\hspace*{23pt}thresholding&2&65--69\\[.23pt]
\Avtors{Shestakov~O.\,V.} The strong law of large numbers for the risk
estimate in the problem of\linebreak
\\[-12pt]
\hspace*{23pt}tomographic image reconstruction from
projections with a correlated noise&3&41--45\\[.23pt]
\Avtors{Shestakov~O.\,V.} see~Zakharova~T.\,V.&&\\[.23pt]
\Avtors{Shnurkov~P.\,V., Gorshenin~A.\,K., and Belousov~V.\,V.}
Analytical solution of~the~optimal control\linebreak
\\[-12pt]
\hspace*{23pt}task of~a~semi-Markov
process with~finite set of~states&4&72--88\\[.23pt]
\Avtors{Shnurkov~P.\,V., Zasypko~V.\,V., Belousov~V.\,V., and
Gorshenin~A.\,K.} Development of the algorithm of numerical solution
of the optimal investment control problem\linebreak
\\[-12pt]
\hspace*{23pt}in the closed dynamical model of three-sector economy&1&82--95\\[.23pt]
\Avtors{Shorgin~S.\,Ya.} see~Gaidamaka~Yu.\,V.&&\\[.23pt]
\Avtors{Shorgin~V.\,S.} see~Agalarov~Ya.\,M.&&\\[.23pt]
\Avtors{Shubnikov~S.\,K.} see~Minin~V.\,A.&&\\[.23pt]
\Avtors{Sidorkin~I.\,I.} see~Arkhipov~O.\,P.&&\\[.23pt]
\Avtors{Sinitsyn~I.\,N.} Analytical modeling of processes in stochastic
systems with complex fractional\linebreak
\\[-12pt]
\hspace*{23pt}order Bessel nonlinearities&3&55--65\\[.23pt]
\Avtors{Sinitsyn~I.\,N.} Orthogonal supoptimal filters for nonlinear
stochastic systems on manifolds&1&34--44\\[.23pt]
\Avtors{Sinitsyn~I.\,N.\ and Korepanov~E.\,R.} Normal Pugachev
conditionally-optimal filters and extra-\linebreak
\\[-12pt]
\hspace*{23pt}polators for state linear stochastic systems&2&14--23\\[.23pt]
\Avtors{Sinitsyn~I.\,N.\ and Sinitsyn~V.\,I.} Analytical modeling of
distributions in stochastic systems on\linebreak
\\[-12pt]
\hspace*{23pt}manifolds based on ellipsoidal approximation&1&45--55\\[.23pt]
\Avtors{Sinitsyn~I.\,N., Sinitsyn~V.\,I., and
Korepanov~E.\,R.} Ellipsoidal suboptimal filters for nonlinear\linebreak
\\[-12pt]
\hspace*{23pt}stochastic systems on manifolds&2&24--35\\[.23pt]
\Avtors{Sinitsyn~V.\,I.} see~Sinitsyn~I.\,N.&&\\[.23pt]
\Avtors{Sinitsyn~V.\,I.} see~Sinitsyn~I.\,N.&&\\[.23pt]
\Avtors{Skvortsov~N.\,A.} see~Stupnikov~S.\,A.&&\\[.23pt]
\Avtors{Sokolov~I.\,A.} see~Chertok~A.\,V.&&\\
\end{tabular}
}
\pagebreak

\def\leftfootline{\small{\textbf{\thepage}
\hfill INFORMATIKA I EE PRIMENENIYA~--- INFORMATICS AND APPLICATIONS\ \ \ 2016\
\ \ volume~10\ \ \ issue\ 4}
}%
 \def\rightfootline{\small{INFORMATIKA I EE PRIMENENIYA~---
INFORMATICS AND APPLICATIONS\ \ \ 2016\ \ \ volume~10\ \ \ issue\ 4
\hfill \textbf{\thepage}}}

\def\leftkol{2016 AUTHOR INDEX} % ENGLISH ABSTRACTS}

\def\rightkol{2016 AUTHOR INDEX} %ENGLISH ABSTRACTS}


{\tabcolsep=3pt
\begin{tabular}{p{382pt}cc}
&\textbf{Issue} & \textbf{Page}\\[6pt]
\Avtors{Sopin~E.\,S.} see~Gaidamaka~Yu.\,V.&&\\
\Avtors{Strijov~V.\,V.} see~Goncharov~A.\,V.&&\\
\Avtors{Strijov~V.\,V.} see~Isachenko~R.\,V.&&\\
\Avtors{Strijov~V.\,V.} see~Karasikov~M.\,E.&&\\
\Avtors{Stupnikov~S.\,A., Briukhov~D.\,O., and Skvortsov~N.\,A.}
Co-lending systemic risk analysis over\linebreak
\\[-12pt]
\hspace*{23pt}heterogeneous data collections&1&23--33\\
\Avtors{Stupnikov~S.\,A.} see~Kalinichenko~L.\,A.&&\\
\Avtors{Suchkov~A.\,P.} see~Zatsarinny~A.\,A.&&\\
\Avtors{Timonina~E.\,E.} see~Grusho~A.\,A.&&\\
\Avtors{Titova~A.\,I.} see~Kudryavtsev~A.\,A.&&\\
\Avtors{Turlikov~A.\,M.} see~Ometov~A.\,Ya.&&\\
\Avtors{Tyrsin~A.\,N.\ and Serebryanskii~S.\,M.} Recognition of
dependences on the basis of inverse\linebreak
\\[-12pt]
\hspace*{23pt}mapping&2&58--64\\
\Avtors{Ulyanov~V.\,V.} see~Markov~A.\,S.&&\\
\Avtors{Ushakov~V.\,G.} Queueing system with working vacations and
hyperexponential input stream&2&92--97\\
\Avtors{Ushakov~V.\,G.} see~Leontyev~N.\,D.&&\\
\Avtors{Volnova~A.\,A.} see~Kalinichenko~L.\,A.&&\\
\Avtors{Yakovlev~O.\,A.\ and Gasilov~A.\,V.} Speeded-up stereo
matching using geodesic support weights&3&\hphantom{1}98--104\\
\Avtors{Zabezhailo~M.\,I.} see~Grusho~A.\,A.&&\\
\Avtors{Zabezhailo~M.\,I.} see~Grusho~A.\,A.&&\\
\Avtors{Zakharova~T.\,V.\ and Shestakov~O.\,V.} Precision analysis of
wavelet processing of aerodynamic\linebreak
\\[-12pt]
\hspace*{23pt}flow patterns&3&46--54\\
\Avtors{Zalizniak~Anna~A.\ and Kruzhkov~M.\,G.} Database
of~Russian impersonal verbal constructions&4&132--141\\
\Avtors{Zasypko~V.\,V.} see~Shnurkov~P.\,V.&&\\
\Avtors{Zatsarinny~A.\,A.\ and Suchkov~A.\,P.} Systems engineering
approaches to~the~establishment of\linebreak
\\[-12pt]
\hspace*{23pt}a~system for~decision support based
on~situational analysis&4&105--113\\
\Avtors{Zatsarinny~A.\,A.} see~Grusho~A.\,A.&&\\
\Avtors{Zatsman~I.\,M., Inkova~O.\,Yu., Kruzhkov~M.\,G., and
Popkova~N.\,A.} Representation of cross-\linebreak
\\[-12pt]
\hspace*{23pt}lingual knowledge about
connectors in supracorpora databases&1&106--118\\
\Avtors{Zatsman~I.\,M.} see~Minin~V.\,A.&&\\
\Avtors{Zeifman~A.\,I.} see~Korolev~V.\,Yu.&&\\
\Avtors{Zeifman~A.\,I.} see~Korolev~V.\,Yu.&&\\
\end{tabular}
}

%\thispagestyle{myheadings}
\def\leftfootline{\small{\textbf{\thepage}
\hfill INFORMATIKA I EE PRIMENENIYA~--- INFORMATICS AND APPLICATIONS\ \ \ 2016\
\ \ volume~10\ \ \ issue\ 4}
}%
 \def\rightfootline{\small{INFORMATIKA I EE PRIMENENIYA~---
INFORMATICS AND APPLICATIONS\ \ \ 2016\ \ \ volume~10\ \ \ issue\ 4
\hfill \textbf{\thepage}}}

 \label{end\stat}

\newpage

%\def\stat{rekl}
%\label{preobr}

%\def\tit{АКАДЕМИК ПУГАЧЁВ  ВЛАДИМИР СЕМЁНОВИЧ\\
%25.03.1911--25.03.1998}


%   \vspace*{-48pt}
%   \begin{center}\LARGE
%Академик Пугачёв  Владимир Семёнович\\ (25.03.1911--25.03.1998)
%   \end{center}
   
   %\vspace*{2.5mm}
   
   \begin{center}

{\prgsh\LARGE
ОБЪЯВЛЕНИЯ О КОНФЕРЕНЦИЯХ}

\end{center}
%\hrule

\vspace*{6pt}

   
   \vspace*{10mm}
   
   \thispagestyle{empty}

\noindent
\begin{tabular}{cc}
%\begin{center}
\multicolumn{1}{c}{\raisebox{-40pt}[0pt][0pt]{\mbox{%
\epsfxsize=33mm
\epsfbox{vspu.eps}
}}}
%\end{center}
&
\tabcolsep=0pt\begin{tabular}{c}
{\prg{\Large\textbf{XII Всероссийское совещание}}}\\[6pt]
{\prg{\Large\textbf{по проблемам управления}}}\\[12pt]
{\prg{\large 16--19 июня 2014~г.}}\\[6pt] 
{\prg{\large Институт проблем управления имени В.\,А.~Трапезникова РАН}}\\[6pt]
{\prg{\large Москва, Россия}}
\end{tabular}
\end{tabular}

\vspace*{60pt}

     
 { %\large    
 XII Всероссийское совещание по проблемам управления (ВСПУ XII), посвященное 75-летию 
Института проблем управления (ИПУ) имени В.\,А.~Трапезникова РАН, проводится 16--19~июня 
2014~г.\ 
в ИПУ РАН (г.~Москва, Россия). ВСПУ XII организуется ИПУ РАН при поддержке РФФИ, Отделения 
энергетики, машиностроения, механики и процессов управления Российской академии наук, 
Российского 
национального комитета по автоматическому управлению, Академии навигации и управ\-ле\-ния 
движением, 
Научного совета РАН по комплексным проблемам управления и автоматизации, Совета по 
мехатронике и робототехнике РАН. Официальный язык Совещания~--- русский.

\vspace*{24pt}
     
     \textbf{Направления работы}
     \begin{enumerate}[1.]
\item Теория систем управления
\item Управление подвижными объектами и навигация
\item Интеллектуальные системы управления
\item Управление в промышленности, транспортом и логистикой
\item Управление системами междисциплинарной природы
\item Средства измерения, вычислений и контроля в управлении
\item Системный анализ и принятие решений в задачах управления
\item Информационные технологии в управлении
\item Проблемы образования в области управления: современное содержание и технологии обучения
\end{enumerate}

\vspace*{24pt}

     Подробная информация о Совещании находится на сайте {\sf http://vspu2014.ipu.ru}. Срок 
окончательной подачи докладов через систему подачи докладов на сайте~--- \textbf{30~ноября} 
2013~г.
}

%\include{rekl-1}

%\end{document}

%\include{nekrolog-rb}


%\end{document}

%\include{IPPM-25}

\def\stat{cont-rus}
{%\hrule\par
%\vskip 7pt % 7pt
\vspace*{-24pt}
\raggedleft\Large \bf%\baselineskip=3.2ex
Правила подготовки рукописей  для публикации в журнале
<<Информатика~и~её~применения>> \vskip 8pt
    \hrule
    \par
\vskip 14pt plus 6pt minus 3pt }

\label{st\stat}

\def\tit{\ }

\def\aut{\ }
\def\auf{\ }

\def\leftkol{\ }
% Правила подготовки рукописей  для публикации в журнале
%<<Информатика и её применения>>

\def\rightkol{\ }
%Правила подготовки рукописей  для публикации в журнале
%<<Информатика и её применения>>}


\titele{\tit}{\aut}{\auf}{\leftkol}{\rightkol}


\vspace*{-60pt}
{ %\small

Журнал <<Информатика и её применения>>
публикует теоретические, обзорные и дискуссионные статьи,
посвященные научным исследованиям и разработкам в области
информатики и ее приложений.

Журнал издается на русском языке. По специальному решению
редколлегии отдельные статьи могут печататься на английском языке.

Тематика журнала охватывает следующие направления:
\begin{itemize}
\item теоретические основы информатики;\\[-15pt]
      \item
математические методы исследования сложных систем и процессов;\\[-15pt]
           \item
информационные системы и сети;\\[-15pt]
                \item
информационные технологии;\\[-15pt]
                     \item
архитектура и программное обеспечение вычислительных комплексов и сетей.\\[-15pt]
\end{itemize}


\noindent
\begin{enumerate}[1.]
\item В журнале печатаются статьи, содержащие результаты, ранее не опубликованные и
не предназначенные к одновременной публикации в других изданиях.

%Публикация не должна нарушать закон об авторских правах.
Публикация предоставленной автором(ами) рукописи не должна нарушать 
положений глав~69, 70 раздела~VII части~IV Гражданского кодекса, 
которые определяют права на результаты интеллектуальной деятельности 
и~средства индивидуализации, в~том числе авторские права, в~РФ.

Ответственность за нарушение авторских прав, в~случае предъявления претензий к~редакции журнала,  
несут авторы статей.



Направляя рукопись в редакцию, авторы сохраняют свои права на данную
рукопись и при этом передают учредителям и редколлегии журнала неисключительные права на
издание статьи на русском языке 
(или на языке статьи, если он отличен от рус\-ско\-го) и~на перевод ее на английский
язык, а~также на
ее распространение в России и за рубежом. 
Каждый автор должен представить в~редакцию подписанный 
с~его стороны <<Лицензионный договор о~передаче неисключительных прав 
на использование произведения>>, текст которого размещен по адресу 
{\sf http://www.ipiran.ru/publications/licence.doc}. 
Этот договор может быть пред\-став\-лен в~бумажном (в~2-х экз.)\ 
или в~электронном виде (отсканированная копия заполненного и~подписанного документа).




Редколлегия вправе запросить у авторов экспертное заключение о возможности
пуб\-ли\-ка\-ции пред\-став\-лен\-ной статьи в открытой печати.\\[-13.5pt]

\item К статье прилагаются данные автора (авторов) (см.\ п.~8). При наличии нескольких
авторов указывается фамилия автора, ответственного за переписку с редакцией.\\[-13.5pt]

\item Редакция журнала осуществляет экспертизу присланных статей в соответствии с
принятой в журнале процедурой рецензирования.

Возвращение рукописи на доработку не означает ее принятия к печати.

Доработанный вариант с ответом на замечания рецензента необходимо прислать в
редакцию.\\[-13.5pt]

\item Решение редколлегии о публикации статьи или ее отклонении сообщается авторам.

Редколлегия может также направить авторам текст рецензии на их статью. Дискуссия по
поводу отклоненных статей не ведется.\\[-13.5pt]

%\pagebreak

\item Редактура статей высылается авторам для просмотра. Замечания к редактуре должны
быть присланы авторами в кратчайшие сроки.\\[-13.5pt]

\item Рукопись предоставляется в электронном виде в форматах MS WORD (.doc или
.docx) или \LaTeX\  (.tex), дополнительно~--- в формате .pdf, на дискете, лазерном диске
или электронной почтой. Предоставление бумажной рукописи необязательно.\\[-13.5pt]

\item При подготовке рукописи в MS Word рекомендуется использовать следующие
настройки.

Параметры страницы:
формат~--- А4; ориентация~--- книжная; поля (см): внутри~--- 2,5, снаружи~--- 1,5,
сверху~--- 2, снизу~--- 2, от края до нижнего колонтитула~--- 1,3.

Основной текст: стиль~--- <<Обычный>>, шрифт~--- Times New Roman, размер~---
14~пунк\-тов, абзацный отступ~--- 0,5~см, 1,5~интервала, выравнивание~--- по ширине.

\pagebreak

\def\leftkol{Правила подготовки рукописей  для публикации в журнале
<<Информатика и её применения>>}

\def\rightkol{Правила подготовки рукописей  для публикации в журнале
<<Информатика и её применения>>}



Рекомендуемый объем рукописи~--- не свыше 10~страниц указанного формата.
При превышении указанного объема редколлегия вправе потребовать от 
автора сокращения объема рукописи.


Сокращения слов, помимо стандартных, не допускаются. Допускается минимальное
количество аббревиатур.


Все страницы рукописи нумеруются.

Шаблоны оформления представлены в интернете:

\noindent
 {\sf
http://www.ipiran.ru/journal/template\_iiep\_ssi\_2024.zip}\\[-14pt]

\item Статья должна содержать следующую информацию на {\bfseries\textit{русском и
английском языках}}:\\[-16pt]

\begin{itemize}
\item название статьи;\\[-15pt]
\item Ф.И.О.\ авторов, на английском можно только имя и фамилию;\\[-15pt]
\item место работы, с указанием почтового адреса организации и электронного адреса каждого
автора;\\[-15pt]
\item сведения об авторах, в соответствии с форматом, образцы которого
представлены на страницах:



\def\leftfootline{\small{\textbf{\thepage}
\hfill ИНФОРМАТИКА И ЕЁ ПРИМЕНЕНИЯ\ \ \ том\ 18\ \ \ выпуск\ 3\ \ \ 2024}
}%
 \def\rightfootline{\small{ИНФОРМАТИКА И ЕЁ ПРИМЕНЕНИЯ\ \ \ том\ 18\ \ \ выпуск\ 3\ \ \ 2024
\hfill \textbf{\thepage}}}



{\sf http://www.ipiran.ru/journal/issues/2013\_07\_01/authors.asp} и

{\sf http://www.ipiran.ru/journal/issues/2013\_07\_01\_eng/authors.asp};
\item аннотация (не менее 100~слов на каждом из языков). Аннотация~--- это краткое
резюме работы, которое может публиковаться отдельно. Она является основным
источником информации в~ин\-фор\-ма\-ци\-он\-ных системах и базах данных. Английская
аннотация должна быть оригинальной, может не быть дословным переводом русского
текста и должна быть написана хорошим английским языком. В~аннотации не должно
быть ссылок на литературу и, по возможности, формул;\\[-15pt]
\item ключевые слова~--- желательно из принятых в мировой
на\-уч\-но-тех\-ни\-че\-ской литературе тематических тезаурусов. Предложения не
могут быть ключевыми словами;\\[-15pt]
\item источники финансирования работы (ссылки на гранты, проекты,
поддерживающие организации и~т.\,п.).
\end{itemize}



%\pagebreak

\item  Требования к спискам литературы.\\[-14pt]

Ссылки на литературу в тексте статьи нумеруются (в квадратных скобках) и
располагаются в каждом из списков литературы в порядке  первых упоминаний. Если источник имеет DOI и/или EDN,
то их необходимо указывать.

Списки литературы представляются в двух вариантах:\\[-14pt]


\noindent
\begin{enumerate}[(1)]
\item \textbf{Список литературы к русскоязычной части}. Русские и английские
работы~---  на языке и в алфавите оригинала;\\[-14.5pt]
\item  \textbf{References}. Русские работы и работы на других языках~--- в латинской
транслитерации с переводом на английский язык; английские работы и работы на других
языках~--- на языке оригинала.
\end{enumerate}

Необходимо для составления списка ``References'' пользоваться размещенной на сайте
{\sf http://www. translit.net/ru/bgn/} бесплатной программой транслитерации русского
 текста в~латиницу. %, при этом в~за\-клад\-ке <<варианты\ldots>> следует выбратьопцию BGN.

Список литературы ``References'' приводится полностью отдельным блоком, повторяя все
позиции из списка литературы к русскоязычной части, независимо от того, имеются или
нет в нем иностранные источники. Если в списке литературы к русскоязычной части есть
ссылки на иностранные публикации, набранные латиницей, они полностью повторяются в
списке ``References''.

Ниже приведены примеры ссылок на различные виды публикаций в списке ``References''.

\def\leftfootline{\small{\textbf{\thepage}
\hfill ИНФОРМАТИКА И ЕЁ ПРИМЕНЕНИЯ\ \ \ том\ 18\ \ \ выпуск\ 3\ \ \ 2024}
}%
 \def\rightfootline{\small{ИНФОРМАТИКА И ЕЁ ПРИМЕНЕНИЯ\ \ \ том\ 18\ \ \ выпуск\ 3\ \ \ 2024
\hfill \textbf{\thepage}}}

{\small

\noindent
\textbf{Описание статьи из журнала:}

\Aue{Zagurenko, A.\,G., V.\,A.~Korotovskikh, A.\,A.~Kolesnikov, A.\,V.~Timonov, and D.\,V.~Kardymon}. 2008.
Tekhniko-ekonomicheskaya optimizatsiya dizayna gidrorazryva plasta [Technical and
economic optimization of the design
of hydraulic fracturing]. \textit{Neftyanoe hozyaystvo} [\textit{Oil Industry}] 11:54--57.

\Aue{Zhang, Z., and D.~Zhu}. 2008. Experimental research on the localized
electrochemical micromachining. \textit{Russ. J.~Electrochem.}  44(8):926--930.
{\sf doi:10.1134/S1023193508080077}.

\noindent
\textbf{Описание статьи из электронного журнала:}

\Aue{Swaminathan, V., E.~Lepkoswka-White, and B.\,P.~Rao}. 1999. Browsers or buyers in cyberspace? An
investigation of electronic factors influencing electronic exchange. \textit{JCMC}
5(2). Available at: {\sf http://www.ascusc.org/jcmc/vol5/issue2/} (accessed April~28, 2011).

\def\leftkol{Правила подготовки рукописей  для публикации в журнале
<<Информатика и её применения>>}

\def\rightkol{Правила подготовки рукописей  для публикации в журнале
<<Информатика и её применения>>}


\noindent
\textbf{Описание статьи из продолжающегося издания (сборника трудов):}

\Aue{Astakhov, M.\,V., and T.\,V.~Tagantsev}. 2006. Eksperimental'noe
issledovanie prochnosti soedineniy ``stal'--kompozit'' [Experimental study of
the strength of joints ``steel--composite'']. \textit{Trudy MGTU
``Matematicheskoe modelirovanie slozhnykh tekh\-ni\-che\-skikh sistem''}
[\textit{Bauman MSTU ``Mathematical Modeling of Complex Technical
Systems'' Proceedings}]. 593:125--130.


\pagebreak



\noindent
\textbf{Описание материалов конференций:}

\Aue{Usmanov, T.\,S., A.\,A.~Gusmanov, I.\,Z.~Mullagalin, R.\,Ju.~Muhametshina, A.\,N.~Chervyakova, and
A.\,V.~Sveshnikov}. 2007. Osobennosti proektirovaniya razrabotki mestorozhdeniy
s primeneniem gidrorazryva
plasta [Features of the design of field development with the use of hydraulic fracturing].
\textit{Trudy 6-go
Mezhdu\-na\-rod\-no\-go Simpoziuma ``Novye resursosberegayushchie tekhnologii nedropol'zovaniya i povysheniya
neftegazootdachi''} [\textit{6th  Symposium (International) ``New Energy Saving Subsoil Technologies and
the Increasing of the Oil and Gas Impact'' Proceedings}]. Moscow. 267--272.



\def\leftfootline{\small{\textbf{\thepage}
\hfill ИНФОРМАТИКА И ЕЁ ПРИМЕНЕНИЯ\ \ \ том\ 18\ \ \ выпуск\ 3\ \ \ 2024}
}%
 \def\rightfootline{\small{ИНФОРМАТИКА И ЕЁ ПРИМЕНЕНИЯ\ \ \ том\ 18\ \ \ выпуск\ 3\ \ \ 2024
\hfill \textbf{\thepage}}}



\noindent
\textbf{Описание книги (монографии, сборники):}



Lindorf, L.\,S., and L.\,G.~Mamikoniants, eds. 1972.
\textit{Ekspluatatsiya turbogeneratorov s neposredstvennym
okhlazhdeniem} [\textit{Operation of turbine generators with direct cooling}].
Moscow: Energy Publs. 352~p.


\Aue{Latyshev, V.\,N.} 2009. \textit{Tribologiya rezaniya. Kn.~1: Friktsionnye protsessy
pri rezanii metallov}
[\textit{Tribology of cutting. Vol.~1: Frictional processes in metal cutting}]. Ivanovo: Ivanovskii
State Univ. 108~p.

\def\leftkol{Правила подготовки рукописей  для публикации в журнале
<<Информатика и её применения>>}

\def\rightkol{Правила подготовки рукописей  для публикации в журнале
<<Информатика и её применения>>}

\noindent
\textbf{Описание переводной книги}
(в списке литературы к русскоязычной части необходимо указать:~/ Пер.\ с англ.~---
после названия книги, а в конце ссылки указать оригинал книги в круглых скобках):
\begin{enumerate}[1.]
\item  В русскоязычной части:

\def\leftfootline{\small{\textbf{\thepage}
\hfill ИНФОРМАТИКА И ЕЁ ПРИМЕНЕНИЯ\ \ \ том\ 18\ \ \ выпуск\ 3\ \ \ 2024}
}%
 \def\rightfootline{\small{ИНФОРМАТИКА И ЕЁ ПРИМЕНЕНИЯ\ \ \ том\ 18\ \ \ выпуск\ 3\ \ \ 2024
\hfill \textbf{\thepage}}}

\Au{Тимошенко С.\,П., Янг Д.\,Х., Уивер~У.}
Колебания в инженерном деле~/ Пер.\ с англ.~--- М.: Машиностроение, 1985. 472~с.
(\Au{Timoshenko~S.\,P., Young~D.\,H., Weaver~W.}
Vibration problems in engineering.~--- 4th ed.~--- New York, NY, USA: Wiley, 1974. 521~p.)\\[-13.5pt]
\item  В англоязычной части:

\Aue{Timoshenko, S.\,P., D.\,H.~Young, and W.~Weaver}.
1974. \textit{Vibration problems in engineering}. 4th ed. New York: 
Wiley. 521~p.
\end{enumerate}

\vspace*{-3pt}


\noindent
\textbf{Описание неопубликованного документа:}


\Aue{Latypov, A.\,R., M.\,M.~Khasanov, and V.\,A.~Baikov}.
2004 (unpubl.). Geologiya i~dobycha (NGT GiD) [Geology and production (NGT GiD)]. Certificate on official registration of the computer program
No.\,2004611198. 

\noindent
\textbf{Описание интернет-ресурса:}


Pravila tsitirovaniya istochnikov [Rules for the citing of sources]. Available at: {\sf
http://www.scribd.com/doc/1034528/} (accessed February~7, 2011).

%\pagebreak

\noindent
\textbf{Описание диссертации или автореферата диссертации:}

\Aue{Semenov, V.\,I.}
2003. Matematicheskoe modelirovanie plazmy v sisteme kompaktnyy tor [Mathematical
modeling of the plasma in the compact torus].  Moscow.  D.Sc.\ Diss. 272~p.

\Aue{Kozhunova, O.\,S.} 2009. Tekhnologiya razrabotki semanticheskogo
slovarya informatsionnogo monitoringa [Technology of development of
semantic dictionary of information monitoring system].  Moscow: IPI RAN. PhD Thesis. 23~p.


\noindent
\textbf{Описание ГОСТа:}

GOST 8.586.5-2005. 2007. Metodika vypolneniya izmereniy. Izmerenie raskhoda i~kolichestva zhidkostey i~gazov
s~pomoshch'yu standartnykh suzhayushchikh ustroystv [Method of measurement.
Measurement of flow rate and volume of liquids and gases by means of orifice devices]. Moscow:
Standardinform  Publs. 10~p.

\noindent
\textbf{Описание патента:}

\Aue{Bolshakov, M.\,V., A.\,V.~Kulakov, A.\,N.~Lavrenov, and M.\,V.~Palkin}.
2006. Sposob orientirovaniya po krenu letatel'nogo
apparata s opti\-che\-skoy golovkoy
samonavedeniya [The way to orient on the roll of aircraft with optical homing head].
Patent RF No.\,2280590.
}

\item Присланные в редакцию материалы авторам не возвращаются.\\[-13.5pt]

\item При отправке файлов по электронной почте просим придерживаться следующих
правил:
\begin{itemize}
\item указывать в поле subject (тема) название журнала и фамилию автора;\\[-13.5pt]
\item указывать в тексте письма название статьи, авторов и~журнал, в~который направляется статья;\\[-13.5pt]
\item использовать attach (присоединение);\\[-13.5pt]
\item в состав электронной версии статьи должны входить: файл, содержащий текст
статьи, и файл(ы), содержащий(е) иллюстрации.\\[-13.5pt]
\end{itemize}

\item Журнал <<Информатика и её применения>> является некоммерческим изданием.
Плата за публикацию не взимается, гонорар авторам не выплачивается.
\end{enumerate}



\def\leftfootline{\small{\textbf{\thepage}
\hfill ИНФОРМАТИКА И ЕЁ ПРИМЕНЕНИЯ\ \ \ том\ 18\ \ \ выпуск\ 3\ \ \ 2024}
}%
 \def\rightfootline{\small{ИНФОРМАТИКА И ЕЁ ПРИМЕНЕНИЯ\ \ \ том\ 18\ \ \ выпуск\ 3\ \ \ 2024
\hfill \textbf{\thepage}}}


\vspace*{-1mm}

\begin{center}

\textbf{Адрес редакции журнала <<Информатика и её применения>>:} \\




Москва 119333, ул.~Вавилова, д.~44, корп.~2, ФИЦ ИУ РАН\\[-10pt]

\

Тел.: +7\,(499)\,135-86-92\ \ Факс:  +7\,(495)\,930-45-05\\[-10pt]

 \

e-mail:   {\sf iiep@frccsc.ru} (Стригина Светлана Николаевна)\\[-10pt]

\

{\sf http://www.ipiran.ru/journal/issues/}
\end{center}
}


\def\leftkol{Правила подготовки рукописей  для публикации в журнале
<<Информатика и её применения>>}

\def\rightkol{Правила подготовки рукописей  для публикации в журнале
<<Информатика и её применения>>}


\def\leftfootline{\small{\textbf{\thepage}
\hfill ИНФОРМАТИКА И ЕЁ ПРИМЕНЕНИЯ\ \ \ том\ 18\ \ \ выпуск\ 3\ \ \ 2024}
}%
 \def\rightfootline{\small{ИНФОРМАТИКА И ЕЁ ПРИМЕНЕНИЯ\ \ \ том\ 18\ \ \ выпуск\ 3\ \ \ 2024
\hfill \textbf{\thepage}}} 
\def\stat{podg-e}
{%\hrule\par
%\vskip 7pt % 7pt
\vspace*{-24pt}
\raggedleft\Large \bf%\baselineskip=3.2ex
Requirements for manuscripts submitted to Journal
``Informatics~and~Applications'' \vskip 8pt
    \hrule
    \par
\vskip 21pt plus 6pt minus 3pt }

\label{st\stat}

\def\tit{\ }

\def\aut{\ }
\def\auf{\ }

\def\leftkol{\ }

\def\rightkol{\ }
%Requirements for manuscripts submitted to Journal
%``Informatics~and~Applications''}

\titele{\tit}{\aut}{\auf}{\leftkol}{\rightkol}

\def\leftfootline{\small{\textbf{\thepage}
\hfill INFORMATIKA I EE PRIMENENIYA~--- INFORMATICS AND APPLICATIONS\ \ \ 2019\
\ \ volume~13\ \ \ issue\ 4}
}%
 \def\rightfootline{\small{INFORMATIKA I EE PRIMENENIYA~--- INFORMATICS AND APPLICATIONS\ \ \ 2019\ \ \ volume~13\ \ \ issue\ 4
\hfill \textbf{\thepage}}}

\vspace*{-60pt}

{\small

\noindent
Journal ``Informatics and Applications'' (Inform.\ Appl.)
publishes theoretical, review, and discussion
articles on the research and development in the
field of informatics and its applications.

The journal is published in Russian.
By a special decision of the editorial
board, some articles can be published in English.


The topics covered include the following areas:
\begin{itemize}
               \item
     theoretical fundamentals of informatics; \\[-14pt]
\item
mathematical methods for studying complex systems and processes; \\[-14pt]
\item
information systems and networks;\\[-14pt]
\item
information technologies; and \\[-14pt]
\item
architecture and software of computational complexes and networks. \\[-14pt]
\end{itemize}

\noindent
\begin{enumerate}[1.]
\item The Journal publishes original articles which have not been published before and are not
intended for simultaneous publication in other editions. An article submitted to the Journal must not violate the
Copyright law. Sending the manuscript to the Editorial Board, the authors retain all rights of the
owners of the manuscript and transfer the nonexclusive rights to publish the article in Russian
(or the language of the article, if not Russian) and its distribution in Russia and abroad to the
Founders and the Editorial Board. Authors should submit a letter to the Editorial Board in the
following form:

{\bfseries\textit{Agreement on the transfer of rights to publish:}}

``\textit{We, the undersigned authors of the manuscript ``\ldots'', pass to the
Founder and the Editorial Board of the Journal ``Informatics and Applications''
the nonexclusive right to publish the manuscript of the article in Russian (or
in English) in both print and electronic versions of the Journal. We affirm
that this publication does not violate the Copyright of other persons or
organizations.}

\textit{Author(s) signature(s): (name(s), address(es), date).}

This agreement should be submitted in paper form or in the form of a scanned copy (signed by
the authors).


%The Editorial Board has the right to request from the authors an official expert conclusion that
%the submitted article has no secret data prohibited for publication. \\[-13.5pt]
\item
A submitted article should be attached with \textbf{the data on the author(s)} (see item~8). If
there are several authors, the contact person should be indicated who is responsible for
correspondence with the Editorial Board and other authors about revisions and final approval
of the proofs.\\[-13.5pt]

\item The Editorial Board of the Journal examines the article according to the established
reviewing procedure. If the authors receive their article for correction after reviewing, it does not
mean that the article is approved for publication. The corrected article should be sent to the
Editorial Board for the subsequent review and approval.\\[-13.5pt]

\item The decision on the article publication or its rejection is communicated to the authors. The
Editorial Board may also send the reviews on the submitted articles to the authors. Any
discussion upon the rejected articles is not possible.\\[-13.5pt]

\item The edited articles will be sent to the authors for proofread. The comments of the authors
to the edited text of the article should be sent to the Editorial Board as soon as possible.\\[-13.5pt]

\item The manuscript of the article should be presented electronically in the MS WORD (.doc or
.docx) or \LaTeX\ (.tex) formats, and additionally in the .pdf format. All documents
 may be sent
by e-mail or provided on a CD or diskette. A~hard copy submission is not necessary.\\[-13.5pt]

\item The recommended typesetting instructions for manuscript.

Pages parameters: format A4, portrait orientation, document margins (cm): left~--- 2.5, right~---
1.5, above~--- 2.0, below~--- 2.0, footer 1.3.

Text: font~---Times New Roman, font size~--- 14, paragraph indent~--- 0.5, line spacing~--- 1.5,
justified alignment.

The recommended manuscript size: not more than 15~pages of the specified format.
If the specified size exceeded, the editorial board is entitled to require the author
to reduce the manuscript.

Use only standard abbreviations. Avoid  abbreviations in the title and
abstract. The full term for which an abbreviation stands should precede
its first use in the text unless it is a standard unit of measurement.

All pages of the manuscript should be numbered.

The templates for the manuscript typesetting are presented on site: {\sf
http://www.ipiran.ru/journal/template.doc}.\\[-13.5pt]


%\def\leftkol{Requirements for manuscripts submitted to Journal
%``Informatics~and~Applications''}

\item The articles should enclose data both in \textbf{Russian and English}:
\begin{itemize}
\item title;\\[-13.5pt]
\item author's name and surname;\\[-13.5pt]
\item affiliation~--- organization, its address with ZIP code, city, country, and
official e-mail address;\\[-13.5pt]
\item data on authors according to the format: (see site)

{\sf http://www.ipiran.ru/journal/issues/2013\_07\_01/authors.asp}  and

{\sf  http://www.ipiran.ru/journal/issues/2013\_07\_01\_eng/authors.asp};\\[-13.5pt]

\pagebreak

\def\leftfootline{\small{\textbf{\thepage}
\hfill INFORMATIKA I EE PRIMENENIYA~--- INFORMATICS AND APPLICATIONS\ \ \ 2019\
\ \ volume~13\ \ \ issue\ 4}
}%
 \def\rightfootline{\small{INFORMATIKA I EE PRIMENENIYA~--- INFORMATICS AND APPLICATIONS\ \ \ 2019\ \ \ volume~13\ \ \ issue\ 4
\hfill \textbf{\thepage}}}


%\def\leftkol{Requirements for manuscripts submitted to Journal
%``Informatics~and~Applications''}

%\def\rightkol{Requirements for manuscripts submitted to Journal
%``Informatics~and~Applications''}



\item abstract (not less than 100 words) both in Russian and in English. Abstract is a short
summary of the article that can be published separately. The abstract is the
main source of information on the article and it could be included in leading information
systems and data bases. The abstract in English has to be an original text and should
not be an exact translation of the Russian one. Good English is required.
In abstracts, avoid references and formulae;\\[-13.5pt]
\item indexing is performed on the basis of keywords. The use of keywords from the
internationally accepted thematic Thesauri is recommended.

%\def\leftkol{Requirements for manuscripts submitted to Journal
%``Informatics~and~Applications''}

%\def\rightkol{Requirements for manuscripts submitted to Journal
%``Informatics~and~Applications''}

Important! Keywords must not be sentences;
\item Acknowledgments.
\end{itemize}

\item References. Russian references have to be presented both in English translation and Latin
transliteration (refer {\sf http://www.translit.net/ru/bgn/}).

Please take into account the following examples of Russian references appearance:

\noindent
\textbf{Article in journal:}

\Aue{Zhang, Z., and D.~Zhu}. 2008. Experimental research on the localized electrochemical
micromachining.
\textit{Rus. J.~Electrochem.}  44(8):926--930. {\sf doi:10.1134/S1023193508080077}.


\noindent
\textbf{Journal article in electronic format:}

\Aue{Swaminathan, V., E.~Lepkoswka-White, and B.\,P.~Rao}. 1999. Browsers or buyers in
cyberspace? An
investigation of electronic factors influencing electronic exchange. \textit{JCMC}
5(2). Available at: {\sf http://www.ascusc.org/jcmc/vol5/issue2/} (accessed April~28, 2011).




\noindent
\textbf{Article from the continuing publication (collection of works, proceedings):}

\Aue{Astakhov, M.\,V., and T.\,V.~Tagantsev}. 2006. Eksperimental'noe
issledovanie prochnosti soedineniy ``stal'--kompozit'' [Experimental study of
the strength of joints ``steel--composite'']. \textit{Trudy MGTU
``Matematicheskoe modelirovanie slozhnykh tekh\-ni\-che\-skikh sistem''}
[\textit{Bauman MSTU ``Mathematical Modeling of Complex Technical
Systems'' Proceedings}]. 593:125--130.

\def\leftfootline{\small{\textbf{\thepage}
\hfill INFORMATIKA I EE PRIMENENIYA~--- INFORMATICS AND APPLICATIONS\ \ \ 2019\
\ \ volume~13\ \ \ issue\ 4}
}%
 \def\rightfootline{\small{INFORMATIKA I EE PRIMENENIYA~--- INFORMATICS AND APPLICATIONS\ \ \ 2019\ \ \ volume~13\ \ \ issue\ 4
\hfill \textbf{\thepage}}}

\def\leftkol{Requirements for manuscripts submitted to Journal
``Informatics~and~Applications''}

\def\rightkol{Requirements for manuscripts submitted to Journal
``Informatics~and~Applications''}

\noindent
\textbf{Conference proceedings:}

\Aue{Usmanov, T.\,S., A.\,A.~Gusmanov, I.\,Z.~Mullagalin, R.\,Ju.~Muhametshina,
A.\,N.~Chervyakova, and
A.\,V.~Sveshnikov}. 2007. Osobennosti proektirovaniya razrabotki mestorozhdeniy
s primeneniem gidrorazryva
plasta [Features of the design of field development with the use of hydraulic fracturing].
\textit{Trudy 6-go
Mezhdu\-na\-rod\-no\-go Simpoziuma ``Novye resursosberegayushchie tekhnologii
nedropol'zovaniya i povysheniya
neftegazootdachi''} [\textit{6th  Symposium (International) ``New Energy Saving Subsoil
Technologies and
the Increasing of the Oil and Gas Impact'' Proceedings}]. Moscow. 267--272.


\noindent
\textbf{Books and other monographs:}




Lindorf, L.\,S., and L.\,G.~Mamikoniants, eds. 1972.
\textit{Ekspluatatsiya turbogeneratorov s neposredstvennym
okhlazhdeniem} [\textit{Operation of turbine generators with direct cooling}].
Moscow: Energy Publs. 352~p.


%\Aue{Latyshev, V.\,N.} 2009. \textit{Tribologiya rezaniya. Kn.~1: Frikcionnye prosessy
%pri rezanii metallov}
%[\textit{Tribology of cutting. Vol.~1: Frictional processes in metal cutting}]. Ivanovo: Ivanovskii
%State Univ. 108~p.


%\noindent
%\textbf{Unpublished material:}

%\Aue{Latypov, A.\,R., M.\,M.~Khasanov, and V.\,A.~Baikov}.
%2004. Geology and production (NGT GiD). Certificate on official registration of the computer
%program
%No.\,2004611198. (In Russian, unpubl.)

%\noindent
%\textbf{Internet-source:}

%APA Style. 2011. Available at: {\sf http://www.apastyle.org/apa-style-help.aspx} (accessed
%February~5, 2011).

%Pravila citirovaniya istochnikov [Rules for the citing of sources]. Available at: {\sf
%http://www.scribd.com/doc/1034528/} (accessed February~7, 2011).


\noindent
\textbf{Dissertation and Thesis:}

%\Aue{Semenov, V.\,I.}
%2003. Matematicheskoe modelirovanie plazmy v sisteme kompaktnyy tor. [Mathematical
%modeling of the plasma in the compact torus]. D.Sc.\ Diss. Moscow. 272~p.

\Aue{Kozhunova, O.\,S.} 2009. Tekhnologiya razrabotki semanticheskogo
slovarya informatsionnogo monitoringa [Technology of development of
semantic dictionary of information monitoring system]. PhD Thesis. Moscow: IPI RAN. 23~p.


\noindent
\textbf{State standards and patents:}

GOST 8.586.5-2005. 2007. Metodika vypolneniya izmereniy. Izmerenie raskhoda i~kolichestva
zhidkostey i gazov 
s~pomoshch'yu standartnykh suzhayushchikh ustroystv [Method of measurement.
Measurement of flow rate and volume of liquids and gases by means of orifice devices]. M.:
Standardinform
Publs. 10~p.

%\noindent
%\textbf{Patent:}

\Aue{Bolshakov, M.\,V., A.\,V.~Kulakov, A.\,N.~Lavrenov, and M.\,V.~Palkin}.
2006. Sposob orientirovaniya po krenu letatel'nogo
apparata s opti\-che\-skoy golovkoy
samonavedeniya [The way to orient on the roll of aircraft with optical homing head].
Patent RF No.\,2280590.

References in Latin transcription are presented in the original language.

References in the text are numbered according to the order of their
first appearance; the number is
placed in square brackets. All items from the reference list should be
cited.\\[-13.5pt]

\item Manuscripts and additional materials are not returned to Authors by the Editorial Board.\\[-13.5pt]

\item Submissions of files by e-mail must include:\\[-13.5pt]
\begin{itemize}
\item   the journal title and author's name in the ``Subject'' field; \\[-13.5pt]
\item   an article and additional materials have to be attached using the ``attach'' function;\\[-13.5pt]
\item   an electronic version of the article should contain the file with the text and a separate file
with figures.\\[-13.5pt]
\end{itemize}

\item ``Informatics and Applications'' journal is not a profit publication. There are no
charges for the authors as well as there are no royalties.\\[-13.5pt]
\end{enumerate}

\def\leftfootline{\small{\textbf{\thepage}
\hfill INFORMATIKA I EE PRIMENENIYA~--- INFORMATICS AND APPLICATIONS\ \ \ 2019\
\ \ volume~13\ \ \ issue\ 4}
}%
 \def\rightfootline{\small{INFORMATIKA I EE PRIMENENIYA~--- INFORMATICS AND APPLICATIONS\ \ \ 2019\ \ \ volume~13\ \ \ issue\ 4
\hfill \textbf{\thepage}}}

\def\leftkol{Requirements for manuscripts submitted to Journal
``Informatics~and~Applications''}

\def\rightkol{Requirements for manuscripts submitted to Journal
``Informatics~and~Applications''}


%\vspace*{5mm}


\begin{center}
\textbf{Editorial Board address:} \\

%ABOUT AUTHORS



FRC CSC RAS, 44, block~2, Vavilov Str., Moscow 119333, Russia\\[-10pt]

\

Ph.: +7\,(499)\,135\,86\,92,\ \ Fax: +7\,(495)\,930\,45\,05\\[-10pt]

\

 e-mail: {\sf rust@ipiran.ru} (to Prof.\ Rustem Seyful-Mulyukov)\\[-10pt]

\

 {\sf http://www.ipiran.ru/english/journal.asp}
\end{center}
 }
%\thispagestyle{myheadings}

\def\leftkol{Requirements for manuscripts submitted to Journal
``Informatics~and~Applications''}

\def\rightkol{Requirements for manuscripts submitted to Journal
``Informatics~and~Applications''}

\def\leftfootline{\small{\textbf{\thepage}
\hfill INFORMATIKA I EE PRIMENENIYA~--- INFORMATICS AND APPLICATIONS\ \ \ 2019\
\ \ volume~13\ \ \ issue\ 4}
}%
 \def\rightfootline{\small{INFORMATIKA I EE PRIMENENIYA~--- INFORMATICS AND APPLICATIONS\ \ \ 2019\ \ \ volume~13\ \ \ issue\ 4
\hfill \textbf{\thepage}}}

 \label{end\stat}

\newpage

%\vspace*{-60pt} {\small
{\baselineskip=9.1pt
\section*{Правила подготовки рукописей статей для публикации в журнале
<<Информатика и её применения>>}

\thispagestyle{empty}

 Журнал <<Информатика и её применения>> публикует
теоретические, обзорные и дискуссионные статьи, посвященные научным
исследованиям и разработкам в области информатики и ее приложений. Журнал
издается на русском языке. По специальному решению редколлегии отдельные статьи,
в виде исключения, могут печататься на английском языке.
Тематика журнала охватывает следующие направления:
\begin{itemize}
\item теоретические основы информатики; %\\[-13.5pt]
\item математические методы исследования сложных систем и процессов; %\\[-13.5pt]
\item информационные системы и сети; %\\[-13.5pt]
\item информационные технологии; %\\[-13.5pt]
\item архитектура и программное
обеспечение вычислительных комплексов и сетей.
\end{itemize}
\begin{enumerate}
\item В журнале печатаются результаты, ранее не
опубликованные и не предназначенные к одновременной публикации в других
изданиях. Публикация не должна нарушать закон об авторских правах. Направляя
свою рукопись в редакцию, авторы автоматически передают учредителям и
редколлегии неисключительные права на издание данной статьи на русском языке и
на ее распространение в России и за рубежом. При этом за авторами сохраняются
все права как собственников данной рукописи. В связи с этим авторами должно
быть представлено в редакцию письмо в следующей форме:
Соглашение о передаче права на публикацию:

\textit{<<Мы, нижеподписавшиеся, авторы рукописи <<$\qquad\qquad$>>, передаем
учредителям и редколлегии журнала <<Информатика и её применения>>
неисключительное право опубликовать данную рукопись статьи на русском языке как
в печатной, так и в электронной версиях журнала. Мы подтверждаем, что данная
публикация не нарушает авторского права других лиц или организаций. Подписи
авторов: (ф.\,и.\,о., дата, адрес)>>.}

Указанное соглашение может быть представлено 
как в бумажном виде, так и в виде отсканированной копии (с подписями авторов).


Редколлегия вправе запросить у авторов экспертное заключение о возможности
опубликования представленной статьи в открытой печати. %\\[-13.5pt]
\item Статья
подписывается всеми авторами. На отдельном листе представляются данные автора
(или всех авторов): фамилия, полные имя и отчество, телефон, факс, e-mail,
почтовый адрес. Если работа выполнена несколькими авторами, указывается фамилия
одного из них, ответственного за переписку с редакцией. %\\[-13.5pt]
\item Редакция журнала
осуществляет самостоятельную экспертизу присланных статей. Возвращение рукописи
на доработку не означает, что статья уже принята к печати. Доработанный вариант
с ответом на замечания рецензента необходимо прислать в редакцию. %\\[-13.5pt]
\item Решение
редакционной коллегии о принятии статьи к печати или ее отклонении сообщается
авторам. Редколлегия не обязуется направлять рецензию авторам отклоненной
статьи. %\\[-13.5pt]
\item Корректура статей высылается авторам для просмотра. Редакция
просит авторов присылать свои замечания в кратчайшие сроки. %\\[-13.5pt]
\item При
подготовке рукописи в MS Word рекомендуется использовать следующие настройки.
Параметры страницы: формат~--- А4; ориентация~--- книжная; поля (см): внутри~---
2,5, снаружи~--- 1,5, сверху~--- 2, снизу~--- 2, от края до нижнего
колонтитула~--- 1,3. Основной текст: стиль~--- <<Обычный>>: шрифт Times New
Roman, размер 14~пунктов, абзацный отступ~--- 0,5~см, 1,5 интервала,
выравнивание~--- по ширине. Рекомендуемый объем рукописи~--- не свыше
25~страниц указанного формата. Ознакомиться с шаблонами, содержащими примеры
оформления, можно по адресу в Интернете:
\textsf{http://www.ipiran.ru/journal/template.doc}.
\item К рукописи, предоставляемой в 2-х
экземплярах, обязательно прилагается электронная версия статьи (как правило, в
форматах MS WORD (.doc) или \LaTeX\ (.tex), а также~--- дополнительно~--- в
формате .pdf) на дискете, лазерном диске или по электронной почте. Сокращения
слов, кроме стандартных, не применяются. Все страницы рукописи должны быть
пронумерованы. %\\[-13.5pt]
\item Статья должна содержать следующую информацию на русском и
английском языках: название, Ф.И.О. авторов, места работы авторов и их
электронные адреса, подробные сведения об авторах, оформленные в соответствии с форматом, 
определяемым файлами {\sf http://www.ipiran.ru/journal/issues/2011\_05\_01/authors.asp} и 
{\sf http://www.ipiran.ru/journal/issues/2011\_01\_eng/authors.asp},
аннотация (не более 100~слов), ключевые слова. Ссылки на
литературу в тексте статьи нумеруются (в квадратных скобках) и располагаются в
порядке их первого упоминания. В~списке литературы не должно быть позиций, на которые нет ссылки в тексте статьи.
Все фамилии авторов, заглавия статей, названия
книг, конференций и~т.\,п.\ даются на языке оригинала, если этот язык
использует кириллический или латинский алфавит. %\\[-13.5pt]
\item Присланные в редакцию материалы авторам не возвращаются.
\item При отправке файлов по электронной
почте просим придерживаться следующих правил:
\begin{itemize}
\item указывать в поле subject (тема) название журнала и фамилию автора; %\\[-13.5pt]
\item использовать attach (присоединение); %\\[-13.5pt]
\item в случае больших объемов информации возможно
использование общеизвестных архиваторов (ZIP, RAR); %\\[-13.5pt]
\item в состав электронной версии статьи должны входить: файл, содержащий текст статьи, и файл(ы),
содержащий(е) иллюстрации. %\\[-13.5pt]
\end{itemize}
\item Журнал <<Информатика и её применения>> является некоммерческим изданием. 
Плата за публикацию с авторов не взимается, гонорар авторам не выплачивается.
\end{enumerate}
\thispagestyle{empty}
\textbf{Адрес редакции:} Москва 119333,
ул.~Вавилова, д.~44, корп.~2, ИПИ РАН\\
\hphantom{\textbf{Адрес редакции:} }Тел.: +7 (499) 135-86-92\ \
Факс:  +7 (495) 930-45-05\ \  E-mail:   rust@ipiran.ru }
}

%\include{ipi-ind}

%\tableofcontents

\end{document}

%\tableofcontents

%\end{document}

%\tableofcontents


\end{document}

\newcommand{\Ack}{\subsection*{\protect\large\bf Acknowledgments}}

\vphantom{\int\limits_0^T }

{ \begin{center}  %fig1
 \vspace*{3pt}
    \mbox{%
 \epsfxsize=79mm 
 \epsfbox{gru-1.eps}
 }

\end{center}

\noindent
{{\figurename~1}\ \ \small{
Временные зависимости данные 
}}}

\vspace*{6pt}

\setcounter{figure}{1}

$\acute{\mbox{о}}$

\linebreak