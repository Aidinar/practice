\newcommand{\Variance}{{\sf D}}
\newcommand{\Prob}{{\sf P}}


\def\stat{shestakov}

\def\tit{АНАЛИЗ НЕСМЕЩЕННОЙ ОЦЕНКИ СРЕДНЕКВАДРАТИЧНОГО РИСКА
МЕТОДА БЛОЧНОЙ ПОРОГОВОЙ ОБРАБОТКИ$^*$}

\def\titkol{Анализ несмещенной оценки среднеквадратичного риска
метода блочной пороговой обработки}

\def\aut{О.\,В.~Шестаков$^1$}

\def\autkol{О.\,В.~Шестаков}

\titel{\tit}{\aut}{\autkol}{\titkol}

\index{Шестаков О.\,В.}
\index{Shestakov O.\,V.}

{\renewcommand{\thefootnote}{\fnsymbol{footnote}} \footnotetext[1]
{Работа выполнена при финансовой поддержке РФФИ (проект 19-07-00352)
и~в~соответствии с~программой Московского цент\-ра фундаментальной и~прикладной математики.}}


\renewcommand{\thefootnote}{\arabic{footnote}}
\footnotetext[1]{Московский государственный университет имени М.\,В.~Ломоносова, 
кафедра математической статистики факультета вы\-чис\-ли\-тель\-ной математики и~кибернетики; 
Институт проб\-лем информатики Федерального исследовательского центра <<Информатика и~управ\-ле\-ние>> 
Российской академии наук, \mbox{oshestakov@cs.msu.ru}}


\vspace*{-12pt}




\Abst{Методы обработки сигналов и~изображений, основанные на вейв\-лет-раз\-ло\-же\-нии и~пороговой обработке, 
приобрели большую популярность при решении задач подавления шума и~компрессии. 
Это объясняется их адаптивностью к~локальным особенностям исследуемых функций, 
высокой скоростью алгоритмов обработки и~оптимальностью получаемых оценок. 
В~данной работе рассмотрен метод блочной пороговой обработки, в~котором коэффициенты 
разложения обрабатываются группами, что позволяет учитывать информацию о~соседних коэффициентах. 
В~модели с~аддитивным шумом проведен анализ несмещенной оценки среднеквадратичного риска и~показано, 
что при определенных условиях регулярности эта оценка является сильно состоятельной и~асимптотически 
нормальной. Данные свойства позволяют использовать оценку риска в~качестве критерия качества 
метода и~строить асимптотические доверительные интервалы для теоретического среднеквадратичного риска.}

\KW{вейвлеты; блочная пороговая обработка; несмещенная оценка риска; асимптотическая нормальность; 
сильная состоятельность}

\DOI{10.14357/19922264210205}

%\vspace*{-3pt}


\vskip 10pt plus 9pt minus 6pt

\thispagestyle{headings}

\begin{multicols}{2}

\label{st\stat}

\section{Введение}

Методы вейвлет-анализа успешно применяются в~задачах непараметрического оценивания функций 
и~задачах обработки и~анализа сигналов\linebreak и~изоб\-ра\-же\-ний. Они приобрели свою популярность благодаря 
адаптивности, вычислительной\linebreak эффективности и~асимптотической оп\-ти\-маль\-ности. Стандартные 
вейв\-лет-ме\-то\-ды используют процедуры пороговой обработки, применяемые отдель\-но к~каж\-до\-му коэффициенту 
вейв\-лет-раз\-ло\-же\-ния. Коэффициент сравнивается с~пороговым значением, и~если его абсолютная 
величина оказывается меньше этого значения, то он обнуляется. Самыми популярными являются 
процедуры жесткой и~мягкой пороговой обработки. Однако они имеют свои недостатки и~часто 
не достигают оп\-тимальных результатов. В~част\-ности, хорошо известный метод VisuShrink~\cite{DonJ94} 
приводит к~слишком сглаженным оценкам функции. В~работе~\cite{HKP99} рассмотрен метод блочной пороговой 
обработки, при котором коэффициенты обрабатываются не отдельно, а группами. Цель 
такого подхода заключается в~использовании информации о~соседних коэффициентах. 
Получаемые оценки имеют оптимальный (в~минимаксном смысле) порядок среднеквадратичного
 риска для различных классов функций~\cite{Cai99}.

Для практического анализа погрешности данного метода можно использовать 
несмещенную оценку среднеквадратичного риска~\cite{St81}, которая зависит только от наблюдаемых 
данных и~дает возможность оценивать качество обработанного сигнала без использования <<эталонных функций>>. 
В~данной работе исследуются статистические свойства этой оценки и~показывается, что 
для довольно широкого класса пространств Бесова она является асимптотически нормальной и~сильно 
состоятельной. Для методов пороговой обработки отдельных коэффициентов подобные исследования 
проводились в~работах~\cite{SH12, SH16-1, SH16-2, PSH19} в~предположении о~принадлежности
 исследуемой функции сигнала классу равномерно регулярных по Липшицу функций.

\section{Метод блочной пороговой обработки}

Для функции сигнала $f\hm\in L^2(\mathbb{R})$ разложение по вейв\-лет-ба\-зи\-су имеет вид:
\begin{equation}
f=\sum\limits_{j,k\in Z}\langle f,\psi_{j,k}\rangle\psi_{j,k},
\label{Wavelet_Decomp}
\end{equation}
где $\psi_{j,k}(t)\hm=2^{j/2}\psi(2^jt-k)$, а $\psi(t)$~--- 
некоторая вейв\-лет-функ\-ция (семейство $\{\psi_{j,k}\}_{j,k\in Z}$ образует 
ортонормированный базис в~$L^2(\mathbb{R})$). Индекс~$j$ в~\eqref{Wavelet_Decomp} 
называется масштабом, а~индекс~$k$~--- сдвигом. Преимущество разложения~\eqref{Wavelet_Decomp} 
заключается в~<<экономном>> представлении функций, т.\,е.\ 
для довольно широкого класса функций лишь относительно небольшое число коэффициентов 
в~\eqref{Wavelet_Decomp} заметно отлично от нуля.

Если вейвлет-функ\-ция~$\psi$ имеет~$r$ непрерывных производных и~$r$~нулевых моментов, 
определим при $0\hm<\gamma\hm<r$ и~$1\hm\leqslant p,q\hm\leqslant\infty$ 
полунорму последовательности вейв\-лет-ко\-эф\-фи\-ци\-ен\-тов выражением
$$
\abs{f}_{B^{\gamma}_{p,q}}=\left(\sum\limits_{j=0}^{\infty}\left(2^{sj}\left(
\sum\limits_{k}\abs{\langle f,\psi_{j,k}\rangle}^p\right)^{1/p}\right)^q\right)^{1/q},
$$
где 
$$
s=\gamma+\fr{1}{2}-\fr{1}{p}\,.
$$

 Далее будем считать, что~$f$ задана на конечном отрезке и~принадлежит
 пространству Бесова $B^{\gamma}_{p,q}(A)$ ($A\hm>0$), т.\,е.\ $\abs{f}_{{B}^{\gamma}_{p,q}}\hm\leqslant A$.

Модель данных, рассматриваемая в~данной работе, предполагает, что функция сигнала задана в~дискретных 
отсчетах и~наблюдения содержат шум:
\begin{equation*}
X_i = f_i + \epsilon_i, \quad i = 1, \dots, 2^J,
%\label{Data_Model}
\end{equation*}
где $2^J$~--- число отсчетов функции сигнала; $f_i$~--- 
незашумленные значения функции сигнала; $\epsilon_i$~--- 
независимые нормально распределенные случайные величины с~нулевым средним и~дисперсией~$\sigma^2$.
После применения дискретного вейв\-лет-пре\-обра\-зо\-ва\-ния получается следующая модель зашумленных 
вейв\-лет-ко\-эф\-фи\-ци\-ен\-тов:
\begin{equation*}
Y_{j,k}=\mu_{j,k}+\epsilon^W_{j,k},\enskip j=0,\ldots,J-1\,,\ k=0,\ldots,2^{j}-1\,,
\end{equation*}
где $\epsilon^W_{j,k}$ независимы и~имеют такое же распределение, как и~$\epsilon_i$, 
а~$\mu_{j,k}\hm\approx 2^{J/2}\langle f,\psi_{j,k}\rangle$.

Для подавления шума коэффициенты $Y_{j,k}$ подвергаются некоторой обработке. 
Самыми распространенными стали методы жесткой и~мягкой пороговой обработки и~их 
модификации~\cite{DonJ94, DonJ95, DonJ98, G98, PK05, LC10, HL10, HX15, ZC15}. 
При использовании этих методов происходит сравнение абсолютной величины каждого 
коэффициента с~некоторым порогом, и~если это значение оказывается меньше порога, то 
коэффициент считается шумом и~обнуляется. Такие методы обрабатывают каждый коэффициент 
отдельно, не используя информацию о других коэффициентах. Блочная пороговая обработка 
применяется к~группам соседних коэффициентов, т.\,е.\ решение об обнулении принимается сразу 
по всем коэффициентам из группы.

Пусть $B_{j,1},\ldots,B_{j,M_j}$~--- разбиение множества индексов $\{0,\ldots,2^j-1\}$ 
на блоки одинаковой длины~$L$ (для удобства предположим, что~$2^j$ делится на~$L$). Пусть 
$S^2_{j,m}\hm=\sum\nolimits_{k\in B_{j,m}}Y^2_{j,k}$. 
Оценки коэффициентов~$\mu_{j,k}$ вычисляются по правилу
\begin{equation*}
\widehat{\mu}_{j,k}=\left(1-\fr{TL\sigma^2}{S^2_{j,m}}\right)_+Y_{j,k},\ 
 j=0,\ldots,J-1,\ k\in B_m,
\end{equation*}
т.\,е.\ если величина $\sum\nolimits_{k\in B_{j,m}}Y^2_{j,k}$ меньше порога~$TL\sigma^2$, 
то все коэффициенты в~рассматриваемом блоке обнуляются.

На качество оценок, получаемых с~помощью блочной пороговой обработки, 
естественно влияют размер блока~$L$ и~значение порога~$T$. В~работе~\cite{Cai99} 
показано, что при $L\hm=\log 2^J$ достигается баланс между локальной и~глобальной 
адаптивностью метода блочной пороговой обработки, и~если при этом $T^*\hm\approx4{,}50524$ 
($T^*$ является корнем уравнения $T\hm-\log  T \hm-3\hm=0$), то среднеквадратичный риск 
оказывается (почти) оптимальным (в~минимаксном смысле). 
В~данной работе рассматриваются именно такие значения~$L$ и~$T$.


\section{Несмещенная оценка среднеквадратичного риска и~ее~свойства}

Среднеквадратичный риск описанного выше метода определяется по формуле:
\begin{equation*}
%\label{Risk}
R_J(T)=\sum\limits_{j=0}^{J-1}\sum\limits_{k=0}^{2^j-1}{\sf E}\left(
\widehat{\mu}_{j,k}-\mu_{j,k}\right)^2.
\end{equation*}
Вычислить его значение на практике нельзя, поскольку~$R_J(T)$ зависит от 
ненаблюдаемых <<чистых>> коэффициентов~$\mu_{j,k}$.

В~\cite{Cai99} показано, что
\begin{equation*}
\sum\limits_{j=0}^{J-1}\sum\limits_{k=0}^{2^j-1}{\sf E}\left(\widehat{\mu}_{j,k}-
\mu_{j,k}\right)^2=\sum\limits_{j=0}^{J-1}\sum\limits_{m=1}^{M_j}{\sf E} F_m[T,L],
\end{equation*}
где
\begin{multline*}
F_m(T,L)=\left[
%\vphantom{\fr{T^2L^2\sigma^4-2TL\sigma^4(L-2)}{S_{j,m}^2}}
L\sigma^2+{}\right.\\
{}+\fr{T^2L^2\sigma^4-2TL\sigma^4(L-2)}{S_{j,m}^2}
\Ik(S^2_{j,m}>TL\sigma^2)+{}\\
\left.{}+\left(S^2_{j,m}-2L\sigma^2\right)\Ik
\left(S^2_{j,m}\leqslant TL\sigma^2\right)\right].
\end{multline*}
Таким образом, величина
\begin{equation}
\label{Risk_Estimate}
\widehat{R}_J(T)=\sum\limits_{j=0}^{J-1}\sum\limits_{m=1}^{M_j} F_m(T,L)
\end{equation}
служит несмещенной оценкой~$R_J(T)$, не зависящей от~$\mu_{j,k}$.


Покажем, что оценка~(\ref{Risk_Estimate}) является асимптотически нормальной.

\smallskip

\noindent
\textbf{Теорема 1.}\ \textit{Пусть $f\in \emph{B}^{\gamma}_{p,q}(A)$ и~задана на конечном отрезке.
 Пусть вейв\-лет-функ\-ция~$\psi$ имеет~$r$ непрерывных производных и~$r$ нулевых моментов, 
 $r\hm>\gamma$. Если $1\hm\leqslant p,q\hm\leqslant\infty$ и~$\gamma\hm>\max(0,1/p-1/2)$, то при} $J\hm\to\infty$
\begin{equation}\label{Normality}
{\sf P}\left(\fr{\widehat{R}_J(T^*)-R_J(T^*)}{\sigma^2\sqrt{2^{J+1}}}<x\right)\to\Phi(x),
\end{equation}
\textit{где $\Phi(x)$~--- функция распределения стандартного нормального закона}.

\smallskip

\noindent
Д\,о\,к\,а\,з\,а\,т\,е\,л\,ь\,с\,т\,в\,о\,.\ \
Выберем $0\hm<d\hm<1$ и~запишем:
\begin{multline}
\label{Two_Sums}
\widehat{R}_J(T^*)-R_J(T^*)={}\\
{}=\sum\limits_{j=0}^{[dJ]}\sum\limits_{m=1}^{M_j} 
\left[F_m(T^*,L)-{\sf E} F_m(T^*,L)\right]+{}\\
{}+
\sum\limits_{j=[dJ]+1}^{J-1}\sum\limits_{m=1}^{M_j} \left[F_m(T^*,L)-{\sf E} F_m(T^*,L)\right].
\end{multline} 
 Число слагаемых в~первой сумме не превосходит~$2^{[dJ]+1}$. Кроме того, существует такая константа $C_F
 \hm>0$, что
\begin{equation}
\label{Bounds}
\abs{F_m(T^*,L)-{\sf E} F_m\left(T^*,L\right)}\leqslant C_FT^*L\;\;\mbox{п.\,в.} % см. формулу 9.3 в~статье Cai
\end{equation}
Применяя неравенство Хеффдинга, получаем, что для любого $\delta\hm>0$ найдется константа $C_\delta
\hm>0$ такая, что
\begin{multline}
\label{Hoeff}
{\sf P}\left(\left\vert \sum\limits_{j=0}^{[dJ]}\sum\limits_{m=1}^{M_j} 
\left[F_m(T^*,L)-{}\right.\right.\right.\\
\left.\left.\left.{}-{\sf E} F_m(T^*,L)\right]
\vphantom{\sum\limits_{j=0}^{[dJ]}\sum\limits_{m=1}^{M_j}}
\right\vert \Bigg / 
\left({\sigma^2\sqrt{2^{J+1}}}\right)>\delta\right)\leqslant {}\\
{}\leqslant
\exp\left\{-C_\delta2^{J-dJ}\right\},
\end{multline} 
т.\,е.
\begin{multline}
\label{First_Sum}
\fr{\sum\nolimits_{j=0}^{[dJ]}\sum\nolimits_{m=1}^{M_j} \left[F_m(T^*,L)-{\sf E} F_m(T^*,L)\right]}
{\sigma^2\sqrt{2^{J+1}}}\stackrel{{\sf P}}{\to}0\\
 \mbox{при}\  J\to\infty.
\end{multline}

При $p_1\leqslant p_2$ справедливы неравенства~\cite{Cai99}:
\begin{multline}
\label{Norms}
\left(\sum\limits_{k=0}^{2^j-1}\abs{\mu_{j,k}}^{p_2}\right)^{1/p_2}
\leqslant
\left(\sum\limits_{k=0}^{2^j-1}\abs{\mu_{j,k}}^{p_1}\right)^{1/p_1}
\leqslant{}\\
{}\leqslant
2^{j(1/p_1-1/p_2)}\left(\sum\limits_{k=0}^{2^j-1}\abs{\mu_{j,k}}^{p_2}\right)^{1/p_2}.
\end{multline}
Так как $f\in B^{\gamma}_{p,q}(A)$, то 
$$
2^{js}\left(\sum\limits_{k=0}^{2^j-1}\abs{\mu_{j,k}}^p\right)^{1/p}\hm\leqslant
 A2^{J/2}
 $$ 
 и~из неравенств~\eqref{Norms} следует, что при $p\hm\geqslant2$
\begin{equation}
\label{Besov2}
\sum\limits_{k=0}^{2^j-1}\mu_{j,k}^2 \leqslant A^22^{-2\gamma j+J}.
\end{equation}
Для произвольного $\eps\hm>0$ при всех $j\hm>dJ$ существует не более $A^22^{J-2\gamma j\hm+\eps j}$ 
слагаемых в~\eqref{Besov2} таких, что $\sum\nolimits_{k\in B_{j,m}}\mu_{j,k}^2>2^{-\eps j}$.
Выделяя эти слагаемые из второй суммы в~(\ref{Two_Sums}) в~отдельную сумму~$S'$ и~применяя к~$S'$ 
неравенство Хеффдинга, аналогичное~\eqref{Hoeff}, получаем, что 
$$
\fr{S'}{\sigma^2\sqrt{2^{J+1}}}\stackrel{{\sf P}}{\to}0\
\mbox{при } J\to\infty\,.
$$

Таким образом, без ограничения общности можно считать, что во второй сумме в~(\ref{Two_Sums}) 
$\sum\nolimits_{k\in B_{j,m}}\mu_{j,k}^2\hm\to0$ при $J\hm\to\infty$ для всех $m\hm=1,\ldots,M_j$ и~$j\hm>dJ$.

Если $p<2$ и~$\gamma\hm>1/p\hm-1/2$, то $s\hm>0$ и~из неравенств~\eqref{Norms} следует, что
\begin{equation}
\label{Besovp}
\sum\limits_{k=0}^{2^j-1}\mu_{j,k}^2 \leqslant A^22^{-2s j+J}.
\end{equation}
Рассуждая аналогично случаю $p\hm\geqslant2$, заключаем, что без ограничения общности можно считать, 
что во второй сумме в~(\ref{Two_Sums}) также $\sum\nolimits_{k\in B_{j,m}}\mu_{j,k}^2\hm\to0$ 
при $J\hm\to\infty$ для всех $m\hm=1,\ldots,M_j$ и~$j>dJ$.

Рассмотрим дисперсии слагаемых во второй сумме в~(\ref{Two_Sums}). 
Имеем 
$$
\Variance F_m(T^*,L)=\Variance \left[S^2_{j,m}-L\sigma^2+U_{j,m}(T^*,L)\right],
$$ где
\begin{multline*}
U_{j,m}(T^*,L)={}\\
{}=\left[\fr{(T^*L)^2\sigma^4-2T^*L\sigma^4(L-2)}{S_{j,m}^2}-S_{j,m}^2+2L\sigma^2\right]\times{}\\
{}\times
\Ik(S^2_{j,m}>T^*L\sigma^2).
\end{multline*}
Пусть $X_{L}$~--- случайная величина, имеющая распределение~$\chi^2_{L}$ с~$L$ степенями свободы, и~$C_1$ 
и~$C_2$~--- некоторые положительные константы.
Тогда в~силу~(\ref{Besov2}) или~(\ref{Besovp}) для некоторой константы 
$1\hm<C_\chi\hm<1\hm+\delta_\chi$ ($0\hm<\delta_\chi\hm<1$)
\begin{multline*}
\Variance\left[S_{j,m}^2\Ik\left(S^2_{j,m}>T^*L\sigma^2\right)\right]\leqslant{}\\
{}\leqslant
 {\sf E} \left[\left(S_{j,m}^2\right)^2\Ik(S^2_{j,m}>T^*L\sigma^2)\right]\leqslant{}\\
{}\leqslant C^2_\chi{\sf E}\left[X^2_{L}\Ik(C_\chi X_{L}>T^*L)\right]\leqslant {}\\
{}\leqslant
C_1(T^*L)^2{\sf P}\left(C_\chi X_{L}>T^*L\right)
\end{multline*}

\vspace*{-6pt}

\noindent
и аналогично
\begin{multline*}
\Variance\left(\left[\fr{(T^*L)^2\sigma^4-2T^*L\sigma^4(L-2)}{S_{j,m}^2}+
2L\sigma^2\right]\times{}\right.\\
\hspace*{-5pt}\left.{}\times
\Ik(S^2_{j,m}>T^*L\sigma^2)\!
\vphantom{\fr{(T^*L)^2\sigma^4-2T^*L\sigma^4(L-2)}{S_{j,m}^2}}
\right)\leqslant C_2(T^*L)^2
{\sf P}\left(C_\chi X_{L}>T^*L
\right)\!.\hspace*{-5.1548pt}
\end{multline*}
В силу леммы~2 из работы~\cite{Cai99} и~вида~$T^*$ и~$L$
$$
\left(T^*L\right)^2{\sf P}\left(C_\chi X_{L}>T^*L\right)\to0 
\mbox{ при } J\to\infty\,.
$$

Таким образом, учитывая очевидное соотношение для дисперсии разности
 случайных величин $X\hm-Y$: 
 $$
 \left(\sqrt{\Variance X}-\sqrt{\Variance Y}\right)^2
\leqslant\Variance(X-Y)\leqslant\left(\sqrt{\Variance X}\hm+\sqrt{\Variance Y}\right)^2,
$$
 получаем
\begin{equation}
\label{Var_Lim1}
\lim\limits_{J\to\infty}\fr{\Variance\sum\nolimits_{j=[dJ]+1}^{J-1}\sum\nolimits_{m=1}^{M_j} 
F_m(T^*,L)}{\Variance\sum\nolimits_{j=[dJ]+1}^{J-1}\sum\nolimits_{k=0}^{2^j-1}Y^2_{j,k}}=1\,.
\end{equation}
Кроме того, поскольку $Y_{j,k}$ независимы и~$\Variance Y^2_{j,k}
\hm=2\sigma^4\hm+4\sigma^2\mu^2_{j,k}$, получаем
\begin{equation}
\label{Var_Lim2}
\lim\limits_{J\to\infty}\fr{\Variance\sum\nolimits_{j=[dJ]+1}^{J-1}\sum\nolimits_{k=0}^{2^j-1}
Y^2_{j,k}}{\sigma^42^{J+1}}=1\,.
\end{equation}

Наконец, выполнено условие Линдеберга: для любого $\epsilon>0$ при $J\to\infty$

\vspace*{-6pt}

\noindent
\begin{multline}
\label{Norm_Cond}
\fr{1}{D^2_J}\sum\limits_{j=[dJ]+1}^{J-1}\sum\limits_{m=1}^{M_j} 
{\sf E}  \bigl[\left( F_m(T^*,L)-{\sf E} F_m(T^*,L)\right)^2\times{}\\
{}\times
\Ik\left( |F_m(T^*,L)-{\sf E} F_m(T^*,L)| >\epsilon D_J\right)\bigr]\rightarrow 0\,,
\end{multline}

\vspace*{-2pt}

\noindent
где 

\vspace*{-2pt}

\noindent
$$
D^2_J=\Variance\sum\limits_{j=[dJ]+1}^{J-1}\sum\limits_{m=1}^{M_j} 
\left[F_m(T^*,L)\hm-{\sf E} F_m(T^*,L)\right].
$$

\vspace*{-2pt}

Действительно, в~силу~(\ref{Bounds}), (\ref{Var_Lim1}) и~(\ref{Var_Lim2}) начиная с~некоторого~$J$ 
все индикаторы в~(\ref{Norm_Cond}) обращаются в~ноль. Объединяя~(\ref{First_Sum}), (\ref{Var_Lim2}) 
и~(\ref{Norm_Cond}), получаем~(\ref{Normality}). Теорема доказана.

\smallskip

Докажем теперь свойство сильной состоятельности оценки (\ref{Risk_Estimate}), справедливое при более слабых ограничениях.

\noindent
\textbf{Теорема 2.} \textit{Пусть $f\hm\in  L^2(\mathbb{R})$ и~задана на конечном отрезке, 
тогда при любом $\alpha\hm>1/2$ имеет место сходимость}
\begin{equation}
\label{Strong_Con}
\fr{\widehat{R}_J(T^*)-R_J(T^*)}{2^{\alpha J}}\rightarrow 0 \mbox{ п.\,в. при } J\rightarrow\infty\,.
\end{equation}

\noindent
Д\,о\,к\,а\,з\,а\,т\,е\,л\,ь\,с\,т\,в\,о\,.\ \ 
  Используя неравенство Хеффдинга, с~учетом~(\ref{Bounds}) получаем, что для любого $\delta\hm>0$ 
  найдется константа $C_\delta\hm>0$ такая, что
  
  \vspace*{-6pt}
  
  \noindent
\begin{multline*}
p_J=\Prob\left(\fr{\abs{\widehat{R}_J(T^*)-R_J(T^*)}}{2^{\alpha J}}>\delta\right)
\leqslant{}\\
{}\leqslant
 \exp\left\{-C_\delta2^{2\alpha J-J}\right\},
\end{multline*} 

\vspace*{-2pt}

\noindent
и, поскольку $\sum\nolimits_{J=1}^{\infty}p_J\hm<\infty$,
в~силу леммы Бореля--Кан\-тел\-ли выполнено~(\ref{Strong_Con}). Теорема доказана.

\smallskip

Теоремы 1 и~2 дают теоретическое обоснование использования значения~$\widehat{R}_J(T^*)$ 
в~качестве оценки неизвестной величины риска (по\-греш\-ности)~$R_J(T^*)$, а~также 
дают возможность строить асимптотические доверительные интервалы для~$R_J(T^*)$.

\vspace*{-9pt}


{\small\frenchspacing
{%\baselineskip=10.8pt
%\addcontentsline{toc}{section}{References}
\begin{thebibliography}{99}

\vspace*{-3pt}

\bibitem{DonJ94}
\Au{Donoho D., Johnstone~I.\,M.} Ideal spatial adaptation via wavelet shrinkage~// 
Biometrika, 1994. Vol.~81. No.\,3.
P.~425--455.

\bibitem{HKP99} 
\Au{Hall P., Kerkyacharian~G., Picard~D.} On the minimax optimality of block thresholded wavelet estimators~// 
Stat. Sinica, 1999. Vol.~9. P.~33--50.

\bibitem{Cai99} 
\Au{Cai T.} Adaptive wavelet estimation: A~block thresholding and oracle inequality approach~// 
Ann. Stat., 1999. Vol.~28. No.\,3. P.~898--924.

\bibitem{St81} 
\Au{Stein C.} Estimation of the mean of a~multivariate normal distribution~//
 Ann. Stat., 1981. Vol.~9. No.\,6. P.~1135--1151.

\bibitem{SH12} 
\Au{Шестаков О.\,В.}  Асимптотическая нормальность оценки риска пороговой обработки вейв\-лет-ко\-эф\-фи\-ци\-ен\-тов 
при выборе адаптивного порога~// Докл.  Акад. наук, 2012. Т.~445. №\,5. С.~513--515.

\bibitem{SH16-1} 
\Au{Shestakov O.\,V.} On the strong consistency of the adaptive risk estimator for wavelet thresholding~// 
J.~Math. Sci., 2016. Vol.~214. No.\,1. P.~115--118.


\bibitem{SH16-2}
\Au{Шестаков О.\,В.} Статистические свойства метода подавления шума, основанного на стабилизированной 
жесткой пороговой обработке~// Информатика и~её применения, 2016.  Т.~10. Вып.~2. С. 65--69.

\bibitem{PSH19}
\Au{Попенова П.\,С., Шестаков~О.\,В.}
 Анализ статистических свойств метода гибридной пороговой обработки~// 
 Вестн. Тверск. ун-та. Сер.: Прикладная математика. 2019. №\,1. С.~15--22.

\bibitem{DonJ95}
\Au{Donoho D., Johnstone I.\,M.} Adapting to unknown smoothness via wavelet shrinkage~// 
J.~Am. Stat. Assoc., 1995. Vol.~90. P.~1200--1224.

\bibitem{DonJ98}
\Au{Donoho D., Johnstone I. M.} Minimax estimation via wavelet shrinkage~// Ann. Stat., 
1998. Vol.~26. No.\,3. P.~879--921.

\bibitem{G98}
\Au{Gao H.-Y.} Wavelet shrinkage denoising using the non-negative garrote~// J.~Comput. Graph. Stat., 
1998. Vol.~7. No.\,4. P.~469--488.

\bibitem{PK05}
\Au{Poornachandra S., Kumaravel~N., Saravanan~T.\,K., Somaskandan~R.} 
WaveShrink using modified hyper-shrinkage function~//
 27th Annual  Conference (International) of the IEEE Engineering in Medicine 
 and Biology Society Proceedings.~--- Piscataway, NJ, USA: IEEE, 2005. P. 30--32.

\bibitem{LC10}
\Au{Lin Y., Cai J.} A~new threshold function for signal denoising based on wavelet transform~// 
Conference (International) on Measuring Technology and Mechatronics Automation  Proceedings.~--- Piscataway, NJ, USA: 
IEEE, 2010. P.~200--203.

\bibitem{HL10}
\Au{Huang H.-C., Lee~T.\,C.\,M.} Stabilized thresholding with generalized sure for image denoising~// 
17th Conference (International) on Image Processing Proceedings.~--- 
Piscataway, NJ, USA: IEEE, 2010. P.~1881--1884.

\bibitem{HX15}
\Au{He C., Xing J., Li~J., Yang~Q., Wang~R.} 
A~new wavelet thresholding function based on hyperbolic tangent function~// 
Math. Probl. Eng., 2015. Vol.~2015. Art.~528656.

\bibitem{ZC15}
\Au{Zhao R.-M., Cui~H.-M.} Improved threshold denoising method based on wavelet transform~// 
 7th  Conference (International) on Modelling, Identification and Control  
Proceedings.~--- Piscataway, NJ, USA: IEEE, 2015. Art.~7409352. 4~p. doi: 10.1109/ICMIC.2015.7409352.

\bibitem{Mall99}
\Au{Mallat S.} A~wavelet tour of signal processing.~--- New York, NY, USA: Academic Press, 1999. 857~p.
 \end{thebibliography}

}
}

\end{multicols}

\vspace*{-8pt}

\hfill{\small\textit{Поступила в~редакцию 27.03.2021}}

\vspace*{5pt}

%\pagebreak

%\newpage

%\vspace*{-28pt}

\hrule

\vspace*{2pt}

\hrule

\vspace*{-3pt}

\def\tit{ANALYSIS OF THE UNBIASED MEAN-SQUARE RISK ESTIMATE OF~THE~BLOCK THRESHOLDING METHOD}


\def\titkol{Analysis of the unbiased mean-square risk estimate of~the~block thresholding method}

\def\aut{O.\,V.~Shestakov$^{1,2}$}

\def\autkol{O.\,V.~Shestakov}


\titel{\tit}{\aut}{\autkol}{\titkol}

\vspace*{-11pt}


\noindent
$^1$Department of Mathematical Statistics, Faculty of Computational Mathematics and Cybernetics,
M.\,V.~Lomo-\linebreak 
$\hphantom{^1}$nosov Moscow State University, 1-52~Leninskie Gory, GSP-1, Moscow 119991, 
Russian Federation

\noindent
$^2$Institute of Informatics Problems, Federal Research Center ``Computer Science and Control'' 
of the Russian\linebreak
$\hphantom{^1}$Academy of Sciences, 44-2~Vavilov Str., Moscow 119333, Russian Federation

 
\def\leftfootline{\small{\textbf{\thepage}
\hfill INFORMATIKA I EE PRIMENENIYA~--- INFORMATICS AND
APPLICATIONS\ \ \ 2021\ \ \ volume~15\ \ \ issue\ 2}
}%
\def\rightfootline{\small{INFORMATIKA I EE PRIMENENIYA~---
INFORMATICS AND APPLICATIONS\ \ \ 2021\ \ \ volume~15\ \ \ issue\ 2
\hfill \textbf{\thepage}}}

\vspace*{3pt}




\Abste{Signal and image processing methods based on wavelet decomposition and thresholding 
have become very popular in solving problems of compression and noise suppression. This is 
due to their ability to adapt to local features of functions, high speed of processing 
algorithms and  optimality of estimates obtained. In this paper, a~block thresholding 
method is considered, in which expansion coefficients are processed in groups, which makes 
it possible to take into account information about neighboring coefficients. 
In the model with additive noise, an unbiased estimate of the mean-square risk is analyzed 
and it is shown that, under certain conditions of regularity, this estimate is strongly 
consistent and asymptotically normal. These properties allow using the risk estimate 
as a~quality criterion for the method and constructing asymptotic confidence intervals for the 
theoretical mean-square risk.}

\KWE{wavelets; block thresholding; mean-square risk estimate; asymptotic normality; strong consistency}



\DOI{10.14357/19922264210205}

\vspace*{-18pt}

 \Ack
\noindent
This research was supported by the Russian Foundation for Basic Research (project 19-07-00352). 
The research was conducted in accordance with the program of the Moscow Center 
for Fundamental and Applied Mathematics.


\vspace*{6pt}

  \begin{multicols}{2}

\renewcommand{\bibname}{\protect\rmfamily References}
%\renewcommand{\bibname}{\large\protect\rm References}

{\small\frenchspacing
 {%\baselineskip=10.8pt
 \addcontentsline{toc}{section}{References}
 \begin{thebibliography}{99}
\bibitem{1-she}
\Aue{Donoho, D., and I.\,M.~Johnstone.} 
1994. Ideal spatial adaptation via wavelet shrinkage. \textit{Biometrika} 81(3):425--455.
\bibitem{2-she}
\Aue{Hall, P., G.~Kerkyacharian, and D.~Picard.}
 1999. On the minimax optimality of block thresholded wavelet estimators. \textit{Stat. Sinica} 9:33--50.
\bibitem{3-she}
\Aue{Cai, T.} 1999. Adaptive wavelet estimation: A~block thresholding and oracle inequality approach. 
\textit{Ann. Stat.} 28(3):898--924.
\bibitem{4-she}
\Aue{Stein, C.} 1981. Estimation of the mean of a multivariate normal distribution. 
\textit{Ann. Stat.} 9(6):1135--1151.
\bibitem{5-she}
\Aue{Shestakov, O.\,V.} 2012. Asymptotic normality of adaptive wavelet thresholding risk estimation. 
\textit{Dokl. Math.} 86(1):556--558.
\bibitem{6-she}
\Aue{Shestakov, O.\,V.} 2016. On the strong consistency of the adaptive risk estimator for wavelet 
thresholding. \textit{J.~Math. Sci.} 214(1):115--118.
\bibitem{7-she}
\Aue{Shestakov, O.\,V.} 2016. Statisticheskie svoystva metoda podavleniya shuma, osnovannogo 
na stabilizirovannoy zhest\-koy porogovoy obrabotke [Statistical properties of the denoising 
method based on the stabilized hard thresholding]. \textit{Informatika i~ee Primeneniya~--- Inform. Appl.}
 10(2):65--69.
\bibitem{8-she}
\Aue{Popenova, P.\,S., and O.\,V.~Shestakov.}
 2019. Analiz sta\-ti\-sti\-che\-skikh svoystv metoda gibridnoy porogovoy obrabotki
  [Analysis of statistical properties of the hybrid thresholding technique]. 
  \textit{Vestnik Tverskogo gosudarstvennogo un-ta. Ser. Prikladnaya matematika}
   [Bull. of the Tverskoy State University. Ser. Appl. Math.] 1:15--22.
\bibitem{9-she}
\Aue{Donoho, D., and I.\,M.~Johnstone.}
 1995. Adapting to unknown smoothness via wavelet shrinkage. \textit{J.~Am. Stat. Assoc.} 90:1200--1224.
\bibitem{10-she}
\Aue{Donoho, D., and I.\,M.~Johnstone.}
 1998. Minimax estimation via wavelet shrinkage. \textit{Ann. Stat.} 26(3):879--921.
\bibitem{11-she}
\Aue{Gao, H.-Y.} 1998. Wavelet shrinkage denoising using the non-negative garrote. 
\textit{J.~Comput. Graph. Stat.} 7(4):469--488.
\bibitem{12-she}
\Aue{Poornachandra, S., N.~Kumaravel, T.\,K.~Saravanan,  and R.~Somaskandan.}
 2005. WaveShrink using modified hyper-shrinkage function. 
 \textit{27th Annual  Conference (International) of the IEEE Engineering in Medicine 
 and Biology Society Proceedings}. Piscataway, NJ: IEEE. 30--32.
\bibitem{13-she}
\Aue{Lin, Y., and J.~Cai.} 2010. A~new threshold function for signal denoising based on wavelet transform. 
\textit{Conference (International) on Measuring Technology and Mechatronics Automation Proceedings}.
 Piscataway, NJ: IEEE. 200--203.
\bibitem{14-she}
\Aue{Huang, H.-C., and T.\,C.\,M.~Lee.}
 2010. Stabilized thresholding with generalized sure for image denoising. 
 \textit{17th Conference (International) on Image Processing Proceedings}. 
 Piscataway, NJ: IEEE. 1881--1884.
\bibitem{15-she}
\Aue{He, C., J.~Xing, J.~Li, Q.~Yang, and R.~Wang.} 2015. 
A~new wavelet thresholding function based on hyperbolic tangent function. 
\textit{Math. Probl. Eng.} 2015:528656. 10~p.
\bibitem{16-she}
\Aue{Zhao, R.-M., and H.-M.~Cui.} 2015. Improved threshold denoising method based on wavelet transform.
\textit{7th Conference (International) on Modelling, Identification and Control Proceedings}. 
Piscataway, NJ: 
IEEE. Art. ID: 7409352. 4~p. doi: 10.1109/ICMIC.2015.7409352.
\bibitem{17-she}
\Aue{Mallat, S.} 1999. \textit{A~wavelet tour of signal processing}. New York, NY: Academic Press. 857~p.
 \end{thebibliography}

 }
 }

\end{multicols}

\vspace*{-3pt}

  \hfill{\small\textit{Received March~27, 2021}}


%\pagebreak

%\vspace*{-8pt}  

\Contrl

\noindent
\textbf{Shestakov Oleg V.} (b.\ 1976)~--- 
Doctor of Science in physics and mathematics, professor, Department of Mathematical Statistics, 
Faculty of Computational Mathematics and Cybernetics, M.\,V.~Lomonosov Moscow State University, 
1-52~Leninskie Gory, GSP-1, Moscow 119991, Russian Federation; senior scientist, 
Institute of Informatics Problems, Federal Research Center ``Computer Science and Control'' 
of the Russian Academy of Sciences, 44-2~Vavilov Str., Moscow 119333, Russian Federation; 
\mbox{oshestakov@cs.msu.su}
\label{end\stat}

\renewcommand{\bibname}{\protect\rm Литература}