

\def\stat{gonch+inkova}

\def\tit{ИЗВЛЕЧЕНИЕ ЗНАНИЙ О~СРЕДСТВАХ ВЫРАЖЕНИЯ ЛОГИКО-СЕМАНТИЧЕСКИХ 
ОТНОШЕНИЙ\\ ПРИ~ПОМОЩИ НАДКОРПУСНОЙ БАЗЫ ДАННЫХ}

\def\titkol{Извлечение знаний о~средствах выражения логико-семантических 
отношений при~помощи НБД} %надкорпусной базы данных}

\def\aut{А.\,А.~Гончаров$^1$, О.\,Ю.~Инькова$^2$}

\def\autkol{А.\,А.~Гончаров, О.\,Ю.~Инькова}

\titel{\tit}{\aut}{\autkol}{\titkol}

\index{Гончаров А.\,А.}
\index{Инькова О.\,Ю.} 
\index{Goncharov A.\,A.}
\index{Inkova O.\,Yu.}

%{\renewcommand{\thefootnote}{\fnsymbol{footnote}} \footnotetext[1]
%{Работа выполнена в~Институте проблем информатики ФИЦ ИУ РАН при поддержке РФФИ (проект  
%20-012-00166).}}


\renewcommand{\thefootnote}{\arabic{footnote}}
\footnotetext[1]{Институт проблем информатики Федерального исследовательского центра <<Информатика 
и~управление>> Российской академии наук, \mbox{a.gonch48@gmail.com}}
\footnotetext[2]{Институт проблем информатики Федерального исследовательского центра <<Информатика 
и~управление>> Российской академии наук, \mbox{olyainkova@yandex.ru}}

\vspace*{-9pt}



\Abst{Цель статьи~--- показать продуктивность использования параллельных текстов и~их 
аннотирования в~надкорпусной базе данных (НБД) коннекторов для извлечения знаний об 
альтернативных средствах выражения ло\-ги\-ко-семантических отношений (ЛСО). На примере 
наиболее известных дискурсивно аннотированных корпусов~--- Penn Discourse Treebank (PDTB), 
Prague Dependency Treebank (PDT) и~Rhetorical Structure Theory  Discourse Treebank (RST-DT)~--- авторы 
показывают, что в~существующих исследованиях нет консенсуса относительно того, какие 
языковые средства относить к~классу коннекторов (прототипических показателей  
ЛСО), а~какие~--- к~альтернативным средствам. 
В~исследовании продемонстрировано, что применение сопоставительного метода 
и~использование возможностей НБД коннекторов позволяет не только извлекать новое 
знание о~средствах выражения ЛСО в~изучаемых языках, но и~создавать тезаурусы таких 
средств, в~том числе альтернативных коннекторам. Кроме того, 
информация, хранящаяся в~НБД, дает возможность получать новые знания о том, какие 
ЛСО могут быть выражены неспециализированными средствами и~какова частотность  
использования этих средств для каждого ЛСО в~каждом из изучаемых языков.}

\KW{надкорпусная база данных; логико-семантические отношения; коннекторы; 
извлечение новых знаний; параллельные тексты}

\DOI{10.14357/19922264210214}

\vspace*{4pt}


\vskip 10pt plus 9pt minus 6pt

\thispagestyle{headings}

\begin{multicols}{2}

\label{st\stat}

\section{Вводные замечания}

\vspace*{-6pt}

Логико-семантические, или, шире, дискурсивные, отношения, 
обеспечивающие связность текста на естественном языке, привлекают 
внимание лингвистов и~специалистов по информатике уже не один десяток 
лет: первые исследования начали появляться в~1970-х~гг.\ (например, работы 
Дж.~Хоббса~[1, 2]). Однако многие вопросы до сих пор остаются 
дискуссионными: это, в~первую очередь, и~само понятие <<дискурсивное 
отношение>>, и~понятие <<коннектор>> (единицы этого класса считаются 
прототипическими эксплицитными показателями таких отношений). Нет 
консенсуса и~относительно того, можно ли создать исчерпывающий список 
коннекторов для исследуемого языка. Тем не менее важность списков 
коннекторов для разработки дискурсивных парсеров и, шире, средств 
автоматической обработки текста и~автоматического извлечения информации 
из текста подчеркивается в~ряде работ (см.,\ например,~[3, с.~55]). В~[4], где 
описываются результаты разработки дискурсивного парсера для русского 
языка, отмечается, что наличие показателя 
ЛСО служит наиболее надежным признаком для определения 
того, каким именно отношением связаны фрагменты текста.

В то же время в~[5] показано, что в~зависимости от типа текста и~вида ЛСО 
коннекторы используются лишь в~30\%--40\% случаев. Остальные случаи 
представляют собой либо имплицитные ЛСО (подробнее об этом понятии 
см., например,~[6]), либо ЛСО, показателем которого являются языковые 
средства, отличные от коннекторов. Следовательно, качество результатов 
автоматической обработки текстов на естественном языке непосредственно 
зависит от уровня наших знаний не только о коннекторах, но и~об этих, 
альтернативных, средствах выражения ЛСО. Цель статьи состоит в~том, 
чтобы показать продуктивность использования параллельных текстов 
и~поисковых возможностей НБД коннекторов, разработанной в~ИПИ ФИЦ 
ИУ РАН (подробнее см.~\cite{7-in, 8-in, 9-in}), для извлечения знаний об 
альтернативных средствах выражения ЛСО.

\vspace*{-8pt}

\section{Существующие подходы}

\vspace*{-1pt}

Отправной точкой для активных исследований средств выражения ЛСО, 
альтернативных коннекторам, стала статья~[10], отражающая подход 
создателей Пенсильванского дискурсивно аннотированного корпуса (PDTB)~[11] к~этому вопросу. При аннотировании 
корпуса выяснилось, что если два фрагмента текста связаны ка\-ки\-м-ли\-бо 
ЛСО (дискурсивным отношением в~терминах PDTB), то это отношение 
может: ($i$)~выражаться коннектором\footnote{К~коннекторам относятся 
сочинительные и~подчинительные союзы, а~также некоторые другие языковые единицы, 
за которыми грамматиками английского языка традиционно признается связующая 
функция.}; ($ii$)~не выражаться коннектором, причем какой-либо коннектор 
можно добавить (=\;импли\-цит\-ные ЛСО); ($iii$)~не выражаться коннектором, 
причем никакой коннектор не может быть добавлен из-за возникающей 
в~этом случае семантической из-\linebreak быточности. Авторы пришли к~выводу, что 
такая\linebreak избыточность вызвана наличием альтернативных коннекторам 
лексических средств выражения ЛСО~--- <<альтернативных 
лексикализаций>> (alternative lexicalizations, AltLex).

Если в~PDTB в~разряд коннекторов попадает ограниченный круг языковых 
единиц, то разработчики Пражского корпуса синтаксических зависимостей 
 (PDT) трактуют понятие коннектор более 
широко, включая в~этот класс большинство лексических средств, которые так 
или иначе могут выражать ЛСО~\cite{3-in, 12-in}. Коннекторы при этом 
разделяются на <<первичные>> (primary) и~<<вторичные>> (secondary), 
довольно разнообразные по своей морфологической природе. Не включаются в~чис\-ло коннекторов лишь так называемые <<неуниверсальные>>, или 
<<свободные связующие сочетания>> (non-\linebreak universal\,/\,free connecting 
phrases), образующие третий класс средств выражения ЛСО. Ср.~(1)--(4) 
из~\cite[с.~51, 68]{3-in}:
{\small
\begin{enumerate}[(1)]
\item Fred didn't stop joking. \textbf{As a result}, his friends enjoyed hilarity throughout the 
evening.

`Фред не переставал шутить. \textbf{В~результате} его друзья смеялись весь 
вечер'.\footnote[2]{Здесь и~далее в~отсутствие других указаний перевод авторов статьи.}

\item  I had all the necessary qualifications. \textbf{Despite this}, I~didn't get the job.

`Я~удовлетворял всем квалификационным требованиям. \textbf{Несмотря на это}, на 
работу меня не приняли'.

\item 
Fred didn't stop joking. This \textbf{caused} hilarity among his friends for the whole evening.

`Фред не переставал шутить. Это \textbf{вызывало} смех его друзей весь вечер'.

\item 
 Fred has pneumonia. \textbf{Because of this illness}, he will be absent from his work for 
two weeks.

`У~Фреда пневмония. \textbf{Вследствие этой болезни} его не будет на работе две 
недели'.
\end{enumerate}
}

Так, в~(1) причинно-следственные ЛСО выражены союзом \textit{as 
a~result}~--- <<первичным коннектором>>. В~(2) и~(3) используются 
<<вторичные коннекторы>>: в~(2) это сочетание предлога \textit{despite} 
с~анафорическим выражением \textit{this}, отсылающим к~ситуации 
\textit{I~had all the necessary qualifications}, которое выражает уступительные 
ЛСО; в~(3) <<вторичным коннектором>> считается глагол \textit{caused}, 
вы\-ра\-жа\-ющий причинные ЛСО. Наконец, в~(4) \textit{because of this illness} 
рассматривается как <<неуниверсальное>>, или <<свободное связующее 
сочетание>>, поскольку оно непосредственно связано с~предыдущим 
контекстом (в~котором упомянута пневмония Фреда), в~отличие от 
\textit{despite this} в~(3), имеющего более общее значение. 

<<Альтернативным лексикализациям>> в~подходе PDTB соответствуют 
второй и~третий классы единиц в~PDT~\cite[с.~54]{3-in}. Такого же широкого 
подхода к~<<коннекторам>> придерживаются \mbox{разработчики} русского 
дискурсивно аннотированного корпуса~\cite{13-in}.

Подчеркнем, что во всех описанных случаях во внимание принимаются лишь 
лексические средства выражения ЛСО. В~\cite[с.~62]{3-in} даже особо 
отмечается, что <<коннекторами>> не считаются синтаксические 
и~морфологические средства, например относительные придаточные или 
деепричастия, которые в~ряде языков способны выражать ЛСО (см.\ ниже). 
Подчеркивая важность создания лексиконов связующих средств, 
разработчики Пражского корпуса не дают, тем не менее, списка 
<<вторичных коннекторов>>.

В последней версии PDTB~(3.0) появился класс показателей ЛСО AltLexC 
(где <<С>> означает \textit{Construction}), включающий лексико-син\-так\-си\-че\-ские 
средства выражения ЛСО~\cite[с.~9--10, 76]{14-in}. Однако 
если для <<первичных коннекторов>> приводится список языковых единиц, 
то ни для <<альтернативных лексикализаций>>, ни для нового класса 
AltLexC списков не дается. В~\cite[с.~75--76]{14-in} приводятся лишь ЛСО, 
которые выражают те единицы этих классов, что зафиксированы в~последней 
версии корпуса PDTB.

В рамках Теории риторической структуры (Rhetorical Structure Theory, RST) 
принципы классификации языковых единиц, способных выражать 
риторические (в~данной терминологии) отношения, даны  
в~работе~\cite[с.~8, 9]{15-in}, которая служит пособием по аннотированию 
показателей риторических отношений в~корпусе RST Discourse Treebank 
(RST-DT). Поскольку объем понятия <<риторическое отношение>> шире, 
чем ЛСО, так как включает не только отношения связности, которые могут 
быть выражены коннектором, то и~набор показателей этих отношений шире. 
К~<<первичным>> коннекторам добавляются показатели самой 
разнообразной природы: лексические, морфологические (временн$\acute{\mbox{ы}}$е формы), 
семантические (синонимия, антонимия и~др.), синтаксические (различные 
виды придаточных и~др.), графические (знаки препинания и~др.) и~т.\,д.; 
причем эти средства могут также комбинироваться друг 
с~другом\footnote[1]{Заметим, что в~более ранних версиях корпусов, размеченных 
в~соответствии с~RST, во всех таких случаях риторическое отношение считалось 
имплицитным. Согласно версии, представленной в~\cite{15-in}, оно оказывается 
эксплицитным, а в~тех случаях, когда никакой из потенциальных показателей отношения 
не может быть идентифицирован, проставляется метка \textit{unsure}.}. Однако 
в~пособии по аннотированию приводятся лишь типы показателей 
риторических отношений, а~не их список.
{\looseness=1

}

\section{Альтернативные средства~выражения логико-семантических отношений в~надкорпусной
базе данных коннекторов}

Если во всех упомянутых выше работах средства выражения ЛСО изучаются 
на одноязычном материале, то НБД коннекторов позволяет проводить 
исследования на материале параллельных текстов, используя методы 
сопоставительной лингвистики. Аннотирование употреблений коннекторов 
в~параллельных текстах позволило заметить, что в~некоторых случаях 
использованный в~оригинале коннектор переведен не коннектором, а~другим 
языковым средством (или наоборот, в~оригинале коннектор отсутствует, но 
появляется в~переводе). С~точки зрения сопоставительного подхода такие 
случаи представляют собой примеры <<дивергентного 
перевода>>\footnote[2]{Термин <<дивергентный перевод>> заимствован из 
работы~\cite{16-in}, посвященной использованию многоязычных корпусов 
в~контрастивных исследованиях; он был уточнен для исследования коннекторов 
в~\cite{17-in}.}.

Для обозначения языковых единиц, не являющихся коннекторами, но 
способными выражать ЛСО, предлагается использовать термин 
<<альтернативные коннекторам средства выражения ЛСО>>. В НБД такие 
средства делятся на ($i$)~лексические; ($ii$)~грамматические 
и~($iii$)~пунктуационные.

\subsection{Лексические средства}

В примере~(5) коннектор \textit{то есть}, выражающий ЛСО 
переформулирования, двумя переводчиками передан альтернативными 
коннекторам лексическими средствами~--- `я~имею в~виду' и~`я~хочу 
сказать' соответственно.
{\small 
\begin{enumerate}[(1)]
\setcounter{enumi}{4}
\item 
 Моя теща, \textbf{то есть} мать жены моей, тоже ничего не видит. [Н.\,В.~Гоголь. Нос 
(1832--1833)]

`Ma belle-m$\grave{\mbox{e}}$re, \textbf{j'entends} la m$\grave{\mbox{e}}$re de ma 
femme, a, elle aussi, la vue faible.' [Tr.\,H.~Mongault (1938)]

`Ma belle-m$\grave{\mbox{e}}$re, \textbf{je veux dire}, la m$\grave{\mbox{e}}$re de ma 
femme, elle non plus, elle n'y voit rien du tout.' [Tr.\,A.~Markowitz (2007)]
\end{enumerate}
}

\subsection{Грамматические средства}

В примере~(6) для передачи коннектора \textit{потому что}, выражающего 
ЛСО причины, также два переводчика используют форму причастия 
настоящего времени, которая во французском языке способна выражать это 
отношение.
{\small
\begin{enumerate}[(1)]
\setcounter{enumi}{5}
\item В~заключение прибавлял, что он <<был бы счастлив, если б удалось ему на себе 
оправдать свое убеждение, но что достичь этого он не надеется, \textbf{потому что} это 
очень трудно>>. [И.\,А.~Гончаров. Обломов (1848--1859)]

`Et il concluait en ajoutant qu'il serait tout $\grave{\mbox{a}}$~fait heureux s'il parvenait 
$\grave{\mbox{a}}$~justifier ses id$\acute{\mbox{e}}$es par son comportement, mais qu'il 
n'esp$\acute{\mbox{e}}$rait pas y~parvenir, cette ad$\acute{\mbox{e}}$quation 
$\acute{\mbox{\textbf{e}}}$\textbf{tant} fort difficile $\grave{\mbox{a}}$~atteindre.' 
[Tr.\,A.~Adamov (1959)]

`En guise de conclusion il ajoutait <<qu'il serait heureux s'il pouvait justifier ses convictions 
par sa propre vie, mais qu'il ne l'esp$\acute{\mbox{e}}$rait pas, cet objectif 
$\acute{\mbox{\textbf{e}}}$\textbf{tant} trop difficile $\grave{\mbox{a}}$~atteindre>>.' 
[Tr.\,L.~Jurgenson (1988)]
\end{enumerate}
}

\subsection{Пунктуационные средства}

В~(7) уже упоминавшийся выше коннектор \textit{потому что} передан 
в~двух переводах двоеточием.
{\small
\begin{enumerate}[(1)]
\setcounter{enumi}{6}
\item
$\ldots$но коллежский асессор Ковалев не мог слышать запаха, \textbf{потому что} 
закрылся платком и~потому что самый нос его находился бог знает в~каких местах. 
[Н.\,В.~Гоголь. Нос (1832--1833)]

`$\ldots$Mais l'assesseur de coll$\grave{\mbox{e}}$ge Kovaliov ne pouvait pas s'en rendre 
compte\textbf{:} il avait cach$\acute{\mbox{e}}$ son visage sous un mouchoir, et d'ailleurs son nez 
se trouvait en cet instant Dieu sait o$\grave{\mbox{u}}$.' [Tr.\,B.~de Schloezer (1925)]

`$\ldots$Cependant le major Kovaliov ne s'en trouvait point incommod$\acute{\mbox{e}}$\textbf{:} 
il tenait son mouchoir sur son visage, et d'ailleurs son nez se promenait$\ldots$ Dieu sait 
o$\grave{\mbox{u}}$.' [Tr.\,H.~Mongault (1938)]
\end{enumerate}
}

Тот факт, что в~примерах~(5)--(7) несколько переводчиков выбирают 
альтернативные средства выражения ЛСО, свидетельствует о~том, что эти 
средства выражают ЛСО на регулярной основе, а~не являются единичными 
переводческими ре\-ше\-ни\-ями.
{\looseness=1

}

\begin{table*}[b]\small %tabl1
\begin{center}
\Caption{Виды ПС для коннекторов русского языка в~НБД (направление перевода 
рус\-ский--фран\-цуз\-ский)}
\vspace*{2ex}

\begin{tabular}{|c|c|c|c|c|}
\hline
Всего&Конгруэнтное ПС&Дивергентное ПС&Конгруэнтно-дивергентное 
ПС&\tabcolsep=0pt\begin{tabular}{c}Эксплицитная\\ языковая\\ единица\\ отсутствует\end{tabular}\\
\hline
11\,175 
(100\%)&8\,948 
(80,07\%)&942 
(8,43\%)&167 
(1,5\%)&1\,118 
(10\%)\\
\hline
\end{tabular}
\end{center}
%\end{table*}
%\begin{table*}[b]\small %tabl2
\vspace*{8pt}
\begin{center}
\parbox{358pt}{\Caption{Наиболее употребительные дивергентные ПС, зафиксированные в~НБД 
(на\-прав\-ле\-ние перевода рус\-ский--фран\-цуз\-ский)}
}

\vspace*{2ex}

\begin{tabular}{|c|l|c|}
\hline
№&\multicolumn{1}{c|}{Средство выражения ЛСО 
в~переводе}&\tabcolsep=0pt\begin{tabular}{c}Число ПС\\ (с коннектором\\ 
в~оригинале)\end{tabular}\\
\hline
1&Придаточное определительное предложение &47\\
2&Форма деепричастия настоящего времени&35\\
3&Конструкция с~местоименным повтором&20\\
4&Форма деепричастия настоящего времени в~сочетании с~\textit{tout}&20\\
5&Форма причастия настоящего времени&17\\
$\ldots$&$\ldots$&$\ldots$\\
46\hphantom{9}&\textit{Il n'y a que$\ldots$\ qui}&\hphantom{9}1\\
\hline
&&246\hphantom{9}\\
\hline
\end{tabular}
\end{center}
\end{table*}




Промежуточное положение между коннекторами и~альтернативными им 
средствами выражения ЛСО занимают языковые единицы, пред\-став\-ля\-ющие 
собой сочетание коннектора и~лексического и/или грамматического средства. 
Так, в~(8) при переводе коннектора \textit{так как} на французский язык\linebreak 
использовано сочетание коннектора \textit{puisque} и~формы причастия 
настоящего времени. Для обо\-значения таких сочетаний используется термин 
<<комбинированные средства выражения ЛСО>>, а~переводное 
соответствие считается кон\-гру\-энт\-но-ди\-вер\-гент\-ным.
{\small
\begin{enumerate}[(1)]
\setcounter{enumi}{7}
\item Предприятия, расположенные в~городе и~севернее города, не выполнили своих 
обязательств перед государством, \textbf{так как} находятся в~районе военных 
действий. [В.\,С.~Гроссман. Жизнь и~судьба (1960)]

`Les entreprises situ$\acute{\mbox{e}}$es dans la ville, ou un peu au nord, n'avaient pu 
remplir leurs obligations envers l'$\acute{\mbox{E}}$tat, \textbf{puisque se trouvant} en 
pleine zone d'op$\acute{\mbox{e}}$rations militaires.' [Tr.\,A.~Berelowitch (1980)]
\end{enumerate}
}

Похожая группа появляется в~последней версии PDTB (3.0)~--- <<AltLex 
Relations Linked with Explicits>>~\cite[с.~80]{14-in}. Она, однако, не 
аналогична классу комбинированных средств выражения ЛСО, так как 
включает единицы, относимые нами к~коннекторам, такие как \textit{and in 
general, but in general}.

В табл.~1 приводятся данные (по состоянию на 02.03.2021) о числе 
зафиксированных в~НБД коннекторов переводных соответствий (далее~--- 
ПС), где в~русском языке (языке оригинала) исследуемая языковая единица 
является коннектором. Для сравнения: в~Пражском корпусе на <<вторичные 
коннекторы>>, при довольно широкой трактовке этого термина, 
приходится~5\%~\cite[с.~456]{12-in}, а~в~PDTB~3.0 на AltLex и~AltLexC~--- 
в~сумме чуть более~3\%~\cite[с.~5]{14-in}.
{\looseness=1

}


По данным НБД, в~дивергентных ПС коннектор чаще всего передается 
лексическими средствами, а~реже всего~--- знаками препинания. На 
грамматические средства, которые совсем не учитываются в~Пражском 
корпусе и~лишь недавно стали аннотироваться в~PDTB, приходится 
более~26\% альтернативных средств. Этим, видимо, можно объяснить более 
высокую долю дивергентных соответствий в~НБД. В~табл.~2 приводятся 
наиболее употребительные грамматические средства выражения ЛСО. На 
данный момент они не разделены на более мелкие подклассы, но фасетная 
классификация, используемая в~НБД (см.~\cite{18-in}), позволяет решить эту 
задачу.
{\looseness=-1

}


В табл.~3 сравниваются, с~одной стороны, данные о том, какие ЛСО могут 
передаваться с~использованием альтернативных средств при переводе 
с~русского языка на французский, и,~с~другой стороны, данные о том, какие 
ЛСО могут выражаться альтернативными средствами в~англоязычном 
корпусе PDTB~3.0. В~работе~\cite{14-in} отсутствуют данные об общем 
числе примеров для каждого ЛСО, поэтому в~табл.~3 приводятся только 
абсолютные цифры. В~четвертом столбце табл.~3 указано общее число 
примеров для AltLex и~AltLexC, так как обе группы соответствуют 
альтернативным средствам выражения ЛСО в~НБД. Данные из табл.~3 
показывают, что использование параллельных текстов позволяет извлечь 
знания об альтернативных средствах выражения большего числа ЛСО, чем 
анализ одноязычного материала.

\begin{table*}\small %tabl3
\begin{center}
\Caption{Альтернативные коннекторам средства выражения в~НБД и~в~PDTB}
\vspace*{2ex}

\tabcolsep=2.5pt
\begin{tabular}{|c|c|c|c|}
\hline
\textbf{Отношение в~НБД}&\tabcolsep=0pt\begin{tabular}{c}\textbf{Дивер-}\\ 
\textbf{гентных}\\ \textbf{ПС}\end{tabular}&\textbf{Отношение в~PDTB}&
\textbf{AltLex}\\
\hline
\tabcolsep=0pt\begin{tabular}{c}Исключение\\  Исключение из 
рассмотрения\end{tabular}&\tabcolsep=0pt\begin{tabular}{c}131\hphantom{9}\\ 27 
\end{tabular}&Exception.Arg2-as-excpt&\hphantom{9}3\\
\hline
<<Вопреки ожидаемому>>&47&\multicolumn{1}{|c|}{\raisebox{-36pt}[0pt][0pt]{Contrast}}&
\multicolumn{1}{|c|}{\raisebox{-36pt}[0pt][0pt]{41}}\\
Сопоставительные&35&&\\
<<Вопреки ожидаемому>> иллокутивные&19&&\\
Возместительное противопоставление&12&&\\
Противительно-уступительные иллокутивные&\hphantom{9}4&&\\
Противительно-уступительные&\hphantom{9}5&&\\
Контраст&\hphantom{9}2&&\\
Противопоставление&\hphantom{9}2&&\\
\hline
\tabcolsep=0pt\begin{tabular}{c}Аддитивные иллокутивные\\ Аддитивные пропозициональные\\
Соединительные\end{tabular}&\tabcolsep=0pt\begin{tabular}{c}45\\ 16\\ 28\end{tabular}&Conjunction&139\hphantom{9}\\
\hline
Замещение&73&\tabcolsep=0pt\begin{tabular}{c}Substitution.Arg1-as-subst;\\ 
Substitution.Arg2-as-subst\end{tabular}&29\\
\hline
Переформулирование&68&Equivalence&10\\
\hline
Спецификация&60&\tabcolsep=0pt\begin{tabular}{c}Instantiation.Arg2-as-instance;\\ 
Level-of-detail.Arg2-as-detail\end{tabular}&106\hphantom{9}\\
\hline
Временные & 53 & Asynchronous.Precedence;&\\ 
\cline{1-2}
Временные 
метаязыковые& \hphantom{9}4&\tabcolsep=0pt\begin{tabular}{c} 
Asynchronous.Succession;\\ Synchronous\end{tabular}&
\tabcolsep=0pt\begin{tabular}{c} 160\hphantom{9}\\ \ \end{tabular}\\
\hline
Пропозициональное сопутствование&42&---&\hphantom{9}0\\
\hline
Уступительные&42&\tabcolsep=0pt\begin{tabular}{c}Concession.Arg1-as-denier; 
\\Concession.Arg2-as-denier\end{tabular}&39\\
\hline
\tabcolsep=0pt\begin{tabular}{c}Условные\\ Метаязыковые 
условные\end{tabular}&\tabcolsep=0pt\begin{tabular}{c}36\\ \hphantom{9}1 
\end{tabular}&\tabcolsep=0pt\begin{tabular}{c}Condition.Arg1-as-cond; \\Condition.Arg2-as-cond\end{tabular}&74\\
\hline
Коррекция&32&---&\hphantom{9}0\\
\hline
Пропозициональная причина&32&Cause.Reason&281\hphantom{9}\\
\hline
Сравнительные&31&Similarity&63\\
\hline
Неединственности&25&---&\hphantom{9}0\\
\hline
Аналогия&18&---&\hphantom{9}0\\
\hline
Иллокутивная причина&16&Cause+Belief.Reason+Belief&\hphantom{9}6\\
\hline
Иллокутивное сопутствование&16&---&\hphantom{9}0\\
\hline
\tabcolsep=0pt\begin{tabular}{c}Пропозициональная альтернатива\\ Гипотетическая альтернатива\end{tabular}
&\tabcolsep=0pt\begin{tabular}{c}15\\ \hphantom{9}1 \end{tabular}&Disjunction&\hphantom{9}0\\
\hline
\tabcolsep=0pt\begin{tabular}{c}Экстенсиональная генерализация\\ Интенсиональная генерализация\end{tabular}
&\tabcolsep=0pt\begin{tabular}{c} \hphantom{9}7\\ \hphantom{9}5 \end{tabular}&Instantiation.Arg1-as-instance&\hphantom{9}1\\
\hline
Тождество&11&---&\hphantom{9}0\\
\hline
Несоответствие&\hphantom{9}6&---&\hphantom{9}0\\
\hline
Обобщающее переформулирование&\hphantom{9}6&Level-of-detail.Arg1-as-detail&16\\
\hline
\tabcolsep=0pt\begin{tabular}{c}Отрицательная альтернатива\\ Оговорка\end{tabular}&
\tabcolsep=0pt\begin{tabular}{c}\hphantom{9}4\\ \hphantom{9}2\end{tabular}&
%\tabcolsep=0pt\begin{tabular}{l}
Negative-condition.Arg2-as-negCond%\end{tabular}
&\hphantom{9}2\\
\hline
Следствие&\hphantom{9}4&\tabcolsep=0pt\begin{tabular}{c}Cause.Result;  Cause.negResult;\\
Cause+Belief.Result+Belief\end{tabular}&663\hphantom{9}\\
\hline
Уступительные иллокутивные&\hphantom{9}4&
%\tabcolsep=0pt\begin{tabular}{l}
Concession+SpeechAct.Arg2-as-denier+SpeechAct%\end{tabular}
&\hphantom{9}1\\
\hline
Отрицание тождества&\hphantom{9}1&---&\hphantom{9}0\\
\hline
Цель&\hphantom{9}0&\tabcolsep=0pt\begin{tabular}{c}Purpose.Arg1-as-goal; \\Purpose.Arg2-as-goal\end{tabular}&35\\
\hline
---&\hphantom{9}0&\tabcolsep=0pt\begin{tabular}{c}Manner.Arg1-as-manner;\\ Manner.Arg2-as-manner\end{tabular}&\hphantom{9}3\\
\hline
&988\hphantom{9}&&1672\hphantom{99}\\
\hline
\end{tabular}
\end{center}
\end{table*}

\section{Заключительные замечания}

Таким образом, исследование ЛСО с~использованием параллельных текстов 
и~аннотирование их показателей в~НБД позволяет, во-пер\-вых, извлекать 
новое знание о~средствах выражения ЛСО (в~том числе альтернативных 
коннекторам) и~создавать тезаурусы таких средств в~изучаемых языках;  
во-вто\-рых, на основе информации, хранящейся в~НБД, получать новые 
знания о~том, какие ЛСО могут быть выражены неспециализированными 
средствами и~какова частность их использования для каждого ЛСО в~каждом 
из изучаемых языков. Все это способно улучшить работу дискурсивных 
парсеров за счет пополнения спектра признаков, на основании которых 
принимаются решения о наличии того или иного ЛСО между фрагментами 
текста.

{\small\frenchspacing
{%\baselineskip=10.8pt
%\addcontentsline{toc}{section}{References}
\begin{thebibliography}{99}
\bibitem{1-in}
\Au{Hobbs J.\,R.} A~computational approach to discourse analyses.~--- 
New York, NY, USA: Department of Computer Science, City College, City University of New 
York, 1976.  Research Report 76-2.
\bibitem{2-in}
\Au{Hobbs J.\,R.} Why is discourse coherent?~--- Menlo Park, CA, 
USA: SRI International, 1978.  SRI Technical Note~176.
\bibitem{3-in}
\Au{Danlos L., Rysov$\acute{\mbox{a}}$~K., Rysov$\acute{\mbox{a}}$~M., Stede~M.} Primary 
and secondary discourse connectives: Definitions and lexicons~// Dialogue Discourse, 2018. 
Vol.~9. No.\,1. P.~50--78.
\bibitem{4-in}
\Au{Chistova~E.\,V., Shelmanov~A.\,O., Kobozeva~M.\,V., Pisarevskaya~D.\,B., Smirnov~I.\,V., 
Toldova~S.\,Yu.} Classification models for RST discourse parsing of texts in Russian~// 
Компьютерная лингвистика и~интеллектуальные технологии: По мат-лам ежегодной 
Междунар. конф. <<Диалог>>.~--- М.: РГГУ, 2019. Вып.~18(25). С.~163--176.
\bibitem{5-in}
\Au{Taboada M.} Discourse markers as signals (or not) of rhetorical relations~// J.~Pragmatics, 
2006. Vol.~38. No.\,4. P.~567--592.
\bibitem{6-in}
\Au{Гончаров А.\,А., Инькова~О.\,Ю.} Имплицитные логико-се\-ман\-ти\-че\-ские отношения 
и~метод их поиска в~параллельных текстах~// Компьютерная лингвистика 
и~интеллектуальные технологии: По мат-лам ежегодной Междунар. конф.  
<<Диалог>>.~--- М.: РГГУ, 2020. Вып.~19(26). С.~310--320.
\bibitem{7-in}
\Au{Зацман И.\,М., Инькова~О.\,Ю., Кружков~М.\,Г., Попкова~Н.\,А.} Представление 
кроссязыковых знаний о~коннекторах в~надкорпусных базах данных~// Информатика и~её 
применения, 2016. Т.~10. Вып.~1. С.~106--118.
\bibitem{8-in}
\Au{Зацман И., Кружков~М., Лощилова~Е.} Методы и~средства информатики для 
описания структуры неоднословных коннекторов~// Структура коннекторов и~методы ее 
описания~/ Под ред. О.\,Ю.~Иньковой.~--- М.: ТОРУС ПРЕСС, 2019. С.~205--230.
\bibitem{9-in}
Семантика коннекторов: количественные методы описания~/
Под ред.\ О.~Иньковой.~--- Bern/Berlin: Peter Lang, 2021. 276~с.
\bibitem{10-in}
\Au{Prasad R., Joshi~A., Webber~B.} Realization of discourse relations by other means: 
Alternative lexicalizations~// 23rd Conference (International) on Computational 
Linguistics Proceedings.~--- 
Beijing, China, 2010. P.~1023--1031. {\sf https://www.aclweb.org/anthology/C10-2118.pdf}.
\bibitem{11-in}
Penn Discourse Treebank Project (PDTB). {\sf https://www.\linebreak seas.upenn.edu/$\sim$pdtb}.
\bibitem{12-in}
\Au{Rysov$\acute{\mbox{a}}$~M., Rysov$\acute{\mbox{a}}$~K.} The centre and periphery of 
discourse connectives~// 28th Pacific Asia Conference on Language, Information and 
Computing Proceedings.~--- Phuket: Department of Linguistics, Chulalongkorn University, 
2014. P.~452--459. {\sf https://www.aclweb.org/ anthology/Y14-1052.pdf}.
\bibitem{13-in}
Ru-RSTreebank. Русскоязычный дискурсивный корпус. {\sf https://rstreebank.ru}.
\bibitem{14-in}
\Au{Webber B., Prasad~R., Lee~A., Joshi~A.} The Penn Discourse Treebank~3.0: Annotation 
Manual, 2019. {\sf https://\linebreak catalog.ldc.upenn.edu/docs/LDC2019T05/PDTB3-Annotation-Manual.pdf}.
\bibitem{15-in}
\Au{Das D., Taboada~M.} RST Signalling Corpus: Annotation Manual, 2014. {\sf 
https://www.sfu.ca/$\sim$mtaboada/docs/\linebreak publications/RST\_Signalling\_Corpus\_Annotation\_\linebreak Manual.pdf}.
\bibitem{16-in}
\Au{Johansson S.} Seeing through multilingual corpora: On the use of corpora in contrastive 
studies.~--- Amsterdam/Philadelphia: John Benjamins, 2007. 377~p.
\bibitem{17-in}
\Au{Инькова О.\,Ю.} Аннотирование параллельных текстов: понятие <<дивергентный 
перевод>>~// Компьютерная лингвистика и~интеллектуальные технологии: По мат-лам 
ежегодной Междунар. конф. <<Диалог>>.~--- М.: РГГУ, 2019. Вып.~18(25). С.~227--238.
\bibitem{18-in}
\Au{Зацман И.\,М., Инькова~О.\,Ю., Нуриев~В.\,А.} По\-стро\-ение классификационных схем: 
методы и~технологии экспертного формирования~// На\-уч\-но-тех\-ни\-че\-ская 
информация. Сер.~2: Информационные процессы и~сис\-те\-мы, 2017. №\,1. С.~8--22.
 \end{thebibliography}

}
}

\end{multicols}

\vspace*{-3pt}

\hfill{\small\textit{Поступила в~редакцию 06.04.2021}}

%\vspace*{8pt}

%\pagebreak

\newpage

\vspace*{-28pt}

%\hrule

%\vspace*{2pt}

%\hrule

%\vspace*{-2pt}

\def\tit{EXTRACTING KNOWLEDGE ABOUT~MEANS\\ OF~EXPRESSION 
 OF~LOGICAL-SEMANTIC RELATIONS\\ FROM~THE~SUPRACORPORA DATABASE}


\def\titkol{Extracting knowledge about~means of~expression 
 of~logical-semantic relations from~the~supracorpora database}

\def\aut{A.\,A.~Goncharov and~O.\,Yu.~Inkova}

\def\autkol{A.\,A.~Goncharov and~O.\,Yu.~Inkova}


\titel{\tit}{\aut}{\autkol}{\titkol}

\vspace*{-11pt}




\noindent
Institute of Informatics Problems, Federal Research Center ``Computer Science and Control''
 of the Russian Academy of Sciences, 44-2~Vavilov Str., Moscow 119333, Russian Federation

 
\def\leftfootline{\small{\textbf{\thepage}
\hfill INFORMATIKA I EE PRIMENENIYA~--- INFORMATICS AND
APPLICATIONS\ \ \ 2021\ \ \ volume~15\ \ \ issue\ 2}
}%
\def\rightfootline{\small{INFORMATIKA I EE PRIMENENIYA~---
INFORMATICS AND APPLICATIONS\ \ \ 2021\ \ \ volume~15\ \ \ issue\ 2
\hfill \textbf{\thepage}}}

\vspace*{3pt}




\Abste{The goal of this paper is to demonstrate how parallel texts annotated with a supracorpora 
database (SCDB) can be efficiently used to extract knowledge about alternative means of 
expression of logical-semantic relations (LSR). The authors review the most prominent 
discursively annotated corpora (Penn Discourse Treebank, Prague Dependency Treebank, 
and Rhetorical Structure Theory Discourse Treebank) to support the observation that 
there is no consensus among the researchers as to which linguistic means are to be considered 
connectives (i.\,e., prototypical markers of LSR) and which means are 
deemed ``alternative.'' The research shows that application of the comparative method while 
leveraging the capabilities of the SCDB of connectives makes it possible not only to extract new 
knowledge about LSR markers but also to create thesauri of various means of LSR expression in 
the languages involved, including the alternative ones. In addition, the SCDB data makes it 
possible to generate new knowledge on correlations between specific LSRs and unconventional 
means of LSR expression and calculate frequencies of utilization of these means for the studied 
languages.}

\KWE{supracorpora database; logical-semantic relations; connectives; knowledge generation; 
parallel texts}

\DOI{10.14357/19922264210214}

%\vspace*{-15pt}

% \Ack
%\noindent


\vspace*{3pt}

  \begin{multicols}{2}

\renewcommand{\bibname}{\protect\rmfamily References}
%\renewcommand{\bibname}{\large\protect\rm References}

{\small\frenchspacing
 {%\baselineskip=10.8pt
 \addcontentsline{toc}{section}{References}
 \begin{thebibliography}{99}
 
 \vspace*{-3pt}
 
\bibitem{1-in-1}
\Aue{Hobbs, J.\,R.} 1976. A~computational approach to discourse analyses.  New York, NY: Department of Computer Science, City College, City University of New 
York.  Research Report  
76-2.
\bibitem{2-in-1}
\Aue{Hobbs, J.\,R.} 1978. Why is discourse coherent?  Menlo Park, 
CA: SRI International. SRI Technical Note~176.
\bibitem{3-in-1}
\Aue{Danlos, L., K.~Rysov$\acute{\mbox{a}}$, M.~Rysov$\acute{\mbox{a}}$, and 
M.~Stede.} 2018. Primary and secondary discourse connectives: Definitions and lexicons. 
\textit{Dialogue Discourse} 9(1):50--78.
\bibitem{4-in-1}
\Aue{Chistova, E.\,V., A.\,O.~Shelmanov, M.\,V.~Kobozeva, D.\,B.~Pisarevskaya, 
I.\,V.~Smirnov, and S.\,Yu.~Toldova.} 2019. Classification models for RST discourse parsing of 
texts in Russian. \textit{Komp'yuternaya lingvistika i~intellektual'nye tekhnologii: po mat-lam 
Mezhdunar. konf. ``Dialog''} [Computational Linguistics and Intellectual Technologies: Papers 
from the Annual Conference (International) ``Dialogue'']. Moscow: RSHI. 18(25):163--176.
\bibitem{5-in-1}
\Aue{Taboada, M.} 2006. Discourse markers as signals (or not) of rhetorical relations. 
\textit{J.~Pragmatics} 38(4):567--592.
\bibitem{6-in-1}
\Aue{Goncharov, A.\,A., and O.\,Yu.~Inkova.} 2020. Implitsitnye logiko-semanticheskie 
otnosheniya i~metod ikh poiska v~parallel'nykh tekstakh [Implicit logical-semantic relations and 
a~method of their identification in parallel texts]. \textit{Komp'yuternaya lingvistika 
i~intellektual'nye tekhnologii: po mat-lam Mezhdunar. konf. ``Dialog''} [Computational 
Linguistics and Intellectual Technologies: Papers from the Annual Conference (International) 
``Dialogue'']. Moscow: RSHI. 19(26):310--320.
\bibitem{7-in-1}
\Aue{Zatsman, I.\,M., O.\,Yu.~Inkova, M.\,G.~Kruzhkov, and N.\,A.~Popkova.} 2016. 
Predstavlenie kross-yazykovykh znaniy o~konnektorakh v~nadkorpusnykh bazakh dannykh 
[Representation of cross-lingual knowledge about connectors in suprocorpora databases]. 
\textit{Informatika i~ee Primeneniya~--- Inform. Appl.} 10(1):106--118.
\bibitem{8-in-1}
\Aue{Zatsman, I., M.~Kruzhkov, and E.~Loshchilova.} 2019. Metody i~sredstva informatiki 
dlya opisaniya struktury neodnoslovnykh konnektorov [Methods and means of informatics for 
multiword connectives structure description]. \textit{Struktura konnektorov i~metody ee 
opisaniya} [Connectives structure and methods of its description]. Ed. O.\,Yu.~Inkova. Moscow: 
TORUS PRESS. 205--230.
\bibitem{9-in-1}
Inkova, O., ed. 2021. \textit{Semantika konnektorov: kolichestvennye metody opisaniya} 
[Semantics of connectives: Quantitative methods of analysis]. Bern/Berlin: Peter Lang. 276~p.
\bibitem{10-in-1}
\Aue{Prasad, R., A.~Joshi, and B.~Webber.} 2010. Realization of discourse relations by other 
means: Alternative lexicalizations. \textit{23rd Conference (International) on Computational 
Linguistics Proceedings}. Beijing, China. 1023--1031.
Available at: {\sf https://www.aclweb.org/anthology/C10-2118.pdf}
(accessed June~15, 2021).
\bibitem{11-in-1}
Penn Discourse Treebank Project. Available at: {\sf https:// www.seas.upenn.edu/$\sim$pdtb/} 
(accessed May~19, 2021).
\bibitem{12-in-1}
\Aue{Rysov$\acute{\mbox{a}}$, M., and K.~Rysov$\acute{\mbox{a}}$.} 2014. The centre and 
periphery of discourse connectives. \textit{28th Pacific Asia Conference on Language, 
Information and Computing Proceedings}. Phuket: Department of Linguistics, Chulalongkorn 
University. 452--459. 
Available at: {\sf https://www.aclweb.org/\linebreak anthology/Y14-1052.pdf}
(accessed June~15, 2021).
\bibitem{13-in-1}
Ru-RSTreebank: Russkoyazychnyy diskursivnyy korpus [Ru-RSTreebank: Russian discourse 
corpus]. Available at: {\sf https://rstreebank.ru/} (accessed May~19, 2021)
\bibitem{14-in-1}
\Aue{Webber, B., R.~Prasad, A.~Lee, and A.~Joshi.} 2019. The Penn Discourse Treebank~3.0: 
Annotation manual. Available at: {\sf  
https://catalog.ldc.upenn.edu/docs/\linebreak LDC2019T05/PDTB3-Annotation-Manual.pdf} (accessed 
May~19, 2021)
\bibitem{15-in-1}
\Aue{Das, D., and M.~Taboada.} 2014. RST signalling corpus: Annotation manual. Available 
at:  {\sf https://www.\linebreak sfu.ca/$\sim$mtaboada/docs/publications/RST\_Signalling\_\linebreak Corpus\_Annotation\_Manual.pdf} 
(accessed May~19, 2021)
\bibitem{16-in-1}
\Aue{Johansson, S.} 2007. \textit{Seeing through multilingual corpora: On the use of corpora in 
contrastive studies}. Amsterdam/Philadelphia: John Benjamins. 377~p.
\bibitem{17-in-1}
\Aue{Inkova, O.\,Yu.} 2019. Annotirovanie parallel'nykh tekstov: ponyatie ``divergentnyy 
perevod'' [Annotation of parallel texts: The concept of divergent translation]. 
\textit{Komp'yuternaya lingvistika i~intellektual'nye tekhnologii: po mat-lam Mezhdunar. konf. 
``Dialog''} [Computational Linguistics and Intellectual Technologies: Papers from the Annual 
Conference (International) ``Dialogue'']. Moscow: RSHI. 18(25):227--238.
\bibitem{18-in-1}
\Aue{Zatsman, I., O.~Inkova, and V.~Nuriev.} 2017. The construction of classification schemes: 
Methods and technologies of expert formation. \textit{Autom. Doc. Math. 
Linguist.} 51(1):27--41.
\end{thebibliography}

 }
 }

\end{multicols}

\vspace*{-3pt}

  \hfill{\small\textit{Received April~6, 2021}}


%\pagebreak

%\vspace*{-8pt}  

\Contr

\noindent
\textbf{Goncharov Alexander A.} (b.\ 1994)~--- junior scientist, Institute of Informatics 
Problems, Federal Research Center``Computer Science and Control'' of the Russian Academy of 
Sciences, 44-2~Vavilov Str., Moscow 119333, Russian Federation; 
\mbox{a.gonch48@gmail.com}

\vspace*{6pt}

\noindent
\textbf{Inkova Olga Yu.} (b.\ 1965)~--- Doctor of Science  in philology, senior scientist, 
Institute of Informatics Problems, Federal Research Center ``Computer Science and Control'' of 
the Russian Academy of Sciences, 44-2~Vavilov Str., Moscow 119333, Russian Federation; 
\mbox{olyainkova@yandex.ru}

\label{end\stat}

\renewcommand{\bibname}{\protect\rm Литература}