\def\stat{kirikov+listopad}

\def\tit{СОГЛАСОВАНИЕ ЦЕЛЕЙ АГЕНТОВ СПЛОЧЕННЫХ ГИБРИДНЫХ 
ИНТЕЛЛЕКТУАЛЬНЫХ МНОГОАГЕНТНЫХ СИСТЕМ$^*$}

\def\titkol{Согласование целей агентов сплоченных гибридных 
интеллектуальных многоагентных систем}

\def\aut{И.\,А.~Кириков$^1$, С.\,В.~Листопад$^2$}

\def\autkol{И.\,А.~Кириков, С.\,В.~Листопад}

\titel{\tit}{\aut}{\autkol}{\titkol}

\index{Кириков И.\,А.}
\index{Листопад С.\,В.}
\index{Kirikov I.\,A.}
\index{Listopad S.\,V.}

{\renewcommand{\thefootnote}{\fnsymbol{footnote}} \footnotetext[1]
{Исследование выполнено при финансовой поддержке РФФИ (проект 20-07-00104а).}}

\renewcommand{\thefootnote}{\arabic{footnote}}
\footnotetext[1]{Калининградский филиал Федерального исследовательского центра <<Информатика и~управление>> 
Российской академии наук, \mbox{baltbipiran@mail.ru}}
\footnotetext[2]{Калининградский филиал Федерального исследовательского центра <<Информатика 
и~управление>> Российской академии наук, \mbox{ser-list-post@yandex.ru}}


%\vspace*{-12pt}

  
  
      
  \Abst{При разработке интеллектуальной системы как сообщества разнородных 
интеллектуальных агентов важна организация их взаимодействия. Для снижения 
трудоемкости этой процедуры предлагается методами сплоченных гибридных 
интеллектуальных многоагентных систем (СГИМАС) моделировать механизмы возникновения 
сплоченности в~коллективах специалистов, решающих проблемы <<за круглым столом>>. 
Агенты таких систем должны быть способны самостоятельно согласовывать цели, модели 
предметной области и~вырабатывать протокол для решения поставленной проблемы. 
В~статье предлагается модель согласования целей агентов \mbox{СГИМАС}.}
  
  \KW{сплоченность; гибридная интеллектуальная многоагентная система; коллектив 
специалистов}
  
\DOI{10.14357/19922264210210}

%\vspace*{-2pt}


\vskip 10pt plus 9pt minus 6pt

\thispagestyle{headings}

\begin{multicols}{2}

\label{st\stat}

\section{Введение}

  Решение проблем малыми коллективами специалистов снижает влияние 
человеческого фактора на вероятность ошибок, обеспечивая возможность 
комплексного, всестороннего рассмотрения проб\-ле\-мы, учета целей различных 
заинтересованных сторон. При этом для обеспечения эффективного 
коллективного решения проб\-лем недостаточно собрать специалистов 
и~обозначить проблему: полярность точек зрений, разнородность знаний, 
отсутствие принятых норм взаимодействия обуслов\-ли\-ва\-ют напряженность 
и~конфликтные ситуации. Для эффективной совместной деятельности малая 
группа в~процессе своего развития должна \mbox{пройти} сложный путь от 
конгломерата незнакомых друг с~другом специалистов без общей цели до 
сплоченного коллектива единомышленников, осуществляющих совместную 
деятельность и~добивающихся результата на основе гармонизации целей, 
интересов и~ценностей~[1].
  
  При моделировании гибридными интеллектуальными многоагентными 
системами (\mbox{ГиИМАС}) совместной работы специалистов по решению 
тех или иных проблем возникают аналогичные сложности. Агенты системы, 
собранные из разных репозиториев и~построенные разными разработчиками, 
могут быть несовместимы по языкам передачи\linebreak сообщений, целям, моделям 
предметной области или протоколам решения проблем. Для преодоления 
данных трудностей в~\cite{2-kir} предложена модель \mbox{СГИМАС}. В~их основу 
положена концепция функциональных гибридных интеллектуальных сис\-тем 
А.\,В.~Колесникова~\cite{3-kir}, обеспечивающая учет проб\-лем\-ной 
неоднородности, многоагентный подход к~построению интеллектуальных 
систем~[4--7], позволяющий имитировать взаимодействие специалистов 
в~коллективе, а также стратометрическая концепция (СК) сплоченности 
А.\,В.~Петровского~\cite{8-kir}, описывающая условия формирования 
сплоченного коллектива агентов, понимающих друг друга, разделяющих общие 
цели и~нормы. Цель настоящей работы~--- разработка модели согласования целей 
агентов \mbox{СГИМАС}.
  
\section{Модель сплоченной гибридной интеллектуальной 
многоагентной системы}

  Предложенная в~\cite{2-kir} \mbox{СГИМАС} моделирует сплоченность 
коллектива специалистов на двух из трех уровней СК 
А.\,В.~Петровского~\cite{8-kir} (из-за отсутствия эмоциональной 
составляющей у агентов уровень эмоциональных межличностных отношений 
СК не рассматривается):
  \begin{enumerate}[(1)]
  \item ценностно-ори\-ен\-та\-ци\-он\-ное единство (ЦОЕ), т.\,е.\ бли\-зость 
основных ценностей и~убеждений, возникающая в~результате совместной 
деятельности;
  \item  ядро~--- единство, обусловленное сходством целей членов коллектива.
  \end{enumerate}
  
  Формально \mbox{СГИМАС}~\cite{2-kir} описывается сле\-ду\-ющим 
выражением:

\vspace*{-6pt}

\noindent
  \begin{multline}
  \mathrm{chimas}={}\\
  {}=\langle \mathrm{AG}, \mathrm{env}, \mathrm{INT}, 
\mathrm{ORG}, \{ \mathrm{glng}, \mathrm{ontng}, \mathrm{protng}\}\rangle\,,
  \end{multline}
  
  \vspace*{-2pt}
  
  \noindent
где $\mathrm{AG}$~--- множество агентов системы, опи\-сы\-ва\-емых 
выражением~(2), включающее подмножество 
$\mathrm{AG}^{\mathrm{sp}}\hm\subseteq \mathrm{AG}$ 
 аген\-тов-спе\-ци\-а\-лис\-тов, моделирующих знания и~рассуждения 
специалистов~--- членов коллектива, агента 
$\mathrm{ag}^{\mathrm{dm}}\hm\in \mathrm{AG}$, принимающего решения 
(АПР), аген\-та-фа\-си\-ли\-та\-то\-ра (АФ) $\mathrm{ag}^{\mathrm{fc}}\hm\in 
\mathrm{AG}$,\linebreak
 отвечающего за организацию эффективного взаимодействия 
агентов системы и~формирование сплоченности, а~также служебных агентов, 
обеспечивающих взаимодействие агентов между собой;\linebreak $\mathrm{env}$~--- 
концептуальная модель внешней среды \mbox{СГИМАС}; $\mathrm{INT}$~--- 
множество элементов для структурирования взаимодействий агентов~(3); 
$\mathrm{ORG}$~--- множество архитектур \mbox{СГИМАС};\linebreak 
$\{\mathrm{glng}, \mathrm{ontng}, \mathrm{protng}\}$~--- множество моделей 
макроуровневых процессов, содержащее модель $\mathrm{glng}$ согласования 
целей агентов, обеспечивающую сплочен\-ность на уровне ядра СК, 
описываемую выражением~(\ref{e5-kir}), модель $\mathrm{ontng}$ 
согласования онтологий (моделей предметной об\-ласти) агентов, 
соответствующая обмену знаниями, опытом и~убеждениями между членами 
коллектива и~формированию сплоченности на уровне ЦОЕ, и~модель 
$\mathrm{protng}$ построения агентами протокола сплоченного решения 
проблем, имитирующая выработку и~интериоризацию членами коллектива 
норм взаимодействия на уровне ЦОЕ. 
  
  Агент $\mathrm{ag}\hm\in \mathrm{AG}$ из формулы~(1) описывается 
выражением:
  \begin{equation}
 \! \mathrm{ag}=\langle \mathrm{id^{ag}}, \mathrm{gl^{ag}}, 
\mathrm{LANG^{ag}}, \mathrm{ont^{ag}}, \mathrm{ACT^{ag}}, 
\mathrm{prot^{ag}}\rangle,\!\!
  \label{e2-kir}
  \end{equation}
где $\mathrm{id^{ag}}$~--- идентификатор (имя) агента;  
$\mathrm{gl^{ag}}$~--- цель агента в~виде нечеткого множества с~функцией 
принадлежности $\mu_{\mathrm{id}} 
(\mathrm{pr}^{\mathrm{cs}}_{\mathrm{id}\,1},\ldots 
\mathrm{pr}^{\mathrm{cs}}_{\mathrm{id}\,\mathrm{Nprid}})$, за\-данной на подмножестве  
кон\-цеп\-тов-свойств $\mathrm{PR_{id}^{cs}}\hm= \{ 
\mathrm{pr^{cs}_{id\,1}},\ldots , \mathrm{pr}^{\mathrm{cs}}_{\mathrm{id}\,\mathrm{Nprid}}\}$ множества 
концептов $\mathrm{PR_{id}^{cs}}\hm\subseteq \mathrm{PR_{id}}\hm\subseteq 
C_{\mathrm{id}}$ онтологии агента~$\mathrm{ont^{ag}}$; 
$\mathrm{LANG^{ag}}\hm\subseteq \mathrm{LANG}$~--- множество языков 
передачи сообщений, которыми <<владеет>> агент; $\mathrm{ont^{ag}}$~--- 
модель предметной области (онтология) агента, описываемая 
выражением~(\ref{e4-kir}); $\mathrm{ACT^{ag}}$~--- множество действий, 
реализуемых агентом; $\mathrm{prot^{ag}}$~--- модель протокола решения 
проблемы, разрабатываемая агентом в~процессе взаимодействия с~другими 
агентами~\cite{4-kir}. 
  
  Множество элементов для структурирования взаимодействий агентов из 
формулы~(1) описывается выражением:
  \begin{equation}
  \mathrm{INT}= \left\{ \mathrm{prot^{bsc}}, \mathrm{PRC}, \mathrm{LANG}, 
\mathrm{ont^{bsc}}, \mathrm{chn}\right\}\,,
  \label{e3-kir}
  \end{equation}
  
  \vspace*{-2pt}
  
  \noindent
где $\mathrm{prot^{bsc}}$~--- базовый протокол, обеспечивающий 
взаимодействие агентов по формированию протокола сплоченного 
взаимодействия для решения поставленных перед \mbox{СГИМАС} проб\-лем; 
$\mathrm{PRC}$~--- множество элементов для конструирования протокола 
решения проблем аген\-та\-ми-спе\-ци\-а\-лис\-та\-ми и~АПР; 
$\mathrm{LANG}$~--- множество языков передачи сообщений агентов; 
$\mathrm{ont^{bsc}}$~--- базовая онтология, обеспечивающая интерпретацию 
агентами семантики сообщений по согласованию собственных целей и~моделей 
предметной области, формированию протокола сплоченного взаимодействия, 
описываемая выражением~(\ref{e4-kir}); $\mathrm{chn}$~--- степень 
сплоченности агентов~\cite{2-kir}, опи\-сы\-ва\-ющая степень сходства целей 
и~онтологий, а также согласованности протокола решения проблем.
  
  Модели онтологий агентов $\mathrm{ont^{ag}}$ и~базовой 
онтологий~$\mathrm{ont_{bsc}}$ из выражений~(\ref{e2-kir}) и~(\ref{e3-kir}) 
соответственно описываются следующим образом:

\vspace*{-8pt}

\noindent
  \begin{multline}
  \mathrm{ont}={}\\
  {}=\langle L, C, R, \mathrm{AT, FC, FR, FA}, H^c, H^r, \mathrm{INST} \rangle\,,
  \label{e4-kir}
  \end{multline}
  
  \vspace*{-2pt}
  
  \noindent
где $L=L^c\cup L^r\cup L^{\mathrm{at}}\cup L^{\mathrm{va}}$~--- лексикон, множество лексем, 
состоящее из подмножеств лексем, обозначающих понятия~$L^c$, 
отношения~$L^r$, атрибуты~$L^{\mathrm{at}}$ и~их значения~$L^{\mathrm{va}}$; $C$~--- 
множество концептов (понятий); $R: C\times C$~--- множество отношений 
между\linebreak концептами, первая компонента кортежа от\-но\-шения называется 
доменом $\mathrm{dm}\,(r)\hm= \mathrm{Пр}_1(r)$,\linebreak а~вторая~--- диапазоном значений отношения 
$\mathrm{rn}\,(r)\hm=\mathrm{Пр}_2(r)$; $\mathrm{AT}: C\times L^{\mathrm{va}}$~--- множество 
атрибутов концептов; $\mathrm{FC}: 2^{L^c}\hm\to 2^C$~--- функция связи лексикона 
с~концептами; $\mathrm{FR}: 2^{L^r}\hm\to 2^R$~--- функция связи лексикона 
с~отношениями; $\mathrm{FA}: L^{\mathrm{at}}\hm\to \mathrm{AT}$~--- функция связи лексикона с~атрибутами; $H^c\hm= C\times C$~--- таксономическая иерархия концептов; 
$H^r\hm= R\times R$~--- иерархия отношений; $\mathrm{INST}$~--- множество 
экземпляров, концептов единичного объема~\cite{9-kir}. Функции~$\mathrm{FC}$ 
и~$\mathrm{FR}$ предполагают, что в~общем случае одна лексема может соответствовать 
нескольким концептам или отношениям и,~наоборот, один концепт или 
отношение может описываться несколькими лексемами.

\vspace*{-5pt}
  
\section{Модель согласования целей агентов}

\vspace*{-2pt}

  Модель согласования целей агентов описывается выражением:
  
  \noindent
  \begin{equation}
  \mathrm{glng} =\langle \mathrm{glest}, \mathrm{glngn}, 
\mathrm{GLNM}\rangle\,,
  \label{e5-kir}
  \end{equation}
где $\mathrm{glest}$~--- модель оценки сходства целей; $\mathrm{glngn}$~--- 
модель оценки необходимости согласования целей; $\mathrm{GLNM}\hm= \{ 
\mathrm{glneg}, \mathrm{glarg}, \mathrm{gldmo}\}$~--- множество методов 
согласования, например путем споров $\mathrm{glarg}$, переговоров  
$\mathrm{glneg}$ или на основе распоряжений АПР $\mathrm{gldmo}$.

  Модель оценки сходства целей пары агентов $\mathrm{ag}_i$ 
и~$\mathrm{ag}_j$, которая подробно рассмотрена в~\cite{10-kir}, может быть 
представлена следующим выражением:
  \begin{multline*}
  \!\!\mathrm{glest} =r_1^{\mathrm{act\,act}} \left(\mathrm{act^{ag}_{cm}}, 
\mathrm{act^{ag}_{cu}}\right) \circ 
r_1^{\mathrm{act\,act}}\left(\mathrm{act^{ag}_{cu}}, \mathrm{act^{ag}_{cvr}}\right) \circ\\
\circ   r_1^{\mathrm{act\,act}}\left(\mathrm{act^{ag}_{cvr}}, 
\mathrm{act^{ag}_{gsmc}}\right)\circ r_1^{\mathrm{act\,pr}}\left(\mathrm{act^{ag}_{cm}}, 
\mathrm{PR}_i^{\mathrm{cs}}\right)\circ\\
\circ  r_1^{\mathrm{act}\,c} 
\left(\mathrm{act^{ag}_{cm}}, C_j\right) \circ 
r_2^{\mathrm{act\,res}}\left(\mathrm{act_{cm}^{ag}}, \mathrm{MP}_{i\,j}\right)\circ\\
  \circ r_1^{\mathrm{act\,res}}\left(\mathrm{act^{ag}_{cu}}, \mathrm{MP}_{i\,j}\right)\circ
  r_1^{\mathrm{act\,res}}\left(\mathrm{act_{cu}^{ag}}, 
\mathrm{ont}_i^{\mathrm{ag}}\right) \circ \\
\circ
r_1^{\mathrm{act\,res}}\left(\mathrm{act^{ag}_{cu}}, \mathrm{ont}_j^{\mathrm{ag}}\right)\circ 
  r_2^{\mathrm{act\,res}}\left(\mathrm{act_{cu}^{ag}}, \mathrm{MP}_{i\,j}^{\prime\prime} 
\right)\circ\\
\circ r_1^{\mathrm{act\,res}}\left(\mathrm{act^{ag}_{cvr}}, 
\mathrm{MP}_{i\,j}^{\prime\prime}\right)\circ 
r_1^{\mathrm{act\,res}}\left(\mathrm{act^{ag}_{cvr}}, \mu_i\right)\circ \\
\circ
r_1^{\mathrm{act\,res}}\left(\mathrm{act^{ag}_{cvr}}, \mu_j\right)\circ
r_2^{\mathrm{act\,res}}\left(\mathrm{act^{ag}_{cvr}}, 
\mu_i^\prime\right) \circ\\
\circ
 r_2^{\mathrm{act\,res}}\left(\mathrm{act^{ag}_{cvr}},\mu_j^\prime\right) \circ \\
 \circ
r_1^{\mathrm{act\,res}}\left(\mathrm{act^{ag}_{gsmc}},\mu_i^\prime\right)\circ 
r_1^{\mathrm{act\,res}}\left (\mathrm{act^{ag}_{gsmc}},\mu_j^\prime\right)\circ\\
  \circ r_1^{\mathrm{act\,res}}\left(\mathrm{act^{ag}_{gsmc}}, 
\mathrm{MP}_{i\,j}^{\prime\prime}\right)\circ 
r_2^{\mathrm{act\,res}}\left(\mathrm{act^{ag}_{gsmc}}, 
\mathrm{gls}_{i\,j}^{\mathrm{ag}}\right)\,,
  \end{multline*}
где $r_1^{\mathrm{act\,act}}$~--- отношение <<следование>> типа  
<<дей\-ст\-вие--дей\-ст\-вие>>~\cite{3-kir}; $\mathrm{act^{ag}_{cm}}$~--- действие по 
уста\-нов\-ле\-нию соответствия~$\mathrm{MP}_{i\,j}$ (сходство пары концептов определяется как среднее 
геометрическое мер лексикографического и~таксономического сходства~\cite{11-kir}) между  
кон\-цеп\-та\-ми-свойст\-ва\-ми~$\mathrm{PR}_i^{\mathrm{cs}}$, на которых определена нечеткая цель 
агента~$\mathrm{ag}_i$, и~концептами~$C_j$ онтологии агента~$\mathrm{ag}_j$, поскольку 
в~онтологиях агентов идентификаторы  
кон\-цеп\-тов-свойств, на которых определены цели, и~их число могут различаться;   
$\mathrm{act_{cu}^{ag}}$~---
действие по выявлению, объединению и~сокращению функционально зависимых концептов 
в~соответствии~$\mathrm{MP}_{i\,j}$, в~результате чего строится модифицированное 
соответствие~$\mathrm{MP}_{i\,j}^{\prime\prime}$ независимых концептов обеих онтологий, на которых 
определены нечеткие цели агентов~$\mathrm{ag}_i$ и~$\mathrm{ag}_j$; 
$\mathrm{act^{ag}_{cvr}}$~--- действие по замене переменных в~функциях 
принадлежности~$\mu_i$ и~$\mu_j$ нечетких целей агентов, в~результате чего формируются 
модифицированные функции принадлежности~$\mu_i^\prime$ и~$\mu_j^\prime$; 
$\mathrm{act^{ag}_{gsmc}}$~--- действие по расчету значения меры 
сходства~$\mathrm{gls}^{\mathrm{ag}}_{i\,j}$ нечетких целей~\cite{10-kir} с~учетом степени сходства  
кон\-цеп\-тов-свой\-ств, на которых они определены; $r_1^{\mathrm{act\,pr}}$~--- отношение 
<<иметь аргументом>> типа  
<<дей\-ст\-вие--свой\-ст\-во>>; $r_1^{\mathrm{act}\,c}$~--- отношение <<иметь аргументом>> 
типа <<дей\-ст\-вие--кон\-цепт>>; $r_2^{\mathrm{act\,res}}$~--- отношение <<иметь 
результатом>> типа <<дей\-ст\-вие--ре\-сурс>>; $r_1^{\mathrm{act\,res}}$~--- отношение 
<<иметь аргументом>> типа  
<<дей\-ст\-вие--ре\-сурс>>;   $\circ$~--- операция склеивания концептов~\cite{3-kir}.
  
  Необходимость согласования целей агентов оценивается АФ в~соответствии 
со своей нечеткой базой знаний, представленной в~\cite{12-kir}. Она позволяет 
АФ организовать работу агентов системы в~соответствии с~моделью ромба 
группового принятия решений С.~Кейнера~\cite{13-kir}, содержащей три 
последовательные фазы: дивергентное коллективное мышление, в~ходе 
которого вырабатываются альтернативные решения проб\-ле\-мы, стадию 
бурления, на которой необходимо повышать <<взаимопонимание>> между 
агентами, сближать их цели, модели предметной об\-ласти и~вырабатывать 
согласованный протокол решения поставленной проб\-ле\-мы,\linebreak
 и~стадию 
конвергентного мышления, когда предложенные альтернативы 
классифицируются, ранжируются и~дорабатываются для принятия 
интегрированного, устраивающего всех агентов \mbox{решения}. Модель оценки 
необходимости согласования целей агентов описывается выражением:
  \begin{multline*}
  \mathrm{glngn}=r_1^{\mathrm{act\,act}}\left(\mathrm{act^{ag}_{dmsa}}, 
\mathrm{act^{ag}_{gnmc}}\right)\circ \\
\circ
r_1^{\mathrm{act\,res}}\left(\mathrm{act^{ag}_{dmsa}}, 
\mathrm{MSG^{sol}}\right)\circ 
r_1^{\mathrm{act\,res}}\left(\mathrm{act^{ag}_{dmsa}}, 
\mathrm{GL^{ag}}\right)\circ\\
  \circ r_1^{\mathrm{act\,st}} \left(\mathrm{act^{ag}_{dmsa}}, \mathrm{pss}^{\mathrm{it}}\right)\circ 
r_2^{\mathrm{act\,st}}\left(\mathrm{act^{ag}_{dmsa}}, \mathrm{pss}^{\mathrm{it}+1}\right)\circ \\
\circ
r_1^{\mathrm{act\,st}}\!\left(\mathrm{act^{ag}_{gnmc}}, \mathrm{pss}^{\mathrm{it}+1}\right)\circ
 r_1^{\mathrm{act\,res}}\!\left(\mathrm{act^{ag}_{gnmc}}, 
\mathrm{GLNM}\right) \circ\\
\circ
 r_1^{\mathrm{act\,res}} \!
\left(\mathrm{act^{ag}_{gnmc}}, \mathrm{glnm}^{\mathrm{it}}\!\right)\circ 
r_2^{\mathrm{act\,res}}\!\left(\mathrm{act^{ag}_{gnmc}}, 
\mathrm{glnm}^{\mathrm{it}+1}\!\right)\!,\hspace*{-8.67668pt}
  \end{multline*}
где $\mathrm{act^{ag}_{dmsa}}$~--- действие по вычислению нечеткой 
переменной <<состояние процесса коллективного решения проблемы>> 
$\mathrm{pss}^{\mathrm{it}+1}$ на основе множества $\mathrm{MSG^{sol}}\hm\subseteq 
\mathrm{MSG} \hm\subseteq \mathrm{INST}$ сообщений~--- решений 
проблемы или ее частей, степени близости целей агентов $\mathrm{GL^{ag}}$ 
и~значения переменной на предыду\-щей итерации $\mathrm{pss}^{\mathrm{it}}$; 
$\mathrm{act^{ag}_{gnmc}}$~--- действие по выбору метода 
$\mathrm{glnm}^{\mathrm{it}+1}$ согласования целей на новой итерации процесса 
решения проблемы из множества $\mathrm{GLNM}$ на основании метода на 
текущей итерации $\mathrm{glnm}^{\mathrm{it}}$ и~вычисленного значения нечеткой 
переменной <<со\-сто\-яние процесса коллективного решения проб\-ле\-мы>> 
$\mathrm{pss}^{\mathrm{it}+1}$; $r_1^{\mathrm{act\,st}}$~--- отношение <<иметь аргументом>> 
типа <<дей\-ст\-вие--со\-сто\-яние>>; $r_2^{\mathrm{act\,st}}$~--- отношение 
<<иметь результатом>> типа <<дей\-ст\-вие--со\-сто\-яние>>.
  
  Методы согласования целей из множества $\mathrm{GLNM}$ описывают 
механизмы изменения целей агентов в~процессе их <<общения>> в~форме 
обмена сообщениями путем споров, переговоров или на основе распоряжений 
АПР. Споры и~переговоры необходимы для повышения спло\-чен\-ности, когда 
знания агентов друг о~друге или о~ре\-ша\-емой проб\-ле\-ме неполны~\cite{14-kir}. 
Корректировка на основе распоряжений АПР имеет серьезный недостаток: АПР 
вмешивается в~систему целеполагания агентов, моделирующих знания 
реальных специалистов по решаемой проблеме, что может привести 
к~нерелевантности предлагаемых \mbox{СГИМАС} решений точкам зрения на 
проблему моделируемых специалистов. В~связи с~этим данный метод может 
быть задействован, только если АПР может получить от агентов достоверные 
сведения об их целях, при этом споры и~переговоры не привели к~желаемому 
результату, т.\,е.\ после проведения споров и~переговоров АФ оценивает 
<<состояние процесса коллективного решения проб\-ле\-мы>> 
$\mathrm{pss}^{\mathrm{it}+1}$ как требующее согласования целей агентов. 
Необходимое условие для выполнения любого из методов согласования целей 
агентов~--- согласование онтологий агентов в~пределах верхних котопий 
кон\-цеп\-тов (множества кон\-цеп\-тов, содержащего все вышележащие кон\-цеп\-ты по 
таксономической иерархии кон\-цеп\-тов~$H^c$ по отношению к~заданному 
концепту и~сам концепт~\cite{11-kir}), на которых определены нечеткие цели.
  
  В ходе споров $\mathrm{glarg}$ агенты обмениваются  
со\-об\-ще\-ни\-ями-ар\-гу\-мен\-та\-ми, которые на\-прав\-ле\-ны на изменение 
целевой функции аген\-та-ад\-ре\-са\-та. На основе анализа рассмотренных 
в~\cite{14-kir} типов аргументов, применяемых в~переговорах специалистов, 
для согласования целей агентами \mbox{СГИМАС} предлагается использовать 
сле\-ду\-ющие механизмы аргументации: 
  \begin{itemize}
  \item примеры и~контрпримеры, демонстрирующие противоречие между 
целями аген\-та-ад\-ре\-са\-та и~результатами реализации предыдущих его 
предложений; 
  \item обращение к~<<сложившейся практике>>, демонстрирующее, что 
агенты, ранее выполнявшие в~\mbox{СГИМАС} роль, занимаемую  
аген\-том-ад\-ре\-са\-том, придерживались предлагаемой цели, что 
способствовало высокому качеству коллективных решений;
\item апелляция к~коллективным целям, чтобы убедить аген\-та-ад\-ре\-са\-та, 
что корректировка его цели позволит принять решение, соответствующее 
поставленным перед сис\-те\-мой целям.
\end{itemize}

  
  Получив аргументированное предложение по корректировке своей цели, 
агент-адресат оценивает изложенные аргументы с~использованием функции 
<<анализ аргументов>> и~в~случае согласия с~ними корректирует свою цель 
в~соответствии с~полученным предложением.
  
  Переговоры агентов по поводу согласования целей $\mathrm{glneg}$ 
заключаются в~формировании со\-об\-ще\-ний-за\-про\-сов на корректировку целей 
и~угроз по корректировке собственной цели в~сторону, невыгодную адресату 
сообщения, если предложение будет отвергнуто, или уступок, выгодных для 
адресата, если предложение будет принято. Для этого может использоваться 
метод монотонных минимальных уступок~\cite{15-kir}: агенты поочередно 
от\-прав\-ля\-ют со\-об\-ще\-ния-пред\-ло\-же\-ния, начиная с~самых выгодных для 
себя, и~в~процессе переговоров монотонно отступают от своих 
первоначальных требований. В~результате множество возможных соглашений 
относительно целей агентов оказывается со\-сто\-ящим из всех индивидуально 
рациональных соглашений, эффективных по Парето~\cite{4-kir}.
  
  В случае необходимости корректировки целей на основе распоряжений АПР  
аген\-ты-спе\-циа\-ли\-сты отправляют ему сообщения о~своих целях на 
текущий момент. Агент, принимающий решения, анализирует множество целей агентов, сопоставляя со 
своей, и~формирует распоряжения для каждого из агентов по корректировке 
его функции принадлежности нечеткой цели. Получив такое распоряжение,  
аген\-ты-спе\-циа\-лис\-ты корректируют свои цели без дальнейших 
обсуждений. 
  
  Таким образом, предложенная модель и~методы согласования целей агентов 
\mbox{СГИМАС} позволят снизить интенсивность конфликтов, обуслов\-лен\-ных 
различиями в~целях агентов, созданных разными группами разработчиков 
и~моделирующих различных специалистов по решаемой проб\-ле\-ме. Благодаря 
наличию модели оценки необходимости согласования целей АФ может 
инициировать релевантные ситуации механизмы согласования целей 
и~приостанавливать их использование для предотвращения таких 
нежелательных эффектов от чрезмерной спло\-чен\-ности, как конформизм. 
  
\section{Заключение}

  Рассмотрены особенности распределенной разработки систем гибридного 
  и~синергетического искусственного интеллекта на примере \mbox{ГиИМАС}. С~целью снижения трудозатрат на 
интеграцию автономных час\-тей интеллектуальной сис\-те\-мы предложено 
моделирование механизмов сплочения коллектива, воз\-ни\-ка\-ющих в~длительно 
существующих группах специалистов, решающих практические проб\-ле\-мы <<за 
круглым столом>>. Для этих целей предложено разработать новый класс 
интеллектуальных сис\-тем~--- сплоченные гибридные интеллектуальные 
системы. Рассмотрена модель системы такого класса и~разработана модель 
согласования целей ее агентов. Механизм согласования целей в~ходе решения 
проблемы позволит агенту вырабатывать решения с~учетом не только 
собственных целей, заложенных разработчиками при моделировании знаний 
и~поведения соответствующего специалиста, но и,~хотя бы час\-тич\-но, учитывать 
цели агентов, мо\-де\-ли\-ру\-ющих других специалистов. Это позволит снизить 
вероятность досрочного завершения коллективного решения проб\-ле\-мы 
и~принятия никого не устраивающего решения из-за несовместимости целей 
и~точек зрения на проблему. 
  
{\small\frenchspacing
{%\baselineskip=10.8pt
%\addcontentsline{toc}{section}{References}
\begin{thebibliography}{99}
  \bibitem{1-kir}
  \Au{Почебут Л.\,Г., Чикер~В.\,А.} Организационная социальная психология.~--- СПб.: Речь, 2002. 298~с.
\bibitem{2-kir}
  \Au{Listopad S.} Modeling team cohesion using hybrid intelligent multi-agent systems~// 2nd 
 Conference (International) on Control Systems, Mathematical Modeling, Automation and Energy 
Efficiency Proceedings.~--- Piscataway, NJ, USA: IEEE, 2020. P.~416--421.
\bibitem{3-kir}
  \Au{Колесников А.\,В., Кириков~И.\,А., Листопад~С.\,В.} Гибридные интеллектуальные 
системы с~самоорганизацией: координация, согласованность, спор.~--- М.: ИПИ РАН, 2014. 
189~с.
\bibitem{4-kir}
  \Au{Тарасов В.\,Б.} Агенты, многоагентные системы, виртуальные сообщества: 
стратегическое направление в~информатике и~искусственном интеллекте~// Новости 
искусственного интеллекта, 1998. №\,2. С.~5--63.
\bibitem{5-kir}
  \Au{Городецкий В.\,И., Грушинский~М.\,С., Хабалов~А.\,В.} Многоагентные системы (обзор)~// 
Новости искусственного интеллекта, 1998. №\,2. С.~64--116.
\bibitem{6-kir}
  \Au{Хорошевский В.\,Ф.} Поведение интеллектуальных агентов: модели и~методы 
реализации~// 4-й Междунар. семинар по прикладной семиотике, семиотическому 
и~интеллектуальному управлению: Сб. научных трудов.~--- Пе\-ре\-славль-За\-лес\-ский: РАИИ, 1999. 
С.~5--20.
\bibitem{7-kir}
  \Au{Wooldridge M.} An introduction to multiagent systems.~--- New York, NY, 
  USA: Wiley,  2009. 484~p.
\bibitem{8-kir}
  \Au{Петровский А.\,В.} Опыт построения социально-пси\-хо\-ло\-ги\-че\-ской концепции 
групповой ак\-тив\-ности~// Вопросы психологии, 1973. №\,5. С.~3--17.
\bibitem{9-kir}
  \Au{Крюков К.\,В., Панкова~Л.\,А., Пронина~В.\,А., Суховеров~В.\,С., Шипилина~Л.\,Б.} Меры 
семантической близости в~онтологии~// Проб\-ле\-мы управ\-ле\-ния, 2010. №\,5. C.~2--14.
\bibitem{10-kir}
  \Au{Listopad S.} Estimating of the similarity of agents' goals in cohesive hybrid intelligent 
multi-agent system~// CEUR Workshop Proceedings, 2020. Vol.~2782. P.~180--185.
\bibitem{11-kir}
  \Au{Maedche A., Zacharias~V.} Clustering ontology-based metadata in the semantic web~// 
  Principles of data mining and knowledge discovery~/
  Eds. T.~Elomaa, H.~Mannila, H.~Toivonen.~---
  Lecture notes in artificial intelligence.~--- Springer, 2002. Vol.~2431.
  P.~348--360.
\bibitem{12-kir}
  \Au{Листопад С.\,В., Румовская~С.\,Б.} Нечеткое управление гетерогенным мышлением 
агентов гибридной интеллектуальной многоагентной системы~// Системы и~средства 
информатики, 2020. Т.~30. №\,4. С.~38--49.
\bibitem{13-kir}
  \Au{Kaner S., Lind~L., Toldi~C., Fisk~S., Beger~D.} The facilitator's guide to participatory 
decision-making.~--- San Francisco, CA, USA: Jossey-Bass, 2011. 368~p.
\bibitem{14-kir}
  \Au{Kraus S., Sycara~K., Evenchik~A.} Reaching agreements through argumentation: A~logical 
model and implementation~// Artif. Intell., 1998. Vol.~104. P.~1--60.
\bibitem{15-kir}
  \Au{Rosenshein J., Zlotkin~G.} Rules of encounter: Designing conventions for automated 
negociation among computers.~--- Cambridge, MA, USA: MIT Press, 1994. 253~p.
  \end{thebibliography}

}
}

\end{multicols}

\vspace*{-6pt}

\hfill{\small\textit{Поступила в~редакцию 05.04.2021}}

\vspace*{6pt}

%\pagebreak

%\newpage

%\vspace*{-28pt}

\hrule

\vspace*{2pt}

\hrule

%\vspace*{-2pt}

\def\tit{COORDINATION OF AGENTS' GOALS IN~COHESIVE HYBRID INTELLIGENT MULTIAGENT SYSTEMS}

\def\titkol{Coordination of agents' goals in~cohesive hybrid intelligent multiagent systems}

\def\aut{I.\,A.~Kirikov and S.\,V.~Listopad}

\def\autkol{I.\,A.~Kirikov and S.\,V.~Listopad}

\titel{\tit}{\aut}{\autkol}{\titkol}

\vspace*{-15pt}


\noindent
   Kaliningrad Branch of the Federal Research Center ``Computer Science and Control'' of the Russian 
Academy of Sciences, 5~Gostinaya Str., Kaliningrad 236000, Russian Federation

 
\def\leftfootline{\small{\textbf{\thepage}
\hfill INFORMATIKA I EE PRIMENENIYA~--- INFORMATICS AND
APPLICATIONS\ \ \ 2021\ \ \ volume~15\ \ \ issue\ 2}
}%
\def\rightfootline{\small{INFORMATIKA I EE PRIMENENIYA~---
INFORMATICS AND APPLICATIONS\ \ \ 2021\ \ \ volume~15\ \ \ issue\ 2
\hfill \textbf{\thepage}}}

\vspace*{3pt}     
  
  
    
   \Abste{When developing an intelligent system as a community of heterogeneous intelligent agents, it is 
important to organize their interaction. To reduce the complexity of this procedure, it is proposed to simulate 
with methods of cohesive hybrid intelligent multiagent systems the mechanisms of cohesion emergence in 
teams of specialists solving problems ``at a~round table.'' Agents of such systems should be able to 
independently coordinate their goals and domain models and develop a protocol to solve the posed problem. 
The article proposes a model for coordinating the goals of agents of cohesive hybrid intelligent multiagent 
systems.}
   
   \KWE{cohesion; hybrid intelligent multiagent system; team of specialists}
   
  
   
\DOI{10.14357/19922264210210}

\vspace*{-15pt}

 \Ack
   \noindent
   The reported study was funded by RFBR, project number 20-07-00104а.

%\vspace*{12pt}

  \begin{multicols}{2}

\renewcommand{\bibname}{\protect\rmfamily References}
%\renewcommand{\bibname}{\large\protect\rm References}

{\small\frenchspacing
 {%\baselineskip=10.8pt
 \addcontentsline{toc}{section}{References}
 \begin{thebibliography}{99}
\bibitem{1-kir-1}
  \Aue{Pochebut, L.\,G., and V.\,A.~Chiker.}
   2002. \textit{Organizatsionnaya sotsial'naya psikhologiya} [Organizational 
social psychology]. St.\ Petersburg: Rech. 298~p.
\bibitem{2-kir-1}
  \Aue{Listopad, S.} 2020. Modeling team cohesion using hybrid intelligent multi-agent systems. 
  \textit{2nd Conference (International) on Control Systems, Mathematical Modeling, Automation 
  and Energy Efficiency Proceedings}. Piscataway, NJ: IEEE. 416--421.
\bibitem{3-kir-1}
  \Aue{Kolesnikov, A.\,V., I.\,A.~Kirikov, and S.\,V.~Listopad.} 2014. \textit{Gibridnye intellektual'nye sistemy 
  s~samoorganizatsiey: koordinatsiya, soglasovannost', spor} [Hybrid intelligent systems with 
  self-organization: Coordination, consistency, dispute]. Moscow: IPI RAN. 189~p.
\bibitem{4-kir-1}
  \Aue{Tarasov, V.\,B.} 1998. Agenty, mnogoagentnye sistemy, virtual'nye soobshchestva: strategicheskoe 
napravlenie v~informatike i~iskusstvennom intellekte [The agents, multi-agent system, virtual communities: 
Strategic direction in computer science and artificial intelligence]. 
\textit{Novosti iskusstvennogo intellekta} [News of 
Artificial Intelligence] 2:5--63.
\bibitem{5-kir-1}
  \Aue{Gorodetskiy, V.\,I., M.\,S.~Grushinskiy, and A.\,V.~Khabalov.} 1998. Mnogoagentnye sistemy (obzor) 
[Multi-agent systems (review)]. \textit{Novosti iskusstvennogo intellekta}
 [News of artificial intelligence] 2:64--116.
\bibitem{6-kir-1}
  \Aue{Khoroshevskiy, V.\,F.} 1999. Povedenie intellektual'nykh agentov: modeli i~metody realizatsii [The 
behavior of intelligent agents: Models and methods of implementation]. 
\textit{4th Workshop (International) on 
Applied Semiotics, Semiotics and Intelligent Management Proceedings}. Pereslavl'-Zalesskiy: RAAI. 5--20.
\bibitem{7-kir-1}
  \Aue{Wooldridge, M.} 2009. \textit{An introduction to multiagent systems}. New York, NY: Wiley. 484~p.
\bibitem{8-kir-1}
  \Aue{Petrovskiy, A.\,V.} 1973. Opyt postroeniya sotsial'no-psikhologicheskoy kontseptsii gruppovoy 
aktivnosti [The experience of building a socio-psychological concept of group activity]. 
\textit{Voprosy psikhologii}  [Psychology Issues] 5:3--17.
\bibitem{9-kir-1}
  \Aue{Kryukov, K.\,V., L.\,A.~Pankova, V.\,A.~Pronina, V.\,S.~Sukhoverov, and L.\,B.~Shipilina.} 2010. Mery 
semanticheskoy blizosti v ontologii [Measures of semantic proximity in ontology]. 
\textit{Problemy upravleniya}  [Control Sciences] 5:2--14.
\bibitem{10-kir-1}
  \Aue{Listopad, S.} 2020. Estimating of the similarity of agents' goals in cohesive hybrid intelligent 
  multi-agent system. \textit{CEUR Workshop Proceedings}. 2782:180--185. 
\bibitem{11-kir-1}
  \Aue{Maedche, A., and V.~Zacharias.} 2002. Clustering ontology-based metadata in the semantic web. 
\textit{Principles of data mining and knowledge discovery}. 
Eds.\ T.~Elomaa, H.~Mannila, and H.~Toivonen. Lecture 
notes in artificial intelligence ser. Springer. 2431:348--360.
\bibitem{12-kir-1}
  \Aue{Listopad, S.\,V., and S.\,B.~Rumovskaya.} 2020. Nechetkoe upravlenie geterogennym myshleniem 
agentov gibridnoy intellektual'noy mnogoagentnoy sistemy [Fuzzy control of heterogeneous thinking of the 
hybrid intelligent multiagent system's agents]. \textit{Sistemy i~Sredstva Informatiki~--- Systems and Means of 
Informatics} 30(4):38--49.
\bibitem{13-kir-1}
  \Aue{Kaner, S., L.~Lind, C.~Toldi, S.~Fisk, and D.~Beger.} 2011. \textit{The facilitator's guide to
   participatory decision-making}. San Francisco, CA: Jossey-Bass. 368 p.
\bibitem{14-kir-1}
  \Aue{Kraus, S., K.~Sycara, and A.~Evenchik.} 1998. Reaching agreements through argumentation: A~logical 
model and implementation. \textit{Artif. Intell.} 104:1--60.
\bibitem{15-kir-1}
  \Aue{Rosenshein, J., and G.~Zlotkin.} 1994. \textit{Rules of encounter: Designing conventions for automated 
negotiation among computers}. Cambridge, MA: MIT Press. 253~p.
   \end{thebibliography}

 }
 }

\end{multicols}

\vspace*{-3pt}

  \hfill{\small\textit{Received April~5, 2021}}


%\pagebreak

%\vspace*{-8pt}     
   
   \Contr
   
   \noindent
   \textbf{Kirikov Igor A.} (b.\ 1955)~--- Candidate of  Science (PhD) in technology, director, 
Kaliningrad Branch of the Federal Research Center ``Computer Science and Control'' of the Russian 
Academy of Sciences, 5~Gostinaya Str., Kaliningrad 236000, Russian Federation; 
\mbox{baltbipiran@mail.ru}
   
   \vspace*{3pt}
   
   \noindent
   \textbf{Listopad Sergey V.} (b.\ 1984)~--- Candidate of Science (PhD) in technology, senior scientist, 
Kaliningrad Branch of the Federal Research Center ``Computer Science and Control'' of the Russian 
Academy of Sciences, 5~Gostinaya Str., Kaliningrad 236000, Russian Federation;  
\mbox{ser-list-post@yandex.ru}
   
    
\label{end\stat}

\renewcommand{\bibname}{\protect\rm Литература}