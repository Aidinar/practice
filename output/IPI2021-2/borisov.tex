\newcommand{\pp}[1]{\mathsf{P}\left\{ #1 \right\}}
\newcommand{\me}[2]{\mathsf{E}_{ #1 }\left\{ \mathop{#2} \right\} }




\def\stat{borisov}

\def\tit{ФИЛЬТРАЦИЯ СОСТОЯНИЙ МАРКОВСКИХ СКАЧКООБРАЗНЫХ ПРОЦЕССОВ ПО~КОМПЛЕКСНЫМ НАБЛЮДЕНИЯМ~I: 
ТОЧНОЕ РЕШЕНИЕ ЗАДАЧИ$^*$}

\def\titkol{Фильтрация состояний марковских скачкообразных процессов по~комплексным наблюдениям~I}
%:  точное решение задачи}

\def\aut{А.\,В.~Борисов$^1$, Д.\,Х.~Казанчян$^2$}

\def\autkol{А.\,В.~Борисов, Д.\,Х.~Казанчян}

\titel{\tit}{\aut}{\autkol}{\titkol}

\index{Борисов А.\,В.}
\index{Казанчян Д.\,Х.}
\index{Borisov A.\,V.}
\index{Kazanchyan D.\,Kh.}

{\renewcommand{\thefootnote}{\fnsymbol{footnote}} \footnotetext[1]
{Работа выполнена при частичной поддержке РФФИ (проект 19-07-00187 А) 
и~в~соответствии с~программой Московского центра фундаментальной и~прикладной математики.}}


\renewcommand{\thefootnote}{\arabic{footnote}}
\footnotetext[1]{Институт проблем информатики Федерального исследовательского центра <<Информатика и~управ\-ле\-ние>> 
Российской академии наук; Московской авиационный институт;
Факультет вычислительной математики и~кибернетики Московского государственного университета имени 
М.\,В.~Ломоносова; Центр фундаментальной и~прикладной математики Московского государственного университета 
имени М.\,В.~Ломоносова, \mbox{ABorisov@frccsc.ru}}
\footnotetext[2]{Факультет вычислительной математики и~кибернетики Московского государственного 
университета имени М.\,В.~Ломоносова, \mbox{Drastamat94@gmail.com}}


\vspace*{6pt}


\Abst{Первая часть цикла посвящена решению задачи оптимальной фильтрации состояний  
марковских скачкообразных процессов (МСП) по совокупности наблюдаемых диффузионных и~считающих процессов. 
Интенсивности шумов в~наблюдаемой диффузии и~скачков в~разрывных наблюдениях зависят от оцениваемого 
состояния. Предложено специальное эквивалентное преобразование наблюдений, приводящее их к~совокупности 
диффузионного процесса с~единичной диффузией, множеству считающих процессов и~совокупности косвенных 
наблюдений, выполненных в~неслучайные дискретные моменты времени. Искомая оптимальная оценка 
представима в~форме решения дис\-крет\-но-не\-пре\-рыв\-ной 
стохастической дифференциальной системы с~преобразованными 
наблюдениями в~правой части. Приведено условие идентифицируемости, при выполнении которого состояние 
МСП может быть восстановлено по косвенным зашумленным наблюдениям точно.}

\KW{марковский скачкообразный процесс; оптимальная фильтрация; мультипликативные шумы 
в~наблюдениях; непрерывные и~считающие наблюдения; условия идентифицируемости}

\DOI{10.14357/19922264210202}

\vspace*{8pt}


\vskip 10pt plus 9pt minus 6pt

\thispagestyle{headings}

\begin{multicols}{2}

\label{st\stat}


 \section{Введение}
 
 
 Данный цикл может рассматриваться как продолжение работ~\cite{B_19_1_IA, B_20_2_IA, B_20_3_IA},
 посвященных тео\-ре\-ти\-че\-скому и~практическому аспектам решения задачи оптимальной фильтрации состояний 
 \textit{марковских скачкообразных процессов} по косвенным зашумленным наблюдениям. На 
 этот раз наблюдения пополнены процессами Кокса, интенсивность которых зависит от оцениваемого состояния 
 МСП. Данная работа посвящена тео\-ре\-ти\-че\-ско\-му решению задачи фильтрации.

 Статья организована следующим образом. 
 
 Раздел~2 содержит формальную постановку задачи 
 оптимальной фильтрации. 
 
 В~разд.~3 предложено специальное преобразование наблюдений, позволяющее 
 решить поставленную задачу, а также утверждение, представляющее дис\-крет\-но-не\-пре\-рыв\-ную 
 стохастическую систему, описывающую искомую оценку. 
 
 Заключительные замечания приведены в~разд.~4.

 \section{Постановка задачи фильтрации}
 
 \vspace*{5pt}
 
  На полном вероятностном пространстве с~фильт\-ра\-цией 
  $(\Omega,\mathcal{F},{\sf P},\{\mathcal{F}_t\}_{t \geqslant 0})$ 
  рассматривается стохастическая динамическая система
  
  \vspace*{-6pt}
  
  \noindent
\begin{align}
 X_t &=X_0 + \int\limits_0^t \Lambda^{\top}(s)X_{s}\,ds + \mu_t^X\,;
 \label{eq:stat_1}
 \\[-2pt]
 Y_t &= \int\limits_{0}^{t}f(s)X_s\,ds+\int\limits_{0}^{t} \sum\limits_{n=1}^NX_s^ng_n^{1/2}(s)\,dW_s;
 \label{eq:cobs_1}
 \\[2pt]
 \displaystyle Z_t& = \int\limits_{0}^{t}h(s)X_s\,ds+\mu_t^Z,
 \label{eq:sobs_1}
 \end{align}
 
   \vspace*{-6pt}
   
   \noindent
 где
 
   \vspace*{-2pt}
   
   \noindent
  \begin{itemize}
  \item
  $X_t \triangleq \mathop{\mathrm{col}}(X_t^1,\ldots,X_t^N) \hm\in \mathbb{S}^N$~--- 
  ненаблюдаемое состояние системы~--- МСП с~множеством состояний $\mathbb{S}^N \hm\triangleq 
  \{e_1,\ldots,e_N\}$ ($\mathbb{S}^N$~--- множество единичных векторов пространства~$\mathbb{R}^N$), 
  мат\-рич\-но\-знач\-ной функцией интенсивностей пе-\linebreak\vspace*{-16pt}
  \end{itemize}
  
  \pagebreak
  
\begin{itemize}
  \item[\ ]
  реходов~$\Lambda(t)$ и~начальным распределением~$\pi$;
  $\mu_t^X \hm \triangleq \mathop{\mathrm{col}}(\mu_t^{X_1},\ldots,\mu_t^{X_N})\hm\in \mathbb{R}^N$~--- 
  $\mathcal{F}_t$-со\-гла\-со\-ван\-ный мартингал;
  \item
  $Y_t \triangleq \mathop{\mathrm{col}}(Y_t^1,\ldots,Y_t^M) \hm\in \mathbb{R}^M$~--- косвенные наблюдения,
зашумленные $\mathcal{F}_t$-со\-гла\-со\-ван\-ным стандартным винеровским процессом $W_t 
\hm\triangleq \mathop{\mathrm{col}}(W_t^1,\ldots,W_t^M) \hm\in \mathbb{R}^M$;  
$f(t)$~--- $(M \times N)$-мер\-ная матричнозначная функция плана наблюдений, а~набор $(M\times M)$-мер\-ных 
симметричных матричнозначных функций
  $\{g_n(t)\}_{n=\overline{1,N}}$ характеризует интенсивности шумов 
  в~зависимости от текущего состояния~$X_t$;
  \item
  $Z_t \triangleq \mathop{\mathrm{col}}(Z_t^1,\ldots,Z_t^K) \hm\in \mathbb{R}^K$~--- 
  косвенные наблюдения, компоненты которого являются считающими процессами: 
  элементы \mbox{$(K \times N)$}-мер\-ной матричнозначной функции~$h(t)$ опре-\linebreak деляют интенсивность скачков 
  отдельных\linebreak компонент в~за\-ви\-си\-мости от текущего со\-сто\-яния~$X_t$; 
  $\mu_t^Z \hm\triangleq \mathop{\mathrm{col}}(\mu_t^{Z_1},\ldots,\mu_t^{Z_K})
  \hm\in \mathbb{R}^N$~--- $\mathcal{F}_t$-со\-гла\-со\-ван\-ный мартингал.
  \end{itemize}

  Обозначим через $\{\mathcal{O}_t\}$ поток $\sigma$-ал\-гебр всех наблюдений,
полученных на отрезке времени $[0,t]$, $\mathcal{O}_0 \hm\triangleq \{\varnothing,\Omega\}$, 
а~$\{\overline{\mathcal{O}}_t\}\; (\mathcal{O}_{t+} \hm\triangleq \cap_{s: s\geqslant t} \mathcal{O}_s)$~--- 
вариант данного потока, замкнутый справа.

\textit{Задача оптимальной фильтрации состояния МСП} $X_t$ заключается в~построении 
\textit{условного математического ожидания} (УМО) $\widehat{X}_t 
\hm\triangleq \me{}{X_t|\overline{\mathcal{O}}_t}$.

Ниже представлены ограничения на ис\-сле\-ду\-емую систему наблюдения~(\ref{eq:stat_1})--(\ref{eq:sobs_1}), 
необходимые для корректного решения поставленной задачи фильтрации.
\begin{itemize}
   \item[А.]
  Поток $\sigma$-под\-ал\-гебр~$\mathcal{F}_t$ непрерывен справа.
  Все траектории МСП $\{X_t\}_{t \hm\geqslant 0}$
  непрерывны справа и~имеют конечные пределы слева, т.\,е.\ являются 
  {c$\acute{\mbox{a}}$dl$\acute{\mbox{a}}$g-про\-цес\-сами}.
  \item[Б.]
  Все компоненты $\Lambda(t)$, $f(t)$, $\{g_n(t)\}_{n=\overline{1,N}}$ и~$h(t)$~--- 
  неслучайные {c$\acute{\mbox{a}}$dl$\acute{\mbox{a}}$g}-про\-цессы.
  \item[В.]
  Шумы в~$Y$ равномерно невырожденны~\cite{LSh_1_01}, т.\,е.\ для любых
   $t \hm\geqslant 0$, $1 \leqslant n \hm\leqslant N$ верно неравенство
  $g_n(t) \hm\geqslant \alpha I \hm > 0$ (здесь и~ниже~$I$ обозначает единичную матрицу подходящей размерности).
  \item[Г.]
  Процессы
  \begin{equation}
  K_{ij}(t) \triangleq \mathbf{I}_{\{\mathbf{0}\}}\left(g_i(t)-g_j(t)\right),\enskip  i,j=\overline{1,N},
  \label{eq:K_def}
  \end{equation}
  имеют локально ограниченную вариацию;
  здесь и~ниже $\mathbf{I}_{\mathcal{A}}(x)$ обозначает индикаторную функцию множества~$\mathcal{A}$, 
  а~$\mathbf{0}$~--- нулевую матрицу подходящей размерности.
  \item[Д.] Компоненты мартингала $\mu^{Z}$ процесса~$Z$~(\ref{eq:cobs_1}) ортогональны друг другу, т.\,е.\
  %\begin{equation}
  $$
  \langle   Z   \rangle_t \hm= \int\limits_0^t \mathrm{diag}\left( h(s)X_s\right)\,ds\,.
  $$
  %\label{eq:Z_qc}
 % \end{equation}
  Мартингалы в~процессе состояния~$\mu^{X}$ и~в~считающих наблюдениях~$\mu^{Z}$ также ортогональны
  %\begin{equation}
  $\langle   X,Z   \rangle_t \hm\equiv \mathbf{0}.$
 % \label{eq:XZ_qc}
  %\end{equation}
\end{itemize}

Условия А--В являются стандартными в~задачах фильтрации~\cite{LSh_1_01, Kall_80, YZL_04}. 
Условие Г обеспечивает регулярность разбиения временн$\acute{\mbox{о}}$й оси на множества, где ка\-кие-ли\-бо 
функции~$\{g_n(\cdot)\}$ совпадают или различны: на любом конечном промежутке времени $[0,t]$ 
для каждой пары $(i,j)$, $i,j\hm=\overline{1,N}$, множество, на котором интенсивности~$g_i(\cdot)$ 
и~$g_j(\cdot)$ совпадают, представимо в~виде объединения конечного набора интервалов. 
В~частности, это условие выполнено в~случае, когда все компоненты~$\{g_n(\cdot)\}$ 
ку\-соч\-но-глад\-кие с~локально ограниченными производными~\cite{MN_2019}.

Ограничение Д также не представляется обременительным: ортогональность 
мартингалов считающих процессов означает, что скачки этих процессов \textit{почти наверное} (п.\,н.)\ 
не совпадают. В~случае если мартингалы в~наблюдаемых процессах не ортогональны, исходные наблюдения 
всегда можно трансформировать так, чтобы обеспечить тре\-бу\-емую ортогональность 
преобразованных наблюдений. Для этого достаточно в~считающих наблюдениях синхронные 
скачки различных компонент выделить в~отдельные процессы. Условие ортогональности 
мартингалов в~состоянии и~наблюдениях
%(\ref{eq:XZ_qc})
позволяет избежать излишней громоздкости формул.

Рассмотрение задач фильтрации относительно замыкания справа~$\{\mathcal{O}_{t}\}$ имеет техническую подоплеку.
 Дело в~том, что математический аппарат стохастического анализа развит для случая, когда поток 
 $\sigma$-ал\-гебр, порожденных наблюдениями, непрерывен справа. В~общем случае п.\,н.\
  непрерывность траекторий процессов не гарантирует не\-пре\-рыв\-ности справа потока $\sigma$-ал\-гебр, 
  порожденных этими процессами~\cite{S_97}. В~\cite{BS_20} показано, что для наблюдений~(\ref{eq:cobs_1}), 
  (\ref{eq:sobs_1}) эта непрерывность нарушается. Переход к~$\{\overline{\mathcal{O}}_t\}$ 
  позволяет успешно решить теоретическую задачу оптимальной фильтрации. Данная 
  техническая особенность не коснется практической реализации решения задачи. 
  Во второй части работы будут предложены аппроксимации теоретических оценок, 
  вычисляемые по исходному незамкнутому потоку $\sigma$-ал\-гебр наблюдений~$\{\mathcal{O}_{t}\}$.

 \section{Преобразование непрерывных наблюдений и~решение задачи фильтрации}
 
В~\cite{LSh_89} представлен фундаментальный результат~--- 
формула УМО специального семимартингала по наблюдениям также в~форме специального семимартингала. 
В~общем случае эта формула не позволяет задать искомое УМО в~виде решения конечной замкнутой системы 
стохастических уравнений. Ситуация упрощается, если наблюдения с~помощью гирсановской 
замены вероятностной меры сводятся к~совокупности винеровских и~пуассоновских процессов~\cite{WH_85, EAM_10}. 
Наблюдения~(\ref{eq:cobs_1}) и~(\ref{eq:sobs_1}) не удовлетворяют этим условиям, однако существует 
преобразование, приводящее их к~эквивалентной совокупности диффузионного процесса с~единичной диффузией, 
набора считающих процессов и~наблюдений, выполняемых в~неслучайные дискретные моменты времени.

Рассмотрим квадратичную характеристику

\vspace*{-4pt}

\noindent
\begin{multline*}
  \langle   Y
  \rangle_t = Y_tY_t^{\top} - \int\limits_0^t Y_s\,dY_s^{\top} -  \int\limits_0^t \,dY_sY_s^{\top} ={}\\
  {}=
  \int\limits_{0}^{t} \sum\limits_{n=1}^NX_{s}^ng_n(s) \,ds\,.
 % \label{eq:YY_qc}
  \end{multline*}
  
  \vspace*{-4pt}

Введем в~рассмотрение $\overline{\mathcal{O}}_t$-со\-гла\-со\-ван\-ный процесс
\begin{equation}
q_t \triangleq \fr{d \langle   Y   \rangle_s}{ds}|_{s=t+} = \sum\limits_{n=1}^N X_{t}^n g_n(t)=q_t\left(X_t\right).
\label{eq:q_def}
\end{equation}

\vspace*{-2pt}

\noindent
Так как на любом конечном отрезке времени процессы~$X$ и~$\{g_n\}$ п.\,н.\
 терпят лишь конечное число скачков, то $\overline{\mathcal{O}}_t$-со\-гла\-со\-ван\-ный процесс
%\begin{equation}
$U_t \hm \triangleq
\int\nolimits_0^t q_s^{-{1}/{2}}\,dY_s$
%\label{eq:U_def}
%\end{equation}
допускает мартингальное разложение

\vspace*{-2pt}

\noindent
$$
U_t =
\int\limits_0^t \overline{f}(s)X_s\,ds \hm+\overline{W}_t,
$$
%\label{eq:U_mr}
%\end{equation}
где

\vspace*{-2pt}

\noindent 
$$
\overline{f}(t) \triangleq \sum\limits_{n=1}^N g_n^{-{1}/{2}}(t)f(t) 
\mathrm{diag} e_n;
$$

\vspace*{-2pt}

\noindent
$\overline{W}_t$~--- некоторый $\mathcal{F}_t$-со\-гла\-со\-ван\-ный 
стандартный винеровский процесс~\cite{LSh_1_01}. Преобразуем наблюдения~$q_t$ так, чтобы 
построить полный прообраз преобразования~(\ref{eq:q_def}) в~форме $N$-мер\-но\-го случайного процесса~$H_t$:

\vspace*{-2pt}

\noindent
  $$
H_t \triangleq \sum\limits_{n=1}^N\mathbf{I}_{\{\mathbf{0}\}}\left(q_t-g_n(t)\right)e_n\,.
$$
Из А и~Б следует, что  $H_t$~--- {c$\acute{\mbox{a}}$dl$\acute{\mbox{a}}$g}-про\-цесс,
представимый в~виде:

\vspace*{2pt}

\noindent
  \begin{equation*}
H_t =  \sum\limits_{n,k=1}^N\mathbf{I}_{\{\mathbf{0}\}}\left(g_k(t+)-g_n(t+)\right)X_t^ke_n=K(t+)X_t\,,
%\label{eq:h_def_2}
\end{equation*}

\vspace*{-2pt}

\noindent
где элементы матричнозначной функции~$K(\cdot)$ определяются формулой~(\ref{eq:K_def}). 
Из определения~$H_t$  и~$K(t)$ следует, что существует такая матрично-\linebreak\vspace*{-12pt}

\columnbreak

\noindent
значная функция~$T(t): 
\mathbb{R}_+ \hm\to \mathbb{R}^{N \times N}$, что процесс
$
V_t \triangleq  T(t)H_t$
обладает следующими свойствами:
\begin{itemize}
\item[(i)]
все компоненты~$V_t$ являются
{c$\acute{\mbox{a}}$dl$\acute{\mbox{a}}$g}-про\-цес\-са\-ми, причем
$\pp{V_t \in \mathbb{S}^N} \hm\equiv 1$;
\item[(ii)]
матрица $J(t) \triangleq T(t)K(t+)$ состоит из {c$\acute{\mbox{a}}$dl$\acute{\mbox{a}}$g}-эле\-мен\-тов, 
является трапециевидной, а ее строки~--- ортогональные векторы, сформированные из~0  и~1.
\end{itemize}
В силу своей структуры процесс~$V_t$ допускает разложение 

\noindent
$$
V_t = D_t\hm+R_t\,.
$$

\vspace*{-2pt}

\noindent
Здесь

\vspace*{-6pt}

\noindent
  \begin{equation*}
D_t = J(0)X_0+\!\!\!\sum\limits_{\substack{{s_j \leqslant t:}\\{\Delta J(s_j) \neq \mathbf{0} }}} 
\!\!\!\!\!\Delta J(s_j)X_{s_j}; \enskip
%\label{eq:D_def}
%\end{equation}
  %\begin{equation}
R_t = \int\limits_0^t J(s)\,dX_s\,,
%\label{eq:D_R_def}
\end{equation*}

\vspace*{-2pt}

\noindent
где $ \Delta J(t) \triangleq  J(t)\hm -  \Delta J(t-)$~--- функция скачков~$J(t)$. По сути, процесс~$D_t$ 
аккумулирует скачки~$V_t$, по\-рож\-ден\-ные изменениями матрицы~$J(t)$ в~неслучайные моменты времени~$s_i$, 
а~$R_t$~--- скачки~$V_t$, порожденные переходами состояния~$X_t$ в~случайные моменты времени~$\tau_j$. 
Заметим, что с~вероятностью~1 множества моментов~$\{s_i\}$ и~$\{\tau_j\}$ не пересекаются.
Процесс~$R_t$ взаимно однозначно определяется процессом $C_t \hm\triangleq \mathop{\mathrm{col}}
(C^1_t,\ldots,C^N_t)$, компоненты которого считают число скачков~$R_t$ в~состояние~$e_n$, $n \hm= 
\overline{1,N}$, произошедших за время $[0,t]$:

\vspace*{2pt}

\noindent
$$
 C^n_t\hm = \int\limits_0^t (1-e^{\top}_nV_{s-})e^{\top}_n\,dR_s.
 $$
 
 %\smallskip
 
 \noindent
 \textbf{Лемма~1.}\
% \begin{lemma} \label{lm:lm_1}
\textit{Для любого $ t \hm\geqslant 0$ верно тождество 
$\sigma\{U_s, Z_s, C_s, D_s:\; 0 \hm\leqslant s  \hm\leqslant t\} \hm\equiv \overline{\mathcal{O}}_t$}.

\vspace*{3pt}

 Истинность леммы~1 следует из взаимно однозначного соответствия траекторий~$Y$ и~$(U, C, D)$:
 
 \vspace*{-5pt}
 
 \noindent
 \begin{align*}
  U_t &\triangleq
 \int\limits_0^t q_s^{-{1}/{2}}dY_s, \enskip
q_t = \fr{d  \langle   Y   \rangle_s}{ds}\Bigl|_{s=t+}; \\[-1pt]
  C_t &=   \displaystyle\!\!\int\limits_0^t \! (I-\mathrm{diag}\, V_{s-})\,dV_s\! -\! \!\!\!\!\!\!\!\!
  \sum\limits_{\substack{{s_j \leqslant t:}\\[-1pt]
  {\Delta J(s_j) \neq \mathbf{0}}}} \!\!\!\!\!\!
  (I\!-\!\mathrm{diag}\, V_{s_j-})\Delta V_{s_j};\\[-1pt]
  D_t &= V_0 + \!\!\!\!\!\!\!\!\sum\limits_{\substack{{s_j \leqslant t:}\\
  {\Delta J(s_j) \neq \mathbf{0}}}}\!\!\! \Delta V_{s_j},  \enskip
  V_t = T(t)H_t, 
  \end{align*}
  
  \vspace*{-3pt}
  
  \noindent
  где
  
\vspace*{-3pt}

\noindent
  $$
  H_t \triangleq \sum\limits_{n=1}^N\mathbf{I}_{\{\mathbf{0}\}}(q_t-g_n(t))e_n;
  $$

 % \vspace*{-6pt}
  
\noindent
\begin{align*}
 V_t &= D_t + \int\limits_0^t \sum\limits_{i,j=1}^N V_{s-}^i\left(e_j-e_i\right)\,dC_s^j; 
\\[-1pt]
 Y_t& = \int\limits_0^t \sum\limits_{n=1}^{N}V_{s}^n g_n^{{1}/{2}}(s)\,dU_s.
\end{align*}

\vspace*{-2pt}

\noindent
 Таким образом исходные наблюдения были преобразованы  в~совокупность:
 \begin{itemize}
 \item непрерывного диффузионного процесса~$U_t$ с~единичной диффузией;
 \item совокупности считающих процессов~$Z_t$ и~$C_t$;
 \item наблюдений~$D_t$, полученных в~дискретные неслучайные моменты времени.
 \end{itemize}

 Ниже используем обозначения
 $J_n(t) \hm\triangleq e_n^{\top} J(t)$  для $n$-й строки~$J(t)$, $h_k(t) 
 \hm\triangleq e_k^{\top} h(t)$ для $k$-й строки~$h(t)$, а также
 \begin{equation*}
\Gamma_n(t) \triangleq \mathrm{diag}\left(J_n(t)\right) \Lambda^{\top}(t)
\left(I-\mathrm{diag}\, J_n(t)\right), \  n=\overline{1,N}.
%\label{eq:Gamma_def}
\end{equation*}

%\vspace*{-12pt}

\noindent
\textbf{Лемма~2.}\
% \begin{lemma} \label{lm:lm_2} %\cite{BS_20}.
 \textit{Верны следующие утверждения}.
 \begin{enumerate}
 \item
  \textit{Процессы $C_t^n,\; n=\overline{1,N}б$, допускают следующее мартингальное разложение}:
  
    \vspace*{-3pt}
  
\noindent
 \begin{multline*}
 C_t^n = \int\limits_0^t \mathbf{1}\Gamma_n(s)X_{s}\,ds +{}\\
 {}+ \int\limits_0^t 
 \left(1 - J_n(s)X_{s-}\right)J_n(s)\,d\mu_s^X\,.
 \label{eq:Cj_dec}
 \end{multline*}
 \item
 $\langle C^n,C^m\rangle_t \equiv 0$  \textit{для любых} $n \hm\neq m$;
 
     \vspace*{-1pt}
     
     \noindent
 \begin{equation*}
 \langle C^n, C^n \rangle_t = \int\limits_0^t \mathbf{1}\Gamma_n(s)X_{s}\,ds\,.
% \label{eq:Cj_char}
 \end{equation*}
 \item
 $\langle C^n, Z^m \rangle_t \equiv 0$  \textit{для любых} $n\hm=\overline{1,N}$, $m\hm=\overline{1,M}$.
 \item
 \textit{Обновляющие процессы}
 
     \vspace*{-4pt}
     
     \noindent
 \begin{align*}
 \nu_t^n &\triangleq \int\limits_0^t \left(dC_s^n - \mathbf{1}\Gamma_n(s)\widehat{X}_{s}\,ds \right), 
 \enskip n=\overline{1,N}\,;
% \label{eq:nij_def}
\\
 \varkappa_t^k &\triangleq \int\limits_0^t \left(dZ_s^k - h_k(s)\widehat{X}_{s}\,ds \right), \enskip 
 k=\overline{1,K}\,,
 %\label{eq:kij_def}
 \end{align*}
 
 \vspace*{-2pt}
 
 \noindent
  \textit{являются $\overline{\mathcal{O}}_t$-со\-гла\-со\-ван\-ны\-ми 
  мартингалами с~квадратичными характеристиками}
 \begin{equation}
 \left.
 \begin{array}{rl}
 \langle \nu^n\rangle_t&= \displaystyle\int\limits_0^t \mathbf{1}\Gamma_n(s)\widehat{X}_{s}\,ds\,; \\[6pt]
 %\label{eq:nij_char}
 %\end{equation}
  %\begin{equation}
 \langle \varkappa^k \rangle_t&=\displaystyle \int\limits_0^t h_k(s)\widehat{X}_{s}\, ds\,,
 \end{array}
 \right\}
 \label{eq:nkij_char}
 \end{equation}
  \textit{а обновляющий процесс}
  
      \vspace*{-3pt}
      
      \noindent
  \begin{equation}\label{eq:filt_2}
 \omega_t \triangleq U_t - \int\limits_0^t \overline{f}(s)\widehat{X}_{s}\,ds
 \end{equation}
  \textit{является $\overline{\mathcal{O}}_t$-со\-гла\-со\-ван\-ным $M$-мер\-ным 
  стандартным винеровским процессом}.
 \end{enumerate}

\noindent
 Д\,о\,к\,а\,з\,а\,т\,е\,л\,ь\,с\,т\,в\,о\ \ леммы~2 
 подобно доказательству соответствующего утверждения в~\cite{BS_20}.
 
 \vspace*{2pt}
 
 \noindent
 \textbf{Теорема~1.}
%  \begin{theorem}\label{th:th_1}
 \textit{Оптимальная оценка фильтрации $\widehat{X}_t$ является сильным 
 решением стохастической дифференциальной системы}
 
     \vspace*{-3pt}
 
 \noindent
 \begin{multline}\label{eq:filt_1}
   \widehat{X}_t =
  \left((D_0)^{\top}J(0)\pi_0\right)^+\mathrm{diag}\left(D_0\right)J(0)\pi_0
   + {}\\
   {}+
   \int\limits_0^t \Lambda^{\top}(s)\widehat{X}_{s}\,ds +
   \int\limits_0^t \mathbf{k}_s \overline{f}^{\top}(s)\,d\omega_s +
   \sum\limits_{n=1}^N\int\limits_{0}^t    \left(
\Gamma_n(s) -{}\right.\\
\left.{}- \mathbf{1}\Gamma_n(s)\widehat{X}_{s-} I
\right)\widehat{X}_{s-}
\left(\mathbf{1}\Gamma_n(s)\widehat{X}_{s-}\right)^+
    \,d\nu_s^n +{} \\
{}    +
   \sum\limits_{k=1}^K\int\limits_{0}^{t}
     \mathbf{k}_{s-}h_k^{\top}(s)
\left( h_k(s)\widehat{X}_{t-}\right)^+ d \varkappa_s^k +{} \\
{}   + \sum\limits_{\substack{{s_j \leqslant t:}\\{\Delta J(s_j) \neq \mathbf{0}}}}
 \left(    \left(
   \Delta D_{s_j}^{\top}\Delta J(s_j)\widehat{X}_{s_j-}\right)^+
   \mathrm{diag}\left(\Delta D_{s_j}\right)\times{}\right.\\
\left.   {}\times \Delta J(s_j) - I
  \vphantom{\left(\Delta D_{s_j}^{\top}\Delta J(s_j)\widehat{X}_{s_j-}\right)^+}
       \right) \widehat{X}_{s_j-},
 \end{multline}
 
     \vspace*{-6pt}
     
     \noindent
 \textit{где}
 
     \vspace*{-3pt}
     
     \noindent
 \begin{equation}
 \mathbf{k}_t = \mathrm{diag}\left(\widehat{X}_{t}\right)- \widehat{X}_{t} \widehat{X}_{t}^{\top},
 \label{eq:k_def}
 \end{equation}
 
     \vspace*{-2pt}
     
     \noindent
 \textit{а $A^+$ обозначает операцию псевдообращения.
Решение системы единственно в~классе неотрицательных кусочно-непрерывных $\overline{\mathcal{O}}_{t}$-согласованных процессов с~точками разрыва, принадлежащими множеству точек разрыва процесса $V$.
}

\noindent
 Д\,о\,к\,а\,з\,а\,т\,е\,л\,ь\,с\,т\,в\,о\ \ теоремы~1 приведено в~приложении.
 
 \vspace*{2pt}

  Теорема~1 позволяет точно восстанавливать состояние МСП~$X_t$ 
  по косвенным зашумленным наблюдениям~(\ref{eq:cobs_1}), (\ref{eq:sobs_1}) 
  в~случае выполнения условий идентифицируемости, представленных в~сле\-ду\-ющей лемме.
  
  \vspace*{2pt}
  
  \noindent
  \textbf{Лемма~3.}
  %\begin{lemma} \label{lm:lm_3}
\textit{Если для  любых $n\hm\neq m$ ($n,m\hm=\overline{1,N}$) неравенства 
$G_n(s) \hm\neq G_m(s)$ верны почти всюду по мере Лебега на $[0,t]$, то $\widehat{X}_t 
\hm= X_t$ ${\sf P}$-п.\,н., и~$X_t$ является решением}~(\ref{eq:filt_1}).

\vspace*{2pt}

\noindent
 Д\,о\,к\,а\,з\,а\,т\,е\,л\,ь\,с\,т\,в\,о\ \ леммы~3 
 аналогично доказательству соответствующего утверждения в~\cite{BS_20}.
 
 \vspace*{-6pt}

 \section{Заключение}
 
 \vspace*{-2pt}
 
 Первая часть цикла содержит теоретическое решение задачи оптимальной фильтрации по
  комплексным (непрерывным и~считающим) наблюдениям при наличии в~непрерывных наблюдениях
  \textit{мультипликативных шумов}, т.\,е.\ шумов, интенсивность которых зависит от 
  оцениваемого состояния.
       Оценка фильтрации определяется решением некоторой замкнутой 
  дис\-крет\-но-не\-пре\-рыв\-ной стохастической  системы, в~правую часть которой входят
   наблюдения, преобразованные специальным образом. Вид оценки имеет важное теоретическое 
   значение, так как позволяет описать важные свойства оценки, в~частности условия полной 
   идентифицируемости состояния, когда его можно восстановить точно, основываясь на косвенных 
   зашумленных наблюдениях. Тем не менее этот вид  малопригоден для конструирования чис\-лен\-ных 
   алгоритмов фильтрации из-за того, что необходимое преобразование наблюдений включает в~себя 
   операцию, подразумевающую двойной предельный переход: во-пер\-вых, вычисление квадратичной
    характеристики~$\langle Y \rangle$ диффузионных наблюдений и,~во-вто\-рых, 
    вы\-чис\-ле\-ние ее правой производной. Последующие исследования посвящены эффективным численным 
    методам решения данной задачи фильтрации.
    
{\small \subsection*{\raggedleft Приложение}

 


\noindent
Д\,о\,к\,а\,з\,а\,т\,е\,л\,ь\,с\,т\,в\,о\ \ теоремы~1.
%\begin{proofoftheorem} {\ref{th:th_1}}
Используем подход из~\cite[Part~III,~Sect.~8.7]{EAM_10}.
 Из правила Байеса следует, что
\begin{multline*}
\widehat{X}_0 = \me{}{X_0|\mathcal{O}_{0+}}\hm = {}\\
{}=\me{}{X_0|D_0}\hm= \left(D_0^{\top}J(0)\pi\right)^+
\mathrm{diag}\left(D_0\right)J(0)\pi\,.
\end{multline*}
Пусть  $s_{j-1}$~--- неслучайный момент $(j-1)$-го дискретного наблюдения~$\Delta D_{s_{j-1}}$. 
Исследуем поведение~$X_t$ на интервале $[s_{j-1}, s_{j})$:
$$
X_t = X_{s_{j-1}} + \int\limits_{s_{j-1}}^t \Lambda^{\top}(s)X_{s}\,ds
 + \mu^X_t - \mu^X_{s_{j-1}}.
$$
Вычисляя УМО обеих частей равенства относительно~$\overline{\mathcal{O}}_t$, можно показать, что
$$
\widehat{X}_t = \widehat{X}_{s_{j-1}} + \int\limits_{s_{j-1}}^t \Lambda^{\top}(s)\widehat{X}_{s}\,ds + \mu_t^1,
$$
где $\{\mu_t^1\}_{t \in [s_{j-1}, s_{j})}$~--- $\overline{\mathcal{O}}_t$-со\-гла\-со\-ван\-ный мартингал.
Для любого $t \hm\in [s_{j-1}, s_{j})$ верно равенство 
\begin{multline*}
\overline{\mathcal{O}}_{t}=
\overline{\mathcal{O}}_{s_{j-1}} \vee \sigma\{U_s,Z_s\;s  \in (s_{j-1},t]\} 
\vee{}\\
{}\vee \sigma\{C_s^n, s \in (s_{j-1},t],\enskip n=\overline{1,N}\}\,.
\end{multline*}
Процесс $\{\omega_t\}$~(\ref{eq:filt_2})  является $\overline{\mathcal{O}}_t$-со\-гла\-со\-ван\-ным 
стандартным винеровским;
$U_t$ является $\overline{\mathcal{O}}_t$-со\-гла\-со\-ван\-ным семимартингалом 
с~условно независимыми относительно~$\mathcal{F}^X$ приращениями, в~то время 
как $\{C^n_t\}_{n=\overline{1,N}}$ и~$\{Z^k_t\}_{k=\overline{1,K}}$~--- 
$\overline{\mathcal{O}}_t$-со\-гла\-со\-ван\-ные точечные процессы. Поэтому мартингал~$\mu_t^1$ 
\textit{допускает интегральное представление}~\cite[Chap. 4, \S\,8, Problem~1]{LSh_89},
т.\,е.

\noindent
\begin{multline}
\widehat{X}_t = \widehat{X}_{s_{j-1}} + \int\limits_{s_{j-1}}^t \Lambda^{\top}(s)\widehat{X}_{s}\,ds
+ \int\limits_{s_{j-1}}^t \alpha_s\, d\omega_s
+{}\\
{}+ \int\limits_{s_{j-1}}^t \sum\limits_{n=1}^N\beta_s^n\, d\nu_s^n+
\int\limits_{s_{j-1}}^t \sum\limits_{k=1}^K\gamma_s^k \,d\varkappa_s^k,
\label{eq:filt_3}
\end{multline}
где $\alpha_t$, $\{\beta_t^n\}_{n=\overline{1,N}}$ и~$\{\gamma_t^k\}_{k=\overline{1,K}}$ представляют собой
$\overline{\mathcal{O}}_t$-пред\-ска\-зу\-емые процессы подходящей размерности, подлежащие определению.

По обобщенному правилу Ито
\begin{multline*}
X_tU_t^{\top}= X_{s_{j-1}}U_{s_{j-1}}^{\top}+{}\\
{}+
\int\limits_{s_{j-1}}^t \left(
\Lambda^{\top}(s)X_{s}U_{s}^{\top} + \mathrm{diag}\left(X_{s}\right)\overline{f}^{\top}(s)
\right)\,ds + \mu_t^2,
\end{multline*}
где $\mu_t^2$ -- некоторый $\mathcal{F}_t$-со\-гла\-со\-ван\-ный мартингал.
Беря УМО от обеих частей равенства относительно  $\overline{\mathcal{O}}_{t}$, можно показать, что
\begin{multline}
\widehat{X}_tU_t^{\top}= \widehat{X}_{s_{j-1}}U_{s_{j-1}}^{\top}+ {}\\
{}+\int\limits_{s_{j-1}}^t \left(
\Lambda^{\top}(s)\widehat{X}_{s}U_{s}^{\top} + \mathrm{diag}\left(\widehat{X}_{s}\right)\overline{f}^{\top}(s)
\right)ds + \mu_t^3,
\label{eq:xy_1}
\end{multline}
где $\mu_t^3$~--- некоторый $\overline{\mathcal{O}}_t$-со\-гла\-со\-ван\-ный мартингал. 
В~то же время из правила Ито, представления~(\ref{eq:filt_3}) и~факта, что $\omega_t$~--- 
винеровский процесс следует, что
\begin{multline}
\widehat{X}_tU_t^{\top}=  \widehat{X}_{s_{j-1}}U_{s_{j-1}}^{\top}+ {}\\
\!\!{}+ \!\!\int\limits_{s_{j-1}}^t\!\! \left(
\Lambda^{\top}(s)\widehat{X}_{s}U_{s}^{\top} + \widehat{X}_{s}\widehat{X}_{s}^{\top}
\overline{f}^{\top}(s) + \alpha_s
\right)ds + \mu_t^4,\!\!
\label{eq:xy_2}
\end{multline}
где $\mu_t^4$~--- некоторый $\overline{\mathcal{O}}_t$-со\-гла\-со\-ван\-ный мартингал. Так
как~(\ref{eq:xy_1}) и~(\ref{eq:xy_2})
являются двумя разложениями одного и~того же специального семимартингала~$\widehat{X}_tU_t^{\top}$, 
то из его единственности следует, что $\overline{\mathcal{O}}_{t}$-пред\-ска\-зу\-емый процесс~$\alpha_t$ 
удовлетворяет равенству
$$
\int\limits_{s_{j-1}}^t
\mathrm{diag}\left(\widehat{X}_{s}\right)\overline{f}^{\top}(s)
ds=\int\limits_{s_{j-1}}^t \left(\widehat{X}_{s}\widehat{X}_{s}^{\top}\overline{f}^{\top}(s) + \alpha_s
\right)ds
$$
и с~учетом (\ref{eq:k_def}) и~свойств~$\omega_t$ может быть выбран в~виде
$
\alpha_t = \mathbf{k}_{t} \overline{f}^{\top}(t).
$

Вновь из правила Ито, свойств МСП $X$ и~$C^n$ по\-лу\-чаем:
\begin{multline*}
X_tC_t^n =
X_{s_{j-1}}C^n_{s_{j-1}} +{}\\
{}+ \int\limits_{s_{j-1}}^t \!\!
\left(
\Lambda^{\top}(s)X_sC^n_s + \Gamma_n(s)X_s
\right)ds  + \mu^5_t,
\end{multline*}
где  $\mu_t^5$~--- $\mathcal{F}_t$-со\-гла\-со\-ван\-ный мартингал. 
Вычисляя УМО  относительно~$\overline{\mathcal{O}}_{t}$, получаем:
\begin{multline}
\widehat{X}_tC_t^n =
\widehat{X}_{s_{j-1}}C^n_{s_{j-1}} +{}\\
{}+\int\limits_{s_{j-1}}^t
\left(
\Lambda^{\top}(s)\widehat{X}_sC^n_s + \Gamma_n(s)\widehat{X}_s
\right)ds + \mu^6_t, 
\label{eq:xz_1}
\end{multline}
где $\mu_t^6$~--- $\overline{\mathcal{O}}_{t}$-со\-гла\-со\-ван\-ный мартингал.

С другой стороны, по правилу Ито и~с~учетом~(\ref{eq:filt_3}) и~(\ref{eq:nkij_char}) можно показать, что
\begin{multline}
\widehat{X}_tC_t^n =
\widehat{X}_{s_{j-1}}C^n_{s_{j-1}}+{}\\
{}+
\int\limits_{s_{j-1}}^t\!
\left(
\Lambda^{\top}(s)\widehat{X}_sC^n_s +
\widehat{X}_s\mathbf{1}\Gamma_n(s)\widehat{X}_s+
\beta^n_s \mathbf{1}\Gamma_n(s)\widehat{X}_s
\right)ds +{}\\
{}+ \mu^7_t,
\label{eq:xz_2}
\end{multline}
где $\mu_t^7$~--- $\overline{\mathcal{O}}_{t}$-со\-гла\-со\-ван\-ный мартингал.  
Из совпадения разложений~(\ref{eq:xz_1}) и~(\ref{eq:xz_2}) можно заключить, что процесс~$\beta^n_s$ 
должен удовлетворять равенству
\begin{equation*}
\int\limits_{s_{j-1}}^t
\Gamma_n(s)\widehat{X}_s\,ds =
\int\limits_{s_{j-1}}^t
\left[
\widehat{X}_s\mathbf{1}\Gamma_n(s)\widehat{X}_s+
\beta^n_s \mathbf{1}\Gamma_n(s)\widehat{X}_s
\right]ds
\end{equation*}
и может быть выбран в~форме
\begin{multline*}
\beta^n_t = \left(
\Gamma_n(t) - \mathbf{1}\Gamma_n(t)\widehat{X}_{t-} I
\right)\widehat{X}_{t-}
\left(\mathbf{1}\Gamma_n(t)\widehat{X}_{t-}\right)^+,\\
n=\overline{1,N}\,.
\end{multline*}


Вновь из правила Ито получаем
\begin{multline*}
X_tZ_t^k =
X_{s_{j-1}}Z^k_{s_{j-1}} +{}\\
{}+\int\limits_{s_{j-1}}^t\!\!
\left(
\Lambda^{\top}(s)X_sZ^k_s + \mathrm{diag}\left(X_s\right)h_k^{\top}(s)
\right)ds
 + \mu^8_t,
\end{multline*}
где  $\mu_t^8$~--- $\mathcal{F}_t$-со\-гла\-со\-ван\-ный мартингал. Вычисляя УМО 
 относительно~$\overline{\mathcal{O}}_{t}$, получаем
\begin{multline}
\widehat{X}_tZ_t^k =
\widehat{X}_{s_{j-1}}Z^k_{s_{j-1}} +{}\\
{}+\int\limits_{s_{j-1}}^t\!\!
\left(
\Lambda^{\top}(s)\widehat{X}_sZ^k_s + \mathrm{diag}\left(\widehat{X}_s\right)h_k^{\top}(s)
\right)ds + \mu^9_t,
\label{eq:xz_3}
\end{multline}
где $\mu_t^9$~--- $\overline{\mathcal{O}}_{t}$-со\-гла\-со\-ван\-ный мартингал.
С~другой стороны, по правилу Ито и~результатам леммы~2 можно получить равенство

\noindent
\begin{multline}
\widehat{X}_tZ_t^k =
\widehat{X}_{s_{j-1}}Z^k_{s_{j-1}}
+{}\\
{}+
\int\limits_{s_{j-1}}^t\!\!
\left(
\Lambda^{\top}(s)\widehat{X}_sZ^k_s +
\widehat{X}_s \widehat{X}_s^{\top} h_k^{\top}(s) +\gamma_s^k h_k(s)\widehat{X}_s
\right)ds  + {}\\
{}+\mu^{10}_t,
\label{eq:xz_4}
\end{multline}
где $\mu_t^{10}$~--- $\overline{\mathcal{O}}_{t}$-со\-гла\-со\-ван\-ный мартингал.  
Из совпадения разложений~(\ref{eq:xz_3}) и~(\ref{eq:xz_4}) следует, что процесс~$\gamma^k_s$ 
должен удовлетворять равенству
$$
\int\limits_{s_{j-1}}^t\!\!\!\!
 \mathrm{diag}\left(\widehat{X}_s\right)h_k^{\top}(s)\,ds =\!\!\!\!
\int\limits_{s_{j-1}}^t \!\!\!\!\left(
\widehat{X}_s \widehat{X}_s^{\top} h_k^{\top}(s) +\gamma_s^k h_k(s)\widehat{X}_s\right)ds
$$
и может быть выбран в~форме
$$
\gamma^k_t\hm =
\mathbf{k}_{t-} h_k^{\top}(t)
\left( h_k(t)\widehat{X}_{t-}\right)^+,\enskip k=\overline{1,K}\,.
$$


Таким образом, на интервале $[s_{j-1},s_{j})$ оптимальная оценка~$\widehat{X}_t$ 
описывается стохастической дифференциальной системой
\begin{multline}
   \widehat{X}_t =
  \widehat{X}_{s_{j-1}}
   + \int\limits_{s_{j-1}}^t\! \Lambda^{\top}(s)\widehat{X}_{s-}\,ds +
   \int\limits_{s_{j-1}}^t \!\mathbf{k}_{s}\overline{f}^{\top}(s)\,d\omega_s +{} \\
   {} +
   \sum\limits_{n=1}^N \,\int\limits_{s_{j-1}}^t\!\!\!\!
   \left(
\Gamma_n(s) - \mathbf{1}\Gamma_n(s)\widehat{X}_{s-} I
\right)\!\widehat{X}_{s-}\!
\left(\mathbf{1}\Gamma_n(s)\widehat{X}_{s-}\right)^+\!\!
    d\nu_s^n + {}\hspace*{-1.07239pt}\\ 
    {}+
     \sum\limits_{k=1}^K\,\int\limits_{s_{j-1}}^t\!
     \mathbf{k}_{s-}h_k^{\top}(s)
\left( h_k(s)\widehat{X}_{s-}\right)^+ \!d \varkappa_s^k.
   \label{eq:f_cut}
 \end{multline}

Так как $\pp{\Delta X_{s_j} = 0}\hm = 1$, то с~вероятностью~1
   \begin{multline*}
   {\sf E}\left\{ X_{s_{j}}|\overline{\mathcal{O}}_{s_{j-1}}
\vee \sigma\left\{U_s,Z_s\;s \in (s_{j-1},s_{j}]\right\}
 \vee {}\right.\\
 \left.{}\vee\sigma\left\{C_s^n,\;s \in (s_{j-1},s_{j}],\enskip n=\overline{1,N}\right\}\right\}
  ={} \\ 
   {}=
 \widehat{X}_{s_{j-1}}
   + \!\int\limits_{s_{j-1}}^{s_{j}} \!\Lambda^{\top}(s)\widehat{X}_{s-}\,ds +
   \int\limits_{s_{j-1}}^{s_{j}}\! \mathbf{k}_{s}\overline{f}^{\top}(s)\,d\omega_s + {}\\ 
  {}+
   \sum\limits_{n=1}^N\int\limits_{s_{j-1}}^{s_{j}}\!\!\!
   \left(
\Gamma_n(s) - \mathbf{1}\Gamma_n(s)\widehat{X}_{s-} I
\right)\!\widehat{X}_{s-}
\left(\mathbf{1}\Gamma_n(s)\widehat{X}_{s-}\!\right)^+
   \! \!d\nu_s^n +{} \\
    {} +
     \sum\limits_{k=1}^K\,\int\limits_{s_{j-1}}^{s_{j}}\!
     \mathbf{k}_{s-}h_k^{\top}(s)
\left( h_k(s)\widehat{X}_{s-}\right)^+ \! d \varkappa_s^k = \widehat{X}_{s_{j}-}.
 \end{multline*}

В силу того что
 \begin{multline*}
   \overline{\mathcal{O}}_{s_{j}} = \overline{\mathcal{O}}_{s_{j-1}}
   \vee \sigma\left\{U_s,Z_s,\;s \in (s_{j-1},s_{j}]\right\}
   \vee{}\\
   {}\vee \sigma\{C_s^n,\;s \in (s_{j-1},s_{j}],\enskip
    n=\overline{1,N}\} \vee \sigma\left\{\Delta D_{s_{j}}\right\},
   \end{multline*}
   по правилу Байеса получаем
   \begin{multline}
   \widehat{X}_{s_{j}} ={}\\
  \!\!\! {}= \left(\Delta D_{s_{j}}^{\top}\Delta J(s_{j})\widehat{X}_{s_{j}-}\right)^+
   \mathrm{diag}\left(\Delta D_{s_{j}}\right)\Delta J(s_{j})\widehat{X}_{s_{j}-}.\!
   \label{eq:Bayes}
   \end{multline}

   Уравнение (\ref{eq:filt_1}) получается путем <<склейки>> локальных решений~(\ref{eq:f_cut}), 
   описывающих изменение~$\widehat{X}_t$ на интервалах $[s_{j-1},s_{j})$, и~формулы~(\ref{eq:Bayes}), 
   описывающей пересчет оценок в~моменты~$s_{j}$ поступления дискретных наблюдений~$\Delta D$.

   Единственность сильного решения в~классе не\-от\-ри\-ца\-тель\-ных 
   ку\-соч\-но-не\-пре\-рыв\-ных 
   $\overline{\mathcal{O}}_t$-со\-гла\-со\-ван\-ных процессов с~моментами скачков, принадлежащими 
   множеству скачков~$V_t$, доказывается аналогично~\cite[Ch.~9, Theorem~9.2]{{LSh_1_01}}.
   Теорема~1 доказана.
}   


{\small\frenchspacing
{%\baselineskip=10.8pt
%\addcontentsline{toc}{section}{References}
\begin{thebibliography}{99}

\bibitem{B_19_1_IA}
  \Au{Борисов А.} Численные схемы фильтрации марковских скачкообразных процессов по дискретизованным 
  наблюдениям~I: характеристики точности~// Информатика и~её применения, 2019. Т.~13. Вып.~4. C.~68--75. 
doi: 10.14357/19922264190411.


     \bibitem{B_20_2_IA}
  \Au{Борисов А.} Численные схемы фильтрации марковских скачкообразных процессов по
   дискретизованным наблюдениям~II: случай аддитивных шумов~// Информатика и~её применения, 2020. Т.~14.
   Вып.~1. C.~17--23. doi: 10.14357/19922264200103.
 
\bibitem{B_20_3_IA}
  \Au{Борисов~А.} Численные схемы фильтрации марковских скачкообразных процессов по 
  дискретизованным наблюдениям~III: случай мультипликативных шумов~// Информатика и~её применения, 
  2020. Т.~14. Вып.~2. C.~10--18. 
  doi: 10.14357/19922264200202.


 \bibitem{LSh_1_01}
 \Au{Liptser R., Shiryaev~A.} Statistics of random processes~I: General theory.~--- 
 Berlin/Heidelberg:~Springer, 2001. 427~p.

 \bibitem{Kall_80} %5     
\Au{Kallianpur G.}  Stochastic filtering theory.~--- New York, NY, USA:~Springer, 1980. 318~p.

\bibitem{YZL_04}
\Au{Yin G., Zhang~Q., Liu~Y.}
Discrete-time approximation of Wonham filters~//
IET Control Theory~A., 2004. No.\,2. P.~1--10. doi: 10.1007/s11768-004-0017-7.


\bibitem{MN_2019}
\Au{Magnus J., Neudecker~H.}
  Matrix differential calculus with applications in statistics and econometrics.~--- 
  New York, NY, USA: Wiley, 2019. 504~p.

  \bibitem{S_97}
\Au{Stoyanov J.} Counterexamples in probability.~--- Hoboken, NJ, USA: Wiley, 1997. 352~p.

  \bibitem{BS_20} %9
 \Au{Borisov A., Sokolov~I.} Optimal filtering of Markov jump processes given 
 observations with state-dependent noises: Exact solution and stable numerical schemes~// Mathematics, 2020. 
 Vol.~8. Iss.~4. Art. No.\,506. 22~p. doi: 10.3390/ math8040506.

 \bibitem{LSh_89}
 \Au{Liptser~R.,  Shiryaev~A.} Theory   of   martingales.~---  Dortrecht: Springer, 1989. 792~p.

 \bibitem{WH_85}
 {\it Wong E., Hajek~B.} Stochastic processes in engineering systems.~--- New York, NY, USA:
 Springer, 1985. 361~p.

 \bibitem{EAM_10}
\Au{Elliott R., Moore~J., Aggoun~L.} Hidden Markov models: Estimation and control.~--- New York, NY, USA:
 Springer, 2010. 382~p.
 %\bibitem{LSh_2_01}
  %{\em Liptser, R.; Shiryaev, A.} Statistics of Random Processes II: Applications; Springer: Berlin/Heidelberg, Germany, 2001.


 \end{thebibliography}

}
}

\end{multicols}

\vspace*{-9pt}

\hfill{\small\textit{Поступила в~редакцию 05.03.2021}}

\vspace*{6pt}

%\pagebreak

%\newpage

%\vspace*{-28pt}

\hrule

\vspace*{2pt}

\hrule

\vspace*{-2pt}

\def\tit{FILTERING OF MARKOV JUMP PROCESSES GIVEN COMPOSITE OBSERVATIONS~I: EXACT SOLUTION}


\def\titkol{Filtering of Markov jump processes given composite observations~I: Exact solution}

\def\aut{A.\,V. Borisov$^{1,2,3,4}$ and D.\,Kh.~Kazanchyan$^{3}$}

\def\autkol{A.\,V. Borisov and D.\,Kh.~Kazanchyan}


\titel{\tit}{\aut}{\autkol}{\titkol}

\vspace*{-15pt}


\noindent
$^1$Institute of Informatics Problems, Federal Research Center 
``Computer Science and Control'' of the Russian\linebreak
$\hphantom{^1}$Academy of Sciences, 
44-2~Vavilov Str., Moscow 119333, Russian Federation

\noindent
$^2$Moscow Aviation Institute (National Research University), 4~Volokolamskoe Shosse, Moscow 125080, 
Russian\linebreak
$\hphantom{^1}$Federation 

\noindent
$^3$Department of Mathematical Statistics, Faculty of Computational Mathematics and Cybernetics,
M.\,V.~Lomo-\linebreak 
$\hphantom{^1}$nosov Moscow State University, 1-52~Leninskiye Gory, GSP-1, Moscow 119991, 
Russian Federation

\noindent
$^4$Moscow Center for Fundamental and Applied Mathematics, M.\,V.~Lomonosov Moscow State University,\linebreak
$\hphantom{^1}$1-52~Leninskiye Gory, GSP-1, Moscow 119991, Russian Federation 



 
\def\leftfootline{\small{\textbf{\thepage}
\hfill INFORMATIKA I EE PRIMENENIYA~--- INFORMATICS AND
APPLICATIONS\ \ \ 2021\ \ \ volume~15\ \ \ issue\ 2}
}%
\def\rightfootline{\small{INFORMATIKA I EE PRIMENENIYA~---
INFORMATICS AND APPLICATIONS\ \ \ 2021\ \ \ volume~15\ \ \ issue\ 2
\hfill \textbf{\thepage}}}

\vspace*{3pt}



\Abste{The first part of the series is devoted to the optimal filtering of the 
finite-state Markov jump processes (MJP) given the ensemble of the diffusion and 
counting observations. The noise intensity in the observable diffusion depends on 
the estimated MJP state. The special equivalent observation transformation 
converts them into the collection of the diffusion process of unit intensity, 
counting processes, and indirect measurements performed at some nonrandom discrete 
instants. The considered filtering estimate is expressed as a~solution to the 
discrete-continuous stochastic differential system with the transformed observations 
on the right-hand side. The identifiability condition, under which
MJP state can be reconstructed from indirect noisy observations precisely, is presented.}

\KWE{Markov jump process; optimal filtering; multiplicative observation noises; stochastic 
differential equation; continuous and counting observations; identifiability condition}




\DOI{10.14357/19922264210202}

%\vspace*{-15pt}

 \Ack
\noindent
The work was supported in part by the Russian Foundation
for Basic Research (project 19-07-00187~A). The research was conducted in accordance with the
program of the Moscow Center for Fundamental and Applied Mathematics.

\vspace*{12pt}

  \begin{multicols}{2}

\renewcommand{\bibname}{\protect\rmfamily References}
%\renewcommand{\bibname}{\large\protect\rm References}

{\small\frenchspacing
 {%\baselineskip=10.8pt
 \addcontentsline{toc}{section}{References}
 \begin{thebibliography}{99}

\bibitem{B_19_1_IA-1}
\Aue{Borisov, A.} 2019. Chislennyye skhemy fil'tratsii mar\-kov\-skikh skachkoobraznykh protsessov po
 diskretizovannym nablyudeniyam~I:  kharakteristiki tochnosti [Numerical schemes of Markov jump process 
 filtering given discretized observations~I: 
 Accuracy characteristics]. \textit{Informatika i~ee Primeneniya~--- Inform.~Appl.}
 13(4):68--75.
  doi: 10.14357/19922264190411.

  \bibitem{B_20_2_IA-1}
\Aue{Borisov, A.} 2020. Chislennye skhemy fil'tratsii mar\-kov\-skikh skachkoobraznykh 
protsessov po diskretizovannym nablyudeniyam~II: sluchay additivnykh shuchmov 
 [Numerical schemes of Markov jump process filtering given discretized observations~II: Additive noises case].
 \textit{Informatika i~ee primeneniya~--- Inform.~Appl.}
 14(1):17--23. doi: 10.14357/19922264200103.

\bibitem{B_20_3_IA-1} %3
  \Aue{Borisov, A.} 2020. Chislennye skhemy fil'tratsii mar\-kov\-skikh skachkoobraznykh 
  protsessov po diskretizovannym nablyudeniyam III:  sluchay mul'tiplikativnykh shumov
  [Numerical schemes of Markov jump process filtering given
discretized observations III: Multiplicative noises case].
\textit{Informatika i~ee primeneniya~--- Inform.~Appl.}
14(2):10--18. doi: 10.14357/19922264200202.


 \bibitem{LSh_1_01-1}
 \Aue{Liptser, R., and A.~Shiryaev.} 2001. \textit{Statistics of random processes~I: General theory.}  
 Berlin/Heidelberg: Springer. 427~p.
 {\looseness=-1
 
 }
 
 
  \bibitem{Kall_80-1}     
 \Aue{Kallianpur, G.} 1980. \textit{Stochastic filtering theory.}  New York, NY: Springer. 318~p.

\bibitem{YZL_04-1}
\Aue{Yin, G., Q.~Zhang, and Y.~Liu.} 2004.
Discrete-time approximation of Wonham filters.
\textit{IET Control Theory~A.} 2:1--10.
doi: 10.1007/s11768-004-0017-7.

\bibitem{MN_2019-1}
 \Aue{Magnus, J., and H.~Neudecker.} 2019.
\textit{Matrix differential calculus with applications in statistics and econometrics.}  
New York, NY: Wiley. 504~p.

  \bibitem{S_97-1}
 \Aue{Stoyanov, J.} 1997. \textit{Counterexamples in probability.} Hoboken, NJ: Wiley. 352~p.


  \bibitem{BS_20-1}
 \Aue{Borisov, A., and  I.~Sokolov.}
  2020. Optimal filtering of Markov jump processes given observations with state-dependent noises: 
  Exact solution and stable numerical schemes. \textit{Mathematics} 8(4):506. 22~p. doi: 10.3390/ math8040506.
 
  \bibitem{LSh_89-1}
 \Aue{Liptser, R., and A.~Shiryaev.} 1989. \textit{Theory of martingales.}  Dortrecht: Springer. 792~p.
 
 
  \bibitem{WH_85-1}
 \Aue{Wong, E., and B.~Hajek.} 1985. \textit{Stochastic processes in engineering systems.} New York, NY:
 Springer. 361~p.

 \bibitem{EAM_10-1}
 \Aue{Elliott, R., J.~Moore, and L.~Aggoun.} 2010. 
 \textit{Hidden Markov models: Estimation and control.} New York, NY: Springer. 382~p.
\end{thebibliography}

 }
 }

\end{multicols}

\vspace*{-3pt}

  \hfill{\small\textit{Received March~5, 2021}}


%\pagebreak

%\vspace*{-8pt}  



\Contr

\noindent
\textbf{Borisov Andrey V.} (b.\ 1965)~--- 
Doctor of Science in physics and mathematics, principal scientist, Institute of Informatics Problems, 
Federal Research Center ``Computer Science and Control'' of the Russian Academy of Sciences, 
44-2~Vavilov Str., Moscow 119333, Russian Federation; professor, Moscow Aviation 
Institute (National Research University), 4~Volokolamskoe Shosse, Moscow 125080, Russian Federation; 
professor, Department of Mathematical Statistics, Faculty of Computational Mathematics and Cybernetics,
M.\,V.~Lomonosov Moscow State University, 1-52~Leninskiye Gory, GSP-1, Moscow 119991, Russian Federation; 
senior scientist, Moscow Center for Fundamental and Applied Mathematics, 
M.\,V.~Lomonosov Moscow State University, 1-52~Leninskiye Gory, GSP-1, Moscow 119991, Russian Federation; 
\mbox{ABorisov@frccsc.ru}

\vspace*{3pt}

\noindent
\textbf{Kazanchyan Drastamat Kh.} (b.\ 1994)~--- 
PhD student, Department of Mathematical Statistics, Faculty of Computational Mathematics and Cybernetics, 
M.\,V.~Lomonosov Moscow State University, 1-52~Leninskiye Gory, GSP-1, Moscow 119991, Russian Federation; 
\mbox{drastamat94@gmail.com} 


\label{end\stat}

\renewcommand{\bibname}{\protect\rm Литература}