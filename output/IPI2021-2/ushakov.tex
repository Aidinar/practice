\def\stat{ushakovi}

\def\tit{МНОГОМЕРНЫЕ РАСПРЕДЕЛЕНИЯ ВЫХОДЯЩИХ ПОТОКОВ В~СИСТЕМЕ С~АБСОЛЮТНЫМ ПРИОРИТЕТОМ$^*$}

\def\titkol{Многомерные распределения выходящих потоков в~системе с~абсолютным приоритетом}

\def\aut{В.\,Г.~Ушаков$^1$, Н.\,Г.~Ушаков$^2$}

\def\autkol{В.\,Г.~Ушаков, Н.\,Г.~Ушаков}

\titel{\tit}{\aut}{\autkol}{\titkol}

\index{Ушаков В.\,Г.}
\index{Ушаков Н.\,Г.}
\index{Ushakov V.\,G.}
\index{Ushakov N.\,G.}

{\renewcommand{\thefootnote}{\fnsymbol{footnote}} \footnotetext[1]
{Работа выполнена при поддержке Министерства науки и~высшего образования
Российской Федерации (проект 075-15-2019-1621).}}

\renewcommand{\thefootnote}{\arabic{footnote}}
\footnotetext[1]{Факультет вычислительной математики и~кибернетики Московского государственного 
университета имени М.\,В.~Ломоносова; 
Федеральный исследовательский центр <<Информатика и~управление>>
  Российской академии наук, \mbox{vgushakov@mail.ru}}
\footnotetext[2]{Институт проблем технологии микроэлектроники и~особочистых материалов Российской академии наук;
Норвежский на\-уч\-но-тех\-но\-ло\-ги\-че\-ский университет, Тронхейм, Норвегия,
\mbox{ushakov@math.ntnu.no}}

\vspace*{3pt}




\Abst{Изучена однолинейная система массового обслуживания с~бесконечным числом мест 
для ожидания, произвольным распределением времени обслуживания и~пуассоновскими входящими 
потоками требований. Между требованиями разных потоков действует дисциплина абсолютного 
приоритета с~обслуживанием заново прерванного требования. Методом вложенных цепей Маркова 
исследуется многомерный случайный процесс, компоненты которого~--- число требований фиксированного 
приоритета в~системе и~длительность интервала времени между последовательными моментами ухода 
из системы требований этого приоритета. Найдены конечномерные распределения указанного процесса. 
В~качестве следствия получены преобразования Лап\-ла\-са--Стилть\-еса 
одномерных и~двумерных распределений выходящего потока требований каждого приоритета в~стационарном режиме.}

\KW{выходящий поток; абсолютный приоритет; %обслуживание заново; 
вложенная цепь Маркова; 
одноканальная система}

\DOI{10.14357/19922264210204}

\vspace*{5pt}


\vskip 10pt plus 9pt minus 6pt

\thispagestyle{headings}

\begin{multicols}{2}

\label{st\stat}

\section{Введение} 

Свойства выходящих из системы обслуживания потоков требований важны при решении многих задач, 
среди  которых исследование сетей массового обслуживания, статистический анализ параметров системы, 
оптимальное управление работой сис\-темы.

Вероятностные свойства выходящих потоков в~приоритетных системах изучены пока недостаточно глубоко. 
В~частности, практически нет работ, посвященных исследованию многомерных распределений. 
Методы, применяемые в~работах~[1--3] при нахождении одномерных распределений, обобщить на 
случай многомерных распределений не удается. В~работе~[4] предложен метод нахождения любых 
конечномерных распределений в~системе с~относительным приоритетом. В~настоящей работе 
этот метод применен для анализа системы с~абсолютным приоритетом и~обслуживанием  
прерванного требования заново.

\vspace*{-1pt}

\section{Обозначения и~определения}


Рассматривается система обслуживания типа $M_r|G_r|1|\infty$ с~абсолютным 
приоритетом и~обслуживанием  прерванного требования заново. Считаем, что 
потоки пронумерованы в~порядке убывания их важности, т.\,е.\ требования из потока с~меньшим 
номером обладают более высоким приоритетом.
Пусть $a_1,\ldots,a_r$~--- интенсивности, а~$B_1(x),\ldots,B_r(x)$~--- функции 
распределения времен обслуживания требований потоков $1,\ldots,r$ соответственно.  Обозначим
\begin{gather*}
  \beta_i(s)=\int\limits_0^{\infty}e^{-sx}dB(x);\enskip \beta_{ij}=\int\limits_0^{\infty}x^jdB_i(x);\\[1pt]
   \sigma_k=a_1+\cdots+a_k,\enskip k=1,\ldots,r;\\ \sigma_0=0;\enskip \sigma=\sigma_r;\enskip 
  \rho_{11}=a_1\beta_{11};
  \end{gather*}
  
  \vspace*{-13pt}
  
  \noindent
  \begin{multline*}
   \rho_{i1}=a_1\beta_{11}+a_2\sigma_1^{-1}
  \left(\beta_2^{-1}(\sigma_1)-1\right)+\cdots\\
  \cdots +
  a_i\sigma_{i-1}^{-1}\left(\beta_i^{-1}(\sigma_{i-1})-1\right),\enskip i=2,\ldots,r.
 \end{multline*}
 
\noindent
Пусть, далее, $t_{iN}$~--- момент ухода из системы \mbox{$N$-го} требования приоритета~$i$ 
(нумерация требований проводится для каждого приоритета отдельно в~порядке их 
ухода из системы), $t_{i0}\hm=0$, $\tau_{iN}\hm=t_{iN}\hm-t_{i,N-1}$, $L_i(t)$~--- 
число требований $i$-го потока (приоритета~$i$) в~системе в~момент времени~$t,$
$i\hm=1,2,\ldots r$, $N\hm=1,2,\ldots$

Всюду в~дальнейшем будем считать выполненным условие эргодичности $\rho_{r1}\hm<1.$
Положим

\noindent
\begin{align*}
P_i\left(n,x\right)&=\lim\limits_{N\rightarrow\infty}\P\left(L_i(t_{iN}+0)=n,\tau_{iN}<x\right);
\\
Q_i\left(n,m,x,y\right)&=\lim\limits_{N\rightarrow\infty}\P\left(L_i(t_{iN}+0)=n,\right.\\
&\hspace*{-20pt}\left.L_i(t_{i,N-1}+0)=m,\tau_{iN}<x,\tau_{i,N-1}<y\right);
\\
p_i\left(z,s\right)&=\int\limits_0^{\infty}e^{-sx}\sum\limits_{n=0}^{\infty}
z^{n}d_xP_i\left(n,x\right);
\\
q_i\left(w,z,s_1,s_2\right)&={}\\
&\hspace*{-63pt}{}=\int\limits_0^{\infty}\!\int\limits_0^{\infty}\!e^{-s_1x}e^{-s_2y}
\sum\limits_{n=0}^{\infty}\sum\limits_{m=0}^{\infty}\!w^{n}z^{m}d_xd_yQ_i\left(n,m,x,y\right);
\\
f_i(s)&=\lim\limits_{N\rightarrow\infty}\int\limits_0^{\infty}e^{-sx}d\P(\tau_{iN}<x);
\\[-2pt]
g_i(s_1,s_2)&={}\\
&\hspace*{-61pt}{}=\lim\limits_{N\rightarrow\infty}\int\limits_0^{\infty}\!\int\limits_0^{\infty}\!
e^{-s_1x}e^{-s_2y}d_xd_y
\P(\tau_{iN}<x,\tau_{i,N-1}<y).
\end{align*}

\section{Предварительные результаты}

При изучении выходящих потоков будут использованы некоторые известные результаты (см.,
 например,~\cite{5-us}) для рассматриваемой системы массового обслуживания.
Пусть $\Pi_k$, $H_k$ и~$\Pi_{ki}$~--- \mbox{$k$-пе}\-ри\-од, $k$-цикл и~$ki$-пе\-ри\-од соответственно:
\begin{equation*}
\pi_k(s)=\mathbb{E} e^{-s\Pi_k};\ h_k(s)=\mathbb{E} e^{-s H_k};\
 \pi_{ki}(s)=\mathbb{E} e^{-s\Pi_{ki}}.
\end{equation*}
Тогда

\vspace*{-6pt}

\noindent
\begin{align*}
\pi_k(s)&=\sum\limits_{i=1}^ka_i\sigma_k^{-1}\pi_{ki}(s);\\
 h_k(s)&=\beta_k\left(s+\sigma_{k-1}\right)\times{}\\
 &{}\times
\left(1-\fr{\sigma_{k-1}\pi_{k-1}(s)}{s+\sigma_{k-1}}
\left(1-\beta_k\left(s+\sigma_{k-1}\right)\right)\right)^{-1},
\end{align*}

\vspace*{-6pt}

\noindent
а $\pi_{ki}(s)$~--- единственное решение системы уравнений

\vspace*{-6pt}

\noindent
\begin{multline*}
\pi_k(s)=\beta_i\left(s+\sigma_k-\sum\limits_{j=i}^ka_j\pi_{kj}(s)\right)-
\sum\limits_{j=1}^{i-1}\pi_{kj}(s)\times\\
\times\pi_{ki}(s)\fr{1-\beta_i\left(s+\sigma_k-\sum\nolimits_{j=i}^ka_j\pi_{kj}(s)\right)}
{\left(s+\sigma_k-\sum\nolimits_{j=i}^ka_j\pi_{kj}(s)\right)},\\
 i=1,\ldots,k.
\end{multline*}

\vspace*{-2pt}

%\columnbreak

Пусть, далее, $W_i(t)$~--- виртуальное время ожидания для требований приоритета~$i$ в~момент времени~$t$:

\noindent
\begin{align*}
\omega_i^{*}(s,t)&=\mathbb{E} e^{-sW_i(t)};\\
 \omega_i(s,q)&=\int\limits_0^{\infty}e^{-qt}
\omega_i^{*}(s,t)\,dt.
\end{align*}



\noindent
Тогда
\begin{align*}
\omega_1(s,q)&=\fr{1-sp_0^{(1)}(q)}{q-\alpha_1(s)},\\
\omega_i(s,q)&=\fr{1-sp_0^{(i)}(q)-(1-\pi_{i-1}(s))p_1^{(i)}(q)}{q-\alpha_i(s)},\\
&\hspace*{40mm} i=2,\ldots,r.
\end{align*}
Здесь
\begin{align*}
\alpha_k(s)&=s-a_k+a_kh_k(s);\\
 p_0^{(k)}(q)&=(q+\sigma_k-\sigma_k\pi_k(q))^{-1},\enskip k=1,\ldots,r;\\
p_1^{(k)}(q)&=\fr{1-(q+a_k-a_k\pi_{kk}(q))p_0^{(k)}(q)}{1-\pi_{k-1}(q+a_k-a_k\pi_{kk}(q))},\\ 
&\hspace*{41mm}k=2,\ldots,r.
\end{align*}

\vspace*{-16pt}


\section{Основные результаты}

\vspace*{-2pt}


В дальнейшем будут изучены выходящие потоки требований приоритетов $2,\ldots,r.$ 
Для требований первого приоритета справедливы результаты
для неприоритетной системы, в~которую поступает только первый поток.
Основные результаты работы содержатся в~приводимых ниже двух теоремах.


\smallskip

\noindent
\textbf{Теорема~1.}\
\textit{Функции $p_i\left(z,s\right)$ и~$q_i\left(z,w,s_1,s_2\right)$ определяются соотношениями}:

\vspace*{-6pt}

\noindent
\begin{multline}
\label{n4}
p_i\left(z,s\right)=\left(z^{-1}\left(p_i(z,0)-p_i(0,0)\right)+a_ip_i(0,0)\times{}\right.\\
{}\times\left.\omega_{i-1}\left(s+a_{i-1}-a_{i-1}\pi_{i-1,i-1}\left(s+a_i-a_iz\right)+{}\right.\right.\\
\left.\left.{}+a_i-a_iz,s+a_i\right)
\right)h_i\left(s+a_i-a_iz\right);
\end{multline}

\vspace*{-14pt}

\noindent
\begin{multline}
\label{n5}
q_i\left(z,w,s_1,s_2\right)={}\\
{}=
\left(z^{-1}\left(p_i(zw,s_2)-p_i(0,s_2)\right)+a_ip_i\left(0,s_2\right)\right.\times{}\\
{}\times\omega_{i-1}\left(s_1+a_{i-1}-a_{i-1}\pi_{i-1,i-1}
\left(s_1+a_i-a_iz\right)+{}\right.\\
\left.\left.{}+a_i-a_iz,s_1+a_i\right)\right)
h_i(s_1+a_i-a_iz).
\end{multline}

\vspace*{-2pt}

\noindent
\textit{Функция $p_i\left(z,0\right)$ равна}

\vspace*{-6pt}

\noindent
\begin{multline}
\label{n6}
p_i\left(z,0\right)=\fr{p_i(0,0)h_i\left(a_i-a_iz\right)}{z-h_i(a_i-a_iz)}
 \left(a_iz\omega_{i-1}
\left(a_{i-1}-{}\right.\right.\\
\left.\left.{}-a_{i-1}\pi_{i-1,i-1}(a_i-a_iz)+a_i-a_iz,a_i\right)-1\right)\,,
\end{multline}

\vspace*{-2pt} 

\noindent
\textit{где}

\noindent
\begin{align*}
p_2(0,0)&=a_2^{-1}\left(1-\rho_{21}\right)\left(\sigma_2-a_1\pi_1(a_2)\right);
\\
p_i(0,0)&=a_i^{-1}\left(1-\rho_{i1}\right)\left(\sigma_i-\sigma_{i-1}\pi_{i-1}(a_i)\right)\times{}\\
&\hspace*{30pt}{}\times
\left( 
\vphantom{\fr{\sigma_{i-1}-\sigma_{i-1}\pi_{i-1}(a_i)-a_{i-1}+a_{i-1}\pi_{i-1,i-1}(a_i)}
{1-\pi_{i-2}(a_i+a_{i-1}-a_{i-1}\pi_{i-1,i-1}(a_i))}}
1-\rho_{i-2,1}+\sigma_{i-2}^{-1}\rho_{i-2,1}\times{}\right.
\end{align*}

\noindent
\begin{align*}
&\hspace*{-35pt}\left.{}\times
\fr{\sigma_{i-1}-\sigma_{i-1}\pi_{i-1}(a_i)-a_{i-1}+a_{i-1}\pi_{i-1,i-1}(a_i)}
{1-\pi_{i-2}(a_i+a_{i-1}-a_{i-1}\pi_{i-1,i-1}(a_i))}\!\right)^{\!-1}\!\!,\\
 &\hspace*{139pt}i=3,\ldots,r\,;
\\
p_i(0,s)&=\left(v_i+a_ip_i(0,0)\omega_{i-1}\times{}\right.\\
&\hspace*{-25pt}\left.{}\times 
\left(s+a_{i-1}-a_{i-1}\pi_{i-1,i-1}(s+a_i)+a_i,s+a_i\right)\right)\times{}\\
&\hspace*{135pt}{}\times h_i\left(s+a_i\right);
\\
v_i&=p_i(0,0)\left(h_i^{-1}(a_i)-a_i\times{}\right.\\
&\hspace*{10pt}\left.{}\times\omega_{i-1} \left(a_{i-1}-a_{i-1}\pi_{i-1,i-1}(a_i)+a_i,a_i\right)\right).
\end{align*}


\noindent
Д\,о\,к\,а\,з\,а\,т\,е\,л\,ь\,с\,т\,в\,о\,.\ \
Рассматривая два соседних момента ухода требований $i$-го приоритета из сис\-те\-мы, имеем
\begin{multline}
\label{n7}
P_i(n,x)={}\\
{}=\sum\limits_{j=1}^{n+1}P_i(j,\infty)
\int\limits_0^xe^{-a_iu}\fr{(a_iu)^{n-j+1}}{(n-j+1)!}\,dH_i(u)+{}\\
{}+P_i(0,\infty)\int\limits_0^xG_i(u,n,x-u)\,d\left(1-e^{-a_iu}\right),
\end{multline}
где $G_i(u,k,v)$~--- вероятность того, что первое требование приоритета~$i$ 
 уйдет из системы через время, меньшее чем $u\hm+v,$ и~в~момент его ухода будет~$k$~требований 
 этого приоритета, при условии что оно поступает в~момент~$u$, 
а~в~начальный момент система свободна.

Положим
$$
g_i(u,z,s)=\sum\limits_{k=0}^{\infty}z^k\int\limits_0^{\infty}e^{-sx}d_xG_i(u,k,x).
$$
Тогда  при $i\geqslant 2$ имеем
\begin{multline*}
\!\!g_i(u,z,s)=\omega_{i-1}^{*}\left(s+a_{i-1}-
a_{i-1}\pi_{i-1,i-1}
\left(s+a_i-{}\right.\right.\hspace*{-1.67781pt}\\
\left.\left.{}-a_iz\right)
+a_i-a_iz,u\right)h_i\left(s+a_i-a_iz\right).
\end{multline*}

Переходя в~\eqref{n7} к~производящим функциям и~преобразованиям Лап\-ла\-са--Стилть\-еса, получаем~\eqref{n4}. 
Из~\eqref{n4} при $s\hm=0$ получаем~\eqref{n6}.
Устремляя в~\eqref{n6}  $z$ к~единице, находим~$p_i(0,0).$

Рассмотрим теперь три последовательных момента ухода из системы требований
 $i$-го приоритета. Имеем
 
 \noindent
\begin{multline}
\label{l1}
Q_i\left(n_1,n_2,x,y\right)={}\\
{}=P_i\left(n_2,y
\right)\int\limits_0^xe^{-a_iu}\fr{(a_iu)^{n_1+1-n_2}}{(n_1+1-n_2)!}\,dH_i(u),\\ 
n_2\geqslant 1\,,\ \  n_1\geqslant n_2-1\,;
\end{multline}

\vspace*{-12pt}

\noindent
\begin{multline}
\label{l2}
Q_i\left(n_1,0,x,y\right)={}\\
{}=Q_i(0,y)\int\limits_0^xG_i\left(u,n_1,x-u\right)\,d\left(1-e^{-a_iu}\right).
\end{multline}
Переходя в~\eqref{l1} и~\eqref{l2} к~производящим функциям и~преобразованиям Лап\-ла\-са--Стилть\-еса, 
получаем~\eqref{n5}.



\smallskip

\noindent
\textbf{Теорема~2.}\
\textit{Справедливы следующие соотношения}:
\begin{multline*}
f_i(s)=\left(1-p_i(0,0)+a_ip_i(0,0)\times{}\right.\\
\left.{}\times\omega_{i-1} \left(s+a_{i-1}-a_{i-1}\pi_{i-1,i-1}(s),s+a_i\right)\right)h_i(s),
\end{multline*}


\vspace*{-12pt}

\noindent
\begin{multline*}
g_i(s_1,s_2)=\left(1-p_i(0,s_2)+a_ip_i(0,s_2)\times{}\right.\\
\left.{}\times\omega_{i-1}\left(s_1+a_{i-1}-a_{i-1}\pi_{i-1,i-1}(s_1),s_1+a_i\right)\right)\times{}\\
{}\times h_i(s_1).
\end{multline*}


\noindent
Д\,о\,к\,а\,з\,а\,т\,е\,л\,ь\,с\,т\,в\,о\ \ непосредственно вытекает из результатов теоремы~1 и~соотношений
$f_i(s)\hm=p_i\left(1,s\right)$ и~$g_i(s_1,s_2)\hm=q_i\left(1,1,s_1,s_2\right).$

{\small\frenchspacing
{%\baselineskip=10.8pt
%\addcontentsline{toc}{section}{References}
\begin{thebibliography}{9}
\bibitem{1-us}
\Au{Nain P.} Interdeparture times from a queuing system with preemptive resume priority~// 
Perform. Evaluation, 1984. Vol.~4. Iss.~2. P.~93--98.
doi: 10.1016/0166-\mbox{5316(84)90003-8}.

\bibitem{2-us}
\Au{Stanford D.\,A.} Interdeparture time distributions in the non-preemptive priority 
$\Sigma\ M_i|G_i|1$ queue~// Perform. Evaluation, 1991. Vol.~12. Iss.~2.   P.~43--60.
%doi: 10.1016/0166-5316(91)90014-T.

\bibitem{3-us}
\Au{Stanford D.\,A.} Waiting and interdeparture times in priority queues with
 Poisson- and general-arrival streams~// Oper.
Res., 1995. Vol.~45. Iss.~5. P.~725--735.
\bibitem{4-us}
\Au{Ушаков В.\,Г., Ушаков Н.\,Г.} Выходящие потоки в~однолинейной системе 
с~относительным приоритетом~// Информатика и~её применения, 2019. Т.~13. Вып.~4. С.~42--47.
 doi: 10.14357/19922264190407.
\bibitem{5-us}
\Au{Матвеев В.\,Ф., Ушаков В.\,Г.} Системы массового
обслуживания.~--- М.: Изд-во Московского ун-та, 1984. 240~с.
 \end{thebibliography}

}
}

\end{multicols}

\vspace*{-3pt}

\hfill{\small\textit{Поступила в~редакцию 07.04.2021}}

%\vspace*{8pt}

%\pagebreak

\newpage

\vspace*{-28pt}

%\hrule

%\vspace*{2pt}

%\hrule

%\vspace*{-2pt}

\def\tit{THE MULTIVARIATE DISTRIBUTIONS OF~OUTPUT STREAMS IN~A~QUEUEING SYSTEM WITH~PREEMPTIVE REPEAT PRIORITY}

\def\titkol{The multivariate distributions of~output streams in~a~queueing system with~preemptive repeat priority}

\def\aut{V.\,G.~Ushakov$^{1,2}$ and~N.\,G.~Ushakov$^{3,4}$}

\def\autkol{V.\,G.~Ushakov and~N.\,G.~Ushakov}

\titel{\tit}{\aut}{\autkol}{\titkol}

\vspace*{-11pt}


\noindent
$^1$Department of Mathematical Statistics, Faculty of Computational Mathematics and Cybernetics,
M.\,V.~Lomo-\linebreak 
$\hphantom{^1}$nosov Moscow State University, 1-52~Leninskie Gory, GSP-1, Moscow 119991, Russian Federation

\noindent
$^2$Institute of Informatics Problems, Federal Research Center ``Computer Science and Control'' of the 
Russian\linebreak
$\hphantom{^1}$Academy of Sciences, 44-2~Vavilov Str., Moscow 119333, Russian Federation

\noindent
$^3$Institute of Microelectronics Technology and High-Purity Materials of the Russian Academy of 
Sciences,\linebreak
$\hphantom{^1}$6~Academician Osipyan Str., Chernogolovka, Moscow Region 142432, Russian Federation

\noindent
$^4$Norwegian University of Science and Technology, 15A~S.\,P.~Andersensvei, Trondheim 7491, Norway

 
\def\leftfootline{\small{\textbf{\thepage}
\hfill INFORMATIKA I EE PRIMENENIYA~--- INFORMATICS AND
APPLICATIONS\ \ \ 2021\ \ \ volume~15\ \ \ issue\ 2}
}%
\def\rightfootline{\small{INFORMATIKA I EE PRIMENENIYA~---
INFORMATICS AND APPLICATIONS\ \ \ 2021\ \ \ volume~15\ \ \ issue\ 2
\hfill \textbf{\thepage}}}

\vspace*{3pt}   




\Abste{The paper studies a single server queuing system with~$r$~types 
of customers, preemptive repeat priority, and an infinite number of positions in the queue.  
The arrival stream of customers of each type is a~Poisson stream. Each type has its own 
generally distributed service time characteristics. The main result is the Laplace--Stieltjes 
transform  of one- and two-dimensional stationary distribution functions of the interdeparture 
times for each type of customers. The analysis of the output process is carried out by the method 
of embedded Markov chains. As embedded times, successive moments of the end of service
 of the same type customers are selected. From a~practical perspective, an accurate characterization 
 of the interdeparture time process is necessary when studying open networks of queues.}


\KWE{output stream; preemptive repeat priority; embedded Markov chain; single server}



\DOI{10.14357/19922264210204}

\vspace*{-15pt}

 \Ack
\noindent
The research was supported by the Ministry of Science and Higher Education of the Russian Federation, 
project No.\,075-15-2019-1621.

\vspace*{12pt}

  \begin{multicols}{2}

\renewcommand{\bibname}{\protect\rmfamily References}
%\renewcommand{\bibname}{\large\protect\rm References}

{\small\frenchspacing
 {%\baselineskip=10.8pt
 \addcontentsline{toc}{section}{References}
 \begin{thebibliography}{9}
 
\vspace*{-4pt}
 
\bibitem{1-us-1}
\Aue{Nain, P.} 1984. Interdeparture times from a queuing system with preemptive resume priority. 
\textit{Perform. Evaluation} 4(2):93--98.
doi: 10.1016/0166-5316(84)90003-8.

\vspace*{-1pt}

\bibitem{2-us-1}
\Aue{Stanford, D.\,A.} 1991. Interdeparture time distributions in the non-preemptive priority 
$\Sigma\ M_i|G_i|1$ queue. \textit{Perform. Evaluation} 12(2):43--60.
%doi: 10.1016/0166-5316(91)90014-T.

\vspace*{-1pt}

 
\bibitem{3-us-1}
\Aue{Stanford, D.\,A.} 1995. Waiting and interdeparture times in priority queues with Poisson-
 and general-arrival streams. \textit{Oper. Res.} 45(5):725--735.


\columnbreak


\bibitem{4-us-1}
\Aue{Ushakov, V.\,G., and  N.\,G.~Ushakov.} 2019. Vykhodyashchie potoki 
v~odnolineynoy sisteme s~otnositel'nym prioritetom [The output streams in the single server 
queueing system with a head of the line priority]. \textit{Informatika i~ee Primeneniya~--- 
Inform. Appl.} 13(4):42--47.
 doi: 10.14357/ 19922264190407.
 
 \vspace*{9pt}

\bibitem{5-us-1}
\Aue{Matveev, V.\,F., and V.\,G.~Ushakov.} 
1984. \textit{Sistemy massovogo obsluzhivaniya } [Queueing systems]. 
Moscow: MSU Publs. 240~p.

\end{thebibliography}

 }
 }

\end{multicols}

\vspace*{-3pt}

  \hfill{\small\textit{Received April~7, 2021}}


%\pagebreak

\vspace*{-12pt}     

\Contr

\noindent
\textbf{Ushakov Vladimir G.} (b.\ 1952)~--- 
Doctor of Science in physics and mathematics, professor, Department of Mathematical Statistics, 
Faculty of Computational Mathematics and Cybernetics, M.\,V.~Lomonosov Moscow State University, 
1-52~Leninskie Gory, GSP-1, Moscow 119991, Russian Federation; senior scientist, Institute of
 Informatics Problems, Federal Research Center ``Computer Science and Control'' of the Russian
  Academy of Sciences, 44-2~Vavilov Str., Moscow 119333, Russian Federation; \mbox{vgushakov@mail.ru}
  
  \vspace*{3pt}
  
  \noindent
  \textbf{Ushakov Nikolai G.} (b.\ 1952)~--- Doctor of Science
   in physics and mathematics, leading scientist, Institute of Microelectronics Technology and 
   High-Purity Materials of the Russian Academy of Sciences, 6~Academician Osipyan Str., 
   Chernogolovka, Moscow Region 142432, Russian Federation; professor, Norwegian
    University of Science and Technology, 15A~S.\,P.~Andersensvei, Trondheim 7491, Norway; 
    \mbox{ushakov@math.ntnu.no}



     
\label{end\stat}

\renewcommand{\bibname}{\protect\rm Литература}