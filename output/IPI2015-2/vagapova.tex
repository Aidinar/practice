\def\stat{vagapova}

\def\tit{СРАВНИТЕЛЬНЫЙ АНАЛИЗ ПРИМЕНЕНИЯ ЭВРИСТИЧЕСКОГО И~МЕТАЭВРИСТИЧЕСКОГО АЛГОРИТМОВ К~ЗАДАЧЕ О~ШКОЛЬНОМ АВТОБУСЕ$^*$}

\def\titkol{Сравнительный анализ применения эвристического и~метаэвристического алгоритмов к задаче о школьном автобусе}

\def\aut{Е.\,М.~Бронштейн$^1$, Д.\,М.~Вагапова$^2$}

\def\autkol{Е.\,М.~Бронштейн, Д.\,М.~Вагапова}

\titel{\tit}{\aut}{\autkol}{\titkol}

\index{Бронштейн Е.\,М.}
\index{Вагапова Д.\,М.}

{\renewcommand{\thefootnote}{\fnsymbol{footnote}} \footnotetext[1]
{Работа выполнена при поддержке РФФИ (проект 13-01-00005).}}


\renewcommand{\thefootnote}{\arabic{footnote}}
\footnotetext[1]{Уфимский государственный авиационный технический университет, bro-efim@yandex.ru}
\footnotetext[2]{Уфимский государственный авиационный технический университет, vagapova-dm@mail.ru}

\Abst{Рассматривается задача о школьном автобусе, которая заключается в~обеспечении доставки школьников по окончании занятий из школы по их
остановкам. Целью является минимизация длины максимального из
маршрутов. Представлен краткий обзор работ по данной
тематике.  Приведена постановка задачи и~формализация.
Описан эвристический алгоритм, предложенный
авторами ранее. Также описан двухэтапный алгоритм на основе
метаэвристики муравьиной колонии: после первоначальной кластеризации
остановок, на которых высаживаются школьники, к каждому кластеру
применяется алгоритм муравьиной колонии с~различными значениями
параметров. Представлены результаты сравнения
эффективности предложенных алгоритмов, а~также результаты работы
программ для двух алгоритмов.}  %В~заключении приводятся выводы.}

\KW{маршрутизация; задача о школьном автобусе; алгоритм муравьиной
колонии; кластеризация}

\DOI{10.14357/19922264150207}

\vskip 14pt plus 9pt minus 6pt

\thispagestyle{headings}

\begin{multicols}{2}

\label{st\stat}

\section{Введение}

     Задача о школьном автобусе (School Bus Routing Problem, SBRP)
заключается в~доставке школьников после окончания занятий до их остановок
за минимальное время. Впервые задача была сформулирована в~\cite{1-va}.
Задача о школьном автобусе входит в~общее семейство задач маршрутизации\linebreak
(Vehicle Routing Problem, VPR). Обзор точных и~эвристических методов
решения задач маршрутизации пред\-став\-лен в~\cite{2-va}. Задача маршрутизации с~одним транспортным средством без дополнительных ограничений является
задачей коммивояжера (Traveling Salesman Problem, TSP).

   Рассматриваемые задачи относятся к классу NP-труд\-ных, поэтому для их
решения целесообразно использовать эвристические и~метаэвристические
алгоритмы. Простой эвристический алгоритм решения задачи о школьном
автобусе предложен в~работе~\cite{3-va}. В~статье разработано решение
данной задачи с~помощью двухэтапного подхода на основе метаэвристического
алгоритма муравьиной колонии (Ant Colony Optimization, ACO). На первом
этапе производится кластеризация остановок, на которых высаживаются
школьники, а~затем к каж\-до\-му кластеру применяется алгоритм муравьиной
колонии. Для расширения области поиска решений итерационно производится
смещение границ кластеров. Число таких итераций равно числу остановок.
Впервые двухэтапный подход такого типа для решения задачи маршрутизации
был предложен в~\cite{4-va}.

     В~\cite{5-va} предложен новый подход к решению задачи о школьном
автобусе в~городе. Рассматривается многомерная целевая функция. Задача
разбивается на две подзадачи. Первая~--- это распределение студентов по
автобусам, и~вторая~--- построение автобусных маршрутов. Для первой
подзадачи используется алгоритм районирования.

     Решение задачи о~школьном автобусе с~по\-мощью модифицированного
алгоритма муравьиной колонии предложено в~[6]. Задача решается на
реальных данных г.~Богота (Колумбия). Используется двухэтапный подход:
сначала производится кластеризация остановок согласно их географическому
расположению (с~севера на юг, затем с~запада на восток), а~потом применяется
модифицированный алгоритм муравьиной колонии для задачи коммивояжера.

     Алгоритм муравьиной колонии является достаточно эффективной
метаэвристикой для нахождения приближенных решений различных задач
дискретной оптимизации. Результаты, полученные с~помощью разработанного
варианта двухэтапного подхода, сравниваются с~результатами, полученными с~помощью эвристического алгоритма из работы~\cite{3-va}.

\section{Постановка задачи}

     Более подробное описание задачи приведено в~\cite{3-va}. По окончании
занятий требуется развезти школьников по их остановкам на нескольких
автобусах. При этом школьников следует доставить за минимально возможное
время. Это условие будет выполнено, если маршрут с~максимальной длиной
будет минимальным среди всех способов доставки. Предполагается, что
автобусы движутся с~постоянной скоростью.

     Для построения математической модели введем соответствующие
величины.
     \begin{enumerate}[1.]
\item Для каждой из $N$ остановок задана величина~$D_n$ ($n\hm=1,\ldots
,N$)~--- число школьников, которых надо доставить на эту остановку. Для
единообразия школу считаем 0-й остановкой.
\item Каждый из $K$ автобусов характеризуется чис\-лом~$T^k$ ($k\hm = 1,\ldots
, K$) посадочных мест.
\item Матрица $C =\parallel c(i, j)\parallel$ содержит минимальные
расстояния (от $i$-й остановки до $j$-й). Элементы матрицы~$C$
удовлетворяют неравенству треугольника.
\item Необходимо найти величины~$x_n^k$ ($n\hm= 1,\ldots, N$; $k\hm=1,\ldots
,K$)~--- число школьников, которых необходимо доставить $k$-м автобусом на
$n$-ю остановку.
\end{enumerate}

     Опишем ограничения задачи.

     На каждую остановку, кроме школы,
должно быть доставлено не\-от\-ри\-ца\-тель\-ное число школьников:
     \begin{equation}
     x_n^k\geq 0 \enskip (n=1,\ldots , N;\ k=1,\ldots , K), \ \mbox{целые}\,.
     \label{e1-va}
     \end{equation}

     На каждую остановку необходимо доставить всех школьников, для
которых это необходимо:
     \begin{equation}
     \sum\limits_{k=1}^K x_n^k=D_n\enskip (n=1,\ldots, N)\,.
     \label{e2-va}
     \end{equation}

     Ограничения на вместимость автобусов:
     \begin{equation}
     \sum\limits_{n=1}^N x_n^k \leq T^k\enskip (k=1,\ldots, K)\,.
     \label{e3-va}
     \end{equation}

     Добавлением виртуальных школьников в~количестве
$\sum\nolimits_{k=1}^K T^k\hm- \sum\nolimits_{n=1}^N D_n$ (в~случае
положитель\-ности этой величины), для которых пунктом доставки является
школа, можно добиться того, что ограничения~(\ref{e3-va}) выполняются в~форме равенств. Таким образом, полагаем, что общее число школьников равно
числу мест в~автобусах: $\sum\nolimits_{n=1}^N D_n\hm= \sum\nolimits_{k=1}^K
T^k$.

     Пусть $R^k=\left\{ i\colon x_i^k\geq1\right\}$~--- множество остановок, на
которых должен высадить пассажиров \mbox{$k$-й} автобус. Следует отметить, что
величинами~$x_n^k$ маршруты автобусов однозначно не определяются. Пусть
$P(R^k)$~--- минимальная длина маршрута, проходящего через все остановки
из~$R^k$, у~которого начальный пункт~--- школа.

     Целевая функция:
     \begin{equation}
     \max\limits_k \left\{ P(R^k)\right\}\to \min\,.
     \label{e4-va}
     \end{equation}

     Результатом решения задачи~(\ref{e1-va})--(\ref{e4-va}) является график
доставки, при котором школьник, покидающий автобус последним, проедет
минимально возможное расстояние. Очевидно, что
     задача~(\ref{e1-va})--(\ref{e4-va}) имеет решение (возможно, не
единственное).

В работе~\cite{3-va} путем добавления новых переменных данная задача
сведена к линейной частично целочисленной форме, для задач малой
раз\-мер\-ности приведено точное решение, а~также предложен эвристический
алгоритм, описанный далее.

\section{Эвристический алгоритм}

     Для простоты полагаем, что все автобусы имеют одну и~ту же
вместимость~$T$.

     Найдем максимальное из расстояний в~графе от нулевой вершины
(школы) до всех вершин. Пусть соответствующий путь имеет вид
     0--$i_1$--$\cdots$--$i_k$.

Заполняем автобус по следующим правилам.

     Если $\sum\nolimits_{s=1}^k D_{i_s} \geq T$, то автобус доставит школьников на
остановки, расположенные на пути 0--$i_1$--$\cdots$--$i_k$, начиная от
последней в~пределах вмес\-ти\-мости (может оказаться, что на одну из остановок
будут доставлены этим автобусом не все школьники, которым это необходимо),
иначе \mbox{найдем} остановки~$i_{s^*}, j^*$, на которых достигается
     $\min\limits_{s,j}\ \left\{
     c(i_k,j)\colon \ \ j\hm\not\in \{i_0,\ldots, i_k\};\ \
     c(i_s,j)\hm+c(j,i_{s+1}) \hm-\right.$\linebreak $\left.-\;c(i_s,i_{s+1})\colon
s\hm=0,\ldots, k-1, j\hm\not\in
\{i_0,\ldots, i_k\}
\right\}$,
где $c(i, j)$~--- минимальное расстояние между остановками, $i_0\hm =0$.

     Если число школьников, которых необходимо доставить на остановки
$i_1,\ldots, i_k, j^*$, не меньше~$T$, то назначаем автобус, который доставит
школьников в~пределах вместимости автобусов на остановки пути
     0--$i_1$--$\cdots$--$i_k$--$j^*$, начиная с~последней, если минимум
достигается на величинах первого вида, и~пути
     0--$i_1$--$\cdots$--$i_{s^*}$--$j^*$--$i_{s^*+1}$--$\cdots$--$i_k$ если
минимум достигается на величинах второго типа.

     В противном случае процесс добавления вершин к пути продолжается
аналогично до заполнения автобуса. В~результате как число свободных
автобусов, так и~число остановок, на которые еще не доставлены все
школьники, уменьшаются. Далее автобусы заполняются последовательно по
тому же алгоритму.

\section{Двухэтапный алгоритм на~основе метаэвристики
муравьиной колонии}

     Алгоритм муравьиной колонии был предложен М.~Дориго~\cite{7-va}.
Алгоритм имитирует поведение му-\linebreak равьев в~колонии. Отдельный муравей не
\mbox{способен} выжить, однако вся колония достигает высокого уровня
самоорганизации благодаря <<со\-ци\-аль\-ности>> муравьев. Цель муравьев~---
найти кратчайший путь от жилища до источника пищи. В~ряде работ
(см., например,~[8--14]) были предложены различные модификации алгоритма.
В~данной работе воспользуемся классическим вариантом данного алгоритма.

     Введем следующие обозначения:
\begin{description}
\item[\,]     $A_{ij}(t)$~--- количество феромона на дуге $(i, j)$ перед $t$-й итерацией
(начальные значения $A_{ij}(1)$ принимаются одинаковыми положительными
величинами для всех дуг);
\item[\,]
     $\tau_{ij}^k(t)$~--- количество феромона, которое $k$-й муравей
оставляет на дуге $(i,j)$ на $t$-й итерации;
\item[\,]
     $B_{ij}(t)=\sum\nolimits_{k=1}^m \tau_{ij}^k(t)$~--- количество феромона,
которое все муравьи оставляют на дуге $(i,j)$ на $t$-й итерации ($m$~--- число
муравьев).
\end{description}

     Для достижения цели муравьи пользуются четырьмя принципами:
     \begin{enumerate}[1.]
\item \textit{Положительная обратная связь. }
\begin{equation}
\tau_{ij}^k(t) =\begin{cases}
\fr{Q}{L^k(t)}\,, &\ \mbox{если } (i,j)\in T^k(t)\,;\\
0 & \ \mbox{иначе}\,,
\end{cases}
\label{e5-va}
\end{equation}
где $T^k(t)$~--- маршрут, пройденный $k$-м му\-равь\-ем на $t$-й итерации;
$L^k(t)$~--- длина маршрута $T^k(t)$;
$Q$~--- масштабирующий параметр, одного порядка с~длиной оптимального
маршрута (в~частности, за $Q$ можно принять длину самого длинного пути от
школы до всех остановок).
\item \textit{Отрицательная обратная связь.} Феромон испаряется со
временем. После каждой итерации количество феромона определяется по
сле\-ду\-ющей формуле:
\begin{equation}
A_{ij}(t+1) =(1-\rho) A_{ij}(t)+B_i(t)\,,
\label{e6-va}
\end{equation}
где $\rho$~--- коэффициент испарения, $\rho \hm\in [0,1]$.
\item \textit{Случайность.} Каждый муравей выбирает дугу случайным
образом. Вероятность перехода \mbox{$k$-го} муравья от $i$-й остановки к $j$-й на
итерации~$t$ вычисляется по следующей формуле:

\noindent
\begin{multline}
\rho_{ij}^k(t)={}\\{}=\begin{cases}\fr{[A_{ij}(t)]^\alpha [1/D_{ij}]^\beta} {\sum_{l\in
J_i^k} [A_{ij}(t)]^\alpha [1/D_{il}]^\beta}\,, &\ \mbox{если } j\in J_i^k\,;\\
0 &\ \mbox{иначе}\,,
\end{cases}
\label{e7-va}
\end{multline}
где $D_{ij}$~--- расстояние от $i$-й до $j$-й остановки;
$\alpha$~--- параметр, регулирующий влияние уровня феромона на вероятность
$p_{ij}^k(t)$, при $\alpha \hm=0$ муравей будет выбирать следующую
остановку, только исходя из расстояния до нее (жадный алгоритм);
$\beta$~--- параметр, регулирующий влияние расстояния между остановками на
вероятность $p_{ij}^k(t)$, при $\beta\hm = 0$ муравей будет выбирать
следующую остановку, только исходя из уровня феромона, что приведет к
быст\-рой схо\-ди\-мости к некоторому субоптимальному решению;
$J_i^k$~--- список остановок, еще не посещенных $k$-м муравьем.

По формуле~(\ref{e7-va}) вычисляются вероятности перехода муравья на
каждую из непосещенных остановок, выбор следующей остановки
производится случайно в~соответствии с~вычисленными вероятностями.
\item \textit{Многократность (множество муравьев)}. Длинные дуги будут
содержать меньшее количество феромона, чем короткие, в~соответствии
с~(\ref{e5-va}). Муравьи будут стремиться к дугам с~большим количеством
феромонов и~тем самым будут предпочитать короткие дуги длинным.
\end{enumerate}

     Двухэтапный алгоритм решения задачи о~школьном автобусе имеет
следующую структуру. Первый этап заключается в~кластеризации пунктов
(остановок). На втором этапе для построения маршрутов будет использоваться
алгоритм муравьиной колонии с~различными значениями параметров~$\alpha$,
$\beta$, $\rho$. Заметим, что для формирования маршрута одного автобуса
запускается колония муравьев.

     \textit{Первый этап~--- кластеризация.}

     Целью данного этапа является распределение школьников по маршрутам
(автобусам). Для этого разобьем всю плоскость, на которой лежат остановки
(рис.~1), на кластеры (штриховые линии~--- границы кластеров), каждый
кластер соответствует одному автобусу. Выбираем начальный луч, проходящий
через некоторую остановку. Двигаемся по часовой стрелке (на рис.~1
обозначена пунктиром). Кластером является минимальный сектор, для
ко-\linebreak\vspace*{-12pt}

\begin{center}  %fig1
\vspace*{-1pt}
 \mbox{%
 \epsfxsize=71.512mm
 \epsfbox{vag-1.eps}
 }


\end{center}

\noindent
{{\figurename~1}\ \ \small{Схема разбиения множества остановок на клас\-теры}}




\vspace*{9pt}


\addtocounter{figure}{1}


\noindent
торого число школьников, которых надо доставить
 на остановки,
принадлежащие этому кластеру, не
меньше вместимости транспортного
средства. Затем формируются следующие кластеры. Если упомянутое число
школьников в~кластере больше вмес\-ти\-мости автобуса, то граничная остановка
кластера войдет в~следующий кластер с~соответственно уменьшенным чис\-лом
школьников. Для по\-стро\-ения начальных лучей при разбиении на кластеры
последовательно перебираются все остановки.



     Приведем пример, который показывает целесообразность изменения
начального луча. Пусть три остановки имеют координаты: $A(1,\,0)$, $B(1,\,1)$,
$C(0,\,1)$; школа расположена в~точке О(0,\,0); вместимость автобуса~---
20~школьников; автобусов~--- два; на остановки~$A$ и~$C$ надо доставить по
10~школьников; на остановку~$B$~--- 20; прямые дороги с~двусторонним
движениям соединяют пары остановок ($OA$), ($AB$), ($BC$), ($CO$). Если в~качестве
начального принять луч~$OB$, то в~один кластер попадает остановка~$B$,
а~в~другой~--- остановки~$A$ и~$C$. Расстояние, необходимое для доставки одного
из школьников во втором кластере, равно~3. Если же в~качестве начального
луча взять~$OA$, то в~один кластер попадут остановки~$A$ и~$B$ (с частичной
доставкой), а~во второй~--- $B$ (с~частичной доставкой) и~$C$. Максимальное
расстояние доставки равно 2.

     \textit{Второй этап~--- маршрутизация.}

     Целью данного этапа является построение маршрутов движения
автобусов, т.\,е.\
установление порядка посещения остановок автобусами
в~каж\-дом кластере (минимальных гамильтоновых путей). Следует отметить, что
маршруты в~общем случае могут выходить за границы кластеров. Для
по\-стро\-ения маршрутов используется алгоритм муравьиной колонии. Автобус
доставляет всех школьников на каждую из остановок кластера, кроме
последней, а~оставшихся школьников на эту остановку до\-став\-ля\-ет автобус,
обслуживающий следующий кластер (возможно, не один). Ниже приводится
описание алгоритма.
     \begin{enumerate}[1.]
\item Цикл по числу кластеров $u\hm= (1$, число автобусов).\\[-13.5pt]
\item Инициализация параметров алгоритма~--- $\alpha$, $\beta$, $\rho$,
$Q$, $A_{ij}(1)$.\\[-13.5pt]
\item Размещение муравьев в~начальной вершине (школе). Число
муравьев равно числу остановок на данном маршруте.\\[-13.5pt]
\item Выбор начального кратчайшего маршрута и~определение длины
этого пути~$L^{u*}$.\\[-13.5pt]
\item Цикл по времени жизни колонии $t\hm=(1$, время жизни колонии).\\[-13.5pt]
\item Цикл по числу муравьев $ k\hm = (1$, число остановок на маршруте
($m$)).\\[-13.5pt]
\item Построение маршрута $T^{ku}(t)$ по формуле~(\ref{e7-va}) и~расчет длины~$L^{ku}(t)$.\\[-13.5pt]
\item Конец цикла по муравьям.\\[-13.5pt]
\item Проверка всех $L^{ku}(t)$ на лучшее решение по сравнению
с~$L^{u*}$.\\[-13.5pt]
\item В случае если решение $L^{ku}(t)$ лучше, обновление~$L^{u*}$
и~$T^{u*}$.\\[-13.5pt]
\item Цикл по всем дугам графа.\\[-13.5pt]
\item Обновление следов феромона на дуге по правилам~(\ref{e5-va})
и~(\ref{e6-va}).\\[-13.5pt]
\item Конец цикла по дугам.\\[-13.5pt]
\item Конец цикла по времени жизни колонии.\\[-13.5pt]
\item Вывод кратчайшего маршрута~$T^{u*}$ и~его длины~$L^{u*}$.\\[-13.5pt]
\item Конец цикла по кластерам.
\end{enumerate}

\vspace*{-9pt}



\section{Вычислительный эксперимент}

\vspace*{-2pt}

     Двухэтапный алгоритм, описанный в~разд.~4, и~эвристический алгоритм,
описанный в~разд.~3, были реализованы в~среде Borland DELPHI~6. Вычисления
проводились на персональном компьютере
(процессор: Intel(R) Core(TM) 2~Duo CPU T5750 2,00~GHz,
ОЗУ (RAM): 3~ГБ, ОС: Windows Vista Home Premium~32). Число автобусов
принималось равным~10 или~20. Вместимость автобусов принималась равной
20, 30 или~60. Число остановок~--- 50, 60, 70 или~80. Параметр~$\alpha$
принимался равным~0, 1, 2 или~6; $\beta$~--- 0, 1, 2,~6; $\rho$~---
0,2, 0,5, 0,7;
$Q\hm = 10\,000$; начальный уровень феромона $A_{ij}(1) \hm= 100$. Время
жизни колонии~--- 200. С~помощью датчика псевдослучайных чисел
генерировались значения
 координат каждой остановки
(расстояние бралось
 евклидово), а~также чис\-ло школьников на каждой\linebreak\vspace*{-12pt}

\pagebreak

\end{multicols}

     \begin{figure} %fig2
         \vspace*{1pt}
 \begin{center}
 \mbox{%
 \epsfxsize=162.28mm
 \epsfbox{vag-2.eps}
 }
\end{center}
 \vspace*{-13pt}
\Caption{Время работы эвристического~(\textit{а})
и~двухэтапного~(\textit{б}) алгоритмов}
%\end{figure*}
%     \begin{figure*} %fig3
              \vspace*{3pt}
 \begin{center}
 \mbox{%
 \epsfxsize=162.28mm
 \epsfbox{vag-4.eps}
 }
\end{center}
 \vspace*{-13pt}
     \Caption{Отношение времени работы двухэтапного алгоритма к времени работы
эвристического алгоритма~(\textit{а})
и~отношение максимальных длин маршрутов, полученных двухэтапным
и~эвристическим алгоритмами~(\textit{б})}
\vspace*{-3pt}
      \end{figure}


\begin{multicols}{2}

\noindent
остановке с~учетом того, что общее число школьников равно общему чис\-лу
посадочных мест. Для каждого
чис\-ла остановок генерировалось 10~примеров.
{\looseness=-1

}

     Рисунки~2 и~3 иллюстрируют зависимость средних значений величин,
полученных в~ходе вы\-чис\-ли\-тельного эксперимента, от числа остановок.
     	

     Отметим, что приводится время расчета (до 4~мин)\ для одного набора
параметров алгоритма муравьиной колонии и~для одного расположения
начального луча на первом этапе, а~именно для тех, при которых получен
лучший результат. При переборе всех наборов величин время вычислений
доходит до 10~ч.

\vspace*{-14pt}


    \section{Заключение}

    \vspace*{-4pt}

В работе рассматривается задача о школьном автобусе, разработан
двухэтапный алгоритм решения, основанный на алгоритме муравьиной
колонии. Производится сравнение его эффективности с~простым
эвристическим алгоритмом.

	Вычислительный эксперимент показал, что
\begin{enumerate}[(1)]
\item время работы эвристического алгоритма слабо
изменяется с~ростом числа остановок;
\item время работы двухэтапного алгоритма и~его отношение к времени
работы эвристического алгоритма резко возрастают с~ростом числа остановок;
\item отношение максимальных длин маршрутов находится в~промежутке
0,75--1,15 и~слабо изменяется с~ростом числа остановок.
\end{enumerate}

Таким образом, простой эвристический алгоритм сопоставим по
эффективности полученного результата с~двухэтапным, но гораздо менее
трудоемок.

\vspace*{-9pt}

{\small\frenchspacing
 {%\baselineskip=10.8pt
 \addcontentsline{toc}{section}{References}
 \begin{thebibliography}{99}
\bibitem{1-va}
\Au{Newton R.\,M., Thomas W.\,H.} Design of school Bus Routes by computer~//
Socio-Economic Planning Science, 1969. Vol.~3. P.~75--85.
\bibitem{2-va}
\Au{Archetti C.} Matheuristics for routing problems. {\sf
http:// www.sintef.no/contentassets/cfb19ab9b7c74d03904c7\linebreak
746ee1d8e77/matheuristics\_routing\_verolog2014\_new.\linebreak pdf}.
\bibitem{3-va}
\Au{Бронштейн Е.\,М., Вагапова Д.\,М., Назмутдинова~А.\,В.} О~построении
семейства маршрутов доставки школьников за минимальное время~//
Автоматика и~телемеханика, 2014. №\,7. С.~43--51.
\bibitem{4-va}
\Au{Fisher M.\,L., Jaikumar R.} A~generalized assignment heuristic for vehicle
routing~// Networks, 1981. Vol.~11. No.\,2. P.~109--124.
\bibitem{5-va}
\Au{Bowerman R., Hall B., Calamai P.} A~multi-objective optimization approach to
urban school Bus Routing: Formulation and solution method~// Transportation
Research Part~A: Policy and Practice, 1995. Vol.~29. No.\,2. P.~107--123.
\bibitem{6-va}
\Au{Arias-Rojas J.\,S., Jimenez J.\,F., Montoya-Torres~J.\,R.} Solving of school bus
routing problem by ant colony~// Revista EIA, 2012. Vol.~9. No.\,17. P.~193--208.
\bibitem{7-va}
\Au{Dorigo M.} Optimization, learning and natural algorithms. PhD Thesis.~---
Milano, Italy: Politecnico di Milano, 1992.
\bibitem{11-va} %8
\Au{Gambardella L.\,M., Dorigo M.} Ant-Q: A~reinforcement learning approach to
the traveling salesman problem~// 12th Conference (International) on Machine
Learning.~--- Tahoe City: Morgan Kaufmann, 1995. P.~252--260.
\bibitem{8-va} %9
\Au{Dorigo M., Maniezzo V., Colorni~A.} The Ant System: Optimization by a
colony of cooperating agents~// IEEE Trans. Systems Man Cybernetics. Part~B,
1996. Vol.~26. No.\,1. P.~29--41.
\bibitem{9-va} %10
\Au{Dorigo M., Gambardella L.\,M.} Ant Colony System: A~cooperative learning
approach to the traveling salesman problem~// IEEE Trans. Evol.
Comput., 1997. Vol.~1. No.\,1. P.~53--66.
\bibitem{10-va} %11
\Au{St$\ddot{\mbox{u}}$tzle T., Hoos H.} MAX-MIN Ant System and local search
for the traveling salesman problem~// IEEE International Conference on Evolutionary
Computation.~--- Indianapolis: IEEE, 1997. P.~309--314.

\bibitem{12-va}
\Au{Штовба С.\,Д.} Муравьиные алгоритмы~// Exponenta Pro. Математика в~приложениях, 2003. №\,4. С.~70--75.
\bibitem{13-va}
\Au{Курейчик В.\,М., Кажаров А.\,А.} О~некоторых модификациях
муравьиного алгоритма~// Известия Южного федерального университета.
Технические науки, 2008. Т.~81. №\,4. С.~7--12.
\bibitem{14-va}
\Au{Сластников С.\,А.} Применение метаэвристических алгоритмов для задачи
маршрутизации транспорта~// Экономика и~математические методы, 2014.
Т.~50. №\,1. С.~117--126.

 \end{thebibliography}

 }
 }

\end{multicols}

\vspace*{-9pt}

\hfill{\small\textit{Поступила в~редакцию 19.02.15}}

%\newpage

\vspace*{7pt}

\hrule

\vspace*{2pt}

\hrule

%\vspace*{12pt}

\def\tit{COMPARATIVE ANALYSIS OF APPLICATION OF~HEURISTIC AND~METAHEURISTIC ALGORITHMS TO~THE~SCHOOL BUS ROUTING PROBLEM}

\def\titkol{Comparative analysis of application of heuristic and metaheuristic algorithms to the school bus routing problem}

\def\aut{E.\,M.~Bronshtein and D.\,M.~Vagapova}

\def\autkol{E.\,M.~Bronshtein and D.\,M.~Vagapova}

\titel{\tit}{\aut}{\autkol}{\titkol}

\index{Bronshtein E.\,M.}
\index{Vagapova D.\,M.}

\vspace*{-12pt}

\noindent
Ufa State Aviation Technical University, 12 K.~Marx Str., Ufa 450000,
Russian Federation


\def\leftfootline{\small{\textbf{\thepage}
\hfill INFORMATIKA I EE PRIMENENIYA~--- INFORMATICS AND
APPLICATIONS\ \ \ 2015\ \ \ volume~9\ \ \ issue\ 2}
}%
 \def\rightfootline{\small{INFORMATIKA I EE PRIMENENIYA~---
INFORMATICS AND APPLICATIONS\ \ \ 2015\ \ \ volume~9\ \ \ issue\ 2
\hfill \textbf{\thepage}}}

\vspace*{2pt}


\Abste{This paper considers the school bus routing problem,
which is to ensure delivery of students after lessons from school to their stops.
The objective function is to minimize the maximum length of the routes.
A~short review of the literature on this theme is provided.
The problem definition and formalization is given. The
heuristic algorithm proposed by the authors earlier is described.
A~two-step algorithm based on ant colony metaheuristics is described.
The algorithm consists of initial clustering of stops at which students drop
off, and subsequent ant colony optimization with different parameters,
which is applied to each cluster. The results of comparing the efficiency of
the proposed algorithms and the performance of the program for two algorithms
are presented.}

\KWE{vehicle routing problem; school bus routing problem; ant colony optimization; clustering}


\DOI{10.14357/19922264150207}

\vspace*{-20pt}

\Ack

\vspace*{-2pt}
\noindent
The research was supported by the Russian Foundation for
Basic Research (project 13-01-00005).


\vspace*{-3pt}


  \begin{multicols}{2}

\renewcommand{\bibname}{\protect\rmfamily References}
%\renewcommand{\bibname}{\large\protect\rm References}



{\small\frenchspacing
 {%\baselineskip=10.8pt
 \addcontentsline{toc}{section}{References}
 \begin{thebibliography}{99}

 \vspace*{-2pt}

\bibitem{1-va-1}
\Aue{Newton, R.\,M., and W.\,H. Thomas}. 1969. Design of school Bus Routes by
computer. \textit{Socio-Economic Planning Science} 3:75--85.
\bibitem{2-va-1}
\Aue{Archetti, C.} Matheuristics for routing problems. Available at: {\sf
http://www.sintef.no/contentassets/cfb19ab9\linebreak b7c74d03904c7746ee1d8e77/matheuristics\_routing\_\linebreak verolog2014\_new.pdf} (accessed March~10, 2015).
\bibitem{3-va-1}
\Aue{Bronshtein, E.\,M., D.\,M. Vagapova, and A.\,V.~Nazmutdinova}. 2014.
O~postroenii semeystva marshrutov dostavki shkol'nikov za minimal'noe vremya
[About creating a set of delivery routes of students in a minimal time].
\textit{Avtomatika i~Telemekhanika} [Automation and Remote Control] 7:43--51.
\bibitem{4-va-1}
\Aue{Fisher, M.\,L., and R.\,A. Jaikumar}. 1981. Generalized assignment heuristic
for vehicle routing. \textit{Networks} 11(2):109--124.
\bibitem{5-va-1}
\Aue{Bowerman, R., B.~Hall, and P.\,A.~Calamai}. 1995. Multi-objective
optimization approach to urban school Bus Routing: Formulation and solution
method. \textit{Transportation Research Part~A: Policy and Practice}
29(2):107--123.
\bibitem{6-va-1}
\Aue{Arias-Rojas, J.\,S., J.\,F. Jimenez, and J.\,R.~Montoya-Torres}. 2012. Solving
of school bus routing problem by ant colony. \textit{Revista EIA} 9(17):193--208.
\bibitem{7-va-1}
\Aue{Dorigo, M.} 1992. \textit{Optimization, learning and natural algorithms}. PhD
Thesis. Milan, Italy: Politecnico di Milano.
\bibitem{11-va-1} %8
\Aue{Gambardella, L.\,M., and M. Dorigo}. 1995. Ant-Q: A~reinforcement learning
approach to the traveling salesman problem. \textit{12th Conference (International)
on Machine Learning}. Tahoe City. 252--260.
\bibitem{8-va-1} %9
\Aue{Dorigo, M., V.~Maniezzo, and A.~Colorni}. 1996. The Ant System:
Optimization by a colony of cooperating agents. \textit{IEEE Trans. Syst. Man
Cybernetics B} 26(1):29--41.
\bibitem{9-va-1} %10
\Aue{Dorigo, M., and L.\,M.~Gambardella}. 1997. Ant Colony System:
A~cooperative learning approach to the traveling salesman problem. \textit{IEEE
Trans. Evol. Comput.} 1(1):53--66.
\bibitem{10-va-1} %11
\Aue{St$\ddot{\mbox{u}}$tzle, T., and H.~Hoos}. 1997. MAX-MIN Ant System
and local search for the traveling salesman problem. \textit{IEEE Conference
(International) on Evolutionary Computation}. Indianapolis. 309--314.

\bibitem{12-va-1}
\Aue{Shtovba, S.\,D.} 2003. Murav'inye algoritmy [Ant colony algorithms].
\textit{Exponenta Pro. Matematika v~prilozheniyakh} [Exponenta-Pro. Mathematics
in Applications] (4):70--75.
\bibitem{13-va-1}
\Aue{Kureychik, V.\,M., and A.\,A.~Kazharov}. 2008. O~nekotorykh
modifikatsiyakh murav'inogo algoritma [About some modifications of the ant colony
algorithm]. \textit{Izvestiya Yuzhnogo Federal'nogo Universiteta. Tekhnicheskie
Nauki} [News of  South Federal University. Technical Sciences] 81(4):7--12.
\bibitem{14-va-1}
\Aue{Slastnikov, S.\,A.} 2014. Primenenie metaevristicheskikh algoritmov dlya
zadachi marshrutizatsii transporta [Application of metaheuristic algorithms for
vehicle routing problem]. \textit{Ekonomika i~Matematicheskie Metody} [Economics
and Mathematical Methods] 50(1):117--126.
\end{thebibliography}

 }
 }

\end{multicols}

\vspace*{-3pt}

\hfill{\small\textit{Received February 19, 2015}}

%\vspace*{-18pt}

\Contr

\noindent
\textbf{Bronshtein Efim M.} (b.\ 1946)~---
Doctor of Science in physics and mathematics; professor, Ufa State Aviation Technical University, 12 K.~Marx Str., Ufa 450000,
Russian Federation; bro-efim@yandex.ru

\vspace*{3pt}

\noindent
\textbf{Vagapova Diana M.} (b.\ 1987)~---
applicant, Ufa State Aviation Technical University, 12 K.~Marx Street, Ufa 450000, Russian Federation; vagapova-dm@mail.ru

\label{end\stat}


\renewcommand{\bibname}{\protect\rm Литература}
