\def\tl{\tilde\lambda}
\def\tB{\widetilde B}
\def\tb{\tilde b}

\def\stat{razum}

\def\tit{ВРЕМЯ ОЖИДАНИЯ В~СИСТЕМЕ ОБСЛУЖИВАНИЯ С~ИНВЕРСИОННЫМ ПОРЯДКОМ
ОБСЛУЖИВАНИЯ И~ОБОБЩЕННЫМ ВЕРОЯТНОСТНЫМ
ПРИОРИТЕТОМ$^*$}

\def\titkol{Время ожидания в~системе обслуживания с~инверсионным порядком
обслуживания} % и~обобщенным вероятностным приоритетом}

\def\aut{Л.\,А.~Мейханаджян$^1$, Т.\,А.~Милованова$^2$,  Р.\,В.~Разумчик$^3$}

\def\autkol{Л.\,А.~Мейханаджян, Т.\,А.~Милованова,  Р.\,В.~Разумчик}

\titel{\tit}{\aut}{\autkol}{\titkol}

\index{Мейханаджян Л.\,А.}
\index{Милованова Т.\,А.}
\index{Разумчик Р.\,В.}

{\renewcommand{\thefootnote}{\fnsymbol{footnote}} \footnotetext[1]
{Работа
выполнена при поддержке РФФИ (проект 13-07-00223).}}


\renewcommand{\thefootnote}{\arabic{footnote}}
\footnotetext[1]{Российский университет дружбы народов, lameykhanadzhyan@gmail.com}
\footnotetext[2]{Российский университет дружбы народов, tmilovanova77@mail.ru}
\footnotetext[3]{Институт проблем информатики Федерального исследовательского
центра <<Информатика и~управление>> Российской академии наук;
Российский
университет дружбы народов, rrazumchik@gmail.com}

\vspace*{-13pt}


\Abst{Рассматривается система массового
обслуживания (СМО) с~одним прибором и~бесконечным
числом мест для ожидания, в~которую поступает пуассоновский
поток заявок. В~системе реализован инверсионный порядок
обслуживания с~обобщенным вероятностным
приоритетом, заключающийся в~следующем.
Предполагается, что в~любой момент времени
известна остаточная длина каждой заявки в~системе.
В~момент поступления в~систему новой заявки ее
исходная длина сравнивается с~остаточной
длиной заявки на приборе, и~в~зависимости
от результатов сравнения либо обе заявки покидают систему,
либо только одна из них (вторая остается на приборе),
либо обе остаются в~системе (причем одна из них помещается на прибор).
Если заявка остается в~системе, она приобретает новую
(случайную) длину.
Предложены математические соотношения
для вычисления распределения времени пребывания заявки в~системе и~периода занятости (ПЗ) (в~терминах преобразования Лап\-ла\-са--Стилть\-еса (ПЛС)),
а~также некоторых временн$\acute{\mbox{ы}}$х характеристик.}

\KW{система массового обслуживания; специальные
дисциплины; инверсионный порядок
обслуживания; вероятностный приоритет}

\DOI{10.14357/19922264150202}

\vspace*{-6pt}


\vskip 12pt plus 9pt minus 6pt

\thispagestyle{headings}

\begin{multicols}{2}

\label{st\stat}

\section{Введение}

Данная работа, являющаяся продолжением~\cite{lataavrv}, посвящена нахождению математических соотношений для стационарных временн$\acute{\mbox{ы}}$х характеристики заявок, поступающих в~одноканальную
СМО неограниченной емкости
с~той же, что и~в~\cite{lataavrv}, дисциплиной~---
инверсионным порядком обслуживания с~обобщенным
вероятностным приоритетом (Last Come First Served with Generalized Probabilistic
Priority, {LCFS GPP}).
Исследования по системам со сложными, специальными дисциплинами обслуживания
начались в~середине XX~в., и~с~тех пор накопился значительный
объем научного материала. Некоторые из результатов, близких к теме
настоящего исследования, можно найти
в~[2--8].
Среди недавних работ по специальным дисциплинам обслуживания
стоит отметить~\cite{jawbz, jawbz1}.

Поводом для написания настоящей статьи стала работа~\cite{lataavrv},
в~которой исследовались стационарные характеристики
в~одноканальной системе обслуживания с~дисциплиной {LCFS GPP},
связанные с~чис\-лом заявок в~системе. Однако,
как известно, если с~точки зрения вероятностных характеристик (например,
средней длины очереди)
выбор дисциплины обслуживания может и~не играть роли, то
с~точки зрения временн$\acute{\mbox{ы}}$х характеристик специфика дисциплины обслуживания
очень важна. Действительно, в~случае дисциплины {LCFS GPP}
поступающая заявка может как мгновенно покинуть систему (выбивая
при этом заявку, обслуживающуюся на приборе), так и~вытеснить заявку с~прибора
или занять первое место в~очереди. При этом каждый раз
заявка, занимая прибор, может покинуть его недообслуженной и/или возвращаться на
него конечное (но сколь угодно большое) число раз. Необходимо
также добавить, что при наступлении каждого события, связанного
с~приходом очередной заявки, длины обслуживающейся заявки (если таковая
имеется) и~поступающей заявки могут меняться в~соответствии
с~заданными законами распределения.

Данное описание не исчерпывает всех возможных событий,
которые могут произойти с~заявкой как только она
поступила в~систему, однако показывает, что важно уметь рассчитывать
и~временн$\acute{\mbox{ы}}$е характеристики системы.

Основной результат работы состоит в~нахождении
стационарного распределения времени пребывания заявки в~системе
из~\cite{lataavrv} в~терминах ПЛС,
а~также нахождения ПЛС ПЗ.
Нахождение моментов
сводится к~задаче нахождения приближенного или точного решения
неоднородных интегральных уравнений (в~некоторых случаях~---
уравнений Фредгольма 2-го рода).

Статья организована следующим образом. Во втором разделе дается
описание рассматриваемой системы. В~разд.~3 приводятся формулы для
ПЗ (в~терминах ПЛС). В~разд.~4 и~5 внимание
уделяется нахождению стационарного распределения времени пребывания
заявки на приборе и~в~системе соответственно. Выражения для
некоторых временн$\acute{\mbox{ы}}$х характеристик приводятся в~разд.~6.
В~заключении работы кратко резюмируются основные результаты и~даются
направления дальнейших исследо-\linebreak ваний.

\vspace*{-6pt}

\section{Описание системы}

Рассмотрим СМО с~входящим потоком заявок,
который для простоты будем называть здесь
потоком пуассоновского типа. Отличие этого потока от пуассоновского
заключается в~следующем: интенсивность поступления заявки равна~$\la$,
если на приборе имеется заявка, и~$\tl$, если система пуста.

Если в~момент поступления заявки в~систему
на приборе имеется заявка, то исходное
распределение времени обслуживания поступающей
заявки является произвольным с~функцией распределения (ФР) $B(x)$.
Если же заявка поступает в~систему в~тот
момент, когда система пуста, то исходное
распределение времени обслуживания поступающей
заявки является произвольным с~ФР $\tB(x)$.

Далее для простоты изложения будем считать,
что ФР $B(x)$ и~$\tB(x)$ имеют непрерывные
ограниченные плотности распределения~$b(x)\hm=B'(x)$ и~$\tb(x)\hm=\tB'(x)$.

Обобщенный инверсионный порядок обслуживания
с~обобщенным вероятностным приоритетом заключается в~следующем.
Предполагается, что в~любой момент времени
известна остаточная длина (далее будем говорить
просто длина) каждой заявки в~системе.
В~момент поступления в~систему новой заявки ее
исходная длина~$u$ сравнивается с~(остаточной)
длиной~$v$ заявки на приборе.
С~вероятностью~$D(x,y|u,v)$, зависящей только от~$u$ и~$v$, обслуживавшаяся
ранее заявка продолжает обслуживаться, причем
ее длина становится меньше~$y$, а~вновь
поступившая становится на первое место в~очереди
и~ее длина становится меньше~$x$.
Кроме того, с~вероятностью~$D^*(x,y|u,v)$,
зависящей только от~$u$ и~$v$, вновь поступившая
заявка занимает прибор, вытесняя обслуживавшуюся
ранее на первое место в~очереди, причем длина
заявки, бывшей ранее на приборе, становится
меньше~$y$, а~вновь поступившей~--- меньше~$x$.

Если на приборе находится заявка остаточной
длины~$v$ и~в~систему поступает заявка
длины~$u$, то с~вероятностью $D_0(x|u,v)$
заявка, находящаяся на приборе, покидает
систему, а~поступившая заявка становится на
прибор, причем ее длина становится меньше~$x$.
Кроме того, с~вероятностью
$D_0^*(y|u,v)$ поступившая заявка сразу же
покидает систему, а~заявка, находящаяся на
приборе, продолжает обслуживаться, причем ее
длина становится меньше~$y$.
Введем также обозначение:
\begin{equation*}
%\label{(2.1)}
D(x|u,v) = D_0(x|u,v) + D_0^*(x|u,v)\,.
\end{equation*}
Здесь $D(x|u,v)$~--- вероятность того, что одна
из двух заявок покинет систему, а~вторая встанет
на прибор и~примет длину меньше $x$.


Наконец, предполагается, что с~вероят\-ностью~$d_0(u,v)$ обе заявки покидают
систему, а~на прибор становится первая заявка
из очереди.

Будем считать для удобства изложения, что все
ФР $D(x,y|u,v)$, $D^*(x,y|u,v)$, $D_0(x|u,v)$,
$D_0^*(y|u,v)$, $D(y|u,v)$ и~$D_0(u,v)$
имеют непрерывные ограниченные плотности:
\begin{gather*}
d(x,y|u,v)=\fr{\partial^2 D(x,y|u,v)}{\partial x\, \partial y}\,;\\
d^*(x,y|u,v)=\fr{\partial^2 D^*(x,y|u,v)}{\partial x\, \partial y}\,;\\
d_0(x|u,v)= \fr{\partial D_0(x|u,v)}{\partial x}\,;\\
d_0^*(y|u,v)=\fr{\partial D_0^*(y|u,v)}{\partial y}\,;\
d(x|u,v)=\fr{\partial D(x|u,v)}{\partial x}\,.
\end{gather*}


Естественно, для любых~$u$ и~$v$ выполнено условие:

\vspace*{-2pt}

\noindent
\begin{multline*}
%\label{(2.1)}
\int\limits_0^\infty \int\limits_0^\infty
\left[d(x,y|u,v) + d^*(x,y|u,v)\right] \,dx\,dy
+{}\\[-6pt]
{}+ \int\limits_0^\infty d(x|u,v) \,dx +
d_0(u,v) ={}\\
{}=
D(\infty,\infty|u,v)+D^*(\infty,\infty|u,v)
+{}\\
{}+ D(\infty|u,v) + d_0(u,v) = 1\,.
   %   \eqno(2.1)
\end{multline*}



Если длина заявки на приборе становится
равной нулю, то она мгновенно покидает систему и~на прибор переходит первая
заявка из очереди. Остальная очередь сдвигается на единицу.


Далее будем предполагать, что система функционирует в~стационарном режиме.
Достаточное условие его существования приведено в~\cite{lataavrv}, где также были найдены аналитические выражения для $p_n(x)$~--- стационарных плотностей вероятностей того, что в~рассматриваемой системе находится~$n$~заявок и~остаточная длина заявки, находящейся на приборе, меньше~$x$.
Далее будем считать, что значения $p_n(x)$ известны.

\section{Период занятости}

Обозначим через $u(s,x)$ ПЛС длительности ПЗ,
открываемого заявкой длины~$x$.
Для нахождения $u(s,x)$ воспользуемся свойствами ПЛС и~рас\-смот\-рим все возможные
события, которые могут произойти после поступления заявки, открывающей ПЗ.
Так, ПЛС длительности ПЗ  $u(s,x)$ равно:
\begin{itemize}
\item  $e^{-s x}$,
если до момента времени~$x$ окончания обслуживания заявки на приборе
новая заявка не поступила (с~вероятностью $e^{-\la x}$);

\item $e^{-s t}$, если в~момент $0\hm<t\hm<x$
поступила новая заявка и~обе заявки покинули
систему (с~плот\-ностью вероятностей
$\la e^{-\la t} \int\limits_0^\infty d_0(y,x\hm-t)\,b(y)\,dy$);

\item $e^{-s t} u(s,v)$, если в~момент времени
$0\hm<t\hm<x$ поступила новая заявка длины~$y$, одна из двух
заявок (поступившая заявка или
заявка на приборе) покинула систему, а~оставшаяся приняла длину~$v$ и~открыла
новый ПЗ системы, ПЛС длительности которого равно $u(s,v)$
(плот\-ность вероятности данного события равна
$ \la e^{-\la t} \int\limits_0^\infty d_0(v|y,x\hm-t)\, b(y)\, dy$);

\item  $e^{-s t} u(s,v) u(s,w)$, если в~момент
времени $0\hm<t\hm<x$ поступила новая заявка длины~$y$,
обе заявки остаются в~системе, причем длина одной из двух заявок (поступившей заявки или заявки на приборе) становится равной~$v$,
а~второй~--- $w$ (с~плот\-ностью вероятностей
$\la e^{-\la t}\times$\linebreak $ \times\int\limits_0^\infty \left[d(v,w|y,x\hm-t)\hm+d^*(v,w|y,x\hm-t)\right] b(y)\, dy $).
\end{itemize}

Теперь по формуле полной вероятности получаем, что
уравнение для нахождения ПЛС $u(s,x)$ имеет вид:
\begin{multline}
u(s,x) = a_1(s,x) + \int\limits_0^\infty u(s,v)\, a_2(s,x,v)\,dv
+ {}\\
{}+\int\limits_0^\infty u(s,v)\, dv \int\limits_0^\infty u(s,w)\, a_3(s,x,v,w)\,dw
\,,
\label{e1-m}
\end{multline}
где
\begin{multline*}
a_1(s,x) = e^{-(s+\la)x}+ {}\\
{}+\int\limits_0^x \la e^{-(s+\la) t}\, dt
\int\limits_0^\infty d_0(y,x-t)\,b(y)\, dy\,;
\end{multline*}

\vspace*{-9pt}

\noindent
\begin{equation*}
a_2(s,x,v) = \int\limits_0^x \la e^{-(s+\la) t} \,dt
\int\limits_0^\infty d_0(v|y,x-t)\, b(y)\, dy\,;
\end{equation*}

\vspace*{-12pt}

\noindent
\begin{multline*}
\hspace*{-3.0176pt}a_3(s,x,v,w) = \int\limits_0^x \la e^{-(\la+s) t}\, dt
\int\limits_0^\infty \left[\vphantom{d^*}
d(v,w|y,x-t)+{}\right.\\
\left.{}+d^*(v,w|y,x-t)\right]\, b(y)\, dy\,.
\end{multline*}

Уравнение~(\ref{e1-m}) представляет собой интегральное уравнение, решение которого, за исключением простейших частных случаев, не удается получить в~явном виде. Для его численного решения,
вообще говоря, можно пользоваться известными численными методами (см., например,~[11--13]).

\section{Стационарное распределение времени пребывания заявки на~приборе}

Обратимся к~задаче нахождения времени пребывания заявки на приборе.
Поскольку вновь поступающие заявки могут вытеснять заявку с~прибора (на первое место в~очереди и~вообще из системы), то время пребывания заявки на приборе можно считать по-разному, а~именно: с~учетом прерываний обслуживания и~без.

\subsection{Время пребывания заявки на~приборе без~учета времени прерываний
обслуживания}

Обозначим через $\varphi(s,x)$ ПЛС времени пребывания заявки длины~$x$, поступившей на прибор без учета времени прерываний ее обслуживания.
Для нахождения $\varphi(s,x)$, так же как и~в~случае ПЛС ПЗ, воспользуемся
свойствами ПЛС и~рассмотрим все возможные события, которые могут произойти с~поступившей на прибор заявкой.
Заявка будет пребывать на приборе время (в~терминах ПЛС), равное:
\begin{itemize}
\item  $e^{-s x}$, если до момента времени~$x$
окончания обслуживания заявки на приборе новая заявка не поступила (с~вероятностью $e^{-\la x}$);

\item $e^{-s t}$, если в~момент $0\hm<t\hm<x$
поступила новая заявка и~обе заявки (новая заявка и~заявка на приборе) покинули
систему (с~плот\-ностью вероятностей $\la e^{-\la t}\int\limits_0^\infty d_0(y,x\hm-t)\,b(y)\,dy$);

\item  $e^{-s t}$, если в~момент времени
$0\hm<t\hm<x$ поступила новая заявка длины~$y$, сама встала на прибор, а~заявка с~прибора покинула систему (с~плот\-ностью вероятностей
$\la e^{-\la t} \int\limits_0^\infty d_0(v|y,x\hm-t)\, b(y)\, dy
$);

\item  $e^{-s t}\varphi(s,w)$, если в~момент времени
$0\hm<t\hm<x$ поступила новая заявка длины~$y$, изменила длину заявки на приборе на~$w$, а~сама покинула систему (с~плот\-ностью вероятностей
$\la e^{-\la t} \int\limits_0^\infty d^*_0(w|y,x\hm-t)\, b(y)\, dy
$);

\item  $e^{-s t}\varphi(s,w)$, если в~момент
времени $0\hm<t\hm<x$ поступила новая заявка длины~$y$, обе заявки (новая
заявка и~заявка на приборе) остались в~системе, новая заявка
получила длину $v$, заявка, бывшая на приборе,~--- длину~$w$ (с~плот\-ностью вероятностей $\la e^{-\la t} \times$ $\times\int\limits_0^\infty
\left[d(v,w|y,x-t)+d^*(v,w|y,x\hm-t)\right] b(y)\, dy $).
\end{itemize}

Воспользовавшись формулой полной вероят\-ности, получаем,
что ПЛС $\varphi(s,x)$ есть решение следующего интегрального уравнения:
\begin{equation}
\label{eq2}
\varphi(s,x)=
b_1(s,x)+\int\limits_0^\infty \varphi(s,w)b_2(s,x,w) \, dw\,,
\end{equation}
где
\begin{multline*}
b_1(s,x) = e^{-(\la+s) x}+{}\\
{}+\int\limits_0^x \la e^{-(\la+s) t}\,dt
\int\limits_0^\infty d_0(y,x-t)\,b(y)\,dy
+{}\\
{} +
\int\limits_0^x \la e^{-(\la+s) t}\,dt \int\limits_0^\infty \, dv
\int\limits_0^\infty d_0(v|y,x-t)\, b(y)\, dy\,;
\end{multline*}

\vspace*{-12pt}

\noindent
\begin{multline*}
\hspace*{-5.58406pt}b_2(s,x,w) = \int\limits_0^x \la e^{-(\la+s) t}\,dt
\int\limits_0^\infty d^*_0(w|y,x-t)\, b(y)\, dy +{}
\\
{}+ \int\limits_0^x \la e^{-(\la+s) t}\,dt
\int\limits_0^\infty \, dv
\int\limits_0^\infty \left[ \vphantom{d^*}
d(v,w|y,x-t)+{}\right.\\
\left.{}+d^*(v,w|y,x-t)\right] b(y)\, dy\,.
\end{multline*}

Уравнение~\eqref{eq2} есть уравнение Фредгольма 2-го рода. Свободный член $b_1(s,x)$ и~ядро $b_2(s,x,w)$ интегрального уравнения являются неотрицательными функциями.
Численные методы решения интегральных уравнений Фредгольма 2-го
рода хорошо известны (см., например,~[11--13]).
В~данном случае, в~принципе, можно воспользоваться итерационным методом, взяв
в~качестве начальной итерации нулевое приближение. Тогда итерации
будут возрастающими. Однако практическая реализация метода является
затруднительной ввиду того, что выражение для ядра уравнения является
весьма сложным.


\subsection{Время пребывания заявки на приборе с~учетом времени прерываний
обслуживания}

Обозначим через $\psi(s,x)$ ПЛС времени пребывания заявки длины~$x$, поступившей на прибор, с~учетом времени прерываний ее обслуживания.
Действуя, как и~ранее, приходим к тому, что время пребывания (в терминах ПЛС) равно:
\begin{itemize}
\item  $e^{-s x}$, если до момента времени~$x$
окончания обслуживания заявки на приборе
новая заявка не поступила (с~вероятностью $e^{-\la x}$);

\item  $e^{-s t}$, если в~момент $0\hm<t\hm<x$
поступила новая заявка и~она вместе с~заявкой на приборе покинула
систему (с~плот\-ностью вероятностей
$\la e^{-\la t} \int\limits_0^\infty d_0(y,x-t)\,b(y)\,dy
$);

\item  $e^{-s t}$, если в~момент времени
$0\hm<t\hm<x$ поступила новая заявка длины~$y$, сама встала на прибор, а~заявка с~прибора покинула систему (с~плот\-ностью вероятностей
$\la e^{-\la t} \int\limits_0^\infty d_0(v|y,x\hm-t)\, b(y)\, dy
$);

\item  $e^{-s t}\psi(s,w)$, если в~момент времени
$0\hm<t\hm<x$ поступила новая заявка длины~$y$,
изменила длину заявки на приборе на~$w$, а~сама покинула систему (с плотностью вероятностей $\la e^{-\la t} \int\limits_0^\infty d^*_0(w|y,x-t)\, b(y)\, dy
$);

\item  $e^{-s t}\psi(s,w)$, если в~момент
времени $0\hm<t\hm<x$ поступила новая заявка длины~$y$, которая встала на первое место в~очереди, причем новая заявка получила новую длину~$v$, а~заявка на приборе~--- новую длину~$w$ (с~плот\-ностью вероятностей $\la e^{-\la t}
\int\limits_0^\infty d(v,w|y,x-t)\, b(y)\, dy
$);

\item  $e^{-s t}u(s,v)\psi(s,w)$, если в~момент
времени $0\hm<t\hm<x$ поступила новая заявка длины~$y$, которая встала на прибор, получив новую длину~$v$, а~заявка с~прибора вытеснена на первое место в~очереди и~получила новую длину~$w$ (с~плот\-ностью вероятностей
$\la e^{-\la t} \int\limits_0^\infty d^*(v,w|y,x-t)\, b(y)\, dy
$).
\end{itemize}

Воспользовавшись снова формулой полной вероятности, получаем, что
уравнение для определения ПЛС $\psi(s,x)$ имеет следующий вид:
\begin{equation}
\label{eq3}
\psi(s,x)= b_1(s,x)+\int\limits_0^\infty\psi(s,w) b_3(s,x,w) \, dw,
\end{equation}
где
\begin{multline*}
b_3(s,x,w) = {}\\
{}=\int\limits_0^x \la e^{-(\la+s) t}\,dt
\int\limits_0^\infty d^*_0(w|y,x-t)\, b(y)\, dy
+{}\\
{}+
\int\limits_0^x \!\la e^{-(\la+s) t}\,dt
\int\limits_0^\infty \, dv
\int\limits_0^\infty d(v,w|y,x-t)\, b(y)\, dy
+{}\\
{}+
\int\limits_0^x\! \la e^{-(\la+s) t}\,dt\!\!
\int\limits_0^\infty\! \!u(s,v)\, dv\!
\int\limits_0^\infty\!\! d^*(v,w|y,x-t) b(y)\, dy.
\end{multline*}

Как видно, вывод уравнения~\eqref{eq3} для $\psi(s,x)$ практически полностью повторяет вывод уравнения~\eqref{eq2} для $\varphi(s,x)$. Единственное отличие состоит в~том, что при расчете $\psi(s,x)$
необходимо учитывать то время, которое проходит с~момента, когда заявка вытесняется с~прибора на первое место в~очереди, и~до момента, когда заявка снова занимает прибор. Это время в~точности совпадает с~длиной ПЗ, открываемого вновь поступающей заявкой, которая прерывает обслуживание заявки на приборе.
Уравнение~\eqref{eq3} также является уравнением Фредгольма 2-го рода, и~к нему применимы те же замечания, что и~к уравнению~\eqref{eq2}.

\section{Стационарное распределение времени пребывания заявки в~системе}


Найдем в~терминах ПЛС $\chi(s,x)$ стационарное распределение времени пребывания в~системе поступающей заявки длины~$x$.

Во-первых, заявка длины $x$ может с~вероят\-ностью $p_0$ поступить в~свободную систему и~тогда время ее пребывания будет совпадать с~полным временем ее обслуживания.

Во-вторых, с~плот\-ностью вероятностей $\sum\nolimits_{n=1}^\infty p_n(y)$ она может застать систему занятой, причем на приборе будет находиться заявка длины~$y$.
В~этом случае возможны следующие вари\-анты:
\begin{itemize}
 \item либо с~вероятностью $d_0(x,y)\hm+\int\limits_0^\infty d_0^*(w|x,y)\,dw$ поступившая заявка покинет систему и~при этом ее время пребывания в~сис\-те\-ме будет равно нулю;

\item  либо с~вероятностью $d_0(v|x,y)\hm+d^*(v,w|x,y)$ она сама встанет на прибор, причем ее длина станет равной~$v$ и~тогда время ее пребывания в~сис\-те\-ме будет совпадать с~полным временем ее обслуживания с~учетом прерываний;

\item  либо с~вероятностью $d(v,w|x,y)$ она станет на первое место в~очередь, получит новую длину~$v$, а~заявка на приборе~--- новую длину~$w$; при этом время пребывания в системе поступившей заявки будет равно сумме двух времен: длины ПЗ, открываемого заявкой длины~$w$, и~полного времени обслуживания заявки длины~$v$ с~учетом прерываний.
    \end{itemize}

Применяя формулу полной вероятности, приходим к следующему выражению для $\chi(s,x)$:
\begin{multline*}
%\label{eq4}
\chi(s,x)= p_0\psi(s,x)+{}\\
{}+
\int\limits_0^\infty\sum\limits_{n=1}^\infty p_n(y) \left [
d_0(x,y)+\int\limits_0^\infty d_0^*(w|x,y)\,dw
+{}\right.
\\
{}+
\int\limits_0^\infty\! \psi(s,v)
\left ( d_0(v|x,y)+\int\limits_0^\infty\! d^*(v,w|x,y)\,dw \right) \,dv
+{}
\\
{}+
\left.\int\limits_0^\infty\!
\int\limits_0^\infty\! \psi(s,v)u(s,w)d(v,w|x,y) \,dv  \, dw
\right] \,dy.
\end{multline*}

Наконец, ПЛС $\chi(s)$ стационарного распределения времени пребывания в~системе заявки произвольной длины получается усреднением по распределению длины заявки $B(x)$ и~равно
$$
\chi(s)=\int\limits_0^\infty\chi(s,x)b(x)\,dx\,.
$$

Для определения моментов времени пребывания заявки в~системе достаточно воспользоваться свойством ПЛС. Расчетные формулы ввиду громоздкости здесь не приводятся.

\section{Некоторые временн$\acute{\mbox{ы}}$е характеристики}

Покажем, как, используя  результаты,
полученные в~предыду\-щих разделах,
можно найти выражения для некоторых вероятностных характеристик
поступающих в~систему заявок.
Ограничимся следующими характеристиками:
\begin{itemize}
\item вероятность $\pi(x)$ того, что заявка будет потеряна при поступлении при условии, что ее исходная длина равнялась~$x$;
\item вероятность ${\widehat{q}_i(x)}$, $i\hm\ge 0$, того, что заявка не будет потеряна при поступлении, не будет
обслужена до конца и~за время пребывания в~сис\-те\-ме сменит длину~$i$~раз при условии, что ее исходная длина равнялась~$x$;
\item вероятность $q_i^*(x)$, $i\hm\ge 0$, того, что заявка не будет потеряна при поступлении, будет обслужена до конца и~за время пребывания в~системе сменит длину~$i$~раз при условии, что ее исходная длина равнялась~$x$;
\item среднее время пребывания в~системе заявки в~расчете на единицу времени обслуживания.
\end{itemize}

Очевидно, что заявка, поступившая в~систему в~момент, когда на приборе
оканчивается обслуживание или нет заявок, не будет потеряна. Тогда из самого определения введенной дисциплины обслуживания следует, что
$$
\pi(x)= \int\limits_0^\infty \! \sum\limits_{n=1}^\infty p_n(y)
\left(\!d_0(x,y)+\!\int\limits_0^\infty\! d_0^*(w|x,y)\, dw\!\right)dy.
$$


Для нахождения вероятностей ${\widehat{q}_i(x)}$, $i\hm\ge 0$,
заметим, что заявка, заставшая в~момент поступления прибор свободным,
сразу начинает обслуживаться и~может покинуть систему не обслужившись,
ни разу не сменив свою исходную длину. Обозначим через
${q}_i(x)$ вероятность того, что заявка не будет обслужена до конца,
сменит длину $i$, $i\ge 0$, раз при условии, что ее исходная длина равнялась~$x$, а~прибор в~момент ее поступления был свободен.
Можно показать, что для ${q}_i(x)$  имеют место следующие рекуррентные соотношения:
\begin{multline*}
{q}_0(x)= \int\limits_0^x \la e^{-\la t}\,dt
\int\limits_0^\infty d_0(y,x-t) b(y)\, dy
+{}\\
{}+
\int\limits_0^x \la e^{-\la t}\,dt
\int\limits_0^\infty
dv \int\limits_0^\infty d_0(v|y,x-t) b(y)\, dy\,;
\end{multline*}

\vspace*{-12pt}

\noindent
\begin{multline*}
{q}_i(x)= \int\limits_0^x \la e^{-\la t}\,dt
\int\limits_0^\infty {q}_{i-1}(w) \, dw\times{}\\
{}\times
\int\limits_0^\infty d^*_0(w|y,x-t)\, b(y)\, dy
+
\int\limits_0^x \la e^{-\la t}\,dt \int\limits_0^\infty \, dv\times{}\\
{}\times
\int\limits_0^\infty {q}_{i-1}(w) \, dw
\int\limits_0^\infty [d(v,w|y,x-t)+{}\\
{}+d^*(v,w|y,x-t)]\, b(y)\, dy\,.
\end{multline*}
Напротив, если поступающая
заявка застанет прибор занятым и~останется в~системе,
то она обязательно хотя бы один раз сменит свою длину\linebreak
 (ввиду предположения
о~непрерывности функций $D(x,y|u,v)$, $D^*(x,y|u,v)$, $D_0(x|u,v)$,
$D_0^*(y|u,v)$ и~$D(y|u,v)$). Воспользовавшись теперь формулой полной вероятности,
получим следующую формулу для расчета ${\widehat{q}_i(x)}$:
\begin{align*}
\widehat{q}_0(x) &= {q}_0(x)p_0\,;
\\
{\widehat{q}_i(x)}&= {q}_i(x)p_0+{}\\
&\hspace*{3mm}{}+
\int\limits_0^\infty \sum\limits_{n=1}^\infty p_n(y)
\left [ \int\limits_0^\infty {q}_{i-1}(v)d_0(v|x,y)\,dv
+{}\right.\\
&\hspace*{10mm}{}+
\int\limits_0^\infty {q}_{i-1}(v)\,dv
\int\limits_0^\infty [d(v,w|x,y)+{}\\
&\hspace*{18mm}\left.{}+d^*(v,w|x,y)]\,dw
\vphantom{\int\limits_0^\infty {q}_{i-1}}
\right]\,dy\,, \ i\ge 1\,.
\end{align*}

Аналогичные рассуждения используются для нахождения вероятностей
$q_i^*(x)$. Обозначим через~$\widetilde{q}_i(x)$ вероятность того,
что заявка будет обслужена до конца, сменит длину~$i$, $i\hm\ge0$, раз
при условии, что ее исходная длина равнялась~$x$, а~в~момент ее
поступления прибор был свободен. Рекуррентные соотношения, которым
удовлетворяют $\widetilde{q}_i(x)$, имеют вид:
\begin{align*}
\widetilde{q}_0(x)&=e^{-\la x}\,,
\\
\widetilde{q}_i(x)&= \int\limits_0^x \la e^{-\la t}\,dt
\int\limits_0^\infty \widetilde{q}_{i-1}(w) \, dw\times{}\\
&{}\times
\int\limits_0^\infty d^*_0(w|y,x-t) b(y)\, dy
\int\limits_0^x \la e^{-\la t}\,dt
\int\limits_0^\infty \, dv\times{}\\
&{}\times
\int\limits_0^\infty \widetilde{q}_{i-1}(w) \, dw
\int\limits_0^\infty \left[ \vphantom{d^*}
d(v,w|y,x-t)+{}\right.\\
&\hspace*{10mm}\left.{}+d^*(v,w|y,x-t)\right]\, b(y)\, dy
\,, \ i \ge 1\,.
\end{align*}
Учитывая, что заявка,
заставшая в~момент поступления прибор занятым (и~оставшаяся в~системе),
обязательно хотя бы один раз сменит свою длину
до того момента, когда будет обслужена до конца,
по формуле полной вероятности получаем следующую рекуррентную формулу для расчета
$q^*_i(x)$, $i\hm\ge0$:
\begin{align*}
{q_0^*(x)}&=
\widetilde{q}_0(x) p_0\,;
\\
{q_i^*(x)}&= \widetilde{q}_i(x) p_0+{}\\
&{}+
\int\limits_0^\infty
\sum\limits_{n=1}^\infty p_n(y)
\left [
\int\limits_0^\infty \widetilde{q}_{i-1}(v)d_0(v|x,y)\,dv
+{}\right.
\\
&\hspace*{-12mm}\left. {}+\!
\int\limits_0^\infty \!\widetilde{q}_{i-1}(v)\,dv
\!\int\limits_0^\infty \!\left[d(v,w|x,y)+d^*(v,w|x,y)\right]\,dw\!
\right]dy.
\end{align*}

Если число раз, которое заявка сменила свою длину
прежде, чем покинуть систему обслуженной (недообслуженной),
не имеет значения, то для нахождения соответствующей вероятности достаточно
просуммировать $q_i^*(x)$ (${\widehat{q}_i(x)}$) по всем значениям~$i$.
Заметим, что данные вероятности можно найти и~другим способом, выписав для них по отдельному уравнению и~решив их.

Обозначим через $W_x$ время пребывания в~системе заявки, которая при
поступлении в~систему имеет длину~$x$. Тогда средний штраф за
предоставление поступающей заявке~$x$ единиц времени обслуживания
определяется как $E(W_x)/x$\footnote{Отношение $W_x/x$ показывает, во
сколько раз время пребывания заявки в~системе отличается от ее
исходной длины. В~предположении, что более длинные заявки должны
находиться в~системе дольше, чем короткие, моменты случайной
величины $W_x/x$ могут использоваться для оценки справедливости
дисциплины обслуживания~\cite{klein, avi}.}. Используя
результаты разд.~5, нетрудно выписать соответствующую расчетную
формулу
$$
\fr{E(W_x)}{x} =- \fr{1}{x}\, \chi'(0,x)\,,
$$
откуда безусловное значения штрафа получается
усреднением по распределению длины заявки~$B(x)$.

\section{Заключение}

В настоящей работе предложен метод нахождения (в терминах ПЛС)
ПЗ и~стационарного распределения времени пребывания заявки в~системе $M/G/1$ с~инверсионным порядком обслуживания и~обобщенным
вероятностным приоритетом. Укажем на проблему расчета временн$\acute{\mbox{ы}}$х характеристик по полученным соотношениям, что
является предметом дальнейших исследований.
Во всех случаях неизвестные величины (например, моменты)
являются решениями неоднородных интегральных уравнений.
Как показывают численные эксперименты,
качество получаемых решений сильно зависит от вида
функций $d(u,v)$, $d^*(x|u,v)$, $d(x|u,v)$, $d(x,y|u,v)$ и~$d^*(x,y|u,v)$
и~применение традиционных методов решения (метода
конечных сумм, метода конечных приближений и~др.)\
за\-час\-тую не дает удовлетворительных результатов.
Чис\-лен\-ные методы Мон\-те Кар\-ло позволяют находить приближенные решения (как
показывают сравнения с~результатами имитационного моделирования)
за конечное время, однако ограничения, присущие этим методам,
не позволяют использовать их для произвольных функций $d(u,v)$, $d^*(x|u,v)$,
$d(x|u,v)$, $d(x,y|u,v)$ и~$d^*(x,y|u,v)$.

{\small\frenchspacing
 {%\baselineskip=10.8pt
 \addcontentsline{toc}{section}{References}
 \begin{thebibliography}{99}

\bibitem{lataavrv} %1
\Au{Мейханаджян Л.\,А., Милованова~Т.\,А., Печинкин~А.\,В., Разумчик~Р.\,В.}
Стационарные вероятности состояний в~системе обслуживания с~инверсионным порядком обслуживания и~обобщенным вероятностным приоритетом~// Информатика и~её применения, 2014. Т.~8. Вып.~3. С.~16--26.

\bibitem{shrage} %2
\Au{Schrage L.} A~proof of the optimality of the shortest remaining processing
time discipline~// Oper.\ Res., 1968. Vol.~16. P.~687--690.



\bibitem{aaa1} %3
\Au{Нагоненко В.\,А.}
О~характеристиках одной нестандартной системы
массового обслуживания.~I; II~// Изв.\ АН СССР. Технич.\ кибернет., 1981.
№\,1. С.~187--195; №\,3. С.~91--99.

\bibitem{aaa3} %4
\Au{Нагоненко В.\,А., Печинкин А.\,В.}
О большой загрузке в~системе с~инверсионным
обслуживанием и~вероятностным приоритетом~//
Изв.\ АН СССР. Технич.\ кибернет., 1982. №\,1. С.~86--94.

\bibitem{aaa2} %5
\Au{Печинкин А.\,В.} Об одной
инвариантной системе массового обслуживания~//
Math.\ Operationsforsch.\ und Statist. Ser.\ Optimization, 1983. Vol.~14. №\,3. P.~433--444.



\bibitem{aaa4} %6
\Au{Нагоненко В.\,А., Печинкин А.\,В.}
О малой загрузке в~сис\-те\-ме с~инверсионным порядком
обслуживания и~вероятностным приоритетом~//
Изв.\ АН СССР. Технич.\ кибернет., 1984. №\,6. С.~82--89.

\bibitem{ppav} %7
\Au{Бочаров  П.\,П., Печинкин А.\,В.}
Теория массового обслуживания.~--- М.: РУДН, 1995. 529~с.

\bibitem{av1} %8
\Au{Печинкин А.\,В., Стальченко И.\,В.}
Система MAP$/G/1/\infty$ с~инверсионным порядком
обслуживания и~вероятностным приоритетом,
функционирующая в~дискретном времени~//
Вестник Российского ун-та дружбы народов.
Сер.\ Математика. Информатика. Физика, 2010. №\,2. С.~26--36.

\bibitem{jawbz} %9
\Au{Nair J., Wierman A., Zwart~B.} Tail-robust scheduling via limited processor sharing~// Perform. Evaluation, 2010. Vol.~67. No.\,11. P.~978--995.

\bibitem{jawbz1} %10
\Au{Wierman A., Zwart B.} Is tail-optimal scheduling possible?~//
Oper.\ Res., 2012. Vol.~60. No.\,5. P.~1249--1257.

\bibitem{jerri} %11
\Au{Jerri A.}
Introduction to integral equations with applications.~---
New York, NY, USA: John Wiley \& Sons, 1999. 272~p.

\bibitem{wh} %12
\Au{Press W.\,H., Teukolsky~S.\,A., Vetterling~W.\,T., Flannery~B.\,P.}
Numerical recipes.~--- 3rd ed.~--- The Art of Scientific Computing, 2007.
1235~p.


\bibitem{adav} %13
\Au{Polyanin, A.\,D., Manzhirov A.\,V.}
Handbook of integral equations.
 Boca Raton\,--\,London: Chapman\,\&\,Hall\,/ CRC Press, 2008. 1108~p.


\bibitem{klein} %14
\Au{Kleinrock L.} Queueing systems: Vol.~II~--- Computer applications.~---
New York, NY, USA: Wiley Interscience, 1976. 576~p.

\bibitem{avi} %15
\Au{Avi-Itzhak B., Brosh~E., Levy~H.}
SQF: A~slowdown queueing fairness measure~//
Perform. Evaluation, 2007. Vol.~64. No.\,9. P.~1121--1136.
 \end{thebibliography}

 }
 }

\end{multicols}

\vspace*{-3pt}

\hfill{\small\textit{Поступила в~редакцию 28.04.15}}

\newpage

%\vspace*{12pt}

%\hrule

%\vspace*{2pt}

%\hrule

\vspace*{-24pt}

\def\tit{STATIONARY WAITING TIME IN~A~QUEUEING SYSTEM
WITH~INVERSE SERVICE ORDER AND~GENERALIZED
PROBABILISTIC PRIORITY}

\def\titkol{Stationary waiting time in a queueing system with inverse service order and generalized probabilistic priority}

\def\aut{L.\,A.~Meykhanadzhyan$^1$,  T.\,A.~Milovanova$^1$, and
R.\,V.~Razumchik$^{1,2}$}

\def\autkol{L.\,A.~Meykhanadzhyan,  T.\,A.~Milovanova, and
R.\,V.~Razumchik}

\titel{\tit}{\aut}{\autkol}{\titkol}

\index{Meykhanadzhyan L.\,A.}
\index{Milovanova T.\,A.}
\index{Razumchik R.\,V.}

\vspace*{-9pt}


\noindent
$^1$Peoples' Friendship University of Russia, 6 Miklukho-Maklaya Str., Moscow
117198, Russian Federation

\noindent
$^2$Institute of Informatics Problems, Federal Research Center ``Computer Science and Control'' of the Russian\linebreak
 $\hphantom{^1}$Academy of Sciences, 44-2 Vavilov Str., Moscow 119333, Russian Federation


\def\leftfootline{\small{\textbf{\thepage}
\hfill INFORMATIKA I EE PRIMENENIYA~--- INFORMATICS AND
APPLICATIONS\ \ \ 2015\ \ \ volume~9\ \ \ issue\ 2}
}%
 \def\rightfootline{\small{INFORMATIKA I EE PRIMENENIYA~---
INFORMATICS AND APPLICATIONS\ \ \ 2015\ \ \ volume~9\ \ \ issue\ 2
\hfill \textbf{\thepage}}}

\vspace*{3pt}


\Abste{The paper considers a single-server queueing system with a buffer of infinite capacity. Customers arrive according to a Poisson process. Service discipline is LIFO (Last In, First Out)
with generalized probabilistic priority. It is assumed that at any instant, the remaining service time of each customer present in the system is known. Upon arrival of a new customer, its service time is compared with the remaining service time of the customer in service. As a~result of the comparison, one of the following occurs: both customers leave the system; one customer leaves the system and the other occupies the server; and both customers stay in the system (one of the two occupies the server). These actions are governed by probabilistic functions. Whenever a customer remains in the system, it acquires a new (random) service time. The paper proposes the methods for calculating customer's sojourn time distribution and busy period (in terms of Laplace--Stieltjes transform) and several performance characteristics.}

\KWE{queueing system; LIFO; probabilistic priority; general service time}


\DOI{10.14357/19922264150202}

\vspace*{-12pt}

\Ack
\noindent
The research was supported by the Russian Foundation for Basic Research (project 13-07-00223).



%\vspace*{3pt}

  \begin{multicols}{2}

\renewcommand{\bibname}{\protect\rmfamily References}
%\renewcommand{\bibname}{\large\protect\rm References}



{\small\frenchspacing
 {%\baselineskip=10.8pt
 \addcontentsline{toc}{section}{References}
 \begin{thebibliography}{99}
\bibitem{sss-1} %1
\Aue{Meykhanadzhyan, L.\,A., T.\,A.~Milovanova, A.\,V.~Pe\-chin\-kin, and R.\,V.~Razumchik}. 2014.
Sta\-tsi\-o\-nar\-nye veroyatnosti sostoyaniy v~sisteme obsluzhivaniya s~inversionnym poryadkom obsluzhivaniya i~obobshchennym veroyat\-no\-st\-nym
prioritetom [Stationary distribution in a~queueing system with inverse service order and generalized probabilistic priority].
\textit{Informatika i~ee Primeneniya}~---
 \textit{Inform. Appl.}  8(3):16--26.

\bibitem{shrage-1} %2
\Aue{Schrage, L.} 1968.
A~proof of the optimality of
the shortest remaining processing time discipline.
\textit{Oper.\ Res.} 16:687--690.



\bibitem{aaa1-1} %3
\Aue{Nagonenko, V.\,A.} 1981.
O~kharakteristikakh odnoy nestandartnoy sistemy
massovogo obsluzhivaniya
[On the characteristics of one nonstandard queuing
system].~I; II.
\textit{Izv.\ AN SSSR. Tekhnich.\ Kibernet.}
[Proceedings of the Academy of Sciences of
the USSR. Technical Cybernetics] (1):187--195; (3):91--99.

\bibitem{aaa3-1} %4
\Aue{Nagonenko, V.\,A., and A.\,V.~Pechinkin}. 1982.
O~bol'shoy zagruzke v~sisteme s~inversionnym
obsluzhivaniem i~ve\-ro\-yat\-no\-st\-nym prioritetom
[On high load in the system with an inversion
procedure service and probabilistic priority].
\textit{Izv.\ AN SSSR. Tekhnich.\ Kibernet.}
[Proceedings of the Academy of Sciences
of the USSR. Technical Cybernetics] (1):86--94.

\bibitem{aaa2-1} %5
\Aue{Pechinkin, A.\,V.} 1983.
Ob odnoy invariantnoy sisteme massovogo
obsluzhivaniya [On an invariant queuing system].
\textit{Math.\ Operationsforsch.\ und Statist.
Ser.\ Optimization} 14(3):433--444.

\bibitem{aaa4-1} %6
\Aue{Nagonenko, V.\,A., and A.\,V.~Pechinkin}. 1984.
O~ma\-loy zagruzke v~sisteme s inversionnym poryadkom
obsluzhivaniya i~veroyatnostnym prioritetom
[On low load in the system with an inversion
procedure service and probabilistic priority].
\textit{Izv.\ AN SSSR. Tekhnich.\ Kibernet.}
[Proceedings of the Academy of Sciences
of the USSR. Technical Cybernetics] (6):82--89.

\bibitem{ppav-1} %7
\Aue{Bocharov,  P.\,P., and A.\,V.~Pechinkin}. 1995.
\textit{Teoriya massovogo obsluzhivaniya} [Queueing theory].
Moscow: RUDN. 529~p.

\bibitem{av1-1} %8
\Aue{Pechinkin, A.\,V., and I.\,V.~Stalchenko}. 2010.
Sis\-te\-ma MAP$/G/1/\infty$ s~inversionnym poryadkom
obsluzhivaniya i~veroyatnostnym prioritetom,
funk\-tsi\-o\-ni\-ru\-yushchaya v~diskretnom vremeni
[The MAP$/G/1/\infty$ discrete-time queueing
system with inversive service order and probabilistic
priority].
\textit{Vestnik Rossiyskogo Universiteta Druzhby
Narodov. Ser.\ Matematika. Informatika. Fizika}
[Bulletin of Peoples' Friendship University
of Russia. Ser. Mathematics. Information Sciences.
Physics] 2:26--36.

\bibitem{jawbz-1} %9
\Aue{Nair, J., A. Wierman, and B.~Zwart}.
2010. Tail-robust scheduling via limited processor sharing.
\textit{Perform. Evaluation} 67(11):978--995.

\bibitem{jawbz1-1} %10
\Aue{Wierman, A., and B.~Zwart}.
2012. Is tail-optimal scheduling possible? \textit{Oper.\ Res.} 60(5):1249--1257.

\bibitem{jerri-1} %11
\Aue{Jerri, A.} 1999.
\textit{Introduction to integral equations with applications}.
 New York, NY: John Wiley \& Sons. 272~p.



\bibitem{wh-1} %12
\Aue{Press, W.\,H., S.\,A.\ Teukolsky, W.\,T.~Vetterling,
and B.\,P.~Flannery}. 2007.
\textit{Numerical recipes}. 3rd ed. The Art of Scientific Computing. 1235~p.

\bibitem{adav-1} %13
\Aue{Polyanin, A.\,D., and A.\,V.~Manzhirov}.
2008. \textit{Handbook of integral equations}.
 Boca Raton\,--\,London: Chapman\,\& Hall\,/\,CRC Press. 1108~p.

\bibitem{klein-1} %14
\Aue{Kleinrock, L.} 1976. \textit{Queueing systems: Vol.~II~--- Computer applications}. New York, NY: Wiley Interscience. 576~p.

\bibitem{avi-1} %15
\Aue{Avi-Itzhak, B., E. Brosh, and H.~Levy}. 2007.
SQF: A~slowdown queueing fairness measure.
\textit{Perform. Evaluation} 64(9):1121--1136.
\end{thebibliography}

 }
 }

\end{multicols}

\vspace*{-3pt}

\hfill{\small\textit{Received April 28, 2015}}

%\vspace*{-18pt}


\Contr

\noindent
\textbf{Meykhanadzhyan Lusine A.} (b.\ 1990)~---
PhD student, Peoples' Friendship University of Russia, 6~Miklukho-Maklaya Str., Moscow 117198, Russian Federation; lameykhanadzhyan@gmail.com

\vspace*{3pt}

\noindent
\textbf{Milovanova Tatiana A.} (b.\ 1977)~---
Candidate of Science (PhD) in physics and mathematics, senior lecturer, Peoples' Friendship University of Russia, 6 Miklukho-Maklaya Str., Moscow 117198, Russian Federation; tmilovanova77@mail.ru

\vspace*{3pt}

\noindent
\textbf{Razumchik Rostislav V.} (b.\ 1984)~---
Candidate of Science (PhD) in physics and mathematics, senior scientist, Institute of Informatics Problems, Federal Research Center ``Computer Science and Control'' of the Russian Academy of Sciences, 44-2 Vavilov Str., Moscow 119333, Russian Federation; associate professor, Peoples' Friendship University of Russia, 6 Miklukho-Maklaya Str., Moscow
117198, Russian Federation; rrazumchik@ieee.org

\label{end\stat}


\renewcommand{\bibname}{\protect\rm Литература} 