\def\stat{vasiliev}

\def\tit{КОАЛИЦИОННО УСТОЙЧИВЫЕ ЭФФЕКТИВНЫЕ РАВНОВЕСИЯ В МОДЕЛЯХ
КОЛЛЕКТИВНОГО ПОВЕДЕНИЯ С ОБМЕНОМ ИНФОРМАЦИЕЙ}

\def\titkol{Коалиционно устойчивые эффективные равновесия в~моделях
коллективного поведения с обменом информацией}

\def\aut{Н.\,С.~Васильев$^1$}

\def\autkol{Н.\,С.~Васильев}

\titel{\tit}{\aut}{\autkol}{\titkol}

\index{Васильев Н.\,С.}

%{\renewcommand{\thefootnote}{\fnsymbol{footnote}} \footnotetext[1]
%{Работа выполнена при финансовой
%поддержке РФФИ (проект 15-07-02652).}}


\renewcommand{\thefootnote}{\arabic{footnote}}
\footnotetext[1]{Московский государственный технический университет им. Н.\,Э.~Баумана, nik8519@yandex.ru}

\vspace*{-3pt}

    \Abst{Теория игр изучает процессы принятия решений в~условиях
неопределенности и~конфликта интересов. В связи с~развитием сетевых технологий
практический интерес вызывают модели операций, в~которых участники
обмениваются информацией с~целями принятия рациональных решений. В статье
проведена аксиоматизация понятия стратегии. Обмен сведениями о стратегиях
моделирует коллективные усилия игроков, направленные на уменьшение
неопределенности, присущей этому процессу. В~основе рационального выбора
стратегий лежат два основных принципа~--- эффективности и~равновесия
(устойчивости), которые, как правило, противоречат один другому. Доказано, что
надлежащий обмен информацией позволяет осуществить выбор устойчивой
эффективной ситуации игры. Исследовано строение эффективных равновесных
стратегий. Введено понятие коалиционно устойчивого решения игры, обобщающее
понятие равновесия Нэша на случай, когда игроки могут вступать в~коалиции.
Доказана достижимость этого решения с~помощью расширения игры введением
дополнительного контролирующего игрока и~надлежащего информационного
обмена.}

    \KW{игра; стратегия; ситуация; исход игры; передача информации; динамика
принятия решений; аксиоматизация; коалиция; кооперативная игра;
характеристическая функция игры; наилучший гарантированный результат;
стратегия наказания; равновесие Нэша; эффективность по Парето}

\DOI{10.14357/19922264150201}

\vspace*{-3pt}

\vskip 12pt plus 9pt minus 6pt

\thispagestyle{headings}

\begin{multicols}{2}

\label{st\stat}

\section{Введение}

    Сложность формализации и~исследования игровых задач, связанных с~принятием рациональных решений, вынуждает ограничиваться изучением игр с~малым числом игроков и~с~простейшими видами стратегий, например
    \textit{стра\-те\-ги\-ями-констан\-та\-ми}. Из-за этого классические модели не
соответствуют приложениям, в~которых наблюдаются весьма разнообразные
способы коллективного поведения игроков. Нельзя заранее преуменьшать уровень
рефлексии игроков. Требуется предусмат\-ри\-вать самые общие классы стратегий. Эти
классы должны включать функционалы, с~помощью которых игроки осуществляют
свой выбор в~зависимости от имеющейся информации.
Исследования в~этом
направлении ранее проводились школой Ю.\,Б.~Гермейера. Речь идет о работах,
посвященных\linebreak
 изучению метаигр и~игр с~фиксированным порядком ходов~[1--3].
Развитие понятия стратегии происходило в~теории динамических игровых задач,
в~позиционных дифференциальных и~многошаговых играх.

Библиографические
ссылки на классические труды по теории игр и~многочисленные примеры
применяемых стратегий содержатся, например, в~работах Дж.~Неймана, Э.~Цермело,
Н.\,Н.~Красовского, А.\,И.~Субботина,
В.\,И.~Горелика, А.\,Ф.~Кононенко, Л.\,А.~Петросяна, В.\,И.~Жуковского~[1--10].
К~данному исследованию наиболее
близки результаты Ю.\,Б.~Гермейера и~Н.\,С.~Кукушкина, полученные для игр
$\Gamma^1$, $\Gamma^2$ и~$\Gamma^3$ двух лиц с~\textit{фиксированным}
порядком ходов и~обменом информацией~\cite{1-vas, 4-vas}.

    Игры многих лиц~$N$, $N\hm>2$, практически не изуче\-ны.
    Это связано с~существенным усложнением задачи изучения рационального
    поведения игроков с~ростом~$N$. Развитие сетевых технологий и~компьютерной
    техники стимулирует
исследование моделей коллективного поведения~\cite{11-vas}. Например, в~игровой
постановке задачи маршрутизации пакетной сети передачи данных число игроков
измеряется десятками тысяч~[12]. Все участники игры заинтересованы
в~рациональном разрешении конфликта интересов. Для этого они могут обмениваться
информацией о~применяемых стратегиях. Можно обеспечить достижимость
ситуаций равновесия, если увеличить информированность игроков и~тем самым
расширить их классы стратегий. Так, с~помощью использования стратегий-функций
может проводиться численный поиск равновесий Нэша матричных игр в~классе
стратегий-констант~[13].

    В теории игр предложены два основных принципа рационального выбора
решений, вообще говоря, противоречащих один другому. Это принципы равновесия
Нэша и~эффективности по Парето~[14]. В работе поставлена и~решена задача о
возможности рационального выбора, удовлетворяющего одновременно обоим этим
принципам. Для этого необходим обмен информацией между игроками.

    Более сложное поведение игроков включает возмож\-ность кооперации. При
<<справедливом дележе>> выигрыша, получаемого коалицией, ее участники могут
рассчитывать на увеличение своих плате\-жей~[1, 6--8].
В~статье введено понятие \textit{коалиционной устойчивости} решения игры,
<<выдерживающего>> возможность объединения игроков в~коалиции. Доказано,
что существуют эффективные коалиционно устойчивые равновесия игры.

\section{Постановка игровой задачи }

    Пусть в~игровой операции принимают участие игроки $i\hm\in I\hm= \{
1,2,\ldots, n+1)$, интересы которых заключаются в~максимизации своих критериев
эффективности:
    \begin{equation}
    w_i= f_i\left( x_1,\ldots ,x_{n+1}\right)\,,\enskip x_i\in X_i,\ i\in I\,.
    \label{e2.1-vas}
    \end{equation}
Один из игроков, пусть это $n\hm+1$, отвечает природной неопределенности,
<<безразличие>> которой моделируется с~по\-мощью функции $f_{n+1}\hm=
const$. Целью каждого $i$-го игрока является выбор контролируемого им
фактора~$x_i$ из некоторого множества значений~$X_i$, характеризующего
возможности этого игрока в~рассматриваемой операции. Не исключено, что
некоторые \textit{исходы} $x\hm= (x_1,\ldots, x_{n+1})$ игры запрещены. Тогда на
выбор игроков дополнительно влияют ограничения вида $x\hm\in Q_i$, в~которых
множества $Q_i\subset X\equiv X_1\times X_2\times \cdots \times X_{n+1}$.

    Согласно~(\ref{e2.1-vas}) результат (выигрыш) любого игрока зависит не
только от его действий, но и~от выбора параметров~$x_j$, $j\not= i$,
контролируемых остальными участниками игры. Поэтому ни один из игроков не
имеет возможности точно прогнозировать величину получаемого им
выигрыша и~вынужден принимать решение о~выборе значения своего
па\-ра\-мет\-ра в~условиях
<<хаоса>> будущих результатов игры. С~по\-мощью фиксирования разнообразных
\textit{классов} стратегий моделируются возможные способы поведения игроков.
Классы применяемых стратегий зависят от информированности игрока об
обстановке игры и~о~поведении партнеров. В~стратегию игроков входит
возможность добровольной передачи информации о своем выборе стратегии, что
влияет на классы стратегий, применяемых партнерами. Выбор стратегии называется
ходом игрока. Ходы всегда делаются в~некоторой очередности и~должны приводить
к~некоторой \textit{ситуации} (\textit{исходу}) $x\hm\in X$.

    За счет введения дополнительного игрока, называемого <<природой>>, обычно
проводится объ\-ективизация обстановки проведения игровой операции, благодаря
которой можно считать, что каждо\-му игроку точно известны цели и~возможности
всех участников игры, заданной в~нормальной форме~(\ref{e2.1-vas}).
Руководствуясь принципом наилучшего гарантированного результата или
располагая знанием вероятностного распределения, определенного на множестве
параметров природы~$X_{n+1}$, обычно устраняют природную неопределенность,
должным образом модифицируя критерии остальных, <<разумных>>
игроков~\cite{1-vas}. Поэтому далее считаем, что $I\hm= \{1,2,\ldots, n\}$ и~функции
выигрыша $f_i:\ X\hm\to R$, $i\hm\in I$, точно известны всем игрокам.

    Каждый игрок может располагать некоторой информацией о факторах~$x_j$,
выбираемых партнерами, например возможен добровольный обмен этой
информацией. Если игрок~$i$ знает~$x_j$, то удобно считать, что значение
фактора~$x_j$ ему сообщает сам выбирающий игрок~$j$. Процесс (динамика)
принятия решений с~учетом информированности игроков формализуется в~понятии
стратегии.

    \textit{Стратегией} $i$-го игрока назовем функционал~$\tilde{x}_i$ вида:
    \begin{equation}
    \tilde{x}_i= x_i\left( x_{J_1(i)}, \tilde{x}_{J_2(i)}\right)\,,\enskip i\in I\,.
    \label{e2.2-vas}
    \end{equation}
Для упрощения записи здесь введено обозначение $x_J\hm= x_{J(i)}$ для вектора
$(x_i,\ j\in J(i))$. Иными словами, стратегия~(\ref{e2.2-vas})~--- это правило выбора
значения контролируемого фактора $x_i\hm\in X_i$ в~зависимости от имеющейся
информации о действиях других игроков~--- результатов их выборов. Индексные
множества обладают свойствами
\begin{multline*}
J_1(i)\cap  J_2(i)=\emptyset\,,\\
 i\not\in J(i) =J_1(i)\cup J_2(i)\,,\ i=1,2,\ldots, n\,,
\end{multline*}
означающими невозможность одновременной передачи информации о выбираемом
значении $x_i\hm\in X_i$ контролируемого фактора и~о~стратегии этого
выбора~$\tilde{x}_i$.

    Выбор допустимых стратегий игроков $\tilde{x}\hm\in \tilde{X}$ должен
приводить к~однозначно определенной ситуации игры~$x$, определяющей их
платежи~(\ref{e2.1-vas}). Состав\-ной частью применяемой стратегии может служить
добровольная передача сведений о своем выборе~\cite{1-vas}. Подобное поведение
зачастую мотивировано близостью интересов взаимодействующих игроков
и~уменьшением неопределенности прогнозируемой ситуации игры с~целями
нахождения рациональных решений~$\tilde{x}^*$. Классам
стратегий~$\tilde{X}_i$~(\ref{e2.2-vas}) отвечает \textit{отношение
предшествования}~$\tilde{D}$, задаваемое на множестве ходов игроков. Каждому
участнику конфликта приходится решать, когда сделать свой \textit{ход}~---
осуществить выбор стратегии и, возможно, сообщить его ко\-му-ли\-бо из партнеров
по игре, еще не определившихся со своим ходом. Переда\-ва\-емые данные влияют на
вид стратегий партнеров (см.~(\ref{e2.2-vas})). Последние делают свой ход позже,
учитывая получаемую информацию. В~иерархических системах и~в~многошаговых
играх классы~$\tilde{X}_i$ заданы правилами игры~\cite{15-vas}.

\subsection{Динамика принятия решений}

    Очередность ходов в~игре и~классы применяемых стратегий формализуем с~помощью некоторого бинарного антисимметричного отношения~$\tilde{D}$,
задание которого определяет классы применяемых стратегий~(\ref{e2.2-vas}). Если
между игроками отсутствует обмен информацией, то берется тривиальное
отношение $\tilde{D}\hm= \emptyset$, означающее, что все игроки выполняют свои
ходы одновременно и~используют лишь стра\-те\-гии-кон\-стан\-ты. В~свою
очередь, отношение~$\tilde{D}$ и~соответствующие информационные обмены
однозначно определены заданием классов стратегий~(\ref{e2.2-vas}). Наличие
аргументов $\tilde{x}_{J_2(i)}$ функционала~$\tilde{x}_i$~(\ref{e2.2-vas}) выделяет
игроков~$j$, $j\hm\in J_2(i)$, которые делают свой ход \textit{раньше} $i$-го игрока.
Отношение предшествования наглядно представимо
\textit{диаграммой}~$\tilde{D}$. В~случае $(j,i)\hm\in \tilde{D}$ будем говорить, что
элемент~$j$ множества~$I$ <<меньше>> элемента $i\hm\in I$, а~элемент~$i$
<<больше>>~$j$. Наличие дуги $(j,i)$ равносильно тому, что игрок~$j$ ходит
раньше игрока~$i$. Транзитивное замыкание отношения~$\tilde{D}$ определяет
частичный порядок на множестве игроков~$I$~\cite{16-vas}. Требование
непротиворечивости процессов, связанных с~обменом информацией и~принятием
решений, накладывает определенные ограничения на классы допустимых стратегий
игроков~(\ref{e2.2-vas}), т.\,е.\ на вид диаграмм~$\tilde{D}$.

    С классами функций~(\ref{e2.2-vas}) связано еще одно отношение~$D$,
\textit{двойственное} к~отношению~$\tilde{D}$. С~частичной
упорядоченностью~$D$ происходит окончательный выбор игроками
контролируемых факторов~$x_i$ как результат сделанных ходов. Выбор величин
$x_j\hm\in X_j$, $j\hm\in J_1(i)$, должен произойти \textit{раньше}~$x_i$, ведь
$x_i$~--- значение функции~$\tilde{x}_i$~(\ref{e2.2-vas}). В~паре $(j,i)\hm\in D$,
образующей дугу графа~$D$, первый элемент, $j$, будем называть <<меньшим>>,
а~второй, $i$,~--- <<большим>> по отношению~$D$. Транзитивное замыкание
отношения~$D$ является двойственным к~$\tilde{D}$ частичным порядком,
в~котором вычисляются значения параметров $x_j\hm\in X_j$. Заметим, что
пересечение отношений $\tilde{D}\cap D\hm= \emptyset$~--- тривиальный порядок
на множестве~$I$. Таким образом, вершины обеих диаграмм ходов отвечают
игрокам. По стрелкам $(j,i)$ передаются данные~$\tilde{x}_j$, $j\hm\in J_2(i)$,
или~$x_j$, $j\hm\in J_1(i)$, в~зависимости от того, о~какой из диаграмм~$\tilde{D}$
или~$D$ идет речь. Задание диаграмм $\tilde{D}$ и~$D$ равносильно определению
классов стратегий~(\ref{e2.2-vas}). Диаграмма~$\tilde{D}$ представляет стратегии
игроков как функционалы. В~двойственной к~$\tilde{D}$ диаграмме~$D$ стратегии
рассматриваются как функции. Ситуация игры, отвечающая выбору стратегий из
классов~(\ref{e2.2-vas}), определяется в~результате вычислений (подстановок),
проводимых в~соответствии с~этими диаграммами.

    \smallskip

    \noindent
    \textbf{Пример 2.1.} Построим диаграммы очередности ходов и~выбора
контролируемых параметров~$\tilde{D}$ и~$D$ для игры трех лиц, в~которой
игроки применяют следующие классы стратегий
      \begin{align*}
    \tilde{x}_1 &= x_1\left( \tilde{x}_2, \tilde{x}_3\right)\,;\\
    \tilde{x}_2 &= x_2\left( {x}_1, \tilde{x}_3\right)\,;\\
    \tilde{x}_3 &= x_3\left( {x}_1, {x}_2\right)\,.
    \end{align*}
Соответствующие диаграммы~$\tilde{D}$ и~$D$ отвечают полным взаимно
обратным ($\tilde{D}^{-1}\hm= D$) порядкам ходов, имеющим соответственно вид:
\begin{center}  %fig
\vspace*{8pt}
\mbox{%
 \epsfxsize=71.415mm
 \epsfbox{vas-1.eps}
 }
\end{center}

\vspace*{6pt}




    \noindent
    \textbf{Замечание 2.1.} Игроки могут сообщать не обязательно полную
информацию о своих стратегиях. Для однозначности <<восстановления>> классов
стратегий~(\ref{e2.2-vas}) по виду диаграмм $\tilde{D}$ и~$D$ достаточно пометить их
дуги теми частичными данными, которые по ним передаются. Если некоторые
игроки блефуют, то пометка соответствующих дуг должна использовать другие
обозначения, не совпадающие с~переменными $x$ и~$\tilde{x}$.

    \smallskip

    \noindent
    \textbf{Пример 2.2} (\textit{игра с~<<контролирующим>> участником}).
Пусть контролирующий игрок $i\hm= n\hm+1$ имеет целевую функцию
$f_{n+1}\hm= const$ и~располагает знанием ходов~$\tilde{x}_j$ остальных игроков $i\hm= 1,2,\ldots, n$. Он
распоряжается выбором параметра $x_{n+1}\hm\in X_{n+1}$, где
    \begin{multline*}
    X_{n+1} =\mathop{\bigcup}\limits_{j=1,2,\ldots, n} \left\{ x_{n+1} =\left( x^1_{n+1},\ldots ,
x^n_{n+1}\right):\right.\\
\left. x^i_{n+1}\in \left\{ j\right\} \times X_j\right\} \cup\ 0\,,\ 0\in R^n\,.
    \end{multline*}
При этом игрок $n+1$ сообщает каждому участнику конфликта~$i$, $i\hm=
1,2,\ldots, n$, лишь $i$-ю координату выбираемого им вектора~$x_{n+1}$. Выбор
значения $x_{n+1}\hm=0$ используется в~качестве признака <<желательного>>
коллективного поведения игроков $i\hm=1,2,\ldots, n$, например, приводящего
к~некоторому, вполне определенному исходу игры. Таким образом, по стрелкам
$(n+1,i)$ диаграммы~$D$ вида
    \begin{center}  %fig
\vspace*{8pt}
\mbox{%
 \epsfxsize=59.927mm
 \epsfbox{vas-2.eps}
 }
\end{center}
\vspace*{6pt}

\noindent
     передаются значения $x_{n+1}^i$. Подобные диаграммы~$D$
характерны для иерархических систем принятия решений~\cite{15-vas}.
Согласно сказанному выше, двойственное отношение $\tilde{D}\hm= D^{-1}$.
Итак, классы допустимых стратегий игроков содержат функции:
    \begin{align*}
    \tilde{x}_{n+1} &= x_{n+1}\left(\tilde{x}_1,\ldots, \tilde{x}_n\right)\,;\\
    \tilde{x}_i &= x_i\left( x^i_{n+1}\right)\,,\enskip i=1,2,\ldots, n\,.
    \end{align*}

    \noindent
    \textbf{Теорема~2.1.} \textit{Критерием непротиворечивости процесса
принятия решений для игры в~нормальной форме}~(\ref{e2.1-vas}) \textit{и классами
стратегий} $\tilde{X}_i\hm= \left\{ \tilde{x}_i\right\}$~(\ref{e2.2-vas}) \textit{является
ацик\-лич\-ность обеих диаграмм $\tilde{D}$ и~$D$. В~этом случае выбор стратегий
$\tilde{x}\hm= \tilde{X}\hm= \tilde{X}_1\times\cdots \times \tilde{X}_{n+1}$ однозначно
определяет исход игры $x\hm\in X$, так что определено отображение}
    \begin{equation}
    s:\ \tilde{X}\to X\,.
    \label{e2.3-vas}
    \end{equation}


    \noindent
    Д\,о\,к\,а\,з\,а\,т\,е\,л\,ь\,с\,т\,в\,о\,.\ \ Пусть отображение вида~(\ref{e2.3-vas})
существует. Построим диаграммы $\tilde{D}$ и~$D$. Если хотя бы в~одной из них
найдем цикл~$C$, то эта диаграмма не может определять отношение частичного
порядка, в~котором проводятся подстановки одних функций~(\ref{e2.2-vas})
в~другие. Тогда вопреки допущению не определено отображение~(\ref{e2.3-vas}).
Установим теперь достаточность условия ацикличности (отсутствие циклов)
диаграмм $\tilde{D}$ и~$D$, воспользовавшись методом математической индукции. Для
диаграмм, содержащих единственный узел, доказываемое утверждение очевидно.
Предположим справедливость теоремы в~случае всех ацикличных диаграмм с~не
более чем $N\hm-1$ вершинами. Докажем, что тогда теорема верна для диаграмм
с~$N$, $N\hm>1$, вершинами. В~ациклическом графе~$D$ существует
вершина~$i_0$, отвечающая минимальному элементу
    отношения~$D$~\cite{16-vas}. Узел~$i_0$ не имеет входных дуг. Если
    элемент~$i_0$ является также минимальным по отношению к~$\tilde{D}$, то игрок~$i_0$
применяет некоторую стра\-те\-гию-кон\-стан\-ту~$x_{i_0}$ (см.~(\ref{e2.2-vas})).
Подставив значение параметра~$x_{i_0}$ во все функции~(\ref{e2.2-vas})), $i\not=
i_0$, исключим вершину~$i_0$ из обеих диаграмм вместе с~исходящими из нее
дугами. Если теперь элемент~$i_0$ не минимален по отношению к~$\tilde{D}$, то
по дугам $(j,i_0)\hm\in \tilde{D}$ передаются стратегии~$\tilde{x}_j$, $j\hm\in
J_2(i_0)$. Для исключения узла~$i_0$ из диаграммы~$D$ достаточно найти
значение параметра~$x_{i_0}$, вычислив значение функции $\tilde{x}_i\hm=
x_i\left( \tilde{x}_{J_2(i)}\right)$, $i\hm= i_0$. Аналогичные рассуждения
применимы к~начальному элементу~$\tilde{i}_0$ диаграммы~$\tilde{D}$,
приводящие к~исключению узла~$\tilde{i}_0$ из диаграммы~$\tilde{D}$.
В~полученных новых диаграммах $\tilde{D}^\prime$ и~$D^\prime$ содержится не более
$N\hm-1$ вершин. Осталось воспользоваться предположением индукции, что
завершает доказательство.
    \smallskip

    \noindent
    \textbf{Следствие 2.1.} Транзитивное замыкание отношений $\tilde{D}$ и~$D$
сохраняет непротиворечивость исходных классов стратегий.

    \smallskip

    Итак, в~теореме~2.1 обоснована проведенная аксиоматизация понятия
стратегии, использующая диаграммы отношений $\tilde{D}$ и~$D$, дуги которых
помечены передаваемыми данными. Классы стратегий игроков назовем
\textit{допустимыми}, если диаграммы отношений $\tilde{D}$ и~$D$ не содержат
циклов. Стратегии из этих классов также называются допустимыми. Они вполне
определены диаграммами $\tilde{D}$ и~$D$ (см.\ пример~2.2).

    В условиях теоремы~2.1 выбор всеми игроками своих
    стратегий~(\ref{e2.2-vas}), $\tilde{x}_i\hm\in \tilde{X}_i$, $i\hm\in I$,
определяет \textit{исход} игры $x\hm= s(\tilde{x}) \hm\in X$, $\tilde{x}\hm\in
\tilde{X}$, и,~как следствие, выигрыши игроков, вычисляемые по
формулам~(\ref{e2.3-vas}), (\ref{e2.1-vas}), определяющие функции $\tilde{f}_i:\
\tilde{X}\hm\to R$,
    \begin{equation}
    \tilde{f}_i\left(\tilde{x}\right) \eqdelta f_i\left(s\left(\tilde{x}\right)\right)
=f_i(x)\,,\enskip i\in I\,.
    \label{e2.1-1-vas}
    \end{equation}
Если имелись запрещенные исходы~$Q_i$, то в~<<новой>>
игре~(\ref{e2.1-1-vas}), $\tilde{x}\hm\in \tilde{X}$, они принимают вид
$\tilde{x}\hm\in s^{-1} Q_i$. Зависимость стратегий~(\ref{e2.2-vas}) от
<<параметров>>~$\tilde{x}_j$ говорит о более высоком уровне рефлексии игроков
по сравнению с~игрой~(\ref{e2.1-vas}), $\tilde{x}\hm\in X^{0}$, в~которой $X^0$~---
множество \textit{стра\-те\-гий-констант}. Дополнительная информированность
$X_i^0\hm\subset \tilde{X}_i$ увеличивает возможности игроков по улучшению
выигрышей.

    В игре~(\ref{e2.1-1-vas}), $f^\prime \hm= \tilde{f}_i$, $X_i^\prime\hm=
\tilde{X}_i$, в~свою очередь, все или часть игроков могут использовать
стратегии~(\ref{e2.2-vas}), $x\hm= x^\prime$, $\tilde{x}\hm= \tilde{x}^\prime$.
Подобное усложнение классов применяемых стратегий приводит к~играм,
называемым \textit{метаиграми}~[1--3].

    \subsection{Рационализация выбора стратегий}

    В теории игр предпринята попытка разрешить конфликт между устойчивостью
поведения игроков (\textit{принципом равновесия}) и~выгодностью для игроков
принимаемых решений (\textit{принципом эффективности}).

    \subsubsection{Принцип эффективности }

    Для операций~(\ref{e2.1-vas}) с~векторным критерием $f:\ X\hm\to R^{n+1}$
рациональный выбор игроками своих стратегий основывается на понятии
\textit{слабо эффективной} (оптимальной по Слейтеру)
    ситуации~$x^*$~\cite{6-vas, 14-vas, 17-vas}. Это решение~$x^*$ таково, что не
существует элемента $x\hm\in X$, для которого $f(x)\hm> f(x^*)$. Множество всех
слабоэффективных решений игры~$S$, как правило, не является одноэлементным.
Поэтому возникает необходимость дальнейшего уточнения выбора игроков.
Предпочтительней рассматривать подмножество  $P\subset S$ \textit{эффективных}
(\textit{оптимальных по Парето}) точек~$x^*$, т.\,е.\ таких векторов, что не
существует элемента $x\hm\in X$, для которого выполнялось бы условие
$$
f(x)\geq f(x^*)\,,\enskip f(x)\not= f(x^*)\,.
$$

    При выполнении стандартных предположений непрерывности $f:\ X\hm\to
R^{n+1}$, компактности или конечности множества~$X$ доказывается, что
множество~$P$ эффективных решений игры не\-пус\-тое. Подобные требования далее
предполагаются выполненными. Из-за неоднозначности выбора $x^*\hm\in P$
целесообразно еще более сузить множество~$P$, ограничившись подмножеством
вида $P_L\hm= \{ x\in P:\ f(x)\hm\geq L\}$. Здесь вектор $L\hm= L_I \hm= (L_1,\ldots,
L_n)$ составлен из \textit{наилучших гарантированных результатов} игроков
    $$
    L_i\left(\tilde{X}_i\right) =
    \max\limits_{\tilde{X}_i}
    \min\limits_{\tilde{x}_j\in \tilde{X}_j,\, j\not=i} \tilde{f}_i(\tilde{x})\,,\enskip  i\in
I\,,
    $$
в игре~(\ref{e2.1-vas}), (\ref{e2.2-vas}). Дело в~том, что никто из игроков <<не
согласится>> получить выигрыш, меньший того, который он может сам себе
гарантировать. Здесь и~далее предполагаем, что если ка\-кой-ли\-бо экстремум не
достигается, то берется $\varepsilon$-ре\-ше\-ние оптимизационной задачи с~некоторой точностью $\varepsilon\hm >0$ по значению критерия.

    Введем обозначение $\tilde{P}$ для множества оптимальных по Парето
стратегий $\tilde{x}\hm\in \tilde{P}$, т.\,е.\ удовлетво\-ря\-ющих включению
$s(\tilde{x})\hm\in P$. Аналогично понимаются множества~$S_L$, $\tilde{S}_L$,
$\tilde{P}_L$.

    \smallskip

    \noindent
    \textbf{Лемма 2.1.} \textit{Пусть все целевые функции $f_i\hm >0$. Тогда для
любых классов допустимых стратегий множество} $\tilde{P}_L\not= \emptyset$.

    \smallskip

    \noindent
    Д\,о\,к\,а\,з\,а\,т\,е\,л\,ь\,с\,т\,в\,о\,.\ \  Рассмотрим какое-либо
решение~$\tilde{x}^S$ оптимизационной задачи

\noindent
    \begin{equation}
    \tilde{x}^S\in \mathrm{Arg}\,\max\limits_{\tilde{X}}
    \min\limits_{i=1,2,\ldots, n+1} \fr{\tilde{f}_i(\tilde{x})}{L_i}\,,
    \label{e2.4-vas}
    \end{equation}
служащей критерием слабой эффективности~\cite{17-vas}. Пусть вектор
$\tilde{x}^\Gamma \hm\in \tilde{X}$ составлен из \textit{гарантирующих} стратегий
игроков~$\tilde{x}_i^\Gamma$, т.\,е.\ таких, что
$$
\tilde{x}^\Gamma_i \in \mathrm{Arg}\,\max\limits_{\tilde{x}_i}
\min\limits_{\tilde{x}_j,\,j\not=i} \tilde{f}_i \left( \tilde{x}_1, \ldots ,
\tilde{x}_{n+1}\right)\,,\ i\in I\,.
$$
Тогда для всех допустимых значений индекса~$i$ выполнены неравенства
$\tilde{f}_i \left( \tilde{x}^\Gamma\right) \hm\geq L_i$. Ввиду~(\ref{e2.4-vas}) для
набора стратегий игроков $\tilde{x}^S\hm\in \tilde{S}$ выполнено

\columnbreak

\noindent
$$
\min\limits_i \fr{\tilde{f}_i \left( \tilde{x}^S\right)}{L_i} \geq \min\limits_i
\fr{\tilde{f}_i\left( \tilde{x}^\Gamma\right)}{L_i} \geq 1\,.
$$
Следовательно, исход игры $x^S\hm =s(\tilde{x}^S)$ содержится в~множестве~$S_L$.
Если $x^S\not\in P_L$, то результат~$x^S$, по определению эффективности, может
быть улучшен выбором вектора~$x^P$, для которого $f\left( x^P\right) \hm\geq f\left(
x^S\right)$. По построению $x^P\hm\in P_L$.

    \subsubsection{Принцип устойчивости }

    В качестве рационального выбора игроков понимается \textit{ситуация
равновесия} Нэша~$\tilde{x}^*$, т.\,е.\ такой набор стратегий игроков, который
определяется условием:
\begin{multline}
(\forall\ i) (\forall\ \tilde{x}_i) \tilde{f}_i \left( \tilde{x}_1^*, \ldots ,  \tilde{x}^*_{i-1}, \tilde{x}_i,
\tilde{x}_{i+1}^*,\ldots , \tilde{x}^*_n\right) \leq{}\\
{}\leq \tilde{f}_i \left( \tilde{x}^*\right)\,.
\label{e2.5-vas}
\end{multline}

    По смыслу соотношений~(\ref{e2.5-vas}) в~ситуации~$\tilde{x}^*$ каж\-до\-му
игроку~$i$, $i\hm\in I$,  не выгодно в~одиночку отклоняться от своей
стратегии~$\tilde{x}^*_i$. Это обеспечивает равновесность ситуации~$\tilde{x}^*$.

    \smallskip

    \noindent
    \textbf{Теорема~2.2.} \textit{Если классы допустимых стратегий таковы, что
$\tilde{X}_i\subset \tilde{X}^{\prime}_i$, $i\hm=1,2,\ldots, N$, то множества
равновесий игр}~(\ref{e2.1-vas}), (4), $\tilde{X}_i$, \textit{и}~(1),
(\ref{e2.1-1-vas}),  $s\hm=s^\prime$,
$\tilde{X}_i^\prime$, \textit{связаны включением} $\tilde{X}^*\subset
    \tilde{X}^{\prime *}$.

    \smallskip

    \noindent
    Д\,о\,к\,а\,з\,а\,т\,е\,л\,ь\,с\,т\,в\,о\,.\ \ Расширение классов стратегий за счет
обмена информацией соответствует добавлению стрелок в~диаграммы
упорядоченности ходов $\tilde{D}$ и~$D$. При этом по теореме~2.1 нельзя нарушать
свойство ацикличности получаемых графов. Поэтому достаточно провести
доказательство методом математической индукции, проводимой по чис\-лу стрелок в~диаграммах~$\tilde{D}$ или~$D$.
Пусть $\tilde{x}^*\hm\in \tilde{X}^*$ и~для
определенности в~диаграмму~$\tilde{D}$ включена новая дуга $(j,i)$. Тогда, по
сравнению с~исходной игрой, участник~$i$ получает дополнительную информацию
о выборе стратегии~$\tilde{x}_j$. Зафиксировав $\tilde{x}_j^*$, $j\not= i$, \mbox{найдем}
стратегию $\tilde{x}_i^\prime$, оптимальную по \mbox{$i$-му} критерию
эффективности~(\ref{e2.1-vas}) при $\tilde{x}_i^\prime\hm\in \tilde{X}_i^\prime$.
\mbox{Изучим} ситуацию $\tilde{x}^\prime \hm= \left( \tilde{x}_1^*,\ldots, \tilde{x}_{i-1}^*,
\tilde{x}_i^\prime, \tilde{x}^*_{i+1},\ldots, \tilde{x}^*_{n+1}\right)$. Подстановка
$\tilde{x}_j\hm= \tilde{x}^*_j$ в~функцию~$\tilde{x}^\prime_i$, исключая
зависимость этой стратегии от переменной~$\tilde{x}_j$, определяет некоторую
стратегию $i$-го игрока~$\tilde{x}_i$ из класса~$\tilde{X}_i$. Поэтому в~ситуации
равновесия Нэша $\tilde{x}^*\hm\in \tilde{X}^*$ имеем
  \begin{multline*}
   \! \!\!\tilde{f}_i \left( \tilde{x}^*\right) =\max\limits_{\tilde{X}_i} \tilde{f}_i \left(
\tilde{x}^*_1, \ldots, \tilde{x}^*_{i-1},\tilde{x}_i, \tilde{x}^*_{i+1},\ldots,
\tilde{x}^*_{n+1}\right) ={}\hspace*{-0.3pt}\\
{}= \max\limits_{\tilde{X}^\prime_i} \tilde{f}_i\left(
\tilde{x}^\prime\right)
    \end{multline*}
и можно считать, что $\tilde{x}^\prime_i\hm= \tilde{x}^*_i$. Аналогичные равенства
справедливы для всех $j\not=i$. Значит, $\tilde{x}^*\hm\in \tilde{X}^{\prime *}$,
т.\,е.\  $\tilde{X}^*\subset \tilde{X}^{\prime *}$, что и~требовалось доказать.

\smallskip


    Следующий результат служит обобщением теоремы Цермело из теории
многошаговых антагонистических игр~\cite{17-vas}. Игру назовем игрой
с~\textit{полной информацией}, если в~ней одно из отношений предшествования
ходов  $\tilde{D}$ и~$D$ является полным.

    \smallskip

    \noindent
    \textbf{Теорема~2.3.} \textit{В игре с~полной информацией существует
ситуация равновесия.}

    \smallskip

    \noindent
    Д\,о\,к\,а\,з\,а\,т\,е\,л\,ь\,с\,т\,в\,о\,.\ \ Пусть для определенности $\tilde{D}$~---
полное отношение порядка. Воспользуемся методом динамического
программирования. По условию теоремы у игроков существуют \textit{абсолютно
оптимальные} стратегии $\tilde{x}_i^a\hm= x_i^a\left( \tilde{x}_1,\tilde{x}_2,\ldots,
\tilde{x}_{i-1}\right)$, которые являются решениями следующих задач:
    \begin{multline}
    \tilde{x}^a_i= \mathrm{Arg}\,\max\limits_{\tilde{x}_i} \tilde{f}_i
    \left( \tilde{x}_1,
    \tilde{x}_2,\ldots{}\right.\\
    \left.{}\ldots, \tilde{x}_{i}, x_{i+1} \left( \tilde{x}_1,
    \ldots\right.
    \tilde{x}_{i-1},\tilde{x}_i\right), \ldots\\
\left.\ldots, x_{n+1}\left( \tilde{x}_1, \ldots,
\tilde{x}_{i-1}, \tilde{x}_i, \tilde{x}_{i+1},\ldots, \tilde{x}_n\right)\right)\,.
    \label{e2.6-vas}
    \end{multline}
Тогда ситуация равновесия Нэша~--- это набор\linebreak функций:
\begin{multline*}
\tilde{x}^*= \left( \tilde{x}^a_1, \tilde{x}^a_2, \ldots, \tilde{x}^a_{n+1}\right)\,,\\
\tilde{x}^a_1\in X_1^0\,,\  \tilde{x}^a_1\equiv x_1^a\,,\ x_1^a\in X_1\,.
\end{multline*}
Проверка формулы~(\ref{e2.5-vas}) не вызывает затруднений. Ситуацию равновесия
$\tilde{x}^*\hm\in \tilde{X}^*$ можно вы\-чис\-лить, последовательно решая
задачи~(\ref{e2.6-vas}) при $i\hm= n, n-1, \ldots , 1$.

    Не исключено, что в~игре~(\ref{e2.1-vas}), (\ref{e2.2-vas}) множество ситуаций
равновесия пустое. Как показывают теоремы~2.2 и~2.3, за счет обмена
информацией можно расширить классы допустимых стратегий так, что в~новой игре
появятся равновесия $\tilde{X}^{\prime *}\not= \emptyset$. \mbox{В~статье}~\cite{13-vas}
рассмотрен численный метод поиска ситуации равновесия бескоалиционной игры
в~классе смешанных стра\-те\-гий-констант с~по\-мощью повторяющейся игры со
смешанными стра\-те\-ги\-ями-функ\-ци\-ями.

    Применение принципа равновесия ограничивается возможной
неэффективностью и~неэквивалентностью для игроков имеющихся ситуаций
равновесия. Эффективная ситуация равновесия вполне заслуживает названия
\textit{оптимального} решения игры.

\section{Существование эффективных ситуаций равновесия}

    Поставим задачу выделения классов стратегий игроков~(\ref{e2.2-vas}), в~которых существует это оптимальное решение.

    \subsection{Эффективные равновесия Нэша}

    Знание обстановки проведения операции позволяет игрокам самостоятельно,
без обмена сведениями, находить свои гарантирующие стратегии. Если все игроки
$i^\prime \hm\in I(i)$ знают стратегию $\tilde{x}_i$, которую выбирает игрок~$i$, то
они могут осуществить <<наказание>> партнера~$i$, применяя стратегию
\textit{наказания} $\tilde{x}^{\mathrm{н}i}\hm
    \equiv \tilde{x}^{\mathrm{н}i}_{I(i)}$~\cite{1-vas}:
$$
\tilde{x}^{\mathrm{н}i} \in \mathrm{Arg}\,\min\limits_{\tilde{x}_j, j\not=i}
\tilde{f}_i \left(
\tilde{x}_1, \ldots , \tilde{x}_{n+1}\right)\,,\enskip \tilde{x}^{\mathrm{н}i} \in
\prod\limits_{j\not= i} \tilde{X}_j\,.
$$

    \noindent
    \textbf{Пример~3.1.} В примере~2.1 первого игрока наказывают выбором
стратегий $\tilde{x}^{\mathrm{н}1}\hm= \left(\tilde{x}^{\mathrm{н}1}_2,
\tilde{x}^{\mathrm{н}1}_3\right)$, равных
    \begin{align*}
    \tilde{x}^{\mathrm{н}1}_2 &= x^{\mathrm{н}1}_2 (x_1,\tilde{x}_3) \in \mathrm{Arg}\,
\min\limits_{x_2} \tilde{f}_1(x_1,x_2,\tilde{x}_3)\,;\\
    \tilde{x}^{\mathrm{н}1}_3 &= x^{\mathrm{н}1}_3 (x_1,x_2) \in \mathrm{Arg}\,
\min\limits_{x_3} f_1(x_1,x_2,x_3)\,.
    \end{align*}

    \noindent
    \textbf{Теорема~3.1.} \textit{Пусть $f_i\hm>0$, $i\hm\in I$, а~диаграммы
$\tilde{D}$  и~$D\hm= \tilde{D}^{-1}$ отвечают полным порядкам на множестве~$I$.
Тогда существует эффективная ситуация равновесия Нэша}~$\tilde{x}^*$.

    \smallskip

    \noindent
    Д\,о\,к\,а\,з\,а\,т\,е\,л\,ь\,с\,т\,в\,о\,.\ \ В~допустимом классе
стратегий~$\tilde{X}_i$~(\ref{e2.2-vas}) определим наилучший гарантированный
результат $L_i\hm= L_i\left[ \tilde{X}_i\right]$ и~гарантирующую
стратегию~$\tilde{x}^\Gamma_i$ произвольного $i$-го игрока. По лемме~2.1
существует эффективная ситуация~--- элемент $x^{\mathrm{э}}\hm\in P_L$.
Проведем необходимое обоснование для случая $n\hm=3$ игроков, как не
ограничивающего общности. Тогда находимся в~условиях и~обозначениях из
примеров~2.1 и~3.1. Дополнительно определим функцию $\overline{x}:\ X_1\hm\to X$
вида $\overline{x}(x_1)\hm=s\left( x_1, \tilde{x}^{\mathrm{н}1}_2,
\tilde{x}^{\mathrm{н}1}_3\right)$ и~результат игры $\overline{\overline{x}}\hm=
s\left( \tilde{x}^{\mathrm{н}2}_1, \tilde{x}^{\mathrm{н}1}_2,
\tilde{x}^{\mathrm{н}2}_3\right)$ для указанной ситуации. Рассмотрим допустимые
стратегии игроков:
    \begin{align}
    \tilde{x}^*_1&=\left\{
    \begin{array}{l}
    x_1^{\mathrm{э}},\ \tilde{x}_j=x_j^*,\ j=2,3\,;\\[6pt]
    \overline{\overline{x}}_1, \ \tilde{x}_2=\tilde{x}^{\mathrm{н}1}_2\,;\\[6pt]
    \tilde{x}^{\mathrm{н}2}_1,\ \tilde{x}_2\not= \tilde{x}^*_2\,;\
\tilde{x}_3=\tilde{x}^*_3\,;\\[6pt]
    \tilde{x}^{\mathrm{н}3}_1,\ \tilde{x}_2=\tilde{x}^*_2,\ \tilde{x}_3\not=
\tilde{x}^*_3\,;
    \end{array}
    \right.
    \label{e3.1-1-vas}
\\
    \tilde{x}^*_2&=\left\{
    \begin{array}{l}
    x_2^{\mathrm{э}}, x_1=x_1^{\mathrm{э}},\  \tilde{x}_3=\tilde{x}^*_3\,;\\[6pt]
    \overline{x}_2(x_1), x_1\not= x_1^{\mathrm{э}},\
\tilde{x}_3=\tilde{x}^*_3\,;\\[6pt]
    \tilde{x}^{\mathrm{н}3}_2,\ \tilde{x}_3\not= \tilde{x}^*_3\,;
    \end{array}
    \right.
    \label{e3.1-2-vas}
   \\
    \tilde{x}^*_3&=\left\{
    \begin{array}{l}
    x_3^{\mathrm{э}}, x_1=x_1^{\mathrm{э}},\  x_2={x}^{\mathrm{э}}_2\,;\\[6pt]
    \tilde{x}^{\mathrm{н}1}_3,\ x_1\not= x_1^{\mathrm{э}}, x_1\not=
\overline{\overline{x}}_1,\ x_2=\overline{x}_2(x_1)\,;\\[6pt]
    \tilde{x}^{\mathrm{н}2}_3, x_1\not= x_1^{\mathrm{э}},\ x_1\not=
\overline{\overline{x}}_1,\ x_2\not= \overline{x}_2(x_1)\,;\\[6pt]
    \tilde{x}^{\mathrm{н}2}_3,\ x_1=\overline{\overline{x}}_1,\
x_2=\overline{\overline{x}}_2\,.
    \end{array}
    \right.
    \label{e3.1-3-vas}
    \end{align}

    Убедимся в~том, что выбор стратегий~(\ref{e3.1-1-vas})--(\ref{e3.1-3-vas})
обладает всеми доказываемыми свойствами $\tilde{x}^*\hm\in \tilde{X}^*\cap
\tilde{P}_L$. Во-пер\-вых, согласно~(\ref{e3.1-1-vas})--(\ref{e3.1-3-vas}),
справедливы неравенства:
\begin{multline*}
\tilde{f}_i(\tilde{x}^*_1,\ldots, \tilde{x}^*_{i-1}, \tilde{x}_i,
\tilde{x}^*_{i+1},\ldots, \tilde{x}_{n+1}^*)={}\\
{}=
\tilde{f}_i(\tilde{x}_1^{\mathrm{н}i}, \ldots, \tilde{x}_{i-1}^{\mathrm{н}i}, \tilde{x}_i,
\tilde{x}_{i+1}^{\mathrm{н}i},\ldots, \tilde{x}_{n+1}^{\mathrm{н}i})={}\\
{}=\min\limits_{\tilde{x}_j, j\not=i} \tilde{f}_i (\tilde{x}_1,\ldots,
\tilde{x}_{n+1})\leq
\max\limits_{\tilde{x}_i} \min\limits_{\tilde{x}_j, j\not=i} f_i(\tilde{x})=L_i
\leq{}\\
{}\leq
\tilde{f}_i(\tilde{x}^*)\,,\ i=1,2,\ldots, n\,.
\end{multline*}
Поэтому $\tilde{x}^*$ является равновесием Нэша. Во-вто\-рых,
$\tilde{f}(\tilde{x}^*) \hm= f(x^{\mathrm{э}}),$ $x^{\mathrm{э}}\hm\in P_L$. Значит,
стратегия $\tilde{x}^*\hm\in \tilde{P}_L$.

\smallskip

    \noindent
    \textbf{Теорема~3.2.} \textit{В~игре с~контролирующим игроком существует
эффективное равновесное решение.}

    \smallskip

    \noindent
    Д\,о\,к\,а\,з\,а\,т\,е\,л\,ь\,с\,т\,в\,о\,.\ \ По лемме~2.1 существует элемент
$x^{\mathrm{э}}\hm\in P_L$. В~условиях примера~2.2 контролирующий игрок
знает, кто из игроков $j\hm= 1,2,\ldots, n$ нарушает условие $\tilde{x}_j\hm=
\tilde{x}^*_j$, где $\tilde{x}^*_1,\ldots , \tilde{x}^*_n$~--- набор заранее
фиксированных стратегий игроков $k\hm= 1,2,\ldots, n$. Тогда он знает значение
параметра $\overline{x}_j\hm= \tilde{x}_j(j,x_j^{\mathrm{э}})$. Докажем, что искомым
решением является следующий выбор допустимых стратегий игроков:
    \begin{align*}
    \tilde{x}^*_{n+1} &= \left( \tilde{x}^{1*}_{n+1}, \ldots \tilde{x}^{n*}_{n+1}\right)\,;
\\
    \tilde{x}^{k*}_{n+1} &=
    x_{n+1}^{k*}\left( \tilde{x}_1,\ldots, \tilde{x}_n\right)={}\\
    &\hspace*{-5pt}{}= \left\{
    \begin{array}{l}
    (j,\overline{x}_j), (\exists !j)\tilde{x}_j\not= \tilde{x}^*_j,\ k\not=j\,;\\[6pt]
    (j,x_j^{\mathrm{э}}),\ k=j\,;\\[6pt]
    0, (\forall\ i) \tilde{x}_i=\tilde{x}^*_i\,,
    \end{array}
    \right.
    k=1,2,\ldots, n\,;
    \\
    \tilde{x}^*_i &= x_i^*(x^i_{n+1}) =x_i^*(j,x_j) ={}\\
    &{}=\left\{
    \begin{array}{l}
    x_i^{\mathrm{н}j}(x_j),\ i\not=j\,;\\[6pt]
    x_i^{\mathrm{э}},\ x_{n+1}^i=0\vee i=j\,,
    \end{array}
    \right.
    \hspace*{9mm}i=1,2,\ldots, n\,.
    \end{align*}
Игрок $n+1$ сообщает игрокам $k\not= j$ величину $x^k_{n+1}\hm=
(j,\overline{x}_j)$. Поэтому становится возможным наказание <<нарушителя>>
совместными действиями остальных игроков. В~результате будет реализован такой
исход игры~$x$, для которого $f_j(x)\hm\leq L_j\hm\leq \tilde{f}_j(\tilde{x}^*)\hm=
f_j(x^{\mathrm{э}})$.

  \subsection{Коалиционно устойчивое эффективное решение}

  В ситуации равновесия Нэша никому из игроков в~одиночку не выгодно
отклоняться от своего решения. За счет коллективных действий игроки вполне
могут рассчитывать на увеличение своих выигрышей. Изучен вопрос существования
ситуаций равновесия, которые нецелесообразно нарушать с~по\-мощью
формирования коалиций. Под кооперативной игрой понимаем игру, в~которой не
исключено образование коалиций.

    \subsubsection{Характеристическая функция кооперативной игры}

    У всякой коалиции $K\subset I$ имеется критерий $f_K(x)$, с~помощью
которого оценивается качество коалиционного решения~$x_K$, совместно
принимаемого игроками $k\hm\in K$. Рациональное поведение коалиции~$K$
заключается в~стремлении по возможности максимизировать коалиционную
функцию выигрыша $f_K(x)$. Обычно рассматривают функции~$f_K$ вида:
\begin{align}
f_K(x)&= \min\limits_{k\in K} f_k(x)\,;\label{e3.2-vas}\\
f_K^\prime(x) &= \sum\limits_{k\in K} f_k(x)\,.\label{e3.3-vas}
\end{align}
В отличие от~(\ref{e3.2-vas}), аддитивный критерий~(\ref{e3.3-vas}) отвечает
кооперативной игре с~неограниченными побочными платежами~\cite{1-vas}. Игроки
$k\hm\in K$ \textit{делят} между собой их суммарный выигрыш~(\ref{e3.3-vas}).
Если решение о~формировании коалиции принимается игроками в~некоторой
ситуации~$x^0$, для которой $f_k(x^0) \hm> 0$, то вместо~(\ref{e3.2-vas})
целесообразно использовать платежные функции
\begin{equation}
f_K^\prime(x^0,x) =\min\limits_{k\in K} \fr{f_k(x)}{f_k(x^0)}\,.
\label{e3.2-1-vas}
\end{equation}
Рациональное поведение коалиции~$K$ заключается в~выборе эффективной по
Парето стратегии $\tilde{x}^*_K(x_{I\backslash K})$ по векторному
критерию~$\tilde{f}$, если ей из\-вес\-тен выбор параметра $ x_{I\backslash K}$
\textit{дополняющей коалиции} $I\backslash K$~\cite{14-vas, 17-vas}.

    Введем характеристическую функцию игры $K\hm\to L_K$, $K\hm\in 2^I$,
которая каждой коалиции ставит в~соответствие ее наилучший гарантированный
результат в~классе стра\-те\-гий-констант~$X^0$. Для коалиционных
критериев~(\ref{e3.2-vas}) характеристическая функция игры равна
    \begin{equation}
    L_K=\max\limits_{x_K} \min\limits_{x_{I\backslash K}} \min\limits_{k\in K}
f_k(x)\,.
    \label{e3.4-vas}
    \end{equation}
При нормировке~(\ref{e3.2-1-vas}) вместо $f_k(x)$ в~(\ref{e3.4-vas}) следует взять
величину $f_k(x)/f_k(x^0)$.

    Кооперативная игра~(\ref{e3.3-vas}) имеет характеристическую функцию
    \begin{equation}
    L_K=\max\limits_{x_K} \min\limits_{x_{I\backslash K}} \sum\limits_{k\in K}
f_k(x)\,.
    \label{e3.5-vas}
    \end{equation}

    Пусть коалиция $K_1$, $K_1\subset K$ представлена игроком~K$_1$, которому
известен вектор $x_{K_2}$, выбира\-емый подкоалицией~$K_2$, $K_2\hm=
K\backslash K_1$, коалиции~$K$. Наилучшим \textit{условным} гарантированным
результатом коалиции~$K_1$ назовем величину $L_{K_1}(x_{K_2})$ наилучшего
гарантированного результата игрока~K$_1$.

    \smallskip

    \noindent
    \textbf{Лемма~3.1.} \textit{Характеристическая функция}~(\ref{e3.5-vas})
\textit{супераддитивна: $L_{K_1}\hm+ L_{K_2}\hm\leq L_K$,
     а~функция}~(\ref{e3.4-vas}) \textit{удовлетворяет неравенству}:
    \begin{multline*}
    \min \left\{ L_{K_1}, L_{K_2}\right\} \leq L_K \leq{}\\
    {}\leq \min \left\{ \max\limits_{x_{K_2}}
L_{K_1}(x_{K_2}), \max\limits_{x_{K_1}} L_{K_2}(x_{K_1})\right\}\,.
    \end{multline*}

    \smallskip

    Найдем наилучший $K$-\textit{коалиционный} результат $L_k(K)$ члена
коалиции~$k$, $k\hm\in K$. Он равен $L_k(K)\hm= \max\limits_{x_K}
\min\limits_{x_{I\backslash K}} f_k(x)$. Аналогичная величина, вычисляемая для
подкоалиции~$K^\prime$, $K^\prime \subset K$, равна $L_{K^\prime}(K) \hm=
\max\limits_{x_K} \min\limits_{x_{I\backslash K}} f_{K^\prime}(x)$ и~является
наилучшим \mbox{$K$-коа}\-ли\-ци\-он\-ным результатом. Введенные величины связаны
соотношениями
    \begin{align*}
    L_k&\equiv L_k(\{k\})\leq L_k(K)\,;\\
    L_{K^\prime} &\equiv L_{K^\prime}
(K^\prime) \leq L_{K^\prime} (K)\,.
    \end{align*}

    Функцию $K^\prime \to L_{K^\prime} (K)$, $K^\prime \subset K$, назовем
    \mbox{$K$-ха}\-рак\-те\-ри\-сти\-че\-ской. Для кооперативных игр~(\ref{e3.2-vas}),
(\ref{e3.3-vas}) $I$-ха\-рак\-те\-ри\-сти\-че\-ская функция совпадает с~обычной~---
(\ref{e3.4-vas}), (\ref{e3.5-vas}).

    \smallskip

    \noindent
    \textbf{Следствие~3.1.} Характеристическая функция~(\ref{e3.4-vas})
удовлетворяет неравенству:
$$
\min\limits_{k\in K} L_k\leq L_K\leq
\min\limits_{k\in K} L_k(K)\eqdelta \overline{L}_K\,.
$$

    \subsubsection{Коалиционная устойчивость результата игры}

    Большая выгодность коллективных действий по сравнению с~индивидуальным
поведением является причиной формирования коалиции. Распад коалиции
происходит тогда, когда она становится невыгодной хотя бы одному из ее членов.
Рас\-смот\-рим исходы игры $x^0\hm= s(\tilde{x})$, приводящие к~выходу некоторых
участников из коалиции~$K$.

    \smallskip

    \noindent
    \textbf{Определение~3.1.} $K$-устой\-чи\-вым назовем такой результат
игры~$x^0$, что $\exists\ k_0 \in K$ $f_{k_0}(x^0)\hm\geq L_{k_0}(K)$.

    \smallskip

    Другими словами, исход игры~$x^0$ устойчив относительно попытки
формирования коалиции~$K$. Результат игры $I$-устой\-чив, если~$x^0$ является
максимумом одной из платежных функций~(\ref{e2.1-vas}) $f_k:\ X\hm\to R$. Выйти
из полной коалиции~$I$ невозможно. Поэтому далее считаем $K\not= I$. В~случае
нормированного коалиционного критерия~(\ref{e3.2-1-vas}) в~определении~3.1
нужно потребовать выполнения формулы $\exists\ k_0 \hm\in K$
$L_{k_0}(K)\hm\leq 1$.

    Если взять стратегии из теорем~3.1 и~3.2, в~которые вместо
вектора~$x^{\mathrm{э}}$ подставлен вектор~$x^0$, то они равновесны по Нэшу и~вдобавок обеспечивают достижимость исхода~$x^0$. В~любом
равновесии~$\tilde{x}^*$ все игроки получают платежи, не меньшие их наилучших
гарантированных результатов $\tilde{f}_i(\tilde{x}^*)\hm\geq L_i[\tilde{X}_i]
\hm\geq L_i[X_i^0]\equiv L_i$. Благодаря определению~3.1, $K\hm= \{k_0\}$, имеем
$f_{k_0}(x^0)\hm\geq L_{k_0}(\{k_0\})\hm= L_{k_0}$, так что указанное
неравенство выполнено для исхода игры~$x^0$.

    Пусть дополняющая коалиция $I\backslash K$ имеет точную информацию о
векторе~$x_K$, выбираемом коалицией~$K$. Тогда при наличии
    $K$-устой\-чи\-во\-го \mbox{исхода} игры~$x^0$ выбором стратегии наказания
$\tilde{x}^{\mathrm{н}k}_{I\backslash K} \hm= X_{I\backslash
K}^{\mathrm{н}k}(x_K)$, $k\hm\in K$, она может противодействовать попытке
формирования коалиции~$K$. В~этом случае для некоторого игрока~$k_0$ более
привлекательным становится результат игры~$x^0$ по сравнению с~исходом $x\hm=
s(\tilde{x}_K, \tilde{x}_{I\backslash K}^{\mathrm{н}k})$:
    $$
    \exists\ k\in K \forall\ x_K \exists\ x_{I\backslash K}  f_k(x_K, x_{I\backslash
K})\leq f_k(x^0)\,.
    $$
Игрок $k_0$ не станет участвовать в~коалиции~$K$, поэтому она не сформируется.

    \smallskip

    \noindent
    \textbf{Определение~3.2.} Исход игры~$x^0$ назовем коалиционно
устойчивым, если он $K$-устой\-чив для любой коалиции $K\subset I$, $K\not=I$.

    \smallskip

    Коалиционно устойчивый результат игры~$x^0$ обладает свойством
    $$
    (\forall\ k) f_k(x^0)\geq L_k\,.
    $$
Он приводит к~изоляционизму поведения игроков, препятствуя созданию коалиций.

\smallskip

\noindent
\textbf{Следствие~3.2.}
\begin{enumerate}[1.]
\item  В игре~(1), (\ref{e3.2-vas}) необходимым условием коалиционной
устойчивости результата~$x^0$ является $(\forall\ K\in 2^I)\wedge (K\not= I)
(\exists\ k\in K) L_K\hm\leq f_k(x^0)$, а~достаточным условием служит выполнение
 $\forall\ K (K\in 2^I) \wedge (K\not= I)\Rightarrow L_K\hm\leq f_K(x^0)$.
    \item  Пусть все $K$-ха\-рак\-те\-ри\-сти\-че\-ские функции в~кооперативной
игре~(1), (\ref{e3.3-vas}), $K\subset I$,
супераддитивны. Тогда условие
$\forall\ K (K\in 2^I)\wedge (K\not=$\linebreak $\not= I)\Rightarrow L_K\hm\leq f_K(x^0)$ достаточно
для коалиционной устойчивости результата игры  .
    \end{enumerate}

    \noindent
    Д\,о\,к\,а\,з\,а\,т\,е\,л\,ь\,с\,т\,в\,о\,.\ \
    \begin{enumerate}[1.]
    \item Если $(\forall K) \overline{L}_K\hm\leq f_K(x^0)$, то по следствию~3.1 для
любой коалиции~$K$ найдется такой игрок $k_0\hm\in K$, что справедливо
неравенство
   \begin{multline*}
    L_{k_0}(K) =\min\limits_{k\in K} L_k(K) \eqdelta \overline{L}_K\leq f_K(x^0)
    \eqdelta{}\\
    {}\eqdelta
\min\limits_{k\in K} f_k(x^0)\leq f_{k_0}(x^0)\,.
    \end{multline*}
Достаточность доказана. По определению~3.2 исход игры~$x^0$ удовлетворяет
условию $(\forall\ K)\exists\ k_0\hm\in K f_{k_0}(x^0)\hm\geq L_{k_0}(K)$. Снова
воспользуемся следствием~3.1:
$$
f_{k_0}(x^0) \geq L_{k_0}(K)\geq \min\limits_{k\in K} L_k(K) \geq L_K\,.
$$
Необходимое условие доказано.
\item Будем рассуждать от противного. Следствие неверно, $\exists\ K \forall\ k\in K
f_k(x^0)\hm< L_k(K)$. Просуммируем эти неравенства и~воспользуемся
супераддитивностью функции $K^\prime\hm\to L_{K^\prime}(K)$:
\begin{multline*}
f_K(x^0) =\sum\limits_{k\in K} f_k(x^0) < \sum\limits_{k\in K} L_k(K) \leq{}\\
{}\leq
L_K(K)=L_K\,.
\end{multline*}
Полученное неравенство $f_K(x^0)\hm< L_K$ противоречит условию следствия.
\end{enumerate}

    Рассмотрим кооперативную игру из примера~2.2, в~которой контролирующий
игрок выбирает параметр из множества
   \begin{multline*}
    X_{n+1} =\mathop{\bigcup}\limits_{K:\ K\subset I, K\not= \emptyset} \left\{
    x_{n+1}=\left( x^1_{n+1},\ldots, x^n_{n+1}\right):\right.\\
    \left. (\forall\ i) x^i_{n+1} \in
K\times \{K\}\times X_K\right\}\cup 0\,,
    \end{multline*}
где $0\in R^n$, $X_K=\prod\limits_{k\in K} X_k$.

    \smallskip

    \noindent
    \textbf{Теорема 3.3.} \textit{Пусть в~игре}~(\ref{e2.1-vas}) \textit{с
участниками $k\hm= 1,2, \ldots, n$ имеется коалиционно устойчивый исход~$x^0$.
Тогда с~помощью контролирующего игрока $n\hm+1$ достигается коалиционно
устойчивое эффективное решение}.

    \noindent
    Д\,\,\,о\,к\,а\,з\,а\,т\,е\,л\,ь\,с\,т\,в\,о\,.\ \ Согласно определению~3.2 можно
считать, что $x^0\hm \in P_L$. Пусть выбор ситуации~$\tilde{x}^*$ приводит
к~исходу игры~$x^0$. Контролирующему игроку известно множество
<<нарушителей>> $K\hm= \{k\in I:\ \tilde{x}_k\not= \tilde{x}^*_k\}$ результата
игры~$x^0$. Согласно определению~3.2 он может найти того из участников
$k_0\hm= k_0(K)$ коалиции~$K$, выигрыш которого не меньше
    \mbox{$K$-коа}\-ли\-ци\-он\-но\-го результата: $L_{k_0}(K)\hm\leq f_{k_0}(x^0)$ Об
этом игроке~$k_0$ он может проинформировать всех участников конфликта. Кроме
того, игроку $n\hm+1$ известен вектор~$\overline{x}_K$ с~координатами
$\overline{x}_k\hm= \tilde{x}_k (k,K,x_K^0)$, $k\hm\in K$, который он может также
сообщить всем участникам дополняющей коалиции $k\hm\in I\backslash K$.
Проверим, что искомым решением игры является
    \begin{align*}
    \tilde{x}^*_{n+1}&= \left( \tilde{x}^{1*}_{n+1},\ldots, \tilde{x}^{n*}_{n+1}\right)\,;
\\
    \tilde{x}^{k*}_{n+1} &= x^{k*}_{n+1}\left( \tilde{x}_1, \ldots, \tilde{x}_n\right)
={}\\
&\hspace*{20mm}{}=\begin{cases}
    \left(k_0, K,\overline{x}_K\right)\,, &\ k\in I\backslash K\,;\\
    \left( k, K, x_K^0\right)\,,& k\in K\,;\\
    0\,, & K=\emptyset\,;
    \end{cases}
    \\
    \tilde{x}_i^*&= x_i^*\left( x^i_{n+1}\right) = {}\\
    &{}=\begin{cases}
    \begin{cases}
    x_i^{\mathrm{н}k_0} (x_K), &i\in I\backslash K;\\
    x_i^0\,, & i\in K,
    \end{cases} &\!\!\! x^i_{n+1}=(k,K,x_K);\\
    x_i^0\,, &\!\!\! x^i_{n+1}=0,
    \end{cases}\\
    & \hspace*{60mm}i\in I\,.
    \end{align*}
Если $K\not= \emptyset$, то дополняющая коалиция, выбирая стратегию наказания
участника коалиции~$k_0$, ограничивает его выигрыш величиной $L_{k_0}(K)$,
меньшей $f_{k_0} (x^0)$:
\begin{multline}
\tilde{f}_{k_0} \left( \tilde{x}_K, \tilde{x}^{\mathrm{н}k_0}_{I\backslash K} \right)
 =
f_{k_0} \left( s\left( \tilde{x}_K, \tilde{x}^{\mathrm{н}k_0}_{I\backslash K}
\right)\right) \leq {}\\
{}\leq\max\limits_{x_K} \min\limits_{x_{I\backslash K}} f_{k0}(x)\eqdelta
L_{k_0}(K) \leq f_{k_0}(x^0)\,.
\label{e3.6-vas}
\end{multline}
Выбор стратегии $\tilde{x}^*$ реализует выполнение неравенства~(\ref{e3.6-vas}),
препятствуя образованию коалиции. При этом $s(\tilde{x})^*\hm=x^0$. Теорема
доказана.

\smallskip

\noindent
\textbf{Пример~3.2.} Применим теоремы~3.2 и~3.3 к~игре, в~которой функции
выигрыша игроков имеют вид
\begin{align*}
f_1(x)&=x_1+x_2-x_3\,;\\
f_2(x) &= x_1-x_2+x_3\,;\\
f_3(x)&= -x_1+x_2+x_3\,;\\
X_k&=[0,\,1]\,,\enskip k=1,2,3\,.
\end{align*}
С помощью определения~3.2 убедимся в~коалиционной устойчивости эффективного
по Парето вектора $x^0\hm= (1,\,1,\,1)$. Это так, потому что характеристическая
функция~(\ref{e3.4-vas}) равна $L_{\{i\}}\hm= 0$, $i\hm= 1,2,3$, $L_{\{1,2\}}\hm=
L_{\{2,3\}} =1/2$, $L_{\{1,3\}}\hm=0$, а~выигрыши игроков $f_k(x^0)\hm= 1$, $k\hm=
1,2,3$. Другие эффективные ситуации равновесия Нэша, построенные по
теореме~3.2, не являются коалиционно устойчивыми, например решение, которое
строится по вектору $x^{\mathrm{э}} \hm= (0,\,1,\,1)$.

    Рассмотрим характеристическую функцию~(\ref{e3.5-vas}) $L^\prime_{\{i\}}\hm=
0$, $i\hm= 1,2,3$, $L^\prime_{\{i,j\}} \hm=2$, $i\not= j$. Исход $x^0\hm= (1,\,1,\,1)$
    по-преж\-не\-му коалиционно устойчив. С~помощью теоремы~3.3 можно
построить коалиционно устойчивое эффективное решение этой игры.


    \smallskip

    \noindent
    \textbf{Пример~3.3.} Функции выигрыша игроков имеют вид:
\begin{multline*}
    f_1=\sum\limits_{j=1}^n x_j,\ f_k=(-1) x_{k-1}+\sum\limits_{j\not=k-1} x_j\,,\\
k\in I\,,\ X_k =[0,1]\,,\ k=2,\ldots, n\,.
\end{multline*}
Рассмотрим оптимальный по Парето вектор $x^0\hm= (1,\ldots ,1)$ как исход
кооперативной игры с~характеристической функцией~(\ref{e3.4-vas}). Вычисления
показывают, что
\begin{align*}
f_K(x^0) &=\begin{cases}
n-1\,, & K\not= \{1\}\,;\\
n\,, & K= \{1\}\;
\end{cases}\\
\overline{L}_K&=\begin{cases}
\vert K\vert -1\,, & (\exists\ k\in K) k-1\in K\,;\\
\vert K\vert -2\,, & (\forall\ k\in K) k-1\in I\backslash K\,,
\end{cases}
\end{align*}
где $\vert K\vert$~--- число членов коалиции~$K$ и~$L_K\hm\leq \overline{L}_K$.
Исход игры~$x^0$ удовлетворяет достаточному условию коалиционной
устойчивости из следствия~3.2. Тогда по теореме~3.3 в~игре с~контролирующим
игроком существует коалиционно устойчивое эффективное решение.

\section{Условия формирования коалиций }

    Вступление в~некоторую коалицию является важной составляющей стратегии
поведения игрока. Всякую коалицию представляет единственный игрок~K,
имеющий критерий эффективности $f_K$ и~гарантирующую стратегию
$x_K^\Gamma\hm\in X_K^0$. За аксиому примем следующее положение,
являющееся необходимым условием формирования коалиции.

    \smallskip

    \noindent
    \textbf{Аксиома~4.1.} \textit{Если формируется коалиция~$K$, то она
удовлетворяет свойству}:
    \begin{equation}
    (\forall\ K^\prime) K^\prime   \subset K\Rightarrow L_{K^\prime} \leq
\min\limits_{x_{I\backslash K}} f_{K^\prime} \left( x_K^\Gamma, x_{I\backslash
K}\right)\,.
    \label{e4.1-vas}
    \end{equation}

Формула~(\ref{e4.1-vas}) запрещает распад создаваемой коалиции~$K$ на части
$K^\prime\not= K$. Формирование одноэлементной коалиции $K\hm= \{k\}$
равносильно изоляционизму в~поведении игрока~$k$ и~применению им своей
гарантирующей стратегии~$x_k^\Gamma$. В~этом случае игрока~$k$ можно
исключить из игры, зная выбор его параметра $x_k\hm= x_k^\Gamma$. Поэтому далее
будем считать, что одноэлементные коалиции не создаются.

    Как показывает неравенство~(\ref{e4.1-vas}), выбором гарантирующей
стратегии $x_K^\Gamma$ любая коалиция должна обеспечивать всем своим
участникам выигрыши, не меньшие их собственных наилучших гарантированных
результатов~$L_k$.

    Пусть игрок K$^\prime$ представляет коалицию~$K^\prime$ и~имеет критерий
эффективности вида~(\ref{e3.2-vas}) или~(\ref{e3.3-vas}). Сформулируем аксиому
\textit{поглощения}.

    \smallskip

    \noindent
    \textbf{Аксиома~4.2.} \textit{Если $\mathrm{K}^\prime\hm\in K$, $K\not= I$, то
формируется коалиция $\overline{K}\hm= K\backslash \left\{ \mathrm{K}^\prime\right\} \cup
K^\prime$.}

    \smallskip
    После поглощения коалиционные платежи и~наилучшие гарантированные
выигрыши сохраняются: $f_{\overline{K}}\hm= f_K$, $L_{\overline{K}}\hm= L_K$.

    \smallskip

    \noindent
    \textbf{Замечание~4.1.} В~игре с~побочными платежами (см.~\cite{1-vas})
всегда можно считать выполненными обе аксиомы~4.1 и~4.2. Это обеспечивается за счет
надлежащего дележа суммарного выигрыша~(\ref{e3.3-vas}).

    \smallskip

    \noindent
    \textbf{Лемма~4.1.} \textit{Если создается некоторая коалиция~$K$, то
существует \mbox{$K$-коа}\-ли\-ци\-он\-но устойчивый исход игры}.

    \noindent
    Д\,о\,к\,а\,з\,а\,т\,е\,л\,ь\,с\,т\,в\,о\,.\ \ По аксиоме~4.1 для произвольной
подкоалиции~$K^\prime$, $K^\prime\subset K$, имеем:
    $$
    L_{K^\prime}\leq \min\limits_{x_{ I\backslash K }}f_{K^\prime} \left(
x_K^\Gamma, x_{ I\backslash K} \right) \leq f_{K^\prime}\left( x_K^\Gamma, x^\Gamma_{
I\backslash K}\right)\,.
     $$
По следствию~3.3 исход игры $x^0\hm= \left( x_K^\Gamma, x^\Gamma_{I\backslash K}\right)$
является $K$-коа\-ли\-ци\-он\-но устойчивым.

    С игрой~(\ref{e2.1-vas}),  $\tilde{X}\hm= X^0$, естественно связать итерационный процесс
поиска коалиционно устойчивого исхода игры~(\ref{e2.1-vas}), (\ref{e2.2-vas}).
На каждом шаге процесса
при формировании коалиции~$K$ происходит замена~$K$ единственным
игроком~K. Игрок~K становится представителем коалиции~$\overline{K}$,
если $K^\prime \hm\in K$ и~игрок~K$^\prime$ представляет коалицию~$K^\prime$
(см.\ аксиому~4.2).

    \smallskip

    \noindent
    \textbf{Теорема~4.1.} \textit{Игровой операции}~(\ref{e2.1-vas}) \textit{с~множеством
игроков~$I$ можно сопоставить игру~$\overline{\Gamma}$, в~которой часть
игроков~$I$ объединены в~попарно непересекающиеся коалиции~$K_r$, $r\hm=
1,2,\ldots , R$. В~игре~$\overline{\Gamma}$ существует эффективный коалиционно
устойчивый исход}.

    \smallskip

    \noindent
    Д\,о\,к\,а\,з\,а\,т\,е\,л\,ь\,с\,т\,в\,о\,.\ \ Рассмотрим указанный итерационный
процесс. Возьмем $x^0\hm= \left( x_1^\Gamma,\ldots, x_n^\Gamma\right) \hm\in X^0$,
где множество
    $$
    X^0= \left\{ x:\ \forall\ k\in I^0 f_k(x)\geq L_k\right\}\,.
    $$
Если исход $x_0$ не является коалиционно устойчивым, то по следствию~3.2 в~игре
$\Gamma^0\hm= \Gamma$, $I^0\hm=I$, существует максимальная по включению
коалиция~$K_1$, для которой $L_{K_1}\hm> f_{K_1}(x^0)$. Построим
игру~$\Gamma^1$ с~множеством участников $I^1\hm= I\backslash K_1 \cup \{K_1\}$,
где игрок~K$_1$ представляет коалицию~$K_1$. В~игре~$\Gamma^1$, как и~выше,
рассмотрим множество исходов
$$
X^1= \left\{ x:\ \forall\ k\in I^1 f_k(x)\geq L_k\right\}\,.
$$
Согласно~(\ref{e4.1-vas}) вектор $x^1\hm= \left( x^\Gamma_{K_1}, x^0_{I\backslash
K_1}\right) \hm\in X^1$. Результат игры~$x^1$ либо коалиционно устойчив, либо это
не так. Тогда по следствию~3.2 найдется максимальная по включению
коалиция~$K_2$, для которой $L_{K_2}\hm> f_{K_2}(x^0)$. Действуя аналогично,
построим игры~$\Gamma^r$, $r\hm=1,2,\ldots$ Ввиду конечности множества
игроков этот процесс завершится на ка\-ком-то шаге~$R$. По построению
вектор~$x^R$ является коалиционно устойчивым исходом игры
$\overline{\Gamma}\hm= \Gamma^R$, причем $\overline{\Gamma}\hm= \Gamma$,
если $R\hm=0$. Если выбор~$x^R$ не эффективен, то найдется исход
игры~$\overline{x}$, для которого выполнено неравенство $f(\overline{x}) \hm\geq
f(x^R)$. Отсюда следует, что обе ситуации $\overline{x},x^R$ удовлетворяют
определению~3.2. Поэтому~$\overline{x}$ является искомым исходом игры.

\section{Заключение}

    Проведена аксиоматизация понятия стратегии. Игровая задача может решаться
в разных классах допустимых стратегий игроков, определяемых их
информированностью. В~тео\-ре\-ме~2.1 обоснована непротиворечивость процесса
принятия решений.

    Под рациональным поведением игроков понимается их стремление к~достижению либо эффективного по Парето решения, либо равновесия по Нэшу.
В~классе стра\-те\-гий-констант ситуация равновесия, вообще говоря, не
достижима, а~эффективное решение определено неоднозначно и,~вообще говоря, не
устойчиво. Даже в~играх с~полной информацией эти принципы не всегда
совместимы.
    В~тео\-ре\-ме~2.4 доказано, что при надлежащей информированности игроков
в играх существуют эффективные равновесия. По теореме~3.1 введение
контролирующего игрока упрощает информационные обмены, необходимые для
достижимости решения игры. Несмотря на выявленную структуру рационального
решения задачи, поиск решения сводится к~минимаксным задачам, требующим
привлечения серьезных вычислительных ресурсов~[18].

    В стратегию игроков входит возможность коалиционных действий.
Следствие~3.2, теоремы~3.3 и~4.1 посвящены изучению свойства коалиционной
устойчивости исхода игры. Эффективная коалиционно устойчивая ситуация игры
сохраняется при коллективных действиях игроков.

{\small\frenchspacing
 {%\baselineskip=10.8pt
 \addcontentsline{toc}{section}{References}
 \begin{thebibliography}{99}

\bibitem{2-vas} %1
\Au{Howard N.} Theory of meta-games~// General Systems,
1966. Vol.~XI. Р.~187--200.
\bibitem{3-vas} %2
\Au{Кукушкин Н.\,С.} Точки равновесия в~метаиграх~// Ж.~вычисл. матем. и~матем.
физ., 1974. Т.~14. №\,2. С.~312--320.
\bibitem{1-vas} %3
\Au{Гермейер Ю.\,Б.} Игры с~непротивоположными интересами.~--- М.: Наука, 1976.
326~с.
\bibitem{4-vas} %4
\Au{Кукушкин Н.\,С.} Роль взаимной информированности сторон в~играх двух лиц с~непротивоположными интересами~// Ж. вычисл. матем. и~матем. физ., 1972. Т.~8.
№\,4. С.~1029--1034.
\bibitem{8-vas} %5
\Au{Красовский Н.\,Н., Субботин А.\,И.} Позиционные дифференциальные игры.~---
М.: Наука, 1974. 458~с.
\bibitem{5-vas} %6
\Au{Горелик В.\,А., Горелов М.\,А., Кононенко~А.\,Ф.} Анализ конфликтных ситуаций
в системах управления.~--- М.: Радио и~связь, 1991. 286~с.
\bibitem{7-vas} %7
\Au{Петросян Л.\,А., Зенкевич Н.\,А., Семина~Е.\,А.} Теория игр.~--- М.: Высшая
школа, 1998. 304~с.
\bibitem{6-vas} %8
\Au{Жуковский В.\,И.} Кооперативные игры при неопределенности и~их
приложения.~--- М.: Эдиториал УРСС, 1999. 336~с.
\bibitem{9-vas}
\Au{Васин А.\,А., Морозов В.\,В.} Теория игр и~модели математической
экономики.~--- М.: МАКС Пресс, 2005. 272~с.
\bibitem{10-vas}
\Au{Колесник Г.\,В., Леонова Н.\,А.} Теория игр в~примерах и~задачах.~--- Тверь:
ТвГУ, 2012. 132~с.
\bibitem{11-vas}
\Au{Опойцев В.\,И.} Равновесие и~устойчивость в~моделях коллективного
поведения.~--- М.: Наука, 1977. 248~с.
\bibitem{12-vas}
\Au{Васильев Н.\,С.} Использование принципа равновесия для управления
маршрутизацией в~транспортных сетях~// Информатика и~её применения, 2014. Т.~8.
Вып.~1. С.~29--36.
\bibitem{13-vas}
\Au{Васильев Н.\,С.} Численное решение бескоалиционных матричных игр~// Наука
и образование: Электронное на\-уч\-но-тех\-ни\-че\-ское издание, 2013. №\,8. doi:
10.7463/0813.058774.
\bibitem{14-vas}
\Au{Подиновский В.\,В., Ногин В.\,Д.} Па\-ре\-то-оп\-ти\-маль\-ные решения
многокритериальных задач.~--- М.: Наука, 1982. 256~с.
\bibitem{15-vas}
\Au{Моисеев Н.\,Н.} Элементы теории оптимальных сис\-тем.~--- М.: Наука, 1975.
527~с.
\bibitem{16-vas}
\Au{Скорняков Л.\,А.} Элементы общей алгебры.~--- М.: Наука, 1983. 272~с.
\bibitem{17-vas}
\Au{Карманов В.\,Г., Федоров В.\,В.} Моделирование в~исследовании операций.~---
М.: Твема, 1996. 102~с.
\bibitem{18-vas}
\Au{Федоров В.\,В.} Численные методы максимина.~--- М.: Наука, 1979. 280~с.
 \end{thebibliography}

 }
 }

\end{multicols}

\vspace*{-3pt}

\hfill{\small\textit{Поступила в~редакцию 10.12.14}}

%\newpage

\vspace*{12pt}

\hrule

\vspace*{2pt}

\hrule

%\vspace*{12pt}

\def\tit{ON AVAILABILITY OF PARETO EFFECTIVE EQUILIBRIUM SITUATIONS
 IN~COLLECTIVE BEHAVIOR MODELS\\ WITH~DATA EXCHANGE}

\def\titkol{On availability of Pareto effective equilibrium situations in collective behavior models with data exchange}

\def\aut{N.\,S.~Vasilyev}

\def\autkol{N.\,S.~Vasilyev}

\titel{\tit}{\aut}{\autkol}{\titkol}

\index{Vasilyev N.\,S.}

\vspace*{-9pt}

\noindent
N.\,E.~Bauman Moscow State Technical University, 5 Baumanskaya 2nd Str.,
Moscow 105005, Russian Federation


\def\leftfootline{\small{\textbf{\thepage}
\hfill INFORMATIKA I EE PRIMENENIYA~--- INFORMATICS AND
APPLICATIONS\ \ \ 2015\ \ \ volume~9\ \ \ issue\ 2}
}%
 \def\rightfootline{\small{INFORMATIKA I EE PRIMENENIYA~---
INFORMATICS AND APPLICATIONS\ \ \ 2015\ \ \ volume~9\ \ \ issue\ 2
\hfill \textbf{\thepage}}}

\vspace*{3pt}


\Abste{Use of network technologies impels investigations of collective behavior
models. Processes of decision making based on data exchange are of utmost interest.
For this purpose, strategy axiomatization is proposed.
Information exchange diminishes uncertainty in the processes and models
collective efforts to achieve rational decisions. Rational behavior uses
the principles of effectiveness and stability usually contradicting one
another.
Rational game solutions' structure is studied. It is discovered that data
exchange allows achieving Pareto effective
situation which is also the
equilibrium one. A~notion of coalitional stable game issue is introduced.\linebreak\vspace*{-12pt}}

\Abstend{The situation prevents from forming coalitions and can simultaneously satisfy
the property of Pareto effectiveness. It can also give Nash equilibrium if
adequate players' strategies are used. An expansion of initial game by means
of additional controlling player shows how the effective coalitional stable
issue can be achieved.}

\KWE{game; strategy; situation; game issue; information exchange; dynamics of decision making; axiomatization; coalition; cooperative game; characteristic function; the best guaranteed result; strategy of punishment; Pareto effectiveness; Nash equilibrium}




\DOI{10.14357/19922264150201}

%\Ack
%\noindent


%\vspace*{3pt}

  \begin{multicols}{2}

\renewcommand{\bibname}{\protect\rmfamily References}
%\renewcommand{\bibname}{\large\protect\rm References}



{\small\frenchspacing
 {%\baselineskip=10.8pt
 \addcontentsline{toc}{section}{References}
 \begin{thebibliography}{99}



\bibitem{2-vas-1} %1
\Aue{Howard, N.} 1966. Theory of meta-games. \textit{General Systems}
11:187--200.
\bibitem{3-vas-1} %2
\Aue{Kukushkin, N.\,S.} 1974. Tochki ravnovesiya v metaigrakh [Equilibria points in
meta-games]. \textit{Zh. Vychisl. Mat. Mat. Fiz.} [Computational Mathematics and
Mathematical Physics] 14(2):312--320.
\bibitem{1-vas-1} %3
\Aue{Germeyer, Yu.\,B.} 1976. \textit{Igry s~neprotivopolozhnymi interesami}
[Games with nonantogonistic interests]. Moscow: Nauka. 326~p.
\bibitem{4-vas-1}
\Aue{Kukushkin, N.\,S.} 1972. Rol' vzaimnoy informirovannosti storon v~igrakh dvuh
lits s~neprotivopolozhnymi interesami [Role of partners mutual information in two
person games with nonantogonistic interests]. \textit{Zh. Vychisl. Mat. Mat. Fiz.}
[Computational Mathematics and Mathematical Physics] 8(4):1029--1034.
\bibitem{8-vas-1} %5
\Aue{Krasovskiy, N.\,N., and A.\,I.~Subbotin}. 1974. \textit{Pozitsionnye
differentsial'nye igry} [Positional differential games]. Moscow: Nauka. 458~p.
\bibitem{5-vas-1} %6
\Aue{Gorelik, V.\,A., M.\,A. Gorelov, and A.\,F.~Kononenko}. 1991. \textit{Analiz
konfliktnykh situatsiy v~sistemakh upravleniya} [Analysis of conflict situations in
control systems]. Moscow: Radio i svyaz'. 286~p.
\bibitem{7-vas-1} %7
\Aue{Petrosyan, L.\,A., N.\,A. Zenkevich, and E.\,A.~Semina}. 1998. \textit{Teoriya
igr} [Theory of games]. Moscow: Vysshaya Shkola. 304~p.
\bibitem{6-vas-1} %8
\Aue{Zhukovskiy, V.\,I.} 1999. \textit{Kooperativnye igry pri neopredelennosti i~ikh
prilozheniya} [Cooperative games under uncertainty and its applications]. Moscow:
Editorial URSS. 336~p.


\bibitem{9-vas-1} %9
\Aue{Vasin, A.\,A., and V.\,V. Morozov}. 2005. \textit{Teoriya igr i~modeli
matematicheskoy ekonomiki} [Theory of games and mathematical economics models].
Moscow: MAKS Press. 272~p.
\bibitem{10-vas-1}
\Aue{Kolesnik, G.\,V., and N.\,A. Leonova}. 2012. \textit{Teoriya igr v~primerakh
i~zadachakh} [Theory of games in examples and tasks]. Tver': Tver' Gos. Univ. 132~p.
\bibitem{11-vas-1}
\Aue{Opoycev, V.\,I.} 1977. \textit{Ravnovesie i~ustoychivost' v~modelyakh
kollektivnogo povedeniya} [Equilibrium and stability in collective behavior models].
Moscow: Nauka. 248~p.
\bibitem{12-vas-1}
\Aue{Vasil'ev, N.\,S.} 2014. Ispol'zovanie printsipa ravnovesiya dlya upravleniya
marshrutizatsiey v~transportnykh setyakh [Equilibrium principle application to routing
control in packet data-transmission networks]. \textit{Informatika i~ee
Primeneniya}~--- \textit{Inform. Appl} 8(1):29--36.
\bibitem{13-vas-1}
\Aue{Vasil'ev, N.\,S.} 2013. Chislennoe reshenie bes\-koa\-li\-tsi\-on\-nykh matrichnykh igr
[Numerical solution of matrix games without coalitions]. \textit{Nauka i~obrazovanie:
Elektronnoe nauchno-tehnicheskoe izdanie} 8. doi: 10.7463/0813.058774.
\bibitem{14-vas-1}
\Aue{Podinovskiy, V.\,V., and V.\,D. Nogin}. 1982. \textit{Pareto-optimal'nye
resheniya mnogokriterial'nykh zadach} [Pareto optimal solutions in multicriteria
problems]. Moscow: Nauka. 256~p.
\bibitem{15-vas-1}
\Aue{Moiseev, N.\,N.} 1975. \textit{Elementy teorii optimal'nykh sistem} [Elements of
optimal systems theory]. Moscow: Nauka. 527~p.
\bibitem{16-vas-1}
\Aue{Skornyakov, L.\,A.} 1983. \textit{Elementy obshchey algebry} [Elements of
general algebra]. Moscow: Nauka. 272~p.
\bibitem{17-vas-1}
\Aue{Karmanov, V.\,G., and V.\,V.~Fedorov}. 1996. \textit{Modelirovanie
v~issledovanii operatsiy} [Modelling in operations research]. Moscow: Tvema. 102~p.
\bibitem{18-vas-1}
\Aue{Fedorov, V.\,V.} 1979. \textit{Chislennye metody maksimina} [Numerical
methods of maximin]. Moscow: Nauka. 280~p.

\end{thebibliography}

 }
 }

\end{multicols}

\vspace*{-3pt}

\hfill{\small\textit{Received December 10, 2014}}

%\vspace*{-18pt}


     \Contrl

     \noindent
     \textbf{Vasilyev Nikolai S.} (b.\ 1952)~---
      Doctor of Science in physics and mathematics, professor,
N.\,E.~Bauman Moscow State Technical University, 5 Baumanskaya 2nd Str.,
Moscow 105005, Russian Federation; nik8519@yandex.ru

\label{end\stat}


\renewcommand{\bibname}{\protect\rm Литература}