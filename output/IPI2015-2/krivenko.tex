\def\stat{krivenko}

\def\tit{МОДЕЛИ ДЛЯ ПРЕДСТАВЛЕНИЯ И~ОБРАБОТКИ РЕФЕРЕНСНЫХ ЗНАЧЕНИЙ}

\def\titkol{Модели для представления и~обработки референсных значений}

\def\aut{М.\,П.~Кривенко$^1$}

\def\autkol{М.\,П.~Кривенко}

\titel{\tit}{\aut}{\autkol}{\titkol}

\index{Кривенко М.\,П.}

%{\renewcommand{\thefootnote}{\fnsymbol{footnote}} \footnotetext[1]
%{Работа выполнена при финансовой
%поддержке РФФИ (проект 15-07-02652).}}


\renewcommand{\thefootnote}{\arabic{footnote}}
\footnotetext[1]{Институт проблем информатики Федерального исследовательского
центра <<Информатика и~управление>> Российской академии наук,
mkrivenko@ipiran.ru}



      \Abst{Рассмотрена задача моделирования референсных значений~--- результатов
наблюдения определенного типа величин, полученных от одного лица или группы лиц, в~соответствии с~заявленным описанием. Для этих целей предлагается использовать смесь
нормальных распределений, которая может эффективно служить как средство
аппроксимации реальных данных и~при этом быть доступной с~точки зрения теоретического
анализа. При оценивании параметров компонентов смеси распределений главную роль
играет метод максимального правдоподобия и~его воплощение в~виде
EM (expectation-maximization)
ал\-го\-рит\-ма.
Для подбора числа компонентов смеси предлагается использовать критерий отношения
правдоподобия и~метод на основе расстояния между распределениями типа хи-квадрат. Их
свойства исследуются с~помощью бут\-стреп-ме\-то\-да. В~качестве экспериментов
рассматривается описание эмпирического распределения данных о~пациентах, включающих
возраст и~измерения PSA (Prostate-Specific Antigen). Предложенные решения имеют явные
преимущества: высокую детализацию по возрастам, сглаживание результатов наблюдений
для различных по объему возрастных групп, возможность формировать предположения
о~характере зависимости между возрастом и~уровнем PSA.}

      \KW{смесь нормальных распределений; оценивание числа компонентов смеси;
референсные значения}

\DOI{10.14357/19922264150208}

%\vspace*{6pt}


\vskip 14pt plus 9pt minus 6pt

\thispagestyle{headings}

\begin{multicols}{2}

\label{st\stat}

\section{Введение}

     Практика диагностической медицины основывается, в~частности, на
данных, собранных во время обследования пациентов и~при проведении
клинических и~дополнительных исследований. Они интерпретируются путем
сравнения с~ранее уста\-нов\-лен\-ны\-ми и~рекомендованными (принятыми для
имеющихся условий диагностирования) данными. Если состояние пациента
похоже на то, которое типично для конкретного заболевания, медицинский
работник может сформировать диагноз на основе только этих данных
(диагностика путем под\-тверж\-де\-ния). Этот диагноз становится более
правдоподобным, если наблюденные симптомы и~признаки не соответствуют
образцам, характеризующим набор альтернативных заболеваний (диагностика
путем исключения).

     Данные, соответствующие определенному состоянию пациентов,
называются референсными значениями. Сопутствующие понятия: референсная
популяция, референсные индивидуумы, референсный интервал.

     На языке задач классификации референсные значения есть обучающая
выборка, относящаяся к~определенному классу. Диагностирование пациента
соответствует задаче классификации, понимаемой как сопоставление данных
предопределенным классам. Если деление на классы отсутствует или является
нечетким, актуальной становится задача классификации как группирования.
В~литературе по распознаванию образов эти два значения термина
\textit{классификация} определяются как распознавание образов без учителя
(классификация как группирование) и~с учителем (классификация как
со\-по\-став\-ление).

     Базовой книгой, в~которой классификация рассматривается как
группирование, является~\cite{1-kri}. В~ней описывается методология
классификации, охватывающая сбор данных, выбор и~кодирование
специфических свойств, нахождение сходства, конструирование иерархических
групп, их формирование с~помощью статистических методов. В~[1, п.~7.3.2]
 также вводится понятие кластерного анализа как статистической
процедуры, распределяющей объекты по группам (кластерам).

     Постановка задачи кластерного анализа является крайне общей, и,~как
следствие, ее решение сопровождается рядом проблем:
     \begin{itemize}
\item формальные статистические оценки, такие как критерии значимости
или точные распределения, редки в~кластеризации; чаще приходится
полагаться на интуицию и~здравый смысл;
\item при решении прикладных задач трудно формально определить понятие
кластера, но при этом у людей присутствует достаточно твердое
представление о том, каков кластер. Соответственно, должны быть
доступны эффективные разведочные методы, которые представляют данные
различными способами так, чтобы пользователь мог сформировать
собственные суждения о присутствии и~конструкции кластеров;
\item существует много методов кластеризации, но явно недостаточно
способов выбора среди них. В~рамках статистического подхода далеки от
решений проблемы принятия вероятностных моделей, оценивания числа
кластеров, надежности принадлежности объектов кластерам;
\item существующие теории и~методы не всегда соответствуют масштабам
многих проблем клас\-те\-ри\-за\-ции. Обеспечивающие группирование данных
структуры данных и~алгоритмы работают на сотнях объектов, но часто
возникает необходимость в~системах классификации для тысяч
изображений, миллионов пациентов и~миллиардов текстов;
\item задача выбора системы признаков часто решается эвристическим
образом, поэтому при наличии избыточного числа признаков надо быть
готовым к~тому, что необходимая классификация может быть <<потеряна>>,
а~при неправильном выборе признаков может получаться классификация,
которая не интересна с~практической точки зрения.
\end{itemize}

     Кластерный анализ референсных значений оказывается эффективным
подходом не только при выявлении скрытых закономерностей в~реальных
данных, но и~при аппроксимации их распределения, служащего для
обеспечения классификации в~смысле сопоставления.

\section{Задачи и~типовые методы обработки референсных
значений}

     Согласно [2, гл.~5] и~исследованиям автора данной работы
можно выделить следующие группы задач статистической обработки
референсных значений:
     \begin{itemize}
\item разбиение (стратификация) референсных значений на подходящие
группы;
\item описание распределений каждой группы;
\item выделение выбросов (аномальных данных);
\item определение референсных пределов;
\item обновление (актуализация) информации о референсных значениях;
\item согласование референсных значений~\cite{3-kri, 4-kri};
\item представление данных и~диагностирование некоторого пациента с~помощью имеющихся наборов референсных значений.
\end{itemize}

     Обычно за основу при обработке референсных значений берутся
следующие модели:
     \begin{itemize}
\item параметрическая, нормальное распределение;
\item непараметрическая.
\end{itemize}

     Параметрический подход при формировании модели данных означает,
что априорные предположения об объекте анализа формулируются в~виде
параметрического семейства распределений. При описании референсных
значений вариант нормального распределения является наиболее
распространенным и~в то же время чаще всего критикуемым (см.,
например,~[5, п.~2.5.1]).

     Отказ от априорных предположений в~виде нормального распределения
приводит к~непараметрической модели, в~рамках которой никакой редукции
данных при обработке референсных значений не осуществляется, они сами
используются для построения статистического вывода.

     В данной работе для описания референсных значений предлагается
использовать модель смеси нормальных распределений. Она за счет
усложнения позволяет лучше описывать реальные распределения данных, но,
оставаясь параметрической моделью, обеспечивать значительное сжатие
информации об объекте анализа. При этом остается в~силе преимущество
относительно простых аналитических решений типовых вероятностных задач в~силу того, что основой по-прежнему остается нормальное распределение.

\section{Модель смеси нормальных распределений}

     Модель конечной смеси нормальных распределений (далее для
краткости~--- просто смеси распределений) означает, что плотность
распределения $f(u)$ представима в~виде:
     $$
     f(u)= \sum\limits_{j=1}^k p_j \varphi\left( u,\vartheta_j\right)\,.
     $$

     В этой модели неизвестными являются все или часть следующих
характеристик (параметров смеси): число~$k$ элементов смеси;
вероятности~$p_j$ появления элементов смеси (веса элементов смеси);
параметры~$\vartheta_j$ нормального распределения.

     Оценивание параметров смеси осуществляется с~помощью метода
максимального правдоподобия
     $
     L(k,p,\vartheta)\hm\to \max\limits_{k,p,\vartheta}$,
где $L(k,p,\vartheta)\equiv L(x,k,p,\vartheta) \hm= \prod\limits_{i=1}^n
\sum\limits_{j=1}^k p_j \varphi\left( x_i,\vartheta_j\right)$~--- функция
правдоподобия.

     Одна из главных трудностей при решении данной задачи связана с~оцениванием целочисленного параметра~$k$~--- числа элементов
анализируемой смеси. Дело в~том, что принцип максимального правдоподобия
сам по себе не дает решения задачи нахождения оценки для~$k$, имеющей
практический смысл. Поэтому обычно используется следующая схема
оценивания параметров смеси. Сначала задается верхняя граница для
возможного числа элементов смеси $k_{\max}$ и~находятся оценки
параметров~$p$ и~$\vartheta$ для последовательности отдельных значений\linebreak
$k\hm= 1,\ldots, k_{\max}$. Затем тем или иным способом, отличным от
максимизации правдоподобия, подбирается <<наилучшее>> значение
$\hat{k}\hm\in \left\{ 1,\ldots, k_{\max}\right\}$,\linebreak
 которое и~выступает в~качестве
оценки числа элемен\-тов смеси. Таким образом, далее при оценивании
параметров элементов смеси будет пред\-по\-лагаться, что число элементов смеси
$k$ является фиксированным, а~неизвестными параметрами остаются
только~$p$ и~$\vartheta$. Тогда согласно принципу максимального
правдоподобия центральным становится решение оптимизационной задачи
вида:

\vspace*{-3pt}

\noindent
     \begin{multline}
     \ln L(x,k,p,\vartheta) ={}\\
     {}= \sum\limits_{i=1}^n \ln \left( \sum\limits_{j=1}^k p_j
\varphi\left( x_i,\vartheta_j\right) \right) \to \max\limits_{p,\vartheta} \,.
     \label{e1-kri}
     \end{multline}

     \vspace*{-4pt}

     Наиболее работоспособную общую схему процедур, позволяющих
находить решения задачи~(\ref{e1-kri}), обычно называют EM-ал\-го\-рит\-мом.
Указанный алгоритм имеет достаточно богатую историю, изложение которой
дано в~[6, разд.~1.13 и~1.18; 7, введение; 8, разд.~1.8].

     Введем в~рассмотрение так называемые апостериорные
вероятности~$q_{ij}$ принадлежности наблюдения~$x_i$ к~$j$-му элементу
смеси. Если известны значения параметров~$p_j$ и~$\vartheta_j$, $j\hm=
1,\ldots ,k$, то при наблюденном значении~$x_i$, $i\hm= 1,\ldots, n$,
апостериорная вероятность принадлежности этого значения к~$j$-му элементу
смеси принимает вид:
     $$
     q_{ij}\left( x_i,p_j,\vartheta_j\right) = \fr{p_j\varphi\left(
x_i,\vartheta_j\right)}{\sum\nolimits_{l=1}^k p_l \varphi\left(x_i,\vartheta_l\right)}\,.
     $$

     Из этого определения следует, что для всех допустимых~$i$ и~$j$
$q_{ij}\hm\geq 0$ и~$\sum\nolimits_{j=1}^k q_{ij} \hm=1$.

     Теперь для нахождения оценки $\left( \hat{p},\hat{\vartheta}\right)$
па\-ра\-мет\-ров смеси распределений появляется возможность дать краткую запись
ЕМ-ал\-го\-рит\-ма, а именно:

\columnbreak
     \begin{enumerate}[(1)]
\item положить $t=0$ и~задаться начальными значениями $\left(
p^{(0)},\vartheta^{(0)}\right)$ для параметров смеси;
\item вычислить значения $q_{ij}^{(t)}\left( x_i, p_j^{(t)},
\vartheta_j^{(t)}\right)$ (\mbox{E-шаг});
\item определить значения $\left( p^{(t+1)},\vartheta^{(t+1)}\right)$ из
условия максимизации функции правдоподобия (\mbox{M-шаг});
\item положить $\left( \hat{p},\hat{\vartheta}\right)\hm= \left( p^{(t+1)},
\vartheta^{(t+1)}\right)$ и~завершить процесс нахождения оценки параметров
смеси, если построенные оценки $\left( p^{(t+1)},\vartheta^{(t+1)}\right)$
являются <<подходящими>>, или положить $t\hm= t\hm+1$ и~перейти к~шагу~2
 в~противном случае. %\hfill~$\blacksquare$
\end{enumerate}

     Применение EM-ал\-го\-рит\-ма в~описанном виде встречает как общие
для итерационных процедур трудности (не определены способ задания
начального приближения на первом шаге и~критерий завершения работы на
последнем шаге алгоритма), так и~порождает специфические проблемы
(устремление функции правдоподобия в~бесконечность при реализации
     М-ша\-га алгоритма).

     Далее кратко дадим возможные, апробированные на
практике варианты решения возникающих в~связи с~этим задач; подробнее
решение указанных вопросов рассмотрено, например, в~[9, п.~2.2.2].

     \paragraph*{Задание начальных приближений.} Для этого в~настоящее
время предложено большое число разнообразных алгоритмов, которые
группируются в~рамках следующих общих подходов:
\begin{itemize}
\item случайный выбор;
\item применение отличных от EM-ал\-го\-рит\-ма, обычно более <<быстрых>>
процедур кластерного анализа;
\item построение разбиения объектов на кластеры в~пространстве сниженной
размерности.
\end{itemize}

В~данной работе в~одномерном случае
применялось приближение на основе равновероятных элементов смеси,
а~в~многомерном случае~--- сначала оценивание параметров смеси для
одномерного случая (например, для одного из показателей или для первой
главной компоненты), а~потом с~по\-мощью матрицы апостериорных
вероятностей переход к~формированию параметров смеси в~многомерном
случае. С~позиций качества получаемых итоговых оценок параметров
элементов смеси эти подходы оказываются практически идентичными. Кроме
того, применялся прием оценивания параметров смеси от большего числа
элементов смеси к~меньшему с~исключением отдельных компонентов смеси
при формировании начальных приближений для оценки с~меньшим
количеством.

     \paragraph*{Завершение работы EM-алгоритма.} Обычным при\-емом
при задании критерия окончания итерационных алгоритмов является сравнение
значений переменных, получающихся на последующих шагах поиска решения
задачи. Сравнение аргументов поиска экстремума функции правдоподобия
в~рассматриваемом случае~--- достаточно сложная задача хотя бы по той причине,
что в~нем должны участвовать разнотипные переменные (веса, средние и~ковариационные матрицы). При этом решение необходимо принимать простое:
продолжать или не продолжать итерационный процесс. Поэтому в~качестве
контролируемой величины обычно берется значение максимума функции
правдоподобия. Несмотря на недостатки этого подхода, обычно используется
следующее правило: как только относительное приращение максимального
значения функции правдоподобия на последовательных шагах итерационного
процесса становится меньше заданного порога, процесс оценивания
прекращается.

\vspace*{-6pt}

     \paragraph*{Реализация M-шага при нарушении условия
ограниченности функции правдоподобия.} Существенным недостатком
     EM-ал\-го\-рит\-ма является возможность обработки только ограниченной
по величине функции правдоподобия. Но для реальных данных отдельные
кластеры могут оказаться вырожденными (либо из-за их дискретности, либо
из-за того, что размерность параметрического пространства оказывается меньше
принятой для совокупности клас\-те\-ров), что противоречит указанному
требованию для применимости EM-ал\-го\-рит\-ма. Это приводит\linebreak
 к~необходимости использования уточненной (невырож\-ден\-ной) модели смеси
нормальных многомерных распределений, в~которой введены ограничения на
множество возможных значений ковариацион\-ных матриц элементов смеси (в
одномерном случае это ограничение снизу на значение дисперсии). Задание
ограничения снизу на собственные значение ковариационных матриц
элементов смеси необходимо для предотвращения устремления функции
правдоподобия в~бесконечность и~потери (превышения) порядка при
выполнении операций с~числами с~плавающей точкой, но оно требует задание
этого ограничения. С~одной стороны, значение должно быть достаточно
большим, чтобы обеспечить выполнение указанных условий, с~другой стороны,
неразумное увеличение этого значения может привести к~снижению качества
кластерного анализа (например, слишком большие значения приведут к~потере
индивидуальности кластеров).

     Охарактеризуем возможности модели смеси распределений в~соответствии с~приведенным ранее списком задач статистической обработки
референсных значений:
     \begin{itemize}
\item смеси распределений являются одной из базовых моделей кластерного
анализа данных, что в~данном случае составляет содержание задачи
разбиения (стратификации) референсных значений на подходящие группы.
В~этой ситуации смесь используется не только для аппроксимации
распределения реальных данных, но и~для представления кластерной
структуры этих данных. Понятно, что интерпретация кластеров является
самостоятельной задачей и~решается в~основном специалистами в~предметной об\-ласти;\\[-9pt]
\item выделение выбросов, а~также определение референсных пределов
может осуществляться \mbox{путем} решения задачи байесовской классификации
данных, причем не только в~случае одномерных референсных значений;\\[-9pt]
\item обновление (актуализация) информации о~референсных значениях
требует наличия всех ранее сформированных данных и~решается путем
переоценки параметров смеси, что при наличии современных средств
вычислительной техники не должно приводить к~ка\-ким-ли\-бо проб\-лемам;
{\looseness=1

}

\item согласование референсных значений на базе смеси распределений
пока сопряжено с~аналитическими трудностями, но вполне успешно может
быть реализовано с~помощью не\-па\-ра\-мет\-ри\-че\-ских методов анализа;\\[-9pt]
\item  диагностирование с~по\-мощью имеющихся наборов
референсных значений решается с~по\-мощью байесовской классификации.
\end{itemize}

     Новые возможности открываются в~задаче разбиения референсных
значений на группы при ис\-поль\-зо\-ва\-нии многомерных (в~частности,
двухмерных) смесей распределения.

Далее наряду
 с~общим случаем,
когда в~записи нормального распределения~$\mu_j$ означает вектор средних,
а~$\Sigma_j$~--- ковариационную матрицу, интерес будут пред\-став\-лять частные
случаи одномерных данных, когда $\vartheta_j\hm= (\mu_j, \sigma_j^2)$,
а~также двухмерных, когда $\vartheta_j\hm= (\mu_{j1}, \mu_{j2}, \sigma_{j1}^2,
\sigma_{j2}^2, \rho_j)$.

     В двухмерном случае плотность нормального распределения имеет вид:
     \begin{multline}
     \varphi\left( x_1,x_2,\left( \mu_{j1}, \mu_{j2}, \sigma_{j1}^2, \sigma_{j2}^2,
\rho_j\right) \right) =\fr{1}{2\pi\sigma_{j1}\sigma_{j2}}\times{}\\
%\hspace*{-1mm}
{} \times\fr{1}{\sqrt{1-\rho_j^2}}\exp  \left\{\! -
\fr{u_{j1}^2 -2\rho_j u_{j1} u_{j2} +u_{j2}^2}{2(1-\rho_j^2)}\right\}\!,\!\!\!
     \label{e2-kri}
     \end{multline}
где $u_{j1}=(x_1-\mu_{j1})/\sigma_{j1}$, $u_{j2}= (x_2-\mu_{j2})/\sigma_{j2}$
и~параметры в~общей и~частной записи нормального распределения связаны как
$$
\Sigma_j= \begin{pmatrix}
\sigma_{j1}^2 & \sigma_{j1} \sigma_{j2} \rho_j\\[3pt]
\sigma_{j1} \sigma_{j2}\rho_j & \sigma_{j2}^2
\end{pmatrix}\,.
$$

Совместное распределение можно разложить на произведение
условного и~безусловного распределений, каждое из которых является также
нормальным, а~именно:
\begin{multline}
\varphi\left( x_1,x_2, \left( \mu_{j1}, \mu_{j2}, \sigma_{j1}^2,
\sigma_{j2}^2,\rho_j\right)\right)={}\\
{}= \varphi \left( x_1\vert x_2, \tilde{\mu}_j,
\tilde{\sigma}_j^2\right) \varphi\left( x_2, \mu_{j2},\sigma_{j2}^2\right)\,,
\label{e3-kri}
\end{multline}
где
\begin{align}
\tilde{\mu}_j &= \mu_{j1} +\rho_j \sigma_{j1} \fr{x_2-
\mu_{j2}}{\sigma_{j2}}\,;\label{e4-kri}\\
\tilde{\sigma}_j^2 &= \sigma^2_{j1} \left( 1-\rho_j^2\right)\,.\label{e5-kri}
\end{align}
Вернемся к~описанию смеси двухмерных нормальных распределений
\begin{multline}
f(u_1,u_2)={}\\
{}=\sum\limits_{j=1}^k p_j \varphi\left( u_1, u_2, \left( \mu_{j1},
\mu_{j2}, \sigma_{j1}^2, \sigma_{j2}^2, \rho_j\right)\right)\,.
\label{e6-kri}
\end{multline}
Оно может стать источником информации о распределении одного показателя,
например описываемого с~помощью переменной~$u_1$, при фиксированном
значении другого (переменная~$u_2$) Для этого наряду с~(\ref{e2-kri}) надо
иметь частное распределение переменной~$u_2$. Непосредственное
интегрирование~(\ref{e6-kri}) дает
$$
f(u_2) = \sum\limits_{j=1}^k p_j \varphi\left( u_2,\left( \mu_{j2},
\sigma_{j2}^2\right) \right)\,.
$$

     Теперь из~(\ref{e2-kri}) с~учетом~(\ref{e3-kri}) получаем:
     \begin{multline*}
     f(u_1\vert u_2) = \fr{f(u_1,u_2)}{f(u_2)}=  {}\\
     {}=\fr{\sum\nolimits_{j=1}^k p_j \varphi
\left(u_1\vert u_2, \tilde{\mu}_j, \tilde{\sigma}_j^2\right) \varphi \left(u_2,
\mu_{j2},\sigma_{j2}^2\right)} {\sum\nolimits_{l=1}^k p_l \varphi\left( u_2, \left(\mu_{l2},
\sigma_{l2}^2\right)\right)}= {}\\
{}= \sum\limits_{j=1}^k \tilde{p}_j \varphi\left( u_1\vert
u_2, \tilde{\mu}_j, \tilde{\sigma}_j^2\right)\,,
     \end{multline*}
где
\begin{equation}
\tilde{p}_j= \fr{p_j \varphi \left(u_2,\mu_{j2}, \sigma_{j2}^2\right)} {\sum\nolimits_{l=1}^k p_l
\varphi \left( u_2, \left( \mu_{l2},\sigma^2_{l2}\right)\right)}\,.
\label{e7-kri}
\end{equation}
Таким образом, доказано, что условное распределение для смеси~(\ref{e6-kri})
есть опять же смесь нормальных распределений с~параметрами,
определяемыми~(\ref{e7-kri}), (\ref{e4-kri}) и~(\ref{e5-kri}).

\section{Оценивание числа элементов смеси}

     Нахождение подходящего числа элементов смеси~$\hat{k}$ является
важной, но крайне сложной задачей; ее обычно относят к~группе задач выбора
модели (определенная модель соответствует некоторому значению~$k$).

В~арсенале соответствующих инструментов анализа данных можно выделить
сле\-ду\-ющие:
     \begin{itemize}
\item графический анализ;
\item информационные критерии;
\item множественная проверка статистических гипотез.
\end{itemize}

\vspace*{-3pt}

     \paragraph*{Графический анализ.} Несмотря на то что принцип
максимального правдоподобия сам по себе не позволяет получить решение
задачи оценивания чис\-ла элементов смеси, информацию о его реальном
<<подходящем>> значении из поведения максимума функции правдоподобия
получить все же можно. Обычно исследуемая функция <<существенно
изменяется>> лишь до некоторого значения~$\hat{k}$, которое и~можно
принять за оценку искомого числа элементов смеси. Попытка формализации
понятия\linebreak <<существенно изменяется>> приводит к~не\-об\-хо\-ди\-мости указать
некоторое конкретное значение порога, при этом качество принятого решения
должно оцениваться с~по\-мощью некоторого внешнего\linebreak
 показателя, например с~помощью вероятности ошибки классификатора, для которого при
аппроксимации распределений элементов классов принимается модель смеси
распределений.

\vspace*{-3pt}

     \paragraph*{Информационные критерии.} Дальнейшим развитием
приема анализа степени изменения $\max\limits_{p,\vartheta} L(k,p,\vartheta)$
являются идеи подбора модели с~помощью некоторого показателя (меры)
адекватности модели, являющегося компромиссом между приспособленностью
модели и~ее сложностью. Перебирая различные варианты модели (различные
значения~$k$), отдают предпочтение тому из них, которому соответствует
экстремум выбранного показателя.

     Сложность модели смеси нормальных $d$-мер\-ных распределений
характеризуется количеством параметров модели, а~именно:
     $$
     v(k) =(k-1)+k\left[ d+\fr{d(d+1)}{2}\right]\,.
     $$

     Рассмотрим далее критерии, связанные с~различными мерами
адекватности модели.

\pagebreak

     Информационный критерий Акаике (Akaike) основывается на
асимптотической теории распределения функции правдоподобия (см.~[6,
п.~6.8.2]) и~определяется показателем сле\-ду\-юще\-го вида:

\noindent
     $$
     \mathrm{AIC}(k) =-2\ln L \left( k,\hat{p}, \hat{\vartheta}\right) +2v(k)\,.
     $$

     Байесовский критерий основывается на аппроксимации интегрального
правдоподобия (см.~[6, п.~6.9.3]) и~определяется показателем
следу\-юще\-го вида:
     $$
     \mathrm{BIC} (k) =-2\ln L \left( k,\hat{p},\hat{\vartheta}\right) +v(k)\ln n\,.
     $$

     Классификационные критерии основываются на предположении, что
модель смеси описывает отделяемые друг от друга кластеры. Представим
логарифм функции правдоподобия как
     $$
     \ln L\left( k,p,\vartheta\right) =C(k,p,\vartheta)+E(k,p,\vartheta)\,,
     $$
где

\vspace*{-3pt}

\noindent
\begin{align*}
C(k,p,\vartheta) &= \sum\limits_{j=1}^k \sum\limits_{i=1}^n q_{ij} \ln p_j \varphi
\left( x_i,\vartheta_j\right)\,;\\
E(k,p,\vartheta) &= -\sum\limits_{j=1}^k \sum\limits_{i=1}^n q_{ij} \ln q_{ij}\,,\
\ E(k,p,\vartheta)\geq 0\,.
\end{align*}
     Тогда получаем:
     \begin{equation}
     C(k,p,\vartheta) =\ln L(k,p,\vartheta) -E(k,p,\vartheta)\,.
     \label{e8-kri}
     \end{equation}

     Если компоненты смеси хорошо отделимы, то матрица апостериорных
вероятностей $Q\hm= (q_{ij})$ определяет разбиение элементов $x_1,\ldots ,
x_n$ и~$E(k,p,\vartheta)\hm\approx 0$. В~противном случае $E(k,p,\vartheta)$
принимает большие значения. Как следствие, энтропия классификационной
матрицы~$Q$ может нести информацию о числе элементов смеси. Заметим,
что уравнение~(\ref{e8-kri}) показывает, что классификационное правдоподобие
$C(k,p,\vartheta)$ может рассматриваться как компромисс между степенью
соответствия модели смеси данным, измеренной с~помощью логарифма
правдоподобия $\ln L (k,p,\vartheta)$, и~способностью модели смеси порождать
разбиение данных, измеренной с~помощью энтропии классификации
$E(k,p,\vartheta)$. Поэтому основой критерия может стать классификационное
правдоподобие в~виде $C(k,\hat{p},\hat{\vartheta})$. Аппроксимация
интегрального классификационного правдоподобия в~духе байесовского
критерия (см.~[6, п.~6.11.4]) приводит к~сле\-ду\-юще\-му показателю:
     $$
     \mathrm{AWE}\,(k) =-2C\left( k,\hat{p},\hat{\vartheta}\right) +2v(k) (1{,}5+\ln n)\,.
     $$

     Из комментариев в~перечисленных разделах~\cite{6-kri} и~опыта автора
данной работы (см., в~частности,~[10,\linebreak\vspace*{-12pt}

\columnbreak

\noindent
 разд.~6.3]) следует, что
перечисленные критерии скорее иллюстрируют наличие кластерной структуры
в данных, нежели дают оценку для числа клас\-те\-ров~--- числа элементов смеси.

\vspace*{-6pt}

     \paragraph*{Множественная проверка статистических гипотез.}
Придать строгость задаче выбора числа элементов смеси можно, если
сформулировать ее на языке проверки гипотез. При этом возникает набор
отдельных задач проверки частных гипотез, связанных тем или иным образом с~конкретным значением числа элементов смеси, и~задача объединения
отдельных решений в~одно общее. В~совокупности это приводит к~проблеме
множественной проверки статистических гипотез.

     Проблему обоснования множественного выбора подходящего
целочисленного значения для оценки~$\hat{k}$ можно заменить выбором
одного из решений более простых задач сравнения двух предположений: число
элементов смеси равно~$k_1$ или~$k_2$, где $k_1\hm <k_2$. Так как при анализе
данных исследователь чаще всего следует принципу минимизации размерности
параметрического пространства, последнюю задачу упрощают и~рассматривают
случай $k_2\hm= k_1+1$. При этом путем последовательного (начиная со
значения $k_1\hm= 1$) перебора предположений ищется такое минимальное число
элементов смеси, более которого не имеет смысл усложнять модель.

     Если $H_k$~--- гипотеза о том, что число элементов смеси равно~$k$, то
задачу оценивания числа элементов смеси будем решать путем анализа
последовательности альтернатив~$A_k$, включающих нулевую
гипотезу~$H_k$ и~конкурирующую гипотезу $H_{k+1}$. В~качестве
оценки~$\hat{k}$ возьмем такое минимальное значение~$k$, для которого при
анализе альтернатив $A_1,\ldots, A_{k-1}$ предпочтение было отдано
конкурирующей гипотезе, а при анализе~$A_k$~--- нулевой гипотезе.

     В рамках принципа максимального правдоподобия для обоснования
выбора частного решения естественно обратиться к~критерию отношения
правдоподобия, статистика которого удобно представить в~виде:
     $$
     s_k= -2\ln \fr{\max\limits_{p,\vartheta} L(k,p,\vartheta)}
{\max\limits_{p,\vartheta} L(k+1,p,\vartheta)}\,.
     $$
     Тогда нулевой гипотезе будет отдаваться предпочтение при малых
значениях этой статистики, а~конкурирующей гипотезе~--- при больших.

     К сожалению, даже в~случае нормального распределения элементов смеси
не выполняются\linebreak условия, при которых статистика критерия отношения
правдоподобия в~подходящем виде имеет привычное
асимптотическое~$\chi^2$-рас\-пре\-де\-ле\-ние с~чис\-лом степеней свободы,
равным разнице в~чис\-ле па\-ра\-мет\-ров при двух конкурирующих гипотезах.
Имеются многочисленные примеры (см., например,~\cite{11-kri}), в~которых
демонстрируется существенная зависимость поведения критерия отношения\linebreak
правдоподобия от способов реализации EM-ал\-горитма (например, от
различных стратегий начальных приближений и~правил остановки
итерационного процесса %\linebreak
 нахождения оценок параметров смесей). Име\-ющи\-еся
попытки аппроксимации распределения статистики критерия отношения
правдоподобия обосно\-вы\-ва\-ют\-ся общими соображениями или методом
моделирования в~частных случаях (в аккумулированном виде данный материал
представлен в~[6, разд.~6.5]).

В~создавшейся ситуации приходится
прибегать к~процедурам, в~основе применения которых лежит использование
бут\-стреп-ме\-то\-да. Рассмотрим одну из них: параметрическую процедуру
оценки качества принятого решения о числе элементов смеси.

     Для реализации слабого критерия значимости необходимо знать
распределение $F_{S_k\vert H_k}$ статистики~$S_k$ критерия при условии
нулевой гипотезы~$H_k$. Тогда в~терминах критического уровня зна\-чи\-мости
решения будут приниматься по величине $\alpha_k^*\hm= 1-F_{S_k\vert
H_k}(s_k^*)$, где $s_k^*$~--- наблюденное значение статистики критерия.

     Построить теоретически распределение $F_{S_k\vert H_k}$ вряд ли
возможно, поэтому необходимо прибегнуть к~бут\-стреп-ме\-то\-ду. Это
означает, что вместо $\alpha_k^*\hm= 1-F_{S_k\vert H_k}(s_k^*)$
рассматривается бут\-стреп-оцен\-ка
     $$
     \alpha_k^B=1-F^B_{S_k\vert H_k}(s_k^*)\,,
     $$
где $F^B_{S_k\vert H_k}$~--- бут\-стреп-рас\-пре\-де\-ление.

     В рассматриваемом случае бут\-стреп-рас\-пре\-де\-ле\-ние
$F^B_{S_k\vert H_k}$ может быть получено только методом моделирования,
схема которого при фиксированном значении~$k$ такова:
     \begin{itemize}
\item нахождение оценок $\hat{p}(k)$ и~$\hat{\vartheta}(k)$
параметров~$p$ и~$\vartheta$ по исходной выборке при условии, что число элементов смеси
равно~$k$;
\item генерация нужного количества наблюденных значений статистики
критерия отношения правдоподобия, состоящая из шага генерации выборки
из смеси распределений с~параметрами~$\hat{p}(k)$ и~$\hat{\vartheta}(k)$ и~шага
подсчета значения статистики~$s_k$ (включающего опять же оценивание
параметров~$p$ и~$\vartheta$, но уже по сгенерированной выборке и~для двух
различных значений числа элементов смеси, а именно: $k$ и~$k\hm+1$);
\item нахождение бут\-стреп-оцен\-ки~$\alpha_k^B$.
\end{itemize}

     Описанный критерий отношения прав\-до\-по\-добия воплощал идею
графических методов последовательного анализа значений функции
правдоподобия. В~качестве дополнительного в~данной \mbox{работе} предлагается
подход, который прямо следует из цели привлечения модели смеси
распределений для описания имеющихся референсных значений. Пусть
референсные значения отвечают некоторому распределению, которое
аппроксимируется смесью $f_k(u)$. Тогда возникает задача проверки
множества нулевых гипотез $\tilde{H}_k$ о том, что распределение данных
есть $f_k(u)$, при конкурирующей гипотезе, что оно иное. Для этого можно
использовать, в~принципе, любой критерий, основанный на расстоянии между
распределениями. Далее будет рассматриваться статистика~$\chi^2$, которая
контролирует согласованность гипотетических вероятностей $p_l(\tilde{H}_k)$
и~эмпирических частот~$v_l$ попадания в~$l$-ю ячейку (разряд, класс), а
именно:
     $$
     \chi^2\left( \tilde{H}_k\right) = n\sum\limits_l \fr{(v_l-p_l(\tilde{H}_k))^2}
{p_l(\tilde{H}_k)}\,.
     $$

Гипотетические вероятности $ p_l(\tilde{H}_k)$ зависят от ряда параметров,
характеризующих смесь из~$k$ элементов, поэтому вместо них подставляются
оценки, найденные с~помощью EM-ал\-го\-рит\-ма. Таким образом может быть
подсчитано значение $\hat{\chi}^2(\tilde{H}_k)$. Нулевая гипотеза будет
приниматься при малых значениях статистики  $\hat{\chi}^2(\tilde{H}_k)$,
а~отвергаться при больших.

     Принятие решений основывается на величине
 $$
 \alpha (\hat{\chi}^2(\tilde{H}_k)=1- F_{\chi^2(\tilde{H}_k)\vert \tilde{H}_k}
 (\hat{\chi}^2(\tilde{H}_k))\,.
 $$

 При использовании бут\-стреп-ме\-то\-да вместо
неизвестного распределения статистики критерия
$F_{\chi^2(\tilde{H}_k)\vert\tilde{H}_k}(\hat{\chi}^2(\tilde{H}_k))$
рассматривается его бут\-стреп-ана\-лог, строящийся путем статистических
испытаний. При этом схема моделирования при фиксированном значении~$k$
такова:
     \begin{itemize}
\item нахождение оценок  $\hat{p}(k)$ и~$\hat{\vartheta}(k)$
параметров $p$ и~$\vartheta$  по исходной выборке;
\item генерирование бут\-стреп-вы\-бор\-ки для значений статистики
$\chi^2(\tilde{H}_k)$,
состоящее из повторений сле\-ду\-ющих шагов: генерация выборки из смеси
распределений с~параметрами  $\hat{p}(k)$ и~$\hat{\vartheta}(k)$,
подсчет по ней набора значений $v_l^B$,
нахождение оценок $\hat{p}^B(k)$ и~$\hat{\vartheta}^B(k)$
па\-ра\-мет\-ров~$p$ и~$\vartheta$  по сгенерированной выборке,
вычисление с~по\-мощью  $\hat{p}^B(k)$ и~$\hat{\vartheta}^B(k)$ набора
$p_l^B(\tilde{H}_k)$, получение с~по\-мощью
наборов $v_l^B$  и~$p_l^B(\tilde{H}_k)$  очередного значения
 $(\chi^2)^B(\tilde{H}_k)$;
\item нахождение бут\-стреп-оцен\-ки  $\alpha^B(\hat{\chi}^2(\tilde{H}_k))$
для $\alpha(\hat{\chi}^2(\tilde{H}_k))$  в~виде
относительной частоты наступления в~ходе генерирования
бут\-стреп-вы\-бор\-ки события
$(\chi^2)^B (\tilde{H}_k)\hm > \hat{\chi}^2(\tilde{H}_k)$.
\end{itemize}

     Рассмотренные два подхода (на основе отношения правдоподобия и~с
использованием рас\-стояния) к~подбору числа элементов смеси реализуют
множественную проверку гипотез при необходи\-мости принять единое решение:
для отдельных гипотез имеются индивидуальные критерии и~проблема состоит
в их комбинировании. При этом возникает следующая проб\-ле\-ма: если для
каж\-до\-го отдельного критерия задать уровень значимости~$\alpha$, то с~ростом
чис\-ла гипотез вероятность одного или более ошибочного отвержения гипотез
при условии их справедливости может существенно вы\-рас\-ти (см.~[12,
разд.~9]). Таким образом, говорить о том, что игнорирование
множественности приводит к~процедуре, контролирующей принятие решений
с~уровнем~$\alpha$, ошибочно. К~сожалению, решения, предложенные
в~\cite{12-kri}, не подходят для рассматриваемой задачи. Основная причина
кроется в~наличии зависимости между решениями частных задач о конкретном
числе элементов смеси, причем эта связь имеет весьма сложный характер.
В~данной работе для того, чтобы получить представление о том, на что можно
рассчитывать при множественной проверке гипотез, рассматривается
следующая процедура принятия решения об определенном значении~$\hat{k}$
числа элементов смеси: $\hat{k}$ таково, что для $k\hm= 1,2,\ldots, \hat{k}\hm-
1$ гипотеза о~том, что число компонент смеси равно~$k$, отвергается, а~для
$k\hm= \hat{k}$~--- принимается. Для исследования статистических свойств этой
процедуры предлагается опять же обращаться к~бут\-стреп-ме\-тоду.

\section{Эксперименты}

     Объектом анализа были данные для 363 мужчин, включающие их возраст
и измерения PSA, опубликованные в~[5, приложение~5.1
гл.~5].

     В соответствии с~общепринятой градацией были выделены возрастные
группы, для которых проводилось оценивание основных характеристик
референсных значений. Проверка независимости проводилась с~помощью
     $\chi^2$-кри\-те\-рия подтверждения независимости в~таблице
сопряженности значений показателей <<возраст>> и~<<PSA>>. Далее
приведены основные результаты:
     \begin{enumerate}[1.]
\item Диапазон возрастов~--- меньше~50, объем группы~--- 68;

характеристики PSA: сред\-нее\;=\;1,0; стандартное отклонение (СО)\;=\;1,0;
минимальное зна\-че\-ние\;=\;0,2;
максимальное зна\-че\-ние\;=\;6,3; количество раз\-лич\-ных\;=\;23;

     критический уровень значимости для проверки независимости
     <<возраст>> и~<<PSA>>\;=\;79\%.
\item Диапазон возрастов~--- $[50,60)$, объем группы~--- 114;

характеристики PSA: сред\-нее\;=\;1,2; СО\;=\;1,1; минимальное зна\-че\-ние\;=\;0,2;
максимальное зна\-че\-ние\;=\;6,7; количество раз\-лич\-ных\;=\;30;

     критический уровень значимости для проверки независимости <<возраст>>
     и~<<PSA>>\;=\;49\%.
\item Диапазон возрастов~--- $[60,70)$, объем группы~--- 126;

характеристики PSA: сред\-нее\;=\;2,0; СО\;=\;1,8; минимальное зна\-че\-ние\;=\;0,2;
максимальное зна\-че\-ние\;=\;11,1; количество раз\-лич\-ных\;=\;45;

     критический уровень значимости для проверки независимости <<возраст>> и~<<PSA>>\;=\;2\%.
\item Диапазон возрастов~--- не меньше~70, объем группы~--- 55;

характеристики PSA: сред\-нее\;=\;2,2; СО\;=\;1,9; минимальное зна\-че\-ние\;=\;0,2;
максимальное зна\-че\-ние\;=\;7,9; количество раз\-лич\-ных\;=\;33;

     критический уровень значимости для проверки независимости <<возраст>> и~<<PSA>>\;=\;76\%.
     \end{enumerate}

     Для сравнения характеристики PSA по всей совокупности данных
следующие: сред\-нее\;=\;1,6; СО\;=\;1,6; минимальное зна\-че\-ние\;=\;0,2; максимальное
зна\-че\-ние\;=\;11,1; количество раз\-лич\-ных\;=\;56;
критический уровень значимости для проверки
независимости <<возраст>> и~<<PSA>>\;=\;0\%.

\begin{table*}\small %tabl1
\begin{center}
\parbox{310pt}{\Caption{Результаты множественной проверки гипотез на основе критерия отношения
правдоподобия}

}

\vspace*{2ex}

\begin{tabular}{|c|c|r|c|c|c|}
\hline
&&&&&\\[-9pt]
$k$&$\max\limits_{p,\vartheta} \ln L(k,p,\vartheta)$&
\multicolumn{1}{c|}{$s_k$}&$\alpha_k^B$,
\%&$\mathrm{Ave}\{s_k\}$&$E\{S_k\}$ ожидаемое\\
&&&&&\\[-9pt]
\hline
1&$-172{,}84$&85,08&\hphantom{9}0&3,29&3,00\\
2&$-130{,}30$&23,98&\hphantom{9}0&2,95&3,00\\
3&$-118{,}31$&3,31&23&2,12&3,00\\
4&$-116{,}65$&1,16&45&1,56&3,00\\
5&$-116{,}08$&0,02&29&0,15&3,00\\
6&$-116{,}07$&$-0{,}00$&93&0,03&3,00\\
7&$-116{,}07$&$-0{,}00$&91&$-0{,}00$\hphantom{$-$}&3,00\\
8&$-116{,}07$&$-0{,}00$&90&0,00&3,00\\
9&$-116{,}07$&$-0{,}00$&85&$-0{,}01$\hphantom{$-$}&3,00\\
10\hphantom{9}&$-116{,}07$&$-0{,}00$&90&$-0{,}00$\hphantom{$-$}&3,00\\
\hline
\end{tabular}
\end{center}
\vspace*{-6pt}
\end{table*}

     В приведенных результатах обращают на себя внимание два момента:
высокая дискретность исходных данных (среди 363 значений PSA различных
всего~56), подтверждение общеизвестного факта зависимости значений PSA от
возраста.
\begin{center}  %fig1
\vspace*{6pt}
\mbox{%
 \epsfxsize=77.98mm
 \epsfbox{kri-1.eps}
 }
\end{center}

\vspace*{3pt}

\noindent
{{\figurename~1}\ \ \small{Зависимость информационных показателей от числа элементов смеси:
\textit{1}~--- $-2\ln L(k)$; \textit{2}~--- AIK; \textit{3}~--- BIC; \textit{4}~--- AWE}}


%\vspace*{9pt}


\addtocounter{figure}{1}

     Для экспериментов с~моделированием распределения одномерных
данных о~PSA была выбрана 2-я возрастная группа, достаточно
представительная по объему и~без признаков зависимости показателей
<<возраст>> и~<<PSA>>.





     Возможности подбора числа элементов смеси с~помощью графических
методов и~критериев информационного типа продемонстрированы на рис.~1:
     \begin{itemize}
     \item
минимумы для AIK ($k\hm=3$), BIC ($k\hm=3$) и~AWE ($k\hm=2$)
выражены слабо;
     \item
принимаемое решение о~$\hat{k}$ не подтверждается ка\-ки\-ми-ли\-бо
количественными характеристиками его качества.
\end{itemize}

     Множественная проверка гипотез на основе критерия отношения
правдоподобия дала результаты, представленные в~табл.~1. В~ней для $k\hm=
1,2,\ldots,10$ приведены значения $\alpha_k^B$ и~средние значения
$\mathrm{Ave}\{s_k\}$ статистики~$S_k$ при условии нулевой
 гипотезы~$H_k$,
которые оценивались с~помощью 10$^3$ экспериментов. В~последнем столбце
приведены ожидаемые значения $E\{S_k\}$, полученные из предположения,
что асимптотическое для статистики отношение правдоподобия
     $\chi^2$-рас\-пре\-де\-ле\-ние имеет число степеней свободы, равное
разнице в~числе параметров при двух конкурирующих гипотезах.



     Из полученных результатов можно сделать следующие выводы:
     \begin{itemize}
\item оценкой для числа элементов смеси может быть значение~3, причем даже
для малых уровней значимости в~1\% этой оценке соответствует отвержение
нулевой гипотезы о том, что $k\hm=2$, в~пользу конкурирующей, что
$k\hm=3$, но принятие нулевой гипотезы о том, что $k\hm=3$, при
конкурирующей, что $k\hm=4$;
\item сравнение двух последних столбцов не оставляет надежды на
использование асимптотических результатов о~распределении статистики
критерия отношения правдоподобии.
\end{itemize}



     Для того чтобы представить, на что можно рассчитывать при
множественной проверке гипотез, задаваясь конкретным уровнем
значимости~$\alpha$ в~частном случае, анализировалась следующая процедура.
Пусть при последовательной проверке частных гипотез ($k\hm= 1,2,\ldots$)
число первых отвергнутых нулевых гипотез равно $N_{\overline{H0}}$, а
гипотеза с~индексом $N_{\overline{H0}}\hm+1$ принимается. В~качестве
оценки для числа элементов смеси примем получившееся $N_{\overline{H0}}$.
Если исходная гипотеза заключается в~том,
что $\hat{k}\hm=3$, то интересно
связать вероятность~$\tilde{\alpha}$ отвергнуть эту гипотезу, если она верна, с~конкретным значением~$\alpha$ частного критерия. Для этого было проведено
10$^3$ бут\-стреп-экс\-пе\-ри\-мен\-тов и~оценены значения~$\tilde{\alpha}$
(табл.~2).


     Заметим, что умеренные значения~$\alpha$ частного критерия
порождают практически те же значения для множественной проверки гипотез,
а~уменьшение~$\alpha$ ($\alpha\hm= 2{,}5$ и~1,0) приводит, вообще говоря,
к~рос\-ту~$\tilde{\alpha}$.

     Высокое качество принимаемых решений относительно числа элементов
смеси подтверждается
 данными табл.~3, где приведены выборочные
характеристики оценки $\hat{k}\hm= N_{\overline{H0}}$.


{\begin{center} %tabl2

\vspace*{-1pt}

{\parbox{39mm}{{\tablename~2}\ \ \small{Результаты бут\-стреп-оцен\-ки связи уровней
значимости частной
и~множественной проверки гипотез на основе критерия отношения правдоподобия}}}

\vspace*{6pt}

{\small\tabcolsep=16pt
\begin{tabular}{|c|c|}
\hline
$\alpha$, \%& $\tilde{\alpha}$, \%\\
\hline
1,0& 5,9\\
2,5& 4,2\\
5,0& 5,4\\
10,0\hphantom{9}& 10,3\hphantom{9}\\
20,0\hphantom{9}&20,4\hphantom{9}\\
\hline
\end{tabular}}
\vspace*{-6pt}
\end{center}}



\addtocounter{table}{1}

\pagebreak

{\begin{center}
\vspace*{1pt}
 %tabl2
{\parbox{68mm}{{\tablename~3}\ \ \small{Результаты бут\-стреп-оцен\-ки моментов оценки числа элементов смеси на
основе критерия отношения правдоподобия}}}

\vspace*{6pt}

{\small \begin{tabular}{|c|c|c|c|c|c|}
      \hline
      \multicolumn{1}{|c|}{\raisebox{-6pt}[0pt][0pt]{Моменты}} &\multicolumn{5}{c|}{$\alpha$, \%}\\
      \cline{2-6}
 &1,0&2,5&5,0&10,0&20,0\\
\hline
Среднее&2,95&2,99&3,04&3,10&3,23\\
СО&0,06&0,04&0,06&0,11&0,24\\
\hline
\end{tabular}}
\end{center}}

\vspace*{6pt}

\addtocounter{table}{1}

{\begin{center} %tabl4
\vspace*{1pt}
 %tabl2
{\parbox{68mm}{{\tablename~4}\ \ \small{Результаты бут\-стреп-оцен\-ки моментов оценки числа элементов смеси на основе
расстояния $\chi^2$ между распределениями}}}

\vspace*{6pt}

{\small\begin{tabular}{|c|c|c|c|c|c|}
      \hline
      \multicolumn{1}{|c|}{\raisebox{-6pt}[0pt][0pt]{Моменты}} &\multicolumn{5}{c|}{$\alpha$, \%}\\
      \cline{2-6}
 &1,0&2,5&5,0&10,0&20,0\\
\hline
Среднее &2,20&2,35&2,49&2,67&2,92\\
СО&0,24&0,29&0,42&0,56&0,93\\
\hline
\end{tabular}}
\end{center}}

\vspace*{12pt}


     Описанные выше действия по исследованию выборочных свойств оценки
числа элементов смеси были повторены для критерия, основанного на
расстоянии $\chi^2$ между распределениями. Эксперименты по множественной
проверке гипотез на основе указанного критерия позволяют сделать следующие
выводы:
     \begin{itemize}
\item оценкой для числа элементов смеси опять же может быть значение~3, так
как  нулевая гипотеза о числе элементов смеси, равном~2, отвергается с~двухпроцентным уровнем значимости, но уже для $k\hm=3$ нулевая гипотеза
принимается при критическом уровне значимости в~62\%;
\item асимптотические результаты о распределении статистики критерия не
применимы.
\end{itemize}

     Выводы о свойствах описанной процедуры множественной проверки
гипотез при заданном уровне значимости частного критерия несколько иные:
     \begin{itemize}
\item значения~$\alpha$ частного критерия порождают большие значения
вероятности~$\tilde{\alpha}$ (так $\tilde{\alpha}\hm= 78\%$ для $\alpha\hm=
1\%$; $\tilde{\alpha}\hm= 66\%$ для $\alpha\hm= 2{,}5\%$;
$\tilde{\alpha}\hm= 56\%$ для $\alpha\hm= 5\%$; $\tilde{\alpha}\hm= 43\%$
для $\alpha\hm= 10\%$; $\tilde{\alpha}\hm= 33\%$ для $\alpha\hm= 20\%$);
\item качество оценки $\hat{k}\hm= N_{\overline{H0}}$ снижается (см.\
табл.~4 для критерия, основанного на расстоянии между распределениями,
и~табл.~3).
\end{itemize}



     Таким образом, данные о~PSA для мужчин в~возрасте от 50 до 60~лет
могут быть описаны с~помощью смеси трех нормальных распределений $\varphi
\left( u,\left(\mu, \sigma^2\right)\right)$, а~именно:
     \begin{multline*}
     f(u) =0{,}459 \varphi\left ( u, (0{,}527, 0{,}050)\right)+{}\\
     {}+ 0{,}446 \varphi\left( u, (1{,}307, 0{,}252)\right) +{}\\
     {}+0{,}095 \varphi\left( u, (3{,}829, 2{,}211)\right)\,.
     \end{multline*}

     Проведенный анализ относится к~определенному возрастному диапазону,
выбранному достаточно произвольно. Последнее можно исключить, обобщая
задачу анализа данных за счет перехода от одномерного случая (описание
условного распределения показателя <<PSA>> при определенных значениях
<<возраст>>) к~двухмерному (описание совместного распределения
показателей <<PSA>> и~<<возраст>>). Теперь объектом анализа совершенно
обоснованно может стать вся совокупность исходных данных.

     Подбор числа элементов смеси с~помощью графических методов и~информационных критериев не дал четких результатов, в~частности минимум
AIK достигается при $k\hm=5$, минимум BIC~--- при $k\hm=3$, минимум
AWE~--- при $k\hm=2$. Множественная проверка гипотез на основе критерия
отношения правдоподобия привела к~оценке $\hat{k}\hm=4$ (для уровня
значимости в~2\% этой оценке соответствует отвержение нулевой гипотезы о
том, что $k\hm=3$, в~пользу конкурирующей, что $k\hm=4$, но принятие
нулевой гипотезы о том, что $k\hm=4$, при конкурирующей, что $k\hm=5$).
Анализ выборочных распределений с~по\-мощью статистики~$\chi^2$ не
проводился, так как соответствующий критерий требует адаптации к~условию
многомерных данных.

     Таким образом, совместное распределение показателей <<PSA>>
(соответствует~$u_1$) и~<<возраст>> (соответствует~$u_2$) может быть
описано с~по\-мощью смеси четырех нормальных распределений $\varphi\left(
x_1,x_2,\left( \mu_{j1}, \mu_{j2}, \sigma^2_{j1}, \sigma^2_{j2},
\rho_j\right)\right)$, а именно:
     {\small\begin{multline*}
     f(u)={}\\
     {}= 0{,}447\varphi\left( u_1,u_2, (0{,}591, 55{,}800, 0{,}068, 83{,}600,
0{,}147)\right) +{}\\
     {}+ 0{,}345\varphi\left( u_1,u_2, (1{,}410, 60{,}400, 0{,}286, 89{,}600,
0{,}126)\right)+{}\\
     {}+ 0{,}187\varphi\left( u_1,u_2, (3{,}750, 63{,}400, 2{,}370, 68{,}900,
0{,}211)\right)+{}\\
     {}+\;0{,}022\varphi\left( u_1,u_2, (5{,}520, 72{,}400, 12{,}100, 25{,}000, -
0{,}989)\right)\!.\hspace*{-4.26pt}
     \end{multline*}}
     Отсюда находятся условные распределения, с~помощью которых можно,
в частности, получить референсные интервалы для отдельных значений
возраста.

На рис.~2 приведены графики оценок правых 95\%-ных границ
референсных интервалов как функции возраста: \textit{1}~--- построенный с~помощью
условных распределений и~\textit{2}~--- на основе
рекомендаций British Association of Urological Surgeons (BAUS). Заметим, что
рекомендации в~зависимости от источника могут отличаться (см.\ для
сравнения\linebreak
 упомянутый~\cite{13-kri} и, например,~\cite{14-kri}). Важно, что\linebreak
настораживающих отличий от рекомендаций не замечено. При этом оценки на
основе условных распределений имеют явные преимущества: высокую
детализацию по возрастам, сглаживание результатов наблюдений для
различных по объему возрастных групп, возможность формировать
предполо-\linebreak\vspace*{-12pt}

\begin{center}  %fig2
\vspace*{1pt}
 \mbox{%
 \epsfxsize=77.189mm
 \epsfbox{kri-2.eps}
 }
\end{center}

\vspace*{3pt}

\noindent
{{\figurename~2}\ \ \small{Зависимость правых 95\%-ных референсных границ от возраста,
найденных с~по\-мощью оценок условных распределений~(\textit{1})
и~рекомендованных BAUS~(\textit{2})}}

\vspace*{9pt}

\noindent
жения о характере зависимости между возрастом и~уровнем PSA.


\section{Заключение}

     Модель смеси нормальных распределений может эффективно
использоваться как средство аппроксимации реальных данных и~при этом быть
доступной с~точки зрения теоретического анализа. Для оценивания параметров
модели могут применяться различные методы: графические, метод моментов,
максимального правдоподобия, методы функции ошибок (методы
минимизации расстояния), байесовские. Среди них метод максимального
правдоподобия и~его воплощение в~виде EM-ал\-го\-рит\-ма занимают
главенствующее место, потому что позволяют решать следующие вопросы:
     \begin{itemize}
\item не привлекать дополнительной априорной информации к~уже не простой
постановке задачи описания данных с~помощью параметрической модели;
\item автоматически и~достаточно эффективно решать задачу
оценивания параметров в~широком диапазоне значений числа
элементов смеси.
\end{itemize}

     Очевидным кандидатом для проверки гипотез, связанных с~моделью
смеси, является критерий отношения правдоподобия. Но в~реальной обстановке
не очень больших объемов данных возникают проблемы в~описании
распределения статистики критерия хотя бы в~условиях одной из принимаемых
гипотез. В~связи с~этим приходится рассчитывать только на бут\-степ-ме\-тод, что
предъявляет повышенные требования к~эффективности реализации
     EM-ал\-го\-ритма.

{\small\frenchspacing
 {%\baselineskip=10.8pt
 \addcontentsline{toc}{section}{References}
 \begin{thebibliography}{99}
\bibitem{1-kri}
\Au{Sokal R.\,R., Sneath P.\,H.\,A.} Principles of numerical taxonomy.~--- San
Francisco\,--\,London: W.\,H.~Freeman and Co., 1963. 359~p.
\bibitem{2-kri}
\Au{Burtis C.\,A., Bruns D.\,E.} Tietz fundamentals of clinical chemistry and
molecular diagnostics.~--- St.\ Louis, MO, USA: Elsevier Health Sciences, 2014. 1104~p.
\bibitem{3-kri}
\Au{Sivkov A., Keshishev N., Krivenko~M., Kovchenko~G., Nikonova~L.}
Comparison of chromogranin-A levels determined by different test systems in
patients with prostate diseases~// Eur. Urol. Suppl., 2014. Vol.~13.
No.\,5. P.~143.
\bibitem{4-kri}
\Au{Кривенко М.\,П.} Сравнительный анализ процедур регрессионного
анализа~// Информатика и~её применения, 2014. Т.~8. Вып.~3. С.~70--78.
\bibitem{5-kri}
\Au{Harris E.\,K., Boyd J.\,C.} Statistical bases of reference values in laboratory
medicine.~--- New York, NY, USA: Marcel Dekker, 1995. 361~p.
\bibitem{6-kri}
\Au{McLachlan G., Peel D.} Finite mixture models.~--- New York,
NY, USA: Wiley\,\&\,Sons,
2000. 456~p.
\bibitem{7-kri}
\Au{Королев В.\,Ю.} ЕМ-ал\-го\-ритм, его модификации и~их применение к~задаче разделения смесей вероятностных распределений: Теоретический
обзор.~--- М.: ИПИ РАН, 2007. 94~с.
\bibitem{8-kri}
\Au{McLachlan G., Krishnan T.} The EM algorithm and extensions.~--- Hoboken,
NJ, USA: Wiley\,\&\,Sons, 2008. 359~p.
\bibitem{9-kri}
\Au{Кривенко М.\,П.} Прикладные методы оценивания распределения
многомерных данных малой выборки.~--- М.: ИПИ РАН, 2011. 146~с.
\bibitem{10-kri}
\Au{Васильев В.\,Г., Кривенко~М.\,П.} Методы автоматизированной обработки
текстов.~--- М.: ИПИ РАН, 2008. 305~с.
\bibitem{11-kri}
\Au{Seidel W., Mosler K., Alker~M.} A~cautionary note on likelihood ratio tests in
mixture models~// Ann. Inst. Statist. Math., 2000. Vol.~52. No.\,3. P.~481--487.
\bibitem{12-kri}
\Au{Lehmann E.\,L., Romano J.\,P.} Testing statistical hypotheses.~---
New York, NY, USA:
Springer, 2005. 792~p.
\bibitem{13-kri}
PSA Measurements.~--- British Association of Urological Surgeons (BAUS). March
2014. {\sf http://www. baus.org.uk}.
\bibitem{14-kri}
\Au{Brosman S.\,A.} Prostate-specific antigen testing~// Medscape, January 13, 2015.
{\sf http://emedicine. medscape.com/article/457394-overview}.
 \end{thebibliography}

 }
 }

\end{multicols}

\vspace*{-3pt}

\hfill{\small\textit{Поступила в~редакцию 19.02.15}}

\newpage

%\vspace*{12pt}

%\hrule

%\vspace*{2pt}

%\hrule

\vspace*{-24pt}

\def\tit{MODELS FOR REPRESENTATION AND TREATMENT\\ OF REFERENCE VALUES}

\def\titkol{Models for representation and treatment of reference values}

\def\aut{M.\,P.~Krivenko}

\def\autkol{M.\,P.~Krivenko}

\titel{\tit}{\aut}{\autkol}{\titkol}

\index{Krivenko M.\,P.}

\vspace*{-9pt}


\noindent
Institute of Informatics Problems, Federal Research
Center ``Computer Science and Control'' of the
Russian Academy of Sciences, 44-2 Vavilov Str., Moscow 119333,
Russian Federation


\def\leftfootline{\small{\textbf{\thepage}
\hfill INFORMATIKA I EE PRIMENENIYA~--- INFORMATICS AND
APPLICATIONS\ \ \ 2015\ \ \ volume~9\ \ \ issue\ 2}
}%
 \def\rightfootline{\small{INFORMATIKA I EE PRIMENENIYA~---
INFORMATICS AND APPLICATIONS\ \ \ 2015\ \ \ volume~9\ \ \ issue\ 2
\hfill \textbf{\thepage}}}

\vspace*{3pt}


\Abste{The article considers the problem of modeling reference values~---
results of a~certain type of quantities obtained from a single individual or a~group
of individuals corresponding to a~stated description. For this purpose, the article
proposes to use a~mixture of normal distributions, which can effectively serve
as a~means of approximating the actual data and to be accessible from the
standpoint of theoretical analysis. In estimating the parameters of mixture
of distributions, the major role is played by the maximum likelihood method
and its embodiment in the form of the expectation-maximization (EM)
algorithm. For assessing the number
of mixture components, the article suggests to use the likelihood ratio test and
a~method based on the chi-square distance between the distributions.
Their properties are investigated using the bootstrap method. As an experiment,
the article considers the description of the empirical distribution of
patient data, including the age and measurements of PSA (Prostate-Specific Antigen).
The proposed solutions have clear advantages: high detail by age,
smoothing the results of observations for age groups which are different
in size, and the opportunity to form assumptions about the nature of the
relationship between age and PSA.}

\KWE{mixture of normal distributions; assessing the number of components
in mixture models; reference values}




\DOI{10.14357/19922264150208}

%\Ack
%\noindent



%\vspace*{3pt}

  \begin{multicols}{2}

\renewcommand{\bibname}{\protect\rmfamily References}
%\renewcommand{\bibname}{\large\protect\rm References}



{\small\frenchspacing
 {%\baselineskip=10.8pt
 \addcontentsline{toc}{section}{References}
 \begin{thebibliography}{99}


\bibitem{1-kri-1}
\Aue{Sokal, R.\,R., and P.\,H.\,A.~Sneath}. 1963. \textit{Principles of numerical
taxonomy}. San Francisco\,--\,London: W.\,H.~Freeman and Co. 359~p.
\bibitem{2-kri-1}
\Aue{Burtis, C.\,A., and D.\,E.~Bruns}. 2014. \textit{Tietz fundamentals of clinical
chemistry and molecular diagnostics}. St.\ Louis, MO: Elsevier Health Sciences.
1104~p.

\bibitem{3-kri-1}
\Aue{Sivkov, A., N.~Keshishev, M.~Krivenko, G.~Kovchenko, and L.~Nikonova}.
2014. Comparison of chromogranin-A levels determined by different test systems in
patients with prostate diseases. \textit{Eur. Urol. Suppl.} 13(5):143.
\bibitem{4-kri-1}
\Aue{Krivenko, M.\,P.} 2014. Sravnitel'nyy analiz protsedur regressionnogo analiza
[Comparative analysis of regression analysis procedures]. \textit{Informatika i~ee
Primeneniya}~--- \textit{Inform. Appl.}  8(3):70--78.
\bibitem{5-kri-1}
\Aue{Harris, E.\,K., and J.\,C.~Boyd}. 1995. \textit{Statistical bases of reference
values in laboratory medicine}. New York, NY: Marcel Dekker. 361~p.
\bibitem{6-kri-1}
\Aue{McLachlan, G., and D.~Peel}. 2000. \textit{Finite mixture models}.
New York, NY:
Wiley\,\&\,Sons.  456~p.
\bibitem{7-kri-1}
\Aue{Korolev, V.\,Yu.} 2007. \textit{EM-algoritm, ego modifikatsii i ikh primenenie
k~zadache razdeleniya smesey veroyatnostnykh raspredeleniy. Teoreticheskiy obzor}
[EM-algorithm. Modifications and their application to the separation of mixtures of
probability distributions. Theoretical review]. Moscow: IPI RAN. 94~p.
\bibitem{8-kri-1}
\Aue{McLachlan, G., and T.~Krishnan}. 2008. \textit{The EM algorithm and
extensions}. Hoboken, NJ: Wiley\,\&\,Sons. 359~p.
\bibitem{9-kri-1}
\Aue{Krivenko, M.\,P.} 2011. \textit{Prikladnye metody otsenivaniya raspredeleniya
mnogomernykh dannykh maloy vyborki} [Applied methods for estimating the
distribution of multi-dimensional data of small sample size]. Moscow: IPI RAN.
146~p.
\bibitem{10-kri-1}
\Aue{Vasil'ev, V.\,G., and M.\,P.~Krivenko}.  2008. \textit{Metody avtomatizirovannoy
obrabotki tekstov} [Methods for automatic text processing]. М.: IPI RAN. 305~p.
\bibitem{11-kri-1}
\Aue{Seidel, W., K.~Mosler, and M.~Alker}. 2000. A~cautionary note on likelihood
ratio tests in mixture models. \textit{Ann. Inst. Statist. Math.} 52(3):481--487.
\bibitem{12-kri-1}
\Aue{Lehmann, E.\,L., and J.\,P.~Romano}. 2005. \textit{Testing statistical
hypotheses}. New York, NY: Springer. 792~p.
\bibitem{13-kri-1}
PSA Measurements. British Association of Urological Surgeons (BAUS). March
2014. Available at: {\sf http:// www.baus.org.uk} (accessed February~11, 2015).
\bibitem{14-kri-1}
\Aue{Brosman, S.\,A.} January~13, 2015  (updated).
Age-specific PSA reference ranges.
2015. Available at: {\sf http:// emedicine.medscape.com/article/457394-overview}
(accessed February~11, 2015).
\end{thebibliography}

 }
 }

\end{multicols}

\vspace*{-4pt}

\hfill{\small\textit{Received February 19, 2015}}

\vspace*{-18pt}



\Contrl

\noindent
\textbf{Krivenko Michail P.} (b.\ 1946)~---
 Doctor of Science in technology, professor, leading scientist, Institute of
Informatics Problems, Federal Research Center ``Computer Science and
Control'' of the Russian Academy of Sciences, 44-2 Vavilov Str., Moscow
119333, Russian Federation; mkrivenko@ipiran.ru
\label{end\stat}


\renewcommand{\bibname}{\protect\rm Литература}
