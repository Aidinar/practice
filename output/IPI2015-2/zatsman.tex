\def\stat{zatsman}

\def\tit{ИНДИКАТОРЫ ТЕМАТИЧЕСКИХ ВЗАИМОСВЯЗЕЙ ОТРАСЛЕЙ НАУКИ
И~ИНФОРМАЦИОННО-КОМПЬЮТЕРНЫХ ТЕХНОЛОГИЙ В~НАЧАЛЕ
XXI~ВЕКА$^*$}

\def\titkol{Индикаторы тематических взаимосвязей отраслей науки
и~информационно-компьютерных технологий} % в~начале XXI~века}

\def\aut{В.\,А.~Минин$^1$, И.\,М.~Зацман$^2$, В.\,А.~Хавансков$^3$,
С.\,К.~Шубников$^4$}

\def\autkol{В.\,А.~Минин, И.\,М.~Зацман, В.\,А.~Хавансков,
С.\,К.~Шубников}

\titel{\tit}{\aut}{\autkol}{\titkol}

\index{Минин В.\,А.}
\index{Зацман И.\,М.}
\index{Хавансков В.\,А.}
\index{Шубников С.\,К.}

{\renewcommand{\thefootnote}{\fnsymbol{footnote}} \footnotetext[1]
{Работа выполнена в~Институте проблем информатики
Федерального исследовательского
центра <<Информатика и~управление>> Российской академии наук при частичной поддержке РГНФ (грант
№\,12-02-12019в).}}


\renewcommand{\thefootnote}{\arabic{footnote}}
\footnotetext[1]{Российский фонд фундаментальных исследований, minin@rfbr.ru}
\footnotetext[2]{Институт проблем информатики Федерального исследовательского
центра <<Информатика и~управление>> Российской академии наук,
iz\_ipi@a170.ipi.ac.ru}
\footnotetext[3]{Институт проблем информатики Федерального исследовательского
центра <<Информатика и~управление>> Российской академии наук, havanskov@a170.ipi.ac.ru}
\footnotetext[4]{Институт проблем информатики Федерального исследовательского
центра <<Информатика и~управление>> Российской академии наук, sergeysh50@yandex.ru}


\Abst{Представлены результаты экспериментальных вычислений
индикаторов тематических взаимосвязей науки
и~ин\-фор\-ма\-ци\-он\-но-компью\-те\-рных технологий. Вычисленные значения
индикаторов получены с~помощью макета аналитической информационной системы (АИС),
который был создан в~рамках проекта Российского гуманитарного научного фонда (РГНФ)
<<Информационная система мониторинга и~оценивания
ин\-но\-ва\-ци\-он\-но-тех\-но\-ло\-ги\-че\-ско\-го потенциала направлений
фундаментальных научных исследований>>. Макет позволяет вычислять значения
индикаторов взаимосвязей отраслей науки и~направлений научных исследований (ННИ)
с~заданным
видом технологий. В~экспериментальных вычислениях в~качестве исходной информации
использовались официальные данные Роспатента об изобретениях по классу G06
Международной патентной классификации (МПК) (Обработка данных; Вычисления; Счет),
опубликованные в~2000--2012~гг. Исходные данные для проведения расчетов не являются
числовой информацией, а~представляют собой полные тексты описаний изобретений на
естественном языке (ЕЯ). Поэтому до начала экспериментальных вычислений индикаторов
выполнялось извлечение из полных текстов изобретений информации о научных
публикациях, цитируемых в~описаниях изобретений, и~определялось число публикаций по
каждой отрасли науки и~ННИ. Полученная числовая
информация является исходной для вычислений значений индикаторов и~дает возможность
экспертам определять степень интенсивности переноса знаний из отраслей
науки и~ННИ в~сферу технологий и~оценивать их с~помощью
количественных индикаторов.}

\KW{взаимосвязи науки и~технологий; информационно-компьютерные технологии;
обработка текста изобретения; регулярные выражения; рубрицирование; расчет значений
индикаторов}

\DOI{10.14357/19922264150212}

\vspace*{6pt}


\vskip 14pt plus 9pt minus 6pt

\thispagestyle{headings}

\begin{multicols}{2}

\label{st\stat}

\section{Введение}

  Данная работа содержит описание итоговых результатов по проекту РГНФ (грант
  №\,12-02-12019в) <<Информационная система мониторинга и~оценивания
  ин\-но\-ва\-ци\-он\-но-тех\-но\-ло\-ги\-че\-ско\-го потенциала направлений
фундаментальных научных исследований>>.

  Методологические проблемы мониторинга, включая вопросы определения значений
индикаторов тематических взаимосвязей науки и~технологий, изложены в~работах~[1--6].
В~\cite{1-zat} приводится описание индикаторов, предназначенных для количественного
оценивания ин\-тен\-сив\-ности процессов передачи знаний от науки к~технологиям. В~качестве
степени интенсивности процессов передачи знаний берется число научных публикаций,
цитируемых экспертами в~отчетах о патентном поиске или авторами изобретений в~их
описаниях. Определение степени интенсивности являлось целью проекта. В~проекте
использовались описания изо\-бре\-те\-ний, опубликованных Федеральной службой по
интеллектуальной собственности (Роспатент) за период с~2000 по 2012~гг.\
и~относящихся к~классу G06 (Обработка данных; Вычисление; Счет) МПК.

  В текстовых описаниях отобранных изобретений были найдены сделанные авторами
ссылки на научные публикации, библиографические данные которых позволяют отнести их
к определенным отраслям науки и~ННИ.
Таким образом, с~помощью ссылок цитирования устанавливалась связь между технологической сферой (в виде
совокупности индексов МПК) и~цитируемыми результатами научных исследований (для
кодирования результатов использовались Государственный рубрикатор
  на\-уч\-но-тех\-ни\-че\-ской информации (ГРНТИ) и~рубрикатор Российского фонда
фундаментальных исследований (РФФИ)).

  Использование информационных ресурсов Роспатента было обусловлено тем, что они
пред\-став\-ле\-ны в~электронном виде и~доступны для автоматизированной обработки.
В~соответствии со Страсбургским соглашением от 24~марта 1971~г.\ о~МПК компетентные
органы стран-участ\-ниц Союза по МПК при классифицировании патентных документов
должны указывать <<полные индексы МПК, присвоенные изобретению,
описанному в~документе>> (ст.~4, п.~3). Это означает, что публикуемые в~Роспатенте сведения о
выданных патентах на изобретения содержат индексы МПК, которые можно использовать
для описания тематики исследуемых групп технологий. Одновременно имеются
полнотекстовые описания изобретений, представляющие собой неструктурированные
текс\-ты, в~которых при изложении сути изобретения авторы нередко ссылаются на
публикации в~научных изданиях. Таким образом, библиографические ссылки на публикации
в описаниях изобретений, привязанные к~одной или нескольким рубрикам
ННИ,\linebreak можно сопоставить с~индексами МПК, описыва\-ющи\-ми
технологическую сферу, к~которой относится изобретение, и~таким образом косвенно
оценить интенсивность процессов передачи знаний от науки к~тех\-но\-ло\-гиям.
{ %\looseness=1

}

  В процессе анализа изобретений объем отобранных полнотекстовых описаний может
достигать несколь\-ких сотен тысяч. Например, в~работе~[7]\linebreak описывается процесс обработки
массива из 656\,695~патентов на изобретения, выданных Патентным ведомством США. Для
установления взаимосвязей между индексами МПК и~кодами рубрик ННИ
 из описаний изобретений в~процессе обработки были выделены
1\,147\,160~ссылок на цитируемые публикации (ссылки на патенты не рассматривались).
Затем из них для дальнейшей обработки были отобраны только те ссылки на журнальные
статьи, для которых удалось идентифицировать название журнала и~соотнести его с~нормативным списком названий журналов, в~котором каждому названию присвоена одна или
несколько рубрик ННИ. При таких ограничениях для анализа
было отобрано 106\,636~ссылок, т.\,е.\ менее 10\% от выделенных ссылок на цитируемые
публикации.

  При реализации подобной методологии для анализа отечественных описаний изобретений
аналогичные зарубежные решения трудно адаптировать в~силу ряда причин, подробно
рассмотренных в~работах~[8, 9], а~именно:
  \begin{itemize}
\item отсутствие в~<<Административном регламенте исполнения Роспатентом приема
заявок на изо\-бре\-те\-ние, их рассмотрения и~экспертизы>> требований именно к~структурированному представлению ссылок на цитируемые публикации (см.\
п.~10.11(12) Регламента~[10]);
\item отсутствие в~опубликованных электронных версиях полнотекстовых описаний
изо\-бре\-те\-ний групп меток, выделяющих ссылки на цитируемые пуб\-ли\-ка\-ции согласно
рекомендациям стандарта Всемирной организации интеллектуальной собственности
(ВОИС) ST.14~[11];
\item отсутствие или неполнота списка нормализованных и~сокращенных названий
журналов, исполь\-зу\-емых в~ссылках на цитируемые пуб\-ли\-ка\-ции, в~системах подготовки
электронных патентных заявок.
\end{itemize}

  Таким образом, при исследовании тематических взаимосвязей технологий и~ННИ
  возникает задача анализа десятков и~сотен тысяч полнотекстовых
описаний изобретений и~поиска в~их текстах на ЕЯ ссылок на
публикации с~последующей их структуризацией и~привязкой ссылок к~руб\-ри\-кам
ННИ. Как следствие, возникает задача автоматизации данного
процесса. При этом необходимо учитывать, что по своему содержанию само
биб\-лио\-гра\-фи\-че\-ское описание является структурированным информационным объектом,
состоящим из нескольких полей, который может размещаться в~любом месте
неструктурированного текста описания изобретения, а~разные поля биб\-лио\-гра\-фи\-че\-ской
информации могут быть, в~общем случае, на разных %\linebreak
языках.
{ %\looseness=1

}

  Целью данной работы является описание результатов завершенного проекта РГНФ и~тех
его задач, результаты решения которых в~основном и~определяют точность вычисленных
значений индикаторов тематических взаимосвязей науки и~технологий, а~именно:
  \begin{itemize}
\item поиск и~выделение ссылок на цитируемые научные публикации в~тексте
изобретений на ЕЯ;
\item рубрицирование выделенных ссылок по заданным классификаторам
ННИ.
\end{itemize}

\section{Поиск ссылок и~их рубрикация}

  Всего для экспериментальных расчетов с~использованием макета АИС~\cite{9-zat} было отобрано 6665~изобрете\-ний,
опубликованных Роспатентом в~2000--2012~гг.\ и~относящихся к~классу~G06 МПК.
В~полнотекстовых описаниях изобретений было выделено и~верифицировано 8847~ссылок
на цитируемые научные публикации. Архитектура АИС, ее описание и~технология
проведения расчетов подробно представлены в~работах~\cite{9-zat, 12-zat}. Полнота и~точность вычисленных значений индикаторов зависят от точности выделения ссылок в~тексте описания изобретения и~полноты перечня найденных ссылок (считается как доля от
общего их числа, которые присутствуют в~описании изобретения).

  Построение шаблона поиска ссылки (в шаблонах используется язык регулярных
выражений~\cite{13-zat}; подробнее о построении шаблонов см.~[14--17])
опирается, в~основном, на требование <<Административного регламента исполнения Роспатентом приема
заявок на изобретение, их рассмотрения и~экспертизы>>, содержащееся
в~п.~10.11(12)~\cite{10-zat}. Это требование звучит следующим образом:
<<Биб\-лио\-гра\-фи\-че\-ские
данные источников информации\linebreak
 указываются таким образом, чтобы источник информации
мог быть по ним обнаружен>>. Как по-\linebreak казывает анализ описаний изобретений, данное
положе\-ние Административного регламента трактуется авторами достаточно широко, от
классического представления ссылок на цитируемые пуб\-ли\-ка\-ции в~научных статьях
(опирающегося на стандарт представления библиографических данных~\cite{18-zat}) до
произвольной формы упоминания цитируемой пуб\-ли\-кации.

  Например, в~патенте 2144211 авторы упоминают источник следующим образом:
<<$\ldots$как он [способ~--- прим.\ авт.] описан в~главе~6 и~главе~12 книги под названием
<<Адаптивная обработка сигналов>> Бернарда Уидроу и~Сэмьюэла Стернса,
опубликованной издательством <<Прентис Холл>>, копирайт 1985~г.>>.

  Таким образом, создать шаблоны поиска ссылок на цитируемые публикации, которые
покрывали бы все возможные варианты представления ссылок на цитируемые публикации,
практически невозможно. Поэтому создаваемые шаблоны ориентированы, как правило, на
поиск ссылок на публикации, приближенных к~требованиям стандарта на представление
библиографических данных публикаций, или нестандартных, но чаще используемых видов
представления ссылок.

  Ввиду того, что и~сам стандарт допускает различные варианты представления
библиографических данных для разных видов публикаций (книги, статьи, доклады на
конференциях и~пр.), воз\-ник\-ла необходимость в~разработке целой коллекции шаб\-ло\-нов
$\{R\}$, каждый из которых нацелен на поиск заданного вида публикации в~некоторой стандартной форме представления библиографических данных или
нестандартной, но частотной форме. В~общем виде структура этих шаб\-ло\-нов опирается на
обобщенную структуру ссылки цитирования пуб\-ли\-ка\-ции, которая может быть представлена
в~следующем виде:
 [автор\{$S_1$\}] [название публикации] [\{$S_2$\}название источника] \{$S_3$\}атрибуты
публикации.

  Наличие квадратных скобок говорит о необязательности присутствия данного элемента
структуры в~реальной ссылке на цитируемую публикацию.

  В процессе поиска ссылок на публикацию в~описании изобретений к~его тексту
применяется коллекция \{$R$\} шаблонов поиска ссылок. Использование классов регулярных
выражений обеспечивает возможность получения непересекающихся фрагментов текста,
содержащих признаки ссылки на пуб\-ли\-ка\-цию, для отдельно взятого шаблона поиска из
коллекции \{$R$\}. Но в~то же время разные шаблоны коллекции могут формировать
пересекающиеся фрагменты текста. Следовательно, после применения к~тексту всей
коллекции \{$R$\} необходимо выполнить процедуру интеграции выделенных каждым
шаблоном фрагментов текста (см.~[17, рис.~1]).

  Каждый шаблон из коллекции \{$R$\} имеет видовую направленность, но заранее вид
публикации неизвестен, вследствие чего при его применении возможны следующие
варианты выделения текста:
  \begin{itemize}
\item[(а)] выделенный текст полностью соответствует ссылке на цитируемую
публикацию в~тексте описания изобретения;
\item[(б)] выделенный текст содержит часть ссылки на цитируемую публикацию в~тексте описания;
\item[(в)] выделенный текст превышает ссылку на цитируемую публикацию в~тексте
описания и~содержит часть содержательного текста описания;
\item[(г)] выделенный текст не содержит ссылки на цитируемую публикацию, но
соответствует шаблону поиска.
\end{itemize}

  Для того чтобы определить качество шаблона для поиска ссылок соответствующего вида,
был сформирован тестовый массив. Каждый элемент этого массива содержит
неструктурированный текст описания изобретения, внутри которого находится одна ссылка
на цитируемую публикацию определенного вида. Таблица~1 включает примеры ссылок на
публикации в~этих текстах.

Всего в~тестовом массиве представлено 277~ссылок на публикации разных видов. Для
каждой ссылки на публикацию указан вид публикации (статья, книга и~пр.)\ и~тип
структуры ссылки. Распределение ссылок на публикации по видам и~типам структуры даны в~табл.~2 и~3 соответственно.

\begin{table*}\small %tabl1
\begin{center}
\Caption{Примеры из тестового массива}
\vspace*{2ex}

\begin{tabular}{|c|p{90mm}|c|}
\hline
\tabcolsep=0pt\begin{tabular}{c}Номер\\ патента\end{tabular}&\multicolumn{1}{c|}{Текст ссылки на публикацию}&Вид публикации\\
\hline
\multicolumn{1}{|c|}{\raisebox{-6pt}[0pt][0pt]{2144274}}&J.\,B.~Postel. Simple mail transfer protokol. August 1982, Information Sciences Institute, University of
Southern California, RFC~821&
\multicolumn{1}{c|}{\raisebox{-6pt}[0pt][0pt]{Стандарт (RFC, ГОСТ, ОСТ и~пр.)}}\\
\hline
2144785&Журнал <<Science>>, 1998. No.\,5. Vol.~280. P.~1723&Статья в~журнале или сборнике \\
\hline
\multicolumn{1}{|c|}{\raisebox{-10pt}[0pt][0pt]{2145466}}&Рекомендация V.110 (1988) <<Голубой книги>> Международного консультативного комитета по
телефонии и~телеграфии (CCITT)&
\multicolumn{1}{c|}{\raisebox{-10pt}[0pt][0pt]{Стандарт (RFC, ГОСТ, ОСТ и~пр.)}} \\
\hline
\multicolumn{1}{|c|}{\raisebox{-10pt}[0pt][0pt]{2146840}}&Баранов С.\,И., Скляров В.\,А. Цифровые устройства на программируемых БИС с~матричной
структурой.~--- М.: Радио и~связь, 1986. С.~43&
\multicolumn{1}{c|}{\raisebox{-10pt}[0pt][0pt]{Книга}} \\
\hline
\multicolumn{1}{|c|}{\raisebox{-6pt}[0pt][0pt]{2148274}}&Горбань А.\,Н.
Обучение нейронных сетей. М.: СП Параграф, 1990&
\multicolumn{1}{c|}{\raisebox{-6pt}[0pt][0pt]{Книга}} \\
\hline
\multicolumn{1}{|c|}{\raisebox{-10pt}[0pt][0pt]{2149450}}&Вопросы проектирования радиоэлектронной аппаратуры. Опыт, результаты, проблемы.~--- Таллин:
\mbox{ЭстНИИНТИ}, 1989. С.~87--90. Рис.~6,\,\textit{б}&
\multicolumn{1}{c|}{\raisebox{-10pt}[0pt][0pt]{Книга}} \\
\hline
\multicolumn{1}{|c|}{\raisebox{-10pt}[0pt][0pt]{2149455}}&Омельченко~В.\,В. Теоретические основы классификации нечетких ситуаций при испытаниях
сложных технических комплексов.~--- М.: МО РФ, 1998. С.~328--351&
\multicolumn{1}{c|}{\raisebox{-10pt}[0pt][0pt]{Книга}} \\
\hline
2150140&Шустер Г. Детерминированный хаос.~--- М.: Мир, 1988. С.~33,~38&Книга \\
\hline
\end{tabular}
\end{center}
\vspace*{3pt}
\end{table*}





 %\begin{table*}
\noindent
  \begin{center}
  {{\tablename~2}\ \ \small{Распределение ссылок по типам публикаций}}
  \vspace*{2.3ex}

 {\small %tabl2
 \tabcolsep=8.2pt
  \begin{tabular}{|l|c|}
  \hline
\multicolumn{1}{|c|}{Тип публикации}&Количество\\
\hline
Книга&100\hphantom{9}\\
Статья в~журнале или сборнике&71\\
Стандарт (ИСО, ГОСТ, ОСТ и~пр.)&31\\
Веб-публикация&29\\
Материалы конференции&25\\
Отчет&16\\
Статья в~энциклопедии&\hphantom{9}5\\
\hline
Всего&277\hphantom{9}\\
\hline
\end{tabular}
}
\end{center}

\vspace*{12pt}
%\end{table*}



%  \begin{table*}
 %tabl3
  \noindent
  {{\tablename~3}\ \ \small{Распределение по типам структур ссылок на публикации}}\\
  \vspace*{-6pt}
  {\small  \begin{center}
  \tabcolsep=3pt
  \begin{tabular}{|l|c|}
  \hline
\multicolumn{1}{|c|}{Тип структуры текста ссылки}&\tabcolsep=0pt\begin{tabular}{c}Коли-\\чество\end{tabular}\\
\hline
Авторы(ФИО)/Название/Источник/Атрибуты&67\\
Название/Источник/Атрибуты&64\\
Авторы(ИОФ)/Название/Источник/Атрибуты&59\\
Произвольный текст&29\\
Источник&18\\
Источник/Атрибуты&16\\
Название/Авторы(ФИО)/Источник/Атрибуты&\hphantom{9}9\\
Название/Авторы(ИОФ)/Источник/Атрибуты&\hphantom{9}4\\
Источник/Название/Атрибуты&\hphantom{9}5\\
Авторы(ИОФ)/Источник/Атрибуты&\hphantom{9}3\\
Авторы(ИОФ)/Название/Атрибуты&\hphantom{9}2\\
Авторы(ФИО)/Название/Атрибуты&\hphantom{9}1\\
\hline
Всего&277\hphantom{9}\\
\hline
\end{tabular}
%\vspace*{3pt}
\end{center}}
%\end{table*}

\addtocounter{table}{2}

\columnbreak





  Для получения сравнительных количественных характеристик качества шаблонов поиска
в рамках макета АИС была разработана отдельная под\-сис\-те\-ма, в~которой реализована
следующая методика анализа качества поиска ссылок на пуб\-ли\-ка\-ции: для
каждого шаб\-ло\-на
поиска ссылок создается задание на его тестирование (рис.~1).


  В задании указывается имя тестируемого шаблона и~формируется тестовый массив
текстов изобре\-те\-ний на основании исходного\footnote{Исходный тестовый массив в~течение
времени может пополняться новыми образцами ссылок на цитируемые публикации. Поэтому
при создании тестового задания для чистоты эксперимента и~последующего корректного
анализа создается копия исходного массива, существующего на момент создания тестового
задания.}, содержащих ссылки (табл.~1 содержит их примеры).
В~тес\-то\-вом массиве помимо
выделенной ссылки на пуб\-ли\-ка\-цию в~описании изобретения указывается стартовая позиция
искомой ссылки на пуб\-ли\-ка\-цию в~текс\-те описания изобре\-те\-ния и~длина
текс\-та ссылки в~символах. Далее запускается программа поиска, которая для
каждого исследуемого описания
изобре\-те\-ния составляет список фрагментов текс\-та, который данный шаблон поиска выделил
как ссылки с~указанием стартовой позиции фрагмента в~текс\-те и~его длины.



  В результате формируется массив фрагментов, в~котором путем сравнения стартовых
позиций и~длин имеющейся искомой ссылки и~выделенного в~процессе поиска фрагмента
текста\linebreak получаются количественные характеристики тес\-ти\-ру\-емо\-го
 шаблона поиска ссылки на
публика-цию, которые
 определяются следующим образом.\linebreak\vspace*{-12pt}

 \pagebreak


\end{multicols}

\begin{figure} %fig1
 \vspace*{1pt}
 \begin{center}
 \mbox{%
 \epsfxsize=160mm
 \epsfbox{zac-1.eps}
 }
\end{center}
 \vspace*{-9pt}
\Caption{Пример задания на тестирование шаблона поиска ссылок}
\vspace*{6pt}
\end{figure}

\begin{multicols}{2}


\noindent
Если выделен\-ный
  фрагмент текста полностью  находится
 за пределами искомой ссылки на публикацию, то считается, что это
ложное выделение (так называемая ошибка 2-го рода). В~противном случае вы\-чис\-ля\-ет\-ся
процент покрытия выделенным фрагментом текс\-та искомого текста ссылки на
пуб\-ли\-кацию;
иными словами, определяется точность выделения ссылки цитирования.

Рисунок~1
иллюстрирует полученные результаты тестирования. Кроме того, дается список найденных
фрагментов текста, который дает возможность качественной оценки работы шаблона для
каждого случая его срабатывания (рис.~2).



  На основе этой оценки предлагаются рекомендации по доработке (уточнению)
существующего шаблона и/или созданию нового шаблона поиска. Проверяя
полноту и~точность каждого шаблона с~использованием тестового массива, можно получить ряд его
характеристик:
  \begin{enumerate}[1.]
\item Точность выделения текста ссылки на цитиру\-емую публикацию.

\columnbreak

\item Полноту поиска ссылок как отношение числа найденных фрагментов
текста с~ненулевой точностью выделения к~общему числу искомых ссылок на публикации, существующих
в тестовом мас\-сиве.
\item Коэффициент ложных выделений как отношение числа найденных фрагментов
текста с~нулевой точностью выделения к~общему числу ссылок на публикацию в~тестовом массиве.
\end{enumerate}

  Если первые две характеристики относятся к~точности и~полноте поиска с~помощью
анализируемого шаблона, то третья характеристика представляет собой частотность ошибок
второго рода, порождаемых этим шаблоном.

\begin{figure*} %fig2
 \vspace*{1pt}
 \begin{center}
 \mbox{%
 \epsfxsize=156.768mm
 \epsfbox{zac-2.eps}
 }
\end{center}
 \vspace*{-9pt}
\Caption{Данные о выделенном фрагменте текста из описания изобретения,
на которое был выдан патент 2152642}
%\end{figure*}
%\begin{figure*} %fig3
 \vspace*{14pt}
 \begin{center}
 \mbox{%
 \epsfxsize=156.768mm
 \epsfbox{zac-3.eps}
 }
\end{center}
 \vspace*{-9pt}
\Caption{Результаты выполнения тестовых заданий на тестирование шаблонов}
\end{figure*}

\begin{table*}[b]\small %tabl4
\vspace*{-6pt}
\begin{center}
\Caption{Частотности связей между индексами МПК и~рубриками ГРНТИ, \%}
\vspace*{2ex}

%\tabcolsep=1.4pt
\begin{tabular}{|c|c|c|c|c|c|c|c|c|c|}
\hline
Код рубрики&
\multicolumn{8}{c|}{Подкласс}&Класс \\
\cline{2-9}
ГРНТИ& G06E&
G06F&
 G06G&
 G06K&
 G06M&
 G06N&
 G06Q&
 G06T&G06\\
\hline
50.00.00&
0&9,58&0,27&4,85&0&0,62&1,01&2,05&18,38\\
%\hline
28.00.00&0&8,40&0,11&4,10&0&0,59&1,05&1,35&15,60\\
%\hline
47.00.00&
0,10&5,16&0,41&3,86&0&0,17&0&0,50&10,20\\
%\hline
45.00.00&0&4,88&0,26&3,49&0&0,19&0&0,35&\hphantom{9}9,17\\
%\hline
20.00.00&0&4,41&0,02&3,25&0&0,12&0&0,05&\hphantom{9}7,85\\
%\hline
30.00.00&0&4,19&0,00&3,23&0&0,11&0&0,04&\hphantom{9}7,57\\
%\hline
29.00.00&0&3,79&0,02&3,17&0&0,01&0&0&\hphantom{9}6,99\\
%\hline
84.00.00&0&3,50&0&3,11&0&0&0,01&0&\hphantom{9}6,62\\
%\hline
27.00.00&0&2,08&0&0,71&0&0,52&1,01&1,30&\hphantom{9}5,62\\
%\hline
Остальные рубрики &0,00&8,69&0,23&1,39&0,01&1,16&0,15&0,37&12,00\\
\hline
\end{tabular}
\end{center}
\end{table*}


  Возможно получение и~еще одной характеристики, определяющей приоритет
использования анализируемого шаблона для поиска ссылок на пуб\-ли\-ка\-ции.
Для этого в~тестовом массиве указаны одновременно тип структуры ссылки и~вид
ци\-ти\-ру\-емой
пуб\-ли\-ка\-ции (книга, статья, доклад  и~пр.). Имея сводную таб\-ли\-цу результатов тестирования
для всех разработанных шаблонов (рис.~3),  можно
 выбрать те шаблоны, которые дают
наилуч-\linebreak шие результаты поиска ссылок. Это позволяет сократить число используемых
шаблонов поиска, выделив базовые, и,~как следствие, увеличить про\-из\-во\-ди\-тель\-ность АИС на
основании результатов тести\-ро\-вания.
{ %\looseness=1

}



  Используемые в~АИС методы определения ННИ, которые
взаимосвязаны с~заданными технологическими областями, предполагают опе\-ри\-рование
классификаторами научной и~на\-уч\-но-тех\-ни\-че\-ской информации
(ГРНТИ, РФФИ и~пр.).\linebreak
Найденные и~структурированные ссылки на цитируемые публикации в~процессе работы
АИС соотносятся с~теми или иными рубриками использу\-емых классификаторов. В~процессе
вычисления значений индикаторов с~помощью макета АИС предполагалось, что руб\-ри\-ки
цитиру\-емых публикаций совпадают с~рубриками изданий, в~которых они были
опубликованы. Это предположение, с~одной стороны, существенно упростило
макетирование. С~другой стороны, оно снизило точность вы\-чис\-лен\-ных значений
индикаторов. Однако, так как основной задачей макета являлась демонстрация реализуемости
методологии вычисления индикаторов взаимосвязей науки и~технологий, то для этапа
макетирования это предположение было допустимым.

  Естественно, что в~промышленном варианте АИС рубрики публикаций и~изданий в~общем
{слу\-чае} совпадать не будут и~должны использоваться именно рубрики публикаций.

Отметим,
что существующие ин\-фор\-ма\-ци\-он\-но-биб\-лио\-теч\-ные сис\-те\-мы предоставляют
возможность использования именно рубрик публикаций. Например, такая возможность есть
в электронных каталогах Государственной публичной на\-уч\-но-тех\-ни\-че\-ской
биб\-лиотеки или
Всероссийского института научной и~технической информации Российской академии наук~\cite{19-zat, 20-zat}.

%\vspace*{-9pt}

\section{Индикаторы взаимосвязей науки и~технологий}

\vspace*{-2pt}

  В проведенном эксперименте с~использованием макета АИС вычислялись значения
следующих двух индикаторов:\\[-14pt]
  \begin{enumerate}[(1)]
\item матрица корреляций между индексами МПК (для класса G06 и~его подклассов)
и~рубриками ННИ ГРНТИ;
\item распределение времени отклика на статью (от момента ее публикации до момента
публикации патента на изобретение, где она цитируется).
\end{enumerate}




  Подробное описание матрицы корреляций представлено в~работе~\cite{9-zat}.
Частотности связей между индексами МПК и~рубриками ГРНТИ,\linebreak вы\-чис\-лен\-ные для всего
массива изобретений по ин\-фор\-ма\-ци\-он\-но-компью\-тер\-ным технологиям,
запатентованных в~РФ в~период 2000--2012~гг.,\linebreak показаны в~ячейках матрицы (табл.~4). Для
ком\-пакт\-ности представления все рубрики ННИ представлены только самым
верхним уровнем ГРНТИ, технологии~--- подклассами класса G06 МПК.

  Таблица~5 отображает распределения цитируемых публикаций и~изобретений для всего
класса G06. Таблица~6 содержит названия индексов ГРНТИ.

  В строках матрицы (см.\ табл.~4) приведены первые девять ННИ
  в~классификации ГРНТИ в~порядке убывания числа связей с~индексами МПК по всему классу~G06 (см.\ последний
столбец таб\-ли\-цы). Эти частотности связей между индексами
 МПК и~руб\-ри\-ка\-ми ГРНТИ были
вычислены впервые в~отечественной на\-уч\-но-тех\-ни\-че\-ской сфере.
 Они определены в~процессе обработки всех статей, цитируемых в~описаниях изобретений по классу~G06,
независимо от того, кем цитируется статья:  экспертами, что обозначается в~описании
изобретения меткой~56, и/или авторами изобретений.
{\looseness=-1

}

\end{multicols}

\begin{table*}\small %tabl5
\begin{center}
\Caption{Распределение изобретений и~статей по подклассам класса G06}
\vspace*{2ex}

\begin{tabular}{|c|p{70mm}|c|c|c|}
\hline
Индекс МПК&\multicolumn{1}{c|}{Название подкласса}&
\tabcolsep=0pt\begin{tabular}{c}Число\\ изобретений\end{tabular}&
\tabcolsep=0pt\begin{tabular}{c}Число\\ статей\end{tabular}&
\tabcolsep=0pt\begin{tabular}{c}Число\\ статей\\ на одно\\ изобретение\end{tabular}\\
\hline
\multicolumn{1}{|c|}{\raisebox{-6pt}[0pt][0pt]{G06C}}&
Механические цифровые вычислительные машины&\hphantom{9}7&\hphantom{9}0&0\\
\hline
\multicolumn{1}{|c|}{\raisebox{-6pt}[0pt][0pt]{G06D}}&
Гидравлические и~пневматические цифровые вычислительные устройства&
\hphantom{9}1&\hphantom{9}0&0\\
\hline
G06E&Оптические вычислительные устройства&52&\hphantom{9}8&0,15\\
\hline
\multicolumn{1}{|c|}{\raisebox{-6pt}[0pt][0pt]{G06F}}&Обработка цифровых данных
с~помощью электрических устройств&3415\hphantom{99}&107\hphantom{9}&0,03\\
\hline
G06G&Аналоговые вычислительные машины\ldots &228\hphantom{9}&14&0,06\\
\hline
G06J&Гибридные вычислительные устройства&\hphantom{9}3&\hphantom{9}0&0\\
\hline
\multicolumn{1}{|c|}{\raisebox{-12pt}[0pt][0pt]{G06K}}&
Распознавание, представление и~воспроизведение данных; манипулирование
носителями информации;
носители информации&681\hphantom{9}&64&0,09\\
\hline
\multicolumn{1}{|c|}{\raisebox{-6pt}[0pt][0pt]{G06M}}&
Счетчики; способы и~устройства для подсчета предметов, не отнесенные к~другим
подклассам&12&\hphantom{9}0&0\\
\hline
\multicolumn{1}{|c|}{\raisebox{-6pt}[0pt][0pt]{G06N}}&Компьютерные системы,
основанные на специфических вычислительных моделях&107\hphantom{9}&\hphantom{9}8&0,07\\
\hline
\multicolumn{1}{|c|}{\raisebox{-12pt}[0pt][0pt]{G06Q}}&
Системы обработки данных или способы, специально предназначенные для
административных,
коммерческих $\langle\ldots\rangle$ целей&417\hphantom{9}&\hphantom{9}5&0,01\\
\hline
\multicolumn{1}{|c|}{\raisebox{-6pt}[0pt][0pt]{G06T}}&
Обработка или генерация данных изображения\ldots&320\hphantom{9}&43&0,13\\
\hline
Всего по классу G06&&5243\hphantom{99}&249\hphantom{9}&0,05\\
\hline
\end{tabular}
\end{center}
\vspace*{6pt}
\end{table*}

\begin{multicols}{2}
%\begin{table*}

\noindent
 %tabl6
{{\tablename~6}\ \ \small{Названия рубрик ГРНТИ из матрицы корреляций}}
%\vspace*{2ex}

\vspace*{2pt}

{\small\begin{center}
\begin{tabular}{|c|l|}
\hline
Код ГРНТИ&\multicolumn{1}{c|}{Название рубрики}\\
\hline
20.00.00&Информатика\\
27.00.00&Математика\\
28.00.00&Кибернетика\\
29.00.00&Физика\\
30.00.00&Механика\\
45.00.00&Электротехника\\
47.00.00&Электроника. Радиотехника\\
50.00.00&Автоматика. Вычислительная техника\\
84.00.00&Стандартизация\\
\hline
\end{tabular}
\end{center}}
%\end{table*}

\vspace*{12pt}





  В рамках проведенного эксперимента вычислялись значения еще одного индикатора~---
распределения времени отклика на статьи (рис.~4). В~процессе вычисления этого
индикатора для каждой пары <<индекс МПК\,--\,руб\-ри\-ка ГРНТИ>> было определено
время отклика как разность между годом публикации патента на изобретение и~годом
публикации статьи, на которую есть ссылка в~описании этого изобретения. Отдельно
отмечались статьи, цитируемые экспертами в~отчетах о патентном поиске. Затем было
построено распределение времени отклика с~учетом авторства ссылок на статьи (эксперты
включили ссылку в~отчет о патентном поиске или авторы изобретения в~его полное
описание).



  Экспериментальные данные позволяют утверж\-дать, что в~патентах на изобретения,
опубликованные в~период 2000--2012~гг., эксперты в~отчетах\linebreak о~поиске и~авторы изобретений
по информационным технологиям наиболее часто цитировали \mbox{статьи}, опубликованные
за~10, 20 и~30~лет до пуб\-ли\-ка\-ции патентов на эти изобретения.
\vspace*{-3pt}

\section{Заключение}


Разработанный макет АИС и~технология его применения впервые в~отечественной
на\-уч\-но-тех\-ни\-че\-ской сфере дали возможность выявить
количественные взаимосвязи отраслей
науки и~ННИ с~заданным видом технологий. С~помощью %\linebreak
макета %\linebreak
 \mbox{были} вычислены значения
индикатора %\linebreak
тематиче\-ских взаимосвязей информационных %\linebreak
технологий, относящихся к~классу
G06 МПК, с~руб\-ри\-ка\-ми ГРНТИ.
Вычисленные значения показывают, что наиболее %\linebreak
 часто в~изобретениях по ин\-фор\-ма\-ци\-он\-но-ком\-пью\-тер\-ным
 технологиям цитируются \mbox{статьи}
по автоматике, вычислительной технике, кибернетике,
элек\-тро\-ни\-ке, радиотехнике, электротехнике\linebreak\vspace*{-12pt}

\pagebreak

\end{multicols}

\noindent
\begin{figure} %fig4
 \vspace*{1pt}
 \begin{center}
 \mbox{%
 \epsfxsize=152.96mm %57.257mm
 \epsfbox{zac-4.eps}
 }
\end{center}
 \vspace*{-9pt}
\Caption{Распределение времени между публикацией статьи и~патента по классу G06:
\textit{1}~--- по полным описаниям изобретений; \textit{2}~--- по спискам документов,
цитируемых в~отчетах об информационном поиске}
\vspace*{-18pt}
\end{figure}

\begin{multicols}{2}




\noindent
и~информатике (см.\ табл.~4). Таким образом, исполь\-зо\-ва\-ние рубрик ГРНТИ дает
в первую очередь прикладной разрез тематических взаимосвязей. Для получения более
полной картины взаимосвязей отраслей науки с~технологиями необходимо также
использовать рубрики фундаментальных наук, например классификатор РФФИ.

  Экспериментальные расчеты, проведенные с~помощью макета АИС, позволяют сделать
вывод о реализуемости методологии определения взаимосвязей отраслей науки и~технологий
с использованием отечественных патентных информационных ресурсов. В~процессе
проведения эксперимента была проведена оценка качества шаблонов поиска, используемых
функциональными подсистемами макета АИС. Были получены численные оценки точности
и полноты поиска ссылок на цитируемые публикации в~полнотекстовых описаниях
изобретений.

  Выявлены наиболее сложные технологические операции поиска ссылок, рубрицирования
публикаций в~автоматическом режиме и~определены подходы к~повышению точности и~полноты выделения ссылок и~рубрицирования.
В~макете АИС преду\-смот\-ре\-на возможность
для экспертного уточнения результатов автоматической рубрикации, выполняемой сейчас с~использованием рубрик изданий, а~не публикаций. Это позволяет уже сегодня использовать
макет АИС для вычисления значений индикаторов взаимосвязей отраслей науки
и~ННИ с~любым заданным видом технологий на ретроспективе
в~12--15~лет и~использовать ретроспективные данные для прогноза изменения значений
этих индикаторов в~краткосрочной перспективе.




{\small\frenchspacing
 {%\baselineskip=10.8pt
 \addcontentsline{toc}{section}{References}
 \begin{thebibliography}{99}
 \bibitem{3-zat} %1
\Au{Schmoch U.} Tracing the knowledge transfer from science to technology as reflected in
patent indicators~// Scientometrics, 1993. Vol.~26. No.\,1. P.~193--211.

 \bibitem{5-zat} %2
\Au{Зацман И.\,М., Веревкин Г.\,Ф.} Информационный мониторинг сферы науки в~задачах
про\-граммно-це\-ле\-во\-го управления~// Системы и~средства информатики, 2006. Вып.~16. С.~164--189.

\bibitem{4-zat} %3
\Au{Зацман И.\,М., Кожунова О.\,С.} Семантический словарь системы информационного
мониторинга в~сфере науки: задачи и~функции~// Системы и~средства информатики, 2007. Вып.~17. С.~124--141.

\bibitem{2-zat} %4
\Au{Архипова М.\,Ю., Зацман И\, М., Шульга~С.\,Ю.} Индикаторы патентной активности
в сфере ин\-фор\-ма\-ци\-он\-но-ком\-му\-ни\-ка\-ци\-он\-ных технологий и~методика их
вычисления~// Экономика, статистика и~информатика. Вестник УМО, 2010. №\,4.
С.~93--104.

\bibitem{6-zat} %5
\Au{Зацман И.\,М., Дурново А.\,А.} Моделирование процессов формирования экспертных
знаний для мониторинга про\-грам\-мно-це\-ле\-вой деятельности~// Информатика и~её
применения, 2011. Т.~5. Вып.~4. С.~84--98.

\bibitem{1-zat} %6
\Au{Минин В.\,А., Зацман И.\,М., Кружков~М.\,Г., Норекян~Т.\,П.} Методологические
основы создания информационных систем для вычисления индикаторов темати\-ческих
взаимосвязей науки и~технологий~// Информатика и~её применения, 2013. Т.~7. Вып.~1.
С.~70--81.
\bibitem{7-zat} %7
\Au{Verbeek А., Debackere K., Luwel~M., Andries~P., Zimmermann~E., Deleus~D.} Linking
science to technology: Using bibliographic references in patents to build linkage schemes~//
Scientometrics, 2002. Vol.~54. No.\,3. P.~399--420.
\bibitem{8-zat}
\Au{Зацман И.\,М., Шубников С.\,К.} Принципы обработки информационных ресурсов
для оценки инновационного потенциала направлений научных исследований~//
Электронные библиотеки: перспективные методы и~технологии, электронные коллекции:
Тр. IX~Всеросс. науч. конф. RCDL'2007.~--- Переславль: Ун-т города
Переславля, 2007. С.~35--44.
\bibitem{9-zat}
\Au{Минин В.\,А., Зацман И.\,М., Хавансков~В.\,А., Шубников~С.\,К.} Индикаторы
тематических взаимосвязей науки и~технологий: от текста к~числам~// Информатика и~её
применения, 2014. Т.~8. Вып.~3. С.~114--125.
\bibitem{10-zat}
Административный регламент исполнения Роспатентом приема заявок на изобретение,
их рас\-смот\-ре\-ния и~экспертизы.~--- М.: ФИПС, 2008. {\sf
http:// www1.fips.ru/wps/wcm/connect/content\_ru/ru/docum\linebreak ents/russian\_laws/order\_minobr/administrative\_regulati\linebreak ons/test\_8}.
\bibitem{11-zat}
Стандарт ВОИС ST.14. Рекомендации по включению ссылок, цитируемых в~патентных
документах. {\sf
http://www.rupto.ru/rupto/nfile/52b8dfc1-1049-11e1-a520-9c8e9921fb2c/03\_14\_01.pdf}.
\bibitem{12-zat}
\Au{Минин В.\,А., Зацман И.\,М., Хавансков~В.\,А., Шубников~С.\,К.} Архитектурные
решения для систем вы\-чис\-ле\-ния индикаторов тематических взаимосвязей науки и~технологий~// Системы и~средства информатики, 2013. Т.~23. №\,2. C.~260--283.
\bibitem{13-zat}
Регулярные выражения в~.NET Framework~// MSDN. Библиотека. {\sf
http://msdn.microsoft.com/ru-ru/library/hs600312.aspx}.

\bibitem{16-zat} %14
\Au{Васильев А., Козлов Д., Самусев~С., Шамина~О.} Извлечение метаинформации и~библиографических ссылок из текстов русскоязычных научных статей~// Электронные
библиотеки: перспективные методы и~технологии, электронные коллекции: Тр.
IX~Всеросс. науч. конф. RCDL'2007.~--- Переславль: Ун-т города Переславля,
2007. С.~175--184.

\bibitem{17-zat} %15
\Au{Васильев А., Козлов Д., Самусев~С., Шамина~О.} Создание электронной библиотеки
русскоязычных научных статей~// Сб. работ стипендиатов гранта
<<Ин\-тер\-нет-ма\-те\-ма\-ти\-ка 2007>>.~--- Екатеринбург: Уральский ун-т,
2007. С.~37--45.

\bibitem{14-zat} %16
\Au{Зацман И.\,М., Хавансков В.\,А., Шубников~С.\,К.} Метод извлечения
библиографической информации из полнотекстовых описаний изобретений~//
Информатика и~её применения, 2013. Т.~7. Вып.~4. С.~52--65.
\bibitem{15-zat} %17
\Au{Хавансков В.\,А., Шубников С.\,К.} Поиск и~рубрицирование ссылок на цитируемые
публикации в~электронных библиотеках полнотекстовых описаний\linebreak изобретений~//
Электронные библиотеки: перспективные методы и~технологии, электронные коллекции:
Тр. XVI~Всеросс. науч. конф. RCDL-2014.~--- Дубна: \mbox{ОИЯИ}, 2014. С.~165--173.


\bibitem{18-zat}
ГОСТ 7.1-2003. Библиографическая запись. Библиографическое описание. Общие
требования и~правила составления. {\sf http://lib.usfeu.ru/index.php/gost-7-1-2003}.
\bibitem{19-zat}
\Au{Сбойчаков К.\,О.} Распределение ключевых слов по руб\-ри\-кам ГРНТИ в~базе данных
Электронного каталога ГПНТБ России~// Библиотеки и~информационные ресурсы в~современном мире науки, культуры, образования и~бизнеса: Тр. XI~Междунар. конф.
<<Крым 2004>>.~--- М., 2004. {\sf http://www.gpntb.ru/win/inter-events/crimea2004/292.pdf}.
\bibitem{20-zat}
\Au{Гиляревский Р.\,С., Шапкин А.\,В., Белоозеров~В.\,Н.} Руб\-ри\-катор как инструмент
информационной навигации.~--- С.-Петербург: Профессия, 2008. 352~с.
 \end{thebibliography}

 }
 }

\end{multicols}

\vspace*{-3pt}

\hfill{\small\textit{Поступила в~редакцию 21.04.15}}

\newpage

%\vspace*{12pt}

%\hrule

%\vspace*{2pt}

%\hrule

\vspace*{-24pt}

\def\tit{INDICATORS FOR THEMATIC LINKAGES BETWEEN SCIENCE
AND~INFORMATION AND~COMPUTER
TECHNOLOGIES AT~THE~BEGINNING OF~THE~XXI~CENTURY}

\def\titkol{Indicators for thematic linkages between science
and~ICT at~the~beginning of~the~XXI~century}

\def\aut{V.\,A. Minin$^1$, I.\,M. Zatsman$^2$, V.\,A. Havanskov$^2$, and S.\,K.~Shubnikov$^2$}

\def\autkol{V.\,A. Minin, I.\,M. Zatsman, V.\,A.~Havanskov, and S.\,K.~Shubnikov}

\titel{\tit}{\aut}{\autkol}{\titkol}

\index{Minin V.\,A.}
\index{Zatsman I.\,M.}
\index{Havanskov V.\,A.}
\index{Shubnikov S.\,K.}

\vspace*{-9pt}

\noindent
$^1$Russian Foundation for Basic Research, 32A~Leninsky
Prosp., Moscow 119991, Russian Federation


\noindent
$^2$Institute of Informatics Problems, Federal Research
Center ``Computer Science and Control'' of the
Russian\linebreak
$\hphantom{^1}$Academy of Sciences, 44-2 Vavilov Str., Moscow 119333,
Russian Federation


\def\leftfootline{\small{\textbf{\thepage}
\hfill INFORMATIKA I EE PRIMENENIYA~--- INFORMATICS AND
APPLICATIONS\ \ \ 2015\ \ \ volume~9\ \ \ issue\ 2}
}%
 \def\rightfootline{\small{INFORMATIKA I EE PRIMENENIYA~---
INFORMATICS AND APPLICATIONS\ \ \ 2015\ \ \ volume~9\ \ \ issue\ 2
\hfill \textbf{\thepage}}}

\vspace*{3pt}


\Abste{Outcomes of experimental evaluation of thematic linkages between
science and information and computer technologies (ICT) are presented.
The indicator values for the linkages are calculated by the testbed of
an analytical information system that was created within the project
of the Russian Foundation for Humanities
``Information system for monitoring and evaluating innovative and technological
potential of the fields of basic research.'' Texts of inventions on the class~G06
(Data processing; Calculations; Account) of the International patent classification
were used. These texts, which were published in 2000--2012 by Rospatent,
are full-text descriptions of inventions in a natural language. Prior to
experimental calculation of indicator values for the linkages, automated
extraction of information on the cited scientific publications was retrieved
from full-text descriptions. A~number of publications was determined for
each field of basic research. Obtained numerical information was used for
quantitative evaluation of thematic science--ICT linkages and gave the
possibility to define an intensity of knowledge transfer from science to
ICT sphere and estimate the linkages by quantitative indicators.}

\KWE{science--technology linkages; information and communication technologies; processing of invention text; regular expressions; classifying; evaluation of indicator values}


\DOI{10.14357/19922264150212}

%\vspace*{-16pt}

\Ack
\noindent
The research was performed at the Institute of Informatics Problems of the
Federal Research
Center ``Computer Science and Control'' of the
Russian Academy of Sciences and was partially supported by the Russian Foundation of Humanities
(grant No.\,12-02-12019в).



%\vspace*{-1pt}

  \begin{multicols}{2}

\renewcommand{\bibname}{\protect\rmfamily References}
%\renewcommand{\bibname}{\large\protect\rm References}



{\small\frenchspacing
 {%\baselineskip=10.8pt
 \addcontentsline{toc}{section}{References}
  \begin{thebibliography}{99}

%  \vspace*{-1pt}




\bibitem{3-zat-1} %1
\Aue{Schmoch, U.} 1993. Tracing the knowledge transfer from science to technology as
reflected in patent indicators. \textit{Scientometrics} 26(1):193--211.

\bibitem{5-zat-1} %2
\Aue{Zatsman, I.\,M., and G.\,F.~Verevkin}. 2006. Informatsionnyy monitoring sfery nauki
v zadachakh programmno-tselevogo upravleniya [Information monitoring in sphere of
science and problems of program-oriented management]. \textit{Sistemy i~Sredstva
Informatiki}~--- \textit{Systems and Means of Informatics} 16:164--189.

\bibitem{4-zat-1} %3
\Aue{Zatsman, I.\,M., and O.\,S.~Kozhunova}. 2007. Seman\-ti\-che\-skiy slovar' sistemy
informatsionnogo monitoringa v sfere nauki: Zadachi i funktsii [The semantic dictionary of
system for information monitoring in science sphere: Tasks and functions].
\textit{Sistemy i Sredstva Informatiki}~---
\textit{Systems and Means of Informatics} 17:124--141.

\bibitem{2-zat-1} %4
\Aue{Arhipova, M.\,Yu., I.\,M. Zatsman, and S.\,Yu.~Shul'ga}. 2010. Indikatory patentnoy
aktivnosti v~sfere informatsionno-kommunikatsionnykh tekhnologiy i~metodika ikh
vychisleniya [Indicators of patent activity in the sphere of information and
communication technologies and technique of their calculation].
\textit{Ekonomika, Statistika i~Informatika.
Vestnik UMO} [Economy, Statistics, and Informatics. Herald of the UMO] 4:93--104.


\bibitem{6-zat-1} %5
\Aue{Zatsman, I.\,M., and A.\,A. Durnovo}. 2011. Modelirovanie protsessov
formirovaniya
ekspertnykh znaniy dlya mo\-ni\-to\-rin\-ga pro\-grammno-tselevoy de\-yatel'\-nosti
[Modeling of
creation processes of expert knowledge for monitoring program-oriented activities].
\textit{Informatika i~ee Primeneniya}~--- \textit{Inform. Appl.}] 5(4):84--98.

\bibitem{1-zat-1} %6
\Aue{Minin, V.\,A., I.\,M. Zatsman, M.\,G.~Kruzhkov, and T.\,P.~Norekyan}. 2013.
Metodologicheskie osnovy sozdaniya informatsionnykh sistem dlya vychisleniya
indikatorov tematicheskikh vzaimosvyazey nauki i~tekhnologiy [Methodological basis for
the creation of information systems for the calculation of indicators
of thematic  linkages between science and technology]. \textit{Informatika i ee Primeneniya}~--- \textit{Inform. Appl.}
7(1):70--81.

\bibitem{7-zat-1} %7
\Aue{Verbeek, A., K. Debackere, M.~Luwel, P.~Andries, E.~Zimmermann, and
D.~Deleus}.
2002. Linking science to technology: Using bibliographic references in patents to build
linkage schemes. \textit{Scientometrics} 54(3):399--420.
\bibitem{8-zat-1}
\Aue{Zatsman, I.\,M., and S.\,K. Shubnikov}. 2007. Printsipy obrabotki informatsionnykh
resursov dlya otsenki in\-no\-va\-tsi\-on\-no\-go
potentsiala napravleniy nauchnykh issledovaniy
[Principles of processing of information resources for assessment of innovative potential of
fields of scientific research]. \textit{Elektronnye biblioteki: Perspektivnye metody
i~tekhnologii, elektronnye kollektsii: Tr. 9-y Vseross. nauch. konf. RCDL'2007}
[Digital Libraries: Perspective Methods and Technologies, Electronic
Collections: 9th
All-Russia Scientific Conference RCDL'2007 Proceedings]. Pereslavl':
Pereslavl University. 35--44.
\bibitem{9-zat-1}
\Aue{Minin, V.\,A., I.\,M. Zatsman, V.\,A.~Havanskov, and S.\,K.~Shubnikov}. 2014.
Indikatory tematicheskikh vzaimosvyazey nauki i tekhnologiy: Ot teksta k~chislam
[Indicators of thematic science--technology linkages: From text to numbers].
\textit{Informatika i ee Primeneniya}~--- \textit{Inform. Appl.}  8(3):114--125.
\bibitem{10-zat-1}
Administrativnyy reglament ispolneniya Rospatentom priema zayavok na izobretenie, ikh
rassmotreniya i~eks\-per\-tizy [Administrative regulations of execution by Rospatent of
demands acceptance for the invention, their considerations and examination].
Available at: {\sf
http:// www1.fips.ru/wps/wcm/connect/content\_ru/ru/docum\linebreak
ents/russian\_laws/order\_minobr/administrative\_regulati\linebreak ons/test\_8/} (accessed April~17, 2015).
\bibitem{11-zat-1}
Standart VOIS ST.14 ``Rekomendatsii po vklyucheniyu ssylok, tsitiruemykh v~patentnykh
dokumentakh'' [WIPO Standard ST.14 ``Recommendations for the Inclusion of References
Cited in Patent Documents'']. Available at:
{\sf
http://www.rupto.ru/rupto/nfile/52b8dfc1-1049-11e1-a520-9c8e9921fb2c/03\_14\_01.pdf}
(accessed April~17, 2015).
\bibitem{12-zat-1}
\Aue{Minin, V.\,A., I.\,M. Zatsman, V.\,A.~Havanskov, and S.\,K.~Shubnikov}. 2013.
Arkhitekturnye resheniya dlya sistem vychisleniya indikatorov tematicheskikh
vza\-imo\-svya\-zey nauki i~tekhnologiy [Information system conceptual decisions
for assessment of linkages between science and technologies]. \textit{Sistemy i Sredstva
Informatiki}~--- \textit{Systems and Means of Informatics} 23(2):260--283.
\bibitem{13-zat-1}
Regulyarnye vyrazheniya v~.NET Framework [Regular expressions within .NET
Framework]. Available at:
{\sf http://msdn.microsoft.com/ru-ru/library/ hs600312.aspx}
(accessed April~17, 2015).

\bibitem{16-zat-1} %14
\Aue{Vasil'ev, A., D. Kozlov, S.~Samusev, and O.~Shamina}. 2007. Izvlechenie
metainformatsii i~bibliograficheskikh ssylok iz tekstov russkoyazychnykh nauchnykh statey
[Extraction of metainformation and bibliographic references from texts of Russian
language scientific articles]. \textit{Elektronnye biblioteki: Perspektivnye metody
i~tekhnologii, elektronnye kollektsii: Tr. 9-y Vseross. nauch. konf. RCDL'2007} [Digital
Libraries: Perspective Methods and Technologies, Electronic Collections: 9th All-Russia
Scientific Conference RCDL'2007 Proceedings]. Pereslavl':
Pereslavl University. 175--184.


\bibitem{17-zat-1} %15
\Aue{Vasil'ev, A., D. Kozlov, S.~Samusev, and O.~Shamina}. 2007. Sozdanie elektronnoy
biblioteki russkoyazychnykh nauchnykh statey [Creation of digital library of Russian
language scientific articles]. \textit{Sb. rabot stipendiatov granta ``Internet-matematika
2007''} [Collection of works of scholars of a~grant ``Internet mathematics 2007''].
Ekaterinburg: Ural University. 37--45.

\bibitem{14-zat-1} %16
\Aue{Zatsman, I.\,M., V.\,A. Havanskov, and S.\,K.~Shubnikov}. 2013. Metod izvlecheniya
bibliograficheskoy infor\-ma\-tsii iz polnotekstovykh opisaniy izobreteniy
[Method of
bibliographic information extraction from full-text descriptions of inventions].
\textit{Informatika i ee Primeneniya}~--- \textit{Inform. Appl.} 7(4):52--65.

\bibitem{15-zat-1} %17
\Aue{Havanskov, V.\,A., and S.\,K.~Shubnikov}. 2013. Poisk i~rub\-ri\-tsirovanie ssylok na
tsitiruemye publikatsii v~elektronnykh bibliotekakh polnotekstovykh opisaniy izobreteniy
[Search and classifying of cited publications in digital libraries
of full-text descriptions of
inventions]. \textit{Elektronnye biblioteki: Perspektivnye metody i~tekhnologii, elektronnye
kollektsii: Tr. 16-y Vseross. nauch. konf.  RCDL'2014} [Digital Libraries:
Perspective
Methods and Technologies, Electronic Collections: 16th All-Russia Scientific Conference
RCDL'2014 Proceedings]. Dubna: Joint Institute for Nuclear Research. 165--173.

\bibitem{18-zat-1}
GOST 7.1-2003. Bibliograficheskaya zapis'. Bibliografi\-che\-skoe opisanie. Obshchie
trebovaniya i~pravila sostav\-le\-niya [Bibliographic record. Bibliographic description. General
requirements and drawing-up rules]. Available at:
{\sf http://lib.usfeu.ru/index.php/gost-7-1-2003} (accessed April~17, 2015).
\bibitem{19-zat-1}
\Aue{Sboychakov, K.\,O.} 2004. Raspredelenie klyuchevykh slov po rubrikam GRNTI
v~baze dannykh Elektronnogo kataloga GPNTB Rossii
[Distribution of keywords on SCSTI
headings in a~database of the Electronic catalog of State Public
Scientific Technical Library of Russia].
\textit{Biblioteki i~informatsionnye resursy v~sovremennom mire nauki, kul'tury,
obrazovaniya i~biznesa: 11-ya Mezhdunar. konf. ``Krym 2004''} [Libraries and
Information
Resources in the Modern World of Science, Culture, Education, and Business: 11th
Conference (International) ``Crimea 2004'']. Moscow.
Available at: {\sf http://www.gpntb.ru/win/inter-events/crimea2004/292.pdf}
(accessed May~27, 2015).
\bibitem{20-zat-1}
\Aue{Gilyarevskiy, R.\,S., A.\,V. Shapkin, and V.\,N.~Beloozerov}. 2008.
\textit{Rubrikator kak instrument informatsionnoy navigatsii} [Subject authority as
instrument of information navigation]. St.\ Petersburg: Professiya. 352~p.
\end{thebibliography}

 }
 }

\end{multicols}

\vspace*{-3pt}

\hfill{\small\textit{Received April 21, 2015}}

%\vspace*{-18pt}

\Contr

\noindent
\textbf{Minin Vladimir A.} (b.\ 1941)~--- Doctor of Science in physics and
mathematics, adviser, Russian Foundation for Basic Research, 32A~Leninsky
Prosp., Moscow 119991, Russian Federation; minin@rfbr.ru

\pagebreak

\noindent
\textbf{Zatsman Igor M.} (b.\ 1952)~---
Doctor of Sciences in technology, Head of Department, Institute of Informatics
Problems, Federal Research Center ``Computer Science and Control'' of the
Russian Academy of Sciences, 44-2 Vavilov Str., Moscow 119333, Russian
Federation; iz\_ipi@a170.ipi.ac.ru

\vspace*{4pt}


\noindent
\textbf{Havanskov Valerij A.} (b.\ 1950)~---
scientist, Institute of Informatics Problems, Federal Research Center ``Computer
Science and Control'' of the Russian Academy of Sciences, 44-2 Vavilov Str.,
Moscow 119333, Russian Federation; havanskov@a170.ipi.ac.ru

\vspace*{4pt}


\noindent
\textbf{Shubnikov Sergej K.} (b.\ 1955)~---
senior scientist, Institute of Informatics Problems, Federal Research Center
``Computer Science and Control'' of the Russian Academy of Sciences, 44-2 Vavilov
Str., Moscow 119333, Russian Federation; sergeysh50@yandex.ru

\label{end\stat}


\renewcommand{\bibname}{\protect\rm Литература}

