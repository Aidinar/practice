\newcommand{\erfc}[1] {\mathrm{Erfc}\,#1 }
\newcommand{\mrm}{\mathrm}

\def\stat{berezin}

\def\tit{ПРИМЕНЕНИЕ УРАВНЕНИЯ ПУГАЧЁВА--СВЕШНИКОВА К~РЕШЕНИЮ ЗАДАЧИ БАКСТЕРА О~ДЛИТЕЛЬНОСТИ ВЫБРОСОВ}

\def\titkol{Применение уравнения Пугачёва--Свешникова к~решению задачи Бакстера о~длительности выбросов}

\def\aut{С.\,В.~Березин$^1$, О.~И.~Заяц$^2$}

\index{Березин С.\,В.}
\index{Заяц О.~И.}

\def\autkol{С.\,В.~Березин, О.~И.~Заяц}

\titel{\tit}{\aut}{\autkol}{\titkol}

%{\renewcommand{\thefootnote}{\fnsymbol{footnote}} \footnotetext[1]
%{Работа выполнена при финансовой
%поддержке РФФИ (проект 15-07-02652).}}


\renewcommand{\thefootnote}{\arabic{footnote}}
\footnotetext[1]{Санкт-Петербургский политехнический университет Петра Великого, servberezin@yandex.ru}
\footnotetext[2]{Санкт-Петербургский политехнический университет Петра Великого, zay.oleg@gmail.com}


\Abst{Вычисляется закон распределения длительности выбросов случайного процесса за подвижную границу, которая перемещается равномерно с заданной скоростью. На выбросы исследуется процесс скошенного броуновского движения (СБД), являющийся одним из важнейших типовых процессов современной стохастической динамики. Этот процесс моделирует динамику броуновской частицы, которая встречает на своем пути полупрозрачный, частично отражающий экран. Задача решается методом уравнения Пу\-га\-чё\-ва-Свеш\-ни\-ко\-ва, которое определенным образом модифицируется с учетом специфических свойств СБД. Указанное уравнение решается аналитически путем сведения к соответствующей краевой задаче Римана. Также описанный метод позволяет получить ряд дополнительных практически интересных характеристик СБД, таких как вероятность невыхода за экран, закон распределения времени первого достижения экрана, моменты времени пребывания за экраном и ряд других.}

\KW{марковский процесс; уравнение Пугачёва; уравнение Пу\-га\-чё\-ва--Свеш\-ни\-ко\-ва; кра\-евая задача Римана; стохастическая механика; скошенное броуновское движение; время пребывания}

\DOI{10.14357/19922264150205}

\vspace*{-6pt}


\vskip 14pt plus 9pt minus 6pt

\thispagestyle{headings}

\begin{multicols}{2}

\label{st\stat}



\section{Введение}

В приложениях теории случайных процессов час\-то приходится иметь дело с~различного \mbox{рода} функционалами от траекторий этих процессов. Важнейшей разновидностью таких функционалов является время пребывания процесса в~заданной области. Оно определяется следующим образом. Пусть задан векторный случайный процесс~$\mathbf {U}(\tau)$, принимающий значения в~$n$-мер\-ном евклидовом пространстве~$\mathbb{R}^n$. Выберем об\-ласть пребывания~$\Omega \subset \mathbb{R}^n$ и~зафиксируем интервал наблюдения~$[0,t]$. Тогда временем пребывания процесса~$\mathbf{U}$ в~$\Omega$ будет называться лебегова мера множества тех моментов наблюдения, для которых~$\mathbf{U}$ не покидал~$\Omega$:
\begin{equation}
  \label{eq1}
  T(t)=\mathrm{mes}\, \{ \tau \in [0, t]|\ \mathbf{ U}(\tau) \in \Omega \}.
\end{equation}

Время пребывания, рассматриваемое как функция от~$t$, представляет собой некоторый случайный процесс. Задача получения закона его распределения (или хотя бы моментов этого распределения) возникает во многих приложениях. Так, в теории надежности область~$\Omega$ отвечает нормативному, штатному режиму работы системы, а~выход из нее~--- переходу к аварийному, форс-ма\-жор\-но\-му режиму. При <<жесткой>> трактовке условий надежности первый же выброс за пределы~$\Omega$ влечет за собой отказ системы, прекращение ее работы. Однако условия надежности можно ставить и в более <<мягкой>> форме, когда отдельные выбросы из~$\Omega$ допускаются, но их суммарная длительность должна быть огра\-ни\-чена.

Функционалы вида~(\ref{eq1}) играют важную роль и~в~стохастической финансовой математике. Платеж\-ные функции ряда нестандартных (экзотических) опционов выражаются через время пребывания курсовой стоимости выше или ниже определенного уровня~\cite{ref1}. Такие же функционалы возникают при решении задач теории массового обслуживания методом диффузионной аппроксимации~\cite{ref2}. В~этом случае приходится вычислять время пребывания винеровского процесса на отрезках единичной длины, ограниченных натуральными числами.

Отметим, что в перечисленных, а также во многих других, подобных им приложениях обычно требуется найти закон распределения~$T$ аналитически, так как этот закон часто служит лишь промежуточным результатом для последующих вычислений. Для процессов~$\mathbf{U}$ общего вида решить такую задачу крайне затруднительно, и приходится ограничивать класс рассматриваемых процессов. В~частности, для гауссовских процессов разработана асимптотическая теория, основанная на предположении, что выбросы за пределы~$\Omega$ являются редкими событиями~\cite{ref3}.

Между тем задачу можно, в принципе, решить точно, если считать~$\mathbf{U}$ непрерывным марковским процессом~\cite{ref4}. Время пребывания~$T$ тогда можно представить в виде решения дифференциального уравнения

\vspace*{2pt}

\noindent
\begin{equation}
  \label{eq2-b}
  \dot T = I_\Omega (\mathbf{U})\,,\quad T(0)=0\,,
\end{equation}

\vspace*{-2pt}

\noindent
где

\noindent
\begin{equation*}
  I_\Omega(\mathbf{x}) =
  \begin{cases}
  1\,, &\ \mathbf{x} \in \Omega\,;\\
  0\,, &\ \mathbf{x} \notin \Omega
\end{cases}
\end{equation*}
обозначает индикаторную функцию множества~$\Omega$. Расширенный процесс~$(\mathbf{U}, T)$ также будет являться марковским, а распределение одной из его компонент~$T$ можно найти, решая соответствующее уравнение Колмогорова.

Последнее обычно оказывается достаточно сложным по своему виду, поэтому применяют и~альтернативный подход, основанный на дискретизации времени, переходе к~схеме случайных блуж\-да\-ний и~предельном переходе, когда шаг блужданий стремится к~нулю~\cite{ref5}.

В настоящее время аналитическое выражение плотности распределения~$f(y; t)$ времени пребывания~$T$ получено лишь для небольшого числа одномерных процессов. При переходе к векторным процессам приходится довольствоваться асимптотическими
и~приближенными формулами~\cite{ref6}.

Наиболее детально изучена задача о времени пребывания одномерного марковского процесса~$U$ на положительной полуоси, когда~(\ref{eq1}) имеет вид:
\begin{equation}
  \label{eq3-b}
  T(t)=\mathrm{mes}\, \{ \tau \in [0, t]|\ U(\tau)>0 \}\,.
\end{equation}
Впервые такую задачу для винеровского процесса решил в 1939~г.\ П.~Леви~\cite{ref7}, который открыл знаменитый закон арксинуса

\noindent
\begin{equation}
  \label{eq4-b}
  f(y;t) = \fr{1}{\pi \sqrt{y(t-y)}} \qquad (0<y<t)\,.
\end{equation}

На этот результат самое пристальное внимание обратил В.~Феллер. Он усовершенствовал вывод~(\ref{eq4-b}), рассмотрел ряд смежных задач, подробно проинтерпретировал этот закон, проверил его экспериментально и~изложил весь круг связанных с~ним вопросов в~известной монографии~\cite{ref5}. С~учетом всего сказанного задачу о~времени пребывания~(\ref{eq3-b}) на положительной полуоси иногда называют задачей Феллера для процесса~$U$.

Помимо случая стандартного винеровского процесса решение этой задачи в настоящее время получено также для винеровского процесса с постоянным сносом. На эту тему имеется целый цикл публикаций, включающий работы
Дж.~Акахори~\cite{ref10}, А.~Дассиоса~\cite{ref9}, М.~Йора~\cite{ref11},  Л.~Такача~\cite{ref8}, А.~Пехт\-ля~\cite{ref12}, ориентированные, в~основном, на приложения к стохастической финансовой математике и~опубликованные в~конце 1990-х~гг.

\columnbreak

В настоящей статье будет рассмотрена иная, более сложная задача, касающаяся времени пребывания
\begin{equation*}
%  \label{eq5}
  T(t)=\mathrm{mes}\, \{ \tau \in [0, t]|\ U(\tau) > b \tau \} \qquad (b > 0)
\end{equation*}
заданного одномерного процесса~$U$ на полуоси~$(b \tau, +\infty)$, граница которой движется равномерно с постоянной скоростью~$b \geqslant 0$. Идея постановки подобной задачи принадлежит Г.~Бакстеру~\cite{ref13}, который впервые решил ее для винеровского процесса еще в~1956~г.

Отметим, что задача Феллера для винеровского процесса с постоянным сносом~[8--12] фактически эквивалентна этой {\it задаче Бакстера} для стандартного винеровского процесса~\cite{ref13}. Винеровский процесс с~постоянным сносом отличается от последнего как раз на линейную функцию времени. Пересечение винеровским процессом с~таким сносом нулевого уровня означает выброс самого винеровского процесса за подвижный линейно рас\-ту\-щий барьер. В~обоих случаях речь идет об одних и~тех же выбросах.

Как было отмечено ранее, обычно применяемые методы решения задач на время пребывания основаны на использовании либо уравнения Колмогорова, либо теории случайных блужданий. Насколько известно авторам, получить с~их помощью решение задачи Феллера для ка\-ких-ли\-бо других процессов~$U$, помимо разобранных в~работах~[7--13], до последнего времени не удавалось, между тем задачи эти интересны и~значимы для многих важных приложений. Поэтому актуальной является разработка альтернативных методов решения.


\section{Уравнение Пугачёва--Свешникова}

Одним из авторов настоящей статьи ранее был предложен метод решения задачи Феллера, основанный на использовании уравнения Пу\-га\-чё\-ва--Свеш\-ни\-ко\-ва~\cite{ref14}. Последнее представляет \mbox{собой} специальный частный случай хорошо известного уравнения Пугачёва для характеристической функции непрерывного марковского процесса~\cite{ref22}. При преобразовании уравнения Пугачёва к форме, предложенной Свешниковым, оно приобретает вид сингулярного интегродифференциального уравнения типа свертки. Вполне естественно, что такое специальное преобразование требует некоторого сужения класса рассматриваемых нелинейных систем.

Сам Свешников первоначально ограничился лишь системами, включающими нелинейности релейного типа~\cite{ref25,ref26}. Свое уравнение Свешников предложил решать приближенно в~качестве альтернативы метода стохастической линеаризации. Впоследст\-вии было установлено, что уравнение Пу\-га\-чё\-ва--Свеш\-ни\-ко\-ва сохраняет силу не только в~классе систем релейного типа, но также и~для некоторых ку\-соч\-но-ли\-ней\-ных систем, в~част\-ности для систем, линейных в~полупространствах~\cite{ref15} и~четвертях пространства~\cite{ref15a}. Кроме того, выяснилось, что это уравнение допускает не только приближенное, но в~целом ряде случаев также и~точное аналитическое решение путем сведения к решению соответствующей краевой задачи типа Римана.

Если исследуемый случайный процесс~$U$ явля\-ется компонентой векторного марковского процесса, описываемого системой ку\-соч\-но-ли\-ней\-ных стохастических дифференциальных уравнений,\linebreak линейной в~полупространствах или четвертьпространствах, то после добавления к этой системе уравнения типа~(\ref{eq2-b}), задающего время пребывания процесса~$U$ на положительной полуоси, приходим к~ку\-соч\-но-ли\-ней\-ной системе именно того вида, который был подробно разобран в~\cite{ref15, ref15a}. Это, в~свою очередь, позволяет решать и~со\-от\-вет\-ст\-ву\-ющую задачу Бакстера. Действительно, эта задача для процесса~$U$ сводится к~решению задачи Феллера для вспомогательного процесса~$V$, задаваемого условием~$V(t)\hm = U(t)\hm - b t$. Если в уравнениях движения исходной задачи выразить~$U$ через~$V$, то вновь полученная система по-прежнему будет принадлежать к~классу ку\-соч\-но-ли\-ней\-ных сис\-тем вида~\cite{ref15, ref15a}, а~значит, допускает применение метода уравнения Пу\-га\-чё\-ва--Свеш\-ни\-кова.

Указанным методом в работе~\cite{ref14} была решена задача Феллера для винеровского процесса, процесса Г.~Уленбека\,--\,Л.~Орнштейна, а~также процесса Т.~Кохи\,--\,Дж.~Дин\-за~\cite{ref16}, играющего важную роль в стохастической механике. Впоследствии независимо от работ~[8--12] методом~\cite{ref14} была решена также и рассмотренная в указанных работах задача Феллера для винеровского процесса с постоянным сносом~\cite{ref17}. В~последнее время также получено решение задачи Феллера для некоторого обобщения процесса Ко\-хи--Дин\-за~\cite{ref15a}.

Далее вначале рассмотрим применение описанного метода к~решению классической задачи Бакстера~\cite{ref13}, касающейся самого винеровского процесса, а~затем обратимся к~более сложному примеру.


\section{Решение классической задачи Бакстера}

Пусть~$W(t)$ обозначает стандартный винеровский процесс. Получим закон распределения суммарного времени, которое для~$0 \hm\leqslant \tau \hm\leqslant t$ проведет процесс~$U(\tau) \hm= a W(\tau)$ выше возрастающего по линейному закону переменного уровня~$b \tau$, где~$a$ и~$b$~--- заданные положительные постоянные.

Положим
\begin{equation*}
%  \label{eq6}
  V(t) = \fr{\sqrt{2}}{a} \left(U(t) - b t\right)\,,\ \mu = \fr{b}{\sqrt{2}a}\,.
\end{equation*}
Тогда задача сведется к исследованию двумерного марковского процесса
\begin{equation}
  \label{eq7}
  \left.
    \begin{aligned}
      &dV = -2 \mu\ dt + \sqrt{2}\ dW\,;\\
      &dT = \fr{1}{2} \left(1 + \mathrm{sign}\,
      V\right)\ dt
    \end{aligned}
  \right\}
\end{equation}
при нулевых начальных условиях
\begin{equation*}
%  \label{eq8}
  V(0) = 0\,;\quad T(0) = 0\,.
\end{equation*}

Вторая компонента~$T$ процесса~(\ref{eq7}) дает длительность выбросов его первой компоненты~$V$ за нулевой уровень, т.\,е.\ совпадает с~дли\-тель\-ностью выбросов исходного процесса~$U$ за линейную границу~$b \tau$.

Тем самым задача Бакстера для процесса~$U$ сведена к задаче Феллера для винеровского процесса~$V$ с постоянным сносом~$-2 \mu$. Последняя задача подробно разобрана в~\cite{ref17}, что позволяет получить следующее обобщение закона арксинуса.

\smallskip

\noindent
\textbf{Утверждение 1.}
\textit{Плотность распределения времени пребывания~$T(t)$ процесса~$U(\tau)$ выше подвижной границы, растущей по линейному закону~$b \tau$, дается формулой
  \begin{multline}
    \label{eq9}
    f(y;t) = \left( \fr{e^{-\mu^2 y}}{\sqrt{\pi y}} - \mu\ \mathrm{Erfc}\left(\mu \sqrt{y}\right) \right)\times{}\\
     {}\times\left( \fr{e^{-\mu^2 (t-y)}}{\sqrt{\pi (t-y)}} + \mu\ \mathrm{Erfc}\left(-\mu \sqrt{t-y}\right) \right),
  \end{multline}
  причем здесь
  \begin{equation*}
%    \label{eq10}
    \erfc{x} = \fr{2}{\sqrt{\pi}} \int \limits_{x}^{+\infty} e^{-s^2}\, ds
  \end{equation*}
  обозначает дополнительную функцию ошибок}.


\smallskip
Вполне естественно, что найденное распределение совпадает с решением Бакстера~\cite{ref13}, а при~$\mu \hm= 0$ оно переходит в закон арксинуса~(\ref{eq4-b}).

Метод получения плотности распределения~(\ref{eq9}), основанный на решении уравнения Пу\-га\-чё\-ва--Свеш\-ни\-ко\-ва, изложен в~\cite{ref17} и~допускает обобщение на ряд процессов более общего вида. Один из примеров такого решения приводится в~следующих разделах статьи и~основывается на понятии СБД, которое по этой причине необходимо разобрать более подробно.

\section{Скошенное броуновское движение}

\vspace*{-3pt}

Скошенное броуновское движение  впервые возникло в статье~\cite{ref18} как обобщение процесса броуновского движения c отражением. Позже было доказано, что СБД описывает движение броуновской частицы при наличии полупроницаемого частично отражающего экрана. Этот процесс не только интересен с математической точки зрения, но и весьма полезен для экономических, биологических, астрономических и разнообразных физических приложений~\cite{ref19}.

Существует несколько подходов к определению СБД~\cite{ref19}, из них для целей настоящей статьи наиболее удобен тот, который использует стохастическое дифференциальное уравнение, включающее локальное время. Рассмотрим одномерное прямолинейное движение частицы. В~точке с~координатой~$x$ расположим полупроницаемый экран. До момента достижения экрана движение частицы будем считать броуновским. Если, достигнув экрана, частица имела положительную скорость, то она свободно пересечет его с~вероятностью~$\beta$, а~с~вероятностью~$1-\beta$ упруго отразится от экрана. При противоположном направлении выхода частицы на экран вероятности пересечения и~отражения соответственно равны~$1-\beta$ и~$\beta$,
т.\,е.\ являются дополнительными для соответствующих вероятностей в~прямом направлении. В~такой ситуации будем называть {\it скошенным броуновским движением} сильное решение стохастического дифференциального уравнения (СДУ)~\cite{ref19}

\vspace*{2pt}

\noindent
\begin{equation}
  \label{eq11}
  dU =(2\beta -1)h^2\ dL^x_U +  h\, dW\,,
\end{equation}

\vspace*{-2pt}

\noindent
где~$W$~---~стандартный винеровский процесс, а~$L^x_U$~--- процесс симметричного локального времени~\cite{ref18} для~$U$ на уровне~$x$, задаваемый формулой:

\noindent
\begin{equation*}
%  \label{eq12}
  L^x_U(t) = \lim \limits_{\varepsilon \to +0} \fr{1}{2\varepsilon}\, \mathrm{mes}\, \{\tau \in [0, t]|\  U(\tau) \in [x- \varepsilon, x+ \varepsilon]\}.
\end{equation*}
Отметим, что данное выше определение СБД корректно в силу доказанной
в~\cite{ref20} теоремы существования и единственности решения уравнения~(\ref{eq11}).

Уравнение~(\ref{eq11}) отличается от соответствующего уравнения для винеровского процесса лишь слагаемым, включающим локальное время. Добавляя подобные же слагаемые и в другие, более сложные уравнения, получаем возможность определить различные обобщения СБД. Одним из простейших таких обобщений является {\it СБД с~постоянным сносом}~\cite{ref21}, получаемое добавлением слагаемого с локальным временем в уравнение броуновского движения с постоянным сносом:

\vspace*{2pt}

\noindent
\begin{equation}
  \label{eq13}
  dU =c\ dt + (2\beta -1)h^2\ dL^x_U + h\,dW\,.
\end{equation}

\noindent
Это слагаемое, как и ранее, моделирует полупроницаемый частично отражающий экран, помещенный в точку~$x$.

При формулировке задачи Бакстера применительно к процессу~(\ref{eq13}) будем теперь считать, что положение частично проницаемого экрана меняется во времени~$t$ по линейному закону~$x(t) \hm= a \hm+ b t$, где~$a > 0$ и~$b \hm\geqslant 0$. Тогда сама задача Бакстера (задача определения длительности выбросов процесса~$U$ за такой подвижный экран) сводится к анализу процесса
\begin{equation}
  \left.
    \begin{array}{rl}
      dU &= c\ dt + (2\beta-1)h^2\ dL^{a + b t}_U + h\ dW\,;\\[6pt]
      dT &=  \fr{1}{2}(1+ \mathrm{sign}\left(U-a-bt\right))\ dt
    \end{array}
  \right\}
    \label{eq14}
\end{equation}
при начальных условиях~$U(0) \hm= 0$ и $T(0)\hm=0$. Необходимо изучить характеристики компоненты~$T$ расширенного процесса~$(U, T)$.

Интересными частными случаями являются~$\beta \hm= 0$ и~1, которые соответствуют упруго отражающему экрану и экра\-ну-ло\-вуш\-ке, пропускающему только частицы, движущиеся с положительной скоростью. В~случае~$\beta \hm= {1}/{2}$ экран вообще не влияет на частицу и~$U$ является винеровским процессом (броуновским движением) с постоянным сносом~\cite{ref17}.

Вводя обозначения
\begin{gather*}
V = \fr{\sqrt{2}}{h}\left(U - a - b t\right)\,;\quad
 \mu  = \fr{b - c}{\sqrt{2}h}\,;\\
 \eta = 2 \beta -1\,;\quad v_0 = -\fr{a \sqrt{2}}{h} < 0\,,
 \end{gather*}
  получим
\begin{equation}
  \left.
    \begin{array}{rl}
      dV &= -2 \mu\ dt + 2\eta\ dL^0_{V} + \sqrt{2}\,dW\,;\\[6pt]
      dT &=  \fr{1}{2}\left(1+ \mathrm{sign}\,V\right)\,dt
    \end{array}
      \right\}
        \label{eq15}
\end{equation}
при начальных условиях
\begin{equation*}
%  \label{eq16}
  V(0) = v_0\,;\quad T(0)=0\,.
\end{equation*}

Система уравнений~(\ref{eq15}) относится к классу ку\-соч\-но-ли\-ней\-ных стохастических систем, линейных в полупространствах, но отличается от стандартных уравнений этого класса~\cite{ref14} наличием локального времени пребывания на границе областей линейности. Несмотря на указанное усложнение постановки задачи, ее решение по-прежнему может быть получено методом, описанным в~\cite{ref14}.

\section{Решение задачи в изображениях}

Повторяя вывод уравнения Пу\-га\-чё\-ва--Свеш\-ни\-ко\-ва~\cite{ref15} с учетом дополнительного слагаемого, содержащего локальное время, получим следующую модификацию указанного уравнения.

\smallskip

\noindent
\textbf{Утверждение 2.}
\textit{Характеристическая функция~$E(z_1, z_2;t)$ системы ординат~$(V(t), T(t))$ процесса}~(\ref{eq15}) \textit{подчиняется уравнению}
  \begin{multline}
    \label{eq17}
    \!\fr{\partial E(z_1,z_2;t)}{\partial t} = -\left( z_1^2 + 2i  \mu  z_1 - \fr{i z_2}{2}\right) E(z_1, z_2; t) +{}\\
     {}+\fr{z_2}{2 \pi}\, \mathrm{v.p.}\ \int \limits_{-\infty}^{+\infty} \fr{E(s, z_2; t)}{s-z_1}\,ds + 2i\eta z_1\ \Phi(z_2; t)
  \end{multline}
  \textit{при начальном условии~$E(z_1, z_2; 0)\hm = e^{i v_0 z_1}$, где функция~$\Phi(z_2; t)$ дается интегралом}
  \begin{equation}
    \label{eq18}
    \Phi(z_2; t) = \fr{1}{2 \pi}\, \mathrm{v.p.}\ \int \limits_{-\infty}^{+\infty} E(s, z_2; t)\, ds\,.
  \end{equation}

Последнее слагаемое в правой части~(\ref{eq17}), отсутствовавшее в~статьях~\cite{ref14,ref15,ref15a,ref17}, появилось из-за наличия в~уравнениях движения~(\ref{eq15}) локального времени. Отметим, что интерпретация интеграла~(\ref{eq18}) в~смысле главного значения по Коши принципиально важна. Как будет показано позже, в~классическом смысле этот интеграл расходится, так как имеет неинтегрируемую особенность на бесконечности.

Метод решения уравнения~(\ref{eq17}) аналогичен~\cite{ref14} и основывается на переходе c помощью формул Ю.\,В.~Сохоцкого к краевой задаче Римана теории функций комплексного переменного. Формулы Сохоцкого имеют вид:
\begin{equation}
\left.
  \begin{array}{rl}
F^+(z_1, z_2; t) - F^-(z_1, z_2; t) &= E(z_1, z_2; t)\,;\\[6pt]
  F^+(z_1, z_2; t) + F^-(z_1, z_2; t) &={}\\
   &\hspace*{-21mm}{}=\displaystyle\fr{1}{\pi i}\, \mathrm{v.p.}\,\displaystyle\int \limits_{-\infty}^{\infty} \fr{E(s, z_2; t)}{s-z_1}\, ds\,.
  \end{array}
  \right\}
    \label{eq19}
\end{equation}
Здесь~$(z_1, z_2) \hm\in \mathbb{R}^2$, а функции~$F^+$ и~$F^-$ являются аналитическими по аргументу~$z_1$ соответственно в верхней и нижней полуплоскостях расширенной комплексной плоскости.

Применяя преобразование Лапласа по~$t$, обозначая аргумент этого преобразования через~$p$, а~изоб\-ра\-же\-ния~--- той же буквой, что и~оригиналы, но с~волной сверху, с~по\-мощью~(\ref{eq19}) перейдем от уравнения~(\ref{eq17}) относительно~$E$ к~краевой задаче Римана относительно
изображений~$\tilde{F}^+$ и~$\tilde{F}^-$ по Лапласу.


\smallskip

\noindent
\textbf{Утверждение 3.}
\textit{Задача Коши для уравнения}~(\ref{eq17}) \textit{эквивалентна краевой задаче Римана, заключающейся в~нахождении пары функций~$\tilde F^\pm$, аналитических по~$z_1$ соответственно в верхней и нижней полуплоскостях и удовлетворяющих при~$\mathrm{Im}\, z_1 \hm= 0$ краевому условию}
  \begin{multline}
    \left(z_1^2 +2i \mu  z_1 + p - i z_2\right) \tilde F^+ = {}\\
    \hspace*{-5mm}{}=\left(z_1^2 +2i \mu  z_1 + p\right) \tilde F^- +
    2 i \eta z_1 \tilde \Phi(z_2;p) + e^{i v_0 z_1},\!\!
        \label{eq20}
  \end{multline}
 \textit{где~$\tilde F^\pm(z_1, z_2; p)$ и~$\tilde \Phi(z_2; p)$ обозначают изображения функций~$F^\pm(z_1, z_2; t)$ и~$\Phi(z_2; t)$ соответственно, причем все изображения существуют при}~$\mathrm{Re}\,{p} > 0$.

 \smallskip


Обратим внимание, что задача~(\ref{eq20}) отличается от аналогичных краевых задач~\cite{ref14,ref15,ref15a,ref17} появлением в краевом условии слагаемого, содержащего~$\tilde \Phi$, связь которого с искомыми краевыми значениями~$\tilde F^\pm$ задается формулами~(\ref{eq18}) и~(\ref{eq19}).

Повторение рассуждений статьи~\cite{ref14} позволяет фактически решить задачу~(\ref{eq20}).

\smallskip

\noindent
\textbf{Утверждение 4.}
  \textit{Краевая задача Римана}~(\ref{eq20}) \textit{эквивалентна обратной задаче подбора функций~$\tilde G_0$ и~$\tilde G_1$, исходя из условий аналитичности по аргументу~$z_1$ правых частей каждого из выражений}
  \begin{equation}
  \left.
  \begin{array}{rl}
     \tilde F^+(z_1, z_2; p) &={}\\[3pt]
      {}&\hspace*{-22mm}=\displaystyle\fr{\tilde G_0(z_2; p) + (\tilde G_1(z_2; p) + i \eta \tilde \Phi(z_2; p))z_1}{z_1^2 +2i \mu   z_1 + p - i z_2}\,;\\[7pt]
     \tilde F^-(z_1, z_2; p) &= {}\\[3pt]
&\hspace*{-24.5mm}{}=\displaystyle\fr{\tilde G_0(z_2; p) \!+\! (\tilde G_1(z_2; p) \!- i \eta \tilde \Phi(z_2; p))z_1 \!-\! e^{i v_0 z_1}}{z_1^2 +2i \mu   z_1 + p}\!    \end{array}\!
    \right\}\!\!\!
        \label{eq21}
  \end{equation}
 \textit{соответственно в верхней и нижней полуплоскостях}.


\smallskip

Обозначая корни знаменателей~(\ref{eq21}) через~$\varkappa^\pm \hm= i( -\mu  \pm \sqrt{ \mu ^2 +p - i z_2})$ и~$\nu^\pm\hm= i( -\mu  \pm \sqrt{ \mu ^2 + p})$, находим условия аналитичности~$\tilde F^\pm$ в виде:
\begin{equation}
\left.
  \begin{array}{l}
\hspace*{-2mm}\tilde G_0\left(z_2; p\right) + \left(\tilde G_1(z_2; p)+i\eta \tilde \Phi(z_2; p)\right)    \varkappa^+ = 0\,;\\[3pt]
\hspace*{-2mm}\tilde G_0\left(z_2; p\right) + \left(\tilde G_1(z_2; p)-i\eta \tilde \Phi(z_2; p)\right)\nu^- = {}\\
\hspace*{52mm}{}= e^{i v_0 \nu^-}\!.
  \end{array}\!
  \right\}\!\!
    \label{eq22}
\end{equation}
Исключение~$\tilde G_0$ из~(\ref{eq21}) и~(\ref{eq22}) дает
\begin{equation}
\left.
  \begin{array}{l}
   \tilde F^+\left(z_1,z_2; p\right) = \fr{\tilde G_1(z_2; p)+i\eta \tilde \Phi(z_2; p)}{z_1-\varkappa^-};\\[6pt]
\tilde F^-\left(z_1,z_2; p\right) = \fr{\tilde G_1(z_2; p)-i\eta \tilde \Phi(z_2; p)}{z_1-\nu^+} + {}\\[6pt]
   \hspace*{31mm}{}+\fr{e^{i v_0 \nu^-} - e^{i v_0 z_1}}{z_1^2 +2i \mu   z_1 + p}\,;\\[6pt]
\left(\nu^- - \varkappa^+\right)\tilde G_1\left(z_2; p\right) - {}\\[6pt]
\hspace*{8mm}{}-i \eta \left(\nu^- + \varkappa^+\right) \tilde \Phi\left(z_2; p\right)= e^{i v_0 \nu^-}.
  \end{array}
\right\}
    \label{eq23}
\end{equation}

Далее с помощью формул~(\ref{eq19}) и~(\ref{eq23}) находим выражение для~$\tilde E$ и подставляем его в~(\ref{eq18}). Разрешая полученные уравнения относительно~$\tilde \Phi$, получаем
\begin{equation*}
%  \label{eq24}
  \tilde \Phi\left(z_2;p\right) = -i \tilde G_1\left(z_2;p\right)\,.
\end{equation*}
Отметим, что интегралы от~$\tilde F^\pm$ по~$z_1$, вообще говоря, расходятся в обычном смысле, но тем не менее существуют в смысле главного значения, а именно это и предполагалось в формуле~(\ref{eq18}). В~итоге приходим к окончательным выражениям для введенных изображений:

\noindent
\begin{equation}
\left.
  \begin{array}{rl}
   \tilde G_1  \left(z_2; p\right) &= \fr{e^{i v_0 \nu^-}}{(1- \eta)\nu^- - (1+\eta)\varkappa^+}\,;\\
   \tilde F^+\left(z_1, z_2; p\right) &= \fr{(1+\eta)\tilde G_1(z_2; p)}{z_1-\varkappa^-}\,;\\
   \tilde F^-\left(z_1, z_2; p\right) &={}\\
    &\hspace*{-15mm}{}=\fr{(1-\eta)\tilde G_1(z_2; p)}{z_1-\nu^+} + \fr{e^{i v_0 \nu^-} - e^{i v_0 z_1}}{z_1^2 +2i \mu   z_1 + p}\,.
  \end{array}
\right\}
    \label{eq25}
\end{equation}
Далее изображение искомой характеристической функции очевидным образом находится по первой формуле Сохоцкого~(\ref{eq19}).
\smallskip

\noindent
\textbf{Утверждение 5.}
\textit{Изображение~$\tilde E$ характеристической функции системы компонент~$(V(t), T(t))$ процесса}~(\ref{eq15}) \textit{выражается в виде}:

\noindent
  \begin{multline}
    \label{eq26}
    \tilde E(z_1, z_2; p) = \left[\fr{1+\eta}{z_1-\varkappa^-} - \fr{1-\eta}{z_1-\nu^+}\right] \tilde G_1(z_2; p) - {}\\
    {}-\fr{e^{i v_0 \nu^-} - e^{i v_0 z_1}}{z_1^2 +2i \mu   z_1 + p}\,,
  \end{multline}
  \textit{где~$\tilde G_1(z_2; p)$ определяется согласно}~(\ref{eq25}).


\smallskip

Полагая здесь~$z_1\hm=0$, получаем изображение маргинальной характеристической функции~$T$.

\smallskip

\noindent
\textbf{Следствие 1.}
  Преобразование Лапласа для характеристической функции искомого времени пребывания~$T$ имеет вид:

  \noindent
  \begin{multline}
    \tilde E_{T}(z_2; p) ={}\\
     \hspace*{-6mm}{}=\fr{(1-\eta)\varkappa^- - (1+\eta)\nu^+}{(1- \eta)\nu^- - (1+\eta)\varkappa^+} \,
     \fr{e^{i v_0 \nu^-}}{\varkappa^- \nu^+} - \fr{e^{i v_0 \nu^-} - 1}{p}\,.\!\!
         \label{eq27}
  \end{multline}

Функция~(\ref{eq27}) представляет собой двукратное преобразование плотности вероятности~$f(y; t)$: по Фурье (по аргументу~$y$) и по Лапласу (по аргументу~$t$). Получению соответствующих оригиналов посвящен следующий раздел.

\vspace*{-4pt}

\section{Решение задачи в оригиналах}

Полученное в предыдущем разделе изображение~(\ref{eq27}) характеристической функции~$\tilde E_T$ за исключением табличного последнего слагаемого, пропорционального~$1/p$, представляет собой произведение экспоненты от аргумента~$\sqrt{\mu^2 + p}$ и~рациональной функции от аргументов~$\sqrt{\mu^2 + p}$ и~$\sqrt{\mu^2 +p - i z_2}$. Такая структура двукратного\linebreak
 преобразования Фурье--Лап\-ла\-са гарантирует по лучение явного выражения оригинала~$f(y;t)$ в~квад-\linebreak\vspace*{-12pt}

 \columnbreak

 \noindent
рату\-рах. Действительно, пользуясь методом двукратного преобразования Лапласа, подробно {описанным} в~\cite{ref17}, общими свойствами преобразования Лапласа и~обобщенной теоремой умножения Эфроса~\cite{ref23}, можно свести исходную задачу обращения к~задаче обращения двукратного преобразования Лапласа, представляющего собой рациональную дробь от аргументов~$q \hm= -i z_2$ и~$p$, которое уже является табличным. Опуская промежуточные выкладки, приходим к следующему конечному результату.

\smallskip

\noindent
\textbf{Утверждение 6.}
\textit{Плотность вероятности времени пребывания~$f(y; t)$ второй компоненты процесса}~(\ref{eq14}) \textit{при~$0\hm<y\hm<t$ дается выражением}:

\noindent
  \begin{multline*}
      f(y; t) = \left[ 1 - \fr{1}{2} \left( e^{-2 v_0 \mu^- } \erfc{ \left( -\fr{v_0 - 2 |\mu| t}{2 \sqrt{t}} \right)} +{}\right.\right.\\
       \left.\left.{}+e^{2 v_0 \mu^+ } \erfc{ \left( -\fr{v_0 + 2 |\mu| t}{2 \sqrt{t}} \right)} \right) \right] \delta(y) +{}\\
      {}+\fr{e^{- \mu^2 t + v_0 \mu}} {4 \pi (y(t-y))^{3/2}}
\left[
%\vphantom{\int\limits_{-v_0}^{+ \infty}}
 \fr{1-\eta}{1 + \eta}
 \int \limits_0^{+ \infty} \int \limits_{-v_0}^{+ \infty} \chi^+(s_1, s_2)\times{}\right.\\
 {}\times e^{\mu  (s_1 + v_0)+(({\mu (3 \eta -1)})/({1 + \eta})) s_2}\ ds_1 ds_2 +{} \\
{} +  \fr{1 + \eta}{1 - \eta}
\int \limits_0^{+ \infty}
\int\limits_{-v_0}^{+ \infty} \chi^-\left(s_1, s_2\right)\times{}\\
\left.{}\times
e^{ (( \mu (3 \eta +1))/( 1-\eta))  \left(s_1 + v_0\right) - \mu  s_2} \, ds_1 ds_2  \vphantom{\int\limits_{-v_0}^{+ \infty}}\right]\,;
\end{multline*}

\vspace*{-12pt}

\noindent
\begin{multline}
     \chi^\pm(s_1, s_2) ={}\\
      {}=\fr{1 \pm \mathrm{sign}\left[(1 + \eta)(s_1+v_0)-(1 - \eta )s_2\right]}{2}\times{}\\
      {}\times s_1 s_2\ e^{- {s_1^2}/(4(t-y)) -{s_2^2}/(4y)}\,,
    \label{eq28}
  \end{multline}
  \textit{где~$\delta(y)$ обозначает дель\-та-функ\-цию Дирака, а~$\mu^\pm \hm= (|\mu| \pm \mu)/2$~--- положительную и отрицательную части числа}~$\mu$.


\smallskip

В частном случае, когда~$\eta \hm= 0$ и~$v_0 \hm= 0$, дельтообразное слагаемое в~(\ref{eq28}) исчезает, а двукратный интеграл преобразуется к виду обычного закона арксинуса~(\ref{eq9}).

Другой важный частный случай, от\-ве\-ча\-ющий~$\mu \hm= 0$ и нулевому начальному условию~$v_0 = 0$, описывает ситуацию, при которой постоянный снос совпадает со скоростью движения экрана. При этом интеграл~(\ref{eq28}) выражается в конечном виде через элементарные функции

\noindent
\begin{multline*}
%  \label{eq29}
  f(y; t) = \fr{(1 - \eta ^2)t}{\pi  \sqrt{y (t - y)} ((1 + \eta)^2t -4 \eta y)} \\
  (0<y<t)\,.
\end{multline*}

\begin{center}  %fig1
\vspace*{-1pt}
\mbox{%
 \epsfxsize=79.301mm
 \epsfbox{ber-1.eps}
 }

\end{center}

%\vspace*{3pt}

\noindent
{{\figurename~1}\ \ \small{График~$f_{\mathcal{T}}(y)$ при различных значениях~$\eta$ ($\mu \hm= 0$): \textit{1}~--- $\eta=-0{,}5$; \textit{2}~--- 0; \textit{3}~--- $\eta\hm=0{,}5$}}



\vspace*{18pt}


\addtocounter{figure}{1}





\noindent
Последняя формула, разумеется, при~$\eta \hm= 0$ переходит в известный закон арксинуса~(\ref{eq4-b}).

Перейдем к безразмерному времени~$\mathcal {T} \hm= T/t$, выраженному в долях интервала наблюдения~$t$, его плотность вероятности~$f_{\mathcal{T}}(y) = t f(ty; t)$ не зависит от~$t$:

\noindent
\begin{multline}
  \label{eq30}
  f_{\mathcal {T}}(y) = \fr{1 - \eta ^2}{\pi  \sqrt{y (1 - y)} ((1 + \eta)^2 -4 \eta y)} \\
  (0<y<1)\,.
\end{multline}
График~$f_{\mathcal T}$ представлен на рис.~1.



Нетрудно убедиться, что доля среднего время, проведенного частицей за движущимся полупроницаемым частично отражающим экраном, дается равенством

\noindent
$$
M[\mathcal{T}] = \fr{\eta + 1}{2} = \beta\,.
 $$
 Этот факт в некотором смысле отражает эргодическую природу рассматриваемого процесса, коль скоро параметр~$\beta$ был определен ранее как вероятность проникновения частицы через экран.

Рассмотрев более общий случай~$v_0 \hm= 0$, $\mu \hm\ne 0$. Получим:

\noindent
\begin{multline*}
\hspace*{-7.21829pt}f(y; t) =\fr{e^{- \mu^2 t}} {4 \pi (y(t-y))^{3/2}}
   \left[ \fr{1-\eta}{1 + \eta} \int\limits_0^{+ \infty} \int\limits_{0}^{+ \infty} \!\!\!\chi^+\left(s_1, s_2\right) \times{}\right.\\
   {}\times e^{\mu s_1 +(({\mu (3 \eta -1)})/({1 + \eta})) s_2}\, ds_1 ds_2 +{} \\
{}  +  \fr{1 + \eta}{1 - \eta} \int\limits_0^{+ \infty}
\int\limits_{0}^{+ \infty}\! \chi^-\left(s_1, s_2\right)\times{}\\
\left.{}\times
e^{ (({ \mu (3 \eta +1) })/(1-\eta)) s_1 - \mu  s_2 }\ ds_1 ds_2  \right].
% \label{eq30a}
\end{multline*}
Соответствующий график для плотности вероятности безразмерного времени~$f_{\mathcal {T}}(y; t)\hm = t f(ty; t)$ представлен на рис.~2.

\columnbreak

\begin{center}  %fig2
\vspace*{-1pt}
\mbox{%
 \epsfxsize=78.901mm
 \epsfbox{ber-2.eps}
 }

\end{center}

\vspace*{1pt}

\noindent
{{\figurename~2}\ \ \small{График~$f_{\mathcal{T}}(y; t)$ при различных значениях~$\eta$ ($\mu \hm= -1, t = 1$): \textit{1}~--- $\eta=-0{,}5$; \textit{2}~--- 0; \textit{3}~--- $\eta\hm=0{,}5$}}



\vspace*{18pt}


\addtocounter{figure}{1}




Обратим внимание, что в отличие от выражения~(\ref{eq30}),
плотность $f_{\mathcal{T}}(y; t)$ теперь уже зависит от~$t$, что очень хорошо иллюстрируется формулой~(\ref{eq35}), которая будет получена в~следующем разделе настоящей статьи.

\section{Дальнейшие результаты}

В этом разделе будет приведен ряд полезных и~качественно интересных результатов, которые удается получить без трудоемкого обращения приведенных выше изображений Лапласа.

Рассмотрим независимый от процесса~(\ref{eq15}) случайный экспоненциальный момент времени~$\tau_\lambda$, где~$\lambda>0$. Нетрудно показать (см., например,~\cite{ref24}), что характеристическая функция~$E_*(z_1, z_2; \lambda)$ случайного вектора~$(V(\tau_\lambda), T(\tau_\lambda))$ легко выражается через~$\tilde E(z_1,z_2; p)$, а именно:
\begin{equation}
  \label{eq31}
  E_*\left(z_1, z_2; \lambda\right) = \lambda \tilde E\left(z_1, z_2; \lambda\right)\,.
\end{equation}
Указанное взаимно однозначное соответствие позволяет, по существу, не делать различия между изображением по Лапласу~$\tilde E(z_1,z_2;p)$ характеристической функции системы величин~$(V(t), T(t))$ и~характеристической функцией~$E_*(z_1,z_2; \lambda)$, вычисленной для величин~$(V(\tau_\lambda), T(\tau_\lambda))$.

Полагая~$z_2\hm=0$ в~(\ref{eq26}), находим:
\begin{multline*}
%  \label{eq32}
    E_{*V}\left(z_1;\lambda\right) = E_*\left(z_1, 0;\lambda\right) = {}\\
    {}=\fr{\lambda}{z_1^2 +2i \mu   z_1 + \lambda} \left( \fr{\eta z_1 e^{v_0( \mu  +\sqrt{ \mu ^2 + \lambda})}}{i (\eta \mu -  \sqrt{\mu^2 + \lambda})}+ e^{i v_0 z_1} \right).
\end{multline*}

Если теперь положить в~(\ref{eq26}) $z_1\hm=z_2\hm=0$, то, воспользовавшись интерпретацией~$\tilde F^\pm$ как односторонних преобразований Фурье~\cite{ref15}, приходим к~паре формул

\noindent
\begin{equation*}
%  \label{eq33}
  P\{ V(\tau_\lambda) \gtrless 0  \} = \pm \lambda \tilde F^\pm(0,0;\lambda)\,,
\end{equation*}
что позволяет сформулировать следующее

\smallskip

\noindent
\textbf{Утверждение 7.}
\textit{Вероятность $P\{ U(\tau_\lambda) \hm> a +b \tau_\lambda\} \hm= P\{ V(\tau_\lambda) \hm> 0 \}$ того, что ордината~$U(\tau_\lambda)$ окажется выше растущей по линейному закону границы, дается формулой}:
  \begin{multline*}
%    \label{eq34}
    P\left\{ U(\tau_\lambda) > a +b \tau_\lambda\right\}={}\\
    {}=
    \fr{(1+\eta) \lambda e^{v_0( \mu +\sqrt{ \mu ^2+\lambda})}}{2(\sqrt{ \mu ^2+\lambda}+ \mu )(\sqrt{ \mu^2+\lambda}-\eta  \mu )}\,.
  \end{multline*}


Отсюда, устремляя параметр~$\lambda$ к нулю, легко находим асимптотику вероятности~$P\{U(t)\hm > a \hm+ b t \}$ при~$t \hm\to \infty$. Полученный результат полностью согласуется со здравым смыслом. Действительно, при~$\eta \hm= -1$ имеем отражающий экран, и~поэтому~$\lim\limits_{t \to \infty}P\{ U(t) \hm> a \hm+ b t \} \hm= 0$, т.\,е.\ частица останется ниже экрана при любом~$\mu$. Пусть теперь~$\eta \hm\ne -1$, тогда будем иметь:
\begin{equation}
  \label{eq35}
  \lim\limits_{t \to \infty} P\{ U(t) > a + b t \}=\begin{cases}
\beta\,, &\ \mu=0\,;\\
1\,, &\ \mu<0\,;\\
0\,, &\ \mu>0\,,
\end{cases}
\end{equation}
что также физически вполне объяснимо: при $\mu\hm<0$ частица рано или поздно обгонит экран, при $\mu\hm>0$ частица никогда его не догонит, а при $\mu\hm=0$ частица окажется выше экрана с вероятностью~$\beta$ (именно так и было введено~$\beta$ выше). Отметим, что последний факт может быть использован при решении задачи идентификации параметра~$\beta$ по опытным данным.

Формула~(\ref{eq27}) позволяет также найти важный в практическом отношении закон распределения времени первого достижения экрана~$\theta$. Действительно, пусть~$\eta \hm\ne -1$. Тогда, анализируя выражение~(\ref{eq27}) для~$\tilde E_T$, легко заметить, что единственным слагаемым, не исчезающим при~$z_2 \hm\to \infty$, является последнее, которое вообще не зависит от~$z_2$. Этому слагаемому отвечает дельтообразная плотность, сосредоточенная в нуле, которая, в~свою очередь, дает вероятность~$P\{ T(\tau_\lambda) \hm= 0\}$. Отсюда получаем:
\begin{multline}
  \label{eq36}
  P \left \{\mathop{\sup}\limits_{\tau \in [0, \tau_\lambda]} (U(\tau)- a - b \tau) < 0 \right\} = P \{ \theta > \tau_\lambda \} ={}\\
   {}=P \{ T(\tau_\lambda) = 0\} = 1-e^{v_0( \mu  + \sqrt{ \mu ^2 +\lambda})},
\end{multline}
и это выражение, естественно, совпадает со случаем броуновского движения с~постоянным сносом~\cite{ref23}. Из физических соображений легко понять, что до момента первого достижения процессом СБД со сносом экрана этот процесс ведет себя в~точности, как аналогичный процесс без экрана, а~значит, время достижения границы вообще не должно зависеть от~$\eta$. Поэтому выражение~(\ref{eq36}) сохраняет силу также и при~$\eta\hm=-1$. Эти рассуждения приводят к~важному выводу.

\smallskip

\noindent
\textbf{Утверждение 8.}
\textit{Вероятность~$P \left \{\sup \limits_{\tau \in [0, \tau_\lambda]} \left(U(\tau) \hm- a \hm-\right.\right.$\linebreak $\left.\left.-\;b \tau\right) \hm< 0 \right\}$ невыхода процесса~$U(t)$ в~течение времени~$\tau_\lambda$ за рас\-ту\-щую по линейному закону границу дается формулой}~(\ref{eq36}).

\smallskip


Учитывая связь~(\ref{eq31}) между изображением характеристической функции~$\tilde E$ и характеристической функцией~$E_*$, после обратного преобразования Лап\-ла\-са получаем аналогичное утверждение для произвольного фиксированного момента времени~$t$.

\smallskip

\noindent
\textbf{Утверждение 9.}
  \textit{Вероятность~$P \left \{\sup \limits_{\tau \in [0, t]} \left(U(\tau) \hm- a\hm +\right.\right.$\linebreak $\left.\left.+\;b \tau\right) < 0 \right\}$ невыхода процесса~$U(t)$ в течение времени~$t$ за растущую по линейному закону границу дается формулой}:
  \begin{multline*}
%    \label{eq37}
     P \left\{\mathop{\sup}\limits_{\tau \in [0, t]} (U(\tau) - a - b t) < 0 \right\} = P \{ \theta > t \} ={}\\
      {}=1 - \fr{1}{2} \left( e^{-2 v_0 \mu^- } \erfc{ \left( -\fr{v_0 - 2 |\mu| t}{2 \sqrt{t}} \right)} +{}\right.\\
\left.       {}+e^{2 v_0 \mu^+ } \erfc{ \left( -\fr{v_0 + 2 |\mu| t}{2 \sqrt{t}} \right)} \right)\,.
  \end{multline*}

Полученное выражение, как легко заметить, фактически уже содержалось в~представлении~(\ref{eq28}).

Найдем также выражение для математического ожидания времени~$T(\tau_\lambda)$. Дифференцируя~$E_*(z_2; \lambda)$ по~$z_2$ и полагая~$z_2\hm=0$, имеем:
\begin{multline*}
%  \label{eq38}
  \tilde m(\lambda) ={\sf M}[T(\tau_\lambda)] ={}\\
   {}=\fr{(1+\eta) e^{v_0( \mu +\sqrt{ \mu ^2+\lambda})}}{2(\sqrt{ \mu ^2+\lambda}+ \mu )(\sqrt{ \mu^2+\lambda}-\eta  \mu )} ={}\\
  {}= \fr{P\{ U(\tau_\lambda) > a +b \tau_\lambda\}}{\lambda}\,.
\end{multline*}
Отсюда с учетом~${\sf M}[\tau_\lambda] \hm= {1}/{\lambda}$ сразу же получаем равенство
\begin{equation}
  \label{eq39}
  \fr{{\sf M}[T(\tau_\lambda)]}
  {{\sf M}[\tau_\lambda]} = P\left\{ U(\tau_\lambda) > a +b \tau_\lambda\right\}\,,
\end{equation}
которое можно интерпретировать как некоторую обобщенную эргодичность.

Как легко проверить, равенство~(\ref{eq39}) можно обобщить на случай векторного процесса~$\mathbf{U}(t)$ и~об\-ласти~$\Omega$ произвольного вида следующим образом.
%\smallskip

\noindent
\textbf{Утверждение 10.}
  \textit{Для произвольного непрерывного процесса $\mathbf{U}(t)$ и произвольной области~$\Omega$ справедливо равенство}:
  \begin{equation*}
%    \label{eq40}
    \fr{{\sf M}[\mathrm{mes}\, \{ \tau \in [0, \tau_\lambda]|\ \mathbf{U}(\tau) \in \Omega \}]}{{\sf M}[\tau_\lambda]} = P\left\{ \mathbf{U}(\tau_\lambda) \in \Omega\right \}\,.
  \end{equation*}


Следует отметить, что все полученные в~данном разделе характеристики, относящиеся к показательному моменту времени~$\tau_\lambda$, в~принципе, могут быть найдены также и~для произвольного фиксированного момента времени~$t$. Соответствующее преобразование Лапласа удается обратить методом, описанным при выводе представления~(\ref{eq28}). Остав\-ляя в~стороне более детальный математический и~физический анализ построенного выше решения, отметим, что изложенный метод позволяет решить и~целый ряд других, более сложных, интересных как с~прикладной, так и~с~вероятностной точки зрения задач.

\section{Заключение}

Метод уравнения Пу\-га\-чё\-ва--Свеш\-ни\-ко\-ва позволяет успешно решать целый ряд задач, ка\-са\-ющих\-ся времени пребывания одномерного процесса на полуоси. Для реализации метода процесс, исследуемый на выбросы, должен описываться сис\-те\-мой стохастических дифференциальных уравнений, которая линейна либо во всем пространстве, либо в~полупространствах. Допускается также линейность и~в~четвертьпространствах, но такая задача более сложна математически. В~предыдущих работах авторов полуось, представляющая область пребывания процесса, имела фиксированную постоянную границу. В~настоящей статье эта граница равномерно движется с заданной скоростью. Кроме того, в~уравнения движения добавлены дополнительные члены, соответствующие локальному времени процесса.

В статье детально изучено распределение\linebreak длительности выбросов типового процесса СБД с~постоянным сносом за равномерно движущуюся границу. Найденное распределение представляет \mbox{собой} некоторое обобщение классического закона арксинуса. Попутно получен ряд дополнительных характеристик движения: вероятность невыхода процесса за подвижную границу, распределение времени ее первого достижения и~т.\,п.

Построенное аналитическое решение пред\-став\-ля\-ет самостоятельный интерес как эталонное точное аналитическое решение типовой задачи статистической динамики. Кроме того, оно может быть использовано для оценки погрешности и~тес\-ти\-ро\-ва\-ния существующих приближенных методов решения уравнения Пугачёва~\cite{ref22}.

{\small\frenchspacing
 {%\baselineskip=10.8pt
 \addcontentsline{toc}{section}{References}
 \begin{thebibliography}{99}
\bibitem{ref1}
\Au{Люу~Ю.\,Д.} Методы и алгоритмы финансовой математики~/ Пер. с англ.~--- М.: Бином, 2007. 752~с. (\Au{Lyuu~Y.\,D.} Financial engineering and computation. --- 1st ed.~--- Cambridge: Cambridge University Press, 2001. 627~p.)
\bibitem{ref2}
\Au{Cohen~J.\,W., Hooghiemstra~G.} Brownian excursion, the $M/M/1$ queue and their occupation times~// Math. Oper. Res., 1981. Vol.~6. No.\,4. P.~608--629.
\bibitem{ref3}
\Au{Berman~S.\,M.} Sojourn and extremes of stochastic processes.~--- Belmond: CRC Press, 1992. 320~p.
\bibitem{ref4}
\Au{Бородин~А.\,Н.} Случайные процессы.~--- СПб.: Лань, 2013. 640~с.
\bibitem{ref5}
\Au{Феллер~В.} Введение в теорию вероятностей и~ее приложения~/ Пер. с англ.~--- М.: Мир, 1967.  Т.~2. 765~с. (\Au{Feller~W.} An introduction to probability theory and its applications.~--- New York, NY, USA: Wiley, 1966.  Vol.~II. 626~p.)
\bibitem{ref6}
\Au{Korpas~A.\,K.} Occupation times of continuous Markov processes. Ph.D. Thesis.~--- Bowling Green: Bowling Green State University, 2006. 92~p.
\bibitem{ref7}
\Au{L$\acute{\mbox{e}}$vy~P.} Sur une
probl{\fontsize{10pt}{10pt}\selectfont\ptb{\!\!\`{e}}}me de Marcinkiewicz~// Comptes rendus Academie sciences Paris, 1939. T.~208. P.~319--321, errata p.~776.

\bibitem{ref10} %8
\Au{Akahori~J.} Some formulae for a new type of path-dependent option~//
Ann. Appl. Probab., 1995. Vol.~5. No.\,2. P.~383--388.

\bibitem{ref9} %9
\Au{Dassios~A.} The distribution of the quantile of a Brownian motion with drift and the pricing of related path-dependent options~// Ann. Appl. Probab., 1995. Vol.~5. No.\,2. P.~389--398.

\bibitem{ref11} %10
\Au{Yor~M.} The distribution of Brownian quantiles~// J.~Appl. Probab., 1995. Vol.~32. P.~405--416.

\bibitem{ref8} %11
\Au{Tak$\acute{\mbox{a}}$cs~L.} On a generalization of the arc-sine law~// Ann. Appl. Probab., 1996. Vol.~6. No.\,3. P.~1035--1040.



\bibitem{ref12}
\Au{Pechtl~A.} Distribution of occupation times of Brownian motion with drift~// J.~Appl. Math. Decision Sci., 1999. Vol.~3. P.~41--62.
\bibitem{ref13}
\Au{Baxter~G.} Wiener process distributions of the arcsine law type~//
Proc. Am. Math. Soc., 1956. Vol.~7. P.~738--741.
\bibitem{ref14}
\Au{Заяц~О.\,И.} Об аналитическом решении задачи Феллера о длительности выбросов~// Труды СПбГТУ. Прикладная математика, 1996. №\,461. С.~92--100.
\bibitem{ref22}
\Au{Пугачёв~В.\,С., Синицын~И.\,Н.} Стохастические дифференциальные системы. Анализ и фильтрация.~--- М.: Наука, 1985 (1-е изд.), 1990 (2-е изд).
\bibitem{ref25}
\Au{Свешников~А.\,А.} Применение теории непрерывных марковских процессов к решению нелинейных задач прикладной гироскопии~// Тр. V~Междунар. конф. по нелинейным колебаниям.~--- Киев: ИМ АН УССР, 1970. T.~3. С.~659--665.
\bibitem{ref26}
\Au{Свешников~А.\,А., Ривкин~С.\,С.} Вероятностные методы в прикладной теории гироскопов.~--- М.: Наука, 1974. 536~с.
\bibitem{ref15}
\Au{Заяц~О.\,И.} Применение уравнения Пу\-га\-чё\-ва--Свеш\-ни\-ко\-ва к~исследованию ку\-соч\-но-ли\-ней\-ных стохастических систем, линейных в полупространствах~// На\-уч\-но-тех\-ни\-че\-ские ведомости СПбГПУ. Физико-математические науки, 2013. №\,4-1. С.~128--142.
\bibitem{ref15a}
\Au{Заяц~О.\,И., Березин~С.\,В.} Применение уравнения Пу\-га\-чё\-ва--Свешникова к исследованию ку\-соч\-но-ли\-ней\-ных стохастических систем, линейных в четвертях пространства~// На\-уч\-но-тех\-ни\-че\-ские СПбГПУ. Информатика. Телекоммуникации. Управление, 2013. №\,6. С.~87--101.
\bibitem{ref16}
\Au{Caughey~T.\,K., Dienes~J.\,K.} Analysis of non-linear first order system with a white noise input~// J.~Appl. Phys., 1961. Vol.~32. No.\,11. P.~2476--2479.
\bibitem{ref17}
\Au{Заяц~О.\,И.} Решение задачи Феллера для винеровского процесса с постоянным сносом~// Труды СПбГТУ. Прикладная математика, 1999. №\,477. С.~67--72.
\bibitem{ref18}
\Au{Ito~K., McKean~H.\,P.} Brownian motion on a half-line~// Illinois J.~Math., 1963. Vol.~7. P.~181--231.
\bibitem{ref19}
\Au{Lejay~A.} On the constructions of the skew Brownian motion~// Probab. Surveys, 2006. Vol.~3. P.~413--466.
\bibitem{ref20}
\Au{Le~Gall~J.-F.} One-dimensional stochastic differential equations involving the local times of the unknown process~// Stochastic analysis and application~/
Eds. A.~Truman, D.\,W.~Williams.~--- Lecture notes in mathematics ser.~---
Berlin--Heidelberg: Springer, 1984. Vol.~1095. P.~51--82.
\bibitem{ref21}
\Au{Appuhamillage~T., Bokil~V., Thomann~E., Waymire~E., Wood~B.} Occupation and local times for skew Brownian motion with applications to dispersion across an interface~// Ann. Appl. Probab., 2011. Vol.~21. No.\,1. P.~183--214.
\bibitem{ref23}
\Au{Лаврентьев~М.\,А., Шабат~Б.\,В.} Методы теории функции комплексного переменного.~--- М.: Наука, 1965. 716~с.
\bibitem{ref24}
\Au{Бородин~А.\,Н., Салминен~П.} Справочник по броуновскому движению: факты и формулы~/ Пер. с англ.~--- СПб.: Лань, 2000. 640~с. (\Au{Borodin~A.\,N., Salminen~P.} Handbook of Brownian motion. Facts and formulae. Probability and its applications. Basel: Birkh$\ddot{\mbox{a}}$user, 1996. 462~p.)
 \end{thebibliography}

 }
 }

\end{multicols}

\vspace*{-3pt}

\hfill{\small\textit{Поступила в~редакцию 02.02.15}}

%\newpage

\vspace*{12pt}

\hrule

\vspace*{2pt}

\hrule

%\vspace*{12pt}

\def\tit{APPLICATION OF~THE~PUGACHEV--SVESHNIKOV EQUATION TO~THE~BAXTER OCCUPATION TIME PROBLEM}

\def\titkol{Application of~the~Pugachev--Sveshnikov equation to~the~Baxter occupation time problem}

\def\aut{S.\,V.~Berezin and O.\,I.~Zayats}

\def\autkol{S.\,V.~Berezin and O.\,I.~Zayats}

\titel{\tit}{\aut}{\autkol}{\titkol}

\index{Berezin S.\,V.}
\index{Zayats O.\,I.}

\vspace*{-9pt}


\noindent
Institute of Applied Mathematics and Mechanics, Peter the Great St.\ Petersburg State Polytechnic University,  29~Politekhnicheskaya Str., St.\ Petersburg 195251, Russian Federation


\def\leftfootline{\small{\textbf{\thepage}
\hfill INFORMATIKA I EE PRIMENENIYA~--- INFORMATICS AND
APPLICATIONS\ \ \ 2015\ \ \ volume~9\ \ \ issue\ 2}
}%
 \def\rightfootline{\small{INFORMATIKA I EE PRIMENENIYA~---
INFORMATICS AND APPLICATIONS\ \ \ 2015\ \ \ volume~9\ \ \ issue\ 2
\hfill \textbf{\thepage}}}

\vspace*{3pt}


\Abste{The Baxter problem, that is, an occupation (sojourn) time above a moving barrier, for a skew Brownian motion is considered. The latter is known as a model of a~semipermeable barrier which permits either movement through it or reflection to the opposite direction with a specified probability. The Pugachev--Sveshnikov equation for a continuous Markov process is used to obtain an analytic solution of the problem. The generic method to solve the Pugachev--Sveshnikov equation for occupation-time type problems involves its reduction to a certain Riemann boundary value problem. This problem is solved, and the occupation time probability density function is derived. Along the way, some additional characteristics of the skew Brownian motion are obtained such as the first passage time, nonexceedance probability, occupation time moments, and some other characteristics.}


\KWE{Markov process; Pugachev equation; Pugachev--Sveshnikov equation; Riemann boundary value problem; stochastic mechanics; skew Brownian motion; occupation time; sojourn time}

\DOI{10.14357/19922264150205}

%\Ack
%\noindent



%\vspace*{3pt}

  \begin{multicols}{2}

\renewcommand{\bibname}{\protect\rmfamily References}
%\renewcommand{\bibname}{\large\protect\rm References}

{\small\frenchspacing
 {%\baselineskip=10.8pt
 \addcontentsline{toc}{section}{References}
 \begin{thebibliography}{99}

\bibitem{ref1-1}
\Aue{Lyuu, Y.\,D.} 2001. \textit{Financial engineering and computation}. 1st ed. Cambridge: Cambridge University Press. 627~p.

\bibitem{ref2-1}
\Aue{Cohen, J.\,W., and G.~Hooghiemstra}. 1981. Brownian excursion, the $M/M/1$ queue and their occupation times. \textit{Math. Oper. Res.} 6(4):608--629.

\bibitem{ref3-1}
\Aue{Berman, S.\,M.} 1992. \textit{Sojourn and extremes of stochastic processes}. Belmond: CRC Press. 320~p.

\bibitem{ref4-1}
\Aue{Borodin, A.\,N.} 2013. \textit{Sluchaynye protsessy} [Stochastic processes]. St.\ Petersburg: Lan'. 640~p.

\bibitem{ref5-1}
\Aue{Feller, W.} 1966. \textit{An introduction to probability theory and its applications}.  New York, NY: Wiley. Vol.~II. 626~p.

\bibitem{ref6-1}
\Aue{Korpas, A.\,K.} 2006. Occupation times of continuous Markov processes. Ph.D. Thesis. Bowling Green: Bowling Green State University. 92~p.

\bibitem{ref7-1}
\Aue{L$\acute{\mbox{e}}$vy, P.} 1939. Sur une \mbox{probl{\fontsize{10pt}{10pt}\selectfont\ptb{\mbox{\hspace*{-3.9pt}\,\`{e}}}}me} de Marcinkiewicz. \textit{Comptes rendus Academie sciences Paris} 208:319--321, errata p. 776.


\bibitem{ref10-1} %8
\Aue{Akahori, J.} 1995. Some formulae for a new type of path-dependent option. \textit{Ann. Appl. Probab}. 5(2):383--388.

\bibitem{ref9-1} %9
\Aue{Dassios, A.} 1995. The distribution of the quantile of a~Brownian motion with drift and the pricing of related path-dependent options. \textit{Ann. Appl. Probab}. 5(2):389--398.

\bibitem{ref11-1} %10
\Aue{Yor, M.} 1995. The distribution of Brownian quantiles. \textit{J.~Appl. Probab}. 32:405--416.

\bibitem{ref8-1} %11
\Aue{Tak$\acute{\mbox{a}}$cs, L}. 1996. On a generalization of the arc-sine law. \textit{Ann. Appl. Probab}. 6(3):1035--1040.

\bibitem{ref12-1} %12
\Aue{Pechtl, A.} 1999. Distribution of occupation times of Brownian motion with drift. \textit{J.~Appl. Math. Decision Sci.} 3:41--62.

\bibitem{ref13-1}
\Aue{Baxter, G.} 1956. Wiener process distributions of the arcsine law type. \textit{Proc. Am. Math. Soc.} 7:738--741.

\bibitem{ref14-1}
\Aue{Zayats, O.\,I.} 1996. Ob analiticheskom reshenii zadachi Fellera o~dlitel'nosti vybrosov [On analytic solution of the Feller problem of upward excursions]. \textit{Trudy \mbox{SPbGTU}. Prikladnaya Matematika} [SPb Polytechnic Univeristy ``Applied Mathematics'' Proceedings] 461:92--100.


\bibitem{ref22-1}
\Aue{Pugachev, V.\,S., and I.\,N.~Sinitsyn}. 1987. \textit{Stochastic differential systems. Analysis and filtering}. Chechester. 549~p.


\bibitem{ref25-1}
\Aue{Sveshnikov, A.\,A.} 1970. Primenenie teorii nepreryvnykh markovskikh protsessov k resheniyu nelineynykh zadach prikladnoy giroskopii [Application of the theory of continuous Markov processes to solution of nonlinear problems of applied gyroscopy]. \textit{Tr. V~Mezhdunar. konf. po nelineynym kolebaniyam} [5th Conference (International) on Nonlinear Vibrations Proceedings]. Kiev. 3:659--665.

\bibitem{ref26-1}
\Aue{Sveshnikov, A.\,A., and S.\,S.~Rivkin}. 1974. \textit{Veroyatnostnye metody v~prikladnoy teorii giroskopov} [Probabilistic methods in the theory of gyroscopy]. Moscow: Nauka. 536~p.

\bibitem{ref15-1}
\Aue{Zayats, O.\,I.} 2013. Primenenie uravneniya Pu\-ga\-che\-va--Svesh\-ni\-ko\-va k~issledovaniyu ku\-soch\-no-li\-ney\-nykh sto\-kha\-sti\-che\-skikh sistem, lineynykh v~poluprostranstvakh [Analysis of piecewise linear stochastic systems in half-spaces by means of the Pugachev--Sveshnikov equation]. \textit{Na\-uch\-no-Tekh\-ni\-che\-skie Vedomosti SPbGPU, Fiziko-Matematicheskie Nauki} [St.\ Petersburg State Polytechnical University
J.~Physics and Mathematics] 4(1):128--142.

\bibitem{ref15a-1}
\Aue{Zayats, O.\,I., and S.\,V.~Berezin}. 2013. Primenenie uravneniya Pu\-ga\-che\-va--Svesh\-ni\-ko\-va k~issledovaniyu \mbox{kusochno}-\mbox{lineynykh} stokhasticheskikh sistem, lineynykh v~chetvertyakh prostranstva [Analysis of piecewise linear stochastic systems in quarter-spaces by means of the Pugachev--Sveshnikov equation]. \textit{Nauchno-tekhnicheskie Vedomosti SPbGPU. Informatika. Telekommunikatsii. Upravlenie} [St. Petersburg State Polytechnical University~J.~Computer Science. Telecommunication and Control Systems] 6:87--101.

\bibitem{ref16-1}
\Aue{Caughey, T.\,K., and J.\,K.~Dienes}. 1961.
Analysis of non-linear first order system with a white noise input.
\textit{J.~Appl. Phys}. 32(11):2476--2479.

\bibitem{ref17-1}
\Aue{Zayats, O.\,I.} 1999. Reshenie zadachi Fellera dlya vinerovskogo protsessa s~postoyannym snosom [Solution of the Feller problem for a~Wiener process with a~constant drift]. \textit{Trudy SPbGTU. Prikladnaya Matematika} [SPb Polytechnic Univeristy ``Applied Mathematics'' Proceedings]. 477:67--72.

\bibitem{ref18-1}
\Aue{Ito, K., and H.\,P. McKean}. 1963. Brownian motion on a~half-line.
\textit{Illinois J.~Math.} 7:181--231.

\bibitem{ref19-1}
\Aue{Lejay, A}.
2006. On the constructions of the skew Brownian motion.
\textit{Probab. Surveys} 3:413--466.

\bibitem{ref20-1}
\Aue{Le Gall, J.-F.} 1984. One-dimensional stochastic differential equations involving the local times of the unknown process. \textit{Stochastic
analysis and application}. Eds.\ A.~Truman and D.\,W.~Williams.
Lecture notes in mathematics ser. Berlin--Heidelberg: Springer.
1095:51--82.

\bibitem{ref21-1}
\Aue{Appuhamillage, T., V.~Bokil, E.~Thomann, E.~Waymire, and B.~Wood}. 2011. Occupation and local times for skew Brownian motion with applications to dispersion across an interface. \textit{ Annals Appl. Pobab}. 21(1):183--214.

\bibitem{ref23-1}
\Aue{Lavrent'ev, M.\,A., and B.\,V.~Shabat}. 1965. \textit{Metody teorii funktsii kompleksnogo peremennogo} [Methods of the theory of functions of a complex variable]. Moscow: Nauka. 716~p.

\bibitem{ref24-1}
\Aue{Borodin, A.\,N., and P.~Salminen}. 1996. \textit{Handbook of Brownian motion. Facts and formulae. Probability and its applications}. Basel: Birkh$\ddot{\mbox{a}}$user. 462~p.
\end{thebibliography}

 }
 }

\end{multicols}

\vspace*{-3pt}

\hfill{\small\textit{Received February 2, 2015}}

%\vspace*{-18pt}



\Contr

\noindent
\textbf{Berezin Sergey V.} (b.\ 1986)~---
senior scientist, Institute of Applied Mathematics and Mechanics, Peter the Great St.\ Petersburg State Polytechnic University,  29~Politekhnicheskaya Str., St.\ Petersburg 195251, Russian Federation; servberezin@yandex.ru

\vspace*{3pt}

\noindent
\textbf{Zayats Oleg I.} (b.\ 1952)~---
 Candidate of Science (PhD) in physics and mathematics, associate professor, Institute of Applied Mathematics and Mechanics, Peter the Great St.\ Petersburg  State Polytechnic University, 29~Politekhnicheskaya Str., St.\ Petersburg 195251, Russian Federation;  zay.oleg@gmail.com


    \label{end\stat}


    \renewcommand{\bibname}{\protect\rm Литература} 