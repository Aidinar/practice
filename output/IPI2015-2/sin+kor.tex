\def\yhxt{\left({\hat X}_t,Y_t, t\right)}
\def\yxtt{\left(X_t,Y_t, t\right)}
\def\yutt{\left(Y_t, {\hat X}_t,t\right)}





\def\stat{sin+kor}

\def\tit{НОРМАЛЬНЫЕ УСЛОВНО-ОПТИМАЛЬНЫЕ
ФИЛЬТРЫ ПУГАЧЁВА ДЛЯ~ДИФФЕРЕНЦИАЛЬНЫХ СТОХАСТИЧЕСКИХ СИСТЕМ,
  ЛИНЕЙНЫХ ОТНОСИТЕЛЬНО СОСТОЯНИЯ$^*$} %\\[-7pt]}

\def\titkol{Нормальные условно-оптимальные
ФП %фильтры Пугачёва
для дифференциальных СтС, %стохастических систем,
  линейных относительно состояния}

\def\aut{И.\,Н.~Синицын$^1$,  Э.\,Р.~Корепанов$^2$}

\def\autkol{И.\,Н.~Синицын,  Э.\,Р.~Корепанов}

\titel{\tit}{\aut}{\autkol}{\titkol}

\index{Синицын И.\,Н.}
\index{Корепанов Э.\,Р.}

{\renewcommand{\thefootnote}{\fnsymbol{footnote}} \footnotetext[1]
{Работа выполнена при  поддержке РФФИ (проект 15-07-02244).}}


\renewcommand{\thefootnote}{\arabic{footnote}}
\footnotetext[1]{Институт проблем информатики Федерального исследовательского
центра <<Информатика и~управление>> Российской академии наук,
sinitsin@dol.ru}
\footnotetext[2]{Институт проблем информатики Федерального исследовательского
центра <<Информатика и~управление>> Российской академии наук,
ekorepanov@ipiran.ru}

\vspace*{-12pt}

\Abst{Рассматриваются вопросы аналитического синтеза нормальных услов\-но-оп\-ти\-маль\-ных фильт\-ров  Пугачёва (НФП) для обработки информации в~дифференциальных негауссовских стохастических системах (СтС), линейных относительно состояния (условия Лип\-це\-ра--Ши\-ря\-ева). Особое внимание уделено синтезу НФП для СтС при условиях Лип\-це\-ра--Ши\-ря\-ева на основе аппроксимации апостериорного распределения нормальным и квазилинейным НФП, основанным на статистической линеаризации нелинейных функций, зависящих от наблюдений. Для СтС высокой размерности  путем выбора структурных функций, отражающих аналитическую природу наблюдаемой системы, можно синтезировать НФП, прос\-ты\-ми в~компьютерной реализации и~для работы в режиме реального времени. Изложенные алгоритмы положены в основу модуля инструментального программного обеспечения <<StS-Filter>>. Даны тестовые примеры. Приводятся некоторые обобщения.}

\vspace*{-8pt}

\KW{метод нормальной аппроксимации (МНА) апостериорной плотности;
метод статистической линеаризации (МСЛ);
нормальный услов\-но-оп\-ти\-маль\-ный фильтр Пугачёва (НФП);
стохастическая система (СтС); дифференциальная СтС;
СтС, линейная относительно состояния;
условия Лип\-це\-ра--Ширяева; фильтр Лип\-це\-ра--Ши\-ря\-ева (ФЛШ)}

\vspace*{-6pt}

\DOI{10.14357/19922264150204}

\vspace*{-6pt}


\vskip 10pt plus 9pt minus 6pt

\thispagestyle{headings}

\begin{multicols}{2}

\label{st\stat}



\section{Введение}

\vspace*{-4pt}

Многие практические задачи обработки информации в~статистических научных исследованиях основаны на использовании теории фильтрации процессов в~СтС, линейных относительно состояния~[1--6]. Первые работы в~этом направлении для гауссовских систем выполнены Липцером и Ширяевым~[7], а~для негауссовских на основе субоптимальной фильтрации~--- Пугачёвым и Синицыным (см.\ обзор~\cite{1-sin}). В~\cite{2-sin, 3-sin} изучены вопросы синтеза и~устойчивости фильтров для линейных СтС с~аддитивными и~мультипликативными негауссовскими шумами.
Статья посвящена вопросам аналитического синтеза и~устойчивости НФП для нелинейных негауссовских СтС, линейных относительно состояния.

\vspace*{-14pt}

\section{Дифференциальные стохастические системы, линейные относительно состояния}

\vspace*{-4pt}

Рассмотрим нелинейную дифференциальную СтС~\cite{1-sin}:

\noindent
    \begin{alignat}{2}
    \dot X_t &=\vrp (X_t, Y_t, t) + \psi (X_t, Y_t, t) V\,, &\ X_{t_0} &= X_0\,;\label{e1-sin}\\
   \dot Y_t &=\vrp_1 (X_t, Y_t, t) + \psi_1 (X_t, Y_t, t) V\,, &\ Y_{t_0}& = Y_0\,,\label{e2-sin}
   \end{alignat}
заданную на многообразиях $\Delta \hm= \Delta^{x,y}$ и $\Delta^V$. Здесь~$X_t$ и $Y_t$~--- векторы состояния и наблюдения размерности~$n_x$ и~$n_y$; $V\hm= \dot W$, $W$~--- векторный процесс с независимыми приращениями, состояний из винеровской $W_0(t)$ и~пуассоновской частей:
    \begin{equation}
     \left.
     \begin{array}{rl}
     W&= W_0 (t) +\displaystyle \iii_{R_0^q} c(\rho) P^0 (t, d\rho)\,;\\[6pt]
     \nu^W &= \nu^{W_0} + \displaystyle \iii_{R_0^q} c(\rho) c(\rho)^{\mathrm{T}} \nu_P (t, \rho)\, d\rho\,,
     \end{array}
     \right\}
     \label{e3-sin}
     \end{equation}
где $c=c(\rho)$~--- векторная функция (той же размерности~$q$, что и~$W$) аргумента~$\rho$, а~интеграл при любом $t\hm\ge t_0$ представляет собой стохастический интеграл по центрированной пуассоновской мере  $P^0 (t, A)$, независимой от~$W_0$  и~име\-ющей независимые значения на попарно непересекающихся множествах;  $A$~--- борелевское множество пространства $R_0^q$ с выколотым началом~$0$; $\nu^W$, $\nu^{W_0}$ и~$\nu_P$~--- интенсивности $W$, $W_0$ и~$P^0$:
$\vrp\hm=\vrp (X_t, Y_t, t)$, $\psi\hm=\psi (X_t, Y_t, t)$, $\vrp_1\hm=\vrp_1 (X_t, Y_t, t)$ и~$\psi_1\hm=\psi_1 (X_t, Y_t, t)$~--- известные функции раз\-мер\-ности $(n_2\times 1)$, $(n_x\times n_v)$, $(n_y \times 1)$ и~$(n_y\times n_v)$, удовлетворяющие следующим условиям Лип\-це\-ра--Ши\-ря\-ева~[7]:
\begin{itemize}
\item  функции $\vrp$ и $\vrp_1$ линейны относительно состояния~$X_t$:
    \begin{align*}
    \vrp (X_t, Y_t, t) &= a_1 (Y_t, t) X_t + a_0 (Y_t, t)\,;\\
    \vrp_1 (X_t, Y_t, t) &= b_1 (Y_t, t) X_t + b_0 (Y_t, t)\,;
\end{align*}

\item функции $\psi$ и $\psi_1$  не зависят от состояния~$X_t$:
    $$
    \psi (X_t, Y_t, t) = \psi (Y_t, t) \,;\enskip \psi_1 (X_t, Y_t, t) = \psi_1  (Y_t, t)\,.
    $$
\end{itemize}

Предполагается, что уравнения СтС~(\ref{e1-sin}) и~(\ref{e2-sin}) понимаются в~смысле Ито и~имеют решение в среднем квадратическом (с.к.)~\cite{1-sin}.
Систему~(\ref{e1-sin})--(\ref{e2-sin}) будем называть гауссовской, если  $V\hm=\dot W_0$, а~$X_0$ и~$Y_0$~--- гауссовские.
Важный частный случай~(\ref{e1-sin}) и~(\ref{e2-sin}) составляют уравнения с~аддитивными шумами, когда
    \begin{equation*}
    \bar\psi\left(Y_t, t\right) = \psi_0 (t)\,;\enskip \bar\psi_1 \left(Y_t, t\right) = \psi_{10} (t)\,.
    %\label{e4-sin}
    \end{equation*}

\noindent
\textbf{Замечание~1.} Для случая, когда в уравнения~(\ref{e1-sin}), (\ref{e2-sin}) входят независимые белые шумы~$V_1$ и~$V_2$, следует принять
    \begin{equation}
    \left.
    \begin{array}{rlrl}
    V&= \lk V_1^{\mathrm{T}}\, V_2^{\mathrm{T}}\rk^{\mathrm{T}}\,,\enskip & \nu&=\begin{bmatrix}
    \nu_1&0\\
    0&\nu_2\end{bmatrix}\,;\\[12pt]
    \psi V &=\psi' V_1\,;\enskip  &\psi_1 V &= \psi_1' V_2\,.
    \end{array}
    \right\}
    \label{e5-sin}
    \end{equation}

\noindent
\textbf{Замечание~2.}
 Для случая, когда уравнения СтС линейны и содержат гауссовские и~негауссовские аддитивные и мультипликативные шумы в~уравнениях состояния  и~наблюдения, соответствующие уравнения приведены в~\cite{2-sin, 3-sin}.



\section{Дифференциальные фильтры Липцера--Ширяева}

Для гауссовской СтС~(\ref{e1-sin})--(\ref{e2-sin}) при условиях Лип\-це\-ра--Ши\-ря\-ева известны следующие точные уравнения нелинейной фильтрации по критерию минимума с.к.\ ошибки~\cite{1-sin, 7-sin}:
 \begin{multline}
 {\dot{\hat X}}_t= \left[a_1 \left(Y_t,t\right) \hat X_t + a_0 \left(Y_t,t\right)\right] + \left[R_tb_1 (Y_t,t)^{\mathrm{T}} +{}\right.\\
\left. {}+\left(\psi\nu_0\psi_1^{\mathrm{T}}\right)
    (Y_t,t)\right] \left(\psi_1\nu_0\psi_1^{\mathrm{T}}\right)^{-1} \left(Y_t,t\right)
\left\{ \dot Y_t -{}\right.\\
{}-\left.\left[ b_1\left(Y_t,t\right) \hat X_t +b_0\left(Y_t,t\right)\right] \right\},\enskip
    \hat X_{t_0} = \hat X_0;\label{e6-sin}
    \end{multline}

    \vspace*{-12pt}

    \noindent
    \begin{multline}
       \dot{R}_t =  a_1\left(Y_t,t\right) R_t + R_t a_1 \left(Y_t,t\right)^{\mathrm{T}} +\left(\psi\nu_0\psi^{\mathrm{T}}\right) \left(Y_t,t\right) -{}\\
       {}-
\left[ R_t b_1 \left(Y_t,t\right)^{\mathrm{T}}
+\left(\psi\nu_0\psi_1^{\mathrm{T}}\right) \left(Y_t,t\right)\right] \left(\psi_1\nu_0\psi_1^{\mathrm{T}}\right)^{-1} \times{}\\
{}\times \left(Y_t,t\right)
 \left[ b_1 \left(Y_t,t\right) R_t+ \left(\psi_1 \nu_0 \psi^{\mathrm{T}}\right)
    \left(Y_t,t\right)\right]\,,\\
     R_{t_0} = R_0\,,\label{e7-sin}
    \end{multline}
где $\hat{X}_t$ --- с.к.\ оценка; $R_t$~--- ковариационная матрица ошибки фильтрации ($X_t\hm- \hat{X}_t$).


Как и в случае линейной фильтрации при аддитивных шумах~\cite{1-sin}, уравнения дифференциального фильтра Лип\-це\-ра--Ширяева (ФЛШ)~\cite{6-sin, 7-sin} пред\-ставляют собой замкнутую систему
уравнений, определяющую  $\hat{X}_t$ и~$R_t$. Поэтому с.к.\  оптимальную
оценку~$\hat{X}$ вектора состояния системы~$X_t$ и его
апостериорную ковариационную матрицу~$R_t$, характеризующую
точность с.к.\ оптимальной оценки ~$\hat{X}_t$, можно вычислять по
мере получения результатов наблюдений совместным интегрированием
уравнений~(\ref{e6-sin}) и~(\ref{e7-sin}). Однако в~противоположность линейной фильтрации
для ФЛШ нельзя вычислить~$R_t$ заранее, до
получения результатов наблюдений, так как от последних зависят
коэффициенты уравнения~(\ref{e2-sin}). Поэтому ФЛШ в~данном случае должен выполнять интегрирование обоих уравнений~(\ref{e6-sin}) и~(\ref{e7-sin}). Это приводит к~существенному повышению
порядка оптимального фильтра. Если линейный фильтр  всегда
описывается уравнениями  порядка~$n_x$, то в~рассматриваемом более общем случае
с.к.\ оптимальный фильтр описывается уравнениями по\-рядка
$$
Q_{\mathrm{ЛШ}}=n_x+\fr{n_x(n_x+1)}{2}= \fr{n_x(n_x+3)}{2}\,.
$$

Таким образом, имеет место следующее утверж\-дение.

\smallskip

\noindent
\textbf{Теорема~1.} \textit{Пусть в гауссовской системе}~(\ref{e1-sin})--(\ref{e2-sin}) \textit{при условиях Лип\-це\-ра--Ши\-ря\-ева диффузионная матрица  $\si_1\hm= \si_1 (Y_t, t) \hm=\psi_1\nu_0\psi_1^{\mathrm{T}} (Y_t, t)$ не вырождена. Тогда с.к.\  оптимальный фильтр определяется уравнением}~(\ref{e6-sin}), \textit{причем его точность оценивается согласно}~(\ref{e7-sin}).

\smallskip

\noindent
\textbf{Замечание~3.}
 Очевидно, что ФЛШ будет совпадать с~обобщенным фильтром Кал\-ма\-на--Бью\-си, фильтрами второго порядка, гауссовыми фильтрами~\cite{1-sin, 6-sin, 8-sin}, если записать условиях Лип\-це\-ра--Ши\-ря\-ева в~виде:
\begin{equation}
    \left.
    \begin{array}{rl}
   \hspace*{-1.5mm}a_1\left(Y_t, t\right) X_t + a_0 \left(Y_t, t\right) &= {}\\[5pt]
    &\hspace*{-37mm}{}=\bar \vrp \left(\hat{X}_t,Y_t, t\right) + \bar\vrp_x \left(\hat{X}_t,Y_t, t\right)^{\mathrm{T}} \left(X_t-\hat{X}_t\right);\\[7pt]
      \hspace*{-1.5mm}b_1\left(Y_t, t\right) X_t + b_0 \left(Y_t, t\right) &={}\\[5pt]
   &\hspace*{-37mm}{}=\bar\vrp_1 \left(\hat{X}_t,Y_t, t\right)+\bar\vrp_{1x}\left(\hat{X}_t,Y_t, t\right)^{\mathrm{T}} \left(X_t-\hat{X}_t\right).
   \end{array}\!
   \right\}\!\!
   \label{e8-sin}
   \end{equation}
Здесь
\begin{align*}
\bar\vrp_x(\hat{X}_t,Y_t, t) &= a_1(Y_t, t)\,;\\
 \bar\vrp_{1x}(\hat{X}_t,Y_t, t) &=b_1(Y_t,t)\,;\\
 \bar\vrp &=  a_1(Y_t, t) \hat{X}_t+a_0(Y_t, t)\,;\\
  \bar\vrp_1 &=  b_1 (Y_t, t) \hat{X}_t+b_0 (Y_t, t)
  \end{align*}
и~необходимо учесть, что $\bar\vrp_x$ и $\bar \vrp_{1x}$ не зависят от~$X_t$.

\section{Нормальные фильтры Пугачёва для~дифференциальных стохастических систем, линейных относительно состояния}

Следуя~\cite{1-sin}, будем искать фильтр для оценки $\hat{X}_t$ в виде следующего уравнения:
\begin{multline}
\hspace*{-2mm}d\hat{X} ={}\\
\hspace*{-1mm}{}=\alp_t \xi \yhxt dt + \beta_t\eta \yhxt dY_t +
    \gamma_t\, dt,\!\!\label{e9-sin}
    \end{multline}
где  $\xi =\xi \yhxt$ и $\eta\hm=\eta\yhxt$~--- некоторые функции
текущих значений наблюдаемого процесса~$Y_t$, оценки $\hat{X}_t$ и
времени~$t$; $\alp_t$, $\beta_t$ и~$\gamma_t$~--- некоторые
функции времени.

Если бы коэффициенты  $\alp_t$, $\beta_t$ и~$\gamma_t$ в~(\ref{e9-sin})
были известными функциями времени, то уравнение~(\ref{e9-sin}) определило
бы фильтр того же  порядка~${n_x}$, что и~уравнение~(\ref{e1-sin}), описывающее поведение системы. Поэтому, естественно,
возникает мысль попытаться непосредственно определить коэффициенты
$\alp_t$, $\beta_t$ и~$\gamma_t$ в~уравнении~(\ref{e9-sin}) как функции
времени из условия минимума с.к.\ ошибки  $\mathrm{M}\lv
\hat{X} \hm- X_t\rrv^2 =\min$ при всех  $t\hm>t_0$. Это приводит к~теории
услов\-но-оп\-ти\-маль\-но\-го фильт\-ра Пугачева (ФП), когда в~уравнение
фильт\-ра задаются заранее и~оптимизируются только коэффициенты
этого уравнения. Итак, мы приходим к~идее
нахождения оптимального фильт\-ра
в~некотором классе допустимых фильт\-ров,
определяемом условием, что поведение фильт\-ра описывается
дифференциальным уравнением заданного порядка и~заданной формы.
Таким образом, мы отказываемся от абсолютной оптимизации
и~ограничиваемся условной оптимизацией в~заданном ограниченном классе фильтров.

Определив класс допустимых фильтров, следует решить вопрос о~том,
какой фильтр в этом классе считается оптимальным. Следуя Пугачёву~\cite{1-sin}, будем считать оптимальным такой фильтр, который дает в~известном смысле наилучшую оценку при всех $t\hm>t_0$. Иными словами, задача оптимизации фильтра при всех
$t\hm>t_0$ является  задачей многокритериальной оптимизации. Такие
задачи, как правило, не имеют решения. Фильтр Кал\-ма\-на--Бью\-си, дающий
оптимальную линейную оценку состояния линейной системы в каждый
момент $t\hm>t_0$, является исключением~\cite{1-sin}. Значит, надо
определить такую оптимальность фильтра, при которой возможно решение
задачи.
Будем считать оптимальным
такой допустимый фильтр, который на каждом бесконечно малом
интервале времени совершает оптимальный переход из того со\-сто\-яния,
в~котором он был в~начале этого интервала, в~новое со\-сто\-яние.
Такой допустимый фильтр будем называть   услов\-но-оп\-ти\-маль\-ным. Тогда задачи фильтрации сведутся к~нахождению оптимальных значений
 $\alp_t$, $\beta_t$ и~$\gamma_t$ в~(\ref{e8-sin}) в~любой момент  $t\hm\ge t_0$, обеспечивающих минимум
с.к.\ ошибки фильтрации $\mathrm{M}\lv \hat X_s \hm-
X_s\rrv^2$  в~бесконечно близкий будущий момент  $s\hm> t$, $s\hm\to t$.

Отметим, что ФП обладает тем свойством, что в~данном
классе допустимых фильтров не существует фильтра, который при
данном начальном распределении  $Y_t$, $X_t$ и~$\hat{X}_t$ в~момент~$t_0$
был бы лучше услов\-но-оп\-ти\-маль\-но\-го при всех  $t\hm>t_0$. Это
значит, по терминологии теории многокритериальной оптимизации, что
ФП представляет собой один из множества
допустимых фильтров~---  оптимальный по Парето~\cite{1-sin}.
Общая теория ФП по с.к.\ критерию развита для уравнений (1), (2)
и подробно изложена в~\cite{1-sin}.
Теория ФП
обладает двумя несомненными преимуществами по сравнению с методами
субоптимальной фильтрации. Во-пер\-вых, она позволяет
получать фильтры более низкого порядка и,~следовательно, более
простые в реализации. Во-вто\-рых, она дает возможность получать
фильтры не меньшей, а~при желании даже большей точности, чем
фильтры, даваемые методами субоптимальной нелинейной
фильтрации~\cite{1-sin, 6-sin, 8-sin}.

Применяя теорию ФП~\cite{1-sin} к~нормальным процессам в~гауссовской СтС~(\ref{e1-sin}), (\ref{e2-sin}) при условиях Лип\-це\-ра--Ши\-ря\-ева и~$V\hm=V_0$, придем к~НФП вида~(\ref{e9-sin}). Входящие в~(\ref{e9-sin}) коэффициенты определяются следующими уравнениями:
    \begin{gather}
    \alp_t m_1 + \beta_t m_2 + \gamma_t=m_0\,;\label{e10-sin}\\
   m_0 = \mathrm{M}^N_\Delta [\varphi]\,;\enskip m_1 = \mathrm{M}^N_\Delta  [\xi]\,;\enskip m_2 = \mathrm{M}^N_\Delta  [\eta]\,;\label{e11-sin}\\
\beta_t =\kappa_{02} \kappa_{22}^{-1}\,;\label{e12-sin}\\
  \kappa_{02} ={}\hspace*{63mm}\notag\\
   {}=\mathrm{M}^N_\Delta \left [\left(X_t - \hat{X}_t\right) \varphi_1^{\mathrm{T}} \eta^{\mathrm{T}}\right]+\mathrm{M}^N_\Delta \left[\psi  \nu_0\psi_1^{\mathrm{T}} \eta^{\mathrm{T}} \right]\,;\label{e12a-sin}\\
   \kappa_{22} = \mathrm{M}^N_\Delta  \eta \psi_1\nu(t) \psi_1^{\mathrm{T}} \eta^{\mathrm{T}}\,.\label{e13-sin}
   \end{gather}

\vspace*{-12pt}

   \noindent
   \begin{multline}
  \alp_t\kappa_{11} + \mathrm{M}^N_\Delta \left[ \left(\hat{X}_t-X_t\right)\left(\xi^{\mathrm{T}} \alp_t^{\mathrm{T}} +\gamma_t^{\mathrm{T}}\right)
\fr{\partial \xi^{\mathrm{T}}}{\partial x}\right]={}\\
{}=\kappa_{01}'-\beta_t\kappa_{21}'\,;\label{e14-sin}
\end{multline}
\begin{equation}
\kappa_{21}' = \mathrm{M}^N_\Delta  \left[ \eta\varphi_1-m_2\right] \xi\,;\label{e15-sin}
\end{equation}

\noindent
\begin{multline*}
   \kappa_{01}' =\kappa_{01} + \mathrm{M}^N_\Delta  \left[ \left(X_t-\hat{X}_t\right)  \fr{\partial \xi^{\mathrm{T}}}{\partial t} \right]+{}\\
   {}+
     \mathrm{M}^N_\Delta \left\{ \left(X_t-\hat{X}_t\right) \varphi_1^{\mathrm{T}} +{}\right.\\
\left.{}+\psi\nu_0\psi_1^{\mathrm{T}} - \beta_t\eta\psi_1\nu_0\psi_1^{\mathrm{T}}
\vphantom{\left(X_t-\hat{X}_t\right)}
\right\} \left(
    \fr{\partial}{\partial y}+\eta^{\mathrm{T}}\beta_t^{\mathrm{T}} \fr{\partial } {\partial x}\right)\xi^{\mathrm{T}}+{}
    \end{multline*}

    \noindent
    \begin{multline}
{}+\fr{1}{2}\, \mathrm{M}^N_\Delta \left(X_t-\hat{X}_t\right) \left\{ \mathrm{tr} \left[ \psi_1\nu_0\psi_1^{\mathrm{T}}
    \left( \fr{\partial}{\partial y}+{}\right.\right.\right.\\
    {}\left.\left.+2 \eta^{\mathrm{T}}\beta_t^{\mathrm{T}} \fr{\partial}{\partial x}\right)
\fr{\partial^{\mathrm{T}}}{\partial y}\right]+{}\\
\left.{}+\mathrm{tr}\left[ \beta_t\eta\psi_1\nu_0\psi_1^{\mathrm{T}}\eta^{\mathrm{T}}\beta_t^{\mathrm{T}} \fr{\partial}{\partial x}\,
\fr{\partial^{\mathrm{T}}}{\partial x}\right]\right\} \xi^{\mathrm{T}} \,;\label{e16-sin}
\end{multline}

%\vspace*{-24pt}

\noindent
\begin{equation}
\left.
\begin{array}{c}
 \kappa_{11} = \mathrm{M}^N_\Delta \left[ \xi - m_1\right] \xi^{\mathrm{T}}\,;\\[6pt]
 \kappa_{21} = \mathrm{M}^N_\Delta \left[ \varphi_1  - m_2\right] \xi^{\mathrm{T}} \,;\\[6pt]
 \kappa_{01} = \mathrm{M}^N_\Delta \left[ \varphi  - m_0\right] \xi^{\mathrm{T}} \,.
 \end{array}
 \right\}
 \label{e17-sin}
 \end{equation}

Точность НФП определяется уравнением:
      \begin{multline}
      \dot R_t = \mathrm{M}^N_\Delta \left[\left( X_t -\hat{X}_t\right) \varphi \yxtt^{\mathrm{T}} +{}\right.\\
      {}+\varphi\yxtt (X_t^{\mathrm{T}} -\hat{X}_t^{\mathrm{T}})-\beta_t \eta \yutt \times{}\\
      {}\times \psi_1 \yxtt \nu_0 \psi_1\yxtt^{\mathrm{T}} \eta \yutt^{\mathrm{T}} \beta_t^{\mathrm{T}} +{}\\
   \left. {}+\psi \yxtt \nu_0 \psi \yxtt^{\mathrm{T}}\right]\,.
   \label{e18-sin}
   \end{multline}

Таким образом, справедливо следующее утверж\-дение.

\smallskip

\noindent
\textbf{Теорема~2.} \textit{Пусть для гауссовской системы}~(\ref{e1-sin})--(\ref{e2-sin}) \textit{при условиях Лип\-це\-ра--Ши\-ря\-ева выполнены условия невырожденности матрицы}~$\kappa_{22}$~(\ref{e13-sin}) \textit{и~конечности величин~$\kappa_{ij}$ $(i,j\hm=0,1,2)$. Тогда НФП определяется уравнением}~(\ref{e9-sin}), \textit{а~$\alp_t$, $\beta_t$ и~$\gamma_t$}~--- \textit{уравнениями}~(\ref{e10-sin})--(\ref{e18-sin}).

\smallskip

Для негауссовской СтС в уравнениях теоремы~2 следует заменить  $\nu_0$ на~$\nu$ согласно~(\ref{e3-sin}), а~в~выражении~(\ref{e16-sin}) учесть два дополнительных интегральных члена:
        \begin{multline*}
    \kappa_{01}'  =\kappa_{01} + \mathrm{M}^N_\Delta  \left[ \left(X_t - \hat X_t\right) \fr{\partial\xi^{\mathrm{T}}}{\partial t} \right] +{}\\
    {}+\mathrm{M}^N_\Delta  \left\{ \left(X_t -\hat X_t\right) \left[ \vrp_1^{\mathrm{T}} -\iii_{R_0^q} c(\rho)^{\mathrm{T}} \nu_P (t,\rho) d\rho \psi_1^{\mathrm{T}} \right] +{}\right.\\
\left.{}+ \psi \nu_0 \psi_1^{\mathrm{T}} - \beta_t \eta \psi_1 \nu_0\psi_1^{\mathrm{T}}
\vphantom{\iii_{R_0^q}}\right\} \left( \fr{\partial}{\partial y} + \eta^{\mathrm{T}} \beta_t^{\mathrm{T}} \fr{\partial}{\partial x}\right) \xi^{\mathrm{T}}+ {}\\
{}+\fr{1}{2} \mathrm{M}^N_\Delta  \left[ \left(X_t -\hat X_t\right)\right] \times{}\\
{}\times\left\{ \mathrm{tr}\, \left[ \psi_1 \nu_0 \psi_1^{\mathrm{T}} \left( \fr{\partial}{\partial y} + 2 \eta^{\mathrm{T}} \beta_t \fr{\partial}{\partial x} \right)\fr{\partial^{\mathrm{T}}}{\partial y} \right]+{}\right.\\
\left.{}+
   \mathrm{tr}\, \left[ \beta_t \eta \psi_1 \nu_0\psi_1^{\mathrm{T}} \eta^{\mathrm{T}} \beta_t^{\mathrm{T}} \fr{\partial}{\partial x} \fr{\partial^{\mathrm{T}}}{\partial x} \right] \right\} \xi^{\mathrm{T}} +{}\\
{}+ \mathrm{M}^N_\Delta \iii_{R_0^q}  \left[ X_t -\hat{X}_t + \left(\psi -\beta_t \eta \psi_1\right) c (\rho) \times{}\right.
\end{multline*}

\noindent
\begin{multline}
{}\times \xi \left(Y_t +\psi c(\rho), \hat{X}_t + \beta_t \eta \psi_1 c(\rho), t\right) -{}\\
\left.{}-\xi^{\mathrm{T}}
\vphantom{\left[ X_t -\hat{X}_t + \left(\psi -\beta_t \eta \psi_1\right) c (\rho) \times{}\right.}
\right]^{\mathrm{T}} \nu_P (t, d\rho)\, d\rho\,,\label{e19-sin}
\end{multline}
где функции $\vrp$, $\vrp_1$, $\psi$ и~$\psi_1$ удовлетворяют условиям Лип\-це\-ра--Ши\-ря\-ева. В~результате имеем следующее утверждение.

\smallskip

\noindent
\textbf{Теорема~3.} \textit{Пусть для негауссовской СтС}~(\ref{e1-sin})--(\ref{e2-sin}) \textit{в~условиях Лип\-це\-ра--Ши\-ря\-ева  матрица~$\kappa_{22}$}~(\ref{e13-sin}) \textit{невырождена,
а~интегралы}~(\ref{e10-sin}), (\ref{e13-sin}), (\ref{e15-sin}), (\ref{e17-sin}) \textit{и}~(\ref{e19-sin}) \textit{конечны.
Тогда НФП определяется уравнением}~(\ref{e9-sin}), \textit{а~коэффициенты}~$\alp_t$, $\beta_t$ и~$\gamma_t$~---
\textit{утверждениями}~(\ref{e10-sin}), (\ref{e12-sin}) \textit{и}~(\ref{e14-sin}).

\smallskip

\noindent
\textbf{Замечание~4.}
Теория НФП (оценивания состояния и~параметров систем) не позволяет получить нормальные с.к.\ оптимальные фильтры. Можно получить только ФП,
которые в~общем случае хуже оптимальных, но зато легко реализуемы.
Однако, если с.к.\ оптимальная оценка~$\hat{X}_t$ вектора~$X_t$
удовлетворяет уравнению допустимого фильтра~(\ref{e9-sin}) при ка\-ких-ли\-бо коэффициентах времени~$\alp_t$,
$\beta_t$ и~$\gamma_t$, то уравнения теорем~2 и~3, конечно, определяют именно эти $\alp_t$, $\beta_t$ и~$\gamma_t$ и~НФП будет  с.к.\ оптимальным
(последний в данном случае будет допустимым и,~следовательно,
оптимальным в классе допустимых фильтров).

\smallskip

\noindent
\textbf{Замечание~5.}
Теория НФП дает возможность оценивать не все компоненты вектора состояния системы (в~общем случае расширенного), а~только некоторые из них. Для этого достаточно
взять структурные функции~$\xi$ и~$\eta$ в~(\ref{e9-sin}), зависящими
лишь от соответствующих компонент вектора  $\hat{X}_t$. Так, например,
взяв~$\xi$ и~$\eta$ в~(\ref{e9-sin}), зависящими лишь от~$Y_t$, $t$ и~оценок неизвестных параметров системы, можно оценивать только
параметры системы, не оценивая ее состояния. В~таких случаях будут
получаться НФП, порядок которых меньше размерности~$n_x$
расширенного вектора состояния.


\section{Дифференциальный нормальный  фильтр Пугачёва на~основе нормальной аппроксимации апостериорного распределения}

Рассмотрим сначала гауссовскую СтС~(\ref{e1-sin})--(\ref{e2-sin}).
Так как гауссовское (нормальное) распределение, аппроксимирующее
апостериорное распределение вектора~$X_t$, полностью определяется
апостериорными математическим ожиданием~$\hat{X}_t$ и~ковариационной матрицей  $R_t$ вектора~$X_t$, то согласно теории нелинейной субоптимальной фильтрации при аппроксимации апостериорного распределения вектора~$X_t$ нормальным
распределением будем иметь следующие стохастические дифференциальные уравнения,
определяющие~$\hat{X}_t$ и~$R_t$~\cite{1-sin}:

\noindent
    \begin{multline}
    \dot{\hat X}_t = f \left(\hat X_t, Y_t,R_t,t\right)+{}\\
    {}+
    h\left(\hat X_t,Y_t, R_t,t\right)
   \left[ \dot Y_t - f^{(1)} \left(\hat X_t,Y_t, R_t,t\right)\right]\,;\label{e20-sin}
    \end{multline}


    \vspace*{-12pt}

    \noindent
    \begin{multline}
    \dot R_t=\left\{ f^{(2)}(\hat X_t, Y_t,R_t,t)-{}\right.\\
    \left.{}-h\left(\hat
    X_t, Y_t,R_t,t\right)\psi_1\nu_0\psi_1^{\mathrm{T}} \left(Y_t,t\right)
 h ({\hat X}_t, Y_t,R_t,t)^{\mathrm{T}}\right\} +{}\\
 \hspace*{-1.7mm}{}+\sss_{r=1}^{n_y}\! \rho_r \left(\hat{X}_t,Y_t, R_t,t\right)\!\!\left[
    \dot Y_r -f_r^{(1)}\left({\hat X}_t,Y_t, R_t,t\right) \right]\!,\!\!\!\label{e21-sin}
    \end{multline}
где
    \begin{multline*}
    f\left(\hat X_t, Y_t,R_t,t\right)=
    \mathrm{M}^N_\Delta \vrp \left(Y_t, \nu,t\right)={}\\
     {}=a_1 \left(Y_t, t\right) \hat{X}_t + a_0 \left(Y_t, t\right)\,; %\label{e22-sin}
    \end{multline*}

    \vspace*{-12pt}

    \noindent
    \begin{multline*}
   f^{(1)}\left(\hat X_t, Y_t,R_t,t\right)=\left\{ f_r^{(1)} \left( \hat X_t, Y_t, R_t, t\right)\right\}={}\\
   {}=\mathrm{M}^N_\Delta \vrp \left(Y_t, \nu,t\right)=b_1 \left(Y_t, t\right) \hat{X}_t+ b_0 \left(Y_t, t\right)\,;
   %\label{e23-sin}
   \end{multline*}

\vspace*{-12pt}

\noindent
   \begin{multline*}
    h\left(\hat X_t, Y_t,R_t,t\right)=\mathrm{M}^N_\Delta \left[ x\varphi_1(Y_t,x,t)^{\mathrm{T}} + {}\right.\\
\left.    {}+\psi\nu_0\psi_1^{\mathrm{T}} \left(Y_t,x,t\right)\right]-
    \hat{X}_t f^{(1)}\left(\hat X_t, Y_t,R_t,t\right)^{\mathrm{T}} %\right\}
    \times{}\\
{}\times \left(\psi_1\nu_0\psi_1^{\mathrm{T}}\right)^{-1} \left(Y_t,t\right)=
\left[ R_t b_1 \left(Y_t, t\right)^{\mathrm{T}} + {}\right.\\
\left.{}+\left(\psi\nu_0\psi_1^{\mathrm{T}}\right)\left(Y_t, t\right)
\vphantom{\left(Y_t, t\right)^{\mathrm{T}}}
\right] \left(\psi\nu_0\psi_1^{\mathrm{T}}\right)^{-1} \left(Y_t, t\right)\,;
%\label{e24-sin}
\end{multline*}

\vspace*{-12pt}

\noindent
\begin{multline*}
   \!\!f^{(2)}\left(\hat X_t, Y_t,R_t,t\right)=
   \mathrm{M}^N_\Delta \left\{  \left(x-\hat{X}_t\right)\varphi\left(Y_t,x,t\right)^{\mathrm{T}} + {}\right.\\
\left.   {}+\varphi \left(Y_t,x,t\right) \left(x^{\mathrm{T}}-\hat{X}_t^{\mathrm{T}}\right) +\psi\nu_0\psi^{\mathrm{T}} \left(Y_t,x,t\right)\right\}={}\\
{}=\left[ R_t b_1 \left(Y_t, t\right)^{\mathrm{T}} + \left(\psi\nu\psi_1^{\mathrm{T}}\right)\left(Y_t, t\right)\right] \left(\psi_1\nu_0\psi_1^{\mathrm{T}}\right)^{-1}\times{}\\
 {}\times \left(Y_t, t\right) \left[ b_1 (Y_t, t) R_t + \left(\psi_1\nu_0\psi^{\mathrm{T}}\right)\left(Y_t, t\right)\right]\,;
%\label{e25-sin}
\end{multline*}

\vspace*{-12pt}

\noindent
\begin{multline*}
   \rho_r\left(\hat{X}_t,Y_t, R_t,t\right)={}\\
   {}=\mathrm{M}^N_\Delta \left\{  \left(x-\hat X_t\right)\left(x^{\mathrm{T}}-\hat X_t^{\mathrm{T}}\right) a_r \left(Y_t,x,t\right)+{}\right.\\
{}+ \left(x-\hat X_t\right) b_r\left(Y_t,x,t\right)^{\mathrm{T}} \left(x^{\mathrm{T}}-\hat X_t^{\mathrm{T}}\right)+{}\\
 \left.{}+b_r \left(Y_t,x,t\right) \left(x^{\mathrm{T}}-\hat{X}_t^{\mathrm{T}}\right)\right\} =0 \enskip (r=1, \ldots, n_y)\,.
%\label{e26-sin}
\end{multline*}
Здесь функции  $a_r$~--- $r$-й элемент мат\-ри\-цы-стро\-ки
$$
A_{\vrp_1} = (\vrp_1^{\mathrm{T}} - \hat\vrp_n^{\mathrm{T}})(\psi_1 \nu_0\psi_1^{\mathrm{T}})^{-1}\,;
$$
 $B_{kr}^{\psi\psi_1}$~--- элемент $k$-й строки и~$r$-го столб\-ца матрицы
 $$
 B^{\psi\psi_1} = (\psi\nu_0\psi_1^{\mathrm{T}})
(\psi_1\nu_0\psi_1^{\mathrm{T}})^{-1}\,;
 $$
 $b_r$~--- $r$-й столбец матрицы $B^{\psi\psi_1}$:
 $$
 b_r^{\psi\psi_1} = \left[ b_{1r}^{\psi\psi_1}\cdots b_{n_x r}^{\psi\psi_1}\right]^{\mathrm{T}}\,.
 $$

Количество уравнений метода нормальной аппроксимации (МНА) одномерного апостериорного распределения
определяется по формуле:
    \begin{equation*}
   Q_{\mathrm{МНА}} = n_x + \fr{n_x (n_x+1)}{2} = \fr{n_x(n_x+3)}{2}\,.
   %\label{e27-sin}
   \end{equation*}

За начальные значения $\hat{X}_t$ и~$R_t$  при интегрировании уравнений~(\ref{e20-sin}) и~(\ref{e21-sin}), естественно, следует принять
условные математическое ожидание и ковариационную матрицу величины~$X_0$ относительно~$Y_0$:
\begin{align*}
\hat{X}_0 &= \mathrm{M}^N_\Delta\left[ X_0 \mid Y_0\right]\,;\\
R_0 &= \mathrm{M}^N_\Delta \left[ \left(X_0 -\hat X_0\right) \left(X_0^{\mathrm{T}} -\hat{X}_0^{\mathrm{T}}\right )\mid Y_0\right]\,.
%\label{e28-sin}
\end{align*}

\noindent
\textbf{Замечание~6.}
Если нет информации об условном распределении~$X_0$ относительно~$Y_0$, то
начальные условия можно взять в виде:
\begin{align*}
\hat{X}_0 &= \mathrm{M}^N_\Delta X_0\,;\\
R_0&= \mathrm{M}^N_\Delta (X_0 \hm-\mathrm{M}^N_\Delta X_0)
(X_0^{\mathrm{T}} - \mathrm{M}^N_\Delta X_0^{\mathrm{T}})\,.
\end{align*}
Если
же и об этих величинах нет никакой информации, то начальные
значения $\hat X_t$ и~$R_t$ приходится задавать произвольно.

Сравнивая  уравнения~(\ref{e20-sin}) и~(\ref{e21-sin}) с~уравнениями ФЛШ, имеем следующий результат.

\smallskip

\noindent
\textbf{Теорема 4.} \textit{Для гауссовской СтС}~(\ref{e1-sin})--(\ref{e2-sin}) \textit{при условиях Лип\-це\-ра--Ши\-ря\-ева НФП (на основе нормальной аппроксимации апостериорной плотности) и~ФЛШ совпадают.}

\smallskip

Для негауссовский СтС~(\ref{e1-sin})--(\ref{e2-sin}) при условиях Лип\-це\-ра--Ши\-ря\-ева, учитывая два дополнительных интегральных члена в~(\ref{e19-sin}), придем к уравнениям~(\ref{e20-sin}), (\ref{e21-sin}). При этом коэффициенты~$\alp_t$ и~$\gamma_t$ определяются~(\ref{e10-sin}), (\ref{e14-sin}), (\ref{e19-sin}) после нахождения~$\beta_t$ по формуле~(\ref{e12-sin}). Таким образом, имеем следующий результат.

\smallskip

\noindent
\textbf{Теорема~5.} \textit{Пусть для негауссовской системы}~(\ref{e1-sin})--(\ref{e2-sin})
\textit{при условиях Лип\-це\-ра--Ши\-ря\-ева выполнены условия невырожденности матрицы~$\kappa_{22}$ и конечности величин}~(\ref{e10-sin})--(\ref{e19-sin}) $(i, j \hm=0,1,2)$.
\textit{Тогда НФП определяется уравнением}~(\ref{e9-sin}),
\textit{а~коэффициенты $\alp_t$, $\beta_t$ и~$\gamma_t$}~---
\textit{уравнениями}~(\ref{e10-sin})--(\ref{e19-sin}).


\section{Квазилинейный нормальный фильтр Пугачёва}


Особое практическое значение имеет случай~(\ref{e1-sin})--(\ref{e2-sin}) при условиях Лип\-це\-ра--Ши\-ря\-ева с~аддитивными (в~общем случае негауссовскими) шумами. Следуя~\cite{1-sin}, проведем статистическую линеаризацию нелинейных функций:
    \begin{align*}
    a_1 \left(Y_t, t\right) X_t &\approx \left(k_{0x}^{a_1 x} - k_{1x}^{a_1 x}\right) m_t^x +{}\\
     &{}+\left( k_{0y}^{a_1 x} - k_{1y}^{a_1 x}\right) m_t^y + k_x^{a_1 y} X_t + k_{0y}^{a_1 x} Y_t\,;
   \\
    b_1 \left(Y_t, t\right) X_t &\approx \left(k_{0x}^{b_1 x} - k_{1x}^{b_1 x}\right) m_t^x + {}\\
    &{}+\left( k_{0y}^{b_1 x} - k_{1y}^{b_1 x}\right) m_t^y + k_x^{b_1 y} X_t + k_{0y}^{b_1 x} Y_t\,;
   \\
    a_0 (Y_t , t) &\approx \left( k_{0y}^{a_0} - k_{1y}^{a_0}\right) m_t^y + k_{0y}^{a_0} Y_t\,;\\
    b_0 \left(Y_t , t\right) &\approx  \left( k_{0y}^{b_0} - k_{1y}^{b_0}\right) m_t^y + k_{0y}^{b_0} Y_t\,.
    %\label{e29-sin}
    \end{align*}
Тогда уравнения~(\ref{e1-sin}) и~(\ref{e2-sin}) приводятся к~эквивалентной гауссовской системе, линейной относительно~$X_t^0$ и~$Y_t^0$ и~нелинейной относительно~$m_t^x$ и~$m_t^y$:
\begin{align}
\dot X_t &= \bar a Y_t + \bar a_1 X_t + \bar a_0 +\bar\psi V\,;\label{e30-sin}\\
\dot Y_t &= \bar b Y_t + \bar b_1 X_t + \bar b_0 +\bar\psi_1 V\,.\label{e31-sin}
\end{align}
Здесь
    \begin{equation}
    \left.
    \begin{array}{rlrl}
    \bar a &= k_y^{a_1x} + k_y^{a_0}\,; &\enskip     \bar b &= k_y^{b_1x} \,;\\
    \bar a_1 &= k_x^{a_1 x}\,;&\enskip \bar b_1 &= k_x^{b_1 x}\,;\\
    a_0 &= (k_{0y}^{a_0} - k_{1y}^{a_0}) m_t^y\,;
 &\enskip b_0 &= (k_{0y}^{b_0} - k_{1y}^{b_0}) m_t^y\,.
    \end{array}
    \right\}
    \label{e32-sin}
    \end{equation}
Правые части~(\ref{e32-sin}) зависят от вероятностных моментов первого и второго порядка и определяются из следующей линейной дифференциальной системы для вектора  $Z_t \hm=\left[ X_t^{\mathrm{T}}\, Y_t^{\mathrm{T}}\right]^{\mathrm{T}}$:
  \begin{align*}
  \bar m_t^z &= cm_t^z + c_0\,; %\label{e33-sin}
  \\
    K_t^{\bar z} &= c K_t^z +K_t^z c^{\mathrm{T}} + l\nu l^{\mathrm{T}}\,, %\label{e34-sin}
    \end{align*}
где
    \begin{equation*}
    c=\begin{bmatrix}
    \bar a_1&\bar a\\
    \bar b_1&\bar b\end{bmatrix}\,; \enskip
    c_0=\begin{bmatrix}
    \bar a_o\\
    \bar b_0\end{bmatrix}\,;\enskip
    l=\begin{bmatrix}
    \bar \psi_t\\
    \bar\psi_{1t}\end{bmatrix}\,.\label{e35-sin}
\end{equation*}

Применяя теорию квазилинейной фильтрации~\cite{1-sin}
к~уравнениям~(\ref{e30-sin}) и~(\ref{e31-sin}), получим следующие уравнения квазилинейного фильтра:
\begin{align}
    \dot{\hat X}_t& = \bar a Y_t +\bar a_1 \hat X_t + \bar a_0 +{}\notag\\
     &\hspace*{15mm}{}+\beta_t \left[ \dot Y_t - \left( \bar b Y_t +\bar b_1 \hat{X}_t + \bar b_0\right) \right]\,;
    \label{e36-sin}
\\
   \beta_t &= R_t \bar b_1^{\mathrm{T}} +\left(\bar\psi\nu\bar\psi_1^{\mathrm{T}}\right) \left(\bar \psi_1 \nu \bar \psi_1^{\mathrm{T}}\right)^{-1}\,;\label{e37-sin}\\
    \dot R_t &= \bar a_1 R_t + R_t \bar a_1^{\mathrm{T}} +\bar\psi\nu\bar\psi^{\mathrm{T}} -{}\notag\\
&\hspace*{-5mm}{}-\left(R_t \bar b_1^{\mathrm{T}} +\bar\psi\nu\bar\psi_1\right)\left(\bar\psi_1 \nu\bar\psi_1^{\mathrm{T}}\right)^{-1} \left( \bar b R_t +\bar\psi_1\nu\bar\psi_1^{\mathrm{T}}\right)\,.
    \label{e38-sin}
    \end{align}

Таким образом, имеем следующий результат.

\smallskip

\noindent
\textbf{Теорема~6.} \textit{Пусть уравнения негауссовской СтС}~(\ref{e1-sin})--(\ref{e2-sin}) \textit{при условиях Лип\-це\-ра--Ши\-ря\-ева и~с~аддитивными шумами  допускают применение метода статистической линеаризации (МСЛ). Тогда уравнения  квазилинейного НФП имеют вид}~(\ref{e36-sin})--(\ref{e38-sin}).

\smallskip

\noindent
\textbf{Замечание~7.}
Согласно~\cite{2-sin, 3-sin}, из уравнения~(\ref{e38-sin}) получаются приближенные условия асимптотической устойчивости в~терминах равномерной наблюда\-емости и~управляемости в~следующем виде:
    \begin{multline*}
    0\le R_t (R_0, t_0) \le u_A (t, t_0) R_0 u_A(t, t_0) +{}\\
     {}+\iii_{t_0}^t u_A (t,\tau) \si_{11} u_A (t, \tau)^{\mathrm{T}}\, d\tau\,,
%    \label{e39-sin}
    \end{multline*}
где $u_A (t,\tau)$~--- фундаментальная матрица однородного уравнения Риккати, причем
   \begin{gather*}
    A =\bar a_1 -\beta_t \bar b_1 - R_t \bar b_1^{\mathrm{T}} \si_{22}^{-1} b_1- \si_{12} \si_{22}^{-1} \bar b_1\,; %\label{e40-sin}
    \\
    \si_{11}= \bar\psi_{10t}\nu\bar\psi_{10}^{\mathrm{T}}\,;\\
     \si_{12}= \bar\psi_{10 t}\nu\bar\psi_{20t}^{\mathrm{T}}\,;\quad
      \si_{22} = \bar\psi_{20t}\nu\bar\psi_{20t}^{\mathrm{T}}\,.\label{e41-sin}
    \end{gather*}

    \vspace*{-18pt}


\section{Заключение}

\vspace*{-4pt}

\noindent
\begin{enumerate}[1.]
\item Рассмотрены вопросы аналитического синтеза услов\-но-оп\-ти\-маль\-ных фильтров Пугачёва для обработки информации в дифференциальных негауссовских СтС, линейных относитель\-но состояния (условия Лип\-це\-ра--Ши\-ря\-ева). Особое внимание уделено синтезу фильт\-ров для СтС при условиях Лип\-це\-ра--Ши\-ря\-ева путем аппроксимации апостериорного распределения нормальным и квазилинейным фильт\-ра\-ми, основанным на статистической линеаризации нелинейных функций, зависящих от наблюдений.

\item Для СтС высокой размерности путем выбора структурных функций, отражающих аналитическую природу наблюдаемой системы, можно синтезировать простые в компьютерной реализации фильтры для работы в режиме реального времени.

Изложенные алгоритмы положены в основу модуля математического обеспечения инструментального программного обеспечения <<StS-Filter>>. Тестирование проведено на основе примеров~[1, 4--6, 8].

\item Полученные результаты  позволяют синтезировать дифференциальным ФП,
 во-пер\-вых, для случая широкополосных шумов в~уравне-\linebreak\vspace*{-12pt}

 \pagebreak

 \noindent
 ниях~(\ref{e1-sin}) и~(\ref{e2-sin}), если аналогично~\cite{3-sin} заменить широкополосный белый шум эквивалентным нормальным (гауссовским) белым шумом и,~во-вто\-рых, для случая автокоррелированных шумов. Результаты также обобщаются на случай дискретных СтС, линейных относительно состояния, если воспользоваться~\cite{4-sin}.
    \end{enumerate}

    \setcounter{equation}{0}
{\small \section*{\raggedleft Приложение}

\renewcommand{\theequation}{П.\arabic{equation}}


\section*{Тестовые примеры}


\textbf{1.} Найти приближенно оптимальный алгоритм оценивания
 состояния~$X_t$ системы, описываемой скалярным уравнением
    \begin{equation}
    \dot X_t =- \theta X_t + V_1\,,\label{p1}
    \end{equation}
 и неизвестного параметра~$\theta$ по результатам наблюдения
 процесса
    \begin{equation}
    \dot Y_t= X_t+ V_2\,,\label{p2}
    \end{equation}
где  $V_2$~--- белый шум, независимый от~$V_1$.

 Заменим параметр  $\theta$ стохастическим процессом~$\Theta_t$,
 определяемым уравнением  $\dot \Theta_t\hm=0$, и примем за расширенный
 вектор состояния пару  $\left[ X_t \Theta_t\right]^{\mathrm{T}}$. Тогда уравнения
 НФП будут иметь вид:
   \begin{align*}
   \dot{\hat{X}}_t&=-\hat {X}_t\hat
    \Theta_t - R_{12} +\nu_2^{-1} R_{11} \left(\dot {Y}_t-\hat {X}_t\right)\,;\\
    \dot{\hat {\Theta}}_t &=\nu_2^{-1} R_{12} \left(\dot{Y}_t- \hat{X}_t\right)\,; %\label{p3}
    \\
\dot R_{11} &= \nu_1 - 2\left(\hat {\Theta}_t R_{11} +\hat X_t R_{12}\right)
    -\nu_2^{-1} R_{11}^2\,; %\label{pIV}
    \\
\dot R_{12} &=- \hat {\Theta}_t
    R_{12} -\hat{X}_t R_{22} -\nu_1^{-1} R_{11} R_{12}\,; \\
    \dot{R}_{22} &=-\nu_2^{-1} R_{12}^2\,, %\label{p4}
    \end{align*}
где  $R_{11}$, $R_{12}$ и~$R_{22}$~--- апостериорные дисперсии и~ковариация ошибок оценок~$\hat{X}_t$ и~$\hat{\Theta}_t$
соответственно. За начальные значения  $\hat{X}_t$,
$\hat{\Theta}_t$, $R_{11}$, $R_{12}$ и~$R_{22}$ следует принять
соответствующие априорные величины, причем~$\hat{\Theta}_0$,
$R_{220}$ и~$R_{120}$ всегда приходится брать произвольно, так как
априорной информации о~параметре~$\theta$ обычно нет, за
исключением, может быть, информации о~возможном диапазоне его значений.
Этот фильтр совпадает с~обобщенным фильтром Калмана~\cite{1-sin, 6-sin, 8-sin}.

\smallskip

\textbf{2.}  Найдем НФП для системы~(\ref{p1}), (\ref{p2}).
За основу класса допустимых фильтров примем первые два уравнения
МНА и соответственно положим, что
\begin{equation*}
    \xi \left(\hat X_t, \hat \Theta_t, t\right) =
    \left[ \hat X_t \hat \Theta_t\, \hat X_t\right] ^{\mathrm{T}} \eta \left(
    \hat X_t, \hat \Theta_t, t\right)=1\,.
    %\label{p5}
    \end{equation*}
Тогда класс допустимых фильтров
будет представлять собой систему двух уравнений:
\begin{align*}
   \dot{\hat{X}}_t &= \alp_{11} \hat{X}_t \hat\Theta_t +\alp_{12} \hat{X}_t +\beta_1 \dot Y_t +\gamma_1\,; %\label{p6}
   \\
   \dot{\hat\Theta}_t &= \alp_{21} \hat{X}_t \hat\Theta_t +\alp_{22} \hat{X}_t +\beta_2 \dot Y_t +\gamma_2\,. %\label{p7}
   \end{align*}
Здесь
    \begin{equation*}
    \kappa_{22} =\nu_2\,; %\label{p8}
    \end{equation*}

    \vspace*{-10pt}

    \noindent
    \begin{multline*}
    \kappa_{02} = \mathrm{M}_N\left[ \left(X_t-\hat{X}_t\right) X_t\left (
    \Theta_t -\hat \Theta_t\right)X_t\right]^{\mathrm{T}} ={}\\
    {}=
    \left[ m_{2000} -m_{1010}\, m_{1100}- m_{1001}\right]^{\mathrm{T}}\,; %\label{p9}
    \end{multline*}
    \begin{equation*}
   \beta_1 =\nu_2^{-1} \left(m_{2000}-m_{1010}\right)\,;\enskip \beta_2 =\nu_2^{-1} \left(m_{1100}-m_{1001}\right)\,, %\label{p10}
   \end{equation*}
где  $m_{pqrs}\hm=\mathrm{M}_N X_t^p \Theta_t^q \hat{X}_t^r\hat\Theta_t^s$.
Уравнения,  определяющие оптимальные
коэффициенты $\alp_{11}$, $\alp_{12}$, $\alp_{21}$, $\alp_{22}$,
$\gamma_1$ и~$\gamma_2$, имеют следующий вид:
    $$
    m_{0011}\alp_{11} +m_{0010}\alp_{12}+\gamma_1 =- m_{1100}-\beta_1 m_{1000}\,; %\eqno({\rm П}.11)
    $$
    $$
    m_{0011}\alp_{21} +m_{0010}\alp_{22}+\gamma_2 =- \beta_1 m_{1000}\,;
    %\eqno({\rm П}.12)
    $$

    \vspace*{-10pt}

    \noindent
    \begin{multline*}
        \left(2m_{0022} -m_{1012}- m_{0011}^2\right)\alp_{11}+{}\\
        {}+\left(2m_{0021} -m_{1012}- m_{0010}m_{0011}\right)\alp_{12}+{}\\
{}+ \left(m_{0031} -m_{1021}\right)\alp_{21}+\left(m_{0030} -m_{0020}\right)\alp_{22}+{}\\
{}+\left(m_{0011} -m_{1001}\right)\gamma_1+ \left(m_{0020} -m_{1010}\right)\gamma_2={}\\
{}=- m_{1111} -2m_{1100}m_{0011}+ \beta_1 \left(m_{2001}-2m_{1011}+{}\right.\\
\left.{}+m_{1000} m_{0011}\right)+ \beta_2 \left(m_{2010}-m_{1020}\right) -\nu_2 \beta_1^2 m_{0001} +{}\\
{}+\nu_2 \beta_1\beta_2
    \left(m_{1000}-2m_{0010}\right)\,;\
    %eqno({\rm П}.13)
    \end{multline*}

     \vspace*{-10pt}

    \noindent
    \begin{multline*}
   \left(2m_{0021}-m_{1011}-m_{0010}m_{0011}\right)\alp_{11} + {}\\
   {}+\left(2m_{0020}-m_{1010}-m_{0010}^2\right)\alp_{12}+{}\\
{}+ \left(m_{0010}-m_{1000}\right)\gamma_1 =- m_{1110}+m_{1100}m_{1010}+{}\\
{}+ \beta_1 \left(m_{2000}- 2m_{1010}+m_{1000}m_{0010}\right)-\nu_2\beta_1^2\,;
%\eqno({\rm П}.14)
\end{multline*}

 \vspace*{-10pt}

    \noindent
    \begin{multline*}
   \left(m_{0013}-m_{0112}\right)\alp_{11} + \left( m_{0012}-m_{0111}\right)\alp_{12} +{}\\
   {}+\left(2m_{0022}-m_{0121}-m_{0011}^2\right) \alp_{21}+{}\\
{}+ \left(2m_{0021}-m_{0120}-m_{0010}m_{0001}\right)\alp_{22} +{}\\
{}+\left(m_{0002}-m_{0101}\right)\gamma_1+ \left(m_{0011}-m_{0110}\right)\gamma_2 ={}\\
{}=\beta_1
\left(m_{1101}-m_{1002}\right) +\beta_2 \left(m_{1110}-2 m_{1011}+{}\right.\\
\!\left.{}+m_{1000}m_{0011}\right)+ \nu_2 \beta_1\beta_2 \left(m_{0100}-2m_{0001}\right)-\nu_2\beta_2^2 m_{0010};
%,\eqno({\rm П}.15)$$
\end{multline*}

 \vspace*{-10pt}

    \noindent
    \begin{multline*}
   \left(m_{0012}-m_{0111}\right) \alp_{11} +\left(m_{0011}-m_{0110}\right)\alp_{12} + {}\\
   {}+\left(m_{0021}-m_{0011}m_{0010}\right)\alp_{21}+{}\\
{}+ \left(m_{0020}-m_{0010}^2\right)\alp_{22} +\left(m_{0001}-m_{0100}\right)\gamma_1={}\\
\!{}= \beta_1 \left(m_{1100}-m_{1001}\right)-\beta_2
\left(m_{1010}-m_{1000}m_{0010}\right) -\nu_2 \beta_1\beta_2.\hspace*{-9.5pt}
%\eqno({\rm П}.16)$$
\end{multline*}
Числовые значения получаются путем решения МНА нормальной апостериорной плот\-ности с~по\-мощью инструментального программного обеспечения <<StS-Filter>>.

\smallskip

\textbf{3.} Построим теперь  НФП для оценивания одного только неизвестного параметра~$\theta$. Для определения класса допустимых фильтров включим
неизвестную функцию времени~$\hat{X}_t$ в~уравнение для~$\dot{\hat
\Theta}_t$ предыдущего примера в~оптимизируемые коэффициенты.
Тогда получим следующее уравнение класса допустимых фильтров:
\begin{align*}
    \dot{\hat \Theta}_t& =\alp _t\hat \Theta_t +\beta_t \dot Y_t +\gamma_t\,;
    %\eqno({\rm П}.17)$$
    \\
    \beta_t&= \nu_2^{-1} (m_{110} -m_{101})\,;
    %,\eqno({\rm П}.18)$$
    \\
    \alp_t&=\fr{-m_{111}+m_{110}m_{001}+\beta_t \left(m_{101}-m_{100}m_{001}\right)}{m_{002}^2-m_{001}^2}\,;
    %\eqno({\rm П}.19)$$
    \\
    \gamma_t&=-\alp m_{001}-\beta_t m_{100}\,.
    %\eqno({\rm П}.20)$$
    \end{align*}

}



{\small\frenchspacing
 {%\baselineskip=10.8pt
 \addcontentsline{toc}{section}{References}
 \begin{thebibliography}{9}

\bibitem{1-sin}
\Au{Синицын И.\,Н.}
Фильтры Калмана и Пугачева.~--- 2-е изд. -- М.: Логос, 2007. 776~с.


\bibitem{5-sin} %2
 \Au{Корепанов Э.\,Р.}
 Стохастические информационные технологии на основе фильтров Пугачева~// Информатика и её применения, 2011. Т.~5. Вып.~2. С.~36--57.

\bibitem{4-sin} %3
\Au{Синицын И.\,Н.}  Параметрическое статистическое и аналитическое моделирование распределений в нелинейных стохастических системах на многообразиях~// Информатика и её применения, 2013. Т.~7. Вып.~2. С.~4--16.



\bibitem{6-sin} %4
\Au{Синицын И.\,Н., Синицын В.\,И.} Лекции по нормальной и эллипсоидальной аппроксимации в стохастических системах.~--- М.: ТОРУС ПРЕСС, 2013. 476~с.

\bibitem{2-sin} %5
\Au{Синицын И.\,Н., Корепанов Э.\,Р.}
Устойчивые линейные условно оптимальные фильтры и экстраполяторы для стохастических систем с~мультипликативными шумами~// Информатика и её применения, 2015. Т.~9. Вып.~1. С.~70--75.

\bibitem{3-sin} %6
\Au{Синицын И.\,Н., Корепанов Э.\,Р.}
Синтез устойчивых линей\-ных фильтров и экстраполяторов Пугачева для стохастических систем с мультипликативными широкополосными шумами~// Системы и~средства информатики, 2015. Вып.~25. №\,1.
С.~108--126.

\bibitem{7-sin}
\Au{Липцер Р.\,Ш., Ширяев А.\,Н.} Статистика случайных процессов.~--- М.: Наука, 1974. 476~с.

\bibitem{8-sin}
\Au{Ройтенберг Я.\,Н.} Автоматическое управление.~--- 3-е изд., перераб. и доп.~--- М.: Наука, 1992. 576~с.
 \end{thebibliography}

 }
 }

\end{multicols}

\vspace*{-3pt}

\hfill{\small\textit{Поступила в~редакцию 31.03.15}}

%\newpage

\vspace*{12pt}

\hrule

\vspace*{2pt}

\hrule

%\vspace*{12pt}

\def\tit{NORMAL PUGACHEV FILTERS FOR STATE LINEAR STOCHASTIC SYSTEMS}

\def\titkol{Normal Pugachev filters for state linear stochastic systems}

\def\aut{I.\,N.~Sinitsyn and E.\,R.~Korepanov}

\def\autkol{I.\,N.~Sinitsyn and E.\,R.~Korepanov}

\titel{\tit}{\aut}{\autkol}{\titkol}

\index{Sinitsyn I.\,N.}
\index{Korepanov E.\,R.}

\vspace*{-9pt}


\noindent
Institute of Informatics Problems, Federal Research
Center ``Computer Science and Control'' of the
Russian Academy of Sciences, 44-2 Vavilov Str., Moscow 119333,
Russian Federation


\def\leftfootline{\small{\textbf{\thepage}
\hfill INFORMATIKA I EE PRIMENENIYA~--- INFORMATICS AND
APPLICATIONS\ \ \ 2015\ \ \ volume~9\ \ \ issue\ 2}
}%
 \def\rightfootline{\small{INFORMATIKA I EE PRIMENENIYA~---
INFORMATICS AND APPLICATIONS\ \ \ 2015\ \ \ volume~9\ \ \ issue\ 2
\hfill \textbf{\thepage}}}

\vspace*{3pt}


\Abste{The applied theory of analytical synthesis of normal conditionally optimal (Pugachev) filters (NPF) in state linear non-Gaussian stochastic systems (StS)
is presented. Special attention is paid to NPF for differential StS satisfying Liptzer--Shiraev conditions based on the normal approximation of \textit{a~posteriori} density and quasi-linear NPF based on statistical linearization of nonlinear functions depending on observations.  For StS of high dimension and real-time problems, NPF are more effective than the
suboptimal filters. The NPF algorithms are the basis of the ``StS-Filters'' software tool. Test examples are given.}


\KWE{Liptser--Shiraev filter (LSF); Liptser--Shiraev conditions; normal approximation method (NAM) for \textit{a~posteriori} density; normal conditionally optimal Pugachev filter (NPF);  stochastic systems (StS);
state linear StS; statistical linearization method (SLM)}

\DOI{10.14357/19922264150204}

\Ack
\noindent
The research was supported by the Russian Foundation for Basic Research (project 15-07-02244).



\vspace*{3pt}

  \begin{multicols}{2}

\renewcommand{\bibname}{\protect\rmfamily References}
%\renewcommand{\bibname}{\large\protect\rm References}



{\small\frenchspacing
 {%\baselineskip=10.8pt
 \addcontentsline{toc}{section}{References}
 \begin{thebibliography}{9}

\bibitem{1-sin-1}
\Aue{Sinitsyn, I.\,N.} 2007. \textit{Fil'try Kalmana i~Pugacheva} [Kalman and Pugachev filters]. 2nd ed. Moscow: Logos.  776~p.


\bibitem{5-sin-1} %2
\Aue{Korepanov, E.\,R.} 2011. Stokhasticheskie informatsionnye tekhnologii na osnove fil'trov Pugacheva [Stochastic informational technologies based on Pugachev filters]. \textit{Informatika i ee Primeneniya}~---
\textit{Inform Appl.}   5(2):36--57.

\bibitem{4-sin-1} %3
\Aue{Sinitsyn, I.\,N.}  2013.
Parametricheskoe statisticheskoe i~analiticheskoe modelirovanie raspredeleniy
v~nelineynykh stokhasticheskikh sistemakh na mnogoobraziyakh
[Parametric statistical and analytical modeling of distributions in stochastic systems on manifolds].
\textit{Informatika i~ee Primeneniya}~---
\textit{Inform. Appl.} 7(2):4--16.

\bibitem{6-sin-1} %4
\Aue{Sinitsyn, I.\,N., and  V.\,I.~Sinitsyn}.  2013.
\textit{Lektsii po normal'noy i ellipsoidal'noy approksimatsii raspredeleniy
v~stokhasticheskikh sistemakh} [Lectures on normal and ellipsoidal
approximation of distributions in stochastic systems]. Moscow: TORUS PRESS. 488~p.

\bibitem{2-sin-1} %5
\Aue{Sinitsyn, I.\,N., and E.\,R.~Korepanov}. 2015.
Ustoychivye lineynye uslovno optimal'nye fil'try i~ekstrapolyatory dlya stokhasticheskikh sistem s mul'tiplikativnymi shumami [Stable linear conditionally optimal filters and extrapolators for stochastic systems with multiplicative noises]. \textit{Informatika i~ee Primeneniya}~---
 \textit{Inform. Appl.}  9(1):70--75.

%\pagebreak

\bibitem{3-sin-1} %6
 \Aue{Sinitsyn, I.\,N., and E.\,R.~Korepanov}. 2015.
Sintez ustoychivykh fil'trov i~ekstrapolyatorov Pugacheva dlya sto\-kha\-sti\-che\-skikh sistem s mul'tiplikativnymi shumami [Synthesis of stable conditionally optimal filters and extrapolators for stochastic systems with multiplicative wide band noises]. \textit{Sistemy i~Sredstva Informatiki}~---
\textit{Systems and Means of Informatics}  25(1):108--126.



\bibitem{7-sin-1}
\Aue{Liptser, R.\,Sh., and A.\,N.~Shiryaev.} 1974. \textit{Statistika sluchaynykh protsessov} [Statistics of stochastic processes].  Moscow: Nauka,  476~p.

\vspace*{-3pt}

\bibitem{8-sin-1}
\Aue{Roytenberg, Ya.\,N.} 1992. \textit{Avtomaticheskoe upravlenie} [Automatic control]. 3rd ed. Moscow: Nauka. 576~p.

\end{thebibliography}

 }
 }

\end{multicols}

\vspace*{-3pt}

\hfill{\small\textit{Received March 31, 2015}}

%\vspace*{-18pt}


\Contr

\noindent
\textbf{Sinitsyn Igor N.} (b.\ 1940)~--- Doctor of Science in technology, professor, Honored scientist of RF, Head of Department, Institute of Informatics Problems, Federal Research Center ``Computer Science and Control'' of the Russian Academy of Sciences, 44-2 Vavilov Str., Moscow 119333, Russian Federation; sinitsin@dol.ru

\vspace*{3pt}

\noindent
\textbf{Korepanov Eduard R.} (b.\ 1966)~--- Candidate of Science (PhD) in technology, Head of Laboratory, Institute of Informatics Problems, Federal Research Center ``Computer Science and Control'' of the Russian Academy of Sciences, 44-2 Vavilov Str., Moscow 119333, Russian Federation; Ekorepanov@ipiran.ru

\label{end\stat}


\renewcommand{\bibname}{\protect\rm Литература} 