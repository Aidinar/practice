\def\stat{shestakov}

\def\tit{НЕПАРАМЕТРИЧЕСКОЕ ОЦЕНИВАНИЕ МНОГОМЕРНОЙ ПЛОТНОСТИ
  С~ПОМОЩЬЮ ВЕЙВЛЕТ-ОЦЕНОК\\ ОДНОМЕРНЫХ ПРОЕКЦИЙ$^*$}

\def\titkol{Непараметрическое оценивание многомерной плотности
  с~помощью вейвлет-оценок одномерных проекций}

\def\aut{О.\,В.~Шестаков$^1$}

\def\autkol{О.\,В.~Шестаков}

\titel{\tit}{\aut}{\autkol}{\titkol}

\index{Шестаков О.\,В.}

{\renewcommand{\thefootnote}{\fnsymbol{footnote}} \footnotetext[1]
{Работа выполнена при финансовой
поддержке РФФИ (проект 15-07-02652).}}


\renewcommand{\thefootnote}{\arabic{footnote}}
\footnotetext[1]{Московский государственный университет им.\ М.\,В.~Ломоносова,
 кафедра математической статистики факультета ВМК;
Институт проблем информатики Федерального исследовательского
центра <<Информатика и~управление>> Российской академии наук,
oshestakov@cs.msu.su}


\Abst{Рассматривается используемый в~реконструктивной
томографии метод обращения преобразования Радона для получения
статистических оценок многомерных вероятностных плотностей.
Этот метод использует нелинейные вейв\-лет-оцен\-ки одномерных проекций
 для построения оценки многомерной плотности. Нелинейные вейв\-лет-оцен\-ки
 обладают возможностью адаптации к локальным свойствам оцениваемой функции
 плотности и,~следовательно, менее чувствительны к наличию сингулярных точек,
 чем линейные оценки. Еще одним важным практическим аспектом рассматриваемого
 метода является его параллельная структура, позволяющая значительно ускорить
 построение оценок на компьютерах, поддерживающих параллельные вычисления.
Также показано, что при выполнении некоторых условий регулярности равномерное
расстояние между построенной оценкой и~истинной многомерной вероятностной
плотностью стремится к нулю в~среднем, и~получены оценки скорости этой сходимости.}

\KW{вейвлеты; многомерная плотность; преобразование Радона}

\DOI{10.14357/19922264150210}

\vspace*{6pt}


\vskip 14pt plus 9pt minus 6pt

\thispagestyle{headings}

\begin{multicols}{2}

\label{st\stat}

\section{Введение}

Проблема непараметрического оценивания мно\-го\-мер\-ной плотности вероятностного
распре\-деления возникает во многих областях статисти\-ческого анализа данных,
например в~задачах классификации и~распознавания изображений.

Существуют
различные подходы к решению этой проб\-ле\-мы~[1], такие как использование
ядерных оценок, оценивание по ближайшим соседям, аппроксимация сплайнами
и~разложение в~орто\-гональные ряды. В~последние годы все более популярны\-ми
становятся методы, основанные на вейв\-лет-раз\-ло\-же\-нии~[2]. Преимуществом
этих методов является адаптивность к локальной гладкости оцениваемой плотности.
Возможные многомерные обобщения вейв\-лет-ме\-то\-дов оценивания плот\-ности
рассмотрены в~работах~[3--6].

В~данной работе рассматривается томографический
подход, основанный на построении оценки многомерной плотности по одномерным
оценкам проекций. Как отмечено в~работе~[7], данный подход не нуждается
в~использовании структурных предположений, таких как мультипликативность
или аддитивность плотности. Кроме того, важным практическим аспектом такого
метода является возможность параллельного вы\-чис\-ле\-ния одномерных оценок.

Для рассматриваемого метода получены оценки скорости сходимости к нулю в~среднем
равномерного расстояния между построенной оценкой и~оцениваемой многомерной
плот\-ностью.

\section{Преобразование Радона}

Преобразование Радона в~$d$-мер\-ном пространстве ставит в~соответствие
функции $f(x)$, $x\hm\in\mathbb{R}^d$, одномерные проекции $R_\theta f(s)$:
\begin{equation*}
R_\theta f(s)=\int\limits_{s\theta=s}f(x)\,dx=
\int\limits_{\theta^\perp}f(s\theta+y)\,dy\,,
\end{equation*}
где $\theta\in S^{d-1}$, а $S^{d-1}$~--- единичная сфера в~$\mathbb{R}^d$.

Зная $R_\theta f(s)$ для всех $\theta\hm\in S^{d-1}$ и~всех $s\hm\in\mathbb{R}$,
функцию $f(x)$ можно вычислить по следующей формуле~[7]:
\begin{multline}
f(x)={}\\
\hspace*{-4mm}{}=\fr{1}{2(2\pi)^{d}}\!\!\!\!\int\limits_{S^{d-1}}
\int\limits_{-\infty}^{\infty}\!\!\!
\abs{\omega}^{d-1}\!\widehat{R_\theta f}(\omega)\exp
\{ i\omega(x \theta)\}\,d\omega d\theta,\!\!\!\!
\label{Inverse_Radon_transform}
\end{multline}
где $\widehat{R_\theta f}(\omega)$~---
преобразование Фурье функции $R_\theta f(s)$ по переменной~$s$:
\begin{equation*}
\widehat{R_\theta f}(\omega)=
\int\limits_{-\infty}^{\infty}R_\theta f(s)\exp\{-i\omega s\}\,ds\,.
\end{equation*}

Метод оценивания многомерной плотности основан на следующих соображениях.
Легко видеть, что если $X$~--- $d$-мер\-ный случайный вектор с~плотностью $f(x)$,
то одномерная случайная величина $X\theta$ имеет плот\-ность
$R_\theta f(s)$~[7]. Таким образом, построив оценки одномерных плотностей
$R_\theta f(s)$ по выборкам $(X\theta)_j$, $j=1,\ldots,N$, можно получить
оценку многомерной плотности $f(x)$, используя методы обращения преобразования
Ра\-дона.

\section{Вейвлет-оценки одномерных вероятностных плотностей}

Вейвлет-базисы приобрели большую популярность в~задачах непараметрического
оценивания функций благодаря своей адаптивности к локальным особенностям
оцениваемых функций. В~работе~[8] предложено использовать вейв\-лет-раз\-ло\-же\-ние
для оценивания вероятностных плотностей.

Рассмотрим неоднородный ортонормированный вейв\-лет-ба\-зис
$\left\{\phi_{j_0,k}(s)\hm=2^{j_0/2}\phi(2^{j_0}s-k)\right\}_{k\in\mathbb{Z}}$
и~$\left\{\psi_{j,k}(s)=2^{j/2}\psi(2^{j}s-k)\right\}_{k\in\mathbb{Z}}$,
$j\hm\geqslant j_0$, где $\phi(s)$ и~$\psi(s)$~--- масштабирующая
и~вейв\-лет-функ\-ции, порождающие некоторый кратномасштабный анализ~[9].
Будем полагать, что функции $\phi(s)$ и~$\psi(s)$ имеют компактный носитель
и~непрерывно дифференцируемы достаточное число раз (этому требованию удовлетворяют,
например, вейвлеты Добеши~[10]). Кроме того, будем полагать, что функция
$\phi(s)$ нормализована следующим образом:
$$
\int\limits_{-\infty}^{\infty}\phi(s)\,ds=1\,.
$$

 Вероятностную плотность $g(s)$
можно формально представить в~виде ряда
\begin{equation}
\label{Density_expansion}
g(s)=\sum\limits_{k\in\mathbb{Z}}a_{j_0,k}\phi_{j_0,k}(s)+
\sum\limits_{j\geqslant j_0}\sum\limits_{k\in\mathbb{Z}}b_{j,k}\psi_{j,k}(s)\,,
\end{equation}
где

\noindent
\begin{equation*}
a_{j_0,k}=\int\limits_{-\infty}^{\infty}g(s)\phi_{j_0,k}(s)\,ds\,;\
b_{j,k}\hm=\int\limits_{-\infty}^{\infty}g(s)\psi_{j,k}(s)\,ds\,.
\end{equation*}

Если $Y_1,\ldots,Y_N$~--- выборка из независимых одинаково распределенных
случайных величин с~плотностью $g(s)$, то

\columnbreak

\noindent
\begin{equation*}
\hat{a}_{j_0,k}=\fr{1}{N}\sum\limits_{m=1}^{N}\phi_{j_0,k}(Y_m)\,;\quad
\hat{b}_{j,k}=\fr{1}{N}\sum\limits_{m=1}^{N}\psi_{j,k}(Y_m)
\end{equation*}
являются несмещенными оценками коэффициентов $a_{j_0,k}$ и~$b_{j,k}$
соответственно. Нелинейная оценка вероятностной плотности $g(s)$ имеет вид~[8]:
\begin{multline}
g_N(s)=\sum\limits_{k\in\mathbb{Z}}\hat{a}_{j_0,k}\phi_{j_0,k}(s)+{}\\
{}+
\sum\limits_{j=j_0}^{J}\sum\limits_{k\in\mathbb{Z}}\hat{b}_{j,k}\mathbf{1}
\left(\hat{b}_{j,k}>T_{j,N}\right)\psi_{j,k}(s)\,,
\label{Density_estimate}
\end{multline}
где $T_{j,N}$~--- <<порог>>, а $J$~--- еще один сглажива\-ющий параметр.
Первая сумма в~(\ref{Density_estimate}) представляет собой несмещенную
оценку первой суммы в~(\ref{Density_expansion}). Во второй сумме необходимо
производить пороговую обработку и~усечение до~$J$. Параметры $T_{j,N}$ и~$J$
позволяют находить компромисс между смещением и~дисперсией.

\section{Оценка многомерной плотности и~ее свойства}

Для построения оценки многомерной плот\-ности нельзя использовать непосредственно
формулу~(\ref{Inverse_Radon_transform}) в~силу некорректности задачи обращения
преобразования Радона. Вместо этого используется регуляризованный вариант
этой формулы, в~которой $\abs{\omega}$ умножается на функцию окна
$W_\alpha(\abs{\omega})$, зависящую от сглаживающего параметра~$\alpha$.
Часто в~качестве $W_\alpha(\abs{\omega})$ выбирается функция
$W_\alpha(\abs{\omega})\hm=\exp(-\alpha^2\omega^2/2)$, т.\,е.\
фактически осуществляется свертка проекций с~нормальным распределением.
В~результате вместо $f(x)$ оценивается
\begin{multline}
\label{Regular_inverse}
f_\alpha(x)=\fr{1}{2(2\pi)^{d}}\int\limits_{S^{d-1}}
\int\limits_{-\infty}^{\infty}\abs{\omega}^{d-1}
\exp\left(-\fr{\alpha^2\omega^2}{2}\right)\times{}\\
{}\times \widehat{R_\theta f}(\omega)
\exp\{ i\omega(x\theta)\}\,d\omega d\theta
\end{multline}
и оценка многомерной плотности вычисляется по формуле:
\begin{multline}
\label{Multi_Density_estimate}
f_{N}(x)=\fr{1}{2(2\pi)^{d}}\int\limits_{S^{d-1}}
\int\limits_{-\infty}^{\infty}\abs{\omega}^{d-1}
\exp\left(-\fr{\alpha^2\omega^2}{2}\right)\times{}\\
{}\times \widehat{R_N f}(\theta,\omega)
\exp\{ i\omega(x\cdot\theta)\}\,d\omega d\theta,
\end{multline}
где $R_N f(\theta,s)$~--- вейвлет-оценки одномерных плотностей
$R_\theta f(s)$, построенные по правилу~(\ref{Density_estimate}).
На практике можно вычислить оценки одномерных плотностей только
для конечного числа направлений $\theta_1,\ldots,\theta_M$. Для случая,
когда $f(x)$ имеет конечный носитель, некоторые равномерные оценки близости
функции плотности, восстановленной по конечному числу проекций, к истинной
функции плотности можно найти в~работах~[11, 12]. Здесь будет предполагаться,
что $M$ достаточно велико ($M\hm\gg N$), чтобы можно было игнорировать ошибку,
возникающую из-за использования конечного числа направлений. Докажем,
что равномерное расстояние между $f_{N}(x)$ и~$f_\alpha(x)$ стремится к нулю
в~среднем.

Пусть $W^r(\mathbb{R}^d)$~--- пространство Соболева, т.\,е.\ пространство всех
таких функций $f(x)$, для которых
\begin{equation*}
\norm{f}_r^2=\int\limits_{\mathbb{R}^d}(1+\abs{\omega}^2)^r
\abs{\widehat{f}(\omega)}^2d\omega<\infty\,.
\end{equation*}
Обозначим через $W^r_0(U)$ пространство всех $f(x)\hm\in W^r$, носитель которых
содержится в~некотором компактном множестве $U\subset\mathbb{R}^d$.

\smallskip

\noindent
\textbf{Теорема 1.} \textit{Пусть $f(x)\hm\in W^r_0(U)$, $r>0$.
Пусть $T_{j,N}\hm=C_r\sqrt{j/N}$ ($C_r>0$~--- константа, зависящая
от~$r$}~[8, 13]), $j_0$ \textit{фиксировано, а $J\hm=J(N)$ выбирается
из условия $2^{J(N)}\hm\approx(N/\log N)^{1/(2r+1)}$. Тогда}
\begin{multline}
\label{Uniform_distance_alpha}
\mathbf{E}\sup\limits_{x\in \mathbb{R}^d}
\abs{f_\alpha(x)-f_{N}(x)}\leqslant {}\\
{}\leqslant C_d\alpha^{1/2-d}N^{-({2r+d-1})/(4r+2d)}\,,
\end{multline}
\textit{где $C_d$~--- некоторая положительная константа}.

\smallskip

\noindent
Д\,о\,к\,а\,з\,а\,т\,е\,л\,ь\,с\,т\,в\,о\,.\ \
Используя формулы~(\ref{Regular_inverse}) и~(\ref{Multi_Density_estimate})
и~неравенство Ко\-ши--Бу\-ня\-ков\-ско\-го, получаем:
\begin{multline*}
\mathbf{E}\sup\limits_{x\in \mathbb{R}^d}\abs{f_\alpha(x)-f_{N}(x)}={}
\\
{}=\mathbf{E}\sup\limits_{x\in \mathbb{R}^d}\fr{1}{2(2\pi)^{d}}
\left\vert \int\limits_{\,S^{d-1}}\int\limits_{-\infty}^{\infty}
\abs{\omega}^{d-1}\exp\left(-\fr{\alpha^2\omega^2}{2}\right)\times{}\right.\\
{}\left.\times
\left(\widehat{R_\theta f}(\omega)-\widehat{R_N f}(\theta,\omega)\right)
\exp\{i\omega(x\theta)\}\,d\omega d\theta
\vphantom{\int\limits_{\,S^{d-1}}\int\limits_{-\infty}^{\infty}}
\right\vert\leqslant{}\\
{}\leqslant\fr{1}{2(2\pi)^{d}}\int\limits_{S^{d-1}}
\int\limits_{-\infty}^{\infty}\abs{\omega}^{d-1}
\exp\left(-\fr{\alpha^2\omega^2}{2}\right)\mathbf{E}\times{}\\
{}\times
\abs{\widehat{R_\theta f}(\omega)-\widehat{R_N f}(\theta,\omega)}
\,d\omega d\theta\leqslant{}\\
{}\leqslant\fr{1}{2(2\pi)^{d}}\!\!\int\limits_{S^{d-1}}\!\left(
\int\limits_{\,-\infty}^{\infty}
\abs{\omega}^{2(d-1)}\exp\left(-\alpha^2\omega^2\right)\,d\omega\right)^{1/2}\!\!\!\times{}\\
{}\times \left(
\int\limits_{\,-\infty}^{\infty}\left(\mathbf{E}
\abs{\widehat{R_\theta f}(\omega)-\widehat{R_N f}(\theta,\omega)}\right)^2d
\omega\right)^{1/2}\, d\theta\leqslant{}
\end{multline*}

\noindent
\begin{multline*}
\hspace*{-4pt}{}\leqslant\fr{1}{2(2\pi)^{d}}\!\int\limits_{S^{d-1}}\left(
\int\limits_{\,-\infty}^{\infty}\abs{\omega}^{2(d-1)}
\exp\left(-\alpha^2\omega^2\right)d\omega\right)^{1/2}\!\!\times{}\\
{}\times \left(
\int\limits_{\,-\infty}^{\infty}\mathbf{E}
\abs{\widehat{R_\theta f}(\omega)-\widehat{R_N f}(\theta,\omega)}^2d
\omega\right)^{1/2}\, d\theta\,.
\end{multline*}
Поскольку $f(x)\in W^r_0(U)$, функция $R_\theta f(s)\hm\in W^{p}(\mathbb{R})$ для
любого $\theta\hm\in S^{d-1}$, где $p\hm=r\hm+(d-1)/2$~\cite{14-sh}.
В~работе~\cite{8-sh} показано, что
\begin{multline*}
\left(\int\limits_{\,-\infty}^{\infty}\mathbf{E}
\abs{\widehat{R_\theta f}(\omega)-\widehat{R_N f}(\theta,\omega)}^2d
\omega\right)^{1/2}={}\\
{}=
\left(\int\limits_{\,-\infty}^{\infty}\mathbf{E}
\abs{R_\theta f(s)-R_N f(\theta,s)}^2ds\right)^{1/2}
\leqslant{}\\
{}\leqslant C_p N^{{p}/(2p+1)},
\end{multline*}
где $C_p>0$~--- некоторая константа. Следовательно, существует константа $C_d\hm>0$
такая, что
\begin{equation*}
\mathbf{E}\sup\limits_{x\in \mathbb{R}^d}
\abs{f_\alpha(x)-f_{N}(x)}\leq C_d\alpha^{1/2-d}N^{-{p}/(2p+1)}\,.
\end{equation*}
Теорема доказана.

\smallskip

Функция $f_\alpha(x)$ аппроксимирует функцию $f(x)$, и~чем меньше~$\alpha$, тем
точнее эта аппроксимация. Если дополнительно предположить, что функция $f(x)$
равномерно регулярна по Липшицу с~показателем $1\hm\geqslant\gamma\hm>0$ на~$U$,
т.\,е.\ существует константа $C_L\hm>0$, такая что
\begin{equation}
\label{Lip_condition}
\sup\limits_{x\in U}\abs{f(x)-f(x+y)}\leqslant
C_L\abs{y}^{\gamma}\ \forall\ y\in\mathbb{R}^d\,,
\end{equation}
то можно оценить скорость стремления к нулю в~среднем равномерного
расстояния между $f_{N}(x)$ и~$f(x)$.

\smallskip

\noindent
\textbf{Теорема~2.} \textit{Пусть выполнены условия теоремы~$1$
и~условие регулярности}~(\ref{Lip_condition}).
\textit{Тогда}
\begin{multline}
\label{Uniform_distance}
\mathbf{E}\sup\limits_{x\in \mathbb{R}^d}
\abs{f(x)-f_{N}(x)}\leqslant{}\\
{}\leqslant C_{d,\gamma}N^{-({(2r+d-1)\gamma})/((4r+2d)(\gamma+d-1/2))}\,,
\end{multline}
\textit{где $C_{d,\gamma}$ -- некоторая положительная константа}.

\smallskip

\noindent
Д\,о\,к\,а\,з\,а\,т\,е\,л\,ь\,с\,т\,в\,о\,.\ \ Имеем
\begin{multline*}
\mathbf{E}\sup\limits_{x\in \mathbb{R}^d}\abs{f(x)-f_{N}(x)}\leqslant
\sup\limits_{x\in \mathbb{R}^d}\abs{f(x)-f_\alpha(x)}+{}\\
{}+
\mathbf{E}\sup\limits_{x\in \mathbb{R}^d}\abs{f_\alpha(x)-f_{N}(x)}\,.
\end{multline*}
Далее
\begin{multline}
\sup\limits_{x\in \mathbb{R}^d}\abs{f(x)-f_\alpha(x)}\leqslant{}\\
{}\leqslant
\fr{1}{(2\pi\alpha^2)^{d/2}}\!\int\limits_{\mathbb{R}^d}\abs{f(x)-f(x+y)}
\exp\left(-\fr{\abs{y}^2}{2\alpha^2}\right)dy\leqslant{}
\\
\hspace*{-1mm}\leqslant\fr{C_L}{(2\pi\alpha^2)^{d/2}}\int\limits_{\mathbb{R}^d}
\abs{y}^\gamma\exp\left(\!
-\fr{\abs{y}^2}{2\alpha^2}\!\right)\,dy\leqslant C_{d,L}\alpha^{\gamma},\!\!
\label{Alpha_distance}
\end{multline}
где $C_{d,L}$~--- положительная константа.
Выберем $\alpha=\alpha(N)=N^{-\beta}$, где
\begin{equation*}
\beta=\frac{2r+d-1}{(4r+d)(\gamma+d-1/2)}\,.
\end{equation*}
При таком выборе $\beta$ степени у~$N$~в~оценках~(\ref{Uniform_distance_alpha})
и~(\ref{Alpha_distance}) равны. Следовательно,
используя~(\ref{Uniform_distance_alpha}) и~(\ref{Alpha_distance}),
получаем~(\ref{Uniform_distance}). Теорема доказана.


{\small\frenchspacing
 {%\baselineskip=10.8pt
 \addcontentsline{toc}{section}{References}
 \begin{thebibliography}{99}

\bibitem{1-sh}
\Au{Silverman B.\,W.} Density estimation for statistics and data analysis.~---
London: Chapman and Hall. 1986. 176~p.
\par
\bibitem{2-sh}
\Au{Vidacovic B.} Statistical modeling by wavelets.~--- New York, NY,
USA: John Wiley and Sons.
1999. 408~p.


\bibitem{6-sh} %3
\Au{Tribouley K.} Practical estimation of multivariate densities using
wavelet methods~// Stat. Neerl., 1995. Vol.~49. P.~41--62.

\bibitem{5-sh} %4
\Au{Masry E.} Multivariate probability density estimation by wavelet
methods: Strong consistency and rates for stationary time series~//
Stoch. Proc. Appl., 1997. Vol.~67. P.~177--193.

\bibitem{4-sh} %5
\Au{Dong J., Jiang R.} Multinomial probability estimation by wavelet
thresholding~// Communication in Statistics. Theory and Methods, 2009.
Vol.~38. P.~1486--1507.

\bibitem{3-sh} %6
\Au{Chesneau C., Dewan I., Doosti~H.} Nonparametric estimation of
a~two dimensional continuous-discrete density function by wavelets~//
Stat. Methodol., 2014. Vol.~18. P.~64--78.

\bibitem{7-sh}
\Au{O'Sullivan F., Pawitan Y.} Multidimensional density estimation
by tomography~// J.~Roy. Stat. Soc. B, 1993. Vol.~55. No.\,2. P.~509--521.

\bibitem{8-sh}
\Au{Donoho D.\,L., Johnstone I.\,M., Kerkyacharian~G., Picard~D.}
Density estimation by wavelet thresholding~// Ann. Stat., 1996.
Vol.~23. P.~508--539.

\bibitem{9-sh}
\Au{Mallat S.} A wavelet tour of signal processing.~--- New York, NY, USA: Academic Press, 1999.
851~p.

\bibitem{10-sh}
\Au{Daubechies I.} Ten lectures on wavelets.~--- Philadelphia, PA, USA: SIAM, 1992. 357~p.

\bibitem{11-us}
\Au{Khalfin L.\,A., Klebanov L.\,B.} A~solution of the computer tomography
paradox and estimating the distances between the
densities of measures with the same marginals~// Ann. Probab., 1994.
Vol.~22. P.~2235--2241.

\bibitem{12-sh}
\Au{Шестаков О.\,В., Савенков Т.\,Ю.} Оценка расстояния между
плотностями вероятностных мер, имеющих близкие проекции~// Вестн.
Моск. ун-та. Сер.~15. Вычисл. матем. и~киберн., 2001. №\,4. С.~44--46.

\bibitem{13-sh}
\Au{Hall P., Patil P.} Formulae for mean integrated squared error of nonlinear wavelet-based density estimators~//
 Ann. Stat., 1995. Vol.~23. No.\,3. P.~905--928.

\bibitem{14-sh}
\Au{Smith K.\,T., Solmon~D.\,C., Wagner~S.\,L.} Practical and mathematical
aspects of the problem of reconstructing objects from radiographs~//
Bull. Amer. Math. Soc., 1977. Vol.~83. P.~1227--1270.
 \end{thebibliography}

 }
 }

\end{multicols}

\vspace*{-3pt}

\hfill{\small\textit{Поступила в~редакцию 02.03.15}}

%\newpage

\vspace*{12pt}

\hrule

\vspace*{2pt}

\hrule

%\vspace*{12pt}

\def\tit{NONPARAMETRIC ESTIMATION OF~MULTIDIMENSIONAL
 DENSITY WITH~THE~USE OF~WAVELET ESTIMATES\\ OF~UNIVARIATE PROJECTIONS}

\def\titkol{Nonparametric estimation of multidimensional density with the use of wavelet estimates of univariate projections}

\def\aut{O.\,V.~Shestakov$^{1,2}$}

\def\autkol{O.\,V.~Shestakov}

\titel{\tit}{\aut}{\autkol}{\titkol}

\index{Shestakov O.\,V.}

\vspace*{-9pt}

\noindent
$^1$Department of Mathematical Statistics, Faculty of Computational Mathematics
and Cybernetics,\linebreak
$\hphantom{^1}$M.\,V.~Lomonosov Moscow State University,
1-52 Leninskiye Gory, GSP-1, Moscow 119991, Russian\linebreak
$\hphantom{^1}$Federation

\noindent
$^2$Institute of Informatics Problems, Federal Research
Center ``Computer Science and Control'' of the
Russian\linebreak
$\hphantom{^1}$Academy of Sciences, 44-2 Vavilov Str., Moscow 119333,
Russian Federation


\def\leftfootline{\small{\textbf{\thepage}
\hfill INFORMATIKA I EE PRIMENENIYA~--- INFORMATICS AND
APPLICATIONS\ \ \ 2015\ \ \ volume~9\ \ \ issue\ 2}
}%
 \def\rightfootline{\small{INFORMATIKA I EE PRIMENENIYA~---
INFORMATICS AND APPLICATIONS\ \ \ 2015\ \ \ volume~9\ \ \ issue\ 2
\hfill \textbf{\thepage}}}

\vspace*{3pt}


\Abste{The paper explores the computerized tomography method of
inverting the Radon transformation for obtaining statistical estimates
of multidimensional probability densities. This method utilizes nonlinear
wavelet estimators of univariate projections to construct the multidimensional
density estimate. Nonlinear wavelet estimators possess the ability to adapt
to the local properties of the estimated density function and, therefore,
are less sensitive to the singular points than linear estimators. Another
important practical feature of the considered\linebreak\vspace*{-12pt}}

\Abstend{method is its parallel structure,
which allows a~considerable speedup of constructing the estimates on the computers
supporting parallel processing. It is also proved that under some regularity
conditions, the uniform distance between the constructed estimate and the true
multidimensional probability density converges to zero in the mean, and some
estimates of the rate of this convergence are obtained.}

\KWE{wavelets; multidimensional density; Radon transformation}




\DOI{10.14357/19922264150210}

\Ack
\noindent
The research was financially supported by the Russian Foundation for
Basic Research (project 15-07-02652).



%\vspace*{3pt}

  \begin{multicols}{2}

\renewcommand{\bibname}{\protect\rmfamily References}
%\renewcommand{\bibname}{\large\protect\rm References}



{\small\frenchspacing
 {%\baselineskip=10.8pt
 \addcontentsline{toc}{section}{References}
 \begin{thebibliography}{99}
\bibitem{1-sh-1}
\Aue{Silverman, B.\,W.} 1986. \textit{Density
estimation for statistics and data analysis.} London: Chapman and Hall. 176~p.

\bibitem{2-sh-1}
\Aue{Vidacovic, B.} 1999. \textit{Statistical modeling by wavelets.} New York, NY:
John Wiley and Sons. 408~p.


 \bibitem{6-sh-1} %3
\Aue{Tribouley, K.} 1995. Practical estimation of multivariate densities
using wavelet methods. \textit{Stat. Neerl.} 49:41--62.

\bibitem{5-sh-1} %4
\Aue{Masry, E.} 1997. Multivariate probability density estimation by wavelet
methods: Strong consistency and rates for stationary time series.
\textit{Stoch. Proc. Appl.} 67:177--193.

\bibitem{4-sh-1} %5
\Aue{Dong, J., and R. Jiang.} 2009. Multinomial probability estimation
by wavelet thresholding. \textit{Communication in Statistics. Theory and Methods}
 38:1486--1507.

\bibitem{3-sh-1} %6
\Aue{Chesneau, C., Dewan, I., and H.~Doosti}. 2014.
Nonparametric estimation of a two dimensional continuous-discrete
density function by wavelets. \textit{Stat. Methodol.} 18:64--78.



\bibitem{7-sh-1}
\Aue{O'Sullivan, F., and Y.~Pawitan.} 1993. Multidimensional density
estimation by tomography. \textit{J.~Roy. Stat. Soc. B} 55(2):509--521.

\bibitem{8-sh-1}
\Aue{Donoho, D.\,L., I.\,M.~Johnstone, G.~Kerkyacharian, and D.~Picard}.
1996. Density estimation by wavelet thresholding.
\textit{Ann. Stat.}  23:508--539.

\bibitem{9-sh-1}
\Aue{Mallat, S.} 1999. \textit{A~wavelet tour of signal processing.}
New York, NY: Academic Press. 851~p.

\bibitem{10-sh-1}
\Aue{Daubechies, I.} 1992. \textit{Ten lectures on wavelets.}
Philadelphia, PA: SIAM. 357~p.

\bibitem{11-sh-1}
\Aue{Khalfin, L.\,A., and L.\,B.~Klebanov}. 1994. A~solution of the
computer tomography paradox and estimating the distances between the
densities of measures with the same marginals.
\textit{Ann. Probab.} 22:2235--2241.

\bibitem{12-sh-1}
\Aue{Shestakov, O.\,V., and T.\,Yu.~Savenkov}.
2001. Otsenka rasstoyaniya mezhdu plotnostyami veroyatnostnykh mer,
imeyushchikh blizkie proektsii [Estimating the distances between
the densities of measures with $\varepsilon$-identical marginals].
\textit{Vestn. Mosk. Un-ta. Ser.~15. Vychisl. matem. i~kibern.}
[\textit{Moscow University Computational Mathematics and Cybernetics}] 4:44--46.

\bibitem{13-sh-1}
\Aue{Hall, P., and P. Patil}. 1995. Formulae for
mean integrated squared error of nonlinear wavelet-based density estimators.
\textit{Ann. Stat.} 23(3):905--928.

\bibitem{14-sh-1}
\Aue{Smith, K.\,T., D.\,C.~Solmon, and S.\,L.~Wagner}.
1977. Practical and mathematical aspects of the problem of reconstructing
objects from radiographs. \textit{Bull. Amer. Math. Soc.} 83:1227--1270.

\end{thebibliography}

 }
 }

\end{multicols}

\vspace*{-3pt}

\hfill{\small\textit{Received March 2, 2015}}

%\vspace*{-18pt}


\Contrl

\noindent
\textbf{Shestakov Oleg V.} (b.\ 1976)~---
Doctor of Science in physics and mathematics, associate professor,
Department of Mathematical Statistics, Faculty of Computational Mathematics
and Cybernetics, M.\,V.~Lomonosov Moscow State University,
1-52 Leninskiye Gory, GSP-1, Moscow 119991, Russian Federation;
senior scientist, Institute of Informatics Problems, Federal Research
Center ``Computer Science and Control'' of the
Russian Academy of Sciences, 44-2 Vavilov Str., Moscow 119333,
Russian Federation; oshestakov@cs.msu.su


\label{end\stat}


\renewcommand{\bibname}{\protect\rm Литература}