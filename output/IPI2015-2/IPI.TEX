\documentclass[10pt]{book}
\usepackage[utf8]{inputenc}

\usepackage{latexsym,amssymb,amsfonts,amsmath,indentfirst,shapepar,%fleqn,%
picinpar,shadow,floatflt,enumerate,multicol,colortbl,ipi}

\usepackage{rotating}
\usepackage{mathrsfs}
\usepackage[noend]{algorithmic}
\usepackage{ulem}

\input{epsf}

%\nofiles

%\includeonly{avtor} %+pdf
%\includeonly{obchak,avtor}
%\includeonly{pred}      %
%\includeonly{podgot-rus,podgot-eng}  %+pdf
%\includeonly{ocherk} %+
%\includeonly{nekrol} %+

%\includeonly{vasil}  %1           %pdf+авт
%\includeonly{meykh} %2            %pdf+автнет
%\includeonly{sinitsini} %3        %pdf+автнет
%\includeonly{sin+kor} %4          %pdf+автнет
%\includeonly{berezin} %5          %pdf+авт
%\includeonly{tuchkova} %6         %pdf %6+автнет
%\includeonly{vagapova} %7         %pdf+авт
%\includeonly{krivenko} %8         %pdf+авт
%\includeonly{stenina}  %9         %4-й рисЦвет %pdfавт
%\includeonly{shest}    %10        %pdf+автнет
%\includeonly{kozerenko} %11       %pdf
%\includeonly{zatsman} %12         %pdf



%\includeonly{toc-rus, toc-en}
%\includeonly{obchak} %,toc-en}

%\includeonly{rekl}
%\includeonly{rekl-1}
%\includeonly{reshal}  %
%\includeonly{eng-index}
%\includeonly{cover3}

\usepackage{acad}
%\usepackage{courier}
\usepackage{decor}
\usepackage{newton}
\usepackage{pragmatica}
\usepackage{zapfchan}
\usepackage{petrotex}
\usepackage{bm}                     % полужирные греческие буквы
\usepackage{upgreek}                % прямые греческие буквы
\usepackage{eufrak}
\usepackage{verbatim}

\renewcommand{\bottomfraction}{0.99}
\renewcommand{\topfraction}{0.99}
\renewcommand{\textfraction}{0.01}

\setcounter{secnumdepth}{1} %здесь - 3 + chapter = 4

\arraycolsep=1.5pt

%\usepackage[pdftex]{graphicx}

%\usepackage{oz}

%NEW COMMANDS


\renewcommand*{\hm}[1]{#1\nobreak\discretionary{}%
            {\hbox{$\mathsurround=0pt #1$}}{}} %% Дублирует знаки операций
                               %при переносе в формуле (перед знаком, который
                               %надо продублировать ставится команда \hm)

%\newcommand{\endproof}{\hfill$\Box$}
\renewcommand{\r}{\mathbb{R}}
\newcommand{\I}{{\rm I\hspace{-0.7mm}I}}
%\newcommand{\Ikl}{{\tt{1}}\hspace*{-1.44mm}\mathtt{1}}
\newcommand{\Ik}{\mbox{{\small \tt {1}}\hspace{-1.5mm}{\tt 1}}}
\newcommand{\argmin}{\mathop{\mathrm{arg}\,\mathrm{min}}}
\newcommand{\argmax}{\mathop{\mathrm{arg}\,\mathrm{max}}}
%\newcommand{\capr}{\mathop{\cap\,}}
%\newcommand{\cupr}{\mathop{\cup\,}}
%\def\argmin{\mathop{arg\,min}}

\def\vrp{\varphi}
\def\prt{\partial}
\def\mm{{\rm M}}
\def\modnop#1{\mathop{#1}\limits_{n}}
\def\eam{\mathbin{{\mathop{=}\limits^{\mathrm{def}}}}}
\def\dey#1#2{#1 (#2)}
\def\deyc#1#2{#1 \cdot  #2}
\def\ra#1{\;\mathop{\to}\limits^{#1}\;}
\def\raz#1{\;\mathop{\longrightarrow}\limits^{\!\!\!#1}\;}
\def\ral#1{\;\mathop{\longrightarrow}\limits^{#1}\;}

\newcommand{\Nor}{\mathcal{N}}
\newcommand{\T}{\mathbb{T}}
\newcommand{\Z}{\mathbb{Z}}



\newcommand{\il}[2]{\int\limits_{#1}^{#2}}%интеграл с пределами #1 и #2

%\def\ss2{\mathop {\sum\limits^p\sum\limits^p}}
\def\sss{\sum\limits}
\def\tr{,\,\ldots\,,\,}
\def\rk{\right]}
\def\lk{\left[}
\def\rf{\right\}}
\def\lf{\left\{}
\def\lv{\,\left\vert}
\def\rv{\right\vert\,}
\def\iii{\int\limits}
\def\iin{\int\limits_{-\infty}^\infty}
\def\rrv{\right\vert}


\def\ee{{\cal E}}
\def\ww{{\cal W}}
\def\yy{{\cal Y}}
\def\vv{{\cal V}}

\newcommand{\R}{\mathbb R}
\newcommand{\E}{\mathbb E}
\newcommand{\N}{\mathbb N}

\renewcommand{\P}{\mathbb{P}}

\newcommand{\h}{{\bf H}}
\newcommand{\p}{{\sf P}}  % вероятность

\newcommand{\e}{{\sf E}}  % мат. ожидание
\newcommand{\D}{{\sf D}}  % дисперсия
\newcommand{\eps}{\varepsilon}
\newcommand{\vp}{{\mathbf p}}
\newcommand{\vz}{{\mathbf z}}
\newcommand{\vx}{{\mathbf x}}
\newcommand{\vf}{{\mathbf f}}
\newcommand{\F}{{\mathcal F}}
\def\ap{{\mathrm{ЭР}}}
\newcommand{\ud}{\Delta_n} %uniform ditance
\newcommand{\nud}{\Delta_n(x)}
\renewcommand{\Re}{\mathrm{Re}\,}

\newcommand{\abs}[1]{\left\vert#1\right\vert}
\newcommand{\norm}[1]{\left\Vert#1\right\Vert}
\def\da{(\Delta_t,A)}

\newcommand{\corr}{\mathrm{corr}}

\newcommand{\cov}{\mathrm{cov}}
\newcommand{\Expect}{\mathbb{E}}

\def\w{\omega}
\def\W{\Omega}

\def\inh{\int\limits_{nh}^{(n+1)h}}

\def\sumin{\sum_{i=1}^N}


\def\bxt{(Y,t)}
\def\xt{(y,t)}

\def\ovth{{\fr{\tau-nh}{h}}}
\def\ov{\overline}
\def\tm{\tilde m}


\DeclareMathOperator{\sign}{sign}

%\newcommand{\gr}{{\geqslant}}


\newcommand{\g}{\mbox{\textit{g}}}

\renewcommand{\la}{\lambda}
\newcommand{\si}{\sigma}
\newcommand{\alp}{\alpha}

%\newcommand{\pto}{\stackrel{P}{\longrightarrow}} % сходимость по веpоятности

\newcommand{\eqd}{\stackrel{\mathrm{d}}{=}} % равенство по pаспpеделению
\newcommand{\eqdelta}{\stackrel{\Delta}{=}} % равенство по pаспpеделению

\def\be#1{\begin{equation}\label{#1}}
\def\ee{\end{equation}}
\def\re#1{(\ref{#1})}

\def\bn{\begin{enumerate}}
\def\en{\end{enumerate}}
\def\bi{\begin{itemize}}
\def\ei{\end{itemize}}
%\def\i{\item}

%\newcommand{\kp}{\kappa}
%\def\Q{{\cal Q}} \def\H{{\cal H}}
%\newcommand{\bet}{\beta_{2+\delta}}


%\newtheorem{definition}{Определение}
%\renewcommand{\thedefinition}{\arabic{definition}.}
%END NEW COMMANDS

%\renewcommand{\baselinestretch}{1.2}

%\pagestyle{myheadings}

\setlength{\textwidth}{167mm}      % 122mm
\setlength{\textheight}{658pt}
%\setlength{\textheight}{635.6pt}
\setlength{\columnsep}{4.5mm}

\setcounter{secnumdepth}{4}

%\addtolength{\headheight}{2pt}
%\addtolength{\headsep}{-2mm}

%\addtolength{\topmargin}{-20mm}  % for printing


%\hoffset=-30mm  % From Yap
\hoffset=-23mm  % From Acrobat

%\voffset=0mm % From Yap
%\voffset=-15mm   % From Acrobat

\addtolength{\evensidemargin}{-9.5mm} % for printing
\addtolength{\oddsidemargin}{9.5mm}  % for printing

%\renewcommand{\thefootnote}{\fnsymbol{footnote}}
%\renewcommand{\thefootnote}{\arabic{footnote}}
\renewcommand{\figurename}{\protect\bf Рис.}
\renewcommand{\tablename}{\protect\bf Таблица}

\newcommand{\Caption}[1]{\caption{\protect\small %\baselineskip=2.5ex
#1}}

\renewcommand{\thefigure}{\arabic{figure}}
\renewcommand{\thetable}{\arabic{table}}
\renewcommand{\theequation}{\arabic{equation}}
\renewcommand{\thesection}{\arabic{section}}

\renewcommand{\contentsname}{СОДЕРЖАНИЕ}
\newcommand{\fr}[2]{\displaystyle\frac{\displaystyle #1\mathstrut}{\displaystyle #2\mathstrut}}

%\renewcommand{\thefootnote}{\fnsymbol{footnote}}
%\newcommand{\g}{\mbox{\textit{g}}}

%\newcommand{\Caption}[1]{\caption{\protect\small\baselineskip=2ex #1}}
\newcounter{razdel}
\setcounter{razdel}{0}


\newcommand{\titel}[4]{%
\

\vspace*{5pt}

\ifodd\therazdel {\raggedright\noindent\Large\textrm\textbf
 \lineskip .75em
  \baselineskip=3.2ex #1 \par}
\vskip 1em {\noindent\large\textrm\textbf #2 \par}
\addcontentsline{toc}{subsection}{{\textrm\textbf #3}\protect\newline #1}
\def\rightheadline{\underline{\noindent\hbox to \textwidth{\hfill\small\textrm{#4}
%\hfill \large\bf\thepage
}}}
\def\leftheadline{\underline{\noindent\parbox{\textwidth}{
%\raggedleft\large\bf\thepage \hfill
\small\textit{#3}\hfill}}}
\def\leftfootline{\small{\textbf{\thepage}
\hfill ИНФОРМАТИКА И ЕЁ ПРИМЕНЕНИЯ\ \ \ том~9\ \ \ выпуск 2\ \ \ 2015}
}%
 \def\rightfootline{\small{ИНФОРМАТИКА И ЕЁ ПРИМЕНЕНИЯ\ \ \ том~9\ \ \ выпуск~2\ \ \ 2015
\hfill \textbf{\thepage}}}
\vskip 2em \setcounter{figure}{0}
\setcounter{table}{0}
\setcounter{equation}{0}
\setcounter{section}{0}
\setcounter{subsection}{0}
\setcounter{subsubsection}{0}
\setcounter{footnote}{0}
\setcounter{razdel}{0}
%\end{flushleft}
\else {
 \raggedright\noindent\Large\textrm\textbf
 \lineskip .75em
\baselineskip=3.2ex #1 \par} \vskip 1em
%\begin{flushleft}
{\noindent\large\textrm\textbf #2 \par}
\addcontentsline{toc}{subsection}{{\textrm\textbf #3}\protect\newline #1}
\def\rightheadline{\underline{\noindent\hbox to \textwidth{\hfill\small\textrm{#4}
%\hfill \large\bf\thepage
}}}
\def\leftheadline{\underline{\noindent\parbox{\textwidth}{%\raggedleft\large\bf\thepage \hfill
\small\textit{#3}\hfill}}}
\def\leftfootline{\small{\textbf{\thepage}
\hfill ИНФОРМАТИКА И ЕЁ ПРИМЕНЕНИЯ\ \ \ том~9\ \ \ выпуск~2\ \ \ 2015}
}%
 \def\rightfootline{\small{ИНФОРМАТИКА И ЕЁ ПРИМЕНЕНИЯ\ \ \ том~9\ \ \ выпуск~2\ \ \ 2015
\hfill \textbf{\thepage}}} \vskip 2em \setcounter{figure}{0}
\setcounter{table}{0} \setcounter{equation}{0} \setcounter{section}{0}
\setcounter{subsection}{0} \setcounter{subsubsection}{0}
\setcounter{footnote}{0}
%\end{flushleft}
\fi}

\newcommand{\titelr}[2]{%
\

\vspace*{5pt}

\ifodd\therazdel {\raggedright\noindent%\Large\textrm\textbf
 \lineskip .75em
  \baselineskip=3.2ex #1 \par}
\vskip 1em {\noindent\normalsize\textrm\textbf #2 \par}
\else {
 \raggedright\noindent\Large\textrm\textbf
 \lineskip .75em
\baselineskip=3.2ex #1 \par} \vskip 1em
%\begin{flushleft}
{\noindent\large\textrm\textbf #2 \par
%\noindent\normalsize\textrm\textbf #2 \par
} \fi}

\newcommand{\titele}[5]{%
\

%\vspace*{5pt}

\ifodd\therazdel {\raggedright\noindent\large
\textrm\textbf
 \lineskip .75em
%  \baselineskip=3.2ex
#1 \par}
\vskip .5em {\noindent\large\textrm\textbf #2 \par}
\vskip .5em
 {\noindent\textrm #3 \par}
\addcontentsline{toc}{subsection}{{\textrm\textbf #1}\protect\newline #2}
\def\rightheadline{\underline{\noindent\hbox to \textwidth{\hfill\small\textrm{#4}
%\hfill \large\bf\thepage
}}}
\def\leftheadline{\underline{\noindent\parbox{\textwidth}{
%\raggedleft\large\bf\thepage \hfill
\small\textrm{#5}\hfill}}}
\def\leftfootline{\small{\textbf{\thepage}
\hfill ИНФОРМАТИКА И ЕЁ ПРИМЕНЕНИЯ\ \ \ том~9\ \ \ выпуск~2\ \ \ 2015}
}%
 \def\rightfootline{\small{ИНФОРМАТИКА И ЕЁ ПРИМЕНЕНИЯ\ \ \ том~9\ \ \ выпуск~2\ \ \ 2015
\hfill \textbf{\thepage}}} \vskip 1em \setcounter{figure}{0}
\setcounter{table}{0} \setcounter{equation}{0} \setcounter{section}{0}
\setcounter{subsection}{0} \setcounter{subsubsection}{0}
\setcounter{footnote}{0} \setcounter{razdel}{0}
%\end{flushleft}
\else {
 \raggedright\noindent\large
 \textrm\textbf
 \lineskip .75em
%\baselineskip=3.2ex
#1 \par} \vskip .5em
%\begin{flushleft}
{\noindent\large\textrm\textbf #2 \par} \vskip .5em
 {\noindent\textrm #3 \par}
\addcontentsline{toc}{subsection}{{\textrm\textbf #1}\protect\newline #2}
\def\rightheadline{\underline{\noindent\hbox to \textwidth{\hfill\small\textrm{#4}
%\hfill \large\bf\thepage
}}}
\def\leftheadline{\underline{\noindent\parbox{\textwidth}{%\raggedleft\large\bf\thepage \hfill
\small\textrm{#5}\hfill}}}
\def\leftfootline{\small{\textbf{\thepage}
\hfill ИНФОРМАТИКА И ЕЁ ПРИМЕНЕНИЯ\ \ \ том~9\ \ \ выпуск~2\ \ \ 2015}
}%
 \def\rightfootline{\small{ИНФОРМАТИКА И ЕЁ ПРИМЕНЕНИЯ\ \ \ том~9\ \ \ выпуск~2\ \ \ 2015
\hfill \textbf{\thepage}}} \vskip 1em \setcounter{figure}{0}
\setcounter{table}{0} \setcounter{equation}{0} \setcounter{section}{0}
\setcounter{subsection}{0} \setcounter{subsubsection}{0}
\setcounter{footnote}{0}
%\end{flushleft}
\fi}

\def\Abst#1{
\begin{center}\small\nwt
\parbox{150mm}{%\baselineskip=2.5ex
\textbf{Аннотация:}\ \
%\hspace*{\parindent}
#1}
\end{center}}
\def\Abste#1{
\begin{center}\small\nwt
\parbox{150mm}{%\baselineskip=2.5ex
\textbf{Abstract:}\ \
%\hspace*{\parindent}
#1}
\end{center}}

\def\DOI#1{
\begin{center}\small\nwt
\parbox{150mm}{%\baselineskip=2.5ex
\textbf{DOI:}\ \
%\hspace*{\parindent}
#1}
\end{center}}

\def\Abstend#1{
\begin{center}\small\nwt
\parbox{150mm}{%\baselineskip=2.5ex
%\hspace*{\parindent}
#1}
\end{center}}


\def\KW#1{
\begin{center}\small\nwt
\parbox{150mm}{%\baselineskip=2.5ex
\textbf{Ключевые слова:}\ \ #1}
\end{center}}

\def\KWE#1{
\begin{center}\small\nwt
\parbox{150mm}{%\baselineskip=2.5ex
\textbf{Keywords:}\ \ #1}
\end{center}}


\def\KWN#1{
%\begin{center}
%\small
%\parbox{150mm}\end{center}
}

\renewcommand{\thesubsection}{\thesection.\arabic{subsection}\hspace*{-5pt}}
\renewcommand{\thesubsubsection}{\thesubsection\hspace*{5pt}.\arabic{subsubsection}\hspace*{-3pt}}

\newcommand{\Ack}{\section*{\protect\rmfamily Acknowledgments}\noindent}
\newcommand{\Contr}{\section*{\protect\rmfamily Contributors}\noindent}
\newcommand{\Contrl}{\section*{\protect\rmfamily Contributor}\noindent}

\makeindex

\begin{document}
\Rus

\nwt
%\ptb


%\renewcommand{\contentsname}{\protect\Large\bf Содержание}

\setcounter{tocdepth}{2}

%\tableofcontents

\renewcommand{\bibname}{\protect\rmfamily Литература}
  \def\Au#1{{\it #1}}
    \def\Aue#1{{#1}}

%\newcommand{\No}{№}
  \newcommand{\tg}{\,\mathrm{tg}\,}
    \newcommand{\ctg}{\,\mathrm{ctg}\,}
  \newcommand{\arctg}{\,\mathrm{arctg}\,}

\def\forallb{\mathop{\forall}}
\def\cupb{\mathop{\cup}}
\def\existsb{\mathop{\exists}}


\newpage
\addtocounter{razdel}{1}
%\def\razd{РЕГУЛИРУЕМЫЙ ЭЛЕКТРОПРИВОД ДЛЯ ЭЛЕКТРОЭНЕРГЕТИКИ}


\setcounter{page}{2}


%   { %\Large  
   { %\baselineskip=16.6pt
   
   \vspace*{-48pt}
   \begin{center}\LARGE
   \textit{Предисловие}
   \end{center}
   
   %\vspace*{2.5mm}
   
   \vspace*{25mm}
   
   \thispagestyle{empty}
   
   { %\small 

    
Вниманию читателей журнала <<Информатика и её применения>> предлагается 
очередной тематический выпуск <<Вероятностно-статистические методы и 
задачи информатики и информационных технологий>>. Предыдущие тематические 
выпуски журнала по данному направлению вышли в 2008~г.\ (т.~2, вып.~2), 
в 2009~г.\ (т.~3, вып.~3) и в 2010~г.\ (т.~4, вып.~2). 

Статьи, собранные в данном журнале, посвящены разработке новых вероятностно-статистических 
методов, ориентированных на применение к решению конкретных задач информатики и информационных 
технологий, а также~--- в ряде случаев~--- и других прикладных задач. Проблематика, охватываемая 
публикуемыми работами, развивается в рамках научного сотрудничества между Институтом проблем 
информатики Российской академии наук (ИПИ РАН) и Факультетом вычислительной математики и 
кибернетики Московского государственного университета им.\ М.\,В.~Ломоносова в ходе работ 
над совместными научными проектами (в том числе в рамках функционирования 
Научно-образовательного центра <<Вероятностно-статистические методы анализа рисков>>). 
Многие из авторов статей, включенных в данный номер журнала, являются активными участниками 
традиционного международного семинара по проблемам устойчивости стохастических моделей, 
руководимого В.\,М.~Золотаревым и В.\,Ю.~Королевым; регулярные сессии этого семинара 
проводятся под эгидой МГУ и ИПИ РАН (в 2011~г.\ указанный семинар проводится в октябре 
в Калининградской области РФ). 

Наряду с представителями ИПИ РАН и МГУ в число авторов данного выпуска журнала входят 
ученые из Научно-исследовательского института системных исследований РАН, Института 
проблем технологии микроэлектроники и особочистых материалов РАН, Института 
прикладных математических исследований Карельского НЦ РАН, Московского 
авиационного института, Вологодского государственного педагогического университета, 
НИИММ им.\ Н.\,Г.~Чеботарева, Казанского государственного университета, Дебреценского 
университета (Венгрия).

Несколько статей выпуска посвящено разработке и применению стохастических методов и 
информационных технологий для решения различных прикладных задач. В~работе В.\,Г.~Ушакова 
и О.\,В.~Шестакова рассмотрена задача определения вероятностных характеристик случайных 
функций по распределениям интегральных преобразований, возникающих в задачах эмиссионной 
томографии. В~статье Д.\,О.~Яковенко и М.\,А.~Целищева рассмотрены некоторые вопросы 
математической теории риска и предложен новый подход к диверсификации инвестиционных 
портфелей. Работа И.\,А.~Кудрявцевой и А.\,В.~Пантелеева посвящена построению и 
исследованию математической модели, описывающей динамику сильноионизованной плазмы. 
В~статье П.\,П.~Кольцова изучается качество работы ряда алгоритмов сегментации изображений. 
Статья А.\,Н.~Чупрунова и И.~Фазекаша посвящена вероятностному анализу числа без\-оши\-бочных 
блоков при помехоустойчивом кодировании; получены усиленные законы больших чисел для указанных 
величин.

В данном выпуске традиционно присутствует тематика, весьма активно разрабатываемая в течение 
многих лет специалистами ИПИ РАН и МГУ,~--- методы моделирования и управления для 
информационно-телекоммуникационных и вычислительных систем, в частности методы 
теории массового обслуживания. В~статье А.\,И.~Зейфмана с соавторами рассматриваются 
модели обслуживания, описываемые марковскими цепями с непрерывным временем в случае 
наличия катастроф. В~работе М.\,М.~Лери и И.\,А.~Чеплюковой рассматриваются случайные 
графы Интернет-типа, т.\,е.\ графы, степени вершин которых имеют степенные распределения; 
такие задачи находят применение при исследовании глобальных сетей передачи данных. 
Работа Р.\,В.~Разумчика посвящена исследованию систем массового обслуживания специального 
вида~--- с отрицательными заявками и хранением вытесненных заявок.

Ряд статей посвящен развитию перспективных теоретических 
вероятностно-статистических методов, которые находят широкое применение в различных 
задачах информатики и информационных технологий. В~работе В.\,Е.~Бенинга, А.\,К.~Горшенина 
и В.\,Ю.~Королева рассмотрена задача статистической проверки гипотез о числе компонент 
смеси вероятностных распределений, приводится конструкция асимптотически наиболее мощного 
критерия. Результаты этой работы найдут применение в ряде прикладных задач, использующих 
математическую модель смеси вероятностных распределений (в информатике, моделировании 
финансовых рынков, физике турбулентной плазмы и~т.\,д.). В~статье В.\,Ю.~Королева, 
И.\,Г.~Шевцовой и С.\,Я.~Шоргина строится новая, улучшенная оценка точности нормальной 
аппроксимации для пуассоновских случайных сумм; как известно, указанные случайные суммы 
широко используются в качестве моделей многих реальных объектов, в том числе в информатике, 
физике и других прикладных областях. Работа В.\,Г.~Ушакова и Н.\,Г.~Ушакова посвящена 
исследованию ядерной оценки плотности распределения; эти результаты могут применяться, 
в част\-ности, при анализе трафика в телекоммуникационных системах. Серьезные приложения 
в статистике могут получить результаты работы О.\,В.~Шестакова, в которой доказаны оценки 
скорости сходимости распределения выборочного абсолютного медианного отклонения к нормальному 
закону. 

\smallskip

Редакционная коллегия журнала выражает надежду, что данный тематический  выпуск 
будет интересен специалистам в области теории вероятностей и математической статистики 
и их применения к решению задач информатики и информационных технологий.
     
     %\vfill 
     \vspace*{20mm}
     \noindent
     Заместитель главного редактора журнала <<Информатика и её 
применения>>,\\
     директор ИПИ РАН, академик  \hfill
     \textit{И.\,А.~Соколов}\\
     
     \noindent
     Редактор-составитель тематического выпуска,\\
     профессор кафедры математической статистики факультета\\
      вычислительной математики и кибернетики МГУ им.\ М.\,В.~Ломоносова,\\
     ведущий научный сотрудник ИПИ РАН,\\ 
доктор физико-математических наук \hfill
      \textit{В.\,Ю.~Королев}
     
     } }
     }

\def\stat{vasil}

\def\tit{ИСПОЛЬЗОВАНИЕ ПРИНЦИПА РАВНОВЕСИЯ ДЛЯ~УПРАВЛЕНИЯ 
МАРШРУТИЗАЦИЕЙ В~ТРАНСПОРТНЫХ СЕТЯХ}

\def\titkol{Использование принципа равновесия для~управления 
маршрутизацией в~транспортных сетях}

\def\autkol{Н.\,С.~Васильев}

\def\aut{Н.\,С.~Васильев$^1$}

\titel{\tit}{\aut}{\autkol}{\titkol}

%{\renewcommand{\thefootnote}{\fnsymbol{footnote}} \footnotetext[1]{Работа 
%выполнена при финансовой поддержке РФФИ (проект 11-01-00515а).}}

\renewcommand{\thefootnote}{\arabic{footnote}}
\footnotetext[1]{Московский государственный технический университет им.\ Н.\,Э.~Баумана, nik8519@yandex.ru} 
   
    
  
  \Abst{Выбор алгоритмов управления передачей должен основываться на 
принципах функциональной эффективности (увеличить быстродействие сети) и 
устойчивости передачи (принцип равновесия). В~сетях передачи данных имеется 
огромное число тяготеющих пар пользователей, каждая из которых заинтересована 
в быстроте доставки своих сообщений. Таким образом, качество функционирования 
сети необходимо оценивать с помощью векторного критерия. (Существуют также и 
другие характеристики сетей.) Поэтому проектирование системы управления 
передачей осуществляется с учетом векторных целевых функций, а принимаемые 
(реализуемые) решения должны искаться методами векторной оптимизации. 
  Стремление улучшить качество передачи с целью наилучшего (по возможности) 
удовлетворения пользователей сети стимулирует поиск новых методов 
маршрутизации сообщений. В~работе предложен метод маршрутизации, 
основанный на применении игрового принципа равновесия (по Нэшу). Игровая 
постановка задачи маршрутизации и указанное понятие решения (равновесие) 
формализуют пред\-став\-ле\-ние об оптимальности управления передачей в 
распределенной системе.
  Использование принципа равновесия предполагает наличие ответа на следующие 
главные вопросы: всегда ли равновесие достижимо, устойчиво ли оно и как его 
найти. При общих предположениях в работе доказано существование равновесия по 
Нэшу. Установлено, что равновесие обладает дополнительными свойствами~--- 
вычислительной устойчивостью и эффективностью (оптимальностью) в смысле 
Парето. Предложен быстрый параллельный (игровой) алгоритм поиска равновесной 
маршрутизации и обоснована его сходимость.} 
    
    \KW{пакетная сеть; потоки в сетях; метрика сети; маршрутизация; векторный 
критерий; многокритериальная оптимизация; игровая задача; равновесие по Нэшу; 
эффективность по Парето}
  
  
  \DOI{10.14357/19922264140104}

\vskip 20pt plus 9pt minus 6pt

      \thispagestyle{headings}

      \begin{multicols}{2}

            \label{st\stat}


  \section{Введение}
   
  Экспоненциальный рост объемов передаваемой по сетям информации 
стимулирует исследования, связанные с совершенствованием не только сетевого 
оборудования, но и алгоритмов управления передачей в пакетных сетях. В~таких 
сетях каждая тяготеющая пара пользователей заинтересована в наискорейшей 
передаче своих сообщений за счет выбора оптимальных маршрутов доставки. 
В~глобальных сетях невозможно обеспечить централизованное управление 
маршрутизацией. Поэтому применяются распределенные параллельные алгоритмы 
(сетевые протоколы). 
  
  В каждый момент времени в сети имеется огромное число тяготеющих пар. 
Значит, задача\linebreak оп\-тимального управ\-ле\-ния передачей (задача марш-\linebreak рутизации~[1]) 
является многокритериальной. Требуется найти маршрутизацию, наилучшим 
образом удовле\-тво\-ря\-ющую всех пользователей сети. Сетевые задачи с векторными 
критериями ранее исследовались, например, в работах~[2--4]. 
  
  В статье поставлена и решена \textit{игровая} многокритериальная 
оптимизационная задача маршрутизации. Одним из подходов к решению задачи 
оптимизации векторного критерия является его сворачивание в скалярный 
критерий. Недостатком этого подхода является то, что распределенная модель 
системы заменяется на централизованную. В~результате строящиеся 
алгоритмы поиска решения не обладают той степенью параллелизма, которая 
допускает распределенную реализацию. 
  
  Переход к задаче математического программирования (в случае 
дифференцируемости целевой функции) позволяет использовать градиентный 
метод поиска оптимального решения, принимаемого в качестве решения исходной 
задачи. Так вы\-нуж\-де\-ны поступать, когда не удается найти подходящие (быстрые) 
алгоритмы поиска решения исходной многокритериальной задачи. 

Указанный 
подход применен в работе~[1] для решения задачи маршрутизации. 
  
  В статье предложен метод сведения игровой задачи оптимизации к 
\textit{эквивалентной} задаче математического программирования с целью 
построения игрового (параллельного) алгоритма поиска решения (равновесия) 
исходной задачи. Этот подход основан на введении новой метрики сети, 
модифицирующей имеющуюся. После этого на итерациях алгоритма поочередно 
для любой пары абонентов сети строится набор маршрутов передачи, 
оптимизирующих время доставки каждого сообщения. Эти вычисления основаны 
на использовании принципа уравнивания Ю.\,Б.~Гермейера, применяемого при 
решении минимаксных задач. 
  
  В отличие от градиентного метода, вычисления маршрутов передачи проводятся 
поочередно для каждой тяготеющей пары в отдельности. Это позволяет реализовать 
алгоритм так, чтобы изменение маршрутизации сети проводить одновременно 
(независимо) для многих пар абонентов.
  
  Управление потоками в пакетных телекоммуникационных системах 
(транспортных сетях~--- ТС) основано на моделях сетей с переменной 
  метрикой~[1--10]. Изменяемая метрика присуща даже однопродуктовой сети 
(имеется единственная пара абонентов) из-за наличия обратной связи между 
потоками и задержкой в передаче пакетов. Напомним, что задержки в линиях связи 
определяют метрику, с помощью которой оценивается быстродействие сети. 
  
  Игнорирование указанной обратной связи при построении параллельных 
алгоритмов маршрутизации не позволяет обеспечить устойчивое управ\-ле\-ние 
потоками ТС. Это наблюдается даже в сетях, имеющих кольцевую архитектуру. 
Так, поочередный выбор кратчайшего маршрута в текущей мет\-ри\-ке сети для 
передачи сообщений между всеми тяготеющими парами может приводить к 
возникновению колебательного процесса~[1]. В~результате в сети возникают 
потоки, вызывающие ее перегрузку, хотя имеется маршрутизация, при которой сеть 
справляется с заданными входными потоками. Для поиска соответствующей 
допустимой маршрутизации сети достаточно применить адекватный (а не 
эвристический, как в~[1]) игровой алгоритм маршрутизации. 
  
  Свойства ТС с переменной метрикой~\cite{10-vasil} ра-\linebreak нее изучались в связи с 
поиском равновесной марш\-рутизации глобальной пакетной сети передачи\linebreak 
данных~\cite{2-vasil, 4-vasil, 7-vasil}. При этом существование равновесной 
маршрутизации удавалось теоретически обосно\-вать лишь для сетей с топологией, 
мало отличающейся от кольцевой. При численном моделировании удавалось 
строить равновесную маршрутизацию для весьма широкого класса сетей.\linebreak Так, в 
результате проведения вы\-чис\-ли\-тель\-ных\linebreak экспериментов равновесие по Нэшу 
достигалось в модели сети Интернет~\cite{7-vasil}. Таким образом, чис\-лен\-ный 
\mbox{поиск} равновесного решения задачи уже прошел экспериментальную апробацию, но 
не получил должного теоретического обоснования. 
  
  Данная статья посвящена доказательству общей теоремы существования 
равновесия и установлению его свойств~--- вычислительной устойчивости и 
эффективности по Парето. 
  
  Доказана сходимость алгоритма поиска равновесной маршрутизации. Простота и 
параллельные свойства алгоритма позволяют надеяться на его применение при 
создании новых сетевых протоколов транспортного уровня~[1], основанных на 
использовании принципа равновесия. 
  
  \section{Оптимизационная модель транспортной системы}
  
  Топологию ТС будем представлять в виде связного неориентированного графа 
$\Gamma\hm = (U,V)$, вдоль ребер $l\hm\in V$, $l\hm= 1, 2,\ldots , n$, которого 
расположены линии (каналы) передачи пакетов, а в узлах $u\hm\in U$ размещены 
источники и стоки передаваемых потоков. (Требование неориентированности графа 
несущественно.) 
  
  Доставка сообщений для каждой \textit{тяготеющей} $k$-й пары 
  (ис\-точ\-ник--сток), $k\hm=1, 2, \ldots ,K$, осуществляется по одному или 
нескольким выбираемым маршрутам графа сети $\{L_j^k\}$, соединяющим эти 
узлы. Входные (случайные) потоки интенсивности $\lambda_k\hm=\lambda_0^k$ 
поступают в узлы-источники, разделяются в них (алгоритмом маршрутизации) на 
маршрутные потоки величины $\{\lambda_j^k\}$ и по маршрутам $\{L_j^k\}$ 
передаются в соответствующие уз\-лы-сто\-ки, из которых покидают~ТС. 
  
   Функционирование сети происходит с задержками на линиях сети $l\hm=1, 2, 
\ldots , n$, равными значениям некоторой функции $f_l(z_l)$, зависящей от величин 
интенсивностей потоков~$z_l$ на этих ли\-ниях.
   
  Например, в пакетных сетях передачи данных величина задержки всякого пакета 
на любой линии складывается из следующих величин~\cite{1-vasil, 8-vasil}:
 \begin{itemize}
\item времени ожидания пакета в очереди; 
\item времени определения направления дальнейшей передачи (в транзитный 
узел сети) с помощью маршрутной таблицы; 
\item времени пересылки пакета по выбранной линии.
\end{itemize}

  Определение функций задержек составляет самостоятельную 
  задачу~\cite{1-vasil, 8-vasil, 9-vasil}. Во всяком случае, эти функции 
неотрицательны, монотонно и неограниченно возрастают при увеличении 
интенсивности потока по линии до величины ее пропускной способности. 
(Согласно теории массового обслуживания при совпадении интенсивностей 
поступления и обслуживания заявок наблюдается неограниченный рост очередей.)
  
  Время доставки продуктов вдоль маршрута~$L$ (или <<длина>> маршрута~$L$) 
равно сумме задержек:
  \begin{equation}
  \rho_f (L,z)=\sum\limits_{l\in L} f_l(z_l)\,.
  \label{e1-vasil}
  \end{equation}
  
  Зафиксировав векторную функцию $f\hm=(f_1,f_2,\ldots , f_n)$, опустим 
индекс~$f$ в обозначении метрики сети~(\ref{e1-vasil}). Потоки в сети задаются в 
единицах измерения интенсивности передачи. По смыслу использованных 
обозначений для всех допустимых значений индексов~$k$ и~$l$ должны 
выполняться балансовые соотношения
  \begin{equation}
  \sum\limits_{j=1}^{J_k} \lambda_j^k=\lambda_0^k\,,\enskip \sum\limits_{j,k\in 
L_j^k} \lambda_j^k=z_l\,.
  \label{e2-vasil}
  \end{equation}
    Через $z_l$ в~(\ref{e2-vasil}) обозначен суммарный поток по\linebreak $l$-й линии ТС, 
который ограничен величиной $\overline{z}_l$~--- пропускной способностью 
линии:
  \begin{equation}
  z_l\leq \overline{z}_l\,,\quad l=1, 2, \ldots ,n\,.
  \label{e3-vasil}
  \end{equation}
  
  Под \textit{допустимой маршрутизацией} (ДМ) сети (для $k$-й тяготеющей 
пары) будем понимать совокупность маршрутов $M^k\hm=\{L_j^k\}$ и 
маршрутных потоков $\{\lambda_j^k\}$ (интенсивностей передачи) для всех 
тяготеющих пар $k\hm=1, 2, \ldots , K$ такую, что выполняются 
соотношения~(\ref{e2-vasil}), (\ref{e3-vasil}) и $\rho(L_j^k,z)\hm<\infty$. 
  
  Векторный входной поток $(\lambda_0^1, \lambda_0^2, \ldots , \lambda_0^K)$ 
называется \textit{допустимым}, если для него найдется~ДМ.
  
  Задавая маршрутизацию, будем перечислять только \textit{применяемые} 
маршруты передачи, для которых маршрутные потоки положительны. В~процессе 
управ\-ле\-ния ТС за счет (принудительного) ограничения входного потока 
обеспечивается его допустимость. Выбор ДМ
  $$
  M=\left\{ \left\{ L_j^k\right\}, \left\{ \lambda_j^k\right\}, k=1, 2, \ldots ,K\right\}
  $$
однозначно задает вектор допустимых потоков по линиям ТС $z\hm= (z_1, z_2, 
\ldots , z_n)$, от которого зависит время передачи сообщений (вместе с 
вектором~$\mathbf{z}$ изменяется метрика сети~(\ref{e1-vasil})). 
  
  Любая $k$-я тяготеющая пара <<заинтересована>> в уменьшении времени 
доставки пакетов, определяемого длинами применяемых маршрутов 
передачи~(\ref{e1-vasil}). В~сети с потоками~$\mathbf{z}$ минимально возможное 
время доставки пакетов $k$-й пары равно
  $$
 \underline{T}^k(z)=\min\limits_{\{L^k\}} \rho(L^k,z)\,,
  $$
  где $\{L^k\}$~--- множество всех маршрутов, соединя\-ющих эту пару.
  
  Маршрутизацию $k$-й тяготеющей пары назовем \textit{оптимальной}, если все 
применяемые маршруты передачи имеют минимальную длину, равную 
$\underline{T}^k(z)$.
  
  Так как вектор~$\mathbf{z}$ зависит от выбираемой маршрутизации 
(см.~(\ref{e2-vasil})), то оптимальным решением этой задачи для $k$-й пары 
является выбор ее кратчайших маршрутов соединения, длина которых оценивается 
с помощью изменяющейся (вместе с сетевыми потоками) 
  мет\-ри\-кой~(\ref{e1-vasil})~\cite{5-vasil, 6-vasil, 10-vasil}.
  
  Далее исследуется \textit{игровая} задача об оп\-ти\-мальной маршрутизации ТС. 
Тяготеющие пары\linebreak $k\hm= 1, 2, \ldots , K$ рассматриваются как игроки в 
бескоалиционной игре, выбирающие свою стратегию~--- 
  маршрутизацию~\cite{2-vasil, 4-vasil, 7-vasil}.
  
  Равновесие по Нэшу в этой игре назовем \textit{равновесной} (оптимальной) 
маршрутизацией ТС. Соответствующий вектор потоков в сети также будем 
называть \textit{равновесным}. 
  
  В равновесии каждой тяготеющей паре \textit{невыгодно} отклоняться от своей 
маршрутизации из-за того, что время передачи потоков только увеличится при 
условии, что все остальные пары придерживаются своих маршрутов и 
интенсивностей передачи. В~этом заключается устойчивость (в игровом смыс\-ле) 
равновесного решения.
  
  На практике выбор маршрутизации может проводиться не самими тяготеющими 
парами, а с по\-мощью некоторого алгоритма, построенного на этом принципе. 
(Таковы сетевые протоколы~[1].)
  
  \section{Существование равновесия}

  Введем вспомогательную задачу математического программирования с целью 
замены более трудной игровой задачи маршрутизации на однокритериальную 
оптимизационную задачу. 
  
  Известные стандартные методы сведения обычно приводят к сложным 
многоэкстремальным задачам, в которых целевые функции не являются 
дифференцируемыми, даже если исходная векторная целевая функция была 
дифференцируемой. 
  
  Особенности строения сетевых критериев ка\-чества позволяют предложить 
следующий способ сведения исходной игровой задачи к выпуклой 
оптимизационной задаче, решить которую проще, чем исходную проблему. 
  
  Определим новые функции задержек 
  $$
  t_l=t_l(z_l)\,,\quad l=1,2, \ldots ,n\,,
  $$
решив набор не связанных между собой одномерных задач Коши для 
обыкновенного дифференциального уравнения (ОДУ) первого порядка:
\begin{equation}
z_l \fr{dt_l}{dz_l}+t_l =f_l(z_l)\,,\quad t_l(0)=0\,.
\label{e4-vasil}
\end{equation}
Пусть $Z$~--- многогранник допустимых потоков по линиям передачи. Ввиду 
равенств~(\ref{e2-vasil}) и (\ref{e3-vasil}) $Z$~--- выпуклый ограниченный 
многогранник. Рассмотрим следующую экстремальную задачу:
\begin{equation}
F(z)=\sum\limits_{l=1}^n z_l t_l(z_l)\to \min\limits_{z\in Z}\,.
\label{e5-vasil}
\end{equation}
  
  \noindent
  \textbf{Теорема~1.}\ \textit{Пусть все функции задержек монотонно возрастают, 
дифференцируемы и}
  $$
  (\forall\ l) f_l(z)\to \infty\,,\quad z\to \overline{z}_l\,.
  $$
  \textit{Тогда если $Z\not= \emptyset$, то минимум целевой 
  функции~$(\ref{e5-vasil})$ достигается в единственной точке~$z^*$, которой 
отвечает некоторая равновесная маршрутизация.}
  
  \medskip
  
  \noindent
  Д\,о\,к\,а\,з\,а\,т\,е\,л\,ь\,с\,т\,в\,о\,.\ \ С~помощью критерия Сильвестра проверим 
положительную определенность матрицы Якоби целевой функции $F(z)$. Условия 
теоремы и определение этой функции (см.~(\ref{e4-vasil}) и (\ref{e5-vasil})) дают: 
  \begin{gather*}
  \fr{\partial^2 F}{\partial z_l^2} = \left( z_l t_l^\prime(z_l) +t_l (z_l)\right)^\prime 
=f_l^\prime(z_l)>0\,;\\
  \fr{\partial^2 F}{\partial z_{l_1} \partial z_{l_1}}=0\,,\quad l_1\not= l_2\,.
  \end{gather*}
(штрихом обозначено дифференцирование). Все условия этого критерия 
выполнены, поэтому целевая функция $F(z)$ строго выпукла. В~условиях теоремы 
это позволяет сделать вывод о том, что решение выпуклой экстремальной 
задачи~(\ref{e5-vasil}) существует и единственно. 

  Докажем, что $z^*$~--- точке минимума~(\ref{e5-vasil})~--- отвечает равновесная 
маршрутизация. Для этого применим теорему Ку\-на--Так\-ке\-ра~\cite{11-vasil}, 
которая в данном случае служит критерием оптимальности потока~$z^*$. В~записи 
условий оптимальности учтем, что ограничения~(\ref{e3-vasil}) неактивны. 
(Согласно наложенным предположениям элемент~$z^*$ удовлетворяет строгим 
неравенствам в~(\ref{e3-vasil}).) 
  
  Выразим произвольный допустимый вектор~$z$ в виде 
$z\hm=\mathbf{A}\lambda$, где вектор $\lambda\hm=(\lambda_j^k)$ входит в 
соотношения~(\ref{e2-vasil}). Через~$\mathbf{A}$ обозначена 
  ($n\times J$)-мат\-ри\-ца всех маршрутов, соединяющих рассматриваемую 
тяготеющую пару. Напомним, что $\mathbf{A}$~--- это (0,\,1)-мат\-ри\-ца 
инциденций реб\-ра--марш\-ру\-ты. Маршруты передачи представлены столбцами 
матрицы~$\mathbf{A}$. Указанное представление вектора потоков $z\hm= 
\mathbf{A}\lambda$ всегда возможно: для неприменяемых маршрутов $L_j^k$ 
полагаем $\lambda_j^k\hm=0$. 
  
  Условия оптимальности из теоремы Ку\-на--Так\-ке\-ра в 
  задаче~(\ref{e2-vasil})--(\ref{e5-vasil}) представляют собой систему соотношений, 
распадающуюся на подсистемы, описывающие оптимальные решения для 
отдельных тяготеющих пар. Поэтому рассмотрим произвольную пару~$k$, 
зафиксировав маршрутизацию остальных пар. Для упрощения записи в 
обозначениях опустим верхний индекс~$k$. Итак, выпишем функцию 
Лагранжа~\cite{11-vasil, 12-vasil}:
  $$
  H(\mu, \lambda) =F(A\lambda)+\left\langle \mu,\lambda_0 -\sum\limits_{j=1}^J 
\lambda_j \right\rangle\,,\quad \lambda\geq 0\,.
  $$
  
  Тогда критерий оптимальности маршрутизации, определяемой вектором
  $$
  \lambda=\lambda^*\,;\quad z^*=A\lambda^*\,,
  $$
для $k$-й тяготеющей пары принимает следующий вид:
\begin{equation}
\left.
\begin{array}{c}
\nabla_\lambda H= fA -\mu (1, \ldots , 1) \geq 0\,;\\[9pt]
(\forall \ j) \lambda_j ((fA)_j-\mu)=0\,,
\end{array}
\right\}
\label{e6-vasil}
\end{equation} 
причем $f=(f_1, f_2, \ldots , f_n)$~--- вектор задержек в~(\ref{e6-vasil})~--- 
вычислен в точке~$z^*$. 

  Если $\lambda_j\not= 0$, то из~(\ref{e6-vasil}) следует равенство $(fA)_j\hm=\mu$. 
Если $\lambda_j\hm=0$, то справедливо неравенство $(fA)_j\hm\geq \mu$. По 
определению матрицы маршрутов и в соответствии с определением 
метрики~(\ref{e1-vasil}) имеем
  $$
  (fA)_j =\rho (L_j, z^*)\,.
  $$
  
  Полученные соотношения означают то, что все маршруты, применяемые 
произвольной $k$-й парой, являются кратчайшими, причем их длина равна~$\mu$. 
Следовательно, решение задачи маршрутизации, отвечающее вектору 
потоков~$z^*$, равновесно. В~соответствии с определением равновесия доказана 
оптимальность рассматриваемой маршрутизации для всех тяготеющих пар.
  
  \medskip
  
  \noindent
  \textbf{Пример 1.} При аппроксимации функций задержек~$f_l$ степенными 
функциями задержки~$t_l$, найден\-ные согласно уравнениям~(\ref{e4-vasil}), с 
точностью до постоянного множителя совпадают с~$f_l$. Таким образом, новая 
метрика сети~$\rho_t$ несущественно отличается от~$\rho_f$.
  
  \medskip
  
  \noindent
  \textbf{Пример 2.} В случае дроб\-но-ли\-ней\-ной аппроксимации
  $$
  f(z)=\fr{az}{\overline{z}-z}\,,\enskip 0\leq z< \overline{z}\,,
  $$
функции задержек вспомогательная метрика сети~$\rho_t$ заметно отличается от 
исходной, так как функция $t\hm=t(z)$, удовлетворяющая 
уравнению~(\ref{e4-vasil}), имеет вид: 
$$
t(z) =-a\left( 1+\fr{\overline{z}}{z}\,\ln (\overline{z}-z)\right)\,,\enskip 0\leq 
z<\overline{z}\,.
$$
  
  \section{Эффективность по~Парето равновесной маршрутизации}
  
   На множестве допустимых маршрутизаций введем отношение эквивалентности. 
Маршрутизации $M_1$ и $M_2$ (всей сети или какой-нибудь отдельной тяготеющей 
пары) \textit{эквивалентны}, если они приводят к одному и тому же вектору 
потоков~$z$ на линиях сети. Класс эквивалентности маршрутизации~$M$ будем 
обозначать $R_z(M)$. 
  
  Для упрощения записи будем опускать индекс~$z$, а множество применяемых 
  $k$-й парой маршрутов $\{L_j^k\}$ при маршрутизации~$M$ обозначать~$M^k$.
  
  Из проведенного доказательства теоремы~1 вытекает
  
  \smallskip
  
  \noindent
  \textbf{Следствие~1.} Равновесной является всякая маршрутизация, которая 
эквивалентна равновесной маршрутизации.
  
  Будем считать, что в сети с потоками~$z$ может реализоваться произвольная 
эквивалентная маршрутизация. (Это предположение имеет место в пакетных сетях, 
в которых управление передачей осущест\-вля\-ет\-ся с помощью маршрутных 
  таб\-лиц~\cite{1-vasil}.) 
  
  Тогда времена доставки отдельных пакетов для $k$-й тяготеющей пары 
абонентов вычисляются по следующей формуле:
  \begin{multline}
  T^k(z;M^k) =\max\limits_{M^\prime\in R_z(M^k)} \max\limits_{L_j^k\in M^\prime} 
\rho(L_j^{\prime\,k}, z)\,, \\ k=1,2,\ldots, K\,.
  \label{e7-vasil}
  \end{multline}
  Подстановка в~(\ref{e7-vasil}) равновесной маршрутизации дает значения 
критериев~(\ref{e7-vasil}), совпадающие с длинами применяемых (кратчайших) 
маршрутов (следствие~1).
  
  \medskip
  
  \noindent
  \textbf{Теорема~2.}\ \textit{Пусть все пары узлов графа сети являются 
тяготеющими и для любых попарно смежных ребер графа сети~$k, l, m$ выполнены 
неравенства треугольника
  $$
  f_k(0)\leq f_l(0)+f_m(0)\,.
  $$
  Тогда равновесная маршрутизация эффективна по Парето.} 
  
  \medskip
  
  \noindent
  Д\,о\,к\,а\,з\,а\,т\,е\,л\,ь\,с\,т\,в\,о\,.\ \ Рассуждая от противного, найдем такую 
допустимую маршрутизацию
  $$
  M=\left\{ \left\{ L_j^k\right\}, \left\{ \lambda_j^k\right\},\ k=1,2,\ldots , K\right\}\,,
  $$
  для которой 
  \begin{equation}
  \left(\forall k \right) T^k(z)\leq T^k(z^*)\,.
  \label{e8-vasil}
  \end{equation}
    В соотношениях~(\ref{e8-vasil}) хотя бы одно неравенство является 
строгим~\cite{13-vasil}, поэтому равновесные потоки~$z^*$ таковы, что 
$z\not=z^*$. 
  
  Из условия теоремы и неравенства треугольника следует, что в ситуации 
равновесия
  $$
  z_l^*> 0\,,\enskip l=1,2,\ldots ,n\,.
  $$
  
  Докажем справедливость неравенства
  \begin{equation}
  z_l\leq z_l^*\,,\enskip l=1,2,\ldots ,n\,.
  \label{e9-vasil}
  \end{equation}
  
  Если в маршрутизации $M$ линия связи~$l$ не применяется для передачи 
пакетов, то, очевидно, соответствующее неравенство в~(\ref{e9-vasil}) выполнено. 
Пусть теперь $z_l\hm> 0$, $l\hm=(a,b)$. Можно считать, что у тяготеющей пары 
$(a,b)$ с номером~$k$ имеется маршрут соединения
  $$
  L\equiv \left\{ (a,b)\right\} \in M^k\,.
  $$
  
  В самом деле, если это не так, то, покажем, существует маршрутизация 
$M^{\prime\,k} \hm\in R(M^k)$, обладающая этим свойством. (Тогда вместо 
маршрутизации~$M$ достаточно будет рассмотреть~$M^\prime$.)
  
  Так как $z_l>0$, то найдется такая пара~$k^\prime$, что $l\hm= (a,b)\hm\in 
L_{j^\prime}^{k^\prime}\hm\in M_{j^\prime}^{k^\prime}$. Выберем величину
  $$
  0<\Delta <\min \left\{ \lambda_j^k,\lambda_j^{k^\prime}\right\}\,,
  $$
в которой поток $\lambda_j^k$ проходит по такому маршруту~$L_j^k$, что $l\notin 
L_j^k\hm\in M^k$. Тогда часть маршрутного потока~$\lambda_j^k$ 
величиной~$\Delta$ перебросим с маршрута~$L_j^k$ на~$L$~--- новый для $k$-й 
пары маршрут соединения. Такую же величину~$\Delta$, являющуюся частью 
потока $\lambda_{j^\prime}^{k^\prime}$, направим по маршруту
$$
L^\prime =\left( L_{j^\prime}^{k^\prime} \backslash \{l\}\right)\cup L_j^k\,,
$$
новому для пары~$k^\prime$. Все остальные <<элементы>> маршрутизации~$M$ 
оставим без изменения. В~результате получена искомая 
маршрутизация~$M^\prime$, для которой $M^\prime\hm\in R(M)$.

  Согласно сделанному допущению~(\ref{e8-vasil}) относительно маршрутизаций 
$M$ и $M^*$ и определению критериев~(\ref{e7-vasil}) имеем:
\begin{multline*}
  \left(\forall\ l=1,2,\ldots , n\right) f_l(z_l) ={}\\
  {}=\rho(L,z)\leq T^k(z)\leq 
T^k(z^*)=f_l(z_l^*)\,.
\end{multline*}
  Так как функции задержек монотонно возрастают, отсюда получаем 
неравенства~(\ref{e9-vasil}), $z\not=z^*$. Целевая функция задачи~(\ref{e5-vasil}) 
монотонно воз\-рас\-та\-ет по каждой переменной~$z_l$. Тогда из~(\ref{e9-vasil}) 
следует неравенство $F(z)\hm<F(z^*)$, противоречащее тому, что $z^*$~--- 
минимум целевой функции~(\ref{e5-vasil}). Теорема~2 доказана.
  
  %\smallskip
  
  \section{Вычислительная устойчивость равновесного решения задачи 
маршрутизации} 
  
  Исходные данные модели обычно обладают некоторой неопределенностью. 
Возникает необходимость исследовать устойчивость искомого решения. 
В~рассматриваемой модели будем варьировать все параметры с помощью 
изменения $c\hm= (\lambda_0,\overline{z})$~--- вектора, составленного из величин 
входных потоков и пропускных способностей линий~ТС. 
  
  Задержки и целевая функция~(\ref{e5-vasil}) есть функции переменных $z$ и $c$, а 
многогранник потоков~--- значение многозначного отображения $c\hm\to Z(c)$. 
Пред\-полагается, что параметр~$c$ изменяется в пределах\linebreak множества~$C$ такого, 
что при справедливости включения $c\hm\in C$ выполняются все условия 
теоремы~1 и, кроме того, все функции задержек непрерывны по совокупности 
переменных. Тогда справедлива
  
  \smallskip
  
  \noindent
  \textbf{Теорема~3.} \textit{Метрика сети непрерывно зависит от параметров 
модели.}
  
  \smallskip
  
  \noindent
  Д\,о\,к\,а\,з\,а\,т\,е\,л\,ь\,с\,т\,в\,о\,.\ \ Из теоремы о непрерывной зависимости 
ОДУ~(\ref{e4-vasil}) от параметра следует, что целевая функция
  $
  F(z,c)$, $z \hm\in Z(c)$, $c\hm\in C$,  экстремальной задачи~(\ref{e5-vasil}) 
непрерывна. Анализ линейных соотношений~(\ref{e2-vasil}) и~(\ref{e3-vasil}) 
показывает, что отображение $c\hm\to Z(c)$ непрерывно по 
  Хаусдорфу~\cite{11-vasil}. По теореме~1 отсюда можно заключить, что 
оптимальное решение задачи~(\ref{e5-vasil}) $z^*\hm=z^*(c)$ непрерывно зависит 
от параметра~$c$. Таким образом, метрика сети также непрерывна как сумма 
непрерывных функций (см.~(\ref{e1-vasil})).
  
  \smallskip
  
  Непосредственным следствием теорем~1 и~2 является вывод о том, что решение 
задачи маршрутизации устойчиво к изменению параметров мо\-дели.
   
  \section{Алгоритм поиска равновесной маршрутизации}
  
  Для поиска оптимальных маршрутов передачи продуктов каждой тяготеющей 
пары будем применять следующую схему алгоритма. 
  
  Произвольно выберем начальную маршрутизацию (шаг $t\hm=0$).
  
  Пусть на шаге $t\hm=0,1,\ldots$ уже построена маршрутизация, обозначаемая 
$M^t$, $M^t\hm= \left( \left\{ L_j^k\right\},\left\{ \lambda_j^k\right\} \right)$.
  
  Ей отвечает вектор потоков~$z^t$. В~сети с фиксированной метрикой 
$\rho(L,z^t)$ найдем кратчайший маршрут~$L$. (Достаточно воспользоваться 
алгоритмом Дейкстры~\cite{14-vasil}.) Пусть также в текущей 
маршрутизации~$M^t$ существует маршрут~$L^t$, имеющий б$\acute{\mbox{о}}$льшую длину по 
сравнению с~$L$. (Иначе решение задачи уже найдено.) Тогда часть 
потока~$\lambda^t$, передаваемого по маршруту~$L^t$, перебросим на 
маршрут~$L$ с целью уменьшения разности длин этих маршрутов. (Это возможно 
ввиду монотонности функций задержек, см.~(\ref{e1-vasil}).) При этом либо 
уравняются длины маршрутов, либо маршрут~$L$ останется по-прежнему короче, а 
маршрут~$L^t$ перестанет использоваться (в случае 
$\lambda\vert_L\hm=\lambda^t$).
  
  Определим очередную маршрутизацию ТС $M^{t+1}$ как результат добавления к 
маршрутизации~$M^t$ пары $L,\lambda\vert_L$ и, возможно, исключения 
маршрута~$L^t$ в случае, когда он перестает применяться. 
  
  Указанные в алгоритме действия~--- применение \textit{принципа уравнивания} 
Ю.\,Б.~Гермейера как метода решения минимаксных задач~\cite{15-vasil}. 
  
  \smallskip
  
  \noindent
  \textbf{Теорема~4.} \textit{Последовательность маршрутизаций $\{M^t,\ 
t=0,1,\ldots\}$, построенная по принципу уравнивания, сходится к равновесному 
решению задачи.}
  
  \smallskip
  
  \noindent
  Д\,о\,к\,а\,з\,а\,т\,е\,л\,ь\,с\,т\,в\,о\,.\ \ Вычислим производную функции 
$F(A\lambda)$ в текущей точке $z^t\hm= A\lambda^t$ по на\-прав\-ле\-нию 
вектора~$\Delta$, все координаты которого равны нулю, за исключением тех, 
которые отвечают маршрутным потокам вдоль~$L$ и~$L^t$, равных 
соответственно~1 и~$-1$:
  $$
  \fr{dF(A\lambda)}{d\Delta} =\left\langle fA,\Delta\right\rangle=\rho(L,z^t)-
\rho(L^t,z^t)<0\,.
  $$
  
  Шаг градиентного метода в направлении~$\Delta$ с целью выравнивания длин 
этих маршрутов приводит к уменьшению значения целевой функции в 
задаче~(\ref{e5-vasil}):
  $$
  F(z^{t+1})< F(z^t)\,,\enskip t=0,1,\ldots
  $$
  
  Монотонная числовая последовательность $\{ F(z^t)\}$ сходится к~$F(z^*)$, где 
$z^*$~--- решение задачи~(\ref{e5-vasil}). Согласно теореме~1 $z^t\hm\to z^*$, 
$t\hm\to\infty$. По следствию~1 в пределе получаем равновесную маршрутизацию.
  
{\small\frenchspacing
{%\baselineskip=10.8pt
\addcontentsline{toc}{section}{References}
\begin{thebibliography}{99}
\bibitem{1-vasil}
\Au{Бертсекас Д., Галлагер Р.} Сети передачи данных~/ 
Пер с англ.~--- М.: Наука, 1989.
(\Au{Bertsecas~D.\,P., Gallager~R}. 
{Data networks}.~--- Englewood Cliffs: Prentice Hall, 1987. 544~p.)
\bibitem{2-vasil}
\Au{Васильев Н.\,С., Федоров В.\,В.} О~равновесной маршрутизации в сетях 
передачи данных~// Вестн. Моск. ун-та. Сер.~15. Вычисл. матем. и киберн., 1996. №\,4. 
С.~39--47.
\bibitem{3-vasil}
\Au{Korilis Y.\,A., Lazar A.\,A., Orda~A.} Capacity allocation under noncooperative 
routing~// IEEE Trans. Automat. Contr., 1997. Vol.~42. No.\,3. P.~309--325.
\bibitem{4-vasil}
\Au{Vasilyev N.\,S.} Nash equilibrium routing in ring networks~// Int. J.~Math. Game 
Theory  Algebra, 1998. Vol.~7. No.\,4. P.~221--234.
\bibitem{5-vasil}
\Au{Васильев Н.\,С.} О~свойствах решений задачи маршрутизации сети с 
виртуальными каналами~// Ж. вычисл. матем. и матем. физ., 1997. Т.~37. №\,7. 
С.~785--793.
\bibitem{6-vasil}
\Au{Васильев Н.\,С.} О~свойствах решений задачи динамической маршрутизации 
сети~// Ж. вычисл. матем. и матем. физ., 1998. Т.~38. №\,1. С.~42--52.

\bibitem{8-vasil}
\Au{Соколов И.\,А., Шоргин С.\,Я.} Модель и математические методы расчета 
характеристик сети, использующей технологии X.25 и Frame relay~// Системы и 
средства информатики. Спец. вып. Математические методы информатики.~--- М.: 
Наука, 2001. С.~43--66.

\bibitem{7-vasil}
\Au{Васильев Н.\,С., Федоров В.\,В.} О~построении алгоритмов маршрутизации 
пакетных сетей на основе векторных критериев~// Известия РАН. Теория и системы 
управления, 2005. №\,3. С.~36--47.

\bibitem{10-vasil} %9
\Au{Васильев Н.\,С.} Задача о кратчайших маршрутах в сетях с переменной 
метрикой~// Вестник МГТУ им.\ Н.\,Э.~Баумана. Сер. Естеств. науки, 2008. №\,1. 
С.~70--75.

\bibitem{9-vasil} %10
\Au{Коновалов М.\,Г.} Оптимизация работы вычислительного комплекса с помощью 
имитационной модели и адаптивных алгоритмов~// Информатика и её применения, 
2012. Т.~6. Вып.~1. С.~37--48.

\bibitem{11-vasil}
\Au{Васильев Ф.\,П.} Численные методы решения экстремальных задач.~--- М.: 
Наука, 1980.
\bibitem{12-vasil}
\Au{Иоффе А.\,Д., Тихомиров В.\,М.} Теория экстремальных задач.~--- М.: Наука, 
1974. 481~c.
\bibitem{13-vasil}
\Au{Подиновкий В.\,В., Ногин В.\,Д.} Па\-ре\-то-оп\-ти\-маль\-ные решения 
многокритериальных задач.~--- М.: Наука, 1982. 256~с.
\bibitem{14-vasil}
\Au{Кристофидес Н.} Теория графов. Алгоритмический подход~/
Пер с англ.~--- М.: Мир, 1978.
(\Au{Cristofides~N.} {Graph theory: An algorithmic approach}.~---
 London: Academic, 1975. 430~p.)
\bibitem{15-vasil}
\Au{Федоров В.\,В.} Численные методы максимина.~--- М.: Наука, 1979.  280~с.
\end{thebibliography}
} }

\end{multicols}

\hfill{\small\textit{Поступила в редакцию 25.05.13}}


\vspace*{12pt}

\hrule

\vspace*{2pt}

\hrule
    
\def\tit{EQUILIBRIUM PRINCIPLE APPLICATION TO~ROUTING CONTROL IN~PACKET DATA TRANSMISSION NETWORKS}

\def\titkol{Equilibrium principle application to routing control in packet data transmission networks}

\def\aut{N.\,S.~Vasilyev}
\def\autkol{N.\,S.~Vasilyev}


\titel{\tit}{\aut}{\autkol}{\titkol}

\vspace*{-12pt}

\noindent
Bauman Moscow State Technical University, 5, 2nd Baumanskaya Str., Moscow 
105005, Russian Federation

 
\def\leftfootline{\small{\textbf{\thepage}
\hfill INFORMATIKA I EE PRIMENENIYA~--- INFORMATICS AND APPLICATIONS\ \ \ 2014\ \ \ volume~8\ \ \ issue\ 1}
}%
 \def\rightfootline{\small{INFORMATIKA I EE PRIMENENIYA~--- INFORMATICS AND APPLICATIONS\ \ \ 2014\ \ \ volume~8\ \ \ issue\ 1
\hfill \textbf{\thepage}}}   

\vspace*{3pt}
  
\Abste{Annual exponential growth of data flows in large scale networks impels to 
search not only network hardware improvements but also more perfect routing control 
algorithms. In networks, it is impossible to use centralized algorithms of 
 routing control. Parallel algorithms choice must be based on the principles 
of functional effectiveness and stability (equilibrium). 
     In large-scale networks, there is a huge number of users' pairs trying to achieve 
     the maximally possible rate of data transmission by routing control. 
     Thus, control must be based on multicriteria optimization ideas and methods. 
    The Nash equilibrium (game formulation of the routing problem) formally presents optimality 
    of transmission control in distributed systems. In the present paper,
     the equilibrium 
    routing is proved to exist under general conditions. The solution is additionally shown to be effective in 
    Pareto sense and computationally stable. An effective (quick and parallel) game theory 
    algorithm is suggested and its convergence is proved.}
    
    \KWE{packet network; data flows; network metric; routing; vector criteria; 
    multicriteria optimization; game problem; Nash equilibrium; Pareto effectiveness}
    
    \DOI{10.14357/19922264140104}

%\Ack
%\noindent
%Работа выполнена при финансовой поддержке РФФИ (проект 11-01-00515а).

  \begin{multicols}{2}
  
  \renewcommand{\bibname}{\protect\rmfamily References}
%\renewcommand{\bibname}{\large\protect\rm References}

{\small\frenchspacing
{%\baselineskip=10.8pt
\addcontentsline{toc}{section}{References}
\begin{thebibliography}{99}
  
  \bibitem{1-vasil1}
  \Aue{Bertsecas,~D.\,P., and R.~Gallager}. 
  1987. \textit{Data networks}. Englewood Cliffs: Prentice Hall. 544~p.
\bibitem{2-vasil-1}
\Aue{Vasilyev, N.\,S., and V.\,V.~Fedorov}. 
1996. O~ravnovesnoy marshrutizatsii v setyakh peredachi dannykh 
[Equilibrium routing in data-transmission networks]. 
\textit{Vestn. Mosk. Univ. Ser.~15: Vychisl. Mat. Kibern.}
[\textit{Bulletin of Moscow State University. Ser.~15: Comput. Math., Cybern.}] 4:47--52.
\bibitem{3-vasil-1}
\Aue{Korilis, Y.\,A., A.\,A.~Lazar, and A.~Orda}. 
1997. Capacity allocation under noncooperative routing. 
\textit{IEEE Trans. Automat. Contr.} 42(3):309--325.
\bibitem{4-vasil-1}
\Aue{Vasilyev, N.\,S.} 1998. 
Nash equilibrious routing in ring networks. \textit{Int. J.~Math. Game Theory Algebra}
7(4):221--234.
{\looseness=1

}
\bibitem{5-vasil-1}
\Aue{Vasilyev, N.\,S.} 1997. 
O~svoystvakh resheniy zadachi marshrutizatsii seti s virtual'nymi kanalami 
[Properties of the solutions to the problem of routing in network with virtual channels]. 
\textit{Zh. Vychisl. Mat. Mat. Fiz.} [\textit{Computational Mathematics and Mathematical 
Physics}] 37(7):785--793.
\bibitem{6-vasil-1}
\Aue{Vasilyev, N.\,S.} 1998. O~svoystvakh resheniy zadachi dinamicheskoy 
marshrutizatsii  seti [Properties of the solutions to the problem of dynamic routing 
in networks]. \textit{Zh. Vychisl. Mat. Mat. Fiz}. 
[\textit{Computational Mathematics and Mathematical Physics}] 38(1):42--52.

\bibitem{8-vasil-1}
\Aue{Sokolov, I.\,A., and S.\,Ya.~Shorgin}. 2001. 
Model' i matematicheskie metody rascheta kharakteristik seti, 
ispol'zuyushchey tekhnologii X.25 i Frame relay 
[Models and mathematical methods of characteristics calculation for X.25 and 
Frame relay network]. \textit{Sistemy i Sredstva Informatiki. Spec. vyp. 
``Matematicheskie metody informatiki''} 
[\textit{Systems and Means of Informatics. Spec. ed. ``Math. Meth. of Informatics''}]. 
Moscow: Nauka, Fizmatlit. 43--66.

\bibitem{7-vasil-1}
\Aue{Vasilyev, N.\,S., and V.\,V.~Fedorov}. 2005. O~postroenii 
algoritmov marshrutizatsii paketnykh setey na osnove vektornykh kriteriev 
[On routing algorithms in packet networks on the base of vector criterias]. 
\textit{Izvestija RAN. Teorija i sistemy upravlenija} 
[\textit{Bulletin of RAS. Theory and Control Systems}] 3:36--47.

\bibitem{10-vasil-1}
\Aue{Vasilyev, N.\,S.} 2008. Zadacha o kratchayshikh marshrutakh v setyakh s 
peremennoy metrikoy [The shortest paths problem in networks with changeable metric]. 
\textit{Vestnik MGTU im. N.\,E.~Baumana, Ser. Estestv. Nauki} 
[\textit{Bulletin of Bauman MSTU. Ser. Natural Sciences}] 1:70--75.


\bibitem{9-vasil-1}
\Aue{Konovalov, M.\,G.} 2012. Optimizatsiya raboty vychislitel'nogo kompleksa s 
pomoshch'yu imitatsionnoy modeli i adaptivnykh algoritmov 
[Optimization of computational complex work on the base of imitational models and 
adaptive algorithms]. 
\textit{Informatika i Ee Primeneniya}~--- \textit{Inform. Appl.} 6(1):37--48.
\bibitem{11-vasil-1}
\Aue{Vasilyev, F.\,P.} 1980. Chislennye metody resheniya ekstremal'nykh zadach 
[Numerical methods for extremum problems]. Moscow: Nauka. 520~p.
\bibitem{12-vasil-1}
\Aue{Ioffe, A.\,D., and V.\,M.~Tihomirov}. 1974. 
\textit{Teoriya ekstremal'nykh zadach} [\textit{Theory of extremum problems}]. 
Moscow: Nauka. 481~p.
\bibitem{13-vasil-1}
\Aue{Podinovskij, V.\,V., and V.\,D.~Nogin}. 1982. 
\textit{Pareto-optimal'nye resheniya mnogokriterial'nykh zadach} 
[\textit{Pareto optimal solutions in multicriteria problems}].  Moscow: Nauka. 256~p.
\bibitem{14-vasil-1}
\Aue{Cristofides, N.} 1975. \textit{Graph theory: An algorithmic approach}.
 London: Academic. 430~p.
\bibitem{15-vasil-1}
\Aue{Fedorov, V.\,V.}  1979. 
\textit{Chislennye metody maksimina} [\textit{Numerical methods of maximin}]. 
Moscow: Nauka. 280~p.



\end{thebibliography}
} }


\end{multicols}

\vspace*{-6pt}

\hfill{\small\textit{Received May 25, 2013}}

\vspace*{-18pt}

\Contrl

\noindent
\textbf{Vasilyev Nikolai S.} (b.\ 1952)~--- Doctor of Science in physics and mathematics,
professor, Bauman Moscow State Technical University,
5, 2nd Baumanskaya Str., Moscow 105005, Russian Federation; nik8519@yandex.ru



 \label{end\stat}
 
 \renewcommand{\bibname}{\protect\rm Литература}  
  %1
\def\l{\lambda}
\def\tl{\tilde\lambda}
\def\tB{\widetilde B}
\def\tb{\tilde b}


\def\stat{razum}

\def\tit{СТАЦИОНАРНЫЕ ВЕРОЯТНОСТИ СОСТОЯНИЙ В~СИСТЕМЕ ОБСЛУЖИВАНИЯ С~ИНВЕРСИОННЫМ ПОРЯДКОМ
ОБСЛУЖИВАНИЯ И~ОБОБЩЕННЫМ ВЕРОЯТНОСТНЫМ
ПРИОРИТЕТОМ$^*$}

\def\titkol{Стационарные вероятности состояний в~системе обслуживания} % с~инверсионным порядком обслуживания и~обобщенным вероятностным приоритетом}

\def\aut{Л.\,А.~Мейханаджян$^1$,  Т.\,А.~Милованова$^2$,
А.\ В.\ Печинкин$^3$,  Р.\,В.~Разумчик$^4$}

\def\autkol{Л.\,А.~Мейханаджян,  Т.\,А.~Милованова,
А.\ В.\ Печинкин,  Р.\,В.~Разумчик}

\titel{\tit}{\aut}{\autkol}{\titkol}

{\renewcommand{\thefootnote}{\fnsymbol{footnote}} \footnotetext[1]
{Работа выполнена при частичной поддержке РФФИ (проект 13-07-00223).}}

\renewcommand{\thefootnote}{\arabic{footnote}}
\footnotetext[1]{Российский университет дружбы народов, lameykhanadzhyan@gmail.com}
\footnotetext[2]{Российский университет дружбы народов, tmilovanova77@mail.ru}
\footnotetext[3]{Институт проблем информатики Российской академии наук,
apechinkin@ipiran.ru}
\footnotetext[4]{Институт проблем информатики Российской академии наук;
Российский университет дружбы народов, rrazumchik@ieee.org}

\Abst{Рассматривается однолинейная система массового
обслуживания (СМО), в которую поступает поток заявок,
называемый здесь потоком пуассоновского типа.
Отличие этого потока от пуассоновского
заключается в том, что интенсивность поступления
заявок равна~$\lambda$, если на приборе имеется
заявка, и $\tl$, если сис\-те\-ма пуста.
Если заявка поступает в сис\-те\-му, в которой
на приборе имеется заявка, то исходное
распределение времени обслуживания поступающей
заявки является произвольным с функцией
распределения (ФР) $B(x)$, в противном случае~---
произвольным с ФР $\tB(x)$.
В~сис\-те\-ме реализован инверсионный порядок
обслуживания с обобщенным вероятностным
приоритетом, заключающийся в сле\-ду\-ющем.
Предполагается, что в любой момент времени
известна остаточная длина каждой заявки в сис\-те\-ме.
В~момент поступления в сис\-те\-му новой заявки ее
исходная длина сравнивается с остаточной
длиной заявки на приборе и в зависимости
от результатов сравнения одна из них становится
на прибор, а другая~--- на первое место в
очередь, причем каждая заявка приобретает новую
(случайную) длину и даже может покинуть систему.
В~статье предложены математические соотношения
для вычисления основных показателей
функционирования системы, связанных со
стационарным распределением числа заявок в ней.}


\KW{система массового обслуживания; специальные
дисциплины; инверсионный порядок
обслуживания; вероятностный приоритет}

\DOI{10.14357/19922264140304}


\vskip 12pt plus 9pt minus 6pt

\thispagestyle{headings}

\begin{multicols}{2}

\label{st\stat}


\section{Введение}

В текущих условиях активного развития
инфотелекоммуникационной отрасли необходимы
аналитические средства создания и анализа
объектов и их совокупностей, используемых для
передачи, хранения и обработки информации.
Традиционно применяемые для описания
функционирования инфотелекоммуникационных
систем (ИТС) СМО
и их комбинации позволяют получать точные
и/или приближенные оценки для ключевых
показателей производительности ИТС, хотя анализ
зачастую связан с определенными
вычислительными трудностями, что может приводить
к бессодержательным результатам.
Другой подход к анализу показателей производительности
ИТС заключается в моделировании реальной системы
с помощью простой модели СМО и применении в ней
сложных дисциплин обслуживания.
Например, предоставление приоритета обслуживания
заявкам с наименьшей остаточной длиной (дисциплина
{SRPT}~--- shortest remaining processing time)~\cite{shrage} дает оптимальную стратегию
с точки зрения минимизации числа заявок в системе.
Но при расчетах показателей функционирования СМО
с дисциплиной {SRPT} необходимо знание времени
обслуживания (длины) каждой поступающей в систему
заявки и использование информации о всех остаточных
длинах.
Подобные ограничения свойственны многим специальным
дисциплинам.

На практике потоки, циркулирующие в ИТС, являются
неоднородными, и требования к обслуживанию у
сообщений различных классов могут быть различными.
Кроме того, могут происходить сбои систем ввиду
флуктуации нагрузки
или появления дестабилизирующих факторов.
Вместо введения в рассмотрение большого числа потоков
можно воспользоваться следующим приемом:
предположить, что длины заявок, а также эффекты
(события), которые связаны с поступлением заявок
той или иной длины, известны лишь с
некоторой вероятностью.
Поэтому представляет интерес рассмотренный в~\cite{aaa1} инверсионный порядок обслуживания с
вероятностным приоритетом (дис\-цип\-ли\-на {LCFS PP}),
при котором при поступлении в систему новой заявки
место на приборе и первое место в очереди
разыг\-ры\-ва\-ют между собой вновь поступившая заявка
и заявка, находившаяся ранее на приборе, причем
с вероятностью, являющейся произвольной функцией
от длины поступившей заявки и остаточной длины
заявки на приборе.


По данной проблематике опубликовано и продолжает
появляться значительное число работ как
теоретического, так и прикладного характера
(см., например,~[3--8]).
%\cite{aaa2,aaa3,aaa4,av1,av2,av3}).
В~частности, в~\cite{aaa2} на примере одноканальной
СМО с пуассоновским входящим потоком была рассмотрена
дисциплина, предо\-став\-ля\-ющая преимущество заявкам с
меньшими остаточными длинами.
Эта дисциплина является в определенном смысле
промежуточной между обычной дисциплиной без
прерывания обслуживания и дисциплиной {SRPT}.
Считаются известными
остаточные времена обслуживания (длины) всех
находящихся в системе заявок.
При поступлении новой заявки ее длина сравнивается
с длиной заявки на приборе, и та из них, длина
которой минимальна, становится на
прибор, оставляя за второй первое место в очереди.


В настоящей статье на основе развития идей
вышеприведенных работ вводится новая дисциплина
(инверсионный порядок обслуживания с обобщенным
вероятностным приоритетом, или LCFS GPP),
согласно которой при поступлении новой заявки система
принимает решение не только о том, что с ней делать
(поставить в очередь, поменять местами с заявкой
на приборе, удалить из системы и~т.\,д.), но и
как долго обслуживать оставшиеся в системе заявки.
Решение принимается путем задания вероятностей
соответствующих событий.
Исследуется СМО с такой дисциплиной, входящим потоком
пуассоновского типа и произвольным распределением
времени обслуживания заявки.
На основе метода исключения состояний получены
в терминах вычислительных алгоритмов и
производящих функций выражения для стационарного
распределения числа заявок в системе.
Как будет показано, вычисление стационарного
распределения (а также его моментов) связано с
решением уравнения Фредгольма 2-го рода.

В следующем разделе дается подробное опи\-сание
системы,
затем выводятся формулы для нахожде\-ния стационарных
вероятностных характеристик, и в заключении
приводятся некоторые примеры численных расчетов,
выполненных по полученным соотношениям.

\section{Описание системы}

Рассмотрим СМО с входящим потоком заявок,
который для простоты будем называть здесь
потоком пуассоновского типа.
Отличие этого потока от пуассоновского
заключается в следующем:
интенсивность поступления заявки равна~$\lambda$,
если на приборе имеется заявка, и~$\tl$, если
система пуста.
Если в момент поступления заявки в систему
на приборе имеется заявка, то исходное
распределение времени обслуживания поступающей
заявки является произвольным с ФР $B(x)$.
Если же заявка поступает в систему в тот
момент, когда система пуста, то исходное
распределение времени обслуживания поступающей
заявки является произвольным с ФР $\tB(x)$.

Далее для простоты изложения будем считать,
что ФР $B(x)$ и $\tB(x)$ имеют непрерывные
ограниченные плотности
распределения~$b(x)\hm=B'(x)$ и $\tb(x)\hm=\tB'(x)$.

Обобщенный инверсионный порядок обслуживания
с вероятностным приоритетом ({LCFS GPP})
заключается в следующем.
Предполагается, что в любой момент времени
известна остаточная длина (далее будем говорить
просто длина) каждой заявки в системе.
В момент поступления в систему новой заявки ее
исходная длина $u$ сравнивается с (остаточной)
длиной $v$ заявки на приборе.
С вероятностью~$D(x,y|u,v)$,
зависящей только от $u$ и~$v$, обслуживавшаяся
ранее заявка продолжает обслуживаться, причем
ее длина становится меньше $y$, а вновь
поступившая становится на первое место в очереди
и ее длина становится меньше~$x$.
Кроме того, с вероятностью~$D^*(x,y|u,v)$,
зависящей только от $u$ и~$v$, вновь поступившая
заявка занимает прибор, вытесняя обслуживавшуюся
ранее на первое место в очереди, причем длина
заявки, бывшей ранее на приборе, становится
меньше~$y$, а вновь поступившей~--- меньше~$x$.

Если на приборе находится заявка остаточной
длины~$v$ и в систему поступает заявка
длины~$u$, то с вероятностью $D_0(x|u,v)$
заявка, находящаяся на приборе, покидает
систему, а поступившая заявка становится на
прибор, причем ее длина становится меньше~$x$.
Кроме того, с вероятностью
$D_0^*(y|u,v)$ поступившая заявка сразу же
покидает систему, а~заявка, находящаяся на
приборе, продолжает обслуживаться, причем ее
длина становится меньше~$y$.
Введем также обозначение:
%$$
\begin{equation*}
%\label{(2.1)}
D(x|u,v) = D_0(x|u,v) + D_0^*(x|u,v)\,.
\end{equation*}
%$$
Здесь $D(x|u,v)$~--- вероятность того, что одна
из двух заявок покинет систему, а вторая встанет
на прибор и примет длину меньше~$x$.

Наконец, предполагается, что с
вероят\-ностью~$d_0(u,v)$ обе заявки покидают
систему, а на прибор становится первая заявка
из очереди.

Будем считать для удобства изложения, что все
ФР $D(x,y|u,v)$, $D^*(x,y|u,v)$, $D_0(x|u,v)$,
$D_0^*(y|u,v)$, $D(y|u,v)$ и $D_0(u,v)$
имеют непрерывные ограниченные плотности
$d(x,y|u,v)\hm=\partial^2 D(x,y|u,v)/
(\partial x\, \partial y)$,\ \
$d^*(x,y|u,v)\hm=\partial^2 D^*(x,y|u,v)/
(\partial x\, \partial y)$,\ \
$d_0(x|u,v)\hm= \partial D_0(x|u,v)/\partial x$,\ \
$d_0^*(y|u,v)\hm=\partial D_0^*(y|u,v)/\partial y$
и
$d(x|u,v)\hm=\partial D(x|u,v)/\partial x$.


Естественно, для любых $u$ и~$v$ выполнено условие:
\begin{multline}
\label{(2.1)}
\int\limits_0^\infty
\int\limits_0^\infty
[d(x,y|u,v) + d^*(x,y|u,v)]
\,dx\,dy
+ {}\\
{}+\int\limits_0^\infty d(x|u,v) \,dx + d_0(u,v) = {}\\
{}= D(\infty,\infty|u,v)+D^*(\infty,\infty|u,v)
+{}\\
 {}+D(\infty|u,v) + d_0(u,v) = 1\,.
\end{multline}



Если длина заявки на приборе становится
равной нулю, то она мгновенно покидает
систему и на прибор переходит первая
заявка из очереди.
Остальная очередь сдвигается на единицу.


Далее будем предполагать, что система
функционирует в стационарном режиме.
К~сожалению, для рассматриваемой СМО не
удается выписать общее необходимое и
достаточное условие существования стационарного
режима функционирования.
Это условие зависит от конкретных параметров
сис\-те\-мы и в каждом отдельном случае нуждается
в специальном исследовании.
Здесь приведем прос\-тое достаточ\-ное условие,
вытекающее из сравнения суммарной имеющейся
работы в описанной СМО и~суммарной работы в
стандартной СМО $M/G/1/\infty$.

Условие состоит из выполнения следующих
соотношений:
\begin{enumerate}[1.]
\item  $\tb =
\int\limits_0^\infty x\,\tb(x)\,dx < \infty$.

\item $\rho = \lambda b =
\lambda \int\limits_0^\infty x\, b(x)\,dx < 1$.

\item $d(x,y|u,v) = 0$ при всех $u,v$ и
$y\hm>v$ или $x\hm>u$.

\item $d^*(x,y|u,v) = 0$ при всех $u,v$ и
$y\hm>v$ или $x\hm>u$.

\item $d(x|u,v) = 0$ при всех $u,v$ и
$x\hm>u$.

\item $d^*(y|u,v) = 0$ при всех $u,v$ и
$y\hm>v$.
\end{enumerate}

Соотношения~{3}--{6} соответствуют
тому факту, что после поступления новой заявки
измененные длины заявок не превышают те длины,
которые были до поступления.
Отметим, что здесь параметр $\rho \hm= \lambda b$ не
является загрузкой в традиционном смысле и может
существенно от нее отличаться.

\section{Система уравнений равновесия}

Обозначим через $\nu(t)$ число заявок в системе
в момент $t$, а через $\vec\xi(t)\hm=
(\xi_{1}(t),\ldots,\xi_{\nu(t)}(t))$~---
вектор, координатой $\xi_{1}(t)$ которой
является (остаточное) время обслуживания
заявки, находящейся в этот момент на приборе,
$\xi_{2}(t)$~--- первой заявки в
очереди$,\ldots,$ $\xi_{\nu(t)-1}(t)$~---
последней, \mbox{$(\nu(t)-1)$-й} заявки в очереди.
При $\nu(t)\hm=0$ вектор $\vec\xi(t)$
не определяется.
Тогда $\eta(t)\hm=(\nu(t),\vec\xi(t))$ представляет
собой марковский процесс, описывающий
поведение числа заявок в рассматриваемой сис\-теме.

Положим
\begin{align*}
p_{0}(t) &= {\bf P}\{\nu(t)=0\} \,;
\\
P_{n}(t;x_1,\ldots,x_{n})
&= {}\\
&\hspace*{-30mm}{}={\bf P}\{\nu(t)=n,\, \xi_{1}(t)<x_{1},\ldots,\xi_{n}(t)<x_{n}\}
\,,\ \ n\ge 1\,.
\end{align*}
Обозначим через
\begin{align*}
p_{0} &= \lim\limits_{t\to\infty} p_{0}(t) \,;
\\
P_{n}(x_1,\ldots,x_{n}) &=
\lim\limits_{t\to\infty} P_{n}(t;x_1,\ldots,x_{n})\,,\ \ n\ge 1\,,
\end{align*}
стационарное распределение процесса $\eta(t)$.
В~силу сделанных в предыдущем разделе
предположений относительно параметров системы
существуют (см., например,~\cite{ppav})
непрерывные ограниченные плотности
\begin{equation*}
p_n(x_1,\ldots,x_{n}) =
\fr{\partial^n}{\partial x_1\cdots \partial x_n}
P_n(x_1,\ldots,x_{n}) \,,\ \ n\ge 1\,.
\end{equation*}

Выпишем систему интегродифференциальных
уравнений, которой удовлетворяют стационарные
плотности $p_n(x_1,\ldots,x_{n})$ и которую
для краткости по аналогии с простейшими СМО
будем называть системой уравнений равновесия (СУР).
Для этого рассмотрим вспомогательную систему
с $(n-1)$ мес\-та\-ми ожидания, отличающуюся от исходной
сис\-те\-мы только тем, что если в очереди
находится $(n-1)$\linebreak заявок, заявка на приборе имеет
остаточную длину~$v$ и поступает новая заявка
длины~$u$, то с вероятностью $d(x,y|u,v)$ на
приборе остается вновь\linebreak поступившая заявка,
длина которой становится равной~$x$, а
обслуживавшаяся ранее заявка покидает
систему и, наоборот, с вероятностью
$d^*(y,x|u,v)$ систему покидает вновь
поступившая заявка, а находившаяся ранее на
приборе заявка продолжает обслуживаться, но
ее длина становится равной~$x$.

В силу метода исключения состояний~\cite{ppav}
стационарные вероятности состояний в исходной
и вспомогательной системах отличаются лишь на
постоянный множитель.
Это дает возможность при составлении
уравнений для $p_n(x_1,\ldots,x_{n})$, $n\hm\ge 1$,
воспользоваться вспомогательной системой и
получить следующие соотношения:
\begin{multline}
\label{(3.1)}
-p'_1(x) =
\tl \tb(x) p_0 - \lambda p_1(x) +{}\\
{}+ \lambda \left(
\int\limits_0^\infty \int\limits_0^\infty
d(x|u,v) b(u) p_1(v) \,du\,dv
+ {}\right.\\
{}+
\int\limits_0^\infty \int\limits_0^\infty \int\limits_0^\infty
[d(x,y|u,v) b(u) p_1(v) +{}\\
\left.{}+ d^*(y,x|u,v) b(u) p_1(v)]
\,dy\,du\,dv \vphantom{\int\limits_0^\infty}
\right)\,;
\end{multline}

\vspace*{-12pt}

\noindent
\begin{multline}
\label{(3.2)}
-p'_{n}(x_1,\ldots,x_n) = {}\\
{}= \lambda \left(
\int\limits_0^\infty \int\limits_0^\infty
\left[d(x_2,x_1|u,v) b(u) p_{n-1}(v,x_3\ldots,x_n)
+ {}\right.\right.\\
\left.\left.{}+
d^*(x_1,x_2|u,v) b(u) p_{n-1}(v,x_3,\ldots,x_n)\right] \,du\,dv
\vphantom{\int\limits_0^\infty}
\right)
- {}\\
{}- \lambda p_{n}(x_1,\ldots,x_n) +{}\\
{}+ \lambda
\left(
\int\limits_0^\infty
\int\limits_0^\infty
d(x_1|u,v) b(u) p_{n}(v,x_2,\ldots,x_n)
\,du\,dv
+ {}\right.\\
{}+ \int\limits_0^\infty
\int\limits_0^\infty
\int\limits_0^\infty
\left[d(x_1,y|u,v) b(u) p_{n}(v,x_2,\ldots,x_n)
+ {}\right.\\
\left.\left.{}+
d^*(y,x_1|u,v) b(u) p_{n}(v,x_2,\ldots,x_n)\right]
\,dy\,du\,dv \vphantom{\int\limits_0^\infty}\right)\,,
\\ n\ge 2\,.
\end{multline}

Остановимся подробнее на выводе уравнения~\eqref{(3.2)}.
Рассмотрим моменты времени~$t$ и $(t\hm+\Delta)$.
Тогда для того чтобы в момент времени
$(t\hm+\Delta)$ в системе
находилось~$n$, $n\hm\ge 2$, заявок, причем
на приборе заявка длины~$x_1$, а в очереди
заявки длин $x_2,\ldots,x_n$, нужно, чтобы
произошло одно из следующих событий:
\begin{itemize}
\item
в момент $t$ в системе находилось $(n-1)$
заявок, причем заявка на приборе имела
длину~$v$, первая заявка в очереди имела
длину $x_3,\ldots,$ последняя заявка в очереди
\mbox{имела} длину~$x_n$ (с плотностью
вероятностей $p_{n-1}(t;v,x_3,\ldots,x_n)$),
и за время~$\Delta$ поступила заявка
(с вероятностью $\lambda\Delta$) длины~$u$
(с плотностью вероятностей $b(u)$).
Заявка на приборе продолжает обслуживаться,
но ее длина становится равной~$x_1$, а вновь
поступившая заявка занимает первое место в
очереди и ее длина становится равной~$x_2$
(с плотностью вероятностей $d(x_2,x_1|u,v)$);
\item
в момент $t$ в системе находилось $(n\hm-1)$
заявок, причем заявка на приборе имела
длину~$v$, первая заявка в очереди имела
длину $x_3,\ldots,$ последняя заявка в
очереди \mbox{имела} длину~$x_n$ (с плот\-ностью
вероятностей $p_{n-1}(t;v,x_3,\ldots,x_n)$),
и за время~$\Delta$ поступила заявка
(с~вероятностью $\lambda\Delta$) длины~$u$
(с~плот\-ностью вероятностей $b(u)$).
Поступившая заявка занимает прибор, и ее длина
становится равной~$x_1$, а заявка,
обслуживавшаяся до поступления новой заявки,
занимает первое мес\-то в очереди, и ее длина
становится равной~$x_2$ (с~плот\-ностью
вероятностей $d^*(x_1,x_2|u,v)$);
\item
в момент $t$ в сис\-те\-ме находилось $n$ заявок,
причем заявка на приборе имела
длину $x_1\hm+\Delta$, первая заявка в очереди
имела длину $x_2,\ldots,$ последняя заявка в
очереди имела длину~$x_n$ (с плотностью
вероятностей
$p_n(t;x_1\hm+\Delta,x_2,\ldots,x_n)$), и за
время~$\Delta$ не поступили заявки (с
вероятностью $(1\hm-\lambda\Delta$));
\item
в момент $t$ в сис\-те\-ме находилось $n$ заявок,
причем заявка на приборе имела длину~$v$,
первая заявка в очереди имела длину
$x_2,\ldots,$ последняя заявка в очереди
имела длину~$x_n$ (с плот\-ностью
вероятностей $p_n(t;v,\ldots,x_n)$), и за время~$\Delta$ поступила заявка (с вероятностью
$\lambda\Delta$), име\-ющая длину~$u$
(с~плот\-ностью вероятностей $b(u)$).
Заявка, находившаяся на приборе, покидает
сис\-те\-му, а поступившая заявка становится на
прибор, причем ее длина становится
равной~$x_1$, или, наоборот, поступившая
заявка сразу же покидает сис\-те\-му, а заявка,
находившаяся на приборе, продолжает
обслуживаться, причем ее длина становится
равной~$x_1$ (с плотностью
вероятностей $d(x_1|u,v)$);
\item
в момент $t$ в сис\-те\-ме находилось $n$ заявок,
причем заявка на приборе имела длину~$v$,
первая заявка в очереди имела длину
$x_2,\ldots,$ последняя заявка в очереди
имела длину~$x_n$ (с плотностью вероятностей
$p_n(t;v,x_2,\ldots,x_n)$), и за время~$\Delta$
поступила заявка (с вероятностью $\lambda\Delta$) длины~$u$
(с плотностью вероятностей $b(u)$).
Заявка, находившаяся на приборе, покинула
сис\-те\-му, а~на прибор встала поступившая
заявка, длина которой стала~$x_1$ (с
плотностью вероятностей $d(x_1,y|u,v)$);
\item
в момент $t$ в сис\-те\-ме находилось $n$ заявок,
причем заявка на приборе имела длину~$v$,
первая заявка в очереди имела длину
$x_2,\ldots,$ последняя заявка в очереди
имела длину~$x_n$ (с плотностью вероятностей
$p_n(t;v,x_2,\ldots,x_n)$), и за время~$\Delta$ поступила заявка (с вероятностью
$\lambda\Delta$) длины~$u$ (с плотностью
вероятностей $b(u)$).
Вновь поступившая заявка покидает сис\-те\-му,
на приборе продолжает обслуживаться заявка,
на\-хо\-див\-шая\-ся на приборе до поступления новой
заявки, но длина ее становится равной~$x_1$
(с плотностью вероятностей $d^*(y,x_1|u,v)$).
\end{itemize}
Вероятности других событий равны $o(\Delta)$.

Применяя формулу полной вероятности, имеем
\begin{multline*}
p_{n}(t+\Delta;x_1,\ldots,x_n)
={}\\
{}= \lambda\Delta \left(
\int\limits_0^\infty \!\int\limits_0^\infty
\left[d(x_2,x_1|u,v) b(u) p_{n-1}(t;v,x_3\ldots,x_n)
+{}\right.\right.\hspace*{-8pt}\\
\left.\left.{}+
d^*(x_1,x_2|u,v) b(u) p_{n-1}(t;v,x_3,\ldots,x_n)\right]
\,du\,dv \vphantom{\int\limits_0^\infty}\right)
+ {}\\
{}+
(1-\lambda\Delta) p_{n}(t;x_1+\Delta,x_2,\ldots,x_n)
+ {}\\
{}+\lambda\Delta \left(
\int\limits_0^\infty \int\limits_0^\infty
d(x_1|u,v) b(u) p_{n}(t;v,x_2,\ldots,x_n)
\,du\,dv + {}\right.\hspace*{-1.28279pt}\\
{}+
\int\limits_0^\infty \int\limits_0^\infty
\int\limits_0^\infty \left[d(x_1,y|u,v) b(u) p_{n}(t;v,x_2,\ldots,x_n)\right.
+{}\\
{}+
\left.\left.d^*(y,x_1|u,v) b(u) p_{n}(t;v,x_2,\ldots,x_n)\right]
\,dy\,du\,dv \vphantom{\int\limits_0^\infty}
\right)+{}\\
{}+o(\Delta)\,, \quad n\ge 2\,,
\end{multline*}
откуда, перенося слагаемое
$p_n(t;x_1+\Delta,x_2,\ldots,x_{n})$ в
левую часть равенства, деля на~$\Delta$,
устремляя~$\Delta$ к нулю и учитывая
стационарный режим функционирования сис\-те\-мы,
получаем уравнение~\eqref{(3.2)}.

Уравнение~\eqref{(3.1)} получается аналогично.

К системе уравнений~\eqref{(3.1)},
\eqref{(3.2)} нужно добавить начальные условия,
которые удобно записать в виде:
\begin{equation}
\label{(3.3)}
p_{1}(\infty) = \lim\limits_{X\to \infty}
p_{1}(X) = 0\,;                         %      \eqno(3.3)
\end{equation}

\vspace*{-12pt}

\noindent
\begin{multline}
p_{n}(\infty,x_2,\ldots,x_n) = \lim\limits_{X\to \infty}
p_{n}(X,x_2,\ldots,x_n) = 0\,,\\  n\ge 2\,.
\label{(3.4)}
\end{multline}

Покажем, как получаются соотношения~\eqref{(3.3)}, \eqref{(3.4)}
на примере~\eqref{(3.3)}.
Интегрируя равенство~\eqref{(3.1)} в пределах
от~0 до~$X$, приходим к формуле
\begin{multline*}
p_1(0) - p_1(X) = \tl \tB(X) p_0 - \lambda P_1(X)
+ {}\\
{}+\lambda \int\limits_0^X \left(
\int\limits_0^\infty
\int\limits_0^\infty
d(x|u,v) b(u) p_1(v)
\,du\,dv
+{}\right.
\end{multline*}

\noindent
\begin{multline*}
{}+
\int\limits_0^\infty
\int\limits_0^\infty
\int\limits_0^\infty
\left[d(x,y|u,v) b(u) p_1(v)
+{}\right.\\
\left.\left.{}+
d^*(y,x|u,v) b(u) p_1(v)\right]
\,dy\,du\,dv \vphantom{\int\limits_0^\infty}
\right) dx\,,
\end{multline*}
которую перепишем в виде:
\begin{multline}
\label{(3.5)}
p_1(X) = p_1(0) - \tl \tB(X) p_0
+ \lambda P_1(X) - {}\\
{}-\lambda \int\limits_0^\infty
\int\limits_0^\infty b(u) p_1(v)
\,du\,dv \int\limits_0^X
\left( \vphantom{\int\limits_0^\infty}
d(x|u,v)
+{}\right.\\
\left.{}+
\int\limits_0^\infty
[d(x,y|u,v) + d^*(y,x|u,v) ]\,dy
\right)\,dx.                         %   \eqno(3.5)
\end{multline}
Правая часть~\eqref{(3.5)} имеет предел
(и даже с учетом~\eqref{(2.1)} конечный)
при $X\hm\to\infty$.
Поэтому $p_1(X)$ также стремится при
$X\hm\to\infty$ к пределу, который для плотности
вероятностей не может быть ничем иным, кроме
нуля, что доказывает справедливость~\eqref{(3.3)}.

Оставшаяся неизвестной стационарная
вероятность~$p_0$ отсутствия заявок в системе
находится, как обычно, из условия нормировки:
\begin{equation}
\label{(3.6)}
\sum\limits_{n=0}^\infty p_n = 1\,,
\end{equation}
где
$p_n=P_n(\infty,\ldots,\infty)$, $n\hm\ge1$,~---
стационарная вероятность наличия в системе~$n$~заявок.

Полученные соотношения~\eqref{(3.1)}--\eqref{(3.6)} позволяют
(теоретически) последовательно по~$n$
находить стационарные плотности вероятностей
$p_n(x_1,\ldots,x_{n})$ (один из методов
решения СУР~\eqref{(3.1)}--\eqref{(3.6)}
будет рассмотрен в следующем разделе на примере маргинальных плотностей).
Но практическое применение такого подхода
невозможно, поскольку при $n\hm\to\infty$ число
аргументов~$x_i$ стационарных плотностей
вероятностей $p_n(x_1,\ldots,x_{n})$ стремится к бесконечности.

Однако в большинстве практических случаев
достаточно знать только маргинальные плотности
\begin{equation*}
p_{n}(x) = \mathop{\int\!\cdots\!\int}\limits_{x_2,\ldots,x_n>0}
\!p_{n}(x,x_2\ldots,x_n) \,dx_2\cdots dx_n\,,
\ n\ge 2\,.
\end{equation*}

Интегрируя~\eqref{(3.2)} по~$x_2,\ldots ,x_n$ в пределах от нуля до
бесконечности и вспоминая равенство~\eqref{(3.1)}, получаем следующее
интегродифференциальное уравнение
для $p_{n}(x)$, $n\hm\ge 1$:
\begin{multline}
\label{(3.7)}
-p'_{n}(x) = a_n(x) - \lambda p_{n}(x)
+ \int\limits_0^\infty K_n(x,v) p_{n}(v)\,dv \,,\\ n\ge 1\,,
\end{multline}
где
\begin{equation*}
a_1(x) = \tl \tb(x) p_0 \,;
\end{equation*}

\vspace*{-12pt}

\noindent
\begin{multline*}
K_1(x,v) = \lambda \int\limits_0^\infty
b(u)\,du \left(\vphantom{\int\limits_0^\infty}
d(x|u,v)
+{}\right.\\
\left.{}+ \int\limits_0^\infty
[d(x,y|u,v) + d^*(y,x|u,v)]\,dy
\right)\,;
\end{multline*}

\vspace*{-12pt}

\noindent
\begin{multline*}
a_{n}(x) = {}\\
{}=\lambda \left(
\int\limits_0^\infty
p_{n-1}(v)\,dv
\int\limits_0^\infty
b(u)\,du
\int\limits_0^\infty
\left[d(y,x|u,v) +{}\right.\right.\\
\left.\left.{}+ d^*(x,y|u,v)\right] \,dy
\vphantom{\int\limits_0^\infty}
\right)\,,\enskip  n\ge 2\,;
\end{multline*}
\vspace*{-12pt}

\noindent
\begin{multline*}
K_n(x,v) = \l \int\limits_0^\infty
b(u)\,du \left( \vphantom{\int\limits_0^\infty}
d(x|u,v)
+ {}\right.\\
\left.{}+\int\limits_0^\infty
[d(x,y|u,v) + d^*(y,x|u,v)]
\,dy \right)\,, \ \ n\ge 2\,.
\end{multline*}
Начальное условие для уравнения~\eqref{(3.7)}
по аналогии с~\eqref{(3.3)} запишем в виде:
\begin{equation}
\label{(3.8)}
p_{n}(\infty) = \lim\limits_{X\to \infty}
p_{n}(X) = 0\,,\ \ n\ge 1\,.
\end{equation}

\section{Метод численного решения системы
уравнений равновесия}

Приведем один из возможных методов решения
интегродифференциального уравнения~\eqref{(3.7)}
с начальным условием~\eqref{(3.8)}.

Решение будем искать в виде:
\begin{equation}
\label{(4.1)}
p_n(x) = e^{\l x} q_n(x) \,.
\end{equation}
Подставляя в~\eqref{(3.7)} вместо $p_n(x)$
ее выражение по формуле~\eqref{(4.1)}, получаем новое
интегродифференциальное уравнение:
\begin{equation*}
- q'_n(x) = e^{-\l x} a_n(x) +
\int\limits_0^\infty e^{\l v} e^{-\l x} K_n(x,v) q_n(v)\, dv\,.
\end{equation*}
Интегрируя последнее равенство по~$x$ в
пределах от~$y$ до~$\infty$ и учитывая
начальное условие \eqref{(3.8)}, получаем
интегральное уравнение Фредгольма 2-го рода:
\begin{equation*}
q_n(y) = b_n(y) + \int\limits_0^\infty
G_n(y,v) q_n(v)\, dv \,,
\end{equation*}
где
\begin{equation*}
b_n(y) = \int\limits_y^\infty e^{-\l x} a_n(x)\, dx\,;
\end{equation*}
\begin{equation}
\label{(4.2)}
G_n(y,v) = \int\limits_y^\infty e^{\l (v-x)} K_n(x,v)\, dx\,.
\end{equation}

Отметим, что свободный член $b_n(y)$ и ядро
$G_n(y,v)$ интегрального уравнения~\eqref{(4.2)}
являются неотрицательными функциями.

Численные методы решения интегральных
уравнений Фредгольма 2-го рода хорошо известны
(см., например,~[10--12]).
Так, в данном случае хороший результат дает
итерационный метод
\begin{equation*}
q^{(s)}_n(y) = b_n(y) + \int\limits_0^\infty
G_n(y,v) q^{(s-1)}_n(v)\, dv \,,
\end{equation*}
причем в качестве начальной итерации
необходимо взять нулевое приближение
${q^{(0)}_n(v)\hm\equiv 0}$.
Тогда итерации будут возрастающими, что
позволит контролировать скорость сходимости к точному решению.

\section{Применение производящей функции}

Введем производящую функцию (ПФ)
$$
\pi(z,x)= \sum\limits_{n=1}^\infty p_n(x)z^n\,.
$$

Умножая уравнение~\eqref{(3.7)} на~$z^n$
и суммируя по всем значениям $n\hm\ge1$, получаем:
\begin{multline}
\label{(5.1)}
-\fr{\partial \pi(z,x)}{\partial x}
= z \tl \tb(x) p_0 - \l \pi(z,x)
+ {}\\
{}+\lambda \int\limits_0^\infty \pi(z,v)\,dv
\int\limits_0^\infty
b(u)\,du \left(\vphantom{\int\limits_0^\infty}
d(x|u,v) +{}\right.
\\
{}+ \int\limits_0^\infty
\left[d(x,y|u,v) + d^*(y,x|u,v)
+ z d(y,x|u,v) +{}\right.\\
\left.\left.{}+ z d^*(x,y|u,v)\right]
\,dy \vphantom{\int\limits_0^\infty}\right)\,.
\end{multline}
Граничное условие для интегродифференциального
уравнения~\eqref{(5.1)}, как и прежде, имеет вид:
\begin{equation*}
\pi(z,\infty) = \lim\limits_{x\to\infty} \pi(z,x)
= 0\,.
\end{equation*}

Трактуя $z$ как параметр, для решения
уравнения~\eqref{(5.1)} можно применить метод,
описанный в предыдущем разделе.


Однако ПФ $\pi(z,x)$, как правило, мало
пригодна для вычисления стационарного
распределения числа заявок в системе.
Но с ее помощью можно найти моменты этого
распределения.


Покажем, как это делается, на примере
математического ожидания ${\sf E}\,\nu$
стационарного распределения числа заявок в
системе.
При этом, не вдаваясь в математические тонкости,
будем считать, что ${\sf E}\,\nu$ существует и
операции дифференцирования, которые будут
применены далее, законны.



Сначала, решая уравнение~\eqref{(5.1)} при $z\hm=1$,
\mbox{найдем} значение $\pi(1,x)$.


Затем продифференцируем уравнение \eqref{(5.1)}
по $z$ в точке $z=1$.
При этом для сокращения записи применим
обозначение:
$$
\fr{\partial \pi(z,x) }{\partial z}\Big|_{z=1} =
\pi'(1,x) \,.
$$
Имеем:
\begin{multline}
-\fr{\partial \pi'(1,x)}{\partial x} = \tl \tb(x) p_0 - \l \pi'(1,x)
+{}\\
{}+ \lambda \int\limits_0^\infty \pi'(1,v)\,dv
\int\limits_0^\infty b(u)\,du
\Bigg( d(x|u,v)
+ {}\\
{}+
\int\limits_0^\infty
\left[d(x,y|u,v) + d^*(y,x|u,v)
+{}\right.\\
\left.{}+ z d(y,x|u,v) + z d^*(x,y|u,v)\right] \,dy
\Bigg)
+ {}\\
{}+
\l \int\limits_0^\infty \pi(1,v)\,dv
\int\limits_0^\infty b(u)\,du
\Bigg( \int\limits_0^\infty
\left[d(y,x|u,v) +{}\right.\\
\left.{}+ d^*(x,y|u,v)\right]
\,dy \Bigg) \,.
\label{(5.3)}
\end{multline}
Граничное условие для интегродифференциального
уравнения~\eqref{(5.3)} определяется прежним равенством:
\begin{equation*}
\pi'(1,\infty) = \lim\limits_{x\to\infty} \pi'(1,x) = 0\,.
\end{equation*}
Уравнение~\eqref{(5.3)} имеет
точно такой же вид, что и уравнение~\eqref{(5.1)},
и может быть решено теми же самыми методами.

Осталось заметить, что
$$
{\sf E}\,\nu = \sum\limits_{n=1}^\infty n
\int\limits_0^\infty p_n(x)\, dx
= \int\limits_0^\infty \pi'(1,x)\, dx \,.
$$

\section{Примеры расчетов}

На основе полученных результатов с помощью
программных средств MATLAB была написана
программа, позволяющая вычислять совместное
стационарное распределение числа заявок в
системе и остаточного времени обслуживания
заявки на \mbox{приборе} и другие связанные с
ним характеристики, а также исследовать
поведение рассматриваемой СМО
в зависимости от значений определяющих ее
исходных параметров.
{Разработанная} программа позволяет
проводить исследование в широкой области
изменения исходных параметров
и при любых заданных аналитически функциях
$d(x,y|u,v)$, $d^*(x,y|u,v)$, $d_0(x|u,v)$,
$d_0^*(x|u,v)$ и $d_0(u,v)$,
удовле\-тво\-ря\-ющих условию существования стационар-\linebreak ного режима.
Однако вид этих функций существенным образом
влияет на эффективность работы\linebreak про\-граммы.

Приведем некоторые результаты расчетов, проведенных с помощью разработанной программы.
При этом будем предполагать, что $\tB(x)\hm=B(x)$,
и среднее время обслуживания~$b$ заявки на приборе положим равным~1.
Параметр~$\l$ далее будет использоваться как аргумент при
построении графиков, поскольку он совпадает с
параметром $\rho\hm=\l b\hm=\l$.

\begin{figure*} %fig1
\begin{minipage}[t]{80mm}
\vspace*{1pt}
\begin{center}
\mbox{%
\epsfxsize=73.96mm
\epsfbox{raz-1.eps}
}
\end{center}
\vspace*{-9pt}
%\label{ris:image2}
\Caption{Пример~1: cреднее число заявок в системе~(\textit{1}) и
СКО числа заявок~(\textit{2})}
\end{minipage}
%\end{figure*}
\hfill
%\begin{figure*} %fig2
\begin{minipage}[t]{81.5mm}
\vspace*{1pt}
\begin{center}
\mbox{%
\epsfxsize=71.986mm
\epsfbox{raz-2.eps}
}
\end{center}
\vspace*{-9pt}
%\label{ris:image2}
\Caption{Пример~2: cреднее число заявок в системе~(\textit{1}) и
СКО числа заявок~(\textit{2})}
\end{minipage}
\end{figure*}

\subsection{Пример~1}

Предположим, что $\l\hm=\tl$, функции $d^*(x,y|u,v)$,
$d(x|u,v)$ и $d_0(u,v)$ тождественно равны нулю
при всех~$x$, $y$, $u$ и~$v$, а функция $d(x,y|u,v)$
при всех~$u$ и~$v$ имеет вид $d(x,y|u,v)\hm=b(x)b(y)$.
Стационарные вероятности состояний этой СМО
совпадают со стационарными вероятностями состояний
СМО $M|G|1$ при следующей дисциплине обслуживания:
поступающая заявка становится на первое место в
очереди, а обслуживавшаяся ранее заявка остается на
приборе, но ее длина разыгрывается заново с той же
функцией распределения $B(x)$.

Если же $d(x,y|u,v)$, $d(x|u,v)$ и $d_0(u,v)$
тождественно равны нулю, а $d^*(x,y|u,v)\hm=b(x)b(y)$,
то с точки зрения стационарных вероятностей
со\-сто\-яний сис\-те\-ма эквивалентна СМО $M|G|1$ с
дис\-цип\-ли\-ной, при которой поступающая заявка
становится на прибор, а обслуживавшаяся ранее заявка
переходит на первое место в очереди и ее длина
разыгрывается заново с функцией распределения $B(x)$.

Для численных расчетов в примере~1 будем считать, что
время обслуживания заявки распределено по
экспоненциальному закону, т.\,е.\ $B(x)\hm=1\hm-e^{-x}$.
Тогда при обоих вариантах выбора $d(x,y|u,v)$
и $d^*(x,y|u,v)$ стационарные вероятности
состояний будут совпадать со стационарными
вероятностями состояний обычной СМО $M|M|1$.
Графики среднего и среднего квадратичного отклонения
(СКО) числа заявок в системе
в зависимости от интенсивности входящего потока~$\lambda$ представлены на
рис.~1.
Как видно, эти данные полностью совпадают
с хорошо известными для СМО $M|M|1$ результатами.

\subsection{Пример 2}

 В следующем примере предположим, что функция
$d(x|u,v)$ тождественно равна нулю при всех~$x$, $u$
и $v$, а функция $d_0(u,v)>0$ при всех $u$ и $v$.
 Это фактически означает, что с вероятностью
$d_0(u,v)$ поступает отрицательная заявка,
которая уничтожает находящуюся на приборе
заявку и вместе с ней покидает систему.
 В частности, если интенсивность поступления
в систему отрицательных заявок равна~$\gamma$,
то $d_0(u,v)= \gamma/\lambda$
при всех $u$, $v$.
 Теперь, если положить $\tl\hm=\lambda - \gamma$,
функцию $d^*(x,y|u,v)\hm=0$ при
всех $x$, $y$, $u$ и $v$, функцию
$d(x,y|u,v)=(\lambda-\gamma) b(x)b(y)/\l$
при всех~$u$, $v$, $x$ и $y$, то получим первый
вариант СМО $M|G|1$ из примера 1, в которую,
наряду с потоком интенсивности $\tl$ обычных
заявок, поступает поток отрицательных заявок
интенсивности~$\gamma$.
Второй вариант СМО из примера~1 с дополнительным
потоком отрицательных заявок получается,
если, наоборот, положить
$d^*(x,y|u,v)\hm=(\lambda\hm-\gamma) b(x)b(y) / \l$
и $d(x,y|u,v)\hm=0$ при тех же самых значениях остальных
параметров.


Положим теперь $B(x)\hm=1-e^{-x}$.  При этом пусть $\gamma\hm=1$.
 Тогда приходим к СМО $M|M|1$ с инверсионным порядком обслуживания
с (без) прерыванием обслуживания и отрицательными заявками.
 Графики среднего и СКО числа заявок в системе в зависимости от интенсивности входящего
потока~$\lambda$ представлены на рис.~2.
 Нетрудно видеть, что данные результаты
полностью совпадают с результатами, которые
легко получить по аналитическим формулам для данной СМО.


\subsection{Пример 3}

 Пусть задана функция распределения $R(x)$
не\-от\-ри\-ца\-тельной случайной величины,
определенная на интервале $(0,v)$, которая имеет ограниченную плотность
распределения~$r(x)\hm=R'(x)$.  Предположим, что функции $d^*(x,y|u,v)$,
$d_0(x|u,v)$ и $d_0(u,v)$ тождественно равны нулю
при всех~$x$, $y$, $u$, $v$.
 При этом функция $d_0^*(y|u,v)\hm=\gamma r(y)/\lambda$
при всех~$u$, $v$ и $y\hm<v$ и $d_0^*(y|u,v)\hm=0$
иначе; функция $d(x,y|u,v)\hm=(\lambda - \gamma) b(x)b(y)/\lambda$
при всех~$u$, $v$ и $x\hm>0$ и $d(x,y|u,v)\hm=0$ иначе.
 Тогда при $\tl\hm=\lambda \hm- \gamma$ систему
можно трактовать как СМО $M|G|1$ с инверсионным порядком обслуживания, в которую
с вероятностью $\gamma/\lambda$ поступают заявки,
приводящие к обслуживанию заявок на приборе
заново с распределением $R(x)$.  С~вероятностью $(\lambda \hm- \gamma)/\lambda$
в систему поступают обычные заявки.

В этом примере положим $B(x)\hm=1\hm-e^{-x}$
и $R(x)\hm=0{,}05 (1\hm- e^{- x})/(1\hm-e^{- v})$.
В~результате получаем СМО $M|M|1$ с инверсионным порядком обслуживания,
в которую с вероятностью 0,05 поступают заявки,
приводящие к обслуживанию заявок на приборе заново.
Графики среднего и СКО числа заявок в системе в зависимости
от интенсивности входящего потока~$\lambda$ представлены на рис.~3.

\begin{figure*} %fig3
\begin{minipage}[t]{80mm}
\vspace*{1pt}
\begin{center}
\mbox{%
\epsfxsize=73.119mm
\epsfbox{raz-3.eps}
}
\end{center}
\vspace*{-9pt}
\Caption{Пример~3: cреднее число заявок в системе~(\textit{1}) и
СКО числа заявок~(\textit{2})}
%\label{ris:image2}
\end{minipage}
%\end{figure*}
\hfill
%\begin{figure*} %fig4
\begin{minipage}[t]{81.5mm}
\vspace*{1pt}
\begin{center}
\mbox{%
\epsfxsize=73.126mm
\epsfbox{raz-4.eps}
}
\end{center}
\vspace*{-9pt}
\Caption{Пример~4: cреднее число заявок в системе~(\textit{1}) и
СКО числа заявок~(\textit{2})}
\label{ris:image2}
\end{minipage}
\end{figure*}




\subsection{Пример 4}


 Предположим, что функция $d^*(x,y|u,v)\hm=0$
при всех $u$, $v$, $y$, $x$, а функции
$d(x,y|u,v)$, $d_0(x|u,v)$, $d_0^*(x|u,v)$
и $d_0(u,v)$ заданы следующим образом:
$$
d(x,y|u,v)=0{,}85 e^{-(x+y)}
(e^{- v}-1)^{-1}(e^{- u}-1)^{-1}
$$
при всех $u$, $v$ и $y\hm<v$, $x\hm<u$
и $d(x,y|u,v)=0$ иначе;
$$
d_0(x|u,v)=0{,}05\fr{1}{u+1}\, \fr{ e^{-(1/(u+1))x}}{
1-e^{-(1/(u+1))u}}
$$
при всех $u$, $v$ и $x<u$ и $d_0(x|u,v)\hm=0$
иначе;
$$
d_0^*(y|u,v)=0{,}05\fr{1}{v+1 }\,
\fr{e^{-(1/(v+1))y} }{1-e^{{-(1/(v+1))}v}}
$$
при всех $u$, $v$ и $y\hm<v$ и $d_0^*(x|u,v)\hm=0$
иначе;
$d_0(u,v)=0{,}05$ при всех $u$,~$v$.

В этом случае имеет место СМО $M|G|1$ с
инверсионным порядком обслуживания, причем
допускается уход из системы вновь
поступившей или недообслуженной заявки (с равными вероятностями $0{,}05$),
а также <<выбивание>>  заявки с прибора
(также с вероятностью $0{,}05$). При данном определении функций $d_0(x|u,v)$
($d_0^*(y|u,v)$) чем больше была длина вновь
поступившей заявки (обслуживаемой заявки),
тем меньше будет средняя длина заявки, которая
останется обслуживаться на приборе.
Отметим, что определенные выше функции
подобраны таким образом, чтобы
удовлетворять достаточному условию существования
стационарного режима.

Для примера~4 графики среднего и СКО числа заявок
в системе в зависимости от интенсивности входящего потока~$\lambda$ представлены на
рис.~4.

\section{Заключение}

 В настоящей работе получены интегродифференциальные уравнения
для стационарных плотностей вероятностей СМО $M/G/1/\infty$ с входящим потоком
пуассоновского типа и инверсионным порядком обслуживания с обобщенным вероятностным
приоритетом ({LCFS GPP}) и предложен метод их решения.
 По полученным соотношениям была разработана программа, которая позволяет проводить исследование
показателей про\-из\-во\-ди\-тель\-ности рассматриваемой
СМО в широкой области изменения исходных параметров.
 Приводятся результаты численных расчетов как
для частных случаев, так и для общего случая рассматриваемой системы.

В дальнейших исследованиях предполагается
обратиться к решению задачи вычисления характеристик,
связанных с временем пребывания заявки в системе.

{\small\frenchspacing
 {%\baselineskip=10.8pt
 \addcontentsline{toc}{section}{References}
 \begin{thebibliography}{99}
\bibitem{shrage} %1
\Au{Schrage L.} A~proof of the
optimality of the shortest remaining processing
time discipline~//
Oper.\ Res., 1968. Vol.~16. P.~687--690.



\bibitem{aaa1} %2
\Au{Нагоненко В.\,А.}
О~характеристиках одной нестандартной системы
массового обслуживания. I, II~//
Изв.\ АН СССР. Технич.\ кибернет., 1981.
№\,1. С.~187--195; №\,3. С.~91--99.

\bibitem{aaa3} %3
\Au{Нагоненко В.\,А., Печинкин~А.\,В.}
О~большой загрузке в системе с инверсионным
обслуживанием и вероятностным приоритетом~//
Изв.\ АН СССР. Технич.\ кибернет., 1982. №\,1. С.~86--94.

\bibitem{aaa2} %4
\Au{Печинкин А.\,В.} Об одной
инвариантной системе массового обслуживания~//
Math.\ Operationsforsch.\ und Statist.
Ser.\ Optimization, 1983. Vol.~14. №\,3. S.~433--444.



\bibitem{aaa4} %5
\Au{Нагоненко~В.\,А., Печинкин~А.\,В.}
О~малой загрузке в системе с инверсионным порядком
обслуживания и вероятностным приоритетом~//
Изв.\ АН СССР. Технич.\ кибернет., 1984. №\,6. С.~82--89.

\bibitem{av1} %6
\Au{Печинкин А.\,В., Стальченко И.\,В.}
Система MAP$/G/1/\infty$ с инверсионным порядком
обслуживания и вероятностным приоритетом,
функционирующая в дискретном времени~//
Вестник Российского ун-та дружбы народов.
Сер.\ Математика. Информатика. Физика, 2010.
№\,2. С.~26--36.

\bibitem{av2} %7
\Au{Касконе А., Манзо~Р., Печинкин~А.\,В., Салерно~С.}
Система MAP$/G/1/\infty$ в дискретном
времени с инверсионной вероятностной дисциплиной
обслуживания~//
Автоматика и телемеханика, 2010. №\,12. С.~57--69.

\bibitem{av3} %8
\Au{Милованова Т.\,А., Печинкин А.\,В.}
Стационарные характеристики системы обслуживания с
инверсионным порядком обслуживания, вероятностным
приоритетом и гистерезисной политикой~//
Информатика и её применения, 2013. Т.~7. Вып.~1. С.~22--36.

\bibitem{ppav} %9
\Au{Бочаров  П.\,П., Печинкин~А.\,В.}
Теория массового обслуживания.~--- М.: РУДН, 1995.
529~с.

\bibitem{jerri} %10
\Au{Jerri A.}
Introduction to integral equations with
applications.~--- N.Y.: John Wiley \& Sons, 1999.
433~p.

\vspace*{-1pt}

\bibitem{wh} %11
\Au{Press W.\,H., Teukolsky~S.\,A.,
Vetterling~W.\,T., Flannery~B.\,P.}
Numerical recipes:
The art of scientific computing.  3rd ed. New York: Cambridge University
Press, 2007. 1235~p.

\vspace*{-1pt}

\bibitem{adav} %12
\Au{Полянин А.\,Д., Манжиров~А.\,В.}
Справочник по интегральным уравнениям. ---
М.: Физматлит, 2003. 608~с.
%Chapman \& Hall, CRC Press, 2008.




 \end{thebibliography}

 }
 }

\end{multicols}

\vspace*{-9pt}

\hfill{\small\textit{Поступила в редакцию 17.06.14}}

%\newpage

\vspace*{10pt}

\hrule

\vspace*{2pt}

\hrule

%\vspace*{12pt}

\def\tit{STATIONARY DISTRIBUTION IN~A~QUEUEING SYSTEM
WITH~INVERSE SERVICE ORDER AND~GENERALIZED
PROBABILISTIC PRIORITY}

\def\titkol{Stationary distribution in a queueing system
with inverse service order and generalized
probabilistic priority}

\def\aut{L.\,A.~Meykhanadzhyan$^1$, T.\,A.~Milovanova$^1$, A.\,V.~Pechinkin$^2$,
and~R.\,V.~Razumchik$^{1,2}$}

\def\autkol{L.\,A.~Meykhanadzhyan, T.\,A.~Milovanova, A.\,V.~Pechinkin,
and~R.\,V.~Razumchik}

\titel{\tit}{\aut}{\autkol}{\titkol}

\vspace*{-9pt}

\noindent
$^1$Peoples' Friendship University of Russia,
6~Miklukho-Maklaya Str., Moscow 117198, Russian Federation

\noindent
$^2$Institute of Informatics Problems, Russian Academy of Sciences,
44-2 Vavilov Str., Moscow 119333, Russian\\
$\hphantom{^1}$Federation


\def\leftfootline{\small{\textbf{\thepage}
\hfill INFORMATIKA I EE PRIMENENIYA~--- INFORMATICS AND
APPLICATIONS\ \ \ 2014\ \ \ volume~8\ \ \ issue\ 3}
}%
 \def\rightfootline{\small{INFORMATIKA I EE PRIMENENIYA~---
INFORMATICS AND APPLICATIONS\ \ \ 2014\ \ \ volume~8\ \ \ issue\ 3
\hfill \textbf{\thepage}}}

\vspace*{3pt}


\Abste{Consideration is given to $M|G|1$ type queueing system.
Inverse service order with generalized probabilistic
priority is implemented in the system.
It is assumed that at any instant, the remaining service
time of each customer residing in the system is known.
Upon arrival of a new customer, the system finds out its service time
and compares it with the remaining service
time of the currently served customer. The result of this comparison
leads to one of the cases: one of them
enters the server and another occupies the first place in the queue;
one of them leaves the system and another enters the server; or
both leave the system.
In each case when customer remains in the system,
its remaining service time may be updated.
An analytical method that allows computing
stationary performance characteristics related to the number
of customers in the system is presented.
Numerical examples based on the developed mathematical relations are provided.}

\KWE{queueing system; special discipline; LIFO; probabilistic priority; general service time}

\DOI{10.14357/19922264140304}

\vspace*{-9pt}

\Ack
\noindent
The research was partially supported by the Russian Foundation for
Basic Research (grant No.\,13-07-00223).

%\vspace*{3pt}

  \begin{multicols}{2}

\renewcommand{\bibname}{\protect\rmfamily References}
%\renewcommand{\bibname}{\large\protect\rm References}

{\small\frenchspacing
 {%\baselineskip=10.8pt
 \addcontentsline{toc}{section}{References}
 \begin{thebibliography}{99}


\bibitem{shrage-1} %1
\Aue{Schrage, L.} 1968.
A~proof of the optimality of
the shortest remaining processing time discipline.
\textit{Oper.\ Res.} 16:687--690.

\bibitem{aaa1-1} %2
\Aue{Nagonenko, V.\,A.} 1981.
O~kharakteristikakh odnoy nestandartnoy sistemy
massovogo obsluzhivaniya
[On the characteristics of one nonstandard queuing
system].~I, II.
\textit{Izv.\ AN SSSR. Tekhnich.\ kibernet.}
[Technical Cybernetics] 1:187--195; 3:91--99.


\bibitem{aaa3-1} %3
\Aue{Nagonenko, V.\,A., and A.\,V.~Pechinkin}. 1982.
O~bol'shoy zagruzke v sisteme s inversionnym
obsluzhivaniem i ve\-ro\-yat\-nostnym prioritetom
[On high load in the system with an inversion
procedure service and probabilistic priority].
\textit{Izv.\ AN SSSR. Tekhnich.\ kibernet.}
[Technical Cybernetics] (1):86--94.

\bibitem{aaa2-1} %4
\Aue{Pechinkin, A.\,V.} 1983.
Ob odnoy invariantnoy sisteme massovogo
obsluzhivaniya [On an invariant queuing system].
\textit{Math.\ Operationsforsch.\ und Statist.
Ser.\ Optimization} 14(3):433--444.


\bibitem{aaa4-1} %5
\Aue{Nagonenko, V.\,A., and A.\,V.~Pechinkin}. 1984.
O~ma\-loy zagruzke v sisteme s inversionnym poryadkom
obsluzhivaniya i veroyatnostnym prioritetom
[On low load in the system with an inversion
procedure service and probabilistic priority].
\textit{Izv.\ AN SSSR. Tekhnich.\ kibernet}
[{Technical Cybernetics}] (6):82--89.

\bibitem{av1-1} %6
\Aue{Pechinkin, A.\,V., and I.\,V.~Stalchenko}. 2010.
Sistema MAP$/G/1/\infty$ s inversionnym poryadkom
obsluzhivaniya i veroyatnostnym prioritetom,
funktsioniruyushchaya v diskretnom vremeni
[The MAP$/G/1/\infty$ discrete-time queueing
system with inversive service order and probabilistic
priority].
\textit{Vestnik Rossiyskogo Un-ta druzh\-by
na-}\linebreak\vspace*{-12pt}

\pagebreak

\noindent
\textit{ro\-dov. Ser.\ Matematika. Informatika. Fizika.}
[Bulletin of Peoples' Friendship University
of Russia. Ser. Mathematics. Information Sciences.
Physics] (2):26--36.

\bibitem{av2-1} %7
\Aue{Cascone, A., R.~Manzo, A.\,V.~Pechinkin,
and S.~Salerno}. 2010.
Sistema MAP$/G/1/\infty$ v diskretnom vremeni s
inversionnoy veroyatnostnoy distsiplinoy
obsluzhivaniya
[Discrete-time MAP$/G/1/\infty$ system with inversive
probabilistic servicing discipline].
\textit{Avtomat.\ i Telemekh.}
[{Automation Remote Control}] (12):57--69.

\bibitem{av3-1} %8
\Aue{Milovanova, T.\,A., and A.\,V.~Pechinkin}. 2013.
Sta\-tsi\-o\-nar\-nye kharakteristiki sistemy obsluzhivaniya
s inversionnym poryadkom obsluzhivaniya,
veroyatnostnym prioritetom i gisterezisnoy politikoy
[Stationary characteristics of queueing system with
an inversion procedure service probabilistic priority
and hysteresis policy]
\textit{Informatika i ee Primeneniya}~---
\textit{Inform. Appl}. 7(1):22--35.

\bibitem{ppav-1} %9
\Aue{Bocharov,  P.\,P., and A.\,V.~Pechinkin}. 1995.
\textit{Teoriya massovogo obsluzhivaniya} [Queueing theory].
Moscow: RUDN. 529~p.


\bibitem{jerri-1} %10
\Aue{Jerri, A.} 1999.
\textit{Introduction to integral equations with applications.}
{John Wiley} \& {Sons}.
433~p.

\bibitem{wh-1} %11
\Aue{Press, W.\,H., S.\,A.~Teukolsky, W.\,T.~Vetterling, and
B.\,P.~Flannery}. 2007.
\textit{Numerical recipes:
The art of scientific computing}.  3rd ed. New York: Cambridge University Press. 1235~p.

\bibitem{adav-1} %12
\Aue{Polyanin, A.\,D., and A.\,V.~Manzhirov}. 2008.
\textit{Handbook of integral equations.} 2nd ed.
Boca Raton\,--\,London: Chapman \& Hall\,/\,CRC Press. 1144~p.



\end{thebibliography}

 }
 }

\end{multicols}

\vspace*{-6pt}

\hfill{\small\textit{Received June 17, 2014}}

\vspace*{-18pt}

\Contr



\noindent
\textbf{Meykhanadzhyan Lusine A.} (b.\ 1990)~---
postgraduate, Peoples' Friendship University of Russia,
6~Miklukho-Maklaya Str., Moscow 117198, Russian Federation;
lameykhanadzhyan@gmail.com

\vspace*{3pt}



\noindent
\textbf{Milovanova Tatiana A.} (b.\ 1977)~---
Candidate of Science (PhD) in physics and
mathematics, senior lecturer,
Peoples' Friendship University of Russia,
6~Miklukho-Maklaya Str., Moscow 117198, Russian Federation;
tmilovanova77@mail.ru


\vspace*{3pt}

\noindent
\textbf{Pechinkin Alexander V.} (b.\ 1946)~--- Doctor
of Science in physics and mathematics; principal
scientist, Institute of Informatics Problems of
the Russian Academy of Sciences, 44-2 Vavilov Str.,
Moscow 119333, Russian Federation; apechinkin@ipiran.ru


\vspace*{3pt}

\noindent
\textbf{Razumchik Rostislav V.} (b.\ 1984)~--- Candidate
of Science (PhD) in physics and mathematics,
senior research fellow, Institute of Informatics
Problems of the Russian Academy of Sciences, 44-2 Vavilov Str.,
Moscow 119333, Russian Federation;
assistant professor,
Peoples' Friendship University of Russia,
6~Miklukho-Maklaya Str., Moscow 117198, Russian Federation;
rrazumchik@ieee.org


\label{end\stat}

\renewcommand{\bibname}{\protect\rm Литература} %2


\def\stat{sinitsini}

\def\tit{МОДЕЛИРОВАНИЕ НОРМАЛЬНЫХ ПРОЦЕССОВ В~СТОХАСТИЧЕСКИХ СИСТЕМАХ СО~СЛОЖНЫМИ
ТРАНСЦЕНДЕНТНЫМИ НЕЛИНЕЙНОСТЯМИ$^*$}

\def\titkol{Моделирование нормальных
процессов в~стохастических системах со сложными трансцендентными нелинейностями}

\def\aut{И.\,Н.~Синицын$^1$, В.\,И.~Синицын$^2$, Э.\,Р.~Корепанов$^3$}

\def\autkol{И.\,Н.~Синицын, В.\,И.~Синицын, Э.\,Р.~Корепанов}

\titel{\tit}{\aut}{\autkol}{\titkol}

\index{Синицын И.\,Н.}
\index{Синицын В.\,И.}
\index{Корепанов Э.\,Р.}

{\renewcommand{\thefootnote}{\fnsymbol{footnote}} \footnotetext[1]
{Работа выполнена при поддержке РФФИ (проект 15-07-02244).}}


\renewcommand{\thefootnote}{\arabic{footnote}}
\footnotetext[1]{Институт проблем информатики Федерального исследовательского
центра <<Информатика и~управление>> Российской академии наук,
sinitsin@dol.ru}
\footnotetext[2]{Институт проблем информатики Федерального исследовательского
центра <<Информатика и~управление>> Российской академии наук,
vsinitsin@ipiran.ru}
\footnotetext[3]{Институт проблем информатики Федерального исследовательского
центра <<Информатика и~управление>> Российской академии наук,
ekorepanov@ipiran.ru}

\vspace*{-6pt}


\Abst{Рассматривается развитие методов аналитического и~статистического моделирования нормальных стохастических процессов (СтП)
на случай непрерывных и~дискретных стохастических систем (СтС) (в~том числе на многообразиях) с~винеровскими и~пуассоновскими шумами и~со сложными трансцендентными нелинейностями (СТН). Даны типовые представления скалярных и~векторных СТН. Получены уравнения методов нормальной аппроксимации (МНА) и~статистической линеаризацией (МСЛ). Представлено алгоритмическое обеспечение МНА (МСЛ) для СтС с~СТН. Приведены тестовые примеры. Рассмотрены возможные обобщения полученных результатов.}


\KW{аналитическое и~статистическое моделирование;
метод нормальной аппроксимации (МНА);
метод статистической линеаризации (МСЛ);
многочлены Эрмита; нормальный стохастический процесс;
сложные трансцендентные нелинейности (СТН); стохастические системы (СтС)}

\DOI{10.14357/19922264150203}

\vspace*{-2pt}


\vskip 14pt plus 9pt minus 6pt

\thispagestyle{headings}

\begin{multicols}{2}

\label{st\stat}

\section{Введение}

Рассмотрим развитие методов аналитического и~статистического моделирования нормальных СтП, приведенных в~[1--4], на случай непрерывных и~дискретных СтС, в~том числе и~на многообразиях с~СТН. В~разд.~2 даны типовые представления трансцендентных функций и~соответствующих  СТН. Раздел~3 посвящен уравнениям МНА и~МСЛ. Алгоритмическое обеспечение МНА (МСЛ) для СтС с~СТН представлено в~разд.~4. Тестовые примеры приведены в~разд.~5. В заключении даны выводы и~рассмотрены некоторые обобщения полученных результатов.

\vspace*{-6pt}

\section{Трансцендентные функции и~нелинейности}

Как известно~[5, 6], трансцендентной аналитической функцией (ТАФ) в~узком смысле слова называется мероморфная функция в~плоскости комплексного переменного $y$, отличная от рациональной. В~частности, сюда относятся целые ТАФ, целые функции, отличные от многочленов, например показательная функция~$e^y$, три\-го\-но\-мет\-ри\-че\=ские функции $\sin y$ и~$\cos y$, гиперболические функции  $\mathrm{sh}\, y$ и~$\mathrm{ch}\, y$. Примерами собственно мероморфных ТАФ могут служить функции
$\tg y$, $\ctg y$, $\mathrm{th}\, y$ и~$\mathrm{cth}\, y$.

В рамках теории элементарных функций к~трансцендентным вычислительным операциям относятся операции взятия тригонометрических (гиперболических) или обратных функций, логарифмирования и~потенцирования~\cite{5-s}.

В широком смысле под ТАФ понимается всякая аналитическая функция, отличная от алгебраической, для вычисления значений которой помимо алгебраических операций над аргументом необходимо применить предельный переход в~той или иной форме. Примерами предельных переходов могут быть различные интегральные, интегродифференциальные
и~другие операторные преобразования.

Примерами СТН, получаемых посредством отрезков сумм элементарных ТАФ, могут служить следующие:
\begin{align}
\vrp^{\mathrm{СТН}} (Y,t) &=\sss_{r=1}^n l_{rt} \vrp^{\mathrm{ТН}}_r (Y)\,;\label{e2.1-s}\\
   \vrp^{\mathrm{СТН}} (Y,t) &=\sss_{r=1}^n l_{rt} \vrp^{\mathrm{ТН}}_r (Y)\vrp_r^{\mathrm{АН}} (Y)\,,\label{e2.2-s}
   \end{align}
а также дроб\-но-ра\-ци\-о\-наль\-ные пред\-став\-ления:
    \begin{align}
    \vrp^{\mathrm{СТН}} (Y,t) &=\fr{\sum\nolimits_{r=1}^{n'} l_{rt}'
      {\vrp'}^{\mathrm{ТН}}_r (Y)}{ \sum\nolimits_{r=1}^{n''} l_{rt}''
   {\vrp''}^{\mathrm{ТН}}_r (Y)}\,;\label{e2.3-s}\\
    \vrp^{\mathrm{СТН}} (Y,t) &=\fr{\sum\nolimits_{r=1}^{n'} l_{rt}'
        {\vrp'}^{\mathrm{ТН}}_r (Y){\vrp'}_r^{\mathrm{АН}} (Y)}{\sum\nolimits_{r=1}^{n''} l_{rt}'' {\vrp''}^{\mathrm{ТН}}_r (Y){\vrp''}_r^{\mathrm{АН}}
    (Y)}\,,\label{e2.4-s}
    \end{align}
где ${\vrp}^{\mathrm{ТН}}_r (Y)$, ${\vrp'}^{\mathrm{ТН}}_r (Y)$ и~${\vrp''}^{\mathrm{ТН}}_r (Y)$~--- элементарные ТАФ; $ l_{rt}$,  $l_{rt}'$  и~$l_{rt}''$~--- коэффициенты, зависящие от времени~$t$; ${\vrp}_r^{\mathrm{АН}} (Y)$, ${\vrp'}_r^{\mathrm{АН}} (Y)$ и~${\vrp''}_r^{\mathrm{АН}} (Y)$~--- алгебраические нелинейности (многочлены, степенные, иррациональные, дроб\-но-ра\-ци\-о\-наль\-ные и~другие функции).

Другими примерами СТН являются нелинейности, получаемые путем соответствующего преобразования аргумента:
    \begin{align}
    \vrp^{\mathrm{СТН}} (Y,t) &=\vrp^{\mathrm{АН}} (\psi^{\mathrm{ТН}} (Y,t), t)\,;
    \label{e2.5-s}\\
   \vrp^{\mathrm{СТН}} (Y,t) &=\vrp^{\mathrm{ТН}} (\psi^{\mathrm{АН}} (Y,t), t)\,,\label{e2.6-s}
   \end{align}
где $\vrp^{\mathrm{АН}}$, $\psi^{\mathrm{АН}}$, $\vrp^{\mathrm{ТН}} (Y,t)$ и~$\psi^{\mathrm{ТН}} (Y,t)$~--- элементарные алгебраические и~трансцендентные нелинейности.

Операция интегрирования ТАФ может выводить из класса элементарных ТАФ, т.\,е.\ интеграл от элементарной ТАФ не всегда может быть выражен через элементарные функции (алгебраические или трансцендентные). В~результате приходится обращаться к~специальным функциям~[5--8]. Прос\-тей\-ши\-ми примерами специальных функций могут служить интегралы от элементарных ТАФ, а~также функций, получающихся из них с~помощью конечного числа вычислительных операций и~операций дифференцирования~\cite{5-s}. Сюда относятся и~функции, обратные указанным.

В качестве примеров скалярных СТН векторного аргумента $Y\hm=\lk Y_1\cdots Y_p\rk^{\mathrm{T}}$ рассмотрим сле\-ду\-ющие:
    \begin{align}
    \vrp^{\mathrm{СТН}} (Y,t) &=\sss_{r=1}^n \prod\limits_{h=1}^H l_{rh,t} \vrp_{rh}^{\mathrm{ТН}} (Y_h)\,;\label{e2.7-s}\\
   \vrp^{\mathrm{СТН}} (Y,t) &=\fr{\sum_{r=1}^{n'} \prod_{h=1}^{H'} l_{rh,t}' {\vrp'}_{rh}^{\mathrm{ТН}} (Y_h)}{\sum_{r=1}^{n''} \prod_{h=1}^{H''} l_{rh,t}'' {\vrp''}_{rh}^{\mathrm{ТН}} (Y_h)}\,.\label{e2.8-s}
   \end{align}

В случае векторных и~матричных СТН формулы~(\ref{e2.1-s})--(\ref{e2.8-s}) имеют место для соответствующих компонент.

В настоящей статье ограничимся рас\-смот\-ре\-нием трансцендентных нелинейностей
(ТН), по\-лу\-ча\-ющихся согласно~(\ref{e2.1-s})--(\ref{e2.6-s}) только из элементарных трансцендент\-ных функций. Случаи специальных аналитических и~разрывных функций будут предметом дальнейшего рассмотрения.

\section{Уравнения методов нормальной аппроксимации и~статистической
 линеаризации со~сложными трансцендентными нелинейностями}

Как известно~[9--11],  уравнения конечномерных непрерывных нелинейных систем со стохастическими возмущениями путем расширения вектора состояния СтС могут быть записаны в~виде сле\-ду\-юще\-го векторного стохастического дифференциального уравнения Ито:
\begin{multline}
   dY_t = a\left(Y_t, t\right) dt + b \left(Y_t, t\right) dW_0+ {}\\
   {}+\iii_{R_0} c \left(Y_t, t, v\right) P^0 (dt, dv)\,,\enskip Y(t_0) = Y_0\,.
   \label{e3.1-s}
   \end{multline}
Здесь $Y_t$~--- $(p\times 1)$-мер\-ный вектор состояния, $Y_t \hm\in \Delta_y$ ($\Delta_y$~--- многообразие состояний);  $a\hm=a(Y_t, t)$ и~$b\hm=b(y_t, t)$~--- известные  $(p\times 1)$- и~$(p\times m)$-мер\-ные функции~$Y_t$ и~$t$;  $W_0\hm= W_0(t)$~--- $(r\times 1)$-мер\-ный винеровский СтП интенсивности  $\nu_0 \hm= \nu_0(t)$; $c(Y_t, t, v)$~--- $(p\times 1)$-мер\-ная функция  $Y_t, t$ и~вспомогательного $(q\times 1)$-мер\-но\-го параметра~$v$; $\iii_{\Delta} dP^0 (t, A)$~--- центрированная пуассоновская мера, опре\-де\-ля\-емая
    \begin{equation*}
    \iii_{\Delta} dP^0 (t, A) = \iii_{\Delta} dP (t,A) =\iii_{\Delta} \nu_P (t,A)\, dt\,.
%    \label{e3.2-s}
    \end{equation*}
В~(\ref{e3.1-s}) принято: $\iii_{\Delta}$~--- число скачков пуассоновского СтП
в интервале времени  $\Delta \hm= (t_1, t_2]$; $\nu_P (t, A)$~---
интенсивность пуассоновского СтП  $P(t,A)$; $A$~--- некоторое
борелевское множество пространства  $R_0^q$ с~выколотым началом.
Начальное значение~$Y_0$ представляет собой случайную величину
(с.в.), не зависящую от приращений $W_0(t)$ и~$P(t,A)$ на
интервалах времени, следующих за~$t_0$, $t_0 \hm\le t_1\hm\le t_2$, для
любого множества~$A$.

Для аддитивных нормальных (гауссовских) и~обобщенных пуассоновских возмущений уравнение~(\ref{e3.1-s}) имеет вид:
    \begin{equation}
    \dot Y = a\left(Y_t,t\right)+ b_0 (t) V\,, \enskip V = \dot W\,,\enskip Y(t_0) = Y_0\,.\label{e3.3-s}
    \end{equation}
Здесь $W$~--- СтП с~независимыми приращениями, представляющий собой смесь нормального и~обобщенного пуассоновского СтП.

Для компонент $\vrp(Y_t, t) \hm= \{a_h, b_{kj}, c_h\}$ функций $a$, $b$ и~$c$,  являющихся СТН, примем представления~(\ref{e2.1-s})--(\ref{e2.6-s}).
Если предположить существование конечных вероятностных моментов второго порядка для моментов времени~$t_1$ и~$t_2$, то уравнения МНА примут следующий вид~[9--11]:
\begin{itemize}
\item  для характеристических функций:
    \begin{equation}
\left.
    \begin{array}{c}
   \hspace*{-10mm}g_1^N (\la;t) =\displaystyle\exp \left[ i\la^{\mathrm{T}} m_t - \fr{1}{2} \la^{\mathrm{T}} K_t \la\right]\!;\\[9pt]
       \hspace*{-32mm}g_{t_1, t_2}^N \left(\la_1, \la_2;t_1, t_2 \right) ={}\\
    \hspace*{11mm}{}=\exp \left[ i\bar \la^{\mathrm{T}} \bar m_2 - \fr{1}{2} \bar \la^{\mathrm{T}} \bar K_2 \la\right]\!,
    \end{array}
    \right\}
    \label{e3.4-s}
    \end{equation}
где
    $$
    \bar \la =\left[ \la_1^{\mathrm{T}}\la_2^{\mathrm{T}}\right]^{\mathrm{T}}\,; \enskip\bar m_2 = \left[ m_{t_1}^{\mathrm{T}} m_{t_2}^{\mathrm{T}}\right]^{\mathrm{T}}\,;\
    $$
    $$
    \bar K_2= \begin{bmatrix}
    K\left(t_1, t_1\right)& K\left(t_1, t_2\right)\\
    K\left(t_2, t_1\right)& K\left(t_2, t_2\right)\end{bmatrix}\,;
    $$

\item для математических ожиданий~$m_t$, ковариационной матрицы~$K_t$ и~матрицы ковариационных функций $K(t_1, t_2)$:
    \begin{equation}
    \dot m_t = a_1 (m_t, K_t, t)\,,\enskip m_0 = m(t_0)\,;\label{e3.5-s}
    \end{equation}
    \begin{equation}
\dot K_t = a_2 \left(m_t, K_t, t\right)\,,\enskip K_0 = K(t_0)\,;\label{e3.6-s}
\end{equation}

\vspace*{-12pt}
\begin{multline}
   \fr{\prt K(t_1, t_2)}{\prt t_2 }= K(t_1, t_2) a_{21} \left(m_{t_2}, K_{t_2}, t_2\right)^{\mathrm{T}}\,,\\
K(t_1, t_1) = K_{t_1}\,.
    \label{e3.7-s}
   \end{multline}
    Здесь приняты следующие обозначения:
    $$
    m_t = {\sf M}_{\Delta_y}^N Y_t\,;\  Y_t^0 = Y_t - m_t\,;\  K_t = {\sf M}_{\Delta_y}^N Y_t^0 Y_t^{0\mathrm{T}}\,;
    $$
    $$
    K\left(t_1, t_2\right) = {\sf M}_{\Delta_y}^N Y_{t_1}^{0} Y_{t_2}^{0\mathrm{T}}\,;
    $$
    $$
    a_1 = a_1 \left(m_t, K_t, t\right) = {\sf M}_{\Delta_y}^N a \left(Y_t, t\right)\,;
    $$

    \vspace*{-12pt}

    \noindent
    \begin{multline*}
    a_2 = a_2 \left(m_t, K_t, t\right) = a_{21} \left(m_t, K_t, t\right)+ {}\\
    {}+a_{21} \left(m_t, K_t, t\right)^{\mathrm{T}} +a_{22}\left(m_t, K_t, t\right)\,;
    \end{multline*}

    \vspace*{-9pt}

    \noindent
    $$
    a_{21} = a_{21}\left(m_t, K_t, t\right)=  {\sf M}_{\Delta_y}^N a\left(Y_t, t\right) Y_{t}^{0\mathrm{T}}\,;
    $$
    $$
    a_{22} = a_{22}\left(m_t, K_t, t\right)= {\sf M}_{\Delta_y}^N \si \left(Y_t, t\right)\,;
    $$

    \vspace*{-12pt}

    \noindent
    \begin{multline*}
    \si \left(Y_t, t\right) = b\left(Y_t, t\right) \nu_0(t) b\left(Y_t, t\right)^{\mathrm{T}} +{}\\
    {}+\iii_{R_0^q}
    c \left(Y_t, t, v\right) c\left(Y_t, t,v\right)^{\mathrm{T}} \nu_P (t, dv)\,, %\label{e3.8-s}
    \end{multline*}
где ${\sf M}_{\Delta_y}^N$~--- символ вычисления математического ожидания для нормальных распределений~(\ref{e3.4-s}).
    \end{itemize}

Для стационарных СтС нормальные стационарные СтП~--- если они
существуют, то  $m_t \hm=\bar m$, $ K_t\;=$\linebreak\vspace*{-12pt}

\columnbreak

\noindent
$=\;\bar K$, $K(t_1, t_2) \hm=
k(\tau)$ $(\tau \hm= t_1-t_2)$,--- определяются уравнениями~[9--11]:
    \begin{equation}
    a_1 (\bar m, \bar K) =0\,;\enskip a_2 (\bar m, \bar K)=0\,;
    \label{e3.9-s}
    \end{equation}

    \vspace*{-14pt}

    \noindent
    \begin{multline}
    \dot k_\tau (\tau) = a_{21} \left(\bar m, \bar K\right)\bar K^{-1} k(\tau)\,,\enskip
    k(0) =\bar K \\
     (\forall \tau >0)\,, \enskip
    k(\tau) = k(-\tau)^{\mathrm{T}} \enskip (\forall\tau <0)\,.\label{e3.10-s}
    \end{multline}
При этом необходимо, чтобы матрица  $a_{21} (\bar m, \bar K)\hm=\bar a_{21}$ была асимптотически устойчивой.

Уравнения МНА в~случае СтС~(\ref{e3.3-s}) переходят в~уравнения МСЛ~[9--11], если принять
    \begin{gather}
    a\left(Y_t,t\right) = a_1 \left(m_t, K_t\right) + k_1^a \left(m_t, K_t\right) Y_t^0\,;\notag
   \\[1pt]
    b\left(Y_t,t\right) = b_0 (t)\,,\enskip \si(Y_t, t)= b_0(t) \nu(t) b_0(t)^{\mathrm{T}} = \si_0(t)\,;\notag   \\[1pt]
    k_1^a (m_t, K_t, t) =\left[ \left(\fr{\prt}{\prt m_t} \right)a_0 \left(m_t, K_t, t\right)^{\mathrm{T}}\right]^{\mathrm{T}}\,;\label{e3.11-s}
    \\
    \dot m_t = a_1 \left(m_t, K_t, t\right)\, ,\enskip m_0 = m\left(t_0\right)\,;\label{e3.12-s}
    \end{gather}

    \vspace*{-12pt}

    \noindent
    \begin{multline}
   \dot K_t = k_1^a \left(m_t, K_t, t\right) K_t + K_t k_1^a \left(m_t, K_t, t\right)^{\mathrm{T}} +\si_0(t)\,;\\
   \hspace*{-3mm}K_0 = K(t_0)\,;\label{e3.13-s}
   \end{multline}

\vspace*{-12pt}

    \noindent
    \begin{multline}
   \fr{\prt K(t_1, t_2)}{\prt t_2} = K\left(t_1, t_2\right) K_{t_2} k_1^a \left(m_{t_2}, K_{t_2}, t_2\right)^{\mathrm{T}}\,,\\
   K\left(t_1, t_2\right) = K_{t_1}\,.\label{e3.14-s}
   \end{multline}

Для стационарных СтС~(\ref{e3.3-s}) при условии асимптотической устойчивости матрицы $k_1^a (\bar m, \bar K)$ в~основе МСЛ лежат уравнения~(\ref{e3.9-s}) и~(\ref{e3.10-s}), записанные в~виде:
    \begin{gather}
    a_1 \left(\bar m, \bar K\right) =0\,; \label{e3.15-s}\\
   k_1^a \left(\bar m, \bar K\right) \bar K + \bar K k_1^a \left(\bar m, \bar K\right)^{\mathrm{T}} +\bar \si_0 =0\,;\label{e3.16-s}
   \end{gather}

   \vspace*{-12pt}

   \noindent
   \begin{multline}
   \dot k_\tau (\tau) = k_1^a \left(\bar m, \bar K\right)k(\tau)\,,\enskip k(0) =\bar K \\ (\forall \tau >0),\enskip k(\tau) = k (-\tau)^{\mathrm{T}} \enskip (\forall \tau <0)\,.\label{e3.17-s}
   \end{multline}

Уравнения~(\ref{e3.4-s})--(\ref{e3.7-s}) лежат в~основе МНА для СтС~(\ref{e3.1-s}), а~урав\-не\-ния~(\ref{e3.11-s})--(\ref{e3.14-s})~--- в~основе МСЛ для СтС~(\ref{e3.3-s}). Для определения стационарных СтП согласно МНА служат соотношения~(\ref{e3.9-s}) и~(\ref{e3.10-s}), а~МСЛ~--- (\ref{e3.15-s})--(\ref{e3.17-s}).

Теперь рассмотрим дискретную СтС с~СТН, описываемую уравнениями
вида~\cite{4-s, 10-s}:
    \begin{equation}
    Y_{k+1} = a_k \left(Y_k\right) + b_k \left(Y_k\right) V_k^d \enskip (k=1,2,\ldots)\,.\label{e3.18-s}
    \end{equation}
Здесь $Y_k$~--- $(p\times 1)$-мер\-ный вектор состояния, $Y_k\hm\in \Delta_y$ ($\Delta_y$~--- многообразие состояний); функции $a_k (Y_k)$ и~$b_k (Y_k)$ имеют размерности $(p\times 1)$ и~$(p\times m)$ соответственно; через $V_k^d$ обозначен векторный дискретный шум, обладающий интенсивностью~$\nu_k^d$.
В~случае аддитивного шума, когда $b_k (Y_k) \hm= b_{0k}$, уравнение~(\ref{e3.18-s}) примет вид:
    \begin{equation}
    Y_{k+1} = a_k Y_k + b_{0k} V_k^d\,.\label{e3.19-s}
    \end{equation}

В~основе МНА лежат следующие соотношения и~уравнения~\cite{4-s, 10-s}:
   \begin{gather*}
        g_{1k}^N (\la) = \exp \left( i\la m_k -\fr {1}{2}\, \la^{\mathrm{T}} K_k \la\right)\,;\\
     g_{k_1k_2}^N = \exp\left( i\bar \la^{\mathrm{T}} \bar m_2 - \fr{1}{ 2}\, \bar \la^{\mathrm{T}} \bar K_2 \bar \la\right)\,;
    %     \label{e3.20-s}
     \\
    m_{k+1}= a_{1k}= {\sf M}_{\Delta_y}^N a_k\,,\enskip m_1 = {\sf M}_{\Delta_y}^NY_1\,; %\label{e3.21-s}
    \end{gather*}

    \vspace*{-10pt}

    \noindent
    \begin{multline*}
    K_{k+1} = a_{2k} ={\sf M}_{\Delta_y}^N \lk a_k a_k^{\mathrm{T}}\rk -\left[ {\sf M}_{\Delta_y}^Na_k\right] \left[ {\sf M}_{\Delta_y}^Na_k^{\mathrm{T}}\right]+{}\\
    {}+
     {\sf M}_{\Delta_y}^N\! \left[ b_k \nu_k^d b_k^{\mathrm{T}}\right], \enskip
         K_1 ={\sf M}_{\Delta_y}^N Y_1^0 Y_1^{0\mathrm{T}}\,;
\end{multline*}

\vspace*{-12pt}

    \noindent
    \begin{multline*}
%\label{e3.22-s}
    K(l,h) = a_{3k} = {\sf M}_{\Delta_y}^N Y_l^0 a_h (Y_h)^{\mathrm{T}}\,,
    \\[7pt]
     K(l,l) = K_l\enskip (l<h)\,,\enskip
    K(l,h) = K(h,l)^{\mathrm{T}} \enskip (l>h)\,.
    %    \label{e3.23-s}
    \end{multline*}

В основе МСЛ для~(\ref{e3.19-s}) после статистической линеаризации функции  $a_k(Y_k)$ согласно
    \begin{equation*}
    a_k\left(Y_k\right) = a_{0k} \left(m_k, K_k\right) + k_{1k}^a \left(m_k, K_k\right) Y_k^0
%    \label{e3.24-s}
    \end{equation*}
будут лежать уравнения~\cite{4-s, 10-s}:
   \begin{gather*}
   m_{k+1} = a_{0k}\,,\enskip m(1) = m_1\,; %\label{e3.25-s}
   \\
    K_{k+1} = k_{1k}^a K_k (k_{1k}^a)^{\mathrm{T}} + b_{0k} \nu_k^d b_{0k}^{\mathrm{T}}\,,\  K(1)= K_1\,; %\label{e3.26-s}
\end{gather*}

\vspace*{-12pt}

\noindent
\begin{multline*}
    K(l,h+1) = K(l,h) (k_{1h}^a)^{\mathrm{T}}\,,
\\
    K(l,l) = K_l \enskip  (l<h)\,,\enskip
     K(l,h) = K(h,l)^{\mathrm{T}}  \enskip (l>h)\,.
    %     \label{e3.27-s}
    \end{multline*}

Для определения стационарных СтП согласно МНА и~МСЛ с~характеристиками
    \begin{equation*}
    m_k = \bar m\,; \enskip K_k =\bar K\,; \enskip K(l,h) = \bar k(r)\enskip (r=h-l) %\label{e3.28-s}
    \end{equation*}
используются уравнения:
    \begin{align*}
    \bar m &= a_{1k} (\bar m,\bar K)\,; %\label{e3.29-s}
    \\[2pt]
    \bar K &= a_{2k} (\bar m, \bar K)\,;%\label{e3.30-s}
    \\[2pt]
    \bar K &= k_1^a \bar K (k_1^a)^{\mathrm{T}} + b_0 \nu_k^d b_0^{\mathrm{T}}\,; %\label{e3.31-s}
    \\[2pt]
    \bar k(r+1)&= k(r) (k_1^a)^{\mathrm{T}}\,,\enskip \bar k(0)=\bar K\,. %\label{e3.32-s}
        \end{align*}

Как следует из уравнений МНА, необходимо уметь вычислять следующие интегралы:
    \begin{align}
    I_{0k}^a &= I_{0k}^a\left (m_k, K_k\right) = {\sf M}_{\Delta_y}^N a_k \left(Y_k\right)\,;\label{e3.33-s}\\[2pt]
   I_{1k}^a &= I_{1k}^a \left(m_k, K_k\right) = {\sf M}_{\Delta_y}^N a_k \left(Y_k\right) Y_k^{0\mathrm{T}}\,;\label{e3.34-s}
\\
   I_{0k}^\si &= I_{0k}^\si \left(m_k, K_k\right) ={\sf M}_{\Delta_y}^N \si \left(Y_k\right)\notag\\[2pt]
   &\hspace*{30mm}\left(\si \left(Y_k\right) = b_k \nu_k^d b_k^{\mathrm{T}}\right)\,.\notag %\label{e3.35-s}
   \end{align}

Для МСЛ достаточно вычислить интеграл~(\ref{e3.33-s}), причем интеграл~(\ref{e3.34-s}) вычисляется по формуле:
    \begin{equation*}
    k_{1k}^a = k_{1k}^a (m_k, K_k) =\left[ \fr{\prt}{\prt m_k}\, I_{0k}^a \left(m_k, K_k\right)^{\mathrm{T}}\right]^{\mathrm{T}}\,. %\label{e3.36-s}
    \end{equation*}

  \section{Алгоритмическое обеспечение аналитического и~статистического моделирования}

  \vspace*{-2pt}

Как следует из~(\ref{e3.15-s}), для МНА необходимо уметь вычислять следующие интегралы:

\noindent
    \begin{align}
    I_0^a &= I_0^a \left(m_t, K_t, t\right) = a_1 \left(m_t, K_t, t\right)={}\notag\\
    &\hspace*{40mm}{}={\sf M}_{\Delta_y}^N a\left(Y_t, t\right)\,; \label{e4.1-s}
\\
   I_1^a &= I_1^a \left(m_t, K_t, t\right)= a_{21}\left(m_t, K_t, t\right)= {}\notag\\
&  \hspace*{35mm}{}=  {\sf M}_{\Delta_y}^N a\left(Y_t , t\right) Y_t^{0\mathrm{T}}\,;\notag %\label{e4.2-s}
   \\
    I_0^{\bar \si} &= I_0^{\bar \si} \left(m_t, K_t, t\right) = a_{22}\left(m_t, K_t, t\right) ={\sf M}_N \bar \si \left(Y_t, t\right)\,.\notag %\label{e4.3-s}
    \end{align}

Для МСЛ достаточно вычислить интеграл~(\ref{e4.1-s}), причем интеграл  $I_1^a$ вычисляется по формуле~[9--11]:

\noindent
    \begin{equation*}
k_1^a = k_1^a \left(m_t, K_t, t\right)=
    \left[ \!\left( \fr{\prt}{\prt m_t}\right) \!I_0^a \left(m_t, K_t, t\right)^{\mathrm{T}}\right]^{\mathrm{T}}\,. %\label{e4.4-s}
    \end{equation*}

При аналитическом моделировании для элементарных одномерных ТН скалярного аргумента, а~также их суперпозиции составлены таблицы формул (см., например,~[9--11]). В~\cite{11-s} приведены также таблицы формул для двух-, трех-, четырех- и~ $n$-мер\-ных аргументов. Соответствующие формулы для круговых ТН даны в~\cite{13-s, 14-s}.

Важно иметь в~виду, что уравнения МНА (МСЛ) содержат интегралы $I_0^a$, $I_1^a$ и~$I_0^\si$ в~виде соответ\-ствующих коэффициентов. Поэтому процедура вычис\-ле\-ния интегралов должна быть согласована с~методом чис\-лен\-но\-го решения обыкновенных дифференциальных уравнений для~$m_t$, $K_t$ и~$K(t_1, t_2)$. Эти коэффициенты допускают дифференцирование по~$m_t$ и~$K_t$, так как под интегралом стоит сглаживающая нормальная плотность.

В~\cite{12-s} изложены алгоритмы аналитического и~статистического
моделирования распределений (в том числе нормальных) в~нелинейных
СтС на многообразиях. Алгоритмы аналитического, статистического
моделирования для СтС с~СТН, а~также смешанные алгоритмы различной
степени точ\-ности относительно шага интегрирования также представлены в~\cite{12-s}.

  \vspace*{-6pt}


\section{Примеры}

  \vspace*{-2pt}

\textbf{1.}\
Для СТН вида

\noindent
   \begin{align*}
    \vrp_1^{\mathrm{СТН}} (Y) &=\sin \left(pY^2 + 2 q Y +r\right)\,,\\
    \vrp_2^{\mathrm{СТН}} (Y) &=\cos \left(pY^2 + 2 q Y +r\right)
    \end{align*}

    \vspace*{-10pt}

    \pagebreak


    \noindent
интегралы~(\ref{e4.1-s}) с~учетом~\cite{7-s, 8-s} имеют следующий вид:

\noindent
    $$
    I_{01}  (m,D) = \fr{1}{\sqrt{2\pi D}} A_1\,, \enskip I_{02}  (m,D) = \fr{1}{\sqrt{2\pi D}} A_2\,,
    $$
где

\noindent
    \begin{multline*}
    A_1 = {}\\
    \hspace*{-1pt}{}=\fr{\sqrt{\pi}}{\root{4}\of{a^2+p^2}} \exp \left[ \fr{a(b^2 - ac) - (a q^2 - 2 bpq + cp^2)}{a^2 + p^2}\right] \times{}\hspace*{-0.74269pt}\\
{}\times\;\sin \left[ \fr{1}{2}\, \mathrm{actg}\, \fr{p}{a} -\fr{p(q^2-pr)-(b^2 p - 2abq +a^2 r)}{a^2 + p^2} \right];\hspace*{-6.21078pt}
\end{multline*}

\vspace*{-12pt}

\noindent
\begin{multline*}
   A_2 ={}\\
    \hspace*{-1.53435pt}{}=\fr{\sqrt{\pi}}{\root{4}\of{a^2+p^2}} \exp \left[ \fr{a(b^2 - ac) - (a q^2 - 2 bpq + cq^2)}{a^2 + p^2}\right] \times{}\\
{}\times \cos \left[ \fr{1}{2}\,\mathrm{actg}\, \fr{p}{a} -\fr{p(q^2-pr)-(b^2 p - 2abq +a^2 r)}{a^2 + p^2} \right]\\
   \left( a=\fr{1}{2D}\,,\enskip b= -\fr{m}{2D}\,,\enskip c=\fr{m^2}{2D}\right)\,.
   \end{multline*}


\textbf{2.}\
Для некоторых СТН при $m\hm=0$, $D\hm\ne 0$ на основе~\cite{7-s, 8-s} получены следующие выражения для интегралов~(\ref{e4.1-s}):
   \begin{enumerate}
   \item[(а)] $
    \vrp^{\mathrm{СТН}} (Y) = Y^{-1} \sin aY\enskip (a>0)\,;
    $

    \vspace*{-12pt}

   $$
    \hspace*{-6mm}I^\vrp_0 (0,D) =\fr{a}{\sqrt{2} \beta} \sss_{j=0}^\infty \fr{(-1)^j}{j(2j+1)} \left(\fr{a}{2\beta}\right)^{2j}\!
    \left(\beta^2 = \fr{1}{2D}\right);
  $$

    \item[(б)] $
    \vrp^{\mathrm{СТН}} (Y) = Y^2 \exp \left(-a Y^{-2}\right)\,; \enskip \mu=\fr{1}{2D}\,;$

    \vspace*{-12pt}
    $$
    \hspace*{-4mm}I^\vrp_0 (0,D) = \fr{1}{2\sqrt{2D}}\, \mu^{-3/2} \left(1+2\sqrt{a\mu}\right) \exp \left(-2\sqrt{a\mu}\right);
    $$

   \item[(в)]
     $\vrp^{\mathrm{СТН}} (Y) = Y^{-2} \exp \left(-a Y^{-2}\right)\,; \enskip \mu=\fr{1}{2D}\,;$

     \vspace*{-12pt}

    $$
    I^\vrp_0 (0,D) = \fr{1}{\sqrt{2aD}} \exp \left(-2\sqrt{a\mu}\right)\,;
    $$

    \item[(г)]
     $ \vrp^{\mathrm{СТН}} (Y) = \mathrm{sh}^{2}\,a Y \enskip (a>0)\,;$

     \vspace*{-12pt}
    $$
    \hspace*{-2mm}I^\vrp_0 (0,D) = \fr{1}{2\sqrt{2\mu D}} \left(\exp \fr{a^2}{\mu}-1\right)\,;
    \enskip \mu=\fr{1}{2D}\,;
    $$
\item[(д)]$\vrp^{\mathrm{СТН}} (Y) = \mathrm{ch}^{2}\, a Y \enskip (a>0)\,;$

\vspace*{-12pt}

         $$
         I^\vrp_0 (0,D) = \fr{1}{2\sqrt{2\mu D}} \left(\exp \fr{a^2}{\mu}+1\right)\,; \enskip \mu=\fr{1}{2D}\,.
         $$
          \end{enumerate}

\textbf{3.}\
Рассмотрим скалярную дифференциальную СтС~(\ref{e3.3-s}) при  $a(Y_t)\hm = \vrp^{\mathrm{СТН}} (Y_t)$. Уравнения МСЛ~(\ref{e3.12-s}), (\ref{e3.13-s}) имеют следующий вид:

\noindent
   \begin{gather*}
    \dot m = a_1 (m,D)\,,\enskip m(t_0) = m_0\,;\\
    \dot D = a_2 (m,D)\,,\enskip D(t_0)=D_0\,.
    \end{gather*}
Воспользуемся известным результатом из теории вычисления интегралов путем разложения по многочленам Эрмита~\cite{7-s, 8-s}:

\noindent
    $$
    \iin e^{-\xi^2} \vrp(\xi)\, d\xi = \sss_{j=1}^n w_j \vrp (\xi_j) + R_n\,.
    $$
Здесь $H_n = H_n (\xi)$~--- многочлен Эрмита; $\xi_j$~--- $j$-й нуль~$H_n$; $w_j$ и $R_n$~--- весовые коэффициенты и~остаточный член, вычисляемые по формулам:

\noindent
    \begin{align*}
    w_j &= \fr{2^{n-1} n! \sqrt{\pi}}{n^2 \lk H_{n-1} (\xi_j)\rk^2}\,;\\
    R_n &= \fr{n! \sqrt{\pi}}{2^n (2n)!}\, \vrp^{2n} (\xi)\enskip (-\infty<\xi<\infty)\,.
    \end{align*}
Опуская промежуточные вычисления, приведем окончательные формулы для вычисления правых частей уравнений МСЛ:
    \begin{gather*}
    a_1(m,D) =\sss_{j=1}^n w_j \bar a \left( m, D;\xi_j\right)\,;
  \\
    \bar a \left( m, D;\xi_j\right)= \fr{a(m+\xi_j \sqrt{2D})}{\sqrt{2\pi D}}\,;
    \\
    a_2(m,D)=\sss_{j=1}^n w_j \bar\si \left( m, D;\xi_j\right)\,;
    \end{gather*}

    \vspace*{-12pt}

    \noindent
    \begin{multline*}
    \bar\si \left( m, D;\xi_j\right)={}\\
    {}=\fr{2a(m+\xi\sqrt{2D})(m+\xi_j\sqrt{2D}) +\si(m+\xi_j\sqrt{2D})}{\sqrt{2\pi D}}.
\end{multline*}
Ограничиваясь небольшом числом многочленов Эрмита $H_n$, получаем алгоритмы МСЛ для различных СТН, приведенных в~примерах 1 и~2.

\vspace*{-6pt}

\section{Заключение}

Рассмотрены  дифференциальные и~разностные СтС (в~том числе и~на многообразиях) с~винеровскими и~пуассоновскими шумами и~с~СТН. Такие модели описывают поведение многих современных нано- и~квантовооптических  технических средств информатики.
Приводятся уравнения МНА и~МСЛ\linebreak
 для моделирования нестационарных и~стационарных нормальных процессов.
Рассматриваются методы вычисления типовых интегралов для одно- и~многомерных  СТН скалярного и~векторного\linebreak
 аргумен\-та, получающихся из суперпозиции элементарных ТН. Обсуждается алгоритмическое обеспечение аналитического и~статистического мо\-делирования.
Особый интерес представляет раз-\linebreak\vspace*{-12pt}

 \pagebreak

 \noindent
витие алгоритмов для ТН на основе специальных функций. Приводятся тестовые примеры.

Результаты допускают обобщение на случай интегродифференциальных и~операторных СтС с~СТН,  в~том числе с~автокоррелированными шумами.


{\small\frenchspacing
 {%\baselineskip=10.8pt
 \addcontentsline{toc}{section}{References}
 \begin{thebibliography}{99}

\bibitem{1-s}
\Au{Синицын И.\,Н., Синицын В.\,И. }
Аналитическое моделирование нормальных процессов в~стохастических системах со сложными нелинейностями~// Информатика и~её применения, 2014. Т.~8. Вып.~3. С.~2--4.

\bibitem{2-s}
\Au{Синицын И.\,Н., Синицын В.\,И., Сергеев~И.\,В., Белоусов~В.\,В., Шоргин~В.\,С.}
Математическое обеспечение аналитического моделирования стохастических систем со сложными нелинейностями~// Системы и~средства информатики, 2014. Т.~24. №\,3. С.~4--29.

\bibitem{3-s}
\Au{Синицын И.\,Н., Синицын В.\,И., Корепанов Э.\,Р.}
Моделирование нормальных процессов в~стохастических системах со сложными иррациональными нелинейностями~// Информатика и~её применения, 2015. Т.~9. Вып.~1. С.~2--8.

\bibitem{4-s}
\Au{Синицын И.\,Н., Синицын~В.\,И., Сергеев~И.\,В.,  Корепанов~Э.\,Р.,
Белоусов~В.\,В., Шоргин~В.\,С.}
Математическое обеспечение моделирования нормальных процессов
в стохастических системах со сложными иррациональными нелинейностями~// Системы и~ средства информатики, 2015.
Т.~25. №\,2. С.~3--19.

\bibitem{5-s}
\Au{Попов Б.\,А., Теслер~Г.\,С. }
Вычисление функций на ЭВМ: Справочник.~--- Киев: Наукова Думка, 1984. 599~с.

\bibitem{6-s}
\Au{Кудрявцев Л.\,Д., Соломенцев~Е.\,Д. }
Трансцендентная функция~// Математическая энциклопедия~/ Гл. ред. И.\,М.~Виноградов.~--- М.: Советская энциклопедия, 1984. C.~425.

\bibitem{7-s}
\Au{Градштейн И.\,С., Рыжик~И.\,М. }
Таблицы интегралов, сумм, рядов и~произведений.~--- М.: ГИФМЛ, 1963. 1100~с.

\bibitem{8-s}
Справочник по специальным функциям~/ Под ред. М.~Абрамовича, И.~Стигана.~--- М.: Наука, 1979. 832~с.

\bibitem{9-s}
\Au{Пугачев В.\,С., Синицын И.\,Н.}
Стохастические дифференциальные системы. Анализ и~фильтрация.~--- М.:
Наука,  1990.  632~с. [\Au{Pugachev V.\,S., Sinitsyn~I.\,N.}. Stochastic differential systems.
Analysis and filtering.~--- Chichester, New York, NY, USA: Jonh Wiley, 1987.
549~p.]

\bibitem{10-s}
\Au{Пугачев В.\,С., Синицын И.\,Н.}
Теория стохастических систем.~--- М.: Логос, 2000; 2004. 1000~с.
%[Англ. пер. \Au{Pugachev V.\,S., Sinitsyn~I.\,N.}. Stochastic systems. Theory and  %applications.~---
%Singapore: World Scientific, 2001. 908~p.].


\bibitem{11-s}
\Au{Синицын И.\,Н.,  Синицын В.\,И. }
Лекции по нормальной и~эллипсоидальной аппроксимации распределений в~стохастических системах.~--- М.: ТОРУС ПРЕСС, 2013. 488~с.


\bibitem{13-s} %12
\Au{Синицын И.\,Н. }
Математическое обеспечение для анализа нелинейных многоканальных круговых стохастических систем, основанное на параметризации распределений~// Информатика и~ её применения, 2012. Т.~6. Вып.~1. С.~12--18.


\bibitem{14-s} %13
\Au{Синицын И.\,Н.,  Корепанов Э.\,Р., Белоусов~В.\,В., Конашенкова~Т.\,Д.}
Развитие математического обеспечения для анализа нелинейных многоканальных круговых стохастических систем~// Системы и~средства информатики, 2012. Вып.~22. №\,1. С.~29--40.

\bibitem{12-s} %14
\Au{Синицын И.\,Н. }
Параметрическое статистическое и~аналитическое моделирование распределений в~ нелинейных стохастических системах на многообразиях~// Информатика и~её применения, 2013. Т.~7. Вып.~2. С.~4--16.

 \end{thebibliography}

 }
 }

\end{multicols}

\vspace*{-3pt}

\hfill{\small\textit{Поступила в~редакцию 25.02.15}}

%\newpage

\vspace*{12pt}

\hrule

\vspace*{2pt}

\hrule

%\vspace*{12pt}

\def\tit{MODELING OF~NORMAL PROCESSES IN~STOCHASTIC SYSTEMS WITH~COMPLEX TRANSCENDENTAL  NONLINEARITIES}

\def\titkol{Modeling of~normal processes in~stochastic systems with~complex transcendental  nonlinearities}

\def\aut{I.\,N.~Sinitsyn, V.\,I.~Sinitsyn, and E.\,R.~Korepanov}

\def\autkol{I.\,N.~Sinitsyn, V.\,I.~Sinitsyn, and E.\,R.~Korepanov}

\titel{\tit}{\aut}{\autkol}{\titkol}

\index{Sinitsyn I.\,N.}
\index{Sinitsyn V.\,I.}
\index{Korepanov E.\,R.}

\vspace*{-9pt}


\noindent
Institute of Informatics Problems, Federal Research
Center ``Computer Science and Control'' of the
Russian Academy of Sciences, 44-2 Vavilov Str., Moscow 119333,
Russian Federation


\def\leftfootline{\small{\textbf{\thepage}
\hfill INFORMATIKA I EE PRIMENENIYA~--- INFORMATICS AND
APPLICATIONS\ \ \ 2015\ \ \ volume~9\ \ \ issue\ 2}
}%
 \def\rightfootline{\small{INFORMATIKA I EE PRIMENENIYA~---
INFORMATICS AND APPLICATIONS\ \ \ 2015\ \ \ volume~9\ \ \ issue\ 2
\hfill \textbf{\thepage}}}

\vspace*{3pt}


\Abste{Development of methods for analytical and statistical modeling for discrete and continuous stochastic systems (StS) with Wiener and Poisson noises and with complex transcendental nonlinearities (CTN) is given. Typical representation of scalar and vector CTN is considered. Equations for the normal approximation method (NAM) and the method of statistical linearization (MSL) are deduced. Also,
NAM and MSL for StS with CTN algorithms are given. Test examples are presented. Some generalizations are given.}

\KWE{analytical and statistical modeling; complex transcendental nonlinearities (CTN); Hermite polynomials; method of statistical linearization (MSL); normal approximation method (NAM); stochastic systems (StS)}


\DOI{10.14357/19922264150203}

\Ack
\noindent
The research was supported by the Russian Foundation for Basic Research
 (project 15-07-02244).



%\vspace*{3pt}

  \begin{multicols}{2}

\renewcommand{\bibname}{\protect\rmfamily References}
%\renewcommand{\bibname}{\large\protect\rm References}



{\small\frenchspacing
 {%\baselineskip=10.8pt
 \addcontentsline{toc}{section}{References}
 \begin{thebibliography}{99}

\bibitem{1-s-1}
\Aue{Sinitsyn, I.\,N., and V.\,I.~Sinitsyn}. 2014.
 Analiticheskoe modelirovanie normal'nykh protsessov v~sto\-kha\-sti\-che\-skikh sistemakh so slozhnymi nelineynostyami [Analytical modeling of normal
  processes in stochastic systems with complex nonlinearities].
  \textit{Informatika i~ee Primeneniya}~--- \textit{Inform. Appl.}  8(3):2--4.

\bibitem{2-s-1}
\Aue{Sinitsyn, I.\,N., V.\,I.~Sinitsyn, I.\,V.~Sergeev, V.\,V.~Belousov, and V.\,S.~Shorgin}. 2014.
Matematicheskoe obespechenie analiticheskogo modelirovaniya stokhasticheskikh sistem so slozhnymi nelineynostyami [Mathematical software for
analytical modeling of stochastic systems with complex nonlinearities].
\textit{Sistemy i~Sredstva Informatiki}~--- \textit{Systems and Means of
Informatics}  24(3):4--29.


\bibitem{3-s-1}
\Aue{Sinitsyn, I.\,N., V.\,I.~Sinitsyn, and E.\,R.~Korepanov}.  2015.
Modelirovanie normal'nykh protsessov v~sto\-kha\-sti\-che\-skikh sistemakh so slozhnymi irratsional'nymi ne\-li\-ney\-no\-stya\-mi [Modeling of normal processes in stochastic systems with complex irrational nonlinearities].   \textit{Informatika i~ee Primeneniya}~--- \textit{Inform. Appl.} 9(1):2--8.

\bibitem{4-s-1}
\Aue{Sinitsyn, I.\,N., V.\,I.~Sinitsyn, I.\,V.~Sergeev, E.\,R.~Korepanov, V.\,V.~Belousov, and V.\,S.~Shorgin}. 2015.
Ma\-te\-ma\-ti\-che\-skoe obespechenie modelirovaniya nor\-mal'\-nykh protsessov v~stokhasticheskikh sistemakh so slozhnymi irratsional'nymi nelineynostyami [Mathematical software for modeling of normal processes in stochsatic systems with complex irrational nonlinearities]. \textit{Sistemy i~Sredstva Informatiki}~--- \textit{Systems and Means of
Informatics} 25(2):3--19.


\bibitem{5-s-1}
\Aue{Popov, B.\,A., and G.\,S.~Tesler}.  1984.
\textit{Vychislenie funktsiy na EVM}: Spravochnik [Computing of functions: Handbook]. Kiev: Naukova Dumka.  599~р.


\bibitem{6-s-1}
\Aue{Kudryavtsev, L.\,D., and E.\,D.~Solomentsev}. 1984.
Transtsepdentnaya funktsiya [Transcendental function].   \textit{Matematicheskaya entsiklopediya} [Mathematical encyclopedia].
Ed.\ I.\,M.~Vinogradov. Moscow: Sovetskaya Entsiklopediya. 425.


\bibitem{7-s-1}
\Aue{Gradshteyn, I.\,S., and I.\,M.~Ryzhik}.  1963.
\textit{Tablitsy integralov, summ, ryadov i~proizvedeniy}.  [Tables of integrals, sums, series, and products]. Moscow: GIFML.  1100~p.

\bibitem{8-s-1}
Abramovich,~M., and I.~Stigan, eds. 1979.
\textit{Spravochnik po spetsial'nym funktsiyam} [Handbook on special functions].  Moscow:  Nauka.  832~p.


\bibitem{9-s-1}
\Aue{Pugachev, V.\,S., and  I.\,N.~Sinitsyn}.  1987.
\textit{Stochastic differential systems. Analysis and filtering.}   Chichester, New York, NY: Jonh Wiley. 549~p.

\bibitem{10-s-1}
\Aue{Pugachev, V.\,S., and I.\,N.~Sinitsyn}. 2001.
\textit{Stochastic systems. Theory and  applications.} Singapore: Worls Scientific. 908~p.

\bibitem{11-s-1}
\Aue{Sinitsyn, I.\,N., and  V.\,I.~Sinitsyn}.  2013.
Lektsii po normal'noy i~ellipsoidal'noy approksimatsii raspredeleniy v~stokhasticheskikh sistemakh [Lectures on normal and ellipsoidal
approximation of distributions in stochastic systems]. Moscow: TORUS PRESS. 488~p.





\bibitem{13-s-1} %12
\Aue{Sinitsyn, I.\,N.} 2012.
Matematicheskoe obespechenie dlya analiza nelineynykh mnogokanal'nykh krugovykh stokhasticheskikh sistem, osnovannoe na parametrizatsii raspredeleniy [Mathematical software for analysis of nonlinear multichannel circular stochastic systems based on parametrization of distributions].
\textit{Informatika i~ee Primeneniya}~---
 \textit{Inform. Appl.} 6(1):12--18.

\bibitem{14-s-1} %13
\Aue{Sinitsyn, I.\,N., E.\,R.~Korepanov, V.\,V.~Belousov, and T.\,D.~Konashenkova}. 2012.
Razvitie matematicheskogo obespecheniya dlya analiza nelineynykh mno\-go\-ka\-nal'\-nykh krugovykh stokhasticheskikh sistem [Development of mathematical software for analysis of nonlinear multichannel circular stochastic systems]. \textit{Sistemy i~Sredstva Informatiki}~--- \textit{Systems and Means of Informatics} 22(1):29--40.

\bibitem{12-s-1} %14
\Aue{Sinitsyn, I.\,N.}  2013.
Parametricheskoe statisticheskoe i~analiticheskoe modelirovanie raspredeleniy v~ne\-li\-ney\-nykh stokhasticheskikh sistemakh na mno\-go\-ob\-ra\-zi\-yakh
[Parametric statistical and analytical modeling of distributions in stochastic systems on manifolds]. \textit{Informatika i~ee Primeneniya}~---
 \textit{Inform. Appl.} 7(2):4--16.
\end{thebibliography}

 }
 }

\end{multicols}

\vspace*{-3pt}

\hfill{\small\textit{Received February 25, 2015}}

\vspace*{-9pt}

\Contr

\noindent
\textbf{Sinitsyn Igor N.} (b.\ 1940)~--- Doctor of Science in 
technology, professor, Honored scientist of RF, Head of Department, 
Institute of Informatics Problems, Federal Research Center ``Computer Science and Control'' of the Russian Academy of Sciences, 44-2 Vavilov Str., Moscow 119333, Russian Federation; sinitsin@dol.ru

\vspace*{3pt}


     \noindent
\textbf{Sinitsyn Vladimir I.} (b.\ 1968)~--- Doctor 
of Science in physics and mathematics, associate professor, 
Head of Department, Institute of Informatics Problems, Federal Research Center ``Computer Science and Control'' of the Russian Academy of Sciences, 44-2 Vavilov Str., Moscow 119333, Russian Federation; VSinitsyn@ipiran.ru

\vspace*{3pt}

    \noindent
\textbf{Korepanov Eduard R.} (b.\ 1966)~--- Candidate of Science (PhD) in technology, Head of Laboratory, Institute of Informatics Problems, Federal Research Center ``Computer Science and Control'' of the Russian Academy of Sciences, 44-2 Vavilov Str., Moscow 119333, Russian Federation; ekorepanov@ipiran.ru


\label{end\stat}


\renewcommand{\bibname}{\protect\rm Литература}  %3
\def\yhxt{\left({\hat X}_t,Y_t, t\right)}
\def\yxtt{\left(X_t,Y_t, t\right)}
\def\yutt{\left(Y_t, {\hat X}_t,t\right)}





\def\stat{sin+kor}

\def\tit{НОРМАЛЬНЫЕ УСЛОВНО-ОПТИМАЛЬНЫЕ
ФИЛЬТРЫ ПУГАЧЁВА ДЛЯ~ДИФФЕРЕНЦИАЛЬНЫХ СТОХАСТИЧЕСКИХ СИСТЕМ,
  ЛИНЕЙНЫХ ОТНОСИТЕЛЬНО СОСТОЯНИЯ$^*$} %\\[-7pt]}

\def\titkol{Нормальные условно-оптимальные
ФП %фильтры Пугачёва
для дифференциальных СтС, %стохастических систем,
  линейных относительно состояния}

\def\aut{И.\,Н.~Синицын$^1$,  Э.\,Р.~Корепанов$^2$}

\def\autkol{И.\,Н.~Синицын,  Э.\,Р.~Корепанов}

\titel{\tit}{\aut}{\autkol}{\titkol}

\index{Синицын И.\,Н.}
\index{Корепанов Э.\,Р.}

{\renewcommand{\thefootnote}{\fnsymbol{footnote}} \footnotetext[1]
{Работа выполнена при  поддержке РФФИ (проект 15-07-02244).}}


\renewcommand{\thefootnote}{\arabic{footnote}}
\footnotetext[1]{Институт проблем информатики Федерального исследовательского
центра <<Информатика и~управление>> Российской академии наук,
sinitsin@dol.ru}
\footnotetext[2]{Институт проблем информатики Федерального исследовательского
центра <<Информатика и~управление>> Российской академии наук,
ekorepanov@ipiran.ru}

\vspace*{-12pt}

\Abst{Рассматриваются вопросы аналитического синтеза нормальных услов\-но-оп\-ти\-маль\-ных фильт\-ров  Пугачёва (НФП) для обработки информации в~дифференциальных негауссовских стохастических системах (СтС), линейных относительно состояния (условия Лип\-це\-ра--Ши\-ря\-ева). Особое внимание уделено синтезу НФП для СтС при условиях Лип\-це\-ра--Ши\-ря\-ева на основе аппроксимации апостериорного распределения нормальным и квазилинейным НФП, основанным на статистической линеаризации нелинейных функций, зависящих от наблюдений. Для СтС высокой размерности  путем выбора структурных функций, отражающих аналитическую природу наблюдаемой системы, можно синтезировать НФП, прос\-ты\-ми в~компьютерной реализации и~для работы в режиме реального времени. Изложенные алгоритмы положены в основу модуля инструментального программного обеспечения <<StS-Filter>>. Даны тестовые примеры. Приводятся некоторые обобщения.}

\vspace*{-8pt}

\KW{метод нормальной аппроксимации (МНА) апостериорной плотности;
метод статистической линеаризации (МСЛ);
нормальный услов\-но-оп\-ти\-маль\-ный фильтр Пугачёва (НФП);
стохастическая система (СтС); дифференциальная СтС;
СтС, линейная относительно состояния;
условия Лип\-це\-ра--Ширяева; фильтр Лип\-це\-ра--Ши\-ря\-ева (ФЛШ)}

\vspace*{-6pt}

\DOI{10.14357/19922264150204}

\vspace*{-6pt}


\vskip 10pt plus 9pt minus 6pt

\thispagestyle{headings}

\begin{multicols}{2}

\label{st\stat}



\section{Введение}

\vspace*{-4pt}

Многие практические задачи обработки информации в~статистических научных исследованиях основаны на использовании теории фильтрации процессов в~СтС, линейных относительно состояния~[1--6]. Первые работы в~этом направлении для гауссовских систем выполнены Липцером и Ширяевым~[7], а~для негауссовских на основе субоптимальной фильтрации~--- Пугачёвым и Синицыным (см.\ обзор~\cite{1-sin}). В~\cite{2-sin, 3-sin} изучены вопросы синтеза и~устойчивости фильтров для линейных СтС с~аддитивными и~мультипликативными негауссовскими шумами.
Статья посвящена вопросам аналитического синтеза и~устойчивости НФП для нелинейных негауссовских СтС, линейных относительно состояния.

\vspace*{-14pt}

\section{Дифференциальные стохастические системы, линейные относительно состояния}

\vspace*{-4pt}

Рассмотрим нелинейную дифференциальную СтС~\cite{1-sin}:

\noindent
    \begin{alignat}{2}
    \dot X_t &=\vrp (X_t, Y_t, t) + \psi (X_t, Y_t, t) V\,, &\ X_{t_0} &= X_0\,;\label{e1-sin}\\
   \dot Y_t &=\vrp_1 (X_t, Y_t, t) + \psi_1 (X_t, Y_t, t) V\,, &\ Y_{t_0}& = Y_0\,,\label{e2-sin}
   \end{alignat}
заданную на многообразиях $\Delta \hm= \Delta^{x,y}$ и $\Delta^V$. Здесь~$X_t$ и $Y_t$~--- векторы состояния и наблюдения размерности~$n_x$ и~$n_y$; $V\hm= \dot W$, $W$~--- векторный процесс с независимыми приращениями, состояний из винеровской $W_0(t)$ и~пуассоновской частей:
    \begin{equation}
     \left.
     \begin{array}{rl}
     W&= W_0 (t) +\displaystyle \iii_{R_0^q} c(\rho) P^0 (t, d\rho)\,;\\[6pt]
     \nu^W &= \nu^{W_0} + \displaystyle \iii_{R_0^q} c(\rho) c(\rho)^{\mathrm{T}} \nu_P (t, \rho)\, d\rho\,,
     \end{array}
     \right\}
     \label{e3-sin}
     \end{equation}
где $c=c(\rho)$~--- векторная функция (той же размерности~$q$, что и~$W$) аргумента~$\rho$, а~интеграл при любом $t\hm\ge t_0$ представляет собой стохастический интеграл по центрированной пуассоновской мере  $P^0 (t, A)$, независимой от~$W_0$  и~име\-ющей независимые значения на попарно непересекающихся множествах;  $A$~--- борелевское множество пространства $R_0^q$ с выколотым началом~$0$; $\nu^W$, $\nu^{W_0}$ и~$\nu_P$~--- интенсивности $W$, $W_0$ и~$P^0$:
$\vrp\hm=\vrp (X_t, Y_t, t)$, $\psi\hm=\psi (X_t, Y_t, t)$, $\vrp_1\hm=\vrp_1 (X_t, Y_t, t)$ и~$\psi_1\hm=\psi_1 (X_t, Y_t, t)$~--- известные функции раз\-мер\-ности $(n_2\times 1)$, $(n_x\times n_v)$, $(n_y \times 1)$ и~$(n_y\times n_v)$, удовлетворяющие следующим условиям Лип\-це\-ра--Ши\-ря\-ева~[7]:
\begin{itemize}
\item  функции $\vrp$ и $\vrp_1$ линейны относительно состояния~$X_t$:
    \begin{align*}
    \vrp (X_t, Y_t, t) &= a_1 (Y_t, t) X_t + a_0 (Y_t, t)\,;\\
    \vrp_1 (X_t, Y_t, t) &= b_1 (Y_t, t) X_t + b_0 (Y_t, t)\,;
\end{align*}

\item функции $\psi$ и $\psi_1$  не зависят от состояния~$X_t$:
    $$
    \psi (X_t, Y_t, t) = \psi (Y_t, t) \,;\enskip \psi_1 (X_t, Y_t, t) = \psi_1  (Y_t, t)\,.
    $$
\end{itemize}

Предполагается, что уравнения СтС~(\ref{e1-sin}) и~(\ref{e2-sin}) понимаются в~смысле Ито и~имеют решение в среднем квадратическом (с.к.)~\cite{1-sin}.
Систему~(\ref{e1-sin})--(\ref{e2-sin}) будем называть гауссовской, если  $V\hm=\dot W_0$, а~$X_0$ и~$Y_0$~--- гауссовские.
Важный частный случай~(\ref{e1-sin}) и~(\ref{e2-sin}) составляют уравнения с~аддитивными шумами, когда
    \begin{equation*}
    \bar\psi\left(Y_t, t\right) = \psi_0 (t)\,;\enskip \bar\psi_1 \left(Y_t, t\right) = \psi_{10} (t)\,.
    %\label{e4-sin}
    \end{equation*}

\noindent
\textbf{Замечание~1.} Для случая, когда в уравнения~(\ref{e1-sin}), (\ref{e2-sin}) входят независимые белые шумы~$V_1$ и~$V_2$, следует принять
    \begin{equation}
    \left.
    \begin{array}{rlrl}
    V&= \lk V_1^{\mathrm{T}}\, V_2^{\mathrm{T}}\rk^{\mathrm{T}}\,,\enskip & \nu&=\begin{bmatrix}
    \nu_1&0\\
    0&\nu_2\end{bmatrix}\,;\\[12pt]
    \psi V &=\psi' V_1\,;\enskip  &\psi_1 V &= \psi_1' V_2\,.
    \end{array}
    \right\}
    \label{e5-sin}
    \end{equation}

\noindent
\textbf{Замечание~2.}
 Для случая, когда уравнения СтС линейны и содержат гауссовские и~негауссовские аддитивные и мультипликативные шумы в~уравнениях состояния  и~наблюдения, соответствующие уравнения приведены в~\cite{2-sin, 3-sin}.



\section{Дифференциальные фильтры Липцера--Ширяева}

Для гауссовской СтС~(\ref{e1-sin})--(\ref{e2-sin}) при условиях Лип\-це\-ра--Ши\-ря\-ева известны следующие точные уравнения нелинейной фильтрации по критерию минимума с.к.\ ошибки~\cite{1-sin, 7-sin}:
 \begin{multline}
 {\dot{\hat X}}_t= \left[a_1 \left(Y_t,t\right) \hat X_t + a_0 \left(Y_t,t\right)\right] + \left[R_tb_1 (Y_t,t)^{\mathrm{T}} +{}\right.\\
\left. {}+\left(\psi\nu_0\psi_1^{\mathrm{T}}\right)
    (Y_t,t)\right] \left(\psi_1\nu_0\psi_1^{\mathrm{T}}\right)^{-1} \left(Y_t,t\right)
\left\{ \dot Y_t -{}\right.\\
{}-\left.\left[ b_1\left(Y_t,t\right) \hat X_t +b_0\left(Y_t,t\right)\right] \right\},\enskip
    \hat X_{t_0} = \hat X_0;\label{e6-sin}
    \end{multline}

    \vspace*{-12pt}

    \noindent
    \begin{multline}
       \dot{R}_t =  a_1\left(Y_t,t\right) R_t + R_t a_1 \left(Y_t,t\right)^{\mathrm{T}} +\left(\psi\nu_0\psi^{\mathrm{T}}\right) \left(Y_t,t\right) -{}\\
       {}-
\left[ R_t b_1 \left(Y_t,t\right)^{\mathrm{T}}
+\left(\psi\nu_0\psi_1^{\mathrm{T}}\right) \left(Y_t,t\right)\right] \left(\psi_1\nu_0\psi_1^{\mathrm{T}}\right)^{-1} \times{}\\
{}\times \left(Y_t,t\right)
 \left[ b_1 \left(Y_t,t\right) R_t+ \left(\psi_1 \nu_0 \psi^{\mathrm{T}}\right)
    \left(Y_t,t\right)\right]\,,\\
     R_{t_0} = R_0\,,\label{e7-sin}
    \end{multline}
где $\hat{X}_t$ --- с.к.\ оценка; $R_t$~--- ковариационная матрица ошибки фильтрации ($X_t\hm- \hat{X}_t$).


Как и в случае линейной фильтрации при аддитивных шумах~\cite{1-sin}, уравнения дифференциального фильтра Лип\-це\-ра--Ширяева (ФЛШ)~\cite{6-sin, 7-sin} пред\-ставляют собой замкнутую систему
уравнений, определяющую  $\hat{X}_t$ и~$R_t$. Поэтому с.к.\  оптимальную
оценку~$\hat{X}$ вектора состояния системы~$X_t$ и его
апостериорную ковариационную матрицу~$R_t$, характеризующую
точность с.к.\ оптимальной оценки ~$\hat{X}_t$, можно вычислять по
мере получения результатов наблюдений совместным интегрированием
уравнений~(\ref{e6-sin}) и~(\ref{e7-sin}). Однако в~противоположность линейной фильтрации
для ФЛШ нельзя вычислить~$R_t$ заранее, до
получения результатов наблюдений, так как от последних зависят
коэффициенты уравнения~(\ref{e2-sin}). Поэтому ФЛШ в~данном случае должен выполнять интегрирование обоих уравнений~(\ref{e6-sin}) и~(\ref{e7-sin}). Это приводит к~существенному повышению
порядка оптимального фильтра. Если линейный фильтр  всегда
описывается уравнениями  порядка~$n_x$, то в~рассматриваемом более общем случае
с.к.\ оптимальный фильтр описывается уравнениями по\-рядка
$$
Q_{\mathrm{ЛШ}}=n_x+\fr{n_x(n_x+1)}{2}= \fr{n_x(n_x+3)}{2}\,.
$$

Таким образом, имеет место следующее утверж\-дение.

\smallskip

\noindent
\textbf{Теорема~1.} \textit{Пусть в гауссовской системе}~(\ref{e1-sin})--(\ref{e2-sin}) \textit{при условиях Лип\-це\-ра--Ши\-ря\-ева диффузионная матрица  $\si_1\hm= \si_1 (Y_t, t) \hm=\psi_1\nu_0\psi_1^{\mathrm{T}} (Y_t, t)$ не вырождена. Тогда с.к.\  оптимальный фильтр определяется уравнением}~(\ref{e6-sin}), \textit{причем его точность оценивается согласно}~(\ref{e7-sin}).

\smallskip

\noindent
\textbf{Замечание~3.}
 Очевидно, что ФЛШ будет совпадать с~обобщенным фильтром Кал\-ма\-на--Бью\-си, фильтрами второго порядка, гауссовыми фильтрами~\cite{1-sin, 6-sin, 8-sin}, если записать условиях Лип\-це\-ра--Ши\-ря\-ева в~виде:
\begin{equation}
    \left.
    \begin{array}{rl}
   \hspace*{-1.5mm}a_1\left(Y_t, t\right) X_t + a_0 \left(Y_t, t\right) &= {}\\[5pt]
    &\hspace*{-37mm}{}=\bar \vrp \left(\hat{X}_t,Y_t, t\right) + \bar\vrp_x \left(\hat{X}_t,Y_t, t\right)^{\mathrm{T}} \left(X_t-\hat{X}_t\right);\\[7pt]
      \hspace*{-1.5mm}b_1\left(Y_t, t\right) X_t + b_0 \left(Y_t, t\right) &={}\\[5pt]
   &\hspace*{-37mm}{}=\bar\vrp_1 \left(\hat{X}_t,Y_t, t\right)+\bar\vrp_{1x}\left(\hat{X}_t,Y_t, t\right)^{\mathrm{T}} \left(X_t-\hat{X}_t\right).
   \end{array}\!
   \right\}\!\!
   \label{e8-sin}
   \end{equation}
Здесь
\begin{align*}
\bar\vrp_x(\hat{X}_t,Y_t, t) &= a_1(Y_t, t)\,;\\
 \bar\vrp_{1x}(\hat{X}_t,Y_t, t) &=b_1(Y_t,t)\,;\\
 \bar\vrp &=  a_1(Y_t, t) \hat{X}_t+a_0(Y_t, t)\,;\\
  \bar\vrp_1 &=  b_1 (Y_t, t) \hat{X}_t+b_0 (Y_t, t)
  \end{align*}
и~необходимо учесть, что $\bar\vrp_x$ и $\bar \vrp_{1x}$ не зависят от~$X_t$.

\section{Нормальные фильтры Пугачёва для~дифференциальных стохастических систем, линейных относительно состояния}

Следуя~\cite{1-sin}, будем искать фильтр для оценки $\hat{X}_t$ в виде следующего уравнения:
\begin{multline}
\hspace*{-2mm}d\hat{X} ={}\\
\hspace*{-1mm}{}=\alp_t \xi \yhxt dt + \beta_t\eta \yhxt dY_t +
    \gamma_t\, dt,\!\!\label{e9-sin}
    \end{multline}
где  $\xi =\xi \yhxt$ и $\eta\hm=\eta\yhxt$~--- некоторые функции
текущих значений наблюдаемого процесса~$Y_t$, оценки $\hat{X}_t$ и
времени~$t$; $\alp_t$, $\beta_t$ и~$\gamma_t$~--- некоторые
функции времени.

Если бы коэффициенты  $\alp_t$, $\beta_t$ и~$\gamma_t$ в~(\ref{e9-sin})
были известными функциями времени, то уравнение~(\ref{e9-sin}) определило
бы фильтр того же  порядка~${n_x}$, что и~уравнение~(\ref{e1-sin}), описывающее поведение системы. Поэтому, естественно,
возникает мысль попытаться непосредственно определить коэффициенты
$\alp_t$, $\beta_t$ и~$\gamma_t$ в~уравнении~(\ref{e9-sin}) как функции
времени из условия минимума с.к.\ ошибки  $\mathrm{M}\lv
\hat{X} \hm- X_t\rrv^2 =\min$ при всех  $t\hm>t_0$. Это приводит к~теории
услов\-но-оп\-ти\-маль\-но\-го фильт\-ра Пугачева (ФП), когда в~уравнение
фильт\-ра задаются заранее и~оптимизируются только коэффициенты
этого уравнения. Итак, мы приходим к~идее
нахождения оптимального фильт\-ра
в~некотором классе допустимых фильт\-ров,
определяемом условием, что поведение фильт\-ра описывается
дифференциальным уравнением заданного порядка и~заданной формы.
Таким образом, мы отказываемся от абсолютной оптимизации
и~ограничиваемся условной оптимизацией в~заданном ограниченном классе фильтров.

Определив класс допустимых фильтров, следует решить вопрос о~том,
какой фильтр в этом классе считается оптимальным. Следуя Пугачёву~\cite{1-sin}, будем считать оптимальным такой фильтр, который дает в~известном смысле наилучшую оценку при всех $t\hm>t_0$. Иными словами, задача оптимизации фильтра при всех
$t\hm>t_0$ является  задачей многокритериальной оптимизации. Такие
задачи, как правило, не имеют решения. Фильтр Кал\-ма\-на--Бью\-си, дающий
оптимальную линейную оценку состояния линейной системы в каждый
момент $t\hm>t_0$, является исключением~\cite{1-sin}. Значит, надо
определить такую оптимальность фильтра, при которой возможно решение
задачи.
Будем считать оптимальным
такой допустимый фильтр, который на каждом бесконечно малом
интервале времени совершает оптимальный переход из того со\-сто\-яния,
в~котором он был в~начале этого интервала, в~новое со\-сто\-яние.
Такой допустимый фильтр будем называть   услов\-но-оп\-ти\-маль\-ным. Тогда задачи фильтрации сведутся к~нахождению оптимальных значений
 $\alp_t$, $\beta_t$ и~$\gamma_t$ в~(\ref{e8-sin}) в~любой момент  $t\hm\ge t_0$, обеспечивающих минимум
с.к.\ ошибки фильтрации $\mathrm{M}\lv \hat X_s \hm-
X_s\rrv^2$  в~бесконечно близкий будущий момент  $s\hm> t$, $s\hm\to t$.

Отметим, что ФП обладает тем свойством, что в~данном
классе допустимых фильтров не существует фильтра, который при
данном начальном распределении  $Y_t$, $X_t$ и~$\hat{X}_t$ в~момент~$t_0$
был бы лучше услов\-но-оп\-ти\-маль\-но\-го при всех  $t\hm>t_0$. Это
значит, по терминологии теории многокритериальной оптимизации, что
ФП представляет собой один из множества
допустимых фильтров~---  оптимальный по Парето~\cite{1-sin}.
Общая теория ФП по с.к.\ критерию развита для уравнений (1), (2)
и подробно изложена в~\cite{1-sin}.
Теория ФП
обладает двумя несомненными преимуществами по сравнению с методами
субоптимальной фильтрации. Во-пер\-вых, она позволяет
получать фильтры более низкого порядка и,~следовательно, более
простые в реализации. Во-вто\-рых, она дает возможность получать
фильтры не меньшей, а~при желании даже большей точности, чем
фильтры, даваемые методами субоптимальной нелинейной
фильтрации~\cite{1-sin, 6-sin, 8-sin}.

Применяя теорию ФП~\cite{1-sin} к~нормальным процессам в~гауссовской СтС~(\ref{e1-sin}), (\ref{e2-sin}) при условиях Лип\-це\-ра--Ши\-ря\-ева и~$V\hm=V_0$, придем к~НФП вида~(\ref{e9-sin}). Входящие в~(\ref{e9-sin}) коэффициенты определяются следующими уравнениями:
    \begin{gather}
    \alp_t m_1 + \beta_t m_2 + \gamma_t=m_0\,;\label{e10-sin}\\
   m_0 = \mathrm{M}^N_\Delta [\varphi]\,;\enskip m_1 = \mathrm{M}^N_\Delta  [\xi]\,;\enskip m_2 = \mathrm{M}^N_\Delta  [\eta]\,;\label{e11-sin}\\
\beta_t =\kappa_{02} \kappa_{22}^{-1}\,;\label{e12-sin}\\
  \kappa_{02} ={}\hspace*{63mm}\notag\\
   {}=\mathrm{M}^N_\Delta \left [\left(X_t - \hat{X}_t\right) \varphi_1^{\mathrm{T}} \eta^{\mathrm{T}}\right]+\mathrm{M}^N_\Delta \left[\psi  \nu_0\psi_1^{\mathrm{T}} \eta^{\mathrm{T}} \right]\,;\label{e12a-sin}\\
   \kappa_{22} = \mathrm{M}^N_\Delta  \eta \psi_1\nu(t) \psi_1^{\mathrm{T}} \eta^{\mathrm{T}}\,.\label{e13-sin}
   \end{gather}

\vspace*{-12pt}

   \noindent
   \begin{multline}
  \alp_t\kappa_{11} + \mathrm{M}^N_\Delta \left[ \left(\hat{X}_t-X_t\right)\left(\xi^{\mathrm{T}} \alp_t^{\mathrm{T}} +\gamma_t^{\mathrm{T}}\right)
\fr{\partial \xi^{\mathrm{T}}}{\partial x}\right]={}\\
{}=\kappa_{01}'-\beta_t\kappa_{21}'\,;\label{e14-sin}
\end{multline}
\begin{equation}
\kappa_{21}' = \mathrm{M}^N_\Delta  \left[ \eta\varphi_1-m_2\right] \xi\,;\label{e15-sin}
\end{equation}

\noindent
\begin{multline*}
   \kappa_{01}' =\kappa_{01} + \mathrm{M}^N_\Delta  \left[ \left(X_t-\hat{X}_t\right)  \fr{\partial \xi^{\mathrm{T}}}{\partial t} \right]+{}\\
   {}+
     \mathrm{M}^N_\Delta \left\{ \left(X_t-\hat{X}_t\right) \varphi_1^{\mathrm{T}} +{}\right.\\
\left.{}+\psi\nu_0\psi_1^{\mathrm{T}} - \beta_t\eta\psi_1\nu_0\psi_1^{\mathrm{T}}
\vphantom{\left(X_t-\hat{X}_t\right)}
\right\} \left(
    \fr{\partial}{\partial y}+\eta^{\mathrm{T}}\beta_t^{\mathrm{T}} \fr{\partial } {\partial x}\right)\xi^{\mathrm{T}}+{}
    \end{multline*}

    \noindent
    \begin{multline}
{}+\fr{1}{2}\, \mathrm{M}^N_\Delta \left(X_t-\hat{X}_t\right) \left\{ \mathrm{tr} \left[ \psi_1\nu_0\psi_1^{\mathrm{T}}
    \left( \fr{\partial}{\partial y}+{}\right.\right.\right.\\
    {}\left.\left.+2 \eta^{\mathrm{T}}\beta_t^{\mathrm{T}} \fr{\partial}{\partial x}\right)
\fr{\partial^{\mathrm{T}}}{\partial y}\right]+{}\\
\left.{}+\mathrm{tr}\left[ \beta_t\eta\psi_1\nu_0\psi_1^{\mathrm{T}}\eta^{\mathrm{T}}\beta_t^{\mathrm{T}} \fr{\partial}{\partial x}\,
\fr{\partial^{\mathrm{T}}}{\partial x}\right]\right\} \xi^{\mathrm{T}} \,;\label{e16-sin}
\end{multline}

%\vspace*{-24pt}

\noindent
\begin{equation}
\left.
\begin{array}{c}
 \kappa_{11} = \mathrm{M}^N_\Delta \left[ \xi - m_1\right] \xi^{\mathrm{T}}\,;\\[6pt]
 \kappa_{21} = \mathrm{M}^N_\Delta \left[ \varphi_1  - m_2\right] \xi^{\mathrm{T}} \,;\\[6pt]
 \kappa_{01} = \mathrm{M}^N_\Delta \left[ \varphi  - m_0\right] \xi^{\mathrm{T}} \,.
 \end{array}
 \right\}
 \label{e17-sin}
 \end{equation}

Точность НФП определяется уравнением:
      \begin{multline}
      \dot R_t = \mathrm{M}^N_\Delta \left[\left( X_t -\hat{X}_t\right) \varphi \yxtt^{\mathrm{T}} +{}\right.\\
      {}+\varphi\yxtt (X_t^{\mathrm{T}} -\hat{X}_t^{\mathrm{T}})-\beta_t \eta \yutt \times{}\\
      {}\times \psi_1 \yxtt \nu_0 \psi_1\yxtt^{\mathrm{T}} \eta \yutt^{\mathrm{T}} \beta_t^{\mathrm{T}} +{}\\
   \left. {}+\psi \yxtt \nu_0 \psi \yxtt^{\mathrm{T}}\right]\,.
   \label{e18-sin}
   \end{multline}

Таким образом, справедливо следующее утверж\-дение.

\smallskip

\noindent
\textbf{Теорема~2.} \textit{Пусть для гауссовской системы}~(\ref{e1-sin})--(\ref{e2-sin}) \textit{при условиях Лип\-це\-ра--Ши\-ря\-ева выполнены условия невырожденности матрицы}~$\kappa_{22}$~(\ref{e13-sin}) \textit{и~конечности величин~$\kappa_{ij}$ $(i,j\hm=0,1,2)$. Тогда НФП определяется уравнением}~(\ref{e9-sin}), \textit{а~$\alp_t$, $\beta_t$ и~$\gamma_t$}~--- \textit{уравнениями}~(\ref{e10-sin})--(\ref{e18-sin}).

\smallskip

Для негауссовской СтС в уравнениях теоремы~2 следует заменить  $\nu_0$ на~$\nu$ согласно~(\ref{e3-sin}), а~в~выражении~(\ref{e16-sin}) учесть два дополнительных интегральных члена:
        \begin{multline*}
    \kappa_{01}'  =\kappa_{01} + \mathrm{M}^N_\Delta  \left[ \left(X_t - \hat X_t\right) \fr{\partial\xi^{\mathrm{T}}}{\partial t} \right] +{}\\
    {}+\mathrm{M}^N_\Delta  \left\{ \left(X_t -\hat X_t\right) \left[ \vrp_1^{\mathrm{T}} -\iii_{R_0^q} c(\rho)^{\mathrm{T}} \nu_P (t,\rho) d\rho \psi_1^{\mathrm{T}} \right] +{}\right.\\
\left.{}+ \psi \nu_0 \psi_1^{\mathrm{T}} - \beta_t \eta \psi_1 \nu_0\psi_1^{\mathrm{T}}
\vphantom{\iii_{R_0^q}}\right\} \left( \fr{\partial}{\partial y} + \eta^{\mathrm{T}} \beta_t^{\mathrm{T}} \fr{\partial}{\partial x}\right) \xi^{\mathrm{T}}+ {}\\
{}+\fr{1}{2} \mathrm{M}^N_\Delta  \left[ \left(X_t -\hat X_t\right)\right] \times{}\\
{}\times\left\{ \mathrm{tr}\, \left[ \psi_1 \nu_0 \psi_1^{\mathrm{T}} \left( \fr{\partial}{\partial y} + 2 \eta^{\mathrm{T}} \beta_t \fr{\partial}{\partial x} \right)\fr{\partial^{\mathrm{T}}}{\partial y} \right]+{}\right.\\
\left.{}+
   \mathrm{tr}\, \left[ \beta_t \eta \psi_1 \nu_0\psi_1^{\mathrm{T}} \eta^{\mathrm{T}} \beta_t^{\mathrm{T}} \fr{\partial}{\partial x} \fr{\partial^{\mathrm{T}}}{\partial x} \right] \right\} \xi^{\mathrm{T}} +{}\\
{}+ \mathrm{M}^N_\Delta \iii_{R_0^q}  \left[ X_t -\hat{X}_t + \left(\psi -\beta_t \eta \psi_1\right) c (\rho) \times{}\right.
\end{multline*}

\noindent
\begin{multline}
{}\times \xi \left(Y_t +\psi c(\rho), \hat{X}_t + \beta_t \eta \psi_1 c(\rho), t\right) -{}\\
\left.{}-\xi^{\mathrm{T}}
\vphantom{\left[ X_t -\hat{X}_t + \left(\psi -\beta_t \eta \psi_1\right) c (\rho) \times{}\right.}
\right]^{\mathrm{T}} \nu_P (t, d\rho)\, d\rho\,,\label{e19-sin}
\end{multline}
где функции $\vrp$, $\vrp_1$, $\psi$ и~$\psi_1$ удовлетворяют условиям Лип\-це\-ра--Ши\-ря\-ева. В~результате имеем следующее утверждение.

\smallskip

\noindent
\textbf{Теорема~3.} \textit{Пусть для негауссовской СтС}~(\ref{e1-sin})--(\ref{e2-sin}) \textit{в~условиях Лип\-це\-ра--Ши\-ря\-ева  матрица~$\kappa_{22}$}~(\ref{e13-sin}) \textit{невырождена,
а~интегралы}~(\ref{e10-sin}), (\ref{e13-sin}), (\ref{e15-sin}), (\ref{e17-sin}) \textit{и}~(\ref{e19-sin}) \textit{конечны.
Тогда НФП определяется уравнением}~(\ref{e9-sin}), \textit{а~коэффициенты}~$\alp_t$, $\beta_t$ и~$\gamma_t$~---
\textit{утверждениями}~(\ref{e10-sin}), (\ref{e12-sin}) \textit{и}~(\ref{e14-sin}).

\smallskip

\noindent
\textbf{Замечание~4.}
Теория НФП (оценивания состояния и~параметров систем) не позволяет получить нормальные с.к.\ оптимальные фильтры. Можно получить только ФП,
которые в~общем случае хуже оптимальных, но зато легко реализуемы.
Однако, если с.к.\ оптимальная оценка~$\hat{X}_t$ вектора~$X_t$
удовлетворяет уравнению допустимого фильтра~(\ref{e9-sin}) при ка\-ких-ли\-бо коэффициентах времени~$\alp_t$,
$\beta_t$ и~$\gamma_t$, то уравнения теорем~2 и~3, конечно, определяют именно эти $\alp_t$, $\beta_t$ и~$\gamma_t$ и~НФП будет  с.к.\ оптимальным
(последний в данном случае будет допустимым и,~следовательно,
оптимальным в классе допустимых фильтров).

\smallskip

\noindent
\textbf{Замечание~5.}
Теория НФП дает возможность оценивать не все компоненты вектора состояния системы (в~общем случае расширенного), а~только некоторые из них. Для этого достаточно
взять структурные функции~$\xi$ и~$\eta$ в~(\ref{e9-sin}), зависящими
лишь от соответствующих компонент вектора  $\hat{X}_t$. Так, например,
взяв~$\xi$ и~$\eta$ в~(\ref{e9-sin}), зависящими лишь от~$Y_t$, $t$ и~оценок неизвестных параметров системы, можно оценивать только
параметры системы, не оценивая ее состояния. В~таких случаях будут
получаться НФП, порядок которых меньше размерности~$n_x$
расширенного вектора состояния.


\section{Дифференциальный нормальный  фильтр Пугачёва на~основе нормальной аппроксимации апостериорного распределения}

Рассмотрим сначала гауссовскую СтС~(\ref{e1-sin})--(\ref{e2-sin}).
Так как гауссовское (нормальное) распределение, аппроксимирующее
апостериорное распределение вектора~$X_t$, полностью определяется
апостериорными математическим ожиданием~$\hat{X}_t$ и~ковариационной матрицей  $R_t$ вектора~$X_t$, то согласно теории нелинейной субоптимальной фильтрации при аппроксимации апостериорного распределения вектора~$X_t$ нормальным
распределением будем иметь следующие стохастические дифференциальные уравнения,
определяющие~$\hat{X}_t$ и~$R_t$~\cite{1-sin}:

\noindent
    \begin{multline}
    \dot{\hat X}_t = f \left(\hat X_t, Y_t,R_t,t\right)+{}\\
    {}+
    h\left(\hat X_t,Y_t, R_t,t\right)
   \left[ \dot Y_t - f^{(1)} \left(\hat X_t,Y_t, R_t,t\right)\right]\,;\label{e20-sin}
    \end{multline}


    \vspace*{-12pt}

    \noindent
    \begin{multline}
    \dot R_t=\left\{ f^{(2)}(\hat X_t, Y_t,R_t,t)-{}\right.\\
    \left.{}-h\left(\hat
    X_t, Y_t,R_t,t\right)\psi_1\nu_0\psi_1^{\mathrm{T}} \left(Y_t,t\right)
 h ({\hat X}_t, Y_t,R_t,t)^{\mathrm{T}}\right\} +{}\\
 \hspace*{-1.7mm}{}+\sss_{r=1}^{n_y}\! \rho_r \left(\hat{X}_t,Y_t, R_t,t\right)\!\!\left[
    \dot Y_r -f_r^{(1)}\left({\hat X}_t,Y_t, R_t,t\right) \right]\!,\!\!\!\label{e21-sin}
    \end{multline}
где
    \begin{multline*}
    f\left(\hat X_t, Y_t,R_t,t\right)=
    \mathrm{M}^N_\Delta \vrp \left(Y_t, \nu,t\right)={}\\
     {}=a_1 \left(Y_t, t\right) \hat{X}_t + a_0 \left(Y_t, t\right)\,; %\label{e22-sin}
    \end{multline*}

    \vspace*{-12pt}

    \noindent
    \begin{multline*}
   f^{(1)}\left(\hat X_t, Y_t,R_t,t\right)=\left\{ f_r^{(1)} \left( \hat X_t, Y_t, R_t, t\right)\right\}={}\\
   {}=\mathrm{M}^N_\Delta \vrp \left(Y_t, \nu,t\right)=b_1 \left(Y_t, t\right) \hat{X}_t+ b_0 \left(Y_t, t\right)\,;
   %\label{e23-sin}
   \end{multline*}

\vspace*{-12pt}

\noindent
   \begin{multline*}
    h\left(\hat X_t, Y_t,R_t,t\right)=\mathrm{M}^N_\Delta \left[ x\varphi_1(Y_t,x,t)^{\mathrm{T}} + {}\right.\\
\left.    {}+\psi\nu_0\psi_1^{\mathrm{T}} \left(Y_t,x,t\right)\right]-
    \hat{X}_t f^{(1)}\left(\hat X_t, Y_t,R_t,t\right)^{\mathrm{T}} %\right\}
    \times{}\\
{}\times \left(\psi_1\nu_0\psi_1^{\mathrm{T}}\right)^{-1} \left(Y_t,t\right)=
\left[ R_t b_1 \left(Y_t, t\right)^{\mathrm{T}} + {}\right.\\
\left.{}+\left(\psi\nu_0\psi_1^{\mathrm{T}}\right)\left(Y_t, t\right)
\vphantom{\left(Y_t, t\right)^{\mathrm{T}}}
\right] \left(\psi\nu_0\psi_1^{\mathrm{T}}\right)^{-1} \left(Y_t, t\right)\,;
%\label{e24-sin}
\end{multline*}

\vspace*{-12pt}

\noindent
\begin{multline*}
   \!\!f^{(2)}\left(\hat X_t, Y_t,R_t,t\right)=
   \mathrm{M}^N_\Delta \left\{  \left(x-\hat{X}_t\right)\varphi\left(Y_t,x,t\right)^{\mathrm{T}} + {}\right.\\
\left.   {}+\varphi \left(Y_t,x,t\right) \left(x^{\mathrm{T}}-\hat{X}_t^{\mathrm{T}}\right) +\psi\nu_0\psi^{\mathrm{T}} \left(Y_t,x,t\right)\right\}={}\\
{}=\left[ R_t b_1 \left(Y_t, t\right)^{\mathrm{T}} + \left(\psi\nu\psi_1^{\mathrm{T}}\right)\left(Y_t, t\right)\right] \left(\psi_1\nu_0\psi_1^{\mathrm{T}}\right)^{-1}\times{}\\
 {}\times \left(Y_t, t\right) \left[ b_1 (Y_t, t) R_t + \left(\psi_1\nu_0\psi^{\mathrm{T}}\right)\left(Y_t, t\right)\right]\,;
%\label{e25-sin}
\end{multline*}

\vspace*{-12pt}

\noindent
\begin{multline*}
   \rho_r\left(\hat{X}_t,Y_t, R_t,t\right)={}\\
   {}=\mathrm{M}^N_\Delta \left\{  \left(x-\hat X_t\right)\left(x^{\mathrm{T}}-\hat X_t^{\mathrm{T}}\right) a_r \left(Y_t,x,t\right)+{}\right.\\
{}+ \left(x-\hat X_t\right) b_r\left(Y_t,x,t\right)^{\mathrm{T}} \left(x^{\mathrm{T}}-\hat X_t^{\mathrm{T}}\right)+{}\\
 \left.{}+b_r \left(Y_t,x,t\right) \left(x^{\mathrm{T}}-\hat{X}_t^{\mathrm{T}}\right)\right\} =0 \enskip (r=1, \ldots, n_y)\,.
%\label{e26-sin}
\end{multline*}
Здесь функции  $a_r$~--- $r$-й элемент мат\-ри\-цы-стро\-ки
$$
A_{\vrp_1} = (\vrp_1^{\mathrm{T}} - \hat\vrp_n^{\mathrm{T}})(\psi_1 \nu_0\psi_1^{\mathrm{T}})^{-1}\,;
$$
 $B_{kr}^{\psi\psi_1}$~--- элемент $k$-й строки и~$r$-го столб\-ца матрицы
 $$
 B^{\psi\psi_1} = (\psi\nu_0\psi_1^{\mathrm{T}})
(\psi_1\nu_0\psi_1^{\mathrm{T}})^{-1}\,;
 $$
 $b_r$~--- $r$-й столбец матрицы $B^{\psi\psi_1}$:
 $$
 b_r^{\psi\psi_1} = \left[ b_{1r}^{\psi\psi_1}\cdots b_{n_x r}^{\psi\psi_1}\right]^{\mathrm{T}}\,.
 $$

Количество уравнений метода нормальной аппроксимации (МНА) одномерного апостериорного распределения
определяется по формуле:
    \begin{equation*}
   Q_{\mathrm{МНА}} = n_x + \fr{n_x (n_x+1)}{2} = \fr{n_x(n_x+3)}{2}\,.
   %\label{e27-sin}
   \end{equation*}

За начальные значения $\hat{X}_t$ и~$R_t$  при интегрировании уравнений~(\ref{e20-sin}) и~(\ref{e21-sin}), естественно, следует принять
условные математическое ожидание и ковариационную матрицу величины~$X_0$ относительно~$Y_0$:
\begin{align*}
\hat{X}_0 &= \mathrm{M}^N_\Delta\left[ X_0 \mid Y_0\right]\,;\\
R_0 &= \mathrm{M}^N_\Delta \left[ \left(X_0 -\hat X_0\right) \left(X_0^{\mathrm{T}} -\hat{X}_0^{\mathrm{T}}\right )\mid Y_0\right]\,.
%\label{e28-sin}
\end{align*}

\noindent
\textbf{Замечание~6.}
Если нет информации об условном распределении~$X_0$ относительно~$Y_0$, то
начальные условия можно взять в виде:
\begin{align*}
\hat{X}_0 &= \mathrm{M}^N_\Delta X_0\,;\\
R_0&= \mathrm{M}^N_\Delta (X_0 \hm-\mathrm{M}^N_\Delta X_0)
(X_0^{\mathrm{T}} - \mathrm{M}^N_\Delta X_0^{\mathrm{T}})\,.
\end{align*}
Если
же и об этих величинах нет никакой информации, то начальные
значения $\hat X_t$ и~$R_t$ приходится задавать произвольно.

Сравнивая  уравнения~(\ref{e20-sin}) и~(\ref{e21-sin}) с~уравнениями ФЛШ, имеем следующий результат.

\smallskip

\noindent
\textbf{Теорема 4.} \textit{Для гауссовской СтС}~(\ref{e1-sin})--(\ref{e2-sin}) \textit{при условиях Лип\-це\-ра--Ши\-ря\-ева НФП (на основе нормальной аппроксимации апостериорной плотности) и~ФЛШ совпадают.}

\smallskip

Для негауссовский СтС~(\ref{e1-sin})--(\ref{e2-sin}) при условиях Лип\-це\-ра--Ши\-ря\-ева, учитывая два дополнительных интегральных члена в~(\ref{e19-sin}), придем к уравнениям~(\ref{e20-sin}), (\ref{e21-sin}). При этом коэффициенты~$\alp_t$ и~$\gamma_t$ определяются~(\ref{e10-sin}), (\ref{e14-sin}), (\ref{e19-sin}) после нахождения~$\beta_t$ по формуле~(\ref{e12-sin}). Таким образом, имеем следующий результат.

\smallskip

\noindent
\textbf{Теорема~5.} \textit{Пусть для негауссовской системы}~(\ref{e1-sin})--(\ref{e2-sin})
\textit{при условиях Лип\-це\-ра--Ши\-ря\-ева выполнены условия невырожденности матрицы~$\kappa_{22}$ и конечности величин}~(\ref{e10-sin})--(\ref{e19-sin}) $(i, j \hm=0,1,2)$.
\textit{Тогда НФП определяется уравнением}~(\ref{e9-sin}),
\textit{а~коэффициенты $\alp_t$, $\beta_t$ и~$\gamma_t$}~---
\textit{уравнениями}~(\ref{e10-sin})--(\ref{e19-sin}).


\section{Квазилинейный нормальный фильтр Пугачёва}


Особое практическое значение имеет случай~(\ref{e1-sin})--(\ref{e2-sin}) при условиях Лип\-це\-ра--Ши\-ря\-ева с~аддитивными (в~общем случае негауссовскими) шумами. Следуя~\cite{1-sin}, проведем статистическую линеаризацию нелинейных функций:
    \begin{align*}
    a_1 \left(Y_t, t\right) X_t &\approx \left(k_{0x}^{a_1 x} - k_{1x}^{a_1 x}\right) m_t^x +{}\\
     &{}+\left( k_{0y}^{a_1 x} - k_{1y}^{a_1 x}\right) m_t^y + k_x^{a_1 y} X_t + k_{0y}^{a_1 x} Y_t\,;
   \\
    b_1 \left(Y_t, t\right) X_t &\approx \left(k_{0x}^{b_1 x} - k_{1x}^{b_1 x}\right) m_t^x + {}\\
    &{}+\left( k_{0y}^{b_1 x} - k_{1y}^{b_1 x}\right) m_t^y + k_x^{b_1 y} X_t + k_{0y}^{b_1 x} Y_t\,;
   \\
    a_0 (Y_t , t) &\approx \left( k_{0y}^{a_0} - k_{1y}^{a_0}\right) m_t^y + k_{0y}^{a_0} Y_t\,;\\
    b_0 \left(Y_t , t\right) &\approx  \left( k_{0y}^{b_0} - k_{1y}^{b_0}\right) m_t^y + k_{0y}^{b_0} Y_t\,.
    %\label{e29-sin}
    \end{align*}
Тогда уравнения~(\ref{e1-sin}) и~(\ref{e2-sin}) приводятся к~эквивалентной гауссовской системе, линейной относительно~$X_t^0$ и~$Y_t^0$ и~нелинейной относительно~$m_t^x$ и~$m_t^y$:
\begin{align}
\dot X_t &= \bar a Y_t + \bar a_1 X_t + \bar a_0 +\bar\psi V\,;\label{e30-sin}\\
\dot Y_t &= \bar b Y_t + \bar b_1 X_t + \bar b_0 +\bar\psi_1 V\,.\label{e31-sin}
\end{align}
Здесь
    \begin{equation}
    \left.
    \begin{array}{rlrl}
    \bar a &= k_y^{a_1x} + k_y^{a_0}\,; &\enskip     \bar b &= k_y^{b_1x} \,;\\
    \bar a_1 &= k_x^{a_1 x}\,;&\enskip \bar b_1 &= k_x^{b_1 x}\,;\\
    a_0 &= (k_{0y}^{a_0} - k_{1y}^{a_0}) m_t^y\,;
 &\enskip b_0 &= (k_{0y}^{b_0} - k_{1y}^{b_0}) m_t^y\,.
    \end{array}
    \right\}
    \label{e32-sin}
    \end{equation}
Правые части~(\ref{e32-sin}) зависят от вероятностных моментов первого и второго порядка и определяются из следующей линейной дифференциальной системы для вектора  $Z_t \hm=\left[ X_t^{\mathrm{T}}\, Y_t^{\mathrm{T}}\right]^{\mathrm{T}}$:
  \begin{align*}
  \bar m_t^z &= cm_t^z + c_0\,; %\label{e33-sin}
  \\
    K_t^{\bar z} &= c K_t^z +K_t^z c^{\mathrm{T}} + l\nu l^{\mathrm{T}}\,, %\label{e34-sin}
    \end{align*}
где
    \begin{equation*}
    c=\begin{bmatrix}
    \bar a_1&\bar a\\
    \bar b_1&\bar b\end{bmatrix}\,; \enskip
    c_0=\begin{bmatrix}
    \bar a_o\\
    \bar b_0\end{bmatrix}\,;\enskip
    l=\begin{bmatrix}
    \bar \psi_t\\
    \bar\psi_{1t}\end{bmatrix}\,.\label{e35-sin}
\end{equation*}

Применяя теорию квазилинейной фильтрации~\cite{1-sin}
к~уравнениям~(\ref{e30-sin}) и~(\ref{e31-sin}), получим следующие уравнения квазилинейного фильтра:
\begin{align}
    \dot{\hat X}_t& = \bar a Y_t +\bar a_1 \hat X_t + \bar a_0 +{}\notag\\
     &\hspace*{15mm}{}+\beta_t \left[ \dot Y_t - \left( \bar b Y_t +\bar b_1 \hat{X}_t + \bar b_0\right) \right]\,;
    \label{e36-sin}
\\
   \beta_t &= R_t \bar b_1^{\mathrm{T}} +\left(\bar\psi\nu\bar\psi_1^{\mathrm{T}}\right) \left(\bar \psi_1 \nu \bar \psi_1^{\mathrm{T}}\right)^{-1}\,;\label{e37-sin}\\
    \dot R_t &= \bar a_1 R_t + R_t \bar a_1^{\mathrm{T}} +\bar\psi\nu\bar\psi^{\mathrm{T}} -{}\notag\\
&\hspace*{-5mm}{}-\left(R_t \bar b_1^{\mathrm{T}} +\bar\psi\nu\bar\psi_1\right)\left(\bar\psi_1 \nu\bar\psi_1^{\mathrm{T}}\right)^{-1} \left( \bar b R_t +\bar\psi_1\nu\bar\psi_1^{\mathrm{T}}\right)\,.
    \label{e38-sin}
    \end{align}

Таким образом, имеем следующий результат.

\smallskip

\noindent
\textbf{Теорема~6.} \textit{Пусть уравнения негауссовской СтС}~(\ref{e1-sin})--(\ref{e2-sin}) \textit{при условиях Лип\-це\-ра--Ши\-ря\-ева и~с~аддитивными шумами  допускают применение метода статистической линеаризации (МСЛ). Тогда уравнения  квазилинейного НФП имеют вид}~(\ref{e36-sin})--(\ref{e38-sin}).

\smallskip

\noindent
\textbf{Замечание~7.}
Согласно~\cite{2-sin, 3-sin}, из уравнения~(\ref{e38-sin}) получаются приближенные условия асимптотической устойчивости в~терминах равномерной наблюда\-емости и~управляемости в~следующем виде:
    \begin{multline*}
    0\le R_t (R_0, t_0) \le u_A (t, t_0) R_0 u_A(t, t_0) +{}\\
     {}+\iii_{t_0}^t u_A (t,\tau) \si_{11} u_A (t, \tau)^{\mathrm{T}}\, d\tau\,,
%    \label{e39-sin}
    \end{multline*}
где $u_A (t,\tau)$~--- фундаментальная матрица однородного уравнения Риккати, причем
   \begin{gather*}
    A =\bar a_1 -\beta_t \bar b_1 - R_t \bar b_1^{\mathrm{T}} \si_{22}^{-1} b_1- \si_{12} \si_{22}^{-1} \bar b_1\,; %\label{e40-sin}
    \\
    \si_{11}= \bar\psi_{10t}\nu\bar\psi_{10}^{\mathrm{T}}\,;\\
     \si_{12}= \bar\psi_{10 t}\nu\bar\psi_{20t}^{\mathrm{T}}\,;\quad
      \si_{22} = \bar\psi_{20t}\nu\bar\psi_{20t}^{\mathrm{T}}\,.\label{e41-sin}
    \end{gather*}

    \vspace*{-18pt}


\section{Заключение}

\vspace*{-4pt}

\noindent
\begin{enumerate}[1.]
\item Рассмотрены вопросы аналитического синтеза услов\-но-оп\-ти\-маль\-ных фильтров Пугачёва для обработки информации в дифференциальных негауссовских СтС, линейных относитель\-но состояния (условия Лип\-це\-ра--Ши\-ря\-ева). Особое внимание уделено синтезу фильт\-ров для СтС при условиях Лип\-це\-ра--Ши\-ря\-ева путем аппроксимации апостериорного распределения нормальным и квазилинейным фильт\-ра\-ми, основанным на статистической линеаризации нелинейных функций, зависящих от наблюдений.

\item Для СтС высокой размерности путем выбора структурных функций, отражающих аналитическую природу наблюдаемой системы, можно синтезировать простые в компьютерной реализации фильтры для работы в режиме реального времени.

Изложенные алгоритмы положены в основу модуля математического обеспечения инструментального программного обеспечения <<StS-Filter>>. Тестирование проведено на основе примеров~[1, 4--6, 8].

\item Полученные результаты  позволяют синтезировать дифференциальным ФП,
 во-пер\-вых, для случая широкополосных шумов в~уравне-\linebreak\vspace*{-12pt}

 \pagebreak

 \noindent
 ниях~(\ref{e1-sin}) и~(\ref{e2-sin}), если аналогично~\cite{3-sin} заменить широкополосный белый шум эквивалентным нормальным (гауссовским) белым шумом и,~во-вто\-рых, для случая автокоррелированных шумов. Результаты также обобщаются на случай дискретных СтС, линейных относительно состояния, если воспользоваться~\cite{4-sin}.
    \end{enumerate}

    \setcounter{equation}{0}
{\small \section*{\raggedleft Приложение}

\renewcommand{\theequation}{П.\arabic{equation}}


\section*{Тестовые примеры}


\textbf{1.} Найти приближенно оптимальный алгоритм оценивания
 состояния~$X_t$ системы, описываемой скалярным уравнением
    \begin{equation}
    \dot X_t =- \theta X_t + V_1\,,\label{p1}
    \end{equation}
 и неизвестного параметра~$\theta$ по результатам наблюдения
 процесса
    \begin{equation}
    \dot Y_t= X_t+ V_2\,,\label{p2}
    \end{equation}
где  $V_2$~--- белый шум, независимый от~$V_1$.

 Заменим параметр  $\theta$ стохастическим процессом~$\Theta_t$,
 определяемым уравнением  $\dot \Theta_t\hm=0$, и примем за расширенный
 вектор состояния пару  $\left[ X_t \Theta_t\right]^{\mathrm{T}}$. Тогда уравнения
 НФП будут иметь вид:
   \begin{align*}
   \dot{\hat{X}}_t&=-\hat {X}_t\hat
    \Theta_t - R_{12} +\nu_2^{-1} R_{11} \left(\dot {Y}_t-\hat {X}_t\right)\,;\\
    \dot{\hat {\Theta}}_t &=\nu_2^{-1} R_{12} \left(\dot{Y}_t- \hat{X}_t\right)\,; %\label{p3}
    \\
\dot R_{11} &= \nu_1 - 2\left(\hat {\Theta}_t R_{11} +\hat X_t R_{12}\right)
    -\nu_2^{-1} R_{11}^2\,; %\label{pIV}
    \\
\dot R_{12} &=- \hat {\Theta}_t
    R_{12} -\hat{X}_t R_{22} -\nu_1^{-1} R_{11} R_{12}\,; \\
    \dot{R}_{22} &=-\nu_2^{-1} R_{12}^2\,, %\label{p4}
    \end{align*}
где  $R_{11}$, $R_{12}$ и~$R_{22}$~--- апостериорные дисперсии и~ковариация ошибок оценок~$\hat{X}_t$ и~$\hat{\Theta}_t$
соответственно. За начальные значения  $\hat{X}_t$,
$\hat{\Theta}_t$, $R_{11}$, $R_{12}$ и~$R_{22}$ следует принять
соответствующие априорные величины, причем~$\hat{\Theta}_0$,
$R_{220}$ и~$R_{120}$ всегда приходится брать произвольно, так как
априорной информации о~параметре~$\theta$ обычно нет, за
исключением, может быть, информации о~возможном диапазоне его значений.
Этот фильтр совпадает с~обобщенным фильтром Калмана~\cite{1-sin, 6-sin, 8-sin}.

\smallskip

\textbf{2.}  Найдем НФП для системы~(\ref{p1}), (\ref{p2}).
За основу класса допустимых фильтров примем первые два уравнения
МНА и соответственно положим, что
\begin{equation*}
    \xi \left(\hat X_t, \hat \Theta_t, t\right) =
    \left[ \hat X_t \hat \Theta_t\, \hat X_t\right] ^{\mathrm{T}} \eta \left(
    \hat X_t, \hat \Theta_t, t\right)=1\,.
    %\label{p5}
    \end{equation*}
Тогда класс допустимых фильтров
будет представлять собой систему двух уравнений:
\begin{align*}
   \dot{\hat{X}}_t &= \alp_{11} \hat{X}_t \hat\Theta_t +\alp_{12} \hat{X}_t +\beta_1 \dot Y_t +\gamma_1\,; %\label{p6}
   \\
   \dot{\hat\Theta}_t &= \alp_{21} \hat{X}_t \hat\Theta_t +\alp_{22} \hat{X}_t +\beta_2 \dot Y_t +\gamma_2\,. %\label{p7}
   \end{align*}
Здесь
    \begin{equation*}
    \kappa_{22} =\nu_2\,; %\label{p8}
    \end{equation*}

    \vspace*{-10pt}

    \noindent
    \begin{multline*}
    \kappa_{02} = \mathrm{M}_N\left[ \left(X_t-\hat{X}_t\right) X_t\left (
    \Theta_t -\hat \Theta_t\right)X_t\right]^{\mathrm{T}} ={}\\
    {}=
    \left[ m_{2000} -m_{1010}\, m_{1100}- m_{1001}\right]^{\mathrm{T}}\,; %\label{p9}
    \end{multline*}
    \begin{equation*}
   \beta_1 =\nu_2^{-1} \left(m_{2000}-m_{1010}\right)\,;\enskip \beta_2 =\nu_2^{-1} \left(m_{1100}-m_{1001}\right)\,, %\label{p10}
   \end{equation*}
где  $m_{pqrs}\hm=\mathrm{M}_N X_t^p \Theta_t^q \hat{X}_t^r\hat\Theta_t^s$.
Уравнения,  определяющие оптимальные
коэффициенты $\alp_{11}$, $\alp_{12}$, $\alp_{21}$, $\alp_{22}$,
$\gamma_1$ и~$\gamma_2$, имеют следующий вид:
    $$
    m_{0011}\alp_{11} +m_{0010}\alp_{12}+\gamma_1 =- m_{1100}-\beta_1 m_{1000}\,; %\eqno({\rm П}.11)
    $$
    $$
    m_{0011}\alp_{21} +m_{0010}\alp_{22}+\gamma_2 =- \beta_1 m_{1000}\,;
    %\eqno({\rm П}.12)
    $$

    \vspace*{-10pt}

    \noindent
    \begin{multline*}
        \left(2m_{0022} -m_{1012}- m_{0011}^2\right)\alp_{11}+{}\\
        {}+\left(2m_{0021} -m_{1012}- m_{0010}m_{0011}\right)\alp_{12}+{}\\
{}+ \left(m_{0031} -m_{1021}\right)\alp_{21}+\left(m_{0030} -m_{0020}\right)\alp_{22}+{}\\
{}+\left(m_{0011} -m_{1001}\right)\gamma_1+ \left(m_{0020} -m_{1010}\right)\gamma_2={}\\
{}=- m_{1111} -2m_{1100}m_{0011}+ \beta_1 \left(m_{2001}-2m_{1011}+{}\right.\\
\left.{}+m_{1000} m_{0011}\right)+ \beta_2 \left(m_{2010}-m_{1020}\right) -\nu_2 \beta_1^2 m_{0001} +{}\\
{}+\nu_2 \beta_1\beta_2
    \left(m_{1000}-2m_{0010}\right)\,;\
    %eqno({\rm П}.13)
    \end{multline*}

     \vspace*{-10pt}

    \noindent
    \begin{multline*}
   \left(2m_{0021}-m_{1011}-m_{0010}m_{0011}\right)\alp_{11} + {}\\
   {}+\left(2m_{0020}-m_{1010}-m_{0010}^2\right)\alp_{12}+{}\\
{}+ \left(m_{0010}-m_{1000}\right)\gamma_1 =- m_{1110}+m_{1100}m_{1010}+{}\\
{}+ \beta_1 \left(m_{2000}- 2m_{1010}+m_{1000}m_{0010}\right)-\nu_2\beta_1^2\,;
%\eqno({\rm П}.14)
\end{multline*}

 \vspace*{-10pt}

    \noindent
    \begin{multline*}
   \left(m_{0013}-m_{0112}\right)\alp_{11} + \left( m_{0012}-m_{0111}\right)\alp_{12} +{}\\
   {}+\left(2m_{0022}-m_{0121}-m_{0011}^2\right) \alp_{21}+{}\\
{}+ \left(2m_{0021}-m_{0120}-m_{0010}m_{0001}\right)\alp_{22} +{}\\
{}+\left(m_{0002}-m_{0101}\right)\gamma_1+ \left(m_{0011}-m_{0110}\right)\gamma_2 ={}\\
{}=\beta_1
\left(m_{1101}-m_{1002}\right) +\beta_2 \left(m_{1110}-2 m_{1011}+{}\right.\\
\!\left.{}+m_{1000}m_{0011}\right)+ \nu_2 \beta_1\beta_2 \left(m_{0100}-2m_{0001}\right)-\nu_2\beta_2^2 m_{0010};
%,\eqno({\rm П}.15)$$
\end{multline*}

 \vspace*{-10pt}

    \noindent
    \begin{multline*}
   \left(m_{0012}-m_{0111}\right) \alp_{11} +\left(m_{0011}-m_{0110}\right)\alp_{12} + {}\\
   {}+\left(m_{0021}-m_{0011}m_{0010}\right)\alp_{21}+{}\\
{}+ \left(m_{0020}-m_{0010}^2\right)\alp_{22} +\left(m_{0001}-m_{0100}\right)\gamma_1={}\\
\!{}= \beta_1 \left(m_{1100}-m_{1001}\right)-\beta_2
\left(m_{1010}-m_{1000}m_{0010}\right) -\nu_2 \beta_1\beta_2.\hspace*{-9.5pt}
%\eqno({\rm П}.16)$$
\end{multline*}
Числовые значения получаются путем решения МНА нормальной апостериорной плот\-ности с~по\-мощью инструментального программного обеспечения <<StS-Filter>>.

\smallskip

\textbf{3.} Построим теперь  НФП для оценивания одного только неизвестного параметра~$\theta$. Для определения класса допустимых фильтров включим
неизвестную функцию времени~$\hat{X}_t$ в~уравнение для~$\dot{\hat
\Theta}_t$ предыдущего примера в~оптимизируемые коэффициенты.
Тогда получим следующее уравнение класса допустимых фильтров:
\begin{align*}
    \dot{\hat \Theta}_t& =\alp _t\hat \Theta_t +\beta_t \dot Y_t +\gamma_t\,;
    %\eqno({\rm П}.17)$$
    \\
    \beta_t&= \nu_2^{-1} (m_{110} -m_{101})\,;
    %,\eqno({\rm П}.18)$$
    \\
    \alp_t&=\fr{-m_{111}+m_{110}m_{001}+\beta_t \left(m_{101}-m_{100}m_{001}\right)}{m_{002}^2-m_{001}^2}\,;
    %\eqno({\rm П}.19)$$
    \\
    \gamma_t&=-\alp m_{001}-\beta_t m_{100}\,.
    %\eqno({\rm П}.20)$$
    \end{align*}

}



{\small\frenchspacing
 {%\baselineskip=10.8pt
 \addcontentsline{toc}{section}{References}
 \begin{thebibliography}{9}

\bibitem{1-sin}
\Au{Синицын И.\,Н.}
Фильтры Калмана и Пугачева.~--- 2-е изд. -- М.: Логос, 2007. 776~с.


\bibitem{5-sin} %2
 \Au{Корепанов Э.\,Р.}
 Стохастические информационные технологии на основе фильтров Пугачева~// Информатика и её применения, 2011. Т.~5. Вып.~2. С.~36--57.

\bibitem{4-sin} %3
\Au{Синицын И.\,Н.}  Параметрическое статистическое и аналитическое моделирование распределений в нелинейных стохастических системах на многообразиях~// Информатика и её применения, 2013. Т.~7. Вып.~2. С.~4--16.



\bibitem{6-sin} %4
\Au{Синицын И.\,Н., Синицын В.\,И.} Лекции по нормальной и эллипсоидальной аппроксимации в стохастических системах.~--- М.: ТОРУС ПРЕСС, 2013. 476~с.

\bibitem{2-sin} %5
\Au{Синицын И.\,Н., Корепанов Э.\,Р.}
Устойчивые линейные условно оптимальные фильтры и экстраполяторы для стохастических систем с~мультипликативными шумами~// Информатика и её применения, 2015. Т.~9. Вып.~1. С.~70--75.

\bibitem{3-sin} %6
\Au{Синицын И.\,Н., Корепанов Э.\,Р.}
Синтез устойчивых линей\-ных фильтров и экстраполяторов Пугачева для стохастических систем с мультипликативными широкополосными шумами~// Системы и~средства информатики, 2015. Вып.~25. №\,1.
С.~108--126.

\bibitem{7-sin}
\Au{Липцер Р.\,Ш., Ширяев А.\,Н.} Статистика случайных процессов.~--- М.: Наука, 1974. 476~с.

\bibitem{8-sin}
\Au{Ройтенберг Я.\,Н.} Автоматическое управление.~--- 3-е изд., перераб. и доп.~--- М.: Наука, 1992. 576~с.
 \end{thebibliography}

 }
 }

\end{multicols}

\vspace*{-3pt}

\hfill{\small\textit{Поступила в~редакцию 31.03.15}}

%\newpage

\vspace*{12pt}

\hrule

\vspace*{2pt}

\hrule

%\vspace*{12pt}

\def\tit{NORMAL PUGACHEV FILTERS FOR STATE LINEAR STOCHASTIC SYSTEMS}

\def\titkol{Normal Pugachev filters for state linear stochastic systems}

\def\aut{I.\,N.~Sinitsyn and E.\,R.~Korepanov}

\def\autkol{I.\,N.~Sinitsyn and E.\,R.~Korepanov}

\titel{\tit}{\aut}{\autkol}{\titkol}

\index{Sinitsyn I.\,N.}
\index{Korepanov E.\,R.}

\vspace*{-9pt}


\noindent
Institute of Informatics Problems, Federal Research
Center ``Computer Science and Control'' of the
Russian Academy of Sciences, 44-2 Vavilov Str., Moscow 119333,
Russian Federation


\def\leftfootline{\small{\textbf{\thepage}
\hfill INFORMATIKA I EE PRIMENENIYA~--- INFORMATICS AND
APPLICATIONS\ \ \ 2015\ \ \ volume~9\ \ \ issue\ 2}
}%
 \def\rightfootline{\small{INFORMATIKA I EE PRIMENENIYA~---
INFORMATICS AND APPLICATIONS\ \ \ 2015\ \ \ volume~9\ \ \ issue\ 2
\hfill \textbf{\thepage}}}

\vspace*{3pt}


\Abste{The applied theory of analytical synthesis of normal conditionally optimal (Pugachev) filters (NPF) in state linear non-Gaussian stochastic systems (StS)
is presented. Special attention is paid to NPF for differential StS satisfying Liptzer--Shiraev conditions based on the normal approximation of \textit{a~posteriori} density and quasi-linear NPF based on statistical linearization of nonlinear functions depending on observations.  For StS of high dimension and real-time problems, NPF are more effective than the
suboptimal filters. The NPF algorithms are the basis of the ``StS-Filters'' software tool. Test examples are given.}


\KWE{Liptser--Shiraev filter (LSF); Liptser--Shiraev conditions; normal approximation method (NAM) for \textit{a~posteriori} density; normal conditionally optimal Pugachev filter (NPF);  stochastic systems (StS);
state linear StS; statistical linearization method (SLM)}

\DOI{10.14357/19922264150204}

\Ack
\noindent
The research was supported by the Russian Foundation for Basic Research (project 15-07-02244).



\vspace*{3pt}

  \begin{multicols}{2}

\renewcommand{\bibname}{\protect\rmfamily References}
%\renewcommand{\bibname}{\large\protect\rm References}



{\small\frenchspacing
 {%\baselineskip=10.8pt
 \addcontentsline{toc}{section}{References}
 \begin{thebibliography}{9}

\bibitem{1-sin-1}
\Aue{Sinitsyn, I.\,N.} 2007. \textit{Fil'try Kalmana i~Pugacheva} [Kalman and Pugachev filters]. 2nd ed. Moscow: Logos.  776~p.


\bibitem{5-sin-1} %2
\Aue{Korepanov, E.\,R.} 2011. Stokhasticheskie informatsionnye tekhnologii na osnove fil'trov Pugacheva [Stochastic informational technologies based on Pugachev filters]. \textit{Informatika i ee Primeneniya}~---
\textit{Inform Appl.}   5(2):36--57.

\bibitem{4-sin-1} %3
\Aue{Sinitsyn, I.\,N.}  2013.
Parametricheskoe statisticheskoe i~analiticheskoe modelirovanie raspredeleniy
v~nelineynykh stokhasticheskikh sistemakh na mnogoobraziyakh
[Parametric statistical and analytical modeling of distributions in stochastic systems on manifolds].
\textit{Informatika i~ee Primeneniya}~---
\textit{Inform. Appl.} 7(2):4--16.

\bibitem{6-sin-1} %4
\Aue{Sinitsyn, I.\,N., and  V.\,I.~Sinitsyn}.  2013.
\textit{Lektsii po normal'noy i ellipsoidal'noy approksimatsii raspredeleniy
v~stokhasticheskikh sistemakh} [Lectures on normal and ellipsoidal
approximation of distributions in stochastic systems]. Moscow: TORUS PRESS. 488~p.

\bibitem{2-sin-1} %5
\Aue{Sinitsyn, I.\,N., and E.\,R.~Korepanov}. 2015.
Ustoychivye lineynye uslovno optimal'nye fil'try i~ekstrapolyatory dlya stokhasticheskikh sistem s mul'tiplikativnymi shumami [Stable linear conditionally optimal filters and extrapolators for stochastic systems with multiplicative noises]. \textit{Informatika i~ee Primeneniya}~---
 \textit{Inform. Appl.}  9(1):70--75.

%\pagebreak

\bibitem{3-sin-1} %6
 \Aue{Sinitsyn, I.\,N., and E.\,R.~Korepanov}. 2015.
Sintez ustoychivykh fil'trov i~ekstrapolyatorov Pugacheva dlya sto\-kha\-sti\-che\-skikh sistem s mul'tiplikativnymi shumami [Synthesis of stable conditionally optimal filters and extrapolators for stochastic systems with multiplicative wide band noises]. \textit{Sistemy i~Sredstva Informatiki}~---
\textit{Systems and Means of Informatics}  25(1):108--126.



\bibitem{7-sin-1}
\Aue{Liptser, R.\,Sh., and A.\,N.~Shiryaev.} 1974. \textit{Statistika sluchaynykh protsessov} [Statistics of stochastic processes].  Moscow: Nauka,  476~p.

\vspace*{-3pt}

\bibitem{8-sin-1}
\Aue{Roytenberg, Ya.\,N.} 1992. \textit{Avtomaticheskoe upravlenie} [Automatic control]. 3rd ed. Moscow: Nauka. 576~p.

\end{thebibliography}

 }
 }

\end{multicols}

\vspace*{-3pt}

\hfill{\small\textit{Received March 31, 2015}}

%\vspace*{-18pt}


\Contr

\noindent
\textbf{Sinitsyn Igor N.} (b.\ 1940)~--- Doctor of Science in technology, professor, Honored scientist of RF, Head of Department, Institute of Informatics Problems, Federal Research Center ``Computer Science and Control'' of the Russian Academy of Sciences, 44-2 Vavilov Str., Moscow 119333, Russian Federation; sinitsin@dol.ru

\vspace*{3pt}

\noindent
\textbf{Korepanov Eduard R.} (b.\ 1966)~--- Candidate of Science (PhD) in technology, Head of Laboratory, Institute of Informatics Problems, Federal Research Center ``Computer Science and Control'' of the Russian Academy of Sciences, 44-2 Vavilov Str., Moscow 119333, Russian Federation; Ekorepanov@ipiran.ru

\label{end\stat}


\renewcommand{\bibname}{\protect\rm Литература}  %4
\newcommand{\erfc}[1] {\mathrm{Erfc}\,#1 }
\newcommand{\mrm}{\mathrm}

\def\stat{berezin}

\def\tit{ПРИМЕНЕНИЕ УРАВНЕНИЯ ПУГАЧЁВА--СВЕШНИКОВА К~РЕШЕНИЮ ЗАДАЧИ БАКСТЕРА О~ДЛИТЕЛЬНОСТИ ВЫБРОСОВ}

\def\titkol{Применение уравнения Пугачёва--Свешникова к~решению задачи Бакстера о~длительности выбросов}

\def\aut{С.\,В.~Березин$^1$, О.~И.~Заяц$^2$}

\index{Березин С.\,В.}
\index{Заяц О.~И.}

\def\autkol{С.\,В.~Березин, О.~И.~Заяц}

\titel{\tit}{\aut}{\autkol}{\titkol}

%{\renewcommand{\thefootnote}{\fnsymbol{footnote}} \footnotetext[1]
%{Работа выполнена при финансовой
%поддержке РФФИ (проект 15-07-02652).}}


\renewcommand{\thefootnote}{\arabic{footnote}}
\footnotetext[1]{Санкт-Петербургский политехнический университет Петра Великого, servberezin@yandex.ru}
\footnotetext[2]{Санкт-Петербургский политехнический университет Петра Великого, zay.oleg@gmail.com}


\Abst{Вычисляется закон распределения длительности выбросов случайного процесса за подвижную границу, которая перемещается равномерно с заданной скоростью. На выбросы исследуется процесс скошенного броуновского движения (СБД), являющийся одним из важнейших типовых процессов современной стохастической динамики. Этот процесс моделирует динамику броуновской частицы, которая встречает на своем пути полупрозрачный, частично отражающий экран. Задача решается методом уравнения Пу\-га\-чё\-ва-Свеш\-ни\-ко\-ва, которое определенным образом модифицируется с учетом специфических свойств СБД. Указанное уравнение решается аналитически путем сведения к соответствующей краевой задаче Римана. Также описанный метод позволяет получить ряд дополнительных практически интересных характеристик СБД, таких как вероятность невыхода за экран, закон распределения времени первого достижения экрана, моменты времени пребывания за экраном и ряд других.}

\KW{марковский процесс; уравнение Пугачёва; уравнение Пу\-га\-чё\-ва--Свеш\-ни\-ко\-ва; кра\-евая задача Римана; стохастическая механика; скошенное броуновское движение; время пребывания}

\DOI{10.14357/19922264150205}

\vspace*{-6pt}


\vskip 14pt plus 9pt minus 6pt

\thispagestyle{headings}

\begin{multicols}{2}

\label{st\stat}



\section{Введение}

В приложениях теории случайных процессов час\-то приходится иметь дело с~различного \mbox{рода} функционалами от траекторий этих процессов. Важнейшей разновидностью таких функционалов является время пребывания процесса в~заданной области. Оно определяется следующим образом. Пусть задан векторный случайный процесс~$\mathbf {U}(\tau)$, принимающий значения в~$n$-мер\-ном евклидовом пространстве~$\mathbb{R}^n$. Выберем об\-ласть пребывания~$\Omega \subset \mathbb{R}^n$ и~зафиксируем интервал наблюдения~$[0,t]$. Тогда временем пребывания процесса~$\mathbf{U}$ в~$\Omega$ будет называться лебегова мера множества тех моментов наблюдения, для которых~$\mathbf{U}$ не покидал~$\Omega$:
\begin{equation}
  \label{eq1}
  T(t)=\mathrm{mes}\, \{ \tau \in [0, t]|\ \mathbf{ U}(\tau) \in \Omega \}.
\end{equation}

Время пребывания, рассматриваемое как функция от~$t$, представляет собой некоторый случайный процесс. Задача получения закона его распределения (или хотя бы моментов этого распределения) возникает во многих приложениях. Так, в теории надежности область~$\Omega$ отвечает нормативному, штатному режиму работы системы, а~выход из нее~--- переходу к аварийному, форс-ма\-жор\-но\-му режиму. При <<жесткой>> трактовке условий надежности первый же выброс за пределы~$\Omega$ влечет за собой отказ системы, прекращение ее работы. Однако условия надежности можно ставить и в более <<мягкой>> форме, когда отдельные выбросы из~$\Omega$ допускаются, но их суммарная длительность должна быть огра\-ни\-чена.

Функционалы вида~(\ref{eq1}) играют важную роль и~в~стохастической финансовой математике. Платеж\-ные функции ряда нестандартных (экзотических) опционов выражаются через время пребывания курсовой стоимости выше или ниже определенного уровня~\cite{ref1}. Такие же функционалы возникают при решении задач теории массового обслуживания методом диффузионной аппроксимации~\cite{ref2}. В~этом случае приходится вычислять время пребывания винеровского процесса на отрезках единичной длины, ограниченных натуральными числами.

Отметим, что в перечисленных, а также во многих других, подобных им приложениях обычно требуется найти закон распределения~$T$ аналитически, так как этот закон часто служит лишь промежуточным результатом для последующих вычислений. Для процессов~$\mathbf{U}$ общего вида решить такую задачу крайне затруднительно, и приходится ограничивать класс рассматриваемых процессов. В~частности, для гауссовских процессов разработана асимптотическая теория, основанная на предположении, что выбросы за пределы~$\Omega$ являются редкими событиями~\cite{ref3}.

Между тем задачу можно, в принципе, решить точно, если считать~$\mathbf{U}$ непрерывным марковским процессом~\cite{ref4}. Время пребывания~$T$ тогда можно представить в виде решения дифференциального уравнения

\vspace*{2pt}

\noindent
\begin{equation}
  \label{eq2-b}
  \dot T = I_\Omega (\mathbf{U})\,,\quad T(0)=0\,,
\end{equation}

\vspace*{-2pt}

\noindent
где

\noindent
\begin{equation*}
  I_\Omega(\mathbf{x}) =
  \begin{cases}
  1\,, &\ \mathbf{x} \in \Omega\,;\\
  0\,, &\ \mathbf{x} \notin \Omega
\end{cases}
\end{equation*}
обозначает индикаторную функцию множества~$\Omega$. Расширенный процесс~$(\mathbf{U}, T)$ также будет являться марковским, а распределение одной из его компонент~$T$ можно найти, решая соответствующее уравнение Колмогорова.

Последнее обычно оказывается достаточно сложным по своему виду, поэтому применяют и~альтернативный подход, основанный на дискретизации времени, переходе к~схеме случайных блуж\-да\-ний и~предельном переходе, когда шаг блужданий стремится к~нулю~\cite{ref5}.

В настоящее время аналитическое выражение плотности распределения~$f(y; t)$ времени пребывания~$T$ получено лишь для небольшого числа одномерных процессов. При переходе к векторным процессам приходится довольствоваться асимптотическими
и~приближенными формулами~\cite{ref6}.

Наиболее детально изучена задача о времени пребывания одномерного марковского процесса~$U$ на положительной полуоси, когда~(\ref{eq1}) имеет вид:
\begin{equation}
  \label{eq3-b}
  T(t)=\mathrm{mes}\, \{ \tau \in [0, t]|\ U(\tau)>0 \}\,.
\end{equation}
Впервые такую задачу для винеровского процесса решил в 1939~г.\ П.~Леви~\cite{ref7}, который открыл знаменитый закон арксинуса

\noindent
\begin{equation}
  \label{eq4-b}
  f(y;t) = \fr{1}{\pi \sqrt{y(t-y)}} \qquad (0<y<t)\,.
\end{equation}

На этот результат самое пристальное внимание обратил В.~Феллер. Он усовершенствовал вывод~(\ref{eq4-b}), рассмотрел ряд смежных задач, подробно проинтерпретировал этот закон, проверил его экспериментально и~изложил весь круг связанных с~ним вопросов в~известной монографии~\cite{ref5}. С~учетом всего сказанного задачу о~времени пребывания~(\ref{eq3-b}) на положительной полуоси иногда называют задачей Феллера для процесса~$U$.

Помимо случая стандартного винеровского процесса решение этой задачи в настоящее время получено также для винеровского процесса с постоянным сносом. На эту тему имеется целый цикл публикаций, включающий работы
Дж.~Акахори~\cite{ref10}, А.~Дассиоса~\cite{ref9}, М.~Йора~\cite{ref11},  Л.~Такача~\cite{ref8}, А.~Пехт\-ля~\cite{ref12}, ориентированные, в~основном, на приложения к стохастической финансовой математике и~опубликованные в~конце 1990-х~гг.

\columnbreak

В настоящей статье будет рассмотрена иная, более сложная задача, касающаяся времени пребывания
\begin{equation*}
%  \label{eq5}
  T(t)=\mathrm{mes}\, \{ \tau \in [0, t]|\ U(\tau) > b \tau \} \qquad (b > 0)
\end{equation*}
заданного одномерного процесса~$U$ на полуоси~$(b \tau, +\infty)$, граница которой движется равномерно с постоянной скоростью~$b \geqslant 0$. Идея постановки подобной задачи принадлежит Г.~Бакстеру~\cite{ref13}, который впервые решил ее для винеровского процесса еще в~1956~г.

Отметим, что задача Феллера для винеровского процесса с постоянным сносом~[8--12] фактически эквивалентна этой {\it задаче Бакстера} для стандартного винеровского процесса~\cite{ref13}. Винеровский процесс с~постоянным сносом отличается от последнего как раз на линейную функцию времени. Пересечение винеровским процессом с~таким сносом нулевого уровня означает выброс самого винеровского процесса за подвижный линейно рас\-ту\-щий барьер. В~обоих случаях речь идет об одних и~тех же выбросах.

Как было отмечено ранее, обычно применяемые методы решения задач на время пребывания основаны на использовании либо уравнения Колмогорова, либо теории случайных блужданий. Насколько известно авторам, получить с~их помощью решение задачи Феллера для ка\-ких-ли\-бо других процессов~$U$, помимо разобранных в~работах~[7--13], до последнего времени не удавалось, между тем задачи эти интересны и~значимы для многих важных приложений. Поэтому актуальной является разработка альтернативных методов решения.


\section{Уравнение Пугачёва--Свешникова}

Одним из авторов настоящей статьи ранее был предложен метод решения задачи Феллера, основанный на использовании уравнения Пу\-га\-чё\-ва--Свеш\-ни\-ко\-ва~\cite{ref14}. Последнее представляет \mbox{собой} специальный частный случай хорошо известного уравнения Пугачёва для характеристической функции непрерывного марковского процесса~\cite{ref22}. При преобразовании уравнения Пугачёва к форме, предложенной Свешниковым, оно приобретает вид сингулярного интегродифференциального уравнения типа свертки. Вполне естественно, что такое специальное преобразование требует некоторого сужения класса рассматриваемых нелинейных систем.

Сам Свешников первоначально ограничился лишь системами, включающими нелинейности релейного типа~\cite{ref25,ref26}. Свое уравнение Свешников предложил решать приближенно в~качестве альтернативы метода стохастической линеаризации. Впоследст\-вии было установлено, что уравнение Пу\-га\-чё\-ва--Свеш\-ни\-ко\-ва сохраняет силу не только в~классе систем релейного типа, но также и~для некоторых ку\-соч\-но-ли\-ней\-ных систем, в~част\-ности для систем, линейных в~полупространствах~\cite{ref15} и~четвертях пространства~\cite{ref15a}. Кроме того, выяснилось, что это уравнение допускает не только приближенное, но в~целом ряде случаев также и~точное аналитическое решение путем сведения к решению соответствующей краевой задачи типа Римана.

Если исследуемый случайный процесс~$U$ явля\-ется компонентой векторного марковского процесса, описываемого системой ку\-соч\-но-ли\-ней\-ных стохастических дифференциальных уравнений,\linebreak линейной в~полупространствах или четвертьпространствах, то после добавления к этой системе уравнения типа~(\ref{eq2-b}), задающего время пребывания процесса~$U$ на положительной полуоси, приходим к~ку\-соч\-но-ли\-ней\-ной системе именно того вида, который был подробно разобран в~\cite{ref15, ref15a}. Это, в~свою очередь, позволяет решать и~со\-от\-вет\-ст\-ву\-ющую задачу Бакстера. Действительно, эта задача для процесса~$U$ сводится к~решению задачи Феллера для вспомогательного процесса~$V$, задаваемого условием~$V(t)\hm = U(t)\hm - b t$. Если в уравнениях движения исходной задачи выразить~$U$ через~$V$, то вновь полученная система по-прежнему будет принадлежать к~классу ку\-соч\-но-ли\-ней\-ных сис\-тем вида~\cite{ref15, ref15a}, а~значит, допускает применение метода уравнения Пу\-га\-чё\-ва--Свеш\-ни\-кова.

Указанным методом в работе~\cite{ref14} была решена задача Феллера для винеровского процесса, процесса Г.~Уленбека\,--\,Л.~Орнштейна, а~также процесса Т.~Кохи\,--\,Дж.~Дин\-за~\cite{ref16}, играющего важную роль в стохастической механике. Впоследствии независимо от работ~[8--12] методом~\cite{ref14} была решена также и рассмотренная в указанных работах задача Феллера для винеровского процесса с постоянным сносом~\cite{ref17}. В~последнее время также получено решение задачи Феллера для некоторого обобщения процесса Ко\-хи--Дин\-за~\cite{ref15a}.

Далее вначале рассмотрим применение описанного метода к~решению классической задачи Бакстера~\cite{ref13}, касающейся самого винеровского процесса, а~затем обратимся к~более сложному примеру.


\section{Решение классической задачи Бакстера}

Пусть~$W(t)$ обозначает стандартный винеровский процесс. Получим закон распределения суммарного времени, которое для~$0 \hm\leqslant \tau \hm\leqslant t$ проведет процесс~$U(\tau) \hm= a W(\tau)$ выше возрастающего по линейному закону переменного уровня~$b \tau$, где~$a$ и~$b$~--- заданные положительные постоянные.

Положим
\begin{equation*}
%  \label{eq6}
  V(t) = \fr{\sqrt{2}}{a} \left(U(t) - b t\right)\,,\ \mu = \fr{b}{\sqrt{2}a}\,.
\end{equation*}
Тогда задача сведется к исследованию двумерного марковского процесса
\begin{equation}
  \label{eq7}
  \left.
    \begin{aligned}
      &dV = -2 \mu\ dt + \sqrt{2}\ dW\,;\\
      &dT = \fr{1}{2} \left(1 + \mathrm{sign}\,
      V\right)\ dt
    \end{aligned}
  \right\}
\end{equation}
при нулевых начальных условиях
\begin{equation*}
%  \label{eq8}
  V(0) = 0\,;\quad T(0) = 0\,.
\end{equation*}

Вторая компонента~$T$ процесса~(\ref{eq7}) дает длительность выбросов его первой компоненты~$V$ за нулевой уровень, т.\,е.\ совпадает с~дли\-тель\-ностью выбросов исходного процесса~$U$ за линейную границу~$b \tau$.

Тем самым задача Бакстера для процесса~$U$ сведена к задаче Феллера для винеровского процесса~$V$ с постоянным сносом~$-2 \mu$. Последняя задача подробно разобрана в~\cite{ref17}, что позволяет получить следующее обобщение закона арксинуса.

\smallskip

\noindent
\textbf{Утверждение 1.}
\textit{Плотность распределения времени пребывания~$T(t)$ процесса~$U(\tau)$ выше подвижной границы, растущей по линейному закону~$b \tau$, дается формулой
  \begin{multline}
    \label{eq9}
    f(y;t) = \left( \fr{e^{-\mu^2 y}}{\sqrt{\pi y}} - \mu\ \mathrm{Erfc}\left(\mu \sqrt{y}\right) \right)\times{}\\
     {}\times\left( \fr{e^{-\mu^2 (t-y)}}{\sqrt{\pi (t-y)}} + \mu\ \mathrm{Erfc}\left(-\mu \sqrt{t-y}\right) \right),
  \end{multline}
  причем здесь
  \begin{equation*}
%    \label{eq10}
    \erfc{x} = \fr{2}{\sqrt{\pi}} \int \limits_{x}^{+\infty} e^{-s^2}\, ds
  \end{equation*}
  обозначает дополнительную функцию ошибок}.


\smallskip
Вполне естественно, что найденное распределение совпадает с решением Бакстера~\cite{ref13}, а при~$\mu \hm= 0$ оно переходит в закон арксинуса~(\ref{eq4-b}).

Метод получения плотности распределения~(\ref{eq9}), основанный на решении уравнения Пу\-га\-чё\-ва--Свеш\-ни\-ко\-ва, изложен в~\cite{ref17} и~допускает обобщение на ряд процессов более общего вида. Один из примеров такого решения приводится в~следующих разделах статьи и~основывается на понятии СБД, которое по этой причине необходимо разобрать более подробно.

\section{Скошенное броуновское движение}

\vspace*{-3pt}

Скошенное броуновское движение  впервые возникло в статье~\cite{ref18} как обобщение процесса броуновского движения c отражением. Позже было доказано, что СБД описывает движение броуновской частицы при наличии полупроницаемого частично отражающего экрана. Этот процесс не только интересен с математической точки зрения, но и весьма полезен для экономических, биологических, астрономических и разнообразных физических приложений~\cite{ref19}.

Существует несколько подходов к определению СБД~\cite{ref19}, из них для целей настоящей статьи наиболее удобен тот, который использует стохастическое дифференциальное уравнение, включающее локальное время. Рассмотрим одномерное прямолинейное движение частицы. В~точке с~координатой~$x$ расположим полупроницаемый экран. До момента достижения экрана движение частицы будем считать броуновским. Если, достигнув экрана, частица имела положительную скорость, то она свободно пересечет его с~вероятностью~$\beta$, а~с~вероятностью~$1-\beta$ упруго отразится от экрана. При противоположном направлении выхода частицы на экран вероятности пересечения и~отражения соответственно равны~$1-\beta$ и~$\beta$,
т.\,е.\ являются дополнительными для соответствующих вероятностей в~прямом направлении. В~такой ситуации будем называть {\it скошенным броуновским движением} сильное решение стохастического дифференциального уравнения (СДУ)~\cite{ref19}

\vspace*{2pt}

\noindent
\begin{equation}
  \label{eq11}
  dU =(2\beta -1)h^2\ dL^x_U +  h\, dW\,,
\end{equation}

\vspace*{-2pt}

\noindent
где~$W$~---~стандартный винеровский процесс, а~$L^x_U$~--- процесс симметричного локального времени~\cite{ref18} для~$U$ на уровне~$x$, задаваемый формулой:

\noindent
\begin{equation*}
%  \label{eq12}
  L^x_U(t) = \lim \limits_{\varepsilon \to +0} \fr{1}{2\varepsilon}\, \mathrm{mes}\, \{\tau \in [0, t]|\  U(\tau) \in [x- \varepsilon, x+ \varepsilon]\}.
\end{equation*}
Отметим, что данное выше определение СБД корректно в силу доказанной
в~\cite{ref20} теоремы существования и единственности решения уравнения~(\ref{eq11}).

Уравнение~(\ref{eq11}) отличается от соответствующего уравнения для винеровского процесса лишь слагаемым, включающим локальное время. Добавляя подобные же слагаемые и в другие, более сложные уравнения, получаем возможность определить различные обобщения СБД. Одним из простейших таких обобщений является {\it СБД с~постоянным сносом}~\cite{ref21}, получаемое добавлением слагаемого с локальным временем в уравнение броуновского движения с постоянным сносом:

\vspace*{2pt}

\noindent
\begin{equation}
  \label{eq13}
  dU =c\ dt + (2\beta -1)h^2\ dL^x_U + h\,dW\,.
\end{equation}

\noindent
Это слагаемое, как и ранее, моделирует полупроницаемый частично отражающий экран, помещенный в точку~$x$.

При формулировке задачи Бакстера применительно к процессу~(\ref{eq13}) будем теперь считать, что положение частично проницаемого экрана меняется во времени~$t$ по линейному закону~$x(t) \hm= a \hm+ b t$, где~$a > 0$ и~$b \hm\geqslant 0$. Тогда сама задача Бакстера (задача определения длительности выбросов процесса~$U$ за такой подвижный экран) сводится к анализу процесса
\begin{equation}
  \left.
    \begin{array}{rl}
      dU &= c\ dt + (2\beta-1)h^2\ dL^{a + b t}_U + h\ dW\,;\\[6pt]
      dT &=  \fr{1}{2}(1+ \mathrm{sign}\left(U-a-bt\right))\ dt
    \end{array}
  \right\}
    \label{eq14}
\end{equation}
при начальных условиях~$U(0) \hm= 0$ и $T(0)\hm=0$. Необходимо изучить характеристики компоненты~$T$ расширенного процесса~$(U, T)$.

Интересными частными случаями являются~$\beta \hm= 0$ и~1, которые соответствуют упруго отражающему экрану и экра\-ну-ло\-вуш\-ке, пропускающему только частицы, движущиеся с положительной скоростью. В~случае~$\beta \hm= {1}/{2}$ экран вообще не влияет на частицу и~$U$ является винеровским процессом (броуновским движением) с постоянным сносом~\cite{ref17}.

Вводя обозначения
\begin{gather*}
V = \fr{\sqrt{2}}{h}\left(U - a - b t\right)\,;\quad
 \mu  = \fr{b - c}{\sqrt{2}h}\,;\\
 \eta = 2 \beta -1\,;\quad v_0 = -\fr{a \sqrt{2}}{h} < 0\,,
 \end{gather*}
  получим
\begin{equation}
  \left.
    \begin{array}{rl}
      dV &= -2 \mu\ dt + 2\eta\ dL^0_{V} + \sqrt{2}\,dW\,;\\[6pt]
      dT &=  \fr{1}{2}\left(1+ \mathrm{sign}\,V\right)\,dt
    \end{array}
      \right\}
        \label{eq15}
\end{equation}
при начальных условиях
\begin{equation*}
%  \label{eq16}
  V(0) = v_0\,;\quad T(0)=0\,.
\end{equation*}

Система уравнений~(\ref{eq15}) относится к классу ку\-соч\-но-ли\-ней\-ных стохастических систем, линейных в полупространствах, но отличается от стандартных уравнений этого класса~\cite{ref14} наличием локального времени пребывания на границе областей линейности. Несмотря на указанное усложнение постановки задачи, ее решение по-прежнему может быть получено методом, описанным в~\cite{ref14}.

\section{Решение задачи в изображениях}

Повторяя вывод уравнения Пу\-га\-чё\-ва--Свеш\-ни\-ко\-ва~\cite{ref15} с учетом дополнительного слагаемого, содержащего локальное время, получим следующую модификацию указанного уравнения.

\smallskip

\noindent
\textbf{Утверждение 2.}
\textit{Характеристическая функция~$E(z_1, z_2;t)$ системы ординат~$(V(t), T(t))$ процесса}~(\ref{eq15}) \textit{подчиняется уравнению}
  \begin{multline}
    \label{eq17}
    \!\fr{\partial E(z_1,z_2;t)}{\partial t} = -\left( z_1^2 + 2i  \mu  z_1 - \fr{i z_2}{2}\right) E(z_1, z_2; t) +{}\\
     {}+\fr{z_2}{2 \pi}\, \mathrm{v.p.}\ \int \limits_{-\infty}^{+\infty} \fr{E(s, z_2; t)}{s-z_1}\,ds + 2i\eta z_1\ \Phi(z_2; t)
  \end{multline}
  \textit{при начальном условии~$E(z_1, z_2; 0)\hm = e^{i v_0 z_1}$, где функция~$\Phi(z_2; t)$ дается интегралом}
  \begin{equation}
    \label{eq18}
    \Phi(z_2; t) = \fr{1}{2 \pi}\, \mathrm{v.p.}\ \int \limits_{-\infty}^{+\infty} E(s, z_2; t)\, ds\,.
  \end{equation}

Последнее слагаемое в правой части~(\ref{eq17}), отсутствовавшее в~статьях~\cite{ref14,ref15,ref15a,ref17}, появилось из-за наличия в~уравнениях движения~(\ref{eq15}) локального времени. Отметим, что интерпретация интеграла~(\ref{eq18}) в~смысле главного значения по Коши принципиально важна. Как будет показано позже, в~классическом смысле этот интеграл расходится, так как имеет неинтегрируемую особенность на бесконечности.

Метод решения уравнения~(\ref{eq17}) аналогичен~\cite{ref14} и основывается на переходе c помощью формул Ю.\,В.~Сохоцкого к краевой задаче Римана теории функций комплексного переменного. Формулы Сохоцкого имеют вид:
\begin{equation}
\left.
  \begin{array}{rl}
F^+(z_1, z_2; t) - F^-(z_1, z_2; t) &= E(z_1, z_2; t)\,;\\[6pt]
  F^+(z_1, z_2; t) + F^-(z_1, z_2; t) &={}\\
   &\hspace*{-21mm}{}=\displaystyle\fr{1}{\pi i}\, \mathrm{v.p.}\,\displaystyle\int \limits_{-\infty}^{\infty} \fr{E(s, z_2; t)}{s-z_1}\, ds\,.
  \end{array}
  \right\}
    \label{eq19}
\end{equation}
Здесь~$(z_1, z_2) \hm\in \mathbb{R}^2$, а функции~$F^+$ и~$F^-$ являются аналитическими по аргументу~$z_1$ соответственно в верхней и нижней полуплоскостях расширенной комплексной плоскости.

Применяя преобразование Лапласа по~$t$, обозначая аргумент этого преобразования через~$p$, а~изоб\-ра\-же\-ния~--- той же буквой, что и~оригиналы, но с~волной сверху, с~по\-мощью~(\ref{eq19}) перейдем от уравнения~(\ref{eq17}) относительно~$E$ к~краевой задаче Римана относительно
изображений~$\tilde{F}^+$ и~$\tilde{F}^-$ по Лапласу.


\smallskip

\noindent
\textbf{Утверждение 3.}
\textit{Задача Коши для уравнения}~(\ref{eq17}) \textit{эквивалентна краевой задаче Римана, заключающейся в~нахождении пары функций~$\tilde F^\pm$, аналитических по~$z_1$ соответственно в верхней и нижней полуплоскостях и удовлетворяющих при~$\mathrm{Im}\, z_1 \hm= 0$ краевому условию}
  \begin{multline}
    \left(z_1^2 +2i \mu  z_1 + p - i z_2\right) \tilde F^+ = {}\\
    \hspace*{-5mm}{}=\left(z_1^2 +2i \mu  z_1 + p\right) \tilde F^- +
    2 i \eta z_1 \tilde \Phi(z_2;p) + e^{i v_0 z_1},\!\!
        \label{eq20}
  \end{multline}
 \textit{где~$\tilde F^\pm(z_1, z_2; p)$ и~$\tilde \Phi(z_2; p)$ обозначают изображения функций~$F^\pm(z_1, z_2; t)$ и~$\Phi(z_2; t)$ соответственно, причем все изображения существуют при}~$\mathrm{Re}\,{p} > 0$.

 \smallskip


Обратим внимание, что задача~(\ref{eq20}) отличается от аналогичных краевых задач~\cite{ref14,ref15,ref15a,ref17} появлением в краевом условии слагаемого, содержащего~$\tilde \Phi$, связь которого с искомыми краевыми значениями~$\tilde F^\pm$ задается формулами~(\ref{eq18}) и~(\ref{eq19}).

Повторение рассуждений статьи~\cite{ref14} позволяет фактически решить задачу~(\ref{eq20}).

\smallskip

\noindent
\textbf{Утверждение 4.}
  \textit{Краевая задача Римана}~(\ref{eq20}) \textit{эквивалентна обратной задаче подбора функций~$\tilde G_0$ и~$\tilde G_1$, исходя из условий аналитичности по аргументу~$z_1$ правых частей каждого из выражений}
  \begin{equation}
  \left.
  \begin{array}{rl}
     \tilde F^+(z_1, z_2; p) &={}\\[3pt]
      {}&\hspace*{-22mm}=\displaystyle\fr{\tilde G_0(z_2; p) + (\tilde G_1(z_2; p) + i \eta \tilde \Phi(z_2; p))z_1}{z_1^2 +2i \mu   z_1 + p - i z_2}\,;\\[7pt]
     \tilde F^-(z_1, z_2; p) &= {}\\[3pt]
&\hspace*{-24.5mm}{}=\displaystyle\fr{\tilde G_0(z_2; p) \!+\! (\tilde G_1(z_2; p) \!- i \eta \tilde \Phi(z_2; p))z_1 \!-\! e^{i v_0 z_1}}{z_1^2 +2i \mu   z_1 + p}\!    \end{array}\!
    \right\}\!\!\!
        \label{eq21}
  \end{equation}
 \textit{соответственно в верхней и нижней полуплоскостях}.


\smallskip

Обозначая корни знаменателей~(\ref{eq21}) через~$\varkappa^\pm \hm= i( -\mu  \pm \sqrt{ \mu ^2 +p - i z_2})$ и~$\nu^\pm\hm= i( -\mu  \pm \sqrt{ \mu ^2 + p})$, находим условия аналитичности~$\tilde F^\pm$ в виде:
\begin{equation}
\left.
  \begin{array}{l}
\hspace*{-2mm}\tilde G_0\left(z_2; p\right) + \left(\tilde G_1(z_2; p)+i\eta \tilde \Phi(z_2; p)\right)    \varkappa^+ = 0\,;\\[3pt]
\hspace*{-2mm}\tilde G_0\left(z_2; p\right) + \left(\tilde G_1(z_2; p)-i\eta \tilde \Phi(z_2; p)\right)\nu^- = {}\\
\hspace*{52mm}{}= e^{i v_0 \nu^-}\!.
  \end{array}\!
  \right\}\!\!
    \label{eq22}
\end{equation}
Исключение~$\tilde G_0$ из~(\ref{eq21}) и~(\ref{eq22}) дает
\begin{equation}
\left.
  \begin{array}{l}
   \tilde F^+\left(z_1,z_2; p\right) = \fr{\tilde G_1(z_2; p)+i\eta \tilde \Phi(z_2; p)}{z_1-\varkappa^-};\\[6pt]
\tilde F^-\left(z_1,z_2; p\right) = \fr{\tilde G_1(z_2; p)-i\eta \tilde \Phi(z_2; p)}{z_1-\nu^+} + {}\\[6pt]
   \hspace*{31mm}{}+\fr{e^{i v_0 \nu^-} - e^{i v_0 z_1}}{z_1^2 +2i \mu   z_1 + p}\,;\\[6pt]
\left(\nu^- - \varkappa^+\right)\tilde G_1\left(z_2; p\right) - {}\\[6pt]
\hspace*{8mm}{}-i \eta \left(\nu^- + \varkappa^+\right) \tilde \Phi\left(z_2; p\right)= e^{i v_0 \nu^-}.
  \end{array}
\right\}
    \label{eq23}
\end{equation}

Далее с помощью формул~(\ref{eq19}) и~(\ref{eq23}) находим выражение для~$\tilde E$ и подставляем его в~(\ref{eq18}). Разрешая полученные уравнения относительно~$\tilde \Phi$, получаем
\begin{equation*}
%  \label{eq24}
  \tilde \Phi\left(z_2;p\right) = -i \tilde G_1\left(z_2;p\right)\,.
\end{equation*}
Отметим, что интегралы от~$\tilde F^\pm$ по~$z_1$, вообще говоря, расходятся в обычном смысле, но тем не менее существуют в смысле главного значения, а именно это и предполагалось в формуле~(\ref{eq18}). В~итоге приходим к окончательным выражениям для введенных изображений:

\noindent
\begin{equation}
\left.
  \begin{array}{rl}
   \tilde G_1  \left(z_2; p\right) &= \fr{e^{i v_0 \nu^-}}{(1- \eta)\nu^- - (1+\eta)\varkappa^+}\,;\\
   \tilde F^+\left(z_1, z_2; p\right) &= \fr{(1+\eta)\tilde G_1(z_2; p)}{z_1-\varkappa^-}\,;\\
   \tilde F^-\left(z_1, z_2; p\right) &={}\\
    &\hspace*{-15mm}{}=\fr{(1-\eta)\tilde G_1(z_2; p)}{z_1-\nu^+} + \fr{e^{i v_0 \nu^-} - e^{i v_0 z_1}}{z_1^2 +2i \mu   z_1 + p}\,.
  \end{array}
\right\}
    \label{eq25}
\end{equation}
Далее изображение искомой характеристической функции очевидным образом находится по первой формуле Сохоцкого~(\ref{eq19}).
\smallskip

\noindent
\textbf{Утверждение 5.}
\textit{Изображение~$\tilde E$ характеристической функции системы компонент~$(V(t), T(t))$ процесса}~(\ref{eq15}) \textit{выражается в виде}:

\noindent
  \begin{multline}
    \label{eq26}
    \tilde E(z_1, z_2; p) = \left[\fr{1+\eta}{z_1-\varkappa^-} - \fr{1-\eta}{z_1-\nu^+}\right] \tilde G_1(z_2; p) - {}\\
    {}-\fr{e^{i v_0 \nu^-} - e^{i v_0 z_1}}{z_1^2 +2i \mu   z_1 + p}\,,
  \end{multline}
  \textit{где~$\tilde G_1(z_2; p)$ определяется согласно}~(\ref{eq25}).


\smallskip

Полагая здесь~$z_1\hm=0$, получаем изображение маргинальной характеристической функции~$T$.

\smallskip

\noindent
\textbf{Следствие 1.}
  Преобразование Лапласа для характеристической функции искомого времени пребывания~$T$ имеет вид:

  \noindent
  \begin{multline}
    \tilde E_{T}(z_2; p) ={}\\
     \hspace*{-6mm}{}=\fr{(1-\eta)\varkappa^- - (1+\eta)\nu^+}{(1- \eta)\nu^- - (1+\eta)\varkappa^+} \,
     \fr{e^{i v_0 \nu^-}}{\varkappa^- \nu^+} - \fr{e^{i v_0 \nu^-} - 1}{p}\,.\!\!
         \label{eq27}
  \end{multline}

Функция~(\ref{eq27}) представляет собой двукратное преобразование плотности вероятности~$f(y; t)$: по Фурье (по аргументу~$y$) и по Лапласу (по аргументу~$t$). Получению соответствующих оригиналов посвящен следующий раздел.

\vspace*{-4pt}

\section{Решение задачи в оригиналах}

Полученное в предыдущем разделе изображение~(\ref{eq27}) характеристической функции~$\tilde E_T$ за исключением табличного последнего слагаемого, пропорционального~$1/p$, представляет собой произведение экспоненты от аргумента~$\sqrt{\mu^2 + p}$ и~рациональной функции от аргументов~$\sqrt{\mu^2 + p}$ и~$\sqrt{\mu^2 +p - i z_2}$. Такая структура двукратного\linebreak
 преобразования Фурье--Лап\-ла\-са гарантирует по лучение явного выражения оригинала~$f(y;t)$ в~квад-\linebreak\vspace*{-12pt}

 \columnbreak

 \noindent
рату\-рах. Действительно, пользуясь методом двукратного преобразования Лапласа, подробно {описанным} в~\cite{ref17}, общими свойствами преобразования Лапласа и~обобщенной теоремой умножения Эфроса~\cite{ref23}, можно свести исходную задачу обращения к~задаче обращения двукратного преобразования Лапласа, представляющего собой рациональную дробь от аргументов~$q \hm= -i z_2$ и~$p$, которое уже является табличным. Опуская промежуточные выкладки, приходим к следующему конечному результату.

\smallskip

\noindent
\textbf{Утверждение 6.}
\textit{Плотность вероятности времени пребывания~$f(y; t)$ второй компоненты процесса}~(\ref{eq14}) \textit{при~$0\hm<y\hm<t$ дается выражением}:

\noindent
  \begin{multline*}
      f(y; t) = \left[ 1 - \fr{1}{2} \left( e^{-2 v_0 \mu^- } \erfc{ \left( -\fr{v_0 - 2 |\mu| t}{2 \sqrt{t}} \right)} +{}\right.\right.\\
       \left.\left.{}+e^{2 v_0 \mu^+ } \erfc{ \left( -\fr{v_0 + 2 |\mu| t}{2 \sqrt{t}} \right)} \right) \right] \delta(y) +{}\\
      {}+\fr{e^{- \mu^2 t + v_0 \mu}} {4 \pi (y(t-y))^{3/2}}
\left[
%\vphantom{\int\limits_{-v_0}^{+ \infty}}
 \fr{1-\eta}{1 + \eta}
 \int \limits_0^{+ \infty} \int \limits_{-v_0}^{+ \infty} \chi^+(s_1, s_2)\times{}\right.\\
 {}\times e^{\mu  (s_1 + v_0)+(({\mu (3 \eta -1)})/({1 + \eta})) s_2}\ ds_1 ds_2 +{} \\
{} +  \fr{1 + \eta}{1 - \eta}
\int \limits_0^{+ \infty}
\int\limits_{-v_0}^{+ \infty} \chi^-\left(s_1, s_2\right)\times{}\\
\left.{}\times
e^{ (( \mu (3 \eta +1))/( 1-\eta))  \left(s_1 + v_0\right) - \mu  s_2} \, ds_1 ds_2  \vphantom{\int\limits_{-v_0}^{+ \infty}}\right]\,;
\end{multline*}

\vspace*{-12pt}

\noindent
\begin{multline}
     \chi^\pm(s_1, s_2) ={}\\
      {}=\fr{1 \pm \mathrm{sign}\left[(1 + \eta)(s_1+v_0)-(1 - \eta )s_2\right]}{2}\times{}\\
      {}\times s_1 s_2\ e^{- {s_1^2}/(4(t-y)) -{s_2^2}/(4y)}\,,
    \label{eq28}
  \end{multline}
  \textit{где~$\delta(y)$ обозначает дель\-та-функ\-цию Дирака, а~$\mu^\pm \hm= (|\mu| \pm \mu)/2$~--- положительную и отрицательную части числа}~$\mu$.


\smallskip

В частном случае, когда~$\eta \hm= 0$ и~$v_0 \hm= 0$, дельтообразное слагаемое в~(\ref{eq28}) исчезает, а двукратный интеграл преобразуется к виду обычного закона арксинуса~(\ref{eq9}).

Другой важный частный случай, от\-ве\-ча\-ющий~$\mu \hm= 0$ и нулевому начальному условию~$v_0 = 0$, описывает ситуацию, при которой постоянный снос совпадает со скоростью движения экрана. При этом интеграл~(\ref{eq28}) выражается в конечном виде через элементарные функции

\noindent
\begin{multline*}
%  \label{eq29}
  f(y; t) = \fr{(1 - \eta ^2)t}{\pi  \sqrt{y (t - y)} ((1 + \eta)^2t -4 \eta y)} \\
  (0<y<t)\,.
\end{multline*}

\begin{center}  %fig1
\vspace*{-1pt}
\mbox{%
 \epsfxsize=79.301mm
 \epsfbox{ber-1.eps}
 }

\end{center}

%\vspace*{3pt}

\noindent
{{\figurename~1}\ \ \small{График~$f_{\mathcal{T}}(y)$ при различных значениях~$\eta$ ($\mu \hm= 0$): \textit{1}~--- $\eta=-0{,}5$; \textit{2}~--- 0; \textit{3}~--- $\eta\hm=0{,}5$}}



\vspace*{18pt}


\addtocounter{figure}{1}





\noindent
Последняя формула, разумеется, при~$\eta \hm= 0$ переходит в известный закон арксинуса~(\ref{eq4-b}).

Перейдем к безразмерному времени~$\mathcal {T} \hm= T/t$, выраженному в долях интервала наблюдения~$t$, его плотность вероятности~$f_{\mathcal{T}}(y) = t f(ty; t)$ не зависит от~$t$:

\noindent
\begin{multline}
  \label{eq30}
  f_{\mathcal {T}}(y) = \fr{1 - \eta ^2}{\pi  \sqrt{y (1 - y)} ((1 + \eta)^2 -4 \eta y)} \\
  (0<y<1)\,.
\end{multline}
График~$f_{\mathcal T}$ представлен на рис.~1.



Нетрудно убедиться, что доля среднего время, проведенного частицей за движущимся полупроницаемым частично отражающим экраном, дается равенством

\noindent
$$
M[\mathcal{T}] = \fr{\eta + 1}{2} = \beta\,.
 $$
 Этот факт в некотором смысле отражает эргодическую природу рассматриваемого процесса, коль скоро параметр~$\beta$ был определен ранее как вероятность проникновения частицы через экран.

Рассмотрев более общий случай~$v_0 \hm= 0$, $\mu \hm\ne 0$. Получим:

\noindent
\begin{multline*}
\hspace*{-7.21829pt}f(y; t) =\fr{e^{- \mu^2 t}} {4 \pi (y(t-y))^{3/2}}
   \left[ \fr{1-\eta}{1 + \eta} \int\limits_0^{+ \infty} \int\limits_{0}^{+ \infty} \!\!\!\chi^+\left(s_1, s_2\right) \times{}\right.\\
   {}\times e^{\mu s_1 +(({\mu (3 \eta -1)})/({1 + \eta})) s_2}\, ds_1 ds_2 +{} \\
{}  +  \fr{1 + \eta}{1 - \eta} \int\limits_0^{+ \infty}
\int\limits_{0}^{+ \infty}\! \chi^-\left(s_1, s_2\right)\times{}\\
\left.{}\times
e^{ (({ \mu (3 \eta +1) })/(1-\eta)) s_1 - \mu  s_2 }\ ds_1 ds_2  \right].
% \label{eq30a}
\end{multline*}
Соответствующий график для плотности вероятности безразмерного времени~$f_{\mathcal {T}}(y; t)\hm = t f(ty; t)$ представлен на рис.~2.

\columnbreak

\begin{center}  %fig2
\vspace*{-1pt}
\mbox{%
 \epsfxsize=78.901mm
 \epsfbox{ber-2.eps}
 }

\end{center}

\vspace*{1pt}

\noindent
{{\figurename~2}\ \ \small{График~$f_{\mathcal{T}}(y; t)$ при различных значениях~$\eta$ ($\mu \hm= -1, t = 1$): \textit{1}~--- $\eta=-0{,}5$; \textit{2}~--- 0; \textit{3}~--- $\eta\hm=0{,}5$}}



\vspace*{18pt}


\addtocounter{figure}{1}




Обратим внимание, что в отличие от выражения~(\ref{eq30}),
плотность $f_{\mathcal{T}}(y; t)$ теперь уже зависит от~$t$, что очень хорошо иллюстрируется формулой~(\ref{eq35}), которая будет получена в~следующем разделе настоящей статьи.

\section{Дальнейшие результаты}

В этом разделе будет приведен ряд полезных и~качественно интересных результатов, которые удается получить без трудоемкого обращения приведенных выше изображений Лапласа.

Рассмотрим независимый от процесса~(\ref{eq15}) случайный экспоненциальный момент времени~$\tau_\lambda$, где~$\lambda>0$. Нетрудно показать (см., например,~\cite{ref24}), что характеристическая функция~$E_*(z_1, z_2; \lambda)$ случайного вектора~$(V(\tau_\lambda), T(\tau_\lambda))$ легко выражается через~$\tilde E(z_1,z_2; p)$, а именно:
\begin{equation}
  \label{eq31}
  E_*\left(z_1, z_2; \lambda\right) = \lambda \tilde E\left(z_1, z_2; \lambda\right)\,.
\end{equation}
Указанное взаимно однозначное соответствие позволяет, по существу, не делать различия между изображением по Лапласу~$\tilde E(z_1,z_2;p)$ характеристической функции системы величин~$(V(t), T(t))$ и~характеристической функцией~$E_*(z_1,z_2; \lambda)$, вычисленной для величин~$(V(\tau_\lambda), T(\tau_\lambda))$.

Полагая~$z_2\hm=0$ в~(\ref{eq26}), находим:
\begin{multline*}
%  \label{eq32}
    E_{*V}\left(z_1;\lambda\right) = E_*\left(z_1, 0;\lambda\right) = {}\\
    {}=\fr{\lambda}{z_1^2 +2i \mu   z_1 + \lambda} \left( \fr{\eta z_1 e^{v_0( \mu  +\sqrt{ \mu ^2 + \lambda})}}{i (\eta \mu -  \sqrt{\mu^2 + \lambda})}+ e^{i v_0 z_1} \right).
\end{multline*}

Если теперь положить в~(\ref{eq26}) $z_1\hm=z_2\hm=0$, то, воспользовавшись интерпретацией~$\tilde F^\pm$ как односторонних преобразований Фурье~\cite{ref15}, приходим к~паре формул

\noindent
\begin{equation*}
%  \label{eq33}
  P\{ V(\tau_\lambda) \gtrless 0  \} = \pm \lambda \tilde F^\pm(0,0;\lambda)\,,
\end{equation*}
что позволяет сформулировать следующее

\smallskip

\noindent
\textbf{Утверждение 7.}
\textit{Вероятность $P\{ U(\tau_\lambda) \hm> a +b \tau_\lambda\} \hm= P\{ V(\tau_\lambda) \hm> 0 \}$ того, что ордината~$U(\tau_\lambda)$ окажется выше растущей по линейному закону границы, дается формулой}:
  \begin{multline*}
%    \label{eq34}
    P\left\{ U(\tau_\lambda) > a +b \tau_\lambda\right\}={}\\
    {}=
    \fr{(1+\eta) \lambda e^{v_0( \mu +\sqrt{ \mu ^2+\lambda})}}{2(\sqrt{ \mu ^2+\lambda}+ \mu )(\sqrt{ \mu^2+\lambda}-\eta  \mu )}\,.
  \end{multline*}


Отсюда, устремляя параметр~$\lambda$ к нулю, легко находим асимптотику вероятности~$P\{U(t)\hm > a \hm+ b t \}$ при~$t \hm\to \infty$. Полученный результат полностью согласуется со здравым смыслом. Действительно, при~$\eta \hm= -1$ имеем отражающий экран, и~поэтому~$\lim\limits_{t \to \infty}P\{ U(t) \hm> a \hm+ b t \} \hm= 0$, т.\,е.\ частица останется ниже экрана при любом~$\mu$. Пусть теперь~$\eta \hm\ne -1$, тогда будем иметь:
\begin{equation}
  \label{eq35}
  \lim\limits_{t \to \infty} P\{ U(t) > a + b t \}=\begin{cases}
\beta\,, &\ \mu=0\,;\\
1\,, &\ \mu<0\,;\\
0\,, &\ \mu>0\,,
\end{cases}
\end{equation}
что также физически вполне объяснимо: при $\mu\hm<0$ частица рано или поздно обгонит экран, при $\mu\hm>0$ частица никогда его не догонит, а при $\mu\hm=0$ частица окажется выше экрана с вероятностью~$\beta$ (именно так и было введено~$\beta$ выше). Отметим, что последний факт может быть использован при решении задачи идентификации параметра~$\beta$ по опытным данным.

Формула~(\ref{eq27}) позволяет также найти важный в практическом отношении закон распределения времени первого достижения экрана~$\theta$. Действительно, пусть~$\eta \hm\ne -1$. Тогда, анализируя выражение~(\ref{eq27}) для~$\tilde E_T$, легко заметить, что единственным слагаемым, не исчезающим при~$z_2 \hm\to \infty$, является последнее, которое вообще не зависит от~$z_2$. Этому слагаемому отвечает дельтообразная плотность, сосредоточенная в нуле, которая, в~свою очередь, дает вероятность~$P\{ T(\tau_\lambda) \hm= 0\}$. Отсюда получаем:
\begin{multline}
  \label{eq36}
  P \left \{\mathop{\sup}\limits_{\tau \in [0, \tau_\lambda]} (U(\tau)- a - b \tau) < 0 \right\} = P \{ \theta > \tau_\lambda \} ={}\\
   {}=P \{ T(\tau_\lambda) = 0\} = 1-e^{v_0( \mu  + \sqrt{ \mu ^2 +\lambda})},
\end{multline}
и это выражение, естественно, совпадает со случаем броуновского движения с~постоянным сносом~\cite{ref23}. Из физических соображений легко понять, что до момента первого достижения процессом СБД со сносом экрана этот процесс ведет себя в~точности, как аналогичный процесс без экрана, а~значит, время достижения границы вообще не должно зависеть от~$\eta$. Поэтому выражение~(\ref{eq36}) сохраняет силу также и при~$\eta\hm=-1$. Эти рассуждения приводят к~важному выводу.

\smallskip

\noindent
\textbf{Утверждение 8.}
\textit{Вероятность~$P \left \{\sup \limits_{\tau \in [0, \tau_\lambda]} \left(U(\tau) \hm- a \hm-\right.\right.$\linebreak $\left.\left.-\;b \tau\right) \hm< 0 \right\}$ невыхода процесса~$U(t)$ в~течение времени~$\tau_\lambda$ за рас\-ту\-щую по линейному закону границу дается формулой}~(\ref{eq36}).

\smallskip


Учитывая связь~(\ref{eq31}) между изображением характеристической функции~$\tilde E$ и характеристической функцией~$E_*$, после обратного преобразования Лап\-ла\-са получаем аналогичное утверждение для произвольного фиксированного момента времени~$t$.

\smallskip

\noindent
\textbf{Утверждение 9.}
  \textit{Вероятность~$P \left \{\sup \limits_{\tau \in [0, t]} \left(U(\tau) \hm- a\hm +\right.\right.$\linebreak $\left.\left.+\;b \tau\right) < 0 \right\}$ невыхода процесса~$U(t)$ в течение времени~$t$ за растущую по линейному закону границу дается формулой}:
  \begin{multline*}
%    \label{eq37}
     P \left\{\mathop{\sup}\limits_{\tau \in [0, t]} (U(\tau) - a - b t) < 0 \right\} = P \{ \theta > t \} ={}\\
      {}=1 - \fr{1}{2} \left( e^{-2 v_0 \mu^- } \erfc{ \left( -\fr{v_0 - 2 |\mu| t}{2 \sqrt{t}} \right)} +{}\right.\\
\left.       {}+e^{2 v_0 \mu^+ } \erfc{ \left( -\fr{v_0 + 2 |\mu| t}{2 \sqrt{t}} \right)} \right)\,.
  \end{multline*}

Полученное выражение, как легко заметить, фактически уже содержалось в~представлении~(\ref{eq28}).

Найдем также выражение для математического ожидания времени~$T(\tau_\lambda)$. Дифференцируя~$E_*(z_2; \lambda)$ по~$z_2$ и полагая~$z_2\hm=0$, имеем:
\begin{multline*}
%  \label{eq38}
  \tilde m(\lambda) ={\sf M}[T(\tau_\lambda)] ={}\\
   {}=\fr{(1+\eta) e^{v_0( \mu +\sqrt{ \mu ^2+\lambda})}}{2(\sqrt{ \mu ^2+\lambda}+ \mu )(\sqrt{ \mu^2+\lambda}-\eta  \mu )} ={}\\
  {}= \fr{P\{ U(\tau_\lambda) > a +b \tau_\lambda\}}{\lambda}\,.
\end{multline*}
Отсюда с учетом~${\sf M}[\tau_\lambda] \hm= {1}/{\lambda}$ сразу же получаем равенство
\begin{equation}
  \label{eq39}
  \fr{{\sf M}[T(\tau_\lambda)]}
  {{\sf M}[\tau_\lambda]} = P\left\{ U(\tau_\lambda) > a +b \tau_\lambda\right\}\,,
\end{equation}
которое можно интерпретировать как некоторую обобщенную эргодичность.

Как легко проверить, равенство~(\ref{eq39}) можно обобщить на случай векторного процесса~$\mathbf{U}(t)$ и~об\-ласти~$\Omega$ произвольного вида следующим образом.
%\smallskip

\noindent
\textbf{Утверждение 10.}
  \textit{Для произвольного непрерывного процесса $\mathbf{U}(t)$ и произвольной области~$\Omega$ справедливо равенство}:
  \begin{equation*}
%    \label{eq40}
    \fr{{\sf M}[\mathrm{mes}\, \{ \tau \in [0, \tau_\lambda]|\ \mathbf{U}(\tau) \in \Omega \}]}{{\sf M}[\tau_\lambda]} = P\left\{ \mathbf{U}(\tau_\lambda) \in \Omega\right \}\,.
  \end{equation*}


Следует отметить, что все полученные в~данном разделе характеристики, относящиеся к показательному моменту времени~$\tau_\lambda$, в~принципе, могут быть найдены также и~для произвольного фиксированного момента времени~$t$. Соответствующее преобразование Лапласа удается обратить методом, описанным при выводе представления~(\ref{eq28}). Остав\-ляя в~стороне более детальный математический и~физический анализ построенного выше решения, отметим, что изложенный метод позволяет решить и~целый ряд других, более сложных, интересных как с~прикладной, так и~с~вероятностной точки зрения задач.

\section{Заключение}

Метод уравнения Пу\-га\-чё\-ва--Свеш\-ни\-ко\-ва позволяет успешно решать целый ряд задач, ка\-са\-ющих\-ся времени пребывания одномерного процесса на полуоси. Для реализации метода процесс, исследуемый на выбросы, должен описываться сис\-те\-мой стохастических дифференциальных уравнений, которая линейна либо во всем пространстве, либо в~полупространствах. Допускается также линейность и~в~четвертьпространствах, но такая задача более сложна математически. В~предыдущих работах авторов полуось, представляющая область пребывания процесса, имела фиксированную постоянную границу. В~настоящей статье эта граница равномерно движется с заданной скоростью. Кроме того, в~уравнения движения добавлены дополнительные члены, соответствующие локальному времени процесса.

В статье детально изучено распределение\linebreak длительности выбросов типового процесса СБД с~постоянным сносом за равномерно движущуюся границу. Найденное распределение представляет \mbox{собой} некоторое обобщение классического закона арксинуса. Попутно получен ряд дополнительных характеристик движения: вероятность невыхода процесса за подвижную границу, распределение времени ее первого достижения и~т.\,п.

Построенное аналитическое решение пред\-став\-ля\-ет самостоятельный интерес как эталонное точное аналитическое решение типовой задачи статистической динамики. Кроме того, оно может быть использовано для оценки погрешности и~тес\-ти\-ро\-ва\-ния существующих приближенных методов решения уравнения Пугачёва~\cite{ref22}.

{\small\frenchspacing
 {%\baselineskip=10.8pt
 \addcontentsline{toc}{section}{References}
 \begin{thebibliography}{99}
\bibitem{ref1}
\Au{Люу~Ю.\,Д.} Методы и алгоритмы финансовой математики~/ Пер. с англ.~--- М.: Бином, 2007. 752~с. (\Au{Lyuu~Y.\,D.} Financial engineering and computation. --- 1st ed.~--- Cambridge: Cambridge University Press, 2001. 627~p.)
\bibitem{ref2}
\Au{Cohen~J.\,W., Hooghiemstra~G.} Brownian excursion, the $M/M/1$ queue and their occupation times~// Math. Oper. Res., 1981. Vol.~6. No.\,4. P.~608--629.
\bibitem{ref3}
\Au{Berman~S.\,M.} Sojourn and extremes of stochastic processes.~--- Belmond: CRC Press, 1992. 320~p.
\bibitem{ref4}
\Au{Бородин~А.\,Н.} Случайные процессы.~--- СПб.: Лань, 2013. 640~с.
\bibitem{ref5}
\Au{Феллер~В.} Введение в теорию вероятностей и~ее приложения~/ Пер. с англ.~--- М.: Мир, 1967.  Т.~2. 765~с. (\Au{Feller~W.} An introduction to probability theory and its applications.~--- New York, NY, USA: Wiley, 1966.  Vol.~II. 626~p.)
\bibitem{ref6}
\Au{Korpas~A.\,K.} Occupation times of continuous Markov processes. Ph.D. Thesis.~--- Bowling Green: Bowling Green State University, 2006. 92~p.
\bibitem{ref7}
\Au{L$\acute{\mbox{e}}$vy~P.} Sur une
probl{\fontsize{10pt}{10pt}\selectfont\ptb{\!\!\`{e}}}me de Marcinkiewicz~// Comptes rendus Academie sciences Paris, 1939. T.~208. P.~319--321, errata p.~776.

\bibitem{ref10} %8
\Au{Akahori~J.} Some formulae for a new type of path-dependent option~//
Ann. Appl. Probab., 1995. Vol.~5. No.\,2. P.~383--388.

\bibitem{ref9} %9
\Au{Dassios~A.} The distribution of the quantile of a Brownian motion with drift and the pricing of related path-dependent options~// Ann. Appl. Probab., 1995. Vol.~5. No.\,2. P.~389--398.

\bibitem{ref11} %10
\Au{Yor~M.} The distribution of Brownian quantiles~// J.~Appl. Probab., 1995. Vol.~32. P.~405--416.

\bibitem{ref8} %11
\Au{Tak$\acute{\mbox{a}}$cs~L.} On a generalization of the arc-sine law~// Ann. Appl. Probab., 1996. Vol.~6. No.\,3. P.~1035--1040.



\bibitem{ref12}
\Au{Pechtl~A.} Distribution of occupation times of Brownian motion with drift~// J.~Appl. Math. Decision Sci., 1999. Vol.~3. P.~41--62.
\bibitem{ref13}
\Au{Baxter~G.} Wiener process distributions of the arcsine law type~//
Proc. Am. Math. Soc., 1956. Vol.~7. P.~738--741.
\bibitem{ref14}
\Au{Заяц~О.\,И.} Об аналитическом решении задачи Феллера о длительности выбросов~// Труды СПбГТУ. Прикладная математика, 1996. №\,461. С.~92--100.
\bibitem{ref22}
\Au{Пугачёв~В.\,С., Синицын~И.\,Н.} Стохастические дифференциальные системы. Анализ и фильтрация.~--- М.: Наука, 1985 (1-е изд.), 1990 (2-е изд).
\bibitem{ref25}
\Au{Свешников~А.\,А.} Применение теории непрерывных марковских процессов к решению нелинейных задач прикладной гироскопии~// Тр. V~Междунар. конф. по нелинейным колебаниям.~--- Киев: ИМ АН УССР, 1970. T.~3. С.~659--665.
\bibitem{ref26}
\Au{Свешников~А.\,А., Ривкин~С.\,С.} Вероятностные методы в прикладной теории гироскопов.~--- М.: Наука, 1974. 536~с.
\bibitem{ref15}
\Au{Заяц~О.\,И.} Применение уравнения Пу\-га\-чё\-ва--Свеш\-ни\-ко\-ва к~исследованию ку\-соч\-но-ли\-ней\-ных стохастических систем, линейных в полупространствах~// На\-уч\-но-тех\-ни\-че\-ские ведомости СПбГПУ. Физико-математические науки, 2013. №\,4-1. С.~128--142.
\bibitem{ref15a}
\Au{Заяц~О.\,И., Березин~С.\,В.} Применение уравнения Пу\-га\-чё\-ва--Свешникова к исследованию ку\-соч\-но-ли\-ней\-ных стохастических систем, линейных в четвертях пространства~// На\-уч\-но-тех\-ни\-че\-ские СПбГПУ. Информатика. Телекоммуникации. Управление, 2013. №\,6. С.~87--101.
\bibitem{ref16}
\Au{Caughey~T.\,K., Dienes~J.\,K.} Analysis of non-linear first order system with a white noise input~// J.~Appl. Phys., 1961. Vol.~32. No.\,11. P.~2476--2479.
\bibitem{ref17}
\Au{Заяц~О.\,И.} Решение задачи Феллера для винеровского процесса с постоянным сносом~// Труды СПбГТУ. Прикладная математика, 1999. №\,477. С.~67--72.
\bibitem{ref18}
\Au{Ito~K., McKean~H.\,P.} Brownian motion on a half-line~// Illinois J.~Math., 1963. Vol.~7. P.~181--231.
\bibitem{ref19}
\Au{Lejay~A.} On the constructions of the skew Brownian motion~// Probab. Surveys, 2006. Vol.~3. P.~413--466.
\bibitem{ref20}
\Au{Le~Gall~J.-F.} One-dimensional stochastic differential equations involving the local times of the unknown process~// Stochastic analysis and application~/
Eds. A.~Truman, D.\,W.~Williams.~--- Lecture notes in mathematics ser.~---
Berlin--Heidelberg: Springer, 1984. Vol.~1095. P.~51--82.
\bibitem{ref21}
\Au{Appuhamillage~T., Bokil~V., Thomann~E., Waymire~E., Wood~B.} Occupation and local times for skew Brownian motion with applications to dispersion across an interface~// Ann. Appl. Probab., 2011. Vol.~21. No.\,1. P.~183--214.
\bibitem{ref23}
\Au{Лаврентьев~М.\,А., Шабат~Б.\,В.} Методы теории функции комплексного переменного.~--- М.: Наука, 1965. 716~с.
\bibitem{ref24}
\Au{Бородин~А.\,Н., Салминен~П.} Справочник по броуновскому движению: факты и формулы~/ Пер. с англ.~--- СПб.: Лань, 2000. 640~с. (\Au{Borodin~A.\,N., Salminen~P.} Handbook of Brownian motion. Facts and formulae. Probability and its applications. Basel: Birkh$\ddot{\mbox{a}}$user, 1996. 462~p.)
 \end{thebibliography}

 }
 }

\end{multicols}

\vspace*{-3pt}

\hfill{\small\textit{Поступила в~редакцию 02.02.15}}

%\newpage

\vspace*{12pt}

\hrule

\vspace*{2pt}

\hrule

%\vspace*{12pt}

\def\tit{APPLICATION OF~THE~PUGACHEV--SVESHNIKOV EQUATION TO~THE~BAXTER OCCUPATION TIME PROBLEM}

\def\titkol{Application of~the~Pugachev--Sveshnikov equation to~the~Baxter occupation time problem}

\def\aut{S.\,V.~Berezin and O.\,I.~Zayats}

\def\autkol{S.\,V.~Berezin and O.\,I.~Zayats}

\titel{\tit}{\aut}{\autkol}{\titkol}

\index{Berezin S.\,V.}
\index{Zayats O.\,I.}

\vspace*{-9pt}


\noindent
Institute of Applied Mathematics and Mechanics, Peter the Great St.\ Petersburg State Polytechnic University,  29~Politekhnicheskaya Str., St.\ Petersburg 195251, Russian Federation


\def\leftfootline{\small{\textbf{\thepage}
\hfill INFORMATIKA I EE PRIMENENIYA~--- INFORMATICS AND
APPLICATIONS\ \ \ 2015\ \ \ volume~9\ \ \ issue\ 2}
}%
 \def\rightfootline{\small{INFORMATIKA I EE PRIMENENIYA~---
INFORMATICS AND APPLICATIONS\ \ \ 2015\ \ \ volume~9\ \ \ issue\ 2
\hfill \textbf{\thepage}}}

\vspace*{3pt}


\Abste{The Baxter problem, that is, an occupation (sojourn) time above a moving barrier, for a skew Brownian motion is considered. The latter is known as a model of a~semipermeable barrier which permits either movement through it or reflection to the opposite direction with a specified probability. The Pugachev--Sveshnikov equation for a continuous Markov process is used to obtain an analytic solution of the problem. The generic method to solve the Pugachev--Sveshnikov equation for occupation-time type problems involves its reduction to a certain Riemann boundary value problem. This problem is solved, and the occupation time probability density function is derived. Along the way, some additional characteristics of the skew Brownian motion are obtained such as the first passage time, nonexceedance probability, occupation time moments, and some other characteristics.}


\KWE{Markov process; Pugachev equation; Pugachev--Sveshnikov equation; Riemann boundary value problem; stochastic mechanics; skew Brownian motion; occupation time; sojourn time}

\DOI{10.14357/19922264150205}

%\Ack
%\noindent



%\vspace*{3pt}

  \begin{multicols}{2}

\renewcommand{\bibname}{\protect\rmfamily References}
%\renewcommand{\bibname}{\large\protect\rm References}

{\small\frenchspacing
 {%\baselineskip=10.8pt
 \addcontentsline{toc}{section}{References}
 \begin{thebibliography}{99}

\bibitem{ref1-1}
\Aue{Lyuu, Y.\,D.} 2001. \textit{Financial engineering and computation}. 1st ed. Cambridge: Cambridge University Press. 627~p.

\bibitem{ref2-1}
\Aue{Cohen, J.\,W., and G.~Hooghiemstra}. 1981. Brownian excursion, the $M/M/1$ queue and their occupation times. \textit{Math. Oper. Res.} 6(4):608--629.

\bibitem{ref3-1}
\Aue{Berman, S.\,M.} 1992. \textit{Sojourn and extremes of stochastic processes}. Belmond: CRC Press. 320~p.

\bibitem{ref4-1}
\Aue{Borodin, A.\,N.} 2013. \textit{Sluchaynye protsessy} [Stochastic processes]. St.\ Petersburg: Lan'. 640~p.

\bibitem{ref5-1}
\Aue{Feller, W.} 1966. \textit{An introduction to probability theory and its applications}.  New York, NY: Wiley. Vol.~II. 626~p.

\bibitem{ref6-1}
\Aue{Korpas, A.\,K.} 2006. Occupation times of continuous Markov processes. Ph.D. Thesis. Bowling Green: Bowling Green State University. 92~p.

\bibitem{ref7-1}
\Aue{L$\acute{\mbox{e}}$vy, P.} 1939. Sur une \mbox{probl{\fontsize{10pt}{10pt}\selectfont\ptb{\mbox{\hspace*{-3.9pt}\,\`{e}}}}me} de Marcinkiewicz. \textit{Comptes rendus Academie sciences Paris} 208:319--321, errata p. 776.


\bibitem{ref10-1} %8
\Aue{Akahori, J.} 1995. Some formulae for a new type of path-dependent option. \textit{Ann. Appl. Probab}. 5(2):383--388.

\bibitem{ref9-1} %9
\Aue{Dassios, A.} 1995. The distribution of the quantile of a~Brownian motion with drift and the pricing of related path-dependent options. \textit{Ann. Appl. Probab}. 5(2):389--398.

\bibitem{ref11-1} %10
\Aue{Yor, M.} 1995. The distribution of Brownian quantiles. \textit{J.~Appl. Probab}. 32:405--416.

\bibitem{ref8-1} %11
\Aue{Tak$\acute{\mbox{a}}$cs, L}. 1996. On a generalization of the arc-sine law. \textit{Ann. Appl. Probab}. 6(3):1035--1040.

\bibitem{ref12-1} %12
\Aue{Pechtl, A.} 1999. Distribution of occupation times of Brownian motion with drift. \textit{J.~Appl. Math. Decision Sci.} 3:41--62.

\bibitem{ref13-1}
\Aue{Baxter, G.} 1956. Wiener process distributions of the arcsine law type. \textit{Proc. Am. Math. Soc.} 7:738--741.

\bibitem{ref14-1}
\Aue{Zayats, O.\,I.} 1996. Ob analiticheskom reshenii zadachi Fellera o~dlitel'nosti vybrosov [On analytic solution of the Feller problem of upward excursions]. \textit{Trudy \mbox{SPbGTU}. Prikladnaya Matematika} [SPb Polytechnic Univeristy ``Applied Mathematics'' Proceedings] 461:92--100.


\bibitem{ref22-1}
\Aue{Pugachev, V.\,S., and I.\,N.~Sinitsyn}. 1987. \textit{Stochastic differential systems. Analysis and filtering}. Chechester. 549~p.


\bibitem{ref25-1}
\Aue{Sveshnikov, A.\,A.} 1970. Primenenie teorii nepreryvnykh markovskikh protsessov k resheniyu nelineynykh zadach prikladnoy giroskopii [Application of the theory of continuous Markov processes to solution of nonlinear problems of applied gyroscopy]. \textit{Tr. V~Mezhdunar. konf. po nelineynym kolebaniyam} [5th Conference (International) on Nonlinear Vibrations Proceedings]. Kiev. 3:659--665.

\bibitem{ref26-1}
\Aue{Sveshnikov, A.\,A., and S.\,S.~Rivkin}. 1974. \textit{Veroyatnostnye metody v~prikladnoy teorii giroskopov} [Probabilistic methods in the theory of gyroscopy]. Moscow: Nauka. 536~p.

\bibitem{ref15-1}
\Aue{Zayats, O.\,I.} 2013. Primenenie uravneniya Pu\-ga\-che\-va--Svesh\-ni\-ko\-va k~issledovaniyu ku\-soch\-no-li\-ney\-nykh sto\-kha\-sti\-che\-skikh sistem, lineynykh v~poluprostranstvakh [Analysis of piecewise linear stochastic systems in half-spaces by means of the Pugachev--Sveshnikov equation]. \textit{Na\-uch\-no-Tekh\-ni\-che\-skie Vedomosti SPbGPU, Fiziko-Matematicheskie Nauki} [St.\ Petersburg State Polytechnical University
J.~Physics and Mathematics] 4(1):128--142.

\bibitem{ref15a-1}
\Aue{Zayats, O.\,I., and S.\,V.~Berezin}. 2013. Primenenie uravneniya Pu\-ga\-che\-va--Svesh\-ni\-ko\-va k~issledovaniyu \mbox{kusochno}-\mbox{lineynykh} stokhasticheskikh sistem, lineynykh v~chetvertyakh prostranstva [Analysis of piecewise linear stochastic systems in quarter-spaces by means of the Pugachev--Sveshnikov equation]. \textit{Nauchno-tekhnicheskie Vedomosti SPbGPU. Informatika. Telekommunikatsii. Upravlenie} [St. Petersburg State Polytechnical University~J.~Computer Science. Telecommunication and Control Systems] 6:87--101.

\bibitem{ref16-1}
\Aue{Caughey, T.\,K., and J.\,K.~Dienes}. 1961.
Analysis of non-linear first order system with a white noise input.
\textit{J.~Appl. Phys}. 32(11):2476--2479.

\bibitem{ref17-1}
\Aue{Zayats, O.\,I.} 1999. Reshenie zadachi Fellera dlya vinerovskogo protsessa s~postoyannym snosom [Solution of the Feller problem for a~Wiener process with a~constant drift]. \textit{Trudy SPbGTU. Prikladnaya Matematika} [SPb Polytechnic Univeristy ``Applied Mathematics'' Proceedings]. 477:67--72.

\bibitem{ref18-1}
\Aue{Ito, K., and H.\,P. McKean}. 1963. Brownian motion on a~half-line.
\textit{Illinois J.~Math.} 7:181--231.

\bibitem{ref19-1}
\Aue{Lejay, A}.
2006. On the constructions of the skew Brownian motion.
\textit{Probab. Surveys} 3:413--466.

\bibitem{ref20-1}
\Aue{Le Gall, J.-F.} 1984. One-dimensional stochastic differential equations involving the local times of the unknown process. \textit{Stochastic
analysis and application}. Eds.\ A.~Truman and D.\,W.~Williams.
Lecture notes in mathematics ser. Berlin--Heidelberg: Springer.
1095:51--82.

\bibitem{ref21-1}
\Aue{Appuhamillage, T., V.~Bokil, E.~Thomann, E.~Waymire, and B.~Wood}. 2011. Occupation and local times for skew Brownian motion with applications to dispersion across an interface. \textit{ Annals Appl. Pobab}. 21(1):183--214.

\bibitem{ref23-1}
\Aue{Lavrent'ev, M.\,A., and B.\,V.~Shabat}. 1965. \textit{Metody teorii funktsii kompleksnogo peremennogo} [Methods of the theory of functions of a complex variable]. Moscow: Nauka. 716~p.

\bibitem{ref24-1}
\Aue{Borodin, A.\,N., and P.~Salminen}. 1996. \textit{Handbook of Brownian motion. Facts and formulae. Probability and its applications}. Basel: Birkh$\ddot{\mbox{a}}$user. 462~p.
\end{thebibliography}

 }
 }

\end{multicols}

\vspace*{-3pt}

\hfill{\small\textit{Received February 2, 2015}}

%\vspace*{-18pt}



\Contr

\noindent
\textbf{Berezin Sergey V.} (b.\ 1986)~---
senior scientist, Institute of Applied Mathematics and Mechanics, Peter the Great St.\ Petersburg State Polytechnic University,  29~Politekhnicheskaya Str., St.\ Petersburg 195251, Russian Federation; servberezin@yandex.ru

\vspace*{3pt}

\noindent
\textbf{Zayats Oleg I.} (b.\ 1952)~---
 Candidate of Science (PhD) in physics and mathematics, associate professor, Institute of Applied Mathematics and Mechanics, Peter the Great St.\ Petersburg  State Polytechnic University, 29~Politekhnicheskaya Str., St.\ Petersburg 195251, Russian Federation;  zay.oleg@gmail.com


    \label{end\stat}


    \renewcommand{\bibname}{\protect\rm Литература}  %5

\def\stat{tuchkova}

\def\tit{ПРЕДЕЛЬНЫЕ РАСПРЕДЕЛЕНИЯ ДЛЯ ХАРАКТЕРИСТИК\\
 ПРИ УСВОЕНИИ ДАННЫХ НАБЛЮДЕНИЙ\\
  В СТАЦИОНАРНОМ РЕЖИМЕ$^*$}

\def\titkol{Предельные распределения для характеристик при усвоении данных наблюдений в~стационарном режиме}

\def\aut{К.\,П.~Беляев$^1$, Н.\,П.~Тучкова$^2$}

\def\autkol{К.\,П.~Беляев, Н.\,П.~Тучкова}

\titel{\tit}{\aut}{\autkol}{\titkol}

\index{Беляев К.\,П.}
\index{Тучкова Н.\,П.}

{\renewcommand{\thefootnote}{\fnsymbol{footnote}} \footnotetext[1]
{Работа выполнена при поддержке РФФИ (проекты 14-05-00363 и~14-07-00037).}}


\renewcommand{\thefootnote}{\arabic{footnote}}
\footnotetext[1]{Институт океанологии им.\ П.\,П.~Ширшова Российской академии наук; Федеральный университет штата Баийя, Сальвадор,  Бразилия, kb@sail.msk.ru}
\footnotetext[2]{Вычислительный центр им.\,А.\,А.\,Дородницына Российской академии наук, tuchkova@ccas.ru}



\Abst{Рассматривается линейная задача усвоения данных наблюдений
в~гидродинамическую модель. Для такой задачи, сформулированной в~терминах
функционирования марковской цепи, исследуется сходимость переходных
распределений цепи к~стационарному распределению, находятся достаточные
условия этой сходимости и~строится функциональное уравнение для характеристической
функции этого распределения. Далее рассматривается схема серий, зависящих от
параметра.  Исследуется сходимость стационарных распределений цепи,
когда параметр стремится  к~нулю. Показывается, что в~такой схеме
в~стационарном режиме марковской цепи существует предельное распределение
параметров,  и~доказывается, что оно будет гауссовым, находятся его
среднее и~дисперсия. Обсуждается, как данная схема может быть применена для
оценки  характеристик при оперативном усвоении данных и~прогнозировании
со\-сто\-яния среды.}


\KW{методы усвоения данных наблюдений; стационарные распределения цепей Маркова;
асимптотическое распределение цепей при малом значении параметра}

\DOI{10.14357/19922264150206}



\vskip 14pt plus 9pt minus 6pt

\thispagestyle{headings}

\begin{multicols}{2}

\label{st\stat}


\section{Введение}

Современная оперативная океанография, оперативный прогноз погоды и~ряд других
областей основаны на сочетании численного моделирования процесса и~усвоения данных
наблюдений в~модель. Задача усвоения состоит в~том, чтобы оптимальным образом
сочетать модельный расчет и~наблюдения, сохранить баланс энергии,
массы, тепла и~других входящих в~модель параметров и~при этом сделать модельный расчет как можно более близким к~наблюдениям в~смысле заданной метрики. Далее модель, стартуя с~оптимального в~этом смысле усвоенного или скорректированного поля, должна дать лучший прогноз, чем в~случае без усвоения данных. При этом качество как самой модели, так и~метода усвоения может быть количественно оценено по степени близости расчета и/или прогноза к~наблюдаемым данным.
Важна также техническая сторона дела, а~именно: срок обработки данных, используемые ресурсы компьютера, память, время работы процессора, скорость передачи информации от получения данных до доставки потребителю, доступ для визуализации результатов и~многое другое. Все эти проблемы прямо
или косвенно решаются в~задачах оперативного усвоения данных наблюдений.

Схема линейного усвоения данных достаточно проста, она реализуется в~системе уравнений следующего вида:
\begin{equation}
\left.
\begin{array}{rl}
X_n&=F\left(X_{n-1}\right)\,;\\[6pt]
X_{n+1}&=X_n+K\left(Y_n-HX_n\right)\,.
\end{array}
\right\}
\label{1-t}
\end{equation}
Здесь и~далее используются следующие обозначения:
\begin{itemize}
\item $X_n$, $n=1,3,5,\ldots,$~--- вектор состояния модели,
 $X_n \hm\in R^r$ , т.\,е.\ $X_n\hm=(x^1_n,\ldots ,x^r_n)$.
 Например, в~моделях динамики океана~$X_n$ может состоять из
 компонент температуры, солености, ско\-рости течений и~т.\,п.
 Для определенности этот вектор в~статье рассматривается в~каждой
 точке пространства области, состоящей из $N$ точек сетки, где задана модель;
 \item $F$~--- оператор модели, т.\,е.\ векторная функция из $R^r \mapsto R^r$.
\item  вектор $Y \hm\in R^{N_{\mathrm{obs}}}$~--- вектор наблюдаемых величин,
его длина на шаге~$n$ обозначается как~$N_{\mathrm{obs}}$.
Обычно $N_{\mathrm{obs}}\hm<r$, так как не все расчетные величины
в~модели могут быть наблюдаемыми. Например, в~моделях циркуляции океана
обычно наблюдаемыми могут быть температура, соленость, уровень океана и~очень
редко компоненты скорости течений;
\item матрица~$K$ в~(\ref{1-t})~--- это $(r\times N_{\mathrm{obs}})$-мат\-ри\-ца,
она имеет в~литературе название весовой матрицы (Kalman gain matrix).
Ее физический смысл состоит в~том, чтобы передать вес наблюдения в~точку
расчетной сетки, т.\,е.\ она передает значения из пространства наблюдений
в~пространство модели;
\item матрица $H$ в~(\ref{1-t})~--- это матрица проекционного оператора,
имеющая размерность $N_{\mathrm{obs}} \times r$ и~осуществляющая обратную операцию,
т.\,е.\ передающая сигнал от пространства модели в~пространство наблюдений.
\end{itemize}

 Таким образом, матрица $KH$ будет квадратной $(r \times r)$-мат\-ри\-цей.
 Расчет начинается с~заданного поля~$X_0$.

Такая ассимиляционная схема описана в~многочисленных книгах и~статьях,
например в~работах [1--3]. Схемы усвоения различаются конкретным выбором модели~$F$
и/или матрицы~$K$. В~частности, хорошо известный фильтр Калмана~\cite{Evensen2}
относится к~данной схеме. Есть и~другие методы определения матрицы~$K$:
например, в~\cite{Belyaev1} матрица~$K$ строится на основе
решений параболических дифференциальных уравнений типа Фок\-ке\-ра--Планка.

В схеме~(\ref{1-t}) последовательность~$X_n$ образует цепь Маркова.
Вектор наблюдений~$Y_n$ предполагается случайным на каждом шаге с~заданным
распределением. Возникает задача построения распределения случайного
вектора~$X_n$ и~изучение его предельного поведения при $n \hm\rightarrow \infty$.
Это достаточно сложная и~до конца не решенная проблема. В~\cite{Belyaev1} приведены некоторые решения для линейной модели $F(X)\hm=\Lambda X$, где $\Lambda$~--- это $(r \times r)$-мат\-рица.

Значительно более сложной задачей будет определение совместного распределения
векторов $X_{n_1}, X_{n_2}, \ldots, X_{n_k}$ для заданного набора индексов
$(n_1, n_2, \ldots , n_k)$. В~работе~\cite{Tanajura} показано,\linebreak что
при физически обоснованных и~про\-ве\-ря\-емых условиях совместное распределение
векторов $X_{n_1}$, $X_{n_2}, \ldots, X_{n_k}$ стремится к~совместному распределению
векторов, определяющих диффузионный процесс, для случая
$X_{n_k}\hm=X_{0+n_k dt}$ при $n_k \hm\rightarrow \infty $ $dt \hm\rightarrow 0$,
когда временной интервал $[0,T]$ разбивается точками
$[0, n_1, n_2, \ldots, n_k, \ldots, T]$.

Важный частный случай имеет место, когда мат\-ри\-ца~$K$ мала. Так, например,
происходит в~случае  ансамблевого фильтра Калмана (EnKF), в~схеме объективного
анализа~\cite{Jazwinski, Evensen2}, когда матрица~$K$ строится на основе
статистики аномалий, т.\,е.\ разницы между расчетным и~средним
многолетним значением ка\-ко\-го-ни\-будь модельного поля, например
поля температуры или уровня океана. Если реальная изменчивость не очень велика,
то модельная разница между конкретным расчетом и~средним многолетним для
температуры в~точке составляет около 2--3~$^\circ$C и~в этом случае матрица~$K$
по порядку величины будет около $10^{-4}$ в~соответствующих единицах.
Поэтому имеет смысл изучить предельное поведение распределений~$X_n$
при малых~$K$, т.\,е.\ когда $K$ можно представить как $K_l\hm=\rho_l K$,
$l\hm=1, 2, 3, \ldots$, и~устремить $\rho_l\hm\rightarrow 0$. Такого рода
задачи активно рассматриваются в~теории массового обслуживания при изучении
пределов характеристик при малой и~большой загрузке.

Цели настоящей работы заключаются в~том, чтобы  получить условия
стационарного режима цепи Маркова~(\ref{1-t}) и~найти распределение этого
режима; получить предельное распределение характеристик в~стационарном режиме
при $K_l\hm=\rho_l K$, $\rho_l\hm\rightarrow 0$. Метод доказательства  теоремы
является оригинальным и~близок к~опубликованному ранее в~работе~\cite{Belyaev2}.


\section{Основные формулировки}

Пусть $X_n$ и~$Y_n$, $n\hm=1, 2, 3, \ldots,$~--- случайные величины,
определенные на заданном вероятностном пространстве. Для определенности
представим~$Y_n$ как $Y_n\hm=HZ_n$, где случайная величина $Z_n \hm\in R^r$.
Как следует из формул~(\ref{1-t}), матрица $KH$ имеет размерность
$r \times r$. Пусть $G_n(x)\hm=\mathrm{Pr}\,(X_n\hm<x)$
и~$\Gamma_n(x)\hm=\partial/\partial x [\mathrm{Pr}\,(Z_n\hm<x)]$~---
их соответствующие функции распределения и~плотности.

\smallskip

\noindent
\textbf{Лемма~1.} \textit{Пусть $n \rightarrow \infty$ и~$\Gamma_n(x)\hm>0 $.
Для того чтобы существовало стационарное распределение~$X_n$,
$\lim P(X_n < x)\hm=G(x)$, достаточно выполнения следующего условия}:
\textit{для любых $x$ и~$y$ существует решение}
\begin{equation}
y=F(x)\,, \quad KHx=y\,.
\label{2-t}
\end{equation}

\noindent
 Д\,о\,к\,а\,з\,а\,т\,е\,л\,ь\,с\,т\,в\,о\,.\ \ Покажем, что для любой
 пары состояний цепи $x, y$  существует такой момент времени~$k$ ,
 что переходные вероятности
\begin{equation}
P(X_{n+k}=x|X_n=y)>0\,.
\label{3-t}
\end{equation}

Действительно, определим $P(X_{n+k}\hm=x)$ в~состоянии $x$ для некоторых~$n$ и~$k$.
Если $n+k$ нечетно, то из условия~(\ref{2-t}) следует, что есть состояние
$\stackrel{\leftarrow}{x}\hm=F^{-1}(x)$ такое, что
$P(X_{n+k-1}\hm=\stackrel{\leftarrow}{x})\hm>0$. Тогда если
$X_{n+k-2}\hm=y$, то для того чтобы цепь в~момент $n\hm+k$ оказалась
в~состоянии~$x$, требуется, чтобы  $KHZ_{n-1}\hm=\stackrel{\leftarrow}{x}-y-KHy$,
вероятность чего больше нуля по условию~(\ref{2-t}).
Аналогично доказывается и~для четных $n\hm+k$ шагов. Следовательно, любые
два состояния цепи являются смежными в~том смысле, что за конечное число
шагов можно попасть из одного состояния в~другое с~ненулевой вероятностью.
Отсюда следует существование стационарного распределения цепи Маркова
 (см.~\cite{Gikhman}).

\smallskip

\noindent
\textbf{Замечание~1.} Доказательство проведено для дискретных цепей, но
легко обобщается на случай непрерывных цепей Маркова. Для опре\-де\-лен\-ности
в~дальнейшем все распределения будем считать непрерывными.

\smallskip

\noindent
\textbf{Замечание~2.} Условия~(\ref{2-t}) и~условие $\Gamma_n(x)\hm>0 $
означают, что данные и~модель не противоречат друг другу. Понятно, например,
что если модель динамики океана дает температуру воды $y\hm=20$~$^\circ$C
и~при этом наблюдения показывают $x\hm=2$~$^\circ$C, то вероятности
перехода~(\ref{3-t}) будут практически нулевые, а~решение уравнения $y\hm=F(x)$,
хотя теоретически и~возможно, физически не реализуемо. Поэтому при практических расчетах важно, чтобы модель адекватно описывала физику, а~данные измерялись в~соответствии с~этой моделью.	

\smallskip

Рассмотрим теперь вложенную цепь Маркова только для нечетных номеров шага
по времени $X_{2k}\hm=X_{2k-1}\hm+K(Y_{2k}\hm-HX_{2k-1})$.

\smallskip

\noindent
\textbf{Лемма~2.} \textit{Пусть выполнены условия леммы~$1$ и~пусть
$\pi(\omega)$ обозначает характеристическую функцию случайной величины~$X_{2k}$
в~стационарном режиме, т.\,е.}
$$
\pi(\omega)=\underbrace{\lim}_{k\rightarrow \infty}
\int\limits^\infty_{-\infty}\!\!e^{i \omega x}\, dP(X_{2k}\hm=x)\,.
$$ \textit{Как обычно}, $i\hm=\sqrt{-1}$.

\textit{Тогда $\pi(\omega)$ удовлетворяет функциональному уравнению}:
\begin{equation}
\pi(\omega)=\pi[(I+KH)\omega]\psi ((KH)^{-1} \omega)\,,
\label{4-t}
\end{equation}
\textit{где $\psi (\omega)$~--- характеристическая функция случайной величины  $Z\hm=\underbrace{\lim}_{k\rightarrow \infty}Z_{2k}$ в~стационарном режиме и~$I$~--- единичная матрица размерности} $r\times r$.

\textit{Обозначим} $\pi^{(s)}(0)=\pi_s$, $\psi^{(s)}(0)\hm=\psi_s$.

\smallskip

\noindent
 Д\,о\,к\,а\,з\,а\,т\,е\,л\,ь\,с\,т\,в\,о\,.\ \
 Из уравнения Кол\-мо\-го\-ро\-ва--Чеп\-ме\-на следует соотношение
 для переходных вероятностей:
\begin{multline*}
dP(X_{2k}=x)={}\\
{}=\!\!\int\limits^\infty_{-\infty}\!\!
dP(X_{2k-1}=y)\,dP(KHZ_{2k-1}=x-y-KHy).\hspace*{-1.2pt}
\end{multline*}
Умножая обе части этого равенства на $e^{i \omega x}$ и~интегрируя
по~$x$ от $-\infty$ до~$\infty$ , после простых преобразований приходим к~соотношению:

\noindent
\begin{multline*}
\int\limits^\infty_{-\infty}\!\!e^{i \omega x} \,dP(X_{2k}=x)={}\\
{}=\int\limits^\infty_{-\infty}\!\!e^{i(I+KH)y\omega}\,dP(X_{2k-1}=y)\times{}\\
{}\times
\int\limits^\infty_{-\infty}\!\!e^{i \omega x} \,dP(X_{2k-1}=(KH)^{-1}x)\,.
\end{multline*}
Из этого равенства переходом к~пределу при $k\hm\rightarrow \infty$
доказывается формула~(\ref{4-t}). В силу единственности стационарного
распределения цепи Маркова уравнение~(\ref{4-t}) справедливо для всей цепи.

\smallskip

В дальнейшем $X, Z$ будут обозначать соответствующие предельные случайные
величины:
$X\hm=\underbrace{\lim}_{k\rightarrow \infty}X_n$,
$Z\hm=\underbrace{\lim}_{k\rightarrow \infty}Z_n$.

Сходимость рассматривается по вероятности:
$P=(|X-\underbrace{\lim}_{k\rightarrow \infty}X_n|>\epsilon)\rightarrow 0$
$\forall\epsilon > 0$, где $|X|$ обозначает любую из векторных норм в~$R^r$.

\smallskip

\noindent
\textbf{Замечание~3.} Равенство~(\ref{4-t})  может рассматриваться
как скалярное или как векторное, где

\noindent
\begin{align*}
\pi(\omega)&=(\pi_1(\omega_1), \pi_2(\omega_2), \ldots ,\pi_r(\omega_r))\,;\\
\pi_i(\omega_i)&=\underbrace{\lim}_{k\rightarrow \infty}
\int\limits^\infty_{-\infty}e^{i \omega y_{ii}}\, dP(X_{n}\hm=y)\,,\
i=1,2,\ldots,r\,,
\end{align*}

\vspace*{-2pt}

\noindent
 причем все координаты кроме~$y_i$ свертываются.

\smallskip

\noindent
\textbf{Замечание~4.} Как видно из~(\ref{4-t}), стационарное распределение
зависит от распределения случайной величины~$Z$ и~от передачи сигнала от~$Z$
к~$X\hm=\underbrace{\lim}_{k\rightarrow \infty} X_n$ через матрицу~$KH$ и~не
зависит напрямую от модели~$F$. Это несколько удивительно, но физически оправдано.
Действительно, неприводимая цепь Маркова аналогична динамической системе с~трением,
в~которой предельный режим если существует, то не зависит от начального состояния,
а~определяется внешним воздействием. В~данном случае со временем состояния цепи
<<перемешиваются>>, а~следовательно, в~пределе стремятся к~равномерному
распределению, независимо от уравнений модели. Поэтому стационарный режим
будет определяться внешним воздействием и~передачей этого воздействия
к~характеристикам цепи Маркова.

\smallskip

\noindent
\textbf{Следствие.}  Моменты стационарного состояния цепи можно найти по формулам:
\begin{multline*}
\hspace*{-6pt}\pi_s=\pi_s(I+KH)^s+ \pi_{s-1}(I+KH)^{s-1} (KH)^{-1}\psi_1+\cdots {}\\
{}\cdots +(KH)^{-s}\psi_s\,.
%\label{5-t}
\end{multline*}
Здесь и~далее обозначено для вектора $a\hm\in R^r$ и~мат\-ри\-цы $B:\ R^r \mapsto R^r$
$$
aB_s=a_i\left(\sum_{j=1}^r B_{ij}\right)\,,\enskip i=1,2,\ldots,r\,.
$$
В частности,

\noindent
\begin{align*}
\pi_1&=(I+KH)^{-2}\psi_1\,;\\
\pi_2&=(I+KH)^{-2}[2\pi_1\psi_1+(I+KH)^{-1}\psi_2]\,.
\end{align*}

В дальнейшем будем рассматривать распределение исходной цепи Маркова
в~стационарном режиме. Без ограничения общности будем предполагать, что
случайная величина~$Z$ имеет нулевое среднее ${\sf E}Z\hm=0$ или $\psi_1\hm=0$.
Понятно, что из-за линейности второго из соотношений~(\ref{1-t})~$X$ можно
представить как ${\sf E}X\hm+X'$, ${\sf E}X'\hm=0$ и~${\sf E}X$
определяется из~(\ref{1-t}) для ненулевого~${\sf E}Z$.

Далее рассматривается схема серий для величин $X$, $Z$ и~$K$,
зависящих от индекса~$l$: $X_l$, $Z_l$ и~$K_l$, $l\hm=1, 2,\ldots$
Вводится также параметр~$\rho_l$ так, что $\rho_l\hm\rightarrow 0$.
Очевидно, что характеристическая функция $\pi(\omega)$ также зависит от~$l$,
что для краткости не показывается.

\smallskip

\noindent
\textbf{Лемма~3.} \textit{Пусть $K_l\hm=\rho_l K$, $Z_l \hm\rightarrow 0$ так, что $\rho_l^{-1} EZ_l^2\hm < \infty$, $\rho_l\hm \rightarrow 0$}.

\textit{Тогда для любого $s > 0$ верно утверждение}:

\vspace*{2pt}

\noindent
\begin{equation}
\rho_l^{s+1}\pi_s\rightarrow 0\,.
\label{6-t}
\end{equation}

\noindent
 Д\,о\,к\,а\,з\,а\,т\,е\,л\,ь\,с\,т\,в\,о\,.\ \ Докажем методом математической
 индукции. Для $s\hm=0$ утверждение~(\ref{6-t}) очевидно. Пусть оно
 справедливо для некоторого $s\hm>0$. Дифференцируя обе
 части~(\ref{4-t}) $s\hm+2$ раза, получаем:

 \noindent
\begin{multline}
\pi^{(s+2)}(0)=\pi^{(s+2)}(0)(I+KH)^{s+2}+{}\\
{}+
\fr{(s+1)(s+2)}{2}\pi^{(s)}(0)(I+KH)^{s}\psi^{(2)}(0) (KH)^{-2}+{}\\
{}+A_s\,,
\label{7-t}
\end{multline}
где $A_s$ содержит слагаемые с~производными $\pi(\omega)$ порядка меньше~$s$.
Для получения формулы~(\ref{7-t}) использовано условие $\psi'\hm=0$.

Уравнение~(\ref{7-t}) можно преобразовать и~получить следующее равенство:

\noindent
\begin{multline*}
\hspace*{-3mm}\pi_{s+2}(KH)(I^{s+1}+(I+(KH))I^s+\cdots +(I+KH)^{s+1})={}\hspace*{-1.96pt}\\
{}=\fr{(s+1)(s+2)}{2}\,\pi_{s} (I+KH)^{s} \psi_{2} (KH)^{-2}+A_s\,.
\end{multline*}
В схеме серий это равенство переписывается как

\noindent
\begin{multline*}
 \pi_{s+2}(K_lH)(I^{s+1}+(I+K_lH))I^s+\cdots\\
 {}\cdots +(I+K_lH)^{s+1})={}
\\
 {}=\fr{(s+1)(s+2)}{2}\pi_{s} (I+K_lH)^{s} \psi_{2} (K_lH)^{-2}+A_s \,.
\end{multline*}

Из условий леммы~3 следует:
\begin{multline}
\pi_{s+2}\rho_l^2(KH)(I^{s+1}+(I+\rho_lKH)I^s+\cdots{}\\
{}\cdots
+(I+\rho_lKH)^{s+1})={}\\
{}=\fr{(s+1)(s+2)}{2}\pi_{s} \rho_l (I+\rho_l KH)^{s} \psi_{2} (KH)^{-2}+{}\\
{}+A_s.
\label{8-t}
\end{multline}
Умножая обе части~(\ref{8-t}) на $\rho_l^{s+1}$, приходим  к~сле\-ду\-юще\-му равенству:
\begin{multline*}
\pi_{s+2}\rho_l^{s+3}(KH)(I^{s+1}+(I+\rho_lKH)I^s+\cdots{}\\
{}\cdots
+(I+\rho_lKH)^{s+1})={}\\
{}=\fr{(s+1)(s+2)}{2}\pi_{s}\rho_l^{s+1} (I+\rho_l KH)^{s}(\rho_l \psi_{2})
(KH)^{-2}+{}\\
{}+A_s \rho_l^{s+1}\,.
\end{multline*}
Если $\rho_l \rightarrow 0$, то правая часть этого уравнения из предположения
индукции и~условий леммы~3 стремится к~$0$. Следовательно, и~левая часть этого
уравнения стремится к~нулю, что и~требовалось доказать.

\smallskip

Введем обозначение  $\lambda_s \hm= \underbrace{\lim}_{\rho_l\rightarrow 0}
\pi_s \rho_l^s$.

\smallskip

\noindent
\textbf{Лемма~4.} \textit{Если выполнены условия леммы~$3$, то
для любого четного~$s$
$$
\lambda_s=(s-1)!!\psi_0^s(KH)^{-2}\,,
$$
где $\psi_0 =\underbrace{\lim}_{\rho_l\rightarrow 0} \rho_l^{-1}\psi_2  (2KH)^{-1}$, \\
и для нечетных}~$s$
 $$
 \lambda_s=0\,.
 $$

\noindent
Д\,о\,к\,а\,з\,а\,т\,е\,л\,ь\,с\,т\,в\,о\,.\ \ Из равенства~(\ref{8-t}) можно получить
\begin{multline}
\pi_{s}\rho_l^2(KH)(I^{s-1}+(I+(\rho_lKH))I^{s-2}+\cdots{}\\
{}\cdots
+(I+\rho_lKH)^{s-1})
={}\\
{}=\fr{s(s-1)}{2}\pi_{s-2} \rho_l^{-1} (I+\rho_l KH)^{s-2}\psi_{2}
(KH)^{-2}+{}\\
{}+A_{s-2}\,.
\label{9-t}
\end{multline}

Умножая обе части~(\ref{9-t}) на $\rho_l^{s-2}$ и~принимая
во внимание утверждение леммы~3, приходим к~следующему равенству:
\begin{equation}
\lambda_s s = \fr{s(s-1)}{2} \,\lambda_{s-2}\left[\rho_l^{-1} \psi_{2}(KH)\right] (KH)^{-2}\,.
\label{10-t}
\end{equation}
Так как $\lambda_0=1$,  $\lambda_1\hm=0$, утверждение леммы~4
непосредственно следует из формулы~(\ref{10-t}).

\smallskip

Теперь сформулируем основное утверждение данной работы.

\smallskip

\noindent
\textbf{Теорема.} Предельное распределение случайной величины~$X$ при
условиях лемм~2 и~3 будет  задаваться формулой:
\begin{equation*}
\underbrace{\lim}_{\rho_l\rightarrow 0} P\left(\rho_l^{-1}(X-{\sf E}X)<x\right)
=
\Phi \left(x; (2KH)^{-2}\psi_0\right)\,.
%\label{11-t}
\end{equation*}

Как обычно,  $\Phi (x;\sigma^2)$ обозначает гауссово распределение
с~нулевым средним и~дисперсией~$\sigma^2$.

\smallskip

\noindent
Д\,о\,к\,а\,з\,а\,т\,е\,л\,ь\,с\,т\,в\,о\ \ следует из разложения Тейлора
для характеристической функции случайной величины~$X$:
$$
\pi((\rho_l \omega) = Ee^{i\omega \rho_l X} = \sum\limits_{j=0}^{\infty}\fr{(i\varpi)^j \rho_l^j \pi_j }{j!}\,.
$$

Поскольку
$$
\sum\limits_{j=0}^{\infty}\fr{(i\varpi)^j \rho_l^j \pi_j }{j!} \rightarrow \sum\limits_{j=0}^{\infty}\fr{(i\omega)^j  }{j!} \lambda_j\ \mbox{при}\  \rho_l \rightarrow 0\,,
$$
 то
\begin{multline*}
\sum\limits_{j=0}^{\infty}\fr{(i\omega)^j  }{j!} \lambda_j ={}\\
{}= \sum\limits_{j=2k, k=0}^{\infty}\fr{(i\omega)^{2k} }{(2k)!} (2k-1)!! \psi_0^{2k} (KH)^{-2k} ={}\\
{} = \sum\limits_{k=0}^{\infty}\fr{(-1)^{k} }{(2k)!!} (\omega \psi_0)^{2k} (KH)^{-2k} ={}\\
{}=
 \sum\limits_{k=0}^{\infty}\fr{(-1)^{k} }{(k)!}  \left(\fr{\omega \psi_0}{2}\right)^{2k} (KH)^{-2k}\,.
 \end{multline*}
Последнее соотношение определяет характеристическую функцию
гауссовой случайной величины с~дисперсией $(2KH)^2$, что и~требовалось показать.

\smallskip

\noindent
\textbf{Замечание~5.} При доказательстве использовалось
существование высших моментов распределений случайных величин $X$ и~$Z$
в~каждой серии. Это ограничение несущественно, так как  в~окончательном ответе
требуется только существование их вторых моментов.

\smallskip

\noindent
\textbf{Замечание~6.} Условие $\rho_l^{-1} {\sf E}Z_l^2 \hm< \infty$,
$l \hm\rightarrow 0$, является стандартным (см., например,~\cite{Gikhman})
для существования предельного распределения Гаусса.

{\small\frenchspacing
 {%\baselineskip=10.8pt
 \addcontentsline{toc}{section}{References}
 \begin{thebibliography}{9}
\bibitem{Jazwinski}
\Au{Jazwinski~A.\,H.} Stochastic processes and filtering theory.~--- New York, NY, USA: Academic Press, 1970. 376~p.

\bibitem{Ghil}
\Au{Ghil~M.,  Malnotte-Rizzoli~P.} Data assimilation in meteorology and oceanography~// Adv. Geophys., 1991. Vol.~33. P.~141--266.

\bibitem{Evensen1}
\Au{Evensen~G.} Sequential data assimilation with a non-linear quasi-geostrophic model using Monte-Carlo methods to forecast error statistics~// J.~Geophys. Res., 1994. Vol.~6. P.~1014--1062.

\bibitem{Evensen2}
\Au{Evensen~G.} The ensemble Kalman filter: Theoretical formulation and practical implementation~// Ocean Dyn., 2003. Vol.~53. P.~343--367.

\bibitem{Belyaev1}
\Au{Belyaev~K., Tanajura~C.\,A.\,S., O'Brien~J.\,J.}  A data assimilation technique with an ocean circulation model and its application to the tropical Atlantic~// Appl. Math. Model., 2001. Vol.~25. P.~655--670.

\bibitem{Tanajura}
\Au{Tanajura~C.\.A.\,S., Belyaev~K.} A sequential data assimilation method based on the properties of diffusion-type process~// Appl. Math. Model., 2009.  Vol.~33. P.~2165--2174.

\bibitem{Belyaev2}
\Au{Belyaev~K., Nazarov~L.} Limit theorems for characteristics of a queuing system with batch processing~// Theory Prob. Appl., 1995.  Vol.~40. No.~4. P.~73--78.

\bibitem{Gikhman}
\Au{Gikhman~I., Skorokhod~A.} An introduction to the theory of random processes.~--- New York, NY, USA: Dover Publ. Inc., 1996. 519~p.
 \end{thebibliography}

 }
 }

\end{multicols}

\vspace*{-3pt}

\hfill{\small\textit{Поступила в~редакцию 19.02.15}}

\newpage

%\vspace*{12pt}

%\hrule

%\vspace*{2pt}

%\hrule

\vspace*{-24pt}

\def\tit{ON A LIMIT DISTRIBUTION OF CHARACTERISTICS IN STATIONARY REGIME FOR THE LINEAR ASSIMILATION PROBLEM}

\def\titkol{On a limit distribution of characteristics in stationary regime for the linear assimilation problem}

\def\aut{K.\,P.~Belyaev$^1$ and N.\,P.~Tuchkova$^2$}

\def\autkol{K.\,P.~Belyaev and N.\,P.~Tuchkova}

\titel{\tit}{\aut}{\autkol}{\titkol}

\index{Belyaev K.\,P.}
\index{Tuchkova N.\,P.}

\vspace*{-9pt}

\noindent
$^1$Shirshov Institute of Oceanology , Russian Academy of Sciences,
36~Nakhimovsky Pr., Moscow 119299, Russian\linebreak
$\hphantom{^1}$Federation

\noindent
$^2$Dorodnicyn Computing Centre, Russian Academy of Sciences,
40~Vavilov Str., Moscow 119333, Russian\linebreak
$\hphantom{^1}$Federation


\def\leftfootline{\small{\textbf{\thepage}
\hfill INFORMATIKA I EE PRIMENENIYA~--- INFORMATICS AND
APPLICATIONS\ \ \ 2015\ \ \ volume~9\ \ \ issue\ 2}
}%
 \def\rightfootline{\small{INFORMATIKA I EE PRIMENENIYA~---
INFORMATICS AND APPLICATIONS\ \ \ 2015\ \ \ volume~9\ \ \ issue\ 2
\hfill \textbf{\thepage}}}

\vspace*{3pt}


\Abste{A commonly used linear assimilation problem when the model
state vector is corrected by observed data through the system of linear
equations is considered. This problem is formulated as a Markov chain problem.
For this problem, convergence of transitional probability of the corresponding
Markov chain is investigated and the sufficient conditions of this
convergence are found out.  A~special case of series depending on a~parameter
when this parameter goes to zero is discussed and the limit theorem about
convergence to Gaussian distribution for analysis of state vector characteristics
for this case is proved. The mean value and variance of this distribution are
determined. The paper discusses how these results can
be applied to practical operational assimilation and forecasting.}

\KWE{data assimilation methods; Markov chains stationary distributions; asymptotic distribution of chains at a small value of a parameter}




\DOI{10.14357/19922264150206}

\Ack
\noindent
The research was supported by the Russian Foundation for Basic Research
 (projects 14-05-00363 and 14-07-00037).



%\vspace*{3pt}

  \begin{multicols}{2}

\renewcommand{\bibname}{\protect\rmfamily References}
%\renewcommand{\bibname}{\large\protect\rm References}



{\small\frenchspacing
 {%\baselineskip=10.8pt
 \addcontentsline{toc}{section}{References}
 \begin{thebibliography}{9}
\bibitem{1-tu}
\Aue{Jazwinski, A.\,H.} 1970. \textit{Stochastic processes and filtering theory}. New York, NY: Academic Press. 376 p.
\bibitem{2-tu}
\Aue{Ghil, M., and P.~Malnotte-Rizzoli}. 1991. Data assimilation in meteorology and oceanography. \textit{Adv. Geophys.} 33:141--266.
\bibitem{3-tu}
\Aue{Evensen, G.} 1994. Sequential data assimilation with a non-linear quasi-geostrophic model using Monte-Carlo methods to forecast error statistics.
\textit{J.~Geophys. Res.} 6:1014--1062.
\bibitem{4-tu}
\Aue{Evensen, G.} 2003. The ensemble Kalman filter: Theoretical
formulation and practical implementation. \textit{Ocean Dyn.} 53:343--367.
\bibitem{5-tu}
\Aue{Belyaev, K., C.\,A.\,S.~Tanajura, and J.\,J.~O'Brien}. 2001.
A~data assimilation technique with an ocean circulation model
and its application to the tropical Atlantic. \textit{Appl. Math. Model.} 25:655--670.
\bibitem{6-tu}
\Aue{Tanajura, C.\,A.\,S., and K.~Belyaev}. 2009. A sequential data assimilation method based on the properties of diffusion-type process. \textit{Appl. Math. Model.} 33:2165--2174.
\bibitem{7-tu}
\Aue{Belyaev, K., and L.~Nazarov}. 1995. Limit theorems for characteristics of a queuing system with batch processing. \textit{Theory Prob. Appl.} 40(4):73--78.
\bibitem{8-tu}
\Aue{Gikhman, I., and A.~Skorokhod}. 1996.  \textit{An introduction to the theory of random processes}. New York, NY: Dover Publ. Inc. 519~p.
\end{thebibliography}

 }
 }

\end{multicols}

\vspace*{-3pt}

\hfill{\small\textit{Received February 19, 2015}}

%\vspace*{-18pt}

\Contr

\noindent
\textbf{Belyaev Konstantin P.} (b.\ 1955)~--- Doctor of Science in physics
and mathematics; leading scientist, Shirshov Institute of Oceanology,
Russian Academy of Sciences, 36~Nakhimovsky Pr., Moscow 119299, Russian Federation;
kosbel55@gmail.com

\vspace*{3pt}

\noindent
\textbf{Tuchkova Natalia P.} (b.\ 1955)~--- Candidate of Science (PhD)
in physics and mathematics; senior scientist, Dorodnicyn Computing Centre,
Russian Academy of Sciences, 40~Vavilov Str., Moscow 119333, Russian Federation;
tuchkova@ccas.ru


\label{end\stat}


\renewcommand{\bibname}{\protect\rm Литература}  %6
\def\stat{vagapova}

\def\tit{СРАВНИТЕЛЬНЫЙ АНАЛИЗ ПРИМЕНЕНИЯ ЭВРИСТИЧЕСКОГО И~МЕТАЭВРИСТИЧЕСКОГО АЛГОРИТМОВ К~ЗАДАЧЕ О~ШКОЛЬНОМ АВТОБУСЕ$^*$}

\def\titkol{Сравнительный анализ применения эвристического и~метаэвристического алгоритмов к задаче о школьном автобусе}

\def\aut{Е.\,М.~Бронштейн$^1$, Д.\,М.~Вагапова$^2$}

\def\autkol{Е.\,М.~Бронштейн, Д.\,М.~Вагапова}

\titel{\tit}{\aut}{\autkol}{\titkol}

\index{Бронштейн Е.\,М.}
\index{Вагапова Д.\,М.}

{\renewcommand{\thefootnote}{\fnsymbol{footnote}} \footnotetext[1]
{Работа выполнена при поддержке РФФИ (проект 13-01-00005).}}


\renewcommand{\thefootnote}{\arabic{footnote}}
\footnotetext[1]{Уфимский государственный авиационный технический университет, bro-efim@yandex.ru}
\footnotetext[2]{Уфимский государственный авиационный технический университет, vagapova-dm@mail.ru}

\Abst{Рассматривается задача о школьном автобусе, которая заключается в~обеспечении доставки школьников по окончании занятий из школы по их
остановкам. Целью является минимизация длины максимального из
маршрутов. Представлен краткий обзор работ по данной
тематике.  Приведена постановка задачи и~формализация.
Описан эвристический алгоритм, предложенный
авторами ранее. Также описан двухэтапный алгоритм на основе
метаэвристики муравьиной колонии: после первоначальной кластеризации
остановок, на которых высаживаются школьники, к каждому кластеру
применяется алгоритм муравьиной колонии с~различными значениями
параметров. Представлены результаты сравнения
эффективности предложенных алгоритмов, а~также результаты работы
программ для двух алгоритмов.}  %В~заключении приводятся выводы.}

\KW{маршрутизация; задача о школьном автобусе; алгоритм муравьиной
колонии; кластеризация}

\DOI{10.14357/19922264150207}

\vskip 14pt plus 9pt minus 6pt

\thispagestyle{headings}

\begin{multicols}{2}

\label{st\stat}

\section{Введение}

     Задача о школьном автобусе (School Bus Routing Problem, SBRP)
заключается в~доставке школьников после окончания занятий до их остановок
за минимальное время. Впервые задача была сформулирована в~\cite{1-va}.
Задача о школьном автобусе входит в~общее семейство задач маршрутизации\linebreak
(Vehicle Routing Problem, VPR). Обзор точных и~эвристических методов
решения задач маршрутизации пред\-став\-лен в~\cite{2-va}. Задача маршрутизации с~одним транспортным средством без дополнительных ограничений является
задачей коммивояжера (Traveling Salesman Problem, TSP).

   Рассматриваемые задачи относятся к классу NP-труд\-ных, поэтому для их
решения целесообразно использовать эвристические и~метаэвристические
алгоритмы. Простой эвристический алгоритм решения задачи о школьном
автобусе предложен в~работе~\cite{3-va}. В~статье разработано решение
данной задачи с~помощью двухэтапного подхода на основе метаэвристического
алгоритма муравьиной колонии (Ant Colony Optimization, ACO). На первом
этапе производится кластеризация остановок, на которых высаживаются
школьники, а~затем к каж\-до\-му кластеру применяется алгоритм муравьиной
колонии. Для расширения области поиска решений итерационно производится
смещение границ кластеров. Число таких итераций равно числу остановок.
Впервые двухэтапный подход такого типа для решения задачи маршрутизации
был предложен в~\cite{4-va}.

     В~\cite{5-va} предложен новый подход к решению задачи о школьном
автобусе в~городе. Рассматривается многомерная целевая функция. Задача
разбивается на две подзадачи. Первая~--- это распределение студентов по
автобусам, и~вторая~--- построение автобусных маршрутов. Для первой
подзадачи используется алгоритм районирования.

     Решение задачи о~школьном автобусе с~по\-мощью модифицированного
алгоритма муравьиной колонии предложено в~[6]. Задача решается на
реальных данных г.~Богота (Колумбия). Используется двухэтапный подход:
сначала производится кластеризация остановок согласно их географическому
расположению (с~севера на юг, затем с~запада на восток), а~потом применяется
модифицированный алгоритм муравьиной колонии для задачи коммивояжера.

     Алгоритм муравьиной колонии является достаточно эффективной
метаэвристикой для нахождения приближенных решений различных задач
дискретной оптимизации. Результаты, полученные с~помощью разработанного
варианта двухэтапного подхода, сравниваются с~результатами, полученными с~помощью эвристического алгоритма из работы~\cite{3-va}.

\section{Постановка задачи}

     Более подробное описание задачи приведено в~\cite{3-va}. По окончании
занятий требуется развезти школьников по их остановкам на нескольких
автобусах. При этом школьников следует доставить за минимально возможное
время. Это условие будет выполнено, если маршрут с~максимальной длиной
будет минимальным среди всех способов доставки. Предполагается, что
автобусы движутся с~постоянной скоростью.

     Для построения математической модели введем соответствующие
величины.
     \begin{enumerate}[1.]
\item Для каждой из $N$ остановок задана величина~$D_n$ ($n\hm=1,\ldots
,N$)~--- число школьников, которых надо доставить на эту остановку. Для
единообразия школу считаем 0-й остановкой.
\item Каждый из $K$ автобусов характеризуется чис\-лом~$T^k$ ($k\hm = 1,\ldots
, K$) посадочных мест.
\item Матрица $C =\parallel c(i, j)\parallel$ содержит минимальные
расстояния (от $i$-й остановки до $j$-й). Элементы матрицы~$C$
удовлетворяют неравенству треугольника.
\item Необходимо найти величины~$x_n^k$ ($n\hm= 1,\ldots, N$; $k\hm=1,\ldots
,K$)~--- число школьников, которых необходимо доставить $k$-м автобусом на
$n$-ю остановку.
\end{enumerate}

     Опишем ограничения задачи.

     На каждую остановку, кроме школы,
должно быть доставлено не\-от\-ри\-ца\-тель\-ное число школьников:
     \begin{equation}
     x_n^k\geq 0 \enskip (n=1,\ldots , N;\ k=1,\ldots , K), \ \mbox{целые}\,.
     \label{e1-va}
     \end{equation}

     На каждую остановку необходимо доставить всех школьников, для
которых это необходимо:
     \begin{equation}
     \sum\limits_{k=1}^K x_n^k=D_n\enskip (n=1,\ldots, N)\,.
     \label{e2-va}
     \end{equation}

     Ограничения на вместимость автобусов:
     \begin{equation}
     \sum\limits_{n=1}^N x_n^k \leq T^k\enskip (k=1,\ldots, K)\,.
     \label{e3-va}
     \end{equation}

     Добавлением виртуальных школьников в~количестве
$\sum\nolimits_{k=1}^K T^k\hm- \sum\nolimits_{n=1}^N D_n$ (в~случае
положитель\-ности этой величины), для которых пунктом доставки является
школа, можно добиться того, что ограничения~(\ref{e3-va}) выполняются в~форме равенств. Таким образом, полагаем, что общее число школьников равно
числу мест в~автобусах: $\sum\nolimits_{n=1}^N D_n\hm= \sum\nolimits_{k=1}^K
T^k$.

     Пусть $R^k=\left\{ i\colon x_i^k\geq1\right\}$~--- множество остановок, на
которых должен высадить пассажиров \mbox{$k$-й} автобус. Следует отметить, что
величинами~$x_n^k$ маршруты автобусов однозначно не определяются. Пусть
$P(R^k)$~--- минимальная длина маршрута, проходящего через все остановки
из~$R^k$, у~которого начальный пункт~--- школа.

     Целевая функция:
     \begin{equation}
     \max\limits_k \left\{ P(R^k)\right\}\to \min\,.
     \label{e4-va}
     \end{equation}

     Результатом решения задачи~(\ref{e1-va})--(\ref{e4-va}) является график
доставки, при котором школьник, покидающий автобус последним, проедет
минимально возможное расстояние. Очевидно, что
     задача~(\ref{e1-va})--(\ref{e4-va}) имеет решение (возможно, не
единственное).

В работе~\cite{3-va} путем добавления новых переменных данная задача
сведена к линейной частично целочисленной форме, для задач малой
раз\-мер\-ности приведено точное решение, а~также предложен эвристический
алгоритм, описанный далее.

\section{Эвристический алгоритм}

     Для простоты полагаем, что все автобусы имеют одну и~ту же
вместимость~$T$.

     Найдем максимальное из расстояний в~графе от нулевой вершины
(школы) до всех вершин. Пусть соответствующий путь имеет вид
     0--$i_1$--$\cdots$--$i_k$.

Заполняем автобус по следующим правилам.

     Если $\sum\nolimits_{s=1}^k D_{i_s} \geq T$, то автобус доставит школьников на
остановки, расположенные на пути 0--$i_1$--$\cdots$--$i_k$, начиная от
последней в~пределах вмес\-ти\-мости (может оказаться, что на одну из остановок
будут доставлены этим автобусом не все школьники, которым это необходимо),
иначе \mbox{найдем} остановки~$i_{s^*}, j^*$, на которых достигается
     $\min\limits_{s,j}\ \left\{
     c(i_k,j)\colon \ \ j\hm\not\in \{i_0,\ldots, i_k\};\ \
     c(i_s,j)\hm+c(j,i_{s+1}) \hm-\right.$\linebreak $\left.-\;c(i_s,i_{s+1})\colon
s\hm=0,\ldots, k-1, j\hm\not\in
\{i_0,\ldots, i_k\}
\right\}$,
где $c(i, j)$~--- минимальное расстояние между остановками, $i_0\hm =0$.

     Если число школьников, которых необходимо доставить на остановки
$i_1,\ldots, i_k, j^*$, не меньше~$T$, то назначаем автобус, который доставит
школьников в~пределах вместимости автобусов на остановки пути
     0--$i_1$--$\cdots$--$i_k$--$j^*$, начиная с~последней, если минимум
достигается на величинах первого вида, и~пути
     0--$i_1$--$\cdots$--$i_{s^*}$--$j^*$--$i_{s^*+1}$--$\cdots$--$i_k$ если
минимум достигается на величинах второго типа.

     В противном случае процесс добавления вершин к пути продолжается
аналогично до заполнения автобуса. В~результате как число свободных
автобусов, так и~число остановок, на которые еще не доставлены все
школьники, уменьшаются. Далее автобусы заполняются последовательно по
тому же алгоритму.

\section{Двухэтапный алгоритм на~основе метаэвристики
муравьиной колонии}

     Алгоритм муравьиной колонии был предложен М.~Дориго~\cite{7-va}.
Алгоритм имитирует поведение му-\linebreak равьев в~колонии. Отдельный муравей не
\mbox{способен} выжить, однако вся колония достигает высокого уровня
самоорганизации благодаря <<со\-ци\-аль\-ности>> муравьев. Цель муравьев~---
найти кратчайший путь от жилища до источника пищи. В~ряде работ
(см., например,~[8--14]) были предложены различные модификации алгоритма.
В~данной работе воспользуемся классическим вариантом данного алгоритма.

     Введем следующие обозначения:
\begin{description}
\item[\,]     $A_{ij}(t)$~--- количество феромона на дуге $(i, j)$ перед $t$-й итерацией
(начальные значения $A_{ij}(1)$ принимаются одинаковыми положительными
величинами для всех дуг);
\item[\,]
     $\tau_{ij}^k(t)$~--- количество феромона, которое $k$-й муравей
оставляет на дуге $(i,j)$ на $t$-й итерации;
\item[\,]
     $B_{ij}(t)=\sum\nolimits_{k=1}^m \tau_{ij}^k(t)$~--- количество феромона,
которое все муравьи оставляют на дуге $(i,j)$ на $t$-й итерации ($m$~--- число
муравьев).
\end{description}

     Для достижения цели муравьи пользуются четырьмя принципами:
     \begin{enumerate}[1.]
\item \textit{Положительная обратная связь. }
\begin{equation}
\tau_{ij}^k(t) =\begin{cases}
\fr{Q}{L^k(t)}\,, &\ \mbox{если } (i,j)\in T^k(t)\,;\\
0 & \ \mbox{иначе}\,,
\end{cases}
\label{e5-va}
\end{equation}
где $T^k(t)$~--- маршрут, пройденный $k$-м му\-равь\-ем на $t$-й итерации;
$L^k(t)$~--- длина маршрута $T^k(t)$;
$Q$~--- масштабирующий параметр, одного порядка с~длиной оптимального
маршрута (в~частности, за $Q$ можно принять длину самого длинного пути от
школы до всех остановок).
\item \textit{Отрицательная обратная связь.} Феромон испаряется со
временем. После каждой итерации количество феромона определяется по
сле\-ду\-ющей формуле:
\begin{equation}
A_{ij}(t+1) =(1-\rho) A_{ij}(t)+B_i(t)\,,
\label{e6-va}
\end{equation}
где $\rho$~--- коэффициент испарения, $\rho \hm\in [0,1]$.
\item \textit{Случайность.} Каждый муравей выбирает дугу случайным
образом. Вероятность перехода \mbox{$k$-го} муравья от $i$-й остановки к $j$-й на
итерации~$t$ вычисляется по следующей формуле:

\noindent
\begin{multline}
\rho_{ij}^k(t)={}\\{}=\begin{cases}\fr{[A_{ij}(t)]^\alpha [1/D_{ij}]^\beta} {\sum_{l\in
J_i^k} [A_{ij}(t)]^\alpha [1/D_{il}]^\beta}\,, &\ \mbox{если } j\in J_i^k\,;\\
0 &\ \mbox{иначе}\,,
\end{cases}
\label{e7-va}
\end{multline}
где $D_{ij}$~--- расстояние от $i$-й до $j$-й остановки;
$\alpha$~--- параметр, регулирующий влияние уровня феромона на вероятность
$p_{ij}^k(t)$, при $\alpha \hm=0$ муравей будет выбирать следующую
остановку, только исходя из расстояния до нее (жадный алгоритм);
$\beta$~--- параметр, регулирующий влияние расстояния между остановками на
вероятность $p_{ij}^k(t)$, при $\beta\hm = 0$ муравей будет выбирать
следующую остановку, только исходя из уровня феромона, что приведет к
быст\-рой схо\-ди\-мости к некоторому субоптимальному решению;
$J_i^k$~--- список остановок, еще не посещенных $k$-м муравьем.

По формуле~(\ref{e7-va}) вычисляются вероятности перехода муравья на
каждую из непосещенных остановок, выбор следующей остановки
производится случайно в~соответствии с~вычисленными вероятностями.
\item \textit{Многократность (множество муравьев)}. Длинные дуги будут
содержать меньшее количество феромона, чем короткие, в~соответствии
с~(\ref{e5-va}). Муравьи будут стремиться к дугам с~большим количеством
феромонов и~тем самым будут предпочитать короткие дуги длинным.
\end{enumerate}

     Двухэтапный алгоритм решения задачи о~школьном автобусе имеет
следующую структуру. Первый этап заключается в~кластеризации пунктов
(остановок). На втором этапе для построения маршрутов будет использоваться
алгоритм муравьиной колонии с~различными значениями параметров~$\alpha$,
$\beta$, $\rho$. Заметим, что для формирования маршрута одного автобуса
запускается колония муравьев.

     \textit{Первый этап~--- кластеризация.}

     Целью данного этапа является распределение школьников по маршрутам
(автобусам). Для этого разобьем всю плоскость, на которой лежат остановки
(рис.~1), на кластеры (штриховые линии~--- границы кластеров), каждый
кластер соответствует одному автобусу. Выбираем начальный луч, проходящий
через некоторую остановку. Двигаемся по часовой стрелке (на рис.~1
обозначена пунктиром). Кластером является минимальный сектор, для
ко-\linebreak\vspace*{-12pt}

\begin{center}  %fig1
\vspace*{-1pt}
 \mbox{%
 \epsfxsize=71.512mm
 \epsfbox{vag-1.eps}
 }


\end{center}

\noindent
{{\figurename~1}\ \ \small{Схема разбиения множества остановок на клас\-теры}}




\vspace*{9pt}


\addtocounter{figure}{1}


\noindent
торого число школьников, которых надо доставить
 на остановки,
принадлежащие этому кластеру, не
меньше вместимости транспортного
средства. Затем формируются следующие кластеры. Если упомянутое число
школьников в~кластере больше вмес\-ти\-мости автобуса, то граничная остановка
кластера войдет в~следующий кластер с~соответственно уменьшенным чис\-лом
школьников. Для по\-стро\-ения начальных лучей при разбиении на кластеры
последовательно перебираются все остановки.



     Приведем пример, который показывает целесообразность изменения
начального луча. Пусть три остановки имеют координаты: $A(1,\,0)$, $B(1,\,1)$,
$C(0,\,1)$; школа расположена в~точке О(0,\,0); вместимость автобуса~---
20~школьников; автобусов~--- два; на остановки~$A$ и~$C$ надо доставить по
10~школьников; на остановку~$B$~--- 20; прямые дороги с~двусторонним
движениям соединяют пары остановок ($OA$), ($AB$), ($BC$), ($CO$). Если в~качестве
начального принять луч~$OB$, то в~один кластер попадает остановка~$B$,
а~в~другой~--- остановки~$A$ и~$C$. Расстояние, необходимое для доставки одного
из школьников во втором кластере, равно~3. Если же в~качестве начального
луча взять~$OA$, то в~один кластер попадут остановки~$A$ и~$B$ (с частичной
доставкой), а~во второй~--- $B$ (с~частичной доставкой) и~$C$. Максимальное
расстояние доставки равно 2.

     \textit{Второй этап~--- маршрутизация.}

     Целью данного этапа является построение маршрутов движения
автобусов, т.\,е.\
установление порядка посещения остановок автобусами
в~каж\-дом кластере (минимальных гамильтоновых путей). Следует отметить, что
маршруты в~общем случае могут выходить за границы кластеров. Для
по\-стро\-ения маршрутов используется алгоритм муравьиной колонии. Автобус
доставляет всех школьников на каждую из остановок кластера, кроме
последней, а~оставшихся школьников на эту остановку до\-став\-ля\-ет автобус,
обслуживающий следующий кластер (возможно, не один). Ниже приводится
описание алгоритма.
     \begin{enumerate}[1.]
\item Цикл по числу кластеров $u\hm= (1$, число автобусов).\\[-13.5pt]
\item Инициализация параметров алгоритма~--- $\alpha$, $\beta$, $\rho$,
$Q$, $A_{ij}(1)$.\\[-13.5pt]
\item Размещение муравьев в~начальной вершине (школе). Число
муравьев равно числу остановок на данном маршруте.\\[-13.5pt]
\item Выбор начального кратчайшего маршрута и~определение длины
этого пути~$L^{u*}$.\\[-13.5pt]
\item Цикл по времени жизни колонии $t\hm=(1$, время жизни колонии).\\[-13.5pt]
\item Цикл по числу муравьев $ k\hm = (1$, число остановок на маршруте
($m$)).\\[-13.5pt]
\item Построение маршрута $T^{ku}(t)$ по формуле~(\ref{e7-va}) и~расчет длины~$L^{ku}(t)$.\\[-13.5pt]
\item Конец цикла по муравьям.\\[-13.5pt]
\item Проверка всех $L^{ku}(t)$ на лучшее решение по сравнению
с~$L^{u*}$.\\[-13.5pt]
\item В случае если решение $L^{ku}(t)$ лучше, обновление~$L^{u*}$
и~$T^{u*}$.\\[-13.5pt]
\item Цикл по всем дугам графа.\\[-13.5pt]
\item Обновление следов феромона на дуге по правилам~(\ref{e5-va})
и~(\ref{e6-va}).\\[-13.5pt]
\item Конец цикла по дугам.\\[-13.5pt]
\item Конец цикла по времени жизни колонии.\\[-13.5pt]
\item Вывод кратчайшего маршрута~$T^{u*}$ и~его длины~$L^{u*}$.\\[-13.5pt]
\item Конец цикла по кластерам.
\end{enumerate}

\vspace*{-9pt}



\section{Вычислительный эксперимент}

\vspace*{-2pt}

     Двухэтапный алгоритм, описанный в~разд.~4, и~эвристический алгоритм,
описанный в~разд.~3, были реализованы в~среде Borland DELPHI~6. Вычисления
проводились на персональном компьютере
(процессор: Intel(R) Core(TM) 2~Duo CPU T5750 2,00~GHz,
ОЗУ (RAM): 3~ГБ, ОС: Windows Vista Home Premium~32). Число автобусов
принималось равным~10 или~20. Вместимость автобусов принималась равной
20, 30 или~60. Число остановок~--- 50, 60, 70 или~80. Параметр~$\alpha$
принимался равным~0, 1, 2 или~6; $\beta$~--- 0, 1, 2,~6; $\rho$~---
0,2, 0,5, 0,7;
$Q\hm = 10\,000$; начальный уровень феромона $A_{ij}(1) \hm= 100$. Время
жизни колонии~--- 200. С~помощью датчика псевдослучайных чисел
генерировались значения
 координат каждой остановки
(расстояние бралось
 евклидово), а~также чис\-ло школьников на каждой\linebreak\vspace*{-12pt}

\pagebreak

\end{multicols}

     \begin{figure} %fig2
         \vspace*{1pt}
 \begin{center}
 \mbox{%
 \epsfxsize=162.28mm
 \epsfbox{vag-2.eps}
 }
\end{center}
 \vspace*{-13pt}
\Caption{Время работы эвристического~(\textit{а})
и~двухэтапного~(\textit{б}) алгоритмов}
%\end{figure*}
%     \begin{figure*} %fig3
              \vspace*{3pt}
 \begin{center}
 \mbox{%
 \epsfxsize=162.28mm
 \epsfbox{vag-4.eps}
 }
\end{center}
 \vspace*{-13pt}
     \Caption{Отношение времени работы двухэтапного алгоритма к времени работы
эвристического алгоритма~(\textit{а})
и~отношение максимальных длин маршрутов, полученных двухэтапным
и~эвристическим алгоритмами~(\textit{б})}
\vspace*{-3pt}
      \end{figure}


\begin{multicols}{2}

\noindent
остановке с~учетом того, что общее число школьников равно общему чис\-лу
посадочных мест. Для каждого
чис\-ла остановок генерировалось 10~примеров.
{\looseness=-1

}

     Рисунки~2 и~3 иллюстрируют зависимость средних значений величин,
полученных в~ходе вы\-чис\-ли\-тельного эксперимента, от числа остановок.
     	

     Отметим, что приводится время расчета (до 4~мин)\ для одного набора
параметров алгоритма муравьиной колонии и~для одного расположения
начального луча на первом этапе, а~именно для тех, при которых получен
лучший результат. При переборе всех наборов величин время вычислений
доходит до 10~ч.

\vspace*{-14pt}


    \section{Заключение}

    \vspace*{-4pt}

В работе рассматривается задача о школьном автобусе, разработан
двухэтапный алгоритм решения, основанный на алгоритме муравьиной
колонии. Производится сравнение его эффективности с~простым
эвристическим алгоритмом.

	Вычислительный эксперимент показал, что
\begin{enumerate}[(1)]
\item время работы эвристического алгоритма слабо
изменяется с~ростом числа остановок;
\item время работы двухэтапного алгоритма и~его отношение к времени
работы эвристического алгоритма резко возрастают с~ростом числа остановок;
\item отношение максимальных длин маршрутов находится в~промежутке
0,75--1,15 и~слабо изменяется с~ростом числа остановок.
\end{enumerate}

Таким образом, простой эвристический алгоритм сопоставим по
эффективности полученного результата с~двухэтапным, но гораздо менее
трудоемок.

\vspace*{-9pt}

{\small\frenchspacing
 {%\baselineskip=10.8pt
 \addcontentsline{toc}{section}{References}
 \begin{thebibliography}{99}
\bibitem{1-va}
\Au{Newton R.\,M., Thomas W.\,H.} Design of school Bus Routes by computer~//
Socio-Economic Planning Science, 1969. Vol.~3. P.~75--85.
\bibitem{2-va}
\Au{Archetti C.} Matheuristics for routing problems. {\sf
http:// www.sintef.no/contentassets/cfb19ab9b7c74d03904c7\linebreak
746ee1d8e77/matheuristics\_routing\_verolog2014\_new.\linebreak pdf}.
\bibitem{3-va}
\Au{Бронштейн Е.\,М., Вагапова Д.\,М., Назмутдинова~А.\,В.} О~построении
семейства маршрутов доставки школьников за минимальное время~//
Автоматика и~телемеханика, 2014. №\,7. С.~43--51.
\bibitem{4-va}
\Au{Fisher M.\,L., Jaikumar R.} A~generalized assignment heuristic for vehicle
routing~// Networks, 1981. Vol.~11. No.\,2. P.~109--124.
\bibitem{5-va}
\Au{Bowerman R., Hall B., Calamai P.} A~multi-objective optimization approach to
urban school Bus Routing: Formulation and solution method~// Transportation
Research Part~A: Policy and Practice, 1995. Vol.~29. No.\,2. P.~107--123.
\bibitem{6-va}
\Au{Arias-Rojas J.\,S., Jimenez J.\,F., Montoya-Torres~J.\,R.} Solving of school bus
routing problem by ant colony~// Revista EIA, 2012. Vol.~9. No.\,17. P.~193--208.
\bibitem{7-va}
\Au{Dorigo M.} Optimization, learning and natural algorithms. PhD Thesis.~---
Milano, Italy: Politecnico di Milano, 1992.
\bibitem{11-va} %8
\Au{Gambardella L.\,M., Dorigo M.} Ant-Q: A~reinforcement learning approach to
the traveling salesman problem~// 12th Conference (International) on Machine
Learning.~--- Tahoe City: Morgan Kaufmann, 1995. P.~252--260.
\bibitem{8-va} %9
\Au{Dorigo M., Maniezzo V., Colorni~A.} The Ant System: Optimization by a
colony of cooperating agents~// IEEE Trans. Systems Man Cybernetics. Part~B,
1996. Vol.~26. No.\,1. P.~29--41.
\bibitem{9-va} %10
\Au{Dorigo M., Gambardella L.\,M.} Ant Colony System: A~cooperative learning
approach to the traveling salesman problem~// IEEE Trans. Evol.
Comput., 1997. Vol.~1. No.\,1. P.~53--66.
\bibitem{10-va} %11
\Au{St$\ddot{\mbox{u}}$tzle T., Hoos H.} MAX-MIN Ant System and local search
for the traveling salesman problem~// IEEE International Conference on Evolutionary
Computation.~--- Indianapolis: IEEE, 1997. P.~309--314.

\bibitem{12-va}
\Au{Штовба С.\,Д.} Муравьиные алгоритмы~// Exponenta Pro. Математика в~приложениях, 2003. №\,4. С.~70--75.
\bibitem{13-va}
\Au{Курейчик В.\,М., Кажаров А.\,А.} О~некоторых модификациях
муравьиного алгоритма~// Известия Южного федерального университета.
Технические науки, 2008. Т.~81. №\,4. С.~7--12.
\bibitem{14-va}
\Au{Сластников С.\,А.} Применение метаэвристических алгоритмов для задачи
маршрутизации транспорта~// Экономика и~математические методы, 2014.
Т.~50. №\,1. С.~117--126.

 \end{thebibliography}

 }
 }

\end{multicols}

\vspace*{-9pt}

\hfill{\small\textit{Поступила в~редакцию 19.02.15}}

%\newpage

\vspace*{7pt}

\hrule

\vspace*{2pt}

\hrule

%\vspace*{12pt}

\def\tit{COMPARATIVE ANALYSIS OF APPLICATION OF~HEURISTIC AND~METAHEURISTIC ALGORITHMS TO~THE~SCHOOL BUS ROUTING PROBLEM}

\def\titkol{Comparative analysis of application of heuristic and metaheuristic algorithms to the school bus routing problem}

\def\aut{E.\,M.~Bronshtein and D.\,M.~Vagapova}

\def\autkol{E.\,M.~Bronshtein and D.\,M.~Vagapova}

\titel{\tit}{\aut}{\autkol}{\titkol}

\index{Bronshtein E.\,M.}
\index{Vagapova D.\,M.}

\vspace*{-12pt}

\noindent
Ufa State Aviation Technical University, 12 K.~Marx Str., Ufa 450000,
Russian Federation


\def\leftfootline{\small{\textbf{\thepage}
\hfill INFORMATIKA I EE PRIMENENIYA~--- INFORMATICS AND
APPLICATIONS\ \ \ 2015\ \ \ volume~9\ \ \ issue\ 2}
}%
 \def\rightfootline{\small{INFORMATIKA I EE PRIMENENIYA~---
INFORMATICS AND APPLICATIONS\ \ \ 2015\ \ \ volume~9\ \ \ issue\ 2
\hfill \textbf{\thepage}}}

\vspace*{2pt}


\Abste{This paper considers the school bus routing problem,
which is to ensure delivery of students after lessons from school to their stops.
The objective function is to minimize the maximum length of the routes.
A~short review of the literature on this theme is provided.
The problem definition and formalization is given. The
heuristic algorithm proposed by the authors earlier is described.
A~two-step algorithm based on ant colony metaheuristics is described.
The algorithm consists of initial clustering of stops at which students drop
off, and subsequent ant colony optimization with different parameters,
which is applied to each cluster. The results of comparing the efficiency of
the proposed algorithms and the performance of the program for two algorithms
are presented.}

\KWE{vehicle routing problem; school bus routing problem; ant colony optimization; clustering}


\DOI{10.14357/19922264150207}

\vspace*{-20pt}

\Ack

\vspace*{-2pt}
\noindent
The research was supported by the Russian Foundation for
Basic Research (project 13-01-00005).


\vspace*{-3pt}


  \begin{multicols}{2}

\renewcommand{\bibname}{\protect\rmfamily References}
%\renewcommand{\bibname}{\large\protect\rm References}



{\small\frenchspacing
 {%\baselineskip=10.8pt
 \addcontentsline{toc}{section}{References}
 \begin{thebibliography}{99}

 \vspace*{-2pt}

\bibitem{1-va-1}
\Aue{Newton, R.\,M., and W.\,H. Thomas}. 1969. Design of school Bus Routes by
computer. \textit{Socio-Economic Planning Science} 3:75--85.
\bibitem{2-va-1}
\Aue{Archetti, C.} Matheuristics for routing problems. Available at: {\sf
http://www.sintef.no/contentassets/cfb19ab9\linebreak b7c74d03904c7746ee1d8e77/matheuristics\_routing\_\linebreak verolog2014\_new.pdf} (accessed March~10, 2015).
\bibitem{3-va-1}
\Aue{Bronshtein, E.\,M., D.\,M. Vagapova, and A.\,V.~Nazmutdinova}. 2014.
O~postroenii semeystva marshrutov dostavki shkol'nikov za minimal'noe vremya
[About creating a set of delivery routes of students in a minimal time].
\textit{Avtomatika i~Telemekhanika} [Automation and Remote Control] 7:43--51.
\bibitem{4-va-1}
\Aue{Fisher, M.\,L., and R.\,A. Jaikumar}. 1981. Generalized assignment heuristic
for vehicle routing. \textit{Networks} 11(2):109--124.
\bibitem{5-va-1}
\Aue{Bowerman, R., B.~Hall, and P.\,A.~Calamai}. 1995. Multi-objective
optimization approach to urban school Bus Routing: Formulation and solution
method. \textit{Transportation Research Part~A: Policy and Practice}
29(2):107--123.
\bibitem{6-va-1}
\Aue{Arias-Rojas, J.\,S., J.\,F. Jimenez, and J.\,R.~Montoya-Torres}. 2012. Solving
of school bus routing problem by ant colony. \textit{Revista EIA} 9(17):193--208.
\bibitem{7-va-1}
\Aue{Dorigo, M.} 1992. \textit{Optimization, learning and natural algorithms}. PhD
Thesis. Milan, Italy: Politecnico di Milano.
\bibitem{11-va-1} %8
\Aue{Gambardella, L.\,M., and M. Dorigo}. 1995. Ant-Q: A~reinforcement learning
approach to the traveling salesman problem. \textit{12th Conference (International)
on Machine Learning}. Tahoe City. 252--260.
\bibitem{8-va-1} %9
\Aue{Dorigo, M., V.~Maniezzo, and A.~Colorni}. 1996. The Ant System:
Optimization by a colony of cooperating agents. \textit{IEEE Trans. Syst. Man
Cybernetics B} 26(1):29--41.
\bibitem{9-va-1} %10
\Aue{Dorigo, M., and L.\,M.~Gambardella}. 1997. Ant Colony System:
A~cooperative learning approach to the traveling salesman problem. \textit{IEEE
Trans. Evol. Comput.} 1(1):53--66.
\bibitem{10-va-1} %11
\Aue{St$\ddot{\mbox{u}}$tzle, T., and H.~Hoos}. 1997. MAX-MIN Ant System
and local search for the traveling salesman problem. \textit{IEEE Conference
(International) on Evolutionary Computation}. Indianapolis. 309--314.

\bibitem{12-va-1}
\Aue{Shtovba, S.\,D.} 2003. Murav'inye algoritmy [Ant colony algorithms].
\textit{Exponenta Pro. Matematika v~prilozheniyakh} [Exponenta-Pro. Mathematics
in Applications] (4):70--75.
\bibitem{13-va-1}
\Aue{Kureychik, V.\,M., and A.\,A.~Kazharov}. 2008. O~nekotorykh
modifikatsiyakh murav'inogo algoritma [About some modifications of the ant colony
algorithm]. \textit{Izvestiya Yuzhnogo Federal'nogo Universiteta. Tekhnicheskie
Nauki} [News of  South Federal University. Technical Sciences] 81(4):7--12.
\bibitem{14-va-1}
\Aue{Slastnikov, S.\,A.} 2014. Primenenie metaevristicheskikh algoritmov dlya
zadachi marshrutizatsii transporta [Application of metaheuristic algorithms for
vehicle routing problem]. \textit{Ekonomika i~Matematicheskie Metody} [Economics
and Mathematical Methods] 50(1):117--126.
\end{thebibliography}

 }
 }

\end{multicols}

\vspace*{-3pt}

\hfill{\small\textit{Received February 19, 2015}}

%\vspace*{-18pt}

\Contr

\noindent
\textbf{Bronshtein Efim M.} (b.\ 1946)~---
Doctor of Science in physics and mathematics; professor, Ufa State Aviation Technical University, 12 K.~Marx Str., Ufa 450000,
Russian Federation; bro-efim@yandex.ru

\vspace*{3pt}

\noindent
\textbf{Vagapova Diana M.} (b.\ 1987)~---
applicant, Ufa State Aviation Technical University, 12 K.~Marx Street, Ufa 450000, Russian Federation; vagapova-dm@mail.ru

\label{end\stat}


\renewcommand{\bibname}{\protect\rm Литература}
 %7
\def\stat{krivenko}

\def\tit{МНОГОМЕРНЫЙ РЕФЕРЕНСНЫЙ РЕГИОН\\ ВЫСОКОЙ ПЛОТНОСТИ}

\def\titkol{Многомерный референсный регион высокой плотности}

\def\aut{М.\,П.~Кривенко$^1$}

\def\autkol{М.\,П.~Кривенко}

\titel{\tit}{\aut}{\autkol}{\titkol}

\index{Кривенко М.\,П.}
\index{Krivenko M.\,P.}


%{\renewcommand{\thefootnote}{\fnsymbol{footnote}} \footnotetext[1]
%{Работа выполнена при финансовой поддержке РФФИ (проекты 16-07-00677 
%и~15-37-20611-мол\_а\_вед).}}


\renewcommand{\thefootnote}{\arabic{footnote}}
\footnotetext[1]{Институт проблем информатики Федерального исследовательского центра <<Информатика и~управление>> Российской академии наук,
\mbox{mkrivenko@ipiran.ru}}

\vspace*{4pt}



\Abst{Рассматриваются принципы построения многомерных референсных регионов
(MRR~--- multivariate reference region). 
Предложен оригинальный метод построения региона на основе областей с~высокой 
плотностью точек и~аппроксимации распределения данных с~помощью смеси нормальных 
распределений. Для оценки порога для плотности распределения используется  
бут\-стреп-ме\-тод. В~качестве эксперимента рассмотрена задача построения 
и~использования эталонной области для прогнозирования типа мочевого камня. Обработка 
реальных данных продемонстрировала преимущества предлагаемых решений.}

\KW{многомерный референсный регион; область высокой плотности; бут\-стреп-ме\-тод; 
смесь многомерных нормальных распределений}

\vspace*{6pt}

\DOI{10.14357/19922264170207} 


\vskip 10pt plus 9pt minus 6pt

\thispagestyle{headings}

\begin{multicols}{2}

\label{st\stat}

\section{Введение}

     Многомерный референсный регион 
был предложен в~литературе по клинической химии в~начале 1970-х~гг.\ как 
альтернатива одномерным референсным интервалам~[1]. Там излагались 
преимущества предлагаемых множественных тестов, хоть и~имеющих 
упрощенный вид, но снижающих (по отношению к~одномерным вариантам) 
число ложных положительных результатов. Появление MRR оказалось 
особенно привлекательным для интерпретации результатов наборов 
медицинских тестов. Тем не менее возникали трудности в~построении 
и~использовании процедур многомерного анализа (см., например,~[2]), 
связанные, в~частности, с~быстрым увеличением числа параметров, которые 
должны быть оценены. Немногие лаборатории использовали MRR в~своей 
практике, причем в~экспериментальном режиме, и,~как следствие, на 
сегодняшний день имеется относительно малое количество соответствующих 
публикаций. 

\vspace*{-6pt}

\section{Многомерный референсный регион на основе расстояния Махалонобиса}

\vspace*{-2pt}

     Одномерный референсный интервал, полученный статистическим путем, 
использует центральную часть значений анализируемого показателя, обычно 
соответствующую~95\% некоторой популяции~--- совокупности особей 
определенного вида (например, здоровой части населения определенного пола 
из некоторого диапазона возрастов). Одномерные референсные интервалы 
применялись в~течение многих лет в~качестве стандартного приема 
интерпретации лабораторных данных. Они легко формируются, хранятся, 
извлекаются и~передаются в~лабораторных информационных системах, просты 
в~понимании, хорошо воспринимаются медицинским сообществом в~ходе 
длительного использования. Тем не менее одномерные референсные интервалы 
при классификации данных могут дать большое число ложно аномальных 
результатов. Этот далеко не единственный недостаток однофакторного 
референсного интервала может быть полностью или частично устранен 
с~помощью MRR.
     
     Простейшим и~весьма распространенным способом построения MRR 
является использование прямого произведения отдельных референсных 
интервалов в~предположении, что они статистически независимы. Пусть 
$(1\hm-\alpha)$~--- вероятность попадания в~MRR, а~$p_0$~--- вероятность 
попадания в~референсный интервал для любого из~$d$~признаков, тогда 
$p_0\hm= \sqrt[d]{1-\alpha}$. С~ростом размерности~$d$ значения~$p_0$ 
быстро приближаются к~1, что фактически лишает смысла применение MRR.
     
     Как и~в одномерном случае, отправной точкой для построения MRR 
может стать нормальное распределение. Идеи центрального расположения 
референсного региона и~заданной вероятности попадания в~него приводят для 
$d$-мер\-но\-го нормального распределения, имеющего плотность 
распределения
     \begin{multline*}
     \varphi(y,\mu,\Sigma) ={}\\
     {}=(2\pi)^{-d/2}\vert\Sigma\vert^{-1/2}\exp \left( -\fr{\left(y-
\mu\right)^{\mathrm{T}} \Sigma^{-1}(y-\mu)}{2}\right),
   \end{multline*}
где величина $(y-\mu)^{\mathrm{T}} \Sigma^{-1} (y-\mu)$ есть квадрат так 
называемого расстояния Махаланобиса между~$y$ и~$\mu$, к~использованию 
многомерного эллипсоида
\begin{multline*}
(2\pi)^{-d/2}\vert\Sigma\vert^{-1/2}\exp \left( -\fr{\left(y-\mu\right)^{\mathrm{T}}
\Sigma^{-1} 
(y-\mu)}{2}\right) ={}\\
{}=const
\end{multline*}
или, что то же самое, 
$$ 
(y-\mu)^{\mathrm{T}} \Sigma^{-1}(y-\mu)=const\,.
$$
Его называют эллипсоидом равной плотности распределения (или просто 
эллипсоидом равной вероятности). 
     
     Если задаться вероятностью $(1\hm-\alpha)$ попадания в~эллипсоид 
равной вероятности вида $(y\hm-\mu)^{\mathrm{T}}\Sigma^{-1} (y\hm-\mu)\hm= 
\rho$, то параметр~$\rho$ удовлетворяет уравнению $\mathrm{Pr}\left\{ 
\chi_d^2\leq \rho\right\} \hm=1\hm-\alpha$.
     
     Использование эллипсоида в~качестве MRR будет оправдано только 
тогда, когда исходное распределение данных есть многомерное нормаль-\linebreak ное. 
Поэтому становятся актуальными критерии\linebreak подгонки, а~также использование 
процедур норма\-ли\-зации распределения данных в~многомерном\linebreak случае.
 Если 
с~помощью тестов выявляется, что распределение не является нормальным, то 
Международная федерация клинической химии и~лабораторной медицины 
рекомендует, согласно~[3], использовать двухступенчатую процедуру 
нормализации. Следует обратить внимание, что многошаговость здесь 
относится не к~многомерности, а касается лишь покоординатного 
преобразования распределения данных к~нормальному.
     
     Первые же попытки применения MRR на основе расстояния 
Махалонобиса (фактически это означает принятие модели нормального 
распределения референсных значений) выявили ряд недостатков (более 
подробно смотри в~\cite[разд.~6.2]{4-kri}):
     \begin{itemize}
\item проявление <<проклятий>> размерности при механическом 
увеличении~$d$, в~особенности если игнорируется этап анализа состава 
признаков~[1, 5, 6];
\item из-за небольших объемов обучающей выборки невысокая устойчивость 
при применении, в~частности чувствительность к~увеличению неточностей 
измерений после того, как регион был установлен~\cite{5-kri, 7-kri}. 
\item предположение о нормальном распределении и~попытки <<подправить>> 
действительность с~помощью преобразований реальных данных для их 
нормализации при увеличении размерности данных становятся все более 
шаткими~\cite{5-kri};
\item представление и~интерпретация выводов на основе MRR трудно 
понимаемы не только для специалистов в~предметной области~[8].
\end{itemize}

\vspace*{-9pt}

\section{Многомерный референсный регион высокой плотности}

\vspace*{-2pt}

     Заметим, что в~случае нормального распределения референсных значений 
для точек внут\-ри построенного эллипсоида значения плотности\linebreak распределения 
больше, чем на границе, а~вне~--- меньше. Это замечание позволяет 
предложить другой подход к~построению MRR.
     
     \smallskip
     
     \noindent
     \textbf{Определение.}\ Eсли плотность распределения референсных 
значений есть $f(y)$, то MRR есть область $A_t\hm= \left\{ y\in 
\mathcal{R}^d\vert f(y)\hm\geq t\right\}$ для некоторого порогового 
значения~$t$. 
     
     \smallskip
     
     Для нормального распределения это уже упомянутый эллипсоид равной 
вероятности. Если задается вероятность $(1\hm-\alpha)$ попадания в~$A_t$, то 
пороговое значение~$t$ есть решение уравнения $\int\nolimits_{A_t} 
f(u)\,du\hm=1\hm-\alpha$, получить которое аналитически в~случае 
произвольной плотности распределения вряд ли возможно. Здесь присутствуют 
две проблемы: вычисление многомерного интеграла и~зависимость области 
интегрирования от неизвестного значения. Для решения их предлагается 
привлечь метод моделирования.
     
     Сгенерируем выборку из $f(y)$, которую обозначим как $Y^f\hm= \left\{ 
y_1^f, \ldots, y_m^f\right\}$. Для оценки $\int\nolimits_{A_t} f(u)\,du$ 
используем отношение:

\noindent
\begin{multline*}
     \fr{\left\vert \left\{ y_i^f\vert y_i^f\in A_t\right\}\right\vert }{m} =
      \fr{\left\vert\left\{ y_i^f\vert 
f\left(y_i^f\right) \geq t\right\}\right\vert }{m} ={}\\
{}= 1-\fr{\left\vert \left\{ y_i^f\vert f(y_i^f)<t\right\}\right\vert }{m}=1-
F_m(t)\,,
     \end{multline*}
где $F_m(t)$~--- эмпирическая функция распределения случайной 
величины~$f(y)$, т.\,е.\ случайной величины, являющейся результатом 
преобразования с~помощью функции~$f(\cdot)$ случайной величины, име\-ющей 
плотность распределения~$f(u)$.

     Таким образом, искомая оценка~$t^*$ должна удовле\-тво\-рять уравнению 
$F_m(t^*)\hm=\alpha$ и~может быть получена как непараметрическая оценка 
квантиля\linebreak\vspace*{-12pt}

\pagebreak

\noindent
 порядка~$\alpha$ из распределения $F_m(\cdot)$. Если обозначить 
$f_i\hm= f(y_i^f)$, то~$t^*$ есть~$f_{(r)}$, где
     $$
     r= \begin{cases}
     m\alpha, &\ m\alpha~\mbox{---~целое}\,;\\
     \lfloor m\alpha+1\rfloor\,, & m\alpha~\mbox{--- не целое}\,.
     \end{cases}
     $$
     Заметим, что для такой оценки можно указать доверительный интервал.
     
     Для построения MRR необходимо знать распределение данных. При 
реализации принципа точек высокой плотности в~первую очередь следует 
обратиться к~параметрическим моделям, в~част\-ности к~смеси нормальных 
распределений, име\-ющей плотность распределения
     $$
     f(u) =\sum\limits_{j=1}^k p_j \varphi\left (u,\mu_j, \Sigma_j\right)\,.
     $$
Если $\hat{f}(u)$~--- оценка смеси, то~$t^*$ строится сле\-ду\-ющим образом:
\begin{itemize}
\item генерируется выборка $\left\{ y_1^f,\ldots , y_m^f\right\}$ из $\hat{f}(u)$ и~
для каждого ее $i$-го элемента подсчитывается значение $\hat{f}\left( 
y_i^f\right)$;
\item в~качестве~$t^*$ берется непараметрическая оценка квантиля 
порядка~$\alpha$ (в случае необходимости дополнительно находится 
непараметрическая оценка доверительного интервала для~$t^*$, что 
может характеризовать правильность выбранного объема для 
генерируемой выборки).
\end{itemize}

     Пусть для $f(u)$ имеется~$A_t$, а также получена $\hat{f}(u)$ 
и~соответствующий MRR вида~$\hat{A}_t$. Качество аппроксимации~$A_t$ 
с~по\-мощью~$\hat{A}_t$ можно оценить с~по\-мощью вероятности совпадения 
этих областей, т.\,е. 
     $$
     P_c= \int\limits_{\{ u\in A_t\}\cup \{u\in \hat{A}_t\}} \hspace*{-6mm}
f(u)\,du+\int\limits_{\{u\not\in A_t\} \cup\{ u\not\in \hat{A}_t\}}\hspace*{-6mm} f(u)\,du\,.
     $$
     
     Для оценки  $P_c$ можно использовать величину
     \begin{multline*}
     \hat{P}_c= \fr{\left\vert \left\{ 
     y_i^f\vert y_i^f \in \left\{\left\{ y_i^f\in A_t\right\}\cup \left\{y_i^f\in 
\hat{A}_t\right\}\right\}\right\}\right\vert}{m}+{}\\
{}+\fr{\left\vert \left\{ y_i^f\vert y_i^f \in \left\{\left\{ y_i^f\not\in A_t\right\}\cup 
\left\{ y_i^f\not\in \hat{A}_t\right\}\right\}\right\}\right\vert}{m}\,.
     \end{multline*}
     
     Использование MRR высокой плотности для диагностирования сводится 
к~реализации так называемого слабого критерия значимости для наблюденного 
значения~$x$: нулевая гипотеза заключается в~том, что $x\hm\in A_t$, 
статистика критерия есть $\hat{f}(x)$ и~решение о~принадлежности 
критической об\-ласти~$A_t$ принимается при больших значениях~$\hat{f}(x)$.
     
     Для медицинской практики важна возможность использования 
референсного региона при интерпретации результатов обследования 
некоторого пациента с~вектором признаков~$x$. В~подобных случаях 
сложившейся практикой для слабых критериев значимости является 
использование критического уровня~$\alpha_{\mathrm{cr}}$ (более распространенным 
в~медицине является употребление термина $p$-зна\-че\-ние)  $\alpha_{\mathrm{cr}}\hm= 
\mathrm{Pr}\left\{ \hat{f}(y)\hm\leq \hat{f}(x)\right\}$, где $y$~--- случайная 
величина, имеющая плотность распределения~$\hat{f}(u)$, а $\hat{f}(x)$~--- 
значение плотности распределения~$\hat{f}(u)$ в~точке~$x$. Эта 
характеристика дает представление о~том, насколько сильно данное 
наблюденное значение~$x$ противоречит гипотезе (или подкрепляет ее) 
о~принадлежности данных MRR. При выбранном же заранее уровне 
значимости с~помощью~$\alpha_{\mathrm{cr}}$ сразу же можно принять конкретное 
решение. 

\vspace*{-9pt}

\section{Эксперименты}

\vspace*{-2pt}

     Для демонстрации возможностей MRR использовались данные по 
прогнозу химического состава мочевых камней по метаболическим 
показателям мочи и~сыворотки крови, а также антропологическим 
характеристикам пациентов~[9]. В качестве исходной классификации камней 
рассматривалась следующая: чисто оксалатные (далее обозначены как O), чисто 
уратные (U), чисто фосфатные (P), смесь только оксалатных и~уратных (OU), 
смесь только оксалатных и~фосфатных (OP), смесь только уратных 
и~фосфатных (UP), все остальные. Данная классификация была построена 
в~[10] на основе доминирующих частот встречаемости основных компонентов. 
В~качестве референсных значений рассматривались наборы метаболических 
и~антропологических показателей (их всего было~14), соответствующих 
определенному классу камней.

\begin{table*}\small
\begin{center}


\begin{tabular}{|c|c|c|c|c|c|c|}
\multicolumn{7}{c}{Качество классификации с~помощью MRR}\\
\multicolumn{7}{c}{\ }\\[-6pt]
\hline
\multicolumn{1}{|c|}{\raisebox{-6pt}[0pt][0pt]{\tabcolsep=0pt\begin{tabular}{c}Тип\\ камня\end{tabular}}}&
\multicolumn{1}{c|}{\raisebox{-6pt}[0pt][0pt]{$N$}}&$(1-\alpha)$, 
&\multicolumn{2}{c|}{MRR(5)}&\multicolumn{2}{c|}{MRR(1)}\\
\cline{4-7}
&&&&&&\\[-9pt]
&&\%&$(1-\hat{\alpha})$, \%&$\hat{\beta}$, \%&$(1-\hat{\alpha})$, \%&$\hat{\beta}$, \%\\
\hline
\multicolumn{1}{|c|}{\raisebox{-18pt}[0pt][0pt]{O}}&
\multicolumn{1}{c|}{\raisebox{-18pt}[0pt][0pt]{82}}
&95&100\hphantom{9}&71&90&24\\
&&85&96&78&89&36\\
&&75&91&85&77&44\\
&&65&76&88&74&50\\
\hline
\multicolumn{1}{|c|}{\raisebox{-18pt}[0pt][0pt]{U}}&
\multicolumn{1}{c|}{\raisebox{-18pt}[0pt][0pt]{76}}&95&100\hphantom{9}&75&91&24\\
&&85&99&85&80&35\\
&&75&82&89&74&48\\
&&65&71&91&68&56\\
\hline
\multicolumn{1}{|c|}{\raisebox{-18pt}[0pt][0pt]{P}}&
\multicolumn{1}{c|}{\raisebox{-18pt}[0pt][0pt]{83}}&95&100\hphantom{9}&66&87&25\\
&&85&94&78&86&33\\
&&75&86&82&82&41\\
&&65&77&87&75&47\\
\hline
\end{tabular}
\end{center}
\end{table*}
     
     
     Для каждого из основных классов O, U, P, OU, OP и~UP перед построением 
MRR проводилась селекция признаков и~принималось то значение размерности 
признакового пространства~$d$ и~соответствующий набор показателей, 
которые позволяли прогнозировать состав камней без потери качества 
(методика описана в~\cite{9-kri} и~привела к~значению $d\hm=9$). В~качестве 
модели данных в~первую очередь рассматривалась смесь многомерных 
нормальных распределений из пяти элементов (подбор числа элементов смеси 
проводился с~по\-мощью AIC~--- Akaike information criterion), для соответствующего региона было принято 
обозначение MRR(5). Для сравнения также использовалась модель 
нормального распределения, которой соответствовал MRR(1). Полученные 
результаты приводятся час\-тич\-но в~таблице, где $N$~--- объем 
классифицируемых данных; $\hat{\alpha}$~--- оценка для~$\alpha$; 
$\hat{\beta}$~--- оценка мощности критерия при определении типа камня на 
основании MRR.


     Одной из базовых характеристик является вероятность попадания в~MRR 
$(1\hm-\alpha)$ и~ее оценка $(1\hm-\hat{\alpha})$. Сравнение соответствующих 
столбцов с~учетом значений~$N$ и~ориентировочных значений разброса 
(стандартные отклонения на основе биномиального распределения) не 
позволило выявить явных отклонений. Необходимо, правда, отметить, что во 
всех проанализированных случаях для MRR(5) оказалось, что $1\hm-
\hat{\alpha}\hm\geq 1\hm-\alpha$.
     
     Назначение MRR, заключающееся в~сжатом представлении референсных 
значений, в~многомерном случае практически не проявляется. Для задания 
MRR(5) необходимо указать следующие величины: $1\hm-\alpha$, $t$, 
$p_1,\ldots, p_{k-1}$, $\mu_1, \Sigma_1,\ldots , \mu_k,\Sigma_k$, общее 
количество которых равно  $[2\hm+ (k\hm-1)\hm+ k(d\hm+ d(d\hm+1)/2)]$ 
и,~в~частности, в~рассматриваемых экспериментах~--- 276. Для MRR(1) это 
значение меньше и~равно~56. При этом для обрабатываемой обучающей 
выборки в~зависимости от класса камней речь идет о~порядка~10$^2$ векторах 
данных (см.\ столбец со значениями~$N$), что приблизительно 
дает~10$^3$~скалярных величин.
     
     Другое назначение MRR состоит в~его использовании для 
диагностирования (классификации). В~этой связи в~первую очередь 
проводился сравнительный анализ MRR(1) (фактически это означает, что 
построение региона осуществляется на основе расстояния Махаланобиса) 
и~MRR(5) (модель смеси нормальных распределений и~предложенный 
в~данной работе метод оценивания па\-ра\-мет\-ров региона). Показателем 
информативности метода построения многомерного региона выступала 
мощность соответствующего слабого критерия значимости, а~именно: 
вероятность не попасть в~MRR при условии, что данные берутся из дополнения 
к~классу, для которого построена MRR. Сравнение соответствующих столбцов 
говорит о~явном преимуществе двух предложенных моментов: усложнение 
модели данных путем перехода от нормального распределения к~смеси 
нормальных распределений и~построение региона высокой плотности.
     
     Использование критического уровня можно продемонстрировать  
с~по\-мощью зависимости результатов сравнения двух классов от того, какой 
класс взять за основу. Введем для возможных значений $p$-ве\-ли\-чи\-ны три 
интервала: $(-\infty, 1\%)$, $[1\%, 5\%)$, $[5\%, 100\%)$ с~соответствующей 
интерпретацией положения наблюденного набора показателей для пациента 
относительно построенного MRR: уверенное непопадание, неуверенное 
попадание, уверенное попадание. Если MRR построить для оксалатных камней, 
то результаты для анализа пациентов с~фосфатными камнями дадут следующий 
вектор относительных частот попадания $p$-ве\-ли\-чин в~указанные 
интервалы: $(60\%, 18\%, 22\%)$. Если же MRR строить для фосфатных 
камней, то получим $(71\%, 5\%, 24\%)$. Таким образом, для классификации 
указанных камней при приблизительно одинаковых частотах попадания в~MRR 
(22\% или~24\%) уверенный отказ от референсного региона происходит чаще, 
если принять за базовый MRR регион для фосфатных камней. Построение 
шкалы, подобной рассмотренной, является прерогативой специалистов 
в~предметной области, в~данной работе она использовалась только для 
иллюстрации. 

\vspace*{-6pt}

\section{Заключение}

\vspace*{-2pt}

     На настоящий момент имеется относительно мало примеров применения 
MRR в~клинической практике. Тому есть несколько причин. Математическое 
обеспечение, необходимое для получения и~применения MRR, не отвечает 
возможностям большинства клинических лабораторий. Лаборатории слабо 
оснащены программными средствами\linebreak для реализации достаточно сложного 
математического аппарата многомерного анализа, а~еще важнее, что 
отсутствуют методики, инструкции по\linebreak использованию соответствующих 
средств. Лишь немногие клинические применения демонстрируют 
преимущества MRR, хотя свидетельств неудачных попыток больше.
     
     Несмотря на сложности внедрения мно\-го\-мерно\-го анализа референсных 
значений, можно сформулировать некоторые рекомендации по иссле\-до\-ва\-нию 
и~разработке MRR. Во-пер\-вых, эффективная размерность в~MRR должна 
быть как можно меньше, чтобы избежать затенения диагностически полезной 
информации тестами, со\-зда\-ющи\-ми шум. Низкая размерность также должна 
уменьшить неблагоприятные последствия увеличения неточности результатов 
в~связи с~ростом числа анализируемых показателей. Во-вто\-рых, показатели 
(тес\-ты), включенные в~MRR, должны быть физиологически релевантными 
исследуемому кругу расстройств, чтобы максимизировать информацию, 
полученную от MRR. В-треть\-их, чтобы учесть эффекты долгосрочной 
лабораторной из\-мен\-чи\-вости, данные, используемые для получения MRR, 
долж\-ны быть собраны и~проанализированы в~течение достаточно большого 
периода времени (от нескольких недель до нескольких месяцев).  
В-чет\-вер\-тых, представление результатов лабораторных исследований 
следует осуществлять в~графическом виде, чтобы помочь врачам лучше понять 
MRR. Различные подходы к~уменьшению размерности помогут выполнить это 
требование.
     
     Необходима дальнейшая разработка пояснительных инструментов, 
способных воспринять результаты анализа MRR. При этом дополнительно 
необходима информация о~том, какие именно тес\-ты являются важнейшими 
факторами нарушения нормы. Надо признать, что соответствующий 
математический аппарат еще предстоит разработать. Решение перечисленных 
вопросов играет важную роль для обеспечения постоянного клинического 
применения MRR. 

\vspace*{-6pt}
     
{\small\frenchspacing
 {%\baselineskip=10.8pt
 \addcontentsline{toc}{section}{References}
 \begin{thebibliography}{99}
 
 \vspace*{-2pt}
 
\bibitem{1-kri}
\Au{Boyd J.\,C.} Reference regions of two or more dimensions~// Clin. Chem. Lab. 
Med., 2004. Vol.~42. No.\,7. P.~739--746.
\bibitem{2-kri}
\Au{Winkel P.} Patterns and clusters~--- multivariate approach for interpreting 
clinical chemistry results~// Clin. Chem., 1973. Vol.~19. No.\,12. P.~1329--1333.
\bibitem{3-kri}
IFCC. Expert panel on theory of reference values. Approved recommendation on the 
theory of reference values. Part~5. Statistical treatment of collected reference values. 
Determination of reference limits~// J.~Clin. Chem. Clin. Biochem., 1987. Vol.~25. 
No.\,9. P.~645--656.
\bibitem{4-kri}
\Au{Кривенко М.\,П.} Статистические методы представления и~предварительной 
обработки референсных значений.~--- М.: ФИЦ ИУ РАН, 2016. 160~с.
\bibitem{5-kri}
\Au{Boyd J.\,C., Lacher~D.\,A.} The multivariate reference range: An alternative 
interpretation of multi-test profiles~// Clin. Chem., 1982. Vol.~28. No.\,2.  
P.~259--265.
\bibitem{6-kri}
\Au{Albert A., Harris~E.\,K.} Multivariate interpretation of clinical laboratory  
data.~--- New York, NY, USA: CRC Press, 1987. 328~p.
\bibitem{7-kri}
\Au{Linnet K.} Influence of sampling variation and analytical errors on the 
performance of the multivariate reference region~// Meth. Inf. Med., 1988. Vol.~27. 
No.\,1. P.~37--42.
\bibitem{8-kri}
\Au{Durbridge T.\,C.} Clinical acceptance of a multi-test reference region for 
biochemical-panel results~// Clin. Chem., 1983. Vol.~29. No.\,10. P.~1724--1726.
\bibitem{9-kri}
\Au{Кривенко М.\,П.} Критерии значимости отбора признаков классификации~// 
Информатика и~её применения, 2016. Т.~10. Вып.~3. С.~32--40.
\bibitem{10-kri}
\Au{Кривенко М.\,П., Голованов~С.\,А., Сивков~А.\,В.} Анализ однородности 
данных о химическом составе камней при уролитиазе~// Информатика и~её 
применения, 2013. Т.~7. Вып.~4. С.~94--104.
 \end{thebibliography}

 }
 }

\end{multicols}

\vspace*{-10pt}

\hfill{\small\textit{Поступила в~редакцию 5.12.16}}

\vspace*{4pt}

%\newpage

%\vspace*{-24pt}

\hrule

\vspace*{2pt}

\hrule

\vspace*{-3pt}


\def\tit{HIGH-DENSITY MULTIVARIATE REFERENCE REGION\\[-5pt]}

\def\titkol{High-density multivariate reference region}

\def\aut{M.\,P.~Krivenko\\[-7pt]}

\def\autkol{M.\,P.~Krivenko}

\titel{\tit}{\aut}{\autkol}{\titkol}

\vspace*{-16pt}


\noindent
Institute of Informatics Problems, Federal Research Center 
``Computer Science and Control'' of the Russian
Academy of Sciences,  44-2~Vavilov Str., Moscow 119333, Russian Federation



\def\leftfootline{\small{\textbf{\thepage}
\hfill INFORMATIKA I EE PRIMENENIYA~--- INFORMATICS AND
APPLICATIONS\ \ \ 2017\ \ \ volume~11\ \ \ issue\ 2}
}%
 \def\rightfootline{\small{INFORMATIKA I EE PRIMENENIYA~---
INFORMATICS AND APPLICATIONS\ \ \ 2017\ \ \ volume~11\ \ \ issue\ 2
\hfill \textbf{\thepage}}}

\vspace*{2pt}




\Abste{The paper considers the principles of construction of multivariate 
reference regions. An original method of construction of 
a~region on the basis of areas of high density of points and approximation 
of data distribution with a~mixture of normal distributions is suggested. 
To estimate the threshold for the probability density, the bootstrap method is used. 
As an experiment, the paper considers the problem of description and use of 
the reference region for predicting the type of urinary stones. 
Real data treatment demonstrated the benefits of the proposed solutions.}

\KWE{multivariate reference region; high-density region; bootstrap method; 
multivariate normal mixture}

\DOI{10.14357/19922264170207} 

%\vspace*{-18pt}

%\Ack
%\noindent



%\vspace*{3pt}

  \begin{multicols}{2}

\renewcommand{\bibname}{\protect\rmfamily References}
%\renewcommand{\bibname}{\large\protect\rm References}

{\small\frenchspacing
 {%\baselineskip=10.8pt
 \addcontentsline{toc}{section}{References}
 \begin{thebibliography}{99}
\bibitem{1-kri-1}
\Aue{Boyd, J.\,C.} 2004. Reference regions of two or more dimensions. \textit{Clin. 
Chem. Lab. Med.} 42(7):739--746.

\bibitem{2-kri-1}
\Aue{Winkel, P.} 1973. Patterns and clusters~--- multivariate approach for interpreting 
clinical chemistry results. \textit{Clin. Chem.} 19(12):1329--1333.
\bibitem{3-kri-1}
IFCC. 1987. Expert panel on theory of reference values. Approved recommendation on the 
theory of reference values. Part~5. Statistical treatment of collected reference values. 
Determination of reference limits. \textit{J.~Clin. Chem. Clin. Biochem.} 
25(9):645--656.
\bibitem{4-kri-1}
\Aue{Krivenko, M.\,P.} 2016. \textit{Statisticheskie metody predstavleniya 
i~predvaritel'noy obrabotki referensnykh znacheniy}
[Statistical methods for representation and preliminary processing of
reference values]. Moscow: FRC CSC RAS. 160~p.

\bibitem{5-kri-1}
\Aue{Boyd, J.\,C., and D.\,A.~Lacher.} 1982. The multivariate reference range: An 
alternative interpretation of multi-test profiles. \textit{Clin. Chem.}  
28(2):259--265.
\bibitem{6-kri-1}
\Aue{Albert, A., and E.\,K.~Harris.} 1987. \textit{Multivariate interpretation of 
clinical laboratory data}. New York, NY: CRC Press. 328~p.
\bibitem{7-kri-1}
\Aue{Linnet, K.} 1988. Influence of sampling variation and analytical errors on the 
performance of the multivariate reference region. \textit{Meth. Inf. Med.}  
27(1):37--42.
\bibitem{8-kri-1}
\Aue{Durbridge, T.\,C.} 1983. Clinical acceptance of a multi-test reference region 
for biochemical-panel results. \textit{Clin. Chem.} 29(10):1724--1726.
\bibitem{9-kri-1}
\Aue{Krivenko, M.\,P.} 2016. Kriterii znachimosti otbora priznakov klassifikatsii
[Significance tests of feature selection for~classification]. \textit{Informatika i~ee 
Primeneniya~--- Inform. Appl.} 10(3):32--40.
\bibitem{10-kri-1}
\Aue{Krivenko, M.\,P., S.\,A.~Golovanov, and A.\,V.~Sivkov}. 2013. Analiz 
odnorodnosti dannykh o~khimicheskom sostave kamney pri urolitiaze
[Analysis of data homogeneity of~the~chemical compositions 
of~stones in~case of~urolithiasis]. \textit{Informatika i~ee Primeneniya~---
Inform Appl.} 7(4):94--104.
\end{thebibliography}

 }
 }

\end{multicols}

\vspace*{-3pt}

\hfill{\small\textit{Received December 5, 2016}}


\Contrl

\noindent
\textbf{Krivenko Michail P.} (b.\ 1946)~--- Doctor of Science in technology, 
professor, leading scientist, Institute of Informatics Problems, Federal Research 
Center ``Computer Science and Control'' of the Russian Academy of Sciences, 
\mbox{44-2}~Vavilov Str., Moscow 119333, Russian Federation; \mbox{mkrivenko@ipiran.ru}

\label{end\stat}


\renewcommand{\bibname}{\protect\rm Литература}  %8
\newcommand{\hx}{\hat{x}}
\newcommand{\hy}{\hat{y}}
\newcommand{\uu}{\mathbf{u}}

\newcommand{\x}{\mathbf{x}}
\newcommand{\hchi}{\hat{\boldsymbol{\chi}}}
\newcommand{\hphi}{\hat{\boldsymbol{\varphi}}}

\newcommand{\weight}{\mathbf{a}}
\newcommand{\pcd}{p(\cdot)}
\newcommand{\q}{q(\cdot)}

\def\stat{stenina}

\def\tit{СОГЛАСОВАНИЕ ПРОГНОЗОВ ПРИ РЕШЕНИИ ЗАДАЧ
ПРОГНОЗИРОВАНИЯ ИЕРАРХИЧЕСКИХ ВРЕМЕННЫХ РЯДОВ$^*$}

\def\titkol{Согласование прогнозов при решении задач
прогнозирования иерархических временных рядов}

\def\aut{М.\,М. Стенина$^1$,  В.\,В.~Стрижов$^2$}

\def\autkol{М.\,М. Стенина,  В.\,В.~Стрижов}

\titel{\tit}{\aut}{\autkol}{\titkol}

\index{Стенина М.\,М.}
\index{Стрижов В.\,В.}

{\renewcommand{\thefootnote}{\fnsymbol{footnote}} \footnotetext[1]
{Работа выполнена при финансовой поддержке РФФИ (проект~13-07-13139).}}


\renewcommand{\thefootnote}{\arabic{footnote}}
\footnotetext[1]{Московский физико-технический институт, Национальный исследовательский университет <<Высшая школа экономики>>, mmedvednikova@gmail.com}
\footnotetext[2]{Вычислительный центр Российской академии наук им.\ А.\,А.~Дородницына, strijov@ccas.com}


\Abst{Рассматривается задача одновременного прогнозирования
набора временн$\acute{\mbox{ы}}$х рядов, объединенных в~иерархическую
многоуровневую структуру. Требуется, чтобы полученные прогнозы
удовлетворяли физическим ограничениям и~структуре иерархии.
Предложен алгоритм согласования прогнозов иерархических
временн$\acute{\mbox{ы}}$х рядов GTOp (Game-theoretically optimal reconciliation),
гарантирующий неухудшение качества прогнозов после проведения
процедуры согласования по сравнению с~качеством прогнозов,
полученных для каждого временн$\acute{\mbox{о}}$го ряда независимо. Подход
базируется на поиске равновесия Нэша в~антагонистической игре
заданного вида и~сводит задачу согласования прогнозов к~задаче
оптимизации с~ограничениями типа равенства и~неравенства.
Доказывается, что при выполнении ряда общих предположений
о~свойствах структуры иерархии, физических ограничений и~функции
потерь в~игре существует равновесие Нэша в~чистых стратегиях.
Работа алгоритма демонстрируется на разных типах иерархических
структур с~использованием данных посуточной загруженности
железнодорожных узлов.}


\KW{иерархические временн$\acute{\mbox{ы}}$е ряды;
согласование прогнозов временн$\acute{\mbox{ы}}$х рядов; антагонистическая игра;
равновесие Нэша}

\DOI{10.14357/19922264150209}

\vspace*{6pt}


\vskip 14pt plus 9pt minus 6pt

\thispagestyle{headings}

\begin{multicols}{2}

\label{st\stat}



\section{Введение}

Рассматривается задача одновременного прогнозирования набора
временн$\acute{\mbox{ы}}$х рядов, связанных в~иерархическую многоуровневую
структуру, в~которой временн$\acute{\mbox{ы}}$е ряды каждого следующего (бо-\linebreak лее
высокого) уровня формируются путем поэлементного суммирования
некоторой части (возможно, всех) временн$\acute{\mbox{ы}}$х рядов предыдущего
уровня.
%
Используемые при решении этой задачи принцип равновесия
и~регрессионные методы широко обсуждаются в~научной литературе,
в~том числе на~страницах журнала
<<Информатика и~её~применения>>~\cite{tokmakova2012hyper, vasilyev2014using}.

Задача прогнозирования иерархических вре\-мен\-н$\acute{\mbox{ы}}$х рядов
возникает в~различных прикладных областях. В~\cite{hong2014global}
описан конкурс, проведенный в~2012~г.\ на Kaggle~\cite{kaggle},
в~котором одной из задач было прогнозирование иерархических
временн$\acute{\mbox{ы}}$х рядов с~требованием согласования прогнозов по иерархии.
В~работе~\cite{hyndman2011optimal} обсуждается задача
прогнозирования туристической активности по регионам и~целям
поездок. В~статье \cite{kuznetsov2011smoothing} решается задача
прогнозирования потребительского спроса на различные группы
товаров в~ряде магазинов. В~\cite{stenina2014reconciliation}
решается задача согласования прогнозов объемов железнодорожных
перевозок различных типов грузов по ряду железнодорожных веток.

В связи с~высоким интересом к~задаче разными авторами предлагаются
различные подходы к~ее решению. Как правило, сперва получают
ка\-ким-ли\-бо образом прогнозы всех (или некоторой части) временн$\acute{\mbox{ы}}$х
рядов независимо друг от друга, а~затем корректируют (согласуют)
эти прогнозы.

Самыми простыми и~самыми распространенными способами
согласования являются нисходящий (top-down) и~восходящий
(bottom-up) подходы~[8--15]. Нисходящий подход
предполагает получение прогноза на верхнем уровне иерархии
(агрегированный временной ряд), а~затем деагрегирование этого
прогноза на сле\-ду\-ющий (более низкий) уровень иерархии на основании
долей, наблюдаемых в~истории. Восходящий подход использует
прогнозы временн$\acute{\mbox{ы}}$х рядов нижнего уровня иерархии
(неагрегированных), из которых получает прогнозы рядов из верхних
уровней путем агрегирования. Также встречаются подходы,
комбинирующие нисходящий и~восходящий.

Нет единой точки зрения на то, какой из этих подходов позволяет
получать более точные прогнозы. Наиболее ранние исследования
проведены в~работе~\cite{grunfeld1960aggregation}, где авторы
считают, что неагрегированные данные содержат много ошибок и~поэтому нисходящее прогнозирование дает более точные прогнозы.
К~таким же выводам приходят авторы работ~\cite{fogarty1991production, narasimhan1995production}. В~\cite{fliedner1999investigation} также утверждается, что
агрегированные прогнозы более точны. С~другой стороны, в~\cite{orcutt1968data, edwards1969should} обсуждается, что основные
потери информации происходят при агрегировании и~поэтому
восходящий подход предпочтительнее. В~\cite{shlifer1979aggregation} сравниваются оба подхода к~согласованию прогнозов и~утверждается, что восходящий
предпочтительнее при выполнении некоторых условий на структуру
иерархии и~горизонт прогноза. В~\cite{schwarzkopf1988top}
исследуется смещение и~устойчивость прогнозов, получаемых с~помощью обоих подходов, и~заключается, что восходящий  надежнее,
за исключением случаев с~пропусками значений и~выбросами на нижних
уровнях иерархии.

Авторы статьи~\cite{hyndman2011optimal} обобщают нисходящий и~восходящий подходы к~согласованию иерархических прогнозов
и предлагают оптимальное согласование с~использованием регрессии,
позволяющей согласовывать одновременно по всем уровням иерархии
любой сложности. Однако предложенный ими алгоритм согласования
предполагает, что полученные независимые прогнозы являются
несмещенными оценками и~что ошибки прогнозов временн$\acute{\mbox{ы}}$х рядов
удовлетворяют структуре иерархии, т.\,е.\ являются согласованными.
Оба эти требования являются достаточно строгими и~сильно
ограничивают применимость этого способа согласования.

Предложенный в~\cite{stenina2014reconciliation} способ
согласования прогнозов не требует несмещенности независимых
прогнозов и~согласованности ошибок. При этом, как
продемонстрировано в~статье, он не уступает методу из~\cite{hyndman2011optimal} по качеству согласованных прогнозов.
Однако и~этот метод имеет ряд недостатков. Для его использования
необходимо оценивать погрешность независимых прогнозов, что не
всегда удается корректно сделать. Метод не гарантирует, что
качество согласованных прогнозов не будет уступать качеству
независимых. И согласование происходит поэтапно по узлам иерархии,
что не позволяет учесть сразу всю информацию о взаимосвязи между
временн$\acute{\mbox{ы}}$ми рядами.

В настоящей статье предлагается обобщение алгоритма согласования
из~\cite{stenina2014reconciliation} с~сохранением его преимуществ
и устранением недостатков. Предлагается алгоритм согласования
прогнозов GTOp,
основанный на идеях из~\cite{vanerven:hal-00920559}. Алгоритм GTOp
не требует оценки погрешности независимых прогнозов временн$\acute{\mbox{ы}}$х
рядов, не требует не\-сме\-щен\-ности независимых прогнозов и~имеет
теоретическое обоснование улучшения качества прогнозов после
проведения согласования. Задача согласования прогнозов
рассматривается как поиск равновесия
Нэша~\cite{petrosyan1998theory, menshikov2010lections}
в~антагонистической игре игрока, выбирающего согласованные прогнозы,
с природой, которая выбирает действительные значения временн$\acute{\mbox{ы}}$х
рядов, и~сводится к~решению оптимизационной задачи с~ограничениями
типа равенства и~неравенства. Вид равновесия Нэша задает параметры
оптимизационной задачи.

Работа предлагаемого алгоритма демонстрируется на данных о
посуточном отправлении вагонов с~37~типами грузов с~98~веток
Российских желез-\linebreak ных дорог (РЖД).
Проведены эксперименты по согласова\-нию прогнозов для различных
типов иерархических структур и~показано, что на практике
действительно наблюдается улучшение качества прогнозов после
проведения согласования алгоритмом GTOp.

Статья включает следующие разделы. В разд.~2 вводятся необходимые
обозначения, в~разд.~3 формулируется задача согласования прогнозов
иерархических временн$\acute{\mbox{ы}}$х рядов. Раздел~4 содержит  необходимые
определения и~факты из теории игр, описание антагонистической
игры, со\-от\-вет\-ст\-ву\-ющей задаче согласования прогнозов, и~доказательство существования в~этой игре равновесия Нэша. Раздел~5
 описывает оптимизационную задачу, к~которой сводится задача
согласования прогнозов, а~также преимущества и~недостатки
алгоритма. В~разд.~6 приводятся примеры функций потерь, для
которых применим алгоритм GTOp. В~разд.~7 приведены результаты
экспериментов. В~разд.~8 подводятся итоги и~делаются выводы.



\section{Обозначения}

В этом разделе вводится система обозначений, которая будет
использоваться в~настоящей работе. Будем обозначать временной ряд
через вектор $\x$, элементы временн$\acute{\mbox{о}}$го ряда будем снабжать
индексом $t$, $t \hm= 1, \ldots, T$,
 где $T$~--- длина истории.
$$
    \x = \{ x_t \}_{t = 1}^T.
$$

Общее количество временн$\acute{\mbox{ы}}$х рядов во всей иерархии будем
обозначать~$d$. Далее для наглядности будем рассматривать иерархию
рядов, изображенную на рис.~1. Она
содержит один временной ряд на верхнем уровне и~$n$~рядов на
нижнем. Для этой иерархии $d \hm= 1\hm + n$.

Временн$\acute{\mbox{ы}}$е ряды нижнего уровня будут обозначаться $\x(i)$, $i \hm= 1,
\ldots, n$, где $n$~--- число временн$\acute{\mbox{ы}}$х рядов на нижнем
уровне. Временной ряд верхнего уровня обозначается как $\x(:)$. Во
избежание\linebreak\vspace*{-12pt}

\pagebreak

\begin{center}  %fig1
\vspace*{-1pt}
\mbox{%
 \epsfxsize=76.721mm
 \epsfbox{ste-1.eps}
 }


\vspace*{3pt}

%\noindent
{{\figurename~1}\ \ \small{Плоская двухуровневая иерархия}}

\end{center}

\vspace*{6pt}


\addtocounter{figure}{1}




\noindent
 путаницы условимся использовать нижние индексы для
обозначения отсчетов времени, а~индексы в~скобках использовать для
обозначения положения временн$\acute{\mbox{ы}}$х рядов в~структуре иерархии.
Элементы временн$\acute{\mbox{ы}}$х рядов, составляющих плоскую двух\-уров\-не\-вую
иерархию, обозначаются соответственно $x_t(i)$, $i \hm= 1, \ldots,
n$, $x_t(:)$. Их соотношение задается формулой
\begin{equation}
\label{eq:2LevelsConstraint}
    x_t(:) = \sum\limits_{i = 1}^n x_t(i)\,, \enskip t = 1, \ldots, T\,.
\end{equation}
Будем называть соотношение~(\ref{eq:2LevelsConstraint})
\textbf{условием согласованности}. Прогнозы этих временн$\acute{\mbox{ы}}$х рядов
будем обозначать <<шляпками>>, опуская нижние индексы, чтобы
избежать излишне громоздких обозначений. Прогнозироваться будет
всегда $(T \hm+ 1)$-е значение временн$\acute{\mbox{о}}$го ряда
$\hat{x}(i)$, $i \hm= 1, \ldots, n$, $\hat{x}(:).$
Согласованные прогнозы будут также обозначаться без нижних
индексов: $\hat{y}(i)$, $i \hm= 1, \ldots, n$, $\hat{y}(:) \hm=
\sum\nolimits_{i = 1}^n \hat{y}(i).$

Запишем все временн$\acute{\mbox{ы}}$е ряды в~матрицу, каждая строка которой
соответствует одному временн$\acute{\mbox{о}}$му ряду. Для иерархии
с~рис.~1 эта матрица будет размера $(1 + n)
\times T$ и~выглядеть следующим образом:
\begin{equation*}
%\label{eq:SeriesMatrix}
    X = \left(%
\begin{array}{cccc}
  x_1(:) & x_2(:) & \cdots & x_T(:) \\[6pt]
  x_1(1) & x_2(1) & \cdots & x_T(1) \\[6pt]
  \cdots & \cdots & \cdots & \cdots \\[6pt]
  x_1(n) & x_2(n) & \cdots & x_T(n)
\end{array}%
\right).
\end{equation*}
Будем называть срезом иерархии в~момент времени~$t$ столбец
матрицы~$X$, соответствующий моменту времени~$t$. Для удобства
записи введем векторы, соответствующие срезу иерархии в~момент
времени~$t$, прогнозам и~согласованным прогнозам. В~этих векторах
значения, соответствующие разным временн$\acute{\mbox{ы}}$м рядам,
записаны в~столбец, начиная с~верхнего уровня иерархии и~заканчивая нижним
уровнем:

\noindent
$$
    \boldsymbol{\chi}_t = \left(%
    \begin{array}{c}
        x_t(:) \\
        x_t(1) \\
        \vdots \\
        x_t(n) \\
    \end{array}%
    \right)\,; \quad
    \hchi = \left(%
    \begin{array}{c}
        \hx(:) \\
        \hx(1) \\
        \vdots \\
        \hx(n) \\
    \end{array}%
    \right)\,; \quad
    \hphi = \left(%
    \begin{array}{c}
        \hy(:) \\
        \hy(1) \\
        \vdots \\
        \hy(n) \\
    \end{array}%
    \right).
$$
Условие~(\ref{eq:2LevelsConstraint}) для векторов
$\boldsymbol{\chi}_t$ и~$\hphi$ запишем, введя мат\-ри\-цу
связей размером $1 \times (n + 1)$:
$$
    S = \left(%
        \begin{array}{cccc}
            -1 & 1 & \cdots & 1 \\
        \end{array}%
        \right).
$$
Тогда условие согласованности запишется кратко
$$
    S \boldsymbol{\chi}_t = 0\,; \quad S \hphi = 0.
$$

В случае, когда иерархия имеет более сложную структуру, чем на
рис.~1, векторы~$\boldsymbol{\chi}_t$,
$\hchi$ и~$\hphi$ имеют размерность $d$, матрица $X$ имеет ровно
$d$ строк, временн$\acute{\mbox{ы}}$е ряды в~ней записываются
от верхних уровней к~нижним. А~размерность матрицы связи~$S$ равна $c \times d$, где
$c$ --- число узлов в~графе иерархии или, другими словами,
количество огра\-ни\-че\-ний-ра\-венств, наложенных на элементы срезов
иерархии~$\boldsymbol{\chi}_t$.



\section{Задача согласования прогнозов иерархических временных рядов}

Сформулируем задачу согласования прогнозов иерархических временн$\acute{\mbox{ы}}$х
рядов. Для этого введем еще ряд необходимых обозначений.

Пусть дан набор из $d$ временн$\acute{\mbox{ы}}$х рядов, значения которых записаны
в матрицу~$X$ размера $d \times T$:
$$
    X = \left(%
\begin{array}{cccc}
  \boldsymbol{\chi}_1 & \boldsymbol{\chi}_2 & \cdots & \boldsymbol{\chi}_T \\
\end{array}%
\right),
$$
где каждый столбец $\boldsymbol{\chi}_t$ соответствует срезу в~момент времени~$t$, а~каждая строка $\x_i$~--- одному
временному ряду. Пусть структура иерархии задана матрицей связи $S$ так, что
для всех $t \hm= 1, \ldots ,T$ выполнено условие согласованности:
$$
    S \boldsymbol{\chi}_t = 0\,.
$$
Пусть даны прогнозы  значений~$\hchi$ для всех временн$\acute{\mbox{ы}}$х рядов в~момент
времени $T\hm + 1$ и~задана функция суммарных потерь при
прогнозировании иерархии
\begin{equation}
\label{eq:HierarchyLoss}
    l_h(\hchi, \boldsymbol{\chi}_{T+1})\,.
\end{equation}

Определим множество
\begin{equation}
\label{eq:SetReconciled}
    \mathcal{A} = \{ \boldsymbol{\chi} \in \mathbb{R}^d \mid S \boldsymbol{\chi} = 0 \}\,,
\end{equation}
где $\boldsymbol{\chi}$~--- произвольный $d$-мерный вектор, а~$S$~--- заданная матрица связи. Отметим, что все срезы $\boldsymbol{\chi}_t$, $t \hm=
1, \ldots, T$, лежат в~множестве~$\mathcal{A}$. Также в~нем должны лежать
согласованные прогнозы $\hphi$.

В ряде задач прогнозы должны удовлетворять некоторым ограничениям,
связанным с~физической природой прогнозируемой величины. В~связи
с чем введем множество
\begin{multline}
\label{eq:SetPhysicsConstraint}
    \mathcal{B} = \{ \boldsymbol{\chi} \in \mathbb{R}^d \mid \boldsymbol{\chi}(i)
    \in [A_i, B_i]\,,\\
     A_i, B_i \in [-\infty, +\infty]\,,\ i = 1, \ldots, d
    \},
\end{multline}
где $\boldsymbol{\chi}$~--- произвольный $d$-мер\-ный вектор; $A_i$
и~$B_i$ задают отрезок, в~котором должна находиться \mbox{$i$-я}
компонента этого вектора. Например, в~случае $A_1 \hm= \cdots = A_d \hm=
0$, $B_1 \hm= \cdots = B_d \hm= +\infty$ вектор~$\boldsymbol{\chi}$
лежит в~положительном октанте, а~конечные значения $A_1, \cdots,
A_d$, $B_1, \cdots, B_d$ могут задавать интервалы, в~которых
должны лежать прогнозы. Отсутствие ка\-ких-ли\-бо ограничений задается
значениями $A_1 \hm= \cdots = A_d \hm= -\infty$, $B_1 \hm= \cdots = B_d \hm=
+\infty$.

Введя все необходимые дополнительные обозначения, можно
сформулировать задачу поиска согласованных прогнозов. Требуется
найти вектор прогнозов~$\hphi$, удовлетворяющий следующим
требованиям.
\begin{description}
    \item[Согласованность:] вектор прогнозов~$\hphi$ должен
    удов\-ле\-тво\-рять структуре иерархии, заданной мат\-ри\-цей связи~$S$,
    т.\,е.\ $\hphi \hm\in \mathcal{A}$.
    \item[Ограничения:] вектор прогнозов~$\hphi$ должен
    удовлетворять наложенным ограничениям, т.\,е.\ $\hphi \hm\in \mathcal{B}$.
    \item[Качество:] общие потери при использовании согласованных
    прогнозов не должны превышать общие потери при использовании
    независимых прогнозов, т.\,е.\ ${l_h(\hphi, \boldsymbol{\chi}_{T + 1}) \hm\leq l_h(\hchi, \boldsymbol{\chi}_{T +    1})}$.
\end{description}


\section{Задача согласования прогнозов как поиск равновесия
в~антагонистической игре}

В этом разделе задача согласования прогнозов, сформулированная
выше, рассматривается как\linebreak антагонистическая игра. Такое
представление не влияет на решение задачи согласования прогнозов
и~направлено лишь на достижение на\-гляд\-ности и~интерпретируемости
полученных ре\-зультатов. В~первой части раздела приводятся
необходимые определения и~факты из теории игр~[17--19],
 во второй части вводится
антагонистическая игра, соответствующая задаче согласования
прогнозов, в~треть\-ей части формулируется и~доказывается теорема о
существовании в~этой игре равновесия Нэша, которая является
обобщением теоремы, доказанной в~\cite{vanerven:hal-00920559},
и~приводится следствие из этой теоремы, однозначно определяющее
выбор оптимального вектора согласованных прогнозов.

\subsection{Понятие антагонистической игры}

\noindent
\textbf{Определение 1.} %\label{def:antagonistic_game}
\textit{Система
$$
    \Gamma = (\mathcal{M}, \mathcal{N}, L)\,,
$$
где $\mathcal{M}$ и~$\mathcal{N}$~--- непустые множества и~функция $L \colon \mathcal{M}
\times \mathcal{N} \to \mathbb{R}$, называется антагонистической игрой
(игрой с~нулевой суммой) в~нормальной форме. Элементы $\mu \hm\in \mathcal{M}$
и~$\nu
\hm\in \mathcal{N}$ называются стратегиями игроков~$1$ и~$2$ соответственно,
$L$~--- функцией потерь игрока~$1$. Потери игрока~$2$ полагаются равными
$-L(\mu, \nu)$}.

\smallskip

\noindent
\textbf{Определение 2.} %\label{def:clear_strategy}
\textit{Говорят, что игра разыгрывается в~чис\-тых стратегиях, если оба
игрока из имеющихся наборов действий $\mathcal{M}$ и~$\mathcal{N}$ выбирают по одному
действию $\mu$ и~$\nu$ соответственно}.


\smallskip

\noindent
\textbf{Определение 3.}
%\label{def:mixed_strategy}
\textit{Введем на множествах стратегий~$\mathcal{M}$ и~$\mathcal{N}$ вероятностные
распределения $p(\mu)$ и~$q(\nu)$ соответственно:
$$
    \int\limits_{\mathcal{M}} p(\mu)\, d\mu = 1\,; \quad
     \int\limits_{\mathcal{N}} q(\nu)\, d\nu =
    1\,.
$$
Распределения $p(\mu)$ и~$q(\nu)$ задают смешанные стратегии в~игре~$\Gamma$,
если игрок~$1$ выбирает действие в~соответствии с~распределением
$p(\mu)$ и~игрок~$2$ выбирает действие в~соответствии с~распределением $q(\nu)$}.

\smallskip


Чистые стратегии являются частным случаем смешанных. Поэтому далее
будут рассматриваться смешанные стратегии, за исключением
специально оговоренных моментов. Обозначать стратегии будем~$\pcd$ и~$\q$.

Математическое ожидание потерь игрока 1 при паре смешанных
стратегий $\pcd$ и~$\q$ обозначим через
$$
    \bar{L}(\pcd, \q) = \int\limits_{\mathcal{M}} \int\limits_{\mathcal{N}} p(\mu) q(\nu) L(\mu,
    \nu)\, d\mu \, d\nu\,.
$$
Игрок 1 преследует цель минимизировать эту величину при любых
действиях игрока~2.

\noindent
\textbf{Определение 4.} %\label{def:Nash_equilibrium}
Пара стратегий $(p(\cdot), q(\cdot))$ называется равновесием Нэша
в~смешанных стратегиях в~игре $\Gamma$, если для любых
$\pcd^{\prime}$ и~$\q^{\prime}$ выполнено неравенство:
$$
    \bar{L}(\pcd, \q^{\prime}) \leq \bar{L}(\pcd, \q) \leq \bar{L}(\pcd^{\prime},
    \q)\,.
$$


При равновесии Нэша ни одному из игроков не выгодно отклоняться от
равновесной стратегии, если второй продолжает придерживаться
равновесной стратегии. При этом игрок~1 минимизирует свои потери в~ситуации, когда игрок~2 действует наиболее выгодным для себя
образом. Отметим также, что равновесие Нэша является седловой
точкой функции $\bar{L}(\pcd, \q)$.

\smallskip

\noindent
\textbf{Теорема~1.} %\label{th:equlibrium_existence}
\textit{В антагонистической игре равновесие Нэша существует тогда и~только
тогда, когда определена величина}
$$
    V = \min\limits_{\pcd^{\prime}} \max\limits_{\q^{\prime}} \bar{L}(\pcd^{\prime}\,,
    \q^{\prime}) =
    \max\limits_{\q^{\prime}} \min\limits_{\pcd^{\prime}} \bar{L}(\pcd^{\prime}\,,
    \q^{\prime}).
$$


Величина $V$ называется ценой игры. Доказательство этой теоремы
в~настоящей статье не приводится, при желании его можно
найти в~[17--19].

\subsection{Антагонистическая игра, описывающая задачу согласования~прогнозов}

Вернемся к~рассмотрению введенных в~разд.~3
множеств~$\mathcal{A}$~(\ref{eq:SetReconciled})
и~$\mathcal{B}$~(\ref{eq:SetPhysicsConstraint}). Напомним, что множество
$\mathcal{A}$ содержит все $d$-мер\-ные векторы, удовлетворяющие структуре
иерархии, заданной матрицей связи~$S$. Множество~$\mathcal{B}$ содержит
$d$-мер\-ные векторы, удовлетворяющие огра\-ни\-че\-ни\-ям-не\-ра\-вен\-ст\-вам
рассматриваемой задачи прогнозирования.

Будем рассматривать антагонистическую иг\-ру~$\Gamma$, в~которой
игрок~1 выбирает вектор согласованных прогнозов~$\hphi$ из
множества $\mathcal{A} \cap \mathcal{B}$, которые одновременно удовлетворяют
структуре иерархии, заданной матрицей связи~$S$,
и~огра\-ни\-че\-ни\-ям-не\-ра\-вен\-ст\-вам,
задающим множество~$\mathcal{B}$, игрок~2 также
выбирает вектор действительных значений $\boldsymbol{\chi}_{T+1}$,
удовлетворяющих структуре иерархии и~физическим ограничениям
(можно считать игрока~2 природой).

Определим множества стратегий $\mathcal{M}$ и~$\mathcal{N}$ игроков~1 и~2 как
пересечение множеств $\mathcal{A} \cap \mathcal{B}$. Функцию потерь игрока~1
определим с~помощью функции потерь по иерархии~(\ref{eq:HierarchyLoss}):
$$
    L(\hphi, \boldsymbol{\chi}_{T+1}) = l_h(\boldsymbol{\chi}_{T+1}, \hphi) - l_h(\boldsymbol{\chi}_{T+1}, \hchi)\,,
$$
где вектор независимых прогнозов $\hchi$ считается заданным и~не
зависит от действий, выбираемых игроками. Считается, что $\hchi
\hm\in \mathcal{B}$. Такой выбор функции потерь игрока~1 связан с~тем
соображением, что при $L(\hphi, \boldsymbol{\chi}_{T+1}) \hm= 0$
качество согласованных прогнозов не хуже, чем качество независимых
прогнозов, а~при $L(\hphi, \boldsymbol{\chi}_{T+1}) \hm< 0$ и~вовсе
превосходит его.

Таким образом, получена антагонистическая игра
\begin{equation}
\label{eq:game}
    \Gamma = (\mathcal{A} \cap \mathcal{B},~\mathcal{A} \cap \mathcal{B},~l_h(\boldsymbol{\chi}_{T+1}, \hphi) - l_h(\boldsymbol{\chi}_{T+1}, \hchi))\,,
\end{equation}
где игрок~1 выбирает вектор согласованных прогнозов $\hphi$ из
множества $\mathcal{A} \cap \mathcal{B}$, а~игрок~2 выбирает вектор действительных
значений элементов временн$\acute{\mbox{ы}}$х рядов $\boldsymbol{\chi}_{T+1}$ из
множества~$\mathcal{A} \cap \mathcal{B}$. При этом первый игрок преследует цель
минимизировать свои потери, выраженные функцией $L(\hphi,
\boldsymbol{\chi}_{T+1})$, при любых действиях игрока~2, т.\,е.\
при любом векторе действительных значений
$\boldsymbol{\chi}_{T+1}$. Эта цель достигается, если игрок~1
воспользуется стратегией, входящей в~равновесие Нэша.

\subsection{Существование равновесия Нэша}

В этой части раздела будет показано, что при выполнении ряда
естественных требований к~множествам $\mathcal{A}$~(\ref{eq:SetReconciled})
и~$\mathcal{B}$~(\ref{eq:SetPhysicsConstraint}) и~функции суммарных потерь
при прогнозировании иерархии $l_h$~(\ref{eq:HierarchyLoss})
в~антагонистической игре (\ref{eq:game}), описывающей задачу
согласования прогнозов, существует равновесие Нэша в~чис\-тых
стратегиях. Также будет показано, что соответствующее этому
равновесию значение функции потерь игрока~1 неположительно, что
гарантирует неухудшение качества прогнозов при переходе от
независимых прогнозов к~согласованным. Рас\-смот\-рим эти требования.

\smallskip

\noindent
\textbf{Определение 5.}
%\label{def:ConvexSet}
Множество $\mathcal{C} \subseteq \mathbb{R}^d$ называется
выпуклым~\cite{boyd2009convex}, если для любых $\boldsymbol{\chi}_1 \hm\in
\mathcal{C}$ и~$\boldsymbol{\chi}_2 \hm\in \mathcal{C}$ и~любого $0
\hm\leq \alpha \hm\leq 1$ выполнено
$$
    \alpha \boldsymbol{\chi}_1 + (1 - \alpha) \boldsymbol{\chi}_2 \in \mathcal{C}\,.
$$


\smallskip

Заметим, что множества $\mathcal{A}$ (\ref{eq:SetReconciled}) и~$\mathcal{B}$~(\ref{eq:SetPhysicsConstraint}) выпуклы и~замкнуты.

\smallskip

\noindent
\textbf{Предположение 1.}
%\label{ass:Convex}
\textit{Будем предполагать, что пересечение множеств
$\mathcal{A}$ и~$\mathcal{B}$ не пусто:
$\mathcal{A} \cap \mathcal{B} \neq \varnothing$}.

\smallskip

Требование непустого пересечения этих множеств естественно, так
как в~противном случае неразрешима задача поиска вектора
согласованных прогнозов~$\hphi$, который должен одновременно
принадлежать обоим множествам. Также отметим, что множество $\mathcal{A}
\cap \mathcal{B}$ является выпуклым и~замкнутым как пересечение двух
выпуклых и~замкнутых множеств~\cite{boyd2009convex}.

\smallskip

\noindent
\textbf{Предположение 2.} %\label{ass:NonNegative}
\textit{Будем считать, что функция суммарных потерь~$l_h$}~(\ref{eq:HierarchyLoss})
\textit{неотрицательна и~равна нулю только при
равенстве аргументов}:
$$
    l_h(\boldsymbol{\chi}_{T+1}, \hchi) \geq 0 \mbox{ для\ всех } \boldsymbol{\chi}_{T+1},
    \hchi\,;
$$
$$
    l_h(\boldsymbol{\chi}_{T+1}, \hchi) = 0 \quad \Leftrightarrow \quad \boldsymbol{\chi}_{T+1} =
    \hchi\,.
$$

\smallskip

Равенство аргументов $\boldsymbol{\chi}_{T+1} \hm= \hchi$
соответствует случаю, когда прогноз полностью совпадает с~действительными значениями. В~этом случае потери равны нулю. Во
всех остальных случаях потери при прогнозе положительные.



\noindent
\textbf{Определение 6.} %\label{def:Projection}
Проекцией точки $\boldsymbol{\chi}_0 \in \mathbb{R}^d$ на множество
$\mathcal{C} \subseteq \mathbb{R}^d$, инициированной функцией расстояния~$f$,
называется точка
$$
    \boldsymbol{\chi}_{\mathrm{proj}} = \argmin\limits_{\boldsymbol{\chi} \in \mathcal{C}} f(\boldsymbol{\chi},
    \boldsymbol{\chi}_0)\,.
$$


%\smallskip

\noindent
\textbf{Предположение 3.}
%\label{ass:Projection}
\textit{Пусть существует проекция точки из $\mathbb{R}^d$, соответствующей вектору
независимых прогнозов $\hchi$, на выпуклое и~замкнутое мно\-жество
$\mathcal{A} \cap \mathcal{B}$, инициированная функцией суммарных потерь~$l_h$}:
$$
    \boldsymbol{\chi}_{\mathrm{proj}} = \argmin\limits_{\boldsymbol{\chi} \in \mathcal{A} \cap \mathcal{B}} l_h(\boldsymbol{\chi},
    \hchi).
$$


\smallskip

\noindent
\textbf{Предположение 4.} %\label{ass:CosinusInequality}
\textit{Пусть $\boldsymbol{\chi}_{\mathrm{proj}} \hm=
\argmin\limits_{\boldsymbol{\chi} \in \mathcal{A} \cap \mathcal{B}}
l_h(\boldsymbol{\chi}, \hchi)$. Будем предполагать, что для всех
$\boldsymbol{\chi} \hm\in \mathcal{B}$ и~для всех $\boldsymbol{\psi} \hm\in \mathcal{A}
\cap \mathcal{B}$ выполняется неравенство}
$$
    l_h(\boldsymbol{\psi}, \boldsymbol{\chi}) \geq
    l_h(\boldsymbol{\psi}, \boldsymbol{\chi}_{\mathrm{proj}}) +
    l_h(\boldsymbol{\chi}_{\mathrm{proj}},
    \boldsymbol{\chi})\,.
$$

\smallskip

Для пояснения этого требования рассмотрим частный случай, когда
$\mathbb{R}^d \hm= \mathbb{R}^3$, $\mathcal{A} \subset \mathbb{R}^2$, $\mathcal{B} \subset \mathbb{R}^3$, $\mathcal{A} \cap \mathcal{B}
\subset \mathbb{R}^2$ и~$l_h$~--- квадрат метрики Евклида
(рис.~2). Точки
$\boldsymbol{\psi}$, $\boldsymbol{\chi}$,
$\boldsymbol{\chi}_{\mathrm{proj}}$ образуют треугольник. Обозначим
$\theta$ угол при вершине $\boldsymbol{\chi}_{\mathrm{proj}}$ и~запишем
теорему косинусов:
\begin{multline*}
    l_h(\boldsymbol{\psi}, \boldsymbol{\chi}) =
    l_h(\boldsymbol{\psi}, \boldsymbol{\chi}_{\mathrm{proj}}) +
    l_h(\boldsymbol{\chi}_{\mathrm{proj}},
    \boldsymbol{\chi}) - {}\\
    {}-2 \sqrt{l_h(\boldsymbol{\psi}, \boldsymbol{\chi}_{\mathrm{proj}})}
    \sqrt{l_h(\boldsymbol{\chi}_{\mathrm{proj}},
    \boldsymbol{\chi})} \cos \theta\,.
\end{multline*}


Поскольку $\boldsymbol{\chi}_{\mathrm{proj}}$ является проекцией, то
угол~$\theta$ не может быть острым. Он прямой, если проецируемая точка
$\boldsymbol{\chi}$ находится <<над>> множеством $\mathcal{A} \cap \mathcal{B}$, и~тупой, если точка находится <<в~стороне>>. Таким образом получаем,
что $\cos \theta \hm\leq 0$, а~значит, последнее слагаемое в~теореме
косинусов неотрицательное. Исключая его и~заменяя знак равенства
на знак нестрогого неравенства, получаем, что предположение~4
соответствует естественным свойствам проекции.

\begin{center}  %fig2
\vspace*{-1pt}
\mbox{%
 \epsfxsize=69.331mm
 \epsfbox{ste-2.eps}
 }


\vspace*{3pt}

%\noindent
{{\figurename~2}\ \ \small{Пояснение к~предположению~4}}

\end{center}


%\vspace*{6pt}


\addtocounter{figure}{1}


%\smallskip

Введя предположения~1--4, сформулируем
теорему.

\smallskip

\noindent
\textbf{Теорема 2.}
%\label{th:saddle_point}
\textit{Пусть выполнены предположения~$1$--$4$. Тогда пара стратегий
$(\boldsymbol{\chi}_{\mathrm{proj}}, \boldsymbol{\chi}_{\mathrm{proj}})$ является
равновесием Нэша в~игре~(\ref{eq:game}) и~седловой точкой функции
$L(\hphi, \boldsymbol{\chi}_{T+1})\hm = l_h(\boldsymbol{\chi}_{T+1},
\hphi) \hm- l_h(\boldsymbol{\chi}_{T+1}, \hchi)$. Цена игры при этом
равна} $V \hm= -l_h(\boldsymbol{\chi}_{\mathrm{proj}}, \hchi)$.


\smallskip

\noindent
Д\,о\,к\,а\,з\,а\,т\,е\,л\,ь\,с\,т\,в\,о\,.\ \
Найдем седловую точку функции $L(\hphi, \boldsymbol{\chi}_{T+1})$
в соответствии с~определением~4  и~теоремой~1. Найдем максимум этой
функции по второму аргументу при $\hphi \hm=
\boldsymbol{\chi}_{\mathrm{proj}}$. Для этого воспользуемся предположением~4:
$$
    l_h(\boldsymbol{\psi}, \boldsymbol{\chi}) \geq l_h(\boldsymbol{\psi},
    \boldsymbol{\chi}_{\mathrm{proj}}) + l_h(\boldsymbol{\chi}_{\mathrm{proj}},
    \boldsymbol{\chi})\,.
$$
Применяя неравенство к~функции потерь~$L$ игрока~1 (подставляем
$\boldsymbol{\psi} \hm= \boldsymbol{\chi}_{T+1}$, $\boldsymbol{\chi}
\hm= \hchi$), получаем:
\begin{multline*}
    L(\boldsymbol{\chi}_{\mathrm{proj}}, \boldsymbol{\chi}_{T+1})
     = l_h(\boldsymbol{\chi}_{T+1}, \boldsymbol{\chi}_{\mathrm{proj}}) -
      l_h(\boldsymbol{\chi}_{T+1}, \hchi) \leq{}\\
      {}\leq
    l_h(\boldsymbol{\chi}_{T+1}, \hchi) -
    l_h(\boldsymbol{\chi}_{\mathrm{proj}}, \hchi) -l_h(\boldsymbol{\chi}_{T+1},
    \hchi) ={}
\\{}
    = -l_h(\boldsymbol{\chi}_{\mathrm{proj}}, \hchi) \end{multline*}
     для всех $\boldsymbol{\chi}_{T+1} \in \mathcal{A} \cap \mathcal{B}$.
Заметим также, что из предположения~2 вытекает
\begin{multline*}
    L(\boldsymbol{\chi}_{\mathrm{proj}}, \boldsymbol{\chi}_{\mathrm{proj}}) =
    l_h(\boldsymbol{\chi}_{\mathrm{proj}}, \boldsymbol{\chi}_{\mathrm{proj}}) -
    l_h(\boldsymbol{\chi}_{\mathrm{proj}}, \hchi) ={}\\
    {}= - l_h(\boldsymbol{\chi}_{\mathrm{proj}}, \hchi)\,.
\end{multline*}
Приходим к~выводу, что
$$
    L(\boldsymbol{\chi}_{\mathrm{proj}}, \boldsymbol{\chi}_{T+1}) \leq
    L(\boldsymbol{\chi}_{\mathrm{proj}}, \boldsymbol{\chi}_{\mathrm{proj}})
    $$
 для всех $\boldsymbol{\chi}_{T+1} \in \mathcal{A} \cap \mathcal{B}$.


Следовательно, максимум по второму аргументу достигается при
$\boldsymbol{\chi}_{T+1} = \boldsymbol{\chi}_{\mathrm{proj}}$.

Минимум по первому аргументу при $\boldsymbol{\chi}_{T+1}\hm =
\boldsymbol{\chi}_{\mathrm{proj}}$ находим, используя
предположение~2, из соотношения
\begin{multline*}
    \argmin\limits_{\hphi \in \mathcal{A} \cap \mathcal{B}} L(\hphi,
    \boldsymbol{\chi}_{\mathrm{proj}}) ={}\\
    {}=
    \argmin\limits_{\hphi \in \mathcal{A} \cap \mathcal{B}}
    l_h(\boldsymbol{\chi}_{\mathrm{proj}}, \hphi)
    - l_h(\boldsymbol{\chi}_{\mathrm{proj}}, \hchi).
\end{multline*}
Второе слагаемое не зависит от~$\hphi$, а~по предположению~2
функция суммарных потерь неотрицательна и~обращается в~ноль только при равенстве аргументов, поэтому
получаем
$$
    \argmin\limits_{\hphi \in \mathcal{A} \cap \mathcal{B}}
    L(\hphi, \boldsymbol{\chi}_{\mathrm{proj}}) =
    \boldsymbol{\chi}_{\mathrm{proj}}\,.
$$

Таким образом, получаем, что
\begin{multline*}
   L(\boldsymbol{\chi}_{\mathrm{proj}}, \boldsymbol{\chi}_{T+1})  \leq
   L(\boldsymbol{\chi}_{\mathrm{proj}}, \boldsymbol{\chi}_{\mathrm{proj}})
   \leq L(\hphi, \boldsymbol{\chi}_{\mathrm{proj}})\,, \\
    \boldsymbol{\chi}_{T+1},~\hphi \in \mathcal{A} \cap \mathcal{B}\,.
\end{multline*}
Следовательно, $(\boldsymbol{\chi}_{\mathrm{proj}},
\boldsymbol{\chi}_{\mathrm{proj}})$~--- седловая точка функции $L$.
И~эта пара является равновесием Нэша в~игре~(\ref{eq:game}), и~цена
игры выражается как

\vspace*{-4pt}

\noindent
\begin{multline*}
    V = \min\limits_{\hphi \in \mathcal{A} \cap \mathcal{B}} \max\limits_{\boldsymbol{\chi}_{T+1} \in \mathcal{A}} L(\hphi, \boldsymbol{\chi}_{T+1}) ={}\\
    {}=
    \max\limits_{\boldsymbol{\chi}_{T+1} \in \mathcal{A}} \min\limits_{\hphi \in \mathcal{A} \cap
    \mathcal{B}} L(\hphi, \boldsymbol{\chi}_{T+1}) ={}\\
    {}=
    L(\boldsymbol{\chi}_{\mathrm{proj}},
    \boldsymbol{\chi}_{\mathrm{proj}})= -l_h(\boldsymbol{\chi}_{\mathrm{proj}}, \hchi)\,.
\end{multline*}

\smallskip


\noindent
\textbf{Следствие 1.} %\label{col:OptimalReconcilation}
\textit{Использование в~качестве вектора согласованных прогнозов~$\hphi$
проекции вектора независимых прогнозов $\hchi\hm \in \mathbb{R}^d$ на
множество $\mathcal{A} \cap \mathcal{B}$, инициированной функцией
суммарных потерь~$l_h$, гарантирует значение функции суммарных потерь не большее,
чем при использовании вектора независимых прогнозов}~$\hchi$.


\noindent
Д\,о\,к\,а\,з\,а\,т\,е\,л\,ь\,с\,т\,в\,о\,.\ \
По теореме~2 цена игры~(\ref{eq:game}) равна
значению функции потерь игрока~1 в~точке
$(\boldsymbol{\chi}_{\mathrm{proj}}, \boldsymbol{\chi}_{\mathrm{proj}})$
и~неположительна в~силу предположения~2

\vspace*{2pt}

\noindent
$$
    V = L(\boldsymbol{\chi}_{\mathrm{proj}}, \boldsymbol{\chi}_{\mathrm{proj}})
    = -l_h(\boldsymbol{\chi}_{\mathrm{proj}}, \hchi)
    \leq 0\,.
$$
А выбор рассматриваемой функции потерь игрока~1 $L(\hphi,
\boldsymbol{\chi}_{T+1})\hm = l_h(\boldsymbol{\chi}_{T+1}, \hphi) \hm-
l_h(\boldsymbol{\chi}_{T+1}, \hchi)$ был обусловлен тем, что ее
знак совпадает со знаком изменения суммарных потерь при переходе
от вектора независимых прогнозов~$\hchi$ к~вектору согласованных
прогнозов~$\hphi$. Следовательно, при $\boldsymbol{\chi}_{T+1} \hm=
\boldsymbol{\chi}_{\mathrm{proj}}$ суммарные потери при согласованных
прогнозах меньше, чем при независимых.

Согласно определению~4 равновесия Нэша

\vspace*{2pt}

\noindent
$$
    L(\boldsymbol{\chi}_{\mathrm{proj}}, \boldsymbol{\chi}_{T+1}) \leq
    L(\boldsymbol{\chi}_{\mathrm{proj}}, \boldsymbol{\chi}_{\mathrm{proj}}) \leq 0
    $$
 для любых  $\boldsymbol{\chi}_{T+1} \in \mathcal{A} \cap \mathcal{B}$.
Поэтому при любом векторе действительных значений
$\boldsymbol{\chi}_{T+1} \hm\in \mathcal{A} \cap \mathcal{B}$ согласованные
прогнозы~$\hphi$ оказываются предпочтительнее независимых
прогнозов~$\hchi$.

\section{Алгоритм согласования прогнозов GTOp}

Согласно следствию~1 оптимальным
выбором вектора согласованных прогнозов~$\hphi$ является проекция
вектора независимых прогнозов~$\hchi$ на множество $\mathcal{A} \cap \mathcal{B}$,
инициированная функцией суммарных потерь~$l_h$. Множество $\mathcal{A} \cap
\mathcal{B}$ содержит векторы размерности~$d$, удовлетворяющие структуре
иерархии, так как множество $\mathcal{A}$ задается
огра\-ни\-че\-ни\-ями-ра\-вен\-ст\-ва\-ми, порожденными матрицей
связи иерархии~$S$. В~то же время $\mathcal{A} \cap \mathcal{B}$ содержит
\mbox{$d$-мер}\-ные векторы,
удовлетворяющие огра\-ни\-че\-ни\-ям-не\-ра\-вен\-ст\-вам из множества~$\mathcal{B}$. Таким
образом, задача поиска проекции~--- это оптимизационная задача с~ограничениями типа равенства и~неравенства:

\noindent
\begin{equation}
\label{eq:OptimProblem}
    \left\{%
\begin{array}{l}
    l_h(\boldsymbol{\chi}, \hchi) \rightarrow \min\limits_{\boldsymbol{\chi}}\,, \\
    \boldsymbol{\chi} \in \mathcal{A}\ \mbox{(ограничения-равенства)}; \\
    \boldsymbol{\chi} \in \mathcal{B}\ \mbox{(ограничения-неравенства)}. \\
\end{array}%
\right.
\end{equation}

Алгоритм согласования прогнозов иерархических временн$\acute{\mbox{ы}}$х рядов GTOp
заключается в~решении оптимизационной задачи~(\ref{eq:OptimProblem}). Достоинства этого алгоритма заключаются в~том, что он требует от вектора независимых прогнозов~$\hchi$ лишь
принадлежности множеству~$\mathcal{B}$, и~не требует не\-сме\-щен\-ности
независимых прогнозов, а~следовательно, для получения независимых
прогнозов можно использовать любой алгоритм прогнозирования. Также
GTOp не требует оценки погрешностей независимых прогнозов. Самое
важное, что GTOp обеспечивает неухудшение качества прогнозирования
при замене независимых прогнозов на согласованные прогнозы. При
этом на структуру иерархии, огра\-ни\-че\-ния-не\-ра\-вен\-ст\-ва на прогнозы и~функцию суммарных потерь накладываются лишь общие ограничения,
гарантирующие существование решения оптимизационной задачи~(\ref{eq:OptimProblem}). Еще одно достоинство алгоритма GTOp в~том, что он позволяет согласовывать прогнозы для иерархий любой
сложности одновременно по всем уровням, учитывая все связи в~иерархии и~решая одну оптимизационную задачу. От сложности
иерархии и~количества временн$\acute{\mbox{ы}}$х рядов и~уровней в~оптимизационной
задаче зависит число переменных и~ограничений.

\section{Дивергенция Брегмана}

В этом разделе будет описано семейство функций двух переменных,
удовлетворяющих предположениям~1--4, и~приведен ряд примеров функций из
этого семейства, которые можно использовать в~задаче согласования
прогнозов в~качестве функции суммарных потерь~$l_h$~(\ref{eq:HierarchyLoss}). Все эти функции двух переменных
называются дивергенциями Брегмана~\cite{cesa2006prediction, bregman1967relaxation}
и~порождаются функциями одной переменной, обладающими следующими свойствами.

\smallskip

\noindent
\textbf{Определение 7.}
%\label{def:Legendre_function}
Функцией Лежандра~\cite{cesa2006prediction} называется функция $F
\colon \mathcal{B} \to \mathbb{R}$, которая удовлетворяет следующим условиям:
\begin{itemize}
    \item $\mathcal{B} \subseteq \mathbb{R}^d$~--- непустое множество, и~внутренность
    $\mathcal{B}$ выпукла;
    \item $F$~--- строго выпуклая функция с~непрерывной первой
    производной на множестве~$\mathcal{B}$;
    \item если $\boldsymbol{\chi}_1, \boldsymbol{\chi}_2, \ldots \in \mathcal{B}$~---
    последовательность, сходящаяся к~граничной точке $\mathcal{B}$, то $\| \nabla F(\boldsymbol{\chi}_n) \| \hm\rightarrow
    \infty$ при $n \hm\rightarrow \infty$.
\end{itemize}


%\smallskip

\noindent
\textbf{Определение 8.} %\label{def:Bregman_divergence}
Дивергенцией Брегмана, порожденной функцией Лежандра $F \colon \mathcal{B}
\hm\to \mathbb{R}$, называется неотрицательная функция
$D_F \colon \mathcal{B} \times
\mathrm{int}\,(\mathcal{B}) \to \mathbb{R}$, определенная как
$$
    D_F(\mathbf{u}, \mathbf{v}) = F(\mathbf{u}) - F(\mathbf{v}) -
    (\mathbf{u} - \mathbf{v}) \nabla F(\mathbf{v})\,.
$$


\paragraph*{Свойства дивергенции Брегмана:}
\begin{itemize}
    \item для всех $\mathbf{u}$ и~$\mathbf{v}$ выполнено $D_F(\mathbf{u}, \mathbf{v}) \geq
    0$. Это следует из выпуклости функции $F$;
    \item $D_F(\mathbf{u}, \mathbf{v})$ выпукла по первому
    аргументу $\mathbf{u}$, но необязательно выпукла по второму
    аргументу~$\mathbf{v}$;
    \item для любых $\alpha, \beta \in \mathbb{R}$ и~любых функций
    Лежандра $F_1$ и~$F_2$ выполнено
    $$
    D_{\alpha F_1 + \beta F_2}(\mathbf{u}, \mathbf{v}) =
    \alpha D_{F_1}(\mathbf{u}, \mathbf{v}) + \beta D_{F_2}(\mathbf{u},
    \mathbf{v})\,.
    $$
\end{itemize}

%\smallskip

\noindent
\textbf{Определение 9.} %\label{def:Bregman_projection}
Пусть $F \colon \mathcal{B} \hm\to \mathbb{R}$~--- функция Лежандра и~$\mathcal{A} \subset
\mathbb{R}^d$~--- замкнутое выпуклое множество, такое что $\mathcal{A} \cap \mathcal{B} \neq
\varnothing$. Проекция Брегмана $\mathbf{w}^{\prime}$ точки
$\mathbf{w} \hm\in \mathrm{int}\,(\mathcal{B})$ на множество~$\mathcal{A}$~--- это
$$
    \mathbf{w}^{\prime} = \argmin\limits_{\mathbf{u} \in \mathcal{A} \cap
    \mathcal{B}} D_F(\mathbf{u}, \mathbf{w})\,.
$$


\smallskip

%\smallskip

\noindent
\textbf{Теорема 3.}
%\label{lemma:Bregman_projection}
\textit{Для всех функций Лежандра $F \colon \mathcal{B} \to \mathbb{R}$, для всех замкнутых
выпуклых множеств $\mathcal{A} \subset \mathbb{R}^d$, имеющих непустое пересечение
$\mathcal{A} \cap \mathcal{B} \neq \varnothing$, и~для всех точек $\mathbf{w} \in
\mathrm{int}\,(\mathcal{B})$ проекция Брегмана точки $\mathbf{w}$ на множество
$\mathcal{A}$ существует и~единственна}.


\smallskip

\noindent
Д\,о\,к\,а\,з\,а\,т\,е\,л\,ь\,с\,т\,в\,о\ \ этой теоремы приведено в~\cite{bregman1967relaxation}.

\smallskip

\noindent
\textbf{Теорема~4.}
%\label{lemma:generalized_pythagorean_inequality}
\textit{Пусть $F$~--- функция Лежандра. Для всех $\mathbf{w} \hm\in
\mathrm{int}\,(\mathcal{B})$ и~для всех замкнутых выпуклых множеств $\mathcal{A}
\subseteq \mathbb{R}^d$ с~непустым пересечением $\mathcal{A} \cap \mathcal{B} \neq
\varnothing$, если $\mathbf{w}^{\prime}$~--- проекция Брегмана
точки $\mathbf{w}$ на множество $\mathcal{A}$ $(\mathbf{w}^{\prime} \hm=
\argmin\limits_{\mathbf{v} \in \mathcal{A} \cap \mathcal{B}} D_F(\mathbf{v},
\mathbf{w}))$, то верно неравенство}:
$$
    D_F(\mathbf{u}, \mathbf{w}) \geq D_F(\mathbf{u},
    \mathbf{w}^{\prime}) + D_F(\mathbf{w}^{\prime}, \mathbf{w})
$$
для всех $\mathbf{u} \in \mathcal{A}$.

\noindent
Д\,о\,к\,а\,з\,а\,т\,е\,л\,ь\,с\,т\,в\,о\ \ этого факта можно найти в~\cite{cesa2006prediction}.

\smallskip

Соотнесем перечисленные свойства дивергенции Брегмана и~предположения~1--4. Определение~8 и~свойство~1 дивергенции Брегмана
обеспечивают выполнение предположения~2 о
знаке функции суммарных потерь. В~определении~9
и~тео\-ре\-ме~3 предполагается выпуклость и~замкнутость
множеств $\mathcal{A}$ и~$\mathcal{B}$ и~их непустое пересечение,
как и~в~предположении~1. Теорема~4 гарантирует
выполнение предположения~3 о~выпуклости
множеств и~существовании и~единственности проекции. Наконец,
теорема~4 гарантирует
выполнение предположения~4.
Следовательно, для функций суммарных потерь $l_h$~(\ref{eq:HierarchyLoss}),
являющихся дивергенциями Брегмана,
выполнены все условия теоремы~2. Использование
в качестве вектора согласованных прогнозов $\hphi$ проекции
Брегмана вектора независимых прогнозов $\hchi \hm\in \mathcal{B}$, где
множество~$\mathcal{B}$ определено по формуле~(\ref{eq:SetPhysicsConstraint}) и~является выпуклым и~замкнутым,
имеющим непустое пересечение
с~множеством $\mathcal{A}$~(\ref{eq:SetReconciled}),
на множество~$\mathcal{A}\cap \mathcal{B}$, включающее векторы
прогнозов, удовлетворяющих структуре иерархии, гарантирует
неухудшение качества прогнозов.

Следующие функции являются дивергенциями Брегмана и~могут быть
использованы в~качестве функций суммарных потерь~(\ref{eq:HierarchyLoss}) при согласовании прогнозов.
\begin{description}
    \item[Квадрат евклидового расстояния] \

    $D_F(\uu, \mathbf{v}) \hm= \| \uu\hm -
    \mathbf{v} \|^2$~--- канонический пример дивергенции Брегмана, порождается
    функцией $F(\uu) \hm= \| \uu \|^2$.
    \item[Квадрат расстояния Махаланобиуса]\

    $D_F(\uu, \mathbf{v})\hm = (1/2)(\uu - \mathbf{v})^{\mathrm{T}} Q (\uu \hm-
    \mathbf{v})$~--- обобщение евклидового расстояния, порождается
    квадратичной формой $F(\uu) \hm= (1/2) \uu^{\mathrm{T}} Q \uu$.
    \item[Обобщенная дивергенция Куль\-ба\-ка--Лейб\-ле\-ра]\

    $D_F(\uu, \mathbf{v}) \hm=
    \sum\nolimits_{i = 1}^n u_i \log ({u_i}/{v_i})
 \hm   - \sum\nolimits_{i = 1}^n u_i\hm + \sum\nolimits_{i = 1}^n v_i$
 по\-рож\-да\-ет\-ся     функцией
 $F(\uu) \hm= \sum\nolimits_{i = 1}^n u_i \log u_i \hm-
    \sum\nolimits_{i = 1}^n
    u_i$.
    \item[Расстояние  Itakura--Saito]\

    $D_F(\uu, \mathbf{v}) = \sum\nolimits_{i = 1}^n \left(
    (u_i/v_i) \hm- \log ({u_i}/{v_i}) \hm- 1    \right)$ порождается функцией
    $F(\uu)\hm = - \sum\nolimits_{i = 1}^n \log
    u_i$.
\end{description}

\begin{table*}\small
\begin{center}
   \begin{tabular}{|c|c|c|c|c|c|c|c|}
   \multicolumn{8}{c}{Вид записи базы данных железнодорожных перевозок}\\
\multicolumn{8}{c}{\ }\\[-4pt]
  \hline
  \tabcolsep=0pt\begin{tabular}{c}Дата\\ погрузки\end{tabular} &
    \tabcolsep=0pt\begin{tabular}{c}Станция\\ отправления\end{tabular} &
      \tabcolsep=0pt\begin{tabular}{c}Станция\\ назначения\end{tabular} &
        \tabcolsep=0pt\begin{tabular}{c}Количество\\ вагонов\end{tabular} &
          \tabcolsep=0pt\begin{tabular}{c}Код\\ груза\end{tabular} &
            \tabcolsep=0pt\begin{tabular}{c}Род\\ вагона\end{tabular} &
              \tabcolsep=0pt\begin{tabular}{c}Суммарный\\ вес груза\end{tabular} &
                \tabcolsep=0pt\begin{tabular}{c}Признак\\ маршрутной\\ отправки\end{tabular} \\
  \hline
  2007-01-01 & 020108 & 932902 & 1 & 1 & 216 & 56 & 9 \\
  \hline
\end{tabular}
\end{center}
%\vspace*{6pt}
\end{table*}



\section{Эксперимент}

В экспериментальной части рассматриваются данные о посуточной
загруженности железнодорожных узлов РЖД. Из имеющихся данных были
сформированы временн$\acute{\mbox{ы}}$е ряды, описывающие отправление 37~различных
типов груза со станций 98~железнодорожных веток посуточно.
Рассмотрены два вида иерархии.

В~случае плоской двухуровневой
иерархии решены задачи согласования прогнозов отправления грузов
по отдельности и~всех грузов в~сумме для каждой ветки, а~так\-же
решены задачи согласования прогнозов отправления груза с~каждой
ветки и~суммарного отправления груза со всех веток для каждого
типа груза.

В~случае неплоской трехуровневой иерархии решена
задача согласования прогнозов всех име\-ющих\-ся временн$\acute{\mbox{ы}}$х рядов.
Демонстрируется уменьшение значения функции суммарных потерь при
переходе от независимых прогнозов к~согласованным. В~качестве
функции суммарных потерь был использован квад\-рат евклидового
расстояния. Независимые прогнозы были получены алгоритмом Hist,
описанным в~работах~\cite{stenina2014reconciliation, medvednikova2012nonparametric}.

\vspace*{-5pt}

\paragraph*{Экспериментальные данные.} В~эксперименте использованы данные о посуточной загруженности
железнодорожных узлов РЖД с~1~января 2007~г.\ по 22~апреля 2008~г. В~таблице
приведен пример записи базы данных.


Коды станций представляют собой шестизначные числа. Станции, в~коде которых две первые цифры совпадают, входят в~одну
железнодорожную ветку. Станций отправления~--- 1566, станций
назначения~--- 1902, веток~--- 98. Код груза~--- натуральное число от~1 до~37; также имеются перевозки, где код груза не указан. Род
вагона~--- натуральное число, в~имеющихся данных~75~различных
родов вагонов.

\vspace*{-5pt}

\paragraph*{Иерархическая структура.}
Экспериментальные данные удовлетворяют структуре, изображенной на
рис.~3.

Как видно из рисунка, иерархия не
является плоской и~содержит три уровня временн$\acute{\mbox{ы}}$х рядов.
Временн$\acute{\mbox{ы}}$е
ряды нижнего уровня этой иерархии\linebreak\vspace*{-12pt}


\begin{center}  %fig3
\vspace*{2pt}
\mbox{%
 \epsfxsize=80mm
 \epsfbox{ste-3.eps}
 }


\vspace*{6pt}

%\noindent
{{\figurename~3}\ \ \small{Неплоская трехуровневая иерархия}}

\end{center}


%\vspace*{6pt}


\addtocounter{figure}{1}





 \noindent
 имеют два индекса, соответствующих
  номеру  ветки и~коду груза: $\x(i, j)$, $i \hm= 1,
\ldots, n$, $j \hm= 1, \ldots, m$,
 где $n$~--- число веток,
а~$m$~--- количество грузов.
 На среднем уровне~--- два семейства
временн$\acute{\mbox{ы}}$х
 рядов. Временн$\acute{\mbox{ы}}$е ряды, соответствующие
суммарному отправлению всех грузов с~каждой ветки,
 обозначаются $\x(i, :)$,
$i = 1, \ldots, n$. Ряды среднего уровня, соответствующие
суммарному отправлению со всех веток каждого из грузов,
обозначаются $\x(:, j)$, $j\hm = 1, \ldots, m$. Временной ряд
верхнего уровня обозначается $\x(:, :)$.

Условие согласованности
для трехуровневой иерархии задается равенствами ($t \hm= 1, \ldots, T$):
\begin{equation}
\left.
\begin{array}{rl}
    x_t(:, :) &= \displaystyle\sum\limits_{i = 1}^n x_t(i, :)\,;\\[6pt]
    x_t(:, :) &= \displaystyle\sum\limits_{j = 1}^m x_t(:, j)\,;\\[6pt]
    x_t(i, :) &= \displaystyle\sum\limits_{j = 1}^m x_t(i, j)\,, \enskip i = 1, \ldots n\,;  \\[6pt]
    x_t(:, j) &= \displaystyle\sum\limits_{i = 1}^n x_t(i, j)\,, \enskip j = 1, \ldots m\,.
    \end{array}
    \right\}
    \label{eq:3LevelConstraint}
\end{equation}
Векторная запись срезов иерархии, независимых прогнозов и~согласованных
прогнозов имеет размерность $d \hm= 1 \hm+ n \hm+ m \hm+ nm$ и~выглядит
сле\-ду\-ющим образом:
$$
    \boldsymbol{\chi}_t = \left(%
        \begin{array}{c}
            x_t(:, :) \\
            x_t(1, :) \\
            \vdots \\
            x_t(n, :) \\
            x_t(:, 1) \\
            \vdots \\
            x_t(:, m) \\
            x_t(1, 1) \\
            \vdots \\
            x_t(1, m) \\
            \vdots \\
            x_t(n, 1) \\
            \vdots \\
            x_t(n, m) \\
        \end{array}%
        \right)\,; \quad
    \hchi = \left(%
        \begin{array}{c}
            \hx(:, :) \\[2pt]
            \hx(1, :) \\
            \vdots \\
            \hx(n, :) \\[2pt]
            \hx(:, 1) \\
            \vdots \\
            \hx(:, m) \\[2pt]
            \hx(1, 1) \\
            \vdots \\
            \hx(1, m) \\
            \vdots \\
            \hx(n, 1) \\
            \vdots \\
            \hx(n, m) \\
        \end{array}%
        \right)\,;
        $$

        \noindent
        $$
    \hphi = \left(%
        \begin{array}{c}
            \hy(:, :) \\
            \hy(1, :) \\
            \vdots \\
            \hy(n, :) \\
            \hy(:, 1) \\
            \vdots \\
            \hy(:, m) \\
            \hy(1, 1) \\
            \vdots \\
            \hy(1, m) \\
            \vdots \\
            \hy(n, 1) \\
            \vdots \\
            \hy(n, m) \\
        \end{array}%
        \right).
$$
Матрица связей для условия~(\ref{eq:3LevelConstraint}) имеет размер
$(2 \hm+ n \hm+ m) \times (1\hm + n \hm+ m \hm+ nm)$:
\begin{multline*}
    S =
\left(%
\begin{array}{c|ccc|ccc|ccc}
  -1 & 1 & \cdots & 1 & 0 & \cdots & 0 & 0 & 0 & \cdots\\
  -1 & 0 & \cdots & 0 & 1 & \cdots & 1 & 0 & 0 & \cdots\\
  \hline
  0 & -1 & \ddots & 0 & 0 & \cdots & 0 & 1 & 1 & \cdots \\
  \vdots & \ddots & \ddots & \ddots & \cdots & \cdots & \cdots & \cdots & \cdots & \cdots\\
  0 & 0 & \ddots & -1 & 0 & \cdots & 0 & 0 & 0 & \cdots\\
  \hline
  0 & 0 & \cdots & 0 & -1 & \ddots & 0 & 1 & 0 & \cdots\\
  \vdots & \ddots & \ddots &\ddots  &\ddots  & \ddots & \ddots &   \ddots &\ddots  & \ddots\\
  %\ddots & \ddots & \ddots & \ddots & \cdots &  \ddots&\ddots  &\ddots  &\ddots  & \ddots & \vdots \\
  0 & 0 & \cdots & 0 & 0 & \cdots & -1 & 0 & 0 & \cdots\\
\end{array}\right.\\[6pt]
\left.
\begin{array}{cc|cccc|c|cccc}%
  \cdots & 0 & 0 & 0 & \cdots & 0 & \cdots & 0 & 0 & \cdots & 0 \\
    \cdots & 0 & 0 & 0 & \cdots & 0 & \cdots & 0 & 0 & \cdots & 0 \\
      \cdots& 1 & 0 & 0 & \cdots & 0 & \cdots & 0 & 0 & \cdots & 0 \\
        \cdots & \cdots &\cdots  &\cdots  & \cdots & \cdots &\cdots  &\cdots  & \cdots & \cdots & \vdots \\
          \cdots & 0 & 0 & 0 & \cdots & 0 & \cdots & 1 & 1 & \cdots & 1 \\
  \cdots & 0 & 1 & 0 & \cdots & 0 & \cdots & 1 & 0 & \cdots & 0 \\
    \ddots & \ddots & \ddots & \ddots & \cdots &  \ddots&\ddots  &\ddots  &\ddots  & \ddots & \vdots \\
      \cdots & 1 & 0 & 0 & \cdots & 1 & \cdots & 0 & 0 & \cdots & 1 \\
\end{array}\right).
\end{multline*}



Требуется, чтобы все прогнозы были неотрицательны, поэтому
множество $\mathcal{B}$~(\ref{eq:SetPhysicsConstraint}) задается как
$$
    \mathcal{B} = \{ \boldsymbol{\chi} \in \mathbb{R}^d \mid \boldsymbol{\chi}(i) \in [0, +\infty], i = 1, \ldots, d
    \}.
$$



В качестве функции суммарных потерь~$l_h$~(\ref{eq:HierarchyLoss})
используется квадрат евклидового расстояния.



\paragraph*{Оптимизационная задача.} Задача, решенная для согласования
прогнозов рассматриваемой иерархии, имеет вид:

\columnbreak


\noindent
$$
    \left\{%
\begin{array}{l}
    \| \boldsymbol{\chi} - \hchi \|^2 \rightarrow \min\limits_{\boldsymbol{\chi}}\,,  \\[9pt]
    S \boldsymbol{\chi} = 0\,,  \\[9pt]
    \boldsymbol{\chi} \geq 0\,.
\end{array}%
\right.
$$


\paragraph*{Результаты эксперимента.}
Для прогноза были использованы временн$\acute{\mbox{ы}}$е ряды, описывающие
суммарный вес отправленных грузов разных типов по каждой ветке.
Значения временн$\acute{\mbox{ы}}$х рядов нижнего уровня иерархии были
отнормированы на отрезок $[0;~1]$ по формуле
\begin{equation*}
    x_t^{\mathrm{norm}}(i, j) = \fr{x_t(i, j) - m(i, j)}{M(i, j) - m(i,
    j)}\,,
    \end{equation*}
    где
    $$
    m = \min\limits_{t = 1, \ldots, T} x_t(i, j)\,, \enskip
    M = \max\limits_{t = 1, \ldots, T} x_t(i, j)\,.
$$

 Прогнозы были
построены и~согласованы для~100~последних точек истории. Для
каждого отсчета времени строился вектор независимых прогнозов~$\hchi$ и~вектор согласованных прогнозов~$\hphi$. Для каждого
вектора вычислялось значение функции суммарных потерь, затем
вычислялись потери игрока~1 в~игре~(\ref{eq:game}), равные
разности суммарных потерь при использовании согласованных
прогнозов и~суммарных потерь при использовании независимых
прогнозов:
$$
    L_t = l_h(\hphi, \boldsymbol{\chi}_t) - l_h(\hchi, \boldsymbol{\chi}_t),
    \quad t = T -
    100 + 1, \ldots, T\,.
$$
Значения этой величины изображены на рис.~\ref{fig:LossDifference}.
% По оси ординат во всех случаях отложены
%номера контрольных точек. По оси абсцисс на рис.~\ref{fig:LossDifference},\,\textit{а} отложены номера железнодорожных веток,
%на рис.~\ref{fig:LossDifference},\,\textit{б}~--- индексы грузов, на рис.~\ref{fig:LossDifference},\,\textit{в} по оси абсцисс только один отсчет,
%соответствующий трехуровневой иерархии.
Теоретические выкладки
подтверждаются на практике. Для плоских двухуровневых иерархий
есть случаи, когда суммарные потери при переходе к~согласованным
прогнозам не изменяются, и~случаи, когда суммарные потери
уменьшаются. Для неплоской трехуровневой иерархии суммарные потери
во всех контрольных точках уменьшаются.


\section{Заключение}

Предложен алгоритм GTOp для согласования прогнозов иерархических
временн$\acute{\mbox{ы}}$х рядов. Алгоритм не требует оценки погрешностей
независимых прогнозов и~не требует их несмещенности. Для любого
набора независимых прогнозов временн$\acute{\mbox{ы}}$х рядов алгоритм согласования
не ухудшает качество прогнозирования. Возможна работа с~иерархическими структурами любой сложности. Все свойства алгоритма
согласования прогнозов GTOp под\-тверж\-да\-ются на практике.

\pagebreak

\end{multicols}

        \begin{figure}%fig4
    \vspace*{1pt}
 \begin{center}
 \mbox{%
 \epsfxsize=163.264mm
 \epsfbox{ste-4.eps}
 }
\end{center}
 \vspace*{-9pt}
    \Caption{Изменение суммарных потерь $L_t$:
    (\textit{а})~для каждой ветки;
(\textit{б})~для каждого груза;
(\textit{в})~вся иерархия}
    \label{fig:LossDifference}
%   \vspace*{-9pt}
\end{figure}

\begin{multicols}{2}


{\small\frenchspacing
 {%\baselineskip=10.8pt
 \addcontentsline{toc}{section}{References}
 \begin{thebibliography}{99}



\bibitem{tokmakova2012hyper}
\Au{Токмакова А.\,А., Стрижов В.\,В.} Оценивание
    гиперпараметров линейных и~регрессионных моделей при отборе
    шумовых и~коррелирующих признаков~// Информатика и~её
    применения, 2012. Т.~6. Вып.~4. С.~66--75.



    \bibitem{vasilyev2014using}
\Au{Васильев Н.\,С.} Использование принципа равновесия для
    управления маршрутизацией в~транспортных сетях~// Информатика
    и~её применения, 2014. Т.~8. Вып.~1. С.~28--35.



\bibitem{hong2014global}
\Au{Hong T., Pinson P., Fan~S.} Global energy forecasting competition 2012~// Int.
    J.~Forecasting, 2014. Vol.~30. No.\,2. P.~357--363.

\bibitem{kaggle}
    Kaggle. {\sf https://www.kaggle.com}.

\bibitem{hyndman2011optimal}
\Au{Hyndman R.\,J., Ahmed R.\,A., Athanasopoulos~G., Shang~H.\,L.} Optimal
    combination forecasts for hierarchical time series~// Comput.
    Stat. Data Anal., 2011. Vol.~55. No.\,9. P.~2579--2589.

\bibitem{kuznetsov2011smoothing}
\Au{Кузнецов М.\,П., Мафусалов А.\,А., Животовский~Н.\,К., Зайцев~Е.\,Ю.,
    Сунгуров~Д.\,С.} Сглаживающие алгоритмы прогнозирования~// Машинное
    обучение и~анализ данных, 2011. Т.~1. Вып.~1. С.~104--112.

\bibitem{stenina2014reconciliation}
\Au{Стенина М.\,М., Стрижов В.\,В.} Согласование агрегированных и~    детализированных прогнозов при решении задач непараметрического
    прогнозирования~//
    Системы и~средства информатики, 2014. Т.~24. Вып.~2. С.~21--34.

\bibitem{grunfeld1960aggregation} %8
\Au{Grunfeld Y., Griliches Z.} Is aggregation necessarily bad?~// Rev. Econ.
Stat., 1960. Vol.~42. No.\,1. P.~1--13.

\bibitem{orcutt1968data} %9
\Au{Orcutt G.\,H., Watts H.\,W., Edwards~J.\,B.}
Data aggregation and information loss~// Am. Econ. Rev., 1968. Vol.~58. No.\,4. P.~773--787.

\bibitem{edwards1969should} %10
\Au{Edwards J.\,B., Orcutt G.\,H.} Should aggregation prior to estimation be the rule?~//
Rev. Econ. Stat., 1969. Vol.~51. No.\,4. P.~409--420.

\bibitem{shlifer1979aggregation} %11
\Au{Shlifer E., Wolff R.\,W.} Aggregation and proration in forecasting~// Manage.
    Sci., 1979. Vol.~25. No.\,6. P.~594--603.

    \bibitem{fogarty1991production} %12
\Au{Fogarty D.\,W., Blackstone J.\,H., Hoffman~T.\,R.} Production and inventory
    management.~--- 2nd ed.~--- Cincinnati, OH, USA: South-Western Publication Co.,
    1990. 880~p.

\bibitem{narasimhan1995production} %13
\Au{Narasimhan S.\,L., McLeavey D.\,W., Billington~P.\,J.} Production planning and inventory
    control.~--- 2nd ed.~--- Englewood Cliffs, NJ, USA: Prentice Hall,
    1995. 716~p.

\bibitem{schwarzkopf1988top} %14
\Au{Schwarzkopf A.\,B., Tersine R.\,J., Morris~J.\,S.} Top-down versus bottom-up forecasting
    strategies~// Int. J.~Prod. Res., 1998.
    Vol.~26. No.\,11. P.~1833--1843.

    \bibitem{fliedner1999investigation} %15
\Au{Fliedner G.} An investigation of aggregate variable time series forecast strategies with
    specific subaggregate time series statistical correlation~//
    Comput. Oper. Res., 1999. Vol.~26. No.\,10--11.
    P.~1133--1149.

\bibitem{vanerven:hal-00920559} %16
\Au{Van Erven T., Cugliari J.}  Game-theoretically optimal reconciliation of contemporaneous
    hierarchical time series forecasts. 2013.
    {\sf https://hal.inria.fr/hal-00920559}.



\bibitem{petrosyan1998theory} %17
\Au{Петросян Л.\,А., Зенкевич Н.\,А., Семина~Е.\,А.}
Теория игр.~--- М.: Университет, 1998. 301~с.

\bibitem{menshikov2010lections} %18
\Au{Меньшиков И.\,С.} Лекции по теории игр и~экономическому моделированию.~--- 2-е
    изд., испр. и~доп.~---
    М.: Контакт Плюс, 2010. 336~с.

\bibitem{cesa2006prediction} %19
\Au{Cesa-Bianchi N., Lugosi G.} Prediction, learning, and games.~--- Cambridge: Cambridge
    University Press, 2006. Vol.~1. 403~p.

\bibitem{boyd2009convex} %20
\Au{Boyd S., Vandenberghe L.} Convex optimization.~---
Cambridge: Cambridge University Press, 2009. 732~p.

\bibitem{bregman1967relaxation} %21
\Au{Bregman L.\,M.} The relaxation method of finding the common point of convex sets and its
    application to the solution of problems in convex programming~//
    USSR Comput. Math. Math. Phys.,
    1967. Vol.~7. No.\,3. P.~200--217.

\bibitem{medvednikova2012nonparametric} %22
\Au{Вальков А.\,С., Кожанов Е.\,М., Медведникова~М.\,М.,
    Хусаинов~Ф.\,И.} Непараметрическое прогнозирование загруженности системы железнодорожных
    узлов по историческим данным~// Машинное обучение и~анализ данных, 2012. Т.~1. Вып.~4. С.~448--465.
     \end{thebibliography}

 }
 }

\end{multicols}

\vspace*{-3pt}

\hfill{\small\textit{Поступила в~редакцию 27.10.14}}

%\newpage

\vspace*{12pt}

\hrule

\vspace*{2pt}

\hrule

%\vspace*{12pt}

\def\tit{FORECASTS RECONCILIATION FOR HIERARCHICAL TIME SERIES FORECASTING PROBLEM}

\def\titkol{Forecasts reconciliation for hierarchical time series forecasting problem}

\def\aut{M.\,M.~Stenina$^1$ and V.\,V.~Strijov$^2$}

\def\autkol{M.\,M.~Stenina and V.\,V.~Strijov}

\titel{\tit}{\aut}{\autkol}{\titkol}

\index{Stenina M.\,M.}
\index{Strijov V.\,V.}

\vspace*{-9pt}


\noindent
$^1$Moscow Institute of Physics and Technology, 9 Institutskiy Per.,
Dolgoprudny, Moscow Region 141700, Russian\linebreak
$\hphantom{^1}$Federation

\noindent
$^2$Dorodnicyn Computing Center, Russian Academy of Sciences, 40~Vavilov Str.,
Moscow 119333, Russian\linebreak
$\hphantom{^1}$Federation


\def\leftfootline{\small{\textbf{\thepage}
\hfill INFORMATIKA I EE PRIMENENIYA~--- INFORMATICS AND
APPLICATIONS\ \ \ 2015\ \ \ volume~9\ \ \ issue\ 2}
}%
 \def\rightfootline{\small{INFORMATIKA I EE PRIMENENIYA~---
INFORMATICS AND APPLICATIONS\ \ \ 2015\ \ \ volume~9\ \ \ issue\ 2
\hfill \textbf{\thepage}}}

\vspace*{3pt}


\Abste{The hierarchical time series forecasting problem is
    researched. Time series forecasts must satisfy the physical
    constraints and the hierarchical structure. In this paper,
    a~new algorithm for hierarchical time series
    forecasts reconciliation is proposed. The algorithm is called GTOp (Game-theoretically
    optimal reconciliation). It guarantees that the quality of reconciled
    forecasts is not worse than the quality of self-dependent forecasts.
    This approach is based on Nash equilibrium search for the
    antagonistic game and turns the forecasts reconciliation problem into
    the optimization problem with equality and inequality
    constraints. It is proved that the Nash
    equilibrium in pure strategies exists in the game if some assumptions
    about the hierarchical structure, the physical
    constraints, and the loss function are satisfied. The algorithm
    performance is demonstrated for different types of
    hierarchical structures of time series.}

\KWE{hierarchical time series; reconciliation of time
    series forecasts; antagonistic game; Nash equilibrium}


\DOI{10.14357/19922264150209}

\Ack
\noindent
The research was financially supported by the Russian Foundation for
Basic Research (project~13-07-13139).



%\vspace*{3pt}

  \begin{multicols}{2}

\renewcommand{\bibname}{\protect\rmfamily References}
%\renewcommand{\bibname}{\large\protect\rm References}



{\small\frenchspacing
 {%\baselineskip=10.8pt
 \addcontentsline{toc}{section}{References}
 \begin{thebibliography}{99}

 \bibitem{tokmakova2012hyper-1}
\Aue{Tokmakova, A.\,A., and V.\,V.~Strizhov}.
2012. Otsenivanie giperparametrov lineynykh i~regressionnykh
    modeley pri otbore shumovykh i~korreliruyushchikh priznakov [Estimation of linear model
    hyperparameters for noise or correlated feature selection problem].
    \textit{Informatika i~ee Primeneniya}~--- \textit{Inform. Appl.} 6(4):66--75.


\bibitem{vasilyev2014using-1}
    \Aue{Vasil'ev, N.\,S.} 2014. Ispol'zovanie printsipa ravnovesiya dlya upravleniya marshrutizatsiey v
    transportnykh setyakh
    [Equilibrium principle application to routing control in packet
    data transmission networks]. \textit{Informatika i~ee Primeneniya}~---
    \textit{Inform. Appl.} 8(1):28--35.

\bibitem{hong2014global-1}
\Aue{Hong, T., P.~Pinson, and S.~Fan}.
2014. Global energy forecasting competition 2012.  \textit{Int.
    J.~Forecasting} 30(2):357--363.

\bibitem{kaggle-1}
    Kaggle. Available at: {\sf https://www.kaggle.com} (accessed May~20, 2015).

\bibitem{hyndman2011optimal-1}
\Aue{Hyndman, R.\,J., R.\,A.~Ahmed, G.~Athanasopoulos, and H.\,L.~Shang}.
2011. Optimal
    combination forecasts for hierarchical time series. \textit{Comput.
    Stat. Data Anal}. 55(9):2579--2589.

\bibitem{kuznetsov2011smoothing-1} %6
\Aue{Kuznetsov, M.\,P., A.\,A.~Mafusalov, N.\,K.~Zhivotovskiy, E.\,Yu.~Zaytsev,
and D.\,S.~Sungurov}. 2011. Sglazhivayushchie algoritmy prognozirovaniya
[Smoothing forecast algorithms]. \textit{Mashinnoe obuchenie i~analiz dannykh}
[J.~Machine Learning Data Anal.] 1(1):104--112.

\bibitem{stenina2014reconciliation-1}
\Aue{Stenina, M.\,M., and V.\,V.~Strizhov}. 2014. Soglasovanie agregirovannykh i~detalizirovannykh
prognozov pri re\-she\-nii zadach neparametricheskogo prognozirovaniya [Reconciliation
    of aggregated and disaggregated time series forecasts in nonparametric
    forecasting problem].
\textit{Sistemy i~Sredstva Informatiki}~--- \textit{Systems and Means of
    Informatics} 24(2):21--34.

\bibitem{grunfeld1960aggregation-1} %8
\Aue{Grunfeld, Y., and Z.~Griliches}.
1960. Is aggregation necessarily bad? \textit{Rev. Econ.
Stat.} 42(1):1--13.



\bibitem{orcutt1968data-1} %9
\Aue{Orcutt, G.\,H., H.\,W.~Watts,  and J.\,B.~Edwards}.
1968. Data aggregation and information loss. \textit{Am. Econ. Rev.} 58(4):773--787.

\bibitem{edwards1969should-1} %10
\Aue{Edwards, J.\,B., and G.\,H.~Orcutt}. 1969.
Should aggregation prior to estimation be the rule? \textit{Rev. Econ. Stat.} 51(4):409--420.

\bibitem{shlifer1979aggregation-1} %11
   \Aue{Shlifer, E., and R.\,W.~Wolff}. 1979. Aggregation and proration
    in forecasting. \textit{Manage.
    Sci.} 25(6):594--603.

    \bibitem{fogarty1991production-1} %12
\Aue{Fogarty, D.\,W., J.\,H.~Blackstone, and T.\,R.~Hoffman}. 1990.
\textit{Production and inventory
    management}. 2nd ed. Cincinnati, OH: South-Western Publication Co. 880~p.

\bibitem{narasimhan1995production-1} %13
\Aue{Narasimhan, S.\,L., D.\,W.~McLeavey, and P.\,J.~Billington}.
1995. \textit{Production planning and inventory
    control}. 2nd ed. Englewood Cliffs, NJ: Prentice Hall. 716~p.

\bibitem{schwarzkopf1988top-1} %14
\Aue{Schwarzkopf, A.\,B., R.\,J.~Tersine, and J.\,S.~Morris}.
1998. Top-down versus bottom-up forecasting
    strategies. \textit{Int. J.~Prod. Res}.
    26(11):1833--1843.

    \bibitem{fliedner1999investigation-1} %15
\Aue{Fliedner, G.} 1999. An investigation of aggregate variable time series forecast strategies with
    specifc subaggregate time series statistical correlation.
    \textit{Comput. Oper. Res.} 26(10--11):1133--1149.

\bibitem{vanerven:hal-00920559-1} %16
    \Aue{Van Erven, T., and J.~Cugliari}.
    2013. Game-theoretically optimal reconciliation of contemporaneous
    hierarchical time series forecasts.
    Available at: {\sf https://hal.inria.fr/hal-00920559}
    (accessed May~20, 2015).

    \bibitem{petrosyan1998theory-1} %17
    \Aue{Petrosyan, L.\,A., N.\,A.~Zenkevich, and E.\,A.~Semina}.
    1998. \textit{Teoriya igr} [Games theory]. Moscow: Knizhnyy Dom
    Universitet. 301~p.

\bibitem{menshikov2010lections-1} %18
\Aue{Men'shikov, I.\,S.} 2010. \textit{Lektsii po teorii igr
i~eko\-no\-mi\-che\-sko\-mu modelirovaniyu}     [Games
    theory and economics modeling lectures]. 2nd~ed.
    Moscow: OOO Kontakt Plyus. 336~p.



\bibitem{cesa2006prediction-1} %19
\Aue{Cesa-Bianchi, N., and G.~Lugosi.} 2006.
\textit{Prediction, learning, and games.} Cambridge: Cambridge
    University Press. Vol.~1. 403~p.

\bibitem{boyd2009convex-1}
\Aue{Boyd, S., and L.~Vandenberghe}. 2009. \textit{Convex optimization}.
Cambridge: Cambridge University Press. 732~p.

\bibitem{bregman1967relaxation-1}
\Aue{Bregman, L.\,M.} 1967. The relaxation method of finding the common point
of convex sets and its application to the solution of problems in convex
programming. \textit{USSR Comput. Math. Math. Phys.} 7(3):200--217.

\bibitem{medvednikova2012nonparametric-1}
\Aue{Val'kov, A.\,S., E.\,M.~Kozhanov, M.\,M.~Medvednikova, and F.\,I.~Khusainov}.
2012.
    Neparametricheskoe prognozirovanie zagruzhennosti sistemy
    zheleznodorozhnykh uzlov po istoricheskim dannym [Nonparametric
    forecasting of railroad stations occupancy according to historical
    data]. \textit{Mashinnoe Obuchenie i~Analiz Dannykh} [J.~Machine
    Learning Data Anal.] 1(4):448--465.

    \end{thebibliography}

 }
 }

\end{multicols}

\vspace*{-3pt}

\hfill{\small\textit{Received October 27, 2014}}

%\vspace*{-18pt}

    \Contr


    \noindent
    \textbf{Stenina Mariya M.} (b.\ 1991)~---
    student, Moscow Institute of Physics and Technology, 9 Institutskiy Per.,
    Dolgoprudny, Moscow Region 141700, Russian Federation; mmedvednikova@gmail.com

    \vspace*{3pt}

    \noindent
    \textbf{Strijov Vadim V.} (b.\ 1967)~--- Doctor of Science
    in physics and mathematics, leading scientist,
    Dorodnicyn Computing Center, Russian Academy of Sciences, 40~Vavilov Str.,
    Moscow 119333, Russian Federation; strijov@gmail.com






\label{end\stat}


\renewcommand{\bibname}{\protect\rm Литература} %9
\def\stat{shestakov}

\def\tit{ОБРАЩЕНИЕ ОДНОРОДНЫХ ОПЕРАТОРОВ С~ПОМОЩЬЮ
СТАБИЛИЗИРОВАННОЙ ЖЕСТКОЙ ПОРОГОВОЙ ОБРАБОТКИ
ПРИ~НЕИЗВЕСТНОЙ ДИСПЕРСИИ ШУМА$^*$}

\def\titkol{Обращение однородных операторов с~помощью
стабилизированной жесткой пороговой обработки}
%при~неизвестной дисперсии шума}

\def\aut{О.\,В.~Шестаков$^1$}

\def\autkol{О.\,В.~Шестаков}

\titel{\tit}{\aut}{\autkol}{\titkol}

\index{Шестаков О.\,В.}
\index{Shestakov O.\,V.}


{\renewcommand{\thefootnote}{\fnsymbol{footnote}} \footnotetext[1]
{Работа выполнена при частичной финансовой поддержке РФФИ (проект 19-07-00352).}}


\renewcommand{\thefootnote}{\arabic{footnote}}
\footnotetext[1]{Московский государственный университет им.\ М.\,В.~Ломоносова, 
кафедра математической статистики факультета вычислительной математики и~кибернетики; 
Институт проб\-лем информатики Федерального исследовательского центра 
<<Информатика и~управ\-ле\-ние>> Российской академии наук, \mbox{oshestakov@cs.msu.su}}


\vspace*{-6pt}


\Abst{При обращении линейных однородных операторов обычно необходимо использовать 
методы регуляризации, поскольку наблюдаемые данные, как правило, зашумлены. 
Для подавления шума часто используется пороговая обработка 
вейвлет-ко\-эф\-фи\-ци\-ен\-тов функции наблюдаемого сигнала. 
Пороговая обработка стала популярным инструментом подавления 
шума благодаря своей простоте, вы\-чис\-ли\-тель\-ной эффективности и~воз\-мож\-ности 
адаптации к~функциям, имеющим на разных участках разную степень регулярности. 
Рассматривается предложенный недавно стабилизированный метод жесткой 
пороговой обработки, в~котором устранены основные недостатки мягкой и~жесткой 
пороговой обработки, и~исследуются статистические свойства этого метода. 
В~модели данных с~аддитивным гауссовским шумом с~неизвестной дисперсией 
проведен анализ несмещенной оценки среднеквадратичного риска и~показано, 
что при определенных условиях данная оценка является асимптотически нормальной, 
при этом дисперсия предельного распределения зависит от способа оценивания 
дисперсии шума.}

\KW{вейвлеты; пороговая обработка; несмещенная оценка риска; 
асимптотическая нормальность; сильная состоятельность}

\DOI{10.14357/19922264190107}
  
%\vspace*{4pt}


\vskip 10pt plus 9pt minus 6pt

\thispagestyle{headings}

\begin{multicols}{2}

\label{st\stat}

\section{Введение}

В медицинских, физических, астрономических и~других научных проблемах часто 
возникает задача получить представление об объекте, который описывается 
некоторой функцией~$f$, имея возможность наблюдать только функцию~$Kf$, где~$K$~--- 
некоторый линейный оператор. При этом часто нельзя просто применить 
к~наблюдаемым данным обратный оператор~$K^{-1}$, поскольку эти данные, как правило, 
содержат шум и~задача обращения оператора~$K$ некорректно поставлена. 
К~тому же обычно дис\-пер\-сия шума неизвестна и~ее необходимо оценивать 
по наблюдаемым данным. 

Одним из популярных инструментов при регуляризации 
процедуры обращения служит вейв\-лет-раз\-ло\-же\-ние с~последующей 
пороговой обработкой вейв\-лет-ко\-эф\-фи\-ци\-ен\-тов. Наиболее распростра\-нен\-ные 
виды пороговой обработки~--- жесткая и~мягкая. В~работе~\cite{HL10} 
был предложен метод стабилизированной жесткой пороговой обработки, который 
объединяет в~себе преимущества этих двух видов. 
В~ситуации, когда дисперсия шума предполагается известной, в~работе~\cite{SH18} 
доказана асимптотическая нормальность оценки среднеквадратичного риска пороговой 
обработки. 

В~данной работе исследуется влияние способов оценивания дисперсии шума 
на характеристики предельного распределения оценки среднеквадратичного риска. 
Для метода мягкой пороговой обработки подобные исследования проводились 
в~работах~\cite{KS11-1, KS11-2}.

\section{Обращение линейных однородных операторов с~помощью вейглет-вейвлет-разложения}

В данной работе рассматривается метод обращения линейных однородных операторов, 
основанный на вейг\-лет-вейв\-лет-раз\-ло\-же\-нии~\cite{AS98}. Линейный оператор~$K$ 
называется однородным, если
$$
K\left[f\left(a\left(x-x_0\right)\right)\right]=a^{-\alpha}(Kf)\left[a\left(x-x_0\right)\right]
$$
для любого $x_0$ и~любого $a\hm>0$. Параметр~$\alpha$ называется показателем 
однородности. Примерами линейных однородных операторов служат оператор 
интегрирования, преобразование Гильберта и~преобразование Абеля.

Относительно наблюдаемой функции~$Kf$ будем предполагать, что она определена на 
конечном отрезке и~равномерно регулярна по Липшицу с~некоторым показателем $\gamma\hm>0$. 
Вейв\-лет-разложение~$Kf$ представляет собой ряд по ортонормированному базису
\begin{equation}
\label{wavelet_decomp}
Kf = \sum\limits_{j,k \in Z} \langle Kf,\psi_{j,k} \rangle \psi_{j,k}\,,
\end{equation}
где $\psi(t)$~--- некоторая материнская вейв\-лет-функ\-ция, 
а~$\psi_{j,k}(t) \hm= 2^{j/2}\psi(2^jt \hm- k)$. Индекс~$j$ в~(\ref{wavelet_decomp}) 
называется масштабом, а~индекс~$k$~--- сдвигом. Если вейв\-лет-функ\-ция 
обладает определенными свойствами регулярности~\cite{Mal99}, 
то для коэффициентов разложения в~(\ref{wavelet_decomp}) справедливо
\begin{equation}
\label{wavelet_decay}
\abs{\langle Kf, \psi_{j,k} \rangle} \leqslant \fr{C_f}
{2^{j \left( \gamma + 1/2 \right)}}\,,
\end{equation}
где $C_f$~--- некоторая положительная константа.

Поскольку оператор~$K$ линеен и~однороден, существуют такие функции~$u_{j,k}$, 
что $\langle f,u_{j,k}\rangle\hm=\langle Kf,\psi_{j,k}\rangle$. При этом функция~$f$ 
представляется в~виде ряда
\begin{equation}
\label{VWD}
f = \sum\limits_{j,k \in Z}\beta_{j,k}\langle Kf,\psi_{j,k}\rangle u_{j,k},
\end{equation}
где $u_{j,k} = K^{-1}\psi_{j,k}/\beta_{j,k}$, $\beta_{j,k}\hm=2^{\alpha j}\beta_{00}$, 
$\beta_{00} \hm= \norm{K^{-1}\psi}$ (функции~$u_{j,k}$, как и~$\psi_{j,k}$, 
представляют собой сдвиги и~растяжения одной материнской функции~$u$ и~называются 
вейглетами). При соответствующем выборе~$\psi(t)$ последовательность~$\{u_{j,k}\}$ 
образует устойчивый базис~\cite{L97}. Формула~(\ref{VWD}) и~есть основа метода 
вейг\-лет-вейв\-лет-раз\-ло\-же\-ния.

\section*{Пороговая обработка эмпирических коэффициентов}

При фактических измерениях значения функции сигнала регистрируются 
в~дискретных отсчетах, при этом такие значения, как правило, зашумлены. 
Рассмотрим сле\-ду\-ющую модель данных \mbox{с~шумом}:
\begin{equation*}
%\label{Data_Model}
X_i = (Kf)_i + \epsilon_i\,, \enskip i = 1, \dots, 2^J\,, %\notag
\end{equation*}
где $2^J$~--- число отсчетов; $(Kf)_i$~--- незашумленные значения функции сигнала; 
$\epsilon_i$~--- независимые нормально распределенные случайные величины с~нулевым 
средним и~дисперсией~$\sigma^2$.
После применения дискретного вейв\-лет-пре\-об\-ра\-зо\-ва\-ния 
получается следующая модель зашумленных вейв\-лет-ко\-эф\-фи\-ци\-ен\-тов:
\begin{equation*}
Y_{j,k}=\mu_{j,k}+\epsilon^W_{j,k},\enskip 
j=0,\ldots,J-1,\ k=0,\ldots,2^{j}-1\,,
\end{equation*}
где $\epsilon^W_{j,k}$ независимы и~распределены так же, как и~$\epsilon_i$, 
а~$\mu_{j,k}\hm= 2^{J/2}\langle Kf,\psi_{j,k}\rangle$~\cite{Mal99}.

Для подавления шума и~построения оценки функции сигнала к~коэффициентам~$Y_{j,k}$ 
обычно применяется функция жесткой пороговой обработки 
$\rho_{H}(y,T)\hm=x\textbf{I}(\abs{y}>T)$ или мягкой пороговой 
обработки $\rho_{S}(y,T)\hm=\textbf{sgn}(x)\left(\abs{y}-T\right)_{+}$ 
с~порогом~$T$. При таком подходе обнуляются коэффициенты, абсолютная величина 
которых ниже порога, так как в~силу~(\ref{wavelet_decay}) основная часть
 полезного сигнала содержится в~относительно небольшом числе больших по 
 модулю коэффициентов.

Каждому из этих видов пороговой обработки присущи свои недостатки. 
Жесткая пороговая функция разрывна, и~это приводит к~отсутствию устойчивости 
при выборе порога~\cite{B96} и~невозможности построения несмещенной оценки 
среднеквадратичного риска~\cite{J01}. При мягкой пороговой обработке в~оценке 
функции появляется дополнительное смещение. Чтобы частично избежать этих недостатков, 
в~работе~\cite{HL10} был предложен новый вид пороговой обработки, представляющий 
собой сглаженный (стабилизированный) аналог жесткой пороговой обработки. 
В~этом методе оценки~$\mu_{j,k}$ вычисляются по формулам:
\begin{equation*}
\widehat{\mu}_{j,k}=\Expect 
\left[\rho_{H}(Y_{j,k}+\lambda\xi_{j,k},T_j)|Y_{j,k}\right], %\notag
\end{equation*}
где случайные величины~$\xi_{j,k}$ имеют стандартное нормальное распределение и~не 
зависят от~$Y_{j,k}$, а~$\lambda\hm>0$~--- 
параметр стабилизации, отвечающий за степень сглаживания. Вычисляя математическое 
ожидание, получаем:
\begin{multline*}
\hspace*{-8.37947pt}\widehat{\mu}_{j,k}=Y_{j,k}\left[\Phi\!\left(-\fr{T_j+Y_{j,k}}
{\lambda}\right)+1-\Phi\left(\fr{T_j-Y_{j,k}}{\lambda}\right)\!\right]+{}\\
{}+
\lambda\left[\phi\left(\fr{T_j-Y_{j,k}}{\lambda}\right)-
\phi\left(\fr{T_j+Y_{j,k}}{\lambda}\right)\right]. %\notag
\end{multline*}
Достоинством такого метода является бесконечная дифференцируемость~$\widehat{\mu}_{j,k}$ 
по~$Y_{j,k}$, что приводит к~более робастным оценкам~\cite{HL10}. Заметим также, 
что при $\lambda\hm\to0$ получается обычный метод жесткой пороговой обработки. 
В~данной работе параметр~$\lambda$ предполагается фиксированным, а~в~качестве~$T_j$ 
для каждого масштаба~$j$ выбирается порог $T_j\hm=\sigma\sqrt{2\ln 2^j}$. 
Такой порог получил название <<универсальный>>, так как он не зависит 
от наблюдаемых данных. И~при жесткой, и~при мягкой пороговой обработке этот 
порог обеспечивает близость среднеквадратичного риска к~минимальному~\cite{Mal99}.

\section{Несмещенная оценка среднеквадратичного риска}

Среднеквадратичный риск метода пороговой обработки определяется по формуле:
\begin{equation}
\label{Risk}
R_J(\sigma)=\sum\limits_{j=0}^{J-1}\sum\limits_{k=0}^{2^j-1}\beta^2_{j,k}
\Expect\left(\widehat{\mu}_{j,k}(\sigma)-\mu_{j,k}\right)^2.
\end{equation}
В~\cite{HL10} показано, что при стабилизированной жесткой пороговой обработке
\begin{multline*}
\Expect\left(\widehat{\mu}_{j,k}(\sigma)-\mu_{j,k}\right)^2={}\\
{}=
\Expect\left[(Y_{j,k}-\widehat{\mu}_{j,k}(\sigma))^2+
2\sigma^2\fr{\partial}{\partial Y_{j,k}}\,\widehat{\mu}_{j,k}(\sigma)\right]-
\sigma^2, %\notag
\end{multline*}
где
\begin{multline*}
\fr{\partial}{\partial Y_{j,k}}\widehat{\mu}_{j,k}(\sigma)={}\\
{}=\Phi\left(-\fr{T_j+Y_{j,k}}{\lambda}\right)+1-
\Phi\left(\fr{T_j-Y_{j,k}}{\lambda}\right)+{}\\
{}+
\fr{T_j}{\lambda}\left[\phi\left(\fr{T_j-Y_{j,k}}{\lambda}\right)+
\phi\left(\fr{T_j+Y_{j,k}}{\lambda}\right)\right]. %\notag
\end{multline*}
Таким образом, величина
\begin{multline}
\label{Risk_Estimate}
\widehat{R}_J(\sigma)=\sum\limits_{j=0}^{J-1}\sum\limits_{k=0}^{2^j-1}
\beta^2_{j,k}
\Bigg[
\left(
Y_{j,k}-
\widehat{\mu}_{j,k}(\sigma)\right)^2+{}\\
{}+2\sigma^2\fr{\partial}{\partial Y_{j,k}}\,\widehat{\mu}_{j,k}(\sigma)-
\sigma^2
\Bigg]
\end{multline}
является несмещенной оценкой~$R_J$, не зависящей от ненаблюдаемых значений~$\mu_{j,k}$.

В работе~\cite{SH18} доказано следующее утверждение, устанавливающее 
асимптотическую нормальность оценки~(\ref{Risk_Estimate}) и~позволяющее строить 
асимптотические доверительные интервалы для риска~(\ref{Risk}).

\smallskip

\noindent
\textbf{Теорема 1.} 
\textit{Пусть $K$~--- линейный однородный оператор с~показателем 
однородности $\alpha\hm>0$, а~$Kf$ задана на конечном отрезке и~равномерно 
регулярна по Липшицу с~показателем $\gamma\hm>0$. Тогда}
\begin{equation*}
%\label{Normality}
{\sf P}\left(\fr{\widehat{R}_J(\sigma)-
R_J(\sigma)}{D_J}<x\right)\Rightarrow\Phi(x)\,, %\notag
\end{equation*}
\textit{где}
$$
D^2_J=\fr{2\sigma^4\beta_{0,0}^4}{2^{4\alpha+1}-1}2^{(4\alpha+1)J}\,.
$$

\section{Виды оценок дисперсии шума}

Как правило, дисперсия~$\sigma^2$ неизвестна и~вместо ее точного значения 
необходимо использовать некоторую оценку~$\hat{\sigma}^2$, которая обычно 
строится по половине всех вейв\-лет-ко\-эф\-фи\-ци\-ен\-тов для $j\hm=J\hm-1$, 
так как в~силу~(\ref{wavelet_decay}) эти коэффициенты фактически содержат только шум. 
При этом порог вычисляется по формуле $\hat{T}_j\hm=\hat{\sigma}\sqrt{2\ln 2^j}$.

В качестве оценки~$\sigma^2$ (или $\sigma$) в~данной работе 
рассматривается выборочная дисперсия
\begin{equation}
\label{SampleVarianceDef}
\widehat{\sigma}_S^2=\fr{1}{2^{J-1}}
\sum\limits_{k=0}^{2^{J-1}-1}Y_{J-1,k}^2-\overline{Y}^2,
\end{equation}
где
\begin{equation*}
\overline{Y}=\fr{1}{2^{J-1}}\sum\limits_{k=0}^{2^{J-1}-1}Y_{J-1,k}\,,
\end{equation*}
а также соответствующим образом нормированный выборочный интерквартильный 
размах~$\widehat{\sigma}_{R}$ и~выборочное абсолютное медианное 
отклонение~$\widehat{\sigma}_{M}$, которые определяются сле\-ду\-ющим образом:
\begin{align}
\widehat{\sigma}_{R}&=\fr{Y_{(J-1,3/4)}-Y_{(J-1,1/4)}}{2\xi_{3/4}}\,;
\label{IQR_Definition}
\\
\widehat{\sigma}_{M}&=\fr{\mathop{\mbox{med}}\limits_{0\leqslant k\leqslant 2^{J-1}-1}|Y_{J-1,k}-\mathop{\mbox{med}}\limits_{0\leqslant l\leqslant 2^{J-1}-1} Y_{J-1,l}|}{\xi_{3/4}}\,.
\label{MAD_Definition}
\end{align}
Здесь $Y_{(J-1,1/4)}$ и~$Y_{(J-1,3/4)}$~--- выборочные квантили порядка~$1/4$ и~$3/4$, 
построенные по выборке из половины всех вейв\-лет-ко\-эф\-фи\-ци\-ен\-тов при 
$j\hm=J\hm-1$; $\xi_{3/4}$~--- теоретическая квантиль порядка~$3/4$ 
стандартного нормального распределения ($\xi_{3/4}\hm\approx0,6745$); $\mbox{med}$ 
обозначает выборочную медиану.

Выборочная дисперсия служит самой популярной оценкой величины~$\sigma^2$, и~в~случае 
отсутствия выбросов она наиболее предпочтительна. Однако в~случае, когда 
оценка дисперсии строится по выборке сигнала, естественно ожидать, 
что выборка не будет однородной. Преимущество использования последних 
двух оценок заключается в~их ро\-баст\-ности, т.\,е.\ нечувствительности к~выбросам.

\section{Предельная дисперсия оценки среднеквадратичного риска}

Способ оценивания дисперсии шума влияет на вид предельной дисперсии 
оценки среднеквадратичного риска. Подобный эффект наблюдается и~при 
мягкой пороговой обработке~[4].

\noindent
\textbf{Теорема~2.}\ \textit{Пусть $Kf$ задана на конечном отрезке и~равномерно 
регулярна по Липшицу с~показателем $\gamma\hm>1/4$, а оценка дисперсии 
шума задана соотношением}~\eqref{SampleVarianceDef}. \textit{Тогда}
\begin{equation}
\label{CLT_Operator_SampleVar_Sigma}
\mathsf{P}\left(\frac{\widehat{R}_J(\widehat{\sigma}_S)-R_J(\sigma)}{D_J}<x\right)
\Rightarrow \Phi_{\Upsilon_1}(x),\notag
\end{equation}
\textit{где $\Phi_{\Upsilon_1}(x)$~--- функция распределения нормального 
закона с~нулевым средним и~дисперсией}
$$
\Upsilon_1^2=\fr{1}{2^{4\alpha+1}}+
\fr{2^{4\alpha+1}-1}{2^{4\alpha+1}\left(2^{2\alpha+1}-1\right)^2}\,.
$$

\noindent
Д\,о\,к\,а\,з\,а\,т\,е\,л\,ь\,с\,т\,в\,о\,.\ \ Обозначим
\begin{multline*}
\widehat{U}_J(\sigma)=\sum\limits_{j=0}^{J-1}\sum\limits_{k=0}^{2^j-1}
\beta^2_{j,k}\Bigg[
\left(Y_{j,k}-\widehat{\mu}_{j,k}(\sigma)\right)^2+{}\\
{}+2\sigma^2\fr{\partial}{\partial Y_{j,k}}\widehat{\mu}_{j,k}(\sigma)\Bigg] %\notag
\end{multline*}
и запишем $\widehat{R}_J(\hat{\sigma}_S)-R_J(\sigma)$ в~виде
\begin{multline*}
%\label{Three_Sums}
\widehat{R}_J(\hat{\sigma}_S)-R_J(\sigma)={}\\
{}=\left[\widehat{U}_J(\hat{\sigma}_S)-\widehat{U}_J(\sigma)\right]+
\left[\widehat{R}_J(\sigma)-R_J(\sigma)\right]+{}\\
{}+
\fr{2^{(2\alpha+1)J}-1}{2^{2\alpha+1}-1}(\sigma^2-\hat{\sigma}^2_S)
\equiv S_1+S_2+S_3\,.
\end{multline*}

Повторяя рассуждения из работ~\cite{KS11-1, KS11-2} и~учитывая, что если $\gamma\hm>1/4$, 
то выполнено $2^{J/2}\overline{Y}^2\stackrel{{\sf P}}{\to} 0$ при 
$J\hm\rightarrow\infty$~\cite{KS11-2}, можно показать, что
\begin{equation*}
{\sf P}\left(\fr{S_2+S_3}{D_J}<x\right)\Rightarrow\Phi_{\Upsilon_1}(x)\,.%\notag
\end{equation*}
% на самом деле с~условием Линдеберга чуть по-другому (без ограниченности слагаемых). Но дисперсия равномерно ограничена -- значит выполнено.

Докажем, что $D_J^{-1}S_1\stackrel{{\sf P}}{\to}0$ при $J\hm\rightarrow\infty$. 
Пусть $C_\delta\hm>0$~--- некоторая константа, а $\delta_J\hm=C_\delta J^{1/2}2^{-J/2}$. 
Запишем
\begin{multline*}
S_1=\mathbf{1}\left(\abs{\sigma^2-\hat{\sigma}^2_S}>\delta_J\right)S_1+{}\\
{}+
\mathbf{1}\left(\abs{\sigma^2-\hat{\sigma}^2_S}\leqslant\delta_J\right)
S_1\equiv S'_1+S''_1. %\notag
\end{multline*}
Для произвольного $\varepsilon\hm>0$
\begin{equation*}
{\sf P}\left(S'_1>\varepsilon\right)\leqslant{\sf P}
\left(\abs{\sigma^2-\hat{\sigma}^2_S}>\delta_J\right). %\notag
\end{equation*}
При выполнении условий теоремы, если константа~$C_\delta$ достаточно велика, 
то найдется константа~$\tilde{C}_\delta>0$ такая, что~\cite{KS11-2}
\begin{equation*}
{\sf P}\left(\abs{\sigma^2-\hat{\sigma}^2_S}>\delta_J\right)
\leqslant\tilde{C}_\delta2^{-J/2}. %\notag
\end{equation*}
%% комментарии по поводу этого неравенства и~загрязнения выборки есть в~диссертации
Следовательно, $S'_1\stackrel{P}{\to}0$ при $J\hm\rightarrow\infty$.

Обозначим слагаемые в~сумме~$S''_1$ через~$F_{j,k}(\hat{\sigma}_S)$. Пусть 
$A_j\hm=\sqrt{A\ln 2^j}$, где $0\hm<A\hm<2(\sigma^2\hm-\delta_J)$. Имеем:

\noindent
\begin{multline*}
\hspace*{-9.9pt}\sum\limits_{j=0}^{J-1}\sum\limits_{k=0}^{2^j-1}F_{j,k}\left(\hat{\sigma}_S\right)=
\sum\limits_{j=0}^{J-1}\sum\limits_{k=0}^{2^j-1}
\mathbf{1}(\abs{Y_{j,k}}\leqslant A_j)F_{j,k}(\hat{\sigma}_S)+{}\\
{}+
\sum\limits_{j=0}^{J-1}\sum\limits_{k=0}^{2^j-1}
\mathbf{1}\left(\abs{Y_{j,k}}>A_j\right)F_{j,k}(\hat{\sigma}_S)
\equiv  W_1+W_2. %\notag
\end{multline*}
Рассмотрим $W_1$. Учитывая определения $\widehat{\mu}_{j,k}(\sigma)$, 
$({\partial}/{\partial Y_{j,k}})\widehat{\mu}_{j,k}(\sigma)$ и~$A_j$, 
можно убедиться, что найдут\-ся константы $C_1\hm>0$ и~$\theta\hm>0$ такие, что
\begin{equation*}
\abs{\mathbf{1}\left(\abs{Y_{j,k}}\leqslant A_J\right)
F_{j,k}(\hat{\sigma}_S)}\leqslant C_1 
J^{5/2}2^{(2\alpha-\theta)j-J/2}\;\;\mbox{п.в.} %\notag
\end{equation*}
% поскольку выполнено \mathbf{1}(\abs{\sigma^2-\hat{\sigma}^2_S}\leqslant\delta_J). В логарифме степень: от Y идет 1, от T идет 1, от \delta_J идет 1/2 но для J, а не для j, поэтому берем для всех J^{5/2}. В степени 2: 2\alpha от \beta{j,k}, \theta из-за выбора A, J/2 от \delta_J
Следовательно, $D_J^{-1}W_1\hm\rightarrow 0$ п.в.\ при $J\hm\rightarrow\infty$.

Далее для слагаемых~$W_2$ имеем:
\begin{multline*}
\left\vert \mathbf{1}\left(
\left\vert Y_{j,k}\right\vert
> A_J\right)F_{j,k}
\left(\hat{\sigma}_S\right)\right\vert
\leqslant{}\\
{}\leqslant C_2 J^{3/2}2^{2\alpha j-J/2} 
\mathbf{1}\left( \left\vert Y_{j,k}\right\vert > A_J\right) 
\left\vert Y_{j,k}\right\vert^2\;\;\mbox{п.в.},
%\notag
\end{multline*}
% поскольку выполнено \mathbf{1}(\abs{\sigma^2-\hat{\sigma}^2_S}\leqslant\delta_J). В логарифме от T идет 1, от \delta_J идет 1/2.
где $C_2>0$~--- некоторая константа. Учитывая распределение~$Y_{j,k}$, 
нетрудно убедиться, что
\begin{equation*}
\Expect\frac{1}{D_J} \sum\limits_{j=0}^{J-1}
\sum\limits_{k=0}^{2^j-1} J^{3/2}2^{2\alpha j-J/2} 
\mathbf{1}\left(\abs{Y_{j,k}}> A_j\right)
\abs{Y_{j,k}}^2\to 0
\end{equation*}
при $J\rightarrow\infty$. %\notag
Следовательно, используя неравенство Маркова, получаем, что
\begin{equation*}
D_J^{-1}W_2\stackrel{{\sf P}}{\to}0\;\;\mbox{при}\;J\rightarrow\infty\,. %\notag
\end{equation*}
Таким образом, $D_J^{-1}S_1\stackrel{{\sf P}}{\to}0$ при $J\hm\rightarrow\infty$.

Теорема доказана.

\smallskip

Рассмотрим теперь ситуацию, когда в~качестве оценки~$\sigma$ используется 
величина~$\widehat{\sigma}_{R}$ или~$\widehat{\sigma}_{M}$. 
В~этом случае повышаются требования к~гладкости функции сигнала.

\smallskip

\noindent
\textbf{Теорема~3.}\
\textit{Пусть~$Kf$ задана на конечном отрезке и~равномерно регулярна по 
Липшицу с~показателем $\gamma\hm>1/2$, а оценка дисперсии шума~$\hat{\sigma}$ 
задана соотношением}~\eqref{IQR_Definition} 
\textit{или соотношением}~\eqref{MAD_Definition}. \textit{Тогда}
\begin{equation*}
\label{CLT_Operator_RobVar_Sigma}
\mathsf{P}\left(\fr{\widehat{R}_J(\widehat{\sigma})-R_J(\sigma)}{D_J}<x\right)
\Rightarrow \Phi_{\Upsilon_2}(x)\,, %\notag
\end{equation*}
где $\Phi_{\Upsilon_2}(x)$~--- функция распределения нормального закона 
с~нулевым средним и~дисперсией
\begin{multline*}
\Upsilon_2^2=1+\fr{2^{4\alpha+1}-1}{4(2^{2\alpha+1}-1)^2
\xi_{3/4}^2(\phi(\xi_{3/4}))^2}-{}\\
{}-
\fr{2^{4\alpha+1}-1 }{2^{2\alpha-1}(2^{2\alpha+1}-1)}\,.
\end{multline*}

\noindent
Д\,о\,к\,а\,з\,а\,т\,е\,л\,ь\,с\,т\,в\,о\,.\ \
Как и~в~предыдущей теореме, запишем
$\widehat{R}_J(\hat{\sigma})\hm-R_J(\sigma)\hm=S_1\hm+S_2\hm+S_3.$
Учитывая,\linebreak\vspace*{-12pt}

\pagebreak

\noindent
 что $\gamma\hm>1/2$, и~поступая, как в~работах~\cite{SH18, KS11-2, SH12}, 
с~использованием разложения Бахадура для выборочных квантилей~\cite{S80} и~выборочного 
абсолютного медианного отклонения~\cite{SM09}, можно показать, что
\begin{equation*}
{\sf P}\left(\fr{S_2+S_3}{D_J}<x\right)\Rightarrow\Phi_{\Upsilon_2}(x)\,. %\notag
\end{equation*}
% на самом деле с~условием Линдеберга чуть по-другому (без ограниченности слагаемых). Но дисперсия равномерно ограничена -- значит выполнено.

Используя экспоненциальные неравенства для выборочных квантилей~\cite{S80} 
и~выборочного абсолютного медианного отклонения~\cite{SM09}, получаем, что при 
выполнении условий теоремы найдется такая константа $C_\delta\hm>0$, что при 
$\delta_J\hm=C_\delta J^{1/2}2^{-J/2}$ для некоторой константы~$\widetilde{C}_\delta>0$ 
выполнено:
\begin{align*}
\mathsf{P}\left(\abs{\widehat{\sigma}_{R}-\sigma}>\delta_J\right)
&\leqslant\widetilde{C}_\delta2^{-J/2}\,;
\\
\mathsf{P}\left(\abs{\widehat{\sigma}_{M}-\sigma}>\delta_J\right)
&\leqslant\widetilde{C}_\delta2^{-J/2}\,. %\notag
\end{align*}
%% комментарии по поводу этого неравенства и~загрязнения выборки есть в~диссертации
Далее, повторяя рассуждения предыдущей теоремы, заключаем, что 
$D_J^{-1}S_1\stackrel{{\sf P}}{\to}0$ при $J\hm\rightarrow\infty$.


Теорема доказана.



{\small\frenchspacing
 {%\baselineskip=10.8pt
 \addcontentsline{toc}{section}{References}
 \begin{thebibliography}{99}

\bibitem{HL10}
\Au{Huang H.-C., Lee~T.\,C.\,M.} 
Stabilized thresholding with generalized sure for image denoising~// 
IEEE 17th  Conference (International) on Image Processing
Proceedings.~--- IEEE, 2010. P.~1881--1884.

\bibitem{SH18}
\Au{Shestakov O.\,V.} 
Nonlinear regularization of inverse problems for linear homogeneous transforms 
by the stabilized hard thresholding~// J.~Math. Sci., 2018. Vol.~234. No.\,6. P.~780--785.

\bibitem{KS11-1}
\Au{Кудрявцев А.\,А., Шестаков~О.\,В.} 
Асимптотика оценки риска при вейг\-лет-вейв\-лет разложении наблюдаемого сигнала~// 
T-Comm~--- телекоммуникации и~транспорт, 2011. №\,2. С.~54--57.

\bibitem{KS11-2}
\Au{Кудрявцев А.\,А., Шестаков~О.\,В.} 
Асимптотическое распределение оценки риска пороговой обработки 
вейг\-лет-ко\-эф\-фи\-ци\-ен\-тов сигнала при неизвестном уровне шума~// 
T-Comm~--- телекоммуникации и~транспорт, 2011. №\,5. С.~24--30.

\bibitem{AS98}
\Au{Abramovich F., Silverman~B.\,W.} 
Wavelet decomposition approaches to statistical inverse problems~// 
Biometrika, 1998. Vol.~85. No.\,1. P. 115--129.

\bibitem{Mal99}
\Au{Mallat S.} A~Wavelet tour of signal processing.~--- 
New York, NY, USA: Academic Press, 1999. 857~p.

\bibitem{L97}
\Au{Lee N.} Wavelet-vaguelette decompositions and homogenous equations.~--- 
West Lafayette, IN, USA: Purdue University, 1997.  PhD Thesis. 103~p.

\bibitem{B96}
\Au{Breiman L.} Heuristics of instability and stabilization in model selection~// 
Ann. Stat., 1996. Vol.~24. No.\,6. P.~2350--2383.

\bibitem{J01}
\Au{Jansen M.} Noise reduction by wavelet thresholding.~--- 
Lecture notes in statistics ser.~--- New York, NY, USA: Springer Verlag,
2001. Vol.~161. 196~p.

\bibitem{SH12}
\Au{Шестаков О.\,В.} О~скорости сходимости оценки риска пороговой обработки 
вейв\-лет-ко\-эф\-фи\-ци\-ен\-тов к~нормальному закону при использовании 
робастных оценок дисперсии~// Информатика и~её применения, 2012. Т.~6. Вып.~2. 
С.~122--128.

\bibitem{S80}
\Au{Serfling R.} Approximation theorems of mathematical statistics.~--- 
New York, NY, USA: John Wiley \& Sons, 1980. 371~p.

\bibitem{SM09}
\Au{Serfling R., Mazumder~S.} 
Exponential probability inequality and convergence results for the median 
absolute deviation and its modifications~// Stat. Probabil. Lett., 2009. 
Vol.~79. No.\,16. P.~1767--1773.
 \end{thebibliography}

 }
 }

\end{multicols}

\vspace*{-3pt}

\hfill{\small\textit{Поступила в~редакцию 14.12.18}}

\vspace*{8pt}

%\pagebreak

%\newpage

%\vspace*{-28pt}

\hrule

\vspace*{2pt}

\hrule

%\vspace*{-2pt}

\def\tit{INVERSION OF~HOMOGENEOUS OPERATORS USING~STABILIZED HARD THRESHOLDING 
WITH~UNKNOWN NOISE VARIANCE}

\def\titkol{Inversion of~homogeneous operators using~stabilized hard thresholding 
with~unknown noise variance}

\def\aut{O.\,V.~Shestakov}

\def\autkol{O.\,V.~Shestakov}

\titel{\tit}{\aut}{\autkol}{\titkol}

\vspace*{-11pt}


\noindent
Department of Mathematical Statistics, Faculty of Computational Mathematics and Cybernetics, M.V. Lomonosov Moscow State University, 1-52 Leninskiye Gory, GSP-1, Moscow 119991, Russian Federation
Institute of Informatics Problems, Federal Research Center 
``Computer Science and Control'' of the Russian Academy of Sciences, 44-2~Vavilov Str., 
Moscow 119333, Russian Federation

\def\leftfootline{\small{\textbf{\thepage}
\hfill INFORMATIKA I EE PRIMENENIYA~--- INFORMATICS AND
APPLICATIONS\ \ \ 2019\ \ \ volume~13\ \ \ issue\ 1}
}%
 \def\rightfootline{\small{INFORMATIKA I EE PRIMENENIYA~---
INFORMATICS AND APPLICATIONS\ \ \ 2019\ \ \ volume~13\ \ \ issue\ 1
\hfill \textbf{\thepage}}}

\vspace*{6pt}



\Abste{When inverting linear homogeneous operators, it is necessary to use 
regularization methods, since observed data are usually noisy. For noise suppression, 
threshold processing of  wavelet coefficients of the observed signal function 
is often used. Threshold processing has become a~popular noise suppression tool 
due to its simplicity, computational efficiency, and ability to adapt to functions 
that have different degrees of regularity at different domains. The paper 
discusses the recently proposed stabilized hard thresholding method that eliminates 
the main
drawbacks of soft and hard thresholding methods and studies statistical 
properties of this method. In the data model\linebreak\vspace*{-12pt}}

\Abstend{with an additive Gaussian noise with 
unknown variance, an unbiased estimate of the mean square risk is analyzed and it 
is shown that under certain conditions, this estimate is asymptotically normal and 
the variance of the limit distribution depends on the type of estimate of noise variance.}


\KWE{wavelets; threshold processing; unbiased risk estimate; asymptotic normality;
strong consistency}




\DOI{10.14357/19922264190107}

%\vspace*{-14pt}

\Ack
\noindent
This research was partly supported by the Russian  
Foundation for Basic Research (project No.\,19-07-00352).




%\vspace*{6pt}

  \begin{multicols}{2}

\renewcommand{\bibname}{\protect\rmfamily References}
%\renewcommand{\bibname}{\large\protect\rm References}

{\small\frenchspacing
 {%\baselineskip=10.8pt
 \addcontentsline{toc}{section}{References}
 \begin{thebibliography}{99}
\bibitem{1-sh-1}
\Aue{Huang, H.-C., and T.\,C.\,M.~Lee.} 2010. 
Stabilized thresholding with generalized sure for image denoising. 
\textit{IEEE 17th Conference (International) on Image Processing}. IEEE. 1881--1884.

 

\bibitem{2-sh-1}
\Aue{Shestakov, O.\,V.} 2018. 
Nonlinear regularization of inverse problems for linear homogeneous transforms 
by the stabilized hard thresholding. 
\textit{J.~Math. Sci.} 234(6):780--785.

\bibitem{3-sh-1}
\Aue{Kudryavtsev, A.\,A., and O.\,V.~Shestakov.} 2011. Аsimptotika otsenki riska pri 
veyglet-veyvlet razlozhenii nablyuda\-emo\-go signala [The average risk assessment 
of the wavelet decomposition of the signal].
\textit{T-Comm~--- Telecommunications and Their Application in
Transport Industry} 2:54--57.

\bibitem{4-sh-1}
\Aue{Kudryavtsev, A.\,A., and O.\,V.~Shestakov.} 2011. Аsimptoticheskoe raspredelenie 
otsenki riska porogovoy ob\-ra\-bot\-ki veyglet-koeffitsientov signala pri 
neizvestnom urovne shuma [Asymptotic distribution of the risk estimate of 
the signal vaguelette coefficients thresholding at the unknown noise level]. 
\textit{T-Comm~--- Telecommunications and Their Application in
Transport Industry} 5:24--30.

\bibitem{5-sh-1}
\Aue{Abramovich, F., and B.\,W.~Silverman.} 1998. Wavelet 
decomposition approaches to statistical inverse problems. 
\textit{Biometrika} 85(1):115--129.

\bibitem{6-sh-1}
\Aue{Mallat, S.} 1999. \textit{A~wavelet tour of signal processing.} New York, NY: 
Academic Press. 857 p.

\bibitem{7-sh-1}
\Aue{Lee, N.} 1997. Wavelet-vaguelette decompositions and homogenous equations. 
 West Lafayette, IN: Purdue University. PhD Thesis. 103~p.

\bibitem{8-sh-1}
\Aue{Breiman, L.} 1996. 
Heuristics of instability and stabilization in model selection. 
\textit{Ann. Stat.} 24(6):2350--2383.

\bibitem{9-sh-1}
\Aue{Jansen, M.} 2001. \textit{Noise reduction by wavelet thresholding.} 
Lecture notes in statistics ser.
New York, NY: Springer Verlag.  Vol.~161. 196~p.

\bibitem{10-sh-1}
\Aue{Shestakov, O.\,V.} 2012. O~skorosti skhodimosti otsenki riska porogovoy 
obrabotki veyvlet-koeffitsientov k~nor\-mal'\-no\-mu zakonu pri ispol'zovanii robastnykh 
otsenok dispersii [On the rate of convergence to the normal law of risk estimate for 
wavelet coefficients thresholding when using robust variance estimates]. 
\textit{Informatika i~ee Primeneniya~--- Inform. Appl.}  6(2):122--128.

\bibitem{11-sh-1}
\Aue{Serfling, R.} 1980. \textit{Approximation theorems of mathematical statistics}.
New York, NY: John Wiley \& Sons. 371~p.

\bibitem{12-sh-1}
\Aue{Serfling, R., and S.~Mazumder.} 2009. Exponential probability inequality 
and convergence results for the median absolute deviation and its modifications. 
\textit{Stat. Probabil. Lett.} 79(16):1767--1773.
\end{thebibliography}

 }
 }

\end{multicols}

\vspace*{-6pt}

\hfill{\small\textit{Received December 14, 2018}}

%\pagebreak

%\vspace*{-18pt}  

\Contrl

\noindent
\textbf{Shestakov Oleg V.} (b.\ 1976)~--- 
Doctor of Science in physics and mathematics, professor, Department of 
Mathematical Statistics, Faculty of Computational Mathematics and Cybernetics, 
M.\,V.~Lomonosov Moscow State University, 1-52~Leninskiye Gory, GSP-1, Moscow 119991, 
Russian Federation; senior scientist, Institute of Informatics Problems, 
Federal Research Center ``Computer Science and Control'' 
of the Russian Academy of Sciences, 44-2~Vavilov Str., Moscow 119333, 
Russian Federation; \mbox{oshestakov@cs.msu.su}
\label{end\stat}

\renewcommand{\bibname}{\protect\rm Литература} 
     %10
\def\stat{kozerenko}

\def\tit{КОГНИТИВНО-ЛИНГВИСТИЧЕСКИЕ ПРЕДСТАВЛЕНИЯ 
В~СИСТЕМАХ ОБРАБОТКИ ТЕКСТОВ}

\def\titkol{Когнитивно-лингвистические представления 
в~системах обработки текстов}

\def\autkol{Е.\,Б.~Козеренко, И.\,П.~Кузнецов}
\def\aut{Е.\,Б.~Козеренко$^1$, И.\,П.~Кузнецов$^2$}

\titel{\tit}{\aut}{\autkol}{\titkol}

%{\renewcommand{\thefootnote}{\fnsymbol{footnote}}\footnotetext[1]
%{Работа выполнена при поддержке Российского фонда фундаментальных
%исследований, проект~10-01-00480. Статья написана на основе материалов доклада, 
%представленного на IV Международном семинаре <<Прикладные задачи теории вероятностей 
%и математической статистики, связанные с моделированием информационных систем>> 
%(зимняя сессия, Аоста, Италия, январь--февраль 2010 г.).}}

\renewcommand{\thefootnote}{\arabic{footnote}}
\footnotetext[1]{Институт проблем информатики Российской академии наук, kozerenko@mail.ru}
\footnotetext[2]{Институт проблем информатики Российской академии наук, igor-kuz@mtu-net.ru}


\Abst{Рассмотрены вопросы проектирования и развития 
семантико-синтаксических и лексико-семантических представлений в 
лингвистических процессорах ряда систем, основанных на аппарате расширенных 
семантических сетей (РСС). Системы этого класса создаются для извлечения знаний из 
текстов на естественных языках, отображения извлеченных сущностей и связей в 
структуры базы знаний (БЗ) и использования знаний для поддержки экспертных 
аналитических решений в различных сферах приложения. В~фокусе внимания 
находятся ин\-же\-нер\-но-линг\-ви\-сти\-че\-ские представления, позволяющие 
построить целостную работающую лингвистическую модель, которая 
модифицируется в зависимости от конкретной задачи: от <<тяжелой>> формы на 
основе детальных глубинных представлений до фокусных редуцированных 
оболочек, настроенных на узкую предметную область (ПО) и ограниченный язык 
общения. Особое внимание уделяется способам описания 
дис\-три\-бу\-тив\-но-транс\-фор\-ма\-ци\-он\-ных признаков языковых объектов.}

\KW{интеллектуальные системы; семантические представления; лингвистические 
процессоры; обработка естественного языка; извлечение знаний}

       \vskip 14pt plus 9pt minus 6pt

      \thispagestyle{headings}

      \begin{multicols}{2}

      \label{st\stat}

\section{Введение}

     Данная работа посвящена проблемам создания\linebreak 
     когни\-тив\-но-линг\-ви\-сти\-че\-ских моделей естественного языка для 
различных классов информационных систем и описанию опыта создания 
линг\-ви\-сти\-че\-ских представлений для интеллектуальных\linebreak технологий 
обработки текстов. Вопросы извлечения знаний из текстов и создания модели 
естественного языка рассматриваются в единстве. В центре внимания будут 
находиться лингвистические процессоры интеллектуальных систем, 
разработанных на основе аппарата \textit{расширенных семантических 
сетей}~[1--5]. %\cite{1koz}--\cite{3koz}, \cite{18koz}--\cite{19koz}. 
Будем 
называть их \textit{РСС-сис\-те\-мы}. Эти системы создавались коллективом 
разработчиков, включая авторов данной статьи в Институте проб\-лем 
информатики РАН на протяжении целого ряда лет в рамках 
исследовательских проектов и прикладных систем, ориентированных на 
конкретные ПО заказчиков. Можно выделить четыре 
поколения РСС-систем. Ко\-гни\-тив\-но-линг\-ви\-сти\-че\-ские 
представления, заложенные в основу систем этого класса, прошли 
определенный эволюционный путь. 
     
     Интеллектуальные РСС-сис\-те\-мы содержат развитые \textit{базы 
знаний}, при этом знания представлены в виде записей на языке 
РСС, называемых 
     \textit{РСС-струк\-ту\-ра\-ми}. Лингвистические знания, таким 
образом, являются частным случаем <<знаний>> и также представлены в 
виде записей на языке РСС. Основным 
конструктивным элементом РСС\linebreak является именованный $N$-мест\-ный 
предикат, на\-зы\-ва\-емый <<\textit{фрагментом}>>. Все множество языковых 
объектов задается в виде системы пре\-ди\-кат\-но-ак\-тант\-ных структур, при этом 
поддерживаются механизмы представления вложенных структур, что дает 
очень мощные изобразительные возможности для описания объектов 
различных языковых уровней. Очень важными факторами являются 
однородность и единообразие лингвистических представлений. 
     
     В процессе анализа и синтеза предложений естественного языка 
используется фор\-маль\-но-грам\-ма\-ти\-че\-ский аппарат, сходный с 
грамматиками зависимостей. При этом подходе опорными элементами 
служат слова и конструкции, выполняющие роль предикатов в предложении, 
и результатом анализа предложения должен стать один предикат, 
соответствующий сказуемому рассматриваемого предложения (т.\,е.\ 
основному глаголу в личной форме или другому основному предикатному 
выражению). Таким образом, в процессе анализа происходит выявление 
\textit{когнитивных опор} предложения: <<слов-дейст\-вий>> и 
     <<слов-от\-но\-ше\-ний>>, т.\,е.\ глаголов и других слов, имеющих 
синтактико-семантические валентности. Примером <<слов-от\-но\-ше\-ний>> 
могут служить, например, слова <<отец>>, <<друг>> и~т.\,п., т.\,е.\ в данном 
случае <<отношения>> (или \textit{функции}~--- в терминах языка логики 
предикатов 1-го порядка)~--- это слова, которые задают сильные, четко 
выраженные син\-так\-ти\-ко-се\-ман\-ти\-че\-ские ожидания. 
     
     Семантический анализ в ин\-же\-нер\-но-линг\-ви\-сти\-че\-ском 
понимании~--- это процесс перевода ес\-тест\-вен\-но-язы\-ко\-вых 
выражений во <<внутренние>> структуры БЗ, в 
рассматриваемой ситуации этими <<внутренними>> структурами являются 
записи на языке РСС. Таким образом, структуры БЗ~--- это код смысла в 
интеллектуальных информационных системах подобного рода. 
     
     В работе рассматриваются ин\-же\-нер\-но-линг\-ви\-сти\-че\-ские 
решения в системах с <<пол\-ным>> линг\-ви\-сти\-че\-ским анализом~--- это 
     сис\-те\-мы 1-го и 2-го поколения: ДИЕС1, ДИЕС2, 
     Логос-Д~\cite{2koz, 3koz}~--- и сис\-те\-мах с <<фактографическим>> 
подходом: интеллектуальных системах поддержки аналитических решений 
(ИСПАР)~\cite{18koz, 19koz}, где целью анализа является выделение 
сущностей и связей из текстов,~--- это системы 3-го и 4-го поколения. 

\section{Процесс концептуально-лингвистического моделирования 
в системах, основанных на аппарате расширенных семантических сетей}
     
\subsection{Центральные вопросы семантического моделирования} %2.1
     
     Концептуально-лингвистическое моделирование (КЛМ)~--- это 
процесс построения ес\-тест\-вен\-но-язы\-ко\-вой модели ПО (рис.~1), синтезирующий в себе подходы 
концептуального и лингвистического моделирования~[1--3]. 
По\-стро\-ение концептуально-лингвистической модели некоторой 
ПО подразделяется на следующие этапы:
     \begin{itemize}
     \item построение собственно концептуальной модели, т.\,е.\ вычленение 
базовых понятий, организация их в ро\-до-ви\-до\-вые деревья и определение 
связей между ними;
     \item разработка идеографического словаря ПО, т.\,е.\ 
лексическое наполнение концептуальной модели;
     \item ввод базовых правил, описывающих на естественном языке 
<<модель мира>>, релевантную данной ПО.
     \end{itemize}
     
     
     Методика КЛМ на 
основе аппарата РСС базируется на следующих принципах:
     \begin{itemize}
\item модель должна быть <<открытой>>, т.\,е.\ поддерживать эффективный 
механизм расширения и обновления информации;
\begin{center} %fig1
%\vspace*{3pt}
\hspace*{-10.7158pt}\mbox{%
\epsfxsize=77.871mm
\epsfbox{koz-1.eps}
}\hspace{10.7158pt}
%\end{center}
\vspace*{4pt}
%\begin{center}
{{\figurename~1}\ \ \small{Процесс КЛМ}}
\end{center}
\vspace*{3pt}

%\bigskip
\addtocounter{figure}{1}
\item модель представления <<смысла>> должна учитывать факты 
экстралингвистической реаль\-ности, которые в виде правил и отношений 
составляют некоторую базовую <<модель мира>>, достраиваемую 
конкретными моделями ПО;
\item модель должна быть практичной, т.\,е.\ не перегруженной детальными 
описаниями связей и отношений между понятиями, чтобы обеспечить 
возможность ее реализации, но в то же время отражать всю релевантную 
конкретной задаче информацию.
\end{itemize}

     \begin{figure*} %fig2
%     \begin{center}
\hspace*{23mm}\{(ВЫРАБАТЫВА895\_\_)(DICSEM)\\
\hspace*{23mm}COORD(PROGNOZ1,RUS,ВЫРАБАТЫВА895\_\_,S50\_31\_51\_20,\%)\\
\hspace*{23mm}SUB(UNIV,0+)~SUB(UNIV,1+)~SUB(UNIV,2+)\\
\hspace*{23mm}ВЫРАБАТЫВ(0-,1-,2-/3+)~INFI(3-)~ПРИДЕТСЯ(3-)~ПРИДЕТСЯ(3$-$/4+) \\
\hspace*{23mm}FUT1(4$-$)~SUB(СРЕД,5+)
%\end{center}
%\vspace*{2pt}
\Caption{Пример записи представления глагола <<вырабатывать>> в семантическом 
словаре
\label{f2koz}}
%\vspace*{6pt}
\end{figure*}

     Реалистичный подход к постановке задачи диктует необходимость 
ограничения моделируемого подмножества естественного языка. Суть 
ограничений сводится к следующему:
     \begin{enumerate}[(1)]
     \item анализируемые текстовые материалы содержат 
экспертные знания из конкретных ПО (в разработанных 
авторами системах это были такие ПО, как диагностика 
брака при изготовлении микросхем, социальное прогнозирование, 
криминалистика и другие);
     \item в целях максимально возможного устранения 
неоднозначности словарь строится по модульному принципу: есть некоторая 
наиболее общая часть (1--2~уровня), которая достраивается специальными 
словарями для каж\-дой отдельной~ПО.
     \end{enumerate}
     
     Предлагаемая модель лексической семантики основана на принципе 
<<ядерного>> значения, реализуемого в контексте данной 
ПО, с последующим индуктивным наращиванием других значений (если 
они актуализируются в рас\-смат\-ри\-ва\-емых контекстах). Также используется 
таксономия, которая реализуется в виде иерархических деревьев классов 
слов. 
     
     Общая <<модель мира>> системы является основой для моделей ПО. 
Элементами этой модели служат классы слов, которые подразделяются на 
понятия/имена, отношения, действия, свойства, характеристики действий, 
временные и пространственные характеристики.
     
     Самым общим понятием является \textit{концепт}, или 
\textit{универсальный класс}, который подразделяется на объект, ситуацию, 
процесс и~др. 
     
     Слова, относящиеся к классам действий и отношений, представлены 
как се\-ман\-ти\-ко-син\-так\-си\-че\-ские фреймы, задающие 
     пре\-ди\-кат\-но-ак\-тант\-ные структуры (модель управления). Однако 
в описываемом подходе (назовем его РСС-под\-хо\-дом) существенно 
расширена область значений актантов. Суть расширения состоит, во-первых, 
в том, что в роли актантов могут выступать не только простые объекты, 
соответствующие отдельным словам, но и структурные объекты, 
представляющие словосочетания и фразы, а во-вторых, в том, что понятие 
падежа включает в себя не только семантические, но и синтаксические 
признаки.
     
     Подход, основанный на РСС, позволяет отражать произвольный 
уровень вложенности структур за счет пропозициональных вершин 
семантической сети. Это обеспечивает представление\linebreak сложных 
синтаксических конструкций фраз\linebreak естественного языка, а также позволяет 
отразить\linebreak структурный характер лексической семантики,\linebreak которая в 
предлагаемой модели имеет иерар\-хи\-че\-ски-се\-те\-вую структуру. 
Линг\-ви\-сти\-че\-ские зна-\linebreak ния пред\-став\-ле\-ны в системном словаре и 
декла\-ра\-тивных модулях линг\-ви\-сти\-че\-ско\-го процессора.\linebreak В РСС-сис\-те\-мах 
так\-же реализована функция динамически форми\-ру\-емо\-го семантического 
словаря, который на основе исходной лингвистической информации 
достраивается системой автоматически в процессе об\-ра\-бот\-ки конкретных 
текстов. На рис.~\ref{f2koz} пред\-став\-ле\-но \mbox{такое} <<внутреннее>> описание 
глагола в семантическом словаре. Этот словарь автоматически генерируется 
РСС-системами ДИЕС2, ЛОГОС-Д, ИКС в процессе обработки 
     естест\-вен\-но-язы\-ко\-вых \mbox{текстов}. 
     {\looseness=1
     
     }
     
     
\subsection{Особенности применения аппарата расширенных семантических сетей 
в~когнитивно-лингвистическом моделировании} %2.2
     
     Дадим краткое описание аппарата РСС и  
обос\-ну\-ем выбор именно этого метода представления для моделирования 
естественного языка. Классическое понятие семантической сети сводится к 
следующему: задаются некоторые вершины, соответствующие объектам,  
вершины связываются дугами, которые помечаются именами отношений. 
Однако с помощью подобных сетей оказывается трудно представлять 
сложные виды информации, например, когда объекты, связанные 
отношениями, образуют агрегаты и когда отношения связываются между 
собой отношениями и~др. Поэтому в сети вводятся вершины, 
соответствующие именам отношений, а также специальный композиционный 
элемент, называемый вершиной связи. Вершина связи как бы <<разрывает>> 
дугу и подсоединяется одним ребром к вершине-отношению, а другими 
ребрами~--- к вершинам-объектам. Расширенная семантическая сеть является развитием такого сорта 
сетей в направлении повышения изобразительных возможностей при 
сохранении свойства однородности.
     
     Основой РСС является множество вершин ($V$), из которых 
составляются элементарные фрагменты (ЭФ) вида
     $
     V_0(V_1,V_2,\ldots ,V_k/V_{k+1})
     $, 
     где
$V_0, V_1, V_2,\ldots , V_k, V_{k+1}>0$.
     
     
     Такой фрагмент представляет $k$-местное отношение. Позиции 
вершин в ЭФ определяют их роли. 
Вершина~$V_0$ ставится в соответствие имени отношения, 
вершины~$V_1$, $V_2$, \ldots , $V_k$~--- объектам, участ\-ву\-ющим в 
отношении, а вершина~$V_{k+1}$, отделенная косой линией,~--- всей 
совокупности упомянутых объектов с учетом их отношения. В~дальнейшем 
будем $V_{k+1}$ называть $C$-вершиной ЭФ.\linebreak 
Множество ЭФ образует РСС. 
С~помощью РСС представляются наборы отношений, различные ситуации, 
сце\-нарии. Сильной стороной РСС-под\-хо\-да является возможность 
однородного пред\-став\-ле\-ния как предметной (концептуальной), так и 
лингвистической информации, что обеспечивает эффективную обработку 
знаний и поддержание непротиворечи\-вости~БЗ.
          \begin{figure*} %fig3
     \vspace*{1pt}
\begin{center}
\mbox{%
\epsfxsize=125.039mm
\epsfbox{koz-3.eps}
}
\end{center}
\vspace*{-9pt}
     \Caption{Семантико-синтаксический анализ без выявления глагольных 
словоформ
      \label{f3koz}}
\vspace*{12pt}
 %     \end{figure*}
%            \begin{figure*} %fig4
           \vspace*{1pt}
\begin{center}
\mbox{%
\epsfxsize=103.129mm
\epsfbox{koz-4.eps}
}
\end{center}
\vspace*{-9pt}
      \Caption{Целостная семантическая структура предложения
      \label{f4koz}}
      \end{figure*}

     
     Посредством РСС в БЗ представлены лингвистические  и 
предметные знания. Обработка этих знаний осуществляется 
продукциями языка ДЕКЛ, на котором реализованы сле\-ду\-ющие шесть 
блоков: морфологического анализа, семанти\-ческого анализа слов, 
син\-так\-ти\-ко-се\-ман\-ти\-че\-ско\-го анализа форм, 
прагматических функций, организации системной активности и 
обратный лингвистический процессор. С~помощью продукций 
осущест\-вля\-ет\-ся последовательное преобразование сети~--- РСС. При этом 
проходятся фазы, соответствующие уровню понимания входного текста. 
Рас\-смот\-рим~их.
     \begin{enumerate}[1.]
     \item На первом шаге анализа строится 
пространственная структура предложения с морфологической информацией 
для каждого слова.\linebreak Каж\-дый член предложения представляется вершиной 
семантической сети. Вместо слова генерируется код (если слово 
многозначно, т.\,е.\ принадлежит к нескольким классам,~--- то более одного 
кода). Основой кода служит корень слова. На этом этапе предложение 
представляется в виде набора фрагментов типа LRR (специальных меток 
результатов 1-го этапа анализа), объединяемых в целостную структуру 
посредством вершины связи. Результат 1-го этапа постоянно обращается к 
словарю: <<Что значит данное слово?>>
     \item На втором этапе каждой вершине сопоставляется семантический 
класс и присваивается новый код. За словами (т.\,е.\ конкретными вершинами 
РСС) система видит объекты, действия, свойства, т.\,е.\ строит 
классификации. Производится се\-ман\-ти\-ко-син\-так\-си\-че\-ский анализ 
без выявления глагольных словоформ, при этом предложение представляется 
в виде совокупности фрагментов типа SEM и SEMD~--- специальных меток 
результатов 2-го этапа анализа (рис.~\ref{f3koz}).
     \item На третьем этапе происходит частичное <<сворачивание>> 
синтаксических структур в более компактные (например, свойство объекта и 
сам объект) с присваиванием нового кода и строится фрагмент для объекта, 
обладающего этим свойством.
     \begin{figure*}[b] %fig5
          \vspace*{12pt}
\begin{center}
\mbox{%
\epsfxsize=147.485mm
\epsfbox{koz-5.eps}
}
\end{center}
\vspace*{-9pt}
     \Caption{Глубинная структура предложений
      \label{f5koz}}
      \end{figure*}      
     \item На четвертом этапе выявляются отношения и действия и 
производится анализ непосредственного контекста на соответствие заданным 
семантическим падежам. Система проверяет, подходят ли объекты 
(концепты, понятия) на аргументные места данного действия или отношения. 
При этом отглагольные существительные (<<делатель>>, т.\,е.\ агент 
действия, или <<делание>>~--- процесс~--- анализируются как слова с 
двойной природой: вначале как действия, а затем как объекты). Результатом 
этого этапа является целостная семантическая структура предложения, 
которая представляется фрагментом типа SEMSTR~--- метки результата 4-го 
этапа анализа (рис.~\ref{f4koz}).
     \item На пятом этапе происходит анализ прагматики: установление 
кореференциальных отношений, частичное восстановление эллиптических 
конструкций, система производит дальнейшие действия с построенными 
фрагментами.
     \end{enumerate}

     
Система ДИЕС допускает ввод полисемичных форм глаголов. Для этого следует 
воспользоваться формальной записью лингвистических знаний. 
     В~сис\-те\-мах, основанных на РСС, все функции реализованы на 
единой основе~--- в рамках языков РСС и ДЕКЛ, которые были разработаны 
с ориентацией на задачи обработки естественного языка.

%\vspace*{-6pt}

\section{Представление семантики глаголов, глубинные 
и~поверхностные структуры}
     
     В процессе анализа выявляются семантические вершины предложения: 
происходит выявление <<слов-дей\-ст\-вий>>, т.\,е.\ глаголов, и 
     <<слов-от\-но\-ше\-ний>>. Что же является конструктивной основой\linebreak 
задания семантических представлений предикатных слов и выражений? Как 
убедительно показано в работе~\cite{4koz}, семантика глагола 
определяется его дис\-три\-бу\-тив\-но-транс\-фор\-ма\-ци\-он\-ны\-ми\linebreak 
свойствами. Поэтому смысл предикатных выражений должен кодироваться с 
учетом их дистрибутивных и трансформационных признаков. 
     
     Выдвинутая рядом лингвистов (Хомский, Филлмор) гипотеза о том, что 
все предложения имеют глубинные и поверхностные 
     структуры~[7--10], явилась очень продуктивным 
источником проектных решений при создании первых РСС-сис\-тем и 
развивалась в дальнейшем. 

В~тео\-ре\-ти\-ко-линг\-ви\-сти\-че\-ском 
понимании глубинная структура~--- это абстракция, содержащая все 
элементы, необходимые для образования поверхностных структур 
предложений со сходной семантикой. 

     В~ин\-же\-нер\-но-линг\-ви\-сти\-че\-ском понимании\linebreak глубинная 
структура~--- это запись на языке БЗ, например на языке РСС, 
которая может быть представлена в <<поверхностном>> виде на одном из 
естественных языков в результате конечного числа определенных 
преобразований. Например, предложения

\noindent
\begin{align*}    
(1)\ &\mbox{\textit{The programmer writes the code}}\\
(2)\ &\mbox{\textit{The code is written by the programmer}}
\end{align*}
имеют истоком одну глубинную структуру:

\medskip

\noindent
     \begin{verbatim}
  Programmer <---- write ----> Code
      agent                   object,
\end{verbatim}

\medskip

\noindent
хотя и отличаются своими поверхностными структурами. В~каждом из них 
имеется агент (the programmer), объект (the code) и действие (write).\linebreak Согласно 
концепции \textit{падежной грамматики} Филлмора~\cite{5koz} глубинная 
структура для обоих предложений инвариантна. Эту структуру можно 
представить в виде скобочной записи $V(\mathrm{AGENT}, \mathrm{OBJECT})$. В~графическом 
виде глубинная структура предложения также может быть представлена 
диаграммой в виде дерева, где отражены инвариантные отношения 
зависимости между предикатной вершиной и актантами (рис.~\ref{f5koz}), 
причем в таком представлении явным образом разграничиваются 
\textit{модальность} (MOD) и \textit{пропозиция} (PROP).
     

     В исходном варианте~\cite{5koz} теория признавала шесть падежей: 
агентив, инструменталис, датив, объектив, локатив и фактитив. По мере 
развития теории~\cite{8koz} происходило увеличение числа падежей, однако 
<<умножение>> количества падежей утяжеляет первоначальную 
конфигурацию, поэтому при построении инженерных семантических 
представлений требуется некоторый <<компромиссный>> вариант, 
сочетающий в себе необходимую полноту, с одной стороны, и простоту и 
гибкость, с другой.

\begin{figure*}[b] %fig6
\vspace*{24pt}
\begin{center}
\mbox{%
\epsfxsize=156.873mm
\epsfbox{koz-6.eps}
}
\end{center}
%\vspace*{-9pt}
\Caption{Обобщенное функциональное представление систем ИСПАР
\label{f6koz}}
\end{figure*}
     
%\vspace*{-6pt}

\section{Некоторые базовые аспекты построения многоязычных 
систем}
     
     Одним из приоритетных направлений развития РСС-сис\-тем является 
обеспечение обработки текстов на нескольких языках, прежде всего для 
рус\-ско-анг\-лий\-ской языковой пары. В системах 2-го поколения~--- ДИЕС2, 
ИКС, ЛОГОС-Д были реализованы лингвистические процессоры и словари 
для русского и английского языка, позволявшие обрабатывать тексты для 
ряда ПО. При этом поддерживался как режим ввода 
лингвистических знаний линг\-вис\-том-ана\-ли\-ти\-ком, так и 
автоматический режим самообучения системы по вводимым \mbox{текстам}. 
{\looseness=1

}

Проводились также эксперименты с итальянским и французским языком. 
При создании многоязычных систем авторы обращались к европейским 
языкам. Очевидно, что европейские языки обладают большим числом общих 
правил, чем любой из них с языками других групп. Но при этом все 
естественные языки обладают общей структурой на самом глубинном 
уровне. На этом уровне располагаются главные элементы естественного 
языка: \textit{предложение}, \textit{модальность}, \textit{пропозиция}.
     
     Моделирование смысловых представлений~--- это процесс, 
развивающийся в направлении от поверхностных семантических структур к 
глубинным. Поиск такого внутреннего представления смысла в условиях 
многоязычной ситуации является на\-прав\-ле\-ни\-ем развития методов 
     КЛМ на базе  РСС. 
     
%     \vspace*{-48pt}
     
\section{Интеллектуальные системы поддержки аналитических 
решений}
     
Системы РСС 3-го и 4-го поколения на\-прав\-ле\-ны на извлечение знаний 
в виде \textit{объектов}, или \textit{сущностей}, и связей между ними из 
пред\-мет\-но-ориен\-ти\-ро\-ван\-ных текстов на русском и английском 
языке~\cite{18koz, 19koz}.

    
В настоящее время во всем мире активно ведутся работы по созданию 
систем извлечения фактов из текстов на естественных языках~[11--14], создаются развитые тезаурусы и 
онтологии~\cite{17koz}. Сис\-те\-мы РСС функционально шире, поскольку 
имеют возможность не только извлекать факты, но и поддерживать 
механизмы логического анализа и экспертного вывода на основе 
извлеченных знаний. Сис\-те\-ма\-ми такого рода являются ИСПАР. В~целом это 
направление исследований требует дальнейшей проработки 
     лек\-си\-ко-се\-ман\-ти\-че\-ских представлений, создания 
     пред\-мет\-но-ориен\-ти\-ро\-ван\-ных семантических словарей. 

Обобщенное функциональное представление систем ИСПАР дано на 
рис.~\ref{f6koz}. 
     
     В рамках ИСПАР на основе РСС 
(\mbox{ИСПАР}--РСС) были реализованы полномасштабные и\linebreak пилотные 
проекты для ряда ПО: криминалистики, управления 
кадрами, мониторинга финансово-экономического кризиса и 
др.~\cite{18koz, 19koz}.

\section{Применение аппарата расширенных семантических сетей в~лингвистических 
исследованиях}
     
     В настоящее время в рамках проектов, на\-прав\-лен\-ных на создание 
открытых лингвистических ресурсов~\cite{20koz} для 
     на\-уч\-но-прак\-ти\-че\-ских целей, ведутся работы по выравниванию 
параллельных текстов научных статей, патентов и 
     фи\-нан\-со\-во-эко\-но\-ми\-че\-ских текстов. В~качестве одного из 
методов выравнивания используется РСС-под\-ход, поскольку он позволяет 
отразить глу\-бин\-но-се\-ман\-ти\-че\-ский уровень языковых структур. 

На  рис.~7 представлен фрагмент первого этапа лингвистического 
анализа в многоязычных системах. Для <<идеальной>> ситуации, когда 
структуры исходного текста и текста перевода практически совпадают, такая 
ситуация имеет место в меньшинстве случаев. Основные трудности 
возникают при наличии переводческих трансформаций в параллельных 
текстах. Особое внимание следует уделять гла\-голь\-но-имен\-ным 
трансформациям, например явлению \textit{номинализации}, поскольку она 
очень продуктивна для всех исследовавшихся языков.

     
     Ключевой задачей при разработке методов сопоставления 
параллельных текстов является выявление и детальное описание тех 
языковых трансформаций, которые имеют место при переводе 
     естест\-вен\-но-язы\-ко\-вых конструкций с одного языка на 
другой~\cite{9koz}, потому что далеко не всегда некое содержание 
передается струк\-тур\-но-по\-доб\-ны\-ми средствами в текстах на разных 
языках. Сравнительное исследование употребления различных частей речи в 
параллельных текстах на разных языках создает основу для выявления и 
описания языковых транс-\linebreak

\begin{center} %fig7
\vspace*{3pt}
\mbox{%
\epsfxsize=79.726mm
\epsfbox{koz-7.eps}
}
\end{center}
\vspace*{4pt}
%\begin{center}
{{\figurename~7}\ \ \small{Первый этап анализа параллельных текстов ($W_n$
обозначает словоформу с номером~$n$, $1\leq n\geq 5$)}}
%\end{center}
%\vspace*{9pt}

%\bigskip
\addtocounter{figure}{1}
      

\noindent 
формаций, при этом центральной трансформацией
является \textit{номинализация}. Явление номинализации
было исследовано в 
ряде работ отечественных и зарубежных лингвистов~[17--20]. 
Ближе всего к правильному, по мнению авторов данной статьи, 
пониманию этого явления следующие определения номинализации: 
<<конструкции\ldots называются номинализованными~--- в том смысле, что 
их естественно рассматривать как результат номинализации конструкций с 
предикативным употреблением глаголов и прилагательных>>; 
<<номинализация~--- это синтаксический процесс, который соотносит 
предложения с именными группами>>~\cite{9koz, 10koz}. Выявление 
номинализованных конструкций в параллельных научных и патентных 
текстах на русском, английском, французском и немецком языках в научных 
и патентных текстах и сопоставительное описание гла\-голь\-но-имен\-ных 
межъязыковых трансформаций~--- одна из центральных задач 
     ин\-же\-нер\-но-линг\-ви\-сти\-че\-ских исследований. 
     
     Следующей базовой трансформацией в исследуемых текстах на 
нескольких европейских языках является адъек\-тив\-но-ад\-вер\-би\-аль\-ное 
преобразование. Это означает, что при переводе с одного языка на другой 
происходит синтаксическое преобразование имен прилагательных в наречия 
и обратное преобразование~--- наречий в прилагательные. Установление 
семантических соответствий между этими языковыми объектами также 
возможно осуществить посредством аппарата~РСС. 
     
     При семантическом выравнивании непараллельных текстов, имеющих 
одну и ту же денотативную составляющую, аппарат РСС позволяет выявить в 
текстах когнитивные опоры (слова с сильной валентностью~--- 
     <<сло\-ва-дейст\-вия>> и <<сло\-ва-от\-но\-ше\-ния>>) и установить 
между ними семантические соответствия.

\section{Заключение}

     В данной работе представлен опыт создания и развития 
     когни\-тив\-но-линг\-ви\-сти\-че\-ских пред\-став\-ле\-ний в 
интеллектуальных информационных сис\-те\-мах, разработанных на основе 
аппарата РСС. Аппарат РСС 
обеспечивает мощные изобразительные возможности для описания всех 
уровней естественного языка, включая уровень 
     глу\-бин\-но-се\-ман\-ти\-че\-ских представлений и межъязыковых 
соответствий. Конкретные лингвистические процессоры, которые были 
созданы на основе этого подхода, прошли определенный путь развития и 
позволили выработать проектные решения для основных задач текущего 
этапа~--- извлечения и обработки содержательных знаний из текстов на 
естественных языках и сопоставления языковых структур в текстах на 
различных языках с учетом базовых трансформаций.
     
     Проблема извлечения и обработки знаний открывает перспективы 
развития интеллектуальных направлений компьютерной лингвистики, 
поскольку ее основной акцент смещен в сторону\linebreak глубинных представлений 
языка, в которых используются как грамматические (морфологические и 
синтаксические), так и семантические атрибуты для описания языковых 
объектов. Проводи-\linebreak мые авторами исследования параллельных текстов 
направлены также на рассмотрение этой проблемы~\cite{20koz}. 
Центральное место в проводящихся линг\-ви\-сти\-че\-ских исследованиях 
занимает изучение и формализация процессов трансформации языковых 
структур, особенно все варианты глагольно-но\-ми\-на\-тив\-ных трансформаций, 
создание развитых дис\-три\-бу\-тив\-но-транс\-фор\-ма\-ци\-он\-ных 
описаний предикатых структур для рассматриваемых языков. 
     
     Для задач извлечения знаний и создания \mbox{ИСПАР} 
     дис\-три\-бу\-тив\-но-транс\-фор\-ма\-ци\-он\-ные описания имеют 
особое значение, поскольку таким образом задаются все возможные способы 
перевода языковых структур в пре\-ди\-кат\-но-ар\-гу\-мент\-ные 
представления, которые затем используются в процедурах обработки знаний.

{\small\frenchspacing
{%\baselineskip=10.8pt
%\addcontentsline{toc}{section}{Литература}
\begin{thebibliography}{99}

     \bibitem{1koz}
     \Au{Кузнецов~И.\,П.}
     Семантические представления.~--- М.: Наука, 1986. 290~с.
     
     \bibitem{2koz}
     \Au{Козеренко~Е.\,Б.}
     Кон\-цеп\-ту\-аль\-но-линг\-вис\-ти\-че\-ское моделирование в среде 
интеллектуального редактора знаний ИКС~// Проблемы проектирования и 
использования баз знаний.~--- Киев: Ин-т кибернетики им.\ В.\,М.~Глушкова, 
1992. C.~73--79.
     
     \bibitem{3koz}
     \Au{Kozerenko~E.\,B.}
     Multilingual processors: A unified approach to semantic and syntactic 
knowledge presentation~// Conference (International ) on Artificial Intelligence 
IC-AI'2001 Proceedings. Las Vegas, Nevada, USA. June 25--28, 2001.~--- Las 
Vegas: CSREA Press, 2001. P.~1277--1282.

     \bibitem{18koz} %4
     \Au{Kuznetsov~I.\,P., Efimov~D.\,A., Kozerenko~E.\,B.}
     Tools for tuning the semantic processor to application areas~// ICAI'09 
Proceedings, WORLDCOMP'09. July 13--16, 2009. Las Vegas, Nevada, USA. 
Vol.~I.~--- Las Vegas: CRSEA Press, 2009. P.~467--472.
     
     \bibitem{19koz} %5
     \Au{Kuznetsov~I.\,P., Kozerenko~E.\,B., Kuznetsov~K.\,I., 
Timonina~N.\,O.}
     Intelligent system for entities extraction (ISEE) from natural language 
texts~// Workshop (International) on Conceptual Structures for Extracting Natural 
Language Semantics (Sense'09) at the 17th Conference 
(International ) on Conceptual Structures (ICCS'09) Proceedings. University Higher School of 
Economics. Moscow, Russia, 2009. P.~17--25.
     
     \bibitem{4koz} %6
     \Au{Апресян~Ю.\,Д.}
     Экспериментальное исследование семантики русского глагола.~--- М.: 
Наука, 1967.  252~с.
     
     \bibitem{5koz} %7
     \Au{Филлмор~Ч.}
     Дело о падеже~// Новое в зарубежной линг\-вистике, 1968. Вып.~X. С.~369--495.
     
     \bibitem{6koz} %8
     \Au{Хомский~Н.}
     Аспекты теории синтаксиса.~--- М.: МГУ, 1972.
     
     \bibitem{7koz} %9
     \Au{Хомский Н.}
     Язык и мышление.~--- М.: МГУ, 1972.
     
     
     \bibitem{8koz} %10
     \Au{Fillmore~C.}
     The case for case reopened~// Syntax and Semantics. Vol.~8.~--- N.Y.: 
Academic Press, 1977. 
     

          \bibitem{15koz} %11
     FASTUS: A cascaded finite-state trasducer for extracting information from 
natural-language text~// AIC, SRI International, Menlo Park, California, 1996. 
     
     \bibitem{16koz} %12
     \Au{Han~J., Pei~Y., Mao~R.}
     Mining frequent patterns without candidate generation: A frequent-pattern 
tree approach~// Data Mining and Knowledge Discovery, 2004. Vol.~8. No.\,1. 
P.~53--87.
     
     
     \bibitem{13koz} %13
     \Au{Cunningham~H.}
     Automatic information extraction~// Encyclopedia of Language and 
Linguistics. 2nd ed.~--- Elsevier, 2005.
     
     \bibitem{14koz} %14
     \Au{Han~J., Kamber~M.}
     Data mining: Concepts and techniques.~--- Morgan Kaufmann, 2006.
     
     
     \bibitem{17koz} %15
     \Au{Добров~Б.\,В., Лукашевич~Н.\,В.}
     Онтологии для автоматической обработки текстов: Описание понятий 
и лексических значений~// Компьютерная лингвистика и интеллектуальные 
технологии: Тр. межд. конф. <<Диалог'06>>. Бекасово, 31~мая\,--\,4~июня 
2006. С.~138--142.

     \bibitem{20koz} %16
     \Au{Kozerenko~E.\,B.}
     INTERTEXT: A multilingual knowledge base for machine translation~// 
Conference (International) on Machine Learning, Models, Technologies and 
Applications Proceedings. June 25--28, 2007. Las Vegas, USA.~--- Las Vegas: 
CSREA Press, 2007. P.~238--243.

     \bibitem{9koz} %17
     \Au{Жолковский~А.\,К., Мельчук~И.\,А.}
     О семантическом синтезе~// Проблемы кибернетики, 1967. Вып.~19.
     
         
     \bibitem{11koz} %18
     \Au{Jacobs~R.\,A., Rosenbaum P.\,S.}
     English transformational grammar.~--- Blaisdell, 1968.
     

\label{end\stat}
     
          \bibitem{12koz} %19
     \Au{Балли~Ш.}
     Общая лингвистика и вопросы французского языка. 2-е изд.~--- М.: 
УРСС, 2001.

\bibitem{10koz} %20
     \Au{Падучева~Е.\,В.}
     О~семантике синтаксиса: Мат-лы к трансформационной 
грамматике русского языка. 2-е изд.~--- М: КомКнига, 2007.  296~с. 
     
 \end{thebibliography}
}
}


\end{multicols} %11
\def\stat{zatsman}

\def\tit{ТРАНСФОРМАЦИИ ОБЪЕКТОВ ПЕРВОГО И~ВТОРОГО ПОРЯДКА 
В~ЛЕКСИКОГРАФИЧЕСКОЙ ИНФОРМАЦИОННОЙ СИСТЕМЕ$^*$}

\def\titkol{Трансформации объектов первого и~второго порядка 
в~лексикографической информационной системе}

\def\aut{И.\,М.~Зацман$^1$}

\def\autkol{И.\,М.~Зацман}

\titel{\tit}{\aut}{\autkol}{\titkol}

\index{Зацман И.\,М.}
\index{Zatsman I.\,M.}


{\renewcommand{\thefootnote}{\fnsymbol{footnote}} \footnotetext[1]
{Исследование выполнено в~ФИЦ ИУ РАН за счет гранта Российского научного фонда №\,24-18-00155, {\sf 
https://rscf.ru/project/24-18-00155}. Работа выполнялась с~использованием инфраструктуры Центра 
коллективного пользования <<Высокопроизводительные вычисления и~большие данные>> (ЦКП 
<<Информатика>>) ФИЦ ИУ РАН (г.\ Москва).}}


\renewcommand{\thefootnote}{\arabic{footnote}}
\footnotetext[1]{ Федеральный исследовательский центр <<Информатика и~управление>> Российской академии наук, 
\mbox{izatsman@yandex.ru}}

\vspace*{-12pt}


  
  \Abst{Рассматриваются теоретические основания проектирования информационных 
технологий (ИТ) интеграции двуязычных словарей и~параллельных корпусов. Дано описание 
первых результатов создания третьего уровня классификации трансформаций объектов 
предметной области информатики, которую предполагается использовать при создании 
концепции лексикографической информационной системы, обеспечивающей интеграцию. 
Все сущности информатики в~статье разделены на два глобальных класса: объекты и~их 
трансформации. Для каждого такого класса конструируется своя классификация. Ранее были 
описаны два верхних уровня классификации трансформаций объектов предметной области. 
В~данной статье рассматривается третий уровень этой классификации. Основанием для 
построения самого верхнего ее уровня служило деление предметной области информатики 
на среды (ментальная, сенсорно воспринимаемая, цифровая и~ряд других сред), каждая из 
которых по определению включает объекты одной природы. Основанием для построения 
второго уровня классификации трансформаций объектов служила типология знаковых  
сис\-тем А.~Соломоника. Цель статьи состоит в~систематизации трансформаций первого 
и~второго порядка объектов предметной области на третьем уровне этой классификации. 
Основанием для систематизации служит средовая версия иерархии Акоффа.}
  
  \KW{объекты предметной области; трансформации объектов; классификация; данные; 
информация; знание; лексикографическая информационная сис\-тема}

\DOI{10.14357/19922264240211}{VZTGVV}
  
\vspace*{3pt}


\vskip 10pt plus 9pt minus 6pt

\thispagestyle{headings}

\begin{multicols}{2}

\label{st\stat}
  
\section{Введение}

\vspace*{-9pt}

  Возникновение параллельных корпусов, в~которых предложениям 
оригинального текста со\-по\-став\-ле\-ны предложения его перевода, обеспечило 
возможность контрастивного лингвистического\linebreak \mbox{анализа} на принципиально 
новом уровне полноты и~точности, недостижимом в~докорпусную эпоху. 
Пионерскими в~этой области стали работы \mbox{1990-х~гг}. Стига Йоханссона  
с~анг\-ло-нор\-веж\-ским корпусом~[1]. В России параллельные корпусы стали 
формироваться в~начале XXI~века в~рамках Национального корпуса русского 
языка~[2].
  
  Создатели двуязычных словарей используют параллельные корпусы для 
сбора материала и~эмпирической проверки своих гипотез, касающихся 
межъязы\-ко\-вой эквивалентности. Ценность параллельных корпусов 
определяется тем, что в~лингвистике этап сбора исходного материала считается 
наиболее трудоемким и~наименее творческим, а~параллельные корпусы 
позволяют значительно сэкономить время и~силы для творческого этапа 
создания словарей~[3].
 % 
  При этом двуязычные словари, создаваемые на основе исходного материала, 
извлеченного из параллельных корпусов, сейчас формируются без связей с~их 
текстами. Другими словами, онлайновые связи созданных словарей 
с~параллельными корпусами, которые служили источниками исходного 
материала, отсутствуют. 

Параллельные корпусы постоянно пополняются 
новыми текстами, в~предложениях которых можно обнаружить новые значения 
слов и~устойчивых словосочетаний. Однако при этом отсутствуют методы 
и~средства оперативного обновления словарей по корпусным данным. 
В~настоящее время проблема установления связей между двуязычными 
словарями и~параллельными корпусами (далее~--- проблема интеграции) 
находится на стадии поиска концептуальных подходов к~их интеграции на 
уровне значений.
  
  Подход к~решению проблемы интеграции, предлагаемый в~статье, учитывает 
  и~появление новых значений слов и~устойчивых словосочетаний, и~динамику 
смысловых значений, которая обусловлена развитием и~пополнением знания 
лингвистов, фиксирующих эти значения в~результате семантического анализа 
пополняемых корпусных данных. Проведенные эксперименты показали, что 
обнаружение нового лингвистического знания обусловливает и~формирование 
дефиниций новых значений, и~пересмотр уже существующих дефиниций~[4, 5].
  
  Например, в~проведенных экспериментах с~использованием ЦКП 
<<Информатика>> ФИЦ ИУ РАН фиксировалась эволюция значений немецких 
модальных глаголов, исходное состояние значений которых было описано 
в~не\-мец\-ко-рус\-ском словаре. В~экспериментальном массиве текстов как 
потенциальных источниках нового знания 16\,268 предложений содержали 
немецкие модальные глаголы и~в~2041 из них встречался глагол sollen. 
В~начале эксперимента в~словаре были описаны~12~значений этого модального 
глагола. По окончании эксперимента лингвисты обнаружили два новых его 
значения, согласовали их дефиниции и~описали эволюцию дефиниций~[6, 7].
  
  Таким образом, для решения проблемы интеграции требуется фиксировать 
новое знание, обнаруженное лингвистами в~текстовых данных параллельных 
корпусов, отслеживать эволюцию знания, представленного в~виде дефиниций 
значений слов и~устойчивых словосочетаний, и,~соответственно, 
актуализировать электронные двуязычные словари. Предлагаемый 
концептуальный подход к~интеграции, который планируется реализовать 
в~процессе проектирования лексикографической информационной сис\-те\-мы, 
фиксирующей эволюцию лингвистического знания, основан на решении 
следующих задач:\\[-14pt]
  \begin{itemize}
  \item категоризация трех базовых понятий информатики, включенных 
  в~иерархию Акоффа~[8] (данные, информация, знание), на объекты 
проектируемой сис\-те\-мы, которая необходима, чтобы фиксировать 
<<кванты>> нового знания и~отслеживать его эволюцию в~этой сис\-теме;\\[-15pt]
  \item  систематизация трансформаций объектов этой сис\-темы.\\[-14pt]
  \end{itemize}
  
  Цель статьи и~состоит в~решении двух задач: категоризации трех базовых 
понятий информатики на объекты лексикографической информационной  
сис\-те\-мы и~сис\-те\-ма\-ти\-за\-ции трансформаций первого и~второго порядка 
ее объектов.
  
  Трансформациями первого порядка, о которых сказано в~формулировке цели 
статьи, называются взаимные преобразования между двумя объектами  
сис\-те\-мы одной природы. Например, перевод в~сис\-те\-ме текста с~русского 
языка на английский относится к~ним. Трансформациями второго порядка 
и~выше называются взаимные преобразования между двумя и~более объектами 
разной природы. Например, кодирование символов текс\-та компьютерными 
кодами и~их декодирование относятся по определению к~трансформациям 
второго порядка.

%\vspace*{-9pt}
  
\section{Процессы трансформаций в~информатике}

%\vspace*{-3pt}

Процессы трансформаций, рассматриваемые в~статье, относятся к~теоретическому ядру информатики, а~не 
только к~проектированию лексикографической информационной сис\-те\-мы. Например, из трех основных 
подходов к~описанию предметной об\-ласти информатики\footnote{В статье предметная область информатики 
трактуется согласно концепции полиадического компьютинга Пола Розенблума~\cite{9-zac}.} (объектный, 
трансформационный и~синтетический) сис\-те\-ма\-ти\-за\-ция трансформаций ближе всего ко второму 
подходу. Примерами первого подхода, в~рамках которого основное внимание уделяется объектам предметной 
области информатики и~в~меньшей степени отношениям\linebreak между ними, могут служить  
работы~\cite{8-zac, 10-zac, 11-zac}; \mbox{примерами} второго подхода, в~рамках которого основное внимание 
уделяется трансформациям и~в~меньшей степени трансформируемым объектам,~---  
работы~\cite{12-zac, 13-zac}; примерами третьего, синтетического подхода, в~котором уделяется внимание 
и~объектам предметной об\-ласти информатики, и~отношениям между ними, могут служить работы~\cite{14-zac, 
15-zac, 16-zac, 17-zac, 18-zac}.

  Таким образом, для описания трансформаций объектов лексикографической 
информационной\linebreak системы предпочтительнее всего трансформационный 
подход, который упоминается и~в определениях информатики. Например, 
в~2009~г.\ П.~Деннинг и~П.~Розенблум сформулировали суть \mbox{информатики} как 
компьютинга следующим образом: <<$\ldots$информатика~--- это не просто 
алгоритмы и~структуры данных; это преобразования [трансформации] 
представлений>>~\cite{12-zac}. Чуть позже, в~контексте краткого описания 
парадигмы информатики как компьютинга, П.~Деннинг и~П.~Фриман изменили 
эту формулировку на такую: <<Центральный объект внимания в~информатике 
можно определить как информационные процессы~--- \textit{естественные или 
искусственные процессы, преобразующие информацию} (курсив мой~--- 
И.\,З.)>>~\cite{13-zac}. Согласно парадигме, предлагаемой авторами этой 
статьи, на начальном этапе проектирования автоматизированных систем 
базовыми элементами моделей их функционирования служат 
\textit{информационные про\-цессы}.
  
  Однако если 15~лет назад в~формулировке из работы~\cite{13-zac} шла речь 
о~процессах, преобразующих информацию, то в~последние~10~лет в~спектр 
процессов трансформаций все чаще стали включать процессы, преобразующие 
не только информацию, но также и~другие объекты автоматизированных 
систем, в~первую очередь данные и~знания~[19--21]. Например, Виктория 
Стодден, позиционируя науку о~данных как одну из дисциплин информатики, 
говорит, что центральный объект исследований в~науке о~данных~--- это 
<<изучение обобщаемого извлечения знания из данных>>~\cite{21-zac}. 
Увеличение и~чис\-ла объектов, и~спект\-ра процессов их трансформаций 
в~автоматизированных сис\-те\-мах обуслов\-ли\-ва\-ет не\-об\-хо\-ди\-мость 
систематизации и~объектов, и~процессов их трансформаций на начальном этапе 
проектирования сис\-тем.
  
  Для создания концепции лексикографической информационной сис\-те\-мы 
и~проектирования ИТ, обеспечивающих интеграцию 
двуязычных словарей и~параллельных корпусов, сначала выполним 
категоризацию на объекты этой сис\-те\-мы трех базовых понятий информатики 
(данные, информация, знание) в~контексте построения классификаций 
сущностей ее предметной об\-ласти.
  
  Необходимость использования классификаций информатики в~процессе 
создания концепции проиллюстрируем, используя иерархию  
Акоффа~\cite{8-zac}. Он использовал принцип их вертикального размещения 
в~иерархии снизу вверх: данные, информация и~знание. Еще в~ней есть термин 
<<мудрость>>, который в~статье не рассматривается. Такое размещение Акофф 
прокомментировал так: <<Каждое из пе\-ре\-чис\-лен\-ных понятий [кроме данных] 
содержит в~себе нижестоящие$\ldots$>>~\cite{8-zac}.
  
  Этому принципу размещения и~комментарию Акоффа свойственны 
недостатки, проанализированные, в~частности, в~работе~\cite{10-zac}. Главный 
вывод, к~которому пришла Роули после изучения иерархии Акоффа, 
заключается в~следующем: <<$\ldots$информация определяется в~терминах 
данных, знание~--- в~терминах информации$\ldots$ но существует меньше 
консенсуса в~описании трансформаций, которые преобразуют сущности, 
расположенные ниже в~иерархии, в~те, которые находятся над ними, что 
приводит к~их терминологической неопределенности>>~\cite{10-zac}. Причина 
этой неопределенности, скорее всего, в~том, что базовые понятия информатики 
включены в~иерархию Акоффа изолированно от общего контекста 
классификаций сущностей ее предметной об\-ласти.

%\vspace*{-9pt}
  
\section{Классификации сущностей информатики}


%\vspace*{-2pt}

  Все сущности предметной области информатики в~работах~[22, 23] 
разделены на два глобальных класса: ее объекты и~их трансформации. Для 
каждого такого класса была предложена своя классификация. 
В~работе~\cite{22-zac} дано описание классификации объектов предметной 
области информатики, первый уровень которой содержит базовые понятия ее 
предметной области (данные, информация, знания и~др.).  
В~работе~\cite{23-zac} дано описание двух верхних уровней классификации 
трансформаций объектов предметной об\-ласти (см.\ рисунок 
в~работе~\cite{23-zac}). Основанием для построения самого верхнего ее уровня послужило деление 
предметной области информатики на среды\footnote{В~работе~\cite{24-zac} дано описание пяти сред 
предметной области информатики (ментальная; сенсорно воспринимаемая, или информационная; 
цифровая; нейро- и~ДНК-среда), каждая из которых по определению включает объекты одной и~той же 
природы.} и~степень разнообразия природы объектов, вовлеченных в~трансформации:
\begin{itemize}
\item  первый класс верхнего уровня классификации включает 
трансформации объектов в~пределах среды только одной природы 
(трансформации первого порядка);
\item  второй класс включает трансформации объектов, относящихся 
к~двум средам разной природы (трансформации второго порядка);
\item третий и~последующие классы включают трансформации объектов, 
относящихся к~трем и~более средам разной природы (трансформации 
третьего и~более высоких порядков).
\end{itemize}

  В работе~\cite{23-zac} были приведены примеры для трех первых классов 
трансформаций, включая пример трансформаций объектов, относящихся 
к~двум средам разной природы (компьютерное кодирование символов текстов 
с~по\-мощью таб\-лиц Unicode).
  
Основанием для построения второго уровня классификации трансформаций объектов послужила типология 
знаковых сис\-тем А.~Соломоника~\cite[c.~131]{25-zac}: естественные знаковые сис\-те\-мы, образные,  
ес\-тест\-вен\-но-язы\-ко\-в$\acute{\mbox{ы}}$е,  
вер\-баль\-но-не\-сло\-вес\-ные сис\-те\-мы записи\footnote{Под системой записи понимается знаковая 
система, сочетающая вербальные знаки с~несловесными (языки нотной записи, карт, таблиц и~др.).} 
и~формализованные знаковые сис\-те\-мы, включая математические. Введем понятие обобщенного текста~--- 
это текст, который может быть создан в~любой из перечисленных знаковых систем. Тогда обобщенные тексты 
могут быть естественными, образными, ес\-тест\-вен\-но-язы\-ко\-в$\acute{\mbox{ы}}$\-ми,  
вер\-баль\-но-не\-сло\-вес\-ны\-ми и~формализованными. Второй уровень классификации трансформаций 
охватывает не все виды объектов предметной  
об\-ласти информатики, а~только перечисленные~5~видов текс\-тов и~их представления, вовлеченные 
в~процессы трансформаций в~одной или более средах вместе с~данными, знанием и~его концептами.

\begin{figure*}[b] %fig1
\vspace*{6pt}
      \begin{center}
     \mbox{%
\epsfxsize=121.191mm 
\epsfbox{zac-1.eps}
}
\end{center}
\vspace*{-6pt}
\Caption{Средовая версия иерархии Акоффа}
\end{figure*}

\section{Классификация трансформаций: построение~третьего 
уровня}

  Основанием для систематизации трансформаций первого и~второго порядка 
на третьем уровне этой классификации служит иерархия Акоффа~\cite{8-zac}, 
на основе которой и~была создана ее средов$\acute{\mbox{а}}$я версия~[26, 
27]. Для создания средов$\acute{\mbox{о}}$й версии была выполнена 
категоризация трех базовых понятий информатики (данные, информация, 
знания) на объекты лексикографической информационной сис\-те\-мы 
в~процессе создания ее концепции\linebreak (рис.~1).
  


  В отличие от классической иерархии Акоффа, в~ее 
средов$\acute{\mbox{о}}$й версии различаются три вида данных: сенсорно 
воспринимаемые, цифровые и~те данные, которые генерируются 
искусственными нейронными сетями (ИНС) в~системах искусственного интеллекта 
(далее~--- ИИ-дан\-ные). Последний вид данных необходим, например, для 
различения входа и~выхода процесса применения обученной 
ИНС в~цифровой модели генерации знания, описанию которой 
посвящена работа~\cite{27-zac}.
  
  Также предлагается различать два вида информации: сенсорно 
воспринимаемая и~цифровая. Кроме знания в~средов$\acute{\mbox{у}}$ю 
версию добавлены концепты и~ментальные образы сенсорно воспринимаемых 
данных. Последние служат промежуточной сущностью между сенсорно 
воспринимаемыми данными и~генерируемым знанием при описании процессов 
извлечения знания из текстовых данных лексикографической информационной 
системы. Описание объектов средов$\acute{\mbox{о}}$й версии иерархии 
Акоффа (см.\ рис.~1) и~отношений между ними дано в~работах~\cite{26-zac, 28-zac}.
  
  В средов$\acute{\mbox{о}}$й версии число объектов равно восьми. Если 
учитывать направления трансформаций, то между восемью объектами на 
рис.~1 она включает~16 их видов (трансформации на границе между сенсорно 
воспринимаемыми данными и~информацией, обозначенные символом~<<?>>, 
в~статье не рас\-смат\-ри\-ва\-ют\-ся). В~будущем число объектов 
в~средов$\acute{\mbox{о}}$й версии, которая выбрана как основание для 
сис\-те\-ма\-ти\-за\-ции трансформаций первого и~второго порядка, может быть 
увеличено. Для построения классификации трансформаций 
важ\-но не возможное увеличение числа объектов 
и~трансформаций между ними, а то, что их виды в~средов$\acute{\mbox{о}}$й 
версии распределены между трансформациями первого и~второго порядка. Из 
16~видов на рис.~1 шесть относятся к~трансформациям первого порядка, это\linebreak 
виды с~номерами~7, 8, 13--16 (далее~--- типология трансформаций первого 
порядка), а~десять~--- к~трансформациям второго порядка, это виды 
с~\mbox{номерами}~1--6 и~9--12 (далее~--- типология трансформаций второго 
порядка). Разместим обе типологии на третьем уровне классификации (см.\ ее 
схему на рис.~2). Перечислим виды трансформаций первой типологии, вводя 
в~скобках их краткие названия, используемые ниже на рис.~3:
  \begin{description}
  \item[\,] 7~--- членение знания на концепты с~помощью одной или нескольких 
знаковых систем (далее~--- членение знания);
  \item[\,] 8~--- формирование знания на основе концептов (формирование 
знания);
  \item[\,] 13~--- обучение ИНС;
  \end{description}
  
  \vspace*{-6pt}
  
  \pagebreak
  
  \end{multicols}
  
  \begin{figure*} %fig2
\vspace*{1pt}
      \begin{center}
     \mbox{%
\epsfxsize=127.513mm 
\epsfbox{zac-2.eps}
}
\end{center}
\vspace*{-9pt}
\Caption{Схема трех верхних уровней классификации трансформаций объектов (объединены 
по три слоя и~для второго, и~для третьего уровней этой классификации)}
\end{figure*}
  
  \begin{multicols}{2}
  
  \noindent
  \begin{description}
  \item[\,] 14~--- восстановление обучающей информации на основе 
содержания обученной ИНС (обращение ИНС);
  \item[\,] 15~--- использование обученной ИНС (использование ИНС);



  \item[\,] 16~--- восстановление исходных данных, соответствующих 
полученным результатам работы обучен\-ной ИНС (восстановление исходных данных 
по результатам ИНС).
  \end{description}
  
  
  Не все виды трансформаций 13--16 поддерживаются в~конкретных системах 
искусственного интеллекта, но с~теоретической точки зрения все их 
предлагается включить в~первую типологию для полноты спектра видов 
трансформаций.
  
  Перечислим виды трансформаций второй типологии:
  \begin{description}
  \item[\,] 1~--- декодирование цифровых данных в~компьютерных системах 
(декодирование данных);
  \item[\,]  2~--- кодирование сенсорно воспринимаемых данных (кодирование 
данных);
  \item[\,] 3~--- ментальное копирование сенсорно воспринимаемых данных 
(ментальное копирование);
  \item[\,] 4~--- восстановление сенсорно воспринимаемых данных по 
ментальным образам (восстановление по образам);
  \item[\,] 5~--- смысловая интерпретация без деления на концепты ментальных 
образов сенсорно воспринимаемых данных (смысловая интерпретация);
  \item[\,] 6~--- восстановление ментальных образов (восстановление образов);
  \item[\,] 9~--- представление концептов в~виде сенсорно воспринимаемой 
информации, например текс\-та\-ми, формулами, таблицами, рисунками и~т.\,д.\ 
(представление концептов);
  \item[\,] 10~--- понимание смысла сенсорно воспринимаемой информации 
(понимание смысла);
  \item[\,] 11~--- кодирование сенсорно воспринимаемой информации 
(кодирование информации);
\end{description}

\vspace*{-6pt}

\pagebreak

\end{multicols}

\begin{figure*} %fig3
\vspace*{1pt}
      \begin{center}
     \mbox{%
\epsfxsize=163mm 
\epsfbox{zac-3.eps}
}
\end{center}
\vspace*{-9pt}
\Caption{Схема частного случая классификации трансформаций объектов (трансформации 
пронумерованы согласно рис.~1)}
\end{figure*}

\begin{multicols}{2}

\noindent
\begin{description}

  \item[\,] 12~--- декодирование цифровой информации (декодирование 
информации).
  \end{description}
  
  Отметим, что в~существующих ИТ
  и~компьютерных системах наиболее часто используются виды 
трансформаций~13 и~15 типологии первого порядка и~1, 2, 11 и~12 типологии 
второго порядка. На рис.~2 в~первом слое третьего уровня классификации 
показаны типологии первого порядка без указания числа трансформаций в~них 
и~без детализации трансформируемых объектов.
  
  Во втором слое третьего уровня классификации условно (без названий) 
показаны типологии второго порядка. Также на рис.~2 в~третьем слое третьего 
уровня классификации условно (также без названий) показаны типологии 
третьего порядка, которые планируется рассмотреть в~отдельной статье. По 
определению они должны включать трансформации между тремя объектами 
разной природы, но средов$\acute{\mbox{а}}$я версия иерархии Акоффа 
включает трансформации только между двумя объектами разной природы. 
Поэтому потребуется другое основание для их систематизации (ранее были 
рассмотрены отдельные примеры трансформаций третьего 
порядка\footnote{Далеко не всегда трансформации третьего и~более высоких порядков можно 
рассматривать как последовательность трансформаций второго порядка. Примером этого могут 
служить трансформации в~процессе обучения пациента пользованию роботизированной рукой, 
охватывающие личностные концепты пациента, релевантные его намерениям, сигналы активности 
мозга как объекты нейросреды и~компьютерные коды~\cite{29-zac}.}~\cite{29-zac}).

\section{Классификация трансформаций: частный~случай}

  Выше было отмечено, что в~будущем число объектов 
в~средов$\acute{\mbox{о}}$й версии иерархии Акоффа может быть увеличено. 
Это означает, что увеличатся и~чис\-ло объектов, и~чис\-ло трансформаций между 
ними в~классификации трансформаций, так как эта средов$\acute{\mbox{а}}$я 
версия служит по определению основанием для систематизации 
трансформаций первого и~второго порядка. Поэтому на третьем уровне рис.~2 
указаны типологии без детализации объектов и~без указания числа 
трансформаций в~каждой из них. С~одной стороны, при таком подходе 
получаем достаточно общий вид этой классификации, так как она не зависит от 
числа объектов в~том или ином варианте средов$\acute{\mbox{о}}$й версии 
(и~это существенно упрощает рис.~2). С~другой стороны, на третьем уровне 
такой общей классификации подразумевается, но не эксплицируется природа 
трансформируемых объектов и~их возможные сочетания в~трансформациях. 

При проектировании лексикографической информационной системы важно 
эксплицировать природу трансформируемых объектов и~их возможные 
сочетания.
  %
  Поэтому в~парадигму информатики~\cite{30-zac} кроме общей 
классификации трансформаций предлагается включать и~ее частные случаи, 
эксплицирующие природу трансформируемых объектов. 

В~этом разделе 
рассмотрим один частный случай, когда используются только естественные 
знаковые сис\-те\-мы из типологии А.~Соломоника~\cite{25-zac} вместе 
с~данными, знанием и~его концептами. Чис\-ло естественных языков при этом не 
ограничено. И~этот частный случай классификации включает только три 
класса природных трансформаций (первого, второго и~третьего порядка, см.\ 
схему классификации на рис.~3).
  
  Первый и~второй уровни схемы общей классификации (см.\ рис.~2) можно 
объединить в~один уровень в~этом частном случае. Ниже этого уровня 
приведено содержание типологий первого и~второго порядка без содержания 
типологий третьего по\-рядка.




  Наполнение типологий первого и~второго порядка соответствует 
средов$\acute{\mbox{о}}$й версии иерархии Акоффа на рис.~1, содержащей 
6~видов трансформаций типологии первого порядка и~10~видов 
трансформаций типологии второго порядка (на рис.~3 стрелки указывают 
направления трансформаций согласно средов$\acute{\mbox{о}}$й версии на рис.~1).
  
  Таким образом, частный случай классификации содержит для этих двух 
типологий 16~теоретически возможных трансформаций, 6 из которых 
в~настоящее время в~существующих ИТ применяются наиболее часто: виды 
трансформаций~1, 2, 11 и~12 типологии второго порядка реализуются 
с~помощью тех или иных методов ко\-ди\-ро\-ва\-ния/де\-ко\-ди\-ро\-ва\-ния 
(например, с~использованием таблиц Unicode), а~виды трансформаций~13 и~15
 в~типологии первого порядка реализуются полностью с~по\-мощью процессов 
цифровой обработки компьютерами.
  
  Остальные виды трансформаций или применяются намного реже (это 
виды~3, 5, 7, 9 и~10), или находятся в~стадии поиска и~разработки (14 и~16) или 
в~настоящее время носят только теоретический характер, обеспечивая полноту 
первой и~второй типологий (4, 6 и~8). Знаком~<<?>> обозначены те виды 
трансформаций, которые по определению не существуют в~используемой 
парадигме информатики~\cite{30-zac}. Однако возможно, что в~других 
будущих подходах к~построению ее парадигмы эти виды трансформаций будут 
существовать.
  
\section{Заключение}

  На сегодняшний день процесс построения классификаций объектов 
предметной области информатики~\cite{22-zac} и~их  
трансформаций~\cite{23-zac} еще не завершен. Однако первые результаты их 
построения уже используются для создания концепции лексикографической 
информационной сис\-те\-мы, обеспечивающей интеграцию двуязычных 
словарей и~параллельных корпусов.
  
  \bigskip
  
  
  Автор признателен рецензентам за помощь в~улучшении статьи.
  
{\small\frenchspacing
 { %\baselineskip=10.6pt
 %\addcontentsline{toc}{section}{References}
 \begin{thebibliography}{99}
\bibitem{1-zac}
\Au{Aijmer K., Altenberg~B.} Advances in corpus-based contrastive linguistics. Studies in honour 
of Stig Johansson.~--- Amsterdam: John Benjamins, 2013. 295~p.  doi: 10.1075/scl.54.
\bibitem{2-zac}
\Au{Добровольский Д.\,О., Кретов~А.\, А., Шаров~С.\,А.} Корпус параллельных текстов~// 
Научная и~техническая информация. Сер.~2: Информационные процессы и~сис\-те\-мы, 2005. 
№\,6. С.~16--27.
\bibitem{3-zac}
\Au{Добровольский Д.\,О.} Корпус параллельных текстов и~сопоставительная 
лексикология~// Труды Института русского языка им.\ В.\,В.~Виноградова, 2015. №\,6. 
С.~413--449. EDN: VJQBHP.
\bibitem{4-zac}
\Au{Гончаров А.\,А., Зацман~И.\,М., Кружков~М.\,Г.} Эволюция классификаций 
в~надкорпусных базах данных~// Информатика и~её применения, 2020. Т.~14. Вып.~4. 
С.~108--116. doi: 10.14357/19922264200415.  
EDN: \mbox{GKWBZT}.
\bibitem{5-zac}
\Au{Гончаров А.\, А., Зацман И. \,М., Кружков~М.\, Г}. Представление новых 
лексикографических знаний в~динамических классификационных сис\-те\-мах~// 
Информатика и~её применения, 2021. Т.~15. Вып.~1. С.~86--93.  doi: 10.14357/19922264210112. EDN: OPEFXW.
\bibitem{6-zac}
\Au{Zatsman I.} Finding and filling lacunas in linguistic typologies~// 15th Forum (International) 
on Knowledge Asset Dynamics Proceedings.~--- Matera, Italy: Institute of Knowledge Asset 
Management, 2020. P.~780--793.
\bibitem{7-zac}
\Au{Zatsman I.} Three-dimensional encoding of emerging meanings in AI-systems~// 21st 
European Conference on Knowledge Management Proceedings.~--- Reading, U.K.: Academic 
Publishing International Ltd., 2020. P.~878--887.
\bibitem{8-zac}
\Au{Ackoff R.} From data to wisdom~// J.~Applied Systems Analysis, 1989. Vol.~16. No.\,1. P.~3--9.
\bibitem{9-zac}
\Au{Rosenbloom P.\,S.} On computing: The fourth great scientific domain.~--- Cambridge, MA, 
USA: MIT Press, 2013. 307~p.
\bibitem{10-zac}
\Au{Rowley J.} The wisdom hierarchy: Representations of the DIKW hierarchy~// J.~Inf. 
Sci., 2007. Vol.~33. Iss.~2. P.~163--180. doi: 10.1177/0165551506070706.
\bibitem{11-zac} 
\Au{Frick$\acute{\mbox{e}}$~M.\,H.} Data--Information--Knowledge--Wisdom (DIKW) pyramid, 
framework, continuum~// Encyclopedia of big data~/ Eds. L.~Schintler, C.~McNeely.~--- Cham: 
Springer, 2018. 4~p. doi: 10.1007/978-3-319-32001-4\_331-1.
\bibitem{12-zac}
\Au{Denning P., Rosenbloom~P.} Computing: The fourth great domain of science~// Commun. 
ACM, 2009. Vol.~52. Iss.~9. P.~27--29.
\bibitem{13-zac}
\Au{Denning P., Freeman~P.} Computing's paradigm~// Commun.  ACM, 2009. Vol.~52. 
Iss.~12. P.~28--30. doi: 10.1145/ 1610252.1610265.
\bibitem{17-zac} %14
\Au{Farradane J.} Knowledge, information, and information science~// J.~Inf. Sci., 
1980. Vol.~2. Iss.~2. P.~75--80. doi: 10.1177/01655515800020020.

\bibitem{15-zac}
\Au{Шрейдер Ю.\,А.} Информация и~знание~// Сис\-тем\-ная концепция информационных 
процессов.~--- М.: ВНИИСИ, 1988. С.~47--52.
\bibitem{16-zac}
\Au{Ingwersen P.} Information and information science~// Enclyclopaedie of library and 
information science~/ Eds. J.\,D.~McDonald, 
M.~Levine-Clark.~--- New York, NY, USA: Marcel Dekker Inc., 1992. Vol.~56. Sup.~19. 
P.~137--174.

\bibitem{14-zac} %17
Информатика как наука об информации: Информационный, документальный, 
технологический, экономический, социальный и~организационный аспекты~/ Под ред. 
Р.\,С.~Гиляревского.~--- М.: Фаир-Пресс, 2006. 592~с.

\bibitem{18-zac}
\Au{Hjorland B.} Library and information science: practice, theory, and philosophical basis~// 
Inform. Process. Manag., 2000. Vol.~36. Iss.~3. P.~501--531. doi:  
10.1016/S0306-\mbox{4573(99)00038-2}.
\bibitem{19-zac}
Deep shift~--- technology tipping points and societal impact.~--- Geneva: WE Forum, 2015. 44~p. 
{\sf http://www3.weforum.org/docs/WEF\_GAC15\_ Technological\_Tipping\_Points\_report\_2015.pdf}.
\bibitem{20-zac}
\Au{Berman F., Rutenbar~R., Hailpern~B., Christensen~H., Davidson~S., Estrin~D., 
Franklin~M., Martonosi~M., Raghavan~P., Stodden~V., Szalay~A.\,S.} Realizing the potential of 
data science~// Commun.  ACM, 2018. Vol.~61. Iss.~4. P.~67--72. doi: 10.1145/3188721.

\bibitem{21-zac}
\Au{Stodden V.} The data science life cycle: A~disciplined approach to advancing data science as 
a~science~// Commun.  ACM, 2020. Vol.~63. Iss.~7. P.~58--66. doi: 10.1145/ 3360646.


\bibitem{23-zac} %22
\Au{Зацман И.\,М.} Научная парадигма информатики: классификация трансформаций 
объектов предметной об\-ласти~// Системы и~средства информатики, 2023. Т.~33. №\,4. 
С.~126--138. doi: 10.14357/08696527230412. EDN: ZIKUWO.

\bibitem{22-zac} %23
\Au{Зацман И.\,М.} Научная парадигма информатики: классификация объектов предметной  
об\-ласти~// Информатика и~её применения, 2023. Т.~17. Вып.~4. С.~96--103. doi: 
10.14357/19922264230413. EDN: FIUQAT.

\bibitem{24-zac}
\Au{Зацман И.\,М.} О~научной парадигме информатики: верхний уровень классификации 
объектов ее предметной об\-ласти~// Информатика и~её применения, 2022. Т.~16. Вып.~4. 
С.~73--79. doi: 10.14357/ 19922264220411. EDN: XZNKVI.

\bibitem{25-zac}
\Au{Соломоник А.\,Б.} Философия знаковых систем и~язык.~--- М.: ЛКИ, 2011. 408~с.
\bibitem{26-zac}
\Au{Зацман И.\,М.} Трансформация иерархии Акоффа в~научной парадигме информатики~// 
Информатика и~её применения, 2023. Т.~17. Вып.~3. С.~107--113. doi: 
10.14357/19922264230315. EDN: UMVRRV.

\bibitem{27-zac}
\Au{Zatsman I.} Building digital spiral models of knowledge generation~// 19th Forum 
(International) on Knowledge Asset Dynamics Proceedings.~--- Matera, Italy: Arts for Business 
Institute, 2024. P.~2185--2196.
\bibitem{28-zac}
\Au{Zatsman I.} Digital spiral model of knowledge creation and encoding its dynamics~// 18th 
Forum (International) on Knowledge Asset Dynamics Proceedings.~--- Matera, Italy: Arts for 
Business Institute, 2023. P.~581--596.
\bibitem{29-zac}
\Au{Зацман И.\,М.} Интерфейсы третьего порядка в~информатике~// Информатика и~её 
применения, 2019. Т.~13. Вып.~3. С.~82--89. doi: 10.14357/19922264190312. EDN: 
EHRQLF.

\bibitem{30-zac}
\Au{Зацман И.\,М.} Научная парадигма информатики как третьей культуры~//  
На\-уч\-но-тех\-ни\-че\-ская информация. Сер.~1: Организация и~методика информационной 
работы, 2023. №\,11. С.~1--14.

\end{thebibliography}

 }
 }

\end{multicols}

\vspace*{-9pt}

\hfill{\small\textit{Поступила в~редакцию 14.04.24}}

\vspace*{4pt}

%\pagebreak

%\newpage

%\vspace*{-28pt}

\hrule

\vspace*{2pt}

\hrule



\def\tit{OBJECT TRANSFORMATIONS OF~THE~FIRST AND~SECOND ORDER
IN~A~LEXICOGRAPHIC INFORMATION SYSTEM\\[-5pt]}


\def\titkol{Object transformations of~the~first and~second order
in~a~lexicographic information system}


\def\aut{I.\,M.~Zatsman}

\def\autkol{I.\,M.~Zatsman}

\titel{\tit}{\aut}{\autkol}{\titkol}

\vspace*{-13pt}


\noindent
Federal Research Center ``Computer Science and Control'' of the Russian Academy of Sciences, 
44-2~Vavilov Str., Moscow 119133, Russian Federation


\def\leftfootline{\small{\textbf{\thepage}
\hfill INFORMATIKA I EE PRIMENENIYA~--- INFORMATICS AND
APPLICATIONS\ \ \ 2024\ \ \ volume~18\ \ \ issue\ 2}
}%
 \def\rightfootline{\small{INFORMATIKA I EE PRIMENENIYA~---
INFORMATICS AND APPLICATIONS\ \ \ 2024\ \ \ volume~18\ \ \ issue\ 2
\hfill \textbf{\thepage}}}

\vspace*{2pt}



\Abste{The theoretical foundations of the design of information technologies used for 
the integration of bilingual dictionaries and parallel corpora are considered. The 
description of the first outcomes of the creation of the third\linebreak\vspace*{-12pt}}

\Abstend{ level of object 
transformations classification in the subject domain of informatics, which is supposed 
to be used
in creating the lexicographic information system providing integration, is 
given. All the entities of informatics are divided into two global classes: objects and 
their transformations. For each such class, its own classification is constructed. 
Previously, the two upper levels of the object transformation classification in the subject 
domain have been described. The present paper discusses the third level of this classification. The 
basis for the construction of its highest level was the division of the subject domain of 
informatics into media (mental, sensory, digital, and a~number of other media), each 
of which by definition includes objects of the same nature. The Solomonick's 
typology of sign systems served as the basis for constructing the second level of the 
object transformation classification. The aim of the paper is to systematize object 
transformations of the first and second orders at the third level of this classification. 
The basis for systematization is the medium version of the Ackoff's hierarchy.}

\KWE{subject domain objects; object transformations; classification; data; 
information; knowledge; lexicographic information system}


\DOI{10.14357/19922264240211}{VZTGVV}

\vspace*{-12pt}

\Ack

\vspace*{-3pt}


\noindent
The reported study was funded by the Russian Science Foundation, project  
No.\,24-18-00155, {\sf 
https://rscf.ru/project/24-18-00155}. The research was carried out using the infrastructure of the Shared 
Research Facilities ``High Performance Computing and Big Data'' (CKP 
``Informatics'') of FRC CSC RAS (Moscow) .
   


  \begin{multicols}{2}

\renewcommand{\bibname}{\protect\rmfamily References}
%\renewcommand{\bibname}{\large\protect\rm References}

{\small\frenchspacing
 {%\baselineskip=10.8pt
 \addcontentsline{toc}{section}{References}
 \begin{thebibliography}{99} 
\bibitem{1-zac-1}
\Aue{Aijmer, K., and B.~Altenberg.} 2013. \textit{Advances in corpus-based 
contrastive linguistics. Studies in honour of Stig Johansson}. Amsterdam: John 
Benjamins. 295~p. doi: 10.1075/scl.54.
\bibitem{2-zac-1}
\Aue{Dobrovolskiy, D.\,O., A.\,A.~Kretov, and S.\,A.~Sharov.} 2005. Korpus 
parallel'nykh tekstov [Corpus of parallel texts]. \textit{Nauchnaya i~tekhnicheskaya 
informatsiya. Ser. 2. Informatsionnye protsessy i~sistemy} [Scientific and Technical 
Information. Ser.~2: Information Processes and Systems] 6:16--27.
\bibitem{3-zac-1}
\Aue{Dobrovolskiy, D.\,O.} 2015. Korpus parallel'nykh tekstov i~sopostavitel'naya 
leksikologiya [The corpus of parallel texts and contrastive lexicology]. \textit{Trudy 
Instituta russkogo yazyka im. V.\,V.~Vinogradova} [Proceedings of the 
V.\,V.~Vinogradov Russian Language Institute] 6:413--449. EDN: VJQBHP.
\bibitem{4-zac-1}
\Aue{Goncharov, A.\,A., I.\,M.~Zatsman, and M.\,G.~Kruzhkov.} 2020. Evolyutsiya 
klassifikatsiy v~nadkorpusnykh ba\-zakh dannykh [Evolution of classifications in 
supracorpora databases]. \textit{Informatika i~ee Primeneniya~--- Inform. \mbox{Appl.}}  
14(4):108--116. doi: 10.14357/19922264200415.  
EDN: GKWBZT.
\bibitem{5-zac-1}
\Aue{Goncharov, A.\,A., I.\,M.~Zatsman, and M.\,G.~Kruzhkov.} 2021. 
Predstavlenie novykh leksikograficheskikh znaniy v~dinamicheskikh 
klassifikatsionnykh sistemakh [Representation of new lexicographical knowledge in 
dynamic classification systems]. \textit{Informatika i~ee Primeneniya~--- Inform. 
Appl.} 15(1):86--93. doi: 10.14357/19922264210112. EDN: OPEFXW.
\bibitem{6-zac-1}
\Aue{Zatsman, I.} 2020. Finding and filling lacunas in linguistic typologies. 
\textit{15th Forum (International) on Knowledge Asset Dynamics Proceedings}. 
Matera, Italy: Institute of Knowledge Asset Management. 780--793.
\bibitem{7-zac-1}
\Aue{Zatsman, I.} 2020. Three-dimensional encoding of emerging meanings in  
AI-systems. \textit{21st European Conference on Knowledge Management 
Proceedings}. Reading, U.K.: Academic Publishing International Ltd. 878--887.
\bibitem{8-zac-1}
\Aue{Ackoff, R.} 1989. From data to wisdom. \textit{J.~Applied Systems Analysis} 
16(1):3--9.
\bibitem{9-zac-1}
\Aue{Rosenbloom, P.\,S.} 2013. \textit{On computing: The fourth great scientific 
domain}. Cambridge, MA: MIT Press. 307~p.
\bibitem{10-zac-1}
\Aue{Rowley, J.} 2007. The wisdom hierarchy: Representations of the DIKW 
hierarchy. \textit{J.~Inf. Sci.} 33(2):163--180. doi: 10.1177/0165551506070706.
\bibitem{11-zac-1}
\Aue{Frick$\acute{\mbox{e}}$, M.\,H.} 2018.  
Data-Information-Knowledge-Wisdom (DIKW) pyramid, framework, continuum. 
\textit{Encyclopedia of big data}. Eds. L.~Schintler and C.~McNeely. Cham: 
Springer. 4~p. doi: 10.1007/978-3-319-32001- 4\_331-1.
\bibitem{12-zac-1}
\Aue{Denning, P., and P.~Rosenbloom.} 2009. Computing: The fourth great domain 
of science. \textit{Commun. ACM} 52(9):27--29.
\bibitem{13-zac-1}
\Aue{Denning, P., and P.~Freeman.} 2009. Computing's paradigm. \textit{Commun. 
ACM} 52(12):28--30. doi: 10.1145/ 1610252.1610265.

\bibitem{17-zac-1} %14
\Aue{Farradane, J.} 1980. Knowledge, information, and information science. 
\textit{J.~Inf. Sci.} 2(2):75--80. doi: 10.1177/ 01655515800020020.

\bibitem{15-zac-1}
\Aue{Shreyder, Yu.\,A.} 1988. Informatsiya i~znanie [Information and knowledge]. 
\textit{Sistemnaya kontseptsiya in\-for\-ma\-tsi\-on\-nykh protsessov} [System concept of 
information processes]. Moscow: VNIISI. 47--52.
\bibitem{16-zac-1}
\Aue{Ingwersen, P.} 1995. Information and information science. 
\textit{Encyclopedia of library and information science}. Eds. J.\,D.~McDonald and 
M.~Levine-Clark. New York, NY: Marcel Dekker Inc. 56(19):137--174.

\bibitem{14-zac-1} %17
Gilyarevskiy, R.\,S., ed. 2006. \textit{Informatika kak nauka ob informatsii: 
informatsionnyy, dokumental'nyy, tekh\-no\-lo\-gi\-che\-skiy, ekonomicheskiy, sotsial'nyy 
i~organizatsionnyy aspekty} [Informatics as information science: Informational, 
documentary, technological, economic, social, and organizational dimensions]. 
Moscow: FAIR-PRESS. 592~p.

\bibitem{18-zac-1}
\Aue{Hjorland, B.} 2000. Library and information science: Practice, theory, and 
philosophical basis. \textit{Inform. Process. Manag.} 36(3):501--531. doi:  
10.1016/S0306-\mbox{4573(99)00038-2}.
\bibitem{19-zac-1}
Deep shift~--- technology tipping points and societal impact. 2015. \textit{World Economic 
Forum}. Geneva. 44~p. Available at: {\sf 
http://www3.weforum.org/docs/WEF\_ GAC15\_Technological\_Tipping\_Points\_report\_2015.pdf} (accessed May~20, 
2024).
\bibitem{20-zac-1}
\Aue{Berman, F., R.~Rutenbar, B.~Hailpern, H.~Christensen, S.~Davidson, 
D.~Estrin, M.~Franklin, M.~Martonosi, P.~Raghavan, V.~Stodden, and 
A.\,S.~Szalay.} 2018. Realizing the potential of data science. \textit{Commun. ACM} 
61(4):67--72. doi: 10.1145/3188721.
\bibitem{21-zac-1}
\Aue{Stodden, V.} 2020. The data science life cycle: A~disciplined approach to 
advancing data science as a~science. \textit{Commun. ACM} 
 63(7):58--66. doi: 10.1145/3360646.

\bibitem{23-zac-1} %22
\Aue{Zatsman, I.\,M.} 2023. Nauchnaya paradigma informatiki: klassifikatsiya 
transformatsiy ob''ektov predmetnoy oblasti [Scientific paradigm of informatics: 
Transformation classification of domain objects]. \textit{Sistemy i~Sredstva 
Informatiki~--- Systems and Means of Informatics} 33(4):126--138. doi: 
10.14357/08696527230412. EDN: ZIKUWO.

\bibitem{22-zac-1} %23
\Aue{Zatsman, I.\,M.} 2023. Nauchnaya paradigma informatiki: klassifikatsiya 
ob''ektov predmetnoy oblasti [Scientific paradigm of informatics: Classification of 
domain objects]. \textit{Informatika i~ee Primeneniya~--- Inform. Appl.} 
 17(4):96--103. doi: 10.14357/19922264230413. EDN: FIUQAT.
 
\bibitem{24-zac-1}
\Aue{   Zatsman, I.\,M.} 2022. O nauchnoy paradigme informatiki: verkhniy uroven' 
klassifikatsii ob''ektov ee predmetnoy oblasti [On the scientific paradigm of 
informatics: The classification high level of its objects]. \textit{Informatika i~ee 
Primeneniya~--- Inform. Appl.} 16(4):73--79. doi: 10.14357/19922264220411. EDN: 
XZNKVI.
\bibitem{25-zac-1}
\Aue{Solomonick, A.\,B.} 2011. \textit{Filosofiya znakovykh system i~yazyk} 
[Philosophy of sign systems and language]. Moscow: LKI. 408~p.
\bibitem{26-zac-1}
\Aue{Zatsman, I.\,M.} 2023. Transformatsiya ierarkhii Akoffa v~nauchnoy 
paradigme informatiki [Transformation of the Ackoff's hierarchy in the scientific 
paradigm of informatics]. \textit{Informatika i~ee Primeneniya~--- Inform. \mbox{Appl.}} 
17(3):107--113. doi: 10.14357/19922264230315. EDN: UMVRRV.
\bibitem{27-zac-1}
\Aue{Zatsman, I.} 2024. Building digital spiral models of knowledge 
generation. \textit{19th Forum (International) on Knowledge Asset Dynamics 
Proceedings}. Matera, Italy: Arts for Business Institute. 2185--2196.
\bibitem{28-zac-1}
\Aue{Zatsman, I.} 2023. Digital spiral model of knowledge creation and encoding its 
dynamics. \textit{18th Forum (International) on Knowledge Asset Dynamics 
Proceedings}. Matera, Italy: Arts for Business Institute. 581--596.
\bibitem{29-zac-1}
\Aue{Zatsman, I.\,M.} 2019. Interfeysy tret'ego poryadka v~informatike 
 [Third-order interfaces in informatics]. \textit{Informatika i~ee Primeneniya~--- 
Inform. Appl.} 13(3):82--89. doi: 10.14357/19922264190312. EDN: EHRQLF.
\bibitem{30-zac-1}
\Aue{Zatsman, I.} 2023. Scientific paradigm of informatics as a~third culture. 
\textit{Scientific Technical Information Processing} 50(4):246--258. doi: 
10.3103/S0147688223040111. EDN: CKHMYS.

\end{thebibliography}

 }
 }

\end{multicols}

\vspace*{-6pt}

\hfill{\small\textit{Received April 14, 2024}} 


\vspace*{-12pt}


\Contrl

\vspace*{-3pt}

\noindent
\textbf{Zatsman Igor M.} (b.\ 1952)~--- Doctor of Science in technology, head of 
department, Federal Research Center ``Computer Science and Control'' of the 
Russian Academy of Sciences, 44-2~Vavilov Str., Moscow 119333, Russian 
Federation; \mbox{izatsman@yandex.ru}





\label{end\stat}

\renewcommand{\bibname}{\protect\rm Литература}  %12



%%%%%%%%%%%%%%%%%%%%%%%%%%%%%%%%%%%%%%%%%%%%%%%

%\def\stat{rez}
{%\hrule\par
%\vskip 7pt % 7pt
\raggedleft\Large \bf%\baselineskip=3.2ex
Р\,Е\,Ц\,Е\,Н\,З\,И\,И \vskip 17pt
    \hrule
    \par
\vskip 6pt plus 6pt minus 3pt }

%\thispagestyle{headings} %с верхним колонтитулом
%\thispagestyle{myheadings} %с нижним колонтитулом, но в верхнем РЕЦЕНЗИИ

\def\tit{НОВАЯ КНИГА И.\,Н.~СИНИЦЫНА, А.\,С.~ШАЛАМОВА <<ЛЕКЦИИ ПО ТЕОРИИ 
ИНТЕГРИРОВАННОЙ ЛОГИСТИЧЕСКОЙ ПОДДЕРЖКИ>> (М.: ТОРУС ПРЕСС, 2012. 624~с.)}

%1
\def\aut{Д.ф.-м.н., профессор С.\,Я.~Шоргин}

\def\auf{\ }

\def\leftkol{\ % РЕЦЕНЗИИ
}

\def\rightkol{ \ } 

%\def\leftkol{\ } % ENGLISH ABSTRACTS}

%\def\rightkol{\ } %ENGLISH ABSTRACTS}

%\def\leftkol{РЕЦЕНЗИИ}

%\def\rightkol{РЕЦЕНЗИИ}

\titele{\tit}{\aut}{\auf}{\leftkol}{\rightkol}
\vspace*{-18pt}


     \label{st\stat}

     \begin{multicols}{2}
     {\small
     {\baselineskip=10.1pt
     

      В книге представлено системное изложение теоретических основ одного из новейших 
направлений в \mbox{об\-ласти} экономики послепродажного обслуживания изделий наукоемкой 
продукции (ИНП) длительного пользования~--- интегрированной логистической поддержки
(ИЛП). 
{\looseness=1

}

Приведены также результаты новых работ, выполненных в Институте проблем информатики 
Российской академии наук в рамках научного направления <<Информационные технологии и 
анализ сложных сис\-тем>>.
 {%\looseness=1

}
     
      Излагаемые в книге научные подходы позво\-ляют карди\-наль\-но реформировать 
существующие системы производства и эксплуатации ИНП путем создания и внед\-ре\-ния 
методов рационального и оптимального управ\-ле\-ния процессами расходования 
вре\-мен\-н$\acute{\mbox{ы}}$х, 
мате\-ри\-аль\-ных, трудовых и других ресурсов на всех стадиях жизненного цикла изделий (ЖЦИ) по 
критериям экономической целесообразности и эф\-фек\-тив\-ности.
  {\looseness=1

}
    
      В книге приведен краткий обзор причин возник\-новения и
      развития CALS-методологии как основы 
современных международных стандартов по созданию и функционированию глобальных 
ин\-фор\-ма\-ци\-он\-но-ком\-му\-ни\-ка\-ци\-он\-ных систем, ее ключевых возможностей и эффективности 
результатов ее использования. 
Авторы %\linebreak 
предлагают ряд научных обоснований для разработки 
единой теории проектирования и управления систем ИЛП для полноценного использования 
преимуществ %\linebreak
 суще\-ст\-ву\-ющей методологии, определяют \mbox{общую} структурную схему 
комплексной системы <<ИНП-СППО>> и необходимость разработки для ее описания 
гибридных стохастических моделей.
{%\looseness=1

}

%\columnbreak
      
      Книга состоит из пяти частей, где последовательно излагается материал по каждой из 
следующих тем: <<Интегрированная логистическая поддержка>>, <<Теория гибридных 
стохастических систем и компьютерная поддержка исследований и разработок>>, <<Основы 
математического моделирования, анализа и синтеза систем послепродажного обслуживания>>, 
<<Определение и анализ показателей экспортного потенциала ИНП при проектировании>>, 
<<Задачи управления поддержкой послепродажного обслуживания>>, а также 
<<Моделирование инвестиционных процессов ИЛП в условиях неравновесных финансовых 
рынков>>. 
   
      В конце каждой главы приведены выводы и даны вопросы и задания для 
самоконтроля. В~приложениях содержатся основные определения по программам работ по 
анализу ИЛП, логистическим базам данных и компьютерным решениям, эквивалентной статистической 
линеаризации нелинейных преобразований ИЛП, справочный материал, а также развернутые 
уравнения для вероятностных характеристик.


      \def\leftkol{РЕЦЕНЗИИ}

\def\rightkol{РЕЦЕНЗИИ} 

      
      Книга заинтересует широкий круг специалистов и может быть использована научными 
проектными организациями в сфере промышленного производства ИНП. Большое количество 
иллюстраций, примеров и вопросов, обращенных к читателю, позволяет использовать книгу 
также в качестве учебного пособия для студентов и аспирантов машиностроительных, 
транспортных и~других специальностей, а также для самостоятельного изучения. 
{%\looseness=-1

}

Книга 
представляет несомненный интерес для специалистов и студентов в области прикладной 
математики и информатики.
    

}

}
\end{multicols}

%\newpage

\def\stat{authorsrus}
{%\hrule\par
%\vskip 7pt % 7pt
\raggedleft\Large \bf%\baselineskip=3.2ex
О\,Б\ \ А\,В\,Т\,О\,Р\,А\,Х \vskip 17pt
    \hrule
    \par
\vskip 21pt plus 8pt minus 4pt }


\def\tit{\ }

\def\aut{\ }

\def\auf{\ }

\def\leftkol{\ } % ENGLISH ABSTRACTS}

\def\rightkol{ОБ АВТОРАХ} %ENGLISH ABSTRACTS}

\titele{\tit}{\aut}{\auf}{\leftkol}{\rightkol}
      
            \label{st\stat}



\vspace*{24pt}

\begin{multicols}{2}




\noindent
\textbf{Архипов Олег Петрович} (р.\ 1948)~---
кандидат технических наук, директор Орловского филиала Института проб\-лем информатики
Российской академии наук
%302025, г.Орел, Московское шоссе, д.137

\vspace*{3pt}

\noindent
\textbf{Бирюкова Татьяна Константиновна} (р.\ 1968)~---
кандидат фи\-зи\-ко-ма\-те\-ма\-ти\-че\-ских наук, старший научный сотрудник Института проб\-лем информатики
Российской академии наук

\vspace*{3pt}

\noindent 
\textbf{Бобков  Сергей Геннадьевич} (р.\ 1955)~---
доктор технических наук,  заведующий отделением На\-уч\-но-ис\-сле\-до\-ва\-тель\-ско\-го 
института системных исследований Российской академии наук
%117218, Москва, Нахимовский просп., 36, к.1 

\vspace*{3pt}

\noindent \textbf{Васильев Николай Семенович} (р.\ 1952)~--- доктор 
фи\-зи\-ко-ма\-те\-ма\-ти\-че\-ских наук, профессор, 
МГТУ им.\ Н.\,Э.~Баумана 
%, Москва 105005, 2-я Бауманская ул., д.~5,

\vspace*{3pt}

\noindent
\textbf{Гершкович Максим Михайлович} (р.\ 1968)~---
старший научный сотрудник Института проб\-лем информатики
Российской академии наук

\vspace*{3pt}

\noindent 
\textbf{Дьяченко Юрий Георгиевич} (р.\ 1958)~--- кандидат технических наук, 
старший научный сотрудник Института проб\-лем информатики
Российской академии наук

\vspace*{3pt}

\noindent 
\textbf{Ерошенко Александр Андреевич} (р.\ 1989)~--- аспирант кафедры 
математической статистики факультета вычисли\-тельной математики и кибернетики 
Московского государственного университета им.\ М.\,В.~Ломоносова
%119991, Москва ГСП-1, Ленинские горы, д.\ 1, стр. 52

\vspace*{3pt}
 
\noindent 
\textbf{Захаров Виктор Николаевич} (р.\ 1948)~--- 
доктор технических наук, доцент, ученый секретарь Института проб\-лем информатики
Российской академии наук

\vspace*{3pt}

\noindent
\textbf{Зейфман Александр Израилевич} (р.\ 1954)~---
доктор фи\-зи\-ко-ма\-те\-ма\-ти\-че\-ских наук, профессор, 
заведующий кафедрой Вологодского государственного университета; 
старший научный сотрудник Института проб\-лем информатики
Российской академии наук; главный научный сотрудник ИСЭРТ Российской академии наук

\vspace*{3pt}

\noindent
\textbf{Зыкин Сергей Владимирович} (р.\ 1959)~--- 
доктор технических наук, профессор, заведующий лабораторией Института математики 
им.\ С.\,Л.~Соболева Сибирского отделения Российской академии наук, Новосибирск 
%630090, пр.\ ак.\ Коптюга, 4 

\vspace*{4pt}

\noindent
\textbf{Киреев Владимир Иванович} (р.\ 1938)~---
доктор фи\-зи\-ко-ма\-те\-ма\-ти\-че\-ских наук, профессор Московского 
государственного горного университета
%Адрес: Россия, 119991, г. Москва, Ленинский проспект, д. 6

%\columnbreak

\vspace*{4pt}

\noindent
\textbf{Козеренко Елена Борисовна} (р.\ 1959)~---
кандидат филологических наук, заведующая лабораторией Института проб\-лем информатики
Российской академии наук

\vspace*{4pt}

\noindent
\textbf{Королев Виктор Юрьевич} (р.\ 1954)~--- доктор
фи\-зи\-ко-ма\-те\-ма\-ти\-че\-ских наук, профессор кафедры математической 
статистики факультета вычисли\-тельной математики и кибернетики 
Московского государственного университета; 
ведущий научный сотрудник Института проб\-лем информатики
Российской академии наук

\vspace*{4pt}

\noindent
\textbf{Коротышева Анна Владимировна} (р.\ 1988)~---
старший преподаватель Вологодского государственного университета

\vspace*{4pt}

\noindent 
\textbf{Кун Де Турк} (р.\ 1981)~--- научный сотрудник 
исследовательской группы SMACS факультета телекоммуникаций и обработки информации
Университета Гента, Бельгия
%В-9000 Гент, Бельгия

\vspace*{4pt}

\noindent
\textbf{Лупенцов Олег Сергеевич} (р.\ 1986)~---
аспирант Омского государственного института сервиса
%Омск 644043, ул.\ Певцова 13

\vspace*{4pt}

\noindent
\textbf{Лучко Олег Николаевич} (р.\ 1961)~---
кандидат педагогических наук, профессор, заведующий кафедрой 
Омского государственного института сервиса
%Омск 644043, ул.\ Певцова 13

\vspace*{4pt}

\noindent
\textbf{Малашенко Юрий Евгеньевич} (р.\ 1946)~---
доктор фи\-зи\-ко-ма\-те\-ма\-ти\-че\-ских наук, заведующий сектором 
Вычислительного центра им.\ А.\,А.~Дородницына Российской академии наук
%Адрес: 119333, Москва, ул. Вавилова, 40,

\vspace*{4pt}

\noindent
\textbf{Маньяков Юрий Анатольевич} (р.\ 1984)~---
кандидат технических наук, научный сотрудник Орловского филиала Института проб\-лем информатики
Российской академии наук
%302025, г.Орел, Московское шоссе, д.137

\vspace*{4pt}

\noindent
\textbf{Маренко Валентина Афанасьевна} (р.\ 1951)~---
кандидат технических наук, доцент, старший научный сотрудник 
Института математики им.\ С.\,Л.~Соболева Сибирского отделения Российской академии наук
%Новосибирск 630090, пр. ак. Коптюга, 4 

\vspace*{3pt}

\noindent 
\textbf{Морозов Евсей Викторович} (р.\ 1947)~--- доктор 
фи\-зи\-ко-ма\-те\-ма\-ти\-че\-ских, профессор, ведущий научный сотрудник 
Института прикладных математических исследований Карельского научного центра Российской
академии наук; 
%%185910 Россия, Республика Карелия, г.\ Петрозаводск, ул.\ Пушкинская, 11
профессор Петрозаводского государственного университета, Петрозаводск
%185910 Россия, Республика Карелия, г.\ Петрозаводск, пр.\ Ленина, 33

%\pagebreak

\vspace*{3pt}

\noindent
\textbf{Назарова Ирина Александровна} (р.\ 1966)~---
кандидат фи\-зи\-ко-ма\-те\-ма\-ти\-че\-ских наук, 
научный сотрудник Вычислительного центра им.\ А.\,А.~Дородницына Российской академии наук 
%Адрес: 119333, Москва, ул. Вавилова, 40

\vspace*{3pt}

\noindent
\textbf{Павлов Игорь Валерианович} (р.\ 1945)~--- 
доктор фи\-зи\-ко-ма\-те\-ма\-ти\-че\-ских наук, профессор МГТУ им.\ Н.\,Э.~Баумана 
%Москва 105005, 2-я Бауманская ул., д.~5 

%\pagebreak

\vspace*{3pt}

\noindent 
\textbf{Потахина Любовь Викторовна} (р.\ 1989)~--- аспирантка
Института прикладных математических исследований Карельского научного центра
Российской академии наук; 
%%185910 Россия, Республика Карелия, г.\ Петрозаводск, ул.\ Пушкинская, 11
инженер Петрозаводского государственного университета, Петрозаводск
%185910 Россия, Республика Карелия, г.\ Петрозаводск, пр.\ Ленина, 33

\vspace*{3pt}

\noindent 
\textbf{Рождественский Юрий Владимирович} (р.\ 1952)~--- 
кандидат технических наук, заведующий сектором Института проб\-лем информатики
Российской академии наук

\vspace*{3pt}

\noindent 
\textbf{Синицын Игорь Николаевич} (р.\ 1940)~--- доктор технических наук,
профессор, заслуженный деятель\linebreak\vspace*{-12pt}

\columnbreak

\noindent
 науки РФ, заведующий отделом Института проб\-лем информатики
Российской академии наук

\vspace*{7pt}


\noindent
\textbf{Сиротинин Денис Олегович} (р.\ 1984)~---
кандидат технических наук, научный сотрудник Орловского филиала Института проб\-лем информатики
Российской академии наук
%302025, г.Орел, Московское шоссе, д.137

\vspace*{7pt}

%\columnbreak

\noindent 
\textbf{Соколов  Игорь Анатольевич} (р.\ 1954)~--- академик (действительный член) Российской 
академии наук, доктор технических наук, директор Института проб\-лем информатики
Российской академии наук

\vspace*{7pt}

\noindent
\textbf{Степченков Юрий Афанасьевич} (р.\ 1951)~---
кандидат технических наук, заведующий отделом Института проб\-лем информатики
Российской академии наук

\vspace*{7pt}

\noindent
\textbf{Сурков Алексей Викторович} (р.\ 1978)~--- 
старший научный сотрудник На\-уч\-но-ис\-сле\-до\-ва\-тель\-ско\-го 
института системных исследований Российской академии наук
%117218, Москва, Нахимовский просп., 36, к.1 

\vspace*{7pt}

\noindent 
\textbf{Шестаков Олег Владимирович} (р.\ 1976)~--- доктор 
фи\-зи\-ко-ма\-те\-ма\-ти\-че\-ских, доцент кафедры математической статистики 
факультета вычисли\-тельной математики и кибернетики Московского 
государственного университета им.\ М.\,В.~Ломоносова; 
%119991, Москва ГСП-1, Ленинские горы, д.\ 1, стр. 52
старший научный сотрудник Института проб\-лем информатики
Российской академии наук
%, Москва 119333, ул. Вавилова, д.~44, корп.~2

\vspace*{7pt}

\noindent 
\textbf{Шоргин Сергей Яковлевич} (р.\ 1952.)~--- доктор
фи\-зи\-ко-ма\-те\-ма\-ти\-че\-ских наук, профессор, заместитель директора Института 
проб\-лем информатики Российской академии наук





%%%%%%%%%%%%%%%%%%%%%%%%%%%%%%%%%%%%%%%%%%%%%%%%%%%%%%%%%%%%%%%%%%%%%%%%%%%%%%%




%\def\rightkol{ОБ АВТОРАХ}
%\def\leftkol{ОБ АВТОРАХ}

 \label{end\stat}





%\def\leftfootline{\small{\textbf{\thepage}
%\hfill ИНФОРМАТИКА И ЕЁ ПРИМЕНЕНИЯ\ \ \ том~7\ \ \ выпуск~1\ \ \ 2013}
%}%
% \def\rightfootline{\small{ИНФОРМАТИКА И ЕЁ ПРИМЕНЕНИЯ\ \ \ том~7\ \ \ выпуск~1\ \ \ 2013
%\hfill \textbf{\thepage}}}


%\thispagestyle{myheadings}



\end{multicols}

\newpage

%\end{document}

%
\def\stat{rekl}
%\label{preobr}

%\def\tit{АКАДЕМИК ПУГАЧЁВ  ВЛАДИМИР СЕМЁНОВИЧ\\
%25.03.1911--25.03.1998}


%   \vspace*{-48pt}
%   \begin{center}\LARGE
%Академик Пугачёв  Владимир Семёнович\\ (25.03.1911--25.03.1998)
%   \end{center}

   %\vspace*{2.5mm}

   \begin{center}

{\prgsh\LARGE
ЮБИЛЕИ}

\end{center}
%\hrule

\vspace*{6pt}


   \vspace*{8mm}

   \thispagestyle{empty}


%\def\stat{emel}


\section*{К 70-летию заместителя директора ИПИ РАН,\\ члена редколлегии журнала
<<Информатика и её применения>>\\ доктора технических наук В.\,И.~Будзко}

\vspace*{18pt}




          \begin{multicols}{2}

%            \label{st\stat}

\begin{center}
\vspace*{1pt}
\mbox{%
\epsfxsize=78mm
\epsfbox{bud-1.eps}
}
\end{center}

\vspace*{12pt}

      14 августа 2014~г.\ исполнилось 70~лет за\-мес\-ти\-те\-лю директора ИПИ РАН по
научной работе доктору технических наук Владимиру Игоревичу Будзко.

      Владимир Игоревич Будзко родился в г.~Москве. Высшее образование получил на факультете
элект\-рон\-но-вы\-чис\-ли\-тель\-ных устройств в Московском
ин\-же\-нер\-но-фи\-зи\-че\-ском институте
(МИФИ), который он окончил в 1968~г., после чего был на\-прав\-лен для прохождения
службы в одну из войс\-ко\-вых частей, где прошел путь от инженера до первого заместителя
командира войсковой части.

      С приходом В.\,И.~Будзко в ИПИ РАН (2001~г.)\ в институте
сформировалось новое научное на\-прав\-ле\-ние теоретических исследований~--- <<Постро\-ение
ин\-фор\-ма\-ци\-он\-но-те\-ле\-ком\-му\-ни\-ка\-ци\-он\-ных\linebreak сис\-тем
высокой до\-ступ\-ности>>. В~рамках этого
направления выполнен широкий круг фундаментальных исследований по поиску подходов и
определению принципов построения средств обеспечения доступности, конфиденциальности
и целостности современных крупномасштабных
ин\-фор\-ма\-ци\-он\-но-те\-ле\-ком\-му\-ни\-ка\-ци\-он\-ных
сис\-тем (ИТС). Разработаны основные сис\-тем\-но-тех\-ни\-че\-ские принципы и базовые
архитектурные решения построения перспективных для условий России ИТС с
централизованной обработкой и хранением информации, сочетающих в себе свойства
высокой доступности, отказо- и катастрофоустойчивости, информационной защищенности.
Определены принципы, методы и математические основы рационального построения и
оптимизации средств восстановления функционирования центров обработки данных (ЦОД)
после возникновения отказов и катастроф, передачи и хранения данных, обеспечения
информационной безопасности при достижении минимальной совокупной стоимости
владения такими системами. Результаты нашли практическое воплощение при реализации
проектов в интересах ряда отечественных государственных и негосударственных
организаций, таких как Банк России (БР), Внешторгбанк, ОАО <<ГМК <<Норильский Никель>>,
<<Газпром>>, Минэкономразвития России, Правительство Москвы, а также ряд силовых
ведомств.

      Под руководством В.\,И.~Будзко начиная с 2001~г.\ выполнен комплекс
      на\-уч\-но-ис\-сле\-до\-ва\-тель\-ских и
      опыт\-но-кон\-ст\-рук\-тор\-ских работ (свыше 100~проектов),
направленных на развитие электронной информационной технологии БР.
Разработаны концепции развития ИТС БР сначала до 2008~г., а затем до 2013~г., которые
были приняты в качестве основы проведения технической политики. За реализацию проекта
<<Катастрофоустойчивая тер\-ри\-то\-ри\-аль\-но-рас\-пре\-де\-лен\-ная
      ин\-фор\-ма\-ци\-он\-но-те\-ле\-ком\-му\-ни\-ка\-ци\-он\-ная сис\-те\-ма централизованной
обработки банковской информации>> В.\,И.~Будзко удостоен Премии Правительства РФ в
области науки и техники за 2010~г.

      В.\,И.~Будзко возглавлял и возглавляет работы по ряду других прикладных проектов,
связанных с созданием, совершенствованием и развитием крупномасштабных ИТС.

      В.\,И.~Будзко~--- генерал-майор, доктор технических наук, член-кор\-рес\-пон\-дент
Академии криптографии РФ, известный ученый в области информатики и применения
информационных технологий при построении территориально распределенных ИТС
различного назначения. Является автором свыше 250~научных работ, опубликованных в
на\-уч\-но-тех\-ни\-че\-ских и специальных изданиях.

    \thispagestyle{empty}

      В.\,И.~Будзко уделяет большое внимание подготовке научных кадров. Под его
руководством защищено 6~диссертаций на соискание ученой степени кандидата
технических наук. Свыше 30~лет он читает лекции в ИКСИ Академии ФСБ, профессор
кафедры НИЯУ МИФИ. Является членом двух диссертационных советов, главным
редактором журнала <<Системы высокой доступности>> и членом редколлегии журнала
<<Информатика и её применения>>.

      \bigskip

      Редакционный совет и Редакционная коллегия журнала <<Информатика и её
применения>> сердечно поздравляют Владимира Игоревича Будзко с 70-ле\-ти\-ем и желают
крепкого здоровья и новых научных достижений.

\end{multicols}

%\def\stat{cont}
{%\hrule\par
%\vskip 7pt % 7pt
\raggedleft\Large \bf%\baselineskip=3.2ex
А\,В\,Т\,О\,Р\,С\,К\,И\,Й\ \ У\,К\,А\,З\,А\,Т\,Е\,Л\,Ь\ \ З\,А\ \ 2\,0\,1\,0 г. \vskip 17pt
    \hrule
    \par
\vskip 21pt plus 6pt minus 3pt }

\label{st\stat}

\def\tit{\ }

\def\aut{\ }
\def\auf{\ }

\def\leftkol{\ } % ENGLISH ABSTRACTS}

\def\rightkol{\ } %АВТОРСКИЙ УКАЗАТЕЛЬ ЗА 2010 г.} %ENGLISH ABSTRACTS}

\titele{\tit}{\aut}{\auf}{\leftkol}{\rightkol}

\vspace*{-12pt}

{\tabcolsep=3pt
\begin{tabular}{p{388pt}rr}
&\textbf{Выпуск} & \textbf{Стр.}\\[6pt]
\hangindent=23pt\noindent\textbf{Арутюнян~А.\,Р.} Моделирование влияния деформаций отпечатков пальцев на 
точность\linebreak
\vspace*{-12pt}\\
\hspace*{23pt}дактилоскопической идентификации$\dotfill$&1&51\\
\hangindent=23pt\noindent\textbf{Архипов~О.\,П., Зыкова~З.\,П.} Интеграция гетерогенной информации о цветных 
пикселях\linebreak
\vspace*{-12pt}\\
\hspace*{23pt}и их цветовосприятии$\dotfill$&4&15\\
\hangindent=23pt\noindent\textbf{Баранов~С.\,И., Френкель~С.\,Л., Захаров~В.\,Н.} Полуформальная верификация 
цифрового устройства с конвейером, основанная на использовании алгоритмических машин\linebreak
\vspace*{-12pt}\\
\hspace*{23pt}состояния$\dotfill$&4&49\\
\textbf{Бекетова~И.\,В.} см.~Каратеев~С.\,Л.&&\\
\textbf{Белоусов~В.\,В.} см.~Синицын~И.\,Н.&&\\
\hangindent=23pt\noindent\textbf{Бенинг~В.\,Е., Королев~Р.\,А.} О предельном поведении мощностей критериев в 
случае\linebreak
\vspace*{-12pt}\\
\hspace*{23pt}распределения Лапласа$\dotfill$&2&63\\
\hangindent=23pt\noindent\textbf{Бенинг~В.\,Е., Сипина~А.\,В.} Асимптотическое разложение для мощности 
критерия,\linebreak
\vspace*{-12pt}\\
\hspace*{23pt}основанного на выборочной медиане, в случае распределения Лапласа$\dotfill$&1&18\\
\textbf{Бондаренко~А.\,В.} см.~Каратеев~С.\,Л.&&\\
\hangindent=23pt\noindent\textbf{Бородина~А.\,В., Морозов~Е.\,В.} Об оценивании асимптотики вероятности 
большого\linebreak
\vspace*{-12pt}\\
\hspace*{23pt}уклонения стационарной регенеративной очереди с одним прибором$\dotfill$&3&29\\
\hangindent=23pt\noindent\textbf{Бунтман~Н.\,В., Минель~Ж.-Л., Ле~Пезан~Д., Зацман~И.\,М.} Типология и 
компьютерное\linebreak
\vspace*{-12pt}\\
\hspace*{23pt}моделирование трудностей перевода$\dotfill$&3&77\\
\textbf{Визильтер~Ю.\,В.} см.~Каратеев~С.\,Л.&&\\
\hangindent=23pt\noindent\textbf{Гавриленко~С.\,В.} Оценки скорости сходимости распределений случайных сумм с 
безгранично делимыми индексами к нормальному закону$\dotfill$&4&81\\
\hangindent=23pt\noindent\textbf{Григорьева~М.\,Е., Шевцова~И.\,Г.} Уточнение неравенства 
Каца--Берри--Эссеена$\dotfill$&2&75\\
\hangindent=23pt\noindent\textbf{Грушо~А.\,А., Грушо~Н.\,А., Тимонина~Е.\,Е.} Поиск конфликтов в политиках 
безопасности: модель случайных графов$\dotfill$&3&38\\
\textbf{Грушо~Н.\,А.} см.~Грушо~А.\,А.&&\\
\hangindent=23pt\noindent\textbf{Гудков~В.\,Ю.} Математические модели изображения отпечатка пальца на основе 
описания линий$\dotfill$&1&58\\
\textbf{Гуртов~А.\,В.} см.~Лукьяненко~А.\,С.&&\\
\textbf{Желтов~С.\,Ю.} см.~Каратеев~С.\,Л.&&\\
\hangindent=23pt\noindent\textbf{Захаров~А.\,А., Серебряков~В.\,А.} Система управления электронной библиотекой 
LibMeta$\dotfill$&4&2\\
\textbf{Захаров~В.\,Н.} см.~Баранов~С.\,И.&&\\
\textbf{Захарова~Т.\,В.} см.~Матвеева~С.\,С.&&\\
\hangindent=23pt\noindent\textbf{Зацаринный~А.\,А., Чупраков~К.\,Г.} Некоторые аспекты выбора технологии для 
постро-\linebreak
\vspace*{-12pt}\\
\hspace*{23pt}ения систем отображения информации ситуационного центра$\dotfill$&3&59\\
\textbf{Зацман~И.\,М.} см.~Бунтман~Н.\,В.&&\\
\hangindent=23pt\noindent\textbf{Зейфман~А.\,И., Коротышева~А.\,В., Сатин~Я.\,А., Шоргин~С.\,Я.} Об 
устойчивости нестаци-\linebreak
\vspace*{-12pt}\\
\hspace*{23pt}онарных систем обслуживания с катастрофами$\dotfill$&3&9\\
\textbf{Зыкова~З.\,П.} см.~Архипов~О.\,П.&&\\
\hangindent=23pt\noindent\textbf{Илюшин~Г.\,Я., Соколов~И.\,А.} Организация управляемого доступа пользователей 
к\linebreak
\vspace*{-12pt}\\
\hspace*{23pt}разнородным ведомственным информационным ресурсам$\dotfill$&1&24\\
\hangindent=23pt\noindent\textbf{Кавагучи~Ю., Ульянов~В.\,В., Фуджикоши~Я.} Приближения для статистик, 
описывающих\linebreak
\vspace*{-12pt}\\
\hspace*{23pt}геометрические свойства данных большой размерности, с оценками 
ошибок$\dotfill$&1&12\\
\hangindent=23pt\noindent\textbf{Каратеев~С.\,Л., Бекетова~И.\,В., Ососков~М.\,В., Князь~В.\,А., 
Визильтер~Ю.\,В., Бондаренко~А.\,В., Желтов~С.\,Ю.} Автоматизированный контроль 
качества цифровых\linebreak
\vspace*{-12pt}\\
\hspace*{23pt}изображений для персональных документов$\dotfill$&1&65\\
\end{tabular}
}

\pagebreak

\def\leftkol{АВТОРСКИЙ УКАЗАТЕЛЬ ЗА 2010 г.} % ENGLISH ABSTRACTS}

\def\rightkol{АВТОРСКИЙ УКАЗАТЕЛЬ ЗА 2010 г.} %ENGLISH ABSTRACTS}

{\tabcolsep=3pt
\begin{tabular}{p{388pt}rr}
&\textbf{Выпуск} & \textbf{Стр.}\\[3pt]
\hangindent=23pt\noindent\textbf{Козеренко~Е.\,Б.} Лингвистические фильтры в статистических моделях машинного\linebreak
\vspace*{-12pt}\\
\hspace*{23pt}перевода$\dotfill$&2&83\\
\hangindent=23pt\noindent\textbf{Козеренко~Е.\,Б., Кузнецов~И.\,П.} Когнитивно-лингвистические представления в 
систе-\linebreak
\vspace*{-12pt}\\
\hspace*{23pt}мах обработки текстов$\dotfill$&3&69\\
\textbf{Князь~В.\,А.} см.~Каратеев~С.\,Л.&&\\
\hangindent=23pt\noindent\textbf{Колесников~А.\,В., Солдатов~С.\,А.} Алгоритм координации для гибридной 
интеллектуальной системы решения сложной задачи оперативно-производственного\linebreak
\vspace*{-12pt}\\
\hspace*{23pt}планирования$\dotfill$&4&61\\
\hangindent=23pt\noindent\textbf{Коновалов~М.\,Г.} О планировании потоков в системах вычислительных 
ресурсов$\dotfill$&2&3\\
\textbf{Конушин~А.\,С.} см.~Конушин~В.\,С.&&\\
\hangindent=23pt\noindent\textbf{Конушин~В.\,С., Кривовязь~Г.\,Р., Конушин~А.\,С.} Алгоритм распознавания людей 
в видео-\linebreak
\vspace*{-12pt}\\
\hspace*{23pt}последовательности по одежде$\dotfill$&1&74\\
\textbf{Корепанов~Э.\, Р.} см.~Синицын~И.\,Н.&&\\
\textbf{Королев~В.\,Ю.} см.~Соколов~И.\,А.&&\\
\textbf{Королев~Р.\,А.} см.~Бенинг~В.\,Е.&&\\
\textbf{Коротышева~А.\,В.} см.~Зейфман~А.\,И.&&\\
\hangindent=23pt\noindent\textbf{Кривенко~М.\,П.} Непараметрическое оценивание элементов байесовского 
клас\-си-\linebreak
\vspace*{-12pt}\\
\hspace*{23pt}фикатора$\dotfill$&2&13\\
\textbf{Кривовязь~Г.\,Р.} см.~Конушин~В.\,С.&&\\
\textbf{Крылов~А.\,С.} см.~Павельева~Е.\,А.&&\\
\hangindent=23pt\noindent\textbf{Крылов~В.\,А.} Моделирование и классификация многоканальных дистанционных\linebreak
\vspace*{-12pt}\\
\hspace*{23pt}изображений с использованием копул$\dotfill$&4&34\\
\hangindent=23pt\noindent\textbf{Крючин~О.\,В.} Разработка параллельных эвристических алгоритмов подбора 
весовых\linebreak
\vspace*{-12pt}\\
\hspace*{23pt}коэффициентов искусственной нейтронной сети$\dotfill$&2&53\\
\hangindent=23pt\noindent\textbf{Кудрявцев~А.\,А., Шоргин~С.\,Я.} Байесовские модели массового обслуживания и 
надеж-\linebreak
\vspace*{-12pt}\\
\hspace*{23pt}ности: характеристики среднего числа заявок в системе $M\vert M \vert 1\vert 
\infty$$\dotfill$&3&16\\
\hangindent=23pt\noindent\textbf{Кузнецов~А.\,А.} Связь между временными и структурно-топологическими 
характери-\linebreak
\vspace*{-12pt}\\
\hspace*{23pt}стиками диаграмм ритма сердца здоровых людей$\dotfill$&4&39\\
\textbf{Кузнецов~И.\,П.} см.~Козеренко~Е.\,Б.&&\\
\textbf{Ле~Пезан~Д.} см.~Бунтман~Н.\,В.&&\\
\hangindent=23pt\noindent\textbf{Лукьяненко~А.\,С., Морозов~Е.\,В., Гуртов~А.\,В.} Анализ сетевого протокола с общей 
функ-\linebreak
\vspace*{-12pt}\\
\hspace*{23pt}цией расширения окна передачи сообщения при конфликтах$\dotfill$&2&46\\
\hangindent=23pt\noindent\textbf{Лямин~О.\,О.} О предельном поведении мощностей критериев в случае обобщенного\linebreak
\vspace*{-12pt}\\
\hspace*{23pt}распределения Лапласа$\dotfill$&3&47\\
\hangindent=23pt\noindent\textbf{Маркин~А.\,В., Шестаков~О.\,В.} Асимптотики оценки риска при пороговой 
обработке\linebreak
\vspace*{-12pt}\\
\hspace*{23pt}вейвлет-вейглет коэффициентов в задаче томографии$\dotfill$&2&36\\
\hangindent=23pt\noindent\textbf{Матвеева~С.\,С., Захарова~Т.\,В.} Сети массового обслуживания с наименьшей 
длиной\linebreak
\vspace*{-12pt}\\
\hspace*{23pt}очереди$\dotfill$&3&22\\
\hangindent=23pt\noindent\textbf{Матюшенко~С.\,И.} Стационарные характеристики двухканальной системы 
обслужива-\linebreak
\vspace*{-12pt}\\
\hspace*{23pt}ния с переупорядочиванием заявок и распределениями фазового типа$\dotfill$&4&68\\
\textbf{Минель~Ж.-Л.} см.~Бунтман~Н.\,В.&&\\
\textbf{Морозов~Е.\,В.} см.~Бородина~А.\,В.&&\\
\textbf{Морозов~Е.\,В.} см.~Лукьяненко~А.\,С.&&\\
\textbf{Ососков~М.\,В.} см.~Каратеев~С.\,Л.&&\\
\hangindent=23pt\noindent\textbf{Павельева~Е.\,А., Крылов~А.\,С.} Поиск и анализ ключевых точек радужной 
оболочки\linebreak
\vspace*{-12pt}\\
\hspace*{23pt}глаза методом преобразования Эрмита$\dotfill$&1&79\\
\textbf{Печинкин~А.\,В.} см.~Френкель~С.\,Л.,&&\\
\hangindent=23pt\noindent\textbf{Протасов~В.\,И.} Составление субъективного портрета с использованием 
эволюционно-\linebreak
\vspace*{-12pt}\\
\hspace*{23pt}го морфинга и квалиметрия метода$\dotfill$&1&83\\
\hangindent=23pt\noindent\textbf{Рудаков~К.\,В., Торшин~И.\,Ю.} Вопросы разрешимости задачи распознавания 
вторичной\linebreak
\vspace*{-12pt}\\
\hspace*{23pt}структуры белка$\dotfill$&2&25\\
\textbf{Сатин~Я.\,А.} см.~Зейфман~А.\,И.&&\\
\hangindent=23pt\noindent\textbf{Сейфуль-Мулюков~Р.\,Б.} Нефть как носитель информации о своем 
происхождении,\linebreak
\vspace*{-12pt}\\
\hspace*{23pt}структуре и эволюции$\dotfill$&1&41\\
\end{tabular}
}

{\tabcolsep=3pt
\begin{tabular}{p{388pt}rr}
&\textbf{Выпуск} & \textbf{Стр.}\\[6pt]
\textbf{Семендяев~Н.\,Н.} см.~Синицын~И.\,Н.&&\\
\textbf{Серебряков~В.\,А.} см.~Захаров~А.\,А.&&\\
\textbf{Синицын~В.\,И.} см.~Синицын~И.\,Н.&&\\
\hangindent=23pt\noindent\textbf{Синицын~И.\,Н., Синицын~В.\,И., Корепанов~Э.\, Р., Белоусов~В.\,В., 
Семендяев~Н.\,Н.} Оперативное построение информационных моделей движения полюса 
Земли\linebreak
\vspace*{-12pt}\\
\hspace*{23pt}методами линейных и линеаризованных фильтров$\dotfill$&1&2\\
\textbf{Сипина~А.\,В.} см.~Бенинг~В.\,Е.&&\\
\hangindent=23pt\noindent\textbf{Соколов~И.\,А.} О работах заслуженного деятеля науки Российской Федерации 
И.\,Н.~Синицына в области информационных технологий и автоматизации (к 70-летию\linebreak
\vspace*{-12pt}\\
\hspace*{23pt}со дня рождения)$\dotfill$&3&84\\
\textbf{Соколов~И.\,А.} см.~Илюшин~Г.\,Я.&&\\
\hangindent=23pt\noindent\textbf{Соколов~И.\,А., Королев~В.\,Ю.} Предисловие$\dotfill$&2&2\\
\textbf{Солдатов~С.\,А.} см.~Колесников~А.\,В.&&\\
\hangindent=23pt\noindent\textbf{Степанов~С.\,Ю.} Использование координатного метода фрагментации 
коммутаторной\linebreak
\vspace*{-12pt}\\
\hspace*{23pt}нейронной сети для сокращения трафика$\dotfill$&2&57\\
\textbf{Тимонина~Е.\,Е.} см.~Грушо~А.\,А.&&\\
\textbf{Торшин~И.\,Ю.} см.~Рудаков~К.\,В.&&\\
\textbf{Ульянов~В.\,В.} см.~Кавагучи~Ю.&&\\
\textbf{Фазекаш~И.} см.~Чупрунов~А.\,Н.&&\\
\textbf{Френкель~С.\,Л.} см.~Баранов~С.\,И.&&\\
\hangindent=23pt\noindent\textbf{Френкель~С.\,Л., Печинкин~А.\,В.} Оценка времени самовосстановления в 
цифровых\linebreak
\vspace*{-12pt}\\
\hspace*{23pt}системах после сбоев, вызываемых переходными помехами$\dotfill$&3&2\\
\textbf{Фуджикоши~Я.} см.~Кавагучи~Ю.&&\\
\hangindent=23pt\noindent\textbf{Цискаридзе~А.\,К.} Математическая модель и метод восстановления позы человека 
по\linebreak
\vspace*{-12pt}\\
\hspace*{23pt}стереопаре силуэтных изображений$\dotfill$&4&27\\
\hangindent=23pt\noindent\textbf{Чупраков~К.\,Г.} К вопросу о размещении коллективных средств отображения в 
ситуа-\linebreak
\vspace*{-12pt}\\
\hspace*{23pt}ционном зале с заданными параметрами$\dotfill$&4&89\\
\textbf{Чупраков~К.\,Г.} см.~Зацаринный~А.\,А.&&\\
\hangindent=23pt\noindent\textbf{Чупрунов~А.\,Н., Фазекаш~И.} Законы повторного логарифма для числа 
безошибочных\linebreak
\vspace*{-12pt}\\
\hspace*{23pt}блоков при помехоустойчивом кодировании$\dotfill$&3&42\\
\textbf{Шевцова~И.\,Г.} см.~Григорьева~М.\,Е.&&\\
\hangindent=23pt\noindent\textbf{Шестаков~О.\,В.} Аппроксимация распределения оценки риска пороговой 
обработки вейвлет-коэффициентов нормальным распределением при использовании 
выбо-\linebreak
\vspace*{-12pt}\\
\hspace*{23pt}рочной дисперсии$\dotfill$&4&73\\
\textbf{Шестаков~О.\,В.} см.~Маркин~А.\,В.&&\\
\textbf{Шоргин~С.\,Я.} см.~Зейфман~А.\,И.&&\\
\textbf{Шоргин~С.\,Я.} см.~Кудрявцев~А.\,А.&&\\
\end{tabular}
}

%\thispagestyle{myheadings}
\def\leftfootline{\small{\textbf{\thepage}
\hfill ИНФОРМАТИКА И ЕЁ ПРИМЕНЕНИЯ\ \ \ том~4\ \ \ выпуск~4\ \ \ 2010}
}%
 \def\rightfootline{\small{ИНФОРМАТИКА И ЕЁ ПРИМЕНЕНИЯ\ \ \ том~4\ \ \ выпуск~4\ \ \ 2010
 \hfill \textbf{\thepage}}}
 \label{end\stat}

%
%Том 10 Выпуск 1-4 Год 2016

\def\stat{cont-e}
{%\hrule\par
%\vskip 7pt % 7pt
\raggedleft\Large \bf%\baselineskip=3.2ex
2\,0\,1\,6\ \ A\,U\,T\,H\,O\,R\ \ I\,N\,D\,E\,X \vskip 17pt
 \hrule
 \par
\vskip 21pt plus 6pt minus 3pt }

\label{st\stat}

\def\tit{\ }

\def\aut{\ }
\def\auf{\ }

\def\leftkol{\ } %2016 AUTHOR INDEX} % ENGLISH ABSTRACTS}

\def\rightkol{\ } %2016 AUTHOR INDEX} %ENGLISH ABSTRACTS}

\titele{\tit}{\aut}{\auf}{\leftkol}{\rightkol}

\def\leftfootline{\small{\textbf{\thepage}
\hfill INFORMATIKA I EE PRIMENENIYA~--- INFORMATICS AND APPLICATIONS\ \ \ 2016\
\ \ volume~10\ \ \ issue\ 4}
}%
 \def\rightfootline{\small{INFORMATIKA I EE PRIMENENIYA~--- INFORMATICS AND APPLICATIONS\ \ \ 2016\ \ \ volume~10\ \ \ issue\ 4
\hfill \textbf{\thepage}}}

\vspace*{-12pt}
\vspace*{-18pt}

{\tabcolsep=2.8pt
\begin{tabular}{p{382pt}cc}
&\textbf{Issue} & \textbf{Page}\\[6pt]
\Avtors{Agalarov~M.\,Ya.} see~Agalarov~Ya.\,M.&&\\
\Avtors{Agalarov~Ya.\,M., Agalarov~M.\,Ya., and
Shorgin~V.\,S.} About the optimal threshold of queue\linebreak
\\[-12pt]
\hspace*{23pt}length in a~particular problem of profit maximization
in the $M/G/1$ queuing system&2&70--79\\
\Avtors{Alexeyevsky~D.\,A.} BioNLP ontology extraction from 
a~restricted language corpus with\linebreak
\\[-12pt]
\hspace*{23pt}context-free grammars&1&119--128\\
\Avtors{Andreev~S.\,D.} see~Gaidamaka~Yu.\,V.&&\\
\Avtors{Andreev~S.\,D.} see~Ometov~A.\,Ya.&&\\
\Avtors{Arkhipov~O.\,P., Arkhipov~P.\,O., and Sidorkin~I.\,I.} The
option to create a~local coordinate\linebreak
\\[-12pt]
\hspace*{23pt}system for synchronization of selected images&3&91--97\\
\Avtors{Arkhipov~P.\,O.} see~Arkhipov~O.\,P.&&\\
\Avtors{Belousov~V.\,V.} see~Shnurkov~P.\,V.&&\\
\Avtors{Belousov~V.\,V.} see~Shnurkov~P.\,V.&&\\
\Avtors{Bening~V.\,E.} Calculation of~the~asymptotic deficiency
of~some statistical procedures based\linebreak
\\[-12pt]
\hspace*{23pt}on~samples with~random sizes&4&34--45\\
\Avtors{Borisov~A.\,V., Bosov~A.\,V., and Miller~G.\,B.} Modeling and
monitoring of VoIP connection&2&\hphantom{1}2--13\\
\Avtors{Bosov~A.\,V.} see~Borisov~A.\,V.&&\\
\Avtors{Briukhov~D.\,O.} see~Stupnikov~S.\,A.&&\\
\Avtors{Callaos~N.\,K.\ and Seyful-Mulyukov~R.\,B.} Complexity and
its information content&1&129--139\\
\Avtors{Chertok~A.\,V., Kadaner~A.\,I., Khazeeva~G.\,T., and
Sokolov~I.\,A.} Regime switching detection\linebreak
\\[-12pt]
\hspace*{23pt}for~the~Levy driven
Ornstein--Uhlenbeck process using CUSUM methods&4&46--56\\
\Avtors{Chichagov~V.\,V.} Asymptotic expansions of mean absolute
error of uniformly minimum variance unbiased and maximum likelihood
estimators on the one-parameter exponential\linebreak
\\[-12pt]
\hspace*{23pt}family model of lattice distributions&3&66--76\\
\Avtors{Danishevsky~V.\,I.} see~Kolesnikov A.\,V.&&\\
\Avtors{Fazliev~A.\,Z.} see~Kalinichenko~L.\,A.&&\\
\Avtors{Fedoseev~A.\,A.} What is behind the concept of ``knowledge in
small packages''&3&105--110\\
\Avtors{Gaidamaka~Yu.\,V., Andreev~S.\,D., Sopin~E.\,S.,
Samouylov~K.\,E., and Shorgin~S.\,Ya.} Interference analysis
of~the~device-to-device communications model with~regard to~a~signal\linebreak
\\[-12pt]
\hspace*{23pt}propagation environment&4&\hphantom{1}2--10\\
\Avtors{Gasilov~A.\,V.} see~Yakovlev~O.\,A.&&\\
\Avtors{Goncharov~A.\,V.\ and Strijov~V.\,V.} Metric time series
classification using weighted dynamic\linebreak
\\[-12pt]
\hspace*{23pt}warping relative to centroids of classes&2&36--47\\
\Avtors{Gordov~E.\,P.} see~Kalinichenko~L.\,A.&&\\
\Avtors{Gorshenin~A.\,K.} Concept of online service for stochastic
modeling of real processes&1&72--81\\
\Avtors{Gorshenin~A.\,K.} see~Shnurkov~P.\,V.&&\\
\Avtors{Gorshenin~A.\,K.} see~Shnurkov~P.\,V.&&\\
\Avtors{Grusho~A.\,A., Grusho~N.\,A., Zabezhailo~M.\,I., and
Timonina~E.\,E.} Integration of statistical and\linebreak
\\[-12pt]
\hspace*{23pt}deterministic methods for
analysis of information security&3&2--8\\
\Avtors{Grusho~A.\,A., Zabezhailo~M.\,I., and Zatsarinny~A.\,A.} On
the advanced procedure to reduce\linebreak
\\[-12pt]
\hspace*{23pt}calculation of Galois closures&4&\hphantom{1}96--104\\
\Avtors{Grusho~N.\,A.} see~Grusho~A.\,A.&&\\
\Avtors{Havanskov~V.\,A.} see~Minin~V.\,A.&&\\
\Avtors{Inkova~O.\,Yu.} see~Zatsman~I.\,M.&&\\
\Avtors{Isachenko~R.\,V.\ and Strijov~V.\,V.} Metric learning in
multiclass time series classification\linebreak
\\[-12pt]
\hspace*{23pt}problem&2&48--57\\
\end{tabular}
}
\pagebreak

\def\leftfootline{\small{\textbf{\thepage}
\hfill INFORMATIKA I EE PRIMENENIYA~--- INFORMATICS AND APPLICATIONS\ \ \ 2016\
\ \ volume~10\ \ \ issue\ 4}
}%
 \def\rightfootline{\small{INFORMATIKA I EE PRIMENENIYA~---
INFORMATICS AND APPLICATIONS\ \ \ 2016\ \ \ volume~10\ \ \ issue\ 4
\hfill \textbf{\thepage}}}

\def\leftkol{2016 AUTHOR INDEX} % ENGLISH ABSTRACTS}

\def\rightkol{2016 AUTHOR INDEX} %ENGLISH ABSTRACTS}


{\tabcolsep=2.83pt
\begin{tabular}{p{382pt}cc}
&\textbf{Issue} & \textbf{Page}\\[6pt]
\Avtors{Kadaner~A.\,I.} see~Chertok~A.\,V.&&\\[.255pt]
\Avtors{Kalinichenko~L.\,A., Volnova~A.\,A., Gordov~E.\,P.,
Kiselyova~N.\,N., Kovaleva~D.\,A., Malkov~O.\,Yu., Okladnikov~I.\,G.,
Podkolodnyy~N.\,L., Pozanenko~A.\,S., Ponomareva~N.\,V.,
Stupnikov~S.\,A.,} \textbf{and Fazliev~A.\,Z.} Data access challenges for data
intensive\linebreak
\\[-12pt]
\hspace*{23pt}research in Russia&1& 2--22\\[.255pt]
\Avtors{Karasikov~M.\,E.\ and Strijov~V.\,V.} Feature-based
time-series classification&4&121--131\\[.255pt]
\Avtors{Khazeeva~G.\,T.} see~Chertok~A.\,V.&&\\[.255pt]
\Avtors{Khokhlov~Yu.\,S.} Multivariate fractional Levy motion and its
applications&2&\hphantom{1}98--106\\[.255pt]
\Avtors{Kirikov~I.\,A., Kolesnikov~A.\,V., Listopad~S.\,V., and
Rumovskaya~S.\,B.} Fine-grained hybrid\linebreak
\\[-12pt]
\hspace*{23pt}intelligent systems. Part 2:
Bidirectional hybridization&1&\hphantom{1}96--105\\[.255pt]
\Avtors{Kirikov~I.\,A., Kolesnikov~A.\,V., Listopad~S.\,V., and
Rumovskaya~S.\,B.} ``Virtual council''~---\linebreak
\\[-12pt]
\hspace*{23pt}source environment
supporting complex diagnostic decision making&3&81--90\\[.255pt]
\Avtors{Kiselyova~N.\,N.} see~Kalinichenko~L.\,A.&&\\[.255pt]
\Avtors{Kolesnikov A.\,V., Listopad~S.\,V., Rumovskaya~S.\,B., and
Danishevsky~V.\,I.} Informal axiomatic\linebreak
\\[-12pt]
\hspace*{23pt}theory of~the~role visual models&4&114--120\\[.255pt]
\Avtors{Kolesnikov~A.\,V.} see~Kirikov~I.\,A.&&\\[.255pt]
\Avtors{Kolesnikov~A.\,V.} see~Kirikov~I.\,A.&&\\[.255pt]
\Avtors{Kolin~K.\,K.} Humanitarian aspects of information
security&3&111--121\\[.255pt]
\Avtors{Konovalov~M.\,G.\ and Razumchik~R.\,V.} Dispatching
to~two parallel nonobservable queues using\linebreak
\\[-12pt]
\hspace*{23pt}only static
information&4&57--67\\[.255pt]
\Avtors{Korchagin~A.\,Yu.} see~Korolev~V.\,Yu.&&\\[.255pt]
\Avtors{Korchagin~A.\,Yu.} see~Korolev~V.\,Yu.&&\\[.255pt]
\Avtors{Korepanov~E.\,R.} see~Sinitsyn~I.\,N.&&\\[.255pt]
\Avtors{Korepanov~E.\,R.} see~Sinitsyn~I.\,N.&&\\[.255pt]
\Avtors{Korolev~V.\,Yu., Korchagin~A.\,Yu., and Zeifman~A.\,I.} The
Poisson theorem for Bernoulli trials\linebreak
\\[-12pt]
\hspace*{23pt}with~a~random probability
of~success and~a~discrete analog of~the~Weibull distribution&4&11--20\\[.255pt]
\Avtors{Korolev~V.\,Yu., Zeifman~A.\,I., and Korchagin~A.\,Yu.}
Asymmetric Linnik distributions as~limit\linebreak
\\[-12pt]
\hspace*{23pt}laws for~random sums
of~independent random variables with~finite variances&4&21--33\\[.255pt]
\Avtors{Koucheryavy~E.\,A.} see~Ometov~A.\,Ya.&&\\[.255pt]
\Avtors{Kovaleva~D.\,A.} see~Kalinichenko~L.\,A.&&\\[.255pt]
\Avtors{Kovalyov~S.\,P.} Metaprogramming to increase
manufacturability of large-scale software-\linebreak
\\[-12pt]
\hspace*{23pt}intensive systems&1&56--66\\[.255pt]
\Avtors{Krivenko~M.\,P.} Significance tests of feature selection for
classification&3&32--40\\[.255pt]
\Avtors{Kruzhkov~M.\,G.} see~Zalizniak~Anna~A.&&\\[.255pt]
\Avtors{Kruzhkov~M.\,G.} see~Zatsman~I.\,M.&&\\[.255pt]
\Avtors{Kudryavtsev~A.\,A.} Bayesian queueing and reliability models:
\textit{A~priori} distributions with\linebreak
\\[-12pt]
\hspace*{23pt}compact support&1&67--71\\[.255pt]
\Avtors{Kudryavtsev~A.\,A.} Characteristics dependent on the balance
coefficient in Bayesian models\linebreak
\\[-12pt]
\hspace*{23pt}with compact support of \textit{a priori}
distributions&3&77--80\\[.255pt]
\Avtors{Kudryavtsev~A.\,A.\ and Palionnaia~S.\,I.} Bayesian recurrent
model of reliability growth:\linebreak
\\[-12pt]
\hspace*{23pt}Parabolic distribution of parameters&2&80--83\\[.255pt]
\Avtors{Kudryavtsev~A.\,A.\ and Titova~A.\,I.} Bayesian queuing
and~reliability models: Degenerate-\linebreak
\\[-12pt]
\hspace*{23pt}Weibull case&4&68--71\\[.255pt]
\Avtors{Leontyev~N.\,D.\ and Ushakov~V.\,G.} Analysis of a queueing
system with autoregressive arrivals\linebreak
\\[-12pt]
\hspace*{23pt}and nonpreemptive priority&3&15--22\\[.255pt]
\Avtors{Listopad~S.\,V.} see~Kirikov~I.\,A.&&\\[.255pt]
\Avtors{Listopad~S.\,V.} see~Kirikov~I.\,A.&&\\[.255pt]
\Avtors{Listopad~S.\,V.} see~Kolesnikov A.\,V.&&\\[.255pt]
\Avtors{Malkov~O.\,Yu.} see~Kalinichenko~L.\,A.&&\\[.255pt]
\Avtors{Markov~A.\,S., Monakhov~M.\,M., and
Ulyanov~V.\,V.} Generalized Cornish--Fisher expansions\linebreak
\\[-12pt]
\hspace*{23pt}for distributions of statistics based on samples
of random size&2&84--91\\[.255pt]
\Avtors{Melnikov~A.\,K.\ and Ronzhin~A.\,F.} Generalized statistical
method of~text analysis based\linebreak
\\[-12pt]
\hspace*{23pt}on~calculation of~probability distributions
of~statistical values&4&89--95\\
\end{tabular}
}
\pagebreak

\def\leftfootline{\small{\textbf{\thepage}
\hfill INFORMATIKA I EE PRIMENENIYA~--- INFORMATICS AND APPLICATIONS\ \ \ 2016\
\ \ volume~10\ \ \ issue\ 4}
}%
 \def\rightfootline{\small{INFORMATIKA I EE PRIMENENIYA~---
INFORMATICS AND APPLICATIONS\ \ \ 2016\ \ \ volume~10\ \ \ issue\ 4
\hfill \textbf{\thepage}}}

\def\leftkol{2016 AUTHOR INDEX} % ENGLISH ABSTRACTS}

\def\rightkol{2016 AUTHOR INDEX} %ENGLISH ABSTRACTS}


{\tabcolsep=3pt
\begin{tabular}{p{381pt}cc}
&\textbf{Issue} & \textbf{Page}\\[6pt]
\Avtors{Meykhanadzhyan~L.\,A.} Stationary characteristics of the finite
capacity queueing system with\linebreak
\\[-12pt]
\hspace*{23pt}inverse service order and generalized
probabilistic priority&2&123--131\\[.23pt]
\Avtors{Miller~G.\,B.} see~Borisov~A.\,V.&&\\[.23pt]
\Avtors{Minin~V.\,A., Zatsman~I.\,M., Havanskov~V.\,A., and
Shubnikov~S.\,K.} Intensity of citation of scientific publications in
inventions on information and computer technologies patented\linebreak
\\[-12pt]
\hspace*{23pt}in Russia by domestic and foreign applicants&2&107--122\\[.23pt]
\Avtors{Monakhov~M.\,M.} see~Markov~A.\,S.&&\\[.23pt]
\Avtors{Naumov~V.\,A.\ and Samouylov~K.\,E.} On relationship
between queuing systems with resources\linebreak
\\[-12pt]
\hspace*{23pt}and Erlang networks&3&\hphantom{1}9--14\\[.23pt]
\Avtors{Okladnikov~I.\,G.} see~Kalinichenko~L.\,A.&&\\[.23pt]
\Avtors{Ometov~A.\,Ya., Andreev~S.\,D., Turlikov~A.\,M., and
Koucheryavy~E.\,A.} Performance analysis of\linebreak
\\[-12pt]
\hspace*{23pt}a wireless data
aggregation system with contention for contemporary sensor
networks&3&23--31\\[.23pt]
\Avtors{Palionnaia~S.\,I.} see~Kudryavtsev~A.\,A.&&\\[.23pt]
\Avtors{Podkolodnyy~N.\,L.} see~Kalinichenko~L.\,A.&&\\[.23pt]
\Avtors{Ponomareva~N.\,V.} see~Kalinichenko~L.\,A.&&\\[.23pt]
\Avtors{Popkova~N.\,A.} see~Zatsman~I.\,M.&&\\[.23pt]
\Avtors{Pozanenko~A.\,S.} see~Kalinichenko~L.\,A.&&\\[.23pt]
\Avtors{Razumchik~R.\,V.} see~Konovalov~M.\,G.&&\\[.23pt]
\Avtors{Ronzhin~A.\,F.} see~Melnikov~A.\,K.&&\\[.23pt]
\Avtors{Rumovskaya~S.\,B.} see~Kirikov~I.\,A.&&\\[.23pt]
\Avtors{Rumovskaya~S.\,B.} see~Kirikov~I.\,A.&&\\[.23pt]
\Avtors{Rumovskaya~S.\,B.} see~Kolesnikov A.\,V.&&\\[.23pt]
\Avtors{Samouylov~K.\,E.} see~Gaidamaka~Yu.\,V.&&\\[.23pt]
\Avtors{Samouylov~K.\,E.} see~Naumov~V.\,A.&&\\[.23pt]
\Avtors{Serebryanskii~S.\,M.} see~Tyrsin~A.\,N.&&\\[.23pt]
\Avtors{Seyful-Mulyukov~R.\,B.} see~Callaos~N.\,K.&&\\[.23pt]
\Avtors{Shestakov~O.\,V.} Statistical properties of the denoising method
based on the stabilized hard\linebreak
\\[-12pt]
\hspace*{23pt}thresholding&2&65--69\\[.23pt]
\Avtors{Shestakov~O.\,V.} The strong law of large numbers for the risk
estimate in the problem of\linebreak
\\[-12pt]
\hspace*{23pt}tomographic image reconstruction from
projections with a correlated noise&3&41--45\\[.23pt]
\Avtors{Shestakov~O.\,V.} see~Zakharova~T.\,V.&&\\[.23pt]
\Avtors{Shnurkov~P.\,V., Gorshenin~A.\,K., and Belousov~V.\,V.}
Analytical solution of~the~optimal control\linebreak
\\[-12pt]
\hspace*{23pt}task of~a~semi-Markov
process with~finite set of~states&4&72--88\\[.23pt]
\Avtors{Shnurkov~P.\,V., Zasypko~V.\,V., Belousov~V.\,V., and
Gorshenin~A.\,K.} Development of the algorithm of numerical solution
of the optimal investment control problem\linebreak
\\[-12pt]
\hspace*{23pt}in the closed dynamical model of three-sector economy&1&82--95\\[.23pt]
\Avtors{Shorgin~S.\,Ya.} see~Gaidamaka~Yu.\,V.&&\\[.23pt]
\Avtors{Shorgin~V.\,S.} see~Agalarov~Ya.\,M.&&\\[.23pt]
\Avtors{Shubnikov~S.\,K.} see~Minin~V.\,A.&&\\[.23pt]
\Avtors{Sidorkin~I.\,I.} see~Arkhipov~O.\,P.&&\\[.23pt]
\Avtors{Sinitsyn~I.\,N.} Analytical modeling of processes in stochastic
systems with complex fractional\linebreak
\\[-12pt]
\hspace*{23pt}order Bessel nonlinearities&3&55--65\\[.23pt]
\Avtors{Sinitsyn~I.\,N.} Orthogonal supoptimal filters for nonlinear
stochastic systems on manifolds&1&34--44\\[.23pt]
\Avtors{Sinitsyn~I.\,N.\ and Korepanov~E.\,R.} Normal Pugachev
conditionally-optimal filters and extra-\linebreak
\\[-12pt]
\hspace*{23pt}polators for state linear stochastic systems&2&14--23\\[.23pt]
\Avtors{Sinitsyn~I.\,N.\ and Sinitsyn~V.\,I.} Analytical modeling of
distributions in stochastic systems on\linebreak
\\[-12pt]
\hspace*{23pt}manifolds based on ellipsoidal approximation&1&45--55\\[.23pt]
\Avtors{Sinitsyn~I.\,N., Sinitsyn~V.\,I., and
Korepanov~E.\,R.} Ellipsoidal suboptimal filters for nonlinear\linebreak
\\[-12pt]
\hspace*{23pt}stochastic systems on manifolds&2&24--35\\[.23pt]
\Avtors{Sinitsyn~V.\,I.} see~Sinitsyn~I.\,N.&&\\[.23pt]
\Avtors{Sinitsyn~V.\,I.} see~Sinitsyn~I.\,N.&&\\[.23pt]
\Avtors{Skvortsov~N.\,A.} see~Stupnikov~S.\,A.&&\\[.23pt]
\Avtors{Sokolov~I.\,A.} see~Chertok~A.\,V.&&\\
\end{tabular}
}
\pagebreak

\def\leftfootline{\small{\textbf{\thepage}
\hfill INFORMATIKA I EE PRIMENENIYA~--- INFORMATICS AND APPLICATIONS\ \ \ 2016\
\ \ volume~10\ \ \ issue\ 4}
}%
 \def\rightfootline{\small{INFORMATIKA I EE PRIMENENIYA~---
INFORMATICS AND APPLICATIONS\ \ \ 2016\ \ \ volume~10\ \ \ issue\ 4
\hfill \textbf{\thepage}}}

\def\leftkol{2016 AUTHOR INDEX} % ENGLISH ABSTRACTS}

\def\rightkol{2016 AUTHOR INDEX} %ENGLISH ABSTRACTS}


{\tabcolsep=3pt
\begin{tabular}{p{382pt}cc}
&\textbf{Issue} & \textbf{Page}\\[6pt]
\Avtors{Sopin~E.\,S.} see~Gaidamaka~Yu.\,V.&&\\
\Avtors{Strijov~V.\,V.} see~Goncharov~A.\,V.&&\\
\Avtors{Strijov~V.\,V.} see~Isachenko~R.\,V.&&\\
\Avtors{Strijov~V.\,V.} see~Karasikov~M.\,E.&&\\
\Avtors{Stupnikov~S.\,A., Briukhov~D.\,O., and Skvortsov~N.\,A.}
Co-lending systemic risk analysis over\linebreak
\\[-12pt]
\hspace*{23pt}heterogeneous data collections&1&23--33\\
\Avtors{Stupnikov~S.\,A.} see~Kalinichenko~L.\,A.&&\\
\Avtors{Suchkov~A.\,P.} see~Zatsarinny~A.\,A.&&\\
\Avtors{Timonina~E.\,E.} see~Grusho~A.\,A.&&\\
\Avtors{Titova~A.\,I.} see~Kudryavtsev~A.\,A.&&\\
\Avtors{Turlikov~A.\,M.} see~Ometov~A.\,Ya.&&\\
\Avtors{Tyrsin~A.\,N.\ and Serebryanskii~S.\,M.} Recognition of
dependences on the basis of inverse\linebreak
\\[-12pt]
\hspace*{23pt}mapping&2&58--64\\
\Avtors{Ulyanov~V.\,V.} see~Markov~A.\,S.&&\\
\Avtors{Ushakov~V.\,G.} Queueing system with working vacations and
hyperexponential input stream&2&92--97\\
\Avtors{Ushakov~V.\,G.} see~Leontyev~N.\,D.&&\\
\Avtors{Volnova~A.\,A.} see~Kalinichenko~L.\,A.&&\\
\Avtors{Yakovlev~O.\,A.\ and Gasilov~A.\,V.} Speeded-up stereo
matching using geodesic support weights&3&\hphantom{1}98--104\\
\Avtors{Zabezhailo~M.\,I.} see~Grusho~A.\,A.&&\\
\Avtors{Zabezhailo~M.\,I.} see~Grusho~A.\,A.&&\\
\Avtors{Zakharova~T.\,V.\ and Shestakov~O.\,V.} Precision analysis of
wavelet processing of aerodynamic\linebreak
\\[-12pt]
\hspace*{23pt}flow patterns&3&46--54\\
\Avtors{Zalizniak~Anna~A.\ and Kruzhkov~M.\,G.} Database
of~Russian impersonal verbal constructions&4&132--141\\
\Avtors{Zasypko~V.\,V.} see~Shnurkov~P.\,V.&&\\
\Avtors{Zatsarinny~A.\,A.\ and Suchkov~A.\,P.} Systems engineering
approaches to~the~establishment of\linebreak
\\[-12pt]
\hspace*{23pt}a~system for~decision support based
on~situational analysis&4&105--113\\
\Avtors{Zatsarinny~A.\,A.} see~Grusho~A.\,A.&&\\
\Avtors{Zatsman~I.\,M., Inkova~O.\,Yu., Kruzhkov~M.\,G., and
Popkova~N.\,A.} Representation of cross-\linebreak
\\[-12pt]
\hspace*{23pt}lingual knowledge about
connectors in supracorpora databases&1&106--118\\
\Avtors{Zatsman~I.\,M.} see~Minin~V.\,A.&&\\
\Avtors{Zeifman~A.\,I.} see~Korolev~V.\,Yu.&&\\
\Avtors{Zeifman~A.\,I.} see~Korolev~V.\,Yu.&&\\
\end{tabular}
}

%\thispagestyle{myheadings}
\def\leftfootline{\small{\textbf{\thepage}
\hfill INFORMATIKA I EE PRIMENENIYA~--- INFORMATICS AND APPLICATIONS\ \ \ 2016\
\ \ volume~10\ \ \ issue\ 4}
}%
 \def\rightfootline{\small{INFORMATIKA I EE PRIMENENIYA~---
INFORMATICS AND APPLICATIONS\ \ \ 2016\ \ \ volume~10\ \ \ issue\ 4
\hfill \textbf{\thepage}}}

 \label{end\stat}

\newpage

%\def\stat{rekl}
%\label{preobr}

%\def\tit{АКАДЕМИК ПУГАЧЁВ  ВЛАДИМИР СЕМЁНОВИЧ\\
%25.03.1911--25.03.1998}


%   \vspace*{-48pt}
%   \begin{center}\LARGE
%Академик Пугачёв  Владимир Семёнович\\ (25.03.1911--25.03.1998)
%   \end{center}
   
   %\vspace*{2.5mm}
   
   \begin{center}

{\prgsh\LARGE
ОБЪЯВЛЕНИЯ О КОНФЕРЕНЦИЯХ}

\end{center}
%\hrule

\vspace*{6pt}

   
   \vspace*{10mm}
   
   \thispagestyle{empty}

\noindent
\begin{tabular}{cc}
%\begin{center}
\multicolumn{1}{c}{\raisebox{-40pt}[0pt][0pt]{\mbox{%
\epsfxsize=33mm
\epsfbox{vspu.eps}
}}}
%\end{center}
&
\tabcolsep=0pt\begin{tabular}{c}
{\prg{\Large\textbf{XII Всероссийское совещание}}}\\[6pt]
{\prg{\Large\textbf{по проблемам управления}}}\\[12pt]
{\prg{\large 16--19 июня 2014~г.}}\\[6pt] 
{\prg{\large Институт проблем управления имени В.\,А.~Трапезникова РАН}}\\[6pt]
{\prg{\large Москва, Россия}}
\end{tabular}
\end{tabular}

\vspace*{60pt}

     
 { %\large    
 XII Всероссийское совещание по проблемам управления (ВСПУ XII), посвященное 75-летию 
Института проблем управления (ИПУ) имени В.\,А.~Трапезникова РАН, проводится 16--19~июня 
2014~г.\ 
в ИПУ РАН (г.~Москва, Россия). ВСПУ XII организуется ИПУ РАН при поддержке РФФИ, Отделения 
энергетики, машиностроения, механики и процессов управления Российской академии наук, 
Российского 
национального комитета по автоматическому управлению, Академии навигации и управ\-ле\-ния 
движением, 
Научного совета РАН по комплексным проблемам управления и автоматизации, Совета по 
мехатронике и робототехнике РАН. Официальный язык Совещания~--- русский.

\vspace*{24pt}
     
     \textbf{Направления работы}
     \begin{enumerate}[1.]
\item Теория систем управления
\item Управление подвижными объектами и навигация
\item Интеллектуальные системы управления
\item Управление в промышленности, транспортом и логистикой
\item Управление системами междисциплинарной природы
\item Средства измерения, вычислений и контроля в управлении
\item Системный анализ и принятие решений в задачах управления
\item Информационные технологии в управлении
\item Проблемы образования в области управления: современное содержание и технологии обучения
\end{enumerate}

\vspace*{24pt}

     Подробная информация о Совещании находится на сайте {\sf http://vspu2014.ipu.ru}. Срок 
окончательной подачи докладов через систему подачи докладов на сайте~--- \textbf{30~ноября} 
2013~г.
}

%\include{rekl-1}

%\end{document}

%   \vspace*{-48pt}

\begin{center}
\vspace*{6pt}
\mbox{%
\epsfxsize=53.502mm
\epsfbox{foto-1.eps}
}
\end{center}

\vspace*{6pt} %Академик


   \begin{center}
\fbox{\Large\textbf{Профессор Игорь Алексеевич Ушаков}}\\[12pt]
\textbf{\large 22.01.1935--27.02.2015}
   \end{center}


   %\vspace*{2.5mm}

   \vspace*{5mm}

   \thispagestyle{empty}

%\

%\vspace*{-12pt}


Редакционный совет и редакционная коллегия журнала <<Информатика и~её применения>> с~глубоким прискорбием извещают, что 27~февраля 2015~г.\ после тяжелой
и~продолжительной болезни скончался Игорь Алексеевич Ушаков~--- доктор технических наук, профессор, член редколлегии журнала <<Информатика и ее применения>>.

Игорь Алексеевич Ушаков окончил Московский авиационный институт, в~1963~г.\ защитил кандидатскую, а~в~1968~г.~--- докторскую диссертацию. С~1958 по 1989~гг.\ работал в~ряде научно-исследовательских организаций СССР, в~том числе руководил отделами в~НИИ АА и~ВЦ АН СССР; с 1969 по 1989 гг. преподавал в~МФТИ (был профессором, а~затем заведующим кафедрой) и~в~МЭИ. С~1989~г.~---- в~США: являлся профессором университета Дж.\ Вашингтона, университета Дж.\ Мэйсона и~Калифорнийского университета, сотрудником компаний MCI, Qualcomm и Hughes.

И.\,А.~Ушаков с момента основания журнала <<Надежность и~контроль качества>> был заместителем ответственного редактора, а~затем на протяжении многих лет членом редколлегии. В~2006~г.\ основал электронный международный журнал ``Reliability: Theory \& Application'', главным редактором которого оставался до конца жизни.

Учебниками и справочниками по теории надежности, написанными И.\,А.~Ушаковым, пользовались и~пользуются несколько поколений ученых и~специалистов в~разных странах мира.

Игорь Алексеевич всегда уделял огромное внимание работе с~молодежью; более~50 его учеников защитили докторские и~кандидатские диссертации.

И.\,А.~Ушаков вел активную научно-про\-све\-ти\-тель\-скую деятельность. В~частности, он был одним из организаторов и~руководителей Московского кабинета качества и~надежности при Политехническом музее (целью этого Кабинета было оказание консультаций работникам промышленных предприятий и~чтение курсов лекций для инженеров, занимающихся проблемой надежности). Находясь в~США, И.\,А.~Ушаков создал международный ин\-тер\-нет-фо\-рум им.\ Б.\,В.~Гнеденко, объединивший около~400~видных специалистов по приложениям теории вероятностей и~математической статистики, преимущественно в~об\-ласти теории надежности и~анализа риска, из десятков стран мира; коллективным членов этого Форума является и~наш журнал. Цели Форума~--- содействие контактам между специалистами из разных стран, организация обмена профессиональными 
новостями и~информацией (новые публикации, предстоящие события и~др.). Также необходимо отметить большое число на\-уч\-но-по\-пу\-ляр\-ных работ, опубликованных И.\,А.~Ушаковым.

И.\,А.~Ушаков обладал большим личным обаянием, имел широкий круг интересов. Все знавшие И.\,А.~Ушакова всегда будут помнить его как замечательного ученого и~прекрасного человека.

\bigskip

Редакционный совет и редакционная коллегия журнала <<Информатика и~её применения>> 
выражают глубокие соболезнования родным и близким покойного, всем, кто его знал и~работал с~ним.



%\end{document}

%\include{IPPM-25}

\def\stat{cont-rus}
{%\hrule\par
%\vskip 7pt % 7pt
\vspace*{-24pt}
\raggedleft\Large \bf%\baselineskip=3.2ex
Правила подготовки рукописей  для публикации в журнале
<<Информатика~и~её~применения>> \vskip 8pt
    \hrule
    \par
\vskip 14pt plus 6pt minus 3pt }

\label{st\stat}

\def\tit{\ }

\def\aut{\ }
\def\auf{\ }

\def\leftkol{\ }
% Правила подготовки рукописей  для публикации в журнале
%<<Информатика и её применения>>

\def\rightkol{\ }
%Правила подготовки рукописей  для публикации в журнале
%<<Информатика и её применения>>}


\titele{\tit}{\aut}{\auf}{\leftkol}{\rightkol}


\vspace*{-60pt}
{ %\small

Журнал <<Информатика и её применения>>
публикует теоретические, обзорные и дискуссионные статьи,
посвященные научным исследованиям и разработкам в области
информатики и ее приложений.

Журнал издается на русском языке. По специальному решению
редколлегии отдельные статьи могут печататься на английском языке.

Тематика журнала охватывает следующие направления:
\begin{itemize}
\item теоретические основы информатики;\\[-15pt]
      \item
математические методы исследования сложных систем и процессов;\\[-15pt]
           \item
информационные системы и сети;\\[-15pt]
                \item
информационные технологии;\\[-15pt]
                     \item
архитектура и программное обеспечение вычислительных комплексов и сетей.\\[-15pt]
\end{itemize}


\noindent
\begin{enumerate}[1.]
\item В журнале печатаются статьи, содержащие результаты, ранее не опубликованные и
не предназначенные к одновременной публикации в других изданиях.

%Публикация не должна нарушать закон об авторских правах.
Публикация предоставленной автором(ами) рукописи не должна нарушать 
положений глав~69, 70 раздела~VII части~IV Гражданского кодекса, 
которые определяют права на результаты интеллектуальной деятельности 
и~средства индивидуализации, в~том числе авторские права, в~РФ.

Ответственность за нарушение авторских прав, в~случае предъявления претензий к~редакции журнала,  
несут авторы статей.



Направляя рукопись в редакцию, авторы сохраняют свои права на данную
рукопись и при этом передают учредителям и редколлегии журнала неисключительные права на
издание статьи на русском языке 
(или на языке статьи, если он отличен от рус\-ско\-го) и~на перевод ее на английский
язык, а~также на
ее распространение в России и за рубежом. 
Каждый автор должен представить в~редакцию подписанный 
с~его стороны <<Лицензионный договор о~передаче неисключительных прав 
на использование произведения>>, текст которого размещен по адресу 
{\sf http://www.ipiran.ru/publications/licence.doc}. 
Этот договор может быть пред\-став\-лен в~бумажном (в~2-х экз.)\ 
или в~электронном виде (отсканированная копия заполненного и~подписанного документа).




Редколлегия вправе запросить у авторов экспертное заключение о возможности
пуб\-ли\-ка\-ции пред\-став\-лен\-ной статьи в открытой печати.\\[-13.5pt]

\item К статье прилагаются данные автора (авторов) (см.\ п.~8). При наличии нескольких
авторов указывается фамилия автора, ответственного за переписку с редакцией.\\[-13.5pt]

\item Редакция журнала осуществляет экспертизу присланных статей в соответствии с
принятой в журнале процедурой рецензирования.

Возвращение рукописи на доработку не означает ее принятия к печати.

Доработанный вариант с ответом на замечания рецензента необходимо прислать в
редакцию.\\[-13.5pt]

\item Решение редколлегии о публикации статьи или ее отклонении сообщается авторам.

Редколлегия может также направить авторам текст рецензии на их статью. Дискуссия по
поводу отклоненных статей не ведется.\\[-13.5pt]

%\pagebreak

\item Редактура статей высылается авторам для просмотра. Замечания к редактуре должны
быть присланы авторами в кратчайшие сроки.\\[-13.5pt]

\item Рукопись предоставляется в электронном виде в форматах MS WORD (.doc или
.docx) или \LaTeX\  (.tex), дополнительно~--- в формате .pdf, на дискете, лазерном диске
или электронной почтой. Предоставление бумажной рукописи необязательно.\\[-13.5pt]

\item При подготовке рукописи в MS Word рекомендуется использовать следующие
настройки.

Параметры страницы:
формат~--- А4; ориентация~--- книжная; поля (см): внутри~--- 2,5, снаружи~--- 1,5,
сверху~--- 2, снизу~--- 2, от края до нижнего колонтитула~--- 1,3.

Основной текст: стиль~--- <<Обычный>>, шрифт~--- Times New Roman, размер~---
14~пунк\-тов, абзацный отступ~--- 0,5~см, 1,5~интервала, выравнивание~--- по ширине.

\pagebreak

\def\leftkol{Правила подготовки рукописей  для публикации в журнале
<<Информатика и её применения>>}

\def\rightkol{Правила подготовки рукописей  для публикации в журнале
<<Информатика и её применения>>}



Рекомендуемый объем рукописи~--- не свыше 10~страниц указанного формата.
При превышении указанного объема редколлегия вправе потребовать от 
автора сокращения объема рукописи.


Сокращения слов, помимо стандартных, не допускаются. Допускается минимальное
количество аббревиатур.


Все страницы рукописи нумеруются.

Шаблоны оформления представлены в интернете:

\noindent
 {\sf
http://www.ipiran.ru/journal/template\_iiep\_ssi\_2024.zip}\\[-14pt]

\item Статья должна содержать следующую информацию на {\bfseries\textit{русском и
английском языках}}:\\[-16pt]

\begin{itemize}
\item название статьи;\\[-15pt]
\item Ф.И.О.\ авторов, на английском можно только имя и фамилию;\\[-15pt]
\item место работы, с указанием почтового адреса организации и электронного адреса каждого
автора;\\[-15pt]
\item сведения об авторах, в соответствии с форматом, образцы которого
представлены на страницах:



\def\leftfootline{\small{\textbf{\thepage}
\hfill ИНФОРМАТИКА И ЕЁ ПРИМЕНЕНИЯ\ \ \ том\ 18\ \ \ выпуск\ 3\ \ \ 2024}
}%
 \def\rightfootline{\small{ИНФОРМАТИКА И ЕЁ ПРИМЕНЕНИЯ\ \ \ том\ 18\ \ \ выпуск\ 3\ \ \ 2024
\hfill \textbf{\thepage}}}



{\sf http://www.ipiran.ru/journal/issues/2013\_07\_01/authors.asp} и

{\sf http://www.ipiran.ru/journal/issues/2013\_07\_01\_eng/authors.asp};
\item аннотация (не менее 100~слов на каждом из языков). Аннотация~--- это краткое
резюме работы, которое может публиковаться отдельно. Она является основным
источником информации в~ин\-фор\-ма\-ци\-он\-ных системах и базах данных. Английская
аннотация должна быть оригинальной, может не быть дословным переводом русского
текста и должна быть написана хорошим английским языком. В~аннотации не должно
быть ссылок на литературу и, по возможности, формул;\\[-15pt]
\item ключевые слова~--- желательно из принятых в мировой
на\-уч\-но-тех\-ни\-че\-ской литературе тематических тезаурусов. Предложения не
могут быть ключевыми словами;\\[-15pt]
\item источники финансирования работы (ссылки на гранты, проекты,
поддерживающие организации и~т.\,п.).
\end{itemize}



%\pagebreak

\item  Требования к спискам литературы.\\[-14pt]

Ссылки на литературу в тексте статьи нумеруются (в квадратных скобках) и
располагаются в каждом из списков литературы в порядке  первых упоминаний. Если источник имеет DOI и/или EDN,
то их необходимо указывать.

Списки литературы представляются в двух вариантах:\\[-14pt]


\noindent
\begin{enumerate}[(1)]
\item \textbf{Список литературы к русскоязычной части}. Русские и английские
работы~---  на языке и в алфавите оригинала;\\[-14.5pt]
\item  \textbf{References}. Русские работы и работы на других языках~--- в латинской
транслитерации с переводом на английский язык; английские работы и работы на других
языках~--- на языке оригинала.
\end{enumerate}

Необходимо для составления списка ``References'' пользоваться размещенной на сайте
{\sf http://www. translit.net/ru/bgn/} бесплатной программой транслитерации русского
 текста в~латиницу. %, при этом в~за\-клад\-ке <<варианты\ldots>> следует выбратьопцию BGN.

Список литературы ``References'' приводится полностью отдельным блоком, повторяя все
позиции из списка литературы к русскоязычной части, независимо от того, имеются или
нет в нем иностранные источники. Если в списке литературы к русскоязычной части есть
ссылки на иностранные публикации, набранные латиницей, они полностью повторяются в
списке ``References''.

Ниже приведены примеры ссылок на различные виды публикаций в списке ``References''.

\def\leftfootline{\small{\textbf{\thepage}
\hfill ИНФОРМАТИКА И ЕЁ ПРИМЕНЕНИЯ\ \ \ том\ 18\ \ \ выпуск\ 3\ \ \ 2024}
}%
 \def\rightfootline{\small{ИНФОРМАТИКА И ЕЁ ПРИМЕНЕНИЯ\ \ \ том\ 18\ \ \ выпуск\ 3\ \ \ 2024
\hfill \textbf{\thepage}}}

{\small

\noindent
\textbf{Описание статьи из журнала:}

\Aue{Zagurenko, A.\,G., V.\,A.~Korotovskikh, A.\,A.~Kolesnikov, A.\,V.~Timonov, and D.\,V.~Kardymon}. 2008.
Tekhniko-ekonomicheskaya optimizatsiya dizayna gidrorazryva plasta [Technical and
economic optimization of the design
of hydraulic fracturing]. \textit{Neftyanoe hozyaystvo} [\textit{Oil Industry}] 11:54--57.

\Aue{Zhang, Z., and D.~Zhu}. 2008. Experimental research on the localized
electrochemical micromachining. \textit{Russ. J.~Electrochem.}  44(8):926--930.
{\sf doi:10.1134/S1023193508080077}.

\noindent
\textbf{Описание статьи из электронного журнала:}

\Aue{Swaminathan, V., E.~Lepkoswka-White, and B.\,P.~Rao}. 1999. Browsers or buyers in cyberspace? An
investigation of electronic factors influencing electronic exchange. \textit{JCMC}
5(2). Available at: {\sf http://www.ascusc.org/jcmc/vol5/issue2/} (accessed April~28, 2011).

\def\leftkol{Правила подготовки рукописей  для публикации в журнале
<<Информатика и её применения>>}

\def\rightkol{Правила подготовки рукописей  для публикации в журнале
<<Информатика и её применения>>}


\noindent
\textbf{Описание статьи из продолжающегося издания (сборника трудов):}

\Aue{Astakhov, M.\,V., and T.\,V.~Tagantsev}. 2006. Eksperimental'noe
issledovanie prochnosti soedineniy ``stal'--kompozit'' [Experimental study of
the strength of joints ``steel--composite'']. \textit{Trudy MGTU
``Matematicheskoe modelirovanie slozhnykh tekh\-ni\-che\-skikh sistem''}
[\textit{Bauman MSTU ``Mathematical Modeling of Complex Technical
Systems'' Proceedings}]. 593:125--130.


\pagebreak



\noindent
\textbf{Описание материалов конференций:}

\Aue{Usmanov, T.\,S., A.\,A.~Gusmanov, I.\,Z.~Mullagalin, R.\,Ju.~Muhametshina, A.\,N.~Chervyakova, and
A.\,V.~Sveshnikov}. 2007. Osobennosti proektirovaniya razrabotki mestorozhdeniy
s primeneniem gidrorazryva
plasta [Features of the design of field development with the use of hydraulic fracturing].
\textit{Trudy 6-go
Mezhdu\-na\-rod\-no\-go Simpoziuma ``Novye resursosberegayushchie tekhnologii nedropol'zovaniya i povysheniya
neftegazootdachi''} [\textit{6th  Symposium (International) ``New Energy Saving Subsoil Technologies and
the Increasing of the Oil and Gas Impact'' Proceedings}]. Moscow. 267--272.



\def\leftfootline{\small{\textbf{\thepage}
\hfill ИНФОРМАТИКА И ЕЁ ПРИМЕНЕНИЯ\ \ \ том\ 18\ \ \ выпуск\ 3\ \ \ 2024}
}%
 \def\rightfootline{\small{ИНФОРМАТИКА И ЕЁ ПРИМЕНЕНИЯ\ \ \ том\ 18\ \ \ выпуск\ 3\ \ \ 2024
\hfill \textbf{\thepage}}}



\noindent
\textbf{Описание книги (монографии, сборники):}



Lindorf, L.\,S., and L.\,G.~Mamikoniants, eds. 1972.
\textit{Ekspluatatsiya turbogeneratorov s neposredstvennym
okhlazhdeniem} [\textit{Operation of turbine generators with direct cooling}].
Moscow: Energy Publs. 352~p.


\Aue{Latyshev, V.\,N.} 2009. \textit{Tribologiya rezaniya. Kn.~1: Friktsionnye protsessy
pri rezanii metallov}
[\textit{Tribology of cutting. Vol.~1: Frictional processes in metal cutting}]. Ivanovo: Ivanovskii
State Univ. 108~p.

\def\leftkol{Правила подготовки рукописей  для публикации в журнале
<<Информатика и её применения>>}

\def\rightkol{Правила подготовки рукописей  для публикации в журнале
<<Информатика и её применения>>}

\noindent
\textbf{Описание переводной книги}
(в списке литературы к русскоязычной части необходимо указать:~/ Пер.\ с англ.~---
после названия книги, а в конце ссылки указать оригинал книги в круглых скобках):
\begin{enumerate}[1.]
\item  В русскоязычной части:

\def\leftfootline{\small{\textbf{\thepage}
\hfill ИНФОРМАТИКА И ЕЁ ПРИМЕНЕНИЯ\ \ \ том\ 18\ \ \ выпуск\ 3\ \ \ 2024}
}%
 \def\rightfootline{\small{ИНФОРМАТИКА И ЕЁ ПРИМЕНЕНИЯ\ \ \ том\ 18\ \ \ выпуск\ 3\ \ \ 2024
\hfill \textbf{\thepage}}}

\Au{Тимошенко С.\,П., Янг Д.\,Х., Уивер~У.}
Колебания в инженерном деле~/ Пер.\ с англ.~--- М.: Машиностроение, 1985. 472~с.
(\Au{Timoshenko~S.\,P., Young~D.\,H., Weaver~W.}
Vibration problems in engineering.~--- 4th ed.~--- New York, NY, USA: Wiley, 1974. 521~p.)\\[-13.5pt]
\item  В англоязычной части:

\Aue{Timoshenko, S.\,P., D.\,H.~Young, and W.~Weaver}.
1974. \textit{Vibration problems in engineering}. 4th ed. New York: 
Wiley. 521~p.
\end{enumerate}

\vspace*{-3pt}


\noindent
\textbf{Описание неопубликованного документа:}


\Aue{Latypov, A.\,R., M.\,M.~Khasanov, and V.\,A.~Baikov}.
2004 (unpubl.). Geologiya i~dobycha (NGT GiD) [Geology and production (NGT GiD)]. Certificate on official registration of the computer program
No.\,2004611198. 

\noindent
\textbf{Описание интернет-ресурса:}


Pravila tsitirovaniya istochnikov [Rules for the citing of sources]. Available at: {\sf
http://www.scribd.com/doc/1034528/} (accessed February~7, 2011).

%\pagebreak

\noindent
\textbf{Описание диссертации или автореферата диссертации:}

\Aue{Semenov, V.\,I.}
2003. Matematicheskoe modelirovanie plazmy v sisteme kompaktnyy tor [Mathematical
modeling of the plasma in the compact torus].  Moscow.  D.Sc.\ Diss. 272~p.

\Aue{Kozhunova, O.\,S.} 2009. Tekhnologiya razrabotki semanticheskogo
slovarya informatsionnogo monitoringa [Technology of development of
semantic dictionary of information monitoring system].  Moscow: IPI RAN. PhD Thesis. 23~p.


\noindent
\textbf{Описание ГОСТа:}

GOST 8.586.5-2005. 2007. Metodika vypolneniya izmereniy. Izmerenie raskhoda i~kolichestva zhidkostey i~gazov
s~pomoshch'yu standartnykh suzhayushchikh ustroystv [Method of measurement.
Measurement of flow rate and volume of liquids and gases by means of orifice devices]. Moscow:
Standardinform  Publs. 10~p.

\noindent
\textbf{Описание патента:}

\Aue{Bolshakov, M.\,V., A.\,V.~Kulakov, A.\,N.~Lavrenov, and M.\,V.~Palkin}.
2006. Sposob orientirovaniya po krenu letatel'nogo
apparata s opti\-che\-skoy golovkoy
samonavedeniya [The way to orient on the roll of aircraft with optical homing head].
Patent RF No.\,2280590.
}

\item Присланные в редакцию материалы авторам не возвращаются.\\[-13.5pt]

\item При отправке файлов по электронной почте просим придерживаться следующих
правил:
\begin{itemize}
\item указывать в поле subject (тема) название журнала и фамилию автора;\\[-13.5pt]
\item указывать в тексте письма название статьи, авторов и~журнал, в~который направляется статья;\\[-13.5pt]
\item использовать attach (присоединение);\\[-13.5pt]
\item в состав электронной версии статьи должны входить: файл, содержащий текст
статьи, и файл(ы), содержащий(е) иллюстрации.\\[-13.5pt]
\end{itemize}

\item Журнал <<Информатика и её применения>> является некоммерческим изданием.
Плата за публикацию не взимается, гонорар авторам не выплачивается.
\end{enumerate}



\def\leftfootline{\small{\textbf{\thepage}
\hfill ИНФОРМАТИКА И ЕЁ ПРИМЕНЕНИЯ\ \ \ том\ 18\ \ \ выпуск\ 3\ \ \ 2024}
}%
 \def\rightfootline{\small{ИНФОРМАТИКА И ЕЁ ПРИМЕНЕНИЯ\ \ \ том\ 18\ \ \ выпуск\ 3\ \ \ 2024
\hfill \textbf{\thepage}}}


\vspace*{-1mm}

\begin{center}

\textbf{Адрес редакции журнала <<Информатика и её применения>>:} \\




Москва 119333, ул.~Вавилова, д.~44, корп.~2, ФИЦ ИУ РАН\\[-10pt]

\

Тел.: +7\,(499)\,135-86-92\ \ Факс:  +7\,(495)\,930-45-05\\[-10pt]

 \

e-mail:   {\sf iiep@frccsc.ru} (Стригина Светлана Николаевна)\\[-10pt]

\

{\sf http://www.ipiran.ru/journal/issues/}
\end{center}
}


\def\leftkol{Правила подготовки рукописей  для публикации в журнале
<<Информатика и её применения>>}

\def\rightkol{Правила подготовки рукописей  для публикации в журнале
<<Информатика и её применения>>}


\def\leftfootline{\small{\textbf{\thepage}
\hfill ИНФОРМАТИКА И ЕЁ ПРИМЕНЕНИЯ\ \ \ том\ 18\ \ \ выпуск\ 3\ \ \ 2024}
}%
 \def\rightfootline{\small{ИНФОРМАТИКА И ЕЁ ПРИМЕНЕНИЯ\ \ \ том\ 18\ \ \ выпуск\ 3\ \ \ 2024
\hfill \textbf{\thepage}}} 
\def\stat{podg-e}
{%\hrule\par
%\vskip 7pt % 7pt
\vspace*{-24pt}
\raggedleft\Large \bf%\baselineskip=3.2ex
Requirements for manuscripts submitted to Journal
``Informatics~and~Applications'' \vskip 8pt
    \hrule
    \par
\vskip 21pt plus 6pt minus 3pt }

\label{st\stat}

\def\tit{\ }

\def\aut{\ }
\def\auf{\ }

\def\leftkol{\ }

\def\rightkol{\ }
%Requirements for manuscripts submitted to Journal
%``Informatics~and~Applications''}

\titele{\tit}{\aut}{\auf}{\leftkol}{\rightkol}

\def\leftfootline{\small{\textbf{\thepage}
\hfill INFORMATIKA I EE PRIMENENIYA~--- INFORMATICS AND APPLICATIONS\ \ \ 2019\
\ \ volume~13\ \ \ issue\ 4}
}%
 \def\rightfootline{\small{INFORMATIKA I EE PRIMENENIYA~--- INFORMATICS AND APPLICATIONS\ \ \ 2019\ \ \ volume~13\ \ \ issue\ 4
\hfill \textbf{\thepage}}}

\vspace*{-60pt}

{\small

\noindent
Journal ``Informatics and Applications'' (Inform.\ Appl.)
publishes theoretical, review, and discussion
articles on the research and development in the
field of informatics and its applications.

The journal is published in Russian.
By a special decision of the editorial
board, some articles can be published in English.


The topics covered include the following areas:
\begin{itemize}
               \item
     theoretical fundamentals of informatics; \\[-14pt]
\item
mathematical methods for studying complex systems and processes; \\[-14pt]
\item
information systems and networks;\\[-14pt]
\item
information technologies; and \\[-14pt]
\item
architecture and software of computational complexes and networks. \\[-14pt]
\end{itemize}

\noindent
\begin{enumerate}[1.]
\item The Journal publishes original articles which have not been published before and are not
intended for simultaneous publication in other editions. An article submitted to the Journal must not violate the
Copyright law. Sending the manuscript to the Editorial Board, the authors retain all rights of the
owners of the manuscript and transfer the nonexclusive rights to publish the article in Russian
(or the language of the article, if not Russian) and its distribution in Russia and abroad to the
Founders and the Editorial Board. Authors should submit a letter to the Editorial Board in the
following form:

{\bfseries\textit{Agreement on the transfer of rights to publish:}}

``\textit{We, the undersigned authors of the manuscript ``\ldots'', pass to the
Founder and the Editorial Board of the Journal ``Informatics and Applications''
the nonexclusive right to publish the manuscript of the article in Russian (or
in English) in both print and electronic versions of the Journal. We affirm
that this publication does not violate the Copyright of other persons or
organizations.}

\textit{Author(s) signature(s): (name(s), address(es), date).}

This agreement should be submitted in paper form or in the form of a scanned copy (signed by
the authors).


%The Editorial Board has the right to request from the authors an official expert conclusion that
%the submitted article has no secret data prohibited for publication. \\[-13.5pt]
\item
A submitted article should be attached with \textbf{the data on the author(s)} (see item~8). If
there are several authors, the contact person should be indicated who is responsible for
correspondence with the Editorial Board and other authors about revisions and final approval
of the proofs.\\[-13.5pt]

\item The Editorial Board of the Journal examines the article according to the established
reviewing procedure. If the authors receive their article for correction after reviewing, it does not
mean that the article is approved for publication. The corrected article should be sent to the
Editorial Board for the subsequent review and approval.\\[-13.5pt]

\item The decision on the article publication or its rejection is communicated to the authors. The
Editorial Board may also send the reviews on the submitted articles to the authors. Any
discussion upon the rejected articles is not possible.\\[-13.5pt]

\item The edited articles will be sent to the authors for proofread. The comments of the authors
to the edited text of the article should be sent to the Editorial Board as soon as possible.\\[-13.5pt]

\item The manuscript of the article should be presented electronically in the MS WORD (.doc or
.docx) or \LaTeX\ (.tex) formats, and additionally in the .pdf format. All documents
 may be sent
by e-mail or provided on a CD or diskette. A~hard copy submission is not necessary.\\[-13.5pt]

\item The recommended typesetting instructions for manuscript.

Pages parameters: format A4, portrait orientation, document margins (cm): left~--- 2.5, right~---
1.5, above~--- 2.0, below~--- 2.0, footer 1.3.

Text: font~---Times New Roman, font size~--- 14, paragraph indent~--- 0.5, line spacing~--- 1.5,
justified alignment.

The recommended manuscript size: not more than 15~pages of the specified format.
If the specified size exceeded, the editorial board is entitled to require the author
to reduce the manuscript.

Use only standard abbreviations. Avoid  abbreviations in the title and
abstract. The full term for which an abbreviation stands should precede
its first use in the text unless it is a standard unit of measurement.

All pages of the manuscript should be numbered.

The templates for the manuscript typesetting are presented on site: {\sf
http://www.ipiran.ru/journal/template.doc}.\\[-13.5pt]


%\def\leftkol{Requirements for manuscripts submitted to Journal
%``Informatics~and~Applications''}

\item The articles should enclose data both in \textbf{Russian and English}:
\begin{itemize}
\item title;\\[-13.5pt]
\item author's name and surname;\\[-13.5pt]
\item affiliation~--- organization, its address with ZIP code, city, country, and
official e-mail address;\\[-13.5pt]
\item data on authors according to the format: (see site)

{\sf http://www.ipiran.ru/journal/issues/2013\_07\_01/authors.asp}  and

{\sf  http://www.ipiran.ru/journal/issues/2013\_07\_01\_eng/authors.asp};\\[-13.5pt]

\pagebreak

\def\leftfootline{\small{\textbf{\thepage}
\hfill INFORMATIKA I EE PRIMENENIYA~--- INFORMATICS AND APPLICATIONS\ \ \ 2019\
\ \ volume~13\ \ \ issue\ 4}
}%
 \def\rightfootline{\small{INFORMATIKA I EE PRIMENENIYA~--- INFORMATICS AND APPLICATIONS\ \ \ 2019\ \ \ volume~13\ \ \ issue\ 4
\hfill \textbf{\thepage}}}


%\def\leftkol{Requirements for manuscripts submitted to Journal
%``Informatics~and~Applications''}

%\def\rightkol{Requirements for manuscripts submitted to Journal
%``Informatics~and~Applications''}



\item abstract (not less than 100 words) both in Russian and in English. Abstract is a short
summary of the article that can be published separately. The abstract is the
main source of information on the article and it could be included in leading information
systems and data bases. The abstract in English has to be an original text and should
not be an exact translation of the Russian one. Good English is required.
In abstracts, avoid references and formulae;\\[-13.5pt]
\item indexing is performed on the basis of keywords. The use of keywords from the
internationally accepted thematic Thesauri is recommended.

%\def\leftkol{Requirements for manuscripts submitted to Journal
%``Informatics~and~Applications''}

%\def\rightkol{Requirements for manuscripts submitted to Journal
%``Informatics~and~Applications''}

Important! Keywords must not be sentences;
\item Acknowledgments.
\end{itemize}

\item References. Russian references have to be presented both in English translation and Latin
transliteration (refer {\sf http://www.translit.net/ru/bgn/}).

Please take into account the following examples of Russian references appearance:

\noindent
\textbf{Article in journal:}

\Aue{Zhang, Z., and D.~Zhu}. 2008. Experimental research on the localized electrochemical
micromachining.
\textit{Rus. J.~Electrochem.}  44(8):926--930. {\sf doi:10.1134/S1023193508080077}.


\noindent
\textbf{Journal article in electronic format:}

\Aue{Swaminathan, V., E.~Lepkoswka-White, and B.\,P.~Rao}. 1999. Browsers or buyers in
cyberspace? An
investigation of electronic factors influencing electronic exchange. \textit{JCMC}
5(2). Available at: {\sf http://www.ascusc.org/jcmc/vol5/issue2/} (accessed April~28, 2011).




\noindent
\textbf{Article from the continuing publication (collection of works, proceedings):}

\Aue{Astakhov, M.\,V., and T.\,V.~Tagantsev}. 2006. Eksperimental'noe
issledovanie prochnosti soedineniy ``stal'--kompozit'' [Experimental study of
the strength of joints ``steel--composite'']. \textit{Trudy MGTU
``Matematicheskoe modelirovanie slozhnykh tekh\-ni\-che\-skikh sistem''}
[\textit{Bauman MSTU ``Mathematical Modeling of Complex Technical
Systems'' Proceedings}]. 593:125--130.

\def\leftfootline{\small{\textbf{\thepage}
\hfill INFORMATIKA I EE PRIMENENIYA~--- INFORMATICS AND APPLICATIONS\ \ \ 2019\
\ \ volume~13\ \ \ issue\ 4}
}%
 \def\rightfootline{\small{INFORMATIKA I EE PRIMENENIYA~--- INFORMATICS AND APPLICATIONS\ \ \ 2019\ \ \ volume~13\ \ \ issue\ 4
\hfill \textbf{\thepage}}}

\def\leftkol{Requirements for manuscripts submitted to Journal
``Informatics~and~Applications''}

\def\rightkol{Requirements for manuscripts submitted to Journal
``Informatics~and~Applications''}

\noindent
\textbf{Conference proceedings:}

\Aue{Usmanov, T.\,S., A.\,A.~Gusmanov, I.\,Z.~Mullagalin, R.\,Ju.~Muhametshina,
A.\,N.~Chervyakova, and
A.\,V.~Sveshnikov}. 2007. Osobennosti proektirovaniya razrabotki mestorozhdeniy
s primeneniem gidrorazryva
plasta [Features of the design of field development with the use of hydraulic fracturing].
\textit{Trudy 6-go
Mezhdu\-na\-rod\-no\-go Simpoziuma ``Novye resursosberegayushchie tekhnologii
nedropol'zovaniya i povysheniya
neftegazootdachi''} [\textit{6th  Symposium (International) ``New Energy Saving Subsoil
Technologies and
the Increasing of the Oil and Gas Impact'' Proceedings}]. Moscow. 267--272.


\noindent
\textbf{Books and other monographs:}




Lindorf, L.\,S., and L.\,G.~Mamikoniants, eds. 1972.
\textit{Ekspluatatsiya turbogeneratorov s neposredstvennym
okhlazhdeniem} [\textit{Operation of turbine generators with direct cooling}].
Moscow: Energy Publs. 352~p.


%\Aue{Latyshev, V.\,N.} 2009. \textit{Tribologiya rezaniya. Kn.~1: Frikcionnye prosessy
%pri rezanii metallov}
%[\textit{Tribology of cutting. Vol.~1: Frictional processes in metal cutting}]. Ivanovo: Ivanovskii
%State Univ. 108~p.


%\noindent
%\textbf{Unpublished material:}

%\Aue{Latypov, A.\,R., M.\,M.~Khasanov, and V.\,A.~Baikov}.
%2004. Geology and production (NGT GiD). Certificate on official registration of the computer
%program
%No.\,2004611198. (In Russian, unpubl.)

%\noindent
%\textbf{Internet-source:}

%APA Style. 2011. Available at: {\sf http://www.apastyle.org/apa-style-help.aspx} (accessed
%February~5, 2011).

%Pravila citirovaniya istochnikov [Rules for the citing of sources]. Available at: {\sf
%http://www.scribd.com/doc/1034528/} (accessed February~7, 2011).


\noindent
\textbf{Dissertation and Thesis:}

%\Aue{Semenov, V.\,I.}
%2003. Matematicheskoe modelirovanie plazmy v sisteme kompaktnyy tor. [Mathematical
%modeling of the plasma in the compact torus]. D.Sc.\ Diss. Moscow. 272~p.

\Aue{Kozhunova, O.\,S.} 2009. Tekhnologiya razrabotki semanticheskogo
slovarya informatsionnogo monitoringa [Technology of development of
semantic dictionary of information monitoring system]. PhD Thesis. Moscow: IPI RAN. 23~p.


\noindent
\textbf{State standards and patents:}

GOST 8.586.5-2005. 2007. Metodika vypolneniya izmereniy. Izmerenie raskhoda i~kolichestva
zhidkostey i gazov 
s~pomoshch'yu standartnykh suzhayushchikh ustroystv [Method of measurement.
Measurement of flow rate and volume of liquids and gases by means of orifice devices]. M.:
Standardinform
Publs. 10~p.

%\noindent
%\textbf{Patent:}

\Aue{Bolshakov, M.\,V., A.\,V.~Kulakov, A.\,N.~Lavrenov, and M.\,V.~Palkin}.
2006. Sposob orientirovaniya po krenu letatel'nogo
apparata s opti\-che\-skoy golovkoy
samonavedeniya [The way to orient on the roll of aircraft with optical homing head].
Patent RF No.\,2280590.

References in Latin transcription are presented in the original language.

References in the text are numbered according to the order of their
first appearance; the number is
placed in square brackets. All items from the reference list should be
cited.\\[-13.5pt]

\item Manuscripts and additional materials are not returned to Authors by the Editorial Board.\\[-13.5pt]

\item Submissions of files by e-mail must include:\\[-13.5pt]
\begin{itemize}
\item   the journal title and author's name in the ``Subject'' field; \\[-13.5pt]
\item   an article and additional materials have to be attached using the ``attach'' function;\\[-13.5pt]
\item   an electronic version of the article should contain the file with the text and a separate file
with figures.\\[-13.5pt]
\end{itemize}

\item ``Informatics and Applications'' journal is not a profit publication. There are no
charges for the authors as well as there are no royalties.\\[-13.5pt]
\end{enumerate}

\def\leftfootline{\small{\textbf{\thepage}
\hfill INFORMATIKA I EE PRIMENENIYA~--- INFORMATICS AND APPLICATIONS\ \ \ 2019\
\ \ volume~13\ \ \ issue\ 4}
}%
 \def\rightfootline{\small{INFORMATIKA I EE PRIMENENIYA~--- INFORMATICS AND APPLICATIONS\ \ \ 2019\ \ \ volume~13\ \ \ issue\ 4
\hfill \textbf{\thepage}}}

\def\leftkol{Requirements for manuscripts submitted to Journal
``Informatics~and~Applications''}

\def\rightkol{Requirements for manuscripts submitted to Journal
``Informatics~and~Applications''}


%\vspace*{5mm}


\begin{center}
\textbf{Editorial Board address:} \\

%ABOUT AUTHORS



FRC CSC RAS, 44, block~2, Vavilov Str., Moscow 119333, Russia\\[-10pt]

\

Ph.: +7\,(499)\,135\,86\,92,\ \ Fax: +7\,(495)\,930\,45\,05\\[-10pt]

\

 e-mail: {\sf rust@ipiran.ru} (to Prof.\ Rustem Seyful-Mulyukov)\\[-10pt]

\

 {\sf http://www.ipiran.ru/english/journal.asp}
\end{center}
 }
%\thispagestyle{myheadings}

\def\leftkol{Requirements for manuscripts submitted to Journal
``Informatics~and~Applications''}

\def\rightkol{Requirements for manuscripts submitted to Journal
``Informatics~and~Applications''}

\def\leftfootline{\small{\textbf{\thepage}
\hfill INFORMATIKA I EE PRIMENENIYA~--- INFORMATICS AND APPLICATIONS\ \ \ 2019\
\ \ volume~13\ \ \ issue\ 4}
}%
 \def\rightfootline{\small{INFORMATIKA I EE PRIMENENIYA~--- INFORMATICS AND APPLICATIONS\ \ \ 2019\ \ \ volume~13\ \ \ issue\ 4
\hfill \textbf{\thepage}}}

 \label{end\stat}

\newpage


%\vspace*{-60pt} {\small
{\baselineskip=9.1pt
\section*{Правила подготовки рукописей статей для публикации в журнале
<<Информатика и её применения>>}

\thispagestyle{empty}

 Журнал <<Информатика и её применения>> публикует
теоретические, обзорные и дискуссионные статьи, посвященные научным
исследованиям и разработкам в области информатики и ее приложений. Журнал
издается на русском языке. По специальному решению редколлегии отдельные статьи,
в виде исключения, могут печататься на английском языке.
Тематика журнала охватывает следующие направления:
\begin{itemize}
\item теоретические основы информатики; %\\[-13.5pt]
\item математические методы исследования сложных систем и процессов; %\\[-13.5pt]
\item информационные системы и сети; %\\[-13.5pt]
\item информационные технологии; %\\[-13.5pt]
\item архитектура и программное
обеспечение вычислительных комплексов и сетей.
\end{itemize}
\begin{enumerate}
\item В журнале печатаются результаты, ранее не
опубликованные и не предназначенные к одновременной публикации в других
изданиях. Публикация не должна нарушать закон об авторских правах. Направляя
свою рукопись в редакцию, авторы автоматически передают учредителям и
редколлегии неисключительные права на издание данной статьи на русском языке и
на ее распространение в России и за рубежом. При этом за авторами сохраняются
все права как собственников данной рукописи. В связи с этим авторами должно
быть представлено в редакцию письмо в следующей форме:
Соглашение о передаче права на публикацию:

\textit{<<Мы, нижеподписавшиеся, авторы рукописи <<$\qquad\qquad$>>, передаем
учредителям и редколлегии журнала <<Информатика и её применения>>
неисключительное право опубликовать данную рукопись статьи на русском языке как
в печатной, так и в электронной версиях журнала. Мы подтверждаем, что данная
публикация не нарушает авторского права других лиц или организаций. Подписи
авторов: (ф.\,и.\,о., дата, адрес)>>.}

Указанное соглашение может быть представлено 
как в бумажном виде, так и в виде отсканированной копии (с подписями авторов).


Редколлегия вправе запросить у авторов экспертное заключение о возможности
опубликования представленной статьи в открытой печати. %\\[-13.5pt]
\item Статья
подписывается всеми авторами. На отдельном листе представляются данные автора
(или всех авторов): фамилия, полные имя и отчество, телефон, факс, e-mail,
почтовый адрес. Если работа выполнена несколькими авторами, указывается фамилия
одного из них, ответственного за переписку с редакцией. %\\[-13.5pt]
\item Редакция журнала
осуществляет самостоятельную экспертизу присланных статей. Возвращение рукописи
на доработку не означает, что статья уже принята к печати. Доработанный вариант
с ответом на замечания рецензента необходимо прислать в редакцию. %\\[-13.5pt]
\item Решение
редакционной коллегии о принятии статьи к печати или ее отклонении сообщается
авторам. Редколлегия не обязуется направлять рецензию авторам отклоненной
статьи. %\\[-13.5pt]
\item Корректура статей высылается авторам для просмотра. Редакция
просит авторов присылать свои замечания в кратчайшие сроки. %\\[-13.5pt]
\item При
подготовке рукописи в MS Word рекомендуется использовать следующие настройки.
Параметры страницы: формат~--- А4; ориентация~--- книжная; поля (см): внутри~---
2,5, снаружи~--- 1,5, сверху~--- 2, снизу~--- 2, от края до нижнего
колонтитула~--- 1,3. Основной текст: стиль~--- <<Обычный>>: шрифт Times New
Roman, размер 14~пунктов, абзацный отступ~--- 0,5~см, 1,5 интервала,
выравнивание~--- по ширине. Рекомендуемый объем рукописи~--- не свыше
25~страниц указанного формата. Ознакомиться с шаблонами, содержащими примеры
оформления, можно по адресу в Интернете:
\textsf{http://www.ipiran.ru/journal/template.doc}.
\item К рукописи, предоставляемой в 2-х
экземплярах, обязательно прилагается электронная версия статьи (как правило, в
форматах MS WORD (.doc) или \LaTeX\ (.tex), а также~--- дополнительно~--- в
формате .pdf) на дискете, лазерном диске или по электронной почте. Сокращения
слов, кроме стандартных, не применяются. Все страницы рукописи должны быть
пронумерованы. %\\[-13.5pt]
\item Статья должна содержать следующую информацию на русском и
английском языках: название, Ф.И.О. авторов, места работы авторов и их
электронные адреса, подробные сведения об авторах, оформленные в соответствии с форматом, 
определяемым файлами {\sf http://www.ipiran.ru/journal/issues/2011\_05\_01/authors.asp} и 
{\sf http://www.ipiran.ru/journal/issues/2011\_01\_eng/authors.asp},
аннотация (не более 100~слов), ключевые слова. Ссылки на
литературу в тексте статьи нумеруются (в квадратных скобках) и располагаются в
порядке их первого упоминания. В~списке литературы не должно быть позиций, на которые нет ссылки в тексте статьи.
Все фамилии авторов, заглавия статей, названия
книг, конференций и~т.\,п.\ даются на языке оригинала, если этот язык
использует кириллический или латинский алфавит. %\\[-13.5pt]
\item Присланные в редакцию материалы авторам не возвращаются.
\item При отправке файлов по электронной
почте просим придерживаться следующих правил:
\begin{itemize}
\item указывать в поле subject (тема) название журнала и фамилию автора; %\\[-13.5pt]
\item использовать attach (присоединение); %\\[-13.5pt]
\item в случае больших объемов информации возможно
использование общеизвестных архиваторов (ZIP, RAR); %\\[-13.5pt]
\item в состав электронной версии статьи должны входить: файл, содержащий текст статьи, и файл(ы),
содержащий(е) иллюстрации. %\\[-13.5pt]
\end{itemize}
\item Журнал <<Информатика и её применения>> является некоммерческим изданием. 
Плата за публикацию с авторов не взимается, гонорар авторам не выплачивается.
\end{enumerate}
\thispagestyle{empty}
\textbf{Адрес редакции:} Москва 119333,
ул.~Вавилова, д.~44, корп.~2, ИПИ РАН\\
\hphantom{\textbf{Адрес редакции:} }Тел.: +7 (499) 135-86-92\ \
Факс:  +7 (495) 930-45-05\ \  E-mail:   rust@ipiran.ru }
}

\end{document}


%\tableofcontents

%\end{document}





%\def\stat{cont}
{%\hrule\par
%\vskip 7pt % 7pt
\raggedleft\Large \bf%\baselineskip=3.2ex
А\,В\,Т\,О\,Р\,С\,К\,И\,Й\ \ У\,К\,А\,З\,А\,Т\,Е\,Л\,Ь\ \ З\,А\ \ 2\,0\,0\,7 г. \vskip 17pt
    \hrule
    \par
\vskip 21pt plus 6pt minus 3pt }

\label{st\stat}

\def\tit{\ }

\def\aut{\ }
\def\auf{\ }

\def\leftkol{\ } % ENGLISH ABSTRACTS}

\def\rightkol{\ } %ENGLISH ABSTRACTS}

\titele{\tit}{\aut}{\auf}{\leftkol}{\rightkol}


\contentsline {chapter}{\ }{Выпуск \quad Стр.} 
\contentsline {section}{\textbf{Батракова Д.\,А., Королев В.\,Ю., Шоргин С.\,Я.}\ \ Новый метод вероятностно-ста\-ти\-сти\-че\-ско\-го анализа информационных потоков в\nobreakspace {}телекоммуникационных сетях}{\qquad 1 \qquad 40} 
\contentsline {section}{\textbf{Борисов А.\,В.}\ \ Байесовское оценивание в системах наблюдения с\nobreakspace {}марковскими скачкообразными процессами: игровой подход}{\qquad 2 \qquad 65}
\contentsline {section}{\textbf{Босов А.\,В., Иванов А.\,В.}\ \ Программная инфраструктура информационного Web-пор\-тала}{\qquad 2 \qquad 50}
\contentsline {section}{\textbf{Захаров В.\,Н., Калиниченко Л.\,А., Соколов И.\,А., Ступников С.\,А.}\ \ Конструирование канонических информационных моделей для интегрированных информационных систем}{\qquad 2 \qquad 15}
\contentsline {section}{\textbf{Захаров В.\,Н., Козмидиади В.\,А.}\ \ Средства обеспечения отказоустойчивости при\-ло\-жений}{\qquad 1 \qquad 14} 
\contentsline {section}{\textbf{Иванов А.\,В.}\ \ см. Босов А.\,В.\hfill\hfill\hfill\hfill\hfill\hfill\hfill\hfill\hfill\hfill\hfill\hfill\hfill\hfill\hfill\hfill\hfill\hfill\hfill\hfill\hfill\hfill\hfill\hfill\hfill\hfill\hfill\hfill\hfill\hfill\hfill\hfill\hfill\hfill\hfill}{\ }
\contentsline {section}{\textbf{Ильин В.\,Д., Соколов И.\,А.}\ \ Символьная модель системы знаний информатики в\nobreakspace {}че\-ло\-ве\-ко-автоматной среде}{\qquad 1 \qquad 66} 
\contentsline {section}{\textbf{Калиниченко Л.\,А.}\ \ см. Захаров В.\,Н.\hfill\hfill\hfill\hfill\hfill\hfill\hfill\hfill\hfill\hfill\hfill\hfill\hfill\hfill\hfill\hfill\hfill\hfill\hfill\hfill\hfill\hfill\hfill\hfill\hfill\hfill\hfill\hfill\hfill\hfill\hfill\hfill\hfill\hfill\hfill}{\ }
\contentsline {section}{\textbf{Козеренко Е.\,Б.}\ \ Лингвистическое моделирование для систем машинного перевода и обработки знаний}{\qquad 1 \qquad 54} 
\contentsline {section}{\textbf{Козмидиади В.\,А.}\ \ см. Захаров В.\,Н.\hfill\hfill\hfill\hfill\hfill\hfill\hfill\hfill\hfill\hfill\hfill\hfill\hfill\hfill\hfill\hfill\hfill\hfill\hfill\hfill\hfill\hfill\hfill\hfill\hfill\hfill\hfill\hfill\hfill\hfill\hfill\hfill\hfill\hfill\hfill }{\ } 
\contentsline {section}{\textbf{Королев В.\,Ю.}\ \ см. Батракова Д.\,А.\hfill\hfill\hfill\hfill\hfill\hfill\hfill\hfill\hfill\hfill\hfill\hfill\hfill\hfill\hfill\hfill\hfill\hfill\hfill\hfill\hfill\hfill\hfill\hfill\hfill\hfill\hfill\hfill\hfill\hfill\hfill\hfill\hfill\hfill\hfill}{\ } 
\contentsline {section}{\textbf{Кудрявцев А.\,А., Шоргин С.\,Я.}\ \ Байесовский подход к\nobreakspace {}анализу систем массового обслуживания и\nobreakspace {}показателей надежности}{\qquad 2 \qquad 76}
\contentsline {section}{\textbf{Печинкин А.\,В., Соколов И.\,А., Чаплыгин В.\,В.}\ \ Многолинейная система массового обслуживания с конечным накопителем и ненадежными приборами}{\qquad 1 \qquad 27} 
\contentsline {section}{\textbf{Печинкин А.\,В., Соколов И.\,А., Чаплыгин В.\,В.}\ \ Стационарные характеристики многолинейной\nobreakspace {}системы массового обслуживания с\nobreakspace {}одновременными отказами приборов}{\qquad 2 \qquad 39}
\contentsline {section}{\textbf{Синицын И.\,Н.}\ \ Корреляционные методы построения аналитических информационных моделей флуктуаций полюса Земли по априорным данным}{\qquad 2 \qquad \hphantom{9}2}
\contentsline {section}{\textbf{Синицын И.\,Н.}\ \ Развитие теории фильтров Пугачева для оперативной обработки информации в стохастических системах}{{\qquad 1 \qquad \hphantom{9}3}} 
\contentsline {section}{\textbf{Соколов И.\,А.}\ \ см. Захаров В.\,Н.\hfill\hfill\hfill\hfill\hfill\hfill\hfill\hfill\hfill\hfill\hfill\hfill\hfill\hfill\hfill\hfill\hfill\hfill\hfill\hfill\hfill\hfill\hfill\hfill\hfill\hfill\hfill\hfill\hfill\hfill\hfill\hfill\hfill\hfill\hfill}{\ }
\contentsline {section}{\textbf{Соколов И.\,А.}\ \ см. Ильин В.\,Д.\hfill\hfill\hfill\hfill\hfill\hfill\hfill\hfill\hfill\hfill\hfill\hfill\hfill\hfill\hfill\hfill\hfill\hfill\hfill\hfill\hfill\hfill\hfill\hfill\hfill\hfill\hfill\hfill\hfill\hfill\hfill\hfill\hfill\hfill\hfill}{\ } 
\contentsline {section}{\textbf{Соколов И.\,А.}\ \ см. Печинкин А.\,В.\hfill\hfill\hfill\hfill\hfill\hfill\hfill\hfill\hfill\hfill\hfill\hfill\hfill\hfill\hfill\hfill\hfill\hfill\hfill\hfill\hfill\hfill\hfill\hfill\hfill\hfill\hfill\hfill\hfill\hfill\hfill\hfill\hfill\hfill\hfill}{\ } 
\contentsline {section}{\textbf{Соколов И.\,А.}\ \ см. Печинкин А.\,В.\hfill\hfill\hfill\hfill\hfill\hfill\hfill\hfill\hfill\hfill\hfill\hfill\hfill\hfill\hfill\hfill\hfill\hfill\hfill\hfill\hfill\hfill\hfill\hfill\hfill\hfill\hfill\hfill\hfill\hfill\hfill\hfill\hfill\hfill\hfill}{\ }
\contentsline {section}{\textbf{Ступников С.\,А.}\ \ см. Захаров В.\,Н.\hfill\hfill\hfill\hfill\hfill\hfill\hfill\hfill\hfill\hfill\hfill\hfill\hfill\hfill\hfill\hfill\hfill\hfill\hfill\hfill\hfill\hfill\hfill\hfill\hfill\hfill\hfill\hfill\hfill\hfill\hfill\hfill\hfill\hfill\hfill}{\ }
\contentsline {section}{\textbf{Чаплыгин В.\,В.}\ \ см. Печинкин А.\,В.\hfill\hfill\hfill\hfill\hfill\hfill\hfill\hfill\hfill\hfill\hfill\hfill\hfill\hfill\hfill\hfill\hfill\hfill\hfill\hfill\hfill\hfill\hfill\hfill\hfill\hfill\hfill\hfill\hfill\hfill\hfill\hfill\hfill\hfill\hfill}{\ } 
\contentsline {section}{\textbf{Чаплыгин В.\,В.}\ \ см. Печинкин А.\,В.\hfill\hfill\hfill\hfill\hfill\hfill\hfill\hfill\hfill\hfill\hfill\hfill\hfill\hfill\hfill\hfill\hfill\hfill\hfill\hfill\hfill\hfill\hfill\hfill\hfill\hfill\hfill\hfill\hfill\hfill\hfill\hfill\hfill\hfill\hfill}{\ }
\contentsline {section}{\textbf{Шоргин С.\,Я.}\ \ см. Батракова Д.\,А.\hfill\hfill\hfill\hfill\hfill\hfill\hfill\hfill\hfill\hfill\hfill\hfill\hfill\hfill\hfill\hfill\hfill\hfill\hfill\hfill\hfill\hfill\hfill\hfill\hfill\hfill\hfill\hfill\hfill\hfill\hfill\hfill\hfill\hfill\hfill}{\ } 
\contentsline {section}{\textbf{Шоргин С.\,Я.}\ \ см. Кудрявцев А.\,А.\hfill\hfill\hfill\hfill\hfill\hfill\hfill\hfill\hfill\hfill\hfill\hfill\hfill\hfill\hfill\hfill\hfill\hfill\hfill\hfill\hfill\hfill\hfill\hfill\hfill\hfill\hfill\hfill\hfill\hfill\hfill\hfill\hfill\hfill\hfill}{\ }
%\thispagestyle{myheadings}
\def\leftfootline{\small{\textbf{\thepage}
\hfill ИНФОРМАТИКА И ЕЁ ПРИМЕНЕНИЯ\ \ \ том~1\ \ \ выпуск~2\ \ \ 2007}
}%
 \def\rightfootline{\small{ИНФОРМАТИКА И ЕЁ ПРИМЕНЕНИЯ\ \ \ том~1\ \ \ выпуск~2\ \ \ 2007
 \hfill \textbf{\thepage}}}
 \label{end\stat}

%\def\stat{cont-e}
{%\hrule\par
%\vskip 7pt % 7pt
\raggedleft\Large \bf%\baselineskip=3.2ex
2\,0\,0\,7\ \ A\,U\,T\,H\,O\,R\ \ I\,N\,D\,E\,X \vskip 17pt
    \hrule
    \par
\vskip 21pt plus 6pt minus 3pt }

\label{st\stat}

\def\tit{\ }

\def\aut{\ }
\def\auf{\ }

\def\leftkol{\ } % ENGLISH ABSTRACTS}

\def\rightkol{\ } %ENGLISH ABSTRACTS}

\titele{\tit}{\aut}{\auf}{\leftkol}{\rightkol}


\contentsline {chapter}{\ }{Issue \quad Page} 
\contentsline {subsection}{\textbf{Batrakova D.\,A., Korolev V.\,Yu., Shorgin S.\,Ya.}\ \ A New Method for the Probabilistic and Statistical Analysis of Information Flows in Telecommunication Networks}{\qquad 1 \qquad 40} 
\contentsline {subsection}{\textbf{Borisov A.\,V.}\ \ Bayesian Estimation in\nobreakspace {}Observation Systems with\nobreakspace {}Markov Jump Processes: Game-Theoretic Approach}{\qquad 2 \qquad 65} 
\contentsline {subsection}{\textbf{Bosov A.\,V., Ivanov A.\,V.}\ \ Linguistic Simulation for Machine Translation and Knowledge Management Systems}{\qquad 2 \qquad 50} 
\contentsline {subsection}{\textbf{Chaplygin V.\,V.} see Pechinkin A.\,V.\hfill\hfill\hfill\hfill\hfill\hfill\hfill\hfill\hfill\hfill\hfill\hfill\hfill\hfill\hfill\hfill\hfill\hfill\hfill\hfill\hfill\hfill\hfill\hfill\hfill\hfill\hfill\hfill\hfill\hfill\hfill\hfill\hfill\hfill\hfill}{\ }
\contentsline {subsection}{\textbf{Chaplygin V.\,V.} see Pechinkin A.\,V.\hfill\hfill\hfill\hfill\hfill\hfill\hfill\hfill\hfill\hfill\hfill\hfill\hfill\hfill\hfill\hfill\hfill\hfill\hfill\hfill\hfill\hfill\hfill\hfill\hfill\hfill\hfill\hfill\hfill\hfill\hfill\hfill\hfill\hfill\hfill}{\ }
\contentsline {subsection}{\textbf{Ilyin V.\,D., Sokolov I.\,A.}\ \ The Symbol Model of Informatics Knowledge System in Human-Automaton Environment}{\qquad 1 \qquad 66} 
\contentsline {subsection}{\textbf{Ivanov A.\,V.} see Bosov A.\,V.\hfill\hfill\hfill\hfill\hfill\hfill\hfill\hfill\hfill\hfill\hfill\hfill\hfill\hfill\hfill\hfill\hfill\hfill\hfill\hfill\hfill\hfill\hfill\hfill\hfill\hfill\hfill\hfill\hfill\hfill\hfill\hfill\hfill\hfill\hfill}{\ }
\contentsline {subsection}{\textbf{Kalinichenko L.\,A.} see Zakharov V.\,N.\hfill\hfill\hfill\hfill\hfill\hfill\hfill\hfill\hfill\hfill\hfill\hfill\hfill\hfill\hfill\hfill\hfill\hfill\hfill\hfill\hfill\hfill\hfill\hfill\hfill\hfill\hfill\hfill\hfill\hfill\hfill\hfill\hfill\hfill\hfill}{\ }
\contentsline {subsection}{\textbf{Korolev V.\,Yu.} see Batrakova D.\,A.\hfill\hfill\hfill\hfill\hfill\hfill\hfill\hfill\hfill\hfill\hfill\hfill\hfill\hfill\hfill\hfill\hfill\hfill\hfill\hfill\hfill\hfill\hfill\hfill\hfill\hfill\hfill\hfill\hfill\hfill\hfill\hfill\hfill\hfill\hfill}{\ }
\contentsline {subsection}{\textbf{Kozerenko E.\,B.}\ \ Linguistic Simulation for Machine Translation and Knowledge Management Systems}{\qquad 1 \qquad 54} 
\contentsline {subsection}{\textbf{Kozmidiady V.\,A.} see Zakharov V.\,N.\hfill\hfill\hfill\hfill\hfill\hfill\hfill\hfill\hfill\hfill\hfill\hfill\hfill\hfill\hfill\hfill\hfill\hfill\hfill\hfill\hfill\hfill\hfill\hfill\hfill\hfill\hfill\hfill\hfill\hfill\hfill\hfill\hfill\hfill\hfill}{\ }
\contentsline {subsection}{\textbf{Kudryavtsev A.\,A., Shorgin S.\,Ya.}\ \ Bayesian Approach to Queueing Systems and Reliability Characteristics}{\qquad 2 \qquad 76} 
\contentsline {subsection}{\textbf{Pechinkin A.\,V., Sokolov I.\,A., Chaplygin V.\,V.}\ \ Multichannel Queuing System with Finite Buffer and Unreliable Servers}{\qquad 1 \qquad 27} 
\contentsline {subsection}{\textbf{Pechinkin A.\,V., Sokolov I.\,A., Chaplygin V.\,V.}\ \ Stationary Characteristics of a Multichannel Queueing System with\nobreakspace {}Simultaneous Refusals of Servers}{\qquad 2 \qquad 39} 
\contentsline {subsection}{\textbf{Shorgin S.\,Ya.} see Batrakova D.\,A.\hfill\hfill\hfill\hfill\hfill\hfill\hfill\hfill\hfill\hfill\hfill\hfill\hfill\hfill\hfill\hfill\hfill\hfill\hfill\hfill\hfill\hfill\hfill\hfill\hfill\hfill\hfill\hfill\hfill\hfill\hfill\hfill\hfill\hfill\hfill}{\ }
\contentsline {subsection}{\textbf{Shorgin S.\,Ya.} see Kudryavtsev A.\,A.\hfill\hfill\hfill\hfill\hfill\hfill\hfill\hfill\hfill\hfill\hfill\hfill\hfill\hfill\hfill\hfill\hfill\hfill\hfill\hfill\hfill\hfill\hfill\hfill\hfill\hfill\hfill\hfill\hfill\hfill\hfill\hfill\hfill\hfill\hfill}{\ }
\contentsline {subsection}{\textbf{Sinitsyn I.\,N.}\ \ Correlational Methods for Analytical Informational Models of the Earth Pole Fluctuations Design Based on a priori Data}{\qquad 2 \qquad \hphantom{9}2}
\contentsline {subsection}{\textbf{Sinitsyn I.\,N.}\ \ Development of Pugachev Filtering for Stochastic Systems}{\qquad 1 \qquad \hphantom{9}3}
\contentsline {subsection}{\textbf{Sokolov I.\,A.} see Ilyin V.\,D.\hfill\hfill\hfill\hfill\hfill\hfill\hfill\hfill\hfill\hfill\hfill\hfill\hfill\hfill\hfill\hfill\hfill\hfill\hfill\hfill\hfill\hfill\hfill\hfill\hfill\hfill\hfill\hfill\hfill\hfill\hfill\hfill\hfill\hfill\hfill}{\ }
\contentsline {subsection}{\textbf{Sokolov I.\,A.} see Pechinkin A.\,V.\hfill\hfill\hfill\hfill\hfill\hfill\hfill\hfill\hfill\hfill\hfill\hfill\hfill\hfill\hfill\hfill\hfill\hfill\hfill\hfill\hfill\hfill\hfill\hfill\hfill\hfill\hfill\hfill\hfill\hfill\hfill\hfill\hfill\hfill\hfill}{\ }
\contentsline {subsection}{\textbf{Sokolov I.\,A.} see Pechinkin A.\,V.\hfill\hfill\hfill\hfill\hfill\hfill\hfill\hfill\hfill\hfill\hfill\hfill\hfill\hfill\hfill\hfill\hfill\hfill\hfill\hfill\hfill\hfill\hfill\hfill\hfill\hfill\hfill\hfill\hfill\hfill\hfill\hfill\hfill\hfill\hfill}{\ }
\contentsline {subsection}{\textbf{Sokolov I.\,A.} see Zakharov V.\,N.\hfill\hfill\hfill\hfill\hfill\hfill\hfill\hfill\hfill\hfill\hfill\hfill\hfill\hfill\hfill\hfill\hfill\hfill\hfill\hfill\hfill\hfill\hfill\hfill\hfill\hfill\hfill\hfill\hfill\hfill\hfill\hfill\hfill\hfill\hfill}{\ }
\contentsline {subsection}{\textbf{Stupnikov S.\,A.} see Zakharov V.\,N.\hfill\hfill\hfill\hfill\hfill\hfill\hfill\hfill\hfill\hfill\hfill\hfill\hfill\hfill\hfill\hfill\hfill\hfill\hfill\hfill\hfill\hfill\hfill\hfill\hfill\hfill\hfill\hfill\hfill\hfill\hfill\hfill\hfill\hfill\hfill}{\ }
\contentsline {subsection}{\textbf{Zakharov V.\,N., Kalinichenko L.\,A., Sokolov I.\,A., Stupnikov S.\,A.}\ \ Development of Canonical Information Models for Integrated Information Systems}{\qquad 2 \qquad 15} 
\contentsline {subsection}{\textbf{Zakharov V.\,N., Kozmidiady V.\,A.}\ \ Means Providing Applications Fault Tolerance}{\qquad 1 \qquad 14} 
\def\leftfootline{\small{\textbf{\thepage}
\hfill ИНФОРМАТИКА И ЕЁ ПРИМЕНЕНИЯ\ \ \ том~1\ \ \ выпуск~2\ \ \ 2007}
}%
 \def\rightfootline{\small{ИНФОРМАТИКА И ЕЁ ПРИМЕНЕНИЯ\ \ \ том~1\ \ \ выпуск~2\ \ \ 2007
 \hfill \textbf{\thepage}}}
 \label{end\stat}


%\tableofcontents


\end{document}

\newcommand{\Ack}{\subsection*{\protect\large\bf Acknowledgments}}