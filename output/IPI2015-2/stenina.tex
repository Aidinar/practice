\newcommand{\hx}{\hat{x}}
\newcommand{\hy}{\hat{y}}
\newcommand{\uu}{\mathbf{u}}

\newcommand{\x}{\mathbf{x}}
\newcommand{\hchi}{\hat{\boldsymbol{\chi}}}
\newcommand{\hphi}{\hat{\boldsymbol{\varphi}}}

\newcommand{\weight}{\mathbf{a}}
\newcommand{\pcd}{p(\cdot)}
\newcommand{\q}{q(\cdot)}

\def\stat{stenina}

\def\tit{СОГЛАСОВАНИЕ ПРОГНОЗОВ ПРИ РЕШЕНИИ ЗАДАЧ
ПРОГНОЗИРОВАНИЯ ИЕРАРХИЧЕСКИХ ВРЕМЕННЫХ РЯДОВ$^*$}

\def\titkol{Согласование прогнозов при решении задач
прогнозирования иерархических временных рядов}

\def\aut{М.\,М. Стенина$^1$,  В.\,В.~Стрижов$^2$}

\def\autkol{М.\,М. Стенина,  В.\,В.~Стрижов}

\titel{\tit}{\aut}{\autkol}{\titkol}

\index{Стенина М.\,М.}
\index{Стрижов В.\,В.}

{\renewcommand{\thefootnote}{\fnsymbol{footnote}} \footnotetext[1]
{Работа выполнена при финансовой поддержке РФФИ (проект~13-07-13139).}}


\renewcommand{\thefootnote}{\arabic{footnote}}
\footnotetext[1]{Московский физико-технический институт, Национальный исследовательский университет <<Высшая школа экономики>>, mmedvednikova@gmail.com}
\footnotetext[2]{Вычислительный центр Российской академии наук им.\ А.\,А.~Дородницына, strijov@ccas.com}


\Abst{Рассматривается задача одновременного прогнозирования
набора временн$\acute{\mbox{ы}}$х рядов, объединенных в~иерархическую
многоуровневую структуру. Требуется, чтобы полученные прогнозы
удовлетворяли физическим ограничениям и~структуре иерархии.
Предложен алгоритм согласования прогнозов иерархических
временн$\acute{\mbox{ы}}$х рядов GTOp (Game-theoretically optimal reconciliation),
гарантирующий неухудшение качества прогнозов после проведения
процедуры согласования по сравнению с~качеством прогнозов,
полученных для каждого временн$\acute{\mbox{о}}$го ряда независимо. Подход
базируется на поиске равновесия Нэша в~антагонистической игре
заданного вида и~сводит задачу согласования прогнозов к~задаче
оптимизации с~ограничениями типа равенства и~неравенства.
Доказывается, что при выполнении ряда общих предположений
о~свойствах структуры иерархии, физических ограничений и~функции
потерь в~игре существует равновесие Нэша в~чистых стратегиях.
Работа алгоритма демонстрируется на разных типах иерархических
структур с~использованием данных посуточной загруженности
железнодорожных узлов.}


\KW{иерархические временн$\acute{\mbox{ы}}$е ряды;
согласование прогнозов временн$\acute{\mbox{ы}}$х рядов; антагонистическая игра;
равновесие Нэша}

\DOI{10.14357/19922264150209}

\vspace*{6pt}


\vskip 14pt plus 9pt minus 6pt

\thispagestyle{headings}

\begin{multicols}{2}

\label{st\stat}



\section{Введение}

Рассматривается задача одновременного прогнозирования набора
временн$\acute{\mbox{ы}}$х рядов, связанных в~иерархическую многоуровневую
структуру, в~которой временн$\acute{\mbox{ы}}$е ряды каждого следующего (бо-\linebreak лее
высокого) уровня формируются путем поэлементного суммирования
некоторой части (возможно, всех) временн$\acute{\mbox{ы}}$х рядов предыдущего
уровня.
%
Используемые при решении этой задачи принцип равновесия
и~регрессионные методы широко обсуждаются в~научной литературе,
в~том числе на~страницах журнала
<<Информатика и~её~применения>>~\cite{tokmakova2012hyper, vasilyev2014using}.

Задача прогнозирования иерархических вре\-мен\-н$\acute{\mbox{ы}}$х рядов
возникает в~различных прикладных областях. В~\cite{hong2014global}
описан конкурс, проведенный в~2012~г.\ на Kaggle~\cite{kaggle},
в~котором одной из задач было прогнозирование иерархических
временн$\acute{\mbox{ы}}$х рядов с~требованием согласования прогнозов по иерархии.
В~работе~\cite{hyndman2011optimal} обсуждается задача
прогнозирования туристической активности по регионам и~целям
поездок. В~статье \cite{kuznetsov2011smoothing} решается задача
прогнозирования потребительского спроса на различные группы
товаров в~ряде магазинов. В~\cite{stenina2014reconciliation}
решается задача согласования прогнозов объемов железнодорожных
перевозок различных типов грузов по ряду железнодорожных веток.

В связи с~высоким интересом к~задаче разными авторами предлагаются
различные подходы к~ее решению. Как правило, сперва получают
ка\-ким-ли\-бо образом прогнозы всех (или некоторой части) временн$\acute{\mbox{ы}}$х
рядов независимо друг от друга, а~затем корректируют (согласуют)
эти прогнозы.

Самыми простыми и~самыми распространенными способами
согласования являются нисходящий (top-down) и~восходящий
(bottom-up) подходы~[8--15]. Нисходящий подход
предполагает получение прогноза на верхнем уровне иерархии
(агрегированный временной ряд), а~затем деагрегирование этого
прогноза на сле\-ду\-ющий (более низкий) уровень иерархии на основании
долей, наблюдаемых в~истории. Восходящий подход использует
прогнозы временн$\acute{\mbox{ы}}$х рядов нижнего уровня иерархии
(неагрегированных), из которых получает прогнозы рядов из верхних
уровней путем агрегирования. Также встречаются подходы,
комбинирующие нисходящий и~восходящий.

Нет единой точки зрения на то, какой из этих подходов позволяет
получать более точные прогнозы. Наиболее ранние исследования
проведены в~работе~\cite{grunfeld1960aggregation}, где авторы
считают, что неагрегированные данные содержат много ошибок и~поэтому нисходящее прогнозирование дает более точные прогнозы.
К~таким же выводам приходят авторы работ~\cite{fogarty1991production, narasimhan1995production}. В~\cite{fliedner1999investigation} также утверждается, что
агрегированные прогнозы более точны. С~другой стороны, в~\cite{orcutt1968data, edwards1969should} обсуждается, что основные
потери информации происходят при агрегировании и~поэтому
восходящий подход предпочтительнее. В~\cite{shlifer1979aggregation} сравниваются оба подхода к~согласованию прогнозов и~утверждается, что восходящий
предпочтительнее при выполнении некоторых условий на структуру
иерархии и~горизонт прогноза. В~\cite{schwarzkopf1988top}
исследуется смещение и~устойчивость прогнозов, получаемых с~помощью обоих подходов, и~заключается, что восходящий  надежнее,
за исключением случаев с~пропусками значений и~выбросами на нижних
уровнях иерархии.

Авторы статьи~\cite{hyndman2011optimal} обобщают нисходящий и~восходящий подходы к~согласованию иерархических прогнозов
и предлагают оптимальное согласование с~использованием регрессии,
позволяющей согласовывать одновременно по всем уровням иерархии
любой сложности. Однако предложенный ими алгоритм согласования
предполагает, что полученные независимые прогнозы являются
несмещенными оценками и~что ошибки прогнозов временн$\acute{\mbox{ы}}$х рядов
удовлетворяют структуре иерархии, т.\,е.\ являются согласованными.
Оба эти требования являются достаточно строгими и~сильно
ограничивают применимость этого способа согласования.

Предложенный в~\cite{stenina2014reconciliation} способ
согласования прогнозов не требует несмещенности независимых
прогнозов и~согласованности ошибок. При этом, как
продемонстрировано в~статье, он не уступает методу из~\cite{hyndman2011optimal} по качеству согласованных прогнозов.
Однако и~этот метод имеет ряд недостатков. Для его использования
необходимо оценивать погрешность независимых прогнозов, что не
всегда удается корректно сделать. Метод не гарантирует, что
качество согласованных прогнозов не будет уступать качеству
независимых. И согласование происходит поэтапно по узлам иерархии,
что не позволяет учесть сразу всю информацию о взаимосвязи между
временн$\acute{\mbox{ы}}$ми рядами.

В настоящей статье предлагается обобщение алгоритма согласования
из~\cite{stenina2014reconciliation} с~сохранением его преимуществ
и устранением недостатков. Предлагается алгоритм согласования
прогнозов GTOp,
основанный на идеях из~\cite{vanerven:hal-00920559}. Алгоритм GTOp
не требует оценки погрешности независимых прогнозов временн$\acute{\mbox{ы}}$х
рядов, не требует не\-сме\-щен\-ности независимых прогнозов и~имеет
теоретическое обоснование улучшения качества прогнозов после
проведения согласования. Задача согласования прогнозов
рассматривается как поиск равновесия
Нэша~\cite{petrosyan1998theory, menshikov2010lections}
в~антагонистической игре игрока, выбирающего согласованные прогнозы,
с природой, которая выбирает действительные значения временн$\acute{\mbox{ы}}$х
рядов, и~сводится к~решению оптимизационной задачи с~ограничениями
типа равенства и~неравенства. Вид равновесия Нэша задает параметры
оптимизационной задачи.

Работа предлагаемого алгоритма демонстрируется на данных о
посуточном отправлении вагонов с~37~типами грузов с~98~веток
Российских желез-\linebreak ных дорог (РЖД).
Проведены эксперименты по согласова\-нию прогнозов для различных
типов иерархических структур и~показано, что на практике
действительно наблюдается улучшение качества прогнозов после
проведения согласования алгоритмом GTOp.

Статья включает следующие разделы. В разд.~2 вводятся необходимые
обозначения, в~разд.~3 формулируется задача согласования прогнозов
иерархических временн$\acute{\mbox{ы}}$х рядов. Раздел~4 содержит  необходимые
определения и~факты из теории игр, описание антагонистической
игры, со\-от\-вет\-ст\-ву\-ющей задаче согласования прогнозов, и~доказательство существования в~этой игре равновесия Нэша. Раздел~5
 описывает оптимизационную задачу, к~которой сводится задача
согласования прогнозов, а~также преимущества и~недостатки
алгоритма. В~разд.~6 приводятся примеры функций потерь, для
которых применим алгоритм GTOp. В~разд.~7 приведены результаты
экспериментов. В~разд.~8 подводятся итоги и~делаются выводы.



\section{Обозначения}

В этом разделе вводится система обозначений, которая будет
использоваться в~настоящей работе. Будем обозначать временной ряд
через вектор $\x$, элементы временн$\acute{\mbox{о}}$го ряда будем снабжать
индексом $t$, $t \hm= 1, \ldots, T$,
 где $T$~--- длина истории.
$$
    \x = \{ x_t \}_{t = 1}^T.
$$

Общее количество временн$\acute{\mbox{ы}}$х рядов во всей иерархии будем
обозначать~$d$. Далее для наглядности будем рассматривать иерархию
рядов, изображенную на рис.~1. Она
содержит один временной ряд на верхнем уровне и~$n$~рядов на
нижнем. Для этой иерархии $d \hm= 1\hm + n$.

Временн$\acute{\mbox{ы}}$е ряды нижнего уровня будут обозначаться $\x(i)$, $i \hm= 1,
\ldots, n$, где $n$~--- число временн$\acute{\mbox{ы}}$х рядов на нижнем
уровне. Временной ряд верхнего уровня обозначается как $\x(:)$. Во
избежание\linebreak\vspace*{-12pt}

\pagebreak

\begin{center}  %fig1
\vspace*{-1pt}
\mbox{%
 \epsfxsize=76.721mm
 \epsfbox{ste-1.eps}
 }


\vspace*{3pt}

%\noindent
{{\figurename~1}\ \ \small{Плоская двухуровневая иерархия}}

\end{center}

\vspace*{6pt}


\addtocounter{figure}{1}




\noindent
 путаницы условимся использовать нижние индексы для
обозначения отсчетов времени, а~индексы в~скобках использовать для
обозначения положения временн$\acute{\mbox{ы}}$х рядов в~структуре иерархии.
Элементы временн$\acute{\mbox{ы}}$х рядов, составляющих плоскую двух\-уров\-не\-вую
иерархию, обозначаются соответственно $x_t(i)$, $i \hm= 1, \ldots,
n$, $x_t(:)$. Их соотношение задается формулой
\begin{equation}
\label{eq:2LevelsConstraint}
    x_t(:) = \sum\limits_{i = 1}^n x_t(i)\,, \enskip t = 1, \ldots, T\,.
\end{equation}
Будем называть соотношение~(\ref{eq:2LevelsConstraint})
\textbf{условием согласованности}. Прогнозы этих временн$\acute{\mbox{ы}}$х рядов
будем обозначать <<шляпками>>, опуская нижние индексы, чтобы
избежать излишне громоздких обозначений. Прогнозироваться будет
всегда $(T \hm+ 1)$-е значение временн$\acute{\mbox{о}}$го ряда
$\hat{x}(i)$, $i \hm= 1, \ldots, n$, $\hat{x}(:).$
Согласованные прогнозы будут также обозначаться без нижних
индексов: $\hat{y}(i)$, $i \hm= 1, \ldots, n$, $\hat{y}(:) \hm=
\sum\nolimits_{i = 1}^n \hat{y}(i).$

Запишем все временн$\acute{\mbox{ы}}$е ряды в~матрицу, каждая строка которой
соответствует одному временн$\acute{\mbox{о}}$му ряду. Для иерархии
с~рис.~1 эта матрица будет размера $(1 + n)
\times T$ и~выглядеть следующим образом:
\begin{equation*}
%\label{eq:SeriesMatrix}
    X = \left(%
\begin{array}{cccc}
  x_1(:) & x_2(:) & \cdots & x_T(:) \\[6pt]
  x_1(1) & x_2(1) & \cdots & x_T(1) \\[6pt]
  \cdots & \cdots & \cdots & \cdots \\[6pt]
  x_1(n) & x_2(n) & \cdots & x_T(n)
\end{array}%
\right).
\end{equation*}
Будем называть срезом иерархии в~момент времени~$t$ столбец
матрицы~$X$, соответствующий моменту времени~$t$. Для удобства
записи введем векторы, соответствующие срезу иерархии в~момент
времени~$t$, прогнозам и~согласованным прогнозам. В~этих векторах
значения, соответствующие разным временн$\acute{\mbox{ы}}$м рядам,
записаны в~столбец, начиная с~верхнего уровня иерархии и~заканчивая нижним
уровнем:

\noindent
$$
    \boldsymbol{\chi}_t = \left(%
    \begin{array}{c}
        x_t(:) \\
        x_t(1) \\
        \vdots \\
        x_t(n) \\
    \end{array}%
    \right)\,; \quad
    \hchi = \left(%
    \begin{array}{c}
        \hx(:) \\
        \hx(1) \\
        \vdots \\
        \hx(n) \\
    \end{array}%
    \right)\,; \quad
    \hphi = \left(%
    \begin{array}{c}
        \hy(:) \\
        \hy(1) \\
        \vdots \\
        \hy(n) \\
    \end{array}%
    \right).
$$
Условие~(\ref{eq:2LevelsConstraint}) для векторов
$\boldsymbol{\chi}_t$ и~$\hphi$ запишем, введя мат\-ри\-цу
связей размером $1 \times (n + 1)$:
$$
    S = \left(%
        \begin{array}{cccc}
            -1 & 1 & \cdots & 1 \\
        \end{array}%
        \right).
$$
Тогда условие согласованности запишется кратко
$$
    S \boldsymbol{\chi}_t = 0\,; \quad S \hphi = 0.
$$

В случае, когда иерархия имеет более сложную структуру, чем на
рис.~1, векторы~$\boldsymbol{\chi}_t$,
$\hchi$ и~$\hphi$ имеют размерность $d$, матрица $X$ имеет ровно
$d$ строк, временн$\acute{\mbox{ы}}$е ряды в~ней записываются
от верхних уровней к~нижним. А~размерность матрицы связи~$S$ равна $c \times d$, где
$c$ --- число узлов в~графе иерархии или, другими словами,
количество огра\-ни\-че\-ний-ра\-венств, наложенных на элементы срезов
иерархии~$\boldsymbol{\chi}_t$.



\section{Задача согласования прогнозов иерархических временных рядов}

Сформулируем задачу согласования прогнозов иерархических временн$\acute{\mbox{ы}}$х
рядов. Для этого введем еще ряд необходимых обозначений.

Пусть дан набор из $d$ временн$\acute{\mbox{ы}}$х рядов, значения которых записаны
в матрицу~$X$ размера $d \times T$:
$$
    X = \left(%
\begin{array}{cccc}
  \boldsymbol{\chi}_1 & \boldsymbol{\chi}_2 & \cdots & \boldsymbol{\chi}_T \\
\end{array}%
\right),
$$
где каждый столбец $\boldsymbol{\chi}_t$ соответствует срезу в~момент времени~$t$, а~каждая строка $\x_i$~--- одному
временному ряду. Пусть структура иерархии задана матрицей связи $S$ так, что
для всех $t \hm= 1, \ldots ,T$ выполнено условие согласованности:
$$
    S \boldsymbol{\chi}_t = 0\,.
$$
Пусть даны прогнозы  значений~$\hchi$ для всех временн$\acute{\mbox{ы}}$х рядов в~момент
времени $T\hm + 1$ и~задана функция суммарных потерь при
прогнозировании иерархии
\begin{equation}
\label{eq:HierarchyLoss}
    l_h(\hchi, \boldsymbol{\chi}_{T+1})\,.
\end{equation}

Определим множество
\begin{equation}
\label{eq:SetReconciled}
    \mathcal{A} = \{ \boldsymbol{\chi} \in \mathbb{R}^d \mid S \boldsymbol{\chi} = 0 \}\,,
\end{equation}
где $\boldsymbol{\chi}$~--- произвольный $d$-мерный вектор, а~$S$~--- заданная матрица связи. Отметим, что все срезы $\boldsymbol{\chi}_t$, $t \hm=
1, \ldots, T$, лежат в~множестве~$\mathcal{A}$. Также в~нем должны лежать
согласованные прогнозы $\hphi$.

В ряде задач прогнозы должны удовлетворять некоторым ограничениям,
связанным с~физической природой прогнозируемой величины. В~связи
с чем введем множество
\begin{multline}
\label{eq:SetPhysicsConstraint}
    \mathcal{B} = \{ \boldsymbol{\chi} \in \mathbb{R}^d \mid \boldsymbol{\chi}(i)
    \in [A_i, B_i]\,,\\
     A_i, B_i \in [-\infty, +\infty]\,,\ i = 1, \ldots, d
    \},
\end{multline}
где $\boldsymbol{\chi}$~--- произвольный $d$-мер\-ный вектор; $A_i$
и~$B_i$ задают отрезок, в~котором должна находиться \mbox{$i$-я}
компонента этого вектора. Например, в~случае $A_1 \hm= \cdots = A_d \hm=
0$, $B_1 \hm= \cdots = B_d \hm= +\infty$ вектор~$\boldsymbol{\chi}$
лежит в~положительном октанте, а~конечные значения $A_1, \cdots,
A_d$, $B_1, \cdots, B_d$ могут задавать интервалы, в~которых
должны лежать прогнозы. Отсутствие ка\-ких-ли\-бо ограничений задается
значениями $A_1 \hm= \cdots = A_d \hm= -\infty$, $B_1 \hm= \cdots = B_d \hm=
+\infty$.

Введя все необходимые дополнительные обозначения, можно
сформулировать задачу поиска согласованных прогнозов. Требуется
найти вектор прогнозов~$\hphi$, удовлетворяющий следующим
требованиям.
\begin{description}
    \item[Согласованность:] вектор прогнозов~$\hphi$ должен
    удов\-ле\-тво\-рять структуре иерархии, заданной мат\-ри\-цей связи~$S$,
    т.\,е.\ $\hphi \hm\in \mathcal{A}$.
    \item[Ограничения:] вектор прогнозов~$\hphi$ должен
    удовлетворять наложенным ограничениям, т.\,е.\ $\hphi \hm\in \mathcal{B}$.
    \item[Качество:] общие потери при использовании согласованных
    прогнозов не должны превышать общие потери при использовании
    независимых прогнозов, т.\,е.\ ${l_h(\hphi, \boldsymbol{\chi}_{T + 1}) \hm\leq l_h(\hchi, \boldsymbol{\chi}_{T +    1})}$.
\end{description}


\section{Задача согласования прогнозов как поиск равновесия
в~антагонистической игре}

В этом разделе задача согласования прогнозов, сформулированная
выше, рассматривается как\linebreak антагонистическая игра. Такое
представление не влияет на решение задачи согласования прогнозов
и~направлено лишь на достижение на\-гляд\-ности и~интерпретируемости
полученных ре\-зультатов. В~первой части раздела приводятся
необходимые определения и~факты из теории игр~[17--19],
 во второй части вводится
антагонистическая игра, соответствующая задаче согласования
прогнозов, в~треть\-ей части формулируется и~доказывается теорема о
существовании в~этой игре равновесия Нэша, которая является
обобщением теоремы, доказанной в~\cite{vanerven:hal-00920559},
и~приводится следствие из этой теоремы, однозначно определяющее
выбор оптимального вектора согласованных прогнозов.

\subsection{Понятие антагонистической игры}

\noindent
\textbf{Определение 1.} %\label{def:antagonistic_game}
\textit{Система
$$
    \Gamma = (\mathcal{M}, \mathcal{N}, L)\,,
$$
где $\mathcal{M}$ и~$\mathcal{N}$~--- непустые множества и~функция $L \colon \mathcal{M}
\times \mathcal{N} \to \mathbb{R}$, называется антагонистической игрой
(игрой с~нулевой суммой) в~нормальной форме. Элементы $\mu \hm\in \mathcal{M}$
и~$\nu
\hm\in \mathcal{N}$ называются стратегиями игроков~$1$ и~$2$ соответственно,
$L$~--- функцией потерь игрока~$1$. Потери игрока~$2$ полагаются равными
$-L(\mu, \nu)$}.

\smallskip

\noindent
\textbf{Определение 2.} %\label{def:clear_strategy}
\textit{Говорят, что игра разыгрывается в~чис\-тых стратегиях, если оба
игрока из имеющихся наборов действий $\mathcal{M}$ и~$\mathcal{N}$ выбирают по одному
действию $\mu$ и~$\nu$ соответственно}.


\smallskip

\noindent
\textbf{Определение 3.}
%\label{def:mixed_strategy}
\textit{Введем на множествах стратегий~$\mathcal{M}$ и~$\mathcal{N}$ вероятностные
распределения $p(\mu)$ и~$q(\nu)$ соответственно:
$$
    \int\limits_{\mathcal{M}} p(\mu)\, d\mu = 1\,; \quad
     \int\limits_{\mathcal{N}} q(\nu)\, d\nu =
    1\,.
$$
Распределения $p(\mu)$ и~$q(\nu)$ задают смешанные стратегии в~игре~$\Gamma$,
если игрок~$1$ выбирает действие в~соответствии с~распределением
$p(\mu)$ и~игрок~$2$ выбирает действие в~соответствии с~распределением $q(\nu)$}.

\smallskip


Чистые стратегии являются частным случаем смешанных. Поэтому далее
будут рассматриваться смешанные стратегии, за исключением
специально оговоренных моментов. Обозначать стратегии будем~$\pcd$ и~$\q$.

Математическое ожидание потерь игрока 1 при паре смешанных
стратегий $\pcd$ и~$\q$ обозначим через
$$
    \bar{L}(\pcd, \q) = \int\limits_{\mathcal{M}} \int\limits_{\mathcal{N}} p(\mu) q(\nu) L(\mu,
    \nu)\, d\mu \, d\nu\,.
$$
Игрок 1 преследует цель минимизировать эту величину при любых
действиях игрока~2.

\noindent
\textbf{Определение 4.} %\label{def:Nash_equilibrium}
Пара стратегий $(p(\cdot), q(\cdot))$ называется равновесием Нэша
в~смешанных стратегиях в~игре $\Gamma$, если для любых
$\pcd^{\prime}$ и~$\q^{\prime}$ выполнено неравенство:
$$
    \bar{L}(\pcd, \q^{\prime}) \leq \bar{L}(\pcd, \q) \leq \bar{L}(\pcd^{\prime},
    \q)\,.
$$


При равновесии Нэша ни одному из игроков не выгодно отклоняться от
равновесной стратегии, если второй продолжает придерживаться
равновесной стратегии. При этом игрок~1 минимизирует свои потери в~ситуации, когда игрок~2 действует наиболее выгодным для себя
образом. Отметим также, что равновесие Нэша является седловой
точкой функции $\bar{L}(\pcd, \q)$.

\smallskip

\noindent
\textbf{Теорема~1.} %\label{th:equlibrium_existence}
\textit{В антагонистической игре равновесие Нэша существует тогда и~только
тогда, когда определена величина}
$$
    V = \min\limits_{\pcd^{\prime}} \max\limits_{\q^{\prime}} \bar{L}(\pcd^{\prime}\,,
    \q^{\prime}) =
    \max\limits_{\q^{\prime}} \min\limits_{\pcd^{\prime}} \bar{L}(\pcd^{\prime}\,,
    \q^{\prime}).
$$


Величина $V$ называется ценой игры. Доказательство этой теоремы
в~настоящей статье не приводится, при желании его можно
найти в~[17--19].

\subsection{Антагонистическая игра, описывающая задачу согласования~прогнозов}

Вернемся к~рассмотрению введенных в~разд.~3
множеств~$\mathcal{A}$~(\ref{eq:SetReconciled})
и~$\mathcal{B}$~(\ref{eq:SetPhysicsConstraint}). Напомним, что множество
$\mathcal{A}$ содержит все $d$-мер\-ные векторы, удовлетворяющие структуре
иерархии, заданной матрицей связи~$S$. Множество~$\mathcal{B}$ содержит
$d$-мер\-ные векторы, удовлетворяющие огра\-ни\-че\-ни\-ям-не\-ра\-вен\-ст\-вам
рассматриваемой задачи прогнозирования.

Будем рассматривать антагонистическую иг\-ру~$\Gamma$, в~которой
игрок~1 выбирает вектор согласованных прогнозов~$\hphi$ из
множества $\mathcal{A} \cap \mathcal{B}$, которые одновременно удовлетворяют
структуре иерархии, заданной матрицей связи~$S$,
и~огра\-ни\-че\-ни\-ям-не\-ра\-вен\-ст\-вам,
задающим множество~$\mathcal{B}$, игрок~2 также
выбирает вектор действительных значений $\boldsymbol{\chi}_{T+1}$,
удовлетворяющих структуре иерархии и~физическим ограничениям
(можно считать игрока~2 природой).

Определим множества стратегий $\mathcal{M}$ и~$\mathcal{N}$ игроков~1 и~2 как
пересечение множеств $\mathcal{A} \cap \mathcal{B}$. Функцию потерь игрока~1
определим с~помощью функции потерь по иерархии~(\ref{eq:HierarchyLoss}):
$$
    L(\hphi, \boldsymbol{\chi}_{T+1}) = l_h(\boldsymbol{\chi}_{T+1}, \hphi) - l_h(\boldsymbol{\chi}_{T+1}, \hchi)\,,
$$
где вектор независимых прогнозов $\hchi$ считается заданным и~не
зависит от действий, выбираемых игроками. Считается, что $\hchi
\hm\in \mathcal{B}$. Такой выбор функции потерь игрока~1 связан с~тем
соображением, что при $L(\hphi, \boldsymbol{\chi}_{T+1}) \hm= 0$
качество согласованных прогнозов не хуже, чем качество независимых
прогнозов, а~при $L(\hphi, \boldsymbol{\chi}_{T+1}) \hm< 0$ и~вовсе
превосходит его.

Таким образом, получена антагонистическая игра
\begin{equation}
\label{eq:game}
    \Gamma = (\mathcal{A} \cap \mathcal{B},~\mathcal{A} \cap \mathcal{B},~l_h(\boldsymbol{\chi}_{T+1}, \hphi) - l_h(\boldsymbol{\chi}_{T+1}, \hchi))\,,
\end{equation}
где игрок~1 выбирает вектор согласованных прогнозов $\hphi$ из
множества $\mathcal{A} \cap \mathcal{B}$, а~игрок~2 выбирает вектор действительных
значений элементов временн$\acute{\mbox{ы}}$х рядов $\boldsymbol{\chi}_{T+1}$ из
множества~$\mathcal{A} \cap \mathcal{B}$. При этом первый игрок преследует цель
минимизировать свои потери, выраженные функцией $L(\hphi,
\boldsymbol{\chi}_{T+1})$, при любых действиях игрока~2, т.\,е.\
при любом векторе действительных значений
$\boldsymbol{\chi}_{T+1}$. Эта цель достигается, если игрок~1
воспользуется стратегией, входящей в~равновесие Нэша.

\subsection{Существование равновесия Нэша}

В этой части раздела будет показано, что при выполнении ряда
естественных требований к~множествам $\mathcal{A}$~(\ref{eq:SetReconciled})
и~$\mathcal{B}$~(\ref{eq:SetPhysicsConstraint}) и~функции суммарных потерь
при прогнозировании иерархии $l_h$~(\ref{eq:HierarchyLoss})
в~антагонистической игре (\ref{eq:game}), описывающей задачу
согласования прогнозов, существует равновесие Нэша в~чис\-тых
стратегиях. Также будет показано, что соответствующее этому
равновесию значение функции потерь игрока~1 неположительно, что
гарантирует неухудшение качества прогнозов при переходе от
независимых прогнозов к~согласованным. Рас\-смот\-рим эти требования.

\smallskip

\noindent
\textbf{Определение 5.}
%\label{def:ConvexSet}
Множество $\mathcal{C} \subseteq \mathbb{R}^d$ называется
выпуклым~\cite{boyd2009convex}, если для любых $\boldsymbol{\chi}_1 \hm\in
\mathcal{C}$ и~$\boldsymbol{\chi}_2 \hm\in \mathcal{C}$ и~любого $0
\hm\leq \alpha \hm\leq 1$ выполнено
$$
    \alpha \boldsymbol{\chi}_1 + (1 - \alpha) \boldsymbol{\chi}_2 \in \mathcal{C}\,.
$$


\smallskip

Заметим, что множества $\mathcal{A}$ (\ref{eq:SetReconciled}) и~$\mathcal{B}$~(\ref{eq:SetPhysicsConstraint}) выпуклы и~замкнуты.

\smallskip

\noindent
\textbf{Предположение 1.}
%\label{ass:Convex}
\textit{Будем предполагать, что пересечение множеств
$\mathcal{A}$ и~$\mathcal{B}$ не пусто:
$\mathcal{A} \cap \mathcal{B} \neq \varnothing$}.

\smallskip

Требование непустого пересечения этих множеств естественно, так
как в~противном случае неразрешима задача поиска вектора
согласованных прогнозов~$\hphi$, который должен одновременно
принадлежать обоим множествам. Также отметим, что множество $\mathcal{A}
\cap \mathcal{B}$ является выпуклым и~замкнутым как пересечение двух
выпуклых и~замкнутых множеств~\cite{boyd2009convex}.

\smallskip

\noindent
\textbf{Предположение 2.} %\label{ass:NonNegative}
\textit{Будем считать, что функция суммарных потерь~$l_h$}~(\ref{eq:HierarchyLoss})
\textit{неотрицательна и~равна нулю только при
равенстве аргументов}:
$$
    l_h(\boldsymbol{\chi}_{T+1}, \hchi) \geq 0 \mbox{ для\ всех } \boldsymbol{\chi}_{T+1},
    \hchi\,;
$$
$$
    l_h(\boldsymbol{\chi}_{T+1}, \hchi) = 0 \quad \Leftrightarrow \quad \boldsymbol{\chi}_{T+1} =
    \hchi\,.
$$

\smallskip

Равенство аргументов $\boldsymbol{\chi}_{T+1} \hm= \hchi$
соответствует случаю, когда прогноз полностью совпадает с~действительными значениями. В~этом случае потери равны нулю. Во
всех остальных случаях потери при прогнозе положительные.



\noindent
\textbf{Определение 6.} %\label{def:Projection}
Проекцией точки $\boldsymbol{\chi}_0 \in \mathbb{R}^d$ на множество
$\mathcal{C} \subseteq \mathbb{R}^d$, инициированной функцией расстояния~$f$,
называется точка
$$
    \boldsymbol{\chi}_{\mathrm{proj}} = \argmin\limits_{\boldsymbol{\chi} \in \mathcal{C}} f(\boldsymbol{\chi},
    \boldsymbol{\chi}_0)\,.
$$


%\smallskip

\noindent
\textbf{Предположение 3.}
%\label{ass:Projection}
\textit{Пусть существует проекция точки из $\mathbb{R}^d$, соответствующей вектору
независимых прогнозов $\hchi$, на выпуклое и~замкнутое мно\-жество
$\mathcal{A} \cap \mathcal{B}$, инициированная функцией суммарных потерь~$l_h$}:
$$
    \boldsymbol{\chi}_{\mathrm{proj}} = \argmin\limits_{\boldsymbol{\chi} \in \mathcal{A} \cap \mathcal{B}} l_h(\boldsymbol{\chi},
    \hchi).
$$


\smallskip

\noindent
\textbf{Предположение 4.} %\label{ass:CosinusInequality}
\textit{Пусть $\boldsymbol{\chi}_{\mathrm{proj}} \hm=
\argmin\limits_{\boldsymbol{\chi} \in \mathcal{A} \cap \mathcal{B}}
l_h(\boldsymbol{\chi}, \hchi)$. Будем предполагать, что для всех
$\boldsymbol{\chi} \hm\in \mathcal{B}$ и~для всех $\boldsymbol{\psi} \hm\in \mathcal{A}
\cap \mathcal{B}$ выполняется неравенство}
$$
    l_h(\boldsymbol{\psi}, \boldsymbol{\chi}) \geq
    l_h(\boldsymbol{\psi}, \boldsymbol{\chi}_{\mathrm{proj}}) +
    l_h(\boldsymbol{\chi}_{\mathrm{proj}},
    \boldsymbol{\chi})\,.
$$

\smallskip

Для пояснения этого требования рассмотрим частный случай, когда
$\mathbb{R}^d \hm= \mathbb{R}^3$, $\mathcal{A} \subset \mathbb{R}^2$, $\mathcal{B} \subset \mathbb{R}^3$, $\mathcal{A} \cap \mathcal{B}
\subset \mathbb{R}^2$ и~$l_h$~--- квадрат метрики Евклида
(рис.~2). Точки
$\boldsymbol{\psi}$, $\boldsymbol{\chi}$,
$\boldsymbol{\chi}_{\mathrm{proj}}$ образуют треугольник. Обозначим
$\theta$ угол при вершине $\boldsymbol{\chi}_{\mathrm{proj}}$ и~запишем
теорему косинусов:
\begin{multline*}
    l_h(\boldsymbol{\psi}, \boldsymbol{\chi}) =
    l_h(\boldsymbol{\psi}, \boldsymbol{\chi}_{\mathrm{proj}}) +
    l_h(\boldsymbol{\chi}_{\mathrm{proj}},
    \boldsymbol{\chi}) - {}\\
    {}-2 \sqrt{l_h(\boldsymbol{\psi}, \boldsymbol{\chi}_{\mathrm{proj}})}
    \sqrt{l_h(\boldsymbol{\chi}_{\mathrm{proj}},
    \boldsymbol{\chi})} \cos \theta\,.
\end{multline*}


Поскольку $\boldsymbol{\chi}_{\mathrm{proj}}$ является проекцией, то
угол~$\theta$ не может быть острым. Он прямой, если проецируемая точка
$\boldsymbol{\chi}$ находится <<над>> множеством $\mathcal{A} \cap \mathcal{B}$, и~тупой, если точка находится <<в~стороне>>. Таким образом получаем,
что $\cos \theta \hm\leq 0$, а~значит, последнее слагаемое в~теореме
косинусов неотрицательное. Исключая его и~заменяя знак равенства
на знак нестрогого неравенства, получаем, что предположение~4
соответствует естественным свойствам проекции.

\begin{center}  %fig2
\vspace*{-1pt}
\mbox{%
 \epsfxsize=69.331mm
 \epsfbox{ste-2.eps}
 }


\vspace*{3pt}

%\noindent
{{\figurename~2}\ \ \small{Пояснение к~предположению~4}}

\end{center}


%\vspace*{6pt}


\addtocounter{figure}{1}


%\smallskip

Введя предположения~1--4, сформулируем
теорему.

\smallskip

\noindent
\textbf{Теорема 2.}
%\label{th:saddle_point}
\textit{Пусть выполнены предположения~$1$--$4$. Тогда пара стратегий
$(\boldsymbol{\chi}_{\mathrm{proj}}, \boldsymbol{\chi}_{\mathrm{proj}})$ является
равновесием Нэша в~игре~(\ref{eq:game}) и~седловой точкой функции
$L(\hphi, \boldsymbol{\chi}_{T+1})\hm = l_h(\boldsymbol{\chi}_{T+1},
\hphi) \hm- l_h(\boldsymbol{\chi}_{T+1}, \hchi)$. Цена игры при этом
равна} $V \hm= -l_h(\boldsymbol{\chi}_{\mathrm{proj}}, \hchi)$.


\smallskip

\noindent
Д\,о\,к\,а\,з\,а\,т\,е\,л\,ь\,с\,т\,в\,о\,.\ \
Найдем седловую точку функции $L(\hphi, \boldsymbol{\chi}_{T+1})$
в соответствии с~определением~4  и~теоремой~1. Найдем максимум этой
функции по второму аргументу при $\hphi \hm=
\boldsymbol{\chi}_{\mathrm{proj}}$. Для этого воспользуемся предположением~4:
$$
    l_h(\boldsymbol{\psi}, \boldsymbol{\chi}) \geq l_h(\boldsymbol{\psi},
    \boldsymbol{\chi}_{\mathrm{proj}}) + l_h(\boldsymbol{\chi}_{\mathrm{proj}},
    \boldsymbol{\chi})\,.
$$
Применяя неравенство к~функции потерь~$L$ игрока~1 (подставляем
$\boldsymbol{\psi} \hm= \boldsymbol{\chi}_{T+1}$, $\boldsymbol{\chi}
\hm= \hchi$), получаем:
\begin{multline*}
    L(\boldsymbol{\chi}_{\mathrm{proj}}, \boldsymbol{\chi}_{T+1})
     = l_h(\boldsymbol{\chi}_{T+1}, \boldsymbol{\chi}_{\mathrm{proj}}) -
      l_h(\boldsymbol{\chi}_{T+1}, \hchi) \leq{}\\
      {}\leq
    l_h(\boldsymbol{\chi}_{T+1}, \hchi) -
    l_h(\boldsymbol{\chi}_{\mathrm{proj}}, \hchi) -l_h(\boldsymbol{\chi}_{T+1},
    \hchi) ={}
\\{}
    = -l_h(\boldsymbol{\chi}_{\mathrm{proj}}, \hchi) \end{multline*}
     для всех $\boldsymbol{\chi}_{T+1} \in \mathcal{A} \cap \mathcal{B}$.
Заметим также, что из предположения~2 вытекает
\begin{multline*}
    L(\boldsymbol{\chi}_{\mathrm{proj}}, \boldsymbol{\chi}_{\mathrm{proj}}) =
    l_h(\boldsymbol{\chi}_{\mathrm{proj}}, \boldsymbol{\chi}_{\mathrm{proj}}) -
    l_h(\boldsymbol{\chi}_{\mathrm{proj}}, \hchi) ={}\\
    {}= - l_h(\boldsymbol{\chi}_{\mathrm{proj}}, \hchi)\,.
\end{multline*}
Приходим к~выводу, что
$$
    L(\boldsymbol{\chi}_{\mathrm{proj}}, \boldsymbol{\chi}_{T+1}) \leq
    L(\boldsymbol{\chi}_{\mathrm{proj}}, \boldsymbol{\chi}_{\mathrm{proj}})
    $$
 для всех $\boldsymbol{\chi}_{T+1} \in \mathcal{A} \cap \mathcal{B}$.


Следовательно, максимум по второму аргументу достигается при
$\boldsymbol{\chi}_{T+1} = \boldsymbol{\chi}_{\mathrm{proj}}$.

Минимум по первому аргументу при $\boldsymbol{\chi}_{T+1}\hm =
\boldsymbol{\chi}_{\mathrm{proj}}$ находим, используя
предположение~2, из соотношения
\begin{multline*}
    \argmin\limits_{\hphi \in \mathcal{A} \cap \mathcal{B}} L(\hphi,
    \boldsymbol{\chi}_{\mathrm{proj}}) ={}\\
    {}=
    \argmin\limits_{\hphi \in \mathcal{A} \cap \mathcal{B}}
    l_h(\boldsymbol{\chi}_{\mathrm{proj}}, \hphi)
    - l_h(\boldsymbol{\chi}_{\mathrm{proj}}, \hchi).
\end{multline*}
Второе слагаемое не зависит от~$\hphi$, а~по предположению~2
функция суммарных потерь неотрицательна и~обращается в~ноль только при равенстве аргументов, поэтому
получаем
$$
    \argmin\limits_{\hphi \in \mathcal{A} \cap \mathcal{B}}
    L(\hphi, \boldsymbol{\chi}_{\mathrm{proj}}) =
    \boldsymbol{\chi}_{\mathrm{proj}}\,.
$$

Таким образом, получаем, что
\begin{multline*}
   L(\boldsymbol{\chi}_{\mathrm{proj}}, \boldsymbol{\chi}_{T+1})  \leq
   L(\boldsymbol{\chi}_{\mathrm{proj}}, \boldsymbol{\chi}_{\mathrm{proj}})
   \leq L(\hphi, \boldsymbol{\chi}_{\mathrm{proj}})\,, \\
    \boldsymbol{\chi}_{T+1},~\hphi \in \mathcal{A} \cap \mathcal{B}\,.
\end{multline*}
Следовательно, $(\boldsymbol{\chi}_{\mathrm{proj}},
\boldsymbol{\chi}_{\mathrm{proj}})$~--- седловая точка функции $L$.
И~эта пара является равновесием Нэша в~игре~(\ref{eq:game}), и~цена
игры выражается как

\vspace*{-4pt}

\noindent
\begin{multline*}
    V = \min\limits_{\hphi \in \mathcal{A} \cap \mathcal{B}} \max\limits_{\boldsymbol{\chi}_{T+1} \in \mathcal{A}} L(\hphi, \boldsymbol{\chi}_{T+1}) ={}\\
    {}=
    \max\limits_{\boldsymbol{\chi}_{T+1} \in \mathcal{A}} \min\limits_{\hphi \in \mathcal{A} \cap
    \mathcal{B}} L(\hphi, \boldsymbol{\chi}_{T+1}) ={}\\
    {}=
    L(\boldsymbol{\chi}_{\mathrm{proj}},
    \boldsymbol{\chi}_{\mathrm{proj}})= -l_h(\boldsymbol{\chi}_{\mathrm{proj}}, \hchi)\,.
\end{multline*}

\smallskip


\noindent
\textbf{Следствие 1.} %\label{col:OptimalReconcilation}
\textit{Использование в~качестве вектора согласованных прогнозов~$\hphi$
проекции вектора независимых прогнозов $\hchi\hm \in \mathbb{R}^d$ на
множество $\mathcal{A} \cap \mathcal{B}$, инициированной функцией
суммарных потерь~$l_h$, гарантирует значение функции суммарных потерь не большее,
чем при использовании вектора независимых прогнозов}~$\hchi$.


\noindent
Д\,о\,к\,а\,з\,а\,т\,е\,л\,ь\,с\,т\,в\,о\,.\ \
По теореме~2 цена игры~(\ref{eq:game}) равна
значению функции потерь игрока~1 в~точке
$(\boldsymbol{\chi}_{\mathrm{proj}}, \boldsymbol{\chi}_{\mathrm{proj}})$
и~неположительна в~силу предположения~2

\vspace*{2pt}

\noindent
$$
    V = L(\boldsymbol{\chi}_{\mathrm{proj}}, \boldsymbol{\chi}_{\mathrm{proj}})
    = -l_h(\boldsymbol{\chi}_{\mathrm{proj}}, \hchi)
    \leq 0\,.
$$
А выбор рассматриваемой функции потерь игрока~1 $L(\hphi,
\boldsymbol{\chi}_{T+1})\hm = l_h(\boldsymbol{\chi}_{T+1}, \hphi) \hm-
l_h(\boldsymbol{\chi}_{T+1}, \hchi)$ был обусловлен тем, что ее
знак совпадает со знаком изменения суммарных потерь при переходе
от вектора независимых прогнозов~$\hchi$ к~вектору согласованных
прогнозов~$\hphi$. Следовательно, при $\boldsymbol{\chi}_{T+1} \hm=
\boldsymbol{\chi}_{\mathrm{proj}}$ суммарные потери при согласованных
прогнозах меньше, чем при независимых.

Согласно определению~4 равновесия Нэша

\vspace*{2pt}

\noindent
$$
    L(\boldsymbol{\chi}_{\mathrm{proj}}, \boldsymbol{\chi}_{T+1}) \leq
    L(\boldsymbol{\chi}_{\mathrm{proj}}, \boldsymbol{\chi}_{\mathrm{proj}}) \leq 0
    $$
 для любых  $\boldsymbol{\chi}_{T+1} \in \mathcal{A} \cap \mathcal{B}$.
Поэтому при любом векторе действительных значений
$\boldsymbol{\chi}_{T+1} \hm\in \mathcal{A} \cap \mathcal{B}$ согласованные
прогнозы~$\hphi$ оказываются предпочтительнее независимых
прогнозов~$\hchi$.

\section{Алгоритм согласования прогнозов GTOp}

Согласно следствию~1 оптимальным
выбором вектора согласованных прогнозов~$\hphi$ является проекция
вектора независимых прогнозов~$\hchi$ на множество $\mathcal{A} \cap \mathcal{B}$,
инициированная функцией суммарных потерь~$l_h$. Множество $\mathcal{A} \cap
\mathcal{B}$ содержит векторы размерности~$d$, удовлетворяющие структуре
иерархии, так как множество $\mathcal{A}$ задается
огра\-ни\-че\-ни\-ями-ра\-вен\-ст\-ва\-ми, порожденными матрицей
связи иерархии~$S$. В~то же время $\mathcal{A} \cap \mathcal{B}$ содержит
\mbox{$d$-мер}\-ные векторы,
удовлетворяющие огра\-ни\-че\-ни\-ям-не\-ра\-вен\-ст\-вам из множества~$\mathcal{B}$. Таким
образом, задача поиска проекции~--- это оптимизационная задача с~ограничениями типа равенства и~неравенства:

\noindent
\begin{equation}
\label{eq:OptimProblem}
    \left\{%
\begin{array}{l}
    l_h(\boldsymbol{\chi}, \hchi) \rightarrow \min\limits_{\boldsymbol{\chi}}\,, \\
    \boldsymbol{\chi} \in \mathcal{A}\ \mbox{(ограничения-равенства)}; \\
    \boldsymbol{\chi} \in \mathcal{B}\ \mbox{(ограничения-неравенства)}. \\
\end{array}%
\right.
\end{equation}

Алгоритм согласования прогнозов иерархических временн$\acute{\mbox{ы}}$х рядов GTOp
заключается в~решении оптимизационной задачи~(\ref{eq:OptimProblem}). Достоинства этого алгоритма заключаются в~том, что он требует от вектора независимых прогнозов~$\hchi$ лишь
принадлежности множеству~$\mathcal{B}$, и~не требует не\-сме\-щен\-ности
независимых прогнозов, а~следовательно, для получения независимых
прогнозов можно использовать любой алгоритм прогнозирования. Также
GTOp не требует оценки погрешностей независимых прогнозов. Самое
важное, что GTOp обеспечивает неухудшение качества прогнозирования
при замене независимых прогнозов на согласованные прогнозы. При
этом на структуру иерархии, огра\-ни\-че\-ния-не\-ра\-вен\-ст\-ва на прогнозы и~функцию суммарных потерь накладываются лишь общие ограничения,
гарантирующие существование решения оптимизационной задачи~(\ref{eq:OptimProblem}). Еще одно достоинство алгоритма GTOp в~том, что он позволяет согласовывать прогнозы для иерархий любой
сложности одновременно по всем уровням, учитывая все связи в~иерархии и~решая одну оптимизационную задачу. От сложности
иерархии и~количества временн$\acute{\mbox{ы}}$х рядов и~уровней в~оптимизационной
задаче зависит число переменных и~ограничений.

\section{Дивергенция Брегмана}

В этом разделе будет описано семейство функций двух переменных,
удовлетворяющих предположениям~1--4, и~приведен ряд примеров функций из
этого семейства, которые можно использовать в~задаче согласования
прогнозов в~качестве функции суммарных потерь~$l_h$~(\ref{eq:HierarchyLoss}). Все эти функции двух переменных
называются дивергенциями Брегмана~\cite{cesa2006prediction, bregman1967relaxation}
и~порождаются функциями одной переменной, обладающими следующими свойствами.

\smallskip

\noindent
\textbf{Определение 7.}
%\label{def:Legendre_function}
Функцией Лежандра~\cite{cesa2006prediction} называется функция $F
\colon \mathcal{B} \to \mathbb{R}$, которая удовлетворяет следующим условиям:
\begin{itemize}
    \item $\mathcal{B} \subseteq \mathbb{R}^d$~--- непустое множество, и~внутренность
    $\mathcal{B}$ выпукла;
    \item $F$~--- строго выпуклая функция с~непрерывной первой
    производной на множестве~$\mathcal{B}$;
    \item если $\boldsymbol{\chi}_1, \boldsymbol{\chi}_2, \ldots \in \mathcal{B}$~---
    последовательность, сходящаяся к~граничной точке $\mathcal{B}$, то $\| \nabla F(\boldsymbol{\chi}_n) \| \hm\rightarrow
    \infty$ при $n \hm\rightarrow \infty$.
\end{itemize}


%\smallskip

\noindent
\textbf{Определение 8.} %\label{def:Bregman_divergence}
Дивергенцией Брегмана, порожденной функцией Лежандра $F \colon \mathcal{B}
\hm\to \mathbb{R}$, называется неотрицательная функция
$D_F \colon \mathcal{B} \times
\mathrm{int}\,(\mathcal{B}) \to \mathbb{R}$, определенная как
$$
    D_F(\mathbf{u}, \mathbf{v}) = F(\mathbf{u}) - F(\mathbf{v}) -
    (\mathbf{u} - \mathbf{v}) \nabla F(\mathbf{v})\,.
$$


\paragraph*{Свойства дивергенции Брегмана:}
\begin{itemize}
    \item для всех $\mathbf{u}$ и~$\mathbf{v}$ выполнено $D_F(\mathbf{u}, \mathbf{v}) \geq
    0$. Это следует из выпуклости функции $F$;
    \item $D_F(\mathbf{u}, \mathbf{v})$ выпукла по первому
    аргументу $\mathbf{u}$, но необязательно выпукла по второму
    аргументу~$\mathbf{v}$;
    \item для любых $\alpha, \beta \in \mathbb{R}$ и~любых функций
    Лежандра $F_1$ и~$F_2$ выполнено
    $$
    D_{\alpha F_1 + \beta F_2}(\mathbf{u}, \mathbf{v}) =
    \alpha D_{F_1}(\mathbf{u}, \mathbf{v}) + \beta D_{F_2}(\mathbf{u},
    \mathbf{v})\,.
    $$
\end{itemize}

%\smallskip

\noindent
\textbf{Определение 9.} %\label{def:Bregman_projection}
Пусть $F \colon \mathcal{B} \hm\to \mathbb{R}$~--- функция Лежандра и~$\mathcal{A} \subset
\mathbb{R}^d$~--- замкнутое выпуклое множество, такое что $\mathcal{A} \cap \mathcal{B} \neq
\varnothing$. Проекция Брегмана $\mathbf{w}^{\prime}$ точки
$\mathbf{w} \hm\in \mathrm{int}\,(\mathcal{B})$ на множество~$\mathcal{A}$~--- это
$$
    \mathbf{w}^{\prime} = \argmin\limits_{\mathbf{u} \in \mathcal{A} \cap
    \mathcal{B}} D_F(\mathbf{u}, \mathbf{w})\,.
$$


\smallskip

%\smallskip

\noindent
\textbf{Теорема 3.}
%\label{lemma:Bregman_projection}
\textit{Для всех функций Лежандра $F \colon \mathcal{B} \to \mathbb{R}$, для всех замкнутых
выпуклых множеств $\mathcal{A} \subset \mathbb{R}^d$, имеющих непустое пересечение
$\mathcal{A} \cap \mathcal{B} \neq \varnothing$, и~для всех точек $\mathbf{w} \in
\mathrm{int}\,(\mathcal{B})$ проекция Брегмана точки $\mathbf{w}$ на множество
$\mathcal{A}$ существует и~единственна}.


\smallskip

\noindent
Д\,о\,к\,а\,з\,а\,т\,е\,л\,ь\,с\,т\,в\,о\ \ этой теоремы приведено в~\cite{bregman1967relaxation}.

\smallskip

\noindent
\textbf{Теорема~4.}
%\label{lemma:generalized_pythagorean_inequality}
\textit{Пусть $F$~--- функция Лежандра. Для всех $\mathbf{w} \hm\in
\mathrm{int}\,(\mathcal{B})$ и~для всех замкнутых выпуклых множеств $\mathcal{A}
\subseteq \mathbb{R}^d$ с~непустым пересечением $\mathcal{A} \cap \mathcal{B} \neq
\varnothing$, если $\mathbf{w}^{\prime}$~--- проекция Брегмана
точки $\mathbf{w}$ на множество $\mathcal{A}$ $(\mathbf{w}^{\prime} \hm=
\argmin\limits_{\mathbf{v} \in \mathcal{A} \cap \mathcal{B}} D_F(\mathbf{v},
\mathbf{w}))$, то верно неравенство}:
$$
    D_F(\mathbf{u}, \mathbf{w}) \geq D_F(\mathbf{u},
    \mathbf{w}^{\prime}) + D_F(\mathbf{w}^{\prime}, \mathbf{w})
$$
для всех $\mathbf{u} \in \mathcal{A}$.

\noindent
Д\,о\,к\,а\,з\,а\,т\,е\,л\,ь\,с\,т\,в\,о\ \ этого факта можно найти в~\cite{cesa2006prediction}.

\smallskip

Соотнесем перечисленные свойства дивергенции Брегмана и~предположения~1--4. Определение~8 и~свойство~1 дивергенции Брегмана
обеспечивают выполнение предположения~2 о
знаке функции суммарных потерь. В~определении~9
и~тео\-ре\-ме~3 предполагается выпуклость и~замкнутость
множеств $\mathcal{A}$ и~$\mathcal{B}$ и~их непустое пересечение,
как и~в~предположении~1. Теорема~4 гарантирует
выполнение предположения~3 о~выпуклости
множеств и~существовании и~единственности проекции. Наконец,
теорема~4 гарантирует
выполнение предположения~4.
Следовательно, для функций суммарных потерь $l_h$~(\ref{eq:HierarchyLoss}),
являющихся дивергенциями Брегмана,
выполнены все условия теоремы~2. Использование
в качестве вектора согласованных прогнозов $\hphi$ проекции
Брегмана вектора независимых прогнозов $\hchi \hm\in \mathcal{B}$, где
множество~$\mathcal{B}$ определено по формуле~(\ref{eq:SetPhysicsConstraint}) и~является выпуклым и~замкнутым,
имеющим непустое пересечение
с~множеством $\mathcal{A}$~(\ref{eq:SetReconciled}),
на множество~$\mathcal{A}\cap \mathcal{B}$, включающее векторы
прогнозов, удовлетворяющих структуре иерархии, гарантирует
неухудшение качества прогнозов.

Следующие функции являются дивергенциями Брегмана и~могут быть
использованы в~качестве функций суммарных потерь~(\ref{eq:HierarchyLoss}) при согласовании прогнозов.
\begin{description}
    \item[Квадрат евклидового расстояния] \

    $D_F(\uu, \mathbf{v}) \hm= \| \uu\hm -
    \mathbf{v} \|^2$~--- канонический пример дивергенции Брегмана, порождается
    функцией $F(\uu) \hm= \| \uu \|^2$.
    \item[Квадрат расстояния Махаланобиуса]\

    $D_F(\uu, \mathbf{v})\hm = (1/2)(\uu - \mathbf{v})^{\mathrm{T}} Q (\uu \hm-
    \mathbf{v})$~--- обобщение евклидового расстояния, порождается
    квадратичной формой $F(\uu) \hm= (1/2) \uu^{\mathrm{T}} Q \uu$.
    \item[Обобщенная дивергенция Куль\-ба\-ка--Лейб\-ле\-ра]\

    $D_F(\uu, \mathbf{v}) \hm=
    \sum\nolimits_{i = 1}^n u_i \log ({u_i}/{v_i})
 \hm   - \sum\nolimits_{i = 1}^n u_i\hm + \sum\nolimits_{i = 1}^n v_i$
 по\-рож\-да\-ет\-ся     функцией
 $F(\uu) \hm= \sum\nolimits_{i = 1}^n u_i \log u_i \hm-
    \sum\nolimits_{i = 1}^n
    u_i$.
    \item[Расстояние  Itakura--Saito]\

    $D_F(\uu, \mathbf{v}) = \sum\nolimits_{i = 1}^n \left(
    (u_i/v_i) \hm- \log ({u_i}/{v_i}) \hm- 1    \right)$ порождается функцией
    $F(\uu)\hm = - \sum\nolimits_{i = 1}^n \log
    u_i$.
\end{description}

\begin{table*}\small
\begin{center}
   \begin{tabular}{|c|c|c|c|c|c|c|c|}
   \multicolumn{8}{c}{Вид записи базы данных железнодорожных перевозок}\\
\multicolumn{8}{c}{\ }\\[-4pt]
  \hline
  \tabcolsep=0pt\begin{tabular}{c}Дата\\ погрузки\end{tabular} &
    \tabcolsep=0pt\begin{tabular}{c}Станция\\ отправления\end{tabular} &
      \tabcolsep=0pt\begin{tabular}{c}Станция\\ назначения\end{tabular} &
        \tabcolsep=0pt\begin{tabular}{c}Количество\\ вагонов\end{tabular} &
          \tabcolsep=0pt\begin{tabular}{c}Код\\ груза\end{tabular} &
            \tabcolsep=0pt\begin{tabular}{c}Род\\ вагона\end{tabular} &
              \tabcolsep=0pt\begin{tabular}{c}Суммарный\\ вес груза\end{tabular} &
                \tabcolsep=0pt\begin{tabular}{c}Признак\\ маршрутной\\ отправки\end{tabular} \\
  \hline
  2007-01-01 & 020108 & 932902 & 1 & 1 & 216 & 56 & 9 \\
  \hline
\end{tabular}
\end{center}
%\vspace*{6pt}
\end{table*}



\section{Эксперимент}

В экспериментальной части рассматриваются данные о посуточной
загруженности железнодорожных узлов РЖД. Из имеющихся данных были
сформированы временн$\acute{\mbox{ы}}$е ряды, описывающие отправление 37~различных
типов груза со станций 98~железнодорожных веток посуточно.
Рассмотрены два вида иерархии.

В~случае плоской двухуровневой
иерархии решены задачи согласования прогнозов отправления грузов
по отдельности и~всех грузов в~сумме для каждой ветки, а~так\-же
решены задачи согласования прогнозов отправления груза с~каждой
ветки и~суммарного отправления груза со всех веток для каждого
типа груза.

В~случае неплоской трехуровневой иерархии решена
задача согласования прогнозов всех име\-ющих\-ся временн$\acute{\mbox{ы}}$х рядов.
Демонстрируется уменьшение значения функции суммарных потерь при
переходе от независимых прогнозов к~согласованным. В~качестве
функции суммарных потерь был использован квад\-рат евклидового
расстояния. Независимые прогнозы были получены алгоритмом Hist,
описанным в~работах~\cite{stenina2014reconciliation, medvednikova2012nonparametric}.

\vspace*{-5pt}

\paragraph*{Экспериментальные данные.} В~эксперименте использованы данные о посуточной загруженности
железнодорожных узлов РЖД с~1~января 2007~г.\ по 22~апреля 2008~г. В~таблице
приведен пример записи базы данных.


Коды станций представляют собой шестизначные числа. Станции, в~коде которых две первые цифры совпадают, входят в~одну
железнодорожную ветку. Станций отправления~--- 1566, станций
назначения~--- 1902, веток~--- 98. Код груза~--- натуральное число от~1 до~37; также имеются перевозки, где код груза не указан. Род
вагона~--- натуральное число, в~имеющихся данных~75~различных
родов вагонов.

\vspace*{-5pt}

\paragraph*{Иерархическая структура.}
Экспериментальные данные удовлетворяют структуре, изображенной на
рис.~3.

Как видно из рисунка, иерархия не
является плоской и~содержит три уровня временн$\acute{\mbox{ы}}$х рядов.
Временн$\acute{\mbox{ы}}$е
ряды нижнего уровня этой иерархии\linebreak\vspace*{-12pt}


\begin{center}  %fig3
\vspace*{2pt}
\mbox{%
 \epsfxsize=80mm
 \epsfbox{ste-3.eps}
 }


\vspace*{6pt}

%\noindent
{{\figurename~3}\ \ \small{Неплоская трехуровневая иерархия}}

\end{center}


%\vspace*{6pt}


\addtocounter{figure}{1}





 \noindent
 имеют два индекса, соответствующих
  номеру  ветки и~коду груза: $\x(i, j)$, $i \hm= 1,
\ldots, n$, $j \hm= 1, \ldots, m$,
 где $n$~--- число веток,
а~$m$~--- количество грузов.
 На среднем уровне~--- два семейства
временн$\acute{\mbox{ы}}$х
 рядов. Временн$\acute{\mbox{ы}}$е ряды, соответствующие
суммарному отправлению всех грузов с~каждой ветки,
 обозначаются $\x(i, :)$,
$i = 1, \ldots, n$. Ряды среднего уровня, соответствующие
суммарному отправлению со всех веток каждого из грузов,
обозначаются $\x(:, j)$, $j\hm = 1, \ldots, m$. Временной ряд
верхнего уровня обозначается $\x(:, :)$.

Условие согласованности
для трехуровневой иерархии задается равенствами ($t \hm= 1, \ldots, T$):
\begin{equation}
\left.
\begin{array}{rl}
    x_t(:, :) &= \displaystyle\sum\limits_{i = 1}^n x_t(i, :)\,;\\[6pt]
    x_t(:, :) &= \displaystyle\sum\limits_{j = 1}^m x_t(:, j)\,;\\[6pt]
    x_t(i, :) &= \displaystyle\sum\limits_{j = 1}^m x_t(i, j)\,, \enskip i = 1, \ldots n\,;  \\[6pt]
    x_t(:, j) &= \displaystyle\sum\limits_{i = 1}^n x_t(i, j)\,, \enskip j = 1, \ldots m\,.
    \end{array}
    \right\}
    \label{eq:3LevelConstraint}
\end{equation}
Векторная запись срезов иерархии, независимых прогнозов и~согласованных
прогнозов имеет размерность $d \hm= 1 \hm+ n \hm+ m \hm+ nm$ и~выглядит
сле\-ду\-ющим образом:
$$
    \boldsymbol{\chi}_t = \left(%
        \begin{array}{c}
            x_t(:, :) \\
            x_t(1, :) \\
            \vdots \\
            x_t(n, :) \\
            x_t(:, 1) \\
            \vdots \\
            x_t(:, m) \\
            x_t(1, 1) \\
            \vdots \\
            x_t(1, m) \\
            \vdots \\
            x_t(n, 1) \\
            \vdots \\
            x_t(n, m) \\
        \end{array}%
        \right)\,; \quad
    \hchi = \left(%
        \begin{array}{c}
            \hx(:, :) \\[2pt]
            \hx(1, :) \\
            \vdots \\
            \hx(n, :) \\[2pt]
            \hx(:, 1) \\
            \vdots \\
            \hx(:, m) \\[2pt]
            \hx(1, 1) \\
            \vdots \\
            \hx(1, m) \\
            \vdots \\
            \hx(n, 1) \\
            \vdots \\
            \hx(n, m) \\
        \end{array}%
        \right)\,;
        $$

        \noindent
        $$
    \hphi = \left(%
        \begin{array}{c}
            \hy(:, :) \\
            \hy(1, :) \\
            \vdots \\
            \hy(n, :) \\
            \hy(:, 1) \\
            \vdots \\
            \hy(:, m) \\
            \hy(1, 1) \\
            \vdots \\
            \hy(1, m) \\
            \vdots \\
            \hy(n, 1) \\
            \vdots \\
            \hy(n, m) \\
        \end{array}%
        \right).
$$
Матрица связей для условия~(\ref{eq:3LevelConstraint}) имеет размер
$(2 \hm+ n \hm+ m) \times (1\hm + n \hm+ m \hm+ nm)$:
\begin{multline*}
    S =
\left(%
\begin{array}{c|ccc|ccc|ccc}
  -1 & 1 & \cdots & 1 & 0 & \cdots & 0 & 0 & 0 & \cdots\\
  -1 & 0 & \cdots & 0 & 1 & \cdots & 1 & 0 & 0 & \cdots\\
  \hline
  0 & -1 & \ddots & 0 & 0 & \cdots & 0 & 1 & 1 & \cdots \\
  \vdots & \ddots & \ddots & \ddots & \cdots & \cdots & \cdots & \cdots & \cdots & \cdots\\
  0 & 0 & \ddots & -1 & 0 & \cdots & 0 & 0 & 0 & \cdots\\
  \hline
  0 & 0 & \cdots & 0 & -1 & \ddots & 0 & 1 & 0 & \cdots\\
  \vdots & \ddots & \ddots &\ddots  &\ddots  & \ddots & \ddots &   \ddots &\ddots  & \ddots\\
  %\ddots & \ddots & \ddots & \ddots & \cdots &  \ddots&\ddots  &\ddots  &\ddots  & \ddots & \vdots \\
  0 & 0 & \cdots & 0 & 0 & \cdots & -1 & 0 & 0 & \cdots\\
\end{array}\right.\\[6pt]
\left.
\begin{array}{cc|cccc|c|cccc}%
  \cdots & 0 & 0 & 0 & \cdots & 0 & \cdots & 0 & 0 & \cdots & 0 \\
    \cdots & 0 & 0 & 0 & \cdots & 0 & \cdots & 0 & 0 & \cdots & 0 \\
      \cdots& 1 & 0 & 0 & \cdots & 0 & \cdots & 0 & 0 & \cdots & 0 \\
        \cdots & \cdots &\cdots  &\cdots  & \cdots & \cdots &\cdots  &\cdots  & \cdots & \cdots & \vdots \\
          \cdots & 0 & 0 & 0 & \cdots & 0 & \cdots & 1 & 1 & \cdots & 1 \\
  \cdots & 0 & 1 & 0 & \cdots & 0 & \cdots & 1 & 0 & \cdots & 0 \\
    \ddots & \ddots & \ddots & \ddots & \cdots &  \ddots&\ddots  &\ddots  &\ddots  & \ddots & \vdots \\
      \cdots & 1 & 0 & 0 & \cdots & 1 & \cdots & 0 & 0 & \cdots & 1 \\
\end{array}\right).
\end{multline*}



Требуется, чтобы все прогнозы были неотрицательны, поэтому
множество $\mathcal{B}$~(\ref{eq:SetPhysicsConstraint}) задается как
$$
    \mathcal{B} = \{ \boldsymbol{\chi} \in \mathbb{R}^d \mid \boldsymbol{\chi}(i) \in [0, +\infty], i = 1, \ldots, d
    \}.
$$



В качестве функции суммарных потерь~$l_h$~(\ref{eq:HierarchyLoss})
используется квадрат евклидового расстояния.



\paragraph*{Оптимизационная задача.} Задача, решенная для согласования
прогнозов рассматриваемой иерархии, имеет вид:

\columnbreak


\noindent
$$
    \left\{%
\begin{array}{l}
    \| \boldsymbol{\chi} - \hchi \|^2 \rightarrow \min\limits_{\boldsymbol{\chi}}\,,  \\[9pt]
    S \boldsymbol{\chi} = 0\,,  \\[9pt]
    \boldsymbol{\chi} \geq 0\,.
\end{array}%
\right.
$$


\paragraph*{Результаты эксперимента.}
Для прогноза были использованы временн$\acute{\mbox{ы}}$е ряды, описывающие
суммарный вес отправленных грузов разных типов по каждой ветке.
Значения временн$\acute{\mbox{ы}}$х рядов нижнего уровня иерархии были
отнормированы на отрезок $[0;~1]$ по формуле
\begin{equation*}
    x_t^{\mathrm{norm}}(i, j) = \fr{x_t(i, j) - m(i, j)}{M(i, j) - m(i,
    j)}\,,
    \end{equation*}
    где
    $$
    m = \min\limits_{t = 1, \ldots, T} x_t(i, j)\,, \enskip
    M = \max\limits_{t = 1, \ldots, T} x_t(i, j)\,.
$$

 Прогнозы были
построены и~согласованы для~100~последних точек истории. Для
каждого отсчета времени строился вектор независимых прогнозов~$\hchi$ и~вектор согласованных прогнозов~$\hphi$. Для каждого
вектора вычислялось значение функции суммарных потерь, затем
вычислялись потери игрока~1 в~игре~(\ref{eq:game}), равные
разности суммарных потерь при использовании согласованных
прогнозов и~суммарных потерь при использовании независимых
прогнозов:
$$
    L_t = l_h(\hphi, \boldsymbol{\chi}_t) - l_h(\hchi, \boldsymbol{\chi}_t),
    \quad t = T -
    100 + 1, \ldots, T\,.
$$
Значения этой величины изображены на рис.~\ref{fig:LossDifference}.
% По оси ординат во всех случаях отложены
%номера контрольных точек. По оси абсцисс на рис.~\ref{fig:LossDifference},\,\textit{а} отложены номера железнодорожных веток,
%на рис.~\ref{fig:LossDifference},\,\textit{б}~--- индексы грузов, на рис.~\ref{fig:LossDifference},\,\textit{в} по оси абсцисс только один отсчет,
%соответствующий трехуровневой иерархии.
Теоретические выкладки
подтверждаются на практике. Для плоских двухуровневых иерархий
есть случаи, когда суммарные потери при переходе к~согласованным
прогнозам не изменяются, и~случаи, когда суммарные потери
уменьшаются. Для неплоской трехуровневой иерархии суммарные потери
во всех контрольных точках уменьшаются.


\section{Заключение}

Предложен алгоритм GTOp для согласования прогнозов иерархических
временн$\acute{\mbox{ы}}$х рядов. Алгоритм не требует оценки погрешностей
независимых прогнозов и~не требует их несмещенности. Для любого
набора независимых прогнозов временн$\acute{\mbox{ы}}$х рядов алгоритм согласования
не ухудшает качество прогнозирования. Возможна работа с~иерархическими структурами любой сложности. Все свойства алгоритма
согласования прогнозов GTOp под\-тверж\-да\-ются на практике.

\pagebreak

\end{multicols}

        \begin{figure}%fig4
    \vspace*{1pt}
 \begin{center}
 \mbox{%
 \epsfxsize=163.264mm
 \epsfbox{ste-4.eps}
 }
\end{center}
 \vspace*{-9pt}
    \Caption{Изменение суммарных потерь $L_t$:
    (\textit{а})~для каждой ветки;
(\textit{б})~для каждого груза;
(\textit{в})~вся иерархия}
    \label{fig:LossDifference}
%   \vspace*{-9pt}
\end{figure}

\begin{multicols}{2}


{\small\frenchspacing
 {%\baselineskip=10.8pt
 \addcontentsline{toc}{section}{References}
 \begin{thebibliography}{99}



\bibitem{tokmakova2012hyper}
\Au{Токмакова А.\,А., Стрижов В.\,В.} Оценивание
    гиперпараметров линейных и~регрессионных моделей при отборе
    шумовых и~коррелирующих признаков~// Информатика и~её
    применения, 2012. Т.~6. Вып.~4. С.~66--75.



    \bibitem{vasilyev2014using}
\Au{Васильев Н.\,С.} Использование принципа равновесия для
    управления маршрутизацией в~транспортных сетях~// Информатика
    и~её применения, 2014. Т.~8. Вып.~1. С.~28--35.



\bibitem{hong2014global}
\Au{Hong T., Pinson P., Fan~S.} Global energy forecasting competition 2012~// Int.
    J.~Forecasting, 2014. Vol.~30. No.\,2. P.~357--363.

\bibitem{kaggle}
    Kaggle. {\sf https://www.kaggle.com}.

\bibitem{hyndman2011optimal}
\Au{Hyndman R.\,J., Ahmed R.\,A., Athanasopoulos~G., Shang~H.\,L.} Optimal
    combination forecasts for hierarchical time series~// Comput.
    Stat. Data Anal., 2011. Vol.~55. No.\,9. P.~2579--2589.

\bibitem{kuznetsov2011smoothing}
\Au{Кузнецов М.\,П., Мафусалов А.\,А., Животовский~Н.\,К., Зайцев~Е.\,Ю.,
    Сунгуров~Д.\,С.} Сглаживающие алгоритмы прогнозирования~// Машинное
    обучение и~анализ данных, 2011. Т.~1. Вып.~1. С.~104--112.

\bibitem{stenina2014reconciliation}
\Au{Стенина М.\,М., Стрижов В.\,В.} Согласование агрегированных и~    детализированных прогнозов при решении задач непараметрического
    прогнозирования~//
    Системы и~средства информатики, 2014. Т.~24. Вып.~2. С.~21--34.

\bibitem{grunfeld1960aggregation} %8
\Au{Grunfeld Y., Griliches Z.} Is aggregation necessarily bad?~// Rev. Econ.
Stat., 1960. Vol.~42. No.\,1. P.~1--13.

\bibitem{orcutt1968data} %9
\Au{Orcutt G.\,H., Watts H.\,W., Edwards~J.\,B.}
Data aggregation and information loss~// Am. Econ. Rev., 1968. Vol.~58. No.\,4. P.~773--787.

\bibitem{edwards1969should} %10
\Au{Edwards J.\,B., Orcutt G.\,H.} Should aggregation prior to estimation be the rule?~//
Rev. Econ. Stat., 1969. Vol.~51. No.\,4. P.~409--420.

\bibitem{shlifer1979aggregation} %11
\Au{Shlifer E., Wolff R.\,W.} Aggregation and proration in forecasting~// Manage.
    Sci., 1979. Vol.~25. No.\,6. P.~594--603.

    \bibitem{fogarty1991production} %12
\Au{Fogarty D.\,W., Blackstone J.\,H., Hoffman~T.\,R.} Production and inventory
    management.~--- 2nd ed.~--- Cincinnati, OH, USA: South-Western Publication Co.,
    1990. 880~p.

\bibitem{narasimhan1995production} %13
\Au{Narasimhan S.\,L., McLeavey D.\,W., Billington~P.\,J.} Production planning and inventory
    control.~--- 2nd ed.~--- Englewood Cliffs, NJ, USA: Prentice Hall,
    1995. 716~p.

\bibitem{schwarzkopf1988top} %14
\Au{Schwarzkopf A.\,B., Tersine R.\,J., Morris~J.\,S.} Top-down versus bottom-up forecasting
    strategies~// Int. J.~Prod. Res., 1998.
    Vol.~26. No.\,11. P.~1833--1843.

    \bibitem{fliedner1999investigation} %15
\Au{Fliedner G.} An investigation of aggregate variable time series forecast strategies with
    specific subaggregate time series statistical correlation~//
    Comput. Oper. Res., 1999. Vol.~26. No.\,10--11.
    P.~1133--1149.

\bibitem{vanerven:hal-00920559} %16
\Au{Van Erven T., Cugliari J.}  Game-theoretically optimal reconciliation of contemporaneous
    hierarchical time series forecasts. 2013.
    {\sf https://hal.inria.fr/hal-00920559}.



\bibitem{petrosyan1998theory} %17
\Au{Петросян Л.\,А., Зенкевич Н.\,А., Семина~Е.\,А.}
Теория игр.~--- М.: Университет, 1998. 301~с.

\bibitem{menshikov2010lections} %18
\Au{Меньшиков И.\,С.} Лекции по теории игр и~экономическому моделированию.~--- 2-е
    изд., испр. и~доп.~---
    М.: Контакт Плюс, 2010. 336~с.

\bibitem{cesa2006prediction} %19
\Au{Cesa-Bianchi N., Lugosi G.} Prediction, learning, and games.~--- Cambridge: Cambridge
    University Press, 2006. Vol.~1. 403~p.

\bibitem{boyd2009convex} %20
\Au{Boyd S., Vandenberghe L.} Convex optimization.~---
Cambridge: Cambridge University Press, 2009. 732~p.

\bibitem{bregman1967relaxation} %21
\Au{Bregman L.\,M.} The relaxation method of finding the common point of convex sets and its
    application to the solution of problems in convex programming~//
    USSR Comput. Math. Math. Phys.,
    1967. Vol.~7. No.\,3. P.~200--217.

\bibitem{medvednikova2012nonparametric} %22
\Au{Вальков А.\,С., Кожанов Е.\,М., Медведникова~М.\,М.,
    Хусаинов~Ф.\,И.} Непараметрическое прогнозирование загруженности системы железнодорожных
    узлов по историческим данным~// Машинное обучение и~анализ данных, 2012. Т.~1. Вып.~4. С.~448--465.
     \end{thebibliography}

 }
 }

\end{multicols}

\vspace*{-3pt}

\hfill{\small\textit{Поступила в~редакцию 27.10.14}}

%\newpage

\vspace*{12pt}

\hrule

\vspace*{2pt}

\hrule

%\vspace*{12pt}

\def\tit{FORECASTS RECONCILIATION FOR HIERARCHICAL TIME SERIES FORECASTING PROBLEM}

\def\titkol{Forecasts reconciliation for hierarchical time series forecasting problem}

\def\aut{M.\,M.~Stenina$^1$ and V.\,V.~Strijov$^2$}

\def\autkol{M.\,M.~Stenina and V.\,V.~Strijov}

\titel{\tit}{\aut}{\autkol}{\titkol}

\index{Stenina M.\,M.}
\index{Strijov V.\,V.}

\vspace*{-9pt}


\noindent
$^1$Moscow Institute of Physics and Technology, 9 Institutskiy Per.,
Dolgoprudny, Moscow Region 141700, Russian\linebreak
$\hphantom{^1}$Federation

\noindent
$^2$Dorodnicyn Computing Center, Russian Academy of Sciences, 40~Vavilov Str.,
Moscow 119333, Russian\linebreak
$\hphantom{^1}$Federation


\def\leftfootline{\small{\textbf{\thepage}
\hfill INFORMATIKA I EE PRIMENENIYA~--- INFORMATICS AND
APPLICATIONS\ \ \ 2015\ \ \ volume~9\ \ \ issue\ 2}
}%
 \def\rightfootline{\small{INFORMATIKA I EE PRIMENENIYA~---
INFORMATICS AND APPLICATIONS\ \ \ 2015\ \ \ volume~9\ \ \ issue\ 2
\hfill \textbf{\thepage}}}

\vspace*{3pt}


\Abste{The hierarchical time series forecasting problem is
    researched. Time series forecasts must satisfy the physical
    constraints and the hierarchical structure. In this paper,
    a~new algorithm for hierarchical time series
    forecasts reconciliation is proposed. The algorithm is called GTOp (Game-theoretically
    optimal reconciliation). It guarantees that the quality of reconciled
    forecasts is not worse than the quality of self-dependent forecasts.
    This approach is based on Nash equilibrium search for the
    antagonistic game and turns the forecasts reconciliation problem into
    the optimization problem with equality and inequality
    constraints. It is proved that the Nash
    equilibrium in pure strategies exists in the game if some assumptions
    about the hierarchical structure, the physical
    constraints, and the loss function are satisfied. The algorithm
    performance is demonstrated for different types of
    hierarchical structures of time series.}

\KWE{hierarchical time series; reconciliation of time
    series forecasts; antagonistic game; Nash equilibrium}


\DOI{10.14357/19922264150209}

\Ack
\noindent
The research was financially supported by the Russian Foundation for
Basic Research (project~13-07-13139).



%\vspace*{3pt}

  \begin{multicols}{2}

\renewcommand{\bibname}{\protect\rmfamily References}
%\renewcommand{\bibname}{\large\protect\rm References}



{\small\frenchspacing
 {%\baselineskip=10.8pt
 \addcontentsline{toc}{section}{References}
 \begin{thebibliography}{99}

 \bibitem{tokmakova2012hyper-1}
\Aue{Tokmakova, A.\,A., and V.\,V.~Strizhov}.
2012. Otsenivanie giperparametrov lineynykh i~regressionnykh
    modeley pri otbore shumovykh i~korreliruyushchikh priznakov [Estimation of linear model
    hyperparameters for noise or correlated feature selection problem].
    \textit{Informatika i~ee Primeneniya}~--- \textit{Inform. Appl.} 6(4):66--75.


\bibitem{vasilyev2014using-1}
    \Aue{Vasil'ev, N.\,S.} 2014. Ispol'zovanie printsipa ravnovesiya dlya upravleniya marshrutizatsiey v
    transportnykh setyakh
    [Equilibrium principle application to routing control in packet
    data transmission networks]. \textit{Informatika i~ee Primeneniya}~---
    \textit{Inform. Appl.} 8(1):28--35.

\bibitem{hong2014global-1}
\Aue{Hong, T., P.~Pinson, and S.~Fan}.
2014. Global energy forecasting competition 2012.  \textit{Int.
    J.~Forecasting} 30(2):357--363.

\bibitem{kaggle-1}
    Kaggle. Available at: {\sf https://www.kaggle.com} (accessed May~20, 2015).

\bibitem{hyndman2011optimal-1}
\Aue{Hyndman, R.\,J., R.\,A.~Ahmed, G.~Athanasopoulos, and H.\,L.~Shang}.
2011. Optimal
    combination forecasts for hierarchical time series. \textit{Comput.
    Stat. Data Anal}. 55(9):2579--2589.

\bibitem{kuznetsov2011smoothing-1} %6
\Aue{Kuznetsov, M.\,P., A.\,A.~Mafusalov, N.\,K.~Zhivotovskiy, E.\,Yu.~Zaytsev,
and D.\,S.~Sungurov}. 2011. Sglazhivayushchie algoritmy prognozirovaniya
[Smoothing forecast algorithms]. \textit{Mashinnoe obuchenie i~analiz dannykh}
[J.~Machine Learning Data Anal.] 1(1):104--112.

\bibitem{stenina2014reconciliation-1}
\Aue{Stenina, M.\,M., and V.\,V.~Strizhov}. 2014. Soglasovanie agregirovannykh i~detalizirovannykh
prognozov pri re\-she\-nii zadach neparametricheskogo prognozirovaniya [Reconciliation
    of aggregated and disaggregated time series forecasts in nonparametric
    forecasting problem].
\textit{Sistemy i~Sredstva Informatiki}~--- \textit{Systems and Means of
    Informatics} 24(2):21--34.

\bibitem{grunfeld1960aggregation-1} %8
\Aue{Grunfeld, Y., and Z.~Griliches}.
1960. Is aggregation necessarily bad? \textit{Rev. Econ.
Stat.} 42(1):1--13.



\bibitem{orcutt1968data-1} %9
\Aue{Orcutt, G.\,H., H.\,W.~Watts,  and J.\,B.~Edwards}.
1968. Data aggregation and information loss. \textit{Am. Econ. Rev.} 58(4):773--787.

\bibitem{edwards1969should-1} %10
\Aue{Edwards, J.\,B., and G.\,H.~Orcutt}. 1969.
Should aggregation prior to estimation be the rule? \textit{Rev. Econ. Stat.} 51(4):409--420.

\bibitem{shlifer1979aggregation-1} %11
   \Aue{Shlifer, E., and R.\,W.~Wolff}. 1979. Aggregation and proration
    in forecasting. \textit{Manage.
    Sci.} 25(6):594--603.

    \bibitem{fogarty1991production-1} %12
\Aue{Fogarty, D.\,W., J.\,H.~Blackstone, and T.\,R.~Hoffman}. 1990.
\textit{Production and inventory
    management}. 2nd ed. Cincinnati, OH: South-Western Publication Co. 880~p.

\bibitem{narasimhan1995production-1} %13
\Aue{Narasimhan, S.\,L., D.\,W.~McLeavey, and P.\,J.~Billington}.
1995. \textit{Production planning and inventory
    control}. 2nd ed. Englewood Cliffs, NJ: Prentice Hall. 716~p.

\bibitem{schwarzkopf1988top-1} %14
\Aue{Schwarzkopf, A.\,B., R.\,J.~Tersine, and J.\,S.~Morris}.
1998. Top-down versus bottom-up forecasting
    strategies. \textit{Int. J.~Prod. Res}.
    26(11):1833--1843.

    \bibitem{fliedner1999investigation-1} %15
\Aue{Fliedner, G.} 1999. An investigation of aggregate variable time series forecast strategies with
    specifc subaggregate time series statistical correlation.
    \textit{Comput. Oper. Res.} 26(10--11):1133--1149.

\bibitem{vanerven:hal-00920559-1} %16
    \Aue{Van Erven, T., and J.~Cugliari}.
    2013. Game-theoretically optimal reconciliation of contemporaneous
    hierarchical time series forecasts.
    Available at: {\sf https://hal.inria.fr/hal-00920559}
    (accessed May~20, 2015).

    \bibitem{petrosyan1998theory-1} %17
    \Aue{Petrosyan, L.\,A., N.\,A.~Zenkevich, and E.\,A.~Semina}.
    1998. \textit{Teoriya igr} [Games theory]. Moscow: Knizhnyy Dom
    Universitet. 301~p.

\bibitem{menshikov2010lections-1} %18
\Aue{Men'shikov, I.\,S.} 2010. \textit{Lektsii po teorii igr
i~eko\-no\-mi\-che\-sko\-mu modelirovaniyu}     [Games
    theory and economics modeling lectures]. 2nd~ed.
    Moscow: OOO Kontakt Plyus. 336~p.



\bibitem{cesa2006prediction-1} %19
\Aue{Cesa-Bianchi, N., and G.~Lugosi.} 2006.
\textit{Prediction, learning, and games.} Cambridge: Cambridge
    University Press. Vol.~1. 403~p.

\bibitem{boyd2009convex-1}
\Aue{Boyd, S., and L.~Vandenberghe}. 2009. \textit{Convex optimization}.
Cambridge: Cambridge University Press. 732~p.

\bibitem{bregman1967relaxation-1}
\Aue{Bregman, L.\,M.} 1967. The relaxation method of finding the common point
of convex sets and its application to the solution of problems in convex
programming. \textit{USSR Comput. Math. Math. Phys.} 7(3):200--217.

\bibitem{medvednikova2012nonparametric-1}
\Aue{Val'kov, A.\,S., E.\,M.~Kozhanov, M.\,M.~Medvednikova, and F.\,I.~Khusainov}.
2012.
    Neparametricheskoe prognozirovanie zagruzhennosti sistemy
    zheleznodorozhnykh uzlov po istoricheskim dannym [Nonparametric
    forecasting of railroad stations occupancy according to historical
    data]. \textit{Mashinnoe Obuchenie i~Analiz Dannykh} [J.~Machine
    Learning Data Anal.] 1(4):448--465.

    \end{thebibliography}

 }
 }

\end{multicols}

\vspace*{-3pt}

\hfill{\small\textit{Received October 27, 2014}}

%\vspace*{-18pt}

    \Contr


    \noindent
    \textbf{Stenina Mariya M.} (b.\ 1991)~---
    student, Moscow Institute of Physics and Technology, 9 Institutskiy Per.,
    Dolgoprudny, Moscow Region 141700, Russian Federation; mmedvednikova@gmail.com

    \vspace*{3pt}

    \noindent
    \textbf{Strijov Vadim V.} (b.\ 1967)~--- Doctor of Science
    in physics and mathematics, leading scientist,
    Dorodnicyn Computing Center, Russian Academy of Sciences, 40~Vavilov Str.,
    Moscow 119333, Russian Federation; strijov@gmail.com






\label{end\stat}


\renewcommand{\bibname}{\protect\rm Литература}