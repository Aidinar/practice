
\def\stat{zhevn}

\def\tit{МЕТОДИКА МОДЕЛИРОВАНИЯ НАГРУЗКИ НА СЕРВЕР В~ОТКРЫТЫХ 
СИСТЕМАХ ОБЛАЧНЫХ ВЫЧИСЛЕНИЙ}

\def\titkol{Методика моделирования нагрузки на сервер в~открытых 
системах облачных вычислений}

\def\autkol{Д.\,В.~Жевнерчук, А.\,В.~Николаев}
\def\aut{Д.\,В.~Жевнерчук$^1$, А.\,В.~Николаев$^2$}

\titel{\tit}{\aut}{\autkol}{\titkol}

%{\renewcommand{\thefootnote}{\fnsymbol{footnote}}\footnotetext[1]
%{Работа выполнена при поддержке РФФИ (гранты 09-07-12098, 09-07-00212-а и
%09-07-00211-а) и Минобрнауки РФ (контракт №\,07.514.11.4001).}}


\renewcommand{\thefootnote}{\arabic{footnote}}
\footnotetext[1]{Чайковский технологический институт (филиал) Ижевского государственного технического университета, 
drevnigeck@yandex.ru}
\footnotetext[2]{Чайковский технологический институт (филиал) Ижевского государственного технического университета, 
elodssa@yandex.ru}

  \Abst{Планирование серверного ресурса систем облачных вычислений является сложной 
задачей. Необходимо учитывать такие факторы, как состав и параметры аппаратной 
платформы, параметры системного программного обеспечения, управляющего выполнением 
прикладных программ, свойства трафика, порождаемого пользователями и определяющего 
режимы функционирования прикладных программ. Предложена методика оценки загрузки 
серверного ресурса открытых систем облачных вычислений (ОСОВ) на основе анализа процессов 
взаимодействия пользователей с программным обеспечением. Обоснована ее достоверность. 
Приведены результаты моделирования нагрузки на серверную часть системы управления 
средами имитационного моделирования.}
  
  \KW{облачные вычисления; имитационное моделирование; человеко-машинное 
взаимодействие} 

   
   \vskip 14pt plus 9pt minus 6pt

      \thispagestyle{headings}

      \begin{multicols}{2}

            \label{st\stat}
  
\section{Введение}
  
  Вопросы проектирования систем поддержки удаленного вычислительного 
эксперимента до \mbox{конца} не решены. В~ходе проектирования ОСОВ 
возникает задача моделирования потока запросов, 
приводящих к загрузке серверной части. Такие модели позволяют получить 
оценку аппаратного ресурса для обслуживания некоторого количества клиентских 
систем при рабочей и пиковой нагрузке. В~работах~[1--3] 
рассматриваются теоретические модели трафика в локальных и глобальных сетях. 
В~основном полученные результаты имеют практическую значимость при 
проектировании средств передачи данных в сети. В~работах~[3--6] 
построены модели трафика, поступающего на вход серверов разного типа, 
таких как веб-серверы, серверы баз данных и~др. На основании обзора работ 
были сделаны следующие выводы:
\begin{enumerate}[1.]
\item Модели описывают трафик систем, построенных на основе определенных 
технологий и/или предназначенных для решения ограниченного круга задач.
\item Процессы формирования запросов моделируются на основании замеров уже 
переданного в сеть трафика.
\item Модели описывают смешанный трафик. 
\end{enumerate}

  Для анализа трафика ОСОВ классические 
методики моделирования трафика не эффективны, поскольку
  \begin{itemize} %[1)]
\item в общем случае ОСОВ обладает свойствами расширения по произвольным 
программно-ап\-па\-рат\-ным платформам, по решаемым задачам, по источникам 
нагрузки; 
\item в ОСОВ постоянно происходят качественные изменения, поэтому на 
основании конечного числа измерений трафика можно построить модели, 
описывающие ОСОВ только в некотором подмножестве состояний;
\item отсутствуют развитые средства автоматизации и механизмы контроля 
перехода ОСОВ в новое качественное состояние.
\end{itemize}

  Таким образом, задача разработки эффективных методик оценки нагрузки 
ОСОВ до конца не решена и является актуальной. 

\begin{figure*}[b] %fig1
 \vspace*{1pt}
 \begin{center}
 \mbox{%
 \epsfxsize=164.097mm
 \epsfbox{zhe-1.eps}
 }
 \end{center}
 \vspace*{-9pt}
\Caption{Методика моделирования нагрузки на сервер}
\end{figure*}
  
\section{Постановка задачи}
  
  Была поставлена задача разработки методики моделирования и построения на 
ее основе моделей нагрузки на серверную часть в ОСОВ. 
К~методике и моделям предъявлены следующие требования:
  \begin{enumerate}[1.]
\item Модели должны описывать дифференци\-ро\-ванный трафик, из которого 
можно выделить\linebreak
 потоки, принадлежащие определенному программному 
обеспечению и связанные с определенными задачами.
\item Источник дифференцированного трафика должен определяться 
процессами человеко-ма\-шин\-но\-го взаимодействия.
\item В модель должны передаваться эмпирические функции распределения 
вероятностей интервалов времени между передачами управляющих сигналов 
серверу, приводящих к существенной загрузке центрального процессора и 
оперативной памяти.
\item Должен проводиться системный анализ процессов решения 
пользовательских задач с применением программного обеспечения и учетом поведения 
пользователя при решении задач с по\-мощью программ.
\item Должна обеспечиваться высокая степень автоматизации процессов сбора 
эмпирических данных.
\end{enumerate}

  Ставилась задача применить представленную методику для изучения системы 
обработки сред имитационного моделирования и проверить достоверность 
построенных моделей нагрузки на сервер.
  
\section{Методика моделирования нагрузки на~сервер}
  
  Предлагаемая методика моделирования нагрузки на сервер включает ряд 
этапов:
  \begin{enumerate}[1.]
\item Сбор сведений о процессе взаимодействия клиента и сервера.
\item Определение последовательности выполнения действий.
\item Построение имитационной модели процесса взаимодействия.
\item Проведение экспериментов с имитационной моделью процесса 
взаимодействия и необходимой настройки.
\item Адаптацию полученного генератора нагрузки к работе с внешней 
системой.
\end{enumerate}
  
  Схема методики представлена на рис.~1.
  
  \begin{table*}[b]\small
\begin{center}
\Caption{Наблюдение за процессом изучения GPSS World Student}
\vspace*{2ex}

\begin{tabular}{|l|l|c|c|} 
\hline
\multicolumn{1}{|c|}{\raisebox{-6pt}[0pt][0pt]{Действие}}&
\multicolumn{1}{c|}{\raisebox{-6pt}[0pt][0pt]{Тип моделей}} & \multicolumn{2}{c|}{Категория}\\
\cline{3-4}
&&Успевающие&Неуспевающие\\
\hline
Количество ошибок компиляции в режиме
&Простые модели&[0--3]&[2--6]\\
 отладки  (1~задание)&Сложные модели&\hphantom{9}[4--12]&\hphantom{9}[8--16]\\
\hline
\multicolumn{2}{|l|}{Поиск ошибки и ее устранение, с}&\hphantom{9}[30--120]&\hphantom{9}[90--200]\\
\hline
&Простые модели&[20--60]&\hphantom{9}[40--120]\\
Анализ итогового отчета, с&Сложные модели&\hphantom{9}[20--120]&\hphantom{9}[90--240]\\
&Первичное ознакомление&\hphantom{9}[60--120]&\hphantom{9}[90--120]\\
\hline
\multicolumn{2}{|l|}{Кодирование новой модели, мин}&[120--300]& [240--420]\\
\multicolumn{2}{|l|}{(подготовка первого варианта кода модели), мин}&[360--720]&\hphantom{9}[600--1080]\\
\hline
\multicolumn{2}{|l|}{
\tabcolsep=0pt\begin{tabular}{l}Работа со средой моделирование по инструкции\\
(время поиска функциональности)\end{tabular}} &[20--40]&[30--90]\\
\hline
\end{tabular}
\end{center}
\end{table*}


  На первом этапе необходимо собрать сведения о взаимодействии клиента и 
сервера. Клиент работает с сервером в режиме за\-прос--от\-вет. Необходимо 
получить данные об интервалах времени между определенными действиями 
клиента. Для упрощения сбора данных было разработано клиент-серверное 
приложение <<Хронометр>>, позволяющее настроить список действий 
пользователя, требующих отметки времени выполнения. После настройки 
клиентская часть <<Хронометра>> начинает замерять интервалы времени, в 
течение которых выполняются действия, и передавать собираемые данные на 
сервер. Такой подход упрощает сбор сведений, так как данные можно собирать 
сразу с группы пользователей.
  
  На втором этапе определяется последовательность действий клиента по 
отношению к серверу. Для этого выполняются следующие шаги. На основе 
собранных данных строятся эмпирические функции распределения вероятностей 
интервалов времени, затраченного на действия клиента. Строятся сценарии 
  че\-ло\-ве\-ко-ма\-шин\-но\-го взаимодействия и вводится классификация 
пользователей, использующих определенные сценарии. Для упрощения работы 
было разработано программное средство <<Редактор сценариев>>, позволяющее 
быстро обработать список действий пользователя и увидеть максимальное, 
минимальное и среднее время выполнения действий. Кроме того, с его помощью 
можно построить эмпирические функции распределения интервалов времени в 
синтаксисе языка {GPSS} (General Purpose Simulation System).
  
  На третьем этапе строится имитационная модель процесса взаимодействия 
клиентов с сервером для оценки времени между событиями прихода запросов от 
клиента, приводящих к существенной загрузке центрального процессора и 
оперативной памяти. Для автоматизации построения имитационной модели было 
разработано программное средство <<Генератор имитационной модели>> (ГИМ). 
С~его помощью можно быстро получить код модели на основании вводимых 
параметров, эмпирических функций и сценариев действий. Программное средство ГИМ использует 
заготовленные шаблоны на языке имитационного моделирования {GPSS}.
  
  Далее проводится эксперимент с имитационной моделью, определение 
необходимых параметров и установка настроек. После этого строится модель 
трафика, поступающего на сервер, включающего запросы от клиента, приводящие 
к существенной загрузке центрального процессора и оперативной памяти. Все 
настройки и текст модели сохраняются в базу данных для последующего 
воспроизведения потока запросов.
  
  На последнем этапе полученную модель генерации запросов пользователя 
адаптируют для работы с внешней системой. 

\section{Ход исследования}
  
  Исследования были проведены для среды моделирования GPSS World 
Student, с которой пользователи работают в режиме обучения. В~построенных 
моделях трафика учитывалась информация о запросах, приводящих к прогонам 
имитационных моделей, что влияет на загрузку аппаратного ресурса. Была 
проверена гипотеза о достоверности построенных моделей. 

\subsection{Сбор данных о~процессе взаимодействия учащегося со~средой 
GPSS World Student}
  
  Был проведен хронометраж действий учащихся по изучению среды 
моделирования с помощью учебных моделей, по кодированию и отладке модели. 
Для формирования журнала действий пользователя использовалось программное 
средство <<Хронометр~1.0>>. 


  Было замечено, что в ходе выполнения учебного задания учащийся сначала 
формирует код модели, далее модель тестируется и отлаживается. Это 
сопровождается определенным числом попыток компиляции модели и 
интервалами времени поиска и устранения ошибок в коде. После отладки 
выполняется разовый контрольный прогон модели, после чего анализируется 
выходной отчет, на основании которого осуществляется поиск логических 
ошибок. Учащимся может быть проведено несколько дополнительных 
исправлений кода. Модель вновь\linebreak\vspace*{-12pt}
\pagebreak

\end{multicols}

\begin{figure} %fig2
 \vspace*{1pt}
 \begin{center}
 \mbox{%
 \epsfxsize=158.423mm
 \epsfbox{zhe-2.eps}
 }
 \end{center}
 \vspace*{-9pt}
\Caption{Число шагов отладки простой~(\textit{а}) и сложной~(\textit{б}) модели успевающим 
(верхний ряд) и  неуспевающим учащимся (нижний ряд)}
\vspace*{6pt}
\end{figure}

  \begin{figure} %fig3
   \vspace*{1pt}
 \begin{center}
 \mbox{%
 \epsfxsize=165.332mm
 \epsfbox{zhe-3.eps}
 }
 \end{center}
 \vspace*{-9pt}
  \Caption{Время написания первого варианта кода простой~(\textit{а}) и сложной~(\textit{б}) 
модели успевающим (верхний ряд) и неуспевающим учащимся (нижний ряд)}
  \end{figure}

\begin{figure} %fig4
 \vspace*{1pt}
 \begin{center}
 \mbox{%
 \epsfxsize=162.78mm
 \epsfbox{zhe-4.eps}
 }
 \end{center}
 \vspace*{-9pt}
\Caption{Время поиска ошибок компиляции успевающим~(\textit{а}) и 
неуспевающим~(\textit{б}) учащимся}
\end{figure}
  

\begin{multicols}{2}

\noindent
 тестируется, отлаживается, и результаты 
итогового прогона снова анализируются. 
  
  В режиме изучения готовой модели или среды моделирования с 
использованием методических указаний основное время тратится на анализ 
инструкций.


  
  Наблюдения проводились за тремя учебными группами общей численностью 
43~чел. Было проведено 5~занятий (10~академических часов). На основе 
полученных данных построена классификация учащихся и задач по времени 
решения. Все учащиеся разделены на две группы: <<успевающие>> и 
<<неуспевающие>>, а задачи~--- на группы <<простые>> и <<сложные>>. 
Полученные граничные оценки интервалов времени действий учащегося в 
режимах обучения и выполнения задания приведены в табл.~1. 
  
  С помощью программы <<Редактор сценариев>> были построены 
необходимые эмпирические законы распределения интервалов времени 
  (рис.~2--5).


  Полученные результаты были использованы при построении имитационных 
моделей взаимодействия учащихся со средой GPSS World.
  
\subsection{Модель взаимодействия учащегося со~средой GPSS World 
Student}

\vspace*{6pt}
  
  Выделим события, приводящие к компиляции и прогону модели, а 
следовательно, и к загрузке центрального процессора. При возникновении 
события первого запуска модели ($e_1$) происходит компиляция, в результате 
которой формируется отчет о готовности прогона. При возникновении события 
<<запуск модели>> ($e_2$) выполняется прогон, в результате которого 
формируется отчет с откликом. В~модели вводится 2~класса задач: простые и 
сложные,~--- и 2~класса учащихся: успевающие и неуспевающие. В~зависимости 
от класса задачи и типа учащегося выбираются построенные в результате 
хронометража функции, определяющие количество ошибок компиляции и время 
задержки при написании кода модели, анализа ошибок компиляции, анализа 
итогового отчета. При построении модели были сделаны следующие допущения:

\noindent
  \begin{enumerate}[1.]
\item Время компиляции модели пренебрежимо мало по сравнению со 
временем прогона модели, а\linebreak
\end{enumerate}



\end{multicols}

\begin{figure} %fig5
 \vspace*{1pt}
 \begin{center}
 \mbox{%
 \epsfxsize=163.628mm
 \epsfbox{zhe-5.eps}
 }
 \end{center}
 \vspace*{-9pt}
\Caption{Анализ отчетов простой~(\textit{а}) и сложной~(\textit{б}) модели успевающим 
(верхний ряд) и  неуспевающим  учащимся (нижний ряд)}
%\vspace*{6pt}
\end{figure}


\vspace*{-24pt}

\begin{multicols}{2}

\noindent
\begin{itemize}
\item[\ ] также с периодами поиска, устранения ошибок и 
анализа отчета и составляет менее 0,1\%.
\item[2.] Время прогона одной учебной модели варьируется в интервале [0,3--4]~c 
в зависимости от алгоритмических свойств модели и от аппаратного ресурса, 
что также пренебрежимо мало в случае однопользовательского режима. 
  \end{itemize}

\section{Сравнение отклика имитационной модели с~реальной 
системой}
  
  Для проведения экспериментальных исследований была разработана система 
моделирования работы комплекса виртуальных лабораторий Open Virtual 
Research Space ({OVRS}), которая представляет собой кли\-ент-сер\-вер\-ное 
приложение для исследования процессов функционирования открытого 
виртуального исследовательского пространства (ОВИП)~\cite{7zh, 8zh}. 
  
  Для оценки достоверности полученной модели был проведен хронометраж 
запросов на запуск имитационного эксперимента в среде {GPSS} тремя 
группами учащихся (табл.~2). 

\vspace*{12pt}

\noindent
{\small
%\begin{center}
{{\tablename~2}\ \ \small{Хронометраж запросов на запуск имитационного эксперимента}}

%\parbox{226pt}{\Caption{Хронометраж запросов на запуск имитационного эксперимента}

\vspace*{-6pt}

\begin{center}
\tabcolsep=6.4pt
\begin{tabular}{|c|c|c|c|c|}
\hline
№&
\multicolumn{2}{c|}{Учащиеся}&\multicolumn{2}{c|}{Задачи}\\
\cline{2-5}
\tabcolsep=0pt\begin{tabular}{c}груп-\\ пы\end{tabular}&
\tabcolsep=0pt\begin{tabular}{c}Успева-\\ ющие\end{tabular}&
\tabcolsep=0pt\begin{tabular}{c}Неуспе-\\ вающие\end{tabular}&Сложные&Простые\\
\hline
1&\hphantom{9}8&12&5&3\\
2&10&\hphantom{9}0&6&2\\
3&\hphantom{9}0&12&4&1\\
\hline
\end{tabular}
\end{center}

}

\vspace*{6pt}
  
  На рис.~6 приведены гистограммы пиковой загрузки реального и модельного 
сервера запросами на запуск имитационной модели. 
  
  Построены доверительные интервалы Велча для среднего значения числа 
заявок, поступающих в  интервалы времени, равные 60~с: ($-0{,}21$,\,0,94), ($-
0{,}83,\,1{,}07$); ($-1{,}17,\,1{,}35$).





  
  \begin{figure*} %fig6
   \vspace*{1pt}
 \begin{center}
 \mbox{%
 \epsfxsize=165.266mm
 \epsfbox{zhe-6.eps}
 }
 \end{center}
 \vspace*{-9pt}
  \Caption{Пиковая загрузка модельного (слева) и реального (справа) сервера по табл.~2: 
(\textit{а})~для группы~1; (\textit{б})~для группы~2; (\textit{в})~для группы~3}
  \end{figure*}
  
  Таким образом, достоверность построенной модели подтверждается наличием 
нуля в каждом интервале.
  
\section{Заключение}
  
  В ходе исследования была разработана методика оценки нагрузки на серверную 
часть ОСОВ,\linebreak особенностью которой является 
моделирование\linebreak потока запросов на основе системного анализа поведения групп 
пользователей при решении определенных задач с применением программного 
обеспечения. 
  
  С использованием методики были построены имитационные модели и 
экспериментальная среда исследования взаимодействия пользователя с сис\-те\-мой 
имитационного моделирования в режиме обучения. 
  
  Построены доверительные интервалы Велча для среднего значения числа 
запросов, поступающих на реальный сервер и его модель в интервалы времени, 
равные 60~с: ($-0{,}21$,\,0,94), ($-0{,}83,\,1,07$); ($-1{,}17,\,1,35$), что 
подтверждает достоверность построенной модели.
  
  Использование методов системного анализа для исследования поведения групп 
пользователей при решении определенных задач с применением средств 
автоматизации позволяет адаптировать предложенную методику для изучения 
серверной нагрузки в ОСОВ.
  
  Методика может быть применена при построении моделей процессов 
взаимодействия пользователей с произвольным программным обеспечением, 
доступ к которому предоставлен системами облачных вычислений. При этом 
модели будут отражать свойства потока запросов к серверной части, приводящих 
к загрузке ресурсов.
  
  Перспективным направлением развития данного исследования является теория 
проектирования систем \textit{cloud computing} и, в частности, вопросы 
зависимости пиковых загрузок от поведения пользователей.

{\small\frenchspacing
{%\baselineskip=10.8pt
\addcontentsline{toc}{section}{Литература}
\begin{thebibliography}{9}

\bibitem{2zh} %1
\Au{Willinger W., Taqqu M.\,S., Erramilli~A.}
A~Bibliographical guide to self-similar traffic and performance modeling for 
modern high-speed networks~// Stochastic networks: Theory and applications.~--- 
Oxford University Press, 1996. P.~282--296.

\bibitem{1zh} %2
\Au{Столлингс В.} Современные компьютерные сети.~--- СПб.: Питер, 2003. 
782~с.

\bibitem{3zh} %3
\Au{Petroff V.} Self-similar network traffic: From chaos and fractals to forecasting 
and QoS~// NEW2AN.~--- St.\ Petersburg, 2004. P.~110--118.

\bibitem{5zh} %4
\Au{Шелухин О.\,И.,  Тенякшев А.\,М.,  Осин А.\,В.}
Фрактальные процессы в телекоммуникациях.~--- М.: Радиотехника, 2003. 
480~с.

\bibitem{4zh} %5
\Au{Dang T.\,D., Sonkoly~B., Molnar~S.}
Fractal analysis and modelling of VoIP traffic~// 11th  Telecommunications Network 
Strategy and Planning Symposium (International) (NETWORKS 2004) 
Proceedings.~--- Vienna, Austria, 2004. P.~123--130.

\bibitem{6zh} %6
\Au{Криштофович А.\,Ю.}
Применение модели трафика сети ОКС №\,7 для управления потоками 
сигнальной нагрузки~// Инфокоммуникационные технологии, 2004. Т.~2. №\,2. 
С.~25--27.

\bibitem{7zh}
\Au{Жевнерчук Д.\,В., Николаев~А.\,В.}
Открытый инструмент проведения дистанционного имитационного 
эксперимента~// Вестник ИжГТУ.~--- Ижевск: ИжГТУ, 2008. №\,2. 
С.~103--108.

\label{end\stat}

\bibitem{8zh}
\Au{Ефимов И.\,Н., Жевнерчук Д.\,В., Николаев~А.\,В.}
Открытые виртуальные исследовательские пространства. Аналитический 
обзор.~--- Екатеринбург: Институт экономики УрО РАН, 2008. 83~с.
 \end{thebibliography}
}
}


\end{multicols}