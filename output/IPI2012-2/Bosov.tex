\def\stat{bosov}

\def\tit{ЗАДАЧИ АНАЛИЗА И ОПТИМИЗАЦИИ ДЛЯ~МОДЕЛИ ПОЛЬЗОВАТЕЛЬСКОЙ 
АКТИВНОСТИ.\\
ЧАСТЬ~3. ОПТИМИЗАЦИЯ ВНЕШНИХ РЕСУРСОВ}

\def\titkol{Задачи анализа и оптимизации для~модели пользовательской 
активности. Часть~3. Оптимизация внешних ресурсов}

\def\autkol{А.\,В.~Босов}
\def\aut{А.\,В.~Босов$^1$}

\titel{\tit}{\aut}{\autkol}{\titkol}

%{\renewcommand{\thefootnote}{\fnsymbol{footnote}}\footnotetext[1]
%{Работа выполнена при поддержке РФФИ (гранты 09-07-12098, 09-07-00212-а и
%09-07-00211-а) и Минобрнауки РФ (контракт №\,07.514.11.4001).}}


\renewcommand{\thefootnote}{\arabic{footnote}}
\footnotetext[1]{Институт проблем информатики Российской академии наук, AVBosov@ipiran.ru}

\vspace*{2pt}
  
  
\Abst{Статья завершает исследование модели описания активности пользователей, предложенной 
автором ранее, и основанных на ней задач оптимизации распределения вычислительных ресурсов. 
Сформулирована и решена задача квадратичной оптимизации распределения <<внешних>> 
ресурсов, используемых информационной системой. Предложены субоптимальные алгоритмы 
оптимизации.}

\vspace*{2pt}

\KW{информационная система; система управления базами данных; стохастическая система 
наблюдения; квадратичный критерий}

\vspace*{2pt}

\vskip 14pt plus 9pt minus 6pt

      \thispagestyle{headings}

      \begin{multicols}{2}

            \label{st\stat}


\section{Введение}
  
  Предложенная в работе~[1] математическая модель описания пользовательской 
активности рассмат\-ри\-ва\-лась в качестве источника для постановки задач оптимизации 
распределения вычислительных ресурсов, используемых некоторой информационной 
системой при обслуживании запросов пользователей. 

В~[2] стохастическая динамическая 
система наблюдения, описывающая эволюцию показателя\linebreak пользовательской активности 
(текущее чис\-ло пользователей~$x_t$) моделью случайного процесса с переключениями, 
порождаемыми значениями показателя, и линейными наблюдениями за ним\linebreak (чис\-ло 
выполненных команд~$y_t$), дополнена квадратичным функционалом качества, задающим 
штраф за использование программой <<внутренних>> вы\-чис\-ли\-тель\-ных ресурсов. 

Целевой 
функционал сформирован в результате анализа алгоритма работы программного 
обеспечения Информационного веб-пор\-та\-ла~[3, 4], <<внут\-рен\-ним>> ресурсом которого 
является пул, поддерживаемый с целью параллельного выполнения\linebreak поступающих запросов.
  
  Внутренний характер рассмотренного в~[2] вида ресурса объясняется тем, что управление 
им полностью подконтрольно программной системе, а выбор стратегии оптимизации никак 
не влияет на состояние внешней среды (активность пользователей, обслуживающие 
системы). В~результате получена оригинальная постановка задачи оптимизации: ни 
состояние, ни наблюдения не зависят непосредственно от выбранной стратегии, обратную 
связь обеспечивает только минимизируемый целевой функционал.
  
  В данной статье использована та же модель активности пользователей и аналогичный 
подход к оптимизации, но уже с целью управления распределением вычислительных 
ресурсов <<внешних>>, т.\,е.\ обслуживающих систем. Как отмечалось в~[2], предложить 
осмысленные постановки для таких задач существенно труднее, чем для <<внутренних>> 
ресурсов. К~последним можно отнести практически любой программный объект, 
создаваемый в процессе функционирования рассматриваемого программного обеспечения (а 
значит, и полностью им контролируемый). Для <<внешних>> ресурсов, которые любой 
программой, несомненно, используются во множестве, нужно иметь доступные средства 
управления их выделением извне, а возможностей такого рода известно немного.
  
  В процессе работы Информационного веб-пор\-та\-ла используется такой <<внешний>> 
ресурс, и поддерживается он применяемой системой управ\-ле\-ния базами данных (СУБД). 
Одним из реализуемых порталом сервисов является личный кабинет пользователя, 
поддерживаемый портальной под\-сис\-те\-мой представления в составе средств персонализации. 
Эффективность функционирования этой подсистемы хотя и не столь критична, как у пула 
запросов, но также допускает постановку задачи оптимизации, решение которой вносит свой 
вклад в повышение качества работы портала в целом.
  
  Функциональность, реализованная в связи с поддержкой личного кабинета, обеспечивает 
пользователя возможностью сохранения результатов выполнения пользовательской 
команды. Информация, отобранная по заданным критериям, может быть перенесена в 
индивидуальное хранилище, и в дальнейшем сохраненные ранее результаты могут быть 
проанализированы вновь в рамках интерфейса личного кабинета.
  
  Совокупность персональных хранилищ данных образует информационный источник 
(пользовательскую базу данных), поддерживаемый в составе собственного хранилища 
портала, в связи с чем возникают традиционные задачи его сопровождения: резервного 
копирования и восстановления, выделения дисковых ресурсов и сборки мусора и~т.\,п.\linebreak 
Предметом оптимизации здесь является процедура обслуживания дискового пространства, 
вы\-де\-ля\-емо\-го под размещение пользовательской базы данных. Накладные расходы, 
влияющие на функционирование портала в целом, в этой связи не\linebreak
 слишком велики, если 
размер пользовательской базы данных мал. Но с ростом этого размера, что рано или 
поздно неизбежно происходит, могут сказываться уже существенно.
  
  Отметим следующие причины, по которым в Информационном веб-пор\-та\-ле 
обслуживание пользовательской базы данных выполняется особым образом:
  \begin{itemize}
\item анализ потребностей потенциальных пользователей привел к отказу от применения 
традиционных принудительных ограничений на пространство, выделяемое под 
персональные данные конкретного пользователя, по причинам (а)~количественных 
характеристик пользовательской аудитории; (б)~неконтролируемого разнообразия 
функциональных потребностей; (в)~недопустимости принудительного вмешательства в 
ход выполняемой пользователем текущей работы из-за нехватки ресурсов;
\item единственным приемлемым способом осво\-бож\-де\-ния персональной базы данных от 
уста\-рев\-ших сведений признаны уведомительные сообщения, формируемые исходя из 
вре\-мен\-н$\acute{\mbox{ы}}$х меток, устанавливаемых на сохраненные данные;
\item традиционные для современных СУБД механизмы автоматического расширения 
дискового пространства при его нехватке на заданный процент от текущего размера 
неприемлемы: при малом размере базы данных операция расширения будет выполняться 
слишком часто, при большом~--- будет выделено неоправданно много места;
\item аналогичная причина заставила отказаться от использования механизма сжатия, 
вы\-пол\-ня\-емо\-го либо автоматически, либо по \mbox{команде} администратора,~--- как будет 
пояснено далее, портальная функциональность предполагает наличие определенного (но 
не слишком большого) свободного пространства в пользовательской базе данных в 
любой момент времени, сжатие же предполагает использование совместно с 
автоматическим расширением.
  \end{itemize}
  
  Цель обслуживания пользовательской базы данных~--- обеспечение компактного 
хранения уже отобран\-ных пользователями данных (а значит, и\linebreak эффективная индексация, и 
доступ) и поддержка небольшого, но приемлемого свободного объема, достаточного для 
удовлетворения потребностей пользователей в каждый момент времени.\linebreak Попутно также 
удается минимизировать деятельность администратора по сопровождению хранилищ 
личных кабинетов.

\section{Используемые обозначения}
  
  Далее в работе будут использованы следующие обозначения:
  \begin{description}
  \item[\,] $\overset{\Delta}{=}$~--- равенство по определению;
\item[\,] $\mathbf{M}[x]$ и $\mathbf{M}[x\vert \mathfrak{J}]$~--- 
соответственно безусловное 
математическое ожидание случайной величины~$x$ и условное математическое 
ожидание~$x$ относительно $\sigma$-ал\-геб\-ры~$\mathfrak{J}$;
\item[\,] $x^{\mathrm{T}}$~--- операция транспонирования вектора (мат\-ри\-цы)~$x$;
\item[\,] $\mathrm{col}\,(x_1,\ldots ,x_n)\overset{\Delta}{=}(x_1,\ldots 
,x_n)^{\mathrm{T}}$~--- век\-тор-стол\-бец с 
элементами $x_1,\ldots ,x_n$;
\item[\,] $\mathrm{row}\,(x_1,\ldots ,x_n) \overset{\Delta}{=}(x_1,\ldots ,x_n)$~--- век\-тор-
стро\-ка с элементами  $x_1,\ldots ,x_n$;
\item[\,] $\mathfrak{J}_t^{y}\overset{\Delta}{=}\sigma\{y_\tau,\tau\hm\leq t\}$~--- 
$\sigma$-ал\-геб\-ра, порожденная наблюдениями~$y_\tau$, $\tau\hm\leq t$;
\item[\,] $\overline{\psi}_x(x,t,j)$, $j\hm=0, 1,\ldots ,$~--- условная плот\-ность вероятности 
$x_{t+j}$ относительно $\sigma$-ал\-геб\-ры~$\mathfrak{J}_{t-1}^y$;
\item[\,] $\hat{\psi}_x(x,t)$~--- условная плотность вероятности~$x_t$ относительно 
$\mathfrak{J}_t^y$.
\end{description}

\section{Модель распределения <<внешних>> ресурсов портала}
  
  Для описания процесса изменения размера пользовательской базы данных воспользуемся 
той же моделью, что и в~[1, 2], т.\,е.\ будем считать, что показатель пользовательской 
активности~$x_t$ за интервал наблюдения $(t-1;t]$ описывается следующим разностным 
стохастическим уравнением:

\noindent
  \begin{multline}
  x_t=a\Theta(x_{t-1})x_{t-1}+q\Theta(x_{t-1})+{}\\
  {}+b\Theta(x_{t-1})v_t\,,\enskip t=1,2, \ldots\,,
  \label{e1bos}
  \end{multline}
предполагая, что область значений~$x_t$ разбита на непересекающиеся интервалы 
$\Delta_k$:

\noindent
\begin{gather*}
-\infty =a_1<a_2<\cdots <a_n<a_{n+1}=+\infty\,;\\
\Delta_k=(a_k,a_{k+1}],\ k=1,\ldots ,n-1;\ \Delta_n=(a_n,+\infty)
\end{gather*}
и текущий режим пользовательской активности задан индикаторной функцией~$\theta(x)$:
\begin{equation}
\left.
\begin{array}{rl}
\Theta(x) &=\mathrm{col}\left( I_{\Delta_1}(x),\ldots ,I_{\Delta_n}(x)\right)\,;\\[6pt]
I_{\Delta_k} (x)&= \begin{cases}
1\,, &\ \mbox{если}\ x\in \Delta_k\,;\\[6pt]
0\,, &\ \mbox{если}\ x\not\in \Delta_k\,;
\end{cases}\\[12pt]
a&=\mathrm{row}\left (a_1,\ldots , a_n\right)\,;\\[6pt]
q&=\mathrm{row}\left (q_1,\ldots  ,q_n\right)\,;\\[6pt]
b&=\mathrm{row}\left (b_1,\ldots ,b_n\right)\,.
\end{array}
\right\}
\label{e2bos}
\end{equation}
  
  В качестве наблюдений~$y_t$ будем использовать объем пользовательских данных, 
помещенных в хранилище в момент времени~$t$. Уравнение для~$y_t$ получим из 
следующих соображений. Заметим, что содержимое хранилища в момент~$t$ должно 
складываться из данных, внесенных пользователями до момента $t-1$ и не удаленных за 
интервал наблюдения $(t-1;t]$, и новых данных, пополнивших хранилище за последний 
интервал наблюдения. Будем предполагать, что часть сохраняемых от шага к шагу данных 
пропорциональна их размеру и составляет в среднем~$dy_t$, т.\,е.\ $d\cdot 100\%$, 
$0\hm<d\hm<1$ пользовательской информации, имеющейся в хранилище в момент $t-1$, 
остается в нем и в момент~$t$, а $(1-d)\cdot100\%$~--- удаляется. Также будем предполагать, что 
объем новых данных, пополняющих хранилище за время $(t-1;t]$, пропорционален числу 
активных пользователей и составляет в среднем~$cx_t$, где параметр~$c$ определяет средний 
объем информации, размещаемой в хранилище персональных данных одним пользователем 
за интервал наблюдения $(t-1;t]$. Таким образом, объем пользовательских данных в 
хранилище можно описывать следующим уравнением:

\noindent
    \begin{equation}
    y_t=dy_{t-1}+cx_t+\sigma w_t\,;\quad y_0=0\,,
    \label{e3bos}
    \end{equation}
где $w_t$~--- возмущение, моделирующее отклонения размера фактически сохраняемых и 
фактически добавляемых данных от заданных средних уровней; $\sigma$~--- среднее 
квадратическое отклонение этого возмущения. Далее предполагается, что $\{w_t\}$~--- 
стандартный дискретный белый шум второго порядка.
  
  Собственно же размер пользовательской базы данных~$z_t$ определяется только тем 
управляющим\linebreak\vspace*{-12pt}
\columnbreak

\noindent
 воздействием~$u_t$, которое будет применено непосредственно к 
пользовательской базе данных средствами администрирования СУБД, а~именно: в момент 
времени $t-1$ должно быть принято решение о расширении (уменьшении) выделенного под 
хранилище персональных данных дискового пространства на величину~$u_t$. Таким 
образом, выход~$z_t$ системы наблюдения~(\ref{e1bos})--(\ref{e3bos}) вычисляется в 
момент $t-1$ как сумма $z_{t-1}$ и управления~$u_t$, которое формируется на основании 
всех наблюдений, выполненных к моменту $t-1$:
  \begin{equation}
  z_t=z_{t-1}+u_t\,;\quad z_0=G_0\,.
  \label{e4bos}
  \end{equation}
  
  Целью оптимизации является определение подходящего размера пользовательской базы 
данных, позволяющего разместить в ней всю пользовательскую информацию, накопленную 
к моменту $t-1$, а также оставить свободное место некоторого объема~$G_t$, достаточного 
для размещения вновь поступающей за время $(t-1;t]$ пользовательской информации. Для 
достижения этого в целевую функцию следует включить следующие слагаемые:
  \begin{equation}
  \mathbf{M}\left[\left( z_t-G_t-y_t\right)^2+u_t^2\right]\,,
  \label{e5bos}
  \end{equation}
т.\,е.\ определять стратегию оптимизации, штрафуя за разницу между предполагаемым $(z_t-
G_t)$ и фактическим~$y_t$ объемом пользовательских данных в момент времени~$t$, а 
также за величину управ\-ля\-юще\-го воздействия~$u_t$. Отметим, что в~(\ref{e5bos}) не 
включено слагаемое, позволяющее учесть штраф за суммарный размер хранилища. 
Включение такого сла\-га\-емо\-го~$z_t^2$ неминуемо привело бы к ограничению на размер 
пользовательской базы, что в условиях портала, как отмечалось выше, неприемлемо. Однако 
нетрудно видеть, что полученный далее результат легко может быть распространен и на 
данный вид штрафа.
  
  Для придания~(\ref{e5bos}) окончательного вида предположим, что задан горизонт 
оптимизации~$N$ (например, сутки или неделя) и дополним выбранные слагаемые 
весовыми коэффициентами. Окончательно получаем целевую функцию следующего вида:
  \begin{multline}
  \mathbf{J}\left(u_0, \ldots , u_N\right) ={}\\
  {}=\sum\limits_{t=1}^N \mathbf{M}\left[ 
  S_t^{(1)}\left( z_t-G_t-y_t\right)^2+S_t^{(2)} u_t^2\right]\,.
  \label{e6bos}
  \end{multline}

\section{Формальная постановка и~решение задачи управления~размером~пула}
  
  Всюду далее будем предполагать, что $\{v_t\}$ из~(\ref{e1bos})~--- стандартный 
дискретный белый шум в узком смысле, сечения которого имеют плотность вероятности 
$\varphi_v(\cdot)$; $x_0$~--- случайная величина, имеющая плот\-ность 
  ве\-ро\-ят\-ности~$\psi_0(\cdot)$; $\{w_t\}$ из~(\ref{e3bos})~--- стандартный дискретный 
белый шум в\linebreak узком смыс\-ле, сечения которого имеют плотность 
ве\-ро\-ят\-ности~$\varphi_w(\cdot)$; $\{v_t\}$, $x_0$, $\{w_t\}$ независимы в совокупности; 
$\mathbf{M}\left[v_t^2\right]\hm<\infty$; $\mathbf{M}\left[x_0^2\right]\hm<\infty$; 
$\mathbf{M}\left[w_t^2\right]\hm<\infty$; параметры $b_k$, $k\hm=1, \ldots ,n$ и~$\sigma$ 
неотрицательны.
  
  Будем предполагать также, что параметры целевого функционала~(\ref{e6bos}) 
$S_t^{(1)}$, $S_t^{(2)}$, $G_t$~--- известные неотрицательные функции~$t$, класс 
допустимых управлений~$U_t$ содержит все $\mathfrak{J}^y_{t-1}$-из\-ме\-ри\-мые 
функции~$u_t:\ \mathbf{M}[u_t^2]\hm<\infty$. Отметим, что отсюда вытекает 
$\mathfrak{J}^y_{t-1}$-из\-ме\-ри\-мость процесса~$z_t$. Таким образом, целью 
оптимизации является поиск закона изменения~$u_t^*$ размера хранилища 
пользовательских данных, удовлетворяющего потребности в ресурсах, описываемых 
наблюдениями~$y_t$, с минимизацией затрат на управляющее воздействие на текущем шаге 
и на всех последующих шагах вплоть до заданного горизонта~$N$:
    \begin{multline}
  \mathrm{col} \left( u_0^*, \ldots ,u_N^*\right) ={}\\
    {}=\argmin\limits_{(u_0,\ldots , u_N)\in U_0\times \cdots\times U_N} 
  \mathbf{J}\left( u_0,\ldots ,u_N\right).
    \label{e7bos}
    \end{multline}
  
  \medskip
  
  \noindent
  \textbf{Теорема.} \textit{Пусть для целевого функционала}~(\ref{e6bos}) 
\textit{выполнено: $S_t^{(1}+S_t^{(2)}\hm>0$, $1\hm\leq t\hm\leq N$.
  Тогда решение~$u_t^*$ задачи оптимизации}~(\ref{e7bos}) \textit{существует и определяется 
соотношением:}
    \begin{multline}
  u_t^* =\fr{1}{R(t)}\left ( \sum\limits_{j=0}^{N-t}
  Q_j(t)\left(\overline{y}_{t+j,t-1}+G_{t+j}\right)-{}\right.\\
\left. \vphantom{\sum\limits_{j=0}^{N-t}} {}-P(t)z_{t-1}\right)\,.
  \label{e8bos}
  \end{multline}
  Здесь
  \begin{align*}
  R(t) &= S_t^{(2)}+P(t)\,,\ 1\leq t\leq N\,;\\
  P(t) &= S_t^{(1)}+L(t+1)\,,\ 1\leq t \leq N\,,
  \end{align*}
  где
  \begin{align*}
  L(t) &= L(t+1)+S_t^{(1)}-\fr{P^2(t)}{R(t)}\,,\ 1\leq t\leq N\,,\\
&  \hspace{37mm}L(N+1)=0\,;\\
  Q_j(t) &=Q_{j-1}(t+1)-Q_{j-1}(t+1)\fr{P(t+1)}{R(t+1)}\,;\\
  Q_0(t) &= S_t^{(1)}\,,\ 1\leq t\leq N\,,\ 0\leq j\leq N-t\,;
  \end{align*}
$\overline{y}_{t+j,t-1}$~--- оптимальные в среднем квадратическом прогнозы 
наблюдений~$y_{t+j}$ по наблюдениям~$y_\tau$, $\tau\hm\leq t-1$.
  \medskip
  
  \noindent
  Д\,о\,к\,а\,з\,а\,т\,е\,л\,ь\,с\,т\,в\,о\,.\ Для решения задачи оптимизации~(\ref{e7bos}) 
воспользуемся методом динамического программирования~\cite{6bos, 5bos}. Обозначим 
через
    \begin{multline*}
    B(t) \overset{\Delta}{=}{}\\
\hspace*{-1pt}  {}\overset{\Delta}{=}\!   \min\limits_{(u_t,\ldots ,u_N)\in U_t\times\cdots \times
  U_N} \sum\limits_{\tau=t}^N \mathbf{M}\left[ S_\tau^{(1)}\left( z_\tau-G_\tau-
y_\tau\right)^2+{}\right.\\
  \left.{}+ S_\tau^{(2)}u_\tau^2\vert \mathfrak{J}^y_{t-1}\right]
  \end{multline*}
функцию Беллмана. При $t\hm= N$ утверждение теоремы очевидно, так как из выражения
\begin{multline*}
B(N) ={}\\
{}=\!\!\!\!\!\min\limits_{u_N\in U_N}\mathbf{M}\left[ S_N^{(1)}\left( z_N-G_N-
y_N\right)^2+
S_N^{(2)} u_N^2\vert \mathfrak{J}^y_{N-1}\right]\hspace*{-1.01462pt}
\end{multline*}
после очевидных преобразований с учетом $\mathfrak{J}^y_{N-1}$-из\-ме\-ри\-мости 
функции~$u_N$, равенства $z_t\hm= z_{t-1}\hm+u_t$, а также обозначения 
$\overline{y}_{N,N-1}\hm=\mathbf{M}\left[ y_N\vert \mathfrak{J}^y_{N-1}\right]$ получается
\begin{multline*}
u_N^* =\argmin\limits_{u_N\in U_N}\left(\left( S_N^{(1)}+S_N^{(2)}\right) u_N^2-{}\right.\\
{}-2S_N^{(1)}\left( \overline{y}_{N,N-1}+G_N-z_N\right) u_N+{}\\
{}+ S_N^{(1)}\left( z_N^2-2z_N\left( \overline{y}_{N,N-1}+G_N\right) +{}\right.\\
\left.\left.{}+\mathbf{M}
\left[ \left( y_N+G_N\right)^2\vert \mathfrak{J}^y_{N-1}\right] \right) \right)\,,
\end{multline*}
откуда с учетом независимости последних двух слагаемых от~$u_N$ и положительности 
коэффициента при $u_N^2$ следует:
\begin{multline*}
u_N^*= \fr{S_N^{(1)}\left(\overline{y}_{N,N-1}+G_N\right) -S_N^{(1)} z_{N-
1}}{S_N^{(1)}+S_N^{(2)}} ={}\\
{}=\fr{1}{R(N)}\left( Q_0(N) \left( \overline{y}_{N,N-1} +G_N\right) -P(N) z_{N-1}\right)\,.
\end{multline*}
  
  Кроме того, получаем и выражение для функции Беллмана:
  \begin{multline*}
  B(N) = L(N) z^2_{N-1} -{}\\
  {}-2Q_1(N-1) z_{N-1}\left( \overline{y}_{N,N-1} +G_N\right) 
+\mathbf{M} \left[ A(N) \vert \mathfrak{J}^y_{N-1}\right],\hspace*{-10.4206pt}
  \end{multline*}
где обозначено:
\begin{multline*}
A(N) =S_N^{(1)} \left( y_N+G_N\right)^2 -{}\\
{}-\fr{1}{R(N)} \left( Q_0(N) \left( \overline{y}_{N,N-
1} +G_N\right)\right)^2\,.
\end{multline*}
  
  Предположим, что утверждение теоремы выполнено для $u^*_{t+1}$ и для функции 
Беллмана $B(t+1)$ имеет место следующее выражение:
  \begin{multline*}
  B(t+1) = L(t+1) z_t^2 -{}\\
{}-2z_t\sum\limits_{j=1}^{N-t} Q_j(t) \left( 
\overline{y}_{t+j,t}+G_{t+j}\right) +\mathrm{M}\left[ A(t+1)\vert \mathfrak{J}^y_t\right]\,,
\end{multline*}
где
\begin{multline*}
A(t+1) = A(t+2) +S_{t+1}^{(1)}\left( y_{t+1} +G_{t+1}\right)^2 -{}\\
{}-\fr{1}{R(t+1)}\left( 
\sum\limits_{j=1}^{N-t} Q_{j-1}\left(t+1\right)\left( \overline{y}_{t+j,t}+G_{t+j}\right) 
\right)^2\,.
  \end{multline*}
  Тогда уравнение Беллмана для $B(t)$ имеет сле\-ду\-ющий вид:
  \begin{multline*}
  B(t) = \min\limits_{u_t\in U_t} \mathbf{M}\left[ S_t^{(1)}\left( z_t-G_t-y_t\right)^2+
  S_t^{(2)} u_t^2+{}\right.\\
{}+L(t+1)z_t^2- 2z_t\sum\limits_{j=1}^{N-t} Q_j(t)\left( 
\overline{y}_{t+j,t}+G_{t+j}\right) +{}\\
  \left.{}+ \mathbf{M} \left[ A(t+1)\Big\vert \mathfrak{J}^y_t\right]\bigg \vert \mathfrak{J}^y_{t-1}
\vphantom{S_t^{(1)}\left( z_t-G_t-y_t\right)^2}\right]\,.
  \end{multline*}
    Преобразовав полученное уравнение с учетом $\mathfrak{J}^y_{t-1}$-из\-ме\-ри\-мости 
функции~$u_t$, формулы полного математического ожидания и равенств 
$\overline{y}_{t+j,t-1}\hm= \mathbf{M}\left[ \overline{y}_{t+j,t}\vert \mathfrak{J}^y_{t-
1}\right] \hm= \mathbf{M}\left[ y_{t+j}\vert \mathfrak{J}^y_{t-1}\right]$ и $z_t\hm= z_{t-
1}\hm+u_t$, запишем:
  \begin{multline*}
  B(t) =\min\limits_{u_t\in U_t}\left[ 
  \left( S_t^{(1)} +S_t^{(2)} +L(t+1)\right)u_t^2-{}\right.\\
  {}-
  2\left( S_t^{(1)} \left( \overline{y}_{t,t-1}+G_t\right) +{}\right.\\
   {}+\sum\limits_{j=1}^{N-t} Q_j(t)\left( 
\overline{y}_{t+j,t-1}+G_{t+j}\right) -{}\\
\left.{}-\left(
  S_t^{(1)}+L(t+1)\right)z_{t-1}\right) u_t+{}\\
 {}+
  \left( S_t^{(1)}+L(t+1)\right) z^2_{t-1} -{}\\
  {}-
  2\left( \vphantom{\sum\limits^{N-t}_{j=1}}
  S_t^{(1)} z_{t-1}\left( \overline{y}_{t,t-1}+G_t\right)+{}\right.\\
 \left.{}+
  z_{t-1}\sum\limits^{N-t}_{j=1} Q_j(t) \left( \overline{y}_{t+j,t-1}+G_{t+j}\right)\right) +{}\\
\left.  {}+
  \mathbf{M}\left[ S_t^{(1)}\left( y_t+G_t\right)^2+A(t+1)\Big\vert \mathfrak{J}^y_{t-
1}\right]\right]\,.
  \end{multline*}

Применяя в последнем выражении введенные в~(\ref{e8bos}) обозначения, получаем:

\noindent
\begin{multline*}
B(t) =\min\limits_{u_t\in U_t}\left[ \vphantom{S_t^{(1)}}
R(t) u_t^2 -{}\right.\\
{}-2\left( \sum\limits_{j=0}^{N-t} Q_j(t) \left( \overline{y}_{t+j,t-1}+G_{t+j}\right) -
P(t)z_{t-1}\right) u_t+{}\\
{}+ \left(
S_t^{(1)}+L(t+1)\right) z^2_{t-1}-{}\\
{}- 2z_{t-1} \sum\limits_{j=0}^{N-t} Q_j(t)
\left( \overline{y}_{t+j,t-1}+G_{t+j}\right) +{}\\
\left.{}+\mathbf{M}\left[
S_t^{(1)} \left( y_t+G_t\right)^2+A(t+1)\Big\vert \mathfrak{J}^y_{t-1}\right]\right]\,.
\end{multline*}
    Поскольку в полученном соотношении три последних слагаемых не зависят от~$u_t$, то в 
предположении положительности~$R(t)$ получается доказываемое 
соотношение~(\ref{e8bos}) для~$u_t^*$. Кроме того, \mbox{подстановкой} $u_t^*$ подтверждается 
справедливость индуктивного предположения относительно функции Беллмана.
  
  Для завершения доказательства остается показать, что $R(t)\hm>0$. Для этого достаточно 
показать, что $L(t+1)\geq 0$. Поскольку $L(N+1)\hm=0$, то требуемое неравенство следует 
из выражения
  $$
  L(t+1)\! =\!\left( L(t+2)+S_{t+1}^{(1)}\right) -\fr{\left( L(t+1)+S^{(1)}_{t+1}\right)^2}
  {L(t+2)+S_{t+1}^{(1)}+S^{(2)}_{t+1}}
  $$
и неотрицательности $S_{t+1}^{(1)}+S_{t+1}^{(2)}$. Теорема доказана.
  
  \medskip
  
  \noindent
  \textbf{Замечание.} В полученном утверждении используются оптимальные в среднем 
квадратическом прогнозы $\overline{y}_{t+j,t-1}$ наблюдений~$y_{t+j}$ по наблюдениям 
$y_\tau$, $\tau\hm\leq t-1$, $j\hm=0,1,\ldots$ В~теореме~2 работы~[1] получены аналогичные 
прогнозы в случае $d\hm=0$. Используя их, нетрудно записать соответствующие 
соотношения для $\overline{y}_{t+j,t-1}$:
  \begin{align*}
  \overline{y}_{t+j,t-1} &= d\overline{y}_{t+j-1,t-1}+{}\\
  &\hspace*{-6mm}{}+\sum\limits_{k=1}^n \int\limits_{\Delta_k}
  c(a_k\xi+q_k)\overline{\psi}_x (\xi,t,j)\,d\xi\,,\enskip j=1,2,\ldots;\\
  \overline{y}_{t,t-1} &= dy_{t-1}+\sum\limits_{k=1}^n \int\limits_{\Delta_k} c( a_k\xi + q_k) 
\hat{\psi}_x (\xi,t-1)\,d\xi\,,
  \end{align*}
где прогнозирующие плотности вероятности определяются соотношениями:

\noindent
\begin{multline*}
\overline{\psi}_x(x,t,j) ={}\\
{}=\sum\limits_{k=1}^n \fr{1}{b_k} \int\limits_{\Delta_k} 
\overline{\psi}_x (\xi, t, j-1)\varphi_v\left( \fr{x-a_k\xi -q_k}{b_k}\right)\,d\xi\,,\\
j=1,2,\ldots;
\end{multline*}
\begin{multline*}
\overline{\psi}_x(x,t,0)\overset{\Delta}{=}{}\\
{}\overset{\Delta}{=} \sum\limits_{k=1}^n \fr{1}{b_k}
\int\limits_{\Delta_k}\hat{\psi}_x(\xi,t-1)\varphi_v\left( \fr{x-a_k\xi-q_k}{b_k}\right)\,d\xi\,;
\end{multline*}

\vspace*{-12pt}

\noindent
\begin{multline*}
\hat{\psi}_x(x,t) = \left(\vphantom{\int\limits_{\Delta_k}}
\varphi_w\left(\fr{y_t-dy_{t-1}-cx}{\sigma}\right)\times\right.\\
\left.{}\times
\sum\limits_{k=1}^n 
\fr{1}{b_k}\int\limits_{\Delta_k} \hat{\psi}_x(\xi,t-1)\varphi_v
\left(\fr{x-a_k\xi-q_k}{b_t^k}\right)\,d\xi\right)\!\Bigg/\\
\Bigg/\left(\sum\limits_{k=1}^n \fr{1}{b_t^k}\int\limits_{R^1} \varphi_w \left(\fr{y_t-dy_{t-1}-
cx}{\sigma}\right)\times{}\right.\\
\left.{}\times\int\limits_{\Delta_k}\hat{\psi}_x(\xi,t-1)\varphi_v
\left(\fr{x-a_k\xi-q_k}{b_k}\right)\,d\xi dx\right)
\,.
\end{multline*}


\section{Субоптимальные управления}
  
  Основной проблемой применения оптимальной стратегии оптимизации пользовательского 
хранилища~(\ref{e8bos}), как и определения размера пула в~[2], является определение 
горизонта~$N$. При этом надо учитывать, что выбор больших значений~$N$ приводит к 
необходимости выполнения вычислительно затратных расчетов большого числа прогнозов, а 
использование малых~$N$ лишает задачу характерных динамических свойств. Преодолевать 
указанную сложность предлагается, как и в~[2], за счет практического применения 
субоптимальной стратегии оптимизации, основанной на принципе локально-оп\-ти\-маль\-но\-го 
(адаптивного) управления~\cite{7bos}. Выполняя локальную оптимизацию целевой 
функции~(\ref{e6bos}), определим в качестве субоптимального решения функцию~$u_t^L$, 
доставляющую минимум функционалу
  \begin{multline*}
  \mathbf{J}_t(u_t)=\mathbf{M}\left[ S_t^{(1)}\left( z_t-G_t-y_t\right)^2 +S_t^{(2)} u_t^2+{}\right.\\
\left.  {}+
  S_{t+1}^{(1)} \left( z_{t+1}-G_{t+1}-y_{t+1}\right)^2 +S_{t+1}^{(2)} u_{t+1}^2\right]\,.
  \end{multline*}
  
  Таким образом, для локально-оптимального решения рассматриваемой задачи 
оптимизации предлагается двухшаговый вариант целевой функции вида~(\ref{e6bos}), 
обновляемый на каждом следующем шаге алгоритма. В~целевую 
функцию~$\mathbf{J}_t(u_t)$, как легко видеть, включены штрафы за ошибку в 
определении размера пользовательского хранилища и за его размер на текущем и 
следующем шаге.
  
  Требующееся выражение для функции $u_t^L\hm=\argmin\limits_{u_t\in U_t} 
\mathbf{J}_t(u_t)$ получаем непосредственно как частный случай~(\ref{e8bos}):

\noindent
  \begin{multline}
u_t^L ={}\\
{}= \fr{1}{S_t^{(1)}+S_t^{(2)}+S_{t+1}^{(1)}-
\left(S_{t+1}^{(1)}\right)^2\!\Big/\!\left(S_{t+1}^{(1)}+S_{t+1}^{(2)}\right)}\times{}\\
{}\times \left( S_t^{(1)}\left( \overline{y}_{t,t-
1}+G_t\right) +
  \left( S_{t+1}^{(1)}-\fr{\left(S_{t+1}^{(1)}\right)^2}{S_{t+1}^{(1)}+S_{t+1}^{(2)}}\right)\times{}\right.\\
  {}\times  \left( 
\overline{y}_{t+1,t-1}+G_{t+1}\right)-{}\\
\left.{}-
  \left( S_t^{(1)}+S_{t+1}^{(1)}-\fr{\left(S_{t+1}^{(1)}\right)^2}{ S_{t+1}^{(1)}+S_{t+1}^{(2)}}\right) 
z_{t-1}\right)\,,
  \label{e9bos}
  \end{multline}
  
  Наконец, как и в~[2], будем использовать самый простой вариант возможного решения 
рассматриваемой задачи оптимизации~--- оптимальную программную стратегию~$u_t^P$ в 
виде:
  \begin{equation}
  u_t^P=\mathbf{M}\left[ u_t^*\right]\,.
  \label{e10bos}
  \end{equation}

\section{Результаты численных экспериментов}
  
  Для сравнения предложенных алгоритмов оптимизации использован незначительно 
измененный модельный пример из~[1]. Заданы три интервала $\Delta_1\hm=(-\infty;3]$, 
$\Delta_2\hm= (3;7]$, $\Delta_3\hm=(7;+\infty)$ и сле\-ду\-ющие параметры 
уравнения~(\ref{e1bos}):


  
  \begin{center}
  \begin{tabular}{|c|c|c|c|c|c|c|c|c|}
  \hline
$a_1$&$a_2$&$a_3$&$q_1$&$q_2$&$q_3$&$b_1$&$b_2$&$b_3$\\
\hline
0,3&0,4&0,7&1,4&3,0&3,0&0,9&1,5&2,5\\
\hline
\end{tabular}
\end{center}

  
  Параметры наблюдений~(\ref{e3bos}): $c\hm=2{,}5$, $d\hm=0{,}5$, $\sigma\hm=1{,}0$. 
Распределения всех возмущений~--- стандартные гауссовские, распределение начального 
условия~$x_0$ также предполагалось гауссовским со средним и дисперсией, равными 
соответствующим моментам предельного распределения (подробнее см.~[1]).
  
  Расчеты проводились для 10~шагов траектории: $t\hm=1, 2, \ldots , 10$, для вычисления 
значения целевой функции использовался пучок из 10\,000~траекторий. Параметры целевой 
функции~(\ref{e6bos}) выбраны следующим образом:
  \begin{gather*}
  S_t^{(1)}=S_t^{(2)} =0{,}1\,,\enskip 1\leq t\leq 10\,;\\
  G_0=0{,}0\,,\enskip G_1=G_2=G_3=0{,}1\,;\\
  G_4=G_5=G_6=10{,}0\,;\\
  G_7=G_8=G_9=G_{10}=40{,}0\,.
  \end{gather*}
  
  Для сравнения качества стратегий оптимизации~(\ref{e8bos})--(\ref{e10bos}) 
  вычислялось значение целевой функции в каждый момент~$t$, т.\,е.\ 
$\mathbf{J}(u_0, \ldots , u_t)$.

  
  \pagebreak

\end{multicols}

\begin{figure} %fig1
\vspace*{1pt}
 \begin{center}
 \mbox{%
 \epsfxsize=157.475mm
 \epsfbox{bos-1.eps}
 }
 \end{center}
 \vspace*{-9pt}
\Caption{Характерные траектории: \textit{1}~--- $x_t$; \textit{2}~--- $y_t$;
\textit{3}~--- $u^*_t$; \textit{4}~--- $u_t^{L1}$; \textit{5}~--- $u_t^{L2}$; 
\textit{6}~--- $u_t^P$}
%\end{figure}
%\begin{figure*} %fig2
\vspace*{12pt}
 \begin{center}
 \mbox{%
 \epsfxsize=157.576mm
 \epsfbox{bos-2.eps}
 }
 \end{center}
 \vspace*{-9pt}
\Caption{Оценки целевых функций: \textit{1}~--- $\hat{J}_1$; \textit{2}~---
$\hat{J}_2$; \textit{3}~--- $\hat{J}_3$; \textit{4}~--- $\hat{J}_4$}
\vspace*{12pt}
\end{figure}


\begin{multicols}{2}

  Локально-оптимальная стратегия вычислялась двумя способами: 
  \begin{itemize} %[(1)]
  \item[(\textit{а})] функция~$u_t^{L1}$ 
минимизировала $\mathbf{J}_t(u_t)$, используя указанные выше значения $S_t^{(1)}$, 
$S_t^{(2)}$, $S_{t+1}^{(1)}$, $S_{t+1}^{(2)}$; 
\item[(\textit{б})] функция~$u_t^{L2}$ 
минимизировала~$\mathbf{J}_t(u_t)$ для таких же $S_t^{(1)}$, $S_t^{(2)}$ и 
$S_{t+1}^{(1)}\hm=S_{t+1}^{(2)}\hm=0$.
\end{itemize}

%\columnbreak

Таким образом, стратегию~$u_t^{L1}$ можно 
назвать ло\-каль\-но-оп\-ти\-маль\-ной двухшаговой, стратегию~$u_t^{L2}$~--- 
  ло\-каль\-но-опти\-маль\-ной одношаговой. 
  
  Результаты расчетов приведены на рис.~1 и~2. 
Рисунок~1 иллюстрирует характерные траектории компонентов системы 
наблюдения~(\ref{e1bos})--(\ref{e3bos}) и функций $u_t^*$, $u_t^{L1}$, $u_t^{L2}$, $u_t^P$, 
на рис.~2 приведены значения целевых функций $\hat{J}_1$ (для~$u_t^*$), $\hat{J}_2$ 
(для~$u_t^{L1}$), $\hat{J}_3$ (для $u_t^{L2}$) и $\hat{J}_4$ (для~$u_t^P$). На 
рис.~2 приведены два графика: на первом изображены кривые для трех функций $\hat{J}_1$, 
$\hat{J}_2$ и~$\hat{J}_3$, на втором~--- все четыре целевых функционала. Это сделано для 
того, чтобы позволить качественно оценить разницу оптимальной стратегии в сравнении с 
локально-оптимальными, которая становится несущественной в масштабе целевой 
функции~$\hat{J}_4$.


  Из приведенных результатов видно, что значения целевой функции, достигаемые при 
использовании оптимальной стратегии оптимизации~$u_t^*$ и ло\-каль\-но-опти\-маль\-ной 
двухшаговой~$u_t^{L1}$, отличаются несущественно. Их преимущество в сравнении с 
локально-оптимальной одношаговой стратегией~$u_t^{L2}$ достигает 25\%--30\%. 
Преимущество же $u_t^*$ в сравнении с $u_t^{L1}$ составляет порядка 5\%--7\% к моменту 
достижения горизонта оптимизации~$N$. При этом очевидно, что вычислительные 
труд\-ности, возникающие в связи с необходимостью расчета большого числа прогнозов, 
весьма велики и можно\linebreak считать, таким образом, что наиболее целесообразно использование 
локально-оптимальной двухшаговой стратегии оптимизации размера персонального 
пользовательского хранилища. Также \mbox{нужно} отметить, что программная стратегия~$u_t^P$ 
существенно проигрывает оптимальным.

\section{Заключение}
  
  В статье завершено исследование модели пользовательской активности, предложенной 
в~[1]. Стохастическая динамическая система наблюдения~(\ref{e1bos})--(\ref{e3bos}), 
описывающая эволюцию числа пользователей, взаимодействующих с некоторой 
информационной системой, и косвенные наблюдения за ними~--- объем персональной 
информации, размещенной в пользовательском хранилище (личном кабинете), дополнено 
уравнением выхода~(\ref{e4bos}), описывающим размер базы данных, используемой для 
размещения пользовательской информации. В~соответствии с предложенной в~[2] 
терминологией используемый таким образом вычислительный ресурс назван <<внешним>>, 
так как его обслуживанием занимается <<внешняя>> обслуживающая система~--- СУБД.
  
  Основная решенная задача состоит в оптимизации в процессе работы программы размера 
пользовательской базы данных на основе квадратичного\linebreak критерия качества, позволяющего 
учесть пользовательские потребности и издержки, опре\-де\-ля\-емые\linebreak размером базы данных. 
В~целом предложенный\linebreak подход аналогичен использованному в~[2] и соответствует 
классической задаче динамического управ\-ле\-ния по квадратичному критерию качества. 
Рассмотренная постановка, однако, имеет существенное отличие: управляющее воздействие 
не влияет на фазовый процесс (текущее число пользователей), а входит аддитивно в 
уравнение наблюдений. Такой результат вполне согласуется с физическим смыслом 
рассматриваемой задачи~--- наилучшим образом распределять ресурсы <<внешней>> 
обслуживающей системы, состояние которой является одним из факторов состояния 
внешней среды.

{\small\frenchspacing
{%\baselineskip=10.8pt
\addcontentsline{toc}{section}{Литература}
\begin{thebibliography}{9}

\bibitem{1bos}
\Au{Босов А.\,В.}
Задачи анализа и оптимизации для модели пользовательской активности. Часть~1. Анализ и 
прогнозирование~// Информатика и её применения, 2011. Т.~5. Вып.~4. С.~40--52.

\bibitem{2bos}
\Au{Босов А.\,В.}
Задачи анализа и оптимизации для модели пользовательской активности. Часть~2. 
Оптимизация внутренних ресурсов~// Информатика и её применения, 2012. Т.~6. Вып.~1. 
С.~18--25.

\bibitem{3bos}
Информационный веб-пор\-тал. Свидетельство об официальной регистрации программы для 
ЭВМ №\,2005612992. Зарегистрировано в Реестре программ для ЭВМ 18.11.2005.

\bibitem{4bos}
\Au{Босов А.\,В.}
Моделирование и оптимизация процессов функционирования Информационного 
web-пор\-та\-ла~// Программирование, 2009. №\,6. С.~53--66.


\bibitem{6bos}
\Au{Флеминг У., Ришел Р.}
Оптимальное управление детерминированными и стохастическими системами.~--- М.: Мир, 
1978.

\bibitem{5bos}
\Au{Бертсекас Д., Шрив С.}
Стохастическое оптимальное управление.~--- М.: Наука, 1985.

\label{end\stat}

\bibitem{7bos}
\Au{Коган М.\,М., Неймарк Ю.\,И.}
Адаптивное ло\-каль\-но-опти\-маль\-ное управ\-ле\-ние~// Автоматика и телемеханика, 1987. 
№\,8. С.~126--136.

  
 \end{thebibliography}
}
}


\end{multicols}