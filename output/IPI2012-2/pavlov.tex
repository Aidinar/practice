\def\stat{pavlov}

\def\tit{РАСЧЕТ И ОПТИМИЗАЦИЯ НЕКОТОРЫХ ХАРАКТЕРИСТИК 
ДЛЯ~МОДЕЛИ ВЫЧИСЛИТЕЛЬНОГО КОМПЛЕКСА}

\def\titkol{Расчет и оптимизация некоторых характеристик 
для модели вычислительного комплекса}

\def\autkol{И.\,В.~Павлов}
\def\aut{И.\,В.~Павлов$^1$}

\titel{\tit}{\aut}{\autkol}{\titkol}

%{\renewcommand{\thefootnote}{\fnsymbol{footnote}}\footnotetext[1]
%{Работа выполнена при поддержке РФФИ (гранты 09-07-12098, 09-07-00212-а и
%09-07-00211-а) и Минобрнауки РФ (контракт №\,07.514.11.4001).}}


\renewcommand{\thefootnote}{\arabic{footnote}}
\footnotetext[1]{Московский государственный технический университет им.\ Н.\,Э.~Баумана, 
ipavlov@bmstu.ru}

\Abst{Рассматривается проблема выбора оптимального размера пакетов при обработке 
информационных задач большого объема для модели вычислительного комплекса с 
учетом возможных отказов или сбоев элементов в процессе решения задачи. Получено 
приближенное асимптотическое решение данной проблемы для случая высоконадежных 
элементов и малого времени пересылки (загрузки) пакетов.}

\KW{оптимальный размер пакета; надежность; интенсивность отказов; время пересылки 
пакетов} 

\vskip 14pt plus 9pt minus 6pt

      \thispagestyle{headings}

      \begin{multicols}{2}

            \label{st\stat}

\section{Введение}

     Пусть имеется система, включающая в себя $l$ основных 
вычислительных элементов. В~систему поступают <<задания>>, каждое из 
которых состоит из некоторого (вообще говоря, случайного) числа   
<<элементарных задач>>, каждая из которых может выполняться 
(обрабатываться) независимо от остальных на любом из этих элементов. Для 
выполнения очередного задания, поступившего в систему, необходимо 
выполнить все составляющие его элементарные задачи. При этом в процессе 
выполнения задание разбивается на некоторое количество $n$  блоков 
(<<пакетов>>) элементарных задач равного объема $\upsilon\hm=L/n$, 
$n\hm\in N$, где $N$~--- множество допустимых значений  (например, 
$N$~---  некоторое подмножество целочисленных значений, кратных~2 
и~т.\,п.). Время~$h$ выполнения одной элементарной задачи на любом из 
элементов далее будем считать равным единице: $h\hm=1$. Соответственно, 
время выполнения одного пакета объемом~$\upsilon$ на любом из элементов 
будет численно совпадать с величиной~$\upsilon$. 
{\looseness=1

}
     
     Выполнение задания происходит путем пересылки пакетов на рабочие 
элементы и дальнейшей их обработки на этих элементах. Время пересылки 
(загрузки) пакета на элемент равно величине $\tau\hm>0$, не зависит от 
размера пакета~$\upsilon$ и от состояния других элементов. Обработка пакета 
после его загрузки на данном элементе занимает время~$\tau$ и происходит 
независимо от состояния других элементов. После завершения обработки 
очередного пакета на том или ином элементе снова происходит его загрузка в 
течение времени~$\tau$ следующим пакетом (из общей очереди всех пакетов 
данного задания) независимо от состояния (работы или загрузки) остальных 
элементов и~т.\,д. Задание считается выполненным после выполнения 
(обработки) всех составляющих его пакетов. Близкие по смыслу модели и 
процессы рассматривались ранее в~[1--5].
     
     В процессе работы любой из элементов может отказывать с постоянной 
(не зависящей от времени) функцией интенсивности отказов 
$\lambda(t)\hm\equiv \lambda$~[6, 7]. Заметим, что более близким к 
реальности было бы предположение о монотонном возрастании 
(неубывании) $\lambda(t)$ по времени. Поэтому фактически здесь 
предполагается, что, по крайней мере в течение времени выполнения одного 
задания, функция интенсивности отказов~$\lambda(t)$ меняется 
незначительно и может считаться приближенно постоянной. Такое 
допущение является естественным, по крайней мере в случае высокой 
надежности элементов, когда вероятность отказа элемента за время 
выполнения в системе одного задания достаточно мала. В~указанных 
допущениях время безотказной работы элемента имеет экспоненциальное 
распределение с функцией надежности $P(t)\hm=e^{-\lambda t}$, а вероятность 
отказа элемента за время~$h$ выполнения одной элементарной задачи равна 
величине $\lambda h\hm+ o(\lambda h)$.
     
     Одной из существенных проблем, возникающих в данной ситуации, 
является выбор оптимального размера пакета~$\upsilon$ с учетом 
возможности отказов (сбоев) элементов при выполнении задания.

\section{Модель со сбоями элементов}

     Рассмотрим случай, когда возможные отказы элементов в системе 
имеют характер <<сбоев>>. Другими словами, в результате отказа (сбоя) 
элемент сам по себе не выходит из строя и продолжает работать, но 
находящийся на нем в момент сбоя пакет считается невыполненным и после 
завершения его обработки снова ставится в очередь необработанных пакетов 
и должен быть полностью обработан заново на этом же или любом другом 
элементе.
     
     Рассмотрим сначала более простой частный случай, когда число 
элементов $l\hm=1$. Обозначим через $p\hm=\exp (-\lambda \upsilon)$ 
вероятность обработки пакета объемом~$\upsilon$ без сбоев и $q\hm=1-p$. 
Время~$\eta$ выполнения всего задания объемом~$L$ имеет вид:
     \begin{equation}
     \eta=(\upsilon+\tau) v\,,
     \label{e1p}
     \end{equation}
где $v$~--- момент (номер шага) первого достижения $n$ <<успехов>> в 
классической схеме независимых ис\-пытаний Бернулли при вероятности 
<<успеха>> (на\linebreak
одном шаге) $p\hm=\exp\left( -\lambda \upsilon\right)$. Задача 
выбора оп\-тимального размера пакета~$\upsilon$ далее сводится к 
минимизации математического ожидания E$\eta$ по параметру~$\upsilon$, 
или, учитывая равенство $\upsilon\hm= L/n$, к\linebreak
 минимизации~E$\eta$ по 
переменной $n\hm\in N$, где $n$~---  чис\-ло пакетов, на которое разбивается 
задание. Случайная величина~$v$ имеет распределение Пас\-каля 
\begin{equation}
P\left( v=m\right) = C_{m-1}^{n-1} p^n q^{m-n}\,,\quad m=n, n+1, \ldots ,
\label{e2p}
\end{equation}
с математическим ожиданием E$v=n/p$, откуда с учетом~(\ref{e1p}) следует, 
что выбор оптимального размера пакета сводится к задаче: найти
\begin{equation}
\min \left( L+n\tau\right)\exp\left( \fr{\lambda L}{n}\right)
\label{e3p}
\end{equation}
по $n\in N$. Далее оптимальный размер пакета~$\tilde{\upsilon}$ 
находится по формуле $\tilde{\upsilon} =L/\tilde{n}$, где $\tilde{n}\hm\in 
N$~--- решение задачи~(\ref{e3p}). 
     
     Оптимизационная задача~(\ref{e3p}) является цело\-чис\-лен\-ной, 
поскольку множество $N$ допустимых значений $n$ содержит только 
целочисленные точки. Введем также дополнительную <<непрерывную>> 
задачу: найти
     \begin{equation}
     \min\left( L+n\tau\right) \exp \left( \fr{\lambda L}{n}\right)
     \label{e4p}
     \end{equation}
по всем (не только целочисленным) значениям $n\hm\geq 1$. Далее 
оптимальный размер пакета~$\upsilon^*$ (без ограничения целочисленности 
$n\hm\in N$) находится как $\upsilon^*=L/n^*$, где $n^*$~--- решение 
задачи~(\ref{e4p}).
     
Теорема~1 дает точное решение оптимизационных 
задач~(\ref{e3p}) и~(\ref{e4p}). Теорема~2 дает асимптотическое выражение 
для оптимального размера пакета~$\upsilon^*$.
     
     \medskip
     
     \noindent
     \textbf{Теорема 1.} \textit{Пусть $\lambda\hm>0$, $\tau\hm>0$ и 
выполняется неравенство}
     \begin{equation}
     \tau\leq \lambda L^2\,.
     \label{e5p}
     \end{equation}
\textit{Тогда минимум}~(\ref{e4p}) \textit{достигается в единственной \mbox{точке}}
$$
n^*=L\sqrt{\fr{\lambda}{\tau}}\left[  
\sqrt{1+\fr{\lambda\tau}{4}}+\fr{\sqrt{\lambda\tau}}{2}\right]\,.
$$
\textit{Минимум}~(\ref{e3p}) \textit{достигается в одной из двух ближайших 
(слева или справа) к точке~$n^*$ целочисленных точек $n\hm\in N$.} 

     \smallskip
     
     \noindent
     Д\,о\,к\,а\,з\,а\,т\,е\,л\,ь\,с\,т\,в\,о\,.\ Введем функцию
     \begin{equation}
     f(n) =\left( L+n\tau\right) \exp\left( \fr{\lambda L}{n}\right)
     \label{e6p}
     \end{equation}
от непрерывного аргумента $n\hm\geq 1$. Нетрудно показать, что знак 
производной этой функции совпадает со знаком многочлена 
$Q(n)\hm=n^2\hm-\lambda L n -\lambda L^2/\tau$, который имеет при 
$n\hm\geq 1$ единственный корень в точке $n\hm=n^*$ и для которого 
справедливы неравенства:
\begin{align*}
Q(n)<0 &\ \ \mbox{при}\ \ 1\leq n\leq n^*\,;\\
Q(n)>0 &\  \ \mbox{при}\ \ n>n^*\,,
\end{align*}
если выполняется условие~(\ref{e5p}), откуда далее и следует теорема~1. 
Теорема доказана.

\smallskip

     \noindent
     \textbf{Теорема~2.} \textit{Пусть $\lambda\hm>0$, $\tau\hm>0$, 
$\tau\hm\leq \lambda L^2$ и $\lambda\tau\hm\rightarrow 0$. Тогда} 

\noindent
     \begin{equation}
     \upsilon^*=\sqrt{\fr{\tau}{\lambda}}\left[ 1+o(1)\right]\,.
     \label{e7p}
     \end{equation}
     
     \smallskip
     
     \noindent
     Д\,о\,к\,а\,з\,а\,т\,е\,л\,ь\,с\,т\,в\,о\ следует из теоремы~1 и равенства 
$\upsilon^*\hm=L/n^*$. 
     
     \medskip
     
     Из~(\ref{e7p}) далее следует приближенная формула для оптимального 
размера пакета при $\lambda\tau\hm\ll 1$:
     $$
     \upsilon^*\cong \sqrt{\tau\theta}\,,
     $$
где $\theta=1/\lambda$~--- математическое ожидание \mbox{времени} безотказной 
работы (средний ресурс) элемента. Другими словами, оптимальный размер 
пакета~$\upsilon^*$ приближенно равен среднему геометрическому между 
временем пересылки (загрузки)~$\tau$ и средним ресурсом элемента~$\theta$ 
(при условии $\lambda\tau\hm\ll 1$). Существенно, что оптимальное 
значение~$\upsilon^*$ не зависит от размера всего задания~$L$, который, 
вообще говоря, может быть неизвестным и случайным.
     
     Рассмотрим далее общий случай $l\hm\geq 1$ элементов.
Для рассматриваемой модели время выполнения задания

\noindent
     \begin{equation}
     \eta=\left(\upsilon+\tau\right) \left(\fr{v}{l}\right)^+\,,
     \label{e8p}
     \end{equation}
где $z^+$~---  величина~$z$, округленная вверх до ближайшего целого. 
Задача сводится к вычислению
\begin{equation}
\min E\eta
\label{e9p}
\end{equation}
по $n\in N$, после чего оптимальный размер пакета~$\tilde{\upsilon}$ находится 
по формуле $\tilde\upsilon\hm=L/\tilde{n}$, где $\tilde{n}\hm\in N$~--- решение 
задачи~(\ref{e9p}). 

\pagebreak
     
     В соответствии с~(\ref{e2p}) и (\ref{e8p})
     $$
     E\eta =\left( \fr{L}{n}+\tau\right) \sum\limits_{m=n}^\infty 
     \left (\fr{m}{l}\right)^+ C_{m-1}^{n-
1} p^n q^{m-n}\,,
     $$
откуда, учитывая, что $m C_{m-1}^{n-1}=nC_m^n$,
\begin{multline}
E\eta = \left( \fr{L}{n}+\tau\right) \sum\limits_{m=n}^\infty 
\left (\fr{m}{l}\right)^+ \fr{n}{m}\,C_m^n 
p^n q^{m-n}={}\\
{}=\left( \fr{1}{l}\right) \left( L+n\tau\right) p^n \sum\limits_{m=n}^\infty 
\fr{(m/l)^+}{m/l}\,C_m^n q^{m-n}\,.
\label{e10p}
\end{multline}

В соответствии с~(\ref{e2p}) 
\begin{equation}
E v = \!\sum\limits_{m=n}^\infty m C_{m-1}^{n-1} p^n q^{m-n} =n p^n\! 
\sum\limits_{m=n}^\infty C_m^n q^{m-n}.
\label{e11p}
\end{equation}
С другой стороны, случайная величина~$v$ является суммой~$n$ 
независимых, одинаково распределенных случайных величин, каждая из 
которых имеет геометрическое распределение с параметром~$p$ и 
математическим ожиданием $1/p$. Соответственно, $Ev \hm= n/p$, откуда с 
учетом~(\ref{e11p}) следует 
\begin{equation}
\sum\limits_{m=n}^\infty C_m^n q^{m-n} =\fr{1}{p^{n+1}}\,.
\label{e12p}
\end{equation}
Из~(\ref{e10p}) и~(\ref{e12p}) следует
$$
E\eta = \fr{1}{l}\left( L+n\tau\right) p^n \sum\limits_{m=n}^\infty \left[ 
1+\fr{(m/l)^\prime}{m/l}\right] C_m^n q^{m-n}\,,
$$
где $z^\prime=z^+-z$, откуда
\begin{equation}
E\eta =\fr{1}{l}\left(L+n\tau\right)\exp \left( \fr{\lambda L}{n}\right) 
\left(1+\delta_l(n)\right)\,,
\label{e13p}
\end{equation}
где 
\begin{equation}
\delta_l(n)=\sum\limits_{m=n}^\infty \alpha_{nm} \fr{(m/l)^\prime}{m/l}\,,
\label{e14p}
\end{equation}
где коэффициенты $\alpha_{nm} =p^{n+1} C_m^n q^{m-n}$, $p\hm= e^{-
\lambda L/n}$, $q\hm=1\hm-p$. При этом в соответствии с~(\ref{e12p}) 
\begin{equation}
\sum\limits_{m=n}^\infty \alpha_{nm}=1\,.
\label{e15p}
\end{equation}
Из~(\ref{e14p}) и (\ref{e15p}) видно, что 
\begin{equation}
0<\delta_l(n)<\fr{l}{n}\,.
\label{e16p}
\end{equation}
     
     Целевая функция~(\ref{e13p}) для общего случая $l\hm\geq 1$ 
совпадает с целевой функцией в~(\ref{e3p}) для случая $l\hm=1$ с точностью 
до множителя $(1/l)\left[ 1+\delta_l(n)\right]$, откуда с учетом~(\ref{e16p}) 
видно, что полученное выше решение для случая $l\hm=1$ практически дает 
и решение для случая $l\hm>1$, если оптимальное число пакетов $n$ 
достаточно велико.
     
     Обозначим через
     \begin{equation}
     f_l(n) =\fr{f(n)}{l}\left[ 1+\delta_l(n)\right]
     \label{e17p}
     \end{equation}
целевую функцию~(\ref{e13p})~--- среднее время выполнения задания при 
данных значениях $n$~--- чис\-ле пакетов и $l$~--- чис\-ле элементов, где 
$f(n)\hm=(L+n\tau)\exp\left(\lambda L/n\right)$~--- целевая функция~(\ref{e6p}) 
для случая $l\hm=1$.
     
     Задача выбора оптимального размера пакета~$\tilde{\upsilon}$ сводится к 
нахождению 
     \begin{equation}
     \min f_l(n) =f_l\left(\tilde{n}_l\right)\,.
     \label{e18p}
     \end{equation}
Здесь минимум берется по всем $n\hm\in N$, где $N$~--- множество 
допустимых значений~$n$ (например, $N$~--- множество це\-ло\-чис\-лен\-ных 
значений~$n$, кратных~2, лежащих в некотором допустимом диапазоне, 
и~т.\,п.). Полагаем $\tilde{\upsilon}\hm=L/\tilde{n}_l$, где 
$\tilde{n}_l$~--- решение задачи~(\ref{e18p}). Введем также дополнительную 
задачу нахождения 
\begin{equation}
\min f_l(n) =f_l(n_l^*)\,,
\label{e19p}
\end{equation}
где минимум берется по всем (не только це\-ло\-чис\-лен\-ным) значениям 
$n\hm\geq 1$. Далее оптимальный размер пакета~$\upsilon_l^*$ (без 
ограничений це\-ло\-чис\-лен\-ности $n\hm\in N$) определим по формуле 
$\upsilon_l^*\hm=L/n_l^*$, где $n_l^*$~--- решение задачи~(\ref{e19p}). Из 
выражений~(\ref{e13p})--(\ref{e16p}) далее следует теорема~3.

\medskip

\noindent
\textbf{Теорема~3.} \textit{Решение оптимизационной задачи}~(\ref{e19p}) 
\textit{удовлетворяет неравенствам}
\begin{equation}
\fr{f_l(n^*)}{1+\varepsilon}\leq \min\limits_{n\geq 1} f_l(n)\leq f_l(n^*)\,,
\label{e20p} 
\end{equation}
\textit{где $n^*$~--- решение этой задачи для случая $l\hm=1$, 
$\varepsilon\hm=\delta_l(n^*)\hm<l/n^*$. Решение оптимизационной 
задачи}~(\ref{e18p}) \textit{удовлетворяет аналогичным неравенствам}
\begin{equation}
\fr{f_l(\tilde{n})}{1+\varepsilon}\leq \min\limits_{n\in N} f_l(n)\leq 
f_l(\tilde{n})\,, 
\label{e21p}
\end{equation}
\textit{где $\tilde{n}$~--- решение этой задачи для случая} $l\hm=1$, 
$\varepsilon\hm=\delta_l(\tilde{n})\hm<l/\tilde{n}$.
     
     \medskip
     
\noindent
     Д\,о\,к\,а\,з\,а\,т\,е\,л\,ь\,с\,т\,в\,о\,.\ Равенство~(\ref{e17p}) при 
$n\hm=n^*$ имеет вид:
     \begin{equation}
     f_l(n^*) =\fr{f(n^*)}{l}\left[ 1+\delta_l(n^*)\right]\,.
     \label{e22p}
     \end{equation}
Из этого же равенства, учитывая, что $\delta_l(n)\hm>0$, следует
$$
f_l(n)\geq \fr{f(n)}{l}\,,
$$
откуда с учетом~(\ref{e22p})
$$
\min\limits_{n\geq 1} f_l(n) \geq \fr{1}{l}\min\limits_{n\geq 1} f(n) 
=\fr{f(n^*)}{l}= \fr{f_l(n^*)}{1+\delta_l(n^*)}\,,
$$
что вместе с~(\ref{e16p}) доказывает левое неравенство в~(\ref{e20p}). 
Правое неравенство очевидно. Доказательство неравенств~(\ref{e21p}) 
аналогично. Теорема доказана.

\medskip

     Таким образом, полученное решение для случая $l\hm=1$ 
практически дает решение и в случае $l\hm>1$, если число пакетов много 
больше по сравнению с количеством элементов~$l$.

\section{Заключение}
     
     Получено решение указанной выше основной проблемы (выбора 
оптимального размера пакета) для модели со сбоями элементов в 
естественной с прикладной точки зрения асимптотике, а именно для случая 
высоконадежных элементов и при малом времени пересылки пакетов. 
Существенно, что полученное решение не зависит от общего объема всего 
задания, что, в частности, позволяет использовать его в ситуации 
неопределенности, когда эта величина, вообще говоря, может быть 
неизвестной и случайной. Отметим также, что представляет интерес 
дальнейшее обобщение полученных результатов на ситуацию, когда 
различные элементы могут иметь существенно различные характеристики 
как производительности, так и надежности, а также на модель с отказами и 
восстановлением (заменой) отказавших элементов. 

{\small\frenchspacing
{%\baselineskip=10.8pt
\addcontentsline{toc}{section}{Литература}
\begin{thebibliography}{9}


\bibitem{3p} %1
\Au{Ронжин А.\,Ф., Суриков В.\,Н.}
О~времени полного перебора~// Обозр. прикл. пром. матем., 2007. Т.~14. 
№\,3. С.~506--508.

\bibitem{4p} %2
\Au{Коновалов М.\,Г., Малашенко Ю.\,Е., Назарова~И.\,А.}
Модели и методы управления заданиями в системах распределенных 
вычислительных ресурсов.~--- М.: ВЦ РАН, 2009. 110~с. (Сообщения по 
прикладной математике.)

\bibitem{5p} %3
\Au{Коновалов М.\,Г., Малашенко Ю.\,Е., Назарова~И.\,А.}
Оперативное управление потоком заданий в системе распределенных 
вычислительных ресурсов~// VI Московская междунар. конф. по 
исследованию операций: ORM-2010: Труды.~--- М.: 
МАКС Пресс, 2010. С.~301--302.

\bibitem{6p} %4
\Au{Козлов М.\,В., Малашенко Ю.\,Е., Назарова~И.\,А., Ронжин~А.\,Ф.}
Анализ режимов управления вычислительным комплексом в условиях 
неопределенности.~--- М.: ВЦ РАН, 2011. 63~с. (Сообщения по прикладной 
математике.)

\bibitem{7p} %5
\Au{Коновалов М.\,Г., Малашенко Ю.\,Е., Назарова~И.\,А.}
Управ\-ле\-ние заданиями в гетерогенных вычислительных сис\-те\-мах~// 
Известия РАН. Теория и системы управления, 2011. №\,2. С.~72--90. 

\bibitem{1p} %6
\Au{Гнеденко Б.\,В., Беляев Ю.\,К., Соловьев~А.\,Д.}
Математические методы в теории надежности.~--- М.: Наука, 1965. 524~с. 

\label{end\stat}

\bibitem{2p} %7
\Au{Gnedenko B.\,V., Pavlov I.\,V., Ushakov~I.\,A.}
Statistical reliability engineering.~--- N.Y.: John Wiley, 1999. 514~p.
 \end{thebibliography}
}
}


\end{multicols}