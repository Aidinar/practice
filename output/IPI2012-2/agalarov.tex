
\def\stat{agalarov}

\def\tit{АЛГОРИТМ ВЫЧИСЛЕНИЯ ХАРАКТЕРИСТИК МОДЕЛИ 
ТЕЛЕКОММУНИКАЦИОННОЙ СЕТИ С~ПОВТОРАМИ 
ПЕРЕДАЧ И~НЕПОЛНОДОСТУПНОЙ СХЕМОЙ УПРАВЛЕНИЯ 
БУФЕРАМИ$^*$}

\def\titkol{Алгоритм вычисления характеристик модели 
телекоммуникационной сети с~повторами 
передач} % и~неполнодоступной схемой управления  буферами}

\def\autkol{Я.\,М.~Агаларов}
\def\aut{Я.\,М.~Агаларов$^1$}

\titel{\tit}{\aut}{\autkol}{\titkol}

{\renewcommand{\thefootnote}{\fnsymbol{footnote}}\footnotetext[1]
{Работа выполнена при поддержке РФФИ, грант 11-07-00112.}}


\renewcommand{\thefootnote}{\arabic{footnote}}
\footnotetext[1]{Институт проблем информатики Российской академии наук, agglar@ya.ru}

\vspace*{2pt}     

      \Abst{Рассмотрена телекоммуникационная сеть со схемой управления буферами 
узлов SMQMA (Sharing with Maximum Queue Length and Minimum Allocation) и 
возможностью повтора передач из источника и в транзитных узлах. Предложен алгоритм 
расчета усредненных характеристик сети (вероятностей блокировок узлов, суммарной 
нагрузки на линиях, среднего числа пакетов в узлах и в сети, среднего числа пакетов, 
ожидающих повтора в источниках, и~др.). Приведены доказательства утверждений о 
свойствах алгоритма и результаты вычислительных экспериментов.}

\vspace*{2pt}
      
      \KW{телекоммуникационная сеть; повторы передач; механизмы управления 
буферами}

%\vspace*{-12pt}

\vskip 14pt plus 9pt minus 6pt

      \thispagestyle{headings}

      \begin{multicols}{2}

            \label{st\stat}
     
\section{Введение}
     
     Эффективность принимаемых на этапе проектирования решений по 
выбору варианта по\-стро\-ения телекоммуникационной сети в значительной 
степени зависит от адекватности ис\-поль\-зу\-емых моде\-лей процессов 
функционирования сетей и точ\-ности методов их расчета. Для обеспечения 
адекватности в модели должны быть учтены все существенные (с точки 
зрения цели исследования) свойства реальной сети. К таким свойствам, в 
частности, относятся: ограниченность буферной памяти узлов коммутации, 
механизмы управления буферами и многоканальность линий связи. 
     
     Ограниченность буферной памяти узлов коммутации является одной из 
причин роста числа повторных передач в сетях с коммутацией пакетов и, как 
следствие, резкого роста нагрузки на отдельные участки сети или сеть в 
целом и снижение ее производительности. 
     
     Значения показателей производительности сети существенным образом 
зависят и от механизмов управления буферами узлов коммутации, из 
множества которых наиболее применяемыми на практике являются 
статические схемы управления~[1, 2]: CS (Complete Sharing), CP (Complete 
Partitioning), SMQ (Sharing with Maximum Queue Length), SMA (Sharing with 
Minimum Allocation), SMQMA. Схема SMQMA является обобщением первых четырех 
схем. Для оценки эффективности указанных схем управления буферами 
разработаны различные модели узлов коммутации (см., например,~[1, 3, 4]) и 
методы расчета их характеристик. 
     
     Необходимость учета многоканальности линий связи вызвана тем, что 
замена в модели многоканальных линий одноканальными может внести 
значительную погрешность в результаты расчета показателей 
производительности сети. 
     
     В качестве модели сети с ограниченными буферами, как правило, 
используется сеть массового обслуживания, узлы которой рассматриваются 
как изолированные системы массового обслуживания (СМО) 
с ограниченными накопителями и пуассоновскими 
входящими потоками. Одной из первых работ, где предлагалась подобная 
модель, является~\cite{5aga}, в которой исследовалась задача выбора 
объемов буферной памяти узлов коммутации сети с датаграммным режимом 
работы в рамках модели сети с одноканальными линиями связи и схемой 
распределения буферов CS. В~этой же работе для расчета модели был 
использован подход, который в дальнейшем нашел применение в работах 
других исследователей (см., например,~[6--10]). 
Cуть подхода заключается в следующем. Для стационарных характеристик 
модели выводится система нелинейных уравнений вида 
$y_i\hm=f_i\left(\overline{y},\overline{a}\right)$, $i\hm=1,\ldots , n$, где 
$\overline{y}\hm=\left(y_1, \ldots ,y_n\right)$~--- вектор неизвестных переменных 
(например, вероятностей блокировки); $\overline{a}$~--- вектор известных 
па\-ра\-мет\-ров; $f_i$~--- обозначение функции. Для решения указанных 
уравнений применялись алгоритмы, основанные, как правило, на 
градиентном методе~\cite{4aga, 5aga} и методе простой итерации~[3, 5--10], 
вопрос сходимости которых, за 
исключением редких случаев, остается открытым. Среди работ, в которых 
для сетей с повторами передач приводится данный подход и доказательство 
сходимости алгоритма, следует выделить публикации~[4, 6--9], 
из которых в~[6, 8] рассматривалась сеть 
коммутации каналов, в~\cite{4aga}~--- случай сети с одноканальными 
линиями связи, повторами передач из источника и схемой CS, в~\cite{8aga, 10aga}~--- 
случаи сети с многоканальными линиями связи, повторами 
передач из источника и схемами CS и SMQ соответственно. 
     
     Ниже предлагается алгоритм расчета усредненных характеристик 
(вероятностей блокировки узлов, суммарной нагрузки линий, среднего числа 
пакетов в узлах и в сети, среднего числа повторов пакета из источника 
и~т.\,д.)\ для модели сети со схемой управления SMQMA, без потерь и 
возможностью повтора пакетов в транзитных узлах и в источниках. 
Алгоритм основан на указанном выше подходе. Приведено доказательство 
сходимости алгоритма и численный пример использования алгоритма. 
     
\section{Модель сети и~постановка задачи}
     
     Модель сети представляется в виде графа, состоящего из $U$ вершин и 
$V$~дуг. Вершины графа отождествляются с узлами связи, дуги~--- с 
линиями связи. Заданы множества узлов-входов (узлов коммутации, в 
которые извне сети поступают пакеты) и узлов-выходов (узлов коммутации, 
через которые пакеты покидают сеть) и множество нециклических 
путей~$L$, соединяющих узлы-входы с уз\-ла\-ми-выходами, причем для 
каждой пары узел-вход и узел-выход существует единственный 
соединяющий их путь. Для каждой пары узел-вход и узел-выход заданы 
интенсивности внешних потоков пакетов, передача которых в сети 
происходит по пути, соединяющему эти узлы. Узлы сети имеют 
ограниченную буферную память со схемой распределения SMQMA, линии 
связи имеют заданное число однородных каналов. 

\begin{figure*} %fig1
\vspace*{1pt}
 \begin{center}
 \mbox{%
 \epsfxsize=132.345mm
 \epsfbox{aga-1.eps}
 }
 \end{center}
 \vspace*{-9pt}
\Caption{Модель узла связи: $j\hm=1, \ldots ,k$~--- номера линий связи;
стрелка~--- возможное направление <<движения>> пакета;
надпись на стрелке~--- вероятность, с которой пакет выбирает указанное 
стрелкой направление;
окружность с номером~$j$~--- сток $j$-й линии; прямоугольник с номером~1 
(блок~1)~--- место в накопителе узла, где хранятся пакеты (копии пакетов), 
стоящие в очереди к линии для отправления в последующий узел;
прямоугольник с номером~2 (блок~2)~--- место в накопителе узла, где 
хранятся неуспешно переданные пакеты, ожидающие повторной передачи; 
прямоугольник с номером~3 (блок~3)~--- место в накопителе, где хранятся 
успешно переданные пакеты, ожидающие подтверждения}
\end{figure*}

     В качестве модели коммутационного узла используется СМО 
     с ограниченным накопителем (буферной памятью) 
и несколькими линиями из однотипных каналов, формализованная структура 
которой приведена на рис.~1. 

  
  Отметим, что накопитель разбит на блоки~1, 2 и~3 условно и пакет 
реально не передается из блока в блок, а хранится в одном месте, 
прикрепляемом условно к одному из блоков в зависимости от состояния 
процесса передачи в последующий узел. 
     
     Обозначим через $v$ линию связи, $u$~--- узел связи, 
     $\Omega_u^+$~--- множество исходящих из узла $u$ линий, по 
которым в последующие узлы передаются пакеты.
  
  В модели выполняются следующие условия:
  \begin{enumerate}[1.]
\item Места в накопителе распределяются согласно схеме SMQMA:
\begin{itemize}
\item за каждой линией $v$ закрепляется $a_v\hm\geq 1$ мест,  
$\sum\limits_{v\in \Omega_u^+} a_v\hm\leq R_{0u}$;
\item оставшиеся $R_u\hm=R_{0u}-\sum\limits_{v\in\Omega_u^+} a_v$ мест 
общедоступны;
\item максимальное число общедоступных мест, которые могут занять 
$v$-пакеты (пакеты из потока, поступающего на линию~$v$), не должно 
превышать заданную величину $0\hm\leq r_v\hm\leq R_u$, $v\hm\in 
\Omega_u^+$, $\sum\limits_{v\in\Omega_u^+}r_v\hm\geq R_u$.
\end{itemize}
\item Поступившему в СМО пакету предоставляется место в блоке~1 (см.\ 
рис.~1), если он передан без ошибок и в момент его поступления в 
накопителе есть доступное свободное место (произошла успешная попытка 
передачи из предыдущего узла), иначе пакет получает отказ (произошла 
неуспешная попытка передачи) и в предыдущем узле может быть повторена 
попытка его передачи.
\item Количество попыток передачи пакета в узле регулируется 
вероятностью повтора передачи, а именно: задается вероятность ($1\hm-a_v$) 
того, что попытка передачи является последней для данного $v$-пакета. 
Если попытка является не последней (происходит с вероятностью~$a_v$), то 
в узле хранится копия $v$-пакета (место хранения которой закрепляется за 
блоком~1 до отправления и передается блоку~2 или~3 после отправления в 
последующий узел) и от последующего узла требуется подтверждение 
успешной передачи, иначе после завершения передачи копия не хранится 
(пакет освобождает место в накопителе) и подтверждения не требуется. Если 
пакет принят в накопитель и предыдущий узел требует подтверждения, то в 
предыдущий узел отправляется подтверждение, где при его получении 
освобождается занятое пакетом место в накопителе (в блоке~3 предыдущего 
узла). Если в течение заданного интервала времени (тайм-аута) с момента 
отправления пакета (в течение этого времени копия хранится в блоке~2, если 
попытка не последняя) узел не получает положительного подтверждения, то 
пакет делает повторную попытку передачи из узла (копия пакета из блока~2 
передается в блок~1), иначе делает повторную попытку через случайный 
интервал времени из источника. 
\item Если в накопителе освобождается место, закрепленное за линией~$v$, 
и хотя бы одно общедоступное место в накопителе занимает $v$-пакет, то 
освободившееся место становится общедоступным, а одно занятое 
$v$-пакетом общедоступное место закрепляется за линией~$v$. 
\item Суммарные потоки $v$-пакетов, поступающих на линии извне узла, 
являются независимыми в совокупности пуассоновскими потоками. Для 
обслуживания $v$-пакета требуется одновременно одно место хранения и 
один канал типа~$v$, $v\hm\in\Omega_u^+$.
\item Принятые в СМО $v$-пакеты обслуживаются в порядке поступления.
\item Время занятия канала $v$-пакетом~--- экспоненциально 
распределенная случайная величина с параметром $0 \hm<\mu_v\hm<\infty$, 
$v\hm\in\Omega_u^+$.
\item Интервал времени $\tau_v>0$ (тайм-аут), через который неуспешно 
переданный $v$-пакет может быть передан повторно,~--- заданная 
детерминированная величина. 
\item Успешно переданный по линии~$v$ пакет освобождает место в 
накопителе через случайное время~$t_v$ с заданным средним 
значением~$\overline{t}_v$, $\tau_v\hm\geq t_v\hm>0$.
\item Источники пакетов (абонентские узлы) имеют неограниченный 
накопитель. Передача пакетов по абонентской линии, соединяющей 
абонентский узел с уз\-лом-вхо\-дом (уз\-лом-вы\-хо\-дом), происходит без 
ошибок. 
\item Подтверждения успешной передачи не те\-ря\-ются.
\end{enumerate}

Введем следующие обозначения:
\begin{description}
\item[\,] $v^+$~--- узел-сток линии~$v$;
\item[\,] $v^-$~--- узел-исток линии~$v$;
\item[\,] $c_v$~--- канальная емкость линии~$v$;
\item[\,] $l$~--- путь, соединяющий узел-вход с узлом-вы\-хо\-дом;
\item[\,] $L_v$~--- множество путей, содержащих линию~$v$, 
$L_v\hm\subseteq L$;
\item[\,] $v_0(l), \ldots , v_{S_l}(l)$~--- линии, составляющие путь~$l$, 
(т.\,е.\ $l\hm=\{v_0, \ldots , v_{S_l}\}$), где $(S_l+1)$~--- число линий на 
пути~$l$; индексы $0, \ldots , S_l$ показывают порядок следования 
элементов на пути; $v_i$~--- линия, исходящая из узла~$u_i$ пути~$l$; 
$V_0(l)$~--- линия, соединяющая источник с узлом-вхо\-дом; $v_{S_l}$~--- 
абонентская линия, исходящая из уз\-ла-вы\-хо\-да~$u_{S_l}$;
\item[\,] $l_v^+$~--- множество различных линий, включающее $v$ и 
линии, следующие после $v$ по направлению к адресату на пути~$l$, 
включая~$v_{S_l}$;
\item[\,] $V_v^+$~--- множество различных линий, включающее 
$\Omega_{v^-}^+$ и линии $v^\prime \in l_v^+$, $v\hm\in \Omega^+_{v^-
}$;
\item[\,] $\lambda(l)$~--- интенсивность потока ($l$-потока) пакетов, 
поступающих из источника на узел-вход и требующих передачи по 
пути~$l$, $\lambda(l)\hm>0$, $l\hm\in L$;
\item[\,] $\delta_v$~--- вероятность безошибочной передачи пакета по 
каналу линии~$v$;
\item[\,] $\Lambda^0_{v_{i(l)}}$~--- интенсивность суммарного $l$-потока 
пакетов на выходе линии $v_{i-1}(l)$, $i\hm=1, \ldots , S_l$;
\item[\,] $\Lambda^*_v(l)$~--- интенсивность суммарного $l$-потока 
пакетов, поступающих на линию~$v$;
\item[\,] $\pi_v$~--- вероятность блокировки узла для пакетов, требующих 
передачи по исходящей линии~$v$.
\end{description}
   
   Пусть $\overline{k}_u=\{\overline{k}_v, v\hm\in\Omega_u^+\}$~--- 
состояние накопителя узла $u\hm\in U$, 
где $\overline{k}_v\hm=\left(k_v,k_v^\prime,k_v^{\prime\prime}\right)$, $k_v$~--- чис\-ло 
пакетов в накопителе узла (в блоке~1), стоящих в очереди к линии~$v$, 
$k_v^\prime$~--- число пакетов в накопителе узла (в блоке~2), неуспешно 
переданных по линии~$v$ и ожидающих повторной передачи, 
$k_v^{\prime\prime}$~--- число пакетов в накопителе узла (в блоке~3), 
успешно переданных по линии~$v$, ожидающих подтверждения:
   \begin{align*}
   A_{R_u,\overline{r}_u, \overline{a}_u} &=\left \{ 
   \overline{k}_u\!: \sum\limits_{v\in\Omega_u^+}\left( 
k_v+k_v^\prime+k_v^{\prime\prime}-a_v\right)^+\leq
   R_u\,,
   \right.\\ 
&\left.    \vphantom{\sum\limits_{v\in\Omega_u^+}}
  \hspace*{10mm}k_v+k_v^\prime+k_v^{\prime\prime}\leq r_v+a_v\,,\enskip v\in 
\Omega_u^+\right\}\,;\\
A^{0,m_v}_{R_u, \overline{r}_u, \overline{a}_u} &= \left\{ \overline{k}_u\in 
A_{ R_u, \overline{r}_u, \overline{a}_u }:\ 
k_v+k_v^\prime+k_v^{\prime\prime}=m_v\right\}\,;\\
A^{1,m_v}_{ R_u, \overline{r}_u, \overline{a}_u }&= \left\{ \overline{k}_u\in 
A_{ R_u, \overline{r}_u, \overline{a}_u}:\ k_v=m_v\right\}\,;\\
A^{2,m_v}_{ R_u, \overline{r}_u, \overline{a}_u} &= \left\{ \overline{k}_u\in 
A_{ R_u, \overline{r}_u, \overline{a}_u}:\ 
k_v^\prime+k_v^{\prime\prime}=m_v\right\}\,,
\end{align*}
где 
\begin{align*}
(b)^+ & =\begin{cases}
b\,, & \mbox{если}\ b\geq 0\,,\\
0\,, & \mbox{если}\ b<0\,,
\end{cases}\\
m_v &= 0, \ldots , r_v+a_v\,.
\end{align*}

   Для данной модели узла с учетом введенных выше обозначений для 
$\pi_v$~--- вероятности блокировки узла для пакетов, поступающих на 
линию~$v$, $v\hm\in\Omega_u^+$, в стационарном режиме работы 
справедлива формула~[2, 3]:
   \begin{multline} 
   \pi_v = {}\\
   {}=\fr{1}{g\left(R_u, \overline{a}_u, \overline{r}_u, 
\overline{\rho}_u^*\right)}\sum\limits_{\overline{k}\in A_{R_u, \overline{r}_u, 
\overline{a}_u}} \hspace*{-3mm}p\left( \overline{c}_u, \overline{\pi}_u, 
\overline{k}_u,\overline{\rho}_u^*\right).
   \label{e1aga}
   \end{multline}
Здесь 
\begin{align*}
u&=v^-\,;\\[3pt]
p\left( \overline{c}_u, \overline{\pi}_u, 
\overline{k}_u,\overline{\rho}_u^*\right)&= \prod\limits_{v\in\Omega_u^+} 
z_v\left(c_v, \overline{\pi}_{v^+}, \overline{\rho}^*_v, k_v, s_v\right)\,,
\end{align*}
где  
$$
s_v\hm=k_v^\prime\hm+k_v^{\prime\prime}\,,
$$

\vspace*{-12pt}

\noindent
\begin{multline*}
z_v\left(c_v, \overline{\pi}_{v^+}, \overline{\rho}^*_v, k_v, s_v\right)={}\\
{}=
\begin{cases} 
\fr{\left(\rho_v^{\prime *}+\rho_v^{\prime\prime 
*}\right)^{s_v}}{s_v!}\,\fr{\rho_v^{*k_v}}{k_v!} &\ \mbox{при}\  k_v<c_v\,,\\
\fr{\left(\rho_v^{\prime *}+\rho_v^{\prime\prime 
*}\right)^{s_v}}{s_v!}\,\fr{\rho_v^{*k_v}}{c_v! c_v^{k_v-c_v}} &\ \mbox{при}\  
k_v\geq c_v\,;
\end{cases}
\end{multline*}

\vspace*{-9pt}

\noindent
$$
g\left(R_u, \overline{a}_u, \overline{r}_u, \overline{\rho}_u^*\right) = 
\sum\limits_{\overline{k_u}\in A_{R_u, \overline{r}_u, \overline{a}_u}} p\left(
\overline{c}_u, \overline{k}_u, \overline{\rho}_u^*\right)\,;
$$
$$
\overline{\rho}_u^*=\left\{ \overline{\rho}_v^*,\, v\in \Omega_u^+\right\}\,,
$$
где
\begin{align*}
\overline{\rho}_v^*&=\left( \rho_v^*, \rho_v^{\prime *}, \rho_v^{\prime\prime 
*}\right)\,,
\\
\rho_v^* &= \sum\limits_{l\in L_v} \fr{\Lambda^*_v(l)}{\mu_v(1-
\alpha_v\beta_v(l))}\,,\\
\rho_v^{\prime *} &=\sum\limits_{l\in L_v} 
\fr{\Lambda_v^*(l)\tau_v\alpha_v\beta_{v}(l)}{1-\alpha_v\beta_{v}(l)}\,,
\\
\rho_v^{\prime\prime *} &=\sum\limits_{l\in L_v} \fr{\Lambda_v^*(l) 
\overline{t}_v\alpha_v(1-\beta_v(l))}{1-\alpha_v\beta_v(l)}\,,
\end{align*}
$$
 \beta_v(l)=1-(1-
\delta_v)\left(1-\pi_{v(l)^+}\right)\,,\ v\in\Omega_u^+\,;
$$
$$
\overline{\pi}_u =\left \{ \overline{\pi}_{v^+},\,v\in \Omega_u^+\right\}\,,
$$
где
$$ 
\overline{\pi}_{v^+} =\left\{ \pi_{v^\prime}, 
v^\prime\in\Omega^+_{v^+}\right\}\,.
$$

   
   Ставится задача расчета значений параметров $\overline{\rho}^*_v$, 
$\pi_v$, $v\hm\in V$.
   
\section{Метод решения}
   
   В установившемся режиме работы рассматриваемой сети для потоков в 
узлах справедливы следующие соотношения:
   \begin{multline}
   \Lambda^0_{v_j(l)}(l)= \fr{\Lambda^0_{v_{j+1}(l)}(l)}{1-\beta_{v_{j-
1}(l)}(l)}\left( 1-\alpha_{v_j(l)}\beta_{v_j(l)}(l)\right)\,,\\ j=1, \ldots ,S_l\,,
   \label{e2aga}
   \end{multline}
   где $\Lambda^0_{v_{S_l+1}(l)}(l)=\lambda(l)$, $\delta_{v_0(l)}\hm=0$, 
$\pi_{v_0(l)}\hm=0$. 
   
   Физический смысл формулы заключается в том, что пакет, принятый в 
накопитель узла~$v_j^-(l)$, делает повторные попытки передачи в 
после\-ду\-ющий узел~$v_j^+(l)$ в среднем $(1-
\alpha_{v_j(l)}\beta_{v_j(l)}(l))^{-1}$ раз и интенсивность потока таких 
попыток, делаемых всеми пакетами $l$-потока, равна величине 
   $\Lambda^*_{v_j(l)}(l)(1-\beta_{v_{j-1}(l)}(l))/\left(1-
\alpha_{v_j(l)}\beta_{v_j(l)}(l)\right)$.
   
   Из~(\ref{e2aga}) получаем формулы:
   \begin{multline}
   \Lambda^*_{v_j(l)}(l) ={}\\
\hspace*{-2mm}{}= \fr{\lambda(l)}{1-
\pi_{v_i(l)}}\prod\limits_{j=1}^{S_l}\left( \fr{1-\alpha_{v_j(l)}}{1-
\beta_{v_j(l)}(l)}+\alpha_{v_j(l)}\right)\,,\ l\in L.
   \label{e3aga}
   \end{multline}
   
   Переобозначив в~(\ref{e1aga}) для краткости изложения $1-\pi_v$ 
через~$y_v$, $z_v\left(c_v, \overline{\pi}_{v^+}, \rho_v^*, k_v, l_v\right)$ через 
$z_v\left(k_v, l_v, \overline{\Lambda}_v^*, \overline{y}_{v^+}\right)$,
   $p\left(\overline{c}_u, \overline{\pi}_u, \overline{k}_u,\overline{\rho}_u^*\right)$ 
через $p_{\overline{k}_u}\left(\overline{\Lambda}_u^*, \overline{y}_u\right)$, $g\left(R_u, 
\overline{a}_u, \overline{r}_u, \overline{\rho}_u^*\right)$ через $g\left(\overline{a}_u, 
\overline{\Lambda}_u^*,\overline{y}_u\right)$, выражение в правой части равенства 
для~$\pi_v$ через $1\hm- q_{R_u, \overline{r}_u, 
\overline{a}_u}(\overline{\Lambda}_u^*, \overline{y}_u)$, где $u\hm=v^-$, 
$\overline{\Lambda}_u^*\hm= \left\{ \overline{\Lambda}_v^*, v\in\Omega_u^+\right\}$, 
$\overline{\Lambda}_v^*\hm=\left\{ \Lambda_v^*(l),l\in L_v\right\}$, 
$\overline{y}_u\hm= \left\{\overline{y}_v, v\in\Omega_u^+\right\}$, $\overline{y}_v\hm= 
\left\{ y_{v^\prime}, v^\prime\in\Omega_u^+\right\}$, получим систему нелинейных 
уравнений относительно неизвестных переменных~$y_v$:
   \begin{equation}
   y_v=q_v\left( \overline{\Lambda}^*_u, \overline{y}_u\right)\,,\enskip v\in 
V\,,\ u=v^-\,.
   \label{e4aga}
   \end{equation}
%
     Заметим, что $ q_v\left( \overline{\Lambda}^*_u, \overline{y}_u
     \right)$~--- функция, зависящая от $y_{v^\prime}$, $v^\prime\hm\in 
V_v^+$ .
     
     Обозначим набор $\{y_v, \ v\in V\}$ через $\overline{y}$. Будем 
говорить, что набор~$\overline{y}$ положителен, если $y_v\hm\in (0,\,1]$ для 
всех $v\in V$.
    
    Пусть задана последовательность $\overline{y}[n]\hm=\{y_v[n],\ v\in 
V\}$, $n\hm\geq0$, где 
$y_v[n+1]\hm=q_v(\overline{\Lambda}_u^*[n],\,\overline{y}_u[n])$, 
$y_v[0]\hm=1$, $v\hm\in V$, а $\overline{\Lambda}_u^*[n]$~--- это 
$\overline{\Lambda}_u^*$, вычисленный при $y_{v^\prime} \hm=1-
\pi_{v^\prime}\hm= y_{v^\prime}[n]$, $v^\prime\hm\in l_v^+$, $l\in L_v$. 
В~дальнейшем будем писать $\overline{y}[n+1]\hm< \overline{y}[n]$, если 
при заданном $n\hm\geq 0$ выполняется неравенство $y_v[n+1]\hm< y_v[n]$ 
для всех $v\hm\in V$.
     
     Введем обозначения: $\overline{a}_u =\{ a_v, v\in\Omega_u^+\}$, 
$\overline{r}_u\hm= \{ r_v, v\hm\in \Omega_u^+\}$,
      $
      \overline{a}_u -\overline{1}_{v^\prime} \hm= \overline{a}_u^\prime = \left\{ 
a^\prime_v, v\hm\in\Omega_u^+\right\}$,
     где 
     $$
     a^\prime_v=\begin{cases}
     a_v & \ \mbox{при}\ v\not=v^\prime\,;\\
     a_v-1 & \ \mbox{при}\ v=v^\prime\,.
     \end{cases}
     $$
     
     В дальнейшем параметры $(\overline{a}_u,\overline{r}_u)$ будем 
называть ограничениями доступа узла (СМО).
     
     \medskip
     
     \noindent
     \textbf{Теорема.} \textit{Существует $\lim\limits_{n\rightarrow\infty} 
\overline{y}[n]\hm=\overline{y}^0\hm\geq 0$. Система}~(\ref{e4aga}) \textit{имеет 
положительное решение тогда и только тогда, когда 
$\lim\limits_{n\rightarrow\infty} \overline{y}[n]\hm=\overline{y}^0\hm>0$.}
     
     \medskip
     
     \noindent
     Д\,о\,к\,а\,з\,а\,т\,е\,л\,ь\,с\,т\,в\,о\,.\ Докажем две вспомогательные 
леммы.
     
     Введем обозначения: $d_{v,\overline{a}_u}\left(\overline{a}_u, 
\overline{\Lambda}_u^*,\overline{y}\right)$~--- среднее число $v$-пакетов в 
узле $u\hm=v^- $ с параметрами $R_u$, $\overline{a}_u$ и $\overline{r}_u$;  
$d_{1, v, \overline{a}_u}\left(\overline{\Lambda}_u^*, \overline{y}_u\right)$~--- среднее 
число $v$-пакетов в блоке~1 узла $u\hm=v^-$ с параметрами $R_u$, 
$\overline{a}_u$ и $\overline{r}_u$; $d_{2,v,\overline{a}_u}\left( 
\overline{\Lambda}_u^*, \overline{y}_u\right)$~--- среднее число $v$-пакетов в 
блоках~2 и~3 узла $u\hm=v^-$ с параметрами $R_u$, $\overline{a}_u$ и
$\overline{r}_u$, $v\hm\in \Omega_u^+$.
     
     Из~(\ref{e1aga}) следуют равенства:
     \begin{multline}
     d_{v,\overline{a}_u} \left( \overline{a}_u, \overline{\Lambda}_u^*, 
\overline{y}_u\right) ={}\\
{}= \fr{1}{g(\overline{a}_u, \overline{\Lambda}_u^*, 
\overline{y}_u)}\sum\limits_{m_v=1}^{r_v+a_v}\! m_v
\sum\limits_{\overline{k}\in A^{0,m_v}_{R_u, \overline{r}_u, 
\overline{a}_u} } \!\!\!\!\!\!\! p_{\overline{k}_u}(\overline{\Lambda}_u^*,\overline{y}_u)\,; 
     \label{e5aga}
     \end{multline}
     
     \vspace*{-9pt}
     
     \noindent
     \begin{multline*}
     d_{1,v,\overline{a}_u} \left( \overline{a}_u, \overline{\Lambda}_u^*, 
\overline{y}_u\right) ={}\\
{}= \fr{1}{g(\overline{a}_u, \overline{\Lambda}_u^*, 
\overline{y}_u)}\sum\limits_{m_v=1}^{r_v+a_v} m_v
\sum\limits_{\overline{k}\in A^{1,m_v}_{R_u, \overline{r}_u, 
\overline{a}_u} }  p_{\overline{k}_u}(\overline{\Lambda}_u^*,\overline{y}_u)\,; 
\end{multline*}

\vspace*{-9pt}

\noindent
\begin{multline*}
     d_{2,v, \overline{a}_u} \left( \overline{a}_u, \overline{\Lambda}_u^*, 
\overline{y}_u\right) ={}\\
\hspace*{-3pt}{}= \fr{1}{g(\overline{a}_u, \overline{\Lambda}_u^*, 
\overline{y}_u)}\sum\limits_{m_v=1}^{r_v+a_v} m_v
\!\!\!\!\!\!\!\!\sum\limits_{\overline{k}\in A^{2,m_v}_{R_u, \overline{r}_u, 
\overline{a}_u} } \!\!\!\!\!\!\!\!\! p_{\overline{k}_u}(\overline{\Lambda}_u^*,\overline{y}_u)\,, 
\ v\in\Omega_u^+.
\end{multline*}
     
     \noindent
     \textbf{Лемма 1.} \textit{В~рамках модели, задаваемой 
соотношениями}~(\ref{e1aga}), \textit{для любого узла~$u$ и любой линии 
$v\hm\in\Omega_u^+$ справедливы неравенства:}
     \begin{gather*}
     d_{1,v^\prime, \overline{a}_{u}}\left( \overline{a}_u, 
\overline{\Lambda}_u^*, \overline{y}_u\right) -d_{1,v^\prime, \overline{a}_u-
1_v}\left( \overline{a}_u, \overline{\Lambda}_u^*, 
\overline{y}_u\right)>0\,;\\
 d_{2,v^\prime, \overline{a}_{u}}\left( \overline{a}_u, 
\overline{\Lambda}_u^*, \overline{y}_u\right) -d_{2,v^\prime, \overline{a}_u-
1_v}\left( \overline{a}_u, \overline{\Lambda}_u^*, 
\overline{y}_u\right)>0\,,\\
\hspace*{35mm}v^\prime\in \Omega_u^+\,.
     \end{gather*}
     
     \noindent
     Д\,о\,к\,а\,з\,а\,т\,е\,л\,ь\,с\,т\,в\,о\,.\ Пусть заданы~$R_u$~--- емкость 
накопителя; $(\overline{a}_u\hm-1_v,\,\overline{r}_u)$~--- параметры 
ограничения доступа узла~$u$. Назовем систему с ограничениями доступа 
$(\overline{a}_u\hm-1_v,\,\overline{r}_u)$ первой СМО, с ограничениями 
$(\overline{a}_u, \overline{r}_u)$~--- второй СМО и обозначим их 
соответственно через СМО$_1$ и СМО$_2$. Выделим в СМО$_2$ 
произвольное место хранения и обозначим его номером 
$\sum\limits_{v\in\Omega_u^+}a_v\hm+R_u$. Остальные места пронумеруем 
числами $1,\ldots , \sum\limits_{v\in\Omega_u^+}a_v\hm+R_u\hm-1$, и пусть 
пакету присваивается номер соответствующего места, где он хранится. 
Место с номером  $\sum\limits_{v\in\Omega_u^+}a_v\hm+R_u$ могут 
занимать только $v$-пакеты. 
     
     Рассмотрим СМО$_3$, отличающуюся от СМО$_2$ только дисциплиной 
обслуживания пакетов, и назовем ее третьей СМО. В~СМО$_3$ $v$-пакеты в 
первую очередь занимают место с номером 
$\sum\limits_{v\in\Omega_u^+}a_v\hm+R_u$, если оно свободно, иначе 
занимают свободные места с номерами $1,\ldots , 
\sum\limits_{v\in\Omega_u^+}a_v\hm+R_u\hm-1$. Номер пакета за время 
пребывания в системе не меняется. Пакеты с номерами $1,\ldots , 
\sum\limits_{v\in\Omega_u^+}a_v\hm+R_u\hm-1$ обслуживаются в каждом из 
блоков~1, 2, 3 в порядке поступления, и $v$-пакеты с этими номерами имеют 
абсолютный приоритет с дообслуживанием перед $v$-пакетом с номером 
$\sum\limits_{v\in\Omega_u^+}a_v\hm+R_u$. Заметим, что $v$-пакет с 
номером $\sum\limits_{v\in\Omega_u^+}a_v\hm+R_u$ поступает на 
обслуживание в соответствующий блок только при отсутствии в очереди к 
этому блоку $v$-пакетов с меньшими номерами (в блоках~2 и~3 очереди 
всегда отсутствуют).
     
     Очевидно, интенсивность потока пакетов каж\-до\-го типа, допускаемых в 
накопитель СМО$_3$, больше, чем в СМО$_1$ (так как занятие $v$-пакетами 
места с номером $\sum\limits_{v\in\Omega_u^+}a_v\hm+R_u$ разгружает 
места с меньшими номерами и не влияет на процессы обслуживания пакетов 
с этими номерами). А~поскольку времена обслуживания в блоках и 
вероятность повтора пакета каждого типа в СМО$_3$ и СМО$_1$ одинаковы, 
то интенсивность выходного потока каждого из блоков~1, 2, 3 в СМО$_3$ 
больше, чем интенсивность выходного потока соответствующего блока 
СМО$_1$. Следовательно, среднее число пакетов каж\-до\-го типа в каждом из 
блоков~1, 2, 3 в СМО$_3$ больше, чем в соответствующем блоке СМО$_1$.
     
     Отметим следующие свойства СМО$_3$:
     \begin{itemize}
\item времена обслуживания пакетов каждого типа с номерами из интервала 
$1,\ldots , \sum\limits_{v\in\Omega_u^+}a_v\hm+R_u\hm-1$ не зависят от их 
номеров;
\item время дообслуживания $v$-пакета с номером 
$\sum\limits_{v\in\Omega_u^+}a_v\hm+R_u$ и время обслуживания 
любого вновь поступившего $v$-пакета~--- одинаково распределенные 
случайные величины.
     \end{itemize}
     
     Из указанных свойств следует, что если в СМО$_3$ в момент 
поступления нового $v$-пакета менять мес\-то его хранения и место хранения 
обслуживаемого или ожидающего обслуживания в этот момент 
неприоритетного $v$-пакета, то процессы изменения числа пакетов каждого 
типа в системе не изменятся. Таким образом, процессы изменения числа 
пакетов каждого типа в СМО$_3$ и СМО$_2$ совпадают. Следовательно, 
среднее число пакетов каждого типа в каждом из блоков~1, 2, 3 в СМО$_2$ 
больше, чем в соответствующем блоке СМО$_1$. 
     
     \medskip
     
     \noindent
     \textbf{Лемма 2.}\ \textit{Для всех $v\in V$ функции 
$q_v(\overline{\Lambda}_u^*,\,\overline{y}_u)$ монотонно возрастают по} 
$y_{v^\prime}$, $v^\prime\hm\in V_v^+$.
     
     \medskip
     
     \noindent
     Д\,о\,к\,а\,з\,а\,т\,е\,л\,ь\,с\,т\,в\,о\,.\ Покажем, что для любых~$v$ и 
$v^\prime\hm\in V_v^+$ справедливо неравенство:
     \begin{equation}
     \fr{dq_v\left(\overline{\Lambda}_u^*, \overline{y}_u 
\right)}{dy_{v^\prime}}>0\,,
     \label{e6aga}
     \end{equation}
где $d$~--- знак производной. 
     
     Фиксируем произвольные линии $v\hm\in V$, $v^\prime\hm\in V_v^+$. 
Приведем ряд вспомогательных равенств~(\ref{e7aga})--(\ref{e12aga}). Как 
следует из определения схемы SMQMA, накопитель системы доступен для 
     $v$-пакета тогда и только тогда, когда 
$k_v\hm+k_v^\prime\hm+k_v^{\prime\prime}\hm<a_v$, или одновременно 
выполняются условия 
$$
a_v\leq  k_v+k_v^\prime+k_v^{\prime\prime}\leq a_v+r_v-1
$$ 
и 
\begin{multline*}
\sum\limits_{\substack{{v^\prime\in\Omega_u^+,}\\{v^\prime\not=v}}}\left( 
k_{v^\prime}+k^\prime_{v^\prime}+k_{v^\prime}^{\prime\prime} - 
a_{v^\prime}\right)^+\leq{}\\
{}\leq R_u-1-
\left(k_v+k_v^\prime+k_v^{\prime\prime}-a_v\right)^+\,.
\end{multline*}
     
     Тогда, использовав обозначения из~(\ref{e1aga}), получим:
     \begin{multline}
     q_v\left(\overline{\Lambda}^*_u, \overline{y}_u\right) = 
\left(\sum\limits_{m_v=0}^{a_v-1}\sum\limits_{\overline{k}_u \in 
A^{0,m_v}_{R_u, \overline{r}_u, \overline{a}_u}}\!\!\!\!\!\! p\left(\overline{c}_u, 
\overline{k}_u, \overline{\rho}_u^*\right)+{}\right.\\
\left.\hspace*{-5pt}{}+\sum\limits_{m_v=a_v}^{r_v+a_v-1} 
\sum\limits_{\overline{k}_u \in A^{0,m_v}_{R_u-1, \overline{r}_u, 
\overline{a}_u}}\!\!\!\!\!\!\!\!\!\!\!
p\left(\overline{c}_u, \overline{k}_u, \overline{\rho}_u^*\right)\right)\!\!\Bigg/\!\!
g(\overline{a}_u, \overline{\Lambda}^*_u, \overline{y}_u)={}\\
{}=
\fr{g(\overline{a}_u-\overline{1}_v,\overline{\Lambda}_u^*, 
\overline{y}_u)}{ g\left(\overline{a}_u, \overline{\Lambda}^*_u, \overline{y}_u\right) }\,.
     \label{e7aga}
     \end{multline}
      
%\noindent
     Для производной функции $q_v(\overline{\Lambda}_u^*, 
\overline{y}_u)$ по $y_{v^\prime}$, $v^\prime\hm\in V_u^+$, 
использовав~(\ref{e7aga}), получим:
     \begin{multline}
     \fr{\partial q_v\left(\overline{\Lambda}_u^*, \overline{y}_u\right)}{\partial 
y_{v^\prime}}=\fr{1}{g^2(\overline{a}_u, \overline{\Lambda}_u^*, 
\overline{y}_u)}\times{}\\
     {}\times \left[ g\left(\overline{a}_u, \overline{\Lambda}_u^*, 
\overline{y}_u\right)\fr{\partial g \left(\overline{a}_u-
\overline{1}_v,\overline{\Lambda}_u,\overline{y}_u\right)}{\partial y_{v^\prime}}-{}\right.\\
\left.{}-
g\left(\overline{a}_u-\overline{1}_v, \overline{\Lambda}_u^*, 
\overline{y}_u\right)\fr{\partial g \left(\overline{a}_u, \overline{\Lambda}_u^*, 
\overline{y}_u\right)}{\partial y_{v^\prime}}\right]\,.
     \label{e8aga}
     \end{multline}
     
     Из определения $g(\overline{a}_u, \overline{\Lambda}_u^*, 
\overline{y}_u)$, данного в~(\ref{e1aga}), получим:
%\end{multicols}

 %    \hrule

  \noindent
  \begin{multline}
     \fr{\partial g \left(\overline{a}_u, \overline{\Lambda}_u^*, 
\overline{y}_u\right)}{\partial y_{v^\prime}}={}\\
\hspace*{-3pt}{}=
\fr{\partial}{\partial 
y_{v^\prime}}\hspace*{-3pt}\sum\limits_{\overline{k} \in A_{R_u, \overline{r}_u, 
\overline{a}_u}} \prod\limits_{v^{\prime\prime\prime}\in\Omega_u^+ } 
\!\!\!\!\!z_{ v^{\prime\prime\prime}}\!\left(k_{ v^{\prime\prime\prime}}, 
s_{v^{\prime\prime\prime}}, \overline{\Lambda}_{ v^{\prime\prime\prime}}^*, 
\overline{y}_{ v^{\prime\prime\prime +}}\right)={}\\
     {}=\!\!\!\!\!\sum\limits_{v^{\prime\prime}\in \Omega_u^+} 
     \sum\limits_{\overline{k}\in A_{R_u, \overline{r}_u, \overline{a}_u}} 
     \prod\limits_{\substack{{v^{\prime\prime\prime}\in \Omega_u^+,}\\ 
{v^{\prime\prime\prime}\not=v^{\prime\prime}}}}
\!\!\!\!\!     z_{ v^{\prime\prime\prime}}\!\left(
     k_{ v^{\prime\prime\prime}},
     s_{ v^{\prime\prime\prime}}, \overline{\Lambda}^*_{ 
v^{\prime\prime\prime}}, \overline{y}_{ v^{\prime\prime\prime +}}\right)\times{}\\
{}\times
     \fr{\partial z_{ v^{\prime\prime}}
     \left(k_{v^{\prime\prime}}, s_{ v^{\prime\prime}}, \overline{\Lambda}_{ 
v^{\prime\prime}}, \overline{y}_{v^{\prime\prime +}}\right)} 
     {\partial y_{v^\prime}}\,.
     \label{e9aga}
     \end{multline}
     
%     \hrule
     
%     \begin{multicols}{2}
   
\noindent
   Для производной функции 
   $z_{ v^{\prime\prime}}\left(k_{ v^{\prime\prime}}, s_{ v^{\prime\prime}}, 
\overline{\Lambda}^*_{ v^{\prime\prime}}, \overline{y}_{ v^{\prime\prime 
+}}\right)$ из~(\ref{e1aga}) следует:
\begin{equation}
\left.
   \begin{array}{rl}
   \fr{\partial z_{ v^{\prime\prime}}\left(k_{ v^{\prime\prime}}, s_{ 
v^{\prime\prime}}, \overline{\Lambda}^*_{ v^{\prime\prime}}, \overline{y}_{ 
v^{\prime\prime +}}\right)}{\partial \rho^*_{ v^{\prime\prime}}}&={}\\[9pt]
&\hspace*{-35mm}{}=
   \fr{k_{ v^{\prime\prime}}}{\rho^*_{ v^{\prime\prime}}}\,z_{ 
v^{\prime\prime}} \left(k_{ v^{\prime\prime}}, s_{ v^{\prime\prime}}, 
\overline{\Lambda}_{ v^{\prime\prime}}^*, \overline{y}_{ v^{\prime\prime +}}\right)\,;
 \\[9pt]
   \fr{\partial z_{ v^{\prime\prime}}\left(k_{ v^{\prime\prime}}, s_{ 
v^{\prime\prime}}, \overline{\Lambda}^*_{ v^{\prime\prime}}, \overline{y}_{ 
v^{\prime\prime +}}\right)}{\partial \rho_{ v^{\prime\prime}}^{\prime *}}&={}\\[9pt]
&\hspace*{-35mm}{}=
   \fr{\partial z_{ v^{\prime\prime}}\left(k_{ v^{\prime\prime}}, 
   s_{ v^{\prime\prime}}, \overline{\Lambda}^*_{ v^{\prime\prime}}, 
\overline{y}_{ v^{\prime\prime +}}\right)}{\partial \rho_{ 
v^{\prime\prime}}^{\prime\prime *}}={}\\[9pt]
&{}\hspace*{-40mm}=\fr{s_{ v^{\prime\prime}}}
   {\rho^{\prime *}_{ v^{\prime\prime}}+\rho_{ 
v^{\prime\prime}}^{\prime\prime *}}\,z_{v^{\prime\prime}}\left(
   k_{ v^{\prime\prime}}, s_{ v^{\prime\prime}}, 
   \overline{\Lambda}_{v^{\prime\prime}}^*, \overline{y}_{ v^{\prime\prime +}}\right)\,.
   \end{array}
   \right\}
   \label{e10aga}
   \end{equation}
   
\noindent
Для производной функции $z_{ v^{\prime\prime}} \left(k_{ v^{\prime\prime}}, 
s_{ v^{\prime\prime}}, \overline{\Lambda}^*_{ v^{\prime\prime}}, 
\overline{y}_{ v^{\prime\prime +}}\right)$ по $y_{v^\prime}$ при $v^\prime\not= 
v^{\prime\prime}$, $v^\prime\not\in \Omega^+_{v^{\prime\prime +}}$ получим:
   \begin{multline}
      \fr{\partial z_{ v^{\prime\prime}}\left(k_{ v^{\prime\prime}}, s_{ 
v^{\prime\prime}}, \overline{\Lambda}^*_{ v^{\prime\prime}}, \overline{y}_{ 
v^{\prime\prime +}}\right)}{\partial y_{v^\prime}}={}\\
{}=
   \fr{\partial z_{ v^{\prime\prime}}\left(k_{ v^{\prime\prime}}, s_{ 
v^{\prime\prime}}, \overline{\Lambda}^*_{ v^{\prime\prime}}, \overline{y}_{ 
v^{\prime\prime +}}\right)}{\partial \rho^*_{ v^{\prime\prime}}}\,\fr{\partial \rho^*_{ 
v^{\prime\prime}}}{\partial y_{v^\prime}}+{}\\
{}+
   \fr{\partial z_{ v^{\prime\prime}}\left(k_{ v^{\prime\prime}},s_{ 
v^{\prime\prime}}, \overline{\Lambda}^*_{ v^{\prime\prime}},
\overline{y}_{v^{\prime\prime +}}\right)}{\partial \rho^{\prime *}_{ v^{\prime\prime}}}\,
   \fr{\partial \rho_{ v^{\prime\prime}}^{\prime *}}{\partial 
y_{v^\prime}}+{}\\
   {}+ \fr{\partial z_{ v^{\prime\prime}}\left(k_{ v^{\prime\prime}}, s_{ 
v^{\prime\prime}}, \overline{\Lambda}^*_{ v^{\prime\prime}}, 
\overline{y}_{v^{\prime\prime +}}\right)}{\partial \rho_{ v^{\prime\prime}}^{\prime\prime *}}\,
\fr{\partial \rho_{v^{\prime\prime}}^{\prime\prime *}}{\partial y_{v^\prime}}\,.
   \label{e11aga}
   \end{multline}
     
     Из определений переменных $\rho_v^*$, $\rho_v^{\prime *}$ и
$\rho_v^{\prime\prime *}$ в~(\ref{e1aga}) и из~(\ref{e3aga}) следуют 
формулы:
     \begin{equation}
     \left.
    \hspace*{-2mm} \begin{array}{rl}
     \rho^*_{v^{\prime\prime}} &=\fr{1}{y_{ v^{\prime\prime}}}\sum\limits_{l 
\in L_{ v^{\prime\prime}} } \lambda(l) \fr{1}{1-\beta_{ v^{\prime\prime}}(l)}\times{}\\[9pt]
&\hspace*{7mm}{}\times
     \prod\limits_{\substack{{v^\prime\in l_{ v^{\prime\prime}}^+,}\\{v^\prime 
\not=v^{\prime\prime}}}}\left( 
     \fr{1-\alpha_{v^\prime}}{1-
\beta_{v^\prime}(l)}+\alpha_{v^\prime}\right)\,;\\[9pt]
 \rho^{\prime *}_{ v^{\prime\prime}}&=\fr{\alpha_{ 
v^{\prime\prime}}\tau_{ v^{\prime\prime}}}{y_{ 
v^{\prime\prime}}}\sum\limits_{l\in L_{ v^{\prime\prime}}}
     \lambda(l) \left( \fr{1}{1-\beta_{ v^{\prime\prime}}(l)}-1\right) \times{}\\[9pt]
&    \hspace*{7mm}{}\times
\prod\limits_{\substack{{v^\prime\in l^+_{ v^{\prime\prime}},}\\ 
{v^\prime\not=v^{\prime\prime}}}}
     \left(\fr{1-\alpha_{v^\prime}}{1-
\beta_{v^\prime}(l)}+\alpha_{v^\prime}\right)\,;\\[9pt]
     \rho_{ v^{\prime\prime}}^{\prime\prime *} &= \fr{\alpha_{ 
v^{\prime\prime}} \overline{t}_{ v^{\prime\prime}}}
{y_{v^{\prime\prime}}}\sum\limits_{l\in L_{ v^{\prime\prime}}}\lambda(l) \times{}\\[9pt]
&\hspace*{7mm}{}\times
\prod\limits_{\substack{{v^\prime\in l^+_{ v^{\prime\prime}},}\\ {v^\prime\not=v^{\prime\prime}}}}\left( \fr{1-\alpha_{v^\prime}}{1-
\beta_{v^\prime}(l)}+\alpha_{v^\prime}\right)\,.
\end{array}
\right\}
\label{e12aga}
\end{equation}
   
   Фиксируем $v^{\prime\prime}\hm\in\Omega_u^+$. Рассмотрим случай 
$v^\prime\not= v^{\prime\prime}$, $v^\prime\not\in\Omega^+_{v^{\prime\prime +}}$. 
Обозначим для $l\hm\in L_v$ через $j(v,l)$ номер линии~$v$ на пути~$l$. 
Взяв производную переменных $\rho_v^*$, $\rho_v^{\prime *}$ и
$\rho_v^{\prime\prime *}$ по~$y_{v^\prime}$, из~(\ref{e1aga}) и~(\ref{e3aga}) 
получим: 
   \begin{multline*}
   \fr{\partial \rho^*_{ v^{\prime\prime}}}{\partial 
y_{v^\prime}}=\sum\limits_{\substack{{l:\ l\in L_{ v^{\prime\prime}},}\\ {v^\prime\in l}}} 
\fr{\partial \Lambda^*_{ v^{\prime\prime}}(l)}
   {\mu_{ v^{\prime\prime}}\left(1-\alpha_{ v^{\prime\prime}}\beta_{ 
v^{\prime\prime}}(l)\right)\partial y_{v^\prime}}={}\\
   {}=
   -\fr{1}{\mu_{ v^{\prime\prime}} y_{v^\prime}} 
   \sum\limits_{\substack{{l:\ l\in L_{ v^{\prime\prime}},}\\ {v^\prime\in l}}} 
   \fr{1}{1-\alpha_{ v^{\prime\prime}} \beta_{v^{\prime\prime}}(l)}\times{}\\
{}\times\fr{1-\alpha_{v_{j(v^\prime,l)-1}}}{1-
\alpha_{v_{j(v^\prime,l)-1}}\beta_{ v_{j(v^\prime,l)-1}}}\, 
\fr{\lambda(l)}{y_{ 
v_{j(v^\prime,l)}}} \times{}\\
{}\times
\prod\limits_{r=i(v^{\prime\prime},l)}^{S_l}\left(
   \fr{1-\alpha_{v_r(l)}}{1-\beta_{v_r(l)}}+\alpha_{v_r(l)}\right)={}\\
   {}=
    -\fr{1}{\mu_{v^{\prime\prime}}y_{v^\prime}}
\sum\limits_{\substack{{l:\ l\in L_{ v^{\prime\prime}},}\\ {v^\prime\in l}}}
      \fr{1}{1-\alpha_{ v^{\prime\prime}}\beta_{ v^{\prime\prime}}(l)}\times{}\\
      {}\times
      \fr{1-\alpha_{ v_{j(v^\prime,l)-1}}}{1-\alpha_{ v_{j(v^\prime,l)-
1}}\beta_{v_{j(v^\prime,l)-1}}}\, \Lambda^*_{ v^{\prime\prime}}(l);
      \end{multline*}
      
      \vspace*{-9pt}
      
      \noindent
  \begin{multline*}
      \fr{\partial \rho_{v^{\prime\prime}}^{\prime *} }
      {\partial y_{v^\prime}} =-\fr{\alpha_{v^{\prime\prime}}\tau_{v^{\prime\prime}}}{y_{v^\prime}}
      \sum\limits_{\substack{{l:\ l\in L_{ v^{\prime\prime}},}\\ {v^\prime\in l}}}  
      \fr{\beta_{ v^{\prime\prime}}(l)}{1-\alpha_{ v^{\prime\prime}}\beta_{ 
v^{\prime\prime}}(l)}\times{}\\
{}\times
      \fr{1-\alpha_{ v_{j(v^\prime,l)-1}}}{1-\alpha_{ v_{j(v^\prime,l)-
1}}\beta_{v_{j(v^\prime,l)-1}}}\, \Lambda^*_{ v^{\prime\prime}}(l)\,;
     \end{multline*}
     
     \vspace*{-12pt}
     
     \noindent
  \begin{multline*}
      \fr{\partial \rho_{}^{\prime\prime *}}{\partial y_{v^{\prime\prime}}}=
      -\fr{\alpha_{ v^{\prime\prime}} \overline{t}_{ v^{\prime\prime}}}{y_{v^\prime}}
      \sum\limits_{\substack{{l:\ l\in L_{ v^{\prime\prime}},}\\ {v^\prime\in l}}}
      \fr{1-\beta_{ v^{\prime\prime}}(l)}
      {1-\alpha_{ v^{\prime\prime}}\beta_{v^{\prime\prime}}(l)}\times{}\\
      {}\times
      \fr{1-\alpha_{v_{j(v^\prime,l)-1}}}{1-\alpha_{ v_{j(v^\prime,l)-1}}\beta_{ 
v_{j(v^\prime,l)-1}}}\, \Lambda^*_{ v^{\prime\prime}}(l)\,.
     \end{multline*}
   
   Подставив правые части последних трех равенств в~(\ref{e11aga}), в 
случае $v^\prime\not= v^{\prime\prime}$, $v^\prime\not\in 
\Omega^+_{v^{\prime\prime +}}$ получим:
   \begin{multline}
   \fr{\partial z_{ v^{\prime\prime}}\left(k_{ v^{\prime\prime}}, s_{ 
v^{\prime\prime}}, \overline{\Lambda}^*_{ v^{\prime\prime}}, \overline{y}_{ 
v^{\prime\prime +}}\right)}{\partial y_{v^\prime}}={}\\
{}=-
\fr{k_{v^{\prime\prime}}}{y_{v^\prime}}\,z_{v^{\prime\prime}}
\left(k_{v^{\prime\prime}}, s_{ v^{\prime\prime}}, 
\overline{\Lambda}_{v^{\prime\prime}}^*, \overline{y}_{ v^{\prime\prime +}}\right)
   \fr{1}{\mu_{ v^{\prime\prime}}\rho^*_{ v^{\prime\prime}}}\times{}\\
   \hspace*{-1pt}{}\times\hspace*{-5.5pt}
   \sum\limits_{\substack{{l:\ l\in L_{ v^{\prime\prime}},}\\ {v^\prime\in l}}} \hspace*{-4pt}
   \fr{1}{1-\alpha_{ v^{\prime\prime}}\beta_{v^{\prime\prime}}(l)}\,
   \fr{1-\alpha_{v_{j(v^\prime,l)-1}}}{1-\alpha_{ v_{j(v^\prime,l)-1}}
   \beta_{{v_{j(v^\prime,l)-1}}}}\,\Lambda^*_{ v^{\prime\prime}}(l)-{}\\
   {}-
   \fr{s_{ v^{\prime\prime}}}{y_{v^\prime}}\,z_{v^{\prime\prime}}
   \left(k_{ v^{\prime\prime}}, s_{ v^{\prime\prime}}, \overline{\Lambda}^*_{ 
v^{\prime\prime}}, \overline{y}_{ v^{\prime\prime +}}\right)\times{}\\
   {}\times \fr{\alpha_{ v^{\prime\prime}}}{\rho_{ v^{\prime\prime}}^{\prime 
*}+\rho_{ v^{\prime\prime}}^{\prime\prime *}}
   \sum\limits_{\substack{{l:\ l\in L_{ v^{\prime\prime}},}\\ {v^\prime\in l}}}
   \fr{\tau_{ v^{\prime\prime}} \beta_{ v^{\prime\prime}}(l) +\overline{t}_{ 
v^{\prime\prime}} \left(1-\beta_{ v^{\prime\prime}}(l)\right)}{1-
\alpha_{v^{\prime\prime}}\beta_{ v^{\prime\prime}}(l)}\times{}\\
{}\times
\fr{1- \alpha_{v_{j(v^\prime,l)-1}}}{1-\alpha_{v_{j(v^\prime,l)-
1}}\beta_{v_{j(v^\prime,l)-1}}}\,\Lambda^*_{ v^{\prime\prime}}(l)\,.
   \label{e13aga}
   \end{multline}
   
   \end{multicols}
   
   Для случая $v^\prime=v^{\prime\prime}$ из~(\ref{e12aga}) получим:
   $$
   \fr{\partial \rho_{v^{\prime\prime}}^*}{\partial y_{v^{\prime\prime}}}=-
\fr{\rho^*_{v^{\prime\prime}}}{y_{v^{\prime\prime}}}\,;\qquad 
   \fr{\partial \rho_{v^{\prime\prime}}^{\prime *}}
   {\partial y_{v^{\prime\prime}}}=
   -\fr{\rho^{\prime *}_{v^{\prime\prime}}}{y_{v^{\prime\prime}}}\,;\qquad
   \fr{\partial \rho^{\prime\prime *}_{v^{\prime\prime}}}
   {\partial y_{v^{\prime\prime}}}=-
   \fr{\rho_{v^{\prime\prime}}^{\prime\prime *}}
   {y_{v^{\prime\prime}}}\,.
   $$
   
   Подставив правые части последних равенств и равенств~(\ref{e10aga}) 
в~(\ref{e11aga}), для случая $v^\prime=v^{\prime\prime}$ получим:
   \begin{equation}
   \fr{dz_{ v^{\prime\prime}}\left(k_{ v^{\prime\prime}}, s_{ v^{\prime\prime}}, 
\overline{\Lambda}^*_{ v^{\prime\prime}}, \overline{y}_{ v^{\prime\prime 
+}}\right)}{\partial y_{ v^{\prime\prime}}}=
-\fr{\left(k_{ v^{\prime\prime}}+s_{ 
v^{\prime\prime}}\right)z_{ v^{\prime\prime}}\left(k_{ v^{\prime\prime}}, 
s_{v^{\prime\prime}}, \overline{\Lambda}^*_{ v^{\prime\prime}}, \overline{y}_{ 
v^{\prime\prime +}}\right)}{y_{ v^{\prime\prime}}}\,.
   \label{e14aga}
   \end{equation}
   
   Пусть $v^\prime\in\Omega^+_{v^{\prime\prime *}}$. Заменив переменные 
$\rho^*_{ v^{\prime\prime}}$, $\rho^{\prime *}_{ v^{\prime\prime}}$ и
$\rho^{\prime\prime *}_{ v^{\prime\prime}}$ соответственно правыми 
частями равенств~(\ref{e12aga}), для производной по 
$v^\prime\hm\in \Omega^+_{v^{\prime\prime +}}$ функции 
   $z_{ v^{\prime\prime}} \left(k_{ v^{\prime\prime}}, s_{ v^{\prime\prime}}, 
\Lambda^*_{ v^{\prime\prime}}, \overline{y}_{ v^{\prime\prime +}}\right)$ для 
случая $v^\prime\hm\in \Omega^+_{v^{\prime\prime +}}$ получим:
   \begin{multline}
   \fr{\partial z_{ v^{\prime\prime}}\left(k_{ v^{\prime\prime}}, 
   s_{v^{\prime\prime}}, 
   \Lambda^*_{ v^{\prime\prime}}, \overline{y}_{ v^{\prime\prime 
+}}\right)}{\partial y_{v^\prime}}= %{}\\{}=
z_{v^{\prime\prime}}\left( k_{ v^{\prime\prime}}, s_{ v^{\prime\prime}}, 
\Lambda^*_{ v^{\prime\prime}}, \overline{y}_{ v^{\prime\prime +}}\right)
   \left[ \fr{s_{ v^{\prime\prime}}}{\rho_{ v^{\prime\prime}}^{\prime 
*}+\rho_{ v^{\prime\prime}}^{\prime\prime *}}\,
   \fr{\partial \rho_{ v^{\prime\prime}}^{\prime *}}{\partial y_{v^\prime}}+
   \fr{k_{ v^{\prime\prime}}}{\rho^*_{ v^{\prime\prime}}}\,
   \fr{\partial \rho^*_{ v^{\prime\prime}}}{\partial y_{v^\prime}}\right]={}\\
   {}= -z_{ v^{\prime\prime}}\left(k_{ v^{\prime\prime}}, s_{ v^{\prime\prime}}, 
\Lambda^*_{ v^{\prime\prime}}, \overline{y}_{ v^{\prime\prime +}}\right)\left[
\vphantom{   \sum\limits_{\substack{{l\in L_{ v^{\prime\prime}},}\\ {v^\prime\in l}}} }
   \fr{s_{ v^{\prime\prime}}}{\rho_{ v^{\prime\prime}}^{\prime *}+\rho_{ 
v^{\prime\prime}}^{\prime\prime *}}\,
   \fr{\alpha_{ v^{\prime\prime}}\tau_{ v^{\prime\prime}}}
   {y_{v^\prime}\{ 1-\alpha_{ v^{\prime\prime}}[1-\left(1-\delta_{ 
v^{\prime\prime}}\right)y_{v^\prime}]\}}
\sum\limits_{\substack{{l\in L_{v^{\prime\prime}},}\\{v^\prime \in l}}}
\Lambda^*_{v^{\prime\prime}(l)}(l)-{}\right.\\
\left.   {}- \fr{k_{ v^{\prime\prime}}}{\rho^*_{ v^{\prime\prime}}}\,
   \fr{1}{\mu_{ v^{\prime\prime}} y_{v^\prime}\{1-\alpha_{ 
v^{\prime\prime}}[1-\left(1-\delta_{ v^{\prime\prime}}\right)y_{v^\prime}]\}}
   \sum\limits_{\substack{{l\in L_{ v^{\prime\prime}},}\\ {v^\prime\in l}}} 
\Lambda^*_{v^{\prime\prime}(l)}(l)\right]={}\\
   {}= -z_{v^{\prime\prime}}\left(k_{ v^{\prime\prime}}, s_{ v^{\prime\prime}}, 
\Lambda^*_{ v^{\prime\prime}}, \overline{y}_{ v^{\prime\prime +}}\right)
   \fr{1}{y_{v^\prime}\{ 1-\alpha_{ v^{\prime\prime}}[1-\left(1-\delta_{ 
v^{\prime\prime}}\right)y_{v^\prime}]\}} 
   \sum\limits_{\substack{{l\in L_{ v^{\prime\prime}},}\\ {v^\prime\in l}}}
   \Lambda^*_{v^{\prime\prime}(l)}(l)\left[
   \fr{k_{ v^{\prime\prime}}}{\mu_{ v^{\prime\prime}} 
   \rho_{v^{\prime\prime}}^*}+\fr{\alpha_{ v^{\prime\prime}}\tau_{ v^{\prime\prime}} 
s_{ v^{\prime\prime}}}{\rho_{ v^{\prime\prime}}^{\prime *}+
\rho_{ v^{\prime\prime}}^{\prime\prime *}}\right]\,.
\label{e15aga}
   \end{multline}
   
   Подставив правые части равенств (\ref{e13aga})--(\ref{e15aga}) 
в~(\ref{e9aga}) и использовав обозначения~(\ref{e5aga}), для произвольных 
$v^\prime\hm\in V_v^+$ получим:
   \begin{equation}
   \fr{\partial g \left(\overline{a}_u, \overline{\Lambda}_u^*, 
\overline{y}_u\right)}{\partial y_{v^\prime}} =-g\left(\overline{a}_u, 
\overline{\Lambda}_u^*, \overline{y}_u\right)\sum\limits_{v^{\prime\prime} \in \Omega_u^+} W_{\overline{a}_u, 
v^{\prime\prime},v^\prime}\,,
   \label{e16aga}
   \end{equation}
где 
\begin{align*}
W_{\overline{a}_u, v^{\prime\prime}, v^\prime}&=
\begin{cases}
\fr{1}{y_{v^\prime}} \left[ \Psi_{1, v^{\prime\prime}, v^\prime} d_{1, 
v^{\prime\prime}, \overline{a}_u}\left(\overline{\Lambda}_u^*, \overline{y}_u\right) 
+\Psi_{2, v^{\prime\prime}, v^\prime} d_{2, v^{\prime\prime}, \overline{a}_u}
\left(\overline{\Lambda}_u^*, \overline{y}_u\right) \right]\,,\ &\hspace*{-20mm} 
v^\prime \not= v^{\prime\prime}\,, v^\prime\not\in\Omega^+_{v^{\prime\prime +}}\,;\\
\fr{1}{y_{v^{\prime\prime\prime}}}\, d_{v^{\prime\prime}, \overline{a}_u} 
\left(\overline{\Lambda}_u^*, \overline{y}_u\right)\,,\ & \hspace*{-20mm}v^\prime=v^{\prime\prime}\,;\\
\fr{1}{y_{v^\prime}}\left[ 
\fr{d_{1, v^{\prime\prime}, \overline{a}_u}\left(\overline{\Lambda}^*_u, 
\overline{y}_u\right)}{\mu_{ v^{\prime\prime}} \rho_{ v^{\prime\prime}}^*}+
\fr{\alpha_{ v^{\prime\prime}}\tau_{ v^{\prime\prime}} d_{2, v^{\prime\prime}, 
\overline{a}_u}\left(\overline{\Lambda}_u^*, \overline{y}_u\right)}
{\rho_{ v^{\prime\prime}}^{\prime *}+\rho_{ v^{\prime\prime}}^{\prime\prime 
*}}\right] \fr{\sum\limits_{l\in L_{ v^{\prime\prime}}, v^\prime \in l} 
\Lambda^*_{v^{\prime\prime}(l)}(l)}{1-\alpha_{ v^{\prime\prime}}[1-\left(1-
\delta_{ v^{\prime\prime}}\right)y_{v^\prime}]}\,,\ & v^\prime\in\Omega^+_{v^{\prime\prime +}}\,;
\end{cases}
\\
   \Psi_{1, v^{\prime\prime}, v^\prime} &=\fr{1}{\mu_{ v^{\prime\prime}} 
\rho^*_{ v^{\prime\prime}}}  %\times{}\\{}\times
\sum\limits_{\substack{{l:\ l\in L_{ v^{\prime\prime}},}\\ {v^\prime\in l}}} \fr{1}{1-\alpha_{ v^{\prime\prime}}\beta_{ 
v^{\prime\prime}}(l)}\,
   \fr{1-\alpha_{v_{j(v^\prime,l)-1}}}{1-\alpha_{ v_{j(v^\prime,l)-1}}
   \beta_{ v_{j(v^\prime,l)-1}}}\, \Lambda^*_{ v^{\prime\prime}}(l)\,;
\\
   \Psi_{2, v^{\prime\prime}, v^\prime} &=
   \fr{\alpha_{ v^{\prime\prime}}}{\rho_{ v^{\prime\prime}}^{\prime 
*}+\rho_{ v^{\prime\prime}}^{\prime\prime *}} %\times{}\\
%{}\times
   \sum\limits_{\substack{{l:\ l\in L_{ v^{\prime\prime}},}\\ {v^\prime\in l}}} 
   \fr{\tau_{ v^{\prime\prime}}\beta_{ v^{\prime\prime}}(l)+\overline{t}_{ 
v^{\prime\prime}}\left(1-\beta_{ v^{\prime\prime}}(l)\right)}
   {1-\alpha_{ v^{\prime\prime}}\beta_{ v^{\prime\prime}}(l)}
   \fr{1-\alpha_{v_{j(v^\prime,l)-1}}}{1-\alpha_{ v_{j(v^\prime,l)-1}}
   \beta_{ v_{j(v^\prime,l)-1}}}\, \Lambda^*_{ v^{\prime\prime}}(l)\,.
   \end{align*}
   
   
   Подставив~(\ref{e16aga}) в~(\ref{e8aga}), получим:
   \begin{equation}
   \fr{\partial q_v\left(\overline{\Lambda}_u^*, \overline{y}_u\right)}{\partial 
y_{v^\prime}}=q_v\left( \overline{\Lambda}_u^*, 
\overline{y}_u\right)\sum\limits_{ v^{\prime\prime}\in \Omega_u^+} 
D_{\overline{a}_u, v^{\prime\prime}, v^\prime}\,,
   \label{e17aga}
   \end{equation}
   
%      \hrule
   
   \begin{multicols}{2}
   
   \noindent
   где 
   \begin{multline*}
   D_{\overline{a}_u, v^{\prime\prime}, v^\prime}={}\\
   {}=
   \begin{cases}
   \fr{1}{y_{v^\prime}}\left[ \Psi_{1, v^{\prime\prime}, v^\prime}
   \Delta_{1, v^{\prime\prime}, \overline{a}_u} \left(\overline{\Lambda}_u^*, 
\overline{y}_u\right)+{}\right.\\
   \hspace*{8mm}\left.{}+\Psi_{2, v^{\prime\prime}, v^\prime}
   \Delta_{2, v^{\prime\prime}, \overline{a}_u} \left(\overline{\Lambda}_u^*, 
\overline{y}_u\right)\right]\,, \\
 \hspace*{33mm}v^\prime\not= v^{\prime\prime}\,,\enskip 
v^\prime\not=\Omega^+_{v^{\prime\prime +}}\,;\\
   \fr{1}{y_{v^\prime}}\,\Delta_{v^{\prime\prime}, 
\overline{a}_u}\left(\overline{\Lambda}_u^*, \overline{y}_u\right)\,,\hspace*{19mm} 
v^\prime=v^{\prime\prime}\,;\\
   \fr{1}{y_{v^\prime}}\left[ 
   \fr{\Delta_{1,v^{\prime\prime}, 
\overline{a}_u}\left(\overline{\Lambda}_u^*,\overline{y}_u\right)}
   {\mu_{v^{\prime\prime}} \rho_{ v^{\prime\prime}}^*}+{}\right.\\
\left.   \hspace*{5mm}{}+
   \fr{\alpha_{ v^{\prime\prime}}\tau_{ v^{\prime\prime}}\Delta_{2, 
v^{\prime\prime}, \overline{a}_u}\left(\overline{\Lambda}_u^*, \overline{y}_u\right)}
   {\rho_{ v^{\prime\prime}}^{\prime *}+\rho_{ v^{\prime\prime}}^{\prime\prime *}}\right] \times{}\\
   {}\hspace*{3mm}\times
   \fr{\sum\limits_{l\in L_{ v^{\prime\prime}}, v^\prime\in l} 
\Lambda^*_{v^{\prime\prime}(l)}(l)}
   {1-\alpha_{ v^{\prime\prime}}[1-\left(1-\delta_{ 
v^{\prime\prime}}\right)y_{v^\prime}]}\,,\ v^\prime \in \Omega^+_{v^{\prime\prime 
+}}\,;
   \end{cases}
   \end{multline*}
   
   \vspace*{-12pt}
   
   \noindent
\begin{multline*}
   \Delta_{v^{\prime\prime},\overline{a}_u}\left(\overline{\Lambda}^*_u, 
\overline{y}_u\right) = d_{ v^{\prime\prime},\overline{a}_u} 
\left(\overline{\Lambda}_u^*, \overline{y}_u\right) -{}\\
{}-d_{ v^{\prime\prime},\overline{a}_u-
1_v}\left(\overline{\Lambda}_u^*, \overline{y}_u\right)\,;
\end{multline*}

\vspace*{-12pt}

\noindent
\begin{multline*}
   \Delta_{1,v^{\prime\prime},\overline{a}_u}\left(\overline{\Lambda}^*_u, 
\overline{y}_u\right) =
   d_{1, v^{\prime\prime},\overline{a}_u} \left(\overline{\Lambda}_u^*, 
\overline{y}_u\right) -{}\\
{}-d_{1, v^{\prime\prime},\overline{a}_u-
1_v}\left(\overline{\Lambda}_u^*, \overline{y}_u\right)\,;
\end{multline*}

\vspace*{-12pt}

\noindent
\begin{multline*}
   \Delta_{2,v^{\prime\prime},\overline{a}_u}\left(\overline{\Lambda}^*_u, 
\overline{y}_u\right) =
   d_{2, v^{\prime\prime},\overline{a}_u} \left(\overline{\Lambda}_u^*, 
\overline{y}_u\right) -{}\\
{}-d_{2, v^{\prime\prime},\overline{a}_u-
1_v}\left(\overline{\Lambda}_u^*, \overline{y}_u\right)\,.
\end{multline*}
   
   Отметим, что согласно лемме~1 
   $\Delta_{v^{\prime\prime}, \overline{a}_u} \left(\overline{\Lambda}_u^*, 
\overline{y}_u\right)\hm>0$,  $\Delta_{1, v^{\prime\prime}, \overline{a}_u} 
\left(\overline{\Lambda}_u^*, \overline{y}_u\right)\hm>0$ и $\Delta_{2, 
v^{\prime\prime}, \overline{a}_u} \left(\overline{\Lambda}_u^*, 
\overline{y}_u\right)\hm>0$. Тогда, так как $0\hm\leq \alpha_{v^{\prime\prime}}\leq 1$,
$0\hm\leq \delta_{v^{\prime\prime}}\hm<1$, $0\hm< y_{ v^{\prime\prime}}\hm\leq 1$, 
$\Psi_{1, v^{\prime\prime}, v^\prime}\hm>0$, $\Psi_{2, v^{\prime\prime}, 
v^\prime}\hm>0$  для всех $v^\prime,  v^{\prime\prime}\hm\in V$ таких, что 
$v^\prime, v^{\prime\prime}\hm\in l$ хотя бы для одного $l\hm\in L$, 
из~(\ref{e17aga}) получим неравенство~(\ref{e6aga}) и, следовательно, 
доказательство леммы~2.
     
     Продолжим доказательство теоремы. Из определения 
последовательности $\overline{y}[n]$, $n\hm\geq 0$, и из леммы~2 следует, 
что $\overline{y}[n+1]\hm< \overline{y}[n]$, $n\hm\geq 0$. Существование 
предела последовательности $\overline{y}[n]$, $n\hm\geq 0$,\linebreak
 следует 
непосредственно из свойства мо\-но\-тон\-ности и ограниченности этой 
последовательности. 
     
     Пусть $\overline{y}^\prime =\{ y_v^\prime \in (0,\,1],\ v\in V\}$~--- 
положительное решение системы уравнений~(\ref{e4aga}), 
$\overline{\Lambda}_u^\prime$~--- значение 
переменной~$\overline{\Lambda}^*_u$ при~$\overline{y}^\prime$, 
$\overline{\Lambda}_u^*[n]$~--- значение 
переменной~$\overline{\Lambda}^*_u$ при $\overline{y}_u[n]$. Очевидно, что
$y_v^\prime \hm<1$, $v\in V$, так как $q_v\left(\overline{\Lambda}_u^*, 
\overline{y}_u\right)\hm<1$ при любых положительных~$\overline{y}$. Пусть 
 $\overline{y}[n]\hm>\overline{y}^\prime$  для 
некоторого $n\hm\geq 0$
(существование такого $n$ вытекает из того, что $y_v[0]\hm=1$ и 
$y_v^\prime \hm<1$, $v\hm\in V$). Тогда, как следует из леммы~2, 
$y_v[n+1]\hm=q_v\left(\overline{\Lambda}^*_u[n], 
\overline{y}_u[n]\right)\hm > q_v\left(\overline{\Lambda}_u^\prime, 
\overline{y}_u^\prime\right)\hm=y_v^\prime$
для 
каждой линии $v\hm\in V$, т.\,е.\ последовательность 
$\overline{y}[n]$, $n\hm\geq 0$, ограничена снизу 
величиной~$\overline{y}^\prime$. Значит, существуют пределы 
$\lim\limits_{n\rightarrow\infty} y_v[n]\hm=y_v^0\hm\geq y_v^\prime\hm>0$ 
для всех $v\hm\in V$. Так как $\overline{\Lambda}^*_u$, 
$q_v\left(\overline{\Lambda}_u^*, \overline{y}_u\right)$, $v\hm\in V$,~--- непрерывные 
по~$y_{v^\prime}$, $v^\prime\hm\in V^+_v$, функции, то можно написать: 
$$
\lim\limits_{n\rightarrow\infty} q_v \left(\overline{\Lambda}_u^*[n], 
\overline{y}_u[n]\right)= q_v\left(\overline{\Lambda}_u^0, 
\overline{y}_u^0\right)=y_v^0\,,
$$ 
где 
$\overline{\Lambda}_u^0$~--- значение 
переменной~$\overline{\Lambda}_u^*$ при~$y^0_{v^\prime}$, 
$v^\prime\hm\in V_v^+$, $u\hm=v^-$, т.\,е.\ $\overline{y}^0\hm=\{ y^0_v\in 
(0,1),\ v\in V\}$~--- положительное решение системы 
уравнений~(\ref{e4aga}). 
     
     Пусть теперь $\lim\limits_{n\rightarrow\infty} y_v[n]\hm=y_v^0\hm>0$ 
для всех $v\hm\in V$. Тогда, так как $\overline{\Lambda}_u^*$, 
$q_v\left(\overline{\Lambda}_u^*, \overline{y}_u\right)$, $v\hm\in V$,~--- непрерывные по 
$y_{v^\prime}$, $v^\prime\hm\in V_v^+$, функции, выполняется 
$q_v\left(\overline{\Lambda}_u^0, \overline{y}_u^0\right)\hm=y_v^0$, т.\,е.\ 
$\overline{y}^0\hm=\{ y_v^0\in (0,\,1), \ v\hm\in V\}$~--- положительное 
решение системы уравнений~(\ref{e4aga}). 
     \medskip
     
     \noindent
     \textbf{Следствие~1.} Система~(\ref{e4aga}) не имеет положительного 
решения тогда и только тогда, когда $\lim\limits_{n\rightarrow\infty} 
y_v[n]\hm=y_v^*\hm=0$ хотя бы для одного $v\hm\in V$.
     
     \smallskip
     
     \noindent
     \textbf{Следствие~2.} Первичные потоки~--- реали\-зу\-емые (т.\,е.\ 
интенсивности первичных входных потоков равны интенсивностям 
соответствующих выходных потоков) тогда и только тогда, когда 
$\lim\limits_{n\rightarrow\infty} y_v[n]\hm=y_v^*\hm>0$ для всех $v\hm\in V$. 
     
\section{Алгоритм и~пример расчета}
     
     Предлагается следующий алгоритм вычисления загруженности 
линий~$\overline{\rho}_v^*$, $v\hm\in V$, и вероятностей блокировки 
пакетов~$\pi_v$, $v\hm\in V$, основывающийся на изложенном выше методе 
поиска решения системы~(\ref{e4aga}) (методе простой итерации). Для 
обозначения значений, вычисляемых на $k$-м шаге алгоритма, к 
обозначениям соответствующих параметров приписывается знак~[$k$]. 
     
     \textbf{Шаг 1.} \textit{Инициализация}. Используя рекуррентную 
формулу (см.~(\ref{e3aga})) 
$$
\Lambda^*_v[0](l)\hm=\lambda(l)\prod\limits_{v^\prime\in l_v^+} \fr{1-
\alpha_{ v^\prime }\delta_{ v^\prime }}{1-\delta_{v^\prime}},\ l\in L_v, \
v\hm\in V\,,
$$
и соответствующие формулы из~(\ref{e1aga}), вычислить 
начальные значения параметров $\overline{\rho}_v^*$, $y_v$, $v\in V$: 
$\overline{\rho}_v^*[0]\hm= \left(\rho_v^* [0], \rho_v^{\prime *}[0], 
\rho_v^{\prime\prime *}[0]\right)$, $y_v[0]\hm=1$.

\begin{figure*} %fig2
\vspace*{1pt}
 \begin{center}
 \mbox{%
 \epsfxsize=148.986mm
 \epsfbox{aga-2.eps}
 }
 \end{center}
 \vspace*{-9pt}
\Caption{Структура сети}
\vspace*{12pt}
\end{figure*}
     
     Положить $k=1$.
     
     \pagebreak
     

     \textbf{Шаг $k$ ($k\geq 1$)}.
     \begin{enumerate}[1.]
\item \textit{Вычисление вероятностей блокировки}. Используя значения 
параметров $\overline{\rho}_v^*[k-1]$, $v\in V$, с\linebreak
помощью 
соответствующих формул из~(\ref{e1aga}) вы\-чис\-лить $y_v[k]\hm=1-\pi_v[k]$, 
$v\hm\in V$. При этом используется метод свертки Базена, позволяющий 
производить рекуррентные вычисления~\cite{3aga}.
\item \textit{Проверка условий останова алгоритма}. Если хотя бы для одной 
$v\hm\in V$ выполняется условие $y_v[k]\hm <\varepsilon$, где 
$\varepsilon\hm >0$~--- значение требуемой точности результатов, алгоритм 
завершает работу, выдав сообщение: <<Первичные потоки заявок не 
реализуемы>>. 
\item \textit{Вычисление значения параметра} 
$\overline{\rho}_v^*[k]$ с по\-мощью рекуррентной формулы~(\ref{e3aga}) и 
формул~(\ref{e1aga}), подставив $y_v\hm=y_v[k]$, $v\hm\in V$.
\item \textit{Проверка условий останова алгоритма}. Если хотя бы для одной 
$v\in V$ выполняется условие
$$
\fr{\left\vert \rho^*_v[k]-\rho^*_v[k-1]\right\vert}{\rho_v^*[k]}>\varepsilon\,,
$$
то перейти к шагу $k+1$, положив~$k$ равным $k+1$, иначе алгоритм 
завершает работу.
     \end{enumerate}
     
     По завершении алгоритма либо выявляется, что система уравнений не 
имеет положительного решения (первичные потоки не реализуемы), либо 
вычисляются загруженности линий $\overline{\rho}_v^*\hm=\left(\rho^*_v, 
\rho_v^{\prime *}, \rho_v^{\prime\prime *}\right)$ и вероятности блокировки узлов 
для $v$-пакетов~$\pi_v$, $v\hm\in V$. Далее с помощью известных формул 
(в том числе формулы~(\ref{e1aga})) и полученных значений параметров 
$\lambda^*_v(l)$, $\overline{\rho}_v^*$, $\pi_v$, $l\hm\in L$, $v\hm\in V$, 
могут быть вычислены другие характеристики сети: среднее значение 
задержки пакета в узлах, среднее число повторов пакета в узлах и из 
источника, среднее число пакетов, находящихся в сети и в ожидании повтора 
из источника, и~др. 

\columnbreak
   
   Число арифметических операций, выпол\-ня\-емых на одном шаге алгоритма 
при использовании метода свертки Базена, имеет порядок
   $$
   Q=K\sum\limits_{u\in U}\left\vert \Omega_u^+\right\vert 
R_u+\sum\limits_{l\in L}\left\vert l\right\vert\,,
   $$
где $K$~--- константа; $\vert l\vert$~--- длина пути~$l$.
     
     В качестве примера использования разработанного алгоритма 
проведены расчеты характеристик сети с четырьмя узлами, топология 
которой задается графом, показанным на рис.~2. Фиксированы следующие 
входные данные: 

\noindent
     \begin{gather*}
     U=\{ 1,2,3,4\}\,;\ V=\{ 1,2,3,4,5,6\}\,;\\
      L=\{ l_1, \ldots ,l_{12}\}\,,\\ 
l_1=\{1\},\ l_2=\{1,2\}\,,\ 
l_3=\{1,2,3\}\,,\\
     l_4=\{2\}\,, \ l_5=\{2,3\}\,, l_6=\{3\}\,,\ l_7=\{4\}\,,\\
      l_8=\{4,5\}\,,\
     l_9=\{4,5,6\}\,,\ l_{10}=\{5\}\,,\ l_{11}=\{5,6\}\,,\\
     l_{12}=\{6\}\,;\ R_1=R_4=20\,;\ R_2=R_3=25\,;\\
      r_1=r_4=20\,;\ 
r_2=r_3=r_5=r_6=25\,;\\
     a_i=0\,;\ \mu_i=1\,; \ \delta_i=0{,}001\,;\ c_i=c_j\,;\\ 
     \alpha_i=\alpha_j\,,\enskip 
i,j=1, \ldots , 10\,;\\
\lambda(l_i)=\lambda(l_j)\,,  i,j=1, \ldots , 12\,.
     \end{gather*}
     
     

     
     
     При расчете $\overline{N}_{\mathrm{повт}}$~--- среднего суммарного 
числа пакетов, ожидающих повтора, использована формула:
     $$
     \overline{N}_{\mathrm{повт}}= \sum\limits_{l\in L}
     \left( \Lambda^*_{v_1(l)}(l)-\lambda(l)\right) t_{\mathrm{повт}}(l)\,,
     $$
     где $v_{1(l)}$~--- первая линия в составе пути~$l$, 
$t_{\mathrm{повт}}(l)$~--- интервал времени ожидания пакетом 
потока~$\lambda(l)$ повторного поступления с момента отказа в передаче, 
$t_{\mathrm{повт}}(l)\hm=10$.



     В таблице и на рис.~3 приведены зависимости от интенсивности 
первичных потоков $\lambda(l)$, $l\hm\in L$, среднего числа пакетов в сети 
(ряды с нечетными номерами), суммарного среднего числа пакетов в сети и в 
источниках в ожидании повтора (ряды с четными номерами). 

\end{multicols}

%\begin{table*}
\noindent
{\small
\begin{center}
\begin{tabular}{|c|c|c|c|c|c|c|c|c|c|c|c|c|}
\multicolumn{13}{c}{Зависимости среднего числа пакетов от интенсивности первичных потоков}\\[6pt]
\hline
Номер &\multicolumn{1}{c|}{\raisebox{-6pt}[0pt][0pt]{$a_i$}}&
\multicolumn{1}{c|}{\raisebox{-6pt}[0pt][0pt]{$c_i$}}&\multicolumn{10}{c|}{$\lambda(l)$}\\
\cline{4-13}
ряда&&&0,05&0,25&0,45&0,65&0,85&1,05&1,25&1,45&1,6&1,8\\
\hline
1&1&8&4,805&24,024&43,277&65,483&$\infty$&$\infty$&$\infty$&$\infty$&$\infty$&$\infty$\\
2&1&8&4,805&24,024&43,327&70,086&$\infty$&$\infty$&$\infty$&$\infty$&$\infty$&$\infty$\\
3&0&8&1,602&\hphantom{9}8,008&14,414&20,828&27,283&33,889&40,917&49,126&59,651&$\infty$\\
4&0&8&1,612&\hphantom{9}8,058&14,504&20,958&27,455&34,15&41,813&54,234&88,368&$\infty$\\
5&0&8&1,603&\hphantom{9}7,999&14,411&20,814&27,325&33,995&40,964&49,215&$\infty$&$\infty$\\
6&0&8&1,619&\hphantom{9}8,083&14,566&21,044&27,632&34,465&42,6\hphantom{99}&55,123&$\infty$&$\infty$\\
7&1&1&3,408&17,13\hphantom{9}&31,036&45,252&61,404&$\infty$&$\infty$&$\infty$&$\infty$&$\infty$\\
8&1&1&3,408&17,130&31,037&45,390&64,929&$\infty$&$\infty$&$\infty$&$\infty$&$\infty$\\
9&0&1&0,204&\hphantom{9}1,114&\hphantom{9}2,207&\hphantom{9}3,546&\hphantom{9}5,229&7,42&10,415&14,83\hphantom{9}&22,201&34,70561\\
10\hphantom{9}&0&1&0,214&\hphantom{9}1,164&\hphantom{9}2,297&\hphantom{9}3,676&5,4\hphantom{9}&7,63&10,669&15,214&23,873&46,70904\\
\hline
\end{tabular}
\end{center}

}     

\vspace*{12pt}

    \begin{multicols}{2}


\noindent
\begin{center} %fig3
\vspace*{1pt}
\mbox{%
   \epsfxsize=77.808mm
 \epsfbox{aga-3.eps}
}
\end{center}
%\begin{center}
\vspace*{3pt}
{{\figurename~3}\ \ \small{Зависимости среднего числа пакетов от интенсивности 
первичных потоков: 1--10~--- ряды~1--10}}
%\end{center}
\vspace*{10pt}

%\smallskip
\addtocounter{figure}{1}

     


\section{Заключение}

     Заметим, что погрешность, которую вносит алгоритм, ограничивается 
заранее задаваемой малой величиной $\varepsilon \hm>0$ и достоверность 
результатов работы алгоритма зависит только от адекватности модели сети. 
Результаты исследований разветвленных сетей показывают, что 
использованные выше в модели упрощающие предположения о 
пуассоновских входных потоках и независимости времен обслуживания в 
узлах на практике подтверждаются (см., например,~[3, 4, 7--10]). 
Отметим также, что анализ результатов вычислительных 
экспериментов, проведенных с использованием имитационных моделей, 
показал слабую зависимость вероятностей блокировки узлов, среднего числа 
пакетов в сети и среднего числа пакетов, ожидающих повтора в источниках, 
от вида функции распределения интервала ожидания повторной передачи 
пакета из источника. 

Открытым остается вопрос единственности решения системы 
уравнений~(\ref{e4aga}). Для некоторых частных случаев вопрос решается 
положительно. Приведем доказательство единственности решения 
системы~(\ref{e4aga}) для следующего частного случая сети. Рассмотрим 
сеть, в которой передача пакетов по каждой линии происходит только в 
одном на\-прав\-ле\-нии и используется полнодоступная схема управ\-ле\-ния 
буферами (CS). Заметим, что для схемы CS при всех $v\hm\in\Omega_u^+$ 
верны $\pi_v\hm=\pi_u$, $\pi_u$~--- вероятность блокировки узла~$u$

Пусть $M_0$~--- множество абонентских узлов-сто\-ков 
(або\-нен\-тов-адре\-са\-тов): $M_i$~--- множество узлов связи $u^\prime 
\not\in \mathop{\cup}\limits_{j=1}^{i-1} M_j$ таких, что для всех 
$v\hm\in\Omega^+_{u^\prime}$ выполняется $v^+\hm\in \mathop{\cup}\limits_{j=1}^{i-
1} M_j$, $i\hm=1, \ldots , l_{\max}$, где $l_{\max} \hm=\max\limits_{l\in L} S_l$~--- 
длина максимально протяженного пути на множестве~$L$. Легко показать, 
что $M_1, \ldots ,M_{l_{\max}}$ обладают свойствами:
\begin{enumerate}[(1)]
\item
   $M_i\frown M_j =\Theta$, $i,j\hm=1, \ldots , l_{\max}$, $i\not=j$;
   \item 
   $\mathop{\cup}\limits_{j=1}^{l_{\max}} M_j=U$;
   \item  максимальная длина пути (число линий, входящих в путь), 
соединяющего $u\hm\in M_i$ с або\-нен\-том-ад\-ре\-са\-том, равна~$i$; 
   \item  множество $\mathop{\cup}\limits_{j=1}^{i-1} M_j$~--- множество всех 
различных узлов, следующих после любого $u\hm\in M_i$ по направлению к 
адресату на путях, содержащих~$u$.
     \end{enumerate}
     
     Пусть $u\in M_1$. Тогда для любой $v\hm\in \Omega_u^+$ верно 
$\Lambda_v^*=\sum\limits_{l\in L_v} \lambda(l)/y_u$ (так как $v$~--- 
абонентская линия). Так как при схеме CS для всех $v\hm\in \Omega_u^+$ 
выполняется $y_v\hm=y_u\hm=1-\pi_u$, то уравнение~(\ref{e4aga}) 
эквивалентно уравнению: 
     \begin{equation}
     y_u=q_u\left(\overline{\Lambda}^*_u, y_u\right)
     \label{e18aga}
     \end{equation}
и имеет одну неизвестную переменную~$y_u$. 
     
     Введем параметры:
     \begin{gather*}
     \Lambda_v(l) =\lambda(l)\prod\limits_{v^\prime\in l_v} \left( \fr{1-
\alpha_{v^\prime}}{1-\beta_{v^\prime}}+\alpha_{v^\prime}\right)\,;\\
     \Lambda_v=\sum\limits_{l\in L_v}(l)\,;\enskip 
\delta_v^\prime=\alpha_v\beta_v\,;\\
     t_v^\prime=\fr{\overline{t}_v\alpha_v\left(1-\beta_v\right)}{1-\alpha_v\beta_v}\,,\ 
\enskip v\in\Omega_u^+\,.
     \end{gather*}
Как видим, введенные параметры для любого $u\hm\in M_i$ однозначно 
определяются переменными~$y_{u^\prime}$, $u^\prime\hm\in 
\mathop{\cup}\limits_{j=1}^{i-1} M_j$ (следует из приведенных выше выражений для 
введенных параметров и из свойства 4 множеств~$M_j$).
     
     Заметим, что~(\ref{e18aga}) сводится к уравнению~(3.3) из 
работы~\cite{9aga}, если в нем положить $\alpha_v\hm=1$, 
$\lambda_v\hm=\Lambda_v$, $\delta_v\hm=\delta_v^\prime$, 
$t_v\hm=t_v^\prime$, $v\hm\in\Omega_u^+$. Тогда из теоремы~1 
работы~\cite{9aga} следует, что в рассматриваемом случае 
уравнение~(\ref{e18aga}) имеет единственное решение.
     
     Фиксируем некоторое $i\hm\in \overline{2,\ldots ,l_{\max}}$. 
Предположим, что для всех $u^\prime\in \mathop{\cup}\limits_{j=1}^{i-1} M_j$ 
существуют единственные решения~$y_{u^\prime}$ уравнений~(\ref{e4aga}) 
(выражения правых частей этих уравнений не включают переменные 
$y_u\not\in \mathop{\cup}\limits_{j=1}^{i-1} M_j$). Тогда для случая любого $u\hm\in M_i$ 
аналогично случаю $u\hm\in M_1$ получим уравнение вида~(\ref{e18aga}), 
которое имеет единственное решение. Следовательно, для любого $u\hm\in 
\mathop{\cup}\limits_{j=1}^{l_{\max}} M_j$ уравнение~(\ref{e18aga}) имеет 
единственное решение. Так как $\mathop{\cup}\limits_{j=1}^{l_{\max}} M_j\hm=U$ 
(см.\ выше свойство~2 множеств $M_j$), то получаем доказательство 
единственности решения системы~(\ref{e4aga}) для рассматриваемого 
случая. 
     
     Отметим также следующие свойства разработанного алгоритма: 
     \begin{itemize}
\item при реализуемых первичных потоках последовательность 
$\overline{y}[n]$, $n\hm\geq 0$, сходится к положительному решению 
системы уравнений~(\ref{e4aga});
\item если последовательность $\overline{y}[n]$, $n\hm\geq 0$, сходится к 
неположительному вектору~$\overline{y}^0$, то входные первичные 
потоки не реализуемы в сети;
\item при реализуемых первичных потоках алгоритм вычисляет 
вероятности блокировки узлов и загруженности узлов и каналов связи с 
приемлемой для предварительного анализа сети точ\-ностью 
(относительная погрешность вероятности блокировки $\sim 0{,}1$) .
   \end{itemize}
   
   Алгоритм рекомендуется использовать для оценки эффективности 
решений по выбору канальных и вычислительных ресурсов, схем управления 
буферами и маршрутизации на этапе предварительного проектирования 
современных сетей коммутации пакетов. 

{\small\frenchspacing
{%\baselineskip=10.8pt
\addcontentsline{toc}{section}{Литература}
\begin{thebibliography}{99}
     
\bibitem{1aga}
\Au{Kamoun F., Kleinrock L.}
Analysis of shared finite storage in a computer networks node environment under 
general traffic conditions~// IEEE Trans. on Communications, 1980. Vol.~28. 
No.\,7. P.~992--1003.

\bibitem{2aga}
\Au{Ефимушкин В.\,А., Ледовских Т.\,В., Салькова~М.\,В.}
Механизмы управления трафиком в сетях АТМ~// Электросвязь, 2003. №\,1. 
С.~39--41.

\bibitem{3aga}
\Au{Башарин Г.\,П., Бочаров П.\,П., Коган~Я.\,А.}
Анализ очередей в вычислительных сетях.~--- М.: Наука, 1989.

\bibitem{4aga}
\Au{Вишневский В.\,М.}
Теоретические основы проектирования компьютерных сетей.~--- М.: 
Техносфера, 2003.

\bibitem{5aga}
\Au{Simon S.\,L.}
Store-and-forward buffer requirements in a packet switching network~// IEEE 
Trans. on Communications, 1976. Vol.~COM-24. No.\,4. P.~394--403.

\bibitem{6aga}
\Au{Kelly F.\,P.}
Blocking probabilities in large circuit-switched networks~// Advances Appl. 
Probability, 1986. Vol.~18. P.~473--505.

\bibitem{8aga} %7
\Au{Агаларов Я.\,М.}
Приближенный метод вычисления характеристик узла 
телекоммуникационной сети с повторными передачами~// Информатика и её 
применения, 2009. Т.~3. Вып.~2. С.~2--10.

\bibitem{7aga} %8
\Au{Степанов С.\,Н.}
Основы телетрафика мультисервисных сетей.~--- М.: Эко-Трендз, 2010. 


\bibitem{10aga}
\Au{Agalarov Y.}
Algorithm of nodes load estimation in the network with repetitions from source 
and static buffer management scheme~// 2010 Congress (International) on Ultra 
Modern Telecommunications and Control Systems (ICUMT) Proceedings.~--- 
M., 2010. P.~1073--1077.

\label{end\stat}

\bibitem{9aga} %10
\Au{Агаларов Я.\,М.}
Об одном численном методе вычисления стационарных характеристик узла 
коммутации с повторными передачами~// Автоматика и телемеханика, 2011. 
№\,1. С.~95--106.
 \end{thebibliography}
}
}


\end{multicols}
       