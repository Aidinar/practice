\def\stat{semenov}

\def\tit{НЕЧЕТКИЕ ПЕРЕМЕННЫЕ КАК СПОСОБ ФОРМАЛИЗАЦИИ 
ХАРАКТЕРИСТИК ПОГРЕШНОСТИ В~ЗАДАЧАХ МАТЕМАТИЧЕСКОЙ 
ОБРАБОТКИ}

\def\titkol{Нечеткие переменные как способ формализации 
характеристик погрешности в~задачах математической 
обработки}

\def\autkol{К.\,К.~Семенов}
\def\aut{К.\,К.~Семенов$^1$}

\titel{\tit}{\aut}{\autkol}{\titkol}

%{\renewcommand{\thefootnote}{\fnsymbol{footnote}}\footnotetext[1]
%{Работа выполнена при поддержке РФФИ (гранты 09-07-12098, 09-07-00212-а и
%09-07-00211-а) и Минобрнауки РФ (контракт №\,07.514.11.4001).}}


\renewcommand{\thefootnote}{\arabic{footnote}}
\footnotetext[1]{Санкт-Петербургский государственный политехнический университет, 
semenov.k.k@gmail.com}

\vspace*{-5pt}

\Abst{В метрологии периодически высказываются предложения об использовании в 
измерительной практике теории нечетких переменных. Известны примеры успешного 
внедрения ее результатов и основных идей, которые убеждают в перспективности подобного 
подхода, хотя и носят частный характер. Важнейшим приложением теории нечетких 
множеств в метрологии является учет и рассмотрение плохо формализуемой информации об 
источниках неопределенности, как правило, традиционными методами не учитывающейся. 
В~настоящей работе представлены результаты теоретического исследования 
принципиальной возможности другого приложения: учета и рассмотрения средствами 
теории нечетких переменных традиционных в метрологии хорошо формализуемых сведений 
о погрешности. Показана возможность единообразного описания погрешности при помощи 
нечетких множеств, не противоречащего сложившейся в отечественной метрологии 
традиции и требованиям норм и стандартов.}

\vspace*{-3pt}

\KW{нечеткие переменные; характеристики погрешности; результаты измерений}

   \vspace*{-1pt}
   
   \vskip 14pt plus 9pt minus 6pt

      \thispagestyle{headings}

      \begin{multicols}{2}

            \label{st\stat}

\section{Формулировка проблемы}
     
     Предложения об использовании теории нечетких множеств или ее 
частных случаев в теории и практике измерений и математической обработки 
появились достаточно давно~[1--4], возможность использования ее средств и 
идей исследовалась и обсуждалась в работах~[5--7].
  
  Согласно~[8], <<погрешность~--- идеализированное понятие, и погрешность 
не может быть известна точно\ldots\  неопределенность результата измерения 
отражает отсутствие точного знания значения измеряемой величины>>. Данное 
утверждение выражает основные предпосылки к использованию теории 
нечетких множеств в метрологии. 
  
  Любые измерительные задачи и задачи, связанные с вычислениями с неточно 
заданными исходными данными, сопряжены с неопределенностью или с 
нечеткостью, складывающейся из многих\linebreak со\-став\-ля\-ющих, обусловленных 
различными причинами. Традиционные методы метрологии позволяют учесть 
только хорошо формализуемую информацию о влияющих факторах, остальные 
сведения, как правило, в расчет не берутся. Появление тео\-рии нечетких 
множеств дало возможность также ввести в рассмотрение плохо 
формализуемую информацию, связанную, например, с опытом 
экспериментатора и позволяющую повысить точность измерений~\cite{5sem}. 
К~попыткам использования теории нечетких множеств в практике измерений 
подталкивает необходимость согласования априорной информации об 
измеряемых величинах, которую зачастую принципиально невозможно 
формализовать в рамках теории вероятностей, с операциями по оценке 
характеристик погрешности результатов измерений, которые выполняются в 
рамках математической статистики. 
  
  На сегодняшний день нечеткая интерпретация неопределенности сведений 
нашла широкое применение в экспертных системах и системах нечеткого 
контроля и управления. Однако используемые в них механизмы работы с 
нечеткими переменными не позволяют решать задачи статистической или иной 
обработки результатов прямых измерений, поскольку плохо соотносятся с 
теорией вероятностей.
  
  В измерительной практике сложилась ситуация, когда основную 
математическую обработку в мет\-ро\-ло\-ги\-че\-ских задачах выполняют 
традиционными методами математической статистики, а при помощи средств и 
методов теории нечетких множеств реализуется обработка плохо 
формализуемой информации и уточнение результатов. Подобное разделение 
может быть преодолено, если удастся \textit{реализовать в рамках теории 
нечетких переменных традиционные методы метрологии}, что и является 
целью настоящей статьи. Достичь этого предполагается представлением 
характеристик погрешности как нечетких переменных, выполненным с учетом 
специфики обработки результатов измерений и принятых на сегодняшний день 
метрологических норм.

\section{Предъявляемые требования к~представлению характеристик 
погрешности результатов измерений нечеткими переменными}
     
     Результат $x$ любого измерения сопровождается погрешностью~$\Delta 
x$, в общем случае складывающейся из систематической 
$\Delta_{\mathrm{сист}} x$ и случайной $\Delta_{\mathrm{случ}} x$ 
составляющих. Точное значение~$x_{\mathrm{ист}}$ измеряемой физической 
величины получено быть не может. Следовательно, не может быть известно и 
фактическое значение $\Delta x \hm= x\hm-x_{\mathrm{ист}} \hm= 
\Delta_{\mathrm{сист}} x\hm+\Delta_{\mathrm{случ}} x$ погрешности 
измерения. В~связи с этим в метрологии используют пределы допускаемых 
значений для погрешности результатов измерений. 

В~качестве подобной 
характеристики для систематической составляющей погрешности 
используются границы интервала $J_1\hm= [-\Delta_{\mathrm{сист}}, 
\Delta_{\mathrm{сист}}]$, гарантированно накрывающего значение 
$\Delta_{\mathrm{сист}} x:\ \vert \Delta_{\mathrm{сист}} x\vert \hm< 
\Delta_{\mathrm{сист}}$. Для случайной же со\-став\-ля\-ющей погрешности 
используют границы интервалов $J_P=[-\Delta_{\mathrm{случ}}, 
\Delta_{\mathrm{случ}}]$, таких что c вероятностью\linebreak не менее~$P$ (обычно 
$0{,}8\div0{,}95$) случайная со\-став\-ля\-ющая погрешности 
$\Delta_{\mathrm{случ}}x$ примет значение,\linebreak
 лежащее внутри~$J_P$. Таким 
образом, основной характеристикой, используемой для погрешности, является 
интервал, границы которого служат пределами допускаемых значений при 
заданной доверительной вероятности, как это уста\-нов\-ле\-но в~\cite{9sem}. 
\textit{Интервальное описание характеристик погрешностей должно быть 
сохранено при по\-стро\-ении их пред\-став\-ле\-ния нечеткими переменными}.
  
  Естественным расширением понятия интервала может служить понятие 
нечеткого интервала, введенное в~\cite{1sem}. Соответствующий ему 
формализм позволяет оперировать с набором интервалов, каждому из которых 
поставлено в соответствие значение $0\hm\leq \alpha \hm\leq 1$ меры, 
именуемой уровнем доверия~\cite{10sem}. Наблюдаемая аналогия с понятиями, 
принятыми в\linebreak теории вероятностей, делает актуальными попытки 
представления в рамках теории нечетких множеств как совокупности $\langle 
\Delta_{\mathrm{сист}}x, \Delta_{\mathrm{случ}}x\rangle$ со\-став\-ля\-ющих 
погрешности, так и самих результатов\linebreak
измерений в целом, поскольку 
сохраняют интервальный принцип их описания. Необходимо только, чтобы при 
этом не возникало противоречий с правилами работы с 
систематической и случайной составляющими погрешности, принятыми в 
метрологии.
  
  На практике для операций с систематическими составляющими погрешности 
$\Delta_{\mathrm{сист}}x $ используется интервальная арифметика, а для 
операций со случайными составляющими погрешности 
$\Delta_{\mathrm{случ}}x$ используются результаты теории вероятностей. 
Унифицировать операции с совокупностью $\langle \Delta_{\mathrm{сист}}x, 
\Delta_{\mathrm{случ}}x\rangle$ так, чтобы в рамках одного из перечисленных 
аппаратов одновременно работать как с систематической составляющей 
погрешности, так и со случайной составляющей не представляется возможным. 
Теория нечетких переменных же позволяет осуществить подобную 
унификацию.
  
  Поскольку обработка составляющих $\Delta_{\mathrm{сист}}x $ и 
$\Delta_{\mathrm{случ}}x$ погрешности выполняется с помощью различных 
правил и операций, то можно сформулировать набор требований, 
предъявляемых к разрабатываемому представлению погрешностей как 
нечетких переменных, следующим образом: 
  \begin{enumerate}[(1)]
\item в рамках подобного представления \textit{должно осуществляться 
разделение систематической и случайной составляющих погрешности}, для 
каж\-дой из которых обеспечиваются соответствующие правила работы с их 
интервальными характеристиками, в частности: 
\item \textit{аппарат теории нечетких переменных при использовании 
указанного представления должен обеспечивать уменьшение 
среднеквадратического отклонения либо интервала неопределенности 
случайной составляющей погрешности в $\sqrt{n}$ раз при усреднении n 
результатов прямых многократных измерений значения одной и той же 
физической величины};
\item \textit{систематическая составляющая погрешности должна 
обрабатываться по правилам интервальной арифметики}.
\end{enumerate}

  Формулировка требования~2 обусловлена тем, что зачастую при 
метрологическом анализе результатов математической обработки неточных 
данных применяют линеаризацию вычисляемых функций. По этой же причине 
в дальнейшем ограничимся рассмотрением только линейных операций с 
нечеткими переменными.
  
  Чтобы обладать практической ценностью в задачах оценки характеристик 
погрешности результатов математической обработки, представление должно 
также удовлетворять следующим требованиям:
  \begin{enumerate}[(1)]
  \setcounter{enumi}{3}
\item \textit{использование представления погрешности результатов прямых 
измерений как нечетких переменных не должно увеличивать время 
обработки данных по сравнению с традиционными методами метрологии};
\item \textit{значения характеристик погрешности конечных результатов 
математической обработки должны совпадать или быть близки к 
оценкам, которые могут быть получены в рамках традиционных методов}.
\end{enumerate}

  Данный набор требований исчерпывающе описывает необходимый для 
практики набор свойств представления погрешностей как нечетких 
переменных. Прежде чем выполнить теоретический анализ перечисленных 
требований, кратко на\-пом\-ним основные понятия теории нечетких переменных.

\section{Основные понятия теории нечетких переменных}

     Нечеткой переменной $\xi$ является совокупность пар $\langle x, 
\mu_\xi(x)\rangle$, где $x\hm\in R$~--- возможные значения~$\xi$, а функция 
$\mu_\xi(x)$, поставленная им в соответствие, принимает значения из 
интервала $[0,\,1]$. Функцию $\mu_\xi(x)$ называют функцией принадлежности, она 
является основной характеристикой нечеткой переменной~$\xi$. 
Носителем~$S_\xi$ нечеткой переменной часто называют интервал $[a,\,b]$, 
такой что для любого числа $x\hm\in [a,\,b]$ известно, что $\mu_\xi(x)\not=0$, а 
для любого числа $y\hm\not\in  [a,\,b]$ известно, что 
$\mu_\xi(y)\hm=0$~\cite{11sem}. Значения~$\mu_\xi(x)$ принято называть 
уровнем доверия. 
  
  В работе~\cite{10sem} введено понятие вложенного интервала~$J_\alpha$ 
функции принадлежности как интервала, границами которого являются две 
точки пересечения прямой $\alpha\hm=const$, параллельной оси~$x$, с 
графиком функции~$\mu_\xi(x)$. Пример функции принадлежности с 
обозначением перечисленных ее характеристик представлен на рис.~1.

\noindent
\begin{center} %fig1
\vspace*{12pt}
\mbox{%
    \epsfxsize=56.756mm
 \epsfbox{sem-1.eps}
}
\end{center}
%\begin{center}
\vspace*{3pt}
{{\figurename~1}\ \ \small{Пример функции принадлежности $\mu_\xi(x)$ нечеткой переменной~$\xi$}}
%\end{center}
%\vspace*{10pt}

%\smallskip
\addtocounter{figure}{1}


\section{Анализ сформулированных требований}

     Памятуя о поставленной в работе цели, можно выполнить представление 
погрешности нечеткой переменной двумя путями:
     \begin{enumerate}[(1)]
\item просто указав соответствие между основными интервальными 
характеристиками ($\Delta_{\mathrm{сист}}$ и~$\Delta_{\mathrm{случ}}$) 
погрешности и сопоставленной ей функцией принадлежности,
\item указав содержательную интерпретацию функции принадлежности как 
варианта описания погрешности в целом.
\end{enumerate}

   Второй вариант является более предпочтительным, поскольку помимо 
реализации традиционных методов метрологии позволяет достичь единства 
понятийного аппарата при работе с погрешностями и нечеткими переменными. 
  
  В~\cite{11sem} отмечается, что функция принадлежности имеет 
субъективный характер и может интерпретироваться на основе понятия 
вероятности: значение $\mu_\xi(x)$ может быть понято как вероятность того, 
что лицо, принимающее решение, отнесет значение~$x$ к множеству 
значений~$\xi$. 
  
  Исходя из данной содержательной физической интерпретации функции 
принадлежности, при представлении погрешности нечеткой переменной 
естественным представляется трактовать значения уровня доверия~$\alpha$ как 
степень уве\-рен\-ности в том, что интервал~$J_\alpha$ полностью накрывается 
интервалом возможных значений погрешности $\Delta x$ или 
интерквантильным промежутком заданной вероятностной меры. При такой 
интерпретации сохраняется интервальный принцип описания характеристик 
погрешности: вложенные интервалы определяют множества возможных 
значений полной погрешности при заданном уровне доверия.
  
  В~связи с указанной трактовкой уместно потребовать, чтобы 
$J_{\alpha_1}\subseteq J_{\alpha_2}$ при $\alpha_1\hm\geq \alpha_2$. Из этого 
следует, что функция принадлежности~$\mu_\xi(x)$, выражающая сведения о 
значениях погрешности, должна принадлежать к классу нормальных функций 
принадлежности~\cite{12sem, 13sem}, т.\,е.\ таких функций, для которых 
выполнены следующие условия:
  \begin{itemize}
  \item существует промежуток значений носителя $(a,\,b],\, a\hm\leq b$, во 
всех точках которого функция принадлежности равна единице 
$\mu_\xi(x_0)\hm=1$, а также 
  \item при отступлении влево от точки $x\hm=a$ или вправо от точки 
$x\hm=b$ функция принад\-леж\-ности~$\mu_\xi(x)$ не возрастает.
  \end{itemize}
  
  Таким образом, функция принадлежности нечеткой переменной, 
выражающей характеристики погрешности результатов прямых измерений, 
должна являться в общем случае криволинейной трапецией.

\subsection{Анализ требования~1} %4.1
  
  Впервые представление погрешности нечеткими переменными было 
выполнено Резником с соавторами в работе~\cite{1sem} и отражало возможное 
наличие систематической $\Delta_{\mathrm{сист}}x$ и случайной 
$\Delta_{\mathrm{случ}}x$ составляющих погрешности результата измерений. 
Вслед за указанной работой в~\cite{14sem} предложено считать, что 
вложенный интервал~$J_1$, т.\,е.\ верхнее основание трапеции функции 
принадлежности, отражает сведения о~$\Delta_{\mathrm{сист}}x$. Например, 
если известно, что $\vert \Delta_{\mathrm{сист}}x\vert \hm \leq 
\Delta_{\mathrm{сист}}$, то $J_1$ целесообразно брать равным $J_1\hm= [-
\Delta_{\mathrm{сист}},\,\Delta_{\mathrm{сист}}]$. 
  
  Вложенные интервалы при прочих значениях уровня доверия $\alpha\not=1$ 
предлагается считать отражающими сведения о характере случайной 
составляющей погрешности $\Delta_{\mathrm{случ}}x$.
  
  Таким образом, \textit{для любой имеющейся функции принадлежности 
всегда может быть осуществлено разделение выражаемых ею сведений о 
характеристиках систематической и случайной составляющих погрешности}. 
Для этого достаточно лишь рас\-смот\-реть по отдельности вложенный интервал 
$J_1$ и вложенные интервалы~$J_\alpha$ при значениях уровня доверия 
$\alpha\not=1$.

\subsection{Анализ требования~2} %4.2
  
  Поскольку положение требования~2 относится только лишь к случайной 
составляющей погрешности, то будем считать, что рассматривается такая 
ситуация, когда систематическая составляющая $\Delta_{\mathrm{сист}}x$ 
погрешности результата измерений~$x$ пренебрежимо мала или вовсе 
отсутствует, т.\,е.\ $\Delta_{\mathrm{сист}}x \hm=0$.
  
Определим свойства функции $\mu_\xi(x)$, необходимые для анализа 
требования~2.
  
  Будем считать, что математическое ожидание случайной величины, которой 
является случайная составляющая погрешности $\Delta_{\mathrm{случ}}x$, 
есть $\mathrm{M}[\Delta_{\mathrm{случ}}x]\hm=0$. В~случае, если 
$\mathrm{M}[\Delta_{\mathrm{случ}}x]\hm=c\not=0$, имеем возможность считать в 
качестве систематической составляющей погрешности величину 
$\Delta_{\mathrm{сист}} x\hm= \mathrm{M}[\Delta_{\mathrm{случ}}x]\hm=c$, а в 
качестве случайной составляющей~--- величину $(\Delta_{\mathrm{случ}} x\hm 
-c)$, математическое ожидание которой уже равно \mbox{нулю}.
  
  В этом случае нечеткая переменная~$\xi$, пред\-став\-ля\-ющая погрешность 
$\Delta x \hm= \Delta_{\mathrm{случ}}x$, будет иметь функцию 
принадлежности $\mu_\xi(x)$, равную~1, только при значении аргумента 
$x\hm= 0$.
  
  На практике чаще всего используют доверительные интервалы для значений 
$\Delta_{\mathrm{случ}}(x)$ с пределами, симметричными относительно 
$\mathrm{M}[\Delta_{\mathrm{случ}}x]\hm=0$. Поэтому будем считать, что функция 
$\mu_\xi(x)$ является симметричной функцией относительно значения 
$x\hm=0$ аргумента. 
  
  Так как функция $\mu_\xi(x)$ является нормальной функцией 
принадлежности, то при $x\hm<0$ она не убывает, а при $x\hm>0$ не 
возрастает.
  
  Потребуем, чтобы функция принадлежности $\mu_\xi(x)$ являлась 
\textit{аналитической функцией} на своем носителе~$S$, что повлечет ее 
гладкость. Действительно, нет оснований приписывать функции 
принадлежности особенности, так как последняя строится экспертами исходя 
из физических соображений, а интуитивно наиболее подходящей на роль 
функции принадлежности представляется достаточно гладкая кривая. Кроме 
того, подобное допущение необходимо сделать в соответствии с требованием 4. 
Чем больше особенностей имеет функция~$\mu_\xi(x)$, тем от большего числа 
параметров она зависит и тем большее время занимают вычисления с ней.
  
Переформулируем требование~$2$ в более удобной для дальнейшего 
анализа форме.
  
  В большинстве случаев на практике в качестве пределов допускаемых 
значений для случайной\linebreak
 составляющей погрешности 
$\Delta_{\mathrm{случ}}x$ выбираются границы интервала $[-
m\sigma,\,m\sigma]$, где $\sigma$~--- значение среднеквадратического 
отклонения $\Delta_{\mathrm{случ}}x$. Величина~$m$ определяется на основе 
соображений, изложенных в~\cite{15sem}, либо по известному неравенству 
П.\,Л.~Чебышёва или его уточнениям для ряда важных практических случаев 
для заданного значения доверительной вероятности~$Q$. Из теории 
вероятностей известно, что среднеквадратическое отклонение суммы двух 
независимых случайных величин с одинаковыми дисперсиями ровно в 
$\sqrt{2}$ раз больше среднеквадратического отклонения каждого из 
слагаемых. 
  
  Рассмотрим две нечеткие переменные~$\xi_1$ и~$\xi_2$, отражающие 
сведения о случайной составляющей погрешности двух результатов 
независимых прямых измерений значения одной и той же величины одним и 
тем же средством измерения. Функции принадлежности $\mu_{\xi_1}(x)$ 
и~$\mu_{\xi_2}(x)$ указанных переменных тождественно равны, т.\,е.\ 
$\mu_{\xi_1}(x)\hm=\mu_{\xi_2}(x)\hm=\mu(x)$, носителем функции $\mu(x)$ 
является интервал~$S$. 
  
  Поскольку $\mu(x)$ является такой симметричной функцией 
принадлежности, что при $\{x:\ x\hm>0, \, x\hm\in S\}$ она не возрастает, а при 
$\{x:\ x\hm\leq 0,\, x\hm\in S\}$ не убывает и при этом $\mu(x)\hm=1$ только 
при $x\hm=0$, то функция $\Delta_\mu(\alpha)\hm=\vert \mu^{-1}(\alpha)\vert$ 
является однозначной при $1\hm\leq \alpha\hm <0$. Заметим, что 
$\Delta_\mu(\mu(x))\hm=\vert x\vert$ и $\mu(\Delta_\mu(\alpha))\hm=\alpha$.
  
  Обозначим $\eta =\xi_1+\xi_2$. Так как функция 
принадлежности~$\mu_\xi(x)$ выражает степень уверенности в том, что 
вложенные интервалы $J_\alpha [\xi_1]$ и~$J_\alpha[\xi_2]$ накрываются 
промежутком $[-m\sigma,\,m\sigma]$, то, соответственно, получаем, 
что функция принадлежности $\mu_\eta(x)$ выражает степень уверенности в 
том, что вложенный интервал $J_\alpha[\eta]$ накрывается промежутком $[-
m\sqrt{2}\,\sigma,\, m\sqrt{2}\,\sigma]\hm= \sqrt{2}[-
m\sigma,\,m\sigma]$. Следовательно, требование~2 может быть 
переформулировано в виде соотношения 
$\Delta_{\mu_\eta}(\alpha)\hm=\sqrt{2}\,\Delta_\mu(\alpha)$, т.\,е.\ вложенные 
интервалы функции принадлежности результата суммирования должны быть в 
$\sqrt{2}$ раз шире,\linebreak чем вложенные интервалы функции при\-над\-леж\-ности 
операндов при том же самом значении уровня доверия.
  
  Подставим в это соотношение значение $\alpha\hm=\mu_\eta(x)$. Получим:
  \begin{align*}
  \Delta_{\mu_\eta}(\mu_\eta(x)) &=\sqrt{2}\,\Delta_\mu(\mu_\eta(x))\,;\\
  \fr{\vert x\vert}{\sqrt{2}} &=\Delta_\mu (\mu_\eta(x))\,;\\
  \mu\left( \fr{\vert x\vert}{\sqrt{2}}\right) &=\mu(\Delta_\mu(\mu_\eta(x)))\,;\\
  \mu\left(\fr{x}{\sqrt{2}}\right) &=\mu_\eta(x)\,.
  \end{align*}
  Отсюда следует, что эквивалентным требованию~2 является 
соблюдение соотношения:
  $$
  \mu_\eta(x) =\mu\left( \fr{x}{\sqrt{2}}\right)\,.
  $$
  
Проанализируем положения требования~2.
  
  Воспользовавшись принципом обобщения Заде~\cite{11sem}, 
представляющим общий вид операций над нечеткими переменными, получим 
соотношение, связывающее~$\mu_\eta(x)$ и~$\mu(x)$ и задающее правило 
сложения нечетких переменных: 
  $$
  \mu_\eta(z)=\sup\limits_{\substack{{x\in S,\ y\in S,}\\{x+y=z}}}
  \left\{ \mu(x)\cap \mu(y)\right\}\,.
  $$
  
  В силу непрерывности и монотонности функции $\mu(x)$ на интервалах $\{x:\ 
x>0, x\in S\}$ и $\{x:\ x\leq 0, x\in S\}$ 
  $$
  \sup\limits_{\substack{{x\in S, y\in S,}\\{x+y=z}}} \{\mu(x)\cap 
\mu(y)\}=\max\limits_{\substack{{x\in S, y\in S,}\\{x+y=z}}} \{\mu(x)\cap \mu(y)\}\,.
$$ 
Таким образом, 
\begin{equation}
\max\limits_{\substack{{x\in S, y\in  S,}\\{x+y=z}}}\{\mu(x)\cap \mu(y)\}=\mu\left(\fr{z}{\sqrt{2}}\right)\,.
\label{e0-sem}
\end{equation}

Среди наиболее 
типовых вариантов введения операции пересечения и, как следствие, правил 
сложения нечетких переменных, встречающихся на практике, 
сле\-ду\-ющие~\cite{13sem}:
  \begin{itemize}
  \item минимаксная операция 
  
    \noindent
$$
\mu_{\xi_1+\xi_2}(z)=\sup\limits_{\substack{{x\in S_1, y\in S_2,}\\{x+y=z}}} 
\{\min\{\mu_{\xi_1}(x),\mu_{\xi_2}(y)\}\}\,;
$$
  \item алгебраическая операция 
  
  \noindent
  $$
  \mu_{\xi_1+\xi_2}(z)= 
\sup\limits_{\substack{{x\in S, y\in S,}\\{x+y=z}}} 
\{\mu_{\xi_1}(x)\mu_{\xi_2}(y)\}\,.
$$
  \end{itemize}
  
  Проверим, удовлетворяют ли перечисленные операции требованию~(1).
  
  Результату минимаксной операции сложения двух нечетких переменных с 
одинаковыми функциями принадлежности будет соответствовать функция 
принадлежности 

\noindent
\begin{multline*}
\mu_\eta(z) = \sup\limits_{\substack{{x\in S, y\in S,}\\{x+y=z}}}\{\min\{\mu(x),\mu(y)\}\}={}\\
{}=
\max\limits_{x\in  S}\{\min\{\mu(x),\mu(z-x)\}\}={}\\
{}= \max\limits_{x\in S}\{\min\{\mu(x),\mu(x-z)\}\}=\mu\left(\fr{z}{2}\right)\,.
\end{multline*}

Действительно, при $x_0\hm=z/2$ имеем 
$\mu(x_0)\hm=\mu(x_0-z)$ и, соответственно, 
$$
\min\{\mu(x_0),\mu(x_0- z)\}=\mu\left(\fr{z}{2}\right)\,.
$$ 


При $\vert x\vert \hm\leq \vert x_0\vert$ имеем $\vert x-z\vert 
\hm\geq \vert z\vert /2$ и, как следствие, 
$$
\mu(z-x)=\mu(\vert z-x\vert ) \leq  \mu\left(\fr{\vert z\vert}{2}\right)=\mu\left(\fr{z}{2}\right)\,.
$$

Соответственно, получаем, что 
$\min\{\mu(x),\mu(x\hm-z)\}\hm\leq \mu(z/2)$. При $\vert x\vert \hm\geq \vert 
x_0\vert$ $\mu(x)\hm\leq \mu(x_0)\hm=\mu(z/2)$ и снова $\min\{\mu(x),\mu(x\hm-
z)\}\hm\leq \mu(z/2)$. Таким образом, минимаксная операция не может 
удовлетворить требованию~2, поскольку $\mu(z/2)\hm\geq \mu(z/\sqrt{2})$, причем 
равенство достигается только при $z\hm=0$.
  
  Алгебраическое же правило 
  
  \noindent
  \begin{multline*}
  \mu_\eta(z) =\sup\limits_{\substack{{x\in S, 
y\in S,}\\{x+y=z}}}\{\mu_{\xi_1}(x)\mu_{\xi_2}(y)\} = {}\\
{}=
\max\limits_{\substack{{x\in S, y\in S,}\\{x+y=z}}}\{\mu_{\xi_1}(x) 
\mu_{\xi_2}(y)\}
\end{multline*} 
оказывается правилом определения операции сложения 
нечетких переменных, потенциально позволяющим удовлетворить 
требованию~2.
  
  Отдельно отметим, что согласно~\cite{11sem} именно алгебраический 
вариант принципа обобщения Заде соответствует вероятностной трактовке 
функции принадлежности и, таким образом, наиболее близок к поставленным в 
настоящей работе задачам.

%\pagebreak
  
  Может быть сформулирована и доказана следующая
  
  \smallskip
  
  \noindent
  \textbf{Теорема.} \textit{Пусть функция $\mu(x)$ является функцией 
принадлежности нечеткой переменной, выражающей}\linebreak\vspace*{-12pt}

\pagebreak

\noindent
 \textit{сведения о случайной 
погрешности результата измерений, и выбрано алгебраическое правило 
сложения нечетких переменных. Требование~$2$ выполняется тогда и только 
тогда, когда} $$
\mu(x)=\exp\left[ -\fr{x^2}{2\sigma^2}\right]\,,
$$
\textit{где} 
$\sigma\hm>0$.
  
  \medskip
  
  Перед тем как привести доказательство пред\-став\-лен\-но\-го утверждения, 
отметим, что имеет место
  
  \medskip
  
  \noindent
  \textbf{Лемма.} \textit{Если функция принадлежности $\mu(x)$ нечеткой 
переменной, выражающей сведения о случайной погрешности результата 
измерений, такова, что $\mu(x/\sqrt{2}) \hm = \mu^2(x/2)$ в некоторой 
окрестности точки $x\hm=0$, то на всем носителе~$S$ выполнено 
$\mu(x)\hm=\exp [-x^2/(2\sigma^2)]$, где} $\sigma\hm>0$.
  
  \medskip
  
  \noindent
  Д\,о\,к\,а\,з\,а\,т\,е\,л\,ь\,с\,т\,в\,о\ \ леммы приведено в приложении к статье.
  
  \smallskip
  
  \noindent
  Д\,о\,к\,а\,з\,а\,т\,е\,л\,ь\,с\,т\,в\,о\ \ теоремы. \textit{Необходимость}. 
Покажем, что из условия $\max\limits_{\substack{{x\in S, y\in 
S,}\\{x+y=z}}}\{\mu(x)\mu(y)\}\hm= \mu(z/\sqrt{2})$ следует, что $\mu(x)\hm= 
e^{\alpha x^2/2}$. Введем функцию $h(x,y)\hm=\mu(x)\mu(y)$. Наложим 
условие $x\hm+y\hm=z$. Рассмотрим задачу поиска условного максимума 
$\varphi(z)\hm= \max\limits_{\substack{{x\in S, y\in S,}\\{x+y=z}}}\left\{\mu(x)\mu(y)\right\} \hm= 
\max\limits_{x\in S} \left \{ \mu(x)\mu(z-x)\right\}\hm=
\max\limits_{x\in S} \{h(x,z-x)\}$.
  
  Экстремумы функции $h(x,z-x)$ при заданном~$z$ достигаются при 
значениях аргумента $x\hm=x_0\in S$, при которых $h^\prime (x_0,z-
x_0)\hm=0$.
  
  Так как $h^\prime(x,z-x)\hm=\mu^\prime(x)\mu(z-x)\hm - \mu(x) 
\mu^{\prime}(z-x)$, то
\begin{gather}
  \mu^\prime(x_0) \mu(z-x_0) -\mu(x_0)\mu^\prime(z-x_0)=0\,;\notag\\
  \mu^\prime(x_0)\mu(z-x_0)=\mu(x_0)\mu^\prime (z-x_0)\,;\notag\\
  \fr{\mu^\prime(x_0)}{\mu(x_0)}=\fr{\mu^\prime(z-x_0)}{\mu(z-x_0)}\,.
  \label{e1sem}
  \end{gather}
  
  Очевидно, что равенство~(\ref{e1sem}) выполняется при $x_0\hm=z/2$, т.\,е.\ в точке с 
указанной абсциссой достигается экстремум функции $h(x,\,z-x)$.
  
  В силу четности функции $\mu(x)$ выполняется равенство $\mu(x)\hm=\mu(-
x)$. Соответственно, функция~$\mu^\prime(x)$ является нечетной и 
$\mu^\prime(x)\hm=-\mu^\prime(-x)$. Следовательно, 
$$
\fr{\mu^\prime(z-x_0)}{\mu(z-x_0)}= -\fr{\mu^\prime(x_0-z)}{\mu(x_0-z)}
$$ 
и  равенство~(\ref{e1sem}) приобретает вид:
  $$
  \fr{\mu^\prime(x_0)}{\mu(x_0)}= -\fr{\mu^\prime(x_0-z)}{\mu(x_0-z)}\,.
  $$
  
  Функция $\mu^\prime(x)/\mu(x)\hm>0$ при $\{x: x<0, x\in S\}$ и 
$\mu^\prime(x)/\mu(x)\hm<0$ при $\{x:\ x>0, x\in S\}$. Отсюда следует, что при 
$z\hm=0$ кривые $\mu^\prime(x)/\mu(x)$ и $-\mu^\prime(x)/\mu(x)$ 
пересекаются только в начале координат. Также отсюда следует, что 
существует такой интервал $[-a,\,a]\subset S$, $a\hm>0$, на котором функция 
$\mu^\prime(x)/\mu(x)$ является монотонно убывающей. Значит, функция 
$-\mu^\prime(x)/\mu(x)$ на этом интервале будет монотонно возрастать и при 
$\vert z \vert \hm\leq a$ уравнение ${\mu^\prime(x_0)}/{\mu(x_0)}\hm= -
{\mu^\prime(x_0-z)}/{\mu(x_0-z)}$ гарантированно имеет единственное 
решение. Как было отмечено выше, им является $x_0\hm=z/2$.
     
     Таким образом, при $\vert z\vert \hm\leq a$ имеем $\mu_\eta(z)\hm= 
\max\limits_{\substack{{x\in S, y\in S,}\\{x+y=z}}}\{ \mu(x)\mu(y)\}\hm= 
\mu(x_0)\mu(z-x_0)\hm= \mu^2(z/2)$. Так как по условию теоремы 
$\mu_\eta(z)\hm= \mu(z/\sqrt{2})$, то $\mu^2(z/2)\hm=\mu(z/\sqrt{2})$. 
В~соответствии с леммой, упомянутой выше, из этого следует, что 
аналитическая функция~$\mu(x)$ совпадает с нормированной гауссианой 
$\exp\left[ -x^2/(2\sigma^2)\right]$ при некотором значении параметра~$\sigma$ 
во всех точках носителя~$S$.
     
  Необходимость доказана.
  \smallskip
  
  \textit{Достаточность}. Покажем, что если $\mu(x)\hm= e^{- 
x^2/(2\sigma^2)}$, то $\max\limits_{\substack{{x\in S, y\in S,}\\{x+y=z}}}\{\mu(x) 
\mu(y)\} \hm= \mu(z/\sqrt{2})$.
  
  Выполним поиск максимума выражения 
  \begin{multline*}
  \max\limits_{\substack{{x\in S, y\in S,}\\{x+y=z}}}\{\mu(x)\mu(y)\}={}\\
  {}= \max\limits_{\substack{{x\in S, y\in S,}\\{x+y=z}}}
  \left\{ e^{-x^2/(2\sigma^2)} e^{-(z-x)^2/(2\sigma^2)}\right\}\,. 
  \end{multline*}
Искомый максимум достигается в точке максимума функции $f(x)\hm=-
x^2/(2\sigma^2)\hm- (z-x)^2/(2\sigma^2)$. Тогда 
  \begin{align*}
  \fr{df(x)}{dx}\Bigg\vert_{x=x_0}&=-\fr{x_0}{\sigma^2}+\fr{z-x_0}{\sigma^2}=0\,;\\
  x_0&=\fr{z}{2}\,.
  \end{align*}
  
  Таким образом, 
  $$
  \max\limits_{\substack{{x\in S, y\in  S,}\\{x+y=z}}}\{\mu(x)\mu(y)\}= \mu^2\left(\fr{z}{2}\right)\,.
  $$
  С~другой стороны, 
  \begin{multline*}
  \mu^2\left(\fr{z}{2}\right)= e^{-(z/2)^2/(2\sigma^2)} e^{-(z/2)^2/(2\sigma^2)}={}\\
  {}= e^{-(z/2)^2/\sigma^2}=e^{-(z/\sqrt{2})^2/(2\sigma^2)}=\mu\left(\fr{z}{\sqrt{2}}\right)\,.
  \end{multline*} 
Следовательно, 
$$
\max\limits_{\substack{{x\in S, y\in 
S,}\\{x+y=z}}}\{\mu(x)\mu(y)\}\hm= \mu\left(\fr{z}{\sqrt{2}}\right)\,.
$$ 

Достаточность доказана. 
  
  \smallskip
  
  Таким образом, для того чтобы предъявленные к разрабатываемому 
представлению требования были выполнены, необходимо, чтобы функции 
принадлежности нечетких переменных, выражающих характеристики 
погрешностей результатов измерений, имели вид криволинейной трапеции, 
боковые стороны которой являются левой и правой половинами 
нормированной гауссианы.
  
  В качестве следствия заметим, что результаты операций сложения и 
вычитания нечетких переменных по правилу 
$$
\mu_{\xi_1+\xi_2}(z)= 
\sup\limits_{\substack{{x\in S, y\in S,}\\{x+y=z}}}
\left\{\mu_{\xi_1}(x)\mu_{\xi_2}(y)\right\}
$$ 
в случае, когда операнды имеют функциями 
принадлежности нормированные гауссианы, также являются нормированными 
гауссианами, \textit{что обеспечивает унификацию операции сложения для 
случайной составляющей погрешности}.
  
  Заметим, что представленные результаты могут быть обобщены на случай, 
когда функция принадлежности случайной составляющей погрешности 
результата прямого измерения не является сим\-мет\-рич\-ной функцией. 

\subsection{Анализ требования~3} %4.3
    
    Пусть $\xi_1$ и~$\xi_2$~--- две нечеткие переменные с функциями 
принадлежности соответственно:
    $$
    \mu_{\xi_1}(x)=\begin{cases}
    1\,, &\ x\in [a_1,b_1]\,;\\
    0\,, &\ x\not\in [a_1,b_1]\,;
    \end{cases}
    $$
    $$
    \mu_{\xi_2}(x)=\begin{cases}
    1\,, & x\in [a_2,b_2]\,;\\
    0\,, & x\not\in [a_2,b_2]\,.
    \end{cases}
    $$
Переменные    $\xi_1$ и $\xi_2$ представляют собой систематические погрешности 
$\Delta_{\mathrm{сист}} x_1$ и $\Delta_{\mathrm{сист}}x_2$ результатов 
измерений~$x_1$ и~$x_2$.
  
  По итогам анализа требования~2 сделан вывод о том, что правилом, 
задающим функцию принадлежности суммы двух нечетких переменных, 
должно являться алгебраическое правило.
  
  Пусть $\eta=\xi_1+\xi_2$. Тогда функция при\-над\-леж\-ности нечеткой 
переменной~$\eta$ есть 

\noindent
  \begin{multline*}
  \mu_\eta (z) =\sup\limits_{\substack{{x\in S, y\in 
S,}\\{x+y=z}}}\{\mu_{\xi_1}(x)\mu_{\xi_2}(y)\} ={}\\
{}=
  \begin{cases}
  1\,, &\ x\in [a_1+a_2,\,b_1+b_2]\,;\\
  0\,, &\ x\not\in [a_1+a_2,\,b_2+b_2]\,.
  \end{cases}
  \end{multline*}
  \noindent
  Границы вложенного интервала $J_1[\eta]$ в точности совпадают с теми, 
которые могут быть получены при помощи интервальной арифметики. 
\textit{Таким образом, алгебраическое правило сохраняет форму функций 
принадлежности нечетких переменных, вы\-ра\-жа\-ющих систематическую 
составляющую погрешности, и позволяет унифицированно их обрабатывать}.
  
  Тот же вывод следует из предложенной интерпретации функции 
принадлежности погрешности, выраженной как нечеткая переменная. 
Поскольку $J_1[\xi_1]\hm=[a_1,b_1]$ и $J_1[\xi_2]\hm=[a_2,b_2]$ 
интерпретируются как такие интервалы, про которые известно, что они лежат 
внутри интервалов возможных значений выражаемых соответственно~$\xi_1$ 
и~$\xi_2$ погрешностей, то и про интервал $J_1[\xi_1+\xi_2]\hm=[a_1+a_2, 
b_1+b_2]$ известно, что он лежит внутри интервала возможных значений 
величины $\xi_1\hm+\xi_2$, что полностью согласуется с требованиями 
интервальной арифметики.

\subsection{Анализ требований~4 и~5} %4.4
    
    Пусть $\xi_{\mathrm{сист}}$~--- нечеткая переменная, функция 
принадлежности $\mu_{\xi_{\mathrm{сист}}}(x)$ которой равна~1 в любой 
точке интервала $[-\Delta_{\mathrm{сист}},\Delta_{\mathrm{сист}}]$ и нулю во 
всех точках вне данного отрезка. Пусть $\xi_{\mathrm{случ}}$~--- нечеткая 
переменная, функция принадлежности $\mu_{\xi_{\mathrm{случ}}}(x)$ 
которой является нормированной гауссианой, т.\,е.\ 
$\mu_{\xi_{\mathrm{случ}}}(x) \hm= \exp \left[-x^2/(2\sigma^2)\right]$. Тогда, 
очевидно, их сумма $\xi\hm=\xi_{\mathrm{сист}}+\xi_{\mathrm{случ}}$ имеет 
функцию принадлежности
\begin{multline*}
    \mu_\xi(z)=\max\limits_{\substack{{x\in S_{\xi_{\mathrm{сист}}}, y\in 
S_{\xi_{\mathrm{случ}}},}\\{x+y=z}}}\{\mu_{\xi_{\mathrm{сист}}}(x)
    \mu_{\xi_{\mathrm{случ}}}(y)\}= {}\\
    {}=
    \begin{cases}
    \mu_{\xi_{\mathrm{случ}}}\left(z+\Delta_{\mathrm{сист}}\right)=\exp\left[ -
\fr{(z+\Delta_{\mathrm{сист}})^2}{2\sigma^2}\right] & \\
&\hspace*{-40mm} \mbox{при}\ z\leq -
\Delta_{\mathrm{сист}}\,;\\
    \mu_{\xi_{\mathrm{сист}}}(z)=1 & \ \hspace*{-41mm}\mbox{при}\ -
\Delta_{\mathrm{сист}}\leq z\leq \Delta_{\mathrm{сист}}\,;\\
    \mu_{\xi_{\mathrm{случ}}}(z-\Delta_{\mathrm{сист}})=\exp\left[ -\fr{(z-
\Delta_{\mathrm{сист}})^2}{2\sigma^2}\right ] & \\
& \hspace*{-40mm}\mbox{при}\ z\geq 
\Delta_{\mathrm{сист}}\,.
    \end{cases}
    \end{multline*}
  
  Учитывая, что было решено считать вложенные интервалы~$J_1$ на уровне 
значимости $\alpha\hm=1$ отражающими сведения о систематической 
составляющей $\Delta_{\mathrm{сист}}x$ погрешности результата измерений, 
а функции принадлежности как у $\xi_{\mathrm{случ}}$~--- отражающими 
сведения о случайной составляющей $\Delta_{\mathrm{случ}}x$, то получаем, 
что нечеткая переменная $\xi\hm= \xi_{\mathrm{сист}}\hm+ 
\xi_{\mathrm{случ}}$ отражает сведения о погрешности в целом. Заметим, что 
при подобном представлении случайная и систематическая составляющие 
погрешности всегда разделяются, как это было отмечено при анализе 
требования~1. Данное свойство иллюстрирует рис.~2.
  
  \begin{figure*} %fig2
  \vspace*{1pt}
 \begin{center}
 \mbox{%
 \epsfxsize=158.687mm
 \epsfbox{sem-2.eps}
 }
 \end{center}
 \vspace*{-6pt}
  \Caption{Разложение нечеткой переменной, выражающей характеристики 
полной погрешности результата измерений~(\textit{а}), в сумму нечетких переменных, описывающих 
систематическую~(\textit{б}) и случайную~(\textit{в}) составляющие погрешности}
\vspace*{6pt}
  \end{figure*}
  
Таким образом, функцией принадлежности нечеткой переменной, 
отражающей сведения о погрешности результата измерений, в общем случае 
является функция, график которой представлен на рис.~1 и~2,\,\textit{а}.
  
  Заметим, что в общем случае функция вида $\mu_\xi(z)$ может быть 
полностью описана при помощи только трех параметров: границ интервала 
$J_1\hm=[a,\,b]$, во всех точках которого она принимает значения, равные~1, и 
параметром масштаба~$\sigma$, описывающим боковые гауссианы. Подобная 
естественная параметризация функции при\-над\-леж\-ности~$\mu_\xi(z)$ позволяет 
заменить алгебраическое правило сложения нечетких переменных, наивная 
реализация которого при вычислениях потребует выполнения достаточно 
большого числа операций, на набор простых правил для параметров.
  
  Действительно, пусть заданы две нечеткие переменные~$\xi_1$ и~$\xi_2$, 
представленные соответственно кортежами чисел $\langle 
[a_1,b_1],\,\sigma_1\rangle$ и $\langle[a_2,b_2],\,\sigma_2\rangle$ и выражающие 
соответственно погрешности~$\Delta x_1$ и~$\Delta x_2$ величин~$x_1$ 
и~$x_2$. Тогда нечеткая переменная $\xi_3\hm=\xi_1\hm+\xi_2$ будет 
характеризоваться кортежем $\langle [a_3,b_3],\,\sigma_3\rangle$, таким что
  \begin{align*}
  a_3&=a_1+a_2\,;\\
  b_3&=b_1+b_2\,;\\
  \sigma_3&=\sqrt{\sigma_1^2+\sigma_2^2}\,,
  \end{align*}
  в чем несложно убедиться, воспользовавшись правилом 
$\mu_{\xi_1+\xi_2}(z)\hm= \sup\limits_{\substack{{x\in S, y\in 
S,}\\{x+y=z}}}\{\mu_{\xi_1}(x)\mu_{\xi_2}(y)\}$. 
  
  Заметим, что представленные соотношения для параметров в точности 
совпадают с правилом вы\-чис\-ле\-ния границ $[a_3,\,b_3]$ интервала допускаемых 
значений для суммы $\Delta_{\mathrm{сист}}x_1+\Delta_{\mathrm{сист}}x_2$ 
систематических составляющих погрешности величин~$x_1$ и~$x_2$ и 
правилом вычисления среднеквадратического значения~$\sigma_3$ суммы 
$\Delta_{\mathrm{случ}}x_1+\Delta_{\mathrm{случ}}x_2$ случайных 
составляющих погрешности. Таким образом, \textit{данные правила в точности 
реализуют общепринятую методику обработки характеристик по\-греш\-ности 
результатов измерений}. Поскольку вычисления предлагается выполнять 
напрямую с параметрами $\langle [a,b],\,\sigma\rangle$, то \textit{использование 
аппарата нечетких переменных не повлечет увеличения времени, 
затрачиваемого на математическую обработку результатов измерений}.
  
  Для случая масштабирования нечеткой переменной, т.\,е.\ для 
$\xi_3\hm=c\xi_1$, где $c$~--- некоторое действительное число, отличное от 
нуля, результат~$\xi_3$ будет характеризоваться кортежем $\langle 
[a_3,b_3],\,\sigma_3\rangle$, где
  \begin{align*}
  a_3&=\min\left( ca_1,\,cb_1\right)\,;\\
  b_3&=\max\left( ca_1,\,cb_1\right)\,;\\
  \sigma_3&=\vert c\vert \sigma_1\,.
  \end{align*}
  
  Действительно, масштабирование нечеткой переменной, по сути, есть 
масштабирование ее носителя, т.\,е.~$\xi_3$ будет соответствовать функция 
принадлежности $\mu_{\xi_1}(c x)$. Представленные правила также в 
точности повторяют соотношения, принятые в метрологической практике.
  
  \textit{Рассмотренного набора линейных операций достаточно для 
применения разработанного представления при решении такой задачи, как 
автоматический контроль погрешности результатов вычислений при 
математической обработке результатов измерений на основе линеаризации 
вычисляемой функции}~\cite{16sem}.
  
  Таким образом, достигается заявленная в начале работы цель.
  
\section{Выводы}
     
     Подытоживая, заметим, что в случае, когда требуется выполнить только 
линейные преобразования с результатами измерений и характеристиками их 
погрешностей, представление погрешности как нечетких переменных не 
затрудняет вычислений. Представленные в предыдущем разделе правила 
работы с параметрами $\langle [a,b],\,\sigma\rangle$ повторяют правила для 
систематической составляющей погрешности, принятые в интервальной 
арифметике, и правила работы с дисперсией и среднеквадратическими 
отклонениями, принятые в теории вероятностей для независимых случайных 
величин. Сохранение вида функции принадлежности для результатов линейных 
операций позволяет унифицировать представление характеристик погрешности 
нечеткими переменными. \textit{Таким образом, представление погрешности 
как нечеткой переменной согласуется с}~\cite{8sem} \textit{и отечественной 
метрологической нормативной базой}.


\vspace*{12pt}


\setcounter{equation}{0}
\renewcommand{\theequation}{П.\arabic{equation}}

{{\small \hfill \textbf{ПРИЛОЖЕНИЕ}}}

\vspace*{-24pt}

{\small

\section*{Доказательство леммы}
  
  Условие леммы позволяет указать следующие свойства функции~$\mu(x)$.
  
  Так как $\mu(x)$ является функцией при\-над\-леж\-ности нечеткой переменной, 
выражающей сведения о случайной погрешности результата измерений,  
следовательно, она является четной относительно прямой $x\hm=0$ функцией и 
при этом $\mu(x)\hm=1$ только при $x\hm=0$. Аналитическая 
функция~$\mu(x)$ не имеет разрывов и является достаточно гладкой во всех 
точках своего носителя~$S$. В~точ\-ке $x\hm=0$ достигается ее единственный 
экстремум.
  
  Из равенства $\mu^2(k_1 x)\hm=\mu(k_2 x)$,  $k_1\hm>0$, 
$k_2\hm>0$, следует, что функция~$\mu(x)$ либо тождественно равна~1, либо 
имеет бесконечное число ненулевых слагаемых в ее разложении в ряд 
Маклорена. Иными словами, $\exists \mu^{(i)}(0)$ $\forall i\in N$ и при этом 
$\forall i\in N:$ $\mu^{(i)}(0)\hm=0$ $\exists j\in N$, $j\hm>i:$ 
$\mu^{(i)}(0)\not=0$.
  
  Действительно, пусть 
  $$
  \mu(x)=\sum\limits_{n=0}^N \fr{\mu^{(n)}(0)}{n!}\, 
x^n\,,
$$
где $N$~--- некоторое натуральное число либо ноль и 
$\mu^{(N)}(0)\not=0$. Тогда 

\noindent
$$
\mu^2(k_1 x) = \sum\limits_{n=0}^{2N} 
a_n k_1^n x^n\,,
$$
причем $a_{2N}\not=0$. В~то же время 
$$\mu(k_2 x)  =\sum\limits_{n=0}^N \fr{k_2^n\mu^{(n)}(0)}{n!}\,x^n\,.$$ 
Из условия леммы 
следует, что функция~$\mu(x)$ константой в области своего носителя~$S$ не 
является; следовательно, $N\hm>0$. Таким образом, функция $\mu^2(k_1 
x)$ является полиномом степени $2N$, а функция $\mu(k_2 x)$~--- 
полиномом степени~$N$. Но поскольку $\mu^2(k_1 x)\hm=\mu(k_2 x)$ 
и $N\hm>0$, получаем противоречие. Следовательно, $N\hm=\infty$. 
  
  Поскольку функция $\mu(x)$ является четной относительно значения 
$x\hm=0$ своего аргумента, $\mu^{(n)}(0)\not=0$ только если $n$~--- 
натуральное четное число, т.\,е.\ в разложении~$\mu(x)$ в ряд Маклорена будут 
ненулевыми только слагаемые при четных индексах: 
$$
\mu(x)= \sum\limits_{n=0}^\infty \fr{\mu^{(2n)}(0)}{(2n)!}\,x^{2n}\,.
$$
  \begin{enumerate}[1.]
  \item Согласно теореме Лейбница о значении высшей производной 
произведения достаточно гладких функций получаем:
  \begin{equation*}
  \fr{d^n(\mu^2(k_1 x))}{dx^n}\Bigg\vert_{x=0}=
k_1^n\sum\limits_{m=0}^m 
C_n^m \mu^{(m)}(0) \mu^{(n-m)}(0)\,.
  \end{equation*} 
  С~другой стороны, в силу равенства $\mu^2(k_1 x)\hm= \mu(k_2 x)$ 
имеем: 
  \begin{equation*}
  \fr{d^n(\mu^2(k_1 x))}{dx^n}\Bigg\vert_{x=0}=\fr{d^n(\mu(k_2 
x))}{dx^n}\Bigg\vert_{x=0}=
k_2^n \mu^{(n)}(0)\,,
  \end{equation*} 
  откуда следует, что 
  \begin{equation}
  \mu^{(n)}(0)=k^n\sum\limits_{m=0}^n C_n^m \mu^{(m)}(0)\mu^{(n-
m)}(0)\,,
  \label{e2sem}
\end{equation}
где
$$
  k=\fr{k_1}{k_2}=\fr{1}{\sqrt{2}}\,.
  $$
  
  Видно, что значения всех производных $d^n\mu(x)/dx^n$ четных 
порядков~$n$ в точке $x\hm=0$ связаны со значением $\mu^{\prime\prime}(0)$ 
простыми алгебраическими соотношениями, например $\mu^{(4)}(0)\hm= 
3\mu^{\prime\prime\,2}(0)$. Отсюда также следует, что 
$\mu^{\prime\prime}\not=0$, в противном случае $\mu(x)\hm\equiv 0$, чего быть 
не может по условиям доказываемого утверждения.
  \item  Рассмотрим функцию $h(x)\hm= \mu^\prime(x)/\mu(x)$ при $x\hm\in S$. 
Она является нечетной функцией относительно значения $x\hm=0$ своего 
аргумента, и, следовательно, $h^{(2n)}(0)\hm=0$ при всех натуральных~$n$, 
т.\,е.\ только производные нечетных порядков функции~$h(x)$ имеют 
ненулевое значение при $x\hm=0$.
  
  Обозначим через $\{a_n\}^{n=\infty}_{n=0}$ коэффициенты в разложении в 
ряд Маклорена функции~$\mu^\prime(x)$, через $\{b_n\}^{n=\infty}_{n=0}$~--- 
коэффициенты в разложении в ряд Мак\-ло\-ре\-на функции~$\mu(x)$, а через 
$\{q_n\}^{n=\infty}_{n=0}$~--- коэффициенты в разложении в ряд Мак\-ло\-ре\-на 
функции~$h(x)$. Как было отмечено выше, $a_{2i}\hm=0$ и $b_{2i+1}\hm=0$ 
при всех $i\in N \cup \{0\}$. Таким образом, коэффициенты~$b_n$ 
образуют последовательность $\{1, 0, {\mu^{(2)}(0)}/{2!},0,
  {\mu^{(4)}(0)}/{4!}, 0, \ldots\}$, а коэффициенты~$a_n$, соответственно, 
образуют последовательность $\{0, {\mu^{(2)}(0)}/{1!}, 0, 
{\mu^{(4)}(0)}/{3!}, 0, \ldots \}$, получаемую из почленного 
дифференцирования ряда Маклорена для функции~$\mu(x)$. 
  
  Покажем методом математической индукции, что все производные нечетных 
порядков   при натуральных $n\hm>1$, т.\,е.\ все производные 
функции~$h(x)$, начиная с третьей, равны нулю в точке $x\hm=0$. Для этого 
достаточно показать, что все коэффициенты $q_n\hm=0$ при $n\geq 3$.
  
  Воспользуемся обращением формулы коэффициентов произведения 
степенных рядов для получения значений коэффициентов~$q_n$:
\begin{description}
%\smallskip
  \item[\,]
  $q_0$: \ $a_0=q_0b_0$, откуда получаем $q_0\hm=a_0/b_0\hm=0$;
  \item[\,]  
  $q_1$: \ $a_1=q_0b_1\hm+q_1b_0$, откуда получаем 
$q_1\hm=a_1/b_0=\mu^{\prime\prime}(0)$;
    \item[\,]
  $q_2$:\: $a_2=q_0b_2\hm+q_1b_1\hm+q_2b_0$, откуда получаем $q_2\hm= 
(a_2-q_1b_1)/b_0\hm=0$;
    \item[\,]
  $q_3$:\  $a_3=q_0b_3\hm+q_1b_2\hm+q_2b_1\hm+q_3b_0$, откуда получаем
  \end{description} 
$$
  \fr{\mu^{(4)}(0)}{6}=\fr{\mu^{\prime\prime}(0)}{2}\,\mu^{\prime\prime}(0)+q
_3\,,\ q_3=\fr{\mu^{(4)}(0)}{6}-\fr{\mu^{\prime\prime\,2}(0)}{2}\,.
  $$
%  \end{description}
  Но поскольку $\mu^{(4)}(0)\hm= 3\mu^{\prime\prime\,2}(0)$, как отмечалось 
в п.~1 настоящего доказательства, $q_3\hm=0$.
  
  \textit{База индукции}: показано, что $h^{(3)}(0)\hm=0$.
  
  \textit{Индукционный переход}: предположим, что доказываемое утверждение 
верно для нечетного $n\hm=2r\hm+1$, где $r\hm\geq 2$~--- натуральное чис\-ло, 
т.\,е.\ выполнено $q_n\hm=0$ или, что то же, $h^{(n)}(0)\hm=0$. Рассмотрим 
утверждение для $n\hm+2$:
\begin{description}
%\smallskip
    \item[\,]
  $q_n$:\ $q_n=0$;
    \item[\,]
  $q_{n+1}$:\ $q_{n+1}=0$;
    \item[\,]
  $q_{n+2}$:\ $a_{n+2}=\sum\limits_{m=0}^{n+2} q_m b_{n-
m+2}\hm=q_1b_{n+1}\hm+q_{n+1}b_0$, откуда получаем 
\begin{multline*}
  q_{n+2} = \fr{a_{n+2}-q_1 b_{n+1}}{b_0}= a_{n+2} - 
\mu^{\prime\prime}(0)b_{n+1}={}\\
{}=
(n+3)b_{n+3}-\mu^{\prime\prime}(0)b_{n+1}={}\\
{}=
  \fr{\mu^{(n+3)}(0)}{(n+2)!}-
\mu^{\prime\prime}(0)\fr{\mu^{(n+1)}(0)}{(n+1)!}\,.
  \end{multline*}
  \end{description}
    Так как $q_n=0$, то $a_n\hm= \sum\limits_{m=0}^n q_m b_{n-m}\hm= 
q_1b_{n-1}$, откуда 
  $$
  \fr{\mu^{(n+1)}(0)}{n!}=\mu^{\prime\prime}(0)\fr{\mu^{(n-1)}(0)}{(n-1)!}\,.
  $$ 
  Следовательно, $\mu^{(n-1)}(0)\hm=\mu^{\prime\prime}(0)\mu^{(n-1)}(0)n$. 
Продолжая по аналогии, получим, что 
  \begin{equation}
  \mu^{(n+1)}(0) ={\mu^{\prime\prime}}^{(n+1)/2}(0) 
\prod\limits_{j=0}^{(n+1)/2-1} (2j+1)\,.
  \label{e3sem}
  \end{equation}
  
  Рассмотрим полученное в п.~1 настоящего доказательства 
соотношение~(\ref{e2sem}) для показателя порядка $n+3$:

\noindent
 \begin{multline*}
  \mu^{(n+3)}(0)={}\\
  {}=\fr{1}{(\sqrt{2})^{n+3}}\sum\limits_{m=0}^{n+3} C^m_{n+3} 
\mu^{(m)}(0) \mu^{(n-m+3)}(0)\,.
  \end{multline*}
  
  Отделим от суммы, стоящей в правой части равенства, первое и последнее 
слагаемое:
  \begin{multline*}
  \mu^{(n+3)}(0) ={}\\
  {}=
  \fr{1}{2^{(n+1)/2+1}} \sum\limits_{m=1}^{n+2} 
C^m_{n+3}\mu^{(m)}(0) \mu^{(n-m+3)}(0)+{}\\
{}+ 2\left( \fr{1}{2^{(n+1)/2+1}}\,\mu^{(n+3)}(0)\right)\,;
  \end{multline*}
  
\vspace*{-13pt}

  \noindent
  \begin{multline*}
  \left( 1-\fr{1}{2^{(n+1)/2}}\right) \mu^{(n+3)}(0)={}\\
  {}=\fr{1}{2^{(n+1)/2+1}}  \sum\limits_{m=1}^{n+2} C^m_{n+3}\mu^{(m)}(0)\mu^{(n-m+3)}(0)\,;
  \end{multline*}
  
  \vspace*{-13pt}
  
  \noindent
  \begin{multline*}
  \left( 2^{(n+1)/2}-1\right) \mu^{(n+3)}(0)={}\\
  {}=\fr{1}{2}\sum\limits_{m=1}^{n+2} 
C^m_{n+3} \mu^{(m)}(0) \mu^{(n-m+3)}(0)\,.
  \end{multline*}
  
  Поскольку $\mu^{(k)}(0)\hm=0$ при всяком нечетном~$k$, удалим из суммы 
нулевые слагаемые:

\noindent
  \begin{multline*}
  \left( 2^{(n+1)/2}-1\right) \mu^{(n+3)}(0) ={}\\
  {}=\fr{1}{2}\sum\limits_{m=1}^{(n+1)/2} C_{n+3}^{2m} \mu^{(2m)}(0)\mu^{(n-
2m+3)}(0)\,.
  \end{multline*}
  
  Подставим~(\ref{e3sem}) в полученное соотношение:
  
  \noindent
  \begin{multline*}
  \left( 2^{(n+1)/2}-1\right) 
\mu^{(n+3)}(0)=
\fr{1}{2}{\mu^{\prime\prime}}^{(n+1)/2+1}(0) \times{}\\
{}\times
\sum\limits_{m=1}^{(n+1)/2} C^{2m}_{n+3} \prod\limits_{j=0}^{m-1} (2j+1) 
\prod\limits_{s=0}^{(n+1)/2-m} (2s+1)\,.
  \end{multline*}
  Поскольку 
  \begin{multline*}
  C^{2m}_{n+3} \prod\limits_{j=0}^{m-1}(2j+1) \prod\limits_{s=0}^{(n+1)/2-
m}(2s+1) ={}\\
{}= (n+3)!\fr{\prod\limits_{j=0}^{m-1}(2j+1)}{(2m)!}\, 
\fr{\prod_{s=0}^{(n+1)/2-m}(2s+1)}{(n-2m+3)!}= {}\\
\hspace*{-2.5pt}{}=\hspace*{-0.8pt}(n+3)! \fr{1}{2^m m!}\,\fr{1}{2^{(n+1)/2-m+1}((n+1)/2-m+1)!}={}\\
{}=\fr{(n+3)!}{2^{(n+1)/2+1} m!((n+1)/2-m+1)!}\,,
  \end{multline*}
  получим:
  
  \pagebreak
  
  \noindent
  \begin{multline*}
  \left( 2^{(n+1)/2}-1\right) \mu^{(n+3)}(0)=
  \fr{1}{2^{(n+1)/2+2}}\times{}\\
  {}\times 
{\mu^{\prime\prime}}^{(n+1)/2+1}(0)\sum\limits_{m=1}^{(n+1)/2} \fr{(n+3)!}
  {m!((n+1)/2-m+1)!}\,;
  \end{multline*}

\vspace*{-12pt}

\noindent
  \begin{multline*}
  \left( 2^{(n+1)/2}-1\right) \mu^{(n+3)}(0)= {}\\
  {}=\fr{1}{2^{(n+1)/2+2}}\, 
{\mu^{\prime\prime}}^{(n+1)/2+1}(0) \fr{(n+3)!}{((n+1)/2+1)!} \times{}\\
{}\times
\sum\limits_{m=1}^{(n+1)/2} \fr{((n+1)/2+1)!}{m!((n+1)/2-m+1)!}\,;
  \end{multline*}
  
  \vspace*{-12pt}
  
  \noindent
  \begin{multline*}
  \left( 2^{(n+1)/2}-1\right) \mu^{(n+3)}(0)= {}\\
  {}=\fr{1}{2^{(n+1)/2+2}}\, 
{\mu^{\prime\prime}}^{(n+1)/2+1}(0) \fr{(n+3)!}{((n+1)/2+1)!} \times{}\\
{}\times
\sum\limits_{m=1}^{(n+1)/2} C^m_{(n+1)/2+1}\,.
  \end{multline*}
  
  Так как согласно биному Ньютона 
  $$
  (1+1)^n-1^n-1^n= \sum\limits_{m=1}^{n-1} C_n^m= 2^n-2
  $$ 
  и так как 
  $$
  \fr{(n+3)!}{((n+1)/2+1)!}=2^{(n+1)/2+1}\prod\limits_{j=0}^{(n+1)/2}(2j+1)\,,
  $$
  получим
  \begin{multline*}
  \left( 2^{(n+1)/2}-1\right) \mu^{(n+3)}(0)={}\\
  {}= \fr{1}{2^{(n+1)/2+2}}\, 
{\mu^{\prime\prime}}^{(n+1)/2+1}(0) \left( 2^{(n+1)/2+1}-2\right)\times{}\\
{}\times  2^{(n+1)/2+1} 
\prod\limits_{j=0}^{(n+1)/2}(2j+1)\,;
  \end{multline*}
  
  \vspace*{-12pt}
  
  \noindent
  \begin{equation*}
  \mu^{(n+3)}(0) ={\mu^{\prime\prime}}^{(n+1)/2+1}(0) 
\prod\limits_{j=0}^{(n+2)/2}(2j+1)\,.
  \end{equation*}
  
  Значит, 
  \begin{multline*}
  q_{n+2}= \fr{\mu^{(n+3)}(0)}{(n+2)!}- \mu^{\prime\prime}(0)
  \fr{\mu^{(n+1)}(0)}{(n+1)!}={}\\
  {}= \fr{{\mu^{\prime\prime}}^{(n+1)/2+1}(0)}{(n+2)!} 
\prod\limits_{j=0}^{(n+1)/2}(2j+1)- {}\\
{}-
\fr{{\mu^{\prime\prime}}^{(n+1)/2+1}(0)}{(n+1)!} \prod\limits_{j=0}^{(n-1)/2} 
(2j+1) ={}\\
  {}= \left( 2\fr{n+1}{2}+1\right) 
\fr{{\mu^{\prime\prime}}^{(n+1)/2+1}(0)}{(n+2)!} \prod\limits_{j=0}^{(n-1)/2} 
(2j+1)-{}\\
{}- \fr{{\mu^{\prime\prime}}^{(n+1)/2+1}(0)}{(n+1)!} 
  \prod\limits_{j=0}^{(n-1)/2} (2j+1)=0\,.
  \end{multline*}
  Таким образом, методом математической индукции показано, что все 
коэффициенты~$q_n$ разложения в ряд Маклорена функции $h(x)\hm= 
\mu^\prime(x)/\mu(x)$ с индексами $n\hm\geq 2$ будут нулевыми.
  \item  Получаем, что исходным условиям леммы удовле\-тво\-ря\-ют решения 
дифференциального уравнения $\mu^\prime(x)/\mu(x)\hm=\alpha x$, где 
$\alpha\hm= \mu^{\prime\prime}(0)$. Найдем их:
  \begin{gather*}
  \fr{d\mu(x)}{\mu(x)}=\alpha x \,dx\,;\\
  \ln \left( \mu(x)\right) =\fr{\alpha x^2}{2}+c\,,
  \end{gather*}
  где $c=const(x)$;  $\mu(x)\hm= e^{(\alpha x^2)/2+c}$.

     
     Поскольку $\mu(0)=1$, то константа $c\hm=0$ и $\mu(x)\hm= e^{\alpha 
x^2/2}$, т.\,е.\ является нормированной гауссианой и описывает семейство 
показательных функций от квадрата аргумента. Из условия существования 
максимума при $x\hm=0$ получаем, что 
$\alpha\hm=\mu^{\prime\prime}(0)\hm<0$. Поскольку степенной ряд 
$\sum\limits_{n=0}^\infty b_n x^n$ имеет бесконечный радиус схо\-ди\-мости, то в 
силу своего аналитического характера функция $\mu(x)\hm= e^{\alpha x^2/2}$ 
на всем своем носителе~$S$.
     \end{enumerate}
     
  Лемма доказана.
  
  }
  
  {\small\frenchspacing
{%\baselineskip=10.8pt
\addcontentsline{toc}{section}{Литература}
\begin{thebibliography}{99}
    

\bibitem{2sem} %1
\Au{Gonella L.}
Proposal for a revision of the measure theory and terminology~// Alta Frequenza, 
1975. Vol.~XLIV. No.\,10.

\bibitem{3sem} %2
\Au{Destouches J.\,L., Fevrier P.}
New trends in expressing results of measurements~// IMEKO Colloquium 
Proceedings.~--- Budapest, 1980.

\bibitem{4sem} %3
\Au{Mari L.}
Notes on Fuzzy set theory as a tool for the measurement theory~// Fundamental 
metrology, measurement theory and education: XII IMEKO World Congress 
Proceedings.~--- Beijing, China. 1991. Vol.~III. P.~70--74.

\bibitem{1sem} %4
\Au{Reznik L.\,K., Jonson W.\,C., Solopchenko~G.\,N.}
Fuzzy interval as a basis for measurement theory~// NASA Conference NAFIPS'94 
Proceedings.~--- San-Antonio, Texas, 1994. P.~405--406.

\bibitem{5sem}
\Au{Резник Л.\,К.}
Математическое обеспечение обработки нечеткой информации 
экспериментатора в ИВК~// Архитектура, модели и программное обеспечение 
ИИС и ИВК: Труды ВНИИЭП.~--- Л.: ВНИИЭП, 1983. С.~45--55.

\bibitem{6sem}
\Au{Reznik L.}
Measurement result uncertainty evaluation: New soft approaches?~// Мягкие 
вычисления и измерения: Сб. трудов междунар. научн. конф. SCM-1999.~--- 
СПб., 1999. С.~21--24.

\bibitem{7sem}
\Au{Брусакова И.\,А.}
Neuro и Fuzzy информационно-из\-ме\-ри\-тель\-ные технологии для анализа 
априорных знаний интеллектуальных измерительных средств~// Мягкие 
вычисления и измерения: Сб. трудов междунар. научн. конф. SCM-2003.~--- 
СПб., 2003. С.~27--32.

\bibitem{8sem}
Руководство по выражению неопределенности измерения~/ Пер. с англ. под 
ред. В.\,А.~Слаева.~--- СПб.: ВНИИМ им.\ Д.\,И.~Менделеева, 1999.

\bibitem{9sem}
МИ 1317-2004. Государственная система обеспечения единства измерений. 
Результаты и характеристики погрешности измерений. Формы представления. 
Способы использования при испытаниях образцов продукции и контроле их 
параметров: Издание официальное.~--- М.: Изд-во стандартов, 2004.

\bibitem{10sem}
\Au{Hung T.~Nguen, Kreinovich~V.}
Nested intervals and sets: Concepts, relations to fuzzy sets, and applications~// 
Application of interval computation~/ Eds. R.\,B.~Kearfott, V.~Kreinovich.~--- 
Dordrecht--Boston--London: Kluver Academic Publs., 1996. P.~245--290.

\bibitem{11sem}
\Au{Борисов А.\,Н., Алексеев А.\,В., Меркурьева~Г.\,В. и~др.}
Обработка нечеткой информации в системах принятия решений.~--- М.: Радио 
и связь, 1989. 304~с.

\columnbreak
\bibitem{12sem}
\Au{Пономарев А.\,С.}
Нечеткие множества в задачах автоматизированного управления и принятия 
решений.~--- Харьков: НТУ ХПИ, 2005. 232~с.

\bibitem{13sem}
\Au{Яхъяева Г.\,Э.}
Нечеткие множества и нейронные сети.~--- М.: 
Ин\-тер\-нет-уни\-вер\-си\-тет информационных технологий, Бином. 
Лаборатория знаний, 2006. 316~с.

\bibitem{14sem}
\Au{Солопченко Г.\,Н.}
Представление измеряемых величин и погрешностей измерений как нечетких 
переменных~// Измерительная техника, 2007. №\,2. C.~3.

\bibitem{15sem}
\Au{Hung T.\,N., Kreinovich~V., Chin-Wang T., Solopchenko~G.\,N.}
Why two sigma? A~theoretical justification~// Soft computing in measurement and 
information acquisition~/ Eds. L.~Reznik, V.~Kreinovich.~--- Berlin--Heidelberg: 
Springer-Verlag, 2003. P.~10--22.

\label{end\stat}

\bibitem{16sem}
\Au{Семенов К.\,К., Солопченко~Г.\,Н.}
Теоретические предпосылки реализации метрологического автосопровождения 
программ обработки результатов измерений~// Измерительная техника, 2010. 
№\,6. C.~9--14.
 \end{thebibliography}
}
}
\end{multicols}