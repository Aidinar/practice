\documentclass[10pt]{book}
\usepackage[utf8]{inputenc}

\usepackage{latexsym,amssymb,amsfonts,amsmath,indentfirst,shapepar,%fleqn,%
picinpar,shadow,floatflt,enumerate,multicol,colortbl,ipi}

\usepackage{rotating}
\usepackage{mathrsfs}

\input{epsf}

%\nofiles

%\includeonly{avtor,avtor-eng} %+pdf
%\includeonly{avtor-eng}
%\includeonly{pred}      %+pdf
%\includeonly{podgot-2str}  %+
%\includeonly{ocherk} %+

%\includeonly{pavlov}       %10+pdf
%\includeonly{shest}        %13+pdf
%\includeonly{dulin}        %4pdf
%
%\includeonly{markova}      %9+pdf
%\includeonly{karpov}      %8 pdf
%\includeonly{sharnin}     %12+pdf
%\includeonly{agalarov}    %1pdf
%\includeonly{zhevn}       %5pdf
%\includeonly{kalin}       %7+pdf
%\includeonly{kalenov}     %6+pdf
%\includeonly{semenov}     %11+pdf
%\includeonly{bosov}       %2+pdf
%\includeonly{gorsh}       %3pdf


%\includeonly{toc-rus, toc-en}
%\includeonly{obchak,toc-en}

%\includeonly{obchak}
%\includeonly{reshal}  %pdf
%\includeonly{eng-index}
%\includeonly{cover3}

\usepackage{acad}
\usepackage{courier}
\usepackage{decor}
\usepackage{newton}
\usepackage{pragmatica}
\usepackage{zapfchan}
\usepackage{petrotex}
\usepackage{bm}                     % полужирные греческие буквы
\usepackage{upgreek}                % прямые греческие буквы
\usepackage{eufrak}
%\usepackage{verbatim}

\renewcommand{\bottomfraction}{0.99}
\renewcommand{\topfraction}{0.99}
\renewcommand{\textfraction}{0.01}

\setcounter{secnumdepth}{1} %здесь - 3 + chapter = 4

\arraycolsep=1.5pt

%\usepackage[pdftex]{graphicx}

%\usepackage{oz}

%NEW COMMANDS


\renewcommand*{\hm}[1]{#1\nobreak\discretionary{}%
            {\hbox{$\mathsurround=0pt #1$}}{}} %% Дублирует знаки операций
                               %при переносе в формуле (перед знаком, который 
                               %надо продублировать ставится команда \hm)


\renewcommand{\r}{\mathbb{R}}
\newcommand{\I}{{\rm I\hspace{-0.7mm}I}}
%\newcommand{\Ikl}{{\tt{1}}\hspace*{-1.44mm}\mathtt{1}}
\newcommand{\Ik}{\mbox{{\small \tt {1}}\hspace{-1.5mm}{\tt 1}}}
\newcommand{\argmin}{\mathop{\mathrm{arg}\,\mathrm{min}}}
%\def\argmin{\mathop{arg\,min}}

\def\vrp{\varphi}
\def\prt{\partial}
\def\mm{{\rm M}}

\newcommand{\il}[2]{\int\limits_{#1}^{#2}}%интеграл с пределами #1 и #2

\def\sss{\sum\limits}
\def\tr{\,,\,\ldots\,,\,}
\def\rk{\right]}
\def\lk{\left[}
\def\rf{\right\}}
\def\lf{\left\{}

\def\ee{{\cal E}}
\def\ww{{\cal W}}
\def\yy{{\cal Y}}
\def\vv{{\cal V}}

\newcommand{\R}{\mathbb R}
\newcommand{\N}{\mathbb N}

\newcommand{\h}{{\bf H}}
\newcommand{\p}{{\sf P}}  % вероятность
\newcommand{\e}{{\sf E}}  % мат. ожидание
\newcommand{\D}{{\sf D}}  % дисперсия
\newcommand{\eps}{\varepsilon}
\newcommand{\vp}{{\mathbf p}}
\newcommand{\vz}{{\mathbf z}}
\newcommand{\vx}{{\mathbf x}}
\newcommand{\vf}{{\mathbf f}}
%\newcommand{\vp}{\mathrm{v.p.}}
\newcommand{\F}{{\mathcal F}}
\def\ap{{\mathrm{ЭР}}}

\newcommand{\abs}[1]{\left\vert#1\right\vert}
\def\w{\omega}
\def\W{\Omega}
\def\iii{\int\limits}
\def\iin{\int\limits_{-\infty}^\infty}

\DeclareMathOperator{\sign}{sign}

%\newcommand{\gr}{{\geqslant}}

\newcommand{\g}{\mbox{\textit{g}}}

\renewcommand{\la}{\lambda}
\newcommand{\si}{\sigma}
\newcommand{\alp}{\alpha}

%\newcommand{\pto}{\stackrel{P}{\longrightarrow}} % сходимость по веpоятности

%\newcommand{\eqd}{\stackrel{d}{=}} % равенство по pаспpеделению

%\newcommand{\kp}{\kappa}
%\def\Q{{\cal Q}} \def\H{{\cal H}}
%\newcommand{\bet}{\beta_{2+\delta}}


%\newtheorem{definition}{Определение}
%\renewcommand{\thedefinition}{\arabic{definition}.}
%END NEW COMMANDS

%\renewcommand{\baselinestretch}{1.2}

%\pagestyle{myheadings}

\setlength{\textwidth}{167mm}      % 122mm
\setlength{\textheight}{658pt}
%\setlength{\textheight}{635.6pt}
\setlength{\columnsep}{4.5mm}

\setcounter{secnumdepth}{4}

%\addtolength{\headheight}{2pt}
%\addtolength{\headsep}{-2mm}

%\addtolength{\topmargin}{-20mm}  % for printing


\hoffset=-30mm  % From Yap
%\hoffset=-20mm  % From Acrobat

%\voffset=0mm % From Yap
%\voffset=-15mm   % From Acrobat

\addtolength{\evensidemargin}{-9.5mm} % for printing
\addtolength{\oddsidemargin}{9.5mm}  % for printing

%\renewcommand{\thefootnote}{\fnsymbol{footnote}}
%\renewcommand{\thefootnote}{\arabic{footnote}}
\renewcommand{\figurename}{\protect\bf Рис.}
\renewcommand{\tablename}{\protect\bf Таблица}

\newcommand{\Caption}[1]{\caption{\protect\small %\baselineskip=2.5ex
#1}}

\renewcommand{\thefigure}{\arabic{figure}}
\renewcommand{\thetable}{\arabic{table}}
\renewcommand{\theequation}{\arabic{equation}}
\renewcommand{\thesection}{\arabic{section}}

\renewcommand{\contentsname}{СОДЕРЖАНИЕ}
\newcommand{\fr}[2]{\displaystyle\frac{\displaystyle #1\mathstrut}{\displaystyle #2\mathstrut}}

%\renewcommand{\thefootnote}{\fnsymbol{footnote}}
%\newcommand{\g}{\mbox{\textit{g}}}

%\newcommand{\Caption}[1]{\caption{\protect\small\baselineskip=2ex #1}}
\newcounter{razdel}
\setcounter{razdel}{0}


\newcommand{\titel}[4]{%
\

\vspace*{5pt}

\ifodd\therazdel {\raggedright\noindent\Large\textrm\textbf
 \lineskip .75em
  \baselineskip=3.2ex #1 \par}
\vskip 1em {\noindent\large\textrm\textbf #2 \par}
\addcontentsline{toc}{subsection}{{\textrm\textbf #3}\protect\newline #1}
\def\rightheadline{\underline{\noindent\hbox to \textwidth{\hfill\small\textrm{#4}
%\hfill \large\bf\thepage
}}}
\def\leftheadline{\underline{\noindent\parbox{\textwidth}{
%\raggedleft\large\bf\thepage \hfill
\small\textit{#3}\hfill}}}
\def\leftfootline{\small{\textbf{\thepage}
\hfill ИНФОРМАТИКА И ЕЁ ПРИМЕНЕНИЯ\ \ \ том~6\ \ \ выпуск 2\ \ \ 2012}
}%
 \def\rightfootline{\small{ИНФОРМАТИКА И ЕЁ ПРИМЕНЕНИЯ\ \ \ том~6\ \ \ выпуск~2\ \ \ 2012
\hfill \textbf{\thepage}}} 
\vskip 2em \setcounter{figure}{0}
\setcounter{table}{0} 
\setcounter{equation}{0} 
\setcounter{section}{0}
\setcounter{subsection}{0} 
\setcounter{subsubsection}{0}
\setcounter{footnote}{0} 
\setcounter{razdel}{0}
%\end{flushleft}
\else {
 \raggedright\noindent\Large\textrm\textbf
 \lineskip .75em
\baselineskip=3.2ex #1 \par} \vskip 1em
%\begin{flushleft}
{\noindent\large\textrm\textbf #2 \par}
\addcontentsline{toc}{subsection}{{\textrm\textbf #3}\protect\newline #1}
\def\rightheadline{\underline{\noindent\hbox to \textwidth{\hfill\small\textrm{#4}
%\hfill \large\bf\thepage
}}}
\def\leftheadline{\underline{\noindent\parbox{\textwidth}{%\raggedleft\large\bf\thepage \hfill
\small\textit{#3}\hfill}}}
\def\leftfootline{\small{\textbf{\thepage}
\hfill ИНФОРМАТИКА И ЕЁ ПРИМЕНЕНИЯ\ \ \ том~6\ \ \ выпуск~2\ \ \ 2012}
}%
 \def\rightfootline{\small{ИНФОРМАТИКА И ЕЁ ПРИМЕНЕНИЯ\ \ \ том~6\ \ \ выпуск~2\ \ \ 2012
\hfill \textbf{\thepage}}} \vskip 2em \setcounter{figure}{0}
\setcounter{table}{0} \setcounter{equation}{0} \setcounter{section}{0}
\setcounter{subsection}{0} \setcounter{subsubsection}{0}
\setcounter{footnote}{0}
%\end{flushleft}
\fi}

\newcommand{\titelr}[2]{%
\

\vspace*{5pt}

\ifodd\therazdel {\raggedright\noindent\large\textrm\textbf
 \lineskip .75em
  \baselineskip=3.2ex #1 \par}
\vskip 1em {\noindent\normalsize\textrm\textbf #2 \par}
\else {
 \raggedright\noindent\large\textrm\textbf
 \lineskip .75em
\baselineskip=3.2ex #1 \par} \vskip 1em
%\begin{flushleft}
{\noindent\normalsize\textrm\textbf #2 \par}
\fi}

\newcommand{\titele}[5]{%
\

%\vspace*{5pt}

\ifodd\therazdel {\raggedright\noindent%\large
\textrm\textbf
 \lineskip .75em
%  \baselineskip=3.2ex
#1 \par}
\vskip .5em {\noindent\large\textrm\textbf #2 \par}
\vskip .5em
 {\noindent\textrm #3 \par}
\addcontentsline{toc}{subsection}{{\textrm\textbf #1}\protect\newline #2}
\def\rightheadline{\underline{\noindent\hbox to \textwidth{\hfill\small\textrm{#4}
%\hfill \large\bf\thepage
}}}
\def\leftheadline{\underline{\noindent\parbox{\textwidth}{
%\raggedleft\large\bf\thepage \hfill
\small\textrm{#5}\hfill}}}
\def\leftfootline{\small{\textbf{\thepage}
\hfill ИНФОРМАТИКА И ЕЁ ПРИМЕНЕНИЯ\ \ \ том~6\ \ \ выпуск~2\ \ \ 2012}
}%
 \def\rightfootline{\small{ИНФОРМАТИКА И ЕЁ ПРИМЕНЕНИЯ\ \ \ том~6\ \ \ выпуск~2\ \ \ 2012
\hfill \textbf{\thepage}}} \vskip 1em \setcounter{figure}{0}
\setcounter{table}{0} \setcounter{equation}{0} \setcounter{section}{0}
\setcounter{subsection}{0} \setcounter{subsubsection}{0}
\setcounter{footnote}{0} \setcounter{razdel}{0}
%\end{flushleft}
\else {
 \raggedright\noindent%\large
 \textrm\textbf
 \lineskip .75em
%\baselineskip=3.2ex
#1 \par} \vskip .5em
%\begin{flushleft}
{\noindent\large\textrm\textbf #2 \par} \vskip .5em
 {\noindent\textrm #3 \par}
\addcontentsline{toc}{subsection}{{\textrm\textbf #1}\protect\newline #2}
\def\rightheadline{\underline{\noindent\hbox to \textwidth{\hfill\small\textrm{#4}
%\hfill \large\bf\thepage
}}}
\def\leftheadline{\underline{\noindent\parbox{\textwidth}{%\raggedleft\large\bf\thepage \hfill
\small\textrm{#5}\hfill}}}
\def\leftfootline{\small{\textbf{\thepage}
\hfill ИНФОРМАТИКА И ЕЁ ПРИМЕНЕНИЯ\ \ \ том~6\ \ \ выпуск~2\ \ \ 2012}
}%
 \def\rightfootline{\small{ИНФОРМАТИКА И ЕЁ ПРИМЕНЕНИЯ\ \ \ том~6\ \ \ выпуск~2\ \ \ 2012
\hfill \textbf{\thepage}}} \vskip 1em \setcounter{figure}{0}
\setcounter{table}{0} \setcounter{equation}{0} \setcounter{section}{0}
\setcounter{subsection}{0} \setcounter{subsubsection}{0}
\setcounter{footnote}{0}
%\end{flushleft}
\fi}

\def\Abst#1{
\begin{center}\small\nwt
\parbox{150mm}{%\baselineskip=2.5ex
\textbf{Аннотация:}\ \
%\hspace*{\parindent}
#1}
\end{center}}
\def\Abste#1{
\begin{center}\small\nwt
\parbox{150mm}{%\baselineskip=2.5ex
\textbf{Abstract:}\ \
%\hspace*{\parindent}
#1}
\end{center}}

\def\KW#1{
\begin{center}\small\nwt
\parbox{150mm}{%\baselineskip=2.5ex
\textbf{Ключевые слова:}\ \ #1}
\end{center}}

\def\KWE#1{
\begin{center}\small\nwt
\parbox{150mm}{%\baselineskip=2.5ex
\textbf{Keywords:}\ \ #1}
\end{center}}


\def\KWN#1{
%\begin{center}
%\small
%\parbox{150mm}\end{center}
}

\renewcommand{\thesubsection}{\thesection.\arabic{subsection}\hspace*{-5pt}}
\renewcommand{\thesubsubsection}{\thesubsection\hspace*{5pt}.\arabic{subsubsection}\hspace*{-3pt}}

\begin{document}
\Rus

\nwt
%\ptb

%\renewcommand{\contentsname}{\protect\Large\bf Содержание}

\setcounter{tocdepth}{2}

%\tableofcontents

\renewcommand{\bibname}{\protect\rmfamily Литература}
  \def\Au#1{{\it #1}}

%\newcommand{\No}{№}
  \newcommand{\tg}{\,\mathrm{tg}\,}
    \newcommand{\ctg}{\,\mathrm{ctg}\,}
  \newcommand{\arctg}{\,\mathrm{arctg}\,}
  
\def\forallb{\mathop{\forall}}
\def\cupb{\mathop{\cup}}
\def\existsb{\mathop{\exists}}

\setcounter{page}{1}

\newpage
\addtocounter{razdel}{1}
%\def\razd{РЕГУЛИРУЕМЫЙ ЭЛЕКТРОПРИВОД ДЛЯ ЭЛЕКТРОЭНЕРГЕТИКИ}
%\newpage
%\def\stat{zakh}
\def\tit{СРЕДСТВА ОБЕСПЕЧЕНИЯ ОТКАЗОУСТОЙЧИВОСТИ ПРИЛОЖЕНИЙ}
\def\titkol{Средства обеспечения отказоустойчивости приложений}

\def\aut{В.\,Н.~Захаров$^1$, В.\,А.~Козмидиади$^2$}
\titel{\razd}{\tit}{\aut}{\titkol}


\Abst{Рассмотрены проблемы построения отказоустойчивых серверов, возникающие в связи с недетерминированностью поведения приложений. Предложена формальная модель, описывающая поведение приложения, основными объектами которой являются ресурсы и события. Предложены алгоритмы протоколирования работы приложения на резервном узле кластера, а также восстановления и продолжения его работы при отказе основного узла. При этом для клиентов сбой остается незаметным, за исключением некоторого увеличения времени обслуживания.}

\KW{сервер приложений $\bullet$ прозрачная отказоустойчивость $\diamond$
 процесс $\diamond$ ресурс $\diamond$ событие $\diamond$ контрольная точка
$\bullet$ детерминированность}

\vskip 12pt plus 6pt minus 3pt

\begin{multicols}{2}

\section*{ВВЕДЕНИЕ}

Средства вычислительной техники стали использоваться в областях,
требующих безотказной работы систем в течение многих лет (24 часа
в сутки, 365 дней в году).

\label{st\stat}

\footnotetext{$^1$ФГУП Центральный институт авиационного моторостроения
им. П.И. Баранова, Москва, Россия}
\footnotetext{$^2$ФГУП Центральный институт авиационного моторостроения
им. П.И. Баранова, Москва, Россия}

К таким областям относятся, например, центры хранения и обработки данных  в сетях (системы резервирования билетов, биллинговые,  банковские и т.д.), массированные распределенные вычисления (GRID-вычисления) и другие.

\thispagestyle{headings}

Обычно в подобных системах применяются частные решения, ориентированные в основном на обеспечение надежного хранения данных (например, файловые серверы, использующие для хранения RAID-контроллеры) и корректного их состояния при отказах (серверы баз данных с транзакционным выполнением запросов). Однако большинство приложений не гарантируют, что не произойдет потери части данных при отказе системы. Обычно предполагается, что клиентские средства должны повторять запросы после восстановления серверов, для того, чтобы данные не были потеряны, или что можно сделать возврат по времени на некоторое время назад и повторить работу с этого места. Однако далеко не все клиентские средства и условия применения приложений допускают это.

Отказоустойчивые системы для критически важных приложений, корректно решающие проблемы восстановления после сбоев,   предлагаемые ведущими производителями, как правило, дороги. Кроме того, они включают специфические серверные и клиентские приложения, не совместимые со стандартными приложениями, не обеспечивающими отказоустойчивость. Примером такого подхода к решению проблемы отказоустойчивости  хранения данных являются системы NetApp FAS компании Network Appliance, работающие на базе специализированной операционной системы Data ONTAP [1].

Построение отказоустойчивых систем, использующих серверы со стандартными приложениями, в свете вышесказанного, является актуальной проблемой, вызывающей значительный интерес. Рассмотрение методов достижения прозрачной отказоустойчивости таких систем и является предметом статьи.
\begin{figure*} %fig1
\vspace*{1pt}
\begin{center}
\mbox{%
\epsfxsize=1.6in
\epsfxsize=100mm
\epsfbox{BbR-1.eps}
}
\end{center}
\vspace*{-9pt}
\Caption{Базовый вариант трубы с разными выходными устройствами
(цилиндрическое, расширяющееся и сужающееся сопло)
\label{f1bab}}
\vspace*{-3pt}
\end{figure*}


\section{ОСНОВНЫЕ ПОНЯТИЯ И ПОДХОДЫ}

Под сервером в данной работе понимается вычислительный центр
(отдельный компьютер или кластер) в сети, предоставляющий клиентам
(пользователям, клиентским компьютерам) определенные услуги, разделяя
между ними свои ресурсы. Подобные серверы названы серверами приложений.
Широко распространенным примером сервера такого типа является файловый сервер, обеспечивающий удаленный коллективный доступ к файловой системе. Часто используются вычислительные серверы, предоставляющие клиентам возможность выполнять на них свои программы (например, в центрах коллективного пользования).


Обычно приложение представляет собой программу или группу программ, работающих в операционной среде, создаваемой операционной системой (в другой терминологии - один или несколько взаимодействующих процессов или потоков (threads)), которые реализуют функциональность сервера. Для построения отказоустойчивых серверов приложений широко используется кластерная технология. Следуя [2], кластером, названа разновидность параллельной или распределенной системы, которая:
\begin{itemize}
\item состоит из нескольких компьютеров (узлов кластера), связанных как минимум необходимыми коммуникационными каналами;
\item используется как единый, унифицированный компьютерный ресурс.
\end{itemize}

Прозрачная отказоустойчивость (Transparent Fault Tolerance, TFT) сервера приложений - это такое его поведение при возникновении аппаратных или программных отказов либо отказов в сети, при котором:
\begin{itemize}
\item отказ не вызывает потери или искажения данных, находящихся в базе данных сервера;
\item сервер продолжает нормально функционировать, несмотря на имевшие место отказы.
\end{itemize}

Клиенты сервера "не замечают" произошедших отказов. Единственным\footnote{допустимым
отклонением сервера от нормального поведения с точки зрения клиента является
некоторое увеличение времени обслуживания} (на несколько секунд или десятков секунд).

Обычно приложения, работающие на серверах приложений, не ориентированы на прозрачную отказоустойчивость. Они "заботятся" лишь о собственной целостности (например, состояния файловой системы или базы данных). Восстановление работоспособности сервера приводит к разрыву соединений с клиентами и потере их запросов. Это замечают клиенты - запросы следует повторять, на что клиентские приложения далеко не всегда рассчитаны. В данной работе предполагается, что приложения (прикладные программные средства), выполняемые на сервере, являются стандартными, то есть не имеют специальных средств, обеспечивающих отказоустойчивость.
\begin{figure*}[b] %fig1
\vspace*{1pt}
\begin{center}
\mbox{%
\epsfxsize=1.6in
\epsfxsize=100mm
\epsfbox{BbR-1.eps}
}
\end{center}
\vspace*{-9pt}
\Caption{Базовый вариант трубы с разными выходными устройствами
(цилиндрическое, расширяющееся и сужающееся сопло)
\label{f1bab}}
\vspace*{-3pt}
\end{figure*}

Серьезные исследования в области обеспечения отказоустойчивости серверов были развернуты после создания вычислительных серверов, предназначенных для решения задач, требующих больших вычислительных ресурсов. Решение этих задач выполняется на суперкомпьютерах, обеспечивающих массово-параллельные вычисления и представляющих собой кластеры из сотен и тысяч узлов (процессоров). Однако даже на этих "монстрах" решение может требовать десятков или сотен часов, и одиночный сбой, если не предприняты специальные меры, может привести к необходимости начинать работу сначала. Обычно решение вычислительной задачи в таких случаях осуществляется в модели относительно редко взаимодействующих между собой процессов, выполняемых на разных узлах кластера. Эти взаимодействия нужны для координации работы процессов, в частности, для обмена данными и промежуточными результатами. Взаимодействия опираются на специальный протокол, называемый MPI (Message-Passing Interface) и представляющий собой стандарт "de facto" [3].

Для преодоления последствий сбоя достаточно давно была разработана и широко применяется технология, опирающаяся на механизм контрольных точек (checkpoints) [4-6]. По этой технологии система должна иметь стабильную память, которая не меняется при отказах. Соответствующие программные средства периодически сохраняют информацию о состоянии процессов приложения в стабильной памяти. Все процессы также имеют доступ к устройству стабильной памяти.  В случае отказа или сбоя, записанная в стабильную память информация используется для повторения вычисления с момента, когда была записана эта информация, то есть выполняется откат назад по времени. Данные, сохранение которых позволяет выполнить откат, называются контрольной точкой. В качестве устройства стабильной памяти может использоваться дисковый том, энергонезависимая оперативная память, память другого узла или узлов кластера. В последнем случае узел, которому требуется сохранить информацию, пересылает ее через быстрый канал связи на другой узел. Стабильная память после отказа одного из узлов должна быть доступной узлу, на котором делается повтор.

Однако решение, опирающееся только на контрольные точки, не является прозрачным, поскольку не скрывает от клиентов факт отказа системы и требует от них выполнения определенных действий. Так как при работе процессы обмениваются сообщениями, возможны два варианта решения проблемы. Первый - все процессы выполняют записи контрольных точек одновременно, что затруднительно. Второй вариант, при несоблюдении синхронности, - возврат в каждом процессе к такому скоординированному набору контрольных точек, при котором невозможна противоречивая ситуация. Такая ситуация возникает, когда один процесс вернулся к контрольной точке, после которой он должен получить сообщение от другого процесса, а этот другой процесс вернулся к точке, которая следует за выдачей этого сообщения. Однако при повторе ожидаемое первым процессом сообщение не поступит. В этом случае  возможен эффект домино, в результате процессы оказываются отброшены как угодно далеко назад.

В этом состоит первая проблема, которую необходимо преодолеть.

Если нужно, чтобы последствия отказа узла не были видны клиенту,  это означает:
\begin{itemize}
\item клиент не должен терять и потом восстанавливать соединения с сервером;
\item клиент не должен повторять свои запросы;
\item клиент не должен повторно получать сообщения, которые он уже получил.
\end{itemize}

Вторая проблема, которую надо решать, связана с недетерминированностью поведения сервера приложений. Приведем пример.  Пусть имеется система продажи билетов на самолеты. Два клиента одновременно обратились к системе с запросом билета на один и тот же рейс. Клиентам безразлично, какие места им зарезервирует система. Система выполняет запросы клиентов параллельно, поэтому в какой-то момент между процессами, обрабатывающими эти запросы, может возникнуть конкуренция за ресурс - в данном случае, скажем, рейс. Один из процессов захватывает ресурс первым, резервирует место и освобождает ресурс. Потом второй процесс проделывает то же самое.

Порядок, в котором в этом примере процессы захватили ресурс, зависит от многих факторов и, в конечном счете, случаен. Однако  это не мешает правильному функционированию системы, поскольку клиентам важно одно - получить билеты, причем на разные места. Однако отсутствие детерминизма в поведении приложения приводит к тому, что при повторном выполнении могут быть получены другие результаты: например, клиенту уже сообщено, что ему зарезервировано место №5, а при повторе может получиться, что зарезервировано место №6. Система должна устранить это несоответствие и сделать его невидимым для клиента.
\begin{figure*} %fig1
\vspace*{1pt}
\begin{center}
\mbox{%
\epsfxsize=1.6in
\epsfxsize=100mm
\epsfbox{BbR-1.eps}
}
\end{center}
\vspace*{-9pt}
\Caption{Базовый вариант трубы с разными выходными устройствами
(цилиндрическое, расширяющееся и сужающееся сопло)
\label{f1bab}}
\vspace*{-3pt}
\end{figure*}

Недетерминированность поведения системы это следствие, по крайней мере, двух обстоятельств. Во-первых, это присущая системам с разделением времени неопределенность в порядке выполнения процессов. Во-вторых, это конкуренция процессов за общие ресурсы. Перечислим некоторые причины недетерминированного поведения приложений:
\begin{itemize}
\item синхронизация процессов с помощью семафоров или атомарных операций над операндами в общей памяти процессов;
\item зависимость от порядка получения клиентских запросов;
\item время, затраченное процессом на обработку полученного запроса;
\item генераторы случайных чисел;
\item системное управление процессами и потоками;
\item локальные таймеры;
\item доступ к реальному времени.
\end{itemize}

По различным  причинам время, которое тратится на выполнение вычислительной задачи с одними и теми же исходными данными, не является константой, то есть повторное выполнение может дать другое время. Процессы используют общие ресурсы, обращение к которым требует организации очередности выполнения (сериализации) - первым пришел, первым захватил. И, наконец,  результат работы процесса может зависеть от состояния ресурса, а это состояние может изменить другой процесс, ранее захвативший ресурс. Все это создает значительные трудности при попытках воспроизведения поведения процессов с сохраненной контрольной точки.

Прозрачная отказоустойчивость серверов приложений обычно осуществляется переносом приложения на другой узел кластера, идентичный первому по конфигурации аппаратных средств и операционной среды. Это делается методом, называемым snapshot/restore. На основном узле (оригинале)  периодически фиксируется состояние приложения на этом узле кластера (так называемый снимок или snapshot). После отказа оригинала на резервном узле (копии) делается восстановление (restore), то есть восстанавливается последнее зафиксированное состояние приложения. Операционная среда при этом приводится в состояние, которое соответствует моменту изготовления снимка. После этого узел-копия продолжает работу с зафиксированного места. Сравнение метода  snapshot/restore с другими подходами приведено в [7].

Ниже рассматриваются информационные  технологии, позволяющие решить ряд принципиальных вопросов, связанных с реализацией прозрачной отказоустойчивости серверов приложений. Ими являются:
\begin{itemize}
\item виртуализация операционной среды, в которой работает серверное приложение;
\item отказоустойчивая реализация протокола TCP;
\item создание контрольных точек состояния приложения и файловой системы, которые делаются внешним по отношению к приложению образом;
\item восстановление серверного приложения на основании контрольной точки.
\end{itemize}
\begin{figure*} %fig1
\vspace*{1pt}
\begin{center}
\mbox{%
\epsfxsize=1.6in
\epsfxsize=100mm
\epsfbox{BbR-1.eps}
}
\end{center}
\vspace*{-9pt}
\Caption{Базовый вариант трубы с разными выходными устройствами
(цилиндрическое, расширяющееся и сужающееся сопло)
\label{f1bab}}
\vspace*{-3pt}
\end{figure*}

\section{МОДЕЛЬ ОПИСАНИЯ ПОВЕДЕНИЯ ПРИЛОЖЕНИЯ}

Предлагаемый подход опирается на построение модели вычислений, связанной с использованием понятия времени в многопроцессных приложениях. Впервые подобные проблемы были изучены в классической работе Л. Лампорта [8].

Многопроцессными приложения называются потому, что в них параллельно работают несколько процессов. Процесс ведет себя детерминированно, пока в предписанном кодом порядке выполняет процессорные инструкции. Конечно, его работа может быть прервана практически в любой момент и процессор передан другому процессу или ядру. Поэтому абсолютное время, которое затрачивает процесс на выполнение определенной работы, не  константа, а случайная  величина. То же  относится к относительному времени, то есть времени, которое процесс занимал процессор,  поскольку одни и те же обращения к операционной среде могут вызвать работы разной длительности, а значит потребовать разное время на свое выполнение.

Кэшированность инструкций и данных, а также длина хэш-списков влияют на действительное время пребывания в операционной среде. Утрачивает смысл понятие одновременность действий, поскольку  нельзя установить, выполнили ли два разных процесса какие-либо действия одновременно или одно из них предшествовало другому. Таким образом, с процессом можно связать только его локальное время, которое линейно упорядочивает события,  происходившие в этом процессе.  Глобальное время, линейно упорядочивающее действия во всех процессах, отсутствует. Расстояние (в этом качестве используется время) между действиями оказывается случайной величиной.

Эти соображения важны, поскольку процессы в интересующих нас приложениях взаимодействуют и используют общие ресурсы. Для взаимодействия они используют средства синхронизации, предоставляемые операционной средой - например, наборы семафоров SVR4 (System V Release 4), POSIX-семафоры, бинарные семафоры и другие примитивы взаимного исключения (POSIX- mutual exclusion locks) и т.д. Подобные средства операционной среды, которые позволяют процессам синхронизировать свою деятельность друг с другом или сериализовать обращения к совместно используемым объектам,  будут ниже  называться ресурсами.

С каждым ресурсом связано свое локальное время, линейно упорядочивающее события в жизни ресурса. Например, в случае двоичных семафоров это создание семафора, а также его захват и освобождение процессом. Заметим, что событие - это не намерение процесса (например, захватить бинарный семафор), а сам факт захвата семафора процессом (т.е. успешное выполнение намерения). От изъявления намерения до его осуществления может многое произойти. Например, семафор, который хочет захватить рассматриваемый процесс, принадлежал другому процессу, потом тот процесс его освободил, но семафор был сначала передан операционной средой третьему процессу, который также на него претендовал, и т.д. Поведение рассматриваемого процесса в это время нас не интересует - он ресурсом еще не овладел, а только его захват определяет его дальнейшее поведение. По причинам,  изложенным выше, расстояние между двумя событиями - случайная величина. Однако, события замечательны тем, что они одновременно присутствуют и в локальном времени процесса, и в локальном времени ресурса. Поэтому все, что произошло в истории процесса или/и ресурса до этого события, предшествует ему. Далее  будет считаться, что истории и ресурсов и процессов состоят только из событий, причем между двумя последовательными событиями в жизни процесса последний ведет себя детерминированно.

Это означает, что на  поведении процесса сказывается только его предыдущая история, то есть состояние ресурсов, с которыми он взаимодействовал. Это свойство процессов ниже будет называться локальной детерминированностью. Этим свойством не обладают ресурсы, поскольку - следующее событие в истории ресурса не определяется однозначно по его предыдущей истории. Утверждение, касающееся детерминированного поведения процессов, неявно опирается на предположение,  что учтены все ресурсы, которые могут привести к  недетерминированности процессов.

Таким образом, описанное нами очень неформально время в многопроцессном комплексе представляет собой отношение частичного порядка, введенное на множестве событий. Зная полное состояние комплекса в некоторый момент времени,  нельзя однозначно определить, какое событие в истории ресурса наступит следующим. Можно говорить только о вероятности наступления того или иного события. Недетерминированность поведения есть следствие двух обстоятельств. Во-первых, это неопределенность времени, которое тратит процесс на переход от одного события к другому. Во-вторых, конкуренция процессов за общие ресурсы.

Выполнение приложения, на множестве событий которого введена частичная упорядоченность, можно описать направленным ациклическим графом выполнения. Вершинами этого графа являются события, с каждым  из которых связаны две входящие в него дуги. Одна дуга начинается в событии, которое непосредственно предшествует данному событию в истории процесса, другая - в истории ресурса.

Построение средств обеспечения прозрачной отказоустойчивости приложений опирается на следующее утверждение: для восстановления работы приложения после отказа достаточно располагать:
\begin{itemize}
\item контрольной точкой, которая отражает на некоторый момент времени состояния процессов и других ресурсов, образующих приложение;
\item графом выполнения приложения, который описывает работу приложения, начинающуюся с контрольной точки и заканчивающуюся отказом. Данные, которые нужны для построения графа выполнения, далее называются протоколом.
\end{itemize}
\begin{figure*} %fig1
\vspace*{1pt}
\begin{center}
\mbox{%
\epsfxsize=1.6in
\epsfxsize=100mm
\epsfbox{BbR-1.eps}
}
\end{center}
\vspace*{-9pt}
\Caption{Базовый вариант трубы с разными выходными устройствами
(цилиндрическое, расширяющееся и сужающееся сопло)
\label{f1bab}}
\vspace*{-3pt}
\end{figure*}

Вся эта информация должна находиться в стабильной памяти, не разрушающейся при отказе.

Ниже неформально описан алгоритм восстановления работы приложения после отказа, который опирается на наличие контрольной точки и графа выполнения. Будем считать, что в распоряжении имеются средства, позволяющие остановить процесс в тот момент, когда он намерен совершить некоторую операцию над ресурсом. Заметим, что событие в графе выполнения соответствует не изъявлению намерения, а его удовлетворению, то есть завершению выполнения операции.

Предварительно сделаем следующее:
\begin{itemize}
\item используя контрольную точку, приведем приложение в состояние, соответствующее этой контрольной точке;
\item в графе выполнения пометим все вершины (события) как "не наступившие". У некоторых вершин графа отсутствуют им непосредственно предшествующие; соответствующие события наступили сразу же после создания контрольной точки. Для каждой такой вершины включим в граф дополнительную вершину, ей предшествующую в истории процесса, и отметим эту дополнительную вершину как "наступившую";
\item разрешим процессам приложения выполняться.
\end{itemize}

Пусть некоторый процесс проявляет намерение выполнить операцию над каким-либо ресурсом. Отыщем для этого процесса в его истории последнее наступившее событие. Следующее в его истории событие - это то, которое соответствует требуемой операции. Посмотрим, наступило ли событие в истории ресурса, которое ему предшествует. Если нет, переведем процесс в состояния ожидания, отметив в предшествующем событии, что данный процесс ожидает его наступления. Если да, разрешим процессу выполняться, то есть выполнить операцию над ресурсом.

Пусть некоторый процесс объявляет, что он выполнил операцию над каким-либо ресурсом (это соответствует моменту протоколирования при оригинальном выполнении). Отыщем для этого процесса в его истории последнее наступившее событие и перейдем к следующему событию в его истории. Это опять то событие, которое мы рассматриваем. Отметим его как "наступившее". Если наступления этого события ожидал какой-нибудь процесс, выведем этот процесс из состояния ожидания. Наконец, разрешим процессу, выполнившему операцию, продолжаться дальше.

Когда выясняется, что наступили все события графа выполнения, повторное выполнение считается законченным.

Естественным следствием из сказанного является следующее утверждение: для того, чтобы размер протокола не рос неограниченно, нужно периодически создавать контрольные точки, очищая при этом протокол.

\section{ФОРМАЛЬНОЕ ОПИСАНИЕ МОДЕЛИ ПОВЕДЕНИЯ МНОГОПРОЦЕССНОГО ПРИЛОЖЕНИЯ}
\begin{figure*} %fig1
\vspace*{1pt}
\begin{center}
\mbox{%
\epsfxsize=1.6in
\epsfxsize=100mm
\epsfbox{BbR-1.eps}
}
\end{center}
\vspace*{-9pt}
\Caption{Базовый вариант трубы с разными выходными устройствами
(цилиндрическое, расширяющееся и сужающееся сопло)
\label{f1bab}}
\vspace*{-3pt}
\end{figure*}

Опишем формально поведение приложения, неформальное описание которого было приведено выше. Рассматриваются два типа объектов:
\begin{itemize}
\item ресурсы (r), например, наборы семафоров (POSIX- или SVR4-семафоры), бинарные семафоры (POSIX-mutex's), таймер реального времени, сокеты (sockets), то есть двусторонние виртуальные соединения с внешним миром;
\item процессы (p), например, процессы или потоки (threads) пользователя.
\end{itemize}

\end{multicols}

\label{end\stat}

%\def\stat{batr}

\def\tit{НОВЫЙ МЕТОД ВЕРОЯТНОСТНО-СТАТИСТИЧЕСКОГО\newline
АНАЛИЗА ИНФОРМАЦИОННЫХ ПОТОКОВ
В~ТЕЛЕКОММУНИКАЦИОННЫХ СЕТЯХ$^*$}
\def\titkol{Новый метод вероятностно-статистического
анализа информационных потоков
в~телекоммуникационных сетях}
\def\autkol{Д.\,А.~Батракова, В.\,Ю.~Королев, С.\,Я.~Шоргин}
\def\aut{Д.\,А.~Батракова$^1$, В.\,Ю.~Королев$^2$, С.\,Я.~Шоргин$^3$}

\titel{\tit}{\aut}{\autkol}{\titkol}

{\renewcommand{\thefootnote}{\fnsymbol{footnote}}\footnotetext[1]{Работа 
выполнена при поддержке РФФИ, проекты №№\,04-01-00671, 05-07-90103.} 
\renewcommand{\thefootnote}{\arabic{footnote}}}
 \footnotetext[1]{ИПИ РАН, 
daria.batrakova@gmail.com} \footnotetext[2]{Факультет вычислительной математики 
и кибернетики МГУ им.~М.\,В.~Ломоносова, ИПИ РАН, vkorolev@comtv.ru} 
\footnotetext[3]{ИПИ РАН, sshorgin@ipiran.ru}



\Abst{В данной работе предлагается метод исследования стохастической структуры
хаотических информационных потоков в сложных телекоммуникационных
сетях. Предлагаемый метод основан на стохастической модели
телекоммуникационной сети, в рамках которой она представляется в виде
суперпозиции некоторых простых последовательно-параллельных структур.
Эта модель естественно порождает смеси гамма-распределений для времени
выполнения (обработки) запроса сетью. Параметры получаемой смеси
гамма-распределений характеризуют стохастическую структуру
информационных потоков в сети. Для решения задачи статистического
оценивания параметров смесей экспоненциальных и гамма-распределений
(задачи разделения смесей) используется ЕМ-алгоритм. Чтобы проследить
изменение стохастической структуры информационных потоков во времени,
ЕМ-алгоритм применяется в режиме скользящего окна. Описывается
программный инструментарий для применения полученного решения к
реальным статистическим данным. Приводится интерпретация результатов.}

\KW{телекоммуникационные сети; информационные потоки;
разделение смесей  распределений;
метод скользящего окна;  программа для разделения смесей}

\vskip 24pt plus 9pt minus 6pt

\thispagestyle{headings}

\begin{multicols}{2}


\label{st\stat}

\section{Введение}

Развитие телекоммуникационных сетей, их усложнение поставило перед
инженерами важную прикладную задачу исследования характеристик
информационных потоков, возникающих в этих сетях. Здесь под
информационным потоком мы будем понимать упорядоченное движение
любого вида информации по сети.

Если на заре эры телекоммуникаций, в эпоху первых телефонных линий и
телеграфа эта проблема не была столь насущной, то со временем, при
постепенном охвате мирового пространства сетями возникла необходимость в
построении и исследовании адекватных моделей сетей и процессов,
происходящих в них.

\thispagestyle{headings}


Современные сети~--- это \textit{конвергентные} сети, т.е.\ совокупность крайне
разнородных как по топологии, так и по физической архитектуре сетей, которые
предлагают конечному пользователю самые разнообразные сервисы. Это~--- огромное
виртуальное и физическое пространство, состоящее из миллионов процессоров,
операционных платформ, линий передачи данных и стыковочных узлов.
%
Существует множество классификаций телекоммуникационных сетей по различным
признакам:
\begin{itemize}
\item масштабу (локальные сети~--- LAN, масштаба города~---
MAN, широкого масштаба~--- WAN);
\item топологии, или логической организации (<<звезда>>,
<<кольцо>>, <<шина>>);
\item физической организации (оптические сети, радио);
\item предлагаемым услугам (сотовые сети, для доступа в
Интернет);
\item назначению (военные, гражданские) и~др.
\end{itemize}


Конвергентная сеть входит во все эти классы, причем, как правило,
обладает всеми этими признаками. Поэтому построение модели для ее анализа
является и очень важной, и очень сложной задачей.

Существуют достаточно многочисленные математические методы, ориентированные на
моделирование и анализ телекоммуникационных сетей. В~большинстве своем они
основываются на теории массового обслуживания, то есть разделе теории
вероятностей, посвященном описанию функционирования сложных систем обслуживания
(в том чис\-ле телекоммуникационных сетей и систем) с помощью стохастических
процессов особого вида и анализу таких процессов. Указанная теория является
весьма развитой и широко применяется на практике. Тем не менее, ее применимость
ограничена~--- во-первых, все возрастающей сложностью структур и дисциплин
обслуживания в рас\-смат\-ри\-ва\-емых реальных сетях. Эта сложность во многих
случаях принципиально не может найти адекватного отображения в моделях
массового обслуживания, даже несмотря на постоянно растущую сложность самих
этих моделей. В результате даже модели, допускающие точный математический
анализ, дают возможность расчета всего лишь приближенных значений характеристик
реальных сетей, ибо предположения, принимаемые при построении моделей, во
многих случаях не соответствуют практике. Во-вторых, для описания
телекоммуникационной сети в виде сети массового обслуживания исследователь
должен располагать детальным описанием структуры сети, что далеко не всегда
имеет мес\-то на практике. В-третьих, разработано крайне мало моделей массового
обслуживания, в которых используется в качестве входной информация о
наблюдаемых (статистических) показателях функционирования сети; в то же время,
такая информация очень часто доступна исследователю, и ее использование при
анализе сети весьма целесообразно.

В данной работе предлагается в определенной степени альтернативный подход к
моделированию сложных телекоммуникационных сетей. Строится и исследуется
вероятностная модель сложной телекоммуникационной сети как суперпозиции
достаточно простых структур. При этом практически никакая априорная информация
о структуре исследуемой сети не используется~--- наоборот, в результате
исследования модели исследователь получает приближенное представление об этой
структуре. Характеристики типовых простых структур, составляющих в совокупности
модель сети, и сети в целом при этом подходе описываются
гам\-ма-рас\-пре\-де\-ле\-ни\-я\-ми. Ставится задача оценки параметров модели
на основе статистических данных о функционировании сети, а также предлагается
математическое решение этой задачи. В статье описан также созданный на основе
разработанных математических методов программный инструментарий и приведены
результаты расчетов для реального трафика. {\looseness=-1

}

\section{Математическая модель и~постановка задачи}

\subsection{Логическая модель сети}
 %1.1

Рассмотрим абстрактную \textit{конвергентную телекоммуникационную
сеть}. Это может быть как крупномасштабная транспортная сеть (WAN), сеть
отдельного оператора масштаба города (MAN) с различными сервисами, так и
локальная сеть (LAN).

Любой из этих случаев можно описать как ($p,\,q$)-\textit{сеть}.

\medskip
\textbf{Определение 1.} В теории графов и сетей под ($p,\,q)$-сетью понимается
набор вида $S =$\linebreak $=(G,\,V^\prime ,\,V^{\prime\prime})$, где $G$~---
граф, а $V^\prime$ и $V^{\prime\prime}$~--- выборки из множества $V(G)$ (вершин
графа) длины~$p$ и $q$ соответственно. При этом выборка $V^\prime$
($V^{\prime\prime}$) считается \textit{входной} (\textit{выходной}) выборкой, а
ее $i$-я вершина называется $i$-\textit{м} \textit{входным} (\textit{выходным})
\textit{полюсом} или, иначе, $i$-\textit{м} \textit{входом} (\textit{выходом})
сети~$S$. Вершины, не участвующие во входной и выходной выборках сети,
считаются ее внутренними вершинами~\cite{1bat}.

Сеть $S$ (рис.~\ref{f1bat}) имеет $p$ точек входа~--- точек соединения
с внешней средой (это могут быть точки стыковки разнородных сетей, сетей
различных операторов, физические подключения к интерфейсам
маршрутизаторов и~т.п.). Под \textit{внешней средой} будем понимать другие
сети, которые передают данные в сеть~$S$. Отдельные <<единицы>> данных
(кадры, сообщения, датаграммы, пакеты) поступают на входы сети~$S$,
обрабатываются и подаются на каждый из $q$ выходов, которые могут быть
соединены как с конечными пользователями, так и с другими сетями.
\begin{figure*} %fig1
\vspace*{1pt}
\begin{center}
\mbox{%
\epsfxsize=139.7mm \epsfbox{bat-1.eps}
%\epsfxsize=139.698mm
%\epsfbox{bek-3.eps}
}
\end{center}
\vspace*{-9pt} \Caption{Абстрактная телекоммуникационная сеть \label{f1bat}}
\end{figure*}

Следует отметить, что структура сложных телекоммуникационных сетей обладает
свойством некоторого самоподобия, т.е.\ на каком бы уровне сетевой архитектуры
мы ни рассматривали поведение информационных потоков, мы можем выделить
некоторые базовые структуры, субпотоки, суперпозицией которых мы можем получить
модель конкретной сети, какой бы уровень <<детализации>> сегментов сети мы ни
взяли. Так, например, физические подключения к интерфейсам оптического
кросс-коннекта в этом смысле подобны <<виртуальным>> подключениям к портам TCP
на сервере приложений.

Итак, независимо от уровня сетевой архитектуры мы можем
рассматривать некоторую величину, характеризующую количество каких-либо
ресурсов сети~$S$, занимаемых в процессе передачи и обработки данных.  Это
могут быть ресурсы, относящиеся как к <<объему>> (памяти сетевого
устройства, количеству занятых линий, размеру пакета), так и ко <<времени>>
(времени обслуживания заявки, времени простоя). Эта величина случайна, т.к.\
мы не можем абсолютно точно сказать для сложной телекоммуникационной
сети, какое сообщение на какой из входов поступит и какого размера оно будет.
Таким образом, случайный характер данной величины определяется
случайностью поведения внешней среды.

Пусть $R$~--- это описанная выше случайная величина, $R>0$. Далее, не
ограничивая общности, будем подразумевать под ней время, необходимое для
какой-либо операции сети (обработки <<единицы>> данных), предполагая, что
время обработки прямо зависит от объема сообщения.

\subsection{Вероятностная модель сети} %1.2.

Даже не зная реальной топологии сети, мы можем описать
функционирование некоторых ее участков как процесс выполнения операций
(задач сети) в последовательном  порядке (например, если доступен только
один канал данных) или как процесс одновременного выполнения субопераций
(когда доступно более одного пути выполнения). Это значит, что мы можем
представить функционирование сложной телекоммуникационной сети как
\textit{суперпозицию} таких <<последовательных>> и <<параллельных>>
блоков.

Для построения вероятностной модели распределения~$R$ используется
комбинация асимптотического подхода, основанного на предельных теоремах
теории вероятностей, и принципа максимальной неопределенности (энтропии).

Рассмотрим следующую модель. Предположим, что мы можем разделить
сеть~$S$ на несколько сегментов $S_i$. Пусть $T$~--- случайная величина,
время выполнения операции в отдельно взятом блоке $S_i$ (сегменте сети).

Если операции выполняются \textit{параллельно}, то время, необходимое
для их выполнения~--- это максимальное время, затрачиваемое на какую-либо
субоперацию:
$$
T = \underset{i}{\max}\, T_i\,,
$$
где $T_i$~--- случайные величины для со\-от\-вет\-ст\-ву\-ющих субопераций.
Модель такого выполнения пред\-став\-ле\-на на рис.~\ref{f2bat}.

\begin{figure*} %fig2
\vspace*{1pt}
\begin{center}
\mbox{%
\epsfxsize=117.271mm
\epsfbox{bat-2.eps}
}
\end{center}
\vspace*{-9pt}
\Caption{Параллельное выполнение
\label{f2bat}}
\end{figure*}

Известно, что предельное распределение экстремальных значений для
выборок ~--- это экспоненциальное распределение с плотностью~\cite{2bat}
$$
f(x) =
\begin{cases}
\lambda e^{-\lambda x}\,, & x>0\,,\\
0\,, & x\leq 0\,,
\end{cases}
$$
где $\lambda >0$~--- параметр масштаба.

 Учитывая также энтропийный подход, естественно будет считать
распределение $T$ экспоненциальным, т.к.\ экспоненциальное распределение
обладает наибольшей энтропией среди всех распределений с $x>0$.

Если же операции сети выполняются \textit{последовательно}, то величина
$T$~--- это сумма времен $T_i$, необходимых для выполнения каждой
субоперации:
$$
T = \sum\limits_i T_i\,,
$$
где $T_i$~--- случайные величины для со\-от\-вет\-ст\-ву\-ющих субопераций.
%
Такая модель представлена на рис.~\ref{f3bat}.

\begin{figure*} %fig3
\vspace*{1pt}
\begin{center}
\mbox{%
\epsfxsize=139.592mm
\epsfbox{bat-3.eps}
}
\end{center}
\vspace*{-9pt}
\Caption{Последовательное  выполнение
\label{f3bat}}
\end{figure*}

Это значит, что распределение общей длительности $T$ выполнения
блока~--- это свертка распределений <<элементарных>> времен $T_i$
(экспоненциально распределенных).

Известно, что результатом свертки экспоненциальных распределений
является гамма-распределение, определяемое плотностью
$$
\g_{\lambda , \alpha} (x) =
\begin{cases}
\fr{\lambda_0^{\alpha_0}}{\Gamma (\alpha_0 )}\,x^{\alpha_0-1}
e^{\lambda_0 x}\,, & x>0\,,\\
0,\, & x\leq 0\,,
\end{cases}
$$
где $\alpha >0$~--- параметр формы,  $\lambda >0$  параметр масштаба, а
$\Gamma (z)$~--- гамма-функция Эйлера:
$$
\Gamma (z) = \int\limits_0^\infty x^{z-1} e^{-x}\,dx\,.
$$

\begin{figure*} %fig4
\vspace*{1pt}
\begin{center}
\mbox{%
\epsfxsize=120.831mm
\epsfbox{bat-4.eps}
}
\end{center}
\vspace*{-9pt}
\Caption{Модель пути  обработки сообщения сетью~$S$
\label{f4bat}}
\end{figure*}

Известно~\cite{2bat}, что класс гамма-распределений замкнут над операцией
свертки, поэтому ре\-зуль\-ти\-ру\-ющее распределение случайной величины~$R$
будет также гамма-распределением
$$
\g_{\lambda , \alpha} (x) =
\begin{cases}
\fr{\lambda^{\alpha}}{\Gamma (\alpha )}\,x^{\alpha -1} e^{-\lambda x}\,, &
x>0\,,\\
0,\, & x\leq 0\,.
\end{cases}
$$

В силу случайного характера ввода данных в сеть~$S$ из внешней среды маршрут
передачи данных становится случайным, что представлено на рис.~\ref{f4bat}. Это
означает, что параметры ре\-зуль\-ти\-ру\-юще\-го распределения~$R$ тоже
случайны. Отсюда имеем следующую модель \textit{смеси
гам\-ма-рас\-пре\-де\-ле\-ний}, определяемой плотностью

\begin{equation} %1
p(x) = \iint \g_{\lambda , \alpha}(x)\,dH (\lambda ,\,\alpha )\,,
\end{equation}
где $H(\lambda , \alpha )$~--- смешивающая функция, функция распределения
входных параметров.

Поясним понятие \textit{смеси распределений}.

\medskip
\textbf{Определение~2.} Пусть имеется двух\-па\-ра\-мет\-ри\-че\-ское
семейство $n$-мерных плотностей  распределения
\begin{equation}
F = \{ f_\omega (x;\, \theta (\omega ))\}\,,
\end{equation}
где одномерный (целочисленный или непрерывный) параметр $\omega$ в
качестве нижнего индекса функции $f$ определяет специфику общего вида
каж\-до\-го компонента~--- распределения смеси, а в качестве аргумента при
многомерном, вообще говоря, параметре $\theta$ определяет зависимость
значений хотя бы части компонентов этого параметра от того, в каком именно
составляющем распределении $f_\omega$ он присутствует. Кроме того, пусть
$P = \{P(\omega )\}$~--- \textit{семейство смешивающих функций}
распределения.

Функция плотности распределения
\begin{equation}
f(x) = \int f_\omega (x;\,\theta(\omega ))\,dP (\omega )
\end{equation}
называется $P$-\textit{смесью} (или просто \textit{смесью})
\textit{распределений} семейства~$F$,  интеграл в~(3) понимается в смысле
Лебега--Стильтьеса~\cite{3bat}.

\medskip
\textbf{Определение 3.} Семейство смесей~(3) называется
\textit{идентифицируемым} (\textit{различимым}), если из равенства
$$
\int f_\omega (x;\,\theta(\omega ))\,dP (\omega ) =\int f_\omega
(x,\,\theta(\omega )) dP^*(\omega )
$$
следует, что $P(\omega ) \equiv P^*(\omega )$ для всех $P \in P(\omega
)$~\cite{3bat}.

\subsection{Постановка задачи} %1.3.

Перед нами встает задача \textit{разделения} такой смеси. Вообще говоря,
задача разделения смесей распределений со смешивающими функциями
общего вида является \textit{некорректно поставленной}, т.к.\ она допускает
существование нескольких решений. Поэтому будем искать решение в классе
\textit{конечных идентифицируемых смесей распределений}, где смешивающая
функция дискретна.

Для этого сузим данное выше определение и будем рассматривать в дальнейшем лишь 
случай конечного числа $k$ возможных значений па\-ра\-мет\-ра~$\omega$, что 
соответствует конечному числу скачков смешивающих функций $P(\omega )$.  
Величины этих скачков как раз и будут играть роль \textit{удельных весов} 
(\textit{априорных вероятностей}) $p_j$ компонентов смеси. Более того, в нашем 
случае мы постулируем также однотипность компонентов распределений $f_j$, т.е.\ 
принадлежность всех $f_j$ к одному общему па\-ра\-мет\-ри\-че\-ско\-му 
семейству $\{ f(X;\,\theta )\}$, где $\theta$~--- многомерный, вообще говоря, 
параметр. Так что~(3) в этом случае может быть записано в виде
\begin{equation} %4
p(x) = \sum\limits^k_{j=1} p_j f_j (x;\,\theta_j )\,.
\end{equation}

Переформулируем понятие идентифицируемости (различимости) смесей
специально применительно к такому виду смесей.

\medskip
\textbf{Определение 4.} \textit{Конечная смесь}~(3) называется
\textit{идентифицируемой} (\textit{различимой}), если из равенства
$$
\sum\limits_{j=1}^k p_j f_j (x;\,\theta_j ) = \sum\limits_{l=1}^{k^*} p_l^* f_l
(x;\,\theta_l^* )
$$
следует, что $k=k^*$ и для любого $j$ ($1\leq j \leq k$) найдется такое $l$ 
($1\leq l \leq k^*$), что $p_j = p_l^*$ и $f_j (x;\,\theta_j ) = f_l 
(x;\,\theta_l^* )$~\cite{3bat}.

Решить эту задачу в выборочном варианте~--- значит по выборке
классифицируемых наблюдений
$X_1,\ldots , X_n, $ извлеченной из генеральной совокупности, яв\-ля\-ющей\-ся смесью~(3)
генеральных совокупностей типа~(2) (при заданном общем виде составляющих
смесь функций $f_j (x;\,\theta_j )$), построить статистические оценки для числа
компонентов смеси~$k$, их удельных весов $p_j$ и, главное, для каждого из
компонентов %f_j (x;\,\theta_j )$ анализируемой смеси. Далее будет считать, что
функции $f_j$ однозначно определяются своими параметрами $\theta_j$: $f_j
=f(x;\,\theta_j)$.

Однако не следует ставить знак тождества между задачей разделения смеси
и задачей статистического оценивания параметров в модели~(4) по выборке $
X_1,\ldots , X_n$, поскольку задача разделения сохраняет смысл и
применительно к генеральным совокупностям, т.е.\ в теоретическом
варианте~\cite{3bat}.

Итак, для статистического анализа на основе реальных данных мы
аппроксимируем нашу общую модель~(1) следующей:
$$
p(x) \approx \hat{p}(x) = \sum\limits_{j=1}^k p_j \g_{\lambda_j , \alpha_j}
(x)\,,
$$
где $p_j$~--- дискретные смешивающие параметры, $\g_{\lambda_j , \alpha_j}
(x)$~--- плотности гамма-распределений.

Такая аппроксимация не только позволяет решить поставленную статистическую
задачу, но и полу\-чить наглядную визуализацию результатов. Существуют
достаточно эффективные методики разделения смесей распределений, среди них~---
семейство так называемых \textit{ЕМ-алгоритмов}
(\textit{Expectation-Maximization Algorithms}).

Полученные результаты могут быть достаточно просто интерпретированы. Число
компонентов смеси символизирует число типичных параллельных или
последовательных структур. Значения параметров составляющих смесь
гам\-ма-рас\-пре\-де\-ле\-ний показывают <<степень параллелизма>>
соответствующей структуры: чем ближе параметр формы к~1, тем выше эта
<<степень>>. И наоборот, чем дальше значение параметра формы от~1, тем больше
последовательных операций выполняется в соответствующем блоке.

Веса компонентов характеризуют примерную долю использования
ресурсов для сообщений, соответствующих каждому распределению входных
данных.

Итак, предложенный подход позволяет получить представление о
стохастической структуре телекоммуникационной сети.

\section{ЕМ-алгоритм разделения смесей распределений}

\subsection{Описание алгоритма} %2.1.

Определяемый ниже итерационный алгоритм решения поставленной в
предыдущем разделе задачи относится к процедурам, базирующимся на
\textit{методе максимального правдоподобия}.

Этот алгоритм позволяет находить максимум логарифмической функции
правдоподобия по параметрам $p_1,\,p_2,\ldots ,\,p_k$, $\theta_1 ,\,\theta_2,\ldots ,\,
\theta_k$ при фиксированном $k$ по выборке $X_1, \ldots , X_n$, т.е.\ решение
оптимизационной задачи вида

\begin{equation} \sum\limits_{i=1}^n \ln \left ( \sum\limits_{j=1}^k p_j f_j
(X_i;\,\theta_j )\right ) \rightarrow \underset{p_j,\,\theta_j}{\max}\,.
\end{equation}

Конкретные алгоритмы, построенные по этой схеме, часто называют
\textit{алгоритмами типа ЕМ}, поскольку в каждом из них можно выделить два
этапа, находящихся по отношению друг к другу в последовательности
итерационного взаимодействия: \textit{оценивание} (\textit{Estimation}) и
\textit{максимизация} (\textit{Maximization})~\cite{4bat}.

Введем в рассмотрение так называемые апостериорные вероятности
$\g_{ij}$ принадлежности наблюдения $X_i$ к $j$-му классу:
\begin{equation} %6
\g_{ij} = \fr{p_j f(X_i;\,\theta_j )}{\sum\limits_{l=1}^k p_l f(X_i;\,\theta_l 
)} \ (i=1,\ldots , n;\ j=1,\ldots ,k)\,.\!\!\end{equation} 
Очевидно, что для 
всех $i=1,\ldots ,n$ и $j=1,\ldots ,k$
$$
\g_{ij} \geq 0,\quad \sum_{j=1}^k \g_{ij} =1\,.
$$


Далее обозначим $\Theta = (p_1,\ldots p_k,\,\theta_1,\ldots ,\theta_k )$ и
представим анализируемую логарифмическую функцию правдоподобия
$$
\ln L(\Theta ) = \sum\limits_{i=1}^n \ln \left (\sum\limits_{j=1}^k p_j f_j
(X_i;\,\theta_j )\right )
$$
в виде
\begin{multline}
\ln L (\Theta ) = \sum\limits_{j=1}^k\sum\limits_{i=1}^n \g_{ij} \ln p_j+{}\\
{}+\sum\limits_{j=1}^k\sum\limits_{i=1}^n \g_{ij} f(X_i;\,\theta_j)-
\sum\limits_{j=1}^k\sum\limits_{i=1}^n \g_{ij} \ln \g_{ij}\,.
\end{multline}

Справедливость этого тождества легко проверяется с учетом
$$
\sum\limits_{j=1}^k \g_{ij} =1\,.
$$

Далее идея построения итерационного алгоритма вычисления оценок
$\hat{\Theta} = (\hat{p}_1,\ldots , \hat{p}_k,\
\hat{\theta}_1,\ldots , \hat{\theta}_k)$
для параметров $\Theta = (p_1,\ldots , p_k,\ \theta_1,\ldots , \theta_k)$ состоит в
следующем:
\begin{enumerate}[1.]
\item Выбирается некоторое \textit{начальное приближение}~$\hat{\Theta}^0$.
\item \textbf{E-step:} вычисляются по формулам~(6) начальные приближения
$\g_{ij}^0$ для апостериорных вероятностей $\g_{ij}$~--- \textit{этап
оценивания}.
\item \textbf{M-step:} затем, возвращаясь к~(7), при вычисленных значениях
$\g^0_{ij}$ следует определить значения $\hat{\Theta}^1$ из условия
максимизации отдельно каждого из первых двух слагаемых правой
части~(7), поскольку первое слагаемое
$$
\sum_{j=1}^k \sum_{i=1}^n \g_{ij} \ln p_j
$$
зависит только от параметров $p_j$, а второе слагаемое
$$
\sum_{j=1}^k \sum_{i=1}^n \g_{ij} f(X_i;\,\theta_j )
$$
зависит только от параметров $\theta_j$~--- \textit{этап максимизации}.
\item Проверяется \textit{условие останова}:
$$
\parallel \Theta^{(t)} - \Theta^{t-1}\parallel <\varepsilon\,,
$$
где $t$~--- номер итерации, а
$\parallel\bullet\parallel$~--- евклидова норма, для некоторого $\varepsilon
>0$.
\end{enumerate}

Очевидно, решение оптимизационной задачи
$$
\sum\limits_{j=1}^k\sum\limits_{i=1}^n \g_{ij}^{(t)}\ln p_j \rightarrow
\underset{p_j}{\max}
$$
дается выражением (с учетом $\sum_{j=1}^k p_j =1$):
$$
p_{ij}^{(t+1)} =\fr{1}{n}\,\sum\limits_{i=1}^n \g_{ij}^{(t)}\,,
$$
где $t$~--- номер итерации, $t = 0$, 1, 2,\,\ldots

Решение оптимизационной задачи
$$
\sum\limits_{j=1}^k \sum\limits_{i=1}^n \g_{ij}^{(t)} f(X_i;\,\theta_j )
\rightarrow \underset{\theta_j}{\max}
$$
получить намного проще решения задачи~(5): выражение для $\theta_j$
записывается с учетом знания конкретного вида функций
$f(X,\,\theta)$~\cite{3bat}.

\subsection{О сходимости алгоритма} %2.2.

В работе М.\,И.~Шлезингера~\cite{5bat}, где эта схема (позднее названная
ЕМ-схемой) впервые предложена, установлены и основные свойства
реа\-ли\-зу\-ющих ее алгоритмов. В частности, было доказано, что при достаточно
широких предположениях \textit{предельные точки} всякой последовательности,
порожденной итерациями ЕМ-алгоритма, являются стационарными точками
оптимизируемой логарифмической функции правдоподобия $\ln L(\Theta )$ и что
найдется неподвижная точка алгоритма, к которой будет сходиться каждая из таких
последовательностей. Если дополнительно потребовать положительной
определенности информационной мат\-ри\-цы Фишера для $\ln L(\Theta )$ при
истинных зна\-че\-ни\-ях па\-ра\-мет\-ра $\Theta$, то можно показать, что
асимптотически по $n\rightarrow\infty$ (т.е.\ при больших выборках) существует
единственное сходящееся (по веро\-ят\-но\-сти) решение $\hat{\Theta}(n)$
уравнений метода максимального правдоподобия и, кроме того, существует в
пространстве параметров $\Theta$ норма, в которой последовательность
$\Theta^{(t)}(n)$, порожденная ЕМ-ал\-го\-рит\-мом, сходится к $\hat{\Theta}
(n)$, если только начальная аппроксимация $\hat{\Theta}^0$ не была слишком
далека от~$\hat{\Theta} (n)$. {%\looseness=1

}

Таким образом, результаты исследования свойств ЕМ-алгоритмов метода
максимального правдоподобия разделения смеси и их практическое
использование показали, что они являются достаточно работоспособными (при
известном чис\-ле компонентов смеси) даже при большом чис\-ле $k$ компонентов и
при высоких размерностях анализируемого признака~$X$~\cite{3bat}.

\subsection{Уравнения для смеси экспоненциальных распределений}
%2.3.

Применим описанный выше алгоритм к разделению смеси
экспоненциальных распределений:
$$
p(x) = \sum\limits_{j=1}^k p_j \lambda_j e^{-\lambda_j x}\,.
$$
Получаем следующие итерационные уравнения:
\begin{align*}
\g_{ij}^{(t+1)} & = \fr{p_j^{(t)}\lambda_j^{(t)}e^{-
\lambda_j^{(t)}X_i}}{\sum\limits_{l=1}^k p_l^{(t)}\lambda_l^{(t)}
e^{-\lambda_l^{(t)}X_i}}\,,\\
p_j^{(t+1)} & = \fr{1}{n}\,\sum\limits_{i=1}^n \g_{ij}^{(t)}\,.
\end{align*}

Чтобы найти  оценки $\lambda_j$, подсчитаем первую производную функции
$$\sum_{j=1}^k\sum_{i=1}^n \g_{ij}^{(t)} \ln (\lambda_j e^{-\lambda_j X_i}):$$
\vspace*{-8pt}
\begin{multline*}
\left ( \sum\limits_{j=1}^k \sum\limits_{i=1}^n
\g_{ij}^{(t)}\ln \left ( \lambda_j
e^{-\lambda_j X_i} \right ) \right )^\prime \lambda_j =\\[-3pt]
{}= \left (
\sum\limits_{j=1}^k\sum\limits_{i=1}^n \g_{ij}^{(t)}\ln (\lambda_j -\lambda_j X_i )
\right )^\prime \lambda_j =\\[-3pt]
{}= \sum\limits_{i=1}^n \g_{ij}^{(t)}\left (
\fr{1}{\lambda_j} - X_i \right )\,.
\end{multline*}

Разрешая уравнение
$$
\sum\limits_{i=1}^n \g_{ij}^{(t)}\left ( \fr{1}{\lambda_j} -X_i\right ) =0
$$
относительно $\lambda_j$, получаем следующее итерационное уравнение:
$$
\lambda_j^{(t+1)} = \fr{\sum\limits_{i=1}^n
\g_{ij}^{(t)}}{\sum\limits_{i=1}^n \g_{ij}^{(t)} X_i}\,.
$$

\subsection{Уравнения для смеси гамма-распределений } %2.4.

Применим теперь ЕМ-алгоритм к смеси гам\-ма-рас\-пре\-де\-ле\-ний вида
$$
p(x) = \sum\limits_{j=1}^k p_j \fr{\alpha_j^{\alpha_j} x^{\alpha_j -
1}}{\lambda_j^{\alpha_j} \Gamma (\alpha_j )}\,e^{-(\alpha_j / \lambda_j)x}\,.
$$

Такая параметризация удобна для нахождения
оценок~$\alpha_j$~\cite{6bat}.

Аналогичным способом выписываются итерационные уравнения:
\begin{align*}
\g_{ij}^{(t+1)} & =   \fr{p_j^{(t)}\fr{(\alpha_j^{\alpha_j} )^{(t)}
x^{\alpha_j - 1}}{(\lambda_j^{\alpha_j} )^{(t)}\Gamma (\alpha_j)}\,
e^{-(\alpha_j /\gamma_j)^{(t)}x}}{\sum\limits_{l=1}^k
p_l^{(t)}\fr{(\alpha_l^{\alpha_l})^{(t)} x^{\alpha_l -
1}}{(\lambda_l^{\alpha_l})^{(t)}\Gamma (\alpha_l )}\,
e^{-(\alpha_l /\lambda_l)^{(t)} x}}\,,\\
p_j^{(t+1)} & = \fr{1}{n}\,\sum\limits_{i=1}^n \g_{ij}^{(t)}\,.
\end{align*}

Далее найдем оценки $\lambda_j$ для данного случая, приравнивая
производную
\begin{equation} %8
\sum\limits_{j=1}^k \sum\limits_{i=1}^n \g_{ij}^{(t)} \ln \left (
\fr{\alpha_j^{\alpha_j} x^{\alpha_j -1}}{\lambda_j^{\alpha_j}\Gamma
(\alpha_j)}\,e^{-(\alpha_j /\lambda_j) x}\right )
\end{equation}
по $\lambda_j$ к нулю и разрешая относительно~$\lambda_j$ уравнение:
$$
\sum\limits_{i=1}^n \g_{ij}^{(t+1)}\left ( \fr{\alpha_j^{(t)}}{\lambda_j^{(t)}}
- \fr{\alpha_j^{(t)}X_i}{\left ( \lambda_j^{(t)}\right )^2}\right ) =0 \,.
$$
Получаем
$$
\lambda_j^{(t+1)} = \fr{\sum\limits_{i=1}^n \g_{ij}^{(t)}
X_i}{\sum\limits_{i=1}^n \g_{ij}^{(t)}}\,.
$$

Для того чтобы получить итерационные уравнения для $\alpha_j$, найдем
первую производную~(8):
\begin{multline*}
\left ( \sum\limits_{j=1}^k\sum\limits_{i=1}^n \g_{ij}^{(t)}\ln \left (
\fr{\alpha_j^{\alpha_j} x^{\alpha_j -1}}{\lambda_j^{\alpha_j}\Gamma (\alpha_j
)}\,e^{-(\alpha_j /\lambda_j ) x} \right ) \right )^\prime \alpha_j ={}\\[-3pt]
{}=\left ( \sum\limits_{j=1}^k\sum\limits_{i=1}^n \g_{ij}^{(t)}\ln \left (
\fr{\alpha_j^{\alpha_j}}{\lambda_j^{\alpha_j}}\right ) - \ln \Gamma (\alpha_j )+{} \right.\\[-3pt]
{}+\left.
(\alpha_j -1 )\ln X_i - \fr{\alpha_j}{\lambda_j}\,X_i \right )^\prime \alpha_j =\\[-3pt]
{}=\sum\limits_{i=1}^n \g_{ij}^{(t)} \left ( \ln \alpha_j +1-\ln \lambda_j -
\fr{\Gamma^\prime (\alpha_j )}{\Gamma (\alpha_j)}\right.+\\[-3pt]
{}+\left. \ln X_i - \fr{X_i}{\lambda_j}\right )\,;
\end{multline*}
\begin{multline*}
\sum\limits_{i=1}^n \g_{ij}^{(t)} \left(  \ln \alpha_j +1 -\ln \lambda_j -{}\right. \\[-3pt]
\left. {}-\fr{\Gamma^\prime (\alpha_j )}{\Gamma (\alpha_j )}+\ln X_i 
-\fr{X_i}{\lambda_j} \right) =0\,;
\end{multline*}
\begin{multline}
\fr{\Gamma^\prime (\alpha_j )}{\Gamma (\alpha_j )} ={}\\[-3pt]
{}= \fr{\sum\limits_{i=1}^n \g_{ij}^{(t)} \left ( \ln \alpha_j +1-\ln\lambda_j 
+\ln X_i -\fr{X_i}{\lambda_j} \right )}{\sum\limits_{i=1}^n \g_{ij}^{(t)}}\,.
\end{multline}
%
Здесь $\Gamma^\prime (\alpha_j ) / \Gamma (\alpha_j )$~--- это
\textit{логарифмическая производная гамма-функции}. Для нее существует так
называемое \textit{разложение Абрамовитца}--\textit{Стигана}~\cite{4bat}:
$$
\fr{\Gamma^\prime (\alpha ) }{ \Gamma (\alpha )} = \mathrm{log}\,\alpha -
\fr{1}{2\alpha }-\fr{1}{12\alpha^2 }+\ldots
$$

Подставим первые три члена разложения в~(9) и разрешим это уравнение
относительно~$\alpha_j$:
$$
\alpha_{ij}^{(t+1)} = \fr{\sum\limits_{i=1}^n
\g_{ij}^{(t+1)}}{2\sum\limits_{i=1}^n \g_{ij}^{(t +1)}\left ( \fr{X_i}{\lambda_j^{(t)}} -
\ln \fr{X_i}{\lambda_j^{(t)}} -1\right )}\,.
$$
В итоге получаем итерационные уравнения для ~$\alpha_j$.

\section{Описание программного обеспечения (программа~ЕМ)}

\subsection{Назначение программы} %3.1.

Разработанная авторами статьи программа ЕМ предназначена для решения задачи
разделения смесей экспоненциальных и гамма-распределений, поставленной в
разд.~2, с использованием ЕМ-ал\-го\-рит\-ма и формул, описанных в разд.~3.

\subsection{Инструменты разработки} %3.2.

Для создания программы была использована среда разработки Microsoft
Visual Studio .NET 2005 и объектно-ориентированный язык C\#. Для
визуализации результатов была использована свободно распространяемая
графическая библиотека ZedGraph~\cite{7bat}.


\subsection{Возможности  программы} %3.3.

\noindent
\begin{itemize}
\item Загрузка выборочных данных из текстового файла
\item Оценивание по выборке параметров смеси экспоненциальных
распределений
\item Оценивание по выборке параметров смеси гамма-распределений
\item Отслеживание изменений параметров смесей распределений во
времени в режиме <<скользящего окна>>
\item Построение гистограммы по выборке
\end{itemize}

\subsection{Входные и выходные данные. Функционирование
программы} %3.4.

В качестве \textit{входных данных} программа ЕМ получает:
\begin{itemize}
\item выборочные данные из текстового файла;
\item число компонентов смеси;
\item размер <<скользящего окна>>;
\item размер класса гистограммы.
\end{itemize}

На \textit{выходе} мы получаем:
\begin{itemize}
\item точечные оценки параметров смеси экспоненциальных
распределений;
\item точечные оценки параметров смеси гамма-распределений;
\item графическое изображение результирующей смеси распределения;
\item графическое изображение компонентов каж\-дой смеси;
\item графическое изображение того, как меняются параметры смесей
распределений с течением времени в режиме <<скользящего окна>>;
\item гистограмма, построенная по выборке;
\item значение статистического теста.
\end{itemize}

Выборочные данные загружаются из текстового файла в память программы и подаются
на вход двум независимо работающим реализациям ЕМ-алгоритма~--- для
идентификации смеси экспоненциальных распределений и для идентификации смеси
гамма-распределений. Результатом их работы являются наборы значений оцениваемых
параметров модели, предложенной в разд.~2. Кроме того, результирующие
распределения визуализируются в виде графиков. В программе можно запустить
режим <<скользящего окна>>, который для всех подвыборок заданного
размера с помощью ЕМ-алгоритма оценивает параметры смесей распределений этих
подвыборок. Все действия программы документируются в окне информации.

\section{Описание тестовых расчетов}

С использованием разработанной программы были проведены тестовые
расчеты на выборочных данных реального сетевого трафика.

На вход программы EM были поданы выборки трафика:
\begin{enumerate}[I]
\item Между лабораторией Lawrence Berkeley (Berkeley, California) и
внешним миром размера примерно 7000~\cite{8bat}~--- \textit{выборка~1}.
\item
Сети радиодоступа ЗАО <<Синтерра>> размера примерно 1000~\cite{9bat}~---
 \textit{выборка~2}.
\end{enumerate}

\subsection{Выборка 1 ``Berkeley''} %5.1.

При числе компонентов смеси~5 и случайном начальном приближении
были получены результаты, представленные в табл.~\ref{t1bat}.


Данные результаты иллюстрирует рис.~\ref{f5bat}.

Гистограмма  на рис.~\ref{f6bat} показывает статистическую значимость
полученных результатов.

Данная выборка обладает той особенностью, что она собиралась в течение
достаточно длительного времени и в ней агрегирован самый разнородный
трафик. Поэтому в ней присутствует не только большое количество
<<коротких>> сообщений (что обычно для выборок из телетрафика), но и
некоторый массив сообщений средней длины, а также определенный
<<выброс>> больших сообщений. Это свидетельствует о \textit{пиковости}
телетрафика на довольно больших промежутках времени.

Как мы видим, ЕМ-алгоритм удачно справился с задачей идентификации
смеси.

\subsection{Выборка~2 ``Synterra''} %5.2.

Результаты применения ЕМ-алгоритма к выборке ``Synterra''
представлены в табл.~\ref{t2bat}.
\begin{table*}\small
\begin{minipage}[t]{76mm}
\begin{center}
\Caption{Результаты применения ЕМ-алго\-рит\-ма к выборке~1 ``Berkeley'' 
\label{t1bat}} \vspace*{2ex}

\tabcolsep=8.7pt
\begin{tabular}{|c|c|c|}
\hline
№&Начальное приближение&Результат\\
\hline
\multicolumn{3}{|c|}{$P$}\\
\hline
0&0,2&0,1896\\
1&0,2&0,1858\\
2&0,2&0,1830\\
3&0,2&0,2259\\
4&0,2&0,2154\\
\hline
\multicolumn{3}{|c|}{$\alpha$}\\
\hline
0&2,7028&10,9783\hphantom{9}\\
1&3,6273&5,8621 \\
2&5,7598&2,7092\\
3&0,2315&1,0235\\
4&0,9110&0,4772\\
\hline
\multicolumn{3}{|c|}{$\lambda$}\\
\hline
0&85,2066&137,1714  \\
1&23,9592&136,7349\\
2&63,8425&132,6482\\
3&\hphantom{9}1,8026&116,7317\\
4&98,3882&102,5278\\
\hline
\end{tabular}
\end{center}
\end{minipage}\hfill
\begin{minipage}[t]{76mm}
%\end{table*}
%\begin{table*}\small
\begin{center}
\Caption{Результаты применения ЕМ-алго\-рит\-ма к выборке~2 ``Synterra'' 
\label{t2bat}} \vspace*{2ex}

\tabcolsep=8.7pt
\begin{tabular}{|c|c|c|}
\hline
№&Начальное приближение&Результат\\
\hline
\multicolumn{3}{|c|}{$P$}\\
\hline
0&0,2&$0{,}3815\hphantom{{}\cdot 10^{-9}}$\\
1&0,2&$0{,}3594\hphantom{{}\cdot 10^{-9}}$\\
2&0,2&$0{,}2589\hphantom{{}\cdot 10^{-9}}$\\
3&0,2&$0{,}4401\cdot 10^{-9}$\\
4&0,2&$0{,}0\hphantom{{}\cdot 10^{-9}999}$\\
\hline
\multicolumn{3}{|c|}{$\alpha$}\\
\hline
0&6,0804&1,5833\\
1&3,1838&0,8554\\
2&1,4886&0,4557\\
3&4,6407&0,2278\\
4&3,7843&0,1139\\
\hline
\multicolumn{3}{|c|}{$\lambda$}\\
\hline
0&17,3387&15,8682\\
1&47,8294&16,9150\\
2&54,1984&19,2866\\
3&\hphantom{1}8,6254&19,2866\\
4&\hphantom{1}5,7252&19,2866\\
\hline
\end{tabular}
\end{center}
\end{minipage}
\end{table*}


Данные результаты иллюстрируют рис.~\ref{f7bat}.


Эти результаты также отражают действительную картину, как показано на
рис.~\ref{f8bat}.


Этот трафик был снят с базовой станции <<Лукойл-Юго-Запад>> сети
широкополосного радиодоступа ЗАО <<Синтерра>>. Сеть радиодоступа
является реализацией так называемой <<последней мили>>, переносящей два
разных вида трафика: данные (Ethernet пакеты) и голос (IP-телефония, VoIP).
Поэтому здесь присутствуют в качестве основной массы короткие, но
интенсивные сообщения (пакеты SIP и голосовые фреймы), а также длинные
сообщения, содержащие данные.

Как мы видим, программная реализация ЕМ-ал\-го\-рит\-ма успешно справилась с
задачей разделения смесей распределений для этих двух выборок, что делает
данную программу удобным инструментом построения стохастической картины
конкретной сети. По полученным данным, используя метод интерпретации,
предложенный в разд.~2, можно получить представление о количестве
последовательных и параллельных структур вероятностной модели сети.

\subsection{Режим <<скользящего окна>>} %5.3.

Результаты для выборки
``Berkeley'' в режиме <<скользящего окна>>  представлены
на рис.~\ref{f9bat}.


Данные графики показывают изменение параметров распределений подвыборок выборки 
``Berkeley''. Видно, что параметры распределений подвыборок не остаются 
неизменными во времени, наоборот, они имеют внешне случайный характер. На 
рис.~\ref{f9bat},\,\textit{в} видна даже своеобразная пульсация первой 
компоненты.
%
На основании расчетов можно сделать вывод о том, что пиковость трафика
обусловливается как формой, так и интенсивностью сообщений.

\section{Заключение}

В данной работе исследована вероятностная модель  информационных потоков,
возникающих в сложных телекоммуникационных конвергентных сетях, построенная с
помощью асимптотического и энтропийного подходов. Эта модель предполагает, что
функционирование сложной телекоммуникационной сети можно представить в виде
суперпозиции довольно простых стохастических структур~--- последовательных и
параллельных, которые по\-рож\-да\-ют смеси гамма-распределений для случайной
величины времени обработки и передачи сообщений в сети. Предложена простая
интерпретация параметров данной модели.
\begin{figure*} %fig5
\vspace*{1pt}
\begin{center}
\mbox{%
\epsfxsize=130mm %145.109mm 
\epsfbox{bat-5.eps} }
\end{center}
\vspace*{-13pt} \Caption{Компоненты смеси начального приближения~(\textit{а}) и 
результата~(\textit{б}) для выборки~1 ``Berkeley'' \label{f5bat}}
%\end{figure*}
%\begin{figure*} %fig6
\vspace*{12pt}
\begin{center}
\mbox{%
\epsfxsize=130mm %148.256mm 
\epsfbox{bat-7.eps} }
\end{center}
\vspace*{-13pt} \Caption{График смеси распределений~(\textit{1}) и гистограмма 
для выборки~1 ``Berkeley''~(\textit{2}) \label{f6bat}}
\end{figure*}



\begin{figure*} %fig7
\vspace*{1pt}
\begin{center}
\mbox{%
\epsfxsize=130mm %144.283mm 
\epsfbox{bat-8.eps} }
\end{center}
\vspace*{-16pt} \Caption{Компоненты смеси начального приближения~(\textit{а}) и 
результата~(\textit{б}) для выборки~2 ``Synterra'' \label{f7bat}}
%\end{figure*}
%\begin{figure*} %fig8
\vspace*{12pt}
\begin{center}
\mbox{%
\epsfxsize=130mm %148.256mm 
\epsfbox{bat-10.eps} }
\end{center}
\vspace*{-11pt} \Caption{График смеси распределений~(\textit{1}) и гистограмма
для выборки~2 ``Synterra''~(\textit{2}) \label{f8bat}}
\end{figure*}

\begin{figure*} %fig9
\vspace*{1pt}
\begin{center}
\mbox{%
\epsfxsize=119.041mm
\epsfbox{bat-11.eps} }
\end{center}
\vspace*{-9pt} \Caption{Изменение  смешивающих параметров~(\textit{а}), 
параметров формы~(\textit{б}) и параметров масштаба~(\textit{в}) во времени для 
выборки~1 ``Berkeley'' \label{f9bat}}
\end{figure*}

Для решения вытекающей из модели задачи предложен итерационный алгоритм,
базирующийся на методе максимального правдоподобия~--- ЕМ-ал\-го\-ритм, для
которого получены формулы для конкретного вида смесей~--- экспоненциальных и
гамма-распределений.
%
Кроме того, разработан программный инструментарий для оценки параметров 
предложенной модели на выборках из реальных трафиковых данных. Проведены 
исследования, которые подтвердили предположения вероятностной модели. 


Получение информации о стохастической структуре
телекоммуникационных сетей и наличие программных инструментов для
выявления более или менее стабильных структур позволит понять причины
возникновения неожиданных больших нагрузок, предотвратить такие нагрузки,
а также поможет в будущем в проектировании надежных, оптимальных по
стоимости и уровню сервиса телекоммуникационных сетей нового поколения.

%\vspace*{-15pt} 
{\small\frenchspacing
{%\baselineskip=10.8pt
\addcontentsline{toc}{section}{Литература}
\begin{thebibliography}{9}
\bibitem{1bat}
Teletraffic Engeneering Handbook. International Telecommunication Union, 
Geneva, 2005 {\sf http://www.itu.int}. \vspace*{5pt} 
\bibitem{2bat}
\Au{Севастьянов~Б.\,А.} Курс теории вероятностей и математической статистики. 
М., 2004. \vspace*{5pt} 
\bibitem{3bat}
\Au{Айвазян~C.\,А., Бухштабер~В.\,М., Енюков~И.\,С, Мешалкин~Л.\,Д.} Прикладная 
статистика. Классификация и снижение размерности~// Финансы и статистика. М., 
1989. \vspace*{5pt} 
\bibitem{4bat}
\Au{Bilmes~J.\,A.} A gentle tutorial of the EM algorithm and its application to 
parameter estimation for Gaussian mixture and hidden Markov models. Berkeley, 
CA, USA: International Computer Science Institute,  1998. \vspace*{5pt} 
\bibitem{5bat}
\Au{Шлезингер~М.\,И.} О самопроизвольном различении образов~// Шлезингер~М.\,И. 
Читающие. автоматы. Киев: Наукова думка, 1965. С.~38--45. \vspace*{5pt} 
\bibitem{6bat}
\Au{Hsiao~I.-T., Rangarajan~A., Gindi~G.}. Joint-MAP 
reconstruction/segmentation for transmission tomography using mixture-models as 
priors. Yale University, 1998. \vspace*{5pt} 
\bibitem{7bat}
{\sf http://zedgraph.org}. \vspace*{4pt} 
\bibitem{8bat}
{\sf http://ita.ee.lbl.gov/html/contrib/LBL-PKT.html}. \vspace*{5pt} 
\bibitem{9bat}
{\sf http://www.synterra.ru}.
\end{thebibliography}

} } \label{end\stat}
\end{multicols}


%\addtocounter{razdel}{1}
%\def\razd{НЕРЕГУЛИРУЕМЫЙ ЭЛЕКТРОПРИВОД ДЛЯ ЭЛЕКТРОЭНЕРГЕТИКИ}

\setcounter{page}{2}

%   { %\Large  
   { %\baselineskip=16.6pt
   
   \vspace*{-48pt}
   \begin{center}\LARGE
   \textit{Предисловие}
   \end{center}
   
   %\vspace*{2.5mm}
   
   \vspace*{25mm}
   
   \thispagestyle{empty}
   
   { %\small 

    
Вниманию читателей журнала <<Информатика и её применения>> предлагается 
очередной тематический выпуск <<Вероятностно-статистические методы и 
задачи информатики и информационных технологий>>. Предыдущие тематические 
выпуски журнала по данному направлению вышли в 2008~г.\ (т.~2, вып.~2), 
в 2009~г.\ (т.~3, вып.~3) и в 2010~г.\ (т.~4, вып.~2). 

Статьи, собранные в данном журнале, посвящены разработке новых вероятностно-статистических 
методов, ориентированных на применение к решению конкретных задач информатики и информационных 
технологий, а также~--- в ряде случаев~--- и других прикладных задач. Проблематика, охватываемая 
публикуемыми работами, развивается в рамках научного сотрудничества между Институтом проблем 
информатики Российской академии наук (ИПИ РАН) и Факультетом вычислительной математики и 
кибернетики Московского государственного университета им.\ М.\,В.~Ломоносова в ходе работ 
над совместными научными проектами (в том числе в рамках функционирования 
Научно-образовательного центра <<Вероятностно-статистические методы анализа рисков>>). 
Многие из авторов статей, включенных в данный номер журнала, являются активными участниками 
традиционного международного семинара по проблемам устойчивости стохастических моделей, 
руководимого В.\,М.~Золотаревым и В.\,Ю.~Королевым; регулярные сессии этого семинара 
проводятся под эгидой МГУ и ИПИ РАН (в 2011~г.\ указанный семинар проводится в октябре 
в Калининградской области РФ). 

Наряду с представителями ИПИ РАН и МГУ в число авторов данного выпуска журнала входят 
ученые из Научно-исследовательского института системных исследований РАН, Института 
проблем технологии микроэлектроники и особочистых материалов РАН, Института 
прикладных математических исследований Карельского НЦ РАН, Московского 
авиационного института, Вологодского государственного педагогического университета, 
НИИММ им.\ Н.\,Г.~Чеботарева, Казанского государственного университета, Дебреценского 
университета (Венгрия).

Несколько статей выпуска посвящено разработке и применению стохастических методов и 
информационных технологий для решения различных прикладных задач. В~работе В.\,Г.~Ушакова 
и О.\,В.~Шестакова рассмотрена задача определения вероятностных характеристик случайных 
функций по распределениям интегральных преобразований, возникающих в задачах эмиссионной 
томографии. В~статье Д.\,О.~Яковенко и М.\,А.~Целищева рассмотрены некоторые вопросы 
математической теории риска и предложен новый подход к диверсификации инвестиционных 
портфелей. Работа И.\,А.~Кудрявцевой и А.\,В.~Пантелеева посвящена построению и 
исследованию математической модели, описывающей динамику сильноионизованной плазмы. 
В~статье П.\,П.~Кольцова изучается качество работы ряда алгоритмов сегментации изображений. 
Статья А.\,Н.~Чупрунова и И.~Фазекаша посвящена вероятностному анализу числа без\-оши\-бочных 
блоков при помехоустойчивом кодировании; получены усиленные законы больших чисел для указанных 
величин.

В данном выпуске традиционно присутствует тематика, весьма активно разрабатываемая в течение 
многих лет специалистами ИПИ РАН и МГУ,~--- методы моделирования и управления для 
информационно-телекоммуникационных и вычислительных систем, в частности методы 
теории массового обслуживания. В~статье А.\,И.~Зейфмана с соавторами рассматриваются 
модели обслуживания, описываемые марковскими цепями с непрерывным временем в случае 
наличия катастроф. В~работе М.\,М.~Лери и И.\,А.~Чеплюковой рассматриваются случайные 
графы Интернет-типа, т.\,е.\ графы, степени вершин которых имеют степенные распределения; 
такие задачи находят применение при исследовании глобальных сетей передачи данных. 
Работа Р.\,В.~Разумчика посвящена исследованию систем массового обслуживания специального 
вида~--- с отрицательными заявками и хранением вытесненных заявок.

Ряд статей посвящен развитию перспективных теоретических 
вероятностно-статистических методов, которые находят широкое применение в различных 
задачах информатики и информационных технологий. В~работе В.\,Е.~Бенинга, А.\,К.~Горшенина 
и В.\,Ю.~Королева рассмотрена задача статистической проверки гипотез о числе компонент 
смеси вероятностных распределений, приводится конструкция асимптотически наиболее мощного 
критерия. Результаты этой работы найдут применение в ряде прикладных задач, использующих 
математическую модель смеси вероятностных распределений (в информатике, моделировании 
финансовых рынков, физике турбулентной плазмы и~т.\,д.). В~статье В.\,Ю.~Королева, 
И.\,Г.~Шевцовой и С.\,Я.~Шоргина строится новая, улучшенная оценка точности нормальной 
аппроксимации для пуассоновских случайных сумм; как известно, указанные случайные суммы 
широко используются в качестве моделей многих реальных объектов, в том числе в информатике, 
физике и других прикладных областях. Работа В.\,Г.~Ушакова и Н.\,Г.~Ушакова посвящена 
исследованию ядерной оценки плотности распределения; эти результаты могут применяться, 
в част\-ности, при анализе трафика в телекоммуникационных системах. Серьезные приложения 
в статистике могут получить результаты работы О.\,В.~Шестакова, в которой доказаны оценки 
скорости сходимости распределения выборочного абсолютного медианного отклонения к нормальному 
закону. 

\smallskip

Редакционная коллегия журнала выражает надежду, что данный тематический  выпуск 
будет интересен специалистам в области теории вероятностей и математической статистики 
и их применения к решению задач информатики и информационных технологий.
     
     %\vfill 
     \vspace*{20mm}
     \noindent
     Заместитель главного редактора журнала <<Информатика и её 
применения>>,\\
     директор ИПИ РАН, академик  \hfill
     \textit{И.\,А.~Соколов}\\
     
     \noindent
     Редактор-составитель тематического выпуска,\\
     профессор кафедры математической статистики факультета\\
      вычислительной математики и кибернетики МГУ им.\ М.\,В.~Ломоносова,\\
     ведущий научный сотрудник ИПИ РАН,\\ 
доктор физико-математических наук \hfill
      \textit{В.\,Ю.~Королев}
     
     } }
     }

\def\stat{agalarov}


\def\tit{ПРИБЛИЖЕННЫЙ МЕТОД ВЫЧИСЛЕНИЯ ХАРАКТЕРИСТИК УЗЛА 
ТЕЛЕКОММУНИКАЦИОННОЙ СЕТИ С~ПОВТОРНЫМИ ПЕРЕДАЧАМИ}
\def\titkol{Приближенный метод вычисления характеристик узла 
телекоммуникационной сети с~повторными передачами} 

\def\autkol{Я.\,М.~Агаларов}
\def\aut{Я.\,М.~Агаларов$^1$}

\titel{\tit}{\aut}{\autkol}{\titkol}

%{\renewcommand{\thefootnote}{\fnsymbol{footnote}}\footnotetext[1]
%{Работа выполнена при поддержке РФФИ, проекты 08--07--00152 и 08--01--00567.}}

\renewcommand{\thefootnote}{\arabic{footnote}}
\footnotetext[1]{Институт проблем
информатики Российской академии наук, agglar@yandex.ru}

%\vspace*{-6pt}


\Abst{Рассмотрена модель узла коммутации пакетов c повторными передачами для двух 
схем распределения буферной памяти: полнодоступной и полного разделения. Предложен 
приближенный метод вычисления интенсивностей потоков и вероятностей блокировок узла. 
Получены необходимые и достаточные условия существования и единственности решения 
уравнения для потоков в узле при установившемся режиме работы и доказана сходимость 
итерационного метода решения указанного уравнения.}

\KW{узел коммутации пакетов; буферная память; повторные передачи; вероятности 
блокировок; итерационный метод}

      \vskip 18pt plus 9pt minus 6pt

      \thispagestyle{headings}

      \begin{multicols}{2}

      \label{st\stat}


\section{Введение}

    Одной из основных задач предварительного анализа 
телекоммуникационных сетей коммутации пакетов с ограниченной буферной 
памятью является расчет характеристик потоков и вероятностей блокировок в 
узлах связи. Важность указанных характеристик определяется тем, что от их 
значений существенным образом зависят другие основные показатели сети 
(пропускная способность, задержки пакетов и~др.). 

    Существует множество различных моделей узлов коммутации пакетов и 
методов их расчета (см., например,~[1--6]). Для моделей, рассматривающих 
узел с ограниченной буферной памятью как систему массового обслуживания 
(CMO) типа 
$
\begin{matrix}
M \\ \lambda
\end{matrix}
\left |
\begin{matrix}
M \\ \lambda
\end{matrix}
\right |
\overline{m} \vert N
$ или  $\vert PH\vert PH\vert 1\vert r$, в предположении отсутствия повторных 
передач пакетов получены точные методы вычисления характеристик 
узлов~[1, 3, 4, 6]. Приближенные методы расчета узлов, учитывающие повторные 
попытки передачи, используют модели типа $\vert PH\vert PH\vert 1\vert r$ или 
$
\begin{matrix}
M \\ \lambda
\end{matrix}
\left |
\begin{matrix}
M \\ \lambda
\end{matrix}
\right |
1 \vert N
$ и являются 
итерационными~[2, 3, 5, 7]. Для моделей типа 
$BM\!AP\vert PH\vert 1$, $M\vert G\vert 1\vert r$ и $M\!AP\vert 
(PH,PH)\vert 1$ с повторными заявками получены точные методы вычисления 
характеристик (например, в работах~[8--10]), которые также могут быть 
использованы при расчете узлов.

    Ниже будут рассмотрены модели узла коммутации пакетов с повторными 
передачами для двух схем распределения буферной памяти: с 
полнодоступными буферами и с полным разделением буферной памяти. 
Предлагается приближенный метод расчета характеристик, который в качестве 
модели узла использует СМО типа $
\begin{matrix}
M \\ \lambda
\end{matrix}
\left |
\begin{matrix}
M \\ \lambda
\end{matrix}
\right |
\overline{m} \vert N
$ с повторными заявками. Доказаны утверждения о 
достаточных и необходимых условиях существования и единственности 
решения уравнения для вероятности блокировки в установившемся режиме 
работы и сходимости предлагаемого итерационного метода. 

\section{Модель узла}

    Математическая модель узла представляется в виде СМО с ограниченной 
буферной памятью и различными потоками заявок, каждая из которых требует 
обслуживания только на одной из многоканальных линий связи. 

    Пусть $0<N<\infty$~--- число мест хранения в буферной памяти, $u$~--- 
узел связи, $v$~--- линия связи, $\Omega_u^+$~--- множество исходящих из 
узла~$u$ линий, $c_v$~--- канальная емкость линии~$v$. Поток заявок, 
тре\-бу\-ющих обслуживания на линии~$v$, назовем $v$-по\-то\-ком, заявки этого 
потока~--- $v$-за\-яв\-ка\-ми.


    Пусть выполняются следующие предположения: 
\begin{enumerate}[1.]
\item Места в буферной памяти распределяются согласно одной из двух 
схем:
\begin{enumerate}[($i$)]
\item полнодоступная схема~--- каждое свободное место хранения доступно 
любой заявке;
\item схема полного разделения памяти~--- $v$-за\-яв\-кам доступны всего 
$N_v$ мест, где $\sum\limits_{v\in\Omega_u^+} N_v=N$.
\end{enumerate}
\item Если в момент поступления $v$-заявки в буферной памяти есть 
доступное свободное место, то она сразу занимает это место. Если в момент 
поступления $v$-заявки в системе нет свободного доступного места 
хранения, то поступившая заявка через некоторое время повторно поступает 
на систему, оставаясь $v$-заявкой. 
\item Интенсивности первичных потоков $v$-заявок~--- заданные величины 
$0<\Lambda_v<\infty$, $v\in \Omega_u^+$. Суммарные потоки первичных и 
повторных $v$-заявок являются независимыми в совокупности 
пуассоновскими потоками. Для обслуживания $v$-заявки требуется 
одновременно одно место хранения и один канал типа~$v$, $v\in 
\Omega_u^+$.
\item Первичные нагрузки~--- реализуемые, т.\,е.\ в данном случае 
интенсивности входных первичных потоков равны интенсивностям 
выходных потоков выполненных заявок. 
\item Принятые в СМО $v$-заявки обслуживаются линией~$v$ в порядке 
поступления. 
\item Время занятия канала $v$-заявкой~--- экспоненциально 
распределенная случайная величина с параметром $0<\mu_v<\infty$, 
$v\in\Omega_u^+$, независимая от других случайных событий в узле.
\item Выполненная $v$-заявка с вероятностью~$B_v$ повторяется через 
заданное время~$\tau_v$ (тайм-аут) и с вероятностью $1-B_v$ покидает 
систему через время~$t_v$ навсегда, сразу освободив занятый канал и место 
буферной памяти.
\end{enumerate}

   Будем говорить, что узел блокирован для $v$-за\-яв\-ки, если в буферной 
памяти отсутствует доступное место хранения. Ставится задача вычисления 
вероятностей блокировок и интенсивностей потоков в узле.

\section{Вычисление вероятности блокировки и~интенсивностей~потоков} 

   Пусть $\Lambda_v^*$~--- интенсивность суммарного потока внешних 
заявок, требующих передачи по линии~$v$, $\pi_v$~--- вероятность блокировки 
узла для заявок, требующих передачи по исходящей из узла линии~$v$. 

    Пусть в узле используется полнодоступная схема распределения 
буферной памяти. Тогда, как следует из описания модели, $\pi_v 
=\pi_{v^\prime},\,v,\,v^\prime\in \Omega_u^+$, и для 
интенсивностей~$\Lambda_v^*$, $v\in\Omega_u^+$, справедливы соотношения:
\begin{equation*}
\Lambda_v^* = \fr{\Lambda_v}{1-\pi}\,,
%\label{e1aga}
\end{equation*}
    где
    $\pi =\pi_v$, $v\in\Omega_u^+$.

    Пусть 
    $\overline{k} = \{\overline{k}_v$, $v\in\Omega_u^+\}$~--- состояние 
буферной памяти узла, $\overline{k}_v =\left ( k_v,\,k_v^\prime,\,k_v^{\prime\prime}\right )$; 
$k_v$~--- число $v$-заявок в буферной 
памяти, ожидающих выполнения линией~$v$; $k^\prime_v$~--- число 
$v$-заявок в буферной памяти, ожидающих тайм-аут и неуспешно переданных 
в последующий узел; $k_v^{\prime\prime}$~--- число $v$-за\-явок в буферной 
памяти, успешно переданных в последующий узел и ожидающих 
потверждения; 
$A_m = \left \{ \overline{k}:\ \sum\limits_{v\in\Omega_u^+} \left ( 
k_v+k_v^\prime + k_v^{\prime\prime}\right ) =m \right \}$~--- множество различных 
состояний, при которых в памяти узла занято ровно $m$~буферов. Тогда с 
учетом введенных выше обозначений и предположений для ве\-ро\-ят\-ности 
блокировки узла можно написать формулу~\cite{1aga, 2aga}:
\begin{equation}
\pi = \fr{1}{G_N}\sum\limits_{\overline{k}\in A_N} 
p\left (\overline{k},\overline{\rho}^*\right )\,,
\label{e2aga}
\end{equation}
где  
\begin{gather}
p(\overline{k},\overline{\rho}^*) = \prod\limits_{v\in\Omega_u^+} z_v (\pi, 
\rho_v , k_v , k_v^\prime , k_v^{\prime\prime})\,;\\
z_v (\pi, \rho_v , k_v , k_v^\prime , k_v^{\prime\prime}) ={}\notag\\
\!\!{}=
\begin{cases}
 \fr{\rho_v^{\prime *k_v^\prime}}{k_v^{\prime}!}\,
\fr{\rho_v^{\prime\prime * k_v^{\prime\prime}}}{ k_v^{\prime\prime}!}  \,
\fr{\rho_v^{*k_v}}{ k_{v}!} 
&\mbox{при}\ k_v<c_v\,,\\
 \fr{\rho_v^{\prime * k_v^\prime}}{k_v^{\prime}!} \,
\fr{\rho_v^{\prime\prime * k_v^{\prime\prime}}} { k_v^{\prime\prime}!} 
\fr{\rho_v^{*k_v}}{ c_{v}!c_v^{k_v- c_v}} 
& \mbox{при}\ k_v\geq c_v\,;
\end{cases}\\
G_N = \sum\limits_{m=0}^N\sum\limits_{\overline{k}\in A_m}
p(\overline{k},\overline{\rho}^*)\,;\\ 
\overline{\rho}^*=\{\rho_v^*,\,v\in\Omega_u^+\}\,;\\
\rho_v^* = \fr{\rho_v}{1-\pi}\,;\quad \rho_v =\fr{\Lambda_v}{\mu_v(1- B_v)}\,;\\
\rho_v^{\prime *} =\rho_v^*\mu_v\tau_vB_v\,;\quad \rho_v^{\prime\prime *}=
p_v^* \mu_vt_v,\,\quad  v\in \Omega_u^+\,.\label{e3aga}
\end{gather}

Переобозначив $1-\pi$ через $y$, выражение в правой части равенства~(2)~--- через 
$p_{\overline{k}}(\overline{\rho},y)$, выражение в правой части равенства~(4)~--- 
через $g_N(\overline{\rho},y)$, а выражение в правой 
части равенства~(1)~--- через $1-q_N (\overline{\rho},y)$, 
где $\overline{\rho} = (\rho_v,\,v\in \Omega_u^+)$, $\rho_v = \rho_v^*y\;=$\linebreak 
$=\;\Lambda_v/(\mu_v(1-B_v))$, $v\in\Omega_u^+$, получим нелинейное уравнение 
относительно неизвестной переменной~$y$:
\begin{equation}
y=q_N(\overline{\rho},y)\,.
\label{e4aga}
\end{equation}

    Решим уравнение~(8). Как следует из~(2)--(7), верно 
равенство
\begin{equation}
q_N(\overline{\rho},y) = \fr{g_{N-1}(\overline{\rho},y )}{g_N(\overline{\rho},y)}\,.
\label{e5aga}
\end{equation}
Введем функцию  $d_n(\overline{\rho} ,y)$ среднего числа заявок в узле с 
буферной памятью емкости $n\geq 0$:
$$
d_n(\overline{\rho} ,y) = 
\fr{1}{g_n(\overline{\rho},y)}\,\sum\limits_{m=0}^n m\sum\limits_{\overline{k}\in 
A_m} p_{\overline{k}}(\overline{\rho},y)\,.
$$
Заметим, что $g_n$, $d_n$ и $q_n$, 
$n\geq 0$,~--- непрерывно-дифференцируемые функции по $y\in (0,\,1]$. Взяв 
производную функции~$g_n$ по~$y$, из~(2)--(7) получим
\begin{multline}
\fr{\partial g_n(\overline{\rho},y)}{\partial y} ={}\\
{}= -\sum\limits_{m=0}^n m 
\sum\limits_{\overline{k}\in A_m}\fr{\prod\limits_{v\in\Omega_u^+} z_n 
(0,\rho_v, k_v, k_v^\prime , k_v^{\prime\prime})}{y^{m+1}}={}\\
{}= -\fr{1}{y}\,g_n (\overline{\rho},y)d_n(\overline{\rho},y)\,.
\label{e6aga}
\end{multline}
Взяв производную функции $q_N$ по $y$, из~(\ref{e5aga}) и~(\ref{e6aga}) 
получим
\begin{equation}
\fr{\partial q_N(\overline{\rho},y)}{\partial y} = \fr{q_N(\overline{\rho},y)}{y}\left 
[ d_N (\overline{\rho},y)-d_{N-1}(\overline{\rho},y)\right ]\,.
\label{e7aga}
\end{equation}
    Докажем несколько утверждений о свойствах 
функции~$q_N(\overline{\rho},y)$.
\medskip

\noindent
\textbf{Утверждение 1.} \textit{Справедливы неравенства}
\begin{multline}
0<d_{n+1}(\overline{\rho},y)-d_n(\overline{\rho},y) <1\,,\\
\ \ \ \ \ \ \ \ \ \ \ \ \ \ \ \ \ \ \ \ y\in (0,\,1]\,, \ n\geq 0\,.
\label{e8aga}
\end{multline}


\noindent

Д\,о\,к\,а\,з\,а\,т\,е\,л\,ь\,с\,т\,в\,о\,.\ Подставив выражение для функции 
$d_n(\overline{\rho},y)$ и проведя преобразования, получим
\begin{multline*}
d_{n+1}(\overline{\rho},y) -d_n(\overline{\rho},y) = 
\fr{\sum\limits_{m=0}^{n+1}m\sum\limits_{\overline{k}\in A_m} 
p_{\overline{k}}(\overline{\rho},y)}
{\sum\limits_{m=0}^{n+1}
\sum\limits_{\overline{k}\in A_m} p_{\overline{k}}(\overline{\rho},y)} - {}\\
{}-
\fr{\sum\limits_{m=0}^n m \sum\limits_{\overline{k}\in A_m} p_{\overline{k}} 
(\overline{\rho},y)}{\sum\limits_{m=0}^n
\sum\limits_{\overline{k}\in A_m}p_{\overline{k}}(\overline{\rho},y)}={}\\
{}=\fr{\sum\limits_{m=1}^n m \sum\limits_{\overline{k}\in 
A_m}p_{\overline{k}}(\overline{\rho},y)+(n+1)\sum\limits_{\overline{k}\in 
A_{n+1}}  p_{\overline{k}}(\overline{\rho},y)}{\sum\limits_{m=0}^n\sum\limits_{\overline{k
}\in A_m}p_{\overline{k}}(\overline{\rho},y)+\sum\limits_{\overline{k}\in 
A_{n+1}}p_{\overline{k}}(\overline{\rho},y)} -{}
\end{multline*}
\begin{multline}
{}-
\fr{\sum\limits_{m=0}^n m 
\sum\limits_{\overline{k}\in A_m}p_{\overline{k}}(\overline{\rho},y)}
{\sum\limits_{m=0}^n\sum\limits_{\overline{k}\in A_m} 
p_{\overline{k}}(\overline{\rho},y)}={}\\
{}=\fr{(n+1)\sum\limits_{\overline{k}\in 
A_{n+1}}p_{\overline{k}}(\overline{\rho},y)g_n(\overline{\rho},y)}{g_{n+1}(\overline{\rho},y) g_n(\overline{\rho},y)} -{}\\
{}-
\fr{\sum\limits_{\overline{k}\in 
A_{n+1}}p_{\overline{k}}(\overline{\rho},y)\sum\limits_{m=0}^n  m 
\sum\limits_{\overline{k}\in A_m} p_{\overline{k}}(\overline{\rho},y) }
{g_{n+1}(\overline{\rho},y) g_n(\overline{\rho},y)}
={}\\
{}=\left [ 1-q_{n+1}(\overline{\rho},y)\right ] \left [n+1-d_n(\overline{\rho},y)\right ]\,.
\label{e9aga}
\end{multline}


    Докажем утверждение~1 методом индукции. При $n = 0$, как следует 
из~(\ref{e9aga}), имеем
$$
d_2(\overline{\rho},y) - d_1 (\overline{\rho},y) =1-q_1(\overline{\rho},y)\,,
$$
    т.\,е.\ утверждение~1 при $n = 0$ справедливо. 

    Пусть неравенства~(\ref{e8aga}) справедливы для некоторого $n > 0$. 
Докажем, что они справедливы и для $n + 1$. Из~(\ref{e9aga}) получаем
\begin{multline*}
d_{n+1}(\overline{\rho},y)- d_n(\overline{\rho},y)={}\\
{}=\left [ 1-
q_{n+1}(\overline{\rho},y)\right ] \left [n+1-d_n(\overline{\rho},y)\right ] ={}\\
{}= \left [ 1-
1-q_{n+1}(\overline{\rho},y)\right ] \left [ n-{}\right.\\
{}-\left. d_{n-1}(\overline{\rho},y)+d_{n-1}(\overline{\rho},y)-
d_n(\overline{\rho},y)+1\right ] ={}\\
{}=\left [ 1-q_{n+1}(\overline{\rho},y)\right ] 
\left [ n-d_{n-1}(\overline{\rho},y)-{}\right.\\
{}-\left. \left ( d_n(\overline{\rho},y)-d_{n-1}(\overline{\rho},y)\right )+1\right] = {}\\
{}=
\left [ 1-q_{n+1}(\overline{\rho},y)\right ]
\left [ 
\fr{d_n(\overline{\rho},y) -d_{n-1}(\overline{\rho},y)}{1-
q_n(\overline{\rho},y)}\right.-{}\\
{}-\left.
\left ( d_n(\overline{\rho},y)-d_{n-1}(\overline{\rho},y)\right )+1
\vphantom{\fr{d_n(\overline{\rho})}{(q_n)}}
\right ]={}\\
{}=
\left [ 1-q_{n+1}(\overline{\rho},y)\right ]
\left [ 
\vphantom{\fr{d_n(\overline{\rho})}{(q_n)}}
\left ( d_n(\overline{\rho},y\right)\right. -{}\\
 {}-\left.
d_{n-1}\left(\overline{\rho},y)\right )\fr{q_n(\overline{\rho},y)}{1-
q_n(\overline{\rho},y)}+1\right ]\,.
\end{multline*}
Так как по предположению $d_n (\overline{\rho},y) -d_{n-1}(\overline{\rho},y) 
>0$, то правая часть последнего равенства больше нуля; следовательно, 
$d_{n+1}(\overline{\rho},y)-d_n(\overline{\rho},y)>0$. 

    Продолжив преобразование правой части последнего равенства и 
учитывая предположение $d_n(\overline{\rho},y) -d_{n-1}(\overline{\rho},y)<1$, 
получим
\begin{multline*}
d_{n+1}((\overline{\rho},y) -d_n(\overline{\rho},y)<{}\\
{}< \left [ 1-
q_{n+1}(\overline{\rho},y)\right ]
\left ( \fr{q_n(\overline{\rho},y)}{1-q_n(\overline{\rho},y)}+1\right )={}\\
{}=
\fr{1-q_{n+1}(\overline{\rho},y)}{1-q_n(\overline{\rho},y)}<1\,,
\end{multline*}
так как $0< q_n(\overline{\rho},y)<q_{n+1}(\overline{\rho},y)<1$, $n>0$, $y\in 
(0,\,1]$.

Следовательно, утверждение~1 доказано.

\medskip

\noindent
\textbf{Утверждение 2.} $q_N(\overline{\rho},y)$~--- \textit{монотонно-воз\-рас\-та\-ющая 
функция по $y\in (0,\,1]$. При этом $0< q_N(\overline{\rho},y)\;\leq $\linebreak 
$\leq\;q_N(\overline{\rho},1) <1$, $y\in (0,\,1]$,  и $\underset{y\rightarrow 
0}{\mathrm{lim}}\,q_N(\overline{\rho},y) =0$}.

\medskip

\noindent
Д\,о\,к\,а\,з\,а\,т\,е\,л\,ь\,с\,т\,в\,о\,.\  Возрастание функции 
$q_N(\overline{\rho},y)$ следует непосредственно из~(\ref{e7aga}) и 
утверж\-де\-ния~1. Доказательство неравенств в условии утверждения очевидно 
следует из~(\ref{e5aga}) и вида функции $g_n (\overline{\rho},y)$, $n\geq 0$. 
Для предела имеем:
\begin{multline*}
\underset{y\rightarrow 0}{\mathrm{lim}}\,q_N(\overline{\rho},y) 
=\underset{y\rightarrow 0}{\mathrm{lim}}\,\fr{g_{N- 1}(\overline{\rho},y)}{g_N(\overline{\rho},y)} = {}\\
{}= \underset{y\rightarrow 0}{\mathrm{lim}}\,\left (
g_{N-1}(\overline{\rho},y)\Bigg / \left ( 
\vphantom{\prod\limits_{v\in\Omega_u^+}}
g_{N-1}(\overline{\rho},y)\right.\right.+{}\\
{}+\left.\left.\sum\limits_{\overline{k}\in A_N}\prod\limits_{v\in\Omega_u^+} 
\fr{z_v(0,\rho_v,k_v,k^\prime_v,k^{\prime\prime}_v)}{y^N}\right )\right ) = {}\\
{}= \underset{y\rightarrow 0}{\mathrm{lim}}\,\left (
y^N g_{N-1}(\overline{\rho},y)\Bigg / 
\left ( 
\vphantom{\prod\limits_{v\in\Omega_u^+}}
y^N g_{N-1}(\overline{\rho},y)+{}\right.\right.\\
{}+\left.\left.\sum\limits_{\overline{k}\in A_N}
\prod\limits_{v\in\Omega_u^+} z_v(0,\rho_v,k_v,k_v^\prime , k_v^{\prime\prime}) 
\right ) \right )=0\,.
\end{multline*}
    
\medskip

\noindent
\textbf{Утверждение 3.} \textit{Производная функции~$q_N (\overline{\rho},y)$ по 
$y\in (0,\,1]$ удовлетворяет следующим соотношениям}:
\begin{align}
\underset{y\rightarrow 0}{\mathrm{lim}}\fr{\partial q_N(\overline{p},y)}
{\partial  y} &= \fr{\sum\limits_{\overline{k}\in A_{N-1}} 
p_{\overline{k}}(\overline{\rho},1)}{\sum\limits_{\overline{k}\in 
A_N}p_{\overline{k}}(\overline{\rho},1)}\,;\label{e10aga}\\
\fr{\partial q_N(\overline{\rho},y)}{\partial y}\Big |_{y=1}&<1\,.\label{e11aga}
\end{align}

\medskip

\noindent
Д\,о\,к\,а\,з\,а\,т\,е\,л\,ь\,с\,т\,в\,о\,.\ Проведя преобразования 
функции~$q_N(\overline{\rho},y)$, получим:
\begin{multline*}
\underset{y\rightarrow 0}{\mathrm{lim}}\fr{q_N(\overline{\rho},y)}{y} = {}\\
\!\!{}=
\underset{y\rightarrow 0}{\mathrm{lim}}
\fr{\sum\limits_{m=0}^{N-1}\sum\limits_{\overline{k}\in A_m}
\!\!\left (\prod\limits_{v\in\Omega_u^+}\!\! 
z_v(0,\rho_v,k_v,k_v^\prime , k_v^{\prime\prime})\right )\!\!\Bigg /\!\! y^m}
{y\sum\limits_{m=0}^{N}\sum\limits_{\overline{k}\in A_m}
\!\!\left(\prod\limits_{v\in\Omega_u^+}\!\! z_v\left (0,\rho_v,k_v,k_v^\prime , 
k_v^{\prime\prime}\right )\right )\!\!\Bigg /\!\!y^m} = \!\!\!
\end{multline*}
\begin{multline*}
\!\!\!\!\!\!{}=\underset{y\rightarrow 0}{\mathrm{lim}}\,
\fr{\sum\limits_{m=0}^{N-1}\sum\limits_{\overline{k}\in A_m}
y^{N-1-m}\prod\limits_{v\in\Omega_u^+} z_v(0,\rho_v,k_v,k_v^\prime , 
k_v^{\prime\prime})}{\sum\limits_{m=0}^{N}\sum\limits_{\overline{k}
\in A_m} y^{N-m}
\prod\limits_{v\in\Omega_u^+} z_v(0,\rho_v,k_v,k_v^\prime , 
k_v^{\prime\prime})}={}\!\\
{}=\fr{\sum\limits_{\overline{k}\in A_{N-1}} p_{\overline{k}}(\overline{\rho},1)}{ 
\sum\limits_{\overline{k}\in A_{N}} p_{\overline{k}}(\overline{\rho},1)}\,.
\end{multline*}
Очевидно, $\underset{y\rightarrow 0}{\mathrm{lim}} \,[d_N (\overline{\rho},y) -
d_{N-1} (\overline{\rho},y)]=1$, так как $\underset{y\rightarrow 
0}{\mathrm{lim}}\,d_n (\overline{\rho},y)=n$, $n>0$.

Следовательно, учитывая~(\ref{e7aga}), получаем~(\ref{e10aga}). 
Справедливость~(\ref{e11aga}) непосредственно следует из~(\ref{e7aga}) и 
утверждения~1.

\medskip

\noindent
\textbf{Утверждение 4.} \textit{Пусть $y^*\in (0,\,1]$~--- решение 
уравнения}~(\ref{e4aga}). \textit{Тогда}
\begin{equation*}
\fr{\partial q_N(\overline{\rho},y)}{\partial y}\Big |_{y=y^*}<1\,.
%\label{e12aga}
\end{equation*}

\medskip

\noindent
Д\,о\,к\,а\,з\,а\,т\,е\,л\,ь\,с\,т\,в\,о\,.\ \ Доказательство следует из~(\ref{e7aga}), 
так как $q_N(\overline{\rho},y^*)/y^* =1$.
\medskip

\noindent
\textbf{Утверждение 5.} \textit{Уравнение}~(\ref{e4aga}) \textit{имеет решение $y^*\in 
(0,\,1)$ тогда и только тогда, когда} 
\begin{equation}
\fr{\sum\limits_{\overline{k}\in A_{N-1}} p_{\overline{k}}(\overline{\rho},1)}{ 
\sum\limits_{\overline{k}\in A_{N}} p_{\overline{k}}(\overline{\rho},1)} >1\,.
\label{e13aga}
\end{equation}
\textit{Если уравнение}~(\ref{e4aga}) \textit{имеет решение $y^*\in (0,\,1)$, то оно 
единственное положительное решение}.
\medskip

\noindent
Д\,о\,к\,а\,з\,а\,т\,е\,л\,ь\,с\,т\,в\,о\,.\ Пусть выполняется 
неравенство~(\ref{e13aga}). Тогда, как следует из утверждения~3, 
$\underset{y\rightarrow 0}{\mathrm{lim}} (\partial q_N(\overline{\rho},y)/\partial y) 
>1$. Кроме того, как следует из утверждения~2, 
$\underset{y\rightarrow 0}{\mathrm{lim}} q_N(\overline{\rho},y)=0$. Тогда, так 
как $q_N(\overline{\rho},y)$~--- непрерывно-дифференцируемая функция по 
$y\in (0,\,1]$, существует значение $y^\prime \in (0,\,1)$ такое, что 
$q_N(\overline{\rho},y)>y$ для всех $y\in (0,\,y^\prime]$ (следует из теоремы о 
конечном приращении~\cite{11aga}). В то же время, согласно утверждению~2, 
$q_N(\overline{\rho},y)<y$ в окрестности точки $y=1$ (рис.~\ref{f1aga},\,\textit{а}). 
Следовательно, кривая $x=q_N(\overline{\rho},y)$ пересекает прямую $x=y$ 
хотя бы в одной точке $y=y^*\in (0,\,1)$, т.\,е.\ уравнение~(\ref{e4aga}) имеет 
хотя бы одно решение $y^*\in (0,\,1)$.

\begin{figure*}
\vspace*{1pt}
\begin{center}
\vspace*{1pt}
\mbox{%
\epsfxsize=158mm
\epsfbox{aga-1.eps}
}
\end{center}
\vspace*{-9pt}
\Caption{Примеры кривых $x=q_N(\overline{\rho},y)$ и $x=y$ (\textit{а})~при существовании решения 
уравнения~(\ref{e5aga}) и (\textit{б})~при выполнении условий~(17)
\label{f1aga}}
\vspace*{6pt}
\end{figure*}

Пусть уравнение~(\ref{e4aga}) имеет решение $y^*\in (0,\,1)$ и 
\begin{equation}
\fr{\sum\limits_{\overline{k}\in A_{N-1}}p_{\overline{k}}(\overline{\rho},1)}{ 
\sum\limits_{\overline{k}\in A_{N}}p_{\overline{k}}(\overline{\rho},1)}\leq 
1\,.\label{e14aga}
\end{equation}
Тогда из условий утверждений~2 и~3 следует, что 
уравнение~(\ref{e4aga}) в интервале $(0,\,1)$ имеет более одного решения, что 
может быть только при существовании решения $y^\prime \in (0,\,1)$ такого, 
что в окрестности точки $y=y^\prime$ выполняются неравенства: 
$q_N(\overline{\rho},y)<y$ при $y<y^\prime$ и $q_N(\overline{\rho},y)>y$ при 
$y>y^\prime$, где $y$ принадлежит указанной окрест\-ности точки~$y^\prime$ 
(рис.~\ref{f1aga},\,\textit{б}). Тогда в точке $y=y^\prime$ производная функции 
$q_N(\overline{\rho},y)$ по $y$ больше~1, что противоречит утверждению~4. 
Следовательно, неравенство~(\ref{e13aga}) справедливо.


Пусть уравнение~(\ref{e4aga}) имеет более одного положительного 
решения. Рассуждая точно так же, как и выше (в случае выполнения 
условий~(\ref{e14aga})), получим противоречие утверждению~4. 
Следовательно, утверждение~5 справедливо.
\medskip

\noindent
\textbf{Следствие.} \textit{Неравенства}
\begin{gather*}
\fr{\mu_v c_v (1-B_v)}{\Lambda_v}>1\,,\quad \fr{1-B_v}{\Lambda_v \tau_v B_v}>1\,,\\ 
\fr{1-B_v}{\Lambda_v t_v}>1\,,\ v\in\Omega_u^+\,,
\end{gather*}
\textit{являются необходимым условием существования решения 
уравнения}~(\ref{e4aga}).

\medskip
\noindent
Д\,о\,к\,а\,з\,а\,т\,е\,л\,ь\,с\,т\,в\,о\,.\ Пусть $\overline{k}_v$~--- это 
набор~$\overline{k}$, у которого $k_v=0$. Преобразовав левую 
часть~(\ref{e13aga}), получим

\noindent
\begin{multline*}
\fr{\sum\limits_{\overline{k}\in A_{N-1}} p_{\overline{k}} (\overline{\rho},1)}
{ \sum\limits_{\overline{k}\in A_{N}} 
 p_{\overline{k}}(\overline{\rho},1)} 
={}
\\
{}=
\fr{\sum\limits_{\overline{k}\in A_{N-1}}\prod\limits_{v\in \Omega_u^+} 
z_v\left(0,\rho_v,k_v,k_v^\prime , k_v^{\prime\prime}\right)}
{\sum\limits_{\overline{k}\in A_{N}}
\prod\limits_{v\in \Omega_u^+} z_v\left (0,\rho_v,k_v,k_v^\prime , k_v^{\prime\prime}\right )} \leq{}
\\
{}\leq
\left ( 
\vphantom{\prod\limits_{v^\prime\in\Omega_u^+\backslash v}}
\fr{\mu_v c_v(1-B_v)}{\Lambda_v}\right. \times{}\\
{}\times \sum\limits_{k_v=0}^{N-1}\sum\limits_{\overline{k}_v\in A_{N-1-k_v}} z_v\left(0,\rho_v,k_v+1,k_v^\prime , 
k_v^{\prime\prime}\right )\times{}\\
{}\times \left.\prod\limits_{v^\prime\in\Omega_u^+\backslash v} z_v^\prime 
\left(0,\rho_v,k_v,k_v^\prime , k_v^{\prime\prime}\right) \right)
\Bigg /{}\\
\Bigg / \left ( 
\vphantom{\prod\limits_{v^\prime\in\Omega_u^+\backslash v}}
\sum\limits_{k_v=0}^{N-1} \sum\limits_{\overline{k}_v\in A_{N-1-k_v}} z_v 
\left (0,\rho_v,k_v+1,k_v^\prime , 
k_v^{\prime\prime}\right )\right. \times{}\\
{}\times \prod\limits_{v^\prime\in\Omega_u^+\backslash v} 
z_{v^\prime}\left(0,\rho_v,k_v,k^\prime , k_v^{\prime\prime}\right)+{}\\
{}+
\sum\limits_{\overline{k}_v\in A_N} z_v\left (0,\rho_v, 0,k_v^\prime , 
k_v^{\prime\prime}\right)\times{}\\
\left.{}\times \prod\limits_{v^\prime\in\Omega_u^+\backslash v}z_{v^\prime} 
\left(0,\rho_v,k_v,k_v^\prime , k_v^{\prime\prime}\right )\right )\,.
\end{multline*}
Как следует из правой части последнего неравенства, если 
$\mu_v c_v (1-B_v)/\Lambda_v \leq 1$, то она меньше~1. Поэтому, чтобы 
выполнилось условие~(\ref{e13aga}), необходимо выполнение первого 
неравенства в условии следствия для каждого $v\in\Omega_u^+$. Точно так же 
доказывается необходимость выполнения второго и третьего неравенств в 
условии следствия.

    Пусть $y[n]$, $n\geq 0$, последовательность, полученная по формуле 
$y[n+1]=q_N(\overline{\rho},y[n])$, $y[0]=1$.

\medskip

\noindent
\textbf{Утверждение 6.} \textit{Пусть $y^*\in (0,\,1)$~--- решение 
уравнения}~(8). \textit{Тогда последовательность $y[n]$, $n\geq 0$, сходится 
к решению~$y^*$}.

\medskip

\noindent
Д\,о\,к\,а\,з\,а\,т\,е\,л\,ь\,с\,т\,в\,о\,.\ Отметим, что $y[1]<y[0]$ (это следует из 
утверждения~2, так как $y[0]=1$). Пусть для некоторого $n>1$ выполняется 
условие $y[n]<y[n-1]$. Тогда, как следует из утверждения~2, указанное условие 
выполняется и для $n+1$, т.\,е.\ по индукции следует, что последовательность 
$y[n]$, $n\geq 0$, монотонно убывает. 

    Пусть для некоторого $n>0$ $y[n]>y^*$ (существование такого $n$ 
следует из равенства $y[0]=1$). Тогда, как следует из утверждения~2, 
$y[n+1]\;=$\linebreak $=\;q_N(\overline{\rho},y[n])>q_N(\overline{\rho},y^*) =y^*$, т.\,е.\ 
последовательность ограничена снизу величиной~$y^*$. Значит, существует 
$\underset{n\rightarrow \infty}{\mathrm{lim}}\,y[n]=y^0\geq y^*$. Так как 
$q_n(\overline{\rho},y)$~--- непрерывная по~$y$ функция, то можно написать 
$\underset{n\rightarrow 
\infty}{\mathrm{lim}}\,q_N(\overline{\rho},y[n])=q_N(\overline{\rho},y^0)=y^0$, 
т.\,е.\ $y^0$~--- решение уравнения~(\ref{e4aga}). Из единственности 
положительного решения уравнения~(\ref{e4aga}) получаем $y^0=y^*$.

    Пусть в узле используется схема полного разделения буферной памяти. 
Тогда для интенсив\-ностей~$\Lambda_v^*$, $v\in\Omega_u^+$, справедливы 
соотношения:
$$
\Lambda_v^* = \fr{\Lambda_v}{1-\pi_v}\,,
$$
где $v\in\Omega_u^+$.


Фиксируем произвольную линию сети~$v$. Пусть $\overline{k}_v = (k_v, 
k_v^\prime, k_v^{\prime\prime})$~--- состояние буферной памяти линии~$v$; 
$k_v$, $k_v^\prime$, $k_v^{\prime\prime}$ определены выше. Тогда с 
учетом введенных ранее предположений и обозначений для вероятности 
блокировки линии справедлива формула~\cite{4aga}:
\begin{equation}
\pi_v = \fr{1}{G_{N_v}}\sum\limits_{k_v=N_v} 
z_v(\pi_v,\rho_v,\overline{k}_v)\,,
\label{e15aga}
\end{equation}
где 
\begin{multline*}
z_v(\pi_v, \rho_v, \overline{k}_v)={}\\
{}=
\begin{cases}
\fr{\rho_v^{\prime * k_v^\prime}}{k_v^\prime !}\,
 \fr{\rho_v^{\prime\prime * k_v^{\prime\prime}}}{k_v^{\prime\prime}!}\,
 \fr{\rho_v^{*k_v}}{k_v !} & \mbox{при}\ k_v<c_v\,,\\
 \fr{\rho_v^{\prime *k_v^\prime}}{k_v^{\prime }! }
 \fr{\rho_v^{\prime\prime * k_v^{\prime\prime}}}{k_v^{\prime\prime}!}
\fr{\rho_v^{*k_v}}{c_v !c_v^{k_v-c_v}} & \mbox{при}\ k_v\geq c_v\,;
\end{cases}
\end{multline*}
\begin{align*}
G_{N_v} &= \sum\limits_{m=0}^{N_v} z_v (\pi_v ,\rho_v , \overline{k}_v)\,;\\ 
\rho_v^*&=\fr{\rho_v}{1-\pi_v}\,;
\end{align*}
$\rho_v$, $\rho_v^{\prime *}$, 
$\rho_v^{\prime\prime *}$, $v\in\Omega_u^+$ определены выше.
    
Пусть $y_v=1-\pi_v$, а $q_{N_v} (\rho_v, y_v)$~--- выражение в правой 
части~(\ref{e15aga}). Тогда из равенств~(\ref{e15aga}), взяв~$y_v$ в качестве 
неизвестной переменной, получим систему независимых уравнений:
\begin{equation}
y_v = q_{N_v}(\rho_v, y_v)\,, \quad v\in \Omega_u^+\,.
\label{e16aga}
\end{equation}
    
    Заметим, что для фиксированной $v$ и заданных параметров $\Lambda_v$, 
$\mu_v$, $\tau_v$, $t_v$, $N_v$, $v\in\Omega_u^+$, уравнение в~(\ref{e16aga}) 
является частным случаем уравнения~(\ref{e4aga}) и совпадает с последним, 
когда число исходящих линий из узла равно~1. Следовательно, для схемы 
полного разделения памяти также справедливы все приведенные выше 
утверждения~1--6 и следствие. Заметим, что неравенство~(\ref{e13aga}) в 
условии утверждения~5 при $B_v=0$ и $t_v=0$ преобразуется в неравенство 
$\Lambda_v / (\mu_v c_v) >1$, $v\in\Omega_u^+$. Последовательность 
$\overline{y}[n]$, $n\geq 0$, в утверждении~6 определяется по формуле:
    \begin{gather*}
    \overline{y}[n] =\{y_v[n],\ v\in\Omega_u^+\}\,,\
    y_v[n+1]=q_{N_v} (\rho_v,\,y_v[n])\,,\\
    y_v[0] =1\,,\quad n\geq 0\,,\quad v\in \Omega_u^+\,.
    \end{gather*}


\section{Алгоритм расчета} %4

    Для вычисления интенсивностей потоков и вероятностей блокировок в 
узле предлагается следующий алгоритм, описывающий изложенную выше 
итерационную процедуру. Введем обозначения:
$y_u^v$~--- вероятность блокировки узла для заявок, направляемых на 
линию~$v$,
\begin{gather*}
y_u^v  = 
\begin{cases}
y_u & \mbox{для}\ v\in\Omega_u^+\ \mbox{при}\\
&\mbox{полнодоступной схеме},\\
y_v & \mbox{при схеме полного распределения}\\
&\mbox{памяти};
\end{cases}
\\
q_N^v(\overline{\rho}_u^{-v}, y_u^v)  = 
\begin{cases}
q_N(\overline{\rho},y) & \mbox{для}\ v\in\Omega_u^+\ \mbox{при пол-}\\ 
&\mbox{нодоступной схеме},\\
q_{N_v}(\rho_v, y_v) & \mbox{при схеме полного}\\
&\mbox{распределения}\\ 
&\mbox{памяти},  v\in\Omega_u^+\,.
\end{cases}
\end{gather*}



Тогда уравнения~(\ref{e4aga}) и~(\ref{e16aga}) записываются в виде:
$$
y_u^v = q_N^v (\overline{\rho}^v_u, y^v_u)\,,\quad v\in \Omega_u^+\,.
$$
Для значений, вычисляемых на $k$-м шаге алгоритма, к 
обозначениям соответствующих параметров приписывается знак~$[k]$.
\pagebreak

\textbf{Шаг 0.} 
\begin{enumerate}[1.]
\item  \textit{Инициализация}. Вычисление начальных значений 
параметров~$\rho_v$, $v\in\Omega_u^+$: $\Lambda_v[0]=\Lambda_v$, 
$\rho_v[0]=\Lambda_v[0]/(\mu_v(1-B_v))$, $y_u^v[0]=1$.
\item \textit{Проверка условий существования решения}. Если для некоторой 
линии $v\in\Omega_u^+$ выполняется хотя бы одно неравенство $(c_v\mu_v(1-
B_v))/\Lambda_v[0]\;\leq$\linebreak $\leq\;1$, или $(1-B_v)/(\Lambda_v\tau_v B_v) \leq 1$, или 
$(t_v(1\;-$\linebreak $-\;B_v))/\Lambda_v[0] \leq 1$, то алгоритм заканчивает работу с 
результатом <<нагрузка не реализуема>>. Если в узле используется 
полнодоступная схема и $(c_v\mu_v(1-B_v))/\Lambda_v[0] > 1$, $(1-
B_v)/(\Lambda_v\tau_v B_v)\;>$\linebreak $>\;1$, $(t_v(1-B_v))/\Lambda_v[0] > 1$ для всех 
$v\in\Omega_u^+$, то проверяется условие~(\ref{e13aga}) для $\Lambda_v =
\Lambda_v[0]$, $v\in\Omega_u^+$, и при невыполнении этого условия алгоритм 
заканчивает работу с результатом <<нагрузка не реализуема>>.
\end{enumerate}

    При вычислении левой части неравенства~(\ref{e13aga}) рекомендуется 
использовать метод свертки Базена (см.~\cite{12aga}), позволяющий 
производить рекуррентные вычисления (подробно этот метод описан также 
в~[1, 3--6]).



\medskip
\textbf{Шаг~$k$} ($k > 0$):
\begin{enumerate}[1.]
\item \textit{Вычисление вероятностей блокировок}. Используя значения 
параметров $\overline{\rho}_u^v[k-1]$, $y_u^v[k-1]$, $v\in\Omega_u^+$, 
вычисление с помощью формул~(1)--(7) значений 
вероятностей $y[k]=1- \pi [k]$~--- в случае полнодоступной памяти, или 
$y_v[k]=1- \pi_v[k]$, $v\in\Omega_u^+$, с помощью формул~(\ref{e15aga})~--- в 
случае полного разделения памяти. При вычислении этих значений 
рекомендуется использовать метод свертки Базена.
    \item \textit{Проверка условий останова алгоритма}. Если хотя бы для 
одной $v\in\Omega_u^+$ для заданного значения точности   выполняется 
условие
$$
\fr{\vert \Lambda_v^*[k]-\Lambda_v^*[k-1]\vert}{\Lambda_v^*[k]}> \varepsilon\,,
$$
то вычисление параметров $\overline{\rho}_u^v[k]$, $v\in\Omega_u^+$, и 
переход к шагу~$k$, положив $k$ равным $k+1$, иначе алгоритм завершает 
работу. 
\end{enumerate}

    По завершении алгоритма либо выявится, что нагрузка в системе не 
реализуема, либо будут вычислены интенсивности потоков, поступающих на 
линии узла, и стационарные вероятности блокировок для заявок каждого типа. 
    
\section{Примеры расчета}

    Для проверки точности вычисления результатов с помощью 
предложенного выше алгоритма и приемлемости введенных предположений 
были проведены вычислительные эксперименты с использованием 
аналитических и имитационных моделей. Во всех рассмотренных ниже 
примерах потоки внешних заявок считаются пуассоновскими. 
В~табл.~1 приведены значения вероятности блокировок вновь 
поступивших извне заявок, полученные на основании точной формулы, 
приведенной в~\cite{4aga} для СМО типа $M\vert M\vert 1\vert 0$ с повторными 
заявками при экспоненциальном распределении интервала времени между 
повторными попытками (первая строка таблицы), алгоритма из подраздела~5 
настоящей статьи (вторая строка) и имитационной модели при постоянном 
интервале времени между повторными попытками, равном~10 (третья строка). 
Расчет табл.~1 проведен для узла с одной исходящей одноканальной 
линией при интенсивности первичного потока $\Lambda =1$ и емкости 
накопителя $N_v=1$. Таблицы~2 и~3 вычислены с помощью 
алгоритма из подраздела~5 и имитационной модели соответственно при одной 
исходящей линии с числом каналов~10.


    В табл.~\ref{t4aga} и~\ref{t5aga} приведены значения вероятности 
блокировки узла с тремя исходящими линиями канальной емкости~10 каждая 
при $\mu_v =0{,}2$, $v\in\Omega_u^+$,  вычисленные с помощью алгоритма из 
подраздела~5 и имитационной модели с интервалом повторной попытки, 
равным~10, соответственно. В табл.~\ref{t4aga} и~\ref{t5aga} знак <<--->> в 
ячейках означает, что предложенная нагрузка $\Lambda_v$, $v\in\Omega_u^+$, 
не реализуема.



В табл.~\ref{t6aga} отражены вероятности блокировки такого же узла с 
накопителем $N = 35$ при экспоненциальном распределении интервала 
времени между повторными попытками со средним значением~$\tau$. 


Результаты вычислительного эксперимента показывают, что с  увеличением 
длины интервала между повторными попытками  вероятность блокировки 
увеличивается и приближается к значению,\linebreak
вычисленному с помощью 
алгоритма из подраздела~5 (см.\ табл.~\ref{t4aga} и~\ref{t6aga}), т.\,е.\ при 
пуассоновском внешнем потоке заявок предположение, что суммарный 
входной поток заявок  является пуассоновским, вполне приемлемо для 
предварительного анализа характеристик узла (например, при  $\tau c_v\mu_v > 
10$). Как показывают табл.~1--3, вероятность блокировки 
узла существенно зависит от\linebreak 

\vspace*{6pt}
\noindent
%\begin{table*}\small %tabl1
{\small
{{\tablename~1}\ \ \small{Вероятности блокировок при одной исходящей одноканальной линии}}
%\label{t1aga}}
\vspace*{-3pt}

\begin{center}
{\tabcolsep=7.3pt
\begin{tabular}{|c|c|c|c|c|c|}
\hline
&\multicolumn{5}{c|}{$\mu$}\\
\cline{2-6}
\multicolumn{1}{|c|}{\raisebox{4pt}[0pt][0pt]{№}}&1{,}1&1{,}2&2&3&4\\
\hline
1&0,9091&0,8333&0,5000&0,3333&0,2500\\
2&0,9091&0,8333&0,5000&0,3333&0,2500\\
3&0,8867&0,8452&0,4944&0,3167&0,2396\\
\hline
\end{tabular}}
\end{center}
%\vspace*{-6pt}
%\end{table*}
}
%\bigskip
%\medskip
\addtocounter{table}{1}
\pagebreak

\end{multicols}

\renewcommand{\figurename}{\protect\bf Таблица}
%\renewcommand{\tablename}{\protect\bf Рис.}
\begin{figure*}
{\small
\begin{minipage}[t]{76mm}
%\begin{table*}\small %tabl2
\begin{center}
\Caption{Вероятности блокировок при одной исходящей многоканальной линии ($\varepsilon 
=0{,}0001$)
\label{t2aga}}
\vspace*{2ex}

\tabcolsep=6.5pt
\begin{tabular}{|c|c|c|c|c|c|}
\hline
&\multicolumn{5}{c|}{$\mu$}\\
\cline{2-6}
\multicolumn{1}{|c|}{\raisebox{4pt}[0pt][0pt]{$N$}}&0{,}11&0{,}12&0{,}2&0{,}3&0{,}4\\
\hline
10&0,4845&0,2935&0,0204&0,0017&0,0002\\
15&0,1181&0,0545&0,0006&0,0000&0,0000\\
20&0,0489&0,0167&0,0000&0,0000&0,0000\\
\hline
\end{tabular}
\end{center}
%\end{table*}
\end{minipage}
\hfill
\begin{minipage}[t]{76mm}
%\begin{table*}\small %tabl3
\begin{center}
\Caption{Вероятности блокировок при одной исходящей линии
\label{t3aga}}
\vspace*{2ex}

\tabcolsep=6.5pt
\begin{tabular}{|c|c|c|c|c|c|}
\hline
&\multicolumn{5}{c|}{$\mu_v$}\\
\cline{2-6}
\multicolumn{1}{|c|}{\raisebox{4pt}[0pt][0pt]{$N$}}&0{,}11&0{,}12&0{,}2&0{,}3&0{,}4\\
\hline
10&0,5247&0,3238&0,0219&0,0019&0,0001\\
15&0,1726&0,0912&0,0004&0,0001&0,0000\\
20&0,1180&0,0371&0,0000&0,0000&0,0000\\
\hline
\end{tabular}
\end{center}
%\end{table*}
\end{minipage}
}
\vspace*{6pt}
\end{figure*}

\renewcommand{\figurename}{\protect\bf Рис.}
\renewcommand{\tablename}{\protect\bf Таблица}
\addtocounter{table}{2}

\begin{table}\small %tabl4
\begin{center}
\parbox{400pt}{\Caption{Вероятности блокировок при трех исходящих линиях, вычисленные алгоритмом из 
подраздела~5 ($\varepsilon =0{,}0001$)
\label{t4aga}}
}

\vspace*{2ex}

\tabcolsep=8pt
\begin{tabular}{|c|c|c|c|c|c|c|c|c|c|}
\hline
&\multicolumn{9}{c|}{$\Lambda_v$}\\
\cline{2-10}
\multicolumn{1}{|c|}{\raisebox{4pt}[0pt][0pt]{$N$}}&1&1{,}1&1{,}2&1{,}3&1{,}4&1{,}5&1{,}6&1{,}7&1{,}8\\
\hline
20&0,0677&0,1423&0,2975&0,7653&---&---&---&---&---\\
25&0,0065&0,0173&0,0394&0,0827&0.1690&0.3827&---&---&---\\
30&0,0005&0,0019&0,0059&0,0155&0.0361&0.0790&0.1792&0,7259&---\\
35&0,0000&0,0002&0,0008&0,0030&0,0089&0,0234&0,0574&0,1505&---\\
40&0,0000&0,0000&0,0001&0,0005&0,0022&0,0075&0,0220&0,0617&0,2449\\
\hline
\end{tabular}
\end{center}
%\end{table}
\vspace*{6pt}
%\begin{table}\small %tabl5
\begin{center}
\parbox{400pt}{\Caption{Вероятности блокировок при трех исходящих линиях, вычисленные с помощью 
имитационной модели
\label{t5aga}}
}

\vspace*{2ex}

\tabcolsep=8pt
\begin{tabular}{|c|c|c|c|c|c|c|c|c|c|}
\hline
&\multicolumn{9}{c|}{$\Lambda_v$}\\
\cline{2-10}
\multicolumn{1}{|c|}{\raisebox{4pt}[0pt][0pt]{$N$}}&1&1{,}1&1{,}2&1{,}3&1{,}4&1{,}5&1{,}6&1{,}7&1{,}8\\
\hline
20&0,0786&0,1695&0,3549&0,7056&---&---&---&---&---\\
25&0,0069&0,0190&0,0441&0,0998&0,2266&0,4583&---&---&---\\
30&0,0007&0,0024&0,0075&0,0184&0,0462&0,1025&0,2380&0,6931&---\\
35&0,0000&0,0003&0,0007&0,0040&0,0129&0,0307&0,0890&0,2981&---\\
40&0,0000&0,0000&0,0000&0,0011&0,0041&0,0095&0,0346&0,0790&0,3179\\
\hline
\end{tabular}
\end{center}
%\end{table}
\vspace*{6pt}
%\begin{table}\small %tabl6
\begin{center}
\parbox{356pt}{\Caption{Зависимость вероятности блокировки при трех исходящих линиях, вы\-чис\-лен\-ные с 
помощью имитационной модели со случайным интервалом между повторными попытками
\label{t6aga}}
}

\vspace*{2ex}

\tabcolsep=8pt
\begin{tabular}{|c|c|c|c|c|c|c|c|c|}
\hline
&\multicolumn{8}{c|}{$\Lambda_v$}\\
\cline{2-9}
\multicolumn{1}{|c|}{\raisebox{6pt}[0pt][0pt]{$\tau$}}&1&1{,}1&1{,}2&1{,}3&1{,}4&1{,}5&1{,}6&1{,}7\\
\hline
\hphantom{9}1&0.0001&0,0001&0,0017&0,0063&0,0210&0,0733&0,1996&0,4222\\
\hphantom{9}5&0.0000&0,0002&0,0016&0,0036&0,0446&0,0159&0,1360&0,3273\\
10&0.0000&0,0002&0,0011&0,0036&0,0101&0,0430&0,0818&0,2774\\
20&0.0000&0,0003&0,0007&0,0029&0,0089&0,0257&0,0863&0,2045\\
     \hline
\end{tabular}
\end{center}
\end{table}


\begin{multicols}{2}


\noindent
числа каналов в линии при равной суммарной 
производительности. Кроме того, как видно из табл.~\ref{t5aga} и~\ref{t6aga}, 
вероятность блокировки в большей степени зависит от среднего значения 
длины интервала между повторными попытками передачи, чем от закона 
распределения длины интервала. Таким образом, предложенный в работе 
алгоритм позволяет вы\-чис\-лить с достаточной точностью вероятность 
блокировки узла, интенсивности повторных передач и предельную величину 
реализуемой нагрузки. Отметим, что полученные в данной статье результаты 
могут быть использованы для расчета нагрузок в телекоммуникационной сети с 
повторами заявок в предыдущем узле или из источника. 


{\small\frenchspacing
{%\baselineskip=10.8pt
\addcontentsline{toc}{section}{Литература}
\begin{thebibliography}{99}    
\bibitem{1aga}
\Au{Kamoun~F., Kleinrock~L.}
Analysis of shared finite storage in a computer networks node environment under 
general traffic conditions~// IEEE Trans. on Commun., 1980. Vol.~28. No.\,7. 
P.~992--1003.

\bibitem{6aga} %2
\Au{Агаларов~Я.\,М., Шоргин~С.\,Я.}
Рекуррентный метод вычисления параметров сетей связи~// Техника средств 
связи, 1986. Сер. <<Системы связи>>. Вып.~6. С.~42--46.

\bibitem{3aga}
\Au{Башарин Г.\,П., Бочаров~П.\,П., Коган~Я.\,А.}
Анализ очередей в вычислительных сетях.~--- М.: Наука, 1989. 

\bibitem{4aga}
\Au{Бочаров~П.\,П., Печинкин~А.\,В.}
Теория массового обслуживания.~--- М.: Изд-во РУДН, 1995. 

\bibitem{5aga}
\Au{Вишневский~В.\,М.} 
Теоретические основы проектирования компьютерных сетей.~--- М.: 
Техносфера, 2003. 

\bibitem{2aga} %6
\Au{Башарин Г.\,П.}
Лекции по математической теории телетрафика.~--- М.: Изд-во РУДН, 2007. 

\bibitem{7aga}
\Au{Таранцев~А.\,А.}
Инженерные методы теории массового обслуживания.~--- М.: Наука, 2007.

\bibitem{9aga} %8
\Au{D'Apice~C., De~Simone~T., Manzo~R., Rizelian~G.}
$M\vert G\vert 1\vert r$ retrial queueing system with priority service of primary 
customers and a customers-searching server~// Distributed Computer and 
Communication Networks. Stochastic Modelling and Optimization.~--- М.: 
Техносфера, 2003. P.~106--117.

\bibitem{8aga} %9
\Au{Klimenok~V.\,I., Kim~C.\,S.}
$BM\!AP$/$PH$/1 retrial system operating in random environment~// Proceedings of 
the 5th St.-Petersburg Workshop on Simulation, St.-Petersburg, June~26\,--\,July~2, 
2005.~--- St.-Petersburg: NII Chemistry St.-Petersburg University Publs., 
2005. P.~367--372.   

\bibitem{10aga}
\Au{Krishnamoorthy~A., Babu~S.}
$M\!AP\vert (PH,PH)/c$ retrial queue with selegeneration of priorities 
and non-preemptive service~// Proceedings of the 14th International Conference on 
Analytical and Stochastic Modeling Techniques and Applications, June~4--6, 
2007. Prague, Czech Republic.~--- Sbr.-Dudweiler: Digitaldruck Pirrot GmbH, 
2007. P.~70--74.

\bibitem{11aga}
\Au{Корн~Г., Корн~Т.}
Справочник по математике.~--- М.: Наука, 1974.

\label{end\stat}


\bibitem{12aga}
\Au{Buzen~J.\,P.}
Computational algorithm for closed queuing networks with exponential servers~// 
Communications ACM, 1973. Vol.~16. No.\,9. P.~527--531.
 \end{thebibliography}
}
}
\end{multicols}
 
 
   %1
\def\stat{bosov+stef}

\def\tit{УПРАВЛЕНИЕ ВЫХОДОМ СТОХАСТИЧЕСКОЙ ДИФФЕРЕНЦИАЛЬНОЙ СИСТЕМЫ 
ПО~КВАДРАТИЧНОМУ КРИТЕРИЮ. I.~ОПТИМАЛЬНОЕ РЕШЕНИЕ МЕТОДОМ 
ДИНАМИЧЕСКОГО ПРОГРАММИРОВАНИЯ$^*$}

\def\titkol{Управление выходом стохастической дифференциальной системы 
по~квадратичному критерию. I}
%.~Оптимальное решение методом 
%динамического программирования}

\def\aut{А.\,В.~Босов$^1$, А.\,И.~Стефанович$^2$}

\def\autkol{А.\,В.~Босов, А.\,И.~Стефанович}

\titel{\tit}{\aut}{\autkol}{\titkol}

\index{Босов А.\,В.}
\index{Стефанович А.\,И.}
\index{Bosov A.\,V.}
\index{Stefanovich A.\,I.}




{\renewcommand{\thefootnote}{\fnsymbol{footnote}} \footnotetext[1]
{Работа выполнена при частичной поддержке РФФИ (проект 16-07-00677).}}


\renewcommand{\thefootnote}{\arabic{footnote}}
\footnotetext[1]{Институт проблем информатики Федерального исследовательского центра <<Информатика 
и~управление>> Российской академии наук, \mbox{AVBosov@ipiran.ru}}
\footnotetext[2]{Институт проблем информатики Федерального исследовательского центра <<Информатика 
и~управление>> Российской академии наук, \mbox{AStefanovich@frccsc.ru}}

%\vspace*{8pt}



  
  \Abst{Решается задача оптимального управления для диффузионного процесса 
Ито и~линейного управ\-ля\-емо\-го выхода. Рассматриваемая постановка близка 
к~классической ли\-ней\-но-квад\-ра\-тич\-ной гауссовской задаче управления 
(linear-quadratic Gaussian (LQG) control). Отличия состоят в~том, что состояние описывается нелинейным 
дифференциальным уравнение Ито $dy_t\hm= A_t(y_t) \,dt\hm+ \Sigma_t(y_t)\,dv_t$ 
и~не зависит от управ\-ле\-ния~$u_t$, оптимизации подлежит управ\-ля\-емый 
линейный выход $dz_t\hm= a_t y_t\,dt\hm+ b_t z_t \,dt\hm+ c_t u_t \,dt\hm+ \sigma_t\, 
dw_t$. Дополнительные обобщения внесены в~квад\-ра\-тич\-ный критерий качества 
с~целью воз\-мож\-ности постановки таких задач, как отслеживание выходом 
состояния или управ\-ле\-ни\-ем~--- линейной комбинации состояния и~выхода. Для 
решения используется метод динамического программирования. Функцию 
Беллмана позволяет найти предположение о~ее структуре вида $V_t(y,z)\hm= 
\alpha_t z^2\hm+ \beta_t(y)z \hm+\gamma_t(y)$. Решение дают три 
дифференциальных уравнения для коэффициентов~$\alpha_t$, $\beta_t(y)$ 
и~$\gamma_t(y)$. Эти уравнения со\-став\-ля\-ют оптимальное решение 
рас\-смат\-ри\-ва\-емой задачи.}
  
  \KW{стохастическое дифференциальное уравнение; оптимальное управ\-ле\-ние; 
динамическое программирование; функция Беллмана; уравнение Риккати; 
линейные уравнения параболического типа}

\DOI{10.14357/19922264180314}
  
%\vspace*{4pt}


\vskip 10pt plus 9pt minus 6pt

\thispagestyle{headings}

\begin{multicols}{2}

\label{st\stat}

\section{Введение}

     Ключевые результаты в~области оптимизации стохастических 
динамических систем, со\-став\-ля\-ющие классическую теорию управления, 
получены более~40~лет назад (такова работа~[1] в~отношении задачи 
управ\-ле\-ния ли\-ней\-но-гаус\-сов\-ски\-ми стохастическими сис\-те\-ма\-ми по 
квад\-ра\-тич\-но\-му критерию). К~классической тео\-рии следует относить 
линейные модели стохастических сис\-тем и~квадратичный критерий качества. 
Это исходный базис, на котором основано множество успешно 
исследованных и~решенных задач стохастического управ\-ле\-ния 
и~оптимизации. 

Дальнейшее развитие~--- это новые модели и~критерии, но 
прежде всего это новые методы: от тео\-рии линейных регуляторов, метода 
динамического программирования и~принципа максимума к~адаптивному 
и~минимаксному подходу, импульсному управ\-ле\-нию и~т.\,д. Множество 
инноваций как в~час\-ти моделей, так и~в~час\-ти математического аппарата, 
имевших мес\-то в~по\-сле\-ду\-ющие годы, существенно обогатили тео\-рию 
управ\-ле\-ния. Но и~до настоящего времени линейные модели и~квадратичный 
критерий, несмотря на всю справедливую критику в~отношении их 
аде\-кват\-ности и~гиб\-кости, сохраняют исследовательский интерес и~находят 
современные области приложения.
     
     Не претендуя на сколь\-ко-ни\-будь полное обосно\-ва\-ние последнего 
тезиса, приведем несколько примеров, показавшихся наиболее ин\-те\-рес\-ными. 

Так, в~[2] решается ли\-ней\-но-квад\-ра\-тич\-ная за\-да\-ча в~игровой 
постановке с~запаздыванием. В~близ\-кой по модели работе~[3] задача 
управ\-ле\-ния ставится в~терминах $H_\infty$-ро\-баст\-ности. Точнее \mbox{называть} 
эту тематику $H_2/H_\infty$-управ\-ле\-ни\-ем, и~работ по этой теме очень 
много. Аккуратности ради следует уточнить, что под линейными 
понимаются модели с~мультипликативными по состоянию воз\-му\-ще\-ниями. 

Совсем другой класс моделей, особо популярных в~по\-след\-ние годы, 
составляют скачкообразные процессы. Например, линейные уравнения 
в~сочетании с~пуассоновскими скачками в~[4] используются в~моделях, 
описывающих различные показатели функционирования сетевых протоколов 
передачи данных транспортного уровня. Телекоммуникации представляют 
в~последние годы самый популярный прикладной материал для 
исследований, работ по этой проб\-ле\-ма\-ти\-ке множество, математические 
техники привлекаются самые разные и~самые современные, но и~линейным 
моделям место находится. Еще один любопытный пример исследования 
скачкообразного процесса и~оптимизации на основе квад\-ра\-тич\-но\-го критерия 
можно найти в~[5] применительно к~задаче инвестирования на финансовом 
рынке. Наконец, упомянем еще работу~[6], подводящую итог исследований 
в~отношении классической детерминированной  
ли\-ней\-но-квад\-ра\-тич\-ной задачи с~использованием техники матричных 
неравенств.
     
     В данной работе также эксплуатируются привлекательные свойства 
линейных моделей и~квад\-ра\-тич\-но\-го критерия, причем в~стохастической 
постановке. На\-прав\-ле\-ни\-ем для обобщения \mbox{выбрана} модель динамики 
сис\-те\-мы: основные усилия на\-прав\-ле\-ны на то, чтобы сделать ее нелинейной. 
Кроме того, пред\-став\-лен\-ная постановка может рас\-смат\-ри\-вать\-ся и~как 
обобщение ранее решенной задачи в~дискретном времени~[7, 8] на время 
непрерывное. В~упомянутых работах помимо собственно модельной 
постановки важна еще и~привлекаемая прикладная об\-ласть~--- 
функционирование сложных программных сис\-тем. Результатов, 
ориентированных непосредственно на такие приложения, к~настоящему 
времени пренебрежимо мало, поэтому~[7, 8]~--- это еще и~прикладное 
обоснование рас\-смат\-ри\-ва\-емой далее задачи.
     
     Оптимизируемая динамическая сис\-те\-ма описывается двумя 
уравнениями. Состояние задается нелинейным стохастическим 
дифференциальным уравнением Ито, не содержащим управ\-ля\-емой 
переменной. Возмущение здесь описывается стандартным винеровским 
процессом, накладываются простые условия существования 
и~един\-ст\-вен\-ности решения. Поскольку состояние не управ\-ля\-ет\-ся, то уместно 
его интерпретировать как слож\-ное внешнее возмущение. Вторая 
переменная~--- управ\-ля\-емый выход~--- задается линейным стохастическим 
дифференциальным уравнением. Цель оптимизации выхода формируется 
квадратичным критерием общего вида. Формальная постановка задачи 
приведена в~сле\-ду\-ющем разделе.
     
     Для решения задачи используется метод динамического 
программирования, решается уравнение Беллмана~[9]. Соответственно, 
в~результате получаются аналитические выражения и~для оптимального 
управ\-ле\-ния, и~для значения функционала качества. Технически 
традиционный, стандартный подход к~задаче обременен, пожалуй, 
единственной проблемой~--- поиском верного пред\-став\-ле\-ния структуры 
функции Беллмана. Справиться с~этой проблемой в~большей степени удается 
за счет результата, полученного при решении дискретного по времени 
аналога рассматриваемой постановки~\cite{8-bos}. Конечные соотношения 
для оптимального решения, как и~во всех подобных задачах, включая 
классическую ли\-ней\-но-квад\-ра\-тич\-ную, содержат решения 
определенных дифференциальных уравнений (обыкновенных и~в~частных 
производных). Вывод этих уравнений и~со\-став\-ля\-ет содержание первой час\-ти 
данной работы. Во второй части будет обсуждаться их приближенное 
чис\-лен\-ное решение и~компьютерные эксперименты.
     
     Кратко обозначим основные положения, при\-вле\-ка\-емые далее 
к~решению задачи, следуя в~основном обозначениям 
и~терминологии~\cite{9-bos}, а~именно: будем рассматривать задачу 
оптимального управления в~стохастической динамической сис\-те\-ме по полной 
информации, применяя метод динамического программирования. В~качестве 
целевого функционала, опре\-де\-ля\-юще\-го качество управ\-ле\-ния $U_0^T\hm= \{ 
u_t,\ 0\leq t\leq T\}$, выступает
     \begin{equation}
     J\left(U_0^T\right)={\sf E}\left\{ \int\limits_0^T L_t \left(x_t, u_t\right)\,dt+ 
l\left(x_T\right)\right\}\,.
     \label{e1-bos}
     \end{equation}
Здесь ${\sf E}\{\cdot\}$~--- оператор математического ожидания; $x_t$~--- 
случайный процесс, описываемый стохастическим дифференциальным 
уравнением Ито
     \begin{equation}
     dx_t=m_t\left( x_t, u_t\right) dt+ \sigma_t\left( x_t\right)dW_t\,,\enskip 
x_0=X\,,
     \label{e2-bos}
     \end{equation}
где $W_t$~--- стандартный винеровский процесс подходящей раз\-мер\-ности; 
$X$~--- случайный вектор.

     $U_0^T$ будем выбирать из класса допустимых неупреждающих (по 
отношению к~$W_t$) управлений~\cite{9-bos}. Соответственно, 
относительно функций сноса и~диффузии~$m_t$ и~$\sigma_t$  
в~(\ref{e2-bos}) будем предполагать выполненными ка\-кие-ли\-бо условия 
существования сильного решения для заданного до\-пус\-ти\-мо\-го управ\-ле\-ния. 
Например, для управ\-ле\-ния с~обратной связью $u_t\hm= u_t(x_t)$ будем 
считать, что $m_t(x,u_t(x))$ и~$\sigma_t(x)$ удовлетворяют условию 
линейного рос\-та и~локальному условию Липшица по~$x$ равномерно 
по~$t$ (т.\,е.\ условиям Ито).
     
     Для поиска оптимального управления, минимизирующего $J(U_0^T)$, 
рас\-смат\-ри\-ва\-ет\-ся функция Беллмана
     \begin{equation}
     V_t(x)=\left.\mathop{\mathrm{inf}}\limits_{U_t^T} {\sf E} \left\{ \int\limits_t^T 
L_t \left( x_t, u_t\right)\,dt+l\left( x_T\right) \right\vert \mathcal{F}_t^x\right\}\,,
     \label{e3-bos}
     \end{equation}
где $\mathcal{F}_t^x$~--- $\sigma$-ал\-геб\-ра, по\-рож\-ден\-ная~$x_\tau$, 
$0\hm\leq \tau\hm\leq t$, ${\sf E}\{\cdot\vert \mathcal{F}\}$~--- оператор условного 
математического ожидания относительно~$\mathcal{F}$. Соответственно, 
в~качестве достаточного условия оп\-ти\-маль\-ности воспользуемся уравнением 
динамического программирования
\begin{multline}
\fr{\partial V_t(x)}{\partial t} +\fr{1}{2}\sum\limits^n_{i,j=1} \sigma^2_{t_{ij}}
\fr{\partial^2 V_t(x)}{\partial x_i \partial x_j}+{}\\
{}+\min\limits_u\left[  
\sum\limits^n_{i=1} m_{t_i} \fr{\partial V_t(x)}{\partial x_i} + L_t(x,u)\right] 
=0\,,\\
V_T(x)=l(x)\,,
\label{e4-bos}
\end{multline}
где $m_{t_i}$~--- $i$-й элемент век\-тор-функ\-ции~$m_t(x,u)$; 
$\sigma^2_{t_{ij}} \hm= \sum\nolimits^m_{k=1} 
\sigma_{t_{ik}}\sigma_{t_{ki}}$, $\sigma_{t_{ij}}$~--- $i$-й по строке, $j$-й 
по столб\-цу элемент мат\-рич\-ной функции~$\sigma_t(x)$; $n$ и~$m$~--- 
размерности~$x_t$ и~$W_t$ соответственно.

     Традиционно в~рамках применения метода динамического 
программирования будем предполагать, что функции~$L_t$, $l$, $m_t$ 
и~$\sigma_t$ обеспечивают существование хотя бы одного решения 
уравнения~(\ref{e4-bos}), а~следовательно, и~оптимального 
управления~$u_t^*$, $0\hm\leq t\hm\leq T$, до\-став\-ля\-юще\-го минимум 
целевому функционалу~(\ref{e1-bos}). Задача оптимизации далее получается 
путем указания конкретных выражений для~$L_t$, $l$, $m_t$ и~$\sigma_t$.

\section{Постановка задачи управления выходом}

     Рассматриваемые далее случайные функции будут предполагаться 
скалярными. Такое упрощение позволит разгрузить выкладки и~итоговые 
выражения от не самых существенных деталей.
     
     Рассмотрим стохастическую дифференциальную сис\-те\-му, со\-сто\-яние 
которой представляет диффузи\-он\-ный процесс~$y_t$, описываемый 
нелинейным стохастическим дифференциальным уравнением Ито
     \begin{equation}
     dy_t=A_t\left( y_t\right) dt +\Sigma_t \left( y_t\right) dv_t\,,\enskip 
y_0=Y\,,
     \label{e5-bos}
     \end{equation}
где $v_t$~--- стандартный (одномерный) винеровский процесс; $Y$~--- 
случайная величина с~конечным вторым моментом; функции~$A_t$ 
и~$\Sigma_t$ удовлетворяют условиям Ито:
\begin{equation*}
\left\vert A_t(y)\right\vert +\left\vert \Sigma_t(y)\right\vert \leq C(1+\vert y\vert )\ 
\mbox{для\ всех } 0\leq t\leq T\,;
\end{equation*}

\vspace*{-12pt}

\noindent
\begin{multline*}
\hspace*{-2.10051pt}\left\vert A_t\left(y_1\right) -A_t \left( y_2\right) \right\vert +\left\vert 
\Sigma_t\left( y_1\right) -\Sigma_t \left(y_2\right)\right\vert \leq
C\left\vert y_1-y_2\right\vert\\
 \mbox{для\ всех\ } 0\leq t\leq T\ \mbox{и } 
y_1,y_2\in \mathbb{R}^1\,,
\end{multline*}
обеспечивающим существование единственного сильного (потраекторного) 
решения уравнения.
     
     Будем считать, что~$y_t$ описывает состояние некоторой 
динамической системы. Соответственно, поведение этой сис\-те\-мы опишем 
выходом, линейно связанным с~со\-сто\-янием:
     \begin{equation}
     dz_t=a_t y_t \,dt+ b_t z_t \,dt+ c_t u_t \,dt+\sigma_t \,dw_t\,,\enskip
     z_0=Z\,.
     \label{e6-bos}
     \end{equation}
Здесь $w_t$~--- не зависящий от~$v_t$, $Y$ и~$Z$ стандартный (одномерный) 
винеровский процесс; $Z$~--- случайная величина с~конечным вторым 
моментом; $u_t$~--- допустимое неупреждающее управ\-ле\-ние, качество 
которого определяется целевым функционалом следующего вида:
\begin{multline}
\!\hspace*{-3.98538pt}J\left( U_0^T\right) ={\sf E}\left\{ \int\limits_0^T \!\left( S_t\left( s_ty_t-g_t z_t -h_t 
u_t\right)^2 +G_t z_t^2+{}\right.\right.\\
\left.\left.{}+ H_t u_t^2
\vphantom{S_t\left( s_ty_t-g_t z_t -h_t 
u_t\right)^2}
\right) dt+S_T\left( s_T y_T -g_T 
z_T\right)^2+G_T z_T^2
\vphantom{\int\limits_0^T}\right\}\,,
\label{e7-bos}
\end{multline}
где $S_t$, $G_t$ и~$H_t$~--- неотрицательные функции\linebreak
$0\hm\leq t\hm\leq T$. 
Такой критерий отражает физический смысл задачи распределения ресурсов 
со\-глас\-но аналогичной~(\ref{e5-bos})--(\ref{e7-bos}) задаче для дис\-крет\-но\-го 
времени, рас\-смот\-рен\-ной в~\cite{7-bos}. В~част\-ности,  
функци\-онал~(\ref{e7-bos}) поз\-во\-ля\-ет ставить задачи отслеживания
 выходом 
со\-сто\-яния сис\-те\-мы, используя сла\-га\-емое $(y_t\hm- z_t)^2$, или 
управлением~--- линейной комбинации со\-сто\-яния и~выхода, сла\-га\-емое типа\linebreak 
$(y_t\hm+ z_t\hm- u_t)^2$. Поскольку задача формулируется 
в~предположении наличия пол\-ной информации о~со\-сто\-янии~$y_t$ 
и~выходе~$z_t$ (соответствующую $\sigma$-ал\-геб\-ру 
обозначим~$\mathcal{F}_t^{y,z}$), то допустимое управ\-ле\-ние ищется 
в~классе~$\mathcal{F}_t^{y,z}$-из\-ме\-ри\-мых неупреждающих функций 
(и,~как будет показано далее, оказывается управ\-ле\-ни\-ем с~обратной связью).

     Функции~$a_t$, $b_t$, $c_t$ и~$\sigma_t$ будем предполагать 
ограниченными: $\vert a_t\vert \hm+ \vert b_t\vert \hm+\vert c_t\vert \hm+ \vert 
\sigma_t \vert \hm\leq C$ для всех $0\hm\leq t\hm\leq T$, процесс  
управления~--- допустимым не\-упреж\-да\-ющим~\cite{9-bos}, обеспечивая, 
таким образом, существование сильного решения урав\-не\-ния~(\ref{e6-bos}) 
для любого допустимого управ\-ления.
     
     Задачу составляет поиск~$u_t^*$~--- допустимого управ\-ле\-ния, 
доставляющего минимум квад\-ра\-тич\-но\-му функционалу~$J(U_0^T)$.
      
     Поставленная задача очевидным образом формулируется в~терминах 
введенных выше в~(\ref{e1-bos})--(\ref{e3-bos}) обозначений, а~именно: 
     требуется обозначить
     \begin{gather*}
      x_t=\begin{pmatrix}
     y_t\\ z_t\end{pmatrix};\quad  m_t(x_t, u_t)=\begin{pmatrix}
     A_t(y_t)\\ a_t y_t +b_t z_t +c_t u_t\end{pmatrix};\\
     \sigma_t(x_t)= \begin{pmatrix}
     \Sigma_t(y_t)& 0\\
     0& \sigma_t\end{pmatrix};\quad W_t=\begin{pmatrix}
     v_t \\ w_t\end{pmatrix}
     %     \label{e8-bos}
     \end{gather*}
для записи уравнения со\-сто\-яния типа~(\ref{e2-bos}) и
\begin{align*}
L_t(x,u)&= L_t(y,z,u) ={}\\
&\hspace*{3mm}{}=S_t\left( s_t y-g_t z -h_t u\right)^2 +G_t z^2 +H_t  u^2\,;\\
l(x)&= l(y,z) =S_T \left( S_T y-g_T z\right)^2 +G_T z^2
%\label{e9-bos}
\end{align*}
для записи целевого функционала в~виде~(\ref{e1-bos}).

     Функция Беллмана~(\ref{e3-bos}) принимает вид 
     $V_t(x)\hm= V_t(y,z)$. Для записи со\-от\-вет\-ст\-ву\-юще\-го~(\ref{e4-bos}) 
уравнения Беллмана для~$V_t(y,z)$ заметим, что
     $$
     \left( \sigma^2_{t_{ij}}\right)_{i,j=1,2}= \begin{pmatrix}
     \Sigma_t^2(y) & 0\\
     0 & \sigma_t^2\end{pmatrix}\,.
     $$
     
     С~учетом перечисленных обозначений урав\-не\-ние динамического 
программирования~(\ref{e4-bos}) принимает вид:
     \begin{multline}
     \fr{\partial V_t(y,z)}{\partial t} +\fr{1}{2}\left( \Sigma_t^2(y) \fr{\partial^2 
V_t(y,z)} {\partial y^2}+\sigma_t^2\fr{\partial^2 V_t(y,z)} {\partial 
z^2}\right)+{}\\
    {}+\min\limits_u\! \left[ A_t(y) \fr{\partial V_t(y,z)}{\partial y}+\left( a_t 
y+b_t z+c_t u\right) \fr{\partial V_t(y,z)}{\partial z} +{}\right.\hspace*{-3pt}\\
\left.{}+ S_t\left( s_t y-g_t z-h_t 
u\right)^2+G_t z^2+H_t u^2
     \vphantom{\fr{\partial V_t(y,z)}{\partial y}}\right] =0\,,\\
     V_T(y,z)=S_T\left( s_T y-g_T z\right)^2+G_T z^2\,.
     \label{e10-bos}
     \end{multline}
     Это и~есть то самое уравнение, которое требуется решить: 
существование решения данного урав\-не\-ния суть достаточное условие 
оптимальности; оптимальное управ\-ле\-ние при этом~--- точ\-ка минимума 
со\-от\-вет\-ст\-ву\-юще\-го сла\-га\-емого.
     
\section{Динамическое программирование и~оптимальное 
управление}

     В рассматриваемой постановке линейность\linebreak выхода и~квадратичность 
критерия дают те же преимущества, что и~в~классической  
ли\-ней\-но-квад\-ра\-тич\-ной задаче управ\-ле\-ния~\cite{1-bos}, а~именно: 
позволяют сразу определить вид оптимального управ\-ле\-ния и~фактические 
условия его существования. Действительно, со\-хра\-няя в~(\ref{e10-bos}) под 
знаком $\min\nolimits_u$ только члены, зависящие от~$u$, получаем
     \begin{multline*}
     \fr{\partial V_t(y,z)}{\partial t} +\fr{1}{2}\left( \Sigma_t^2(y) \fr{\partial^2 
V_t(y,z)} {\partial y^2}+\sigma_t^2\fr{\partial^2 V_t(y,z)} {\partial 
z^2}\right)+{}\\
     {}+A_t(y)\fr{\partial V_t(y,z)}{\partial y}+\left( a_t y+b_t z\right) 
\fr{\partial V_t(y,z)}{\partial z}+{}\\
{}+S_t\left( s_t y-g_t z\right)^2 +G_t z^2+{}
\end{multline*}

\noindent
\begin{multline*}
     {}+\min\limits_u \left[ \left( c_t \fr{\partial V_t(y,z)}{\partial z}-2S_t \left( 
s_t y-g_t z\right) h_t\right)u +{}\right.\\
\left.{}+\left( S_t h_t^2+H_t\right) u^2
\vphantom{\fr{\partial V_t(y,z)}{\partial z}}
\right]=0\,,
     %\label{e11-bos}
     \end{multline*}
откуда в~предположении $S_t h_t^2\hm+ H_t\hm>0$ следует, что существует 
оптимальное управ\-ле\-ние, которое определяется равенством
\begin{multline}
u_t^* = u_t^*(y,z)=-\fr{1}{2}\left( S_t h_t^2 +H_t\right)^{-1} \left( c_t 
\fr{\partial V_t(y,z)}{\partial z}-{}\right.\\
\left.{}-2S_t\left( s_t y-g_t z\right) h_t
\vphantom{\fr{\partial V_t(y,z)}{\partial z}}
\right)
\label{e12-bos}
\end{multline}
и доставляет минимум соответствующему сла\-га\-емо\-му в~урав\-не\-нии Беллмана, 
равный
$-\left( S_t h_t^2\hm+\right.$\linebreak
$\left.{}+H_t\right)^{-1} \left( c_t 
{\partial V_t(y,z)}/{\partial 
z}\hm-2S_t\left( s_t y \hm-g_t z\right) h_t \right)^2/4.
$ 
     
     Отметим, что, как и~в~классической ли\-ней\-но-квад\-ра\-тич\-ной 
задаче, управ\-ле\-ние из класса до\-пус\-ти\-мых не\-упреж\-да\-ющих получилось 
управ\-ле\-ни\-ем с~обратной связью.
     
     Таким образом, функция Беллмана описывается сле\-ду\-ющим 
дифференциальным уравнением:
     \begin{multline}
     \fr{\partial V_t(y,z)}{\partial t} +\fr{1}{2}\left( \Sigma_t^2(y) \fr{\partial^2 
V_t(y,z)} {\partial y^2}+\sigma_t^2\fr{\partial^2 V_t(y,z)} {\partial 
z^2}\right)+{}\\
     {}+ A_t(y) \fr{\partial V_t(y,z)}{\partial y}+\left( a_t y+b_t z\right) 
\fr{\partial V_t(y,z)}{\partial z}+{}\\
{}+ S_t \left( s_t y- g_t z\right)^2 +G_t z^2-
 \fr{1}{4}\left( S_t h_t^2+H_t\right)^{-1}\times{}\\
 {}\times \left( c_t \fr{\partial V_t(y,z)} 
{\partial z}-2S_t\left( s_t y -g_t z\right) h_t \right)^2=0\,.
     \label{e13-bos}
     \end{multline}
     
     Возводя в~квадрат по\-след\-нее сла\-га\-емое в~(\ref{e13-bos}), перепишем 
его в~виде:
     \begin{multline}
     \fr{\partial V_t(y,z)}{\partial t} +\fr{1}{2}\left( \Sigma_t^2(y) \fr{\partial^2 
V_t(y,z)} {\partial y^2}+\sigma_t^2\fr{\partial^2 V_t(y,z)} {\partial 
z^2}\!\right)+{}\\
{}+A_t(y) \fr{\partial V_t(y,z)}{\partial y}
+ \left( 
\vphantom{\left( S_t h_t^2 +H_t\right)^{-1}}
a_t y+b_t z+{}\right.\\
\left.{}+\left( S_t h_t^2 +H_t\right)^{-1}
 c_t S_t \left( s_t y-g_t z\right) h_t
\right) 
     \fr{\partial V_t(y,z)}{\partial z}+{}\\
     {}+\left( S_t-\left( S_t h_t^2 +H_t\right)^{-1} S_t^2 h_t^2\right)\left( s_t y -
g_t z\right)^2+{}\\
     \!\!{}+
     G_t z^2 -\fr{1}{4}\left( S_t h_t^2+H_t\right)^{-1}\! c_t^2
     \left(\fr{\partial V_t(y,z)}{\partial z}\right)^{\!2}=0\,.\!\!
     \label{e14-bos}
     \end{multline}
     
     Рассматривая полученное уравнение, заметим, что его решение может 
быть пред\-став\-ле\-но в~виде:
   \begin{equation}
     V_t(y,z)= \alpha_t z^2+\beta_t(y) z +\gamma_t(y)\,,
     \label{e15-bos}
     \end{equation}
т.\,е.\ будем искать решение при дополнительном предположении 
о~квад\-ра\-тич\-ности функции Белл\-ма\-на по переменной~$z$, и~сведем, таким 
образом, поиск оптимального решения к~уравнениям относительно функций 
$\alpha_t$, $\beta_t(y)$ и~$\gamma_t(y)$. Отметим сразу, что явный вид 
функции~$\gamma_t(y)$ для реализации оптимального управ\-ле\-ния не 
требуется, однако далее будет предложен вариант вы\-чис\-ле\-ния и~этой 
функции, что пред\-став\-ля\-ет\-ся небесполезным, поскольку позволит выполнять 
расчет минимума целевого функционала. Источником для 
предложения~(\ref{e15-bos}) является уже упоминавшаяся аналогичная 
задача для случая дис\-крет\-но\-го времени~\cite{7-bos, 8-bos}. В~той задаче 
выражение для функции Беллмана получается формально без 
дополнительных усилий. При этом форма~(\ref{e15-bos}) обнаруживается 
как свойство оптимального решения. В~рассматриваемом случае 
непрерывного времени~(\ref{e15-bos}) постулируется, а~пра\-виль\-ность 
постулата под\-тверж\-да\-ет\-ся далее ре\-зуль\-ти\-ру\-ющи\-ми уравнениями 
для~$\alpha_t$, $\beta_t(y)$ и~$\gamma_t(y)$ Кроме того, данное 
предположение пред\-став\-ля\-ет\-ся вы\-те\-ка\-ющим из линейной структуры задачи 
в~отношении переменной~$z$, в~част\-ности, тем фактом, что такой вид 
функции Беллмана обеспечивает линейность оптимального 
управ\-ле\-ния~(\ref{e12-bos}) по~$z$.

     Граничное условие при выбранном предположении~(\ref{e15-bos}) 
принимает вид:

\noindent
     \begin{multline*}
     V_T(y,z)= S_T\left( s_T y- g_T z\right)^2+G_T z^2 ={}\\[-0.5pt]
     {}=\alpha_T z^2 
+\beta_T(y) z +\gamma_T(y)\,,
    \end{multline*}
т.\,е.

\noindent
\begin{align*}
\alpha_T&= S_T g_T^2 +G_T\,;\\[-0.5pt]
\beta_T(y)&=-2S_T s_T g_T y\,;\\[-0.5pt]
\gamma_T(y)&=S_T s_T^2 y^2\,.
%\label{e16-bos}
\end{align*}
          При этом само оптимальное управ\-ле\-ние, определенное 
выражением~(\ref{e12-bos}), оказывается управ\-ле\-ни\-ем с~обратной связью 
по~$y_t$ и~$z_t$:

\noindent
     \begin{multline}
     u_t^*=u_t^*(y,z) ={}\\[-0.5pt]
     {}=
     -\fr{1}{2}\left( S_t h_t^2 +H_t\right)^{-1}
     \left( c_t \left( 2\alpha_t z +\beta_t(y)\right) +{}\right.\\[-0.5pt]
    \left. {}+2S_t\left( s_t y-g_t z\right) 
h_t\right)\,.
     \label{e17-bos}
     \end{multline}
          Подставляем $V_t(y,z)\hm= \alpha_t z^2 \hm+ \beta_t(y) 
z\hm+\gamma_t(y)$ в~(\ref{e14-bos}):

\noindent
     \begin{multline*}
     \fr{\partial \alpha_t}{\partial t}\, z^2 +
     \fr{\partial \beta_t(y)}{\partial t}\,z +
     \fr{\partial \gamma_t(y)}{\partial t}+{}\\[-0.5pt]
     {}+\fr{1}{2}\left( \Sigma_t^2(y) \left( 
\fr{\partial^2\beta_t(y)}{\partial y^2}\,z +\fr{\partial^2 \gamma_t(y)}{\partial 
y^2}\right) +2\sigma_t^2\alpha_t\right)+{}\\[-0.5pt]
 {}+A_t(y)\left(\fr{\partial \beta_t(y)}{\partial y}\,z + \fr{\partial 
\gamma_t(y)}{\partial y}\right) +{}\\[-0.5pt]
\hspace*{-0.22987pt}{}+\left( a_t y+b_t z+\left( S_t h_t^2 +H_t\right)^{-1} c_t S_t \left( s_t y-
g_t z\right) h_t\right)\times{}
\end{multline*}

\noindent
\begin{multline*}
         {}\times \left( 2\alpha_t z+\beta_t(y)\right)+{}\\
     {}+\left( S_t-\left( S_t h_t^2 +H_t\right)^{-1} S_t^2 h_t^2\right)\left( s_t y-
g_t z\right)^2+{}\\
     {}+ G_t z^2 -\fr{1}{4}\left( S_t h_t^2 +H_t\right)^{-1} c_t^2 \left( 
2\alpha_t z+\beta_t(y)\right)^2=0\,.
     \end{multline*}
          Далее выделяем слагаемые при~$z^2$, $z$ и~$z^0$
          
          \noindent
     \begin{multline*}
     \fr{\partial \alpha_t}{\partial t}\, z^2 +\fr{\partial \beta_t(y)}{\partial t}\,z +
     \fr{\partial \gamma_t(y)}{\partial 
t}+\fr{1}{2}\,\Sigma_t^2(y)\fr{\partial^2\beta_t(y)}{\partial y^2}\,z+ {}\\
{}+
\fr{1}{2}\,\Sigma_t^2(y)\fr{\partial^2\gamma_t(y)}{\partial 
y^2}+\sigma_t^2\alpha_t+A_t(y)\fr{\partial \beta_t(y)}{\partial y}\,z +{}\\
{}+A_t(y) \fr{\partial 
\gamma_t(y)}{\partial y}+{}\\
{}+ 2\alpha_t \left( b_t -\left( S_t h_t^2+H_t\right)^{-1} c_t 
S_t h_t g_t \right)z^2+{}\\
     {}+
     \left( 2\alpha_t\left( \alpha_t+\left( S_t h_t^2+H_t\right)^{-1} c_t S_t h_t 
s_t\right)y +{}\right.\\
\left.{}+\beta_t(y) \left( b_t-\left( S_t h_t^2+H_t\right)^{-1} c_t S_t h_t 
g_t\right) \right) z+{}\\
     {}+\beta_t(y)\left( a_t +\left( S_t h_t^2+H_t\right)^{-1} c_t S_t h_t s_t\right) 
y+{}\\
{}+ \left( S_t -\left( S_t h_t^2+H_t\right)^{-1} S_t^2 h_t^2\right) g_t^2 z^2-{}\\
     {}- 2\left( S_t -\left( S_t h_t^2+H_t\right)^{-1} S_t^2 h_t^2\right) s_t g_t yz 
+{}\\
{}+
     \left( S_t-\left( S_t h_t^2+H_t\right)^{-1} S_t^2 h_t^2\right) s_t^2 y^2+{}\\
     {}+G_t z^2 -\left( S_t h_t^2 +H_t\right)^{-1} c_t^2 \alpha_t^2 z^2 -{}\\
     {}-\left( 
S_t h_t^2+H_t\right)^{-1} c_t^2 \alpha_t \beta_t(y) z-{}\\
{}-
\fr{1}{4}\left( S_t h_t^2+H_t\right)^{-1}  c_t^2 \beta_t^2(y)=0\,,
     \end{multline*}
группируем их и~получаем сле\-ду\-ющие уравнения:
\begin{itemize}
\item  для~$\alpha_t$:

\noindent
\begin{multline}
\fr{\partial\alpha_t}{\partial t}+2\alpha_t\left( b_t-\left( S_t h_t^2+H_t\right)^{-1} c_t 
S_t h_t g_t\right)+{}\\
{}+ \left( S_t- \left( S_t h_t^2+H_t\right)^{-1} S_t^2 h_t^2\right) 
g_t^2+G_t-{}\\
\hspace*{-8mm}{}-\left( S_t h_t^2+H_t\right)^{-1} c_t^2 \alpha_t^2 =0\,,\enskip \alpha_T=S_T 
g_t^2+G_T\,;\!\!
\label{e18-bos}
\end{multline}
\item для $\beta_t$:

\noindent
\begin{multline}
\fr{\partial\beta_t(y)}{\partial 
t}+\fr{1}{2}\,\Sigma_t^2(y)\fr{\partial^2\beta_t(y)}{\partial y^2} 
+A_t(y)\fr{\partial \beta_t(y)}{\partial y}+{}\\
{}+ 2\alpha_t\left( a_t +\left( S_t h_t^2+H_t\right)^{-1} c_t S_t h_t s_t\right) y+{}\\
{}+
\beta_t(y)\left( b_t -\left( S_t h_t^2 +H_t\right)^{-1} c_t S_t h_t g_t\right)-{}\\
{}-2\left( S_t-\left( S_t h_t^2+H_t\right)^{-1} S_t^2 h_t^2\right) s_t g_t y-{}
\\
{}-
\left( S_t h_t^2+H_t\right)^{-1} c_t^2 \alpha_t \beta_t(y)=0\,,\\
\beta_T(y)=-2S_T s_T g_T y\,;
\label{e19-bos}
\end{multline}
\item  для $\gamma_t$:
\begin{multline}
\hspace*{-0.8pt}\fr{\partial \gamma_t(y)}{\partial t}+\fr{1}{2}\,\Sigma_t^2(y)
\fr{\partial^2 \gamma_t(y)}{\partial y^2} +\sigma_t^2 \alpha_t +A_t(y)
\fr{\partial \gamma_t(y)}{\partial y}+{}\\
{}+ \beta_t(y)\left( a_t +\left( S_t h_t^2+H_t\right)^{-1} c_t S_t h_t s_t\right) y+{}\\
{}+
\left( S_t-\left( S_t h_t^2+H_t\right)^{-1} S_t^2 h_t^2\right)  s_t^2 y^2-{}\\
{}-\fr{1}{4}\left( S_t h_t^2+H_t\right)^{-1} c_t^2 \beta_t^2(y) =0\,,\\
\gamma_T(y)=S_T s_T^2 y^2\,.
\label{e20-bos}
\end{multline}
\end{itemize}
     
     Уравнение~(\ref{e18-bos}), легко заметить, является уравнением 
Риккати, которое в~силу сформулированного выше условия   
имеет единственное неотрицательное решение для всех $0\hm\leq t\hm\leq T$. 
Этот факт требует дополнительного комментария. Для получения 
уравнения~(\ref{e18-bos}) рас\-смот\-рим исходную задачу при дополнительных 
условиях $a_t\hm=0$ и~$s_t\hm=0$ для всех $0\hm\leq t\hm\leq T$. Нетрудно 
видеть, что эти условия рассматриваемую по\-ста\-нов\-ку сводят фактически 
к~классической ли\-ней\-но-квад\-ра\-тич\-ной задаче. Имеющуюся 
в~рассматриваемой формулировке чуть более общую форму целевой 
функции (принципиального значения это обобщение, конечно, не имеет) 
сведем к~классической еще одним предположением: $S_t\hm=0$ для всех 
$0\hm\leq t\hm\leq T$. Теперь уравнение для~$\alpha_t$ принимает хорошо 
известный вид:
     \begin{equation}
     \fr{\partial \alpha_t}{\partial t}+2\alpha_t b_t +G_t- H_t^{-1} c_t^2 
\alpha_t^2=0\,,\enskip \alpha_T=G_T\,.
     \label{e21-bos}
     \end{equation}

     В таком случае, как известно~\cite{10-bos}, существует единственное 
оптимальное управление~--- линейное с~обратной связью по выходу~$z_t$, 
с~коэффициентом усиления, опи\-сы\-ва\-емым уравнением  
Риккати~(\ref{e21-bos}). Именно этот результат дают  
уравнения~(\ref{e18-bos})--(\ref{e20-bos}) и~описываемая ими функция 
Беллмана~(\ref{e15-bos}), так как из $a_t\hm=0$ и~$s_t\hm=0$ немедленно 
следует, что $\beta_t(y)\hm=0$, откуда, в~свою очередь, с~учетом 
не\-за\-ви\-си\-мости решения от~$y_t$ следует, что $\gamma_t(y)\hm=\gamma_t$, 
т.\,е.\ не зависит от~$y$ и~задается уравнением: 
     $$
     \fr{\partial \gamma_t(y)}{\partial t} +\sigma^2_t \alpha_t=0\,,\enskip 
\gamma_T=0\,.
     $$ 
     Оптимальное управ\-ле\-ние при этом 
     $$
     u_t^*= -H_t^{-1} c_t \alpha_t z_t\,,
     $$
      т.\,е.\ все полностью совпадает с~известным классическим решением.
     
     С уравнениями~(\ref{e19-bos}) и~(\ref{e20-bos}) ситуация, естественно, 
обстоит сложнее. Это линейные уравнения второго порядка параболического 
типа, поскольку\linebreak
 $\Sigma_t^2(y)\hm>0$. Фактически отсутствуют 
конструктивные условия, гарантирующие существование их\linebreak
 решений 
(требовать, чтобы все фигурирующие в~уравнениях коэффициенты были 
представлены аналитическими функциями на всем пространстве значений, 
вряд ли целесообразно), поэтому далее будем предполагать, что данные 
уравнения имеют на рас\-смат\-ри\-ва\-емом интервале $0\hm\leq t\hm\leq T$ хотя 
бы одно ограниченное решение и~именно эти условия будем рас\-смат\-ри\-вать 
как достаточные условия существования оптимального решения 
рассматриваемой задачи.
     
     Таким образом, доказана следующая тео\-рема.
     
     \smallskip
     
     \noindent
     \textbf{Теорема.}\ \textit{Пусть для диффузионного 
процесса}~(\ref{e5-bos}) \textit{выполнены условия Ито, для 
     процесса}~(\ref{e6-bos})~--- \textit{ограничены коэффициенты, 
уравнения}~(\ref{e18-bos})--(\ref{e20-bos}) \textit{имеют ограниченные 
решения для $0\hm\leq t\hm\leq T$. Тогда минимум  
функционалу}~(\ref{e7-bos}) \textit{доставляет оптимальное 
управ\-ле\-ние}~(\ref{e17-bos}), \textit{где} $y\hm= y_t$; $z\hm=z_t$.
     
\section{Заключение}

     Рассмотренная задача оптимизации в~целом близка и~по модели, и~по 
критерию к~классической ли\-ней\-но-квад\-ра\-тич\-ной постановке. 
Принципиальным отличием является нелинейная модель для описания 
со\-сто\-яния динамической сис\-те\-мы, в~которой отсутствует управ\-ля\-ющее 
воздействие.\linebreak
 Такую модель наряду с~традиционной интер\-пре\-тацией  
<<со\-сто\-яние--вы\-ход>> мож\-но понимать как\linebreak модель неконтролируемого 
слож\-но\-го внешнего воздействия. Небольшое дополнительное отличие дает 
предложенная форма квад\-ра\-тич\-но\-го критерия, поз\-во\-ля\-ющая, в~част\-ности, 
ставить такие задачи, как отслеживание выходом или управ\-ле\-ни\-ем со\-сто\-яния 
сис\-те\-мы или ее выхода.
     
     Поскольку обсуждать возможности точного решения уравнений, 
определяющих оптимальное управ\-ле\-ние, не имеет смыс\-ла, наиболее 
актуальной далее является задача их приближенного чис\-лен\-но\-го решения 
и~анализа воз\-мож\-ности практической реализации. Этому посвящена вторая 
часть данной работы, пла\-ни\-ру\-емая к~выходу в~ближайшее время.

{\small\frenchspacing
 {%\baselineskip=10.8pt
 \addcontentsline{toc}{section}{References}
 \begin{thebibliography}{99}
\bibitem{1-bos}
\Au{Athans M.} Editorial on the LQG problem~// IEEE~T. Automat. Contr., 1971. Vol.~16. 
No.\,6. P.~528--552. doi: 10.1109/TAC.1971.1099845.
\bibitem{2-bos}
\Au{Wu Z.} Forward-backward stochastic differential equations, linear quadratic stochastic 
optimal control and nonzero sum differential games~// J.~Syst. Sci. Complex., 2005. Vol.~18. 
No.\,2. P.~179--192.
\bibitem{3-bos}
\Au{Chen B.\,S., Zhang~W.} Stochastic H2/H1 control with state-dependent noise~// IEEE 
T.~Automat. Contr., 2004. Vol.~49. No.\,1. P.~45--56. doi: 10.1109/TAC.2003.821400.
\bibitem{4-bos}
\Au{Bohacek S.} A~stochastic model of TCP and fair video transmission~// IEEE 
INFOCOM, 2003. Vol.~2. P.~1134--1144. doi: 10.1109/INFCOM.2003.1208950.
\bibitem{5-bos}
\Au{Домбровский В.\,В., Объедко~Т.\,Ю.} Управление с~прогнозированием системами 
с~марковскими скачками при ограничениях и~применение к~оптимизации 
инвестиционного портфеля~// Автомат. телемех., 2011. №\,5. С.~96--112. doi: 
10.1134/S0005117911050079.
\bibitem{6-bos}
\Au{Баландин Д.\,В., Коган~М.\,М.} Оптимальное линейно-квад\-ра\-тич\-ное управление: от 
матричных уравнений к~линейным матричным неравенствам~// Автомат. телемех., 2011. 
№\,11. С.~60--69. doi: 10.1134/ S0005117911110038.
\bibitem{7-bos}
\Au{Босов А.\,В.} Обобщенная задача распределения ресурсов программной системы~// 
Информатика и~её применения, 2014. Т.~8. Вып.~2. С.~39--47. doi: 
10.14357/19922264140204.
\bibitem{8-bos}
\Au{Босов А.\,В.} Управление линейным выходом дискретной стохастической системы по 
квадратичному критерию~// Изв. РАН. Теория и~системы управления, 2016. №\,3.  
С.~19--35. doi: 10.1134/S1064230716030060.
\bibitem{9-bos}
\Au{Флеминг У., Ришел~Р.} Оптимальное управление детерминированными 
и~стохастическими системами~/ Пер. с~англ.~--- М.: Мир, 1978. 316~с. 
(\Au{Fleming~W.\,H., Rishel~R.\,W.} Deterministic and stochastic optimal control.~--- New 
York, NY, USA: Springer-Verlag, 1975. 222~p.)
\bibitem{10-bos}
\Au{Девис М.\,Х.\,А.} Линейное оценивание и~стохастическое управление~/ Пер. с~англ.~--- 
М.: Наука, 1984. 206~с. (\Au{Davis~M.\,H.\,A.} Linear estimation and stochastic control.~--- 
London: Chapman and Hall, 1977. 224~p.)

 \end{thebibliography}

 }
 }

\end{multicols}

\vspace*{-6pt}

\hfill{\small\textit{Поступила в~редакцию 30.03.18}}

\vspace*{4pt}

%\newpage

%\vspace*{-24pt}

\hrule

\vspace*{2pt}

\hrule

\vspace*{-2pt}


\def\tit{STOCHASTIC DIFFERENTIAL SYSTEM OUTPUT CONTROL 
BY~THE~QUADRATIC CRITERION.~I.~DYNAMIC\\ PROGRAMMING 
OPTIMAL SOLUTION}


\def\titkol{Stochastic differential system output control 
by~the~quadratic criterion. I.~Dynamic programming 
optimal solution}

\def\aut{A.\,V.~Bosov and~A.\,I.~Stefanovich}

\def\autkol{A.\,V.~Bosov and~A.\,I.~Stefanovich}

\titel{\tit}{\aut}{\autkol}{\titkol}

\vspace*{-11pt}


\noindent
Institute of Informatics Problems, Federal Research Center ``Computer Science 
and Control'' of the Russian Academy of Sciences, 44-2~Vavilov Str., Moscow 
119333, Russian Federation


\def\leftfootline{\small{\textbf{\thepage}
\hfill INFORMATIKA I EE PRIMENENIYA~--- INFORMATICS AND
APPLICATIONS\ \ \ 2018\ \ \ volume~12\ \ \ issue\ 3}
}%
 \def\rightfootline{\small{INFORMATIKA I EE PRIMENENIYA~---
INFORMATICS AND APPLICATIONS\ \ \ 2018\ \ \ volume~12\ \ \ issue\ 3
\hfill \textbf{\thepage}}}

\vspace*{3pt}



\Abste{The problem of optimal control for the Ito diffusion 
process and a~controlled linear output is solved. The considered 
statement is close to the classical linear-quadratic Gaussian 
control  (LQG control) problem. Differences consist in the fact 
that the state is described by the nonlinear differential Ito equation  $dy_y = A_t(y_t) 
\,dt+\Sigma_t(y_t)\,dv_t$ and does not depend on the control~$u_t$, 
optimization subject is controlled linear output 
 $dz_t=a_ty_t\,dt +b_tz_t\,dt +c_t u_t\,dt +\sigma_t \,dw_t$. 
Additional generalizations are included in the quadratic 
quality criterion for the purpose of statement such problems 
as state tracking by output or a linear combination of state 
and output tracking by control. The method of dynamic programming 
is used for the solution. 
The assumption about Bellman function in the form  $V_t(y,z)= \alpha_t 
z^2+\beta_t(y) z+\gamma_t(y)$ allows one to find it. 
Three differential equations for the coefficients $\alpha_t$,  $\beta_t(y)$,
and $\gamma_t(y)$ give the solution. 
These equations constitute the optimal solution of the problem under consideration.}

\KWE{stochastic differential equation; optimal control; dynamic programming; 
Bellman function; Riccati equation; linear differential equations of parabolic type}


\DOI{10.14357/19922264180314}

\vspace*{-12pt}

\Ack
\noindent
This work was partially supported by the Russian Science Foundation (grant  
16-07-00677).



%\vspace*{6pt}

  \begin{multicols}{2}

\renewcommand{\bibname}{\protect\rmfamily References}
%\renewcommand{\bibname}{\large\protect\rm References}

{\small\frenchspacing
 {%\baselineskip=10.8pt
 \addcontentsline{toc}{section}{References}
 \begin{thebibliography}{99}
\bibitem{1-bos-1}
\Aue{Athans, M.} 1971. Editorial on the LQG problem. \textit{IEEE~T. 
Automat. Contr.} 16(6):528--552. doi: 10.1109/ TAC.1971.1099845.
\bibitem{2-bos-1}
\Aue{Wu, Z.} 2005. Forward-backward stochastic differential equations, linear 
quadratic stochastic optimal control and\linebreak\vspace*{-12pt}

\columnbreak

\noindent
 nonzero sum differential games. 
\textit{J.~Syst. Sci. Complex.} 18(2):179--192.
\bibitem{3-bos-1}
\Aue{Chen, B.\,S. and W.~Zhang.} 2004. Stochastic H2/H1 control with  
state-dependent noise. \textit{IEEE~T. Automat. Contr.} 49(1):45--56.
doi: 10.1109/TAC.2003.821400.
\bibitem{4-bos-1}
\Aue{Bohacek, S.} 2003. A~stochastic model of TCP and fair video 
transmission. \textit{IEEE INFOCOM}. 2:1134--1144.
doi: 10.1109/INFCOM.2003.1208950.
\bibitem{5-bos-1}
\Aue{Dombrovskii, V.\,V., and T.\,Yu.~Ob''edko.} 2011. Predictive control of 
systems with Markovian jumps under constraints and its application to the 
investment portfolio optimization. \textit{Automat. Rem. Contr.}  
72(5):989--1003.
\bibitem{6-bos-1}
\Aue{Balandin, D.\,V., and M.\,M.~Kogan.} 2011. Optimal linear-quadratic 
control: From matrix equations to linear matrix inequalities. \textit{Automat. 
Rem. Contr.} 72(11):2276--2284.
\bibitem{7-bos-1}
\Aue{Bosov, A.\,V.} 2014. Obobshchennaya zadacha raspredeleniya resursov 
programmnoy sistemy [The generalized problem of software system resources 
distribution]. \textit{Informatika i~ee Primeneniya~--- Inform. Appl.}  
8(2):39--47. doi: 
10.14357/19922264140204.
\bibitem{8-bos-1}
\Aue{Bosov, A.\,V.} 2016. Discrete stochastic system linear output control 
with respect to a quadratic criterion. \textit{J.~Comput. Syst. Sc. 
Int.} 55(3):349--364.
\bibitem{9-bos-1}
\Aue{Fleming, W.\,H., and R.\,W.~Rishel.} 1975. \textit{Deterministic and 
stochastic optimal control.} New York, NY: Springer-Verlag. 222~p.
\bibitem{10-bos-1}
\Aue{Davis, M.\,H.\,A.} 1977. \textit{Linear estimation and stochastic 
control.} London: Chapman and Hall. 224~p.
\end{thebibliography}

 }
 }

\end{multicols}

\vspace*{-6pt}

\hfill{\small\textit{Received March 30, 2018}}

%\pagebreak

%\vspace*{-18pt}
     
     \Contr
     
       \noindent
       \textbf{Bosov Alexey V.} (b.\ 1969)~--- Doctor of Science in technology, 
principal scientist, Institute of Informatics Problems, Federal Research 
Center ``Computer Science and Control'' of the Russian Academy of Sciences, 
44-2~Vavilov Str., Moscow 119333, Russian Federation; 
\mbox{AVBosov@ipiran.ru}
       
       \vspace*{3pt}
       
       \noindent
       \textbf{Stefanovich Alexey I.} (b.\ 1983)~--- principal specialist, 
Institute of Informatics Problems, Federal Research Center ``Computer Science 
and Control'' of the Russian Academy of Sciences, 44-2~Vavilov Str., Moscow 
119333, Russian Federation; \mbox{AStefanovich@frccsc.ru}
\label{end\stat}

\renewcommand{\bibname}{\protect\rm Литература}       

        %2
\def\stat{gorsh}

\def\tit{ОБ УСТОЙЧИВОСТИ СДВИГОВЫХ СМЕСЕЙ НОРМАЛЬНЫХ ЗАКОНОВ ПО~ОТНОШЕНИЮ 
К~ИЗМЕНЕНИЯМ СМЕШИВАЮЩЕГО РАСПРЕДЕЛЕНИЯ}

\def\titkol{Об устойчивости сдвиговых смесей нормальных законов по~отношению 
к~изменениям смешивающего распределения}

\def\autkol{А.\,К.~Горшенин}
\def\aut{А.\,К.~Горшенин$^1$}

\titel{\tit}{\aut}{\autkol}{\titkol}

{\renewcommand{\thefootnote}{\fnsymbol{footnote}}\footnotetext[1]
{Работа
выполнена при поддержке Российского фонда фундаментальных
исследований (проекты 11-01-12026-офи-м и 12-07-00115).}}


\renewcommand{\thefootnote}{\arabic{footnote}}
\footnotetext[1]{Институт проблем информатики
Российской академии наук, agorshenin@ipiran.ru}


\Abst{Работа посвящена изучению устойчивости конечных сдвиговых смесей
нормальных законов относительно изменений параметров
смешивающего распределения. Результаты формулируются для моделей
добавления и расщепления компоненты, которые используются в задачах
проверки статистических гипотез о числе компонент смеси.}


\KW{сдвиговые смеси нормальных законов; метрика Леви}

\vskip 14pt plus 9pt minus 6pt

      \thispagestyle{headings}

      \begin{multicols}{2}

            \label{st\stat}
     


\newcommand*{\E}{\mathbb E}

\section{Введение}

В работе рассмотрены две модели конечных сдвиговых смесей
нормальных распределений: добавления и расщепления компоненты
(подробнее см.\ работы~\cite{Gorshenin2011, Gorshenin2011Mod2}).
Данные модели являются весьма удобными и информативными при
проведении статистического анализа данных с помощью различных
итерационных процедур разделения смесей вероятностных
распределений и позволяют эффективно решать задачи проверки гипотез о числе компонент смеси.

Модель добавления компоненты удобно использовать для проверки
значимости компоненты с малым весом.  Дело в том, что при
статистическом определении параметров в модели типа
смеси вероятностных распределений может появиться ком\-по\-нен\-та, вес
которой значительно меньше весов остальных компонент. В такой
ситуации необходимо убедиться в статистической значимости этой
компоненты, чтобы избежать влияния погрешностей вычисления на
итоговый результат.

Модель расщепления компоненты может применяться для в некотором
смысле обратной задачи, когда в силу вычислительных ошибок
компонента с малым весом может быть ошибочно отнесена к одной из
компонент с б$\acute{\mbox{о}}$льшим весом. Наличие устойчивости играет важную
роль при практическом использовании данных моделей, так как
гарантирует корректность полученных результатов.

В работе~\cite{Gorshenin2012SSI} были получены результаты для
конечных масштабных смесей нормальных законов. Однако в ряде
ситуаций оказывается полезным рассмотрение сдвиговых конечных
нормальных смесей. Модели такого типа возникают, например, при
решении оптимизационных задач для управлении запасами, при
моделировании потоков страховых выплат, при прогнозировании
надежности различных систем. Доказательству теорем устойчивости для
сдвиговых конечных смесей нормальных законов  и посвящена настоящая
статья.

\section{Постановка задачи}

Предположим, что каждое из независимых наблюдений
$\mathbf{X}_n\hm=(X_1,\ldots,X_n)$ имеет распределение,
представимое в виде конечной сдвиговой смеси нормальных законов,
т.\,е.\
\begin{equation}
\label{Stab_G} 
G(x)=\sum\limits_{i=1}^k p_i\Phi(x-a_i)\,,
\end{equation}
где
\begin{equation*}
\sum\limits_{i=1}^{k}p_i=1\,,\quad p_i\geqslant0\,,\enskip a_i\in\R\,,\enskip  i=1,\ldots,k,
\end{equation*}
а через $\Phi(\cdot)$ обозначена функция распределения
стандартного нормального закона
\begin{equation*}
\Phi(x)=\fr{1}{\sqrt{2\pi}}\int\limits_{-\infty}^x\exp{\left\{
-\fr{t^2}2\right\}}\,dt\,.
\end{equation*}

Также в дальнейшем будет использоваться функция плотности
стандартного нормального закона
$$
\varphi(x)=\fr{1}{\sqrt{2\pi}}\exp{\left\{ -\fr{x^2}2\right\}}\,.
$$

Очевидно, что функция распределения $G(x)$ из
соотношения~\eqref{Stab_G} может быть представлена в виде:
$$
G(x)=\E\Phi(x-V)\,,
$$
где $V$~--- дискретная случайная величина, принимающая значения
$a_i$ с вероятностями~$p_i$, т.\,е.\
\begin{equation}
\label{Stab_V} V:
\begin{tabular}{cccc}
$a_1$&$a_2$&$\cdots$&$a_k$\\
$p_1$&$p_2$&$\cdots$&$p_k$\,.
\end{tabular}
\end{equation}

Обозначим  через $\rho(F,G)$ равномерное расстояние между
функциями распределения $F(x)$ и~$G(x)$:
\begin{equation}
\label{Stab_ro}
\rho(F,G)=\sup\limits_{x\in\R}|F(x)-G(x)|\,.
\end{equation}

Известно, что для решения задачи устойчивости для конечных
сдвиговых смесей нормальных законов равномерная
метрика~\eqref{Stab_ro} является не вполне корректной (можно привести
пример весьма близких функций распределения, для которых
равномерная метрика будет давать расстояние, равное единице). Поэтому
необходимо рассматривать мет\-ри\-ки, метризующие слабую сходимость,
например метрику Леви~$L(F,G)$ между функциями распределения
$F(x)$ и~$G(x)$, определяемую соотношением:
\begin{multline*}
L(F,G)=\inf\left\{h:\, G(x-h)-h\leqslant{}\right.\\
\left.{}\leqslant F(x)\leqslant G(x+h)+h,
\forall x\in\R\right\}\,.
\end{multline*}

Модели добавления и расщепления компоненты могут быть
представлены в виде:
$$
G_p(x)=\E\Phi(x-V_p)\,,
$$
где дискретная случайная величина~$V_p$ определяется для каждой из
моделей по-раз\-но\-му. Необходимо получить соотношения, связывающие
расстояния Леви между смешивающими распределениями и смесями.
Перейдем к рассмотрению каждой из моделей.

\section{Модель добавления компоненты}

Модель добавления компоненты формализуется следующим образом.
Предполагается, что каждое из независимых наблюдений
$\mathbf{X}_n\hm=(X_1,\ldots,X_n)$ имеет распределение,
представимое в виде:
\begin{equation}
\label{Stab_G_p1}
G_p(x)=(1-p)\sum\limits_{i=1}^k
p_i\Phi(x-a_i)+p\Phi(x-a)\,,
\end{equation}
где все величины $a_i\in\R$, $p_i\hm\geqslant 0$, $i\hm=1,\ldots,k,$ считаются
известными, а~$a$ и~$p$ являются параметрами модели, при этом
$a\hm\in\R$, $0\hm\leqslant p\hm\leqslant 1$. Без ограничения общности для
определенности будем считать, что выполнены соотношения
\begin{equation}
\label{Stab_a1}
a_0\leqslant a\leqslant a_1\leqslant a_2\leqslant\cdots\leqslant a_k\,.
\end{equation}
%
Левое неравенство означает достаточно естест\-вен\-ное для практики
предположение, что рас\-смат\-ри\-ва\-ют\-ся конечные математические
ожидания. Поэтому в дальнейшем считаем~$a_0$ известным
параметром модели (так как он может быть указан из некоторых разумных
предположений для каждого конкретного случая).

В модели добавления компоненты дискретная случайная величина~$V_p$
имеет следующий вид:
\begin{equation}
\label{Stab_V_p1} 
V_p:\!\!
\begin{tabular}{ccccc}
$a$\!&$a_1$\!&$a_2$\!&$\cdots$\!\!&$a_k$\\
$p$\!&$p_1(1-p)$\!&$p_2(1-p)$\!&$\cdots$\!\!&$p_k(1-p)$.
\end{tabular}\!\!\!
\end{equation}

Отметим, что расстояние Леви $L(V,V_p)$ не превосходит величины~$p$,
так как расстояние между ступеньками функций распределения
составляет в точности $p$ на сегменте $[a,a_1]$ и $p p_i$ на сегментах
$[a_i,a_{i+1}]$, $i=1,\ldots,k-1$. Изменяться могут лишь параметры~$a$ и
$p$, величины $a_i$, $p_i$, $i\hm=1,\ldots,k,$ считаем постоянными. Однако
при фиксированном параметре $p$ и при $a\hm\to a_1$
очевидно, что $L(V,V_p)$ к нулю не стремится. Таким образом,
без ограничения общности считаем, что $0\hm\leqslant p\hm\leqslant a_1- a$. Поэтому
\begin{equation}
\label{Stab_Cond1} 
L(V,V_p)=p\,.
\end{equation}

Тогда справедлива следующая теорема.

\medskip

\noindent
\textbf{Теорема 1.}
\textit{В~рамках модели добавления компоненты~\eqref{Stab_G_p1} при
выполнении условий~\eqref{Stab_a1} и~\eqref{Stab_Cond1} расстояние
Леви $L(V,V_p)$ между смешивающими распределениями~$V$ из
соотношения~\eqref{Stab_V} и~$V_p$ из соотношения~\eqref{Stab_V_p1}
и расстояние Леви $L(G,G_p)$ между истинным распределением~$G(x)$
из соотношения~\eqref{Stab_G} и приближающей смесью $G_p(x)$ из
соотношения~\eqref{Stab_G_p1} связывают неравенства}
\begin{multline*}
C_1^{[1]}(a_0,a_k)L(G,G_p)\leqslant L(V,V_p)\leqslant{}\\
{}\leqslant
C_2^{[1]}(a_0,a_k)L^{1/2}(G,G_p)\,, 
\end{multline*}
\textit{где коэффициенты $C_j^{[1]}(a_0,a_k)$, $j=1,2$, зависящие
только от известных величин $a_k$ и $a_0$, имеют вид:}
\begin{align}
C_1^{[1]}(a_0,a_k)&=\max
\left\{1,\fr{\sqrt{2\pi}}{a_k-\min\{0,a_0\}}\right\}\,;
\label{Stab_C11}\\
C_2^{[1]}(a_0,a_k)&=\varphi^{-1/2}\left(a_k+|a_k|-{}\right.\notag\\
&\left.{}-\min\{0,a_0\}\right)
\left(1+\fr{1}{\sqrt{2\pi}}\right)^{1/2}\,.
\label{Stab_C12}
\end{align}


\medskip

\noindent
Д\,о\,к\,а\,з\,а\,т\,е\,л\,ь\,с\,т\,в\,о\,.\ 
Запишем оценки снизу для равномерного расстояния между функциями
распределения $G(x)$ и $G_p(x)$, воспользовавшись формулой
Лагранжа:
\begin{multline}
\rho(G,G_p)=\sup\limits_x|G(x)-G_p(x)|={}\\
{}=\sup\limits_x|G(x)-
G(x)+p(G(x)-\Phi(x-a))|={}\\
{}=p\sup\limits_x|G(x)-\Phi(x-a)|\geqslant
p|G(x_0-a_i)-\Phi(x_0-a)|={}\\
{}=p\left|\sum\limits_{i=1}^k
p_i\left(\Phi(x_0-a_i)-\Phi(x_0-a)\right)\right|={}\\
\!\!{}=\!p\left|\sum\limits_{i=1}^k
p_i(a-a_i)\varphi(\theta_i(x_0-a_i)\!+\!(1-\theta_i)(x_0-a))\right|\!={}\\
{}=p\left|\sum\limits_{i=1}^k p_i(a_i-a)\varphi(x_0-a-\theta_i(a_i-a))\right|\,.
\label{rho}
\end{multline}

Неравенство в соотношении~\eqref{rho} справедливо для любого~$x_0$.
Выберем значение данной величины так, чтобы воспользоваться
свойством монотонного убывания плотности стандартного нормального
распределения~$\varphi(x)$ от положительного аргумента. А~именно
потребуем выполнения условия
$$
x_0-a-\theta_i(a_i-a)\geqslant 0\,,
$$
откуда следует (с учетом того, что выражение в скобках
в силу условий~\eqref{Stab_a1} неотрицательно и $0\hm\leqslant\theta_i\hm\leqslant 1$),
что
\begin{equation}
x_0\geqslant a_i 
\label{x0Cond}
\end{equation}
сразу для всех номеров~$i$. Тогда в качестве  $x_0$ возьмем
величину
\begin{equation}
x_0=a_k+|a_k|\,. 
\label{x0}
\end{equation}

Очевидно, что условие~\eqref{x0Cond} выполняется,
при этом $x_0\hm\geqslant 0$ и $x_0-a\hm\geqslant 0$. Тогда, продолжая~\eqref{rho} с
учетом соотношений~\eqref{Stab_a1} и~\eqref{Stab_Cond1}, получим:
\begin{multline*}
\rho(G,G_p)\geqslant{}\\
{}\geqslant p\left|\sum\limits_{i=1}^k
p_i(a_i-a)\varphi\left(a_k+|a_k|-a-\theta_i(a_i-a)\right)\right|\geqslant{}\\
{}\geqslant p\left|\sum\limits_{i=1}^k p_i(a_i-a)\varphi\left(a_k+|a_k|-a\right)\right|
\geqslant{}\\
{}\geqslant p\left|\sum\limits_{i=1}^k
p_i(a_i-a)\varphi\Big(a_k+|a_k|-\min\{0,a_0\}\Big)\right|\geqslant{}\\
{}\geqslant p\sum\limits_{i=1}^k p_i(a_1-a)\varphi\left(a_k+|a_k|-\min\{0,a_0\}\right)
={}\\
{}=p(a_1-a)\varphi\left(a_k+|a_k|-\min\{0,a_0\}\right)\geqslant{}\\
{}\geqslant L^2(V,V_p)\varphi\left(a_k+|a_k|-\min\{0,a_0\}\right)\,.
\end{multline*}

Воспользуемся известным неравенством для мет\-ри\-ки Леви (см.,
например, книгу~\cite{Zolotarev1986}):
\begin{multline}
\label{Stab_Lrho} 
L(G,G_p)\leqslant\rho(G,G_p)\leqslant{}\\
{}\leqslant (1+\max\limits_x
G'(x))L(G,G_p)\,.
\end{multline}

Воспользуемся правым неравенством из соотношения~\eqref{Stab_Lrho}.
Имеем
\begin{multline*}
L^2(V,V_p)\varphi\left(a_k+|a_k|-\min\{0,a_0\}\right)\leqslant\rho(G,G_p)\leqslant{}\\
{}\leqslant (1+
\max\limits_x G'(x))L(G,G_p)={}\\
{}=\left(1+\max\limits_x \left(\sum\limits_{i=1}^k
p_i\varphi(x-a_i) \right)\right)L(G,G_p)\leqslant{}\\
{}\leqslant
\left(1+\sum\limits_{i=1}^k
p_i \fr {1}{\sqrt{2\pi}}\right)L(G,G_p)={}\\
{}=\left(1+\fr{1}{\sqrt{2\pi}}\right)L(G,G_p)\,.
\end{multline*}

Окончательно получаем следующую оценку сверху для
$L(V,V_p)$:
\begin{multline*}
L(V,V_p)\leqslant\varphi^{-1/2}\left(a_k+|a_k|-\min\{0,a_0\}\right)\times{}\\
{}\times
\left(1+\fr{1}{\sqrt{2\pi}}\right)^{1/2}L^{1/2}(G,G_p)={}\\
{}=
C_2^{[1]}(a_0,a_k)L^{1/2}(G,G_p)\,.
\end{multline*}

Оценка снизу для $L(V,V_p)$ может быть найдена из соотношений
\begin{multline*}
L(G,G_p)\leqslant\rho(G,G_p)=\sup\limits_x|G(x)-G_p(x)|={}\\
{}=p\sup\limits_x\left|\sum\limits_{i=1}^k
p_i(\Phi(x-a_i)-\Phi(x-a))\right|
\leqslant{}\\
{}\leqslant
 p\sup\limits_x\sum\limits_{i=1}^k
p_i|\Phi(x-a_i)-\Phi(x-a)|\leqslant{}\\
{}\leqslant p\sum\limits_{i=1}^k
p_i\sup\limits_x|\Phi(x-a_i)-\Phi(x-a)|\leqslant{}\\
{}\leqslant
p\sum\limits_{i=1}^k p_i= L(V,V_p)\,.
\end{multline*}

Однако можно провести оценивание и другим путем. Найдем точки
экстремума функции $\Phi(x-a)\hm-\Phi(x-a_i)$ из условия
$$
\varphi(x-a)-\varphi(x-a_i)=0\,.
$$

Максимум достигается в точке
$$
x_i^*=\fr{a+a_i}{2}\,.
$$

\pagebreak

\noindent
Тогда, учитывая четность функции $\varphi(x)$, получим:
\begin{multline*}
p\sum\limits_{i=1}^k
p_i\sup\limits_x|\Phi(x-a_i)-\Phi(x-a)|\leqslant{}\\
{}\leqslant
p\sup\limits_x\sum\limits_{i=1}^k
p_i|\Phi(x-a_i)-\Phi(x-a)|\leqslant{}\\
{}\leqslant p\sup\limits_x\sum\limits_{i=1}^k
p_i|\Phi(x_i^*-a_i)-\Phi(x_i^*-a)|
={}\\
{}= p\sum\limits_{i=1}^k
p_i(a_i-a)\varphi\left(\theta(x_i^*-a)+(1-\theta)(x_i^*-a_i)\right)\leqslant{}\\
{}\leqslant p\fr{a_k-\min\{0,a_0\}}{\sqrt{2\pi}}
=L(V,V_p)\fr{a_k-\min\{0,a_0\}}{\sqrt{2\pi}}\,.
\end{multline*}
%
Окончательно
\begin{multline*}
L(V,V_p)\geqslant\max\left\{1,\fr{\sqrt{2\pi}}{a_k-\min\{0,a_0\}}\right\}
L(G,G_p)={}\\
{}=C_1^{[1]}(a_0,a_k)L(G,G_p)\,.~~\square
\end{multline*}


\medskip

Рассмотрим следующее обобщение модели~\eqref{Stab_G_p1}. Пусть
имеется еще одна смесь данного типа, отличающаяся
от~\eqref{Stab_G_p1} только весом, т.\,е.\ (при этом $0\hm\leqslant q\hm\leqslant 1$)
\begin{equation}
\label{Stab_G_q1}
G_q(x)=(1-q)\sum\limits_{i=1}^k
p_i\Phi(x-a_i)+q\Phi(x-a)\,.
\end{equation}

Для $G_q(x)$ дискретная случайная величина $V_q$ имеет следующий
вид:
\begin{equation}
\label{Stab_V_q1} 
V_q:
\begin{tabular}{ccccc}
$a$&$a_1$&$a_2$&$\cdots$&$a_k$\\
$q$&$p_1(1-q)$&$p_2(1-q)$&$\cdots$&$p_k(1-q)$\,.
\end{tabular}
\end{equation}

Рассуждая как описано выше, получим, что $|p-q|\hm\leqslant a_1-a$. В~этом
случае расстояние Леви $L(V_p,V_q)$ примет вид:
\begin{equation}
\label{Stab_Cond1q}
L(V_p,V_q)=|p-q|\,.
\end{equation}

Тогда справедлива следующая теорема.

\medskip

\noindent
\textbf{Теорема~2.}
\textit{В рамках модели добавления
компоненты~\eqref{Stab_G_p1} при выполнении
условий~\eqref{Stab_a1} и~\eqref{Stab_Cond1q} расстояние
Леви $L(V_p,V_q)$ между смешивающими распределениями $V_p$ из
соотношения~\eqref{Stab_V_p1} и $V_q$ из
соотношения~\eqref{Stab_V_q1} и расстояние Леви $L(G_p,G_q)$
между распределениями $G_p(x)$ из соотношения~\eqref{Stab_G_p1} и
$G_q(x)$ из соотношения~\eqref{Stab_G_q1} связывают неравенства:}
\begin{multline*}
%\label{Stab_Ineqv1q}
C_1^{[1]}(a_0,a_k)L(G_p,G_q)\leqslant L(V_p,V_q)\leqslant{}\\
{}\leqslant
C_2^{[1]}(a_0,a_k)L^{1/2}(G_p,G_q)\,,
\end{multline*}
\textit{где коэффициенты $C_j^{[1]}(a_0,a_k)$, $j\hm=1,2$, зависящие
только от известных величин $a_k$ и $a_0$, определяются
формулами}~\eqref{Stab_C11} и~\eqref{Stab_C12}.

\medskip


\noindent
Д\,о\,к\,а\,з\,а\,т\,е\,л\,ь\,с\,т\,в\,о\,.\ 
Рассуждая аналогично доказательству теоремы~1,
найдем оценки снизу для равномерного расстояния между функциями
распределения $G_p(x)$ и~$G_q(x)$. Имеем:
\begin{multline*}
\rho(G_p,G_q)={}\\
{}=\sup\limits_x|(q-p)\sum\limits_{i=1}^k
p_i\Phi(x-a_i)+(p-q)\Phi(x-a)|={}\\
{}=|p-q|\sup\limits_x\left|\sum\limits_{i=1}^k
p_i\Phi(x-a_i)-\Phi(x-a)\right|\geqslant{}\\
{}\geqslant |p-q|\left|\sum\limits_{i=1}^k
p_i(\Phi(x-a_i)-\Phi(x-a))\right|\geqslant{}\\
{}\geqslant L^2(V_p,V_q)\varphi\left(a_k+|a_k|-\min\{0,a_0\}\right)\,.
\end{multline*}

Оценим максимум производной для функций~$G_p$ и~$G_q$. Запишем
выражения, например, для функции~$G_p$ (для функции~$G_q$ оценка
получается аналогично). Имеем:
\begin{multline*}
\max\limits_x G_p'(x)={}\\
{}=\max\limits_x \left((1-p)\sum\limits_{i=1}^k
p_i\varphi(x-a_i)+p\varphi(x-a)\right)\leqslant{}\\
{}\leqslant
\fr{1-p}{\sqrt{2\pi}}\sum\limits_{i=1}^k p_i+
\fr {p}{\sqrt{2\pi}}=\fr{1}{\sqrt{2\pi}}\,.
\end{multline*}

Пользуясь правым неравенством в формуле~\eqref{Stab_Lrho},
приходим к следующему результату:
\begin{multline*}
L(V_p,V_q)\leqslant\varphi^{-1/2}\left(a_k+|a_k|-\min\{0,a_0\}\right)\times{}\\
{}\times
\left(1+\fr{1}{\sqrt{2\pi}}\right)^{1/2}L^{1/2}(G_p,G_q)
={}\\
{}=C_2^{[1]}(a_0,a_k)L^{1/2}(G_p,G_q)\,.
\end{multline*}

Оценка снизу для $L(V_p,V_q)$ может быть найдена из следующих
соотношений:
\begin{multline*}
L(G_p,G_q)\leqslant\rho(G_p,G_q)={}\\
{}=|p-q|\sup\limits_x\left|\sum\limits_{i=1}^k
p_i\Phi(x-a_i)-\Phi(x-a)\right|\leqslant{}\\
{}\leqslant |p-q|\sup\limits_x\sum\limits_{i=1}^k
p_i|\Phi(x-a_i)-\Phi(x-a)|\leqslant{}
\end{multline*}

\noindent
\begin{multline*}
{}\leqslant |p-q|\sum\limits_{i=1}^k
p_i\sup\limits_x|\Phi(x-a_i)-\Phi(x-a)|
\leqslant{}\\
{}\leqslant |p-q|\sum\limits_{i=1}^k p_i= L(V_p,V_q)\,.
\end{multline*}
Аналогично доказательству теоремы~1 получим:
\begin{multline*}
\hspace*{-7.22379pt}L(V_p,V_q)\geqslant\max\left\{1,\fr{\sqrt{2\pi}}{a_k-\min\{0,a_0\}}\right\}
L(G_p,G_q)={}\\
{}=C_1^{[1]}(a_0,a_k)L(G_p,G_q)\,.~~\square
\end{multline*}


\section{Модель расщепления компоненты}

Модель расщепления компоненты формализуется следующим образом.
Предполагается, что каждое из независимых наблюдений
$\mathbf{X}_n\hm=(X_1,\ldots,X_n)$ имеет распределение,
представимое в виде:
\begin{multline}
\label{Stab_G_p2}
G_p(x)=\sum\limits_{i=1}^{k-1}
p_i\Phi(x-a_i)+{}\\
{}+(p_k-p)\Phi(x-a_k)+p\Phi(x-a)\,,
\end{multline}
где все величины $a_i\hm\in\R$, $0\hm\leqslant p_i\hm\leqslant 1$, $i\hm=1,\ldots,k,$ считаются
известными, $a$ и $p$ являются параметрами модели, при этом
$0\hm\leqslant p\hm\leqslant p_k$. Без ограничения общности для
определенности будем считать, что выполнены соотношения:
\begin{equation}
\label{Stab_sigma2}
a_1\leqslant a_2\leqslant\cdots\leqslant a_{k-1}\leqslant a\leqslant a_k\,.
\end{equation}

Для данной модели дискретная случайная величина $V_p$ имеет вид:
\begin{equation}
\label{Stab_V_p2} 
V_p:
\begin{tabular}{ccccc}
$a_1$&$a_2$&$\cdots$&$a$&$a_k$\\
$p_1$&$p_2$&$\cdots$&$p$&$p_k-p$\,.
\end{tabular}
\end{equation}

Воспользовавшись геометрической интерпретацией расстояния Леви,
можно получить, что
\begin{equation}
\label{Stab_L2} 
L(V,V_p)=\min\{a_k-a,p\}\,.
\end{equation}

В этой ситуации оба условия: $a\hm\to a_k$ при фиксированном
параметре~$p$ и $p\hm\to 0$ при фиксированном~$a$~--- влекут
справедливость соотношения $L(V,V_p)\hm\to 0$. Тогда справедлива
следующая тео\-рема.

\medskip

\noindent
\textbf{Теорема~3.}
\textit{В~рамках модели расщепления
компоненты~\eqref{Stab_G_p2} при выполнении
условий~\eqref{Stab_sigma2} расстояние Леви $L(V,V_p)$
из соотношения~\eqref{Stab_L2} между смешивающими
распределениями $V$ из соотношения~\eqref{Stab_V} и $V_p$ из
соотношения~\eqref{Stab_V_p2} и расстояние Леви $L(G,G_p)$ между
истинным распределением $G(x)$ из соотношения~\eqref{Stab_G} и
приближающей смесью $G_p(x)$ из соотношения~\eqref{Stab_G_p2}
связывают неравенства:}
\begin{multline*}
C_1^{[2]}(a_{k-1},a_k)L(G,G_p)\leqslant
L(V,V_p)\leqslant{}\\
{}\leqslant
 C_2^{[2]}(a_{k-1},a_k)L^{1/2}(G,G_p)\,,
\end{multline*}
\textit{где коэффициенты $C_j^{[2]}(a_{k-1},a_k)$, $j=1,2$, не зависят
от величин $a$, $p$ и имеют вид:}
\begin{align}
\label{Stab_C21}
C_1^{[2]}(a_{k-1},a_k)&=\fr{\sqrt{2\pi}}{\max\{1,a_k-a_{k-1}\}}\,,\\
\label{Stab_C22}
C_2^{[2]}(a_{k-1},a_k)&=\varphi^{-1/2}
\left(a_k+|a_k|-\min\{0,a_{k-1}\}\right)\times{}\notag\\
&\hspace*{20mm}{}\times \left(1+\fr{1}{\sqrt{2\pi}}\right)^{1/2}\,.
\end{align}


\medskip

\noindent
Д\,о\,к\,а\,з\,а\,т\,е\,л\,ь\,с\,т\,в\,о\,.\ 
Запишем оценки снизу для равномерного расстояния между функциями
распределения $G(x)$ и $G_p(x)$, воспользовавшись формулой
Лагранжа, свойством монотонного убывания плотности стандартного
нормального распределения $\varphi(x)$ от положительного аргумента и
соотношениями~\eqref{x0}, \eqref{Stab_sigma2} и~\eqref{Stab_L2}:
\begin{multline*}
\rho(G,G_p)=\sup\limits_x|G(x)-G_p(x)|={}\\
{}=\sup\limits_x\left|\sum\limits_{i=1}^k
p_i\Phi(x-a_i)-\sum\limits_{i=1}^k
p_i\Phi(x-a_i)+{}\right.\\
\left.{}+p\Phi(x-a_k)-p\Phi(x-a)
\vphantom{\sum\limits_{i=1}^k}\right|={}\\
{}=p\sup\limits_x|\Phi(x-a_k)-\Phi(x-a)|\geqslant{}\\
{}\geqslant
p|\Phi(x_0-a_k)-\Phi(x_0-a)|={}\\
{}=p|(a_k-a)\varphi(\theta(x_0-a_k)+(1-\theta)(x_0-a))|\geqslant{}\\
{}\geqslant p(a_k-a)\varphi\left(a_k+|a_k|-\min\{0,a_{k-1}\}\right)\geqslant{}\\
{}\geqslant
L^2(V,V_p)\varphi\left(a_k+|a_k|-\min\{0,a_{k-1}\}\right)\,.
\end{multline*}

Чтобы оценить сверху $L(V,V_p)$, воспользуемся правым
неравенством из соотношения~\eqref{Stab_Lrho} и найденной в
доказательстве теоремы~1 оценкой для
максимума производной, а также неравенствами~\eqref{Stab_sigma2}.
Имеем:
\begin{multline*}
L^2(V,V_p)\varphi\left(a_k+|a_k|-\min\{0,a_{k-1}\}\right)\leqslant{}\\
{}\leqslant
\left(1+\fr{1}{\sqrt{2\pi}}\right)L(G,G_p)\,,
\end{multline*}
%
откуда

\noindent
\begin{multline*}
L(V,V_p)\leqslant\varphi^{-1/2}
\left(a_k+|a_k|-\min\{0,a_{k-1}\}\right)\times{}\\
{}\times
\left(1+\fr{1}{\sqrt{2\pi}}\right)^{1/2}L^{1/2}(G,G_p)={}\\
{}=C_2^{[2]}(a_{k-1},a_k)L^{1/2}(G,G_p)\,.
\end{multline*}

Выпишем оценку снизу для $L(V,V_p)$. С этой целью заметим, что
\begin{multline}
\label{Inqv}
\hspace*{-5.16743pt}L(G,G_p)\leqslant\rho(G,G_p)=p\sup\limits_x|\Phi(x-a_k)-\Phi(x-a)|
={}\\
{}=p\sup\limits_x\left(\Phi(x-a)-\Phi(x-a_k)\right)\,.
\end{multline}


Найдем точки экстремума функции $\Phi(x-a)\hm-\Phi(x-a_k)$ из условия
$$
\varphi(x-a)-\varphi(x-a_k)=0\,.
$$

Максимум достигается в точке:
$$
x^*=\fr{a+a_k}{2}\,.
$$

Подставляя это значение в~\eqref{Inqv}, получим (учитывая четность
функции~$\varphi(x)$)
\begin{multline*}
p\sup\limits_x\left(\Phi(x-a)-\Phi(x-a_k)\right)
={}\\
{}=p\left(\Phi(x^*-a)-\Phi(x^*-a_k)\right)={}\\
{}=p(a_k-a)\varphi\left(\theta(x^*-a)+(1-\theta)(x^*-a_k)\right)={}\\
{}= p(a_k-a)\varphi\left((a_k-a)\left\vert\theta-\fr{1}{2}\right\vert\right)\leqslant{}\\
{}\leqslant L(V,V_p)\max\{p,a_k-a\}\fr{1}{\sqrt{2\pi}}
\leqslant{}\\
{}\leqslant L(V,V_p)\max\{1,a_k-a_{k-1}\}\fr{1}{\sqrt{2\pi}}\,.
\end{multline*}
%
Окончательно
\begin{multline*}
L(V,V_p)\geqslant\fr{\sqrt{2\pi}}{\max\{1,a_k-a_{k-1}\}}L(G,G_p)
={}\\
{}=C_1^{[2]}(a_{k-1},a_k)L(G,G_p)\,.~\square
\end{multline*}

\medskip

Рассмотрим следующее обобщение модели~\eqref{Stab_G_p2}. Пусть
имеется еще одна смесь данного типа, отличающаяся
от~\eqref{Stab_G_p2} только весом, т.\,е.\ (при этом $0\hm\leqslant q\hm\leqslant p_k$)
\begin{multline}
\label{Stab_G_q2}
G_q(x)=\sum\limits_{i=1}^{k-1}
p_i\Phi(x-a_i)+{}\\
{}+(p_k-q)\Phi(x-a_k)+q\Phi(x-a)\,.
\end{multline}

Для $G_q(x)$ дискретная случайная величина~$V_q$ имеет вид

\noindent
\begin{equation}
\label{Stab_V_q2} 
V_q:
\begin{tabular}{ccccc}
$a_1$&$a_2$&$\cdots$&$a$&$a_k$\\
$p_1$&$p_2$&$\cdots$&$q$&$p_k-q$
\end{tabular}\,.
\end{equation}

Воспользовавшись геометрической интерпретацией расстояния Леви,
можно получить, что
\begin{equation}
\label{Stab_L2q} 
L(V_p,V_q)=\min\{a_k-a,|p-q|\}\,.
\end{equation}

Тогда справедлива следующая теорема.

\medskip

\noindent
\textbf{Теорема~4.}
\textit{В~рамках модели расщепления
компоненты~\eqref{Stab_G_p2} при выполнении
условий~\eqref{Stab_sigma2} расстояние Леви
$L(V_p,V_q)$ из соотношения~\eqref{Stab_L2q} между смешивающими
распределениями $V_p$ из соотношения~\eqref{Stab_V_p2} и $V_q$ из
соотношения~\eqref{Stab_V_q2} и расстояние Леви $L(G_p,G_q)$
между распределениями $G_p(x)$ из соотношения~\eqref{Stab_G_p2} и
$G_q(x)$ из соотношения~\eqref{Stab_G_q2} связывают неравенства:}
\begin{multline*}
%\label{Stab_Ineqv2q}
C_1^{[2]}(a_{k-1},a_k)L(G_p,G_q)\leqslant
L(V_p,V_q)\leqslant{}\\
{}\leqslant
 C_2^{[2]}(a_{k-1},a_k)L^{1/2}(G_p,G_q)\,,
\end{multline*}
\textit{где коэффициенты $C_j^{[2]}(a_{k-1},a_k)$, $j=1,2$, не зависят от
величин $a$, $p$ и определяются формулами~\eqref{Stab_C21}
и}~\eqref{Stab_C22}.


\medskip

\noindent
Д\,о\,к\,а\,з\,а\,т\,е\,л\,ь\,с\,т\,в\,о\,.\ 
Рассуждая аналогично доказательству теоремы~3,
найдем оценки снизу для равномерного расстояния между функциями
распределения $G_p(x)$ и $G_q(x)$. Имеем:
\begin{multline*}
\rho(G_p,G_q)=\sup\limits_x|G_p(x)-G_q(x)|={}\\
{}=|p-q|\sup\limits_x|\Phi(x-a_k)-\Phi(x-a)|\geqslant{}\\
{}\geqslant
p|\Phi(x_0-a_k)-\Phi(x_0-a)|\geqslant{}\\
{}\geqslant L^2(V_p,V_q)\varphi\left(a_k+|a_k|-\min\{0,a_{k-1}\}\right)\,.
\end{multline*}

Оценим максимум производной для функций~$G_p$ и~$G_q$. Имеем:
\begin{multline*}
\max\limits_x G_p'(x)=\max\limits_x\left(\sum\limits_{i=1}^{k-1}
p_i\varphi(x-a_i)+{}\right.\\
\left.{}+(p_k-p)\varphi(x-a_k)+p\varphi(x-a)
\vphantom{\sum\limits_{i=1}^{k-1}}\right)\leqslant{}\\
{}\leqslant \fr{1}{\sqrt{2\pi}}\sum\limits_{i=1}^{k-1}
p_i+\fr{(p_k-p)}{\sqrt{2\pi}}+\fr{p}{\sqrt{2\pi}}
=\fr{1}{\sqrt{2\pi}}\,.
\end{multline*}

Пользуясь правым неравенством в формуле~\eqref{Stab_Lrho},
приходим к следующему результату:
\begin{multline*}
L(V_p,V_q)\leqslant\varphi^{-1/2}\left(a_k+|a_k|-\min\{0,a_{k-1}\}\right)\times{}\\
{}\times
\left(1+\fr{1}{\sqrt{2\pi}}\right)^{1/2}L^{1/2}(G_p,G_q)={}\\
{}=C_2^{[2]}(a_{k-1},a_k)L^{1/2}(G_p,G_q)\,.
\end{multline*}

Оценка снизу для $L(V_p,V_q)$ может быть найдена из следующих
соотношений:
\begin{multline*}
L(G_p,G_q)\leqslant\rho(G_p,G_q)={}\\
{}=|p-q|\sup\limits_x|
\Phi(x-a_k)-\Phi(x-a)|\,.
\end{multline*}

Повторяя рассуждения из доказательства
теоремы~3, получаем, что
\begin{multline*}
L(V_p,V_q)\geqslant\fr{\sqrt{2\pi}}{\max\{1,a_k-a_{k-1}\}}L(G,G_p)
={}\\
{}=C_1^{[2]}(a_{k-1},a_k)L(G_p,G_q)\,.~~\square
\end{multline*}


\section{Выводы}

В рамках двух рассмотренных моделей возмущений параметров смеси~---
моделей добавления и расщепления компоненты~--- получены оценки
устойчивости смесей нормальных законов по отношению к изменениям
смешивающего параметра. Для каждой из моделей получены
двусторонние оценки, связывающие расстояния Леви между смесями и
смешивающими законами. Данные оценки, в частности, являются
количественными характеристиками идентифицируемости конечных
сдвиговых смесей нормальных законов.

В то же время доказанные
теоремы~1--4 уста\-нав\-ли\-ва\-ют
взаимно однозначное соответствие между значением параметра веса и
числом компонент в смеси. Данный результат удобно использовать при
построении асимптотически наиболее мощных критериев для моделей
добавления и расщепления компоненты для случая произвольных
конечных сдвиг-мас\-штаб\-ных смесей в качестве обоснования вида гипотез
в задаче статистической проверки чис\-ла компонент смеси вероятностных
распределений (подробнее об этом см.\ в
работах~\cite{Gorshenin2011, Gorshenin2011MUCMC}).

{\small\frenchspacing
{%\baselineskip=10.8pt
\addcontentsline{toc}{section}{Литература}
\begin{thebibliography}{9}


\bibitem{Gorshenin2011} 
\Au{Бенинг~В.\,Е., Горшенин~А.\,К.,
Королев~В.\,Ю.}
Асимптотически оптимальный критерий проверки гипотез о числе
компонент смеси вероятностных распределений~// Информатика и её
применения, 2011. Т.~5. Вып.~3. C.~4--16.

\bibitem{Gorshenin2011Mod2}\Au{Горшенин~А.\,К.}
Проверка статистических гипотез в модели расщепления компоненты~//
Вестник Московского университета. Сер.~15. Вычисл. матем. и киберн.,
2011. №\,4. С.~26--32.

\bibitem{Gorshenin2012SSI} \Au{Горшенин~А.\,К.}
Устойчивость масштабных смесей нормальных законов по отношению к
изменениям смешивающего распределения~// Системы и средства
информатики, 2012. Т.~22. Вып.~1. С.~136--148.

\bibitem{Zolotarev1986} 
\Au{Золотарев~В.\,М.} Современная теория
суммирования независимых случайных величин.~--- М.: Наука, 1986. 417~с.

\label{end\stat}

\bibitem{Gorshenin2011MUCMC}\Au{Gorshenin~A.\,K.}
Testing of statistical hypotheses in the splitting component
model~// Moscow University Computational Mathematics and
Cybernetics, 2011. Vol.~35. No.\,4. P.~176--183.
 \end{thebibliography}
}
}


\end{multicols}  %3
\renewcommand{\figurename}{\protect\bf Figure}
\renewcommand{\tablename}{\protect\bf Table}

\def\stat{dulin}


\def\tit{INFORMATION FUSION OF~DOCUMENTS}

\def\titkol{Information fusion of~documents}

\def\autkol{S.\,K.~Dulin, N.\,G.~Dulina, and~P.\,V.~Ermakov}

\def\aut{ S.\,K.~Dulin$^1$, N.\,G.~Dulina$^2$, and~P.\,V.~Ermakov$^3$}

\titel{\tit}{\aut}{\autkol}{\titkol}



\renewcommand{\thefootnote}{\arabic{footnote}}
\footnotetext[1]{Institute of Informatics Problems, Federal Research Center ``Computer Science and Control'' 
of the Russian Academy of Sciences, 44-2~Vavilov Str., Moscow 119333, Russian Federation, 
skdulin@mail.ru}
\footnotetext[2]{A.\,A.~Dorodnicyn Computing Center, Federal Research Center ``Computer Science and 
Control'' of the Russian Academy of Sciences, 40~Vavilov Str., Moscow 119333, Russian Federation, 
ngdulina@mail.ru}
\footnotetext[3]{ TeleRetail GmbH, 30~\mbox{Markenstra{\!\ptb{\ss}}e}, 
D$\ddot{\mbox{u}}$sseldorf 40227,  Germany; petcazay@gmail.com}


\index{Dulin S.\,K.}
\index{Dulina N.\,G.}
\index{Ermakov P.\,V.}
\index{Дулин С.\,К.}
\index{Дулина Н.\,Г.}
\index{Ермаков П.\,В.}


\def\leftfootline{\small{\textbf{\thepage}
\hfill INFORMATIKA I EE PRIMENENIYA~--- INFORMATICS AND
APPLICATIONS\ \ \ 2020\ \ \ volume~14\ \ \ issue\ 1}
}%
 \def\rightfootline{\small{INFORMATIKA I EE PRIMENENIYA~---
INFORMATICS AND APPLICATIONS\ \ \ 2020\ \ \ volume~14\ \ \ issue\ 1
\hfill \textbf{\thepage}}}

%\vspace*{-2pt}



       \Abste{The paper considers the problems associated with the 
creation of an expert base of documents that require prompt 
processing of incoming information and, as a consequence, 
restructuring of the knowledge base. The authors propose procedures 
that reduce the search of the optimal consistent state of 
interrelated documents. An approach to assessing the relationship of 
text documents and informational messages as poorly structured 
objects was developed. The practical implementation of this approach 
is described.}
      
      \KWE{information fusion; controlled data and knowledge consistency; 
knowledge base restructuring}
      
\DOI{10.14357/19922264200117} 
      
      %\vspace*{8pt}
      
      
      \vskip 12pt plus 9pt minus 6pt
      
       \thispagestyle{myheadings}
      
       \begin{multicols}{2}
      
       \label{st\stat}
     
     \section{Introduction}
      
     \noindent
     Combining information of various origins for integrative analysis and 
processing has been called ``Information Fusion''[1], implying that the synthesized 
data carrying information combine type properties of source data and possess 
more information than merely conjunction of information sources considered 
separately. The main difficulty of the synthesis problem is that information sources 
contain heterogeneous data represented by various formats and structures and 
employed in different types of platforms.
     
     The main factors of data heterogeneity and their sources are: various types 
of data, diversity in data origin, various models of database representation, various 
data presentation formats, differentiating in the organization of data storage 
systems, differences in the degree of reliability and accuracy of data, and
variety of  a~degree and form of data structure.
     
     The process of information fusion is a~multilevel process that includes five 
basic stages~\cite{2-d, 3-d, 4-d}:
     \begin{itemize}
\item zero stage~--- the stage of combining sensor signals, designed to obtain 
data indicating semantically clear and interpretable attributes of objects and 
participating in the applications of the research being performed;
\item the first stage is aimed at processing data of the zero stage in order to 
make a decision on the classes of the objects in question and the states of these 
objects;
\item the second stage of Information Fusion, designed to assess the situation, 
including the zero and the first stages. It is used to assess the situational 
interaction of objects considered as a whole;
\item the third stage~--- the stage of evaluation of the interaction ``Impact 
Assessment,'' designed to perform an antagonistic assessment, based on the 
prediction of the situation;
\item the fourth stage~--- the stage of feedbacks, evaluating the possibility of 
using feedbacks in the system in question; and
\item the fifth stage~--- the final stage, the level of man--machine interaction, 
performing correctional actions of the operator for the sake of the system 
control.
\end{itemize}

     Research in the field of Information Fusion mainly focuses on the synthesis 
of data represented by digital images and arrays of data and  
documents~\cite{1-d, 4-d, 5-d}.
     
     Current trends in the development of corporative informational systems 
show that, along with traditional informational resources, the results of intelligent 
activity of experts and analysts become very important for the successful operation 
of large and middle-sized companies. A~unified informational environment of the 
company incorporates these formalized results in an accumulated form such that 
all executives can jointly use this resource in the context of their assignments. The 
role played by the knowledge accumulated in such a~way in the enterprise-wide 
systems allows us to consider this knowledge as very valuable and a~notably 
important resource for a~company, which, together with the traditional resources, 
such as financial, material, human, etc., characterizes the reliability of the 
company. The totality of this knowledge, presented mainly in text form, is the 
intelligent assets of the company, and the competitiveness of the company and its 
adaptability to changing the business environment depends on how efficiently this 
resource is used.
{\looseness=-1

}
     
     An intelligent asset is a~specific resource that requires specialized 
knowledge management systems. These systems enable the search, accumulation, 
and processing of knowledge by experts in solving various analytical problems. 
This tendency in knowledge engineering appeared relatively recently, but interest 
in the development and usage of such systems is permanently growing. This is 
largely due to the significant results achieved by some companies that have 
successfully implemented knowledge management systems into their 
manufacturing activity.
     
     Complex technological solutions designed to support various stages of 
composition and usage of corporative data and knowledge have been embodied in 
the knowledge management systems. At each of these stages, individual problems 
are solved, with the most important of them being associated with tasks related to 
searching, processing documents, and extracting knowledge from them.
     
     Text processing tasks are solved in practically all fields of human activity, 
and the analysis of the current environment is an integral part of practically 
each 
corporative management system securing a timely and adequate reaction to 
changes in the business environment. Actually, operativeness is the basic 
characteristics of monitoring problems, which distinguishes them from the problems 
related to prediction, planning, etc., because the main goal of the monitoring is the 
timely reaction of corresponding management subsystems of the general 
technological scheme of company functioning to changes of internal or external 
factors.
     
     In the general case, the purpose of text processing tasks is to accumulate 
necessary information from different sources, process it analytically, and, on this 
basis, generate corresponding decisions. The character of text processing tasks is 
permanent in the sense that the environment and the parameters of the company 
operation are subject to permanent changes, which requires regular (or periodic) 
sampling of ever changing information.
     
     Text processing tasks can conventionally be divided into two classes: internal 
monitoring and external monitoring.
     
     Internal monitoring is associated mainly with the monitoring of internal 
operation parameters, e.\,g., regular monitoring of the operation of complex installa-
tions, cargo moving, etc. Possible examples are control systems for energy plants, 
freight management, etc. The typical feature of these problems is a relatively 
constant set of parameters used to estimate the state of the process (production, 
physical parameters of an installation, etc.).
     
     In contrast to the internal monitoring, the external monitoring is mainly 
related to the estimation of the state of the environment and external conditions of 
the company operation. As an example, an analysis of consumer demand carried 
out by a commodity-producing company falls into this category. The typical 
feature of these problems is that, first, the parameters to be estimated are poorly 
formalized and, second, the set of these parameters is variable. The latter factor 
requires the restructuring of the analyst knowledge according to the changed 
conditions. All this makes us consider the ``restructurability'' of the expert 
knowledge base as one of the characteristic features of the problems of external 
monitoring.
     
     In the problems of external monitoring, special requirements must be 
imposed on the sources of information used by experts for the localization of 
required knowledge and data. The development of informational technologies 
during recent years has strongly suggested that the Internet is gradually becoming 
the most important source of information in solving analytical problems in 
practically all areas of human activity. Coming up to printed and electronic mass 
media, Internet is often ranked first in operativeness, which makes the Internet the 
most valuable information source in monitoring problems. It is for this reason that, 
in this work, special attention is paid to the solution of monitoring problems 
associated with search and processing of text information in Internet.
     
     \section{Approach to~Provision of~Knowledge Consistency}
     
     \noindent
     In previous works (see~\cite{4-d, 7-d, 6-d}), the authors put forward a procedure 
providing the consistency of the knowledge base dynamically formed by an 
expert, which is based on the analysis of structural interrelations between separate 
components of the knowledge base with subsequent restructuring of it aimed at 
reducing existing inconsistency. In so doing, the basic criterion of structural 
consistency was a concept of polyconsonance of power~$n$~\cite{2-d}.
     
     Consider a knowledge base formed on the basis of search and analysis of 
Internet information. In solving the monitoring problems associated with the 
formation of such a knowledge base, the application of this procedure faces certain 
difficulties resulting from poor formalization and an obscure or ambiguous 
structure of the data (text or multimedia documents). Besides, for the monitoring 
problems considered here, a large number of informational messages directed to the 
expert for analytical processing and replenishment of the knowledge base are 
characteristic. As a result, the amount of resources (especially, time) required for 
the restructuring of a dynamically changing knowledge base is increased 
significantly, which is, perhaps, the main obstacle to the successful practical 
implementation of any procedure of the above type.
     
     One of the major disadvantages of the algorithm proposed in~\cite{4-d} is 
that it is oriented to problems of the search type; that is why, the authors made 
special efforts to reduce the search and thus increase the algorithm efficiency in its 
practical implementation. The results presented below are aimed at the solution of 
the latter problem.
     
     Consider a set of mutually related objects $O = \{o_i\}$ with a similarity 
function~$f$~\cite{3-d} satisfying the condition
     $$
     0\leq f\left( o_i, o_j\right)\leq 1\,.
     $$
     
     Numbers $\alpha$ and~$\beta$ will denote the lower and upper similarity 
thresholds, respectively, satisfying the condition
     $$
     0\leq \alpha\leq \beta\leq 1\,.
     $$
     
    Now, let us introduce the concepts of a negative, positive, and indifferent link 
between two arbitrary elements~$o_i$ and~$o_j$ of the set~$O$. The link is called 
``negative'' if its value does not exceed the lower similarity threshold: $0\leq 
f(o_i,o_j)\leq \alpha$; it is called ``positive'' if the value of the similarity function is 
not less than the upper similarity threshold: $\beta\leq f(o_i,o_j)\leq 1$; and, if 
$\alpha<f<\beta$, it is called ``indifferent'' (zero).
     
     Consider a partition of the given set into a number of nonempty subsets 
$K_1,\ldots , K_n$.
     
     A link between two arbitrary elements~$o_i$ and~$o_j$ of the entire 
set~$O$ is called ``bad'' if one of the following conditions is satisfied:
     \begin{enumerate}[(1)] 
     \item the elements~$o_i$ and~$o_j$ belong to the same subset~$K_x$, and 
the link between them is negative; or
\item the elements~$o_i$ and~$o_j$ belong to different subsets~$K_1$ 
and~$K_2$, and the link between them is positive.
\end{enumerate}

     Using this definition, let us to each object~$o_k$ from the set 
considered   assign the number~$v_k$ of its bad links for a~given partition into subsets. 
Now, let us construct a~vector~$V$ consisting of these values (this vector has 
a~dimension equal to the number of objects in the set) and call it the nodewise 
difference vector (NDV)~\cite{4-d}. The sum of the elements of this vector is 
denoted by $S_{\mathrm{NDV}}$.
     
     Clearly, different partitions of the original set correspond to different NDVs 
and different values of $S_{\mathrm{NDV}}$. According to the algorithm considered, 
the main problem is to find a partition of the given set~$O$ such that the sum 
$S_{\mathrm{NDV}}$ 
takes its minimal value; i.\,e., the total number of bad links tends to zero.
     
     The algorithm~\cite{4-d} developed by the authors consists in 
successive transformations of the set of informational objects on the basis of the 
condition
     $$
     S_{\mathrm{NDV}} > \fr{n(N-n)}{2}
     $$
     where $S_{\mathrm{NDV}}$ is the sum of nodewise differences for the given 
set of~$n$ elements belonging to a pair of consonant subsets of the total 
cardinality~$N$.  If this condition is fulfilled, then the restructuring of the 
considered set results in a decrease of the total sum~$S_{\mathrm{NDV}}$.
     \smallskip
     
     \noindent
     \textbf{Theorem~1.} \textit{Let~$K_1$ and~$K_2$ be two subsets of 
a~given set of mutually related objects~$O$}:
     \begin{align*}
     K_1 &= \left\{ o_i\right\}\,,\ i=1,\ldots, n_1\,;\\
     K_2&= \left\{ o_j\right\}\,, \ j=1,\ldots , n_2\,.
     \end{align*}
     
     \textit{A set containing~$m$~elements from these two subsets satisfies the 
condition of the algorithm if, and only if, the set consisting of all remaining 
elements of these two subsets satisfies the same condition.}
     
     \smallskip
     
     \noindent
     P\,r\,o\,o\,f\,.\ \  First, let us prove the necessity. Let the set of 
objects~$\{o_k\}$, $k = 1,\ldots , m$, satisfy the condition of the algorithm:
     $$
     \sum v_k> \fr{m(n_1+n_2-m)}{2}
     $$
     where $v_k$ are the NDV values for the element with the number~$k$. 
This formula can be transformed to the form:
     $$
     \sum v_k > \fr{\left(n_1+n_2-m\right)
     \left(\left(n_1+n_2\right)-\left(n_1+n_2-m\right)\right)}{2}
     $$
     which means that the set of $n_1+n_2-m$ vectors not belonging to the 
original set also satisfies the condition of the algorithm.
     
     The sufficiency of the condition is proved similarly. The theorem is proved.
     
     \smallskip
     
     \noindent
     \textbf{Corollary.} In order to find a set of objects from two given subsets 
that satisfies the condition of the algorithm, it is sufficient to check the fulfillment 
of this condition only for the subsets consisting of $(n_1+n_2)/2$ objects. In other 
words, only subsets with cardinalities not exceeding half of the sum of the 
cardinalities of the original subsets~$K_1$ and~$K_2$ should be checked.
     \smallskip
     
     \noindent
     P\,r\,o\,o\,f\,.\ \ Indeed, if some set consisting of more than $(n_1+n_2)/2$ 
elements satisfies the condition, then the complement to it also satisfies this 
condition, with the cardinality of the complement being not greater than 
$(n_1+n_2)/2$.

\begin{figure*}[b] %fig1
\vspace*{1pt}
    \begin{center}  
  \mbox{%
 \epsfxsize=160.967mm 
 \epsfbox{dul-1.eps}
 }
\end{center}
\vspace*{-10pt}
\Caption{Determination of vocabulary groups}
\end{figure*}

     
     \smallskip
     
     \noindent
     \textbf{Theorem~2.}\  \textit{Let~$K_1$ and~$K_2$ be two subsets of 
a~given set of mutually related objects~$O$}:
     \begin{align*}
     K_1 &= \left\{o_i\right\}\,,\ i=1,\ldots , n_1\,;\\
     K_2&= \left\{o_j\right\}\,,\ i=1,\ldots , n_2\,.
     \end{align*}
     \textit{Let a set $\{o_k\}$ of $m < (n_1 + n_2)/2$ elements belonging to 
these two subsets satisfy the condition of the algorithm. If a zero NDV element 
corresponds to some element~$o_x$ from this set, then the set of the vectors 
corresponding $O^*=\{o_1, \ldots, o_{x-1}, o_{x+1}, \ldots, o_m\}$ also satisfies the 
condition of the algorithm.}
     
     \smallskip
     
     \noindent
     P\,r\,o\,o\,f\,.\ \ According to the assumption of the theorem, the sum 
$S^*_{\mathrm{NDV}}$ for the set $O^*=\{o_1, \ldots\linebreak
\ldots, o_{x-1}, o_{x+1}, \ldots, o_m\}$ is 
equal to the sum $S_{\mathrm{NDV}}$ of the original set of the elements from the two 
subsets~$K_1$ and~$K_2$:
     $$
S^*_{\mathrm{NDV}} = S_{\mathrm{NDV}}\,.
$$
     
     Denote by~$N$ the total cardinality of the considered subsets: $N = 
n_1+n_2$. Then,
     $$
     (m-1)(N-(m-1)) = m(N-m)+(2m-N-1)\,.
     $$
     
     According to the assumption of the theorem, $m \leq N/2$; hence, $2m-N-1 
< 0$. To complete the proof, let us write the following inequality:
     \begin{multline*}
     S^*_{\mathrm{NDV}} = S_{\mathrm{NDV}} = \sum v_k >\fr{m(N - 
m)}{2} >{}\\
{}> \fr{(m-1)(N - (m-1))}{ 2}
   \end{multline*}
     which means that the set $\{o_1, \ldots, o_{x-1}, o_{x+1}, \ldots, o_m\}$ satisfies 
the condition of the algorithm.
     
     Obviously enough, it follows from this theorem that, in the practical 
implementation of the proposed algorithm, it is sufficient to search for a set of 
elements for the next iteration among those with nonzero NDV values.
{\looseness=1

}
     
\section{Thematic Role of~Similarity}

     \noindent
     The most significant factor affecting the operation of the algorithm 
considered is the similarity function on the basis of which interrelations between 
different elements of a given set are determined. As far as the support of 
monitoring problems is considered, with the texts (in particular, news) and the 
Internet being the elements and the main information source, respectively, the 
construction of the similarity function becomes a fairly difficult problem. Perhaps, 
one of the solutions to this problem could be the use of various methods of 
linguistic analysis to determine the degree of ``likeness'' of two different 
documents, although these methods are not free from some shortcomings 
associated with the hardship of their implementation, adjustment, etc. To 
determine the similarity function in practical applications, the authors have put 
forward another approach. One of the advantages of this new approach is the 
simplicity of implementation and the ``notional transparency.''
     
     The basis of this approach schematically shown in Fig.~1 is the 
determination of vocabulary groups~\cite{7-d}, which denote the sets of keywords 
defined by the expert. The expert assorts the keywords according to 
some criterion, e.\,g., ``thematic meaning:''
     $$
     G_k= \left\{w_i\right\},\enskip i = 1,\ldots ,n_k.
     $$

\begin{figure*}[b] %fig2
\vspace*{1pt}
    \begin{center}  
  \mbox{%
 \epsfxsize=94.043mm 
 \epsfbox{dul-2.eps}
 }
\end{center}
\vspace*{-10pt}
\Caption{A general scheme of operation of iiProcessor system}
\end{figure*}

     Consider an arbitrary element~$o_j$ from a given set~$O$. This object is a 
text document; so, it can be represented as an aggregate of lexical units, i.\,e., 
words. For~$o_j$, let us define its coefficient of correspondence with the dictionary 
group~$G_i$ as the ratio $S(G_i)_j$ of the number of keywords specified in this 
dictionary group and available in the text of the information object itself, to the 
total number of keywords from all dictionary groups, $S(G)_j$ found in this text. 
Then, one can define the factor of correspondence of the object~$o_j$ to the 
vocabulary group~$G_i$ as
     $$
     L^i_j = \fr{S(G_i)_j}{S(G)_j}.
        $$
     
     On the basis of these coefficients, let us define the degree of thematic coupling 
between two arbitrary informational objects as follows:
     \begin{itemize}
     \item[(A)] $f(o_k, o_l) = 1$ if $ S(G)_k = 0$ and  $S(G)_l = 0$;
     \item[(B)] $f(o_k, o_l) = 0$ if  $S(G)_k\not= S(G)_l$ 
     and $S(G)_k S(G)_l\linebreak = 0$; and
     \item[(C)] $f(o_k, o_l) = \max\left( \min\left(L^i_k, L^i_l\right) \right)$, $i = 1, 
\ldots, n$,  for $S(G)_k  S(G)_l\not= 0$
     where $n$ is the number of the vocabulary groups.
     \end{itemize}

     
     Note that the similarity function defined above takes the values on the 
interval from~0 to~1 but lacks associativity, because $0 \leq f(o_i, o_j) \leq 1$. In 
the works devoted to the theoretical grounds of the considered algorithm of 
structural transformations of a set of objects, the associativity of the similarity 
function has not been used; therefore, the fact that the function introduced above is 
not associative does not require any changes in the proposed algorithm. Moreover, 
the lack of associativity here has an additional meaning, which makes it possible to 
treat the function introduced above as a~\textit{thematic} similarity function.
     
     Indeed, if, in the considered text, there are keywords from different 
vocabulary groups, then all the coefficients~$L^i_j$ for this element will be less 
than one. Hence, the value of the similarity function~$f$ will also be less than one, 
and the more the number of the vocabulary groups, the less this value. In practice, 
this could mean that the considered document is of a review nature and, most 
probably, has no distinct ``thematic meaning.''

\begin{figure*} %fig3
\vspace*{1pt}
    \begin{center}  
  \mbox{%
 \epsfxsize=156.872mm 
 \epsfbox{dul-3.eps}
 }
\end{center}
\vspace*{-10pt}
\Caption{Example of use of vocabulary group technique to establish
links between different documents}
\end{figure*}
     
\section{Consistency Controlling Module iiProcessor}

     \noindent
     The authors' technique for providing structural consistency of the knowledge 
base in solving monitoring problems has been implemented in a specialized system 
called an iiProcessor. This system is designed to compose expert knowledge bases 
for social, political, and international sciences. The knowledge bases are 
constructed from the information supplied by various mass media through their 
Internet servers. The main purpose of the system is to accumulate informational 
messages (news) related to the themes of user's interest from various Internet 
sources, to integrate the information into a unified knowledge base, to create links 
between different elements of the knowledge base, and to make subsequent 
restructuring of the knowledge base on the basis of these links, with the result of 
this restructuring being the representation of the body of the information 
accumulated as a logical system of classes. The latter system can be treated as an 
informational model of the problem examined by the expert (for example, the 
social and political situation in a particular region of the world). A~general scheme 
of operation of the system is shown in Fig.~2.



     As a source of information, this system uses the CNN Internet site ({\sf 
http://cnn.com}). Several times a day, this site publishes information covering many 
aspects of social and political life in many countries. In most cases, the 
informational messages are weakly-structured text documents. In order to establish 
links between different documents, the vocabulary group technique described 
above is used (Fig.~3). If various informational messages contain common 
keywords belonging to different vocabulary groups, this technique estimates the 
``likeness'' of the messages. The similarity function classifies these links as 
positive or negative, which makes it possible to construct a~connectivity matrix on 
the set of the informational messages received by the user (see Fig.~3).
     
    


     The mode of ``Keywords'' allows one to get~10 of the most significant key 
words for a~given document with an indication of their weighting factors (Fig.~4).
     
     
     The mode of interrelations (``Correlations'') will allow to get several 
documents that have the greatest interrelations with selected document. This mode 
works only if the loaded document belongs to the current project of the iiProcessor 
system, in which the relationship was evaluated (Fig.~5).
    
     The choice of the CNN server as a source of information is explained by the 
fact that this server is one of the most informationally abundant servers providing 
real-time information. Of course, the choice of the sources of information is 
strongly determined by the character of the problem considered. In this sense, the 
CNN server is not universal. In view of the above considerations, the Restructor 
system is implemented as a~complex of two program modules. The rsn.exe module 
is the basic one. An auxiliary iip.class module executes a real-time search for new 
information in a specified information source in the Internet. With such an 
architecture, this\linebreak\vspace*{-12pt}

{ \begin{center}  %fig4
 \vspace*{-7pt}
     \mbox{%
 \epsfxsize=79mm 
 \epsfbox{dul-4.eps}
 }


\vspace*{4pt}


\noindent
{{\figurename~4}\ \ \small{``Keywords'' mode}}
\end{center}
}

\vspace*{2pt}


{ \begin{center}  %fig5
 \vspace*{-1pt}
    \mbox{%
 \epsfxsize=79mm 
 \epsfbox{dul-5.eps}
 }


\vspace*{4pt}


\noindent
{{\figurename~5}\ \ \small{``Correlation'' mode}}
\end{center}
}

%\vspace*{3pt}



\noindent
 system can be adopted to operation with any informational 
servers in the Internet (and beyond) by replacing only the auxiliary module, 
without changing its kernel where the major mathematical results of the authors' 
approach are implemented.
     
\section{Concluding Remarks}

\noindent
The implementation of the results of Theorems~1 and~2 in the inference engine 
made it possible to considerably reduce the time expenses of the built-in algorithm 
for restructuring the database. The use of the connectivity matrix as the major 
visualization means for the informational objects improved the clearness of the 
representation of the information model of the problem considered by an expert. 
The system has been tested in analyzing the events related to NASA's 
aerospace research.
     
    % \Ack
    % \noindent
    % This work was supported by the Russian Foundation for Basic Research, 
%project No.\,20-07-00329~А.
     
     \renewcommand{\bibname}{\protect\rmfamily References}
     
     
     \vspace*{-9pt}
     
     {\small\frenchspacing
     {\baselineskip=10.45pt
     \begin{thebibliography}{99}
     
     \bibitem{1-d} %1
\Aue{Dasarathy, B.} 2001. Information fusion~--- what, where, why, when, and how? 
\textit{Inform. Fusion} 2(2):75--76.
     
     \bibitem{4-d} %2
\Aue{Dulin, S.\,K.} 1995. The approach to structural consistency of situations' models in 
an active knowledge base. \textit{Workshop of 10th IEEE Symposium 
(International) on Intelligent Control Proceedings}. Monterey, CA: AdRem, Inc. 
253--258.

\bibitem{3-d} %3
\Aue{Duckham, M., and M.~Worboys.} 2007. Automated geographic information 
fusion and ontology alignment. \textit{Spatial data on the Web}. Eds. A.~Belussi, 
B.~Catania, E.~Clementini, and E.~Ferrari.
Berlin: Springer. Ch.~6:109--132. 

\bibitem{2-d} %4
\Aue{Pravia, M.} 2008. Generation of a~fundamental data set for hard/soft information 
fusion. \textit{11th Conference (International) on Information Fusion Proceedings}. 
Cologne: International Society of Information Fusion. 134--145.





\bibitem{5-d} %5
\Aue{Landauer, T.\,K., K.~Kireyev, and C.~Panaccione.} 2011. Word maturity: A~new 
metric for word knowledge. \textit{Sci. Stud. Read.} 15(1):92--108. 

\bibitem{7-d} %6
\Aue{Dulina, N., and O.~Kozhunova.} 2010. Information monitoring system: 
A~problem 
of linguistic resources consistency and verification. \textit{Problems of Cybernetics and 
Informatics: 3rd Conference (International) Proceedings}. Baku.  
56--58.
\bibitem{6-d} %7
\Aue{Dulin, S.\,K., and  N.\,G.~Dulina.} 2018. Ispol'zovanie disseminatsionnykh 
algoritmov dlya formirovaniya nestrukturirovannoy tekstovoy informatsii v~baze 
geodannykh [Using dissemination algorithms for the formation of unstructured textual 
information in the geodatabase]. \textit{Sistemy i~Sredstva Informatiki~--- Systems and 
Means of Informatics} 28(2):42--59.

\end{thebibliography}}}

\end{multicols}

\vspace*{-6pt}

\hfill{\small\textit{Received February 26, 2019}}

\vspace*{-16pt}

\Contr

%\vspace*{-3pt}

\noindent
\textbf{Dulin Sergey K.} (b.\ 1950)~--- Doctor of Science in technology, 
professor, leading scientist, Institute of Informatics Problems, Federal Research 
Center ``Computer Science and Control'' of the Russian Academy of Sciences,  
44-2~Vavilov Str., Moscow 119333, Russian Federation; principal scientist, 
Research \& Design Institute for Information Technology, Signalling and 
Telecommunications on Railway Transport (JSC NIIAS), 27-1~Nizhegorodskaya 
Str., Moscow 109029, Russian Federation; \mbox{skdulin@mail.ru} 

\vspace*{3pt}

\noindent
\textbf{Dulina Natalia G.} (b.\ 1947)~--- Candidate of Science (PhD) in 
technology, leading programmer, A.\,A.~Dorodnicyn Computing Center, Federal 
Research Center ``Computer Science and Control'' of the Russian Academy of 
Sciences, 40~Vavilov Str., Moscow 119333, Russian Federation; 
\mbox{ngdulina@mail.ru}
\vspace*{3pt}

\noindent
\textbf{Ermakov Petr V.} (b.\ 1985)~--- Senior Software Developer, TeleRetail 
GmbH, 30~\mbox{Markenstra{\!\ptb{\ss}}e}, D$\ddot{\mbox{u}}$sseldorf 
40227,  Germany; \mbox{petcazay@gmail.com}

 

%\newpage

\vspace*{8pt}

\hrule

\vspace*{2pt}

\hrule

%\vspace*{-7pt}

%\newpage

%\vspace*{-28pt}

\def\tit{ИНФОРМАЦИОННЫЙ СИНТЕЗ ДОКУМЕНТОВ}

\def\titkol{Информационный синтез документов}

\def\aut{С.\,К.~Дулин$^1$, Н.\,Г.~Дулина$^2$, П.\,В.~Ермаков$^3$}

\def\autkol{С.\,К.~Дулин, Н.\,Г.~Дулина, П.\,В.~Ермаков}

%{\renewcommand{\thefootnote}{\fnsymbol{footnote}} \footnotetext[1]
%{Работа was supported by the Russian Foundation for Basic Research, project No.\,20-07-00329~А.}}



\titel{\tit}{\aut}{\autkol}{\titkol}

\vspace*{-11pt}

\noindent
$^1$Институт проблем информатики Федерального исследовательского центра <<Информатика 
и~управление>>\linebreak
$\hphantom{^1}$Российской академии наук, \mbox{skdulin@mail.ru}

\noindent
$^2$Вычислительный центр им.\ А.\,А.~Дородницына Федерального исследовательского центра 
<<Информатика\linebreak
$\hphantom{^1}$и~управление>> Российской академии наук, \mbox{ngdulina@mail.ru}

\noindent
$^3$TeleRetail GmbH, D$\ddot{\mbox{u}}$sseldorf, Germany

\vspace*{1pt}

\def\leftfootline{\small{\textbf{\thepage}
\hfill ИНФОРМАТИКА И ЕЁ ПРИМЕНЕНИЯ\ \ \ том\ 14\ \ \ выпуск\ 1\ \ \ 2020}
}%
 \def\rightfootline{\small{ИНФОРМАТИКА И ЕЁ ПРИМЕНЕНИЯ\ \ \ том\ 14\ \ \ 
выпуск\ 1\ \ \ 2020
\hfill \textbf{\thepage}}}

\vspace*{-1pt}



\Abst{Рассматриваются проблемы, связанные с созданием экспертной 
базы документов, требующей оперативной обработки поступающей 
информации и, как следствие, реструктуризации базы знаний. 
Предложены процедуры, уменьшающие время поиска оптимального 
согласованного состояния взаимосвязанных документов. Был 
разработан подход к~оценке взаимосвязи текстовых документов 
и~информационных сообщений как плохо структурированных 
объектов. Описана практическая реализация этого подхода.}

\KW{информационный синтез; контролируемая согласованность 
данных и~знаний; реструктуризация базы знаний}


\DOI{10.14357/19922264200117} 

%\vspace*{-3pt}


 \begin{multicols}{2}

\renewcommand{\bibname}{\protect\rmfamily Литература}
%\renewcommand{\bibname}{\large\protect\rm References}

{\small\frenchspacing
{\baselineskip=10.5pt
\begin{thebibliography}{99}
%\vspace*{-3pt} 

\bibitem{1-d-1} %1
\Au{Dasarathy B.} Information fusion~--- what, where, why, when, and how?~// 
Inform. Fusion, 2001. Vol.~2. Iss.~2. P.~75--76.

\bibitem{4-d-1} %2
\Au{Dulin S.\,K.} The approach to structural consistency of situations' models in an 
active knowledge base~// Workshop of 10th IEEE Symposium 
(International) on Intelligent Control Proceedings.~--- Monterey, CA, USA: AdRem, 
Inc., 1995. P.~253--258.

\bibitem{3-d-1} %3
\Au{Duckham M., Worboys~M.} Automated geographic information fusion and 
ontology alignment~// Spatial data on the Web~/ Eds. A.~Belussi, B.~Catania, 
E.~Clementini, E.~Ferrari.~--- Berlin: Springer, 2007. Ch.~6. P.~109--132. 

\bibitem{2-d-1} %4
\Au{Pravia M.} Generation of a fundamental data set for hard/soft information 
fusion~// 11th Conference (International) on Information Fusion.~--- Cologne: 
International Society of Information Fusion, 2008. P.~134--145.




\bibitem{5-d-1} %5
\Au{Landauer T.\,K., Kireyev~K., Panaccione~C.} Word maturity: A~new metric 
for word knowledge~// Sci. Stud. Read., 2011. Vol.~15. Iss.~1. 
P.~92--108. 

\bibitem{7-d-1} %6
\Au{Dulina N., Kozhunova~O.} Information monitoring system: A~problem of 
linguistic resources consistency and verification~// Problems of Cybernetics and 
Informatics: 3rd Conference (International) Proceedings.~--- Baku, 2010.  
P.~56--58.
\bibitem{6-d-1} %7
\Au{Дулин С.\,К., Дулина~Н.\,Г.} Использование диссеминационных 
алгоритмов для формирования неструктурированной текстовой информации 
в базе геоданных~// Системы и средства информатики, 2018. Т.~28. №\,2. 
С.~42--59. 

\end{thebibliography}
} }

\end{multicols}

 \label{end\stat}

 \vspace*{-9pt}

\hfill{\small\textit{Поступила в~редакцию 26.02.2019}}


%\renewcommand{\bibname}{\protect\rm Литература}
\renewcommand{\figurename}{\protect\bf Рис.}
\renewcommand{\tablename}{\protect\bf Таблица}  %4

\def\stat{zhevn}

\def\tit{МЕТОДИКА МОДЕЛИРОВАНИЯ НАГРУЗКИ НА СЕРВЕР В~ОТКРЫТЫХ 
СИСТЕМАХ ОБЛАЧНЫХ ВЫЧИСЛЕНИЙ}

\def\titkol{Методика моделирования нагрузки на сервер в~открытых 
системах облачных вычислений}

\def\autkol{Д.\,В.~Жевнерчук, А.\,В.~Николаев}
\def\aut{Д.\,В.~Жевнерчук$^1$, А.\,В.~Николаев$^2$}

\titel{\tit}{\aut}{\autkol}{\titkol}

%{\renewcommand{\thefootnote}{\fnsymbol{footnote}}\footnotetext[1]
%{Работа выполнена при поддержке РФФИ (гранты 09-07-12098, 09-07-00212-а и
%09-07-00211-а) и Минобрнауки РФ (контракт №\,07.514.11.4001).}}


\renewcommand{\thefootnote}{\arabic{footnote}}
\footnotetext[1]{Чайковский технологический институт (филиал) Ижевского государственного технического университета, 
drevnigeck@yandex.ru}
\footnotetext[2]{Чайковский технологический институт (филиал) Ижевского государственного технического университета, 
elodssa@yandex.ru}

  \Abst{Планирование серверного ресурса систем облачных вычислений является сложной 
задачей. Необходимо учитывать такие факторы, как состав и параметры аппаратной 
платформы, параметры системного программного обеспечения, управляющего выполнением 
прикладных программ, свойства трафика, порождаемого пользователями и определяющего 
режимы функционирования прикладных программ. Предложена методика оценки загрузки 
серверного ресурса открытых систем облачных вычислений (ОСОВ) на основе анализа процессов 
взаимодействия пользователей с программным обеспечением. Обоснована ее достоверность. 
Приведены результаты моделирования нагрузки на серверную часть системы управления 
средами имитационного моделирования.}
  
  \KW{облачные вычисления; имитационное моделирование; человеко-машинное 
взаимодействие} 

   
   \vskip 14pt plus 9pt minus 6pt

      \thispagestyle{headings}

      \begin{multicols}{2}

            \label{st\stat}
  
\section{Введение}
  
  Вопросы проектирования систем поддержки удаленного вычислительного 
эксперимента до \mbox{конца} не решены. В~ходе проектирования ОСОВ 
возникает задача моделирования потока запросов, 
приводящих к загрузке серверной части. Такие модели позволяют получить 
оценку аппаратного ресурса для обслуживания некоторого количества клиентских 
систем при рабочей и пиковой нагрузке. В~работах~[1--3] 
рассматриваются теоретические модели трафика в локальных и глобальных сетях. 
В~основном полученные результаты имеют практическую значимость при 
проектировании средств передачи данных в сети. В~работах~[3--6] 
построены модели трафика, поступающего на вход серверов разного типа, 
таких как веб-серверы, серверы баз данных и~др. На основании обзора работ 
были сделаны следующие выводы:
\begin{enumerate}[1.]
\item Модели описывают трафик систем, построенных на основе определенных 
технологий и/или предназначенных для решения ограниченного круга задач.
\item Процессы формирования запросов моделируются на основании замеров уже 
переданного в сеть трафика.
\item Модели описывают смешанный трафик. 
\end{enumerate}

  Для анализа трафика ОСОВ классические 
методики моделирования трафика не эффективны, поскольку
  \begin{itemize} %[1)]
\item в общем случае ОСОВ обладает свойствами расширения по произвольным 
программно-ап\-па\-рат\-ным платформам, по решаемым задачам, по источникам 
нагрузки; 
\item в ОСОВ постоянно происходят качественные изменения, поэтому на 
основании конечного числа измерений трафика можно построить модели, 
описывающие ОСОВ только в некотором подмножестве состояний;
\item отсутствуют развитые средства автоматизации и механизмы контроля 
перехода ОСОВ в новое качественное состояние.
\end{itemize}

  Таким образом, задача разработки эффективных методик оценки нагрузки 
ОСОВ до конца не решена и является актуальной. 

\begin{figure*}[b] %fig1
 \vspace*{1pt}
 \begin{center}
 \mbox{%
 \epsfxsize=164.097mm
 \epsfbox{zhe-1.eps}
 }
 \end{center}
 \vspace*{-9pt}
\Caption{Методика моделирования нагрузки на сервер}
\end{figure*}
  
\section{Постановка задачи}
  
  Была поставлена задача разработки методики моделирования и построения на 
ее основе моделей нагрузки на серверную часть в ОСОВ. 
К~методике и моделям предъявлены следующие требования:
  \begin{enumerate}[1.]
\item Модели должны описывать дифференци\-ро\-ванный трафик, из которого 
можно выделить\linebreak
 потоки, принадлежащие определенному программному 
обеспечению и связанные с определенными задачами.
\item Источник дифференцированного трафика должен определяться 
процессами человеко-ма\-шин\-но\-го взаимодействия.
\item В модель должны передаваться эмпирические функции распределения 
вероятностей интервалов времени между передачами управляющих сигналов 
серверу, приводящих к существенной загрузке центрального процессора и 
оперативной памяти.
\item Должен проводиться системный анализ процессов решения 
пользовательских задач с применением программного обеспечения и учетом поведения 
пользователя при решении задач с по\-мощью программ.
\item Должна обеспечиваться высокая степень автоматизации процессов сбора 
эмпирических данных.
\end{enumerate}

  Ставилась задача применить представленную методику для изучения системы 
обработки сред имитационного моделирования и проверить достоверность 
построенных моделей нагрузки на сервер.
  
\section{Методика моделирования нагрузки на~сервер}
  
  Предлагаемая методика моделирования нагрузки на сервер включает ряд 
этапов:
  \begin{enumerate}[1.]
\item Сбор сведений о процессе взаимодействия клиента и сервера.
\item Определение последовательности выполнения действий.
\item Построение имитационной модели процесса взаимодействия.
\item Проведение экспериментов с имитационной моделью процесса 
взаимодействия и необходимой настройки.
\item Адаптацию полученного генератора нагрузки к работе с внешней 
системой.
\end{enumerate}
  
  Схема методики представлена на рис.~1.
  
  \begin{table*}[b]\small
\begin{center}
\Caption{Наблюдение за процессом изучения GPSS World Student}
\vspace*{2ex}

\begin{tabular}{|l|l|c|c|} 
\hline
\multicolumn{1}{|c|}{\raisebox{-6pt}[0pt][0pt]{Действие}}&
\multicolumn{1}{c|}{\raisebox{-6pt}[0pt][0pt]{Тип моделей}} & \multicolumn{2}{c|}{Категория}\\
\cline{3-4}
&&Успевающие&Неуспевающие\\
\hline
Количество ошибок компиляции в режиме
&Простые модели&[0--3]&[2--6]\\
 отладки  (1~задание)&Сложные модели&\hphantom{9}[4--12]&\hphantom{9}[8--16]\\
\hline
\multicolumn{2}{|l|}{Поиск ошибки и ее устранение, с}&\hphantom{9}[30--120]&\hphantom{9}[90--200]\\
\hline
&Простые модели&[20--60]&\hphantom{9}[40--120]\\
Анализ итогового отчета, с&Сложные модели&\hphantom{9}[20--120]&\hphantom{9}[90--240]\\
&Первичное ознакомление&\hphantom{9}[60--120]&\hphantom{9}[90--120]\\
\hline
\multicolumn{2}{|l|}{Кодирование новой модели, мин}&[120--300]& [240--420]\\
\multicolumn{2}{|l|}{(подготовка первого варианта кода модели), мин}&[360--720]&\hphantom{9}[600--1080]\\
\hline
\multicolumn{2}{|l|}{
\tabcolsep=0pt\begin{tabular}{l}Работа со средой моделирование по инструкции\\
(время поиска функциональности)\end{tabular}} &[20--40]&[30--90]\\
\hline
\end{tabular}
\end{center}
\end{table*}


  На первом этапе необходимо собрать сведения о взаимодействии клиента и 
сервера. Клиент работает с сервером в режиме за\-прос--от\-вет. Необходимо 
получить данные об интервалах времени между определенными действиями 
клиента. Для упрощения сбора данных было разработано клиент-серверное 
приложение <<Хронометр>>, позволяющее настроить список действий 
пользователя, требующих отметки времени выполнения. После настройки 
клиентская часть <<Хронометра>> начинает замерять интервалы времени, в 
течение которых выполняются действия, и передавать собираемые данные на 
сервер. Такой подход упрощает сбор сведений, так как данные можно собирать 
сразу с группы пользователей.
  
  На втором этапе определяется последовательность действий клиента по 
отношению к серверу. Для этого выполняются следующие шаги. На основе 
собранных данных строятся эмпирические функции распределения вероятностей 
интервалов времени, затраченного на действия клиента. Строятся сценарии 
  че\-ло\-ве\-ко-ма\-шин\-но\-го взаимодействия и вводится классификация 
пользователей, использующих определенные сценарии. Для упрощения работы 
было разработано программное средство <<Редактор сценариев>>, позволяющее 
быстро обработать список действий пользователя и увидеть максимальное, 
минимальное и среднее время выполнения действий. Кроме того, с его помощью 
можно построить эмпирические функции распределения интервалов времени в 
синтаксисе языка {GPSS} (General Purpose Simulation System).
  
  На третьем этапе строится имитационная модель процесса взаимодействия 
клиентов с сервером для оценки времени между событиями прихода запросов от 
клиента, приводящих к существенной загрузке центрального процессора и 
оперативной памяти. Для автоматизации построения имитационной модели было 
разработано программное средство <<Генератор имитационной модели>> (ГИМ). 
С~его помощью можно быстро получить код модели на основании вводимых 
параметров, эмпирических функций и сценариев действий. Программное средство ГИМ использует 
заготовленные шаблоны на языке имитационного моделирования {GPSS}.
  
  Далее проводится эксперимент с имитационной моделью, определение 
необходимых параметров и установка настроек. После этого строится модель 
трафика, поступающего на сервер, включающего запросы от клиента, приводящие 
к существенной загрузке центрального процессора и оперативной памяти. Все 
настройки и текст модели сохраняются в базу данных для последующего 
воспроизведения потока запросов.
  
  На последнем этапе полученную модель генерации запросов пользователя 
адаптируют для работы с внешней системой. 

\section{Ход исследования}
  
  Исследования были проведены для среды моделирования GPSS World 
Student, с которой пользователи работают в режиме обучения. В~построенных 
моделях трафика учитывалась информация о запросах, приводящих к прогонам 
имитационных моделей, что влияет на загрузку аппаратного ресурса. Была 
проверена гипотеза о достоверности построенных моделей. 

\subsection{Сбор данных о~процессе взаимодействия учащегося со~средой 
GPSS World Student}
  
  Был проведен хронометраж действий учащихся по изучению среды 
моделирования с помощью учебных моделей, по кодированию и отладке модели. 
Для формирования журнала действий пользователя использовалось программное 
средство <<Хронометр~1.0>>. 


  Было замечено, что в ходе выполнения учебного задания учащийся сначала 
формирует код модели, далее модель тестируется и отлаживается. Это 
сопровождается определенным числом попыток компиляции модели и 
интервалами времени поиска и устранения ошибок в коде. После отладки 
выполняется разовый контрольный прогон модели, после чего анализируется 
выходной отчет, на основании которого осуществляется поиск логических 
ошибок. Учащимся может быть проведено несколько дополнительных 
исправлений кода. Модель вновь\linebreak\vspace*{-12pt}
\pagebreak

\end{multicols}

\begin{figure} %fig2
 \vspace*{1pt}
 \begin{center}
 \mbox{%
 \epsfxsize=158.423mm
 \epsfbox{zhe-2.eps}
 }
 \end{center}
 \vspace*{-9pt}
\Caption{Число шагов отладки простой~(\textit{а}) и сложной~(\textit{б}) модели успевающим 
(верхний ряд) и  неуспевающим учащимся (нижний ряд)}
\vspace*{6pt}
\end{figure}

  \begin{figure} %fig3
   \vspace*{1pt}
 \begin{center}
 \mbox{%
 \epsfxsize=165.332mm
 \epsfbox{zhe-3.eps}
 }
 \end{center}
 \vspace*{-9pt}
  \Caption{Время написания первого варианта кода простой~(\textit{а}) и сложной~(\textit{б}) 
модели успевающим (верхний ряд) и неуспевающим учащимся (нижний ряд)}
  \end{figure}

\begin{figure} %fig4
 \vspace*{1pt}
 \begin{center}
 \mbox{%
 \epsfxsize=162.78mm
 \epsfbox{zhe-4.eps}
 }
 \end{center}
 \vspace*{-9pt}
\Caption{Время поиска ошибок компиляции успевающим~(\textit{а}) и 
неуспевающим~(\textit{б}) учащимся}
\end{figure}
  

\begin{multicols}{2}

\noindent
 тестируется, отлаживается, и результаты 
итогового прогона снова анализируются. 
  
  В режиме изучения готовой модели или среды моделирования с 
использованием методических указаний основное время тратится на анализ 
инструкций.


  
  Наблюдения проводились за тремя учебными группами общей численностью 
43~чел. Было проведено 5~занятий (10~академических часов). На основе 
полученных данных построена классификация учащихся и задач по времени 
решения. Все учащиеся разделены на две группы: <<успевающие>> и 
<<неуспевающие>>, а задачи~--- на группы <<простые>> и <<сложные>>. 
Полученные граничные оценки интервалов времени действий учащегося в 
режимах обучения и выполнения задания приведены в табл.~1. 
  
  С помощью программы <<Редактор сценариев>> были построены 
необходимые эмпирические законы распределения интервалов времени 
  (рис.~2--5).


  Полученные результаты были использованы при построении имитационных 
моделей взаимодействия учащихся со средой GPSS World.
  
\subsection{Модель взаимодействия учащегося со~средой GPSS World 
Student}

\vspace*{6pt}
  
  Выделим события, приводящие к компиляции и прогону модели, а 
следовательно, и к загрузке центрального процессора. При возникновении 
события первого запуска модели ($e_1$) происходит компиляция, в результате 
которой формируется отчет о готовности прогона. При возникновении события 
<<запуск модели>> ($e_2$) выполняется прогон, в результате которого 
формируется отчет с откликом. В~модели вводится 2~класса задач: простые и 
сложные,~--- и 2~класса учащихся: успевающие и неуспевающие. В~зависимости 
от класса задачи и типа учащегося выбираются построенные в результате 
хронометража функции, определяющие количество ошибок компиляции и время 
задержки при написании кода модели, анализа ошибок компиляции, анализа 
итогового отчета. При построении модели были сделаны следующие допущения:

\noindent
  \begin{enumerate}[1.]
\item Время компиляции модели пренебрежимо мало по сравнению со 
временем прогона модели, а\linebreak
\end{enumerate}



\end{multicols}

\begin{figure} %fig5
 \vspace*{1pt}
 \begin{center}
 \mbox{%
 \epsfxsize=163.628mm
 \epsfbox{zhe-5.eps}
 }
 \end{center}
 \vspace*{-9pt}
\Caption{Анализ отчетов простой~(\textit{а}) и сложной~(\textit{б}) модели успевающим 
(верхний ряд) и  неуспевающим  учащимся (нижний ряд)}
%\vspace*{6pt}
\end{figure}


\vspace*{-24pt}

\begin{multicols}{2}

\noindent
\begin{itemize}
\item[\ ] также с периодами поиска, устранения ошибок и 
анализа отчета и составляет менее 0,1\%.
\item[2.] Время прогона одной учебной модели варьируется в интервале [0,3--4]~c 
в зависимости от алгоритмических свойств модели и от аппаратного ресурса, 
что также пренебрежимо мало в случае однопользовательского режима. 
  \end{itemize}

\section{Сравнение отклика имитационной модели с~реальной 
системой}
  
  Для проведения экспериментальных исследований была разработана система 
моделирования работы комплекса виртуальных лабораторий Open Virtual 
Research Space ({OVRS}), которая представляет собой кли\-ент-сер\-вер\-ное 
приложение для исследования процессов функционирования открытого 
виртуального исследовательского пространства (ОВИП)~\cite{7zh, 8zh}. 
  
  Для оценки достоверности полученной модели был проведен хронометраж 
запросов на запуск имитационного эксперимента в среде {GPSS} тремя 
группами учащихся (табл.~2). 

\vspace*{12pt}

\noindent
{\small
%\begin{center}
{{\tablename~2}\ \ \small{Хронометраж запросов на запуск имитационного эксперимента}}

%\parbox{226pt}{\Caption{Хронометраж запросов на запуск имитационного эксперимента}

\vspace*{-6pt}

\begin{center}
\tabcolsep=6.4pt
\begin{tabular}{|c|c|c|c|c|}
\hline
№&
\multicolumn{2}{c|}{Учащиеся}&\multicolumn{2}{c|}{Задачи}\\
\cline{2-5}
\tabcolsep=0pt\begin{tabular}{c}груп-\\ пы\end{tabular}&
\tabcolsep=0pt\begin{tabular}{c}Успева-\\ ющие\end{tabular}&
\tabcolsep=0pt\begin{tabular}{c}Неуспе-\\ вающие\end{tabular}&Сложные&Простые\\
\hline
1&\hphantom{9}8&12&5&3\\
2&10&\hphantom{9}0&6&2\\
3&\hphantom{9}0&12&4&1\\
\hline
\end{tabular}
\end{center}

}

\vspace*{6pt}
  
  На рис.~6 приведены гистограммы пиковой загрузки реального и модельного 
сервера запросами на запуск имитационной модели. 
  
  Построены доверительные интервалы Велча для среднего значения числа 
заявок, поступающих в  интервалы времени, равные 60~с: ($-0{,}21$,\,0,94), ($-
0{,}83,\,1{,}07$); ($-1{,}17,\,1{,}35$).





  
  \begin{figure*} %fig6
   \vspace*{1pt}
 \begin{center}
 \mbox{%
 \epsfxsize=165.266mm
 \epsfbox{zhe-6.eps}
 }
 \end{center}
 \vspace*{-9pt}
  \Caption{Пиковая загрузка модельного (слева) и реального (справа) сервера по табл.~2: 
(\textit{а})~для группы~1; (\textit{б})~для группы~2; (\textit{в})~для группы~3}
  \end{figure*}
  
  Таким образом, достоверность построенной модели подтверждается наличием 
нуля в каждом интервале.
  
\section{Заключение}
  
  В ходе исследования была разработана методика оценки нагрузки на серверную 
часть ОСОВ,\linebreak особенностью которой является 
моделирование\linebreak потока запросов на основе системного анализа поведения групп 
пользователей при решении определенных задач с применением программного 
обеспечения. 
  
  С использованием методики были построены имитационные модели и 
экспериментальная среда исследования взаимодействия пользователя с сис\-те\-мой 
имитационного моделирования в режиме обучения. 
  
  Построены доверительные интервалы Велча для среднего значения числа 
запросов, поступающих на реальный сервер и его модель в интервалы времени, 
равные 60~с: ($-0{,}21$,\,0,94), ($-0{,}83,\,1,07$); ($-1{,}17,\,1,35$), что 
подтверждает достоверность построенной модели.
  
  Использование методов системного анализа для исследования поведения групп 
пользователей при решении определенных задач с применением средств 
автоматизации позволяет адаптировать предложенную методику для изучения 
серверной нагрузки в ОСОВ.
  
  Методика может быть применена при построении моделей процессов 
взаимодействия пользователей с произвольным программным обеспечением, 
доступ к которому предоставлен системами облачных вычислений. При этом 
модели будут отражать свойства потока запросов к серверной части, приводящих 
к загрузке ресурсов.
  
  Перспективным направлением развития данного исследования является теория 
проектирования систем \textit{cloud computing} и, в частности, вопросы 
зависимости пиковых загрузок от поведения пользователей.

{\small\frenchspacing
{%\baselineskip=10.8pt
\addcontentsline{toc}{section}{Литература}
\begin{thebibliography}{9}

\bibitem{2zh} %1
\Au{Willinger W., Taqqu M.\,S., Erramilli~A.}
A~Bibliographical guide to self-similar traffic and performance modeling for 
modern high-speed networks~// Stochastic networks: Theory and applications.~--- 
Oxford University Press, 1996. P.~282--296.

\bibitem{1zh} %2
\Au{Столлингс В.} Современные компьютерные сети.~--- СПб.: Питер, 2003. 
782~с.

\bibitem{3zh} %3
\Au{Petroff V.} Self-similar network traffic: From chaos and fractals to forecasting 
and QoS~// NEW2AN.~--- St.\ Petersburg, 2004. P.~110--118.

\bibitem{5zh} %4
\Au{Шелухин О.\,И.,  Тенякшев А.\,М.,  Осин А.\,В.}
Фрактальные процессы в телекоммуникациях.~--- М.: Радиотехника, 2003. 
480~с.

\bibitem{4zh} %5
\Au{Dang T.\,D., Sonkoly~B., Molnar~S.}
Fractal analysis and modelling of VoIP traffic~// 11th  Telecommunications Network 
Strategy and Planning Symposium (International) (NETWORKS 2004) 
Proceedings.~--- Vienna, Austria, 2004. P.~123--130.

\bibitem{6zh} %6
\Au{Криштофович А.\,Ю.}
Применение модели трафика сети ОКС №\,7 для управления потоками 
сигнальной нагрузки~// Инфокоммуникационные технологии, 2004. Т.~2. №\,2. 
С.~25--27.

\bibitem{7zh}
\Au{Жевнерчук Д.\,В., Николаев~А.\,В.}
Открытый инструмент проведения дистанционного имитационного 
эксперимента~// Вестник ИжГТУ.~--- Ижевск: ИжГТУ, 2008. №\,2. 
С.~103--108.

\label{end\stat}

\bibitem{8zh}
\Au{Ефимов И.\,Н., Жевнерчук Д.\,В., Николаев~А.\,В.}
Открытые виртуальные исследовательские пространства. Аналитический 
обзор.~--- Екатеринбург: Институт экономики УрО РАН, 2008. 83~с.
 \end{thebibliography}
}
}


\end{multicols} %5

\def\stat{kalenov}

\def\tit{ПРОБЛЕМЫ СЕТЕВОГО ДОСТУПА К НАУЧНЫМ ЖУРНАЛАМ}

\def\titkol{Проблемы сетевого доступа к научным журналам}

\def\autkol{А.\,В.~Глушановский, Н.\,Е.~Калёнов}

\def\aut{А.\,В.~Глушановский$^1$, Н.\,Е.~Калёнов$^2$}

\titel{\tit}{\aut}{\autkol}{\titkol}

%{\renewcommand{\thefootnote}{\fnsymbol{footnote}}\footnotetext[1] {Статья 
%рекомендована к публикации в журнале Программным комитетом конференции 
%<<Электронные библиотеки: перспективные методы и технологии, электронные 
%коллекции>> (RCDL-2012).}}

\renewcommand{\thefootnote}{\arabic{footnote}}
\footnotetext[1]{Библиотека по естественным наукам Российской академии наук, 
avglush@benran.ru} 
\footnotetext[2]{Библиотека по естественным наукам 
Российской академии наук, nek@benran.ru}



\Abst{Рассматриваются проблемы организации сетевого доступа российских ученых к 
научным журналам и базам данных. В соответствии с мировой практикой организацию 
такого доступа осуществляют научные биб\-лио\-те\-ки, объединяющиеся в 
консорциумы для получения выгодных финансовых условий. Описывается существующая 
в России практика организации доступа к зарубежным научным ресурсам через 
посредство Российского
фонда фундаментальных исследований (РФФИ) и <<Национального электронно-информационного 
консорциума>> (НЭИКОН). Приведена статистика востребованности пользователями 
Российской академии наук (РАН) научных журналов, предоставляемых через НЭИКОН. 
Предложены организационные действия для решения задачи оптимизации доступа к 
коммерческим сетевым научным ресурсам в условиях существующих в РАН финансовых 
ограничений.}


\KW{научные журналы; информация; Интернет; удаленный доступ; библиотеки; 
консорциум}

\vskip 14pt plus 9pt minus 6pt

      \thispagestyle{headings}

      \begin{multicols}{2}

            \label{st\stat}

     Анализ информационных потребностей ученых РАН показывает, что 
по-прежнему одним из важнейших источников научной информации для них 
остаются научные журналы (в первую очередь~--- иностранные). 
В~настоящее время наряду с традиционной печатной формой все более 
широкое распространение получил доступ к научным журналам через сеть 
Интернет.
     
     Технически такой доступ не представляет затруднений, что создает 
впечатление легкой доступности полных текстов статей. На самом деле все 
обстоит несколько сложнее. Большинство ведущих зарубежных издательств 
и научных обществ, таких как Elsevier, Springer, American Physical Society, 
American Chemical Society и других, представляют в свободном доступе 
только биб\-лио\-гра\-фи\-че\-скую информацию (описание статей) и (в лучшем 
случае) их рефераты. Доступ к полным текстам является платным и требует 
заключения соответствующего договора с издательством, причем суммы 
таких договоров, во-пер\-вых, весьма значительны, а во-вто\-рых, весьма 
заметно варьируются в зависимости от числа пользователей в организации, 
количества подключаемых компьютеров и ряда других па\-ра\-мет\-ров.
     
     Организация сетевого доступа к коммерческим\linebreak
      источникам научной 
информации требует значительной по объему и сложности специфической\linebreak 
работы, связанной с выбором нужных ресурсов, проведением переговоров с 
поставщиками, согласованием условий предоставления ресурсов, 
заключением контрактов и оформлением лицензионных соглашений, 
предоставлением IP-ад\-ре\-сов и контролем выполнения договорных 
обязательств. Для научных сотрудников такая деятельность не является 
характерной, поэтому сложившаяся мировая практика организации сетевого 
доступа к научной информации состоит в том, что ею занимаются 
биб\-лио\-те\-ки университетов, научных центров и других научных и учебных 
организаций. Биб\-лио\-те\-ки в силу специфики своей деятельности лучше 
знакомы с издательским миром, имеют опыт взаимодействия как с 
издательствами, так и с пользователями информации, и работа по 
информационному обеспечению научных исследований является их прямой 
обязанностью 
     
     Подобная практика сложилась и в России, в частности в РАН. 
Центральные академические биб\-лио\-те\-ки (такие как Библиотека Российской 
академии наук (БАН) в Санкт-Пе\-тер\-бур\-ге, Биб\-лио\-те\-ка по естественным наукам 
РАН (БЕН РАН) в Москве, Государственная публичная научнотехническая биб\-лио\-те\-ка Сибирского 
отделения РАН (\mbox{ГПНТБ} СО РАН)
 в Новосибирске, Центральная научная биб\-лио\-те\-ка 
Уральского отделения РАН (ЦНБ УрО РАН) в Екатеринбурге, Центральная научная 
биб\-лио\-тека Дальневос\-точ\-ного отделения РАН (ЦНБ ДвО РАН) во Вла\-ди\-востоке), 
обеспечивающие информационные потребности многих институтов РАН, 
тематика исследований которых в значительной мере пересекается, могут 
получить\linebreak значительно более выгодные условия доступа к \mbox{научным} журналам 
и базам данных (БД), нежели отдельные институты, заключающие 
самостоятельные договора с поставщиками. Кроме того, биб\-лио\-те\-ки (и/или 
их объединения~--- консорциумы) берут на себя организационные вопросы 
(переговоры с издательствами, заключение и оплата договоров, оформление 
лицензионных соглашений, сбор IP-адресов и организацию их подключения 
и~т.\,д.). Библиотеки также ведут анализ фактического использования 
доступа и оптимизируют подписку (в условиях жестких финансовых 
ограничений) для своих систем в целом.
 %    
     Как принято в мировой практике, для оптимизации финансовых 
условий доступа биб\-лио\-те\-ки объединяются в консорциумы, выступающие 
как единое юридическое лицо в отношениях с из\-да\-ющи\-ми организациями 
(или поставщиками ресурсов).
     
     Обычно в консорциумы объединяются биб\-лио\-те\-ки исходя из двух 
положений~--- либо предоставить узкотематическую информацию как можно 
более широкому кругу пользователей (объединяются организации с 
близкими научными интересами) и получить скидки (в расчете на одного 
участника консорциума) за счет значительного числа пользователей данного 
ресурса, либо предоставить пользователям консорциума как можно более 
широкий спектр информационных ресурсов (объединяются организации с 
различными тематическими интересами) и получить скидки (в расчете на 
одного участника консорциума) за счет увеличения объема предоставляемых 
ресурсов.
     
     В мире существует значительное число различных биб\-лио\-теч\-ных 
консорциумов. Международное объединение биб\-лио\-теч\-ных консорциумов 
(The International Coalition of Library Consortia~--- ICOLC)~[1] объединяет 
более 200~биб\-лио\-теч\-ных консорциумов, созданных на основе коалиций 
биб\-лио\-тек по тематическому или территориальному принципу. 
Консорциумы бывают разной величины и типа. Например, в Финляндии 
практически все университетские биб\-лио\-те\-ки, биб\-лио\-те\-ки научных 
учреждений и пуб\-лич\-ные биб\-лио\-те\-ки объединены в FinELib~[2]~--- 
национальный консорциум, по\-став\-ля\-ющий более 70\% всей электронной 
информации~[3]. Различные типы европейских биб\-лио\-теч\-ных консорциумов 
описаны в~[4].

В России в 1990--2000-е~гг.\ сложилась аналогичная практика 
организации доступа к научным журналам~[5]. С~1997~г.\ такой доступ 
предоставлялся в рамках консорциума, созданного по инициативе БЕН РАН 
и включавшего \mbox{РФФИ} и 
14~крупнейших научных биб\-лио\-тек. Финансирование консорциума 
осуществлял \mbox{РФФИ} в рамках принятой в конце 1996~г.\ <<Программы 
поддержки российских научных биб\-лио\-тек>>. В~рамках этой программы 
была создана научная электронная биб\-лио\-те\-ка (НЭБ). В~соответствии с 
принципами ее организации электронные версии журналов поступали из 
издательств в \mbox{РФФИ} и загружались на специальный сервер НЭБ и его 
зеркала в Казани и Новосибирске. Доступ предоставлялся всем 
пользователям биб\-лио\-тек, входящих в консорциум, а поскольку в 
консорциум входили все центральные академические биб\-лио\-те\-ки (БАН, БЕН 
РАН, \mbox{ГПНТБ} СО РАН, ЦНБ УрО РАН и ЦНБ ДвО РАН), любой сотрудник 
Академии наук мог читать основные научные журналы мира. К~началу 
2002~г.\ на серверы НЭБ было загружено около 2000~наименований (около 
75\,000~выпусков) журналов наиболее значимых научных издательств 
мира~[6]. Научная электронная библиотека пользовалась большой популярностью у специалистов~--- за 
год в начале \mbox{2000-х}~гг.\ из нее выгружалось около четверти миллиона 
статей. 
     
     Соглашение о консорциуме НЭБ, подписанное РФФИ и ведущими 
биб\-лио\-те\-ка\-ми, сопровождалось рядом условий, выдвинутых издательствами 
и направленных, в частности, на сохранение перечня приобретаемых 
биб\-лио\-те\-ка\-ми печатных версий журналов (по условиям участия в 
консорциуме организация должна была выписать для себя в печатном виде 
не менее 5~журналов, не входящих в подписку консорциума~[7]). Имелся (и 
сохраняется до сих пор во всех подобного рода консорциумах) ряд 
ограничений на выгрузку и распространение полученных текстов 
(запрещается сплошное копирование номера журнала, распространение 
полученных материалов за пределами ор\-га\-ни\-за\-ции-участ\-ника).
     
     К сожалению, в 2004~г.\ НЭБ РФФИ прекратила свое существование в 
том виде, который преду\-смат\-ри\-вал\-ся соглашениями 1996~г. Причинами 
этого стали несколько факторов, в частности проверка РФФИ со стороны 
Счетной палаты. Проверка вы\-яви\-ла нарушения Устава РФФИ, согласно 
которому последний не имеет права финансировать что-ли\-бо без 
проведения конкурсов. Это по\-влек\-ло за собой проблемы финансирования 
поддержки технологии функционирования НЭБ (обработка и загрузка 
массивов данных, поддержка серверов). 

С~вступлением в силу 94-го 
Федерального закона о закупках фактически были ликвидированы 
механизмы координированной работы биб\-лио\-тек по приобретению научных 
ресурсов. Из-за распада консорциума наиболее значимые научные 
издательства отказались передавать журналы российской стороне. 
В~результате уже загруженные на сервер журналы НЭБ были юридически 
переданы ООО <<Научная электронная биб\-лио\-те\-ка>> с условием 
бесплатного предоставления на ее сервере ({\sf http://www.elibrary.ru}), чем в 
настоящее время могут пользоваться российские ученые.
     
     Российский фонд фундаментальных исследований заключил новые договора с рядом зарубежных издательств о 
доступе к их журналам, но уже в режиме онлайн и только для своих 
грантодержателей.
     
     С этого периода и по настоящее время в России существуют два 
основных централизованных канала сетевого доступа учреждений РАН к 
зарубежной\linebreak научной информации~--- за счет \mbox{РФФИ} при посредстве 
Внешнеэкономического
объединения <<Академинторг>> и за счет средств Мин\-обр\-на\-у\-ки при посредстве 
\mbox{НЭИКОН}. 
Кроме централизованных\linebreak источников подписки в масштабах страны доступ 
к зарубежным научным журналам и БД приобретают 
вышеперечисленные центральные биб\-лио\-те\-ки РАН за счет средств, 
выделяемых Президиумом РАН и руководством ее региональных отделений, 
а также некоторые академические институты за счет своих средств. Однако 
количество ресурсов, приобретаемых академическими организациями, в 
десятки раз меньше количества ресурсов, приобретаемых РФФИ и НЭИКОН. 
     
     В настоящее время РФФИ финансирует своим грантодержателям (на 
уровне организаций, через которые осуществляется оплата средств по 
грантам) доступ к журналам шести издательств: Wiley (1600~журналов), The 
American Mathematical Society (предоставляется реферативная БД MathSciNet
(MSN), включающая около двух миллионов описаний статей), American 
Physical Society (9~журналов), Institute of Physics~(49 журналов), The Royal 
Society of Chemistry (6~журналов), Elsevier (Freedom Collection~--- около 
1700~журналов). До 2011~г.\ предо\-став\-лял\-ся также доступ к журналам 
издательства Springer, но в 2012~г.\ \mbox{РФФИ} отказался от этой подписки, 
мотивируя это решение нехваткой финансовых средств (одновременно было 
сокращено число доступных журналов The Royal Society of Chemistry с 23 
до~6). С~2011~г.\ грантодержателям \mbox{РФФИ} стали доступны журналы одной 
из коллекций издательства Elsevier (Freedom Collection~--- более 
1700~журналов).
     
     За счет средств РФФИ также организован доступ пяти крупнейших 
академических биб\-лио\-тек к известной БД Web of Knowledge, которая широко 
используется для определения публикационной активности и уровня 
цитирования научных публикаций.
     
     Следует заметить, что, лишившись в 2012~г.\ доступа к текущим 
журналам издательства Springer, пользователи РФФИ лишились и доступа к 
журналам предыдущих лет издания, подписка на которые была ранее 
оплачена. Согласно условиям контракта, при прекращении подписки для 
доступа к ранее оплаченным журналам каждая организация должна 
заплатить поставщику определенную сумму в качестве компенсации затрат 
на поддержку его серверов. 
     
     Как указывалось выше, каждый поставщик в зависимости от суммы 
контракта формулирует свои условия предоставления доступа к своим 
ресурсам. В~частности, ограничивает число пользователей, IP-ад\-ре\-сов или 
количество доступных журналов. Это обусловливает ограничения для 
грантодержателей \mbox{РФФИ} в получении доступа к сетевым ресурсам. Каж\-дый 
грантодержатель в начале 2012~г.\ должен был выбрать из предложенного 
\mbox{РФФИ} списка от одного до четырех издательств, журналы которых ему 
необходимы, и сообщить о своем выборе \mbox{РФФИ}. Последний, в зависимости 
от возможностей, диктуемых контрактами, принимал окончательное 
решение, кому и какие ресурсы предоставить. 
     
     <<Национальный электронно-информационный консорциум>>, 
     включающий в свой состав несколько сот организаций 
науки и образования, предо\-став\-ля\-ет им в 2012~г.\ за счет средств 
Министерства образования и науки доступ к полным текстам журналов 
следующих издательств, представляющих интерес для РАН: American 
Chemical Society (ACS~--- 38~журналов), American Institute of Physics 
     (AIP~--- 10~журналов), Annual Reviews Sciences Collection (AR~--- 
37~журналов), Business Source Complete (BSC~--- около 3500~журналов), 
Computers \& Applied Sciences Complete (CASC~--- около 950~журналов), 
Nature Publishing Group (NPG~--- 8~журналов), Oxford University Press 
(OUP~--- более 200~журналов), Optical Society of America (OSA~--- 
14~журналов), Sage STM (Science, Technology \& Medicine~--- более 
100~журналов), SPIE~--- International Society for Optics and Photonics 
(6~журналов и материалы конференций), Taylor \& Francis (T\&F~--- более 
1000~журналов), Georg Thieme Verlag KG (Thieme~--- 5~журналов) The 
American Association for the Advancement of Science (AAAS~--- журнал 
Science).
     
     Для сравнения~--- БЕН РАН на средства, выделенные ей Президиумом 
РАН в рамках целевого финансирования на приобретение научной 
литературы в 2011~г., смогла приобрести права сетевого доступа на 2012~г.\ 
лишь к 142~наименованиям журналов, отсутствующих в списках \mbox{РФФИ} и 
\mbox{НЭИКОН} (по соглашениям с поставщиками доступ предоставляется не 
только из центрального здания БЕН РАН, но и из ее отделов, расположенных 
в научных учреждениях РАН).
     
<<Национальный электронно-информационный консорциум>>, 
работая по контракту с Минобрнауки, уделяет серьезное 
внимание анализу использования ресурсов, предоставляемых научным 
организациям. Эту работу, касающуюся академических учреждений, с 
2010~г.\ по договору с НЭИКОН проводит БЕН РАН. В~ходе проводимого 
анализа был получен ряд интересных предварительных (работы 
заканчиваются в 2013~г.)\  результатов, которые приведены ниже.



 \begin{figure*}[b]
     \vspace*{1pt}
 \begin{center}
 \mbox{%
 \epsfxsize=99mm
 \epsfbox{glu-1.eps}
 }
 \end{center}
 \vspace*{-6pt}
\begin{center}
{\small Распределение числа выгрузок по издательствам}
\end{center}
     \end{figure*}

     
     По 14 издательствам, используемым в учреждениях РАН в 2010~г., в 
среднем в месяц выгружалось\linebreak\vspace*{-12pt}

\pagebreak

\begin{center}
 \vspace*{-6pt}
{{\tablename~1}\ \ \small{Активность использования ресурсов}}

\vspace*{6pt}

      \begin{tabular}{|l|c|c|}
      \hline
\multicolumn{1}{|c|}{Ресурс}&\tabcolsep=0pt\begin{tabular}{c}Количество\\ журналов\end{tabular} &
\tabcolsep=0pt\begin{tabular}{c}Количество\\ выгрузок\\ в месяц\end{tabular}\\
\hline
ACS &\hphantom{9}38&23\,806\hphantom{9}\\
AIP &\hphantom{9}10 &12\,930\hphantom{9}\\
NPG & \hphantom{99}8 &6378\\
T\&F &1547\hphantom{9}&3236\\
AAAS (Science) &\hphantom{99}1 &3015\\
OSA &\hphantom{9}14&2505\\
OUP\_Full &217&2106\\
Thieme &\hphantom{99}5&2035\\
SPIE &\hphantom{99}6&1568\\
Cell &\hphantom{9}15 &1286\\
Annual Review &\hphantom{9}37&\hphantom{9}402\\
SAGE &382&\hphantom{9}240\\
ACM &420&\hphantom{99}91\\
BSC &3345\hphantom{9}&\hphantom{99}42\\
\hline
\end{tabular}
\end{center}

%\pagebreak

\vspace*{12pt}

   


\addtocounter{table}{1}
\setcounter{figure}{0}

\noindent
 59\,642~статьи. Ресурсы использовались 
186~организациями РАН (журнал Science~--- 111~организаций, журналы 
AIP~--- 106~организаций, журналы ACS~--- 
82~организации, журналы группы Nature (NPG~--- Nature Publishing Group)~--- 
70~организаций и~т.\,д.). В целом, в 2010~г.\ активность учреж\-де\-ний РАН, 
измеряемая числом выгрузок полных текстов статей в месяц, выглядит 
следующим образом (табл.~1).
     

     
     По данным табл.~1 представлен график (см.\ рисунок).
     
    
     Данный график имеет две точки перегиба (после NPG 
и после группы журналов Cell). Суммарное число выгрузок до первой точки 
перегиба составляет 72\% от общего количества статей, выгруженных РАН, а 
до второй~--- 97\%.
     
      Наибольшим спросом у ученых РАН пользуются журналы ACS, AIP и NPG. 
Наименьшим спросом~--- журналы издательств Business Source Complete, 
Association for Computing Machinery (ACM), Sage и Annual Review. В средней 
части таблицы~--- пользующиеся, тем не менее, заметным спросом журналы 
издательств T\&F, AAAS (Science), OSA, 
OUP, Thieme, SPIE и Cell. 

\begin{table*}[b]\small
\vspace*{-6pt}
\begin{center}
\Caption{Использование ресурсов ОНИТ РАН}
\vspace*{2ex}

\begin{tabular}{|l|c|}
\hline
\multicolumn{1}{|c|}{Ресурс}&Число выгрузок
в месяц\\
\hline
American Institute of Physics (AIP)&413\hphantom{9}\\
Optical Society of America (OSA)&331\hphantom{9}\\
Society of Photographic Instrumentation Engineers (SPIE) &94\\
American Chemical Society (ACS)&42\\
AAAS (Science)&36\\
Nature&31\\
Sage &18\\
Nature Physics &17\\
Nature Nanotechnology&13\\
Association for Computing Machinery (ACM)&10\\
Nature Photonics &\hphantom{9}8\\
Nature Materials&\hphantom{9}7\\
Nature Chemistry &\hphantom{9999}0,25\\
Nature Methods &\hphantom{9999}0,17\\
Business Source Complete &\hphantom{9}0\\
Cell&\hphantom{9}0\\
Nature Biotechnology &\hphantom{9}0\\
Oxford University Press. Mathematics \& Computing &\hphantom{9}0\\
Oxford University Press BioMed &\hphantom{9}0\\
Oxford University Press Life &\hphantom{9}0\\
Oxford University Press Med &\hphantom{9}0\\
Oxford University Press STM &\hphantom{9}0\\
Taylor \& Francis Bio &\hphantom{9}0\\
Taylor \& Francis Chem &\hphantom{9}0\\
Taylor \& Francis Earth &\hphantom{9}0\\
Taylor \& Francis. Natural Sciences &\hphantom{9}0\\
Taylor \& Francis Med &\hphantom{9}0\\
Taylor \& Francis. Other &\hphantom{9}0\\
Taylor \& Francis. Physics \& Mathematics &\hphantom{9}0\\
Taylor \& Francis. Technique&\hphantom{9}0\\
\hline
\end{tabular}
\end{center}
\end{table*}
     
     До настоящего времени доступ к журналам, выписываемым через 
НЭИКОН, предоставлялся бесплатно. Со второй половины 2012~г.\ 
НЭИКОН планирует (по указанию Минобрнауки) взимать часть стоимости 
ресурсов с получателей. В~этой ситуации станет актуальным вопрос, не 
дешевле ли будет вместо оплаты доступа к базам малоспрашиваемых 
журналов заказывать электронные копии отдельных статей с 
<<постатейной>> оплатой. Этот вид сервиса достаточно хорошо развит за 
рубежом, им пользуются как отдельные ученые, так и научные биб\-лио\-те\-ки 
по заказам своих пользователей. Как правило, к оглавлениям и аннотациям 
статей из научных журналов предоставляется свободный доступ. Функции 
<<посредника>>, осуществляющего прием заказов на статьи от ученых РАН, 
контакты с поставщиками, оплату заказов в валюте могли бы взять на себя 
центральные академические биб\-лио\-те\-ки. По данным БЕН РАН, стоимость 
электронной копии статьи объемом до 20~страниц в биб\-лио\-те\-ках 
континентальной Европы составляет в среднем около 10~евро, что при 
годовых объемах 50--60~статей будет существенно дешевле, чем оплата 
доступа ко всем журналам издательства, не являющегося приоритетным для 
РАН. Очевидно, что такой подход потребует некоторого перераспределения 
средств и организационной перестройки биб\-лио\-теч\-ных служб, но он может 
оказаться достаточно эффективным.
     
     <<Национальный электронно-информационный консорциум>> достаточно оперативно реагирует на изменения 
спроса на журналы. Так, по \mbox{результа\-там} проведенного анализа с 2011~г.\ 
была прекращена подписка на журналы ACM. 
Что касается журналов издательств Business Source Complete, они 
пользуются значительным спросом у второй большой группы 
     ор\-га\-ни\-за\-ций--поль\-зо\-ва\-те\-лей \mbox{НЭИКОН}: российских университетов. 
Издательство Sage, которое также оказалось в нижней части рейтинговой 
таб\-ли\-цы, в 2010~г.\ было пред\-став\-ле\-но в \mbox{НЭИКОН} только журналами по 
гуманитарным и социальным наукам, хотя выпускает оно и другую научную 
литературу. В~настоящее время НЭИКОН предполагает дополнить этот 
ресурс журналами группы Sage STM (Science, Technology \& Medicine), и, 
возможно, результаты его востребованности академическими организациями 
изменятся.
     
     Получив полные данные о спросе на журналы НЭИКОН, авторы статьи 
провели анализ их использования сотрудниками отделений РАН. В~табл.~2 
представлены показатели Отделения нанотехнологий и информационных 
технологий (ОНИТ). 


     
     В табл.~2 более подробно, чем в предыдущей, раскрыты журналы 
группы Nature, а также журналы издательств Taylor \& Francis и Oxford 
University Press, поэтому таблица формально включает 30~ресурсов, но 
фактически это те же 14~ресурсов, раскрытых более подробно.
     
     Как видно из табл.~2, наибольший интерес для сотрудников ОНИТ 
     представляют журналы  AIP и OSA. Далее со 
значительным отрывом следуют\linebreak журна\-лы Society of Photographic 
Instrumentation Engineers. Среднюю группу (30--40~обращений к полным 
текстам в месяц) составляют журналы American Chemical Society, AAAS 
(Science) и основной журнал группы Nature. Значительно меньшим спросом 
пользуются остальные журналы группы Nature, журналы же остальных 
издательств не представляют для ОНИТ никакого интереса. 
     
     Россиский фонд фундаментальных исследований, в отличие от \mbox{НЭИКОН}, не 
     пред\-остав\-ля\-ет пользователям 
статистики использования пред\-став\-ля\-емых Фондом ресурсов (хотя БЕН РАН 
\mbox{обращалась} по этому поводу к руководству \mbox{РФФИ} и получила 
принципиальное согласие, но данные пока не получила), поэтому 
аналогичный анализ по этим ресурсам пока невозможен. Однако, по 
наблюдениям авторов статьи, журналы, предлагаемые через РФФИ, 
пользовались также весьма заметным спросом. Так, статьи из журналов 
издательства Springer в 2011~г.\ только в БЕН РАН выгружались в среднем 
158~раз в месяц, в ГПНТБ СО РАН~--- 489~раз; статьи издательства Elsevier 
в 2011~г.\ выгружались в БЕН РАН в среднем 1536~раз в месяц.
     
     Таким образом, в настоящее время в России созда\-на действующая 
система доступа к полным текстам нескольких тысяч зарубежных научных 
журналов. Эта система охватывает подавляющее большинство научных 
организаций РАН и пользуется значительной популярностью. Однако она не 
охватывает (в первую очередь в силу недостаточного финансирования) весь 
необходимый объем научной информации, требуемый для эффективного 
функционирования научных институтов и центров РАН. В~существующих 
условиях, к сожалению, не представляется возможным обеспечить доступ с 
каждого рабочего места сотрудника РАН ко всем необходимым ему 
журналам. Оптимизировать систему возможно за счет серьезного анализа 
фактического спроса, создания ранжированного списка наиболее 
востребованных журналов, выявления необходимых издательств и 
централизованного (на базе существующих или вновь создаваемых 
консорциумов) заключения договоров с этими из\-да\-тель\-ст\-вами.
     
     Другим фактором оптимизации системы является сокращение числа 
пользователей (IP-ад\-ре\-сов) каждого научного института (научного 
центра), получающих доступ к тому или иному ресурсу, что позволит 
снизить стоимость договоров с поставщиками. В~этом плане представляется 
целесообразным подход, реализованный в БЕН РАН, которая при 
заключении договоров оговаривает права доступа к ресурсам не только из 
здания Центральной биб\-лио\-те\-ки, но и из ее отделов (биб\-лио\-тек) в 
на\-уч\-но-ис\-сле\-до\-ва\-тель\-ских учреждениях (НИУ) РАН. 
При этом поставщикам официально сообщаются IP-ад\-ре\-са биб\-лио\-теч\-ных 
компьютеров, с которых сотрудники институтов могут читать журналы. 
Такая схема работает уже несколько лет и позволяет без существенных 
затрат обеспечивать важнейшей информацией (хотя и не с каждого 
компьютера института) сотрудников более 40~институтов и научных центров 
Москвы и Московского региона. Существенным ограничением этой схемы 
является требование поставщиков, чтобы отделы БЕН в НИУ РАН имели 
компьютеры с выделенными IP-ад\-ре\-са\-ми, поэтому биб\-лио\-те\-кам, 
работающим через про\-кси-сер\-ве\-ры институтов, доступ к ресурсам 
предоставлен быть не может.
     
     Необходимо отметить, что в России и, в част\-ности, в РАН доля 
финансирования, выделяемая на информационное обеспечение науки, 
существенно меньше принятой в развитых и развивающихся странах. 
Согласно мировой практике эта доля составляет от~8\% до~12\% от 
ассигнований на научные исследования. У~нас она не достигает и~1\%.
     
     В существующих условиях, когда биб\-лио\-те\-кам катастрофически не 
хватает централизованно выделяемых РАН средств на приобретение 
информационных ресурсов, необходимо и впредь развивать идеи создания 
академических и межведомственных консорциумов по доступу к научной 
информации, интеграции финансов, выделяемых биб\-лио\-те\-кам, и 
собственных финансов НИУ, перехода на новые системы информационного 
обслуживания пользователей.


{\small\frenchspacing
{%\baselineskip=10.8pt
\addcontentsline{toc}{section}{Литература}
\begin{thebibliography}{9}
     
     
\bibitem{1-g}
The International Coalition of Library Consortia (ICOLC). {\sf 
http://www.library.yale.edu/consortia}.
\bibitem{2-g}
FinELib, the National Electronic Library. The National Library of Finland. {\sf 
http://www.nationallibrary.fi/\linebreak libraries/finelib/finelibconsortium.html}.
\bibitem{3-g}
\Au{H$\ddot{\mbox{o}}$kli E.} Libraries in Finland establish consortia~// Liber 
Quarterly: The J.~European Research Libraries, 2001. Vol.~11. No.\,1. P.~53--59.
\bibitem{4-g}
\Au{Hormia-Poutanen K., Xenidou-Dervou~C., Kupryte~R., Stange~K., 
Kuznetsov~A., Woodward~H.} Consortia in Europe: Describing the various 
solutions through four country examples~// Library Trends, 2006. Vol.~54. No.\,3. 
{\sf https://dspace.lib.cranfield.ac.uk/handle/1826/1014}.
\bibitem{5-g}
\Au{Литвинова Н.\,Н.} Электронные документы: отбор, использование и 
хранение~// Библиотека, 2005. №\,6. С.~6--9.


\label{end\stat}

\bibitem{7-g}
\Au{Никаньшин Д.\,П., Туриянский~И.\,Е., Астафьев~М.\,Н.} О~развитии 
зеркального сервера научной электронной биб\-лио\-те\-ки РФФИ~// 
Исследования по информатике, 2003. Вып.~5. С.~133--142.

\bibitem{6-g}
\Au{Хельферих П., Красикова~О.\,Л.} Научная информация для российских 
биб\-лио\-тек~// Библиотеки и ассоциации в меняющемся мире: новые 
технологии и новые формы сотрудничества: Мат-лы 7-й Междунар. 
конф.~--- Судак, Крым, Украина, 2000.~--- Т.~2. С.~127--128.


\end{thebibliography} } }

\end{multicols} %6
\renewcommand{\figurename}{\protect\bf Figure}
\renewcommand{\tablename}{\protect\bf Table}

\def\stat{kalinich}


\def\tit{CONCEPTUAL MODELING OF~MULTIDIALECT WORKFLOWS}

\def\titkol{Conceptual modeling of~multidialect workflows}

\def\autkol{L.~Kalinichenko, S.~Stupnikov, A.~Vovchenko,
and~D.~Kovalev}

\def\aut{L.~Kalinichenko$^{1,2}$, S.~Stupnikov$^1$, A.~Vovchenko$^1$,
and~D.~Kovalev$^1$}

\titel{\tit}{\aut}{\autkol}{\titkol}

%{\renewcommand{\thefootnote}{\fnsymbol{footnote}}
%\footnotetext[1] {}}

\renewcommand{\thefootnote}{\arabic{footnote}}
\footnotetext[1]{Institute of Informatics Problems, Russian Academy of Sciences,
44-2 Vavilov Str., Moscow 119333, Russian Federation}
\footnotetext[2]{Faculty of
Computational Mathematics and Cybernetics, M.\,V.~Lomonosov Moscow State University,
1-52~Leninskiye Gory, GSP-1, Moscow 119991, Russian Federation}


%\vspace*{6pt}

\def\leftfootline{\small{\textbf{\thepage}
\hfill INFORMATIKA I EE PRIMENENIYA~--- INFORMATICS AND APPLICATIONS\ \ \ 2014\ \ \ volume~8\ \ \ issue\ 4}
}%
 \def\rightfootline{\small{INFORMATIKA I EE PRIMENENIYA~--- INFORMATICS AND APPLICATIONS\ \ \ 2014\ \ \ volume~8\ \ \ issue\ 4
\hfill \textbf{\thepage}}}

%\vspace*{6pt}


\Abste{This paper contributes to the techniques for conceptual representation of
data analysis algorithms and data integration facilities as well as processes to
specify data and behavior semantics in one paradigm. An investigation of a~novel
approach for applying a~combination of semantically different
platform-independent rule-based languages (dialects) for interoperable conceptual
specifications over various rule-based systems (RSs) relying on the rule-based
program transformation technique recommended by the W3C Rule Interchange
Format (RIF) is extended here. Such approach is combined with the facilities
aimed at the semantic rule-based mediation intended for the heterogeneous data
base integration. This paper extends a~previous research of the authors in the
direction of workflow modeling for definition of compositions of algorithmic
modules in a~process structure. A~capability of the multidialect workflow
support specifying the tasks in semantically different languages mostly suited to
the task orientation is presented. A~practical workflow use case, the
interoperating tasks of which are specified in several rule-based languages
(RIF-CASPD, RIF-BLD, RIF-PRD), is introduced. In addition, OWL~2 is used
for the conceptual schema definition, RIF-PRD is used also for the workflow
orchestration. The use case implementation infrastructure includes a~production
rule-based system (IBM ILOG), a~logic rule-based system (DLV), and
a~mediation system.}

\KWE{conceptual specification; workflow; RIF; production rule languages;
database integration; mediators; PRD; multidialect infrastructure}

\DOI{10.14357/19922264140413}

%\vspace*{6pt}


\vskip 12pt plus 9pt minus 6pt

      \thispagestyle{myheadings}

      \begin{multicols}{2}

                  \label{st\stat}

\section{Introduction}

  \noindent
  This work keeps on the intention of developing the facilities for conceptual
declarative problem specification and solving in data intensive domains (DID). In [1]
it was claimed that conceptual data semantics alone (e.\,g., formalized in ontology
languages based on description logic) are insufficient, so that conceptual
representation of data analysis algorithms as well as processes for problem solving are
required to specify data and behavior semantics in one paradigm.

The results presented in this paper\footnote[3]{This paper is an extended for the journal version of
the results presented in the ``Multidialect Workflows'' report at the ADBIS'2014
Conference.} extend the research~[1] aimed at the definition and implementation of the
facilities for conceptually-driven problems specification and solving in DID aiming at
ensuring eventually the following capabilities for expressing the specifications:
\begin{enumerate}[(1)]
\item an ability to provide complete and precise specification of the abstract
structure and behavior of the domain entities, their consistency, relationship, and
interaction;
\item well-grounded diversity of semantics of the modeling facilities providing for
the best attainable expressiveness, compactness, and precision of the definition of
the problem solving algorithm specifications;
\item arrangements for the extensions of the modeling facilities satisfying the
changing technological and practical needs;
\item specification independence from implementation platforms (languages,
systems);
\item specification independence from concrete information resources (IRs)
(databases,
services, ontologies, etc.)\ combined with facilities for their semantic
integration and interoperability; and
\item built-in methodologies for creation of unifying specification languages
providing for construction of semantics-preserving mappings of conceptual
specifications into their implementations in specific platforms.
\end{enumerate}

  The research reported in~[1] investigated the conceptual modeling facilities for
DID applying rule-based declarative logic languages possessing different,
complementary semantics and capabilities combined with the methods and languages
for heterogeneous data mediation and integration. Two fundamental techniques were
combined: ($i$)~constructing of the unifying extensible language providing for
semantics-preserving mapping into it of various IR
specification languages (e.\,g., such as data definition (DDL) and
data manipulation (DML) languages for databases); and ($ii$)~creation of
the unified extensible family of rule-based languages (dialects) and a~model of
interoperability of the programs expressed in such dialects with different semantics.

  The first technique is based on the experience obtained in course of the
SYNTHESIS language development~[2]. The kernel of the SYNTHESIS language is
based on the object-frame data model used together with the declarative rule-based
facilities in the logic language similar to a~stratified Datalog with functions and
negation. The extensions of the kernel are constructed in such a~way that each
extension together with the kernel is a~result of semantic preserving mapping of some
IR language into the SYNTHESIS~[2]. The canonical information model is
constructed as a~union of the kernel with such extensions defined for various resource
languages. Canonical model is used for development of \textit{mediators} positioned
between the users, conceptually formulating problems in terms of the mediators, and
distributed resources. A~schema of a~subject mediator for a~class of problems
includes the specification of the domain concepts defined by the respective
ontologies.
{ %\looseness=1

}

  Another, multidialect technique for rule-based programs interoperability applied is
based on the RIF standard [3] of W3C. The RIF standard introduces a~unified family of rule-based
languages together with a~methodology for constructing of semantic preserving
mappings of specific languages used in various RSs into RIF
dialects. Examples of RSs include {SILK}, {OntoBroker}, {DLV},
{IBM Websphere ILOG JRules}, {RIF4J}\;+\;{IRIS}, and others
(more examples can be found at {\sf
http:// www.w3.org/2005/rules/wiki/Implementations}). From the RIF point of view, an
IR is a~program developed in a~specific language of some RS.

  In [1], the first results obtained were presented including the description of an
approach and an infrastructure supporting:
  \begin{itemize}
\item application domain conceptual specification and problem solving algorithms
definitions based on the combination of the heterogeneous database mediation
technique and the rule-based multidialect facilities;
\item interoperability of distributed multidialect rule-based programs and mediators
integrating heterogeneous databases; and
\item rule delegation approach for the peer interactions in the multidialect
environment.
\end{itemize}

  The proof-of-concept prototype of the infrastructure based on the SYNTHESIS
environment and RIF standards has been implemented. The approach for multidialect
conceptualization of a~problem domain, rule delegation, rule-based programs, and
mediators interoperability were explained in detail and illustrated on an use-case in
the finance domain~[1]. For the conceptual definition of the use-case problem, the
OWL was used for the domain concepts definition and two RIF logic dialects
RIF-BLD~[4] and RIF-CASPD~[5] were used and mapped for implementation into the
SYNTHESIS formula language and the ASP (answer set programming)
based DLV~[6] language, respectively.

  The results obtained so far are quite encouraging for future work: they show that
the mentioned in the beginning capabilities~(1)--(6) sought for conceptual modeling
become feasible. This paper reports the results of extending the research in the
direction of modeling of the processes for the problem solving following the approach
briefly outlined above. These results include extensions of the infrastructure and
specification languages considered in~[1] to the workflow level keeping the same
approach and paradigm as well as aiming at the capabilities of the
conceptualization~(1)--(6) that were stated in~[1] and mentioned in the beginning of
the introduction.

  For investigation of such extension with respect to the choice of rule-based languages, it was
decided not to go outside the limits of the existing set of the published RIF dialects.
Such decision would allow to retain well-defined semantics of the conceptual
rule-based languages with a~possibility to check preservation of their semantics by various
languages of the implementing systems.

  The production rule dialect RIF PRD~[7] has been chosen as the language for the
workflow modeling in such a~way that the tasks of the workflow can have
multidialect rule-based representation (as defined in~[1]). This paper reporting the results
of such investigation is structured as follows. To make the paper self-contained, the
next section provides a~brief overview of the infrastructure supporting multidialect
programming defined in details in~[1]. Here, it is stressed
 that this infrastructure is
suitable for the workflow tasks specification. Workflow-oriented extension of the
multidialect infrastructureis considered in section~3. Use case implementation in
the proof-of-concept prototype is given in section~4. Related works are reviewed
in section~5. Concluding Remarks summarize contributions of the research.

%\vspace*{-24pt}

\begin{figure*} %fig1
\vspace*{1pt}
 \begin{center}
 \mbox{%
 \epsfxsize=164.734mm
 \epsfbox{kal-1.eps}
 }
 \end{center}
 \vspace*{-9pt}
\Caption{Conceptual schema and peer specifications }
\vspace*{-2pt}
  \end{figure*}

\section{Basic Principles of the Workflow Tasks Representation
in~the~Multidialect Infrastructure}

  \noindent
  Each workflow task (besides those that for pragmatic reasons are defined as
externally specified functions) is assumed to be represented in the novel infrastructure
defined in details in~[1]. Conceptual programming of tasks is performed using the
RIF dialects (now not only logic but also PRDs can be
used).

Conceptual tasks are implemented by their transformation into the rule-based
programs of the respective RSs and mediation systems (MSs). \textit{Conceptual
specification of a~task} is defined in the context of a~subject domain and consists of
a~set of RIF-documents (document is a~specification unit of RIF). The
\textit{conceptual schema} of the domain is defined using OWL 2~\cite{8-kal}
ontologies. Such usage of ontology is analogous to~\cite{22-kal}; however, it is
specifically important in the multidialect environment due to the formally defined
compatibility between RIF and OWL. The ontologies contain entities of the domain
and their relationships (Fig.~1, right-hand part). Conceptual specification of a~task is
defined over conceptual schema. Ontologies are imported into the RIF-documents
specifying an import profile, for instance, {OWL Direct}. Documents
\textit{import} other documents having the same semantics (the \textit{Import}
directive), \textit{link} documents defined using other dialects and having different
semantics (remote module directive \textit{Module}) or \textit{refer} to entities
contained in other documents using \textit{external terms}.
{\looseness=1

}

  Semantics of a~conceptual task definition in such setting becomes a~multidialect
one. The specification modules of a~task are treated as peers. Mediation modules are
assumed to be defined in RIF-BLD for representation of the mediator rules (to be
interpreted in SYNTHESIS) supporting schema mapping and semantic integration of
the IRs. Multidialect task is implemented by means of
transformation of conceptual specifications into modular, component-based
peer-to-peer (P2P)
program represented in the languages of the MSs and RSs
with the respective semantics. Interoperability of logic rule components of such
distributed program is carried out by means of the delegation technique [1,
section~3.3]. Production rule components are considered as external functions,
interoperability is achieved through the mechanism of external terms.

  A schema $S_R$ of a~peer~$R$ is a~set of entities (classes or relations and their
attributes) corresponding to extensional and intensional predicates of the resource
implementing the peer~$R$.

  The RS or the MS of each peer~$R$ should be
a~conformant~$D_R$ consumer where~$D_R$ is the~respective RIF dialect (Fig.~1,
left-hand part). Conformance is formally defined using formula entailment and
language mappings~[3].

  The peer $R$ is relevant to a~RIF-document~$d$ of a~conceptual specification of a~problem
  (Fig.~1, right-hand part) if ($i$)~$D_R$ is a~subdialect of the document~$d$
dialect (subdialect is a~language obtained from some dialect by removing certain
syntactic constructsand imposing respective restrictions on its semantics~[4]; each
program that conforms with the subdialect also conforms with the dialect) and
($ii$)~entities of the peer schema~$S_R$ (if they exist) are \textit{ontologically
relevant} to entities of the conceptual schema the names of which are used
in~$d$~for extensional predicates.

  The schema of a~relevant peer is mapped into the conceptual schema. The mapping
establishes the correspondence of the conceptual entities referred in the
document~$d$ to their expressions in terms of entities of the schema~$S_R$ using
rules of the $D_R$ dialect. These schema mapping rules constitute separate
  RIF-document (Fig.~1, middle part).

  Peers communicate using a~technique for distributed execution of the rule-based
programs. The basic notion of the technique is delegation-transferring facts and
rules from one peer to another. A~peer is installed on a~node of the multidialect
infrastructure. A~node is a~combination of a~wrapper, an RS or an MS, and a~peer
(for the details, refer~[1, Fig.~3]). A~wrapper
transforms programs and facts from the specific RIF dialect into the language of the
RS or MS and \textit{vice versa}. A~wrapper also implements the delegation mechanism.
Transferring facts and rules among peers is performed in the RIF dialects.

  A~special component (\textit{Supervisor}) of the architecture defined in~[1] stores
shared information of the environment, i.\,e., conceptual specifications related to the
domain and to the problem, a~list of the relevant resources, RIF-documents combining
rules for the conceptual specification and a~resource schema mapping.

  Implementation of the conceptual specification includes the following steps:
  \begin{enumerate}[(1)]
\item rewriting of the conceptual documents into the RIF-programs of the peers
performed by the \textit{Supervisor}. The rewriting includes also ($i$)~replacing the
document identifiers (used to mark predicates) by peer identifiers and ($ii$)~adding
schema mapping rules to programs (Fig.~1, middle part); %\\[-14pt]
\item a transfer of the rewritten programs to nodes containing peers relevant to the
respective conceptual documents. The transfer is performed by the \textit{Supervisor}
by calling the method \textit{loadRules} of the respective node wrappers; %\\[-14pt]
\item a transformation of the RIF-programs into the concrete RS or MS languages.
The transformation is performed by the \textit{NodeWrapper} or by the RS or MS
itself (if the RS or MS supports the respective RIF dialect); and %\\[-14pt]
\item an execution of the produced programs in P2P environment.
\end{enumerate}

  During the process of rewriting of the conceptual schema into the resource
programs, the relationships between RIF-documents of the conceptual schema defined
by remote or imported terms are replaced by relationships between peers also defined
by remote or imported terms. To implement remote and imported terms, a~\textit{rule
delegation} mechanism is used to transfer facts and rules from one peer to another.
The details of rule delegation approach including description of the related algorithms
are provided in~[1].

\vspace*{-7pt}

\section{Workflow-Oriented Extension of~the~Multidialect
Infrastructure}

\vspace*{-2pt}

  \noindent
  The aim of the infrastructure proposed is a~conceptual programming of problems in
the RIF-dialects and an implementation of conceptual specifications using rule-based
languages of the RSs and MSs. One of the objectives of this particular paper is to
introduce an extension of the existing multidialect infrastructure~[1] aiming at the
conceptual specification of rule-based workflows.

  Conceptual specification of a~problem (class of problems) is defined in the context
of a~subject domain and consists of a~set of
  RIF-documents. Besides the documents expressed in the logic dialects of RIF, the
documents expressed in the production rule dialect (RIF-PRD) also can be a~part of
conceptual specification of a~problem. In particular, these documents are aimed to
express a~process of solving the problem as the production rule-based workflow.

\vspace*{-4pt}

\subsection{Specification of~workflow orchestration}

  A workflow consists of a~set of tasks orchestrated by specific constructs
(\textit{workflow patterns}~\cite{9-kal}, for instance, \textit{sequence}, \textit{split},
\textit{join}) defining the order of tasks execution. The specification of such
orchestration is called here a~\textit{workflow skeleton}. A~skeleton is defined using
RIF-PRD production rules. Workflows and workflow patterns can be represented
using production rules in various ways, e.\,g., as in~\cite{9-kal, 17-kal}. The
approach applied in this paper to represent workflows requires the extension of
  RIF-PRD dialect by several built-in predicates (they are considered to be a~part of
\textit{wkfl} namespace referenced by
  {\sf http://www.w3.org/2014/rif-workflow-predicate\#} URI similarly to
\textit{func} and \textit{pred} namespaces defined in~\cite{21-kal} for built-in
functions and predicates of RIF):
  \begin{itemize}
\item predicate {\sf wkfl:end-of-task(?arg)} where \textit{\sf ?arg} is an identifier of a~task. The value space of
{\sf ?arg} is the XML-Schema built-in data type
{\sf xsd:Name} representing XML names. The predicate turns into true if a~task
\textit{?arg} has been completed;
\item predicate {\sf wkfl:variable-definition(?arg1\, ?arg2)} where {\sf ?arg1}
is the identifier of a~variable and {\sf ?arg2} is the identifier of a~type of the
variable.
The value space for both arguments is {\sf xsd:Name}. Turning the predicate into
true means that a~variable {\sf ?arg1} of type {\sf ?arg2} is defined in the
context of a~workflow;
\item predicate {\sf wkfl:variable-value(?arg1\,?arg2)} where {\sf ?arg1} is the
identifier of a~variable and {\sf ?arg2} is the value of the variable. The value space for
the first argument is {\sf xsd:Name}, the value space for the second argument is the
union of value spaces of all RIF built-in datatypes. Turning the predicate into true
means that a~variable {\sf ?arg1} has the value {\sf ?arg2};
\item predicate {\sf wkfl:parameter-definition(?arg1\,?arg2\,
?arg3)} where
{\sf ?arg1} is the identifier of a~workflow parameter; {\sf ?arg2} is the identifier
of a~type of the parameter; and {\sf ?arg3} is the direction of the parameter. The value
space for the first and for the second arguments is {\sf xsd:Name}. The value space
for the third argument is {\sf \{IN, OUT, IN\_OUT\}}
(\textit{input}, \textit{output}, or
\textit{input--output} parameter).
Turning the predicate into true means that a~parameter {\sf ?arg1} of
type {\sf ?arg2}, and direction {\sf ?arg3} is defined
for a~workflow; and
\item predicate {\sf wkfl:parameter-value(?arg1\,?arg2)} defines values of
workflow parameters in the same way as {\sf wkfl:variable-value} defines values
of workflow variables.
  \end{itemize}

  Predicates {\sf wkfl:variable-definition} and
  {\sf wkfl:}\linebreak {\sf variable-value} allow
to specify workflow variables and their values and thus to organize the data flow
within a~workflow. Predicates {\sf wkfl:parameter-definition} and
{\sf wkfl:parameter-value} allow to specify workflow parameters and their values
and thus to define the interface of a~workflow in terms of input and output parameters.
Using of workflow parameters and variables is illustrated in the Appendix.

  The predicate {\sf wkfl:end-of-task(?arg)} allows to orchestrate the order of
execution of workflows tasks using conditions and actions of production rules. In this
section, the template rules intended for representation of several basic workflow
patterns (Fig.~2) are provided.

\begin{center}  %fig2
\vspace*{6pt}
\mbox{%
 \epsfxsize=76.913mm
 \epsfbox{kal-2.eps}
 }
  \vspace*{2pt}

{{\figurename~2}\ \ \small{Basic workflow patterns}}
  \end{center}

\vspace*{6pt}


\addtocounter{figure}{1}

  Three well-known workflow patterns are considered below: {\sf Sequence},
{\sf AND-Split}, and {\sf AND-Join}.

  The \textit{AND-Split}\footnote{In this paper, the simplified \textit{presentation
syntax}~\cite{7-kal} is used.} workflow pattern is represented in RIF-PRD by the
following production ruletemplate using {\sf wkfl:end-of-task} predicate:
  \begin{verbatim}
If Not(External(wkfl:end-of-task(A)))
Then Do (Act(A)
 Assert(External(wkfl:end-of-task(A))))
If And(Not(External(wkfl:end-of-task(B)))
 External(wkfl:end-of-task(A)))
Then Do (Act(B)
 Assert(External(wkfl:end-of-task(B))))
If And(Not(External(wkfl:end-of-task(C)))
 External(wkfl:end-of-task(A)))
Then Do (Act(C)
 Assert(External(wkfl:end-of-task(C))))
\end{verbatim}

  The template includes three rules for tasks~$A$, $B$, and~$C$, respectively.
${\sf Act}(A)$, ${\sf Act}(B)$, and ${\sf Act}(C)$ denote \textit{actions} associated with tasks~$A$,
$B$, and $C$. Orchestration (tasks~$B$ and~$C$ are executed concurrently right after
task~$A$ is completed) is specified using {\sf wkfl:end-of-task} predicate in
conditions and {\sf Assert} actions of rules.

  Similarly, the {\sf AND-Split} pattern is represented in RIF-PRD by the
following production rule template:

\vspace*{-1.5pt}

\noindent
  \begin{verbatim}
If Not(External(wkfl:end-of-task(A)))
Then Do (Act(A)
 Assert(External(wkfl:end-of-task(A))))
If And(Not(External(wkfl:end-of-task(B)))
 External(wkfl:end-of-task(A)))
Then Do (Act(B)
 Assert(External(wkfl:end-of-task(B))))
If And(Not(External(wkfl:end-of-task(C)))
 External(wkfl:end-of-task(A)))
Then Do (Act(C)
 Assert(External(wkfl:end-of-task(C))))
\end{verbatim}

\vspace*{-1.5pt}

  The {\sf Sequence} pattern is represented in RIF-PRD by the following
production rule template:

\vspace*{-1.5pt}

\noindent
  \begin{verbatim}
If Not(External(wkfl:end-of-task(A)))
Then Do (Act(A)
 Assert(External(wkfl:end-of-task(A))))
If And(Not(External(wkfl:end-of-task(B)))
 External(wkfl:end-of-task(A)))
Then Do (Act(B)
 Assert(External(wkfl:end-of-task(B))))
\end{verbatim}

\vspace*{-1.5pt}

  More complicated patterns like OR-, XOR- splits and joins, structured loops,
subflows, and others are represented in RIF-PRD similarly.

\vspace*{-6pt}

\subsection{Workflow tasks specification}

  \noindent
  Workflow taskscan be specified as:
  \begin{itemize}
\item separate RIF-documents in various logic RIF-dialects (this is the way how
multidialect infrastructure~[1] is extended with workflow capabilities);\\[-15pt]
\item separate RIF-documents in the RIF-PRD dialect;\\[-15pt]
\item set of production rules embedded into the workflow skeleton; and\\[-15pt]
\item external functions treated as ``black boxes.''
\end{itemize}

  Semantics of tasks specified as multidialect logic programs are defined in
accordance with the RIF-FLD~[3] standard and standards for the respective
  RIF-dialects (BLD, CASPD, etc.). Semantics of tasks specified as production
rule programs are defined in accordance with the RIF-PRD standard. Semantics of
external functions ``are assumed to be specified externally in some document''~[3].

  All kinds of tasks (except those that are embedded into a~workflow skeleton) are
referenced in the workflow skeleton as \textit{external terms}~[3] like
${\sf External}\left({\sf t}\right)$
where term~{\sf t} is defined by an external resource identified by internationalized
resource identifier (IRI)~[3].

\begin{figure*} %fig3
\vspace*{1pt}
 \begin{center}
 \mbox{%
 \epsfxsize=163.675mm
 \epsfbox{kal-3.eps}
 }
 \end{center}
 \vspace*{-9pt}
\Caption{Extended multidialect infrastructure}
\end{figure*}


\vspace*{-6pt}

\subsection{Workflow implementation infrastructure}

  \noindent
  Workflows defined in the conceptual specification are implemented in the
environment shown in Fig.~3. Peer-\linebreak\vspace*{-12pt}

\pagebreak

\noindent
to-peer environment~[1] intended to implement logic
programs is extended with a~production rule-based system (PRS) (for
instance, a~production system compliant with the OMG Production Rule
Representation~\cite{16-kal}) and with external functions, implemented as
  web-services. Implementation of the conceptual specification includes the
following steps:
  \begin{enumerate}[(1)]
\item transfer of the conceptual RIF-documents constituting a~workflow skeleton to
the production rule-based system node (performed by the \textit{Supervisor} component);\\[-14pt]
\item transformation of the conceptual RIF-documents constituting a~workflow
skeleton into the language of the production rule-based system (performed by the PRS
Wrapper component);\\[-14pt]
\item transferring RIF logic programs related to tasks to the relevant nodes of the
environment and transformation of the RIF-programs into the concrete RS or MS
languages~[1]; and\\[-14pt]
\item execution of the workflow.
\end{enumerate}


  The interface of the \textit{Supervisor} includes methods for submitting and
executing a~workflow represented as a~set of RIF-documents, and for getting the
result of the workflow execution.

  To provide a~proof of the multidialect infrastructure concept, a~use case in the
financial domain has been implemented. The problem to be solved in the use case is
called the \textit{investment portfolio diversification problem}. The detailed
description of the use case is included in the Appendix.

\vspace*{-9pt}

\section{Related Work}

  \noindent
  Two types of workflow models, namely, abstract and concrete, were
identified~\cite{15-kal}. In the abstract model, a~workflow is described in an abstract
form, without re-\linebreak\vspace*{-12pt}
\columnbreak

\noindent
ferring to specific resources. In this paper, workflow
representation in abstract and platform-independent  form is suggested.

  A classification model for scientific workflow characteristics~\cite{9-kal}
contributes to better understanding of scientific workflow requirements. The list of
structural patterns discovered during this analysis (including sequential, parallel,
parallel-split, parallel-merge, and mesh) influenced the choice of the required workflow
patterns.

  The OMG standard~\cite{16-kal} reflects an attitude to production rules from the
industrial side providing an OMG MDA (model-driven architecture)
platform-independent model (PIM)  with a~high probability of
support at the PSM (platform-specific model) level from the rule engine vendors.
Similar capabilities though formally defined are used as the basis for the
RIF-PRD~\cite{7-kal}.

  Some vendors of such production rule engines have extended their languages with
the workflow specification capabilities. IBM has extended ILOG to provide the
ruleflow capability. Microsoft supports Windows Workflow Foundation as a~platform
providing the workflow and rules capabilities. The examples of specific formalisms for
PIM rule-based process specifications are also provided in~\cite{17-kal}.

  Comparing to the known variants of the PIM production rule representations,
  selection of the RIF-PRD is considered to be well grounded:
  \begin{enumerate}[(1)]
\item the RIF-PRD is formally defined;
\item RIF ensures support of interoperability of modules written in different
rule-based dialects with different semantics;
\item RIF provides foundations for PIM to PSM semantic preserving
transformation; and
\item RIF also provides ability for specification of the concepts in application
domain terms combining rule-based specifications with the OWL ontologies.
\end{enumerate}

  Importance of providing the interdialect interoperation is advocated
  in~\cite{18-kal} for combining the functionalities of production systems and logic
programs for abductive logic programming (ALP). The ALP framework gives a~model-theoretic semantics to both kinds of rules and provides them with powerful
proof procedures, combining backward and forward reasoning.

  Papers related to RIF-PRD experimentations are focused mainly on the issue of the
PRD programs transformation to an implementation system. In~\cite{19-kal}, a~case
study of bridging the ILOG Rule Language (IRL) to RIF-PRD and vice versa is
considered. In~\cite{20-kal}, implementation of RIF-PRD in three different
paradigms: Answer Set Programming, Production Rules, and Logic Programming
(XSB) is investigated.

  The contribution of this paper with regard to previous works of the authors~[1] consists in
extensions of the infrastructure and specification languages considered in~[1] to the
workflow level.

\section{Concluding Remarks}

  \noindent
  Progress in the investigation of the infrastructure~[1] for the conceptual
multidialect interoperable programming in the abstract, rule-based,
platform-independent notations is reported. An extension of the coherent
combination of the multidialect rule-based programming technique recommended by
the W3C RIF with the approach for unifying modeling of heterogeneous data bases
for their semantic mediation is presented. The extension of the infrastructure and specification
languages considered in~[1] in the direction of the workflow modeling is presented.

  Sticking to the limits of the existing set of the published RIF dialects,
   a~capability of the multidialect workflow support
   is presented with the tasks specified in
semantically different languages mostly suited to the task orientation.
Also, a~realistic problem solving use case containing the interoperating tasks
specified in
several platform-independent rule-based languages: RIF-CASPD, RIF-BLD,
  RIF-PRD, is presented. In addition, OWL~2 is used for the conceptual schema definition,
  RIF-PRD is applied for the workflow orchestration. The platforms selected for
implementation of the tasks include: DLV, SYNTHESIS, IBM ILOG. Such approach
retains well-defined semantics of the platform-independent rule-based languages with
a possibility to check preservation of their semantics by various languages of the
implementing systems. The principle of independence of tasks from the specific IRs is
carried out by the heterogeneous database mediation facilitates contributing to the
  reuse of tasks and workflows. Alongside with the further extension of the
approach, in the future work, the authors plan to apply the conceptual multidialect
programming philosophy for support of the experiments in data intensive sciences. In
particular, they plan to investigate modeling hypotheses in astronomy representing them
as a~set of rules applying the multiplicity of the dialects required.


%\setcounter{equation}{0}

\vspace*{12pt}

{{\hfill \textbf{APPENDIX A}}}

\vspace*{-12pt}

\subsection*{MULTIDIALECT WORKFLOW USE CASE}


{\small


 %\section*{\raggedleft Appendix~A.\\ Multidialect Workflow Use-Case}


%\renewcommand{\thesection}{A\arabic{equation}}

\subsection*{A.1\ Investment portfolio diversification\\
\hspace{20pt}problem extended}

  \noindent
  Motivation of the use case that illustrates the proposed approach comes from the
finance area. The use case extends the \textit{investment portfolio diversification
problem} defined in~[1, Appendix] by adding workflow orchestration applying the
RIF-PRD. The idea of the portfolio diversification problem is
as follows. The portfolio is a~collection of securities of companies, and its size is the
number of securities in the portfolio. The problem is to build a~diversified portfolio of
maximum size. Diversification means that the prices of the securities in portfolio
should be almost independent of each other. If the price of one security falls, it will
not significantly affect the prices of others. Thus, the risk of a~portfolio sharp decrease
is reduced.

  The input data for the problem is a~set of securities and respective time series of
indicators of the security price for each security. Time series for each security is a~set
of pairs $(d, v)$ where $d$ is a~date and~$v$ is an indicator of the security price (for
instance, closing price). The financial services \textit{Google Finance} ({\sf
https://www.google.com/finance}) and \textit{Yahoo! Finance} ({\sf
http://finance.yahoo.com/}) are considered. They include various indicators of the
security price for all trading days of the last decades. For the diversified portfolio, the
securities having noncorrelated time series should be used. Noncorrelation of the
time series means that their correlation is less than some predetermined price
correlation value. The output data for the problem is a~set of subsets of securities of
the maximum size, for which the pair wise correlation will be less than the
predetermined one.

  The maximum satisfying subset of securities is calculated in the following way.
Let~$G$ be a~graph where the vertices are the securities. An edge between two
securities exists if absolute value of their correlation is less than a~specified number.
So, any two securities connected by an edge are considered as noncorrelated. In such
case, the problem of finding the portfolio of the maximum size is exactly the problem
of finding a~maximum clique in an undirected graph. A~maximal clique is a~maximal
portfolio. Note that several different maximal portfolios can be found.

  The conceptual specification of the use case~[1] used two RIF-dialects: RIF-BLD
and RIF-CASPD. The use case was implemented in the environment containing a~mediation system used as a~platform for RIF-BLD~[4] and ASP-based DLV
system~\cite{6-kal}~--- a~platform for RIF-CASPD. The RIF-BLD was used to
specify the problem of data integration, and RIF-CASPD~--- the problem of finding
a~maximum clique in an undirected graph.

In this work, the portfolio use case is extended in the following way. The goal is
not only to build a~set of diversified portfolios, but
also to choose the ``best'' of them
according to some criteria. There are several approaches to choose the most
appropriate portfolio.

  The most recognized one is based on the Markovitz portfolio theory~\cite{10-kal}.
The idea is to choose the portfolio, which has the maximum risk/return ratio. The
most well-known metric to operate with risk/return is Sharpe-ratio~\cite{11-kal}:
  $(r_p -r_f)/\sigma^2$. Here, $r_p$ denotes the expected return of the portfolio;
$r_f$ denotes the~risk free rate; and $\sigma^2$ denotes the~portfolio standard deviation (risk).
The more the Sharpe-ratio is, the better the investment is.

  Another approach is based on an idea that with the advent of social networks, it
became possible to monitor ideas, sentiments, actions of people and lots of available
information has to do with the markets and investments. In~\cite{12-kal}, Bollen
\textit{et al.}\ draw the connection between the mood of investor tweets and the move
of Dow Jones Index, stating that correlation between them is more than 80\%. The
idea of using tweets to assess market movements has been implemented in several
hedge funds.

  Combining these two strategies could provide benefits of both of them, which
leads to the following problem statement: having S\&P500 (a stock market index
maintained by the Standard\,\&\,Poor's, comprising 500~large-cap American
companies) list of companies, compute the diversified portfolio of maximum size
with the best risk/return and sentiment ratios.

%\vspace*{-6pt}

\subsection*{A.2\	Conceptual specification\\
\hspace*{20pt}of~the~application domain\\
\hspace*{20pt}and~the~problem}

%\vspace*{-2pt}

  \noindent
  Conceptual schema (ontology) of the application domain of historical prices of
securities is written in the simplified OWL functional syntax~\cite{8-kal}
({\sf Declaration} keyword is omitted; {\sf property}, {\sf domain}, and {\sf range}
declarations are combined).
  \begin{verbatim}
Ontology(<http://synthesis.ipi.ac.ru/portfolio/
    ontology>
 Class(Portfolio)
  ObjectProperty(securities domain(Portfolio)
   range(Portfolio))
  DataProperty(expected_return domain(Portfolio)
   range(xsd:double))
  DataExactCardinality(1 expected_return
   Portfolio)
  DataProperty(std_dev domain(Portfolio)
   range(xsd:double))
  DataExactCardinality(1 std_dev Portfolio)
  DataProperty(sharpe_ratio domain(Portfolio)
   range(xsd:double))
  DataExactCardinality(1 sharpe_ratio Portfolio)
  DataProperty(twitter_positive_ratio
   domain(Portfolio) range(xsd:double))
  DataExactCardinality(1 twitter_positive_ratio
   Portfolio)
  DataProperty(risk_free_rate domain(Portfolio)
   range(xsd:double))
  DataExactCardinality(1 risk_free_rate
   Portfolio)
  DataProperty(recommended domain(Portfolio)
   range(xsd:boolean))
  DataExactCardinality(1 recommended Portfolio)

 Class(Security)
  DataProperty(ticker  domain(Security)
   range(xsd:string))
  DataExactCardinality(1 ticker Security)
  DataProperty(rates  domain(Security)
   range(StockRate))
  DataProperty(positive_tweets domain(Security)
   range(xsd:double))
  DataExactCardinality(1 positive_tweets
   Security)
  DataProperty(sec_expected_return
   domain(Security) range(xsd:double))
  DataExactCardinality(1 sec_expected_return
   Security)
  DataProperty(sec_std_dev domain(Security)
   range(xsd:double))
  DataExactCardinality(1 sec_std_dev Security)

 Class(StockRate)
  DataProperty(date domain(StockRate)
   range(xsd:date))
  DataExactCardinality(1 date StockRate)
  DataProperty(price domain(StockRate)
   range(xsd:double))
  DataExactCardinality(1 price StockRate)
)
  \end{verbatim}

  \vspace*{-6pt}

  A~portfolio (the {\sf Portfolio} class) is characterized by a~set of securities
({\sf securities} attribute) contained in the portfolio, by several metrics: expected
return ({\sf expected\_return} attribute), standard deviation ({\sf std\_dev}
attribute), Sharpe ratio ({\sf sharpe\_ratio attribute}), risk free rate
({\sf risk\_free\_rate} attribute), and
ratio of positive tweets mentioning securities of
the portfolio ({\sf twitter\_positive\_ratio} attribute).

  A security (the {\sf Security} class) is characterized by identifier ({\sf ticker}
attribute), time series of historical prices (attribute {\sf
rates}), ratio of positive
tweets mentioning the security ({\sf positive\_tweets} attribute), expected return
({\sf sec\_expected\_return} attribute), and standard deviation ({\sf sec\_std\_dev}
attribute).

\begin{figure*} %fig4
\vspace*{1pt}
 \begin{center}
 \mbox{%
 \epsfxsize=126.24mm
 \epsfbox{kal-4.eps}
 }
 \end{center}
 \vspace*{-11pt}
\Caption{Portfolio workflow}
\vspace*{-6pt}
  \end{figure*}

The workflow of the extended portfolio problem is demonstrated in Fig. 4. The workflow
contains six tasks\footnote{To save space, specifications are provided only for
{\sf getPortfolios}, {\sf getPositiveTweetRatio}, and
{\sf computePortfolioTwitterMetrics} tasks.}:
\begin{enumerate}[(1)]
\item {\sf getPortfolios}. A~set of diversified portfolio candidates is computed. The
multidialect task specification consists of two RIF-documents in BLD and CASPD
dialects~[1, Appendix]. Portfolios received as a~result contain only security tickers,
they have to be augmented by financial and sentiments ratios;
\item {\sf getPositiveTweetRatio}. This task is responsible for computing a~sentiment ratio of tweets for every security. Every tweet is assessed
to be positive,
negative, or neutral. The task is specified as a~call of external function;
\item {\sf computePortfolioTwitterMetrics}. The portfolio sentiment ratio is
computed as the average of its securities sentiment ratio. The task is specified using
RIF-PRD;
{\looseness=1

}
\item {\sf getSecurityFinancialMetrics}. For every security in a~portfolio the
financial rates (the {\sf expected return} and the {\sf standard deviation}) are
calculated on the basis of historical rates of securities specified as an OWL~2 class of
the ontology of the application domain. The task is specified using RIF-BLD dialect;
\item {\sf computePortfolioFinancialMetrics}. The computation of the portfolio
expected return, risk, and Sharpe-ratio is done within this task. The task is specified
using RIF-PRD dialect; and
\item {\sf choosePortfolio}. The best portfolio is chosen according to maximizing
the (\textit{Sharpe ratio * sentiment ratio}) coefficient. The task is specified using
RIF-PRD dialect.
  \end{enumerate}

  Workflow skeleton is specified as a~RIF-PRD document importing the ontology of
the application domain:
  \begin{verbatim}
Document( Dialect(RIF-PRD)
 Base(<http://synthesis.ipi.ac.ru/portfolio/
  workflow#>)
 Import(<http://synthesis.ipi.ac.ru/portfolio/
  ontology#>
 <http://www.w3.org/ns/entailment/OWL-Direct>)
Prefix(ont<http://synthesis.ipi.ac.ru/portfolio/
 ontology#>)
Prefix(ofws<http://synthesis.ipi.ac.ru/
 synthesis/projects/RuleInt/OpinionFinderWS#>)
Prefix(mws<http://synthesis.ipi.ac.ru/
 synthesis/projects/RuleInt/MediatorWS#>)

Group 2 (
 Do(
  Assert(External(wkfl:parameter-definition(
   startDatexsd:string IN)))
  Assert(External(wkfl:parameter-definition(
   endDatexsd:string IN)))
  Assert(External(wkfl:parameter-definition(
   bestPortfolioont:Portfolio OUT)))
  Assert(External(wkfl:variable-definition(
   ps  List<ont:Portfolio> IN)))
  Assert(External(wkfl:
   variable-value(ps List())))
 )
)
\end{verbatim}

\noindent
\begin{verbatim}

Group 1 (
 Forall ?sd ?ed such that (
  External(wkfl:parameter-value(startDate ?sd))
  External(wkfl:parameter-value(endDate ?ed))  )
( If Not(External(wkfl:
   end-of-task(getPortfolios)))
  Then
   Do( Modify(External(wkfl:variable-value(ps
    External(mws:getPortfolios(?sd ?ed) )))
   Assert(External(wkfl:
    end-of-task(getPortfolios))) )
 )

 Forall ?ps ?p ?scs ?s ?t such that (
  External(wkfl:variable-value(ps ?ps))
  ?p#?ps  ?p[securities->?scs]
   ?s#?scs ?s[ticker->?t] )
( If And( Not(External(wkfl:
     end-of-task(getTweets)))
   External(wkfl:end-of-task(getPortfolios)))
  Then
  Do( Modify(?s[positive_tweets->
   External(ofws:computeSecPosTweets(?t))] )
   Assert(External(wkfl:
    end-of-task(getTweets))) )
)

Forall ?ps ?p such that (
 External(wkfl:variable-value(ps  ?ps))
  ?p#?ps)
( If And(Not(External(wkfl:
   end-of-task(countTwitterMetrics)))
   External(wkfl:end-of-task(getTweets)) )
  Then Do(
   Modify(?p[twitter_positive_ratio->
    External(func:numeric-divide(
    Sum{?pt | Exists
     ?scs ?s(?p[securities->?scs]
     ?s#?scs  ?s[positive_tweets->?pt])}
    External(func:count(?ps))))])
   Assert(External(wkfl:
    end-of-task(countTwitterMetrics)))
)	)) )
\end{verbatim}

\begin{figure*}[b] %fig5
\vspace*{-4pt}
 \begin{center}
 \mbox{%
 \epsfxsize=162.319mm
 \epsfbox{kal-5.eps}
 }
 \end{center}
 \vspace*{-9pt}
\Caption{Portfolio problem implementation infrastructure}
  \end{figure*}

  Production rules of the document are divided into two groups. The first group with
priority~2 contains rules defining workflow parameters and variable. Input parameters
are \textit{start date} and \textit{end date} of historical rates used for calculation of
\textit{portfolio metrics}. Workflow variable {\sf ps} denotes a~set containing
\textit{portfolio candidates}.

  The second group with priority~1 contains the orchestration rules~--- workflow
skeleton. The only orchestration rule provided in the example above corresponds to
the task {\sf getPortfolios}. The external function {\sf getPortfolios}
encapsulates a~multidialect logic program calculating portfolio candidates~[1,
Appendix]. A~{\sf Modify} action is used to call the function and to put the
returned result into the {\sf ps} variable.

\vspace*{-6pt}

\subsection*{A.3\	Revised portfolio problem infrastructure}

  \noindent
  The implementation structure of the use case is shown in Fig.~5.




  The RIF-PRD workflow skeleton was transformed into a~program (rule set) in the
ILOG~\cite{13-kal} language combining production rules and workflow facilities
(like {\sf fork} and {\sf sequence}). The ILOG program was executed in the
{IBM Operational Decision Manager} tool~\cite{24-kal}. In order to execute
ILOG programs, the underlying execution model (XOM)~\cite{25-kal}
was defined as a~set of
Java classes: {\sf Portfolio}, {\sf Security}, and {\sf StockRate}. The
{\sf Portfolio class} corresponds to a~financial portfolio and contains as attributes a~set of
securities in it, its expected return, standard deviation, Sharpe ratio, and twitter
positive ratio. Code of this class is provided below:
  \begin{verbatim}
public class Portfolio {
 private Collection<Security> securities;
 private double expected_return;
 private double std_dev;
 private double sharpe_ratio;
 private double twitter_positive_ratio;
 // as of 05.04.14 US 5-year treasuries
 private static double risk_free_rate = 0.0169;
 private boolean recommended;
}
\end{verbatim}

  Class {\sf Security} corresponds to real world financial securities. The class
contains as attributes a~ticker, ratio of positive tweet number to the sum of positive
and negative tweets, a~set of stock rates, security's standard deviation, and expected
return. These attributes are set as responses to corresponding web services queries:

\vspace*{-2pt}

\noindent
  \begin{verbatim}
public class Security {
 public String ticker;
 public double positive_tweets;
 public Collection<StockRate> rates;
 public double std_dev;
 public double expected_return;
 public static int number_of_periods = 5;
}	
\end{verbatim}

\vspace*{-2pt}

{\sf StockRate} is a~simple class and contains just two attributes~--- price and date:

\vspace*{-2pt}

\noindent
  \begin{verbatim}
public class StockRate {
 public float price;
 public String date;
}
\end{verbatim}

\vspace*{-2pt}

  It is easy to see that the one-to-one mapping exists between conceptual schema
entities and execution model entities.

  Parameters of RIF-PRD workflow skeleton ({\sf startDate}, {\sf endDate}, and
{\sf bestPortfolio}) are mapped into the respective parameters of ILOG rule set
(Fig.~6).

\begin{figure*} %fig6
\vspace*{1pt}
 \begin{center}
 \mbox{%
 \epsfxsize=115mm
 \epsfbox{kal-6.eps}
 }
 \end{center}
 \vspace*{-9pt}
\Caption{Rule set parameters}
  \end{figure*}

  The variable of RIF-PRD workflow skeleton ({\sf ps}) is mapped into a~local variable
of the rule set. Specification of the variable looks as follows:

\vspace*{-2pt}

\noindent
  \begin{verbatim}
<?xml version="1.0" encoding="UTF-8"?>
<ilog.rules.studio.model.base:VariableSetxmi:
  version="2.0"
xmlns:xmi="http://www.omg.org/XMI"
  xmlns:ilog.rules.studio.model.base =
"http://ilog.rules.studio/model/base.ecore">
 <name>local_vars</name>
 <variables name="ps" type="java.util.ArrayList"
   initialValue=""verbalization="ps"/>
</ilog.rules.studio.model.base:VariableSet>
\end{verbatim}

  Rules of the RIF-PRD workflow skeleton are mapped into ILOG
\textit{ruleflow}~\cite{25-kal}:

\vspace*{-6pt}

\noindent
  \begin{verbatim}
flowtask portfolio$_$flow {
 property mainflowtask = true;
 property ilog.rules.business_name =
  "portfolio_flow";
 body {
  portfolio$_$flow#getPortfolios;
  fork {
   portfolio$_$flow#getRates;
   portfolio$_$flow
   #computePortfolioFinancialMetrics;} &&
  { portfolio$_$flow#getTweets;
   portfolio$_$flow#
    computePortfolioTwitterMetrics;}
  portfolio$_$flow#choosePortfolio;
 }
};

ruletask portfolio$_$flow#getPortfolios {
 property ilog.rules.business_name =
   "portfolio_flow>getPortfolios";
 body { getPortfolios.*}
};

ruletask portfolio$_$flow#
   computePortfolioTwitterMetrics {
 propertyilog.rules.business_name =
  "portfolio_flow>
   computePortfolioTwitterMetrics";
 body { computePortfolioTwitterMetrics.* }
};

ruletask portfolio$_$flow#getTweets {
 property ilog.rules.business_name =
  "portfolio_flow>getTweets";
 property ilog.rules.package_name = "";
 body {getTweets.*}
};
\end{verbatim}

  The {\sf computePortfolioTwitterMetrics},
{\sf computePortfolioFinancialMetrics}, and {\sf choosePortfolio} tasks are
implemented as production rules in ILOG:

\vspace*{-6pt}

\noindent
  \begin{verbatim}
package computePortfolioTwitterMetrics {
 use ps;
 import portfolio.*;

 rule computePortfolioTwitterMetrics {
  property status = "new";
  when {	IlrContext() from ?context;	}
  then {
   foreach (Portfolio p in ps) {
    double ?twitter_metrics = 0;
    int ?length = 0;
     foreach (Security security
       in p.securities) {
      ?twitter_metrics= ?twitter_metrics +
       security.positive_tweets;
      ?length = ?length + 1; }
     p.twitter_positive_ratio=
      ?twitter_metrics / ?length;
}}}}
\end{verbatim}

  The {\sf getPortfolios} and {\sf computeSecurityFinancialMetrics} tasks are
implemented by the following production rules in ILOG:


\noindent
  \begin{verbatim}
package getPortfolios {
 use ps;
 import portfolio.*;

 rule getPortfolios {
  when { IlrContext() from ?context; }
  then {
   ps = Supervisor.getPortfolios(startDate,
    endDate);
} } }
\end{verbatim}

\begin{table*}\small
\begin{center}
\Caption{Metrics for the securities}
  \vspace*{2ex}

  \begin{tabular}{cccc}
  \hline
Security identifier&Expected return&Standard deviation&Positive tweet ratio\\
\hline
COG&0.163&0.201&0.507\\
DO&0.015&0.019&0.651\\
EQR&0.150&0.022&0.846\\
FOSL&0.513&0.030&0.579\\
SCG&0.050&0.010&0.622\\
\hline
\end{tabular}
\end{center}
%\end{table*}
%\begin{table*}\small
\begin{center}
\Caption{Metrics for the portfolio candidates}
  \vspace*{2ex}

  \begin{tabular}{lcccccc}
  \hline
\multicolumn{1}{c}{\tabcolsep=0pt\begin{tabular}{c}Portfolio\\ identifier\end{tabular}}&
\tabcolsep=0pt\begin{tabular}{c}Expected\\ return\end{tabular}&
\tabcolsep=0pt\begin{tabular}{c}Standard\\ deviation\end{tabular}&
\tabcolsep=0pt\begin{tabular}{c}Risk free\\ rate\end{tabular}&
\tabcolsep=0pt\begin{tabular}{c}Sharpe\\ ratio\end{tabular}&
\tabcolsep=0pt\begin{tabular}{c}Positive\\ tweet ratio\end{tabular}&
\tabcolsep=0pt\begin{tabular}{c}Sharpe ratio\\$\times$\;Positive tweet ratio\end{tabular}\\
\hline
1&0.111&0.008&0.0169&11.755&0.660&7.758\\
2&2.400&0.507&0.0169&\hphantom{9}4.701&0.508&2.388\\
3&2.381&0.508&0.0169&\hphantom{9}4.662&0.557&2.597\\
4&2.347&0.505&0.0169&\hphantom{9}4.606&0.708&3.261\\
5 (best)&0.178&0.011&0.0169&14.227&0.641&9.120\\
6&0.147&0.008&0.0169&15.577&0.521&8.166\\
\hline
\end{tabular}
\end{center}
\vspace*{-3pt}
\end{table*}



\noindent
  Here, the {\sf Supervisor} is the~Java class wrapping execution of logic programs
in multidialect infrastructure including two nodes~[1]. The nodes correspond to the
mediation system (which integrates \textit{Google Finance} and the \textit{Yahoo!
Finance} services) and to a~rule-based programming system DLV.

  The {\sf getSecurityFinancialMetrics} task uses the same instance of the
mediation system as the {\sf getPortfolios} task. The reason is that financial
metrics are calculated using the historical rates of the securities. This is exactly the
information that is extracted by the mediation system from {Google Finance}
and {Yahoo! Finance}. The difference between two tasks is that the
{\sf getPortfolios} is implemented as a~submission of a~query to the DLV node, but
the {\sf getSecurityFinancialMetrics} is implemented as a~submission of a~different
query to the Mediation Node.

  The {\sf getPositiveTweetRatio} task is implemented by the following
production rule in ILOG:

%\pagebreak

\noindent
  \begin{verbatim}
package getTweets {
 use ps;
 import portfolio.*;

 rule getTweets {
  when { IlrContext() from ?context; }
  then {
   foreach (Portfolio p in ps) {
    foreach (Security s in p.securities) {
     s.positive_tweets =
      WebServices.computeSecPosTweets(s.ticker);
} } } } }
\end{verbatim}




\noindent
Here, {\sf WebServices} is the~Java-class wrapping invocation of a~web-service.
The WSDL specification of the web-service can be found at {\sf
http://synthesis.ipi.ac.ru/synthesis/ projects/RuleInt/OpinionFinderWS}. The
  web-service, in its turn, encapsulates a~Java-program. The program first collects
tweets using the {\sf Twitter Streaming API}. After that, a~sentiment analysis is
done by the {\sf Polarity Classifier} of the {\sf OpinionFinder}
  tool~\cite{14-kal} which assesses if tweet is positive, negative, or neutral. Finally,
the sentiment ratio for every security in a~portfolio is calculated and returned as the
result.

\vspace*{-6pt}

\subsection*{A.4\	Result of~the~use case workflow execution}

  \noindent
  The results obtained by one of the use case runs are as follows. The task
{\sf getPortfolios} computes portfolio candidates on the basis of historical rates of
daily closing prices of securities from S\&P500 list for the 2011--2013. Six portfolios
of size~5 were calculated. Each portfolio is a~set of identifiers (tickers) of
companies:
  \begin{verbatim}
Candidate 1: { ALXN, BF.B, EW, POM, VNO }
Candidate 2: { BMC, JBL, LUK, MNST, POM }
Candidate 3: { AVP, BMC, JPL, MNST, POM }
Candidate 4: { ALTR, BF.B, BMC, DGX, PEG }
Candidate 5: { COG, DO, EQR, FOSL, SCG }
Candidate 6: { ADSK, GILD, INTC, POM, TJX }
\end{verbatim}

  The task {\sf getSecurityFinancialMetrics} computes the expected return and
the standard deviation for every security mentioned in portfolio candidates.
 The task
{\sf getPositiveTweetRatio} computes positive sentiment ratios for every security
mentioned in portfolio candidates (500~latest tweets for every security were used for
the computation). Financial and twitter metrics for several securities are provided in
Table~1.



  The task {\sf computePortfolioFinancialMetrics} computes financial metrics for
every portfolio candidate on the basis of respective metrics for
securities in a~portfolio. The task {\sf computePortfolioTwitterMetrics} computes sentiment
metrics for every portfolio candidate on the basis of sentiment metrics for securities in
a~portfolio. Financial and twitter metrics for portfolio candidates are provided in
Table~2. The task {\sf choosePortfolio} identifies the best portfolio by maximum
value of the products of Sharpe ratio and positive tweet ratio obtained for every
portfolio (see Table~2).


}

\vspace*{-9pt}

\Ack
\noindent
This research has been done under the support of the \mbox{RFBR} (projects13-07-00579,
14-07-00548) and the Program for Basic Research of the Presidium of RAS.

\renewcommand{\bibname}{\protect\rmfamily References}

\vspace*{-9pt}

{\small\frenchspacing
{%\baselineskip=10.8pt
\begin{thebibliography}{99}

\bibitem{1-kal}
\Aue{Kalinichenko, L.\,A., S.\,A.~Stupnikov, A.\,E.~Vovchenko, and D.\,Y.~Kovalev}.
2013. Conceptual declarative problem specification and solving in data intensive domains.
\textit{Informatics and Applications}~--- \textit{Inform \mbox{Appl.}} 7(4):112--139.
Available at: {\sf http://synthesis.ipi.ac.\linebreak ru/synthesis/publications/13ia-multidialect}
 (accessed December~9, 2014).
\bibitem{2-kal}
\Aue{Kalinichenko, L.\,A., S.\,A.~Stupnikov, and D.\,O.~Martynov}. 2007.
\textit{SYNTHESIS: A~language for canonical information modeling and mediator
definition for problem solving in heterogeneous information resource environments}.
Moscow: IPIRAN. 171~p.
\bibitem{3-kal}
Boley, H., and M.~Kifer, eds. 2013. {RIF framework for logic dialects. W3C
recommendation}. 2nd ed. Available at:
{\sf http://www.w3.org/TR/2013/REC-rif-fld-20130205/}
(accessed December~9, 2014).

\bibitem{4-kal}
Boley, H., and M.~Kifer, eds. 2013. {RIF basic logic dialect. W3C Recommendation}.
2nd ed. Available at:
{\sf http://www.w3.org/TR/2013/REC-rif-bld-20130205/}
(accessed December~9, 2014).


\bibitem{5-kal}
Heymans, S., and M.~Kifer, eds. 2009. {RIF core answer set programming dialect}.
Available at: {\sf http:// ruleml.org/rif/RIF-CASPD.html} (accessed November~5, 2014).
\bibitem{6-kal}
\Aue{Leone, N., G.~Pfeifer, W.~Faber, T.~Eiter,  G.~Gottlob, S.~Perri, and F.~Scarcello}.
2006. The DLV system for knowledge representation and reasoning. \textit{ACM Trans.
Comput. Logic} 7(3):499--562.
\bibitem{7-kal}
DeSante, M.\,C., G.~Hallmark, and A.~Paschke, eds. 2013. {RIF production rule
dialect. W3C Recommendation}. 2nd ed.
Available at: {\sf http://www.w3.org/TR/2013/REC-rif-prd-20130205/}
(accessed December~9, 2014).

\bibitem{8-kal}
Motik, B., P.\,F.~Patel-Schneider, and B.~Parsia, eds.
2012. {OWL~2 Web Ontology Language structural
specification and functional-style syntax. W3C Recommendation}. 2nd ed.
Available at:
{\sf http://www.w3.org/TR/owl2-syntax/} (accessed November~5, 2014).
\bibitem{22-kal} %9
\Aue{Calvanese, D., G.~De Giacomo, D.~Lembo, M.~Lenzerini, A.~Poggi, and
R.~Rosati}.
 2007. Ontology-based database access. \textit{15th
Italian Symposium on Advanced Database Systems Proceedings}. 324--331.

\bibitem{9-kal} %10
\Aue{Ramakrishnan, L., and B.~Plale}. 2010. A~multi-dimensional classification model for
scientific workflow\linebreak characteristics. \textit{1st Workshop (International) on Workflow
Approaches to New Data-Centric Science Proceedings}. New York: ACM.
Article No.\,4. 12~p.
Available at: {\sf http://dl.acm.org/citation.cfm?id=1833402}\linebreak
(accessed December~9, 2014).

\bibitem{17-kal} %11
\Aue{Boukhebouze, M, Y.~Amghar, A.-N.~Benharkat,  and Z.~Maamar}. 2011.
A~rule-based approach to model and verify flexible business processes. \textit{Int.
J.~Business Process Integration Management} 5(4):287--307.

\bibitem{21-kal} %12
Polleres, A., H.~Boley, and M.~Kifer, eds. 2013. {RIF datatypes and Built-Ins~1.0
W3C Recommendation.} 2nd ed.
Available at: {\sf http://www.w3.org/TR/2013/REC-rif-dtb-20130205/}
(accessed December~9, 2014).




\bibitem{16-kal} %13
Production Rule Representation (PRR), Version 1.0. OMG Document Number:
formal/2009-12-01. Available at: {\sf http://www.omg.org/spec/PRR/1.0} (accessed
November~5, 2014).

\bibitem{15-kal} %14
\Aue{Yu,~J., and R.~Buyya}. 2005. A~taxonomy of scientific workflow systems for grid
computing. \textit{ACM SIGMOD Records} 34(3):44--49.

\bibitem{18-kal} %15
\Aue{Kowalski, R., and F.~Sadri}. 2009. Integrating logic programming and production
systems in abductive logic programming agents.
\textit{Web reasoning and rule systems}. Eds. A.~Polleres and T.~Swift.
Lecture notes in computer science ser. Springer.
5837:1--23.

\bibitem{19-kal} %16
\Aue{Cosentino, V., M.\,D.~Del Fabro, and A.~El Ghali}. 2012. A~model driven approach
for bridging ILOG rule language and RIF. \textit{6th Symposium (International) on Rules
RuleML Proceedings}. CEUR-WS.org. 874:96--102.
\bibitem{20-kal} %17
\Aue{Veiga, F.\,D.\,J.} 2011. Implementation of the RIF-PRD.  Universidade
Nova de Lisboa. Master Thesis. Available at: {\sf
http://run.unl.pt/bitstream/10362/6310/1/Veiga\_\linebreak 2011.pdf} (accessed November~5, 2014).

\bibitem{10-kal} %18
\Aue{Markowitz, H.\,M.} 1991. \textit{Portfolio selection: Efficient diversification of
investments}. Wiley. 402~p.
\bibitem{11-kal} %19
\Aue{Sharpe, W.\,F.} 1966. Mutual fund performance. \textit{J.~Business}
39(S1):119--138.
\bibitem{12-kal} %20
\Aue{Bollen,~J., H.~Maoa, and X.~Zeng}. 2011. Twitter mood predicts the stockmarket.
\textit{J.~Comput. Sci.} 2(1):1--8.
\bibitem{13-kal} %21
IBM WebSphere ILOG JRules Version~7.0. Online documentation. Available at: {\sf
http://pic.dhe.ibm.com/\linebreak infocenter/brjrules/v7r0/index.jsp} (accessed November~5, 2014).

\bibitem{24-kal} %22
IBM Operational Decision Manager. Available at:
{\sf http:// www-03.ibm.com/software/products/en/odm} (accessed November~5, 2014).
\bibitem{25-kal} %23
IBM Operational Decision Manager Version~8.5 Information Center. Available at: {\sf
http://pic.dhe.ibm.com/\linebreak infocenter/dmanager/v8r5/index.jsp} (accessed November~5, 2014).


\bibitem{14-kal} %24
\Aue{Wilson, T., J.~Wiebe, and P.~Hoffmann}. 2005.
Recognizing contextual polarity in phrase-level sentiment Analysis. \textit{Conference on
Human Language Technology and Empirical Methods in Natural Language Processing
Proceedings}. Stroudsburg: Association for Computational Linguistics. 347--354.


%\bibitem{23-kal}
%Bock, C., \textit{et. al.}, eds. 2012. \textit{OWL~2 Web Ontology Language Structural
%Specification and Functional-Style Syntax. W3C Recommendation}. 2nd ed.


\end{thebibliography} } }

\end{multicols}

\vspace*{-9pt}

\hfill{\small\textit{Received November 3, 2014}}

\vspace*{-18pt}

\Contr

\noindent
\textbf{Kalinichenko Leonid A.} (b.\ 1937)~---
 Doctor of Science in physics and mathematics, professor;
 Head of Laboratory, Institute of Informatics Problems, 44-2 Vavilov Str.,
 Moscow 119333, Russian Federation; professor,
 Faculty of Computational Mathematics and Cybernetics, M.\,V.~Lomonosov Moscow
 State University, 1-52 Leninskiye Gory, GSP-1, Moscow 119991,
 Russian Federation; leonidandk@gmail.com

 \vspace*{3pt}

 \noindent
 \textbf{Stupnikov Sergey A.} (b.\ 1978)~---
 Candidate of Science (PhD) in technology, senior scientist,
 Institute of Informatics Problems, Russian Academy of Sciences,
 44-2 Vavilov Str.,
 Moscow 119333, Russian Federation; ssa@ipi.ac.ru

 \vspace*{3pt}

 \noindent
 \textbf{Vovchenko Alexey E.} (b.\ 1984)~---
 Candidate of Science (PhD) in technology, senior scientist,
 Institute of Informatics Problems, Russian Academy of Sciences,
 44-2 Vavilov Str.,
 Moscow 119333, Russian Federation; itsnein@gmail.com

 \vspace*{3pt}

 \noindent
 \textbf{Kovalev Dmitry Yu.} (b.\ 1988)~---
 junior scientist, Institute of Informatics Problems, Russian Academy of Sciences,
 44-2 Vavilov Str.,
 Moscow 119333, Russian Federation; dm.kovalev@gmail.com


%\vspace*{24pt}

%\hrule

%\vspace*{2pt}

%\hrule

%\vspace*{-6pt}

\newpage


\def\tit{КОНЦЕПТУАЛЬНОЕ МОДЕЛИРОВАНИЕ МУЛЬТИДИАЛЕКТНЫХ ПОТОКОВ РАБОТ$^*$}

\def\aut{Л.\,А.~Калиниченко$^{1,2}$, С.~Ступников$^1$, А.~Вовченко$^1$, Д.~Ковалев$^1$}


\def\titkol{Концептуальное моделирование мультидиалектных потоков работ}

\def\autkol{Л.\,А.~Калиниченко, С. Ступников, А. Вовченко, Д. Ковалев}

{\renewcommand{\thefootnote}{\fnsymbol{footnote}}
\footnotetext[1]{Работа выполнена при поддержке РФФИ (проекты
13-07-00579, 14-07-00548) и~Программы фундаментальных исследований Президиума РАН.}}


\titel{\tit}{\aut}{\autkol}{\titkol}

\vspace*{-12pt}

\noindent
$^1$Институт проблем информатики Российской академии наук

\noindent
$^2$Московский государственный университет им.\ М.\,В.~Ломоносова, факультет вычислительной
матема-\linebreak
$\hphantom{^1}$тики и~кибернетики

\vspace*{6pt}

\def\leftfootline{\small{\textbf{\thepage}
\hfill ИНФОРМАТИКА И ЕЁ ПРИМЕНЕНИЯ\ \ \ том\ 8\ \ \ выпуск\ 4\ \ \ 2014}
}%
 \def\rightfootline{\small{ИНФОРМАТИКА И ЕЁ ПРИМЕНЕНИЯ\ \ \ том\ 8\ \ \ выпуск\ 4\ \ \ 2014
\hfill \textbf{\thepage}}}


\Abst{Рассматриваются методы концептуального представления
алгоритмов анализа данных, средств интеграции данных, а~также процессов,
направленных на спецификацию семантики данных и~поведения в~единой парадигме.
Расширяется новый подход к~применению комбинации семантически различных
плат\-фор\-мо\-не\-за\-ви\-си\-мых языков на правилах (диалектов) для создания
интероперабельных концептуальных спецификаций над различными системами на правилах.
Подход опирается на методику трансформации программ на правилах, рекомендованную
стандартом W3C Rule Interchange Format (RIF). Подход, предлагаемый в~стандарте RIF,
сочетается со технологией семантической интеграции неоднородных баз данных
в~предметных посредниках. Статья расширяет предыдущие исследования авторов
в~направлении моделирования потоков работ для определения композиций
алгоритмических модулей в~процессной структуре. Рассмотрены возможности
спецификации задач в~мультидиалектных потоках работ с~применением семантически
различных языков, наиболее подходящих для конкретных задач. Приведен практический
пример потока работ, задачи которого специфицированы с~использованием нескольких
 языков на правилах (RIF-CASPD, RIF-BLD, RIF-PRD). Для определения концептуальной
 схемы использован язык OWL~2, для оркестровки потока работ использован язык
 RIF-PRD. Инфраструктура реализации примера включает систему на продукционных
 правилах (IBM ILOG), систему на логических правилах (DLV) и~предметный посредник.}

\KW{концептуальные спецификации; потоки работ; RIF; языки продукционных правил;
интеграция баз данных; посредники; PRD; мультидиалектная инфраструктура}

\DOI{10.14357/19922264140413}

\vspace*{6pt}


 \begin{multicols}{2}

\renewcommand{\bibname}{\protect\rmfamily Литература}
%\renewcommand{\bibname}{\large\protect\rm References}

{\small\frenchspacing
{%\baselineskip=10.8pt
\begin{thebibliography}{99}

\bibitem{1-kal-1}
\Au{Kalinichenko L.\,A., Stupnikov S.\,A.. Vovchenko~A.\,E.,
Kovalev~D.\,Y.}
Conceptual declarative problem specification and solving in data intensive domains~//
Информатика и~её применения, 2013. Т.~7. Вып.~4. С.~112--139.
{\sf http://synthesis.ipi.ac.ru/synthesis/publications/13ia-multidialect}.
\bibitem{2-kal-1}
\Au{Kalinichenko L.\,A., Stupnikov~S.\,A., Martynov~D.\,O.}
 SYNTHESIS: A~language for canonical information modeling and mediator definition
 for problem solving in heterogeneous information resource environments.~---
 Moscow: IPI RAN, 2007. 171~p.
\bibitem{3-kal-1}
RIF framework for logic dialects. W3C Recommendation~/
Eds. H.~Boley, M.~Kifer. 2nd ed.
{\sf http:// www.w3.org/TR/2013/REC-rif-fld-20130205/}.
\bibitem{4-kal-1}
RIF basic logic dialect. W3C Recommendation~/
Eds. H.~Boley, M.~Kifer. 2nd ed.
{\sf http://www.w3.org/ TR/2013/REC-rif-bld-20130205/}.
\bibitem{5-kal-1}
RIF core answer set programming dialect~/
Eds.\ S.~Heymans, M.~Kifer, 2009. {\sf  http://ruleml.org/rif/RIF-CASPD.html}.
\bibitem{6-kal-1}
\Au{Leone N., Pfeifer G., Faber~W., Eiter~T., Gottlob~G., Perri~S., Scarcello~F.}
The DLV system for knowledge representation and reasoning~//
 ACM Trans. Comput. Logic, 2006. Vol.~7. No.\,3. P.~499--562.
\bibitem{7-kal-1}
RIF production rule dialect. W3C Recommendation~/
Eds.\ De Sante Marie~C., Hallmark~G., A.~Paschke.~ 2nd ed.
{\sf http://www.w3.org/TR/2013/REC-rif-prd-20130205/}.
\bibitem{8-kal-1}
OWL~2 Web Ontology Language Structural Specification and Functional-Style Syntax.
W3C Recommendation~/ Eds. B.~Motik, P.\,F.~Patel-Schneider, B.~Parsia. 2nd ed.
{\sf http://www.w3.org/TR/owl2-syntax/}.

\bibitem{22-kal-1} %9
\Au{Calvanese, D., De Giacomo~G., Lembo~D., Lenzerini~M., Poggi~A.,
Rosati~R.}
Ontology-based database access~// 15th Italian Symposium on Advanced
Database Systems Proceedings, 2007. P.~324--331.

\bibitem{9-kal-1} %10
\Au{Ramakrishnan L., Plale~B.}
A~multi-dimensional classification model for scientific workflow characteristics~//
1st  Workshop (International) on Workflow Approaches to New Data-Centric Science
Proceedings.  New York: ACM, 2010. Aricle No.\,4. 12~p.
{\sf http://dl. acm.org/citation.cfm?id=1833402}.
\pagebreak

\bibitem{17-kal-1} %11
\Au{Boukhebouze M., Amghar~Y., Benharkat~A.-N., Maamar~Z.}
A~rule-based approach to model and verify flexible business processes~//
Int. J.~Business Process Integration Management, 2011. Vol.~5. No.\,4. P.~287--307.

\bibitem{21-kal-1} %12
RIF Datatypes and Built-Ins 1.0. W3C Recommendation~/
Eds.\ A.~Polleres, H.~Boley, M.~Kifer. 2nd ed.
{\sf http://www.w3.org/TR/2013/REC-rif-dtb-20130205/}.



%\pagebreak

\bibitem{16-kal-1} %13
Production Rule Representation (PRR), Version~1.0.
OMG Document Number: formal/2009-12-01. {\sf http:// www.omg.org/spec/PRR/1.0}.

\bibitem{15-kal-1} %14
\Au{Yu J., Buyya~R.}
 A~taxonomy of scientific workflow systems for grid computing~//
 ACM SIGMOD Records, 2005. Vol.~34. No.\,3. P.~44--49.

 \bibitem{18-kal-1} %15
\Au{Kowalski R., Sadri~F.}
Integrating logic programming and production systems in abductive logic programming
agents~//
Web reasoning and rule systems~/ Eds. A.~Polleres, T.~Swift.
Lecture notes in computer science ser.~--- Springer, 2009. Vol.~5837. P.~1--23.
\bibitem{19-kal-1} %16
\Au{Cosentino V., Del Fabro~M.\,D., El Ghali~A.}
 A~model driven approach for bridging ILOG rule language and RIF~//
 6th  Symposium (International) on Rules, RuleML 2012 Proceedings.  2012.
 CEUR-WS.org. Vol.~874. P.~96--102.
\bibitem{20-kal-1} %17
\Au{Veiga F.\,D.\,J.}
Implementation of the RIF-PRD. Universidade Nova de Lisboa, 2011. Master Thesis.


\bibitem{10-kal-1} %18
\Au{Markowitz H.\,M.}
Portfolio selection: Efficient diversification of investments. Wiley, 1991.
402~p.
\bibitem{11-kal-1} %19
\Au{Sharpe, W.\,F.}
Mutual fund performance~// J.~Business, 1966. Vol.~39(S1). P.~119--138.

\bibitem{12-kal-1} %20
\Au{Bollen J., Maoa H., Zeng~X.}
Twitter mood predicts the stock market~// J.~Comput. Sci., 2011. Vol.~2. No.\,1.
P.~1--8.

\bibitem{13-kal-1} %21
IBM WebSphere ILOG JRules Version 7.0. Online documentation.
{\sf http://pic.dhe.ibm.com/infocenter/\linebreak brjrules/v7r0/index.jsp}.

 \bibitem{24-kal-1} %22
 IBM Operational Decision Manager.
 {\sf http://www-03.\linebreak ibm.com/software/products/en/odm}.
\bibitem{25-kal-1} %23
IBM Operational Decision Manager Version~8.5 Information Center.
{\sf http://pic.dhe.ibm.com/infocenter/\linebreak dmanager/v8r5/index.jsp}.

\bibitem{14-kal-1} %24
\Au{Wilson T., Wiebe~J., Hoffmann~P.}
Recognizing contextual polarity in phrase-level sentiment analysis.
\textit{Conference on Human Language Technology and Empirical Methods in Natural
Language Processing Proceedinhgs}.
Stroudsburg: Association for Computational Linguistics, 2005. P.~347--354.

\end{thebibliography}
} }

\end{multicols}

 \label{end\stat}

 \vspace*{-3pt}

\hfill{\small\textit{Поступила в редакцию 03.11.2014}}
%\renewcommand{\bibname}{\protect\rm Литература}
\renewcommand{\figurename}{\protect\bf Рис.} %7
\def\stat{karpov}

\def\tit{КОГНИТИВНЫЕ ИССЛЕДОВАНИЯ АССИСТИВНОГО 
МНОГОМОДАЛЬНОГО ИНТЕРФЕЙСА ДЛЯ~БЕСКОНТАКТНОГО 
ЧЕЛОВЕКО-МАШИННОГО ВЗАИМОДЕЙСТВИЯ$^*$}

\def\titkol{Когнитивные исследования ассистивного 
многомодального интерфейса для~бесконтактного 
%человеко-машинного 
взаимодействия}

\def\autkol{А.\,А.~Карпов}
\def\aut{А.\,А.~Карпов$^1$}

\titel{\tit}{\aut}{\autkol}{\titkol}

{\renewcommand{\thefootnote}{\fnsymbol{footnote}}\footnotetext[1]
{Работа выполнена при поддержке Минобрнауки РФ в рамках ФЦП <<Исследования и 
разработки>>, госконтракт №\,11.519.11.4025; Совета по грантам Президента РФ, проект МК-1880.2012.8; фонда 
<<Научный Потенциал>> и Комитета по науке и высшей школе Правительства Санкт-Петербурга.}}


\renewcommand{\thefootnote}{\arabic{footnote}}
\footnotetext[1]{Санкт-Петербургский институт информатики и автоматизации 
Российской академии наук (СПИИРАН), karpov@iias.spb.su}



\Abst{Представлены результаты исследований многомодального пользовательского 
интерфейса, предназначенного для бесконтактного управления персональным компьютером 
при помощи речевого ввода и указательных жестов/движений головой. Данный 
многомодальный интерфейс использует низкобюджетное аудио- и видеооборудование для 
одновременного захвата многоканальных сигналов и обеспечивает универсальный доступ к 
компьютерным сис\-те\-мам как обычных операторов для бесконтактной (без использования 
рук) работы с компьютером, так и пользователей с ограниченными физическими 
возможностями (с проблемами двигательных функций рук или даже не имеющих 
рук/пальцев). Описаны методики и результаты количественной оценки 
производительности бесконтактного человеко-машинного взаимодействия с применением 
элементов когнитивных экспериментов и сравнение с результатами для стандартных 
контактных способов указательного ввода информации.}

\KW{многомодальный интерфейс; распознавание речи; машинное зрение; ассистивные 
информационные технологии}

\vskip 14pt plus 9pt minus 6pt

      \thispagestyle{headings}

      \begin{multicols}{2}

            \label{st\stat}

\section{Введение}
  
  Многие люди не могут полноценно работать с компьютерными системами 
(печатать тексты, работать в Интернете, рисовать, и~т.\,д.) по причине 
физических ограничений, например ампутации рук в результате войн, аварий, 
врожденных дефектов или парализации рук в результате болезней. Для таких 
людей и создается многомодальный пользовательский интерфейс 
бесконтактного взаимодействия с компьютером посредством речевого ввода и 
отслеживания осмысленных движений (жестов) головы или тела человека. 
Согласно общепринятому определению, <<жест>> (от лат.\ \textit{gestus}~--- движение 
тела)~--- это некоторое действие или движение человеческого тела или его части 
(например, рук, головы или глаз), имеющее определенное значение или смысл. 
В~этом смысле жестом может являться кивок, покачивание или наклон головы, 
а также указательные жесты, когда пользователь головой показывает на 
определенное направление движения. 
  
  Разработанный за последние годы в лаборатории речевых и многомодальных 
интерфейсов \mbox{СПИИРАН} ассистивный (предназначенный для помощи) 
пользовательский интерфейс получил сокращенное название \mbox{ICanDo} (<<Я могу 
делать>>), что рас\-шиф\-ро\-вы\-ва\-ет\-ся как ``Intellectual Computer AssistaNt for 
Disabled Operators'' (<<Интеллектуальный компьютерный помощник для 
операторов-инвалидов>>)~[1]. Он снабжен программными технологиями 
автоматического распознавания русской речи\,/\,го\-ло\-со\-вых команд и 
технического зрения для отслеживания движений головы (указательных 
жес\-тов) с целью управления курсором \mbox{мыши} на экране дисплея, что повышает 
естественность и эффективность че\-ло\-ве\-ко-машинного взаимодействия. Речевое 
взаимодействие является наилучшей альтернативой любым устройствам ввода 
для\linebreak
 задачи набора текста на компьютере как для поль\-зовате\-лей-ин\-ва\-ли\-дов, так 
и для обычных пользователей. Видео- и аудиосигналы одновременно и 
па\-раллельно захватываются одним аппаратным\linebreak устройством~--- цифровой 
видеокамерой (веб-ка\-ме\-рой) и синхронно обрабатываются в многомодальном 
интерфейсе.
  
  Альтернативой программному человеко-ма\-шин\-но\-му интерфейсу для 
пользователей без верхних конечностей могут служить различные аппа\-ратные 
устройства для управления графическим интерфей\-сом компьютера, например 
ап\-па\-рат\-но-про\-грам\-мные устройства слежения~--- трекеры головы (зарубежная 
сис\-те\-ма InterTrax, которая использует гироскоп; сис\-те\-ма SmartNav, которой 
\mbox{необходим} инфракрасный приемопередатчик; оптическая сис\-те\-ма HeadMouse 
Extreme). Чтобы использовать данные системы для управления курсором мыши 
на экране дисплея, пользователь должен надеть на голову специальное 
устройство (шлем или очки виртуальной реальности со встроенным 
микроминиатюрным гироскопом в случае InterTrax, либо специальную 
конструкцию со светоотражающими метками в случае SmartNav или 
HeadMouse Extreme). Кроме того, для этой задачи могут также применяться 
специальные устройства со светодиодами и аккумуляторами, например 
комплект для ассистивного управления компьютером КАУ-09-1, 
разработанный в ЗАО НПК ФАТУМ, или цветными реперными 
(контрольными) точками-мишенями, которые крепятся на специальном шлеме, 
на\-де\-ва\-емом на голову, например аппаратная система <<Шлемомышь>>~[2], 
разработанная лабораторией компьютерной графики факультета 
вычислительной математики и кибернетики МГУ. Реперные точки на таких 
устройствах отслеживаются посредством инфракрасной либо цифровой 
видеокамеры. Однако пользователи и психологи говорят о том, что люди не 
хотят использовать для человеко-машинного взаимодействия специальные, 
носимые на голове или теле аппаратные устройства, значительно снижающие 
естественность взаимодействия и мобильность передвижения из-за наличия 
проводов, кабелей, аккумуляторов для автономной работы, их общей 
громоздкости и технических сложностей в калибровке и установке. Кроме того, 
люди без рук не могут надеть такое устройство сами себе на голову.
  
  Для некоторых задач и для определенных категорий 
  пользователей-инвалидов (например, парализованных лежачих людей) 
перспективно применение аппаратно-программных систем для\linebreak трекинга 
направления взгляда. Такие системы в мире существуют, например зарубежные 
трекеры глаз Eyegaze System или Visual Mouse, но их использование и 
внедрение на практике осложняется тем, что необходимо использовать очень 
дорогие высокоскоростные цифровые видеокамеры высокой четкости с 
большим разрешением, так как область глаза незначительна по размеру и 
сложна в распознавании. Кроме того, как показывают когнитивные 
исследования~[3], использование отслеживания направления взгляда для 
управления курсором мыши намного сложнее для обучения и хуже, чем 
отслеживание движений головы, по следующим показателям: 
производительность, эмоциональная нагрузка на пользователя, удобство 
использования, эргономичность.
  
  Кроме того, в качестве альтернатив бесконтактному взаимодействию можно 
упомянуть управление манипулятором-мышью с использованием ног вместо 
рук или специальный тактильный манипулятор, функционирующий за счет 
изменения положения центра масс тела человека, сидящего на специальной 
<<подушке>>~[4]. В~будущем, возможно, будут доступны и системы 
взаимодействия на основе прямого интерфейса мозг--ком\-пьютер, во всяком 
случае, разработки в области нейроинформатики активно ведутся как за 
рубежом, так и в России.
  
  В ассистивном интерфейсе ICanDo, которому посвящена данная статья, 
реализованы и применены программные средства компьютерного зрения для 
обнаружения лица человека в оптическом потоке на основе характерных 
органов/черт лица (нос, глаза, губы) без использования искусственных 
маркеров/мишеней и специализированных, носимых человеком, устройств, что 
выгодно отличает его от имеющихся аналогов. Программная система 
взаимодействия не накладывает дополнительных ограничений на пользователя 
и обеспечивает естественность и комфорт при бесконтактной работе с 
компьютером. Применяемые в интерфейсе голосовые команды, 
распознаваемые автоматической сис\-те\-мой, являются прекрасной альтернативой 
стандартным органам ввода информации (клавиатура) как для инвалидов без 
рук или пальцев рук, так и для обычных пользователей.

\section{Архитектура ассистивного многомодального интерфейса 
пользователя}
  
  Ассистивный интерфейс человеко-машинного взаимодействия относится к 
классу многомодальных пользовательский интерфейсов~[5] и использует две 
естественные входные модальности: речь на русском языке и указательные 
жесты~--- движения головы (вверх, вниз, вправо, влево и в любых 
промежуточных направлениях). Обе модальности являются активными~[6] и 
инициируются напрямую человеком, поэтому они непрерывно отслеживаются 
и обрабатываются интеллектуальными подсистемами интерфейса. Каждая из 
модальностей передает свою семантическую информацию: положение головы 
определяет положение курсора мыши на рабочем столе компьютера в 
конкретный момент времени, а речевой сигнал передает информацию о 
действии, которое должно быть выполнено с некоторым объектом 
графического пользовательского интерфейса. На рис.~1 представлена 
архитектура ап\-па\-рат\-но-программного комплекса ассистивного 
многомодального интерфейса.

\begin{figure*} %fig1
 \vspace*{1pt}
 \begin{center}
 \mbox{%
 \epsfxsize=161.784mm
 \epsfbox{kar-1.eps}
 }
 \end{center}
 \vspace*{-9pt}
\Caption{Архитектура ассистивного многомодального пользовательского интерфейса}
\end{figure*}

  Многомодальный интерфейс способен распознавать несколько десятков 
голосовых команд для управления компьютером (например, <<открыть>>, 
<<сохранить>>, <<левая>>, <<правая>>, <<ввод>> и~т.\,д.). Всего система 
содержит 40~голосовых команд, которые являются наиболее часто 
используемыми командами при работе с графическим пользовательским 
интерфейсом русскоязычного варианта операционной системы семейства 
Microsoft Windows. Теоретически возможно работать с компьютером, 
используя лишь левую и правую кнопку мыши (команды <<левая>> и 
<<правая>>), однако введение дополнительных голосовых команд позволяет 
серьезно ускорить и упростить процесс человеко-машинного взаимодействия и 
производительность работы.
  
  Подаваемые голосовые команды захватываются встроенным в видеокамеру 
дистанционным микрофоном, передаются в цифровом виде в компьютер и 
интерпретируются автоматической системой распознавания русской речи. 
Помимо этой системы для управления курсором мыши используется 
технология компьютерного зрения, от\-сле\-жи\-ва\-ющая перемещения головы 
пользователя. Координаты курсора мыши <<привязываются>> к координатам 
кончика носа и иных лицевых органов, автоматически определяемых на 
изображении, таким образом любые движения головы вызывают смещение 
курсора на экране в соответствующем направлении. Объединение такого 
интерфейса с речевым интерфейсом позволяет пользователям не только 
бесконтактно работать с графическим интерфейсом компьютера, подавая 
отдельные команды голосом, но и набирать текст в любом существующем 
редакторе или форме, используя виртуальную клавиатуру или произнося текст 
по буквам. Альтернативно для ввода текста может использоваться интерфейс 
Dasher~[7], который является сторонней ассистивной программой для 
указательного ввода букв на разных языках.
  
\subsection{Автоматическая обработка аудиовизуальных сигналов}
  
  Интерфейс способен обрабатывать одноканальный аудиосигнал от 
микрофона и распознавать голосовые команды на русском языке, разработаны 
также аналогичные версии для английского и французского языка. Для 
распознавания русской речи применяется оригинальная система 
автоматического распознавания речи, получившая название SIRIUS (SPIIRAS 
Interface for Recognition and Integral Understanding of Speech)~[8]. В~системе для 
параметризации звука используется разновидность спектральной обработки 
сигнала~--- мел-час\-тот\-ные кепстральные коэффициенты с их первой и 
второй производными. Акустическое моделирование звуков речи в системе 
производится с применением непрерывных скрытых марковских моделей 
(СММ) первого порядка~[9] и смесей нормальных (гауссовских) распределений 
плотностей вероятностей векторов наблюдений в состояниях СММ. Для 
лучшего учета вариативности разговорной речи каждое слово преобразуется в 
последовательность произносимых фонем (звуков речи) и строится 
вероятностная модель для каждой фонемы. С~помощью алгоритма Витерби 
вычисляется вероятность принадлежности последовательности векторов 
наблюдений СММ некоторого слова~[9]. 
  
  Для задачи голосового управления (работа с персональным компьютером 
относится к этой категории приложений), где применяется малый словарь 
распознавания, лексикон системы представляет собой линейный список всех 
команд с их фонематическими транскрипциями и может достаточно прос\-то 
дополняться. Все голосовые команды \mbox{ICanDo} можно условно разделить на 
четыре класса по их функциональному назначению: 
  \begin{enumerate}[(1)]
\item команды, заменяющие управление кнопками и регуляторами 
манипулятора-мыши (например, <<левая>>, <<правая>>, <<двойной клик>>, 
<<прокрутка вниз>> и~т.\,д.); 
\item команды, заменяющие нажатие клавиш клавиатуры (например, <<ввод>>, 
<<удалить>>, <<регистр>>, цифры, буквы и~т.\,д.); 
\item команды управления графическим пользовательским интерфейсом 
(например, <<открыть>>, <<сохранить>>, <<печать>>, <<пуск>> и~т.\,д.); 
\item специальные команды (<<калибровка>>). 
\end{enumerate}
  
  Нужно отметить, что лишь команды, заменяющие работу мыши, являются 
фактически многомодальными, так как они используют информацию о 
положении курсора мыши в текущий момент времени. Все остальные являются 
исключительно речевыми одномодальными командами, и при их выполнении 
положение курсора не учитывается.
  
  В многомодальном интерфейсе для управления курсором мыши используется 
подсистема компьютер\-ного зрения, отслеживающая указательные движения 
головы пользователя. Применяется программный модуль для отслеживания 
движений головы пользователя, реализованный на основе базового алгоритма 
Лукаса--Канаде (Lukas--Kanade)~\cite{10kar} и его более поздней 
пирамидальной модификации~[11] для анализа оптического потока, т.\,е.\ 
изображение видимого движения объектов, поверхностей или краев сцены, 
получаемое в результате перемещения наблюдателя относительно сцены или, 
наоборот, сцены относительно наблюдателя. В~системе производится 
автоматическое отслеживание пяти естественных точек на лице: центр верхней 
губы, кончик носа, точка между глаз, зрачок правого глаза и зрачок левого 
глаза. Первоначальный поиск головы человека на статических изображениях 
(последовательных ви\-део\-кад\-рах с разрешением $640\times480$~пикселей и 
частотой до 25~кадров в секунду, получаемых от видеокамеры) производится 
методом AdaBoost с применением алгоритма Вио\-лы--Джон\-са 
  (Viola--Jones)~[12]. Изображение сканируется рамкой-окном заданного 
размера и строится пирамида копий объектов. Построенная пирамида 
анализируется заранее обучен\-ны\-ми каскадами Хаара, и на изображении 
находятся графические области, отвечающие заданной визуальной 
модели~[13]. Реализованный метод детекции головы находит прямоугольные 
графические области на изображении, с высокой степенью вероятности 
содержащие изображение лица человека. Введено ограничение: размер такой 
области должен быть не менее $220\times250$~пикселей, чтобы захватывать 
только одно лицо в кадре, достаточно близко расположенное по отношению к 
видеокамере, а кроме того, это ускоряет процесс обработки видеопотока. 
  
  В отличие от имеющей аналогичное предназначение канадской системы 
Nouse~[14], в которой отслеживается только положение кончика носа для 
управления движением курсора мыши, в ICanDo для более робастного 
слежения за перемещением головы оператора используется набор из 
5~естественных лицевых объектов.

\subsection{Синхронизация сигналов и~объединение информации}

  В интерфейсе для объединения информации и выполнения многомодальной 
команды необходимо учитывать координаты указателя мыши, актуальные для 
момента времени непосредственно\linebreak
 перед произнесением голосовой команды 
пользователем, т.\,е.\ должна сохраняться определенная история координат 
положения курсора. Если же использовать координаты указателя, актуальные 
на момент окончания произнесения голосовой команды, то многомодальная 
команда может оказаться неверной, так как курсор может сместиться от 
запланированного положения из-за непроизвольных перемещений головы (а 
они всегда существуют при говорении). В~этом аспекте состоит 
принципиальное отличие указателя, управляемого движениями головы, от 
управляемого аппаратными манипуляторами наподобие мыши, трекбола, 
сенсорного экрана и~т.\,д.
  
  Звуковой сигнал, непрерывно захватываемый дистанционным стационарным 
микрофоном и передаваемый в компьютер посредством звуковой платы, 
обрабатывается модулем автоматического распознавания речи. Процесс 
распознавания речи запускается программным модулем детекции границ речи, 
который обнаруживает наличие в звуковом сигнале речевого фрагмента, 
отличного от тишины или постоянного фонового шума. Процесс распознавания 
заканчивается после получения наилучшей гипотезы распознавания голосовой 
команды из автоматической системы. Синхронизация модальностей 
производится следующим образом: текущее положение курсора сохраняется в 
буфере системы в первый момент определения наличия речи оператора 
(срабатывания алгоритма поиска границ речи по значению энергии сегментов 
сигнала). По окончании процесса распознавания команды модуль 
распознавания речи дает сигнал на объединение информации и выполнение 
многомодальной команды. Таким образом, именно модуль распознавания речи 
осуществляет синхронизацию модальностей в бимодальном интерфейсе.
  
  Для объединения информации, поступающей от двух модальностей, 
используется фреймовый метод позднего объединения, когда поля 
определенной структуры (фрейма) заполняются данными по мере их 
поступления, а по окончании процесса распознавания выполняется 
многомодальная команда. Поля семантического фрейма следующие: текст 
голосовой команды, абсцисса точки положения указателя мыши, ордината 
точки положения указателя, тип речевой команды (многомодальная или 
одномодальная). Если распознанная команда является 
многомодальной, она объединяется в единую команду с сохраненными 
координатами курсора и автоматически отсылается сообщение виртуальному 
устройству мыши о выполнении нужного действия. Если же голосовая команда 
одномодальна, то посылается соответствующее сообщение виртуальному 
устройству клавиатуры с кодом клавиши или сочетанием кодов. Движения 
головы сами по себе не могут подавать команд управления графическим 
пользовательским интерфейсом, однако они могут использоваться, например, 
для создания изображений в графических редакторах.
  
\section{Когнитивные исследования пользовательского интерфейса}
  
  При помощи многомодального интерфейса ICanDo был проведен ряд 
экспериментов, которые были ориентированы на изучение организации 
бесконтактного взаимодействия человека с машиной и использовали элементы 
когнитивных исследований.

\subsection{Методика исследований}

  Экспериментально была проведена оценка скорости и производительности 
работы пользователей с бесконтактным интерфейсом при указании на объекты 
графического пользовательского интерфейса. Для оценки скорости ввода 
информации была использована методология международного стандарта 
ISO~9241-9:2000 ``Requirements for non-keyboard input devices'' (Требования 
к неклавитаурным устройствам ввода информации)~[15], которая базируется на 
экспериментах и законах, разработанных в середине XX~в.\ американским 
пси\-хо\-ло\-гом-когни\-ти\-ви\-стом Полом Фиттсом (Paul Morris Fitts) и впоследствии 
развитых другими учеными~[16]. Применяемая в данном исследовании 
методика оценки интерфейса состоит в следующем. Тестеры, используя 
предоставленное им устройство указательного ввода, должны насколько 
возможно быстро отмечать на экране набор целей-объектов (последовательно 
кликнуть на них, т.\,е.\ дать голосовую команду <<левая>> для нажатия левой 
кнопки мыши), появляющихся по круговой схеме на мониторе. При этом 
порядок целей задается программой автоматически таким образом, чтобы 
пользователь последовательно выделял наиболее удаленно расположенные 
друг от друга объекты, совершая движения указателем в различных 
направлениях~[17]. Когда нажатием на кнопку происходит подтверждение 
выделения текущего объекта-цели на экране, отображается следующая цель. 
При этом автоматически вычисляется индекс сложности задачи ID
(\textit{index of difficulty}), измеряемый в битах согласно формуле~[18]:
  \begin{equation}
\mathrm{ID}=\log_2 \left( \fr{D}{W}+1\right)\,,
  \label{e1kar}
  \end{equation}
где $D$~--- расстояние между центрами целей; $W$~--- диаметр цели. 
  
  Однако координаты точки, где происходит щелчок кнопкой мыши, зависят 
как от фактического (\textit{effective}) расстояния между точками кликов, так и 
от фактического диаметра самих целей (т.\,е.\ чем меньше цель, тем сложнее 
попасть по ее центру). Поэтому фактический индекс сложности выражается 
следующей формулой~[18]: 
  \begin{equation}
\mathrm{ID}_e=\log_2\left( \fr{D_e}{W_e}+1\right)\,.
  \label{e2kar}
  \end{equation}
Здесь $D_e$~--- фактическое расстояние между точками кликов двух последних 
целей; $W_e$~--- фактический диаметр (или ширина) цели, определяемый 
в~[18] как
\begin{equation}
W_e=4{,}133 \sigma\,,
\label{e3kar}
\end{equation}
где $\sigma$~--- среднеквадратическое отклонение координат точки выделения 
(клика), проецируемой на ось, которая соединяет центры начальной и конечной 
целей. Получаемые значения ID$_e$ отличаются от значений ID, более точно 
учитывая качество выполнения тестового задания пользователем.

\begin{figure*} %fig2
 \vspace*{1pt}
 \begin{center}
 \mbox{%
 \epsfxsize=157.675mm
 \epsfbox{kar-2.eps}
 }
 \end{center}
 \vspace*{-9pt}
\Caption{Схема и порядок расположения целей на экране для проведения когнитивных 
экспериментов с интерфейсом по методу Фиттса~(\textit{а}) и реальный пример траектории 
движения курсора мыши на экране при бесконтактном выполнении задания~(\textit{б})}
\end{figure*}
 
  
  Для проведения эксперимента было разработано соответствующее 
программное обеспечение, которое позволяет произвольно задавать 
значения~$D$ и~$W$, а также фиксировать результаты прохождения теста. 
Программа для ЭВМ предлагает пользователю последовательно кликнуть на 
16~целей, которые по очереди появляются на экране согласно рис.~2,\,\textit{а}. 
На рис.~2,\,\textit{б} показан реальный пример получившейся траектории 
движения курсора мыши на экране при бесконтактном выполнении задания 
посредством ICanDo. Здесь можно видеть, что данная задача для пользователя 
не была простой, но ошибок выделения (непопаданий по целям) он не 
допустил.


\subsection{Анализ результатов экспериментов}

  Для выполнения тестового задания были привлечены четыре пользователя-новичка, не имевших ранее опыта работы с многомодальным интерфейсом, и 
два пользователя-эксперта, принимавших участие в ее разработке и отладке. 
Каждым пользователем были проведены серии по 10~тестов с 
последовательным изменением диаметра цели $W$ в пределах от 32 до 
128~пикселей и среднего расстояния~$D$ между целями в пределах 
  96--650~пикселей (использовалось разрешение экрана $1280\times1024$), 
т.\,е.\ показатель~ID варьировался от 1,32 до 4,4~бит. Каждый тест занимал в 
среднем 30--60~с.
  
  Рисунок~3 показывает полученный в результате экспериментов и 
усредненный по всем пользователям график зависимости отношения значений 
ID$_e$ (фактический индекс сложности) и ID (теоретически рассчитанный 
индекс сложности) при разных значениях~$D$ и~$W$. Характерно, что данный 
график лежит выше пунктирной ли\-нии-нор\-ма\-ли (ожи\-да\-емый теоретически 
индекс сложности выполнения задачи), а это означает, что выполнение данной 
задачи оказалось несколько сложнее, чем планировалось. В~противном случае, 
если бы график зависимости лежал ниже нормали, то можно было бы говорить 
о том, что предлагаемая тестерам задача оказалась легче расчетной сложности.

\begin{figure*} %fig3-4
 \vspace*{1pt}
 \begin{center}
 \mbox{%
 \epsfxsize=156.193mm
 \epsfbox{kar-3.eps}
 }
 \end{center}
 \vspace*{-9pt}
 \begin{minipage}[t]{78mm}
\Caption{График зависимости значений фактической сложности ID$_e$ и теоретической 
сложности~ID выполнения задачи и его отклонение от нормали}
\end{minipage}
\hfill
\begin{minipage}[t]{78mm}
\Caption{Графики зависимостей времени движения MT от фактического индекса 
сложности ID$_e$ задачи отдельно для новичков~(\textit{1}) и экспертов~(\textit{2})}
\end{minipage}
\end{figure*}
    
  Согласно экспериментам по методике Фиттса, время движения 
 MT (\textit{movement time}) между двумя целями линейно зависит от индекса 
слож\-ности~ID~[19]. Полученное в ходе экспериментов среднее значение 
MT для всех тестеров равнялось 2550~мс, т.\,е.\ около 2,5~с между 
речевыми <<нажатиями>> цели. Рисунок~4 показывает два 
аппроксимирующих графика зависимости времени движения~MT от 
фактической сложности задачи~ID$_e$ отдельно для поль\-зо\-ва\-те\-лей-но\-вич\-ков, не 
работавших ранее с интерфейсом, и для обученных поль\-зо\-ва\-те\-лей-экс\-пер\-тов. 
Хорошо заметно, что эффект обучения положительно сказывается на 
увеличении скорости бесконтактной работы с компьютером. Также разброс 
значений~MT для новичков оказался значительнее, они выполняли тесты 
менее стабильно. На основании результатов экспериментов можно сказать, что 
новички начинают уверенно работать с компьютером бесконтактно при 
помощи многомодального интерфейса уже через 10--15~мин тренировки 
(исключая этап настройки системы распознавания речи на голосовые 
характеристики\linebreak пользователя), что, конечно же, несколько больше, чем при 
первоначальном овладении мышкой и клавиатурой. Однако через день работы с 
системой пользователь уже может считаться экспертом в бесконтактном 
человеко-машинном взаимодействии.


  В применяемой методике экспериментов \mbox{Фиттса} основным показателем 
оценки интерфейса\linebreak является общая производительность работы пользователя с 
системой~TP (\textit{throughput})~[20], определяющая компромисс между 
временем движения\linebreak (скоростью выполнения задания) и точностью выделения 
цели и измеряемая в битах в секунду согласно следующей формуле: 
  \begin{equation}
\mathrm{TP}=\fr{\mathrm{ID}_e}{\mathrm{MT}}\,.
  \label{e4kar}
  \end{equation}
  
  Полученное в ходе экспериментов среднее значение TP для всех тестеров 
составило 1,2~бит/c, максимальное значение~TP для одного тестера~--- 
2,0~бит/c.
  
  Также в ходе когнитивных исследований была проведена сравнительная 
оценка контактных устройств для ввода/указания, таких как сенсорный экран 
17$^{\prime\prime}$, джойстик, трекбол, сенсорная панель (\textit{touchpad}) 3$^{\prime\prime}$ 
и  стандартный ма\-ни\-пу\-ля\-тор-мышь. Двумя пользователями были проведены 
серии по 10~тестов для каждого устройства с последовательным изменением 
диаметра цели~$W$ в пределах от 32 до 128~пикселей и среднего 
расстояния~$D$ между целями в пределах от 96 до 650~пикселей. Таблица~1 
приводит результаты экспериментов и сравнения всех вышеуказанных 
устройств по трем основным количественным критериям: 
  \begin{enumerate}[(1)]
\item среднее время движения MT между двумя целями; 
\item процент ошибок выделения целей (непопадание курсором в цель); 
\item общая производительность указательного интерфейса~TP. 
\end{enumerate}


  
  Таблица~1 показывает, что наилучшие резуль\-таты по производительности 
интерфейсов были показа\-ны сенсорным монитором, так как рука тес\-те\-ра 
свободно перемещается по воздуху. Управ\-ление курсором посредством 
многомодального интерфейса, отслеживающего движения головы,\linebreak
 уступает по 
производительности практически всем аппаратным контактным средствам 
ввода информации, кроме джойстика (который весьма непривычен для 
управления курсором), однако имеет то преимущество, что является 
бесконтактным способом управления курсором и может применяться 
категориями потенциальных пользователей, для которых стандартные средства 
ввода информации недоступны.

\vspace*{12pt}

\noindent
%\begin{center} %fig3
{{\tablename~1}\ \ \small{Сравнительная оценка эффективности интерфейсов для указательного ввода 
информации с использованием методики Фиттса}}
%\end{center}
%\vspace*{3pt}

%\smallskip
    
{\small
\begin{center}
%\begin{table*}\small
\begin{tabular}{|l|c|c|c|}
\hline
\multicolumn{1}{|c|}{Устройство ввода}&\tabcolsep=0pt\begin{tabular}{c}MT,\\ с\end{tabular}&
\tabcolsep=0pt\begin{tabular}{c}Ошибка\\ выделения,\\  \%\end{tabular}&
\tabcolsep=0pt\begin{tabular}{c}TP,\\ бит/с\end{tabular}\\
\hline
Джойстик&2,01&7,00&1,54\\
Трекбол&1,03&3,83&3,51\\
Сенсорная панель 3$^{\prime\prime}$&0,85&4,50&3,72\\
Манипулятор-мышь&0,49&3,17&6,65\\
Сенсорный экран 17$^{\prime\prime}$&0,50&6,17&7,85\\
Интерфейс ICanDo&1,98&7,33&1,59\\
  \hline
  \end{tabular}
  \end{center}
%  \end{table*}

}

\vspace*{9pt}



\addtocounter{table}{1}

\begin{figure*}[b] %fig5
 \vspace*{1pt}
 \begin{center}
 \mbox{%
 \epsfxsize=118.652mm
 \epsfbox{kar-5.eps}
 }
 \end{center}
 \vspace*{-9pt}
\Caption{Распределение относительной частоты использования голосовых 
команд тестером в ходе эксперимента}
\end{figure*}
    
    Тестирование интерфейса в реальной задаче бесконтактной работы с 
компьютером было также проведено тремя добровольными пользователями. 
Пользователям предлагался определенный сценарий~--- последовательность 
операций, которую\linebreak\vspace*{-12pt}

\pagebreak

\noindent
%\begin{center} %fig3
{{\tablename~2}\ \ \small{Сравнение бесконтактного и контактного интерфейсов человеко-машинного 
взаимодействия}}
%\end{center}

\vspace*{4pt}

%\smallskip
    
{\small
\begin{center}
%\begin{table*}\small
%\begin{center}
\begin{tabular}{|c|c|c|}
\hline
Точность&
\multicolumn{2}{c|}{Время выполнения}\\
распознавания &\multicolumn{2}{c|}{тестового сценария,  с}\\
\cline{2-3}
\tabcolsep=0pt\begin{tabular}{c}голосовых\\ команд, \%\end{tabular}&
\tabcolsep=0pt\begin{tabular}{c}Интерфейс\\ ICanDo\end{tabular}&Мышь\;+\;клавиатура\\
\hline
96&82&43\\
\hline
\end{tabular}
\end{center}
%\end{table*}
}

\vspace*{14pt}
    

\noindent
 пользователи должны были выполнить двумя способами 
(многомодальным~--- посредством \mbox{ICanDo} и стандартным~--- при помощи 
ма\-ни\-пу\-ля\-то\-ра-мы\-ши). Тестовая задача включала в себя элементарные операции 
с текстовым редактором MS Word, а также поиск заданной информации в 
Интернете посредством MS Internet Explorer. Конкретнее: пользователю нужно 
было найти информацию о программе передач на ин\-тер\-нет-пор\-та\-ле Рамблер, 
скопировать интересующий фрагмент этой страницы, открыть текстовый 
редактор MS Word, вставить в пустой документ информацию из буфера, 
сохранить файл на рабочем столе и распечатать данный файл. Таблица~2 
показывает количественные результаты экспериментов и сравнение двух 
способов человеко-машинного взаимодействия (среднее время, требуемое для 
выполнения всего тестового сценария и точность распознавания речи в 
дикторозависимом режиме). 
    

  Многомодальный бесконтактный способ ввода оказался в 1,9~раз медленнее, 
чем стандартный контактный способ, что было очевидно. При этом точность 
распознавания голосовых команд составила свыше 96\% в дикторозависимом 
режиме работы. Однако, если учесть, что аудиосигнал, по\-лу\-ча\-емый от 
встроенного в видеокамеру микрофона, характеризуется невысоким 
отношением сигнал/шум (SNR, signal-to-noise ratio), то полученный результат по точности 
распознавания можно считать приемлемым. Полученная скорость работы 
бесконтактного интерфейса вполне достаточна, так как он разрабатывается для 
помощи людям с физическими ограничениями, в частности для людей без рук 
или с парализованными руками. 
  
  Также был проведен анализ статистики бес\-контактной работы пользователя-эксперта 
  с ин\-тер\-фейсом ICanDo в течение одного дня в задаче навигации 
(серфинга) в Интернете посредством\linebreak браузера MS Internet Explorer. 
Последующий анализ журнала статистики показал, что всего пользователь 
сделал более 750~голосовых команд, при этом некоторые команды были более 
частотными, чем другие, а некоторые команды не использовались вовсе. 
Диаграмма на рис.~5 показывает распределение частотности голосовых команд, 
примененных пользователем. 


  Легко было предсказать заранее, что наиболее популярной окажется команда 
<<левая>> (клик левой кнопкой мыши), которая использовалась более чем в 
трети случаев, включая и ввод текста при помощи специального программного 
обеспечения~--- экранной виртуальной клавиатуры. Однако необходимо сказать, 
что при работе с мышкой и клавиатурой это значение еще выше для подобной 
задачи, так как, работая бесконтактно, пользователи стараются избегать работы 
со сложными многоуровневыми меню стандартных офисных прикладных 
программ, заменяя их <<горячими клавишами>> для быстрого доступа к 
действиям. Все остальные команды распределены более-менее равномерно 
среди оставшихся 62\%. При этом 64\% всех голосовых команд было подано 
многомодально (совместно с движениями головы для выделения графических 
объектов или ссылок на экране), а оставшиеся 36\% команд~--- одномодально. 
  
  Видеодемонстрации бесконтактной работы пользователей с компьютером, в 
том числе и одного человека, не имеющего верхних конечностей, посредством 
ассистивного многомодального интерфейса ICanDo можно посмотреть на 
ин\-тер\-нет-сай\-те лаборатории речевых и многомодальных интерфейсов 
СПИИРАН~\cite{21kar}. 

\section{Заключение}

  В статье представлены результаты иссле\-до\-ваний бесконтактного 
  человеко-машинного вза\-и\-модействия, реализуемого посредством 
ассис\-тив\-но\-го многомодального интерфейса ICanDo,\linebreak
предназначенного 
специально для работы че\-ло\-ве\-ка-опе\-ра\-то\-ра с ЭВМ без использования рук. 
Опи\-сана общая архитектура ассистивного многомодального интерфейса, 
автоматическая обработка\linebreak аудио- и видеосигналов, а также механизмы 
синхронизации и объединения модальностей. В~данном ассистивном 
пользовательском интерфейсе для робастного отслеживания указательных 
жес\-тов/дви\-же\-ний головы оператора используется массив из пяти естественных 
точек на лице: центр верхней губы, кончик носа, точка между глаз, зрачок 
правого глаза и зрачок левого глаза. Применяются голосовые команды для 
бесконтактного управ\-ле\-ния прикладным и системным про\-грам\-мным 
обеспечением компьютера. Результаты проведенных исследований с 
использованием методики %\linebreak
 Фиттса и элементов иных когнитивных 
экспериментов позволяют заключить, что данный многомодальный интерфейс 
обеспечивает приемлемую скорость и производительность работы пользователя 
с компьютером, не сильно отличающуюся от аналогичных показателей для 
стандартных контактных интерфейсов~--- устройств ввода, и может успешно 
применяться для бесконтактного управления как обычными операторами, так и 
потенциальными поль\-зо\-ва\-те\-ля\-ми-ин\-ва\-ли\-да\-ми с грубыми моторными 
нарушениями в функционировании рук и даже вовсе без верхних конечностей.
  
  Применение ассистивного пользовательского интерфейса позволит повысить 
социоэкономическую интеграцию инвалидов в информационном\linebreak
обществе и 
сделает их более независимыми от помощи со стороны других лиц. 
Предложенный бесконтактный интерфейс позволит пользователям\linebreak самим 
выбирать доступные им средства взаимодействия с компьютером, компенсируя 
недоступные модальности альтернативными коммуникативными каналами.

{\small\frenchspacing
{%\baselineskip=10.8pt
\addcontentsline{toc}{section}{Литература}
\begin{thebibliography}{99}

  
\bibitem{1kar}
\Au{Карпов А.\,А.}
ICanDo: Интеллектуальный помощник для пользователей с ограниченными 
физическими возможностями~// Вестник компьютерных и информационных 
технологий, 2007. №\,7. С.~32--41.

\bibitem{2kar}
\Au{Кричевец А.}
Шлемомышь~// Компьютерра, 2002. №\,434. С.~48--51. {\sf 
www.computerra.ru/offline/ 2002/434/16588/}.

\bibitem{3kar}
\Au{Bates R., Istance H.\,O.}
Why are eye mice unpopular? A~detailed comparison of head and eye controlled 
assistive technology pointing devices~// 1st Cambridge Workshop on Universal 
Access and Assistive Technology Proceedings.~--- USA, 2002.

\bibitem{4kar}
\Au{Аграновский А.\,В., Евреинов Г.\,Е., Яшкин~А.\,С.}
Ап\-па\-рат\-но-программные инструментальные средства проектирования 
виртуальных акустических объектов и сцен для слепых пользователей 
персональных компьютеров~// Информационные технологии в образовании: 
Мат-лы IX Междунар. конф.-выс\-тав\-ки.~--- М., 1999.
\bibitem{5kar}
\Au{Карпов А.\,А., Ронжин А.\,Л.}
Многомодальные интерфейсы в автоматизированных системах управления~// 
Известия высших учебных заведений. Приборостроение, 2005. Т.~48. №\,7. 
С.~9--14.

\bibitem{6kar}
\Au{Ронжин А.\,Л., Карпов А.\,А.}
Проектирование интерактивных приложений с многомодальным 
интерфейсом~// Докл. Томского гос. ун-та сис\-тем управ\-ле\-ния 
и радиоэлектроники (ТУСУР), 2010. №\,1. Ч.~1. С.~124--127.

\bibitem{7kar}
\Au{Ward D., Blackwell A., MacKay~D.}
Dasher: A data entry interface using continuous gestures and language models~// 
ACM Symposium on User Interface Software and Technology UIST'2000 
Proceedings.~--- New York: ACM Press, 2000. P.~129--137.

\bibitem{8kar}
\Au{Ronzhin A.\,L., Karpov A.\,A.}
Russian voice interface~// Pattern Recognition and Image Analysis (Advances in 
Mathematical Theory and Applications), 2007. Т.~17. №\,2. С.~321--336.

\bibitem{9kar}
\Au{Карпов А.\,А.}
Аудиовизуальный речевой интерфейс для систем управления и оповещения~// 
Известия Южного федерального ун-та. Технические науки, 2010. 
№\,3(104). С.~218--222.

\bibitem{10kar}
\Au{Lucas B.\,D., Kanade T.}
An iterative image registration technique with an application to stereo vision~// 7th 
Joint Conference (International) on Artificial Intelligence \mbox{IJCAI} Proceedings.~--- 
Vancouver, Canada, 1981. P.~674--679.

\bibitem{11kar}
\Au{Bouguet J.-Y.}
Pyramidal implementation of the Lucas--Kanade feature tracker description of the 
algorithm~// Intel Corporation Microprocessor Research Labs: Report.~--- New 
York, USA, 2000.

\bibitem{12kar}
\Au{Viola P., Jones M.}
Rapid object detection using a boosted cascade of simple features~// IEEE Conference
(International) on Computer Vision and Pattern Recognition Conference 
(CVPR) Proceedings.~--- Kauai, HI, USA, 2001.

\bibitem{13kar}
\Au{Lienhart R., Maydt J.}
An extended set of Haar-like features for rapid object detection~// IEEE 
 Conference (International) on Image Processing (ICIP'2002) Proceedings.~--- Rochester, New York, 
USA, 2002. P.~900--903.

\bibitem{14kar}
\Au{Gorodnichy D., Roth G.}
Nouse `Use your nose as a mouse' perceptual vision technology for hands-free games 
and interfaces~// Image and Vision Computing, 2004. Vol.~22. No.\,12. P.~931--942.

\bibitem{15kar}
ISO 9241-9:2000(E) Ergonomic Requirements for Office Work with Visual Display 
Terminals (VDTs). Part~9: Requirements for Non-Keyboard Input Devices.~--- 
International Standards Organization, 2000.

\bibitem{16kar}
\Au{Soukoreff R.\,W., MacKenzie I.\,S.}
Towards a standard for pointing device evaluation, perspectives on 27~years of Fitts' 
law research in HCI~// Intern. J.~Human Computer Studies, 2004. Vol.~61. No.\,6. 
P.~751--789.

\bibitem{17kar}
\Au{Zhang X., MacKenzie I.\,S.}
Evaluating eye tracking with ISO 9241 Part~9~// Human--Computer Interaction 
 Conference (International) (HCII 2007) Proceedings.~--- Beijing, China: Springer Verlag LNCS 
4552, 2007. P.~779--788.

\bibitem{18kar}
\Au{Carbini S., Viallet J.\,E.}
Evaluation of contactless multimodal pointing devices~//  2nd IASTED Conference (International) 
on Human--Computer Interaction Proceedings.~--- Chamonix, France, 
2006. P.~226--231.

\bibitem{19kar}
\Au{De Silva G.\,C., Lyons M.\,J., Kawato~S., Tetsutani~N.}
Human factors evaluation of a vision-based facial gesture interface~// Workshop on 
Computer Vision and Pattern Recognition for Computer Human Interaction 
Proceedings.~--- Madison, USA, 2003.

\bibitem{20kar}
\Au{Wilson A., Cutrell E.}
FlowMouse: A computer vision-based pointing and gesture input device~// 
Human--Computer Interaction INTERACT Conference Proceedings.~--- Rome, Italy, 
2005. P.~565--578.

\label{end\stat}

\bibitem{21kar}
Видеодемонстрации с интернет-сайта лаборатории речевых и многомодальных 
интерфейсов \mbox{СПИИРАН}. {\sf www.spiiras.nw.ru/speech/demo/demo\_ new.avi}, {\sf 
www.spiiras.nw.ru/speech/demo/ort.avi}.
 \end{thebibliography}
}
}


\end{multicols} %8
\def\stat{markova}

\def\tit{ЛОГИКА БИОГРАФИЧЕСКИХ ФАКТОВ}

\def\titkol{Логика биографических фактов}

\def\autkol{Н.\,А.~Маркова}
\def\aut{Н.\,А.~Маркова$^1$}

\titel{\tit}{\aut}{\autkol}{\titkol}

%{\renewcommand{\thefootnote}{\fnsymbol{footnote}}\footnotetext[1]
%{Работа выполнена при поддержке РФФИ (гранты 09-07-12098, 09-07-00212-а и
%09-07-00211-а) и Минобрнауки РФ (контракт №\,07.514.11.4001).}}


\renewcommand{\thefootnote}{\arabic{footnote}}
\footnotetext[1]{Институт проблем информатики Российской академии наук, nMarkova@ipiran.ru}

  
  \Abst{Предложен метод формализации биографических фактов в виде логических 
формул, который позволяет интегрировать и анализировать данные, получаемые из разных 
источников, а также служит основой повышения эффективности 
справочно-информационного аппарата биографических ресурсов.}
  
  \KW{биографическое исследование; информационный поиск; формализация; 
биографический факт}

\vskip 14pt plus 9pt minus 6pt

      \thispagestyle{headings}

      \begin{multicols}{2}

            \label{st\stat}
  
\section{Введение}
  
  Использование информационных технологий для проведения 
биографических исследований играет важную роль для удовлетворения 
объективно существующей общественной потребности в изучении и 
распространении сведений биографического характера. А~потребность эта 
растет по мере роста общественного интереса к различным вариантам 
<<частной>> истории, в том числе истории науки, краеведения, истории семьи.
  
  Колоссальный объем биографических сведе\-ний хранится в архивах или 
опубликован в труд\-но\-доступных изданиях. Массовая электронная пуб\-ли\-кация 
источников существенно повышает их\linebreak
 доступность, однако потенциал 
информационных технологий по обеспечению эффективности\linebreak
 биографического 
поиска далеко не исчерпан. Име\-ющи\-еся биографические ресурсы (БР) плохо 
сис\-те\-ма\-тизированы, их средства поиска неэффективны, данные противоречивы, 
а часто и недостоверны. Проб\-ле\-мы эти (подробно рассмотренные в работах~[1, 2]) 
усугубляются, когда в процессе исследования объединяются данные, 
получаемые из разных источников. Для того чтобы процесс исследования был 
эффективен, требуется методическая и инструментальная поддержка как со 
стороны спра\-воч\-но-ин\-фор\-ма\-ци\-он\-но\-го аппарата БР, так и со стороны 
организации и систематизации работы исследователя. 
  
  В работе предложен способ представления формализуемых биографических 
данных в виде логических формул, который, с одной стороны~---\linebreak
стороны 
исследователя, позволяет интегрировать данные, получаемые из разных 
источников, проверять непротиворечивость, интерполировать, корректно 
ставить новые исследовательские вопросы. С~другой стороны~--- стороны БР, 
может\linebreak служить концептуальной основой для <<стандартов операционной 
совместимости, метаданных, средств упорядочения информационного 
содержания, интерфейсов доступа к массивам данных в\linebreak
 цифровом формате, 
средств поиска и средств сохранения>>~--- достижения целей, определяемых 
программой ЮНЕСКО <<Информация для всех>>~[3].
  
  В разд.~2 уточняются задачи биографического поиска, 
являющиеся важным звеном в исторических исследованиях самой разной 
направленности. В~разд.~3 рассматриваются существующие модели 
биографических данных, предназначенные для систематизации и упорядочения 
сведений и обмена данными. В~разд.~4 предлагаются основные концепции 
новой модели, конкретизируемые в разд.~5 в виде формул логики 
биографических фактов. В~разд.~6 основные операции 
биографического поиска пред\-став\-ле\-ны в терминах формул логики 
биографических фактов.

\section{Задачи биографического поиска}
  
  Задача биографа~--- собрать и обобщить биографические факты, под 
которыми понимаются высказывания, являющиеся <<ответом на вопросы типа 
кто? что? когда?>>~\cite[с.~53]{4mar}, упорядочить их определенным образом, 
связать между собой и с внешними объектами. Задача обобщения~--- 
интеграции данных~--- включает анализ их непротиво\-ре\-чи\-вости и разрешение 
противоречий, интерполяцию, экстраполяцию, что может приводить к 
постановке новых исследовательских вопросов.
  
  Объект исследования биографического поиска~--- это человек или группа 
людей, принадлежащих к определенному кругу, информация о которых 
сохранилась в источниках. 

Предмет исследования биографического поиска~--- 
это конкретные биографические характеристики, отношения и события, 
связанные с изуча\-емы\-ми людьми. 

Биографический поиск~--- это задача, 
начинающаяся со слова <<найти>>, которая может быть как независимой, так и 
включаться в виде определенного этапа в те или иные исторические 
исследования. В~рамках предлагаемого рассмотрения в биографический поиск 
не включаются задачи при\-чин\-но-след\-ст\-вен\-но\-го анализа исторических явлений, 
теоретических обобщений и художественных построений.
  
  Эффективность биографических исследований во многом определяется 
доступностью информации, в том числе предоставляемыми БР возможностями 
поиска и навигации, а также наглядностью представления найденных данных. 
В~рамках традиционных бумажных БР задача повышения доступности решается 
библиографами, архивистами и редакторами. Сведения об основных лицах, 
относящихся к документам, содержатся (во всяком случае, должны 
содержаться) в каталогах библиотек и описях архивов. Сведения о 
многочисленных\linebreak
 лицах, упоминаемых в монографиях,~--- в соответствующих 
именных указателях. Библиографические справочники сопоставляют именам 
(со\-про\-вож\-да\-емым, возможно, краткими сведениями о\linebreak
 лицах) указания на 
источники, в которых может быть найде\-на соответствующая информация. 
  
  Если направление деятельности библиографа~--- от источника к персонажам, то 
исследователь идет от персонажа к источникам и фактам и от объекта или 
явления~--- к персонажам. В~качестве объекта или явления могут выступать 
как организации, общества, исторические события, так и лица, связи с 
которыми изучаются (<<корреспонденты Гоголя>>, <<учителя Пушкина>>, 
<<ученики Ключевского>>). Аналогичные задачи стоят в рамках изучения 
истории любой сферы деятельности, отрасли, организации, края. В~задачах 
генеалогии (или шире~--- истории семьи) изучаются как объекты <<род>>, 
<<семья>>, так и отдельные лица. 
  
  Отметим взаимосвязь описываемых сторон: для описания источников 
требуется выполнить задачу биографического поиска, а публикация 
результатов исследований приводит к созданию нового источника.
  
  Биографический поиск~--- это многошаговая процедура. На основании 
исходных данных ставится некий исследовательский вопрос, ищутся 
источники, анализируются найденные документы, выявляются факты (или 
констатируется, что источник не содержит релевантных фактов), которые затем 
сопоставляются и интегрируются с ранее обнаруженными. На основе новой 
информации формулируются новые вопросы и~т.\,д. Многие задачи и 
проблемы биографического поиска инвариантны по отношению к виду 
исследований.

\section{Существующие биографические модели}

\vspace*{-6pt}
  
  Определить в общем случае, что должно входить в биографию, невозможно, 
однако в рамках проб\-лем\-ных областей, где возможна частичная формализация, 
такая задача имеет практические решения: от рекомендаций по подготовке 
повествовательных текстов до строгих стандартов представления данных. 
Рассмотрим основные виды биографических моделей, отмечая их достоинства 
и недостатки с точки зрения задач биографического поиска.
  
  Упорядочение работы авторов биографического словаря определяется 
наличием рекомендаций по содержанию и форме представления данных. 
Например, грандиозная работа по созданию биографического словаря русских 
фольклористов методически обеспечена монументальным трудом~\cite{5mar}, 
в котором наряду со специфическими рекомендациями есть и правила 
представления общих биографических и связанных с ними библиографических 
сведений. Такого рода БР рассчитаны на бумажную публикацию, в них нет 
учета возможностей информационных технологий.
  
  Расширение горизонта рассмотрения от конкретной сферы деятельности до 
всех значимых в истории лиц вместе с освоением новых технологических 
возможностей (\textit{wiki}), с одной стороны, и подключением широчайшего 
круга авторов, с другой, демонстрирует портал Персоналии в 
Википедии~\cite{6mar}. Методическим обеспечением для ав\-то\-ров-со\-ста\-ви\-те\-лей 
статей служат шаблоны~--- своего рода биографические модели, каждая из 
которых ориентирована на некоторый круг лиц. Например, шаблон 
<<Ученый>> представляет собой список из двух десятков анкетных статей, 
каждая из которых раскрывается отдельным шаблоном-регламентом. 
Возможность ссылаться (гиперссылками) на другие статьи, относящиеся как к 
людям, так и к другим объектам, а также к доступным в сети источникам~--- 
колоссальное преимущество Википедии. Часть биографических данных можно 
почерпнуть в связанных статьях. \textit{Wiki}-технология дает возможность 
совместно представлять формализованное и неформализованное знание. При 
этом формализация (в отличие от реляционных баз данных) может\linebreak 
осуществляться по ходу накопления данных: подключением новых или 
изменением старых шаб\-ло\-нов. Несовершенства Википедии, возможно, 
час\-тич\-но будут сниматься по мере ее развития.\linebreak
 Отметим некоторые из них. 

Связь между статьями дается простой гиперссылкой, целесообразно расширить 
ее некоторым\linebreak\vspace*{-12pt}

\pagebreak

\noindent
семантическим содержанием (пометкой <<друг>>,
 <<отец>>), 
что уже предполагают современные микроформаты~\cite{7mar}. Другое 
необходимое расширение, которое, к сожалению, пока даже не декларируют 
создатели семантического \textit{Web},~--- это определение динамики связей. 
Привязка событий и связей в жизни человека к временн$\acute{\mbox{о}}$й оси~--- важнейшее 
условие успешности биографического поиска.
  
  Важным шагом в сторону эффективного представления биографических 
данных является новая редакция Российского коммуникативного формата 
представления авторитетных данных в машиночитаемой форме 
(RUSMARC)~\cite{8mar}. 

Для лиц, причастных к созданию документов или 
упоми\-на\-емых в них, записи RUSMARC определяют имя и идентифицирующие 
признаки (даты жизни, специальность, область деятельности, титулы, звания, 
степени и~т.\,п.). Аналогичные сведения полагаются для объектов 
библиографического описания <<Род>> (семья) и <<Организация>>. 
Предполагается фиксация связей по родству, работе, культурной общности, 
местожительству и~т.\,п. Как и Википедия, RUSMARC позволяет 
фиксировать связи между людьми и объектами другой природы. И~эти связи 
также, к сожалению, не помечаются хронологическими рамками. Большое 
внимание в RUSMARC уделяется вариативности в именованиях лиц, 
организаций, документов~--- типичной причины проблем биографического 
поиска. 
  
  Основная сложность внедрения RUSMARC~--- отсутствие необходимого 
числа библиографов, которые могли бы заполнить соответствующие записи, в 
то время как успех Википедии во многом определя\-ет\-ся подключением 
широчайшего круга лиц к созданию, проверке, редактированию\linebreak статей. 
  
  Стандартом де-фак\-то для представления 
биографической информации на протяжении долгих лет является давно 
устаревшая модель Ge\-ne\-a\-logi\-cal 
Data Communications (\mbox{GEDCOM}), революционное, основанное на \textit{xml} 
обновление которой \mbox{GEDCOM}~6.0~\cite{9mar} было выпущено в 2002~г., но 
до сих пор фактически никем не используется. Конкурентом \mbox{GEDCOM}, тоже 
определяемым как спецификация формата обмена данными между 
генеалогическими программами, является стандарт GenXML~\cite{10mar}. 
Помимо более строгой структурной упорядоченности он несет в себе 
несколько принципиально новых положений, отражающих практику 
биографических исследований. В~частности, в нем явным образом 
определяется процесс исследования: введены понятия <<свидетельство>> и 
<<заключение>>. Но главное, GenXML открыт для добавления новых типов 
атрибутов и событий. К~сожалению, ни в GEDCOM, ни в GenXML не 
отражены важнейшие свойства биографической информации: временн$\acute{\mbox{а}}$я 
изменчивость и взаимная зависимость характеристик. 
  
  В рамках просопографических исследований~\cite{11mar} задача построения 
обобщенной модели и не ставится. Каждое исследование предполагает свою 
проблемно-ориентированную информационную модель. 
  
  Таким образом, в настоящее время концептуальных моделей, в полном 
объеме отражающих специфику биографических исследований, не существует.

\section{Концептуальная модель биографических данных} 
  
  Построим концептуальную модель, описы\-ва\-ющую биографические данные, 
для которых возмож\-но формализованное представление. Постараемся учесть 
все недочеты и достоинства существующих моделей. Дадим общее 
неформальное описание проблемной области биографических исследований.
  
  Человек рождается, умирает, действует сам или подвергается воздействиям 
окружения в исторической реальности. Некоторая часть сведений о нем 
фиксируется документально и попадает в информационное пространство 
(ИП)~--- на бумажные и другие твердые носители, а в последние десятилетия и 
в электронные ресурсы. Информационное пространство 
является компонентом исторической реальности. 
Выделенные компоненты ИП~--- биографии конкретных лиц~--- существенно 
различаются по объему: от нескольких слов до нескольких томов. Кроме них 
биографические сведения рассыпаны по ИП~--- они содержатся в биографиях 
лиц из круга общения, в исторических описаниях событий и явлений, в 
документах учета, в библиографических списках и~т.\,д. Задача 
биографического поиска~--- собрать рассыпанные в ИП данные, касающиеся 
конкретных лиц, конкретных событий, объектов, явлений.
  
  Уточним вводимые понятия и термины.

\subsection{Историческая реальность}
  
  С биографической точки зрения объектами исторической реальности 
  ($b$-объектами) являются:
  \begin{itemize}
\item люди (персонажи, лица);
\item общественные образования (государства, учреж\-де\-ния, 
общества и др.);
\item физические объекты (географические, технические, естественные);
\item исторические события и процессы; 
\item отрасли деятельности.
\end{itemize}

  Между $b$-объектами существует объективно или могут быть определены в 
рамках той или иной интерпретации различные отношения ($b$-отношения). 
Объективны биологические $b$-отношения, например отношение 
  <<ребенок--родители>>. Влияние на творчество писателя произведений его 
пред\-ше\-ст\-вен\-ни\-ка~--- пример субъективно интер\-пре\-ти\-ру\-емо\-го $b$-отношения.
  
  Как выделение $b$-объектов, так и определение их свойств и $b$-отношений 
является результатом абстрагирования, необходимого для целей сбора и 
сис\-те\-ма\-ти\-за\-ции данных. Будем рассматривать только данные, для которых 
возможно формализованное представление. Интерпретация тонких вопросов, 
связанных с психологией, этикой, мировоззрением, творчеством,~--- задача 
соответствующих профессионалов.
  
  Все $b$-объекты, а также большинство $b$-от\-но\-ше\-ний существуют и 
изменяются во времени. Собственно <<биографией>> является 
информационный объект, в котором в динамике или интегрально\linebreak представлены 
свойства и характеристики некоторого лица, а также его $b$-отношения с 
другими\linebreak
 $b$-объек\-та\-ми и в какой-то мере их свойства и характеристики. 
Характеристики $b$-объек\-та или $b$-от\-но\-ше\-ния будем называть 
  $b$-ха\-рак\-те\-ри\-сти\-ка\-ми. Личными $b$-ха\-рак\-те\-ри\-сти\-ка\-ми являются 
составляющие генотипа и фенотипа человека, в частности состояние здоровья. 
Гражданское состояние, имущественное состояние, сословие, чин, звание, сан, 
титул характеризуют не только человека, но и соответствующую социально-правовую 
организацию общества, а точнее~--- $b$-от\-но\-ше\-ние между ними.
  
  Примером $b$-отношения, на первый взгляд не зависящего от исторического 
контекста, является отношение местопребывания, связывающее человека и 
объект географического пространства. Формально в конкретный момент 
времени местопребывание может быть охарактеризовано координатами. 
Однако на практике оно определяется в терминах названий населенных 
пунктов, сопоставление которых с координатами~--- не всегда тривиальная 
задача исторической географии.
  
  Важнейшая $b$-характеристика~--- официальное именование~--- 
определяется на отношении лица и $b$-объекта-государства. 
Другие $b$-объекты могут использовать другие имена данного лица, в част\-ности в 
домашнем обращении или в отношении авторства (псевдонимы).
  
  Между $b$-характеристиками существуют зависимости, регламентируемые 
законами природы или нормативными законами: правилами, юридическими 
актами, традициями. Примерами регламентов являются законодательные 
документы, уставы обществ, штатное расписание учреждения и~т.\,п. 
Существуют закономерности, определяющие до\-пус\-ти\-мые последовательности 
событий~--- смены значений $b$-характеристик, задающие некий шаб\-лон, 
сценарий или набор ограничений. Перечислим несколько очевидных: смерть 
следует за рож\-де\-ни\-ем; имеются допустимые пределы разницы между 
рождением человека и границами жизни его родителей; в каждый конкретный 
момент времени человек может находиться только в одной точке 
географического пространства. Для определения большинства нормативных 
законов требуется конкретно-историческое знание.

\subsection{Информационное пространство}
  
  Информационное пространство без ограничения общности можно 
представить как совокупность хранилищ документов. Для пользователей 
электронных хранилищ, реализованных, возможно, в виде баз данных, их 
содержание пред\-став\-ля\-ют виртуальные документы, визуализируемые на 
экране. Документы, в свою очередь, делятся на час\-ти/фрагменты: разделы, 
страницы, абзацы и~т.\,п.\linebreak
 Фрагмент документа редко независим, для его\linebreak 
корректной интерпретации требуется контекст. Элементы ИП также являются 
объектами исторической реальности: у них есть время жизни, они связаны с 
людьми отношениями <<автор>>, <<адресат>>, <<упоминаемое лицо>>.
  
  Документы, их фрагменты или их совокупности~--- хранилища, и их разделы 
идентифицируются адресами. Для электронных хранилищ адрес~--- это 
URL/URI (Uniform Resource Locator/Identifier), ключи базы данных, имена закладок и~т.\,п. Для архивных 
хранилищ~--- номера фонда, описи, дела, листа. Сложнее дело обстоит с 
печатными изданиями. Библиографическая ссылка, вообще говоря, не является 
адресом~--- книгу еще требуется \mbox{найти} в хранилище-библиотеке, где адресом 
ее будут служить соответствующие шиф\-ры хранения. Возможно, однако, что 
книга уже оциф\-ро\-ва\-на, тогда адрес ее~--- тот же URL.
  
  Идеальное решение задач биографического поиска состояло бы во всеобщем 
справочно-ин\-фор\-ма\-ци\-он\-ном аппарате, в котором $b$-объекты были бы 
соотнесены с элементами ИП. На базе такого аппарата, используя возможности 
современных информационных технологий, удалось бы добиться качественно 
нового уровня эффективности биографических исследований. На пути к этому 
идеалу стоят сложнейшие задачи.
  
  Прежде всего, задача описания существующих источников (как бумажных, 
так и электронных)~--- идентификации взаимоотношений между элементами 
ИП и $b$-объектами, их свойствами, хронологией~--- далека от реализации. 
<<Концепция информатизации архивного дела России>>~\cite{12mar}, 
декларируя необходимость обеспечения прав граждан на информацию, на 
самом деле только намечает подходы к решению задач научного описания 
архивных материалов. В~практике отечественных архивов электронные 
описания в основном присутствуют разве что на уровне их крупных единиц~--- 
фондов. 
  
  Существуют два аспекта задачи описания источников: систематизация и 
наполнение. В~части наполнения многое могли бы сделать пользователи, 
читатели, исследователи. Примером деятельности по обмену биографическими 
ссылками (и фактографическими данными) может служить сайт 
ВГД~\cite{13mar}. Пользователь, обладающий доступом к труднодоступному 
источнику, выкладывает его описание на общедоступный ресурс. Но эта 
коммуникация ведется бессистемно: в виде обмена текстовыми сообщениями 
на форуме сайта.
  
  Необходимым условием создания справочно-ин\-фор\-ма\-ци\-он\-но\-го аппарата, 
соотносящего элементы ИП с $b$-объектами, является систематизация их 
описаний, сведение формализуемой части касающихся их сведений в единый 
формат метаданных. 

\section{Биографические характеристики и~логика~фактов}
     
     Для того чтобы иметь возможность сопоставить факты, содержащиеся в 
источниках, оценить их 
не-\linebreak противоречивость, сделать выводы, необходимо 
привести их к некоторому общему, нормализованному представлению. 
Представим биографические сведения в виде совокупности взаимосвязанных 
значений $b$-характеристик на общей временн$\acute{\mbox{о}}$й оси. 

\subsection{Биографические характеристики}
  
  Существенная часть $b$-характеристик формализуема, их можно измерить, 
например, в терминах социологической~\cite{14mar} или 
психологической~\cite{15mar}\linebreak
стратификации. Значения $b$-характеристик~---\linebreak 
статусы (атрибуты), как правило, изменяются во времени. Например, 
отношение сотрудника и учреж\-де\-ния характеризует должность, которая 
изменяется по мере карьерного роста. В~некоторых случаях новое значение не 
заменяет предыдущее, а присоединяется к списку ранее имевшихся 
компонентов значения (пример~--- награды). Ряд $b$-ха\-рак\-те\-ри\-стик имеет 
простые количественные значения, например рост и вес. Артериальное 
давление измеряется парой чисел.
  
  Значения $b$-характеристик в именных шкалах, где присутствуют 
синонимы, многозначны. Существует вариативность именования лиц и 
организаций, даже если речь идет о конкретном моменте времени. Например, 
<<собор Покрова Пресвятой Богородицы, что на Рву>> эквивалентен <<собору 
Василия Блаженного>>. Причиной многозначности могут быть также те или 
иные варианты искажений, а также в целом субъективный характер оценок (про 
рост профессора Ловецкого Герцен утверждал: <<был высокий\ldots\ 
мужчина>>, а Пирогов~--- <<небольшого роста>>).
  
  $B$-характеристика~--- это отображение (в общем случае многозначная 
функция), динамически сопоставляющее $b$-объек\-ту или паре $b$-объек\-тов 
некоторое значение. Область определения $b$-ха\-рак\-те\-рис\-ти\-ки соответствует 
некоторому $b$-от\-но\-шению.
  
  Одно и то же $b$-отношение может быть оценено разными, но 
взаимозависимыми $b$-ха\-рак\-те\-ри\-сти\-ка\-ми в разных шкалах, например рост 
измеряется в саженях, футах или сантиметрах (или же ему дается неформальная 
словесная оценка). Кроме того, между значениями различных 
  $b$-ха\-рак\-те\-ристик связанных между собой $b$-объек\-тов существует 
взаимосвязь, в частности, являясь сотрудником подразделения, человек 
автоматически является и сотрудником учреждения в целом. С~другой 
стороны, являясь сотрудником учреждения, человек работает в некотором 
подразделении, о котором, если оно неизвестно, может быть поставлен 
исследовательский вопрос.
  
  Как области определения, так и области значений для подавляющего 
большинства $b$-ха\-рак\-те\-ри\-стик меняются во времени, само наличие 
  $b$-ха\-рак\-те\-ри\-сти\-ки ограничено определенными временн$\acute{\mbox{ы}}$ми рамками, для 
конкретной исторической ситуации их определяет научное знание 
соответствующей специальной исторической дис\-цип\-ли\-ны, а также социологии, 
антропологии, психологии, биологии. Они же определяют зависимости между 
$b$-ха\-рак\-те\-ри\-сти\-ка\-ми, варианты возможной синонимии значений и другие 
общие для рассматриваемых классов $b$-объек\-тов закономерности. 
Формулировки соответствующих законов природы, нормативных актов, 
традиций в виде зависимостей между значениями $b$-ха\-рак\-те\-ри\-стик будем 
называть $b$-нор\-ма\-ля\-ми. Лишь в редких случаях $b$-нор\-маль может быть 
сформулирована формально, большинство из них задается неформальными 
текстами, проверка соблюдения их правил~--- <<ручная>> процедура. 

\subsection{Нормализованный факт}
 
  Предложим формализацию понятия $b$-ха\-рак\-те\-ри\-сти\-ки. Высказывание, 
фиксирующее, что в данный момент данная $b$-характеристика для данного 
  $b$-объекта (объектов) имела данное значение, назовем формализованным 
фактом. Такое выражение может иметь логическое значение~--- ИСТИНА или 
ЛОЖЬ, быть неизвестным, а может быть оценено неким промежуточным 
образом: <<Скорее, ИСТИНА, чем ЛОЖЬ>>. Если информация об 
интересующем $b$-объекте или группе объектов будет представлена в виде 
набора формализованных фактов, тогда, применяя соответствующие 
  $b$-нормали, можно их сопоставить, выявить и разрешить противоречия, 
сформировать новые факты-следствия, интегрировать данные в общую 
картину.
  
  Большинство $b$-характеристик сохраняет свои значения на протяжении 
некоторого промежутка времени.  Чтобы отразить это 
фундаментальное свойство исторической реальности, введем понятие 
нормализованный факт (НФ). Назовем нормализованным фактом  следующую 
логическую формулу:
  \begin{equation}
  \left( \forall t\in \Delta t\right) \beta(p,q,t)=a\,.
  \label{e1mar}
  \end{equation}
  Здесь 
  $\beta$~--- $b$-характеристика;
  $p$ и $q$~--- $b$-объекты;
  $a\hm\in  \mathrm{Im}\left(\beta\right)$~--- конкретное значение $b$-характеристики из области 
ее значений;
  $t$~--- время;
  $\Delta t$~--- период времени, когда $b$-характеристика неизменна.
  
  Компоненты НФ полагаем некоторым информационным представлением 
соответствующих сущностей, например в виде идентификаторов, текстов, 
чисел.
  
  В формуле~(\ref{e1mar}) представлена характеристика двуместного 
  $b$-от\-но\-ше\-ния. Для одноместных $b$-от\-но\-ше\-ний будем использовать 
нотацию~$\beta (p, t)$. Трехместные отношения, а также отношения большей 
местности с помощью логических формул могут быть сведены к двуместным.
  
  Если $b$-характеристика принимает логическое значение, будем опускать 
его значение в записи, полагая $\beta (p, q, t)$ эквивалентным 
$$\beta (p, q, t) = \mbox{ИСТИНА}\,.
$$ 

Наконец, для краткости в формуле~(\ref{e1mar}) 
будем опускать время или использовать нотацию 
$$
\beta (p, q, \Delta t) = a\,.
$$ 

\subsection{Формулы логики фактов}
  
  Расширим понятие НФ для различных предикатов. Помимо <<$=$>> в 
формуле~(\ref{e1mar}) будем использовать <<$\not=$>>; <<$<$>> и 
  <<$>$>>~--- для упорядоченных значений\linebreak
   $b$-ха\-рак\-те\-ри\-стик; а также 
<<$\in$>> и <<$\not\in$>>~--- если в правой части не единичное значение, а 
множество. Кроме того, правая часть может представлять значение 
  $b$-характеристики для другого $b$-объекта, что соответствует текстам: 
<<был одноклассником>>, <<служил в той же должности>>, <<был старше 
чином>>.
  
  Нормализованные факты связаны друг с другом. Формально эти связи представимы в виде 
логических формул. Назовем их формулами логики фактов~--- FF 
(\textit{Fact Formula}). Для сигнатуры формул логики фактов применимы как 
выражения обычной логики предикатов, так и специальные темпоральные 
аппараты. В~любом случае FF включает пропозициональные связки ($\lnot$, 
$\neq$, $\vee$, $\rightarrow$) и кванторы ($\forall$, $\exists$). 
  $B$-характеристики выступают в роли функций, а в качестве предикатов 
используются $b$-характеристики с логическим значением или оценки 
значений $b$-характеристик ($=$, $\not=$, $\in$, $\not\in$, $<$, $>$,\ \ldots).
  
  Дадим индуктивное определение формулы логики фактов~--- FF в 
терминах логики предикатов.
  
  Переменными и константами являются $b$-объ\-ек\-ты $(p, q)$, элементы и 
подмножества из множеств значений $b$-характеристик ($a\hm\in \mathrm{Im}\left(\beta\right)$, 
$A\hm\subset \mathrm{Im}\left(\beta\right)$) и время ($t$), а также различные варианты 
временн$\acute{\mbox{ы}}$х периодов (отрезок, интервал, полуинтервал):
  \begin{align*}
\mathrm{Term}&::= \beta (p, q, t) \vert a \vert  A\vert t \vert \\
& \vert [t_1 - t_2] \vert (t_1 - t_2) \vert 
[t_1 - t_2) \vert (t_1 - t_2]\,;\\
 \mathrm{Atom}&::= \mathrm{Term}\ \rho\  \mathrm{Term} \ \left(\rho\in \{=, \not=, \in, \not\in, 
 <,  >,\ldots\}\right)\\
\mathrm{FF}&::=\mathrm{Atom} \vert \neg \mathrm{FF} \vert\\
  &\vert \mathrm{FF}_1 \wedge 
  \mathrm{FF}_2\vert  \mathrm{FF}_1 \vee \mathrm{FF}_2 \vert \forall x\ 
\mathrm{FF} \vert  \exists x\ \mathrm{FF}\,.
%  \label{e2mar}
  \end{align*}
  
  Приведем формулировки некоторых $b$-нор\-ма\-лей в терминах логики 
фактов.
  \smallskip
  
  \textbf{Симметрия.} Для большинства двуместных $b$-от\-но\-ше\-ний 
значению  
$\beta (p, q, t)$ однозначно соответствует $\beta^\prime (q, p, t)$: 

\smallskip
   
\textit{МестоРаботы}(<<Иванов>>, 
<<Контора>>)\;$\leftrightarrow$\\[-9pt]

\hspace*{10mm}$\leftrightarrow$\;\textit{Сотрудник}(<<Контора>>,  <<Иванов>>)\,.
  
  \medskip
  
  \textbf{Транзитивность.} Факты, основанные на таких\linebreak
   $b$-ха\-рак\-теристиках, 
как иерархия и местоположение, обладают свойством транзитивности. 
  %
  \noindent
  Например: 
  
  \smallskip
  
  \noindent
  \textit{МестоРаботы}(<<Иванов>>, 
<<Контора>>)\;$\wedge$\\[-9pt]

$\wedge$\;\textit{Местопребывание}(<<Контора>>, 
<<Москва>>)\;$\rightarrow$\\[-9pt]

\hspace*{5mm}$\rightarrow$\;\textit{Местопребывание}(<<Иванов>>, <<Москва>>).

\smallskip
  
 
 \textbf{Вариативность.} Для вариативных $b$-ха\-рак\-те\-ри\-стик, например 
именования, значения разбиваются на классы эквивалентности, определяемые 
$b$-нор\-ма\-ля\-ми, а формула принимает вид принадлежности данному классу. 
Приведем $b$-нор\-маль для имено-\linebreak\vspace*{-12pt}

\pagebreak

\noindent
вания в современной отечественной практике 
(без учета знаков препинания и грамматических форм, в нестрогой форме):

%\pagebreak

\smallskip

\noindent
Имя($x$) = <<Имя>>\;$\wedge$\; %\\

\noindent
\ \ \ $\wedge$\;Фамилия($x$)\;=\;<<Фамилия>>\;$\wedge$ %\\[-9pt]

\noindent
\ \ \ $\wedge$\;Отчество($x$)\;=\;<<Отчество>>\;$\rightarrow$ %\\
 %\\[-9pt]

\noindent
\ \ \ $\rightarrow$\;Именование($x$)\;=\;<<Фамилия\ Имя\ Отчество>>\;$\vee$ %\\[-9pt]

\noindent
\ \ \ $\vee$\;<<Имя\  Отчество\ Фамилия>>\;$\vee$ %\\[-9pt]

\noindent
\ \ \ $\vee$\;<<Фамилия Имя>>\;$\vee$ %\\

\noindent
\ \ \ $\vee$\;<<Имя Фамилия>>\;$\vee$ %\\[-9pt]

\noindent
\ \ \ $\vee$\;<<Имя\ Отчество>>\;$\vee$ %\\

\noindent
\ \ \ $\vee$\;<<И\ О\ Фамилия>>\;$\vee$ %\\[-9pt]

\noindent
\ \ \ $\vee$\;<<Фамилия\ И\ О>>\;$\vee$

\noindent
\ \ \ $\vee$\;<<И\ Фамилия>>\;$\vee$

\noindent
\ \ \ $\vee$\;<<Фамилия\ И>>\,. 

%\smallskip

\subsection{Виды нормализованных фактов}
  
  Рассмотрим два направления классификации НФ: по динамике и по 
определенности.
  \begin{enumerate}[1.]
\item Нормализованные факты представляют различные варианты динамики значений  
$b$-ха\-рак\-те\-ристик:
\begin{itemize}
\item НФ-событие: период соответствует <<мгновению>>: $\Delta 
t\hm=\{t^\prime\}$;
\item НФ-состояние: период протяжен~--- $\Delta t\hm= [t_s, t_b)$, он 
включает начало и не включает конец временн$\acute{\mbox{о}}$го промежутка;
\item цепочка НФ~--- процесс, дизъюнкция формул отдельных состояний: 
$\vee_{i\in [0,\,n-1]} \beta(p,q,[t_i,\,t_{i+1}))$~--- в моменты~$t_i$ 
$b$-характеристика принимает новое значение.
\end{itemize}

  Факт-состояние~--- это элементарный процесс, включающий событие~--- 
переход в данное состояние и ограниченный событием~--- выходом из 
данного состояния. С~другой стороны, интегральное состояние может 
уточняться детальным процессом. 
\item Нормализованные факты представляют раз\-лич\-ные варианты определенности значений\linebreak 
$b$-ха\-рак\-те\-ристик:
\begin{itemize}
\item рамочный НФ~--- факт, часть компонентов которого не определена;
\item строгий НФ~--- факт, для которого все компоненты определены.
  \end{itemize}
  
  Под компонентами факта понимаются время, характеризуемые объекты и 
значения $b$-ха\-рак\-те\-ри\-стики.
  
  Заметим, что строгий факт совсем не обязательно корректен. Произвольная 
формула логики фактов является строгой, если все ее компоненты~--- строгие 
факты, и рамочной во всех других случаях.
  \end{enumerate}
  
  Неполные и неточные данные, оформленные в виде рамочных НФ и 
основанных на них рамочных формул, пусть с пропусками, размытыми 
значениями, неформальными комментариями при накоплении, систематизации, 
анализе способны стать основой для реконструкции биографий.

%\vspace*{-6pt}

\section{Биографический поиск в~терминах логики фактов}
  
  Рамочные НФ представляют удобный механизм для формулировки 
исследовательских вопросов, ответы на которые, возможно, хранятся в 
до\-ку\-мен\-тах-ис\-точ\-ни\-ках. Как интерпретация источников, так и выводы из 
имеющихся фактов, как правило, являются неформальными процедурами. Для 
их выполнения требуется экспертное знание. Тем не менее сам факт 
выполнения операции, ее вход и выход удобно представить как специальные 
формулы логики фактов. Благодаря формализованному представлению ручных 
операций сведения о целях исследования, его текущем состоянии и дальнейших 
шагах складываются в единую картину. 

%\vspace*{-6pt}

\subsection{Формулировка исследовательских вопросов}
  
  Рамочный НФ связан с исследовательским вопросом, касающимся уточнения 
значений его компонентов. Рассмотрим основные варианты таких вопросов.
  \smallskip
  
  \textbf{Хронологические рамки.} Важнейший вопрос биографического 
поиска~--- <<когда?>>. Например, $\mbox{Жизнь}(\mbox{<<Иванов>>}, [t_1 - t_2))$, где 
$t_1$ и~$t_2$ неизвестны. При его постановке, как правило, существуют 
ориентиры, оценки: начало не ранее, конец не позднее~--- или известен некоторый 
промежуток, входящий в искомый. 
  
  \smallskip
  \textbf{Рамки значений.} Исследовательский вопрос состоит в выяснении, 
какое значение принимала данная $b$-характеристика для данного лица в 
данное время. Пример: 

\vspace*{-6pt}

\noindent
\begin{multline*}
\mbox{\textit{СотрудникДолжность}}(\mbox{<<Контора>>},\\ 
\mbox{<<Иванов>>}, \mbox{1890}) \hm= x.
\end{multline*}

  
\noindent
 Область значений: 
 
 \noindent
$$
x\in \mbox{\textit{СписокДолжностей}}(\mbox{<<Контора>>}, 
\mbox{1890})\,.$$ 
  
  Уточнение рамок возможно, если известны значения $b$-ха\-рак\-те\-ри\-сти\-ки в 
некоторые моменты до и после данного времени и, кроме того, ее значения 
упорядочены или известно значение другой $b$-ха\-рак\-те\-ри\-сти\-ки, а между 
  $b$-характеристиками существует зависимость, определяемая $b$-нормалью.
  
  \smallskip
  \textbf{Объектные рамки.} Неопределенность этого рода~--- ответ на 
вопросы <<кто?>>~--- $\mbox{\textit{Муж}}(x, \mbox{<<Петров>>},$ $1909)$ или <<что?>>~--- 
$\mbox{\textit{МестоРаботы}}(\mbox{<<Иванов>>}, y,$ $1890)$. Запись в анкете <<женат>>, 
предполагает наличие $b$-объек\-та~$x$~--- жены, про которую известно лишь 
то, что в указанное время она была замужем за заполнителем анкеты. 
Собственно поиск объекта сводится к поиску его $b$-ха\-рак\-те\-ри\-стик, по 
крайней мере, идентификационных (именования, времени жизни). Какие-то 
ограничения на именование и время жизни содержат уже имеющиеся данные о 
связываемом $b$-объекте. 

%\vspace*{-6pt}

\subsection{Интерпретация текстов источника}
  
  Лишь незначительная часть ИП представляет формализованные факты в 
явном виде. Это в основном фактографические (генеалогические или 
просопографические) базы данных, в которых $b$-ха\-рак\-те\-ри\-сти\-ка 
представляется доменом (колонкой в таблице), а шкала ее значений, возможно, 
выделена в отдельную вспомогательную таб\-ли\-цу-сло\-варь. 
  
  Там, где документ-источник осмысленно размечен, например в анкетных 
группах~--- шаблонах Википедии, формализованные факты представлены 
метаданными. 
  
  Кроме того, несложно выявить формализованные факты, исходя из:
  \begin{itemize}
\item структурной организацией документа-ис\-точ\-ни\-ка, отвечающей 
структуре $b$-объектов;
\item в той или иной степени структурированных текстовых определений, 
сопоставляющих именам $b$-объектов имена связанных с ними $b$-объек\-тов 
или $b$-характеристики;
\item идеографических схем (пример~--- генеалогическое древо), 
определяющих связи между $b$-объ\-ек\-тами.
\end{itemize}

  Структура справочного издания <<Памятная книга губернии>>~\cite{16mar} 
отражает деление губернии на уезды и населенные пункты, принадлежность 
учреждений как административным единицам, так и ведомствам, структуру 
учреждений и должности, звания, чины служащих в них лиц. 
  
  Однако большинство фактов извлечь из текстов может только исследователь. 
Определим операцию интерпретации ($\Rightarrow$), недоступную или 
неэффективную для автоматического логического анализа и выполняемую 
исследователем, которая выводит формулы логики фактов из содержания 
компонента ИП. Например, то, что в тексте речь идет об указанном промежутке 
времени, представимо следующей формулой: $(text) \hm\Rightarrow \exists t \in 
[t_{\min},\,t_{\max}] (text)$.
  
  Другой пример~--- текст содержит сведения о $b$-характеристике~$\beta$ и 
некотором подмножестве ее значений~$A$: $(text) \hm\Rightarrow \exists x (\beta 
(x) \in A) (text)$.
  %
  В~част\-ности, если текст относится к $b$-объекту, именуемому <<$N\!N$>>: 
$ (text) \hm\Rightarrow \exists x (\mbox{\textit{Именование}}(x) \hm = \mbox{<<}N\!N\mbox{>>}) 
(text)$.
  
  Максимально точная оценка биографических границ для текста, реализуемая 
как внутренняя разметка или как внешнее индексирование, должна 
способствовать эффективности биографического поиска.
  
  Интерпретация источника может выявить новые неизвестные 
  $b$-отношения, которые на новом шаге позволят найти дополнительные 
сведения об искомом $b$-объекте. Например, в воспоминаниях одноклассника 
искомого лица мы находим сведения об их учителе (неизвестное ранее 
  $b$-отношение), а уже в переписке учителя~--- сведения об искомом лице.
  
  Немаловажный факт, извлекаемый из текста, состоит в утверждении, что в 
нем нет сведений о данном $b$-объекте или о данной $b$-характеристике.

%\vspace*{-6pt}

\subsection{Выборка биографических сведений из~источника}
  
  Выборку текста так же, как его интерпретацию, представим в терминах 
логики фактов. Введем следующее обозначение: $text \hm= \mbox{\textit{Контент}}(l)$~--- 
интерпретируемый текст~--- это содержание документа (фрагмента документа) 
по адресу~$l$. Расширим множество констант и переменных логики фактов 
произвольными текстами и адресами в~ИП.
  
  Важнейшим фактом, извлекаемым из текста, является отсылка к другим 
компонентам ИП. Пусть из интерпретации текста по адресу~$l_0$ следует,\linebreak что 
сведения относительно $b$-характеристики~$\beta$\linebreak
 $b$-объек\-та~$x$ можно 
найти в документе по адресу~$l_1$, тогда: $\mbox{\textit{Контент}}(l_0) \hm \Rightarrow 
\mbox{\textit{Контент}}(l_1) \wedge \beta (x)$.
  
  Доступ к архивным материалам или редким изданиям затруднен. В~то же 
время необходимый %\linebreak
 для анализа в конкретном исследовании объем данных, как 
правило, бывает много меньше до\-ку\-мен\-та-ис\-точ\-ни\-ка в целом. Поэтому как в 
случае труднодоступных источников, так и для анализа легкодоступных, но 
объемных материалов, применяется практика создания вторичных документов: 
копии фрагмента, выписки, реферата, аннотации. В~терминах логики фактов 
то, что по отношению к совокупности исследовательских вопросов (ff) при 
\textit{Редукции}~--- создании вторичного документа~--- не выпущено ничего 
существенного, фиксируется\linebreak  так:

\smallskip

\noindent
  $$
(\mbox{\textit{Контент}}(l) \wedge\mbox{ff} ) = 
(\mbox{\textit{Редукция}}(\mbox{\textit{Контент}}(l)) 
\wedge \mbox{ff})\,.
$$

%\vspace*{-6pt}

\subsection{Вывод формул логики фактов}
  
  Появление новой формулы логики фактов в багаже исследователя возможно 
и как следствие интерпретации некоторого текста, и как логический вывод из 
имеющихся формул. И~в том, и в другом случае возможны неточности и 
огрехи, вследствие как ошибок исследователя, так и некорректности или 
неполноты исходных данных. В~любом случае полезно делать предположения, 
строить рабочие гипотезы. Для того чтобы отличить гипотезу от 
установленного факта, имеет смысл воспользоваться категориями нечеткой 
логики, т.\,е.\ вместо значений ИСТИНА и ЛОЖЬ, употреблять оценку 
правдоподобия в виде числа в диапазоне от~0 (ЛОЖЬ) до~1 (ИСТИНА). 
В~виде формул нечеткой логики могут быть сформулированы опирающиеся на 
статистику экспертные оценки. Пример: 
  \begin{align*}
&\mbox{\textit{Оценка}}(\mbox{\textit{Служба}}(\mbox{<<Леонтий\ Кириллович>>},\\
& \mbox{<<церковь\ села\  Ловцы>>}, 1780\mbox{--}1790) \wedge{}\\
 &\wedge \mbox{\textit{Служба}}(\mbox{<<Кирилл\ Васильевич>>},\\
 &\mbox{<<церковь\ села\ Ловцы>>}, 
1755\mbox{--}1760) \wedge{}\\
&\wedge \mbox{\textit{ЭкспертнаяОценка}}(\mbox{\textit{Имя}}(\mbox{духовное\ лицо}) ={}\\
&=\; 
\mbox{<<Кирилл>>}) = 0{,}003 \wedge{}\\
&\wedge \mbox{\textit{ЭкспертнаяОценка}}((\mbox{\textit{Служба}}(x, c) \wedge{}\\
&{}\wedge 
\mbox{\textit{Родня}}(x, y)) \rightarrow \mbox{\textit{Служба}}(y, c)) = 
0{,}75\rightarrow{}\\
 & {}\rightarrow \mbox{\textit{Отец}}(\mbox{<<Леонтий\ Кириллович>>},\\
 & \mbox{<<Кирилл\ Васильевич>>})) = 0{,}9\,.
  \end{align*}
  
  Нечеткую логику рационально использовать и\linebreak
  для оценки противоречивых 
фактов. Нормализованный факт с оценкой, отличной от~0 и~1, пред\-став\-ля\-ет собой вариант 
исследовательского вопроса. Важно сохранять все, даже отвергнутые (с\linebreak 
оценкой~0), факты. На новом этапе исследователь, может быть, вернется к ним, 
в том числе для анализа намеренных искажений, из которых также могут быть 
выявлены некоторые факты или дана предопределенная оценка извлекаемых из 
некорректного источника новых фактов.

\section{Заключение}
  
  Подытожим статью, перечислив основные черты предложенной модели 
представления биографической информации:
  \begin{itemize}
\item в рассмотрение включаются не только лица, но и объекты иной природы 
(организации, населенные пункты, исторические события) вместе с их 
структурой и историей;
\item помимо характеристик отдельного объекта рассматриваются 
характеристики отношений между объектами;
\item характеристики рассматриваются в динамике изменения их значений;
\item наличие характеристик, их возможные значения и взаимозависимости 
определяются конкретно историческими знаниями, большая часть которых 
также изменяется во времени;
\item для представления совокупности установленных фактов, намеченных к 
рассмотрению вопросов, а также шагов процесса исследования предложена 
единая форма~--- формулы логики биографических фактов. 
\end{itemize}
  
  Предложенная модель не противоречит существующим биографическим 
моделям, а, скорее, дополняет и уточняет их, расставляя несколько другие 
акценты. В~частности, приоритет динамических характеристик отношений 
между объектами позволяет создать целостную непротиворечивую картину 
биографии отдельного лица и в то же время обеспечивает повторное 
использование фактов при освещении биографий связанных с ним лиц или 
истории объектов другой природы. 
  
  То, что номенклатура рассматриваемых характеристик, их возможные 
значения и допустимые отношения между ними не закреплены жестко, а 
определяются динамически изменяемыми нормалями, обеспечивает открытость 
модели, возможность применения ее для решения задач из самых разных 
проблемных областей.
  
  На основе предложенной модели возможна организация эффективного 
доступа к БР. Для этого определение метаданных, с одной стороны, и 
интерфейса поисковых запросов~--- с другой, следует осуществлять в виде 
формул логики фактов. Такой подход применим не только при создании новых 
БР, но и для модификации существующих открытых БР, а также для разработки 
специализированных оболочек закрытых БР. 
  
  Важнейшим приложением модели должно стать создание рабочего поля 
биографического исследования~--- своеобразного <<нового>> БР, в котором 
востребованы возможности работы с неточными, противоречи\-вы\-ми, 
непроверенными данными. Их постепенное накопление, систематизация, 
выдвижение и опровержение гипотез, сопоставление, обосно\-ва\-ние, постановка 
новых вопросов~--- все то, что необходимо в процессе исследования, удобно 
представимо с помощью аппарата логики фактов.

{\small\frenchspacing
{%\baselineskip=10.8pt
\addcontentsline{toc}{section}{Литература}
\begin{thebibliography}{99}

\bibitem{1mar}
\Au{Маркова Н.\,А., Адамович И.\,М.}
Электронные биографические ресурсы~// Электронные библиотеки: перспективные методы 
и технологии, электронные коллекции (RCDL'2010): Труды XII Всеросс. научн. конф.~--- 
Казань: КГУ, 2010. С.~168--180.
\bibitem{2mar}
\Au{Маркова Н.\,А., Адамович И.\,М.}
Коллекции персоналий~// Системы и средства информатики. Вып.~20. №\,2. Методы и 
технологии, применяемые в научных исследованиях информатики.~--- М.: ИПИ РАН, 2010. 
С.~178--198.

\bibitem{3mar}
Стратегический план Программы ЮНЕСКО <<Информация для всех>> (2008--2013~гг.).~--- 
М.: Межрегиональный центр библиотечного сотрудничества, 2009. 48~с.

\bibitem{4mar}
\Au{Валевский В.\,Л.}
Биографика как дисциплина гуманитарного цикла~// Лица: биографический альманах.~--- 
СПб.: Феникс, 1995. Вып.~6. С.~33--68.

\bibitem{5mar}
Русские фольклористы: Биобиблиографический словарь. Пробный выпуск~/ Отв. ред. 
Т.\,Г.~Иванова и А.\,Л.~Топорков.~--- М.: ПРОБЕЛ-2000, 2010. 240~с. 

\bibitem{6mar}
Портал Персоналии. Материал из Википедии~--- свободной энциклопедии. {\sf 
http://ru.wikipedia.org/\linebreak wiki/Портал:Персоналии} (дата обращения: 29.01.2011).

\bibitem{7mar}
Микроформаты. {\sf http://microformats.org/} (дата обращения: 29.01.2011).

\bibitem{8mar}
Российский коммуникативный формат.~--- Министерство культуры Российской Федерации, 
Российская библиотечная ассоциация, Национальная служба развития системы форматов 
RUSMARC. {\sf http://www.rba.ru/rusmarc/}.

\bibitem{9mar}
GEDCOM XML Specification, Release~6.0. 
{\sf http:// xml.coverpages.org/Gedcom-XMLv60.pdf} (дата обращения: 29.01.2010).

\bibitem{10mar}
GenXML 3.0 16.06.2010. {\sf http://www.cosoft.org/\linebreak genxml/GenXML30.pdf} (дата 
обращения: 29.01.2011).

\bibitem{11mar}
\Au{Юмашева Ю.\,Ю.}
Историография просопографии~// Известия Уральского государственного университета.~--- 
Екатеринбург: УрГУ, 2005. №\,39. Гуманитарные науки. Вып.~10. С.~95--127.

\bibitem{12mar}
Концепция информатизации архивного дела России. Утверждена Росархивом в 1995~г. {\sf 
http://\linebreak www.rusarchives.ru/informatization/conseption.shtml} (дата обращения: 29.01.2011).

\bibitem{13mar}
Всероссийское генеалогическое древо (ВГД). {\sf http://baza.vgd.ru/} (дата обращения: 
29.01.2011).

\bibitem{14mar}
\Au{Кравченко А.\,И.}
Социология. Общий курс: Учебное пособие для вузов.~--- М.: ПЕРСЭ; Логос, 2002. 640~с.

\bibitem{15mar}
\Au{Ганзен В.\,А.}
Системные описания в психологии.~--- Л.: ЛГУ, 1984.  175~с.

\label{end\stat}

\bibitem{16mar}
Памятные книжки губерний и областей Российской империи. {\sf 
http://www.nlr.ru/pro/inv/mem\_buks.htm} (дата обращения: 29.01.2011).
 \end{thebibliography}
}
}


\end{multicols} %9
\def\stat{pavlov}

\def\tit{РАСЧЕТ И ОПТИМИЗАЦИЯ НЕКОТОРЫХ ХАРАКТЕРИСТИК 
ДЛЯ~МОДЕЛИ ВЫЧИСЛИТЕЛЬНОГО КОМПЛЕКСА}

\def\titkol{Расчет и оптимизация некоторых характеристик 
для модели вычислительного комплекса}

\def\autkol{И.\,В.~Павлов}
\def\aut{И.\,В.~Павлов$^1$}

\titel{\tit}{\aut}{\autkol}{\titkol}

%{\renewcommand{\thefootnote}{\fnsymbol{footnote}}\footnotetext[1]
%{Работа выполнена при поддержке РФФИ (гранты 09-07-12098, 09-07-00212-а и
%09-07-00211-а) и Минобрнауки РФ (контракт №\,07.514.11.4001).}}


\renewcommand{\thefootnote}{\arabic{footnote}}
\footnotetext[1]{Московский государственный технический университет им.\ Н.\,Э.~Баумана, 
ipavlov@bmstu.ru}

\Abst{Рассматривается проблема выбора оптимального размера пакетов при обработке 
информационных задач большого объема для модели вычислительного комплекса с 
учетом возможных отказов или сбоев элементов в процессе решения задачи. Получено 
приближенное асимптотическое решение данной проблемы для случая высоконадежных 
элементов и малого времени пересылки (загрузки) пакетов.}

\KW{оптимальный размер пакета; надежность; интенсивность отказов; время пересылки 
пакетов} 

\vskip 14pt plus 9pt minus 6pt

      \thispagestyle{headings}

      \begin{multicols}{2}

            \label{st\stat}

\section{Введение}

     Пусть имеется система, включающая в себя $l$ основных 
вычислительных элементов. В~систему поступают <<задания>>, каждое из 
которых состоит из некоторого (вообще говоря, случайного) числа   
<<элементарных задач>>, каждая из которых может выполняться 
(обрабатываться) независимо от остальных на любом из этих элементов. Для 
выполнения очередного задания, поступившего в систему, необходимо 
выполнить все составляющие его элементарные задачи. При этом в процессе 
выполнения задание разбивается на некоторое количество $n$  блоков 
(<<пакетов>>) элементарных задач равного объема $\upsilon\hm=L/n$, 
$n\hm\in N$, где $N$~--- множество допустимых значений  (например, 
$N$~---  некоторое подмножество целочисленных значений, кратных~2 
и~т.\,п.). Время~$h$ выполнения одной элементарной задачи на любом из 
элементов далее будем считать равным единице: $h\hm=1$. Соответственно, 
время выполнения одного пакета объемом~$\upsilon$ на любом из элементов 
будет численно совпадать с величиной~$\upsilon$. 
{\looseness=1

}
     
     Выполнение задания происходит путем пересылки пакетов на рабочие 
элементы и дальнейшей их обработки на этих элементах. Время пересылки 
(загрузки) пакета на элемент равно величине $\tau\hm>0$, не зависит от 
размера пакета~$\upsilon$ и от состояния других элементов. Обработка пакета 
после его загрузки на данном элементе занимает время~$\tau$ и происходит 
независимо от состояния других элементов. После завершения обработки 
очередного пакета на том или ином элементе снова происходит его загрузка в 
течение времени~$\tau$ следующим пакетом (из общей очереди всех пакетов 
данного задания) независимо от состояния (работы или загрузки) остальных 
элементов и~т.\,д. Задание считается выполненным после выполнения 
(обработки) всех составляющих его пакетов. Близкие по смыслу модели и 
процессы рассматривались ранее в~[1--5].
     
     В процессе работы любой из элементов может отказывать с постоянной 
(не зависящей от времени) функцией интенсивности отказов 
$\lambda(t)\hm\equiv \lambda$~[6, 7]. Заметим, что более близким к 
реальности было бы предположение о монотонном возрастании 
(неубывании) $\lambda(t)$ по времени. Поэтому фактически здесь 
предполагается, что, по крайней мере в течение времени выполнения одного 
задания, функция интенсивности отказов~$\lambda(t)$ меняется 
незначительно и может считаться приближенно постоянной. Такое 
допущение является естественным, по крайней мере в случае высокой 
надежности элементов, когда вероятность отказа элемента за время 
выполнения в системе одного задания достаточно мала. В~указанных 
допущениях время безотказной работы элемента имеет экспоненциальное 
распределение с функцией надежности $P(t)\hm=e^{-\lambda t}$, а вероятность 
отказа элемента за время~$h$ выполнения одной элементарной задачи равна 
величине $\lambda h\hm+ o(\lambda h)$.
     
     Одной из существенных проблем, возникающих в данной ситуации, 
является выбор оптимального размера пакета~$\upsilon$ с учетом 
возможности отказов (сбоев) элементов при выполнении задания.

\section{Модель со сбоями элементов}

     Рассмотрим случай, когда возможные отказы элементов в системе 
имеют характер <<сбоев>>. Другими словами, в результате отказа (сбоя) 
элемент сам по себе не выходит из строя и продолжает работать, но 
находящийся на нем в момент сбоя пакет считается невыполненным и после 
завершения его обработки снова ставится в очередь необработанных пакетов 
и должен быть полностью обработан заново на этом же или любом другом 
элементе.
     
     Рассмотрим сначала более простой частный случай, когда число 
элементов $l\hm=1$. Обозначим через $p\hm=\exp (-\lambda \upsilon)$ 
вероятность обработки пакета объемом~$\upsilon$ без сбоев и $q\hm=1-p$. 
Время~$\eta$ выполнения всего задания объемом~$L$ имеет вид:
     \begin{equation}
     \eta=(\upsilon+\tau) v\,,
     \label{e1p}
     \end{equation}
где $v$~--- момент (номер шага) первого достижения $n$ <<успехов>> в 
классической схеме независимых ис\-пытаний Бернулли при вероятности 
<<успеха>> (на\linebreak
одном шаге) $p\hm=\exp\left( -\lambda \upsilon\right)$. Задача 
выбора оп\-тимального размера пакета~$\upsilon$ далее сводится к 
минимизации математического ожидания E$\eta$ по параметру~$\upsilon$, 
или, учитывая равенство $\upsilon\hm= L/n$, к\linebreak
 минимизации~E$\eta$ по 
переменной $n\hm\in N$, где $n$~---  чис\-ло пакетов, на которое разбивается 
задание. Случайная величина~$v$ имеет распределение Пас\-каля 
\begin{equation}
P\left( v=m\right) = C_{m-1}^{n-1} p^n q^{m-n}\,,\quad m=n, n+1, \ldots ,
\label{e2p}
\end{equation}
с математическим ожиданием E$v=n/p$, откуда с учетом~(\ref{e1p}) следует, 
что выбор оптимального размера пакета сводится к задаче: найти
\begin{equation}
\min \left( L+n\tau\right)\exp\left( \fr{\lambda L}{n}\right)
\label{e3p}
\end{equation}
по $n\in N$. Далее оптимальный размер пакета~$\tilde{\upsilon}$ 
находится по формуле $\tilde{\upsilon} =L/\tilde{n}$, где $\tilde{n}\hm\in 
N$~--- решение задачи~(\ref{e3p}). 
     
     Оптимизационная задача~(\ref{e3p}) является цело\-чис\-лен\-ной, 
поскольку множество $N$ допустимых значений $n$ содержит только 
целочисленные точки. Введем также дополнительную <<непрерывную>> 
задачу: найти
     \begin{equation}
     \min\left( L+n\tau\right) \exp \left( \fr{\lambda L}{n}\right)
     \label{e4p}
     \end{equation}
по всем (не только целочисленным) значениям $n\hm\geq 1$. Далее 
оптимальный размер пакета~$\upsilon^*$ (без ограничения целочисленности 
$n\hm\in N$) находится как $\upsilon^*=L/n^*$, где $n^*$~--- решение 
задачи~(\ref{e4p}).
     
Теорема~1 дает точное решение оптимизационных 
задач~(\ref{e3p}) и~(\ref{e4p}). Теорема~2 дает асимптотическое выражение 
для оптимального размера пакета~$\upsilon^*$.
     
     \medskip
     
     \noindent
     \textbf{Теорема 1.} \textit{Пусть $\lambda\hm>0$, $\tau\hm>0$ и 
выполняется неравенство}
     \begin{equation}
     \tau\leq \lambda L^2\,.
     \label{e5p}
     \end{equation}
\textit{Тогда минимум}~(\ref{e4p}) \textit{достигается в единственной \mbox{точке}}
$$
n^*=L\sqrt{\fr{\lambda}{\tau}}\left[  
\sqrt{1+\fr{\lambda\tau}{4}}+\fr{\sqrt{\lambda\tau}}{2}\right]\,.
$$
\textit{Минимум}~(\ref{e3p}) \textit{достигается в одной из двух ближайших 
(слева или справа) к точке~$n^*$ целочисленных точек $n\hm\in N$.} 

     \smallskip
     
     \noindent
     Д\,о\,к\,а\,з\,а\,т\,е\,л\,ь\,с\,т\,в\,о\,.\ Введем функцию
     \begin{equation}
     f(n) =\left( L+n\tau\right) \exp\left( \fr{\lambda L}{n}\right)
     \label{e6p}
     \end{equation}
от непрерывного аргумента $n\hm\geq 1$. Нетрудно показать, что знак 
производной этой функции совпадает со знаком многочлена 
$Q(n)\hm=n^2\hm-\lambda L n -\lambda L^2/\tau$, который имеет при 
$n\hm\geq 1$ единственный корень в точке $n\hm=n^*$ и для которого 
справедливы неравенства:
\begin{align*}
Q(n)<0 &\ \ \mbox{при}\ \ 1\leq n\leq n^*\,;\\
Q(n)>0 &\  \ \mbox{при}\ \ n>n^*\,,
\end{align*}
если выполняется условие~(\ref{e5p}), откуда далее и следует теорема~1. 
Теорема доказана.

\smallskip

     \noindent
     \textbf{Теорема~2.} \textit{Пусть $\lambda\hm>0$, $\tau\hm>0$, 
$\tau\hm\leq \lambda L^2$ и $\lambda\tau\hm\rightarrow 0$. Тогда} 

\noindent
     \begin{equation}
     \upsilon^*=\sqrt{\fr{\tau}{\lambda}}\left[ 1+o(1)\right]\,.
     \label{e7p}
     \end{equation}
     
     \smallskip
     
     \noindent
     Д\,о\,к\,а\,з\,а\,т\,е\,л\,ь\,с\,т\,в\,о\ следует из теоремы~1 и равенства 
$\upsilon^*\hm=L/n^*$. 
     
     \medskip
     
     Из~(\ref{e7p}) далее следует приближенная формула для оптимального 
размера пакета при $\lambda\tau\hm\ll 1$:
     $$
     \upsilon^*\cong \sqrt{\tau\theta}\,,
     $$
где $\theta=1/\lambda$~--- математическое ожидание \mbox{времени} безотказной 
работы (средний ресурс) элемента. Другими словами, оптимальный размер 
пакета~$\upsilon^*$ приближенно равен среднему геометрическому между 
временем пересылки (загрузки)~$\tau$ и средним ресурсом элемента~$\theta$ 
(при условии $\lambda\tau\hm\ll 1$). Существенно, что оптимальное 
значение~$\upsilon^*$ не зависит от размера всего задания~$L$, который, 
вообще говоря, может быть неизвестным и случайным.
     
     Рассмотрим далее общий случай $l\hm\geq 1$ элементов.
Для рассматриваемой модели время выполнения задания

\noindent
     \begin{equation}
     \eta=\left(\upsilon+\tau\right) \left(\fr{v}{l}\right)^+\,,
     \label{e8p}
     \end{equation}
где $z^+$~---  величина~$z$, округленная вверх до ближайшего целого. 
Задача сводится к вычислению
\begin{equation}
\min E\eta
\label{e9p}
\end{equation}
по $n\in N$, после чего оптимальный размер пакета~$\tilde{\upsilon}$ находится 
по формуле $\tilde\upsilon\hm=L/\tilde{n}$, где $\tilde{n}\hm\in N$~--- решение 
задачи~(\ref{e9p}). 

\pagebreak
     
     В соответствии с~(\ref{e2p}) и (\ref{e8p})
     $$
     E\eta =\left( \fr{L}{n}+\tau\right) \sum\limits_{m=n}^\infty 
     \left (\fr{m}{l}\right)^+ C_{m-1}^{n-
1} p^n q^{m-n}\,,
     $$
откуда, учитывая, что $m C_{m-1}^{n-1}=nC_m^n$,
\begin{multline}
E\eta = \left( \fr{L}{n}+\tau\right) \sum\limits_{m=n}^\infty 
\left (\fr{m}{l}\right)^+ \fr{n}{m}\,C_m^n 
p^n q^{m-n}={}\\
{}=\left( \fr{1}{l}\right) \left( L+n\tau\right) p^n \sum\limits_{m=n}^\infty 
\fr{(m/l)^+}{m/l}\,C_m^n q^{m-n}\,.
\label{e10p}
\end{multline}

В соответствии с~(\ref{e2p}) 
\begin{equation}
E v = \!\sum\limits_{m=n}^\infty m C_{m-1}^{n-1} p^n q^{m-n} =n p^n\! 
\sum\limits_{m=n}^\infty C_m^n q^{m-n}.
\label{e11p}
\end{equation}
С другой стороны, случайная величина~$v$ является суммой~$n$ 
независимых, одинаково распределенных случайных величин, каждая из 
которых имеет геометрическое распределение с параметром~$p$ и 
математическим ожиданием $1/p$. Соответственно, $Ev \hm= n/p$, откуда с 
учетом~(\ref{e11p}) следует 
\begin{equation}
\sum\limits_{m=n}^\infty C_m^n q^{m-n} =\fr{1}{p^{n+1}}\,.
\label{e12p}
\end{equation}
Из~(\ref{e10p}) и~(\ref{e12p}) следует
$$
E\eta = \fr{1}{l}\left( L+n\tau\right) p^n \sum\limits_{m=n}^\infty \left[ 
1+\fr{(m/l)^\prime}{m/l}\right] C_m^n q^{m-n}\,,
$$
где $z^\prime=z^+-z$, откуда
\begin{equation}
E\eta =\fr{1}{l}\left(L+n\tau\right)\exp \left( \fr{\lambda L}{n}\right) 
\left(1+\delta_l(n)\right)\,,
\label{e13p}
\end{equation}
где 
\begin{equation}
\delta_l(n)=\sum\limits_{m=n}^\infty \alpha_{nm} \fr{(m/l)^\prime}{m/l}\,,
\label{e14p}
\end{equation}
где коэффициенты $\alpha_{nm} =p^{n+1} C_m^n q^{m-n}$, $p\hm= e^{-
\lambda L/n}$, $q\hm=1\hm-p$. При этом в соответствии с~(\ref{e12p}) 
\begin{equation}
\sum\limits_{m=n}^\infty \alpha_{nm}=1\,.
\label{e15p}
\end{equation}
Из~(\ref{e14p}) и (\ref{e15p}) видно, что 
\begin{equation}
0<\delta_l(n)<\fr{l}{n}\,.
\label{e16p}
\end{equation}
     
     Целевая функция~(\ref{e13p}) для общего случая $l\hm\geq 1$ 
совпадает с целевой функцией в~(\ref{e3p}) для случая $l\hm=1$ с точностью 
до множителя $(1/l)\left[ 1+\delta_l(n)\right]$, откуда с учетом~(\ref{e16p}) 
видно, что полученное выше решение для случая $l\hm=1$ практически дает 
и решение для случая $l\hm>1$, если оптимальное число пакетов $n$ 
достаточно велико.
     
     Обозначим через
     \begin{equation}
     f_l(n) =\fr{f(n)}{l}\left[ 1+\delta_l(n)\right]
     \label{e17p}
     \end{equation}
целевую функцию~(\ref{e13p})~--- среднее время выполнения задания при 
данных значениях $n$~--- чис\-ле пакетов и $l$~--- чис\-ле элементов, где 
$f(n)\hm=(L+n\tau)\exp\left(\lambda L/n\right)$~--- целевая функция~(\ref{e6p}) 
для случая $l\hm=1$.
     
     Задача выбора оптимального размера пакета~$\tilde{\upsilon}$ сводится к 
нахождению 
     \begin{equation}
     \min f_l(n) =f_l\left(\tilde{n}_l\right)\,.
     \label{e18p}
     \end{equation}
Здесь минимум берется по всем $n\hm\in N$, где $N$~--- множество 
допустимых значений~$n$ (например, $N$~--- множество це\-ло\-чис\-лен\-ных 
значений~$n$, кратных~2, лежащих в некотором допустимом диапазоне, 
и~т.\,п.). Полагаем $\tilde{\upsilon}\hm=L/\tilde{n}_l$, где 
$\tilde{n}_l$~--- решение задачи~(\ref{e18p}). Введем также дополнительную 
задачу нахождения 
\begin{equation}
\min f_l(n) =f_l(n_l^*)\,,
\label{e19p}
\end{equation}
где минимум берется по всем (не только це\-ло\-чис\-лен\-ным) значениям 
$n\hm\geq 1$. Далее оптимальный размер пакета~$\upsilon_l^*$ (без 
ограничений це\-ло\-чис\-лен\-ности $n\hm\in N$) определим по формуле 
$\upsilon_l^*\hm=L/n_l^*$, где $n_l^*$~--- решение задачи~(\ref{e19p}). Из 
выражений~(\ref{e13p})--(\ref{e16p}) далее следует теорема~3.

\medskip

\noindent
\textbf{Теорема~3.} \textit{Решение оптимизационной задачи}~(\ref{e19p}) 
\textit{удовлетворяет неравенствам}
\begin{equation}
\fr{f_l(n^*)}{1+\varepsilon}\leq \min\limits_{n\geq 1} f_l(n)\leq f_l(n^*)\,,
\label{e20p} 
\end{equation}
\textit{где $n^*$~--- решение этой задачи для случая $l\hm=1$, 
$\varepsilon\hm=\delta_l(n^*)\hm<l/n^*$. Решение оптимизационной 
задачи}~(\ref{e18p}) \textit{удовлетворяет аналогичным неравенствам}
\begin{equation}
\fr{f_l(\tilde{n})}{1+\varepsilon}\leq \min\limits_{n\in N} f_l(n)\leq 
f_l(\tilde{n})\,, 
\label{e21p}
\end{equation}
\textit{где $\tilde{n}$~--- решение этой задачи для случая} $l\hm=1$, 
$\varepsilon\hm=\delta_l(\tilde{n})\hm<l/\tilde{n}$.
     
     \medskip
     
\noindent
     Д\,о\,к\,а\,з\,а\,т\,е\,л\,ь\,с\,т\,в\,о\,.\ Равенство~(\ref{e17p}) при 
$n\hm=n^*$ имеет вид:
     \begin{equation}
     f_l(n^*) =\fr{f(n^*)}{l}\left[ 1+\delta_l(n^*)\right]\,.
     \label{e22p}
     \end{equation}
Из этого же равенства, учитывая, что $\delta_l(n)\hm>0$, следует
$$
f_l(n)\geq \fr{f(n)}{l}\,,
$$
откуда с учетом~(\ref{e22p})
$$
\min\limits_{n\geq 1} f_l(n) \geq \fr{1}{l}\min\limits_{n\geq 1} f(n) 
=\fr{f(n^*)}{l}= \fr{f_l(n^*)}{1+\delta_l(n^*)}\,,
$$
что вместе с~(\ref{e16p}) доказывает левое неравенство в~(\ref{e20p}). 
Правое неравенство очевидно. Доказательство неравенств~(\ref{e21p}) 
аналогично. Теорема доказана.

\medskip

     Таким образом, полученное решение для случая $l\hm=1$ 
практически дает решение и в случае $l\hm>1$, если число пакетов много 
больше по сравнению с количеством элементов~$l$.

\section{Заключение}
     
     Получено решение указанной выше основной проблемы (выбора 
оптимального размера пакета) для модели со сбоями элементов в 
естественной с прикладной точки зрения асимптотике, а именно для случая 
высоконадежных элементов и при малом времени пересылки пакетов. 
Существенно, что полученное решение не зависит от общего объема всего 
задания, что, в частности, позволяет использовать его в ситуации 
неопределенности, когда эта величина, вообще говоря, может быть 
неизвестной и случайной. Отметим также, что представляет интерес 
дальнейшее обобщение полученных результатов на ситуацию, когда 
различные элементы могут иметь существенно различные характеристики 
как производительности, так и надежности, а также на модель с отказами и 
восстановлением (заменой) отказавших элементов. 

{\small\frenchspacing
{%\baselineskip=10.8pt
\addcontentsline{toc}{section}{Литература}
\begin{thebibliography}{9}


\bibitem{3p} %1
\Au{Ронжин А.\,Ф., Суриков В.\,Н.}
О~времени полного перебора~// Обозр. прикл. пром. матем., 2007. Т.~14. 
№\,3. С.~506--508.

\bibitem{4p} %2
\Au{Коновалов М.\,Г., Малашенко Ю.\,Е., Назарова~И.\,А.}
Модели и методы управления заданиями в системах распределенных 
вычислительных ресурсов.~--- М.: ВЦ РАН, 2009. 110~с. (Сообщения по 
прикладной математике.)

\bibitem{5p} %3
\Au{Коновалов М.\,Г., Малашенко Ю.\,Е., Назарова~И.\,А.}
Оперативное управление потоком заданий в системе распределенных 
вычислительных ресурсов~// VI Московская междунар. конф. по 
исследованию операций: ORM-2010: Труды.~--- М.: 
МАКС Пресс, 2010. С.~301--302.

\bibitem{6p} %4
\Au{Козлов М.\,В., Малашенко Ю.\,Е., Назарова~И.\,А., Ронжин~А.\,Ф.}
Анализ режимов управления вычислительным комплексом в условиях 
неопределенности.~--- М.: ВЦ РАН, 2011. 63~с. (Сообщения по прикладной 
математике.)

\bibitem{7p} %5
\Au{Коновалов М.\,Г., Малашенко Ю.\,Е., Назарова~И.\,А.}
Управ\-ле\-ние заданиями в гетерогенных вычислительных сис\-те\-мах~// 
Известия РАН. Теория и системы управления, 2011. №\,2. С.~72--90. 

\bibitem{1p} %6
\Au{Гнеденко Б.\,В., Беляев Ю.\,К., Соловьев~А.\,Д.}
Математические методы в теории надежности.~--- М.: Наука, 1965. 524~с. 

\label{end\stat}

\bibitem{2p} %7
\Au{Gnedenko B.\,V., Pavlov I.\,V., Ushakov~I.\,A.}
Statistical reliability engineering.~--- N.Y.: John Wiley, 1999. 514~p.
 \end{thebibliography}
}
}


\end{multicols} %10
\include{semenov} %11
\def\stat{sharnin}

\def\tit{ОСОБЕННОСТИ СЕМАНТИЧЕСКОГО ПОИСКА ИНФОРМАЦИОННЫХ ОБЪЕКТОВ\\ НА~ОСНОВЕ 
ТЕХНОЛОГИИ БАЗ ЗНАНИЙ}

\def\titkol{Особенности семантического поиска информационных объектов на~основе 
технологии баз знаний}

\def\autkol{М.\,М.~Шарнин, И.\,П.~Кузнецов}
\def\aut{М.\,М.~Шарнин$^1$, И.\,П.~Кузнецов$^2$ }

\titel{\tit}{\aut}{\autkol}{\titkol}

%{\renewcommand{\thefootnote}{\fnsymbol{footnote}}\footnotetext[1]
%{Работа выполнена при поддержке РФФИ (гранты 09-07-12098, 09-07-00212-а и
%09-07-00211-а) и Минобрнауки РФ (контракт №\,07.514.11.4001).}}


\renewcommand{\thefootnote}{\arabic{footnote}}
\footnotetext[1]{Институт проблем информатики Российской академии наук, keywen1@mail.ru}
\footnotetext[2]{Институт проблем информатики Российской академии наук,
igor-kuz@mtu-net.ru}


     
     \Abst{Рассматривается система семантического поиска информации в больших 
массивах документов на естественном языке (ЕЯ). Поиск основан на использовании 
лингвистического процессора, обеспечивающего автоматическое выделение из текстов 
информационных объектов (именованных сущностей), их признаков, связей и участие в 
действиях. В~результате формируются структуры знаний. Аналогичным образом 
формируется структура запроса. Поиск, называемый семантическим, обеспечивается за счет 
сопоставления таких структур, где учитываются связи объектов, а также их участие в 
событиях, действиях.}
     
     \KW{семантический поиск; семантико-ориентированный лингвистический процессор; 
извлечение знаний из текстов; база знаний}

\vskip 14pt plus 9pt minus 6pt

      \thispagestyle{headings}

      \begin{multicols}{2}

            \label{st\stat}
    
    \section{Введение}
    
    Одной из актуальных задач в области информационных технологий 
является поиск информации в больших массивах документов~--- текстов на 
\textit{естественном языке}. Для многих профессиональных 
пользователей поиск определяется их задачами. Например, задачи 
следователей-аналитиков (из области <<Криминалистика>>) непосредственно 
связаны с поиском фигурантов, их адресов, деяний, связей между фигурантами, 
поиском по приметам, поиском похожих фигурантов и происшествий и многим 
другим. Для поиска используются документы криминальной полиции, 
имеющие вид текстов на ЕЯ: сводки происшествий и~др.
    
    Другой пример~--- задачи кадровых агентств, где документами являются 
резюме людей, желающих получить работу. Такие резюме часто пишутся в 
свободной форме~--- в виде текстов на ЕЯ. В~резюме даются анкетные данные, 
места учебы и работы с указанием периодов и организаций или учебных 
заведений и~т.\,д. Задачи кадровых агентств~--- поиск лиц по запросам 
клиентов, которые часто задаются на~ЕЯ.
    
    Следует отметить, что профессиональных пользователей интересует 
определенного сорта информация, которая зависит от предметной области. 
В~приведенных выше примерах это лица, где они работают (организации), кем 
(профессии), чем занимаются (служебные обязанности) или в каких событиях 
участвовали (деяния лиц) и~т.\,д. Подобную информацию будем называть 
\textit{информационными объектами} или просто \textit{объектами} (другое 
название~--- \textit{именованные сущности}).
    
    Поиск информационных объектов~--- это самостоятельная задача. 
Типовые поисковые машины ({Google}, Яндекс и~др.)\ ищут ресурсы, 
содержащие слова запроса. Они не учитывают семантическую 
    составляющую~--- наличие объектов, их связи.
    
    Для поиска объектов требуется предварительная формализация текстов на 
ЕЯ~--- выделение не только объектов, но и всего, что с ними связано. 
Возникают структуры знаний. 

В~данной статье рассматривается поиск, 
основанный на сопоставлении таких структур. \textit{Поиск} осуществляется не 
на уровне слов, а на уровне структур знаний, и поэтому является 
\textit{семантическим}.
    
    В настоящее время проблема семантического поиска приобретает все 
большую актуальность. Следует отметить семантическую поисковую сис\-тему 
{AskNet} ({\sf http://asknet.ru}), которая <<автоматически выбирает 
смысловые ответы на запросы пользователя>>, систему {Hakia} ({\sf 
http://hakia.com}),\linebreak основанную на хранилище семантической информации и 
технологии ранжирования найденных текстов по смыслу, а также системы 
Wolfram Alpha, Powerset и~др. Во многих из них 
рассматриваются смыс\-ло\-вые связи между терминами, для представления 
которых разрабатываются специальные фор\-ма\-лизмы.
    
    Цель данной статьи~--- описание технологии поиска информационных 
объектов на основе структур знаний, для извлечения которых используется 
семантико-ориентированный лингвистический процессор~[1--3]. 
Такой поиск не является универсальным (как в системах Яндекс, 
Google). Его организация требует настройки лингвистического 
процессора на выделение объектов в определенной предметной области. Набор 
таких объектов ограничен. Соответственно, система настраивается давать 
точные ответы на определенный круг запросов.
    
    В основе семантических поисков лежит технология баз знаний (БЗ), 
разработанная в рамках проектов ИПИ РАН~\cite{3sha}. Она включает в себя 
формализацию текстов (извлечение структур знаний), формализмы 
представления и хранения знаний (выделенных <<смысловых элементов>>) в 
БЗ, а также методы сопоставления запроса и имеющейся информации на уровне 
структур знаний.
    
    Для организации соответствующего технологического комплекса 
требуются формализмы, которые должны обладать определенными свойствами: 
быть как можно более простыми (в синтаксическом плане), обладать высокими 
изобразительными возможностями для представления знаний и обеспечивать в 
широких пределах логико-лингвистическую обработку~\cite{1sha}. Данными 
свойствами обладает язык расширенных семантических сетей (РСС) и 
продукционный язык их обработки~--- ДЕКЛ. На этой основе разработан 
инструментальный комплекс, ориентированный на обработку структур 
знаний~\cite{2sha, 8sha}. Комплекс использован для построения класса 
семантико-ориентированных лингвистических процессоров, преобразующих 
тексты на ЕЯ в формализм РСС, организации на этой основе БЗ и для 
разработки множества прикладных программ, обеспечивающих 
идентификацию объектов, выявление имплицитной информации, 
преобразование представлений, семантический поиск, принятие экспертных 
решений и~др. Все эти задачи дополняют одна другую и решаются на одном 
уровне~--- структур знаний~\cite{3sha}.
    
    Отметим, что структуры знаний в виде РСС автоматически отображаются 
на языке XML~\cite{9sha} и могут быть использованы для построения 
прикладных программ на различных языках программирования. Подобный 
подход при соответствующей технологической доработке может быть основой 
крайне перспективного направления информатики~--- <<Семантического 
Интернета>>.

\vspace*{-9pt}
    
\section{Особенности обработки в~базе~знаний}

\vspace*{-3pt}
    
    Технология семантического поиска объектов на уровне структур знаний 
была разработана при построении систем <<Аналитик>> и <<Криминал>>. 
Последняя была создана для ГУВД г.~Москвы~\cite{4sha, 3sha, 8sha}. Эти системы 
ориентированы на работу с текстами на ЕЯ в определенной предметной 
области. В~частности, система <<Криминал>> ориентирована на работу с 
большими потоками документов в области криминальной полиции: сводками 
происшествий, справками по уголовным делам, обвинительными 
заключениями, записными книжками фигурантов и~др. Тексты автоматически 
формализуется с помощью семантико-ориентированного лингвистического 
процессора. При этом выделяются информационные объекты (фигуранты, их 
приметы, адреса, телефоны, даты, оружие, автотранспорт со всеми атрибутами 
и~др.), а также связи между ними и разного рода деяниями, событиями. 
Участие объектов в одном действии считается одним из видов связи. Более 
того, сами действия~--- это тоже информационные объекты, которые 
связываются с временем, местом, а также причинно-следственными и другими 
отношениями. В~результате возникают сложные структуры. На основе каждого 
документа формируется семантическая сеть (РСС), называ\-емая 
\textit{содержательным портретом документа}~\cite{6sha, 7sha}. Такие 
портреты образует \textit{базу предметных знаний}, которая 
запоминается, а сами портреты связываются с соответствующими текстами.
    
    Семантические поиски идут на уровне структур БЗ и включают в себя 
логический анализ признаков, связей. Например, поиск ответа на запрос в 
свободной форме (т.\,е.\ на ЕЯ) обеспечивается путем сопоставления 
содержательного портрета, построенного на основе запроса, и содержимого БЗ, 
т.\,е.\ сводится к поиску соответствующей структуры в БЗ. При этом широко 
используются онтологии, представленные в виде РСС, а также дополнительная 
информация, которая характеризует поисковый объект или ситуацию, но 
которая дается в тексте в неявной форме~--- как имплицитная информация, 
которую нужно восстанавливать~\cite{10sha}.
    
    В данной статье в качестве примера использования технологии БЗ 
рассматриваются задачи поиска похожих происшествий и лиц (фигурантов). 
При поиске похожих происшествий учитываются все действия и объекты, 
составляющие данное происшествие. При поиске похожего фигуранта 
учитывается только то, что связано с фигурантом. Эти задачи относятся к 
наиболее важным в об\-ласти криминальной полиции. Они необходимы для 
идентификации лиц, установления их связей, порождения и проверки 
различных гипотез, планирования следственных действий. В~данной статье 
рассматриваются методики и алгоритмы решения этих задач на структурном 
уровне, т.\,е.\ на основе различных видов связей с учетом особенностей 
описываемых объектов, событий, происшествий. Ориентация сделана на 
использование семантических связей, а также методов логического анализа и 
нечеткого вывода. Отметим, что подобные методики использованы для 
семантического поиска других информационных объектов.
    
    Задача поиска похожих происшествий и фигурантов решалась в рамках 
логико-аналитической системы <<Криминал>> с учетом ее задач и 
особенностей~\cite{3sha, 8sha}.
    
    В системе <<Криминал>> онтологии представлены в виде РСС и образуют 
\textit{онтологическую базу}, которая находится в отдельном файле и 
объединяется с БЗ в процессе поиска. Онтологическая база определяет 
семантическое пространство терминов и признаков~--- с учетом их смысловой 
близости, синонимии и взаимоотрицания. За счет этого расширяется 
пространство поиска, повышается точность и надежность результатов, 
обеспечивается достаточная свобода использования слов и терминов в запросах 
и заданиях системе.
    
    Все документы и полученные на их основе структуры знаний 
(содержательные портреты) помещаются в собственную базу данных, 
ориентированную на большие потоки информации и обеспечивающую их 
быстрый выбор~--- за счет индексных файлов (базы данных служат для 
хранения документов и структур знаний). Эти структуры по мере 
необходимости подкачиваются в оперативную память и вместе с 
онтологической базой образуют \textit{оперативную базу знаний} (ОБЗ), где и 
осуществляется поиск. При этом допускается наличие множества баз данных 
(со своими БЗ) на различных компьютерах, связанных в сеть. Таким образом 
обеспечивается работа распределенных~БЗ.

\vspace*{-6pt}
    
\section{Содержательные портреты документов}

\vspace*{-3pt}
    
    Сеть (РСС), представляющая объекты и связи документа, образует его 
содержательный портрет, где все слова представлены в канонической форме. 
Такие портреты служат основой для семантического поиска.
    
    \smallskip
    
    \noindent
    \textbf{Пример 1.} Типовой документ (с номером~221) из сводок 
происшествий:
    \textit{1.05.98~г.\ в 7.10 Фирсова Владимира Николаевича 1953~г.р.\ прож.: 
ул.~Глаголева 25-1-273, работает АОЗТ <<ХДУ>>, зам.\ директора, о том, 
что 1-05-98~г.\ неизвестные от д.~22 кор.~3 по ул.~Тухачевского, похитили а/м 
ГАЗ~31029, черная, 1995~г/в, дв. 402-0019476}\ldots
    
    Его содержательный портрет имеет вид:
    
\smallskip

{\footnotesize

\noindent    
     ДОК\_(221,`TEXT\_98.TXT',`S\_CRI.NL')

\noindent     
     ДАТА\_(\#1.5.1998,1998,МАЙ,$\sim$1,7.1/4+)

\noindent     
     FIO(ФИРСОВ,ВЛАДИМИР,НИКОЛАЕВИЧ,1953/5+)
     
\noindent     
 АДР\_(УЛ.,ГЛАГОЛЕВА,25,1,273/6+)
     
\noindent     
     ПРОЖ.(5-,6-/7+)
     
\noindent     
     ОРГ\_(АОЗТ,ХДУ/8+)
     
\noindent     
     РАБ\_(5-,8-,ЗАМ.,ДИРЕКТОР/9+)
     
\noindent     
     FIO(``\,'',``\,'',``\,'',НЕСКОЛЬКО/10+) \mbox{НЕИЗВЕСТНЫЙ(10-)}
     
\noindent     
     АВТО\_(ГАЗ,31029,ЧЕРНЫЙ,1995,Г/В,ДВ.,402-0019476/11+)
     

\noindent     
    УГНАТЬ(10-,11-/12+)
     
\noindent     
     ДАТА\_(\#1.5.1998,1998,МАЙ,$\sim$1/14+)
     
\noindent     
     КОГДА(12-,14-)
     
\noindent     
     АДР\_(УЛ.,ТУХАЧЕВСКОГО,ДОМ,22,КОРП.,3/15+)
     
\noindent     
     ГДЕ(12-,15-)
     
\noindent     
     ПРЕДЛ\_(221,4-,5-,6-,8-,9-,О,ТОМ,12-,14-,15-)
     }
     
     \smallskip
    
    Первый фрагмент ДОК\_(221,`TEXT\_98.TXT',\linebreak `S\_CRI.NL') указывает на 
то, что содержательный портрет построен на основе документа~221 из файла 
`TEXT\_98.TXT'. При этом были использованы лингвистические знания 
`S\_CRI.NL'. Второй фрагмент представляет дату. Третий фрагмент 
представляет фигуранта~--- \textit{Фирсова Владимира Николаевича}. Ему 
сопоставлен внутренний код 5+, с помощью которого представлено, где он 
проживает~--- ПРОЖ.\mbox{(5-,6-/7+)}, где <<5->>~--- код адреса. Здесь же 
представлены и другие объекты~--- организация (ОРГ\_), место работы (РАБ\_), 
автотранспорт (АВТО\_) и~др. Фрагмент УГНАТЬ(10-,11-/12+) представляет 
действие неизвестного лица (с кодом 10+), который \textit{похитил} (\mbox{=\ 
\textit{угнал}}) автомашину (с кодом 11+). Последний фрагмент 
ПРЕДЛ\_(221,\ldots) содержит коды других фрагментов и представляет порядок 
расположения соответствующей информации в тексте документа.
    
    Такие портреты (в виде РСС) запоминаются в БЗ. Поиск сводится к 
сопоставлению таких портретов~--- запроса и содержимого БЗ~\cite{3sha, 8sha}. 
При поиске похожих фигурантов и происшествий важную роль играют не 
только объекты, но и действия типа УГНАТЬ и~др. Помимо этого используется 
дополнительная информация, представленная в виде аналитических 
фрагментов (см.\ разд.~4).
    
\section{Оценка документа по~ключевым позициям}
    
    При организации семантических поисков важную роль играют признаки, 
задающие общий характер происшествий (\textit{способы проникновения}, 
\textit{совершения преступления} и~др.), особенности фигуранта (их приметы) 
или особенности любого другого информационного объекта. Они могут в 
явном виде не присутствовать в тексте и требуют специального логического 
анализа для их выявления. С~этой целью в процессе ввода документов с их 
формализацией производится оценка документа по ключевым позициям. Она 
необходима для быстрого и качественного поиска, а также для выдачи 
информации в сжатом виде и объяснения результатов.
    
    Оценка документа по ключевым позициям осуществляется на уровне 
структур знаний с помощью специальной программы постлингвистической\linebreak 
обработки, реализующей идеологию семантических
    фильтров~\cite{5sha, 6sha, 7sha}. Оценка заключается в выделении 
особенностей описанного в документе происшест\-вия (или особенностей 
какого-либо информационного объекта) и его соотнесении с 
соответствующими ветвями типовых классификаторов, находящих\-ся в 
онтологической базе. Такое соотнесение осуществляется автоматически~--- на 
основе анализа содержательного портрета документа.
    
    В результате строятся так называемые \textit{аналитические 
фрагменты}, которые представляют в сжатом виде наиболее значимую 
информацию об объекте или происшествии и которые дополняют 
содержательный портрет документа. Они играют важную роль при поиске и 
аналитической обработке.
    
    Рассмотрим примеры работы программы постлингвистической обработки 
на документах из об\-ласти криминальной полиции.
    
    \medskip
    
    \noindent
    \textbf{Пример~2.} Формирование по тексту описания словесного 
портрета фигуранта.
    
    Основные классы онтологической базы, характеризующие фигурантов 
(лиц): \textit{пол}, \textit{возраст}, \textit{рост}, \textit{особые приметы}, 
\textit{индивидуальные особенности}, \textit{телосложение}, \textit{тип лица}, 
\textit{волосы}, \textit{глаза}, \textit{лоб}, \textit{брови}, \textit{нос}, \textit{рот}, 
\textit{губы}, \textit{зубы}, \textit{подбородок}, \textit{уши}, \textit{одежда}.
    
    Текст на входе:
    
    \ldots\textit{На вид 45~лет, рост 170--175~см, полного т/сл., одет в 
рыжую лохматую шапку, зеленый пуховик, черные брюки, зимние ботинки 
коричневого цвета}\ldots
    
    На выходе~--- аналитический фрагмент, представляющий в 
формализованном виде следующую информацию:

\smallskip
     
{\footnotesize

\noindent
     \textit{ВОЗРАСТ}: 45,
     
\noindent
     \textit{РОСТ}: 170, 175,
     
\noindent
     \textit{ТЕЛОСЛОЖЕНИЕ}: ПОЛНЫЙ,
     
\noindent
\textit{ОДЕЖДА}: ШАПКА (РЫЖИЙ, ЛОХМАТЫЙ),

\noindent ПУХОВИК (ЗЕЛЕНЫЙ), БРЮКИ (ЧЕРНЫЙ),

\noindent 
БОТИНОК (ЗИМНИЙ, КОРИЧНЕВЫЙ).

}

\smallskip
     
    Каждое слово с двоеточием представляет класс. Далее следуют подклассы. 
Слова в скобках поясняют или уточняют эти подклассы.
    
    Подобное формализованное описание играет роль реферата. Оно строится 
по аналитическому фрагменту автоматически с помощью обратного 
лингвистического процессора. В~данном случае программа 
постлингвистической обработки осуществляет автоматическое построение 
словесного портрета по тексту описания с его формализацией.
    
    \medskip
    
    \noindent
    \textbf{Пример~3.} Выявление из текста описания основных 
характеристик происшествия.
    
    Основные классы онтологической базы, характеризующие криминальные 
происшествия: \textit{предварительные действия}, \textit{способ 
проникновения}, \textit{способ совершения преступления}, \textit{преступные 
действия}, \textit{предлог}, \textit{организация}, \textit{оружие}, 
\textit{транспортные средства}, \textit{ценные бумаги}, \textit{драгоценные 
изделия}, \textit{ценные изделия}.
    
    Текст на входе:
    
    \ldots\textit{Найдена а/м ВАЗ 2109 темно-вишневого цвета г.~н.\ К 939 ЕМ 
70, в которой на передних сиденьях находятся два трупа мужчин кавказской 
национальности на вид 30--35~лет. Исследование показало, что смерть 
данных лиц наступила от огнестрельного ранения. На месте преступления 
были найдены и изъяты стреляные 2~гильзы от пистолета ТТ и 5~стреляных 
гильз, пуля от ПМ}.
    
    На выходе~--- аналитический фрагмент, представляющий в 
формализованном виде следующую информацию:

     \smallskip
     
{\small 
\noindent
     \textit{Преступные действия}: РАНЕНИЕ (ОГНЕСТРЕЛЬНЫЙ),
     
\noindent
     \textit{ЛИЧНОСТЬ}: НАЦИОНАЛЬНОСТЬ (Лицо кавказской национальности),
     
\noindent
\textit{ОРУЖИЕ}: ПИСТОЛЕТ (ТТ,ПМ),
     
\noindent
     \textit{АВТОМАШИНА}: ВАЗ.
     
     }
     
     \smallskip
     
    Подобное описание (как и в предыдущем примере) строится 
автоматически и играет роль сжатого описания или реферата.
    
\section{Этапы поиска}
    
    Поиск похожих происшествий и фигурантов (как и других 
информационных объектов) осуществляется по запросам и заключается в 
анализе содержательных портретов документов на предмет их совпадения с 
содержательным портретом запроса~\cite{3sha, 8sha, 9sha}. Анализ 
осуществляется на уровне структур знаний, находящихся в оперативной 
БЗ. Вначале выделяются объекты запроса (например, фигуранта), их 
признаки~--- \textit{приметы}, \textit{деяния}, а также связанные с ними 
\textit{адреса}, \textit{телефоны}, \textit{машины} и~др. Анализ сводится к 
проверке наличия в документах объектов (фигурантов) с аналогичными 
признаками и связями. При этом используются следующие признаки и связи:
    \begin{itemize}
    \item первичные признаки (значимые слова запроса в каноническом виде);
    \item вторичные признаки (близкие по смыслу слова, уточняющие слова 
и~др.), порожденные первичными признаками за счет информации 
онтологической базы;
    \item аналитические признаки (\textit{способ проникновения}, \textit{способ 
совершения преступления} и~др.), взятые из аналитических фрагментов;
    \item свойства объектов (например, для фигурантов~--- 
\textit{неизвестный}, \textit{потерпевший}, \textit{безработный}, для действий 
и событий~--- \textit{время}, \textit{место});
    \item участие объектов в действиях.
    \end{itemize}
    
    Отметим, что в качестве запроса может быть взят любой документ или 
словесный портрет фигуранта. Тогда вначале будет сформирован 
содержательный портрет, в котором будут все объекты запроса. Далее 
пользователь может выбрать любой из них. Тогда система на содержательном 
уровне будет искать похожие объекты. Если пользователь выберет документ, то 
будет инициирован поиск похожих происшествий. Поиск является нечетким, 
так как не требуется полного совпадения слов-признаков. Находится только то, 
что является общим и объединяет запрашиваемый и найденный объекты. Это 
важно, так как точный поиск часто не дает результата.
    
\subsection*{Этапы поиска похожих происшествий и~фигурантов}
    
    \textbf{Первый этап.} Выделение значимых слов-приз\-на\-ков из 
содержательного портрета запроса с присвоением им весов. Они образуют 
первичные признаки. Под значимыми понимаются слова, которые не являются 
предлогами, союзами, понятиями широкого объема, вспомогательными 
глаголами и~др. Напомним, что значимые слова блоком морфологического 
анализа приводятся в каноническую форму~\cite{4sha}.
    
    Если стоит задача поиска похожих про\-ис\-шест\-вий, то используются все 
значимые слова запроса, дополненные аналитическими признаками. Если 
решается задача поиска похожих фигурантов, то из запроса выделяются только 
те части, которые относятся к указанному фигуранту: его ФИО, а также 
\textit{приметы}, \textit{адрес}, \textit{деяния} и~т.\,д. Только из этих частей 
берутся слова, которые в дальнейшем будут играть роль первичных признаков.
    
    Выделенным словам-признакам присваиваются веса в зависимости от 
уникальности слова и вида информации (куда отнесено слово~--- к приметам, 
адресам и~др.). Наибольшие веса присваиваются аналитическим признакам, 
относящимся к характеру происшествия и к фигурантам, описанным в запросе.
    
    Отметим, что веса фактически отражают степень значимости той или иной 
информации при анализе степени сходства. В~системе <<Криминал>> имеются 
специальные настроечные фреймы, да\-ющие возможность пользователю 
изменять веса аналитических данных (\textit{способ проникновения}, 
\textit{оружие} и~т.\,п.) и категорий (\textit{приметы}, \textit{адреса} и~т.\,п.). 
Соответственно, будут меняться веса аналитических и других признаков. Таким 
способом акцентируется внимание на определенных моментах. Пример настроечного фрейма, 
задающего веса при поиске похожих происшествий
(веса получены экспериментальным путем):
\begin{center} %fig1
\vspace*{6pt}
\mbox{%
 \epsfxsize=78mm
 \epsfbox{sha-1.eps}
}
\end{center}
\begin{center}
\vspace*{-9pt}
%{{\figurename~1}\ \ \small{Изображение отпечатка пальца}}
\end{center}
%\vspace*{11pt}

%\smallskip
%\addtocounter{figure}{1}


 

        
    Аналогичный вид (с другими категориями и весами) имеет настроечный 
фрейм, определяющий веса слов-признаков при поиске похожих фигурантов.
    
    Например, если дать высокий вес классу <<\textit{способ 
проникновения}>> то система будет присваивать высокие веса происшествиям 
(документам) со способом проникновения, описанным в запросе. Если дать 
высокий вес адресам, то большой вес будет присваиваться документам с 
такими же улицами, номерами домов и квартир, как в запросе.
    
    \textbf{Второй этап.} Дополнение набора признаков за счет 
онтологической базы. Это необходимо для расширения пространства поиска. 
Нужно учесть различные способы и средства описания того, что есть в запросе. 
На базе имеющихся признаков запроса порождаются вторичные признаки:
    \begin{itemize}
    \item близкие по смыслу термины (на основе фрагментов NEAR);
    \item поясняющие термины (на основе фрагментов SUB);
    \item противоречивые признаки (на основе фрагментов OR\_OR).
    \end{itemize}
    
    Примеры фрагментов, взятых из онтологической базы:
    
    \smallskip
     

{\small

\noindent
     NEAR(АТЛЕТИЧЕСКИЙ,МОГУЧИЙ,МОЩНЫЙ,
     
     \noindent
     \hspace*{5mm}БОГАТЫРСКИЙ,СПОРТИВНЫЙ,К
РЕПКИЙ)

     
\noindent
     SUB(ДОКУМЕНТ,ПАСПОРТ)
     
\noindent
     SUB(ДОКУМЕНТ,УДОСТОВЕРЕНИЕ)
     
\noindent
     SUB(ДОКУМЕНТ,``Водительские права'')
     
\noindent
     SUB(ДОКУМЕНТ,``Воинский билет'')
     
\noindent
     SUB(ДОКУМЕНТ,МЕТРИКА)
     
\noindent
     SUB(ДОКУМЕНТ,ПРОПУСК)
     
\noindent
     OR\_OR(МОЛОДОЙ,``средний возраст'', ПОЖИЛОЙ)
     
\noindent
     NEAR(ПОЖИЛОЙ,СТАРЫЙ)
     
\noindent
     \ldots
     
     }
     
     \smallskip
     
    Поясним роль этих фрагментов на примерах. Если в запросе встретился 
признак БОГАТЫРСКИЙ (относящийся к фигуранту), то за счет фрагмента 
NEAR будут сформированы вторичные признаки: АТЛЕТИЧЕСКИЙ, 
МОГУЧИЙ, МОЩНЫЙ, СПОРТИВНЫЙ, КРЕПКИЙ, которые будут 
принимать участие при поиске.
    
    Если в запросе встретился термин ДОКУМЕНТ, то за счет фрагментов 
типа SUB будут сформированы способы его расшифровки: это может быть 
ПАСПОРТ, УДОСТОВЕРЕНИЕ, <<Водительские права>>, <<Воинский 
билет>>, МЕТРИКА, ПРОПУСК. Они будут также учитываться при поиске.
    
    Если в запросе фигурант был охарактеризован как ПОЖИЛОЙ (это могла 
сделать и сама система путем анализа возраста), то за счет фрагментов типа 
OR\_OR будут сформированы опровергающие признаки: МОЛОДОЙ, 
<<средний возраст>>, которые используются при оценке степени сходства со 
знаком минус.
    
    Вторичным признакам также присваиваются веса~--- в зависимости от веса 
признака, который их породил.
    
    При наличии в запросе фигурантов производится анализ их ФИО. 
Отметим, что полные имена и отчества при построении содержательного 
портрета уже преобразуются к единому виду~--- в каноническую форму. На 
данном этапе на их основе порождаются инициалы, которые тоже играют роль 
признаков. И~наоборот, по инициалам порождаются возможные имена и 
отчества. Это позволяет при поиске более полно охватить возможные случаи 
написания ФИО.
    
    \textbf{Третий этап.} Быстрый поиск (по индексным файлам) в базах 
данных содержательных портретов документов с указанными признаками. 
Поиск осуществляется на основе выделенных из запроса слов-признаков и 
заключается в подсчете взвешенной суммы весов совпавших признаков. 
В~качестве результата выдаются номера найденных документов~--- в порядке 
взвешенных сумм.
    
    При наличии в запросе фигурантов с ФИО производится дополнительный 
поиск документов, при котором обязательными признаками делаются пары: 
полные фамилия и имя или полные имя и отчество. Словом, находятся 
документы, где есть и то, и другое. Это позволяет избежать потерь при поиске и 
идентификации фигурантов.
    
    \textbf{Четвертый этап.} Подкачка из базы данных в оперативную память 
семантических сетей~--- содержательных портретов документов с 
наибольшими взвешенными суммами. В~результате (вместе с онтологической 
базой) образуется ОБЗ, представляющая собой 
большую семантическую сеть, доступную для быстрого выполнения сложных 
операций сравнения и логического анализа.
    
    \textbf{Пятый этап.} Детальный анализ на совпадение слов-признаков, 
связанных с объектами (или выбранным объектом) запроса и объектами, 
находящимися в ОБЗ. При этом сравнение идет по категориям: ФИО фигуранта 
из запроса сравнивается с ФИО фигурантов из ОБЗ, связанные с ними приметы 
сравниваются с приметами, свойства~--- со свойствами, действия~--- с 
действиями и~т.\,д. В~результате находятся похожие объекты. Подсчитывается 
их вес, который определяет степень сходства с объектами (или объектом) 
запроса. Выбираются объекты с наибольшими весами. При этом учитываются 
следующие факторы:
    \begin{itemize}
    \item веса совпавших порожденных аналитических признаков, 
определяющих характер объекта или особенности фигуранта;
    \item веса совпавших слов-признаков (в том числе вторичных);
    \item соотнесенность признаков к той или иной категории;
    \item связь признаков, заданная в онтологической базе (близкие по смыслу 
или поясняющие);
    \item сильное совпадение по какой-либо категории признаков (например, 
совпадает большинство примет);
    \item наличие противоречивых признаков.
    \end{itemize}
    
    Каждое совпадение дополняет общий вес выбранного объекта~--- к нему 
добавляется вес совпавшего признака. При наличии противоречивых признаков 
их веса вычитаются.
    
    При анализе чисел и интервалов на их совпадение (например, ВОЗРАСТ, 
РОСТ, номер дома, квартиры, год и~др.)\ рассматриваются различные 
варианты:
    \begin{itemize}
    \item равенство чисел;
    \item число входит в интервал;
    \item пересечение интервалов;
    \item близость числовых значений.
    \end{itemize}
    
    В зависимости от варианта совпадения и от категории (приметы, адрес, 
время и~др.)\ к общему весу документа добавляется определенная величина.
    
    При поиске похожих происшествий в первую очередь учитывается 
сходство аналитических признаков и криминальных действий, а уже затем 
объектов, участвующих в этих действиях. Для \mbox{этого} категориям присваиваются 
соответствующие веса. Общий вес выбранного (из ОБЗ) происшествия 
подсчитывается как сумма весов входящих в него признаков и объектов 
(действие~--- это тоже объект).
    
    При поиске фигурантов различается два случая. 
    
    Первый случай~--- когда в 
запросе заданы ФИО фигуранта. Тогда в ОБЗ производится поиск фигурантов с 
аналогичными ФИО. При этом учитываются случаи совпадения инициалов с 
полными именами или отчествами (такое совпадение дает меньший вес). 
Подсчитывается общая степень совпадения~--- в зависимости от совпавших 
признаков.
    
    Множество найденных фигурантов с высокими весами является основой 
для дальнейшего анализа. К~ним добавляются веса, полученные от совпадения 
свойств, а также от совпадения связанных с фигурантами примет, адресов, 
телефонов и~др. В~результате находятся фигуранты с высокими весами, 
отражающими степень сходства с лицом, описанным в запросе.
    
    Если в ОБЗ не найдено фигурантов с ФИО, заданными в запросе, то 
ищутся фигуранты, у которых может быть другое имя или отчество. Возникают 
противоречивые признаки, которые уменьшают вес анализируемого фигуранта. 
При этом акцент смещается на сравнение связей.
    
    Второй случай~--- когда запрашивается неизвестное лицо (фигурант). 
Тогда поиск и сравнение идет по связанным с этим лицом приметам, 
действиям, адресам и другим объектам. В~ОБЗ ищутся лица с аналогичными 
связями.
    
    При поиске похожих происшествий найденные фигуранты с их степенями 
совпадения запоминаются, а сами степени дополняют вес соответ\-ст\-ву\-юще\-го 
документа. Помимо этого учитываются веса других совпавших признаков. 
В~результате находятся происшествия (документы) с высокими весами, 
отражающими степень сходства с запросом.
    
    \textbf{Шестой этап.} Выдача похожих происшествий или фигурантов, 
упорядоченных по степени сходства, в виде списка или меню.
    
    \textbf{Седьмой этап.} Выдача объяснений. Пользователь может выбрать 
из упомянутого меню любой пункт, соответствующий происшествию или 
фигуранту. Система на основе совпавших признаков формирует объяснение 
сходства в виде понятного текста на русском языке.
    
\section{Выдача и~объяснение результатов}
    
    Как отмечалось ранее, вся обработка в системе <<Криминал>> 
осуществляется на уровне семантических сетей в рамках специально 
созданного для этого инструментария языка~--- ДЕКЛ~\cite{2sha, 8sha}. Находятся 
фрагменты семантической сети, представляющие похожие происшествия или 
фигурантов с совпавшими признаками. При выдаче соответствующего меню и 
объяснении результатов такие фрагменты преобразуются на понятный 
пользователю язык~--- естественный. Это делается с помощью обратного 
лингвистического процессора.
    
    При формировании меню формируются краткие описания происшествий 
или фигурантов. При объяснении результатов (когда пользователь выбирает из 
меню интересующее его происшествие или фигуранта) дается краткое описание 
выбранного происшествия (фигуранта), указываются совпавшие и 
противоречивые признаки, а также дается сам текст описания. Этого 
достаточно, чтобы помочь пользователю самому оценить степень сходства или 
адекватности запросу.
    
    \medskip
    
    \noindent
    \textbf{Пример~4.} Проиллюстрируем сказанное на примере выдачи 
результатов поиска похожих криминальных происшествий и похожих 
фигурантов.
    
    Текст на входе (взят из документа с номером~63):
    \ldots\textit{На лестничной площадке 3-го этажа двое неизвестных из 
неустановленного оружия нанесли два сквозных ранения в голову и живот 
Лихомову Владимиру Ивановичу, 1954~г.р., неработающий, прож.: 
Тюменская\ldots С~места происшествия изъято: 1 пуля и 1 гильза калибра 
7.62~мм предположительно от пистолета ТТ}\ldots
    
    Меню похожих происшествий выглядит следующим образом:
    
     На документ 63 содержательно похожи:
     \begin{itemize}
     \item Док-т 1231 (БЗ-1) УБИЙСТВО 29.3.1996 (вес~142);
     \item Док-т 4323 (БЗ-1) ОГРАБЛЕНИЕ 20.6.1996 (вес~111);
     \item Док-т 81 (БЗ-2) УБИЙСТВО 1.7.1995 (вес~92);
     \item \ldots
     \end{itemize}
    
    При выборе пункта~1 данного меню на экран будет выдано объяснение 
причин сходства документов~63 и~1231:

         Похожее происшествие~--- 1231 из БЗ-1 (вес~142).
         
         В происшествии~1231 встретились те же приз\-наки:
         \pagebreak
         
%         \smallskip
     
     \textit{Преступные действия}: РАНЕНИЕ, ГОЛОВА.
     
     \textit{Оружие}: ПИСТОЛЕТ, ТТ.
     
     \textit{Фабула}: УБИЙСТВО.
     
     \textit{Работа}: НЕРАБОТАЮЩИЙ.
     
     \textit{Действие}: ИЗЪЯТЬ ГИЛЬЗА.
     
     \textit{Общие сведения}: РЕЗУЛЬТАТ, ПРОВЕДЕНИЕ, 
     
     СОТРУДНИК, ОКАЗАТЬСЯ, КАЛИБР.
     
\noindent
     $<$Текст документа 1231 из БЗ-1$>$.
     
     \medskip
     
     \noindent
     \textbf{Пример~5.} Меню похожих фигурантов выглядит следующим 
образом:
    
     На фигуранта ЛИХОМОВ ВЛАДИМИР ИВАНОВИЧ похожи:
     \begin{itemize}
     \item ЛИХОМОВ ВЛАДИМИР ПЕТРОВИЧ 1943, док. 4437 из БЗ-1 (вес~42);
     \item без ФИО в кол-ве 1, док. 81 из БЗ-1 (вес~35);
     \item КОВАЛЕВ ВЛАДИМИР ИВАНОВИЧ 1956, док. 24 из БЗ-2 (вес~30);
     \item \ldots
     \end{itemize}
     
    При выборе пункта~1 данного меню на экран будет выдано объяснение 
причин сходства фигурантов ЛИХОМОВ ВЛАДИМИР ИВАНОВИЧ и 
ЛИХОМОВ ВЛАДИМИР ПЕТРОВИЧ:

     \smallskip
    
\noindent
     Похожий фигурант (вес 44)~--- ЛИХОМОВ ВЛАДИМИР ПЕТРОВИЧ 1943.
     
     \textit{Особые приметы}: ОТМЕТИНА (РАНЕНИЕ),
     
     \textit{ТЕЛОСЛОЖЕНИЕ}: ТОЛСТЫЙ,
     
     \textit{СТАТУС}: ПОТЕРПЕВШИЙ (РАНЕНИЕ).
     
          \smallskip
     
\noindent
     У фигуранта встретились те же признаки:
     
     \smallskip
     
     \textit{ТЕЛОСЛОЖЕНИЕ}: ТОЛСТЫЙ,
     
     \textit{СТАТУС}: ПОТЕРПЕВШИЙ РАНЕНИЕ,
     
     \textit{Работа}: НЕРАБОТАЮЩИЙ.
     
     ФИО: ЛИХОМОВ ВЛАДИМИР.
     
     \textit{Не совпадают ФИО}: 1943 (\textit{было} 1954).
     
     \textit{Не совпадают ФИО}: ПЕТРОВИЧ (\textit{было} ИВА-
     
     НОВИЧ).
     
     %\noindent
     \textit{Входит в документ с номером 4437 из БЗ-2}.
     
    \noindent
     $<$Текст документа 4437$>$.
     
          \smallskip
     
    Отметим некоторые важные моменты.
    
    Во-первых, в содержательных портретах представлены объекты и 
действия. Их сопоставление играет важную роль при поиске похожих 
происшествий, при классификации лиц как \textit{потерпевших} или 
\textit{преступников} и во многих других случаях.
    
    Во-вторых, дается оценка документа по ключевым позициям, 
представляющая в сжатом виде наиболее значимую информацию и играющая 
роль реферата.
    
    В-третьих, при анализе степени сходства запроса и документа 
используются признаки типа <<\textit{Преступные действия}>>, <<\textit{Угроза 
оружием}>> и~др., которые в явном виде могут отсутствовать в тексте и 
которые выявляются системой в процессе постлингвистической обработки. 
Соответственно, такие признаки вводятся в объяснения.
    
    В-четвертых, допускается поиск похожих фигурантов без ФИО (поиск 
неизвестных лиц) по связанной информации, например по словесному 
портрету.
    
    И последнее: использование привычных человеку классификаторов (они 
представлены в онтологической базе) делает результат реферирования и 
объяснения более понятным.
    
\section{Заключение}
    
    Особенность предлагаемых в данной статье методик и алгоритмов 
семантического поиска состоит в следующем:
    \begin{enumerate}[(1)]
    \item вся обработка осуществляется на уровне структур знаний, т.\,е.\ 
содержательных портретов\linebreak документов. Они образуют БЗ, 
которая вмес\-те с правилами преобразования (продукци-\linebreak ями языка ДЕКЛ) 
образует законченный технологический комплекс, ориентированный на 
сложные задачи, связанные с логическим выводом, преобразованием 
пред\-став\-ле\-ний, экспертными решениями. В~результате обеспечивается анализ 
высокой степени глубины и сложности;
    \item выделяются и используются разнообразные признаки. Учитывается 
наличие множества объектов (лиц, телефонов, оружия и~т.\,п.~--- до 40~типов) 
и аналитических признаков, характеризующих происшествия и фигурантов. 
Для расширения пространства поиска используется онтологическая база;
    \item допускается работа с многими БЗ, связанными через сеть или 
Интернет. Они образуют распределенную БЗ.
    \end{enumerate}
    
    Описанные методики и алгоритмы семантического поиска реализованы в 
рамках систем <<Аналитик>>, <<Криминал>>, <<Поток>> и апробированы при 
работе с различными корпусами текстов, среди которых: сообщения СМИ, 
сводки происшествий, обвинительные заключения, записные книжки 
фигурантов и~др. Эти методики использованы в различных приложениях и 
показали высокую степень универсальности. В~перспективе они могут 
послужить основой для создания комплекса поисковых программ, 
составляющих <<Семантический Интернет>>.
    
       
    {\small\frenchspacing
{%\baselineskip=10.8pt
\addcontentsline{toc}{section}{Литература}
\begin{thebibliography}{99}

     \bibitem{1sha}
     \Au{Кузнецов И.\,П.}
     Семантические представления.~--- М.: Наука, 1986. 290~с.
     
          \bibitem{4sha} %2
     \Au{Кузнецов И.\,П., Кузнецов В.\,П., Мацкевич~А.\,Г.}
     Система выявления из документов значимой информации на основе лингвистических 
знаний в форме семантических сетей~// Диалог-2000: Труды Междунар. семинара по 
компьютерной лингвистике и ее приложениям.~--- Протвино, 2000.  Т.~2.
     
     \bibitem{5sha} %3
     \Au{Kuznetsov I., Matskevich A.}
     System for extracting semantic information from natural language text~// Диалог-2002: 
Труды Междунар. семинара по компьютерной лингвистике и ее приложениям (Протвино).~--- 
М.: Наука, 2002. Т.~2.
     
          \bibitem{3sha} %4
     Лаборатория компьютерной лингвистики ИПИ РАН: Официальный сайт. {\sf 
www.IpiranLogos.com}.
     
     \bibitem{2sha} %5
     \Au{Кузнецов И.\,П., Шарнин М.\,М.}
     Продукционный язык программирования ДЕКЛ~// Система обработки декларативных 
структур знаний Деклар-2.~--- М.: ИПИ РАН, 1988.

 \bibitem{8sha} %6
     \Au{Кузнецов И.\,П., Мацкевич А.\,Г.}
     Семантико-ори\-ен\-ти\-ро\-ван\-ные системы на основе баз знаний.~--- М.: \mbox{МТУСИ}, 2007. 
173~с.
     
     \bibitem{9sha} %7
     \Au{Kuznetsov I.\,P., Kozerenko E.\,B.}
     Linguistic processor Semantix for knowledge extraction from natural texts in Russian and 
English~// ICAI 2008: 2008 Conference (International) on Artificial Intelligence Proceedings.~--- 
Las Vegas: CSREA Press, 2008. P.~835--841.
     
    
     \bibitem{6sha} %8
     \Au{Кузнецов И.\,П., Мацкевич А.\,Г.}
     Особенности организации базы предметных и лингвистических знаний в системе 
АНАЛИТИК~// Диа\-лог-2003: Труды Междунар. конф. по компьютерной лингвистике и 
интеллектуальным технологиям.~--- Протвино, 2003. С.~373--378.
     
     \bibitem{7sha} %9
     \Au{Кузнецов И.\,П., Мацкевич А.\,Г.}
     Англоязычная вер-\linebreak сия системы автоматического выявления значимой\linebreak
      информации из 
текстов естественного языка~// Диа\-лог-2005: Труды Междунар. конф. по компьютерной 
лингвистике и интеллектуальным технологиям (Звенигород).~--- М.: Наука, 2005. 
     С.~303--311.
     
    

\label{end\stat}
     
     \bibitem{10sha}
     \Au{Kuznetsov I.\,P.}
     Identifying role functions of persons on the basis of knowledge structures)~// Компьютерная 
лингвистика и интеллектуальные технологии: По мат-лам ежегодной 
Междунар. конф. <<Диалог 2011>>.~--- М.: РГГУ, 2011.  Вып.~10(17). С.~391--402.
    
 \end{thebibliography}
}
}


\end{multicols} %12
\def\stat{shestakov}

\def\tit{ОБРАЩЕНИЕ ОДНОРОДНЫХ ОПЕРАТОРОВ С~ПОМОЩЬЮ
СТАБИЛИЗИРОВАННОЙ ЖЕСТКОЙ ПОРОГОВОЙ ОБРАБОТКИ
ПРИ~НЕИЗВЕСТНОЙ ДИСПЕРСИИ ШУМА$^*$}

\def\titkol{Обращение однородных операторов с~помощью
стабилизированной жесткой пороговой обработки}
%при~неизвестной дисперсии шума}

\def\aut{О.\,В.~Шестаков$^1$}

\def\autkol{О.\,В.~Шестаков}

\titel{\tit}{\aut}{\autkol}{\titkol}

\index{Шестаков О.\,В.}
\index{Shestakov O.\,V.}


{\renewcommand{\thefootnote}{\fnsymbol{footnote}} \footnotetext[1]
{Работа выполнена при частичной финансовой поддержке РФФИ (проект 19-07-00352).}}


\renewcommand{\thefootnote}{\arabic{footnote}}
\footnotetext[1]{Московский государственный университет им.\ М.\,В.~Ломоносова, 
кафедра математической статистики факультета вычислительной математики и~кибернетики; 
Институт проб\-лем информатики Федерального исследовательского центра 
<<Информатика и~управ\-ле\-ние>> Российской академии наук, \mbox{oshestakov@cs.msu.su}}


\vspace*{-6pt}


\Abst{При обращении линейных однородных операторов обычно необходимо использовать 
методы регуляризации, поскольку наблюдаемые данные, как правило, зашумлены. 
Для подавления шума часто используется пороговая обработка 
вейвлет-ко\-эф\-фи\-ци\-ен\-тов функции наблюдаемого сигнала. 
Пороговая обработка стала популярным инструментом подавления 
шума благодаря своей простоте, вы\-чис\-ли\-тель\-ной эффективности и~воз\-мож\-ности 
адаптации к~функциям, имеющим на разных участках разную степень регулярности. 
Рассматривается предложенный недавно стабилизированный метод жесткой 
пороговой обработки, в~котором устранены основные недостатки мягкой и~жесткой 
пороговой обработки, и~исследуются статистические свойства этого метода. 
В~модели данных с~аддитивным гауссовским шумом с~неизвестной дисперсией 
проведен анализ несмещенной оценки среднеквадратичного риска и~показано, 
что при определенных условиях данная оценка является асимптотически нормальной, 
при этом дисперсия предельного распределения зависит от способа оценивания 
дисперсии шума.}

\KW{вейвлеты; пороговая обработка; несмещенная оценка риска; 
асимптотическая нормальность; сильная состоятельность}

\DOI{10.14357/19922264190107}
  
%\vspace*{4pt}


\vskip 10pt plus 9pt minus 6pt

\thispagestyle{headings}

\begin{multicols}{2}

\label{st\stat}

\section{Введение}

В медицинских, физических, астрономических и~других научных проблемах часто 
возникает задача получить представление об объекте, который описывается 
некоторой функцией~$f$, имея возможность наблюдать только функцию~$Kf$, где~$K$~--- 
некоторый линейный оператор. При этом часто нельзя просто применить 
к~наблюдаемым данным обратный оператор~$K^{-1}$, поскольку эти данные, как правило, 
содержат шум и~задача обращения оператора~$K$ некорректно поставлена. 
К~тому же обычно дис\-пер\-сия шума неизвестна и~ее необходимо оценивать 
по наблюдаемым данным. 

Одним из популярных инструментов при регуляризации 
процедуры обращения служит вейв\-лет-раз\-ло\-же\-ние с~последующей 
пороговой обработкой вейв\-лет-ко\-эф\-фи\-ци\-ен\-тов. Наиболее распростра\-нен\-ные 
виды пороговой обработки~--- жесткая и~мягкая. В~работе~\cite{HL10} 
был предложен метод стабилизированной жесткой пороговой обработки, который 
объединяет в~себе преимущества этих двух видов. 
В~ситуации, когда дисперсия шума предполагается известной, в~работе~\cite{SH18} 
доказана асимптотическая нормальность оценки среднеквадратичного риска пороговой 
обработки. 

В~данной работе исследуется влияние способов оценивания дисперсии шума 
на характеристики предельного распределения оценки среднеквадратичного риска. 
Для метода мягкой пороговой обработки подобные исследования проводились 
в~работах~\cite{KS11-1, KS11-2}.

\section{Обращение линейных однородных операторов с~помощью вейглет-вейвлет-разложения}

В данной работе рассматривается метод обращения линейных однородных операторов, 
основанный на вейг\-лет-вейв\-лет-раз\-ло\-же\-нии~\cite{AS98}. Линейный оператор~$K$ 
называется однородным, если
$$
K\left[f\left(a\left(x-x_0\right)\right)\right]=a^{-\alpha}(Kf)\left[a\left(x-x_0\right)\right]
$$
для любого $x_0$ и~любого $a\hm>0$. Параметр~$\alpha$ называется показателем 
однородности. Примерами линейных однородных операторов служат оператор 
интегрирования, преобразование Гильберта и~преобразование Абеля.

Относительно наблюдаемой функции~$Kf$ будем предполагать, что она определена на 
конечном отрезке и~равномерно регулярна по Липшицу с~некоторым показателем $\gamma\hm>0$. 
Вейв\-лет-разложение~$Kf$ представляет собой ряд по ортонормированному базису
\begin{equation}
\label{wavelet_decomp}
Kf = \sum\limits_{j,k \in Z} \langle Kf,\psi_{j,k} \rangle \psi_{j,k}\,,
\end{equation}
где $\psi(t)$~--- некоторая материнская вейв\-лет-функ\-ция, 
а~$\psi_{j,k}(t) \hm= 2^{j/2}\psi(2^jt \hm- k)$. Индекс~$j$ в~(\ref{wavelet_decomp}) 
называется масштабом, а~индекс~$k$~--- сдвигом. Если вейв\-лет-функ\-ция 
обладает определенными свойствами регулярности~\cite{Mal99}, 
то для коэффициентов разложения в~(\ref{wavelet_decomp}) справедливо
\begin{equation}
\label{wavelet_decay}
\abs{\langle Kf, \psi_{j,k} \rangle} \leqslant \fr{C_f}
{2^{j \left( \gamma + 1/2 \right)}}\,,
\end{equation}
где $C_f$~--- некоторая положительная константа.

Поскольку оператор~$K$ линеен и~однороден, существуют такие функции~$u_{j,k}$, 
что $\langle f,u_{j,k}\rangle\hm=\langle Kf,\psi_{j,k}\rangle$. При этом функция~$f$ 
представляется в~виде ряда
\begin{equation}
\label{VWD}
f = \sum\limits_{j,k \in Z}\beta_{j,k}\langle Kf,\psi_{j,k}\rangle u_{j,k},
\end{equation}
где $u_{j,k} = K^{-1}\psi_{j,k}/\beta_{j,k}$, $\beta_{j,k}\hm=2^{\alpha j}\beta_{00}$, 
$\beta_{00} \hm= \norm{K^{-1}\psi}$ (функции~$u_{j,k}$, как и~$\psi_{j,k}$, 
представляют собой сдвиги и~растяжения одной материнской функции~$u$ и~называются 
вейглетами). При соответствующем выборе~$\psi(t)$ последовательность~$\{u_{j,k}\}$ 
образует устойчивый базис~\cite{L97}. Формула~(\ref{VWD}) и~есть основа метода 
вейг\-лет-вейв\-лет-раз\-ло\-же\-ния.

\section*{Пороговая обработка эмпирических коэффициентов}

При фактических измерениях значения функции сигнала регистрируются 
в~дискретных отсчетах, при этом такие значения, как правило, зашумлены. 
Рассмотрим сле\-ду\-ющую модель данных \mbox{с~шумом}:
\begin{equation*}
%\label{Data_Model}
X_i = (Kf)_i + \epsilon_i\,, \enskip i = 1, \dots, 2^J\,, %\notag
\end{equation*}
где $2^J$~--- число отсчетов; $(Kf)_i$~--- незашумленные значения функции сигнала; 
$\epsilon_i$~--- независимые нормально распределенные случайные величины с~нулевым 
средним и~дисперсией~$\sigma^2$.
После применения дискретного вейв\-лет-пре\-об\-ра\-зо\-ва\-ния 
получается следующая модель зашумленных вейв\-лет-ко\-эф\-фи\-ци\-ен\-тов:
\begin{equation*}
Y_{j,k}=\mu_{j,k}+\epsilon^W_{j,k},\enskip 
j=0,\ldots,J-1,\ k=0,\ldots,2^{j}-1\,,
\end{equation*}
где $\epsilon^W_{j,k}$ независимы и~распределены так же, как и~$\epsilon_i$, 
а~$\mu_{j,k}\hm= 2^{J/2}\langle Kf,\psi_{j,k}\rangle$~\cite{Mal99}.

Для подавления шума и~построения оценки функции сигнала к~коэффициентам~$Y_{j,k}$ 
обычно применяется функция жесткой пороговой обработки 
$\rho_{H}(y,T)\hm=x\textbf{I}(\abs{y}>T)$ или мягкой пороговой 
обработки $\rho_{S}(y,T)\hm=\textbf{sgn}(x)\left(\abs{y}-T\right)_{+}$ 
с~порогом~$T$. При таком подходе обнуляются коэффициенты, абсолютная величина 
которых ниже порога, так как в~силу~(\ref{wavelet_decay}) основная часть
 полезного сигнала содержится в~относительно небольшом числе больших по 
 модулю коэффициентов.

Каждому из этих видов пороговой обработки присущи свои недостатки. 
Жесткая пороговая функция разрывна, и~это приводит к~отсутствию устойчивости 
при выборе порога~\cite{B96} и~невозможности построения несмещенной оценки 
среднеквадратичного риска~\cite{J01}. При мягкой пороговой обработке в~оценке 
функции появляется дополнительное смещение. Чтобы частично избежать этих недостатков, 
в~работе~\cite{HL10} был предложен новый вид пороговой обработки, представляющий 
собой сглаженный (стабилизированный) аналог жесткой пороговой обработки. 
В~этом методе оценки~$\mu_{j,k}$ вычисляются по формулам:
\begin{equation*}
\widehat{\mu}_{j,k}=\Expect 
\left[\rho_{H}(Y_{j,k}+\lambda\xi_{j,k},T_j)|Y_{j,k}\right], %\notag
\end{equation*}
где случайные величины~$\xi_{j,k}$ имеют стандартное нормальное распределение и~не 
зависят от~$Y_{j,k}$, а~$\lambda\hm>0$~--- 
параметр стабилизации, отвечающий за степень сглаживания. Вычисляя математическое 
ожидание, получаем:
\begin{multline*}
\hspace*{-8.37947pt}\widehat{\mu}_{j,k}=Y_{j,k}\left[\Phi\!\left(-\fr{T_j+Y_{j,k}}
{\lambda}\right)+1-\Phi\left(\fr{T_j-Y_{j,k}}{\lambda}\right)\!\right]+{}\\
{}+
\lambda\left[\phi\left(\fr{T_j-Y_{j,k}}{\lambda}\right)-
\phi\left(\fr{T_j+Y_{j,k}}{\lambda}\right)\right]. %\notag
\end{multline*}
Достоинством такого метода является бесконечная дифференцируемость~$\widehat{\mu}_{j,k}$ 
по~$Y_{j,k}$, что приводит к~более робастным оценкам~\cite{HL10}. Заметим также, 
что при $\lambda\hm\to0$ получается обычный метод жесткой пороговой обработки. 
В~данной работе параметр~$\lambda$ предполагается фиксированным, а~в~качестве~$T_j$ 
для каждого масштаба~$j$ выбирается порог $T_j\hm=\sigma\sqrt{2\ln 2^j}$. 
Такой порог получил название <<универсальный>>, так как он не зависит 
от наблюдаемых данных. И~при жесткой, и~при мягкой пороговой обработке этот 
порог обеспечивает близость среднеквадратичного риска к~минимальному~\cite{Mal99}.

\section{Несмещенная оценка среднеквадратичного риска}

Среднеквадратичный риск метода пороговой обработки определяется по формуле:
\begin{equation}
\label{Risk}
R_J(\sigma)=\sum\limits_{j=0}^{J-1}\sum\limits_{k=0}^{2^j-1}\beta^2_{j,k}
\Expect\left(\widehat{\mu}_{j,k}(\sigma)-\mu_{j,k}\right)^2.
\end{equation}
В~\cite{HL10} показано, что при стабилизированной жесткой пороговой обработке
\begin{multline*}
\Expect\left(\widehat{\mu}_{j,k}(\sigma)-\mu_{j,k}\right)^2={}\\
{}=
\Expect\left[(Y_{j,k}-\widehat{\mu}_{j,k}(\sigma))^2+
2\sigma^2\fr{\partial}{\partial Y_{j,k}}\,\widehat{\mu}_{j,k}(\sigma)\right]-
\sigma^2, %\notag
\end{multline*}
где
\begin{multline*}
\fr{\partial}{\partial Y_{j,k}}\widehat{\mu}_{j,k}(\sigma)={}\\
{}=\Phi\left(-\fr{T_j+Y_{j,k}}{\lambda}\right)+1-
\Phi\left(\fr{T_j-Y_{j,k}}{\lambda}\right)+{}\\
{}+
\fr{T_j}{\lambda}\left[\phi\left(\fr{T_j-Y_{j,k}}{\lambda}\right)+
\phi\left(\fr{T_j+Y_{j,k}}{\lambda}\right)\right]. %\notag
\end{multline*}
Таким образом, величина
\begin{multline}
\label{Risk_Estimate}
\widehat{R}_J(\sigma)=\sum\limits_{j=0}^{J-1}\sum\limits_{k=0}^{2^j-1}
\beta^2_{j,k}
\Bigg[
\left(
Y_{j,k}-
\widehat{\mu}_{j,k}(\sigma)\right)^2+{}\\
{}+2\sigma^2\fr{\partial}{\partial Y_{j,k}}\,\widehat{\mu}_{j,k}(\sigma)-
\sigma^2
\Bigg]
\end{multline}
является несмещенной оценкой~$R_J$, не зависящей от ненаблюдаемых значений~$\mu_{j,k}$.

В работе~\cite{SH18} доказано следующее утверждение, устанавливающее 
асимптотическую нормальность оценки~(\ref{Risk_Estimate}) и~позволяющее строить 
асимптотические доверительные интервалы для риска~(\ref{Risk}).

\smallskip

\noindent
\textbf{Теорема 1.} 
\textit{Пусть $K$~--- линейный однородный оператор с~показателем 
однородности $\alpha\hm>0$, а~$Kf$ задана на конечном отрезке и~равномерно 
регулярна по Липшицу с~показателем $\gamma\hm>0$. Тогда}
\begin{equation*}
%\label{Normality}
{\sf P}\left(\fr{\widehat{R}_J(\sigma)-
R_J(\sigma)}{D_J}<x\right)\Rightarrow\Phi(x)\,, %\notag
\end{equation*}
\textit{где}
$$
D^2_J=\fr{2\sigma^4\beta_{0,0}^4}{2^{4\alpha+1}-1}2^{(4\alpha+1)J}\,.
$$

\section{Виды оценок дисперсии шума}

Как правило, дисперсия~$\sigma^2$ неизвестна и~вместо ее точного значения 
необходимо использовать некоторую оценку~$\hat{\sigma}^2$, которая обычно 
строится по половине всех вейв\-лет-ко\-эф\-фи\-ци\-ен\-тов для $j\hm=J\hm-1$, 
так как в~силу~(\ref{wavelet_decay}) эти коэффициенты фактически содержат только шум. 
При этом порог вычисляется по формуле $\hat{T}_j\hm=\hat{\sigma}\sqrt{2\ln 2^j}$.

В качестве оценки~$\sigma^2$ (или $\sigma$) в~данной работе 
рассматривается выборочная дисперсия
\begin{equation}
\label{SampleVarianceDef}
\widehat{\sigma}_S^2=\fr{1}{2^{J-1}}
\sum\limits_{k=0}^{2^{J-1}-1}Y_{J-1,k}^2-\overline{Y}^2,
\end{equation}
где
\begin{equation*}
\overline{Y}=\fr{1}{2^{J-1}}\sum\limits_{k=0}^{2^{J-1}-1}Y_{J-1,k}\,,
\end{equation*}
а также соответствующим образом нормированный выборочный интерквартильный 
размах~$\widehat{\sigma}_{R}$ и~выборочное абсолютное медианное 
отклонение~$\widehat{\sigma}_{M}$, которые определяются сле\-ду\-ющим образом:
\begin{align}
\widehat{\sigma}_{R}&=\fr{Y_{(J-1,3/4)}-Y_{(J-1,1/4)}}{2\xi_{3/4}}\,;
\label{IQR_Definition}
\\
\widehat{\sigma}_{M}&=\fr{\mathop{\mbox{med}}\limits_{0\leqslant k\leqslant 2^{J-1}-1}|Y_{J-1,k}-\mathop{\mbox{med}}\limits_{0\leqslant l\leqslant 2^{J-1}-1} Y_{J-1,l}|}{\xi_{3/4}}\,.
\label{MAD_Definition}
\end{align}
Здесь $Y_{(J-1,1/4)}$ и~$Y_{(J-1,3/4)}$~--- выборочные квантили порядка~$1/4$ и~$3/4$, 
построенные по выборке из половины всех вейв\-лет-ко\-эф\-фи\-ци\-ен\-тов при 
$j\hm=J\hm-1$; $\xi_{3/4}$~--- теоретическая квантиль порядка~$3/4$ 
стандартного нормального распределения ($\xi_{3/4}\hm\approx0,6745$); $\mbox{med}$ 
обозначает выборочную медиану.

Выборочная дисперсия служит самой популярной оценкой величины~$\sigma^2$, и~в~случае 
отсутствия выбросов она наиболее предпочтительна. Однако в~случае, когда 
оценка дисперсии строится по выборке сигнала, естественно ожидать, 
что выборка не будет однородной. Преимущество использования последних 
двух оценок заключается в~их ро\-баст\-ности, т.\,е.\ нечувствительности к~выбросам.

\section{Предельная дисперсия оценки среднеквадратичного риска}

Способ оценивания дисперсии шума влияет на вид предельной дисперсии 
оценки среднеквадратичного риска. Подобный эффект наблюдается и~при 
мягкой пороговой обработке~[4].

\noindent
\textbf{Теорема~2.}\ \textit{Пусть $Kf$ задана на конечном отрезке и~равномерно 
регулярна по Липшицу с~показателем $\gamma\hm>1/4$, а оценка дисперсии 
шума задана соотношением}~\eqref{SampleVarianceDef}. \textit{Тогда}
\begin{equation}
\label{CLT_Operator_SampleVar_Sigma}
\mathsf{P}\left(\frac{\widehat{R}_J(\widehat{\sigma}_S)-R_J(\sigma)}{D_J}<x\right)
\Rightarrow \Phi_{\Upsilon_1}(x),\notag
\end{equation}
\textit{где $\Phi_{\Upsilon_1}(x)$~--- функция распределения нормального 
закона с~нулевым средним и~дисперсией}
$$
\Upsilon_1^2=\fr{1}{2^{4\alpha+1}}+
\fr{2^{4\alpha+1}-1}{2^{4\alpha+1}\left(2^{2\alpha+1}-1\right)^2}\,.
$$

\noindent
Д\,о\,к\,а\,з\,а\,т\,е\,л\,ь\,с\,т\,в\,о\,.\ \ Обозначим
\begin{multline*}
\widehat{U}_J(\sigma)=\sum\limits_{j=0}^{J-1}\sum\limits_{k=0}^{2^j-1}
\beta^2_{j,k}\Bigg[
\left(Y_{j,k}-\widehat{\mu}_{j,k}(\sigma)\right)^2+{}\\
{}+2\sigma^2\fr{\partial}{\partial Y_{j,k}}\widehat{\mu}_{j,k}(\sigma)\Bigg] %\notag
\end{multline*}
и запишем $\widehat{R}_J(\hat{\sigma}_S)-R_J(\sigma)$ в~виде
\begin{multline*}
%\label{Three_Sums}
\widehat{R}_J(\hat{\sigma}_S)-R_J(\sigma)={}\\
{}=\left[\widehat{U}_J(\hat{\sigma}_S)-\widehat{U}_J(\sigma)\right]+
\left[\widehat{R}_J(\sigma)-R_J(\sigma)\right]+{}\\
{}+
\fr{2^{(2\alpha+1)J}-1}{2^{2\alpha+1}-1}(\sigma^2-\hat{\sigma}^2_S)
\equiv S_1+S_2+S_3\,.
\end{multline*}

Повторяя рассуждения из работ~\cite{KS11-1, KS11-2} и~учитывая, что если $\gamma\hm>1/4$, 
то выполнено $2^{J/2}\overline{Y}^2\stackrel{{\sf P}}{\to} 0$ при 
$J\hm\rightarrow\infty$~\cite{KS11-2}, можно показать, что
\begin{equation*}
{\sf P}\left(\fr{S_2+S_3}{D_J}<x\right)\Rightarrow\Phi_{\Upsilon_1}(x)\,.%\notag
\end{equation*}
% на самом деле с~условием Линдеберга чуть по-другому (без ограниченности слагаемых). Но дисперсия равномерно ограничена -- значит выполнено.

Докажем, что $D_J^{-1}S_1\stackrel{{\sf P}}{\to}0$ при $J\hm\rightarrow\infty$. 
Пусть $C_\delta\hm>0$~--- некоторая константа, а $\delta_J\hm=C_\delta J^{1/2}2^{-J/2}$. 
Запишем
\begin{multline*}
S_1=\mathbf{1}\left(\abs{\sigma^2-\hat{\sigma}^2_S}>\delta_J\right)S_1+{}\\
{}+
\mathbf{1}\left(\abs{\sigma^2-\hat{\sigma}^2_S}\leqslant\delta_J\right)
S_1\equiv S'_1+S''_1. %\notag
\end{multline*}
Для произвольного $\varepsilon\hm>0$
\begin{equation*}
{\sf P}\left(S'_1>\varepsilon\right)\leqslant{\sf P}
\left(\abs{\sigma^2-\hat{\sigma}^2_S}>\delta_J\right). %\notag
\end{equation*}
При выполнении условий теоремы, если константа~$C_\delta$ достаточно велика, 
то найдется константа~$\tilde{C}_\delta>0$ такая, что~\cite{KS11-2}
\begin{equation*}
{\sf P}\left(\abs{\sigma^2-\hat{\sigma}^2_S}>\delta_J\right)
\leqslant\tilde{C}_\delta2^{-J/2}. %\notag
\end{equation*}
%% комментарии по поводу этого неравенства и~загрязнения выборки есть в~диссертации
Следовательно, $S'_1\stackrel{P}{\to}0$ при $J\hm\rightarrow\infty$.

Обозначим слагаемые в~сумме~$S''_1$ через~$F_{j,k}(\hat{\sigma}_S)$. Пусть 
$A_j\hm=\sqrt{A\ln 2^j}$, где $0\hm<A\hm<2(\sigma^2\hm-\delta_J)$. Имеем:

\noindent
\begin{multline*}
\hspace*{-9.9pt}\sum\limits_{j=0}^{J-1}\sum\limits_{k=0}^{2^j-1}F_{j,k}\left(\hat{\sigma}_S\right)=
\sum\limits_{j=0}^{J-1}\sum\limits_{k=0}^{2^j-1}
\mathbf{1}(\abs{Y_{j,k}}\leqslant A_j)F_{j,k}(\hat{\sigma}_S)+{}\\
{}+
\sum\limits_{j=0}^{J-1}\sum\limits_{k=0}^{2^j-1}
\mathbf{1}\left(\abs{Y_{j,k}}>A_j\right)F_{j,k}(\hat{\sigma}_S)
\equiv  W_1+W_2. %\notag
\end{multline*}
Рассмотрим $W_1$. Учитывая определения $\widehat{\mu}_{j,k}(\sigma)$, 
$({\partial}/{\partial Y_{j,k}})\widehat{\mu}_{j,k}(\sigma)$ и~$A_j$, 
можно убедиться, что найдут\-ся константы $C_1\hm>0$ и~$\theta\hm>0$ такие, что
\begin{equation*}
\abs{\mathbf{1}\left(\abs{Y_{j,k}}\leqslant A_J\right)
F_{j,k}(\hat{\sigma}_S)}\leqslant C_1 
J^{5/2}2^{(2\alpha-\theta)j-J/2}\;\;\mbox{п.в.} %\notag
\end{equation*}
% поскольку выполнено \mathbf{1}(\abs{\sigma^2-\hat{\sigma}^2_S}\leqslant\delta_J). В логарифме степень: от Y идет 1, от T идет 1, от \delta_J идет 1/2 но для J, а не для j, поэтому берем для всех J^{5/2}. В степени 2: 2\alpha от \beta{j,k}, \theta из-за выбора A, J/2 от \delta_J
Следовательно, $D_J^{-1}W_1\hm\rightarrow 0$ п.в.\ при $J\hm\rightarrow\infty$.

Далее для слагаемых~$W_2$ имеем:
\begin{multline*}
\left\vert \mathbf{1}\left(
\left\vert Y_{j,k}\right\vert
> A_J\right)F_{j,k}
\left(\hat{\sigma}_S\right)\right\vert
\leqslant{}\\
{}\leqslant C_2 J^{3/2}2^{2\alpha j-J/2} 
\mathbf{1}\left( \left\vert Y_{j,k}\right\vert > A_J\right) 
\left\vert Y_{j,k}\right\vert^2\;\;\mbox{п.в.},
%\notag
\end{multline*}
% поскольку выполнено \mathbf{1}(\abs{\sigma^2-\hat{\sigma}^2_S}\leqslant\delta_J). В логарифме от T идет 1, от \delta_J идет 1/2.
где $C_2>0$~--- некоторая константа. Учитывая распределение~$Y_{j,k}$, 
нетрудно убедиться, что
\begin{equation*}
\Expect\frac{1}{D_J} \sum\limits_{j=0}^{J-1}
\sum\limits_{k=0}^{2^j-1} J^{3/2}2^{2\alpha j-J/2} 
\mathbf{1}\left(\abs{Y_{j,k}}> A_j\right)
\abs{Y_{j,k}}^2\to 0
\end{equation*}
при $J\rightarrow\infty$. %\notag
Следовательно, используя неравенство Маркова, получаем, что
\begin{equation*}
D_J^{-1}W_2\stackrel{{\sf P}}{\to}0\;\;\mbox{при}\;J\rightarrow\infty\,. %\notag
\end{equation*}
Таким образом, $D_J^{-1}S_1\stackrel{{\sf P}}{\to}0$ при $J\hm\rightarrow\infty$.

Теорема доказана.

\smallskip

Рассмотрим теперь ситуацию, когда в~качестве оценки~$\sigma$ используется 
величина~$\widehat{\sigma}_{R}$ или~$\widehat{\sigma}_{M}$. 
В~этом случае повышаются требования к~гладкости функции сигнала.

\smallskip

\noindent
\textbf{Теорема~3.}\
\textit{Пусть~$Kf$ задана на конечном отрезке и~равномерно регулярна по 
Липшицу с~показателем $\gamma\hm>1/2$, а оценка дисперсии шума~$\hat{\sigma}$ 
задана соотношением}~\eqref{IQR_Definition} 
\textit{или соотношением}~\eqref{MAD_Definition}. \textit{Тогда}
\begin{equation*}
\label{CLT_Operator_RobVar_Sigma}
\mathsf{P}\left(\fr{\widehat{R}_J(\widehat{\sigma})-R_J(\sigma)}{D_J}<x\right)
\Rightarrow \Phi_{\Upsilon_2}(x)\,, %\notag
\end{equation*}
где $\Phi_{\Upsilon_2}(x)$~--- функция распределения нормального закона 
с~нулевым средним и~дисперсией
\begin{multline*}
\Upsilon_2^2=1+\fr{2^{4\alpha+1}-1}{4(2^{2\alpha+1}-1)^2
\xi_{3/4}^2(\phi(\xi_{3/4}))^2}-{}\\
{}-
\fr{2^{4\alpha+1}-1 }{2^{2\alpha-1}(2^{2\alpha+1}-1)}\,.
\end{multline*}

\noindent
Д\,о\,к\,а\,з\,а\,т\,е\,л\,ь\,с\,т\,в\,о\,.\ \
Как и~в~предыдущей теореме, запишем
$\widehat{R}_J(\hat{\sigma})\hm-R_J(\sigma)\hm=S_1\hm+S_2\hm+S_3.$
Учитывая,\linebreak\vspace*{-12pt}

\pagebreak

\noindent
 что $\gamma\hm>1/2$, и~поступая, как в~работах~\cite{SH18, KS11-2, SH12}, 
с~использованием разложения Бахадура для выборочных квантилей~\cite{S80} и~выборочного 
абсолютного медианного отклонения~\cite{SM09}, можно показать, что
\begin{equation*}
{\sf P}\left(\fr{S_2+S_3}{D_J}<x\right)\Rightarrow\Phi_{\Upsilon_2}(x)\,. %\notag
\end{equation*}
% на самом деле с~условием Линдеберга чуть по-другому (без ограниченности слагаемых). Но дисперсия равномерно ограничена -- значит выполнено.

Используя экспоненциальные неравенства для выборочных квантилей~\cite{S80} 
и~выборочного абсолютного медианного отклонения~\cite{SM09}, получаем, что при 
выполнении условий теоремы найдется такая константа $C_\delta\hm>0$, что при 
$\delta_J\hm=C_\delta J^{1/2}2^{-J/2}$ для некоторой константы~$\widetilde{C}_\delta>0$ 
выполнено:
\begin{align*}
\mathsf{P}\left(\abs{\widehat{\sigma}_{R}-\sigma}>\delta_J\right)
&\leqslant\widetilde{C}_\delta2^{-J/2}\,;
\\
\mathsf{P}\left(\abs{\widehat{\sigma}_{M}-\sigma}>\delta_J\right)
&\leqslant\widetilde{C}_\delta2^{-J/2}\,. %\notag
\end{align*}
%% комментарии по поводу этого неравенства и~загрязнения выборки есть в~диссертации
Далее, повторяя рассуждения предыдущей теоремы, заключаем, что 
$D_J^{-1}S_1\stackrel{{\sf P}}{\to}0$ при $J\hm\rightarrow\infty$.


Теорема доказана.



{\small\frenchspacing
 {%\baselineskip=10.8pt
 \addcontentsline{toc}{section}{References}
 \begin{thebibliography}{99}

\bibitem{HL10}
\Au{Huang H.-C., Lee~T.\,C.\,M.} 
Stabilized thresholding with generalized sure for image denoising~// 
IEEE 17th  Conference (International) on Image Processing
Proceedings.~--- IEEE, 2010. P.~1881--1884.

\bibitem{SH18}
\Au{Shestakov O.\,V.} 
Nonlinear regularization of inverse problems for linear homogeneous transforms 
by the stabilized hard thresholding~// J.~Math. Sci., 2018. Vol.~234. No.\,6. P.~780--785.

\bibitem{KS11-1}
\Au{Кудрявцев А.\,А., Шестаков~О.\,В.} 
Асимптотика оценки риска при вейг\-лет-вейв\-лет разложении наблюдаемого сигнала~// 
T-Comm~--- телекоммуникации и~транспорт, 2011. №\,2. С.~54--57.

\bibitem{KS11-2}
\Au{Кудрявцев А.\,А., Шестаков~О.\,В.} 
Асимптотическое распределение оценки риска пороговой обработки 
вейг\-лет-ко\-эф\-фи\-ци\-ен\-тов сигнала при неизвестном уровне шума~// 
T-Comm~--- телекоммуникации и~транспорт, 2011. №\,5. С.~24--30.

\bibitem{AS98}
\Au{Abramovich F., Silverman~B.\,W.} 
Wavelet decomposition approaches to statistical inverse problems~// 
Biometrika, 1998. Vol.~85. No.\,1. P. 115--129.

\bibitem{Mal99}
\Au{Mallat S.} A~Wavelet tour of signal processing.~--- 
New York, NY, USA: Academic Press, 1999. 857~p.

\bibitem{L97}
\Au{Lee N.} Wavelet-vaguelette decompositions and homogenous equations.~--- 
West Lafayette, IN, USA: Purdue University, 1997.  PhD Thesis. 103~p.

\bibitem{B96}
\Au{Breiman L.} Heuristics of instability and stabilization in model selection~// 
Ann. Stat., 1996. Vol.~24. No.\,6. P.~2350--2383.

\bibitem{J01}
\Au{Jansen M.} Noise reduction by wavelet thresholding.~--- 
Lecture notes in statistics ser.~--- New York, NY, USA: Springer Verlag,
2001. Vol.~161. 196~p.

\bibitem{SH12}
\Au{Шестаков О.\,В.} О~скорости сходимости оценки риска пороговой обработки 
вейв\-лет-ко\-эф\-фи\-ци\-ен\-тов к~нормальному закону при использовании 
робастных оценок дисперсии~// Информатика и~её применения, 2012. Т.~6. Вып.~2. 
С.~122--128.

\bibitem{S80}
\Au{Serfling R.} Approximation theorems of mathematical statistics.~--- 
New York, NY, USA: John Wiley \& Sons, 1980. 371~p.

\bibitem{SM09}
\Au{Serfling R., Mazumder~S.} 
Exponential probability inequality and convergence results for the median 
absolute deviation and its modifications~// Stat. Probabil. Lett., 2009. 
Vol.~79. No.\,16. P.~1767--1773.
 \end{thebibliography}

 }
 }

\end{multicols}

\vspace*{-3pt}

\hfill{\small\textit{Поступила в~редакцию 14.12.18}}

\vspace*{8pt}

%\pagebreak

%\newpage

%\vspace*{-28pt}

\hrule

\vspace*{2pt}

\hrule

%\vspace*{-2pt}

\def\tit{INVERSION OF~HOMOGENEOUS OPERATORS USING~STABILIZED HARD THRESHOLDING 
WITH~UNKNOWN NOISE VARIANCE}

\def\titkol{Inversion of~homogeneous operators using~stabilized hard thresholding 
with~unknown noise variance}

\def\aut{O.\,V.~Shestakov}

\def\autkol{O.\,V.~Shestakov}

\titel{\tit}{\aut}{\autkol}{\titkol}

\vspace*{-11pt}


\noindent
Department of Mathematical Statistics, Faculty of Computational Mathematics and Cybernetics, M.V. Lomonosov Moscow State University, 1-52 Leninskiye Gory, GSP-1, Moscow 119991, Russian Federation
Institute of Informatics Problems, Federal Research Center 
``Computer Science and Control'' of the Russian Academy of Sciences, 44-2~Vavilov Str., 
Moscow 119333, Russian Federation

\def\leftfootline{\small{\textbf{\thepage}
\hfill INFORMATIKA I EE PRIMENENIYA~--- INFORMATICS AND
APPLICATIONS\ \ \ 2019\ \ \ volume~13\ \ \ issue\ 1}
}%
 \def\rightfootline{\small{INFORMATIKA I EE PRIMENENIYA~---
INFORMATICS AND APPLICATIONS\ \ \ 2019\ \ \ volume~13\ \ \ issue\ 1
\hfill \textbf{\thepage}}}

\vspace*{6pt}



\Abste{When inverting linear homogeneous operators, it is necessary to use 
regularization methods, since observed data are usually noisy. For noise suppression, 
threshold processing of  wavelet coefficients of the observed signal function 
is often used. Threshold processing has become a~popular noise suppression tool 
due to its simplicity, computational efficiency, and ability to adapt to functions 
that have different degrees of regularity at different domains. The paper 
discusses the recently proposed stabilized hard thresholding method that eliminates 
the main
drawbacks of soft and hard thresholding methods and studies statistical 
properties of this method. In the data model\linebreak\vspace*{-12pt}}

\Abstend{with an additive Gaussian noise with 
unknown variance, an unbiased estimate of the mean square risk is analyzed and it 
is shown that under certain conditions, this estimate is asymptotically normal and 
the variance of the limit distribution depends on the type of estimate of noise variance.}


\KWE{wavelets; threshold processing; unbiased risk estimate; asymptotic normality;
strong consistency}




\DOI{10.14357/19922264190107}

%\vspace*{-14pt}

\Ack
\noindent
This research was partly supported by the Russian  
Foundation for Basic Research (project No.\,19-07-00352).




%\vspace*{6pt}

  \begin{multicols}{2}

\renewcommand{\bibname}{\protect\rmfamily References}
%\renewcommand{\bibname}{\large\protect\rm References}

{\small\frenchspacing
 {%\baselineskip=10.8pt
 \addcontentsline{toc}{section}{References}
 \begin{thebibliography}{99}
\bibitem{1-sh-1}
\Aue{Huang, H.-C., and T.\,C.\,M.~Lee.} 2010. 
Stabilized thresholding with generalized sure for image denoising. 
\textit{IEEE 17th Conference (International) on Image Processing}. IEEE. 1881--1884.

 

\bibitem{2-sh-1}
\Aue{Shestakov, O.\,V.} 2018. 
Nonlinear regularization of inverse problems for linear homogeneous transforms 
by the stabilized hard thresholding. 
\textit{J.~Math. Sci.} 234(6):780--785.

\bibitem{3-sh-1}
\Aue{Kudryavtsev, A.\,A., and O.\,V.~Shestakov.} 2011. Аsimptotika otsenki riska pri 
veyglet-veyvlet razlozhenii nablyuda\-emo\-go signala [The average risk assessment 
of the wavelet decomposition of the signal].
\textit{T-Comm~--- Telecommunications and Their Application in
Transport Industry} 2:54--57.

\bibitem{4-sh-1}
\Aue{Kudryavtsev, A.\,A., and O.\,V.~Shestakov.} 2011. Аsimptoticheskoe raspredelenie 
otsenki riska porogovoy ob\-ra\-bot\-ki veyglet-koeffitsientov signala pri 
neizvestnom urovne shuma [Asymptotic distribution of the risk estimate of 
the signal vaguelette coefficients thresholding at the unknown noise level]. 
\textit{T-Comm~--- Telecommunications and Their Application in
Transport Industry} 5:24--30.

\bibitem{5-sh-1}
\Aue{Abramovich, F., and B.\,W.~Silverman.} 1998. Wavelet 
decomposition approaches to statistical inverse problems. 
\textit{Biometrika} 85(1):115--129.

\bibitem{6-sh-1}
\Aue{Mallat, S.} 1999. \textit{A~wavelet tour of signal processing.} New York, NY: 
Academic Press. 857 p.

\bibitem{7-sh-1}
\Aue{Lee, N.} 1997. Wavelet-vaguelette decompositions and homogenous equations. 
 West Lafayette, IN: Purdue University. PhD Thesis. 103~p.

\bibitem{8-sh-1}
\Aue{Breiman, L.} 1996. 
Heuristics of instability and stabilization in model selection. 
\textit{Ann. Stat.} 24(6):2350--2383.

\bibitem{9-sh-1}
\Aue{Jansen, M.} 2001. \textit{Noise reduction by wavelet thresholding.} 
Lecture notes in statistics ser.
New York, NY: Springer Verlag.  Vol.~161. 196~p.

\bibitem{10-sh-1}
\Aue{Shestakov, O.\,V.} 2012. O~skorosti skhodimosti otsenki riska porogovoy 
obrabotki veyvlet-koeffitsientov k~nor\-mal'\-no\-mu zakonu pri ispol'zovanii robastnykh 
otsenok dispersii [On the rate of convergence to the normal law of risk estimate for 
wavelet coefficients thresholding when using robust variance estimates]. 
\textit{Informatika i~ee Primeneniya~--- Inform. Appl.}  6(2):122--128.

\bibitem{11-sh-1}
\Aue{Serfling, R.} 1980. \textit{Approximation theorems of mathematical statistics}.
New York, NY: John Wiley \& Sons. 371~p.

\bibitem{12-sh-1}
\Aue{Serfling, R., and S.~Mazumder.} 2009. Exponential probability inequality 
and convergence results for the median absolute deviation and its modifications. 
\textit{Stat. Probabil. Lett.} 79(16):1767--1773.
\end{thebibliography}

 }
 }

\end{multicols}

\vspace*{-6pt}

\hfill{\small\textit{Received December 14, 2018}}

%\pagebreak

%\vspace*{-18pt}  

\Contrl

\noindent
\textbf{Shestakov Oleg V.} (b.\ 1976)~--- 
Doctor of Science in physics and mathematics, professor, Department of 
Mathematical Statistics, Faculty of Computational Mathematics and Cybernetics, 
M.\,V.~Lomonosov Moscow State University, 1-52~Leninskiye Gory, GSP-1, Moscow 119991, 
Russian Federation; senior scientist, Institute of Informatics Problems, 
Federal Research Center ``Computer Science and Control'' 
of the Russian Academy of Sciences, 44-2~Vavilov Str., Moscow 119333, 
Russian Federation; \mbox{oshestakov@cs.msu.su}
\label{end\stat}

\renewcommand{\bibname}{\protect\rm Литература} 
     %13


%   { %\Large  
   { %\baselineskip=16.6pt
   
   \vspace*{-48pt}
   \begin{center}\LARGE
   \textit{Предисловие}
   \end{center}
   
   %\vspace*{2.5mm}
   
   \vspace*{25mm}
   
   \thispagestyle{empty}
   
   { %\small 

    
Вниманию читателей журнала <<Информатика и её применения>> предлагается 
очередной тематический выпуск <<Вероятностно-статистические методы и 
задачи информатики и информационных технологий>>. Предыдущие тематические 
выпуски журнала по данному направлению вышли в 2008~г.\ (т.~2, вып.~2), 
в 2009~г.\ (т.~3, вып.~3) и в 2010~г.\ (т.~4, вып.~2). 

Статьи, собранные в данном журнале, посвящены разработке новых вероятностно-статистических 
методов, ориентированных на применение к решению конкретных задач информатики и информационных 
технологий, а также~--- в ряде случаев~--- и других прикладных задач. Проблематика, охватываемая 
публикуемыми работами, развивается в рамках научного сотрудничества между Институтом проблем 
информатики Российской академии наук (ИПИ РАН) и Факультетом вычислительной математики и 
кибернетики Московского государственного университета им.\ М.\,В.~Ломоносова в ходе работ 
над совместными научными проектами (в том числе в рамках функционирования 
Научно-образовательного центра <<Вероятностно-статистические методы анализа рисков>>). 
Многие из авторов статей, включенных в данный номер журнала, являются активными участниками 
традиционного международного семинара по проблемам устойчивости стохастических моделей, 
руководимого В.\,М.~Золотаревым и В.\,Ю.~Королевым; регулярные сессии этого семинара 
проводятся под эгидой МГУ и ИПИ РАН (в 2011~г.\ указанный семинар проводится в октябре 
в Калининградской области РФ). 

Наряду с представителями ИПИ РАН и МГУ в число авторов данного выпуска журнала входят 
ученые из Научно-исследовательского института системных исследований РАН, Института 
проблем технологии микроэлектроники и особочистых материалов РАН, Института 
прикладных математических исследований Карельского НЦ РАН, Московского 
авиационного института, Вологодского государственного педагогического университета, 
НИИММ им.\ Н.\,Г.~Чеботарева, Казанского государственного университета, Дебреценского 
университета (Венгрия).

Несколько статей выпуска посвящено разработке и применению стохастических методов и 
информационных технологий для решения различных прикладных задач. В~работе В.\,Г.~Ушакова 
и О.\,В.~Шестакова рассмотрена задача определения вероятностных характеристик случайных 
функций по распределениям интегральных преобразований, возникающих в задачах эмиссионной 
томографии. В~статье Д.\,О.~Яковенко и М.\,А.~Целищева рассмотрены некоторые вопросы 
математической теории риска и предложен новый подход к диверсификации инвестиционных 
портфелей. Работа И.\,А.~Кудрявцевой и А.\,В.~Пантелеева посвящена построению и 
исследованию математической модели, описывающей динамику сильноионизованной плазмы. 
В~статье П.\,П.~Кольцова изучается качество работы ряда алгоритмов сегментации изображений. 
Статья А.\,Н.~Чупрунова и И.~Фазекаша посвящена вероятностному анализу числа без\-оши\-бочных 
блоков при помехоустойчивом кодировании; получены усиленные законы больших чисел для указанных 
величин.

В данном выпуске традиционно присутствует тематика, весьма активно разрабатываемая в течение 
многих лет специалистами ИПИ РАН и МГУ,~--- методы моделирования и управления для 
информационно-телекоммуникационных и вычислительных систем, в частности методы 
теории массового обслуживания. В~статье А.\,И.~Зейфмана с соавторами рассматриваются 
модели обслуживания, описываемые марковскими цепями с непрерывным временем в случае 
наличия катастроф. В~работе М.\,М.~Лери и И.\,А.~Чеплюковой рассматриваются случайные 
графы Интернет-типа, т.\,е.\ графы, степени вершин которых имеют степенные распределения; 
такие задачи находят применение при исследовании глобальных сетей передачи данных. 
Работа Р.\,В.~Разумчика посвящена исследованию систем массового обслуживания специального 
вида~--- с отрицательными заявками и хранением вытесненных заявок.

Ряд статей посвящен развитию перспективных теоретических 
вероятностно-статистических методов, которые находят широкое применение в различных 
задачах информатики и информационных технологий. В~работе В.\,Е.~Бенинга, А.\,К.~Горшенина 
и В.\,Ю.~Королева рассмотрена задача статистической проверки гипотез о числе компонент 
смеси вероятностных распределений, приводится конструкция асимптотически наиболее мощного 
критерия. Результаты этой работы найдут применение в ряде прикладных задач, использующих 
математическую модель смеси вероятностных распределений (в информатике, моделировании 
финансовых рынков, физике турбулентной плазмы и~т.\,д.). В~статье В.\,Ю.~Королева, 
И.\,Г.~Шевцовой и С.\,Я.~Шоргина строится новая, улучшенная оценка точности нормальной 
аппроксимации для пуассоновских случайных сумм; как известно, указанные случайные суммы 
широко используются в качестве моделей многих реальных объектов, в том числе в информатике, 
физике и других прикладных областях. Работа В.\,Г.~Ушакова и Н.\,Г.~Ушакова посвящена 
исследованию ядерной оценки плотности распределения; эти результаты могут применяться, 
в част\-ности, при анализе трафика в телекоммуникационных системах. Серьезные приложения 
в статистике могут получить результаты работы О.\,В.~Шестакова, в которой доказаны оценки 
скорости сходимости распределения выборочного абсолютного медианного отклонения к нормальному 
закону. 

\smallskip

Редакционная коллегия журнала выражает надежду, что данный тематический  выпуск 
будет интересен специалистам в области теории вероятностей и математической статистики 
и их применения к решению задач информатики и информационных технологий.
     
     %\vfill 
     \vspace*{20mm}
     \noindent
     Заместитель главного редактора журнала <<Информатика и её 
применения>>,\\
     директор ИПИ РАН, академик  \hfill
     \textit{И.\,А.~Соколов}\\
     
     \noindent
     Редактор-составитель тематического выпуска,\\
     профессор кафедры математической статистики факультета\\
      вычислительной математики и кибернетики МГУ им.\ М.\,В.~Ломоносова,\\
     ведущий научный сотрудник ИПИ РАН,\\ 
доктор физико-математических наук \hfill
      \textit{В.\,Ю.~Королев}
     
     } }
     }

%%%%%%%%%%%%%%%%%%%%%%%%%%%%%%%%%%%%%%%%%%%%%%%


                       


\def\stat{rez}
{%\hrule\par
%\vskip 7pt % 7pt
\raggedleft\Large \bf%\baselineskip=3.2ex
Р\,Е\,Ц\,Е\,Н\,З\,И\,И \vskip 17pt
    \hrule
    \par
\vskip 6pt plus 6pt minus 3pt }

%\thispagestyle{headings} %с верхним колонтитулом
%\thispagestyle{myheadings} %с нижним колонтитулом, но в верхнем РЕЦЕНЗИИ

\def\tit{НОВАЯ КНИГА И.\,Н.~СИНИЦЫНА, А.\,С.~ШАЛАМОВА <<ЛЕКЦИИ ПО ТЕОРИИ 
ИНТЕГРИРОВАННОЙ ЛОГИСТИЧЕСКОЙ ПОДДЕРЖКИ>> (М.: ТОРУС ПРЕСС, 2012. 624~с.)}

%1
\def\aut{Д.ф.-м.н., профессор С.\,Я.~Шоргин}

\def\auf{\ }

\def\leftkol{\ % РЕЦЕНЗИИ
}

\def\rightkol{ \ } 

%\def\leftkol{\ } % ENGLISH ABSTRACTS}

%\def\rightkol{\ } %ENGLISH ABSTRACTS}

%\def\leftkol{РЕЦЕНЗИИ}

%\def\rightkol{РЕЦЕНЗИИ}

\titele{\tit}{\aut}{\auf}{\leftkol}{\rightkol}
\vspace*{-18pt}


     \label{st\stat}

     \begin{multicols}{2}
     {\small
     {\baselineskip=10.1pt
     

      В книге представлено системное изложение теоретических основ одного из новейших 
направлений в \mbox{об\-ласти} экономики послепродажного обслуживания изделий наукоемкой 
продукции (ИНП) длительного пользования~--- интегрированной логистической поддержки
(ИЛП). 
{\looseness=1

}

Приведены также результаты новых работ, выполненных в Институте проблем информатики 
Российской академии наук в рамках научного направления <<Информационные технологии и 
анализ сложных сис\-тем>>.
 {%\looseness=1

}
     
      Излагаемые в книге научные подходы позво\-ляют карди\-наль\-но реформировать 
существующие системы производства и эксплуатации ИНП путем создания и внед\-ре\-ния 
методов рационального и оптимального управ\-ле\-ния процессами расходования 
вре\-мен\-н$\acute{\mbox{ы}}$х, 
мате\-ри\-аль\-ных, трудовых и других ресурсов на всех стадиях жизненного цикла изделий (ЖЦИ) по 
критериям экономической целесообразности и эф\-фек\-тив\-ности.
  {\looseness=1

}
    
      В книге приведен краткий обзор причин возник\-новения и
      развития CALS-методологии как основы 
современных международных стандартов по созданию и функционированию глобальных 
ин\-фор\-ма\-ци\-он\-но-ком\-му\-ни\-ка\-ци\-он\-ных систем, ее ключевых возможностей и эффективности 
результатов ее использования. 
Авторы %\linebreak 
предлагают ряд научных обоснований для разработки 
единой теории проектирования и управления систем ИЛП для полноценного использования 
преимуществ %\linebreak
 суще\-ст\-ву\-ющей методологии, определяют \mbox{общую} структурную схему 
комплексной системы <<ИНП-СППО>> и необходимость разработки для ее описания 
гибридных стохастических моделей.
{%\looseness=1

}

%\columnbreak
      
      Книга состоит из пяти частей, где последовательно излагается материал по каждой из 
следующих тем: <<Интегрированная логистическая поддержка>>, <<Теория гибридных 
стохастических систем и компьютерная поддержка исследований и разработок>>, <<Основы 
математического моделирования, анализа и синтеза систем послепродажного обслуживания>>, 
<<Определение и анализ показателей экспортного потенциала ИНП при проектировании>>, 
<<Задачи управления поддержкой послепродажного обслуживания>>, а также 
<<Моделирование инвестиционных процессов ИЛП в условиях неравновесных финансовых 
рынков>>. 
   
      В конце каждой главы приведены выводы и даны вопросы и задания для 
самоконтроля. В~приложениях содержатся основные определения по программам работ по 
анализу ИЛП, логистическим базам данных и компьютерным решениям, эквивалентной статистической 
линеаризации нелинейных преобразований ИЛП, справочный материал, а также развернутые 
уравнения для вероятностных характеристик.


      \def\leftkol{РЕЦЕНЗИИ}

\def\rightkol{РЕЦЕНЗИИ} 

      
      Книга заинтересует широкий круг специалистов и может быть использована научными 
проектными организациями в сфере промышленного производства ИНП. Большое количество 
иллюстраций, примеров и вопросов, обращенных к читателю, позволяет использовать книгу 
также в качестве учебного пособия для студентов и аспирантов машиностроительных, 
транспортных и~других специальностей, а также для самостоятельного изучения. 
{%\looseness=-1

}

Книга 
представляет несомненный интерес для специалистов и студентов в области прикладной 
математики и информатики.
    

}

}
\end{multicols}

%\newpage

%\end{document}

\include{obchak}

%\end{document}



\def\stat{authorsrus}
{%\hrule\par
%\vskip 7pt % 7pt
\raggedleft\Large \bf%\baselineskip=3.2ex
О\,Б\ \ А\,В\,Т\,О\,Р\,А\,Х \vskip 17pt
    \hrule
    \par
\vskip 21pt plus 8pt minus 4pt }


\def\tit{\ }

\def\aut{\ }

\def\auf{\ }

\def\leftkol{\ } % ENGLISH ABSTRACTS}

\def\rightkol{ОБ АВТОРАХ} %ENGLISH ABSTRACTS}

\titele{\tit}{\aut}{\auf}{\leftkol}{\rightkol}
      
            \label{st\stat}



\vspace*{24pt}

\begin{multicols}{2}




\noindent
\textbf{Архипов Олег Петрович} (р.\ 1948)~---
кандидат технических наук, директор Орловского филиала Института проб\-лем информатики
Российской академии наук
%302025, г.Орел, Московское шоссе, д.137

\vspace*{3pt}

\noindent
\textbf{Бирюкова Татьяна Константиновна} (р.\ 1968)~---
кандидат фи\-зи\-ко-ма\-те\-ма\-ти\-че\-ских наук, старший научный сотрудник Института проб\-лем информатики
Российской академии наук

\vspace*{3pt}

\noindent 
\textbf{Бобков  Сергей Геннадьевич} (р.\ 1955)~---
доктор технических наук,  заведующий отделением На\-уч\-но-ис\-сле\-до\-ва\-тель\-ско\-го 
института системных исследований Российской академии наук
%117218, Москва, Нахимовский просп., 36, к.1 

\vspace*{3pt}

\noindent \textbf{Васильев Николай Семенович} (р.\ 1952)~--- доктор 
фи\-зи\-ко-ма\-те\-ма\-ти\-че\-ских наук, профессор, 
МГТУ им.\ Н.\,Э.~Баумана 
%, Москва 105005, 2-я Бауманская ул., д.~5,

\vspace*{3pt}

\noindent
\textbf{Гершкович Максим Михайлович} (р.\ 1968)~---
старший научный сотрудник Института проб\-лем информатики
Российской академии наук

\vspace*{3pt}

\noindent 
\textbf{Дьяченко Юрий Георгиевич} (р.\ 1958)~--- кандидат технических наук, 
старший научный сотрудник Института проб\-лем информатики
Российской академии наук

\vspace*{3pt}

\noindent 
\textbf{Ерошенко Александр Андреевич} (р.\ 1989)~--- аспирант кафедры 
математической статистики факультета вычисли\-тельной математики и кибернетики 
Московского государственного университета им.\ М.\,В.~Ломоносова
%119991, Москва ГСП-1, Ленинские горы, д.\ 1, стр. 52

\vspace*{3pt}
 
\noindent 
\textbf{Захаров Виктор Николаевич} (р.\ 1948)~--- 
доктор технических наук, доцент, ученый секретарь Института проб\-лем информатики
Российской академии наук

\vspace*{3pt}

\noindent
\textbf{Зейфман Александр Израилевич} (р.\ 1954)~---
доктор фи\-зи\-ко-ма\-те\-ма\-ти\-че\-ских наук, профессор, 
заведующий кафедрой Вологодского государственного университета; 
старший научный сотрудник Института проб\-лем информатики
Российской академии наук; главный научный сотрудник ИСЭРТ Российской академии наук

\vspace*{3pt}

\noindent
\textbf{Зыкин Сергей Владимирович} (р.\ 1959)~--- 
доктор технических наук, профессор, заведующий лабораторией Института математики 
им.\ С.\,Л.~Соболева Сибирского отделения Российской академии наук, Новосибирск 
%630090, пр.\ ак.\ Коптюга, 4 

\vspace*{4pt}

\noindent
\textbf{Киреев Владимир Иванович} (р.\ 1938)~---
доктор фи\-зи\-ко-ма\-те\-ма\-ти\-че\-ских наук, профессор Московского 
государственного горного университета
%Адрес: Россия, 119991, г. Москва, Ленинский проспект, д. 6

%\columnbreak

\vspace*{4pt}

\noindent
\textbf{Козеренко Елена Борисовна} (р.\ 1959)~---
кандидат филологических наук, заведующая лабораторией Института проб\-лем информатики
Российской академии наук

\vspace*{4pt}

\noindent
\textbf{Королев Виктор Юрьевич} (р.\ 1954)~--- доктор
фи\-зи\-ко-ма\-те\-ма\-ти\-че\-ских наук, профессор кафедры математической 
статистики факультета вычисли\-тельной математики и кибернетики 
Московского государственного университета; 
ведущий научный сотрудник Института проб\-лем информатики
Российской академии наук

\vspace*{4pt}

\noindent
\textbf{Коротышева Анна Владимировна} (р.\ 1988)~---
старший преподаватель Вологодского государственного университета

\vspace*{4pt}

\noindent 
\textbf{Кун Де Турк} (р.\ 1981)~--- научный сотрудник 
исследовательской группы SMACS факультета телекоммуникаций и обработки информации
Университета Гента, Бельгия
%В-9000 Гент, Бельгия

\vspace*{4pt}

\noindent
\textbf{Лупенцов Олег Сергеевич} (р.\ 1986)~---
аспирант Омского государственного института сервиса
%Омск 644043, ул.\ Певцова 13

\vspace*{4pt}

\noindent
\textbf{Лучко Олег Николаевич} (р.\ 1961)~---
кандидат педагогических наук, профессор, заведующий кафедрой 
Омского государственного института сервиса
%Омск 644043, ул.\ Певцова 13

\vspace*{4pt}

\noindent
\textbf{Малашенко Юрий Евгеньевич} (р.\ 1946)~---
доктор фи\-зи\-ко-ма\-те\-ма\-ти\-че\-ских наук, заведующий сектором 
Вычислительного центра им.\ А.\,А.~Дородницына Российской академии наук
%Адрес: 119333, Москва, ул. Вавилова, 40,

\vspace*{4pt}

\noindent
\textbf{Маньяков Юрий Анатольевич} (р.\ 1984)~---
кандидат технических наук, научный сотрудник Орловского филиала Института проб\-лем информатики
Российской академии наук
%302025, г.Орел, Московское шоссе, д.137

\vspace*{4pt}

\noindent
\textbf{Маренко Валентина Афанасьевна} (р.\ 1951)~---
кандидат технических наук, доцент, старший научный сотрудник 
Института математики им.\ С.\,Л.~Соболева Сибирского отделения Российской академии наук
%Новосибирск 630090, пр. ак. Коптюга, 4 

\vspace*{3pt}

\noindent 
\textbf{Морозов Евсей Викторович} (р.\ 1947)~--- доктор 
фи\-зи\-ко-ма\-те\-ма\-ти\-че\-ских, профессор, ведущий научный сотрудник 
Института прикладных математических исследований Карельского научного центра Российской
академии наук; 
%%185910 Россия, Республика Карелия, г.\ Петрозаводск, ул.\ Пушкинская, 11
профессор Петрозаводского государственного университета, Петрозаводск
%185910 Россия, Республика Карелия, г.\ Петрозаводск, пр.\ Ленина, 33

%\pagebreak

\vspace*{3pt}

\noindent
\textbf{Назарова Ирина Александровна} (р.\ 1966)~---
кандидат фи\-зи\-ко-ма\-те\-ма\-ти\-че\-ских наук, 
научный сотрудник Вычислительного центра им.\ А.\,А.~Дородницына Российской академии наук 
%Адрес: 119333, Москва, ул. Вавилова, 40

\vspace*{3pt}

\noindent
\textbf{Павлов Игорь Валерианович} (р.\ 1945)~--- 
доктор фи\-зи\-ко-ма\-те\-ма\-ти\-че\-ских наук, профессор МГТУ им.\ Н.\,Э.~Баумана 
%Москва 105005, 2-я Бауманская ул., д.~5 

%\pagebreak

\vspace*{3pt}

\noindent 
\textbf{Потахина Любовь Викторовна} (р.\ 1989)~--- аспирантка
Института прикладных математических исследований Карельского научного центра
Российской академии наук; 
%%185910 Россия, Республика Карелия, г.\ Петрозаводск, ул.\ Пушкинская, 11
инженер Петрозаводского государственного университета, Петрозаводск
%185910 Россия, Республика Карелия, г.\ Петрозаводск, пр.\ Ленина, 33

\vspace*{3pt}

\noindent 
\textbf{Рождественский Юрий Владимирович} (р.\ 1952)~--- 
кандидат технических наук, заведующий сектором Института проб\-лем информатики
Российской академии наук

\vspace*{3pt}

\noindent 
\textbf{Синицын Игорь Николаевич} (р.\ 1940)~--- доктор технических наук,
профессор, заслуженный деятель\linebreak\vspace*{-12pt}

\columnbreak

\noindent
 науки РФ, заведующий отделом Института проб\-лем информатики
Российской академии наук

\vspace*{7pt}


\noindent
\textbf{Сиротинин Денис Олегович} (р.\ 1984)~---
кандидат технических наук, научный сотрудник Орловского филиала Института проб\-лем информатики
Российской академии наук
%302025, г.Орел, Московское шоссе, д.137

\vspace*{7pt}

%\columnbreak

\noindent 
\textbf{Соколов  Игорь Анатольевич} (р.\ 1954)~--- академик (действительный член) Российской 
академии наук, доктор технических наук, директор Института проб\-лем информатики
Российской академии наук

\vspace*{7pt}

\noindent
\textbf{Степченков Юрий Афанасьевич} (р.\ 1951)~---
кандидат технических наук, заведующий отделом Института проб\-лем информатики
Российской академии наук

\vspace*{7pt}

\noindent
\textbf{Сурков Алексей Викторович} (р.\ 1978)~--- 
старший научный сотрудник На\-уч\-но-ис\-сле\-до\-ва\-тель\-ско\-го 
института системных исследований Российской академии наук
%117218, Москва, Нахимовский просп., 36, к.1 

\vspace*{7pt}

\noindent 
\textbf{Шестаков Олег Владимирович} (р.\ 1976)~--- доктор 
фи\-зи\-ко-ма\-те\-ма\-ти\-че\-ских, доцент кафедры математической статистики 
факультета вычисли\-тельной математики и кибернетики Московского 
государственного университета им.\ М.\,В.~Ломоносова; 
%119991, Москва ГСП-1, Ленинские горы, д.\ 1, стр. 52
старший научный сотрудник Института проб\-лем информатики
Российской академии наук
%, Москва 119333, ул. Вавилова, д.~44, корп.~2

\vspace*{7pt}

\noindent 
\textbf{Шоргин Сергей Яковлевич} (р.\ 1952.)~--- доктор
фи\-зи\-ко-ма\-те\-ма\-ти\-че\-ских наук, профессор, заместитель директора Института 
проб\-лем информатики Российской академии наук





%%%%%%%%%%%%%%%%%%%%%%%%%%%%%%%%%%%%%%%%%%%%%%%%%%%%%%%%%%%%%%%%%%%%%%%%%%%%%%%




%\def\rightkol{ОБ АВТОРАХ}
%\def\leftkol{ОБ АВТОРАХ}

 \label{end\stat}





%\def\leftfootline{\small{\textbf{\thepage}
%\hfill ИНФОРМАТИКА И ЕЁ ПРИМЕНЕНИЯ\ \ \ том~7\ \ \ выпуск~1\ \ \ 2013}
%}%
% \def\rightfootline{\small{ИНФОРМАТИКА И ЕЁ ПРИМЕНЕНИЯ\ \ \ том~7\ \ \ выпуск~1\ \ \ 2013
%\hfill \textbf{\thepage}}}


%\thispagestyle{myheadings}



\end{multicols}

\newpage


%\vspace*{-48pt}
\begin{center}\LARGE
\textit{About Authors}
\end{center}

\thispagestyle{empty}
\def\tit{\ }

\def\aut{\ }

\def\auf{\ }


\def\leftkol{ABOUT AUTHORS}

\def\rightkol{ABOUT AUTHORS}

\vspace*{-18pt}

\titele{\tit}{\aut}{\auf}{\leftkol}{\rightkol}

%\vspace*{36pt}

\def\rightmark{{\noindent\hbox to \textwidth{\hfill\small ABOUT AUTHORS
%\hfill \large\bf\thepage
}}}
\def\leftmark{{\noindent\parbox{\textwidth}{
%\raggedleft\large\bf\thepage \hfill
\small\textrm{ABOUT AUTHORS}\hfill}}}


\def\leftfootline{\small{\textbf{\thepage}
\hfill ИНФОРМАТИКА И ЕЁ ПРИМЕНЕНИЯ\ \ \ том~6\ \ \ выпуск~2\ \ \ 2012}
}%
 \def\rightfootline{\small{ИНФОРМАТИКА И ЕЁ ПРИМЕНЕНИЯ\ \ \ том~6\ \ \ выпуск~2\ \ \ 2012
\hfill \textbf{\thepage}}}


\begin{multicols}{2}

\noindent
\textbf{Agalarov Yaver M.} (b.\ 1952)~--- Candidate of Science (PhD)
in technology, 
leading scientist, Institute of Informatics Problems, Russian Academy of Sciences

\vspace*{5pt}


  \noindent
\textbf{Bosov Alexey V.} (b.\ 1969)~--- Doctor of Science in technology, Head of
Laboratory, Institute of Informatics Problems, Russian Academy of Sciences

\vspace*{5pt}


\noindent
\textbf{Dulin Sergey K.} (b.\ 1950)~--- Doctor of Science in technology, 
professor, senior scientist, Institute of Informatics Problems, Russian Academy of Sciences

\vspace*{5pt}

\noindent
\textbf{Gorshenin Andrey K.}~--- (b.\ 1986)~--- Candidate of Science (PhD)
in physics and mathematics,
senior scientist, Institute of Informatics Problems, Russian Academy of Sciences

\vspace*{5pt}

\noindent
\textbf{Kalenov Nikolay E.}  (b.\ 1945)~--- Doctor of Science in technology,
professor, Director, Library for Natural Sciences,  Russian Academy of Sciences 

\vspace*{5pt}

\noindent
\textbf{Kalinichenko Leonid A.} (b.\ 1937)~--- Doctor of Science in physics and mathematics, 
professor, Honored scientist of RF, 
Head of Laboratory, Institute of Informatics Problems, Russian Academy of Sciences 

\vspace*{5pt}

\noindent
\textbf{Karpov Alexey A.} (b.\ 1978)~--- Candidate of Science (PhD) in technology, 
senior scientist, St.\ Petersburg Institute for
Informatics and Automation,  Russian Academy of Sciences

\vspace*{5pt}

\noindent
\textbf{Kuznetsov Igor P.} (b.\ 1938)~--- Doctor of Science in technology, 
professor, principal scientist, Institute of Informatics Problems, Russian Academy of Sciences

\vspace*{5pt}


\noindent
\textbf{Markova Natalia A.} (b.\ 1950)~--- Candidate of Science (PhD) in
physics and mathematics, leading scientist,  
Institute of Informatics Problems, Russian Academy of Sciences

\vspace*{5pt}

\noindent
\textbf{Nikolaev Andrey V.} (b.\ 1985)~--- Candidate of Science (PhD) in technology, 
senior lecturer, Tchaikovsky Technological Institute, Branch of the Izhevsk State Technical 
University

\vspace*{6pt}

\noindent
\textbf{Pavlov Igor V.} (b.\ 1945)~---  Doctor of Science in physics and mathematics,
professor, Bauman Moscow State Technical University

\vspace*{6pt}

%\columnbreak

\noindent
\textbf{Rozenberg Igor N.} (b.\ 1965)~--- Doctor of Science in technology, 
First Deputy Director General, Research \& Design Institute for Information 
Technology, Signalling and Telecommunications on Railway Transport (JSC NIIAS)

\vspace*{6pt}


\noindent
\textbf{Semenov Konstantin K.} (b.\ 1986)~--- MPhil, 
associate professor, St.\ Petersburg State Polytechnical University

\vspace*{6pt}

\noindent
\textbf{Sharnin Mikhail M.} (b.\ 1959)~--- Candidate of Science (PhD) 
in technology, senior scientist, Institute of Informatics Problems, Russian Academy of Sciences

\vspace*{6pt}

\noindent 
\textbf{Shestakov Oleg V.} (b.\ 1976)~--- Candidate of Science (PhD) in physics and mathematics,
associate professor, Department of Mathematical Statistics, Faculty of Computational Mathematics and Cybernetics,
M.\,V.~Lomonosov Moscow State University; senior scientist, Institute of Informatics Problems, 
Russian Academy of Sciences

\vspace*{6pt}

\noindent
\textbf{Stupnikov Sergey A.} (b.\ 1978)~--- Candidate of Science (PhD) in technology, 
senior scientist, Institute of Informatics Problems, Russian Academy of Sciences 

\vspace*{6pt}

\noindent
\textbf{Umansky Vladimir I.} (b.\ 1954)~--- Candidate of Science (PhD) in technology, 
Director General, ``IntechGeoTrans'' Closed Joint Stock Company

\vspace*{6pt}

\noindent
\textbf{Zhevnerchuk Dmitry V.} (b.\ 1978)~--- Candidate of Science (PhD) in technology, 
associate professor, Tchaikovsky Technological Institute, Branch of the Izhevsk State 
Technical University

%\vspace*{6pt}

\def\leftfootline{\small{\textbf{\thepage}
\hfill ИНФОРМАТИКА И ЕЁ ПРИМЕНЕНИЯ\ \ \ том~6\ \ \ выпуск~2\ \ \ 2012}
}%
 \def\rightfootline{\small{ИНФОРМАТИКА И ЕЁ ПРИМЕНЕНИЯ\ \ \ том~6\ \ \ выпуск~2\ \ \ 2012
\hfill \textbf{\thepage}}}



%\thispagestyle{myheadings}

\end{multicols}
\newpage

%   \vspace*{-48pt}

\begin{center}
\vspace*{6pt}
\mbox{%
\epsfxsize=53.502mm
\epsfbox{foto-1.eps}
}
\end{center}

\vspace*{6pt} %Академик


   \begin{center}
\fbox{\Large\textbf{Профессор Игорь Алексеевич Ушаков}}\\[12pt]
\textbf{\large 22.01.1935--27.02.2015}
   \end{center}


   %\vspace*{2.5mm}

   \vspace*{5mm}

   \thispagestyle{empty}

%\

%\vspace*{-12pt}


Редакционный совет и редакционная коллегия журнала <<Информатика и~её применения>> с~глубоким прискорбием извещают, что 27~февраля 2015~г.\ после тяжелой
и~продолжительной болезни скончался Игорь Алексеевич Ушаков~--- доктор технических наук, профессор, член редколлегии журнала <<Информатика и ее применения>>.

Игорь Алексеевич Ушаков окончил Московский авиационный институт, в~1963~г.\ защитил кандидатскую, а~в~1968~г.~--- докторскую диссертацию. С~1958 по 1989~гг.\ работал в~ряде научно-исследовательских организаций СССР, в~том числе руководил отделами в~НИИ АА и~ВЦ АН СССР; с 1969 по 1989 гг. преподавал в~МФТИ (был профессором, а~затем заведующим кафедрой) и~в~МЭИ. С~1989~г.~---- в~США: являлся профессором университета Дж.\ Вашингтона, университета Дж.\ Мэйсона и~Калифорнийского университета, сотрудником компаний MCI, Qualcomm и Hughes.

И.\,А.~Ушаков с момента основания журнала <<Надежность и~контроль качества>> был заместителем ответственного редактора, а~затем на протяжении многих лет членом редколлегии. В~2006~г.\ основал электронный международный журнал ``Reliability: Theory \& Application'', главным редактором которого оставался до конца жизни.

Учебниками и справочниками по теории надежности, написанными И.\,А.~Ушаковым, пользовались и~пользуются несколько поколений ученых и~специалистов в~разных странах мира.

Игорь Алексеевич всегда уделял огромное внимание работе с~молодежью; более~50 его учеников защитили докторские и~кандидатские диссертации.

И.\,А.~Ушаков вел активную научно-про\-све\-ти\-тель\-скую деятельность. В~частности, он был одним из организаторов и~руководителей Московского кабинета качества и~надежности при Политехническом музее (целью этого Кабинета было оказание консультаций работникам промышленных предприятий и~чтение курсов лекций для инженеров, занимающихся проблемой надежности). Находясь в~США, И.\,А.~Ушаков создал международный ин\-тер\-нет-фо\-рум им.\ Б.\,В.~Гнеденко, объединивший около~400~видных специалистов по приложениям теории вероятностей и~математической статистики, преимущественно в~об\-ласти теории надежности и~анализа риска, из десятков стран мира; коллективным членов этого Форума является и~наш журнал. Цели Форума~--- содействие контактам между специалистами из разных стран, организация обмена профессиональными 
новостями и~информацией (новые публикации, предстоящие события и~др.). Также необходимо отметить большое число на\-уч\-но-по\-пу\-ляр\-ных работ, опубликованных И.\,А.~Ушаковым.

И.\,А.~Ушаков обладал большим личным обаянием, имел широкий круг интересов. Все знавшие И.\,А.~Ушакова всегда будут помнить его как замечательного ученого и~прекрасного человека.

\bigskip

Редакционный совет и редакционная коллегия журнала <<Информатика и~её применения>> 
выражают глубокие соболезнования родным и близким покойного, всем, кто его знал и~работал с~ним.


\vspace*{-60pt} {\small
{\baselineskip=9.1pt
\section*{Правила подготовки рукописей статей для публикации в журнале
<<Информатика и её применения>>}

\thispagestyle{empty}

 Журнал <<Информатика и её применения>> публикует
теоретические, обзорные и дискуссионные статьи, посвященные научным
исследованиям и разработкам в области информатики и ее приложений. Журнал
издается на русском языке. По специальному решению редколлегии отдельные статьи,
в виде исключения, могут печататься на английском языке.
Тематика журнала охватывает следующие направления:
\begin{itemize}
\item теоретические основы информатики; %\\[-13.5pt]
\item математические методы исследования сложных систем и процессов; %\\[-13.5pt]
\item информационные системы и сети; %\\[-13.5pt]
\item информационные технологии; %\\[-13.5pt]
\item архитектура и программное
обеспечение вычислительных комплексов и сетей.
\end{itemize}
\begin{enumerate}
\item В журнале печатаются результаты, ранее не
опубликованные и не предназначенные к одновременной публикации в других
изданиях. Публикация не должна нарушать закон об авторских правах. Направляя
свою рукопись в редакцию, авторы автоматически передают учредителям и
редколлегии неисключительные права на издание данной статьи на русском языке и
на ее распространение в России и за рубежом. При этом за авторами сохраняются
все права как собственников данной рукописи. В связи с этим авторами должно
быть представлено в редакцию письмо в следующей форме:
Соглашение о передаче права на публикацию:

\textit{<<Мы, нижеподписавшиеся, авторы рукописи <<$\qquad\qquad$>>, передаем
учредителям и редколлегии журнала <<Информатика и её применения>>
неисключительное право опубликовать данную рукопись статьи на русском языке как
в печатной, так и в электронной версиях журнала. Мы подтверждаем, что данная
публикация не нарушает авторского права других лиц или организаций. Подписи
авторов: (ф.\,и.\,о., дата, адрес)>>.}

Указанное соглашение может быть представлено 
как в бумажном виде, так и в виде отсканированной копии (с подписями авторов).


Редколлегия вправе запросить у авторов экспертное заключение о возможности
опубликования представленной статьи в открытой печати. %\\[-13.5pt]
\item Статья
подписывается всеми авторами. На отдельном листе представляются данные автора
(или всех авторов): фамилия, полные имя и отчество, телефон, факс, e-mail,
почтовый адрес. Если работа выполнена несколькими авторами, указывается фамилия
одного из них, ответственного за переписку с редакцией. %\\[-13.5pt]
\item Редакция журнала
осуществляет самостоятельную экспертизу присланных статей. Возвращение рукописи
на доработку не означает, что статья уже принята к печати. Доработанный вариант
с ответом на замечания рецензента необходимо прислать в редакцию. %\\[-13.5pt]
\item Решение
редакционной коллегии о принятии статьи к печати или ее отклонении сообщается
авторам. Редколлегия не обязуется направлять рецензию авторам отклоненной
статьи. %\\[-13.5pt]
\item Корректура статей высылается авторам для просмотра. Редакция
просит авторов присылать свои замечания в кратчайшие сроки. %\\[-13.5pt]
\item При
подготовке рукописи в MS Word рекомендуется использовать следующие настройки.
Параметры страницы: формат~--- А4; ориентация~--- книжная; поля (см): внутри~---
2,5, снаружи~--- 1,5, сверху~--- 2, снизу~--- 2, от края до нижнего
колонтитула~--- 1,3. Основной текст: стиль~--- <<Обычный>>: шрифт Times New
Roman, размер 14~пунктов, абзацный отступ~--- 0,5~см, 1,5 интервала,
выравнивание~--- по ширине. Рекомендуемый объем рукописи~--- не свыше
25~страниц указанного формата. Ознакомиться с шаблонами, содержащими примеры
оформления, можно по адресу в Интернете:
\textsf{http://www.ipiran.ru/journal/template.doc}.
\item К рукописи, предоставляемой в 2-х
экземплярах, обязательно прилагается электронная версия статьи (как правило, в
форматах MS WORD (.doc) или \LaTeX\ (.tex), а также~--- дополнительно~--- в
формате .pdf) на дискете, лазерном диске или по электронной почте. Сокращения
слов, кроме стандартных, не применяются. Все страницы рукописи должны быть
пронумерованы. %\\[-13.5pt]
\item Статья должна содержать следующую информацию на русском и
английском языках: название, Ф.И.О. авторов, места работы авторов и их
электронные адреса, подробные сведения об авторах, оформленные в соответствии с форматом, 
определяемым файлами {\sf http://www.ipiran.ru/journal/issues/2011\_05\_01/authors.asp} и 
{\sf http://www.ipiran.ru/journal/issues/2011\_01\_eng/authors.asp},
аннотация (не более 100~слов), ключевые слова. Ссылки на
литературу в тексте статьи нумеруются (в квадратных скобках) и располагаются в
порядке их первого упоминания. В~списке литературы не должно быть позиций, на которые нет ссылки в тексте статьи.
Все фамилии авторов, заглавия статей, названия
книг, конференций и~т.\,п.\ даются на языке оригинала, если этот язык
использует кириллический или латинский алфавит. %\\[-13.5pt]
\item Присланные в редакцию материалы авторам не возвращаются.
\item При отправке файлов по электронной
почте просим придерживаться следующих правил:
\begin{itemize}
\item указывать в поле subject (тема) название журнала и фамилию автора; %\\[-13.5pt]
\item использовать attach (присоединение); %\\[-13.5pt]
\item в случае больших объемов информации возможно
использование общеизвестных архиваторов (ZIP, RAR); %\\[-13.5pt]
\item в состав электронной версии статьи должны входить: файл, содержащий текст статьи, и файл(ы),
содержащий(е) иллюстрации. %\\[-13.5pt]
\end{itemize}
\item Журнал <<Информатика и её применения>> является некоммерческим изданием. 
Плата за публикацию с авторов не взимается, гонорар авторам не выплачивается.
\end{enumerate}
\thispagestyle{empty}
\textbf{Адрес редакции:} Москва 119333,
ул.~Вавилова, д.~44, корп.~2, ИПИ РАН\\
\hphantom{\textbf{Адрес редакции:} }Тел.: +7 (499) 135-86-92\ \
Факс:  +7 (495) 930-45-05\ \  E-mail:   rust@ipiran.ru }
}


\end{document}

%\include{IPPM-25}

\def\stat{cont}
{%\hrule\par
%\vskip 7pt % 7pt
\raggedleft\Large \bf%\baselineskip=3.2ex
А\,В\,Т\,О\,Р\,С\,К\,И\,Й\ \ У\,К\,А\,З\,А\,Т\,Е\,Л\,Ь\ \ З\,А\ \ 2\,0\,1\,0 г. \vskip 17pt
    \hrule
    \par
\vskip 21pt plus 6pt minus 3pt }

\label{st\stat}

\def\tit{\ }

\def\aut{\ }
\def\auf{\ }

\def\leftkol{\ } % ENGLISH ABSTRACTS}

\def\rightkol{\ } %АВТОРСКИЙ УКАЗАТЕЛЬ ЗА 2010 г.} %ENGLISH ABSTRACTS}

\titele{\tit}{\aut}{\auf}{\leftkol}{\rightkol}

\vspace*{-12pt}

{\tabcolsep=3pt
\begin{tabular}{p{388pt}rr}
&\textbf{Выпуск} & \textbf{Стр.}\\[6pt]
\hangindent=23pt\noindent\textbf{Арутюнян~А.\,Р.} Моделирование влияния деформаций отпечатков пальцев на 
точность\linebreak
\vspace*{-12pt}\\
\hspace*{23pt}дактилоскопической идентификации$\dotfill$&1&51\\
\hangindent=23pt\noindent\textbf{Архипов~О.\,П., Зыкова~З.\,П.} Интеграция гетерогенной информации о цветных 
пикселях\linebreak
\vspace*{-12pt}\\
\hspace*{23pt}и их цветовосприятии$\dotfill$&4&15\\
\hangindent=23pt\noindent\textbf{Баранов~С.\,И., Френкель~С.\,Л., Захаров~В.\,Н.} Полуформальная верификация 
цифрового устройства с конвейером, основанная на использовании алгоритмических машин\linebreak
\vspace*{-12pt}\\
\hspace*{23pt}состояния$\dotfill$&4&49\\
\textbf{Бекетова~И.\,В.} см.~Каратеев~С.\,Л.&&\\
\textbf{Белоусов~В.\,В.} см.~Синицын~И.\,Н.&&\\
\hangindent=23pt\noindent\textbf{Бенинг~В.\,Е., Королев~Р.\,А.} О предельном поведении мощностей критериев в 
случае\linebreak
\vspace*{-12pt}\\
\hspace*{23pt}распределения Лапласа$\dotfill$&2&63\\
\hangindent=23pt\noindent\textbf{Бенинг~В.\,Е., Сипина~А.\,В.} Асимптотическое разложение для мощности 
критерия,\linebreak
\vspace*{-12pt}\\
\hspace*{23pt}основанного на выборочной медиане, в случае распределения Лапласа$\dotfill$&1&18\\
\textbf{Бондаренко~А.\,В.} см.~Каратеев~С.\,Л.&&\\
\hangindent=23pt\noindent\textbf{Бородина~А.\,В., Морозов~Е.\,В.} Об оценивании асимптотики вероятности 
большого\linebreak
\vspace*{-12pt}\\
\hspace*{23pt}уклонения стационарной регенеративной очереди с одним прибором$\dotfill$&3&29\\
\hangindent=23pt\noindent\textbf{Бунтман~Н.\,В., Минель~Ж.-Л., Ле~Пезан~Д., Зацман~И.\,М.} Типология и 
компьютерное\linebreak
\vspace*{-12pt}\\
\hspace*{23pt}моделирование трудностей перевода$\dotfill$&3&77\\
\textbf{Визильтер~Ю.\,В.} см.~Каратеев~С.\,Л.&&\\
\hangindent=23pt\noindent\textbf{Гавриленко~С.\,В.} Оценки скорости сходимости распределений случайных сумм с 
безгранично делимыми индексами к нормальному закону$\dotfill$&4&81\\
\hangindent=23pt\noindent\textbf{Григорьева~М.\,Е., Шевцова~И.\,Г.} Уточнение неравенства 
Каца--Берри--Эссеена$\dotfill$&2&75\\
\hangindent=23pt\noindent\textbf{Грушо~А.\,А., Грушо~Н.\,А., Тимонина~Е.\,Е.} Поиск конфликтов в политиках 
безопасности: модель случайных графов$\dotfill$&3&38\\
\textbf{Грушо~Н.\,А.} см.~Грушо~А.\,А.&&\\
\hangindent=23pt\noindent\textbf{Гудков~В.\,Ю.} Математические модели изображения отпечатка пальца на основе 
описания линий$\dotfill$&1&58\\
\textbf{Гуртов~А.\,В.} см.~Лукьяненко~А.\,С.&&\\
\textbf{Желтов~С.\,Ю.} см.~Каратеев~С.\,Л.&&\\
\hangindent=23pt\noindent\textbf{Захаров~А.\,А., Серебряков~В.\,А.} Система управления электронной библиотекой 
LibMeta$\dotfill$&4&2\\
\textbf{Захаров~В.\,Н.} см.~Баранов~С.\,И.&&\\
\textbf{Захарова~Т.\,В.} см.~Матвеева~С.\,С.&&\\
\hangindent=23pt\noindent\textbf{Зацаринный~А.\,А., Чупраков~К.\,Г.} Некоторые аспекты выбора технологии для 
постро-\linebreak
\vspace*{-12pt}\\
\hspace*{23pt}ения систем отображения информации ситуационного центра$\dotfill$&3&59\\
\textbf{Зацман~И.\,М.} см.~Бунтман~Н.\,В.&&\\
\hangindent=23pt\noindent\textbf{Зейфман~А.\,И., Коротышева~А.\,В., Сатин~Я.\,А., Шоргин~С.\,Я.} Об 
устойчивости нестаци-\linebreak
\vspace*{-12pt}\\
\hspace*{23pt}онарных систем обслуживания с катастрофами$\dotfill$&3&9\\
\textbf{Зыкова~З.\,П.} см.~Архипов~О.\,П.&&\\
\hangindent=23pt\noindent\textbf{Илюшин~Г.\,Я., Соколов~И.\,А.} Организация управляемого доступа пользователей 
к\linebreak
\vspace*{-12pt}\\
\hspace*{23pt}разнородным ведомственным информационным ресурсам$\dotfill$&1&24\\
\hangindent=23pt\noindent\textbf{Кавагучи~Ю., Ульянов~В.\,В., Фуджикоши~Я.} Приближения для статистик, 
описывающих\linebreak
\vspace*{-12pt}\\
\hspace*{23pt}геометрические свойства данных большой размерности, с оценками 
ошибок$\dotfill$&1&12\\
\hangindent=23pt\noindent\textbf{Каратеев~С.\,Л., Бекетова~И.\,В., Ососков~М.\,В., Князь~В.\,А., 
Визильтер~Ю.\,В., Бондаренко~А.\,В., Желтов~С.\,Ю.} Автоматизированный контроль 
качества цифровых\linebreak
\vspace*{-12pt}\\
\hspace*{23pt}изображений для персональных документов$\dotfill$&1&65\\
\end{tabular}
}

\pagebreak

\def\leftkol{АВТОРСКИЙ УКАЗАТЕЛЬ ЗА 2010 г.} % ENGLISH ABSTRACTS}

\def\rightkol{АВТОРСКИЙ УКАЗАТЕЛЬ ЗА 2010 г.} %ENGLISH ABSTRACTS}

{\tabcolsep=3pt
\begin{tabular}{p{388pt}rr}
&\textbf{Выпуск} & \textbf{Стр.}\\[3pt]
\hangindent=23pt\noindent\textbf{Козеренко~Е.\,Б.} Лингвистические фильтры в статистических моделях машинного\linebreak
\vspace*{-12pt}\\
\hspace*{23pt}перевода$\dotfill$&2&83\\
\hangindent=23pt\noindent\textbf{Козеренко~Е.\,Б., Кузнецов~И.\,П.} Когнитивно-лингвистические представления в 
систе-\linebreak
\vspace*{-12pt}\\
\hspace*{23pt}мах обработки текстов$\dotfill$&3&69\\
\textbf{Князь~В.\,А.} см.~Каратеев~С.\,Л.&&\\
\hangindent=23pt\noindent\textbf{Колесников~А.\,В., Солдатов~С.\,А.} Алгоритм координации для гибридной 
интеллектуальной системы решения сложной задачи оперативно-производственного\linebreak
\vspace*{-12pt}\\
\hspace*{23pt}планирования$\dotfill$&4&61\\
\hangindent=23pt\noindent\textbf{Коновалов~М.\,Г.} О планировании потоков в системах вычислительных 
ресурсов$\dotfill$&2&3\\
\textbf{Конушин~А.\,С.} см.~Конушин~В.\,С.&&\\
\hangindent=23pt\noindent\textbf{Конушин~В.\,С., Кривовязь~Г.\,Р., Конушин~А.\,С.} Алгоритм распознавания людей 
в видео-\linebreak
\vspace*{-12pt}\\
\hspace*{23pt}последовательности по одежде$\dotfill$&1&74\\
\textbf{Корепанов~Э.\, Р.} см.~Синицын~И.\,Н.&&\\
\textbf{Королев~В.\,Ю.} см.~Соколов~И.\,А.&&\\
\textbf{Королев~Р.\,А.} см.~Бенинг~В.\,Е.&&\\
\textbf{Коротышева~А.\,В.} см.~Зейфман~А.\,И.&&\\
\hangindent=23pt\noindent\textbf{Кривенко~М.\,П.} Непараметрическое оценивание элементов байесовского 
клас\-си-\linebreak
\vspace*{-12pt}\\
\hspace*{23pt}фикатора$\dotfill$&2&13\\
\textbf{Кривовязь~Г.\,Р.} см.~Конушин~В.\,С.&&\\
\textbf{Крылов~А.\,С.} см.~Павельева~Е.\,А.&&\\
\hangindent=23pt\noindent\textbf{Крылов~В.\,А.} Моделирование и классификация многоканальных дистанционных\linebreak
\vspace*{-12pt}\\
\hspace*{23pt}изображений с использованием копул$\dotfill$&4&34\\
\hangindent=23pt\noindent\textbf{Крючин~О.\,В.} Разработка параллельных эвристических алгоритмов подбора 
весовых\linebreak
\vspace*{-12pt}\\
\hspace*{23pt}коэффициентов искусственной нейтронной сети$\dotfill$&2&53\\
\hangindent=23pt\noindent\textbf{Кудрявцев~А.\,А., Шоргин~С.\,Я.} Байесовские модели массового обслуживания и 
надеж-\linebreak
\vspace*{-12pt}\\
\hspace*{23pt}ности: характеристики среднего числа заявок в системе $M\vert M \vert 1\vert 
\infty$$\dotfill$&3&16\\
\hangindent=23pt\noindent\textbf{Кузнецов~А.\,А.} Связь между временными и структурно-топологическими 
характери-\linebreak
\vspace*{-12pt}\\
\hspace*{23pt}стиками диаграмм ритма сердца здоровых людей$\dotfill$&4&39\\
\textbf{Кузнецов~И.\,П.} см.~Козеренко~Е.\,Б.&&\\
\textbf{Ле~Пезан~Д.} см.~Бунтман~Н.\,В.&&\\
\hangindent=23pt\noindent\textbf{Лукьяненко~А.\,С., Морозов~Е.\,В., Гуртов~А.\,В.} Анализ сетевого протокола с общей 
функ-\linebreak
\vspace*{-12pt}\\
\hspace*{23pt}цией расширения окна передачи сообщения при конфликтах$\dotfill$&2&46\\
\hangindent=23pt\noindent\textbf{Лямин~О.\,О.} О предельном поведении мощностей критериев в случае обобщенного\linebreak
\vspace*{-12pt}\\
\hspace*{23pt}распределения Лапласа$\dotfill$&3&47\\
\hangindent=23pt\noindent\textbf{Маркин~А.\,В., Шестаков~О.\,В.} Асимптотики оценки риска при пороговой 
обработке\linebreak
\vspace*{-12pt}\\
\hspace*{23pt}вейвлет-вейглет коэффициентов в задаче томографии$\dotfill$&2&36\\
\hangindent=23pt\noindent\textbf{Матвеева~С.\,С., Захарова~Т.\,В.} Сети массового обслуживания с наименьшей 
длиной\linebreak
\vspace*{-12pt}\\
\hspace*{23pt}очереди$\dotfill$&3&22\\
\hangindent=23pt\noindent\textbf{Матюшенко~С.\,И.} Стационарные характеристики двухканальной системы 
обслужива-\linebreak
\vspace*{-12pt}\\
\hspace*{23pt}ния с переупорядочиванием заявок и распределениями фазового типа$\dotfill$&4&68\\
\textbf{Минель~Ж.-Л.} см.~Бунтман~Н.\,В.&&\\
\textbf{Морозов~Е.\,В.} см.~Бородина~А.\,В.&&\\
\textbf{Морозов~Е.\,В.} см.~Лукьяненко~А.\,С.&&\\
\textbf{Ососков~М.\,В.} см.~Каратеев~С.\,Л.&&\\
\hangindent=23pt\noindent\textbf{Павельева~Е.\,А., Крылов~А.\,С.} Поиск и анализ ключевых точек радужной 
оболочки\linebreak
\vspace*{-12pt}\\
\hspace*{23pt}глаза методом преобразования Эрмита$\dotfill$&1&79\\
\textbf{Печинкин~А.\,В.} см.~Френкель~С.\,Л.,&&\\
\hangindent=23pt\noindent\textbf{Протасов~В.\,И.} Составление субъективного портрета с использованием 
эволюционно-\linebreak
\vspace*{-12pt}\\
\hspace*{23pt}го морфинга и квалиметрия метода$\dotfill$&1&83\\
\hangindent=23pt\noindent\textbf{Рудаков~К.\,В., Торшин~И.\,Ю.} Вопросы разрешимости задачи распознавания 
вторичной\linebreak
\vspace*{-12pt}\\
\hspace*{23pt}структуры белка$\dotfill$&2&25\\
\textbf{Сатин~Я.\,А.} см.~Зейфман~А.\,И.&&\\
\hangindent=23pt\noindent\textbf{Сейфуль-Мулюков~Р.\,Б.} Нефть как носитель информации о своем 
происхождении,\linebreak
\vspace*{-12pt}\\
\hspace*{23pt}структуре и эволюции$\dotfill$&1&41\\
\end{tabular}
}

{\tabcolsep=3pt
\begin{tabular}{p{388pt}rr}
&\textbf{Выпуск} & \textbf{Стр.}\\[6pt]
\textbf{Семендяев~Н.\,Н.} см.~Синицын~И.\,Н.&&\\
\textbf{Серебряков~В.\,А.} см.~Захаров~А.\,А.&&\\
\textbf{Синицын~В.\,И.} см.~Синицын~И.\,Н.&&\\
\hangindent=23pt\noindent\textbf{Синицын~И.\,Н., Синицын~В.\,И., Корепанов~Э.\, Р., Белоусов~В.\,В., 
Семендяев~Н.\,Н.} Оперативное построение информационных моделей движения полюса 
Земли\linebreak
\vspace*{-12pt}\\
\hspace*{23pt}методами линейных и линеаризованных фильтров$\dotfill$&1&2\\
\textbf{Сипина~А.\,В.} см.~Бенинг~В.\,Е.&&\\
\hangindent=23pt\noindent\textbf{Соколов~И.\,А.} О работах заслуженного деятеля науки Российской Федерации 
И.\,Н.~Синицына в области информационных технологий и автоматизации (к 70-летию\linebreak
\vspace*{-12pt}\\
\hspace*{23pt}со дня рождения)$\dotfill$&3&84\\
\textbf{Соколов~И.\,А.} см.~Илюшин~Г.\,Я.&&\\
\hangindent=23pt\noindent\textbf{Соколов~И.\,А., Королев~В.\,Ю.} Предисловие$\dotfill$&2&2\\
\textbf{Солдатов~С.\,А.} см.~Колесников~А.\,В.&&\\
\hangindent=23pt\noindent\textbf{Степанов~С.\,Ю.} Использование координатного метода фрагментации 
коммутаторной\linebreak
\vspace*{-12pt}\\
\hspace*{23pt}нейронной сети для сокращения трафика$\dotfill$&2&57\\
\textbf{Тимонина~Е.\,Е.} см.~Грушо~А.\,А.&&\\
\textbf{Торшин~И.\,Ю.} см.~Рудаков~К.\,В.&&\\
\textbf{Ульянов~В.\,В.} см.~Кавагучи~Ю.&&\\
\textbf{Фазекаш~И.} см.~Чупрунов~А.\,Н.&&\\
\textbf{Френкель~С.\,Л.} см.~Баранов~С.\,И.&&\\
\hangindent=23pt\noindent\textbf{Френкель~С.\,Л., Печинкин~А.\,В.} Оценка времени самовосстановления в 
цифровых\linebreak
\vspace*{-12pt}\\
\hspace*{23pt}системах после сбоев, вызываемых переходными помехами$\dotfill$&3&2\\
\textbf{Фуджикоши~Я.} см.~Кавагучи~Ю.&&\\
\hangindent=23pt\noindent\textbf{Цискаридзе~А.\,К.} Математическая модель и метод восстановления позы человека 
по\linebreak
\vspace*{-12pt}\\
\hspace*{23pt}стереопаре силуэтных изображений$\dotfill$&4&27\\
\hangindent=23pt\noindent\textbf{Чупраков~К.\,Г.} К вопросу о размещении коллективных средств отображения в 
ситуа-\linebreak
\vspace*{-12pt}\\
\hspace*{23pt}ционном зале с заданными параметрами$\dotfill$&4&89\\
\textbf{Чупраков~К.\,Г.} см.~Зацаринный~А.\,А.&&\\
\hangindent=23pt\noindent\textbf{Чупрунов~А.\,Н., Фазекаш~И.} Законы повторного логарифма для числа 
безошибочных\linebreak
\vspace*{-12pt}\\
\hspace*{23pt}блоков при помехоустойчивом кодировании$\dotfill$&3&42\\
\textbf{Шевцова~И.\,Г.} см.~Григорьева~М.\,Е.&&\\
\hangindent=23pt\noindent\textbf{Шестаков~О.\,В.} Аппроксимация распределения оценки риска пороговой 
обработки вейвлет-коэффициентов нормальным распределением при использовании 
выбо-\linebreak
\vspace*{-12pt}\\
\hspace*{23pt}рочной дисперсии$\dotfill$&4&73\\
\textbf{Шестаков~О.\,В.} см.~Маркин~А.\,В.&&\\
\textbf{Шоргин~С.\,Я.} см.~Зейфман~А.\,И.&&\\
\textbf{Шоргин~С.\,Я.} см.~Кудрявцев~А.\,А.&&\\
\end{tabular}
}

%\thispagestyle{myheadings}
\def\leftfootline{\small{\textbf{\thepage}
\hfill ИНФОРМАТИКА И ЕЁ ПРИМЕНЕНИЯ\ \ \ том~4\ \ \ выпуск~4\ \ \ 2010}
}%
 \def\rightfootline{\small{ИНФОРМАТИКА И ЕЁ ПРИМЕНЕНИЯ\ \ \ том~4\ \ \ выпуск~4\ \ \ 2010
 \hfill \textbf{\thepage}}}
 \label{end\stat}


%Том 10 Выпуск 1-4 Год 2016

\def\stat{cont-e}
{%\hrule\par
%\vskip 7pt % 7pt
\raggedleft\Large \bf%\baselineskip=3.2ex
2\,0\,1\,6\ \ A\,U\,T\,H\,O\,R\ \ I\,N\,D\,E\,X \vskip 17pt
 \hrule
 \par
\vskip 21pt plus 6pt minus 3pt }

\label{st\stat}

\def\tit{\ }

\def\aut{\ }
\def\auf{\ }

\def\leftkol{\ } %2016 AUTHOR INDEX} % ENGLISH ABSTRACTS}

\def\rightkol{\ } %2016 AUTHOR INDEX} %ENGLISH ABSTRACTS}

\titele{\tit}{\aut}{\auf}{\leftkol}{\rightkol}

\def\leftfootline{\small{\textbf{\thepage}
\hfill INFORMATIKA I EE PRIMENENIYA~--- INFORMATICS AND APPLICATIONS\ \ \ 2016\
\ \ volume~10\ \ \ issue\ 4}
}%
 \def\rightfootline{\small{INFORMATIKA I EE PRIMENENIYA~--- INFORMATICS AND APPLICATIONS\ \ \ 2016\ \ \ volume~10\ \ \ issue\ 4
\hfill \textbf{\thepage}}}

\vspace*{-12pt}
\vspace*{-18pt}

{\tabcolsep=2.8pt
\begin{tabular}{p{382pt}cc}
&\textbf{Issue} & \textbf{Page}\\[6pt]
\Avtors{Agalarov~M.\,Ya.} see~Agalarov~Ya.\,M.&&\\
\Avtors{Agalarov~Ya.\,M., Agalarov~M.\,Ya., and
Shorgin~V.\,S.} About the optimal threshold of queue\linebreak
\\[-12pt]
\hspace*{23pt}length in a~particular problem of profit maximization
in the $M/G/1$ queuing system&2&70--79\\
\Avtors{Alexeyevsky~D.\,A.} BioNLP ontology extraction from 
a~restricted language corpus with\linebreak
\\[-12pt]
\hspace*{23pt}context-free grammars&1&119--128\\
\Avtors{Andreev~S.\,D.} see~Gaidamaka~Yu.\,V.&&\\
\Avtors{Andreev~S.\,D.} see~Ometov~A.\,Ya.&&\\
\Avtors{Arkhipov~O.\,P., Arkhipov~P.\,O., and Sidorkin~I.\,I.} The
option to create a~local coordinate\linebreak
\\[-12pt]
\hspace*{23pt}system for synchronization of selected images&3&91--97\\
\Avtors{Arkhipov~P.\,O.} see~Arkhipov~O.\,P.&&\\
\Avtors{Belousov~V.\,V.} see~Shnurkov~P.\,V.&&\\
\Avtors{Belousov~V.\,V.} see~Shnurkov~P.\,V.&&\\
\Avtors{Bening~V.\,E.} Calculation of~the~asymptotic deficiency
of~some statistical procedures based\linebreak
\\[-12pt]
\hspace*{23pt}on~samples with~random sizes&4&34--45\\
\Avtors{Borisov~A.\,V., Bosov~A.\,V., and Miller~G.\,B.} Modeling and
monitoring of VoIP connection&2&\hphantom{1}2--13\\
\Avtors{Bosov~A.\,V.} see~Borisov~A.\,V.&&\\
\Avtors{Briukhov~D.\,O.} see~Stupnikov~S.\,A.&&\\
\Avtors{Callaos~N.\,K.\ and Seyful-Mulyukov~R.\,B.} Complexity and
its information content&1&129--139\\
\Avtors{Chertok~A.\,V., Kadaner~A.\,I., Khazeeva~G.\,T., and
Sokolov~I.\,A.} Regime switching detection\linebreak
\\[-12pt]
\hspace*{23pt}for~the~Levy driven
Ornstein--Uhlenbeck process using CUSUM methods&4&46--56\\
\Avtors{Chichagov~V.\,V.} Asymptotic expansions of mean absolute
error of uniformly minimum variance unbiased and maximum likelihood
estimators on the one-parameter exponential\linebreak
\\[-12pt]
\hspace*{23pt}family model of lattice distributions&3&66--76\\
\Avtors{Danishevsky~V.\,I.} see~Kolesnikov A.\,V.&&\\
\Avtors{Fazliev~A.\,Z.} see~Kalinichenko~L.\,A.&&\\
\Avtors{Fedoseev~A.\,A.} What is behind the concept of ``knowledge in
small packages''&3&105--110\\
\Avtors{Gaidamaka~Yu.\,V., Andreev~S.\,D., Sopin~E.\,S.,
Samouylov~K.\,E., and Shorgin~S.\,Ya.} Interference analysis
of~the~device-to-device communications model with~regard to~a~signal\linebreak
\\[-12pt]
\hspace*{23pt}propagation environment&4&\hphantom{1}2--10\\
\Avtors{Gasilov~A.\,V.} see~Yakovlev~O.\,A.&&\\
\Avtors{Goncharov~A.\,V.\ and Strijov~V.\,V.} Metric time series
classification using weighted dynamic\linebreak
\\[-12pt]
\hspace*{23pt}warping relative to centroids of classes&2&36--47\\
\Avtors{Gordov~E.\,P.} see~Kalinichenko~L.\,A.&&\\
\Avtors{Gorshenin~A.\,K.} Concept of online service for stochastic
modeling of real processes&1&72--81\\
\Avtors{Gorshenin~A.\,K.} see~Shnurkov~P.\,V.&&\\
\Avtors{Gorshenin~A.\,K.} see~Shnurkov~P.\,V.&&\\
\Avtors{Grusho~A.\,A., Grusho~N.\,A., Zabezhailo~M.\,I., and
Timonina~E.\,E.} Integration of statistical and\linebreak
\\[-12pt]
\hspace*{23pt}deterministic methods for
analysis of information security&3&2--8\\
\Avtors{Grusho~A.\,A., Zabezhailo~M.\,I., and Zatsarinny~A.\,A.} On
the advanced procedure to reduce\linebreak
\\[-12pt]
\hspace*{23pt}calculation of Galois closures&4&\hphantom{1}96--104\\
\Avtors{Grusho~N.\,A.} see~Grusho~A.\,A.&&\\
\Avtors{Havanskov~V.\,A.} see~Minin~V.\,A.&&\\
\Avtors{Inkova~O.\,Yu.} see~Zatsman~I.\,M.&&\\
\Avtors{Isachenko~R.\,V.\ and Strijov~V.\,V.} Metric learning in
multiclass time series classification\linebreak
\\[-12pt]
\hspace*{23pt}problem&2&48--57\\
\end{tabular}
}
\pagebreak

\def\leftfootline{\small{\textbf{\thepage}
\hfill INFORMATIKA I EE PRIMENENIYA~--- INFORMATICS AND APPLICATIONS\ \ \ 2016\
\ \ volume~10\ \ \ issue\ 4}
}%
 \def\rightfootline{\small{INFORMATIKA I EE PRIMENENIYA~---
INFORMATICS AND APPLICATIONS\ \ \ 2016\ \ \ volume~10\ \ \ issue\ 4
\hfill \textbf{\thepage}}}

\def\leftkol{2016 AUTHOR INDEX} % ENGLISH ABSTRACTS}

\def\rightkol{2016 AUTHOR INDEX} %ENGLISH ABSTRACTS}


{\tabcolsep=2.83pt
\begin{tabular}{p{382pt}cc}
&\textbf{Issue} & \textbf{Page}\\[6pt]
\Avtors{Kadaner~A.\,I.} see~Chertok~A.\,V.&&\\[.255pt]
\Avtors{Kalinichenko~L.\,A., Volnova~A.\,A., Gordov~E.\,P.,
Kiselyova~N.\,N., Kovaleva~D.\,A., Malkov~O.\,Yu., Okladnikov~I.\,G.,
Podkolodnyy~N.\,L., Pozanenko~A.\,S., Ponomareva~N.\,V.,
Stupnikov~S.\,A.,} \textbf{and Fazliev~A.\,Z.} Data access challenges for data
intensive\linebreak
\\[-12pt]
\hspace*{23pt}research in Russia&1& 2--22\\[.255pt]
\Avtors{Karasikov~M.\,E.\ and Strijov~V.\,V.} Feature-based
time-series classification&4&121--131\\[.255pt]
\Avtors{Khazeeva~G.\,T.} see~Chertok~A.\,V.&&\\[.255pt]
\Avtors{Khokhlov~Yu.\,S.} Multivariate fractional Levy motion and its
applications&2&\hphantom{1}98--106\\[.255pt]
\Avtors{Kirikov~I.\,A., Kolesnikov~A.\,V., Listopad~S.\,V., and
Rumovskaya~S.\,B.} Fine-grained hybrid\linebreak
\\[-12pt]
\hspace*{23pt}intelligent systems. Part 2:
Bidirectional hybridization&1&\hphantom{1}96--105\\[.255pt]
\Avtors{Kirikov~I.\,A., Kolesnikov~A.\,V., Listopad~S.\,V., and
Rumovskaya~S.\,B.} ``Virtual council''~---\linebreak
\\[-12pt]
\hspace*{23pt}source environment
supporting complex diagnostic decision making&3&81--90\\[.255pt]
\Avtors{Kiselyova~N.\,N.} see~Kalinichenko~L.\,A.&&\\[.255pt]
\Avtors{Kolesnikov A.\,V., Listopad~S.\,V., Rumovskaya~S.\,B., and
Danishevsky~V.\,I.} Informal axiomatic\linebreak
\\[-12pt]
\hspace*{23pt}theory of~the~role visual models&4&114--120\\[.255pt]
\Avtors{Kolesnikov~A.\,V.} see~Kirikov~I.\,A.&&\\[.255pt]
\Avtors{Kolesnikov~A.\,V.} see~Kirikov~I.\,A.&&\\[.255pt]
\Avtors{Kolin~K.\,K.} Humanitarian aspects of information
security&3&111--121\\[.255pt]
\Avtors{Konovalov~M.\,G.\ and Razumchik~R.\,V.} Dispatching
to~two parallel nonobservable queues using\linebreak
\\[-12pt]
\hspace*{23pt}only static
information&4&57--67\\[.255pt]
\Avtors{Korchagin~A.\,Yu.} see~Korolev~V.\,Yu.&&\\[.255pt]
\Avtors{Korchagin~A.\,Yu.} see~Korolev~V.\,Yu.&&\\[.255pt]
\Avtors{Korepanov~E.\,R.} see~Sinitsyn~I.\,N.&&\\[.255pt]
\Avtors{Korepanov~E.\,R.} see~Sinitsyn~I.\,N.&&\\[.255pt]
\Avtors{Korolev~V.\,Yu., Korchagin~A.\,Yu., and Zeifman~A.\,I.} The
Poisson theorem for Bernoulli trials\linebreak
\\[-12pt]
\hspace*{23pt}with~a~random probability
of~success and~a~discrete analog of~the~Weibull distribution&4&11--20\\[.255pt]
\Avtors{Korolev~V.\,Yu., Zeifman~A.\,I., and Korchagin~A.\,Yu.}
Asymmetric Linnik distributions as~limit\linebreak
\\[-12pt]
\hspace*{23pt}laws for~random sums
of~independent random variables with~finite variances&4&21--33\\[.255pt]
\Avtors{Koucheryavy~E.\,A.} see~Ometov~A.\,Ya.&&\\[.255pt]
\Avtors{Kovaleva~D.\,A.} see~Kalinichenko~L.\,A.&&\\[.255pt]
\Avtors{Kovalyov~S.\,P.} Metaprogramming to increase
manufacturability of large-scale software-\linebreak
\\[-12pt]
\hspace*{23pt}intensive systems&1&56--66\\[.255pt]
\Avtors{Krivenko~M.\,P.} Significance tests of feature selection for
classification&3&32--40\\[.255pt]
\Avtors{Kruzhkov~M.\,G.} see~Zalizniak~Anna~A.&&\\[.255pt]
\Avtors{Kruzhkov~M.\,G.} see~Zatsman~I.\,M.&&\\[.255pt]
\Avtors{Kudryavtsev~A.\,A.} Bayesian queueing and reliability models:
\textit{A~priori} distributions with\linebreak
\\[-12pt]
\hspace*{23pt}compact support&1&67--71\\[.255pt]
\Avtors{Kudryavtsev~A.\,A.} Characteristics dependent on the balance
coefficient in Bayesian models\linebreak
\\[-12pt]
\hspace*{23pt}with compact support of \textit{a priori}
distributions&3&77--80\\[.255pt]
\Avtors{Kudryavtsev~A.\,A.\ and Palionnaia~S.\,I.} Bayesian recurrent
model of reliability growth:\linebreak
\\[-12pt]
\hspace*{23pt}Parabolic distribution of parameters&2&80--83\\[.255pt]
\Avtors{Kudryavtsev~A.\,A.\ and Titova~A.\,I.} Bayesian queuing
and~reliability models: Degenerate-\linebreak
\\[-12pt]
\hspace*{23pt}Weibull case&4&68--71\\[.255pt]
\Avtors{Leontyev~N.\,D.\ and Ushakov~V.\,G.} Analysis of a queueing
system with autoregressive arrivals\linebreak
\\[-12pt]
\hspace*{23pt}and nonpreemptive priority&3&15--22\\[.255pt]
\Avtors{Listopad~S.\,V.} see~Kirikov~I.\,A.&&\\[.255pt]
\Avtors{Listopad~S.\,V.} see~Kirikov~I.\,A.&&\\[.255pt]
\Avtors{Listopad~S.\,V.} see~Kolesnikov A.\,V.&&\\[.255pt]
\Avtors{Malkov~O.\,Yu.} see~Kalinichenko~L.\,A.&&\\[.255pt]
\Avtors{Markov~A.\,S., Monakhov~M.\,M., and
Ulyanov~V.\,V.} Generalized Cornish--Fisher expansions\linebreak
\\[-12pt]
\hspace*{23pt}for distributions of statistics based on samples
of random size&2&84--91\\[.255pt]
\Avtors{Melnikov~A.\,K.\ and Ronzhin~A.\,F.} Generalized statistical
method of~text analysis based\linebreak
\\[-12pt]
\hspace*{23pt}on~calculation of~probability distributions
of~statistical values&4&89--95\\
\end{tabular}
}
\pagebreak

\def\leftfootline{\small{\textbf{\thepage}
\hfill INFORMATIKA I EE PRIMENENIYA~--- INFORMATICS AND APPLICATIONS\ \ \ 2016\
\ \ volume~10\ \ \ issue\ 4}
}%
 \def\rightfootline{\small{INFORMATIKA I EE PRIMENENIYA~---
INFORMATICS AND APPLICATIONS\ \ \ 2016\ \ \ volume~10\ \ \ issue\ 4
\hfill \textbf{\thepage}}}

\def\leftkol{2016 AUTHOR INDEX} % ENGLISH ABSTRACTS}

\def\rightkol{2016 AUTHOR INDEX} %ENGLISH ABSTRACTS}


{\tabcolsep=3pt
\begin{tabular}{p{381pt}cc}
&\textbf{Issue} & \textbf{Page}\\[6pt]
\Avtors{Meykhanadzhyan~L.\,A.} Stationary characteristics of the finite
capacity queueing system with\linebreak
\\[-12pt]
\hspace*{23pt}inverse service order and generalized
probabilistic priority&2&123--131\\[.23pt]
\Avtors{Miller~G.\,B.} see~Borisov~A.\,V.&&\\[.23pt]
\Avtors{Minin~V.\,A., Zatsman~I.\,M., Havanskov~V.\,A., and
Shubnikov~S.\,K.} Intensity of citation of scientific publications in
inventions on information and computer technologies patented\linebreak
\\[-12pt]
\hspace*{23pt}in Russia by domestic and foreign applicants&2&107--122\\[.23pt]
\Avtors{Monakhov~M.\,M.} see~Markov~A.\,S.&&\\[.23pt]
\Avtors{Naumov~V.\,A.\ and Samouylov~K.\,E.} On relationship
between queuing systems with resources\linebreak
\\[-12pt]
\hspace*{23pt}and Erlang networks&3&\hphantom{1}9--14\\[.23pt]
\Avtors{Okladnikov~I.\,G.} see~Kalinichenko~L.\,A.&&\\[.23pt]
\Avtors{Ometov~A.\,Ya., Andreev~S.\,D., Turlikov~A.\,M., and
Koucheryavy~E.\,A.} Performance analysis of\linebreak
\\[-12pt]
\hspace*{23pt}a wireless data
aggregation system with contention for contemporary sensor
networks&3&23--31\\[.23pt]
\Avtors{Palionnaia~S.\,I.} see~Kudryavtsev~A.\,A.&&\\[.23pt]
\Avtors{Podkolodnyy~N.\,L.} see~Kalinichenko~L.\,A.&&\\[.23pt]
\Avtors{Ponomareva~N.\,V.} see~Kalinichenko~L.\,A.&&\\[.23pt]
\Avtors{Popkova~N.\,A.} see~Zatsman~I.\,M.&&\\[.23pt]
\Avtors{Pozanenko~A.\,S.} see~Kalinichenko~L.\,A.&&\\[.23pt]
\Avtors{Razumchik~R.\,V.} see~Konovalov~M.\,G.&&\\[.23pt]
\Avtors{Ronzhin~A.\,F.} see~Melnikov~A.\,K.&&\\[.23pt]
\Avtors{Rumovskaya~S.\,B.} see~Kirikov~I.\,A.&&\\[.23pt]
\Avtors{Rumovskaya~S.\,B.} see~Kirikov~I.\,A.&&\\[.23pt]
\Avtors{Rumovskaya~S.\,B.} see~Kolesnikov A.\,V.&&\\[.23pt]
\Avtors{Samouylov~K.\,E.} see~Gaidamaka~Yu.\,V.&&\\[.23pt]
\Avtors{Samouylov~K.\,E.} see~Naumov~V.\,A.&&\\[.23pt]
\Avtors{Serebryanskii~S.\,M.} see~Tyrsin~A.\,N.&&\\[.23pt]
\Avtors{Seyful-Mulyukov~R.\,B.} see~Callaos~N.\,K.&&\\[.23pt]
\Avtors{Shestakov~O.\,V.} Statistical properties of the denoising method
based on the stabilized hard\linebreak
\\[-12pt]
\hspace*{23pt}thresholding&2&65--69\\[.23pt]
\Avtors{Shestakov~O.\,V.} The strong law of large numbers for the risk
estimate in the problem of\linebreak
\\[-12pt]
\hspace*{23pt}tomographic image reconstruction from
projections with a correlated noise&3&41--45\\[.23pt]
\Avtors{Shestakov~O.\,V.} see~Zakharova~T.\,V.&&\\[.23pt]
\Avtors{Shnurkov~P.\,V., Gorshenin~A.\,K., and Belousov~V.\,V.}
Analytical solution of~the~optimal control\linebreak
\\[-12pt]
\hspace*{23pt}task of~a~semi-Markov
process with~finite set of~states&4&72--88\\[.23pt]
\Avtors{Shnurkov~P.\,V., Zasypko~V.\,V., Belousov~V.\,V., and
Gorshenin~A.\,K.} Development of the algorithm of numerical solution
of the optimal investment control problem\linebreak
\\[-12pt]
\hspace*{23pt}in the closed dynamical model of three-sector economy&1&82--95\\[.23pt]
\Avtors{Shorgin~S.\,Ya.} see~Gaidamaka~Yu.\,V.&&\\[.23pt]
\Avtors{Shorgin~V.\,S.} see~Agalarov~Ya.\,M.&&\\[.23pt]
\Avtors{Shubnikov~S.\,K.} see~Minin~V.\,A.&&\\[.23pt]
\Avtors{Sidorkin~I.\,I.} see~Arkhipov~O.\,P.&&\\[.23pt]
\Avtors{Sinitsyn~I.\,N.} Analytical modeling of processes in stochastic
systems with complex fractional\linebreak
\\[-12pt]
\hspace*{23pt}order Bessel nonlinearities&3&55--65\\[.23pt]
\Avtors{Sinitsyn~I.\,N.} Orthogonal supoptimal filters for nonlinear
stochastic systems on manifolds&1&34--44\\[.23pt]
\Avtors{Sinitsyn~I.\,N.\ and Korepanov~E.\,R.} Normal Pugachev
conditionally-optimal filters and extra-\linebreak
\\[-12pt]
\hspace*{23pt}polators for state linear stochastic systems&2&14--23\\[.23pt]
\Avtors{Sinitsyn~I.\,N.\ and Sinitsyn~V.\,I.} Analytical modeling of
distributions in stochastic systems on\linebreak
\\[-12pt]
\hspace*{23pt}manifolds based on ellipsoidal approximation&1&45--55\\[.23pt]
\Avtors{Sinitsyn~I.\,N., Sinitsyn~V.\,I., and
Korepanov~E.\,R.} Ellipsoidal suboptimal filters for nonlinear\linebreak
\\[-12pt]
\hspace*{23pt}stochastic systems on manifolds&2&24--35\\[.23pt]
\Avtors{Sinitsyn~V.\,I.} see~Sinitsyn~I.\,N.&&\\[.23pt]
\Avtors{Sinitsyn~V.\,I.} see~Sinitsyn~I.\,N.&&\\[.23pt]
\Avtors{Skvortsov~N.\,A.} see~Stupnikov~S.\,A.&&\\[.23pt]
\Avtors{Sokolov~I.\,A.} see~Chertok~A.\,V.&&\\
\end{tabular}
}
\pagebreak

\def\leftfootline{\small{\textbf{\thepage}
\hfill INFORMATIKA I EE PRIMENENIYA~--- INFORMATICS AND APPLICATIONS\ \ \ 2016\
\ \ volume~10\ \ \ issue\ 4}
}%
 \def\rightfootline{\small{INFORMATIKA I EE PRIMENENIYA~---
INFORMATICS AND APPLICATIONS\ \ \ 2016\ \ \ volume~10\ \ \ issue\ 4
\hfill \textbf{\thepage}}}

\def\leftkol{2016 AUTHOR INDEX} % ENGLISH ABSTRACTS}

\def\rightkol{2016 AUTHOR INDEX} %ENGLISH ABSTRACTS}


{\tabcolsep=3pt
\begin{tabular}{p{382pt}cc}
&\textbf{Issue} & \textbf{Page}\\[6pt]
\Avtors{Sopin~E.\,S.} see~Gaidamaka~Yu.\,V.&&\\
\Avtors{Strijov~V.\,V.} see~Goncharov~A.\,V.&&\\
\Avtors{Strijov~V.\,V.} see~Isachenko~R.\,V.&&\\
\Avtors{Strijov~V.\,V.} see~Karasikov~M.\,E.&&\\
\Avtors{Stupnikov~S.\,A., Briukhov~D.\,O., and Skvortsov~N.\,A.}
Co-lending systemic risk analysis over\linebreak
\\[-12pt]
\hspace*{23pt}heterogeneous data collections&1&23--33\\
\Avtors{Stupnikov~S.\,A.} see~Kalinichenko~L.\,A.&&\\
\Avtors{Suchkov~A.\,P.} see~Zatsarinny~A.\,A.&&\\
\Avtors{Timonina~E.\,E.} see~Grusho~A.\,A.&&\\
\Avtors{Titova~A.\,I.} see~Kudryavtsev~A.\,A.&&\\
\Avtors{Turlikov~A.\,M.} see~Ometov~A.\,Ya.&&\\
\Avtors{Tyrsin~A.\,N.\ and Serebryanskii~S.\,M.} Recognition of
dependences on the basis of inverse\linebreak
\\[-12pt]
\hspace*{23pt}mapping&2&58--64\\
\Avtors{Ulyanov~V.\,V.} see~Markov~A.\,S.&&\\
\Avtors{Ushakov~V.\,G.} Queueing system with working vacations and
hyperexponential input stream&2&92--97\\
\Avtors{Ushakov~V.\,G.} see~Leontyev~N.\,D.&&\\
\Avtors{Volnova~A.\,A.} see~Kalinichenko~L.\,A.&&\\
\Avtors{Yakovlev~O.\,A.\ and Gasilov~A.\,V.} Speeded-up stereo
matching using geodesic support weights&3&\hphantom{1}98--104\\
\Avtors{Zabezhailo~M.\,I.} see~Grusho~A.\,A.&&\\
\Avtors{Zabezhailo~M.\,I.} see~Grusho~A.\,A.&&\\
\Avtors{Zakharova~T.\,V.\ and Shestakov~O.\,V.} Precision analysis of
wavelet processing of aerodynamic\linebreak
\\[-12pt]
\hspace*{23pt}flow patterns&3&46--54\\
\Avtors{Zalizniak~Anna~A.\ and Kruzhkov~M.\,G.} Database
of~Russian impersonal verbal constructions&4&132--141\\
\Avtors{Zasypko~V.\,V.} see~Shnurkov~P.\,V.&&\\
\Avtors{Zatsarinny~A.\,A.\ and Suchkov~A.\,P.} Systems engineering
approaches to~the~establishment of\linebreak
\\[-12pt]
\hspace*{23pt}a~system for~decision support based
on~situational analysis&4&105--113\\
\Avtors{Zatsarinny~A.\,A.} see~Grusho~A.\,A.&&\\
\Avtors{Zatsman~I.\,M., Inkova~O.\,Yu., Kruzhkov~M.\,G., and
Popkova~N.\,A.} Representation of cross-\linebreak
\\[-12pt]
\hspace*{23pt}lingual knowledge about
connectors in supracorpora databases&1&106--118\\
\Avtors{Zatsman~I.\,M.} see~Minin~V.\,A.&&\\
\Avtors{Zeifman~A.\,I.} see~Korolev~V.\,Yu.&&\\
\Avtors{Zeifman~A.\,I.} see~Korolev~V.\,Yu.&&\\
\end{tabular}
}

%\thispagestyle{myheadings}
\def\leftfootline{\small{\textbf{\thepage}
\hfill INFORMATIKA I EE PRIMENENIYA~--- INFORMATICS AND APPLICATIONS\ \ \ 2016\
\ \ volume~10\ \ \ issue\ 4}
}%
 \def\rightfootline{\small{INFORMATIKA I EE PRIMENENIYA~---
INFORMATICS AND APPLICATIONS\ \ \ 2016\ \ \ volume~10\ \ \ issue\ 4
\hfill \textbf{\thepage}}}

 \label{end\stat}

\newpage


\vspace*{-60pt} {\small
{\baselineskip=9.1pt
\section*{Правила подготовки рукописей статей для публикации в журнале
<<Информатика и её применения>>}

\thispagestyle{empty}

 Журнал <<Информатика и её применения>> публикует
теоретические, обзорные и дискуссионные статьи, посвященные научным
исследованиям и разработкам в области информатики и ее приложений. Журнал
издается на русском языке. По специальному решению редколлегии отдельные статьи,
в виде исключения, могут печататься на английском языке.
Тематика журнала охватывает следующие направления:
\begin{itemize}
\item теоретические основы информатики; %\\[-13.5pt]
\item математические методы исследования сложных систем и процессов; %\\[-13.5pt]
\item информационные системы и сети; %\\[-13.5pt]
\item информационные технологии; %\\[-13.5pt]
\item архитектура и программное
обеспечение вычислительных комплексов и сетей.
\end{itemize}
\begin{enumerate}
\item В журнале печатаются результаты, ранее не
опубликованные и не предназначенные к одновременной публикации в других
изданиях. Публикация не должна нарушать закон об авторских правах. Направляя
свою рукопись в редакцию, авторы автоматически передают учредителям и
редколлегии неисключительные права на издание данной статьи на русском языке и
на ее распространение в России и за рубежом. При этом за авторами сохраняются
все права как собственников данной рукописи. В связи с этим авторами должно
быть представлено в редакцию письмо в следующей форме:
Соглашение о передаче права на публикацию:

\textit{<<Мы, нижеподписавшиеся, авторы рукописи <<$\qquad\qquad$>>, передаем
учредителям и редколлегии журнала <<Информатика и её применения>>
неисключительное право опубликовать данную рукопись статьи на русском языке как
в печатной, так и в электронной версиях журнала. Мы подтверждаем, что данная
публикация не нарушает авторского права других лиц или организаций. Подписи
авторов: (ф.\,и.\,о., дата, адрес)>>.}

Указанное соглашение может быть представлено 
как в бумажном виде, так и в виде отсканированной копии (с подписями авторов).


Редколлегия вправе запросить у авторов экспертное заключение о возможности
опубликования представленной статьи в открытой печати. %\\[-13.5pt]
\item Статья
подписывается всеми авторами. На отдельном листе представляются данные автора
(или всех авторов): фамилия, полные имя и отчество, телефон, факс, e-mail,
почтовый адрес. Если работа выполнена несколькими авторами, указывается фамилия
одного из них, ответственного за переписку с редакцией. %\\[-13.5pt]
\item Редакция журнала
осуществляет самостоятельную экспертизу присланных статей. Возвращение рукописи
на доработку не означает, что статья уже принята к печати. Доработанный вариант
с ответом на замечания рецензента необходимо прислать в редакцию. %\\[-13.5pt]
\item Решение
редакционной коллегии о принятии статьи к печати или ее отклонении сообщается
авторам. Редколлегия не обязуется направлять рецензию авторам отклоненной
статьи. %\\[-13.5pt]
\item Корректура статей высылается авторам для просмотра. Редакция
просит авторов присылать свои замечания в кратчайшие сроки. %\\[-13.5pt]
\item При
подготовке рукописи в MS Word рекомендуется использовать следующие настройки.
Параметры страницы: формат~--- А4; ориентация~--- книжная; поля (см): внутри~---
2,5, снаружи~--- 1,5, сверху~--- 2, снизу~--- 2, от края до нижнего
колонтитула~--- 1,3. Основной текст: стиль~--- <<Обычный>>: шрифт Times New
Roman, размер 14~пунктов, абзацный отступ~--- 0,5~см, 1,5 интервала,
выравнивание~--- по ширине. Рекомендуемый объем рукописи~--- не свыше
25~страниц указанного формата. Ознакомиться с шаблонами, содержащими примеры
оформления, можно по адресу в Интернете:
\textsf{http://www.ipiran.ru/journal/template.doc}.
\item К рукописи, предоставляемой в 2-х
экземплярах, обязательно прилагается электронная версия статьи (как правило, в
форматах MS WORD (.doc) или \LaTeX\ (.tex), а также~--- дополнительно~--- в
формате .pdf) на дискете, лазерном диске или по электронной почте. Сокращения
слов, кроме стандартных, не применяются. Все страницы рукописи должны быть
пронумерованы. %\\[-13.5pt]
\item Статья должна содержать следующую информацию на русском и
английском языках: название, Ф.И.О. авторов, места работы авторов и их
электронные адреса, подробные сведения об авторах, оформленные в соответствии с форматом, 
определяемым файлами {\sf http://www.ipiran.ru/journal/issues/2011\_05\_01/authors.asp} и 
{\sf http://www.ipiran.ru/journal/issues/2011\_01\_eng/authors.asp},
аннотация (не более 100~слов), ключевые слова. Ссылки на
литературу в тексте статьи нумеруются (в квадратных скобках) и располагаются в
порядке их первого упоминания. В~списке литературы не должно быть позиций, на которые нет ссылки в тексте статьи.
Все фамилии авторов, заглавия статей, названия
книг, конференций и~т.\,п.\ даются на языке оригинала, если этот язык
использует кириллический или латинский алфавит. %\\[-13.5pt]
\item Присланные в редакцию материалы авторам не возвращаются.
\item При отправке файлов по электронной
почте просим придерживаться следующих правил:
\begin{itemize}
\item указывать в поле subject (тема) название журнала и фамилию автора; %\\[-13.5pt]
\item использовать attach (присоединение); %\\[-13.5pt]
\item в случае больших объемов информации возможно
использование общеизвестных архиваторов (ZIP, RAR); %\\[-13.5pt]
\item в состав электронной версии статьи должны входить: файл, содержащий текст статьи, и файл(ы),
содержащий(е) иллюстрации. %\\[-13.5pt]
\end{itemize}
\item Журнал <<Информатика и её применения>> является некоммерческим изданием. 
Плата за публикацию с авторов не взимается, гонорар авторам не выплачивается.
\end{enumerate}
\thispagestyle{empty}
\textbf{Адрес редакции:} Москва 119333,
ул.~Вавилова, д.~44, корп.~2, ИПИ РАН\\
\hphantom{\textbf{Адрес редакции:} }Тел.: +7 (499) 135-86-92\ \
Факс:  +7 (495) 930-45-05\ \  E-mail:   rust@ipiran.ru }
}

\end{document}


%\tableofcontents

%\end{document}





%\def\stat{cont}
{%\hrule\par
%\vskip 7pt % 7pt
\raggedleft\Large \bf%\baselineskip=3.2ex
А\,В\,Т\,О\,Р\,С\,К\,И\,Й\ \ У\,К\,А\,З\,А\,Т\,Е\,Л\,Ь\ \ З\,А\ \ 2\,0\,0\,7 г. \vskip 17pt
    \hrule
    \par
\vskip 21pt plus 6pt minus 3pt }

\label{st\stat}

\def\tit{\ }

\def\aut{\ }
\def\auf{\ }

\def\leftkol{\ } % ENGLISH ABSTRACTS}

\def\rightkol{\ } %ENGLISH ABSTRACTS}

\titele{\tit}{\aut}{\auf}{\leftkol}{\rightkol}


\contentsline {chapter}{\ }{Выпуск \quad Стр.} 
\contentsline {section}{\textbf{Батракова Д.\,А., Королев В.\,Ю., Шоргин С.\,Я.}\ \ Новый метод вероятностно-ста\-ти\-сти\-че\-ско\-го анализа информационных потоков в\nobreakspace {}телекоммуникационных сетях}{\qquad 1 \qquad 40} 
\contentsline {section}{\textbf{Борисов А.\,В.}\ \ Байесовское оценивание в системах наблюдения с\nobreakspace {}марковскими скачкообразными процессами: игровой подход}{\qquad 2 \qquad 65}
\contentsline {section}{\textbf{Босов А.\,В., Иванов А.\,В.}\ \ Программная инфраструктура информационного Web-пор\-тала}{\qquad 2 \qquad 50}
\contentsline {section}{\textbf{Захаров В.\,Н., Калиниченко Л.\,А., Соколов И.\,А., Ступников С.\,А.}\ \ Конструирование канонических информационных моделей для интегрированных информационных систем}{\qquad 2 \qquad 15}
\contentsline {section}{\textbf{Захаров В.\,Н., Козмидиади В.\,А.}\ \ Средства обеспечения отказоустойчивости при\-ло\-жений}{\qquad 1 \qquad 14} 
\contentsline {section}{\textbf{Иванов А.\,В.}\ \ см. Босов А.\,В.\hfill\hfill\hfill\hfill\hfill\hfill\hfill\hfill\hfill\hfill\hfill\hfill\hfill\hfill\hfill\hfill\hfill\hfill\hfill\hfill\hfill\hfill\hfill\hfill\hfill\hfill\hfill\hfill\hfill\hfill\hfill\hfill\hfill\hfill\hfill}{\ }
\contentsline {section}{\textbf{Ильин В.\,Д., Соколов И.\,А.}\ \ Символьная модель системы знаний информатики в\nobreakspace {}че\-ло\-ве\-ко-автоматной среде}{\qquad 1 \qquad 66} 
\contentsline {section}{\textbf{Калиниченко Л.\,А.}\ \ см. Захаров В.\,Н.\hfill\hfill\hfill\hfill\hfill\hfill\hfill\hfill\hfill\hfill\hfill\hfill\hfill\hfill\hfill\hfill\hfill\hfill\hfill\hfill\hfill\hfill\hfill\hfill\hfill\hfill\hfill\hfill\hfill\hfill\hfill\hfill\hfill\hfill\hfill}{\ }
\contentsline {section}{\textbf{Козеренко Е.\,Б.}\ \ Лингвистическое моделирование для систем машинного перевода и обработки знаний}{\qquad 1 \qquad 54} 
\contentsline {section}{\textbf{Козмидиади В.\,А.}\ \ см. Захаров В.\,Н.\hfill\hfill\hfill\hfill\hfill\hfill\hfill\hfill\hfill\hfill\hfill\hfill\hfill\hfill\hfill\hfill\hfill\hfill\hfill\hfill\hfill\hfill\hfill\hfill\hfill\hfill\hfill\hfill\hfill\hfill\hfill\hfill\hfill\hfill\hfill }{\ } 
\contentsline {section}{\textbf{Королев В.\,Ю.}\ \ см. Батракова Д.\,А.\hfill\hfill\hfill\hfill\hfill\hfill\hfill\hfill\hfill\hfill\hfill\hfill\hfill\hfill\hfill\hfill\hfill\hfill\hfill\hfill\hfill\hfill\hfill\hfill\hfill\hfill\hfill\hfill\hfill\hfill\hfill\hfill\hfill\hfill\hfill}{\ } 
\contentsline {section}{\textbf{Кудрявцев А.\,А., Шоргин С.\,Я.}\ \ Байесовский подход к\nobreakspace {}анализу систем массового обслуживания и\nobreakspace {}показателей надежности}{\qquad 2 \qquad 76}
\contentsline {section}{\textbf{Печинкин А.\,В., Соколов И.\,А., Чаплыгин В.\,В.}\ \ Многолинейная система массового обслуживания с конечным накопителем и ненадежными приборами}{\qquad 1 \qquad 27} 
\contentsline {section}{\textbf{Печинкин А.\,В., Соколов И.\,А., Чаплыгин В.\,В.}\ \ Стационарные характеристики многолинейной\nobreakspace {}системы массового обслуживания с\nobreakspace {}одновременными отказами приборов}{\qquad 2 \qquad 39}
\contentsline {section}{\textbf{Синицын И.\,Н.}\ \ Корреляционные методы построения аналитических информационных моделей флуктуаций полюса Земли по априорным данным}{\qquad 2 \qquad \hphantom{9}2}
\contentsline {section}{\textbf{Синицын И.\,Н.}\ \ Развитие теории фильтров Пугачева для оперативной обработки информации в стохастических системах}{{\qquad 1 \qquad \hphantom{9}3}} 
\contentsline {section}{\textbf{Соколов И.\,А.}\ \ см. Захаров В.\,Н.\hfill\hfill\hfill\hfill\hfill\hfill\hfill\hfill\hfill\hfill\hfill\hfill\hfill\hfill\hfill\hfill\hfill\hfill\hfill\hfill\hfill\hfill\hfill\hfill\hfill\hfill\hfill\hfill\hfill\hfill\hfill\hfill\hfill\hfill\hfill}{\ }
\contentsline {section}{\textbf{Соколов И.\,А.}\ \ см. Ильин В.\,Д.\hfill\hfill\hfill\hfill\hfill\hfill\hfill\hfill\hfill\hfill\hfill\hfill\hfill\hfill\hfill\hfill\hfill\hfill\hfill\hfill\hfill\hfill\hfill\hfill\hfill\hfill\hfill\hfill\hfill\hfill\hfill\hfill\hfill\hfill\hfill}{\ } 
\contentsline {section}{\textbf{Соколов И.\,А.}\ \ см. Печинкин А.\,В.\hfill\hfill\hfill\hfill\hfill\hfill\hfill\hfill\hfill\hfill\hfill\hfill\hfill\hfill\hfill\hfill\hfill\hfill\hfill\hfill\hfill\hfill\hfill\hfill\hfill\hfill\hfill\hfill\hfill\hfill\hfill\hfill\hfill\hfill\hfill}{\ } 
\contentsline {section}{\textbf{Соколов И.\,А.}\ \ см. Печинкин А.\,В.\hfill\hfill\hfill\hfill\hfill\hfill\hfill\hfill\hfill\hfill\hfill\hfill\hfill\hfill\hfill\hfill\hfill\hfill\hfill\hfill\hfill\hfill\hfill\hfill\hfill\hfill\hfill\hfill\hfill\hfill\hfill\hfill\hfill\hfill\hfill}{\ }
\contentsline {section}{\textbf{Ступников С.\,А.}\ \ см. Захаров В.\,Н.\hfill\hfill\hfill\hfill\hfill\hfill\hfill\hfill\hfill\hfill\hfill\hfill\hfill\hfill\hfill\hfill\hfill\hfill\hfill\hfill\hfill\hfill\hfill\hfill\hfill\hfill\hfill\hfill\hfill\hfill\hfill\hfill\hfill\hfill\hfill}{\ }
\contentsline {section}{\textbf{Чаплыгин В.\,В.}\ \ см. Печинкин А.\,В.\hfill\hfill\hfill\hfill\hfill\hfill\hfill\hfill\hfill\hfill\hfill\hfill\hfill\hfill\hfill\hfill\hfill\hfill\hfill\hfill\hfill\hfill\hfill\hfill\hfill\hfill\hfill\hfill\hfill\hfill\hfill\hfill\hfill\hfill\hfill}{\ } 
\contentsline {section}{\textbf{Чаплыгин В.\,В.}\ \ см. Печинкин А.\,В.\hfill\hfill\hfill\hfill\hfill\hfill\hfill\hfill\hfill\hfill\hfill\hfill\hfill\hfill\hfill\hfill\hfill\hfill\hfill\hfill\hfill\hfill\hfill\hfill\hfill\hfill\hfill\hfill\hfill\hfill\hfill\hfill\hfill\hfill\hfill}{\ }
\contentsline {section}{\textbf{Шоргин С.\,Я.}\ \ см. Батракова Д.\,А.\hfill\hfill\hfill\hfill\hfill\hfill\hfill\hfill\hfill\hfill\hfill\hfill\hfill\hfill\hfill\hfill\hfill\hfill\hfill\hfill\hfill\hfill\hfill\hfill\hfill\hfill\hfill\hfill\hfill\hfill\hfill\hfill\hfill\hfill\hfill}{\ } 
\contentsline {section}{\textbf{Шоргин С.\,Я.}\ \ см. Кудрявцев А.\,А.\hfill\hfill\hfill\hfill\hfill\hfill\hfill\hfill\hfill\hfill\hfill\hfill\hfill\hfill\hfill\hfill\hfill\hfill\hfill\hfill\hfill\hfill\hfill\hfill\hfill\hfill\hfill\hfill\hfill\hfill\hfill\hfill\hfill\hfill\hfill}{\ }
%\thispagestyle{myheadings}
\def\leftfootline{\small{\textbf{\thepage}
\hfill ИНФОРМАТИКА И ЕЁ ПРИМЕНЕНИЯ\ \ \ том~1\ \ \ выпуск~2\ \ \ 2007}
}%
 \def\rightfootline{\small{ИНФОРМАТИКА И ЕЁ ПРИМЕНЕНИЯ\ \ \ том~1\ \ \ выпуск~2\ \ \ 2007
 \hfill \textbf{\thepage}}}
 \label{end\stat}

%\def\stat{cont-e}
{%\hrule\par
%\vskip 7pt % 7pt
\raggedleft\Large \bf%\baselineskip=3.2ex
2\,0\,0\,7\ \ A\,U\,T\,H\,O\,R\ \ I\,N\,D\,E\,X \vskip 17pt
    \hrule
    \par
\vskip 21pt plus 6pt minus 3pt }

\label{st\stat}

\def\tit{\ }

\def\aut{\ }
\def\auf{\ }

\def\leftkol{\ } % ENGLISH ABSTRACTS}

\def\rightkol{\ } %ENGLISH ABSTRACTS}

\titele{\tit}{\aut}{\auf}{\leftkol}{\rightkol}


\contentsline {chapter}{\ }{Issue \quad Page} 
\contentsline {subsection}{\textbf{Batrakova D.\,A., Korolev V.\,Yu., Shorgin S.\,Ya.}\ \ A New Method for the Probabilistic and Statistical Analysis of Information Flows in Telecommunication Networks}{\qquad 1 \qquad 40} 
\contentsline {subsection}{\textbf{Borisov A.\,V.}\ \ Bayesian Estimation in\nobreakspace {}Observation Systems with\nobreakspace {}Markov Jump Processes: Game-Theoretic Approach}{\qquad 2 \qquad 65} 
\contentsline {subsection}{\textbf{Bosov A.\,V., Ivanov A.\,V.}\ \ Linguistic Simulation for Machine Translation and Knowledge Management Systems}{\qquad 2 \qquad 50} 
\contentsline {subsection}{\textbf{Chaplygin V.\,V.} see Pechinkin A.\,V.\hfill\hfill\hfill\hfill\hfill\hfill\hfill\hfill\hfill\hfill\hfill\hfill\hfill\hfill\hfill\hfill\hfill\hfill\hfill\hfill\hfill\hfill\hfill\hfill\hfill\hfill\hfill\hfill\hfill\hfill\hfill\hfill\hfill\hfill\hfill}{\ }
\contentsline {subsection}{\textbf{Chaplygin V.\,V.} see Pechinkin A.\,V.\hfill\hfill\hfill\hfill\hfill\hfill\hfill\hfill\hfill\hfill\hfill\hfill\hfill\hfill\hfill\hfill\hfill\hfill\hfill\hfill\hfill\hfill\hfill\hfill\hfill\hfill\hfill\hfill\hfill\hfill\hfill\hfill\hfill\hfill\hfill}{\ }
\contentsline {subsection}{\textbf{Ilyin V.\,D., Sokolov I.\,A.}\ \ The Symbol Model of Informatics Knowledge System in Human-Automaton Environment}{\qquad 1 \qquad 66} 
\contentsline {subsection}{\textbf{Ivanov A.\,V.} see Bosov A.\,V.\hfill\hfill\hfill\hfill\hfill\hfill\hfill\hfill\hfill\hfill\hfill\hfill\hfill\hfill\hfill\hfill\hfill\hfill\hfill\hfill\hfill\hfill\hfill\hfill\hfill\hfill\hfill\hfill\hfill\hfill\hfill\hfill\hfill\hfill\hfill}{\ }
\contentsline {subsection}{\textbf{Kalinichenko L.\,A.} see Zakharov V.\,N.\hfill\hfill\hfill\hfill\hfill\hfill\hfill\hfill\hfill\hfill\hfill\hfill\hfill\hfill\hfill\hfill\hfill\hfill\hfill\hfill\hfill\hfill\hfill\hfill\hfill\hfill\hfill\hfill\hfill\hfill\hfill\hfill\hfill\hfill\hfill}{\ }
\contentsline {subsection}{\textbf{Korolev V.\,Yu.} see Batrakova D.\,A.\hfill\hfill\hfill\hfill\hfill\hfill\hfill\hfill\hfill\hfill\hfill\hfill\hfill\hfill\hfill\hfill\hfill\hfill\hfill\hfill\hfill\hfill\hfill\hfill\hfill\hfill\hfill\hfill\hfill\hfill\hfill\hfill\hfill\hfill\hfill}{\ }
\contentsline {subsection}{\textbf{Kozerenko E.\,B.}\ \ Linguistic Simulation for Machine Translation and Knowledge Management Systems}{\qquad 1 \qquad 54} 
\contentsline {subsection}{\textbf{Kozmidiady V.\,A.} see Zakharov V.\,N.\hfill\hfill\hfill\hfill\hfill\hfill\hfill\hfill\hfill\hfill\hfill\hfill\hfill\hfill\hfill\hfill\hfill\hfill\hfill\hfill\hfill\hfill\hfill\hfill\hfill\hfill\hfill\hfill\hfill\hfill\hfill\hfill\hfill\hfill\hfill}{\ }
\contentsline {subsection}{\textbf{Kudryavtsev A.\,A., Shorgin S.\,Ya.}\ \ Bayesian Approach to Queueing Systems and Reliability Characteristics}{\qquad 2 \qquad 76} 
\contentsline {subsection}{\textbf{Pechinkin A.\,V., Sokolov I.\,A., Chaplygin V.\,V.}\ \ Multichannel Queuing System with Finite Buffer and Unreliable Servers}{\qquad 1 \qquad 27} 
\contentsline {subsection}{\textbf{Pechinkin A.\,V., Sokolov I.\,A., Chaplygin V.\,V.}\ \ Stationary Characteristics of a Multichannel Queueing System with\nobreakspace {}Simultaneous Refusals of Servers}{\qquad 2 \qquad 39} 
\contentsline {subsection}{\textbf{Shorgin S.\,Ya.} see Batrakova D.\,A.\hfill\hfill\hfill\hfill\hfill\hfill\hfill\hfill\hfill\hfill\hfill\hfill\hfill\hfill\hfill\hfill\hfill\hfill\hfill\hfill\hfill\hfill\hfill\hfill\hfill\hfill\hfill\hfill\hfill\hfill\hfill\hfill\hfill\hfill\hfill}{\ }
\contentsline {subsection}{\textbf{Shorgin S.\,Ya.} see Kudryavtsev A.\,A.\hfill\hfill\hfill\hfill\hfill\hfill\hfill\hfill\hfill\hfill\hfill\hfill\hfill\hfill\hfill\hfill\hfill\hfill\hfill\hfill\hfill\hfill\hfill\hfill\hfill\hfill\hfill\hfill\hfill\hfill\hfill\hfill\hfill\hfill\hfill}{\ }
\contentsline {subsection}{\textbf{Sinitsyn I.\,N.}\ \ Correlational Methods for Analytical Informational Models of the Earth Pole Fluctuations Design Based on a priori Data}{\qquad 2 \qquad \hphantom{9}2}
\contentsline {subsection}{\textbf{Sinitsyn I.\,N.}\ \ Development of Pugachev Filtering for Stochastic Systems}{\qquad 1 \qquad \hphantom{9}3}
\contentsline {subsection}{\textbf{Sokolov I.\,A.} see Ilyin V.\,D.\hfill\hfill\hfill\hfill\hfill\hfill\hfill\hfill\hfill\hfill\hfill\hfill\hfill\hfill\hfill\hfill\hfill\hfill\hfill\hfill\hfill\hfill\hfill\hfill\hfill\hfill\hfill\hfill\hfill\hfill\hfill\hfill\hfill\hfill\hfill}{\ }
\contentsline {subsection}{\textbf{Sokolov I.\,A.} see Pechinkin A.\,V.\hfill\hfill\hfill\hfill\hfill\hfill\hfill\hfill\hfill\hfill\hfill\hfill\hfill\hfill\hfill\hfill\hfill\hfill\hfill\hfill\hfill\hfill\hfill\hfill\hfill\hfill\hfill\hfill\hfill\hfill\hfill\hfill\hfill\hfill\hfill}{\ }
\contentsline {subsection}{\textbf{Sokolov I.\,A.} see Pechinkin A.\,V.\hfill\hfill\hfill\hfill\hfill\hfill\hfill\hfill\hfill\hfill\hfill\hfill\hfill\hfill\hfill\hfill\hfill\hfill\hfill\hfill\hfill\hfill\hfill\hfill\hfill\hfill\hfill\hfill\hfill\hfill\hfill\hfill\hfill\hfill\hfill}{\ }
\contentsline {subsection}{\textbf{Sokolov I.\,A.} see Zakharov V.\,N.\hfill\hfill\hfill\hfill\hfill\hfill\hfill\hfill\hfill\hfill\hfill\hfill\hfill\hfill\hfill\hfill\hfill\hfill\hfill\hfill\hfill\hfill\hfill\hfill\hfill\hfill\hfill\hfill\hfill\hfill\hfill\hfill\hfill\hfill\hfill}{\ }
\contentsline {subsection}{\textbf{Stupnikov S.\,A.} see Zakharov V.\,N.\hfill\hfill\hfill\hfill\hfill\hfill\hfill\hfill\hfill\hfill\hfill\hfill\hfill\hfill\hfill\hfill\hfill\hfill\hfill\hfill\hfill\hfill\hfill\hfill\hfill\hfill\hfill\hfill\hfill\hfill\hfill\hfill\hfill\hfill\hfill}{\ }
\contentsline {subsection}{\textbf{Zakharov V.\,N., Kalinichenko L.\,A., Sokolov I.\,A., Stupnikov S.\,A.}\ \ Development of Canonical Information Models for Integrated Information Systems}{\qquad 2 \qquad 15} 
\contentsline {subsection}{\textbf{Zakharov V.\,N., Kozmidiady V.\,A.}\ \ Means Providing Applications Fault Tolerance}{\qquad 1 \qquad 14} 
\def\leftfootline{\small{\textbf{\thepage}
\hfill ИНФОРМАТИКА И ЕЁ ПРИМЕНЕНИЯ\ \ \ том~1\ \ \ выпуск~2\ \ \ 2007}
}%
 \def\rightfootline{\small{ИНФОРМАТИКА И ЕЁ ПРИМЕНЕНИЯ\ \ \ том~1\ \ \ выпуск~2\ \ \ 2007
 \hfill \textbf{\thepage}}}
 \label{end\stat}


%\tableofcontents


\end{document}