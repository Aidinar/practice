\def\stat{abstr}
{%\hrule\par
%\vskip 7pt % 7pt
\raggedleft\Large \bf%\baselineskip=3.2ex
A\,B\,S\,T\,R\,A\,C\,T\,S \vskip 17pt
    \hrule
    \par
\vskip 21pt plus 6pt minus 3pt }

\label{st\stat}

%\def\rightmark{\ }

%1
\def\tit{ALGORITHM FOR COMPUTATION CHARACTERISTICS OF~TELECOMMUNICATION NETWORK 
WITH~RESUBMITS AND~INCOMPLETE CIRCUIT BUFFER MANAGEMENT MODEL}

\def\aut{Ya.\,M.~Agalarov}

\def\auf{IPI RAN, agglar@ya.ru}

\def\leftkol{\ } % ENGLISH ABSTRACTS}
\def\rightkol{\ } %ENGLISH ABSTRACTS}

\titele{\tit}{\aut}{\auf}{\leftkol}{\rightkol}

%\vspace*{-2pt}

\noindent
Telecommunication network with SMQMA (Sharing with Maximum Queue Length and Minimum Allocation) 
nodes buffer management scheme and the ability to repeat the transmission from the source and 
transit nodes are discussed. The computation algorithm for average network 
characteristics such as probability of nodes blocking, total load in lines, average number 
of packets in nodes and network, average number of packets, waiting for repeat in sources
is suggested. Proofs of statements concerning the properties of the algorithm and the results 
of computational experiments are presented.



%\vspace*{-5pt}

\KWN{telecommunication networks; repeat of transmission; mechanisms of buffers management}

%\thispagestyle{myheadings}


\vskip 12pt plus 6pt minus 3pt

%2
\def\tit{ANALYSIS AND OPTIMIZATION PROBLEMS FOR~SOME USERS ACTITITY MODEL. 
PART~3.~EXTERNAL~RESOURCES OPTIMIZATION

}

\def\aut{A.\,V.~Bosov}

\def\auf{IPI RAN, AVBosov@ipiran.ru}


\def\leftkol{\ } % ENGLISH ABSTRACTS}

\def\rightkol{\ } %ENGLISH ABSTRACTS}

\titele{\tit}{\aut}{\auf}{\leftkol}{\rightkol}

%\vspace*{-2pt}

\noindent
This paper completes the mathematical model describing the activity of users and optimization problems 
for distribution of internal computational resources proposed by the author earlier. 
The optimization problem for  the
distribution of external resources is formulated and solved. 
Suboptimal optimization algorithms are proposed.

%\vspace*{-5pt}

\KWN{information systems; database management system; 
stochastic observation system; quadratic criterion}

\vskip 12pt plus 6pt minus 3pt


%3


\def\tit{ON STABILITY OF~NORMAL LOCATION MIXTURES WITH RESPECT TO~VARIATIONS 
IN~MIXING~DISTRIBUTION}

\def\aut{A.\,K.~Gorshenin}

\def\auf{IPI RAN, agorshenin@ipiran.ru}


\def\leftkol{\ } % ENGLISH ABSTRACTS}

\def\rightkol{\ } %ENGLISH ABSTRACTS}

\titele{\tit}{\aut}{\auf}{\leftkol}{\rightkol}

%\vspace*{-2pt}

\noindent
The stability of finite location mixtures of normal 
distributions with respect to parameter variations of the mixing distribution
is studied. The 
results are stated for the models of addition and splitting of components which are used 
to test statistical hypotheses about the number of mixture components.

%\vspace*{-5pt}

\KWN{location mixtures of normal distributions; L$\acute{\mbox{e}}$vy metric}


%\pagebreak

 \vskip 12pt plus 6pt minus 3pt

%\pagebreak

\def\leftkol{\ } % ENGLISH ABSTRACTS}
\def\rightkol{\ } %ENGLISH ABSTRACTS}

 %4
\def\tit{GEOSPATIAL INFORMATION PROCESSING ON~THE~BASE OF~GIS REPOSITORY}

\def\aut{S.\,K.~Dulin$^1$, I.\,N.~Rozenberg$^2$, and V.\,I.~Umansky$^3$}

\def\auf{$^1$IPI RAN, s.dulin@ccas.ru\\[1pt]
$^2$Research \& Design Institute for Information Technology, Signalling and Telecommunications 
on Railway\\ $\hphantom{^1}$Transport (JSC NIIAS),
I.Rozenberg@gismps.ru\\[1pt]
$^3$``IntechGeoTrans'' Close Corporation, umanvi@yandex.ru}



\titele{\tit}{\aut}{\auf}{\leftkol}{\rightkol}


\noindent
Effective integration of the data describing the interacting components is 
a key to successful management of the entire system of operating objects. 
Lack of data integration can lead to significant inefficiencies of operational, tactical, 
and long-term management strategies. The integrated management system can serve 
to overcome this inefficiency and improve coordination and cost-effectiveness of solutions. 
Data Integration is no doubt the most responsible action for the implementation of 
successful management strategies.
An approach to the use of the centralized repository of corporate data which allows 
to combine spatial and nonspatial data object descriptions is presented. 
Repository is the environment for data sharing and pooling of data and 
software products in action.


%\vspace*{-2pt}

\KWN{facilities management; data integration; GIS; repository}

 \vskip 12pt plus 6pt minus 3pt

%5
\def\tit{TECHNIQUE FOR MODELING THE~SERVER LOADING IN~OPEN CLOUD COMPUTING SYSTEMS}

\def\aut{D.\,V.~Zhevnerchuk$^1$ and A.\,V.~Nikolaev$^2$}

\def\auf{$^1$Tchaikovsky Technological Institute (branch) the Izhevsk State 
Technical University, drevnigeck@yandex.ru\\[1pt]
$^2$Tchaikovsky Technological Institute (branch) the Izhevsk State Technical University,
elodssa@yandex.ru}

%\def\leftkol{ENGLISH ABSTRACTS}
%\def\rightkol{ENGLISH ABSTRACTS}

\titele{\tit}{\aut}{\auf}{\leftkol}{\rightkol}

%\vspace*{-2pt}

\def\leftkol{ENGLISH ABSTRACTS}

\def\rightkol{ENGLISH ABSTRACTS}

\noindent
Planning resource of server system of cloud computing is a challenging task. 
Such factors as composition and parameters of the hardware platform, 
parameters of the system software that manages the execution of applications, 
the properties of traffic generated by users and determining the modes of operation of 
applications should be taken into account. 
Technique of estimating the server loading in open cloud computing systems 
is proposed. The technique is based on the analysis of processes of user interaction 
with the software. Reliability of the technique is justified. The 
results of modeling the load on the server side of management system simulation are presented.


%\vspace*{-2pt}

\KWN{cloud computing; imitation modeling; human--computer interaction}


 \vskip 12pt plus 6pt minus 3pt

%6
\def\tit{RUSSIAN ACADEMY OF~SCIENCES LIBRARIES TASKS AND~FUNCTIONS IN~MODERN CONDITIONS}

\def\aut{N.\,E.~Kalenov}

\def\auf{Library for Natural Sciences, Russian Academy of Sciences, nek@benran.ru}

%\def\leftkol{ENGLISH ABSTRACTS}
%\def\rightkol{ENGLISH ABSTRACTS}

\titele{\tit}{\aut}{\auf}{\leftkol}{\rightkol}

%\vspace*{-2pt}

\def\leftkol{ENGLISH ABSTRACTS}

\def\rightkol{ENGLISH ABSTRACTS}

\noindent
The problems of scientific libraries activities changes due to the development of 
network technology, the burgeoning growth of Internet available electronic publications 
and databases are being considered. It is shown that academic libraries are an integral 
part of the scientific infrastructure in the modern situation. The traditional tasks 
such as science information support remain the prerogative of the libraries, but they must 
be based on new solutions and performed with an extensive use of modern network technologies. 
New approaches to the traditional tasks are illustrated on the example of Library for Natural 
Sciences of the Russian Academy of Sciences (LNS RAS)
({\sf http://www.benran.ru}). In addition to the traditional tasks, 
academic libraries are ready for new challenges in the area of bibliometrics research, digitization of printed publication materials, 
etc.

%\vspace*{-2pt}

\KWN{academic libraries; automation; informatics; user service; information providing; 
computer technologies; digital libraries; automation workplaces; LNS RAS}

 \vskip 12pt plus 6pt minus 3pt

%7
\def\tit{UNIFICATION OF THE RULE-BASED SYSTEM LANGUAGES TO~PROVIDE 
INTEROPERABILITY OF~DECLARATIVE PROGRAMS}



\def\aut{L.\,A.~Kalinichenko$^1$ and S.\,A.~Stupnikov$^2$}

\def\auf{$^1$IPI RAN, leonidk@synth.ipi.ac.ru\\[1pt]
$^2$IPI RAN, ssa@ipi.ac.ru}

%\def\leftkol{ENGLISH ABSTRACTS}
%\def\rightkol{ENGLISH ABSTRACTS}

\titele{\tit}{\aut}{\auf}{\leftkol}{\rightkol}

%\vspace*{-2pt}

\def\leftkol{ENGLISH ABSTRACTS}

\def\rightkol{ENGLISH ABSTRACTS}

\noindent
The W3C standard RIF (Rule Interchange Format) that is provided for the interoperability 
of various rule based systems by the introduction of the extensible family of unified 
languages (dialects) oriented on creation of semantic preserving mappings of rule based 
languages of various systems into the dialects is analyzed. To characterize the motivation 
for the RIF project, a short survey of development and application of rule based languages 
and systems in the areas of knowledge representation, deductive databases, and
logical reasoning is made. Various semantics of logical rule based languages that influenced 
the RIF decisions are also analyzed. Main classes of application cases of the interoperable 
rule based programs used for development of the requirements for RIF are considered. Finally, 
the main decisions of the RIF project are overviewed.
 

%\vspace*{-2pt}

\KWN{language unification; language extensibility; logic programming systems; active rule systems;
production systems; knowledge representation; deductive databases; logical models of reasoning;
stratified semantics; stable model of a logic program; well-founded semantics;
RIF dialects; RIF Framework}

%\pagebreak

%8
\def\tit{COGNITIVE STUDY OF AN ASSISTIVE MULTIMODAL USER INTERFACE 
FOR~HANDS-FREE~HUMAN--COMPUTER INTERACTION}

\def\aut{A.\,A.~Karpov} 


\def\auf{St.\ Petersburg Institute for Informatics and Automation, 
Russian Academy of Sciences (SPIIRAS),\\ karpov@iias.spb.su}

%\def\leftkol{ENGLISH ABSTRACTS}
%\def\rightkol{ENGLISH ABSTRACTS}

\titele{\tit}{\aut}{\auf}{\leftkol}{\rightkol}

%\vspace*{-2pt}

\def\leftkol{ENGLISH ABSTRACTS}

\def\rightkol{ENGLISH ABSTRACTS}

\noindent
The process of research and development of a multimodal 
user interfaces aimed at hands-free interaction with a personal computer by means 
of the natural speech input and pointing gestures/movements by user's head is described. 
The proposed interface uses a low-cost audio-visual equipment for information input and 
provides an universal access to computer systems both for regular human-operators 
for contactless (without using hands) personal computer control, and for  handicapped 
users having difficulties with hand motion or even without arms. The methods 
and results of quantitative evaluation of speech and performance and contactless human--computer 
interaction are described and comparison with contact-based means of information input
is made.

%\vspace*{-2pt}

\KWN{multimodal user interface; automatic speech recognition; computer vision; cognitive research}


%9
\def\tit{A LOGIC OF~BIOGRAPHICAL FACTS}

\def\aut{N.\,A.~Markova} 


\def\auf{IPI RAN, nMarkova@ipiran.ru}

%\def\leftkol{ENGLISH ABSTRACTS}
%\def\rightkol{ENGLISH ABSTRACTS}

\titele{\tit}{\aut}{\auf}{\leftkol}{\rightkol}

%\vspace*{-2pt}

\def\leftkol{ENGLISH ABSTRACTS}

\def\rightkol{ENGLISH ABSTRACTS}

\noindent
A method for formalizing biographical facts in the form of logical formulas is proposed.
The method enables to integrate and analyze data obtained from heterogeneous sources and
allows to improve the efficiency of the reference-informational service of biographical resources.

%\vspace*{-2pt}

\KWN{biographical research; information retrieval; formalization; biographical fact}

 \vskip 12pt plus 6pt minus 3pt
 


%10
\def\tit{CALCULATION AND OPTIMIZATION OF~SOME CHARACTERISTICS 
OF~THE~MODEL FOR~COMPUTATIONAL COMPLEX}

\def\aut{I.\,V.~Pavlov}

\def\auf{Bauman Moscow State Technical University, ipavlov@mbstu.ru}


\def\leftkol{ENGLISH ABSTRACTS}

\def\rightkol{ENGLISH ABSTRACTS}

\titele{\tit}{\aut}{\auf}{\leftkol}{\rightkol}

%\vspace*{-2pt}

\noindent
The problem of optimal packet sizing while processing large volume information
is considered. In the model of a computer system, possible failures or faults of elements 
during tasks execution are taken into account. An asymptotic solution is obtained for 
the case of highly reliable components and small packets transfer time.
%\vspace*{-3pt}

\KWN{optimal packet size; reliability; failure rate; transfer time}
\pagebreak

%\vskip 12pt plus 6pt minus 3pt

%11
\def\tit{FUZZY VARIABLES AS~A~TOOL FOR~EXPRESSING ERROR CHARACTERISTICS IN~DATA
PROCESSING}

\def\aut{K.\,K.~Semenov}

\def\auf{St.\ Petersburg State Polytechnic University, semenov.k.k@gmail.com}


\def\leftkol{ENGLISH ABSTRACTS}

\def\rightkol{ENGLISH ABSTRACTS}

\titele{\tit}{\aut}{\auf}{\leftkol}{\rightkol}

%\vspace*{-2pt}

\noindent
The main application of fuzzy variables
theory is to take into account badly formalized information about
uncertainty sources. In this paper, the results of another application are
presented: to take into account well-formalized information about errors 
that is classical for metrology. The principal possibility of error 
characteristics representation
as fuzzy variables that is compatible with metrological norms and
standards has been shown.
 
%\vspace*{-2pt}

\KWN{fuzzy variables; error characteristics; measurement results}

%\pagebreak

\vskip 12pt plus 6pt minus 3pt

% \vskip 12pt plus 6pt minus 3pt

%12
\def\tit{SEMANTIC SEARCH OF NATURAL LANGUAGE INFORMATION ON~THE~BASIS 
OF~KNOWLEDGE BASE TECHNOLOGY}

\def\aut{M.\,M.~Sharnin$^1$ and I.\,P.~Kuznetsov$^2$}

\def\auf{$^1$IPI RAN, keywen1@mail.com\\[1pt]
$^2$IPI RAN, igor-kuz@mtu-net.ru}


\def\leftkol{ENGLISH ABSTRACTS}

\def\rightkol{ENGLISH ABSTRACTS}

\titele{\tit}{\aut}{\auf}{\leftkol}{\rightkol}

%\vspace*{-2pt}

\noindent 
A system for semantic search in the natural language texts is considered. 
The search is based on the use of a linguistic processor which analyzes 
the text in natural language and extracts from it the information objects (named entities), 
their signs, links, and participation in actions. As a result, the processor forms the 
semantic structures in the Knowledge Base. The structure of quiries is formed by analogy. 
The semantic search consists in comparison of such structures and finding the unknown objects. 
In the process, the connections between objects and their participation in the
actions are taken into account.

%\vspace*{-2pt}

\KWN{semantic search; semantics oriented linguistic processor; 
knowledge extraction from texts; Knowledge Base}

  \vskip 12pt plus 6pt minus 3pt
  
  %13
\def\tit{ON THE RATE OF CONVERGENCE TO~THE~NORMAL LAW OF~RISK ESTIMATE 
FOR~WAVELET COEFFICIENTS THRESHOLDING WHEN USING ROBUST VARIANCE ESTIMATES}

\def\aut{O.\,V.~Shestakov}

\def\auf{Department of Mathematical Statistics, Faculty of Computational Mathematics and Cybernetics,\\
M.\,V.~Lomonosov Moscow State University; IPI RAN, oshestakov@cs.msu.su}

\def\leftkol{ENGLISH ABSTRACTS}

\def\rightkol{ENGLISH ABSTRACTS}

\titele{\tit}{\aut}{\auf}{\leftkol}{\rightkol}

%\vspace*{-2pt}

\noindent
The asymptotic properties of risk estimate for thresholding wavelet coefficients 
of signal function are analyzed. Some estimates for the rate of convergence to the 
normal law are obtained.


 \label{end\stat}

%\vspace*{-2pt}

\KWN{wavelets; thresholding; risk estimate; normal distribution; rate of convergence}






\newpage