\def\stat{markova}

\def\tit{ЛОГИКА БИОГРАФИЧЕСКИХ ФАКТОВ}

\def\titkol{Логика биографических фактов}

\def\autkol{Н.\,А.~Маркова}
\def\aut{Н.\,А.~Маркова$^1$}

\titel{\tit}{\aut}{\autkol}{\titkol}

%{\renewcommand{\thefootnote}{\fnsymbol{footnote}}\footnotetext[1]
%{Работа выполнена при поддержке РФФИ (гранты 09-07-12098, 09-07-00212-а и
%09-07-00211-а) и Минобрнауки РФ (контракт №\,07.514.11.4001).}}


\renewcommand{\thefootnote}{\arabic{footnote}}
\footnotetext[1]{Институт проблем информатики Российской академии наук, nMarkova@ipiran.ru}

  
  \Abst{Предложен метод формализации биографических фактов в виде логических 
формул, который позволяет интегрировать и анализировать данные, получаемые из разных 
источников, а также служит основой повышения эффективности 
справочно-информационного аппарата биографических ресурсов.}
  
  \KW{биографическое исследование; информационный поиск; формализация; 
биографический факт}

\vskip 14pt plus 9pt minus 6pt

      \thispagestyle{headings}

      \begin{multicols}{2}

            \label{st\stat}
  
\section{Введение}
  
  Использование информационных технологий для проведения 
биографических исследований играет важную роль для удовлетворения 
объективно существующей общественной потребности в изучении и 
распространении сведений биографического характера. А~потребность эта 
растет по мере роста общественного интереса к различным вариантам 
<<частной>> истории, в том числе истории науки, краеведения, истории семьи.
  
  Колоссальный объем биографических сведе\-ний хранится в архивах или 
опубликован в труд\-но\-доступных изданиях. Массовая электронная пуб\-ли\-кация 
источников существенно повышает их\linebreak
 доступность, однако потенциал 
информационных технологий по обеспечению эффективности\linebreak
 биографического 
поиска далеко не исчерпан. Име\-ющи\-еся биографические ресурсы (БР) плохо 
сис\-те\-ма\-тизированы, их средства поиска неэффективны, данные противоречивы, 
а часто и недостоверны. Проб\-ле\-мы эти (подробно рассмотренные в работах~[1, 2]) 
усугубляются, когда в процессе исследования объединяются данные, 
получаемые из разных источников. Для того чтобы процесс исследования был 
эффективен, требуется методическая и инструментальная поддержка как со 
стороны спра\-воч\-но-ин\-фор\-ма\-ци\-он\-но\-го аппарата БР, так и со стороны 
организации и систематизации работы исследователя. 
  
  В работе предложен способ представления формализуемых биографических 
данных в виде логических формул, который, с одной стороны~---\linebreak
стороны 
исследователя, позволяет интегрировать данные, получаемые из разных 
источников, проверять непротиворечивость, интерполировать, корректно 
ставить новые исследовательские вопросы. С~другой стороны~--- стороны БР, 
может\linebreak служить концептуальной основой для <<стандартов операционной 
совместимости, метаданных, средств упорядочения информационного 
содержания, интерфейсов доступа к массивам данных в\linebreak
 цифровом формате, 
средств поиска и средств сохранения>>~--- достижения целей, определяемых 
программой ЮНЕСКО <<Информация для всех>>~[3].
  
  В разд.~2 уточняются задачи биографического поиска, 
являющиеся важным звеном в исторических исследованиях самой разной 
направленности. В~разд.~3 рассматриваются существующие модели 
биографических данных, предназначенные для систематизации и упорядочения 
сведений и обмена данными. В~разд.~4 предлагаются основные концепции 
новой модели, конкретизируемые в разд.~5 в виде формул логики 
биографических фактов. В~разд.~6 основные операции 
биографического поиска пред\-став\-ле\-ны в терминах формул логики 
биографических фактов.

\section{Задачи биографического поиска}
  
  Задача биографа~--- собрать и обобщить биографические факты, под 
которыми понимаются высказывания, являющиеся <<ответом на вопросы типа 
кто? что? когда?>>~\cite[с.~53]{4mar}, упорядочить их определенным образом, 
связать между собой и с внешними объектами. Задача обобщения~--- 
интеграции данных~--- включает анализ их непротиво\-ре\-чи\-вости и разрешение 
противоречий, интерполяцию, экстраполяцию, что может приводить к 
постановке новых исследовательских вопросов.
  
  Объект исследования биографического поиска~--- это человек или группа 
людей, принадлежащих к определенному кругу, информация о которых 
сохранилась в источниках. 

Предмет исследования биографического поиска~--- 
это конкретные биографические характеристики, отношения и события, 
связанные с изуча\-емы\-ми людьми. 

Биографический поиск~--- это задача, 
начинающаяся со слова <<найти>>, которая может быть как независимой, так и 
включаться в виде определенного этапа в те или иные исторические 
исследования. В~рамках предлагаемого рассмотрения в биографический поиск 
не включаются задачи при\-чин\-но-след\-ст\-вен\-но\-го анализа исторических явлений, 
теоретических обобщений и художественных построений.
  
  Эффективность биографических исследований во многом определяется 
доступностью информации, в том числе предоставляемыми БР возможностями 
поиска и навигации, а также наглядностью представления найденных данных. 
В~рамках традиционных бумажных БР задача повышения доступности решается 
библиографами, архивистами и редакторами. Сведения об основных лицах, 
относящихся к документам, содержатся (во всяком случае, должны 
содержаться) в каталогах библиотек и описях архивов. Сведения о 
многочисленных\linebreak
 лицах, упоминаемых в монографиях,~--- в соответствующих 
именных указателях. Библиографические справочники сопоставляют именам 
(со\-про\-вож\-да\-емым, возможно, краткими сведениями о\linebreak
 лицах) указания на 
источники, в которых может быть найде\-на соответствующая информация. 
  
  Если направление деятельности библиографа~--- от источника к персонажам, то 
исследователь идет от персонажа к источникам и фактам и от объекта или 
явления~--- к персонажам. В~качестве объекта или явления могут выступать 
как организации, общества, исторические события, так и лица, связи с 
которыми изучаются (<<корреспонденты Гоголя>>, <<учителя Пушкина>>, 
<<ученики Ключевского>>). Аналогичные задачи стоят в рамках изучения 
истории любой сферы деятельности, отрасли, организации, края. В~задачах 
генеалогии (или шире~--- истории семьи) изучаются как объекты <<род>>, 
<<семья>>, так и отдельные лица. 
  
  Отметим взаимосвязь описываемых сторон: для описания источников 
требуется выполнить задачу биографического поиска, а публикация 
результатов исследований приводит к созданию нового источника.
  
  Биографический поиск~--- это многошаговая процедура. На основании 
исходных данных ставится некий исследовательский вопрос, ищутся 
источники, анализируются найденные документы, выявляются факты (или 
констатируется, что источник не содержит релевантных фактов), которые затем 
сопоставляются и интегрируются с ранее обнаруженными. На основе новой 
информации формулируются новые вопросы и~т.\,д. Многие задачи и 
проблемы биографического поиска инвариантны по отношению к виду 
исследований.

\section{Существующие биографические модели}

\vspace*{-6pt}
  
  Определить в общем случае, что должно входить в биографию, невозможно, 
однако в рамках проб\-лем\-ных областей, где возможна частичная формализация, 
такая задача имеет практические решения: от рекомендаций по подготовке 
повествовательных текстов до строгих стандартов представления данных. 
Рассмотрим основные виды биографических моделей, отмечая их достоинства 
и недостатки с точки зрения задач биографического поиска.
  
  Упорядочение работы авторов биографического словаря определяется 
наличием рекомендаций по содержанию и форме представления данных. 
Например, грандиозная работа по созданию биографического словаря русских 
фольклористов методически обеспечена монументальным трудом~\cite{5mar}, 
в котором наряду со специфическими рекомендациями есть и правила 
представления общих биографических и связанных с ними библиографических 
сведений. Такого рода БР рассчитаны на бумажную публикацию, в них нет 
учета возможностей информационных технологий.
  
  Расширение горизонта рассмотрения от конкретной сферы деятельности до 
всех значимых в истории лиц вместе с освоением новых технологических 
возможностей (\textit{wiki}), с одной стороны, и подключением широчайшего 
круга авторов, с другой, демонстрирует портал Персоналии в 
Википедии~\cite{6mar}. Методическим обеспечением для ав\-то\-ров-со\-ста\-ви\-те\-лей 
статей служат шаблоны~--- своего рода биографические модели, каждая из 
которых ориентирована на некоторый круг лиц. Например, шаблон 
<<Ученый>> представляет собой список из двух десятков анкетных статей, 
каждая из которых раскрывается отдельным шаблоном-регламентом. 
Возможность ссылаться (гиперссылками) на другие статьи, относящиеся как к 
людям, так и к другим объектам, а также к доступным в сети источникам~--- 
колоссальное преимущество Википедии. Часть биографических данных можно 
почерпнуть в связанных статьях. \textit{Wiki}-технология дает возможность 
совместно представлять формализованное и неформализованное знание. При 
этом формализация (в отличие от реляционных баз данных) может\linebreak 
осуществляться по ходу накопления данных: подключением новых или 
изменением старых шаб\-ло\-нов. Несовершенства Википедии, возможно, 
час\-тич\-но будут сниматься по мере ее развития.\linebreak
 Отметим некоторые из них. 

Связь между статьями дается простой гиперссылкой, целесообразно расширить 
ее некоторым\linebreak\vspace*{-12pt}

\pagebreak

\noindent
семантическим содержанием (пометкой <<друг>>,
 <<отец>>), 
что уже предполагают современные микроформаты~\cite{7mar}. Другое 
необходимое расширение, которое, к сожалению, пока даже не декларируют 
создатели семантического \textit{Web},~--- это определение динамики связей. 
Привязка событий и связей в жизни человека к временн$\acute{\mbox{о}}$й оси~--- важнейшее 
условие успешности биографического поиска.
  
  Важным шагом в сторону эффективного представления биографических 
данных является новая редакция Российского коммуникативного формата 
представления авторитетных данных в машиночитаемой форме 
(RUSMARC)~\cite{8mar}. 

Для лиц, причастных к созданию документов или 
упоми\-на\-емых в них, записи RUSMARC определяют имя и идентифицирующие 
признаки (даты жизни, специальность, область деятельности, титулы, звания, 
степени и~т.\,п.). Аналогичные сведения полагаются для объектов 
библиографического описания <<Род>> (семья) и <<Организация>>. 
Предполагается фиксация связей по родству, работе, культурной общности, 
местожительству и~т.\,п. Как и Википедия, RUSMARC позволяет 
фиксировать связи между людьми и объектами другой природы. И~эти связи 
также, к сожалению, не помечаются хронологическими рамками. Большое 
внимание в RUSMARC уделяется вариативности в именованиях лиц, 
организаций, документов~--- типичной причины проблем биографического 
поиска. 
  
  Основная сложность внедрения RUSMARC~--- отсутствие необходимого 
числа библиографов, которые могли бы заполнить соответствующие записи, в 
то время как успех Википедии во многом определя\-ет\-ся подключением 
широчайшего круга лиц к созданию, проверке, редактированию\linebreak статей. 
  
  Стандартом де-фак\-то для представления 
биографической информации на протяжении долгих лет является давно 
устаревшая модель Ge\-ne\-a\-logi\-cal 
Data Communications (\mbox{GEDCOM}), революционное, основанное на \textit{xml} 
обновление которой \mbox{GEDCOM}~6.0~\cite{9mar} было выпущено в 2002~г., но 
до сих пор фактически никем не используется. Конкурентом \mbox{GEDCOM}, тоже 
определяемым как спецификация формата обмена данными между 
генеалогическими программами, является стандарт GenXML~\cite{10mar}. 
Помимо более строгой структурной упорядоченности он несет в себе 
несколько принципиально новых положений, отражающих практику 
биографических исследований. В~частности, в нем явным образом 
определяется процесс исследования: введены понятия <<свидетельство>> и 
<<заключение>>. Но главное, GenXML открыт для добавления новых типов 
атрибутов и событий. К~сожалению, ни в GEDCOM, ни в GenXML не 
отражены важнейшие свойства биографической информации: временн$\acute{\mbox{а}}$я 
изменчивость и взаимная зависимость характеристик. 
  
  В рамках просопографических исследований~\cite{11mar} задача построения 
обобщенной модели и не ставится. Каждое исследование предполагает свою 
проблемно-ориентированную информационную модель. 
  
  Таким образом, в настоящее время концептуальных моделей, в полном 
объеме отражающих специфику биографических исследований, не существует.

\section{Концептуальная модель биографических данных} 
  
  Построим концептуальную модель, описы\-ва\-ющую биографические данные, 
для которых возмож\-но формализованное представление. Постараемся учесть 
все недочеты и достоинства существующих моделей. Дадим общее 
неформальное описание проблемной области биографических исследований.
  
  Человек рождается, умирает, действует сам или подвергается воздействиям 
окружения в исторической реальности. Некоторая часть сведений о нем 
фиксируется документально и попадает в информационное пространство 
(ИП)~--- на бумажные и другие твердые носители, а в последние десятилетия и 
в электронные ресурсы. Информационное пространство 
является компонентом исторической реальности. 
Выделенные компоненты ИП~--- биографии конкретных лиц~--- существенно 
различаются по объему: от нескольких слов до нескольких томов. Кроме них 
биографические сведения рассыпаны по ИП~--- они содержатся в биографиях 
лиц из круга общения, в исторических описаниях событий и явлений, в 
документах учета, в библиографических списках и~т.\,д. Задача 
биографического поиска~--- собрать рассыпанные в ИП данные, касающиеся 
конкретных лиц, конкретных событий, объектов, явлений.
  
  Уточним вводимые понятия и термины.

\subsection{Историческая реальность}
  
  С биографической точки зрения объектами исторической реальности 
  ($b$-объектами) являются:
  \begin{itemize}
\item люди (персонажи, лица);
\item общественные образования (государства, учреж\-де\-ния, 
общества и др.);
\item физические объекты (географические, технические, естественные);
\item исторические события и процессы; 
\item отрасли деятельности.
\end{itemize}

  Между $b$-объектами существует объективно или могут быть определены в 
рамках той или иной интерпретации различные отношения ($b$-отношения). 
Объективны биологические $b$-отношения, например отношение 
  <<ребенок--родители>>. Влияние на творчество писателя произведений его 
пред\-ше\-ст\-вен\-ни\-ка~--- пример субъективно интер\-пре\-ти\-ру\-емо\-го $b$-отношения.
  
  Как выделение $b$-объектов, так и определение их свойств и $b$-отношений 
является результатом абстрагирования, необходимого для целей сбора и 
сис\-те\-ма\-ти\-за\-ции данных. Будем рассматривать только данные, для которых 
возможно формализованное представление. Интерпретация тонких вопросов, 
связанных с психологией, этикой, мировоззрением, творчеством,~--- задача 
соответствующих профессионалов.
  
  Все $b$-объекты, а также большинство $b$-от\-но\-ше\-ний существуют и 
изменяются во времени. Собственно <<биографией>> является 
информационный объект, в котором в динамике или интегрально\linebreak представлены 
свойства и характеристики некоторого лица, а также его $b$-отношения с 
другими\linebreak
 $b$-объек\-та\-ми и в какой-то мере их свойства и характеристики. 
Характеристики $b$-объек\-та или $b$-от\-но\-ше\-ния будем называть 
  $b$-ха\-рак\-те\-ри\-сти\-ка\-ми. Личными $b$-ха\-рак\-те\-ри\-сти\-ка\-ми являются 
составляющие генотипа и фенотипа человека, в частности состояние здоровья. 
Гражданское состояние, имущественное состояние, сословие, чин, звание, сан, 
титул характеризуют не только человека, но и соответствующую социально-правовую 
организацию общества, а точнее~--- $b$-от\-но\-ше\-ние между ними.
  
  Примером $b$-отношения, на первый взгляд не зависящего от исторического 
контекста, является отношение местопребывания, связывающее человека и 
объект географического пространства. Формально в конкретный момент 
времени местопребывание может быть охарактеризовано координатами. 
Однако на практике оно определяется в терминах названий населенных 
пунктов, сопоставление которых с координатами~--- не всегда тривиальная 
задача исторической географии.
  
  Важнейшая $b$-характеристика~--- официальное именование~--- 
определяется на отношении лица и $b$-объекта-государства. 
Другие $b$-объекты могут использовать другие имена данного лица, в част\-ности в 
домашнем обращении или в отношении авторства (псевдонимы).
  
  Между $b$-характеристиками существуют зависимости, регламентируемые 
законами природы или нормативными законами: правилами, юридическими 
актами, традициями. Примерами регламентов являются законодательные 
документы, уставы обществ, штатное расписание учреждения и~т.\,п. 
Существуют закономерности, определяющие до\-пус\-ти\-мые последовательности 
событий~--- смены значений $b$-характеристик, задающие некий шаб\-лон, 
сценарий или набор ограничений. Перечислим несколько очевидных: смерть 
следует за рож\-де\-ни\-ем; имеются допустимые пределы разницы между 
рождением человека и границами жизни его родителей; в каждый конкретный 
момент времени человек может находиться только в одной точке 
географического пространства. Для определения большинства нормативных 
законов требуется конкретно-историческое знание.

\subsection{Информационное пространство}
  
  Информационное пространство без ограничения общности можно 
представить как совокупность хранилищ документов. Для пользователей 
электронных хранилищ, реализованных, возможно, в виде баз данных, их 
содержание пред\-став\-ля\-ют виртуальные документы, визуализируемые на 
экране. Документы, в свою очередь, делятся на час\-ти/фрагменты: разделы, 
страницы, абзацы и~т.\,п.\linebreak
 Фрагмент документа редко независим, для его\linebreak 
корректной интерпретации требуется контекст. Элементы ИП также являются 
объектами исторической реальности: у них есть время жизни, они связаны с 
людьми отношениями <<автор>>, <<адресат>>, <<упоминаемое лицо>>.
  
  Документы, их фрагменты или их совокупности~--- хранилища, и их разделы 
идентифицируются адресами. Для электронных хранилищ адрес~--- это 
URL/URI (Uniform Resource Locator/Identifier), ключи базы данных, имена закладок и~т.\,п. Для архивных 
хранилищ~--- номера фонда, описи, дела, листа. Сложнее дело обстоит с 
печатными изданиями. Библиографическая ссылка, вообще говоря, не является 
адресом~--- книгу еще требуется \mbox{найти} в хранилище-библиотеке, где адресом 
ее будут служить соответствующие шиф\-ры хранения. Возможно, однако, что 
книга уже оциф\-ро\-ва\-на, тогда адрес ее~--- тот же URL.
  
  Идеальное решение задач биографического поиска состояло бы во всеобщем 
справочно-ин\-фор\-ма\-ци\-он\-ном аппарате, в котором $b$-объекты были бы 
соотнесены с элементами ИП. На базе такого аппарата, используя возможности 
современных информационных технологий, удалось бы добиться качественно 
нового уровня эффективности биографических исследований. На пути к этому 
идеалу стоят сложнейшие задачи.
  
  Прежде всего, задача описания существующих источников (как бумажных, 
так и электронных)~--- идентификации взаимоотношений между элементами 
ИП и $b$-объектами, их свойствами, хронологией~--- далека от реализации. 
<<Концепция информатизации архивного дела России>>~\cite{12mar}, 
декларируя необходимость обеспечения прав граждан на информацию, на 
самом деле только намечает подходы к решению задач научного описания 
архивных материалов. В~практике отечественных архивов электронные 
описания в основном присутствуют разве что на уровне их крупных единиц~--- 
фондов. 
  
  Существуют два аспекта задачи описания источников: систематизация и 
наполнение. В~части наполнения многое могли бы сделать пользователи, 
читатели, исследователи. Примером деятельности по обмену биографическими 
ссылками (и фактографическими данными) может служить сайт 
ВГД~\cite{13mar}. Пользователь, обладающий доступом к труднодоступному 
источнику, выкладывает его описание на общедоступный ресурс. Но эта 
коммуникация ведется бессистемно: в виде обмена текстовыми сообщениями 
на форуме сайта.
  
  Необходимым условием создания справочно-ин\-фор\-ма\-ци\-он\-но\-го аппарата, 
соотносящего элементы ИП с $b$-объектами, является систематизация их 
описаний, сведение формализуемой части касающихся их сведений в единый 
формат метаданных. 

\section{Биографические характеристики и~логика~фактов}
     
     Для того чтобы иметь возможность сопоставить факты, содержащиеся в 
источниках, оценить их 
не-\linebreak противоречивость, сделать выводы, необходимо 
привести их к некоторому общему, нормализованному представлению. 
Представим биографические сведения в виде совокупности взаимосвязанных 
значений $b$-характеристик на общей временн$\acute{\mbox{о}}$й оси. 

\subsection{Биографические характеристики}
  
  Существенная часть $b$-характеристик формализуема, их можно измерить, 
например, в терминах социологической~\cite{14mar} или 
психологической~\cite{15mar}\linebreak
стратификации. Значения $b$-характеристик~---\linebreak 
статусы (атрибуты), как правило, изменяются во времени. Например, 
отношение сотрудника и учреж\-де\-ния характеризует должность, которая 
изменяется по мере карьерного роста. В~некоторых случаях новое значение не 
заменяет предыдущее, а присоединяется к списку ранее имевшихся 
компонентов значения (пример~--- награды). Ряд $b$-ха\-рак\-те\-ри\-стик имеет 
простые количественные значения, например рост и вес. Артериальное 
давление измеряется парой чисел.
  
  Значения $b$-характеристик в именных шкалах, где присутствуют 
синонимы, многозначны. Существует вариативность именования лиц и 
организаций, даже если речь идет о конкретном моменте времени. Например, 
<<собор Покрова Пресвятой Богородицы, что на Рву>> эквивалентен <<собору 
Василия Блаженного>>. Причиной многозначности могут быть также те или 
иные варианты искажений, а также в целом субъективный характер оценок (про 
рост профессора Ловецкого Герцен утверждал: <<был высокий\ldots\ 
мужчина>>, а Пирогов~--- <<небольшого роста>>).
  
  $B$-характеристика~--- это отображение (в общем случае многозначная 
функция), динамически сопоставляющее $b$-объек\-ту или паре $b$-объек\-тов 
некоторое значение. Область определения $b$-ха\-рак\-те\-рис\-ти\-ки соответствует 
некоторому $b$-от\-но\-шению.
  
  Одно и то же $b$-отношение может быть оценено разными, но 
взаимозависимыми $b$-ха\-рак\-те\-ри\-сти\-ка\-ми в разных шкалах, например рост 
измеряется в саженях, футах или сантиметрах (или же ему дается неформальная 
словесная оценка). Кроме того, между значениями различных 
  $b$-ха\-рак\-те\-ристик связанных между собой $b$-объек\-тов существует 
взаимосвязь, в частности, являясь сотрудником подразделения, человек 
автоматически является и сотрудником учреждения в целом. С~другой 
стороны, являясь сотрудником учреждения, человек работает в некотором 
подразделении, о котором, если оно неизвестно, может быть поставлен 
исследовательский вопрос.
  
  Как области определения, так и области значений для подавляющего 
большинства $b$-ха\-рак\-те\-ри\-стик меняются во времени, само наличие 
  $b$-ха\-рак\-те\-ри\-сти\-ки ограничено определенными временн$\acute{\mbox{ы}}$ми рамками, для 
конкретной исторической ситуации их определяет научное знание 
соответствующей специальной исторической дис\-цип\-ли\-ны, а также социологии, 
антропологии, психологии, биологии. Они же определяют зависимости между 
$b$-ха\-рак\-те\-ри\-сти\-ка\-ми, варианты возможной синонимии значений и другие 
общие для рассматриваемых классов $b$-объек\-тов закономерности. 
Формулировки соответствующих законов природы, нормативных актов, 
традиций в виде зависимостей между значениями $b$-ха\-рак\-те\-ри\-стик будем 
называть $b$-нор\-ма\-ля\-ми. Лишь в редких случаях $b$-нор\-маль может быть 
сформулирована формально, большинство из них задается неформальными 
текстами, проверка соблюдения их правил~--- <<ручная>> процедура. 

\subsection{Нормализованный факт}
 
  Предложим формализацию понятия $b$-ха\-рак\-те\-ри\-сти\-ки. Высказывание, 
фиксирующее, что в данный момент данная $b$-характеристика для данного 
  $b$-объекта (объектов) имела данное значение, назовем формализованным 
фактом. Такое выражение может иметь логическое значение~--- ИСТИНА или 
ЛОЖЬ, быть неизвестным, а может быть оценено неким промежуточным 
образом: <<Скорее, ИСТИНА, чем ЛОЖЬ>>. Если информация об 
интересующем $b$-объекте или группе объектов будет представлена в виде 
набора формализованных фактов, тогда, применяя соответствующие 
  $b$-нормали, можно их сопоставить, выявить и разрешить противоречия, 
сформировать новые факты-следствия, интегрировать данные в общую 
картину.
  
  Большинство $b$-характеристик сохраняет свои значения на протяжении 
некоторого промежутка времени.  Чтобы отразить это 
фундаментальное свойство исторической реальности, введем понятие 
нормализованный факт (НФ). Назовем нормализованным фактом  следующую 
логическую формулу:
  \begin{equation}
  \left( \forall t\in \Delta t\right) \beta(p,q,t)=a\,.
  \label{e1mar}
  \end{equation}
  Здесь 
  $\beta$~--- $b$-характеристика;
  $p$ и $q$~--- $b$-объекты;
  $a\hm\in  \mathrm{Im}\left(\beta\right)$~--- конкретное значение $b$-характеристики из области 
ее значений;
  $t$~--- время;
  $\Delta t$~--- период времени, когда $b$-характеристика неизменна.
  
  Компоненты НФ полагаем некоторым информационным представлением 
соответствующих сущностей, например в виде идентификаторов, текстов, 
чисел.
  
  В формуле~(\ref{e1mar}) представлена характеристика двуместного 
  $b$-от\-но\-ше\-ния. Для одноместных $b$-от\-но\-ше\-ний будем использовать 
нотацию~$\beta (p, t)$. Трехместные отношения, а также отношения большей 
местности с помощью логических формул могут быть сведены к двуместным.
  
  Если $b$-характеристика принимает логическое значение, будем опускать 
его значение в записи, полагая $\beta (p, q, t)$ эквивалентным 
$$\beta (p, q, t) = \mbox{ИСТИНА}\,.
$$ 

Наконец, для краткости в формуле~(\ref{e1mar}) 
будем опускать время или использовать нотацию 
$$
\beta (p, q, \Delta t) = a\,.
$$ 

\subsection{Формулы логики фактов}
  
  Расширим понятие НФ для различных предикатов. Помимо <<$=$>> в 
формуле~(\ref{e1mar}) будем использовать <<$\not=$>>; <<$<$>> и 
  <<$>$>>~--- для упорядоченных значений\linebreak
   $b$-ха\-рак\-те\-ри\-стик; а также 
<<$\in$>> и <<$\not\in$>>~--- если в правой части не единичное значение, а 
множество. Кроме того, правая часть может представлять значение 
  $b$-характеристики для другого $b$-объекта, что соответствует текстам: 
<<был одноклассником>>, <<служил в той же должности>>, <<был старше 
чином>>.
  
  Нормализованные факты связаны друг с другом. Формально эти связи представимы в виде 
логических формул. Назовем их формулами логики фактов~--- FF 
(\textit{Fact Formula}). Для сигнатуры формул логики фактов применимы как 
выражения обычной логики предикатов, так и специальные темпоральные 
аппараты. В~любом случае FF включает пропозициональные связки ($\lnot$, 
$\neq$, $\vee$, $\rightarrow$) и кванторы ($\forall$, $\exists$). 
  $B$-характеристики выступают в роли функций, а в качестве предикатов 
используются $b$-характеристики с логическим значением или оценки 
значений $b$-характеристик ($=$, $\not=$, $\in$, $\not\in$, $<$, $>$,\ \ldots).
  
  Дадим индуктивное определение формулы логики фактов~--- FF в 
терминах логики предикатов.
  
  Переменными и константами являются $b$-объ\-ек\-ты $(p, q)$, элементы и 
подмножества из множеств значений $b$-характеристик ($a\hm\in \mathrm{Im}\left(\beta\right)$, 
$A\hm\subset \mathrm{Im}\left(\beta\right)$) и время ($t$), а также различные варианты 
временн$\acute{\mbox{ы}}$х периодов (отрезок, интервал, полуинтервал):
  \begin{align*}
\mathrm{Term}&::= \beta (p, q, t) \vert a \vert  A\vert t \vert \\
& \vert [t_1 - t_2] \vert (t_1 - t_2) \vert 
[t_1 - t_2) \vert (t_1 - t_2]\,;\\
 \mathrm{Atom}&::= \mathrm{Term}\ \rho\  \mathrm{Term} \ \left(\rho\in \{=, \not=, \in, \not\in, 
 <,  >,\ldots\}\right)\\
\mathrm{FF}&::=\mathrm{Atom} \vert \neg \mathrm{FF} \vert\\
  &\vert \mathrm{FF}_1 \wedge 
  \mathrm{FF}_2\vert  \mathrm{FF}_1 \vee \mathrm{FF}_2 \vert \forall x\ 
\mathrm{FF} \vert  \exists x\ \mathrm{FF}\,.
%  \label{e2mar}
  \end{align*}
  
  Приведем формулировки некоторых $b$-нор\-ма\-лей в терминах логики 
фактов.
  \smallskip
  
  \textbf{Симметрия.} Для большинства двуместных $b$-от\-но\-ше\-ний 
значению  
$\beta (p, q, t)$ однозначно соответствует $\beta^\prime (q, p, t)$: 

\smallskip
   
\textit{МестоРаботы}(<<Иванов>>, 
<<Контора>>)\;$\leftrightarrow$\\[-9pt]

\hspace*{10mm}$\leftrightarrow$\;\textit{Сотрудник}(<<Контора>>,  <<Иванов>>)\,.
  
  \medskip
  
  \textbf{Транзитивность.} Факты, основанные на таких\linebreak
   $b$-ха\-рак\-теристиках, 
как иерархия и местоположение, обладают свойством транзитивности. 
  %
  \noindent
  Например: 
  
  \smallskip
  
  \noindent
  \textit{МестоРаботы}(<<Иванов>>, 
<<Контора>>)\;$\wedge$\\[-9pt]

$\wedge$\;\textit{Местопребывание}(<<Контора>>, 
<<Москва>>)\;$\rightarrow$\\[-9pt]

\hspace*{5mm}$\rightarrow$\;\textit{Местопребывание}(<<Иванов>>, <<Москва>>).

\smallskip
  
 
 \textbf{Вариативность.} Для вариативных $b$-ха\-рак\-те\-ри\-стик, например 
именования, значения разбиваются на классы эквивалентности, определяемые 
$b$-нор\-ма\-ля\-ми, а формула принимает вид принадлежности данному классу. 
Приведем $b$-нор\-маль для имено-\linebreak\vspace*{-12pt}

\pagebreak

\noindent
вания в современной отечественной практике 
(без учета знаков препинания и грамматических форм, в нестрогой форме):

%\pagebreak

\smallskip

\noindent
Имя($x$) = <<Имя>>\;$\wedge$\; %\\

\noindent
\ \ \ $\wedge$\;Фамилия($x$)\;=\;<<Фамилия>>\;$\wedge$ %\\[-9pt]

\noindent
\ \ \ $\wedge$\;Отчество($x$)\;=\;<<Отчество>>\;$\rightarrow$ %\\
 %\\[-9pt]

\noindent
\ \ \ $\rightarrow$\;Именование($x$)\;=\;<<Фамилия\ Имя\ Отчество>>\;$\vee$ %\\[-9pt]

\noindent
\ \ \ $\vee$\;<<Имя\  Отчество\ Фамилия>>\;$\vee$ %\\[-9pt]

\noindent
\ \ \ $\vee$\;<<Фамилия Имя>>\;$\vee$ %\\

\noindent
\ \ \ $\vee$\;<<Имя Фамилия>>\;$\vee$ %\\[-9pt]

\noindent
\ \ \ $\vee$\;<<Имя\ Отчество>>\;$\vee$ %\\

\noindent
\ \ \ $\vee$\;<<И\ О\ Фамилия>>\;$\vee$ %\\[-9pt]

\noindent
\ \ \ $\vee$\;<<Фамилия\ И\ О>>\;$\vee$

\noindent
\ \ \ $\vee$\;<<И\ Фамилия>>\;$\vee$

\noindent
\ \ \ $\vee$\;<<Фамилия\ И>>\,. 

%\smallskip

\subsection{Виды нормализованных фактов}
  
  Рассмотрим два направления классификации НФ: по динамике и по 
определенности.
  \begin{enumerate}[1.]
\item Нормализованные факты представляют различные варианты динамики значений  
$b$-ха\-рак\-те\-ристик:
\begin{itemize}
\item НФ-событие: период соответствует <<мгновению>>: $\Delta 
t\hm=\{t^\prime\}$;
\item НФ-состояние: период протяжен~--- $\Delta t\hm= [t_s, t_b)$, он 
включает начало и не включает конец временн$\acute{\mbox{о}}$го промежутка;
\item цепочка НФ~--- процесс, дизъюнкция формул отдельных состояний: 
$\vee_{i\in [0,\,n-1]} \beta(p,q,[t_i,\,t_{i+1}))$~--- в моменты~$t_i$ 
$b$-характеристика принимает новое значение.
\end{itemize}

  Факт-состояние~--- это элементарный процесс, включающий событие~--- 
переход в данное состояние и ограниченный событием~--- выходом из 
данного состояния. С~другой стороны, интегральное состояние может 
уточняться детальным процессом. 
\item Нормализованные факты представляют раз\-лич\-ные варианты определенности значений\linebreak 
$b$-ха\-рак\-те\-ристик:
\begin{itemize}
\item рамочный НФ~--- факт, часть компонентов которого не определена;
\item строгий НФ~--- факт, для которого все компоненты определены.
  \end{itemize}
  
  Под компонентами факта понимаются время, характеризуемые объекты и 
значения $b$-ха\-рак\-те\-ри\-стики.
  
  Заметим, что строгий факт совсем не обязательно корректен. Произвольная 
формула логики фактов является строгой, если все ее компоненты~--- строгие 
факты, и рамочной во всех других случаях.
  \end{enumerate}
  
  Неполные и неточные данные, оформленные в виде рамочных НФ и 
основанных на них рамочных формул, пусть с пропусками, размытыми 
значениями, неформальными комментариями при накоплении, систематизации, 
анализе способны стать основой для реконструкции биографий.

%\vspace*{-6pt}

\section{Биографический поиск в~терминах логики фактов}
  
  Рамочные НФ представляют удобный механизм для формулировки 
исследовательских вопросов, ответы на которые, возможно, хранятся в 
до\-ку\-мен\-тах-ис\-точ\-ни\-ках. Как интерпретация источников, так и выводы из 
имеющихся фактов, как правило, являются неформальными процедурами. Для 
их выполнения требуется экспертное знание. Тем не менее сам факт 
выполнения операции, ее вход и выход удобно представить как специальные 
формулы логики фактов. Благодаря формализованному представлению ручных 
операций сведения о целях исследования, его текущем состоянии и дальнейших 
шагах складываются в единую картину. 

%\vspace*{-6pt}

\subsection{Формулировка исследовательских вопросов}
  
  Рамочный НФ связан с исследовательским вопросом, касающимся уточнения 
значений его компонентов. Рассмотрим основные варианты таких вопросов.
  \smallskip
  
  \textbf{Хронологические рамки.} Важнейший вопрос биографического 
поиска~--- <<когда?>>. Например, $\mbox{Жизнь}(\mbox{<<Иванов>>}, [t_1 - t_2))$, где 
$t_1$ и~$t_2$ неизвестны. При его постановке, как правило, существуют 
ориентиры, оценки: начало не ранее, конец не позднее~--- или известен некоторый 
промежуток, входящий в искомый. 
  
  \smallskip
  \textbf{Рамки значений.} Исследовательский вопрос состоит в выяснении, 
какое значение принимала данная $b$-характеристика для данного лица в 
данное время. Пример: 

\vspace*{-6pt}

\noindent
\begin{multline*}
\mbox{\textit{СотрудникДолжность}}(\mbox{<<Контора>>},\\ 
\mbox{<<Иванов>>}, \mbox{1890}) \hm= x.
\end{multline*}

  
\noindent
 Область значений: 
 
 \noindent
$$
x\in \mbox{\textit{СписокДолжностей}}(\mbox{<<Контора>>}, 
\mbox{1890})\,.$$ 
  
  Уточнение рамок возможно, если известны значения $b$-ха\-рак\-те\-ри\-сти\-ки в 
некоторые моменты до и после данного времени и, кроме того, ее значения 
упорядочены или известно значение другой $b$-ха\-рак\-те\-ри\-сти\-ки, а между 
  $b$-характеристиками существует зависимость, определяемая $b$-нормалью.
  
  \smallskip
  \textbf{Объектные рамки.} Неопределенность этого рода~--- ответ на 
вопросы <<кто?>>~--- $\mbox{\textit{Муж}}(x, \mbox{<<Петров>>},$ $1909)$ или <<что?>>~--- 
$\mbox{\textit{МестоРаботы}}(\mbox{<<Иванов>>}, y,$ $1890)$. Запись в анкете <<женат>>, 
предполагает наличие $b$-объек\-та~$x$~--- жены, про которую известно лишь 
то, что в указанное время она была замужем за заполнителем анкеты. 
Собственно поиск объекта сводится к поиску его $b$-ха\-рак\-те\-ри\-стик, по 
крайней мере, идентификационных (именования, времени жизни). Какие-то 
ограничения на именование и время жизни содержат уже имеющиеся данные о 
связываемом $b$-объекте. 

%\vspace*{-6pt}

\subsection{Интерпретация текстов источника}
  
  Лишь незначительная часть ИП представляет формализованные факты в 
явном виде. Это в основном фактографические (генеалогические или 
просопографические) базы данных, в которых $b$-ха\-рак\-те\-ри\-сти\-ка 
представляется доменом (колонкой в таблице), а шкала ее значений, возможно, 
выделена в отдельную вспомогательную таб\-ли\-цу-сло\-варь. 
  
  Там, где документ-источник осмысленно размечен, например в анкетных 
группах~--- шаблонах Википедии, формализованные факты представлены 
метаданными. 
  
  Кроме того, несложно выявить формализованные факты, исходя из:
  \begin{itemize}
\item структурной организацией документа-ис\-точ\-ни\-ка, отвечающей 
структуре $b$-объектов;
\item в той или иной степени структурированных текстовых определений, 
сопоставляющих именам $b$-объектов имена связанных с ними $b$-объек\-тов 
или $b$-характеристики;
\item идеографических схем (пример~--- генеалогическое древо), 
определяющих связи между $b$-объ\-ек\-тами.
\end{itemize}

  Структура справочного издания <<Памятная книга губернии>>~\cite{16mar} 
отражает деление губернии на уезды и населенные пункты, принадлежность 
учреждений как административным единицам, так и ведомствам, структуру 
учреждений и должности, звания, чины служащих в них лиц. 
  
  Однако большинство фактов извлечь из текстов может только исследователь. 
Определим операцию интерпретации ($\Rightarrow$), недоступную или 
неэффективную для автоматического логического анализа и выполняемую 
исследователем, которая выводит формулы логики фактов из содержания 
компонента ИП. Например, то, что в тексте речь идет об указанном промежутке 
времени, представимо следующей формулой: $(text) \hm\Rightarrow \exists t \in 
[t_{\min},\,t_{\max}] (text)$.
  
  Другой пример~--- текст содержит сведения о $b$-характеристике~$\beta$ и 
некотором подмножестве ее значений~$A$: $(text) \hm\Rightarrow \exists x (\beta 
(x) \in A) (text)$.
  %
  В~част\-ности, если текст относится к $b$-объекту, именуемому <<$N\!N$>>: 
$ (text) \hm\Rightarrow \exists x (\mbox{\textit{Именование}}(x) \hm = \mbox{<<}N\!N\mbox{>>}) 
(text)$.
  
  Максимально точная оценка биографических границ для текста, реализуемая 
как внутренняя разметка или как внешнее индексирование, должна 
способствовать эффективности биографического поиска.
  
  Интерпретация источника может выявить новые неизвестные 
  $b$-отношения, которые на новом шаге позволят найти дополнительные 
сведения об искомом $b$-объекте. Например, в воспоминаниях одноклассника 
искомого лица мы находим сведения об их учителе (неизвестное ранее 
  $b$-отношение), а уже в переписке учителя~--- сведения об искомом лице.
  
  Немаловажный факт, извлекаемый из текста, состоит в утверждении, что в 
нем нет сведений о данном $b$-объекте или о данной $b$-характеристике.

%\vspace*{-6pt}

\subsection{Выборка биографических сведений из~источника}
  
  Выборку текста так же, как его интерпретацию, представим в терминах 
логики фактов. Введем следующее обозначение: $text \hm= \mbox{\textit{Контент}}(l)$~--- 
интерпретируемый текст~--- это содержание документа (фрагмента документа) 
по адресу~$l$. Расширим множество констант и переменных логики фактов 
произвольными текстами и адресами в~ИП.
  
  Важнейшим фактом, извлекаемым из текста, является отсылка к другим 
компонентам ИП. Пусть из интерпретации текста по адресу~$l_0$ следует,\linebreak что 
сведения относительно $b$-характеристики~$\beta$\linebreak
 $b$-объек\-та~$x$ можно 
найти в документе по адресу~$l_1$, тогда: $\mbox{\textit{Контент}}(l_0) \hm \Rightarrow 
\mbox{\textit{Контент}}(l_1) \wedge \beta (x)$.
  
  Доступ к архивным материалам или редким изданиям затруднен. В~то же 
время необходимый %\linebreak
 для анализа в конкретном исследовании объем данных, как 
правило, бывает много меньше до\-ку\-мен\-та-ис\-точ\-ни\-ка в целом. Поэтому как в 
случае труднодоступных источников, так и для анализа легкодоступных, но 
объемных материалов, применяется практика создания вторичных документов: 
копии фрагмента, выписки, реферата, аннотации. В~терминах логики фактов 
то, что по отношению к совокупности исследовательских вопросов (ff) при 
\textit{Редукции}~--- создании вторичного документа~--- не выпущено ничего 
существенного, фиксируется\linebreak  так:

\smallskip

\noindent
  $$
(\mbox{\textit{Контент}}(l) \wedge\mbox{ff} ) = 
(\mbox{\textit{Редукция}}(\mbox{\textit{Контент}}(l)) 
\wedge \mbox{ff})\,.
$$

%\vspace*{-6pt}

\subsection{Вывод формул логики фактов}
  
  Появление новой формулы логики фактов в багаже исследователя возможно 
и как следствие интерпретации некоторого текста, и как логический вывод из 
имеющихся формул. И~в том, и в другом случае возможны неточности и 
огрехи, вследствие как ошибок исследователя, так и некорректности или 
неполноты исходных данных. В~любом случае полезно делать предположения, 
строить рабочие гипотезы. Для того чтобы отличить гипотезу от 
установленного факта, имеет смысл воспользоваться категориями нечеткой 
логики, т.\,е.\ вместо значений ИСТИНА и ЛОЖЬ, употреблять оценку 
правдоподобия в виде числа в диапазоне от~0 (ЛОЖЬ) до~1 (ИСТИНА). 
В~виде формул нечеткой логики могут быть сформулированы опирающиеся на 
статистику экспертные оценки. Пример: 
  \begin{align*}
&\mbox{\textit{Оценка}}(\mbox{\textit{Служба}}(\mbox{<<Леонтий\ Кириллович>>},\\
& \mbox{<<церковь\ села\  Ловцы>>}, 1780\mbox{--}1790) \wedge{}\\
 &\wedge \mbox{\textit{Служба}}(\mbox{<<Кирилл\ Васильевич>>},\\
 &\mbox{<<церковь\ села\ Ловцы>>}, 
1755\mbox{--}1760) \wedge{}\\
&\wedge \mbox{\textit{ЭкспертнаяОценка}}(\mbox{\textit{Имя}}(\mbox{духовное\ лицо}) ={}\\
&=\; 
\mbox{<<Кирилл>>}) = 0{,}003 \wedge{}\\
&\wedge \mbox{\textit{ЭкспертнаяОценка}}((\mbox{\textit{Служба}}(x, c) \wedge{}\\
&{}\wedge 
\mbox{\textit{Родня}}(x, y)) \rightarrow \mbox{\textit{Служба}}(y, c)) = 
0{,}75\rightarrow{}\\
 & {}\rightarrow \mbox{\textit{Отец}}(\mbox{<<Леонтий\ Кириллович>>},\\
 & \mbox{<<Кирилл\ Васильевич>>})) = 0{,}9\,.
  \end{align*}
  
  Нечеткую логику рационально использовать и\linebreak
  для оценки противоречивых 
фактов. Нормализованный факт с оценкой, отличной от~0 и~1, пред\-став\-ля\-ет собой вариант 
исследовательского вопроса. Важно сохранять все, даже отвергнутые (с\linebreak 
оценкой~0), факты. На новом этапе исследователь, может быть, вернется к ним, 
в том числе для анализа намеренных искажений, из которых также могут быть 
выявлены некоторые факты или дана предопределенная оценка извлекаемых из 
некорректного источника новых фактов.

\section{Заключение}
  
  Подытожим статью, перечислив основные черты предложенной модели 
представления биографической информации:
  \begin{itemize}
\item в рассмотрение включаются не только лица, но и объекты иной природы 
(организации, населенные пункты, исторические события) вместе с их 
структурой и историей;
\item помимо характеристик отдельного объекта рассматриваются 
характеристики отношений между объектами;
\item характеристики рассматриваются в динамике изменения их значений;
\item наличие характеристик, их возможные значения и взаимозависимости 
определяются конкретно историческими знаниями, большая часть которых 
также изменяется во времени;
\item для представления совокупности установленных фактов, намеченных к 
рассмотрению вопросов, а также шагов процесса исследования предложена 
единая форма~--- формулы логики биографических фактов. 
\end{itemize}
  
  Предложенная модель не противоречит существующим биографическим 
моделям, а, скорее, дополняет и уточняет их, расставляя несколько другие 
акценты. В~частности, приоритет динамических характеристик отношений 
между объектами позволяет создать целостную непротиворечивую картину 
биографии отдельного лица и в то же время обеспечивает повторное 
использование фактов при освещении биографий связанных с ним лиц или 
истории объектов другой природы. 
  
  То, что номенклатура рассматриваемых характеристик, их возможные 
значения и допустимые отношения между ними не закреплены жестко, а 
определяются динамически изменяемыми нормалями, обеспечивает открытость 
модели, возможность применения ее для решения задач из самых разных 
проблемных областей.
  
  На основе предложенной модели возможна организация эффективного 
доступа к БР. Для этого определение метаданных, с одной стороны, и 
интерфейса поисковых запросов~--- с другой, следует осуществлять в виде 
формул логики фактов. Такой подход применим не только при создании новых 
БР, но и для модификации существующих открытых БР, а также для разработки 
специализированных оболочек закрытых БР. 
  
  Важнейшим приложением модели должно стать создание рабочего поля 
биографического исследования~--- своеобразного <<нового>> БР, в котором 
востребованы возможности работы с неточными, противоречи\-вы\-ми, 
непроверенными данными. Их постепенное накопление, систематизация, 
выдвижение и опровержение гипотез, сопоставление, обосно\-ва\-ние, постановка 
новых вопросов~--- все то, что необходимо в процессе исследования, удобно 
представимо с помощью аппарата логики фактов.

{\small\frenchspacing
{%\baselineskip=10.8pt
\addcontentsline{toc}{section}{Литература}
\begin{thebibliography}{99}

\bibitem{1mar}
\Au{Маркова Н.\,А., Адамович И.\,М.}
Электронные биографические ресурсы~// Электронные библиотеки: перспективные методы 
и технологии, электронные коллекции (RCDL'2010): Труды XII Всеросс. научн. конф.~--- 
Казань: КГУ, 2010. С.~168--180.
\bibitem{2mar}
\Au{Маркова Н.\,А., Адамович И.\,М.}
Коллекции персоналий~// Системы и средства информатики. Вып.~20. №\,2. Методы и 
технологии, применяемые в научных исследованиях информатики.~--- М.: ИПИ РАН, 2010. 
С.~178--198.

\bibitem{3mar}
Стратегический план Программы ЮНЕСКО <<Информация для всех>> (2008--2013~гг.).~--- 
М.: Межрегиональный центр библиотечного сотрудничества, 2009. 48~с.

\bibitem{4mar}
\Au{Валевский В.\,Л.}
Биографика как дисциплина гуманитарного цикла~// Лица: биографический альманах.~--- 
СПб.: Феникс, 1995. Вып.~6. С.~33--68.

\bibitem{5mar}
Русские фольклористы: Биобиблиографический словарь. Пробный выпуск~/ Отв. ред. 
Т.\,Г.~Иванова и А.\,Л.~Топорков.~--- М.: ПРОБЕЛ-2000, 2010. 240~с. 

\bibitem{6mar}
Портал Персоналии. Материал из Википедии~--- свободной энциклопедии. {\sf 
http://ru.wikipedia.org/\linebreak wiki/Портал:Персоналии} (дата обращения: 29.01.2011).

\bibitem{7mar}
Микроформаты. {\sf http://microformats.org/} (дата обращения: 29.01.2011).

\bibitem{8mar}
Российский коммуникативный формат.~--- Министерство культуры Российской Федерации, 
Российская библиотечная ассоциация, Национальная служба развития системы форматов 
RUSMARC. {\sf http://www.rba.ru/rusmarc/}.

\bibitem{9mar}
GEDCOM XML Specification, Release~6.0. 
{\sf http:// xml.coverpages.org/Gedcom-XMLv60.pdf} (дата обращения: 29.01.2010).

\bibitem{10mar}
GenXML 3.0 16.06.2010. {\sf http://www.cosoft.org/\linebreak genxml/GenXML30.pdf} (дата 
обращения: 29.01.2011).

\bibitem{11mar}
\Au{Юмашева Ю.\,Ю.}
Историография просопографии~// Известия Уральского государственного университета.~--- 
Екатеринбург: УрГУ, 2005. №\,39. Гуманитарные науки. Вып.~10. С.~95--127.

\bibitem{12mar}
Концепция информатизации архивного дела России. Утверждена Росархивом в 1995~г. {\sf 
http://\linebreak www.rusarchives.ru/informatization/conseption.shtml} (дата обращения: 29.01.2011).

\bibitem{13mar}
Всероссийское генеалогическое древо (ВГД). {\sf http://baza.vgd.ru/} (дата обращения: 
29.01.2011).

\bibitem{14mar}
\Au{Кравченко А.\,И.}
Социология. Общий курс: Учебное пособие для вузов.~--- М.: ПЕРСЭ; Логос, 2002. 640~с.

\bibitem{15mar}
\Au{Ганзен В.\,А.}
Системные описания в психологии.~--- Л.: ЛГУ, 1984.  175~с.

\label{end\stat}

\bibitem{16mar}
Памятные книжки губерний и областей Российской империи. {\sf 
http://www.nlr.ru/pro/inv/mem\_buks.htm} (дата обращения: 29.01.2011).
 \end{thebibliography}
}
}


\end{multicols}