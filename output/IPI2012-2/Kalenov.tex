\def\stat{kalenov}

\def\tit{ЗАДАЧИ И ФУНКЦИИ БИБЛИОТЕК РАН\\ В~СОВРЕМЕННЫХ УСЛОВИЯХ}

\def\titkol{Задачи и функции библиотек РАН в~современных условиях}

\def\autkol{Н.\,Е. Калёнов}
\def\aut{Н.\,Е. Калёнов$^1$}

\titel{\tit}{\aut}{\autkol}{\titkol}

%{\renewcommand{\thefootnote}{\fnsymbol{footnote}}\footnotetext[1]
%{Работа выполнена при поддержке РФФИ (гранты 09-07-12098, 09-07-00212-а и
%09-07-00211-а) и Минобрнауки РФ (контракт №\,07.514.11.4001).}}


\renewcommand{\thefootnote}{\arabic{footnote}}
\footnotetext[1]{Библиотека по естественным наукам Российской академии наук, 
nek@benran.ru}

\vspace*{2pt}


\Abst{Рассмотрены вопросы, связанные с изменениями в деятельности академических 
библиотек в связи с развитием сетевых технологий, бурным ростом числа 
электронных публикаций и баз данных (БД), доступных через Интернет. Показано, что и в 
современных условиях академические библиотеки являются неотъемлемой частью 
научной инфраструктуры. Традиционные задачи по информационному со\-про\-вож\-де\-нию 
научных исследований остаются прерогативой библиотек, однако должны выполняться на 
базе новых технологических решений с широким использованием современных сетевых 
технологий. Новые подходы к решению традиционных задач иллюстрируются на примере 
Библиотеки по естественным наукам Российской академии наук (БЕН РАН) ({\sf http://www.benran.ru}). Наряду с 
традиционными задачами академические библиотеки готовы к решению новых задач, 
связанных с проведением библиометрических исследований, оцифровкой печатных 
изданий и~т.\,п.}

\vspace*{2pt}
 
\KW{академические библиотеки; информатизация; автоматизация; обслуживание 
пользователей; информационное обеспечение; компьютерные технологии; 
автоматизированное рабочее место; электронные библиотеки; БЕН РАН}



  
   \vskip 16pt plus 9pt minus 6pt

      \thispagestyle{headings}

      \begin{multicols}{2}

            \label{st\stat}

\section{Введение}

   За последние годы произошли серьезные изменения в мировом информационном 
пространстве, которые непосредственно связаны с научными биб\-лио\-те\-ка\-ми. Бурное 
развитие электронных биб\-лио\-тек, широкое распространение новых издательских и 
биб\-лио\-теч\-ных технологий, кажущаяся доступность через Интернет всей научной 
информации породили мнение, к сожалению, раз\-де\-ля\-емое рядом ведущих российских 
ученых, о бес\-смыс\-лен\-ности существования академических биб\-лио\-тек. В~этой связи 
представляется необходимым проанализировать их деятельность в современных условиях, 
рассмотреть перспективы их существования и развития. 
   
   С самого начала своего существования академические биб\-лио\-те\-ки решали две 
основные тесно взаимосвязанные задачи: (а)~информационное обеспечение научных 
исследований и (б)~сохранение знаний. 
   
   Первый вопрос, на который необходимо ответить: актуальны ли эти задачи в 
современных условиях? 
   
   Начнем со второй задачи, ответ по которой очевиден~--- знания необходимо 
сохранять. При этом в мире пока не придумано системы, альтернативной опубликованию 
результатов научных исследований, отражающих накапливаемые знания. Это могут быть 
публикации в научных журналах, монографиях, зарегистрированные патенты и авторские 
свидетельства. Под публикацией не обязательно рассматривать печатные материалы~--- 
это могут быть электронные носители, фото- или киноматериалы. Важно, что научный 
результат прошел экспертизу некоторым признанным научным сообществом~--- будь то 
редколлегия журнала, рецензенты издательств или ученый совет вуза. 
   
   Вплоть до настоящего времени сохранность знаний обеспечивают научные 
биб\-лио\-те\-ки, осуществляя комплектование своих фондов (в том числе и материалами на 
электронных носителях). Но если библиотеки перестанут существовать~--- кто и в каком 
виде будет сохранять знания? Теоретически это могут быть издательства или 
производители информации~--- ученые. Но издательства~--- коммерческие структуры, и 
их деятельность полностью определяется экономической конъюнктурой. Ежегодно в мире 
происходит слияние и разделение научных издательств, банкротство одних и появление 
других. Поскольку в сохранности знаний заинтересовано общество в целом, поручать эту 
серьезную задачу организациям, заинтересованным в получении максимальной прибыли, 
вряд ли целесообразно. 

Еще более сомнительным представляется вариант обеспечения 
сохранности знаний их произ\-водителями~--- научными организациями или отдельными 
учеными. Очевидно, что при этом, во-пер\-вых, существенную роль будет играть 
субъективизм, а во-вто\-рых, если ученые будут заниматься проблемами организации 
хранения и\linebreak
 предоставления информации, им не останется време\-ни на проведение 
собственно научных ис\-следований и неминуемо встанет вопрос о создании для этого 
специальной структуры, которая фактически и будет являться библиотекой. 

Возможен 
еще третий вариант, когда полученные знания формируются в электронном виде и 
загружаются в Интернет авторами, с тем чтобы любой же\-ла\-ющий мог их оттуда 
<<выудить>>. Этот вариант также не выдерживает серьезной критики, поскольку не 
обеспечивает экспертизу качества полученных результатов, оставляет открытым вопрос 
научного приоритета, требует наличия специальных служб, обеспечивающих поддержку и
сохран\-ность ресурсов (фактически аналогов биб\-лио\-теки). 
{\looseness=1

}
   
   Таким образом, для обеспечения сохранения и предоставления знаний необходимо 
существование специальной структуры, которую логично отождествить с библиотечной 
системой, поскольку по\-дав\-ля\-ющее большинство физических носителей знаний до 
настоящего времени хранится именно в библиотеках. Очевидно, что библиотеки как 
хранители знаний должны претерпеть серьезные изменения и вместе (а в дальнейшем, 
возможно, и вместо) с книжными стеллажами обладать мощными 
вычислительными средствами. Соответственно, и специалисты, работающие в 
библиотеках, должны получить необходимую квалификацию. 
   
   Перейдем к проблеме информационного обеспечения научных исследований. Нужны 
ли в РАН библиотеки для ее решения? 
   
   Оппоненты утверждают, что, поскольку вся важная научная информация имеется в 
Интернете, а научные издательства переходят на технологию \textit{print on demand} 
(печать по требованию), надобность в библиотеках как структурах, обслуживающих 
информацией сотрудников РАН, отпадает~--- необходимую информацию каждый ученый 
может найти и получить самостоятельно через Интернет. 
   
   Предположим, что это так, но при этом возникает несколько проблем. 
   
   Первая из них~--- отбор и приобретение прав доступа к научным ресурсам 
(большинство ведущих мировых издательств и информационных центров предоставляют 
свои информационные ресурсы по подписке на платной основе). Очевидно, что отдельные 
ученые для себя лично этим заниматься не будут. Речь может идти о приобретении 
ресурсов для определенного коллектива исследователей уровня лаборатории, института, 
научного центра, отделения или РАН в целом. В~условиях ограниченных финансовых 
ресурсов для оптимального решения этой проблемы необходима большая работа по 
анализу мирового информационного рынка, проведение переговоров с производителями 
(или поставщиками) ресурсов. Кроме того, как показывает практика, в процессе работы с 
удаленными ресурсами возникает множество технических и организационных вопросов, 
которые кто-то должен решать. Вряд ли ученым следует заниматься этими проблемами, 
которые отнимают достаточно много времени и требуют специальных навыков. Логично 
поручить эту работу библиотечным специалистам, которые во многих академических 
организациях ее уже выполняют. 
   
   Вторая проблема~--- работа с информационными ресурсами. До настоящего времени 
библиотечные работники РАН были <<посредниками>> между учеными и 
информационной средой~--- они осуществляли поиск ресурсов, отвечающих тематике 
исследований обслуживаемых групп ученых, информировали о появлении новых 
материалов и~т.\,п. Если такие посредники исчезнут, то каждый научный сотрудник 
должен будет самостоятельно отслеживать появление новой информации, 
соответствующей тематике его исследований, а это, учитывая огромные объемы 
информационных потоков и их постоянный рост, потребует значительных затрат времени 
в ущерб собственно научным исследованиям. 
   
   Таким образом, роль академических библиотек в решении обеих вышеприведенных 
задач ничуть не уменьшилась по сравнению с предыдущими годами. Существенно должна 
измениться технология решения этих задач, что и будет показано ниже путем 
сравнительного анализа <<традиционных>> функций академических библиотек и их 
аналогов в современных условиях. 
   
   Традиционные функции академических биб\-лиотек включают: 
   \begin{itemize}
\item анализ информационных потребностей ученых; 
\item анализ мирового информационного рынка и приобретение необходимых научных 
ресурсов; 
\item формирование и доведение до пользователей (научных сотрудников) вторичной 
информации (сведений о поступивших изданиях, оглав\-ле\-ний журналов, аннотаций статей и~т.\,д.); 
\item предоставление материалов (оригиналов изданий, копий статей)
по запросам пользователей; 
\item организацию и хранение фондов. 
\end{itemize}
Каждая функция подразумевает ряд основных процессов. 

Рассмотрим традиционные и 
современные (применительно к практике БЕН РАН) способы их выполнения.
   
\section{Анализ информационных потребностей ученых}
   
   Если исходить из того, что каждый отдельный ученый не в состоянии обеспечить себя 
самостоятельно в полной мере необходимой ему научной информацией и для этого 
требуется специальная структура, очевидно, что без знания потребностей обслуживаемых 
ученых эта структура действовать не может. 
   
   Согласно традиционной библиотечной технологи информационные потребности 
ученых, обслуживаемых данной библиотекой, отражаются в 
   те\-ма\-ти\-ко-ти\-по\-ло\-ги\-че\-ском плане комплектования библиотеки (ТТПК), 
который представляет собой перечень тематических рубрик и типов изданий (научные, 
справочные и~т.\,п.) по каждой из них.\linebreak
 В~XX~в.\ 
ТТПК академических библиотек формировались в печатном виде библиотечными 
специалистами совместно с сотрудниками обслуживаемого 
   на\-уч\-но-ис\-сле\-до\-ва\-тель\-ско\-го учреждения (НИУ) и утверждались 
руководством НИУ. В~дальнейшем библиотека отбирала литературу для своих фондов в 
соответствии с ТТПК. При изменении тематики исследований ТТПК перепечатывался. 
В~централизованной библиотечной системе (ЦБС) БЕН РАН комплектование биб\-лио\-тек 
НИУ осуществляется централизованно, поэтому ТТПК отдельных биб\-лио\-тек 
передавались в центральную библиотеку (ЦБ), где на их основе формировался сводный 
печатный ТТПК ЦБС, который использовался комплектаторами при отборе литературы, 
подлежащей заказу. В~современных условиях ТТПК остается по-преж\-не\-му 
необходимым материалом для библиотек, отражающим информационные потребности 
ученых НИУ, однако ведется он в виде БД. В~БЕН\linebreak
РАН разработаны 
специальные программные\linebreak
средства, обеспечивающие формирование ТТПК по отдельным 
библиотекам и сводного ТТПК по ЦБС в целом. Сис\-те\-ма обладает дружественным 
интерфейсом и полным фор\-маль\-но-ло\-ги\-че\-ским контролем, обеспечивает 
унифицированный ввод информации в БД, предоставляет авторизованным 
комплектаторам развитые средства редактирования и выборки 
   данных~[1--3]. 
   
   Для получения дополнительных сведений о реальных информационных потребностях 
пользователей библиотеки анализируют спрос на издания из своих фондов (в первую 
очередь спрос на журналы, поскольку полученные данные могут служить основой для 
корректировки подписки). Если раньше анализ спроса проводился вручную путем 
обработки читательских требований, то в современной системе БЕН РАН все заказы 
вводятся и обрабатываются в автоматизированном режиме, а для получения 
статистических данных и формирования оптимального заказа на журналы используются 
специально разработанные программные средства~\cite{7kale, 8kale}. 
   
   Таким образом, принципиально процессы изуче\-ния информационных потребностей 
ученых остаются необходимой составляющей библиотечной деятельности, однако 
реализуются они на базе новых технологий, требующих соответствующей квалификации 
библиотечного персонала. 
   
\section{Анализ мирового информационного рынка и~приобретение 
научных ресурсов}
   
   В XX~в.\ основным источником информации об отечественном книжном рынке 
служили печатные тематические планы издательств (ТПИ), которые выпускались большими 
тиражами всеми издательствами страны и поступали в крупные библиотеки. Библиотеки 
ЦБС получали ТПИ через ЦБ, отмечали совместно с учеными необходимые издания и 
возвращали в ЦБ, где на их основе (после контроля по ТТПК) формировался и 
направлялся в издательства сводный заказ. 
   
   По зарубежным книгам БЕН выпускала ежемесячный указатель <<Новые зарубежные 
книги>>, формируемый специалистами библиотеки на основе анализа материалов, 
поступающих из зарубежных издательств, и сопоставления их с ТТПК. Указатель 
рассылался и обрабатывался аналогично ТПИ. В~масштабах страны информирование 
библиотек о новой зарубежной на\-уч\-но-тех\-ни\-че\-ской литературе осуществлялось 
путем выпуска многотиражного издания, подготавливаемого ГПНТБ СССР на основе 
материалов, передаваемых крупнейшими библиотеками. 
   
   Сейчас такая технология представляется анахронизмом: ТПИ в печатном виде не 
издаются, а многие издательства присылают свои планы и прайс-лис\-ты в библиотеки по 
электронной почте. Кроме того, Российская книжная палата (РКП) формирует в 
электронном виде библиографическую информацию обо всех изданиях, по\-сту\-па\-ющих по 
обязательному экземпляру. Используя эти данные, БЕН РАН разработала и внедрила в 
практику экспертную систему комплектования, основанную на сетевых 
   технологиях~\cite{4kale, 5kale, 6kale}. Два раза в месяц в специальную БД, 
поддерживаемую на сервере БЕН РАН, вводится новая информация, поступающая из РКП 
(на договорных условиях) и из ряда издательств (безвозмездно). Предварительно из 
общего массива данных программно отбираются издания, соответствующие сводному 
ТТПК, а также осуществляется сверка на дублетность (описания изданий, поступившие из 
РКП, могли ранее поступить из издательств). Авторизованные эксперты, официально 
выделенные руководством НИУ РАН из числа квалифицированных научных сотрудников, 
имеют возможность войти в БД, выбрать интересующую тематику и оценить издания с 
точки зрения целесообразности приобретения для библиотеки НИУ или ЦБС в целом. 
Полученные оценки обрабатываются специальными про\-грам\-мны\-ми средствами и 
используются, в совокупности с ТТПК, для централизованного комплектования фондов 
биб\-лио\-тек НИУ. 
   
   Информация о зарубежном информационном рынке формируется специалистами БЕН 
РАН с использованием электронных каталогов издательств и базы данных Global 
Books-in-Print, конвертируется, загружается в экспертную систему и далее оценивается и 
обрабатывается аналогично отечественной информации. 
   
   Таким образом, пользователь участвует в отборе информации, подлежащей заказу; в 
определенном смысле реализуется технология, аналогичная традиционной, но на 
современном уровне. Очевидно, что эффективность такой технологии существенно выше 
традиционной, однако она требует от комплектаторов специальных навыков~--- они 
должны уметь работать с прикладными программными средствами, искать и 
обрабатывать информацию в Интернете, причем не только на русском, но и на других 
языках. Необходимо отметить, что работа по анализу мирового информационного рынка 
через сеть может быть существенно упрощена, если библиотека получит доступ к 
специальным БД, аккумулирующим издательскую информацию, однако доступ 
к подобным БД является платным и достаточно дорогим. 
   
   Что касается собственно приобретения ресурсов, то в современных условиях 
академические библиотеки все большее внимание уделяют приобретению прав доступа 
для своих пользователей к сетевым научным ресурсам. Поэтому наряду с традиционными 
процессами, связанными с получением материалов на физических носителях (заказ, 
контроль поступлений, регистрация и распределение изданий), библиотечные 
специалисты решают задачи, связанные с определением соотношения в приобретении 
печатных и сетевых ресурсов и организационным обеспечением доступа к последним 
(оформление лицензионных соглашений, сбор и передача поставщикам IP-ад\-ре\-сов 
подключаемых НИУ, решение часто возникающих текущих проб\-лем). 
   
   Как и другие направления деятельности, связанные с приобретением информационных 
материалов, процессы подключения пользователей к сетевым ресурсам определяются, в 
первую очередь, объемом финансовых ресурсов, выделяемых на эти цели. Права доступа 
к сетевым версиям по\-дав\-ля\-юще\-го большинства серьезных научных журналов (так же, как 
и их печатные версии) предоставляются за плату, причем ее величина может существенно 
меняться в зависимости от того, с какого количества IP-ад\-ре\-сов предоставляется 
доступ, сколько сотрудников имеется в данной организации и~т.\,п. К~сожалению, 
объемы финансирования, вы\-де\-ля\-емые на информационное сопровождение научных 
исследований в России (в част\-ности, в РАН) и в большинстве стран, занимающихся 
наукой, несопоставимы. Так, годовой объем средств, получаемых БЕН РАН на 
информационное обеспечение около 120~институтов и научных центров, в 8~раз меньше 
средств, выделяемых на подобные цели библиотеке только одного факультета 
Гарвардского университета США. Как и любая коммерческая сфера, торговля доступом к 
научным ресурсам пред\-усмат\-ри\-ва\-ет различные скидки, нелинейную зависимость 
стоимости от количества пользователей и объема ресурсов и~т.\,д. Проблемы с 
финансированием имеются практически во всех научных библиотеках мира, поэтому 
многие из них объединяются в консорциумы по доступу к информационным ресурсам, 
что позволяет экономить финансы для каждой из них. Достаточно подробно современные 
подходы к этим вопросам изложены в докладе американских библиотечных специалистов 
на последней конференции ИФЛА~\cite{14kale}, где описываются подходы к экономии 
средств у двух консорциумов американских библиотек \mbox{ORBIS} и \mbox{CIC}, 
вклю\-ча\-ющих соответственно 36~биб\-лио\-тек университетов и колледжей штатов Орегон 
и Вашингтон и 13~биб\-лио\-тек университетов северных штатов США. Объем 
финансирования этих библиотек колеблется в диапазоне от 6 до 20~млн долларов в 
год (цифры, несопоставимые с финансированием российских академических библиотек), 
тем не менее авторы утверждают, что средств на приобретение 
ин\-фор\-ма\-ци\-он\-но-биб\-лио\-теч\-ных ресурсов катастрофически не хватает. 
Задачу информационной поддержки 
своих пользователей они решают, сокращая до одного экземпляра подписку на печатные 
журналы внутри консорциума (или отказываясь вообще от печатной версии), координируя 
приобретение печатных книг, анализируя спрос на приобретенные ресурсы и корректируя 
подписку на основе этого анализа. 
   
   В этом направлении Россия нисколько не отстает от зарубежных стран. Подобный 
консорциум был организован в 1996~г.\ по инициативе БЕН РАН и включил РФФИ и 
14~крупнейших библиотек страны. В~рамках консорциума была организована 
скоординированная подписка на печатные версии журналов (финансируемая каждой 
библиотекой) и обеспечен доступ к научным журналам практически всех ведущих 
издательств мира (финансируемый РФФИ), которые передавали электронные версии 
журналов российской стороне, загружающей их на свои серверы и предоставляющей 
свободный доступ к ним всем желающим. С~2004~г.\ эта система перестала существовать 
по ряду причин (издательства отказались передавать информацию, а цены на 
   он\-лайн-до\-ступ существенно повысили, РФФИ\linebreak
   изменил свою <<политику>>, 
передав сопровожде-\linebreak ние уже загруженных массивов коммерческой\linebreak
 структуре ООО 
<<Научная электронная библиотека>>~--- НЭБ). В~настоящее время информация, 
поступившая в НЭБ до 2004~г., находится в свободном доступе на сайте {\sf 
www.elibrary.ru}, все же новые журналы (в том числе и журналы РАН) предоставляются 
на коммерческой основе. 
   
   В последние годы РФФИ финансирует доступ своих грантодержателей к научным 
журналам основных зарубежных издательств, однако с 2012~г.\ эта деятельность также в 
определенной степени сворачивается (в частности, РФФИ из-за нехватки 
финансовых средств
отказался  от приобретения доступа к журналам второго по значимости в мире 
издательства Springer-Verlag). 
   
   В этой связи перспективы развития системы доступа к научным ресурсам со всех 
рабочих мест сотруд\-ни\-ков РАН представляются достаточно туманными. Централизация 
же библиотек (когда структурные подразделения центральных библиотек располагаются в 
институтах и научных центрах и обслуживают их сотрудников) позволяет существенно 
экономить средства на приобретение доступа к журналам и БД путем 
ограничения числа рабочих мест компьютерами библиотеки. Именно такой подход в 
сложившихся условиях применяет БЕН РАН, обеспечивая в рамках выделенных 
ассигнований доступ к максимальному количеству важнейших журналов с компьютеров 
ЦБ и библиотек~--- ее отделений в НИУ РАН. 
   
   Острота проблемы доступа к важнейшим научным журналам прошлых лет, вероятно, в 
ближайшие годы будет снижена благодаря тому, что Минобрнауки планирует 
профинансировать приобретение их электронных архивов, размещение на отечественных 
серверах и доступ к ним всех граж\-дан России.
   
\section{Формирование и~доведение до~ученых вторичной информации}

%\vspace*{-9pt}
 
   На протяжении многих десятилетий академическими библиотеками отрабатывались 
наиболее эффективные формы и методы информационного обслуживания ученых. 
Обеспечивающие его процессы можно разделить на две группы~--- информирование об 
изданиях, имеющихся в библиотеках (раскрытие фондов), и поиск вторичной информации 
по тематике исследований ученых. Первая группа традиционно включала в себя 
карточные (реже~--- печатные) каталоги фондов (алфавитные,\linebreak
 сис\-те\-ма\-ти\-че\-ские, предметные), 
печатные указатели подписки и бюллетени новых поступлений. Вторая группа включала 
выпуск текущих и ретро\-спективных указателей литературы по различным тематическим 
направлениям, ведение темати\-че\-ских картотек, избирательное распространение 
информации (ИРИ), поиск библиографии по заданной пользователями теме и~т.\,п. 
   
   В современных условиях на смену карточным каталогам пришли электронные, 
обеспечивающие все виды библиографического поиска; вместо печатных указателей на 
сайтах академических биб\-лио\-тек представлены разделы, отражающие новые поступления 
изданий. Наряду с информацией об изданиях, имеющихся на физических носителях, на 
сайтах академических библиотек представлены сведения о сетевых ресурсах, доступных 
их пользователям. Так, на сайте БЕН РАН ({\sf http://www.benran.ru}) выделен раздел 
<<Электронные версии книг и сериальных изданий, доступные пользователям БЕН 
РАН>>, который содержит подразделы <<Cериальные издания>>, <<Книги>>, 
<<Справочники и энциклопедии>>; в сводном каталоге журналов содержится перечень 
всех выпусков каждого журнала, поступивших в ЦБС БЕН РАН с 1990~г., а также ссылки 
на сайты журналов, перейдя по которым пользователь может знакомиться с оглавлениями 
журналов, аннотациями статей и читать полные тексты (если имеет соответствующие 
права). В~каталоге пред\-став\-ле\-ны не только журналы, выписанные\linebreak непосредственно БЕН, 
но и журналы, доступ к которым предоставлен ученым в рамках электронной библиотеки 
РФФИ и через Национальный электронный информационный
консорциум (\mbox{НЭИКОН}), реализующий соответствующую программу Минобрнауки. 
   
   Что касается текущего и ретроспективного информационного обслуживания, то, как 
показывают опросы пользователей, эти процессы по-прежнему востребованы учеными, но 
формы их реализации %\linebreak\vspace*{-12pt}
%\pagebreak
%\noindent
 существенно изменились~--- вместо тематических картотек и 
указателей создаются проб\-лем\-но-ориен\-ти\-ро\-ван\-ные БД; информацию 
для удовле\-тво\-ре\-ния запросов пользователей библиотечные специалисты ищут в 
Интернете и результаты поиска на\-прав\-ля\-ют пользователям по электронной почте. 
Многие 
библиотеки создают на своих сайтах разделы, содержащие ссылки на сетевые ресурсы по 
тематике исследований своих институтов. В~качестве примеров наиболее активных в 
этом направлении библиотек ЦБС БЕН РАН можно привести Цент\-раль\-ную библиотеку 
Пущинского научного цент\-ра (ПНЦ) РАН ({\sf http://cbp.iteb.psn.ru}) и библиотеку 
Математического института им.\ В.\,А.~Стеклова РАН ({\sf http://libserv.mi.ras.ru}). 
   
   БЕН РАН поддерживает на своем сайте раздел <<Естественные науки в Интернете>>, 
который представляет собой набор <<метауказателей>> ресурсов по основным разделам 
естественных наук. Каждый метауказатель (например, <<Физика в Интернете>>, 
<<Биология в Интернете>> и~т.\,п.)\ содержит сведения об указателях ресурсов, 
представленных в Интернете по соответствующим разделам науки~--- дается описание 
указателя, сведения об организации, формирующей указатель, и ссылка на него. 
Специальные сотрудники библиотеки обеспечивают поддержку метауказателей в 
актуальном состоянии.

   
   В современных условиях одной из информационных функций сотрудников 
академических\linebreak
 биб\-лио\-тек (по аналогии с традиционными указа-\linebreak телями новых 
поступлений) должно стать от\-сле\-живание новых выпусков журналов, появляющихся\linebreak в 
Интернете, к которым имеется доступ у со\-труд\-ников обслуживаемых НИУ. Библиотечные 
ра\-ботники должны регулярно просматривать сайты\linebreak
 журналов, соответствующих 
информационным интересам ученых, и информировать их (по электронной почте или на 
сайте) о выходе нового выпуска. Такой сервис, ориентированный на подразделения НИУ, 
позволяет существенно экономить время научных сотрудников, высвобождая его для 
исследовательской работы. 
   
   Одна из традиционных функций академических библиотек~--- ведение картотек 
трудов сотрудников. В~современных условиях картотеки заменяются БД. На 
сайте БЕН РАН представлены БД пуб\-ли\-ка\-ций сотрудников ряда НИУ РАН, 
поддерживаемые биб\-лио\-теч\-ны\-ми специалистами с помощью унифицированного 
программного обеспечения, разработанного в БЕН. Потребность в таких БД не 
только не уменьшается, а существенно возрастает в связи с введением критериев оценки 
эффективности научной деятельности, в том числе связанных с количеством и качеством 
публикаций. В~этой связи необходимо отметить роль библиотек в проведении 
биб\-лио\-мет\-ри\-че\-ско\-го анализа пуб\-ли\-ка\-ций сотрудников институтов с использованием 
мировых БД, таких как WEB of Science и \mbox{SCOPUS}. Многолетний 
опыт работы с подобными БД, име\-ющий\-ся в БЕН РАН и других центральных 
академических библиотеках~[9--12], показывает, что для 
достижения корректных результатов как по отдельным авторам, так и по коллективам 
ученых необходимо проведение достаточно серьезной и большой работы, требующей 
знания специфики данной БД. Дилетантский подход при оценке пуб\-ли\-ка\-ци\-он\-ной 
активности приводит к ошибочным выводам, которые могут быть чреваты серьезными 
последствиями для науки. Библиометрический анализ публикаций ученых РАН должны 
проводить квалифицированные специалисты, для которых эта деятельность является 
профессиональной. В РАН такими специалистами являются, в первую очередь, сотрудники 
библиотек. Наиболее активную работу в этом направлении в ЦБС БЕН РАН ведет 
Центральная библиотека ПНЦ, о которой уже шла речь выше. Ее сотрудники 
периодически проводят анализ публикаций ученых НИУ ПНЦ и представляют на своем 
сайте результаты этого анализа. 

\section{Предоставление материалов по~запросам пользователей} 
   
   Традиционные формы обслуживания читателей академическими библиотеками 
постепенно уступают место новым технологиям. В~БЕН РАН все заказы на оригиналы 
изданий и копии материалов принимаются только в автоматизированном режиме. Каждый 
авторизованный читатель может сформировать свой заказ через Интернет или через 
компьютер в зале каталогов ЦБ. Сотрудники биб\-лио\-те\-ки регистрируют в базе данных все 
отказы, поступающие из отдела фондов. Это позволяет с помощью специальных 
программных средств проводить 100\%-ный анализ спроса на литературу и причин отказов, 
что, в свою очередь, является основой для корректировки комплектования ЦБС и 
перераспределения изданий между библиотеками внутри единого фонда. Если 10~лет 
назад основными видами выполнения заказов являлось пред\-остав\-ле\-ние оригинала издания 
или ксерокопии\linebreak статьи, то сейчас значительная доля заказов выполняется в виде 
электронных копий. Действующая в БЕН РАН сис\-те\-ма электронной доставки документов 
позволяет авторизованному пользователю осуществлять заказ непосредственно из 
Ин\-тер\-нет-ка\-та\-ло\-гов, представленных на сайте библиотеки, и получить материалы в 
течение нескольких часов. 
{\looseness=1

}
   
   Наряду с предоставлением материалов из собственных фондов академические 
библиотеки (как центральные, так и многие библиотеки НИУ) организуют доступ своих 
пользователей к сетевым информационным ресурсам. 
%
Как показывает практика 
(подтвержденная результатами анкетирования, проведенного БЕН РАН среди своих 
пользователей), многие поль\-зо\-ва\-те\-ли\,--\,со\-труд\-ни\-ки РАН испытывают 
определенные затруднения при работе с сетевыми ресурсами (что усугубляется разными 
интерфейсами, принятыми у разных поставщиков). Поэтому одной из функций 
академических библиотек должно стать обучение пользователей работе с сетевыми 
ресурсами. Эта функция, как и многие другие, не является чем-то новым для 
   биб\-лио\-тек~--- раньше они обучали читателей пользованию каталогами, теперь~---  
пользованию сетевыми ресурсами. Меняются средства работы библиотечных 
специалистов, повышаются требования к их квалификации, но их функции как 
<<проводников>> ученых в мировом информационном пространстве остаются. 

\vspace*{-6pt}

\section{Организация и~хранение фондов}

\vspace*{-2pt}
   
   Пока продолжается выпуск научных материалов в печатной форме, традиционные 
функции библиотек, связанные с обеспечением сохранности литературы, созданием 
условий для быстрого поиска и выдачи нужного издания сохраняются. В~современных 
условиях к ним добавляются процессы, связанные с обеспечением сохранности 
электронных носителей информации. 
%
Здесь достаточно много неясностей~--- нет четкого 
представления о том, как часто надо перезаписывать данные, чтобы защититься от их 
физического износа; как обеспечить возможность считывания данных с таких носителей, 
как CD- или DVD-дис\-ки, через несколько лет после записи, когда принципиально 
поменяются носители и технология записи на них данных (например, уже сейчас 
достаточно трудно найти компьютеры для чтения 3,5-дюй\-мо\-вых дискет, не говоря уже про 
5-дюй\-мо\-вые); как долго надо хранить электронные издания (печатные издания в 
большинстве библиотек периодически списываются) и~т.\,д. Технически и программно 
многие из этих вопросов решаются, но необходимо отдавать отчет в том, что если 
академические библиотеки будут оцифровывать и хранить электронные издания, то они 
должны обладать соответствующими техническими ресурсами. 

Проблемы, связанные с 
электронными научными ресурсами, должны решаться на серьезном уровне, а не 
декларациями о переводе всех фондов библиотек в электронную форму в течение 
ближайших лет. К~этим проблемам вплотную примыкают вопросы, связанные с 
формированием электронных библиотек (ЭБ) путем оцифров-\linebreak ки печатных изданий. 
Многие традиционные\linebreak биб\-лио\-те\-ки мира участвуют в формировании ЭБ, реализуются 
международные проекты в этой об\-ласти. К~наиболее успешным можно \mbox{отнести}\linebreak проект 
Европейской электронной (циф\-ро\-вой) биб\-лио\-те\-ки {Europeana}, реализованный при 
поддержке Европейской комиссии в конце 2008~г.\ ({\sf http://www.europeana.eu/portal}). 
В~этой ЭБ представлены ресурсы, отражающие культурное наследие различных стран, в 
том числе (хотя и в небольшой степени) России.
   
   Что касается академических библиотек России, то они активно вовлечены в работы по 
созданию ЭБ <<Научное наследие России>> ({\sf 
   http://e-heritage.ru}), которые ведутся в рамках одноименной целевой программы 
Президиума РАН с конца 2006~г.~\cite{15kale, 16kale}. Библиотеки формируют и вводят в 
ЭБ разнородную информацию о российских ученых, оцифровывают наиболее важные их 
работы, опубликованные в виде монографий и статей. Наполнение ЭБ осуществляется в 
хронологическом порядке начиная с XVIII~в., что в первую очередь связано с 
законодательством об охране авторских прав. Согласно 4-й части Граж\-дан\-ско\-го кодекса 
РФ библиотека не имеет права оцифровывать печатные издания и предоставлять их в 
сетевой доступ без письменного согласия авторов и других владельцев авторских прав, 
если со дня смерти автора прошло менее 70~лет. Тем не менее создание электронных 
библиотек является одним из перспективных направлений деятельности академических 
библиотек.

\section{Заключение }
   
   Выше были перечислены функции, которые должны выполнять академические 
библиотеки в современных условиях, оставаясь востребованными учеными. Многие из 
этих функций в той или иной мере развивают традиционные 
ин\-фор\-ма\-ци\-он\-но-биб\-лио\-теч\-ные технологии. Но появился ряд новых задач, которые с успехом могли бы 
выполнять библиотеки. Это уже упомянутые работы по проведению библиометрических 
исследований, работы по созданию электронных библиотек. Кроме того, одним из 
направлений деятельности академических библиотек могла бы стать поддержка в 
актуальном состоянии информации, представленной в Едином научном информационном 
пространстве (ЕНИП) РАН. Разработанная программная оболочка ЕНИП <<Научный 
институт РАН>>~\cite{13kale} позволяет вводить, редактировать и осуществлять 
многоаспектный поиск данных о деятельности НИУ РАН, в том числе о публикациях 
сотрудников с возможностью подключения полных тексов. Работа эта требует 
значительных усилий и временн$\acute{\mbox{ы}}$х затрат. По мнению автора, ЕНИП РАН не получает 
должного развития в том числе и из-за отсутствия в НИУ РАН сотрудников, 
обеспечивающих ввод и поддержку в актуальном состоянии всего информационного 
комплекса. Эта работа <<не по профилю>> ни системному администратору, ни 
администратору сайта института. В~ряде НИУ она возложена на ученых секретарей, но у 
многих из них есть свои научные интересы, и поддержка информационных ресурсов 
является для них второстепенной задачей. Кроме того, объем работы в этом направлении 
достаточно велик (особенно для крупных институтов), и ученый секретарь с ней не 
справляется физически. В~то же время библиотечные работники с современной 
подготовкой могли бы выполнять эти обязанности, поскольку они им достаточно близки в 
профессиональном плане, а свойственная им скрупулезность обеспечит качественное их 
выполнение. 
   
   Таким образом, в современных условиях для биб\-лио\-теч\-ной системы РАН продолжают 
существовать свои направления деятельности, работая в которых она вносит свой вклад в 
развитие академической науки. 

{\small\frenchspacing
{%\baselineskip=10.8pt
\addcontentsline{toc}{section}{Литература}
\begin{thebibliography}{99}

%\bibitem{1kale}
%\Au{Калёнов Н.\,Е.} Функции библиотек РАН в современных условиях~// 
%Информационное обеспечение науки. Новые технологии: Сб. научн. тр.~--- М.: БЕН 
%РАН, 2005. С.~6--16. 

\bibitem{2kale}
\Au{Бочарова Е.\,Н., Докторов Я.\,Я.}
Автоматизированная система формирования и ведения 
те\-ма\-ти\-ко-ти\-по\-ло\-ги\-че\-ских планов в практике комплектования ЦБС БЕН 
РАН~// Информационное обеспечение науки. Новые технологии: Сб. научн. тр.~--- 
М.: Научный мир, 2009. С.~200--207. 

\bibitem{3kale}
\Au{Бочарова Е.\,Н., Кочукова Е.\,В., Докторов~Я.\,Я.}
Актуализация сводного тематико-типологического плана комплектования ЦБС БЕН 
РАН~// Библиосфера, 2009. №\,2. С.~87--89. 

\bibitem{4kale} %3
\Au{Власова С.\,А., Докторов Я.\,Я., Калёнов~Н.\,Е., Кочукова~Е.\,В., Павлова~О.\,В.}
Ин\-тер\-нет-тех\-но\-ло\-гии в развитии системы комплектования ЦБС БЕН РАН~// 
Науковi пpацi Нацiональноi бiблiотекi Укpаiны iменi В.\,I.~Веpнадського, 2010. Вып.~28. 
С.~41--55. 

\bibitem{7kale} %4
\Au{Варакин В.\,П., Каленов Н.\,Е.}
Управление ресурсами централизованной библиотечной системы~// Информационные 
ресурсы России, 2010. №\,115. С.~2--11. 

\bibitem{8kale} %5
\Au{Варакин В.\,П., Власова С.\,А., Каленов~Н.\,Е.} 
Современные информационные технологии в задачах обслуживания читателей ЦБС БЕН 
РАН~// Вклад ин\-фор\-ма\-ци\-он\-но-библиотечной системы РАН в развитие отечественного 
библиотековедения, информатики и книговедения: Юбилейный научный сборник, 
посвященный 100-ле\-тию ИБС РАН.~---  Новосибирск, 2011. С.~187--203. 


\bibitem{5kale} %6
\Au{Власова С.\,А., Васильчиков В.\,В., Калёнов~Н.\,Е., Левнер~М.\,В.}
Использование экспертных оценок для комплектования централизованных библиотечных 
сис\-тем~// Научно-техническая информация. Сер.~1, 2007. №\,5. С.~22--26. 

\bibitem{6kale} %7
\Au{Власова С.\,А., Кочукова Е.\,В.}
Экспертная система ЦБС БЕН РАН~// Библиотеки национальных академий наук: 
проблемы функционирования, тенденции развития: Научн.-практ. и 
теор. сб.~--- Киев: Наукова Думка, 2010. Вып.~8. С.~79--85. 

\bibitem{14kale} %8
\Au{Armstrong K., Starratt J.}
Even cowgirls get the blues: How research libraries are coping with reductions in their 
collections budgets~// World Library and Information Congress: 77th IFLA General Conference 
and Assembly. Puerto Rico, 2011.
{\sf http://\linebreak conference.ifla.org/sites/default/files/files/papers/\linebreak ifla77/113-armstrong-en.pdf}.


\bibitem{9kale}
\Au{Глушановский А.\,В., Калёнов~Н.\,Е., Лексикова~Е.\,Е.}
База данных <<SCIENCE CITATION INDEX>> на CD-ROM.~--- М.: Биоинформсервис, 
1993. 38~с. 

\bibitem{10kale}
\Au{Лаврентьева М.\,В., Мелконян М.\,К., Смирнов~С.\,Н.}
Характеризация некоторых научных направлений Института кристаллографии РАН в базе 
данных <<Web of Science>>~// Новые технологии в информационном обеспечении науки: 
Сб. научн. тр.~--- М.: Биоинформсервис, 2001. С.~127--131. 

\bibitem{12kale} %11
\Au{Елепов Б., Лаврик О., Свирюкова~В.}
К~подсчету индексов готовы~// Наука в Сибири, 2007. №\,35(2620). С.~8.

\bibitem{11kale} %12
\Au{Трескова П.\,П.}
Наука в информационном измерении: анализ публикационной активности ученых с 
использованием баз данных <<Web of Science>> и <<SCOPUS>>~// Информационное 
обеспечение науки. Новые технологии: Сб. научн. тр.~--- М.: Научный мир, 2009. 
С.~253--262. 


\bibitem{15kale} %13
\Au{Каленов Н.\,Е., Савин Г.\,И., Сотников~А.\,Н.}
Электронная библиотека <<Научное наследие России>>: технология наполнения~// Новые 
технологии в информационном обеспечении науки: Сб. научн. тр.~--- М.: Научный 
мир, 2007. C.~40--48.

\bibitem{16kale} %14
\Au{Каленов Н.\,Е., Савин Г.\,И., Сотников~А.\,Н.}
Электронная библиотека <<Научное наследие России>> как составляющая 
интеграционных процессов~// Вестник Библиотечной Ассамблеи Евразии, 2011. №\,3. 
С.~52--55.

\label{end\stat} 

\bibitem{13kale} %15
\Au{Бездушный А.\,Н, Бездушный А.\,А., Нестеренко~А.\,К., Серебряков~В.\,А., 
Сысоев~Т.\,М., Теймуразов~К.\,Б., Филиппов~В.\,И.}
Информационная Web-сис\-те\-ма <<Научный институт на платформе ЕНИП>>.~--- М.: 
ВЦ РАН, 2007. 


 \end{thebibliography}
}
}


\end{multicols}


%Abstract: Problems of scientific libraries activities changings due to the development of 
%network technology, the burgeoning growth of Internet available electronic publications and 
%databases, are being considered. It is shown that academic libraries are an integral part of the 
%scientific infrastructure in the modern situation. The traditional tasks as such as science 
%information support remain the prerogative of the libraries, but they must be based on new 
%solutions and performed with an extensive use of modern network technologies. New 
%approaches to the traditional tasks are illustrated on the example of Library for Natural 
%Sciences of RAS (http://www.benran.ru). In addition to the traditional tasks academic libraries 
%are ready for new challenges in the area of bibliometrics research, digitization of printed 
%publications materials etc.
%
%Keywords: academic libraries, automation, informatics, user service, information providing, 
%computer technologies, digital libraries, automation workplaces, LNS of RAS