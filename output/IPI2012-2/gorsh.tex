\def\stat{gorsh}

\def\tit{ОБ УСТОЙЧИВОСТИ СДВИГОВЫХ СМЕСЕЙ НОРМАЛЬНЫХ ЗАКОНОВ ПО~ОТНОШЕНИЮ 
К~ИЗМЕНЕНИЯМ СМЕШИВАЮЩЕГО РАСПРЕДЕЛЕНИЯ}

\def\titkol{Об устойчивости сдвиговых смесей нормальных законов по~отношению 
к~изменениям смешивающего распределения}

\def\autkol{А.\,К.~Горшенин}
\def\aut{А.\,К.~Горшенин$^1$}

\titel{\tit}{\aut}{\autkol}{\titkol}

{\renewcommand{\thefootnote}{\fnsymbol{footnote}}\footnotetext[1]
{Работа
выполнена при поддержке Российского фонда фундаментальных
исследований (проекты 11-01-12026-офи-м и 12-07-00115).}}


\renewcommand{\thefootnote}{\arabic{footnote}}
\footnotetext[1]{Институт проблем информатики
Российской академии наук, agorshenin@ipiran.ru}


\Abst{Работа посвящена изучению устойчивости конечных сдвиговых смесей
нормальных законов относительно изменений параметров
смешивающего распределения. Результаты формулируются для моделей
добавления и расщепления компоненты, которые используются в задачах
проверки статистических гипотез о числе компонент смеси.}


\KW{сдвиговые смеси нормальных законов; метрика Леви}

\vskip 14pt plus 9pt minus 6pt

      \thispagestyle{headings}

      \begin{multicols}{2}

            \label{st\stat}
     


\newcommand*{\E}{\mathbb E}

\section{Введение}

В работе рассмотрены две модели конечных сдвиговых смесей
нормальных распределений: добавления и расщепления компоненты
(подробнее см.\ работы~\cite{Gorshenin2011, Gorshenin2011Mod2}).
Данные модели являются весьма удобными и информативными при
проведении статистического анализа данных с помощью различных
итерационных процедур разделения смесей вероятностных
распределений и позволяют эффективно решать задачи проверки гипотез о числе компонент смеси.

Модель добавления компоненты удобно использовать для проверки
значимости компоненты с малым весом.  Дело в том, что при
статистическом определении параметров в модели типа
смеси вероятностных распределений может появиться ком\-по\-нен\-та, вес
которой значительно меньше весов остальных компонент. В такой
ситуации необходимо убедиться в статистической значимости этой
компоненты, чтобы избежать влияния погрешностей вычисления на
итоговый результат.

Модель расщепления компоненты может применяться для в некотором
смысле обратной задачи, когда в силу вычислительных ошибок
компонента с малым весом может быть ошибочно отнесена к одной из
компонент с б$\acute{\mbox{о}}$льшим весом. Наличие устойчивости играет важную
роль при практическом использовании данных моделей, так как
гарантирует корректность полученных результатов.

В работе~\cite{Gorshenin2012SSI} были получены результаты для
конечных масштабных смесей нормальных законов. Однако в ряде
ситуаций оказывается полезным рассмотрение сдвиговых конечных
нормальных смесей. Модели такого типа возникают, например, при
решении оптимизационных задач для управлении запасами, при
моделировании потоков страховых выплат, при прогнозировании
надежности различных систем. Доказательству теорем устойчивости для
сдвиговых конечных смесей нормальных законов  и посвящена настоящая
статья.

\section{Постановка задачи}

Предположим, что каждое из независимых наблюдений
$\mathbf{X}_n\hm=(X_1,\ldots,X_n)$ имеет распределение,
представимое в виде конечной сдвиговой смеси нормальных законов,
т.\,е.\
\begin{equation}
\label{Stab_G} 
G(x)=\sum\limits_{i=1}^k p_i\Phi(x-a_i)\,,
\end{equation}
где
\begin{equation*}
\sum\limits_{i=1}^{k}p_i=1\,,\quad p_i\geqslant0\,,\enskip a_i\in\R\,,\enskip  i=1,\ldots,k,
\end{equation*}
а через $\Phi(\cdot)$ обозначена функция распределения
стандартного нормального закона
\begin{equation*}
\Phi(x)=\fr{1}{\sqrt{2\pi}}\int\limits_{-\infty}^x\exp{\left\{
-\fr{t^2}2\right\}}\,dt\,.
\end{equation*}

Также в дальнейшем будет использоваться функция плотности
стандартного нормального закона
$$
\varphi(x)=\fr{1}{\sqrt{2\pi}}\exp{\left\{ -\fr{x^2}2\right\}}\,.
$$

Очевидно, что функция распределения $G(x)$ из
соотношения~\eqref{Stab_G} может быть представлена в виде:
$$
G(x)=\E\Phi(x-V)\,,
$$
где $V$~--- дискретная случайная величина, принимающая значения
$a_i$ с вероятностями~$p_i$, т.\,е.\
\begin{equation}
\label{Stab_V} V:
\begin{tabular}{cccc}
$a_1$&$a_2$&$\cdots$&$a_k$\\
$p_1$&$p_2$&$\cdots$&$p_k$\,.
\end{tabular}
\end{equation}

Обозначим  через $\rho(F,G)$ равномерное расстояние между
функциями распределения $F(x)$ и~$G(x)$:
\begin{equation}
\label{Stab_ro}
\rho(F,G)=\sup\limits_{x\in\R}|F(x)-G(x)|\,.
\end{equation}

Известно, что для решения задачи устойчивости для конечных
сдвиговых смесей нормальных законов равномерная
метрика~\eqref{Stab_ro} является не вполне корректной (можно привести
пример весьма близких функций распределения, для которых
равномерная метрика будет давать расстояние, равное единице). Поэтому
необходимо рассматривать мет\-ри\-ки, метризующие слабую сходимость,
например метрику Леви~$L(F,G)$ между функциями распределения
$F(x)$ и~$G(x)$, определяемую соотношением:
\begin{multline*}
L(F,G)=\inf\left\{h:\, G(x-h)-h\leqslant{}\right.\\
\left.{}\leqslant F(x)\leqslant G(x+h)+h,
\forall x\in\R\right\}\,.
\end{multline*}

Модели добавления и расщепления компоненты могут быть
представлены в виде:
$$
G_p(x)=\E\Phi(x-V_p)\,,
$$
где дискретная случайная величина~$V_p$ определяется для каждой из
моделей по-раз\-но\-му. Необходимо получить соотношения, связывающие
расстояния Леви между смешивающими распределениями и смесями.
Перейдем к рассмотрению каждой из моделей.

\section{Модель добавления компоненты}

Модель добавления компоненты формализуется следующим образом.
Предполагается, что каждое из независимых наблюдений
$\mathbf{X}_n\hm=(X_1,\ldots,X_n)$ имеет распределение,
представимое в виде:
\begin{equation}
\label{Stab_G_p1}
G_p(x)=(1-p)\sum\limits_{i=1}^k
p_i\Phi(x-a_i)+p\Phi(x-a)\,,
\end{equation}
где все величины $a_i\in\R$, $p_i\hm\geqslant 0$, $i\hm=1,\ldots,k,$ считаются
известными, а~$a$ и~$p$ являются параметрами модели, при этом
$a\hm\in\R$, $0\hm\leqslant p\hm\leqslant 1$. Без ограничения общности для
определенности будем считать, что выполнены соотношения
\begin{equation}
\label{Stab_a1}
a_0\leqslant a\leqslant a_1\leqslant a_2\leqslant\cdots\leqslant a_k\,.
\end{equation}
%
Левое неравенство означает достаточно естест\-вен\-ное для практики
предположение, что рас\-смат\-ри\-ва\-ют\-ся конечные математические
ожидания. Поэтому в дальнейшем считаем~$a_0$ известным
параметром модели (так как он может быть указан из некоторых разумных
предположений для каждого конкретного случая).

В модели добавления компоненты дискретная случайная величина~$V_p$
имеет следующий вид:
\begin{equation}
\label{Stab_V_p1} 
V_p:\!\!
\begin{tabular}{ccccc}
$a$\!&$a_1$\!&$a_2$\!&$\cdots$\!\!&$a_k$\\
$p$\!&$p_1(1-p)$\!&$p_2(1-p)$\!&$\cdots$\!\!&$p_k(1-p)$.
\end{tabular}\!\!\!
\end{equation}

Отметим, что расстояние Леви $L(V,V_p)$ не превосходит величины~$p$,
так как расстояние между ступеньками функций распределения
составляет в точности $p$ на сегменте $[a,a_1]$ и $p p_i$ на сегментах
$[a_i,a_{i+1}]$, $i=1,\ldots,k-1$. Изменяться могут лишь параметры~$a$ и
$p$, величины $a_i$, $p_i$, $i\hm=1,\ldots,k,$ считаем постоянными. Однако
при фиксированном параметре $p$ и при $a\hm\to a_1$
очевидно, что $L(V,V_p)$ к нулю не стремится. Таким образом,
без ограничения общности считаем, что $0\hm\leqslant p\hm\leqslant a_1- a$. Поэтому
\begin{equation}
\label{Stab_Cond1} 
L(V,V_p)=p\,.
\end{equation}

Тогда справедлива следующая теорема.

\medskip

\noindent
\textbf{Теорема 1.}
\textit{В~рамках модели добавления компоненты~\eqref{Stab_G_p1} при
выполнении условий~\eqref{Stab_a1} и~\eqref{Stab_Cond1} расстояние
Леви $L(V,V_p)$ между смешивающими распределениями~$V$ из
соотношения~\eqref{Stab_V} и~$V_p$ из соотношения~\eqref{Stab_V_p1}
и расстояние Леви $L(G,G_p)$ между истинным распределением~$G(x)$
из соотношения~\eqref{Stab_G} и приближающей смесью $G_p(x)$ из
соотношения~\eqref{Stab_G_p1} связывают неравенства}
\begin{multline*}
C_1^{[1]}(a_0,a_k)L(G,G_p)\leqslant L(V,V_p)\leqslant{}\\
{}\leqslant
C_2^{[1]}(a_0,a_k)L^{1/2}(G,G_p)\,, 
\end{multline*}
\textit{где коэффициенты $C_j^{[1]}(a_0,a_k)$, $j=1,2$, зависящие
только от известных величин $a_k$ и $a_0$, имеют вид:}
\begin{align}
C_1^{[1]}(a_0,a_k)&=\max
\left\{1,\fr{\sqrt{2\pi}}{a_k-\min\{0,a_0\}}\right\}\,;
\label{Stab_C11}\\
C_2^{[1]}(a_0,a_k)&=\varphi^{-1/2}\left(a_k+|a_k|-{}\right.\notag\\
&\left.{}-\min\{0,a_0\}\right)
\left(1+\fr{1}{\sqrt{2\pi}}\right)^{1/2}\,.
\label{Stab_C12}
\end{align}


\medskip

\noindent
Д\,о\,к\,а\,з\,а\,т\,е\,л\,ь\,с\,т\,в\,о\,.\ 
Запишем оценки снизу для равномерного расстояния между функциями
распределения $G(x)$ и $G_p(x)$, воспользовавшись формулой
Лагранжа:
\begin{multline}
\rho(G,G_p)=\sup\limits_x|G(x)-G_p(x)|={}\\
{}=\sup\limits_x|G(x)-
G(x)+p(G(x)-\Phi(x-a))|={}\\
{}=p\sup\limits_x|G(x)-\Phi(x-a)|\geqslant
p|G(x_0-a_i)-\Phi(x_0-a)|={}\\
{}=p\left|\sum\limits_{i=1}^k
p_i\left(\Phi(x_0-a_i)-\Phi(x_0-a)\right)\right|={}\\
\!\!{}=\!p\left|\sum\limits_{i=1}^k
p_i(a-a_i)\varphi(\theta_i(x_0-a_i)\!+\!(1-\theta_i)(x_0-a))\right|\!={}\\
{}=p\left|\sum\limits_{i=1}^k p_i(a_i-a)\varphi(x_0-a-\theta_i(a_i-a))\right|\,.
\label{rho}
\end{multline}

Неравенство в соотношении~\eqref{rho} справедливо для любого~$x_0$.
Выберем значение данной величины так, чтобы воспользоваться
свойством монотонного убывания плотности стандартного нормального
распределения~$\varphi(x)$ от положительного аргумента. А~именно
потребуем выполнения условия
$$
x_0-a-\theta_i(a_i-a)\geqslant 0\,,
$$
откуда следует (с учетом того, что выражение в скобках
в силу условий~\eqref{Stab_a1} неотрицательно и $0\hm\leqslant\theta_i\hm\leqslant 1$),
что
\begin{equation}
x_0\geqslant a_i 
\label{x0Cond}
\end{equation}
сразу для всех номеров~$i$. Тогда в качестве  $x_0$ возьмем
величину
\begin{equation}
x_0=a_k+|a_k|\,. 
\label{x0}
\end{equation}

Очевидно, что условие~\eqref{x0Cond} выполняется,
при этом $x_0\hm\geqslant 0$ и $x_0-a\hm\geqslant 0$. Тогда, продолжая~\eqref{rho} с
учетом соотношений~\eqref{Stab_a1} и~\eqref{Stab_Cond1}, получим:
\begin{multline*}
\rho(G,G_p)\geqslant{}\\
{}\geqslant p\left|\sum\limits_{i=1}^k
p_i(a_i-a)\varphi\left(a_k+|a_k|-a-\theta_i(a_i-a)\right)\right|\geqslant{}\\
{}\geqslant p\left|\sum\limits_{i=1}^k p_i(a_i-a)\varphi\left(a_k+|a_k|-a\right)\right|
\geqslant{}\\
{}\geqslant p\left|\sum\limits_{i=1}^k
p_i(a_i-a)\varphi\Big(a_k+|a_k|-\min\{0,a_0\}\Big)\right|\geqslant{}\\
{}\geqslant p\sum\limits_{i=1}^k p_i(a_1-a)\varphi\left(a_k+|a_k|-\min\{0,a_0\}\right)
={}\\
{}=p(a_1-a)\varphi\left(a_k+|a_k|-\min\{0,a_0\}\right)\geqslant{}\\
{}\geqslant L^2(V,V_p)\varphi\left(a_k+|a_k|-\min\{0,a_0\}\right)\,.
\end{multline*}

Воспользуемся известным неравенством для мет\-ри\-ки Леви (см.,
например, книгу~\cite{Zolotarev1986}):
\begin{multline}
\label{Stab_Lrho} 
L(G,G_p)\leqslant\rho(G,G_p)\leqslant{}\\
{}\leqslant (1+\max\limits_x
G'(x))L(G,G_p)\,.
\end{multline}

Воспользуемся правым неравенством из соотношения~\eqref{Stab_Lrho}.
Имеем
\begin{multline*}
L^2(V,V_p)\varphi\left(a_k+|a_k|-\min\{0,a_0\}\right)\leqslant\rho(G,G_p)\leqslant{}\\
{}\leqslant (1+
\max\limits_x G'(x))L(G,G_p)={}\\
{}=\left(1+\max\limits_x \left(\sum\limits_{i=1}^k
p_i\varphi(x-a_i) \right)\right)L(G,G_p)\leqslant{}\\
{}\leqslant
\left(1+\sum\limits_{i=1}^k
p_i \fr {1}{\sqrt{2\pi}}\right)L(G,G_p)={}\\
{}=\left(1+\fr{1}{\sqrt{2\pi}}\right)L(G,G_p)\,.
\end{multline*}

Окончательно получаем следующую оценку сверху для
$L(V,V_p)$:
\begin{multline*}
L(V,V_p)\leqslant\varphi^{-1/2}\left(a_k+|a_k|-\min\{0,a_0\}\right)\times{}\\
{}\times
\left(1+\fr{1}{\sqrt{2\pi}}\right)^{1/2}L^{1/2}(G,G_p)={}\\
{}=
C_2^{[1]}(a_0,a_k)L^{1/2}(G,G_p)\,.
\end{multline*}

Оценка снизу для $L(V,V_p)$ может быть найдена из соотношений
\begin{multline*}
L(G,G_p)\leqslant\rho(G,G_p)=\sup\limits_x|G(x)-G_p(x)|={}\\
{}=p\sup\limits_x\left|\sum\limits_{i=1}^k
p_i(\Phi(x-a_i)-\Phi(x-a))\right|
\leqslant{}\\
{}\leqslant
 p\sup\limits_x\sum\limits_{i=1}^k
p_i|\Phi(x-a_i)-\Phi(x-a)|\leqslant{}\\
{}\leqslant p\sum\limits_{i=1}^k
p_i\sup\limits_x|\Phi(x-a_i)-\Phi(x-a)|\leqslant{}\\
{}\leqslant
p\sum\limits_{i=1}^k p_i= L(V,V_p)\,.
\end{multline*}

Однако можно провести оценивание и другим путем. Найдем точки
экстремума функции $\Phi(x-a)\hm-\Phi(x-a_i)$ из условия
$$
\varphi(x-a)-\varphi(x-a_i)=0\,.
$$

Максимум достигается в точке
$$
x_i^*=\fr{a+a_i}{2}\,.
$$

\pagebreak

\noindent
Тогда, учитывая четность функции $\varphi(x)$, получим:
\begin{multline*}
p\sum\limits_{i=1}^k
p_i\sup\limits_x|\Phi(x-a_i)-\Phi(x-a)|\leqslant{}\\
{}\leqslant
p\sup\limits_x\sum\limits_{i=1}^k
p_i|\Phi(x-a_i)-\Phi(x-a)|\leqslant{}\\
{}\leqslant p\sup\limits_x\sum\limits_{i=1}^k
p_i|\Phi(x_i^*-a_i)-\Phi(x_i^*-a)|
={}\\
{}= p\sum\limits_{i=1}^k
p_i(a_i-a)\varphi\left(\theta(x_i^*-a)+(1-\theta)(x_i^*-a_i)\right)\leqslant{}\\
{}\leqslant p\fr{a_k-\min\{0,a_0\}}{\sqrt{2\pi}}
=L(V,V_p)\fr{a_k-\min\{0,a_0\}}{\sqrt{2\pi}}\,.
\end{multline*}
%
Окончательно
\begin{multline*}
L(V,V_p)\geqslant\max\left\{1,\fr{\sqrt{2\pi}}{a_k-\min\{0,a_0\}}\right\}
L(G,G_p)={}\\
{}=C_1^{[1]}(a_0,a_k)L(G,G_p)\,.~~\square
\end{multline*}


\medskip

Рассмотрим следующее обобщение модели~\eqref{Stab_G_p1}. Пусть
имеется еще одна смесь данного типа, отличающаяся
от~\eqref{Stab_G_p1} только весом, т.\,е.\ (при этом $0\hm\leqslant q\hm\leqslant 1$)
\begin{equation}
\label{Stab_G_q1}
G_q(x)=(1-q)\sum\limits_{i=1}^k
p_i\Phi(x-a_i)+q\Phi(x-a)\,.
\end{equation}

Для $G_q(x)$ дискретная случайная величина $V_q$ имеет следующий
вид:
\begin{equation}
\label{Stab_V_q1} 
V_q:
\begin{tabular}{ccccc}
$a$&$a_1$&$a_2$&$\cdots$&$a_k$\\
$q$&$p_1(1-q)$&$p_2(1-q)$&$\cdots$&$p_k(1-q)$\,.
\end{tabular}
\end{equation}

Рассуждая как описано выше, получим, что $|p-q|\hm\leqslant a_1-a$. В~этом
случае расстояние Леви $L(V_p,V_q)$ примет вид:
\begin{equation}
\label{Stab_Cond1q}
L(V_p,V_q)=|p-q|\,.
\end{equation}

Тогда справедлива следующая теорема.

\medskip

\noindent
\textbf{Теорема~2.}
\textit{В рамках модели добавления
компоненты~\eqref{Stab_G_p1} при выполнении
условий~\eqref{Stab_a1} и~\eqref{Stab_Cond1q} расстояние
Леви $L(V_p,V_q)$ между смешивающими распределениями $V_p$ из
соотношения~\eqref{Stab_V_p1} и $V_q$ из
соотношения~\eqref{Stab_V_q1} и расстояние Леви $L(G_p,G_q)$
между распределениями $G_p(x)$ из соотношения~\eqref{Stab_G_p1} и
$G_q(x)$ из соотношения~\eqref{Stab_G_q1} связывают неравенства:}
\begin{multline*}
%\label{Stab_Ineqv1q}
C_1^{[1]}(a_0,a_k)L(G_p,G_q)\leqslant L(V_p,V_q)\leqslant{}\\
{}\leqslant
C_2^{[1]}(a_0,a_k)L^{1/2}(G_p,G_q)\,,
\end{multline*}
\textit{где коэффициенты $C_j^{[1]}(a_0,a_k)$, $j\hm=1,2$, зависящие
только от известных величин $a_k$ и $a_0$, определяются
формулами}~\eqref{Stab_C11} и~\eqref{Stab_C12}.

\medskip


\noindent
Д\,о\,к\,а\,з\,а\,т\,е\,л\,ь\,с\,т\,в\,о\,.\ 
Рассуждая аналогично доказательству теоремы~1,
найдем оценки снизу для равномерного расстояния между функциями
распределения $G_p(x)$ и~$G_q(x)$. Имеем:
\begin{multline*}
\rho(G_p,G_q)={}\\
{}=\sup\limits_x|(q-p)\sum\limits_{i=1}^k
p_i\Phi(x-a_i)+(p-q)\Phi(x-a)|={}\\
{}=|p-q|\sup\limits_x\left|\sum\limits_{i=1}^k
p_i\Phi(x-a_i)-\Phi(x-a)\right|\geqslant{}\\
{}\geqslant |p-q|\left|\sum\limits_{i=1}^k
p_i(\Phi(x-a_i)-\Phi(x-a))\right|\geqslant{}\\
{}\geqslant L^2(V_p,V_q)\varphi\left(a_k+|a_k|-\min\{0,a_0\}\right)\,.
\end{multline*}

Оценим максимум производной для функций~$G_p$ и~$G_q$. Запишем
выражения, например, для функции~$G_p$ (для функции~$G_q$ оценка
получается аналогично). Имеем:
\begin{multline*}
\max\limits_x G_p'(x)={}\\
{}=\max\limits_x \left((1-p)\sum\limits_{i=1}^k
p_i\varphi(x-a_i)+p\varphi(x-a)\right)\leqslant{}\\
{}\leqslant
\fr{1-p}{\sqrt{2\pi}}\sum\limits_{i=1}^k p_i+
\fr {p}{\sqrt{2\pi}}=\fr{1}{\sqrt{2\pi}}\,.
\end{multline*}

Пользуясь правым неравенством в формуле~\eqref{Stab_Lrho},
приходим к следующему результату:
\begin{multline*}
L(V_p,V_q)\leqslant\varphi^{-1/2}\left(a_k+|a_k|-\min\{0,a_0\}\right)\times{}\\
{}\times
\left(1+\fr{1}{\sqrt{2\pi}}\right)^{1/2}L^{1/2}(G_p,G_q)
={}\\
{}=C_2^{[1]}(a_0,a_k)L^{1/2}(G_p,G_q)\,.
\end{multline*}

Оценка снизу для $L(V_p,V_q)$ может быть найдена из следующих
соотношений:
\begin{multline*}
L(G_p,G_q)\leqslant\rho(G_p,G_q)={}\\
{}=|p-q|\sup\limits_x\left|\sum\limits_{i=1}^k
p_i\Phi(x-a_i)-\Phi(x-a)\right|\leqslant{}\\
{}\leqslant |p-q|\sup\limits_x\sum\limits_{i=1}^k
p_i|\Phi(x-a_i)-\Phi(x-a)|\leqslant{}
\end{multline*}

\noindent
\begin{multline*}
{}\leqslant |p-q|\sum\limits_{i=1}^k
p_i\sup\limits_x|\Phi(x-a_i)-\Phi(x-a)|
\leqslant{}\\
{}\leqslant |p-q|\sum\limits_{i=1}^k p_i= L(V_p,V_q)\,.
\end{multline*}
Аналогично доказательству теоремы~1 получим:
\begin{multline*}
\hspace*{-7.22379pt}L(V_p,V_q)\geqslant\max\left\{1,\fr{\sqrt{2\pi}}{a_k-\min\{0,a_0\}}\right\}
L(G_p,G_q)={}\\
{}=C_1^{[1]}(a_0,a_k)L(G_p,G_q)\,.~~\square
\end{multline*}


\section{Модель расщепления компоненты}

Модель расщепления компоненты формализуется следующим образом.
Предполагается, что каждое из независимых наблюдений
$\mathbf{X}_n\hm=(X_1,\ldots,X_n)$ имеет распределение,
представимое в виде:
\begin{multline}
\label{Stab_G_p2}
G_p(x)=\sum\limits_{i=1}^{k-1}
p_i\Phi(x-a_i)+{}\\
{}+(p_k-p)\Phi(x-a_k)+p\Phi(x-a)\,,
\end{multline}
где все величины $a_i\hm\in\R$, $0\hm\leqslant p_i\hm\leqslant 1$, $i\hm=1,\ldots,k,$ считаются
известными, $a$ и $p$ являются параметрами модели, при этом
$0\hm\leqslant p\hm\leqslant p_k$. Без ограничения общности для
определенности будем считать, что выполнены соотношения:
\begin{equation}
\label{Stab_sigma2}
a_1\leqslant a_2\leqslant\cdots\leqslant a_{k-1}\leqslant a\leqslant a_k\,.
\end{equation}

Для данной модели дискретная случайная величина $V_p$ имеет вид:
\begin{equation}
\label{Stab_V_p2} 
V_p:
\begin{tabular}{ccccc}
$a_1$&$a_2$&$\cdots$&$a$&$a_k$\\
$p_1$&$p_2$&$\cdots$&$p$&$p_k-p$\,.
\end{tabular}
\end{equation}

Воспользовавшись геометрической интерпретацией расстояния Леви,
можно получить, что
\begin{equation}
\label{Stab_L2} 
L(V,V_p)=\min\{a_k-a,p\}\,.
\end{equation}

В этой ситуации оба условия: $a\hm\to a_k$ при фиксированном
параметре~$p$ и $p\hm\to 0$ при фиксированном~$a$~--- влекут
справедливость соотношения $L(V,V_p)\hm\to 0$. Тогда справедлива
следующая тео\-рема.

\medskip

\noindent
\textbf{Теорема~3.}
\textit{В~рамках модели расщепления
компоненты~\eqref{Stab_G_p2} при выполнении
условий~\eqref{Stab_sigma2} расстояние Леви $L(V,V_p)$
из соотношения~\eqref{Stab_L2} между смешивающими
распределениями $V$ из соотношения~\eqref{Stab_V} и $V_p$ из
соотношения~\eqref{Stab_V_p2} и расстояние Леви $L(G,G_p)$ между
истинным распределением $G(x)$ из соотношения~\eqref{Stab_G} и
приближающей смесью $G_p(x)$ из соотношения~\eqref{Stab_G_p2}
связывают неравенства:}
\begin{multline*}
C_1^{[2]}(a_{k-1},a_k)L(G,G_p)\leqslant
L(V,V_p)\leqslant{}\\
{}\leqslant
 C_2^{[2]}(a_{k-1},a_k)L^{1/2}(G,G_p)\,,
\end{multline*}
\textit{где коэффициенты $C_j^{[2]}(a_{k-1},a_k)$, $j=1,2$, не зависят
от величин $a$, $p$ и имеют вид:}
\begin{align}
\label{Stab_C21}
C_1^{[2]}(a_{k-1},a_k)&=\fr{\sqrt{2\pi}}{\max\{1,a_k-a_{k-1}\}}\,,\\
\label{Stab_C22}
C_2^{[2]}(a_{k-1},a_k)&=\varphi^{-1/2}
\left(a_k+|a_k|-\min\{0,a_{k-1}\}\right)\times{}\notag\\
&\hspace*{20mm}{}\times \left(1+\fr{1}{\sqrt{2\pi}}\right)^{1/2}\,.
\end{align}


\medskip

\noindent
Д\,о\,к\,а\,з\,а\,т\,е\,л\,ь\,с\,т\,в\,о\,.\ 
Запишем оценки снизу для равномерного расстояния между функциями
распределения $G(x)$ и $G_p(x)$, воспользовавшись формулой
Лагранжа, свойством монотонного убывания плотности стандартного
нормального распределения $\varphi(x)$ от положительного аргумента и
соотношениями~\eqref{x0}, \eqref{Stab_sigma2} и~\eqref{Stab_L2}:
\begin{multline*}
\rho(G,G_p)=\sup\limits_x|G(x)-G_p(x)|={}\\
{}=\sup\limits_x\left|\sum\limits_{i=1}^k
p_i\Phi(x-a_i)-\sum\limits_{i=1}^k
p_i\Phi(x-a_i)+{}\right.\\
\left.{}+p\Phi(x-a_k)-p\Phi(x-a)
\vphantom{\sum\limits_{i=1}^k}\right|={}\\
{}=p\sup\limits_x|\Phi(x-a_k)-\Phi(x-a)|\geqslant{}\\
{}\geqslant
p|\Phi(x_0-a_k)-\Phi(x_0-a)|={}\\
{}=p|(a_k-a)\varphi(\theta(x_0-a_k)+(1-\theta)(x_0-a))|\geqslant{}\\
{}\geqslant p(a_k-a)\varphi\left(a_k+|a_k|-\min\{0,a_{k-1}\}\right)\geqslant{}\\
{}\geqslant
L^2(V,V_p)\varphi\left(a_k+|a_k|-\min\{0,a_{k-1}\}\right)\,.
\end{multline*}

Чтобы оценить сверху $L(V,V_p)$, воспользуемся правым
неравенством из соотношения~\eqref{Stab_Lrho} и найденной в
доказательстве теоремы~1 оценкой для
максимума производной, а также неравенствами~\eqref{Stab_sigma2}.
Имеем:
\begin{multline*}
L^2(V,V_p)\varphi\left(a_k+|a_k|-\min\{0,a_{k-1}\}\right)\leqslant{}\\
{}\leqslant
\left(1+\fr{1}{\sqrt{2\pi}}\right)L(G,G_p)\,,
\end{multline*}
%
откуда

\noindent
\begin{multline*}
L(V,V_p)\leqslant\varphi^{-1/2}
\left(a_k+|a_k|-\min\{0,a_{k-1}\}\right)\times{}\\
{}\times
\left(1+\fr{1}{\sqrt{2\pi}}\right)^{1/2}L^{1/2}(G,G_p)={}\\
{}=C_2^{[2]}(a_{k-1},a_k)L^{1/2}(G,G_p)\,.
\end{multline*}

Выпишем оценку снизу для $L(V,V_p)$. С этой целью заметим, что
\begin{multline}
\label{Inqv}
\hspace*{-5.16743pt}L(G,G_p)\leqslant\rho(G,G_p)=p\sup\limits_x|\Phi(x-a_k)-\Phi(x-a)|
={}\\
{}=p\sup\limits_x\left(\Phi(x-a)-\Phi(x-a_k)\right)\,.
\end{multline}


Найдем точки экстремума функции $\Phi(x-a)\hm-\Phi(x-a_k)$ из условия
$$
\varphi(x-a)-\varphi(x-a_k)=0\,.
$$

Максимум достигается в точке:
$$
x^*=\fr{a+a_k}{2}\,.
$$

Подставляя это значение в~\eqref{Inqv}, получим (учитывая четность
функции~$\varphi(x)$)
\begin{multline*}
p\sup\limits_x\left(\Phi(x-a)-\Phi(x-a_k)\right)
={}\\
{}=p\left(\Phi(x^*-a)-\Phi(x^*-a_k)\right)={}\\
{}=p(a_k-a)\varphi\left(\theta(x^*-a)+(1-\theta)(x^*-a_k)\right)={}\\
{}= p(a_k-a)\varphi\left((a_k-a)\left\vert\theta-\fr{1}{2}\right\vert\right)\leqslant{}\\
{}\leqslant L(V,V_p)\max\{p,a_k-a\}\fr{1}{\sqrt{2\pi}}
\leqslant{}\\
{}\leqslant L(V,V_p)\max\{1,a_k-a_{k-1}\}\fr{1}{\sqrt{2\pi}}\,.
\end{multline*}
%
Окончательно
\begin{multline*}
L(V,V_p)\geqslant\fr{\sqrt{2\pi}}{\max\{1,a_k-a_{k-1}\}}L(G,G_p)
={}\\
{}=C_1^{[2]}(a_{k-1},a_k)L(G,G_p)\,.~\square
\end{multline*}

\medskip

Рассмотрим следующее обобщение модели~\eqref{Stab_G_p2}. Пусть
имеется еще одна смесь данного типа, отличающаяся
от~\eqref{Stab_G_p2} только весом, т.\,е.\ (при этом $0\hm\leqslant q\hm\leqslant p_k$)
\begin{multline}
\label{Stab_G_q2}
G_q(x)=\sum\limits_{i=1}^{k-1}
p_i\Phi(x-a_i)+{}\\
{}+(p_k-q)\Phi(x-a_k)+q\Phi(x-a)\,.
\end{multline}

Для $G_q(x)$ дискретная случайная величина~$V_q$ имеет вид

\noindent
\begin{equation}
\label{Stab_V_q2} 
V_q:
\begin{tabular}{ccccc}
$a_1$&$a_2$&$\cdots$&$a$&$a_k$\\
$p_1$&$p_2$&$\cdots$&$q$&$p_k-q$
\end{tabular}\,.
\end{equation}

Воспользовавшись геометрической интерпретацией расстояния Леви,
можно получить, что
\begin{equation}
\label{Stab_L2q} 
L(V_p,V_q)=\min\{a_k-a,|p-q|\}\,.
\end{equation}

Тогда справедлива следующая теорема.

\medskip

\noindent
\textbf{Теорема~4.}
\textit{В~рамках модели расщепления
компоненты~\eqref{Stab_G_p2} при выполнении
условий~\eqref{Stab_sigma2} расстояние Леви
$L(V_p,V_q)$ из соотношения~\eqref{Stab_L2q} между смешивающими
распределениями $V_p$ из соотношения~\eqref{Stab_V_p2} и $V_q$ из
соотношения~\eqref{Stab_V_q2} и расстояние Леви $L(G_p,G_q)$
между распределениями $G_p(x)$ из соотношения~\eqref{Stab_G_p2} и
$G_q(x)$ из соотношения~\eqref{Stab_G_q2} связывают неравенства:}
\begin{multline*}
%\label{Stab_Ineqv2q}
C_1^{[2]}(a_{k-1},a_k)L(G_p,G_q)\leqslant
L(V_p,V_q)\leqslant{}\\
{}\leqslant
 C_2^{[2]}(a_{k-1},a_k)L^{1/2}(G_p,G_q)\,,
\end{multline*}
\textit{где коэффициенты $C_j^{[2]}(a_{k-1},a_k)$, $j=1,2$, не зависят от
величин $a$, $p$ и определяются формулами~\eqref{Stab_C21}
и}~\eqref{Stab_C22}.


\medskip

\noindent
Д\,о\,к\,а\,з\,а\,т\,е\,л\,ь\,с\,т\,в\,о\,.\ 
Рассуждая аналогично доказательству теоремы~3,
найдем оценки снизу для равномерного расстояния между функциями
распределения $G_p(x)$ и $G_q(x)$. Имеем:
\begin{multline*}
\rho(G_p,G_q)=\sup\limits_x|G_p(x)-G_q(x)|={}\\
{}=|p-q|\sup\limits_x|\Phi(x-a_k)-\Phi(x-a)|\geqslant{}\\
{}\geqslant
p|\Phi(x_0-a_k)-\Phi(x_0-a)|\geqslant{}\\
{}\geqslant L^2(V_p,V_q)\varphi\left(a_k+|a_k|-\min\{0,a_{k-1}\}\right)\,.
\end{multline*}

Оценим максимум производной для функций~$G_p$ и~$G_q$. Имеем:
\begin{multline*}
\max\limits_x G_p'(x)=\max\limits_x\left(\sum\limits_{i=1}^{k-1}
p_i\varphi(x-a_i)+{}\right.\\
\left.{}+(p_k-p)\varphi(x-a_k)+p\varphi(x-a)
\vphantom{\sum\limits_{i=1}^{k-1}}\right)\leqslant{}\\
{}\leqslant \fr{1}{\sqrt{2\pi}}\sum\limits_{i=1}^{k-1}
p_i+\fr{(p_k-p)}{\sqrt{2\pi}}+\fr{p}{\sqrt{2\pi}}
=\fr{1}{\sqrt{2\pi}}\,.
\end{multline*}

Пользуясь правым неравенством в формуле~\eqref{Stab_Lrho},
приходим к следующему результату:
\begin{multline*}
L(V_p,V_q)\leqslant\varphi^{-1/2}\left(a_k+|a_k|-\min\{0,a_{k-1}\}\right)\times{}\\
{}\times
\left(1+\fr{1}{\sqrt{2\pi}}\right)^{1/2}L^{1/2}(G_p,G_q)={}\\
{}=C_2^{[2]}(a_{k-1},a_k)L^{1/2}(G_p,G_q)\,.
\end{multline*}

Оценка снизу для $L(V_p,V_q)$ может быть найдена из следующих
соотношений:
\begin{multline*}
L(G_p,G_q)\leqslant\rho(G_p,G_q)={}\\
{}=|p-q|\sup\limits_x|
\Phi(x-a_k)-\Phi(x-a)|\,.
\end{multline*}

Повторяя рассуждения из доказательства
теоремы~3, получаем, что
\begin{multline*}
L(V_p,V_q)\geqslant\fr{\sqrt{2\pi}}{\max\{1,a_k-a_{k-1}\}}L(G,G_p)
={}\\
{}=C_1^{[2]}(a_{k-1},a_k)L(G_p,G_q)\,.~~\square
\end{multline*}


\section{Выводы}

В рамках двух рассмотренных моделей возмущений параметров смеси~---
моделей добавления и расщепления компоненты~--- получены оценки
устойчивости смесей нормальных законов по отношению к изменениям
смешивающего параметра. Для каждой из моделей получены
двусторонние оценки, связывающие расстояния Леви между смесями и
смешивающими законами. Данные оценки, в частности, являются
количественными характеристиками идентифицируемости конечных
сдвиговых смесей нормальных законов.

В то же время доказанные
теоремы~1--4 уста\-нав\-ли\-ва\-ют
взаимно однозначное соответствие между значением параметра веса и
числом компонент в смеси. Данный результат удобно использовать при
построении асимптотически наиболее мощных критериев для моделей
добавления и расщепления компоненты для случая произвольных
конечных сдвиг-мас\-штаб\-ных смесей в качестве обоснования вида гипотез
в задаче статистической проверки чис\-ла компонент смеси вероятностных
распределений (подробнее об этом см.\ в
работах~\cite{Gorshenin2011, Gorshenin2011MUCMC}).

{\small\frenchspacing
{%\baselineskip=10.8pt
\addcontentsline{toc}{section}{Литература}
\begin{thebibliography}{9}


\bibitem{Gorshenin2011} 
\Au{Бенинг~В.\,Е., Горшенин~А.\,К.,
Королев~В.\,Ю.}
Асимптотически оптимальный критерий проверки гипотез о числе
компонент смеси вероятностных распределений~// Информатика и её
применения, 2011. Т.~5. Вып.~3. C.~4--16.

\bibitem{Gorshenin2011Mod2}\Au{Горшенин~А.\,К.}
Проверка статистических гипотез в модели расщепления компоненты~//
Вестник Московского университета. Сер.~15. Вычисл. матем. и киберн.,
2011. №\,4. С.~26--32.

\bibitem{Gorshenin2012SSI} \Au{Горшенин~А.\,К.}
Устойчивость масштабных смесей нормальных законов по отношению к
изменениям смешивающего распределения~// Системы и средства
информатики, 2012. Т.~22. Вып.~1. С.~136--148.

\bibitem{Zolotarev1986} 
\Au{Золотарев~В.\,М.} Современная теория
суммирования независимых случайных величин.~--- М.: Наука, 1986. 417~с.

\label{end\stat}

\bibitem{Gorshenin2011MUCMC}\Au{Gorshenin~A.\,K.}
Testing of statistical hypotheses in the splitting component
model~// Moscow University Computational Mathematics and
Cybernetics, 2011. Vol.~35. No.\,4. P.~176--183.
 \end{thebibliography}
}
}


\end{multicols}