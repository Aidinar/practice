\def\stat{dulin}

\def\tit{ОБРАБОТКА ГЕОПРОСТРАНСТВЕННОЙ ИНФОРМАЦИИ НА~БАЗЕ  
РЕПОЗИТОРИЯ ГЕОИНФОРМАЦИОННОЙ СИСТЕМЫ}

\def\titkol{Обработка геопространственной информации на~базе  
репозитория геоинформационной системы}

\def\autkol{С.\,К.~Дулин, И.\,Н.~Розенберг, В.\,И.~Уманский}
\def\aut{С.\,К.~Дулин$^1$, И.\,Н.~Розенберг$^2$, В.\,И.~Уманский$^3$}

\titel{\tit}{\aut}{\autkol}{\titkol}

%{\renewcommand{\thefootnote}{\fnsymbol{footnote}}\footnotetext[1]
%{Работа выполнена при поддержке РФФИ (гранты 09-07-12098, 09-07-00212-а и
%09-07-00211-а) и Минобрнауки РФ (контракт №\,07.514.11.4001).}}


\renewcommand{\thefootnote}{\arabic{footnote}}
\footnotetext[1]{Институт проблем информатики Российской академии наук, s.dulin@ccas.ru}
\footnotetext[2]{Научно-исследовательский и проектно-конструкторский институт информатизации, автоматизации и связи 
на железнодорожном транспорте (ОАО <<НИИАС>>), I.Rozenberg@gismps.ru}
\footnotetext[3]{ЗАО <<ИнтехГеоТранс>>, umanvi@yandex.ru}




   \Abst{Эффективная интеграция данных, описывающих взаимодействующие компоненты,~--- 
ключ к успешному управ\-ле\-нию всей системой функционирующих объектов. Нехватка интеграции 
данных может привести к существенной неэффективности оперативных, тактических и 
   долгосрочных стратегий управления. Интегрированные системы управления могут помочь 
преодолеть эту неэффективность и улучшить координацию и рентабельность решений. 
Интеграция данных~--- безусловно, самое ответственное мероприятие для осуществления 
успешных стратегий управления.
   В работе описан подход к использованию централизованного репозитория корпоративных 
данных, позволяющего объединить пространственные и непространственные данные описания 
объектов. Репозиторий представляется средой для совместного использования и объединения 
данных и используемых программных продуктов.}
   
   \KW{управление объектами; интеграция данных; ГИС; репозиторий}
   
   \vskip 14pt plus 9pt minus 6pt

      \thispagestyle{headings}

      \begin{multicols}{2}

            \label{st\stat}

    
\section{Введение}
      
      Информационное обеспечение системы функционирования объектов должно 
учитывать много сложных, взаимозависимых и перегруженных процессов. 

Успешная 
реализация стратегий управ\-ле\-ния в значительной степени определяется: совместным 
использованием и управлением данными жизненного цикла объектов и способностью 
поддерживать\linebreak и координировать процессы многофункциональной работы на тактическом 
и стратегическом уровнях. Недостаток интеграции данных при управ\-ле\-нии объектами 
может привести к существенной\linebreak неэффективности. Интегрированные сис\-те\-мы управ\-ле\-ния 
объектами могут преодолеть эту неэффективность и улучшить координацию и 
рентабельность решений управ\-ле\-ния~[1].
      
      Данные описания объектов характеризуются большим объемом, сложностью, 
взаимозависимостями и динамизмом. Эти данные могут существовать в несопоставимых 
форматах из-за наличия разнообразных источников данных и программных систем. 
Объединение этих данных в согласованную и унифицированную форму считается 
критическим мероприятием для успешного управления~[2, 3].
      
      Большинство разработанных в последние десятилетия инструментальных средств 
функционируют как автономные системы с ограниченной возмож\-ностью совместного 
использования информации с другими системами. Быстрое увеличение числа этих 
инструментальных средств создало так называемые <<острова информации>> и появление 
противоречивых моделей данных в несопоставимых программных продуктах.
      
      Эффективная интеграция данных может значительно улучшить рентабельность 
принятия решений при управлении объектами на оперативном, тактическом и 
стратегическом уровнях.\linebreak
 В~результате интеграции данных достигаются: 
пригодность/доступность данных; своевременность; точность, корректность и целостность; 
согласованность и ясность; завершенность; сокращение дуб\-ли\-ро\-ва\-ния; ускорение 
обработки и сокращение времени ожидания; уменьшение стоимости сбора и хранения 
данных; всесторонняя обоснованность решения и интегрированное принятие 
решений~[4].
      
      Даная статья посвящена обоснованию необходимости использования репозитория 
как интегратора разнородных ресурсов геоинформационной\linebreak системы (ГИС) и 
формулированию требований к архитектуре информационных ресурсов ГИС с\linebreak учас\-тием 
репозитория. 

План изложения материала следующий: в разд.~2 описывается роль 
централизованного репозитория\linebreak в формировании геоинформационного пространства; в 
разд.~3 обсуждаются возможности репозитория в определении пространственных 
классов\linebreak ограничения це\-лост\-ности геоданных, являющиеся важным аспектом управ\-ле\-ния 
качеством геоданных, а также архитектура и концептуальная модель репозитория, 
обеспечивающая контроль ограничений це\-лост\-ности со стороны широкого круга 
пользователей; в разд.~4 рассматривается композиционный подход к 
конструированию репозитория на основе формирования системы компонентных объектов 
ГИС, и завершается раздел обсуждением принятой в настоящее время архитектуры 
обработки геоданных с участием репозитория. 

\section{Репозиторий~--- интегратор данных о состоянии объектов 
и~процессов}
      
      Пространственные данные составляют ядро большинства ГИС и оказывают 
наибольшее влияние на многие процессы принятия решений при управлении объектами. 
Данные объектов железнодорожного транспорта всегда идентифицируются своим 
географическим местоположениям и пространственными отношениями, поэтому ГИС и 
пространственный анализ данных являются важнейшими средствами поддержки 
процессов управ\-ле\-ния объектами.
      
      В большинстве реализаций ГИС до настоящего времени пространственные данные 
сохранялись и обрабатывались в персональных или ведомственных базах геоданных, 
которые ограничивали совместное использование и редактирование данных. 
Возрастающие требования к совместной обработке пространственных данных для 
различных приложений выявили острую потребность в масштабируемости ГИС и 
создании геоинформационного пространства~\cite{1dul, 5dul}. 

В~данной статье обсуждается 
возможность усиления роли ГИС в поддержке развития интегрированных систем 
управления объектами на базе централизованного репозитория. Под репозиторием здесь 
понимается пред\-мет\-но-ориен\-ти\-ро\-ван\-ная информационная корпоративная база данных, 
специально разработанная для поддержки принятия решений. Репозиторий строится на 
базе кли\-ент-сер\-вер\-ной архитектуры, реляционной системы управления базами
данных (СУБД) и утилит поддержки принятия 
решений. 
      
      Ключевой задачей совершенствования средств обработки геоинформационного 
контента является создание геоинформационного пространства, позволяющего 
осуществить интеграцию про\-стран\-ст\-вен\-но-рас\-пре\-де\-лен\-ной информации (семантической, 
метрической и топологической), с которой имеет дело ГИС, с данными высокоточного 
спутникового позиционирования, объединив их в единой геоинформационной базе 
данных отрасли. Это позволит представить все виды геоинформационных ресурсов 
отрасли в виде геореляционных структур, рассматривать их во взаимосвязи и оперативно 
получать информацию необходимого вида и содержания. 
      
      Создание геоинформационного пространства с концептуальной точки зрения 
может быть разделено на два основных направления:
      \begin{enumerate}[(1)]
\item разработка принципов интеграции геопространственной информации на 
растровой и/или векторной основе с присоединенными базами фактографических 
данных, метаданных, данных высокоточного спутникового позиционирования, а также 
пространственно-опре\-де\-лен\-ной вербальной информации;
\item разработка принципов и подходов к многоуровневому семантическому 
моделированию и согласованной интеграции геопространственной и 
пространственно-определенной вербальной информации с одновременным 
использованием электронных знаковых форм представления геотекстов, а также 
растровой и/или векторной формы их представления.
      \end{enumerate}
      
      Геоинформационное пространство, интегрированное в прикладные 
геоинформационные и автоматизированные системы, способствует решению ряда задач, 
комплексно повышая уровень безопасности, бесперебойности и надежности 
функционирования железнодорожного транспорта.
      
      Можно выделить три основных сегмента потребителей данных 
геоинформационного про\-стран\-ства:
\begin{enumerate}[(1)] 
\item  задачи проектирования, 
реконструкции и текущего содержания объектов;
\item задачи управления 
процессами, базирующимися на различных технологиях, в том числе на технологиях 
спутниковых радионавигационных систем (СРНС); 
\item обеспечение сис\-тем безопасности 
функционирования координатной информации, которая служит вторичным 
ин\-фор\-ма\-ци\-он\-но-управ\-ля\-ющим контуром сис\-тем. 
\end{enumerate}
      
      Каждый из этих сегментов предполагает разработку специальных методов 
интеграции геоинформационных ресурсов, обеспечивающих поддержку технологий 
управления железнодорожным транспортом. Более подробно с существующими 
железнодорожными технологиями можно ознакомиться в~\cite{6dul}.
      
      Актуализация геоинформационного пространства, поддерживаемого 
инструментальной средой ГИС, в современных условиях предполагает разработку и 
сопровождение централизованных репозиториев, основанных на интегрированных 
моделях\linebreak\vspace*{-12pt}

\pagebreak

\end{multicols}

\begin{figure} %fig1
 \vspace*{1pt}
 \begin{center}
 \mbox{%
 \epsfxsize=161.258mm
 \epsfbox{dul-1.eps}
 }
 \end{center}
 \vspace*{-9pt}
\Caption{Роль репозитория как интегратора данных о состоянии объектов и процессов}
\end{figure}

\begin{multicols}{2}

\noindent
 данных. Эти модели данных описывают характеристики объектов 
инфраструктуры и результаты функционирования подсистем объектов, отражая тем 
самым различные аспекты жизненного цикла объектов (рис.~1). Репозиторий может 
потенциально улучшить эффективность, рентабельность и координацию различных 
процессов управления объектами~\cite{7dul}.
 

     
      Централизованный репозиторий данных об объектах на основе ГИС построен как 
внешний мо-\linebreak дуль реляционной СУБД и поэтому может обеспе\-чить широкий диапазон 
совместного использования данных, интеграции и сервисов управления,\linebreak типа управления 
версиями, многопользовательского параллельного доступа и редактирования, 
безопас\-ности и авторизации, а также сервисов метаданных. Репозиторий должен 
гарантировать согласованность и целостность данных и объединить различные форматы 
данных во всестороннее и непротиворечивое представление всей системы 
инфраструктуры. Обеспечивая единственный вход для доступа ко всем данным объектов, 
репозиторий может значительно улучшить сбор, организацию, управление и 
распределение данных всюду по жизненному циклу объекта. Интерфейс ГИС с 
репозиторием расширит возможности пользователей, позволив точно определить свои 
требования, сделать запрос и проанализировать данные объектов в пространственном 
контексте.
      
      Репозиторий может обеспечить поддержку управления взаимозависимыми 
подсистемами инфраструктуры транспортных сетей интегрированным способом. Этими 
подсистемами инфраструктуры обычно управляют сепарабельно, при этом данные 
хранятся в отдельных и, возможно, несовместимых базах данных. Централизованный 
репозиторий может поддерживать перекрестные ссылки и отношения между различными 
подсистемами инфраструктуры, обеспечивающими взаимодействие между 
      cо-расположенными или накладывающимися объектами. Различные рабочие 
группы при этом будут в состоянии объединиться на основе согласованного 
представления данных.
      
      Репозиторий допускает функциональную совместимость и эффективное 
совместное использование данных несовместимыми подсистемами управления объектами. 
Репозиторий может также поддержать использование распределенных источников данных 
и обеспечить доступ к этим источникам локально или через Ин\-тер\-нет/Интра\-нет (в 
доступной через сеть архитектуре кли\-ент--сер\-вер) распределенным приложениям-кли\-ен\-там. 
Можно перечислить некоторые другие преимущества централизованных 
репозиториев:
      \begin{itemize}
\item эффективное хранение, индексация, запрос и анализ данных объекта с 
параллельным поиском и редактированием этих данных для множественных 
приложений;
\item поддержка координации и упрощение процессов управления объектами, 
уве\-ли\-чи\-ва\-ющая операционную эффективность и рас\-ши\-ря\-ющая коммуникации между 
различными департаментами и заинтересованными пользовате\-лями;
\item возможность многократного совместного использования данных и, таким 
образом, устранение дублирования усилий, потенциальной несогласованности и 
избыточности при сборе, верификации и хранении данных об объектах;
\item использование согласованных, интегрированных и стандартизированных 
моделей данных и форматов;
\item легко выполнимая интеграция инструментальных средств программного 
обеспечения, предварительно сформированных в виде отдельных модулей, в единую 
инструментальную среду.
\end{itemize}

\section{Проектирование репозитория для~управления метаданными 
ограничений целостности геоданных}
      
      Один из недостатков существующих коммерческих ГИС, предоставляющих 
настраиваемый набор услуг,~--- невозможность обеспечить адекватную операционную 
среду для конечных пользователей, которые являются экспертами в своей прикладной\linebreak 
области, но обладают минимальным опытом в настройке программного обеспечения и 
формулировании требований к проекту базы геоданных~\cite{6dul}.\linebreak
Такие пользователи 
лишены возможности использования многих из особенностей коммерческой ГИС. Другой 
недостаток~--- отсутствие средств наложения ограничений целостности данных, что 
ставит под угрозу качество геоданных. Проектирование базы геоданных с контролем 
целостности при помощи существующих инструментальных средств требует знания 
некоторого языка сценариев. Конечные пользователи редко обладают этим типом знания.
      
      Репозиторий ГИС предоставляет конечным пользователям средство для 
определения подмножества пространственных классов ограничения целостности 
геоданных без потребности в программировании~\cite{5dul, 7dul}.
      
      Если первые шаги в разработке базы геоданных для ГИС ставили в качестве 
главной цели приобретение данных и размещение их в релевантное место в сис\-те\-ме, то в 
настоящее время акцент смещается в сторону эффективной организации и анализа 
геоданных, хотя никто не отрицает важности совершенствования технологии сбора 
геоданных. На ранней стадии разработки непространственных сис\-тем управления базами 
данных контролю целостности было уделено недостаточно внимания. В~результате этого 
и пространственные наборы данных создаются с сомнительным качеством данных, что 
приводит к соответствующим результатам анализа, выполненного с использованием 
таких данных.
      
      Учитывая высокую стоимость фиксации данных в формате ГИС, уже недостаточно 
неавтоматизированным способом обрабатывать определяемые пользователем ограничения 
целостности, которые влияют на качество данных. 

\begin{figure*} %fig2
 \vspace*{1pt}
 \begin{center}
 \mbox{%
 \epsfxsize=114.043mm
 \epsfbox{dul-2.eps}
 }
 \end{center}
 \vspace*{-9pt}
\Caption{Интегрированная пространственная среда программирования}
\vspace*{6pt}
\end{figure*}
      
      Есть два основных подхода к проблеме качества данных. Первый~--- уменьшить 
ошибки в пространственных наборах данных, второй~--- вооружить пользователей 
знанием качества наборов данных и их содержания. Чтобы устанавливать ограничения 
целостности до ввода данных, необходимо включить ограничения целостности в язык 
определения данных базы данных так, чтобы они автоматически контролировались во 
время выполнения загрузки данных. Метаданные для этого процесса~--- словарь 
метаданных, описывающих характеристики, отношения и структуры. Каталог метаданных 
описывает происхождение и качество данных. Со-\linebreak\vspace*{-12pt}

\pagebreak



\noindent
поставление информации типа <<кто 
ввел данные?>> и <<каково их происхождение?>> представляет собой важную задачу при 
вводе данных для пользова-\linebreak теля. 
      
      Репозиторий метаданных, который поддерживает хранение и обновление каталога 
метаданных, важен при проектировании геопространственных систем. Ограничения 
целостности могут быть неявно включены при проектировании, что позволит 
автоматически включить их в спроектированную систему. Пространственные и 
непространственные отношения также сохраняются вместе с ограничениями целостности и 
данными, касающимися преобразования объектов. 
      
      Репозиторий активизируется при вводе геоданных, контролируя соблюдение 
ограничений на данные или же фиксируя такие нарушения в файле регистрации 
нарушений.
      
      Важно различать ГИС и пространственную инфор\-ма\-ци\-он\-ную системную среду 
разработки. Дополнительная функциональность часто обеспечивается для разработчика 
приложения ГИС объединением ГИС с внешними программами или написанием 
специальной прикладной программы. 

Особую важность в современных исследованиях 
представляет инструмент разработки ГИС, который позволяет пользователю 
специфицировать семантические ограничения на бинарные топологические отношения без 
потребности в программировании. Проектирование при этом не зависит от конкретной 
ГИС, но может быть связано с одной из них. Оно должно помочь пользователям в 
разработке пространственных приложений; один и тот же инструмент может 
использоваться, чтобы многократно строить проекты в различных масштабах. Одна из 
главных особенностей такого проектирования~--- создание репозитория для 
пространственных ограничений целостности, которые могут быть использованы во 
множестве приложений. Репозиторий отличается от различных словарей данных, 
поддерживающих ГИС, которые являются файлами или наборами файлов, 
ориентированных на одно конкретное приложение. 

Интегрированная пространственная 
среда разработки программного обеспечения проиллюстрирована на рис.~2. Ключевой 
элемент в этой среде~--- репозиторий метаданных, который является средством контроля 
всех проектных изменений так же, как и репозиторий инструментальных средств 
разработки программного обеспечения в непространственных средах разработки. 

\begin{figure*} %fig3
 \vspace*{1pt}
 \begin{center}
 \mbox{%
 \epsfxsize=150.726mm
 \epsfbox{dul-3.eps}
 }
 \end{center}
 \vspace*{-9pt}
\Caption{Пространственные ограничения целостности}
\end{figure*}

\begin{figure*}[b] %fig4
 \vspace*{1pt}
 \begin{center}
 \mbox{%
 \epsfxsize=101.181mm
 \epsfbox{dul-4.eps}
 }
 \end{center}
 \vspace*{-9pt}
\Caption{Архитектура системы управления пространственными данными}
\end{figure*}


      При проектировании репозитория формулируются следующие цели:
      \begin{enumerate}[1.]
\item Исследовать спецификацию правил и ограничений целостности в существующих 
пространственных и непространственных средах разработки.
\item Оценить существующую возможность инструментальных средств разработки ГИС 
пред\-став\-лять определяемые пользователем пространственные ограничения целостности.
\item Определить методы регистрации и применения этих ограничений средствами, 
доступными для конечных пользователей.
\item Проверить эффективность этих методов.
      \end{enumerate}
      
      Репозиторий хранит и контролирует подмножество классов ограничения 
целостности. Ограничения фиксируются двумя способами: посредством предложений 
языка определения данных СУБД и интеграцией с существующим программным 
обеспечением ГИС. Репозиторий хранит элементы модели геоданных, или данных о 
геоданных, средства генерации логических моделей данных из этих метаданных и 
поддержки проекта базы геоданных. В~частности, должно быть средство для включения 
ограничений целостности в базу данных разрабатываемой системы. Метаданные качества 
и происхождения также сохранены в репозитории. 
      
      Проблеме качества пространственных данных, связанной с ограничениями 
целостности, в последнее время уделяется большое внимание. Ограничения целостности 
структурируются, как показано на рис.~3. Они подразделяются соответственно трем типам 
возможных ошибок: структурным, гео\-мет\-рическим и 
      то\-по-се\-ман\-ти\-че\-ским. 
      
      То\-по-се\-ман\-ти\-че\-ские ограничения подразделяются 
на семантические ограничения и опреде\-ля\-емые пользователем ограничения целостности. 

      Рисунок~4 иллюстрирует архитектуру предлагаемого репозитория. В~качестве 
инструментальных средств можно использовать любой тип инструментальных средств, 
доступных в интегрированной среде разработки. Репозиторий управляет всем 
программным обеспечением в среде и представляет собой интерфейс между 
пользователем и инструментальными средствами. 

\begin{figure*} %fig5
 \vspace*{1pt}
 \begin{center}
 \mbox{%
 \epsfxsize=127.755mm
 \epsfbox{dul-5.eps}
 }
 \end{center}
 \vspace*{-9pt}
\Caption{Модель данных репозитория}
\vspace*{6pt}
\end{figure*}

      Рисунок~5 представляет концептуальную модель репозитория, хранящего 
данные на ме\-та\-уров\-не. 
При проектировании неявно предполагается, что геометрия может 
иметь много {графических} представлений. Визуализация спецификаций вклю\-че\-на, чтобы 
позволить пользователю определить графические стили для объектов в ГИС, предполагая, 
что наличие визуальных команд вызова программы улучшило бы ясность относительно 
того, что собой представляют объекты, и таким образом уменьшило бы ошибку ввода 
данных. Это задача, которая непосредственно контролируется репозиторием. Для этого 
используются метаданные, касающиеся схематического изображения и использующиеся 
для генерации схемы, со\-хра\-ня\-емой на том же самом метауровне.
      
      Сущность диаграммы является внутренней для репозитория и содержит 
пространственную диаграмму связей. Логическая схема репозитория содержит таблицы 
хранения каталога метаданных, возвращаемых репозиторию от ГИС через программный 
интерфейс Open Database Connectivity (ODBC) доступа к данным, и файл нарушений 
корректности (целостности и согласованности). Правило (атрибута) иллюстрирует 
концептуальную модель хранения атрибутивных правил. 



      Центральная часть изображения модели репозитория содержит основную 
информацию для представления сущности и ее отношений. Репозиторий обрабатывает 
проекты через отношение сущность--проект этой части модели. Все записи в базе 
геоданных отмечены идентификатором проекта посредством значения внешнего ключа в 
сущности. В~рабочем режиме репозиторий нужен для того, чтобы 
проекты совместно использовали допустимые правила. Но это требует дальнейшего 
развития модели, так как при этом необходима обработка отношений <<многие к многим>> 
между проектом и сущностью. 
      
      Топологические ограничения неявно поддерживаются средствами ГИС, с которой 
связан репозиторий. Правила представлены как предложения языка определения данных в 
языке структурированных запросов Structured Query Language (SQL) для правил атрибута 
и как структурированный текст для правил отношений (связи).


      Система репозитория проектируется, чтобы облегчить разработку системы не 
вполне под\-го\-тов\-лен\-ны\-ми пользователями. При этом главная задача~--- обеспечить 
спецификацию реальных объектов пользователями при существующих ограничениях на 
способ, которым данные об этих объектах могут быть введены. Эти ограничения задаются, 
чтобы управлять качеством данных. Система репозитория обеспечивает пользователям 
интерфейс, позволяющий устанавливать статические ограничения целостности на значения 
атрибута и определяемые пользователем ограничения целостности на пространственные 
отношения. Они автоматически\linebreak\vspace*{-12pt}

\pagebreak

\end{multicols}

\begin{figure} %fig6
 \vspace*{1pt}
 \begin{center}
 \mbox{%
 \epsfxsize=156.638mm
 \epsfbox{dul-6.eps}
 }
 \end{center}
 \vspace*{-9pt}
\Caption{Структура меню системы репозитория}
%\vspace*{6pt}
\end{figure}

\begin{multicols}{2}

\noindent
 перетранслируются в предложения языка определения 
данных или ограничения, выраженные как запросы к ГИС. Таким образом, не приходится 
нагружать программированием пользователей. Интерфейс репозитория также обеспечивает 
шлюз к ГИС. Управление в этом случае переходит к ГИС. Геоинформационная система 
способна реагировать на 
результат, который указывает, что имеет место нарушение правила. Такие нарушения при 
вводе геоданных приводят к откату входных данных или фиксируются в файле нарушений 
коррект\-ности репозитория. Кроме того, могут быть собраны автоматически и также 
предоставляться репозиторием метаданные об авторе, дате, пространственных границах, 
масштабе, проекции и системе коор\-динат.
{\looseness=1

}
      
      На рис.~6 изображена структура меню системы репозитория. Цель операции по 
выбору проекта состоит в том, чтобы находить, вводить или изменять основные детали 
проекта. Эта операция начинается непосредственно после инициализации системы 
репозитория. Однако она может быть задействована на любой стадии выполнения проекта, 
когда пользователь пожелает перейти на другой проект. При этом вводятся детали проекта, 
включая координаты и масштаб.



      Описание базы геоданных и триггеров (встроенных процедур) позволяет 
пользователю (1)~описывать реальные объекты, их признаки и их графическое 
представление и (2)~генерировать таб\-ли\-цы, представляющие эти объекты. Есть два 
обеспечивающих ее процесса: ввод/редактирование метаданных и создание/обновление 
объектов. В~первом случае пользователь вводит имя сущности, ее атрибуты и правила, 
которые применяются к ним, ее геометрию и графический стиль, связанный с ней. 
{\looseness=1

}
      
      Функция ввода топологического бизнес-пра\-ви\-ла дает возможность пользователю 
определить\linebreak пользовательские правила, ограничивающие от\-ношения, в которых сущности 
могут принять\linebreak учас\-тие~\cite{8dul}. Первоначально создается правило, основанное на 
сущностях, вовлеченных в связь и являющихся непосредственно связью. 
      
      Если заданы метаданные, которые сохраняются согласно существующим таблицам, 
то данные о правилах собираются на основе существующих в базе геоданных ограничениях 
целостности объектов. Сообщение об ошибке формируется во время производимого 
пользователем выбора. Условия, на которых базируются ограничения целостности, 
сохранены в двух областях в репозитории. В~первой условия сохраняются в соответствии с 
определенными атрибутами, а во второй~--- в соответствии с пространственными 
отношениями. 
      
      Когда пользователю требуются средства представления пространственных объектов, 
репозиторий инициирует запуск ГИС со всеми таблицами для выбранного открытого 
проекта. Репозиторий управляет всеми топологическими ограничениями и атрибутивными 
условиями. Ограничения проверяются ГИС при добавлении каждого пространственного 
объекта. Кроме того, при каждой транзакции проверяются все атрибутивные условия и 
топологические ограничения и в случае ошибок формируется соответствующее сообщение.
      
      Существующие стандарты метаданных в пространственных информационных 
системах прежде всего отражены в каталоге метаданных. Существенная выгода 
использования подхода на основе репозитория в том, что репозиторий является активным и 
при разработке системы, и при ее эксплуатации. 
      
Геоинформационная система 
обеспечивает сообщения (отчеты) о сущностях/атрибутах и дает список всех 
сущностей в проекте, их атрибутов и ограничений на эти атрибуты. Файл нарушений 
корректности геоданных дает список всех ошибок, которые произошли с 
пользовательскими правилами, идентификатор рас\-смат\-ри\-ва\-емо\-го объекта и его 
координаты. 
      
      Сообщения, обеспечиваемые репозиторием, включают также сообщения о 
топологическом правиле и о правиле атрибута. Топологическое сообщение о 
правиле включает все топологические правила проекта и условия выполнения этих правил. 
Сообщение о правиле атрибута включает атрибуты проектных сущностей, правила, 
связанные с ними, и текстовое правило, которое по\-став\-ля\-ет\-ся с сообщением об ошибке.

\section{Конструирование системы репозитория компонентных объектов 
геоинформационной системы}
      
      Поскольку основной поток программной продукции в последнее время 
ориентирован на раз\-работку отдельных модулей, удовле\-тво\-ря\-ющих\linebreak требованиям модели 
компонентных объектов,
 программная индустрия в ГИС также становится модульно-ориентированной, обеспечивая как крупномасштабное развитие приложений ГИС, так и 
создание небольших гибких производственных систем. Для эффективной разработки 
таких систем необходимо глубокое исследование не только разработки компонентных 
объектов, но и управления ими.
      
      Компонентные объекты ГИС перед каталогизацией в репозитории должны быть 
классифицированы на основе их специфики. Поэтому необходима система регистрации и 
поиска компонентов ГИС, в результате которой каталогизированный компонент позволит 
прикладным разработчикам найти нужный компонент или создать новый компонент 
посредством модификации и комбинации существующих компонентов и системы в целом. 
Основная цель разработки репозитория компонентных объектов ГИС состоит в том, чтобы 
обеспечить возможность использования во многих приложениях и функциональную 
совместимость разработанных компонентных объектов ГИС.
      
      Чтобы реализовать ГИС более эффективно с точки зрения стоимости и времени, 
разработчики при конструировании репозитория компонентных объектов ГИС уделяют 
основное внимание возможности многоцелевого использования компонентных объектов и 
их функциональной совмести\-мости. Это обусловлено тем, что \mbox{большинство}\linebreak проектов 
ГИС представляет собой настройку на конкретное приложение типа административной 
информационной сис\-те\-мы, интеллектуальной информационной сис\-те\-мы, кадастровой 
информационной сис\-те\-мы, сис\-те\-мы управ\-ле\-ния в чрезвычайных ситуациях и адаптивной 
производственной сис\-те\-мы. При этом каждое приложение нуждается в одних и тех же 
стандартных функциональных возможностях типа отображения геоданных и реализации 
запроса или специальных функциональных возможностях типа трехмерного средства 
просмотра и обработки данных сис\-те\-мы GPS (Global Positioning System). 
Поэтому если существует универсальный 
репозиторий, хранящий компонентные объекты ГИС, и сис\-тем\-ные проектировщики или 
разработчики в режиме реального времени могут определить, где необходимый 
компонентный объект расположен, то они могут легко получить его, модифицировать или 
присоединить к своей сис\-теме. 
      
      Система компонентных объектов репозитория ГИС для регистрации и нахождения 
компонентных объектов ГИС строится в зависимости от их метаданных.


      На рис.~7 показана общая концептуальная диаграмма системы компонентных 
объектов репози-\linebreak\vspace*{-12pt}

\pagebreak

\end{multicols}

\begin{figure} %fig7
 \vspace*{1pt}
 \begin{center}
 \mbox{%
 \epsfxsize=133.07mm
 \epsfbox{dul-7.eps}
 }
 \end{center}
 \vspace*{-9pt}
\Caption{Концепция исследования}
\vspace*{6pt}
\end{figure}

\begin{multicols}{2}

\noindent
тория ГИС. Пользователи вводят свои ключевые слова в интерфейсе 
поиска, и эти ключевые слова проверяются на соответствие логическому описанию СУБД 
через использование спецификаций метаданных компонентных объектов ГИС. 
Спецификации метаданных компонентных объектов ГИС подразделяются на три большие 
категории. 
\begin{enumerate}[1.]
  \item
 Первая категория~--- это компоненты источников геоданных ГИС, которые 
должны обеспечить функциональную совместимость пространственных форматов 
данных, находящихся в обращении в гетерогенной среде геоинформационного контента. 
\item Ко второй категории относятся компоненты функциональных возможностей ГИС, 
которые\linebreak
 могут использоваться как ядро ГИС при разработке определенного прикладного 
програм\-много обеспечения ГИС. При этом следует заметить, что существуют как 
специфические \mbox{функции}, так и функции, общие для всех ГИС, такие как вывод на дисплей 
карты и атрибутов геоданных, или же анализ геометрических характеристик, трехмерный 
анализ и авторизация. 
\item Третья категория~--- компоненты приложений ГИС, которые 
указывают соответствие нескольким областям применения ГИС, типа административной 
информационной системы, системы управ\-ле\-ния ресурсами, системы управ\-ле\-ния в 
чрезвычайных ситуациях, ин\-тел\-лек\-ту\-аль\-ной транспортной сис\-те\-мы, городской 
информационной системы или кадастровой информационной сис\-темы. 
\end{enumerate}



Эта 
функциональная архитектура классификации позволяет указать место компонентных 
объектов ГИС и каталогизировать их в репозитории. 
      
      Рисунок~8 демонстрирует основную идею сис\-тем\-но\-го проекта с участием 
репозитория. На нем показана логика реализации системы в плане описания 
последовательности действий регистрации и поиска компонентных объектов ГИС.
      
      В настоящее время данные о функционировании объектов передаются между 
различными программными инструментальными средствами, главным образом, двумя 
основными способами:

\end{multicols}

\begin{figure}[b] %fig8
 \vspace*{1pt}
 \begin{center}
 \mbox{%
 \epsfxsize=149.712mm
 \epsfbox{dul-8.eps}
 }
 \end{center}
 \vspace*{-9pt}
\Caption{Функциональная схема с участием репозитория}
\end{figure}



\begin{multicols}{2}


 

\noindent
\begin{enumerate}[(1)]
\item с помощью трансляторов~--- чтобы преобразовывать 
данные при переходах между различными специализированными моделями данных и 
форматами. Процесс трансляции при этом связан с громадным объемом избыточного 
поиска данных, интерпретацией и повторным вводом данных и, как известно, допускает 
ошибки интерпретации и отображения результатов. Кроме того, специфика трансляторов 
налагает ненужные ограничения, требуя использования специализированных моделей 
данных и программных систем;
\item с помощью ряда прикладных программных 
интерфейсов (API~--- applied program interface)~--- чтобы обратиться к внутренней модели данных приложения и 
вводить или извлекать данные непосредственно из приложения.\linebreak
 Хотя некоторые из этих 
API фактически соответствуют промышленному стандарту, многие API являются 
специально разработанными и ориентированными под конкретную\linebreak программную среду. 
Использование специально разработанных методов~--- очевидное препятствие на пути 
интеграции данных об объектах инфраструктуры.
\end{enumerate}

 Использование централизованных 
репозиториев, основанных на независимых от программной среды стандартных моделях 
данных,~--- по-ви\-ди\-мо\-му, самая жизнеспособная идея для интеграции данных управления 
объектами и программной функциональной совместимости.


      Главная задача при формировании централизованного репозитория данных~--- 
разработать\linebreak
 модель данных и соответствующую схему базы данных, чтобы представить 
данные жизненного цик\-ла объектов объединенным, всесторонним и пред\-почтитель\-но 
стандартизированным способом. 

Программные инструментальные средства, 
соответствующие стандартам, могут быть легко\linebreak интегрированы в репозиторий без 
необходимости разрабатывать специальные адаптеры для транслирования данных 
объектов в модель данных репозитория и, таким образом, будут способствовать развитию 
и развертыванию интегрированных систем управления объектами.
      
      В отличие от традиционных моделей автома\-тизированного проектирования 
модели ГИС обеспе\-чи\-вают определение и использование семантически богатых 
      объект\-но-ори\-ен\-ти\-ро\-ван\-ных моделей, которые поддерживаются реляционной СУБД 
(РСУБД), предназначенной для хранения и управ\-ле\-ния атрибутивными данными. 
Объединяя пространственные и непространственные данные,\linebreak модели ГИС допускают 
эффективную автоматизированную проверку корректности данных, гарантируя качество и 
надежность геоданных. Кроме того, архитектура кли\-ент--сер\-вер большинства ГИС дает 
возможность тонким клиентам эффективно обращаться к геоданным по сетям 
Интернет/Интранет, обеспечивая массовую публикацию пространственных данных в 
различных департаментах достаточно рентабельным способом. 


\begin{figure*} %fig9
 \vspace*{1pt}
 \begin{center}
 \mbox{%
 \epsfxsize=135.585mm
 \epsfbox{dul-9.eps}
 }
 \end{center}
 \vspace*{-9pt}
\Caption{Архитектура обработки геоданных с участием репозитория
(SDE~--- Spatial Database Engine)}
\vspace*{6pt}
\end{figure*}                    
      
      Модель базы геоданных дает возможность выполнить проверку ограничений 
целостности на геоданные и использовать функции SQL реляционной СУБД для доступа к 
геоданным, обновления и управ\-ле\-ния транзакциями. Кроме того, модель позволяет 
определять специальные объекты, которые воплощают определяемую пользователем 
семантику, а также поддерживает сложные пространственные отношения типа сетей, 
топологий и ландшафтов. 
     
     Сервер приложений ГИС (рис.~9) представляет собой интерфейс, который 
позволяет управлять пространственными данными и хранить их. Важнейшее его 
преимущество состоит в возможности совместного доступа (чтения, записи, 
обновления, удаления) к используемым данным. Он распределяет пространственные 
данные для различного рода приложений, а также поставляет пространственные данные 
через глобальные сети по протоколу TCP/IP (Transmission Control Protocol\,/\,Internet
Protocol). 
     
     Сервер приложений служит интерфейсом между ГИС и РСУБД для организации 
совместного доступа и управления пространственными данными как таблицами. 
В~среде разнотипных баз данных, созданных различными
организациями или отдельными пользователями, он обеспечивает общую 
модель хранения географической информации и значительно улучшает характеристики 
всей ГИС за счет распределения функций приложения ГИС между сервером базы 
данных и клиентом.


      Сервер приложений управляет набором заданных таблиц (или системным словарем 
данных), которые хранят метаданные о пространственных данных, таких как 
пространственные ссылки, имена классов признаков и структур и пространственную 
индексацию. 
      
      Репозиторий поддерживает версионирование базы геоданных, что обеспечивает 
слежение за хронологией обновления геоданных и откат до прежнего уровня изменений, 
если в этом возникает потребность. Чтобы оптимизировать использование ресурсов 
памяти, изменения хранятся только в дель\-та-таб\-ли\-цах. Эти таблицы используются вмес\-то 
копирования всей базы геоданных. Проведенные изменения в итоге приводят к одной 
версии, если все изменения согласованы. 

\section{Заключение}

      В большинстве реализаций ГИС до настоящего времени пространственные данные 
сохранялись и обрабатывались в персональных или ведомственных базах геоданных, 
которые ограничивали совместное использование и редактирование данных. 
Возрастающие требования к совместной обработке пространственных данных для 
различных приложений выявили острую потребность в масштабируемости ГИС и 
создании геоинформационного пространства. В~связи с этим следует обратить внимание 
на усиление роли ГИС в поддержке развития интегрированных систем управления 
объектами на базе централизованного репозитория.
      
      В данной статье описан подход к проектированию репозитория для управления 
пространственными бизнес-правилами и его роль в разработке интегрированной среды 
программного обеспечения для ГИС. Следует отметить, что метаданные являются больше, 
чем средством каталогизации наборов данных. Они могут также эффективно 
использоваться при проектировании базы геоданных. Через контролируемый ввод 
геоданных репозиторий имеет возможность улучшить качество данных ГИС. Репозиторий 
активен при эксплуатации сис\-те\-мы, проверяя ввод данных. Все нарушения ограничений 
регистрируются с выдачей сообщений о качестве геоданных. Репозиторий может 
обеспечить метаданные описания идентичности и происхождения наборов данных, 
введенных в систему. Эти два средства сообщения улучшают осведомленность о качестве 
рассматриваемого набора данных и поэтому предотвращают некорректное использование. 
Наконец, можно получить полный отчет о содержании репозитория, который помогает в 
администрировании базы данных. Качество геоданных в существующих ГИС часто 
невысокое. Существующие ГИС обеспечивают в лучшем случае только поддержку 
корректности ввода геоданных. Кроме того, инструментальные средства, используемые для 
разработки таких систем, не ориентированы на участие в разработке системы конечных 
пользователей. Основная цель представленной работы~--- определить главные особенности 
развития ГИС, допускающей конечных пользователей к учас\-тию в со\-зда\-нии 
интегрированной среды, которая поз\-во\-ли\-ла бы пользователям задавать их собственные 
ограничения и получать качественные отчеты, соответствующие стандартам на 
метаданные.

{\small\frenchspacing
{%\baselineskip=10.8pt
\addcontentsline{toc}{section}{Литература}
\begin{thebibliography}{9}

\bibitem{1dul}
\Au{Розенберг И.\,Н., Дулин С.\,К.}
Геоинформационный портал отрасли. Гарантировать достоверность данных~// 
Железнодорожный транспорт, 2010. №\,2. С.~12--17.


\bibitem{3dul} %2
\Au{Дулин С.\,К., Розенберг И.\,Н.}
Об одном подходе к структурной согласованности геоданных~// Мир транспорта, 2005. 
№\,3. С.~16--29.

\bibitem{2dul} %3
\Au{Дулина Н.\,Г., Уманский В.\,И.}
Структуризация проблемы улучшения пространственной согласованности баз 
геоданных.~--- М.: ВЦ РАН, 2009. 40~с.

\bibitem{4dul}
\Au{Дулин С.\,К., Розенберг И.\,Н.}
Согласованное пополнение геоинформационного портала неструктурированными 
данными~// Системы и средства информатики.~--- М.: Наука, 2005. Вып.~15. С.~194--218.

\bibitem{5dul}
\Au{Longley P.\,A., Goodchild M.\,F., Maguire~D.\,J., Rhind~D.\,W.}
Geographic information systems and science.~--- 2nd ed.~--- New York: Wiley, 2005.

\bibitem{6dul}
\Au{Розенберг И.\,Н., Цветков В.\,Я., Матвеев~С.\,И., Дулин~С.\,К.}
Интегрированная система управления железной дорогой~/ Под ред. В.\,И.~Якунина.~--- 
2-е изд., перераб. и доп.~--- М.: ИПЦ <<Дизайн. Информация. Картография>>, 2008. 
144~с.

\bibitem{7dul}
\Au{Baird M.\,P., Frome R.\,J.}
Large-scale repository design~// Cell Preservation Technol., 2005. Vol.~3. No.\,4. P.~256--266.


\label{end\stat}

\bibitem{8dul}
\Au{Orriens B., Yang J., Papazoglou~M.\,P.}
A framework for business rule driven service composition~// 4th  Workshop (International) TES 
2003~/ Eds. B.~Benatallah, M.-C.~Shan.~--- Springer, 2003. P.~14--27.
 \end{thebibliography}
}
}


\end{multicols}