

%\newcommand{\norm}[1]{\left\Vert#1\right\Vert}
\newcommand{\abs}[1]{\left\vert#1\right\vert}
\newcommand{\set}[1]{\left\{#1\right\}}
%\newcommand{\Real}{\mathbb R}
%\newcommand{\eps}{\varepsilon}
%\newcommand{\To}{\longrightarrow}
%\newcommand{\BX}{\mathbf{B}(X)}
%\newcommand{\A}{\mathcal{A}}

\def\stat{shest}

\def\tit{ВОССТАНОВЛЕНИЕ ВЕРОЯТНОСТНЫХ ХАРАКТЕРИСТИК
СЛУЧАЙНЫХ ФУНКЦИЙ В ЗАДАЧАХ ОДНОФОТОННОЙ
ЭМИССИОННОЙ ТОМОГРАФИИ}
\def\titkol{Восстановление вероятностных характеристик
случайных функций} % в задачах однофотонной
%эмиссионной томографии}

\def\autkol{В.\,Г.~Ушаков, О.\,В.~Шестаков}
\def\aut{В.\,Г.~Ушаков$^1$, О.\,В.~Шестаков$^2$}

\titel{\tit}{\aut}{\autkol}{\titkol}

{\renewcommand{\thefootnote}{\fnsymbol{footnote}}\footnotetext[1]
{Работа выполнена при
финансовой поддержке РФФИ, грант 08-01-00567.}}

\renewcommand{\thefootnote}{\arabic{footnote}}
\footnotetext[1]{Московский государственный университет им. М.\,В.~Ломоносова, 
кафедра математической статистики факультета ВМиК; 
Институт проблем информатики Российской академии наук, vgushakov@mail.ru}
\footnotetext[2]{Московский государственный университет им. М.\,В.~Ломоносова, 
кафедра математической статистики факультета ВМиК, oshestakov@cs.msu.su}

\Abst{В работе рассматривается задача восстановления
вероятностных характеристик объекта, случайным образом меняющего свою структуру в процессе регистрации
проекционных данных. В~рамках предложенной модели томографического эксперимента разрабатывается метод восстановления распределений случайной функции по распределениям проекций в случае, когда
случайная функция имеет не более чем счетное число состояний.}

\KW{однофотонная эмиссионная томография; стохастическая томография; экспоненциальное преобразование Радона; 
случайные функции; проекционные данные}

           \vskip 24pt plus 9pt minus 6pt

      \thispagestyle{headings}

      \begin{multicols}{2}

      \label{st\stat}

\section{Введение}

В задачах однофотонной эмиссионной томографии возникает проблема
обращения обобщенного преобразования Радона, имеющего вид
\begin{multline}
R_\mu(s,\theta)=\int\limits_{x\theta=s}f(x)e^{D\mu(x,\theta^{\perp})}\,dl\,,\\
\theta\in S^1\,,\ s\in\mathbf{R}\,,
\label{e1u}
\end{multline} 
где $f(x)$~--- непрерывная функция
с компактным носителем, имеющая смысл интенсивности излучения,
интеграл берется вдоль прямой $x\theta=s$, $S^1$~---
множество направлений, задаваемых единичными векторами в
$\mathbf{R}^2$ с центром в начале координат, а~$D\mu(x,\theta^{\perp})$~--- весовая функция, равная
$$
D\mu(x,\theta^{\perp})=\int\limits_{0}^{\infty}\mu(x+t\theta^{\perp})\,dt\,.
$$
Здесь $\mu(x)$~--- известная функция с компактным носителем,
$\theta=(\cos\alpha,\sin\alpha)$, а
$\theta^{\perp}=(-\sin\alpha,\cos\alpha)$. Функция $-\mu(x)$ имеет
смысл коэффициента поглощения. Если $\mu(x)=0$, то
$R_\mu(s,\theta)$ превращается в классическое преобразование
Радона, а если~$\mu(x)$ равна константе на носителе $f(x)$,
$R_\mu(s,\theta)$ превращается в экспоненциальное преобразование
Радона. Формулы обращения для классического и экспоненциального
преобразований Радона получены довольно давно, однако вопрос
возможности обращения преобразования~(1) оставался открытым более
20~лет. В последние годы проблема была успешно решена, и были
получены различные формулы обращения~[1--3].

Часто в подобных задачах естественно считать функцию, описывающую интенсивность излучения, случайной.
При этом состояния (реализации) этой функции меняются во время
процесса получения проекций. Это приводит к тому, что
восстановление даже одной реализации случайной функции обычными
томографическими методами становится невозможно.

В работах~[4--8] рассматривается задача определения вероятностных характеристик двумерных случайных функций
по характеристикам одномерных проекций без учета поглощения и в
предположении, что коэффициент поглощения равен константе.
Показывается, что в общем эта задача характеризуется сильной
неоднозначностью, и если не накладывать ограничений на вид
реализаций случайной функции, то содержательные результаты удается
получить лишь в том случае, когда случайная функция имеет не более
чем счетное число состояний. В работах~\cite{6u, 8u} для класса таких
функций разрабатывается метод восстановления распределений
двумерных случайных функций.

В данной работе задача определения вероятностных характеристик случайных
функций рассматривается в предположении, что коэффициент
поглощения равен известной функции $-\mu(x)$, имеющей компактный
носитель и принадлежащей классу Гёльдера (с некоторым параметром).

\section{Постановка задачи}

Пусть стохастический
объект описывается двумерной случайной функцией $\xi(x)$. Предполагается выполнение следующих условий: 
\begin{enumerate}[(1)]
\item почти все значения
$\xi(x)$ непрерывны и интегрируемы, \item 
$\xi(x)$ имеет компактный
носитель (без потери общ\-ности будем считать, что этим носителем
является единичный круг: $U=\{x\in R^2:x_1^2+x_2^2\leq 1\}$).
\end{enumerate}
Функции, совпадающие всюду за исключением множеств нулевой
лебеговой меры, будем считать эквивалентными.

При выполнении этих условий определены проекции
функции $\xi(x)$ с учетом поглощения~--- одномерные случайные
функции вида
\begin{multline}
R_\mu\xi_{\theta}(s)=\int\limits_{x\theta=s}\xi(x)e^{D\mu(x,\theta^{\perp})}\,dl\,,\\
 \theta\in S^1,\ s\in\mathbf{R}\,.
\label{e2u}
\end{multline}

В задачах стохастической томографии предполагается, что
имеется некоторая информация о вероятностных характеритиках
учитывающих поглощение проекций (всех или некоторого множества).
Задача состоит в нахождении определенных вероятностных
характеристик случайной функции $\xi(x)$. Первый вопрос касается
однозначности соответствия между рассматриваемыми характеристиками
двумерной случайной функции и ее проекций. Как и в случае
отсутствия поглощения, этот вопрос может возникать, например, в
следующих формах:
\begin{enumerate}[1.]
\item  Можно ли однозначно определить
совместные распределения $\xi(x_1)$,\ \ldots , $\xi(x_n)$, если
известны совместные распределения $R_\mu\xi_{\theta}(s_1)$, \ldots ,
$R_\mu\xi_{\theta}(s_m)$ для всех $m=1$, 2,\ \ldots  и всех $\theta\in
S^1$?
\item Можно ли однозначно определить дис\-пер\-сии $\mathrm{var}\,(\xi(x))$, если известны дисперсии 
$\mathrm{var}\,(R_\mu\xi_{\theta}(s))$ для всех $\theta\in S^1$?
\item Можно
ли однозначно восстановить дис\-пе\-рсии $\mathrm{var}\,(\xi(x))$, если
известны все совместные распределения величин
$R_\mu\xi_{\theta}(s_1)$, \ldots , $R_\mu\xi_{\theta}(s_m)$ для всех
$m=1$, 2,\ \ldots  и всех $\theta\in S^1$?
\item  Имеется ли связь между
<<изменчивостью>> (величиной дисперсии) двумерной случайной
функции в некоторой точке и изменчивостью проекций в точках,
являющихся проекциями этой точки?
\end{enumerate}

Проблема восстановления математического
ожидания двумерной случайной функции по математическим ожиданиям
учитывающих поглощение проекций не рассматривается, по\-с\-ко\-ль\-ку, как легко
видеть, эта проблема эквивалентна обычной (нестохастической)
томографии: 
\begin{multline*}
{\rm E}R_\mu\xi_{\theta}(s)={\rm
E}\int\limits_{x\theta=s}\xi(x)e^{D\mu(x,\theta^{\perp})}\,dl={}\\
{}=
\int\limits_{x\theta=s}{\rm
E}\xi(x)e^{D\mu(x,\theta^{\perp})}\,dl\,.
\end{multline*}

\vspace*{-12pt}

\section{Общий случай}

В данном разделе будет приведено
несколько утверждений, относящихся к сформулированным выше вопросам~1--4 и показывающих, что в общем случае ответы
на, по крайней мере, вопросы~2 и~4 отрицательны. По-видимому, в
общем случае отрицательными будут ответы и на вопросы~1 и~3, хотя
пока не удалось построить соответствующие примеры.

\smallskip
\noindent
\textbf{Утверждение 1.} \textit{Существуют две случайные функции $\xi(x)$ и
$\eta(x)$, определенные в единичном круге
$x_{1}^{2}+x_{2}^{2}\leq1$, такие, что $\mathrm{var}\,(\xi(x))\not=\mathrm{var}\,(\eta(x))$ 
для всех $x$, удовлетворяющих неравенству
$x_{1}^{2}+x_{2}^{2}<1$, в то время как $\mathrm{var}\,(R_{\mu}\xi_{\theta}(s))=
\mathrm{var}\,(R_{\mu}\eta_{\theta}(s))$
для всех $\theta\in S^1$ и $s\in R^1$.}

\smallskip

В ряде задач бывает достаточно установить, в каких точках значения
реализаций двумерной случайной функции имеют большой разброс по
сравнению с другими точками. Другими словами, нет необходимости
восстанавливать численные значения дисперсии, а достаточно на
качественном уровне указать те точки, в которых значения дисперсии
относительно велики. В связи с этим возникает вопрос о связи
величины разброса значений реализаций в некоторой точке и
величинами разброса реализаций проекций с учетом поглощения в
проекциях этой точки. Следующие два утверждения показывают, что в
общем случае такая связь отсутствует.

\smallskip

\noindent
\textbf{Утверждение 2.} \textit{Существует случайная функция $\xi(x)$,
$x_1^2+x_2^2\leq 1$, такая, что все учитывающие поглощение
проекции наиболее изменчивой точки (точки, име\-ющей максимальную
дисперсию) являются наименее изменчивыми точками соответсвующих
одномерных учитывающих поглощение проекций функции $\xi(x)$.}

\smallskip
\noindent
\textbf{Утверждение 3.} 
\textit{Существует случайная функция $\xi(x)$,
$x_1^2+x_2^2\leq 1$, такая, что все учитывающие поглощение
проекции наименее изменчивой точки (точки, имеющей минимальную
дисперсию) являются наиболее изменчивыми точками соответсвующих
одномерных учитывающих поглощение проекций функции $\xi(x)$.}

\smallskip

Сформулированные утверждения справедливы даже в том случае, когда
коэффициент поглощения равен константе. Соответствующие
доказательства можно найти в работе~\cite{7u}.

\section{Класс $T$ случайных функций}

Как следует из предыдущего раздела, в случае, когда случайная
функция $\xi(x)$ произвольна, по-ви\-ди\-мо\-му, невозможно получить
какие-либо содержательные результаты. Однако ситуация изменится,
если наложить на нее некоторые ограничения. В данном разделе будет введен
 класс случайных функций, с одной стороны, достаточный для
многих приложений, а с другой~--- позволяющий единственным образом
восстанавливать вероятностные характеристики стохастических
объектов по характеристикам проекций.

Пусть $T$~--- множество всех двумерных случайных функций
$\xi(x)$ вида 
$$
\xi(x) = f_{\nu}(x)\,,
$$ 
где
$f_{1}(x),f_{2}(x),\;\ldots$~--- последовательность непрерывных
интегрируемых функций, определенных в единичном круге
$U=\{(x)\in\mathbf{R}^{2}\; : \;x_{1}^{2}+x_{2}^{2}\leq1\}$, а
$\nu$~--- случайная величина, принимающая целые положительные значения.

Вероятностная структура случайных функций из класса $T$ полностью определяется распределением,
т.\,е.\ набором 
$(f_{1}(x),f_{2}(x),\ldots;p_{1},p_{2},\ldots)$,
где $p_{i}=P(\xi(x) = f_{i}(x)),$ $i = 1$, 2,\ \ldots, 
$\sum\limits_{i=1}^{\infty}p_{i} = 1$. Распределение $\xi(x)$ будем
обозначать $P_\xi$.

Оказывается, что в рамках этой модели, как и в случае постоянного
коэффициента поглощения, рассмотренного в работе~\cite{7u},
распределение двумерной случайной функции полностью определяется
распределениями проекций. Однако, если для модели с постоянным
коэффициентом поглощения удалось разработать метод
реконструкции распределения двумерной случайной функции по
распределениям проекций~\cite{8u}, то в модели с неравномерным
поглощением пока удалось лишь установить принципиальную
возможность такой реконструкции.

\medskip
\noindent
\textbf{Теорема.} \textit{Пусть $\xi(x)\in T$, $\eta(x)\in T$ и
$P_{R_{\mu}\xi_{\theta}}=P_{R_{\mu}\eta_{\theta}}$ для всех
$\theta\in\Lambda$, где $\Lambda$~--- подмножество $S^1$, имеющее
положительную меру. Тогда}
$$
P_\xi=P_\eta\,.
$$

\medskip

Другими словами, в классе $T$ распределение любой двумерной
случайной функции однозначно определяется распределениями
проекций, зарегистрированных в любом сколь угодно узком диапазоне
углов обзора.

\medskip
\noindent
Д\,о\,к\,а\,з\,а\,т\,е\,л\,ь\,с\,т\,в\,о.\ \, 
Покажем, что если функция $f(x)$ непрерывна и имеет компактный носитель,
функция $\mu(x)$ принадлежит классу Гёльдера (с некоторым
параметром) и также имеет компактный носитель, а
$R_{\mu}f(s,\theta)=0$ для всех $s\in\mathbf{R}$ и всех $\theta\in
E$, где $E$~--- подмножество $S^1$, имеющее положительную меру, то
$f(x)$ тождественно равна нулю. (При доказательстве этого факта будем следовать идеям работы~[9].)

Разложим функцию
$$
D\mu(x,\theta^{\perp})=\int\limits_{0}^{\infty}\mu(x+t\theta)\,dt
$$
на четную и нечетную компоненты (по $\theta$):
$D\mu(x,\theta^{\perp})=u(x,\theta)+v(x,\theta)$, где
\begin{align*}
u(x,\theta)&=\fr{1}{2}(D\mu(x,\theta^{\perp})+D\mu(x,-\theta^{\perp}))\,;\\[6pt]
v(x,\theta)&=\fr{1}{2}(D\mu(x,\theta^{\perp})-D\mu(x,-\theta^{\perp}))\,.
\end{align*}
Легко видеть, что функция $2u(x,\theta)$ равна преобразованию
Радона функции $\mu(x)$, вычисленному в точке
$(s,\theta)=(x\cdot\theta,\theta)$. Поэтому эта функция постоянна на прямых
$x\cdot\theta=s$, и если $\mu(x)$ принадлежит классу Гёльдера и
имеет компактный носитель, то~$u(x,\theta)$ также принадлежит
классу Гёльдера. Для данной непрерывной функции $\psi(\theta)$
обозначим за~$\tilde{\psi}(\theta)$ такую функцию от $\theta$ с
нулевым средним, для которой функция
$\psi(\theta)+i\tilde{\psi}(\theta)$ является граничным значением
аналитической функции, определенной в единичном круге. Функция
$\tilde{\psi}(\theta)$ называется функцией, сопряженной к
$\psi(\theta)$, и ее можно вычислить с помощью свертки
$$
\tilde{\psi}(\theta)=\fr{1}{2\pi}\,\int\limits_{0}^{2\pi}\psi(\theta(\alpha-\beta))\ctg\fr{\beta}{2}\,d\beta\,,
$$
где $\theta(\alpha)=(\cos\alpha,\sin\alpha)$. По теореме Привалова~[10], 
если $\psi(\theta)$ принадлежит классу Гёльдера, 
то~$\tilde{\psi}(\theta)$ также принадлежит классу Гёльдера. Найдем
сопряженную функцию для нечетной компоненты~$v(x,\theta)$:
\begin{multline*}
\tilde{v}(x,\theta)=\fr{1}{2}\left( \tilde{D\mu}(x,\theta^{\perp})-\tilde{D\mu}\left(x,-\theta^{\perp}\right)\right)={}\\
{}=\fr{1}{4\pi}\int\limits_{0}^{2\pi}D\mu(x,\theta^{\perp}(\alpha-\beta))\ctg\fr{\beta}{2}\,d\beta-{}
\end{multline*}

\noindent
\begin{equation*}
\ \ \ \ \ \ \ \ \ \ \ \ \ \ \ \ \ \ {}-
\fr{1}{4\pi}\int\limits_{0}^{2\pi}D\mu(x,-\theta^{\perp}(\alpha-\beta))\ctg\fr{\beta}{2}\,d\beta\,,
\end{equation*}
где $\theta^{\perp}(\alpha)=(-\sin\alpha,\cos\alpha)$. Далее,
воспользовавшись формулой
$$
\fr{1}{\cos\beta}=\fr{1}{2}\left(\ctg\fr{\beta+\pi/2}{2}-\ctg\fr{\beta-\pi/2}{2}
\right)\,,
$$ 
получаем
\begin{multline*}
\tilde{v}(x,\theta)=\fr{1}{2\pi}\int\limits_{S^1}\fr{1}{\theta\omega}\,D\mu(x,\omega)\,d\omega={}\\
{}=
\fr{1}{2\pi}\int\limits_{S^1}\fr{1}{\theta \omega}\int\limits_{0}^{\infty}\mu(x+t\omega)\,dt\,.
\end{multline*}
Переходя в последнем интеграле к декартовым координатам
$x+t\omega=y$, имеем
\begin{multline*}
\tilde{v}(x,\theta)=-\fr{1}{2\pi}\int\limits_{\mathbf{R}^2}\fr{\mu(y)}{(x-y)\theta}\,dy={}\\
{}=
-\fr{1}{2\pi}\int\limits_{-\infty}^{\infty}\fr{1}{x\theta-s}\left(\ 
\int\limits_{\,y\theta=s}\mu(y)\,dl\right)\,ds={}\\
{}=-\fr{1}{2}\,HR\mu(x\theta,\theta)\,,
\end{multline*} 
где
$HR\mu(x\theta,\theta)$~--- преобразование Гильберта 
функции~$R\mu(\theta,s)$ по первой переменной, вычисленное в точке 
$(x\theta)$. (Преобразование Гильберта функции~$q(s)$ задается
формулой
$$
Hq(t)=\fr{1}{\pi}\int\limits_{-\infty}^{\infty}\fr{1}{t-s}q(s)\,ds\,,
$$ 
где интеграл понимается в смысле главного значения.)

 Следовательно, $\tilde{v}(x,\theta)$
постоянна на прямых $x\theta=s$. Функции $u(x,\theta)$ и
$v(x,\theta)$ принадлежат классу Гёльдера, и по теореме Привалова
$\tilde{v}(x,\theta)$ также принадлежит классу Гёльдера, а функция
$v(x,\theta)+i\tilde{v}(x,\theta)$ является граничным значением
аналитической функции, определенной в единичном круге.

Введем функцию
$$
a(x,\theta)=e^{D\mu(x,\theta^{\perp})}b(x,\theta)\,,
$$ 
где
$$
b(x,\theta)=e^{(1/2)\left(-R\mu(\theta,x\theta)-iHR\mu(\theta,x\theta)\right)}\,.
$$
Таким образом,
$$
a(x,\theta)=e^{v(x,\theta)+i\tilde{v}(x,\theta)}\,.
$$ 
Поскольку $b(x,\theta)$ постоянна на прямых $x\theta=s$, из того, что
$R_{\mu}f(s,\theta)=0$ для $\theta\in E$, следует
$$
R_{a}f(s,\theta)=\int\limits_{x\theta=s}f(x)a(x,\theta)\,dx=0\,,\quad
\theta\in E,\  s\in\mathbf{R}\,.
$$ 
Возьмем произвольный многочлен $p(s)$ и рас\-смот\-рим функцию
\begin{multline*}
W_p(\theta)=\int\limits_{-\infty}^{\infty}p(s)R_{a}f(s,\theta)\,ds={}\\
{}=\int\limits_{\mathbf{R}^2}p(x \theta)f(x)a(x,\theta)\,dx\,.
\end{multline*}
Функция $W_p(\theta)=0$ при $\theta\in E$, и, поскольку\linebreak подынтегральное
выражение при каждом фиксированном~$x$ является граничным
значением аналитической функции, а функция $f(x)$ имеет компактный
носитель, $W_p(\theta)=0$ при всех $\theta\in S^1$. Функция
$R_{a}f(s,\theta)$ для любого $\theta$ имеет носитель в некотором
интервале $\abs{s}<A$, и, так как множество многочленов плотно в
пространстве непрерывных на интервале функций,
$R_{a}f(s,\theta)=0$ для всех $\theta$ и всех $s$. Так как
$b(x,\theta)$ постоянна (и не равна нулю) на прямых
$x\theta=s$, должно также выполняться равенство $R_{\mu}f(s,\theta)=0$
для всех $\theta$ и всех $s$. Функция $f(x)$ однозначно
определяется значениями $R_{\mu}f(s,\theta)$ для $\theta\in S^1$ и
$s\in\mathbf{R}$ по формуле Новикова~\cite{2u} или Наттерера~\cite{3u} и,
следовательно, $f(x)\equiv0$.

Предположим теперь, что $P_\xi\not=P_\eta$. Это значит, что
существует непрерывная функция $f(x)$ с носителем в круге $U$
такая, что 
$$
P(\xi(x)=f(x))\not=P(\eta(x)=f(x))\,.
$$ 
Обозначим через
$f_1(x)$, $f_2(x),\ldots$ значения случайной функции $\xi(x)$,
отличные от $f(x)$, и  через
$g_1(x),g_2(x),\ldots$ обозначим аналогичные значения $\eta(x)$ (таким образом, $f(x)\not\equiv f_i(x)$ и
$f(x)\not\equiv g_i(x)$, $i=1$, 2,\ \ldots). Для каждого фиксированного
$i=1$, 2,\ \ldots пусть $A_i$ обозначает множество всех
$\theta\in\Lambda$, для которых $$R_{\mu}f(s,\theta)\equiv
R_{\mu}f_i(s,\theta),$$ и, соответственно, $B_i$~--- множество всех $\theta\in\Lambda$, для которых
$$
R_{\mu}f(s,\theta)\equiv R_{\mu}g_i(s,\theta)\,.
$$ 
Каждое из множеств $A_i$ и $B_i$ имеет меру нуль. Если для некоторого $i$
это было бы не так, то $R_{\mu}f(s,\theta)$ совпадала бы с
$R_{\mu}f_i(s,\theta)$ или с $R_{\mu}g_i(s,\theta)$ на множестве
положительной меры и в силу доказанного выше это влекло бы за
собой совпадение функции $f(x)$ с функцией $f_i(x)$ или с функцией
$g_i(x)$.

Таким образом, множество
$$
C=\bigcup\limits_{i=1}^{\infty}\left(A_i\bigcup\limits B_i\right)
$$ 
имеет меру нуль, а следовательно, множество $\Lambda\setminus C$ непусто.
Возьмем произвольное $\theta\in\Lambda\setminus C$. Поскольку для
этого $\theta$ выполнено условие $R_{\mu}f(s,\theta)\;\not\equiv$\linebreak
$\not\equiv\;R_{\mu}f_i(s,\theta)$, $i=1$, 2,\ \ldots, следовательно,
$$
P(R_{\mu}\xi_{\theta}(s)=R_{\mu}f(s,\theta))=P(\xi(x)=f(x))
$$ 
и аналогично
$$
P(R_{\mu}\eta_{\theta}(s)=R_{\mu}f(s,\theta))=P(\eta(x)=f(x))\,.
$$
Полученное противоречие доказывает теорему.

{\small\frenchspacing
{%\baselineskip=10.8pt
\addcontentsline{toc}{section}{Литература}
\begin{thebibliography}{99}

\bibitem{1u}
\Au{Arbuzov~E.\,V., Bukhgeim~A.\,L., Kazantsev~S.\,G.}
Two-dimensional tomogra\-phy problems and the theory of $A$-analytic
functions~// Siberian Adv. Math., 1998. Vol.~8. P.~1--20.

\bibitem{3u} %2
\Au{Natterer F.} 
Inversion of the attenuated Radon transform~//
Inverse Problems, 2001. Vol.~17. P.~113--119.

\bibitem{2u} %3
\Au{Novikov R.\,G.} 
An inversion formula for the attenuated X-ray
transformation~// Ark. Mat., 2002. Vol.~40. P.~145--167.


\bibitem{4u}
\Au{Ушаков В.\,Г., Ушаков~Н.\,Г.} 
Восстановление вероятностных
характеристик многомерных случайных функций по проекциям~// Вестн.
Моск. ун-та. Сер.~15. Вычисл. матем. и киберн., 2001. №\,4. C.~32--39.

\bibitem{6u} %5
\Au{Shestakov O.\,V.} 
An algorithm to reconstruct probabilistic
distributions of multivariate random functions from the
distributions of their projections~// J. of Mathematical
Sciences, 2002. Vol.~112. No.\,2. P.~4198--4204.

\bibitem{5u} %6
\Au{Шестаков О.\,В.} 
О единственности восстановления вероятностных
характеристик многомерных случайных функций по вероятностным
характеристикам их проекций~// Вестн. Моск. ун-та. Сер.~15. Вычисл.
матем. и киберн., 2003. №\,3. С.~37--41.

\bibitem{7u}
\Au{Ушаков В.\,Г., Шестаков~О.\,В.} 
Экспоненциальное преобразование
Радона случайных функций~// Вестн. Моск. ун-та. Сер.~15. Вычисл.
матем. и киберн., 2005. №\,1. C.~49--55.

\bibitem{8u}
\Au{Shestakov O.\,V.} 
Inversion of exponential Radon transform of
random func\-tions~// Transactions of XXV Seminar on Stability
Problems for Stochastic Models, 2005. P.~264--269.

\bibitem{9u}
\Au{Boman~J., Stromberg~J.-O.} 
Novikov's inversion formula for
the attenuated radon transform -- a new approach~// J. Geom. Anal., 2004.
Vol.~14. P.~185--198.

 \label{end\stat}

\bibitem{10u}
\Au{Привалов И.\,И.} 
Граничные свойства аналитических функций.~--- М.: ГИТТЛ, 1950.
\end{thebibliography}
}
}
\end{multicols}  