\def\stat{egorov}

\def\tit{КОНЦЕПЦИЯ СОЗДАНИЯ ОТЕЧЕСТВЕННЫХ 
ИНТЕГРИРОВАННЫХ КОММУНИКАЦИОННЫХ МИКРОКОНТРОЛЛЕРОВ 
ДЛЯ ПАКЕТНОЙ КОММУТАЦИИ}
\def\titkol{Концепция создания отечественных 
ИКМ для пакетной коммутации}

\def\autkol{В.\,Б.~Егоров}
\def\aut{В.\,Б.~Егоров$^1$}

\titel{\tit}{\aut}{\autkol}{\titkol}

%{\renewcommand{\thefootnote}{\fnsymbol{footnote}}\footnotetext[1]
%{Работа выполнена при поддержке
%Российского фонда фундаментальных исследований,
%гранты 06-07-89056 и 08-07-00152.}}

\renewcommand{\thefootnote}{\arabic{footnote}}
\footnotetext[1]{Институт проблем
информатики Российской академии наук, vegorov@ipiran.ru}

\vspace*{12pt}


\Abst{В статье предлагается концепция упрощенных интегрированных 
коммуникационных мик\-ро\-контроллеров (ИКМ), которые могли бы найти применение в 
различных отечественных устройствах пакетной коммутации и маршрутизации, 
расширяя их функциональные возможности и упрощая разработку.}

\KW{интегрированный коммуникационный микроконтроллер; PowerQUICC; 
децентрализованная коммутация; маршрутизирующий коммутатор}

           \vskip 24pt plus 9pt minus 6pt

      \thispagestyle{headings}

      \begin{multicols}{2}

      \label{st\stat}
   

\section{Современные интегрированные коммуникационные 
микроконтроллеры}
     
     Интегрированные коммуникационные микроконтроллеры~--- один из наиболее 
динамично развивающихся секторов рынка коммуникационной 
     микроэлектроники~\cite{4e, 5e}. Лидеры этого направления~--- компания 
\textit{Motorola} и ее преемница \textit{Freescale Semiconductor}~--- уже выпустили на рынок 
шесть поколений ИКМ, а общее число предлагаемых модификаций таких приборов 
перевалило за сотню~\cite{10e}. Популярность ИКМ объясняется как высокой степенью 
интеграции электронных компонентов, так и удачной интеграцией функций, в совокупности 
обеспечивающими ИКМ уникальные свойства и эффективное применение в самых 
разнообразных приложениях, связанных с телекоммуникацией вообще и коммутацией в 
частности, причем как каналов, так и пакетов.
     
     Типичный ИКМ компании \textit{Freescale} пред\-став\-ля\-ет собой <<систему на 
кристалле>>, объ\-еди\-ня\-ющую в единое целое современный мощный универсальный 
микроконтроллер с различными %\linebreak 
сопро\-цес\-со\-ра\-ми и коммуникационными 
модулями (рис.~\ref{f1e}).
{\looseness=1

}


\begin{figure*} %fig1
\vspace*{1pt}
\begin{center}
\mbox{%
\epsfxsize=164.253mm
\epsfbox{ego-1.eps}
}
\end{center}
\vspace*{-9pt}
\Caption{Обобщенная структура ИКМ
\label{f1e}}
\end{figure*}

     Микроконтроллерная часть ИКМ включает в себя свободно программируемое 
процессорное ядро и модуль системной интеграции.
     
     Программируемое ядро, на котором выполняются пользовательские программы, 
представляет собой мощный универсальный процессор с архитектурой PowerPC. Ядра 
комплектуются диспетчером памяти, раздельными кэшами инструкций и~данных, 
дополняемыми в некоторых моделях %\linebreak 
кэшем второго уровня и процессором плавающей 
запя\-той. Процессорные ядра самых производительных ИКМ способны работать на частотах 
до полутора гигагерц, обеспечивая производительность до трех с половиной миллиардов 
инструкций в секунду по эталонному тесту Dhrystone (DMIPS).
     
     Модуль системной интеграции объединяет разнообразные средства организации 
системы: контроллеры различных памятей, включая DDR SDRAM (double data rate 
synchronous dynamic %\linebreak 
\mbox{random} access memory) и NAND-flash; адаптеры внешних системных 
шин, в частности шины PCI (peripheral component interconnect), и стандартных 
последовательных интерфейсов вво\-да-вы\-во\-да I$^2$C (inter integrated circuits) и SPI (serial 
peripheral interface); контроллеры современных высокоскоростных последовательных 
интерфейсов PCIe (PCI express), SRIO (serial RapidIO) и SATA (serial advanced technology 
attachment).
     
     Большинство ИКМ имеет встроенный сопроцессор безопасности, выполняющий 
шифрование пользовательских данных <<на лету>> по всем основным принятым в мировой 
практике стандартам. Некоторые модели дополнительно комплектуются сопроцессорами, 
упрощающими анализ контекста заголовков пакетов и/или табличный поиск, в частности с 
использованием хеширования адресов.
     
     Однако основной изюминкой ИКМ являются встроенные коммуникационные модули 
двух видов: универсальные и специализированные. Универсальный коммуникационный 
модуль, как %\linebreak 
прави\-ло, единственный в ИКМ, обслуживает несколь\-ко программно 
конфигурируемых среднескоростных коммуникационных интерфейсов и поддерживает 
целый ряд соответствующих им %\linebreak 
ши\-ро\-ко распространенных в мире коммуникационных 
протоколов канального уровня благодаря тому, что включает отдельный RISC-про\-цес\-сор со 
своим встроенным фирменным программным обеспечением (ФПО). Специализированные 
коммуникационные модули обслуживают один--два однотипных высокоскоростных 
интерфейса и тоже могут содержать процессоры и ФПО.
     
     Так как ИКМ долгое время отслеживали эволюцию телекоммуникационных 
технологий, все историческое многообразие интерфейсов и протоколов нашло в них 
отражение различными типами коммуникационных модулей и широким ассортиментом 
конфигурируемых интерфейсов универсальных модулей. На разных этапах развития в 
номенклатуру поддерживаемых ИКМ коммуникационных интерфейсов и протоколов вошли:
     \begin{itemize}
\item асинхронные интерфейсы типа UART (universal asynchronous receiver/transmitter);
\item синхронные интерфейсы на основе протоколов SDLC/HDLC (synchronous data link 
control\,/\,high-level data link control) и BISYNC (binary synchronous communications);
\item иерархия интерфейсов ISDN (integrated services digital network), включая как basic rate 
(IDL, GCI), так и primary rate (T1/E1, T3/E3);
\item интерфейсы ATM (asynchronous transfer mode), в том числе UTOPIA (universal test and 
operations physical interface for ATM);
\item интерфейс Ethernet, причем по мере совершенствования коммуникационных модулей 
скорость передачи данных последовательно росла с~10~Мбит/с до~1~Гбит/с.
     \end{itemize}
     
     Обогащение номенклатуры интерфейсов и %\linebreak 
     протоколов в ИКМ параллельно 
сопровождалось расши\-ре\-ни\-ем перечня коммуникационных технологий, поддерживаемых 
аппаратурой и ФПО коммуникационных модулей. В результате современные ИКМ находят 
применение и в сетях цифровой телефонии, в частности ISDN, и в пакетных сетях Token 
ring, X.25, Frame relay, ATM, Ethernet, Packet over SONET (synchronous optical networking) и 
других.

     
     Для ИКМ характерны универсальность и <<дружелюбие>> к пользователю. Непросто 
найти область телекоммуникации и пакетной коммутации, где ИКМ оказался бы не к месту. 
Для любого не слишком экзотического приложения найдутся приборы с соответствующими 
внешними интерфейсами и %\linebreak 
поддержкой требуемых коммуникационных протоколов, которые 
в совокупности с ис\-чер\-пы\-ва\-ющи\-ми возможностями универсального мик\-ро\-кон\-т\-рол\-ле\-ра 
действительно оказываются эффективной %\linebreak 
закон\-чен\-ной телекоммуникационной <<систе\-мой 
на кристалле>>, позволяющей решать самые разные по характеру и сложности задачи 
исключительно внут\-рен\-ни\-ми ресурсами одного единственного прибора, а следовательно, 
предельно быст\-ро и с минимальными затратами.
     
     Но обратная сторона универсальности и <<дружелюбия>> ИКМ~--- его сложность. В 
ИКМ заложен целый ряд технических решений, представляющих собой интеллектуальную 
собственность компании \textit{Freescale} самой высокой пробы в различных областях 
электроники, вычислительной техники и телекоммуникации. Ее главные 
составляющие~\cite{8e}:
     \begin{itemize}
\item современный высокопроизводительный универсальный процессор, оснащенный 
эффективными средствами разработки программного обеспечения;
\item сверхреактивный RISC-процессор и особенно его уникальное многоплановое ФПО, 
поддерживающее все основные современные телекоммуникационные технологии и их 
эффективное взаимодействие (interworking) между собой;

\item многофункциональные адаптеры практически всех интерфейсов, нашедших 
применение в компьютерах и телекоммуникации.
\end{itemize}

\section{Концепции пакетных коммутаторов на основе интегрированных коммутационных
 микроконтроллеров}
     
     С учетом имеющейся на мировом рынке номенклатуры ИКМ и опыта их 
применения~\cite{3e, 6e} можно выделить три концептуальных подхода к созданию 
устройств пакетной коммутации и маршрутизации на основе некоего обобщенного 
ИКМ~\cite{1e, 2e}:
     \begin{enumerate}[(1)]
\item централизованный коммутатор как <<система на кристалле>>, предполагающий 
мощный высокоинтегрированный и высокопроизводительный многопортовый ИКМ;
\item децентрализованный коммутатор на основе гомогенной <<сверхлокальной>> сети 
тесно связанных малопортовых ИКМ средней производительности;

\item децентрализованный коммутатор на основе гетерогенной <<сверхлокальной>> сети с 
общей разделяемой буферной памятью и множеством ИКМ ограниченной 
производительности.
\end{enumerate}

     Ярким примером высокопроизводительной <<системы на кристалле>> является ИКМ 
MPC8568E компании \textit{Freescale}~\cite{11e}. В рамках общей структуры, показанной на 
рис.~\ref{f1e}, он включает универсальный микроконтроллер, сопроцессоры безопас\-ности и 
табличного поиска, а также три коммуникационных модуля. Микроконтроллер содержит 
свободно программируемое ядро с рабочей частотой до 1,33~ГГц, 64-раз\-ряд\-ный интерфейс 
памяти DDR2 SDRAM с частотой до~533~МГц, 32-раз\-ряд\-ную параллельную локальную 
шину и 32-раз\-ряд\-ный интерфейс PCI, работающий в режимах host и agent на частоте до~66~МГц, 
а также последовательные интерфейсы PCIe, SRIO, два I$^2$C и два UART.
     
     Универсальный коммуникационный модуль MPC8568E с двумя RISC-про\-цес\-со\-ра\-ми и 
мощным ФПО обслуживает 8~одинаковых коммуникационных портов, 8~портов ISDN и два 
порта SPI. Универсальные порты поддерживают технологии ISDN, X.25, Frame relay, ATM, 
Ethernet и другие. В~режиме ATM максимальная скорость передачи данных через два 
16-раз\-ряд\-ных интерфейса UTOPIA составляет~622~Мбит/с (OC-12); в режиме Ethernet все 
8~портов способны работать на скоростях~10 и~100~Мбит/с, а три порта~--- на ско\-рости~1~Гбит/с. 
Фирменное программное обеспечение модуля поддер\-жи\-ва\-ет все современные теле\-ком\-му\-ни\-ка\-ци\-он\-ные технологии 
и обеспечивает эффек\-тив\-ное взаимодействие между ними. Бла\-года\-ря этому универ\-саль\-ный 
коммуникационный модуль реализует исключительно внутренними ресурсами 
восьмипортовый коммутатор L2 и обеспечивает ин\-кап\-су\-ля\-цию трафика Ethernet в трафик 
ATM.
{\looseness=-1

}
     
     Помимо универсального коммуникационного модуля в ИКМ MPC8568E входят два 
одинаковых специализированных <<интеллектуальных>> коммуникационных модуля 
трехскоростного (10, 100 %\linebreak
 и~1000~Мбит/с) Ethernet с аппаратной поддержкой протокола IP, 
что дает возможность прикладному программному обеспечению эффективно %\linebreak
реали\-зо\-вы\-вать 
функции коммутатора L3, граничного маршрутизатора и устройства сетевой безопас\-ности.
     
     В целом ИКМ MPC8568E позволяет создать с минимальной дополнительной 
аппаратурой (необходимые внешние памяти и трансиверы Ethernet) законченный 
централизованный пакетный коммутатор максимально с 10~внешними портами (из них до 
пяти гигабитных), который также может работать в качестве граничного маршрутизатора и 
устройства безопасности региональных и глобальных сетей, в том числе сетей Metro Ethernet 
и MPLS (multiple protocol label switching)~\cite{4e, 5e}. Этот подход весьма привлекателен~\cite{8e}, но здесь он не 
обсуждается, поскольку, к сожалению, появление в обозримом будущем отечественного 
ИКМ, подобного MPC8568E, нереально~\cite{7e}. Далее внимание уделено альтернативным 
подходам к созданию устройств пакетных коммутаторов и маршрутизаторов, а именно 
основанным на децентрализованной архитектуре пакетной коммутации~\cite{9e} с 
использованием более простых ИКМ.

\begin{figure*} %fig2
\vspace*{1pt}
\begin{center}
\mbox{%
\epsfxsize=129.293mm
\epsfbox{ego-2.eps}
}
\end{center}
\vspace*{-9pt}\Caption{Гомогенный коммутатор с узлами на основе ИКМ
\label{f2e}}
\end{figure*}
     
     Первой альтернативой является децентрализованная коммутация в гомогенной 
<<сверхлокальной>> сети из множества тесно связанных одина\-ковых малопортовых ИКМ 
средней производительности, оснащенных небольшим числом коммуникационных портов. 
Примерами таких ИКМ %\linebreak 
могут служить приборы  \textit{Freescale} MPC8542 
(2~коммуникационных порта трехскоростного Ethernet) и MPC8548 (4~порта 
Ethernet)~\cite{12e}, снабженные также адаптерами интерфейсов SRIO и PCIe. На 
рис.~\ref{f2e} показана примерная организация гомогенного децентрализованного 
коммутатора на базе \mbox{неких} малопортовых ИКМ средней про\-из\-во\-ди\-тель\-ности, 
функционально близких, например, к прибору MPC8542.
     
   
     В гомогенном коммутаторе <<сверхлокальная>> сеть реализуется на 
высокоскоростном интерфейсе, рассчитанном на объединение множества ИКМ в пределах 
одной платы или одного шасси и допускающем скорости передачи данных, не уступающие 
суммарной скорости передачи на внешних портах одного ИКМ. Этим условиям в полной 
мере отвечает интерфейс SRIO, скорость передачи данных на котором может достигать 
10~Гбит/с. Принципиально важно, что спецификация SRIO включает механизм 
инкапсуляции чужих протоколов для потокового транзита инкапсулированных блоков 
данных через коммутационные структуры с учетом их приоритетов и классов обслуживания, 
причем с минимальными накладными расходами инкапсуляции и задержками 
транзита~\cite{13e}.
     
     Данная концепция предполагает децентрализованную коммутацию, при которой 
решение о продвижении каждого входящего в коммутатор пакета принимает тот узел 
<<сверхлокальной>> сети, в нашем случае ИКМ, в который входит этот пакет. Система 
гомогенна в том смысле, что все узлы распределенного коммутатора одинаковы и 
равноправны. Оценочно, на основании опыта компании \textit{Freescale}, для собственно 
продвижения принятых пакетов с выполнением сопутствующих действий по их фильт\-ра\-ции 
и квалификации, а также других функций вплоть до распределенной маршрутизации 
требуется производительность свободно про\-грам\-ми\-ру\-емо\-го ядра ИКМ не ниже 
1000~DMIPS. Коммутация пакетов в реальном времени может быть обеспечена таким ядром 
вплоть до суммарно гигабитной скорости передачи данных на внешних коммуникационных 
портах. В свою очередь, при тех же условиях пропускная способность интерфейса SRIO 
достаточна не только для доставки скоммутированных пакетов из входного узла 
<<сверхлокальной>> сети в выходной, но и для обмена маршрутной и служебной 
информацией между ИКМ.
     
     Концепция реализуется всего двумя базовыми элементами: ИКМ средней 
производительности в качестве узлов <<сверхлокальной>> сети и одним коммутатором, 
реализующим коммутационную структуру. Требования к ИКМ такого рода подробно 
обсуждаются ниже. Что же касается коммутатора, то, например, коммутаторы SRIO 
выпускаются несколькими западными фирмами, например компаниями \textit{IDT} и 
\textit{Tundra Semiconductor}~\cite{14e, 15e}. Однако типичный коммутатор для интерфейса 
SRIO или ему подобного, функционально достаточный для реализации концепции 
<<сверхлокальной>> сети, может быть реализован на программируемых логических интегральных схемах
(ПЛИС) умеренной сложности или, в 
перспективе, в виде отечественной заказной СБИС (сверхбольшой интегральной схемы).
     
     Другая альтернатива коммутации внутри <<сис\-те\-мы на кристалле>>~--- концепция 
централизованной коммутации в гетерогенном распределенном коммутаторе с общей 
буферной памятью. Она %\linebreak 
также базируется на <<сверхлокальной>> сети и пред\-по\-ла\-га\-ет в 
коммуникационных узлах ком\-му\-та\-тора множество ИКМ, но более простых, в пределе 
однопортовых и относительно небольшой %\linebreak 
производительности. Суть этой концепции 
поясняет  рис.~\ref{f3e}.
{\looseness=1

}     

\begin{figure*} %fig3
\vspace*{1pt}
\begin{center}
\mbox{%
\epsfxsize=166.336mm
\epsfbox{ego-3.eps}
}
\end{center}
\vspace*{-9pt}
\Caption{Гетерогенный коммутатор с общей буферной памятью
\label{f3e}}
\vspace*{6pt}
\end{figure*}

     Гетерогенный коммутатор предполагает общую для всего коммутатора буферную 
память, множество коммуникационных узлов, обес\-пе\-чи\-ва\-ющих коммуникационные порты, 
и выделенные узлы для выполнения коммутации, маршрутизации и управ\-ле\-ния 
коммутатором, причем все узлы также %\linebreak 
объеди\-ня\-ют\-ся некой <<сверхлокальной>> \mbox{сетью}. 
Коммуникационные узлы, непосредственно участ\-ву\-ющие в продвижении пакетов,~--- узлы 
data plane~--- реализуются на относительно прос\-тых ИКМ (на рис.~\ref{f3e} ИКМ/DP$_{1\ldots 
N}$) с %\linebreak 
процессорным ядром умеренной производительности 300$\ldots$500~DMIPS. Все 
решения по коммутации и маршрутизации, а также общему управлению коммутатором, т.\,е.\
функции control plane, принимаются выделенными ИКМ повышенной 
производительности (на рис.~\ref{f3e} ИКМ/CP$_{1\ldots K}$).
     
     Принципиально новым элементом архитектуры коммутатора является общая буферная 
память, например типа QDR SRAM (quadruple data rate static random access memory), со своим контроллером, 
который подключается к коммутационной структуре звеном интерфейса 
<<сверхлокальной>> сети с высокой пропускной способностью. В частности, в случае 
интерфейсов SRIO и/или PCIe должны быть использованы варианты звеньев максимальной 
<<ширины>> и наибольших скоростей передачи данных. Контроллер общей буферной 
памяти может быть реализован на ПЛИС средней сложности, предоставляющей 
возможность подключения современных синхронных памятей. В частном, но, как будет 
показано ниже, важном случае контроллер памяти может быть совмещен с коммутационной 
структурой в виде некоторого коммутаторного ядра (на рис.~\ref{f3e} оно обведено 
     штрих-пунктиром), для реализации которого, разумеется, потребуется ПЛИС большей 
сложности.
     

\section{Современные тенденции в~области пакетной коммутации}

\vspace*{-9pt}

     Во второй половине прошлого века телекоммуникация в высоком темпе преодолела 
несколько знаковых рубежей. Сначала переход от аналоговой технологии связи к цифровой 
увенчался объединением различных услуг, таких как телефон, факс и передача данных 
между компьютерами, в единой цифровой сети интегрированных сервисов \mbox{ISDN}, 
основанной на идее временн$\acute{\mbox{о}}$го разделения традиционных телефонных каналов связи между 
несколькими соединениями. Однако распространение технологии ISDN на 
высокоскоростные каналы связи, вылившееся в сложные синхронные и плезиохронные 
иерархии цифровых каналов, не успев толком завершиться, показало свою относительную 
неэффективность вследствие плохого использования дорогих широкополосных линий связи. 
Осознание этого принципиального недостатка технологии коммутации каналов 
стимулировало в телекоммуникации решительный переход к технологиям коммутации 
пакетов.
     
     Пакетные сети первых поколений, изначально ориентированные только на передачу 
данных между компьютерами, отличались широким много\-об\-ра\-зи\-ем. В области локальных 
сетей конкурировало множество технологий, постепенно пол\-ностью вытесненных, однако, 
технологией Ethernet. Несколько меньшее исходное разнообразие региональных пакетных 
сетей, реализуемых на традиционных телефонных каналах, также постепенно свелось к 
одной доминирующей технологии X.25. Однако\linebreak\vspace*{-12pt}
\pagebreak

\noindent
идея интеграции услуг, уже успешно 
опробованная в %\linebreak 
сетях ISDN, плохо ложилась на пакетные сети~X.25 из-за ненадежности 
каналов связи и низких скоростей передачи данных в них. Поэтому с внедре\-нием в 
региональные и глобальные сети более %\linebreak 
быст\-ро\-действующих и надежных каналов, в 
частности %\linebreak
 оптоволоконных, появились и более гибкие телекоммуникационные технологии, 
адекватные требованиям интеграции услуг. Отличительной особенностью новых пакетных 
технологий стала %\linebreak
 возможность организации виртуальных соединений, аналогичных 
аналоговым, т.\,е., по существу, наложения на пакетные сети виртуальных сетей с 
коммутацией каналов. В таких сетях передача компьютерных данных могла по-прежнему 
выполняться традиционными методами, в частности дейтаграммами с динамической 
маршрутизацией, но для критических сервисов строились прямые виртуальные тракты 
<<точка--точка>> от одного конечного абонента к другому и резервировались требуемые 
полосы пропускания во всех звеньях этих трактов.
     
     Первая такого рода гибридная технология Frame relay, поддерживающая организацию 
как постоянных, так и коммутируемых виртуальных соединений, наглядно 
продемонстрировала правильность подхода к пакетным сетям и эффективно решила 
проблему передачи, в частности, человеческой речи. Затем для передачи через пакетные сети 
в реальном времени изображения, <<живой>> телевизионной картинки, потребовались не 
только линии связи с более широкой полосой пропускания, но и новые более гибкие 
коммуникационные технологии, обеспечивающие доставку последовательности пакетов 
изображения получателю в нужном порядке и с постоянным ритмом. Основной в этом 
классе стала технология ATM. Еще совсем недавно успешное решение с ее помощью 
практических проблем организации разнородных трафиков в пакетных сетях регионального 
масштаба позволило экспертам предсказывать всеобщую <<АТМизацию>> глобальных 
пакетных сетей, несмотря на сложности реализации и дороговизну этой технологии. Однако 
жизнь нашла другое решение и заставила переоценить ценности под влиянием трех мощных 
факторов технического прогресса:
     \begin{enumerate}[(1)]
\item широчайшего распространения персональных компьютеров и объединяющих их 
локальных сетей Ethernet;
\item впечатляющего рывка техники передачи данных с мегабитных на гигабитные скорости, 
т.\,е.\ на три порядка~(!) менее чем за четверть века;
\item огромной популярности Всемирной паутины, сумевшей за два десятилетия охватить 
практически всю планету.
\end{enumerate}
     
     Персональные компьютеры и локальные сети превратили технологию Ethernet, по 
первоначальным прогнозам самую дорогую из конкурировавших, в самую дешевую и 
массово доступную на практике технологию физического и канального уровней. Прогресс в 
технике передачи данных позволил в исторически короткие сроки создать целую гамму 
стандартов Ethernet, покрывших диапазон скоростей передачи данных от 10~Мбит/с 
до~10~Гбит/с и позволивших вывести Ethernet из области сугубо локальной на 
региональный и даже глобальный уровень. С другой стороны, популярность Интернета 
сделала его стандарты сетевого и транспортного уровня, <<стек>> TCP/IP (transport control 
protocol\,/\,Internet protocol), абсолютно доминирующими в мире. В результате сегодня 
связка Ethernet\;+\;IP становится превалирующей в пакетных сетях любого масштаба, от 
локальных до глобальных, практически лишая шансов все конкурирующие технологии.
     
     Однако отказ от технологий, ориентированных на виртуальные соединения 
     <<точка--точка>>, таких как Frame relay и ATM, потребовал для поддержки 
критических трафиков существенной доработки протоколов Интернета. Так, например, в 
протокол IP были введены понятия типа обслуживания (в версии~4) или класса трафика (в 
версии~6). Однако это не решило всех проблем. Для быстрого продвижения пакетов в 
глобальных дейта\-грам\-мных сетях через множество коммутаторов все же пришлось 
вернуться к идее виртуальных <<туннелей>> и добавить к связке Ethernet\;+\;IP еще один 
промежуточный уровень (уровень $2^1/_2$)~--- технологию MPLS, совместившую в себе простоту Frame relay и гарантированное 
качество 
обслуживания QoS (quality of %\linebreak
service) ATM. В результате минимально необходимый и 
практически достаточный комплект телекоммуникационных технологий, которые 
необходимо поддерживать современным пакетным %\linebreak
 комму\-та\-то\-рам, свелся к триаде Ethernet, 
MPLS и IP. Соответственно, разработка любого нового ИКМ, особенно упрощенного и не 
предполагающего непременной обратной совместимости с длинным рядом 
предшественников, могла бы быть ориентирована только на эту триаду.

\section{Внутренние интерфейсы <<сверхлокальной>> сети}
     
     Важным элементом обеих концепций <<распределенного>> коммутатора является его 
внутренний интерфейс <<сверхлокальной>> сети, характеристики которого в значительной 
степени определяют возможности и параметры всего коммутатора. Исходная концепция 
децентрализованной коммутации %\linebreak
основывалась на некой шине или системе параллельных 
шин, слотовых или блочных~\cite{9e}, но совре\-мен\-ные скорости передачи данных смещают %\linebreak 
акценты в сторону высокоскоростных последовательных интерфейсов. На сегодняшний день 
в качестве такого рода интерфейсов конкурируют интерфейсы SRIO и PCIe, в целом очень 
похожие и близкие по своим характеристикам. Оба интерфейса основаны на пакетной 
передаче данных по последовательным высокоскоростным каналам. У них сходная 
физическая реализация: одна или несколько дорожек (lanes), по которым передаются 
низковольтные дифференциальные сигналы~--- с той лишь небольшой разницей, что в SRIO 
принята стандартная сигнализация для витой пары 10-gigabit Ethernet, а PCIe использует 
собственную оригинальную. Возможности PCIe и SRIO практически равны в части 
адресации памяти: PCIe допускает длину адреса 32 и 64~разряда, а SRIO~--- 34, 50 и~66~разрядов. 
Схожи у обоих интерфейсов механизмы контроля целостности передаваемых 
данных и подтверждения передачи пакета на звене.
     
     Но есть у этих интерфейсов и существенные различия, вытекающие из их назначения. 
Интерфейс PCIe предназначен, в первую очередь, для замены традиционной шины PCI, вследствие чего 
ориентирован на подключение быст\-ро\-дей\-ст\-ву\-ющей периферии к центральному процессору 
и подразумевает древовидную топологию с центральным процессором в корне дерева. При 
этом подключение других процессоров представляет определенные сложности и возможно 
только через дополнительные %\linebreak
 нестан\-дарт\-ные и непрозрачные порты промежуточных 
коммутаторов. В противовес этому SRIO изначально создавался как интерфейс 
взаимодействия множества равноправных процессоров и допус\-ка\-ет самые разнообразные 
топологические воплощения объединяющих такие процессоры %\linebreak
 коммутационных структур. 
Ниже более подробно рассмотрены вытекающие из этих осново\-по\-ла\-га\-ющих различий 
особенности обоих интерфейсов.
     
     Безусловным преимуществом интерфейса PCIe является существенно 
б$\acute{\mbox{о}}$льшая максимальная пропускная способность звена. Интегральная 
пропускная способность обоих интерфейсов зависит от их ширины, т.\,е.\ числа 
задействованных дорожек, и скорости передачи данных по дорожке. В самом широком 
варианте с 16~дорожками (принятое обозначение $\times 16$) и на максимальной скорости 
5~Гбод PCIe способен обеспечить на звене <<чистую>> пользовательскую скорость 
64~Гбит/с. Интерфейс SRIO в самом широком варианте с четырьмя дорожками (принятое 
обозначение $4\times$) и на максимальной скорости 3,125~Гбод предоставляет 
пользовательскую скорость только 10~Гбит/с.
     
     Формально интерфейс PCIe, использующий общую адресную сетку для адресации 
памяти и подключаемых устройств ввода-вывода, может адресовать практически 
бесконечное число або\-нентов. Однако вряд ли это следует рассматривать как %\linebreak 
преиму\-ще\-ст\-во, 
поскольку 8- или 16-раз\-ряд\-ные идентификаторы SRIO накрывают все реальные потребности 
адресации конечных абонентов <<сверх\-ло\-каль\-ной>> сети и значительно упрощают 
реализацию коммутационных структур.
     
     Аналогичным образом как относительное можно оценить преимущество механизма 
реализации QoS на интерфейсе PCIe. Формально восемь классов трафика (traffic classes), 
отражаемые в восемь виртуальных каналов, и гибкая система арбитража между ними 
выглядят гораздо более внушительно, чем простые три уровня приоритета пакетов в SRIO. 
Однако эта система сложна в реализации и требует б$\acute{\mbox{о}}$льших объемов 
буферов, сепаратно выделяемых в конечном пользователе каждому задействованному 
сочетанию класса трафика и виртуального канала. В результате большинство практических 
реализаций интерфейса PCIe поддерживают только один-единственный класс трафика и 
один виртуальный канал. Между тем простая приоритетная схема SRIO, дополняемая 
механизмом CoS (classes of service), допускает в системе тысячи одновременных потоков и 
десятки тысяч виртуальных каналов для каждого конечного абонента. При этом она 
ориентирована на общий для пакетов любых приоритетов и потому меньший по суммарному 
объему буфер в конечном абоненте.
     
     Интерфейс SRIO обладает безусловным преимуществом при взаимодействии 
множества равноправных ИКМ между собой и в использовании ими общей памяти. 
Адресный доступ к общей памяти как разделяемому ресурсу поддерживается 
<<неделимыми>> (atomic) операциями. 
\begin{table*}\small
\begin{center}
\Caption{Сравнение интерфейсов PCIe и SRIO
\label{t1e}}
\vspace{2ex}

\begin{tabular}{|l|c|c|}
\hline
\multicolumn{1}{|c|}{Параметр}&PCIe&SRIO\\
\hline
Топология&древовидная&любая\\
Число конечных абонентов&очень большое&256 или 65 536\\
Подтверждение операции записи&нет&есть\\
Поддержка обмена сообщениями&весьма ограниченная&полная и многоадресная\\
Поддержка режима дейтаграмм&нет&есть\\
Поля, заменяемые коммутатором &3 поля&нет\\
Число сигнальных пар&1, 2, 4, 8, 12, 16&1, 4\\
Скорость передачи данных&2,5; 5,0 Гбод&1,25; 2,5; 3,125 Гбод\\
Пропускная способность звена&до 64 Гбит/с&до 10 Гбит/с\\
Механизм QoS&8 классов трафика&3 приоритета пакетов\\
Аппаратура реализации QoS&более сложная&более простая\\
Механизмы управления потоком&только на уровне звена&на уровне звена и абонента\\
\hline
\end{tabular}
\end{center}
\end{table*}
Но интерфейс SRIO помимо традиционного 
адресного доступа предоставляет два дополнительных гибких механизма: обмен 
сообщениями, исключающий насильственные вторжения в память других конечных 
абонентов, и инкапсуляцию сквозных потоков (streaming flows), позволяющую осуществлять 
эффективный транзит высокоскоростных трафиков с произвольными протоколами от одного 
конечного абонента к другому. Оба этих механизма задействуют процедуру SAR 
(segmentation and reassembly) как для потоков данных, так и длинных сообщений, что 
позволяет ограничить длину пакетов 256~байтами и тем самым упростить их буферирование 
внутри коммутационных структур. Протокол звена интерфейса SRIO настолько прост, что 
компонентам коммутационной структуры не нужно менять, в отличие от коммутаторов 
PCIe, поля в заголовках пакетов и, соответственно, пересчитывать контрольные суммы. 
Гораздо богаче возможности SRIO в управ\-ле\-нии потоком: помимо схожих с интерфейсом 
PCIe механизмов синхронизации на звене интерфейс SRIO дополнительно включает 
протокольные средства управления потоком между конечными абонентами, в том числе 
традиционный механизм xon/xoff.

     
     Сводные характеристики обоих интерфейсов представлены в табл.~\ref{t1e}, из которой видно, 
что в целом интерфейс SRIO обладает абсолютными преимуществами для реализации 
<<сверхлокальной>> сети гомогенных коммутаторов, но в случае гетерогенного 
коммутатора он проигрывает интерфейсу PCIe в одном-единственном пункте~--- 
пропускной способности звена подключения к коммутационной структуре контроллера 
общей буферной па\-мяти.
     

     Картину может изменить недавно анонсированная архитектура <<сверхлокальной>> 
сети ASI (advanced switching interconnect), представляющая собой транспортную надстройку 
на интерфейс PCIe. Сохраняя тем самым основные достоинства физического уровня PCIe, в 
первую очередь высокую потенциальную пропускную способность звена, ASI одновременно 
предоставляет пользователю целый ряд возможностей, ранее свойственных только 
логическому уровню интерфейса SRIO, в числе которых многоадресная рассылка, 
инкапсуляция произвольных протоколов, в том числе с механизмом SAR, туннелирование 
потоков инкапсулированных данных и управление коммутационными структурами.
     
     Кроме того, в гипотетических отечественных ИКМ возможно появление оригинальных 
высокоскоростных последовательных интерфейсов, совмещающих в себе достоинства SRIO 
и PCIe, что позволит им успешно конкурировать с интерфейсами уровня ASI, по крайней 
мере, в плане реализации рассматриваемых концепций децентрализованной коммутации. Но 
такие решения имеют право на жизнь, если только не иметь в виду возможность 
объединения в одной <<сверхлокальной>> сети этих гипотетических отечественных ИКМ с 
реально существующими и будущими приборами компании \textit{Freescale}. В любом 
случае рассмотрение подобных интерфейсов выходит за рамки настоящей статьи.

\section{Упрощенные интегрированные коммутационные микроконтроллеры для~пакетной~коммутации}

\begin{figure*} %fig4
\vspace*{1pt}
\begin{center}
\mbox{%
\epsfxsize=141.78mm
\epsfbox{ego-4.eps}
}
\end{center}
\vspace*{-9pt}
\Caption{ Структура ИКМ-Б
\label{f4e}}
\end{figure*}

     В рамках двух концепций децентрализованного пакетного коммутатора~--- гомогенной 
и гетерогенной~--- были, по существу, обозначены два варианта упрощенных ИКМ для 
децентрализованного коммутатора: большей (свыше 1000~DMIPS) и меньшей (порядка 
500~DMIPS) про\-из\-во\-ди\-тель\-ности. Они различаются процессорным ядром и некоторыми 
системными компонентами, но должны иметь минимум два одинаковых функциональных 
блока: коммуникационный модуль внешнего интерфейса и модуль внутреннего интерфейса 
<<сверх\-ло\-каль\-ной>> сети. С учетом выводов предыдущих разделов в качестве первого 
необходимо и достаточно иметь трехскоростной (10, 100 и 1000~Мбит/с) Ethernet, а в 
качестве второго могут выступать интерфейсы SRIO или PCIe. Поскольку интерфейс SRIO 
обладает рядом существенных преимуществ перед PCIe, далее для определенности он 
принят в рассматриваемых упрощенных ИКМ в качестве интерфейса <<сверхлокальной>> 
сети.

     
     Возможная структура упрощенного ИКМ б$\acute{\mbox{о}}$льшей 
производительности на 1000~DMIPS и более (далее ИКМ-Б) приведена на рис.~\ref{f4e}. В 
целом она не противоречит традиционному подходу (см.\ рис.~\ref{f1e}), но отличается 
существенно меньшей сложностью, сохраняя при этом достаточно высокую степень 
универсальности, позволяющую применять такой ИКМ в различных приложениях, в том 
числе в обеих рассмотренных выше концепциях: в качестве универсального узла 
<<сверхлокальной>> сети в гомогенном коммутаторе либо узлов коммутации, 
маршрутизации и управления гетерогенного коммутатора, т.\,е.\ ИКМ/CP на 
рис.~\ref{f3e}.

     В качестве процессорного ядра в ИКМ-Б может быть задействован процессор любой 
архитектуры с суммарной производительностью не хуже 1000~DMIPS. Хорошим решением 
может стать процессорное ядро с открытой архитектурой ARM, например, как показано на 
рис.~\ref{f4e}, Cortex-A8~\cite{16e}. Лицензию и технологию производства различных ядер ARM 
можно приобретать, в том числе на условиях royalty free, у компании \textit{ARM}.
     
     Обязательным элементом любого ИКМ, в том числе ИКМ-Б, должен быть модуль 
трехскоростного Ethernet, включающий MAC-контроллер, контроллер прямого доступа в 
память DMA (direct memory access) и специфический блок отделения и слияния заголовков 
HDAM (headers detach and merge). Внешним интерфейсом может служить любой 
стандартный MAC-интерфейс гигабитного Ethernet, в частности GMII (gigabit media 
independent interface), RGMII (reduced GMII) или, в целях экономии числа выводов прибора, 
последовательный интерфейс SGMII (serial GMII), как показано на рис.~\ref{f4e}.

\begin{figure*} %fig5
\vspace*{1pt}
\begin{center}
\mbox{%
\epsfxsize=141.78mm
\epsfbox{ego-5.eps}
}
\end{center}
\vspace*{-9pt}
\Caption{Структура ИКМ-М
\label{f5e}}
\end{figure*}
     
     Второй обязательный элемент всех упрощенных ИКМ для пакетной коммутации~--- 
это модуль высокоскоростного последовательного интерфейса <<сверхлокальной>> сети, в 
частности модуль SRIO. Он обязательно должен включать контроллеры DMA и звена 
интерфейса (link control), обработчики доступа к общей буферной памяти (access handler) и 
обмена сообщениями (message handler). Также весьма желательно для улучшения условий 
доставки инкапсулированных блоков данных через коммутационные структуры, чтобы в 
него входил блок SAR. Сегментация, с одной стороны, упростит коммутационную 
структуру, а с другой~--- ускорит доставку пакетов между узлами <<сверхлокальной>> сети 
и облегчит реализацию сопутствующих требований QoS.

     
     В предположении, что контроллеры DMA умеют работать в свойственном 
классическим ИКМ активном режиме, используя механизм цепочек описателей 
буферов~\cite{8e}, модули Ethernet и SRIO могут включать небольшого объема двупортовые 
памяти DPM (dual-port memories) для циркулярных цепочек такого рода описателей.
     
     Специфическим именно для ИКМ-Б блоком должен быть контроллер внешней памяти 
большого объема, например DDR3, необходимой для буферирования блоков данных при 
реализации узла гомогенного коммутатора или хранения маршрутной и служебной 
информации при реализации коммутирующего процессора гетерогенного коммутатора.
     
     Для ускорения адресной фильтрации и классификации входящих кадров в ИКМ-Б 
может быть предусмотрен блок быстрого доступа к адресным таблицам LUT (look-up table) 
и/или другие блоки, выполняющие функции, аналогичные функциям сопроцессоров 
табличного и контекстного поиска в классических ИКМ (см.\ рис.~\ref{f1e}). Блок LUT 
может также включать небольшую, размером в несколько элементов, ассоциативную память 
CAM (context addressable memory) для хранения текущих MAC-ад\-ре\-сов и как возможное 
расширение~--- интерфейс для подключения внешней табличной памяти.
     

\begin{table*}\small
\begin{center}
\Caption{Сравнение ядер Cortex-A8 и Cortex-R4
\label{t2e}}
\vspace*{2ex}

\begin{tabular}{|l|c|c|}
\hline
\multicolumn{1}{|c|}{Параметр}&Cortex-A8&Cortex-R4\\
\hline
L1 кэш&переменный&0$\ldots$64 K\\
L2 кэш&0$\ldots$1 M&нет\\
Тесно связанная память&нет&есть\\
Внешняя шина&AMBA 3 AXI&2\;$\times$\;AMBA 3 AXI\\
Технологии NEON и Jazelle RCT&есть&нет\\
Процесс&65 нм (GP)&90 нм (HS)\\
Максимальная частота&1,1 ГГц&475 МГц\\
Максимальное быстродействие&2000 DMIPS&760 DMIPS\\
Напряжение питания&1,0 В&1,0 В\\
Занимаемая площадь кристалла&4 мм$^2$&1,74 мм$^2$\\
Потребляемая мощность&0,5 Вт&150 мВт\\
\hline
\end{tabular}
\end{center}
\end{table*}

     Общая организация гомогенного коммутатора и место в нем ИКМ-Б достаточно 
очевидны. Каждый ИКМ работает здесь как сетевой узел, используя любую известную 
сетевую технологию продвижения пакетов по <<сверхлокальной>> сети, возможно, 
адаптированную к особенностям конкретного интерфейса~--- SRIO или другого. Все ИКМ 
коммутатора совокупно выполняют маршрутизацию, поддерживая соответствующие 
маршрутизирующие протоколы на своих коммуникационных портах и обмениваясь между 
собой маршрутной информацией в форме сообщений.
     
     На рис.~\ref{f5e} показана структура ИКМ меньшей производительности на 
500$\ldots$1000 DMIPS (далее ИКМ-М), которые могут использоваться в узлах продвижения 
пакетов, т.\,е.\ в качестве ИКМ/DP на рис.~\ref{f3e}, и, возможно, некоторых узлах 
менеджмента гетерогенных коммутаторов, т.\,е.\ в качестве ИКМ/CP на рис.~\ref{f3e}.
     
     В ИКМ-М может быть задействовано более прос\-тое процессорное ядро, чем в ИКМ-Б, 
например ядро Cortex-R4~\cite{17e} с производительностью порядка 750~DMIPS. 
Сравнительные характеристики обоих ядер ARM, Cortex-A8 и Cortex-R4, приведены в 
табл.~\ref{t2e}.
     
     Еще одно отличие ИКМ-М от ИКМ-Б может заключаться в отказе от использования 
внешней памяти и, соответственно, исключении аппаратуры ее интерфейсов. Разумеется, 
такой шаг предполагает наличие в ИКМ-М достаточных ресурсов внутрикристальной 
памяти. Это могут быть памяти программ (ROM или Flash объемом до 512~Kбайт) и данных 
(SRAM до 64~Kбайт). Память данных может дополняться блоками DPM модулей Ethernet и 
SRIO, а также блоком LUT с небольшого объема памятью CAM для текущих MAC-адресов.







     Необходимо отметить, что исключение из ИКМ интерфейсов внешних памятей резко 
сокращает чис\-ло выводов корпуса ИКМ-М и, следовательно, в значительной степени 
стоимость прибора. В~структуре, показанной на рис.~\ref{f5e}, ИКМ-М требуется всего 
полтора десятка~(!)\ сигнальных выводов. Для сравнения современные ИКМ компании 
\textit{Freescale} выпускаются в корпусах с числом выводов, переваливающим за тысячу.
     
     Если организация гомогенного коммутатора и использование в ней ИКМ были 
тривиальны, то работа гетерогенного коммутатора требует пояснений.
     
     Дезинтегрированная архитектура пакетной коммутации~\cite{9e} предполагает, что 
входные порты расчленяют входящие Ethernet-кадры: заголовки кадра остаются в 
микроконтроллере порта (в случае ИКМ-М~--- в его внутренней памяти данных), а 
<<обезглавленное>> тело кадра сохраняется в общей буферной памяти коммутатора. В тех 
случаях, когда адрес назначения в оторванном заголовке принятого кадра имеется во 
внутренней CAM, ИКМ-М входного узла способен сам принять решение по классификации 
и продвижению этого кадра. После принятия решения о продвижении ИКМ-М отправляет в 
соответствующий выходной узел \textit{сообщение продвижения} с модифицированным 
заголовком скоммутированного кадра и ссылкой на буфер в общей памяти, где сохранено 
тело этого кадра. В остальных случаях ИКМ-М входного узла отправляет 
\textit{сообщение коммутации}, включающее оторванный заголовок и ссылку на тело 
соотнесенного кадра, коммутирующему процессору. Коммутирующий процессор, в 
типичном случае ИКМ-Б, после определения порта назначения высылает в соответствующий 
выходной узел \textit{сообщение продвижения} с новым заголовком и ссылкой на тело 
соотнесенного кадра, а во входной узел~--- \textit{сообщение обновления}, которое 
позволит ИКМ-М входного узла обновить содержимое своей CAM и в дальнейшем 
коммутировать кадры с теми же MAC-адресами самостоятельно. ИКМ-М выходного узла, 
получив \textit{сообщение продвижения}, ставит полученный заголовок со ссылкой на 
соотнесенное тело в одну из очередей на отправку в соответствии с требованиями QoS. 
Когда очередь доходит до этого заголовка, ИКМ-М склеивает его с телом кадра из буферной 
памяти и включает ссылку на готовый кадр в выходную цепочку описателей буферов блока 
Ethernet. После отправки кадра (или в зависимости от протокола после подтверждения его 
получения адресатом) ИКМ выходного узла отправляет \textit{сообщение освобождения} 
буфера в общей буферной памяти входному узлу, после чего процесс коммутации 
завершается.
     
     Если гетерогенный коммутатор включает отдельный узел или даже несколько узлов 
общего управления и менеджмента, то в таких узлах могут использоваться в зависимости от 
сложности решаемых задач как ИКМ-М, так и ИКМ-Б.
     
     Таким образом, и децентрализованная коммутация в гомогенном коммутаторе, и 
централизованная коммутация в гетерогенном коммутаторе предполагают на интерфейсе 
<<сверхлокальной>> сети:
     \begin{itemize}
\item потоковые пересылки больших инкапсулированных блоков данных (кадров между 
ИКМ или тел кадров между ИКМ и общей буферной памятью);
\item передачу между ИКМ коротких высокоприоритетных сообщений.
\end{itemize}

     Длина инкапсулированных блоков данных не имеет принципиального значения при 
условии их фрагментации, что характерно, например, для интерфейса SRIO. Длина 
сообщений определяется в основном размером оторванных заголовков кадров. В 
зависимости от уровня коммутации в коммутирующем процессоре это могут быть заголовки 
только L2 (кадра Ethernet) или также заголовки L3 и даже L4 (пакетов IP и TCP/UDP). 
Соответственно, длина сообщений может варьироваться от 32~байт в случае коммутации L2 
до 128~байт при коммутации L4 и версии~6 протокола IP. Все это вполне соответствует 
возможностям интерфейса SRIO, который поддерживает эффективный обмен сообщениями 
размером до 256~байт, а с использованием механизма SAR~--- до 4096~байт.
     
     Суммарная производительность гетерогенного коммутатора определяется двумя 
узкими местами: быстродействием общей буферной памяти и пропускной способностью 
звена интерфейса между контроллером буферной памяти с коммутационной структурой. 

Современная синхронная память способна пропустить через себя десятки гигабайт 
информации в секунду. Например, 64-раз\-ряд\-ная память QDR SRAM (4~прибора 
IDT71P74604), работающая на %\linebreak
частоте 250~МГц, пропускает через себя дуплексный поток 
32~Гбит/с и, следовательно, может обслу\-жить до 32~дуплексных гигабитных портов %\linebreak 
Ethernet. Однако максимальная пропускная способность звена SRIO $4\times$ составляет 
всего 10~Гбит/с, что ограничивает число гигабитных портов гетерогенного коммутатора 
десятью. Увеличивать пропускную способность интерфейса буферной памяти можно, 
добавляя звенья интерфейса SRIO, как показано пунктиром на рис.~\ref{f3e}. Однако такое 
решение предпо\-ла\-га\-ет динамическую маршрутизацию коммутатором SRIO потоков между 
параллельными звеньями, что нетипично для существующих коммутаторов SRIO. 

Другое 
решение заключается в замене для передачи тел блоков данных интерфейса SRIO 
интерфейсом PCIe, который обладает гораздо большей пропускной способностью: на звене 
$\times 16$ она достигает 64~Гбит/с, что позволяет обслуживать до 64~гигабитных портов. 
Но PCIe не поддерживает необходимый в децентрализованном пакетном коммутаторе обмен 
сообщениями, поэтому такое решение заставляет вводить в ИКМ-М в дополнение к модулю 
SRIO еще и блок PCIe, что выглядит неоправданно избыточным. 

Проблема может быть 
автоматически решена в будущем новыми версиями спецификации интерфейса SRIO, в 
которых пропускная способность звена будет существенно повышена до конкурентных с 
PCIe величин, либо переходом на архитектуру ASI. Но возможно и другое решение \textit{ad hoc} 
разработкой в виде отдельной микросхемы контроллера общей буферной памяти 
гетерогенного коммутатора, совмещенного с объединяющим ИКМ коммутатором SRIO, т.\,е.\ 
коммутаторного ядра на рис.~\ref{f3e}.

\section{Заключение}

     Интегрированные коммутационные микроконтроллеры~--- общепризнанный строительный материал для средств телекоммуникации и 
пакетной коммутации, который особенно удобен для типового <<крупноблочного>> 
строительства. Устройства на основе ИКМ могут быть широко востребованы в самых 
разных приложениях. Но вряд ли устройства, реализованные на ИКМ, достигнут рекордных 
показателей в своем классе. Характеристики таких устройств, скорее всего, окажутся на 
приличном среднем уровне, который достаточен для большинства типичных применений. 
Зато разработка на базе ИКМ может быть завершена быстро и с минимальными затратами.
     
     Хотя приборы ведущего поставщика ИКМ компании \textit{Freescale} становятся все 
более популярными и в России, сдерживающим фактором их активного использования в 
отечественных разработках останется цена, которая в зависимости от сложности 
микросхемы колеблется от десятков до сотен долларов США. Кроме того, есть области, 
связанные с телекоммуникацией и пакетной коммутацией, в которых применение 
импортных комплектующих,\linebreak в том числе ИКМ, нежелательно или невозможно по 
различным причинам, в том числе нетехнического 
характера. В этих областях были бы 
активно востребованы приборы отечественного производства. %\linebreak 
Одна\-ко разработка в России 
современных\linebreak ИКМ, эквивалентных по своим возможностям лучшим образцам от компании 
\textit{Freescale}, сегодня совершенно нереальна. Настоящая статья указывает один из путей 
решения проблемы разработкой ограниченной номенклатуры отечественных ИКМ, 
существенно более простых, чем приборы компании \textit{Freescale}, но позволяющих, тем 
не менее, %\linebreak 
решать серь\-ез\-ные задачи на современном техническом уровне как минимум в 
области пакетной коммутации.
     
     Таким решением могли бы стать два ИКМ, меньшей и большей производительности, с 
программируемыми процессорными ядрами семейства Cortex, лицензионно поставляемых 
компанией \textit{ARM}, и оснащенных по минимуму все\-го-на\-всего двумя 
специализированными коммуникационными модулями: трехскоростным Ethernet и 
интерфейсом (типа) SRIO. С использованием этих двух вариантов ИКМ можно было бы 
строить высокопроизводительные <<распределенные>> маршрутизирующие коммутаторы с 
гомогенной и гетерогенной %\linebreak
организацией. Такие коммутаторы, способные предо\-ста\-вить 
пользователю до нескольких десятков гигабитных портов Ethernet, могли бы, помимо 
коммутации на уровнях 2$\ldots$4, также выполнять %\linebreak
функции граничных маршрутизаторов в 
\mbox{сетях} Metro Ethernet и MPLS, обеспечивать сетевую защиту и эффективно решать целый ряд 
других смежных задач по мере развития собственного прикладного программного 
обеспечения.
     
     Можно ожидать, что предлагаемые упрощенные ИКМ при нормально организованном 
крупносерийном производстве могли бы оказаться существенно дешевле более сложных 
ИКМ компании \textit{Freescale} и вполне конкурентоспособны во многих приложениях, 
причем не только на внутреннем, но и на мировом рынке.

{\small\frenchspacing
{%\baselineskip=10.8pt
\addcontentsline{toc}{section}{Литература}
\begin{thebibliography}{99}

\bibitem{4e} %1
\Au{Егоров~В.\,Б.} 
Новое поколение коммуникационных микроконтроллеров компании Freescale 
Semiconductor~// Chip News, 2007. No.\,3. С.~14--18.

\bibitem{5e} %2
\Au{Егоров~В.\,Б.}
Интегрированные коммуникационные процессоры компании Freescale 
Semiconductor~// Электронные компоненты, 2007. №\,8. С.~85--89.

\bibitem{10e} %3
PowerQUICC Communications Processors~// Сайт компании Freescale Semiconductor: 
{\sf http://www.freescale. com/webapp/sps/site/homepage.jsp?nodeId= 0162468rH3bTdGJk19}.

\bibitem{8e} %4
\Au{Егоров~В.\,Б.}
Интегрированные коммуникационные микроконтроллеры компании <<Freescale 
Semiconductor>> и их применение в пакетной коммутации.~--- М.: ИПИ РАН, 2008.  129~с.

\bibitem{3e} %5
\Au{Соколов~И.\,А., Гребенщиков~В.\,И., Егоров~В.\,Б. и~др.}
Прототип многофункционального 
устройства пакетной коммутации~// Системы и средства автоматики. Вып. 16.~--- М.: Наука, 
2006. С.~449--462.

\bibitem{6e} %6
\Au{Егоров~В.\,Б.}
Опыт разработки пакетных коммутаторов и маршрутизаторов на основе интегрированных 
коммуникационных процессоров~// Материалы Международной научно-технической 
конференции ``Intermatic-2007''. Москва. Ч.~2.~--- М.: МИРЭА, 2007. С.~148--151.

\bibitem{1e} %7
\Au{Егоров~В.\,Б.}
Принципы создания коммутационной аппаратуры на основе специализированных 
микроконтроллеров~// Системы и средства автоматики. Вып.~9.~--- М.: Наука, 1999. С. 44--55.

\bibitem{2e} %8
\Au{Соколов~И.\,А., Егоров В.\,Б.}
Дезинтеграционный подход к архитектуре универсального 
процессора коммутации пакетов~// 
Информационные технологии и вычислительные системы, 
2005. №\,2. С.~76--85.

\bibitem{11e} %9
MPC8568E PowerQUICC III Processor~// Сайт компании Freescale Semiconductor: 
{\sf 
http://www.freescale. com/webapp/sps/site/prod\_summary.jsp?code= MPC8568E\&nodeId=0162468rH3bTdGJk196465}.


\bibitem{7e} %10
\Au{Егоров~В.\,Б.}
О возможном подходе к созданию отечественных технических средств пакетной 
коммутации двойного назначения~// Информационные технологии управления информационными 
ресурсами двойного применения (III)~/ Препринт (ДСП).~--- М.: ИПИ РАН, 2007. 
С.~26--34.


\bibitem{9e} %11
\Au{Соколов~И.\,А., Егоров~В.\,Б.}
Дезинтегрированная архитектура пакетной коммутации~// 
Информатика и её применения, 2008. Т.~2. Вып.~4. С.~2--11.


\bibitem{12e}
MPC8548E PowerQUICC III Processors~// Сайт компании Freescale Semiconductor: 
{\sf http://www. freescale.com/webapp/sps/site/prod\_summary.jsp?code =MPC8548E\&nodeId=0162468rH3bTdGJk196465}.

\bibitem{13e}
RapidIO Specifications~// RapidIO WEB site: {\sf http:// www.rapidio.org/specs/current}.

\bibitem{14e}
Serial RapidIO Preprocessing Switches~// Сайт компании IDT: {\sf http://www.idt.com/?catID=18388262}.

\bibitem{15e}
RapidIO Switches~// Сайт компании Tundra: {\sf http:// www.tundra.com/products/rapidio-switches}.

\bibitem{16e}
Application Processors~// Сайт компании ARM: {\sf http:// www.arm.com/products/CPUs/application.html}.

  \label{end\stat}
  
\bibitem{17e}
Embedded Processors~// Сайт компании ARM: {\sf http:// www.arm.com/products/CPUs/embedded.html}.
\end{thebibliography}
}
}
\end{multicols}  
 
 
 
 
 