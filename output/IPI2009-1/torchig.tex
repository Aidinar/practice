\def\stat{torch}

\def\tit{ОБ ОДНОМ ПОДХОДЕ К ФОРМИРОВАНИЮ ИЗОБРАЖЕНИЙ 
БЕЗ~ИСПОЛЬЗОВАНИЯ ЭКРАНА}
\def\titkol{Об одном подходе к формированию изображений 
без использования экрана}

\def\autkol{А.\,В. Торчигин}
\def\aut{А.\,В. Торчигин$^1$}

\titel{\tit}{\aut}{\autkol}{\titkol}

%{\renewcommand{\thefootnote}{\fnsymbol{footnote}}\footnotetext[1]
%{Работа выполнена при поддержке
%Российского фонда фундаментальных исследований,
%гранты 06-07-89056 и 08-07-00152.}}

\renewcommand{\thefootnote}{\arabic{footnote}}
\footnotetext[1]{Институт проблем
информатики Российской академии наук, torchigin\_a@mail.ru}

  
\vspace*{-12pt}
        
   \Abst{Рассматриваются свойства изображений, возникающих при наблюдении в 
колеблющемся зеркале линейки модулируемых по яркости светодиодов. 
Проанализированы возможные области применения этого подхода.}
   
   \KW{формирование изображений; виртуальное окружение; виртуальная реальность; 
объемное изображение}

           \vskip 24pt plus 9pt minus 6pt

      \thispagestyle{headings}

      \begin{multicols}{2}

      \label{st\stat}
      
%      \vspace*{-12pt}
   
\section{Введение}

        \vspace*{-3pt}
  
  Привычные изображения обычно формируются на некотором основании. Это может 
быть скала, на которую первобытный человек наносил свои примитивные рисунки. Это 
может быть холст, на котором гениальные художники эпохи Возрождения создавали свои 
шедевры. Это может быть экран телевизора или монитора. В любом из подобных случаев 
изображение представляет собой двумерную картину. С недавнего времени стали 
известны стереоизображения, которые невозможно совместить с плоским экраном. 
Однако и такие изображения создаются с помощью экрана, на который проецируют два 
изображения соответственно для левого и правого глаза. Зритель снабжается очками, 
пропускающими  свет от изображения, приготовленного специально для левого и правого 
глаза. Этот способ требует больших помещений, двух мощных проекторов и специального 
дорогостоящего экрана. 
  
  Примером изображения, которое существует без экрана, является изображение картины 
реального мира в обыкновенном плоском зеркале. Разглядывая такое изображение в 
зеркале, зритель видит объемную картину, которая находится в глубине зеркала. Размеры 
этой картины могут существенно превосходить размеры зеркала. Например, в зеркале над 
умывальником зритель может видеть себя в полный рост. В предлагаемой статье 
рассматриваются возможности существующих технических средств для формирования 
подобных изображений без использования экранов и соответствующих проекторов. 

        \vspace*{-6pt}

\section{Принцип действия}

        \vspace*{-3pt}

  Разглядывая в зеркале отражение светящихся светодиодов, расположенных вдоль 
  \columnbreak
некоторой вертикальной линии на одинаковом минимальном расстоянии друг от друга, 
зритель видит светящуюся вертикальную линейку светодиодов. При повороте зеркала 
вокруг вертикальной оси эта линейка сдвигается в зеркале в горизонтальном направлении. 
В зеркале, совершающем вращательные колебания вокруг вертикальной оси, зритель 
увидит светящийся прямоугольник, ширина которого зависит от амплитуды колебаний 
зеркала. Если при этом светодиоды модулируются во времени соответствующим образом, 
то вместо однородно светящегося прямоугольника зритель может видеть 
соответствующую картину. Эта картина расположена не в плоскости зеркала (плоскость 
колеблется с частотой около 25~Гц), а в глубине зеркала на удалении, равном расстоянию 
от зеркала до линейки светодиодов. 

\begin{figure*} %fig1
\vspace*{1pt}
\begin{center}
\mbox{%
\epsfxsize=145.645mm
\epsfbox{tor-1.eps}
}
\end{center}
\vspace*{-9pt}
\Caption{Фотографии одного и того же изображения, формируемого в прикрепленном к стене 
колеблющемся зеркале линейкой модулируемых по яркости светодиодов. Правая фотография сделана 
фотоаппаратом, находящимся на меньшем расстоянии от зеркала, чем левая 
\label{f1t}}
\end{figure*}
  
  Если амплитуда колебаний зеркала достаточно велика, то зритель будет видеть в 
зеркале только часть картины. Однако если повернуть колеблющееся зеркало вокруг 
вертикальной оси на некоторый угол, то зритель увидит другой фрагмент формируемой 
картины. Подобно человеку, который разглядывает в карманное зеркало различные 
фрагменты отражения реального мира путем изменения ориентации этого зеркала, 
зритель, изменяя ориентацию колеблющегося зеркала, может разглядывать различные 
фрагменты формируемого изображения. Более того, он может не изменять ориентацию 
зеркала, а смотреть на него под разными углами. В этом случае зритель также будет 
видеть различные фрагменты формируемой картины. Приближая зеркало к глазам, он 
будет видеть картину под большим углом зрения, т.\,е.\ будет видеть больший фрагмент 
картины. При этом сам размер изображения остается неизменным. В качестве 
иллюстрации на рис.~\ref{f1t} показаны фотографии одного и того же изображения, 
сделанные фотоаппаратом, находящимся на разном расстоянии от колеблющегося 
зеркала. При приближении фотоаппарата к зеркалу увеличивается фиксируемый 
аппаратом фрагмент надписи. 
  
 
  Зеркало можно рассматривать как окно в виртуальный мир. Находясь в реальном мире 
и подходя к колеблющемуся зеркалу, зритель видит не зеркало (зеркало он видеть не 
может, так как оно колеблется и подобно спицам вращающегося колеса становится 
невидимым), а нечто, нарушающее привычную картину реального мира. Заглядывая в это 
нечто, зритель видит картину виртуального мира, созданного модулируемыми по яркости 
светодиодами. Для зрителя это нечто представляется в виде некоторого отверстия в 
реальном мире, через которое он может разглядывать виртуальный мир. Чем ближе его 
глаза находятся к этому отверстию, тем больше телесный угол, в котором виден 
виртуальный мир. Подобно тому как, рассматривая через обычное отверстие реальный 
мир, зритель может видеть при изменении позиции относительно отверстия разные 
фрагменты реального мира, так и разглядывая через это нечто виртуальный мир, зритель 
может видеть при изменении позиции относительно этого нечто разные фрагменты 
виртуального мира. Отличие лишь в том, что зритель может изменять ориентацию этого 
нечто и видеть при этом различные фрагменты виртуального мира. Приведенный способ 
формирования изображений описан в патенте автора~\cite{1t}. 

\section{Теоретические предпосылки}

  Построим изображение светодиода при различных угловых положениях вращающегося 
зеркала, показанного жирной линией на рис.~2. %\ref{f2t}. 
Имея в виду,\columnbreak

\vspace*{1pt}

% \noindent
%\label{f1s}}
%\end{figure*}
%\vspace*{2ex}
\begin{center}
\mbox{%
\epsfxsize=41.463mm
\epsfbox{tor-2.eps}
}
\end{center}
%\vspace*{6pt}
{{\figurename~2}\ \ \small{Траектория~\textit{3} перемещения мнимого изображения~\textit{4} источника~\textit{2} при 
вращении зеркала~\textit{1}}}


\bigskip
\addtocounter{figure}{1}  



\noindent
что изображение 
светодиода в зеркале расположено в точке, симметричной светодиоду относительно 
плоскости зеркала, получим, что изображение светодиода~\textit{2} при положении 
зеркала~\textit{1} окажется в точке~\textit{4}. При вращении зеркала~\textit{1} 
изображение светодиода~\textit{2} описывает окружность~\textit{3} с центром~$O$ на оси 
вращающегося зеркала~\textit{1}. Радиус этой окружности равен расстоянию от оси 
вращающегося зеркала~\textit{1} до светодиода~\textit{2}. Однако зритель, 
расположенный в некоторой точке~\textit{5}, может не увидеть всю окружность по 
нескольким причинам. 

  Во-первых, угол обзора ограничен строением глаз человека, который, не поворачивая 
головы, может видеть в секторе около 90$^\circ$. 
  
  Во-вторых, угол обзора изображения ограничен размерами зеркала и расстоянием от 
оси вращения зеркала до зрителя. Для того чтобы изображение светодиода было видно 
зрителю, необходимо, чтобы прямая линия, соединяющая изображение светодиода в 
точке~\textit{4} с точкой~\textit{5}, где находится зритель, пересекала зеркало~\textit{1}. 

%  \begin{figure*} %fig3

\begin{center}
\vspace*{3pt}

\mbox{%
\epsfxsize=41.685mm
\epsfbox{tor-3.eps}
}
\end{center}
%\vspace*{-9pt}
{{\figurename~3}\ \ \small{Дуга, которую видит зритель~\textit{5} во вращающемся зеркале~\textit{1} 
ограниченных размеров}}


\bigskip
\addtocounter{figure}{1}  
%  \label{f3t}}
%  \end{figure*}
 
  
  На рис.~3 %\ref{f3t} 
  показан в качестве примера для конкретного положения зрителя и 
размера зеркала фрагмент окружности между точками~\textit{6} и~\textit{7}, по
которому 
перемещается изображение светодиода. Только этот фрагмент виден зрителю, 
расположенному в точке~\textit{5}. Чем ближе зритель расположен к вращающемуся 
зеркалу и чем больше горизонтальный размер $2R$ вращающегося зеркала, тем больший 
фрагмент окружности может видеть зритель. 
  
  На рисунках показана лишь одна окружность, создаваемая одним светодиодом. В том 
случае, если сразу несколько светодиодов расположено на прямой, параллельной оси 
вращения, изображение каждого из них при вращении зеркала будет описывать 
идентичную окружность, расположенную в плоскости, параллельной плоскости 
рас\-смот\-рен\-ной окружности. Набор таких окружностей формирует изображение, подобно 
тому как набор прямых линий в виде строк формирует изображение на экране телевизора. 
Формируемое изображение состоит из набора вертикальных линеек светящихся 
светодиодов, сдвинутых относительно друг друга в горизонтальном направлении на 
некоторое расстояние. Сдвиг образуется из-за того, что линейка светодиодов отражается в 
колеблющемся зеркале.
  
  Угол, в котором зритель может видеть формируемое изображение, определяется 
отношением $s/r$, где $s$~--- размер зеркала в горизонтальном направлении, $r$~--- 
расстояние от центра зеркала до зрачков зрителя. Это означает, что при приближении 
зеркала к глазам зрителя поперечные размеры зеркала могут быть уменьшены. В случае 
если колеблющееся зеркало расположено в непосредственной близости от зрачков 
зрителя, его размеры могут быть сделаны сравнимыми с размером стекол в очках. При 
этом угол формируемого изображения может даже превосходить угол, в котором глаза 
зрителя способны видеть изображение. 
  
  Использование зеркал, совершающих вращательные колебания вокруг оси вращения 
(ко\-леб\-лю\-щих\-ся зеркал), имеет несколько преимуществ перед использованием 
вращающихся зеркал. Во-пер\-вых, при использовании вращающихся зеркал картина 
формируется только в течение небольшого времени, когда положение зеркал таково, что 
лучи света от модулируемых светодиодов попадают в глаза зрителя. Обычно это время 
составляет менее 10\% от периода вращения. Это приводит к тому, что б$\acute{\mbox{о}}$льшую часть 
времени свет от модулируемых светодиодов не может попадать в глаза зрителя, что 
снижает максимальную яркость формируемого изображения. Во-вторых, если в 
помещении находятся другие источники света, например окна, лампы освещения, то при 
определенном положении вращающихся зеркал свет от них попадает в глаза зрителя. Это 
приводит к тому, что изображение формируется на неконтролируемом светлом фоне. 
  В-третьих, колеблющиеся зеркала можно располагать ближе к зрачкам зрителя, чем 
вращающиеся. Это позволяет увеличить максимальный телесный угол, в котором может 
формироваться изображение. 
  
  Если ось колеблющегося зеркала закрепить на платформе, жестко связанной с головой 
зрителя, например на шлеме или очках, то при повороте головы зритель будет видеть 
другой фрагмент окружности на рис.~3. У него создастся впечатление, что он со 
всех сторон окружен виртуальным миром. При повороте глаз зритель начинает видеть 
другой фрагмент виртуального мира, точно так же как в реальном мире при повороте глаз 
он видит другой фрагмент реального мира. Как известно, движение зрачков происходит 
непроизвольно и достаточно быстро. Это движение не регистрируется известными 
системами создания виртуальной реальности, которые обычно реагируют только на 
поворот головы.

  \begin{figure*} %fig4
\vspace*{1pt}
\begin{center}
\mbox{%
\epsfxsize=152.511mm
\epsfbox{tor-4.eps}
}
\end{center}
\vspace*{-9pt}
  \Caption{Схема формирования различных изображений для левого и правого глаза от 
одного комплекта светодиодов. При первом положении зеркал~(\textit{а}) светодиоды 
видны левым глазом и не видны правым. При втором положении зеркал~(\textit{б}) 
светодиоды видны правым  глазом и не видны левым
  \label{f4t}}
  \end{figure*}
  
  До настоящего времени речь шла об одном зрителе. Однако все вышесказанное 
справедливо для любого другого зрителя, который разглядывает линейку светодиодов 
через свою систему зеркал. Таким образом, в виртуальный мир может быть одновременно 
погружено сколь угодно много зрителей, каждый из которых будет иметь перед глазами 
свой фрагмент виртуального мира, точно так же как многие зрители в реальном мире 
видят каждый свой фрагмент, который им доступен из их местоположения. Однако, так 
же как и в реальном мире, иногда зрители могут мешать друг другу, закрывая собой 
некоторые фрагменты виртуальной картины. 
  
  Рассмотренный подход можно легко приспособить для формирования 
стереоизображений. Схема получения стереоизображения показана на рис.~\ref{f4t}. 
Здесь светодиоды~\textit{1} формируют изображение для левого глаза~\textit{3} в тот 
интервал времени, когда свет от источников попадает в левый глаз, отражаясь от 
зеркала~\textit{4}, совершающего вращательные колебания. При этом свет от тех же 
светодиодов не попадает в правый глаз, так как от зеркала~\textit{5} он отражается в 
другую сторону (рис.~\ref{f4t},\,\textit{а}). В следующий интервал времени~--- через 
полпериода колебаний зеркал~--- последние принимают положение, показанное на 
рис.~\ref{f4t},\,\textit{б}. В этом случае ситуация изменяется на противоположную. Свет 
от светодиодов, отражаясь от колеблющегося зеркала~\textit{5}, попадает в правый 
глаз~\textit{2} и не попадает в левый после отражения от зеркала~\textit{4}. Таким 
образом, каждый глаз видит формируемую для него картину, а у зрителя создается 
впечатление, что он видит стереоизображение.
  
  Отметим, что в рассматриваемом подходе при желании могут быть использованы все те 
средства, которые используются в существующих сис\-те\-мах. Действительно, можно 
использовать два комплекта светодиодов, предназначенных соответственно для левого и 
правого глаза. Один комплект покрыт прозрачной пленкой, пропускающей поляризацию, 
предназначенную для левого глаза, а другой~--- для правого. Дополнительно к очкам с 
колеблющимися зеркалами добавляются обычные поляризационно-чувствительные очки. 
В этом случае интенсивность формируемой картины увеличивается вдвое, так как оба 
глаза воспринимают одновременно предназначенные для них изоб\-ра\-же\-ния. В этом случае 
отпадает необходимость в дорогостоящем экране, сохраняющем поляризацию 
отраженного света, и мощных двух проекторах, обеспечивающих формирование на экране 
двух изображений. Возможны и другие варианты получения необходимых изображений 
для левого и правого глаза. 
  
  
  Следует обратить внимание, что при изменении местоположения зрителя картина 
виртуального мира, которую он видит, изменяется. Таким образом, если несколько 
зрителей снабжены очками с колеблющимися зеркалами, то каждый зритель оказывается 
погруженным в виртуальное окружение. Это обстоятельство позволяет погружать в 
виртуальное окружение большие зрительские аудитории. При этом, в отличие от 
существующих систем, требующих больших помещений, экранов большой площади, 
мощных дорогостоящих проекторов, в рассматриваемом подходе виртуальное окружение 
может быть создано в любом пространстве. Для этого достаточно лишь поместить в 
любую комнату вертикальную линейку светодиодов. Из любого мес\-та, откуда видна эта 
линейка, зритель, надев очки, погружается в виртуальное окружение. 
  

  
  
  Весьма интересен подход, который позволяет создавать стереоизображение для многих 
зрителей, не требуя при этом, чтобы зрители надевали какие-либо очки. Рассмотрим 
простейший пример. На рис.~5 %\ref{f5t} 
показаны две линейки светодиодов~\textit{1} 
и~\textit{2}, расположенные на разных расстояниях от вращающегося или колеблющегося 
зеркала. Если, например, с помощью линейки~\textit{1} формировать изображение одного 
человека, а с помощью линейки~\textit{2}~--- изоб\-ра\-же\-ние другого человека, то зритель 
будет видеть\linebreak


%  \begin{figure*} %fig5
%\vspace*{1pt}
\begin{center}
\mbox{%
\epsfxsize=38.754mm
\epsfbox{tor-5.eps}
}
\end{center}
%\vspace*{-9pt}
{{\figurename~5}\ \ \small{Зритель~\textit{3} видит две картины на разных расстояниях от зеркала. Без каких-либо очков 
получается объемное изображение}}


\bigskip
\addtocounter{figure}{1}  
%  \label{f5t}}
%  \end{figure*}

\noindent
 в зеркале двух людей, расположенных от него на разных расстояниях. 
Действительно, в обычном зеркале зритель прекрасно отличает человека, стоящего вблизи 
зеркала, от человека, расположенного от зеркала на большом расстоянии. 
     \begin{figure*}[b] %fig6
%\vspace*{6pt}
\begin{center}
\mbox{%
\epsfxsize=145.268mm
\epsfbox{tor-6.eps}
}
\end{center}
\vspace*{-9pt}
   \Caption{Аппаратные средства для управления светодиодами
   \label{f6t}}
   \end{figure*}

   Аналогичным 
образом в колеблющемся зеркале зритель отличает изображение, созданное линейкой 
светодиодов, расположенной вблизи зеркала, от изображения, сформированного 
линейкой, расположенной на значительном удалении от зер-\linebreak
кала. 


  Обобщая этот прием, можно расположить $N$ линеек светодиодов на расстояниях $R$, 
$R + \Delta R$, \ldots , $R + (N-1)\Delta R$ от оси вращения зеркала, а отоб\-ра\-жа\-емую 
объемную картину разбить на систему колец толщиной $\Delta R$ и внутренними 
диаметрами $R$, $R + \Delta R$, \ldots , $R + (N-1)\Delta R$. При этом фрагмент 
изображения, попадающий в $i$-е кольцо, следует показывать с помощью $i$-й линейки 
светодиодов. Так как $i$-я линейка формирует изображение, отстоящее от оси вращения 
на расстояние $R + (i-1)\Delta R$, то всеми линейками светодиодов будут сформированы 
фрагменты изображения, отстоящие от оси вращения на разные расстояния, т.\,е.\ будет 
сформировано изображение, фрагменты которого находятся на разном удалении от оси 
вращения. Иными словами, будет сформировано объемное изображение. Как следует из 
проведенного выше рас\-смот\-ре\-ния, это объемное изображение могут наблюдать без 
  каких-либо очков зрители, находящиеся в любом месте, откуда видны в колеблющемся 
зеркале линейки светодиодов. Имеются ограничения на формируемые таким образом 
объемные изображения, однако подобные ограничения существуют и для известных 
систем формирования стереоизображений. В простейшем частном случае двумя 
линейками светодиодов можно создавать передний и задний планы, которые 
воспринимаются зрителем в виде объемной картины. 

\section{Особенности реализации}

  Любая из рассмотренных систем включает в себя три основных компонента:
\begin{enumerate}[(1)]
\item линейку светодиодов с системой, обеспечи\-ва\-ющей модуляцию их яркости во 
времени; 
\item зеркало, совершающее вращательно-ко\-ле\-ба\-тель\-ные движения; 
\item средства, обеспечивающие синхронизацию колебаний зеркала с модуляцией 
светодиодов.  
\end{enumerate}
  
  Казалось бы, наиболее сложным является первый компонент. Однако при тщательном 
анализе оказалось, что для его реализации с успехом могут быть использованы 
аппаратные средства, разработанные для жидкокристаллических (ЖК) мониторов, структурная 
схема которых приведена на рис.~\ref{f6t}. 
  
  Здесь подготовленная в персональном компьютере (ПК) информация для показа на 
экране ЖК-мо\-ни\-то\-ра передается по стандартному интерфейсу на вход контроллера 
монитора, который формирует соответствующие сигналы, поступающие на 
горизонтальные и вертикальные шины ЖК-панели. С целью 
сокращения числа проводов между контроллером и панелью (ЖК-па\-нель имеет более 
1000~вертикальных шин) сигналы от контроллера передаются последовательно по 
меньшему числу проводов (52~провода при представлении сигналов в виде стандартных 
сигналов транзисторно-транзисторной логики (TTL~--- transistor-transistor logic)
либо около 20~проводов при представлении сигналов в виде 
парофазных низковольтных дифференциальных сигналов (LVDS~--- low voltage differetial signal)). 
На ЖК-па\-не\-ли в качестве ее 
неразрывной части изготавливается специальный драйвер, преобразующий 
последовательность входных сигналов в выходные, поступающие на вертикальные и 
горизонтальные шины ЖК-матрицы. Вообще говоря, для управления яркостью 
светодиодов в линейке можно было бы использовать эти сигналы. Однако три 
обстоятельства, относящиеся к технической реализации, препятствуют этому. 
  
  
  Во-первых, электрические параметры этих сигналов (мощность и напряжение) не 
соответствуют требованиям светодиодов. Необходимо ставить дополнительный 
преобразователь для каждого такого сигнала. 
  
  Во-вторых, на выходе драйвера шаг проводников, по которым сигналы поступают на 
шины ЖК-матрицы, составляет около 0,05~мм. Подсоединиться к таким проводникам без 
специального технологического оборудования не представляется возможным. 
  
  В-третьих, различаются способы получения полутонов для жидкокристаллических 
ячеек и светодиодов. 
  
  Таким образом, только чисто технические проблемы не позволяют использовать уже 
имеющееся оборудование для модуляции светодиодов. Что касается алгоритмов работы 
аппаратуры монитора, то они такие же, как и для линейки светодиодов. Действительно, 
существующая аппаратура обеспечивает подачу сигналов на вертикальные шины 
  ЖК-мат\-ри\-цы и одну из горизонтальных шин. При этом высвечивается одна строка 
пикселов ЖК-мат\-ри\-цы. Затем в следующий временной интервал высвечивается 
следующая строка пикселов. 
  
  При получении изображения с помощью светодиодов линейка светодиодов 
соответствует одной строке матрицы пикселов. Следующая строка изображения 
формируется той же линейкой светодиодов за счет того, что в следующем временном 
интервале изображение линейки светодиодов оказывается сдвинутым из-за колебаний 
зеркала. Таким образом, требования к аппаратуре со стороны линейки светодиодов даже 
слабее, чем требования к аппаратуре ЖК-монитора, так как отсутствует необходимость 
подачи сигналов на горизонтальные шины (такие шины просто отсутствуют). Число 
светодиодов в линейке приблизительно в 1000~раз (по числу строк в мониторе) меньше 
числа пикселов в мониторе. 
  
  С учетом приведенных соображений наиболее целесообразно использовать контроллер 
ЖК-мо\-ни\-то\-ра, а драйвер для светодиодов разработать с учетом требований со стороны 
последних и име\-юще\-го\-ся стандартного оборудования, выпускаемого в настоящее время 
многими фирмами для управ\-ле\-ния яркостью светодиодов. На альтернативное 
оборудование, не входящее в состав монитора, указывают на рис.~\ref{f6t} пунктирные 
стрелки. В настоящее время развитие техники ЖК-мониторов достигло такого уровня, что 
стали коммерчески доступны контроллеры для ЖК-мониторов. Примером может служить 
появившийся в последнее время контроллер типа SVT-1920. До этого каждая фирма, 
производящая ЖК-мониторы, разрабатывала собственные контроллеры, правда на основе 
стандартных, предназначенных для этих целей сверхбольших интегральных схем (СБИС). 
  
  Аппаратные средства светодиодных табло типа бегущая строка также могут быть 
использованы для модуляции линейки светодиодов. В этом случае табло также 
представляется в виде матрицы, число строк в которой ограничено несколькими 
десятками. Формирование изображения происходит точно так же, как и в ЖК-матрице. 
Однако в бегущих строках отсутствует реализация полутонов. Использование техники 
широтно-импульсной модуляции (ШИМ) позволяет решить эту проблему. Таким образом, уже существующие в настоящее 
время аппаратные средства можно приспособить для формирования изображений от тех 
же источников, что и на экране монитора ПК (телевизионный приемник, видеокамера, CD, 
Интернет и~т.\,д.). 
  
  Что касается второго компонента рассматриваемого устройства, т.\,е.\ зеркала, 
совершающего вращательно-колебательные движения, то такое зеркало может быть 
самых разных размеров в зависимости от области применения. В работе~\cite{2t} 
приведены фотографии нескольких конструкций таких зеркал. Различные зеркала могут 
использоваться с одной и той же линейкой светодиодов. Необходимо лишь обеспечить, 
чтобы период и фаза колебаний различных зеркал были одинаковы. 
  
  Для этих целей предназначен третий компонент, состоящий из радиопередатчика, 
связанного с линейкой светодиодов и радиоприемниками, по одному около каждого 
колеблющегося зеркала. При начале показа очередного кадра на линейке светодиодов 
передатчик выдает радиосигнал, который принимается каждым приемником и поступает 
на систему автоподстройки частоты, обеспечивающей определенную фазу колебаний 
зеркала в момент прихода этого сигнала. Любая из существующих в настоящее время 
систем беспроводной связи может быть использована для решения этой простейшей 
задачи. 
  
  Некоторые сомнения у потенциальных потребителей вызывает то обстоятельство, что в 
устройстве используется механическое движение в виде вращательных колебаний. Такой 
тип механического движения рассматривается исключительно в целях наглядности. 
Сплошное плоское зеркало может быть заменено уже существующими и широко 
используемыми в DLP (digital light processing) проекторах DMD (digital micromirror device) матрицами~\cite{3t}. Эти 
матрицы пред\-став\-ля\-ют собой множество плоских микрозеркал, угол поворота каждого из 
которых задается соответствующим электрическим сигналом. Такие зеркала в состоянии 
совершать колебательно-вращательные движения с частотой в десятки килогерц. 
   В рассматриваемом применении условия работы таких зеркал гораздо проще. Все они 
должны синхронно совершать одинаковые вращательно-колебательные движения с 
частотой около 100~Гц. 

  
  Кроме того, необходимое изменение угла отражения лучей от зеркала может быть 
получено и без использования механических перемещений. С этой целью колеблющиеся 
зеркала могут быть заменены неподвижными дифракционными решетками, период 
которых управляется электрическими сигналами. Угол отклонения проходящих через 
дифракционную решетку лучей зависит от ее периода. При изменении периода 
изменяется угол отклонения лучей и у зрителя создается впечатление, что лучи попадают 
в его глаза по разным направлениям. Один из вариантов такой дифракционной решетки 
описан в патенте~\cite{4t}. Известно также несколько более ранних патентов с описанием 
дифракционных решеток, в которых угол отклонения светового луча управляется 
электрическими сигналами~\cite{5t}. 

%\vspace*{-6pt}

\section{Области применения }

%\vspace*{-3pt}

  Изложенный подход к формированию изображений может быть использован в 
различных применениях. Рассмотрим некоторые из них. 

%\vspace*{-3pt}

\subsection{Просмотр телевизионных передач}
  
  Как известно, в настоящее время громоздкие телевизоры с электронно-лучевыми 
трубками постепенно вытесняются еще более громоздкими и дорогостоящими 
телевизорами с использованием жидкокристаллических или плазменных плоских панелей. 
Габариты телевизора определяются размером его экрана, который уже может достигать 
двух метров по диагонали. Такой телевизор весит более 100~кг, и для него требуется 
достаточно большое помещение. Вместе с тем рассмотренный подход позволяет 
формировать телевизионные изображения, размер которых значительно превышает этот 
предел и превосходит даже размеры комнаты. 
  
  
  На рис.~7 показана схема такого формирования. Для этого требуется линейка 
светодиодов и колеблющееся зеркало. В частном случае линейка светодиодов может быть 
прикреплена к потолку, а колеблющееся зеркало размещено на столе. Размеры этого 
зеркала сравнимы с тарелкой, которая\linebreak\vspace*{-12pt}
%\columnbreak
%  \begin{figure*} %[b] %fig7

\begin{center}
\vspace*{1pt}
\mbox{%
\epsfxsize=79.924mm
\epsfbox{tor-7.eps}
}
\end{center}
%\vspace*{-6pt}
{{\figurename~7}\ \ \small{Схема просмотра телевизионной передачи зрителем~\textit{1} в колеблющемся 
зеркале~\textit{2}, в котором отражается линейка светодиодов~\textit{3}}}


\vspace*{6pt}
\bigskip
\addtocounter{figure}{1}  

%   \label{f7t}}
%   \end{figure*}
   
     


\noindent
 представляется зрителю в виде волшебного блюдца, 
в котором он может наблюдать изображение, превосходящее размеры комнаты. 
Действительно, если зеркало колеблется вокруг горизонтальной оси, параллельной 
линейке светодиодов, то горизонтальный размер изображения равен длине этой линейки, 
а вертикальный размер определяется амплитудой колебаний зеркала и может быть 
значительно больше. Весьма ценным свойством такой установки является то 
обстоятельство, что зритель при просмотре телевизионных программ не мешает другим 
обитателям помещения. Ему предоставлено отверстие в виртуальный мир, через которое 
он может наблюдать крупноформатное изображение. При желании другие зрители могут 
наблюдать виртуальный мир через то же или другое колеблющееся зеркало. Следует 
отметить, что колеблющееся зеркало представляет собой чрезвычайно простое и дешевое 
устройство, состоящее из обычного плоского зеркала и электромотора от компьютерного 
вентилятора, обеспечивающего изменение ориентации зеркала. Что касается линейки 
светодиодов, то она может использоваться не только для формирования изображений, но 
и в качестве осветительного прибора. По существующим прогнозам, светодиодные 
источники света придут на смену традиционным. В~этом случае линейка светодиодов 
может использоваться в качестве источника света, который может регулироваться в 
широких пределах по интенсивности, спектральному составу и местоположению. 


\subsection{Полиэкранный монитор}

  В некоторых применениях одного экрана обычного монитора оказывается 
недостаточно. На\-при\-мер, рабочее место брокера на бирже окружено многочисленными 
экранами, которые\linebreak позволяют ему оперативно отслеживать текущую ситуацию. При 
применении рассмотренного устройства брокер получит возможность наблюдать 
фрагмент крупноформатного изображения, состоящего из изображений многих экранов. 
Чтобы перевести взгляд с одного экрана на другой, ему достаточно изменить угол зрения 
или ориентацию зеркала. Отметим, что сложность обычных мониторов пропорциональна 
числу пикселов, т.\,е.\ пропорциональна квадрату линейного размера. Сложность 
рас\-смат\-ри\-ва\-емо\-го устройства пропорциональна линейному размеру. При этом увеличение 
числа пикселов по направлению движения изображений светодиодов не требует 
дополнительного оборудования, при условии что используемая аппаратура имеет запас по 
быстродействию. 

\subsection{Экран мобильного телефона}
  
  Рассмотренный подход позволяет преодолеть основное противоречие современных 
мобильных телефонов между необходимостью иметь малогабаритный аппарат, 
умещающийся в кармане, и\linebreak желанием рассматривать с помощью телефона 
крупноформатные изображения. Если на экране мобильного телефона~\textit{1} 
(рис.~8) реализовать отверстие в виртуальный мир, то через это отверстие можно 
рассматривать крупноформатные изображения~(\textit{22}), размеры которых превосходят длину 
линейки светодиодов~(\textit{2}). Такая линейка размером в несколько десятков сантиметров 
может быть размещена на голове зрителя. В настоящее время многие слушатели 
аудиоплееров размещают на голове наушники. Зрители видеопрограмм или читатели 
книг, газет и журналов могут размещать на голове линейку светодиодов. Так они смогут 
просматривать электронные тексты не на экране размером со спичечную коробку, а на 
площади развернутой кни-

%\columnbreak

%\begin{figure*} %fig8

\begin{center}
\vspace*{12pt}
\mbox{%
\epsfxsize=70.623mm
\epsfbox{tor-8.eps}
}
\end{center}
\vspace*{2pt}
{{\figurename~8}\ \ \small{Схема получения крупноформатного изображения на экране мобильного телефона}}


\bigskip
\addtocounter{figure}{1}  
 %  \label{f8t}}
%   \end{figure*}

\noindent
ги. При этом места в пространстве для размещения этой 
площади не требуется. 

\vspace*{-4pt}

\subsection{Рекламный щит в виде окна в~виртуальный мир}

  В этом приложении размеры колеблющегося зеркала желательно иметь максимально 
возможными, чтобы получить <<окно>> в виртуальный мир максимального размера. 
Однако при этом возникают трудности, связанные с обеспечением необходимого размаха 
колебаний зеркала большого размера. Эти трудности могут быть преодолены несколькими 
способами. 
 
  
  Во-первых, плоское зеркало может быть со\-став\-ле\-но из нескольких плоских зеркал 
меньшего размера, как показано на рис.~9,\,\textit{а}. Зеркала вращаются  вдоль оси, 
перпендикулярной плоскости рисунка. Нетрудно убедиться, что при отклонении зеркал от 
горизонтального положения изображение светодиода будет сначала видно в одном 
зеркале и при его вращении будет перемещаться от одного края к другому. После этого 
изображение будет видно в соседнем зеркале, где оно будет также перемещаться в том же 
направлении. В результате такая система
зеркал обеспечивает перемещение изображения 
светодиода в горизонтальном направлении и, следовательно, точно так же может быть 
использована для формирования изображения. 

  
  Во-вторых, может быть использована система зеркал, составленная из правильных 
треугольных призм, как показано на рис.~9,\,\textit{б}. Призмы вращаются вокруг 
осей, перпендикулярных их основаниям и проходящих через центр основания. Боковые 
грани призм образованы плоскими зеркалами. В~этом случае, так же как и в предыдущем 
варианте, изоб\-ра\-жение светодиода перемещается в горизонтальном направлении и 
переходит последовательно от одной призмы к соседней. За один оборот призмы\linebreak 

%   \begin{figure*} %fig9
%\vspace*{1pt}
\begin{center}
\vspace*{-2pt}
\mbox{%
\epsfxsize=73.764mm
\epsfbox{tor-9.eps}
}
\end{center}
\vspace*{-3pt}
{{\figurename~9}\ \ \small{Варианты замены колеблющегося сплошного зеркала: (\textit{а})~множество одинаковых 
синхронно колеблющихся зеркал; (\textit{б})~множество одинаковых правильных призм с зеркальными 
боковыми поверхностями, синхронно вращающихся в одном направлении}}



%\bigskip
\addtocounter{figure}{1}  
%   \label{f9t}}
%   \end{figure*}

\begin{center}
\vspace*{3pt}
\mbox{%
\epsfxsize=64.42mm
\epsfbox{tor-10.eps}
}
\end{center}
%\vspace*{-9pt}
{{\figurename~10}\ \ \small{Взаимное расположение вращающихся призм, линейки светодиодов и зрителя, при котором для 
него создается <<окно>> в виртуальный мир}}

\bigskip
\addtocounter{figure}{1}  
%   \label{f10t}}
%   \end{figure}

\noindent
изображение светодиода перемещается в горизонтальном направлении 3~раза. Это 
обстоятельство позволяет втрое уменьшить угловую скорость вращения призм по 
сравнению со скоростью вращения плоских зеркал на рис.~9,\,\textit{а}. Заметим, 
что устройства, аналогичные представленным на рис.~9,\,\textit{б}, широко 
используются в рекламе для поочередного показа трех изображений, фрагменты которых 
прикреплены к соответствующим боковым граням призм. При этом формируются 
картины стандартного для рекламных щитов размера $6\times 3$~м. 

  
  На рис.~10 показана схема формирования динамических изображений на 
стандартном рекламном щите с использованием рассматриваемого подхода. Рекламный 
щит состоит из вращающихся призм с зеркальными боковыми стенками, показанными на 
рис.~9,\,\textit{б}. В зеркалах отражается линейка светодиодов~\textit{2}. Эту 
линейку могут наблюдать зрители, один из которых показан на рис.~10. Высота 
получающегося при этом изображения может несколько превосходить высоту зеркальных 
призм. Заметим, что формируемое изображение может быть видно практически из любого 
места полупространства, удаленного от рекламного щита на расстояние, превосходящее 
некоторый предел $R$. При этом чем выше расположены вращающиеся призмы, тем 
больше~$R$. 
  

  
  Например, если расстояние между светодиодами 3~мм, то линейка из 1000 таких 
светодиодов имеет длину 3~м. В этом случае формируется изображение в 3~м высотой с 
разрешением 1000~пикселов по вертикали. Размеры изображения и разрешение по 
горизонтали могут быть в несколько раз больше. 
  
  У зрителя создается впечатление, что он наблюдает изображение в виртуальном мире 
через отверстие, площадь которого совпадает с площадью рекламного щита. Это 
впечатление усиливается, когда он движется относительно щита. В этом случае для него 
постепенно открываются новые фрагменты изображения и пропадают некоторые из уже 
открытых. Аналогичная ситуация имеет место при наблюдении с разных позиций в 
обычном неподвижном зеркале отраженных предметов реального мира. 
  
  Представлено лишь несколько областей применения рассматриваемого подхода с 
целью демонстрации его перспектив. В настоящее время создаются опытные образцы, 
чтобы исследовать его\linebreak особенности. Результаты этих исследований предполагается 
опубликовать в отдельном сообщении. Однако уже в настоящее время на основе 
проведенного теоретического анализа и первых испытаний экспериментальных образцов 
можно утверждать, что рассматриваемый подход обладает\linebreak многими весьма 
привлекательными свойствами, которые могут быть использованы в различных 
применениях. 


     
{\small\frenchspacing
{%\baselineskip=10.8pt
\addcontentsline{toc}{section}{Литература}
\begin{thebibliography}{9}

  \bibitem{1t}
  \Au{Торчигин А.\,В.}
  Способ формирования изображений и устройство для его осуществления. Патент РФ 
RU №\,2328024~C2 от 29.12.2003. 
  
  \bibitem{2t}
  \Au{Торчигин А.\,В.}
  Формирование изображений движущимися источниками света.~--- М.: ИПИ РАН, 
2008. 
  
  \bibitem{3t}
  \Au{Бурняшев~А.}
  Современные мощные светодиоды и их оптика~// Современная электроника, 2006. 
№\,1. С.~24--27.
  
  \bibitem{4t}
  \Au{Торчигин~В.\,П.}
  Способ формирования стереоизображений. Патент РФ RU №\,2337386~C2 от 
02.11.2006.

  \label{end\stat}
  
  \bibitem{5t}
  Патенты США №\,7420737~B2 (2008), №\,7286292~B2 (2007), №\,6903872~B2 (2005), 
№\,5151814 (1992).
  
  
 
%  \bibitem{6t}
%  \Au{Пушков~А.}
%  Домашний кинотеатр на ПК.~--- СПб.: БВХ-Петербург, 2001.
\end{thebibliography}
}
}
\end{multicols}  
 
 
 