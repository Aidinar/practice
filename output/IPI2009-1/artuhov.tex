\newcommand{\+}{^{+}}
%\renewcommand{\-}{^{-}}
\renewcommand{\*}{^{*}}
\newcommand{\T}{^{T}}


\def\stat{art}

\def\tit{ОЦЕНКИ СКОРОСТИ СХОДИМОСТИ РАСПРЕДЕЛЕНИЙ ЭКСТРЕМУМОВ ОБОБЩЕННЫХ ПРОЦЕССОВ КОКСА
С~НЕНУЛЕВЫМ СРЕДНИМ К СДВИГОВЫМ СМЕСЯМ
НОРМАЛЬНЫХ ЗАКОНОВ}
\def\titkol{Оценки скорости сходимости распределений экстремумов обобщенных процессов Кокса}
% с ненулевым средним к сдвиговым смесям нормальных законов}

\def\autkol{С.\,В.~Артюхов}
\def\aut{С.\,В.~Артюхов$^1$}

\titel{\tit}{\aut}{\autkol}{\titkol}

%{\renewcommand{\thefootnote}{\fnsymbol{footnote}}\footnotetext[1]
%{Работа выполнена при поддержке
%Российского фонда фундаментальных исследований,
%гранты 06-07-89056 и 08-07-00152.}}

\renewcommand{\thefootnote}{\arabic{footnote}}
\footnotetext[1]{Московский государственный университет им. М.\,В.~Ломоносова,
факультет ВМиК, ArtyuhovSV@yandex.ru}


\Abst{В статье рассматриваются математические модели
катастрофически накапливающихся эффектов, связанных с
неоднородными хаотическими потоками экстремальных событий, ---
экстремумы обобщенных дважды стохастических пуассоновских
процессов (обобщенных процессов Кокса) с ненулевым средним.
Получены оценки скорости сходимости в предельных теоремах для
экстремумов обобщенных процессов Кокса. Приведен пример
существования нетривиального предела одномерных распределений
экстремумов таких процессов с бесконечной дисперсией при
нормировке, традиционной для сумм с конечной дисперсией.}

\KW{экстремум; обобщенный дважды стохастический
пуассоновский процесс; обобщенный процесс Кокса; сдвиговая смесь
нормальных законов; оценка скорости сходимости}


           \vskip 24pt plus 9pt minus 6pt

      \thispagestyle{headings}

      \begin{multicols}{2}

      \label{st\stat}
   

В данной статье рассматриваются математические модели
катастрофически на\-кап\-ли\-ва\-ющих\-ся эффектов, связанных с
неоднородными хаотическими потоками экстремальных событий.\linebreak В
качестве таких моделей берутся экстремумы обобщенных дважды
стохастических пуассоновских процессов. Как показано, например, в
книге~\cite{1a}, такие модели, во-первых, могут быть вполне адекватны
при оценивании рисков некоторых природных катастроф и, во-вторых,
дают более реалистичные оценки по сравнению с классическими
моделями, недооценивающими риски в случае существенно непостоянной
интенсивности потока экстремальных событий.

Пусть $X_1,X_2,\ldots$~--- одинаково распределенные случайные
величины, а $N(t)$~--- дважды стохастический пуассоновский процесс
(процесс Кокса), управляемый процессом $\Lambda(t)$, т.\,е.\
$$
N(t)=N_1\big(\Lambda(t)\big)\,,\quad t\ge 0\,,
$$
где $\Lambda(t)$~--- случайный процесс с неубывающими непрерывными
справа траекториями, выходящими из нуля, $N_1(t)$~--- стандартный
пуассоновский процесс (однородный пуассоновский процесс с
единичной интенсивностью), причем процессы $N_1(t)$ и $\Lambda(t)$
независимы.

Предположим, что при каждом $t\geq0$ случайные величины
$N(t),X_1,X_2,\ldots$ независимы. Процесс

\noindent
\begin{equation}
S(t)=\sum_{j=1}^{N(t)}X_j\,,\quad t\geq 0\,,
\label{e1a}
\end{equation}
назовем \textit{обобщенным процессом Кокса} (при этом для определенности
будем считать, что $\sum_{j=1}^{0} =0$). Процессы вида~(1) играют
чрезвычайно важную роль во многих прикладных задачах. Достаточно
сказать, что при $\Lambda(t) \equiv \ell t$ с $\ell > 0$ процесс
$S(t)$ превращается в классический обобщенный пуассоновский
процесс, широко используемый при моделировании многих явлений в
физике, теории надежности, финансовой и актуарной деятель\-ности,
биологии и~т.\,д. Большое число разнообразных прикладных задач,
приводящих к обобщенным пуассоновским процессам, описано в книгах~[1--3]. 
Обобщенные процессы Кокса играют важную роль\linebreak при
моделировании характеристик неоднородных хаотических
стохастических потоков случайных событий. При этом особую важность
при использовании обобщенных процессов Кокса вида~(1),\linebreak скажем в
страховой математике при моделировании неоднородных потоков
страховых выплат или в теории управления запасами при
моделировании неоднородных потоков заявок на поставку некоторого
продукта, имеет тот случай, когда математическое ожидание
слагаемых $X_j$ в сумме~(1) отлично от нуля (в частности, в
условиях приведенных примеров~--- положительно).

В данной статье изучается асимптотическое поведение экстремумов
обобщенного процесса Кокса. Более точно, объектом изучения
является процесс
$$
\overline{S}(t) = \max_{0\leq\tau\leq t} S(t)\,,
$$
где процесс $S(t)$ определен в~(1). Всюду далее предполагается, что
$\e X_1 = a \neq 0$, $\D X_1 = \sigma^2 < \infty$.

В качестве модели для катастрофически накапливающихся
неблагоприятных воздействий в неоднородных хаотических потоках
событий рас\-смат\-ри\-ва\-ют\-ся экстремумы обобщенных дважды\linebreak
стохастических пуассоновских процессов. Целесообразность
рассмотрения подобных моделей при прогнозировании рисков катастроф
диктуется следующими примерами.

\smallskip

\noindent
\textbf{Пример~1}. Эволюция финансовых индексов хорошо описывается
обобщенным дважды стохастическим пуассоновским процессом,
накопленная интенсивность потока событий в котором определяется
потоком новостей (см., например,~[4]). При этом, как известно,
если изменение этого индекса в течение биржевого дня оказывается
слишком большим, то во избежание слишком больших потерь (т.\,е.\
финансовых катастроф) торги автоматически прекращаются. Другими
словами, когда экстремум процесса, описывающего динамику
финансового индекса, достигает критического значения, торги
прекращаются.


\smallskip

\noindent
\textbf{Пример~2}. Для снабжения некоторой отрасли некоторого региона
в течение фиксированного периода времени (например, квартала или
зимнего периода) на склад (хранилище) выделяется определенное
количество некоторого ресурса (скажем, топлива). Если {\it
суммарный $($накопленный$)$} расход этого ресурса в течение
указанного времени превысит выделенный лимит, то в данной отрасли
в данном регионе наступит катастрофический коллапс. При этом
естественно предположить, что ресурс отпускается потребителям
партиями, вообще говоря, случайного объема согласно запросам,
возникающим, вообще говоря, в случайные моменты времени. Поскольку
интенсивность потока требований обусловлена не поддающимися
абсолютно точному прогнозированию факторами (например, погодными
условиями или политическими событиями), наиболее разумной моделью
потока требований является обобщенный процесс\linebreak Кокса.
{ %\looseness=1

}

\smallskip

\noindent
\textbf{Пример~3}. Количество воды в некотором резервуаре
(водохранилище, бассейне реки, озере и~т.\,п.) изменяется случайным
образом: оно увеличивается за счет выпадения осадков (в случайные
моменты времени) и уменьшается за счет испарения. Если накопленное
количество воды выходит за критический уровень, то происходят
события катастрофического характера. Избыток воды вызывает
наводнения, ее недостаток~--- засуху. При этом испарение или забор
воды из водохранилища происходит со случайной интенсивностью,
обусловленной, к примеру, погодными условиями.

\smallskip

С формальной точки зрения целью данной \mbox{статьи} является изучение
скорости сходимости в предельной теореме для экстремумов случайных
сумм независимых одинаково распределенных случайных величин с
ненулевым средним, в которых число слагаемых также случайно и
изменяется во времени в соответствии с некоторым процессом Кокса.

В книге~\cite{1a} доказан следующий результат.

\smallskip

\noindent
\textbf{Теорема~1.} {\it Пусть $a\neq0$. Предположим, что ${\sf
E}\Lambda(t) \equiv t$ и $\Lambda(t)\pto\infty$ при $t\to\infty$.
Тогда одномерные распределения неслучайно центрированных и
нормированных обобщенных процессов Кокса слабо сходятся к
распределению некоторой случайной величины $Z$ при $t\to\infty$,
т.\,е.\
$$
\fr{\overline{S}(t)-at}{\sqrt{t(a^2+\sigma^2)}}\Longrightarrow
Z\quad (t\to\infty)\,,
$$
тогда и только тогда, когда существует случайная величина $V$
такая, что
\begin{equation}
{\sf P}(Z<x)={\sf
E}\Phi\left(x-\fr{aV}{\sqrt{\si^2+a^2}}\right)\,,\quad
x\in \mathrm{I}\hspace*{-0.7mm}\mathrm{R}\,,
\label{e2a}
\end{equation}
и}
$$\fr{\Lambda(t)-t}{\sqrt t}\Longrightarrow V\quad 
(t\to\infty)\,.
$$

\smallskip

\noindent 
Здесь и далее $\Phi(x)$~--- стандартная нормальная
функция распределения, $x\in\r$. Соотношение~(2) означает, что
$$
Z\eqd X+\fr{a}{\sqrt{a^2+\sigma^2}}\, V\,,
$$
где $X$~--- случайная величина со стандартным нормальным
распределением, независимая от случайной величины $V$ (символ
$\eqd$ обозначает совпадение распределений).

В книге~\cite{1a} также приведены некоторые оценки скорости сходимости в
теореме~1, но они справедливы для довольно узкого класса
распределений предельной случайной величины $V$, довольно
громоздки, содержат трудно вычисляемые характеристики и неудобны
для анализа и применения. Здесь будет приведена легко
интерпретируемая оценка скорости сходимости в теореме~1 в
традиционных терминах.

Прежде чем сформулировать соответствующие результаты, введем
дополнительные обозначения:
\begin{align*}
\mu^3 &= \e|X_1-a|^3\,;\\
L_3 &= \fr{\mu^3}{\sigma^3}\,;\\
L_3\* & = L_3+\fr{\sigma^2}{a^2}\,;\\
F_t(x) &= \p\left(\fr{\overline{S}(t)-at}{\sigma\sqrt{t}}<
x\right)\,;
\\
%\Delta_t =
%\sup_{v}\left|\p\left(\frac{\Lambda(t)-t}{\sqrt{t}}<v\right)-\p
%\left(V <v\right)\right|,$$
\rho_t &=\sup_x \left|F_t(x) - \e \Phi\left(\fr{\sigma x -
aV}{{\sqrt{a^2 + \sigma^2}}}\right) \right|\,;\\
\Delta_t &=
\sup_{v}\left|\,\p\left(\fr{\Lambda(t)-t}{\sqrt{t}}<v\right)-\p
\left(V <v\right)\right|\,.
\end{align*}
Как видно из введенных обозначений, впредь будет использоваться
несколько иная нормировка, нежели в теореме~1. Это сделано для
удобства вычислений. Более того, в теореме~1 величина
$\overline{S}(t)$ нормируется не ее дисперсией. Как будет показано
ниже, для существования нетривиального предела экстремум
обобщенного процесса Кокса можно нормировать просто величиной
$\sqrt{t}$, т.\,е.\ приводимые результаты справедливы и для
случая, когда дисперсия процесса $\overline{S}(t)$ бесконечна (за
счет бесконечности дисперсии управляющего процесса).

Для получения оценки скорости сходимости в теореме~1 используется
следующее утверждение.

\smallskip

\noindent
\textbf{Лемма 1.} {\it Пусть $\mu^3 < \infty$. Пусть $N_\lambda$~---
случайная величина, имеющая пуассоновское распределение с
параметром $\lambda>0$ и независимая от по\-сле\-до\-ва\-тель\-ности
$\{X_j\}_{j\geq 1}$ независимых одинаково распределенных случайных
величин. Тогда существует конечная положительная абсолютная
константа $C$ такая, что}
\begin{multline*}
\sup_x \left|\p\left(\fr{\overline{S}_{N_\lambda}(t) -a\lambda}{{\sigma\sqrt{\lambda}}} < x\right)
- \Phi\left(\fr{\sigma x }{{\sqrt{a^2 + \sigma^2}}}\right) \right|
\leq{}\\
{}\leq  \fr{C}{{\lambda}}\,L_3^*\,.
\end{multline*}

\smallskip

Д\,о\,к\,а\,з\,а\,т\,е\,л\,ь\,с\,т\,в\,о \ cм.\ в книге~\cite{1a}, с.~103.

\smallskip

Главным результатом данной статьи является следующее утверждение.

\smallskip

\noindent
\textbf{Теорема~2}. {\it Пусть $\mu^3<\infty$, $\e |V| < \infty$.
Тогда справедлива оценка
\begin{multline*}
\rho_t \leq \Delta_t
+\fr{1}{\sqrt{t}}\,\inf_{\epsilon\in(0,1)}\left\{\fr{CL_3\*}{\sqrt{1-\epsilon}}+\fr{{\sf
E}|V|}{\epsilon}+{}\right.\\
\left.{}+Q(\epsilon){\sf
E}\left |\fr{\Lambda(t)-t}{\sqrt{t}}\right |\right\}\,,
\end{multline*}
где}
$$
Q(\epsilon)=\max\left\{\fr{1}{\epsilon},\,\fr{\sqrt{1+\epsilon}}{\sqrt{2\pi
e(1-\epsilon)}\big(1+\sqrt{1-\epsilon}\big)}\right\}\,.
$$

\smallskip

Д\,о\,к\,а\,з\,а\,т\,е\,л\,ь\,с\,т\,в\,о.\ \, Имеем
\begin{multline*}
\!\!\!\!\rho_t=\sup_x \left| \p\left(\fr{\overline{S}(t)
-at}{{\sigma\sqrt{t}}}<x\right)\!-\!\e \Phi\left( \fr{\sigma x -
aV}{{\sqrt{a^2 + \sigma^2}}}\right) \right|={}
\\
{} =\sup_x
\left| \int\limits_0^\infty
\p\left(\fr{\overline{S}_{N_\lambda}(t)
-at}{{\sigma\sqrt{t}}} < x\right)\,d\p \left(\Lambda(t) <
\lambda\right) -{}\right.\\
\left.{}- \e \Phi\left(\fr{\sigma x - aV}{{\sqrt{a^2 +
\sigma^2}}}\right)\right|\,.
\end{multline*}
Полученное представление величины $\rho_t$ позволяет провести
дальнейшее доказательство практически по тому же пути, которым
доказана теорема~2 в работе~\cite{5a} (устанавливающая аналогичные
оценки для скорости сходимости распределений самих обобщенных
процессов Кокса). Доказательства различаются лишь
переобозначениями и тем, что вмес\-то неравенства Берри--Эссеена для
обобщенных пуассоновских распределений (см., например,~\cite{6a}),
использованного в работе~\cite{5a}, здесь используется лемма~1.

\smallskip

Отметим, что оценка, полученная в теореме~2, имеет компактную
запись, легко вычислима и удобна для дальнейшего анализа. Однако,
к сожалению, до настоящего момента конкретное значение абсолютной
константы $C$ не найдено.

Если дополнительно предположить, что семейство случайных величин
$\left\{\left|(\Lambda(t)-t)/\sqrt{t}\right|\right\}_{t>0}
$
равномерно интегрируемо, то с помощью неравенства Ляпунова можно
получить неравенство
\begin{multline}
{\sf E}|V|=\lim_{t\to\infty}{\sf E}\left |\fr{\Lambda(t)-
t}{\sqrt{t}}\right |\le{}\\
{}\le\lim_{n\to\infty}\sqrt{{\sf
D}\left (\fr{\Lambda(t)-t}{\sqrt{t}}\right )}=1\,.
\label{e3a}
\end{multline}
При этом из теоремы~2 очевидным образом вытекает следующий
результат.

\smallskip

\noindent
\textbf{Следствие~1.} {\it Пусть в дополнение к условиям тео\-ре\-мы~$1$
при каждом $t>0$ выполнено $(3)$. Тогда справедлива оценка
$$
\rho_t \leq \Delta_t
+\fr{1}{\sqrt{t}}\,\inf_{\epsilon\in(0,1)}\left \{\fr{CL^*_3}{\sqrt{1-\epsilon}}+
\fr{1}{\epsilon}+Q(\epsilon)\right \}\,,
$$
где}
$$
Q(\epsilon)=\max\left \{\fr{1}{\epsilon},\,\fr{\sqrt{1+\epsilon}}{\left (1+\sqrt{1-\epsilon}\right )\sqrt{2\pi
e(1-\epsilon)}}\right \}\,.
$$

\medskip

\noindent
\textbf{Замечание~1.} Пусть, к примеру, процесс $\Lambda(t)$ является
пуассоновским с единичной интенсивностью. В таком случае
распределение случайной величины~$N(t)$ (см.~(1)) при каждом $t>0$
является так называемым {\it пуассон-пуассоновским} или
({\it инфекционным}) {\it распределением Неймана
типа А}. Класс таких распределений введен Ю.~Нейманом в работе~\cite{9a}
в связи с некоторыми задачами из области бактериологии и
энтомологии. В этом случае величина $\Delta_t$ также имеет порядок
$O\!\left(t^{-1/2}\right)$, поскольку в условиях примера справедлива
оценка
$$
\Delta_t\le\fr{C_0}{\sqrt{t}}\,,
$$
где $C_0$~--- абсолютная постоянная, $C_0\le 0{,}7005$, причем
случайная величина $V$ имеет стандартное нормальное распределение
(см., например,~[6, 8, 9]), а следовательно, распределение
случайной величины $Z$ также является стандартным нормальным. 
В~таком случае имеет место оценка
$$
\rho_t \leq \fr{1}{\sqrt{t}}\,\inf_{\epsilon\in(0,1)}\left\{C_0+
\fr{CL^*_3}{\sqrt{1-\epsilon}}+\fr{1}{\epsilon}+Q(\epsilon)\right\}\,.
$$

\medskip

\noindent
\textbf{Замечание~2.} Другим примером управляющего процесса
$\Lambda(t)$, при котором величина $\Delta_t$ также имеет порядок
$O\big(t^{-1/2}\big)$, служит ситуация, когда~$\Lambda(t)$
является гамма-процессом Леви, приращение которого на единичном
интервале имеет гам\-ма-рас\-пре\-де\-ле\-ние с параметром масштаба
$\lambda=\ell$ и параметром формы $\alpha=\ell^2$. В таком случае\linebreak
распределение случайной величины $N(t)$ (см.~(1)) при каждом $t>0$
является отрицательным биномиальным. При этом случайная величина
$V$ имеет стандартное нормальное распределение (см., например,~\cite{6a}), 
а следовательно, распределение случайной величины $Z$
опять-таки является стандартным нормальным.

\medskip

\noindent
\textbf{Замечание~3.} Если случайный процесс $N(t)$ является
однородным пуассоновским, то логично считать, что распределение
случайной величины $V$ является вырожденным в нуле и $\Delta_t=0$.
Поэтому в таком случае оценка, приведенная в теореме~2,
естественно переходит в оценку скорости схо\-ди\-мости распределений
пуассоновских случайных сумм к нормальному закону, приведенную,
например, в работе~\cite{7a} с константой $C_0$, уточненной в работе~\cite{8a}.
{\looseness=1

}

\medskip

\noindent
\textbf{Замечание~4.} Теорема~2 верна и в некоторых ситуациях, в
которых дисперсия управляющего процесса~$\Lambda(t)$ бесконечна
(очевидно, в таком случае бесконечна и дисперсия процесса
$\overline S(t)$). В качестве примера ситуации, на которую
распространяется действие теоремы~2, можно привести такую, в
которой
$$
\Lambda(t)=\max\{0,\sqrt{t}\cdot V+t\}+\fr{1}{2t^{\alpha/2}}\left(\fr{2\alpha+1}{\alpha}\sqrt{t}-1\right )\,,
$$
где $2<\alpha<3$, а $V$~--- случайная величина с плот\-ностью
$$
p(x)=\fr{\alpha+1}{2(|x|+1)^{\alpha}}\,,\quad x\in\r\,.
$$
Несложно проверить, что ${\sf E}\Lambda(t)=t$ при любом $t>0$, но
второй момент случайной величины $\Lambda(t)$ бесконечен
вследствие бесконечности второго момента случайной величины $V$
(и, следовательно, бесконечен и второй момент экстремума
обобщенного процесса Кокса $S(t)$, управляемого таким процессом
$\Lambda(t)$). Однако при этом, как легко видеть,
{\looseness=1

}

\noindent
\begin{multline*}
\fr{\Lambda(t)-t}{\sqrt{t}}=
\max\left \{-\sqrt{t},\,V\right \}+{}\\
{}+
\fr{1}{2t^{(\alpha+1)/2}}\left (\fr{2\alpha+1}{\alpha}\sqrt{t}-1\right)
\Longrightarrow V
\end{multline*}
при $t\to\infty$. Этот случай является наглядной иллюстрацией
очень интересного и нетривиального факта: в отличие от
классической теории суммирования, для экстремумов сумм со
случайным числом слагаемых (в частности, для обобщенных процессов
Кокса) с бесконечной дисперсией существование нетривиальных слабых
пределов возможно и при нормировке порядка $t^{1/2}$, являющейся
<<стандартной>> в классической теории лишь для сумм с конечной
дисперсией.


{\small\frenchspacing
{%\baselineskip=10.8pt
\addcontentsline{toc}{section}{Литература}
\begin{thebibliography}{9}
\bibitem{1a}
\Au{Королёв В.\,Ю., Соколов~И.\,А.} 
Математические модели
неоднородных потоков экстремальных событий.~--- М.: ТОРУС-ПРЕСС, 2008. %[1]

\bibitem{2a}
\Au{Gnedenko B.\,V., Korolev~V.\,Yu.} 
Random summation: Limit theorems
and applications.~--- Boca Raton: CRC Press, 1996. %[2]

\bibitem{3a}
\Au{Bening V.\,E., Korolev~V.\,Yu.} 
Generalized Poisson models and
their applications in insurance and finance.~--- Utrecht: VSP, 2002. %[3]

\bibitem{4a}
\Au{Королёв В.\,Ю.} 
Вероятностно-статистический анализ
хаотических процессов с помощью смешанных гауссовских моделей.
Декомпозиция волатильности финансовых индексов и турбулентной
плазмы.~--- М.: ИПИ РАН, 2007. %[4]

\bibitem{5a}
\Au{Артюхов С.\,В., Королёв~В.\,Ю.} 
Оценки скорости сходимости
распределений обобщенных дважды стохастических пуассоновских
процессов с ненулевым средним к сдвиговым смесям нормальных
законов~// Обозрение промышленной и прикладной
математики, 2008. Т.~15. Вып.~6. С.~988--998. %[5]

\bibitem{6a}
\Au{Бенинг~В.\,Е., Королёв~В.\,Ю., Шоргин~С.\,Я.} 
Математические основы теории риска.~--- М.: Физматлит, 2007. %[6]

\bibitem{9a} %7
\Au{Neyman~J.} 
On a new class of ``contagious'' distributions, applicable in
entomology and bacteriology~ // Ann. Math. Statist., 1939. Vol.~10. P.~35--57. %[9]

\bibitem{7a} %8
\Au{Шевцова~И.\,Г.} 
Уточнение структуры оценок ско\-рости сходимости в
центральной предельной теореме для сумм независимых случайных
величин. Дис.\ \ldots %на соискание ученой степени 
канд.\ физ.-мат. наук.~--- М.: МГУ, 2006. %[7]

 \label{end\stat}

\bibitem{8a} %9
\Au{Шевцова~И.\,Г.} 
Об абсолютной постоянной в неравенстве
Берри--Эссеена~// Сб.\ статей молодых ученых факультета ВМиК
МГУ. Вып.~5.~--- М.: Изд-во факультета ВМиК МГУ, 2008. С.~101--110. %[8]



\end{thebibliography}
}
}
\end{multicols}  
 