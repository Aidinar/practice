\def\stat{abstr}
{%\hrule\par
%\vskip 7pt % 7pt
\raggedleft\Large \bf%\baselineskip=3.2ex
A\,B\,S\,T\,R\,A\,C\,T\,S \vskip 17pt
    \hrule
    \par
\vskip 21pt plus 6pt minus 3pt }


%1
\def\tit{METHODS FOR INFORMATION MODEL BUILDING FOR THE EARTH TIDAL
HEREDITARY IRREGULAR ROTATION}

\def\aut{I.\,N. Sinitsyn}
\def\auf{IPI RAN, sinitsin@dol.ru
}

\def\leftkol{\ } %ENGLISH ABSTRACTS}

\def\rightkol{\ } %ENGLISH ABSTRACTS}


\titele{\tit}{\aut}{\auf}{\leftkol}{\rightkol}

\noindent
Methods for information hereditary model building for the Earth
tidal rotation fluctuations based on \textit{a priori} and
\textit{a posteriori} data
are considered. Linear and quasi-linear methods are developed.
Equivalence of different hereditary disturbances is discussed.
Experimental software is the part of the informational resources
``Statistical dynamics of the Earth rotation.''

\KWN{\textit{a priori}  and \textit{a posteriori} data;
informational model; informational
 resources; quasi-linear methods; spectral-correlational
 characteristics; hereditary fluctuations of the Earth rotation;
 hereditary kernel}

\vskip 14pt plus 6pt minus 3pt

%2
\def\tit{PARALLEL COMPUTING IN LARGE-SCALE MULTIMODAL BIOMETRIC SYSTEMS}


\def\aut{O.\,S. Ushmaev}
\def\auf{IPI RAN, oushmaev@ipiran.ru}


%\def\leftkol{\ } %ENGLISH ABSTRACTS}

%\def\rightkol{\ } %ENGLISH ABSTRACTS}


\titele{\tit}{\aut}{\auf}{\leftkol}{\rightkol}

\noindent
The main topic is parallel computing in large-scale biometric identification systems.
Approach of a biometric cluster throughput estimation, a cluster configuration estimation is proposed.
Methods of organization of parallel computing for multimodal biometrics are developed.

\KWN{biometric identification; multimodal biometrics; parallel computing
}

\vskip 14pt plus 6pt minus 3pt

%2
\def\tit{DEVELOPMENT OF SUPERRESOLUTION-BASED FACE VIDEO ENHANCEMENT}


\def\aut{A.\,V.~Nasonov$^1$, A.\,S.~Krylov$^2$, and O.\,S.~Ushmaev$^3$}
\def\auf{$^1$M.\,V.~Lomonosov Moscow State University,
Faculty of Computational Mathematics and Cybernetics,\\
$\hphantom{^1}$nasonov@cs.msu.ru\\[1pt]
$^2$M.\,V.~Lomonosov Moscow State University,
Faculty of Computational Mathematics and Cybernetics,\\
$\hphantom{^1}$kryl@cs.msu.ru\\[1pt]
$^3$IPI RAN, oushmaev@ipiran.ru
}


%\def\leftkol{\ } %ENGLISH ABSTRACTS}

%\def\rightkol{\ } %ENGLISH ABSTRACTS}


\titele{\tit}{\aut}{\auf}{\leftkol}{\rightkol}

\noindent
General superresolution-based method of face image enhancement
for video data has been suggested. The superresolution is modeled
as inverse problem to image downsampling, i.e.,
it finds an image that gives the minimal value of the quadratic
discrepancy with initial low-resolution images after the motion
dependent downsampling. High-quality superresolution method and
fast superresolution methods are considered. Special deringing
method for fast superresolution is proposed. New multiscale
motion estimation method has been developed.

\KWN{superresolution; deringing; facial video sequence; multiscale motion
estimation; fast superresolution}

\pagebreak

\vskip 14pt plus 6pt minus 3pt


%3
\def\tit{RECONSTRUCTION OF PROBABILISTIC CHARACTERISTICS
OF RANDOM FUNCTIONS IN SPECT PROBLEMS}

\def\aut{V.\,G.~Ushakov$^1$ and O.\,V.~Shestakov$^2$}
\def\auf{$^1$Department of Mathematical Statistics, Faculty of Computational
Mathematics and Cybernetics,\\
$\hphantom{^1}$M.\,V.~Lomonosov Moscow State University; IPI RAN, vgushakov@mail.ru\\[1pt]
$^2$Department of Mathematical Statistics, Faculty of Computational
Mathematics and Cybernetics,\\
$\hphantom{^1}$M.\,V.~Lomonosov Moscow State University, oshestakov@cs.msu.su}

%\def\leftkol{\ } %ENGLISH ABSTRACTS}

%\def\rightkol{\ } %ENGLISH ABSTRACTS}

\titele{\tit}{\aut}{\auf}{\leftkol}{\rightkol}

\noindent
The problem of reconstruction of probabilistic characteristics of an object
which structure changes in a random manner during the process of projection data aquisition is considered.
Within the frames of proposed tomography experiment model, a method to reconstruct
distributions of a random function from distributions of projections in the case when random
function has at most denumerable number of states is developed.


\KWN{single-photon emission computer tomography (SPECT);
stochastic tomography; exponential Radon transform; random functions; projection data}
%\pagebreak


%\vfil
% \vskip 18pt plus 6pt minus 3pt
% \vskip 24pt plus 9pt minus 6pt

%\pagebreak


\vskip 14pt plus 6pt minus 3pt


%
\def\tit{A DESIGN CONCEPT OF DOMESTIC INTEGRATED COMMUNICATION MICROCONTROLLERS FOR PACKET SWITCHING}

\def\aut{V.\,B. Egorov}
\def\auf{$^1$IPI RAN, vegorov@ipiran.ru}


%\def\leftkol{\ } %ENGLISH ABSTRACTS}

%\def\rightkol{\ } %ENGLISH ABSTRACTS}

\titele{\tit}{\aut}{\auf}{\leftkol}{\rightkol}

\noindent
A concept of simplified integrated communication microcontrollers, which
could be applied in various domestic packet switching and routing devices,
with leveraging their functionality and facilitating development
is suggested.

\KWN{integrated communication microcontroller; PowerQUICC;
decentralized switching; routing switch}
%\pagebreak

%\vful

 \vskip 14pt plus 6pt minus 3pt

%5
\def\tit{ON NONSTATIONARY QUEUEING SYSTEMS WITH CATASTROPHES}

\def\aut{A.\,I.~Zeifman$^1$, Ya.\,A.~Satin$^2$, and A.\,V.~Chegodaev$^3$}

\def\auf{$^1$Vologda State Pedagogical University;
IPI RAN; VSCC CEMI RAN, a\_zeifman@mail.ru\\[1pt]
$^2$Vologda State Pedagogical University,  yacovi@mail.ru\\[1pt]
$^3$Vologda State Pedagogical University, cheg\_al@mail.ru}

\def\leftkol{ENGLISH ABSTRACTS}
%
%\def\rightkol{ENGLISH ABSTRACTS}

%\def\leftkol{\ } %ENGLISH ABSTRACTS}

%\def\rightkol{\ } %ENGLISH ABSTRACTS}

\titele{\tit}{\aut}{\auf}{\leftkol}{\rightkol}

\noindent
Nonstationary birth and death processes with
catastrophes are considered. The bounds of the rate of convergence to the limit
regime and the estimates of the limit probabilities are obtained.
Also, the bounds for the mean of the process are studied and a
queueing example is considered.

\KWN{nonstationary queues; Markovian models with
catastrophes; weak ergodicity; bounds; limiting characteristics;
approximations}

%\vskip 18pt plus 6pt minus 3pt

\pagebreak

 \vskip 14pt plus 6pt minus 3pt


\def\tit{BAYESIAN QUEUING AND RELIABILITY MODELS: AN EXPONENTIAL-ERLANG CASE}

%6
\def\aut{A.\,A.~Kudriavtsev$^1$ and S.\,Ya.~Shorgin$^2$}

\def\auf{$^1$Faculty of Computational Mathematics and Cybernetics,\\
$\hphantom{^1}$M.\,V.~Lomonosov Moscow State University, nubigena@hotmail.com\\[1pt]
$^2$IPI RAN, sshorgin@ipiran.ru}

%\def\leftkol{\ } % ENGLISH ABSTRACTS}

%\def\rightkol{\ } %ENGLISH ABSTRACTS}

\titele{\tit}{\aut}{\auf}{\leftkol}{\rightkol}


\noindent
The investigation of Bayesian queuing and reliability models is continued in
the paper. The method provides the randomization of system characteristics
with regard to \textit{a priori} distributions of input parameters. This
approach could be used to calculate average values of performance and
reliability characteristics for the large groups of systems or devices.
The new results are presented for a case of exponential and
Erlang \textit{a priori} distributions.


%\label{st\stat}

 \KWN{Bayesian approach; queuing systems; reliability; mixed distributions;
modeling; Erlang distribution; exponential distribution}

 \vskip 14pt plus 6pt minus 3pt


\def\tit{ON ONE APPROACH TO IMAGE PRODUCTION WITHOUT SCREENS}

%7
\def\aut{A.\,V.~Torchigin}

\def\auf{IPI RAN, torchigin\_a@mail.ru}

\def\leftkol{ENGLISH ABSTRACTS}

\def\rightkol{ENGLISH ABSTRACTS}

\titele{\tit}{\aut}{\auf}{\leftkol}{\rightkol}

\noindent
Properties of images observed in an oscillating mirror, where LEDs
modulated by brightness are reflected, are considered. Possible areas of application of this approach are analyzed.

%\label{st\stat}

\KWN{image production; virtual environment; virtual reality; stereo images}

%\pagebreak

% \thispagestyle{headings}

\vskip 14pt plus 6pt minus 3pt

\def\tit{CONVERGENCE RATE ESTIMATES OF DISTRIBUTIONS OF EXTREMA OF COMPOUND
COX PROCESSES WITH NONZERO MEANS TO LOCATION MIXTURES OF NORMAL LAWS}

%8
\def\aut{S.\,V.~Artyukhov} % and ??????V.\,Yu.~Korolev}


\def\auf{Faculty of Computational Mathematics and Cybernetics,\\
M.\,V.~Lomonosov Moscow State University, ArtyuhovSV@yandex.ru}

%\def\leftkol{\ } % ENGLISH ABSTRACTS}

%\def\rightkol{\ } %ENGLISH ABSTRACTS}

\titele{\tit}{\aut}{\auf}{\leftkol}{\rightkol}


\noindent
Mathematical models of catastrophically accumulating effects related to
nonhomogeneous chaotic flows of extremal events are considered, namely,
extrema of compound doubly stochastic Poisson processes (compound Cox
processes) with nonzero expectation. Convergence rate estimates are
obtained in limit theorems for extrema of compound Cox processes. An example
is given of existence of nontrivial limit of one-dimensional distributions
of extrema of such processes with infinite variance under normalization
which is traditional for sums with finite variance.


\label{st\stat}

 \KWN{extremum; compound doubly stochastic Poisson process; compound
Cox process; location mixture of normal laws; convergence rate estimates}

%\vfil


% \pagebreak

% \label{end\stat}