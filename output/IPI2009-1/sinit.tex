

%\def\ss{\textstyle}
%\def\kk{\kappa}
\def\tr{\,,\,\ldots\,,\,}
\def\rv{\right\vert\,}
\def\rrv{\right\vert}
\def\lv{\,\left\vert}
\def\rk{\,\right]}
\def\lk{\left[\,}
%\def\rf{\right\}}
%\def\lf{\left\{}
\def\prl{\,\parallel}
\def\prr{\parallel\,}
\def\paar{\parallel}  
%\def\sbs{\subset}
%\def\sps{\supset}
\def\eps{\varepsilon}
\def\si{\sigma}
\def\la{\lambda}
\def\alp{\alpha}
\def\w{\omega}  
\def\W{\Omega}
%\def\sssd{\mathop{\sum\limits^2\sum\limits^2}}
%\def\sssn{\mathop{\sum\limits^n\sum\limits^n}}
%\def\liminf{\mathop{\cup\,inf}} 
%\def\limsup{\mathop{\cup\,sup}}
\def\iint{\int\limits_{-\infty}^{\infty}}
%\def\iii{\int\limits}
%\def\sss{\sum\limits}
%\def\prt{\partial}
\def\mm{{\rm M}}


\def\stat{sinit}

\def\tit{МЕТОДЫ ПОСТРОЕНИЯ ИНФОРМАЦИОННЫХ МОДЕЛЕЙ ЭРЕДИТАРНЫХ
ФЛУКТУАЦИЙ НЕРАВНОМЕРНОСТИ ВРАЩЕНИЯ ЗЕМЛИ$^*$}
\def\titkol{Методы построения информационных моделей эредитарных
флуктуаций неравномерности вращения Земли}

\def\autkol{И.\,Н.~Синицын}
\def\aut{И.\,Н.~Синицын$^1$}

\titel{\tit}{\aut}{\autkol}{\titkol}

{\renewcommand{\thefootnote}{\fnsymbol{footnote}}\footnotetext[1]
{Работа выполнена при финансовой поддержке РФФИ
(проект №\,07-07-00031) и программы ОНИТ РАН <<Информационные
технологии и анализ сложных систем>> (проект 1.5).}}

\renewcommand{\thefootnote}{\arabic{footnote}}
\footnotetext[1]{Институт проблем
информатики Российской академии наук, sinitsin@dol.ru}


\Abst{Рассматриваются методы построения информационных
моделей флуктуаций   неравномерности вращения Земли вследствие
эредитарных гравитационно-приливных диссипативных возмущений.
Изучены вопросы эквивалентности различных эредитарных возмущений.
Созданное методическое и экспериментальное программное обеспечение
включено в состав информационных ресурсов по проблеме
 <<статистическая динамика вращения Земли>>. }

\KW{априорные и апостериорные данные;
информационная модель; информационные ресурсы; квазилинейные методы;
спектрально-корреляционные характеристики; эредитарные флуктуации
неравномерности вращения Земли; эредитарное ядро}

           \vskip 24pt plus 9pt minus 6pt

      \thispagestyle{headings}

      \begin{multicols}{2}

      \label{st\stat}
   

\section{Введение}

Как известно~[1--7], одной из важных современных задач
статистической динамики вращения Земли является построение
информационных\linebreak моделей флуктуаций неравномерности враще-\linebreak ния Земли.
Как показывают результаты астрометрических измерений, такие
возмущения на внутригодовых интервалах времени оказываются
\mbox{существенными}. Разработанные в~[7] модели\linebreak учитывают только
нелинейные стохастические механизмы гравитационно-при\-лив\-ных
диссипативных возмущений. Дадим обобщение результатов~[7] на случай
эредитарных стохастических гравитационно-приливных диссипативных
возмущений.

\section{Эредитарные стохастические интегродифференциальные уравнения 
флуктуаций неравномерности вращения Земли}

Обобщая математические модели~[1--7]  флуктуаций угловой скорости
собственного вращения Земли на случай воздействия как аддитивных,
так и параметрических гармонических и широкополосных случайных
гравитационно-приливных и эредитарных флуктуационно-диссипативных
возмущений на внутригодовых интервалах времени, представим
дифференциальное уравнение изменения угла собственного вращения
Земли $\delta \varphi$ в сле\-ду\-ющем виде:
\begin{multline}
\delta \ddot\varphi (t) = M_{0} (t) +X^S(t) + X^L(t) -{}\\
{}-\mu_1 \lk 1+ \tilde \mu_1 (t) + X_1^{\mu S} (t) + X^{\mu L} (t)\rk \delta \dot \varphi(t)-{}\\
{}-
    \int\limits_{t-T_0}^t \mu_1^H (t,\tau) \delta\dot\varphi (\tau) d\tau+
 \mu_n \lk 1+ \tilde \mu_n +{}\right.\\
\left. {}+ X^{\mu S}_n (t) + X_n^{\mu L} (t)\rk F_n (\delta\varphi(t),
    \delta\dot\varphi(t))\,,
\label{e1s}
\end{multline}
где
    \begin{multline}
\tilde M_{0t} = M_{10}^S \cos\left( 2\pi f_\Gamma t +\chi_1^S\right)+{}\\[3pt]
{}+M_{20}^S
    \cos\left( 4\pi f_\Gamma t +\chi_2^S\right)+{}\\[3pt]
{}+M_{m0}^L \cos\left(2\pi \nu_m t +\chi_m^L\right)+{}\\[3pt]
{}+ M_{f0}^L
    \cos\left( 2\pi \nu_f t+\chi_f^L\right )\,;\label{e2s}
    \end{multline}
    \vspace*{-24pt}
    
    \noindent
    \begin{multline}
\tilde\mu_{lt} = \mu_l\left[ \pi_{l1}^{\mu S} \cos\left( 2\pi f_\Gamma t +\chi_{l1}^{\mu S}\right )+{}\right.\\[3pt]
{}+
    \pi_{l2}^{\mu S} \cos\left ( 4\pi f_\Gamma t +\chi_{l2}^{\mu S}\right )+{}\\[3pt]
{}+\pi_{lm}^{\mu L} \cos\left( 2\pi \nu_m t +\chi_{lm}^{\mu L}\right )+{}\\[3pt]
\left. {}+   \pi_{lf}^{\mu L} \cos\left( 2\pi \nu_{lf} t +\chi_{lf}^{\mu L}\right)\right]\enskip (l=1,n)\,.
\label{e3s}
\end{multline}
Здесь введены следующие обозначения: $\delta\varphi\; =$\linebreak $=\;\varphi - r_*
\w_*^{-1} t$~--- изменение угла собственного вращения Земли; 
$\varphi$~--- угол собственного вращения Земли, значения которого
определены на дату $t$ (он является параметром, характеризующим
вращение земной системы координат по отношению к не\-бе\-сной); $r_* =
7{,}292115 \cdot 10^{-5}$~рад/с~--- постоянная средняя составляющая
угловой скорости собственного вращения Земли; $\w_*$~---  угловая
скорость обращения Земли по орбите; $t$~--- безразмерное время,
из\-ме\-ря\-емое стандартными годами; $f_\Gamma = 1$, $ 2 f_\Gamma$,
$\nu_m= 13{,}28$, $\nu_f =26{,}28$, $M_{10}^S$, $M_{20}^S$, $M_{m0}^L$,
$M_{f0}^L$ и $\chi_1^S$, $\chi_2^S$, $\chi_m^L$, $\chi_f^L$~---
частоты, амплитуды и начальные фазы аддитивных гармонических
возмущений от Солнца $(S)$ и Луны $(L)$, соответствующие годовому,
полугодовому, месячному и двухнедельному циклам; $\mu_h$ $(h=1,n)$~--- 
коэффициенты диссипативных моментов сил, обуслов\-лен\-ных
разнообразием геофизических процессов (приливное трение океанических
и земных приливов, атмосферные воздействия, океанические течения,
перераспределение водных масс и~т.\,п.); $ \pi_{h1}^{\mu S}$,
$\pi_{h2}^{\mu S}$, $\pi_{hm}^{\mu L}$, $ \pi_{hf}^{\mu L}$ и
$\chi_{h1}^{\mu S}$, $\chi_{h2}^{\mu S}$, $\chi_{hm}^{\mu L}$,
$\chi_{hf}^{\mu_L}$ $(h=1,\,n)$~--- амплитуды и начальные фазы
параметрических гармонических диссипативных моментов сил на частотах
$f_\Gamma$, $2f_\Gamma$, $\nu_m$, $\nu_f$; $F_n=F_n (\delta\varphi,
\delta\dot\varphi)$~--- нелинейная со\-став\-ля\-ющая диссипации (в случае
рэлеевского механизма диссипации $F_n (\delta \dot\varphi)
=\delta\dot\varphi^3$); $X^S (t)$, $ X^L (t)$ и $X_h^{\mu S} (t)$,
$X_h^{\mu L} (t)$~--- нормальные (гауссовские) широкополосные
аддитивные и параметрические\linebreak случайные возмущения с известными
математическими ожиданиями и ковариационными характеристиками;
$\mu_1^H =\mu_1^H(t,\tau)$~--- нестационарные эредитарные ядра,
определяющие эредитарность\linebreak (<<память>>) гравитационно-приливных
диссипативных сил.

Примерами стационарных эредитарных ядер служат функции вида $\mu_1^H
(t,\tau) = \mu_1^H (\rho)$, $\rho = t-\tau$, где
\begin{align}
\mu_1^H(\rho) &= \mu_{10}^H e^{-\alp \lv\rho\rv} {\bf 1}(\rho)\,;\\[3pt]
\mu_1^H(\rho) &= \mu_{10}^H e^{-\alp \lv\rho\rv} (1+\alp\lv \rho\rv){\bf 1}(\rho)\,;\\[3pt]
\mu_1^H(\rho) &= \mu_{10}^H e^{-\alp \lv\rho\rv} \left( 1+\alp\lv \rho\rv +\fr{1}{3}\, \alp^2 \rho^2\right ){\bf 1}(\rho)\,;\\[3pt]
\mu_1^H(\rho) &= \mu_{10}^H e^{-\alp \lv\rho\rv} \cos \w \rho\cdot {\bf 1}(\rho)\,;\\[3pt]
\mu_1^H(\rho) &= \mu_{10}^H e^{-\alp \lv\rho\rv} (\cos \w \rho+\gamma \sin \w\lv \rho\rv){\bf 1}(\rho)\,,
\end{align}
допускающие производные по  $\rho$ различных порядков. Ядра~(4)--(6)
определяют чисто затухающую эредитарность, а ядра (7) и (8)~---
затухающую колебательную эредитарность на частоте $\w$.

\section{Априорные методы построения~информационных корреляционных эредитарных моделей}

Основываясь на уравнениях~(1)--(4), составим приближенные уравнения
для математических ожиданий, дисперсий, ковариаций и ковариационных
функций основных переменных при сле\-ду\-ющих основных допущениях.
\begin{enumerate}[1{$^\circ$}]
\item Аддитивный возмущающий момент допускает представление
$X^S (t) + X^L (t) = m_0^{SL}+V_1$, где $V_1 $ является скалярным
нормальным белым шумом интенсивности $\nu_1 = \nu_1(t)$, происходящим
от первого источника~[7].
\item
Возмущающие моменты  $X_3 = X^{\mu S}_1 (t) + X_1^{\mu L} (t)$
и $X_4 = X_n^{\mu S} (t) + X_n^{\mu L}(t)$ происходят от второго
источника нормального белого шума  $V_2$ единичной интенсивности
$(\nu_2 =1)$. Они имеют конечные математические ожидания  $m_3$ и
$m_4$ и дисперсии  $\si_3^2, \si_4^2$ и удовлетворяют следующим
скалярным уравнениям формирующего фильт\-ра~[8]:
\begin{align*}
\dot X_l^0 &=-\alp_l X_l^0 + \si_l \sqrt{2\alp_l}\cdot V_2\,;\\
 X_l^0 &= X_l- m_l^*\,,\quad l=3,4\,,\quad \alp_l>0\,,
\end{align*}
где  $m_l^*$~--- постоянные значения  $m_l$.
     \item
Нелинейные функции  $X_2 X_3$, $F_n = F_n (X_1, X_2)$, $ F_n'=
F_n' ( X_1, X_2, X_4)= F_n (X_1, X_2) X_4$, входящие в~(1),
допускают статистическую линеаризацию нелинейностей по формулам~[8]:
    \begin{align*}
X_2 X_3 &\approx m_2 m_3 + k_{23}+ m_2 X_3^0 + m_3 X_2^0\, ;\\
F_n &= F_n (X_1,X_2)\approx F_{n0}+F_{n1} X_1^0+ F_{n2} X_2^0\,;\\
F_2' &= F_1 (X_1, X_2) X_4 \approx F_{n0}' +F_{n1}' X_1^0+{}\\
&\ \ \ \ \ \ \ \ \ \ \ \ \ \ \ \ \ \ \ \ \ \ \ \ \ \ \ \ \ \ \ \ {}+ F_{n2}' X_2^0 + F_{n4}' X_4^0\,.
\end{align*}
Здесь $m_i$ и $k_{ij}$~--- математические ожидания и ковариационные
моменты  переменных $X_i$ и $ X_i^0=X_i - m_i$;
\begin{align*}
F_{n0} &= F_{n0} (m_1, m_2, k_{11}, k_{12}, k_{22}) ={}\\
&\ \ \ \ \ \ \ \ \ \ \ \ \ \ \ \ \ \ \ \ \ {}= \mm_N^{(1,2)} \lk F_n (X_1, X_2)\rk\,;\\
F_{n1} &=\fr{\partial F_{n0}}{\partial m_1}\,;\quad F_{n2} =\fr{\partial F_{n0}}{\partial m_2}\,;\\
F_{n0}' &={}\\
&\!\!\!\!\!{}= F_{n0}' (m_1, m_2, m_4, k_{11}, k_{12}, k_{22}, k_{14}, k_{24}, k_{44})={}\\
&\ \ \ \ \ \ \ \ \ \ \ \ \ \ \ \ \ \ \ \ \ {}= \mm_N^{(1,2,4)} \lk F_n' (X_1, X_2, X_4)\rk\,;\\
F_{n1}' &=\fr{\partial F_{n0}'}{\partial m_1}\,;\quad F_{n2}' =\fr{\partial F_{n0}'}{\partial m_2}\,;
\quad F_{n4}' =\fr{\partial F_{n0}'}{\partial m_4}\,,
\end{align*}
где  $\mm_N^{(1,2)} \lk\, \cdot\, \rk$ и $\mm_N^{(1,2,4)} \lk\,
\cdot\, \rk$~--- символы вероятностного осреднения для двумерного и
трехмерного нормального распределения переменных $X_1, X_2$  и
$X_1,X_2, X_4$.
\item
Эредитарные ядра  $\mu_1^H (t,\tau)$ имеют вид~(4), причем при
$t<t_0+T_0$ влиянием значений $X_2(\tau)$ при  $\tau \in (t-T_0,
t_0)$ можно пренебречь, положив  $t- T_0 \approx t_0$. Тогда, 
следуя~[8--10] и приняв
    \begin{equation*}
\int\limits_{t_0}^t \mu_{10}^H e^{-\alp_5 \lv t-\tau\rv} X_2 (\tau) d\tau = X_5\,,
\end{equation*}
будем иметь следующие дифференциальные уравнения, связывающие
вспомогательную переменную  $X_5$ с исходными переменными:
\begin{equation*}
\dot X_5 = -\alp_5 X_5 -\mu_{10}^H X_2\,.
\end{equation*}
\end{enumerate}


В результате стохастическое нелинейное дифференциальное уравнение
второго порядка~(1) будет эквивалентно нелинейной системе для
математических ожиданий $m_i = \mm X_i$ $(i=\overline{1,5})$:
\begin{align}
\dot m &= A^m,\enskip m= \lk m_1 m_2 m_3 m_4 m_5\rk^T\,;\\[3pt]
 A^m &=\lk A_1^m A_2^m A_3^m
    A_4^m A_5^m\rk^T\,,
\end{align}
где
\begin{align*}
A_1^m &= m_2\,;\quad A_2^m = M_0^{\mathrm{э}}\,;\\
M_0^{\mathrm{э}}& =\tilde M_{0t} + m_0^{SL} - ( \mu_1 +\tilde \mu_{1t}) m_2 -{}\\[3pt]
&{}- \mu_1 (m_2m_3 + k_{23}) +(\mu_n +\tilde \mu_{nt}) F_{n0} +\mu_n F_{n0}' -{}\\[3pt]
&\ \ \ \ \ \ \ \ \ \ \ \ \ \ \ \ \ \ \ \ \ \ \ \ \ \ \ \ \ \ \ \ \ \ \ \ \ \ \ \ \ \ \ \ \ \ \  {}- m_5 +m_5\,;\\[3pt]
A_3^m &=-\alp_3 (m_3- m_3^*)\,,\quad
A_4^m = -\alp_4 (m_4 - m_4^*)\,;\\[3pt]
A_5^m &=-\alp_5 m_5 +\mu_{10}^H m_2\,,
\end{align*}
и линейной дифференциальной системе уравнений для центрированных
составляющих:
    \begin{align*}
\dot X_1^0 &= X_2^0\,;\\[3pt]
\dot X_2^0 &= V_1 + \lk (\mu_n +\tilde\mu_{nt}) F_{n1} + \mu_n F_{n1}'\rk
    X_1^0-{}\\[3pt]
  &{}- \lk (\mu_1 +\tilde\mu_{1t}) + \mu_1m_3 -(\mu_n +\tilde\mu_{nt})F_{n2} -{}\right.\\[3pt]
&\left.  {}- \mu_n F_{n2}'\rk
    X_2^0- \mu_1 m_2 X_3^0 + \mu_n F_{n4}' X_4^0-X_5^0\,;\\[3pt]
    \dot X_l^0 &=-\alp_l X_l^0 +\si_l \sqrt{2\alp_l} V_2\quad
    (l=3, 4)\,.
\end{align*}
Обозначая через  $\eps$ и $\beta$ матрицы размерности $(5\times 5)$ и
$(2\times 5)$
    \begin{equation*}
\eps = \eps(t)= \lk\eps_{ij}\rk\,;\quad
    \beta =\lk\beta_{ij}\rk\,,
\end{equation*}
где
    \begin{align*}
 \eps_{21} &= (\mu_n +\tilde\mu_{nt}) F_{n1} +\mu_n F_{n1}'\,;\\
 \eps_{22} &= -(\mu_1 +\tilde\mu_{1t})-\mu_1 m_3+  (\mu_n +\tilde\mu_{nt}) F_{n2} +{}\\
&\ \ \ \ \ \ \ \ \ \ \ \ \ \ \ \ \ \ \ \ \ \ \ \ \ \ \ \ \ \ \ \ \ \ \ \ \ \ \ \ \ \ \ \ \ \ \ \ \ \ \ \ \ \ \ \ \ 
 {}+\mu_n F_{n2}'\,;\\
 \eps_{23} &=-\mu_1m_2\,;\quad \eps_{24} = \mu_n F_{n1}'\,;\\
\eps_{25}&= -1\,; \quad \eps_{33} = -\alp_3\,;\\
  \eps_{44} &= -\alp_4\,;\quad \eps_{52}= -\mu_{10}^H\,;\quad \eps_{55}= -\alp_5\,;\\
\beta_{21} &=1\,;\quad \beta_{32} =\sigma_3\sqrt{2\alp_3}\,;\quad 
\beta_{42} =\sigma_4\sqrt{2\alp_4}\,,
\end{align*}
представим уравнения  для ковариационной матрицы  $K(t)=\lk
k_{ij}(t)\rk$, $k_{ij}(t) =\mm \lk X_i^0 (t) X_j^0 (t)\rk$ и
мат\-ри\-цы для ковариационных функций $K(t_1, t_2)\; =$\linebreak $=\; \lk K_{ij} (t_1,
t_2)\rk$, $ K_{ij} (t_1, t_2) =\mm \lk X_i^0 (t_1) X_j^0 (t_2)\rk$
$(i,j=\overline{1,5})$ в следующем виде~[8]:
\begin{align}  
\dot K(t) &= \eps (t) K(t) + K(t) \eps^T(t) +\beta \nu \beta^T\,;\\
k_{ij} (0) &= k_{ij0}\,;\\
\fr{\partial K(t_1, t_2)}{\partial t_2} &= K(t_1, t_2) \eps^T(t_2)\,;\\
K_{ij} (t, t') &= k_{ij} (t)\,,
\end{align}
где $\nu =\lk \nu_{ij}\rk$ $( \nu_{11} = \nu_1 (t)$, $\nu_{12}
=\nu_{12} (t)$, $ \nu_{22}=1)$~--- матрица интенсивностей белых шумов
$V_1$ и $V_2$.

Совокупность уравнений~(9)--(14) определяет аналитическую
квазилинейную гауссовскую корреляционную модель флуктуаций приливной
внут\-ри\-го\-до\-вой не\-рав\-но\-мер\-ности вращения Земли при экспоненциальной
затухающей эредитарности.

Аналогично выписываются уравнения для эредитарных ядер~(5)--(8), а
также модулированных частотами $f_{\mathrm{г}},\nu_m,\nu_t$
эредитарных ядер $\mu_1^H (t,\tau)$.

Усредняя по периодам основных гармоник, отвечающих частотам
$f_\Gamma$, $2f_\Gamma$, $\nu_m$ и $\nu_f$ для стационарных регулярных и
нерегулярных колебаний и\linebreak постоянных $m_h = m_h^*$ $(h= 2,3,4)$, из
уравнений~(9)--(14) находим соответствующие урав\-не\-ния для
математических ожиданий, ковариационной мат\-ри\-цы, мат\-ри\-цы
ковариационных функ\-ций и мат\-ри\-цы спектральных плотностей.

Уравнения~(9)--(14)  применимы для нелинейных функций $F_n  =
F_n (X_1, X_2)$, допускающих угловые точки и даже разрывы. В случае
гладких функций  $F_n$ рассмотренный квазилинейный метод\linebreak переходит в
метод непосредственной линеаризации в окрестностях математических
ожиданий  $m_1$ и $m_2$.

Для негауссовских возмущений в~(1), как показано в~[8], можно
воспользоваться методом эквивалентной линеаризации, взяв в качестве
осредняющего вероятностного распределения отрезок\linebreak
 параметризованного
(моментами, квазимоментами, семиинвариантами и~др.) разложения одно-
и двумерных плотностей.

\section{Апостериорные информационные модели эредитарных флуктуаций}

Примем за информационные переменные $Z_1$ и~$Z_2$, допускающие
измерение  переменных  $X_1=\delta \varphi$ и $X_2=\delta
\dot\varphi$. Положим
\begin{equation}
Z_1 = X_1+V_3,\enskip Z_2 = X_2 + V_4\,,
\end{equation}
где  $V_3$ и $V_4$~--- независимые нормальные белые шумы с
интенсивностями $\nu_5$ и $\nu_6$, соответственно. Тогда
совокупность уравнений~(1)--(4) и~(15) будет представлять собой
исходную систему уравнений для синтеза фильтра для обработки
информации о флуктуациях неравномерности вращения Земли по
апостеорным данным, т.\,е.\ результатам измерения~$Z_1$ и~$Z_2$.

Для построения квазилинейного нормального фильтра, согласно~[11, 12],
перепишем уравнения~(1)--(4) и~(15) в следующем стандартном виде:
\begin{equation}
\dot X = a (X,t) + b(t) \bar V_1\,;\quad Z= a_1 (X, t) +\bar V_2\,.
\end{equation}
Здесь
\begin{align}
X&=\left [ X_1 \ldots X_6\right ]^T\,; \quad Z= \left [ Z_1 Z_2\right ]^T\,;\notag\\[3pt]
\bar V_1& = \left [ V_1 V_2\right ]^T\,;\quad \bar V_2 =\left [ V_3 V_4\right ]^T\,;\notag\\[3pt]
a &= \left [ a_1\ldots a_5\right ]^T\,\;\notag\\[3pt]
a_1 &=\left [ a_{11}\ \ a_{12}\right ]^T\,;\quad a_{11}= X_1\,;\quad 
a_{12} = X_2\,;\notag\\[3pt]
a_2 &= \tilde M_{0t} -\left [ (\mu_1+\tilde\mu_{1t}) +
\mu_1 X_3\right ] X_2  +{}\notag\\[3pt]
&{}+\left [(\mu_n +\tilde \mu_{nt}) +\mu_n X_4\right ] F_n(X_1,
        X_2)- X_5\,;\notag\\[3pt]
a_3 &= -\alp_3 (X_3 - m_3^*)\,;\quad a_4 =-\alp_4(X_4 - m_4^*)\,;\\
a_5 &=-\alp_5 X_5 +\mu_{10}^H X_2\,;   \\[6pt]
b&=b(t) =
\begin{bmatrix}
    0&0\\
    1&0\\
    0&\si_3\sqrt{2\alp_3}\\
    0&\si_4 \sqrt{2\alp_4}
\end{bmatrix}\,;\notag\\[6pt]
\bar\nu_1 &=
\begin{bmatrix}
    \nu_1&0\\
    0&\nu_2
\end{bmatrix}\,;\notag\\[6pt]
\bar\nu_2&=
\begin{bmatrix}
    \nu_3&0\\
    0&\nu_4
\end{bmatrix}\,.\notag
\end{align}

Заменим~(16) статистически линеаризованной системой уравнений,
нелинейной относительно математических ожиданий  $m^x =\lk m_1^x
\ldots m_6^x\rk^T$, $m^z=\lk m_1^z m_2^z\rk$ и линейной относительно
центрированных составляющих  $X^0 = X-m^x$:
    
\noindent
\begin{align}
\dot m^x &= a_{00}\,; & m^z &= a_{10}\,;\\
\dot X^0 &= a_{01} X^0 +\psi (t) \bar V_1\,; & Z^0 &= a_{11} X^0 + V_2\,.
\end{align}
Здесь  $a_{ij} = a_{ij} (m^x, K^x,t)$ $(i,j=0,1)$~--- коэффициенты
статистической линеаризации функций $a=a(X,t)$ и $a_1= a_1 (X,t)$.
При этом ковариационная матрица $K^x$ определяется уравнением вида~(11):
\begin{equation}
\dot K^x = a_{01} K^x + K^x a_{01}^T + b\bar\nu_1 b^T\,.
\end{equation}
Применяя к~(19) и~(20) уравнения фильтра Кал\-ма\-на--Бьюси~[11],
получим
\begin{align}
{\dot{\hat X}} &= a_{00} - a_{01} m^x + a_{01} \hat X +{}\notag\\[3pt]
&  {}+R a_{11} \bar\nu_2^{-1} (Z-a_{11} \hat X - a_{10} + a_{11} m^x)\,;\\[3pt]
\hat X_0 &=\mm X(t_0)\,;\\[3pt]
    \dot R &= a_{01} R + R a_{01}^T - R a_{11}^T \bar \nu_2^{-1} a_{11} R +{}\notag\\[3pt]
    &\ \ \ \ \ \ \ \ \ \ \ \ \ \ \ \ \ \ \ \ \ \ \ \ \ \ \ \ \ \ \ \ \ \ \ \ \ {}+ b\bar \nu_1 b^T\,;\\[3pt]
    R_0 &=\mm \lk (X_0 -\hat X_0)(X_0 -\hat X_0)^T\rk\,.
\end{align}
Совокупность фильтрационных уравнений~(22)--(25) при условиях~(19) и~(21) 
определяет искомый квазилинейный нормальный фильтр для
обработки информации о флуктуациях неравномерности вращения Земли по
апостериорным данным, в том числе в темпе получения результатов
наблюдений. Коэффициенты статистической линеаризации $a_{00}$ и
$a_{01}$ приведены в разд.~3, а  $a_{10}=0$ и $ a_{11} = I_2$ в
силу линейности второго уравнения~(30).

Фильтрационные уравнения~(22)--(25) для произвольного линейного
наблюдения, когда  $a_1(X,t) = b_1 (t) X + b_0$, $ a_{11} = b_1
(t)$, если через  $\beta$ обозначить коэффициент усиления фильтра
$\beta = R b_1 (t)^T \bar\nu_2^{-1}$, принимают следующий вид:
   \begin{align*}
{\dot{\hat X}} &= a_{00} - a_{01} m^x + a_{01} \hat X + \beta \bigl[ Z- b_1(t) \hat X -{}\\[3pt]
&\ \ \ \ \ \ \ \ \ \ \ \ \ \ \  \ \ {}-b_0(t) + b_1(t) m^x\bigr] \,,\quad \hat X_0 =\mm X_0\,;\\[3pt]
\dot R &= a_{01} R + R a_{01}^T - \beta b_1(t) R + b(t) \bar\nu_1 (t) b(t)^T\,;\\[3pt]
R_0 &=\mm \lk (X_0 -\hat X_0)(X_0 -\hat X_0)^T\rk\,.
\end{align*}

Важно отметить, что коэффициенты статистической линеаризации
$a_{00}, a_{01}$ и вспомогательная инструментальная матрица
ошибки фильтрации $R$ не содержат результатов наблюдений и могут
быть определены отдельно (до получения результатов наблюдений).
Таким образом, возможна априорная оценка точности квазилинейного
фильтра.

\section{Об эквивалентности моделей эредитарных флуктуаций неравномерности вращения Земли}

Будем считать известными совместные канонические разложения (КР)
возмущений
\begin{align*}
X_{0t}^{SL} &= \tilde M_0(t) + X^s(t) + X^L(t)\,;\\
X_{1t}^{SL} &= \tilde \mu_1(t)+ X_1^{\mu S}(t) + X_1^{\mu L} (t)\,.
\end{align*}
Запишем их в виде
   \begin{equation}
X_{ht}^{SL} = m_h^{SL} (t) +\sum\limits_\nu V_\nu x_{\nu h}^{SL} (t) \quad
(h=\overline{0,\,n})\,.
\end{equation}
Здесь $m_h^{SL} (t)$~--- математические ожидания воз\-мущений; $V_\nu$~--- 
случайные независимые (не\-коррелированные) коэффициенты с
известными распределениями (дисперсиями $D_\nu$); $x_{\nu
n}^{SL}(t)$~--- детерминированные координатные функции. Тогда
уравнения~(1)--(3) совместно  с~(4)--(8) для различных эредитарных
ядер с помощью методов~ [9, 10] приводятся к уравнениям вида
\begin{align} 
\dot X_1 &= X_2\,;\\
\dot X_2 &=-\mu_1 X_2 +{\cal F}(X_1, X_2, X_{ht}^{SL}, X_{lt}^{I})\,,
\end{align}
где входные возмущения  $X_{ht}^{SL}(t)$ определяются~(26);
$X_{lh}^{SL}(t)$ $(l=1,2,\ldots)$~---  инструментальные
вспомогательные переменные, удовлетворяющие дифференциальным
уравнениям, задаваемым типом эредитарного ядра; ${\cal F}(\bullet)$~--- 
в общем случае нелинейная функция отмеченных переменных.

Эффективное значение линейного коэффициента диссипации
$\mu_1^{\mbox{э}}$ (по математическому ожиданию)
определяется по формуле
\begin{equation*}
\mu_1^{\mbox{э}} =\mu_1 +\fr{\partial}{\partial m_2^x} {\rm M} {\cal F}\,.
\end{equation*}

Эффективное значение линейного коэффициента диссипации (по случайной
составляющей) находится из рассмотрений уравнений~(27) и~(28) для
центрированных переменных.

При использовании КР~(26) аналогично вы\-чис\-ля\-ют\-ся эффективные
коэффициенты диссипации по случайным величинам  $V_\nu$ или
координатным функциям входных возмущений  $x_{\nu h}^{SL}(t)$.

Принцип эквивалентной замены исходной эредитарной стохастической
системы на систему с эквивалентной линейной диссипацией позволяет
вычислять $\mu_1^{\mbox{э}}$ не только для различных типов
эредитарных ядер, но и для эредитарных нелинейностей вида
    $\int\limits_{t-T_0}^t \mu_n^H (t,\tau) F_n (\delta\varphi(\tau), \delta\dot\varphi(\tau))\,d \tau$.
В некоторых практических задачах применяется принцип
эквивалентности, связанный с критериями совпадения параметров
систематического и флуктуационного дрейфа по переменной
$\delta\varphi$~[7].

\section{Тестовые примеры}

\noindent
\textbf{Пример~1.} Сохраняя в уравнениях~(9)--(12)
 постоянное возмущение  $m_0^{SL}$, вязкое и эредитарное вязкое
 трение ($\mu_1\ne 0$, $ \mu_{10}^H \ne 0$, $\alp_5 \ne 0$), получим
 уравнения:
\begin{align}
\dot m_1 &=m_2\,;\\
\dot m_2 &= m_0^{SL} - \mu_1 m_2 - m_5\,;\\
\dot m_5 & =     -\alp_5 m_5 +\mu_{10}^H m_2\,;\\
\dot K_{11} &= 2 K_{12}\,;\quad \dot K_{12} = K_{22} -\mu_1
    K_{12} - K_{15}\,;\\
    \dot K_{15} &= K_{25} - \alp_5 K_{15}
    +\mu_{10}^H K_{12}\,;\\
\dot K_{22} &=\nu_1 - 2( \mu_1 K_{22} + K_{25})\,;\notag\\ 
\dot     K_{25} &= -(\mu_1+\alp_5) K_{25} - K_{55} +\mu_{10}^H K_{22}\,;\notag\\
\dot K_{55} &=- 2 (\alp_5 K_{55} -\mu_{10}^H K_{25})\,.
\end{align}
Отсюда, во-первых, следует, что для переменных~$X_{2}$ и~$X_5$ имеют
место стационарные решения, отве\-ча\-ющие постоянным значениям
математических ожиданий
\begin{align}
m_2^* &= m_0^{SL} \mu_{11}^{\mathrm{э}}\,;\quad 
m_5^* = m_2^* \mu_{10}^H \alp_5^{-1}\,;\notag\\
\mu_{11}^{\mathrm{э}} &=\mu_1 +\mu_{10}^H \alp_5^{-1}
\end{align}
и постоянным значениям дисперсий и ковариаций
\begin{align}    
K_{22}^* &=\fr{\nu_1}{2 \mu_{12}^{\mathrm{э}}}\,;\quad 
K_{25}^* =\fr{\mu_{10}^H}{(\mu_1 +\alp_5) +\mu_{10}^H \alp_5^{-1}}\,;\notag\\
K_{55}^* &= \mu_{10}^H \alp_5^{-1} K_{25}^*\,;\notag\\
\mu_{12}^{\mathrm{э}} &=\fr{\mu_1 (\mu_1 +\alp_5) +\mu_{10}^H (1+ \mu_1 \alp_5^{-1})}
{   (\mu_1 +\alp_5) +\mu_{10}^H \alp_5^{-1}}\,.
\end{align}

Во-вторых, имеет место систематический и флуктуационный дрейф по
$X_1 = \delta\varphi$, опре\-де\-ля\-емый уравнениями~(29), (32) и~(33). Наличие
эредитарного вязкого трения приводит к увеличению эффективного
вязкого трения как по математическому ожиданию
$\mu_{11}^{\mathrm{э}}$, так и по случайной со\-став\-ля\-ющей~$\mu_{12}^{\mathrm{э}}$. 
Момент  $m_0^{SL}$ определяет
величину смещенных колебаний и не влияет на корреляционные
характеристики случайных колебаний. При чисто эредитарном вязком
трении его эффективные коэффициенты  будут определяться
формулами
\begin{equation}
\mu_{11}^{\mathrm{э}} = \fr{\mu_{10}^H}{ \alp_5}\,,\quad 
\mu_{12}^{\mathrm{э}} = \fr{\mu_{10}^H}{ \alp_5 +\mu_{10}^H
    \alp_5^{-1}}\,.
\end{equation}

\bigskip
\noindent
\textbf{Пример~2.}  Если в условиях примера~1
дополнительно учесть аддитивный полигармонический возмущающий момент
$\tilde M_{0t}$, определяемый формулой~(2), то уравнения~(29)--(31)
примут следующий вид:
  \begin{align*}
\dot m_1 &= m_2\,;\\
\dot m_2 &=- \mu_1 m_2 - m_5 +\tilde M_{0t}\,;\\
\dot m_5 &= -\alp_5 m_5 +\mu_{10}^H m_2\,
\end{align*}
а уравнения~(32)--(34) останутся без изменения.
Амплитудно-частотные характеристики вы\-нуж\-ден\-ных полигармонических
колебаний находятся по известным формулам стационарных\linebreak линейных
дифференциальных систем~[8]. По переменной  $X_1 =\delta \varphi$
возможно накопление систематической ошибки, зависящее от
амплитудно-частотных характеристик. При этом могут быть использованы
формулы~(37) и результаты~[7].

\bigskip
\noindent
\textbf{Пример~3.} Разработанные в~[7] тестовые
примеры~3--10 были использованы для моделирования переменных с учетом~(35) и~(36).

\section{Заключение}

Разработанные  методы построения информационных моделей
эредитарных  флуктуаций неравномерности вращения Земли по
априорным и апостериорным данным реализованы в виде\linebreak
экспериментального программного обеспечения в среде  MATLAB.
Проведено тестирование программного обеспечения на десяти примерах.
Разработаны критерии эквивалентности эредитарных и неэредитарных
моделей.

Методы, алгоритмы, программное обеспечение и тестовые примеры
включены в состав информационных ресурсов по фундаментальной
проблеме <<Статистическая динамика вращения Земли>>.

Квазилинейные методы, как показали вычислительные эксперименты и
сравнение с результатами статистического моделирования, обеспечивают
высокую точность фильтрации скорости $\delta \dot\varphi$ (примерно
2\%--3\% для априорной стационарной информации и 0,5\%--1\% при
апостериорной информации). Из-за отсутствия возвращающей силы по
$\delta \varphi$ появляются дрейфы и накапливающиеся ошибки. Поэтому
так необходимы точные измерения $\delta \varphi$.

Направления дальнейших исследований:
\begin{enumerate}[(1)]
\item учет влияния автокоррелированности эредитарных ядер различных
возмущений в уравнениях~(1);
\item
оценка негауссовости распределений возмущения для оценок больших
уклонений по $\delta \varphi$;
\item
оценивание и распознавание возмущений в уравнениях~(1) на основе
апостериорной информации от нескольких нелинейных измерительных
систем различной точности;
\item
разработка комплексных статистических моделей вращения Земли,
учитывающих эредитарные флуктуации полюса и неравномерности вращения
Земли.
\end{enumerate}

\bigskip
Автор благодарен Н.\,Н.~Семендяеву за выполненные вычислительные
эксперименты.


{\small\frenchspacing
{%\baselineskip=10.8pt
\addcontentsline{toc}{section}{Литература}
\begin{thebibliography}{99}
\bibitem{1sin}
\Au{Марков Ю.\,Г., Синицын И.\,Н.} 
Спектрально-кор\-ре\-ля\-ци\-он\-ные модели флуктуаций вращательного движения Земли~// 
ДАН,  2003.  Т.~393.  №\,5. С.~618--623.

\bibitem{2sin}
\Au{Марков Ю.\,Г., Синицын И.\,Н.} 
Спектрально-кор\-ре\-ля\-ци\-он\-ные и кинетические модели движения Земли~// 
Астрон. журнал,  2004.  Т.~81,  №\,2. С.~184--192.

\bibitem{3sin}
\Au{Синицын И.\,Н.}  
Стохастические модели флуктуаций движения
Земли в условиях пуассоновских возмущений~// Системы и средства
информатики. Спец. вып. <<Геоинформационные технологии>>.~--- М.: ИПИ
РАН, 2004.  С.~39--55.

\bibitem{4sin}
\Au{Марков Ю.\,Г., Дасаев~Р.\,Р., Перепёлкин~В.\,В., Синицын~И.\,Н., 
Синицын~В.\,И.}
Стохастические модели  вращения Земли с учетом влияния Луны и планет~// 
Космические исследования, 2005. Т.~43.  №\,1. С.~54--66.

\bibitem{5sin}
\Au{Акуленко~Л.\,Д., Марков~Ю.\,Г., Перепёлкин~В.\,В.} 
Внутригодовые неравномерности вращения Земли~// ДАН,
2007.  Т.~417. №\,2. С.~118--121.

\bibitem{6sin}
\Au{Марков Ю.\,Г., Синицын~И.\,Н.} 
Корреляционная модель приливной неравномерности вращения Земли~//
ДАН, 2008. Т.~49. №\,3. С.~338--341.

\bibitem{7sin}
\Au{Синицын И.\,Н.} 
Квазилинейные методы построения
информационных моделей флуктуаций неравномерности вращения Земли~//
Информатика и её применения, 2008. Т.~2. Вып.~1. С.~35--43.

\bibitem{8sin}
\Au{Пугачёв В.\,С., Синицын И.\,Н.} Теория стохастических систем. 2-е
изд.~--- М.:  Логос, 2004.

\bibitem{9sin}
\Au{Sinitsyn I.\,N.} 
Stochastic hereditary control systems~//  Пробл. упр. и теория информации, 
1986. Vol.~15. No.\,4. P.~287--298.

\bibitem{10sin}
\Au{Пугачёв В.\,С., Синицын И.\,Н.} 
Стохастические дифференциальные системы.  Анализ и фильтрация. 2-е изд.~---
М.: Наука, 1990.

\bibitem{11sin}
\Au{Синицын И.\,Н.} 
Фильтры Калмана и Пугачёва. 2-е. изд.~--- М.: Логос, 2007.

  \label{end\stat}

\bibitem{12sin}
\Au{Синицын И.\,Н.} 
Развитие теории фильтров Пугачёва для оперативной обработки 
информации в стохастических системах~//
Информатика и её применения, 2007. Т.~1. Вып.~1. С.~3--13.
\end{thebibliography}
}
}
\end{multicols}  