\def\stat{agalarov}


\def\tit{АЛГОРИТМ ВЫЧИСЛЕНИЯ ЗАГРУЖЕННОСТИ 
ТЕЛЕКОММУНИКАЦИОННОЙ СЕТИ С~ПОВТОРНЫМИ ПЕРЕДАЧАМИ$^*$}
\def\titkol{Алгоритм вычисления загруженности 
телекоммуникационной сети с~повторными передачами} 

\def\autkol{Я.\,М.~Агаларов}
\def\aut{Я.\,М.~Агаларов$^1$}

\titel{\tit}{\aut}{\autkol}{\titkol}

{\renewcommand{\thefootnote}{\fnsymbol{footnote}}\footnotetext[1]
{Работа выполнена при частичной поддержке РФФИ, проекты 08-07-00152-а и
      09-07-12032-офи-м.}}

\renewcommand{\thefootnote}{\arabic{footnote}}
\footnotetext[1]{Институт проблем
информатики Российской академии наук, agglar@yandex.ru}


\end{document}
      \vskip 18pt plus 9pt minus 6pt

      \thispagestyle{headings}

      \begin{multicols}{2}

      \label{st\stat}

    
      \Abst{Рассмотрены модели сети коммутации пакетов c повторными попытками передачи 
пакетов для двух схем распределения буферной памяти: полнодоступной и полного разделения. 
Предложен итерационный метод расчета интенсивностей потоков в сети и вероятностей блокировок 
узлов, где в качестве модели узла используется СМО типа $
      \begin{matrix}
      M \\ \vec{\lambda}
      \end{matrix}
      \left |
      \begin{matrix}
      M \\ \vec{\lambda}
      \end{matrix}
      \right |
      \vec{m} \vert N$. Получено необходимое условие существования решения системы уравнений 
сохранения баланса потоков в установившемся режиме работы сети и доказана монотонная 
сходимость последовательности значений интенсивностей потоков и вероятностей блокировок, 
получаемых предлагаемым методом, к решению указанной системы.}
      
      \KW{слова: сеть коммутации пакетов; буферная память; повторные передачи; вероятность 
блокировки; итерационный метод}

     
\section{Введение}

     Одной из важных проблем, решаемых на этапе проектирования 
телекоммуникационных сетей, является задача предварительного анализа сети 
на предмет возникновения локальных и глобальных перегрузок.
     
     Причинами перегрузок наряду с другими могут быть:
     \begin{itemize}
     \item ограничение объема буферной памяти коммутационного 
оборудования;
     \item блокировка конечных терминалов;
     \item недостаточная производительность вычислительных ресурсов и 
пропускная способность каналов связи.
     \end{itemize}
     
     Ограничение буферной памяти в реальных сетях вызвано не только 
желанием разработчиков снизить себестоимость коммутаторов, но и 
требованиями к параметрам качества обслуживания (среднее время задержки и 
его разброс, вероятность потери пакетов и~др.). В то же время ограничение 
объема буферной памяти узлов является одной из причин роста числа 
повторных передач в сетях с коммутацией пакетов и, как следствие, резкого 
роста нагрузки на отдельных участках сети или сети в целом. Поэтому одной из 
задач предварительного анализа сетей является оценка загруженности узлов и 
каналов связи с учетом ограниченного объема буферной памяти.
     
     Используемые точные методы анализа сетей с коммутацией пакетов (см., 
например,~[1, 2]) разработаны в рамках экспоненциальных СеМО (сетей 
массового обслуживания) со стохастическими маршрутными матрицами и 
неограниченной буферной памятью. Однако предположение о неограниченной 
буферной памяти исключает возможность учета блокировок узлов из-за 
нехватки буферной памяти, которые и являются одной из основных причин 
возникновения повторных передач пакетов в сети.
     
     Большое число работ в последнее время посвящается системам массового 
обслуживания с повторными заявками, одним каналом (обслуживающим 
устройством), ограниченным накопителем и более общими предположениями 
относительно входящих потоков заявок и длительностей обслуживания~[2--7], 
чем при исследовании СеМО. Однако использование этих моделей при расчете 
сетей вызывает очень большие вычислительные трудности.
     
     Из множества приближенных методов расчета сетей с ограниченной 
буферной памятью в узлах следует выделить методы, используемые в теории 
второго порядка для СеМО~[2, 8], и методы, рассматривающие узлы как 
изолированные СМО с пуассоновскими входящими потоками~[9, 10]. Первые 
из них предполагают: 
     \begin{enumerate}[1)]
     \item внешний поток заявок~--- рекуррентный с известными первым и 
вторым моментами интервалов между поступлениями; 
     \item  узел~--- СМО с одним прибором и накопителем;
     \item  время обслуживания в узле~--- независимое с произвольным 
распределением с известными первым и вторым моментами;
     \item движение заявок по сети происходит согласно неразложимой 
стохастической матрице с возможностью случайного ухода из сети в каждом 
узле. Сущность этих методов состоит в том, что они при расчетах используют 
первые и вторые моменты соответствующих распределений интервалов 
поступления и обслуживания заявок. Второй подход отличается тем, что 
потоки, образованные в узлах суперпозицией внешнего потока, повторениями 
по сети не доставленных пакетов и потоками от других узлов,~--- 
пуассоновские потоки, а времена обслуживания~--- экспоненциальные. Общее 
у этих методов то, что они являются итерационными, причем каждая итерация 
выполняется в два этапа: на первом этапе вычисляются характеристики потоков 
в узлах, на втором~--- уточняются другие характеристики сети (вероятности 
блокировок узлов, моменты времени задержки пакетов в узле и~др.). В этих 
методах в качестве моделей линий связи использованы одноканальные СМО. 
     \end{enumerate}
     
     Ниже будут рассмотрены модели сетей коммутации пакетов с повторами 
из источника и из предыдущего узла. Предлагается итерационный метод 
расчета сетей, который реализует второй из упомянутых выше подходов и в 
качестве модели узла использует СМО типа $
      \begin{matrix}
      M \\ \vec{\lambda}
      \end{matrix}
      \left |
      \begin{matrix}
      M \\ \vec{\lambda}
      \end{matrix}
      \right |
      \vec{m} \vert N$. Получено необходимое условие существования решения 
системы уравнений сохранения баланса потоков в установившемся режиме 
работы сети и доказана монотонная сходимость последовательности значений 
интенсивностей потоков и вероятностей блокировок, получаемых 
предлагаемым методом, к решению системы.
     
\section{Общее описание моделей сети}
     
     Рассматривается модель сети с коммутацией пакетов в виде графа, 
состоящего из $U$~вершин и $V$~дуг. Вершины графа отождествляются с 
узлами связи, дуги~--- с линиями связи. Имеется множество источников и 
получателей пакетов, каждый из которых соединен с одним из узлов связи, 
который называется узлом-входом, если соединен с источником, и узлом-
выходом, если соединен с получателем. Передача пакета в сети происходит по 
заданному пути~$l$, соединяющему узел-вход с узлом-выходом. Будем считать 
(без потери общности), что каждый узел сети входит хотя бы в один путь сети и 
множество путей не разбивается на непересекающиеся подмножества. Известна 
интенсивность потока (первичного потока) пакетов, поступающих извне на 
каждый путь~$l$. Узлы сети имеют ограниченную буферную память с заданной 
схемой распределения, линии связи имеют заданное число однородных 
каналов. Поступивший в промежуточный узел пакет принимается в буферную 
память (занимает одно место буферной памяти), если согласно заданной схеме 
распределения ему можно выделить место в буферной памяти (узел не 
блокирован для данного пакета) и он передан без ошибок, иначе он передается 
повторно согласно процедуре повторов (из источника или из предыдущего 
узла), пока не будет успешно передан адресату. Под блокировкой узла (линии) 
понимается такое состояние узла (линии), когда согласно принятой схеме 
распределения памяти поступивший пакет не может быть принять в буфер 
данного узла (линии). Под успешной передачей (попыткой передачи) пакета 
понимается передача (попытка), когда переданный пакет принимается 
последующим узлом в буферную память. При неуспешной попытке передачи 
по линии пакета занятый им буфер освобождается сразу в случае сети с 
повторными попытками из источника и сохраняется за ним в случае повторов 
из предыдущего узла. После успешной передачи пакета занятое им место в 
буферной памяти через заданное время освобождается. Предполагается, что 
пакеты в сети не теряются.
     
     Введем обозначения:
     
     \noindent
     $v$ (или $v_i$, $i = 1$, 2,\ldots)~--- линия связи;
     
     
     \noindent
     $v^+$~--- узел-сток линии~$v$;
     
     \noindent
     $u$ (или $u_i$, $i = 1$, 2,\ldots)~--- узел связи;
     
     \noindent
     $\Omega_u^+$~--- множество исходящих из узла $u$ линий;
     
     \noindent
     $c_v$~--- канальная емкость линии~$v$;
     
     \noindent
     $N_v$~--- емкость буферной памяти, выделенной для линии~$v$;
     
     \noindent
     $N_u$~--- емкость общей буферной памяти узла~$u$;
     
     \noindent
     $L$~--- заданное множество нециклических путей;
     
     \noindent
     $L_v$~--- множество путей, содержащих линию~$v$, ($L_v\subseteq L$);
     
     \noindent
     $l=\{v_1,\ldots ,v_{S_l}\}$~--- путь, содержащий линии $v_1,\ldots 
,v_{S_l}$, где $S_l$~--- число линий в пути~$l$, индексы 1,\ldots , $S_l$ 
показывают порядок следования линий в пути, $v_l$~--- линия, исходящая из 
     узла-входа, $v_{S_l}$~--- линия, входящая в узел-выход;
     
     \noindent
     $u_{S_l+1}$~--- абонентский узел, соединенный с узлом-выходом 
пути~$l$;
     
     \noindent
     $U_u^+$~--- множество различных узлов, следующих после узла~$u$ по 
направлению к адресату в путях, проходящих через узел~$u$;
     
     \noindent
     $V_v^+$~--- множество различных линий, следующих после линии~$v$ 
по направлению к адресату в путях, проходящих по линии~$v$;
     
     \noindent
     $\lambda(l)$~--- интенсивность потока ($l$-потока) пакетов, поступающих 
из источника на узел-вход и требующих передачи на узел-выход, $\lambda(l) 
>0$, $l\in L$;
     
     \noindent
     $\mu_v$~--- интенсивность обслуживания пакета каналом линии~$v$;
     
     \noindent
     $\delta_v$~--- вероятность безошибочной передачи пакета по линии~$v$;
     
     \noindent
     $\Lambda_v$~--- интенсивность потока пакетов, успешно передаваемых 
по линии~$v$;
     
     \noindent
     $\Lambda_v^*$~--- интенсивность суммарного потока пакетов, 
требующих передачи по линии~$v$;
     
     \noindent
     $\Lambda_v^*(l)$~--- интенсивность $l$-потока, поступающего на 
линию~$v$;
     
     \noindent
     $\pi_u$~--- вероятность блокировки узла~$u$;
     $\pi_v$~--- вероятность блокировки узла для пакетов, требующих 
передачи по исходящей из узла линии~$v$.
     
     Во всех рассматриваемых ниже моделях узла коммутации 
предполагается, что потоки $\Lambda_v^*$~--- пуассоновские, а времена 
обслуживания пакетов каналами связи~--- экспоненциальные с параметрами 
$\mu_v$, $v\in V$. Предполагается также, что внешние нагрузки~--- 
реализуемые, т.\,е.\ в стационарном режиме работы сети интенсивности 
первичных входных потоков равны интенсивностям соответствующих 
выходных (покидающих сеть) потоков. Всюду ниже сеть рассматривается в 
стационарном режиме.
     
\section{Сеть с повторами из источника}
     
     В качестве модели коммутационного узла используется СМО с 
ограниченным накопителем (буферной памятью) и несколькими линиями из 
однотипных каналов, в которой сделаны также следующие предположения:
     \begin{enumerate}[1.]
\item Места в буферной памяти распределяются согласно одной из двух 
схем:
\begin{itemize}
\item полнодоступная схема (CS)~--- каждое свободное место хранения 
доступно любой заявке (пакету);
\item схема полного разделения памяти (CP)~--- заявке, требующей передачи 
по линии~$v$ ($v$-заявке), доступны всего~$N_v$ мест, где 
$\sum\limits_{v\in\Omega_u^+} N_v = N$.
\end{itemize}
\item Суммарные потоки первичных и повторных $v$-заявок являются 
независимыми в совокупности пуассоновскими потоками. Для 
обслуживания $v$-заявки требуется одновременно одно место хранения и 
один канал типа~$v$, $v\in\Omega_u^+$.
\item Поступившей в СМО заявке предоставляется место в накопителе, если 
она передана без ошибок и в момент ее поступления в накопителе есть 
доступное свободное место, иначе заявка получает отказ.
\item Принятые в СМО $v$-заявки обслуживаются линией~$v$ в порядке 
поступления.
\item Время занятия канала $v$-заявкой~--- экспоненциально 
распределенная случайная величина с параметром $0<\mu_v<\infty$, $v\in 
\Omega_u^+$, независимая от других случайных событий в системе.
\item Обслуженная заявка освобождает сразу место в накопителе СМО.
\item Заявка, получившая отказ, повторяется через заданное время из 
источника.
\end{enumerate}

     Пусть во всех узлах сети распределение буферной памяти происходит по 
схеме CS. При полнодоступной схеме и повторах из источника в 
установившемся режиме работы сети справедливы следующие соотношения 
для потоков в узлах:
     \begin{align}
     \Lambda_v(l) =\Lambda_v^*(l)\left (1-\pi_v\right )\,,\quad
     \Lambda^*_{v_i}(l) =\Lambda_{v_{i-1}}(l)\,,\quad
     \Lambda_{v_{S_l}} =\lambda(l)\,,\quad l\in L\,,\notag\\
     \Lambda_v=\sum\limits_{l\in L}\Lambda_v(l)\,,\quad 
\Lambda_v^*=\sum\limits_{l\in L}\Lambda_v^*(l)\,,\quad v\in V\,.
     \end{align}
     
     Из~(1) для вычисления параметра $\Lambda_{v_i}(l), $i=1,\ldots ,S_l$, 
$l\inL$, получаем формулу
     \begin{equation}
     \Lambda_{v_i}(l) =\fr{\Lambda_{v_{i+1}}(l)}{(1-
\pi_{u_{i+1}})\delta_{v_{i+1}}} =\fr{\lambda(l)}{(1-
\pi_{u_{S_l+1}})\prod\limits_{}^{S_l} (1-\pi_{u_j})\delta_{v_j}}\,,\quad i=1,\ldots 
,S_l-1\,.
     \end{equation}
     
     Здесь и далее по тексту статьи считается, что при заданных~$\lambda(l)$, 
$l\inL$, величины $\pi_{u_{S_l+1}}(l)$, $l\in L$, заранее вычислены.
     
     Пусть $\overline{k}_u =\{k_v,\ v\in\Omega_u^+\}$~--- состояние буферной 
памяти узла $u\in U$, $k_v$~--- число пакетов в буферной памяти узла, 
передаваемых по линии~$v$, $A_m = 
\{\overline{k}_u:\sum\limits_{v\in\Omega_u^+} k_v=m\}$~--- множество 
различных состояний, при которых в памяти узла заняты ровно $m$~буферов. 
Тогда с учетом введенных выше обозначений и предположений для 
стационарной вероятности блокировки узла можем написать формулу~[2, 11]
\begin{equation}
\pi_u = \fr{1}{G_{N_u}}\sum\limits_{\overline{k}\in A_N} p\left ( 
\overline{k}_u,\overline{\rho}_u^*\right )\,,
\end{equation}
где 
\begin{align}
p(\overline{k}_u,\overline{\rho}_u^*) & = 
\prod\limits_{v\in \Omega_u^+} z_v(\rho_v^*,k_v)\,,\notag\\
Z_v(\rho_v^*,k_v) & = 
\begin{cases}
\fr{\rho_v^{*k_v}}{k_v!} & \mbox{при}\ k_v<c_v\,,\\
\fr{\rho_v^{*k_v}}{c_v ! c_v^{k_v-c_v}} & \mbox{при}\ k_v\geq c_v\,,
\end{cases}\\
G_{N_u} & = \sum\limits_{m=0}^N \sum\limits_{\overline{k}\in A_m} 
p(\overline{k}_u, \overline{\rho}_u^* )\,,\quad \overline{\rho}_u^*=\{\rho_v^*,\ v\in 
\Omega_u^*\}\quad \rho_v^* =\fr{\Lambda_v^*}{\mu_v}\,,\quad v\in\Omega_u^+\,.
     \end{align}
     
     Таким образом, из соотношений~(1)--(5) относительно неизвестных 
величин~$\pi_u$, $u\in U$, получаем систему нелинейных уравнений вида
     \begin{equation*}
     \pi_u = f_u\left ( 
\overline{\lambda},\overline{\mu},\overline{N},\overline{\pi}\right )\,,\quad u\in 
U\,,
     \end{quation*}
     где $\overline{\lambda} =\{\lambda (l),\ l\in\L_u\}$, $\overline{mu} 
=\{\mu_{u^\prime},\ u^\prime\in u\cup U_u^+\}$, $\overline{N}=\{N_{u^\prime},\  
u^\prime \in u\cup U_u^+\}$, $\overline{\pi} = \{\pi_{u^\prime},\ u^\prime \in u\cup 
U_u^+\}$.
     
     Переобозначив $1-\pi+u$ через~$v_u$, выражение в правой части 
равенства для~$p(\overline{k}_u, \overline{\rho}_u^*)$ из (4)~--- через 
$p_{\overline{k}}(\overline{\rho}_u, y_u)$, выражение в правой части равенства 
для~$\pi_u$ из (3)~--- через $1-q_{N_u}(\overline{\rho}_u, y_u)$, где 
$\overline{\rho}_u = (\rho_v, \ v\in\Omega_u^+)$, $\rho_v = \rho_v^* y_u = 
\Lambda_v/\mu_v$, $v\in\Omega_u^+$, получим систему нелинейных уравнений 
относительно неизвестных переменных~$y_u$
     \begin{equation}
     y_u = q_{N_u}\left ( \overline{\rho}_u,y_u\right ),\quad u\in U\,.
     \end{equation}
     
     Отметим, что $\overline{\rho}_u = \{\rho_v, v\in\Omega_u^+\}$ где 
$\rho_v$~--- функция переменных $\overline{y}_u =\{y_{u^\prime},\ u^\prime \in 
U_u^+\}$.
     
     Обозначим набор $\{y_u, u\in U\}$ через~$\overline{y}$. Будем говорить, 
что решение~$\overline{y}$ положительное, если $y_u\in (0,\,1]$ для всех $u\in 
U$.
     
     \medskip
     
     \noindent
     \textbf{Утверждение~1.} \textit{Если} 
     $\overline{y}^\prime = \{u_u^\prime \in (0,\,],\ u\in U\}~--- \textit{решение 
системы уравнений}~(6), \textit{то необходимо выполнение для всех} $u\in U$ 
\textit{условия}
     \begin{equation}
     \fr{\sum\limits_{\overline{k}\in A_{N_u-1}} 
p_{\overline{k}}(\overline{p}_u^\prime , 1)}
     {\sum\limits_{\overline{k}\in A_{N_u}} 
p_{\overline{k}}(\overline{\rho}_u^\prime, 1)} >1\,,
     \label{e7ag}
     \end{equation}
     \textit{где} $\overline{\rho}_u^\prime$~--- \textit{значение 
переменной}~$\overline{\rho}_u$ \textit{при} $\overline{y} 
=\overline{y}^\prime$.
     
     \medskip
     
     \noindent
     Д\,о\,к\,а\,з\,а\,т\,е\,л\,ь\,с\,т\,в\,о\,.\ Пусть $\overline{y}^\prime = 
\{y_u^\prime \in (0,\,1]$, u\in U\}$~--- решение системы~(5). Фиксируем 
произвольный узел~$u$ и положим $u_{u^\prime} = y_u^\prime$ для всех 
$u^\prime\not= u$, $u^\prime\in U$. Отметим, что значение 
переменной~$\overline{\rho}_u$ при заданных значениях~$\lambda(l)$, $l\in L$, 
однозначно определяется переменными~$y_{u^\prime}$, $u^\prime\not= u$, 
$u^\prime\in U$ (см.~(2)). Рассмотрим уравнение
\begin{equation}
Y_u = q_{N_u} (\overline{\rho}_u^\prime, y_u)\,.
\label{e8ag}
\end{equation}
     
     Из работы~[12] (см.\ утверждение~4) следует, что уравнение~(\ref{e8ag}) 
имеет положительное решение тогда и только тогда, когда в узле~$\u$ 
выполняется условие~(\ref{e7ag}), при этом оно будет единственным 
положительным решением. Так как узел $u$~--- произвольный, то получаем, 
что неравенство~(\ref{e7ag}) должно выполняться для всех $u\in U$.
     
     \medskip
     
     \noindent
     \textbf{Следствие.} \textit{Выполнение неравенств} $\mu_v c_v / 
\Lambda_v >1$, $v\in V$, \textit{является необходимым условием 
существования положительного решения системы уравнений}~(\ref{e6ag}).
     
     \medskip
     
     \noindent
     Д\,о\,к\,а\,з\,а\,т\,е\,л\,ь\,с\,т\,в\,о\ непосредственно вытекает из следствия 
утверждения~4 в~[12].
     
     \smallskip
     Пусть задана последовательность $\overline{y}[n] =\{ y_u [n],\ u\in U\}$, 
$n\geq 0, где $y_n[n+1]=q_{N_u}[n+1]=q_{N_u}(\overline{\rho}_u[n],y_u[n])$, 
y_u[0]=1$, $u\in U$, а $\overline{\rho}_u[n]$~--- это~$\overline{\rho}_u$, 
вычисленное при $y_u =1-\pi_u =y_u[n]$. В дальнейшем будем писать 
$\overline{y}[n+1]<\overline{y}[n]$, если для заданного $n\geq 0$ выполняется 
$y_u[n+1]<y_u[n]$ для всех $u\in U$.
     
     \medskip
     
     \noindent
     \textbf{Утверждение 2.} \textit{Для всех} $n\geq 0$ \textit{верно} 
$\overline{y}[n+1] <\overline{y}[n]$.
     \medskip
     
     \noindent
     Д\,о\,к\,а\,з\,а\,т\,е\,л\,ь\,с\,т\,в\,о\,.\ Докажем, что для любых~$u$, 
$u^\prime \in U$, принадлежащих одновременно хотя бы одному пути, 
справедливо неравенство
     \begin{equation}
     \fr{\partial q_{N_u}(\overline{\rho}_u,y_u)}{\partial y_{u^\prime}}>0\,.
     \end{equation}
     
     Взяв производную от $y_u =q_{N_u}(\overline{\rho}_u,y_u)$ как от 
сложной функции, получаем
     \begin{equation}
     \fr{\partial q_{N_u}(\overline{\rho}_u,y_u)}{\partial y_{u^\prime}} = 
\sum\limits_{v\in\Omega_u^+}\fr{\partial q_{N_u}(\overline{\rho}_u,y_u)}{\partial 
\Lambda}\,\fr{\partial \Lambda_v}{\partial y_{u^\prime}}.
     \end{equation}
     
     Введем обозначения:
        .
     Из (3)--(5), взяв производную по~$\Lambda_v$, имеем
     \begin{equation}
     \fr{\partial q_{N_u}(\overline{\rho}_u,y_u)}{\partial\Lambda_v} = 
\fr{1}{\Lambda_v}\,q_{N_u}(\overline{\rho}_u,y_u)\left [ d_{N_u-
1}(\overline{\rho}_u,y_u)-d_{N_u}(\overline{\rho}_u,y_u)\right ]\,.
     \end{equation}
     
     Из (1) и~(2), взяв производную по~$y_{u^\prime}$, получаем
     \begin{equation}
     
      . (12)
     
     Подставив~(11) и~(12) в~(10), имеем
     \begin{equation*}
     \fr{\partial q_{N_u}(\overline{\rho}_u,y_u)}{\partial y_{u^\prime}} = 
\fr{1}{}\,q_{N_u}(\overline{\rho}_u, y_u)\left [d_{N_u}(\overline{\rho}_u,y_u)-
d_{N_u -1} (\overline{\rho}_u,y_u)\right 
]\sum\limits_{v\in\Omega_u^+}\fr{1}{\Lambda_v} \sum\limits_{l:l\in L_v,\, 
\mu^\prime\in l} \Lambda_v(l)\,.
     \end{equation*}
     
     Так как справедливо неравенство $d_{N_u}(\overline{\rho}_u,y_u)-d_{N_u 
-1} (\overline{\rho}_u,y_u) >0$ (см.\ утверждение~1 из~[12]), то из последнего 
равенства следует доказательство неравенства~(9). Из определения 
последовательности $\overline{y}[n]$, $n\geq 0$, и из~(9) следует 
доказательство утверждения~2.
     
     \medskip
     
     \noindent
     \textbf{Утверждение 3.} \textit{Последовательность}~$\overline{y}$, 
$n\geq 0$, \textit{сходится к положительному решению системы}~(6) 
\textit{тогда и только тогда, когда существует положительное решение 
системы}~(6).
     
     \medskip
     
     \noindent
     Д\,о\,к\,а\,з\,а\,т\,е\,л\,ь\,с\,т\,в\,о\,.\ Пусть $\overline{y}^* = \{y_u^*\in 
(0,\,1],\ u\in U\}$~--- решение системы уравнений~(6), $\overline{p}_u^*$~--- 
значение переменной~$\overline{\rho}_u$ при~$\overline{y}^*$. Очевидно, 
$u_u^*<1$, $u\in U$, так как $q_{N_u}(\overline{\rho}_u,y_u)<1$ при любых 
$y_u\in (0,\,1]$, $u\in U$. Пусть для некоторого $n\geq 0$ 
$\overline{y}[n]>\overline{y}^*$ (существование такого~$n$ вытекает из того, 
что $u_n[0] =1$ и $y_u^*<1$, $u\in U$). Тогда, как следует из~(9), для каждого 
узла $u\in U$ $y_u[n+1]=q_{N_u} (\overline{\rho}_u[n],y_u[n]) > q_{N_u}
     (\overline{\rho}_u^*,y_u^*)=y_u^*$, т.\,е.\ последовательность~$u_u[n]$, 
$n\geq 0$, ограничена снизу величиной~$\overline{y}_u^*$. Значит, 
существуют пределы $\lim\limits_{n\rightarrow\infty} y_u[n]=y_u^0\geq 
y_u^*>0$ для всех $u\in U$. Так как $q_{N_u}(\overline{\rho}_u,y_u), \rho_v$, 
$v\in\Omega_u^+$,~--- непрерывные по $y_{u^\prime}$, $u^\prime\in u\cup 
U_u^+$ функции, то можно написать $\lim\limits_{n\rightarrow\infty} q_{N_u} 
(\overline{\rho}_u[n],y_u[n])=q_{N_u}(\overline{\rho}_u^0,y_u^0)=y_u^0$, где 
$\overline{\rho}_u^0$~--- значение переменной~$\overline{\rho}_u$ при 
$y^0_{u^\prime}$, $u^\prime\in U_u^+$, т.\,е.\ $\overline{y}^0 =\{y_u^0\in (0,\,1),\ 
u\in U\}$~--- положительное решение уравнения~(6).
     
     Пусть теперь $\lim\limits_{n\rightarrow\infty} y_u[n] =y_u^*>0$ для всех 
$u\in U$. Тогда, как показано в первой части доказательства утверждения, 
$\overline{y}^0 = \{y_u^0\in (0,\,1),\ u\inU\}$~--- положительное решение 
уравнения~(6). Утверждение~3 доказано.
     
     \medskip
     
     \noindent
     \textbf{Следствие 2.} \textit{Система}~(6) \textit{не имеет 
положительного решения тогда и только тогда, когда} 
$\lim\limits_{n\rightarrow\infty} y_u[n]=y_u^*=0$ \textit{для всех} $u\in U$.
     
     Пусть во всех узлах сети распределение буферной памяти происходит по 
схеме CP. Тогда формула вероятности блокировки узла $u\inU$ для $v$-заявки 
($v\in \Omega_u^*$) записывается в том же виде, что и~(3)--(5), с заменой 
всюду индекса~$u$ на~$v$, за исключением обозначения~$\Omega_u^*$. 
Нетрудно заметить, что в случае этой схемы система уравнений~(6) примет вид
     \begin{equation}
     y_v = q_{N_u}(\rho_v,y_v)\,,\quad v\in V\,,
     \end{equation}
     где $\rho_v$ является функцией~$y_{v^\prime}$, $v^\prime \in V_v^+$, 
которая обладает всеми свойствами системы~(6), использованными при 
доказательстве утверждений~1, 2 и~3 и следствий. Неравенства вида~(6) в 
данном случае преобразуются в $\mu_v c_v/\Lambda_v >1$, $v\in V$.
     
     Заметим также, что все рассуждения, приведенные выше для сети с одной 
только из указанных выше схем распределения буферной памяти, справедливы 
и в смешанном случае, когда в узлах используется любая из этих схем.
     
\section{Сеть с повторами из предыдущего узла}
     
     Рассмотрим сеть с полнодоступной буферной памятью и повторами из 
предыдущего узла. В качестве модели узла используется СМО, отличающаяся 
от СМО, определенной в предыдущем разделе, только пунктами~6 и~7. Вместо 
действий, указанных в этих пунктах, реализуется следующее: выполненная 
     $v$-заявка с заданной вероятностью~$B_{v^+}$ (вероятность блокировки 
последующего узла или ошибки при передаче пакета по линии~$v$) 
повторяется в данном узле через заданное время~$\tau_v$ (тайм-аут) и с 
вероятностью~$1-B_{v^+}$ покидает систему через время~$t_v$ навсегда, 
сразу освободив занятый канал и место в буферной памяти. Для такой модели 
существует более общая формула для вычисления вероятности блокировки 
системы для $v$-заявок (см.~[11, 12]), которая задает зависимость вероятности 
блокировки узла в виде функции от вероятностей блокировок последующих 
узлов~$B_{v^+}$, $v\in \Omega_u^+$, при заданных значениях остальных 
параметров, в частности~$\Lambda_v$, $v\in\Omega_u^+$.
     
     В сети с повторами из предыдущего узла при установившемся режиме 
работы справедливы следующие уравнения баланса потоков:
     \begin{align*}
     \lambda (l) & = \Lambda_v^*(l) (1-\pi_v)\delta_v\,,\quad l\in L_v\,,\\
     \Lambda_v & = \sum\limits_{l\in L_v} \lambda_v(l)\,,\quad 
\Lambda_v^*=\sum\limits_{l\in L_v} \Lambda_v^*(l)\,,\quad v\in V\,.
     \end{align*}
     
     Тогда с учетом введенных выше обозначений и формул~(1)--(7) и~(18) 
из~[12], заменив обозначение~$B_{v^+}$ на~$y_{v^+}$, $v\in\Omega_u^+$, 
можем написать систему нелинейных уравнений
     \begin{align}
     y_u &= q_{N_u}(\overline{\rho}_u,y_u)\ \mbox{при схеме распределения 
CP
     }\,,\\
     y_v &= q_{N_u}(\rho_v,y_v)\ \mbox{ при схеме распределения CS
     },\, \ v\in\Omega_u^+\,\ u\inU\,,
     \end{align}
где компоненты~$\rho_v$ набора $\overline{\rho}_u$~--- функции 
переменных~$y_{v^+}$, $v\in \Omega_u^+$.

     Нетрудно видеть, что системы~(14) и~(15) обладают всеми свойствами 
системы~(6), использованными при доказательстве утверждений~1, 2, 3 и 
следствий, т.\,е.\ для систем~(13) и~(14) также справедливы утверждения~1, 2, 
3 и следствия.
     
\section{Алгоритм расчета}
     
     Для вычисления характеристик потоков в узлах и вероятностей 
блокировок пакетов предлагается следующий алгоритм, описывающий 
изложенную выше итерационную процедуру. Для описания значений, 
вычисляемых на $k$-м шаге алгоритма, к обозначениям соответствующих 
параметров приписывается знак~$[k]$. Введем новые обозначения:
     
     \noindent
     $y_u^v$~--- вероятность отсутствия блокировки узла $u\in U$ для 
пакетов, направляемых на линию $v\in \Omega_u^+$;
     \begin{align*}
     y_u^v & = \begin{cases}
     y_u & \mbox{для}\ v\in\Omega_u^+\ \mbox{при схеме}\ CS\,,\\
     y_v & \mbox{при схеме распределения CP}\,;
     \end{cases}\\
     \overline{\rho}_u^v & = 
     \begin{cases}
     \overline{\rho}_u & \mbox{для}\ v\in\Omega_u^+\ \mbox{при схеме}\ 
CS\,,\\
     \rho_v & \mbox{при схеме распределения CP}\,;
     \end{cases}\\
     q_{N_u}^v(\overline{\rho}_u^v, y_u^v) & = 
     \begin{cases}
     q_{N_u}(\overline{\rho}_u,y_u) & \mbox{для}\ v\in\Omega_u^+\ 
\mbox{при схеме}\ CS\,,\\
     q_{N_v}(\rho_v,y_v)& \mbox{при схеме распределения CP}\,;
     \end{cases}
     \end{align*}
     
     Тогда система уравнений для смешанного варианта сети, аналогичная 
системам~(6), (13)--(15), записывается в виде
     $$
     y_u^v = q_{N_u}^v(\overline{\rho}_u^v,\overline{y}_u^v)\,,\quad u\in 
U\,,\quad v\in\Omega_u^+\,.
     $$

\textbf{Шаг 1.} \textit{Инициализация}. Вычисление начальных значений 
параметров~$\rho_v^*$, $v\in V$: $\Lambda_v[0]=\sum\limits_{l\in L_v} 
\lambda(l)/((1-\pi_{u_{S_l+1}}(l)\prod\limits_{v^\prime\in 
V^+}\delta_{v^\prime}$, $\rho_v^*[0]=\Lambda_v[0]/\mu_v$, $v\in V$, 
$y_u^v[0]=1$, $u\in U$, $v\in\Omega_u^+$.
     
     \textbf{Шаг} $k$ ($k>1$).
     \begin{enumerate}[1.]
\item \textit{Проверка необходимых условий существования решения}. 
Если для некоторой линии $v\in V$ выполняется условие 
$c_v\mu_v/(\Lambda_v[k-1])\leq 1$, то алгоритм заканчивает работу с 
результатом <<система не имеет решения>>. Если в некотором узле~$u$, 
в котором используется полнодоступная схема, условие 
$c_v\mu_v/(\Lambda_v[k-1])> 1$ выполняется для всех $v\in\Omega_u^+$, 
то проверяется условие~(7) заданных $\Lambda_v[k-1]$, $v\in V$, и при 
невыполнении этого условия алгоритм заканчивает работу с результатом 
<<система не имеет решения>>.
     \item \textit{Вычисление вероятностей блокировок}. Используя 
значения параметров $\overline{\rho}_u^v[k-1]$, $y_u^v[k-1]$, $u\in U$, 
$v\in\Omega_u^+$, с помощью соответствующих формул~(3)--(5) или 
формул~(1)--(7) и~(18) из~[12] (в зависимости от типа схемы 
распределения буферной памяти и процедуры повторов передач) 
вычисляется $y_u^v[k]=1-\pi_u[k]$, $u\in U$, $v\in\Omega_u^+$. При этом 
рекомендуется использовать метод свертки Базена (см.~[13]), 
позволяющий производить рекуррентные вычисления (подробно этот 
метод описан также в~[2, 9]).
     \item \textit{Вычисление значений параметров} $\Lambda_v[k]$, $v\in V$:
     \begin{enumerate}[$i$)]
     \item в случае повторов от источника
     \begin{gather*}
     \Lambda_{v_{S_l}}[k]=\lambda(l)\,,\ \Lambda_{v_i}^*(l)[k]=
     \fr{\Lambda_{v_i}(l)[k]}{y^{v_i}_{u_i}[k-1]\delta_{v_i}}\,,
     \Lambda_{v_i-1}(l)[k]=\Lambda_{v_i}^*(l)[k]\,,\ i=1,\ldots ,S_l\,,\ l\in 
L\,,\\
     \Lambda_v^=[k] = \sum\limits_{l\in L_v} \Lambda_v^=(l)[k]\,,\quad 
v\in V\,;
     \end{gather*}
     \item в случае повторов из предыдущего узла
     \begin{equation*}
     \Lambda_v^*[k]=\fr{\Lambda_v[0]}{y_u^v[k-1]\delta_v}\,,\quad 
v\in\Omega_u^+\,,\quad u\in U\,.
     \end{equation*}
     \end{enumerate}
     \item \textit{Проверка условий останова алгоритма}. Если хотя бы для 
одной $v\in V$ для заданного значения точности $\varepsilon >0$ выполняется 
условие
     $$
     \fr{\vert \Lambda_v^*[k]-\Lambda_v^*[k-1]\vert}{\Lambda_v^*[k]} 
>\varepsilon\,,
     $$
     то вычисляются параметры $\overline{\rho}_u^v[k]$, $u\in U$, 
$v\in\Omega_u^+$, и осуществляется переход к шагу~$k$, положив $k$ 
равным~$k+1$, иначе алгоритм завершает работу.
     \end{enumerate}
     
     По завершении алгоритма либо выявляется, что система уравнений не 
имеет положительного решения, либо вычисляются интенсивности потоков, 
поступающих в узлы и на линии сети, и вероятности блокировок узлов для 
пакетов. Далее эти характеристики могут быть использованы для вычисления 
других характеристик сети (средних задержек, среднего числа повторов в узлах, 
узких участков сети и~др.).
     
\section{Примеры расчета}

     В качестве примера рассматривается сеть с тремя узлами, топология 
которой задается графом, показанным на рис.~1. В рассматриваемой сети 
предполагается полнодоступная схема распределения буферной памяти и 
процедура повторных передач из источника. Для вычисления вероятностей 
блокировок узлов и интенсивностей потоков, поступающих на линии связи, 
был использован алгоритм, представленный в разд.~5, и имитационная модель 
сети. В табл.~1 и на рис.~2 приведены результаты вычислений при следующих 
значениях входных параметров: емкости накопителей $N_u = 15$ для всех 
узлов, множество путей
     $$
     L = \{l_1, l_2, l_3, l_4, l_5,l_6\}\,,\  l_1=\{v_1\}\,,\ l_2=\{v_2\}\,,\ 
l_3=\{v_3\}\,,\ l_4=\{v_2,v_3\}\,,\ l_5=\{v_1,v_2\}\,,\ l_6=\{v_3,v_1\}\,,
     $$
     интенсивности первичных потоков $\lambda (l) =2$, 2,5, 2,7, 2,8, 2,9, 3, 
3,1, 3,2, $l \in L$.  Строки~1, 3 в табл.~1 и графики~\textsl{1}, \textsl{3} на 
рис.~2 соответствуют вариантам расчетов с помощью предложенного 
алгоритма, а \textsl{2}, \textsl{4}, \textsl{5}~--- с помощью имитационной 
модели. В вариантах~1 и~2 канальные емкости $c_v = 10$ для всех линий, 
параметр экспоненциального времени обслуживания   для всех линий, 
интервал повтора для всех заявок равен~10, в вариантах~3 и~4 емкости $c_v = 
1$ для всех линий, параметр экспоненциального времени обслуживания $\mu_v 
= 10$ для всех линий, интервал повтора равен~10, в варианте~5 емкости $c_v = 
10$ для всех линий, время обслуживания пакета каналом связи равно~10, 
интервал повтора равен~10. Во всех вариантах первичные потоки~--- 
пуассоновские, $\pi_{u_{S_l+1}}(l) =0$, $\delta_v =1$ $v\in V$, $l\in 
\begin{figure*} %fig1
     \Caption{Граф сети
     \label{f1ag}}
     \end{figure}
     
\begin{table}\small
\begin{center}
\Caption{Зависимость вероятности отсутствия блокировки узла от интенсивности первичных 
потоков
\label{t1ag}}
\vspace*{1ex}

\begin{tabular}{ccccccccc}
\hline
& \multicolumn{8}{$\lambda (l)$\\
\cline{2-9}
    
&2&2.5&2.7&2.8&2.9&3&3.1&3.2\\
\hline
1&0,9967&0,9758&0,9504&0,9272&0,8825&0,0000&0,0000&0\\
2&0,9905&0,9882&0,9257&0,9211&0,6928&0,0000&0,0000&0\\
3&0,9998&0,9964&0,9904&0,9844&0,9746&0,9568&0,9018&0\\
4&1,0000&0,9954&0,9934&0,9873&0,972&0,949&0,8787&0\\
5&0,9986&0,9917&0,9718&0,9677&0,9569&0,8018&0,0000&0\\
     \hline
     \end{tabular}
     \end{center}
     \end{table}
     
     Как показывают результаты, отраженные в табл.~1 и на рис.~2, а также 
другие вычислительные эксперименты, оценки вероятностей блокировок узлов, 
полученные с помощью представленного алгоритма, дают вполне приемлемые 
по точности значения для предварительного анализа сети на реализуемость 
первичных потоков пакетов.
     
     \begin{figure} %fig2
     \Caption{Зависимость вероятности отсутствия блокировки узла от 
интенсивности первичных потоков
     \label{f2ag}}
     \end{figure}
     
     Кроме того, точность результатов, полученных с помощью предлагаемого 
итерационного метода, увеличивается с ростом разветвленности сети и 
увеличением интервала повторов передач пакета. Эксперименты также 
показывают, что, как правило, погрешность, вносимая заменой многоканальной 
линии одноканальной с равной пропускной способностью, больше, чем 
вносимая предположением о пуассоновских входных потоках и 
экспоненциальных временах обслуживания.
     
\section{Заключение}
     
     Проведенные исследования показали, что алгоритм расчета сетей, 
предложенный в данной статье, обладает следующими достоинствами:
     \begin{enumerate}[1.]
     \item Использует в качестве модели сети СеМО, представляющие собой 
совокупность общепринятых СМО типа $
      \begin{matrix}
      M \\ \vec{\lambda}
      \end{matrix}
      \left |
      \begin{matrix}
      M \\ \vec{\lambda}
      \end{matrix}
      \right |
\vec{m} \vert N$ со схемами распределения CS или CP, связанных 
уравнениями баланса потоков в узлах.
\item При реализуемых первичных потоках сходится к положительному 
решению системы уравнений баланса потоков в узлах.
\item При реализуемых первичных потоках вычисляет вероятности 
блокировок узлов и загруженности узлов и каналов связи с приемлемой 
для предварительного анализа сети точностью (относительная 
погрешность вероятности блокировки $\sim 0.1$) .
\item Позволяет определить реализуемость первичных входных потоков.
\item При использовании алгоритма Базена требует для выполнения 
одного шага порядка $\sum\limits_{u\in U} (N_uK_u+N_u^2/2)$ 
арифметических операций, где $K_u$~--- степень узла~$u$.
    \end{enumerate}


{\small\frenchspacing
{%\baselineskip=10.8pt
\addcontentsline{toc}{section}{Литература}
\begin{thebibliography}{99}    
\bibitem{1ag}
\Au{Клейнрок~Л.}
Теория массового обслуживания.~--- М.: Машиностроение, 1979.

\bibitem{2ag}
\Au{Башарин~Г.\,П., Бочаров~П.\,П., Коган~Я.\,А.}
Анализ очередей в вычислительных сетях.~--- М.: Наука, 1989.

\bibitem{3ag}
\Au{Бочаров~П.\,П., Д'Апиче~Ч., Мандзо~Р., Фонг~Н.\,Х.}
Об обслуживании многомерного пуассоновского потока на одном 
приборе с конечным накопителем и повторными заявками~// Проблемы 
передачи информации, 2001. Т.~37. Вып.~4. С.~130--140.

\bibitem{4ag}
\Au{Tsitsiashvili~G.\,Sh., Osipova~M.\,A.}
Construction of queueing networks with stationary product distributions~// 
Proceedings of 5th Workshop (International ) on Retrial Queues.~--- Seoul: 
Korea University, 2004. Р.~111--115.

\bibitem{5ag}
\Au{Моисеева~С.\,П., Морозова~А.\,С., Назаров~А.\,А.}
Исследования СМО с повторным обращением и неограниченным числом 
обслуживающих приборов методом предельной декомпозиции~// 
Вычислительные технологии, 2008. Т.~13. Спец. вып.~5. С.~88--92.

\bibitem{6ag}
\Au{Wuechner~P., Meer~H., Bolc~G., Roszik~J., Sztrik~J.}
Modeling finite-source retrial queueing systems with unreliable heterogeneous 
servers and different service policies using MOSEL~// Proceedings of the 14th 
International conference on analytical and stochastic modeling techniques and 
applications, 2007, Prague, Czech Republic.~--- Sbr.-Dudweiler: Digitaldruck 
Pirrot GmbH, 2007. P.~75--80.

\bibitem{7ag}
\Au{Artalejo~J., G\'{o}mez-Corral~A.}
Retrial queueing systems. A computational approach.~--- Berlin: Springer 
Berlin Heidelberg, 2008.

\bibitem{8ag}
\Au{Бочаров~П.\,П.}
Приближенный метод расчета разомкнутых неэкспоненциальных сетей 
массового обслуживания конечной емкости с потерями или 
блокировками~// Автоматика и телемеханика, 1987. №\,1. C.~55--65.

\bibitem{9ag}
\Au{Вишневский~В.\,М.}
Теоретические основы проектирования компьютерных сетей.~--- М.: 
Техносфера, 2003.

\bibitem{10ag}
\Au{Таранцев~А.\,А.}
Инженерные методы теории массового обслуживания.~--- М.: Наука, 
2007.

\bibitem{11ag}
\Au{Kamoun~F., Kleinrock~L.}
Analysis of shared finite storage in a computer networks node environment 
under general traffic conditions~// IEEE Trans. on Commun., 1980. Vol.~28. 
No.\,7. P.~992--1003.

\bibitem{12ag}
\Au{Агаларов~Я.\,М.}
Приближенный метод вычисления характеристик узла 
телекоммуникационной сети с повторными передачами~// Информатика 
и её применения, 2009. Т.~3. Вып.~2. С.~2--10.

\label{end\stat}


\bibitem{13ag}
\Au{Buzen~J.\,P.}
Computational algorithm for closed queuing networks with exponential 
servers~// Communications ACM, 1973. Vol.~16. No.\,9. P.~527--531.

 \end{thebibliography}
}
}
\end{multicols}