\def\stat{ushmaev}

\def\tit{ПРОБЛЕМЫ РАСПАРАЛЛЕЛИВАНИЯ БИОМЕТРИЧЕСКИХ ВЫЧИСЛЕНИЙ 
В~КРУПНОМАСШТАБНЫХ ИДЕНТИФИКАЦИОННЫХ СИСТЕМАХ}
\def\titkol{Проблемы распараллеливания биометрических вычислений 
в~крупномасштабных идентификационных системах} 

\def\autkol{О.\,С.~Ушмаев}
\def\aut{О.\,С.~Ушмаев$^1$}

\titel{\tit}{\aut}{\autkol}{\titkol}

%{\renewcommand{\thefootnote}{\fnsymbol{footnote}}\footnotetext[1]
%{Работа выполнена при поддержке РФФИ, проекты 08--07--00152 и 08--01--00567.}}

\renewcommand{\thefootnote}{\arabic{footnote}}
\footnotetext[1]{Институт проблем информатики Российской академии наук, oushmaev@ipiran.ru
}

 
\Abst{Рассматриваются проблемы распараллеливания биометрических вычислений в 
крупномасштабных идентификационных системах. Предлагается методика вычисления 
производительности биометрического кластера, методы определения необходимой мощности 
вычислительных средств. Разрабатываются методы оптимизации и распараллеливания 
вычислений в задаче мультибиометрической идентификации.}

\KW{мультибиометрическая идентификация; 
параллельные вычисления}

  \vskip 30pt plus 9pt minus 6pt

      \thispagestyle{headings}

      \begin{multicols}{2}

      \label{st\stat}


     \section{Введение}
     
     К настоящему времени накоплен значительный опыт создания 
крупномасштабных био\-мет\-риче\-ских систем. Большинство из них являются 
авто\-ма\-ти\-зи\-ро\-ван\-ны\-ми дактилоскопическими сис\-те\-ма\-ми (АДИС). Выбор 
отпечатков пальцев обуслов\-лен множеством факторов. Основным из них %\linebreak
 является традиционное 
использование отпечатков пальцев в криминальном учете и высокий потенциал 
дактилоскопии с точки зрения ошибок распознавания. Для практически 
безошибочного распознавания людей в масштабах населения страны %\linebreak  
достаточно
использования всех десяти отпечатков пальцев.  При 
использовании меньшего чис\-ла отпечатков соотношение ошибок 1-го и 2-го  родов 
является удовлетворительных во многих при\-ло\-же\-ни\-ях~[1, 2]. Однако использование 
дак\-ти\-ло\-ско\-пии имеет недостатки. Накопленный опыт реализации АДИС для 
криминалистической и %\linebreak 
гражданской идентификации позволяет выделить следующие 
наиболее критичные направления развития~[3--6]:
     \begin{itemize}
\item идентификация людей, не обладающих пригодными для распознавания 
отпечатками пальцев (инвалиды, плохое состояние кожи);
\item увеличение производительности.  
\end{itemize}

     Первая задача возникает из непосредственных требований к биометрическим 
системам, а именно: она должна автоматически идентифицировать личность по 
предъявляемым биометрическим образцам. Соответственно люди с плохим качеством 
отпечатков пальцев не могут надежно идентифицироваться средствами АДИС.
     
     Вторая задача~--- увеличение про\-из\-во\-ди\-тель\-ности~--- связана с тем, что при создании 
крупномасштабных биометрических систем более 75\% средств затрачиваются на 
аппаратные средства  вы\-чис\-ли\-тель\-ных узлов~[7].  Уже в настоящее время потребность 
в вы\-чис\-ли\-тель\-ных мощностях со\-времен\-ных АДИС достигает тысяч серверов, что 
требу\-ет параллельных вычислений и супер\-компь\-ю\-тер\-ных мощностей. При 
пол\-но\-масштаб\-ном внед\-ре\-нии систем гражданской идентификации, таких как 
пас\-порт\-но-ви\-зо\-вые до\-кумен\-ты %\linebreak 
нового %\linebreak 
по\-коле\-ния, био\-мет\-ри\-че\-ские массивы
 вырас\-тут 
многократно, поэтому проблемы увеличения производительности \mbox{будут} становиться 
все более актуальными. В такой ситуации первоочередной проблемой становится 
орга\-ни\-за\-ция вы\-чис\-ли\-тель\-но\-го кластера биометрической системы и параллельных 
вычислений~[2, 7--9].
   
     Проблемы организации параллельных вычислений усугубляются тем, что многие 
перспективные крупномасштабные биометрические системы планируются как 
мультибиометрические с привлечением одновременно нескольких биометрик, 
отличных по своим эксплуатационным показателям и используемым вычислительным 
средствам. В~\cite{2ush} показано, что привлечение нескольких биометрик может 
увеличить скорость идентификации при правильной организации вычислений. В то же 
время это создает дополнительные трудности при проектировании. В частности, 
определение необходимой производительности вычислительных средств 
мультибиометрической системы становится довольно сложной методологической 
проблемой.
    
     


     Решение проблемы параллельных вычислений в мультибиометрических 
идентификационных сис\-те\-мах требует решения следующих задач:
     \begin{itemize}
\item выработка методологии оценки производительности биометрического 
кластера;
\item определение потребности в вычислительных средствах в зависимости от 
состава биометрического программного обеспечения;
\item организация параллельных вычислений.
\end{itemize}


     В статье рассмотрена проблема параллельных вычислений в 
мультибиометрической системе. В~настоящем разделе поставлена задача. В~разд.~2 
дана методология оценки производительности биомет\-ри\-че\-ской идентификации. 
Особенности оценки производительности при муль\-ти\-био\-мет\-ри\-че\-ской идентификации 
представлены в разд.~3. %\linebreak
 Раздел~4 посвящен определению потребности в 
вы\-чис\-ли\-тель\-ных средствах. В разд.~5 изложены подходы к организации параллельных 
вы\-чис\-ле\-ний. Пример применения разработанных методов на %\linebreak
 примере 
идентификационной системы, использующей отпечатки пальцев и изображение лица, 
представлены в разд.~6. Основные выводы и рекомендации приведены в заключении.
     
     
     \section{Оценка производительности биометрического кластера}
     
     Основным показателем производительности биометрической системы является 
проектное время ожидания отклика. Для крупномасштабных систем, таких как 
паспортно-визовые системы или системы криминального учета, время отклика
обычно 
устанавливается в 24~ч для суточного цик\-ла 
(в редких случаях 168~ч для недельного цикла)
функционирования системы~\cite{10ush}. Био\-мет\-ри\-че\-ская\linebreak\vspace*{-12pt}

%\begin{figure*}[b] %fig1
%\vspace*{1pt}
\begin{center}
\vspace*{1pt}
\mbox{%
\epsfxsize=80.45mm
\epsfbox{ush-1.eps}
}
\end{center}
%\vspace*{-9pt}
%\label{f1ush}}
%\end{figure*}
{{\figurename~1}\ \ \small{Пример графика потока заявок в биометрическую систему}}

\bigskip
\addtocounter{figure}{1}

\noindent
система 
должна успевать обрабатывать заявки на идентификацию, поступающие в течение 
суток или недели соответственно. Отказ от реального времени в пользу суточного 
цикла обычно связан с тем, что поток заявок имеет прогнозируемую неравномерную 
структуру. Примерный вид графика интенсивности запроса приведен на 
рис.~1. Максимальный участок соответствует времени, когда функционирует 
большинство пунктов сбора биометрической информации во всех часовых поясах. С 
точки зрения экономии ресурсов целесообразно производить вычисления равномерно в 
течение суток, а не согласно графику поступления.
   
     В таком случае проектное время обработки заявок, поступающих в течение суток, 
рассчитывается по следующей формуле:
     \begin{equation}
T_{\mathrm{сут}} = r_{\mathrm{сут}} t\,,
     \label{e1ush}
     \end{equation}
где $ r_{\mathrm{сут}}$~--- максимальный проектный поток заявок в течение суток; 
$t$~--- среднее время идентификации по биометрической базе.

     
      В случае АДИС (или другой однофакторной биометрии) время $t$ обычно 
линейно пропорционально количеству записей в базе данных (БД), поскольку в ходе 
идентификации предъявляемые образцы последовательно сравниваются с каждым 
хранимым, т.\,е.\ в большинстве приложений 
      \begin{equation}
      t=\fr{Nt_{\mathrm{ср}}} {W}\,,
      \label{e2ush}
      \end{equation}
где $N$~--- число записей в БД; $t_{\mathrm{ср}}$~--- среднее время сравнения пары 
биометрических образцов на единицу мощности вычислительных средств (с); $W$~--- 
мощность вычислительных средств, нор\-ми\-ро\-ванная на единичную номинальную 
мощность %\linebreak 
(например, на один процессор с тактовой частотой~1~ГГц).

     Формула~(\ref{e2ush}) верна, если пренебречь потерями, связанными с 
распараллеливанием вычислений.
     
     Чтобы система идентификации справлялась с плановым потоком заявок в течение 
суток, накладывается ограничение $T_{\mathrm{сут}} < 24$~ч. Резерв $R$ системы 
(избыточность) определяется как 
     \begin{equation*}
     R = \fr{24 -T_{\mathrm{сут}}}{24} =
     1-\fr{Nt_{\mathrm{ср}}}{W}\,\fr{r_{\mathrm{сут}}}{24} =1-
r\fr{Nt_{\mathrm{ср}}}{W} =1-rt\,,
%     \label{e3ush}
     \end{equation*}
     где $r$~--- интенсивность потока заявок.
     
     
          \begin{figure*}[b] %fig2
\vspace*{1pt}
\begin{center}
\mbox{%
\epsfxsize=134.727mm
\epsfbox{ush-2.eps}
}
\end{center}
\vspace*{-9pt}
     \Caption{Независимые процессы идентификации
     \label{f2ush}}
     \end{figure*}
     
     Избыточность в основном необходима в следующих случаях: 
     \begin{itemize}
\item сезонные колебания и резкие скачки нагрузки на биометрические серверы;
\item выход из строя части вычислительных мощностей;
\item плановый профилактический вывод из эксплуатации части вычислительных 
мощностей; 
\item сбой системы, приводящий к необходимости повторной обработки запросов.
\end{itemize}
     
     В случае систем оперативной идентификации, где время ожидания ограничено 
минутами, помимо средней способности системы обеспечить обработку потока заявок 
требуется ограничить дисперсию времени ожидания, чтобы в моменты пиковой 
загрузки вычислительных мощностей проектное время ожидания не было превышено. 
В таком случае требуется проводить специальное исследование структуры потока 
заявок, чтобы определить максимально возможную интенсивность.
     
     Как видно из приведенной формулы~(\ref{e2ush}), скорость сравнения 
биометрических образцов является значимым фактором при определении 
производительности.
     
     Вторым ограничивающим фактором является объем адресуемой памяти. Как 
показывает анализ современных тенденций развития аппаратных средств, темпы 
прироста мощности вычислительных средств опережают темпы прироста скорости 
доступа к оперативной памяти и жестким дискам. В такой ситуации становится 
критичным ор\-га\-ни\-за\-ция доступа к памяти. В крупномасштабных %\linebreak 
систе\-мах, где 
востребованы отпечатки пальцев, радужная оболочка глаза или изображение лица, 
единст\-вен\-ным способом быстрого сравнения является кэширование биометрической 
базы в оперативной памяти. Хранение на жестком диске %\linebreak
 нежелательно из-за низкой 
скорости доступа, которая многократно превышает время сравнения 
$t_{\mathrm{ср}}$. Таким образом, получаем, что на одном сервере биометрического 
кластера можно разместить $V/s_{\mathrm{tpl}}$~биометрических шаблонов, где $V$~--- размер 
доступной оперативной памяти, $s_{\mathrm{tpl}}$~--- размер биометрического шаблона. 
      
      Для 32-битной версии максимальное значение~$V$ составляет 4~ГБ. Реально за 
вычетом памяти, занятой вычислительным процессом и операционной сис\-те\-мой (ОС), остается порядка 
      3,5--3,7~ГБ. В случае 64-бит\-ной ОС максимальное адресуемое пространство 
составляет~8~ГБ.
   

     В случае одномодальной системы единственно возможной конфигурацией 
кластера, реализующей эффективное распараллеливание, является распределение 
биометрической базы на нескольких %\linebreak 
\mbox{узлах}. 
{\looseness=1

}
     
     \section{Мультибиометрическая идентификация}
     
     При мультибиометрической идентификации оценка времени 
сравнения~(\ref{e2ush}) является отдельной задачей. При привлечении нескольких 
техно\-ло\-гий биометрической идентификации возможны несколь\-ко вариантов 
реализации муль\-ти\-био\-мет\-ри\-че\-ской сис\-те\-мы, позволяющих корректировать 
эксплуатационные показатели. 
     
     В качестве базового примера рассмотрим бимодальные технологии, 
использующие две биометрики. В таком случае возможны две основные схема 
интеграции: 
     \begin{enumerate}[(1)]
\item процессы сравнения независимы (рис.~\ref{f2ush}); 
\item процессы сравнения зависимы (рис.~\ref{f3ush}).
\end{enumerate}

\begin{figure*} %fig3
\vspace*{1pt}
\begin{center}
\mbox{%
\epsfxsize=145.046mm
\epsfbox{ush-3.eps}
}
\end{center}
\vspace*{-9pt}
\Caption{Зависимые процессы идентификации
     \label{f3ush}}
     %\vspace*{24pt}
     \end{figure*}
     

     В~[10--15] разработана методология интеграции биометрических систем в случае 
независимого сравнения. Тогда достигаются минимально возможные ошибки 
распознавания. Однако это приводит к потерям производительности. Несложно 
показать, что время сравнения такой системы получается суммированием времени 
сравнения по каж\-дой биометрике.
     
     Во многих задачах, решаемых современными АДИС, качество идентификации 
является приемлемым в отличие от производительности, которая остается достаточно 
низкой. Поэтому далее в разделе мы сосредоточим основное внимание на схеме 
интеграции с зависимыми процессами идентификации, которая позволяет достичь 
прироста в производительности с приемлемыми потерями в качестве распознавания 
относительно независимого процесса сравнения.
     
     Рассмотрим реализацию идентификации с зависимыми процессами сравнения 
более детально. На вход функции одномодальной био\-мет\-ри\-че\-ской идентификации 
поступают предъявляемая био\-мет\-ри\-че\-ская запись или образец и био\-мет\-ри\-че\-ская 
БД, или линейный список био\-мет\-ри\-че\-ских записей. На выходе мы получаем 
меры сходства предъ\-яв\-ля\-емой и хранимых в БД записей. На основе этой информации 
принимается решение, принадлежат ли записи одному человеку или нет. Большинство 
систем идентификации по отпечаткам пальцев и изображению лица используют 
пороговые методы принятия решения.
     

     При идентификации по двум био\-мет\-ри\-кам увеличения производительности можно достичь, 
если по результатам идентификации по более быст\-рой из двух технологий принимать 
решение о целе\-со\-об\-раз\-ности дальнейшего поиска по второй био\-мет\-ри\-ке (схема 
реализации функции сравнения приведена на рис.~\ref{f4ush}). 


     Как видно из рис.~\ref{f4ush}, в такой функции сравнения биометрических 
образцов есть четыре терминальных состояния:
     \begin{itemize}
\item 2~--- при сравнении по первой биометрике принято решение об идентичности 
образцов (Accept), так как результат сравнения $m_1$ превышает определенный 
порог ~$A_1$;
\item 4~--- при сравнении по первой биометрике принято решение о различности 
образцов (Reject), так как мера сходства $m_1$ меньше некоторого минимального 
порога~$R_1$;
\item 5~--- после сравнения по второй биометрике принято решение об идентичности 
образцов, что суммарная мера сходства $m_2$  больше порога  заданного 
порога~$A_2$, проблемы построения
интегральной меры сходства при 
мульти\-био\-мет\-ри\-че\-ской идентификации изложены в~[11--15];
\item 6~--- по результатам сравнения принято решение о различности образцов.
\end{itemize}

     Оценим статистические характеристики временных показателей выполнения 
функции мультибиометрического сравнения. Следует разделить следующие два случая:
     \begin{enumerate}[(1)]
\item образцы принадлежат одному человеку (обозначим время сравнения $t^g$);
\item образцы принадлежат разным людям ($t^i$). 
\end{enumerate}

     В первом случае вероятности события $m_1\geq A_1$ и вероятность $m_1<R_1$ 
являются стандартными показателями качества распознавания и обозначаются 
TAR($A_1$), True Acceptance Rate, и $\mathrm{FRR}(R_1)\;=$ %\linebreak
 $=\;1- \mathrm{TAR}(R_1)$, 
False Rejection 
Rate, ошибка первого рода. Среднее время сравнения <<своих>>\ \ будет %\linebreak \vspace*{-12pt}

\end{multicols}

\begin{figure} %fig4
     \vspace*{1pt}
\begin{center}
\mbox{%
\epsfxsize=133.51mm
\epsfbox{ush-4.eps}
}
\end{center}
\vspace*{-9pt}
\Caption{Реализации функции мультибиометрического сравнения
\label{f4ush}}
%\end{figure*}
%\begin{figure*} %fig5
\vspace*{12pt}
\begin{center}
\mbox{%
\epsfxsize=138.849mm
\epsfbox{ush-5.eps}
}
\end{center}
\vspace*{-9pt}
\Caption{Ввероятности перехода: (\textit{а})~свои сравнения; (\textit{б})~чужие 
сравнения
\label{f5ush}}
\end{figure}

\begin{multicols}{2}


\noindent
складываться из 
двух слагаемых: времени выполнения сравнения по первой биометрике и времени
сравнения по второй биометрике. Причем сравнение отпечатков пальцев будет 
проводиться только в случае $R_1\leq m_1 <A_1$. Вероятности переходов схематично 
представлены на рис.~\ref{f5ush}. 

   
Соответственно среднее суммарное время определяется по формуле:
     \begin{multline}
     t^g = t_1^g + (1-\mathrm{TAR}_1(A_1)-\mathrm{FRR}_1(R_1))t_2^g ={}\\
     {}= t_1^g 
+(\mathrm{FRR}_1(A_1)-\mathrm{FRR}_1(R_1))t_2^g\,,
     \label{e4ush}
     \end{multline}
где $t_1^g$~--- среднее время выполнения <<своих>> сравнений для первой биометрики; 
$t_2^g$~---  среднее время выполнения <<своих>> сравнений для второй био\-мет\-рики.

     Во втором случае вероятности $m_1\geq A_1$ и $m_1\;<$\linebreak $<\;R_1$ выражаются 
аналогичным образом \mbox{через} ошибку второго рода FAR($A_1$), False Acceptance 
     Rate~--- ошибка второго рода, и $\mathrm{TRR}(R_1)=1\;-$\linebreak
     $-\;\mathrm{FAR}(R_1)$, True Rejection 
Rate. Среднее суммарное время определяется как
     \begin{multline}
     t^i = t_1^i +(1 - \mathrm{FAR}_1(A_1) -\mathrm{TRR}_1(R_1)) t^i_2 = {}\\
     {}=
t_1^i+(\mathrm{FAR}_1(R_1) -\mathrm{FAR}_1(A_1))t_2^i\,,
     \label{e5ush}
     \end{multline}
где $t_1^i$~--- среднее время выполнения <<чужих>> сравнений для изображения 
лица; $t_2^i$~--- среднее время выполнения <<чужих>> сравнений для отпечатков 
пальцев.
     
     Таким образом, получаем, что время сравнения~(\ref{e5ush}) очевидно меньше 
времени сравнения при независимых процессах сравнения. Более того, при 
определенных соотношениях $t_1^i$ и $t_2^i$ время~(\ref{e5ush}) меньше времени 
$t_2^i$ сравнения по одной биометрики. Данный результат может применяться, 
например, для ускорения дактилоскопической идентификации~\cite{2ush}. 
    
     \section{Определение потребности в~вычислительных средствах}
     
     В задаче массовой идентификации, когда размер 
базы достаточно велик, одного сервера или ПЭВМ недостаточно. Поэтому требуется 
решение задачи определения вычислительной мощности. 
     
     Массовость сравнения позволяет при оценке производительности в значительной 
степени ориентироваться на формулы~(\ref{e4ush}) и~(\ref{e5ush}), так как при 
сравнении по большой базе входящие в формулу вероятности дают достаточно точную 
оценку времени идентификации. При выполнении операции массовой идентификации 
доминируют операции сравнения <<чужих>>. Так, если в базе зарегистрированы по 
одному образцу для $N$ субъектов, то в процессе идентификации будет выполнено 
$N$ <<чужих>> и 1 <<свое>> сравнение. При большом $N$ доля времени $t^g$ в 
процессе идентификации ничтожно мала, ей можно пренебречь. Таким образом из 
формулы~(\ref{e1ush}) получаем, что требуемая мощность вычислительных средств 
составляет
     \begin{equation}
     W = \fr{Nt_{\mathrm{ср}}}{t}\,.
     \label{e6ush}
     \end{equation}
     
     С точки зрения производительности схема интеграции с зависимыми процессами 
сравнения заведомо медленнее одномодальной биометрической системы на базе первой 
технологии и заведомо быст\-рее бимодальной с независимыми процессами сравнения. 
Как было отмечено выше, при определенных условиях бимодальная система с 
зависимыми процессами сравнения может быть производительней одномодальной с 
использованием второй биометрики (важный пример: интеграция в АДИС изображения 
лица). Если дополнительная биометрика дает существенный выигрыш в 
производительности, то ее использование целесообразно. Если выигрыша нет, то 
требуются дополнительные аргументы в пользу мультибиометрии.
     
     Для оценки выигрыша в производительности введем обозначения для 
относительного изменения времени сравнения $a(A_1,R_1)$. На одном 
вы\-чис\-литель\-ном узле относительное изменение  произ\-во\-ди\-тель\-ности при зависимых 
процессах идентификации с выбранными порогами $A_1$ и~$R_1$ можно %\linebreak
 вы\-чис\-лить 
как отношение~(\ref{e5ush}) к времени идентификации $t_1^i+t_2^i$ при 
независимых процессах идентификации:

\noindent
     \begin{multline}
     a(A_1,R_1) = \fr{t^i}{t_1^i+t_2^i} ={}\\
     {}= \fr{t^i_1+(\mathrm{FAR}_1(R_1)-
\mathrm{FAR}_1(A_1))t_2^i}{t_1^i+t_2^i}\,.
     \label{e7ush}
     \end{multline}
     
     Значение $a(A_1,R_1)$ изменяется в пределах от  $t_1^i/(t_1^i+t_2^i)$ до~1.
     
     При измерении ускорения относительно одномодальной системы 
выражение~(\ref{e7ush}) модифицируется следующим образом:
     \begin{equation}
     a(A_1,R_1) =\fr{t^i}{t_2^i} =\fr{t_1^i}{t_2^i}+(\mathrm{FAR}_1(R_1)-
\mathrm{FAR}_1(A_1))\,.
     \label{e8ush}
     \end{equation}
     
     Выражения~(\ref{e7ush}) и~(\ref{e8ush}) применимы, когда процессы физически 
функционируют на одних и тех же вычислительных средствах и выполняются 
последовательно. Но во многих случаях для различных биометрик потребуются 
отдельные вычислительные средства (обозначим их долю через $g_i$). В таком случае 
можно считать, что в продолжительном интервале времени процессы идентификации 
исполняются параллельно со средним временем сравнения $t_1^i/g_1$ для первой 
биометрики и $t_2^i/g_2$ для второй в случае независимых процессов сравнения. 
Среднее время выполнения определяется самым медленным звеном, т.\,е.\ итоговое 
среднее время сравнения равно $\max \left ( t_1^i/g_1,t_2^i/g_2\right )$. При зависимых 
процессах на вторую биометрику приходится меньшая вычислительная нагрузка, 
поскольку до сравнения по второй биометрике доходит в среднем только 
$(\mathrm{FAR}_1(R_1)-\mathrm{FAR}_1(A_1))$ образцов.
     
     Распределение вычислительных мощностей $g_1$ и $g_2$ (балансировка 
кластера) для каждой био\-мет\-ри\-ки может производиться из следующего основного 
соображения. Суммарное время выполнения идентификации по каждой из биометрик 
должно быть одинаковым на достаточно длинном интервале (например, сутки). Если 
одна из биометрик систематически отрабатывает быстрее, то часть ее мощностей 
целесообразно отдать второй.
     
     При независимых процессах сравнения вы\-чис\-ли\-тельные мощности находятся из 
соотношений (первое уравнение является условием равной загруженности биометрик, 
второе~--- условием того, что все мощности распределены):
     \begin{equation}
     \begin{cases} 
     &\fr{t_1^i}{g_1} -\fr{t_2^i}{g_2}  =0\,;\\
     &g_1+g_2  =1\,.
     \end{cases}
     \label{e9ush}
     \end{equation}
     
     \begin{figure*} %[b] %fig6
     \vspace*{1pt}
\begin{center}
\mbox{%
\epsfxsize=134.676mm
\epsfbox{ush-6.eps}
}
\end{center}
\vspace*{-9pt}
\Caption{Распараллеливание биометрических вычислений
\label{f6ush}}
\end{figure*}
    
\noindent
В таком случае получаем
     \begin{equation}
     g_1  = \fr{t_1^i}{t_1^i+t_2^i}\,;\quad 
     g_2  = \fr{t_2^i}{t_1^i+t_2^i}\,.
          \label{e10ush}
     \end{equation}
     
     При зависимых процессах распределение вы\-чис\-лительных мощностей 
определяется как
     \begin{equation}
     \begin{cases}
     &\fr{t_1^i}{g_1} -\left (\mathrm{FAR}_1(R_1)-\mathrm{FAR}_1(A_1)\right 
)\fr{t_2^i}{g_2} =0\,;\\
     &g_1+g_2=1\,.
     \end{cases}
     \label{e11ush}
     \end{equation}
Следовательно,
     \begin{align*}
     g_1 & = \fr{t_1^i}{t_1^i+(\mathrm{FAR}_1(R_1)-
\mathrm{FAR}_1(A_1))t_2^i}\,;\\
     g_2 & = \fr{(\mathrm{FAR}_1(R_1)-
\mathrm{FAR}_1(A_1))t_2^i}{t_1^i+(\mathrm{FAR}_1(R_1)-
\mathrm{FAR}_1(A_1))t_2^i}\,.
     \end{align*}
     
     Изменение производительности при параллельной организации вычислений в 
точности соответствует формулам~(\ref{e7ush}) и~(\ref{e8ush}), выведенных в 
предположении о функционировании двух биометрик на одних аппаратных средствах. 
При дефиците памяти распределение различных био\-мет\-рик по различным 
аппаратным средствам является пред\-почтительным. Соответственно при определении 
вычислительной нагрузки следует руководствоваться следующим алгоритмом:
     \begin{itemize}
\item вычисление необходимого времени идентификации на основе размера 
базы, интенсивности потока заявок и требуемой избыточности методами 
разд.~2 с учетом поправок~(\ref{e6ush}) и~(\ref{e7ush}) на 
производительность;
\item вычисление доли серверов, специализиру\-ющих\-ся на каждом из 
биометрических методов, по формулам~(\ref{e8ush})--(\ref{e11ush}).
     \end{itemize}
     
     При большем числе биометрик следует руководствоваться тем же принципом 
равной нагрузки.
     
     \section{ Конфигурация кластера}

 \vspace*{-6pt}    

     При значительном чис\-ле зарегистрированных пользователей био\-мет\-ри\-че\-ской
     сис\-те\-ме необхо\-ди\-мо более одного сервера.
     Это требует решения проблемы распараллеливания вы\-чис\-ле\-ний.

     
     Возможны два принципиально методологически отличных подхода к данной 
проблеме. Первый заключается в распараллеливании операций био\-мет\-ри\-че\-ской 
идентификации. В таком случае организацию параллельных вычислений можно 
переложить на штатные средства кластера. Однако такой подход практически не 
используется. Во-первых, качество распараллеливания отдельных операций вряд ли 
будет высоким. Во-вторых, биометрическая идентификация является процессом с 
высокой внутренней степенью параллелизма, который следует максимально 
использовать. 
     
     Второй подход к организации параллельных вычислений заключается в 
комплектовании био\-мет\-ри\-че\-ской системы специализированными средствами 
управления параллельными вы\-чис\-ле\-ниями. 
   {\looseness=1
   
   }
   
     Базовая идея организации параллельных вы\-чис\-ле\-ний приведена на 
рис.~\ref{f6ush}. В самой простой ситуации каждый сервер можно ассоциировать с 
одним хранимым биометрическим образцом.  Тогда идентификация будет 
осуществляться следующим образом. Предъявляемый образец сравнивается с каждым 
хранимым на отдельном сервере. В таком случае без учета потерь на передачу время 
идентификации равно времени сравнения пары образцов.%\linebreak

\end{multicols}

%\end{document}


\begin{figure} %fig7
\vspace*{1pt}
\begin{center}
\mbox{%
\epsfxsize=103.684mm
\epsfbox{ush-7.eps}
}
\end{center}
\vspace*{-6pt}
     \Caption{Топология мультибиометрического кластера (вариант отдельных 
вертикалей)
     \label{f7ush}}
     \vspace*{12pt}
     \end{figure}
               \begin{figure} %fig8
          \vspace*{12pt}
\begin{center}
\mbox{%
\epsfxsize=103.684mm
\epsfbox{ush-8.eps}
}
\end{center}
\vspace*{-9pt}
     \Caption{Топология мультибиометрического кластера (вариант репликации)
     \label{f8ush}}
          %\vspace*{6pt}
          \end{figure}
     
     \begin{multicols}{2}
     
\noindent
В общем случае при $n$ 
серверах на каждом можно размещать $1/n$ всей базы. 

          Помимо вычислительных серверов и сер\-веров управления в конфигурации 
кластера требуется предусмотреть специализированные аппаратные средства для 
мониторинга и управления функционированием, хранения биометрической базы, 
технологические серверы (доменные контроллеры, серверы авторизации) и~пр. 
Определение требований и потребностей в данных категориях выходят за рамки данной 
статьи. 

\begin{table*}\small %tabl1
%\vspace*{-6pt}
\begin{center}
\Caption{Эксплуатационные показатели современных биометрических технологий
\label{t1ush}}
\vspace*{2ex}

\begin{tabular}{|l|c|c|c|c|}
\hline
\multicolumn{1}{|c|}{Технология}&\tabcolsep=0pt\begin{tabular}{c}Скорость\\ сравнения {$t^{-1}_{\mathrm{ср}}$}$^*$,\\ 
сравнений/с \end{tabular} &
\tabcolsep=0pt\begin{tabular}{c}Размер\\ шаблона,\\ байт\end{tabular}&
\tabcolsep=0pt\begin{tabular}{c}Размер\\ образца,\\ КБ\end{tabular}&
\tabcolsep=0pt\begin{tabular}{c}Время\\ создания\\ шаблона$^{*}$,\\ с\end{tabular}\\
\hline
\multicolumn{5}{|c|}{\textbf{Отпечатки пальцев}}\\
\hline
Biolink ATK &\hphantom{9}9\,000&600--3000&100 &0,2\\
\cline{1-3}
\cline{5-5}
Neurotechnologija MegaMatcher&10\,000&600--3000&&0,3\\
\hline
\multicolumn{5}{|c|}{\textbf{Изображение лица}}\\
\hline
Neurotechnologija 
VeriLook&60\,000&3000&200&0,6\\
\hline
\multicolumn{5}{|c|}{\textbf{Радужная оболочка глаза}}\\
\hline
Biolink Idenium&200\,000\hphantom{9}&3400 &150&0,1\\
\hline
\multicolumn{5}{|c|}{\textbf{Голос}}\\
\hline
ЦРТ Трал&1 &55\,296&N/A&5\hphantom{0,}\\
\hline
\multicolumn{5}{p{100mm}}{\footnotesize $^*$В расчете на одно 
ядро тактовой частотой 1~ГГц.}
\end{tabular}
\end{center}
\end{table*}

     
     С точки зрения организации вычислительного кластера можно выделить две 
основные топологии кластера: организация нескольких независимых вертикалей 
(рис.~\ref{f7ush}) и репликация мультибиометрических кластеров меньшего размера 
(рис.~\ref{f8ush}). 
     

     С точки зрения требований к функционированию каждый из вариантов имеет 
свои преимущества и недостатки. Отдельные вертикали более удобны в профилактике 
и модернизации и больше подходят для систем с независимыми процессами 
сравнения. Объединение более аналогичных сис\-тем в рамках большей более 
перспективно в задачах, связанных с интеграцией биометрической информации из 
различных ведомств или территорий. 
 {\looseness=1
 
 }    
     
     При применении обеих перечисленных топологий в задаче идентификации с 
зависимыми процессами сравнения возникает значимая проблема с балансировкой 
нагрузки, так как отобранные при сравнении первой биометрики образцы 
необязательно будут равномерно распределены по аппаратным средствам второй 
биометрики. В таком случае может потребоваться дополнительная избыточность. 


     
     \section{Пример использования}
     
     В качестве примера рассмотрим часть процесса проектирования системы 
мультибиометрической идентификации по отпечаткам пальцев и изображению лица. 
Данный пример имеет важное практическое значение, потому что такая комбинация 
биометрических идентификаторов является основой многих современных 
высокопроизводительных биометрических систем. Как видно из табл.~\ref{t1ush}, 
идентификация по изображению лица является более быстрой по сравнению даже с 
одним отпечатком %\linebreak
 пальца, не говоря уже о более трудоемких четырех- и 
десятипальцевых дактокартах. В то же время отпечатки пальцев являются более 
надежной био\-мет\-ри\-ей с точки зрения качества идентификации. %\linebreak
 Такое соотношение 
про\-из\-во\-ди\-тель\-ность--на\-деж\-ность идентификации делает привлекательным совместное 
использование этих биометрик. При этом распознавание по изображению лица будет 
первой технологией предварительного поиска, окончательное решение принимается на 
основе идентификации по отпечаткам пальцев. 
     
     Схематично идентификация представлена на рис.~\ref{f9ush}. Размеры $A_1$, 
$R_1$, $A_3$, $R_3$ определяются вероятностями перехода в соответствующие состояния 
графа (рис.~\ref{f9ush}). Априори, в связи с высокой надежностью бимодальной 
биометрии по дактокарте и изображению лица можно считать, что переходы из 
узла~\textit{3} выполняются без ошибок, и основной \mbox{целью} усиления АДИС лицевой 
биометрией является увеличение производительности.
     

     Учитывая реальные требования к современным системам биометрической 
идентификации  вероятность перехода в узел~\textit{2} (рис.~\ref{f9ush}) будет очень 
редким событием, поскольку на ошибку второго рода накладываются серьезные 
ограничения $\mathrm{FAR}_{\mathrm{face}}(A_1)<10^{-3}\div 10^{-8}$, из которого 
определяется допустимое значение порога~$A_1$. Соответственно в чужих сравнениях 
вероятность перехода прак\-тиче\-ски нулевая и не влияет на среднее \mbox{время}. Поэто\-му 
следует отметить, что в боль\-шинст\-ве приложений при оптимизации 
про\-из\-во\-ди\-тель\-ности %\linebreak 
придет\-ся отказаться от возможности принятия положительного 
решения на основе идентификации только по изображению лица, так как прирост 
минимальный, а риски неправильной оценки FAR$_{\mathrm{face}}$ высоки. 
В~то же время порог~$R_1$ будет регулироваться другим ограничением: максимально 
допустимым уровнем ошибки 1-го рода, $\mathrm{FRR}_{\mathrm{face}} (R_1)\approx 
0$. 
{%\looseness=1

}

В качестве конкретного примера определения конфигурации серверов выберем 
следующие пока-

\end{multicols}

\begin{figure} %fig9
     \vspace*{1pt}
\begin{center}
\mbox{%
\epsfxsize=144.976mm
\epsfbox{ush-9.eps}
}
\end{center}
\vspace*{-9pt}
\Caption{Идентификация по лицу  и отпечаткам пальцев
\label{f9ush}}
%\end{figure*}
 \vspace*{6pt}  
 \renewcommand{\figurename}{\protect\bf Таблица}
 \setcounter{figure}{1}
%\renewcommand{\tablename}{\protect\bf Таблица}  
         % \begin{figure} %\small
 %    \vspace*{2pt}
     \begin{center}
     \Caption{Модельные параметры мультибиометрического кластера
     \label{t2ush}}
     \vspace*{2ex}
     
     {\small
     \begin{tabular}{|l|l|}
     \hline
\multicolumn{1}{|c|}{Параметр}&\multicolumn{1}{c|}{Значение}\\
\hline
Размер шаблона дактокарты&20~Кб\\
Скорость сравнения дактокарт&5000 сравнений/с\\
Размер шаблона изображения лица&2 Кб\\
Скорость сравнения изображения лица &250\,000 сравнений/с\\
Размер базы (записей)&20\,000\,000\\
Суточный поток заявок с учетом избыточности&60\,000 \\
FAR$_{\mathrm{face}} (R_1)-\mathrm{FAR}_{\mathrm{face}}(A_1)$ &10\%\\
\hline
\end{tabular}}
\end{center}
\vspace*{3pt}
\end{figure}

\renewcommand{\figurename}{\protect\bf Рис.}
%\renewcommand{\tablename}{\protect\bf Таблица}

\begin{multicols}{2}

\noindent     
затели стационарного режима\footnote{Режим, в котором размер базы и поток 
заявок достигли своего максимального штатного значения.} (табл.~\ref{t2ush}), с 
параметрами российских биометрических технологий Biolink (отпечатки пальцев) и 
IIT (изображение \mbox{лица}).  
      
     Получаем, что всего за сутки требуется про\-из\-вес\-ти $60\cdot 10^3\cdot 20\cdot 10^6 
=1{,}2\cdot 10^{12}$ сравнений мультибиометрических образцов. Среднее время 
сравнения 
     \begin{multline*}
     t^i_{\mathrm{face}} +t^i_{\mathrm{finger}} 
(\mathrm{FAR}_{\mathrm{face}}(R_1)-\mathrm{FAR}_{\mathrm{face}}(A_1)) ={}\\
{}=
     \fr{1}{250\,000}+0{,}10\fr{1}{5000}=\fr{1+0{,}10\cdot 
50}{250\,000}=\fr{6}{250\,000}\,.
     \end{multline*}
     
     Скорость сравнения равна примерно 41\,600~сравнений/с (или $3{,}6\cdot 
10^9$~срав\-не\-ний/сут), %\linebreak 
что значительно выше скорости сравнения по %\linebreak 
отпечаткам пальцев. 
Суммарная потребность в вы\-чис\-ли\-тель\-ных мощностях равна: $1{,}2\cdot 
10^{12}/(3{,}6\cdot 10^9)=$ 
$= 0{,}333\cdot 10^3\sim 333$~ядра, при четырехъядерном процессоре~--- 
84~сервера.
     
     На лицо приходится 1/6 вычислительных ресурсов, на отпечаток пальца~--- 5/6. 
Всего 14~лицевых серверов и 70~дактилоскопических. Суммарный объем адресуемой 
памяти лицевых серверов не менее 100~ГБ, дакто~--- 504~ГБ (при 64-битной 
адресации). Общая потребность в памяти: 40~ГБ на лицевую биометрию и 400~ГБ на 
дактилоскопию (в частности, лицевая биометрия допускает использование 32-бит\-ных 
версий). 
     
     При таком составе вычислительного узла возможны две крайние конфигурации: 
репликация минимальной конфигурации 1~лицевой сервер и 5~дактосерверов (всего 
14~комплектов) или две вертикали: 14~лицевых серверов и 70~дактосерверов. Данные 
конфигурации различаются с точки зрения организации вычислительного кластера. 
При этом с точки зрения балансировки нагрузки они идентичны. Плановая загрузка 
вычислительных мощностей при максимальной нагрузке в 60\,000~за\-про\-сов/сут 
составляет 99,2\%. Данное значение получается как соотношение плановой нагрузки в 
$1{,}2\cdot 10^{12}$ сравнений/сут и максимальной производительности лицевого 
кластера ($1{,}2096\cdot 10^{12}$ сравнений/с) и дактилоскопического кластера 
($1{,}2096\;\times$\linebreak $\times\;10^{11}$). Дактилоскопический кластер в приведенной конфигурации 
планово выполняет только 10\% сравнений, оставшихся после лицевой био\-мет\-рии, 
т.\,е.\ всего $1{,}2\cdot 10^{11}$ сравнений/сут.
     
     Столь высокая нагрузка в модельном примере определяется оптимальным 
соотношением памяти и производительности. При снижении плановой загрузки до 
30\,000 запросов/сут произойдет эффект избытка памяти. При размере базы в 20~млн 
записей необходимо иметь минимум 56~серверов дактобиометрии для поддержки 
необходимых 400~ГБ адресуемой памяти. В таком случае конфигурация кластера будет 
далека от оптимальной (7~лицевых серверов и 56~дактилоскопических). Плановая 
нагрузка составит 99\% для лицевых серверов и 62\% для дактилоскопических 
соответственно. 
     
     \section{Заключение}
     
     В статье выработан подход к распараллеливанию вычислений в 
мультибиометрической системе. В качестве преимуществ данного подхода можно 
отметить следующее:
     \begin{itemize}
\item эффективное распараллеливание биометрических вычислений;
\item разделение вычислительных мощностей между различными биометриками;
\item оптимизация производительности мультибиометрического кластера;
\item возможность регулирования соотношения производительность~--- ошибка 
первого рода;
\item возможность оптимального выбора конфигурации кластера в случае 
распределенных вы\-чис\-лений.
\end{itemize}

     С практической точки зрения разработанный подход позволяет проводить 
быструю оценку конфигурации мультибиометрической системы на этапе эскизного 
проектирования.
     
     
     {\small\frenchspacing
{%\baselineskip=10.8pt
\addcontentsline{toc}{section}{Литература}
\begin{thebibliography}{99}

\bibitem{1ush}
\Au{Ushmaev~O.\,S., Novikov~S.\,O.} 
Integral criteria for large-scale multiple fingerprint solutions~// Biometric technology for human 
identification~/ Edited by A.\,K.~Jain, N.\,K.~Ratha. SPIE Proceedings.  
Berlingham, WA: SPIE, 2004. Vol.~5404. P.~534--543.

\bibitem{2ush}
\Au{Ушмаев~О.\,С.}
Информационная технология интеграции идентификации по изображению лица для ускорения 
автоматической дактилоскопической идентификации~// Информатика и её применения, 2008. Т.~2. Вып.~4.
С.~66--73.

\bibitem{5ush} %3
\Au{Болл Р.\,М., Коннел~Дж.\,Х., Панканти~Ш., Ратха~Н.\,К., Сеньор~Э.\,У.} 
Руководство по био\-мет\-рии.~--- М.: Техносфера, 2007.

\bibitem{3ush} %4
\Au{Ушмаев О.\,С.}
Концепция муль\-ти\-био\-мет\-ри\-че\-ской идентификации в информационно-аналитических сис\-те\-мах~// 
Паспортные и правоохранительные сис\-те\-мы  2008, Интерполитех-2008. 
{\sf http://www.dancom. ru/rus/AIA/Archive/RUXIX-IPIRAN-Ushmaev-MultimodalBiometricsFramework.ppt}. 

\bibitem{4ush} %5
\Au{Ушмаев О.\,С., Босов~А.\,В.}
Реализация концепции многофакторной био\-мет\-ри\-че\-ской 
идентификации в интегрированных аналитических системах~// Бизнес и безопасность в России, 2008. 
№\,49. С.~104--105.

\bibitem{6ush}
\Au{Синицын~И.\,Н., Губин~А.\,В., Ушмаев~О.\,С.} 
Метрологические и биометрические технологии и 
сис\-те\-мы~// История науки и техники, 2008.  №\,7. С.~41--44.

\bibitem{7ush}
\Au{Dizard III, Wilson~P.}
FBI plans major database upgrade~// Government Computer News. Available at
{\sf http://www.gcn.com/print/25\_26/41792-1.html? page=1}.


\bibitem{9ush} %8
\Au{Ушмаев~О.\,С., Синицын~И.\,Н.} 
Опыт проектирования многофакторных био\-мет\-ри\-че\-ских систем, 
Труды \mbox{VIII} международной научно-технической конференции <<Кибернетика и высокие 
технологии XXI~века>>, 2007. Т.~1. С.~17--28.

\bibitem{8ush} %9
\Au{Ушмаев~О.\,С.}
Сервисно-ориентированный подход к разработке муль\-ти\-био\-мет\-ри\-ческих 
технологий~// Информатика и её применения, 2008.  Т.~2. Вып.~3. С.~41--53.

\bibitem{10ush}
FBI~--- next generation identification. {\sf http://www.fbi. gov/hq/cjisd/ngi.htm}.

\bibitem{11ush}
\Au{Синицын И.\,Н., Новиков~С.\,О., Ушмаев~О.\,С.}
Развитие технологий интеграции био\-мет\-ри\-че\-ской 
информации~// Системы и средства информатики, 2004. Вып.~14. С.~5--36.

\bibitem{12ush}
\Au{Ushmaev~O., Novikov~S.} 
Biometric fusion: Robust approach~// MMUA~06 Proceedings. Toulouse, France. 
May 11--12, 2006.  

\bibitem{17ush} %13
Biometric fusion: Robust approach~// MMUA~06 Proceedings. Toulose, France. May 11--12, 2006.

\bibitem{13ush} %14
\Au{Ушмаев~О.\,С., Босов~А.\,В.} 
Реализация концепции многофакторной био\-мет\-ри\-че\-ской 
идентификации в интегрированных аналитических системах~// Системы высокой доступности, 4, 
2007. Т.~3. С.~13--23.

\label{end\stat}

\bibitem{14ush} %15
\Au{Ушмаев О.\,С., Синицын~И.\,Н.}
 Программная реализация муль\-ти\-био\-мет\-ри\-че\-ской идентификации в 
интегрированных аналитических приложениях~// Труды IX международной научно-технической 
конференции <<Кибернетика и высокие технологии XXI~ века>>. Воронеж, 13--15~мая 2008. Т.~2. 
С.~735--746.


\end{thebibliography}
}
}
\end{multicols}