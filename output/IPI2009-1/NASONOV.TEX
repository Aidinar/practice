\def\stat{nasonov}

\def\tit{РАЗВИТИЕ МЕТОДОВ ПОВЫШЕНИЯ КАЧЕСТВА ИЗОБРАЖЕНИЙ ЛИЦ В~ВИДЕОПОТОКЕ$^*$}
\def\titkol{Развитие методов повышения качества изображений лиц в~видеопотоке}

\def\autkol{А.\,В.~Насонов, А.\,С.~Крылов, О.\,С.~Ушмаев}
\def\aut{А.\,В.~Насонов$^1$, А.\,С.~Крылов$^2$, О.\,С.~Ушмаев$^3$}

\titel{\tit}{\aut}{\autkol}{\titkol}

{\renewcommand{\thefootnote}{\fnsymbol{footnote}}\footnotetext[1]
{Работа поддержана грантами РФФИ (проекты 09-07-00173 и 09-01-00703) и Программой ОНИТ РАН
<<Информационные технологии и методы анализа сложных систем>>.}}

\renewcommand{\thefootnote}{\arabic{footnote}}
\footnotetext[1]{Московский государственный университет им. М.\,В.~Ломоносова, факультет ВМиК,
nasonov@cs.msu.ru}
\footnotetext[2]{Московский государственный университет им. М.\,В.~Ломоносова, факультет ВМиК,
kryl@cs.msu.ru}
\footnotetext[3]{Институт проблем информатики Российской академии наук,
oushmaev@ipiran.ru}


\Abst{В работе предложен метод улучшения качества изображений лиц на видео, основанный на
использовании общего метода суперразрешения. Суперразрешение моделируется как обратная
задача к задаче понижения разрешения, т.\,е.\ осуществляется поиск изображения, которое,
будучи уменьшенным с учетом движения, дает минимальное суммарное квадратичное
отклонение от исходных изображений низкого разрешения. Рассмотрены качественный и
быстрый варианты метода суперразрешения. Предложен специальный метод подавления
эффекта Гиббса для метода быстрого суперразрешения. Предложен новый многомасштабный
метод оценки движения для последовательных кадров.}

\KW{суперразрешение; подавление эффекта Гиббса; изображения лиц; многомасштабный
метод оценки движения; быстрый метод}

\vskip 30pt plus 9pt minus 6pt

      \thispagestyle{headings}

      \begin{multicols}{2}

      \label{st\stat}


     \section{Введение}

     В настоящее время активно развиваются информационные технологии, связанные
с интеллектуальной обработкой видеоинформации~[1--8]. Одним из наиболее
динамично развивающихся\linebreak
на\-прав\-ле\-ний интеллектуальной обработки является анализ
динамических последовательностей лица (видеоряда)~[9, 10] с целью последующей
идентификации.

     Существенной особенностью идентификации по видеоряду является то, что на
этапе регистрации используется один источник (статичное фото, фоторобот и~т.\,д.), в
то время как при идентифи\-кации доступна информация из принципиально другого\linebreak
источника (видеокамера). Последний отличается, с одной стороны, худшим качеством
изоб\-ра\-же\-ния и потенциальной некооперативностью человека. С~другой стороны, при
относительно\linebreak длительном нахождении в кадре лицо человека меняет ракурс, что
теоретически устраняет значимую проблему идентификации человека по изображению
лица, а именно: стремительную деградацию качества распознавания при различных
ракурсах и низком разрешении.

     На практике при анализе видео используют два принципиально различных
подхода. Первый заключается в использовании длины видеоряда. А~именно,
современные библиотеки распознавания по изоб\-ра\-же\-нию лица позволяют сравнивать
изображения низкого разрешения, но с значительными ошибкам распознавания.
Однако длина видеоряда позволяет в определенной степени компенсировать
негативные факторы за счет массового сравнения (рис.~\ref{f1nas}).


      Второй подход заключается в интеграции до этапа сравнения (рис.~\ref{f2nas}) с
целью приблизить качество видеокадра к уровню статической фотографии. А~именно,
интеграция происходит на уровне выделения признаков, влияющих на качество
распознавания. Например, по длинной серии изоб\-ра\-же\-ний можно определить
положение зрачков с шагом, значительно меньшим размер пикселя камеры.


Наиболее перспективным направлением в рамках второго подхода является
применение методов суперразрешения, которые позволяют получить изображение
высокого разрешения на основе серии  изображений низкого разрешения. Такой подход
имеет следующие преимущества. 

Во-первых, он решает ряд проблем
автоматизированной идентификации. Например, около 90\% данных видеозаписи не
пригодны для фотопортретной экспертизы~[11]. Технологии offline и online повышения
разрешения
 видеоизображения могут обеспечить определенный прогресс в этом
на\-прав\-ле\-нии. 

\end{multicols}

\begin{figure} %fig1
\vspace*{1pt}
\begin{center}
\mbox{%
\epsfxsize=129.299mm
\epsfbox{nas-1.eps}
}
\end{center}
\vspace*{-9pt}
\Caption{Первый подход в использовании нескольких кадров в задаче распознавания:
интеграция результатов сравнений
\label{f1nas}}
%\end{figure*}
\vspace*{18pt}
%\begin{figure*} %fig2
%\vspace*{1pt}
\begin{center}
\mbox{%
\epsfxsize=129.3mm
\epsfbox{nas-2.eps}
}
\end{center}
\vspace*{-9pt}
\Caption{Второй подход в использовании нескольких кадров в задаче распознавания:  \label{f2nas}
сравнение результата интеграции
}
\vspace*{6pt}
\end{figure}


\begin{multicols}{2}


Во-вторых, он имеет самостоятельную ценность и может быть
интегрирован в произвольные видеосистемы. 

В-третьих, использование такого подхода
позволяет улучшить качество идентификации по изоб\-ра\-жению лица.

     В разд.~2 дано общее описание методов суперразрешения. В разд.~3 описывается
математическая модель задачи суперразрешения. В данной работе предлагаются
методы качественного и быстрого суперразрешения, описанные соответственно в
разд.~4 и~5. Метод подавления эффекта Гиббса для быстрого суперразрешения описан
в разд.~6. Новый алгоритм определения векторов движения при решении задачи
суперразрешения по видеоданным приведен в разд.~7. Результаты и сравнение качества
предложенных методов проиллюстрированы в разд.~8. В заключении представлены
основные выводы.

     \section{Суперразрешение}


     \begin{figure*} %fig3
\vspace*{1pt}
\begin{center}
\mbox{%
\epsfxsize=153.319mm
\epsfbox{nas-3.eps}
}
\end{center}
\vspace*{-9pt}
      \Caption{Соответствие между пикселями изображений низкого разрешения
(вверху) и изображения высокого разрешения (внизу)
      \label{f3nas}}
%      \vspace*{3pt}
      \end{figure*}

     В отличие от интерполяции и увеличения (re-sampling) изображений,
повышающих разрешение, но не вносящих в изображение новой информации,
суперразрешение использует информацию сразу с~нескольких изображений, поэтому
резуль\-ти\-ру\-ющее изображение высокого разрешения содержит в себе больше полезной
информации. Это поло\-жительно влияет на качество методов идентификации.

     Основной источник информации для суперразрешения~--- это изображения
одного и того же объекта, незначительно движущегося на последовательных кадрах.
Пиксели камеры, получающей изоб\-ра\-же\-ние, имеют ненулевой размер, поэтому
наблюдаемое значение пикселя соответствует не значению в конкретной точке на
реальном изоб\-ра\-же\-нии, а является усреднением по некоторой окрестности точки.
Объект смещается, как правило, на нецелое число пикселей, поэтому на разных кадрах
усреднение производится по разным окрестностям (рис.~\ref{f3nas}). Если
движение объекта известно, то можно использовать информацию со всех кад\-ров для
построения одного изображения высокого разрешения.


     В реальных ситуациях движение объекта неизвестно, поэтому его необходимо
сначала вы\-чис\-лить. Близкие друг к другу точки обычно движутся одинаково, поэтому
обычно используется предположение о том, что движение в небольшой окрестности
точки, в которой вычисляется движение, является плоскопараллельным. В этом случае
повышается точность методов оценки движения и устойчивость к шуму, но в случае
резкого изменения движения, например в случае движущегося объекта на
неподвижном фоне, метод будет давать неточные результаты.

     Задача суперразрешения обычно ставится в виде задачи минимизации: найти
такое изображение, которое, будучи уменьшенным с учетом движения, даст
минимальное суммарное квадратичное\linebreak отклонение от исходных изображений низкого\linebreak
разрешения. Эта задача является некорректно поставленной. Для ее решения
применяется ре\-гу\-ля\-ри\-зи\-ру\-ющий метод, основанный на методе регуляризации
Тихонова~[12]. При этом некорректно поставленная задача заменяется на близкую к
ней корректно поставленную задачу путем добавления ограничений.

     \section{Математическая модель}

     Перед исследованием методов суперразрешения рассмотрим обратную задачу
построения кад\-ров низкого разрешения при известном изображении высокого
разрешения $z$ (двумерная матрица интенсивности пикселей) и известных операторах
движения $F_k$ (отображение одного изображения на последующее) для каждого
кадра. Эти операторы задают соответствия между точками первого и $k$-го кадров.
При этом кадры низкого разрешения $w_k$ можно представить в следующем виде:
     \begin{equation}
w_k = A_k z=DHF_kz\,,\quad k=\overline{1,\,N}\,,
     \label{e1nas}
     \end{equation}
где $DH$ является оператором понижения разрешения. Этот оператор осуществляет
последовательное применение фильтра высоких частот~$H$, пред\-став\-ля\-юще\-го собой
свертку с фильтром Гаусса и оператора прореживания~$D$. Схематично это может
быть проиллюстрировано на рис.~\ref{f4nas}.

\begin{figure*} %fig4
\vspace*{1pt}
\begin{center}
\mbox{%
\epsfxsize=120.835mm
\epsfbox{nas-4.eps}
}
\end{center}
\vspace*{-9pt}
\Caption{Схема построения кадров низкого разрешения~(1)
\label{f4nas}}
\vspace*{6pt}
\end{figure*}

     Задача суперразрешения ставится в виде обратной задачи к задаче понижения
разрешения~(1). Таким образом, происходит поиск изображение высокого разрешения
$z$ при известных $w_k$ и~$F_k$. Задача качественного суперразрешения
формулируется в виде задачи минимизации ошибки~$z_R$
     \begin{equation}
z_R = \mathrm{arg}\,\underset{z}{\min} \sum\limits_{k=1}^N \parallel A_kz-
w_k\parallel_1\,,
     \label{e2nas}
     \end{equation}
где $\parallel z\parallel_1=\sum\limits_{i,j} \vert z_{i,j}\vert$\,.

     Решением задачи является изображение высокого разрешения, которое наиболее
близко ко всей последовательности $w_k$ одновременно.

     В качестве оценки ошибки вместо~(2) может быть использована стандартная
евклидова норма:
     \begin{equation}
     Z_R =\mathrm{arg}\,\underset{z}{\min}\sum\limits_{k=1}^N\parallel A_k-
w_k\parallel_2^2\,.
     \label{e3nas}
     \end{equation}

     \section{Качественное суперразрешение}

     Задачи минимизации (2) и~(3) являются некорректно поставленными, поэтому
для их решения необходимо использовать регуляризирующие алгоритмы. Для задачи
качественного суперразрешения~(2) мы минимизируем функционал вида
     \begin{equation}
z_R = \mathrm{arg}\,\underset{z}{\min}\left (\sum\limits_{k=1}^N\parallel A_kz-
w_k\parallel_1+\alpha\Omega(z)\right )\,.
     \label{e4nas}
     \end{equation}

     В качестве стабилизатора $\Omega(z)$ используется функционал Bilateral Total
Variation~\cite{13nas}
$$
\Omega(z) = \sum\limits_{-p\leq x,y \leq p} \gamma^{\vert x\vert
+\vert y\vert}\parallel S_{x,y}z-z\parallel_1\,,
$$
где $S_{x,y}$~--- операторы сдвига по
горизонтали на $x$ и по вертикали на $y$ пикселей соответственно, $\gamma = 0{,}8$,
$p=1$.

     Минимизация функционала~(4) производится с помощью субградиентного
метода~\cite{14nas, 15nas}.

     \section{ Быстрое суперразрешение}

     Решение задачи в постановке~(4) требует больших вычислительных затрат. Часто
является важным быстрое построение приближенного решения задачи
суперразрешения. Мы используем подход, близкий к~\cite{16nas}. В рамках такого
подхода мы заменяем регуляризирующий алгоритм обращения оператора $DH$ из
формулы~(1) на прямой оператор повышения разрешения~$U$. В качестве этого
оператора мы в данной работе используем оператор повышения разрешения,
основанный на гауссовской фильтрации~\cite{17nas}.

     Данная реализация алгоритма быстрого суперразрешения включает в себя
следующие шаги:
     \begin{enumerate}[1.]
     \item Взять первый кадр $w_1$ и вычислить движение между ним и остальными
кадрами $w_k$, $k=2, 3, \ldots , N$. Движение между $w_1$ и $w_1$ принимается
равным нулю, т.\,е.\ $z=F_1z$.
     \item Увеличить каждый кадр $w_k$ и применить оператор движения,
переводящий $k$-й кадр в первый:
     \begin{equation}
     W_k=F_k U w_k\,,
     \label{e5nas}
     \end{equation}
где $U$~--- оператор повышения разрешения, основанный на гауссовской фильтрации.
Быст\-рое вычисление~(5) описано в~\cite{17nas}.
     \item Произвести усреднение $z=(1/N)\sum\limits_{k=1}^N W_k$.
     \item Поднять резкость изображения при помощи нерезкой маски и адаптивного
фильтра~\cite{18nas}.
     \end{enumerate}

     \section{Подавление эффекта Гиббса}

     При поднятии резкости изображения в быстром суперразрешении возникает
побочный эффект, называемый эффектом Гиббса и выражающийся в появлении
ореолов возле резких границ.

     Мы используем метод, аналогичный методу подавления эффекта Гиббса после
увеличения изображений~\cite{19nas}. При этом коэффициент увеличения $s$ для
полученного методом быстрого суперразрешения изображения определяется выбором
оператора~$U$ в~(5). Основная идея метода заключается в ограничении
максимального значения функционала суммарной вариации, в дискретном двумерном
случае задаваемом в виде
     $$
     TV(z) = \sum\limits_{i,j} \vert  z_{i+1,j}-z_{i,j} \vert +
\sum\limits_{i,j} \vert z_{i,j+1}- z_{i,j}\vert\,.
     $$

     Ограничение заключается в том, что при увеличении в $s$ раз суммарная
вариация не может увеличиться больше, чем в $s$ раз. Если это условие не
выполняется, то считается проекция на множество изображений $M$ с суммарной
вариацией, удовлетворяющей этому ограничению:
     $$
z_R =\mathrm{arg}\,\underset{z\in M}{\min}\parallel z-z^*\parallel_2^2\,,
     $$
где $z^*$~--- результат интерполяции или быстрого суперразрешения. Для задачи
суперразрешения это множество задается в виде
$$
M = \left \{ z\vert TV(z)\leq \fr{1}{N}\, s\sum\limits_{k=1}^N TV(w_k)\right \}\,.
$$

     Иллюстрация применения метода приведена ниже в разд.~8.

     \section{Оценка оператора движения}

     Одной из подзадач в задаче суперразрешения является вычисление оператора~---
вектора движения точек $F_k$ для пар изображений. Результаты качественного
суперразрешения существенно зависят от его точности.

     Существует много методов вычисления векторов движения. Среди них
можно выделить группу методов, вычисляющих движение на основе совмещения
особых точек. В качестве особых точек могут использоваться, например, угловые
точки~\cite{20nas}, элементы лиц~\cite{21nas}. Эти методы не дают результата с
субпиксельной точностью, поэтому его приходится уточнять. Другая группа
     методов~--- методы, использующие вариационный подход.

     В качестве базового метода используется метод Канаде--Лукаса~\cite{22nas},
относящийся к группе вариационных методов. Этот метод использует пред\-став\-ле\-ние
изображения $J(x,y)$ через $I(x,y)$ в виде
     \begin{equation}
     I(x+u(x,y)\,,y+v(x,y)) = J(x,y)\,,
     \label{e6nas}
     \end{equation}
где $u(x,y)$ и $v(x,y)$~--- компоненты вектора движения $\vec{V} =(u,\,v)$.
Изображения $I(x,y)$ и $J(x,y)$ считаются достаточно гладкими и
диф\-фе\-рен\-ци\-ру\-емы\-ми, $I(x+u,\,y+v)$ раскладывается в ряд Тейлора с отбрасыванием
старших производных
$$
I(x+u,\,y+v)\approx I(x,y)+\fr{\partial I(x,y)}{\partial x}\,u+\fr{\partial I(x,y)}{\partial
y}\,v\,.
$$
При этом~(6) принимает вид
\begin{equation*}
I_x (x,y)u(x,y)+I_y(x,y)v(x,y)=I_t(x,y)\,,
%\label{e7nas}
\end{equation*}
где $I_t(x,y) =J(x,y)-I(x,y)$. Так как векторы движения соседних точек близки друг к
другу, то в методе Канаде--Лукаса при вычислении вектора $\vec{V}$ в конкретной
точке $(x_0,y_0)$ предполагается, что в окрестности этой точки
$$
\Omega (x_0,y_0) =
\left \{(x,y):\ (x_0-x)^2+(y-y_0)^2\leq \varepsilon^2\right \}
$$

\end{multicols}

\begin{figure*} %fig5
\vspace*{1pt}
\begin{center}
\mbox{%
\epsfxsize=158.307mm
\epsfbox{nas-5.eps}
}
\end{center}
\vspace*{-6pt}
\Caption{Иллюстрация простого и многомасштабного методов оценки движения:
(\textit{а})~пара изображений, для которых вычислялся оператор движения;
(\textit{б})~обычный метод Канаде--Лукаса; (\textit{в})~многомасштабный метод
\label{f5nas}}
\vspace*{3pt}
\end{figure*}

\begin{multicols}{2}

\noindent
вектор движения постоянен:
\begin{multline}
I_x(x,y) u(x_0,y_0)+I_y(x,y)v(x_0,y_0)=I_t(x,y)\,,\\
(x,y)\in \Omega(x_0,y_0)\,.
\label{e8nas}
\end{multline}

     Это повышает устойчивость метода, но делает невозможным резкие смены
векторов движения. Далее используется модификация метода Канаде--Лукаса,
использующая помимо самих изображений и их производные~\cite{17nas}. Это
увеличивает точность работы метода в случае переменной освещенности кадров
видеопоследовательности.

     Метод Канаде--Лукаса дает относительно точные результаты только в случае
небольших векторов движения. Для преодоления этой проблемы нами предлагается
многомасштабный поход, заключающийся в последовательном уточнении векторов
движения, предварительно найденных для уменьшенных изображений.

     Предположим, что известны векторы движения $\vec{V}_2=(u_2,v_2)$ для пары
изображений вдвое меньшего разрешения. Тогда для изображений оригинального
разрешения эти вектора движения можно представить в виде $\vec{V}^*=(u^*,v^*)$,
где $u^*(x,y)=2u_2(x/2,y/2)$, $v^*(x,y)=2v_2(x/2,y/2)$. Заменяя $u(x,y)=u^*(x,y)+\delta
u(x,y)$ и $v(x,y)\;=$\linebreak $=\;v^*(x,y)+\delta v(x,y)$, переписываем~(6) в виде
     \begin{multline}
     I(x+u^*(x,y)+\delta u(x,y)\,,y+v^*(x,y)+\delta v(x,y)) ={}\\
     {}= J(x,y)\,.
     \label{e9nas}
     \end{multline}

     Пользуясь тем, что метод Канаде--Лукаса предполагает постоянность вектора
движения в окрестности точки, в которой считается вектор движения, заменяем~(8) на
     \begin{multline*}
     I(x+\delta u(x,y),\,y+\delta v(x,y)) = {}\\
     {}=J(x-u^*(x,y),\,y-v^*(x,y))
%     \label{e10nas}
     \end{multline*}
и вычисляем $\delta u(x,y)$ и $\delta v(x,y)$ через~(7). Затем под\-став\-ля\-ем эти значения в
$u(x,y) =u^*(x,y)+\delta u(x,y)$ и $v(x,y) = v^*(x,y)+\delta v(x,y)$.

     \section{Эксперименты и обсуждение результатов}

     В этой работе приводятся результаты следующих экспериментов:
     \begin{enumerate}[1.]
     \item \textit{Сравнение простого и многомасштабного методов Канаде--Лукаса}.
Были взяты два изображения, сдвинутые друг относительно друга примерно на
20~пикселей по горизонтали и~5~по вертикали. Для этих изображений были
применены обычный и многомасштабный методы Канаде--Лукаса. Результаты их
работы приведены на рис.~\ref{f5nas}. Многомасштабный метод довольно точно
определил смешение в отличие от простого метода, который такое смещение не
обнаружил.
     \item \textit{Синтетический тест}. Было взято изображение лица, к нему
применены несколько произвольных сдвигов, затем сдвинутые изображения были
уменьшены в 4~раза. Всего было построено 16~таких изображений. К уменьшенным
изоб\-ра\-же\-ни\-ям были применены методы интерполяции, использующие информацию
только об одном изображении, и предложенные методы суперразрешения. Результаты
приведены на рис.~\ref{f6nas}. Для вычисления среднеквадратичного отклонения
использовались полученное и исходное изображения высокого разрешения.
     \item \textit{Применение метода подавления эффекта Гиббса} для метода
быстрого суперразрешения. Результаты представлены на рис.~\ref{f7nas}.
     \item \textit{Обработка реального видеопотока}. Результаты приведены на
рис.~\ref{f8nas}.
\end{enumerate}

\begin{figure*} %fig6
\vspace*{1pt}
\begin{center}
\mbox{%
\epsfxsize=138.178mm
\epsfbox{nas-6.eps}
}
\end{center}
\vspace*{-9pt}
\Caption{Результаты работы методов увеличения изображений, использующих только
один кадр, и предложенных методов суперразрешения для синтетической
видеопоследовательности: (\textit{а})~оригинал; (\textit{б})~примеры уменьшенных
изображений; (\textit{в})~бикубическая интерполяция,
$\mathrm{MSE} = 608{,}63$; (\textit{г})~интерполяция, основанная на
регуляризации~\cite{23nas}, $\mathrm{MSE} = 480{,}07$; (\textit{д})~быстрое
суперразрешение,
$\mathrm{MSE} = 232{,}76$; (\textit{е})~качественное суперразрешение,
$\mathrm{MSE} = 137{,}88$
\label{f6nas}}
\end{figure*}


\begin{figure*} %fig7
\vspace*{1pt}
\begin{center}
\mbox{%
\epsfxsize=126.924mm
\epsfbox{nas-7.eps}
}
\end{center}
\vspace*{-9pt}
\Caption{Подавление эффекта Гиббса для быстрого суперразрешения:
(\textit{а})~быстрое суперразрешение,
     $\mathrm{MSE} = 232{,}76$; (\textit{б})~быстрое суперразрешение с подавлением
эффекта Гиббса,
$\mathrm{MSE} = 197{,}81$
\label{f7nas}}
\end{figure*}


\begin{figure*} %fig8
\vspace*{1pt}
\begin{center}
\mbox{%
\epsfxsize= 111mm %119.456mm
\epsfbox{nas-8.eps}
}
\end{center}
\vspace*{-9pt}
\Caption{Применение метода суперразрешения к реальным видеоданным:
(\textit{а})~примеры исходных кадров; (\textit{б})~метод <<ближайшего соседа>>;
(\textit{в})~билинейная интерполяция; (\textit{г})~интерполяция, основанная на
регуляризации; (\textit{д})~быстрое суперразрешение; (\textit{е})~качественное
суперразрешение
\label{f8nas}}
\end{figure*}

     \section{Заключение}

     Таким образом, в работе рассмотрено применение общего подхода
суперразрешения видеопоследовательности к задаче улучшения качества изображений
лиц. Анализ предложенных качественного и быстрого методов суперразрешения и
нового многомасштабного метода оценки движения показал практическую
применимость разработанных алгоритмов для задач видеонаблюдения. Рассмотрена
задача постобработки результата суперразрешения видеоданных, получаемого
быстрым методом с \mbox{целью} подавления эффекта Гиббса.

     Основными направлениями дальнейших ис\-следований являются создание
комбинированного метода оценки движения, использующего, в том\linebreak
 чис\-ле, характерные
точки лица, и разработка алгоритмов для случая зашумленных видеоданных.


     {\small\frenchspacing
{%\baselineskip=10.8pt
\addcontentsline{toc}{section}{Литература}
\begin{thebibliography}{99}

\bibitem{7nas} %1
\Au{McNeese~M.\,D.}
How video informs cognitive systems engineering: Making experience count~//
Cognition, Technology \& Work, 2004. Vol.~6. No.\,3. P.~186--194.

\bibitem{6nas} %2
\Au{Samanich~N.}
Analytics driving transition in video~// Security Technology \& Design, 2006.
September. P.~96--98.

\bibitem{2nas} %3
\Au{Ушмаев~О.\,С.}
Реализации концепции многофак\-тор\-ной биометрической идентификации в
пра\-во\-ох\-ра\-ни\-тельных системах. Интерполитех-2007.
{\sf http://www. dancom.ru/rus/AIA/Archive/RUVI\_BioLinkSolutions\_ MultimodalBiometricsConcept.pdf}.

\bibitem{5nas} %4
Обработка и анализ цифровых изображений с примерами на LabVIEW и IMAQ Vision: [учебный
курс]~/ Ю.\,В.~Визильтер, С.\,Ю.~Желтов, В.\,А.~Князь и~др.~--- М. : ДМК Пресс, 2007.  464~с.


\bibitem{1nas} %5
\Au{Синицын~И.\,Н., Губин~А.\,В., Ушмаев~О.\,С.}
Метрологические и биометрические технологии и
системы~// История науки и техники, 2008.  №\,7. С.~41--44.

\bibitem{3nas} %6
\Au{Ушмаев~О.\,С., Босов~А.\,В.} 
Реализация концепции многофакторной биометрической 
идентификации в интегрированных аналитических системах~// Бизнес и безопасность в России, 2008.
№\,49, январь. С.~104--105.

\bibitem{4nas} %7
\Au{Ушмаев~О.\,С., Синицын~И.\,Н.}
Информационные технологии распознавания лиц в потоке~// 
Оптико-элек\-трон\-ные приборы и устройства в системах распознавания образов. Ч.~2.~--- Курск, 2008. 
С.~138--140.

\bibitem{8nas} %8
\Au{Niels H., Venetiaren~P.\,L., Lipton~A.} 
The evolution of video surveillance: An overview~// 
Machine Vision and Applications, 2008. Vol.~19. No.\,5--6. P.~279--290.

\bibitem{9nas} %9
Face Recognition Vendor Test. {\sf http://www.frvt.org}.

\bibitem{10nas} %10
Face Recognition Vendor Test  2006. {\sf http://frvt.org/ FRVT2006/}.

\bibitem{11nas} %11
Портретная экспертиза~/ Под ред.\ Зинина А.\,М.~---  М.: Право и Закон, Экзамен, 2004. 160~с.

\bibitem{12nas}
\Au{Тихонов А.\,Н., Арсенин~В.\,Я.}
Методы решения некорректных задач. М., 1979.

\bibitem{13nas}
\Au{Farsiu~S., Robinson~D., Elad~M., Milanfar~P.}
Fast and robust multi-frame super-resolution~// 
IEEE Trans. On Image Processing, 2004. Vol.~13. No.\,10. P.~1327--1344.

\bibitem{14nas}
\Au{Boyd~S., Xiao~L., Mutapcic~A.}
Subgradient methods. Lecture notes of EE392.~--- Stanford 
University, 2003.

\bibitem{15nas}
\Au{Krylov~A.\,S., Nasonov~A.\,V., Sorokin~D.\,V.}
Face image super-resolution from video data with non-uniform illumination~//
Conference (International) Graphicon~2008 Proceedings, 2008. P.~150--155.

\bibitem{16nas}
\Au{Lin~F., Fookes~C., Chandran~V., Sridharan~S.}
Investigation into optical flow super-resolution 
for surveillance applications~// APRS Workshop on Digital Image Computing Proceedings, 
2005. P.~ 73--78.

\bibitem{17nas}
\Au{Krylov~A., Nasonov~A.}
Fast super-resolution from video data using optical flow estimation~// 
IEEE ICSP'08 Proceedings. Beijing, 2008. P.~853--856.

\bibitem{18nas}
\Au{Krylov~A., Nasonov~A., Ushmaev~O.}
Image super-resolution using fast deconvolution~// 
9th Conference on Pattern Recognition and Image Analysis: New Information Technologies Proceedings. Nizhni 
Novgorod, 2008. Vol.~1. No.\,2. P.~362--364.

\bibitem{19nas}
\Au{Krylov~A., Nasonov~A.}
Adaptive total variation deringing method for image interpolation~// 
ICIP'08 Proceedings. San Diego, 2008. P.~2608--2611.

\bibitem{20nas}
\Au{Sung Won Park, Savvides~M.}
Breaking the limitation of manifold analysis for super-resolution of facial images~//
IEEE  Conference (International) on Acoustics, Speech and Signal Processing, 2007.
Vol.~1. P.~573--576.

\bibitem{21nas}
\Au{Jiangang Yu}. 
Super-resolution and facial expressions for face recognition in video. PhD.\ Thesis. 
Riverside: University of California, 2007.

\bibitem{22nas}
\Au{Lucas~B.\,D., Kanade~T.}
An iterative image registration technique with an application to stereo vision~// 
Imaging Understanding Workshop Proceedings, 1981. P.~121--130.

\label{end\stat}

\bibitem{23nas}
\Au{Lukin~S., Krylov~A.\,S., Nasonov~A.\,V.} 
Image interpolation by super-resolution~// Conference (International) 
Graphicon~2006 Proceedings, 2006. P.~239--242.

\end{thebibliography}
}
}
\end{multicols}  

 
 
 