%%%%%%%%%%%%%%%%%%%%%%%%%%%%%%%%%%%%%%%%%%%%%%%%%%%%%%%%%%%%%%%%%%%%%%%%%%%%


%\newcommand{\be}{\begin{equation}}
%\newcommand{\ee}{\end{equation}}
\newcommand{\vp}{{\mathbf p}}
%\newcommand{\ber}{\begin{eqnarray}}
%\newcommand{\eer}{\end{eqnarray}}
\newcommand{\A}{{\mathbf A}}

%\newcommand{\nin}{\noindent}
\newcommand{\non}{\nonumber}
\newcommand{\half}{\fr{1}{2}}
\newcommand{\quarter}{\fr{1}{4}}

\def\stat{zeifmann}

\def\tit{О НЕСТАЦИОНАРНЫХ СИСТЕМАХ ОБСЛУЖИВАНИЯ С~КАТАСТРОФАМИ$^*$}
\def\titkol{О нестационарных системах обслуживания с катастрофами} 

\def\autkol{А.\,И.~Зейфман, Я.\,А.~Сатин, А.\,В.~Чегодаев}
\def\aut{А.\,И.~Зейфман$^1$, Я.\,А.~Сатин$^2$, А.\,В.~Чегодаев$^3$}

\titel{\tit}{\aut}{\autkol}{\titkol}

{\renewcommand{\thefootnote}{\fnsymbol{footnote}}\footnotetext[1]
{Исследование поддержано грантом РФФИ 06-01-00111 и научным грантом
Вологодской области.}}

\renewcommand{\thefootnote}{\arabic{footnote}}
\footnotetext[1]{Вологодский государственный
педагогический университет,  Институт проблем информатики РАН и ВНКЦ
ЦЭМИ РАН, a\_zeifman@mail.ru}
\footnotetext[2]{Вологодский государственный педагогический
университет, yacovi@mail.ru}
\footnotetext[3]{Вологодский государственный педагогический
университет, cheg\_al@mail.ru}


\Abst{Рассматриваются модели обслуживания, описываемые
процессами рождения и гибели (ПРГ) с катастрофами. Получены как
оценки скорости сходимости к предельному режиму, так и оценки
различных характеристик этого предельного режима. Рассмотрены также
вопросы построения предельных характеристик и пример конкретной
системы обслуживания.}

\KW{нестационарные системы обслуживания;
марковские модели с катастрофами; слабая эргодичность; оценки;
предельные характеристики; аппроксимация}

      \vskip 24pt plus 9pt minus 6pt

      \thispagestyle{headings}

      \begin{multicols}{2}

      \label{st\stat}

\section{Введение}

Простейшие модели систем массового обслуживания с катастрофами начали изучаться 
несколько лет назад (см., например, подробную мотивацию и первые результаты в 
работах~\cite{KK, Di}). Несколько другой подход для изучения близких моделей был 
применен в~\cite{DZ}. Обсуждение современных исследований в этой области и 
некоторые новые результаты для общих стационарных ПРГ с катастрофами приведены 
в~\cite{Di08}. Нестационарная система обслуживания типа  $M(t)/M(t)/S$ с 
катастрофами была изучена недавно в~\cite{z08}. В настоящей работе будет 
рассмотрен более общий класс моделей, опи\-сы\-ва\-емых нестационарными ПРГ с 
катастрофами.

Предлагаемый подход базируется на методе, возможность применения
которого впервые была отмечена в заметке Б.\,В.~Гнеденко и И.\,П.~Макарова~\cite{gm},  
а развитие его было проведено в работах одного из авторов настоящей 
 статьи~\cite{z85, z95b}.

Пусть $X=X(t)$, $t\geq 0$~--- ПРГ с катастрофами, а $\lambda_n(t)$,
$\mu_n(t)$ и $\xi (t)$~--- интенсивности рождения, гибели и
катастрофы  соответственно.

Обозначим через 
\begin{multline*}
p_{ij}(s,t)=Pr\left\{ X(t)=j\left| X(s)=i\right.
\right\}\,,\\
i,j \ge 0\,, \ 0\leq s\leq t\,,
\end{multline*} 
переходные вероятности
процесса $X=X(t)$, а через  $p_i(t)=Pr\left\{ X(t) =i \right\}$~---
его вероятности состояний.

Тогда вероятности состояний (при выполнении некоторых естественных
дополнительных условий) удовлетворяют прямой системе
дифференциальных уравнений Колмогорова
\begin{equation}
\begin{cases}
\fr{dp_0}{dt} &= -\left(\lambda_0 (t) + \xi(t)\right)p_0 +\mu_1(t)p_1 + \xi(t) ,  \\
\fr{dp_k}{dt} &= \lambda_{k-1} (t)p_{k-1} -\left(\lambda_{k} (t) +
\mu_{k}(t)+{}\right.\\
&\left.{}+
\xi(t)\right)p_k +\mu(t)_{k+1}p_{k+1}, \quad k \ge 1\,.
\end{cases}
\label{eq111}
\end{equation}

Обозначим через 
$$
\vp(t)=\left(p_0(t),p_1(t),\dots\right)^T\,,\quad t>0\,,
$$
вектор-столбец вероятностей состояний, а через
$\A(t)=\left\{a_{ij}(t),\: t\geq 0\right\}$~--- матрицу, порождаемую
сис\-те\-мой~(\ref{eq111}), при этом элементы  матрицы $\A(t)$
определяются по формулам
\begin{equation*}
a_{ij} (t) = 
\begin{cases}
\lambda _{i-1}\left( t\right) & \mbox {при} \quad j=i-1\,, \\
\mu _{i+1}\left( t\right) & \mbox {при} \quad j=i+1\,,\\
-\left( \lambda _i\left( t\right) +\mu _i\left( t\right) +\xi(t) \right) & \mbox {при} \quad j=i\,, \\
0 &   \mbox{в остальных}\\
& \mbox{случаях}\,.
\end{cases}
%\label{eq111,5}
\end{equation*}

Ограничимся здесь рассмотрением только таких процессов,
интенсивности которых можно представить в следующем виде:
\begin{equation*}
\lambda _n\left( t\right) =\nu_n \lambda \left( t\right),\  \mu
_n\left( t\right) =\eta_n \mu \left( t\right), \quad t\ge 0\,, \quad n\in E\,,
% \label{eq101}
\end{equation*}
предполагая, что числовые множители ограничены, т.\,е.\ $ 0 \le
\eta_n \leq M$,  $0 \leq \nu_n \leq M $ (см.\
подробное рассмотрение в~\cite{z06}).

Теперь можно систему~(\ref{eq111}) рассмотреть как дифференциальное
уравнение
\begin{equation}
\fr{d\vp}{dt}=\A\left( t\right) \vp  +{\bf g} (t)\,, \quad t\ge 0\,,
\label{eq112}
\end{equation}
в пространстве последовательностей  $l_1$, где 
$$
{\bf g}(t)=\left(\xi (t),0,0, \dots\right)^T\,.
$$

Обозначим через $\Omega=\left\{{\bf x}: \: {\bf x}\geq 0,\: \|{\bf
x}\|_1=1\right\}$ множество всех стохастических векторов.

Далее всюду будем считать <<базисные>> функции $\lambda(t)$,
$\mu(t)$ и $\xi(t)$ локально интегрируемыми на $[0;\infty)$. Более
того, для простоты вычислений будем предполагать эти функции
ограниченными, т.\,е.\ будем считать, что при некотором $L$
выполняется неравенство
\begin{equation*}
\lambda(t) + \mu(t) + \xi(t) \le L < \infty 
%\label{eq112b}
\end{equation*}
почти при всех $t \ge 0$.

Тогда
\begin{equation*}
\|A(t)\|_1  = \sup_{j}\sum_i|a_{ij}(t)| \le 2M L  
%\label{eq112bb}
\end{equation*}
почти при всех $t \ge 0$, а значит, задача Коши для
уравнения~(\ref{eq112}) с начальным условием $\vp(0)$ имеет
единственное решение
\begin{equation*}
{\bf p}(t) =  U(t){\bf p}(0) + \int\limits_0^t U(t,\tau){\bf
g}(\tau)\, d\tau \,, 
%\label{eq112bc}
\end{equation*}
где $U(t,s)$~--- оператор Коши уравнения~(\ref{eq112}). При
этом если ${\bf p}(s) \in \Omega$, то и ${\bf p}(t) \in \Omega$ при
любом  $t \ge s$.

\section{Оценки, случай большой интенсивности катастроф}

\smallskip

\noindent
\textbf{Теорема~1.} \textit{Пусть}
\begin{equation}
\int\limits_0^\infty \xi(t)\, dt = \infty\,. 
\label{cat01}
\end{equation}

\textit{Тогда $X(t)$ слабо эргодичен в равномерной операторной топологии. При этом справедлива оценка}
\begin{equation*}
\|{\bf p^*}(t) - {\bf p^{**}}(t)\| \le 2 e^{- \int\limits_0^t
\xi(\tau)\, d\tau} 
%\label{cat02}
\end{equation*}
\textit{для любых начальных условий} ${\bf p^*}(0)$, ${\bf
p^{**}(0)}$.

\smallskip


\noindent
Д\,о\,к\,а\,з\,а\,т\,е\,л\,ь\,с\,т\,в\,о.\ \, Здесь и далее будет использоваться понятие
логарифмической нормы операторной функции и соответствующие оценки
(см.~\cite{z06, gz04}. В условиях теоремы~1 имеем (через
$\gamma \left(A(t)\right)_{1}$ обозначена логарифмическая норма в
пространстве~$l_1$):
\begin{equation*}
\gamma \left(A(t)\right)_{1} = \sup_i \left(a_{ii}(t) + \sum_{j\neq
i} a_{ji}(t)\right) = -\xi(t)\,.
%\label{cat03}
\end{equation*}
А тогда справедлива следующая (точная!) оценка:
\begin{equation*}
\|U(t,s)\| \le  e^{-\int\limits_s^t \xi(\tau)\, d\tau}
%\label{cat04}
\end{equation*}
для всех $0 \le s \le t$, а значит, справедливо и требу\-емое утверждение.

\smallskip

\noindent
\textbf{Замечание 1}. 
В общем случае можно рассматривать  {\it любой} режим распределения вероятностей состояний ${\bf p^*}(t)$ в качестве предельного.
Однако если все интенсивности ($\lambda(t)$, $\mu(t)$ и $\xi(t)$)
1-периодичны, то существует  1-периодический предельный режим,
скажем ${\bf \pi}(t) = \left(\pi_0(t),\pi_1(t),\dots\right)^T$.


\smallskip


Далее будем изучать математические ожидания типа
\begin{equation*}
E_{\vp(0)}(t) =  E_{\vp(0)}\left\{X(t)\right\} =
E\left\{X(t)\left|{\vp(0)}\right. \right\} 
\end{equation*} 
и, в частности, 
\begin{equation*} 
E_k(t) = E\left\{X(t)\left|X(0)=k\right.\right\}\,.
%\label{cat11}
\end{equation*}

\medskip

Выпишем первые простейшие оценки.

\smallskip

\noindent
\textbf{Теорема 2.} \label{mean01}
\textit{Пусть выполнено условие~{\rm (\ref{cat01})}. Тогда}
\begin{multline}
e^{-\int\limits_0^t \xi(\tau)\, d\tau}E_{\vp(0)}(0) - M
\int\limits_0^t \mu(\tau)
e^{-\int\limits_{\tau}^t \xi(s)\, ds} \, d\tau \le {} \\
{}\le E_{\vp(0)}(t) \le e^{-\int\limits_0^t \xi(\tau)\,
d\tau}E_{\vp(0)}(0) +{}\\
{}+
 M\int\limits_0^t \lambda(\tau)
e^{-\int\limits_{\tau}^t \xi(s)\, ds} \, d\tau 
\label{cat13}
\end{multline}
\textit{при всех} $t \ge 0$ \textit{и любом} $\vp(0)$.

\smallskip

\noindent
Д\,о\,к\,а\,з\,а\,т\,е\,л\,ь\,с\,т\,в\,о\,.\ \, Из~(\ref{eq111}) получаем:
\begin{equation*}
\fr{d E_{\vp(0)}(t)}{dt} =  \sum_{k \ge
0}\left(\lambda_k(t)-\mu_k(t)-k\xi(t)\right) p_k(t)\,. 
%\label{cat141}
\end{equation*}
Отсюда
\begin{equation}
\fr{d E_{\vp(0)}(t)}{dt} \le  M\lambda(t)  - \xi(t) E_{\vp(0)}(t)
 \label{cat142}
\end{equation}
и
\begin{equation}
\fr{d E_{\vp(0)}(t)}{dt} \ge  - M\mu (t) - \xi(t) E_{\vp(0)}(t) \,.
\label{cat143}
\end{equation}
Теперь~(\ref{cat13}) вытекает из~(\ref{cat142}) и~(\ref{cat143}).

\smallskip

\noindent
\textbf{Определение 1.} 
\textit{Будем говорить, что марковская цепь~$X(t)$ имеет \emph{предельное среднее} $\varphi (t),$ если}
\begin{equation*}
 \lim_{t \to \infty }  \left(\varphi (t) - E_k(t)\right) = 0 
% \label{cat44}
\end{equation*}
\textit{при любом $k$.}

\smallskip


\noindent
\textbf{Теорема 3.}  \label{mean02}
\textit{Пусть интенсивности процесса  $\lambda(t)$, $\mu(t)$, $\xi(t)$
1-периодичны. Пусть, далее,}
\begin{equation}
\int\limits_0^1\xi(t)\, dt >0\,. \label{cat501}
\end{equation}
\textit{Тогда $X(t)$ имеет 1-периодическое предельное среднее~$\varphi (t)$.}


\smallskip

\noindent
Д\,о\,к\,а\,з\,а\,т\,е\,л\,ь\,с\,т\,в\,о\,.\ \, 
Прежде всего отметим, что в условиях теоремы найдется $\delta >1$, при котором
\begin{equation}
\int\limits_0^1\left(\xi(t)- M\left(\delta
-1\right)\lambda(t)\right)\, dt>0\,. 
\label{cat51}
\end{equation}
Рассмотрим матрицу
\begin{equation*}
D=\mathrm{diag}\,\left( 1, \delta, \delta^2,\dots \right)  
%\label{cat52}
\end{equation*}
и пространство последовательностей $\cal B$ таких, что 
$$
\|{\bf x}\|_{\cal B} = \sum\limits_{i=0}^{\infty} \delta^i |x_i| < \infty\,.
$$ 
Будем исследовать теперь прямую систему 
Колмогорова~(\ref{eq112}) как уравнение в пространстве $\cal B$.  Тогда логарифмическую норму  $\gamma
\left(A(t)\right)$ в $\cal B$ можно оценить следующим образом:
\begin{multline*}
\gamma \left(A\right)_{\cal B} = \sup\limits_{i \ge 0} \left(\delta
 \lambda_i(t) - \left(\lambda_i(t) +\mu_i(t)
+\xi(t)\right) +{}\right.\\ 
\left.{}+
\delta^{-1} \mu_i(t) \right) \le 
\sup\limits_{i \ge 0} \left(\left(\delta-1\right)
 \lambda_i(t) -\xi(t)\right) \le {}\\
 {}\le - \left(\xi(t)- M\left(\delta
-1\right)\lambda(t)\right)\,. 
%\label{cat53}
\end{multline*}
А тогда имеем
\begin{equation*}
\|U(t,s)\|_{\cal B} \le  e^{-\int\limits_s^t \left(\xi(\tau)-
M\left(\delta -1\right)\lambda(\tau)\right)\, d\tau} 
%\label{cat54}
\end{equation*}
при всех $0 \le s \le t$.

Следовательно,
\begin{multline}
\|{\bf p^*}(t) - {\bf p^{**}}(t)\|_{\cal B}  \le {}\\
{}\le e^{-\int\limits_0^t
\left(\xi(\tau)- M\left(\delta -1\right)\lambda(\tau)\right)\,
d\tau}\|{\bf p^*}(0) - {\bf p^{**}}(0)\|_{\cal B}
\label{cat55}
\end{multline}
для любых допустимых начальных условий ${\bf p^*}(0)$, 
${\bf p^{**}}(0)$.

С другой стороны, используя условия теоремы, получаем следующую оценку:
\begin{multline*}
\limsup\limits_{t \to \infty} \|{\bf p}(t)\|_{\cal B} \le
\limsup\limits_{t \to \infty} \left( \|U(t){\bf p}(0) \|_{\cal B} +{}\right.\\
\left.{}+ \int\limits_0^t \| U(t,\tau){\bf g}(\tau)\, d\tau \|_{\cal
 B}\right) \le {} \\
 {}\le
\limsup\limits_{t \to \infty} \int\limits_0^t \xi(\tau)
e^{-\int\limits_{\tau}^t \left(\xi(u)- M\left(\delta
-1\right)\lambda(u)\right) \, du} \, d\tau ={}\\
{}= \textsf{M} < \infty
%\label{cat56}
\end{multline*}
при любом ${\bf p}(0)$.

Положим
\begin{equation*}
W = \sup_{n}\fr{n}{\delta^n} < \infty\,. 
%\label{cat57}
\end{equation*}
Тогда получаем
\begin{multline*}
\limsup\limits_{t \to \infty} E_{\vp(0)}(t) =  \limsup\limits_{t \to
\infty} \sum_{k=0}^{\infty} kp_k(t) \le{}\\
{}\le W \limsup\limits_{t \to
\infty} \|{\bf p}(t)\|_{\cal B} \le  W \textsf{M} 
\end{multline*}
при любом допустимом $\vp(0)$.

Выберем теперь ${\bf p^*}(0) = {\bf \pi}(0)$, $\varphi(t) =
\sum\limits_{k=0}^{\infty}k\pi_k(t)$ и ${\bf p^{**}}(0) = {\bf p}(0)
= {\bf e}_0$. Тогда в~(\ref{cat55}) имеем:
\begin{multline*}
\|{\bf \pi}(t) - {\bf p}(t)\|_{\cal B}  \le{}\\
{}\le  e^{-\int\limits_0^t
\left(\xi(\tau)- M\left(\delta -1\right)\lambda(\tau)\right)\,
d\tau} \sum_{k=0}^{\infty} \delta^k \pi_k(0) \le{}\\
{}\le  \textsf{M}
e^{-\int\limits_0^t \left(\xi(\tau)- M\left(\delta
-1\right)\lambda(\tau)\right)\, d\tau} \,. 
%\label{cat59}
\end{multline*}
Окончательно получаем следующую оценку:
\begin{equation}
|\varphi (t) - E_0(t)| \le  W \textsf{M} e^{-\int\limits_0^t
\left(\xi(\tau)- M\left(\delta -1\right)\lambda(\tau)\right)\,
d\tau} \,. 
\label{cat591}
\end{equation}

Правая часть~(\ref{cat591}) стремится к нулю при  $t \to
\infty$ в соответствии с~(\ref{cat51}). Легко убедиться, что тогда и  $|\varphi
(t) - E_k(t)| \to 0$ при $t \to \infty$ для {\it любого} $k$. А значит,
$\varphi (t)$~--- 1-периодическое предельное среднее для $X(t)$.

\bigskip

\noindent
\textbf{Следствие 1.}
{\it Пусть выполнены условия теоремы~3. Тогда справедлива
оценка скорости сходимости к предельному среднему~(\ref{cat591})}.

\smallskip

\noindent
\textbf{Замечание 2.}
Предельное среднее существует, в принципе, независимо от свойств интенсивностей как функций времени.
Однако в общем случае в качестве предельного среднего можно выбирать
{\it любое} среднее, поскольку никаких его особых свойств
гарантировать нельзя. Достаточным условием существования в этом
общем случае будет~(\ref{cat51}) вмес\-то~(\ref{cat501}).

\section{Оценки, общий случай}

Пусть $d_i$~--- некоторые положительные числа.

Положим
\begin{multline}
\alpha_{k}\left( t\right) = \lambda _k\left( t\right) +\mu
_{k+1}\left( t\right) -{}\\
{}-\fr{d_{k+1}}{d_k} \lambda _{k+1}\left(
t\right) -\fr{d_{k-1}}{d_k} \mu _k\left( t\right)\,, \quad k \ge 0\,,
\label{g01}
\end{multline}
и
\begin{equation*}
\alpha\left( t\right) = \inf_{k\geq 0} \alpha_{k}\left( t\right)\,.
%\label{g02}
\end{equation*}

\smallskip

\noindent
\textbf{Теорема 4.}
\textit{Пусть заданы интенсивности}  $\lambda_k(t)$ \textit{и}~$\mu_k(t)$ 
\textit{для процесса рождения и гибели с катастрофа\-ми}~$X(t)$.
 
\textit{Пусть существует последовательность положительных чисел}  $\{d_j\}$
\textit{такая, что} $1 \le d_1 \le d_2 \le \dots$\textit{, и при этом}
$ %\begin{equation*}
\int\limits_0^\infty \alpha(t)\, dt = + \infty$. 
%\label{g03}
%\end{equation*}
\textit{Тогда  $X(t)$ слабо эргодичен для любой $\xi(t)$ и
справедлива следующая оценка:}
\begin{multline*}
\|{\bf p^*}(t) - {\bf p^{**}}(t)\|_{\cal B}  \le{}\\
{}\le
 2 e^{-\int\limits_s^t \alpha(\tau)\, d\tau}\|{\bf p^*}(s) - {\bf
p^{**}}(s)\|_{1D} 
%\label{g04}
\end{multline*}
\textit{при всех $s,t, \ 0 \le s \le t$, и любых допустимых
начальных условиях ${\bf p^*}(s)$,  ${\bf p^{**}}(s)$}.


\smallskip


\noindent
Д\,о\,к\,а\,з\,а\,т\,е\,л\,ь\,с\,т\,в\,о\,.\ \, Поскольку  ${\bf p}(t) \in \Omega$ при всех $t
\ge s$, можно положить
 $p_0(t) = 1 - \sum\limits_{i \ge 1} p_i(t)$ (для обычных ПРГ этот подход подробно описан, например, в~\cite{z06}),
 и тогда из~(\ref{eq112}) получим систему

 \end{multicols}
 
 \hrule
 
 \vspace*{6pt}
 
 \noindent
\begin{equation*}
\begin{pmatrix}
 \fr{dp_1}{dt}\\
 \fr{dp_2}{dt}\\
 \vdots\\
 \fr{dp_n}{dt}\\
\vdots
\end{pmatrix}=
\begin{pmatrix}
 -(\lambda_0+\lambda_1+\mu_1+\xi_1) & (\mu_2 -\lambda_0) & -\lambda_0 & -\lambda_0 & \cdots & \cdots\\
  \lambda_1 & -(\lambda_2+\mu_2+\xi_2) & \mu_3 & 0 &  0 &\cdots\\
 0 & \lambda_2 & -(\lambda_3+\mu_3+\xi_3) & \mu_4  & 0  & \cdots \\
\vdots & \vdots & \vdots & \vdots & \vdots & \ddots \\
\end{pmatrix}
\begin{pmatrix}
p_1 \\
p_2 \\
\vdots \\
p_n \\
\vdots
\end{pmatrix}
+
\begin{pmatrix}
\lambda_0 \\
0 \\
\vdots \\
0 \\
\vdots
\end{pmatrix}
%\label{eq:12.02}
\end{equation*} 

\hrule

\begin{multicols}{2}

или, в векторном виде,
\begin{equation}
\label{eq:12.03}
\fr{d{\bf z}(t)}{dt}=B(t){\bf z}(t)+{\bf f}(t)\,.
\end{equation}


Решение этого неоднородного уравнения можно записать в виде
\begin{equation*}
{\bf z}(t)= V(t,0){\bf z}(0)+\int\limits_0^t{}V(t,z){\bf f}(z)\,dz\,,
%\label{12.}
\end{equation*}
где $V(t,z)$~--- оператор Коши уравнения~(\ref{eq:12.03}).

Рассмотрим матрицу
\begin{equation*}
  D=
\begin{pmatrix}
  d_0 & d_0 & d_0 & \cdots \\
  0   & d_1 & d_1 & \cdots \\
  0   & 0   & d_2 & \cdots \\
  \vdots & \vdots & \ddots & \ddots
\end{pmatrix}
%\label{05}
\end{equation*}
и пространство последовательностей
\begin{multline*}
\ell_{1D}={}\\
{}=
\left\{{\bf z} =(p_1,p_2,\ldots):\|{\bf z}\|_{1D}=
\|D{\bf
z}\|_1<\infty\right\}\,. 
%\label{06}
\end{multline*}
Имеем
\begin{equation*}
D^{-1}=
\begin{pmatrix}
d_0^{-1} & -d_1^{-1} & 0 & \ldots &\ldots  \\
0 & d_1^{-1} & -d_2^{-1} & 0 & \ddots \\
\vdots & 0 & d_2^{-1} & \ddots & \ddots \\
\vdots& \ddots & \ddots & \ddots & \ddots \\
\vdots&\ldots  & \ddots & \ddots & \ddots
\end{pmatrix}\,.
\end{equation*}

Рассмотрим теперь логарифмическую норму $\gamma
\left(B(t)\right)_{1_D}=  \gamma \left(D B(t)D^{-1}\right)_{1}$.
Имеем
\begin{multline*}
DB(t)D^{-1}={}\\
{}=\!
\begin{pmatrix}
-\left( \lambda _0+\mu _1 +\xi\right) \!\!&\!\! \fr{d_0 \mu_1}{d_1}\!\!
&\!\! 0\!\! &\!\!\ldots\!\!
&\!\! \ldots \\
\fr{d_1  \lambda _1}{d_0} & \!\!-\left( \lambda _1+\mu _2
+\xi\right)\!\! &\!\!
\fr{d_1\mu _2}{ d_2} \!\!&\!\! 0\!\! &\!\! \ddots \\
0\!\!&\!\! \fr{d_2  \lambda _2}{d_1} \!\!&\!\! \ddots\!\! &\!\! \ddots\!\! &\!\! \ddots \\
\vdots\!\! &\!\! 0\!\! &\!\! \ddots\!\! &\!\! \ddots\!\! &\!\! \ddots \\
\vdots\!\!&\!\! \ldots\!\! &\!\! \ddots\!\! &\!\! \ddots\!\! &\!\! \ddots
\end{pmatrix}\!\!
%\label{eqS8}
\end{multline*}
и, следовательно,

\noindent
\begin{multline*}
\gamma \left(B\right)_{1D} = \sup\limits_{i \ge 0}
\left(\fr{d_{i+1}}{d_i}
 \lambda_{i+1}(t) - \left(\lambda_i(t) +\mu_{i+1}(t)
+{}\right.\right.\\
\left.\left. {}+\xi(t)\right) +  \fr{d_{i-1}}{d_i} \mu_i(t) \right) 
\le -
\alpha(t) 
%\label{g05}
\end{multline*}
в соответствии с~(\ref{g01}). Теперь, используя
доказательство теоремы~1 из~\cite{z06}, получаем
\begin{multline*}
\|{\bf p^*}(t) - {\bf p^{**}}(t)\|_{1D}  \le {}\\
{}\le  e^{-\int\limits_s^t
\alpha(\tau)\, d\tau}\|{\bf p^*}(s) - {\bf p^{**}}(s)\|_{1D}\,.
%\label{g06}
\end{multline*}

Сравним теперь нормы вектора ${\bf z} = \left(z_1, z_2, \dots
\right)^T$ в пространствах  $\cal B$ и $l_{1D}$. Имеем
\begin{multline*}
\|{\bf z}\|_{\cal B} = \sum_{i \ge 1} d_iz_i =
d_1\left(\left|\sum_{i \ge 1} z_i + \sum_{i \ge 2} -
z_i\right|\right) +{}\\
{}+ d_2\left(\left|\sum_{i \ge 2} z_i + \sum_{i \ge
3} - z_i\right|\right) + \dots \le{}  \\
{}\le d_1\left|\sum_{i \ge
1} z_i \right| + 2d_2\left|\sum_{i \ge 2} z_i \right| + \dots \le
2\|{\bf z}\|_{1D}\,, 
%\label{g07}
\end{multline*}
откуда в итоге и вытекает требуемое утверждение.

\smallskip

\noindent
\textbf{Следствие 2.}
\textit{Пусть при выполнении условий теоремы последовательность $d_i$
возрастает достаточно быстро так, что $\inf\limits_{k \ge 1}
d_k/k = \omega > 0$. Тогда $X(t)$ имеет предельное среднее
 $\phi^*(t)$ и справедлива следующая оценка:}
\begin{equation*}
\left|\phi^*(t) - E_k(t)\right| \le \fr{2}{\omega}
e^{-\int\limits_0^t \alpha(\tau)\, d\tau}\|{\bf p^*}(0) - {\bf
e_k}\|_{1D}\,. 
%\label{g08}
\end{equation*}

\medskip


\noindent
\textbf{Теорема 5.}
\textit{Пусть при выполнении условий предыду\-щего следствия все интенсивности
1-периодичны. Тогда существуют 1-периодический предельный режим
распределения вероятностей состояний  
$$
{\bf \pi}(t) = \left(\pi_0(t),\pi_1(t),\dots\right)^T
$$ 
и соответствующее ему
предельное 1-периодическое среднее $\phi(t)$. Кроме того,
справедливы следующие оценки:}
\begin{align}
\|{\bf p}(t) - {\bf \pi}(t)\|_{\cal B}&  \le 2 e^{-\int\limits_0^t
\alpha(\tau)\, d\tau}\|{\bf p}(0) - {\bf \pi}(0)\|_{1D}\,; \label{g09}\\
\left|\phi(t) - E_k(t)\right| & \le \fr{2}{\omega}
e^{-\int\limits_0^t \alpha(\tau)\, d\tau}\|{\bf \pi}(0) - {\bf
e_k}\|_{1D}\,. \label{g10}
\end{align}


%\bigskip

К сожалению, полученные оценки~(\ref{g09}) и~(\ref{g10}) имеют один
существенный недостаток, связанный с отсутствием информации о ${\bf
\pi}(0)$, а значит, и возможностью реального применения. Рассмотрим
способ, позволяющий получить такую информацию. Пусть $X(0) = 
k$, тогда имеем $\|{\bf e_k}\|_{1D} =\sum\limits_{i=1}^k d_i$ при $k \ge
1$ и $\|{\bf e_0}\|_{1D} = 0$. Далее, используя подход из~\cite{z06}, 
можно оценить $\|{\bf \pi}(0)\|_{1D}$ следующим образом.
Имеем
%\vspace*{-3pt}

\noindent
\begin{equation*}
\sup_{|t-s| \le 1} \int\limits_s^t \alpha(\tau)\, d\tau = K <
\infty\,, 
%\label{g11}
\end{equation*}
далее получаем
\begin{multline*}
\limsup_{t \to \infty} \|{\bf \pi} (t)\|_{1D} \le \left\|\int\limits_0^t
V(t,\tau){\bf f}(\tau) d\tau\right\|_{1D} \le{}\\
{}\le  L \nu_0 \int\limits_0^t e^{-\int\limits_{\tau}^t \alpha (u)\, du} d\tau \le  {}\\
{}\le
L e^K\nu_0 \int\limits_0^t e^{- \alpha^*\left(t-\tau\right)} d\tau \le
\fr{L e^K\nu_0}{\alpha^*}\,, 
%\label{g12}
\end{multline*}
где $\alpha^* = \int\limits_0^1 \alpha (u)\, du$.

Кроме того, с учетом 1-периодичности ${\bf \pi} (t)$ справедливо
неравенство 

%\vspace*{-3pt}

\noindent
$$
 \|{\bf \pi} (0)\|_{1D} \le \limsup\limits_{t \to \infty}
\|{\bf \pi} (t)\|_{1D}\,.
$$

А тогда получаем
\begin{equation*}
\|{\bf \pi}(0) - {\bf e_k}\|_{1D} \le \limsup_{t \to \infty} \|{\bf
\pi} (t)\|_{1D} + \|{\bf e_k}\|_{1D} 
%\label{g13}
\end{equation*}
и следующее утверждение.

\smallskip

\noindent
\textbf{Следствие 3.}
\textit{Пусть в условиях теоремы~5 выполняется равенство $X(0) =
k$. Тогда справедливы следующие оценки скорости сходимости:}
\begin{align*}
\|{\bf p}(t) - {\bf \pi}(t)\|_{\cal B}  %\le{}\\
&\le 2 e^{-\int\limits_0^t
\alpha(\tau)\, d\tau}\left( \sum_{i=1}^k d_i + \fr{L
e^K\nu_0}{\alpha^*}\right) %\label{g14}
%\end{equation*}
\\
\intertext{и}
%\begin{equation*}
\left|\phi(t) - E_k(t)\right| % \le {} \\
&\le \fr{2}{\omega}
e^{-\int\limits_0^t \alpha(\tau)\, d\tau}\left( \sum_{i=1}^k d_i +
\fr{L e^K\nu_0}{\alpha^*}\right)\,. %\label{g15}
\end{align*}



\vspace*{-6pt}
\section{Аппроксимации}

\vspace*{-3pt}

Рассмотрим семейство <<усеченных>> процессов $X_n(t), \ n \ge S$, с
фазовыми пространствами $E_n\; =$\linebreak\vspace*{-12pt}
\pagebreak

\noindent
 $=\;\{0,1,\dots,n\}$, теми же
интенсивностями при  $k\; \le$\linebreak $\le\; n$ и матрицами интенсивностей $A_n(t)$.

Пусть $\{h_k\}$~--- последовательность положительных чисел такая, что
$1 = h_1 \le h_2 \le \dots$, и
\begin{equation*}
w_n = \sup_{k\ge n} \fr{h_k}{d_k}\,.
\end{equation*}

Обозначим через $\|{\bf z}\|_{{\cal B}_d}$ и $\|{\bf z}\|_{{\cal
B}_h}$ нормы, соответствующие пространствам ${\cal B}$ для
последовательностей $\{d_k\}$ и $\{h_k\}$ соответственно.


\medskip

\noindent
\textbf{Теорема 6.}
\textit{Пусть при выполнении условий теоремы~5 дополнительно
выполнено условие  $\lim\limits_{n \to \infty} w_n = 0.$ Пусть $X(0) = X_n
(0) = 0.$ Тогда}
\begin{equation*}
\|{\bf p} (t) - {\bf p}_n (t)\|_{{\cal B}_h} \le \fr{6L^2Mw_n
e^K\nu_0t}{\alpha^*}
%\label{cat05}
\end{equation*}
\textit{при всех $t \ge 0$ и любом $n$.}


\medskip

\noindent
Д\,о\,к\,а\,з\,а\,т\,е\,л\,ь\,с\,т\,в\,о\,.\ \,Будем отождествлять векторы
$\left(x_1,\dots,x_n,0,0,\dots\right)^T$ и $\left(x_1,\dots,x_n
\right)^T$. Рас\-смот\-рим прямую систему Колмогорова~(\ref{eq112}) для
исходного процесса  в следующей форме:
\begin{equation*}
\fr{d\mathbf{p}}{dt}=A_n(t) \mathbf{p} + {\bf g}(t) +\left(A(t) -
A_n(t) \right) \mathbf{p}\,, 
%\label{cat06}
\end{equation*}
а также соответствующую систему
\begin{equation}
\fr{d\mathbf{p_n}}{dt}=A_n(t) \mathbf{p_n} +  {\bf g}(t)
\label{cat061}
\end{equation}
для усеченного процесса.

%\smallskip

Имеем
\begin{equation*}
{\bf p}_n (t)= U_n(t){\bf p} (0) + \int\limits_0^t U_n(t,\tau){\bf
g}(\tau)\, d\tau 
%\label{cat062}
\end{equation*}
при ${\bf p} (0) = {\bf p}_n (0)$ и
\begin{multline*}
{\bf p} (t)= U_n \left(t\right) {\bf p} (0) + \int\limits_0^t
U_n(t,\tau){\bf g}(\tau)\, d\tau + {}\\
{}+
\int\limits_0^t U_n \left(t,
\tau\right) \left(A(\tau) - A_n(\tau) \right) {\bf p} (\tau)\,
d\tau\,. 
%\label{eq3316}
\end{multline*}

Тогда (в любой норме) получаем
\begin{multline}
\left\|{\bf p} (t) - {\bf p}_n (t)\right\| = {}\\
{}= 
\left\|\int\limits_0^t U_n \left(t,
 \tau\right) \left(A(\tau) -
 A_n(\tau) \right) {\bf p} (\tau)\, d\tau \right\|\,.
\label{cat063}
\end{multline}

Рассмотрим матрицу Коши
\begin{equation*}
U_n =
\begin{pmatrix}
  u_{00}^n & \ldots & u_{0n}^n  & 0 & 0 & \ldots \\
u_{10}^n & \ldots & u_{1n}^n  & 0 & 0 & \ldots \\
\ldots & \ldots& \ldots & \vdots & \vdots & \ddots \\
u_{n0}^n & \ldots & u_{nn}^n  & 0 & 0 & \ddots \\
0 & \ldots & 0 & 1 & 0 & \ddots \\
0 & \ldots & 0 & 0 & 1 & \ddots \\
\ldots & \ldots& \ddots & \ddots & \ddots &\ddots
\end{pmatrix}
%\label{18}
\end{equation*}
Тогда
\begin{multline*}
\!\!\!\!\left(A -A_n\right) {\bf p} = \left(0,\dots,0,-\lambda_np_n +
\mu_{n+1}p_{n+1}, \lambda_np_n -{}\right.\\
\left.{}- \left(\lambda_{n+1} + \mu_{n+1}+\xi
\right)p_{n+1} + \mu_{n+2}p_{n+2},\dots \right)^T 
%\label{20}
\end{multline*}
и, следовательно,
\begin{multline*}
U_n\left(A -A_n\right) {\bf p} ={}\\
\!=\!
\begin{pmatrix}
u_{0n}^n\left(-\lambda_np_n +
\mu_{n+1}p_{n+1}\right)\\
u_{1n}^n\left(-\lambda_np_n +
\mu_{n+1}p_{n+1}\right) \\ \vdots \\
u_{nn}^n\left(-\lambda_np_n +
 \mu_{n+1}p_{n+1}\right) \\ \!\lambda_np_n -\left(\lambda_{n+1} +
\mu_{n+1}+\xi\right)p_{n+1} + \mu_{n+2}p_{n+2} \\ \vdots
\end{pmatrix}.
%\label{21}
\end{multline*}

С учетом неравенств $ u_{ij}^n (t,\tau) \ge 0$ (при всех $i,j,
t,\tau $) и равенств $\sum\limits_i u_{ij}^n (t,\tau) = 1$ (при всех $j,
t,\tau $) получаем оценку
\begin{multline}
\|U_n \left(A -A_n \right) {\bf p}\|_{{\cal B}_h} ={}\\[2pt]
{}=
\left|-\lambda_np_n + \mu_{n+1}p_{n+1}\right|\sum_{k\leq n} h_k
u_{kn}^n +  {}\\[2pt]
{}+ \sum\limits_{k > n}\! h_{k+1} \left| \lambda_kp_k
 - \left(\lambda_{k+1} + \mu_{k+1} + \xi \right)p_{k+1} +{}\right.\\[2pt]
\!\!\!\left.{}+\mu_{k+2}p_{k+2} \right| \le
 3LM  \sum_{k \ge n} h_k p_k = 
 3LM\sum_{k \ge n} \fr{h_k}{d_k} d_k p_k \le{}\\[2pt]
 {}\le 3LMw_n
\limsup_{t \to \infty} \|{\bf \pi} (t)\|_{{\cal B}_d} \le
\fr{6L^2Mw_n e^K\nu_0}{\alpha^*} \,. 
\label{21'}
\end{multline}
Теперь из~(\ref{cat063}) и~(\ref{21'}) получаем
\begin{equation*}
\left\|{\bf p} (t) - {\bf p}_n (t)\right\|_{{\cal B}_h} %\le{}\\
\le  3LM w_n
\int\limits_0^t \limsup_{t \to \infty} \|{\bf \pi} (t)\|_{{\cal
B}_d}\, d\tau 
%\label{2201}
\end{equation*}
и требуемое утверждение.

\smallskip

\begin{figure*}[b] %fig1
\vspace*{1pt}
\begin{center}
\mbox{%
\epsfxsize=165.71mm
\epsfbox{zei-1.eps}
}
\end{center}
\vspace*{-9pt}
\begin{minipage}[t]{80mm}
\Caption{Предельное среднее
\label{f1z}}
\end{minipage}
\hfill
\begin{minipage}[t]{80mm}
\Caption{Вероятность $Pr\left\{ X(t) =0 \right\}$
\label{f2z}}
\end{minipage}
\end{figure*}

\noindent
\textbf{Замечание 3.}
По-видимому, наиболее интересные оценки получаются, если выбрать  $h_k=1$ или $h_k=k$ (при всех $k$).
Первый случай дает возможность построить 1-периодические предельные
вероятности, а второй~--- предельное среднее.

\smallskip

\noindent
\textbf{Следствие 4.}
\textit{Пусть выполнены условия теоремы~6, а $X(0) = X_n (0) =
0.$ Тогда при всех  $t \ge 0$ и любом $n$ справедливы оценки}
\begin{multline}
\|{\bf \pi}(t) - {\bf p}_n (t)\|_{1} \le 2 e^{-\int\limits_0^t
\alpha(\tau)\, d\tau} \fr{L e^K\nu_0}{\alpha^*} +{}\\
{}+ \fr{6L^2Mw_n^1
e^K\nu_0t}{\alpha^*} 
\label{2202}
\end{multline}
и
\begin{multline*}
\left|\phi(t) - E_{0,n}(t)\right| \le \fr{2}{\omega}
e^{-\int\limits_0^t \alpha(\tau)\, d\tau} \fr{L
e^K\nu_0}{\alpha^*} + {}\\
{}+\fr{6L^2Mw_n^2 e^K\nu_0t}{\alpha^*},
%\label{2203}
\end{multline*}
где $
w_n^1 = \sup\limits_{k\ge n} 1/d_k$, $ w_n^2 = \sup\limits_{k\ge n}
k/d_k$ и $E_{0,n} =E_k(t)\; =$\linebreak $=\;E\left\{X_n(t)\left|X_n(0)=0\right.\right\}$.


\medskip


Пусть $J_k(t) = \Pr \left(X(t) \le k\right) $~--- вероятность того,
что число требований в системе в момент $t$ не превышает $k$.
Полученные оценки позволяют приближенно вычислить предельные
1-периодические $J_k(t)$ и предельное 1-периодическое среднее
следующим образом.

\medskip

Пусть $\varepsilon$ -- произвольное положительное число.

\noindent
\begin{enumerate}
\item Выбираем целое  $m$ так, чтобы первое слагаемое в правой
части~(\ref{2202}) было меньше $ \varepsilon/3$ при всех $t \ge m$.
\item
Находим $n$ так, чтобы второе слагаемое в правой части~(\ref{2202}) 
было меньше $ \varepsilon/3$ при всех $t \le m+1$.
\item
Тогда решение задачи Коши для усеченной прямой системы
Колмогорова~(\ref{cat061}) с начальным условием ${\bf e}_0$ на
отрезке $[m;m+1]$ (вычисленное с погрешностью $\varepsilon/3$) дает
предельный 1-периодический режим ${\bf \pi}(t) =
\left(\pi_0(t),\pi_1(t),\dots\right)^T$ с погрешностью меньше
$\varepsilon$.
\item
Наконец, предельное 1-периодическое выражение для $J_k(t) =
\Pr\{X(t)\le k\}$ вычисляется как $\sum\limits_{i=0}^{k}\pi_i(t)$ с той же
погрешностью $\varepsilon$.
\end{enumerate}

%\medskip
Предельное среднее находится аналогично.

\section{Пример}

Рассмотрим систему обслуживания $M(t)/M(t)/100$ с катастрофами,
считая, что интенсивности задаются следующим образом: 
\begin{align*}
\lambda(t) &= 3+\sin 2\pi t \,;\\
\mu(t) &= 2 + \cos 2\pi t \,;\\
\xi(t) &=2 + \sin 4 \pi t\,,
\end{align*}
и вычислим предельное среднее $\phi(t)$, а также предельные
величины  $J_k(t)$ при некоторых значениях~$k$. В частности,
$J_0(t)$~--- вероятность того, что в момент~$t$ очередь пуста, т.\,е.\
в системе обслуживания нет ни одного требования.

\medskip

\begin{figure*} %fig3-4
\vspace*{1pt}
\begin{center}
\mbox{%
\epsfxsize=164.864mm
\epsfbox{zei-3.eps}
}
\end{center}
\vspace*{-9pt}
\begin{minipage}[t]{80mm}
\Caption{Вероятность $Pr\left\{ X(t) \le 5 \right\}$
\label{f3z}}
\end{minipage}
\hfill
\begin{minipage}[t]{80mm}
\Caption{Вероятность $Pr\left\{ X(t) \le 10 \right\}$
\label{f4z}}
\end{minipage}
\end{figure*}


Используя подход, описанный в предыдущих параграфах, выберем $d=1{,}3$
и $d_k = d^k$. Тогда будет: $L=10, \ M=100$, $\nu_0 = 1$, $w_n^1 =
1{,}3^{-n}$, $w_n^2 = n/1{,}3^{n}$, далее $\alpha(t) = \mu(t)
-(d-1)\lambda(t)\;=$\linebreak $=\; 1{,}1 +\cos 2\pi t - 0{,}3 \sin 2\pi t$, $K \le 4$ и
$\alpha^* = 1{,}1$.

Пусть $\varepsilon = 10^{-6}$. Тогда, как следует из полученных
оценок, можно выбрать $m = 30$ и $n = 155$. При этом предельное
среднее  $\phi(t)$ и все предельные величины $J_k(t)$ строятся с
точностью до $\varepsilon = 10^{-6}$ как соответствующие
характеристики решения с начальным условием ${\bf e}_0$ задачи Коши
для соответствующей усеченной прямой системы Колмогорова на отрезке
$[m,m+1]$. Приведенные рис.~1--4 дают приближенные с описанной
погрешностью предельные характеристики $\phi(t)$ и $J_0(t),
J_{5}(t), J_{10}(t)$ соответственно. Отметим, что при $k \ge 11$
получается уже $J_k(t) \approx 1$.

\medskip

\noindent
\textbf{Замечание 4.}
Отметим в заключение,
что аналогично можно рассмотреть несколько более общую ситуацию,
когда интенсивности катастроф зависят от состояния процесса.
Основные изменения в этом случае будут касаться результатов
параграфа~2.

{\small\frenchspacing
{%\baselineskip=10.8pt
\addcontentsline{toc}{section}{Литература}
\begin{thebibliography}{99}

\bibitem{KK} %1
\Au{Krishna Kumar~B.,  Arivudainambi~D.} 
Transient solution of an
$M\vert M\vert 1$ queue with catastrophes~//  Comput. Math. Appl., 2000. Vol.~40. 
P.~1233--1240.

\bibitem{Di}  %2
\Au{Di Crescenzo~A., Giorno~V., Nobile~A.\,G., Ricciardi~L.\,M.}  
On the $M\vert M\vert 1$ queue with catastrophes and its continuous approximation~//
 Queueing Syst., 2003. Vol.~43. P.~329--347.
 
 \bibitem{DZ}  %3
\Au{Van Doorn~E.\,A., Zeifman~A.}
Extinction probability in a birth--death process with killing~// J.
Appl. Probab., 2005. Vol.~42. P.~185--198.

\bibitem{Di08}  %4
\Au{Di Crescenzo~A., Giorno~V., Nobile~A.\,G., Ricciardi~L.\,M.}
A~note on birth--death processes with catastrophes~//
Statist. Probab. Lett., 2008. Vol.~78. P.~2248--2257.


\bibitem{z08} %5
\Au{Zeifman A.,  Satin~Ya., Chegodaev~A., Bening~V., Shorgin~V.}
Some bounds for $M(t)/M(t)/S$ queue with catastrophes~// 
SMCTools08 Proceedings.  Athens, Greece, 2008.

\bibitem{gm}   %6
\Au{Гнеденко Б.\,В., Макаров~И.\,П.}
Свойства решений задачи с потерями в случае периодических
интенсивностей~// Дифф. уравнения, 1971. Т.~7. С.~1696--1698.

\bibitem{z85}  %7
\Au{Zeifman A.\,I.} 
Stability for contionuous-time
nonhomogeneous Markov chains~// Lect. Notes Mathem., 1985. Vol.~1155. P.~401--414.

\bibitem{z95b} %8
\Au{Zeifman A.\,I.} 
Upper and lower bounds on the rate of
convergence for nonhomogeneous birth and death proc\-esses~// Stoch.
Proc.  Appl., 1995. Vol.~59. P.~157--173.

\bibitem{z06} %9
\Au{Zeifman A., Leorato~S., Orsingher~E., Satin~Ya., Shilova~G.}
Some universal limits for nonhomogeneous birth and death processes~//
Queueing Syst., 2006. Vol.~52. P.~139--151.

\label{end\stat}

\bibitem{gz04} %10
\Au{Granovsky~B., Zeifman~A.}
Nonstationary queues: Estimation of the rate of convergence~//
Queueing Syst.,  2004. Vol.~46. P.~363--388.


\end{thebibliography}
}
}
\end{multicols}