\newcommand {\ebd}{\triangleq}
\newcommand {\ppp}{{\mathcal P}}
\newcommand {\qqq}{{\mathcal Q}}
\newcommand{\me}[2]{\mathsf{E}_{ #1 }\left\{ \mathop{#2} \right\} }



\def\stat{borisov}

\def\tit{РЫНОК С~МАРКОВСКОЙ СКАЧКООБРАЗНОЙ ВОЛАТИЛЬНОСТЬЮ I: МОНИТОРИНГ ЦЕНЫ РИСКА\\
 КАК ЗАДАЧА ОПТИМАЛЬНОЙ ФИЛЬТРАЦИИ$^*$}

\def\titkol{Рынок с~марковской скачкообразной волатильностью I: мониторинг цены риска как задача оптимальной фильтрации}

\def\aut{А.\,В.~Борисов$^1$}

\def\autkol{А.\,В.~Борисов}

\titel{\tit}{\aut}{\autkol}{\titkol}

\index{Борисов А.\,В.}
\index{Borisov A.\,V.}


{\renewcommand{\thefootnote}{\fnsymbol{footnote}} \footnotetext[1]
{Работа выполнена с~использованием инфраструктуры Центра коллективного пользования <<Высокопроизводительные вы\-чис\-ле\-ния и~большие данные>> 
(ЦКП <<Информатика>>) ФИЦ ИУ РАН (г.~Москва).}}


\renewcommand{\thefootnote}{\arabic{footnote}}
\footnotetext[1]{Федеральный исследовательский центр <<Информатика и~управление>> Российской академии наук;
\mbox{aborisov@frccsc.ru}}

\vspace*{-8pt}



\Abst{Первая часть цикла посвящена исследованию задачи определения рыночной цены риска 
в~финансовой системе, содержащей безрисковый актив~--- банковский вклад, базовые рисковые инструменты и~их деривативы. 
Модель эволюции базовых активов, описываемая стохастической дифференциальной системой (СДС), содержит стохастическую волатильность, 
порожденную скрытым марковским скачкообразным процессом (МСП). Исследуемый рынок предполагается неполным с~отсутствием возможности арбитража. 
Рыночная цена риска, соответствующая преобладающей мартингальной мере, выражается через скрытый марковский процесс и~не может быть восстановлена точно. 
Тем не менее она может быть оценена оптимальным образом по наблюдениям цен активов. Исходя из наличия преобладающей мартингальной меры в~статье 
получена СДС в~частных производных, описывающая эволюцию во времени цены деривативов и~являющаяся аналогом
 классического уравнения Блэ\-ка--Шоул\-за. Задача оценки рыночной цены риска сформулирована в~терминах оптимальной фильт\-ра\-ции 
 состояния СДС наблюдения. Также обсуждаются вопросы численной реализации решения данной задачи оценивания.}


\KW{марковский скачкообразный процесс; оптимальная фильтрация; стохастическая волатильность; рыночная цена риска; преобладающая мартингальная мера}

  \DOI{10.14357/19922264230204}{GAXCHQ}
  
\vspace*{-2pt}


\vskip 10pt plus 9pt minus 6pt

\thispagestyle{headings}

\begin{multicols}{2}

\label{st\stat}


 \section{Введение}

 Одна из практических проблем, привлекающих  решение задачи фильтрации состояний стохастических систем,~--- 
 мониторинг внутренней волатильности базовых финансовых инструментов по разнородным наблюдениям их цен~\cite{EMT_02, GKL_05, CLR_06}. 
 Помимо цен основных инструментов в~качестве доступной статистической информации присутствует богатейший объем данных 
 о~торгах производными инструментами (деривативами). В~работе~\cite{R_04} предложен один из способов использования такого массива данных.
  Для этого была выбрана концепция неполных безарбитражных рынков~\cite{B_98, D_01}. Зависимость\linebreak цены деривативов от текущих цен базовых активов 
  и~времени до погашения определялась относительно преобладающей мартингальной меры с~по\-мощью формулы Фейн\-ма\-на--Ка\-ца. 
  Эта мера\linebreak \mbox{определялась} в~терминах рыночной цены риска (Market Price of Risk, MPR)~--- некоторого <<внеш\-не\-го>> ненаблюдаемого процесса, 
  связывающего\linebreak мгновенную процентную ставку и~внутреннюю волатильность. Таким образом, решение задачи\linebreak оценивания MPR в~реальном масштабе 
  времени поз\-во\-ля\-ет получить и~оценки внутренней во\-ла\-тиль\-ности.
  { %\looseness=1
  
  }

 Автор~\cite{R_04} распространяет концепцию MPR, ранее использовавшуюся преимущественно для описания колебаний процентных ставок 
 по бескупонным облигациям, ставок по кредитам и~пр.\ (см., например,~\cite{DS_00} и~ссылки внутри), на класс базовых рисковых активов. 
 В~качестве возможных моделей MPR были выбраны диффузионные процессы~--- решения СДС с~винеровскими процессами в~правой 
 части. Основной результат~\cite{R_04} заключается в~формулировке задачи оценивания MPR по наблюдениям базовых активов и~деривативов в~виде 
 задачи фильт\-ра\-ции состояний СДС в~самом общем виде. Эта проб\-ле\-ма весьма сложна для ее как аналитического, так и~численного решения.

 Представляемая работа посвящена исследованию упомянутой задачи в~случае, когда MPR определяется поведением внешнего 
 ненаблюдаемого МСП с~конечным числом состояний. Цель статьи заключается в~исследовании задачи 
 мониторинга MPR по имеющимся наблюдениям цен базовых активов и~деривативов в~рамках математического аппарата стохастического анализа. 
 Статья имеет следующую структуру. В~разд.~2 представлено описание исследуемого класса финансовых систем, включающего банковский депозит, 
 базовые рисковые активы и~их деривативы. Ставка по депозиту является детерминированной, а~цены базовых активов и~деривативов описываются 
 наблюдаемыми случайными процессами. Мгновенные ставки и~внутренние волатильности базовых активов зависят от скрытого МСП. 
 Уравнения динамики деривативов отсутствуют.

 Относительно исследуемого класса сис\-тем известно, что он соответствует неполному без\-ар\-бит\-раж\-но\-му рынку.
  Данные условия гарантируют \mbox{существование} некоторой преобладающей мартингальной меры, относительно которой 
  все базовые активы имеют одинаковую процентную ставку, совпадающую с~депозитной. Существование такой меры позволило характеризовать 
  временн$\acute{\mbox{у}}$ю эволюцию деривативов в~форме решения системы линейных дифференциальных уравнений в~частных производных~--- 
  некоторого обобщения уравнения Блэ\-ка--Шоул\-за~\cite{Sh_98}. Таким образом, корректно построена стохастическая дифференциальная модель наблюдения, 
  в~которой в~качестве ненаблюдаемого состояния выступает скрытый фак\-тор-про\-цесс и~модели наблюдаемых цен базовых активов и~их деривативов 
  пол\-ностью определены. Оценивание MPR в~реальном масштабе времени эквивалентно фильтрации состояний скрытого МСП. Эти выкладки содержатся в~разд.~3.

 Полученная система наблюдения обладает некоторыми свойствами, препятствующими непосредственному применению известных методов 
 оптимальной нелинейной фильтрации для построения оценок МСП: поток $\sigma$-под\-ал\-гебр, порожденный наблюдениями, не является непрерывным справа,
  шумы в~наблюдениях вырождены и~пр. Раздел~4 статьи посвящен перспективам теоретических исследований и~численной реализации полученной задачи фильт\-ра\-ции.
  
  \vspace*{-6pt}
  
  \section{Описание модели финансовой системы}
  
  На отрезке времени $[0,T]$ рассматривается финансовая сис\-те\-ма, со\-сто\-ящая из
  банковского вклада,
$N$ базовых финансовых инструментов~$S$ и~$M$ деривативов~$F$.

  Ставка~$r_t$ банковского вклада известна и~неслучайна.
  Эволюция цен базовых активов определяется на базисе с~фильтрацией $(\Omega, \F, \ppp, \{\F_t\}_{t \in [0,T]})$ 
  как наблюдаемый случайный процесс $S_t \hm\ebd \mathrm{col}\left(S_t^1,\ldots,S_t^N\right)$~--- единственное сильное решение СДС:
  \begin{multline}
  dS_t = \mathrm{diag}\, S_t a\left(t,Z_t\right)dt + \mathrm{diag}\, S_t \sigma(t,Z_t)\,dw_t,\\ t \in (0,T], \enskip S_0 \sim \pi_0(s),
  \label{eq:S_sys_P}
  \end{multline}
  где
  \begin{itemize}
  \item   $S_0$~--- $\F_0$-из\-ме\-ри\-мое начальное условие;
  \item
  $w_t = \mathrm{col}\left(w_t^1,\ldots,w_t^K\right)\hm \in \mathbb{R}^K$~--- $\F_t$-со\-гла\-со\-ван\-ный стандартный винеровский процесс ($K \hm\geqslant N$);
  \item
  $a(\cdot,\cdot)$, $\sigma(\cdot,\cdot)$~--- $(N \times 1)$- и~$(N \times K)$-мер\-ные 
  функции мгновенной процентной ставки  и~внутренней волатильности базовых инструментов~$S$; 
  $\sigma(\cdot,\cdot)$ имеет полный строчный ранг $\ppp$-п.~н.\ почти везде по мере Лебега на $[0,T]$;
  \item
  $Z_t$~--- $\F_t$-со\-гла\-со\-ван\-ный ненаблюдаемый (скрытый) процесс, описывающий действие на финансовую систему неконтролируемых внешних факторов.
  \end{itemize}
 
 В~(\ref{eq:S_sys_P}) $Z_t \hm= \mathrm{col}\left(Z_t^1, \ldots, Z_t^{\ell}\right) \hm\in \mathbb{S}^L\hm = \{e_1,\ldots, e_{\ell}\}$ 
 представляет собой неоднородный МСП с~конечным множеством состояний~--- набором координатных ортов евклидова пространства~$\mathbb{R}^L$, 
 известной мат\-рич\-но\-знач\-ной функцией интенсивностей переходов~$\Lambda(\cdot)$ и~начальным распределением~$\pi_0^Z$. 
 Дополнительно о~$Z$ предполагается, что все компоненты его распределения строго положительны на $[0,T]$. Известно~\cite{EAM_10}, что
  МСП $Z_t$~--- единственное сильное решение СДС
  \begin{equation}
  dZ_t = \Lambda^{\top}(t)Z_t\,dt + dM_t,\enskip t \in (0,T], \enskip Z_0 \sim \pi_0^Z,
  \label{eq:Z_sys}
  \end{equation}
  где
  $M_t = \mathrm{col}\,(M_t^1,\ldots,M_t^L) \hm\in \mathbb{R}^L$~--- некоторый $\F_t$-со\-гла\-со\-ван\-ный мартингал.
  

  Для исследуемой финансовой сис\-те\-мы предполагается отсутствие арбитража~\cite{Criens_20} 
  и~существование на измеримом пространстве $(\Omega, \F)$ преобладающей мартингальной меры $\qqq$, $\qqq \hm\sim \ppp$,
   для которой выполнены следующие условия.
  \begin{enumerate}[1.]
  \item
  Процесс $M_t$ является мартингалом относительно меры~$\qqq$.
  \item
  Относительно $\qqq$ процесс $S_t$~--- единственное сильное решение СДС
  \begin{multline}
  dS_t =r_t S_t \,dt + \mathrm{diag}\,S_t \sigma(t,Z_t)\,dw_t^{\qqq},\\
   t \in (0,T], \enskip S_0 \sim \pi_0(s),
  \label{eq:S_sys_Q}
  \end{multline}
  в~которой $w_t^{\qqq} \hm\in \mathbb{R}^K$~--- $\F_t$-со\-гла\-со\-ван\-ный стандартный винеровский процесс.
  \end{enumerate}
  Согласно обобщенной теореме Гирсанова~\cite{KE_15} винеровский процесс~$w_t^{\qqq}$ связан с~исходным~$w_t$ равенством
   \begin{equation}
  dw_t^{\qqq} = \theta_t\,dt + dw_t,
  \label{eq:w_con}
  \end{equation}
  где $\theta_t \in \mathbb{R}^K$~--- недоступный наблюдению $\F_t$-со\-гла\-со\-ван\-ный процесс MPR.

Момент погашения всех деривативов совпадает с~$T$ и~соответствует
платежному требованию 
$$
H(S_T) = \mathrm{col}\,(H^1(S_T),\ldots, H^M(S_T)),
$$
 где $H(s) \hm= \mathrm{col}\,(H^1(s),\ldots, H^M(s))$~--- 
известная детерминированная век\-тор-функ\-ция.
Эволюция цен производных финансовых инструментов является наблюдаемым $\F_t$-со\-гла\-со\-ван\-ным специальным семимартингалом 
$F_t \hm= \mathrm{col}\,(F_t^1, \ldots, F_t^M)$,  таким что $F_T \hm= H(S_T))$ $\ppp$-п.~н.

Данный рынок является неполным и~означает неединственность мартингальной вероятностной меры. 
Точное знание MPR~$\theta_t$ эквивалентно знанию преобладающей вероятностной меры~$\qqq$ и~позволяет решить ряд актуальных финансовых задач, 
в~том числе задачу построения хеджирующего портфеля~\cite{B_98, Sh_98}. В~рас\-смат\-ри\-ва\-емой задаче MPR 
недоступна прямому безошибочному наблюдению. Все данные, доступные на момент времени~$t$, описываются $\sigma$-ал\-геб\-рой 
наблюдений 
$$
\mathcal{O}_t \ebd \sigma\{S_u, F_u: \enskip 0  \leqslant u \hm\leqslant t\},
$$ 
поэтому при наличии этой информации очевидным представляется поиск условного среднего MPR относительно имеющихся наблюдений: 
$$
\widehat{\theta}_t \ebd \me{\ppp}{\theta_t | \mathcal{O}_t}.
$$

\section{Сведение задачи определения цены риска к~задаче оптимальной фильтрации}

Из уравнений~(\ref{eq:S_sys_P}), (\ref{eq:S_sys_Q}) и~(\ref{eq:w_con}) можно определить связь между функциями~$r$, $a$ и~$\sigma$, 
с~одной стороны, и~MPR $\theta$~--- с~другой. Подстановкой~(\ref{eq:w_con}) в~(\ref{eq:S_sys_Q}) можно получить СДС~(\ref{eq:S_sys_P}) 
в~виде представления специального семимартингала. В~силу единственности этого представления равенство
$$
\mathrm{diag}\, S_t a(t,Z_t) = r_t S_t +\mathrm{diag}\, S_t a(t,Z_t) \sigma (t,Z_t)\theta_t
$$
верно $\ppp$-п.~н. почти везде по мере Лебега на $[0,T]$. Так как 
все компоненты вектора $S_t$ $\ppp$-п.~н. строго положительны, последнее равенство преобразуется к~виду
   \begin{equation}
  \sigma (t,Z_t)\theta_t = a(t,Z_t) - r_t \mathbf{1} ,
  \label{eq:theta_1}
  \end{equation}
  где $\mathbf{1}$~--- век\-тор-стол\-бец подходящей раз\-мер\-ности, со\-став\-лен\-ный из единиц.

  Далее, в~силу того что $Z_t \hm\in \mathbb{S}^L$,  функции~$a$ и~$\sigma$ представимы в~виде
  $$
  a(t,Z_t) = \sum\limits_{\ell=1}^L Z_t^{\ell} a^{\ell}(t); \enskip  \sigma(t,Z_t) = \sum\limits_{\ell=1}^L Z_t^{\ell}  \sigma^{\ell}(t),
$$
  где 
  $\{a^{\ell}(t)\}_{\ell= \overline{1,L}}$ и~$\{\sigma^{\ell}(t)\}_{\ell= \overline{1,L}}$~--- наборы известных
  неслучайных функций~--- <<алфавиты>> мгновенных ставок и~внут\-рен\-них волатильностей: 
  $$
  a^{\ell}(t) \hm\ebd a(t,e_{\ell});\enskip \sigma^{\ell}(t) \hm\ebd \sigma(t,e_{\ell}).
  $$ 
  
  В~этом случае сис\-те\-ма~(\ref{eq:theta_1}) принимает вид:
 \begin{equation*}
 \sum\limits_{\ell=1}^L Z_t^{\ell} \sigma^{\ell}(t)\theta_t = \sum\limits_{\ell=1}^L Z_t^{\ell} a^{\ell}(t) - r_t \mathbf{1}\,,
% \label{eq:theta_2}
 \end{equation*}
 откуда в~силу условий $\pi^Z(t)\hm>0$ и~$\mathrm{rk}\left(\sigma(\cdot)\right) \hm\equiv K$ следует, что MPR $\theta$ имеет вид:
 \begin{multline}
\theta_t = \sum\limits_{\ell=1}^L Z_t^{\ell} \left(\sigma^{\ell}(t)\right)^+ \left( a^{\ell}(t) - r_t \mathbf{1} \right) +{}\\
{}+
\sum\limits_{\ell=1}^L Z_t^{\ell}
\left(I-
\left(\sigma^{\ell}(t)\right)^+\sigma^{\ell}(t) \right)\xi_t,
 \label{eq:theta_3}
 \end{multline}
 где $I$~--- единичная матрица подходящей размерности; $A^+$~--- мат\-ри\-ца, псев\-до\-об\-рат\-ная мат\-ри\-це~$A$; 
 $\xi_t \hm\in \mathbb{R}^L$~--- произвольный $\F_t$-со\-гла\-со\-ван\-ный случайный процесс.
 Второе слагаемое в~(\ref{eq:theta_3}) аннулируется умножением справа на~$\sigma(t,Z_t)$ и~не влияет на превалирующую меру~$\qqq$, 
 поэтому ниже в~работе считается, что оно в~(\ref{eq:theta_3}) отсутствует. Таким образом, MPR $\theta_t$  является линейным преобразованием МСП~$Z_t$.

Вид функций $F^m(t, S_t, Z_t)$ ($m\hm=\overline{1,M}$) цен деривативов определяется, исходя из отсутствия\linebreak в~исследуемой 
финансовой сис\-те\-ме арбитража и~наличия мартингальных вероятностных мер, как 
\mbox{дисконтированное} условное среднее платежного требования~$H$ на момент погашения~$T$, 
вычисленное относительно преобладающей меры~$\qqq$:
\begin{equation*}
F^m\left(t, S_t, Z_t\right) = e^{-\int\nolimits_t^T r_s\,ds}\me{\qqq}{H^m(S_T)|\F_t}.
%\label{eq:F_m_1}
\end{equation*}
Заметим, что $\Pi_t^m \ebd \me{\qqq}{H^m(S_T)|\F_t}$~--- мартингал относительно~$\qqq$~--- представим в~виде
$$
\Pi^m = e^{\int\nolimits_t^T r_sds}\sum\limits_{\ell=1}^L Z_t^{\ell} F^{m\ell}(t, S_t),
$$
где $F^{m\ell}(t, S_t) \ebd F^m(t, S_t, e_{\ell})$.

Предположим, что $F^{m\ell}$ обладает достаточной степенью гладкости по своим переменным, тогда~$\Pi_t^m$ 
допускает следующий стохастический дифференциал (ниже в~выкладках зависимость функций от своих аргументов для краткости опущена):
\begin{multline*}
d\Pi^m = e^{\int_t^T r\,ds} \left( -r \sum\limits_{\ell=1}^L Z^{\ell} F^{m\ell} \,dt +{}\right.\\
\left.{}+\sum\limits_{\ell=1}^L dZ^{\ell} F^{m\ell} +  \sum\limits_{\ell=1}^L Z^{\ell} \,dF^{m\ell} \right).
\end{multline*}
Рассмотрим отдельно второе и~третье слагаемое в~скобках последнего выражения:

\vspace*{-3pt}
\noindent
\begin{multline*}
\sum\limits_{\ell=1}^L dZ^{\ell} F^{m\ell} = \sum\limits_{\ell=1}^L e_{\ell}^{\top}\left( \Lambda^{\top}Z\,dt + dM\right) F^{m\ell} ={}\\
{}=
\sum\limits_{i,j,\ell=1}^L e_{\ell}^i  \Lambda^{ji}Z^{j} F^{m\ell} \,dt + \sum\limits_{\ell=1}^L F^{m\ell} \,dM^{\ell} ={} \\
{} =
\sum\limits_{\ell=1}^L Z^{\ell} \sum\limits_{j=1}^L \Lambda^{\ell j}F^{mj}\,dt + \sum\limits_{\ell=1}^L F^{m\ell} \,dM^{\ell};
\end{multline*}

\vspace*{-12pt}

\noindent
\begin{multline*}
\sum\limits_{\ell=1}^L Z^{\ell}\, dF^{m\ell} = \sum\limits_{\ell=1}^L Z^{\ell}
\Bigg(
F_t^{m\ell}\,dt + \sum\limits_{n=1}^N F_{s^n}^{m\ell} \,dS^n +{}\\
{}+ \fr{1}{2}  \sum\limits_{i,j=1}^N F_{s^i,s^j}^{m\ell}\, d\langle S^i,S^j \rangle \Bigg)
= 
\sum\limits_{\ell=1}^L Z^{\ell}\Bigg(
F_t^{m\ell}\,dt +{}\\
{}+ \sum\limits_{n=1}^NF_{s^n}^{m\ell}\left(
S^n a^{\ell}_n \,dt + S^n \sum\limits_{k=1}^K \sigma_{nk}^{\ell}\,dw^k
\right)
+{}\\
{}+
\fr{1}{2}  \sum\limits_{i,j=1}^N F_{s^i,s^j}^{m\ell}\,d \langle S^i,S^j \rangle
\Bigg),
\end{multline*}
где

\vspace*{-3pt}

\noindent
\begin{align*}
F_t^{m\ell} &\ebd \fr{\partial F^{m\ell}(t,s)}{\partial t}\Bigl|_{(t,S_t)};\\
F_{s^n}^{m\ell} &\ebd \fr{\partial F^{m\ell}(t,s)}{\partial s^n}\Bigl|_{(t,S_t)}; \\
F_{s^i,s^j}^{m\ell}& \ebd \fr{\partial^2 F^{m\ell}(t,s)}{\partial s^i \partial s^j}\Bigl|_{(t,S_t)}.
\end{align*}
Используем в~третьем слагаемом формулы~(\ref{eq:w_con}), (\ref{eq:theta_3}) и~обозначение
$\nabla_s F^{m\ell} \ebd \mathrm{col} \left(
F_{s^1}^{m\ell},\ldots, F_{s^N}^{m\ell}
\right)$:
\begin{multline*}
\sum\limits_{\ell=1}^L Z^{\ell} \sum\limits_{n=1}^N \sum\limits_{k=1}^K
F_{s^n}^{m\ell} S^n \sigma_{nk}^{\ell}\, dw^k ={}\\
{}= \sum\limits_{\ell=1}^L Z^{\ell}
\left( \nabla_s F^{m\ell} \right)^{\top} \mathrm{diag}\, (S) \sigma^{\ell}\,dw = {}\\ 
{}=
\sum\limits_{\ell=1}^L Z^{\ell}
\left( \nabla_s F^{m\ell} \right)^{\top} \mathrm{diag}\,(S) \sigma^{\ell}(dw^{\qqq}-\theta \,dt) ={} \\
{} =
\sum\limits_{\ell=1}^L Z^{\ell}
\left( \nabla_s F^{m\ell} \right)^{\top} \mathrm{diag}\, (S)(r \mathbf{1} - a^{\ell})\,dt+{}\\
{}+
\sum\limits_{\ell=1}^L Z^{\ell}
\left( \nabla_s F^{m\ell} \right)^{\top} \mathrm{diag}\, (S) \sigma^{\ell}\,dw^{\qqq}.
\end{multline*}
Если 
$$
B^{\ell}(t) \ebd \sigma^{\ell}(t)\left(\sigma^{\ell}(t)\right)^{\top}\hm=\|B_{ij}^{\ell}(t)\|_{i,j=\overline{1,K}},
$$
то~$\Pi^m$ допускает дифференциал:

\columnbreak

\noindent
\begin{multline*}
d\Pi^m = e^{\int_t^T r\,ds}
\Bigg(
\sum\limits_{\ell=1}^L
F^{m\ell}\,dM^{\ell} +{}\\
{}+ \sum\limits_{\ell=1}^L Z^{\ell}
\left( \nabla_s F^{m\ell} \right)^{\top} \mathrm{diag}\,(S) \sigma^{\ell}\,dw^{\qqq}
\Bigg) + {}\\
{}+
e^{\int\nolimits_t^T r\,ds}
\sum\limits_{\ell=1}^L Z^{\ell}
\Bigg[ -rF^{m\ell} + \sum\limits_{j=1}^L \Lambda^{\ell j}F^{mj} + F_t^{m\ell}+{}\\
{}+
\left(\nabla_s F^{m\ell} \right)^{\top} \mathrm{diag}\, (S)(r \mathbf{1} - a^{\ell})
+{}\\
{}+\fr{1}{2} \sum\limits_{i,j=1}^N F_{s^i,s^j}^{m\ell}S^iS^jB_{ij}^{\ell}
\Bigg]dt.
\end{multline*}
Первое слагаемое в~нем представляет собой дифференциал мартингала относительно меры~$\qqq$, второе~--- 
дифференциал процесса с~ограниченной вариацией. Мартингальное свойство процессов~$\{\Pi^m\}$ относительно~$\qqq$, 
а~также поэлементное выполнение неравенств $\pi^Z(t)\hm > 0$ на отрезке $[0,T]$ позволяют определить функции цены~$\{F^{m\ell}(t,s)\}$ 
как решение сис\-те\-мы уравнений Колмогорова~\cite{GS_3_75}:
\begin{equation}
\left.
\begin{array}{l}
F_t^{m\ell} = rF^{m\ell} - \displaystyle\sum\limits_{j=1}^L \Lambda^{\ell j}F^{mj} - {}\\[6pt]
{}-\displaystyle \sum\limits_{n=1}^N F_{s^n}^{m\ell}s^n(r - a_n^{\ell}) - 
 \fr{1}{2}  \sum\limits_{i,j=1}^N s^is^j F_{s^i,s^j}^{m\ell}B_{ij}^{\ell},\\[6pt]
 \hspace*{18.5mm}  \ell=\overline{1,M}, \enskip m=\overline{1,M}, \enskip
t \in [0,T];\\[6pt]
F^{m\ell}(T,s) = H^m(s).
\label{eq:F_m_eq}
\end{array}
\right\}
\end{equation}
Таким образом, деривативы~$F$ допускают стохастический дифференциал
\begin{multline}
dF^{m} =
 \sum\limits_{\ell=1}^L Z^{\ell}
 \left[
 r F^{m\ell} + \sum\limits_{n=1}^N F_{s^n}^{m\ell} S^n(a_n^{\ell} - r)
\right]dt + {}\\
{}+ \sum\limits_{\ell=1}^L F^{m\ell}\,dM^{\ell} +
 \sum\limits_{\ell=1}^L Z^{\ell} \sum\limits_{n=1}^N \sum\limits_{k=1}^K F_{s^n}^{m\ell} S^n \sigma_{nk}^{\ell}\,dw^k,\\
 m=\overline{1,M},\enskip \ell=\overline{1,L}, \enskip F^{m\ell}= F^{m\ell}(t,S_t).
\label{eq:F_m_obs}
\end{multline}
Итак, определение MPR $\theta_t$ в~описанной финансовой системе сводится к~решению задачи оптимальной фильтрации состояний МСП~$Z$~(\ref{eq:Z_sys}) 
по совокупности наблюдений~$S$~(\ref{eq:S_sys_P}) и~$F$~(\ref{eq:F_m_obs}).

\vspace*{-6pt}

 \section{Заключение}
 
 \vspace*{-2pt}
 
Формулы (\ref{eq:Z_sys}), (\ref{eq:S_sys_P}) и~(\ref{eq:F_m_obs}) задают СДС наблюдения 
с~МСП в~качестве состояния и~мультипликативными шумами в~наблюдениях. Несмотря на то что теоретическое решение похожей задачи фильтра-\linebreak\vspace*{-12pt}

\pagebreak

\noindent
ции представлено 
в~\cite{BS_20}, а комплекс алгоритмов численного решения~--- в~\cite{B_20_2_ARC_, B_20_2_ARC}, при решении поставленной в~данной статье задачи возникает
 ряд теоретических и~реализационных проб\-лем. Прежде всего, обратимся к~теоретическим вопросам.

\textbf{Первое:} поток $\sigma$-ал\-гебр $\{\mathcal{O}_t\}_{t \in [0,T]}$, порожденный наблюдениями~(\ref{eq:S_sys_P}) и~(\ref{eq:F_m_obs}), 
не является непрерывным справа. Для корректного использования аппарата стохастического анализа можно <<сгладить cправа>> исходные
 $\sigma$-ал\-геб\-ры наблюдений, т.\,е.\ рас\-смат\-ри\-вать~$\mathcal{O}_{t+}$ вмес\-то~$\mathcal{O}_t$, 
 и~трансформировать задачу фильтрации в~задачу прогнозирования на малый фиксированный шаг $\delta\hm>0$, т.\,е.\ 
 в~задачу поиска $\me{\ppp}{Z_t|\mathcal{O}_{(t-\delta)+}}$.

\textbf{Второе:} система наблюдения (\ref{eq:Z_sys}), (\ref{eq:S_sys_P}) и~(\ref{eq:F_m_obs})\linebreak формально не принадлежит классу систем наблюдения, 
для которых задача фильтрации состояния МСП решена. Уравнение~(\ref{eq:F_m_obs}) цен деривативов нелинейно: коэффициенты сноса 
и~диффузии зависят от случайного процесса~$S_t$. Тем не менее результат~\cite{BS_20} может быть распространен и~на данную систему: процесс~$S_t$ имеет 
$\ppp$-п.~н.\ траектории и~доступен наблюдению, поэтому вместо детерминированных коэффициентов можно использовать их наблюдаемые случайные аналоги. 
Уместна аналогия с~услов\-но-гаус\-сов\-ски\-ми процессами~\cite{LSh_74}, оптимальная фильт\-ра\-ция со\-сто\-яний 
которых может быть выполнена с~помощью фильтра Кал\-ма\-на--Бью\-си со случайными, но наблюдаемыми па\-ра\-мет\-рами.

\textbf{Третье:} из уравнений~(\ref{eq:S_sys_P}) и~(\ref{eq:F_m_obs}) следует, что шумы в~этих наблюдениях  вырожденные, и~путем 
линейных преобразований из них можно выделить наблюдаемые компоненты, не содержащие шумов. 
В~\cite{BS_20} предложен прием, преобразующий подобные наблюдения в~совокупность считающих процессов и~косвенных наблюдений МСП,
 полученных в~известные детерминированные моменты времени.

Следующие вопросы относятся к~области чис\-лен\-ной реализации решения поставленной задачи фильтрации. 
Продолжим <<сквозную>> нумерацию проблем.

\textbf{Четвертое:} численные алгоритмы фильтрации~\cite{B_20_2_ARC} разработаны для систем наблюдения, в~которых параметры динамики и~наблюдений 
не зависят от времени. В~рас\-смот\-рен\-ной финансовой задаче это принципиально не так: с~приближением к~моменту погашения деривативов
 их цена изменяется даже при условии постоянства цены базовых активов. Предложенные 
 в~\cite{B_20_2_ARC} численные алгоритмы применимы и~в этом случае, однако из-за нестационарности сис\-те\-мы наблюдения они будут порождать 
 дополнительные ошибки, наличие которых следует принимать во внимание.
 
 \columnbreak

\textbf{Пятое:} определение функции $F$ цены деривативов как решение сис\-те\-мы~(\ref{eq:F_m_eq}) представляется самостоятельной нетривиальной задачей.
Аналитическое решение ее отсутствует, а~применяемые\linebreak чис\-лен\-ные методы должны обеспечивать достаточную точ\-ность аппроксимации 
не только самих функций~$F^{m\ell}$, но и~их частных производных~$F^{m\ell}_{s^n}$, так как они входят в~мат\-ри\-цу наблюдений~(\ref{eq:F_m_obs}).

\textbf{Шестое:} наблюдения $F$ (\ref{eq:F_m_obs})~--- это сумма непрерывных и~скачкообразных процессов, которую тео\-ре\-ти\-че\-ски легко разделить. Однако
непрерывные наблюдения являются идеализацией. В~действительности доступны наблюдения, дискретизованные по времени, либо результаты высокочастотных 
измерений в~случайные моменты времени~\cite{KCKG_13}. В~таких наблюдениях выделить скачки не пред\-став\-ля\-ет\-ся возможным, нужны новые 
варианты алгоритмов чис\-лен\-ной фильт\-ра\-ции со\-сто\-яний МСП.

Перечисленные проблемы не представляются непреодолимыми, основа их решения заложена в~работе~\cite{BS_20}.
По\-сле\-ду\-ющие час\-ти цик\-ла будут посвящены вопросам чис\-лен\-ной реализации решения задачи мониторинга MPR, 
начиная с~моделирования цен соответствующих активов, эво\-лю\-цио\-ни\-ру\-ющих в~соответствии со скачкообразным изменением MPR, и~заканчивая 
алгоритмами фильт\-ра\-ции скрытой MPR по име\-ющим\-ся разнородным наблюдениям цен базовых бумаг и~деривативов.


{\small\frenchspacing
 {\baselineskip=12pt
 %\addcontentsline{toc}{section}{References}
 \begin{thebibliography}{99}
\bibitem{EMT_02} %1
\Au{Elliott R., Malcolm~W., Tsoi~A.} HMM volatility estimation~// 
41st IEEE Conference on Decision and Control Proceedings, 2002.~--- Piscataway, NJ, USA: IEEE, 2003. Vol.~1. P.~398--404.
doi: 10.1109/CDC.2002.1184527.

\bibitem{GKL_05}
\Au{Голдентаер~Л., Клебанер~Ф., Липцер~Р.} Слежение за функцией волатильности~// Проб\-ле\-мы передачи информации, 2005. Т.~41. Вып.~3. С.~32--50.

\bibitem{CLR_06} %3
\Au{Cvitani$\acute{\mbox{c}}$~J., Liptser~R., Rozovskii~B.} A~filtering approach to tracking volatility from prices observed at random times~// 
Ann. Appl. Probab.,  2006. Vol.~16. No.\,3. P.~1633--1652. doi: 10.1214/105051606000000222.

\bibitem{R_04} %4
\Au{Runggaldier W.} Estimation via stochastic filtering in financial market models~// Contemp. Math., 2004. Vol.~351.  P.~309--318.
doi: 10.1090/conm/351/06412.

\bibitem{B_98} %5
\Au{Bj$\ddot{\mbox{o}}$rk T.} Arbitrage theory in continuous time.~--- New York, NY, USA: Oxford University Press, 1998. 324~p.

\bibitem{D_01} %6
\Au{Duffie D.} Dynamic asset pricing theory.~--- Princeton, NJ, USA: Princeton University Press, 2001. 472~p.

\bibitem{DS_00} %7
\Au{Dai Q., Singleton~K.} Specification analysis of affine term structure models~//
J.~Financ., 2000. Vol.~55. No.\,5. P.~1943--1978. doi: 10.1111/0022-1082.00278.

 \bibitem{Sh_98} %8
 \Au{Shiryayev A.} Essentials of stochastic finance: Facts, models, theory.~--- New Jersey, NJ, USA: World Scientific, 1999. 834~p.

 

 \bibitem{EAM_10} %9
 \Au{Elliott R., Moore~J., Aggoun~L.} Hidden Markov models: Estimation and control.~--- New York, NY, USA: Springer, 2010. 382~p.
 
  \bibitem{Criens_20} %10
\Au{Criens D.} No arbitrage in continuous financial markets~// Math. Financ. Econ., 2020. Vol.~14. P.~461--506.


 \bibitem{KE_15}
 \Au{Cohen S., Elliott~R.} Stochastic calculus and applications.~--- New York,~NY, USA:~Birkh$\ddot{\mbox{a}}$user, 2015. 666~p.

 \bibitem{GS_3_75}
 \Au{Gihman I., Skorohod~A.} The theory of stochastic processes III.~--- New York, NY, USA: Springer, 1979.~388~p.

   \bibitem{BS_20}
 \Au{Borisov A., Sokolov~I.} Optimal filtering of Markov jump processes given observations with state-dependent noises: Exact 
 solution and stable numerical schemes~// Mathematics, 2020. Vol.~8. No.\,4. Art. No.\,506.

%   \bibitem{B_20_3_ARC}
%{\it Борисов A.} Алгоритм робастной фильтрации марковских скачкообразных процессов по высокочастотным считающим наблюдениям~// Автоматика и~телемеханика,~2020.~Вып.~4.~С.~3--20.

%\bibitem{S_55}
%{\it Smith W.}  Regenerative stochastic processes // Proc. Royal Society of London. Ser. A. Mathematical and Physical Sciences. 1955. V. 232. No. 1188. P.~\mbox{6--31}.

   \bibitem{B_20_2_ARC_}
\Au{Борисов A.} Численные схемы фильтрации марковских скачкообразных процессов по дискретизованным наблюдениям II: случай аддитивных шумов~// 
Информатика и~её применения, 2020. Т.~14. Вып.~1. С.~17--23.

  \bibitem{B_20_2_ARC}
\Au{Борисов A.} Численные схемы фильт\-ра\-ции марковских скачкообразных процессов по дискретизованным наблюдениям III: случай мультипликативных шумов~// 
Информатика и~её применения, 2020. Т.~14. Вып.~2. С.~10--18.

     \bibitem{LSh_74}
 \Au{Liptser R., Shiryaev~A.} Statistics of random processes II: Applications.~--- Berlin/Heidelberg: Springer, 2001. 402~p.

  \bibitem{KCKG_13}
\Au{Королев В., Черток~А., Корчагин~А., Горшенин~А.} 
Ве\-ро\-ят\-ност\-но-ста\-ти\-сти\-че\-ское моделирование информационных потоков в~слож\-ных финансовых сис\-те\-мах 
на основе высокочастотных данных~// Информатика и~её применения, 2013. Т.~7. Вып.~1. С.~12--21.
\end{thebibliography}

 }
 }

\end{multicols}

\vspace*{-6pt}

\hfill{\small\textit{Поступила в~редакцию 05.10.22}}

\vspace*{8pt}

%\pagebreak

%\newpage

%\vspace*{-28pt}

\hrule

\vspace*{2pt}

\hrule

%\vspace*{-2pt}

\def\tit{MARKET WITH MARKOV JUMP VOLATILITY I: PRICE OF~RISK MONITORING AS~AN~OPTIMAL FILTERING PROBLEM}


\def\titkol{Market with Markov jump volatility I: Price of~risk monitoring as~an~optimal filtering problem}


\def\aut{A.\,V.~Borisov}

\def\autkol{A.\,V.~Borisov}

\titel{\tit}{\aut}{\autkol}{\titkol}

\vspace*{-10pt}


\noindent
Federal Research Center ``Computer Science and Control'' of the Russian Academy 
of Sciences, 44-2~Vavilov Str., Moscow 119333, Russian Federation


\def\leftfootline{\small{\textbf{\thepage}
\hfill INFORMATIKA I EE PRIMENENIYA~--- INFORMATICS AND
APPLICATIONS\ \ \ 2023\ \ \ volume~17\ \ \ issue\ 2}
}%
 \def\rightfootline{\small{INFORMATIKA I EE PRIMENENIYA~---
INFORMATICS AND APPLICATIONS\ \ \ 2023\ \ \ volume~17\ \ \ issue\ 2
\hfill \textbf{\thepage}}}

\vspace*{3pt}




\Abste{The first part of series is devoted to investigating the market price of risk in a~financial system. 
It contains riskless bank deposits, risky base assets, and their derivatives. The model of the underlying price evolution represents 
a~stochastic differential system with stochastic volatility which is a~hidden Markov jump process. The investigated market is incomplete and has no 
arbitrage possibilities. The market price of risk, which corresponds to a~prevailing martingale measure, can be characterized via the 
hidden Markov jump process but can not be restored precisely. However, it can be estimated optimally using the observations of 
both the derivative and underlying prices. Using the concept of the prevailing martingale measure existence, one can derive 
a~system of the partial differential equations which describes an evolution of the derivative prices and represents some analog of the classic Black--Sholes 
equation. Then, one can convert the calculation problem for the market price of risk to the optimal state filtering in 
a~differential stochastic observation system. The paper also discusses various aspects of the numerical realization for the stated estimation problem.}

\KWE{Markov jump process; optimal filtering; diffusion and counting observations; multiplicative observation noise; numerical approximation accuracy}




  \DOI{10.14357/19922264230204}{GAXCHQ}

\vspace*{-18pt}

\Ack
\noindent
The research was carried out using the infrastructure of the Shared Research Facilities 
``High Performance Computing and Big Data'' (CKP ``Informatics'') of FRC CSC RAS (Moscow).
  

\vspace*{6pt}

  \begin{multicols}{2}

\renewcommand{\bibname}{\protect\rmfamily References}
%\renewcommand{\bibname}{\large\protect\rm References}

{\small\frenchspacing
 {%\baselineskip=10.8pt
 \addcontentsline{toc}{section}{References}
 \begin{thebibliography}{99} 
  \bibitem{EMT_02-1} %1
\Aue{Elliott, R., W.~Malcolm, and A.~Tsoi.} 2002. 
HMM volatility estimation. \textit{41st IEEE Conference on Decision and Control Proceedings}. Piscataway, NJ: IEEE.
1:398--404. doi: 10.1109/CDC.2002.1184527.

\bibitem{GKL_05-1}
\Aue{Goldentayer, L., F.\,K.~Klebaner, and R.\,Sh.~Liptser.} 2005. Tracking volatility.
\textit{Probl. Inf. Transm.} 41(3):212--229. doi: 10.1007/s11122-005-0026-2.

\bibitem{CLR_06-1}
\Aue{Cvitani$\acute{\mbox{c}}$,~J., R.~Liptser, and B.~Rozovskii.} 2006. A~filtering approach to tracking volatility from prices observed at random times. 
\textit{Ann. Appl. Probab.} 16:1633--1652. doi: 10.1214/105051606000000222.

\bibitem{R_04-1}
\Aue{Runggaldier, W.} 2004. Estimation via stochastic filtering in financial market models. \textit{Contemp. Math.} 351:309--318. 
doi: 10.1090/conm/351/06412.

\bibitem{B_98-1}
\Aue{Bj$\ddot{\mbox{o}}$rk, T.} 1998. \textit{Arbitrage theory in continuous time}. New York, NY: Oxford University Press. 324~p.

\bibitem{D_01-1}
\Aue{Duffie, D.} 2001. \textit{Dynamic asset pricing theory}. Princeton, NJ: Princeton University Press. 472~p.

\bibitem{DS_00-1} %7
\Aue{Dai, Q., and K.~Singleton.} 2000. Specification analysis of affine term structure models.  \textit{J.~Financ.} 55(5):1943--1978.
doi: 10.1111/0022-1082.00278.

\bibitem{Sh_98-1} %8
\Aue{Shiryayev, A.} 1999. \textit{Essentials of stochastic finance: Facts, models, theory}. New Jersey, NJ: World Scientific. 834~p.



\bibitem{EAM_10-1} %9
\Aue{Elliott, R., J.~Moore, and L.~Aggoun.} 2010. \textit{Hidden Markov models: Estimation and control}. New York, NY:~Springer. 382~p.

\bibitem{Criens_20-1} %10
\Aue{Criens, D.} 2020. No arbitrage in continuous financial markets. \textit{Math. Financ. Econ.} 14:461--506.
doi: 10.1007/ s11579-020-00262-1.
 
\bibitem{KE_15-1}
\Aue{Cohen, S., and R.~Elliott.} 2015. \textit{Stochastic calculus and applications}. New York, NY: Birkh$\ddot{\mbox{a}}$user. 666~p.
 
\bibitem{GS_3_75-1}
\Aue{Gihman, I., and A.~Skorohod.} 1979. \textit{The theory of stochastic processes III}. New York, NY: Springer.~388~p.
  
\bibitem{BS_20-1} %13
\Aue{Borisov, A., and I.~Sokolov.}
 2020. Optimal filtering of Markov jump processes given observations with state-dependent noises: Exact solution and stable numerical schemes. 
 \textit{Mathematics} 8(4):506.  doi: 10.3390/ math8040506.
 
\bibitem{B_20_1_ARC-1}
\Aue{Borisov, A.} 2020. Chislennye skhemy fil'tratsii markovskikh skachkoobraznykh protsessov po diskretizovannym nablyudeniyam II: sluchay additivnykh shumov  
[Numerical schemes of Markov jump process filtering given discretized observations II: Additive noise case].
\textit{Informatika i~ee primeneniya~--- Informatics and Applications} 14(1): 117--23. doi: 10.14357/19922264200103.


\bibitem{B_20_1_ARC-2}
\Aue{Borisov, A.} 2020. Chislennye skhemy fil'tratsii mar\-kov\-skikh skachkoobraznykh protsessov po dis\-kre\-ti\-zo\-van\-nym nablyudeniyam III: 
sluchay mul'tiplikativnykh shu\-mov [Numerical schemes of Markov jump process filtering given discretized observations III: Multiplicative noises case]
\textit{Informatika i~ee primeneniya~--- Informatics and Applications} 14(2)10--18. doi: 10.14357/19922264200202.

\bibitem{LSh_74-1}
\Aue{Liptser, R., and A.~Shiryaev.} 2001. \textit{Statistics of random processes II: Applications}. Berlin/Heidelberg: Springer.~402~p.
 
\bibitem{KCKG_13-1}
\Aue{Korolev, V., A.~Chertok, A.~Korchagin, and A.~Gorshenin.} 2013. 
Veroyatnostno-statisticheskoe modelirovanie informatsionnykh potokov v~slozhnykh finansovykh sistemakh na osnove vysokochastotnykh dannykh
[Probability and statistical modeling of information flows in complex financial systems based on high-frequency data].
\textit{Informatika i~ee primeneniya~--- Informatics and Applications} 7(1)12--21. 
\end{thebibliography}

 }
 }

\end{multicols}

\vspace*{-6pt}

\hfill{\small\textit{Received October 5, 2022}} 


\Contrl

\noindent
\textbf{Borisov Andrey V.} (b.\ 1965)~--- 
Doctor of Science in physics and mathematics, principal scientist, Federal Research Center ``Computer Science and Control''
 of the Russian Academy of Sciences, 44-2~Vavilov Str., Moscow 119333, Russian Federation; \mbox{aborisov@frccsc.ru}


 


\label{end\stat}

\renewcommand{\bibname}{\protect\rm Литература} 
    