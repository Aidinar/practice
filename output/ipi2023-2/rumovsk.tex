\def\stat{rumovskaya}

\def\tit{ПОДХОДЫ К ПОДБОРУ СПЕЦИАЛИСТОВ ПРИ~ОРГАНИЗАЦИИ КОЛЛЕКТИВНОГО 
РЕШЕНИЯ ПРОБЛЕМ$^*$}

\def\titkol{Подходы к~подбору специалистов при~организации коллективного 
решения проблем}

\def\aut{С.\,Б.~Румовская$^1$}

\def\autkol{С.\,Б.~Румовская}

\titel{\tit}{\aut}{\autkol}{\titkol}

\index{Румовская С.\,Б.}
\index{Rumovskaya S.\,B.}


{\renewcommand{\thefootnote}{\fnsymbol{footnote}} \footnotetext[1]
{Исследование выполнено за счет гранта РНФ №\,23-21-00218.}}


\renewcommand{\thefootnote}{\arabic{footnote}}
\footnotetext[1]{Федеральный исследовательский центр <<Информатика и~управление>> Российской 
академии наук, \mbox{sophiyabr@gmail.com}}

%\vspace*{-12pt}

 

 

  \Abst{Исследование малых групп (коллективов, команд), их особенностей, проблем, 
динамики и~особенностей подбора специалистов стоит на стыке психологии в~управлении 
персоналом и~социальной психологии. Особое место в~широком спектре направлений 
современной науки занимает моделирование взаимодействия людей в~малых коллективах 
специалистов, в~частности в~рамках многоагентного подхода. При этом, разрабатывая 
интеллектуальные сис\-те\-мы (ИС) (искусственные гетерогенные коллективы) для решения 
практических проблем, сейчас требуется объединять в~составе системы модели специалистов 
(агентов), созданных различными командами разработчиков, имеющих несовместимые цели 
и~модели предметной об\-ласти. Отбор специалистов в~естественные и~моделей
 специалистов в~искусственные гетерогенные коллективы~--- важная задача, результаты решения которой 
влияют на дальнейший процесс принятия решений. Представлен анализ методов 
и~подходов к~подбору специалистов и~комплектования малых групп (коллективов, команд), 
измерительные инструменты которых должны подвергаться оценке качества.}
  
  \KW{группа; малый коллектив специалистов; команда; методы подбора специалистов 
  и~формирования малых групп; командообразование}

\DOI{10.14357/19922264230214}{VJWNOE} 
  
%\vspace*{-8pt}


\vskip 10pt plus 9pt minus 6pt

\thispagestyle{headings}

\begin{multicols}{2}

\label{st\stat}
  
\section{Введение}

  Взаимосвязь психологии в~управлении персоналом и~социальной психологии 
состоит в~том, что объект исследования первой~--- как отдельно взятая 
личность, так и~малые группы~--- один из сложнейших феноменов социальной 
психологии~[1]. <<Малая группа>>~\cite{2-r}~--- элементарное звено 
структуры\linebreak социальных отношений, обретающее через непосредственные 
межличностные контакты структурные, динамические и~феноменологические 
характеристики, отражающие признаки группы как \mbox{целостной} системы 
социальных и~психологических отношений. В~организациях руководители 
получают значительный эффект, создавая малые группы с~учетом групповой 
сплоченности, единства и~других социально-психологических феноменов. 

Отбор специалистов~--- важная задача, результаты выполнения которой влияют 
на дальнейшую работу группы (коллектива, команды), ре\-ша\-ющей в~различных 
сферах проб\-ле\-мы, \mbox{осложненные} слабой формализацией, комплексным 
строением, сетевым характером условий и~целей, неопределенностью, 
субъективностью и~ди\-на\-мич\-ностью. Подобный коллектив, который называют 
естественным гетерогенным коллективным интеллектом поддержки принятия 
решений~\cite{3-r},~--- это малая группа экспертов (специалистов), которой\linebreak 
присущи не\-од\-но\-род\-ность, разнообразие, сотрудничество, до\-пол\-ни\-тель\-ность 
и~относительность знаний. 

Основная форма организации малых  
коллективов~--- совещания, построенные по принципу круглого стола. Особое 
место в~современной науке занимает моделирование взаимодействия людей 
в~таких коллективах, в~частности методами многоагентных систем. Сейчас для 
решения практических проблем требуется объединять в~составе 
ИС модели специалистов (агентов), созданных 
различными командами разработчиков и~имеющих несовместимые цели 
и~модели предметной области. В~этой связи для учета субъективности 
и~динамического характера проблем предполагается разработать новый класс 
ИС~--- реф\-лек\-сив\-но-ак\-тив\-ные сис\-те\-мы искусственных гетерогенных 
интеллектуальных агентов (\mbox{РАСИГИА}), в~которых агенты будут 
взаимно моделировать рефлексивные позиции друг друга, динамически 
вырабатывать стратегии своего поведения, по мере необходимости в~процессе 
решения проблем привлекать новых агентов из пула доступных агентов от 
различных разработчиков и~исключать существующих. 

Результат работы 
\mbox{РАСИГИА} зависит от состава выбираемых для включения агентов, 
а~значит, методы подбора специалистов, разрабатываемые в~об\-ласти 
психологии управ\-ле\-ния персоналом, должны быть проанализированы на 
предмет возможности адаптации для \mbox{РАСИГИА}. 
  
  В настоящей работе проанализированы методы подбора специалистов 
и~комплектования малых групп (коллективов и~команд), разработанные 
в~области психологии управления персоналом.

\section{Малые высокоорганизованные группы (коллективы, 
команды)} 
  
  В работах~\cite{4-r, 5-r, 6-r, 7-r} о малых группах специалистов, 
рассматриваемых в~данном исследовании как естественный гетерогенный 
коллектив, говорят как о~коллективах~--- высокоразвитой форме организации 
групповой дея\-тель\-ности, при которой связи и~отношения между индивидами 
опосредованы общественно значимыми целями. Коллективы в~отечественной 
литературе рас\-смат\-ри\-ва\-ют\-ся как высший уровень развития группы, 
ха\-рак\-те\-ри\-зу\-емый\linebreak
 высокой степенью спло\-чен\-ности, единством,  
цен\-ност\-но-нор\-ма\-тив\-ной ориентации, глубокой идентификацией 
индивида с~группой и~от\-вет\-ст\-вен\-ностью за результаты совместной групповой 
\mbox{деятельности}~\cite{5-r}. В~работах~\cite{8-r, 9-r, 10-r} говорится о~стадиях 
зрелости коллектива (команды) в~рамках более широкого процесса жизни 
малой группы, которые можно свес\-ти к~таким, как (1)~притирка 
и~формирование; (2)~конфликтная (образуются подгруппы, появляются 
разногласия); (3)~экспериментирование в~методах и~средствах; (4)~появление 
спло\-чен\-ности (границы подгрупп стираются, успешное решение задач, 
творчество); (5)~высокий уровень спло\-чен\-ности (формируются прочные связи, 
роли и~полномочия динамично согласовываются, личные разногласия быст\-ро 
устраняются). Катценбах и~Смит~\cite{11-r} определяют команду как 
малое чис\-ло людей (от~2 до~25~человек, но обычно не более~10) 
с~взаимодополняющими умениями, связанных единым за\-мыс\-лом, стремящихся 
к~общим целям и~ответственных за их достижение. Команде присущи 
постоянство со\-ста\-ва, жест\-кое распределение ролей, ясная и~формальная цель, 
а~члены команды сыгранны и~действуют одинаково по отношению 
к~окру\-же\-нию~\cite{12-r}. В~\cite{4-r} отмечается, что в~современной трактовке 
команды много общего с~описанием коллектива в~работах отечественных 
авторов прош\-лых лет. Команда как группа высокого уровня развития, 
сравнительно с~пониманием коллектива, более реалистична, прагматична, 
лишена идеологических ярлыков. Таким образом, понятия <<малый коллектив 
специалистов>> и~<<команда>> идентичны, поэтому актуально исследование 
формирования и~коллективов, и~команд. 
  
\section{Методы и~подходы к~отбору~специалистов }

  \textbf{Методы комплектования малых групп и~коллективов}~\cite{13-r}. 
Определяются оптимальные количественные соотношения между работниками 
в~малых группах и~коллективах. Выделяют два взаимодополняющих принципа 
отбора: сра\-бо\-тан\-ность и~со\-вмес\-ти\-мость~\cite{14-r}. Сра\-бо\-тан\-ность 
характеризуется высокой со\-гла\-со\-ван\-ностью у~членов группы~\cite{15-r} и~ее\linebreak 
про\-дук\-тив\-ностью и~базируется на  
про\-фес\-сио\-наль\-но-ква\-ли\-фи\-ка\-ци\-он\-ной до\-пол\-ня\-емости. 
Совместимость~--- оптимальное сочетание свойств участников, 
обес\-пе\-чи\-ва\-ющее их эффективное \mbox{существование} и~спо\-соб\-ность 
оптимизировать свои взаимоотношения и~согласовывать свои 
действия~\cite{15-r}. Выделяются три уровня со\-вмес\-ти\-мости.\\[-14pt]
  \begin{enumerate}[1.]
\item Согласованность функ\-ци\-о\-наль\-но-ро\-ле\-вых ожи\-да\-ний~--- выделяют 
своеобразные роли, которые в~совместной дея\-тель\-ности дополнительно 
к~основным играют люди, например выделяют целевые и~под\-дер\-жи\-ва\-ющие~\cite{16-r}.\\[-14pt]
\item Ценностно-ори\-ен\-та\-ци\-он\-ное единст\-во (ЦОЕ) по 
А.\,В.~Петровскому~\cite{17-r}~--- сход\-ст\-во мнений, позиций членов группы 
по отношению к~объектам, наиболее значимым для группы в~целом. Есть ряд 
процедур оценки ЦОЕ.\\[-14pt]
\item Психофизиологическая со\-вмес\-ти\-мость:\\[-14pt]
\begin{itemize}
\item первичное комплектование малых групп~\cite{18-r}. Применяют метод 
изуче\-ния характерных особенностей индивидуальной ориентации человека по 
отношению к~другим людям, диагностируемых опросником межличностных 
отношений (ОМО) В.~Шультца, который определяет межличностную 
со\-вмес\-ти\-мость как отношения между двумя или более индивидами, при 
которых достигается та или иная степень взаимного удовле\-тво\-ре\-ния 
меж\-лич\-ност\-ных по\-треб\-но\-стей. Так\-же применяется соционический метод, 
например Е.\,С.~Филатовой и~Ю.\,В.~Иванова~\cite{19-r, 20-r}, ба\-зи\-ру\-ющий\-ся 
на том, что отношения между сотрудниками группы можно пред\-ста\-вить в~виде 
парных взаимодействий, а~психологический тип человека проявляется во 
взаимодействии с~другими людьми;\\ [-14pt]
\item перекомплектование~--- предполагает, что члены об\-сле\-ду\-емо\-го 
коллектива достаточно хорошо знают друг друга по со\-вмест\-ной дея\-тель\-ности. 
Тогда мож\-но использовать со\-цио\-мет\-ри\-че\-ский тест Дж.~Морено~\cite{21-r}, 
пред\-став\-ля\-ющий собой процедуру пе\-ре\-крест\-но\-го опроса членов группы друг 
о~друге по вопросам или критериям, которые на\-прав\-ле\-ны на выявление 
особенностей их взаимоотношений, взаимных оценок тех или иных качеств 
лич\-ности и~поведения. Данные ответов кодируются в~специальные мат\-ри\-цы 
и~анализируются ве\-ро\-ят\-ност\-но-ста\-ти\-сти\-че\-ски\-ми методами. 
В~\cite{22-r, 23-r, 24-r} рас\-смот\-рен ряд приемов экспертизы психологической 
со\-вмес\-ти\-мости в~сло\-жив\-ших\-ся малых группах.\\[-14pt]
\end{itemize}
\end{enumerate}
  
  \textbf{Профессиональный отбор}~\cite{7-r} предполагает выбор по 
критериям профессиональной под\-го\-тов\-лен\-ности и~опыта, уровню и~профилю 
образования. 
  \begin{enumerate}[1.]
\item Формирование профиля должности~\cite{6-r}. Каждая должностная 
позиция, в~том чис\-ле и~в~со\-ста\-ве коллектива, ре\-ша\-юще\-го проб\-ле\-му, 
предъявляет специалисту ряд требований, информацию о~которых 
структурируют и~сводят в~единую сис\-те\-му в~профиле долж\-ности. Используют 
несколько видов критериев~\cite{15-r}: 
\begin{itemize}
\item квалификационные; 
\item объективные, кон\-ста\-ти\-ру\-ющие соответствие реальных достижений 
оце\-ни\-ва\-емых субъектов некоторым количественным и~качественным 
показателям; 
\item внеш\-ние, ха\-рак\-те\-ри\-зу\-ющие наличие качеств, поз\-во\-ля\-ющих 
добиваться высоких результатов; 
\item психологические, раз\-ра\-ба\-ты\-ва\-емые на 
осно\-ве профессиограммы со\-от\-вет\-ст\-ву-\linebreak юще\-го вида дея\-тель\-ности, которая 
пред\-став\-ля\-ет собой сис\-те\-му при\-зна\-ков, опи\-сы\-ва\-ющих профессию, перечень 
норм и~\mbox{требований}~\cite{7-r}; 
\item тес\-то\-вые по ин\-ди\-ви\-ду\-аль\-но-пси\-хо\-ло\-ги\-че\-ским 
характеристикам.
\end{itemize}

 В~профиль включают факторы 
приоритетов при принятии решений, основные мотивации и~др. В~профиле 
долж\-ны быть сформулированы предельно конкретно со шкалами~измерений 
компетенции, которые необходимы, желательны или безразличны, в~част\-ности 
для роли специалиста в~коллективе, фор\-ми\-ру\-ющем\-ся для выработки решения 
по некоторой проб\-ле\-ме~\cite{6-r, 25-r}.\\[-14pt]
\item Первичный отбор~\cite{7-r} начинается с~анализа спис\-ка кандидатов 
с~точ\-ки зрения их соответствия общим требованиям по\-сред\-ст\-вом 
анкетирования, тес\-ти\-ро\-ва\-ния или испытания, графологического анализа 
(экспертиза почерка и~стиля изложения), морфологического анализа и~анализа 
по фотографии.\\ [-14pt]
\item На втором этапе отбора в~основном используют~\cite{26-r, 27-r}:
\begin{itemize}
\item комплексные исследования в~цент\-рах оценки персонала~--- 
оценки одних и~тех же критериев в~разных ситуациях и~различными методами, а~также 
ролевые и~имитационные деловые игры и~анализ конкретных ситуаций 
(моделируются существенные моменты дея\-тель\-ности и~оцениваются реальные 
достижения ис\-пы\-ту\-емых или де\-монст\-ри\-ру\-емое поведение)~\cite{27-r};\\[-14pt]
\item тесты: 
\begin{itemize}
\item[(а)] на проф\-при\-год\-ность~--- оценка психофизиологических качеств 
человека, умений выполнять определенную дея\-тель\-ность;
\item[(б)] общие~--- оценка 
общего уров\-ня развития и~особенностей мыш\-ле\-ния, внимания, памяти и~других 
высших психических функций; 
\item[(в)] био\-гра\-фи\-че\-ские тес\-ты и~изучение 
био\-гра\-фии~--- анализируются семейные отношения, характер образования, 
физическое развитие, глав\-ные по\-треб\-но\-сти и~интересы, осо\-бен\-но\-сти 
интеллекта, об\-щи\-тель\-ность; 
\item[(г)] личностные тес\-ты~--- психодиагностические 
тес\-ты на оцен\-ку уровня отдельных качеств и~пред\-рас\-по\-ло\-жен\-ность 
к~определенному типу поведения;\\[-14pt]
\end{itemize}
\item интервью~--- беседа, на\-прав\-лен\-ная на сбор информации об опыте 
и~уров\-не знаний и~на оценку профессионально важ\-ных качеств претендента;
\item также анализируют рекомендации (их источники и~оформление) 
и~используют: полиграф, медицинские тес\-ты, психоанализ.\\[-14pt]
\end{itemize}
\end{enumerate}
  
  \textbf{Методика подбора персонала О.\,С.~Насташевской с~использованием 
психограмм и~профессиограмм}~\cite{28-r}. Методика базируется на выделении 
типов лич\-ности по уровню отклонений от тео\-ре\-ти\-че\-ской психограммы через 
сис\-те\-му отбора. Сначала разрабатывается психограмма на основе 
\mbox{профессиограммы} специалиста, на которого объявлен отбор,~--- пе\-ре\-чис\-ля\-ют\-ся 
психологические профессионально важ\-ные качества специалиста~\cite{7-r}. 
Затем выбираются диагностические методики и~проводится со\-бе\-се\-до\-ва\-ние-тес\-ти\-ро\-ва\-ние, 
по результатам которого определяют степень соответствия 
кандидатов требованиям психограммы и~по ним выделяют типы лич\-ности 
(кластерный анализ, $k$-сред\-них). Кластеризация предполагает неаддитивную 
модель учета соответствия психограмме, что повышает адекват\-ность 
получаемого решения. В~итоге кандидаты, тип лич\-ности которых 
с~максимальной степенью соответствует требованиям, об\-суж\-да\-ют\-ся 
руководством.
  
  \textbf{Психологическая оценка персонала при выдвижении в~кад\-ро\-вый 
резерв}~\cite{29-r}.
  Кадровый резерв составляет группа сотрудников, которая долж\-на быть 
обучена внут\-рен\-ни\-ми и~внеш\-ни\-ми экспертами. Создается для обеспечения 
гиб\-кости в~замещении сотрудников. Для выдвижения в~кадровый резерв 
проводятся: оценка профессиональных компетенций, психологическое 
тес\-ти\-ро\-ва\-ние и~экспертное оценивание в~ходе деловых игр (ассесс\-мент-центр). 
Принцип включения в~кадровый резерв базируется на двух основных 
измерениях: профессиональном и~управ\-лен\-че\-ском потенциале, а~также 
эф\-фек\-тив\-ности в~дея\-тель\-ности. Оценивается уровень сле\-ду\-ющих базовых 
психологических характеристик: интеллектуальный уровень (тест структуры 
интеллекта Р.~Амтхауэра, методика оценки социального интеллекта 
Дж.~Гилфорда); лидерский потенциал (калифорнийский личностный опросник
(КЛО); стандартизированный метод исследования лич\-ности Л.\,Н.~Собчик 
(СМИЛ));\linebreak коммуникабельность (КЛО, СМИЛ); психологическая устой\-чи\-вость 
(опросник эмоционального интеллекта Д.\,В.~Люсина, СМИЛ); этич\-ность, 
по\-ря\-доч\-ность во взаимодействии с~коллегами, \mbox{подчиненными}, руководством 
(экспертная оценка и~сбор информации). Со\-труд\-ни\-ка мож\-но за\-чис\-лять 
в~кадровый резерв уже при одновременном наличии сред\-них показателей по 
эф\-фек\-тив\-ности и~потенциалу.
  
  \textbf{Формирование команд}~\cite{10-r}. Выделяют:
  \begin{enumerate}[(1)]
  \item динамический 
подход, на\-прав\-лен\-ный на развитие социоэмоциональных и~инструментальных 
отношений в~команде по\-сред\-ст\-вом различных тренингов~--- базируются на 
стадиях развития групп (M.~Kelly, W.~Wellins, B.\,W.~Tuckman, 
Т.\,Ю.~Базаров)~--- модели здесь дескриптивные; 
\item специально 
организованные со\-ци\-аль\-но-пси\-хо\-ло\-ги\-че\-ские технологии 
формирования коллективного субъекта дея\-тель\-ности~--- команды, а~именно: 
командные испытания для развития эмоциональных отношений; тренинги 
навыков командной работы; командный коучинг; деловые игры, тренинги по 
разработке общего видения. 
\end{enumerate}
  
  При комплектовании/пе\-ре\-комп\-лек\-то\-ва\-нии команд превалирует 
ориентация на формирование гетерогенных групп~\cite{30-r}: 
\begin{enumerate}[(1)]
\item по полу, 
возрасту и~профессии~--- не вызывает трудностей; 
\item по интеллекту 
и~личностным чертам~--- используются тес\-ты оценки IQ, методики 
диагностики когнитивного стиля и~креативности мышления, учитывающие 
специфику дея\-тель\-ности фор\-ми\-ру\-ющей\-ся команды (например, тесты 
Фланагана и~Векслера), а~так\-же методики, основанные на типологическом 
подходе К.\,Г.~Юнга (модели Май\-ерс--Бриггс и~Кейрси~--- включают 
си\-ту\-а\-ци\-он\-но-по\-ве\-ден\-че\-ское тес\-ти\-ро\-ва\-ние и/или глубинное био\-гра\-фи\-че\-ское 
ин\-тервью), и~методики, по\-стро\-ен\-ные на базе концепции командных ролей 
Р.\,М.~Белбина~--- используют со\-по\-став\-ле\-ние результатов применения 
опросника и~внеш\-них оценок, полученных от коллег,~--- 360-гра\-дус\-ная 
об\-рат\-ная связь.
\end{enumerate}
 Диагностика ценностной ориентации осуществляется, 
например, с~по\-мощью проективных методик, об\-ра\-ща\-ющих\-ся непосредственно 
к~символическому содержанию, в~котором объединены и~образ, и~отношение 
к~организации, и~личный опыт, и~ценности, и~переживания. 
  
  За рубежом выделяют четыре основных подхода к~образованию 
команд~\cite{31-r}:
  \begin{enumerate}[(1)]
  \item  основанный на развитии и~согласовании целей команды (E.\,A.~Locke, 
E.~Weldon и~др.)~--- развитие спо\-соб\-ности группы людей достигать своих 
целей;
  \item интерперсональный, ориентированный на анализ процессов 
  и~улучшение межличностных отношений (C.~Argyris, W.~Schutz и~др.); 
  \item ролевой подход~--- улучшение работы команды за счет увеличения 
яс\-ности ролей и,~как следствие, увеличения организационной эф\-фек\-тив\-ности. 
Пред\-ва\-ри\-тел\-ьно сотрудников тес\-ти\-ру\-ют, определяют типы их поведения, 
а~далее со\-еди\-ня\-ют в~команды по принципу взаи\-мо\-до\-пол\-ня\-емости (модели 
Р.\,М.~Белбина и~Т.\,Ю.~Базарова; взаи\-мо\-до\-пол\-ня\-ющая команда по 
И.\,А.~Адизес)~\cite{3-r, 32-r};
  \item проблемно-ори\-ен\-ти\-ро\-ван\-ный подход (W.\,G.~Dyer, R.~Kilman, 
I.~Kilman и~др.)~--- более общий, может вклю\-чать все предыду\-щие. Считается, 
что команда становится более эффективной в~результате совместного решения 
проб\-лем. 
  \end{enumerate}
  
  Результаты анализа подходов из разд.~3 сведены в~таб\-лицу.
  
  
  В таблице курсивом отмечены подходы, которые могут быть 
в~адаптированном виде использованы при моделировании коллективного 
принятия\linebreak\vspace*{-12pt}

\pagebreak

\end{multicols}

 \begin{table*}\small
  \begin{center}
 \tabcolsep=1pt
  \begin{tabular}{|l|c|c|c|c|c|}
  \multicolumn{6}{c}{Анализ отечественных и~зарубежных подходов к~отбору 
специалистов}\\
   \multicolumn{6}{c}{\ }\\[-6pt]
   \hline
\multicolumn{1}{|c|}{Основные подходы к~отбору 
специалистов}&\tabcolsep=0pt\begin{tabular}{c}Качест-\\венные\end{tabular}&
\tabcolsep=0pt\begin{tabular}{c}Количе-\\ ственные\end{tabular}&
\tabcolsep=0pt\begin{tabular}{c}Индиви-\\дуальные\end{tabular}&\tabcolsep=0pt\begin{tabular}{c}Груп-\\ повые\end{tabular}& 
\tabcolsep=0pt\begin{tabular}{c}Возможность\\ моделирования\\ в~рамках\\ 
РАСИГИА\end{tabular}\\
\hline
\textit{Согласованность функционально-ролевых  
ожиданий}~\cite{15-r, 16-r}&+&&&+&$\pm$\\
\hline
\textit{Ценностно-ориентационное единство по А.\,В.~Петровскому}~\cite{17-r}&&+&&+&$\pm$\\
\hline
Комплектование групп с~использованием ОМО~\cite{18-r}&&+&&+&\\
\hline
Соционический метод~\cite{19-r, 20-r}&&+&&+&\\
\hline
Социометрическая теория Я.~Морено~\cite{21-r} &&+&+&+&\\
\hline
\textit{Метод формирования профиля должности}~\cite{7-r}&&+&+&&$\pm$\\
\hline
\textit{Анкетирование и~тестирование}~\cite{7-r}&+&+&+&&$\pm$\\
\hline
Графологический и~морфологический анализ~\cite{7-r} &+&&+&&\\
\hline
\textit{Комплексные исследования в~центрах оценки персонала}~\cite{27-r}&+&+&+&&$\pm$\\
\hline
Биографические тесты и~изучение биографии~\cite{27-r}&+&&+&&\\
\hline
Интервью~\cite{24-r, 25-r, 26-r, 27-r}&+&+&+&&\\
\hline
Ролевые и~имитационные деловые игры~\cite{27-r}&+&+&+&+&\\
\hline
\textit{Методика подбора персонала О.\,С.~Насташевской}~\cite{28-r}&&+&&&$\pm$\\
\hline
Психологическая оценка персонала для кадрового резерва~\cite{29-r} &+&+&+&&\\
\hline
\textit{Динамический подход к~командообразованию}~\cite{10-r}&+&+&&+&$\pm$\\
\hline
\tabcolsep=0pt\begin{tabular}{l}Специальные организованные социально-психологические\\ технологии формирования 
команд~\cite{10-r}\end{tabular}&+&&&+&\\
\hline
\tabcolsep=0pt\begin{tabular}{l}Тесты и~методики оценки когнитивной сферы для\\
формирования гетерогенных групп~\cite{30-r}\end{tabular} &+&+&+&&\\
\hline
Проективные методики~\cite{30-r}&+&+&+&&\\
\hline
\textit{Развитие и~согласование целей команды}~\cite{31-r}&+&+&&+&$\pm$\\
\hline
Интерперсональный подход к~командообразованию~\cite{31-r}&+&+&&+&\\
\hline
\textit{Ролевой подход объединения в~команды}~\cite{30-r, 32-r}&+&+&&+&$\pm$\\
\hline
\textit{Проблемно-ориентированный подход к~формированию команд}&+&+&&&$\pm$\\
\hline
\multicolumn{6}{p{162mm}}{\footnotesize \hspace*{2mm}\textbf{Обозначения:} $\pm$~---  в~рамках 
соответствующего подхода есть ограничения и/или противоречия по выделенному критерию.}
\end{tabular}
\end{center}
\vspace*{-6pt}
\end{table*}
 

\begin{multicols}{2}

\noindent
 решения в~РАСИГИА для отбора моделей специалистов (агентов).

\section{Качество измерительного инструмента отбора 
специалистов}

  Одна из проблем отбора~--- повышение прог\-но\-стич\-ности, на\-деж\-ности 
процедуры выделения таких характеристик человека, которые укажут на его 
по\-сле\-ду\-ющую успеш\-ную профессиональную дея\-тель\-ность и~соответствие 
запросам организации, т.\,е.\ необходимо обоснование качества вы\-бран\-но\-го 
измерительного инструмента (анкеты, тес\-та и~т.\,д.)~\cite{14-r, 28-r}. При этом 
оцениваются:
  \begin{itemize}
\item эмпирическая валидность инструмента, т.\,е.\ соответствие его 
результатов характеристике, для измерения которой он разработан,~--- 
проводится пилотное исследование, в~ходе которого респонденты оценивают 
ис\-сле\-ду\-емый объект при помощи альтернативных анкет. Связь между 
результатами измерения определяется расчетом коэффициента ранговой 
корреляции Спирмена~\cite{14-r};
\item надежность инструмента: 
\begin{enumerate}[(1)]
\item с позиции со\-гла\-со\-ван\-ности, т.\,е.\ степени 
однородности со\-ста\-ва вопросов (заданий) с~точ\-ки зрения из\-ме\-ря\-емой 
характеристики~--- определяется связью каждого конкретного элемента 
инструмента с~общим результатом~\cite{33-r}; 
\item с позиции устой\-чи\-вости~--- 
проводится несколько измерений с~некоторым промежутком времени одним 
и~тем же инструментом. Инструмент устойчив, если имеет мес\-то 
статистически значимое значение коэффициента корреляции между данными 
измерений и~статистически незначимые различия в~сред\-них значениях, 
полученных при измерениях.
\end{enumerate}
\end{itemize}

\vspace*{-6pt}

\section{Заключение}

  В работе представлены результаты исследования, которое по материалам 
открытой печати выявило большое разнообразие подходов к~отбору 
специалистов на долж\-ность и~в~группу (команду, коллектив)~--- групповые, 
индивидуальные, качественные, количественные и~комбинированные. При этом 
инструменты оценки кандидатов должны подвергаться проверке на ва\-лид\-ность
 и~на\-деж\-ность. На практике часто комбинируют несколько методов, чтобы 
повысить качество отбора. По результатам анализа так\-же были выделены 
несколько подходов (при условии их адаптации), которые могут быть в~той или 
иной степени использованы для отбора моделей специалистов (агентов) 
в~искусственный гетерогенный коллектив \mbox{РАСИГИА}\linebreak из всего пула 
интеллектуальных агентов: со\-гла\-со\-ван\-ность функ\-цио\-наль\-но-ро\-ле\-вых 
ожиданий; цен\-ност\-но-ори\-ен\-та\-ци\-он\-ное единство; формирование\linebreak 
профиля долж\-ности; анкетирование и~тес\-ти\-ро\-ва\-ние; принципы комплексных 
исследований в~цент\-рах; методика О.\,С.~На\-ста\-шев\-ской; развитие 
и~согласование целей команды; динамический, \mbox{ролевой} и~проб\-лем\-но-ори\-ен\-ти\-ро\-ван\-ный подходы.
  
{\small\frenchspacing
 {%\baselineskip=10.8pt
 %\addcontentsline{toc}{section}{References}
 \begin{thebibliography}{99}
\bibitem{1-r}
\Au{Куроедова Е.\,О.} Ин\-тер\-нет-курс по дисциплине <<Психология в~управлении  персоналом>>. 
{\sf   http://www.\linebreak \mbox{e-biblio}.ru/book/bib/04\_pravo/psiholog\_v\_uprav\_\linebreak personalom/sg\_online.html}.
\bibitem{2-r}
\Au{Кричевский Р.\,Л., Ду\-бов\-ская Е.\,М.} Социальная психология малой группы.~--- М.: Аспект Пресс, 2001. 318~с.
\bibitem{3-r}
\Au{Колесников А.\,В.} Гетерогенные естественные и~искусственные сис\-те\-мы~// 
Интегрированные модели и~мягкие вы\-чис\-ле\-ния в~искусственном интеллекте.~--- М.: Физматлит, 2013. Т.~1. 
С.~86--103.
\bibitem{4-r}
\Au{Андреева Г.\,М.} Социальная психология.~--- М.: Аспект Пресс, 2009. 393~с.
\bibitem{5-r}
\Au{Меньшиков А.\,А.} Основы интегрированных коммуникаций.~---  
Ком\-со\-мольск-на-Аму\-ре: КнАГУ, 2012. 101~с.
\bibitem{6-r}
\Au{Коноваленко В.\,А., Коноваленко~М.\,Ю., Соломатин~А.\,А.} Психология управления 
персоналом.~--- М.: Юрайт, 2014. 477~с.
\bibitem{7-r}
Психология управ\-ле\-ния персоналом~/ Под ред. Е.\,И.~Рогова.~--- М.: 
Юрайт, 2023. 350~с.



\bibitem{9-r} %8
Социальная психология в~современном мире~/ Под ред. 
Г.\,М.~Андреевой, А.\,И.~Донцова.~--- М.: Аспект Пресс, 2002. 335~с.
\bibitem{10-r} %9
\Au{Короткина Е.\,Д.} Современные технологии создания команды в~организации~// 
Вестник Санкт-Пе\-тер\-бург\-ско\-го университета. Сер.~12. Психология.  Социология. Педагогика, 2009. Т.~3. №\,2. С.~46--53.

\bibitem{8-r} %10
\Au{Нургалиева А.\,М., Ахметшина~А.\,Р., Сайфудинова~Н.\,З.} Современные методики 
формирования эффективной команды в~организации~// СКИФ. Вопросы студенческой науки, 
2018. Вып.~11(27). С.~221--230.

\bibitem{11-r}
\Au{Katzenbach J.\,R., Smith D.\,K.} The discipline of teams~// Harward Business Review, 1993. 
Vol.~71. Iss.~2. P.~111--120.
\bibitem{12-r}
\Au{Картушина T.\,Н.} Командообразование как потребность в~современном процессе 
управления персоналом~// Cо\-ци\-аль\-но-эко\-но\-ми\-че\-ские явления и~процессы, 2013. 
№\,5(051). С.~99--102. 
\bibitem{13-r}
\Au{Жаглин А.\,В., Ульянов А.\,Д.} Основы управ\-ле\-ния и~делопроизводства в~органах 
внут\-рен\-них дел: Альбом схем.~--- М.: Юни\-ти-Да\-на, 2014. 191~с.
\bibitem{14-r}
\Au{Горленко О.\,А., Ерохин Д.\,В., Можаева~Т.\,П.} Управление персоналом.~--- М.: Юрайт, 2023. 217~с.
\bibitem{15-r}
\Au{Кабаченко Т.\,С.} Психология в~управ\-ле\-нии человеческими ресурсами.~--- СПб.: Питер, 2003. 400~с.
\bibitem{16-r}
\Au{Mescon M.\,H., Albert~M., Khedouri~F.} Management.~--- New York, NY, USA: Harper \& 
Row Publs., 1988. 777~p.
\bibitem{17-r}
Психологическая теория коллектива~/ Под ред. А.\,В.~Пет\-ров\-ско\-го.~--- М.: Педагогика, 1979. 
240~с.
\bibitem{18-r}
Рабочая книга практического психолога~/ Под ред. А.\,А.~Бодалева, А.\,А.~Деркача, Л.\,Г.~Лаптева.~--- М.: Изд-во 
Института психотерапии, 2001. 640~с.
\bibitem{19-r}
\Au{Иванов Ю.\,В.} Деловая соционика.~--- М.: Топ-пер\-со\-нал, 2004. 200~с.
\bibitem{20-r}
\Au{Филатова Е.\,С.} Соционика в~портретах и~примерах.~--- М.: Черная белка, 2009. 443~с.
\bibitem{21-r}
\Au{Миронова Е.\,Е.} Сборник психологических тес\-тов. Часть~I: Пособие.~--- Мн.: Женский 
институт \mbox{ЭНВИЛА}, 2005. 155~с.


\bibitem{23-r} %22
Лучшие психологические тесты для профотбора и~профориентации~/ Отв. ред. 
А.\,Ф.~Кудряшов.~--- Петрозаводск: Петроком, 1992. 318~с.
\bibitem{24-r} %23
Психологические тесты~/ Под ред. А.\,А.~Карелина: в~2~т.~--- М.: Владос, 2002. Т.~1. 312~с. 
Т.~2. 246~с.
\bibitem{22-r} %24
\Au{Елисеев О.\,П.} Практикум по психологии личности.~--- М.: Юрайт, 
2023. 390~с.

\bibitem{25-r}
\Au{Иванова С.\,В.} Искусство подбора персонала: как оценить человека за час.~--- М.: 
Альпина Паблишер, 2012. 269~с.
\bibitem{26-r}
\Au{Мякушкин Д.\,Е.} Отбор и~под\-бор персонала.~--- Челябинск: ЮУрГУ, 2006. 
26~с.
\bibitem{27-r}
\Au{Базаров Т.\,Ю.} Технология цент\-ров оцен\-ки персонала: процессы и~результаты. 
Практическое пособие.~--- М.: КноРус, 2021. 301~с.
\bibitem{28-r}
\Au{Насташевская О.\,С.} Психологические аспекты технологии подбора персонала для 
торговой организации~// Вестник Самарской гуманитарной академии. Сер. Психология, 
2015. №\,1(17). С.~11--29.
\bibitem{29-r}
\Au{Васильева И.\,В.} Психотехники и~пси\-хо\-диа\-гно\-сти\-ка в~управ\-ле\-нии персоналом: 
Практическое пособие.~--- М.: Юрайт, 2023. 122~с.
\bibitem{30-r}
\Au{Жуков Ю.\,М., Журавлев А.\,В., Павлова~Е.\,Н.} Технологии командообразования.~--- М.: 
Аспект Пресс, 2008. 320~с.

\pagebreak

\bibitem{31-r}
\Au{Безрукова Е.\,Ю.} Информационно-ме\-то\-ди\-че\-ское обеспечение процесса 
командообразования:\linebreak Дисс.\ \ldots\  канд. псих. наук.~--- М., 1998. 289~с.
\bibitem{32-r}
\Au{Семина А.\,П.} Анализ моделей и~подходов в~формировании команды компании~// 
Вестник Алтайской академии экономики и~права, 2020. №\,12-2. С.~399--404. doi: 
10.17513/vaael.1526.
\bibitem{33-r}
\Au{Яхонтова Е.\,С.} Стратегическое управ\-ле\-ние персоналом.~--- М.: 
Дело, 2013. 378~с.

\end{thebibliography}

 }
 }

\end{multicols}

\vspace*{-6pt}

\hfill{\small\textit{Поступила в~редакцию 05.04.23}}

\vspace*{8pt}

%\pagebreak

%\newpage

%\vspace*{-28pt}

\hrule

\vspace*{2pt}

\hrule

%\vspace*{-2pt}

\def\tit{SELECTION OF SPECIALISTS IN~THE~ORGANIZATION OF~COLLECTIVE SOLVING 
PROBLEMS}


\def\titkol{Selection of specialists in~the~organization of~collective solving 
problems}


\def\aut{S.\,B.~Rumovskaya}

\def\autkol{S.\,B.~Rumovskaya}

\titel{\tit}{\aut}{\autkol}{\titkol}

\vspace*{-10pt}


\noindent
Federal Research Center ``Computer Science and Control'' of the Russian Academy 
of Sciences, 44-2~Vavilov Str., Moscow 119333, Russian Federation


\def\leftfootline{\small{\textbf{\thepage}
\hfill INFORMATIKA I EE PRIMENENIYA~--- INFORMATICS AND
APPLICATIONS\ \ \ 2023\ \ \ volume~17\ \ \ issue\ 2}
}%
 \def\rightfootline{\small{INFORMATIKA I EE PRIMENENIYA~---
INFORMATICS AND APPLICATIONS\ \ \ 2023\ \ \ volume~17\ \ \ issue\ 2
\hfill \textbf{\thepage}}}

\vspace*{3pt}
   
   
      
   \Abste{The study of small groups (collectives, teams), their characteristics, problems, dynamics, and 
features of selection of specialists stands at the intersection of psychology of personnel management and 
social psychology. A~special place in a~wide range of areas of modern science is occupied by modeling the 
interaction of people in small collectives of specialists, in particular, within the framework of a~multiagent 
approach. At the same time, when developing intelligent systems (artificial heterogeneous collectives) 
to solve practical problems, it is now required to combine in the system the models of specialists (agents) 
with incompatible goals and domain models. These agents are created by different development teams. The 
selection of specialists in natural and models of specialists in artificial heterogeneous teams is an important 
task, the results of which influence the further decision-making process. The paper presents an analysis of 
methods and approaches to the selection of specialists and the acquisition of small groups (collectives, 
teams) whose measuring tools should be exposed to quality assessment.}
   
   \KWE{group; small collective of specialists; team; methods of selecting specialists and forming small 
groups; teambuilding}
   
   
   
\DOI{10.14357/19922264230214}{VJWNOE} 

\vspace*{-11pt}

\Ack
   \noindent
   The research was supported by the Russian Science Foundation (project No.\,23-21-00218).
  

%\vspace*{4pt}

  \begin{multicols}{2}

\renewcommand{\bibname}{\protect\rmfamily References}
%\renewcommand{\bibname}{\large\protect\rm References}

{\small\frenchspacing
 {%\baselineskip=10.8pt
 \addcontentsline{toc}{section}{References}
 \begin{thebibliography}{99} 
\bibitem{1-r-1}
   \Aue{Kuroedova, E.\,O.} Internet-kurs po dis\-tsip\-li\-ne ``Psi\-kho\-lo\-giya v~uprav\-le\-nii per\-so\-na\-lom'' 
[Online course on the discipline ``Psychology in personnel management'']. Available at: {\sf 
http://www.e-biblio.ru/book/bib/04\_pravo/ psiholog\_v\_uprav\_personalom/sg\_online.html} (accessed 
May~11, 2023).
\bibitem{2-r-1}
   \Aue{Krichevskiy, R.\,L., and E.\,M.~Dubovskaya}. 2001. \textit{So\-tsi\-al'\-naya psi\-kho\-lo\-giya 
ma\-loy grup\-py} [Social psychology of a~small group]. Moscow: 
Aspect Press. 318~p.
\bibitem{3-r-1}
   \Aue{Kolesnikov, A.\,V.} 2013. Ge\-te\-ro\-gen\-nye es\-test\-ven\-nye i~is\-kus\-stven\-nye sis\-te\-my [Natural 
and artificial heterogeneous systems]. \textit{In\-teg\-ri\-ro\-van\-nye mo\-de\-li i~myag\-kie vy\-chis\-le\-niya 
v~iskusstvennom intellekte} [Integrated models and oft computing in artificial intelligence]. 
Moscow: Fizmatlit. 1:86--103.
\bibitem{4-r-1}
   \Aue{Andreeva, G.\,M.} 2009. \textit{So\-tsi\-al'\-naya psi\-kho\-lo\-giya} [Social psychology]. Moscow: 
Aspect Press. 393~p.
\bibitem{5-r-1}
\Aue{Men'shikov, A.\,A.} 2012. \textit{Os\-no\-vy in\-teg\-ri\-ro\-van\-nykh kom\-mu\-ni\-ka\-tsiy} 
[Fundamentals of integrated communications]. Komsomolsk-on-Amur: 
KnAGU. 101~p.
\bibitem{6-r-1}
\Aue{Konovalenko, V.\,A., M.\,Yu.~Konovalenko, and A.\,A.~Solomatin.} 2014. 
\textit{Psi\-kho\-lo\-giya uprav\-le\-niya per\-so\-na\-lom} 
[Psychology of human resources management]. 
Moscow: Yurayt. 477 p.
\bibitem{7-r-1}
   Rogov, E.\,I., ed.  2023. \textit{Psi\-kho\-lo\-giya uprav\-le\-niya per\-so\-na\-lom} 
   [Psychology of human resources management]. Moscow: 
Yurayt. 350~p.


\bibitem{9-r-1} %8
   Andreeva, G.\,M., and A.\,I.~Dontsov, eds. 2002. \textit{So\-tsi\-al'\-naya psi\-kho\-lo\-giya 
v~so\-vre\-men\-nom mi\-re} [Social psychology in the modern world]. Moscow: Aspect Press. 335~p.
\bibitem{10-r-1} %9
   \Aue{Korotkina, E.\,D.} 2009. So\-vre\-men\-nye tekh\-no\-lo\-gii so\-zda\-niya ko\-man\-dy v~or\-ga\-ni\-za\-tsii 
[Modern approaches to teambuilding in organization]. \textit{Vestnik Sankt-Peterburgskogo 
universiteta. Ser.~12. Psikhologiya. Sotsiologiya. Pedagogika} [Vestnik of Saint Petersburg University. Ser.~12.
Psychology. Sociology. Pedagogy] 3(2):46--53.

\bibitem{8-r-1} %10
   \Aue{Nurgalieva, A.\,M., A.\,R.~Akhmetshina, and N.\,Z.~Sayfudinova.} 2018. So\-vre\-men\-nye 
me\-to\-di\-ki for\-mi\-ro\-va\-niya ef\-fek\-tiv\-noy ko\-man\-dy v~or\-ga\-ni\-za\-tsii [Modern methods of forming an 
effective team in the organization]. \textit{Skif. Voprosy stu\-den\-che\-skoy na\-u\-ki} [Skif. Issues of 
Student Science] 11(27):221--230.

\bibitem{11-r-1}
   \Aue{Katzenbach, J.\,R., and D.\,K.~Smith.} 1993. The discipline of teams. \textit{Harward 
Business Review} 71(2):111--120.
\bibitem{12-r-1}
   \Aue{Kartushina, T.\,N.} 2013. Ko\-man\-do\-ob\-ra\-zo\-va\-nie kak po\-treb\-nost' v~so\-vre\-men\-nom 
pro\-tses\-se uprav\-le\-niya per\-so\-na\-lom [Teambuilding as need in modern HR management]. 
\textit{Sotsial'no-ekonomicheskie yavleniya i~protsessy} [Social-Economic Phenomena and 
Processes] 5(051):99--102.
\bibitem{13-r-1}
   \Aue{Zhaglin, A.\,V., and A.\,D.~Ul'yanov.} 2014. \textit{Osno\-vy uprav\-le\-niya 
i~de\-lo\-pro\-iz\-vod\-st\-va v~or\-ga\-nakh vnut\-ren\-nikh del: Al'bom skhem} 
[Fundamentals of management and office work in the internal affairs bodies: Album of schemes]. Moscow: Unity-Dana. 191~p.
\bibitem{14-r-1}
   \Aue{Gorlenko, O.\,A., D.\,V.~Erokhin, and T.\,P.~Mozhaeva.} 2023. \textit{Uprav\-le\-nie 
per\-so\-na\-lom} [Human resource 
management]. Moscow: Yurayt. 217~p.
\bibitem{15-r-1}
   \Aue{Kabachenko, T.\,S.} 2003. \textit{Psi\-kho\-lo\-giya v~uprav\-le\-nii che\-lo\-ve\-che\-ski\-mi re\-sur\-sa\-mi} 
   [Psychology in human resource management]. Saint Petersburg: 
Piter Publishing House. 400~p.
\bibitem{16-r-1}
   \Aue{Mescon, M.\,H., M.~Albert, and F.~Khedouri.} 1988. \textit{Management}. New York, 
NY: Harper \& Row Publs. 777~p.
\bibitem{17-r-1}
   Petrovskiy, A.\,V., ed. 1979. \textit{Psi\-kho\-lo\-gi\-che\-skaya teo\-riya kol\-lek\-ti\-va} [Psychological 
theory of the team]. Moscow: Pedagogika. 240~p.
\bibitem{18-r-1}
   Bodalev, A.\,A., A.\,A.~Derkach, and L.\,G.~Laptev, eds. 2001. \textit{Ra\-bo\-chaya kni\-ga 
prak\-ti\-che\-sko\-go psi\-kho\-lo\-ga} [Practical 
psychologist's workbook]. Moscow: Publishing house of the 
Institute of Psychotherapy Publs. 640~p.
\bibitem{19-r-1}
   \Aue{Ivanov, Yu.\,V.} 2004. \textit{De\-lo\-vaya so\-tsi\-o\-ni\-ka} [Business socionics]. Moscow:  
Top-personal. 200~p.
\bibitem{20-r-1}
   \Aue{Filatova, E.\,S.} 2009. \textit{So\-tsi\-o\-ni\-ka v~portre\-takh i~pri\-me\-rakh} [Socionics in portraits 
and examples]. Moscow: Chernaya belka. 443~p.
\bibitem{21-r-1}
   \Aue{Mironova, E.\,E.}  2005. \textit{Sbor\-nik psi\-kho\-lo\-gi\-che\-skikh tes\-tov. Chast'~I: Posobie} 
[Collection of psychological tests. Part~I: Manual].  Minsk: Zhenskiy Institut ENVILA
[ENVIL Women's Institute]. 155~p.

\bibitem{23-r-1} %22
   Kudryashov, A.\,F., ed. 1992. \textit{Luch\-shie psi\-kho\-lo\-gi\-che\-skie tes\-ty dlya prof\-ot\-bo\-ra 
i~prof\-ori\-en\-ta\-tsii} [The best psychological tests for vocational selection and vocational guidance]. 
Petrozavodsk: Petrokom. 318~p.
\bibitem{24-r-1} %23
   Karelin, A.\,A., ed. 2002. \textit{Psi\-kho\-lo\-gi\-che\-skie tes\-ty}: v~2 tomakh [Psychological tests in 
two volumes]. Moscow: Vlados. Vol.~1. 312~p. Vol.2. 246~p.

\bibitem{22-r-1} %24
   \Aue{Eliseev, O.\,P.} 2023. \textit{Prak\-ti\-kum po psi\-kho\-lo\-gii lich\-nosti} 
[Practical work on personality psychology]. Moscow: 
Yurayt. 390~p.

\bibitem{25-r-1}
   \Aue{Ivanova, S.\,V.} 2012. \textit{Is\-kus\-stvo pod\-bo\-ra per\-so\-na\-la: kak otse\-nit' che\-lo\-ve\-ka za 
chas} [The art of recruiting: How to evaluate a~person in an hour]. Moscow: Alpina Publisher. 
269~p.
\bibitem{26-r-1}
   \Aue{Myakushkin, D.\,E.} 2006. \textit{Ot\-bor i~pod\-bor personala} [Selection 
and recruitment of personnel]. Chelyabinsk: Publishing Center of South Ural State 
University. 26~p.
\bibitem{27-r-1}
   \Aue{Bazarov, T.\,Yu.} 2021. \textit{Tekh\-no\-lo\-giya tsent\-rov otsen\-ki per\-so\-na\-la: pro\-tses\-sy 
i~re\-zul'\-ta\-ty. Prakticheskoe posobie} [Technology of personnel assessment centers: Processes and 
results. Practical guide]. Moscow: KnoRus. 301~p.
\bibitem{28-r-1}
   \Aue{Nastashevskaya, O.\,S.} 2015. Psi\-kho\-lo\-gi\-che\-skie as\-pek\-ty tekh\-no\-lo\-gii pod\-bo\-ra per\-so\-na\-la 
dlya tor\-go\-voy or\-ga\-ni\-za\-tsii [Psychological aspects of technology recruitment for a~trade 
organization]. \textit{Vest\-nik Sa\-mar\-skoy gu\-ma\-ni\-tar\-noy aka\-de\-mii. Ser. Psi\-kho\-lo\-giya} [Bulletin of 
Samara Academy for the Humanities. Ser. Psychology] 1(17):11--29.
\bibitem{29-r-1}
   \Aue{Vasil'eva, I.\,V.} 2023. \textit{Psi\-kho\-tekh\-ni\-ki i~psi\-kho\-diag\-no\-sti\-ka v~uprav\-le\-nii 
per\-so\-na\-lom: Prakticheskoe posobie} [Psychotechnics and psychodiagnostics in personnel 
management: A~practical guide]. Moscow: Yurayt. 122~p.
\bibitem{30-r-1}
   \Aue{Zhukov, Yu.\,M., A.\,V.~Zhuravlev, and E.\,N.~Pavlova.} 2008. \textit{Tekh\-no\-lo\-gii 
ko\-man\-do\-ob\-ra\-zo\-va\-niya} [Teambuilding technologies]. Moscow: Aspect Press. 320~p.
\bibitem{31-r-1}
   \Aue{Bezrukova, E.\,Yu.} 1998. Informatsionno-metodicheskoe obes\-pe\-che\-nie pro\-tses\-sa 
ko\-man\-do\-ob\-ra\-zo\-va\-niya [Information and methodological support of the teambuilding process]. 
Moscow. PhD Diss. 289~p.
\bibitem{32-r-1}
   \Aue{Semina, A.\,P.} 2020. Ana\-liz mo\-de\-ley i~pod\-kho\-dov v~for\-mi\-ro\-vanii ko\-man\-dy kom\-pa\-nii 
[Analysis of models and approaches in the formation of team in company]. \textit{Vest\-nik Al\-tay\-skoy 
aka\-de\-mii eko\-no\-mi\-ki i~pra\-va} [Bulletin of the Altai Academy of Economics and Law]  
12-2:399--404. doi: 10.17513/vaael.1526.
\bibitem{33-r-1}
   \Aue{Yakhontova, E.\,S.} 2013. \textit{Stra\-te\-gi\-che\-skoe uprav\-le\-nie per\-so\-na\-lom} 
   [Strategic human resources management]. Moscow: Delo. 378~p.
   \end{thebibliography}

 }
 }

\end{multicols}

\vspace*{-6pt}

\hfill{\small\textit{Received April 5, 2023}} 
      
   
   \Contrl
   
   \noindent
   \textbf{Rumovskaya Sophiya B.} (b.\ 1985)~--- Candidate of Science (PhD) in technology, senior 
scientist, Kaliningrad Branch of the Federal Research Center ``Computer Science and Control'' of the 
Russian Academy of Sciences, 5~Gostinaya Str., Kaliningrad 236000, Russian Federation; 
\mbox{sophiyabr@gmail.com}
    
\label{end\stat}

\renewcommand{\bibname}{\protect\rm Литература} 
    