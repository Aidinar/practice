\def\stat{vasiliev}

\def\tit{КОМПОЗИЦИОНАЛЬНОЕ ПРЕДСТАВЛЕНИЕ СТРУКТУРЫ ИГРЫ МНОГИХ ЛИЦ 
В~МОНОИДАЛЬНОЙ КАТЕГОРИИ БИНАРНЫХ ОТНОШЕНИЙ}

\def\titkol{Композициональное представление структуры игры многих лиц 
в~моноидальной категории бинарных отношений}

\def\aut{Н.\,С.~Васильев$^1$}

\def\autkol{Н.\,С.~Васильев}

\titel{\tit}{\aut}{\autkol}{\titkol}

\index{Васильев Н.\,С.}
\index{Vasilyev N.\,S.}


%{\renewcommand{\thefootnote}{\fnsymbol{footnote}} \footnotetext[1]
%{Работа выполнена с~использованием инфраструктуры Центра коллективного пользования <<Высокопроизводительные вы\-чис\-ле\-ния и~большие данные>> 
%(ЦКП <<Информатика>>) ФИЦ ИУ РАН (г.~Москва).}}


\renewcommand{\thefootnote}{\arabic{footnote}}
\footnotetext[1]{Московский государственный технический университет им.\ Н.\,Э.~Баумана, \mbox{nik8519@yandex.ru}}

\vspace*{-10pt}


    \Abst{Предложен системный подход к~решению игры многих лиц, отвечающий 
современным сетевым технологиям. Он поз\-во\-ля\-ет оптимизировать 
функционирование муль\-ти\-агент\-ных сис\-тем. Моноидальная категория бинарных 
отношений применяется как средство описания правил игры, исследования 
и~модификации поведения игроков. Игровая проб\-ле\-ма со\-сто\-ит в~том, чтобы по 
воз\-мож\-ности максимизировать отношения предпочтения всех участ\-ни\-ков игры. 
В~соответствии с~правилами игры их композиция определяет ре\-зуль\-ти\-ру\-ющее 
отношение игры (РОИ). Поиск рационального поведения игроков сведен 
к~на\-хож\-де\-нию максимальных элементов РОИ. Формализовано использование 
разнообразных классов до\-пус\-ти\-мых стратегий, процессов обмена информацией 
меж\-ду игроками и~формирование коалиций. Доказано существование РОИ 
и~изуче\-на структура его максимальных элементов, со\-кра\-ща\-ющая поиск. Выяснено 
значение отношений предшествования ходов и~абсолютно оптимальных 
предпочтений игроков в~процессе формирования коалиций.}
    
    \KW{отношения предпочтения: абсолютно оптимальное, гарантированное, 
предшествования ходов; граф игры; до\-пус\-ти\-мая стратегия; рациональное решение; 
характеристическое отношение коалиции; ре\-зуль\-ти\-ру\-ющее отношение игры; 
моноидальная категория; ком\-по\-зи\-ци\-о\-наль\-ность}

  \DOI{10.14357/19922264230203}{GPMZTS}
  
\vspace*{2pt}


\vskip 10pt plus 9pt minus 6pt

\thispagestyle{headings}

\begin{multicols}{2}

\label{st\stat}
    
\section{Введение}

    Игровой задаче со~многими участниками присуще большое разнообразие 
по\-ста\-но\-вок, которые приходится динамически уточ\-нять в~процессе исследования, 
проводя анализ результатов воз\-мож\-ных рациональных решений~[1--5]. 
Востребовано новое, композициональное, пред\-став\-ле\-ние игры, от\-ве\-ча\-ющее сетевым 
технологиям мультиагентных сис\-тем, программно ре\-а\-ли\-зу\-емое для оптимизации их 
работы~[6]. Предложено формализовать игру средствами моноидальной категории 
бинарных отношений~[7, 8], так как традиционные нормальная\linebreak форма и~развернутое 
представление игры не обладают нуж\-ным качеством~\cite{1-vas, 9-vas}. В~отличие 
от применения платежных функ\-ций теперь решения принимаются на базе 
отношений на множестве \mbox{ситуаций} игры, которые учитывают правила игры 
и~динамику поведения парт\-не\-ров. Сход\-ны\-ми соображениями руководствуются 
в~многокритериальных задачах~\cite{3-vas} и~экономических приложениях тео\-рии 
игр~\cite{4-vas, 10-vas, 11-vas}. 
    
    \subsection*{Переход от~нормальной к~композициональной форме 
игры}

    Пусть каждый участник игровой задачи $i\hm \in I\hm= \{1, 2, \ldots ,n\}$ 
стремится по воз\-мож\-ности максимизировать свой критерий эф\-фек\-тив\-ности $w_i: 
X\hm\to R, X\hm\equiv X_1\times X_2\times \cdots \times X_n$, выбирая свой 
конт\-ро\-ли\-ру\-емый фактор $x_i\hm\in X_i$, $i\hm\in I$~\cite{1-vas, 2-vas, 5-vas}. Поиск 
рационального решения игры на множестве ситуаций $x\hm\in X$ проводится 
в~условиях не\-опре\-де\-лен\-ности, обуслов\-лен\-ной различием интересов игроков 
и~не\-воз\-мож\-ностью точ\-но\-го прогноза результата игры из-за незнания действий 
парт\-не\-ров. Поэтому агенты вы\-нуж\-де\-ны расширять классы до\-пус\-ти\-мых 
стратегий~$\tilde{X}_i$, $X_i\hm\subset \tilde{X}_i$, чтобы учитывать по\-сту\-па\-ющую 
информацию и~коалиционное поведение игроков~\cite{1-vas, 5-vas}. 

Срав\-не\-ние 
ситуаций игры мож\-но проводить не только с~по\-мощью критериев эф\-фек\-тив\-ности 
$w_i\hm= w_i(x)$, но и~применяя отношения предпочтения игроков $\rho_i\hm\subset 
X^2$, $i\hm\in I$. Точ\-ное знание интересов моделируется линейным порядком, 
а~нечеткие пред\-став\-ле\-ния~--- бинарными отношениями общего вида. 
 По-преж\-не\-му игроки стремятся по воз\-мож\-ности выбирать максимально 
предпочтительные для них ситуации игры. 
    
    Рациональное поведение агентов определено правилами игры, по\-рож\-да\-ющи\-ми 
из исходных~$\rho_i$,\linebreak $i\hm\in I$, вспомогательные бинарные отношения, \mbox{которые} 
выражают принципы оптимального поведения, открытые в~тео\-рии игр~[1--5], 
и~руководят действиями игроков. Производные отношения \mbox{учитывают} кооперацию 
и~ин\-фор\-ми\-ро\-ван\-ность игроков. Они конструируются посредством ал\-геб\-ра\-и\-че\-ских 
операций ${A}\hm= (\circ, \cup, \cap, {}^{\mathrm{op}}, \times; \sigma, \varnothing)$ 
и~композиции морфизмов категории бинарных отношений REL~[7, 8]. Так, обмен 
информацией между агентами всегда приводит к~сужению их отношений 
предпочтения $\rho\vert_A \hm= \rho\cap A^2$ на некоторую часть~$A$ множества 
ситуаций игры.
    
    \textbf{Пример~1.1.} В~бескоалиционной игре $\Gamma\hm= \Gamma^0$ 
участ\-ни\-ки не имеют никакой информации о~стратегиях партнеров и~отсутствует 
до\-го\-во\-рен\-ность об оче\-ред\-ности ходов. Тогда \textit{ре\-зуль\-ти\-ру\-ющим} отношением, 
которое в~конечном итоге требуется оптимизировать, будет сле\-ду\-ющая 
дизъ\-юнк\-тив\-ная сумма~[7, 8]:
    \begin{equation}
    \rho^{\Gamma}= \coprod\limits_{i\in I} \rho_i.
    \label{e1.1-vas}
    \end{equation}
    
    Коалиции игроков~$C$ создаются за счет коммуникации. Им отвечают 
некоторые подыг\-ры~$\Gamma^\prime$ исходной игры~$\Gamma$, в~которых 
фиксированы стратегии агентов $i \not= C$. Интересы коалиции пред\-став\-ле\-ны 
\textit{характеристическим} отношением~$\rho_C$, сов\-па\-да\-ющим 
с~РОИ~$\Gamma^\prime$. Конструкция~$\rho_C$ 
определяется исходя из правил игры и~принципа рационального поведения 
участ\-ни\-ков коалиции. 
    
    \textbf{Пример~1.2.} В коалиции Парето $C\hm= C^P$ все участники обладают 
полной информацией о~действиях друг друга и~со\-вмест\-но выбирают общий 
конт\-ро\-ли\-ру\-емый фактор $x_C\hm= (x_i, i\in C)$, стремясь к~максимизации 
сле\-ду\-юще\-го от\-но\-ше\-ния~\cite{3-vas}:
    \begin{equation}
    \rho^{\Gamma^\prime} = \rho_C = \mathop{\bigcap}\limits_{i\in C} \rho_i\,.
    \label{e1.2-vas}
    \end{equation}
    
    \textit{Абсолютно оптимальным} отношением $i$-го игрока назовем сужение 
его отношения предпочтения $a_i \hm= \rho_i\vert_{x_i}$ при фиксированных 
значениях всех остальных па\-ра\-мет\-ров $x_j\hm\in X_j$, $j\not= i$. Включение 
$(x,x^\prime)\hm\in a_i$ означает до\-сти\-жи\-мость более предпочтительной 
ситуации~$x^\prime$ из~$x$ ходом $i$-го игрока. Для игры в~нормальной форме 
этому понятию соответствует применение игроком абсолютно оптимальной 
стратегии~\cite{1-vas}.
    
    \textbf{Пример~1.3.} В~игре~$\Gamma^{0,n}$ участники коалиции $C\hm= 
C(\pi, \varnothing)$, $\pi \hm= \{ (l, l\hm+1), l=1,\ldots , n\hm-1\}$, не обмениваясь 
данными ($D\hm= \varnothing$), выполняют ходы в~линейном по\-рядке
\begin{equation}
\pi:\boxed{1}--\rightarrow \boxed{2}--\rightarrow\cdots--\rightarrow\boxed{n}\,.
\label{e1.3-vas}
\end{equation}
Тогда предпочтения коалиции задаются характеристическим от\-но\-ше\-нием
\begin{equation}
\rho^{\Gamma^{0,n}} =\rho_{C(\pi,\varnothing)} =a_n \circ a_{n-1} \circ \cdots \circ 
a_1\,.
\label{e1.4-vas}
\end{equation}
    
    В~(\ref{e1.4-vas}) использована композиция морфизмов категории 
REL~\cite{8-vas}. Рациональное поведение всякого игрока $i\hm\in C(\pi, 
\varnothing)$~--- это использование абсолютно оптимальной стратегии, от\-ве\-ча\-ющей 
стремлению вы\-брать ситуацию из множества~MAX\,$a_i$. Коалиция 
$C(\pi,\varnothing)$ сможет окончательно вы\-брать оптимальное решение~$x_C^*$ 
лишь на последнем ($n\hm-1$)-м шаге, завершив по\-стро\-ение 
суперпозиции~(\ref{e1.4-vas}). При этом произведение морфизмов берется 
в~противоположном порядке~$\pi^{op}$. 
    
    Категориальный подход согласует правила игры, процессы формирования 
коалиций, классы до\-пус\-ти\-мых стратегий и~поиск рационального решения  
(см.~(\ref{e1.1-vas})--(\ref{e1.4-vas})).
    
    Наличие отношения эк\-ви\-ва\-лент\-ности $x\sim x^\prime$ на множестве~$X$ 
поз\-во\-ля\-ет упрос\-тить игру за счет факторизации~[7, 8]. Раз\-би\-ение ситуаций игры на 
классы эк\-ви\-ва\-лент\-ности строится с~по\-мощью критериев $x\sim x^\prime 
\Leftrightarrow w_i(x)\hm= w_i(x^\prime)$. То же самое делается с~применением 
отношения то\-ле\-рант\-ности $\tau\hm\subset X^2$, которое по определению обладает 
свойствами реф\-лек\-сив\-ности и~сим\-мет\-рич\-ности.
    
    \smallskip
    
    \noindent
    \textbf{Лемма~1.1.} \textit{Пусть семейство $\{ x: x\tau y\}$ со\-сто\-ит из 
открытых множеств. Тогда компакт $X\hm\subset R^n$ можно заменить 
конечным множеством.}
     
  \subsection*{Игровая задача как максимизация результирующего 
отношения игры}
   
    Интересы игроков $i\hm\in I$ пред\-став\-ле\-ны бинарными отношениями 
предпочтения $\rho_i\hm\subset X^2$, $i\hm\in I$, заданными на конечном 
множестве ситуаций игры~$X$. Они реф\-лек\-сив\-ны и~тран\-зи\-тив\-ны. Все агенты ведут 
себя рационально. Их стратегии включают обмен данными с~другими участниками 
конфликта, решение о~вступ\-ле\-нии в~одну или несколько коалиций и~выбор момента 
выполнения своего хода~\cite{1-vas, 5-vas}. \textit{Ход} игрока~--- это принятие 
решения вида $\tilde{x}_i : X\hm\to X_i$, со\-про\-вож\-да\-емо\-го, воз\-мож\-но, сообщением 
стратегии~$\tilde{x}_i$ одному или нескольким парт\-не\-рам по игре. 
    
    Коллективные действия игроков порождают правила игры, которые согласуют 
классы до\-пус\-ти\-мых стратегий, процессы формирования коалиций, вводят 
\textit{отношение предшествования} ходов~$\pi$ и~уточ\-ня\-ют содержание данных, 
которыми обмениваются игроки~\cite{5-vas} (см.\ примеры~1.1 и~1.3). 
Со\-гла\-со\-ван\-ность означает отсутствие противоречия в~процессе принятия решений. 
Благодаря этому существует ре\-зуль\-ти\-ру\-ющее отношение игры~$\rho^{\Gamma(S)}$. 
Композициональная структура $\rho^{\Gamma(S)} \hm\subset X^2(S)$ формализует 
процесс поиска рационального решения. В~нее входят модули, от\-ве\-ча\-ющие 
подыграм~$\Gamma_r$, $r\hm= 1,2, \ldots , R$, проходящим в~коалициях $C\hm= 
C_r$, пред\-став\-лен\-ных характеристическими отношениями~$\rho_{C_r}$, $r\hm= 
1,2,\ldots , R$. Все подыгры \textit{на\-сле\-ду\-ют} правила исходной игры, в~част\-ности 
отношение пред\-шест\-во\-ва\-ния ходов~$\pi_C$.
    
    \textit{Рациональное} решение игры~$\Gamma(S)$ в~классе стратегий~$S$~--- 
это выбор ситуации $x^*\hm\in X(S)$, яв\-ля\-ющей\-ся мак\-си\-маль\-ным элементом 
\textit{ре\-зуль\-ти\-ру\-юще\-го} отношения

\noindent
    \begin{equation}
    x^* \in \mathrm{MAX}\,\rho^{\Gamma(S)}.
    \label{e1.5-vas}
    \end{equation}
    
    \vspace*{-6pt}
    
    На множестве классов $\overset{\frown}{X}(S)$ эквивалентных ситуаций $x\sim x^\prime 
\hm\Leftrightarrow (x,x^\prime), (x^\prime,x)\hm\in \rho^{\Gamma(S)}$, введем  
фак\-тор-от\-но\-ше\-ние $\overset{\frown}{\rho}^{\Gamma(S)} \hm\subset \overset{\frown}{X}^2(S)$~[7,~8].
    
    \smallskip
    
    \noindent
    \textbf{Теорема~1.1.}\ \textit{Пусть отношение~$\rho^{\Gamma(S)}$ 
реф\-лек\-сив\-но и~тран\-зи\-тив\-но. Тогда с~точ\-ностью до эк\-ви\-ва\-лент\-ности существует 
рациональное решение игры.}
    
    \smallskip
    
    \noindent 
    Д\,о\,к\,а\,з\,а\,т\,е\,л\,ь\,с\,т\,в\,о\,.\ \  В~час\-тич\-но упорядоченном 
множестве $(\overset{\frown}{X},\overset{\frown}{\rho}), \overset{\frown}{\rho}\hm= \overset{\frown}{\rho}^{\Gamma(S)}$, 
возьмем произвольный элемент $\overset{\frown}{x}\hm\in \overset{\frown}{X}$. Если $\overset{\frown}{x}\hm= 
\mathrm{MAX}\,\overset{\frown}{\rho}^{\Gamma(S)}$, то тео\-ре\-ма доказана. Искомая ситуация~$x^*$, 
$x^*\hm\in \overset{\frown}{x}$.\linebreak Иначе решением игры будет по\-след\-ний элемент 
$\overset{\frown}{x}_k\hm= \overset{\frown}{x}^*$ максимальной цепи $\mathrm{Ch}\hm= \left( \overset{\frown}{x}_0, 
\overset{\frown}{x}_1, \ldots , \overset{\frown}{x}_k\right)$, в~которой $\overset{\frown}{x}_0\hm= \overset{\frown}{x}$ 
и~$\left( \overset{\frown}{x}_l\overset{\frown}{x}_{l+1}\right)\hm\in \overset{\frown}{\rho}^{\Gamma(S)}$ для всех 
$l\hm= 0, 1, \ldots ,k-1$. 
    
    \smallskip
    
    \noindent
    \textbf{Следствие~1.1.} \textit{Конечное ациклическое ан\-ти\-сим\-мет\-рич\-ное 
отношение содержит максимальный и~минимальный элементы}.
    
    \smallskip
    
    Наряду с~эффективностью~(\ref{e1.5-vas}) в~игровых задачах используется 
принцип устой\-чи\-вости вы\-би\-ра\-емой ситуации. \textit{Равновесием} в~коалиционной 
игре $\Gamma(S)$ назовем ситуацию 

\noindent
    \begin{equation}
    x^* \in \mathop{\bigcap}\limits_r \mathrm{MAX}\,\rho_{\rho_{C_r}}.
    \label{e1.6-vas}
    \end{equation}
    
        \vspace*{-6pt}
        
        \noindent
Вообще говоря, принципы~(\ref{e1.5-vas}), (\ref{e1.6-vas}) противоречат друг 
другу~[1--5].

\vspace*{-4pt}

\section{Характеристическое отношение коалиции}

\vspace*{-4pt}
 
    Всякий раз, когда имеется нетривиальное отношение пред\-шест\-во\-ва\-ния ходов, 
возникают иерархически организованные коалиции~\cite{1-vas, 2-vas, 5-vas} (см.\ 
пример~1.3). При этом их участники, вообще говоря, обмениваются 
информацией~$D$. Иерархия внут\-ри коалиции по\-рож\-да\-ет подкоалиции. Коалиции 
передают другим игрокам $i\notin C$ некоторые данные $x_C\hm\in D$, которые 
ранее сообщали их участники $k\hm\in C$. Сообщение стра\-те\-гий-функ\-ций 
будем выражать в~форме включения $\tilde{x}_C\hm\in \tilde{D}$. 
    
    \textbf{Пример~2.1.} В~игре~$\Gamma^{1,n}$ образуется коалиция $C(\pi, 
D)$, $D\hm= \{\overline{x}_1, \ldots , \overline{x}_{n-1}\}$, в~результате 
до\-го\-во\-рен\-ности о~порядке ходов~(\ref{e1.3-vas}) и~передачи игроком $l\hm= 1, 
\ldots , n\hm-1$ величины $\overline{x}_l\hm\in D$ партнеру~$l\hm+1$. 
Характеристическим отношением коалиции будет 
     \begin{equation}
     \rho^{\Gamma^{1,n}}=\rho_{C(\pi,D)} =a_1\circ a_2\circ \cdots\circ a_n\,.
     \label{e2.1-vas}
     \end{equation}
    
    В обосновании формулы~(\ref{e2.1-vas}) лежат те же причины, что 
и~у~схемы~(\ref{e1.4-vas}). Чтобы определиться с~выбором пе\-ре\-да\-ва\-емой 
величины~$\overline{x}_l$, $l\hm= 1,2, \ldots ,n\hm-1$, всякий игрок~$l$ учитывает 
ра\-ци\-о\-наль\-ность поведения сле\-ду\-юще\-го игрока $l\hm+1$. Поэтому 
суперпозиция~(\ref{e2.1-vas}) строится в~порядке ходов~$\pi$. 
 Формулы~(\ref{e1.4-vas}) и~(\ref{e2.1-vas}) отвечают принципу \textit{динамического} 
программирования. 
    
    \textbf{Пример~2.2.} В~двухуровневой иерархической сис\-те\-ме ведущий 
игрок~1 пер\-вым делает свой ход~\cite{2-vas}. <<Подчиненные>> игроки $2, \ldots , 
n$ между собой не обмениваются информацией. В~за\-ви\-си\-мости от того, передает 
пер\-вый агент значение~$\overline{x}_1$ или нет, ре\-зуль\-ти\-ру\-ющее отношение 
со\-от\-вет\-ст\-ву\-ющей игры рав\-но (см.~(\ref{e1.1-vas}), (\ref{e1.4-vas}) и~(\ref{e2.1-vas})):
    \begin{multline}
    D_1=\varnothing\Rightarrow \rho^{\Gamma^{0;1,n-1}} =  \left( 
\coprod\limits_{l=2}^n a_l\right) \circ a_1 \\
  \mbox{либо\ }  D_1\not= \varnothing\Rightarrow \rho^{\Gamma^{1;1,n-1}} =a_1\circ \left( 
\coprod\limits^n_{l=2} a_l\right) .
\label{e2.2-vas}
    \end{multline}
    
    Подчиненные игроки могли бы объединиться в~коалицию Парето~$C$, сводя 
ре\-ша\-емую проб\-ле\-му к~иерархической игре двух лиц~(1 и~$C$) с~ре\-зуль\-ти\-ру\-ющим 
отношением $\rho_{\{1,C\} (\pi,\varnothing)} \hm= a_C \circ a_1$ или 
$\rho_{\{1,C\}(\pi,D)} \hm= a_1\circ a_C$ в~за\-ви\-си\-мости от того, множество 
$D_1\hm=\varnothing$ или нет. Здесь $a_C\hm= \rho_C\vert_{x_C}$~--- абсолютно 
оптимальное отношение коалиции (см.~(\ref{e1.2-vas})). 
    
    Рассмотрим теперь игры $\Gamma^{2,n}$, $n\hm\geq 2$, в~которых партнерам 
последовательно передаются данные вида $\tilde{x}_i \hm\in \tilde{D}_i$, 
$\tilde{x}_i: X\hm\to X_i$, $i\hm= 1,\ldots n\hm-1$~\cite{1-vas, 5-vas}. Обрат\-ная связь 
может приводить к~ситуациям \textit{равновесия}~(\ref{e1.6-vas}). 
    
    \textbf{Пример~2.3.} В~обоб\-щен\-ной игре  
Гермейера~$\Gamma^{2,2}$~\cite{1-vas} граф ходов и~характеристическое 
отношение коалиции $C(\pi, \tilde{D})$ име\-ют сле\-ду\-ющий вид:
    \begin{multline}
    \pi, \tilde{D}: \boxed{1} \overset{\tilde{x}_1}{\to} 
\boxed{2}\sim\rho^{\Gamma^{2,2}} = \rho_{C(\pi,\tilde{D})} ={}\\
    {}= \rho_2\circ \left( \rho_1\cup \rho_2^G\right), \quad \rho_2^G 
=\rho_2\vert_{\mathrm{MIN}\,\rho_2\vert_{X_1}}.
\label{e2.3-vas}
\end{multline}

    В формуле~(\ref{e2.3-vas}) использовано \textit{гарантированное} отношение 
второго игрока $\rho_2^G\hm\subset \rho_2$, с~по\-мощью которого срав\-ни\-ва\-ют\-ся 
результаты <<по\-слой\-ной>> минимизации отношения $\rho_2\vert_{X_1}$ при 
любом фиксированном па\-ра\-мет\-ре $x_2\hm\in X_2$:
    \begin{equation}
    \rho_2^G \triangleq \left\{ (x,x^\prime): x\rho_2 x^\prime;\enskip  x, x^\prime\in \mathrm{MIN}\,\rho_2\vert_{X_1}\right\}.
    \label{e2.4-vas}
    \end{equation}
В~тео\-ре\-ме~3.2 \mbox{изучен} общий случай сис\-те\-мы $\Gamma^{2,n}$, $n\hm\geq 2$,  
и~доказана формула~(\ref{e2.3-vas}). 

    \textbf{Пример~2.4.} В~двухуровневой иерархической игре~$\Gamma^{1;1,n-
1}$, $\pi \hm= \{ (1,2), \ldots , (1, n-1)\}$, ведущий игрок~1 не может сообщить  
стра\-те\-гию-функ\-цию  $\tilde{x}_1: X_2\times \cdots \times X_n\hm\to X_1$ своим 
партнерам $\{2,3,\ldots , n\}$. Отсутствие коммуникации меж\-ду ними ведет 
к~противоречию в~процессе принятия решений: в~подкоалициях $\{1,2\}, \ldots , \{1, 
n-1\}$ нельзя одновременно разыг\-рать игры~$\Gamma^{2,2}$. Класс 
функции~$\tilde{D}_1$ недопустим. Поэтому требуется изменить правила игры: 
игрок~1 сообщает парт\-не\-рам лишь значения $\overline{x}_1\hm\in D_1 \hm\subset 
\tilde{D}_1$. Тогда в~подкоалициях решаются подыг\-ры~$\Gamma^{1,2}$, 
а~в~целом~--- игра~$\Gamma^{1;1,n-1}$ (см.~(\ref{e2.1-vas}) и~(\ref{e2.2-vas})). 
Следовательно, интересы коалиции $C\hm= C(\pi, \tilde{D}_1)$ характеризуются 
отношением 
    \begin{equation}
    \rho^{\Gamma^{1;1,n-1}} =\rho_C= a_1\circ a_2\coprod \cdots \coprod a_1\circ 
a_n\,,
    \label{e2.5-vas}
    \end{equation}
    поэтому необходимо контролировать до\-пус\-ти\-мость применяемых стратегий. 
    
\section{Композициональность игры}

    Развиваемый подход базируется на свойстве ком\-по\-зи\-ци\-о\-наль\-ности бинарных 
отношений, при\-ме\-ня\-емых в~игровых операциях. Основой служит моноидальная 
категория бинарных отношений REL~[7, 8]. Объектами REL вы\-сту\-па\-ют 
конечные множества $X, Y, Z, \ldots$, а~морфизмами~--- бинарные отношения $\alpha: 
X\hm\to Y, \beta: Y\hm\to Z, \ldots ,$ заданные на произведениях их областей 
и~кообластей $X\times Y, Y\times Z, \ldots$ Напомним~\cite{8-vas}, что 
композицией морфизмов $\alpha: X\hm\to Y, \beta: Y\hm\to Z$ в~категории REL 
является суперпозиция $\alpha\circ \beta \hm\subset X\times Z$, $\alpha\circ\beta: 
X\hm\to Z$ этих отношений, \mbox{равная}
    \begin{equation}
    \alpha\circ\beta =\left\{ (x,z): \exists_{y\in Y} x\alpha y\wedge y\beta z\right\}.
    \label{e3.1-vas}
    \end{equation}
Вместо записи $(x,y)\hm\in \alpha\hm\subset X\times Y$ здесь использовано 
инфиксное обозначение $x\alpha y$. Единичными морфизмами для операции 
произведения~(\ref{e3.1-vas}) служат тривиальные порядки~$\sigma_X$ 
и~$\sigma_Y$, а~$\sigma_Z\hm= \{ (z,z): z\hm\in Z\}$ для любого мно\-же\-ст\-ва~$Z$. 

    В категории REL определена операция дизъюнктивной суммы морфизмов 
     $\alpha\coprod \beta : (X\coprod Y) \hm\to (Y\coprod Z)$, на\-зы\-ва\-емая так\-же 
моноидальным произведением: 
    \begin{equation}
    \alpha\coprod\beta =\left\{ (x,y;1): x\alpha y\right\} \cup \{ (y,z;2): y\beta z\}.
    \label{e3.2-vas}
    \end{equation}
Операции~(\ref{e3.1-vas}) и~(\ref{e3.2-vas}) ас\-со\-ци\-а\-тив\-ны, а~сумма~(\ref{e3.2-vas}) 
обладает свойством коммутативности и~имеет единицу~$\varnothing$. Поэтому 
REL пред\-став\-ля\-ет собой моноидальную категорию~\cite{8-vas}.

    В рассматриваемых игровых задачах у~каж\-до\-го морфизма совпадают области 
и~кообласти, например это так у~исходных отношений предпочтения игроков $\rho: 
X\hm\to X$. При по\-стро\-ении вспомогательных бинарных отношений по\-сред\-ст\-вом 
моноидального произведения $\rho_1 \coprod\rho_2$ изменяются области 
и~кообласти по\-лу\-ча\-емых морфизмов. Тем не менее всегда можно считать 
до\-пус\-ти\-мы\-ми произвольные композиции морфизмов. Так, под левыми час\-тя\-ми 
выражений 
    \begin{align*}
    \left(\gamma_1\coprod\gamma_2\right) \circ\gamma 
&=\gamma_1\circ\gamma\coprod\gamma_2\circ\gamma\,;\\
    \gamma\circ\left(\gamma_1\coprod\gamma_2\right)&=\gamma\circ\gamma_1\coprod 
\gamma\circ \gamma_2
    \end{align*}
следует понимать сле\-ду\-ющие композиции морфизмов соответственно: 
\begin{align*}
\left(\gamma_1\coprod\gamma_2\right)&\circ\left(\gamma\coprod\gamma\right); \\
\left(\gamma\coprod\gamma\right)&\circ\left( \gamma_1\coprod\gamma_2\right).
\end{align*}
    
    Вообще говоря, РОИ $\rho^{\Gamma(S)} 
\hm\subset Z^2$ определено на множестве $Z\hm= X(S)\not= X$. Поэтому найденное 
решение задачи~(\ref{e1.5-vas}) нуж\-но дополнительно преобразовать для получения 
до\-пус\-ти\-мой ситуации игры $x^*\hm\in X$. Это делается проектированием $Z\hm\to 
X$ (см.\ определение~(\ref{e3.2-vas})). 
    
    \subsection*{Задание правил игры}

    Договоренности между игроками могут приводить к~формированию коалиций 
до начала их ходов. \textit{Заранее} объ\-яв\-ля\-емые коалиции Парето (см.\ пример~1.2) 
или иерархические коалиции (см.\ примеры~2.1--2.4) имеют приоритет при 
выполнении ходов, приводящий к~изменению исходного отношения~$\pi$. После 
этого коалиции трактуются как отдельные игроки с~отношениями 
предпочтения~$\rho_C$ (см.~(\ref{e1.2-vas}) и~(\ref{e1.4-vas})). 
    
    Выбираемое агентами отношение предшествования ходов~$\pi$ управ\-ля\-ет 
созданием иерархически организованных коалиций в~процессе игры Оно должно 
обладать свойствами ан\-ти\-сим\-мет\-рич\-ности, реф\-лек\-сив\-ности 
    и~ацик\-лич\-ности~\cite{5-vas}. В~конечном итоге игра проходит между 
коалициями разных типов, интересы которых пред\-став\-ле\-ны характеристическими 
отношениями. Проходящие внут\-ри коалиций~$C$ подыгры наследуют отношение 
предшествования ходов, обозна\-ча\-емое~$\pi_C$.
    
    Всякая коалиция~$C$ действует как игрок, пе\-ре\-да\-ющий информацию 
$D_C\hm= \left\{ \tilde{x}_i, \overline{x}_j, i,j\hm\in C, i\not=j\right\}$ одновременно 
всем участникам конфликта, которые были адресатами сообщений ее членов. Если 
адресаты $r,k,\ldots$ вошли в~некоторые коалиции\linebreak $C^\prime\hm= C$, то данные 
$x_C\hm= \left\{ \tilde{x}_i,\overline{x}_j\right\}$ по\-сту\-па\-ют игроку~$C^\prime$ 
и~распределяются по на\-зна\-че\-нию. 
    
    В понятие стратегии входит участие игрока в~разработке правил игры, 
ини\-ци\-иро\-ва\-нии момента \textit{хода}, выборе конт\-ро\-ли\-ру\-емо\-го фак\-то\-ра 
и,~воз\-мож\-но, коммуникации~--- сообщении информации некоторым из партнеров. 
Размеченный пе\-ре\-да\-ва\-емы\-ми данными граф~$G_\pi$ отношения предшествования 
ходов назовем \textit{графом игры} (см.~(\ref{e1.3-vas}) и~(\ref{e2.3-vas})). Так как все 
участники конфликта заинтересованы в~выборе рационального решения, то они 
используют лишь \textit{допустимые} клас\-сы стратегий. Выбор графа игры должен 
сопровождаться анализом клас\-сов стратегий, которые намерены применять игроки, 
и~в~случае не\-об\-хо\-ди\-мости их корректировать. Изменение правил игры направлено 
на соблюдение требования не\-про\-ти\-во\-ре\-чи\-вости процесса поиска решения (см.\
пример~2.4). 
    
    Противоречие в~обмене данными устраняется с~по\-мощью \textit{допустимой} 
разметки графа игры. Никто из игроков не может одновременно сообщить 
парт\-не\-рам и~процедуру выбора фактора~$\tilde{x}$, и~его значение~$\overline{x}$. 
Для адресатов $j\hm\in T_i \hm= \{j: i\pi j, i\not=j\}$ игрока~$i$ либо дуги $i\hm\to j$ 
графа~$G_\pi$ вовсе не помечаются, если $\exists_l l\hm\in T_i \wedge j\pi l$ (см., 
например, (\ref{e1.3-vas})), либо имеют раз\-мет\-ку~$\overline{x}_i$, если мощ\-ность 
$\# T_i\hm>1$, или~$\tilde{x}_i$, если $\# T_i\hm=1$. 
    
    \textbf{Пример~3.1.} В~левой час\-ти формулы
    
    \vspace*{-29pt}
    
    \noindent
      \begin{equation} 
\setlength{\unitlength}{1mm}\thicklines %(3.3)-1
\begin{picture}(16,20)
\put(2,2.5){$\boxed{1}$}
\put(7,3.5){\vector(1,0){6}}
\put(14,2.5){$\boxed{2}$}
\put(8,5.5){\small{$\tilde{x}_1$}}
\put(4,0.5){\vector(0,-1){6}}
\put(0,-3){\small{$\tilde{x}_1$}}
\put(14,-3.5){\small{$\tilde{x}_2$}}
\put(2,-10){$\boxed{3}$}
\put(16,0.5){\vector(-4,-3){8}}
\end{picture}\enskip \enskip ; \quad %\enskip\enskip\enskip
\setlength{\unitlength}{1mm}\thicklines % (3.3)-2
\begin{picture}(16,20)
\put(0,2.5){$\boxed{1}$}
\put(5,3.5){\vector(1,0){6}}
\put(12,2.5){$\boxed{2}$}
\put(6,5.5){\small{$\tilde{x}_1$}}
\put(2,0.5){\vector(0,-1){6}}
\put(12,-3.5){\small{$\tilde{x}_2$}}
\put(0,-10){$\boxed{3}$}
\put(14,0.5){\vector(-4,-3){8}}
\end{picture}
\label{e3.3-vas}
\end{equation}

\vspace*{29pt}

\noindent
изображена 
не\-до\-пус\-ти\-мая раз\-мет\-ка графа~$G_\pi$. В~правой час\-ти~(\ref{e3.3-vas}) показан 
<<ис\-прав\-лен\-ный>> граф игры. 
  Разметка~$\tilde{x}_1$ дуги $1\hm\to 3$ допустима, но игнорирует заданный 
порядок ходов $(2,3)\hm\in \pi$. 

    \subsection*{Построение результирующего отношения игры}

    Пусть граф игры обладает до\-пус\-ти\-мой раз\-мет\-кой дуг. Докажем, что имеется 
РОИ~$\rho^{\Gamma(S)}$, которое однозначно 
определено в~соответствии с~правилами\linebreak игры, при этом оно пред\-став\-ля\-ет собой 
композицию исходных и~производных бинарных отношений в~категории REL. 
Индуктивное конструирование~$\rho^{\Gamma(S)}$ опирается на отношение 
\mbox{пред\-шест\-во\-ва\-ния} ходов~$\pi$, моделируя процесс формирования иерархических 
коалиций вслед\-ст\-вие коммуникации игроков.
    
    \smallskip
    
    \noindent
    \textbf{Теорема~3.1.} \textit{Во всякой игре существует ре\-зуль\-ти\-ру\-ющее 
отношение}. 
    
    \smallskip
    
    \noindent
    Д\,о\,к\,а\,з\,а\,т\,е\,л\,ь\,с\,т\,в\,о\,.\ \  Применим метод математической 
индукции, проводимой по чис\-лу участников игры. При $n\hm\leq 2$ 
РОИ существует (см.\ примеры~1.1--1.3 и~2.1--2.4). Если 
$\pi\hm=\sigma$, то для всех значений $n\hm= 1,2,\ldots$ ре\-зуль\-ти\-ру\-ющее 
отношение вы\-чис\-ля\-ет\-ся по фор\-му\-ле~(\ref{e1.1-vas}). 
    
    Пусть $\pi\not=\sigma$. Согласно следствию~1.1 и~ацик\-лич\-ности 
отношения~$\pi$ найдется элемент $i\hm\in \mathrm{MIN}\, \pi$, для которого $\exists_j (j\not= 
i) \wedge i\pi j$. Сформируем коалицию $C_1\hm= \{j: i\pi j\}$. По свойству 
реф\-лек\-сив\-ности~$\pi$ игрок $i\hm\in C_1$. Воспользуемся предположением 
индукции: существует характеристическое отношение~$\rho_{C_1}$. Не 
исключено, что при сле\-ду\-ющих ходах внут\-ри коалиции~$C_1$ будут образованы 
подкоалиции~$C_1^k$, $k\hm= 1,2,\ldots , K$, с~чис\-лом участников $r_k\hm <n$. 
Тогда~$\rho_{C_1}$ строится композицией морфизмов~$\rho_{C_1^k}$ категории 
REL.
    
    Построим новый граф игры~$G_{\pi^\prime}^\prime$. Пусть из вершины $i\hm\in G_\pi$ 
исходит~$l$, $l\hm\geq 2$, дуг, которые имеют до\-пус\-ти\-мую разметку. Заменим 
одним агентом коалицию~$C_1$. Ис\-клю\-чим из графа игры~$G_\pi$ все дуги 
$(i,j)\hm\in\pi$, от\-ве\-ча\-ющие коммуникациям внут\-ри коалиции~$C_1$. Из новой 
вершины $C_1\hm\in G_{\pi^\prime}^\prime$ проведем дуги $(C_1,r)\hm\in \pi^\prime$, где 
$(j,r)\hm\in \pi$, $j\hm\in C_1$. Таким образом, по\-стро\-ен наследник~$\pi^\prime$ 
отношения предшествования ходов~$\pi$. Разметим дуги $(C_1,r)$ так же, как 
и~$(j,r)\hm\in\pi$, обеспечив до\-пус\-ти\-мость графа игры~$G^\prime_{\pi^\prime}$. 
В~$\Gamma^\prime$ чис\-ло участников $m\hm<n$. По предположению индукции 
существует ре\-зуль\-ти\-ру\-ющее отношение~$\rho^{\Gamma^\prime}$, стро\-яще\-еся 
композицией соответствующих морфизмов. 
    
    Композиционный процесс завершается построением бескоалиционной игры 
$\pi^{(k)} \hm=\sigma$, ре\-зуль\-ти\-ру\-ющим отношением которой выступает 
моноидальное произведение~(\ref{e1.1-vas}). Про\-ил\-люст\-ри\-ру\-ем\linebreak тео\-ре\-му~3.1 на 
примере графа игры~(\ref{e3.3-vas}). Сначала в~подкоалиции $C\hm= \{1,2\}$ 
строится характеристическое отношение~(\ref{e2.3-vas}), а~затем для игры 
$C\hm\to 3$~--- ре\-зуль\-ти\-ру\-ющее (см.\ фор\-му\-лу~(\ref{e1.4-vas})):
    \begin{equation}
    \rho^{\Gamma^{2,3(\pi^\prime)}} =a_3\circ \left(\rho_2\circ \left( \rho_1\cup 
\rho_2^G\right)\right).
    \label{e3.3-1-vas}
    \end{equation}
    
    \textbf{Пример~3.2.} Рассмотрим граф игры~$\Gamma$, опи\-сы\-ва\-ющий 
функционирование трехуровневой иерархической сис\-темы

\vspace*{-26pt}

\noindent
\begin{center}
\setlength{\unitlength}{1mm}\thicklines %ПРИМЕР 3.2
\begin{picture}(28,16)
\put(0,-2){$\boxed{1}$}
\put(5,-1){\vector(1,0){6}}
\put(12,-2){$\boxed{2}$}
\put(6,1){\small{$\tilde{x}_1$}}
\put(17,0){\vector(4,3){6}}
\put(24,4){$\boxed{3}$}
\put(17,4){\small{$\bar{x}_2$}}
\put(17,-7){\small{$\bar{x}_2$}}
\put(24,-7.5){$\boxed{4}$}
\put(17,-1){\vector(4,-3){6}}
\end{picture}~~~~~.
\end{center}

  \vspace*{26pt} 

\noindent
На первом шаге построения отношения~$\rho^{\Gamma}$ формируется 
иерархическая коалиция $C_1\hm=\{1,2\}$, в~которой разыгрывается 
операция~$\Gamma^{2,2}$ с~па\-ра\-мет\-ра\-ми $x_3$ и~$x_4$. Ее характеристическое 
отношение~$\rho_{C_1}$ задается формулой~(\ref{e2.3-vas}). При втором ходе идет 
игра трех лиц $\{C_1, 3, 4\}$ (см.\ примеры~2.2 и~2.4 и~формулу~(\ref{e2.5-vas})). 
По\-этому 
\begin{multline}
\rho^{G} =\rho_{C_1} \circ a_C={}\\
{}=\left( \rho_2\circ\left( \rho_1\cup 
\rho_2^G\right)\right)\circ \left( \rho_3\coprod\rho_4\right)\Big\vert_{X_3\times X_4}.
\label{e3.4-vas}
\end{multline}
    
    Если в~правилах игры оговорить заранее, что игроки~3 и~4 создадут коалицию 
Парето~$C^P$ (см.~(\ref{e1.2-vas})), то второй ход приведет к~игре двух лиц 
$\Gamma^\prime\hm= \left\{ C_1, C^P,\pi^\prime: C_1\hm\to C^P\right\}$. 
Не\-оп\-ре\-де\-лен\-ность\linebreak выбора рационального решения уменьшится по сравнению 
с~оптимизацией~(\ref{e3.4-vas}), так как теперь исследуется ре\-зуль\-ти\-ру\-ющее 
отношение
    $$
    \rho^{\Gamma^\prime} =\rho_{C_1} \circ a_{C^P} = \left( 
\rho_2\circ\left(\rho_1\cup \rho_2^G\right)\right)\circ \rho_3\cap \rho_4.
    $$
    
    \textbf{Пример~3.3.} На первом шаге по\-стро\-ения~\mbox{РОИ}

\begin{center}
\setlength{\unitlength}{1mm}\thicklines % ПРИМЕР 3.3
\begin{picture}(18,18)
\put(2,12.5){$\boxed{1}$}
\put(13,13.5){\vector(-1,0){6}}
\put(14,12.5){$\boxed{2}$}
\put(8,15.5){\small{$\bar{x}_2$}}
\put(4,4.5){\vector(0,1){6}}
\put(0,7){\small{$\bar{x}_3$}}
\put(14,6.5){\small{$\bar{x}_2$}}
\put(2,0){$\boxed{3}$}
\put(16,10.5){\vector(-4,-3){8}}
\end{picture}
\end{center}


\noindent
участники объединяются в~одну иерархическую коалицию~$C_1$ с~ведущим 
игроком~2. На втором шаге формируется подкоалиция $C_1^1\hm= \{1,3\}$ 
с~игроком~1. Согласно примеру~2.1, $\rho_{C_1^1}\hm= a_3\circ a_1$. 
Характеристическое отношение коалиции $C_1\hm= \{2, C_1^1\}$ рав\-но 
$\rho^\Gamma \hm= \rho_{C_1} \hm= a_2\circ \rho_{C_1^1} \hm= a_2\circ a_3 \circ 
a_1$ (см.~(\ref{e2.1-vas})).
    
    \subsection*{Обобщение игры Гермейера}

    Докажем формулы~(\ref{e2.3-vas}) и~(\ref{e2.4-vas}). Под 
\textit{гарантированным} отношением игрока~2 будем понимать сле\-ду\-ющее 
суже\-ние его отношения предпочтения:
    \begin{equation}
    \rho_2^\Gamma =\rho_2\vert_{\mathrm{MIN}\, \rho_2\vert_{X_1}}.
    \label{e3.5-vas}
    \end{equation}
Пусть $\pi_{X_1}: X\hm\to X_1$~--- проектирование. Любая функция 
$\tilde{x}_1^{\mathrm{н}}: X_2\hm\to \pi_{X_1} \mathrm{MIN}\, \rho_2\vert_{X_1}$ называется 
стратегией \textit{наказания} 2-го игрока~\cite{1-vas}, а~$x_2^G\hm\in 
\pi_{X_2} \mathrm{MAX}\, \rho_2^{G}$~--- \textit{га\-ран\-ти\-ру\-ющей} стратегией. 
    
    \smallskip
    
    \noindent
    \textbf{Теорема~3.2.} \textit{Пусть $\rho_2$~--- транзитивное отношение. 
Тогда в~игре двух лиц $\Gamma^{2,2}$ ре\-зуль\-ти\-ру\-ющее отношение равно}
    \begin{equation}
    \rho^{\Gamma^{2,2}} =\rho_2\circ\left( \rho_1\cup \rho_2^G\right).
    \label{e3.6-vas}
    \end{equation}
    
    \noindent
    Д\,о\,к\,а\,з\,а\,т\,е\,л\,ь\,с\,т\,в\,о\,.\ \  Пусть $x_2^*\hm= \pi_{X_2} x^*$, 
$x^*\hm\in \mathrm{MAX}\,\rho_1$. Если игрок~2 выберет $x_2\not= x_2^*$, то игрок~1 
применит стратегию наказания~$\tilde{x}_1^{\mathrm{н}}$. Поэтому перед 
игроком~2 стоит альтернатива: либо применять $x_2\hm= x_2^*$, либо быть 
наказанным. Угроза действенна лишь при условии $(\tilde{x}_1^{\mathrm{н}}(x_2^G), x_2^G)\rho_2 x^*$, когда не помогает 
использование игроком~2 га\-ран\-ти\-ру\-ющей стратегии $x_2\hm= x_2^G$. В~самом 
деле, определение~$\rho_2^G$ и~тран\-зи\-тив\-ность~$\rho_2$ \mbox{дают}
    \begin{multline*}
    \left( \tilde{x}_1^{\mathrm{н}}(x_2), x_2\right)\rho_2 \left( 
\tilde{x}_1^{\mathrm{н}}(x_2^G),x_2^G\right) \wedge \left( \tilde{x}_1^{\mathrm{н}} 
(x_2^G), x_2^G\right) \rho_2 x^*\Rightarrow{}\hspace*{-2.17662pt}\\
{}\Rightarrow
    \left( \tilde{x}_1^{\mathrm{н}}(x_2),x_2\right) \rho_2 x^*.
    \end{multline*}
<<Выгоднее>> для игрока~2 вы\-брать~$x_2^*$, а~не~$x_2^G$. 
Отсюда~$\rho^{\Gamma^{2,2}}$ имеет вид~(\ref{e3.6-vas}) (см.~(\ref{e2.3-vas})). 

\smallskip

\noindent
\textbf{Следствие~3.1.}\ \textit{Иерархическая игра~$\Gamma^{2,n}$ с~$n\hm= 2k$ 
или $n\hm=2k\hm+ 1$ участ\-ни\-ками}
\begin{equation}
\boxed{1} \overset{\tilde{x}_1}{\to} \cdots \overset{\tilde{x}_r}{\to} 
\boxed{r+1}\overset{\tilde{x}_{r+1}}{\to} \cdots \overset{\tilde{x}_{n-1}}{\to} \boxed{n}
\label{e3.7-vas}
\end{equation} 
\textit{имеет следующие ре\-зуль\-ти\-ру\-ющие отношения}:
\begin{multline}
\rho^{\Gamma^{2,2k}} =\rho_{2k}\circ \left( a_{2k-1}\circ \rho^{\Gamma^{2,2k-2}} 
\cup \rho_{2k}^G\right);\\
\rho^{\Gamma^{2,2k+1}} = a_{2k+1}\circ \rho^{\Gamma^{2,2k}};\\
\rho^{\Gamma^{2,0}} =\sigma,\enskip k=1,2,\ldots
\label{e3.8-vas}
\end{multline}
    
    \noindent
    Д\,о\,к\,а\,з\,а\,т\,е\,л\,ь\,с\,т\,в\,о\,.\ \ В~со\-от\-вет\-ст\-вии с~тео\-ре\-мой~3.1 
с~каж\-дым ходом игроков происходит расширение коалиции $C_{k+1}\hm= C_k \cup 
\{k+1\}$, $C_1\hm= \{1\}$, и~поочередное применение формул~(\ref{e1.4-vas}) 
и~(\ref{e3.6-vas}) вы\-чис\-ле\-ния характеристических отношений. 
    
    \textbf{Пример~3.4.} Пусть в~игре~(\ref{e3.7-vas}) имеются предварительные 
договоренности о~формировании подкоалиций $C_{2k} \hm= \{2k-1,2k\}$, 
$\rho_{C_{2k}} \hm= \rho^{\Gamma^{2,2}}\vert_{X_{2k-1}\times X_{2k}}$ 
(см.~(\ref{e3.6-vas})). Тогда 
    \begin{multline*}
    \rho_C^{\Gamma^{2,2k}} =\rho_{2k} \circ\left( \rho_{2k-1} \cup \rho^G_{2k} 
\right)\circ\cdots\circ \rho_2 \circ\left( \rho_1\cup \rho_2^G\right);\\
    \rho_C^{\Gamma^{2,2k+1}} =a_{2k+1}\circ \rho_C^{\Gamma^{2,2k}}.
  \end{multline*}

\section{Поиск рационального решения игры}

    Композиционное строение отношения $\rho^{\Gamma(S)}$ со\-кра\-ща\-ет перебор 
в~задаче~(\ref{e1.5-vas}). 
    
    \smallskip
    
    \noindent
    \textbf{Теорема~4.1.} \textit{Имеют мес\-то свойства}
    \begin{equation}
    \left.
    \begin{array}{rl}
    \mathrm{MAX}\rho_2\vert_{\mathrm{MAX}\,\rho_1} &\subset \mathrm{MAX} \left(\rho_2\circ\rho_1\right);\\[6pt]
    \mathrm{MAX}\left(\rho_1\coprod\rho_2\right) &=\mathrm{MAX}\,\rho_1\cup \mathrm{MAX}\, \rho_2.
    \end{array}
    \right\}
    \label{e4.1-vas}
    \end{equation}
 \textit{Если отношения $\rho_1$ и~$\rho_2$ транзитивны и~рефлексивны, то} 
 $\mathrm{MAX}\,\rho_2\vert_{\mathrm{MAX}\,\rho_1} \hm= \mathrm{MAX} \left(\rho_2\circ\rho_1\right)$.
 
 \smallskip
 
 \noindent
    Д\,о\,к\,а\,з\,а\,т\,е\,л\,ь\,с\,т\,в\,о\,.\ \ Вторая из формул~(\ref{e4.1-vas}) 
непосредственно следует из определения~(\ref{e3.2-vas}) дизъ\-юнк\-тив\-ной суммы. 
Докажем пер\-вую. По определению максимального элемента $x^*\hm= \mathrm{MAX}\,\rho_2\vert_{\mathrm{MAX}\,\rho_1}$ \mbox{имеем}
    \begin{multline*}
    \forall_{x_1^*} \forall_x \left( \left( x_1^*,x\right)\in\rho_1 \Rightarrow 
x=x_1^*\right) \wedge{}\\
{}\wedge  \left(\left( x^*,x_1^*\right)\in\rho_2\Rightarrow x_1^*=x^*\right),
 \end{multline*}
что эквивалентно формуле 
$$
\forall_{x_1^*} \forall_x \left(\left( x_1^*, x\right)\in \rho_1\right) \wedge \left(\left( x^*, 
x_1^*\right)\in \rho_2\right) \Rightarrow x=x^*.
$$
Переписывая левую часть импликации в~форме суперпозиции отношений, приходим 
к~сле\-ду\-юще\-му вы\-воду: 
$$
\forall_x \left( x^*, x\right) \in \rho_2\circ\rho_1\Rightarrow x=x^*.
$$
 Значит, доказано, что 
$$
\mathrm{MAX}\,\rho_2\vert_{\mathrm{MAX}\,\rho_1} \subset \mathrm{MAX} \left(\rho_2\circ\rho_1\right).
$$
    
    Пусть отношения $\rho_1$ и~$\rho_2$ транзитивны и~рефлексивны. Включение 
$x^*\hm\in \mathrm{MAX}\,(\rho_2\circ\rho_1)$ рав\-но\-силь\-но фор\-муле 
    $$
    \forall_x \left( x^*,x\right) \in \rho_2\circ\rho_1 \Rightarrow x=x^*.
    $$
     Согласно определению композиции~(\ref{e3.1-vas}), имеем 
    $$
    \forall_x \exists_{x_1^*} \left( x^*, x_1^*\right) \in \rho_2\wedge \left( x_1^*, 
x\right) \in\rho_1\Rightarrow x=x^*.
    $$
     Как следствие, имеем
    $$
    \forall_x \forall_{x_1^*} \left( x^*, x_1^*\right) \in\rho_2\wedge \left( x_1^*, 
x\right) \in\rho_1\Rightarrow x=x^*.
    $$
Опираясь на реф\-лек\-сив\-ность отношения~$\rho_2$, под\-ста\-вим сюда $x_1^*\hm= 
x^*$. Тогда вер\-на формула 
$$
\left( x^*,x\right)\in\rho_1\hm\Rightarrow x= x^*,
$$ 
озна\-ча\-ющая $x^*\hm\in \mathrm{MAX}\,\rho_1$. Рас\-суж\-дая аналогично, под\-ста\-нов\-кой $x\hm= 
x_1^*$ получим, что $x^*\hm= \mathrm{MAX}\,\rho_2$ и~тем более $x^*\hm\in \mathrm{MAX}\,\rho_2\vert_{\mathrm{MAX}\,\rho_1}$. Итак, противоположное вложение 
$\mathrm{MAX}\,(\rho_2\circ\rho_1)\hm\subset \mathrm{MAX}\,\rho_2\vert_{\mathrm{MAX}\,\rho_1}$ так\-же имеет место. 
    
    Из соображений двой\-ст\-вен\-ности вытекает сле\-ду\-ющий результат.
    
    \smallskip
    
    \noindent
    \textbf{Следствие~4.1.} \textit{Справедливы соотношения}:
    \begin{equation}
    \left.
    \begin{array}{rl}
   \mathrm{MIN}\,\rho_2\vert_{\mathrm{MIN}\,\rho_1} &\subset \mathrm{MIN}\left( \rho_1\circ\rho_2\right),\\[6pt]
    \mathrm{MIN}\,\rho_1 \displaystyle \coprod \rho_2 &=\mathrm{MIN}\,\rho_1\cup \mathrm{MIN}\,\rho_2.
     \end{array}
     \right\}
    \label{e4.2-vas}
    \end{equation}
 \textit{Если $\rho_1$ и~$\rho_2$~--- транзитивные и~рефлексивные отношения, то} 
$\mathrm{MIN}\,\rho_2\vert_{\mathrm{MIN}\,\rho_1} \hm= \mathrm{MIN}\,(\rho_1\circ\rho_2)$.
    
    \smallskip
    
    В теореме~4.1 и~следствии~4.1 обобщен метод динамического 
программирования применительно к~бинарным отношениям. Воспользуемся 
свойством ас\-со\-циа\-тив\-ности композиции морфизмов и~формулами~(\ref{e1.4-vas}), 
(\ref{e2.1-vas}), (\ref{e3.6-vas}), (\ref{e3.8-vas})--(\ref{e4.2-vas}) (см.\ 
также пример~2.4), и~методом математической индукции докажем сле\-ду\-ющее 
    
    \smallskip
    
    \noindent
    \textbf{Следствие~4.2.} \textit{Рациональные решения игр $\Gamma^{0,n}$, 
$\Gamma^{1,n};$ $\Gamma^{0;1,n-1}$, $\Gamma^{1;1,n-1}$, $\Gamma^{2,n}$ 
удовле\-тво\-ря\-ют сле\-ду\-ющим свойствам}:
    \begin{equation}
    \left.
    \begin{array}{l}
    \mathrm{MAX}\, a_n\vert_{\mathrm{MAX}\, a_{n-1}\vert\cdots \vert_{\mathrm{MAX} \,a_1}} \subset \mathrm{MAX}\, 
\rho^{\Gamma^{0,n}};\\[6pt]
 \mathrm{MAX} \,a_1\vert_{\mathrm{MAX} \,a_2\vert\cdots {}_{\vert_{\mathrm{MAX}\, a_n}}}\subset 
\mathrm{MAX} \,\rho^{\Gamma^{1,n}}\,;\\[6pt]
    \mathop{\bigcup}\limits^n_{l=2} \mathrm{MAX} \,a_l\vert_{\mathrm{MAX} \,a_1} = \mathrm{MAX} \,\rho^{\Gamma^{0; 1,n-1}}; \\[6pt]
 \mathrm{MAX}\, a_1\vert_{\mathop{\bigcup}_{l=2}^n \mathrm{MAX} a_l} =\mathrm{MAX} \,\rho^{\Gamma^{1;1,n-1}}\,;\\[6pt]
    \mathrm{MAX} \,a_{2k+1}\vert_{\mathrm{MAX}\, \rho^{\Gamma^{2,2k}}} \subset \mathrm{MAX}\,\rho^{\Gamma^{2,2k+1}};\\[6pt]
 \mathrm{MAX}\,\rho_{2k}\vert_{\mathrm{MAX}\, a_{2k-1} \cup \mathrm{MAX}\,\rho^G_{2k}\big\vert_{\mathrm{MAX}\, \rho^{\Gamma^{2,2k-2}}}} \subset{}\\[6pt]
\hspace*{38mm}{}\subset \mathrm{MAX}\,  \rho^{\Gamma^{2,2k+2}}.
\end{array}
\right\}
\label{e4.3-vas}
    \end{equation}

    
    \textbf{Пример~4.1.} Рас\-смот\-рим игру~$\Gamma^{2,3}$, в~которой 
предпочтения участников являются рефлексивными транзитивными замыканиями 
сле\-ду\-ющих отношений: 
    \begin{align*}
    \rho_1&= \{ (0,1), (0,5), (4,0), (3,4), (3,2), (7,6)\}\,;\\
    \rho_2 &= \{ (0,2), (1,0), (2,4), (3,2), 3,5), (4,5), (6,4),\\
    &\hspace*{65mm} (6,7)\}\,;\\
    \rho_3 &= \{ (2,0), (4,0), (3,5), (3,1), (6,4), (7,6)\}\,,
    \end{align*}
заданных на бинарном кубе $\underline{8}\hm\simeq \{0,1\}^3$. Ситуации 
$(x_1,x_2,x_3)\hm\in \underline{8}$ пред\-став\-ле\-ны чис\-ла\-ми $x\hm\simeq x_1\hm+ 
2x_2\hm+ 4x_3$, записанными в~двоичной сис\-те\-ме. Ре\-зуль\-ти\-ру\-ющее отношение 
игры имеет вид~(\ref{e3.3-1-vas}) (см.~(\ref{e3.8-vas}), $k\hm=1$). Найдем 
рациональное решение~$x^*$, со\-кра\-тив перебор воз\-мож\-ных вариантов 
в~задаче~(\ref{e1.5-vas}) с~по\-мощью формул~(\ref{e4.3-vas}). 
    
    Характеристическое отношение коалиции $\{1,2\}$ равно $\rho_2\circ 
(\rho_1\cup \rho_2^G)$ (см.~(\ref{e3.5-vas}) и~(\ref{e3.6-vas})). Сначала вы\-чис\-лим 
$\rho_1\cup \rho_2^G \hm= \{ (0,1), (0,5), (4,0), (3,4), (3,2),\linebreak (7,6)\} \hm\cup \{ (1,4), (3,4), 
(6,4)\}$, а~затем~--- максимальные элементы отношений, входящих 
в~композицию~$\rho^{\Gamma^{2,3}}$: 

\vspace*{-3pt}

\noindent
     \begin{align*}
     M_1&\triangleq \mathrm{MAX}\, \rho_1\cup \rho_2^G =\{ 2,5\};\\
     M_2&\triangleq \mathrm{MAX}\,\rho_2\vert_{M_1} =\{5\};\\
     \mathrm{MAX}\, a_3\vert_{M_2} &=\{5\}\Rightarrow x^* =5\simeq (1,0,1).
   \end{align*}
   
   \vspace*{-10pt}

\section{Заключение}

\vspace*{-1pt}

    Композициональная структура РОИ многих лиц 
выражает формализацию правил игры, процессов формирования коалиций и~выбора 
игроками до\-пус\-ти\-мых стратегий. Подобное представление иг-\linebreak ры сокращает перебор 
при поиске рациональных решений. С~его по\-мощью получено обобщение 
классической тео\-ре\-мы Гермейера. Мо\-дуль\-ность композиции ре\-зуль\-ти\-ру\-юще\-го 
отношения упро\-щает разработку и~оптимизацию муль\-ти\-агент\-ных сис\-тем. 
Пред\-ло\-жен\-ный метод исследования и~чис\-лен\-но\-го решения игровой задачи 
нуж\-да\-ет\-ся в~алгоритмизации. 
    
{\small\frenchspacing
 {%\baselineskip=12pt
 %\addcontentsline{toc}{section}{References}
 \begin{thebibliography}{99}

\bibitem{2-vas}
\Au{Моисеев Н.\,Н.} Элементы теории оптимальных сис\-тем.~--- М.: Наука, 1974. 
526~с.

\bibitem{1-vas}
\Au{Гермейер Ю.\,Б.} Игры с~непротивоположными интересами.~--- М.: Наука, 1976. 
326~с.

\bibitem{3-vas}
\Au{Подиновский~В.\,В., Ногин~В.\,Д.} Па\-ре\-то-оп\-ти\-маль\-ные решения 
многокритериальных задач.~--- М.: Наука, 1982. 256~с.
\bibitem{4-vas}
\Au{Розен~В.\,В.} Применение тео\-рии бинарных отношений к~об\-щей тео\-рии игр~// 
Математические методы решения экономических задач.~--- Новосибирск: Наука, 1982. С.~127--152.
\bibitem{5-vas}
\Au{Васильев~Н.\,С.} Коалиционно устойчивые эффективные равновесия в~моделях 
коллективного поведения с~обменом информацией~// Информатика и~её 
применения, 2015. Т.~9. Вып.~2. С.~2--13. doi: 10.14357/19922264150201.
\bibitem{6-vas}
\Au{Bai~Q., Ren~F., Fujita~K., Znang~M.} Multi-agent and complex systems.~--- Studies in 
computational intelligence ser.~--- Luxembourg: Springer, 2016. 210~p.
\bibitem{7-vas}
\Au{Скорняков~Л.\,А.} Элементы общей ал\-геб\-ры.~--- М.: Наука, 1983. 272~с.
\bibitem{8-vas}
\Au{Маклейн~С.} Категории для ра\-бо\-та\-юще\-го математика~/
Пер. с~англ.~--- М.: Физматлит, 2004.  352~с.
(\Au{Mac Lane~S.} Categories for the working mathematician.~---  
Berlin\,--\,Heidelberg\,--\,New York: Springer, 1978. 317~p.)
\bibitem{9-vas}
\Au{Shoham~Y., Leyton-Brown~R.} Multiagent systems: Algorithmic, game-theoretic, and 
logical foundations.~--- Cambridge University Press, 2010. 532~p.

\bibitem{11-vas}
\Au{Dixit~A.\,K., Natebuff~B.\,J.} The art of strategy.~--- New York, London: 
W.\,W.~Norton \& Co., 2008. 446~p.

\bibitem{10-vas}
\Au{Dixit~A.\,K., Skeath~S., Reiley~D.\,H., Jr.} Games of strategy.~--- New York, 
London: W.\,W.~Norton \& Co., 2017. 880~p.

\end{thebibliography}

 }
 }

\end{multicols}

\vspace*{-8pt}

\hfill{\small\textit{Поступила в~редакцию 12.03.23}}

\vspace*{6pt}

%\pagebreak

%\newpage

%\vspace*{-28pt}

\hrule

\vspace*{2pt}

\hrule

%\vspace*{-2pt}

\def\tit{MULTIPLAYERS' GAMES COMPOSITIONAL STRUCTURE IN~THE~MONOIDAL CATEGORY 
OF BINARY RELATIONS}


\def\titkol{Multiplayers' games compositional structure in~the~monoidal category 
of binary relations}


\def\aut{N.\,S.~Vasilyev}

\def\autkol{N.\,S.~Vasilyev}

\titel{\tit}{\aut}{\autkol}{\titkol}

\vspace*{-14pt}


\noindent
N.\,E.~Bauman Moscow State Technical University, 5-1  Baumanskaya 2nd Str., Moscow 105005, 
Russian Federation


\def\leftfootline{\small{\textbf{\thepage}
\hfill INFORMATIKA I EE PRIMENENIYA~--- INFORMATICS AND
APPLICATIONS\ \ \ 2023\ \ \ volume~17\ \ \ issue\ 2}
}%
 \def\rightfootline{\small{INFORMATIKA I EE PRIMENENIYA~---
INFORMATICS AND APPLICATIONS\ \ \ 2023\ \ \ volume~17\ \ \ issue\ 2
\hfill \textbf{\thepage}}}

\vspace*{3pt}
     


\Abste{ The system approach is suggested for multiplayers' games solution that meets 
up-to-date network technologies. It allows to optimize the functionality of multiagent systems. 
The monoidal category of binery relations is applied to make games rules 
description and players' behavior study and modification. The game problem is to
maximize, if possible, the preference relations of all participants in the game. Their composition in the monoidal 
binary relations category in correspondence with games rules defines resulting game 
relation (RGR). Players' rational behavior search is reduced to RGR maximum elements 
choice. The author formalizes the use of various classes of permissible strategies, 
information exchange processes, and coalitions formation. The RGR existence is proved and 
maximum RGR elements structure is studied. Moves priority and absolutely optimal 
preference relations significance are clarified for the coalitions formation process.}

\KWE{player's preference relations: absolutely optimal relation, guarantied relation, moves priority relation;
game graph; permissible strategy; rational solution; 
coalition characteristic relation;  resulting game relation; monoidal category;  compositionality}

  \DOI{10.14357/19922264230203}{GPMZTS}

%\vspace*{-18pt}

%\Ack
%\noindent
  

%\vspace*{12pt}

  \begin{multicols}{2}

\renewcommand{\bibname}{\protect\rmfamily References}
%\renewcommand{\bibname}{\large\protect\rm References}

{\small\frenchspacing
 {%\baselineskip=10.8pt
 \addcontentsline{toc}{section}{References}
 \begin{thebibliography}{99} 

\bibitem{2-vas-1}
\Aue{Moiseev, N.\,N.} 1975. \textit{Ele\-men\-ty teo\-rii op\-ti\-mal'\-nykh sis\-tem} [Elements of optimal systems 
theory]. Moscow: Nauka. 527~p.

\bibitem{1-vas-1}
\Aue{Germeyer, Yu.\,B.} 1976. \textit{Ig\-ry s~nep\-ro\-ti\-vo\-po\-lozh\-ny\-mi in\-te\-re\-sa\-mi} 
 [Games with non-opposite interests]. Moscow: Nauka. 326~p.
 
\bibitem{3-vas-1}
\Aue{Podinovskiy, V.\,V., and V.\,D.~Nogin.} 1982. \textit{Pa\-re\-to-optimal'nye re\-she\-niya 
mno\-go\-kri\-te\-ri\-al'\-nykh za\-dach} [Pareto optimal solutions in multicriteria problems]. Moscow: Nauka. 
256~p.
\bibitem{4-vas-1}
\Aue{Rozen, V.\,V.} 1982. Pri\-me\-ne\-nie teo\-rii bi\-nar\-nykh ot\-no\-she\-niy k~ob\-shchey teo\-rii igr [Application 
of the theory of binary relations to general game theory]. \textit{Matematicheskie metody resheniya 
ekonomicheskikh zadach} [Mathematical methods for solving economic problems]. Novosibirsk: Nauka. 127--152.
\bibitem{5-vas-1}
\Aue{Vasilyev, N.\,S.} 2014. Ko\-a\-li\-tsi\-on\-no us\-toy\-chi\-vye ef\-fek\-tiv\-nye rav\-no\-ve\-siya v~mo\-de\-lyakh 
kol\-lek\-tiv\-no\-go po\-ve\-de\-niya s~ob\-me\-nom in\-for\-ma\-tsi\-ey [On availability of Pareto effective equilibrium 
situations in collective behavior models with data exchange]. \textit{Informatika i~ee Primeneniya~--- 
Inform. Appl.} 9(2):2--13. doi: 10.14357/19922264150201.
\bibitem{6-vas-1}
\Aue{Bai, Q., F.~Ren, K.~Fujita, and M.~Znang.} 2016. \textit{Multi-agent and complex systems}. 
Studies in computational intelligence ser. Luxembourg: Springer. 210~p.
\bibitem{7-vas-1}
\Aue{Skornyakov, L.\,A.} 1983. \textit{Ele\-men\-ty ob\-shchey al\-geb\-ry} [Elements of general algebra]. 
Moscow: Nauka. 272~p.
\bibitem{8-vas-1}
\Aue{Mac Lane, S.} 1978. \textit{Categories for the working mathematician}.  
Berlin\,--\,Heidelberg\,--\,New York: Springer. 317~p.
\bibitem{9-vas-1}
\Aue{Shoham, Y., and R.~Leyton-Brown.} 2010. \textit{Multiagent systems: Algorithmic, game-theoretic, 
and logical foundations}. Cambridge University Press. 532~p.

\bibitem{11-vas-1}
\Aue{Dixit, A.\,K., and B.\,J.~Nalebuff.} 2008. \textit{The art of strategy}. New York, London: 
W.\,W.~Norton \& Co. 446~p.

\bibitem{10-vas-1}
\Aue{Dixit, A.\,K., S.~Skeath, and  D.\,H.~Reiley, Jr.} 2017. \textit{Games of strategy}. New York, 
London: W.\,W.~Norton \& Co. 880~p.
\end{thebibliography}

 }
 }

\end{multicols}

\vspace*{-6pt}

\hfill{\small\textit{Received March 12, 2023}} 

\Contrl

\noindent
\textbf{Vasilyev Nikolai S.} (b.\ 1952)~--- Doctor of Science in physics and mathematics, professor, 
N.\,E.~Bauman Moscow State Technical University, 5-1~Baumanskaya 2nd Str., Moscow 105005, 
Russian Federation; \mbox{nik8519@yandex.ru}
     



\label{end\stat}

\renewcommand{\bibname}{\protect\rm Литература} 