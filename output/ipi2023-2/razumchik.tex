\def\stat{razumchik}

\def\tit{ВЕРОЯТНОСТНАЯ МОДЕЛЬ ДЛЯ~ОЦЕНКИ ОСНОВНЫХ
ХАРАКТЕРИСТИК ПРОИЗВОДИТЕЛЬНОСТИ\\ МАРКОВСКОЙ МОДЕЛИ 
СУПЕРКОМПЬЮТЕРА$^*$}

\def\titkol{Вероятностная модель для оценки %основных
характеристик производительности марковской модели 
суперкомпьютера}

\def\aut{Р.\,В.~Разумчик$^1$, А.\,С.~Румянцев$^2$, Р.\,М.~Гаримелла$^3$}

\def\autkol{Р.\,В.~Разумчик, А.\,С.~Румянцев, Р.\,М.~Гаримелла}

\titel{\tit}{\aut}{\autkol}{\titkol}

\index{Разумчик Р.\,В.}
\index{Румянцев А.\,С.}
\index{Гаримелла Р.\,М.}
\index{Razumchik R.\,V.}
\index{Rumyantsev A.\,S.}
\index{Garimella R.\,M.}


{\renewcommand{\thefootnote}{\fnsymbol{footnote}} \footnotetext[1]
{Исследование выполнено за счет гранта Российского 
научного фонда №\,21-71-10135. Работа проводилась с~использованием 
инфраструктуры Центра коллективного пользования <<Высокопроизводительные 
вычисления и~большие данные>> (ЦКП <<Информатика>>) ФИЦ ИУ РАН (г.~Москва).}}


\renewcommand{\thefootnote}{\arabic{footnote}}
\footnotetext[1]{Федеральный исследовательский центр <<Информатика и~управление>> Российской 
академии наук, \mbox{rrazumchik@frccsc.ru}}
\footnotetext[2]{Институт прикладных математических исследований 
Карельского научного центра Российской академии наук, \mbox{ar0@krc.karelia.ru}}
\footnotetext[3]{Университет Махиндра, Индия,  \mbox{rama.murthy@mahindrauniversity.edu.in}}

%\vspace*{-12pt}


\Abst{Рассматривается известная модель суперкомпьютера в~виде
двухлинейной экспоненциальной неконсервативной системы массового обслуживания (СМО)
с~очередью неограниченной ем\-кости и~с~одновременным занятием и~одновременным освобождением заявкой
случайного числа приборов. Впервые доказано,
что ее основные вероятностные характеристики обслуживания
могут быть получены из принципиально другой модели,
представляющей собой однолинейную СМО неограниченной емкости
с~потерями поступающих заявок, но без искусственных простоев прибора.
C~привлечением развитого аналитического аппарата анализа обобщенных
процессов размножения и~гибели (ПРГ) показано, что в~важных частных случаях совместное нестационарное
распределение вероятностей ее состояний пред\-став\-ля\-ет\-ся в~мат\-рич\-но-гео\-мет\-ри\-че\-ской форме и~может быть найдено 
(в~терминах преобразования Лапласа (ПЛ))
с~помощью метода на основе информации о~пересечении уровней.
Приведены примеры численных расчетов, иллюстрирующие
некоторые характеристики установленной связи между двумя моделями.}


\KW{суперкомпьютер; система массового обслуживания (СМО); 
неконсервативность дис\-цип\-ли\-ны обслуживания; нестационарный режим}

\DOI{10.14357/19922264230209}{KXYHPO}
  
\vspace*{3pt}


\vskip 10pt plus 9pt minus 6pt

\thispagestyle{headings}

\begin{multicols}{2}

\label{st\stat}


\section{Введение}

\vspace*{-3pt}


В области вычислительных и~телекоммуникационных систем широко применяются 
системы параллельного обслуживания, такие как многопроцессорные вычислительные 
устройства общего назначения, суперкомпьютеры, сис\-те\-мы передачи данных и~сис\-те\-мы 
хранения. Для исследования характеристик производительности таких сис\-тем 
вследствие случайного характера по\-сту\-па\-ющей нагрузки и~зачастую наличия 
очередей применяются методы теории массового обслуживания и~многолинейные 
СМО. В~этом классе СМО важную роль играет подкласс 
систем с~одновременным занятием и~одновременным освобождением заявкой случайного 
числа приборов на одно и~то же случайное\linebreak время. Сейчас в~научном сообществе по 
исследованию операций он~активно применяется для моделирования и~анализа 
характеристик производительности современных суперкомпьютеров и~\mbox{центров} 
обработки данных~\cite{ee0, ee1, o1, o2, o3}. Однако ранее СМО этого класса 
применялись также к~исследованию сис\-тем социального обслуживания~\cite{lg1,ssk}. 
Особенность этих\linebreak сис\-тем заключается в~неконсервативности про\-цесса 
нагрузки\footnote[4]{Или, другими словами, неконсервативности
дисциплины обслуживания. Напомним (см., например,~\cite[с.~49]{yash}), что 
свойство консервативности дисциплины обслуживания означает, что дли\-тель\-ность 
обслуживания заявки не зависит от дисциплины обслуживания и~нет искусственных 
простоев прибора.}: приборы могут простаивать при не\-пус\-той очереди.
Это обстоятельство существенно затрудняет анализ даже сис\-тем малого размера~\cite{ee4, chak}, 
и,~как следствие, многие вопросы здесь остаются либо открытыми, 
либо недостаточно разработанными (см. более подробно~\cite{ee311}).

Эта статья посвящена, главным образом, модели суперкомпьютера, представляющей 
собой двухлинейную
неконсервативную марковскую СМО с~двумя типами заявок и~дисциплиной обслуживания
в~порядке поступления (см., например,~\cite{tlm} и~подробное описание ниже, в~начале разд.~3).
Хотя эта модель в~научной литературе исследована достаточно подробно~\cite{ee1,o1,o2,o3,lg1,ssk,ee4,chak,tlm,mor22,melikov,unwin,afanaseva19,grosof},
здесь, по-ви\-ди\-мо\-му, впервые показано, что ее основные вероятностные 
характеристики обслуживания могут быть получены из принципиально другой модели,
представляющей собой СМО, в~которой, во-пер\-вых, нет
искусственных простоев прибора и,~во-вто\-рых, имеют место потери заявок
при неограниченной емкости очереди. Для наглядности ее описание
(см.\ разд.~2) дано в~рамках дискретной системы поллинга (о~них см., например,~\cite{viii}),
в~которой анализу подвергается только очередь высшего приоритета.
В~отличие от исходной модели, в~которой процесс обслуживания имеет конкретное физическое наполнение
(заключающееся в~выполнении одного или одновременно нескольких суперкомпьютерных
заданий), в~новой модели каждый акт обслуживания является виртуальным
и~однозначного физического наполнения не~имеет. 

Таким образом,
новая сис\-те\-ма пред\-став\-ля\-ет собой отвлеченную математическую
модель, результаты анализа которой требуют интерпретации.
Для распределения очереди она дана в~разд.~3 (см.\ соотношение~\eqref{loc}).
Для временн$\acute{\mbox{ы}}$х характеристик вопрос остается открытым,
но вы\-чис\-ли\-тель\-ные эксперименты указывают на наличие здесь
некоторых закономерностей (см.\ разд.~5).
Судя по публикациям в~открытой печати, внимание подробному исследованию
предложенной в~разд.~2 СМО не уделялось\footnote{И связано это, во-видимому,
с~тем, что с~точки зрения вероятностных характеристик она представляет
собой лишь частный случай хорошо известной СМО $\mathrm{SM}/\mathrm{MSP}/n/r$ (см.,
например,~\cite{nn1, nn3}).}. Поэтому в~разд.~4 внимание уделено
стандартному вопросу расчета совместного нестационарного
распределения вероятностей ее состояний (и,~ввиду~\eqref{loc}, состояний
связанной с~ней суперкомпьютерной модели) и~показано,
что в~некоторых частных случаях оно может быть найдено
(в~терминах ПЛ) известным, но нестандартным методом 
на основе информации о~пересечении уровней~\cite{zhang}.

Далее используются следующие обозначения: $\mathbf{I}$~--- тождественная мат\-ри\-ца;
$\mathbf{0}$~--- нулевая мат\-ри\-ца; $\vec 1$~--- вектор, состоящий из единиц.
Там, где раз\-мер\-ность векторов и~мат\-риц не~ясна из контекста, она
определяется нижним индексом.


\section{Описание модели и~ее вероятностные характеристики}
%\label{sec2}


Пусть имеется некоторая дискретная система поллинга с~одним обслуживающим 
прибором, ${N\hm\ge 1}$ очередями и~приоритетным порядком их обхода.
Переключения прибора между очередями происходят мгновенно.
Будем рассматривать только очередь, имеющую наивысший приоритет. Предположим, 
что число мест для ожидания в~ней неограниченно
и~в~нее поступает пуассоновский поток групп заявок интенсивности~$\lambda$.
Число заявок, приходящих в~каждой группе, равно двум.
Если в~момент поступления группы в~очереди находится нечетное число заявок,
то~поступающая группа целиком помещается в~нее, иначе только одна заявка из группы (другая считается потерянной).
Внутри очереди заявки обслуживаются в~порядке поступления.
Дисциплина обслуживания очереди прибором~--- исчерпывающая,
т.\,е.\ очередь обслуживается до тех пор, пока не опустеет.

Процесс обслуживания относится к~марковскому типу и~определяется следующим 
образом. Если в~очереди находятся~$k$, ${k\hm\ge 0}$, заявок, то процесс обслуживания
может находиться в~одном из~$l_k$, ${1 \hm\le l_k\hm<\infty}$, состояний (фаз 
обслуживания), причем ${l_0\hm=1}$.
Тогда если в~некоторый момент обслуживания в~очереди
находятся ${k\hm\ge 1}$ заявок и~фаза обслуживания
равна~$i$, ${1\hm\le i \hm\le l_k}$, то за <<малое>> время~$\Delta$ с~вероятностью
${\lambda_{ij}^{(k)} \Delta \hm+ o(\Delta)}$ ни одна заявка не покинет очередь
и~фаза обслуживания изменится на~$j$, ${1\hm \le j \hm\le l_{k}}$,
а~с~ве\-ро\-ят\-ностью ${n_{ij}^{(k)} \Delta \hm+ o(\Delta)}$
обслуживание закончится,
причем если~$k$~--- нечетное число, то очередь мгновенно покинет
одна заявка, иначе две, и~(в~обоих случаях) фаза обслуживания изменится
на~$j$, ${1 \hm\le j \hm\le l_{k-1-(k-1\,\mathrm{mod}\,2)}}$.
Матрицы из элементов $\lambda_{ij}^{(k)}$ и~${n_{ij}^{(k)}}$
будем обозначать соответственно ${\bm{\Lambda}}^{(k)}$ и~${\mathbf{N}}^{(k)}$, ${k \hm\ge 1}$.
Кроме того, будем предполагать, что начиная с~некоторого номера~$k$ все~$l_k$ равны, 
${{\bm{\Lambda}}^{(k)}\hm={\mathbf{L}}}$, ${{\mathbf{N}}^{(k)}={\mathbf{N}}}$, мат\-ри\-ца 
${{\bm{\Lambda}}+ {\mathbf{N}}}$ неразложимая, а~ма\-тр\-ица ${\mathbf{N}}$ ненулевая.
Наконец, будем предполагать, что если в~момент поступления в~очереди
имеются ${k \hm\ge 0}$ других заявок и~фаза обслуживания равна~$i$, ${1\hm \le i \hm\le  l_k}$, то
после поступления она с~вероятностью~$\omega_{ij}^{(k)}$
изменится на~$j$, ${1 \hm\le j \hm\le l_{k+1+(k\,\mathrm{mod}\,2)}}$.
Матрицы из элементов~$\omega_{ij}^{(k)}$ будем обозначать ${\bm{\Omega}}^{(k)}$, 
${k \hm\ge 0}$,
и~считать, что ${{\bm{\Omega}}^{(k)}={\bm{\Omega}}}$ начиная с~некоторого  номера~$k$.


Обозначим через~$\hat{X}(t)$ число заявок в~очереди, а~через $\hat{Y}(t)$~--- фазу обслуживания в~момент~$t$.
Положим 
$$
\hat{Z}(t)=\left(\hat{X}(t),\hat{Y}(t)\right).
$$
При сделанных предположениях относительно входящего потока
и~процесса обслуживания случайный процесс
$\{\hat{Z}(t), t \hm\ge 0\}$ является марковским.
Множество его состояний имеет
вид ${\{(k,j),k \hm \ge 0,\ 1 \hm\le j \hm\le l_k \}}$,
где индексы~$k$ и~$j$ указывают соответственно
число заявок в~очереди и~фазу обслуживания.
Следуя терминологии обобщенных~ПРГ, назовем уровнем процесса $\{\hat{Z}(t),\ t \ge 0\}$
число заявок в~очереди, а~его фазой~--- фазу обслуживания.
Из описания модели следует, что, если рассматриваемый процесс
находится на уровне~$k$, возможны переходы либо с~текущего уровня на один или 
на два уровня выше, либо внут\-ри текущего уровня,
либо с~текущего уровня на один или два уровня ниже.
Так, если~$k$, $k\hm\ge 1$, есть чис\-ло нечетное, то
строки матрицы интенсивностей переходов\footnote{Первая строка
матрицы~$\hat{\mathbf{Q}}$, соответствующая нулевому уровню,
имеет вид ${(- \lambda \, \, \lambda {\bm{\Omega}}_0 \,\, \mathbf{0} \,\, \cdots  )}$.}
$\hat{\mathbf{Q}}$,
соответствующие $k$-му и~$(k+1)$-му уровням процесса $\{\hat{Z}(t), t \ge 0\}$,  
имеют вид:

\vspace*{-3pt}

\noindent
\begin{multline}
\hat{\mathbf{Q}}
={}\\
\!\!\!\!\!=\!\!
\begin{pmatrix}\!
\vdots  & \ddots  &   \vdots  &   \vdots &   \vdots &  \vdots  &   \vdots \\
\underbrace{\mathbf{0} \, \cdots \,  \mathbf{0}}_{k-1} & \mathbf{N}^{(k)} &  \mathbf{M}^{(k)} 
&   \mathbf{0} &   \lambda \bm{\Omega}^{(k)} &  \mathbf{0} &   \cdots \\
\underbrace{\mathbf{0} \, \cdots \,  \mathbf{0}}_{k-2} & \mathbf{N}^{(k+1)} &    \mathbf{0}  
&  \mathbf{M}^{(k+1)} &  \lambda \bm{\Omega}^{(k+1)} &  \mathbf{0} &   \cdots \\
\vdots  & \vdots  &   \vdots  &   \vdots &   \ddots &  \vdots  &   \vdots \!\!
\end{pmatrix}\!,\!
\label{e1}
\end{multline}
где матрицы $\mathbf{M}^{(k)}$ вычисляются по формуле 
$$
\mathbf{M}^{(k)}=\bm{\Lambda}^{(k)}-\mathrm{diag}
\left(\lambda \bm{\Omega}^{(k)} \vec{1} \right).
$$

Введем обозначение 
$$
\vec{\hat {p}}_k(t)\hm=({\hat p}_{k,1}(t),\ldots\linebreak \ldots,{\hat p}_{k,l_k}(t)),
$$
 где ${\hat p}_{(k,j)}(t)\hm=\mathsf{P} 
(\hat{Z}(t)\hm=(k,j))$~--- вероятность того, что процесс $\{\hat{Z}(t), t \hm\ge 0\}$ находится в~состоянии~$(k,j)$ в~момент~$t$.
Принципиально вопрос\linebreak расчета совместного распределения $\vec{\hat {p}}(t)\hm=({\hat { p}}_0(t),\vec{\hat { p}}_1(t),\dots)$
решается путем путем группировки элементов~\eqref{e1} в~блоки вида
$$
\begin{pmatrix} \mathbf{0} &  \mathbf{N}^{(k)} \\ \mathbf{0}& \mathbf{N}^{(k+1)} 
\end{pmatrix};\enskip 
\begin{pmatrix} \mathbf{M}^{(k)} & \mathbf{0} \\ \mathbf{0}& \mathbf{M}^{(k+1)} 
\end{pmatrix}; \enskip
\begin{pmatrix} 
\lambda \bm{\Omega}^{(k)} & \mathbf{0} \\  \lambda \bm{\Omega}^{(k+1)} & \mathbf{0}
\end{pmatrix},
$$
которые с~некоторого номера~$k$ становятся регулярными,
и~далее применением одного из~множества известных методов расчета распределений 
очередей в~СМО,
описываемых обобщенными ПРГ (для стационарного случая см., например,~\cite[теорема~2]{nn2}, \cite[с.~242]{ppav}; 
для нестационарного, например,~\cite{nn0,nn0rrr}).

Умея вычислять совместное распределение по стандартным алгоритмам, нетрудно
рассчитать (хотя бы приближенно) и~основные вероятностные характеристики 
обслуживания. Так, вероятность~${\hat {q}}(t)$ потери заявки\footnote[2]{Считая в~покидающей очередь
группе одну из двух заявок необслуженной, можно ввести такие характеристики, как (виртуальная) вероятность 
(не)обслуживания. Их расчет более сложен и~требует отдельного анализа.} в~момент~$t$
равна 
$$
{\hat { q}}(t)\hm={\sum\nolimits_{k=0}^{\infty} \vec{\hat { p}}_{2k}(t){\vec { 1}}}\,.
$$

\vspace*{-12pt}

\columnbreak

\noindent
Из классической формулы Литтла можно получить стационарное\footnote[3]{Критерий 
существования которого может быть выписан в~явном виде, так как после указанной выше 
группировки матрица~$\hat{\mathbf{Q}}$ оказывается трехдиагональной c~повторяющимися блоками на <<верхних  уровнях>>.}
среднее  время~${\hat V}$ пребывания заявки в~очереди, равное
${\hat V}\hm=\sum\nolimits_{k=1}^{\infty} k \vec{\hat { p}}_{k}(\infty){\vec { 1}}/(\lambda 
(2-{\hat { q}}(\infty)))$.
 Другие временн$\acute{\mbox{ы}}$е характеристики требуют 
дополнительных построений, и~их нахождение, по крайней мере в~явном виде, 
остается открытым вопросом.


\section{Связь с~моделью суперкомпьютера}

Рассмотрим модель суперкомпьютера (см., например,~\cite{tlm,ee1}) в~виде СМО 
с~двумя идентичными приборами (единичной скорости),
обслуживающими заявки (в~порядке поступления) из \mbox{единственной} очереди 
неограниченной емкости. Заявки поступают в~очередь по пуассоновскому закону с~параметром~$\lambda$.
Для выполнения каждой заявке с~ве\-ро\-ят\-ностью ${p_1\hm\in (0,1)}$ требуется один 
прибор, а~с~дополнительной ве\-ро\-ят\-ностью $p_2\hm= 1-p_1$~--- два; конкретное число становится 
известным в~момент поступления заявки в~сис\-те\-му и~остается фиксированным для данной заявки.
Времена обслуживания заявок являются независимыми случайными величинами и~имеют экспоненциальное распределение
с~параметром~$\mu_i$, где~$i$~--- число тре\-бу\-емых заявкой приборов (тип заявки).
Предполагается, что занятие (и~освобождение) приборов заявкой второго типа
происходит одновременно. Заявка поступает на обслуживание, во-пер\-вых, когда
подошла ее очередь и,~во-вто\-рых, когда требуемое ей число приборов свободно.


Обозначим через ${X}(t)$ общее число заявок в~сис\-те\-ме в~момент~$t$,
через~${Y}(t)$~--- комбинацию типов двух старейших заявок в~сис\-те\-ме 
в~момент~$t$,
под\-разу\-ме\-вая, что 
$$
Y(t)=\begin{cases}
1\,, &\!\! \mbox{если\ в~момент~$t$\ обе\ старейшие\ заявки}\\
&\!\!\mbox{первого\ типа};\\
2\,, &\!\! \mbox{если\ в~момент~$t$\ старейшая\ заявка}\\
&\!\!\mbox{первого\ типа,\ а~вторая\ 
старейшая~---}\\
&\!\!\mbox{второго\ типа\ (ожидает\ на\ первой}\\ 
&\!\!\mbox{позиции\ в~очереди)};\\
3 &\!\! \mbox{в~остальных\ случаях}.
\end{cases}
$$

 Заметим, что если $X(t)\hm=1$, то~$Y(t)$~--- 
тип единственной заявки в~сис\-те\-ме в~момент~$t$. Положим\footnote[1]{При $X(t)=0$ вторая компонента процесса 
опускается.} ${Z}(t)\hm=({X}(t),{Y}(t))$. Напомним
(см., например,~\cite{threel}), что мат\-ри\-ца интенсивностей
$\mathbf{Q}$ переходов процесса $\{{Z}(t), t \hm\ge 0\}$ имеет вид:
\begin{equation}
\mathbf{Q} =
\begin{pmatrix}
\mathbf{A}^{0,0} &  \mathbf{A}^{0,1}  & \mathbf{0} & \mathbf{0} & \mathbf{0}   & \cdots\\
\mathbf{A}^{1,0}  & \mathbf{A}^{1,1} & \mathbf{A}^{1,2}  & \mathbf{0} & \mathbf{0} &    \cdots\\
\mathbf{0} & \mathbf{A}^{2,1}  &  \mathbf{A}^{(0)}  & \mathbf{A}^{(1)} & \mathbf{0}  & \cdots\\
\mathbf{0} & \mathbf{0} & \mathbf{A}^{(-1)} & \mathbf{A}^{(0)} & \mathbf{A}^{(1)}  & \cdots\\
\vdots & \vdots & \vdots& \vdots& \ddots&  \vdots\\
\end{pmatrix},
\label{QBD_Q2}
\end{equation}
 и, значит, $\{{Z}(t), t \ge 0\}$~--- обобщенный ПРГ,
причем ${X}(t)$~--- уровень процесса; ${Y}(t)$~--- его фаза.
Ненулевые блоки в~\eqref{QBD_Q2} имеют сле\-ду\-ющую структуру:
\begin{gather*}
\mathbf{A}^{0,0}=-\lambda; \quad \mathbf{A}^{0,1}=\lambda (p_1 \,\, p_2 );\\
\mathbf{A}^{1,0} =
\begin{pmatrix}
\mu_1 \\
\mu_2
\end{pmatrix};
\quad
\mathbf{A}^{1,1} \!=\!
-\mathrm{diag} \left ( \mathbf{A}^{1,0} \!+\! \mathbf{A}^{1,2} {\vec 1} \right );
\\
\mathbf{A}^{1,2}
=
\begin{pmatrix}
\lambda p_1 & \lambda p_2 & 0\\
0 & 0 & \lambda
\end{pmatrix};
\quad \mathbf{A}^{2,1}=
\begin{pmatrix}
2 \mu_1 & 0\\
0 & \mu_1 \\
p_1 \mu_2 & p_2 \mu_2
\end{pmatrix};
\\
\mathbf{A}^{(0)} =
-\mathrm{diag} \left ( \mathbf{A}^{2,1} \!+\! \mathbf{A}^{(1)} {\vec 1} \right );
\quad
\mathbf{A}^{(1)}=\lambda \mathbf{I}_3;
\\
\mathbf{A}^{(-1)} =
\begin{pmatrix}
2 \mu_1 p_1 & 2 \mu_1 p_2 & 0\\
0 & 0 & \mu_1 \\
p_1^2 \mu_2 & p_1 p_2 \mu_2 & \mu_2 p_2
\end{pmatrix}.
\end{gather*}

Введем обозначения:
\begin{align*}
\vec{{p}}_1(t) &= \left({p}_{1,1}(t),{p}_{1,2}(t)\right);\\
\vec{{ p}}_k(t)&=({p}_{k,1}(t),{p}_{k,2}(t),{p}_{k,3}(t)),\ 
k \ge 2;\\
\vec{{ p}}(t)&=(p_0(t),\vec{{p}}_1(t),\dots),
\end{align*}
где ${p}_{(k,j)}(t)\hm=\mathsf{P}({Z}(t)=(k,j))$~---
вероятность того, что процесс $\{{Z}(t), t \hm\ge 0\}$
находится в~состоянии $(k,j)$ в~момент~$t$.

Рассмотрим бесконечную матрицу $\mathbf{T}$ вида:
$$
\mathbf{T} \!=\!
\mathrm{diag}
\begin{pmatrix}
1,
{\vec 1}_2
\otimes
\mathbf{I}_2,
{\vec 1}_2
\otimes
\mathbf{I}_3
,
{\vec 1}_2
\otimes
\mathbf{I}_3,
\dots
\end{pmatrix},
$$
где $\otimes$~--- кронекерово произведение. Отметим, что $\mathbf{T}$~--- это мат\-ри\-ца Теплица~\cite[с. 86]{gelb}.
Как можно убедиться непосредственной проверкой, если элементы матрицы~$\hat{\mathbf{Q}}$ выбраны
таким образом, что
\begin{multline}
\mathbf{N}^{(1)} = \mathbf{N}^{(2)} = \mathbf{A}^{1,0};
\quad
\mathbf{N}^{(3)} = \mathbf{N}^{(4)} = \mathbf{A}^{2,1};
\\
\mathbf{N}^{(k)} = \mathbf{A}^{(-1)},\enskip k \ge 5;
\\
{\bm {\Omega}}^{(0)} = \mathbf{A}^{0,1};
\quad
{\bm{\Omega}}^{(1)} =\bm{\Omega}_2 =  \mathbf{A}^{1,2};\\
{\bm{\Omega}}^{(k)} = \mathbf{A}^{(1)}, \enskip
 k \ge 3,
\label{para}
\end{multline}
то имеет место тождество $\hat{\mathbf{Q}} \mathbf{T}=\mathbf{T}\mathbf{Q}$.
Поскольку вероятности $\vec{\hat {p}}(t) \mathbf{T}$
удовлетворяют системе дифференциальных уравнений
$({d}/{dt}) \vec{\hat {p}}(t) \mathbf{T}\hm ={\vec{\hat {p}}}(t){\hat{\mathbf{Q}}}\mathbf{T}$,
то с~учетом цепочки равенств $\vec{\hat {p}}(t) (\hat{\mathbf{Q}} \mathbf{T})\hm=
\vec{\hat {p}}(t) (\mathbf{T}\mathbf{Q})\hm =(\vec{\hat {p}}(t) \mathbf{T})\mathbf{Q}$, ассоциативность умножения в~которой
гарантируется конечным числом ненулевых элементов в~каждом столбце матрицы $\mathbf{T}$,
покомпонентно имеем:
\begin{equation}
\label{loc}
\vec{{ p}}(t)=\vec{\hat {p}}(t) \mathbf{T}, \enskip t \ge 0\,.
\end{equation}

\noindent 
Таким образом, вероятности состояний
рассматриваемой суперкомпьютерной модели совпадают с~вероятностями
укрупненных (с~по\-мощью мат\-ри\-цы~$\mathbf{T}$) состояний
частного случая модели из разд.~2.

\section{Нестационарное распределение очереди}


Пусть элементы $\hat{\mathbf{Q}}$ заданы согласно~\eqref{para}.
Нетрудно видеть, что мат\-ри\-ца ${{\hat{\mathbf{Q}}}\hm -s \mathbf{I}}$ допускает пред\-став\-ле\-ние
\begin{equation}
\label{qhatnew}
\begin{pmatrix}
\mathbf{B}_0 & \mathbf{B}_1 & \mathbf{0} & \mathbf{0} &\cdots \\
\mathbf{0} & \mathbf{C} & \mathbf{B} & \mathbf{0} &\cdots \\
\mathbf{0} & \mathbf{0} & \mathbf{C} & \mathbf{B} & \cdots \\
\vdots & \vdots & \vdots & \ddots  & \vdots
\end{pmatrix},
\end{equation}
в котором (не все квадратные) блоки $\mathbf{B}_0$, $\mathbf{B}_1$, $\mathbf{B}$ и~$\mathbf{C}$ зависят от~$s$ и~имеют вид:
$$
\mathbf{C}=
\begin{pmatrix}
\mathbf{A}^{(-1)} & \mathbf{A}^{(0)} \!-\! s \mathbf{I} \\
\mathbf{A}^{(-1)} &\mathbf{0}
\end{pmatrix};
$$
$$
\mathbf{B}
\!=\!
\begin{pmatrix}
\mathbf{0} & \mathbf{A}^{(1)} \\
\mathbf{A}^{(0)} \!-\! s \mathbf{I} & \mathbf{A}^{(1)}
\end{pmatrix};\quad
\ \
\mathbf{B}_1
\!=\!
\begin{pmatrix}
\mathbf{0}  \\
\mathbf{B}
\end{pmatrix};
$$
$$ 
\mathbf{B}_0
\!=\!
\begin{pmatrix}
    \mathbf{A}^{0,0} \!-\! s \mathbf{I} & \mathbf{A}^{0,1} & \mathbf{0} & \mathbf{0}\\
    \mathbf{A}^{1,0} & \mathbf{A}^{1,1}\!-\!s \mathbf{I} & \mathbf{0} & \mathbf{A}^{1,2} \\
    \mathbf{A}^{1,0} & \mathbf{0} & \mathbf{A}^{1,1} \!-\! s \mathbf{I} & \mathbf{A}^{1,2} \\
    \mathbf{0} & \mathbf{0} & \mathbf{A}^{2,1} &\mathbf{A}^{(0)} \!-\! s \mathbf{I}\\
    \mathbf{0} & \mathbf{0} & \mathbf{A}^{2,1} & \mathbf{0}
    \end{pmatrix}.
    $$
    
    \begin{figure*}[b] %fig1
\vspace*{8pt}
\begin{minipage}[t]{80mm}
\begin{center}
   \mbox{%
\epsfxsize=79mm 
\epsfbox{raz-1.eps}
}

\end{center}
\vspace*{-9pt}
\Caption{Условные вероятности вынужденного простоя прибора (\textit{а}) 
и~вероятности потери заявки~(\textit{б}) как функции времени. Среднее время 
обслуживания заявок разных типов одинаковое: \textit{1}~--- $p_1\hm= 8/9$; \textit{2}~--- $2/3$; \textit{3}~--- $p_1\hm= 1/3$}
\end{minipage}
%\end{figure*}
\hfill
%\begin{figure*} %fig2
\vspace*{1pt}
\begin{minipage}[t]{80mm}
\begin{center}
   \mbox{%
\epsfxsize=79mm 
\epsfbox{raz-2.eps}
}

\end{center}
\vspace*{-9pt}
\Caption{Условные вероятности вынужденного простоя прибора~(\textit{а}) и~вероятности потери заявки~(\textit{б}) как функции времени.
Среднее время обслуживания заявок разных типов различное:
\textit{1}~--- $p_1\hm= 8/9$; \textit{2}~--- $2/3$; \textit{3}~--- $p_1\hm= 1/3$}
\end{minipage}
\end{figure*}

    
    \noindent 
    Обозначим распределение вероятностей
     состояний обобщенного ПРГ с~матрицей интенсивностей (при ${s=0}$) 
переходов~\eqref{qhatnew}
     через ${\vec{{\check p}}(t)\hm=(\vec{\check { p}}_0(t),\vec{\check { p}}_1(t),\dots)}$,
     а~его покомпонентное преобразование Лапласа в~точке~$s$, $s\hm\ge 0$,~--- 
     через ${\vec{{\check \pi}}(s)\hm=(\vec{\check { \pi}}_0(s),\vec{\check 
{\pi}}_1(s),\dots)}$. Далее без ограничения общности можно считать, что в~начальный момент процесс 
находится в~нулевом состоянии,  т.\,е.\
 ${{\check p}_0(0)\hm=1}$.
Поскольку ${\vec{{\check \pi}}(s)}$ удовлетворяет системе линейных 
алгебраических уравнений 
$$
{\vec{{\check \pi}}(s) ({\hat{\mathbf{Q}}}-s \mathbf{I})=-\vec{{\check  p}}(0)},
$$
то, если мат\-ри\-ца $\mathbf{C}$ не вырождена,
расчет ${\vec{\check {\pi}}_k(s)}$, ${k\hm\ge 2}$, можно осуществить по
мат\-рич\-но-ре\-кур\-рент\-ной формуле. Однако $\mathbf{C}$ обратимой не~является
(определитель ${\mathbf{A}^{(-1)}}$ равен нулю).
%
Выполним, следуя~\cite{zhang}, операции над столбцами матрицы \eqref{qhatnew},
не изменяющие решения.
Выделим в~\eqref{qhatnew} все группы столбцов, содержащие целиком мат\-ри\-цу~$\mathbf{C}$,
домножим в~каждой группе второй столбец на~${p_2/p_1}$ и~вычтем из первого 
столбца.
Столбцы мат\-ри\-цы~$\mathbf{B}$ автоматически претерпят аналогичные изменения.\linebreak 
В~результате преобразований первый столбец каж\-дой мат\-ри\-цы~$\mathbf{C}$ будет 
содержать только нули. 

Перегруппируем теперь элементы мат\-ри\-цы~\eqref{qhatnew} 
с~сохранением блочной структуры и~обозначений \mbox{сле\-ду\-ющим} образом.
Добавим в~мат\-ри\-цу~$\mathbf{B}_0$ (справа) первый столбец~$\mathbf{B}_1$. Новую мат\-ри\-цу~$\mathbf{B}_1$ 
<<начнем>> со второго столбца прежней и~дополним ее справа нулевым 
столбцом. 

В~качестве новой мат\-ри\-цы~$\mathbf{C}$ возьмем  прежнюю, но начиная со 
второго столбца, и~дополним ее (справа) первым столбцом прежней
мат\-ри\-цы~$\mathbf{B}$. Новую мат\-ри\-цу~$\mathbf{B}$ определим аналогичным образом, 
начиная со второго столбца прежней, дополнив нулевым столбцом (справа). Теперь 
определитель мат\-ри\-цы~$\mathbf{C}$,
равный $\mu_1 \mu_2 p_1 p_2  (\lambda\hm+s)(\lambda\hm+\mu_2\hm+s)(\lambda\hm+\mu_1\hm+s)(\lambda\hm+2\mu_1+s)$,
всегда отличен от нуля и,~таким образом, ПЛ распределения вероятностей состояний~$\vec{{\check \pi}}(s)$
представляется в~мат\-рич\-но-гео\-мет\-ри\-че\-ской форме
$$
{\vec{\check { \pi}}_{k+1}(s)\hm=\vec{\check { \pi}}_1(s)}(- \mathbf{B}\mathbf{C}^{-1})^{k}\,\enskip
{k\ge 1}\,,
$$
 а~векторы $\vec{\check { \pi}}_0(s)$ и~$\vec{\check { \pi}}_1(s)$
определяются единственным образом из системы уравнений
$$\vec{\check { \pi}}_0(s) \mathbf{B}_0 \hm= -(1,0,\dots,0);\quad
{\vec{\check { \pi}}_0(s)\mathbf{B}_1 \hm= - \vec{\check { \pi}}_1(s) \mathbf{C}}.
$$
По~построению число уравнений в~этой системе на~два меньше числа неизвестных.
Как следует из~\cite[теорема 4.1]{zhang}, недостающие два уравнения имеют вид:
$$
\vec{\check {\pi}}_1(s){\vec {\check \psi}}=0\,;\quad 
 \vec{{\pi}}_1(s){\vec {\phi}}\hm=0,
 $$
где $\vec {\psi}$ и~$\vec {\phi}$~--- правые собственный векторы,
соответствующие тем (двум из шести) собственным значениям матрицы ${ - \mathbf{B}\mathbf{C}^{-1}}$,
которые лежат на границе или вне единичного круга.

Хотя полученный результат и~позволяет находить (в~терминах ПЛ)
нестационарные вероятностные характеристики (обеих моделей, рас\-смот\-рен\-ных в~разд.~2 и~3),
окончательные формулы\linebreak не являются замкнутыми в~полном смысле этого слова
и~требуют применения численных методов.
Не~останавливаясь на обсуждении достоинств и~недостатков описанного
решения\footnote{Которое можно назвать спектральным, поэтому см.~\cite[Ch.~13]{161}.}
по \mbox{сравнению} с~традиционными, проиллюстрируем сказанное двумя примерами.
Верхняя группа кривых на рис.~1 и~2 показывает поведение
условной вероятности~${q}(t)$ вынужденного простоя прибора\footnote{Вычисляемой по формуле
${q}(t)={\sum\nolimits_{k=2}^\infty {p}_{k,2}(t)/(2p_0(t)\hm+{p}_{1,1}(t)\hm+\sum\nolimits_{k=2}^\infty 
{p}_{k,2}(t))}$.}
с~ростом~$t$ при трех соотношениях между типами
заявок, когда приборы загружены в~среднем наполовину\footnote{То есть их стационарная средняя загрузка 
$\mathsf{E}U\hm=1\hm-p_0(\infty)\hm-0{,}5 {p}_{1,1}(\infty)\hm-0{,}5\sum\nolimits _{k=2}^\infty 
{p}_{k,2}(\infty)\hm=0{,}5$.}.
Рисунок~1 соответствует однородному случаю, когда ${\mu_1\hm=\mu_2\hm=1}$,
рис.~2~--- неоднородному, когда ${\mu_2\hm=3 \mu_1\hm=3}$.
Нижняя группа кривых на каждом рисунке показывает
соответствующие значения вероятности потерь ${{\hat {q}}(t)}$
в~модели из разд.~2, па\-ра\-мет\-ры которой выбраны согласно~\eqref{para}.



Отметим немонотонный характер изменения вероятности ${q}(t)$ (при 
каждом~$t$) как функции от~$p_1$:
в~одном случае доля времени, в~течение которого прибор вынужденно простаивает,
достигает наибольшего значения, когда существенно преобладает один из
двух типов заявок, в~другом случае~--- когда такого преобладания нет.
Аналогичное по характеру замечание (но, как видно из рис.~2, уже
с~учетом значения~$t$) справедливо
и~для ве\-ро\-ят\-ности~${\hat {q}}(t)$.

\vspace*{-6pt}

\section{Заключение}

В этой статье впервые установлена связь (см.\ соотношение~\eqref{loc}),
существующая между хорошо исследованной в~литературе моделью суперкомпьютера
в~виде двухлинейной неконсервативной СМО\linebreak неограниченной ем\-кости
и отвлеченной математической моделью, представляющей собой
однолинейную СМО с~групповым поступлением, механизмом
активного управ\-ле\-ния очередью (\mbox{приводящего} к~потерям поступающих~заявок)
и групповым обслуживанием, зависящим от со\-сто\-яния очереди.
Характер этой связи требует дальнейшего прояснения.
В~част\-ности, неясно, существует ли для условной
ве\-ро\-ят\-ности вынужденного простоя прибора в~первой СМО
эквивалентная характеристика
во~второй:
как видно из рис.~1 и~2, вероятность~${{\hat {q}}(t)}$\linebreak\vspace*{-12pt}

{ \begin{center}  %fig3
 \vspace*{-3pt}
    \mbox{%
\epsfxsize=79.483mm 
\epsfbox{raz-3.eps}
}

\end{center}

\vspace*{-4pt}

\noindent
{{\figurename~3}\ \ \small{Стационарное среднее время $V$~(\textit{1}) ($\hat V$~(\textit{2}) и~${\hat V}^*$~(\textit{3})) пребывания 
заявки в~модели из разд.~3 (разд.~2) при стационарной средней загрузке 
приборов ${\mathsf{E}\,U\hm=0{,}5}$~(\textit{а}) и~$0{,}6$~(\textit{б}) и~различных 
значениях~$p_1$. Среднее время обслуживания заявок разных типов одинаковое и~равно~1
}}}

\vspace*{12pt}

\noindent
 потери поступающей заявки таковой
не является.
  Как показывают вычислительные эксперименты, с~точки зрения такой характеристики,
как стационарное среднее время пребывания заявки в~сис\-те\-ме,
вторая СМО (при дисциплине FIFO~--- first in, first out) мажорирует первую сверху (рис.~3),
и эти оценки лучше, по край\-ней мере известных из~литературы,
<<наивных>> (со\-от\-вет\-ст\-ву\-ющих случаю ${p_1\hm=0}$). Вместе с~тем вопрос
о~стохастической упо\-ря\-до\-чен\-ности со\-от\-вет\-ст\-ву\-ющих случайных
величин остается открытым.




Что касается описанной в~разд.~2 модели, то, как уже
отмечалось выше, она обобщается на случай полумарковского входящего потока
и~теоретический\footnote[4]{А~с~учетом установленной в~этой статье связи
с~суперкомпьютерными моделями и~практический, поскольку
для последних проблема анализа и~оптимизации
временн$\acute{\mbox{ы}}$х характеристик не исчерпана
(см., например,~\cite{lg1}).}
интерес здесь представляет получение временн$\acute{\mbox{ы}}$х
характеристик при различных\footnote[1]{Обратимся к~примерам на рис.~3.
Если при удалении из очереди двух заявок считать обслуженной
ту, что стоит на последнем месте, то стационарное среднее время пребывания
уменьшается с~$\hat V$ до~${\hat V}^*$.} правилах
учета и~выбора заявок из очереди на обслуживание. Стандартные результаты
для обобщенных ПРГ (см., например,~\cite{ozawa06}) оказываются здесь ма\-ло\-при\-год\-ными.




{\small\frenchspacing
 { \baselineskip=11.5pt
 %\addcontentsline{toc}{section}{References}
 \begin{thebibliography}{99}
\bibitem{ee0}
\Au{Морозов Е.\,В.,  Румянцев А.\,С.} Модели многосерверных систем для анализа 
вычислительного
кластера~// Труды Карельского научного центра РАН, 2011. №\,5. С.~75--85.

\bibitem{ee1}
\Au{Rumyantsev A.,  Morozov~E.} Stability criterion of a~multiserver model with 
simultaneous service~// Ann. Oper. Res., 2017.
Vol.~252. P.~29--39. doi: 10.1007/s10479-015-1917-2.

\bibitem{o1} %3
\Au{Hong Y., Wang~W.} Sharp waiting-time bounds for multiserver jobs~// 
23rd  Symposium (International) on Theory, Algorithmic 
Foundations, and Protocol Design for Mobile Networks and Mobile Computing Proceedings.~--- 
ACM, 2022. P.~161--170. doi: 10.1145/3492866.3549717.

\bibitem{o2} %4
\Au{Grosof I.}  Optimal scheduling in the multiserver-job model under heavy 
traffic~// Proceedings ACM Measurement Analysis Computing Systems, 2022. Vol.~3. No.\,6. Art.~51. 
doi: 10.1145/3570612.

\bibitem{o3} %5
\Au{Wang W., Xie~Q., Harchol-Balter~M.} Zero queueing for multi-server jobs~// 
ACM Sigmetrics Performance Evaluation Review, 2022. Vol.~1. No.\,49. P.~13--14. doi: 
10.1145/ 3543516.3453924.

\bibitem{ssk} %6
\Au{Kim S.} ${M/M/s}$ queueing system where customers demand multiple server 
use: PhD Diss.~--- Dallas, TX, USA: Southern Methodist University, 1979. 104~p.


\bibitem{lg1} %7
\Au{Brill P.\,H., Green~L.} Queues in which customers receive simultaneous 
service from a~random number of servers: A system point approach~// Manage. Sci.,
1984. Vol.~30. No.\,1. P.~51--68.



\bibitem{yash} %8
\Au{Яшков С.\,Ф.} Анализ очередей в~ЭВМ.~--- М.: Радио и~связь, 1989. 216~с.

\bibitem{ee4} %9
\Au{Filippopoulos D.,  Karatza~H.} An ${M/M/2}$ parallel system model with pure
space sharing among rigid jobs~// Math. Comput. Model.,
2007. Vol.~45. P.~491--530.

\bibitem{chak} %10
\Au{Chakravarthy S.\,R., Karatza H.\,D.} Two-server parallel system with pure 
space sharing and Markovian arrivals~// Comput. Oper. Res., 2013.
Vol.~40. No.\,1. P.~510--519.

\bibitem{ee311} %11
\Au{Harchol-Balter M.} Open problems in queueing theory inspired by datacenter
computing~// Queueing Syst., 2021. Vol.~97. P.~3--37.

\bibitem{tlm} %12
\Au{Rumyantsev A., Basmadjian~R., Golovin~A., Astafiev~S.}
 A~three-level modelling approach for asynchronous speed\linebreak\vspace*{-12pt}
 
 \columnbreak
 
 \noindent
  scaling in high-performance data centres~// 
 12th Conference (International) on Future Energy 
Systems e-Energy Proceedings.~--- ACM, 2021. P.~417--423.

\bibitem{unwin} %13
\Au{Unwin A.\,R.}
Results for dual resource queues~// Modelling and performance evaluation methodology~/
Eds. F.~Baccelli, G.~Fayolle.~--- Lecture notes in control and information sciences ser. --- 
Berlin/Heidelberg: Springer-Verlag, 1984.  Vol.~60.
P.~351--370. doi: 10.1007/BFb0005182.

\bibitem{melikov} %14
\Au{Melikov A.\,Z.} Computation and optimization methods for multiresource 
queues~// Cybern.  Syst. Anal., 1996. Vol.~32. No.\,6. P.~821--836. doi: 
10.1007/BF02366862.


\bibitem{afanaseva19} %15
\Au{Afanaseva L., Bashtova~E., Grishunina~S.}
Stability analysis of a~multi-server model with simultaneous service and a~regenerative input flow~//
Methodol. Comput. Appl., 2020. Vol.~22. P.~1439--1455. doi: 
10.1007/s11009-019-09721-9.

%\bibitem{green84}  Brill, P. H.;  Green, L. ``Queues in which customers receive 
%simultaneous service from a random number of servers: a system point approach.'' 
%Management Science, vol. 30, no. 1, pp. 51?68, 1984. 
%https://doi.org/10.1287/mnsc.30.1.51

\bibitem{grosof} %16
\Au{Grosof I., Harchol-Balter~M., Scheller-Wolf~A.}
Stability for two-class multiserver-job systems.~--- Cornell University,  2020. 29~p. \mbox{arXiv}:2010.00631 [cs.PF].

\bibitem{mor22} %17
\Au{Harchol-Balter M.} The multiserver job queueing model~// Queueing Syst., 
2022. Vol.~100. No.\,3-4. P.~201--203. doi: 10.1007/s11134-022-09762-x.



\bibitem{viii} %18
\Au{Вишневский В.\,М.,  Семенова~О.\,В.} Математические методы исследования 
систем поллинга~// Автомат. и~телемех., 2006. Вып.~2. С.~3--56.
%Automation and Remote Control, 2006, Volume 67, Issue 2, Pages 173--220


\bibitem{nn3} %19
\Au{Печинкин А.\,В., Чаплыгин~В.\,В.} Стационарные характеристики системы 
массового обслуживания ${\mathrm{SM}/\mathrm{MSP}/n/r}$~//
Автомат. и~телемех., 2004. Вып.~9. С.~85--100.

\bibitem{nn1} %20
\Au{Dudin A.\,N., Klimenok~V.\,I., Vishnevsky~V.\,M.} The theory of queuing 
systems with correlated flows.~--- Cham, Switzerland: Springer, 2020. 410~p.


\bibitem{zhang} %21
\Au{Zhang J., Coyle~E.\,J.} Transient analysis of quasi-birth-death processes~//
Commun. Stat. Stochastic Models, 1989. Vol.~5. No.\,3. P.~459--496. 
doi: 10.1080/ 15326348908807119.

\bibitem{nn2} %22
\Au{Вишневский В.\,М., Дудин А.\,Н.} Системы массового обслуживания с~коррелированными
входными потоками и~их применение для моделирования телекоммуникационных сетей~// Автомат. и~телемех., 2017. Вып.~8. С.~3--59.


\bibitem{ppav}  %23
\Au{Бочаров  П.\,П., Печинкин А.\,В.}
Теория массового обслуживания.~--- М.: РУДН, 1995. 529~с.

\bibitem{nn0} %24
\Au{Le Ny L.\,M.,  Sericola B.}
Transient analysis of the ${\mathrm{BMAP}/\mathrm{PH}/1}$ queue~//
Int. J.~Simulation Systems Science Technology, 2002. 
Vol.~3. No.\,3. P.~4--14.

\bibitem{nn0rrr} %25
\Au{Hiroyuki M., Takine~T.}
Algorithmic computation of the time-dependent solution of
structured Markov chains and its application to queues~//
Stoch. Models, 2005. Vol.~21. P.~885--912.

\pagebreak

\bibitem{threel} %26
\Au{Rumyantsev A., Basmadjian~R., As\-ta\-fiev~S., Golovin~A.}
Three-level modeling of a~speed-scaling supercomputer~// Ann. Oper. Res., 2022. 
doi: 10.1007/s10479-022-\linebreak \mbox{04830-0}.

\bibitem{gelb} %27
\Au{Гелбаум Б., Олмстед~Д.} Контрпримеры в~анализе~/ Пер. с~англ.~--- М.: Мир, 1967. 251~с.
(\Au{Gelbaum~B.\,R., Olmsted~J.\,M.\,H.} 
{Counterexamples in analysis}.~--- 1st ed.~---
 Dover books on matematics ser.~--- Dover Publicatrions, 1964. 194~p.)


\bibitem{161} %28
\Au{Dshalalow J.\,H.} Advances in queueing: Theory, methods and open problem.~--- London: CRC Press, 1995. 528~p.

\bibitem{ozawa06} %29
\Au{Ozawa T.} Sojourn time distributions in the queue defined by a~general QBD process~//
Queueing Syst., 2006. Vol.~53. No.\,4. P.~203--211. doi: 10.1007/s11134-006-7651-3.

\end{thebibliography}

 }
 }

\end{multicols}

\vspace*{-6pt}

\hfill{\small\textit{Поступила в~редакцию 15.04.23}}

\vspace*{8pt}

%\pagebreak

%\newpage

%\vspace*{-28pt}

\hrule

\vspace*{2pt}

\hrule

%\vspace*{-2pt}

\def\tit{A QUEUEING SYSTEM FOR~PERFORMANCE EVALUATION OF~A~MARKOVIAN SUPERCOMPUTER MODEL}


\def\titkol{A queueing system for~performance evaluation of~a~Markovian supercomputer model}


\def\aut{R.\,V.~Razumchik$^1$, A.\,S.~Rumyantsev$^2$, and~R.\,M.~Garimella$^3$}

\def\autkol{R.\,V.~Razumchik, A.\,S.~Rumyantsev, and~R.\,M.~Garimella}

\titel{\tit}{\aut}{\autkol}{\titkol}

\vspace*{-10pt}


\noindent
$^1$Federal Research Center ``Computer Science and Control'' of the Russian Academy of Sciences, 44-2~Vavilov\linebreak
$\hphantom{^1}$Str., Moscow 119333, Russian Federation

\noindent
$^2$Institute of Applied Mathematical Research of the Karelian Research Center of the Russian Academy of Sciences,\linebreak 
$\hphantom{^1}$11~Pushkinskaya Str., Petrozavodsk 185910, Russian Federation

\noindent
$^3$Mahindra University, 62/1A~Bahadurpally Jeedimetla, Hyderabad 500043, India

\def\leftfootline{\small{\textbf{\thepage}
\hfill INFORMATIKA I EE PRIMENENIYA~--- INFORMATICS AND
APPLICATIONS\ \ \ 2023\ \ \ volume~17\ \ \ issue\ 2}
}%
 \def\rightfootline{\small{INFORMATIKA I EE PRIMENENIYA~---
INFORMATICS AND APPLICATIONS\ \ \ 2023\ \ \ volume~17\ \ \ issue\ 2
\hfill \textbf{\thepage}}}

\vspace*{3pt}




\Abste{Consideration is given to the well-known supercomputer model in the form of a~Markovian nonwork-conserving 
two-server queueing system with unlimited queue capacity, in which customers are served by a~random number 
of servers simultaneously. For the first time, it is shown that its basic probabilistic characteristics 
can be calculated from an~unrelated single-server queueing system with infinite capacity, work conserving scheduling, and forced customers' 
losses. Based on the known matrix-analytic techniques for quasi-birth-and-death processes, it is shown that in certain special cases, the transient 
queue-size distribution can be found (in terms of Laplace transform) using the Level Crossing Information method and has a~matrix-geometric form. 
Numerical examples which illustrate some properties of the established connection between the two queueing systems are provided. } 



\KWE{supercomputer model; queueing system; nonwork-conserving scheduling; transient regime}




\DOI{10.14357/19922264230209}{KXYHPO} 

\vspace*{-18pt}

\Ack

\vspace*{-3pt}


\noindent
The research was funded by the Russian Science Foundation, project No.\,21-71-10135. 
The research was carried out using the infrastructure of the Shared Research Facilities ``High Performance Computing and Big Data'' 
(CKP ``Informatics'') of FRC CSC RAS (Moscow).

%\vspace*{4pt}

  \begin{multicols}{2}

\renewcommand{\bibname}{\protect\rmfamily References}
%\renewcommand{\bibname}{\large\protect\rm References}

{\small\frenchspacing
 {%\baselineskip=10.8pt
 \addcontentsline{toc}{section}{References}
 \begin{thebibliography}{99} 
\bibitem{1-raz}
\Aue{Morozov, E.\,V., and A.\,S.~Rumyantsev.} 2011. Modeli mnogoservernykh sistem dlya analiza vychislitel'nogo 
klastera [Multi-server models to analyze high performance cluster]. \textit{Transactions of the Karelian Research Centre of the
Russian Academy of Sciences} 5:75--85.  


\bibitem{2-raz}
\Aue{Rumyantsev, A., and E.~Morozov.} 2017. Stability criterion of a~multiserver model with simultaneous service. 
\textit{Ann. Oper. Res.} 252:29--39. doi: 10.1007/s10479-015-1917-2.

\bibitem{3-raz}
\Aue{Hong, Y., and W.~Wang.} 2022. Sharp waiting-time bounds for multiserver jobs. 
\textit{23rd  Symposium (International) on Theory, Algorithmic Foundations, and Protocol Design for Mobile Networks and Mobile Computing
Proceedings}. ACM. 161--170. doi: 10.1145/3492866.3549717.

\bibitem{4-raz}
\Aue{Grosof, I.} 2022. Optimal scheduling in the multiserver-job model under heavy traffic.
 \textit{Proceedings ACM Measurement Analysis Computing Systems} 3(6):51. 32~p. doi: 10.1145/ 3570612.

\bibitem{5-raz}
\Aue{Wang, W., Q.~Xie, and M.~Harchol-Balter.} 2022. Zero queueing for multi-server jobs.
\textit{ACM Sigmetrics Performance Evaluation Review} 1(49):13--14. doi: 10.1145/ 3543516.3453924.

\bibitem{7-raz} %6
\Aue{Kim, S.} 1979. ${M/M/s}$ queueing system where customers demand multiple server use.
 Dallas, TX: Southern Methodist University, 1979. PhD Diss. 104~p.

\bibitem{6-raz} %7
\Aue{ Brill, P.\,H., and L.~Green.} 1984. Queues in which customers receive simultaneous service from a~random number of 
 servers: A~system point approach. \textit{Manage. Sci.}  30(1): 51--68.



\bibitem{8-raz}
\Aue{Yashkov, S.\,F.} 1989.
\textit{Analiz ocheredey v~EVM} [Analysis of queues in computer systems]. Moscow: Radio i~svyaz'. 216~p.

\bibitem{9-raz}
\Aue{Filippopoulos, D., and H.~Karatza.} 2007. An ${M/M/2}$ parallel system model with pure
space sharing among rigid jobs.
\textit{Math. Comput. Model.} 45:491--530.

\bibitem{10-raz}
\Aue{Chakravarthy, S.\,R., and H.\,D.~Karatza.} 2013.
Two-server parallel system with pure space sharing and Markovian arrivals.
\textit{Comput. Oper. Res.}   40(1):510--519.

\bibitem{11-raz}
\Aue{Harchol-Balter, M.} 2021. Open problems in queueing theory inspired by datacenter
computing. \textit{Queueing Syst.} 97:3--37.

\bibitem{12-raz}
\Aue{Rumyantsev, A., R.~Basmadjian, A.~Golovin, and S.~As\-tafiev.} 2021.
 A~three-level modelling approach for asynchronous speed scaling in high-performance data centres.
 \textit{12th  Conference (International) on Future Energy Systems \mbox{e-Energy} Proceedings}.  
 ACM. 417--423.
 
 \bibitem{15-raz} %13
\Aue{Unwin, A.\,R.} 1984. Results for dual resource queues.
\textit{Modelling and performance evaluation methodology}. Eds. F.~Baccelli and G.~Fayolle.
Lecture notes in control and information sciences ser. Berlin/Heidelberg: Springer-Verlag.
60:351-370. doi: 10.1007/BFb0005182.



\bibitem{14-raz} %14
\Aue{Melikov, A.\,Z.} 1996.
Computation and optimization methods for multiresource queues. 
\textit{Cybern. Syst. Anal.} 
32(6):821--836. doi: 10.1007/BF02366862.



\bibitem{16-raz} %15
\Aue{Afanaseva, L., E.~Bashtova, and S.~Gri\-shu\-ni\-na.} 2020.
Stability analysis of a multi-server model with simultaneous service and a~regenerative input 
flow. \textit{Methodol. Comput. Appl.} 22:1439--1455. doi: 10.1007/s11009-019-09721-9.

\bibitem{17-raz} %16
\Aue{Grosof, I., M.~Harchol-Balter, and A.~Scheller-Wolf.} 2020.
Stability for two-class multiserver-job systems.
\textit{\mbox{arXiv}.org}. 29~p. 
Available at: {\sf https://arxiv.org/abs/2010.00631} (accessed April~30, 2023).

\bibitem{13-raz} %17
\Aue{Harchol-Balter, M.} 2022.
The multiserver job queueing model.
\textit{Queueing Syst.} 100(3-4):201--203. doi: 10.1007/ s11134-022-09762-x.

\bibitem{18-raz} %18
\Aue{Vishnevskiy, V.\,M., and O.\,V.~Se\-me\-no\-va.} 2006.
Mathematical methods to study the polling systems.
\textit{Automat. Rem. Contr.} 67(2):173--220.

\bibitem{20-raz} %19
\Aue{Pechinkin, A.\,V., and V.\,V.~Chap\-ly\-gin.} 2004. 
 Stationary characteristics of the ${\mathrm{SM}/\mathrm{MSP}/n/r}$ queuing system.
\textit{Automat. Rem. Contr.} 65(9):1429--1443.

\bibitem{19-raz} %20
\Aue{Dudin, A.\,N., V.\,I.~Kli\-me\-nok, and V.\,M.~Vish\-nev\-sky.} 2020.
\textit{The theory of queuing systems
with correlated flows}. Cham, Switzerland: Springer. 410~p.



\bibitem{21-raz}
\Aue{Zhang, J., and E.\,J.~Coyle.} 1989.
Transient analysis of quasi-birth-death processes.
\textit{Commun. Stat. Stochastic Models} 5(3):459--496. 
doi: 10.1080/15326348908807119.

\bibitem{22-raz}
\Aue{Vishnevskii, V.\,M., and A.\,N.~Dudin.} 2017.
Queueing systems with correlated arrival flows and their applications to modeling telecommunication network. 
\textit{Automat. Rem. Contr.} 78:1361--1403.

\bibitem{23-raz}
\Aue{Bocharov,  P.\,P., and A.\,V.~Pechinkin.} 1995.
\textit{Teoriya massovogo obsluzhivaniya} [Queueing theory].
Moscow: RUDN. 529~p.

\bibitem{24-raz}
\Aue{Le Ny, L.~M., and B.~Sericola.} 2002.
Transient analysis of the ${\mathrm{BMAP}/\mathrm{PH}/1}$ queue.
\textit{Int. J. Simulation Systems Science Technology}
3(3):4--14.

\bibitem{25-raz}
\Aue{Hiroyuki, M., and T.~Takine.} 2005. 
Algorithmic computation of the time-dependent solution of 
structured Markov chains and its application to queues.
\textit{Stoch. Models} 21:885--912.

\bibitem{26-raz}
\Aue{Rumyantsev, A., R.~Basmadjian, S.~Astafiev, and A.~Go\-lo\-vin.} 2022.
Three-level modeling of a~speed-scaling supercomputer.
\textit{Ann. Oper. Res.} doi: 10.1007/s10479-022-04830-0.

\bibitem{27-raz}
\Aue{Gelbaum, B.\,R., and J.\,M.\,H.~Olmsted.} 2003.
\textit{Counterexamples in analysis}. Courier Corporation. 195~p.

%\bibitem{Telek}
%Numerical inverse Laplace transformation using concentrated matrix exponential
%distributions / G. Horvth, I. Horvath, S.A. Almousa, M. Telek // Perform.
%Eval. 2020. Vol. 137. P. 102067.

\bibitem{28-raz}
\Aue{Dshalalow, J.\,H.} 1995.
\textit{Advances in queueing: Theory, methods and open problem}.
London: CRC Press. 528~p. 

\bibitem{29-raz}
\Aue{Ozawa, T.} 2006.
Sojourn time distributions in the queue defined by a~general QBD process.
\textit{Queueing Syst.} 53(4):203--211.  doi: 10.1007/s11134-006-7651-3.
\end{thebibliography}

 }
 }

\end{multicols}

\vspace*{-6pt}

\hfill{\small\textit{Received April 15, 2023}} 

\Contr


\noindent
\textbf{Razumchik Rostislav V.} (b.\ 1984)~---
Doctor of Science in physics and mathematics, leading scientist,
Institute of Informatics Problems, Federal Research Center ``Computer Science and Control'' 
of the Russian Academy of Sciences, 44-2~Vavilov Str., Moscow 119333, Russian Federation; \mbox{rrazumchik@ipiran.ru}


\vspace*{3pt}

\noindent
\textbf{Rumyantsev Alexander S.} (b.\ 1986)~---
Doctor of Science in physics and mathematics, senior scientist,
Institute of Applied Mathematical Research of the 
``Karelian Research Center of the Russian Academy of Sciences, 11~Pushkinskaya Str., Petrozavodsk 185910, Russian Federation; \mbox{ar0@krc.karelia.ru}

\vspace*{3pt}

\noindent
\textbf{Garimella Rama Murthy} (b.\ 1962) ---
PhD in computer engineering, professor,
Department of Computer Science and Engineering, Mahindra University, 62/1A
Bahadurpally Jeedimetla, Hyderabad 500043, India; \mbox{rama.murthy@mahindrauniversity.edu.in}



   
\label{end\stat}

\renewcommand{\bibname}{\protect\rm Литература} 