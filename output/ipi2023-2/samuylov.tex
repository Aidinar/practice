\def\stat{samouylov}

\def\tit{К МОДЕЛИРОВАНИЮ ЭФФЕКТОВ ОБСЛУЖИВАНИЯ МНОГОАДРЕСНОГО 
ТРАФИКА В~СЕТЯХ 5G NR$^*$}

\def\titkol{К моделированию эффектов обслуживания многоадресного 
трафика в~сетях 5G NR}

\def\aut{А.\,К.~Самуйлов$^1$, А.\,А.~Платонова$^2$, В.\,С.~Шоргин$^3$, 
Ю.\,В.~Гайдамака$^4$}

\def\autkol{А.\,К.~Самуйлов, А.\,А.~Платонова, В.\,С.~Шоргин, Ю.\,В.~Гайдамака}

\titel{\tit}{\aut}{\autkol}{\titkol}

\index{Самуйлов А.\,К.}
\index{Платонова А.\,А.}
\index{Шоргин В.\,С.}
\index{Гайдамака Ю.\,В.}
\index{Samouylov A.\,K.}
\index{Platonova A.\,A.}
\index{Shorgin V.\,S.} 
\index{Gaidamaka Yu.\,V.}


{\renewcommand{\thefootnote}{\fnsymbol{footnote}} \footnotetext[1]
{Исследование выполнено за счет Российского научного фонда (грант №\,21-79-00142).}}


\renewcommand{\thefootnote}{\arabic{footnote}}
\footnotetext[1]{Российский университет дружбы народов, samuylov-ak@rudn.ru}
\footnotetext[2]{Российский университет дружбы народов, platonova-aa@rudn.ru}
\footnotetext[3]{Федеральный исследовательский центр <<Информатика и~управление>> Российской академии наук, 
\mbox{vshorgin@ipiran.ru}}
\footnotetext[4]{Российский университет дружбы народов; Федеральный исследовательский 
центр <<Информатика и~управ\-ле\-ние>> Российской академии наук, 
\mbox{gaydamaka-yuv@rudn.ru}}

\vspace*{-12pt}
 
 
     
  
  \Abst{Многоадресная передача данных в~сетях беспроводного доступа поз\-во\-ля\-ет 
эффективно предостав\-лять услугу группе абонентов и~оказывается полезной для сокращения 
ресурса, необходимого для обслуживания пользователей, за\-пра\-ши\-ва\-ющих одни и~те же 
данные. Поддержка этой функции в~современной технологии 5G~New Radio (NR) и~будущих 
субтерагерцевых сис\-те\-мах~6G сталкивается с~особенностями, связанными с~использованием 
фазированных антенных решеток (ФАР), фор\-ми\-ру\-ющих на\-прав\-лен\-ные лучи. Пред\-став\-лен\-ная 
 модель обслуживания многоадресного и~одноадресного трафика поз\-во\-ля\-ет 
исследовать области значений па\-ра\-мет\-ров сети связи~5G/6G для снижения плот\-ности 
размещения базовых станций (БС) при поддержании качества предостав\-ле\-ния услуг абонентам.}
   
  \KW{5G; 6G; многоадресная передача; мил\-ли\-мет\-ро\-вые волны; терагерцевые час\-то\-ты; 
фазированные антенные решетки; технологии радиодоступа; математическое 
моделирование}

\DOI{10.14357/19922264230210}{SLMGZU} 
  
\vspace*{3pt}


\vskip 10pt plus 9pt minus 6pt

\thispagestyle{headings}

\begin{multicols}{2}

\label{st\stat}
  
\section{Введение}

  Приложения недалекого будущего, такие как дополненная (AR, augmented reality) и~виртуальная 
реальность (VR, virtual reality), телеприсутствие, видео~8/16~K, требуют резкого 
увеличения ско\-рости передачи по радиоинтерфейсу~[1]. Значительное 
увеличение ско\-рости до~20~Гбит/с на БС ожидается 
в~сис\-те\-мах 5G~NR за счет использования мил\-ли\-мет\-ро\-во\-го 
диапазона длин волн (mmWave) (с~час\-то\-та\-ми~24--52,6~ГГц), а~для сотовых 
сис\-тем~6G планируется использовать ниж\-нюю часть диапазона 
мил\-ли\-мет\-ро\-вых волн (с~час\-то\-та\-ми~52,6--100~ГГц) и~даже субтерагерцевый 
диапазон час\-тот (100--300~ГГц)~[2], потенциально повышая ско\-рость доступа 
до~100~Гбит/с на одну БС.

\begin{figure*}[b] %fig1
\vspace*{-6pt}
\begin{center}
   \mbox{%
\epsfxsize=106.848mm 
\epsfbox{sam-1.eps}
}

\end{center}
\vspace*{-9pt}
\Caption{Схема СМО для модели совместного обслуживания многоадресного 
и~одноадресного трафика в~сетях 5G/6G}
\end{figure*}
  
  Многоадресная передача является важ\-ной возможностью сетей 
беспроводного доступа, по\-вы\-ша\-ющей эф\-фек\-тив\-ность использования ресурсов 
при обслуживании абонентов сети, за\-пра\-ши\-ва\-ющих одни и~те же данные~[3]. 
Если для предоставления услуги одноадресной передачи необходимо 
организовать отдельную сессию для каждого абонента, то для одновременного 
предоставления услуги многоадресной передачи нескольким абонентам из 
группы многоадресной до\-став\-ки информации достаточно организовать одну 
многоадресную сессию. Под сессией понимается процесс непрерывной 
передачи данных от БС, ини\-ци\-иро\-ван\-ный запросом абонента на 
предостав\-ле\-ние со\-от\-вет\-ст\-ву\-ющей услуги и~завершающийся освобождением 
выделенного на БС радиоресурса~[4]. По сравнению с~проводными сетями 
в~беспроводных сис\-те\-мах одновременная передача данных нескольким 
абонентским устройствам (АУ) за\-труд\-не\-на не только из-за от\-ли\-ча\-ющих\-ся 
условий распространения радиосигнала до разных АУ, но и~вследствие 
использования ФАР, фор\-ми\-ру\-ющих 
несколько вы\-со\-ко\-на\-прав\-лен\-ных лучей, что приводит к~не\-воз\-мож\-ности 
обслуживания всех АУ из группы многоадресной до\-став\-ки 
одним лучом. Преимущество вы\-со\-ко\-на\-прав\-лен\-ных лучей при этом заключается 
в~увеличении радиуса по\-кры\-ва\-емо\-го лучом сек\-то\-ра, что поз\-во\-ля\-ет размещать 
БС на большем расстоянии друг от друга и~снижает капительные затраты 
оператора сети. Таким образом, возникает задача исследования особенностей 
обслуживания многоадресного трафика в~сетях 5G~NR с~точ\-ки зрения баланса 
для оператора сети между эф\-фек\-тив\-ностью использования радиоресурсов, 
качеством предостав\-ле\-ния услуг и~за\-тра\-та\-ми на развертывание сети.

\vspace*{-6pt}
  
\section{Формализация модели}

  Рассмотрим сценарий с~одной БС, предоставляющей $M\hm= \vert 
\mathcal{M}\vert$ услуг, тре\-бу\-ющих многоадресную, и~$K\hm= 
\vert\mathcal{K}\vert$ услуг, тре\-бу\-ющих одноадресную доставку информации. 
Абонент формирует запрос на услугу многоадресной передачи с~ве\-ро\-ят\-ностью~$p_M$ 
  и~на услугу одноадресной передачи с~ве\-ро\-ят\-ностью~$p_U$, 
$p_M\hm+p_U\hm=1$. Назовем\linebreak $({I}, m)$-сес\-си\-ей многоадресную 
сессию класса~$m$, со\-от\-вет\-ст\-ву\-ющую услуге~$m$, и~будем считать, что 
по\-сту\-пив\-ший от абонента запрос с~ве\-ро\-ят\-ностью~$p_{M,m}$ пред\-став\-ля\-ет 
собой запрос на предостав\-ле\-ние услуги~$m$, $m\hm= 1,2,\ldots , M$. 
Аналогично $({II}, k)$-сес\-сия~--- одноадресная сессия класса~$k$, 
со\-от\-вет\-ст\-ву\-ющая услуге~$k$ и~ве\-ро\-ят\-ности~$p_{U,k}$, $k\hm= 1,2,\ldots, K$. 
При этом $\sum\nolimits^M_{m=1} p_{M,m} \hm= p_M$, 
$\sum\nolimits^K_{k=1} p_{U,k} \hm= p_U$.
  
  Пусть $\lambda_{UE}$~--- интенсивность поступления запросов на 
установление сессии от абонента, запрашивающего одну из~$M$ 
многоадресных или одну из~$K$ одноадресных услуг. Тогда $\Lambda\hm= 
\lambda_{UE}N_{UE}$~--- общая ин\-тен\-сив\-ность по\-ступ\-ле\-ния запросов 
от~$N_{UE}$ абонентов в~зоне покрытия БС. Для оценки числа~$N_{UE}$ при 
заданной плот\-ности абонентов~$\lambda_B$ делаем предположение 
о~равномерном распределении АУ в~зоне покрытия антенны БС, 
об\-слу\-жи\-ва\-ющей сектор 120$^\circ$ с~радиусом~$r$~[5]. В~этом случае 
$N_{UE}\hm= \lambda_B \pi r^2/3$, а~интенсивности по\-ступ\-ле\-ния на БС 
запросов на предостав\-ле\-ние услуги класса~$m$ и~класса~$k$ равны 
$\lambda_m\hm= p_{M,m}\Lambda$ и~$\nu_k\hm= p_{U,k}\Lambda$ 
соответственно. Кроме этого, для всех классов заданы сред\-ние длительности 
предостав\-ле\-ния услуги абоненту~$\mu_m^{-1}$ и~$\kappa_k^{-1}$, а~так\-же 
объем требуемого ресурса~$b_m$ и~$d_k$ в~первичных ресурсных блоках (РБ), причем два 
последних па\-ра\-мет\-ра, а~так\-же общее чис\-ло~$V$ доступных РБ 
на БС зависят от размера блока, спект\-раль\-ной эф\-фек\-тив\-ности, полосы 
пропускания БС и~могут быть вы\-чис\-ле\-ны, как показано в~[6].


  Описанный сценарий моделируется мультисервисной сис\-те\-мой массового 
обслуживания (СМО) с~потерями~[7], схема которой пред\-став\-ле\-на на рис.~1. Сис\-те\-ма 
поз\-во\-ля\-ет моделировать предостав\-ле\-ние двух типов услуг многоадресной 
передачи: услуг передачи хранимых данных (например, мультимедийное 
оповещение для целей общественной безопас\-ности, массовые обновления 
программного обеспечения и~про\-шив\-ки в~интернете вещей) и~услуг \mbox{передачи} 
данных в~реальном времени (например, средства массовой информации, 
развлечения, включая вещание в~дополненной и~виртуальной ре\-аль\-ности, 
создание профессионального \mbox{контента} в~беспроводной студии с~несколькими 
камерами, иммерсивное видеопроизводство в~реальном времени)~[8--10]. 
Ключевое отличие заключается в~дли\-тель\-ности сессии для предостав\-ле\-ния 
многоадресной услуги на стороне БС, т.\,е.\ \mbox{интервала} за\-ня\-тости ресурса, 
выделенного на БС для непрерывного предостав\-ле\-ния услуги абонентам из 
группы многоадресной до\-став\-ки информации. Про\-дол\-жи\-тель\-ность интервала 
за\-ня\-тости отражает\linebreak особенности многоадресной передачи хранимых данных 
и~данных в~реальном времени и~моделируется в~СМО двумя дис\-цип\-ли\-на\-ми 
<<прозрачного>> обслуживания~[11]~--- Tr$_1$ для многоадресной передачи 
хранимых данных и~Tr$_2$ для многоадресной передачи данных в~реальном 
времени. Изменение чис\-ла абонентов $\xi^{(1)}(t)$ для~Tr$_1$ и~$\xi^{(2)}(t)$ 
для~Tr$_2$, об\-слу\-жи\-ва\-емых в~течение одной многоадресной сессии, показано 
на рис.~2.




 
  
  Для обеих дисциплин запрос абонента на предостав\-ле\-ние услуги класса 
$({I}, m)$ принимается к~обслуживанию, когда при по\-ступ\-ле\-нии он не 
находит на обслуживании запросов такого же класса и~свободны не 
менее~$b_m$ РБ. В~этом случае $({I}, m)$-за\-прос ини\-ци\-иру\-ет 
интервал за\-ня\-тости ресурса запросами класса $({I}, m)$ и~занимает~$b_m$ РБ в~течение случайного интервала времени, не зависящего от 
моментов по\-ступ\-ле\-ния и~длительностей обслуживания многоадресных 
и~одноадресных запросов других классов. Все $({I}, m)$-за\-про\-сы, 
по\-сту\-пив\-шие в~течение этого интервала за\-ня\-тости, т.\,е.\ в~период, когда на 
обслуживании находится один или несколько запросов класса $({I}, 
m)$, принимают-\linebreak\vspace*{-12pt}

{ \begin{center}  %fig2
 \vspace*{-1pt}
   \mbox{%
\epsfxsize=79mm 
\epsfbox{sam-2.eps}
}

\end{center}



\noindent
{{\figurename~2}\ \ \small{Интервалы за\-ня\-тости при передаче хранимых данных (Tr$_1$)~(\textit{а}) 
и~при передаче данных в~реальном времени (Tr$_2$)~(\textit{б})
}}}

\vspace*{9pt}

\addtocounter{figure}{1}

\noindent
ся в~сис\-те\-му и~получают обслуживание без выделения 
дополнительных РБ. Для обеих дис\-цип\-лин до\-ступ в~сис\-те\-му  
$({I}, m)$-за\-про\-са блокируется, если в~сис\-те\-ме нет запросов этого 
класса, а~свободного ресурса недостаточно для начала этой сессии.




  
  Существует принципиальное различие в~моделировании момента окончания 
интервала за\-ня\-тости для дис\-цип\-лин~Tr$_1$ и~Tr$_2$. Для дис\-цип\-ли\-ны~Tr$_1$\linebreak 
все об\-слу\-жи\-ва\-ющи\-еся $({I}, m)$-за\-про\-сы покидают\linebreak сис\-те\-му 
одновременно в~момент окончания обслуживания первого запроса, 
ини\-ци\-иро\-вав\-ше\-го период за\-ня\-тости~[12]. Для дис\-цип\-лины~Tr$_2$ 
об\-слу\-жи\-ва\-ющи\-еся $({I}, m)$-за\-про\-сы покидают сис\-те\-му 
в~\mbox{произвольные} моменты, а~интервал за\-ня\-тости заканчивается, когда все 
выделенные РБ осво\-бож\-да\-ют\-ся при окончании обслуживания последнего 
$({I}, m)$-за\-про\-са, не оставившего после себя в~сис\-те\-ме запросов 
того же класса~[13]. Под ниж\-ни\-ми горизонтальными стрелками на схемах 
рис.~2 указаны сред\-ние значения дли\-тель\-ности интервала за\-ня\-тости для 
обеих дис\-цип\-лин. Здесь $\rho_m\hm= (1\hm+ \rho_{UE,m})^{p_{M,m}N_{UE}} 
\hm-1$ со\-от\-вет\-ст\-ву\-ет пред\-ла\-га\-емой нагрузке запросами класса $({I}, m)$ 
от всех АУ в~секторе покрытия антенны, при этом $\rho_{UE,m}\hm= p_{M,m} 
\lambda_{UE}/\mu_m$~--- пред\-ла\-га\-емая на\-груз\-ка запросами класса 
$({I}, m)$ от одного абонента, $m\hm\in \mathcal{M}$.


  Обслуживание запросов абонентов на предо\-став\-ле\-ние одноадресной услуги 
класса $({II}, k)$ моделируется классической СМО с~потерями~[14], для которой задана ин\-тен\-сив\-ность~$\nu_k$ 
входящего пуассоновского потока, экспоненциальное случайное время 
обслуживания запроса $\kappa_k^{-1}$ и~требование к~ресурсу для 
обслуживания запроса~$d_k$ РБ. Пред\-ла\-га\-емая на\-груз\-ка класса $({II}, k)$ 
обозначена $a_k\hm= \nu_k/\kappa_k$, $k\hm\in \mathcal{K}$.
  
\section{Метрики качества предоставления услуг}

  Функционирование СМО описывается марковским процессом (МП) 
$$
\mathbf{Z}(t)= \left(Y_1(t), \ldots , Y_M(t),N_1(t),\ldots , N_K(t)\right),\enskip 
t\geq 0\,. 
$$
Здесь $Y_m(t)$, $t\hm\geq 0$, $m\hm\in \mathcal{M}$,~--- индикатор наличия в~сис\-те\-ме 
$({I},m)$-за\-про\-сов в~момент~$t$: 
$$
Y_m(t)=\begin{cases}
1, & \mbox{если\ в~момент\ $t$\ обслуживается}\\
&\mbox{хотя\ бы\ один\ $({I}, m)$-за\-прос};\\
0 & \mbox{иначе}.
\end{cases}
$$
 Марковский процесс $N_k(t)$, 
$t\hm\geq0$, отражает чис\-ло  
$({II}, k)$-за\-про\-сов в~момент~$t$, $N_k(t)\hm\in \{ 0,1,2,\ldots , \lfloor 
V/d_k\rfloor\}$. Со\-сто\-яние МП $\mathbf{Z}(t)$ имеет вид 
$(\mathbf{y},\mathbf{n})$, а~пространство со\-сто\-яний опре\-де\-ля\-ет\-ся~как 
  \begin{multline*}
  \mathcal{Z}= \left\{
  \vphantom{\sum\limits^M_{m=1}}
  (\mathbf{y},\mathbf{n})\in \{0,1\}^M \times 
\{0,1,2,\ldots\}^K :\right.\\ 
\left.  \sum\limits^M_{m=1} b_m y_m +\sum\limits^K_{k=1} d_k n_k\leq V\right\}.
  \end{multline*}
  
  \begin{figure*}[b] %fig3
\vspace*{-3pt}
\begin{center}
   \mbox{%
\epsfxsize=160.643mm 
\epsfbox{sam-4.eps}
}

\end{center}
\vspace*{-11pt}
\Caption{Вероятность блокировки доступа в~сис\-те\-му многоадресных~(\textit{а}) 
и~одноадресных~(\textit{б}) запросов в~за\-ви\-си\-мости от ISD: \textit{1}~--- 
$\mathrm{ISD}\hm= 321$; \textit{2}~--- 446; \textit{3}~--- 620; \textit{4}~--- 
$\mathrm{ISD}\hm= 863$}
\end{figure*}
  
  В~[15] показано, что стационарное распределение МП $\mathbf{Z}(t)$ имеет 
мультипликативное пред\-став\-ление

\noindent
  \begin{equation}
  p(\mathbf{y},\mathbf{n}) =G^{-1}(\mathcal{Z}) \!\prod\limits^M_{m=1} 
\gamma_m^{y_m} \prod\limits^k_{k=1} \fr{a_k^{n_k}}{n_k!}\,, \\
  (\mathbf{y},\mathbf{n})\in \mathcal{Z}\,,
  \label{e1-sam}\!\!
  \end{equation}
  
    \vspace*{-2pt}
    
    \noindent
с нормировочной константой $G(\mathcal{Z})$, пред\-став\-лен\-ной в~виде

\vspace*{1pt}

\noindent
$$
G(\mathcal{Z}) =\sum\limits_{\mathbf{z}\in\mathcal{Z}} \prod\limits^M_{m=1} 
\gamma_m^{y_m} \prod\limits^K_{k=1} \fr{a_k^{n_k}}{n_k !}\,.
$$

  \vspace*{-1pt}

\noindent
Здесь $\gamma_m$ имеет смысл загрузки БС многоадресными 
сессиями, которая различается для услуг передачи хранимых данных и~услуг 
передачи данных в~реальном времени и~опре\-де\-ля\-ет\-ся~как 
$$
\gamma_m= \begin{cases}
\rho_m & \mbox{для}\ \mathrm{Tr}_1\,;\\
e^{\rho_m}-1 &\mbox{для}\ \mathrm{Tr}_2\,.
\end{cases}
$$

  \vspace*{-2pt}
  
  Основные характеристики качества предостав\-ле\-ния услуг мож\-но найти 
суммированием стационарных вероятностей~(\ref{e1-sam}) по со\-от\-вет\-ст\-ву\-ющим 
подмножествам~$\Omega$ пространства со\-сто\-яний~$\mathcal{Z}$:
  \begin{equation}
  {\sf P}\{ (\mathbf{y},\mathbf{n})\in \Omega\} =\!\!\!\!\sum\limits_{(\mathbf{y},\mathbf{n}) 
\in \Omega} \!\!\! p(\mathbf{y},\mathbf{n})= \fr{G(\Omega)}{G(\mathcal{Z})}\,,\enskip 
\Omega\subseteq \mathcal{Z}\,.\!\!
\label{e2-sam}
  \end{equation}
  
  \vspace*{-2pt}

  Так, множество состояний блокировки доступа в~сис\-те\-му для 
многоадресного запроса класса~$({I},m)$ определяется выражением

\vspace*{-6pt}

\noindent
  \begin{multline*}
  \hspace*{-8pt}\mathcal{B}_m^I= \!\left\{ (\mathbf{y},\mathbf{n})\in\mathcal{Z} : 
\sum\limits^M_{m=1}\! b_my_m +\!\sum\limits^K_{k=1} \!d_k n_k+b_m > V\,,\right.\\  
\left.y_m=0
\vphantom{\sum\limits^M_{m=1}}
\right\},\enskip m\in \mathcal{M}\,,
 % \label{e3-sam}
  \end{multline*}
а~для одноадресного запроса класса $({II},k)$~--- выражением

\vspace*{-2pt}

\noindent
\begin{multline*}
\hspace*{-7pt}\mathcal{B}_k^{II} =\!\left\{ (\mathbf{y},\mathbf{n}) \in\mathcal{Z} : 
\sum\limits^M_{m=1} \!b_m y_m +\sum\limits^K_{k=1}\! d_k n_k +d_k 
\!>\!V\!\right\},\hspace*{-1.89pt}\\
 k\in\mathcal{K}\,.
%\label{e4-sam}
\end{multline*}
  
  Соответствующие вероятности $B_m^I\hm= {\sf P}\{ (\mathbf{y}, 
\mathbf{n})\hm\in \mathcal{B}^I_m\}$ и~$B_k^{II} \hm= {\sf P}\{ 
(\mathbf{y},\mathbf{n})\hm\in \mathcal{B}_k^{II}\}$ можно найти  
с~по\-мощью~(\ref{e2-sam}).

\section{Результаты численного эксперимента}

\vspace*{-3pt}

  Для иллюстрации эффектов обслуживания многоадресного трафика при 
использовании час\-тот мил\-ли\-мет\-ро\-во\-го диапазона длин волн 
и~субтерагерцевого диапазона час\-тот рас\-смот\-ре\-на сота сети 5G~NR 
с~техническими па\-ра\-мет\-ра\-ми из~[16], радиус которой в~за\-ви\-си\-мости от ФАР 
(от $4\times4$ до~$32\times 4$) может варь\-и\-ро\-вать\-ся от~107 до~288~м. 
Абоненты получают одну одноадресную услугу и~одну услугу многоадресной 
передачи данных в~реальном времени, при этом для по\-след\-ней высокая 
на\-прав\-лен\-ность лучей в~технологии 5G~NR может привести к~не\-об\-хо\-ди\-мости 
поддерживать одновременно несколько многоадресных сессий, чис\-ло которых 
ограничено чис\-лом антенных элементов ФАР. 
  
  На рис.~3 приведены графики ве\-ро\-ят\-ности блокировки доступа в~сис\-те\-му 
для многоадресного и~одноадресного трафика при $p_M\hm= 0{,}5$ 
в~за\-ви\-си\-мости от плот\-ности~$\lambda_B$ абонентов в~зоне покрытия БС 
для~4~значений рас\-сто\-яния меж\-ду БС (ISD, inter-site distance). 
  

  Заметим, что увеличение ин\-тен\-сив\-ности по\-ступ\-ле\-ния запросов от 
абонентов из зоны покры-\linebreak тия сначала приводит к~увеличению ве\-ро\-ят\-ности\linebreak 
блокировки доступа в~сис\-те\-му запросов обоих классов. Однако начиная 
с~определенной ин\-тен\-сив\-ности ве\-ро\-ят\-ность блокировки многоадресных 
запросов начинает уменьшаться, как показано на рис.~3,\,\textit{а}. Объяснение 
заключается в~том, что воз\-рас\-та\-ет ве\-ро\-ят\-ность обслуживания в~сис\-те\-ме\linebreak\vspace*{-12pt}

\pagebreak

\noindent
 многоадресной сессии, которая занимает ресурс,
 в~том чис\-ле для обслуживания 
в~рамках текущей сессии всех сле\-ду\-ющих по\-сту\-па\-ющих запросов на эту 
многоадресную услугу. Вышеупомянутый эффект
 приводит к~резкому 
увеличению ве\-ро\-ят\-ности блокировки доступа для одноадресных запросов, как 
показано на рис.~3,\,\textit{б}, вплоть до момента, когда сис\-те\-ма начинает 
обслуживать почти исключительно многоадресные запросы. Подобный эффект 
обычно наблюдается, когда пред\-ла\-га\-емая нагрузка многоадресных запросов 
увеличивается или когда падает пред\-ла\-га\-емая нагрузка одноадресных запросов, 
и~усиливается для ФАР с~большим чис\-лом элементов, со\-от\-вет\-ст\-ву\-ющих 
большим значениям рас\-сто\-яния между БС соседних сот сети.
  
  Проиллюстрированная неявная приоритизация многоадресных сессий не 
всегда может быть предпочтительной для оператора сети, поскольку она может 
блокировать запросы на одноадресные услуги с~более высоким приоритетом. 
Однако фактический баланс между вероятностями блокировки доступа 
одноадресных и~многоадресных запросов существенно зависит от функции 
по\-лез\-ности сетевого оператора. Последний может обеспечить соблюдение 
необходимого баланса,  введя, например, приоритеты на этапе приема запросов 
на обслуживание.
  
\section{Заключение}

  В работе предложена математическая модель обслуживания 
  многоадресного и~одноадресного трафика на БС в~сис\-те\-мах~5G/6G при использовании 
вы\-со\-ко\-на\-прав\-лен\-ных антенных решеток, характерных для сетей 5G~NR. 
Показано, что со\-вмест\-ная передача многоадресного и~одноадресного трафика 
на радиоинтерфейсе приводит к~ряду эффектов, связанных с~использованием 
ресурсов этими классами трафика. Предложенная модель поз\-во\-ля\-ет 
анализировать области значений па\-ра\-мет\-ров технической сис\-те\-мы для 
эффективного применения ФАР при обслуживании многоадресного трафика, 
в~том чис\-ле оценивать ограничения на рас\-сто\-яние между БС
в~таких сетях.
  
{\small\frenchspacing
 {\baselineskip=11.5pt
 %\addcontentsline{toc}{section}{References}
 \begin{thebibliography}{99}
\bibitem{1-sam}
\Au{David K., Berndt~H.} 6G vision and requirements: Is there any need for beyond 5G?~// IEEE 
Veh. Technol. Mag., 2018. Vol.~13. Iss.~3. P.~72--80. doi: 10.1109/MVT.2018. 2848498.
\bibitem{2-sam}
\Au{Petrov V., Kurner~T., Hosako~I.} IEEE 802.15. 3d: First standardization efforts for  
sub-terahertz band communications toward~6G~// IEEE Commun. Mag., 2020. Vol.~58. Iss.~11. 
P.~28--33. doi: 10.1109/MCOM.001.2000273. 
\bibitem{3-sam}
\Au{Kompella V.\,P., Pasquale~J.\,C., Polyzos~G.\,C.} Multicasting for multimedia applications~// 
Conference on Computer Communications.~--- IEEE, 1992. P.~2078--2085. doi: 
10.1109/INFCOM.1992.263480. 
\bibitem{4-sam}
Multimedia Broadcast/Multicast Service (MBMS); Stage~1 (Release~16): Technical Specification 
22.146 V16.0.0.~--- 3GPP, 2020. {\sf  
https://www.3gpp.org/ftp/ Specs/archive/22\_series/22.146/22146-g00.zip}.
\bibitem{5-sam}
\Au{Moltchanov D.} Distance distributions in random networks~// Ad Hoc Netw., 2012. 
Vol.~10. Iss.~6. P.~1146--1166. doi: 10.1016/j.adhoc.2012.02.005.
\bibitem{6-sam}
\Au{Kovalchukov R., Moltchanov~D., Gai\-da\-ma\-ka~Y., Bob\-ri\-ko\-va~E.} An accurate approximation 
of resource request distributions in millimeter wave 3GPP New Radio systems~// Internet of 
things, smart spaces, and next generation networks and systems~/
Eds.\ O.~Galinina, S.~Andreev, S.~Balandin, Y.~Koucheryavy.~--- Lecture notes in computer science ser.~--- 
Springer, 2019. Vol.~11660. P.~572--585. doi: 10.1007/978-3-030-30859-9\_50.
\bibitem{7-sam}
\Au{Basharin G., Gaidamaka~Y., Samouylov~K.} Mathematical theory of teletraffic and its 
application to the analysis of multiservice communication of next generation networks~// Autom. 
Control Comp.~S., 2013. Vol.~47. P.~62--69. doi:  10.3103/S0146411613020028.

\bibitem{9-sam} %8
\Au{Araniti G., Condoluci~M., Scopelliti~P., Mo\-li\-na\-ro~A., Iera~A.} Multicasting over emerging 
5G Networks: Challenges and perspectives~// IEEE Network, 2017. Vol.~31. No.\,2. P.~80--89. 
doi: 10.1109/MNET.2017.1600067NM.

\bibitem{8-sam} %9
Study on architectural enhancements for 5G multicast-broadcast services (Release~17): Technical 
Report 23.757 V1.2.0.~--- 3GPP, 2020. {\sf 
https://www.3gpp.org/ftp/ Specs/archive/23\_series/23.757/23757-120.zip}.

\bibitem{10-sam}
\Au{Tran T., Navr$\acute{\mbox{a}}$til~D., Sanders~P., Hart~J., Odar\-chen\-ko~R., Barjau~C., 
Altman~B., Burdinat~C., Gomez-Barquero~D.} Enabling multicast and broadcast in the 5G core for 
converged fixed and mobile networks~// IEEE T. Broadcast., 2020. Vol.~66. No.\,2.  
P.~428--439. doi: 10.1109/ TBC.2020.2991548.
\bibitem{11-sam}
\Au{Рыков В.\,В., Самуйлов~К.\,Е.} К~анализу вероятностей блокировок ресурсов сети 
с~динамическими многоадресными со\-еди\-не\-ни\-ями~// Электросвязь, 2000. №\,10. С.~27--30.
\bibitem{12-sam}
\Au{Karvo J., Martikainen~O., Virtamo~J., Aalto~S.} Blocking of dynamic multicast 
connections~// Telecommun. Syst., 2001. Vol.~16. P.~467--481. doi: 10.1023/A:1016631431617.
\bibitem{13-sam}
\Au{Boussetta K., Belyot~A.-L.} Multirate resource sharing for unicast and multicast connections~// 
Broadband communications~/ Eds.\ D.\,H.\,K.~Tsang, P.\,J.~K$\ddot{\mbox{u}}$hn.~--- Boston, MA, USA: Springer, 2000. Vol.~30. P.~561--570. doi: 
10.1007/978-0-387-35579-5\_47.

\pagebreak

\bibitem{14-sam}
\Au{Kelly F.\,P.} Loss networks~// Ann. Appl. Probab., 1991. Vol.~1. Iss.~3.  
P.~319-- 378.  doi: 
10.1214/aoap/1177005872.
\bibitem{15-sam}
\Au{Naumov V., Gaidamaka~Y., Yar\-ki\-na~N., Sa\-mouy\-lov~K.} Matrix and analytical methods for 
performance analysis of telecommunication systems.~--- Springer Nature, 2021. 308~p.
\bibitem{16-sam}
\Au{Samuylov A., Moltchanov~D., Ko\-val\-chu\-kov~R., Pir\-ma\-go\-me\-dov~R., Gai\-da\-ma\-ka~Y., 
And\-re\-ev~S., Kou\-che\-rya\-vy~Y., Samouylov~K.} Characterizing resource allocation trade-offs in 5G 
NR serving multicast and unicast traffic~// IEEE T. Wirel. Commun., 2020. Vol.~19. No.\,5. 
P.~3421--3434. doi: 10.1109/TWC.2020.2973375.

\end{thebibliography}

 }
 }

\end{multicols}

\vspace*{-6pt}

\hfill{\small\textit{Поступила в~редакцию 15.04.23}}

\vspace*{8pt}

%\pagebreak

%\newpage

%\vspace*{-28pt}

\hrule

\vspace*{2pt}

\hrule

%\vspace*{-2pt}

\def\tit{ON MODELING THE EFFECTS OF~MULTICAST TRAFFIC SERVICING IN~5G~NR 
NETWORKS}


\def\titkol{On modeling the effects of multicast traffic servicing in 5G NR 
networks}


\def\aut{A.\,K.~Samouylov$^1$, A.\,A.~Platonova$^1$, V.\,S.~Shorgin$^2$, 
and~Yu.\,V.~Gaidamaka$^{1,2}$}

\def\autkol{A.\,K.~Samouylov, A.\,A.~Platonova, V.\,S.~Shorgin, 
and~Yu.\,V.~Gaidamaka}

\titel{\tit}{\aut}{\autkol}{\titkol}

\vspace*{-10pt}


\noindent
    $^1$RUDN University, 6~Miklukho-Maklaya Str., Moscow 117198, Russian Federation
    
    \noindent
    $^2$Federal Research Center ``Computer Science and Control'' of the Russian Academy of 
Sciences; 44-2~Vavilov\linebreak
$\hphantom{^1}$Str., Moscow 119133, Russian Federation

\def\leftfootline{\small{\textbf{\thepage}
\hfill INFORMATIKA I EE PRIMENENIYA~--- INFORMATICS AND
APPLICATIONS\ \ \ 2023\ \ \ volume~17\ \ \ issue\ 2}
}%
 \def\rightfootline{\small{INFORMATIKA I EE PRIMENENIYA~---
INFORMATICS AND APPLICATIONS\ \ \ 2023\ \ \ volume~17\ \ \ issue\ 2
\hfill \textbf{\thepage}}}

\vspace*{3pt}
  
  
    
  \Abste{Multicasting in wireless access networks allows efficient provision of a~service to 
  a~group of subscribers and is useful for reducing the resource required to serve user equipments 
requesting the same data. The support of this feature in current 5G New Radio (NR) technology and 
future subterahertz (sub-THz) 6G systems faces challenges associated with the use of the 
directional beamforming phased array antennas. The presented multicast and unicast traffic service model 
allows one to explore the range of 5G/6G network parameters to reduce the 
density of base stations while maintaining the quality of services provided to subscribers.}
  
  \KWE{5G; 6G; multicasting; millimeter wave; terahertz; multibeam antennas; 
 multi-RAT; numerical simulation}
  
  
  
\DOI{10.14357/19922264230210}{SLMGZU} 

\vspace*{-14pt}

\Ack
  \noindent
  The reported study was funded by the Russian Science Foundation, project No.\,21-79-00142.

\vspace*{4pt}

  \begin{multicols}{2}

\renewcommand{\bibname}{\protect\rmfamily References}
%\renewcommand{\bibname}{\large\protect\rm References}

{\small\frenchspacing
 {%\baselineskip=10.8pt
 \addcontentsline{toc}{section}{References}
 \begin{thebibliography}{99}
  \bibitem{1-sam-1}
  \Aue{David, K., and H.~Berndt.} 2018. 6G vision and requirements: Is there any need for 
beyond 5G? \textit{IEEE Veh. Technol. Mag.} 13(3):72--80. doi: 10.1109/MVT.2018.2848498.
  \bibitem{2-sam-1}
  \Aue{Petrov, V., T.~Kurner, and I.~Hosako.} 2020. IEEE 802.15. 3d: First standardization 
efforts for sub-terahertz band communications toward 6G. \textit{IEEE Commun. Mag.} 
58(11):28--33. doi: 10.1109/MCOM.001.2000273. 
  \bibitem{3-sam-1}
  \Aue{Kompella, V.\,P., J.\,C.~Pasquale, and G.\,C.~Polyzos.} 1992. Multicasting for multimedia 
applications. \textit{Conference on Computer Communications}. IEEE. 2078--2085. doi: 
10.1109/INFCOM.1992.263480. 
  \bibitem{4-sam-1}
Multimedia broadcast/multicast service (MBMS); Stage~1 (Release~16): Technical specification 
22.146 V16.0.0. 3GPP. Available at:  {\sf 
https://www.3gpp.org/ftp/Specs/ archive/22\_series/22.146/22146-g00.zip} (accessed May~20, 2023).
  \bibitem{5-sam-1}
\Aue{Moltchanov, D.} 2012. Distance distributions in random networks. \textit{AD Hoc Netw.} 
10(6):1146--1166. doi: 10.1016/ j.adhoc.2012.02.005.
  \bibitem{6-sam-1}
  \Aue{Kovalchukov, R., D.~Moltchanov, Y.~Gai\-da\-ma\-ka, and E.~Bob\-ri\-ko\-va.} 2019. An accurate 
approximation of resource request distributions in millimeter wave 3GPP New Radio systems. 
\textit{Internet of things, smart spaces, and next generation networks and systems}. Eds.\ O.~Galinina, S.~Andreev, S.~Balandin,
and Y.~Koucheryavy. Lectures notes 
in computer science ser. Springer. 11660:572--585. doi: 10.1007/978-3-030-30859-9\_50.
  \bibitem{7-sam-1}
  \Aue{Basharin, G., Y.~Gai\-da\-ma\-ka, and K.~Sa\-mouy\-lov.} 2013. Mathematical theory of 
teletraffic and its application to the analysis of multiservice communication of next generation 
networks. \textit{Autom. Control Comp.~S.}  47:62--69. doi:  10.3103/S0146411613020028.
  
  \bibitem{9-sam-1} %8
  \Aue{Araniti, G., M.~Condoluci, P.~Scopelliti, A.~Mo\-li\-na\-ro, and A.~Iera.} 2017. Multicasting 
over emerging 5G networks: Challenges and perspectives. \textit{IEEE Network} 31(2):80--89. doi: 
10.1109/MNET.2017.1600067NM.

\bibitem{8-sam-1} %9
  Study on architectural enhancements for 5G multicast-broadcast services (Release 17): Technical 
Report 23.757 V1.2.0. 3GPP. Available at:  {\sf 
https://www.3gpp.org/ ftp/Specs/archive/23\_series/23.757/23757-120.zip} (accessed May~20, 2023).
  \bibitem{10-sam-1}
  \Aue{Tran, T., D.~Nav$\acute{\mbox{a}}$til, P.~Sanders, J.~Hart, R.~Odar\-chen\-ko, C.~Barjau, 
B.~Altman, C.~Burdinat, and D.~Gomez-Barquero.} 2020. Enabling multicast and broadcast in the 
5G core for converged fixed and mobile networks. \textit{IEEE T. Broadcast.} 66(2):428--439. doi: 
10.1109/ TBC.2020.2991548.
  \bibitem{11-sam-1}
  \Aue{Rykov, V.\,V., and K.\,E.~Samuylov.} 2000. K~analizu ve\-ro\-yat\-no\-stey blokirovok 
resursov seti s~dinamicheskimi mno\-go\-ad\-res\-ny\-mi so\-edi\-ne\-ni\-yami [To the analysis of blocking 
probabilities in a~network with dynamic multicast connections]. \textit{Elektrosvyaz'}  
[Electrosvyaz Magazine] 10:\linebreak 27--30.
  \bibitem{12-sam-1}
  \Aue{Karvo, J., O.~Martikainen, J.~Virtamo, and S.~Aalto.} 2001. Blocking of dynamic 
multicast connections. \textit{Telecommun. Syst.} 16:467--481. doi: 10.1023/A:1016631431617.
  \bibitem{13-sam-1}
  \Aue{Boussetta, K., and A.-L.~Belyot.} 2000. Multirate resource sharing for unicast and 
multicast connections. \textit{Broadband communications}. Eds. D.\,H.\,K.~Tsang and P.\,J.~K$\ddot{\mbox{u}}$hn.
Boston, MA: Springer. 30:561--570. doi: 10.1007/978-0-387-35579-5\_47.
  \bibitem{14-sam-1}
  \Aue{Kelly, F.\,P.} 1991. Loss networks. \textit{Ann. Appl. Probab.} 1(3):319--378. doi: 
10.1214/aoap/1177005872.
  \bibitem{15-sam-1}
  \Aue{Naumov, V., Y.~Gaidamaka, N.~Yar\-ki\-na, and K.~Sa\-mouy\-lov.} 2021. \textit{Matrix and 
analytical methods for performance analysis of telecommunication systems}. Springer Nature. 
308~p.
  \bibitem{16-sam-1}
  \Aue{Samuylov, A., D.~Moltchanov, R.~Ko\-val\-chu\-kov, R.~Pir\-ma\-go\-me\-dov, Y.~Gai\-da\-ma\-ka, 
S.~And\-re\-ev, Y.~Kou\-che\-rya\-vy, and K.~Sa\-mouy\-lov.} 2020. Characterizing resource allocation 
trade-offs in 5G NR serving multicast and unicast traffic. \textit{IEEE T. Wirel. Commun.}  
19(5):3421--3434. doi: 10.1109/TWC.2020.2973375.
  \end{thebibliography}

 }
 }

\end{multicols}

\vspace*{-6pt}

\hfill{\small\textit{Received April 15, 2023}} 
  
  \Contr
  
  \noindent
  \textbf{Samuylov Andrey K.} (b.\ 1988)~--- Candidate of Science (PhD) in physics and 
mathematics, associate professor, Department of Applied Probability and Informatics, RUDN 
University, 6~Miklukho-Maklaya Str., Moscow 117198, Russian Federation;  
\mbox{samuylov-ak@rudn.ru}
  
  \vspace*{3pt}
  
  \noindent
  \textbf{Platonova Anna A.} (b.\ 1996)~--- PhD student, Department of Applied Probability and 
Informatics, RUDN University, 6~Miklukho-Maklaya Str., Moscow 117198, Russian Federation; 
\mbox{platonova-aa@rudn.ru}
  
  \vspace*{3pt}
  
  \noindent
  \textbf{Shorgin Vsevolod S.} (b.\ 1978)~--- Candidate of Science (PhD) in technology, senior 
scientist, Institute of Informatics Problems, Federal Research Center ``Computer Science and 
Control'' of the Russian Academy of Sciences, 44-2~Vavilov Str., Moscow 119333, Russian 
Federation; \mbox{vshorgin@ipiran.ru}
  
  \vspace*{3pt}
  
  \noindent
  \textbf{Gaidamaka Yuliya V.} (b.\ 1971)~--- Doctor of Science in physics and mathematics, 
professor, Department of Applied Probability and Informatics, RUDN University,  
6~Miklukho-Maklaya Str., Moscow 117198, Russian Federation; senior scientist, Institute of 
Informatics Problems, Federal Research Center ``Computer Science and Control'' of the Russian 
Academy of Sciences, 44-2~Vavilov Str., Moscow 119333, Russian Federation;  
\mbox{gaydamaka-yuv@rudn.ru}
  
\label{end\stat}

\renewcommand{\bibname}{\protect\rm Литература} 