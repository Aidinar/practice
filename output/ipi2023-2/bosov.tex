\def\stat{bosov}

\def\tit{ИССЛЕДОВАНИЕ РОБАСТНОСТИ ЧИСЛЕННЫХ АППРОКСИМАЦИЙ ФИЛЬТРА 
ВОНЭМА$^*$}

\def\titkol{Исследование робастности численных аппроксимаций фильтра 
Вонэма}

\def\aut{А.\,В.~Босов$^1$}

\def\autkol{А.\,В.~Босов}

\titel{\tit}{\aut}{\autkol}{\titkol}

\index{Босов А.\,В.}
\index{Bosov A.\,V.}


{\renewcommand{\thefootnote}{\fnsymbol{footnote}} \footnotetext[1]
{Исследование выполнено за счет гранта Российского научного фонда (проект 22-28-00588) с~использованием инфраструктуры Центра коллективного пользования 
<<Высокопроизводительные вычисления и~большие данные>> (ЦКП <<Информатика>>) ФИЦ ИУ 
РАН (г.\ Москва).}}


\renewcommand{\thefootnote}{\arabic{footnote}}
\footnotetext[1]{Федеральный исследовательский центр <<Информатика и~управление>> Российской 
академии наук, \mbox{ABosov@frccsc.ru}}

%\vspace*{-10pt}




\Abst{Исследуются свойства решения задачи оптимальной фильт\-ра\-ции со\-сто\-яния непрерывной 
цепи Маркова по линейным наблюдениям, за\-шум\-лен\-ным винеровским процессом, 
в~предположении неполной информации о~его ин\-тен\-сив\-ности. Не\-опре\-де\-лен\-ность сис\-те\-мы 
наблюдения задается верх\-ней границей ин\-тен\-сив\-ности шума. Чис\-лен\-ная реализация оптимального 
решения в~по\-ста\-нов\-ке с~полной информацией, обес\-пе\-чи\-ва\-емо\-го фильт\-ром Вонэма, не гарантирует 
устой\-чи\-вости. Показано, что фильтр Вонэма в~постановке с~не\-опре\-де\-лен\-ностью является 
робастным по отношению к~ин\-тен\-сив\-ности шума, если па\-ра\-мет\-ры модели не приводят к~его 
рас\-хо\-ди\-мости. В~общем случае не\-устой\-чи\-вость чис\-лен\-ной схе\-мы Эй\-ле\-ра--Ма\-ру\-ямы фильт\-ра 
Вонэма со\-хра\-ня\-ет\-ся. Прос\-тые эвристические приемы, обес\-пе\-чи\-ва\-ющие устойчивые 
аппроксимации фильт\-ра Вонэма, показывают ра\-бо\-то\-спо\-соб\-ность для более широкого набора 
па\-ра\-мет\-ров. Однако в~по\-ста\-нов\-ке с~не\-опре\-де\-лен\-ностью удается привести примеры, когда такие 
эвристические фильт\-ры показывают неприемлемо низ\-кое качество. Лучшее решение дают 
дискретизованные фильт\-ры~--- аппроксимации фильт\-ра Вонэма, реализованные для дискретной 
модели, ап\-прок\-си\-ми\-ру\-ющей исходную непрерывную сис\-те\-му наблюдения. На серии чис\-лен\-ных 
экспериментов показано, что данные фильт\-ры обладают ро\-баст\-ностью для всех наборов 
па\-ра\-мет\-ров. Если в~расчетах среди смоделированных траекторий фильт\-ра Вонэма нет 
расходящихся, то дискретизованные фильт\-ры немного проигрывают. Если рас\-хо\-дя\-щи\-еся 
траектории есть, то выигрыш дискретизованных фильт\-ров носит абсолютный характер.}

\KW{марковский скачкообразный процесс; стохастическая фильт\-ра\-ция; робастное оценивание; 
фильтр Вонэма; чис\-лен\-ная схема Эй\-ле\-ра--Ма\-ру\-ямы; дискретизованные фильтры}

\DOI{10.14357/19922264230206}{BGILKR}
  
\vspace*{4pt}


\vskip 10pt plus 9pt minus 6pt

\thispagestyle{headings}

\begin{multicols}{2}

\label{st\stat}

\section{Введение}

\vspace*{-2pt}

     Задача стохастической фильтрации~--- оценивания со\-сто\-яния 
динамической сис\-те\-мы по косвенным наблюдениям~--- одна из самых 
востребованных задач стохастического анализа. Наиболее яркие решения, 
такие как фильтр Кал\-ма\-на--Бьюси~[1], фильтр Вонэма~[2],  
услов\-но-гаус\-сов\-ский фильтр~[3], были и~остаются источниками 
актуальных по\-ста\-но\-вок для фундаментальных и~при\-клад\-ных исследований. 
Одно из наиболее известных на\-прав\-ле\-ний развития классических методов 
оценивания дают по\-ста\-нов\-ки с~не\-опре\-де\-лен\-ностью, т.\,е.\ неполной 
априорной информацией о~па\-ра\-мет\-рах сис\-те\-мы наблюдения. В~этих 
постановках, как правило, исследуются методы, обес\-пе\-чи\-вающие 
устой\-чи\-вость оценок по отношению к~не\-опре\-де\-лен\-ности, т.\,е.\linebreak об\-ла\-да\-ющие 
свойством ро\-баст\-ности. Среди робастных оценок значительное мес\-то 
занимают те, которые обладают минимаксным свойством. Эти оценки дают 
наилучший результат в~предположении, что реализуется наихудший 
сценарий не\-опре\-де\-лен\-ности, что и~поз\-во\-ля\-ет считать их ро\-баст\-ными.
     
     В фокусе данной работы неопределенность, связанная с~отсутствием 
точной априорной информации об ин\-тен\-сив\-ности шума в~наблюдениях. 
Самые известные и~содержательные результаты для такого типа 
не\-опре\-де\-лен\-ности дает фильтр Калмана~[4--12]. Вне за\-ви\-си\-мости от того, 
как задана не\-опре\-де\-лен\-ность и~какова постановка (фильт\-ра\-ция, управ\-ле\-ние, 
оценивание по интегральному критерию), решение ми\-ни\-макс\-ной задачи 
обеспечивается исключительными свойствами фильтра Калмана~--- 
ли\-ней\-ностью оценки и~уравнением Риккати для ковариации ее ошиб\-ки. 

Предполагалось исследовать аналогичные свойства в~по\-ста\-нов\-ке, 
от\-ли\-ча\-ющей\-ся от фильт\-ра\-ции Кал\-ма\-на--Бью\-си моделью со\-сто\-яния тем, 
что вмес\-то линейного гауссовского процесса состояние задается дискретной 
цепью Маркова, т.\,е.\ решается задача фильт\-ра\-ции Вонэма~[2]. Не\-смот\-ря на 
большое сходство фильт\-ров Кал\-ма\-на--Бью\-си и~Вонэма, между ними есть 
принципиальная разница. По\-след\-ний~--- существенно нелинейный, 
и~записать его точ\-ность не удается. По этой причине непосредственное 
решение минимаксной задачи не пред\-став\-ля\-ет\-ся воз\-мож\-ным. Но 
с~практической точ\-ки зрения это обстоятельство не играет особой роли, 
поскольку в~связи с~применением фильт\-ра Вонэма становится гораздо 
актуальнее задача обеспечения устой\-чи\-вости его чис\-лен\-ной реализации. 
Именно этот аспект ставится предметом исследования в~статье 
и~детализирован в~сле\-ду\-ющем разделе с~по\-ста\-нов\-кой задачи. По\-сколь\-ку 
возможности тео\-ре\-ти\-че\-ско\-го аппарата здесь ограничены, то акцент сделан на 
практическое исследование.
     
\section{Постановка задачи робастной фильтрации}

     На каноническом вероятностном пространстве $(\Omega, \mathcal{F}, 
\mathcal{P}, \mathcal{F}_t)$, $t\hm\in [0,T]$, рас\-смот\-рим сто\-ха\-сти\-че\-скую 
сис\-те\-му наблюдения с~вектором со\-сто\-яния~$y_t$ и~линейными 
наблюдениям~$z_t$ сле\-ду\-юще\-го вида:
     \begin{alignat}{2}
     dy_t&= \Lambda_t^\prime y_t \,dt +d\Lambda_t^y, &\enskip y_0&=Y\,;
     \label{e1-bos}\\
     dz_t &= a_t y_t\, dt +b_t z_t\, dt +\sigma_t\, dw_t, &\enskip z_0&=Z\,.
     \label{e2-bos}
     \end{alignat}
     
     Уравнение~(\ref{e1-bos}) определяет марковский скачкообразный процесс~--- 
цепь с~конечным чис\-лом со\-сто\-яний и~значениями в~множестве $\{ e_1, \ldots , 
e_{n_y}\}$, со\-сто\-ящем из единичных координатных векторов в~евклидовом 
пространстве~$\mathbb{R}^{n_y}$. Предполагается, что начальное 
со\-сто\-яние~$Y$ имеет известное распределение~$\pi$, $\Lambda_t$~--- 
мат\-ри\-ца интенсивностей переходов, $\Lambda_t^\prime$~--- 
транспонированная мат\-ри\-ца~$\Lambda_t$, а $\Lambda_t^y$~---  
$\mathcal{F}_t$-со\-гла\-со\-ван\-ный мартингал~\cite{13-bos}. Уравнение~(\ref{e2-bos}) 
пред\-став\-ля\-ет косвенные наблюдения $z_t\hm\in \mathbb{R}^{n_z}$ за 
со\-сто\-яни\-ем цепи~$y_t$, по\-рож\-да\-емую ими $\sigma$-ал\-геб\-ру будем 
обозначать~$\mathcal{F}_t^z$ и~предполагать, что 
$\mathcal{F}_t^z\hm\subseteq \mathcal{F}_t\hm\subseteq \mathcal{F}$. Далее, 
$w_t\hm\in \mathbb{R}^{n_w}$~--- не зависящий от $\Lambda_t^y$, $Y$ и~$Z$ 
стандартный векторный винеровский процесс; $Z\hm\in 
     \mathbb{R}^{n_z}$~--- гауссовский случайный вектор с~известными 
моментами $\mathbb{E}\{Z\}$ и~$\mathbb{E}\{Z Z^\prime\}$; мат\-рич\-ные 
функции $a_t\hm\in \mathbb{R}^{n_z\times n_y}$, $b_t\hm\in 
\mathbb{R}^{n_z\times n_z}$ и~$\sigma_t\hm\in \mathbb{R}^{n_z\times n_w}$ 
предполагаются ограниченными:$\vert a_t\vert \hm+ \vert b_t\vert\hm+ \vert 
\sigma_t\vert \hm\leq C$ для всех $0\hm\leq t\hm\leq T$, что обеспечивает 
существование решения уравнения~(\ref{e2-bos}). Кроме того, ошибки наблюдений 
предполагаются не\-вы\-рож\-ден\-ны\-ми, $\sigma_t \sigma_t^\prime \hm>0$. Через 
$\mathbb{E}\{\cdot \}$ и~$\mathbb{E}\{\cdot\vert\cdot\}$ обозначены 
операторы безуслов\-но\-го и~услов\-но\-го математического ожидания 
соответственно.
     
     Основное условие рассматриваемой задачи~--- предположение об 
отсутствии точ\-ной информации об ин\-тен\-сив\-ности ошибок наблюдений, 
вы\-ра\-жа\-емое неравенством $\sigma_t \sigma_t^\prime \hm\leq \Sigma_t$, 
в~котором известна положительно определенная мат\-ри\-ца~$\Sigma_t$, име\-ющая 
\mbox{смысл} верх\-ней границы ковариации шума. Задачей ставится анализ свойств 
алгоритмов оценивания состояния цепи~$y_t$ (фильт\-ров) 
по~$\mathcal{F}_t^z$. Точ\-ное решение задачи минимаксной фильт\-рации
     $$
     \hat{y}_t^{\min\max} =\argmin\limits_{\hat{y}_t} \max\limits_{\sigma_t: 
\sigma_t \sigma_t^\prime \leq \Sigma_t} \mathbb{E} \left\{ \left\vert \hat{y}_t -
y_t\right\vert^2\right\}
     $$
     
\vspace*{-2pt}

     \noindent
получить затруднительно, но для дальнейших целей можно исходить из 
интуитивного предположения, что 
$$
\hat{y}_t^{\min\max} \approx \hat{y}_t^W(\Sigma_t),
$$

\noindent
 где $\hat{y}_t^W(\sigma_t \sigma_t^\prime) \hm= 
\mathbb{E}\{ y_t\vert \mathcal{F}_t^z, \sigma_t\}$, т.\,е.\ задается выражением 
оптимального фильт\-ра в~задаче без не\-опре\-де\-лен\-ности (при известной 
ковариации $\sigma_t\sigma_t^\prime$), в~котором эта неизвестная 
ковариация заменена \mbox{максимально} воз\-мож\-ной~$\Sigma_t$. На самом деле, 
даже если бы удалось доказать такое утверж\-де\-ние, это никак не при\-бли\-зи\-ло 
бы к~практическому решению задачи фильт\-ра\-ции, поскольку переход от 
записи оптимального фильт\-ра Вонэма к~его типовой чис\-лен\-ной реализации 
по схеме Эй\-ле\-ра--Ма\-ру\-ямы~\cite{14-bos}\linebreak приводит к~неустойчивой 
процедуре~\cite{15-bos}, и,~соответственно, гораздо важ\-нее предложить пусть 
неоптимальные в~минимаксном смыс\-ле, но работоспособные устойчивые 
алгоритмы. Такими \mbox{алгоритмами} служат дискретизованные фильтры, 
предложенные в~\cite{17-bos, 18-bos} и~подробно исследованные  
в~\cite{19-bos}. В~на\-сто\-ящей статье результаты дополнены анализом свойств 
этих фильт\-ров в~по\-ста\-нов\-ке с~не\-опре\-де\-лен\-ностью ин\-тен\-сив\-ности шума 
в~наблюдениях.

\vspace*{-6pt}

\section{Некоторые сведения об~оптимальных фильтрах}

     Для понимания дальнейшего обратим внимание на сле\-ду\-ющие 
известные для рас\-смат\-ри\-ва\-емой задачи факты. Во-пер\-вых, при условии 
$\sigma_t\sigma_t^\prime \hm= \Sigma_t$, т.\,е.\ при отсутствии 
не\-опре\-де\-лен\-ности, известное оптимальное решение $\hat{y}_t^W$ 
определяется фильт\-ром Вонэма~\cite{2-bos, 13-bos}:
     \begin{multline}
     d\hat{y}_t^W = \Lambda_t^\prime \hat{y}_t^W \,dt+ \left( 
\mathrm{diag}\left( \hat{y}_t^W\right) -{}\right.\\
\left.{}-\hat{y}_t^W \left( 
\hat{y}_t^W\right)^\prime \right) a_t^\prime \Sigma_t^{-1} \left( dz_t -b_t z_t \,dt -
a_t \hat{y}_t^W dt\right).
     \label{e3-bos}
     \end{multline}
     
     Во-вторых, это общее уравнение оптимальной фильт\-ра\-ции в~терминах 
об\-нов\-ля\-ющих процессов~\cite{3-bos}, записанное с~учетом модели сис\-те\-мы 
наблюдения~(\ref{e1-bos}), (\ref{e2-bos}):
     \begin{multline}
     d\hat{y}_t^{\mathrm{Opt}} =\Lambda_t^\prime \hat{y}_t^{\mathrm{Opt}} dt+\mathbb{E} 
\left\{ \left( y_t -\hat{y}_t^{\mathrm{Opt}}\right) y_t^\prime \vert \mathcal{F}_t^z, 
\sigma_t\right\} \times{}\\
{}\times a_t^\prime \Sigma_t^{-1} \left( dz_t - b_t z_t\, dt -a_t 
\hat{y}_t^{\mathrm{Opt}} \,dt\right).
     \label{e4-bos}
     \end{multline}
     
     И, наконец, уравнение фильтра  
Кал\-ма\-на--Бью\-си~\cite{19-bos, 20-bos} для модели~(\ref{e1-bos}), (\ref{e2-bos}), записанное 
в~предположении, что вмес\-то цепи~$y_t$ наблюдается гауссовский процесс, 
т.\,е.\ мартингал~$\Lambda_t^y$ предполагается винеровским процессом: 
     \begin{multline}
     d\hat{y}_t^K ={}\\
     \!{}=\Lambda_t^\prime \hat{y}_t^K dt+ P_t^K a_t^\prime 
\Sigma_t^{-1} \left( dz_t -b_t z_t \,dt -a_t \hat{y}_t^K dt\right),\!
     \label{e5-bos}
     \end{multline}
где матрица усиления $P_t^{K}$ не зависит от~$\mathcal{F}_t^z$ 
и~определяется обыкновенным дифференциальным уравнением Риккати.
     
     Сравнивая (\ref{e3-bos}) и~(\ref{e4-bos}) и~учитывая не\-сме\-щен\-ность оценок, не\-труд\-но 
увидеть, что 
\begin{multline*}
\mathbb{E}\left\{ \left( y_t- \hat{y}_t^W\right) \left( y_t - 
\hat{y}_t^W\right)^\prime \vert \mathcal{F}_t^z, \sigma_t\right\} = {}\\
{}=
\mathrm{diag}\left( \hat{y}_t^W\right)- \hat{y}_t^W\left( 
\hat{y}_t^W\right)^\prime
\end{multline*} 
для традиционной по\-ста\-нов\-ки фильт\-ра\-ции 
Вонэма. Аналогично, сравнивая~(\ref{e4-bos}) и~(\ref{e5-bos}), нетрудно увидеть, что 
$$
\mathbb{E}\left\{ \left( y_t- \hat{y}_t^K\right) \left( y_t- 
\hat{y}_t^K\right)^\prime \vert \mathcal{F}_t^z, \sigma_t\right\} = P_t^K
$$ 
в~традиционной по\-ста\-нов\-ке фильт\-ра\-ции Калмана. Поскольку здесь~$P_t^K$ не 
зависит от наблюдений, то и~общее качество калмановской фильт\-ра\-ции 
$$
\mathbb{E}\left\{ \left( y_t- \hat{y}_t^K\right) \left( y_t- 
\hat{y}_t^K\right)^\prime\right\} = P_t^K\,.
$$
 Это обстоятельство вместе 
с~ли\-ней\-ностью оценки~$\hat{y}_t^K$ и~становится тем движущим 
свойством, которое обеспечивает фильт\-ру Калмана еще и~ми\-ни\-макс\-ность  
в~ли\-ней\-но-га\-ус\-сов\-ской модели с~не\-опре\-де\-лен\-ностью $\sigma_t 
\sigma_t^\prime \hm\leq \Sigma_t$, а~именно:
     \begin{equation}
\hat{y}_t^K(\Sigma_t) =\argmin\limits_{\hat{y}_t}\! \max\limits_{\sigma_t: 
\sigma_t\sigma_t^\prime \leq \Sigma_t} \mathbb{E} \left\{ \left\vert \hat{y}_t^K 
(\sigma_t) -y_t \right\vert^2 \right\}.\!\!
     \label{e6-bos}
     \end{equation}
     
     Вычислить безусловную ковариацию 
     $\mathbb{E}\left\{ \left( y_t\hm- \hat{y}_t^W\right) \left( y_t\hm- \hat{y}_t^W\right)^\prime\right\}$ 
     в~модели фильт\-ра\-ции Вонэма, к~сожалению, не удается. Кроме того, сама оценка~(\ref{e3-bos}) 
существенно нелинейна, поэтому получить аналогичное~(\ref{e6-bos}) свойство 
минимаксности для фильт\-ра Вонэма пока не удалось. В~качестве 
фундаментального результата это был бы большой успех. В~практических 
целях особого значения это не имеет из-за не\-устой\-чи\-вости чис\-лен\-ных схем 
фильт\-ра Вонэма, и~минимаксная по\-ста\-нов\-ка ничего здесь не изменит.
     
\section{Дискретизованные фильтры для~модели Вонэма} 

      Устойчивые аппроксимации фильтра Вонэма~\cite{17-bos, 18-bos} 
реализованы для модели, которая получается из~(\ref{e1-bos}), (\ref{e2-bos}) дискретизацией на 
некотором интервале $t\hm\in [0,T]$ с~заданным временн$\acute{\mbox{ы}}$м шагом~$\delta$:
     \begin{multline*}
      0=t_0;\ \ t_1= t_0+\delta;\ \ \ldots ;\  \ t_i= t_{i-1}+\delta;\ \ \ldots\\
      \ldots  ;\ \  t_{T/\delta-1} 
+\delta= t_{T/\delta}=T\,.
    \end{multline*}
      
      Кроме того, предполагается, что $a_t\hm\equiv const$, $b_t\hm\equiv const$ 
и~$\sigma_t\hm\equiv const$ на интервалах дискретизации $[t_{i-1}, t_i]$. 
Это предположение лег\-ко реализуется заменой функций~$a_t$, $b_t$ 
и~$\sigma_t$ на их ку\-соч\-но-по\-сто\-ян\-ные аппроксимации. Чтобы 
привести наблюдения~(\ref{e2-bos}) к~каноническому виду~\cite{16-bos}, мож\-но 
рас\-смот\-реть процесс~$z_t^0$, пред\-став\-ля\-ющий собой не\-упреж\-да\-ющее 
преобразование исходного наблюдаемого выхода~$z_t$ (т.\,е.\ выполнено 
равенство $\hat{y}_t\hm= \mathbb{E}\left\{ y_t\vert \mathcal{F}_t^z\right\} \hm= 
\mathbb{E}\left\{ {y_t\vert \mathcal{F}_t^z}^0\right\})$:
      \begin{equation}
      z_t^0 = \!\int\limits_0^t \!\left( dz_\tau -b_\tau z_\tau \,d\tau \right) = \!
      \int\limits_0^t \!\left( a_\tau y_\tau \,d\tau +\sigma_\tau \,dw_\tau\right).\!
      \label{e7-bos}
      \end{equation}
      
      Далее введем новые наблюдения, дискретизованные по времени 
с~шагом~$\delta$: 
$$
\Delta z^0_{t_i}= \int\limits^{t_i}_{t_{i-1}} (a_\tau y_\tau \,d\tau + \sigma_\tau \,dw_\tau).
$$
 Это приращения~$z_t^0$ на 
интервалах дискретизации, и~они по\-рож\-да\-ют $\sigma$-ал\-геб\-ру 
$$
\mathcal{F}_{t_i}^{\Delta z^0} = \sigma \left\{ \Delta z^0_{t_j}, 1\leq j\leq i\right\}.
$$
 Если обозначить через 
 $$
 \mu_i=  \int\limits^{t_i}_{t_{i-1}} y_\tau \,d\tau \hm= \left( \mu_i^1, \ldots , \mu_i^{n_y}\right)^\prime
 $$ случайный вектор, компоненты которого рав\-ны 
времени пребывания марковской цепи~$y_t$ в~каждом из воз\-мож\-ных 
со\-сто\-яний на интервале времени $(t_{i-1}, t_i]$, а~через $\mathcal{N}(z; 
m,\sigma^2)$ гауссовскую плот\-ность ве\-ро\-ят\-ности со сред\-ним~$m$ 
и~дисперсией~$\sigma^2$, вы\-чис\-лен\-ную в~точ\-ке~$z$, то оценка  
$\hat{y}_{t_i} \hm= \mathbb{E}\left\{ y_t\vert \mathcal{F}_{t_i}^{\Delta 
z^0}\right\}$ находится с~по\-мощью сле\-ду\-ющей рекуррентной 
процедуры~\cite{17-bos}:
      \begin{equation}
      \left.
      \begin{array}{rl}
     \!\! \hat{y}^{\mathrm{opt}}_{t_i}& =\left( \mathbf{1} \hat{q}_{t_i}^\prime 
\hat{y}^{\mathrm{opt}}_{t_{i-1}} \right)^{-1} \left( \hat{q}^\prime_{t_i} 
\hat{y}^{\mathrm{opt}}_{t_{i-1}} \right);\\[6pt]
     \!\! \hat{q}^{k,j}_{t_i}&=\mathbb{E}\left\{ \mathcal{N}\left( \Delta z^0_{t_i}; 
a \mu_i, \delta \sigma\sigma^\prime\right) y^j_{t_i}\vert y_{t_{i-1}} =e_k\right\},
\end{array}\!
\right\}\!
      \label{e8-bos}
      \end{equation}
где $\mathbf{1} \hm= (1, \ldots , 1)^\prime \hm\in \mathbb{R}^{n_y}$~--- 
вектор из единиц; начальное условие $\hat{y}_0^{\mathrm{opt}} \hm= \pi_0$; мат\-ри\-ца 
$\hat{q}_{t_i}\hm= \left\| \hat{q}_{t_i}^{k,j}\right\|^{n_y}_{k,j=1}$. Величины 
$\hat{q}^{k,j}_{t_i}$ можно аппроксимировать, используя для за\-да\-ющих их 
интегралов одну из аппроксимаций~\cite{18-bos}:
\begin{itemize}
     \item схему <<левых>> прямоугольников (порядок точ\-ности $1/2$);
     \item схему <<средних>> прямоугольников (порядок точ\-ности~1);
     \item схему, основанную на квад\-ра\-ту\-рах Гаусса (порядок точ\-ности~2).
     \end{itemize}
     
     Полностью соотношения приведены также в~\cite{19-bos}, где 
показано, что в~большинстве экспериментов разница меж\-ду оценками 
дис\-кре\-ти\-зо\-ванных фильт\-ров, связанная с~выбором схемы интегрирования, 
незначительна и~может не учитываться в~рас\-смат\-ри\-ва\-емом модельном 
примере, поэтому в~расчетах все три оценки пред\-став\-ле\-ны пер\-вой,~$\breve{y}_{t_i}^{\delta^{1/2}}$.
{\looseness=-1

}
     
     Отдельного пояснения требует преобразование~(\ref{e7-bos}). Его смысл со\-сто\-ит 
в~том, что в~модели~(\ref{e2-bos}) без ограничения общ\-ности можно считать 
$b_t\hm=0$, причем как для дискретизованных фильт\-ров, так и~для фильт\-ра 
Вонэма~(\ref{e3-bos}). Однако для чис\-лен\-ной реализации это оказывается не так. Как 
показали пред\-став\-лен\-ные далее рас\-че\-ты, наблюдения, фор\-ми\-ру\-емые 
неустойчивым процессом~$z_t$, который мож\-но получить именно 
с~по\-мощью мат\-ри\-цы~$b_t$, оказывают ре\-ша\-ющее влияние на устой\-чи\-вость 
аппроксимации фильт\-ра Вонэма.
     
\section{Экспериментальное исследование}

\subsection{Модель}

     Основу для численных экспериментов дала модель механического 
привода, использованная в~\cite{19-bos}. Отличие со\-сто\-ит в~том, что здесь 
привод не управ\-ля\-ет\-ся, поскольку решается только задача фильт\-ра\-ции, 
поэтому модель сис\-те\-мы наблюдения имеет вид
     \begin{equation}
     \left.
     \begin{array}{rl}
     dx_t&=v_t\,dt,\quad t\in (0,T]\,;\\[3pt]
      dv_t &=ax_t \,dt +bv_t \,dt+ cy_y \,dt+ \sqrt{g}\,dw_t\,,
      \end{array}
      \right\}
     \label{e9-bos}
     \end{equation}
где $x_t$~--- положение привода на горизонтальной оси; $v_t$~---  
ско\-рость. Физический смысл модели со\-сто\-ит в~том, чтобы стабилизировать 
привод в~положениях, за\-да\-ва\-емых со\-сто\-яни\-ями цепи, т.\,е.\ в~точках $-c_i/a$ 
для $y_t\hm= e_i$. Если~$y_t$ неизвестно, то его надо оценивать, 
используя~$v_t$ в~качестве наблюдений.
      
      Марковская цепь~$y_t$ в~разных расчетах будет иметь три или четыре 
со\-сто\-яния. Для $n_y\hm=3$ по\-сто\-ян\-ная мат\-ри\-ца интенсивностей 
$\Lambda_t\hm=\Lambda$ отвечает модели прос\-то\-го процесса  
рож\-де\-ния-ги\-бе\-ли: 
$$
\Lambda = 
      \begin{pmatrix}
       -0{,}5 & 0{,}5 & 0\\
      0{,}5 & -1 & 0{,}5\\
      0& 0{,}5 & -0{,}5
      \end{pmatrix} 
      \mbox{\ \ или\ \ } 
      \Lambda= 
      \begin{pmatrix}
      -5 & 5 &0\\
      5 &-1& 5\\
      0&5&-5
      \end{pmatrix}.
      $$
       Для $n_y\hm=4$ также 
       
       \noindent
       $$
       \Lambda_t= \Lambda = 
\begin{pmatrix}
 -0{,}5 &0{,}5& 0&0\\
      0{,}5& -1 & 0{,}5 &0\\
      0& 0{,}5& -1& 0{,}5\\
      0& 0& 0{,}5& -0{,}5
      \end{pmatrix}.
      $$
      
      \vspace*{-2pt}
      
      \noindent
       Начальное распределение~$\pi$ во всех расчетах 
задает $y_0\hm= Y\hm= e_1$
      
      Скаляры $a$, $b$ и~$g$~--- известные по\-сто\-ян\-ные и~меняются в~раз\-ных 
расчетах; строки~$c$ известны и~рав\-ны соответственно $(c_1, c_2, c_3)\hm= 
(-1, 0, 1)$ и~$(c_1, c_2, c_3, c_4)\hm= (-1{,}5; -0{,}5; 0{,}5; 1{,}5)$; $w_t$~--- 
стандартный винеровский процесс.
      
      Наблюдения~(\ref{e9-bos}), как видно, сами пред\-став\-ля\-ют собой сис\-те\-му. Эта 
линейная сис\-те\-ма устойчива, если $b\hm<0$ и~$b^2\hm+ 4a \hm<0$, 
поскольку~$b$ и~$b^2\hm+ 4a$~--- собственные чис\-ла мат\-ри\-цы сис\-те\-мы 
$b_t= 
\begin{pmatrix}
      0&1\\
      a&b
      \end{pmatrix},
      $
      и~неустойчива иначе.
      
      Интегрирование во всех расчетах как для сис\-те\-мы~(\ref{e9-bos}), так и~для 
фильт\-ров~(\ref{e3-bos}) и~(\ref{e8-bos}) выполнялось методом Эйлера с~шагом $\delta\hm= 
0{,}001$. Дискретная цепь, ап\-прок\-си\-ми\-ру\-ющая~$y_t$, моделировалась 
независимыми экспоненциальными величинами для каж\-до\-го из 100 
интервалов интегрирования длины~$\delta$, т.\,е.\ использовалась выборка из 
распределения $E(0{,}00001)$. Для моделирования сис\-те\-мы наблюдения, 
т.\,е.\ переменных $x_t$ и~$v_t$, шаг интегрирования~$\delta$ так\-же 
разбивался на 100~интервалов длины $\delta/100$.
      
      В~\cite{19-bos} есть подробные пояснения и~примеры рас\-хо\-ди\-мости 
аппроксимаций фильт\-ра Вонэма, полученных в~этих расчетах с~по\-мощью 
схемы Эй\-ле\-ра--Ма\-ру\-ямы. В~опи\-сы\-ва\-емых здесь расчетах также 
в~качестве признака рас\-хо\-ди\-мости оценки~$\hat{y}_t^W$ использовалось 
условие $\left\vert \left( \hat{y}_t^W\right)_k\right\vert \hm>1$ хотя бы для 
одного~$k$, и~при выполнении этого условия оценка 
фильт\-ра\-ции~$\hat{y}_t^W$ возвращалась в~предельное со\-сто\-яние: 

\vspace*{-3pt}

\noindent
      $$
      \hat{y}_\tau^W = \pi_\infty =
      \begin{cases}
       \left( \fr{1}{3}, \fr{1}{3}, \fr{1}{3}\right)^\prime & \mbox{при } n_y=3\,;\\
        \left( \fr{1}{4}, \fr{1}{4}, \fr{1}{4},  \fr{1}{4}\right)^\prime & \mbox{при } n_y=4
        \end{cases}
        $$
        
        \vspace*{-2pt}
        
        \noindent
         для момента времени~$\tau$, в~который 
выполнилось это условие. Так получается~$\hat{y}_t^{\mathrm{lim}}$. Второй вариант 
предотвратить рас\-хо\-ди\-мость~--- это замена текущей оценки на оценку 
предыду\-ще\-го шага, т.\,е.\ $\hat{y}_\tau^W \hm= \hat{y}^W_{\tau-\delta}$. Так 
получается $\hat{y}_t^{\mathrm{del}}$. Имен\-но эти оцен\-ки участвуют далее 
в~расчетах и~анализируются на пред\-мет ро\-баст\-ности в~отношении 
сформулированной задачи.
      
      В каждом расчете моделировалось по 1000 траекторий сис\-те\-мы~(\ref{e9-bos}) 
и~оценок фильт\-ров. Для характеризации качества оценок фильт\-ра\-ции 
$\hat{y}_t^{\mathrm{lim}}$, $\hat{y}_t^{\mathrm{del}}$ 
и~$\breve{y}_{t_i}^{\delta^{1/2}}$ рассчитывались интегральные 
квадратичные ошибки $\hat{D}\!\left( \hat{y}_t^{\lim}\right)$, $\hat{D}\!\left( 
\hat{y}_t^{\mathrm{del}}\right)$ и~$\hat{D}\!\left(\breve{y}_{t_i}^{\delta^{1/2}} 
\right)$ для 
$$
\hat{D}\left( \tilde{y}_{t_i}\right)= \hat{\mathbb{E}} \left\{ 
\fr{\delta}{T} \sum\limits_{i=1}^{T/\delta} \left( cy_{t_i} - 
c\tilde{y}_{t_i}\right)^2\right\},
$$

\vspace*{-2pt}

\noindent
 где $\hat{\mathbb{E}}$  обозначает 
усред\-не\-ние по пуч\-ку~1000~траекторий.
      
      Далее, в~каждом из расчетов выбирался <<базовый>> сценарий без 
не\-опре\-де\-лен\-ности с~известной ин\-тен\-сив\-ностью шума $\sqrt{g}\hm= 0{,}1$ 
Расчеты для анализа свойства ро\-баст\-ности выполнялись в~каж\-дом случае для 
этого значения~$g$, но в~предположении, что известна лишь 
граница~$\Sigma_t$, для которой выбирались два варианта: $\Sigma_t\hm= 
3g$ и~$10g$. Наконец, для оцен\-ки верх\-ней границы точ\-ности 
в~задаче с~не\-опре\-де\-лен\-ностью расчет повторялся в~предположении, что 
интенсивности из\-вест\-ны, т.\,е.\ для $g\hm= 0{,}03$ и~$0{,}1$.

\vspace*{-2pt} 

\subsection{Устойчивые наблюдения}

%\vspace*{-2pt}

      Устойчивый вариант системы~(\ref{e9-bos}) описывается значениями $a\hm=-1$ 
и~$b\hm= -0{,}5$ Для этого случая эксперименты выполнены для цепей 
раз\-мер\-ности $n_y\hm=3$ и~$4$ и~для всех трех матриц 
интенсивностей, всего три эксперимента, в~каж\-дом по пять расчетов. Для 
удобства па\-ра\-мет\-ры каж\-дой модели повторены в~табл.~1--3, 
пред\-став\-ля\-ющих ре\-зуль\-таты.





      
      Рассмотрим табл.~1. Первая строка, со\-от\-вет\-ст\-ву\-ющая базовому 
сценарию, повторяет основные выводы, сделанные в~\cite{19-bos}. Среди 
смоделированных траекторий не слишком час\-то, но срабатывали условия 
      $\vert (\hat{y}_t^W)_k \vert \hm>1$, т.\,е.\ фильтр Вонэма <<не 
справ\-лял\-ся>>. На них по-раз\-но\-му реагировали аппроксимации 
$\hat{y}_t^{\mathrm{lim}}$ и~$\hat{y}_t^{\mathrm{del}}$ (об этом свидетельствует их
разная точ\-ность). Дискретизованные фильт\-ры дали результат лучше. Но в~тех двух 
расчетах, что должны дать ответ на основной вопрос о~наличии робаст-\linebreak\vspace*{-12pt}

%\begin{table*}[b]\small %tabl1
\vspace*{9pt}
\begin{center}
\parbox{78mm}{{{\tablename~1}\ \ \small{Фильтрация по устойчивым наблюдениям. Низкая интенсивность 
переключений
}}
}


      \vspace*{6pt}
      
  {\small    
  \tabcolsep=7.5pt
      \begin{tabular}{|c|c|c|c|c|}
      \hline
&&&&\\[-9pt]
$g$& $\Sigma_t$& $\hat{D}\left(\hat{y}_t^{\mathrm{lim}}\right)$  &  $\hat{D}\left(\hat{y}_t^{\mathrm{del}}\right)$ 
&$\hat{D}\left(\breve{y}_{t_i}^{\delta^{1/2}}\right)$\\
&&&&\\[-9pt]
\hline
 & 0{,}01  &0,0540&0,0495&0,0486\\
0{,}01&0{,}3\hphantom{9}&\textbf{0,0612}&\textbf{0,0612}&\textbf{0,0729}\\
&0{,}1\hphantom{9}&\textbf{0,0911}&\textbf{0,0911}&\textbf{0,1225}\\
\hline
0{,}03 & 0{,}03&0,0987&0,0987&\textbf{0,0986}\\
\hline
0{,}1\hphantom{9} & 0{,}1\hphantom{9}&0,1837&0,1837&\textbf{0,1837}\\
\hline
\end{tabular}

\smallskip

\parbox{78mm}{\footnotesize{\hspace*{5mm}\textbf{Параметры:} $a=-1$; $b=-0{,}5$; $(c_1,c_2,c_3)=(-1, 0, 1)$; $\Lambda= 
\begin{pmatrix}
-0{,}5& 0{,}5& 0\\
0{,}5& -1& 0{,}5\\
0& 0{,}5& -0{,}5
\end{pmatrix}$.
}}
}
\end{center}
%\vspace*{-6pt}
%\end{table*}

%\begin{table*}\small %tabl2
\begin{center}
\parbox{78mm}{{\tablename~2}\ \ \small{Фильтрация по устойчивым наблюдениям. Высокая ин\-тен\-сив\-ность 
переключений
}}


      \vspace*{6pt}
      
     {\small  
     \tabcolsep=8pt
     \begin{tabular}{|c|c|c|c|c|}
      \hline
&&&&\\[-9pt]
$g$ &$\Sigma_t$&$\hat{D}\left(\hat{y}_t^{\mathrm{lim}}\right)$ & $\hat{D}\left(\hat{y}_t^{\mathrm{del}}\right)$ &$\hat{D}\left(\breve{y}_{t_i}^{\delta^{1/2}}\right)$\\
&&&&\\[-9pt]
\hline
 & 0{,}01&0,1959&0,1956&0,1948\\
0{,}01 & 0{,}03&\textbf{0,2164}&\textbf{0,2164}&\textbf{0,2286}\\
&  0{,}1\hphantom{9}&\textbf{0,3247}&\textbf{0,3247}&\textbf{0,3250}\\
\hline
0{,}03 &  0{,}03&0,3055&0,3055&\textbf{0,3064}\\
\hline
0{,}1\hphantom{9} &  0{,}1\hphantom{9}&0,4477&0,4477&\textbf{0,4485}\\
\hline
\end{tabular}
\smallskip

\parbox{78mm}{\footnotesize{\hspace*{5mm}\textbf{Параметры:} $a=-1$; $b=-0{,}5$; $(c_1, c_2, c_3)=(-1,0,1)$; $\Lambda = 
\begin{pmatrix}
-5& 5&0\\
5 &-1 &5\\
0&5&5
\end{pmatrix}$.
}}
}
%\end{center}

%\begin{table*}\small %tabl3
%\begin{center}

\vspace*{12pt}

\parbox{78mm}{{\tablename~3}\ \ \small{Фильтрация по устойчивым наблюдениям. Цепь размерности~4
}}

      \vspace*{6pt}
      
      \tabcolsep=8pt
     {\small  \begin{tabular}{|c|c|c|c|c|}
      \hline
      &&&&\\[-9pt]
$g$& $\Sigma_t$ & $\hat{D}\left(\hat{y}_t^{\mathrm{lim}}\right)$ & $\hat{D}\left(\hat{y}_t^{\mathrm{del}}\right)$&$\hat{D}\left(\breve{y}_{t_i}^{\delta^{1/2}}\right)$\\
&&&&\\[-9pt]
\hline
 &  0{,}01 &0,3615&0,3562&0,0534\\
0,01 & 0{,}03&\textbf{0,3583}&\textbf{0,3583}&\textbf{0,0811}\\
& 0{,}1\hphantom{9}&\textbf{0,3751}&\textbf{0,3751}&\textbf{0,1384}\\
\hline
0{,}03 & 0{,}03&0,3957&0,3957&\textbf{0,1086}\\
\hline
0{,}1\hphantom{9} & 0{,}1\hphantom{9}&0,4767&0,4767&\textbf{0,2058}\\
\hline
\end{tabular}

}

\smallskip

\parbox{78mm}{\footnotesize{\hspace*{5mm}\textbf{Параметры:} $a=-1$; $b=-0{,}5$; $(c_1, c_2, c_3, c_4)\hm=(-1{,}5; -0{,}5; 0{,}5; 1{,}5)$; 
$\Lambda= \begin{pmatrix}
-0{,}5 & 0{,}5& 0&0\\
0{,}5& -1 & 0{,}5& 0\\
0& 0{,}5& -1& 0{,}5\\
0& 0& 0{,}5 & -0{,}5
\end{pmatrix}$.
}}

\end{center}

\vspace*{3pt}



\noindent
 ных свойств (строки выделены полужирным), этого уже нет. В~остальных 
расчетах расходящихся траекторий не было (точ\-ность $\hat{y}_t^{\mathrm{lim}}$ 
и~$\hat{y}_t^{\mathrm{del}}$ одинаковая). Дискретизованные фильт\-ры дали результат 
хуже.







\begin{figure*}[b] %fig1
\vspace*{6pt}
\begin{center}
   \mbox{%
\epsfxsize=163mm
\epsfbox{bos-1.eps}
}
\end{center}
\vspace*{-11pt}
      \Caption{Характерные траектории для $g\hm= 0{,}01$ и~$\Sigma_t\hm= 0{,}03$: 
\textit{1}~--- цепь~$y_t$; \textit{2}~--- оценка $\hat{y}_t^{\mathrm{lim}}$; \textit{3}~--- оценка 
$\hat{y}_t^{\mathrm{del}}$; \textit{4}~--- оцен\-ка~$\protect\breve{y}_{t_i}^{\delta^{1/2}}$}
     % \end{figure*}
    % \end{figure*}
     %\begin{figure*} %fig2
     \vspace*{6pt}
\begin{center}
   \mbox{%
\epsfxsize=163mm
\epsfbox{bos-2.eps}
}
\end{center}
\vspace*{-11pt}
     \Caption{Характерные траектории для $g\hm= 0{,}01$ и~$\Sigma_t \hm= 0{,}1$: 
\textit{1}~--- цепь~$y_t$; \textit{2}~--- оценка $\hat{y}_t^{\mathrm{lim}}$; \textit{3}~--- оценка 
$\hat{y}_t^{\mathrm{del}}$; \textit{4}~--- оцен\-ка~$\protect\breve{y}_{t_i}^{\delta^{1/2}}$
}
\vspace*{-3pt}
      \end{figure*}

      
      Надо отметить, что в~этом эксперименте па\-ра\-мет\-ры оказались такими, 
что обеспечили идеальные условия для фильт\-ра Вонэма. Ситуация, когда 
расходящихся траекторий нет, как показано в~\cite{19-bos}, возникает не 
слишком час\-то. Вмес\-те с~тем ро\-баст\-ностью дискретизованные фильт\-ры 
обладают, хотя и~дают худший результат, чем оценки $\hat{y}_t^{\mathrm{lim}}$ 
и~$\hat{y}_t^{\mathrm{del}}$.


      

      Обратимся к~табл.~2. Этот расчет повторил те же выводы, что 
и~предыду\-щий: робастные свойства есть у~всех фильт\-ров, но аппроксимации 
фильт\-ра Вонэма выигрывают, хотя и~менее заметно.


      
      
      Перейдем к~табл.~3. Па\-ра\-мет\-ры этого эксперимента принципиально 
отличаются от предыду\-щих тем, что уже в~базовом сценарии аппроксимации 
фильт\-ра Вонэма проигрывают очень сильно. Модель с~неопределенностью 
<<добавляет>> устой\-чи\-вости оценкам $\hat{y}_t^{\mathrm{lim}}$ и~$\hat{y}_t^{\mathrm{del}}$ 
(их качество совпадает, расходящихся траекторий нет), но на превосходство 
дискретизованных фильт\-ров это не влияет.


      
\subsection{Неустойчивые наблюдения}

\vspace*{-2pt}

      Еще один эксперимент был выполнен для формально неустойчивой 
сис\-те\-мы~(\ref{e9-bos}) со значениями $a\hm=0$ и~$b\hm=0$. В~ре\-аль\-ности такая сис\-те\-ма 
ведет себя довольно инерт\-но и~обнаружить в~мо\-де\-ли\-ру\-емом пучке 
траектории сис\-те\-мы, которые успевают разойтись за время $T\hm=10$, не 
удается. Расчеты с~теми же параметрами, что и~в~табл.~1, привели к~тем же в~точ\-ности результатам, 
т.\,е.\ в~табл.~1 мож\-но считать $a\hm=0$ и~$b\hm=0$. 
Это отвечает замене~(\ref{e7-bos}) и~эквивалентной записи~(\ref{e2-bos}) с~$b_t\hm=0$. Но 
сле\-ду\-ющий эксперимент показывает, что на практике так сделать можно не 
всегда. Единственное отличие по\-след\-не\-го об\-суж\-да\-емо\-го эксперимента 
со\-сто\-ит в~том, что для неустойчивой сис\-те\-мы~(\ref{e9-bos}) использованы значения 
$a\hm=1$ и~$b\hm=0{,}5$. Результаты приведены в~табл.~4.
      

      
      Тенденция ухудшения качества аппроксимаций фильт\-ра Вонэма, 
наметившаяся в~эксперименте из табл.~3, здесь продолжилась вплоть до того, 
что оценки $\hat{y}_t^{\mathrm{lim}}$ и~$\hat{y}_t^{\mathrm{del}}$ перестали иметь смысл, 
потому что уста\-но\-вив\-ше\-еся значение $\mathbb{E}\{ \vert y_t\vert^2\} \hm= 
2/3$,
 а~значит, тривиальная оценка $\mathbb{E}\{y_t\}\hm=0$ становится 
лучше этих оценок. При этом с~дискретизованными\linebreak\vspace*{-12pt}

    %tabl4
\begin{center}
\parbox{78mm}{{\tablename~4}\ \ \small{Фильтрация по неустойчивым наблюде\-ниям
}}

      \vspace*{6pt}
      
      {\small 
      \tabcolsep=8pt
      \begin{tabular}{|c|c|c|c|c|}
      \hline
&&&&\\[-9pt]
$g$&$\Sigma_t$ & $\hat{D}\left(\hat{y}_t^{\mathrm{lim}}\right)$ &$\hat{D}\left(\hat{y}_t^{\mathrm{del}}\right)$ &
$\hat{D}\left(\breve{y}_{t_i}^{\delta^{1/2}}\right)$\\
&&&&\\[-9pt]
\hline
 &  0{,}01 &0,4860&0,6432&0,0486\\
0{,}01 &  0{,}03&\textbf{0,5491}&\textbf{0,6407}&\textbf{0,0729}\\
 & 0{,}1\hphantom{9}&\textbf{0,6107}&\textbf{0,6398}&\textbf{0,1225}\\
 \hline
0{,}03 & 0{,}03&0,5660&0,6569&\textbf{0,0986}\\
\hline
0{,}1\hphantom{9} &  0{,}1\hphantom{9}&0,6496&0,6796&\textbf{0,1837}\\
\hline
\end{tabular}
}

\smallskip

\parbox{78mm}{\footnotesize{\hspace*{5mm}\textbf{Параметры:} $a=1$; $b=0{,}5$; $(c_1, c_2, c_3)= (-1, 0, 1)$; $\Lambda= 
\begin{pmatrix}
-0{,}5 & 0{,}5 &0\\
0{,}5 &-1& 0{,}5\\
0 & 0{,}5& -0{,}5
\end{pmatrix}$.
}}

\end{center}

%\vspace*{6pt}

\noindent
 фильт\-ра\-ми не произошло 
ничего~--- качество осталось ров\-но таким же, как и~в~устойчивом случае. 
Это в~равной степени касается и~их робастных свойств в~отношении 
неопределенности границы~$\Sigma_t$.



     
     

    
    

  
      Интересно понять, что именно происходит с~аппроксимациями 
фильт\-ра Вонэма. Пред\-став\-ле\-ние об этом дают рис.~1 и~2, на 
которых приведены примеры одной и~той же траектории цепи и~всех 
оценок в~двух расчетах, по\-свя\-щен\-ных анализу ро\-баст\-ности.
{\looseness=1

}
      
      
      
      Видно, что оценка дискретизованного фильт\-ра остается 
содержательной на всем интервале. При этом траектория 
$\breve{y}_{t_i}^{\delta^{1/2}}$, видимо, хуже отслеживает~$y_t$ 
для большего значения~$\Sigma_t$, но фильтр работает. Обе аппроксимации 
фильт\-ра Вонэма начинают проигрывать с~самого начала, а~в~итоге 
<<разваливаются>>, причем делают это по-раз\-но\-му: оценке 
$\hat{y}_t^{\mathrm{lim}}$ свойственны резкие многократные колебания, а~оценка 
$\hat{y}_t^{\mathrm{del}}$ в~итоге <<зависает>> и~перестает работать. Надо отметить, 
что такое поведение имеют все смоделированные траектории. 
Окончательные итоги подведены в~за\-клю\-че\-нии.



\section{Заключение}
      
      Статья продолжает исследования свойств и~приложений 
дискретизованных фильт\-ров~--- эффективных дискретных аппроксимаций 
фильтра Вонэма, обес\-пе\-чи\-ва\-ющих устой\-чи\-вость чис\-лен\-ной реализации. 
Рас\-смот\-рен\-ная по\-ста\-нов\-ка включает не\-опре\-де\-лен\-ность ин\-тен\-сив\-ности шума в~наблюдениях,\linebreak
 типичную для минимаксных задач, 
успешно решенных для линейных сис\-тем. В~отсутствие подходящих аналитических результатов 
анализ алгоритмов выполнен на объемном чис\-лен\-ном \mbox{эксперименте}. 

Результаты под\-твер\-ди\-ли ожи\-да\-емую ро\-баст\-ность оценок всех фильт\-ров~--- 
самого фильт\-ра Вонэма, его эмпирических устойчивых аппроксимаций 
и~дискретизованных фильт\-ров. При этом отсутствие устой\-чи\-вости схемы  
Эй\-ле\-ра--Ма\-ру\-ямы в~по\-ста\-нов\-ке с~не\-опре\-де\-лен\-ностью остается такой же 
проб\-ле\-мой для фильт\-ра Вонэма, как и~в~задаче с~пол\-ной информацией. 
Преимущество дискретизованных фильт\-ров, т.\,е.\ их чис\-лен\-ная 
устойчивость для любых комбинаций па\-ра\-мет\-ров сис\-те\-мы наблюдения, 
принимает абсолютный характер в~задаче с~не\-опре\-де\-лен\-ностью, когда 
сис\-те\-ма наблюдения сама становится неустойчивой.

{\small\frenchspacing
 {\baselineskip=12pt
 %\addcontentsline{toc}{section}{References}
 \begin{thebibliography}{99}
\bibitem{1-bos}
      \Au{Kalman R.\,E., Bucy R.\,S.} New results in linear filtering and prediction theory~// 
J.~Basic Eng.~--- T. ASME, 1961. Vol.~83. P.~95--108. doi: 10.1115/1.3658902.
\bibitem{2-bos}
      \Au{Wonham W.\,M.} Some applications of stochastic differential equations to optimal 
nonlinear filtering~// SIAM J. Control, 1965. Vol.~2. No.\,3. P.~347--369. doi: 10.1137/ 030202.
\bibitem{3-bos}
\Au{Липцер Р.\,Ш., Ширяев~А.\,Н.} Статистика случайных процессов (нелинейная 
фильт\-ра\-ция и~смеж\-ные во\-про\-сы).~--- М.: Наука, 2001. 696~с.
\bibitem{4-bos}
\Au{D'Appolito J.\,A., Hutchinson~C.\,E.} A~minimax approach to the design of low sensitivity 
state estimators~// Automatica, 1972. Vol.~8. No.\,5. P.~599--608. doi:  
10.1016/0005-1098(72)90031-3.

\bibitem{6-bos} %5
      \Au{Morris J.} The Kalman filter: A~robust estimator for some classes of linear quadratic 
problems~// IEEE T. Inform. Theory, 1976. Vol.~22. No.\,5. P.~526--534. doi: 
10.1109/ TIT.1976.1055611.

\bibitem{5-bos} %6
      \Au{Vandelinde V.} Robust properties of solutions to linear-quadratic estimation and 
control problems~// IEEE T. Automat. Contr., 1977. Vol.~22. No.\,1. P.~138--139. doi: 
10.1109/TAC.1977.1101433.

\bibitem{7-bos}
\Au{Poor V., Looze~D.} Minimax state estimation for linear stochastic systems with noise 
uncertainty~// IEEE T. Automat. Contr., 1981. Vol.~26. No.\,4. P.~902--906. doi: 
10.1109/TAC.1981.1102756.
\bibitem{8-bos}
      \Au{Verdu S., Poor~H.} Minimax linear observers and regulators for stochastic systems 
with uncertain second-order statistics~// IEEE T. Automat. Contr., 1984. Vol.~29. No.\,6.  
P.~499--511. doi: 10.1109/TAC.1984.1103576.
\bibitem{9-bos}
\Au{Verdu S., Poor~H.} On minimax robustness: A~general approach and applications~// IEEE 
T. Inform. Theory, 1984. Vol.~30. No.\,2. P.~328--340. doi: 10.1109/ TIT.1984.1056876.
\bibitem{10-bos}
\Au{Панков А.\,Р.} Стратегии управ\-ле\-ния в~линейной сто\-ха\-сти\-че\-ской сис\-те\-ме 
с~негауссовскими возмущениями~// Автомат. и~телемех., 1994. №\,6. C.~74--83.
\bibitem{11-bos}
\Au{Миллер Г.\,Б., Панков~А.\,Р.} Минимаксная фильт\-ра\-ция в~линейных  
не\-опре\-де\-лен\-но-сто\-ха\-сти\-че\-ских дис\-крет\-но-не\-пре\-рыв\-ных сис\-те\-мах~// 
Автомат. и~телемех., 2006. №\,3. C.~77--93.
\bibitem{12-bos}
\Au{Миллер Г.\,Б., Панков~А.\,Р.} Минимаксное управ\-ле\-ние процессом в~линейной  
не\-опре\-де\-лен\-но-сто\-ха\-сти\-че\-ской сис\-те\-ме с~неполными данными~// Автомат. 
и~телемех., 2007. №\,11. C.~164--177.
\bibitem{13-bos}
\Au{Elliott R.\,J., Aggoun~L., Moore~J.\,B.} Hidden Markov models: Estimation and control.~--- 
New York, NY, USA: Springer-Verlag, 1995. 396~p.
\bibitem{14-bos}
\Au{Kloeden P.\,E., Platen~E.} Numerical solution of stochastic differential equations.~--- 
Berlin: Springer, 1992. 636~p.
\bibitem{15-bos}
      \Au{Yin G., Zhang~Q., Liu~Y.} Discrete-time approximation of Wonham filters~// 
J.~Control Theory Applications, 2004. Vol.~2. P.~1--10. doi: 10.1007/s11768-004-0017-7.

\bibitem{18-bos} %16
\Au{Борисов А.\,В.} Численные схемы фильтрации марковских скачкообразных 
процессов по дискретизованным наблюдениям II: случай аддитивных шумов~// 
Информатика и~её применения, 2020. Т.~14. Вып.~1. C.~17--23. doi: 
10.14357/19922264200103.


\bibitem{17-bos}
\Au{Борисов А.\,В.} L1-оп\-ти\-маль\-ная фильтрация марковских скачкообразных 
процессов II: численный анализ конкретных схем~// Автомат. и~телемех., 2020. №\,12. 
С.~24--49.

\bibitem{19-bos} %18
\Au{Борисов А.\,В., Босов~А.\,В.} Практическая реализация решения задачи стабилизации 
линейной сис\-те\-мы со скачкообразным случайным дрейфом по косвенным наблюдениям~// 
Автомат. и~телемех., 2022. №\,9. С.~109--127.
\bibitem{20-bos} %19
\Au{Davis M.\,H.\,A.} Linear estimation and stochastic control.~--- London: Chapman and Hall, 
1977. 224~p.

\bibitem{16-bos} %20
\Au{Борисов А.\,В.} Фильтрация со\-сто\-яний марковских скачкообразных процессов по 
дискретизованным наблюдениям~// Информатика и~её применения, 2018. Т.~12. Вып.~3. 
C.~115--121. doi: 10.14357/ 19922264180316.


\end{thebibliography}

 }
 }

\end{multicols}

\vspace*{-6pt}

\hfill{\small\textit{Поступила в~редакцию 15.03.23}}

\vspace*{8pt}

%\pagebreak

%\newpage

%\vspace*{-28pt}

\hrule

\vspace*{2pt}

\hrule

%\vspace*{-2pt}

\def\tit{ROBUSTNESS INVESTIGATION OF~THE~NUMERICAL APPROXIMATION 
OF~THE~WONHAM FILTER}


\def\titkol{Robustness investigation of~the~numerical approximation 
of~the~Wonham filter}


\def\aut{A.\,V.~Bosov}

\def\autkol{A.\,V.~Bosov}

\titel{\tit}{\aut}{\autkol}{\titkol}

\vspace*{-10pt}


\noindent
Federal Research Center ``Computer Science and Control'' of the Russian Academy 
of Sciences, 44-2~Vavilov Str., Moscow 119333, Russian Federation


\def\leftfootline{\small{\textbf{\thepage}
\hfill INFORMATIKA I EE PRIMENENIYA~--- INFORMATICS AND
APPLICATIONS\ \ \ 2023\ \ \ volume~17\ \ \ issue\ 2}
}%
 \def\rightfootline{\small{INFORMATIKA I EE PRIMENENIYA~---
INFORMATICS AND APPLICATIONS\ \ \ 2023\ \ \ volume~17\ \ \ issue\ 2
\hfill \textbf{\thepage}}}

\vspace*{3pt}
      
      
      
      \Abste{The properties of the optimal continuous Markov chain state filtering problem 
decision given be the linear observations noisy Wiener process, assuming incomplete information about 
its intensity, are investigated. The uncertainty of the observation system is set by the upper bound of the 
noise intensity. Numerical implementation of the optimal solution in the statement with complete 
information provided by the Wonham filter does not guarantee stability. It is shown that the Wonham 
filter in the statement with uncertainty is robust with respect to the noise intensity if the model parameters 
do not lead to its divergence. In the general case, the instability of the Euler--Maruyama numerical scheme 
of the Wonham filter is preserved. Simple heuristic techniques that provide stable approximations of the 
Wonham filter show the workability for a~wider set of parameters. However, in the statement with 
uncertainty, it is possible to give examples when such heuristic filters show unacceptably low quality. The 
best solution is provided by discretized filters, approximations of the Wonham filter implemented for 
a~specific model approximating the initial continuous observation system. A~series of numerical 
experiments has shown that these filters have robustness for all sets of parameters. If there are no 
divergent trajectories among the modeled trajectories of the Wonham filter in the calculations, then the 
discretized filters lose a~little. If there are divergent trajectories, then the gain of discretized filters is 
absolute.}
      
      \KWE{Markov jump process; stochastic filtering; robust estimation; Wonham filter;  
Euler--Maruyama numerical scheme; discretized filters}
      
      
\DOI{10.14357/19922264230206}{BGILKR}

\vspace*{-18pt}

\Ack

\vspace*{-3pt}

\noindent
This work was supported by the Russian Science Foundation, project No.\,22-28-00588. The 
research was carried out using the infrastructure of the Shared Research Facilities ``High Performance 
Computing and Big Data'' (CKP ``Informatics'') of FRC CSC RAS (Moscow).
  

\vspace*{6pt}

  \begin{multicols}{2}

\renewcommand{\bibname}{\protect\rmfamily References}
%\renewcommand{\bibname}{\large\protect\rm References}

{\small\frenchspacing
 {%\baselineskip=10.8pt
 \addcontentsline{toc}{section}{References}
 \begin{thebibliography}{99} 
\bibitem{1-bos-1}
      \Aue{Kalman, R.\,E., and R.\,S.~Bucy.} 1961. New results in linear filtering and prediction 
theory. \textit{J. Basic Eng.~--- T. ASME} 83(1):95--108. doi: 10.1115/1.3658902.
\bibitem{2-bos-1}
      \Aue{Wonham, W.\,M.} 1965. Some application of stochastic differential equations to optimal 
nonlinear filtering. \textit{SIAM J. Control} 2(3):347--369. doi: 10.1137/ 030202.
\bibitem{3-bos-1}
      \Aue{Liptser, R.\,S., and A.\,N.~Shiryaev.} 2001. \textit{Statistics of random processes II. 
Applications.} Berlin: Springer-Verlag. 402~p.
\bibitem{4-bos-1}
      \Aue{D'Appolito, J.\,A., and C.\,E.~Hutchinson.} 1972. A~minimax approach to the design of 
low sensitivity state estimators. \textit{Automatica} 8(5):599--608. doi: 10.1016/0005-1098(72)90031-3.

\bibitem{6-bos-1} %5
      \Aue{Morris, J.} 1976. The Kalman filter: A~robust estimator for some classes of linear 
quadratic problems. \textit{IEEE T. Inform. Theory} 22(5):526--534. doi: 10.1109/TIT.1976.1055611.

\bibitem{5-bos-1} %6
      \Aue{Vandelinde, V.} 1977. Robust properties of solutions to linear-quadratic estimation and 
control problems. \textit{IEEE T. Automat. Contr.} 22(1):138--139. doi: 10.1109/ TAC.1977.1101433.

\bibitem{7-bos-1}
      \Aue{Poor, V., and D.~Looze.} 1981. Minimax state estimation for linear stochastic systems with 
noise uncertainty. \textit{IEEE T. Automat. Contr.} 26(4):902--906. doi: 10.1109/ TAC.1981.1102756.
\bibitem{8-bos-1}
      \Aue{Verdu, S., and H.~Poor.} 1984. Minimax linear observers and regulators for stochastic 
systems with uncertain second-order statistics. \textit{IEEE T. Automat. Contr.} 29(6):499--511. doi: 
10.1109/TAC.1984.1103576.
\bibitem{9-bos-1}
      \Aue{Verdu, S., and H.~Poor.} 1984. On minimax robustness: A~general approach and 
applications. \textit{IEEE T. Inform. Theory} 30(2):328--340. doi: 10.1109/TIT.1984.1056876.
\bibitem{10-bos-1}
      \Aue{Pankov, A.\,R.} 1994. Control strategies in a~linear stochastic system with non-Gaussian 
perturbations. \textit{Automat. Rem. Contr.} 55(6):832--840. 
\bibitem{11-bos-1}
      \Aue{Miller, G.\,B., and A.\,R.~Pankov.} 2006. Minimax filtering in linear stochastic uncertain 
discrete-continuous systems. \textit{Automat. Rem. Contr.} 67(3):413--427.
\bibitem{12-bos-1}
      \Aue{Miller, G.\,B., and A.\,R.~Pankov.} 2007. Minimax control of a~process in a linear 
uncertain-stochastic system with incomplete data. \textit{Automat. Rem. Contr.} 68(11):2042--2055.
\bibitem{13-bos-1}
      \Aue{Elliott, R.\,J., L.~Aggoun, and J.\,B.~Moore.} 1995. \textit{Hidden Markov models: 
Estimation and control}. New York, NY: Springer-Verlag. 396~p.
\bibitem{14-bos-1}
      \Aue{Kloden, P.\,E., and E.~Platen.} 1992. \textit{Numerical solution of stochastic differential 
equations}. Berlin: Springer. 636~p.
\bibitem{15-bos-1}
      \Aue{Yin, G., Q.~Zhang, and Y.~Liu.} 2004. Discrete-time approximation of Wonham filters. 
\textit{J.~Control Theory Applications}  2:1--10. doi: 10.1007/s11768-004-0017-7.

\bibitem{18-bos-1} %16
      \Aue{Borisov, A.\,V.} 2020. Chis\-len\-nye skhe\-my fil't\-ra\-tsii mar\-kov\-skikh skach\-ko\-ob\-raz\-nykh 
pro\-tses\-sov po dis\-kre\-ti\-zo\-van\-nym nab\-lyu\-de\-niyam II: Slu\-chay ad\-di\-tiv\-nykh shu\-mov [Numerical schemes of 
Markov jump process filtering given discretized observations II: Additive noise case]. \textit{Informatika 
i~ee Primeneniya~--- Inform. Appl.} 14(1):17--23. doi: 10.14357/19922264200103.

\bibitem{17-bos-1}
      \Aue{Borisov, A.\,V.} 2020. L1-optimal filtering of Markov jump processes. II. Numerical 
analysis of particular realizations schemes. \textit{Automat. Rem. Contr.} 81(12):2160--2180.

\bibitem{19-bos-1} %18
      \Aue{Borisov, A.\,V., and A.\,V.~Bosov.} 2022. Practical implementation of the stabilization 
problem solution for a~linear system with discontinuous random drift by indirect observations. 
\textit{Automat. Rem. Contr.} 83(9):1417--1432. doi: 10.31857/S0005231022090069.
\bibitem{20-bos-1} %19
      \Aue{Davis, M.\,H.\,A.} 1977. \textit{Linear estimation and stochastic control}. London: 
Chapman and Hall. 224~p.

\bibitem{16-bos-1} %20
      \Aue{Borisov, A.\,V.}  2018. Fil't\-ra\-tsiya so\-sto\-yaniy \mbox{mar\-kov\-skikh} skach\-ko\-ob\-raz\-nykh pro\-tses\-sov 
po dis\-kre\-ti\-zo\-van\-nym nab\-lyu\-de\-niyam [Filtering of Markov jump processes by discretized observations]. 
\textit{Informatika i~ee Primeneniya~--- Inform. Appl.} 12(3):115--121. doi: 10.14357/ 19922264180316.

\end{thebibliography}

 }
 }

\end{multicols}

\vspace*{-6pt}

\hfill{\small\textit{Received March 15, 2023}} 
      
      \Contrl
      
      \noindent
      \textbf{Bosov Alexey V.} (b.\ 1969)~--- Doctor of Science in technology, principal scientist, 
Institute of Informatics Problems, Federal Research Center ``Computer Science and Control'' of the 
Russian Academy of Sciences, 44-2~Vavilov Str., Moscow 119333, Russian Federation; 
\mbox{AVBosov@ipiran.ru}



   
\label{end\stat}

\renewcommand{\bibname}{\protect\rm Литература} 
      