\documentclass[10pt]{book}
\usepackage[utf8]{inputenc}

\usepackage{latexsym,amssymb,amsfonts,amsmath,amsxtra,dsfont,
indentfirst,shapepar,%fleqn,%
picinpar,shadow,floatflt,enumerate,multicol,colortbl,moreverb,cite,ipi}

\usepackage{rotating}
\usepackage{mathrsfs}
\usepackage[noend]{algorithmic}
\usepackage{ulem}
\usepackage{graphicx}
%\usepackage{algorithm2e}
\usepackage[linesnumbered,boxed,ruled]{algorithm2e}
%\usepackage{xypic}
\usepackage{oldgerm}
\usepackage{epic}
\usepackage{eepic}

\SetAlgorithmName{Algorithm}{алгоритм}{Список алгоритмов}

%из Дюковой

\newcommand{\algKeyword}[1]{{\bf #1}}
\newcommand{\Proc}[1]{\text{\tt #1}}
\def\CALL{\algKeyword{call}~}

\newenvironment{AlgProcedure}[1]
{
\small
\medskip
%    \hrule
\medskip
\algKeyword{PROCEDURE} #1
\begin{algorithmic}[1]}
{\end{algorithmic}
%    \hrule
\bigskip
}

\def\CALL{\algKeyword{call}~}

%конец для Дюковой

%\RequirePackage[ruled]{algorithm}


\input{epsf}

%\nofiles

%\includeonly{avtor}    %pdf
%\includeonly{podgot-rus-site,podgot-eng-site}  
%\includeonly{podgot-rus,podgot-eng}  
%\includeonly{ipi-ind} 
%\includeonly{index-16i}
%\includeonly{toc-rus, toc-en}
%\includeonly{toc-rus}
%\includeonly{toc-en} 
%\includeonly{popravka}



%\includeonly{bosov}          %pdf+авт+
%\includeonly{rumovsk}        %pdf+авт+
%\includeonly{razumchik}      %pdf+авт+
%\includeonly{ushakov}        %pdf+авт+
%\includeonly{shestakov-vor}  %pdf+авт+
%\includeonly{kovalev}        %pdf+авт+
%\includeonly{torshin}        %pdf+авт+
%\includeonly{krivenko}       %pdf+авт+
%\includeonly{grusho}         %pdf+авт+
%\includeonly{borisov}        %pdf+авт
%\includeonly{samuylov}       %pdf+авт
%\includeonly{nuriev}         %+авт
%\includeonly{hachumovi}      %+pdf+авт+
%\includeonly{vasilev}        %+pdf+авт


%%%%%%%%%%%%%%%%%%%\includeonly{nekrolog-new}



%\includeonly{rekl}




\usepackage{acad}
%\usepackage{courier}
\usepackage{decor}
\usepackage{newton}
\usepackage{pragmatica}
\usepackage{zapfchan}
\usepackage{petrotex}
\usepackage{bm}                     % полужирные греческие буквы
\usepackage{upgreek}                % прямые греческие буквы \upalpha
\usepackage{eufrak}
\usepackage{verbatim}

\renewcommand{\bottomfraction}{0.99}
\renewcommand{\topfraction}{0.99}
\renewcommand{\textfraction}{0.01}

\setcounter{secnumdepth}{1} %здесь - 3 + chapter = 4

\arraycolsep=1.5pt

%\usepackage[pdftex]{graphicx}

%\usepackage{oz}

%NEW COMMANDS



\renewcommand*{\hm}[1]{#1\nobreak\discretionary{}%
            {\hbox{$\mathsurround=0pt #1$}}{}} %% Дублирует знаки операций
                               %при переносе в формуле (перед знаком, который
                               %надо продублировать ставится команда \hm)
                               
                               \newcommand{\PRB}{\begin{picture}(22.5,11)
      \spline(1,8)(4,10)(7,10.5)(10,11)(13,11)(16,10.5)(19,10)(22,8)
               \put(0,0){$P_{i-1}P_{t_{t-1}}$} \end{picture}}

\newcommand{\prb}{\begin{picture}(15.5,9)
      \spline(1,6)(3,8)(5,8.5)(7,9)(9,9)(11,8.5)(13,8)(15,6)
               \put(0,0){$PP_t$} \end{picture}}
               
                 \newcommand{\PRDN}{\begin{picture}(40,11)
      \spline(4,11.5)(7,10.5)(12,10)(16,9)(20,9)(24,10)(29,10.5)(32,11.5)
               \put(0,0){$P_{i-1}P_{t_{t-1}}$} \end{picture}}

\newcommand{\prdn}{\begin{picture}(18,11)
      \spline(3,10.5)(4,10)(6,9)(8,8.5)(10,8.5)(12,9)(14,10)(15,10.5)
               \put(0,0){$PP_t$} \end{picture}}




%\newcommand{\endproof}{\hfill$\Box$}
%\renewcommand{\r}{\mathbb{R}}
%\newcommand{\I}{{\rm I\hspace{-0.7mm}I}}
%\newcommand{\Ikl}{{\tt{1}}\hspace*{-1.44mm}\mathtt{1}}
%\newcommand{\Ik}{\mbox{{\small \tt {1}}\hspace{-1.3mm}{\tt 1}}}
\newcommand{\Ik}{\mbox{{{\tt 1}}\hspace{-1.3mm}{\sf 1}}}
\newcommand{\argmin}{\mathop{\mathrm{arg}\,\mathrm{min}}}
\newcommand{\argmax}{\mathop{\mathrm{arg}\,\mathrm{max}}}
%\newcommand{\capr}{\mathop{\cap\,}}
%\newcommand{\cupr}{\mathop{\cup\,}}
%\def\argmin{\mathop{arg\,min}}

\def\vrp{\varphi}
\def\prt{\partial}
\def\mm{{\sf M}}
\def\modnop#1{\mathop{#1}\limits_{n}}
\def\eam{\mathbin{{\mathop{=}\limits^{\mathrm{def}}}}}
\def\dey#1#2{#1 (#2)}
\def\deyc#1#2{#1 \cdot  #2}
\def\ra#1{\;\mathop{\to}\limits^{#1}\;}
\def\raz#1{\;\mathop{\longrightarrow}\limits^{\!\!\!#1}\;}
\def\ral#1{\;\mathop{\longrightarrow}\limits^{#1}\;}





\newcommand{\il}[2]{\int\limits_{#1}^{#2}}%интеграл с пределами #1 и #2

\def\sm2{\mathop {\sum\limits^{n^\Theta}\sum\limits^{n^\Theta}}}
\def\sss{\sum\limits}
\def\tr{,\,\ldots\,,\,}
\def\rk{\right]}
\def\lk{\left[}
\def\rf{\right\}}
\def\lf{\left\{}
\def\lv{\,\left\vert}
\def\rv{\right\vert\,}
\def\iii{\int\limits}
\def\iin{\int\limits_{-\infty}^\infty}
\def\rrv{\right\vert}


\def\ee{{\cal E}}
\def\ww{{\cal W}}
\def\yy{{\cal Y}}
\def\vv{{\cal V}}

\newcommand{\R}{\mathbb R}
\newcommand{\E}{\mathbb E}
\newcommand{\N}{\mathbb N}
\newcommand{\T}{\mathbb{T}}
\newcommand{\Z}{\mathbb{Z}}

\renewcommand{\P}{\mathbb{P}}

\newcommand{\Nor}{\mathcal{N}}

\newcommand{\h}{{\bf H}}
\newcommand{\p}{{\sf P}}  % вероятность
\newcommand{\e}{{\sf E}}  % мат. ожидание
\newcommand{\D}{{\sf D}}  % дисперсия



\newcommand{\vw}{{\mathbf w}}
\newcommand{\vp}{{\mathbf p}}
\newcommand{\vz}{{\mathbf z}}
\newcommand{\vx}{{\mathbf x}}
\newcommand{\vf}{{\mathbf f}}
\newcommand{\F}{{\mathcal F}}
\def\ap{{\mathrm{ЭР}}}
\newcommand{\ud}{\Delta_n} %uniform ditance
\newcommand{\nud}{\Delta_n(x)}
%\renewcommand{\Re}{\mathrm{Re}\,}

\newcommand{\abs}[1]{\left\vert#1\right\vert}

\newcommand{\norm}[1]{\left\Vert#1\right\Vert}
\def\da{(\Delta_t,A)}

\newcommand{\corr}{\mathrm{corr}}

\newcommand{\cov}{\mathrm{cov}}
\newcommand{\Expect}{\mathbb{E}}

\def\w{\omega}
\def\W{\Omega}


\def\inh{\int\limits_{nh}^{(n+1)h}}

\def\sumin{\sum_{i=1}^N}


\def\bxt{(Y,t)}
\def\xt{(y,t)}

\def\ovth{{\fr{\tau-nh}{h}}}
\def\ov{\overline}
\def\tm{\tilde m}
\def\tl{\tilde\lambda}
\def\tB{\widetilde B}
\def\tb{\tilde b}
\def\ld{\ldots}
\def\cd{\cdots}


\DeclareMathOperator{\sign}{sign}



\newcommand{\g}{\mbox{\textit{g}}}

\renewcommand{\la}{\lambda}
\newcommand{\si}{\sigma}
\newcommand{\eps}{\varepsilon}
\newcommand{\alp}{\alpha}

\newcommand{\pto}{\stackrel{P}{\longrightarrow}} % сходимость по веpоятности

\newcommand{\eqd}{\stackrel{\mathrm{d}}{=}} % равенство по pаспpеделению
\newcommand{\eqdelta}{\stackrel{\triangle}{=}} % равенство по pаспpеделению

\def\be#1{\begin{equation}\label{#1}}
\def\ee{\end{equation}}
\def\re#1{(\ref{#1})}

\def\bn{\begin{enumerate}}
\def\en{\end{enumerate}}
\def\bi{\begin{itemize}}
\def\ei{\end{itemize}}
%\def\i{\item}

%\newcommand{\kp}{\kappa}
%\def\Q{{\cal Q}} \def\H{{\cal H}}
%\newcommand{\bet}{\beta_{2+\delta}}




%\renewcommand{\baselinestretch}{1.2}

%\pagestyle{myheadings}

\setlength{\textwidth}{167mm}      % 122mm
\setlength{\textheight}{658pt}
%\setlength{\textheight}{635.6pt}
\setlength{\columnsep}{4.5mm}

\setcounter{secnumdepth}{4}

%\addtolength{\headheight}{2pt}
%\addtolength{\headsep}{-2mm}

\addtolength{\topmargin}{-7mm}  % for printing


%\hoffset=-30mm  % From Yap
\hoffset=-23mm  % From Acrobat

%\voffset=0mm % From Yap
\voffset=-5mm   % From Acrobat

%\addtolength{\evensidemargin}{-2.5mm} % for printing
%\addtolength{\oddsidemargin}{2.5mm}  % for printing

\addtolength{\evensidemargin}{-12mm} % for printing
\addtolength{\oddsidemargin}{8mm}  % for printing

%\renewcommand{\thefootnote}{\fnsymbol{footnote}}
%\renewcommand{\thefootnote}{\arabic{footnote}}
\renewcommand{\figurename}{\protect\bf Рис.}
\renewcommand{\tablename}{\protect\bf Таблица}

\newcommand{\Caption}[1]{\caption{\protect\small %\baselineskip=2.5ex
#1}}

\renewcommand{\thefigure}{\arabic{figure}}
\renewcommand{\thetable}{\arabic{table}}
\renewcommand{\theequation}{\arabic{equation}}
\renewcommand{\thesection}{\arabic{section}}

\renewcommand{\contentsname}{СОДЕРЖАНИЕ}
\newcommand{\fr}[2]{\displaystyle\frac{\displaystyle #1\mathstrut}{\displaystyle #2\mathstrut}}

%\renewcommand{\thefootnote}{\fnsymbol{footnote}}
%\newcommand{\g}{\mbox{\textit{g}}}

%\newcommand{\Caption}[1]{\caption{\protect\small\baselineskip=2ex #1}}
\newcounter{razdel}
\setcounter{razdel}{0}

\def\god{2023}
\def\tom{17}
\def\vyp{2}


\newcommand{\titel}[4]{%
\

\vspace*{5pt}

\ifodd\therazdel {\raggedright\noindent\Large\textrm\textbf
 \lineskip .75em
  \baselineskip=3.2ex #1 \par}
\vskip 1em {\noindent\large\textrm\textbf #2 \par}
\addcontentsline{toc}{subsection}{{\textrm\textbf #1}\protect\newline #2}
\def\rightheadline{\underline{\noindent\hbox to \textwidth{\hfill\small\textrm{#4}
%\hfill \large\bf\thepage
}}}
\def\leftheadline{\underline{\noindent\parbox{\textwidth}{
%\raggedleft\large\bf\thepage \hfill
\small\textit{#3}\hfill}}}
\def\leftfootline{\small{\textbf{\thepage}
\hfill ИНФОРМАТИКА И ЕЁ ПРИМЕНЕНИЯ\ \ \ том~\tom\ \ \ выпуск~\vyp\ \ \ \god}
}%
 \def\rightfootline{\small{ИНФОРМАТИКА И ЕЁ ПРИМЕНЕНИЯ\ \ \ том~\tom\ \ \ выпуск~\vyp\ \ \ \god
\hfill \textbf{\thepage}}}
\vskip 2em \setcounter{figure}{0}
\setcounter{table}{0}
\setcounter{equation}{0}
\setcounter{section}{0}
\setcounter{subsection}{0}
\setcounter{subsubsection}{0}
\setcounter{footnote}{0}
\setcounter{razdel}{0}
%\end{flushleft}
\else {
 \raggedright\noindent\Large\textrm\textbf
 \lineskip .75em
\baselineskip=3.2ex #1 \par} \vskip 1em
%\begin{flushleft}
{\noindent\large\textrm\textbf #2 \par}
\addcontentsline{toc}{subsection}{{\textrm\textbf #1}\protect\newline #2}
\def\rightheadline{\underline{\noindent\hbox to \textwidth{\hfill\small\textrm{#4}
%\hfill \large\bf\thepage
}}}
\def\leftheadline{\underline{\noindent\parbox{\textwidth}{%\raggedleft\large\bf\thepage \hfill
\small\textit{#3}\hfill}}}
\def\leftfootline{\small{\textbf{\thepage}
\hfill ИНФОРМАТИКА И ЕЁ ПРИМЕНЕНИЯ\ \ \ том~\tom\ \ \ выпуск~\vyp\ \ \ \god}
}%
 \def\rightfootline{\small{ИНФОРМАТИКА И ЕЁ ПРИМЕНЕНИЯ\ \ \ том~17\ \ \ выпуск~\vyp\ \ \ 2023
\hfill \textbf{\thepage}}} \vskip 2em \setcounter{figure}{0}
\setcounter{table}{0} \setcounter{equation}{0} \setcounter{section}{0}
\setcounter{subsection}{0} \setcounter{subsubsection}{0}
\setcounter{footnote}{0}
%\end{flushleft}
\fi}

\newcommand{\titelr}[2]{%
\

\vspace*{5pt}

\ifodd\therazdel {\raggedright\noindent%\Large\textrm\textbf
 \lineskip .75em
  \baselineskip=3.2ex #1 \par}
\vskip 1em {\noindent\normalsize\textrm\textbf #2 \par}
\else {
 \raggedright\noindent\Large\textrm\textbf
 \lineskip .75em
\baselineskip=3.2ex #1 \par} \vskip 1em
%\begin{flushleft}
{\noindent\large\textrm\textbf #2 \par
%\noindent\normalsize\textrm\textbf #2 \par
} \fi}

\newcommand{\titele}[5]{%
\

%\vspace*{5pt}

\ifodd\therazdel {\raggedright\noindent\large
\textrm\textbf
 \lineskip .75em
%  \baselineskip=3.2ex
#1 \par}
\vskip .5em {\noindent\large\textrm\textbf #2 \par}
\vskip .5em
 {\noindent\textrm #3 \par}
\addcontentsline{toc}{subsection}{{\textrm\textbf #1}\protect\newline #2}
\def\rightheadline{\underline{\noindent\hbox to \textwidth{\hfill\small\textrm{#4}
%\hfill \large\bf\thepage
}}}
\def\leftheadline{\underline{\noindent\parbox{\textwidth}{
%\raggedleft\large\bf\thepage \hfill
\small\textrm{#5}\hfill}}}
\def\leftfootline{\small{\textbf{\thepage}
\hfill ИНФОРМАТИКА И ЕЁ ПРИМЕНЕНИЯ\ \ \ том~17\ \ \ выпуск~2\ \ \ 2023}
}%
 \def\rightfootline{\small{ИНФОРМАТИКА И ЕЁ ПРИМЕНЕНИЯ\ \ \ том~17\ \ \ выпуск~2\ \ \ 2023
\hfill \textbf{\thepage}}} \vskip 1em \setcounter{figure}{0}
\setcounter{table}{0} \setcounter{equation}{0} \setcounter{section}{0}
\setcounter{subsection}{0} \setcounter{subsubsection}{0}
\setcounter{footnote}{0} \setcounter{razdel}{0}
%\end{flushleft}
\else {
 \raggedright\noindent\large
 \textrm\textbf
 \lineskip .75em
%\baselineskip=3.2ex
#1 \par} \vskip .5em
%\begin{flushleft}
{\noindent\large\textrm\textbf #2 \par} \vskip .5em
 {\noindent\textrm #3 \par}
\addcontentsline{toc}{subsection}{{\textrm\textbf #1}\protect\newline #2}
\def\rightheadline{\underline{\noindent\hbox to \textwidth{\hfill\small\textrm{#4}
%\hfill \large\bf\thepage
}}}
\def\leftheadline{\underline{\noindent\parbox{\textwidth}{%\raggedleft\large\bf\thepage \hfill
\small\textrm{#5}\hfill}}}
\def\leftfootline{\small{\textbf{\thepage}
\hfill ИНФОРМАТИКА И ЕЁ ПРИМЕНЕНИЯ\ \ \ том~17\ \ \ выпуск~2\ \ \ 2023}
}%
 \def\rightfootline{\small{ИНФОРМАТИКА И ЕЁ ПРИМЕНЕНИЯ\ \ \ том~17\ \ \ выпуск~2\ \ \ 2023
\hfill \textbf{\thepage}}} \vskip 1em \setcounter{figure}{0}
\setcounter{table}{0} \setcounter{equation}{0} \setcounter{section}{0}
\setcounter{subsection}{0} \setcounter{subsubsection}{0}
\setcounter{footnote}{0}
%\end{flushleft}
\fi}

\def\Abst#1{
\begin{center}\small\nwt
\parbox{150mm}{%\baselineskip=2.5ex
\textbf{Аннотация:}\ \
%\hspace*{\parindent}
#1}
\end{center}}
\def\Abste#1{
\begin{center}\small\nwt
\parbox{150mm}{%\baselineskip=2.5ex
\textbf{Abstract:}\ \
%\hspace*{\parindent}
#1}
\end{center}}

%\def\DOI#1{
%\begin{center}\small\nwt
%\parbox{150mm}{%\baselineskip=2.5ex
%\textbf{DOI:}\ \
%%\hspace*{\parindent}
%#1}
%\end{center}}

\def\Abstend#1{
\begin{center}\small\nwt
\parbox{150mm}{%\baselineskip=2.5ex
%\hspace*{\parindent}
#1}
\end{center}}

\newcommand{\DOI}[2]{\begin{center}\small\nwt
\parbox{150mm}{%\baselineskip=2.5ex
\textbf{DOI:}\ \
%\hspace*{\parindent}
#1 \hfill \textbf{EDN:}\ \
#2}
\end{center}}




\def\KW#1{
\begin{center}\small\nwt
\parbox{150mm}{%\baselineskip=2.5ex
\textbf{Ключевые слова:}\ \ #1}
\end{center}}

\def\KWE#1{
\begin{center}\small\nwt
\parbox{150mm}{%\baselineskip=2.5ex
\textbf{Keywords:}\ \ #1}
\end{center}}


\def\KWN#1{
%\begin{center}
%\small
%\parbox{150mm}\end{center}
}

\newcommand{\Avtors}[1]{%\smallskip
%\vspace*{.5pt}
\hangindent=23pt\noindent
%\nwt
{\bfseries#1}\
}


\renewcommand{\thesubsection}{\thesection.\arabic{subsection}\hspace*{-5pt}}
\renewcommand{\thesubsubsection}{\thesubsection\hspace*{5pt}.\arabic{subsubsection}\hspace*{-3pt}}

\newcommand{\Ack}{\section*{\protect\rmfamily Acknowledgments}\noindent}
\newcommand{\Contr}{\section*{\protect\rmfamily Contributors}\noindent}
\newcommand{\Contrl}{\section*{\protect\rmfamily Contributor}\noindent}

\makeindex


\begin{document}
\Rus

\nwt
%\ptb


%\renewcommand{\contentsname}{\protect\Large\bf Содержание}

\setcounter{tocdepth}{2}

%\tableofcontents

\renewcommand{\bibname}{\protect\rmfamily Литература}
  \def\Au#1{{\it #1}}
    \def\Aue#1{{#1}}

%\newcommand{\No}{№}
  \newcommand{\tg}{\,\mathrm{tg}\,}
    \newcommand{\ctg}{\,\mathrm{ctg}\,}
  \newcommand{\arctg}{\,\mathrm{arctg}\,}

\def\forallb{\mathop{\forall}}
\def\cupb{\mathop{\cup}}
\def\existsb{\mathop{\exists}}


\newpage
\addtocounter{razdel}{1}
%\def\razd{РЕГУЛИРУЕМЫЙ ЭЛЕКТРОПРИВОД ДЛЯ ЭЛЕКТРОЭНЕРГЕТИКИ}


\setcounter{page}{2}

%   { %\Large  
   { %\baselineskip=16.6pt
   
   \vspace*{-48pt}
   \begin{center}\LARGE
   \textit{Предисловие}
   \end{center}
   
   %\vspace*{2.5mm}
   
   \vspace*{25mm}
   
   \thispagestyle{empty}
   
   { %\small 

    
Вниманию читателей журнала <<Информатика и её применения>> предлагается 
очередной тематический выпуск <<Вероятностно-статистические методы и 
задачи информатики и информационных технологий>>. Предыдущие тематические 
выпуски журнала по данному направлению вышли в 2008~г.\ (т.~2, вып.~2), 
в 2009~г.\ (т.~3, вып.~3) и в 2010~г.\ (т.~4, вып.~2). 

Статьи, собранные в данном журнале, посвящены разработке новых вероятностно-статистических 
методов, ориентированных на применение к решению конкретных задач информатики и информационных 
технологий, а также~--- в ряде случаев~--- и других прикладных задач. Проблематика, охватываемая 
публикуемыми работами, развивается в рамках научного сотрудничества между Институтом проблем 
информатики Российской академии наук (ИПИ РАН) и Факультетом вычислительной математики и 
кибернетики Московского государственного университета им.\ М.\,В.~Ломоносова в ходе работ 
над совместными научными проектами (в том числе в рамках функционирования 
Научно-образовательного центра <<Вероятностно-статистические методы анализа рисков>>). 
Многие из авторов статей, включенных в данный номер журнала, являются активными участниками 
традиционного международного семинара по проблемам устойчивости стохастических моделей, 
руководимого В.\,М.~Золотаревым и В.\,Ю.~Королевым; регулярные сессии этого семинара 
проводятся под эгидой МГУ и ИПИ РАН (в 2011~г.\ указанный семинар проводится в октябре 
в Калининградской области РФ). 

Наряду с представителями ИПИ РАН и МГУ в число авторов данного выпуска журнала входят 
ученые из Научно-исследовательского института системных исследований РАН, Института 
проблем технологии микроэлектроники и особочистых материалов РАН, Института 
прикладных математических исследований Карельского НЦ РАН, Московского 
авиационного института, Вологодского государственного педагогического университета, 
НИИММ им.\ Н.\,Г.~Чеботарева, Казанского государственного университета, Дебреценского 
университета (Венгрия).

Несколько статей выпуска посвящено разработке и применению стохастических методов и 
информационных технологий для решения различных прикладных задач. В~работе В.\,Г.~Ушакова 
и О.\,В.~Шестакова рассмотрена задача определения вероятностных характеристик случайных 
функций по распределениям интегральных преобразований, возникающих в задачах эмиссионной 
томографии. В~статье Д.\,О.~Яковенко и М.\,А.~Целищева рассмотрены некоторые вопросы 
математической теории риска и предложен новый подход к диверсификации инвестиционных 
портфелей. Работа И.\,А.~Кудрявцевой и А.\,В.~Пантелеева посвящена построению и 
исследованию математической модели, описывающей динамику сильноионизованной плазмы. 
В~статье П.\,П.~Кольцова изучается качество работы ряда алгоритмов сегментации изображений. 
Статья А.\,Н.~Чупрунова и И.~Фазекаша посвящена вероятностному анализу числа без\-оши\-бочных 
блоков при помехоустойчивом кодировании; получены усиленные законы больших чисел для указанных 
величин.

В данном выпуске традиционно присутствует тематика, весьма активно разрабатываемая в течение 
многих лет специалистами ИПИ РАН и МГУ,~--- методы моделирования и управления для 
информационно-телекоммуникационных и вычислительных систем, в частности методы 
теории массового обслуживания. В~статье А.\,И.~Зейфмана с соавторами рассматриваются 
модели обслуживания, описываемые марковскими цепями с непрерывным временем в случае 
наличия катастроф. В~работе М.\,М.~Лери и И.\,А.~Чеплюковой рассматриваются случайные 
графы Интернет-типа, т.\,е.\ графы, степени вершин которых имеют степенные распределения; 
такие задачи находят применение при исследовании глобальных сетей передачи данных. 
Работа Р.\,В.~Разумчика посвящена исследованию систем массового обслуживания специального 
вида~--- с отрицательными заявками и хранением вытесненных заявок.

Ряд статей посвящен развитию перспективных теоретических 
вероятностно-статистических методов, которые находят широкое применение в различных 
задачах информатики и информационных технологий. В~работе В.\,Е.~Бенинга, А.\,К.~Горшенина 
и В.\,Ю.~Королева рассмотрена задача статистической проверки гипотез о числе компонент 
смеси вероятностных распределений, приводится конструкция асимптотически наиболее мощного 
критерия. Результаты этой работы найдут применение в ряде прикладных задач, использующих 
математическую модель смеси вероятностных распределений (в информатике, моделировании 
финансовых рынков, физике турбулентной плазмы и~т.\,д.). В~статье В.\,Ю.~Королева, 
И.\,Г.~Шевцовой и С.\,Я.~Шоргина строится новая, улучшенная оценка точности нормальной 
аппроксимации для пуассоновских случайных сумм; как известно, указанные случайные суммы 
широко используются в качестве моделей многих реальных объектов, в том числе в информатике, 
физике и других прикладных областях. Работа В.\,Г.~Ушакова и Н.\,Г.~Ушакова посвящена 
исследованию ядерной оценки плотности распределения; эти результаты могут применяться, 
в част\-ности, при анализе трафика в телекоммуникационных системах. Серьезные приложения 
в статистике могут получить результаты работы О.\,В.~Шестакова, в которой доказаны оценки 
скорости сходимости распределения выборочного абсолютного медианного отклонения к нормальному 
закону. 

\smallskip

Редакционная коллегия журнала выражает надежду, что данный тематический  выпуск 
будет интересен специалистам в области теории вероятностей и математической статистики 
и их применения к решению задач информатики и информационных технологий.
     
     %\vfill 
     \vspace*{20mm}
     \noindent
     Заместитель главного редактора журнала <<Информатика и её 
применения>>,\\
     директор ИПИ РАН, академик  \hfill
     \textit{И.\,А.~Соколов}\\
     
     \noindent
     Редактор-составитель тематического выпуска,\\
     профессор кафедры математической статистики факультета\\
      вычислительной математики и кибернетики МГУ им.\ М.\,В.~Ломоносова,\\
     ведущий научный сотрудник ИПИ РАН,\\ 
доктор физико-математических наук \hfill
      \textit{В.\,Ю.~Королев}
     
     } }
     }

\def\stat{torshin}

\def\tit{О ПОРОЖДЕНИИ СИНТЕТИЧЕСКИХ ПРИЗНАКОВ НА~ОСНОВЕ~ОПОРНЫХ ЦЕПЕЙ 
И~ПРОИЗВОЛЬНЫХ МЕТРИК В~РАМКАХ~ТОПОЛОГИЧЕСКОГО ПОДХОДА 
К~АНАЛИЗУ ДАННЫХ.\\ ЧАСТЬ~2.~ЭКСПЕРИМЕНТАЛЬНАЯ АПРОБАЦИЯ\\ НА~ЗАДАЧАХ ФАРМАКОИНФОРМАТИКИ$^*$}

\def\titkol{О порождении синтетических признаков на основе опорных цепей 
и~произвольных метрик} % в~рамках топологического подхода  к~анализу данных. Часть~2. Экспериментальная апробация на  задачах фармакоинформатики}

\def\aut{И.\,Ю.~Торшин$^1$}

\def\autkol{И.\,Ю.~Торшин}

\titel{\tit}{\aut}{\autkol}{\titkol}

\index{Торшин И.\,Ю.}
\index{Torshin I.\,Yu.}


{\renewcommand{\thefootnote}{\fnsymbol{footnote}} \footnotetext[1]
{Работа выполнена при поддержке гранта РНФ (проект №\,23-21-00154) с~использованием инфраструктуры 
Центра коллективного пользования <<Высокопроизводительные вычисления и~большие данные>> (ЦКП 
<<Информатика>>) ФИЦ ИУ РАН (г.~Москва).}}


\renewcommand{\thefootnote}{\arabic{footnote}}
\footnotetext[1]{Федеральный исследовательский центр <<Информатика и~управление>> Российской академии наук, 
\mbox{tiy135@yahoo.com}}

\vspace*{-12pt}


\Abst{Рассмотрение прецедентных отношений между признаками и~таргетной переменной в~виде наборов элементов булевой решетки указывает на возможность порождения 
синтетических признаков с~использованием метрических функций расстояния. 
Сформулированы подходы к~(1)~оценке релевантности (<<информативности>>) метрик 
по отношению к~решаемым задачам, (2)~порождению и~(3)~отбору синтетических 
признаков, более информативных, чем исходные признаковые описания. Представленные 
результаты топологического анализа 2400~выборок данных  
<<мо\-ле\-ку\-ла--свойство>> из ProteomicsDB позволили получить достаточно 
эффективные алгоритмы прогнозирования свойств молекул (ранговая корреляция  
в~кросс-ва\-ли\-да\-ции~--- $0{,}90\pm0{,}23$). На данной выборке задач установлены 
метрики, которые наиболее часто порождают информативные синтетические признаки: 
максимальное уклонение Колмогорова, <<косое>> расстояние, метрики Lp, Реньи, фон 
Мизеса. Для решения изученного комплекса задач показано преимущество полиномных 
корректоров по сравнению с~нейросетевыми и~с~корректорами типа <<случайный 
лес>>.}

\KW{топологический анализ данных; теория решеток; алгебраический подход 
Ю.\,И.~Жу\-рав\-лё\-ва; фармакоинформатика}

\DOI{10.14357/19922264240207}{OTXCUD}
  
\vspace*{-1pt}


\vskip 10pt plus 9pt minus 6pt

\thispagestyle{headings}

\begin{multicols}{2}

\label{st\stat}

\section{Введение}

     В первой части работы~[1] принимается, что задано регулярное 
множество прецедентов 
$$
\mathbf{Q}\hm= \{\mathrm{D}(x_i)\vert x_i\in 
\mathbf{X}\}
$$ 
на решетке $L(T(\mathbf{X}))$, по\-рож\-ден\-ное на основе 
множества исходных описаний объектов $\mathbf{X}\hm= \{ x_1, \ldots , 
x_{N_0}\}$. Для индивидуального объекта\linebreak $x_i\hm\in \mathbf{X}$ 
прецедентному соотношению между значениями признаками 
$\Gamma_k(x_i)$ и~\mbox{$t$-й} таргетной переменной соответствует множество пар 
$\{(\{\Gamma_k^{-1}(\Gamma_k(x_i)),\linebreak k\hm=\overline{1, ,n}\}, \Gamma_t^{-1}(\Gamma_t(x_i))), i\hm=\overline{1,N_0},\
 k\hm=\overline{1,n},\linebreak t\hm=\overline{n+1, n+l}\}$, где $l$~--- 
число таргетных переменных. В~рамках топологической теории 
распознавания прецедентное соотношение между множествами $\{ 
\Gamma_k^{-1}(\Gamma_k(x_i))\}$ и~$\Gamma_t^{-1}(\Gamma_t(x_i))$ 
моделируется как со\-от\-вет\-ст\-ву\-ющие массивы расстояний, по\-рож\-да\-емые той 
или иной мет\-ри\-кой~$\rho_m$: $L(T(\mathbf{X}))^2\hm\to [0\ldots 1]$, 
$m\hm= \overline{1, m_0}$. В~[1] предложены способы <<встра\-и\-ва\-ния>> 
в~формализм полуэмирических рас\-сто\-яний на множествах $a\hm\in 
L(T(\mathbf{X}))$, векторах $\vec{v}_\alpha [a] \hm= ( v_{\alpha_1}[a], 
v_{\alpha_2}[a], \ldots , v_{\alpha_i}[a],\ldots)$ и~функциях 
$\hat{\phi}(x)\bm{\Gamma}_t(u)$. 
     
     Здесь для практического приложения формализма сформулированы 
подходы к~исследованию свойств~$\rho_m$, способы оценки релевантности 
функций~$\rho_m$ по отношению к~решаемым задачам, способы 
порождения и~отбора синтетических признаков, основанных на~$\rho_m$. 
Представлены результаты экспериментальной апробации на задачах 
фармакоинформатики.
     
\section{Об исследовании свойств функций расстояния~$\rho_m$}

    Рабочая гипотеза настоящего исследования со\-сто\-ит в~том, что для 
порождения более <<информативных>> признаков могут использоваться 
полуэмпирические функционалы расстояния на \mbox{множествах}, векторах, 
функциях~[2]. Метрические свойства ис\-поль\-зу\-емых функций 
расстояния~$\rho_m$ могут исследоваться аналитически или комбинаторно 
с~использованием аксиом метрики~[3]. Для анализа свойств этих 
функционалов в~топологической теории распознавания вводится следующее 
понятие.

\smallskip

\noindent
\textbf{Определение~1.} Обобщенной оценочной функцией расстояния 
будем называть конструкцию вида 
$$
\rho(a,b) = f(g ( v[a\vee b]) - g(v[a\wedge b])),
$$
 в~которой~$f$ и~$g$~--- функции, монотонные на 
соответствующих участках действительной оси; $v:\ L\hm\to R^+$~--- 
изотонная оценка, для которой выполнено условие оценки (\textbf{уО}: $\forall_L 
a,b: v[a]\hm+v[b]\hm= v[a \wedge b]\hm+ v[a\vee b]$) и~изотонности 
(\textbf{уИ}:  $\forall_L a,b: a\supseteq b \hm\Rightarrow v[a]\hm\geq v[b]$). 

\smallskip

\noindent
\textbf{Теорема~1.} \textit{Функция расстояния~$\rho$ считается 
обобщенной оценочной функцией расстояния тогда и~только тогда, когда 
$\rho(a,b)\hm= \rho(a\vee b, a\wedge b)$, а~термы от $a$ и~$b$ в~формуле для 
$\rho(a,b)$ представляют собой композицию монотонной функции 
и~изотонной оценки}. 

\smallskip

Необходимость следует из  $a\vee b\hm= (a\vee b)\vee (a\wedge b)$ и~$a\wedge b \hm= (a\vee b) \wedge (a\wedge b)$  при 
подстановке $a\vee b$ и~$a\wedge b$ вместо $a$ и~$b$ в~определение~1. 
Эквивалентность $\rho(a,b)$ и~$\rho(a\vee b, a\wedge b)$ указывает на то, что 
в~выражение для вычисления~$\rho$ входят тер\-мы-функ\-ци\-о\-на\-лы, 
содержащие выражения $a\vee b$ и~$a\wedge b$, взаимозаменяемые с~$a$ 
и~$b$, т.\,е.\ термы вида $g^\prime (a\vee b)$ и~$g^\prime(a\wedge b)$. По 
условию теоремы эти термы включают монотонную функцию от изотонной 
оценки, т.\,е.~$g^\prime$ монотонна. Так как $\rho$~--- функция расстояния, 
то $g^\prime$-тер\-мы не могут входить в~выражение для~$\rho$ в~виде 
произведения, суммы, отношения, степени или суммы, а~только в~виде 
разности, т.\,е.\
$$
\rho(a,b) = f\left(g^\prime(a\vee b) \hm- g^\prime (a\wedge b)\right),
$$ 
из чего следует достаточность. Теорема доказана.

\smallskip

\noindent
\textbf{Следствие~1.} Для обобщенной оценочной~$\rho$ 
\begin{multline*}
\forall \ell \subseteq L(T(\mathbf{X})): \Delta_{\vee\wedge}(\ell)\equiv 0,\\ 
\Delta_{\vee\wedge}(\ell)=  \sum\limits_{a,b\in \ell} \vert\rho(a,b)- 
\rho(a\vee b, a\wedge b)\vert \fr{2}{\vert\ell\vert/(\vert\ell\vert -1)}\,.
\end{multline*}

\smallskip

\noindent
\textbf{Следствие~2.} Выберем <<опорное>> множество $a\hm\in 
L(T(\mathbf{X}))$ и~обобщенную оценочную~$\rho$. При $f(x)\hm= g(x)\hm= x$ 
$v_{a,\rho}[b]\hm= \rho(a,b)\hm= \rho(a\vee b, a\wedge b)$~--- изотонная 
оценка. 

Следует из того, что любая линейная комбинация изотонных оценок~--- 
изотонная оценка при условии положительной определенности (теорема~2 
в~[4]). Также проверяется прямой подстановкой $v_{a,\rho}[b]$ в~уО и~уИ. 

\smallskip

\noindent
\textbf{Следствие~3.} Расстояния Фре\-ше--Ни\-ко\-ди\-ма, Амана,  
Рэн\-да/Ще\-ка\-нов\-ско\-го, Со\-ка\-ла--Сни\-са (варианты~1, 2 и~3),  
Рас\-се\-ла--Рао, Род\-же\-ра--Та\-ни\-мо\-то, Фейта, Тверского и~Юле 
могут служить обобщенными оценочными функциями расстояния. 

\smallskip

\noindent
\textbf{Следствие~4.} Расстояния Симпсона, Бра\-у\-на--Блан\-ке, 
Андерберга и~Говера-2  не входят в~число обобщенных оценочных функций 
расстояния.

\smallskip

     Теорема~1 со следствиями предоставляет аналитический 
и~комбинаторный инструментарий для исследования свойств 
полуэмпирических функций расстояния. Если заданная~$\rho$ служит 
обобщенной оценочной функцией расстояния, то могут быть получены 
соответствующие аналитические выражения для функций~$f$ и~$g$. 
Например, расстояние Со\-ка\-ла--Сни\-са-2
$$
\rho(a,b) = 1- \fr{\vert a\cap 
b\vert }{\vert a\cup b\vert + \vert a\Delta b\vert}
$$ 
выступает 
обобщенным оценочным расстоянием с~$f(x)\hm= (e^x\hm-1)/(0{,}5e^x\hm-1)$ и~$g(x)\hm=\ln (x)$. При невозможности аналитической проверки 
свойства~$\rho$ как обобщенной оценочной могут быть изучены на 
подмножествах~$\ell$ решетки $L(T(\mathbf{X}))$ посредством вычисления 
значений функционала $\Delta_{\vee\wedge}(\ell)$ (следствие~1). 

\section{О способах оценки релевантности метрик~$\rho_m$ по~отношению к~задаче клас\-сификации/прогнозирования}

     Биекция между множеством прецедентов~$\mathbf{Q}$ и~множеством 
исходных описаний объектов~$\mathbf{X}$, существующая при выполнении 
условия регулярности по Журавлёву ($\forall \mathrm{x}\hm\in \mathbf{X}, 
\mathrm{x}\hm= D^{-1}(D(\mathrm{x}))$, гарантирует однозначность 
соответствия описаний~$x_i$ и~$q_i$. Это делает возможным рассматривать 
прецедентные соотношения, заданные на~$\mathbf{Q}$, в~терминах 
множеств $\{ \Gamma_k^{-1}(\Gamma_k(x_i))\}$ и~$\Gamma_t^{-1}( 
\Gamma_t(x_i))$ с~использованием расстояний~$\rho_m$ на подмножествах 
множества~$\mathbf{X}$~[1].
     
     Пусть таргетный класс объектов $\mathbf{c}_{\bm{\alpha}}$ задан 
посредством $\alpha$-го значения $t$-й переменной $\lambda_{t\alpha}\hm\in 
\mathrm{I}_t$, $t\hm= \overline{n+1,  n+l}$, как $\mathbf{c}_{{\bm 
\alpha}} \hm= \Gamma_t^{-1}(\lambda_{t\alpha})$. В~случае числовой 
переменной за $\mathbf{c}_{\bm{\alpha}}$ может приниматься каждый из 
элементов $u(\lambda_{t\alpha})$ цепи~$A_t$. Так как 
$L(T(\mathbf{X}))$ булева, то дополнение множества 
$\mathbf{c}_{\bm{\alpha}}$, $\overline{\mathbf{c}}_{\bm{\alpha}} \hm= 
\mathbf{X}\backslash \Gamma_t^{-1}(\lambda_{t\alpha})$, определено 
однозначно. Таким образом, выделение класса $\mathbf{c}_{\bm{\alpha}}$ 
порождает задачу классификации $\mathbf{c}_{\bm{\alpha}}/ 
\overline{\mathbf{c}}_{\bm{\alpha}}$. Любая задача числового 
прогнозирования может быть сведена к~последовательности корректно 
решаемых задач $\mathbf{c}_{\bm{\alpha}}/ 
\overline{\mathbf{c}}_{\bm{\alpha}}$~\cite{5-tor}.
     
     Пусть задано подмножество признаков~$p \hm\subseteq [1\ldots n]$ 
     и~элемент решетки $c\in L(T(\mathbf{X}))$. Определим функцию 
$$
\bm{\rho}_{\mathbf{mc}} (x_i, c, {p}) \hm= \{ \rho_m(c, \Gamma_k^{-1}(\Gamma_k (x_i)),\ k\hm\in {p})\}.
$$
 При заданных~$\rho_m$, $p$, 
$\mathbf{c}_{\bm{\alpha}}$ и~$\overline{\mathbf{c}}_{\bm{\alpha}}$ 
для~$x_i$ вычислимы множества расстояний $\bm{\rho}_{\mathbf{mc}}(x_i, 
\mathbf{c}_{\bm{\alpha}}, {p})$ и~$\bm{\rho}_{\mathbf{mc}}(x_i, 
\overline{\mathbf{c}}_{\bm{\alpha}}, {p})$. Обозначим 
\begin{align*}
\bm{\rho}_{\mathbf{m}\bm{\alpha}}(x_i) &=  \bm{\rho}_{\mathbf{mc}} 
(x_i, \mathbf{c}_{\bm{\alpha}}, [1\ldots n]); \\
\bm{\rho}_{\mathbf{m}\overline{\bm{\alpha}}} (x_i) &= 
\bm{\rho}_{\mathbf{mc}}(x_i, \overline{\mathbf{c}}_{\bm{\alpha}} , [1\ldots n]).
\end{align*}
 Для $x_i\hm\in \mathbf{X}$ 
определено множество 
\begin{multline*}
\bm{\rho}_{\mathbf{m}}(x_i,{p})=\left \{ \rho_{mk_1k_2}(x_i, {p}) = {}\right.\\
{}\rho_m\left(\Gamma^{-1}_{k_1}\left(\Gamma_{k_1}(x_i), \Gamma^{-1}_{k_2}\left(\Gamma_{k_2}(x_i)\right)\right)\right),\\
\left. k_1, k_2\hm \in {p},\  k_1\not= k_2\right\},\ \bm{\rho}_{\mathbf{m}}(x_i)=  \bm{\rho}_{\mathbf{m}}(x_i, [1\ldots n]).
\end{multline*}
     
     На основе $\bm{\rho}_{\mathbf{m}{\bm{\alpha}}}(x_i)$ 
и~$\bm{\rho}_{\mathbf{m}\overline{\bm{\alpha}}}(x_i)$ вводятся оценки 
релевантности~$\rho_m$. По отношению к~задаче $\mathbf{c}_{\bm{\alpha}}/ 
\overline{\mathbf{c}}_{\bm{\alpha}}$ более релевантна или 
<<информативна>> такая мет\-ри\-ка~$\rho_m$, которая для всех $x\hm\in 
\mathbf{c}_{\bm{\alpha}}$ минимизирует расстояния в~списке 
$\bm{\rho}_{\mathbf{m}{\bm{\alpha}}}(x)$ и~максимизирует расстояния 
в~списке $\bm{\rho}_{\mathbf{m}\overline{\bm{\alpha}}}(x)$ (т.\,е.\ 
<<приближает>> объекты к~их классам). Выделены два взаимосвязанных 
направления дальнейших исследований: 
\begin{enumerate}[(1)]
\item нахождение подмножеств $p$ 
признаков, <<более информативных>> для~$\rho_m$;  
\item на\-строй\-ка/вы\-бор~$\rho_m$ при фиксированном~$p$.
\end{enumerate}
     
     Для $c^\prime\hm\in L(T(\mathbf{X}))$ определим 
$\vartheta_{\mathbf{mc}}$, операцию слияния списков 
$\bm{\rho}_{\mathbf{mc}}$:
$$
\vartheta_{\mathbf{mc}}(c^\prime, c, 
{p})\hm= \bigcup\limits_{y\in c^\prime} \bm{\rho}_{\mathbf{mc}} (y,c, 
{p}).
$$
 Обозначим 
 $$
 \vartheta_{\mathbf{m}\bm{\alpha}}(\mathbf{c}, 
{p}) \!=\! \vartheta_{\mathbf{mc}}(\mathbf{c}, 
\mathbf{c}_{\bm{\alpha}}, {p});\ 
\vartheta_{\mathbf{m}\bm{\alpha}}(\mathbf{c},{p})\!=\! 
\vartheta_{\mathrm{mc}}(\mathbf{c}, \overline{\mathbf{c}}_{\bm \alpha}, 
{p}),
$$
 вычислим множества $\vartheta_{\mathbf{m}{\bm \alpha}} 
(\mathbf{c}_{\bm \alpha}, {p})$ и~$\vartheta_{\mathbf{m}{\bm \alpha}} 
(\overline{\mathbf{c}}_{\bm \alpha},{p})$ и~сформируем 
эмпирические функции распределения (э.ф.р.)\ $\hat{\phi}(x) 
\vartheta_{\mathbf{m}{\bm \alpha}} (\mathbf{c}_{\bm \alpha}, {p})$ 
и~$\hat{\phi}(x) \vartheta_{\mathbf{m}{\bm \alpha}} 
(\overline{\mathbf{c}}_{\bm \alpha}, {p})$. На пространстве 
однородных монотонно возрастающих функций 
\begin{multline*}
\mathbf{M}^+_{0\ldots1} ={}\\
{}= 
\{f: [0\ldots 1]\hm\to [0\ldots 1],\ x\geq y\hm\Rightarrow f(x)\geq f(y)\}
\end{multline*}
введем 
функционал расстояния $d_f$: $\mathbf{M}^+_{0..1}\hm\to [0\ldots 1]$ 
(максимальное уклонение Колмогорова $D(f(x), g(x))\hm= \mathrm{sup}_x 
\vert f(x)\hm- g(x)\vert$, метрики фон Мизеса, Реньи и~др.). Выбор~$d_f$ 
делает возможной постановку ряда задач топологического анализа данных:
     \begin{enumerate}[(1)]
\item количественные оценки релевантности~$\rho_m$ как 
$d_f(\hat{\phi}(x)\vartheta_{\mathbf{m}{\bm \alpha}}(\mathbf{c}_{\bm \alpha}, 
{p}), \hat{\phi}(x)\vartheta_{\mathbf{m}{\bm 
\alpha}}(\overline{\mathbf{c}}_{\bm \alpha}, {p}))$ для 
разных~$\mathbf{c}_{\bm \alpha}$, $\lambda_{t\alpha} \hm\in \mathrm{I}_t$, 
$\alpha \hm= \overline{1, \vert \mathrm{I}_t\vert}$;
\item задачи оптимизации для увеличения разделения классов 
$\mathbf{c}_{\bm \alpha}/\overline{\mathbf{c}}_{\bm \alpha}$ 
($\argmax_{\rho_m,{p}} d_f(\hat{\phi}\vartheta_{\mathbf{m}{\bm \alpha}}(\overline{\mathbf{c}}_{\bm \alpha},{p}), 
\hat{\phi}\vartheta_{\mathbf{m}{\bm \alpha}}(\mathbf{c}_{\bm \alpha}, {p}))$,
$\argmax_{\rho_m,{p}} d_f(\hat{\phi}\vartheta_{\mathbf{m}\overline{\bm{\alpha}}}, (\overline{\mathbf{c}}_{\bm \alpha}, {p}), 
\hat{\phi}\vartheta_{\mathbf{m}\overline{\bm \alpha}}
(\mathbf{c}_{\bm \alpha}, {p}))$  и~др.);
\item определение $\rho_q$-мет\-рик на пространстве объектов~[2, с.~184--199] 
(например, в~виде $d_f (\hat{\phi}\bm{\rho}_{\mathbf{m}{\bm \alpha}}(x, 
{p}), \hat{\phi}\bm{\rho}_{\mathbf{m}{\bm \alpha}}(y, {p})), 
d_f (\hat{\phi}\bm{\rho}_{\mathbf{m}}(x,{p})$, 
$\hat{\phi}\bm{\rho}_{\mathbf{m}}(y, {p}))$); 
\item оценка близости метрик~$\rho_q$ к~метрике разреза по классам 
$\mathbf{c}_{\bm{\alpha}}/ \overline{\mathbf{c}}_{\bm{\alpha}}$; 
\item формулировка критериев раз\-ре\-ши\-мости/ре\-гу\-ляр\-ности задачи 
$\mathbf{c}_{\bm{\alpha}}/ \overline{\mathbf{c}}_{\bm{\alpha}}$~[6]; 
\item оценки компактности классов $\mathbf{c}_{\bm{\alpha}}$  
и~$\overline{\mathbf{c}}_{\bm{\alpha}}$~[3]. 
\end{enumerate}

\section{О способах порождения и~отбора синтетических 
признаков на~основании функций расстояния}

     Множества $\bm{\rho}_{\mathbf{m}{\bm \alpha}}(x_i,{p})$, 
$\bm{\rho}_{\mathbf{m}{\overline{\bm \alpha}}}(x_i, {p})$ 
и~$\bm{\rho}_{\mathbf{m}}(x_i)$ и~отдельные $\rho_m(\mathbf{c}_{\bm 
\alpha}, \Gamma_k^{-1}(\Gamma_k(x_i))$ используются для формирования 
синтетических числовых признаков $\Gamma_{k^\prime}(x_i)$ 
объекта~$x_i$, $k^\prime\hm= \overline{n+ l+1, n+l+n_S}$. 
Значение синтетического признака~$\Gamma_{k^\prime}(x_i)$ зависит от 
выбора~$\rho_m$, классов $\mathbf{c}_{\bm{\alpha}}$ 
и~$\overline{\mathbf{c}}_{\bm{\alpha}}$  и~от способа его вы\-чис\-ле\-ния: 
\begin{enumerate}[(1)]
\item $\rho_m(\mathbf{c}_{\bm \alpha}, \Gamma_k^{-1}(\Gamma_k(x_i))$; 
\item $\rho_m(\overline{\mathbf{c}}_{\bm \alpha}, \Gamma_k^{-1}(\Gamma_k(x_i))$; 
\item $\rho_m(\mathbf{c}_{\bm \alpha}, \ldots ) \hm- \rho_m(\overline{\mathbf{c}}_{\bm \alpha}, \ldots)$;
\item $1\hm- \rho_m(\mathbf{c}_{\bm \alpha}, \ldots)$;
\item значения э.ф.р.\ 
$\hat{\phi}(x)\bm{\rho}_{\mathbf{m}{\bm \alpha}}(x_i,{p})$ при 
разных~$x$ (например, соответствующих процентилям 
$\hat{\phi}\bm{\rho}_{\mathbf{m}{\bm \alpha}}(x_i,{p})$); 
\item значения $\hat{\phi}(x)\bm{\rho}_{\mathbf{m}\overline{\bm{\alpha}}} 
(x_i, {p})$ при разных~$x$;
\item $\hat{\phi}(x\hm+ \Delta x) 
\bm{\rho}_{\mathbf{m}{\bm \alpha}}(x_i,p) \hm- 
\hat{\phi}(x)\bm{\rho}_{\mathbf{m}{\bm \alpha}} (x_i, {p})$ 
и~$\hat{\phi}(x\hm+ \Delta x) \bm{\rho}_{\mathbf{m}{\overline{\bm \alpha}}} 
(x_i,{p}) \hm- \hat{\phi}(x)\bm{\rho}_{\mathbf{m}\overline{\bm 
\alpha}} (x_i, {p})$, где $\Delta x$~--- шаг.
\end{enumerate}
     
     Кроме того, $\mathbf{c}_{\bm{\alpha}}$ может определяться как 
$\Gamma_t^{-1}(\lambda_{t\alpha})$ или как $u(\lambda_{t\alpha})$; если 
$\mathbf{c}_{\bm \alpha} \hm= \Gamma_t^{-1}(\lambda_{t\alpha})$, то 
$\overline{\mathbf{c}}_{\bm{\alpha}}$ может быть равно $\Gamma^{-1}_t 
(\lambda_{t\alpha+1})$; классы $\mathbf{c}_{\bm{\alpha}}/ 
\overline{\mathbf{c}}_{\bm{\alpha}}$  
$t$-й переменной могут определяться с~использованием раз\-би\-ений на 
различные процентили (которые определяются как подвыборка значений 
$\lambda_{t\alpha} \hm\in \mathrm{I}_t$) и~т.\,д. 
     
     Таким образом, предлагаемые схемы порождают значительное число 
синтетических признаков $\Gamma_{k^\prime}(x_i)$ ($10n$ и~более при $n$ 
исходных признаках $\Gamma_k$), что делает необходимым введение 
процедур отбора признаков. Таргетная переменная $\Gamma_t(x_i)$~--- 
чис\-ло\-вая, и~по\-рож\-да\-емые признаки $\Gamma_{k^\prime}(x_i)$~--- также 
чис\-ло\-вые. Для данного случая в~прикладной математике имеется несколько 
различных подходов к~оценке взаимосвязи $\Gamma_t(x_i)$ 
и~$\Gamma_{k^\prime}(x_i)$: корреляционные оценки (для линейных 
закономерностей), полиномная аппроксимация с~оценкой качества (для 
нелинейных закономерностей) и~методы теории  
ве\-ро\-ят\-но\-стей\,/\,ма\-те\-ма\-ти\-че\-ской статистики, не зависящие от 
вида закономерности (в~том числе на основе <<взаимной 
информации>>~[7]).
{\looseness=1

}
     
     Наиболее фундаментальным представляется тес\-ти\-ро\-ва\-ние взаимосвязи 
двух переменных на осно\-ве <<нулевой гипотезы>> об их независимости. 
Пусть заданы пары тестируемых значений, $(x_i, y_i)$,\linebreak $i\hm= \overline{1,\mathbf{n}_{(\mathrm{x,y})}}$, э.ф.р.~$F_{xy}(x,y)$ характеризует 
совместное распределение~$x$ и~$y$, а~э.ф.р.~$F_{{x}}(x)$ 
и~$F_{{y}}(y)$~--- индивидуальные распределения переменных. 
Эмпирическая функция распределения нулевой \mbox{гипотезы} (независимость~$x$ и~$y$) определяется как 
$F_{{x}}(x)F_{{y}}(y)$. 
     
     Для оценки отличий между $F_{{xy}}(x,y)$\linebreak 
и~$F_{{x}}(x) F_{{y}}(y)$ необходимо ввести расстояние 
меж-\linebreak ду такими функциями (так называемую <<статисти-\linebreak ку>>) и~оценить 
достоверность различий посред\-ст\-вом \mbox{того} или иного статистического\linebreak \mbox{тес\-та}. 
В~качестве расстояния можно использовать функции~$d_f$, адап\-ти\-ро\-ван\-ные 
для 2-мер\-но\-го случая (например, макси\-маль\-ное уклонение 
     $D(\mathrm{F}_{{xy}}(x,y), \mathrm{F}_{{x}}(x) 
\mathrm{F}_{{y}}(y)) \hm= \max ( \vert 
\mathrm{F}_{{xy}}(x_i,y_i) \hm- \mathrm{F}_{{x}}(x_i) 
\mathrm{F}_{{y}}(y_i)\vert )$) и~статистический тест  
Кол\-мо\-го\-ро\-ва--Смир\-но\-ва 
$P_{\mathrm{КС}}$ $(D 
(\mathrm{F}_{{xy}}(x,y), \mathrm{F}_{{x}}(x) 
\mathrm{F}_{{y}}(y)), n_{(x,y)})$. Тогда $1\hm- 
P_{\mathrm{КС}}$ характеризует <<информативность>>~$x$ 
относительно~$y$. 
     
     Более универсальным подходом к~оценке достоверности различий 
между $\mathrm{F}_{{xy}}(x,y)$ и~$\mathrm{F}_{{x}}(x) 
\mathrm{F}_{{y}}(y)$ считается прямое вычисление выбранной 
статистики~$d_f$ на множествах пар значений $(x_i, y_i)$, полученных 
датчиком случайных чисел. 
     
     Пусть \textit{оператор $\hat{\zeta}$, семплирующий} 
множество~$\mathbf{X}$, формирует набор семплов 
$$
\hat{\zeta}\mathbf{X}\hm= \{a_1, a_2, \ldots , a_k, \ldots , 
a_{\vert\hat{\zeta}X\vert}\vert a_k\hm\subset \mathbf{X}\},
$$
 а~процедура 
random~--- датчик случайных чисел (в~диапазоне $[0\ldots 1]$). Для каждого 
семпла~$a_k$ принимается, что ${n}_{({x,y})} \hm= \vert 
a_k\vert$, и~вычисляется множество значений~$d_f$ для случайных 
выборок, 

\noindent
\begin{multline*}
\mathrm{rnd}\,(\hat{\zeta}\mathbf{X}, d_f)= \left\{ 
\vphantom{i=\overline{1,\left\vert \hat{\zeta} X\right\vert }}
d_f\left(
\vphantom{\overline{1, \vert a_i\vert }}
\mathrm{F}_{{xy}}(x_{ij}, y_{ij}), 
\mathrm{F}_{{x}}(x_{ij}) \mathrm{F}_{{y}}(y_{ij}),\right.\right.\\
\left.\left. x_{ij}, 
y_{ij}= \mathrm{random},\  j=\overline{1, \vert a_i\vert }\right),\ i=\overline{1,\left\vert \hat{\zeta} X\right\vert }\right\}.
\end{multline*}

 Для $a\hm\in \hat{\zeta} \mathbf{X}$ значение 
${P}(d_f, \hat{\zeta}\mathbf{X}, a, k^\prime, t)\hm= 1\hm-
\hat{\phi}(d_f(\mathrm{F}_{k^\prime t}(\Gamma_{k^\prime}(z), \Gamma_t(z)), F_{k^\prime}(\Gamma_{k^\prime}(z)) 
\mathrm{F}_t(\Gamma_t(z)))\vert z\hm\in a) \mathrm{rnd}\,(\hat{\zeta}\mathbf{X}, d_f)$~--- статистическая достоверность 
<<зависимости>> $\Gamma_t(z)$ и~$\Gamma_{k^\prime}(z)$ по 
статистике~$d_f$ на семпле~$a$, а~$1\hm- P(d_f, 
\hat{\zeta}\mathbf{X}, a, k^\prime, t)$ количественно оценивает зависимость.
    

При заданном способе оценки зависимости $1\hm- P(d_f, 
\hat{\zeta}\mathbf{X}, a, k^\prime, t)$ задача отбора информативных 
признаков решается посредством так называемого\linebreak  
В-ал\-го\-рит\-ма, исходно разработанного для построения оптимальных 
словарей финальных ин\-фор\-маций (чему и~соответствует литера~<<В>>)~[8]. 
\mbox{Данный} алгоритм, основанный на критерии раз\-ре\-ши\-мости по Журавлёву, 
позволяет выбирать множества финальных информаций на основе 
максимального час\-тич\-но\-го покрытия при минимуме\linebreak элементов покрытия. 
Замена мощности пересечения множеств на $1\hm- P(d_f, 
\hat{\zeta}\mathbf{X}, a, k^\prime, t)$ приведет к~тому, что  
В-ал\-го\-ритм будет выбирать минимум признаков с~максимальной 
<<информативностью>>\linebreak (наиболее информативные признаки, см.\ 
теоремы~1, 7  и~8 работы~[8]).

    Таким образом, в~рамках развиваемого формализма синтез более 
информативных синтетических~$\Gamma_{k^\prime}(x_i)$ осуществляется 
в~5~стадий: 
\begin{enumerate}[(1)]
\item определяется набор исходных (как правило, 
<<низкоинформативных>>) признаков~$\Gamma_k(x_i)$ и~таргетная 
переменная~$\Gamma_t(x_i)$;
\item вводится набор метрик~$\rho_m$, 
оценивается их релевантность $d_f(\hat{\phi}(x)\vartheta_{\mathbf{m}{\bm 
\alpha}}(\mathbf{c}_{\bm \alpha},{p})$,\linebreak 
$\hat{\phi}(x)\vartheta_{\mathbf{m}{\bm \alpha}}(\overline{\mathbf{c}}_{\bm 
\alpha}, {p}))$ для каждого класса~$\mathbf{c}_{\bm \alpha}$ 
значений $t$-й переменной и~отбираются наиболее релевантные~$\rho_m$; 
\item посредством каждой из отобранных~$\rho_m$ по\-рож\-да\-ют\-ся 
синтетические признаки~$\Gamma_{k^\prime}(x_i)$;
\item посредством 
вычислений $1\hm- P(d_f, \hat{\zeta}\mathbf{X}, a, k^\prime, t)$  
и~В-ал\-го\-рит\-ма отбирается минимальное чис\-ло признаков максимальной 
<<ин\-фор\-ма\-тив\-ности>>;
\item применяется алгоритм прогнозирования 
таргетной переменной (корректор по Жу\-рав\-лё\-ву--Ру\-да\-кову). 
\end{enumerate}

\begin{table*}\small
\begin{center}
\begin{tabular}{|l|c|c|}
\multicolumn{3}{p{140mm}}{Ранговые корреляции между экспериментальными 
и~расчетными значениями $EC_{50}$ и~других величин хемокиномного анализа: $r$~--- 
коэффициент ранговой корреляции на обучении; $r_c$~---  на контроле. Усреднение~$r$ 
и~$r_c$ проводилось по 2400~выборкам хемокиномных данных}\\
\multicolumn{3}{c}{\ }\\[-6pt]
\hline
\multicolumn{1}{|c|}{{Эксперимент}}&$r$&$r_c$\\
\hline
{\boldmath $f_{\theta_k}$}\textbf{-алгоритмы, корректор~--- нейросеть}&\boldmath{$0{,}88\pm 
0{,}15$}&\boldmath{$0{,}86\pm0{,}20$}\\
Синтетические $\Gamma_{k^\prime}(x_i)$, корректор~--- нейросеть (2~слоя)&$0{,}45\pm 
0{,}22$&$0{,}22\pm 0{,}21$\\
Синтетические $\Gamma_{k^\prime}(x_i)$, корректор~--- нейросеть 
(10~слоев)&$0{,}52\pm 0{,}25$&$0{,}21\pm 0{,}20$\\
Синтетические $\Gamma_{k^\prime}(x_i)$, корректор~--- <<случайный лес>>, 
вариант~1&$0{,}98\pm 0{,}15$&$0{,}67\pm 0{,}31$\\
Синтетические $\Gamma_{k^\prime}(x_i)$, корректор~--- <<случайный лес>>, 
вариант~2&$0{,}99\pm 0{,}14$&$0{,}71\pm 0{,}35$\\
\textbf{Синтетические {\boldmath $\Gamma_{k^\prime}(x_i)$}, полиномные корректоры, 
вариант~1}&\boldmath{$0{,}93\pm 0{,}11$}&\boldmath{$0{,}90\pm 0{,}23$}\\
\textbf{Синтетические {\boldmath $\Gamma_{k^\prime}(x_i)$}, полиномные корректоры, 
вариант~2}&\boldmath{$0{,}95\pm0{,}08$}&\boldmath{$0{,}86\pm 0{,}27$}\\
\hline
\end{tabular}
\end{center}
\end{table*}

\section{Экспериментальная апробация }

    Формализм апробирован на комплексе задач\linebreak фармакоинформатики: 
получение количественных оценок ингибирования киназ протеома 
перспективными лекарствами (хемокиномный анализ)~[9]. Использованы 
2400~выборок данных <<\mbox{мо\-ле\-ку\-ла}--свой\-ст\-во>> из ProteomicsDB; 
свойства молекул включили константы $EC_{50}$ и~активности для 
концентраций~$(E_j(C_i))$.

     Исходные признаки $\Gamma_k(x_i)$ определялись как булевы 
инварианты над множествами $\chi$-це\-пей и~$\chi$-уз\-лов 
хемографов~$x_i$, как и~в~[9]. Таргетная $\Gamma_t(x_i)$ определялась как 
числовое значение прогнозируемого свойства. В~качестве~$\rho_m$ 
использовались функции расстояния на множествах, векторах и~э.ф.р.\ (всего 
65~функций из справочника~[2]). Классы~$\mathbf{c}_{\bm{\alpha}}$ 
определялись как квартили значений~$\Gamma_t$. Векторы элементов 
$L(T(\mathbf{X}))$ формировались из оценок $v^+_\alpha$, $v^-_\alpha$ 
и~$d_\alpha$~\cite{4-tor} для каждого~$\mathbf{c}_{\bm{\alpha}}$. 
Релевантность~$\rho_m$ по $d_f(\hat{\phi}(x),\vartheta_{\mathbf{m}{\bm 
\alpha}}(\mathbf{c}_{\bm{\alpha}},{p}), 
\hat{\phi}(x)\vartheta_{\mathbf{m}{\bm \alpha}} 
(\overline{\mathbf{c}}_{\bm{\alpha}}, {p}))$ оценивалась для 
каждого~$\mathbf{c}_{\bm{\alpha}}$, $d_f$~--- максимальное уклонение. 
Синтетические признаки~$\Gamma_{k^\prime}(x_i)$ по\-рож\-да\-лись всеми 
перечисленными выше способами; их отбор проводился В-ал\-го\-рит\-мом 
с~использованием $1\hm- {P}(d_f, \hat{\zeta}\mathbf{X}, a, 
     k^\prime, t)$. 
     
     В качестве корректоров использовались нейронные сети с~несколькими 
слоями (от~2 до~10) с~функцией активации softmax, полиномы различных 
конструкций (более 20~формул, в~том числе квазиполиномные модели 
с~элементарными функциями) и~<<случайные леса>> решающих деревьев. 
Оператор семплирования~$\hat{\zeta}$ был реализован как десятикратная  
кросс-ва\-ли\-да\-ция с~делением каждой выборки объектов на 80\% 
(обучение) и~20\% (конт\-роль). Результаты экспериментов суммированы 
в~таблице.
     

     
     Наилучший результат применения нового <<топологического>> 
формализма с~полиномным корректором ($r_c\hm=0{,}90\hm\pm0{,}23$) 
немного превзошел наилучший результат применения \mbox{метода} опорных 
функций (композиций вида $f_{\theta_k} \hm= g(f_1(\sum \omega_k^j x_k), 
\ldots\linebreak \ldots , f_l(\sum \omega_k^j x_k))$, см.~[9]), для которого 
$r_c\hm=0{,}86\hm\pm0{,}20$. Полиномными формулами, наиболее часто 
показывавшими наилучший результат, оказались полиномы 1-й или 2-й 
степеней с~произведениями переменных первой степени, полиномы 5-й 
степени, квазиполиномы 5-й степени с~сигмоидами и~Фурье-по\-ли\-но\-мы  
3-й степени.
     
     Нейросетевые корректоры всех использованных конфигураций 
отличались крайне низкими показателями ($r\hm=0{,}45\hm\pm0{,}22$, 
$r_c\hm=0{,}22\hm\pm0{,}21$), а~<<случайный лес>> приводил 
к~существенному переобучению (см.\ таб\-ли\-цу). При этом в~290 
из~2400~выборок данных (12\%) <<случайный лес>> приводил к~улучшению 
результатов по сравнению с~наилучшими полиномными корректорами, 
а~в~1670 из 2400~выборок данных (70\%)~--- к~ухудшению.
     
     
     Анализ синтетических признаков $\Gamma_{k^\prime}(x_i)$, 
вошедших в~наилучшие полиномные модели, показал, что среди более 
информативных (по оценке $1\hm- P(d_f, \hat{\zeta}\mathbf{X},  
a, k^\prime, t)$) признаков чаще всего встречались признаки, порождаемые 
с~использованием э.ф.р.\ на основе опорных цепей (теорема~1 в~1-й части 
работы~[1]), среди наименее информативных~--- исходные признаки 
$\Gamma_k(x_i)$ и~признаки на основе отдельных расстояний 
$\rho_m(\mathbf{c}_{\bm{\alpha}} , \Gamma_k^{-1}(\Gamma_k(x_i))$. 
Функциями~$\rho_m$, наиболее часто порождающими информативные 
$\Gamma_{k^\prime}(x_i)$ на пространстве э.ф.р., оказались максимальное 
уклонение Колмогорова, <<косое>> расстояние, метрики $\mathrm{Lp}$, 
Реньи, $\chi2$, фон Мизеса, инженерная~\cite{2-tor}. В~среднем по всем 
выборкам данных эти~7~разновидностей~$\rho_m$ порождали более 50\% 
самых информативных признаков~$\Gamma_{k^\prime}(x_i)$, отобранных  
В-ал\-го\-рит\-мом.

\vspace*{-6pt}

\section{Заключение}

\vspace*{-2pt}

    Предлагаемый подход к~порождению информативных синтетических 
признаков подразумевает последовательные трансформации описаний 
объекта:\\[-13pt]
\begin{enumerate}[(1)]
\item исходное множество значений признаков;\\[-13.5pt]
\item множество 
соответствующих элементов решетки;\\[-13.5pt] 
\item ~множество расстояний 
(измеряемых посредством~$\rho_m$) между элементами решетки, 
соответствующими классам и~признакам;\\[-13.5pt]
\item множество э.ф.р.\ расстояний, 
измеренных заданными~$\rho_m$;\\[-13.5pt] 
\item множество синтетических признаков 
объ-\linebreak екта.
\end{enumerate}

\noindent
 Использование многочисленных метрик на стадии порождения 
признаков позволяет рассматривать развиваемый формализм как вариант 
развития идеологии АВО (алгоритмы вычисления \mbox{оценок}) научной школы 
Ю.\,И.~Журавлёва. Экспериментальная апробация предлагаемого подхода на 
2400~однородных задачах фармакоинформатики позволила повысить 
аккуратность и~обобщающую способность алгоритмов. 


{\small\frenchspacing
 {\baselineskip=10.6pt
 %\addcontentsline{toc}{section}{References}
 \begin{thebibliography}{99}
  
  \bibitem{1-tor}
\Au{Торшин И.\,Ю.} О~порождении синтетических признаков на основе опорных цепей 
и~произвольных метрик в~рамках топологического подхода к~анализу данных. Часть~1. 
Включение в~формализм эмпирических функций расстояния~// Информатика и~её 
применения, 2024. Т.~18. Вып.~1. С.~71--77. doi: 10.14357/19922264240110. EDN: 
RIVOXR.
  \bibitem{2-tor}
  \Au{Деза Е.\,И., Деза~М.\,М.} Энциклопедический словарь расстояний~/ Пер. с~англ.~--- М.: Наука, 
2008. 444~с. (\Au{Deza~E.\,I., Deza~M.\,M.} {Dictionary of distances}.~--- North-Holland: 
Elsevier, 2006. 412~p. doi: 10.1016/B978-0-444-52087-6.X5000-8.)
  \bibitem{3-tor}
  \Au{Torshin I.\,Y., Rudakov~K.\,V.} Combinatorial analysis of the solvability properties of 
the problems of recognition and completeness of algorithmic models. Part~2: Metric approach 
within the framework of the theory of classification of feature values~// Pattern Recognition Image 
Analysis, 2017. Vol.~27. No.\,2. P.~184--199. doi: 10.1134/S1054661817020110.
  \bibitem{4-tor}
\Au{Торшин И.\,Ю.} О~формировании множеств прецедентов на основе таблиц 
разнородных признаковых описаний методами топологической теории анализа данных~// 
Информатика и~её применения, 2023. Т.~17. Вып.~3. С.~2--7. doi: 
10.14357/19922264230301. EDN: AQEUYO.
  \bibitem{5-tor}
  \Au{Torshin I.\,Yu., Rudakov~K.\,V.} On the procedures of generation of numerical features 
over partitions of sets of objects in the problem of predicting numerical target variables~// 
Pattern Recognition Image Analysis, 2019. Vol.~29. No.\,4. P.~654--667. doi: 
10.1134/S1054661819040175. 
  \bibitem{6-tor}
  \Au{Torshin I.\,Y., Rudakov~K.\,V.} Combinatorial analysis of the solvability properties of 
the problems of recognition and completeness of algorithmic models. Part~1: Factorization 
approach~// Pattern Recognition Image Analysis, 2017. Vol.~27. No.\,1. P.~16--28. doi: 
10.1134/S1054661817010151.
  \bibitem{7-tor}
  \Au{Sosa-Cabrera G., G$\acute{\mbox{o}}$mez-Guerrero~S.,  
Garc$\acute{\iota}$a-Torres~M., Schaerer~C.\,E.} Feature selection: A~perspective on inter-attribute 
cooperation~// Int. J. Data Science Analytics, 2024. Vol.~17. P.~139--151. doi:  
10.1007/s41060-023-00439-z.
  \bibitem{8-tor}
  \Au{Torshin I.\,Y.} Optimal dictionaries of the final information on the basis of the solvability 
criterion and their applications in bioinformatics~// Pattern Recognition Image Analysis, 2013. 
Vol.~23. No.\,2. P.~319--327. doi: 10.1134/S1054661813020156.
  \bibitem{9-tor}
\Au{Торшин И.\,Ю.} О~задачах оптимизации, воз\-ни\-ка\-ющих при применении 
топологического анализа данных к~поиску алгоритмов прогнозирования 
с~фиксированными корректорами~// Информатика и~её применения, 2023. Т.~17. Вып.~2. 
С.~2--10. doi: 10.14357/19922264230201. EDN: IGSPEW.

\end{thebibliography}

 }
 }

\end{multicols}

\vspace*{-8pt}

\hfill{\small\textit{Поступила в~редакцию 09.04.24}}

\vspace*{6pt}

%\pagebreak

%\newpage

%\vspace*{-28pt}

\hrule

\vspace*{2pt}

\hrule



\def\tit{ON THE GENERATION OF~SYNTHETIC FEATURES BASED~ON~SUPPORT~CHAINS 
AND~ARBITRARY METRICS\\ WITHIN THE~FRAMEWORK OF~A~TOPOLOGICAL 
APPROACH\\ TO~DATA ANALYSIS. PART~2. EXPERIMENTAL TESTING 
ON~PHARMACOINFORMATICS PROBLEMS}


\def\titkol{On the generation of~synthetic features based on~support chains 
and~arbitrary metrics} % within the~framework of~a~topological  approach to~data analysis. Part~2. Experimental testing  on~pharmacoinformatics problems}


\def\aut{I.\,Yu.~Torshin}

\def\autkol{I.\,Yu.~Torshin}

\titel{\tit}{\aut}{\autkol}{\titkol}

\vspace*{-15pt}


\noindent
Federal Research Center ``Computer Science and Control'' of the Russian Academy of 
Sciences, 44-2~Vavilov Str., Moscow 119333, Russian Federation

\def\leftfootline{\small{\textbf{\thepage}
\hfill INFORMATIKA I EE PRIMENENIYA~--- INFORMATICS AND
APPLICATIONS\ \ \ 2024\ \ \ volume~18\ \ \ issue\ 2}
}%
 \def\rightfootline{\small{INFORMATIKA I EE PRIMENENIYA~---
INFORMATICS AND APPLICATIONS\ \ \ 2024\ \ \ volume~18\ \ \ issue\ 2
\hfill \textbf{\thepage}}}

\vspace*{3pt}
  
  


\Abste{Consideration of precedent relationships between features and a target variable in the 
form of sets of Boolean lattice elements indicates the possibility of generating synthetic features 
using metric distance functions. Approaches to ($i$)~assessing the relevance (``informativeness'') 
of metrics in relation to the problems being solved; ($ii$)~generating; and ($iii$)~selecting synthetic 
features that are more informative than the original feature descriptions are formulated. The 
results of topological analysis of~2400~samples of ``molecule--property'' data
from 
ProteomicsDB made it possible to obtain fairly effective algorithms for 
predicting the properties of molecules (rank correlation in cross-validation is~$0.90\pm 0.23$). 
Using this sample of problems, metrics have been established\linebreak\vspace*{-12pt}}

\Abstend{that most often generate 
informative synthetic features: maximum Kolmogorov deviation, ``oblique'' distance, and Lp, Renyi, 
and von Mises metrics. To solve the studied set of problems, the advantage of polynomial 
correctors compared to neural network and random forest correctors is shown.}

\KWE{topological data analysis; lattice theory; algebraic approach of Yu.\,I.~Zhuravlev; 
pharmacoinformatics}




\DOI{10.14357/19922264240207}{OTXCUD}

%\vspace*{-12pt}

\Ack

\vspace*{-3pt}


\noindent
The research was funded by the Russian Science Foundation, project No.\,23-21-00154. The 
research was carried out using the infrastructure of the Shared Research Facilities ``High 
Performance Computing and Big Data'' (CKP ``Informatics'') of FRC CSC RAS (Moscow).
 


  \begin{multicols}{2}

\renewcommand{\bibname}{\protect\rmfamily References}
%\renewcommand{\bibname}{\large\protect\rm References}

{\small\frenchspacing
 {%\baselineskip=10.8pt
 \addcontentsline{toc}{section}{References}
 \begin{thebibliography}{9} 
 
 %\vspace*{-3pt}
  \bibitem{1-tor-1}
\Aue{Torshin, I.\,Yu.} 2024. O~porozhdenii sinteticheskikh priznakov na osno\-ve opor\-nykh 
tsepey i~proizvol'nykh metrik v~ram\-kakh topologicheskogo podkhoda k~analizu dannykh. 
Chast'~1. Vklyuchenie v~formalizm empiricheskikh funktsiy rasstoyaniya [On the generation 
of synthetic features based on support chains and arbitrary metrics within a~topological approach 
to data analysis. Part~1. Inclusion of empirical distance functions into the formalism]. 
\textit{Informatika i~ee Primeneniya~--- Inform Appl.} 18(1):71--77. doi: 
10.14357/19922264240110. EDN: RIVOXR.
  \bibitem{2-tor-1}
\Aue{Deza, E.\,I., and M.\,M.~Deza.} 2006. \textit{Dictionary of distances}. North-Holland: 
Elsevier. 412~p. doi: 10.1016/B978-0-444-52087-6.X5000-8.
  \bibitem{3-tor-1}
\Aue{Torshin, I.\,Yu., and K.\,V.~Rudakov.} 2017. Combinatorial analysis of the solvability 
properties of the problems of recognition and completeness of algorithmic models. Part~2: 
Metric approach within the framework of the theory of classification of feature values. 
\textit{Pattern Recognition Image Analysis} 27(2):184--199. doi: 10.1134/S1054661817020110.
  \bibitem{4-tor-1}
\Aue{Torshin, I.\,Yu.} 2023. O~formirovanii mnozhestv pretsedentov na osnove tablits 
raznorodnykh priznakovykh opisaniy metodami topologicheskoy teorii analiza dannykh [On the 
formation of sets of precedents based on tables of heterogeneous feature descriptions by methods 
of topological theory of data analysis]. \textit{Informatika i~ee Primeneniya~--- Inform Appl.} 
17(3):2--7. doi: 10.14357/19922264230301. EDN: AQEUYO.
  \bibitem{5-tor-1}
\Aue{Torshin, I.\,Yu., and K.\,V.~Rudakov.} 2019. On the procedures of generation of 
numerical features over partitions of sets of objects in the problem of predicting numerical target 
variables. \textit{Pattern Recognition Image Analysis} 29(4):654--667. doi: 
10.1134/S1054661819040175.
  \bibitem{6-tor-1}
\Aue{Torshin, I.\,Y., and K.\,V.~Rudakov.} 2017. Combinatorial analysis of the solvability of 
the problems of recognition, completeness of algorithmic models. Part~1: Factorization 
approach. \textit{Pattern Recognition Image Analysis} 27(1):16--28. doi: 
10.1134/S1054661817010151.
  \bibitem{7-tor-1}
\Aue{Sosa-Cabrera, G., S.~Gуmez-Guerrero, \mbox{M.~Garc$\acute{\!\mbox{{\ptb{\i}}}}$a}-Torres, 
and C.\,E.~Schaerer.} 2024. Feature selection: A~perspective on inter-attribute cooperation. \textit{Int. J. 
Data Science Analytics} 17:139--151. doi: 10.1007/s41060-023-00439-z.
  \bibitem{8-tor-1}
\Aue{Torshin, I.\,Y.} 2013. Optimal dictionaries of the final information on the basis of the 
solvability criterion and their applications in bioinformatics. \textit{Pattern Recognition Image 
Analysis}  23(2):319--327. doi: 10.1134/ S1054661813020156.
  \bibitem{9-tor-1}
\Aue{Torshin, I.\,Yu.} 2023. O~zadachakh optimizatsii, voznikayushchikh pri primenenii 
topologicheskogo analiza dannykh k~poisku algoritmov prognozirovaniya s~fiksirovannymi 
korrektorami [On optimization problems arising from the application of topological data analysis 
to the search for forecasting algorithms with fixed correctors]. \textit{Informatika i~ee 
Primeneniya~--- Inform Appl.} 17(2):2--10. doi: 10.14357/19922264230201. EDN: IGSPEW.

\end{thebibliography}

 }
 }

\end{multicols}

\vspace*{-6pt}

\hfill{\small\textit{Received April 9, 2024}} 

\vspace*{-12pt}


\Contrl

\vspace*{-3pt}

\noindent
\textbf{Torshin Ivan Y.} (b.\ 1972)~--- Candidate of Science (PhD) in physics and mathematics, 
Candidate of Science (PhD) in chemistry, leading scientist, Federal Research Center ``Computer 
Science and Control'' of the Russian Academy of Sciences, 44-2~Vavilov Str, Moscow 119333, 
Russian Federation; \mbox{tiy135@yahoo.com}
  
  



\label{end\stat}

\renewcommand{\bibname}{\protect\rm Литература}        %1
\def\stat{kovalev}

\def\tit{МЕТОДЫ ТЕОРИИ КАТЕГОРИЙ В~МОДЕЛЬНО-ОРИЕНТИРОВАННОЙ СИСТЕМНОЙ 
ИНЖЕНЕРИИ}

\def\titkol{Методы теории категорий в~модельно-ориентированной системной 
инженерии}

\def\aut{С.\,П.~Ковалёв$^1$}

\def\autkol{С.\,П.~Ковалёв}

\titel{\tit}{\aut}{\autkol}{\titkol}

\index{Ковалёв С.\,П.}
\index{Kovalyov S.\,P.}


%{\renewcommand{\thefootnote}{\fnsymbol{footnote}} \footnotetext[1]
%{Исследование выполнено при финансовой поддержке Российского научного фонда (проект 16-11-10227).}}


\renewcommand{\thefootnote}{\arabic{footnote}}
\footnotetext[1]{Институт проблем управления им.\ В.\,А.~Трапезникова 
Российской академии наук,  \mbox{kovalyov@nm.ru}}

%\vspace*{-18pt}

\Abst{Предложен математический аппарат на базе теории категорий, который позволяет 
формально описывать и~строго исследовать процедуры применения моделей в~инженерной 
деятельности, составляющие сущность мо\-дель\-но-ори\-ен\-ти\-ро\-ван\-ной системной 
инженерии (Model-Based Systems Engineering, MBSE). В~основе аппарата лежит 
математическое представление сборочных чертежей (мегамоделей сис\-тем) диаграммами 
в~категориях, объектами которых служат модели, а~морфизмы представляют действия по 
сборке моделей сис\-тем из моделей компонентов. Адекватность аппарата обоснована исходя 
из требований стандартов, регламентирующих описание структуры систем, в~том числе 
IEC~81346. Предложены и~исследованы тео\-ре\-ти\-ко-ка\-те\-гор\-ные методы решения ряда 
практических задач сборки систем. Приведены примеры решения таких задач в~категориях, 
представляющих две ключевые области применения MBSE: гео\-мет\-ри\-че\-ское моделирование 
изделий сложной формы и~дис\-крет\-но-со\-бы\-тий\-ное имитационное моделирование 
поведения технических систем.}

\KW{модельно-ориентированная системная инженерия; мегамодель; теория категорий; 
копредел}



\DOI{10.14357/19922264170305} 


\vspace*{6pt}

\vskip 10pt plus 9pt minus 6pt

\thispagestyle{headings}

\begin{multicols}{2}

\label{st\stat}

\section{Введение}

   Модельно-ориентированная системная инженерия состоит в~формализованном применении моделирования в~
поддержке жизненного цикла сис\-тем, включая сбор требований, 
проектирование, проверку и~приемку, другие стадии~[1]. Модели, 
разрабатываемые в~ходе процедур MBSE, пригодны к~автоматической 
обработке на компьютерах. Это позволяет сначала задавать, верифицировать 
и~оптимизировать проектные решения на моделях <<в циф\-ре~и только потом 
воплощать <<в железе>>, снижая затраты на организацию жизненного цикла 
изделий и~сокращая сроки выполнения работ~[2].
   
   И все же внедрение технологий MBSE в~инженерную деятельность 
происходит медленно. Это связано во многом с~нехваткой единой 
концептуальной базы инженерного моделирования: предлагается много 
частных языков и~технологий, слабо совместимых друг с~другом и~плохо 
приспособленных для совместной разработки моделей большими 
мультидисциплинарными коллективами~[3]. Тем самым затрудняется переход 
от набора электронных чертежей к~полноценному электронно-цифровому 
макету (digital mock-up) промышленного изделия.
   
   Естественный, хотя и~<<трудный>>, подход к~получению результатов 
общего характера, унифи\-ци\-ру\-ющих разнородные технологии, состоит в~том, 
чтобы как можно более строго формализовать процедуры моделирования. 
Формализация позволит совершенствовать процедуры MBSE и~передавать их 
на исполнение компьютеру без пробелов и~искажений. Самый высокий уровень 
строгости достигается при привлечении математического аппарата, поскольку 
математика позволяет надежно доказывать или опровергать утверждения, 
ха\-рак\-те\-ри\-зу\-ющие корректность и~эффективность процедур.
   
   В настоящей работе предложен аппарат, основанный на математическом 
представлении сборочных чертежей (<<мегамоделей>> систем) 
ориенти-\linebreak рованными графами (диаграммами). Узлы такого\linebreak графа помечаются 
обозначениями моделей час\-тей, а~реб\-ра помечаются обозначениями действий\linebreak 
(activities), посредством которых части собираются в~систему. Представление 
структуры систем графами регламентируется, в~частности, стандартом 
IEC~81346~[4]. Естественным источником математических методов 
конструирования и~анализа мегамоделей служит теория категорий (см., 
например,~[5, 6]). Модели рассматриваются как объекты подходящих 
категорий, а~действия формально описываются морфизмами. Строятся 
и~исследу-\linebreak ются тео\-ре\-ти\-ко-ка\-те\-гор\-ные конструкции, опи\-сы\-ва\-ющие процедуры 
MBSE на абстрактном кон-\linebreak цептуальном уровне. Определенный опыт такого\linebreak 
исследования был накоплен в~инженерии программного обеспечения~[7] 
и~теперь может быть обобщен для системной инженерии в~целом. Например, 
сборке системы согласно некоторой мегамодели отвечает построение 
копредела диаграммы~--- универсальной конструкции~\cite{5-kov}.
   
   Статья построена следующим образом. В~разд.~2 приведен обзор 
принципов описания структуры сис\-тем согласно стандарту IEC~81346. 
Раздел~3 посвящен практическим проб\-ле\-мам мегамоделирования и~сборке 
сис\-тем. В~разд.~4 вводятся конструкции тео\-рии категорий, позволяющие 
формально решать задачи мегамоделирования. В~заключении приводятся 
выводы и~намечаются направления дальнейших исследований.

\section{Структура систем и~стандарт~IEC~81346}

   Важной проблемой MBSE, отмеченной во введении, является слабая 
совместимость языков и~инструмен\-тов моделирования от разных поставщиков. 
Основным подходом к~достижению совместимости является стандартизация~--- 
принятие обязывающих документов, устанавливающих требования и~принципы 
взаимозаменяемости инструментов. Многие стандарты определяют конкретные 
форматы машиночитаемой записи моделей, нейтральные относительно 
разработчиков инструментов MBSE. Примером служит формат описания 
твердотельных геометрических моделей STEP, стандартизованный семейством 
ISO~10303. Однако для формализации MBSE в~целом интерес представляют 
в~первую очередь стандарты более общего плана, унифицирующие принципы 
и~методы применения моделей в~жизненном цикле систем независимо от 
способа записи моделей. С~этой точки зрения внимания заслуживает 
международный стандарт IEC 81346-1:2009 <<Промышленные системы, 
установки и~обору\-до\-ва\-ние~--- принципы структурирования и~ссылочные 
обозначения~--- часть~1: основные правила>> (<<Industrial Systems, 
Installations and Equipment and Industrial Products~--- Structuring Principles and 
Reference Designations~--- Part~1: Basic Rules>>)~\cite{4-kov}. Стандарт не 
принят в~России, однако ряду его положений в~области структуры систем 
соответствует российский ГОСТ~2.053-2013 <<ЕСКД. Электронная структура 
изделия. Общие положения>>.
   
   В стандарте IEC~81346 рассматривается ряд вопросов моделирования 
структуры систем и~идентификации отдельных единиц в~составе систем. 
Системная единица названа в~стандарте объектом, причем принципиально не 
проводится различие между объектами реального мира, составляющими 
реально существующие системы, и~объектами мыслительной деятельности~--- 
моделями единиц, составляющими модели систем. Таким образом, стандарт 
выходит за рамки MBSE и~рассматривает ряд вопросов системной инженерии 
вообще. Иерар\-хи\-че\-ская структура системы (холархия~\cite{3-kov}) 
изображается деревом, узлы которого помечены обозначениями объектов. 
Важным достижением стандарта является выявление того факта, что одна и~та 
же система задается не одной, а несколькими в~общем случае различными 
иерархическими структурами, возникающими в~результате декомпозиции 
согласно различным принципам (аспектам). В~их числе:
   \begin{itemize}
\item функциональная (function-oriented) структура, отвечающая разделению 
системных единиц по выполняемым ими функциям в~составе сис\-темы;
\item продуктовая (product-oriented), или модульная, структура, отражающая 
сборочную (технологическую) конфигурацию сис\-темы;
\item структура размещения (location-oriented), в~соответствии с~которой 
единицы располагаются в~физическом пространстве.
\end{itemize}

   Ясно, что один и~тот же объект может входить в~несколько структур и~при 
этом находиться на различных уровнях. В~то же время в~некоторых аспектах 
объект может никак не проявлять себя и~вследствие этого отсутствовать 
в~соответствующих структурах. Полное идентифицирующее ссылочное 
обозначение объекта (reference designation) конструируется путем 
последовательного перечисления всех объектов, находящихся на пути от корня 
дерева рассматриваемой структуры до дан\-ного объекта включительно. 
Наименование каж\-до\-го объекта в~этом перечислении составляется из 
символьного обозначения аспекта, буквенного обозначения класса (типа), 
к~которому относится  объект, и~порядкового номера объекта среди 
экземпляров своего класса. Таким путем обеспечивается\linebreak  уникальность 
наименования любой единицы\linebreak
 в~пределах системы. Например, функциональная 
структура обозначается символом <<=>>, а~функциональный класс 
переключателей потоков ресурсов обозначается буквами QA, так что первая по 
порядку единица, выполняющая функцию переключения, называется =QA1, 
а~ее полное ссылочное обозначение может выглядеть как =WP1=WC1=QA1. 
Если объект присутствует в~нескольких структурах, то он может иметь 
несколько ссылочных обозначений, как показано на рис.~1~\cite{4-kov}.

\begin{figure*} %fig1
    \vspace*{1pt}
\begin{center}
\mbox{%
\epsfxsize=165mm
\epsfbox{kov-1.eps}
}
\end{center}
\vspace*{-9pt}
\Caption{Пример ссылочных обозначений структурных единиц системы}
\vspace*{9pt}
\end{figure*}

   С~точки зрения практики системной инженерии большой интерес 
представляет описание эволюции структурного представления системы по ходу 
жизненного цикла, приведенное в~приложении~B к~стандарту IEC~81346. 
<<Строительный материал>> для структур имеет вид (виртуального) 
справочника или каталога объектов, из которого выбираются объекты для 
включения в~структуру. 

В~начале жизненного цикла системы на основе 
исходных требований к~ней конструктор строит ее функциональную структуру. 
Затем определяется пространственное положение функциональных объектов, 
в~результате чего создается структура размещения. На следующей стадии 
формируются закупочные спецификации, образующие продуктовую структуру. 
В~ходе последующих стадий жизненного цикла эти структуры могут 
трансформироваться. На каждой стадии могут происходить замена, слияние 
и~расщепление объектов. Таким образом, объекты разных структур системы 
связаны отношением вида <<многие ко многим>>, вдоль которого 
прослеживаются (трассируются) исходные требования.
   
   В то же время стандарт не предусматривает указа\-ние способов, какими 
объекты собраны в~сис\-те\-мы. Поэтому структуру сис\-те\-мы можно рас\-смат\-ри\-вать 
как эскизный проект, в~котором отражены лишь факты вхождения системных 
единиц более низкого уровня иерархии в~единицы более высокого уровня. 


Проект такого рода поступает на вход технологу, который определяет 
конкретные операции сборки каждой единицы каждого уровня иерархии. При 
необходимости технолог вносит изменения в~конструкцию объектов (такие как 
нарезка резьбы) и~добавляет связующие интерфейсные объекты (такие как 
клей, трансформатор и~др.). В~результате для каждого составного объекта 
формируется сборочный чертеж, на котором указаны все со\-став\-ля\-ющие 
объекты и~действия по их соединению в~целях получения сис\-те\-мы. 
Технологическая проработка требуется на всех стадиях жизненного цикла, на 
которых формируется либо изменяется ка\-кая-ли\-бо из структур системы.

%\vspace*{-6pt}

\section{Мегамоделирование и~сборка~систем}

   В MBSE объекты, образующие 
структуры\linebreak
 сис\-тем, описываются формализованными ком\-пьютерными моделями 
различных видов: геометрическими фигурами и~телами, численными 
аппроксимациями дифференциальных уравнений, оснащенными графами и~
т.\,д. При этом, как свидетель\-ст\-ву\-ют стандарты типа IEC~81346, для анализа 
структуры систем и~организации сборки необходимо знать не столько 
внутреннюю структуру моделей, сколько ассортимент их возможностей 
соединяться с~другими моделями в~целях формирования моделей составных 
объектов. Иными словами, модели рассматриваются как <<черные ящики>> 
с~известным поведением по отношению к~другим моделям. Каталог объектов, 
упоминавшийся в~предыду\-щем разделе, в~условиях применения \mbox{MBSE} 
составляется из моделей и~описаний действий по их соединению.
   
   Структуры систем и~сборочные чертежи представляют собой частные 
случаи мегамоделей (mega\-mod\-el)~--- моделей, состоящих из моделей и~связей 
между ними~\cite{8-kov}. Мегамодель, в~которой связи описывают соединение 
моделей, образующих некоторую сис\-те\-му, называется конфигурацией этой 
сис\-те\-мы~\cite{5-kov}. Существуют и~другие виды мегамоделей, 
предназначенные для описания других процедур \mbox{MBSE}, таких как 
формирование модели согласно заданной метамодели  
(instantiating)~\cite{9-kov}. Но в~настоящей работе сосредоточимся на 
конфигурациях и~сборке систем.
   
   Например, в~моделировании механических сис\-тем, состоящих из твердых 
тел, моделями деталей и~сборочных единиц служат геометрические тела, 
которые могут быть представлены для компьютерной обработки различными 
способами: конструктивным, воксельным, граничным~\cite{10-kov}. Объекты, 
составляющие механические системы, т.\,е.\ представления экземпляров тел, 
получаются из моделей путем аффинных изометрий и~растяжений. Так, из 
набора цилиндров разных размеров составляется модель штанги (спортивного 
снаряда). В~функциональной структуре штанги по IEC~81346 цилиндры 
представлены разными объектами, поскольку они выполняют разные функции, 
хотя порождаются одной и~той же геометрической моделью. Соответственно, 
в~каталоге моделей содержится тело в~форме цилиндра, допускающее 
несколько разных действий по включению в~состав штанги.
   
   В качестве еще одного примера рассмотрим дис\-крет\-но-со\-бы\-тий\-ное 
имитационное моделирование, поддержка которого относится к~числу 
важнейших достижений MBSE~\cite{1-kov}. Здесь модель имеет вид 
сценария~--- фрагмента предполагаемой истории поведения моделируемой 
системы, пред\-став\-лен\-но\-го потоком дискретных событий различных видов. 
Некоторые события могут вызывать либо запрещать возникновение других 
событий. Описания действий по сборке сценариев поведения систем отражают 
вклад сценариев поведения составляющих. Так, сценарий работы цеха 
составляется из сценариев работы станков, связанных друг с~другом согласно 
маршрутным картам~\cite{11-kov}.
   
   Сформулируем задачу мегамоделирования сборки систем в~общем виде 
следующим образом. По мегамодели, представляющей конфигура\-цию 
некоторой системы, требуется сконструировать модель системы как целого 
и~рассчитать для нее моделируемые параметры, в~том числе эмерджентные~--- 
не присущие никакой из со\-став\-ля\-ющих единиц в~отдельности. Принцип 
конструирования модели системы легко усмотреть из организации 
структур-\linebreak\vspace*{-12pt}

\columnbreak

 { \begin{center}  %fig1
 \vspace*{1pt}
\mbox{%
\epsfxsize=57.246mm
\epsfbox{kov-2.eps}
}


\vspace*{12pt}


\noindent
{{\figurename~2}\ \ \small{Схема склеивания}}
\end{center}
}

\vspace*{18pt}

\addtocounter{figure}{1}

\noindent
ного представления: система должна находиться на иерархическом 
уровне, располагающемся непосредственно над уровнем со\-став\-ля\-ющих ее 
объектов. Иными словами, модель системы должна включать в~себя модели 
всех составляющих с~учетом их конфигурационных связей и~в~то же время 
включаться в~любые модели, включающие в~себя модели всех составляющих 
конфигурации.
   
   Поясним этот принцип на простом примере. Предположим, что нужно 
объединить в~систему два объекта~$P$ и~$S$ и~что технолог решил сделать это 
с~по\-мощью клея~--- третьего объекта~$G$, который может быть соединен 
и~с~$P$, и~с~$S$. Действие клея описывается конфигурацией следующего 
вида: объекты~$G$ и~$P$ порождают в~результате соединения известный 
промежуточный комплексный объект~$P_G$, содержащий их, а~объекты~$G$ 
и~$S$ порождают объект~$S_G$. Система~$R$, полученная путем 
склеивания~$P$ с~$S$ при помощи~$G$, отбирается среди объектов, 
содержащих~$P_G$ и~$S_G$, по следующему структурному критерию: 
объект~$R$ должен содержаться в~любом объекте~$T$, содержащем~$P_G$ 
и~$S_G$. Схематически этот критерий изображен на рис.~2.


   Если объект $R$, удовлетворяющий указанному структурному критерию, 
существует, то он действительно отвечает системе, которая собрана из~$S$ 
и~$P$ путем склеивания посредством~$G$ (и~не содержит ничего 
<<лишнего>>). Более того, легко видеть, что такой объект~$R$ определяется, 
по существу, однозначно в~том смысле, что любые два объекта~$R$ 
и~$R^\prime$, удовлетворяющие структурному критерию, содержатся друг 
в~друге. Если же нужного объекта~$R$ не существует, то делается вывод, что 
технолог ошибся: клей~$G$ не способен соединить объекты~$P$ и~$S$.
   
   В структурное представление, выполненное по стандарту IEC~81346 либо по 
ГОСТу 2.053-2013, входят только объекты~$P$, $S$ и~$R$ и~две композитные 
стрелки: $P\hm\to R$, проходящая через~$P_G$, и~$S\hm\to R$, проходящая 
через~$S_G$ (так что мегамодель склеивания~--- это часть схемы, ограниченная 
треугольником~$PSR$). Кроме того, стрелки на схеме склеивания, в~отличие от 
структуры, представляют не просто факты включения объектов друг в~друга, 
а~конкретные действия по их соединению. При этом соблюдается следующее 
естественное условие структурной корректности: если из одного объекта 
можно прийти в~другой разными путями по схеме, то эти пути задают одно и~то 
же композитное действие. Например, клей~$G$ включается в~состав 
системы~$R$ единственным способом, несмотря на наличие двух путей $G 
\hm\to  P_G \hm\to R$ и~$G \hm\to S_G \hm\to R$: в~действительности не имеет 
значения, через какой промежуточный объект <<прослеживается>> включение 
клея в~систему. Таким образом, мегамодель сборки содержит больше 
информации, чем иерархическая структура системы.
   
   Если модели содержат значения тех или иных параметров, а описание 
действий по их соединению позволяет выявить правила преобразования 
значений, то по мегамодели сборки можно вы\-чис\-лить значения параметров для 
системы. Известны примеры вычислений такого рода в~области разработки 
новых композиционных материалов~\cite{12-kov}. Осредненные 
(эффективные) физические характеристики композитов, такие как модуль Юнга и~коэффициент Пуассона, сложным образом зависят от характеристик 
компонентов и~способов изготовления композита из них. При помощи методов 
теории упру\-гости эти зависимости задаются в~форме линеаризованных 
матричных соотношений, которые приписываются к~стрелкам мегамоделей, 
пред\-став\-ля\-ющим включение компонентов в~композиты. Появляется 
возможность рассчитывать на компьютере свойства композитов по базе данных 
компонентов, без проведения дорогостоящих физических экспериментов.
   
   В заключение раздела отметим, что хотя прямой расчет системы по 
конфигурации имеет большое значение, в~MBSE он играет вспомогательную 
роль. Согласно стандарту IEC~81346 и~практикам системной инженерии, 
система обычно проектируется сверху вниз~--- от корня структурной иерархии 
к~составляющим~\cite{13-kov}. Это означает, что технолог в~основном решает 
не прямую, а~обратную задачу: модель системы, которую нужно собрать, 
известна, а~нужно построить (восстановить) конфигурацию, из которой такая 
система может быть получена путем сборки, с~учетом различных ограничений. 
Формальные математические постановки и~методы решения обратных задач 
мегамоделирования представляют собой крупную перспективную тему 
исследований, выходящую за рамки настоящей статьи.

\section{Теория категорий в~мегамоделировании}

   Как указывалось во введении, естественным источни\-ком математических 
методов кон\-стру\-ирова\-ния и~анализа мегамоделей служит теория категорий. 
Категорией называется коллекция абстрактных объектов, попарно связанных 
морфизмами (стрелками). Точное определение занимает буквально несколько 
строк~\cite{14-kov}: категория~$C$ состоит из совокупности 
объектов~$\mathrm{Ob}\,C$ и~совокупности морфизмов~$\mathrm{Mor}\,C$, 
на которых заданы следующие операции:
\begin{enumerate}[(1)]
\item каждому морфизму~$f$ 
сопоставляется два объекта: область $\mathrm{dom}\,f$ и~кообласть 
$\mathrm{codom}\,f$ (соотношения вида $\mathrm{dom}\,f \hm= A$ и~
$\mathrm{codom}\,f \hm= B$ наглядно записываются в~форме стрелки~$f$: 
$A\hm\to B$, а множество всех морфизмов, удовлетворяющих этим 
соотношениям, обозначается через $\mathrm{Mor}(A, B))$;
\item для 
любой пары морфизмов~$f, g$, удовлетворяющей условию 
$\mathrm{codom}\,f\hm = \mathrm{dom}\,g$, определена композиция~--- 
морфизм $g \circ f : \mathrm{dom}\,f \hm\to  \mathrm{codom}\,g$, причем она 
ассоциативна: для любой тройки морфизмов~$f, g, h$, удовлетворяющей 
условиям $\mathrm{codom}\,f \hm= \mathrm{dom}\,g$ и~$\mathrm{codom}\,g 
\hm= \mathrm{dom}\,h$, выполняется соотношение $h \circ (g \circ f) \hm= (h 
\circ g) \circ f$;
\item любой объект~$A$ обладает тождественным 
морфизмом~$1_A : A \to A$ таким, что для любого морфизма~$f : A\hm\to B$ 
выполняется соотношение $f \circ 1_A \hm= 1_B \circ  f \hm= f$.
\end{enumerate}

Классическим 
примером категории служит $\mathbf{Set}$, состоящая из всех множеств и~всех 
их отображений: закон композиции отображений задается стандартной 
подстановкой, а тождественным морфизмом произвольного множества служит 
его тождественное отображение на себя.
   
   Вместе с~категорией вводится понятие функтора~--- отображения категорий, 
сохраняющего структуру. Функтор $\mathrm{fun}\,: C \hm\to D$, действующий из 
категории~$C$ в~$D$,~--- это пара одноименных отображений $\mathrm{fun}\,: 
\mathrm{Ob}\,C \hm\to \mathrm{Ob}\,D$, $\mathrm{fun}\,: \mathrm{Mor}\,C \hm\to 
\mathrm{Mor}\,D$, удовлетворяющая следующим условиям (для произвольных 
$C$-мор\-физ\-мов~$f, g$ и~$C$-объ\-ек\-та~$A$): 
\begin{enumerate}[(1)]
\item $\mathrm{fun}\,(\mathrm{dom}\,f) 
\hm= \mathrm{dom}\,\mathrm{fun}\,(f), \mathrm{fun}\,(\mathrm{codom}\,f)\hm = 
\mathrm{codom}\,\mathrm{fun}\,(f)$;  
\item $\mathrm{fun}\,(g \circ f) = \mathrm{fun}\,(g) \circ \mathrm{fun}\,(f)$, 
если композиция $g \circ f$ определена; 
\item $\mathrm{fun}\,(1_A) \hm= 1_{\mathrm{fun}\,(A)}$.
\end{enumerate}
 Все категории и~все функторы образуют 
(формальную) категорию~$\mathbf{CAT}$. Чтобы исследовать взаимосвязь 
между функторами, вводится следующее понятие: естественным 
преобразованием~$\varepsilon$ функтора $\mathrm{fun}\, : C\hm\to D$ в~$\mathrm{fun}^\prime\, : C 
\hm\to D$ называется любое семейство $D$-мор\-физ\-мов~$\varepsilon_A : 
\mathrm{fun}\,(A) \hm\to \mathrm{fun}^\prime (A)$, $A \hm\in \mathrm{Ob}\,C$, 
такое что для любого 
\mbox{$C$-мор}\-физ\-ма $f : A\hm\to B$ выполняется соотношение $\varepsilon_B \circ 
\mathrm{fun}\,(f) \hm= \mathrm{fun}^\prime(f) \circ \varepsilon_A$:

%\begin{figure*} %рис
\vspace*{1pt}
\begin{center}
\mbox{%
\epsfxsize=54.473mm
\epsfbox{kov-3.eps}
}
\end{center}
%\vspace*{-9pt}
%\end{figure*}

   Эффективность применения теории категорий в~качестве математического 
аппарата \mbox{MBSE} обуслов\-ле\-на тем, что любой каталог моделей представляет 
собой не что иное, как категорию. Действительно, любая цепочка действий по 
соединению моделей порождает композитное действие (процесс) и, кроме того, 
любая модель допускает пустое действие над самой собою, не 
подразумевающее никаких изменений (процедура <<ничегонеделания>>). 
Например, в~твердотельном моделировании механических систем объектами 
категории\linebreak моделей выступают тела~--- подмножества в~$\mathbb{R}^3$, 
которые являются ограниченными, регулярными\linebreak
 (совпадают с~замыканием 
своей внутренности) и~полуаналитическими (допускают представление 
конечными булевыми комбинациями множеств вида $\{(x, y, z) \vert  F_i(x, y, 
z)\hm\leq 0\}$, где~$F_i : \mathbb{R}^3\hm\to \mathbb{R}$ является 
вещественной аналитической функцией для всех~$i$)~\cite{10-kov}. Чтобы 
было возможно задавать процедуры типа склеивания участков поверхности тел, в~категорию геометрических моделей добавляются ограниченные регулярные 
полуаналитические подмножества в~$\mathbb{R}^n$, $0 \hm\leq n \hm\leq 2$, 
при помощи стандартного вложения~$\mathbb{R}^n$ в~$\mathbb{R}^3$. Далее 
выполняется факторизация: отождествляются друг с~другом все множества, 
переходящие друг в~друга под действием аффинных изометрий. Морфизмы 
таких классов эквивалентности, описывающие действия по сборке составных 
механических сис\-тем, порождаются изометрическими вложениями множеств 
и~растяжениями. Получается подкатегория в~\textbf{Set}, которую будем обозначать 
через $\mathbf{MBS}$ (от Multibody Systems).
   
   Для многих известных технологий MBSE формальное описание каталогов 
поддерживаемых моделей приводит к~категориям множеств со структурой~--- 
алгебраических систем, топологических пространств, графов и~т.\,д. 
Морфизмами в~таких категориях служат отображения множеств, со\-вмес\-ти\-мые 
со структурой. На любой такой категории действует канонический функтор 
в~$\mathbf{Set}$, <<забывающий>> структуру. 

В~качестве примера приведем  
дис\-крет\-но-со\-бы\-тий\-ное моделирование, в~котором математической 
моделью сценария служит множество событий, час-\linebreak тич\-но упорядоченное  
при\-чин\-но-след\-ст\-вен\-ны\-ми зависимостями и~размеченное видами 
событий~\cite{15-kov}. Действия по сборке сложных сценариев задаются 
монотонными отображениями, сохраняющими разметку, поскольку ни 
события, ни зависимости, ни метки не могут быть <<потеряны>> при 
соединении сценариев поведения компонентов в~сценарии поведения систем. 
Получается категория~$\mathbf{Pomset}$, состоящая из всех помеченных 
частично упорядоченных множеств и~всех их монотонных отображений, 
сохраняющих разметку. Имеется функтор $\vert \mbox{--} \vert : 
\mathbf{Pomset}\hm\to \mathbf{Set} : S \mapsto \vert S\vert$, <<забывающий>> 
порядок и~разметку.
   
   Зафиксируем произвольную категорию~$C$, представляющую некоторый 
каталог моделей. Как и~для любой алгебраической системы, определена 
конструкция подкатегории в~$C$~--- это пара, состоящая из подкласса 
в~$\mathrm{Ob}\,C$ и~подкласса в~$\mathrm{Mor}\,C$, замкнутых 
относительно унаследованных из~$C$ операций. Подкатегория в~$C$ 
называется полной, если любой \mbox{$C$-мор}\-физм, область и~кообласть которого 
содержатся в~ней, сам содержится в~ней. Например, подкатегориями 
описываются различные аспекты структурного представления систем согласно 
стандарту IEC~81346. Действительно, композиция двух морфизмов, 
представляющих действия по формированию некоторого аспекта структуры, 
также должна входить в~этот аспект, поскольку стандарт предписывает строить 
цепочки для идентификации объектов в~структуре системы. Кроме того, если 
объект присутствует в~аспекте, то его тождественный морфизм формально 
должен быть включен в~этот аспект. В~то же время подкатегории, 
опи\-сы\-ва\-ющие все аспекты, не обязаны образовывать в~совокупности разбиение 
категории~$C$: как показывает рис.~1, возможны как действия, входящие 
в~несколько аспектов одновременно, так и~композитные действия с~переходом 
между структурами, не входящие ни в~один аспект. Требуется лишь, чтобы 
объединение классов объектов всех этих подкатегорий совпадало 
с~$\mathrm{Ob}\,C$, поскольку не имеет смысла вводить модели, не входящие 
ни в~одну структуру.
   
   Категории можно получать из графов: любой ориентированный мультиграф 
порождает категорию, объектами в~которой служат все узлы, а морфизмами~--- 
все пути. Областью и~кообластью морфизма являются соответственно начало 
и~конец пути, композиция морфизмов действует как конкатенация путей, 
а~тождественным морфизмом узла~$a$ является пустой путь из~$a$ в~$a$, не 
содержащий ни одного ребра. Отсюда получается фундаментальное понятие  
$C$-диа\-грам\-мы~--- это функтор вида~$\Delta : X \hm\to C$, где~$X$~--- 
категория, порожденная некоторым графом и~называемая схемой диаграммы. 
Все $C$-диа\-грам\-мы образуют категорию~$\mathbf{D}C$ (ковариантная 
категория <<сверхзапятой>>~\cite{14-kov}), в~которой морфизмом 
диаграммы~$\Delta : X \hm\to C$ в~$\Xi : Y \hm\to C$ служит любая пара 
вида $\langle\gamma, fd\rangle$, состоящая из функтора~$fd : X\hm\to Y$ 
и~естественного преобразования~$\gamma : \Delta\hm\to \Xi \circ fd$; закон 
композиции морфизмов диаграмм имеет вид:
$$
\langle \gamma, fd\rangle \circ 
\langle \varphi, gd\rangle \hm = \langle \gamma_{gd(-)} \circ \varphi, fd \circ 
gd\rangle\,.
$$ 
В~тео\-рии категорий накоплен богатый арсенал алгебраических 
методов конструирования и~анализа диаграмм.
   
   Любая мегамодель задается $C$-диа\-грам\-мой, так что категорное 
представление каталогов моделей позволяет формально решать задачи 
мегамоделирования. Морфизмы диаграмм описывают структурные 
преобразования мегамоделей, выполняемые при помощи инструментов MBSE. 
Покажем, как решаются средствами теории категорий прямые задачи 
мегамоделирования. Здесь применяется одна из основных  
тео\-ре\-ти\-ко-ка\-те\-гор\-ных конструкций~--- копредел  
диаграммы~\cite{5-kov}, который строится следующим образом. Обозначим 
через~$\mathbf{1}$ категорию,\linebreak состоящую из одного объекта~0 и~одного 
морфизма~$1_0$. Из любой категории~$X$ имеется в~точ\-ности один 
функтор~$!_X : X \hm\to \mathbf{1}$, сопоставляющий объект~0  
любому~$X$-объ\-ек\-ту (иными словами, $\mathbf{1}$ является терминальным 
$\mathbf{CAT}$-объ\-ек\-том). Имеется вложение (инъективный функтор) 
$\ulcorner \mbox{--}\urcorner : C \hookrightarrow \mathbf{D}C$, сопоставляющее 
произвольному $C$-объ\-ек\-ту $Q$~точку~--- диаграмму $\ulcorner Q\urcorner : 
\mathbf{1}\hm\to  C : 0 \mapsto Q$. Коконусом (cocone) называется 
$\mathbf{D}C$-мор\-физм, имеющий точку в~качестве кообласти. Можно 
изобразить коконус $\langle \sigma, !_X\rangle : \Delta\hm\to \ulcorner 
Q\urcorner$ над диаграммой $\Delta : X\hm\to C$ в~виде диаграммы, 
<<пририсовав>> к~$\Delta$ дополнительную вершину, помеченную 
объектом~$Q$, и~набор ребер~--- стрелок, по одной для каждого узла $I\hm\in 
\mathrm{Ob}\,X$, направленной из~$I$ в~вершину и~помеченной морфизмом 
$\sigma_I : \Delta (I) \hm\to Q$. Копределом (colimit) диаграммы~$\Delta$ 
называется коконус $\mathrm{colim}\,\Delta : \Delta\hm\to \ulcorner R\urcorner$, 
универсальный в~том смысле, что для любых \mbox{$C$-объ}\-ек\-та~$T$ 
и~коконуса~$\delta : \Delta\hm\to\ulcorner T\urcorner$ существует единственный 
$C$-мор\-физм~$w : R \hm\to T$ такой, что $\delta\hm= \ulcorner w\urcorner \circ  
\mathrm{colim}\,\Delta$. Легко видеть, что это условие универсальности 
представляет собой в~точности структурный критерий из разд.~3. Таким 
образом, конструирование копредела конфигурации~$\Delta$ описывает на 
строгом математическом языке сборку системы, которой отвечает 
вершина~$R$. В~категориях типа $\mathbf{MBS}$ и~$\mathbf{Pomset}$ 
построение копредела сводится к~факторизации раздельных объединений 
объектов, представляющих компоненты системы, по отношениям 
эквивалентности, индуцированным моделями клея и~других средств сборки.
   
   Копредел любой диаграммы, если он сущест\-вует, определяется однозначно 
   с~точностью до изомор\-физма. Более того, можно описать сборку сис\-тем из 
конфигураций в~виде функтора. Пусть $Cd$~--- некоторый класс  
$C$-диа\-грамм, имеющих копределы. Он порождает полную подкатегорию 
в~$\mathbf{D}C$, из которой в~$C$ действует функтор копредела $\mathrm{colim}$, 
сопоставляя каждой диаграмме из~$Cd$~вершину некоторого ее копредела, а 
каждому \mbox{$\mathbf{D}C$-мор}\-физ\-му~$\theta : \Delta\hm\to \Xi$, 
где~$\Delta, \Xi\hm\in Cd$~--- стрелку копредела $\mathrm{colim}\,(\theta)$ такую, что 
$\mathrm{colim}\,\Xi \circ \theta \hm= \ulcorner \mathrm{colim}\,(\theta)\urcorner \circ 
\mathrm{colim}\,\Delta$.

%\begin{figure*}
\vspace*{1pt}
\begin{center}
\mbox{%
\epsfxsize=56.127mm
\epsfbox{kov-4.eps}
}
\end{center}
%\vspace*{-9pt}
%\end{figure*}

   Например, в~категории \textbf{Set} любая диаграмма имеет 
копредел~\cite[упражнение~5.1.8]{14-kov}, поэтому имеется функтор $\mathrm{colim}\, : 
\mathbf{D}(\mathbf{Set})\hm\to \mathbf{Set}$. Примечательно, что этот функтор 
является рефлектором: он сопряжен слева с~вложением $\ulcorner \mbox{--}\urcorner : 
\mathbf{Set}\hookrightarrow \mathbf{D}(\mathbf{Set})$, причем 
единица рефлексии состоит из $\mathbf{D}(\mathbf{Set})$-мор\-физ\-мов 
$\mathrm{colim}\,\Delta : \Delta\hm\to \ulcorner\mathrm{colim}\,(\Delta)\urcorner$, 
$\Delta\hm\in \mathrm{Ob}\ \mathbf{D}(\mathbf{Set})$. Напомним, что единица 
рефлексии~--- это естественное преобразование тождественного функтора 
в~композицию рефлектора и~вложения (в~данном случае, естественное 
преобразование функтора $1_{\mathbf{D}(\mathbf{Set})}$ в~$\ulcorner \mathrm{colim}\,(  
\mbox{--})\urcorner)$, состоящее из универсальных  
стрелок~\cite[разд.~4.3]{14-kov}. И~для произвольного класса~$Cd$, 
содержащего достаточное количество одноточечных диаграмм, функтор 
$\mathrm{colim}$ сопряжен слева с~ограничением  
вложения~$\ulcorner \mbox{--}\urcorner$ на подходящую полную подкатегорию 
в~$C$. А~поскольку сопряженный функтор задается однозначно с~точностью 
до изоморфизма~\cite[разд.~4.1]{14-kov}, можно сделать вывод, что сборка 
систем в~некотором смысле <<зашифрована>> в~процедуре построения 
одноточечных диаграмм~--- моделей систем как целого без раскрытия 
струк\-туры. 

Так наглядно проявляется двойственность прямых и~обратных задач 
мегамоделирования.

\section{Заключение}

   Аппарат теории категорий обладает большим потенциалом в~области 
повышения полезной отдачи от MBSE, в~том числе путем математически 
строгого решения задач мегамоделирования. Так, базовая процедура системной 
инженерии~--- сборка\linebreak
 системы из заданной конфигурации взаимо\-свя\-занных 
компонентов~--- формально описывается тео\-ретико-ка\-те\-гор\-ной 
конструкцией копредела диа\-граммы. Более сложные конструкции отвечают\linebreak 
сложным процедурам сборки, таким как связывание (weaving) общесистемных 
функций, рассеянных по всем компонентам (crosscutting concerns), например 
мониторинговых или защитных~\cite{16-kov}. Математического представления 
требуют и~другие процедуры MBSE, в~частности коллективная модификация 
мегамоделей и~составляющих моделей, восстановление конфигурации заданной 
системы, оценка взаимозаменяемости компонентов. 

Актуальны вопросы 
внедрения аппарата теории категорий в~практику, в~том числе путем развития 
программных инструментов моделирования и~мегамоделирования. Здесь 
открывается широкий спектр направлений для дальнейших исследований.
   
{\small\frenchspacing
 {%\baselineskip=10.8pt
 \addcontentsline{toc}{section}{References}
 \begin{thebibliography}{99}
\bibitem{1-kov}
Modeling and simulation-based systems engineering handbook~/
Eds.\ D.~Gianni,  A.~D'Ambrogio, A.~Tolk.~--- London: CRC Press, 2014. 513~p.
\bibitem{2-kov}
\Au{Ковалёв С.\,П., Толок~А.\,В.} Применение модельно-ори\-ен\-ти\-ро\-ван\-но\-го подхода 
в~управ\-ле\-нии жизненным циклом технических изделий~// Информационные технологии 
в~проектировании и~производстве, 2015. №\,2. С.~3--9.
\bibitem{3-kov}
\Au{Левенчук А.\,И.} Системноинженерное мышление.~--- М.: TechInvestLab, 2015. 305~с.
\bibitem{4-kov}
IEC 81346-1:2009. Industrial Systems, Installations and Equipment and Industrial Products~--- 
Structuring Principles and Reference Designations~--- Part~1: Basic Rules.~--- Geneva: ISO, 2009. 
168~p.
\bibitem{5-kov}
\Au{Ginali S., Goguen~J.} A~categorical approach to general systems~// 
 Conference (International) on Applied General Systems 
Research Proceedings~/
Ed. G.\,J.~Klir.~--- NATO conference series.~--- New York, NY, USA: Plenum 
Press, 1978. Vol.~5. P.~257--270.
\bibitem{6-kov}
\Au{Mabrok M.\,A., Ryan M.\,J.} Category theory as a~formal mathematical foundation for  
model-based systems engineering~// Appl. Math. Inform. Sci., 2017. Vol.~11. No.\,1. P.~43--51.
\bibitem{7-kov}
\Au{Ковалёв С.\,П.} Тео\-ре\-ти\-ко-ка\-те\-гор\-ный подход к~проектированию программных 
сис\-тем~// Фундаментальная и~прикладная математика, 2014. Т.~19. Вып.~3. С.~111--170.
\bibitem{8-kov}
\Au{B$\acute{\mbox{e}}$zivin J., Jouault~F., Rosenthal~P., Valduriez~P.} Modeling in the large 
and modeling in the small~// Model Driven Architecture: European MDA Workshops on 
Foundations and Applications Proceedings~/
Eds.\ U.~A{\!\ptb{\ss}}mann, M.~Aksit,  A.~Rensink.~--- 
Lecture notes in computer science ser.~--- Springer, 2005. Vol.~3599. 
P.~33--46.
\bibitem{9-kov}
\Au{Diskin Z., Kokaly~S., Maibaum~T.} Mapping-aware mega\-mod\-eling: Design patterns and 
laws~// Software Language Engineering: 6th Conference (International) Proceedings~/
Eds.\ M.~Erwig, R.\,F.~Paige, E.~Van Wyk.~--- 
Lecture notes  in computer science ser.~--- Springer, 2013. Vol.~8225. P.~322--343.
\bibitem{10-kov}
\Au{Requicha A.\,G.} Representations for rigid solids: Theory, methods, and systems~// 
ACM  Comput. Surv., 1980. Vol.~12. Iss.~4. P.~437--464.
\bibitem{11-kov}
\Au{K$\acute{\mbox{a}}$d$\acute{\mbox{a}}$r B., Pfeiffer~A., Monostori~L.} Discrete event 
simulation for supporting production planning and scheduling decisions in digital
 factories~//  37th 
CIRP Seminar (International) on Manufacturing Systems Proceedings.~--- Budapest, 2004.  
P.~444--448.
\bibitem{12-kov}
\Au{Giesa T., Spivak D.\,I., Buehler~M.\,J.} Category theory based solution for the building block 
replacement problem in materials design~// Adv. Eng. Mater., 2012. Vol.~14. 
Iss.~9. P.~810--817.
\bibitem{13-kov}
\Au{Косяков А., Свит У., Сеймур~С., Бимер~С.} Системная инженерия. Принципы 
и~практика~/ Пер. с~англ.~--- М.: ДМК-Пресс, 2014. 636~с. (\Au{Kossiakoff~A., Sweet~W.\,N., 
Seymour~S., Biemer~S.\,M.} Systems engineering principles and practice.~--- 2nd ed.~--- New 
York, NY, USA: John Wiley, 2011. 560~p.)
\bibitem{14-kov}
\Au{Маклейн С.} Категории для работающего математика~/ Пер. с~англ.~--- М.: Физматлит, 
2004. 352~с. (\Au{Mac Lane~S.} Categories for the working mathematician.~--- New York, NY, 
USA: Springer, 1978. 317~p.)
\bibitem{15-kov}
\Au{Pratt V.\,R.} Modeling concurrency with partial orders~// Int. J.~Parallel 
Prog., 1986. Vol.~15. No.\,1. P.~33--71.
\bibitem{16-kov}
\Au{Ковалёв С.\,П.} Семантика ас\-пект\-но-ори\-ен\-ти\-ро\-ван\-но\-го моделирования 
данных и~процессов~// Информатика и~её применения, 2013. Т.~7. Вып.~3. С.~70--80.
 \end{thebibliography}

 }
 }

\end{multicols}

\vspace*{-3pt}

\hfill{\small\textit{Поступила в~редакцию 16.01.17}}

%\vspace*{8pt}

\newpage

\vspace*{-30pt}

%\hrule

%\vspace*{2pt}

%\hrule

%\vspace*{8pt}


\def\tit{METHODS OF CATEGORY THEORY IN~MODEL-BASED SYSTEMS ENGINEERING\\[-7pt]}

\def\titkol{Methods of category theory in~model-based systems engineering}

\def\aut{S.\,P.~Kovalyov\\[-12pt]}

\def\autkol{S.\,P.~Kovalyov}

\titel{\tit}{\aut}{\autkol}{\titkol}

\vspace*{-14pt}


\noindent
Institute of Control Sciences, Russian Academy of Sciences, 65~Profsoyuznaya Str., 
Moscow 117997, Russian Federation



\def\leftfootline{\small{\textbf{\thepage}
\hfill INFORMATIKA I EE PRIMENENIYA~--- INFORMATICS AND
APPLICATIONS\ \ \ 2017\ \ \ volume~11\ \ \ issue\ 3}
}%
 \def\rightfootline{\small{INFORMATIKA I EE PRIMENENIYA~---
INFORMATICS AND APPLICATIONS\ \ \ 2017\ \ \ volume~11\ \ \ issue\ 3
\hfill \textbf{\thepage}}}

\vspace*{1pt}

 

\Abste{A mathematical device based on the category theory is proposed to formally describe and 
rigorously explore procedures of employing models in engineering that constitute the contents of 
model-based systems engineering (MBSE). The essence of the device consists in mathematical 
representation of assembly drawings (megamodels of systems) as diagrams in categories whose 
objects are models, and morphisms represent actions associated with assembling system models 
from component models. The soundness of the device is justified on the basis of standards that 
govern description of the systems' structure such as IEC~81346. Category-theoretical methods for 
solving a number of practical problems of assembling systems are proposed and explored. 
Examples of solving such problems are provided in categories that represent two key application 
areas for MBSE: geometric modeling of complex shapes and discrete-event simulation of the 
behavior of industrial systems.}

\KWE{ model-based systems engineering; megamodel; category theory; colimit}

\DOI{10.14357/19922264170305} 

%\vspace*{-18pt}

%\Ack
%\noindent




\vspace*{-7pt}

  \begin{multicols}{2}

\renewcommand{\bibname}{\protect\rmfamily References}
%\renewcommand{\bibname}{\large\protect\rm References}

{\small\frenchspacing
 {%\baselineskip=10.8pt
 \addcontentsline{toc}{section}{References}
 \begin{thebibliography}{99}
\bibitem{1-kov-1}
Gianni, D., A.~D'Ambrogio, and A.~Tolk, eds. 2014. \textit{Modeling and simulation-based 
systems engineering handbook}. London: CRC Press. 513~p.
\bibitem{2-kov-1}
\Aue{Kovalyov, S.\,P., and A.\,V.~Tolok.} 2015. Primenenie model'no-orientirovannogo podkhoda 
v~upravlenii zhiznennym tsiklom tekhnicheskikh izdeliy [Applying model-based approach 
to product lifecycle management].\linebreak \textit{Informatsionnye tekhnologii v~proektirovanii 
i~proizvod\-st\-ve} [Information Technologies in Design and Industry] 2(158):3--9.
\bibitem{3-kov-1}
\Aue{Levenchuk A.\,I.} 2015. 
\textit{Sistemnoinzhenernoe myshlenie} [Systems engineering thinking]. 
Moscow: TechInvestLab. 305~p.
\bibitem{4-kov-1}
IEC 81346-1:2009. 2009. 
Industrial Systems, Installations and Equipment and Industrial 
Products~--- Structuring Principles and Reference Designations~--- 
Part~1: Basic Rules. Geneva:  ISO. 168~p.
\bibitem{5-kov-1}
\Aue{Ginali, S., and J.~Goguen.} 1978. 
A~categorical approach to general systems. \textit{Conference 
(International) on Applied General Systems Research Proceedings}. Ed.\
 G.\,J.~Klir. \mbox{NATO}  conference ser. Plenum Press. 5:257--270.
\bibitem{6-kov-1}
\Aue{Mabrok, M.\,A., and M.\,J.~Ryan}. 
2017. Category theory as a~formal mathematical foundation for 
model-based systems engineering. \textit{Appl. Math.  Inform. Sci.} 11(1):43--51.
\bibitem{7-kov-1}
\Aue{Kovalyov, S.\,P.} 2016. 
Category-theoretic approach to software systems design. \textit{J.~Math. Sci.} 
214(6):814--853.
\bibitem{8-kov-1}
\Aue{B$\acute{\mbox{e}}$zivin, J., F.~Jouault, P.~Rosenthal, and P.~Valduriez.}
 2005. Modeling in 
the large and modeling in the small. 
\textit{Model Driven Architecture: European MDA Workshops on 
Foundations and Applications Proceedings.} 
Eds.\ U.~\mbox{A{\!\ptb{\ss}}mann}, M.~Aksit, and A.~Rensink. 
Lecture notes in computer science ser. Springer. 3599:33--46.
\bibitem{9-kov-1}
\Aue{Diskin, Z., S.~Kokaly, and T.~Maibaum.} 2013. 
Mapping-aware megamodeling: Design patterns 
and laws. \textit{6th Conference (International) on Software Language Engineering 
Proceedings}. Eds.\ M.~Erwig, R.\,F.~Paige, and E.~Van Wyk. 
Lecture notes in computer science ser. Springer. 
8225:322--343.
\bibitem{10-kov-1}
\Aue{Requicha, A.\,G.} 1980. Representations for rigid solids: 
Theory, methods, and systems. \textit{ACM 
Comput. Surv.} 12(4):437--464.
\bibitem{11-kov-1}
\Aue{K$\acute{\mbox{a}}$d$\acute{\mbox{a}}$r,~B., A.~Pfeiffer, and L.~Monostori.}
2004. Discrete 
event simulation for supporting production planning and scheduling decisions in 
digital factories. \textit{37th CIRP Seminar (International) on Manufacturing 
Systems Proceedings}. Budapest.  444--448.
\bibitem{12-kov-1}
\Aue{Giesa, T., D.\,I.~Spivak, and M.\,J.~Buehler.} 2012. 
Category theory based solution for the building 
block replacement problem in materials design. 
\textit{Adv. Eng. Mater.} 14(9):810--817.
\bibitem{13-kov-1}
\Aue{Kossiakoff, A., W.\,N.~Sweet, S.~Seymour, and S.\,M.~Bie\-mer.}
2011. \textit{Systems engineering 
principles and practice}. 2nd ed. New York, NY: John Wiley. 560~p.
\bibitem{14-kov-1}
\Aue{Mac Lane, S.} 1978. \textit{Categories for the working mathematician}. 
New York, NY: Springer. 317~p.
\bibitem{15-kov-1}
\Aue{Pratt, V.\,R.} 1986. Modeling concurrency with partial orders. 
\textit{Int. J.~Parallel Prog.} 15(1):33--71.
\bibitem{16-kov-1}
\Aue{Kovalyov, S.\,P.} 2013. 
Semantika aspektno-ori\-en\-ti\-ro\-van\-no\-go modelirovaniya dannykh 
i~protsessov [Semantics of aspect-oriented modeling of data and processes]. 
\textit{Informatika i~ee  Primeneniya~--- Inform. Appl.} 7(3):70--80.
\end{thebibliography}

 }
 }

\end{multicols}

\vspace*{-9pt}

\hfill{\small\textit{Received January 16, 2017}}

\vspace*{-18pt}

\Contrl

\noindent
\textbf{Kovalyov Sergey P.} (b.\ 1972)~--- Doctor of Science in physics and 
mathematics, leading scientist, Institute of Control Problems, Russian 
Academy of Sciences, 65~Profsoyuznaya Str., Moscow 117997, Russian 
Federation Federation; \mbox{kovalyov@nm.ru} 

\label{end\stat}


\renewcommand{\bibname}{\protect\rm Литература}        %2
\def\stat{vasiliev}

\def\tit{КОМПОЗИЦИОНАЛЬНОЕ ПРЕДСТАВЛЕНИЕ СТРУКТУРЫ ИГРЫ МНОГИХ ЛИЦ 
В~МОНОИДАЛЬНОЙ КАТЕГОРИИ БИНАРНЫХ ОТНОШЕНИЙ}

\def\titkol{Композициональное представление структуры игры многих лиц 
в~моноидальной категории бинарных отношений}

\def\aut{Н.\,С.~Васильев$^1$}

\def\autkol{Н.\,С.~Васильев}

\titel{\tit}{\aut}{\autkol}{\titkol}

\index{Васильев Н.\,С.}
\index{Vasilyev N.\,S.}


%{\renewcommand{\thefootnote}{\fnsymbol{footnote}} \footnotetext[1]
%{Работа выполнена с~использованием инфраструктуры Центра коллективного пользования <<Высокопроизводительные вы\-чис\-ле\-ния и~большие данные>> 
%(ЦКП <<Информатика>>) ФИЦ ИУ РАН (г.~Москва).}}


\renewcommand{\thefootnote}{\arabic{footnote}}
\footnotetext[1]{Московский государственный технический университет им.\ Н.\,Э.~Баумана, \mbox{nik8519@yandex.ru}}

\vspace*{-10pt}


    \Abst{Предложен системный подход к~решению игры многих лиц, отвечающий 
современным сетевым технологиям. Он поз\-во\-ля\-ет оптимизировать 
функционирование муль\-ти\-агент\-ных сис\-тем. Моноидальная категория бинарных 
отношений применяется как средство описания правил игры, исследования 
и~модификации поведения игроков. Игровая проб\-ле\-ма со\-сто\-ит в~том, чтобы по 
воз\-мож\-ности максимизировать отношения предпочтения всех участ\-ни\-ков игры. 
В~соответствии с~правилами игры их композиция определяет ре\-зуль\-ти\-ру\-ющее 
отношение игры (РОИ). Поиск рационального поведения игроков сведен 
к~на\-хож\-де\-нию максимальных элементов РОИ. Формализовано использование 
разнообразных классов до\-пус\-ти\-мых стратегий, процессов обмена информацией 
меж\-ду игроками и~формирование коалиций. Доказано существование РОИ 
и~изуче\-на структура его максимальных элементов, со\-кра\-ща\-ющая поиск. Выяснено 
значение отношений предшествования ходов и~абсолютно оптимальных 
предпочтений игроков в~процессе формирования коалиций.}
    
    \KW{отношения предпочтения: абсолютно оптимальное, гарантированное, 
предшествования ходов; граф игры; до\-пус\-ти\-мая стратегия; рациональное решение; 
характеристическое отношение коалиции; ре\-зуль\-ти\-ру\-ющее отношение игры; 
моноидальная категория; ком\-по\-зи\-ци\-о\-наль\-ность}

  \DOI{10.14357/19922264230203}{GPMZTS}
  
\vspace*{2pt}


\vskip 10pt plus 9pt minus 6pt

\thispagestyle{headings}

\begin{multicols}{2}

\label{st\stat}
    
\section{Введение}

    Игровой задаче со~многими участниками присуще большое разнообразие 
по\-ста\-но\-вок, которые приходится динамически уточ\-нять в~процессе исследования, 
проводя анализ результатов воз\-мож\-ных рациональных решений~[1--5]. 
Востребовано новое, композициональное, пред\-став\-ле\-ние игры, от\-ве\-ча\-ющее сетевым 
технологиям мультиагентных сис\-тем, программно ре\-а\-ли\-зу\-емое для оптимизации их 
работы~[6]. Предложено формализовать игру средствами моноидальной категории 
бинарных отношений~[7, 8], так как традиционные нормальная\linebreak форма и~развернутое 
представление игры не обладают нуж\-ным качеством~\cite{1-vas, 9-vas}. В~отличие 
от применения платежных функ\-ций теперь решения принимаются на базе 
отношений на множестве \mbox{ситуаций} игры, которые учитывают правила игры 
и~динамику поведения парт\-не\-ров. Сход\-ны\-ми соображениями руководствуются 
в~многокритериальных задачах~\cite{3-vas} и~экономических приложениях тео\-рии 
игр~\cite{4-vas, 10-vas, 11-vas}. 
    
    \subsection*{Переход от~нормальной к~композициональной форме 
игры}

    Пусть каждый участник игровой задачи $i\hm \in I\hm= \{1, 2, \ldots ,n\}$ 
стремится по воз\-мож\-ности максимизировать свой критерий эф\-фек\-тив\-ности $w_i: 
X\hm\to R, X\hm\equiv X_1\times X_2\times \cdots \times X_n$, выбирая свой 
конт\-ро\-ли\-ру\-емый фактор $x_i\hm\in X_i$, $i\hm\in I$~\cite{1-vas, 2-vas, 5-vas}. Поиск 
рационального решения игры на множестве ситуаций $x\hm\in X$ проводится 
в~условиях не\-опре\-де\-лен\-ности, обуслов\-лен\-ной различием интересов игроков 
и~не\-воз\-мож\-ностью точ\-но\-го прогноза результата игры из-за незнания действий 
парт\-не\-ров. Поэтому агенты вы\-нуж\-де\-ны расширять классы до\-пус\-ти\-мых 
стратегий~$\tilde{X}_i$, $X_i\hm\subset \tilde{X}_i$, чтобы учитывать по\-сту\-па\-ющую 
информацию и~коалиционное поведение игроков~\cite{1-vas, 5-vas}. 

Срав\-не\-ние 
ситуаций игры мож\-но проводить не только с~по\-мощью критериев эф\-фек\-тив\-ности 
$w_i\hm= w_i(x)$, но и~применяя отношения предпочтения игроков $\rho_i\hm\subset 
X^2$, $i\hm\in I$. Точ\-ное знание интересов моделируется линейным порядком, 
а~нечеткие пред\-став\-ле\-ния~--- бинарными отношениями общего вида. 
 По-преж\-не\-му игроки стремятся по воз\-мож\-ности выбирать максимально 
предпочтительные для них ситуации игры. 
    
    Рациональное поведение агентов определено правилами игры, по\-рож\-да\-ющи\-ми 
из исходных~$\rho_i$,\linebreak $i\hm\in I$, вспомогательные бинарные отношения, \mbox{которые} 
выражают принципы оптимального поведения, открытые в~тео\-рии игр~[1--5], 
и~руководят действиями игроков. Производные отношения \mbox{учитывают} кооперацию 
и~ин\-фор\-ми\-ро\-ван\-ность игроков. Они конструируются посредством ал\-геб\-ра\-и\-че\-ских 
операций ${A}\hm= (\circ, \cup, \cap, {}^{\mathrm{op}}, \times; \sigma, \varnothing)$ 
и~композиции морфизмов категории бинарных отношений REL~[7, 8]. Так, обмен 
информацией между агентами всегда приводит к~сужению их отношений 
предпочтения $\rho\vert_A \hm= \rho\cap A^2$ на некоторую часть~$A$ множества 
ситуаций игры.
    
    \textbf{Пример~1.1.} В~бескоалиционной игре $\Gamma\hm= \Gamma^0$ 
участ\-ни\-ки не имеют никакой информации о~стратегиях партнеров и~отсутствует 
до\-го\-во\-рен\-ность об оче\-ред\-ности ходов. Тогда \textit{ре\-зуль\-ти\-ру\-ющим} отношением, 
которое в~конечном итоге требуется оптимизировать, будет сле\-ду\-ющая 
дизъ\-юнк\-тив\-ная сумма~[7, 8]:
    \begin{equation}
    \rho^{\Gamma}= \coprod\limits_{i\in I} \rho_i.
    \label{e1.1-vas}
    \end{equation}
    
    Коалиции игроков~$C$ создаются за счет коммуникации. Им отвечают 
некоторые подыг\-ры~$\Gamma^\prime$ исходной игры~$\Gamma$, в~которых 
фиксированы стратегии агентов $i \not= C$. Интересы коалиции пред\-став\-ле\-ны 
\textit{характеристическим} отношением~$\rho_C$, сов\-па\-да\-ющим 
с~РОИ~$\Gamma^\prime$. Конструкция~$\rho_C$ 
определяется исходя из правил игры и~принципа рационального поведения 
участ\-ни\-ков коалиции. 
    
    \textbf{Пример~1.2.} В коалиции Парето $C\hm= C^P$ все участники обладают 
полной информацией о~действиях друг друга и~со\-вмест\-но выбирают общий 
конт\-ро\-ли\-ру\-емый фактор $x_C\hm= (x_i, i\in C)$, стремясь к~максимизации 
сле\-ду\-юще\-го от\-но\-ше\-ния~\cite{3-vas}:
    \begin{equation}
    \rho^{\Gamma^\prime} = \rho_C = \mathop{\bigcap}\limits_{i\in C} \rho_i\,.
    \label{e1.2-vas}
    \end{equation}
    
    \textit{Абсолютно оптимальным} отношением $i$-го игрока назовем сужение 
его отношения предпочтения $a_i \hm= \rho_i\vert_{x_i}$ при фиксированных 
значениях всех остальных па\-ра\-мет\-ров $x_j\hm\in X_j$, $j\not= i$. Включение 
$(x,x^\prime)\hm\in a_i$ означает до\-сти\-жи\-мость более предпочтительной 
ситуации~$x^\prime$ из~$x$ ходом $i$-го игрока. Для игры в~нормальной форме 
этому понятию соответствует применение игроком абсолютно оптимальной 
стратегии~\cite{1-vas}.
    
    \textbf{Пример~1.3.} В~игре~$\Gamma^{0,n}$ участники коалиции $C\hm= 
C(\pi, \varnothing)$, $\pi \hm= \{ (l, l\hm+1), l=1,\ldots , n\hm-1\}$, не обмениваясь 
данными ($D\hm= \varnothing$), выполняют ходы в~линейном по\-рядке
\begin{equation}
\pi:\boxed{1}--\rightarrow \boxed{2}--\rightarrow\cdots--\rightarrow\boxed{n}\,.
\label{e1.3-vas}
\end{equation}
Тогда предпочтения коалиции задаются характеристическим от\-но\-ше\-нием
\begin{equation}
\rho^{\Gamma^{0,n}} =\rho_{C(\pi,\varnothing)} =a_n \circ a_{n-1} \circ \cdots \circ 
a_1\,.
\label{e1.4-vas}
\end{equation}
    
    В~(\ref{e1.4-vas}) использована композиция морфизмов категории 
REL~\cite{8-vas}. Рациональное поведение всякого игрока $i\hm\in C(\pi, 
\varnothing)$~--- это использование абсолютно оптимальной стратегии, от\-ве\-ча\-ющей 
стремлению вы\-брать ситуацию из множества~MAX\,$a_i$. Коалиция 
$C(\pi,\varnothing)$ сможет окончательно вы\-брать оптимальное решение~$x_C^*$ 
лишь на последнем ($n\hm-1$)-м шаге, завершив по\-стро\-ение 
суперпозиции~(\ref{e1.4-vas}). При этом произведение морфизмов берется 
в~противоположном порядке~$\pi^{op}$. 
    
    Категориальный подход согласует правила игры, процессы формирования 
коалиций, классы до\-пус\-ти\-мых стратегий и~поиск рационального решения  
(см.~(\ref{e1.1-vas})--(\ref{e1.4-vas})).
    
    Наличие отношения эк\-ви\-ва\-лент\-ности $x\sim x^\prime$ на множестве~$X$ 
поз\-во\-ля\-ет упрос\-тить игру за счет факторизации~[7, 8]. Раз\-би\-ение ситуаций игры на 
классы эк\-ви\-ва\-лент\-ности строится с~по\-мощью критериев $x\sim x^\prime 
\Leftrightarrow w_i(x)\hm= w_i(x^\prime)$. То же самое делается с~применением 
отношения то\-ле\-рант\-ности $\tau\hm\subset X^2$, которое по определению обладает 
свойствами реф\-лек\-сив\-ности и~сим\-мет\-рич\-ности.
    
    \smallskip
    
    \noindent
    \textbf{Лемма~1.1.} \textit{Пусть семейство $\{ x: x\tau y\}$ со\-сто\-ит из 
открытых множеств. Тогда компакт $X\hm\subset R^n$ можно заменить 
конечным множеством.}
     
  \subsection*{Игровая задача как максимизация результирующего 
отношения игры}
   
    Интересы игроков $i\hm\in I$ пред\-став\-ле\-ны бинарными отношениями 
предпочтения $\rho_i\hm\subset X^2$, $i\hm\in I$, заданными на конечном 
множестве ситуаций игры~$X$. Они реф\-лек\-сив\-ны и~тран\-зи\-тив\-ны. Все агенты ведут 
себя рационально. Их стратегии включают обмен данными с~другими участниками 
конфликта, решение о~вступ\-ле\-нии в~одну или несколько коалиций и~выбор момента 
выполнения своего хода~\cite{1-vas, 5-vas}. \textit{Ход} игрока~--- это принятие 
решения вида $\tilde{x}_i : X\hm\to X_i$, со\-про\-вож\-да\-емо\-го, воз\-мож\-но, сообщением 
стратегии~$\tilde{x}_i$ одному или нескольким парт\-не\-рам по игре. 
    
    Коллективные действия игроков порождают правила игры, которые согласуют 
классы до\-пус\-ти\-мых стратегий, процессы формирования коалиций, вводят 
\textit{отношение предшествования} ходов~$\pi$ и~уточ\-ня\-ют содержание данных, 
которыми обмениваются игроки~\cite{5-vas} (см.\ примеры~1.1 и~1.3). 
Со\-гла\-со\-ван\-ность означает отсутствие противоречия в~процессе принятия решений. 
Благодаря этому существует ре\-зуль\-ти\-ру\-ющее отношение игры~$\rho^{\Gamma(S)}$. 
Композициональная структура $\rho^{\Gamma(S)} \hm\subset X^2(S)$ формализует 
процесс поиска рационального решения. В~нее входят модули, от\-ве\-ча\-ющие 
подыграм~$\Gamma_r$, $r\hm= 1,2, \ldots , R$, проходящим в~коалициях $C\hm= 
C_r$, пред\-став\-лен\-ных характеристическими отношениями~$\rho_{C_r}$, $r\hm= 
1,2,\ldots , R$. Все подыгры \textit{на\-сле\-ду\-ют} правила исходной игры, в~част\-ности 
отношение пред\-шест\-во\-ва\-ния ходов~$\pi_C$.
    
    \textit{Рациональное} решение игры~$\Gamma(S)$ в~классе стратегий~$S$~--- 
это выбор ситуации $x^*\hm\in X(S)$, яв\-ля\-ющей\-ся мак\-си\-маль\-ным элементом 
\textit{ре\-зуль\-ти\-ру\-юще\-го} отношения

\noindent
    \begin{equation}
    x^* \in \mathrm{MAX}\,\rho^{\Gamma(S)}.
    \label{e1.5-vas}
    \end{equation}
    
    \vspace*{-6pt}
    
    На множестве классов $\overset{\frown}{X}(S)$ эквивалентных ситуаций $x\sim x^\prime 
\hm\Leftrightarrow (x,x^\prime), (x^\prime,x)\hm\in \rho^{\Gamma(S)}$, введем  
фак\-тор-от\-но\-ше\-ние $\overset{\frown}{\rho}^{\Gamma(S)} \hm\subset \overset{\frown}{X}^2(S)$~[7,~8].
    
    \smallskip
    
    \noindent
    \textbf{Теорема~1.1.}\ \textit{Пусть отношение~$\rho^{\Gamma(S)}$ 
реф\-лек\-сив\-но и~тран\-зи\-тив\-но. Тогда с~точ\-ностью до эк\-ви\-ва\-лент\-ности существует 
рациональное решение игры.}
    
    \smallskip
    
    \noindent 
    Д\,о\,к\,а\,з\,а\,т\,е\,л\,ь\,с\,т\,в\,о\,.\ \  В~час\-тич\-но упорядоченном 
множестве $(\overset{\frown}{X},\overset{\frown}{\rho}), \overset{\frown}{\rho}\hm= \overset{\frown}{\rho}^{\Gamma(S)}$, 
возьмем произвольный элемент $\overset{\frown}{x}\hm\in \overset{\frown}{X}$. Если $\overset{\frown}{x}\hm= 
\mathrm{MAX}\,\overset{\frown}{\rho}^{\Gamma(S)}$, то тео\-ре\-ма доказана. Искомая ситуация~$x^*$, 
$x^*\hm\in \overset{\frown}{x}$.\linebreak Иначе решением игры будет по\-след\-ний элемент 
$\overset{\frown}{x}_k\hm= \overset{\frown}{x}^*$ максимальной цепи $\mathrm{Ch}\hm= \left( \overset{\frown}{x}_0, 
\overset{\frown}{x}_1, \ldots , \overset{\frown}{x}_k\right)$, в~которой $\overset{\frown}{x}_0\hm= \overset{\frown}{x}$ 
и~$\left( \overset{\frown}{x}_l\overset{\frown}{x}_{l+1}\right)\hm\in \overset{\frown}{\rho}^{\Gamma(S)}$ для всех 
$l\hm= 0, 1, \ldots ,k-1$. 
    
    \smallskip
    
    \noindent
    \textbf{Следствие~1.1.} \textit{Конечное ациклическое ан\-ти\-сим\-мет\-рич\-ное 
отношение содержит максимальный и~минимальный элементы}.
    
    \smallskip
    
    Наряду с~эффективностью~(\ref{e1.5-vas}) в~игровых задачах используется 
принцип устой\-чи\-вости вы\-би\-ра\-емой ситуации. \textit{Равновесием} в~коалиционной 
игре $\Gamma(S)$ назовем ситуацию 

\noindent
    \begin{equation}
    x^* \in \mathop{\bigcap}\limits_r \mathrm{MAX}\,\rho_{\rho_{C_r}}.
    \label{e1.6-vas}
    \end{equation}
    
        \vspace*{-6pt}
        
        \noindent
Вообще говоря, принципы~(\ref{e1.5-vas}), (\ref{e1.6-vas}) противоречат друг 
другу~[1--5].

\vspace*{-4pt}

\section{Характеристическое отношение коалиции}

\vspace*{-4pt}
 
    Всякий раз, когда имеется нетривиальное отношение пред\-шест\-во\-ва\-ния ходов, 
возникают иерархически организованные коалиции~\cite{1-vas, 2-vas, 5-vas} (см.\ 
пример~1.3). При этом их участники, вообще говоря, обмениваются 
информацией~$D$. Иерархия внут\-ри коалиции по\-рож\-да\-ет подкоалиции. Коалиции 
передают другим игрокам $i\notin C$ некоторые данные $x_C\hm\in D$, которые 
ранее сообщали их участники $k\hm\in C$. Сообщение стра\-те\-гий-функ\-ций 
будем выражать в~форме включения $\tilde{x}_C\hm\in \tilde{D}$. 
    
    \textbf{Пример~2.1.} В~игре~$\Gamma^{1,n}$ образуется коалиция $C(\pi, 
D)$, $D\hm= \{\overline{x}_1, \ldots , \overline{x}_{n-1}\}$, в~результате 
до\-го\-во\-рен\-ности о~порядке ходов~(\ref{e1.3-vas}) и~передачи игроком $l\hm= 1, 
\ldots , n\hm-1$ величины $\overline{x}_l\hm\in D$ партнеру~$l\hm+1$. 
Характеристическим отношением коалиции будет 
     \begin{equation}
     \rho^{\Gamma^{1,n}}=\rho_{C(\pi,D)} =a_1\circ a_2\circ \cdots\circ a_n\,.
     \label{e2.1-vas}
     \end{equation}
    
    В обосновании формулы~(\ref{e2.1-vas}) лежат те же причины, что 
и~у~схемы~(\ref{e1.4-vas}). Чтобы определиться с~выбором пе\-ре\-да\-ва\-емой 
величины~$\overline{x}_l$, $l\hm= 1,2, \ldots ,n\hm-1$, всякий игрок~$l$ учитывает 
ра\-ци\-о\-наль\-ность поведения сле\-ду\-юще\-го игрока $l\hm+1$. Поэтому 
суперпозиция~(\ref{e2.1-vas}) строится в~порядке ходов~$\pi$. 
 Формулы~(\ref{e1.4-vas}) и~(\ref{e2.1-vas}) отвечают принципу \textit{динамического} 
программирования. 
    
    \textbf{Пример~2.2.} В~двухуровневой иерархической сис\-те\-ме ведущий 
игрок~1 пер\-вым делает свой ход~\cite{2-vas}. <<Подчиненные>> игроки $2, \ldots , 
n$ между собой не обмениваются информацией. В~за\-ви\-си\-мости от того, передает 
пер\-вый агент значение~$\overline{x}_1$ или нет, ре\-зуль\-ти\-ру\-ющее отношение 
со\-от\-вет\-ст\-ву\-ющей игры рав\-но (см.~(\ref{e1.1-vas}), (\ref{e1.4-vas}) и~(\ref{e2.1-vas})):
    \begin{multline}
    D_1=\varnothing\Rightarrow \rho^{\Gamma^{0;1,n-1}} =  \left( 
\coprod\limits_{l=2}^n a_l\right) \circ a_1 \\
  \mbox{либо\ }  D_1\not= \varnothing\Rightarrow \rho^{\Gamma^{1;1,n-1}} =a_1\circ \left( 
\coprod\limits^n_{l=2} a_l\right) .
\label{e2.2-vas}
    \end{multline}
    
    Подчиненные игроки могли бы объединиться в~коалицию Парето~$C$, сводя 
ре\-ша\-емую проб\-ле\-му к~иерархической игре двух лиц~(1 и~$C$) с~ре\-зуль\-ти\-ру\-ющим 
отношением $\rho_{\{1,C\} (\pi,\varnothing)} \hm= a_C \circ a_1$ или 
$\rho_{\{1,C\}(\pi,D)} \hm= a_1\circ a_C$ в~за\-ви\-си\-мости от того, множество 
$D_1\hm=\varnothing$ или нет. Здесь $a_C\hm= \rho_C\vert_{x_C}$~--- абсолютно 
оптимальное отношение коалиции (см.~(\ref{e1.2-vas})). 
    
    Рассмотрим теперь игры $\Gamma^{2,n}$, $n\hm\geq 2$, в~которых партнерам 
последовательно передаются данные вида $\tilde{x}_i \hm\in \tilde{D}_i$, 
$\tilde{x}_i: X\hm\to X_i$, $i\hm= 1,\ldots n\hm-1$~\cite{1-vas, 5-vas}. Обрат\-ная связь 
может приводить к~ситуациям \textit{равновесия}~(\ref{e1.6-vas}). 
    
    \textbf{Пример~2.3.} В~обоб\-щен\-ной игре  
Гермейера~$\Gamma^{2,2}$~\cite{1-vas} граф ходов и~характеристическое 
отношение коалиции $C(\pi, \tilde{D})$ име\-ют сле\-ду\-ющий вид:
    \begin{multline}
    \pi, \tilde{D}: \boxed{1} \overset{\tilde{x}_1}{\to} 
\boxed{2}\sim\rho^{\Gamma^{2,2}} = \rho_{C(\pi,\tilde{D})} ={}\\
    {}= \rho_2\circ \left( \rho_1\cup \rho_2^G\right), \quad \rho_2^G 
=\rho_2\vert_{\mathrm{MIN}\,\rho_2\vert_{X_1}}.
\label{e2.3-vas}
\end{multline}

    В формуле~(\ref{e2.3-vas}) использовано \textit{гарантированное} отношение 
второго игрока $\rho_2^G\hm\subset \rho_2$, с~по\-мощью которого срав\-ни\-ва\-ют\-ся 
результаты <<по\-слой\-ной>> минимизации отношения $\rho_2\vert_{X_1}$ при 
любом фиксированном па\-ра\-мет\-ре $x_2\hm\in X_2$:
    \begin{equation}
    \rho_2^G \triangleq \left\{ (x,x^\prime): x\rho_2 x^\prime;\enskip  x, x^\prime\in \mathrm{MIN}\,\rho_2\vert_{X_1}\right\}.
    \label{e2.4-vas}
    \end{equation}
В~тео\-ре\-ме~3.2 \mbox{изучен} общий случай сис\-те\-мы $\Gamma^{2,n}$, $n\hm\geq 2$,  
и~доказана формула~(\ref{e2.3-vas}). 

    \textbf{Пример~2.4.} В~двухуровневой иерархической игре~$\Gamma^{1;1,n-
1}$, $\pi \hm= \{ (1,2), \ldots , (1, n-1)\}$, ведущий игрок~1 не может сообщить  
стра\-те\-гию-функ\-цию  $\tilde{x}_1: X_2\times \cdots \times X_n\hm\to X_1$ своим 
партнерам $\{2,3,\ldots , n\}$. Отсутствие коммуникации меж\-ду ними ведет 
к~противоречию в~процессе принятия решений: в~подкоалициях $\{1,2\}, \ldots , \{1, 
n-1\}$ нельзя одновременно разыг\-рать игры~$\Gamma^{2,2}$. Класс 
функции~$\tilde{D}_1$ недопустим. Поэтому требуется изменить правила игры: 
игрок~1 сообщает парт\-не\-рам лишь значения $\overline{x}_1\hm\in D_1 \hm\subset 
\tilde{D}_1$. Тогда в~подкоалициях решаются подыг\-ры~$\Gamma^{1,2}$, 
а~в~целом~--- игра~$\Gamma^{1;1,n-1}$ (см.~(\ref{e2.1-vas}) и~(\ref{e2.2-vas})). 
Следовательно, интересы коалиции $C\hm= C(\pi, \tilde{D}_1)$ характеризуются 
отношением 
    \begin{equation}
    \rho^{\Gamma^{1;1,n-1}} =\rho_C= a_1\circ a_2\coprod \cdots \coprod a_1\circ 
a_n\,,
    \label{e2.5-vas}
    \end{equation}
    поэтому необходимо контролировать до\-пус\-ти\-мость применяемых стратегий. 
    
\section{Композициональность игры}

    Развиваемый подход базируется на свойстве ком\-по\-зи\-ци\-о\-наль\-ности бинарных 
отношений, при\-ме\-ня\-емых в~игровых операциях. Основой служит моноидальная 
категория бинарных отношений REL~[7, 8]. Объектами REL вы\-сту\-па\-ют 
конечные множества $X, Y, Z, \ldots$, а~морфизмами~--- бинарные отношения $\alpha: 
X\hm\to Y, \beta: Y\hm\to Z, \ldots ,$ заданные на произведениях их областей 
и~кообластей $X\times Y, Y\times Z, \ldots$ Напомним~\cite{8-vas}, что 
композицией морфизмов $\alpha: X\hm\to Y, \beta: Y\hm\to Z$ в~категории REL 
является суперпозиция $\alpha\circ \beta \hm\subset X\times Z$, $\alpha\circ\beta: 
X\hm\to Z$ этих отношений, \mbox{равная}
    \begin{equation}
    \alpha\circ\beta =\left\{ (x,z): \exists_{y\in Y} x\alpha y\wedge y\beta z\right\}.
    \label{e3.1-vas}
    \end{equation}
Вместо записи $(x,y)\hm\in \alpha\hm\subset X\times Y$ здесь использовано 
инфиксное обозначение $x\alpha y$. Единичными морфизмами для операции 
произведения~(\ref{e3.1-vas}) служат тривиальные порядки~$\sigma_X$ 
и~$\sigma_Y$, а~$\sigma_Z\hm= \{ (z,z): z\hm\in Z\}$ для любого мно\-же\-ст\-ва~$Z$. 

    В категории REL определена операция дизъюнктивной суммы морфизмов 
     $\alpha\coprod \beta : (X\coprod Y) \hm\to (Y\coprod Z)$, на\-зы\-ва\-емая так\-же 
моноидальным произведением: 
    \begin{equation}
    \alpha\coprod\beta =\left\{ (x,y;1): x\alpha y\right\} \cup \{ (y,z;2): y\beta z\}.
    \label{e3.2-vas}
    \end{equation}
Операции~(\ref{e3.1-vas}) и~(\ref{e3.2-vas}) ас\-со\-ци\-а\-тив\-ны, а~сумма~(\ref{e3.2-vas}) 
обладает свойством коммутативности и~имеет единицу~$\varnothing$. Поэтому 
REL пред\-став\-ля\-ет собой моноидальную категорию~\cite{8-vas}.

    В рассматриваемых игровых задачах у~каж\-до\-го морфизма совпадают области 
и~кообласти, например это так у~исходных отношений предпочтения игроков $\rho: 
X\hm\to X$. При по\-стро\-ении вспомогательных бинарных отношений по\-сред\-ст\-вом 
моноидального произведения $\rho_1 \coprod\rho_2$ изменяются области 
и~кообласти по\-лу\-ча\-емых морфизмов. Тем не менее всегда можно считать 
до\-пус\-ти\-мы\-ми произвольные композиции морфизмов. Так, под левыми час\-тя\-ми 
выражений 
    \begin{align*}
    \left(\gamma_1\coprod\gamma_2\right) \circ\gamma 
&=\gamma_1\circ\gamma\coprod\gamma_2\circ\gamma\,;\\
    \gamma\circ\left(\gamma_1\coprod\gamma_2\right)&=\gamma\circ\gamma_1\coprod 
\gamma\circ \gamma_2
    \end{align*}
следует понимать сле\-ду\-ющие композиции морфизмов соответственно: 
\begin{align*}
\left(\gamma_1\coprod\gamma_2\right)&\circ\left(\gamma\coprod\gamma\right); \\
\left(\gamma\coprod\gamma\right)&\circ\left( \gamma_1\coprod\gamma_2\right).
\end{align*}
    
    Вообще говоря, РОИ $\rho^{\Gamma(S)} 
\hm\subset Z^2$ определено на множестве $Z\hm= X(S)\not= X$. Поэтому найденное 
решение задачи~(\ref{e1.5-vas}) нуж\-но дополнительно преобразовать для получения 
до\-пус\-ти\-мой ситуации игры $x^*\hm\in X$. Это делается проектированием $Z\hm\to 
X$ (см.\ определение~(\ref{e3.2-vas})). 
    
    \subsection*{Задание правил игры}

    Договоренности между игроками могут приводить к~формированию коалиций 
до начала их ходов. \textit{Заранее} объ\-яв\-ля\-емые коалиции Парето (см.\ пример~1.2) 
или иерархические коалиции (см.\ примеры~2.1--2.4) имеют приоритет при 
выполнении ходов, приводящий к~изменению исходного отношения~$\pi$. После 
этого коалиции трактуются как отдельные игроки с~отношениями 
предпочтения~$\rho_C$ (см.~(\ref{e1.2-vas}) и~(\ref{e1.4-vas})). 
    
    Выбираемое агентами отношение предшествования ходов~$\pi$ управ\-ля\-ет 
созданием иерархически организованных коалиций в~процессе игры Оно должно 
обладать свойствами ан\-ти\-сим\-мет\-рич\-ности, реф\-лек\-сив\-ности 
    и~ацик\-лич\-ности~\cite{5-vas}. В~конечном итоге игра проходит между 
коалициями разных типов, интересы которых пред\-став\-ле\-ны характеристическими 
отношениями. Проходящие внут\-ри коалиций~$C$ подыгры наследуют отношение 
предшествования ходов, обозна\-ча\-емое~$\pi_C$.
    
    Всякая коалиция~$C$ действует как игрок, пе\-ре\-да\-ющий информацию 
$D_C\hm= \left\{ \tilde{x}_i, \overline{x}_j, i,j\hm\in C, i\not=j\right\}$ одновременно 
всем участникам конфликта, которые были адресатами сообщений ее членов. Если 
адресаты $r,k,\ldots$ вошли в~некоторые коалиции\linebreak $C^\prime\hm= C$, то данные 
$x_C\hm= \left\{ \tilde{x}_i,\overline{x}_j\right\}$ по\-сту\-па\-ют игроку~$C^\prime$ 
и~распределяются по на\-зна\-че\-нию. 
    
    В понятие стратегии входит участие игрока в~разработке правил игры, 
ини\-ци\-иро\-ва\-нии момента \textit{хода}, выборе конт\-ро\-ли\-ру\-емо\-го фак\-то\-ра 
и,~воз\-мож\-но, коммуникации~--- сообщении информации некоторым из партнеров. 
Размеченный пе\-ре\-да\-ва\-емы\-ми данными граф~$G_\pi$ отношения предшествования 
ходов назовем \textit{графом игры} (см.~(\ref{e1.3-vas}) и~(\ref{e2.3-vas})). Так как все 
участники конфликта заинтересованы в~выборе рационального решения, то они 
используют лишь \textit{допустимые} клас\-сы стратегий. Выбор графа игры должен 
сопровождаться анализом клас\-сов стратегий, которые намерены применять игроки, 
и~в~случае не\-об\-хо\-ди\-мости их корректировать. Изменение правил игры направлено 
на соблюдение требования не\-про\-ти\-во\-ре\-чи\-вости процесса поиска решения (см.\
пример~2.4). 
    
    Противоречие в~обмене данными устраняется с~по\-мощью \textit{допустимой} 
разметки графа игры. Никто из игроков не может одновременно сообщить 
парт\-не\-рам и~процедуру выбора фактора~$\tilde{x}$, и~его значение~$\overline{x}$. 
Для адресатов $j\hm\in T_i \hm= \{j: i\pi j, i\not=j\}$ игрока~$i$ либо дуги $i\hm\to j$ 
графа~$G_\pi$ вовсе не помечаются, если $\exists_l l\hm\in T_i \wedge j\pi l$ (см., 
например, (\ref{e1.3-vas})), либо имеют раз\-мет\-ку~$\overline{x}_i$, если мощ\-ность 
$\# T_i\hm>1$, или~$\tilde{x}_i$, если $\# T_i\hm=1$. 
    
    \textbf{Пример~3.1.} В~левой час\-ти формулы
    
    \vspace*{-29pt}
    
    \noindent
      \begin{equation} 
\setlength{\unitlength}{1mm}\thicklines %(3.3)-1
\begin{picture}(16,20)
\put(2,2.5){$\boxed{1}$}
\put(7,3.5){\vector(1,0){6}}
\put(14,2.5){$\boxed{2}$}
\put(8,5.5){\small{$\tilde{x}_1$}}
\put(4,0.5){\vector(0,-1){6}}
\put(0,-3){\small{$\tilde{x}_1$}}
\put(14,-3.5){\small{$\tilde{x}_2$}}
\put(2,-10){$\boxed{3}$}
\put(16,0.5){\vector(-4,-3){8}}
\end{picture}\enskip \enskip ; \quad %\enskip\enskip\enskip
\setlength{\unitlength}{1mm}\thicklines % (3.3)-2
\begin{picture}(16,20)
\put(0,2.5){$\boxed{1}$}
\put(5,3.5){\vector(1,0){6}}
\put(12,2.5){$\boxed{2}$}
\put(6,5.5){\small{$\tilde{x}_1$}}
\put(2,0.5){\vector(0,-1){6}}
\put(12,-3.5){\small{$\tilde{x}_2$}}
\put(0,-10){$\boxed{3}$}
\put(14,0.5){\vector(-4,-3){8}}
\end{picture}
\label{e3.3-vas}
\end{equation}

\vspace*{29pt}

\noindent
изображена 
не\-до\-пус\-ти\-мая раз\-мет\-ка графа~$G_\pi$. В~правой час\-ти~(\ref{e3.3-vas}) показан 
<<ис\-прав\-лен\-ный>> граф игры. 
  Разметка~$\tilde{x}_1$ дуги $1\hm\to 3$ допустима, но игнорирует заданный 
порядок ходов $(2,3)\hm\in \pi$. 

    \subsection*{Построение результирующего отношения игры}

    Пусть граф игры обладает до\-пус\-ти\-мой раз\-мет\-кой дуг. Докажем, что имеется 
РОИ~$\rho^{\Gamma(S)}$, которое однозначно 
определено в~соответствии с~правилами\linebreak игры, при этом оно пред\-став\-ля\-ет собой 
композицию исходных и~производных бинарных отношений в~категории REL. 
Индуктивное конструирование~$\rho^{\Gamma(S)}$ опирается на отношение 
\mbox{пред\-шест\-во\-ва\-ния} ходов~$\pi$, моделируя процесс формирования иерархических 
коалиций вслед\-ст\-вие коммуникации игроков.
    
    \smallskip
    
    \noindent
    \textbf{Теорема~3.1.} \textit{Во всякой игре существует ре\-зуль\-ти\-ру\-ющее 
отношение}. 
    
    \smallskip
    
    \noindent
    Д\,о\,к\,а\,з\,а\,т\,е\,л\,ь\,с\,т\,в\,о\,.\ \  Применим метод математической 
индукции, проводимой по чис\-лу участников игры. При $n\hm\leq 2$ 
РОИ существует (см.\ примеры~1.1--1.3 и~2.1--2.4). Если 
$\pi\hm=\sigma$, то для всех значений $n\hm= 1,2,\ldots$ ре\-зуль\-ти\-ру\-ющее 
отношение вы\-чис\-ля\-ет\-ся по фор\-му\-ле~(\ref{e1.1-vas}). 
    
    Пусть $\pi\not=\sigma$. Согласно следствию~1.1 и~ацик\-лич\-ности 
отношения~$\pi$ найдется элемент $i\hm\in \mathrm{MIN}\, \pi$, для которого $\exists_j (j\not= 
i) \wedge i\pi j$. Сформируем коалицию $C_1\hm= \{j: i\pi j\}$. По свойству 
реф\-лек\-сив\-ности~$\pi$ игрок $i\hm\in C_1$. Воспользуемся предположением 
индукции: существует характеристическое отношение~$\rho_{C_1}$. Не 
исключено, что при сле\-ду\-ющих ходах внут\-ри коалиции~$C_1$ будут образованы 
подкоалиции~$C_1^k$, $k\hm= 1,2,\ldots , K$, с~чис\-лом участников $r_k\hm <n$. 
Тогда~$\rho_{C_1}$ строится композицией морфизмов~$\rho_{C_1^k}$ категории 
REL.
    
    Построим новый граф игры~$G_{\pi^\prime}^\prime$. Пусть из вершины $i\hm\in G_\pi$ 
исходит~$l$, $l\hm\geq 2$, дуг, которые имеют до\-пус\-ти\-мую разметку. Заменим 
одним агентом коалицию~$C_1$. Ис\-клю\-чим из графа игры~$G_\pi$ все дуги 
$(i,j)\hm\in\pi$, от\-ве\-ча\-ющие коммуникациям внут\-ри коалиции~$C_1$. Из новой 
вершины $C_1\hm\in G_{\pi^\prime}^\prime$ проведем дуги $(C_1,r)\hm\in \pi^\prime$, где 
$(j,r)\hm\in \pi$, $j\hm\in C_1$. Таким образом, по\-стро\-ен наследник~$\pi^\prime$ 
отношения предшествования ходов~$\pi$. Разметим дуги $(C_1,r)$ так же, как 
и~$(j,r)\hm\in\pi$, обеспечив до\-пус\-ти\-мость графа игры~$G^\prime_{\pi^\prime}$. 
В~$\Gamma^\prime$ чис\-ло участников $m\hm<n$. По предположению индукции 
существует ре\-зуль\-ти\-ру\-ющее отношение~$\rho^{\Gamma^\prime}$, стро\-яще\-еся 
композицией соответствующих морфизмов. 
    
    Композиционный процесс завершается построением бескоалиционной игры 
$\pi^{(k)} \hm=\sigma$, ре\-зуль\-ти\-ру\-ющим отношением которой выступает 
моноидальное произведение~(\ref{e1.1-vas}). Про\-ил\-люст\-ри\-ру\-ем\linebreak тео\-ре\-му~3.1 на 
примере графа игры~(\ref{e3.3-vas}). Сначала в~подкоалиции $C\hm= \{1,2\}$ 
строится характеристическое отношение~(\ref{e2.3-vas}), а~затем для игры 
$C\hm\to 3$~--- ре\-зуль\-ти\-ру\-ющее (см.\ фор\-му\-лу~(\ref{e1.4-vas})):
    \begin{equation}
    \rho^{\Gamma^{2,3(\pi^\prime)}} =a_3\circ \left(\rho_2\circ \left( \rho_1\cup 
\rho_2^G\right)\right).
    \label{e3.3-1-vas}
    \end{equation}
    
    \textbf{Пример~3.2.} Рассмотрим граф игры~$\Gamma$, опи\-сы\-ва\-ющий 
функционирование трехуровневой иерархической сис\-темы

\vspace*{-26pt}

\noindent
\begin{center}
\setlength{\unitlength}{1mm}\thicklines %ПРИМЕР 3.2
\begin{picture}(28,16)
\put(0,-2){$\boxed{1}$}
\put(5,-1){\vector(1,0){6}}
\put(12,-2){$\boxed{2}$}
\put(6,1){\small{$\tilde{x}_1$}}
\put(17,0){\vector(4,3){6}}
\put(24,4){$\boxed{3}$}
\put(17,4){\small{$\bar{x}_2$}}
\put(17,-7){\small{$\bar{x}_2$}}
\put(24,-7.5){$\boxed{4}$}
\put(17,-1){\vector(4,-3){6}}
\end{picture}~~~~~.
\end{center}

  \vspace*{26pt} 

\noindent
На первом шаге построения отношения~$\rho^{\Gamma}$ формируется 
иерархическая коалиция $C_1\hm=\{1,2\}$, в~которой разыгрывается 
операция~$\Gamma^{2,2}$ с~па\-ра\-мет\-ра\-ми $x_3$ и~$x_4$. Ее характеристическое 
отношение~$\rho_{C_1}$ задается формулой~(\ref{e2.3-vas}). При втором ходе идет 
игра трех лиц $\{C_1, 3, 4\}$ (см.\ примеры~2.2 и~2.4 и~формулу~(\ref{e2.5-vas})). 
По\-этому 
\begin{multline}
\rho^{G} =\rho_{C_1} \circ a_C={}\\
{}=\left( \rho_2\circ\left( \rho_1\cup 
\rho_2^G\right)\right)\circ \left( \rho_3\coprod\rho_4\right)\Big\vert_{X_3\times X_4}.
\label{e3.4-vas}
\end{multline}
    
    Если в~правилах игры оговорить заранее, что игроки~3 и~4 создадут коалицию 
Парето~$C^P$ (см.~(\ref{e1.2-vas})), то второй ход приведет к~игре двух лиц 
$\Gamma^\prime\hm= \left\{ C_1, C^P,\pi^\prime: C_1\hm\to C^P\right\}$. 
Не\-оп\-ре\-де\-лен\-ность\linebreak выбора рационального решения уменьшится по сравнению 
с~оптимизацией~(\ref{e3.4-vas}), так как теперь исследуется ре\-зуль\-ти\-ру\-ющее 
отношение
    $$
    \rho^{\Gamma^\prime} =\rho_{C_1} \circ a_{C^P} = \left( 
\rho_2\circ\left(\rho_1\cup \rho_2^G\right)\right)\circ \rho_3\cap \rho_4.
    $$
    
    \textbf{Пример~3.3.} На первом шаге по\-стро\-ения~\mbox{РОИ}

\begin{center}
\setlength{\unitlength}{1mm}\thicklines % ПРИМЕР 3.3
\begin{picture}(18,18)
\put(2,12.5){$\boxed{1}$}
\put(13,13.5){\vector(-1,0){6}}
\put(14,12.5){$\boxed{2}$}
\put(8,15.5){\small{$\bar{x}_2$}}
\put(4,4.5){\vector(0,1){6}}
\put(0,7){\small{$\bar{x}_3$}}
\put(14,6.5){\small{$\bar{x}_2$}}
\put(2,0){$\boxed{3}$}
\put(16,10.5){\vector(-4,-3){8}}
\end{picture}
\end{center}


\noindent
участники объединяются в~одну иерархическую коалицию~$C_1$ с~ведущим 
игроком~2. На втором шаге формируется подкоалиция $C_1^1\hm= \{1,3\}$ 
с~игроком~1. Согласно примеру~2.1, $\rho_{C_1^1}\hm= a_3\circ a_1$. 
Характеристическое отношение коалиции $C_1\hm= \{2, C_1^1\}$ рав\-но 
$\rho^\Gamma \hm= \rho_{C_1} \hm= a_2\circ \rho_{C_1^1} \hm= a_2\circ a_3 \circ 
a_1$ (см.~(\ref{e2.1-vas})).
    
    \subsection*{Обобщение игры Гермейера}

    Докажем формулы~(\ref{e2.3-vas}) и~(\ref{e2.4-vas}). Под 
\textit{гарантированным} отношением игрока~2 будем понимать сле\-ду\-ющее 
суже\-ние его отношения предпочтения:
    \begin{equation}
    \rho_2^\Gamma =\rho_2\vert_{\mathrm{MIN}\, \rho_2\vert_{X_1}}.
    \label{e3.5-vas}
    \end{equation}
Пусть $\pi_{X_1}: X\hm\to X_1$~--- проектирование. Любая функция 
$\tilde{x}_1^{\mathrm{н}}: X_2\hm\to \pi_{X_1} \mathrm{MIN}\, \rho_2\vert_{X_1}$ называется 
стратегией \textit{наказания} 2-го игрока~\cite{1-vas}, а~$x_2^G\hm\in 
\pi_{X_2} \mathrm{MAX}\, \rho_2^{G}$~--- \textit{га\-ран\-ти\-ру\-ющей} стратегией. 
    
    \smallskip
    
    \noindent
    \textbf{Теорема~3.2.} \textit{Пусть $\rho_2$~--- транзитивное отношение. 
Тогда в~игре двух лиц $\Gamma^{2,2}$ ре\-зуль\-ти\-ру\-ющее отношение равно}
    \begin{equation}
    \rho^{\Gamma^{2,2}} =\rho_2\circ\left( \rho_1\cup \rho_2^G\right).
    \label{e3.6-vas}
    \end{equation}
    
    \noindent
    Д\,о\,к\,а\,з\,а\,т\,е\,л\,ь\,с\,т\,в\,о\,.\ \  Пусть $x_2^*\hm= \pi_{X_2} x^*$, 
$x^*\hm\in \mathrm{MAX}\,\rho_1$. Если игрок~2 выберет $x_2\not= x_2^*$, то игрок~1 
применит стратегию наказания~$\tilde{x}_1^{\mathrm{н}}$. Поэтому перед 
игроком~2 стоит альтернатива: либо применять $x_2\hm= x_2^*$, либо быть 
наказанным. Угроза действенна лишь при условии $(\tilde{x}_1^{\mathrm{н}}(x_2^G), x_2^G)\rho_2 x^*$, когда не помогает 
использование игроком~2 га\-ран\-ти\-ру\-ющей стратегии $x_2\hm= x_2^G$. В~самом 
деле, определение~$\rho_2^G$ и~тран\-зи\-тив\-ность~$\rho_2$ \mbox{дают}
    \begin{multline*}
    \left( \tilde{x}_1^{\mathrm{н}}(x_2), x_2\right)\rho_2 \left( 
\tilde{x}_1^{\mathrm{н}}(x_2^G),x_2^G\right) \wedge \left( \tilde{x}_1^{\mathrm{н}} 
(x_2^G), x_2^G\right) \rho_2 x^*\Rightarrow{}\hspace*{-2.17662pt}\\
{}\Rightarrow
    \left( \tilde{x}_1^{\mathrm{н}}(x_2),x_2\right) \rho_2 x^*.
    \end{multline*}
<<Выгоднее>> для игрока~2 вы\-брать~$x_2^*$, а~не~$x_2^G$. 
Отсюда~$\rho^{\Gamma^{2,2}}$ имеет вид~(\ref{e3.6-vas}) (см.~(\ref{e2.3-vas})). 

\smallskip

\noindent
\textbf{Следствие~3.1.}\ \textit{Иерархическая игра~$\Gamma^{2,n}$ с~$n\hm= 2k$ 
или $n\hm=2k\hm+ 1$ участ\-ни\-ками}
\begin{equation}
\boxed{1} \overset{\tilde{x}_1}{\to} \cdots \overset{\tilde{x}_r}{\to} 
\boxed{r+1}\overset{\tilde{x}_{r+1}}{\to} \cdots \overset{\tilde{x}_{n-1}}{\to} \boxed{n}
\label{e3.7-vas}
\end{equation} 
\textit{имеет следующие ре\-зуль\-ти\-ру\-ющие отношения}:
\begin{multline}
\rho^{\Gamma^{2,2k}} =\rho_{2k}\circ \left( a_{2k-1}\circ \rho^{\Gamma^{2,2k-2}} 
\cup \rho_{2k}^G\right);\\
\rho^{\Gamma^{2,2k+1}} = a_{2k+1}\circ \rho^{\Gamma^{2,2k}};\\
\rho^{\Gamma^{2,0}} =\sigma,\enskip k=1,2,\ldots
\label{e3.8-vas}
\end{multline}
    
    \noindent
    Д\,о\,к\,а\,з\,а\,т\,е\,л\,ь\,с\,т\,в\,о\,.\ \ В~со\-от\-вет\-ст\-вии с~тео\-ре\-мой~3.1 
с~каж\-дым ходом игроков происходит расширение коалиции $C_{k+1}\hm= C_k \cup 
\{k+1\}$, $C_1\hm= \{1\}$, и~поочередное применение формул~(\ref{e1.4-vas}) 
и~(\ref{e3.6-vas}) вы\-чис\-ле\-ния характеристических отношений. 
    
    \textbf{Пример~3.4.} Пусть в~игре~(\ref{e3.7-vas}) имеются предварительные 
договоренности о~формировании подкоалиций $C_{2k} \hm= \{2k-1,2k\}$, 
$\rho_{C_{2k}} \hm= \rho^{\Gamma^{2,2}}\vert_{X_{2k-1}\times X_{2k}}$ 
(см.~(\ref{e3.6-vas})). Тогда 
    \begin{multline*}
    \rho_C^{\Gamma^{2,2k}} =\rho_{2k} \circ\left( \rho_{2k-1} \cup \rho^G_{2k} 
\right)\circ\cdots\circ \rho_2 \circ\left( \rho_1\cup \rho_2^G\right);\\
    \rho_C^{\Gamma^{2,2k+1}} =a_{2k+1}\circ \rho_C^{\Gamma^{2,2k}}.
  \end{multline*}

\section{Поиск рационального решения игры}

    Композиционное строение отношения $\rho^{\Gamma(S)}$ со\-кра\-ща\-ет перебор 
в~задаче~(\ref{e1.5-vas}). 
    
    \smallskip
    
    \noindent
    \textbf{Теорема~4.1.} \textit{Имеют мес\-то свойства}
    \begin{equation}
    \left.
    \begin{array}{rl}
    \mathrm{MAX}\rho_2\vert_{\mathrm{MAX}\,\rho_1} &\subset \mathrm{MAX} \left(\rho_2\circ\rho_1\right);\\[6pt]
    \mathrm{MAX}\left(\rho_1\coprod\rho_2\right) &=\mathrm{MAX}\,\rho_1\cup \mathrm{MAX}\, \rho_2.
    \end{array}
    \right\}
    \label{e4.1-vas}
    \end{equation}
 \textit{Если отношения $\rho_1$ и~$\rho_2$ транзитивны и~рефлексивны, то} 
 $\mathrm{MAX}\,\rho_2\vert_{\mathrm{MAX}\,\rho_1} \hm= \mathrm{MAX} \left(\rho_2\circ\rho_1\right)$.
 
 \smallskip
 
 \noindent
    Д\,о\,к\,а\,з\,а\,т\,е\,л\,ь\,с\,т\,в\,о\,.\ \ Вторая из формул~(\ref{e4.1-vas}) 
непосредственно следует из определения~(\ref{e3.2-vas}) дизъ\-юнк\-тив\-ной суммы. 
Докажем пер\-вую. По определению максимального элемента $x^*\hm= \mathrm{MAX}\,\rho_2\vert_{\mathrm{MAX}\,\rho_1}$ \mbox{имеем}
    \begin{multline*}
    \forall_{x_1^*} \forall_x \left( \left( x_1^*,x\right)\in\rho_1 \Rightarrow 
x=x_1^*\right) \wedge{}\\
{}\wedge  \left(\left( x^*,x_1^*\right)\in\rho_2\Rightarrow x_1^*=x^*\right),
 \end{multline*}
что эквивалентно формуле 
$$
\forall_{x_1^*} \forall_x \left(\left( x_1^*, x\right)\in \rho_1\right) \wedge \left(\left( x^*, 
x_1^*\right)\in \rho_2\right) \Rightarrow x=x^*.
$$
Переписывая левую часть импликации в~форме суперпозиции отношений, приходим 
к~сле\-ду\-юще\-му вы\-воду: 
$$
\forall_x \left( x^*, x\right) \in \rho_2\circ\rho_1\Rightarrow x=x^*.
$$
 Значит, доказано, что 
$$
\mathrm{MAX}\,\rho_2\vert_{\mathrm{MAX}\,\rho_1} \subset \mathrm{MAX} \left(\rho_2\circ\rho_1\right).
$$
    
    Пусть отношения $\rho_1$ и~$\rho_2$ транзитивны и~рефлексивны. Включение 
$x^*\hm\in \mathrm{MAX}\,(\rho_2\circ\rho_1)$ рав\-но\-силь\-но фор\-муле 
    $$
    \forall_x \left( x^*,x\right) \in \rho_2\circ\rho_1 \Rightarrow x=x^*.
    $$
     Согласно определению композиции~(\ref{e3.1-vas}), имеем 
    $$
    \forall_x \exists_{x_1^*} \left( x^*, x_1^*\right) \in \rho_2\wedge \left( x_1^*, 
x\right) \in\rho_1\Rightarrow x=x^*.
    $$
     Как следствие, имеем
    $$
    \forall_x \forall_{x_1^*} \left( x^*, x_1^*\right) \in\rho_2\wedge \left( x_1^*, 
x\right) \in\rho_1\Rightarrow x=x^*.
    $$
Опираясь на реф\-лек\-сив\-ность отношения~$\rho_2$, под\-ста\-вим сюда $x_1^*\hm= 
x^*$. Тогда вер\-на формула 
$$
\left( x^*,x\right)\in\rho_1\hm\Rightarrow x= x^*,
$$ 
озна\-ча\-ющая $x^*\hm\in \mathrm{MAX}\,\rho_1$. Рас\-суж\-дая аналогично, под\-ста\-нов\-кой $x\hm= 
x_1^*$ получим, что $x^*\hm= \mathrm{MAX}\,\rho_2$ и~тем более $x^*\hm\in \mathrm{MAX}\,\rho_2\vert_{\mathrm{MAX}\,\rho_1}$. Итак, противоположное вложение 
$\mathrm{MAX}\,(\rho_2\circ\rho_1)\hm\subset \mathrm{MAX}\,\rho_2\vert_{\mathrm{MAX}\,\rho_1}$ так\-же имеет место. 
    
    Из соображений двой\-ст\-вен\-ности вытекает сле\-ду\-ющий результат.
    
    \smallskip
    
    \noindent
    \textbf{Следствие~4.1.} \textit{Справедливы соотношения}:
    \begin{equation}
    \left.
    \begin{array}{rl}
   \mathrm{MIN}\,\rho_2\vert_{\mathrm{MIN}\,\rho_1} &\subset \mathrm{MIN}\left( \rho_1\circ\rho_2\right),\\[6pt]
    \mathrm{MIN}\,\rho_1 \displaystyle \coprod \rho_2 &=\mathrm{MIN}\,\rho_1\cup \mathrm{MIN}\,\rho_2.
     \end{array}
     \right\}
    \label{e4.2-vas}
    \end{equation}
 \textit{Если $\rho_1$ и~$\rho_2$~--- транзитивные и~рефлексивные отношения, то} 
$\mathrm{MIN}\,\rho_2\vert_{\mathrm{MIN}\,\rho_1} \hm= \mathrm{MIN}\,(\rho_1\circ\rho_2)$.
    
    \smallskip
    
    В теореме~4.1 и~следствии~4.1 обобщен метод динамического 
программирования применительно к~бинарным отношениям. Воспользуемся 
свойством ас\-со\-циа\-тив\-ности композиции морфизмов и~формулами~(\ref{e1.4-vas}), 
(\ref{e2.1-vas}), (\ref{e3.6-vas}), (\ref{e3.8-vas})--(\ref{e4.2-vas}) (см.\ 
также пример~2.4), и~методом математической индукции докажем сле\-ду\-ющее 
    
    \smallskip
    
    \noindent
    \textbf{Следствие~4.2.} \textit{Рациональные решения игр $\Gamma^{0,n}$, 
$\Gamma^{1,n};$ $\Gamma^{0;1,n-1}$, $\Gamma^{1;1,n-1}$, $\Gamma^{2,n}$ 
удовле\-тво\-ря\-ют сле\-ду\-ющим свойствам}:
    \begin{equation}
    \left.
    \begin{array}{l}
    \mathrm{MAX}\, a_n\vert_{\mathrm{MAX}\, a_{n-1}\vert\cdots \vert_{\mathrm{MAX} \,a_1}} \subset \mathrm{MAX}\, 
\rho^{\Gamma^{0,n}};\\[6pt]
 \mathrm{MAX} \,a_1\vert_{\mathrm{MAX} \,a_2\vert\cdots {}_{\vert_{\mathrm{MAX}\, a_n}}}\subset 
\mathrm{MAX} \,\rho^{\Gamma^{1,n}}\,;\\[6pt]
    \mathop{\bigcup}\limits^n_{l=2} \mathrm{MAX} \,a_l\vert_{\mathrm{MAX} \,a_1} = \mathrm{MAX} \,\rho^{\Gamma^{0; 1,n-1}}; \\[6pt]
 \mathrm{MAX}\, a_1\vert_{\mathop{\bigcup}_{l=2}^n \mathrm{MAX} a_l} =\mathrm{MAX} \,\rho^{\Gamma^{1;1,n-1}}\,;\\[6pt]
    \mathrm{MAX} \,a_{2k+1}\vert_{\mathrm{MAX}\, \rho^{\Gamma^{2,2k}}} \subset \mathrm{MAX}\,\rho^{\Gamma^{2,2k+1}};\\[6pt]
 \mathrm{MAX}\,\rho_{2k}\vert_{\mathrm{MAX}\, a_{2k-1} \cup \mathrm{MAX}\,\rho^G_{2k}\big\vert_{\mathrm{MAX}\, \rho^{\Gamma^{2,2k-2}}}} \subset{}\\[6pt]
\hspace*{38mm}{}\subset \mathrm{MAX}\,  \rho^{\Gamma^{2,2k+2}}.
\end{array}
\right\}
\label{e4.3-vas}
    \end{equation}

    
    \textbf{Пример~4.1.} Рас\-смот\-рим игру~$\Gamma^{2,3}$, в~которой 
предпочтения участников являются рефлексивными транзитивными замыканиями 
сле\-ду\-ющих отношений: 
    \begin{align*}
    \rho_1&= \{ (0,1), (0,5), (4,0), (3,4), (3,2), (7,6)\}\,;\\
    \rho_2 &= \{ (0,2), (1,0), (2,4), (3,2), 3,5), (4,5), (6,4),\\
    &\hspace*{65mm} (6,7)\}\,;\\
    \rho_3 &= \{ (2,0), (4,0), (3,5), (3,1), (6,4), (7,6)\}\,,
    \end{align*}
заданных на бинарном кубе $\underline{8}\hm\simeq \{0,1\}^3$. Ситуации 
$(x_1,x_2,x_3)\hm\in \underline{8}$ пред\-став\-ле\-ны чис\-ла\-ми $x\hm\simeq x_1\hm+ 
2x_2\hm+ 4x_3$, записанными в~двоичной сис\-те\-ме. Ре\-зуль\-ти\-ру\-ющее отношение 
игры имеет вид~(\ref{e3.3-1-vas}) (см.~(\ref{e3.8-vas}), $k\hm=1$). Найдем 
рациональное решение~$x^*$, со\-кра\-тив перебор воз\-мож\-ных вариантов 
в~задаче~(\ref{e1.5-vas}) с~по\-мощью формул~(\ref{e4.3-vas}). 
    
    Характеристическое отношение коалиции $\{1,2\}$ равно $\rho_2\circ 
(\rho_1\cup \rho_2^G)$ (см.~(\ref{e3.5-vas}) и~(\ref{e3.6-vas})). Сначала вы\-чис\-лим 
$\rho_1\cup \rho_2^G \hm= \{ (0,1), (0,5), (4,0), (3,4), (3,2),\linebreak (7,6)\} \hm\cup \{ (1,4), (3,4), 
(6,4)\}$, а~затем~--- максимальные элементы отношений, входящих 
в~композицию~$\rho^{\Gamma^{2,3}}$: 

\vspace*{-3pt}

\noindent
     \begin{align*}
     M_1&\triangleq \mathrm{MAX}\, \rho_1\cup \rho_2^G =\{ 2,5\};\\
     M_2&\triangleq \mathrm{MAX}\,\rho_2\vert_{M_1} =\{5\};\\
     \mathrm{MAX}\, a_3\vert_{M_2} &=\{5\}\Rightarrow x^* =5\simeq (1,0,1).
   \end{align*}
   
   \vspace*{-10pt}

\section{Заключение}

\vspace*{-1pt}

    Композициональная структура РОИ многих лиц 
выражает формализацию правил игры, процессов формирования коалиций и~выбора 
игроками до\-пус\-ти\-мых стратегий. Подобное представление иг-\linebreak ры сокращает перебор 
при поиске рациональных решений. С~его по\-мощью получено обобщение 
классической тео\-ре\-мы Гермейера. Мо\-дуль\-ность композиции ре\-зуль\-ти\-ру\-юще\-го 
отношения упро\-щает разработку и~оптимизацию муль\-ти\-агент\-ных сис\-тем. 
Пред\-ло\-жен\-ный метод исследования и~чис\-лен\-но\-го решения игровой задачи 
нуж\-да\-ет\-ся в~алгоритмизации. 
    
{\small\frenchspacing
 {%\baselineskip=12pt
 %\addcontentsline{toc}{section}{References}
 \begin{thebibliography}{99}

\bibitem{2-vas}
\Au{Моисеев Н.\,Н.} Элементы теории оптимальных сис\-тем.~--- М.: Наука, 1974. 
526~с.

\bibitem{1-vas}
\Au{Гермейер Ю.\,Б.} Игры с~непротивоположными интересами.~--- М.: Наука, 1976. 
326~с.

\bibitem{3-vas}
\Au{Подиновский~В.\,В., Ногин~В.\,Д.} Па\-ре\-то-оп\-ти\-маль\-ные решения 
многокритериальных задач.~--- М.: Наука, 1982. 256~с.
\bibitem{4-vas}
\Au{Розен~В.\,В.} Применение тео\-рии бинарных отношений к~об\-щей тео\-рии игр~// 
Математические методы решения экономических задач.~--- Новосибирск: Наука, 1982. С.~127--152.
\bibitem{5-vas}
\Au{Васильев~Н.\,С.} Коалиционно устойчивые эффективные равновесия в~моделях 
коллективного поведения с~обменом информацией~// Информатика и~её 
применения, 2015. Т.~9. Вып.~2. С.~2--13. doi: 10.14357/19922264150201.
\bibitem{6-vas}
\Au{Bai~Q., Ren~F., Fujita~K., Znang~M.} Multi-agent and complex systems.~--- Studies in 
computational intelligence ser.~--- Luxembourg: Springer, 2016. 210~p.
\bibitem{7-vas}
\Au{Скорняков~Л.\,А.} Элементы общей ал\-геб\-ры.~--- М.: Наука, 1983. 272~с.
\bibitem{8-vas}
\Au{Маклейн~С.} Категории для ра\-бо\-та\-юще\-го математика~/
Пер. с~англ.~--- М.: Физматлит, 2004.  352~с.
(\Au{Mac Lane~S.} Categories for the working mathematician.~---  
Berlin\,--\,Heidelberg\,--\,New York: Springer, 1978. 317~p.)
\bibitem{9-vas}
\Au{Shoham~Y., Leyton-Brown~R.} Multiagent systems: Algorithmic, game-theoretic, and 
logical foundations.~--- Cambridge University Press, 2010. 532~p.

\bibitem{11-vas}
\Au{Dixit~A.\,K., Natebuff~B.\,J.} The art of strategy.~--- New York, London: 
W.\,W.~Norton \& Co., 2008. 446~p.

\bibitem{10-vas}
\Au{Dixit~A.\,K., Skeath~S., Reiley~D.\,H., Jr.} Games of strategy.~--- New York, 
London: W.\,W.~Norton \& Co., 2017. 880~p.

\end{thebibliography}

 }
 }

\end{multicols}

\vspace*{-8pt}

\hfill{\small\textit{Поступила в~редакцию 12.03.23}}

\vspace*{6pt}

%\pagebreak

%\newpage

%\vspace*{-28pt}

\hrule

\vspace*{2pt}

\hrule

%\vspace*{-2pt}

\def\tit{MULTIPLAYERS' GAMES COMPOSITIONAL STRUCTURE IN~THE~MONOIDAL CATEGORY 
OF BINARY RELATIONS}


\def\titkol{Multiplayers' games compositional structure in~the~monoidal category 
of binary relations}


\def\aut{N.\,S.~Vasilyev}

\def\autkol{N.\,S.~Vasilyev}

\titel{\tit}{\aut}{\autkol}{\titkol}

\vspace*{-14pt}


\noindent
N.\,E.~Bauman Moscow State Technical University, 5-1  Baumanskaya 2nd Str., Moscow 105005, 
Russian Federation


\def\leftfootline{\small{\textbf{\thepage}
\hfill INFORMATIKA I EE PRIMENENIYA~--- INFORMATICS AND
APPLICATIONS\ \ \ 2023\ \ \ volume~17\ \ \ issue\ 2}
}%
 \def\rightfootline{\small{INFORMATIKA I EE PRIMENENIYA~---
INFORMATICS AND APPLICATIONS\ \ \ 2023\ \ \ volume~17\ \ \ issue\ 2
\hfill \textbf{\thepage}}}

\vspace*{3pt}
     


\Abste{ The system approach is suggested for multiplayers' games solution that meets 
up-to-date network technologies. It allows to optimize the functionality of multiagent systems. 
The monoidal category of binery relations is applied to make games rules 
description and players' behavior study and modification. The game problem is to
maximize, if possible, the preference relations of all participants in the game. Their composition in the monoidal 
binary relations category in correspondence with games rules defines resulting game 
relation (RGR). Players' rational behavior search is reduced to RGR maximum elements 
choice. The author formalizes the use of various classes of permissible strategies, 
information exchange processes, and coalitions formation. The RGR existence is proved and 
maximum RGR elements structure is studied. Moves priority and absolutely optimal 
preference relations significance are clarified for the coalitions formation process.}

\KWE{player's preference relations: absolutely optimal relation, guarantied relation, moves priority relation;
game graph; permissible strategy; rational solution; 
coalition characteristic relation;  resulting game relation; monoidal category;  compositionality}

  \DOI{10.14357/19922264230203}{GPMZTS}

%\vspace*{-18pt}

%\Ack
%\noindent
  

%\vspace*{12pt}

  \begin{multicols}{2}

\renewcommand{\bibname}{\protect\rmfamily References}
%\renewcommand{\bibname}{\large\protect\rm References}

{\small\frenchspacing
 {%\baselineskip=10.8pt
 \addcontentsline{toc}{section}{References}
 \begin{thebibliography}{99} 

\bibitem{2-vas-1}
\Aue{Moiseev, N.\,N.} 1975. \textit{Ele\-men\-ty teo\-rii op\-ti\-mal'\-nykh sis\-tem} [Elements of optimal systems 
theory]. Moscow: Nauka. 527~p.

\bibitem{1-vas-1}
\Aue{Germeyer, Yu.\,B.} 1976. \textit{Ig\-ry s~nep\-ro\-ti\-vo\-po\-lozh\-ny\-mi in\-te\-re\-sa\-mi} 
 [Games with non-opposite interests]. Moscow: Nauka. 326~p.
 
\bibitem{3-vas-1}
\Aue{Podinovskiy, V.\,V., and V.\,D.~Nogin.} 1982. \textit{Pa\-re\-to-optimal'nye re\-she\-niya 
mno\-go\-kri\-te\-ri\-al'\-nykh za\-dach} [Pareto optimal solutions in multicriteria problems]. Moscow: Nauka. 
256~p.
\bibitem{4-vas-1}
\Aue{Rozen, V.\,V.} 1982. Pri\-me\-ne\-nie teo\-rii bi\-nar\-nykh ot\-no\-she\-niy k~ob\-shchey teo\-rii igr [Application 
of the theory of binary relations to general game theory]. \textit{Matematicheskie metody resheniya 
ekonomicheskikh zadach} [Mathematical methods for solving economic problems]. Novosibirsk: Nauka. 127--152.
\bibitem{5-vas-1}
\Aue{Vasilyev, N.\,S.} 2014. Ko\-a\-li\-tsi\-on\-no us\-toy\-chi\-vye ef\-fek\-tiv\-nye rav\-no\-ve\-siya v~mo\-de\-lyakh 
kol\-lek\-tiv\-no\-go po\-ve\-de\-niya s~ob\-me\-nom in\-for\-ma\-tsi\-ey [On availability of Pareto effective equilibrium 
situations in collective behavior models with data exchange]. \textit{Informatika i~ee Primeneniya~--- 
Inform. Appl.} 9(2):2--13. doi: 10.14357/19922264150201.
\bibitem{6-vas-1}
\Aue{Bai, Q., F.~Ren, K.~Fujita, and M.~Znang.} 2016. \textit{Multi-agent and complex systems}. 
Studies in computational intelligence ser. Luxembourg: Springer. 210~p.
\bibitem{7-vas-1}
\Aue{Skornyakov, L.\,A.} 1983. \textit{Ele\-men\-ty ob\-shchey al\-geb\-ry} [Elements of general algebra]. 
Moscow: Nauka. 272~p.
\bibitem{8-vas-1}
\Aue{Mac Lane, S.} 1978. \textit{Categories for the working mathematician}.  
Berlin\,--\,Heidelberg\,--\,New York: Springer. 317~p.
\bibitem{9-vas-1}
\Aue{Shoham, Y., and R.~Leyton-Brown.} 2010. \textit{Multiagent systems: Algorithmic, game-theoretic, 
and logical foundations}. Cambridge University Press. 532~p.

\bibitem{11-vas-1}
\Aue{Dixit, A.\,K., and B.\,J.~Nalebuff.} 2008. \textit{The art of strategy}. New York, London: 
W.\,W.~Norton \& Co. 446~p.

\bibitem{10-vas-1}
\Aue{Dixit, A.\,K., S.~Skeath, and  D.\,H.~Reiley, Jr.} 2017. \textit{Games of strategy}. New York, 
London: W.\,W.~Norton \& Co. 880~p.
\end{thebibliography}

 }
 }

\end{multicols}

\vspace*{-6pt}

\hfill{\small\textit{Received March 12, 2023}} 

\Contrl

\noindent
\textbf{Vasilyev Nikolai S.} (b.\ 1952)~--- Doctor of Science in physics and mathematics, professor, 
N.\,E.~Bauman Moscow State Technical University, 5-1~Baumanskaya 2nd Str., Moscow 105005, 
Russian Federation; \mbox{nik8519@yandex.ru}
     



\label{end\stat}

\renewcommand{\bibname}{\protect\rm Литература}       %3
%\newcommand {\ff}{{\mathcal F}}
\newcommand {\ebd}{\triangleq}
\newcommand{\me}[2]{\mathbf{E}_{ #1 }\left\{ \mathop{#2} \right\} }



\def\stat{borisov}

\def\tit{ФИЛЬТРАЦИЯ СОСТОЯНИЙ МАРКОВСКИХ СКАЧКООБРАЗНЫХ ПРОЦЕССОВ 
ПО~ДИСКРЕТИЗОВАННЫМ НАБЛЮДЕНИЯМ$^*$}

\def\titkol{Фильтрация состояний марковских скачкообразных процессов 
по~дискретизованным наблюдениям}

\def\aut{А.\,В.~Борисов$^1$}

\def\autkol{А.\,В.~Борисов}

\titel{\tit}{\aut}{\autkol}{\titkol}

\index{Борисов А.\,В.}
\index{Borisov A.\,A.}




{\renewcommand{\thefootnote}{\fnsymbol{footnote}} \footnotetext[1]
{Работа выполнена при частичной поддержке РФФИ (проект 16-07-00677).}}


\renewcommand{\thefootnote}{\arabic{footnote}}
\footnotetext[1]{Институт проблем информатики Федерального исследовательского центра <<Информатика 
и~управление>> Российской академии наук,
\mbox{aborisov@frccsc.ru}}

%\vspace*{8pt}



\Abst{Статья посвящена решению задачи оптимальной 
фильтрации состояний однородного марковского скачкообразного процесса (МСП). 
Наблюдения представляют собой приращения случайных процессов~--- интегральных 
преобразований состояний, зашумленные винеровскими процессами, интенсивность 
которых также зависит от оцениваемого состояния. Оптимальная оценка в~моменты 
получения нового наблюдения вычисляется как функция предыдущей оценки и~новых 
наблюдений, а~между моментами наблюдений~--- простейшим прогнозом в~силу системы 
уравнений Колмогорова. Рекуррентная формула пересчета ресурсозатратна, так как 
содержит  интегралы~--- мас\-штаб\-но-сдви\-го\-вые смеси многомерных гауссиан, 
где в~качестве смешивающих выступают распределения времени пребывания 
состояния в~каждом из возможных значений. Предложены более простые аппроксимации, 
основанные на предположении об ограниченности числа скачков состояния за время между 
наблюдениями. Получены универсальные локальная и~глобальная характеристики точности 
аппроксимаций, зависящие от па\-ра\-мет\-ров оцениваемого процесса, величины 
временн$\acute{\mbox{о}}$го шага  между наблюдениями и~максимального числа учитываемых скачков.}

\KW{марковский скачкообразный процесс; оптимальная фильтрация; мультипликативные 
шумы в~наблюдениях; стохастическое дифференциальное уравнение; численная аппроксимация}

\DOI{10.14357/19922264180316}
  
%\vspace*{4pt}


\vskip 10pt plus 9pt minus 6pt

\thispagestyle{headings}

\begin{multicols}{2}

\label{st\stat}



 \section{Введение}
 
 Фильтр Вонэма~\cite{Won_65}~--- один из редких удачных случаев, когда 
 оценка оптимальной фильтрации состо\-яния стохастической системы наблюдения 
 выражается в~виде решения некоторой замк\-ну\-той\linebreak конечномерной сис\-те\-мы 
 стохастических дифференциальных уравнений. 
 
 Алгоритм данного фильт\-ра 
 позволяет вычислить оценку фильт\-ра\-ции со\-сто\-яния \textit{марковского скачкообразного 
 процесса} с~\mbox{конечным} множеством состояний по наблюдениям в~присутствии 
 аддитивных винеровских шумов. Теоретически оптимальная оценка со\-сто\-яния~--- 
 его условное распределение в~текущий момент времени~--- 
 обладает очевидными свойствами неотрицательности и~нормировки. 
 При чис\-лен\-ной реализации данного фильтра классическим методом 
 Эй\-ле\-ра--Ма\-ру\-ямы~\cite{KP_92} данные свойства могут не сохраняться и~процедура 
 вы\-чис\-ле\-ния становится неустойчивой.  В~связи с~этим обстоятельством разрабатывались 
 другие алгоритмы чис\-лен\-но\-го решения уравнения фильтра Вонэма, обладающие 
 требуемыми свойствами устойчивости (см.~\cite{YZL_04, PR_10} и~библиографию в~них). 
 В~час\-ти этих работ доказана лишь слабая сходимость пред\-ла\-га\-емых аппроксимационных 
 схем к~оценке фильт\-ра Вонэма, в~то время как ка\-кая-ли\-бо 
 характеризация точ\-ности этих приближений отсутствует.
 
 В~\cite{B_18} было представлено распространение фильт\-ра Вонэма на случай 
 наблюдений с~мультипликативными шумами. При этом уравнение обобщенного 
 фильт\-ра содержит в~правой части квадратическую характеристику шумов в~наблюдениях. 
 Данный процесс на практике никогда не наблюдается непосредственно, а~является лишь 
 некоторым нелинейным интегральным преобразованием наблюдений. Очевидно, что 
 имеющиеся в~настоящий момент времени алгоритмы приближенного вычисления оценки 
 фильтрации Вонэма для данной системы не подходят. 
 
 Целью предлагаемой работы является ис\-поль\-зование результатов оптимальной 
 фильтрации со\-стояний сис\-тем с~дискретным временем для аппроксимации решения 
 аналогичной задачи для\linebreak стохастических дифференциальных сис\-тем. 
 
 Статья организована следующим образом. Раздел~2 содержит формальную постановку 
 задачи фильт\-ра\-ции со\-сто\-яний однородного МСП с~конечным множеством со\-сто\-яний 
 по наблюдениям, полученным путем временн$\acute{\mbox{о}}$й дискретизации процессов с~непрерывным 
 временем~--- интегральных преобразований со\-сто\-яния сис\-те\-мы в~присутствии 
 мультипликативных винеровских шумов.\linebreak
  В~разд.~3 пред\-став\-ле\-но решение поставленной 
 задачи фильт\-ра\-ции: пересчет оценок со\-сто\-яний в~момент получения новых 
 дискретизованных наблюдений выполняется в~соответствии с~некоторыми\linebreak 
 рекуррентными интегральными соотношениями, в~то время как между 
 моментами наблюдений оценка корректируется в~соответствии с~прогнозом в~силу 
 сис\-те\-мы уравнений Колмогорова. Вы\-чис\-ли\-тель\-ная слож\-ность 
 упомянутых выше интегральных\linebreak 
 соотношений связана с~тем, что в~расчет принимается воз\-мож\-ность того, что между 
 моментами наблюдений оцениваемое со\-сто\-яние может совершить произвольное чис\-ло 
 скачков. В~разд.~4 пред\-став\-лен более простой алгоритм приближенного вы\-чис\-ле\-ния 
 оценки фильт\-ра\-ции, основанный на ограничении возможного числа учитываемых скачков 
 МСП. Доказана тео\-ре\-ма, опре\-де\-ля\-ющая как\linebreak
  локальную (одношаговую), так и~глобальную 
 (многошаговую) характеристики точ\-ности предложенного при\-бли\-же\-ния~--- 
 $\ell_1$-нор\-мы ошибки аппроксимации. Полученные характеристики являются\linebreak 
 универсальными, т.\,е.\ не асимптотическими по шагу дискретизации, и~зависят от характеристик 
 самого МСП, %\linebreak
  шага временн$\acute{\mbox{о}}$й дискретизации и~чис\-ла
  скачков со\-сто\-яния, учи\-ты\-ва\-емых 
 на шаге. Об\-суж\-де\-ние результатов и~заключительные комментарии пред\-став\-ле\-ны 
 в~разд.~5.
 
 \section{Постановка задачи фильтрации}
 
 На полном вероятностном пространстве с~фильт\-ра\-цией 
 $(\Omega,\mathcal{F},\mathcal{P},\{\mathcal{F}_{t}\}_{t \geqslant 0})$ рассматривается система наблюдений
\begin{equation}
 \left.
 \begin{array}{rl}
 \displaystyle X_t &=X_0 +  \displaystyle
 \int\limits_0^t \Lambda^{\top}X_{s}\,ds + \mu_s\,;  \\[6pt]
 \displaystyle Y_k &=  \displaystyle\int\limits_{t_{k-1}}^{t_k}fX_s\,ds+
 \int\limits_{t_{k-1}}^{t_k} 
 \sum\limits_{n=1}^NX_s^ng_n \,dW_s, \\[6pt]
 &\hspace*{10mm}\{t_k\}_{k \geqslant 0}: \; 0 = t_0 < t_1 < t_2\cdots,
 \end{array}
 \right\}
 \label{eq:obsys_1}
 \end{equation}
 где
  \begin{itemize}
  \item
  $X_t \ebd \mathrm{col}\left(X_t^1,\ldots,X_t^N\right) \hm\in \mathbb{S}^N$~--- 
  ненаблюда\-емое состояние системы, являющееся однородным МСП с~конечным 
  множеством состояний $ \mathbb{S}^N \ebd$\linebreak $\ebd \{e_1,\ldots,e_N\}$ ($\mathbb{S}^N$~--- 
  множество единичных векторов евклидова пространства~$\mathbb{R}^N$), 
  матрицей интенсивностей переходов~$\Lambda$ и~начальным распределением~$\pi$;
  \item
  $\mu_t \ebd \mathrm{col}\left(
  \mu_t^1,\ldots,\mu_t^N\right)\hm\in \mathbb{R}^N$~--- 
  ${\mathcal{F}}_t$-со\-гла\-со\-ван\-ный мартингал;
  \item
  $\{Y_k\}_{k \in \mathbb{N}}:\;  Y_k \ebd \mathrm{col}\left(Y_k^1,\ldots,Y_k^M\right) 
  \hm\in \mathbb{R}^M$~--- последовательность дискретизованных наблюдений, 
  доступных в~известные неслучайные  моменты времени~$\{t_k\}_{k \in \mathbb{N}}$,
в~которых $W_t \ebd$\linebreak $\ebd \mathrm{col}\left(W_t^1,\ldots,W_t^M\right) \hm\in \mathbb{R}^M$
 является ${\mathcal{F}}_t$-со\-гла\-со\-ван\-ным стандартным винеровским процессом, 
 определяющим шумы в~наблюдениях,\linebreak  $f$~--- $(M \times N)$-мер\-ная 
 мат\-ри\-ца плана наблюдений, а~набор мат\-риц~$\{g_n\}_{n=\overline{1,N}}$ 
 характеризует интенсивности шумов в~зависимости от текущего состояния~$X_t$.
  \end{itemize}
  
  Введем также в~рассмотрение неубывающие семейства $\sigma$-ал\-гебр 
  $\mathcal{O}_k \ebd \sigma\{ Y_{\ell}: \; 1 \hm\leqslant \ell \hm\leqslant k\}$ 
  и~$\mathcal{O}_t \ebd  \mathcal{O}_{k(t)}$, где 
  $k(t) \ebd \sum\nolimits_{j \in \mathbb{N}}\mathbf{I}(t-t_{j})$; 
  $\mathcal{O}_0 \ebd \{\varnothing,\; \Omega\}$.
  
   \textit{Задача оптимальной фильтрации состояния~$X$ по наблюдениям~$Y$} 
   заключается в~нахождении \textit{условного математического ожидания} (УМО)
  \begin{equation*}
  \widehat{X}_t \ebd {\sf E}\left\{X_t|\mathcal{O}_{t} \right\}\,.
 % \label{eq:fest_1}
  \end{equation*}
  
  Относительно системы~(\ref{eq:obsys_1})  сделаны следующие предположения:
   \begin{itemize}
 \item[(а)]
 ${\mathcal{F}}_t \equiv {\mathcal{F}}_{t}^X \bigvee 
 {\mathcal{F}}_{t}^W $ для любого $t \hm\geqslant 0$;
 \item[(б)]
 шумы в~наблюдениях равномерно невырожденные, т.\,е.\
  $g_ng_n^{\top} \hm\geqslant \alpha I \hm> 0$ для всех $n\hm=\overline{1,N}$ 
  и~некоторого $\alpha\hm>0$.
% \item
 % Верно неравенство
  %\begin{equation}
  %\min_{1\leqslant k \leqslant N}|\lambda_{kk}| > 0.
  %\label{eq:ineq_0}
  % \end{equation}
 %\item
 %Для любого $t \geqslant 0$ все компоненты вектора $p_t \ebd \me{}{X_t}$ строго %положительны. 
 \end{itemize} 

 \section{Уравнения оптимального фильтра} 
 
 Для получения уравнений оптимального фильт\-ра воспользуемся подходом, 
 применяемым для решения аналогичной задачи в~стохастических сис\-те\-мах 
 наблюдения с~дискретным временем~\cite{BSh_85}. 
 Воспользу\-ем\-ся методом математической индукции. 
 
 При $r=0$ 
 \begin{equation}
 \widehat{X}_{t_0}={\sf E}\{X_0|\mathcal{O}_0\}={\sf E}\{X_0\}=\pi\,.
 \label{eq:in_cond}
 \end{equation} 
 
 Пусть для некоторого $ r \hm\geqslant 0$ известна оценка оптимальной 
 фильтрации~$\widehat{X}_{t_r} \hm= {\sf E}{X_{t_r} |\mathcal{O}_r}$. 
 Определим оценку оптимальной фильтрации~$\widehat{X}_{t} $ для $t\hm \in (t_r,t_{r+1}]$. 
 
 Для произвольного момента $t \hm\in (t_r,t_{r+1})$ в~силу мартингального 
 разложения МСП~$X_t$ и~свойств УМО верна следующая цепочка равенств:
 \begin{multline*}
 \widehat{X}_{t} = {\sf E}\left\{X_t | \mathcal{O}_r\right\}={}\\
 {}=
 {\sf E}\left\{{\cal P}^{\top}(t_r,t)X_{t_r}+
 \int\limits_{t_r}^t{\cal P}^{\top}(t_r,s)\,dM_s\big\vert \mathcal{O}_r\right\} = {}
\end{multline*}

\noindent
   \begin{multline}
 \hspace*{-11.66pt}{}=\mathcal{P}^{\top}(t_r,t)\widehat{X}_{t_r} + {\sf E}\hspace*{-2pt}
 \left\{{\sf E}\hspace*{-2pt}\left\{\int\limits_{t_r}^t\hspace*{-2pt}\mathcal{P}^{\top}(t_r,s)\,dM_s |
 {\mathcal{F}}_{t_r}\right\}\!\big\vert 
 \mathcal{O}_r\!\right\} ={}\hspace*{-4.24124pt}\\
 {}=
  \mathcal{P}^{\top}(t_r,t)\widehat{X}_{t_r}\,,
 \label{eq:bw_obs}
 \end{multline}
 где $\mathcal{P}(s,t)$ $(s \hm\leqslant t)$~--- матрица переходной ве\-ро\-ят\-ности МСП 
 на промежутке $[s,t]$, являющаяся решением сис\-те\-мы дифференциальных 
 уравнений Колмогорова
 \begin{equation*}
 \mathcal{P}'_t(s,t) = \mathcal{P}(s,t) \Lambda, \enskip t > s, \enskip \mathcal{P}(s,s) = I.
 \end{equation*}
 В случае однородного МСП $\mathcal{P}(s,t) \hm= e^{(t-s)\Lambda}$.
 
 Далее необходимо определить совместное распределение $(X_{t_{r+1}},Y_{r+1})$ 
 относительно~$ \mathcal{O}_r$. Из модели наблюдений следует, что 
 распределение~$Y_{r+1}$ относительно 
 $\sigma$-ал\-геб\-ры~$\mathcal{F}^X_{t_{r+1}} \vee \mathcal{O}_r$~---
 гауссовское с~параметрами 
 \begin{align*}
{\sf E}\left\{Y_{r+1}|{\mathcal{F}}^X_{t_{r+1}}\right\}& = f \tau_{r+1}\,; \\[6pt]
 \mathrm{cov} \left(Y_{r+1},Y_{r+1}|{\mathcal{F}}^X_{t_{r+1}}\right) &= 
 \displaystyle\sum\limits_{n=1}^N \tau_{r+1}^n g_ng_n^{\top}\,,
% \label{eq:occup_1}
 \end{align*}
 где $\tau_{r+1} \hm= \tau_{r+1}(X(\omega))=
 \mathrm{col}\left(\tau_{r+1}^1,\ldots,\tau_{r+1}^N\right) \ebd$\linebreak
 $\ebd 
 \int\nolimits_{t_r}^{t_{r+1}}X_s\,ds$~--- случайный вектор, $n$-я 
 компонента которого равна времени пребывания процесса~$X$ в~со\-сто\-янии~$e_n$ 
 на  интервале времени $[t_r, t_{r+1}]$. 
 Обозначим через $\mathcal{D}_{r+1} \ebd \{u=\mathrm{col}\,(u^1,\ldots,u^N):\; 
 u_m \hm\geqslant 0,\; \sum\nolimits_{m=1}^Mu_m\hm= t_{r+1}-t_r\}$ $(M-1)$-мер\-ный 
 симплекс в~пространстве~$\mathbb{R}^M$, являющийся носителем распределения 
 вектора~$\tau_{r+1}$. Пусть $\rho^{k,\ell}_{r+1}(du)$~--- 
 распределение вектора $\tau_{r+1} X_{t_{r+1}}^{\ell}$ при условии $X_{t_r}\hm=e_k$, 
 т.\,е.\ 
 для любого $\mathcal{A} \hm\in \mathcal{B}(\mathbb{R}^M)$ верно тождество:
\begin{multline*}
 \mathbf{P}\left\{\omega: \; X_{t_{r+1}}(\omega)=e_{\ell},\right.\\
 \left. 
 \tau_{r+1}(X(\omega)) \in \mathcal{A}\;|\;X_{t_r}=e_k\right\} \equiv
   \rho^{k,\ell}_{r+1}(\mathcal{A})\,.
\end{multline*}
 
Обозначим через
\begin{multline*}
 \mathcal{N}(y,m,K) \ebd (2\pi)^{-M/2} \mathrm{ det}^{-1/2} K\times{}\\
 {}\times\exp
 \left\{ -\fr{1}{2}\left(y-m\right)^{\top}K^{-1}(y-m)\right\}
\end{multline*}
 $M$-мер\-ную плот\-ность гауссовского распределения с~математическим 
 ожиданием~$m$ и~ковариационной матрицей~$K$.
 
 Из марковского свойства  $\{X_{t_{r}},Y_{r})\}_{r \geqslant 0}$ 
 относительно~${\mathcal{F}}_{t_{r}}$~\cite{ZhSh_95} и~теоремы Фубини следует, что 
 для любого  множества $\mathcal{A} \hm\in \mathcal{B}(\mathbb{R}^M)$ 
 верна следующая цепочка равенств:
 \begin{multline*}
 {\sf E}\left\{X_{t_{r+1}}\mathbf{I}_{\mathcal{A}}
 \left(Y_{r+1}\right)\big|\mathcal{O}_r\right\}={}\\
 {}=
{\sf E}\left\{{\sf E}\left\{X_{t_{r+1}}\mathbf{I}_{\mathcal{A}}
\left(Y_{r+1}\right)\big|
\mathcal{F}^X_{t_{r+1}} \vee \mathcal{O}_r\right\}
 \big|\mathcal{O}_r\right\} = {}
\end{multline*}

\noindent
\begin{multline*}
 %{}=
% {\sf E}\left\{{\sf E}\left\{X_{t_{r+1}}\mathbf{I}_{\mathcal{A}}
% \left(Y_{r+1}\right)\vert X_{t_r}\right\}
% \vert\mathcal{O}_r\right\} = {}\\
% {}=
%{\sf E}\left\{\sum\limits_{k=1}^N {\sf E}\left\{X_{t_{r+1}}\mathbf{I}_{\mathcal{A}}
%\left(Y_{r+1}\right)  \big| X_{t_r}=e_k\right\}X_{t_r}^k
% \big|\mathcal{O}_r\right\} = {}\\ 
% {}=
% \sum\limits_{k=1}^N{\sf E}
% \left\{X_{t_{r+1}}\mathbf{I}_{\mathcal{A}}\left(Y_{r+1}\right)\bigl| X_{t_r}=e_k\right\} 
% \widehat{X}_{t_r}^k ={}\\
% {}=\!
% \sum\limits_{k=1}^N{\sf E}
% \left\{{\sf E}\left\{X_{t_{r+1}}\mathbf{I}_{\mathcal{A}}
% \left(Y_{r+1}\right)\!\bigl| {\mathcal{F}}_{t_{r+1}}\right\}\!\bigl| 
% X_{t_r}\!=e_k\right\} \widehat{X}_{t_r}^k ={}\\
% {}=
% \sum\limits_{k=1}^N {\sf E}\left\{
% \vphantom{\int\limits_A\left(\sum\limits_{p=1}^N\right)}
% X_{t_{r+1}} \times{}\right.\\
% {}\times\int\limits_{\mathcal{A}}  
% \mathcal{N}\left(y,f \tau_{r+1}(X),\sum\limits_{p=1}^N \tau_{r+1}^p(X) g_pg_p^{\top}\right)dy
% \Biggl| X_{t_r}={}\\
%\left. {}=e_k
% \vphantom{\int\limits_A\left(\sum\limits_{p=1}^N\right)}
%\right\} \widehat{X}_{t_r}^k = 
% \sum\limits_{k=1}^N \int\limits_{\mathcal{A}}{\sf E}\left\{ 
% \vphantom{\sum\limits_{p=1}^N}
% X_{t_{r+1}} \times{}\right.\\
% {}\times\mathcal{N}\left(y,f \tau_{r+1}(X),\sum\limits_{p=1}^N \tau_{r+1}^p(X) 
% g_p g_p^{\top}\right)
% \Biggl| X_{t_r}={}\\
%\left. {}=e_k
%\vphantom{\sum\limits^N_{p=1}}
%\right\} \widehat{X}_{t_r}^k\, dy
 %={}\\
 {}=
 \sum\limits_{\ell=1}^N e_{\ell} \int\limits_{\mathcal{A}} 
 \left[ \sum\limits_{k=1}^N 
 \int\limits_{\mathcal{D}_{r+1}} 
 \mathcal{N}\left(y,f u,\sum_{p=1}^N u^p g_pg_p^{\top}\right)\times{}\right.\\
\left. {}\times
 \rho^{k,\ell}_{r+1}(du)\widehat{X}_{t_r}^k
 \vphantom{\int\limits_A\sum\limits_{p=1}^N}
 \right] 
 dy,
 \end{multline*}
 из чего следует, что интегранд в~квадратных скобках в~последнем выражении 
 определяет искомое совместное распределение $(X_{t_{r+1}},Y_{r+1})$ 
 относительно~$ \mathcal{O}_r$. Оценка~$\widehat{X}_{t_{r+1}}$ покомпонентно 
 определяется~\cite{BSh_85} с~помощью обобщенного варианта формулы Байеса:
 \begin{multline}
 \widehat{X}_{t_{r+1}}^j = {}\\
 \hspace*{-1mm}{}=
 \fr{\int\nolimits_{\mathcal{D}_{r+1}}\hspace*{-6mm} 
 \mathcal{N}\left(Y_{r+1},f u,\sum\nolimits_{p=1}^N \hspace*{-2mm}
 u^p g_pg_p^{\top}\!\right)\hspace*{-1mm}
 \sum\nolimits_{k=1}^N \hspace*{-2mm}
 \widehat{X}_{t_r}^k
 \rho^{k,j}_{r+1}(du)
 }
 { \int\nolimits_{\mathcal{D}_{r+1}} \hspace*{-6mm}
 \mathcal{N}\left(Y_{r+1},f v,\sum\nolimits_{q=1}^N \hspace*{-2mm}
 v^q g_qg_q^{\top}\!\right)\hspace*{-1mm}
 \sum\nolimits_{i,\ell=1}^N \hspace*{-2mm}
 \widehat{X}_{t_r}^i
 \rho^{i,\ell}_{r+1}(dv)
  },  \\ 
  j = \overline{1,N}\,.
 \label{eq:filt_1}
 \end{multline}
 Таким образом, доказана следующая
 
 %\smallskip
 
 \noindent
 \textbf{Лемма~1.}
\textit{Если для системы наблюдения}~(\ref{eq:obsys_1}) 
\textit{верны условия~(а) и~(б), то оценка~$\widehat{X}_t$ оптимальной фильтрации 
определяется формулой}~(\ref{eq:in_cond}) 
\textit{при $t\hm=0$, рекуррентным соотношением}~(\ref{eq:filt_1})~---
\textit{в~моменты~$t_{r+1}$ получения наблюдений~$Y_{r+1}$ 
и~формулой}~(\ref{eq:bw_obs})~--- 
\textit{в~промежутках времени между моментами получения наблюдений}.


\smallskip
 

 
 Несмотря на компактную запись~(\ref{eq:filt_1}), их прямая численная реализация 
 ресурсозатратна. Во-пер\-вых, в~(\ref{eq:filt_1}) требуется вычислять 
 распределения мас\-штаб\-но-сдви\-го\-вых смесей многомерных нормальных 
 распределений, что является трудоемкой\linebreak процедурой. Во-вто\-рых, 
 распределения~$\rho^{k,j}_{r+1}$ вре-\linebreak мени пребывания представляют собой 
 сумму\linebreak бесконечного ряда, слагаемые которого вычис\-ляются с~помощью 
 некоторой рекуррентной про\-це\-дуры~\cite{S_00}. В-третьих, 
 распределения~$\rho^{k,j}_{r+1}$ не являются абсолютно непрерывными 
 относительно меры Ле\-бега.
 { %\looseness=1
 
 }
 
 Следующий раздел посвящен численной аппроксимации~(\ref{eq:filt_1}) и~исследованию 
 ее точностных характеристик.
 
 \section{Приближенное вычисление оценки фильтрации}
 
 Без ограничения общности будем считать, что сетка~$\{t_r\}_{r \geqslant 0}$ 
 является равномерной с~шагом~$\Delta$, т.\,е.\ $t_r \hm= r \Delta$ 
 и~$\mathcal{D}_r \hm\equiv \mathcal{D}$.
 Обозначим через~$N_{r+1}$ об-\linebreak\vspace*{-12pt}
 
 \pagebreak
 
 \noindent
 щее число скачков процесса~$X_t$, имевших место 
 на промежутке $(t_r,t_{r+1}]$. Тогда из формулы полной вероятности следует, 
 что~(\ref{eq:filt_1}) представима в~виде:
 \begin{multline}
 \widehat{X}_{t_{r+1}}^j =  \left(
 \int\limits_{\mathcal{D}} 
 \mathcal{N}\left(Y_{r+1},f u,\sum\limits_{p=1}^N u^p g_pg_p^{\top}\right)\times{}\right.\\
\left. {}\times
 \sum\limits_{h=0}^{\infty}\sum\limits_{k=1}^N \widehat{X}_{t_r}^k
 \rho^{k,j,h}_{r+1}(du)
 \right)\Bigg/ \\
 \left(
 \vphantom{\sum\limits_{m=0}^{\infty}
 \sum\limits_{i,\ell=1}^N \widehat{X}_{t_r}^i
 \rho^{i,\ell,m}_{r+1}(dv)}
 \int\limits_{\mathcal{D}} 
 \mathcal{N}\left(Y_{r+1},f v,\sum\limits_{q=1}^N v^q g_qg_q^{\top}\right)\times{}\right.\\
\left.{}\times \sum\limits_{m=0}^{\infty}
 \sum\limits_{i,\ell=1}^N \widehat{X}_{t_r}^i
 \rho^{i,\ell,m}_{r+1}(dv)
 \right)
  \,, \enskip j = \overline{1,N}\,,
  \label{eq:filt_1_1}
 \end{multline}
 где 
 $ \rho^{k,j,h}_{r+1}(du)$~--- распределение вектора 
 $\tau_{r+1}X_{t_{r+1}}^{j}\mathbf{I}_{\{h\}}(N_{r+1})$ при 
 условии $X_{t_r}\hm=e_k$, т.\,е.\ 
 для любого $\mathcal{A} \hm\in \mathcal{B}(\mathbb{R}^M)$ верно тождество
\begin{multline*}
 \mathbf{P}\left\{\omega: \; X_{t_{r+1}}(\omega)=e_{j}, \; N_{r+1} = h,\right.\\ 
\left. \tau_{r+1}(X(\omega)) \in \mathcal{A}\;|\;X_{t_r}=e_k\right\} \equiv
  \rho^{k,j,h}_{r+1}(\mathcal{A}).
\end{multline*}
В качестве аппроксимации оценок можно использовать  
 $\overline{X}_{t_{r+1}}^n \ebd 
 \mathrm{col}\,(\overline{X}_{t_{r+1}}^{n,1},\ldots,\overline{X}_{t_{r+1}}^{n,N})$, 
 полученные из~(\ref{eq:filt_1_1}) путем урезания сумм ряда в~числителе и~знаменателе:
 
 \noindent
 \begin{multline}
 \overline{X}_{t_{r+1}}^{n,j} = 
 \left(
 \int\limits_{\mathcal{D}} 
 \mathcal{N}\left(Y_{r+1},f u,\sum\limits_{p=1}^N u^p g_pg_p^{\top}\right)\times{}\right.\\[-1pt]
\left.{}\times \sum\limits_{h=0}^{n}\sum\limits_{k=1}^N \overline{X}_{t_r}^k
 \rho^{k,j,h}_{r+1}(du)
 \right)\Bigg/ \\[-1pt]
 \left(
 \int\limits_{\mathcal{D}} 
 \mathcal{N}\left(Y_{r+1},f v,\sum\limits_{q=1}^N v^q g_qg_q^{\top}\right)\times{}\right.\\[-1pt]
\left. {}\times
 \sum\limits_{m=0}^{n}
 \sum\limits_{i,\ell=1}^N \overline{X}_{t_r}^i
 \rho^{i,\ell,m}_{r+1}(dv)
  \right)\,, \enskip
   j = \overline{1,N}.
  \label{eq:filt_2}
 \end{multline}
 Ниже по формуле полной вероятности получены интегралы из~(\ref{eq:filt_2}) для 
 $h\hm=0,1,2$:
 
\vspace*{-3pt}

 \noindent
  \begin{multline*}
 \int\limits_{\mathcal{D}}  \mathcal{N}
 \left(Y_{r+1},f u,\sum\limits_{p=1}^N u^p g_pg_p^{\top}\right) 
 \rho^{k,j,0}_{r+1}(du) = {}\\[-1pt]
 {}=
 \delta_{kj}\mathcal{N}\left(Y_{r+1},\Delta f^j,\Delta g_jg_j^{\top}\right)
 e^{\lambda_{jj}\Delta};
 %\label{eq:h0}
\\[-1pt]
 \int\limits_{\mathcal{D}}  \mathcal{N}\left(
 Y_{r+1},f u,\sum\limits_{p=1}^N u^p g_pg_p^{\top}\right) 
 \rho^{k,j,1}_{r+1}(du) ={} 
 \end{multline*}
 
 \noindent
 \begin{multline}
 \hspace*{-6.7pt}{}=\left(1-\delta_{kj}\right)\lambda_{kj}e^{\lambda_{jj}\Delta}
\! \int\limits_0^{\Delta}\!
 e^{(\lambda_{kk}-\lambda_{jj})u^k}
 \mathcal{N}\left(Y_{r+1},u^kf^k +{}\right.\hspace*{-0.28818pt}\\[-1pt]
\hspace*{-3mm}\left. {}+ \left(\Delta - u^k\right)f^j, u^k g_kg_k^{\top}+
 \left(\Delta-u^k\right)g_jg_j^{\top}\right)\,du^k;
 \label{eq:h1}
 \end{multline}
 
 \vspace*{-12pt}
 
 \noindent
 \begin{multline}
 \int\limits_D \mathcal{N}\left( 
Y_{r+1},f u,\sum\limits_{p=1}^N u^p g_pg_p^{\top}\right)du ={}\\[-1pt]
{}=
\sum\limits_{\substack{{\ell:\ell \neq k,}\\ {\ell \neq j}}}
 \lambda_{k\ell}\lambda_{\ell j} e^{\lambda_{jj}\Delta}\times {}\\[-1pt] 
 {}\times
 \int\limits_0^{\Delta} \int\limits_0^{\Delta-u^k} \!
e^{(\lambda_{kk}-\lambda_{\ell\ell})u^k+(\lambda_{\ell\ell}-
 \lambda_{jj})u^{\ell}}\times{} \\[-1pt] 
{}  \times
 \mathcal{N}\left(Y_{r+1},u^k f^k+u^{\ell}f^{\ell}+\left(
 \Delta-u^k-u^{\ell} \right)f^j,\right.\\[-1pt]
 \hspace*{-1mm}\left.
 u^k g_kg_k^{\top}+u^{\ell}g_{\ell}g_{\ell}^{\top}+\left(
 \Delta-u^k-u^{\ell} \right)
 g_jg_j^{\top}
 \right) du^{\ell}du^{k}, \!\!
  \label{eq:h2}
 \end{multline} 
 
\vspace*{-2pt}
 
 \noindent
  где  $\delta_{ij}$~--- символ Кронекера. Интегралы для $h\hm>2$ также могут 
  быть получены в~явном виде, однако их сложность резко возрастает.
 

   Так как система~(\ref{eq:obsys_1}) является автономной, то в~качестве локальной 
   характеристики бли\-зости~$\{\overline{X}_{t_r}\}$ 
   к~$\{\widehat{X}_{t_r}\}$ может быть выбрана величина
   
\noindent
 \begin{multline*}
 \overline{\sigma}(\pi) \ebd {\sf E}\left\{
 \|\widehat{X}_{t_{1}}(\pi, Y_{1}) - \overline{X}_{t_{1}}
 \left(\pi,Y_{1}\right)\|_{1}\right\} = {}\\
 {}=
 \sum\limits_{j=1}^N{\sf E}
 \left\{\left\vert \widehat{X}^j_{t_{1}}\left(\pi, Y_{1}\right) - \overline{X}^{n,j}_{t_{1}}
 \left(\pi,Y_{1}\right)\right\vert\right\}.
 %\label{eq:prec_1}
 \end{multline*}
 При этом начальное распределение $\pi \hm\in \mathcal{D}_1 \ebd $\linebreak $\ebd
 \{\mathrm{col}\,(\pi^1,\ldots,\pi^N):\;\pi^j > 0$, 
 $\sum\nolimits_{j=1}^N\pi^j\hm=1\}$ является начальным условием применения 
 одного шага рекурсии~(\ref{eq:filt_1}) или~(\ref{eq:filt_2}) для вычисления 
 оценки~$\widehat{X}_{t_{1}}$
   или~$\overline{X}_{t_{1}}$ соответственно. Фактически, 
 характеристика~$\overline{\sigma}(\pi)$ определяет, насколько сильно 
 рекурсивные схемы~(\ref{eq:filt_1}) и~(\ref{eq:filt_2}) разойдутся за 
 один шаг, стартуя из общей точки~$\pi$.
 
 Рекуррентные схемы~(\ref{eq:filt_1}) и~(\ref{eq:filt_2}), примененные~$r$~раз, 
 позволяют вычислить оценки~$\widehat{X}_{t_r}$ и~$\overline{X}_{t_r}$ 
 в~точке~$t_r$. В~качестве характеристики точности глобальной аппроксимации в~этом 
 случае естественно рассмотреть величину
 
 \vspace*{-2pt}
 
 \noindent
 \begin{equation*}
 \overline{\Sigma}_{t_r}(\pi) \ebd {\sf E}
 \left\{\|\widehat{X}_{t_{r}} - \overline{X}_{t_{r}}\|_{1}\right\} = 
 \!\sum\limits_{j=1}^N\!{\sf E}
 \left\{\left\vert \widehat{X}^j_{t_{r}} - 
 \overline{X}^{n,j}_{t_{r}}\right\vert \right\}.
% \label{eq:prec_2}
 \end{equation*}
 
 Следующее утверждение определяет оценки локальной и~глобальной 
 точности схемы аппроксимации~(\ref{eq:filt_2}).
 
 %\smallskip
 
 \noindent
 \textbf{Теорема~1.}\
\textit{Выполняются неравенства} 

%\vspace*{-2pt}

\noindent
 \begin{equation}
 \sup_{\pi \in \mathcal{D}_1} \overline{\sigma}(\pi) 
 \leqslant 2 \fr{(\overline{\lambda}\Delta)^{n+1}}{(n+1)!}\,;
 \label{eq:prec_loc}
\end{equation}

\noindent
\begin{align}
  \sup\limits_{\pi \in \mathcal{D}_1} \overline{\Sigma}_{t_r}(\pi)
   &\leqslant 2r \fr{(\overline{\lambda}\Delta)^{n+1}}{(n+1)!} +{}\notag\\[-0.5pt]
   &\hspace*{-20mm}{}+
  r(r-1)\left(
  \fr{(\overline{\lambda}\Delta)^{n+1}}{(n+1)!}
  \right)^2
  \left(
  1-\fr{(\overline{\lambda}\Delta)^{n+1}}{(n+1)!}
  \right)^{r-2},
 \label{eq:prec_glob}
 \end{align}
 
 \vspace*{-2pt}
 
 \noindent
 \textit{где} $\overline{\lambda} \ebd \max_{1 \leqslant j \leqslant N}|\lambda_{jj}|$.


%\smallskip

 Доказательство теоремы~1 приведено в~приложении.
 
 Данное утверждение представляет полезные оценки точности. Во-пер\-вых, 
 они являются равномерными по начальному распределению $\pi \hm\in \mathcal{D}_1$. 
 Во-вто\-рых, оценки носят универсальный, а~не асимптотический характер. Это 
 существенно в~практических задачах оценивания по дискретизованным 
 наблюдениям с~физическими или алгоритмическими ограничениями на шаг 
 по времени. Например, в~случае наблюдаемого процесса восстановления в~силу 
 центральной предельной теоремы для процессов восстановления~\cite{B_80} его
  приращения можно рассматривать как гауссовские случайные величины. 
  Однако данная аппроксимация обладает удовлетворительной точностью 
  только в~случае, когда шаг дискретизации по времени достаточно большой. 
 %
 В-третьих, неравенство~(\ref{eq:prec_glob}) позволяет получить порядок 
 аппроксимации при $\Delta \hm\to 0$. Зафиксируем момент времени $t\hm=T$ и~рассмотрим 
 характеристику $\sup\nolimits_{\pi \in \mathcal{D}_1} 
 \overline{\Sigma}_{T}(\pi)$ при $r\hm={T}/{\Delta}$ и~$\Delta \hm\to 0$. 
 Как только~$\Delta$ становится настолько мало, что 
 $\max\left({(\overline{\lambda}\Delta)^{n+1}}/{(n+1)!}, 
 \Delta ({T\lambda^{n+1}}/{(n+1)!})\right)\hm< 1$, из~(\ref{eq:prec_glob}) 
 следует неравенство
  %\begin{equation}
  $\sup\nolimits_{\pi \in \mathcal{D}_1} \overline{\Sigma}_{T}(\pi) 
  \hm\leqslant  ({3\overline{\lambda}^{n+1}}/{(n+1)!}) T\Delta^n.$
 %\label{eq:prec_asympt}
 %\end{equation}
 Это значит, что с~ростом времени~$T$ 
 ошибка аппроксимации копится пропорционально~$T$ и~при этом порядок точности 
 по~$\Delta$ равен~$n$.
 
 %\vspace*{-7pt}
 
  \section{Заключение}
  
  \vspace*{-4pt}
 
  В работе решена задача оценивания состояния однородного МСП по 
  дискретизованным наблюдениям. Получено аналитическое решение и~его 
  чис\-лен\-ные аппроксимации. Локальные и~глобальные показатели точ\-ности этих 
  приближений в~статье так\-же пред\-став\-ле\-ны. Примечательно, что  част\-ный случай 
  аппроксимаций~(\ref{eq:filt_2}) при $n\hm=0$ и~$\Lambda\hm=0$ был ранее 
  пред\-став\-лен в~\cite{B_17_1,B_17_2} для решения задачи байесовской классификации 
  случайного вектора по непрерывным наблюдениям с~мультипликативными шумами. 
 % 
Алгоритм оптимальной фильт\-ра\-ции и~его субоптимальные версии могут 
рас\-смат\-ри\-вать\-ся в~качестве основы чис\-лен\-ной реализации обобщения фильт\-ра 
Вонэма для сис\-тем с~мультипликативными шумами в~наблюдениях. 
Однако для их непосредственного использования необходимо решить 
следующие проб\-ле\-мы. Во-пер\-вых, в~(\ref{eq:h1}) и~(\ref{eq:h2}) присутствуют
 многомерные интегралы. Следует выяснить, какую результирующую погрешность 
 будут вносить ошибки их вы\-чис\-ле\-ния. Во-вто\-рых, представляется интересным 
 определить характеристики точ\-ности оптимальной фильт\-ра\-ции по дискретизованным 
 наблюдениям по отношению к~оптимальной фильт\-ра\-ции по непрерывным наблюдениям: 
 каков порядок точ\-ности по шагу временной дискретизации~$\Delta$? Для случая 
 вы\-чис\-ле\-ния классического фильт\-ра Вонэма с~по\-мощью алгоритма Эй\-ле\-ра--Ма\-ру\-ямы 
 подобный результат известен: порядок глобальной ошибки равен~${1}/{2}$. 
 Перечисленные задачи являются предметом дальнейших исследований.
 
 
  \vspace*{-10pt}
 
{\small
\subsection*{\raggedleft Приложение} 

\vspace*{-2pt}


\noindent
Д\,о\,к\,а\,з\,а\,т\,е\,л\,ь\,с\,т\,в\,о\ \ теоремы~1.\ \ Введем следующие 
обозначения для случайных величин и~мат\-риц, составленных из них:
\begin{align*}
\xi^{ji}(\ell)&\ebd 
\sum\limits_{h=0}^n \int\limits_{\mathcal{D}} 
 \mathcal{N}\left(Y_{\ell},f u,\sum\limits_{p=1}^N u^p g_pg_p^{\top}\right)
 \rho^{j,i,h}_{1}(du)\,; \\
  \theta^{ji}(\ell)&\ebd 
\sum\limits_{h=n+1}^{\infty} \int\limits_{\mathcal{D}} 
 \mathcal{N}\left(Y_{\ell},f u,\sum\limits_{p=1}^N u^p g_pg_p^{\top}\right)
 \rho^{j,i,h}_{1}(du)\,;
\\
 \xi(\ell)&\ebd \|\xi^{ji}(\ell)\|_{j,i=\overline{1,N}}\,,\quad 
 \Xi(r) \ebd \xi(r) \xi(r-1)\cdots \xi(1)\,;
 \\
 \theta(\ell)&\ebd \|\theta^{ji}(\ell)\|_{j,i=\overline{1,N}}\,, \quad 
 \Theta(r) \ebd \theta(r) \theta(r-1)\cdots \theta(1)\,.
%\label{eq:not_1}
\end{align*}
 
 Рекуррентные формулы~(\ref{eq:filt_1}) и~(\ref{eq:filt_2}) можно записать в~явной 
 форме
 
 
\noindent
\begin{align*}
 \widehat{X}_{t_r}& = \left( \mathbf{1}\left(\Xi(r) + 
 \Theta(r)\right)\pi\right)^{-1} \left(\Xi(r) + \Theta(r)\right)\pi\,;
\\
 \overline{X}_{t_r} &= \left( \mathbf{1}\Xi(r)\pi\right)^{-1} \Xi(r) \pi,
\end{align*}

\vspace*{-2pt}

\noindent
где $\mathbf{1} \ebd (1,\ldots,1)$~--- век\-тор-стро\-ка 
подходящей раз\-мер\-ности, составленная из единиц.

%Далее для краткости записи зависимость от~$r$ в~обозначениях~$\Xi(r)$ 
%и~$\Theta(r)$ будет опущена. 
Верна следующая цепочка неравенств:

 \vspace*{-3pt}

\noindent
\begin{multline}
\overline{\Sigma}_{t_r}(\pi)=%
%\me{}{\left\| 
%\widehat{X}_{t_r}(\pi, Y_1,\ldots,Y_r) - \overline{X}_{t_r}(\pi, Y_1,\ldots,Y_r)
%\right\|_1} =\\=
{\sf E}\left\{\left\| 
\fr{1}{\mathbf{1}\left(\Xi(r) + \Theta(r)\right)\pi} \left(\Xi(r) +{}\right.\right.\right.\\[-1pt]
\left.\left.\left.{}+ \Theta(r)\right)\pi
- \fr{1}{\mathbf{1}\Xi(r)\pi}\,\Xi(r) \pi
\right\|_1\right\} ={} \\[-1pt]
{}=
{\sf E}\left\{\fr{1}{\mathbf{1}\left(\Xi(r) + \Theta(r)\right)\pi \mathbf{1}\Xi(r)\pi}
\left\|
 \mathbf{1}\Xi(r) \pi \Theta(r)\pi -{}\right.\right.\\[-1pt]
\left.\left. {}- \mathbf{1}\Theta(r)\pi \Xi(r) \pi
 \right\|_1
 \vphantom{\fr{1}{\mathbf{1}\left(\Xi(r) + \Theta(r)\right)\pi \mathbf{1}\Xi(r)\pi}}
\right\} \leqslant {}\\[-1pt]
{}\leqslant 
{\sf E}\left\{\fr{1}{\mathbf{1}\left(\Xi(r) + \Theta(r)\right)\pi \mathbf{1}\Xi(r)\pi}
\left(
\mathbf{1}\Xi(r)\pi \| \Theta(r)\pi \|_1 +{}\right.\right.\\[-1pt]
\left.\left.{}+ \mathbf{1}\Theta(r)\pi 
\|
\Xi(r) \pi
\|_1
\right)
 \vphantom{\fr{1}{\mathbf{1}\left(\Xi(r) + \Theta(r)\right)\pi \mathbf{1}\Xi(r)\pi}}
\right\} ={}\\[-1pt]
{}=
2\,{\sf E}\left\{\fr{1}{\mathbf{1}\left(\Xi(r) + \Theta(r)\right)\pi}\mathbf{1}\Theta(r)\pi 
\right\}.
\label{eq:ineq_1}
\end{multline}

 
 \noindent
 Рассмотрим случайные события $a_{\ell} \ebd \{\omega \in \Omega: 
 N_{\ell}(\omega) \hm\leqslant n\}$, $\ell \hm= \overline{1,r}$, и~$A_r \ebd \{
 \omega\hm \in \Omega: \max_{1 \leqslant {\ell} \leqslant r}N_{\ell}(\omega) 
 \hm\leqslant n
 \}\hm=\prod\nolimits_{\ell=1}^r a_{\ell}$ и~оценку 
 $
 \widetilde{X}_{t_r}(\pi, Y_1,\ldots,Y_r)\ebd$\linebreak $\ebd
 {\sf E}\left\{X_{t_r}(\omega)\mathbf{I}_{A_r}(\omega)|\mathcal{O}_r\right\}.
 $
 Используя введенные выше обозначе\-ния и~абстрактный вариант формулы Байеса, 
 получаем, что
 
 \noindent
\begin{align}
\widetilde{X}_{t_r}& = \fr{1}{{\mathbf{1}\left(\Xi(r) + 
 \Theta(r)\right)\pi}}\,\Xi(r)\pi\,;\notag
 \\
\widehat{X}_{t_r} - \widetilde{X}_{t_r} &=
{\sf E}\left\{X_{t_r}(\omega)\mathbf{I}_{\overline{A}_r}(\omega)|\mathcal{O}_r\right\} ={}\notag\\[-1pt]
&\hspace*{17mm}{}= 
\fr{1}{\mathbf{1}\left(\Xi(r) + \Theta(r)\right)\pi}\Theta(r)\pi\,. 
\label{eq:eq_2}
 \end{align}
 Из (\ref{eq:ineq_1}) и~(\ref{eq:eq_2}) для $r\hm=1$ следует, что
 
 \vspace*{-4pt}
 
 \noindent
 \begin{multline}
 \overline{\sigma}(\pi) \leqslant 2\,{\sf E}
 \left\{\|{\sf E}\left\{X_{t_1}(\omega)\mathbf{I}_{\overline{a}_1}(\omega)|\mathcal{O}_1
 \right\}\|_1
 \right\} ={}\\[-1.5pt]
 {}=
 2\,{\sf E}\left\{\sum\limits_{n=1}^N {\sf E}
 \left\{X^n_{t_1}(\omega)\mathbf{I}_{\overline{a}_1}
 (\omega)|\mathcal{O}_1\right\}\right\} ={} \\[-2pt] 
 {}=
  2\,{\sf E}\left\{{\sf E}\left\{\mathbf{I}_{\overline{a}_1}(\omega)|\mathcal{O}_1
  \right\}\right\} =
   2 \mathbf{P}\left\{\overline{a}_1(\omega)\right\}.
\label{eq:ineq_3}
\end{multline}

 \vspace*{-2pt}
 
 \noindent
 Процесс $N^X_t$ общего числа скачков состояния~$X_t$ является считающим, и~его
  квадратическая характеристика равна 
  
\vspace*{-2pt}
  
  \noindent
 $$
 \langle N^X, N^X\rangle_t = - \int\limits_0^t \sum\limits_{n=1}^N \lambda_{nn} X_s^n\,ds\,,
 $$
 поэтому искомая вероятность ограничена сверху:
 $$ 
 \mathbf{P}\left\{\overline{a}_1(\omega)\right\} \leqslant 
 e^{-\overline{\lambda}\Delta}\sum\limits_{k=n+1}^{\infty} 
 \fr{(\overline{\lambda}\Delta)^{k}}{k!} <
 \fr{(\overline{\lambda}\Delta)^{n+1}}{(n+1)!}.
 $$
 
  \vspace*{-2pt}
  
  \noindent
 Из последнего неравенства и~(\ref{eq:ineq_3}) следует, что  для любого 
 начального распределения~$\pi$ выполняется неравенство $\overline{\sigma}(\pi)  
 \hm< 2({(\overline{\lambda}\Delta)^{n+1}}/{(n+1)!})$, т.\,е.\ 
 локальная оценка~(\ref{eq:prec_loc}) верна.
 
 С помощью марковского свойства пары $(X_t, N^X_t)$ и~последнего 
 неравенства можно оценить сверху вероятность 
 $\mathbf{P}\left\{\overline{A}_r(\omega)\right\}$:
 
  \vspace*{-2pt}
 
 \noindent
 \begin{multline*}
 \mathbf{P}\left\{\overline{A}_r(\omega)\right\} \leqslant 1 - \left(
 1- \fr{(\overline{\lambda}\Delta)^{n+1}}{(n+1)!}
 \right)^r \leqslant r \fr{(\overline{\lambda}\Delta)^{n+1}}{(n+1)!} + {}\\[-1pt]
 {}+\left|
 \sum\limits_{k=2}^r C_r^k \left(-\fr{(\overline{\lambda}\Delta)^{n+1}}{(n+1)!}
 \right)^k
 \right| \leqslant
 r \fr{(\overline{\lambda}\Delta)^{n+1}}{(n+1)!} +{}\\[-1pt]
 {}+\fr{r(r-1)}{2}
 \left(
 \fr{(\overline{\lambda}\Delta)^{n+1}}{(n+1)!}
 \right)^2
 \left(
 1-\fr{(\overline{\lambda}\Delta)^{n+1}}{(n+1)!}
 \right)^{r-2},
 \end{multline*} 
 из чего следует истинность глобальной оценки~(\ref{eq:prec_glob}).
Теорема~1 доказана.

}

%\vspace*{-12pt}

{\small\frenchspacing
 {%\baselineskip=10.8pt
 \addcontentsline{toc}{section}{References}
 \begin{thebibliography}{99}

\bibitem{Won_65}
\Au{Wonham W.} 
Some applications of stochastic differential equations to optimal
  nonlinear filtering~//
SIAM~J.~Control, 1965. Vol.~2. P.~347--369. 

\bibitem{KP_92}
\Au{Kloeden P., Platen E.} Numerical solution of stochastic
differential equations.~--- Berlin: Springer, 1992.~636~p.

\bibitem{YZL_04}
\Au{Yin G., Zhang Q., Liu Y.} 
Discrete-time approximation of Wonham filters~//
J.~Control Theory Applications, 2004. Iss.~2. P.~1--10.

\bibitem{PR_10}
\Au{Platen E., Rendek R.}
Quasi-exact approximation of hidden Markov chain filters~//
Communicat.~Stoch.~Analys., 2010. Vol.~4. Iss.~1. P.~129--142.

\bibitem{B_18}
\Au{Борисов А.} Фильтрация Вонэма по наблюдениям с~мультипликативными шумами~// 
Автоматика и~телемеханика, 2018.
№~1. C.~52--65. 
 
  \bibitem{BSh_85} %6
\Au{Бертсекас Д., Шрив С.} Стохастическое оптимальное управление. 
Случай дискретного времени~/ Пер. с~англ.~--- М.: Наука, 1985.~280~c.
(\Au{Betsekas~D.\,P., Shreve~S.\,E.} Stochastic optimal control:
The discrete-time case.~--- Orlando, FL, USA:
Academic Press Inc., 1978. 323~p.)

  \bibitem{ZhSh_95} %7
\Au{Жакод Ж., Ширяев А.} Предельные теоремы для случайных процессов,~I.~/
Пер. с~англ.~--- 
М.: Физматлит, 1995.~544~c.
(\Au{Jacod~J., Shiryaev~A.} Limit theorems for stochastic processes.~---
Berlin: Springer, 2003. 664~p.)

\bibitem{S_00}
\Au{Sericola B.} Occupation times in Markov processes~//
Commun. Stat. Stochastic Models, 2000. Vol.~16. Iss.~5. P.~479--510. 

  \bibitem{B_80}
\Au{Боровков А.} Асимптотические методы в~тео\-рии массового обслуживания.~--- 
М.: Физматлит, 1995.~384~c.

  \bibitem{B_17_1}
\Au{Борисов А.} Классификация по непрерывным наблюдениям с~мультипликативными шумами.~I. 
Формулы байесовской оценки~// Информатика и~её применения, 2017. Т.~11. Вып.~1. C.~11--19.
doi: 10.14357/19922264170102.

  \bibitem{B_17_2}
\Au{Борисов А.} Классификация по непрерывным наблюдениям с~мультипликативными 
шумами.~II. Алгоритм численной реализации оценки~// Информатика и~её 
применения, 2017. Т.~11. Вып.~2. C.~33--41.
doi: 10.14357/19922264170204.

 \end{thebibliography}

 }
 }

\end{multicols}

\vspace*{-4pt}

\hfill{\small\textit{Поступила в~редакцию 10.07.18}}

\vspace*{6pt}

%\pagebreak

%\newpage

%\vspace*{-28pt}

\hrule

\vspace*{2pt}

\hrule

%\vspace*{-2pt}

\def\tit{FILTERING OF~MARKOV JUMP PROCESSES\\ BY~DISCRETIZED OBSERVATIONS}

\def\titkol{Filtering of Markov jump processes by discretized observations}

\def\aut{A.\,V.~Borisov}

\def\autkol{A.\,V.~Borisov}

\titel{\tit}{\aut}{\autkol}{\titkol}

\vspace*{-11pt}


\noindent
Institute of Informatics Problems, Federal Research Center ``Computer Science 
and Control'' of the Russian Academy of Sciences, 44-2~Vavilov Str., Moscow 
119333, Russian Federation


\def\leftfootline{\small{\textbf{\thepage}
\hfill INFORMATIKA I EE PRIMENENIYA~--- INFORMATICS AND
APPLICATIONS\ \ \ 2018\ \ \ volume~12\ \ \ issue\ 3}
}%
 \def\rightfootline{\small{INFORMATIKA I EE PRIMENENIYA~---
INFORMATICS AND APPLICATIONS\ \ \ 2018\ \ \ volume~12\ \ \ issue\ 3
\hfill \textbf{\thepage}}}

\vspace*{6pt}



\Abste{The article is devoted to a~solution of the optimal filtering problem 
of a~homogenous Markov
jump process state. The available observations represent 
time increments of the integral transformations of the Markov\linebreak\vspace*{-12pt}}

\Abstend{state corrupted by 
Wiener processes. The noise intensity is also state-dependent. At the instant of 
the consecutive
observation obtaining, the optimal estimate is calculated recursively 
as a~function of previous estimate and the new observation, meanwhile between 
observations the filtering estimate is a simple forecast by virtue of the Kolmogorov 
differential system. The recursion is rather expensive because of  need to calculate 
the integrals, which are the location-scale mixtures of Gaussians. The mixing 
distributions represent the occupation of the state in each of possible values 
during the mid-observation intervals. The paper contains numerically cheaper 
approximations, based on the restriction of the state transitions number between 
the observations. Both the local and global characteristics of approximation 
accuracy are obtained as functions of the dynamics parameters, mid-observation 
interval length, and upper bound of transitions number.}

\KWE{Markov jump process; optimal filtering; multiplicative observation noises; 
stochastic differential equation; numerical approximation}




\DOI{10.14357/19922264180316}

%\vspace*{-14pt}

\Ack
\noindent
The work was supported in part by the Russian Foundation
for Basic Research (Project No.\,16-07-00677).



%\vspace*{6pt}

  \begin{multicols}{2}

\renewcommand{\bibname}{\protect\rmfamily References}
%\renewcommand{\bibname}{\large\protect\rm References}

{\small\frenchspacing
 {%\baselineskip=10.8pt
 \addcontentsline{toc}{section}{References}
 \begin{thebibliography}{99}
\bibitem{Won_65-1}
\Aue{Wonham, W.} 1965.
Some applications of stochastic differential equations to optimal
  nonlinear filtering.
\textit{SIAM~J.~Control} 2:347--369. 

\bibitem{KP_92-1}
\Aue{Kloeden,~P., and E.~Platen.} 1992. \textit{Numerical solution of stochastic
differential equations.} Berlin: Springer. 636~p.

\bibitem{YZL_04-1}
\Aue{Yin,~G., Q.~Zhang, and Y.~Liu.} 2004.
Discrete-time approximation of Wonham filters.
\textit{J.~Control Theory Applications} 2:1--10.

\bibitem{PR_10-1}
\Aue{Platen, E., and R.~Rendek.} 2010.
Quasi-exact approximation of hidden Markov chain filters.
\textit{Communicat. Stoch. Analys.} 4(1):129--142.

\bibitem{B_18-1}
\Aue{Borisov, A.} 2018. Wonham filtering by observations
with multiplicative noises. \textit{Automat.~Rem.~Contr.} 79(1):39--50.  
doi: 10.1134/ S0005117918010046.
 
  \bibitem{BSh_85-1}
\Aue{Bertsekas, D., and S.~Shreve.} 1996.
\textit{Stochastic optimal control: The discrete-time case}.
Nashua, NH: Athena Scientific. 330~p.
  
  \bibitem{ZhSh_95-1}
  \Aue{Jacod,~J., and A.~Shiryaev.} 2003.
\textit{Limit theorems for stochastic processes.}
Berlin: Springer. 664~p.

\bibitem{S_00-1}
\Aue{Sericola, B.}
2000. Occupation times in Markov processes.
\textit{Commun. Stat.} 16(5):479--510. 

  \bibitem{B_80-1}
\Aue{Borovkov, A.} 1984.
 \textit{Asymptotic methods in queueing theory}. 
 Hoboken, NJ: Wiley-Blackwell.~304~p.

  \bibitem{B_17_1-1}
  \Aue{Borisov, A.} 2017. 
  Klassifikatsiya po ne\-pre\-ryv\-nym nablyu\-de\-miyam s~mul'tiplikativnymi shumami. I. 
  Formuly bayesov\-skoy otsenki [Classification by continuous-time observations
in multiplicative noise. I.~Formulae for Bayesian 
estimate]. \textit{Informatika i~ee Primeneniya~--- Inform.~Appl.}
11(1):11--19. doi: 10.14357/19922264170102.

  \bibitem{B_17_2-1}
\Aue{Borisov, A.} 2017. Klassifikatsiya po nepreryvnym nablyudemiyam 
s~mul'tiplikativnymi summami. II.~Formuly bayesovskoy otsenki 
[Classification by continuous-time observations
in multiplicative noise. II.~Numerical algorithm].
\textit{Informatika i~ee Primeneniya~--- Inform.~Appl.}
11(2):33--41. doi: 10.14357/19922264170204.

\end{thebibliography}

 }
 }

\end{multicols}

\vspace*{-6pt}

\hfill{\small\textit{Received July 10, 2018}}

%\pagebreak

%\vspace*{-18pt}

\Contrl

\noindent
\textbf{Borisov Andrey V.} (b.\ 1965)~--- 
Doctor of Science in physics and mathematics, principal scientist, Institute of
Informatics Problems, Federal Research Center ``Computer Science and Control''
 of the Russian Academy of
Sciences, 44-2 Vavilov Str., Moscow 119333, Russian Federation; 
\mbox{aborisov@frccsc.ru}
\label{end\stat}

\renewcommand{\bibname}{\protect\rm Литература}              %4
\def\stat{shest}

\def\tit{СРЕДНЕКВАДРАТИЧНЫЙ РИСК FDR-ПРОЦЕДУРЫ\\ В~УСЛОВИЯХ
СЛАБОЙ ЗАВИСИМОСТИ}

\def\titkol{Среднеквадратичный риск FDR-процедуры в~условиях
слабой зависимости}

\def\aut{М.\,О.~Воронцов$^1$,  О.\,В.~Шестаков$^2$}

\def\autkol{М.\,О.~Воронцов,  О.\,В.~Шестаков}

\titel{\tit}{\aut}{\autkol}{\titkol}

\index{Воронцов М.\,О.}
\index{Шестаков О.\,В.}
\index{Vorontsov M.\,O.}
\index{Shestakov O.\,V.}


%{\renewcommand{\thefootnote}{\fnsymbol{footnote}} \footnotetext[1]
%{Исследование выполнено за счет гранта Российского научного фонда (проект 22-28-00588). Работа 
%РАН, Москва).}}


\renewcommand{\thefootnote}{\arabic{footnote}}
\footnotetext[1]{Факультет вычислительной математики и кибернетики,  Московский государственный университет 
имени М.\,В.~Ломоносова;
Московский центр фундаментальной и прикладной математики, \mbox{m.vtsov@mail.ru}}
\footnotetext[2]{Факультет вычислительной математики и кибернетики, Московский государственный университет 
имени М.\,В.~Ломоносова; 
Федеральный исследовательский центр <<Информатика и~управ\-ле\-ние>> Российской 
академии наук; Московский центр фундаментальной и прикладной математики, 
\mbox{oshestakov@cs.msu.ru}}

%\vspace*{-12pt}





\Abst{Во многих прикладных областях возникает задача обработки больших массивов 
данных. При этом часто перед обработкой массив данных подвергается некоторому 
преобразованию, приводящему к <<разреженному>>, или <<экономному>>, 
представлению, при котором абсолютное значение большинства элементов массива 
равно нулю (или достаточно мало).
Кроме того, в~результате помех при получении и~передаче данных в них попадает 
шум, который при дальнейшей обработке желательно некоторым образом удалить. 
Возникающая при этом задача математически эквивалентна некоторым задачам 
множественной проверки гипотез.
Ранее для решения указанной задачи в~условиях нормальности, независимости 
и~разреженности данных была предложена процедура, основанная на методе контроля средней доли ложных 
отклонений (False Discovery Rate, FDR) гипотез.
В~настоящей работе исследуется асимптотика риска указанной процедуры в~случае 
наличия слабой за\-ви\-си\-мости в~данных.}

\KW{пороговая обработка; множественная проверка гипотез; 
среднеквадратичный риск}


\DOI{10.14357/19922264230205}{AVJZDX}
  
%\vspace*{-8pt}


\vskip 10pt plus 9pt minus 6pt

\thispagestyle{headings}

\begin{multicols}{2}

\label{st\stat}


\section{Введение}


В современных приложениях статистики за\-час\-тую требуется обрабатывать большие 
массивы зашумленных данных~-- источниками шума могут выступать помехи 
и~несовершенство оборудования. Примерами служат исследования в~об\-ласти генетики 
с~возникающими в них задачами множественной проверки гипотез~\cite{MultipleTesting}, задачи обработки изображений 
с~высоким разрешением~\cite{FDRImage} и другие прикладные проб\-ле\-мы. В~связи с этим рассмотрим задачу 
нахождения оценки неизвестного вектора~$\mu$ как функции~$x$ в~модели данных
$$
x_i = \mu_i + z_i, \enskip i=1,\ldots,n,
$$
где $\mu_i\in\mathbb{R}$, $z_i \sim N(0,\sigma^2)$ для всех~$i$.

Приведенная задача может рассматриваться как частный случай задачи множественной 
проверки гипотез, а~именно: пусть построено~$n$ статистик~$x_i$ для проверки 
нулевых гипотез~$H_{0,i}$ против альтернатив~$H_{1,i\,}$, причем при 
верной гипотезе~$H_{0, i}$ (соответственно~$H_{1, i}$) распределение~$x_i$ 
известно и~равно $N(0,\sigma^2)$ (соответственно $N(\mu_i,\sigma^2)$, 
$\mu_i\hm\neq0$ и неизвестно). Принятие гипотезы $H_{0,i}$ в такой постановке 
равносильно заключению $\mu_i\hm=0$.

В работе~\cite{AdaptingFDR} для решения рассматриваемой задачи в условиях 
независимости компонент вектора~$x$ и~разреженности вектора~$\mu$ была 
предложена процедура построения оценки~$\hat{\mu}_F$ вектора~$\mu$, основанная 
на методе контроля средней доли ложных отклонений (FDR) 
гипотез при помощи алгоритма Бен\-жа\-ми\-ни--Хоч\-бер\-га,
и~было проведено исследование асимптотики риска построенной оценки.

В то же время в определенных приложениях, например
при анализе полученных в результате использования ДНК-мик\-ро\-чи\-пов 
данных~\cite{ResultsOnFDRUnderDependence}, исследовании геофизических процессов 
и~анализе помех в телекоммуникационных каналах, условие независимости компонент 
вектора~$x$ может не выполняться. В~данной работе исследуется асимптотика риска 
предложенной в~\cite{AdaptingFDR} оценки~$\hat{\mu}_F$ в случае, когда 
компоненты вектора~$x$ слабо зависимы, а~$\mu$ принадлежит классу разреженности
$$
l_0[\eta] = \left\{\mu\in\mathbb{R}^n\,:\, ||\mu||_0 \leq \eta n\right\}, \enskip
\eta\in(0;1).
$$


\section{Обработка вектора данных с~помощью FDR-процедуры}

Предложенная в~\cite{AdaptingFDR} процедура заключается в~жесткой пороговой 
обработке компонент вектора~$x$ порогом $\hat{t}_F = \hat{t}_F(x)$, и~ее 
результат~--- оценка~$\hat{\mu}_F$ вектора~$\mu$ с компонентами

\noindent
\begin{equation*}
(\hat{\mu}_F)_i  = p(x_i,\hat{t}_F) \equiv
 \begin{cases}
   x_i, & |x_i| > \hat{t}_F;\\
   0, & |x_i| \leq \hat{t}_F,
 \end{cases}
\end{equation*}
где
$$
\hat{t}_F = \sigma z\left(\fr{q \hat{k}_F}{2n}\right).
$$
Здесь
$z(\alpha)$~--- квантиль уровня $(1-\alpha)$ стандартного нормального 
распределения; $q\in(0;1)$~--- управ\-ля\-ющий па\-ра\-метр FDR-ме\-то\-да;
$$
\hat{k}_F \hm= \max \left\{k \, :\, |x|_{(k)} \geq t_k \right\}, 
$$
где
$|x|_{(k)}$~--- $k$-й элемент вектора, получаемого в~результате 
упорядочения вектора~$|x|$ по невозрастанию:
$$
|x|_{(1)} \geq |x|_{(2)} \geq \cdots \geq |x|_{(n)},
$$
$$
 t_k = \sigma z\left(\fr{q  k}{2n}\right).
$$
 Далее полагается, что $q\hm\equiv q_n$ зависит от~$n$.

В~\cite{AdaptingFDR} для среднеквадратичного риска
$$
\rho(\hat{\mu}_F, \mu) = \e ||\hat{\mu}_F-\mu||^2 = \e\sum\limits_{i=1}^n 
\left(p(x_i,\hat{t}_F)-\mu_i\right)^2
$$
оценки $\hat{\mu}_F$ получен сле\-ду\-ющий результат.


\smallskip

\noindent
\textbf{Теорема~1.}
\textit{Пусть $x_i$, $i=1,\ldots,n$, независимы, $\liminf q_n \ln n > 0$, $\limsup q_n 
<1$, а также $\eta_n$ лежит в интервале $[n^{-1}\ln^5 n; n^{-\delta}]$, 
$\delta>0$. Тогда при $n\to\infty$}
\begin{multline*}
\sup\limits_{\mu\in l_0[\eta_n]}\rho(\hat{\mu}_F, \mu)  \leq {}\\
{}\leq
R_n(l_0[\eta_n])\left(1+2 q_n(1-q_n)^{-1}+o(1)\right),
\end{multline*}
\textit{где}
\begin{equation*}
R_n(l_0[\eta_n]) = \inf\limits_{\hat{\mu}=\hat{\mu}(x)} \sup\limits_{\mu\in 
l_0[\eta_n]}\rho(\hat{\mu}, \mu).
\end{equation*}



В работе также приведена асимптотика
$$
R_n(l_0[\eta_n])\sim c n\eta_n \ln \eta_n^{-1},
$$
где $c=c(\sigma)$.

При пороговой обработке иногда также используется так называемый универсальный 
порог $T_U \hm= \sigma \sqrt{2\ln n}$, предложенный в работе~\cite{spatialAdaptation}. Исследования в~\cite{AdaptingSURE, ExactRisk} 
показали, что порог~$T_U$ в определенном смысле максимальный и~рассматривать 
пороги выше него не имеет смысла. Более того, нетрудно показать, что $t_k\hm < T_U$ 
для всех~$k$ и всех достаточно больших~$n$.  В~связи с~этим всюду далее 
полагаем, что порог~$\hat{t}_F$ выбирается на отрезке $[0; T_U]$.

\section{Асимптотика среднеквадратичного риска FDR-процедуры в~условиях слабой 
зависимости}

Перейдем к исследованию асимптотики риска оценки~$\hat{\mu}_F$ в случае, когда 
компоненты вектора~$x$ слабо зависимы, а~именно: имеют достаточно быст\-ро 
убывающий коэффициент сильного перемешивания~\cite{Bosq}
\begin{multline*}
\alpha(k) = \sup\limits_{1\leq m\leq n}\alpha\left(\sigma(x_i, i\leq m), 
\sigma(x_i, i\geq m+k)\right), \\ k=1,\ldots,n-1,
\end{multline*}
где
$$
\alpha(\mathcal{B},\mathcal{C}) = \sup\limits_{B\in\mathcal{B}, 
C\in\mathcal{C}} \left|\p(BC)-\p(B)\p(C)\right|.
$$
Отметим, что для любой измеримой функции~$f(\cdot)$ коэффициент сильного 
перемешивания набора $f(x_1),\ldots,f(x_n)$ не больше коэффициента сильного 
перемешивания набора $x_1,\ldots,x_n$.



Покажем справедливость следующего вспомогательного утверж\-де\-ния.

\smallskip

\noindent
\textbf{Утверждение~1.}
\textit{Пусть для набора действительных случайных величин $X_1,\dots,X_n$ с 
коэффициентом сильного перемешивания $\alpha(\cdot)$ выполняется $\e X_i \hm= 0$, 
$|X_i| \leq b$, $i = 1,\dots,n$. Тогда для любого целого числа $m\in[1;n/2]$ и 
любого $\eps > 0$ справедливо}
\begin{multline}
\p\left(\left|\sum\limits_{i=1}^n X_i\right| > \eps\right) \leq 2 \exp\left\{-
\fr{\eps^2}{32 v_0}\, B\left(\fr{nb\eps}{8m v_0}\right)\right\}+{}\\
{}+ 2 \exp\left\{-\fr{\eps^2}{32 v_1} B\left(\fr{nb\eps}{8m  v_1}\right)\right\} +{}\\
{}+ 22\left(1+\fr{4bn}{\eps}\right)^{1/2} m \alpha\left(\left[\fr{n}{2m}\right]\right), 
\label{statement1}
\end{multline}
\textit{где}
\begin{multline*}
v_0 = \sum\limits_{j=1}^m \e\left(([2(j-1)p]+1-2(j-1)p)\right. \times{}\\
{}\times X_{[2(j-1)p]+1}+X_{[2(j-1)p]+2}+\cdots + X_{[(2j-1)p]} +{}\\
\left.{}+ ((2j-1)p-[(2j-1)p])X_{[(2j-1)p]+1}\right)^2 ;
\end{multline*}

\vspace*{-12pt}

\noindent
\begin{multline*}
v_1 = \sum\limits_{j=1}^m \e\left(([(2j-1)p]+1-(2j-1)p)\times{} \right.\\
{}\times X_{[(2j- 1)p]+1}+X_{[(2j-1)p]+2}+\cdots + X_{[2jp]} +{}\\
\left.{}+ (2jp-[2jp])X_{[2jp]+1}\right)^2;
\end{multline*}

\noindent
$$
p=\fr{n}{2m}\,; \enskip B(\lambda) = 2\lambda^{-2}((1+\lambda)\ln(1+\lambda)-
\lambda), \enskip \lambda>0\,.
$$


\noindent
Д\,о\,к\,а\,з\,а\,т\,е\,л\,ь\,с\,т\,в\,о\,.\ \ При доказательстве теоремы~1.3 из~\cite{Bosq} 
показано, что
\begin{multline*}
\p \left(\left|\sum\limits_{j=1}^m V_j\right| > \fr{\eps}{2} \right) \leq 
\p \left(\left|\sum\limits_{j=1}^m W_j\right| > \fr{\eps}{4}\right) + {}\\
{}+
11\left(1+\fr{4bn}{\eps}\right)^{1/2} m 
\alpha\left(\left[\fr{n}{2m}\right]\right),
\end{multline*}
где
\begin{multline*}
V_j = ([2(j-1)p]+1-2(j-1)p)\times{}\\
{}\times X_{[2(j-1)p]+1}+X_{[2(j-1)p]+2}+\cdots + X_{[(2j-1)p]} + {}\\
{}+((2j-1)p-[(2j-1)p])X_{[(2j-1)p]+1},\\
  W_j {\overset{d}{=}} V_j, \enskip j = 1,\ldots,m\,,
  \end{multline*}
а случайные величины $W_1,\ldots,W_m$ независимы. Применяя для случайных величин 
$W_1,\ldots,W_m$ неравенство Беннета~\cite{Pollard}, получим
$$
\p \left(\left|\sum\limits_{j=1}^m W_j\right| > \fr{\eps}{4}\right) \leq 2 
\exp\left\{-\fr{\eps^2}{32 v_0}\, 
B\left(\fr{pb\eps}{4v_0}\right)\right\}.
$$
Проводя аналогичные рассуждения для случайных величин
\begin{multline*}
V'_j = ([(2j-1)p]+1-(2j-1)p) \times{}\\
{}\times X_{[(2j-1)p]+1}+X_{[(2j-1)p]+2}+\cdots + X_{[2jp]} +{}\\
{}+ (2jp-[2jp])X_{[2jp]+1}, \enskip j = 1,\ldots,m,
\end{multline*}
и объединяя результаты, с учетом
\begin{multline*}
\p \left(\left|\sum\limits_{i=1}^n X_i \right| > \eps \right) \leq {}\\
{}\leq 
\p \left(\left|\sum\limits_{i=1}^m V_i \right| > \fr{\eps}{2} \right) + 
\p \left(\left|\sum\limits_{i=1}^m V'_i \right| > \fr{\eps}{2} \right)
\end{multline*}
получаем требуемое.\hfill$\square$


\smallskip

\noindent
\textbf{Замечание.}
Из непрерывности правой части неравенства~(\ref{statement1}) по~$\eps$ следует, 
что в левой части можно заменить знак~$>$ на~$\geq$.


\smallskip

Введем следующие обозначения:

\vspace*{-6pt}

$$
k_n = [\eta_n n]; \enskip \gamma_n = (\ln\ln n)^{-1}; \enskip \kappa_n = (1 - q_n - \gamma_n)^{-1} k_n;
$$

\vspace*{-12pt}

\noindent
\begin{multline*}
p_i = \p(|x_i|\geq t_k), \enskip X_i = \Ik(|x_i|\geq t_k) - p_i, \\ i=1,\ldots,n;
\end{multline*}

\vspace*{-3pt}

\noindent
$$
N(t_k) = \#\{i : |x_i| \geq t_k\};\enskip M = \e N(t_k) = \sum\limits_{i=1}^n  p_i.
$$
Заметим, что $\e X_i \hm= 0$, $|X_i| \hm< 1$, $\D X_i \hm= p_i(1\hm-p_i)$ для всех~$i, k$.

\smallskip

\noindent
\textbf{Лемма~1.}
\textit{Пусть $\eta_n \leq b<1$, $m\in[1;n/2]\cap\mathbb{N}$, а~$\alpha(\cdot)$~--- 
коэффициент сильного перемешивания компонент вектора~$x$. Для некоторого 
$N\hm\in\mathbb{N}$ при $n \hm\geq N$ справедливо}
\begin{multline*}
\hspace*{-7.9pt}\sup\limits_{\mu\in l_0[\eta_n]} \p \left(\hat{k}_F \!\geq\! \kappa_n \right) \leq 
4 n \exp\left\{ \!-\fr{(1-b)m}{64n}  \kappa_n q_n \gamma_n^2   \! \right\}+{}\\[3pt]
{}+ 
22\left(1+\fr{4n}{(1-b)\kappa_n q_n \gamma_n}\right)^{1/2} n m  \alpha\left(\left[\fr{n}{2m}\right]\right).
\end{multline*}

\noindent
Д\,о\,к\,а\,з\,а\,т\,е\,л\,ь\,с\,т\,в\,о\,.\ \ Фиксируем $\mu\hm\in l_0[\eta_n]$. Имеем
\begin{equation}
\label{lem1eq1}
\p(\hat{k}_F\geq \kappa_n) \leq \sum\limits_{k\geq\kappa_n} \p(N(t_k)\geq k).
\end{equation}
Фиксируем $k \geq \kappa_n$; задача~-- ограничить вероятность $\p(N(t_k)\hm\geq k)$ 
сверху. Ниже показано, что $M\hm<k$ для всех $k \hm\geq \kappa_n$ и всех достаточно 
больших~$n$. По утверждению~1 имеем:
\begin{multline}
\p(N(t_k)\geq k) = \p\left(\sum\limits_{i=1}^n X_i \geq k-M\right) \leq {}\\[3pt]
{}\leq
\p\left(\left|\sum\limits_{i=1}^n X_i \right| \geq k-M\right) \leq{}\\[3pt]
{}\leq 2 \exp\left\{-\fr{(k-M)^2}{32 v_0}\, B\left(\fr{n(k-M)}{8m  v_0}\right)\right\}+{}\\[3pt]
{}+2 \exp\left\{-\fr{(k-M)^2}{32 v_1}\, B\left(
\fr{n(k-M)}{8m v_1}\right)\right\} +{}\\[3pt]
{}+ 22\left(1+\fr{4n}{(k-M)}\right)^{1/2} m  \alpha\left(\left[\fr{n}{2m}\right]\right).
\label{lem1eq2}
\end{multline}
Для произвольного набора центрированных случайных величин $\xi_1, \ldots, \xi_l$ 
с~конечными дисперсиями справедливо
$$
\e(\xi_1 + \cdots + \xi_l)^2 \leq l\sum\limits_{i=1}^l \D \xi_i,
$$
откуда
$$
v_{0,\,1} \leq \left(\left[\fr{n}{2m}\right]+1\right) \sum\limits_{i=1}^n  p_i(1-p_i) \leq \fr{nM}{m}\,.
$$

Рассмотрим первое слагаемое в~(\ref{lem1eq2}). Пусть сначала $n(k-M)/(8mv_0) 
\hm\leq 1$. Так как функция~$B(\lambda)$ убывает по~$\lambda$ и $v_0 \hm\leq nM/m$, то

\noindent
\begin{multline*}
\fr{(k-M)^2}{32 v_0}\, B\left(\fr{n(k-M)}{8m v_0}\right) \geq \fr{(k-M)^2 m}{32nM}\, B(1) ={}\\
{}=
\fr{mM}{32n} \left(\fr{k}{M}-1\right)^2 B(1).
\end{multline*}
Если же $n(k-M)/(8mv_0)\hm > 1$, то, поскольку $\lambda B(\lambda)$ возрастает по~$\lambda$ при $\lambda\hm\geq1$,
\begin{multline*}
\fr{(k-M)^2}{32 v_0}\, B\left(\fr{n(k-M)}{8m v_0}\right) \geq  \fr{(k-M) m}{4n}\, B(1) ={}\\
{}=  \fr{mM}{4n} \left(\fr{k}{M}-1\right) B(1).
\end{multline*}
Объединяя данные результаты, с учетом $B(1)\hm>1/2$ получим

\noindent
\begin{multline*}
2 \exp\left\{-\fr{(k-M)^2}{32 v_0}\, B\left(\fr{n(k-M)}{8m  v_0}\right)\right\} \leq{}\\
{}\leq 2 \exp\left\{-\fr{mM}{64n}
\min\left\{\left(\fr{k}{M}-1\right)^2,\left(\fr{k}{M}- 1\right)\right\}\right\}
\end{multline*}
и~аналогично для слагаемого с~$v_1$.

Перейдем к поиску границ возможных значений~$M$. Вспомним, что в векторе $\mu\hm\in 
l_0[\eta_n]$ не более $k_n \hm= [\eta_n n] \leq bn$ ненулевых, а~следовательно, и 
не менее $n\hm-k_n$ нулевых компонент, откуда при $k\hm\geq \kappa_n$ для~$M$ получим 
оценку снизу:
\begin{multline*}
M = \sum\limits_{i=1}^n p_i \geq k_n \cdot 0 + (n-k_n)  \fr{kq_n}{n} 
\geq   \fr{(n-k_n)}{n}\, kq_n \geq{}\\
{}\geq (1-b)\kappa_n q_n.
\end{multline*}
С другой стороны,
$$
M \leq k_n \cdot 1 + (n-k_n)  \fr{kq_n}{n} =  k_n + \left(1-\fr{k_n}{n}\right) k q_n.
$$
Рассмотрим функцию 
$$
g(x) = \fr{x}{k_n \hm+ (1\hm-k_n/n) x q_n}\,.
$$
 Тогда $k/M \hm\geq g(k)$ для 
любого~$k$. Заметим, что функция~$g(x)$ возрастает по~$x$, поэтому при 
$k\hm\geq\kappa_n$ имеем
\begin{multline*}
g(k) \geq g(\kappa_n) =\fr{\kappa_n}{k_n+ (1-k_n/n) \kappa_n q_n} ={}\\
{}= \left(1- \gamma_n- \fr{k_n q_n}{n}\right)^{-1} > 1 + \gamma_n,
\end{multline*}
откуда
$$
\left(\fr{k}{M}-1\right) > \gamma_n.
$$
Также здесь показано, что $k>M$.

\columnbreak

Наконец, заметим, что неравенство 
$$
\left(\fr{k}{M}-1\right)^2 > \fr{k}{M}-1
$$
выполняется лишь в случае 
$$
\fr{k}{M}-1 > 1\,.
$$
 Но тогда тем более 
 $$
 \fr{k}{M}- 1>\gamma_n^2,
 $$
  откуда
$$
\min\left\{\left(\fr{k}{M}-1\right)^2,\left(\fr{k}{M}-1\right)\right\}  > \gamma_n^2.
$$

Объединяя выписанные неравенства, получим
\begin{multline*}
\p(N(t_k)\geq k) \leq 4 \exp\left\{-\fr{(1-b)m}{64n}\,  \kappa_n q_n  \gamma_n^2    \right\}+ {}\\
{}+ 22\left(1+\fr{4n}{(1-b)\kappa_n q_n 
\gamma_n}\right)^{1/2} m \alpha\left(\left[\fr{n}{2m}\right]\right),
\end{multline*}
что вместе с~(\ref{lem1eq1}) дает утверждение леммы.\hfill$\square$

\smallskip


Обозначим $T_1 = \sigma\sqrt{2\ln \eta_n^{-1}}$.

\smallskip

\noindent
\textbf{Лемма~2.}
\textit{Пусть $\eta_n\hm\geq n^{-\delta_1},$ $\delta_1\hm<1;$ $\mathrm{lim}\,\eta_n\hm=0;$ $q_n\hm\leq 1/2;$ 
$\liminf q_n \ln n \hm\geq 2 c_3 > 0;$ а~также существуют такие константы $c_1, 
c_2\hm>0,$ что $\alpha(k) \hm\leq c_1 k^{-1-(9/2)\delta_1/(1\hm-\delta_1)\hm-c_2}$ для 
любого $k\hm\in\mathbb{N}$. Тогда при} $n\hm\to\infty$
$$
\sup\limits_{\mu\in l_0 [\eta_n]} \p\left(\hat{t}_F \leq T_1\right) = o\left(\eta_n^2\right).
$$


\noindent
Д\,о\,к\,а\,з\,а\,т\,е\,л\,ь\,с\,т\,в\,о\,.\ \ Используя требование $q_n\hm\leq1/2$ и свойства 
квантилей нормального распределения~\cite{AdaptingFDR}, можно показать, что при 
всех достаточно больших~$n$ справедливо 
$$
t_{\kappa_n}\equiv \sigma z\left( \fr{q_n \kappa_n}{2n}\right) > T_1,
$$
 откуда
$$
\p\left(\hat{t}_F \leq T_1\right) \leq \p\left(\hat{t}_F \leq 
t_{\kappa_n}\right) = \p \left(\hat{k}_F \geq \kappa_n \right).
$$
Заметив, что $\gamma_n \hm> \ln^{-1} n$, $\kappa_n \hm> \eta_n n/2$, $q_n \hm> c_3 \ln^{-1} n$ 
для всех достаточно больших~$n$, и применив лемму~1 с $m\hm= n^{\delta_1} \ln^5 n$, получим требуемое.\hfill$\square$

\smallskip

Следующее утверждение доказано в~\cite{AdaptingFDR} для $\sigma\hm=1$ и $T_1 \hm\geq 
3^{1/4}$, приведенное ниже обобщение элементарно.


\smallskip

\noindent
\textbf{Лемма~3.}
\textit{
Пусть $\hat{t}$~--- произвольный случайный порог, $\eta_n\hm\leq b\hm<1$, $x_i\sim 
N(\mu_i, \sigma^2)$, $(\hat{\mu})_i \hm= p(x_i, \hat{t})$, $i\hm=1,\ldots,n$. Тогда с 
некоторой константой} $c \hm\equiv c(\sigma, b)$
$$
\e \left|\left| \hat{\mu} - \mu\right|\right|^2 \Ik(\hat{t}\leq T_1) \leq c 
 T_1^2  n \left(\p (\hat{t}\leq T_1)\right)^{1/2}.
$$



Перейдем, наконец, к основному утверждению работы.


\smallskip

\noindent
\textbf{Теорема~2.}
\textit{Пусть выполнены требования леммы}~2. \textit{При} $n\hm\to\infty$
$$
\sup\limits_{\mu\in l_0[\eta_n]}\rho(\hat{\mu}_F, \mu)  \leq n \eta_n T_U^2 (1+o(1)).
$$


\noindent
Д\,о\,к\,а\,з\,а\,т\,е\,л\,ь\,с\,т\,в\,о\,.\ \ Пусть $\mu \hm\in l_0[\eta_n]$. Имеем
\begin{multline}
\label{risk1}
\rho(\hat{\mu}_F, \mu) = \e \left|\left| \hat{\mu}_F - \mu\right|\right|^2 
\Ik(\hat{t}_F\leq T_1) + {}\\
{}+\e \left|\left| \hat{\mu}_F - \mu\right|\right|^2 
\Ik(\hat{t}_F > T_1).
\end{multline}
Используя леммы~2 и~3, для первого слагаемого в~(\ref{risk1}) 
получим
\begin{equation}
\label{proof1}
\e \left|\left| \hat{\mu}_F - \mu\right|\right|^2 \Ik(\hat{t}_F\leq T_1) \leq n 
\, o(\eta_n) \ln \eta_n^{-1}.
\end{equation}
Заметим, что
\begin{equation*}
\left(p(x_i, t)-\mu_i\right)^2 =
 \begin{cases}
   (x_i-\mu_i)^2, & |x_i| > t;\\
   \mu_i^2, & |x_i| \leq t.
 \end{cases}
\end{equation*}
Отсюда для второго слагаемого в~(\ref{risk1})
\begin{multline*}
\e \left|\left| \hat{\mu}_F - \mu\right|\right|^2 \Ik(\hat{t}_F > T_1) = {}\\
{}=
\e  \sum\limits_{i=1}^n \left(p(x_i, \hat{t}_F)-\mu_i\right)^2  \Ik(T_1 < \hat{t}_F  
\leq T_U) ={}\\
{}= \e  \sum\limits_{i=1}^n \left((x_i-\mu_i)^2 \Ik(|x_i| >\hat{t}_F) + {}\right.\\
\left.{}+\mu_i^2 
\Ik(|x_i| \leq \hat{t}_F)\right) \Ik(T_1 < \hat{t}_F  \leq T_U) \leq{}\\
\!{}\leq \e  \sum\limits_{i=1}^n \left((x_i-\mu_i)^2 \Ik(|x_i| >T_1) + \mu_i^2 
\Ik(|x_i| \leq T_U)\right) \equiv{}\\
{}\equiv E_1+E_2,
\end{multline*}
где
\begin{multline*}
E_1 ={}\\
\!{}= \e \!\! \sum\limits_{i : |\mu_i| > 0}\!\!\! \left((x_i-\mu_i)^2 \Ik(|x_i| >T_1) + 
\mu_i^2 \Ik(|x_i| \leq T_U)\right);\hspace*{-6.19644pt}
\end{multline*}
$$
E_2 = \e  \sum\limits_{i : |\mu_i| =0} x_i^2 \Ik(|x_i| >T_1).
$$
Пусть $\xi \sim N(0,1)$, $x\hm>0$, тогда
$$
\e \xi^2 \Ik
(|\xi|>x)  \leq 2\left(x + \fr{1}{x}\right)\varphi(x),
$$
где использовано неравенство $1\hm-\Phi(x)\hm\leq\varphi(x)/x$, $x\hm>0$ ($\Phi(x)$ 
и~$\varphi(x)$~--- соответственно функция распределения и плот\-ность~$N(0,1)$). 
Отсюда

\columnbreak

\noindent
\begin{multline}
\label{proof2}
E_2 \leq \sqrt{\fr{2}{\pi}} \, n \fr{T_1}{\sigma}\, e^{-T_1^2/(2\sigma^2)}(1 + o(1)) = {}\\
{}=
O\left(n \eta_n \sqrt{\ln \eta_n^{-1}}\right). 
\end{multline}
Пусть далее $\xi \sim N(\mu,\sigma)$, тогда если $|\mu| \hm\leq T_U$, то $\mu^2 \p 
(|\xi|\hm\leq T_U) \hm\leq T_U^2$.
Если же $\mu \hm> T_U$ (для $\mu\hm < -T_U$ аналогично), используя $2(1\hm-\Phi(x))\hm \leq 
e^{-x^2/2}$ для $x\hm\geq 0$, получим
\begin{multline*}
\mu^2 \p (|\xi|\leq T_U) < \mu^2 \left(1-\Phi\left(\fr{\mu- T_U}{\sigma}\right)\right) \leq{}\\
{}\leq \fr{\mu^2}{2} e^{-(\mu-T_U)^2/(2\sigma^2)}  \leq \fr{T_U^2}{2}+O(T_U),
\end{multline*}
где последнее неравенство можно получить, исследуя выражение в левой части на 
экстремум по~$\mu$.
Из приведенных соотношений следует, что
\begin{multline}
\label{proof3}
E_1 \leq n\eta_n \sigma^2 + \sum\limits_{i : |\mu_i| > 0} \mu_i^2 \p (|x_i|\leq 
T_U)\leq{}\\
{}\leq  n \eta_n T_U^2 (1+o(1)).
\end{multline}
Объединяя~(\ref{proof1})--(\ref{proof3}), получаем утверждение тео\-ре\-мы.\hfill$\square$

\smallskip

От степени разреженности вектора~$\mu$ (ско\-рости убывания~$\eta_n$) зависит 
асимптотический порядок верхней границы риска, полученной в тео\-ре\-ме~2. Например, при $\eta_n\hm = n^{-\delta}$, $\delta\hm\in(0,1)$, получим
$$
\sup\limits_{\mu\in l_0[\eta_n]}\rho(\hat{\mu}_F, \mu)  \leq 2\sigma^2 \, 
n^{1-\delta} \ln n \, (1+o(1));
$$
если же $\eta_n\hm = (\ln n)^{-r}$, $r\hm>0$, то
$$
\sup\limits_{\mu\in l_0[\eta_n]}\rho(\hat{\mu}_F, \mu)  \leq 2\sigma^2 \, n 
(\ln n)^{1-r} \, (1+o(1)).
$$


{\small\frenchspacing
 {%\baselineskip=10.8pt
 %\addcontentsline{toc}{section}{References}
 \begin{thebibliography}{9}
\bibitem{MultipleTesting}
\Au{Menyhart~O., Weltz~B., Gyorffy~B.}
MultipleTesting.com: A~tool for life science researchers for multiple hypothesis 
testing correction~// PLoS One, 2021. Vol.~16. No.\,6. Art.~0245824.

\bibitem{FDRImage}
\Au{Krylov V.\,A., Moser~G., Serpico~S.\,B., Zerubia~J.}
False discovery rate approach to unsupervised image change detection~// IEEE 
T. Image Process., 2016. Vol.~25. No.~10. P.~4704--4718.

\bibitem{AdaptingFDR}
\Au{Abramovich~F., Benjamini~Y., Donoho~D., Johnstone~I.}
Adapting to unknown sparsity by controlling the false discovery rate~//  
Ann. Stat., 2006. Vol.~34. No.\,2. P.~584--653.

\bibitem{ResultsOnFDRUnderDependence}
\Au{Farcomeni~A.\/}
Some results on the control of the false discovery rate under dependence~// 
Scand. J. Stat., 2007. Vol.~34. No.\,2. P.~275--297.

\bibitem{spatialAdaptation}
\Au{Donoho~D., Johnstone~I.}
Ideal spatial adaptation via wavelet shrinkage~// Biometrika, 1994. Vol.~81. 
No.\,3. P.~425--455.

\bibitem{AdaptingSURE}
\Au{Donoho D., Johnstone~I.\,M.}
Adapting to unknown smoothness via wavelet shrinkage~// J.~Am. Stat. Assoc., 
1995. Vol.~90. P.~1200--1224.

\bibitem{ExactRisk}
\Au{Marron J.\,S., Adak S., Johnstone~I.\,M., Neumann~M.\,H., Patil~ P.}
Exact risk analysis of wavelet regression~// J.~Comput. Graph. Stat., 1998. 
Vol.~7. P.~278--309.

\bibitem{Bosq}
\Au{Bosq~D.}
Nonparametric statistics for stochastic processes: Estimation and prediction.~--- 
Lecture notes in statistics ser.~--- New York, NY, USA: Springer, 1996. Vol.~110. 
188~p.

\bibitem{Pollard}
\Au{Pollard~D.}
Convergence of stochastic processes.~--- Springer ser. in statistics.~--- New 
York, NY, USA: Springer, 1984. 215~p.
\end{thebibliography}

 }
 }

\end{multicols}

\vspace*{-6pt}

\hfill{\small\textit{Поступила в~редакцию 05.12.22}}

\vspace*{8pt}

%\pagebreak

%\newpage

%\vspace*{-28pt}

\hrule

\vspace*{2pt}

\hrule

%\vspace*{-2pt}

\def\tit{MEAN-SQUARE RISK OF~THE~FDR PROCEDURE\\ UNDER~WEAK DEPENDENCE}


\def\titkol{Mean-square risk of~the~FDR procedure under~weak dependence}


\def\aut{M.\,O.~Vorontsov$^{1,2}$ and~O.\,V.~Shestakov$^{1,2,3}$}

\def\autkol{M.\,O.~Vorontsov and~O.\,V.~Shestakov}

\titel{\tit}{\aut}{\autkol}{\titkol}

\vspace*{-10pt}


\noindent
$^{1}$M.\,V.~Lomonosov Moscow State University, 1-52~Leninskie Gory, GSP-1, Moscow 119991, Russian Federation

\noindent
$^{2}$Moscow Center for Fundamental and Applied Mathematics, M.\,V.~Lomonosov Moscow State University,\linebreak
$\hphantom{^1}$1~Leninskie Gory, GSP-1, Moscow 119991, Russian Federation

\noindent
$^{3}$Federal Research Center ``Computer Science and Control'' of the Russian Academy of Sciences, 44-2~Vavilov\linebreak
$\hphantom{^1}$Str., Moscow 119333, Russian Federation


\def\leftfootline{\small{\textbf{\thepage}
\hfill INFORMATIKA I EE PRIMENENIYA~--- INFORMATICS AND
APPLICATIONS\ \ \ 2023\ \ \ volume~17\ \ \ issue\ 2}
}%
 \def\rightfootline{\small{INFORMATIKA I EE PRIMENENIYA~---
INFORMATICS AND APPLICATIONS\ \ \ 2023\ \ \ volume~17\ \ \ issue\ 2
\hfill \textbf{\thepage}}}

\vspace*{3pt}



\Abste{In many application areas, the problem of processing large amounts of data arises. 
In this case, before processing, the data array is often subjected to some transformation leading to a~``sparse'' or ``economical'' 
representation in which the absolute value of most elements of the array is equal to zero (or sufficiently small). 
In addition, as a~result of interference when receiving and transmitting data, they become corrupted with noise and it is desirable 
to remove this noise during further processing. The resulting task is mathematically equivalent to some multiple hypothesis testing problems. 
Previously, to solve this problem under conditions of normality, independence, and sparsity of data, a procedure based on the method 
of controlling the average proportion of erroneously rejected hypotheses was proposed (False Discovery Rate, FDR). 
In this paper, the authors study the asymptotics of the mean-square risk of this procedure in the case of a~weak dependence in the data.}

\KWE{thresholding; multiple hypothesis testing; mean-square risk}



\DOI{10.14357/19922264230205}{AVJZDX}

%\vspace*{-11pt}

%\Ack
%\noindent

  

%\vspace*{4pt}

  \begin{multicols}{2}

\renewcommand{\bibname}{\protect\rmfamily References}
%\renewcommand{\bibname}{\large\protect\rm References}

{\small\frenchspacing
 {%\baselineskip=10.8pt
 \addcontentsline{toc}{section}{References}
 \begin{thebibliography}{9} 
\bibitem{1-she-1}
\Aue{Menyhart, O., B.~Weltz, and B.~Gyorffy.} 2021.
MultipleTesting.com: A~tool for life science researchers for multiple hypothesis testing correction. \textit{PLoS One} 16(6):0245824. doi: 10.1371/journal.pone.0245824.

\bibitem{2-she-1}
\Aue{Krylov, V.\,A., G.~Moser, S.\,B.~Serpico, and J.~Zerubia.} 2016.
False discovery rate approach to unsupervised image change detection. \textit{IEEE T. Image Process.} 25(10):4704--4718. doi: 10.1109/TIP.2016.2593340.
\bibitem{3-she-1}
\Aue{Abramovich, F., Y.~Benjamini, D.~Donoho, and I.\,M.~Johnstone.} 2006. 
Adapting to unknown sparsity by controlling the false discovery rate. \textit{Ann. Stat.} 34(2):584--653. doi: 10.1214/009053606000000074.

\bibitem{4-she-1}
\Aue{Farcomeni, A.} 2007. 
Some results on the control of the false discovery rate under dependence. \textit{Scand. J. Stat.} 34(2):275--297.
doi: 10.1111/j.1467-9469.2006.00530.x.

\bibitem{5-she-1}
\Aue{Donoho, D., and I.\,M.~Johnstone.} 1994. 
Ideal spatial adaptation via wavelet shrinkage. \textit{Biometrika} 81(3):425--455. doi: 10.1093/biomet/81.3.425.

\bibitem{6-she-1}
\Aue{Donoho, D., and I.\,M.~Johnstone.} 1995.
Adapting to unknown smoothness via wavelet shrinkage. \textit{J.~Am. Stat. Assoc.} 90(432):1200--1224. doi: 10.1080/01621459. 1995.10476626.

\bibitem{7-she-1}
\Aue{Marron, J.\,S., S.~Adak, I.\,M.~Johnstone, M.\,H.~Neumann, and P.~Patil.} 1998. 
Exact risk analysis of wavelet regression. \textit{J.~Comput. Graph. Stat.} 7(3):278--309. doi: 10.1080/ 10618600.1998.10474777.

\bibitem{8-she-1}
\Aue{Bosq, D.} 1996.
\textit{Nonparametric statistics for stochastic processes: Estimation and prediction}. Lecture notes in statistics ser. New York, NY: Springer Verlag. 188~p.


\bibitem{9-she-1}
\Aue{Pollard, D.} 1984. 
\textit{Convergence of stochastic processes.} Springer ser. in statistics. New York, NY: Springer. 215~p.
\end{thebibliography}

 }
 }

\end{multicols}

\vspace*{-6pt}

\hfill{\small\textit{Received December 5, 2022}} 
\Contr

\noindent
\textbf{Vorontsov Mikhail O.} (b.\ 1996)~--- PhD student, Department of Mathematical Statistics, Faculty of Computational Mathematics and Cybernetics, 
M.\,V.~Lomonosov Moscow State University, 1-52~Leninskie Gory, GSP-1, Moscow 119991, Russian Federation;
mathematician,  Moscow Center for Fundamental and Applied Mathematics, 
M.\,V.~Lomonosov Moscow State University, 1~Leninskie Gory, GSP-1, Moscow 119991, Russian Federation;
\mbox{m.vtsov@mail.ru}

\vspace*{5pt}

\noindent
\textbf{Shestakov Oleg V.} (b.\ 1976)~--- Doctor of Science in physics and mathematics, professor, Department of Mathematical Statistics, 
Faculty of Computational Mathematics and Cybernetics, M.\,V.~Lomonosov Moscow State University, 1-52~Leninskie Gory, GSP-1, Moscow 119991, 
Russian Federation; senior scientist, Institute of Informatics Problems, Federal Research Center ``Computer Science and Control'' of 
the Russian Academy of Sciences, 44-2~Vavilov Str., Moscow 119333, Russian Federation; leading scientist, Moscow Center for Fundamental and Applied Mathematics, 
M.\,V.~Lomonosov Moscow State University, 1~Leninskie Gory, GSP-1, Moscow 119991, Russian Federation; \mbox{oshestakov@cs.msu.su}
\label{end\stat}

\renewcommand{\bibname}{\protect\rm Литература}  %5
\def\stat{bosov+stef}

\def\tit{УПРАВЛЕНИЕ ВЫХОДОМ СТОХАСТИЧЕСКОЙ ДИФФЕРЕНЦИАЛЬНОЙ СИСТЕМЫ 
ПО~КВАДРАТИЧНОМУ КРИТЕРИЮ. I.~ОПТИМАЛЬНОЕ РЕШЕНИЕ МЕТОДОМ 
ДИНАМИЧЕСКОГО ПРОГРАММИРОВАНИЯ$^*$}

\def\titkol{Управление выходом стохастической дифференциальной системы 
по~квадратичному критерию. I}
%.~Оптимальное решение методом 
%динамического программирования}

\def\aut{А.\,В.~Босов$^1$, А.\,И.~Стефанович$^2$}

\def\autkol{А.\,В.~Босов, А.\,И.~Стефанович}

\titel{\tit}{\aut}{\autkol}{\titkol}

\index{Босов А.\,В.}
\index{Стефанович А.\,И.}
\index{Bosov A.\,V.}
\index{Stefanovich A.\,I.}




{\renewcommand{\thefootnote}{\fnsymbol{footnote}} \footnotetext[1]
{Работа выполнена при частичной поддержке РФФИ (проект 16-07-00677).}}


\renewcommand{\thefootnote}{\arabic{footnote}}
\footnotetext[1]{Институт проблем информатики Федерального исследовательского центра <<Информатика 
и~управление>> Российской академии наук, \mbox{AVBosov@ipiran.ru}}
\footnotetext[2]{Институт проблем информатики Федерального исследовательского центра <<Информатика 
и~управление>> Российской академии наук, \mbox{AStefanovich@frccsc.ru}}

%\vspace*{8pt}



  
  \Abst{Решается задача оптимального управления для диффузионного процесса 
Ито и~линейного управ\-ля\-емо\-го выхода. Рассматриваемая постановка близка 
к~классической ли\-ней\-но-квад\-ра\-тич\-ной гауссовской задаче управления 
(linear-quadratic Gaussian (LQG) control). Отличия состоят в~том, что состояние описывается нелинейным 
дифференциальным уравнение Ито $dy_t\hm= A_t(y_t) \,dt\hm+ \Sigma_t(y_t)\,dv_t$ 
и~не зависит от управ\-ле\-ния~$u_t$, оптимизации подлежит управ\-ля\-емый 
линейный выход $dz_t\hm= a_t y_t\,dt\hm+ b_t z_t \,dt\hm+ c_t u_t \,dt\hm+ \sigma_t\, 
dw_t$. Дополнительные обобщения внесены в~квад\-ра\-тич\-ный критерий качества 
с~целью воз\-мож\-ности постановки таких задач, как отслеживание выходом 
состояния или управ\-ле\-ни\-ем~--- линейной комбинации состояния и~выхода. Для 
решения используется метод динамического программирования. Функцию 
Беллмана позволяет найти предположение о~ее структуре вида $V_t(y,z)\hm= 
\alpha_t z^2\hm+ \beta_t(y)z \hm+\gamma_t(y)$. Решение дают три 
дифференциальных уравнения для коэффициентов~$\alpha_t$, $\beta_t(y)$ 
и~$\gamma_t(y)$. Эти уравнения со\-став\-ля\-ют оптимальное решение 
рас\-смат\-ри\-ва\-емой задачи.}
  
  \KW{стохастическое дифференциальное уравнение; оптимальное управ\-ле\-ние; 
динамическое программирование; функция Беллмана; уравнение Риккати; 
линейные уравнения параболического типа}

\DOI{10.14357/19922264180314}
  
%\vspace*{4pt}


\vskip 10pt plus 9pt minus 6pt

\thispagestyle{headings}

\begin{multicols}{2}

\label{st\stat}

\section{Введение}

     Ключевые результаты в~области оптимизации стохастических 
динамических систем, со\-став\-ля\-ющие классическую теорию управления, 
получены более~40~лет назад (такова работа~[1] в~отношении задачи 
управ\-ле\-ния ли\-ней\-но-гаус\-сов\-ски\-ми стохастическими сис\-те\-ма\-ми по 
квад\-ра\-тич\-но\-му критерию). К~классической тео\-рии следует относить 
линейные модели стохастических сис\-тем и~квадратичный критерий качества. 
Это исходный базис, на котором основано множество успешно 
исследованных и~решенных задач стохастического управ\-ле\-ния 
и~оптимизации. 

Дальнейшее развитие~--- это новые модели и~критерии, но 
прежде всего это новые методы: от тео\-рии линейных регуляторов, метода 
динамического программирования и~принципа максимума к~адаптивному 
и~минимаксному подходу, импульсному управ\-ле\-нию и~т.\,д. Множество 
инноваций как в~час\-ти моделей, так и~в~час\-ти математического аппарата, 
имевших мес\-то в~по\-сле\-ду\-ющие годы, существенно обогатили тео\-рию 
управ\-ле\-ния. Но и~до настоящего времени линейные модели и~квадратичный 
критерий, несмотря на всю справедливую критику в~отношении их 
аде\-кват\-ности и~гиб\-кости, сохраняют исследовательский интерес и~находят 
современные области приложения.
     
     Не претендуя на сколь\-ко-ни\-будь полное обосно\-ва\-ние последнего 
тезиса, приведем несколько примеров, показавшихся наиболее ин\-те\-рес\-ными. 

Так, в~[2] решается ли\-ней\-но-квад\-ра\-тич\-ная за\-да\-ча в~игровой 
постановке с~запаздыванием. В~близ\-кой по модели работе~[3] задача 
управ\-ле\-ния ставится в~терминах $H_\infty$-ро\-баст\-ности. Точнее \mbox{называть} 
эту тематику $H_2/H_\infty$-управ\-ле\-ни\-ем, и~работ по этой теме очень 
много. Аккуратности ради следует уточнить, что под линейными 
понимаются модели с~мультипликативными по состоянию воз\-му\-ще\-ниями. 

Совсем другой класс моделей, особо популярных в~по\-след\-ние годы, 
составляют скачкообразные процессы. Например, линейные уравнения 
в~сочетании с~пуассоновскими скачками в~[4] используются в~моделях, 
описывающих различные показатели функционирования сетевых протоколов 
передачи данных транспортного уровня. Телекоммуникации представляют 
в~последние годы самый популярный прикладной материал для 
исследований, работ по этой проб\-ле\-ма\-ти\-ке множество, математические 
техники привлекаются самые разные и~самые современные, но и~линейным 
моделям место находится. Еще один любопытный пример исследования 
скачкообразного процесса и~оптимизации на основе квад\-ра\-тич\-но\-го критерия 
можно найти в~[5] применительно к~задаче инвестирования на финансовом 
рынке. Наконец, упомянем еще работу~[6], подводящую итог исследований 
в~отношении классической детерминированной  
ли\-ней\-но-квад\-ра\-тич\-ной задачи с~использованием техники матричных 
неравенств.
     
     В данной работе также эксплуатируются привлекательные свойства 
линейных моделей и~квад\-ра\-тич\-но\-го критерия, причем в~стохастической 
постановке. На\-прав\-ле\-ни\-ем для обобщения \mbox{выбрана} модель динамики 
сис\-те\-мы: основные усилия на\-прав\-ле\-ны на то, чтобы сделать ее нелинейной. 
Кроме того, пред\-став\-лен\-ная постановка может рас\-смат\-ри\-вать\-ся и~как 
обобщение ранее решенной задачи в~дискретном времени~[7, 8] на время 
непрерывное. В~упомянутых работах помимо собственно модельной 
постановки важна еще и~привлекаемая прикладная об\-ласть~--- 
функционирование сложных программных сис\-тем. Результатов, 
ориентированных непосредственно на такие приложения, к~настоящему 
времени пренебрежимо мало, поэтому~[7, 8]~--- это еще и~прикладное 
обоснование рас\-смат\-ри\-ва\-емой далее задачи.
     
     Оптимизируемая динамическая сис\-те\-ма описывается двумя 
уравнениями. Состояние задается нелинейным стохастическим 
дифференциальным уравнением Ито, не содержащим управ\-ля\-емой 
переменной. Возмущение здесь описывается стандартным винеровским 
процессом, накладываются простые условия существования 
и~един\-ст\-вен\-ности решения. Поскольку состояние не управ\-ля\-ет\-ся, то уместно 
его интерпретировать как слож\-ное внешнее возмущение. Вторая 
переменная~--- управ\-ля\-емый выход~--- задается линейным стохастическим 
дифференциальным уравнением. Цель оптимизации выхода формируется 
квадратичным критерием общего вида. Формальная постановка задачи 
приведена в~сле\-ду\-ющем разделе.
     
     Для решения задачи используется метод динамического 
программирования, решается уравнение Беллмана~[9]. Соответственно, 
в~результате получаются аналитические выражения и~для оптимального 
управ\-ле\-ния, и~для значения функционала качества. Технически 
традиционный, стандартный подход к~задаче обременен, пожалуй, 
единственной проблемой~--- поиском верного пред\-став\-ле\-ния структуры 
функции Беллмана. Справиться с~этой проблемой в~большей степени удается 
за счет результата, полученного при решении дискретного по времени 
аналога рассматриваемой постановки~\cite{8-bos}. Конечные соотношения 
для оптимального решения, как и~во всех подобных задачах, включая 
классическую ли\-ней\-но-квад\-ра\-тич\-ную, содержат решения 
определенных дифференциальных уравнений (обыкновенных и~в~частных 
производных). Вывод этих уравнений и~со\-став\-ля\-ет содержание первой час\-ти 
данной работы. Во второй части будет обсуждаться их приближенное 
чис\-лен\-ное решение и~компьютерные эксперименты.
     
     Кратко обозначим основные положения, при\-вле\-ка\-емые далее 
к~решению задачи, следуя в~основном обозначениям 
и~терминологии~\cite{9-bos}, а~именно: будем рассматривать задачу 
оптимального управления в~стохастической динамической сис\-те\-ме по полной 
информации, применяя метод динамического программирования. В~качестве 
целевого функционала, опре\-де\-ля\-юще\-го качество управ\-ле\-ния $U_0^T\hm= \{ 
u_t,\ 0\leq t\leq T\}$, выступает
     \begin{equation}
     J\left(U_0^T\right)={\sf E}\left\{ \int\limits_0^T L_t \left(x_t, u_t\right)\,dt+ 
l\left(x_T\right)\right\}\,.
     \label{e1-bos}
     \end{equation}
Здесь ${\sf E}\{\cdot\}$~--- оператор математического ожидания; $x_t$~--- 
случайный процесс, описываемый стохастическим дифференциальным 
уравнением Ито
     \begin{equation}
     dx_t=m_t\left( x_t, u_t\right) dt+ \sigma_t\left( x_t\right)dW_t\,,\enskip 
x_0=X\,,
     \label{e2-bos}
     \end{equation}
где $W_t$~--- стандартный винеровский процесс подходящей раз\-мер\-ности; 
$X$~--- случайный вектор.

     $U_0^T$ будем выбирать из класса допустимых неупреждающих (по 
отношению к~$W_t$) управлений~\cite{9-bos}. Соответственно, 
относительно функций сноса и~диффузии~$m_t$ и~$\sigma_t$  
в~(\ref{e2-bos}) будем предполагать выполненными ка\-кие-ли\-бо условия 
существования сильного решения для заданного до\-пус\-ти\-мо\-го управ\-ле\-ния. 
Например, для управ\-ле\-ния с~обратной связью $u_t\hm= u_t(x_t)$ будем 
считать, что $m_t(x,u_t(x))$ и~$\sigma_t(x)$ удовлетворяют условию 
линейного рос\-та и~локальному условию Липшица по~$x$ равномерно 
по~$t$ (т.\,е.\ условиям Ито).
     
     Для поиска оптимального управления, минимизирующего $J(U_0^T)$, 
рас\-смат\-ри\-ва\-ет\-ся функция Беллмана
     \begin{equation}
     V_t(x)=\left.\mathop{\mathrm{inf}}\limits_{U_t^T} {\sf E} \left\{ \int\limits_t^T 
L_t \left( x_t, u_t\right)\,dt+l\left( x_T\right) \right\vert \mathcal{F}_t^x\right\}\,,
     \label{e3-bos}
     \end{equation}
где $\mathcal{F}_t^x$~--- $\sigma$-ал\-геб\-ра, по\-рож\-ден\-ная~$x_\tau$, 
$0\hm\leq \tau\hm\leq t$, ${\sf E}\{\cdot\vert \mathcal{F}\}$~--- оператор условного 
математического ожидания относительно~$\mathcal{F}$. Соответственно, 
в~качестве достаточного условия оп\-ти\-маль\-ности воспользуемся уравнением 
динамического программирования
\begin{multline}
\fr{\partial V_t(x)}{\partial t} +\fr{1}{2}\sum\limits^n_{i,j=1} \sigma^2_{t_{ij}}
\fr{\partial^2 V_t(x)}{\partial x_i \partial x_j}+{}\\
{}+\min\limits_u\left[  
\sum\limits^n_{i=1} m_{t_i} \fr{\partial V_t(x)}{\partial x_i} + L_t(x,u)\right] 
=0\,,\\
V_T(x)=l(x)\,,
\label{e4-bos}
\end{multline}
где $m_{t_i}$~--- $i$-й элемент век\-тор-функ\-ции~$m_t(x,u)$; 
$\sigma^2_{t_{ij}} \hm= \sum\nolimits^m_{k=1} 
\sigma_{t_{ik}}\sigma_{t_{ki}}$, $\sigma_{t_{ij}}$~--- $i$-й по строке, $j$-й 
по столб\-цу элемент мат\-рич\-ной функции~$\sigma_t(x)$; $n$ и~$m$~--- 
размерности~$x_t$ и~$W_t$ соответственно.

     Традиционно в~рамках применения метода динамического 
программирования будем предполагать, что функции~$L_t$, $l$, $m_t$ 
и~$\sigma_t$ обеспечивают существование хотя бы одного решения 
уравнения~(\ref{e4-bos}), а~следовательно, и~оптимального 
управления~$u_t^*$, $0\hm\leq t\hm\leq T$, до\-став\-ля\-юще\-го минимум 
целевому функционалу~(\ref{e1-bos}). Задача оптимизации далее получается 
путем указания конкретных выражений для~$L_t$, $l$, $m_t$ и~$\sigma_t$.

\section{Постановка задачи управления выходом}

     Рассматриваемые далее случайные функции будут предполагаться 
скалярными. Такое упрощение позволит разгрузить выкладки и~итоговые 
выражения от не самых существенных деталей.
     
     Рассмотрим стохастическую дифференциальную сис\-те\-му, со\-сто\-яние 
которой представляет диффузи\-он\-ный процесс~$y_t$, описываемый 
нелинейным стохастическим дифференциальным уравнением Ито
     \begin{equation}
     dy_t=A_t\left( y_t\right) dt +\Sigma_t \left( y_t\right) dv_t\,,\enskip 
y_0=Y\,,
     \label{e5-bos}
     \end{equation}
где $v_t$~--- стандартный (одномерный) винеровский процесс; $Y$~--- 
случайная величина с~конечным вторым моментом; функции~$A_t$ 
и~$\Sigma_t$ удовлетворяют условиям Ито:
\begin{equation*}
\left\vert A_t(y)\right\vert +\left\vert \Sigma_t(y)\right\vert \leq C(1+\vert y\vert )\ 
\mbox{для\ всех } 0\leq t\leq T\,;
\end{equation*}

\vspace*{-12pt}

\noindent
\begin{multline*}
\hspace*{-2.10051pt}\left\vert A_t\left(y_1\right) -A_t \left( y_2\right) \right\vert +\left\vert 
\Sigma_t\left( y_1\right) -\Sigma_t \left(y_2\right)\right\vert \leq
C\left\vert y_1-y_2\right\vert\\
 \mbox{для\ всех\ } 0\leq t\leq T\ \mbox{и } 
y_1,y_2\in \mathbb{R}^1\,,
\end{multline*}
обеспечивающим существование единственного сильного (потраекторного) 
решения уравнения.
     
     Будем считать, что~$y_t$ описывает состояние некоторой 
динамической системы. Соответственно, поведение этой сис\-те\-мы опишем 
выходом, линейно связанным с~со\-сто\-янием:
     \begin{equation}
     dz_t=a_t y_t \,dt+ b_t z_t \,dt+ c_t u_t \,dt+\sigma_t \,dw_t\,,\enskip
     z_0=Z\,.
     \label{e6-bos}
     \end{equation}
Здесь $w_t$~--- не зависящий от~$v_t$, $Y$ и~$Z$ стандартный (одномерный) 
винеровский процесс; $Z$~--- случайная величина с~конечным вторым 
моментом; $u_t$~--- допустимое неупреждающее управ\-ле\-ние, качество 
которого определяется целевым функционалом следующего вида:
\begin{multline}
\!\hspace*{-3.98538pt}J\left( U_0^T\right) ={\sf E}\left\{ \int\limits_0^T \!\left( S_t\left( s_ty_t-g_t z_t -h_t 
u_t\right)^2 +G_t z_t^2+{}\right.\right.\\
\left.\left.{}+ H_t u_t^2
\vphantom{S_t\left( s_ty_t-g_t z_t -h_t 
u_t\right)^2}
\right) dt+S_T\left( s_T y_T -g_T 
z_T\right)^2+G_T z_T^2
\vphantom{\int\limits_0^T}\right\}\,,
\label{e7-bos}
\end{multline}
где $S_t$, $G_t$ и~$H_t$~--- неотрицательные функции\linebreak
$0\hm\leq t\hm\leq T$. 
Такой критерий отражает физический смысл задачи распределения ресурсов 
со\-глас\-но аналогичной~(\ref{e5-bos})--(\ref{e7-bos}) задаче для дис\-крет\-но\-го 
времени, рас\-смот\-рен\-ной в~\cite{7-bos}. В~част\-ности,  
функци\-онал~(\ref{e7-bos}) поз\-во\-ля\-ет ставить задачи отслеживания
 выходом 
со\-сто\-яния сис\-те\-мы, используя сла\-га\-емое $(y_t\hm- z_t)^2$, или 
управлением~--- линейной комбинации со\-сто\-яния и~выхода, сла\-га\-емое типа\linebreak 
$(y_t\hm+ z_t\hm- u_t)^2$. Поскольку задача формулируется 
в~предположении наличия пол\-ной информации о~со\-сто\-янии~$y_t$ 
и~выходе~$z_t$ (соответствующую $\sigma$-ал\-геб\-ру 
обозначим~$\mathcal{F}_t^{y,z}$), то допустимое управ\-ле\-ние ищется 
в~классе~$\mathcal{F}_t^{y,z}$-из\-ме\-ри\-мых неупреждающих функций 
(и,~как будет показано далее, оказывается управ\-ле\-ни\-ем с~обратной связью).

     Функции~$a_t$, $b_t$, $c_t$ и~$\sigma_t$ будем предполагать 
ограниченными: $\vert a_t\vert \hm+ \vert b_t\vert \hm+\vert c_t\vert \hm+ \vert 
\sigma_t \vert \hm\leq C$ для всех $0\hm\leq t\hm\leq T$, процесс  
управления~--- допустимым не\-упреж\-да\-ющим~\cite{9-bos}, обеспечивая, 
таким образом, существование сильного решения урав\-не\-ния~(\ref{e6-bos}) 
для любого допустимого управ\-ления.
     
     Задачу составляет поиск~$u_t^*$~--- допустимого управ\-ле\-ния, 
доставляющего минимум квад\-ра\-тич\-но\-му функционалу~$J(U_0^T)$.
      
     Поставленная задача очевидным образом формулируется в~терминах 
введенных выше в~(\ref{e1-bos})--(\ref{e3-bos}) обозначений, а~именно: 
     требуется обозначить
     \begin{gather*}
      x_t=\begin{pmatrix}
     y_t\\ z_t\end{pmatrix};\quad  m_t(x_t, u_t)=\begin{pmatrix}
     A_t(y_t)\\ a_t y_t +b_t z_t +c_t u_t\end{pmatrix};\\
     \sigma_t(x_t)= \begin{pmatrix}
     \Sigma_t(y_t)& 0\\
     0& \sigma_t\end{pmatrix};\quad W_t=\begin{pmatrix}
     v_t \\ w_t\end{pmatrix}
     %     \label{e8-bos}
     \end{gather*}
для записи уравнения со\-сто\-яния типа~(\ref{e2-bos}) и
\begin{align*}
L_t(x,u)&= L_t(y,z,u) ={}\\
&\hspace*{3mm}{}=S_t\left( s_t y-g_t z -h_t u\right)^2 +G_t z^2 +H_t  u^2\,;\\
l(x)&= l(y,z) =S_T \left( S_T y-g_T z\right)^2 +G_T z^2
%\label{e9-bos}
\end{align*}
для записи целевого функционала в~виде~(\ref{e1-bos}).

     Функция Беллмана~(\ref{e3-bos}) принимает вид 
     $V_t(x)\hm= V_t(y,z)$. Для записи со\-от\-вет\-ст\-ву\-юще\-го~(\ref{e4-bos}) 
уравнения Беллмана для~$V_t(y,z)$ заметим, что
     $$
     \left( \sigma^2_{t_{ij}}\right)_{i,j=1,2}= \begin{pmatrix}
     \Sigma_t^2(y) & 0\\
     0 & \sigma_t^2\end{pmatrix}\,.
     $$
     
     С~учетом перечисленных обозначений урав\-не\-ние динамического 
программирования~(\ref{e4-bos}) принимает вид:
     \begin{multline}
     \fr{\partial V_t(y,z)}{\partial t} +\fr{1}{2}\left( \Sigma_t^2(y) \fr{\partial^2 
V_t(y,z)} {\partial y^2}+\sigma_t^2\fr{\partial^2 V_t(y,z)} {\partial 
z^2}\right)+{}\\
    {}+\min\limits_u\! \left[ A_t(y) \fr{\partial V_t(y,z)}{\partial y}+\left( a_t 
y+b_t z+c_t u\right) \fr{\partial V_t(y,z)}{\partial z} +{}\right.\hspace*{-3pt}\\
\left.{}+ S_t\left( s_t y-g_t z-h_t 
u\right)^2+G_t z^2+H_t u^2
     \vphantom{\fr{\partial V_t(y,z)}{\partial y}}\right] =0\,,\\
     V_T(y,z)=S_T\left( s_T y-g_T z\right)^2+G_T z^2\,.
     \label{e10-bos}
     \end{multline}
     Это и~есть то самое уравнение, которое требуется решить: 
существование решения данного урав\-не\-ния суть достаточное условие 
оптимальности; оптимальное управ\-ле\-ние при этом~--- точ\-ка минимума 
со\-от\-вет\-ст\-ву\-юще\-го сла\-га\-емого.
     
\section{Динамическое программирование и~оптимальное 
управление}

     В рассматриваемой постановке линейность\linebreak выхода и~квадратичность 
критерия дают те же преимущества, что и~в~классической  
ли\-ней\-но-квад\-ра\-тич\-ной задаче управ\-ле\-ния~\cite{1-bos}, а~именно: 
позволяют сразу определить вид оптимального управ\-ле\-ния и~фактические 
условия его существования. Действительно, со\-хра\-няя в~(\ref{e10-bos}) под 
знаком $\min\nolimits_u$ только члены, зависящие от~$u$, получаем
     \begin{multline*}
     \fr{\partial V_t(y,z)}{\partial t} +\fr{1}{2}\left( \Sigma_t^2(y) \fr{\partial^2 
V_t(y,z)} {\partial y^2}+\sigma_t^2\fr{\partial^2 V_t(y,z)} {\partial 
z^2}\right)+{}\\
     {}+A_t(y)\fr{\partial V_t(y,z)}{\partial y}+\left( a_t y+b_t z\right) 
\fr{\partial V_t(y,z)}{\partial z}+{}\\
{}+S_t\left( s_t y-g_t z\right)^2 +G_t z^2+{}
\end{multline*}

\noindent
\begin{multline*}
     {}+\min\limits_u \left[ \left( c_t \fr{\partial V_t(y,z)}{\partial z}-2S_t \left( 
s_t y-g_t z\right) h_t\right)u +{}\right.\\
\left.{}+\left( S_t h_t^2+H_t\right) u^2
\vphantom{\fr{\partial V_t(y,z)}{\partial z}}
\right]=0\,,
     %\label{e11-bos}
     \end{multline*}
откуда в~предположении $S_t h_t^2\hm+ H_t\hm>0$ следует, что существует 
оптимальное управ\-ле\-ние, которое определяется равенством
\begin{multline}
u_t^* = u_t^*(y,z)=-\fr{1}{2}\left( S_t h_t^2 +H_t\right)^{-1} \left( c_t 
\fr{\partial V_t(y,z)}{\partial z}-{}\right.\\
\left.{}-2S_t\left( s_t y-g_t z\right) h_t
\vphantom{\fr{\partial V_t(y,z)}{\partial z}}
\right)
\label{e12-bos}
\end{multline}
и доставляет минимум соответствующему сла\-га\-емо\-му в~урав\-не\-нии Беллмана, 
равный
$-\left( S_t h_t^2\hm+\right.$\linebreak
$\left.{}+H_t\right)^{-1} \left( c_t 
{\partial V_t(y,z)}/{\partial 
z}\hm-2S_t\left( s_t y \hm-g_t z\right) h_t \right)^2/4.
$ 
     
     Отметим, что, как и~в~классической ли\-ней\-но-квад\-ра\-тич\-ной 
задаче, управ\-ле\-ние из класса до\-пус\-ти\-мых не\-упреж\-да\-ющих получилось 
управ\-ле\-ни\-ем с~обратной связью.
     
     Таким образом, функция Беллмана описывается сле\-ду\-ющим 
дифференциальным уравнением:
     \begin{multline}
     \fr{\partial V_t(y,z)}{\partial t} +\fr{1}{2}\left( \Sigma_t^2(y) \fr{\partial^2 
V_t(y,z)} {\partial y^2}+\sigma_t^2\fr{\partial^2 V_t(y,z)} {\partial 
z^2}\right)+{}\\
     {}+ A_t(y) \fr{\partial V_t(y,z)}{\partial y}+\left( a_t y+b_t z\right) 
\fr{\partial V_t(y,z)}{\partial z}+{}\\
{}+ S_t \left( s_t y- g_t z\right)^2 +G_t z^2-
 \fr{1}{4}\left( S_t h_t^2+H_t\right)^{-1}\times{}\\
 {}\times \left( c_t \fr{\partial V_t(y,z)} 
{\partial z}-2S_t\left( s_t y -g_t z\right) h_t \right)^2=0\,.
     \label{e13-bos}
     \end{multline}
     
     Возводя в~квадрат по\-след\-нее сла\-га\-емое в~(\ref{e13-bos}), перепишем 
его в~виде:
     \begin{multline}
     \fr{\partial V_t(y,z)}{\partial t} +\fr{1}{2}\left( \Sigma_t^2(y) \fr{\partial^2 
V_t(y,z)} {\partial y^2}+\sigma_t^2\fr{\partial^2 V_t(y,z)} {\partial 
z^2}\!\right)+{}\\
{}+A_t(y) \fr{\partial V_t(y,z)}{\partial y}
+ \left( 
\vphantom{\left( S_t h_t^2 +H_t\right)^{-1}}
a_t y+b_t z+{}\right.\\
\left.{}+\left( S_t h_t^2 +H_t\right)^{-1}
 c_t S_t \left( s_t y-g_t z\right) h_t
\right) 
     \fr{\partial V_t(y,z)}{\partial z}+{}\\
     {}+\left( S_t-\left( S_t h_t^2 +H_t\right)^{-1} S_t^2 h_t^2\right)\left( s_t y -
g_t z\right)^2+{}\\
     \!\!{}+
     G_t z^2 -\fr{1}{4}\left( S_t h_t^2+H_t\right)^{-1}\! c_t^2
     \left(\fr{\partial V_t(y,z)}{\partial z}\right)^{\!2}=0\,.\!\!
     \label{e14-bos}
     \end{multline}
     
     Рассматривая полученное уравнение, заметим, что его решение может 
быть пред\-став\-ле\-но в~виде:
   \begin{equation}
     V_t(y,z)= \alpha_t z^2+\beta_t(y) z +\gamma_t(y)\,,
     \label{e15-bos}
     \end{equation}
т.\,е.\ будем искать решение при дополнительном предположении 
о~квад\-ра\-тич\-ности функции Белл\-ма\-на по переменной~$z$, и~сведем, таким 
образом, поиск оптимального решения к~уравнениям относительно функций 
$\alpha_t$, $\beta_t(y)$ и~$\gamma_t(y)$. Отметим сразу, что явный вид 
функции~$\gamma_t(y)$ для реализации оптимального управ\-ле\-ния не 
требуется, однако далее будет предложен вариант вы\-чис\-ле\-ния и~этой 
функции, что пред\-став\-ля\-ет\-ся небесполезным, поскольку позволит выполнять 
расчет минимума целевого функционала. Источником для 
предложения~(\ref{e15-bos}) является уже упоминавшаяся аналогичная 
задача для случая дис\-крет\-но\-го времени~\cite{7-bos, 8-bos}. В~той задаче 
выражение для функции Беллмана получается формально без 
дополнительных усилий. При этом форма~(\ref{e15-bos}) обнаруживается 
как свойство оптимального решения. В~рассматриваемом случае 
непрерывного времени~(\ref{e15-bos}) постулируется, а~пра\-виль\-ность 
постулата под\-тверж\-да\-ет\-ся далее ре\-зуль\-ти\-ру\-ющи\-ми уравнениями 
для~$\alpha_t$, $\beta_t(y)$ и~$\gamma_t(y)$ Кроме того, данное 
предположение пред\-став\-ля\-ет\-ся вы\-те\-ка\-ющим из линейной структуры задачи 
в~отношении переменной~$z$, в~част\-ности, тем фактом, что такой вид 
функции Беллмана обеспечивает линейность оптимального 
управ\-ле\-ния~(\ref{e12-bos}) по~$z$.

     Граничное условие при выбранном предположении~(\ref{e15-bos}) 
принимает вид:

\noindent
     \begin{multline*}
     V_T(y,z)= S_T\left( s_T y- g_T z\right)^2+G_T z^2 ={}\\[-0.5pt]
     {}=\alpha_T z^2 
+\beta_T(y) z +\gamma_T(y)\,,
    \end{multline*}
т.\,е.

\noindent
\begin{align*}
\alpha_T&= S_T g_T^2 +G_T\,;\\[-0.5pt]
\beta_T(y)&=-2S_T s_T g_T y\,;\\[-0.5pt]
\gamma_T(y)&=S_T s_T^2 y^2\,.
%\label{e16-bos}
\end{align*}
          При этом само оптимальное управ\-ле\-ние, определенное 
выражением~(\ref{e12-bos}), оказывается управ\-ле\-ни\-ем с~обратной связью 
по~$y_t$ и~$z_t$:

\noindent
     \begin{multline}
     u_t^*=u_t^*(y,z) ={}\\[-0.5pt]
     {}=
     -\fr{1}{2}\left( S_t h_t^2 +H_t\right)^{-1}
     \left( c_t \left( 2\alpha_t z +\beta_t(y)\right) +{}\right.\\[-0.5pt]
    \left. {}+2S_t\left( s_t y-g_t z\right) 
h_t\right)\,.
     \label{e17-bos}
     \end{multline}
          Подставляем $V_t(y,z)\hm= \alpha_t z^2 \hm+ \beta_t(y) 
z\hm+\gamma_t(y)$ в~(\ref{e14-bos}):

\noindent
     \begin{multline*}
     \fr{\partial \alpha_t}{\partial t}\, z^2 +
     \fr{\partial \beta_t(y)}{\partial t}\,z +
     \fr{\partial \gamma_t(y)}{\partial t}+{}\\[-0.5pt]
     {}+\fr{1}{2}\left( \Sigma_t^2(y) \left( 
\fr{\partial^2\beta_t(y)}{\partial y^2}\,z +\fr{\partial^2 \gamma_t(y)}{\partial 
y^2}\right) +2\sigma_t^2\alpha_t\right)+{}\\[-0.5pt]
 {}+A_t(y)\left(\fr{\partial \beta_t(y)}{\partial y}\,z + \fr{\partial 
\gamma_t(y)}{\partial y}\right) +{}\\[-0.5pt]
\hspace*{-0.22987pt}{}+\left( a_t y+b_t z+\left( S_t h_t^2 +H_t\right)^{-1} c_t S_t \left( s_t y-
g_t z\right) h_t\right)\times{}
\end{multline*}

\noindent
\begin{multline*}
         {}\times \left( 2\alpha_t z+\beta_t(y)\right)+{}\\
     {}+\left( S_t-\left( S_t h_t^2 +H_t\right)^{-1} S_t^2 h_t^2\right)\left( s_t y-
g_t z\right)^2+{}\\
     {}+ G_t z^2 -\fr{1}{4}\left( S_t h_t^2 +H_t\right)^{-1} c_t^2 \left( 
2\alpha_t z+\beta_t(y)\right)^2=0\,.
     \end{multline*}
          Далее выделяем слагаемые при~$z^2$, $z$ и~$z^0$
          
          \noindent
     \begin{multline*}
     \fr{\partial \alpha_t}{\partial t}\, z^2 +\fr{\partial \beta_t(y)}{\partial t}\,z +
     \fr{\partial \gamma_t(y)}{\partial 
t}+\fr{1}{2}\,\Sigma_t^2(y)\fr{\partial^2\beta_t(y)}{\partial y^2}\,z+ {}\\
{}+
\fr{1}{2}\,\Sigma_t^2(y)\fr{\partial^2\gamma_t(y)}{\partial 
y^2}+\sigma_t^2\alpha_t+A_t(y)\fr{\partial \beta_t(y)}{\partial y}\,z +{}\\
{}+A_t(y) \fr{\partial 
\gamma_t(y)}{\partial y}+{}\\
{}+ 2\alpha_t \left( b_t -\left( S_t h_t^2+H_t\right)^{-1} c_t 
S_t h_t g_t \right)z^2+{}\\
     {}+
     \left( 2\alpha_t\left( \alpha_t+\left( S_t h_t^2+H_t\right)^{-1} c_t S_t h_t 
s_t\right)y +{}\right.\\
\left.{}+\beta_t(y) \left( b_t-\left( S_t h_t^2+H_t\right)^{-1} c_t S_t h_t 
g_t\right) \right) z+{}\\
     {}+\beta_t(y)\left( a_t +\left( S_t h_t^2+H_t\right)^{-1} c_t S_t h_t s_t\right) 
y+{}\\
{}+ \left( S_t -\left( S_t h_t^2+H_t\right)^{-1} S_t^2 h_t^2\right) g_t^2 z^2-{}\\
     {}- 2\left( S_t -\left( S_t h_t^2+H_t\right)^{-1} S_t^2 h_t^2\right) s_t g_t yz 
+{}\\
{}+
     \left( S_t-\left( S_t h_t^2+H_t\right)^{-1} S_t^2 h_t^2\right) s_t^2 y^2+{}\\
     {}+G_t z^2 -\left( S_t h_t^2 +H_t\right)^{-1} c_t^2 \alpha_t^2 z^2 -{}\\
     {}-\left( 
S_t h_t^2+H_t\right)^{-1} c_t^2 \alpha_t \beta_t(y) z-{}\\
{}-
\fr{1}{4}\left( S_t h_t^2+H_t\right)^{-1}  c_t^2 \beta_t^2(y)=0\,,
     \end{multline*}
группируем их и~получаем сле\-ду\-ющие уравнения:
\begin{itemize}
\item  для~$\alpha_t$:

\noindent
\begin{multline}
\fr{\partial\alpha_t}{\partial t}+2\alpha_t\left( b_t-\left( S_t h_t^2+H_t\right)^{-1} c_t 
S_t h_t g_t\right)+{}\\
{}+ \left( S_t- \left( S_t h_t^2+H_t\right)^{-1} S_t^2 h_t^2\right) 
g_t^2+G_t-{}\\
\hspace*{-8mm}{}-\left( S_t h_t^2+H_t\right)^{-1} c_t^2 \alpha_t^2 =0\,,\enskip \alpha_T=S_T 
g_t^2+G_T\,;\!\!
\label{e18-bos}
\end{multline}
\item для $\beta_t$:

\noindent
\begin{multline}
\fr{\partial\beta_t(y)}{\partial 
t}+\fr{1}{2}\,\Sigma_t^2(y)\fr{\partial^2\beta_t(y)}{\partial y^2} 
+A_t(y)\fr{\partial \beta_t(y)}{\partial y}+{}\\
{}+ 2\alpha_t\left( a_t +\left( S_t h_t^2+H_t\right)^{-1} c_t S_t h_t s_t\right) y+{}\\
{}+
\beta_t(y)\left( b_t -\left( S_t h_t^2 +H_t\right)^{-1} c_t S_t h_t g_t\right)-{}\\
{}-2\left( S_t-\left( S_t h_t^2+H_t\right)^{-1} S_t^2 h_t^2\right) s_t g_t y-{}
\\
{}-
\left( S_t h_t^2+H_t\right)^{-1} c_t^2 \alpha_t \beta_t(y)=0\,,\\
\beta_T(y)=-2S_T s_T g_T y\,;
\label{e19-bos}
\end{multline}
\item  для $\gamma_t$:
\begin{multline}
\hspace*{-0.8pt}\fr{\partial \gamma_t(y)}{\partial t}+\fr{1}{2}\,\Sigma_t^2(y)
\fr{\partial^2 \gamma_t(y)}{\partial y^2} +\sigma_t^2 \alpha_t +A_t(y)
\fr{\partial \gamma_t(y)}{\partial y}+{}\\
{}+ \beta_t(y)\left( a_t +\left( S_t h_t^2+H_t\right)^{-1} c_t S_t h_t s_t\right) y+{}\\
{}+
\left( S_t-\left( S_t h_t^2+H_t\right)^{-1} S_t^2 h_t^2\right)  s_t^2 y^2-{}\\
{}-\fr{1}{4}\left( S_t h_t^2+H_t\right)^{-1} c_t^2 \beta_t^2(y) =0\,,\\
\gamma_T(y)=S_T s_T^2 y^2\,.
\label{e20-bos}
\end{multline}
\end{itemize}
     
     Уравнение~(\ref{e18-bos}), легко заметить, является уравнением 
Риккати, которое в~силу сформулированного выше условия   
имеет единственное неотрицательное решение для всех $0\hm\leq t\hm\leq T$. 
Этот факт требует дополнительного комментария. Для получения 
уравнения~(\ref{e18-bos}) рас\-смот\-рим исходную задачу при дополнительных 
условиях $a_t\hm=0$ и~$s_t\hm=0$ для всех $0\hm\leq t\hm\leq T$. Нетрудно 
видеть, что эти условия рассматриваемую по\-ста\-нов\-ку сводят фактически 
к~классической ли\-ней\-но-квад\-ра\-тич\-ной задаче. Имеющуюся 
в~рассматриваемой формулировке чуть более общую форму целевой 
функции (принципиального значения это обобщение, конечно, не имеет) 
сведем к~классической еще одним предположением: $S_t\hm=0$ для всех 
$0\hm\leq t\hm\leq T$. Теперь уравнение для~$\alpha_t$ принимает хорошо 
известный вид:
     \begin{equation}
     \fr{\partial \alpha_t}{\partial t}+2\alpha_t b_t +G_t- H_t^{-1} c_t^2 
\alpha_t^2=0\,,\enskip \alpha_T=G_T\,.
     \label{e21-bos}
     \end{equation}

     В таком случае, как известно~\cite{10-bos}, существует единственное 
оптимальное управление~--- линейное с~обратной связью по выходу~$z_t$, 
с~коэффициентом усиления, опи\-сы\-ва\-емым уравнением  
Риккати~(\ref{e21-bos}). Именно этот результат дают  
уравнения~(\ref{e18-bos})--(\ref{e20-bos}) и~описываемая ими функция 
Беллмана~(\ref{e15-bos}), так как из $a_t\hm=0$ и~$s_t\hm=0$ немедленно 
следует, что $\beta_t(y)\hm=0$, откуда, в~свою очередь, с~учетом 
не\-за\-ви\-си\-мости решения от~$y_t$ следует, что $\gamma_t(y)\hm=\gamma_t$, 
т.\,е.\ не зависит от~$y$ и~задается уравнением: 
     $$
     \fr{\partial \gamma_t(y)}{\partial t} +\sigma^2_t \alpha_t=0\,,\enskip 
\gamma_T=0\,.
     $$ 
     Оптимальное управ\-ле\-ние при этом 
     $$
     u_t^*= -H_t^{-1} c_t \alpha_t z_t\,,
     $$
      т.\,е.\ все полностью совпадает с~известным классическим решением.
     
     С уравнениями~(\ref{e19-bos}) и~(\ref{e20-bos}) ситуация, естественно, 
обстоит сложнее. Это линейные уравнения второго порядка параболического 
типа, поскольку\linebreak
 $\Sigma_t^2(y)\hm>0$. Фактически отсутствуют 
конструктивные условия, гарантирующие существование их\linebreak
 решений 
(требовать, чтобы все фигурирующие в~уравнениях коэффициенты были 
представлены аналитическими функциями на всем пространстве значений, 
вряд ли целесообразно), поэтому далее будем предполагать, что данные 
уравнения имеют на рас\-смат\-ри\-ва\-емом интервале $0\hm\leq t\hm\leq T$ хотя 
бы одно ограниченное решение и~именно эти условия будем рас\-смат\-ри\-вать 
как достаточные условия существования оптимального решения 
рассматриваемой задачи.
     
     Таким образом, доказана следующая тео\-рема.
     
     \smallskip
     
     \noindent
     \textbf{Теорема.}\ \textit{Пусть для диффузионного 
процесса}~(\ref{e5-bos}) \textit{выполнены условия Ито, для 
     процесса}~(\ref{e6-bos})~--- \textit{ограничены коэффициенты, 
уравнения}~(\ref{e18-bos})--(\ref{e20-bos}) \textit{имеют ограниченные 
решения для $0\hm\leq t\hm\leq T$. Тогда минимум  
функционалу}~(\ref{e7-bos}) \textit{доставляет оптимальное 
управ\-ле\-ние}~(\ref{e17-bos}), \textit{где} $y\hm= y_t$; $z\hm=z_t$.
     
\section{Заключение}

     Рассмотренная задача оптимизации в~целом близка и~по модели, и~по 
критерию к~классической ли\-ней\-но-квад\-ра\-тич\-ной постановке. 
Принципиальным отличием является нелинейная модель для описания 
со\-сто\-яния динамической сис\-те\-мы, в~которой отсутствует управ\-ля\-ющее 
воздействие.\linebreak
 Такую модель наряду с~традиционной интер\-пре\-тацией  
<<со\-сто\-яние--вы\-ход>> мож\-но понимать как\linebreak модель неконтролируемого 
слож\-но\-го внешнего воздействия. Небольшое дополнительное отличие дает 
предложенная форма квад\-ра\-тич\-но\-го критерия, поз\-во\-ля\-ющая, в~част\-ности, 
ставить такие задачи, как отслеживание выходом или управ\-ле\-ни\-ем со\-сто\-яния 
сис\-те\-мы или ее выхода.
     
     Поскольку обсуждать возможности точного решения уравнений, 
определяющих оптимальное управ\-ле\-ние, не имеет смыс\-ла, наиболее 
актуальной далее является задача их приближенного чис\-лен\-но\-го решения 
и~анализа воз\-мож\-ности практической реализации. Этому посвящена вторая 
часть данной работы, пла\-ни\-ру\-емая к~выходу в~ближайшее время.

{\small\frenchspacing
 {%\baselineskip=10.8pt
 \addcontentsline{toc}{section}{References}
 \begin{thebibliography}{99}
\bibitem{1-bos}
\Au{Athans M.} Editorial on the LQG problem~// IEEE~T. Automat. Contr., 1971. Vol.~16. 
No.\,6. P.~528--552. doi: 10.1109/TAC.1971.1099845.
\bibitem{2-bos}
\Au{Wu Z.} Forward-backward stochastic differential equations, linear quadratic stochastic 
optimal control and nonzero sum differential games~// J.~Syst. Sci. Complex., 2005. Vol.~18. 
No.\,2. P.~179--192.
\bibitem{3-bos}
\Au{Chen B.\,S., Zhang~W.} Stochastic H2/H1 control with state-dependent noise~// IEEE 
T.~Automat. Contr., 2004. Vol.~49. No.\,1. P.~45--56. doi: 10.1109/TAC.2003.821400.
\bibitem{4-bos}
\Au{Bohacek S.} A~stochastic model of TCP and fair video transmission~// IEEE 
INFOCOM, 2003. Vol.~2. P.~1134--1144. doi: 10.1109/INFCOM.2003.1208950.
\bibitem{5-bos}
\Au{Домбровский В.\,В., Объедко~Т.\,Ю.} Управление с~прогнозированием системами 
с~марковскими скачками при ограничениях и~применение к~оптимизации 
инвестиционного портфеля~// Автомат. телемех., 2011. №\,5. С.~96--112. doi: 
10.1134/S0005117911050079.
\bibitem{6-bos}
\Au{Баландин Д.\,В., Коган~М.\,М.} Оптимальное линейно-квад\-ра\-тич\-ное управление: от 
матричных уравнений к~линейным матричным неравенствам~// Автомат. телемех., 2011. 
№\,11. С.~60--69. doi: 10.1134/ S0005117911110038.
\bibitem{7-bos}
\Au{Босов А.\,В.} Обобщенная задача распределения ресурсов программной системы~// 
Информатика и~её применения, 2014. Т.~8. Вып.~2. С.~39--47. doi: 
10.14357/19922264140204.
\bibitem{8-bos}
\Au{Босов А.\,В.} Управление линейным выходом дискретной стохастической системы по 
квадратичному критерию~// Изв. РАН. Теория и~системы управления, 2016. №\,3.  
С.~19--35. doi: 10.1134/S1064230716030060.
\bibitem{9-bos}
\Au{Флеминг У., Ришел~Р.} Оптимальное управление детерминированными 
и~стохастическими системами~/ Пер. с~англ.~--- М.: Мир, 1978. 316~с. 
(\Au{Fleming~W.\,H., Rishel~R.\,W.} Deterministic and stochastic optimal control.~--- New 
York, NY, USA: Springer-Verlag, 1975. 222~p.)
\bibitem{10-bos}
\Au{Девис М.\,Х.\,А.} Линейное оценивание и~стохастическое управление~/ Пер. с~англ.~--- 
М.: Наука, 1984. 206~с. (\Au{Davis~M.\,H.\,A.} Linear estimation and stochastic control.~--- 
London: Chapman and Hall, 1977. 224~p.)

 \end{thebibliography}

 }
 }

\end{multicols}

\vspace*{-6pt}

\hfill{\small\textit{Поступила в~редакцию 30.03.18}}

\vspace*{4pt}

%\newpage

%\vspace*{-24pt}

\hrule

\vspace*{2pt}

\hrule

\vspace*{-2pt}


\def\tit{STOCHASTIC DIFFERENTIAL SYSTEM OUTPUT CONTROL 
BY~THE~QUADRATIC CRITERION.~I.~DYNAMIC\\ PROGRAMMING 
OPTIMAL SOLUTION}


\def\titkol{Stochastic differential system output control 
by~the~quadratic criterion. I.~Dynamic programming 
optimal solution}

\def\aut{A.\,V.~Bosov and~A.\,I.~Stefanovich}

\def\autkol{A.\,V.~Bosov and~A.\,I.~Stefanovich}

\titel{\tit}{\aut}{\autkol}{\titkol}

\vspace*{-11pt}


\noindent
Institute of Informatics Problems, Federal Research Center ``Computer Science 
and Control'' of the Russian Academy of Sciences, 44-2~Vavilov Str., Moscow 
119333, Russian Federation


\def\leftfootline{\small{\textbf{\thepage}
\hfill INFORMATIKA I EE PRIMENENIYA~--- INFORMATICS AND
APPLICATIONS\ \ \ 2018\ \ \ volume~12\ \ \ issue\ 3}
}%
 \def\rightfootline{\small{INFORMATIKA I EE PRIMENENIYA~---
INFORMATICS AND APPLICATIONS\ \ \ 2018\ \ \ volume~12\ \ \ issue\ 3
\hfill \textbf{\thepage}}}

\vspace*{3pt}



\Abste{The problem of optimal control for the Ito diffusion 
process and a~controlled linear output is solved. The considered 
statement is close to the classical linear-quadratic Gaussian 
control  (LQG control) problem. Differences consist in the fact 
that the state is described by the nonlinear differential Ito equation  $dy_y = A_t(y_t) 
\,dt+\Sigma_t(y_t)\,dv_t$ and does not depend on the control~$u_t$, 
optimization subject is controlled linear output 
 $dz_t=a_ty_t\,dt +b_tz_t\,dt +c_t u_t\,dt +\sigma_t \,dw_t$. 
Additional generalizations are included in the quadratic 
quality criterion for the purpose of statement such problems 
as state tracking by output or a linear combination of state 
and output tracking by control. The method of dynamic programming 
is used for the solution. 
The assumption about Bellman function in the form  $V_t(y,z)= \alpha_t 
z^2+\beta_t(y) z+\gamma_t(y)$ allows one to find it. 
Three differential equations for the coefficients $\alpha_t$,  $\beta_t(y)$,
and $\gamma_t(y)$ give the solution. 
These equations constitute the optimal solution of the problem under consideration.}

\KWE{stochastic differential equation; optimal control; dynamic programming; 
Bellman function; Riccati equation; linear differential equations of parabolic type}


\DOI{10.14357/19922264180314}

\vspace*{-12pt}

\Ack
\noindent
This work was partially supported by the Russian Science Foundation (grant  
16-07-00677).



%\vspace*{6pt}

  \begin{multicols}{2}

\renewcommand{\bibname}{\protect\rmfamily References}
%\renewcommand{\bibname}{\large\protect\rm References}

{\small\frenchspacing
 {%\baselineskip=10.8pt
 \addcontentsline{toc}{section}{References}
 \begin{thebibliography}{99}
\bibitem{1-bos-1}
\Aue{Athans, M.} 1971. Editorial on the LQG problem. \textit{IEEE~T. 
Automat. Contr.} 16(6):528--552. doi: 10.1109/ TAC.1971.1099845.
\bibitem{2-bos-1}
\Aue{Wu, Z.} 2005. Forward-backward stochastic differential equations, linear 
quadratic stochastic optimal control and\linebreak\vspace*{-12pt}

\columnbreak

\noindent
 nonzero sum differential games. 
\textit{J.~Syst. Sci. Complex.} 18(2):179--192.
\bibitem{3-bos-1}
\Aue{Chen, B.\,S. and W.~Zhang.} 2004. Stochastic H2/H1 control with  
state-dependent noise. \textit{IEEE~T. Automat. Contr.} 49(1):45--56.
doi: 10.1109/TAC.2003.821400.
\bibitem{4-bos-1}
\Aue{Bohacek, S.} 2003. A~stochastic model of TCP and fair video 
transmission. \textit{IEEE INFOCOM}. 2:1134--1144.
doi: 10.1109/INFCOM.2003.1208950.
\bibitem{5-bos-1}
\Aue{Dombrovskii, V.\,V., and T.\,Yu.~Ob''edko.} 2011. Predictive control of 
systems with Markovian jumps under constraints and its application to the 
investment portfolio optimization. \textit{Automat. Rem. Contr.}  
72(5):989--1003.
\bibitem{6-bos-1}
\Aue{Balandin, D.\,V., and M.\,M.~Kogan.} 2011. Optimal linear-quadratic 
control: From matrix equations to linear matrix inequalities. \textit{Automat. 
Rem. Contr.} 72(11):2276--2284.
\bibitem{7-bos-1}
\Aue{Bosov, A.\,V.} 2014. Obobshchennaya zadacha raspredeleniya resursov 
programmnoy sistemy [The generalized problem of software system resources 
distribution]. \textit{Informatika i~ee Primeneniya~--- Inform. Appl.}  
8(2):39--47. doi: 
10.14357/19922264140204.
\bibitem{8-bos-1}
\Aue{Bosov, A.\,V.} 2016. Discrete stochastic system linear output control 
with respect to a quadratic criterion. \textit{J.~Comput. Syst. Sc. 
Int.} 55(3):349--364.
\bibitem{9-bos-1}
\Aue{Fleming, W.\,H., and R.\,W.~Rishel.} 1975. \textit{Deterministic and 
stochastic optimal control.} New York, NY: Springer-Verlag. 222~p.
\bibitem{10-bos-1}
\Aue{Davis, M.\,H.\,A.} 1977. \textit{Linear estimation and stochastic 
control.} London: Chapman and Hall. 224~p.
\end{thebibliography}

 }
 }

\end{multicols}

\vspace*{-6pt}

\hfill{\small\textit{Received March 30, 2018}}

%\pagebreak

%\vspace*{-18pt}
     
     \Contr
     
       \noindent
       \textbf{Bosov Alexey V.} (b.\ 1969)~--- Doctor of Science in technology, 
principal scientist, Institute of Informatics Problems, Federal Research 
Center ``Computer Science and Control'' of the Russian Academy of Sciences, 
44-2~Vavilov Str., Moscow 119333, Russian Federation; 
\mbox{AVBosov@ipiran.ru}
       
       \vspace*{3pt}
       
       \noindent
       \textbf{Stefanovich Alexey I.} (b.\ 1983)~--- principal specialist, 
Institute of Informatics Problems, Federal Research Center ``Computer Science 
and Control'' of the Russian Academy of Sciences, 44-2~Vavilov Str., Moscow 
119333, Russian Federation; \mbox{AStefanovich@frccsc.ru}
\label{end\stat}

\renewcommand{\bibname}{\protect\rm Литература}       

               %6
\def\stat{krivenko}

\def\tit{МНОГОМЕРНЫЙ РЕФЕРЕНСНЫЙ РЕГИОН\\ ВЫСОКОЙ ПЛОТНОСТИ}

\def\titkol{Многомерный референсный регион высокой плотности}

\def\aut{М.\,П.~Кривенко$^1$}

\def\autkol{М.\,П.~Кривенко}

\titel{\tit}{\aut}{\autkol}{\titkol}

\index{Кривенко М.\,П.}
\index{Krivenko M.\,P.}


%{\renewcommand{\thefootnote}{\fnsymbol{footnote}} \footnotetext[1]
%{Работа выполнена при финансовой поддержке РФФИ (проекты 16-07-00677 
%и~15-37-20611-мол\_а\_вед).}}


\renewcommand{\thefootnote}{\arabic{footnote}}
\footnotetext[1]{Институт проблем информатики Федерального исследовательского центра <<Информатика и~управление>> Российской академии наук,
\mbox{mkrivenko@ipiran.ru}}

\vspace*{4pt}



\Abst{Рассматриваются принципы построения многомерных референсных регионов
(MRR~--- multivariate reference region). 
Предложен оригинальный метод построения региона на основе областей с~высокой 
плотностью точек и~аппроксимации распределения данных с~помощью смеси нормальных 
распределений. Для оценки порога для плотности распределения используется  
бут\-стреп-ме\-тод. В~качестве эксперимента рассмотрена задача построения 
и~использования эталонной области для прогнозирования типа мочевого камня. Обработка 
реальных данных продемонстрировала преимущества предлагаемых решений.}

\KW{многомерный референсный регион; область высокой плотности; бут\-стреп-ме\-тод; 
смесь многомерных нормальных распределений}

\vspace*{6pt}

\DOI{10.14357/19922264170207} 


\vskip 10pt plus 9pt minus 6pt

\thispagestyle{headings}

\begin{multicols}{2}

\label{st\stat}

\section{Введение}

     Многомерный референсный регион 
был предложен в~литературе по клинической химии в~начале 1970-х~гг.\ как 
альтернатива одномерным референсным интервалам~[1]. Там излагались 
преимущества предлагаемых множественных тестов, хоть и~имеющих 
упрощенный вид, но снижающих (по отношению к~одномерным вариантам) 
число ложных положительных результатов. Появление MRR оказалось 
особенно привлекательным для интерпретации результатов наборов 
медицинских тестов. Тем не менее возникали трудности в~построении 
и~использовании процедур многомерного анализа (см., например,~[2]), 
связанные, в~частности, с~быстрым увеличением числа параметров, которые 
должны быть оценены. Немногие лаборатории использовали MRR в~своей 
практике, причем в~экспериментальном режиме, и,~как следствие, на 
сегодняшний день имеется относительно малое количество соответствующих 
публикаций. 

\vspace*{-6pt}

\section{Многомерный референсный регион на основе расстояния Махалонобиса}

\vspace*{-2pt}

     Одномерный референсный интервал, полученный статистическим путем, 
использует центральную часть значений анализируемого показателя, обычно 
соответствующую~95\% некоторой популяции~--- совокупности особей 
определенного вида (например, здоровой части населения определенного пола 
из некоторого диапазона возрастов). Одномерные референсные интервалы 
применялись в~течение многих лет в~качестве стандартного приема 
интерпретации лабораторных данных. Они легко формируются, хранятся, 
извлекаются и~передаются в~лабораторных информационных системах, просты 
в~понимании, хорошо воспринимаются медицинским сообществом в~ходе 
длительного использования. Тем не менее одномерные референсные интервалы 
при классификации данных могут дать большое число ложно аномальных 
результатов. Этот далеко не единственный недостаток однофакторного 
референсного интервала может быть полностью или частично устранен 
с~помощью MRR.
     
     Простейшим и~весьма распространенным способом построения MRR 
является использование прямого произведения отдельных референсных 
интервалов в~предположении, что они статистически независимы. Пусть 
$(1\hm-\alpha)$~--- вероятность попадания в~MRR, а~$p_0$~--- вероятность 
попадания в~референсный интервал для любого из~$d$~признаков, тогда 
$p_0\hm= \sqrt[d]{1-\alpha}$. С~ростом размерности~$d$ значения~$p_0$ 
быстро приближаются к~1, что фактически лишает смысла применение MRR.
     
     Как и~в одномерном случае, отправной точкой для построения MRR 
может стать нормальное распределение. Идеи центрального расположения 
референсного региона и~заданной вероятности попадания в~него приводят для 
$d$-мер\-но\-го нормального распределения, имеющего плотность 
распределения
     \begin{multline*}
     \varphi(y,\mu,\Sigma) ={}\\
     {}=(2\pi)^{-d/2}\vert\Sigma\vert^{-1/2}\exp \left( -\fr{\left(y-
\mu\right)^{\mathrm{T}} \Sigma^{-1}(y-\mu)}{2}\right),
   \end{multline*}
где величина $(y-\mu)^{\mathrm{T}} \Sigma^{-1} (y-\mu)$ есть квадрат так 
называемого расстояния Махаланобиса между~$y$ и~$\mu$, к~использованию 
многомерного эллипсоида
\begin{multline*}
(2\pi)^{-d/2}\vert\Sigma\vert^{-1/2}\exp \left( -\fr{\left(y-\mu\right)^{\mathrm{T}}
\Sigma^{-1} 
(y-\mu)}{2}\right) ={}\\
{}=const
\end{multline*}
или, что то же самое, 
$$ 
(y-\mu)^{\mathrm{T}} \Sigma^{-1}(y-\mu)=const\,.
$$
Его называют эллипсоидом равной плотности распределения (или просто 
эллипсоидом равной вероятности). 
     
     Если задаться вероятностью $(1\hm-\alpha)$ попадания в~эллипсоид 
равной вероятности вида $(y\hm-\mu)^{\mathrm{T}}\Sigma^{-1} (y\hm-\mu)\hm= 
\rho$, то параметр~$\rho$ удовлетворяет уравнению $\mathrm{Pr}\left\{ 
\chi_d^2\leq \rho\right\} \hm=1\hm-\alpha$.
     
     Использование эллипсоида в~качестве MRR будет оправдано только 
тогда, когда исходное распределение данных есть многомерное нормаль-\linebreak ное. 
Поэтому становятся актуальными критерии\linebreak подгонки, а~также использование 
процедур норма\-ли\-зации распределения данных в~многомерном\linebreak случае.
 Если 
с~помощью тестов выявляется, что распределение не является нормальным, то 
Международная федерация клинической химии и~лабораторной медицины 
рекомендует, согласно~[3], использовать двухступенчатую процедуру 
нормализации. Следует обратить внимание, что многошаговость здесь 
относится не к~многомерности, а касается лишь покоординатного 
преобразования распределения данных к~нормальному.
     
     Первые же попытки применения MRR на основе расстояния 
Махалонобиса (фактически это означает принятие модели нормального 
распределения референсных значений) выявили ряд недостатков (более 
подробно смотри в~\cite[разд.~6.2]{4-kri}):
     \begin{itemize}
\item проявление <<проклятий>> размерности при механическом 
увеличении~$d$, в~особенности если игнорируется этап анализа состава 
признаков~[1, 5, 6];
\item из-за небольших объемов обучающей выборки невысокая устойчивость 
при применении, в~частности чувствительность к~увеличению неточностей 
измерений после того, как регион был установлен~\cite{5-kri, 7-kri}. 
\item предположение о нормальном распределении и~попытки <<подправить>> 
действительность с~помощью преобразований реальных данных для их 
нормализации при увеличении размерности данных становятся все более 
шаткими~\cite{5-kri};
\item представление и~интерпретация выводов на основе MRR трудно 
понимаемы не только для специалистов в~предметной области~[8].
\end{itemize}

\vspace*{-9pt}

\section{Многомерный референсный регион высокой плотности}

\vspace*{-2pt}

     Заметим, что в~случае нормального распределения референсных значений 
для точек внут\-ри построенного эллипсоида значения плотности\linebreak распределения 
больше, чем на границе, а~вне~--- меньше. Это замечание позволяет 
предложить другой подход к~построению MRR.
     
     \smallskip
     
     \noindent
     \textbf{Определение.}\ Eсли плотность распределения референсных 
значений есть $f(y)$, то MRR есть область $A_t\hm= \left\{ y\in 
\mathcal{R}^d\vert f(y)\hm\geq t\right\}$ для некоторого порогового 
значения~$t$. 
     
     \smallskip
     
     Для нормального распределения это уже упомянутый эллипсоид равной 
вероятности. Если задается вероятность $(1\hm-\alpha)$ попадания в~$A_t$, то 
пороговое значение~$t$ есть решение уравнения $\int\nolimits_{A_t} 
f(u)\,du\hm=1\hm-\alpha$, получить которое аналитически в~случае 
произвольной плотности распределения вряд ли возможно. Здесь присутствуют 
две проблемы: вычисление многомерного интеграла и~зависимость области 
интегрирования от неизвестного значения. Для решения их предлагается 
привлечь метод моделирования.
     
     Сгенерируем выборку из $f(y)$, которую обозначим как $Y^f\hm= \left\{ 
y_1^f, \ldots, y_m^f\right\}$. Для оценки $\int\nolimits_{A_t} f(u)\,du$ 
используем отношение:

\noindent
\begin{multline*}
     \fr{\left\vert \left\{ y_i^f\vert y_i^f\in A_t\right\}\right\vert }{m} =
      \fr{\left\vert\left\{ y_i^f\vert 
f\left(y_i^f\right) \geq t\right\}\right\vert }{m} ={}\\
{}= 1-\fr{\left\vert \left\{ y_i^f\vert f(y_i^f)<t\right\}\right\vert }{m}=1-
F_m(t)\,,
     \end{multline*}
где $F_m(t)$~--- эмпирическая функция распределения случайной 
величины~$f(y)$, т.\,е.\ случайной величины, являющейся результатом 
преобразования с~помощью функции~$f(\cdot)$ случайной величины, име\-ющей 
плотность распределения~$f(u)$.

     Таким образом, искомая оценка~$t^*$ должна удовле\-тво\-рять уравнению 
$F_m(t^*)\hm=\alpha$ и~может быть получена как непараметрическая оценка 
квантиля\linebreak\vspace*{-12pt}

\pagebreak

\noindent
 порядка~$\alpha$ из распределения $F_m(\cdot)$. Если обозначить 
$f_i\hm= f(y_i^f)$, то~$t^*$ есть~$f_{(r)}$, где
     $$
     r= \begin{cases}
     m\alpha, &\ m\alpha~\mbox{---~целое}\,;\\
     \lfloor m\alpha+1\rfloor\,, & m\alpha~\mbox{--- не целое}\,.
     \end{cases}
     $$
     Заметим, что для такой оценки можно указать доверительный интервал.
     
     Для построения MRR необходимо знать распределение данных. При 
реализации принципа точек высокой плотности в~первую очередь следует 
обратиться к~параметрическим моделям, в~част\-ности к~смеси нормальных 
распределений, име\-ющей плотность распределения
     $$
     f(u) =\sum\limits_{j=1}^k p_j \varphi\left (u,\mu_j, \Sigma_j\right)\,.
     $$
Если $\hat{f}(u)$~--- оценка смеси, то~$t^*$ строится сле\-ду\-ющим образом:
\begin{itemize}
\item генерируется выборка $\left\{ y_1^f,\ldots , y_m^f\right\}$ из $\hat{f}(u)$ и~
для каждого ее $i$-го элемента подсчитывается значение $\hat{f}\left( 
y_i^f\right)$;
\item в~качестве~$t^*$ берется непараметрическая оценка квантиля 
порядка~$\alpha$ (в случае необходимости дополнительно находится 
непараметрическая оценка доверительного интервала для~$t^*$, что 
может характеризовать правильность выбранного объема для 
генерируемой выборки).
\end{itemize}

     Пусть для $f(u)$ имеется~$A_t$, а также получена $\hat{f}(u)$ 
и~соответствующий MRR вида~$\hat{A}_t$. Качество аппроксимации~$A_t$ 
с~по\-мощью~$\hat{A}_t$ можно оценить с~по\-мощью вероятности совпадения 
этих областей, т.\,е. 
     $$
     P_c= \int\limits_{\{ u\in A_t\}\cup \{u\in \hat{A}_t\}} \hspace*{-6mm}
f(u)\,du+\int\limits_{\{u\not\in A_t\} \cup\{ u\not\in \hat{A}_t\}}\hspace*{-6mm} f(u)\,du\,.
     $$
     
     Для оценки  $P_c$ можно использовать величину
     \begin{multline*}
     \hat{P}_c= \fr{\left\vert \left\{ 
     y_i^f\vert y_i^f \in \left\{\left\{ y_i^f\in A_t\right\}\cup \left\{y_i^f\in 
\hat{A}_t\right\}\right\}\right\}\right\vert}{m}+{}\\
{}+\fr{\left\vert \left\{ y_i^f\vert y_i^f \in \left\{\left\{ y_i^f\not\in A_t\right\}\cup 
\left\{ y_i^f\not\in \hat{A}_t\right\}\right\}\right\}\right\vert}{m}\,.
     \end{multline*}
     
     Использование MRR высокой плотности для диагностирования сводится 
к~реализации так называемого слабого критерия значимости для наблюденного 
значения~$x$: нулевая гипотеза заключается в~том, что $x\hm\in A_t$, 
статистика критерия есть $\hat{f}(x)$ и~решение о~принадлежности 
критической об\-ласти~$A_t$ принимается при больших значениях~$\hat{f}(x)$.
     
     Для медицинской практики важна возможность использования 
референсного региона при интерпретации результатов обследования 
некоторого пациента с~вектором признаков~$x$. В~подобных случаях 
сложившейся практикой для слабых критериев значимости является 
использование критического уровня~$\alpha_{\mathrm{cr}}$ (более распространенным 
в~медицине является употребление термина $p$-зна\-че\-ние)  $\alpha_{\mathrm{cr}}\hm= 
\mathrm{Pr}\left\{ \hat{f}(y)\hm\leq \hat{f}(x)\right\}$, где $y$~--- случайная 
величина, имеющая плотность распределения~$\hat{f}(u)$, а $\hat{f}(x)$~--- 
значение плотности распределения~$\hat{f}(u)$ в~точке~$x$. Эта 
характеристика дает представление о~том, насколько сильно данное 
наблюденное значение~$x$ противоречит гипотезе (или подкрепляет ее) 
о~принадлежности данных MRR. При выбранном же заранее уровне 
значимости с~помощью~$\alpha_{\mathrm{cr}}$ сразу же можно принять конкретное 
решение. 

\vspace*{-9pt}

\section{Эксперименты}

\vspace*{-2pt}

     Для демонстрации возможностей MRR использовались данные по 
прогнозу химического состава мочевых камней по метаболическим 
показателям мочи и~сыворотки крови, а также антропологическим 
характеристикам пациентов~[9]. В качестве исходной классификации камней 
рассматривалась следующая: чисто оксалатные (далее обозначены как O), чисто 
уратные (U), чисто фосфатные (P), смесь только оксалатных и~уратных (OU), 
смесь только оксалатных и~фосфатных (OP), смесь только уратных 
и~фосфатных (UP), все остальные. Данная классификация была построена 
в~[10] на основе доминирующих частот встречаемости основных компонентов. 
В~качестве референсных значений рассматривались наборы метаболических 
и~антропологических показателей (их всего было~14), соответствующих 
определенному классу камней.

\begin{table*}\small
\begin{center}


\begin{tabular}{|c|c|c|c|c|c|c|}
\multicolumn{7}{c}{Качество классификации с~помощью MRR}\\
\multicolumn{7}{c}{\ }\\[-6pt]
\hline
\multicolumn{1}{|c|}{\raisebox{-6pt}[0pt][0pt]{\tabcolsep=0pt\begin{tabular}{c}Тип\\ камня\end{tabular}}}&
\multicolumn{1}{c|}{\raisebox{-6pt}[0pt][0pt]{$N$}}&$(1-\alpha)$, 
&\multicolumn{2}{c|}{MRR(5)}&\multicolumn{2}{c|}{MRR(1)}\\
\cline{4-7}
&&&&&&\\[-9pt]
&&\%&$(1-\hat{\alpha})$, \%&$\hat{\beta}$, \%&$(1-\hat{\alpha})$, \%&$\hat{\beta}$, \%\\
\hline
\multicolumn{1}{|c|}{\raisebox{-18pt}[0pt][0pt]{O}}&
\multicolumn{1}{c|}{\raisebox{-18pt}[0pt][0pt]{82}}
&95&100\hphantom{9}&71&90&24\\
&&85&96&78&89&36\\
&&75&91&85&77&44\\
&&65&76&88&74&50\\
\hline
\multicolumn{1}{|c|}{\raisebox{-18pt}[0pt][0pt]{U}}&
\multicolumn{1}{c|}{\raisebox{-18pt}[0pt][0pt]{76}}&95&100\hphantom{9}&75&91&24\\
&&85&99&85&80&35\\
&&75&82&89&74&48\\
&&65&71&91&68&56\\
\hline
\multicolumn{1}{|c|}{\raisebox{-18pt}[0pt][0pt]{P}}&
\multicolumn{1}{c|}{\raisebox{-18pt}[0pt][0pt]{83}}&95&100\hphantom{9}&66&87&25\\
&&85&94&78&86&33\\
&&75&86&82&82&41\\
&&65&77&87&75&47\\
\hline
\end{tabular}
\end{center}
\end{table*}
     
     
     Для каждого из основных классов O, U, P, OU, OP и~UP перед построением 
MRR проводилась селекция признаков и~принималось то значение размерности 
признакового пространства~$d$ и~соответствующий набор показателей, 
которые позволяли прогнозировать состав камней без потери качества 
(методика описана в~\cite{9-kri} и~привела к~значению $d\hm=9$). В~качестве 
модели данных в~первую очередь рассматривалась смесь многомерных 
нормальных распределений из пяти элементов (подбор числа элементов смеси 
проводился с~по\-мощью AIC~--- Akaike information criterion), для соответствующего региона было принято 
обозначение MRR(5). Для сравнения также использовалась модель 
нормального распределения, которой соответствовал MRR(1). Полученные 
результаты приводятся час\-тич\-но в~таблице, где $N$~--- объем 
классифицируемых данных; $\hat{\alpha}$~--- оценка для~$\alpha$; 
$\hat{\beta}$~--- оценка мощности критерия при определении типа камня на 
основании MRR.


     Одной из базовых характеристик является вероятность попадания в~MRR 
$(1\hm-\alpha)$ и~ее оценка $(1\hm-\hat{\alpha})$. Сравнение соответствующих 
столбцов с~учетом значений~$N$ и~ориентировочных значений разброса 
(стандартные отклонения на основе биномиального распределения) не 
позволило выявить явных отклонений. Необходимо, правда, отметить, что во 
всех проанализированных случаях для MRR(5) оказалось, что $1\hm-
\hat{\alpha}\hm\geq 1\hm-\alpha$.
     
     Назначение MRR, заключающееся в~сжатом представлении референсных 
значений, в~многомерном случае практически не проявляется. Для задания 
MRR(5) необходимо указать следующие величины: $1\hm-\alpha$, $t$, 
$p_1,\ldots, p_{k-1}$, $\mu_1, \Sigma_1,\ldots , \mu_k,\Sigma_k$, общее 
количество которых равно  $[2\hm+ (k\hm-1)\hm+ k(d\hm+ d(d\hm+1)/2)]$ 
и,~в~частности, в~рассматриваемых экспериментах~--- 276. Для MRR(1) это 
значение меньше и~равно~56. При этом для обрабатываемой обучающей 
выборки в~зависимости от класса камней речь идет о~порядка~10$^2$ векторах 
данных (см.\ столбец со значениями~$N$), что приблизительно 
дает~10$^3$~скалярных величин.
     
     Другое назначение MRR состоит в~его использовании для 
диагностирования (классификации). В~этой связи в~первую очередь 
проводился сравнительный анализ MRR(1) (фактически это означает, что 
построение региона осуществляется на основе расстояния Махаланобиса) 
и~MRR(5) (модель смеси нормальных распределений и~предложенный 
в~данной работе метод оценивания па\-ра\-мет\-ров региона). Показателем 
информативности метода построения многомерного региона выступала 
мощность соответствующего слабого критерия значимости, а~именно: 
вероятность не попасть в~MRR при условии, что данные берутся из дополнения 
к~классу, для которого построена MRR. Сравнение соответствующих столбцов 
говорит о~явном преимуществе двух предложенных моментов: усложнение 
модели данных путем перехода от нормального распределения к~смеси 
нормальных распределений и~построение региона высокой плотности.
     
     Использование критического уровня можно продемонстрировать  
с~по\-мощью зависимости результатов сравнения двух классов от того, какой 
класс взять за основу. Введем для возможных значений $p$-ве\-ли\-чи\-ны три 
интервала: $(-\infty, 1\%)$, $[1\%, 5\%)$, $[5\%, 100\%)$ с~соответствующей 
интерпретацией положения наблюденного набора показателей для пациента 
относительно построенного MRR: уверенное непопадание, неуверенное 
попадание, уверенное попадание. Если MRR построить для оксалатных камней, 
то результаты для анализа пациентов с~фосфатными камнями дадут следующий 
вектор относительных частот попадания $p$-ве\-ли\-чин в~указанные 
интервалы: $(60\%, 18\%, 22\%)$. Если же MRR строить для фосфатных 
камней, то получим $(71\%, 5\%, 24\%)$. Таким образом, для классификации 
указанных камней при приблизительно одинаковых частотах попадания в~MRR 
(22\% или~24\%) уверенный отказ от референсного региона происходит чаще, 
если принять за базовый MRR регион для фосфатных камней. Построение 
шкалы, подобной рассмотренной, является прерогативой специалистов 
в~предметной области, в~данной работе она использовалась только для 
иллюстрации. 

\vspace*{-6pt}

\section{Заключение}

\vspace*{-2pt}

     На настоящий момент имеется относительно мало примеров применения 
MRR в~клинической практике. Тому есть несколько причин. Математическое 
обеспечение, необходимое для получения и~применения MRR, не отвечает 
возможностям большинства клинических лабораторий. Лаборатории слабо 
оснащены программными средствами\linebreak для реализации достаточно сложного 
математического аппарата многомерного анализа, а~еще важнее, что 
отсутствуют методики, инструкции по\linebreak использованию соответствующих 
средств. Лишь немногие клинические применения демонстрируют 
преимущества MRR, хотя свидетельств неудачных попыток больше.
     
     Несмотря на сложности внедрения мно\-го\-мерно\-го анализа референсных 
значений, можно сформулировать некоторые рекомендации по иссле\-до\-ва\-нию 
и~разработке MRR. Во-пер\-вых, эффективная размерность в~MRR должна 
быть как можно меньше, чтобы избежать затенения диагностически полезной 
информации тестами, со\-зда\-ющи\-ми шум. Низкая размерность также должна 
уменьшить неблагоприятные последствия увеличения неточности результатов 
в~связи с~ростом числа анализируемых показателей. Во-вто\-рых, показатели 
(тес\-ты), включенные в~MRR, должны быть физиологически релевантными 
исследуемому кругу расстройств, чтобы максимизировать информацию, 
полученную от MRR. В-треть\-их, чтобы учесть эффекты долгосрочной 
лабораторной из\-мен\-чи\-вости, данные, используемые для получения MRR, 
долж\-ны быть собраны и~проанализированы в~течение достаточно большого 
периода времени (от нескольких недель до нескольких месяцев).  
В-чет\-вер\-тых, представление результатов лабораторных исследований 
следует осуществлять в~графическом виде, чтобы помочь врачам лучше понять 
MRR. Различные подходы к~уменьшению размерности помогут выполнить это 
требование.
     
     Необходима дальнейшая разработка пояснительных инструментов, 
способных воспринять результаты анализа MRR. При этом дополнительно 
необходима информация о~том, какие именно тес\-ты являются важнейшими 
факторами нарушения нормы. Надо признать, что соответствующий 
математический аппарат еще предстоит разработать. Решение перечисленных 
вопросов играет важную роль для обеспечения постоянного клинического 
применения MRR. 

\vspace*{-6pt}
     
{\small\frenchspacing
 {%\baselineskip=10.8pt
 \addcontentsline{toc}{section}{References}
 \begin{thebibliography}{99}
 
 \vspace*{-2pt}
 
\bibitem{1-kri}
\Au{Boyd J.\,C.} Reference regions of two or more dimensions~// Clin. Chem. Lab. 
Med., 2004. Vol.~42. No.\,7. P.~739--746.
\bibitem{2-kri}
\Au{Winkel P.} Patterns and clusters~--- multivariate approach for interpreting 
clinical chemistry results~// Clin. Chem., 1973. Vol.~19. No.\,12. P.~1329--1333.
\bibitem{3-kri}
IFCC. Expert panel on theory of reference values. Approved recommendation on the 
theory of reference values. Part~5. Statistical treatment of collected reference values. 
Determination of reference limits~// J.~Clin. Chem. Clin. Biochem., 1987. Vol.~25. 
No.\,9. P.~645--656.
\bibitem{4-kri}
\Au{Кривенко М.\,П.} Статистические методы представления и~предварительной 
обработки референсных значений.~--- М.: ФИЦ ИУ РАН, 2016. 160~с.
\bibitem{5-kri}
\Au{Boyd J.\,C., Lacher~D.\,A.} The multivariate reference range: An alternative 
interpretation of multi-test profiles~// Clin. Chem., 1982. Vol.~28. No.\,2.  
P.~259--265.
\bibitem{6-kri}
\Au{Albert A., Harris~E.\,K.} Multivariate interpretation of clinical laboratory  
data.~--- New York, NY, USA: CRC Press, 1987. 328~p.
\bibitem{7-kri}
\Au{Linnet K.} Influence of sampling variation and analytical errors on the 
performance of the multivariate reference region~// Meth. Inf. Med., 1988. Vol.~27. 
No.\,1. P.~37--42.
\bibitem{8-kri}
\Au{Durbridge T.\,C.} Clinical acceptance of a multi-test reference region for 
biochemical-panel results~// Clin. Chem., 1983. Vol.~29. No.\,10. P.~1724--1726.
\bibitem{9-kri}
\Au{Кривенко М.\,П.} Критерии значимости отбора признаков классификации~// 
Информатика и~её применения, 2016. Т.~10. Вып.~3. С.~32--40.
\bibitem{10-kri}
\Au{Кривенко М.\,П., Голованов~С.\,А., Сивков~А.\,В.} Анализ однородности 
данных о химическом составе камней при уролитиазе~// Информатика и~её 
применения, 2013. Т.~7. Вып.~4. С.~94--104.
 \end{thebibliography}

 }
 }

\end{multicols}

\vspace*{-10pt}

\hfill{\small\textit{Поступила в~редакцию 5.12.16}}

\vspace*{4pt}

%\newpage

%\vspace*{-24pt}

\hrule

\vspace*{2pt}

\hrule

\vspace*{-3pt}


\def\tit{HIGH-DENSITY MULTIVARIATE REFERENCE REGION\\[-5pt]}

\def\titkol{High-density multivariate reference region}

\def\aut{M.\,P.~Krivenko\\[-7pt]}

\def\autkol{M.\,P.~Krivenko}

\titel{\tit}{\aut}{\autkol}{\titkol}

\vspace*{-16pt}


\noindent
Institute of Informatics Problems, Federal Research Center 
``Computer Science and Control'' of the Russian
Academy of Sciences,  44-2~Vavilov Str., Moscow 119333, Russian Federation



\def\leftfootline{\small{\textbf{\thepage}
\hfill INFORMATIKA I EE PRIMENENIYA~--- INFORMATICS AND
APPLICATIONS\ \ \ 2017\ \ \ volume~11\ \ \ issue\ 2}
}%
 \def\rightfootline{\small{INFORMATIKA I EE PRIMENENIYA~---
INFORMATICS AND APPLICATIONS\ \ \ 2017\ \ \ volume~11\ \ \ issue\ 2
\hfill \textbf{\thepage}}}

\vspace*{2pt}




\Abste{The paper considers the principles of construction of multivariate 
reference regions. An original method of construction of 
a~region on the basis of areas of high density of points and approximation 
of data distribution with a~mixture of normal distributions is suggested. 
To estimate the threshold for the probability density, the bootstrap method is used. 
As an experiment, the paper considers the problem of description and use of 
the reference region for predicting the type of urinary stones. 
Real data treatment demonstrated the benefits of the proposed solutions.}

\KWE{multivariate reference region; high-density region; bootstrap method; 
multivariate normal mixture}

\DOI{10.14357/19922264170207} 

%\vspace*{-18pt}

%\Ack
%\noindent



%\vspace*{3pt}

  \begin{multicols}{2}

\renewcommand{\bibname}{\protect\rmfamily References}
%\renewcommand{\bibname}{\large\protect\rm References}

{\small\frenchspacing
 {%\baselineskip=10.8pt
 \addcontentsline{toc}{section}{References}
 \begin{thebibliography}{99}
\bibitem{1-kri-1}
\Aue{Boyd, J.\,C.} 2004. Reference regions of two or more dimensions. \textit{Clin. 
Chem. Lab. Med.} 42(7):739--746.

\bibitem{2-kri-1}
\Aue{Winkel, P.} 1973. Patterns and clusters~--- multivariate approach for interpreting 
clinical chemistry results. \textit{Clin. Chem.} 19(12):1329--1333.
\bibitem{3-kri-1}
IFCC. 1987. Expert panel on theory of reference values. Approved recommendation on the 
theory of reference values. Part~5. Statistical treatment of collected reference values. 
Determination of reference limits. \textit{J.~Clin. Chem. Clin. Biochem.} 
25(9):645--656.
\bibitem{4-kri-1}
\Aue{Krivenko, M.\,P.} 2016. \textit{Statisticheskie metody predstavleniya 
i~predvaritel'noy obrabotki referensnykh znacheniy}
[Statistical methods for representation and preliminary processing of
reference values]. Moscow: FRC CSC RAS. 160~p.

\bibitem{5-kri-1}
\Aue{Boyd, J.\,C., and D.\,A.~Lacher.} 1982. The multivariate reference range: An 
alternative interpretation of multi-test profiles. \textit{Clin. Chem.}  
28(2):259--265.
\bibitem{6-kri-1}
\Aue{Albert, A., and E.\,K.~Harris.} 1987. \textit{Multivariate interpretation of 
clinical laboratory data}. New York, NY: CRC Press. 328~p.
\bibitem{7-kri-1}
\Aue{Linnet, K.} 1988. Influence of sampling variation and analytical errors on the 
performance of the multivariate reference region. \textit{Meth. Inf. Med.}  
27(1):37--42.
\bibitem{8-kri-1}
\Aue{Durbridge, T.\,C.} 1983. Clinical acceptance of a multi-test reference region 
for biochemical-panel results. \textit{Clin. Chem.} 29(10):1724--1726.
\bibitem{9-kri-1}
\Aue{Krivenko, M.\,P.} 2016. Kriterii znachimosti otbora priznakov klassifikatsii
[Significance tests of feature selection for~classification]. \textit{Informatika i~ee 
Primeneniya~--- Inform. Appl.} 10(3):32--40.
\bibitem{10-kri-1}
\Aue{Krivenko, M.\,P., S.\,A.~Golovanov, and A.\,V.~Sivkov}. 2013. Analiz 
odnorodnosti dannykh o~khimicheskom sostave kamney pri urolitiaze
[Analysis of data homogeneity of~the~chemical compositions 
of~stones in~case of~urolithiasis]. \textit{Informatika i~ee Primeneniya~---
Inform Appl.} 7(4):94--104.
\end{thebibliography}

 }
 }

\end{multicols}

\vspace*{-3pt}

\hfill{\small\textit{Received December 5, 2016}}


\Contrl

\noindent
\textbf{Krivenko Michail P.} (b.\ 1946)~--- Doctor of Science in technology, 
professor, leading scientist, Institute of Informatics Problems, Federal Research 
Center ``Computer Science and Control'' of the Russian Academy of Sciences, 
\mbox{44-2}~Vavilov Str., Moscow 119333, Russian Federation; \mbox{mkrivenko@ipiran.ru}

\label{end\stat}


\renewcommand{\bibname}{\protect\rm Литература}       %7
\def\stat{kondranin+ushakov}

\def\tit{СИСТЕМА ОБСЛУЖИВАНИЯ С~ОТНОСИТЕЛЬНЫМ ПРИОРИТЕТОМ  И~ПРОФИЛАКТИКАМИ ПРИБОРА$^*$}

\def\titkol{Система обслуживания с~относительным приоритетом  и~профилактиками прибора}

\def\aut{Е.\,С.~Кондранин$^1$,  В.\,Г.~Ушаков$^2$}

\def\autkol{Е.\,С.~Кондранин,  В.\,Г.~Ушаков}

\titel{\tit}{\aut}{\autkol}{\titkol}

\index{Кондранин Е.\,С.}
\index{Ушаков В.\,Г.}
\index{Kondranin E.\,S.}
\index{Ushakov V.\,G.}




{\renewcommand{\thefootnote}{\fnsymbol{footnote}} \footnotetext[1]
{Работа выполнена при финансовой поддержке РФФИ (проект 18-07-00678).}}


\renewcommand{\thefootnote}{\arabic{footnote}}
\footnotetext[1]{Факультет вычислительной математики и~кибернетики Московского государственного 
университета им.\ М.\,В.~Ломоносова, \mbox{ekondranin@yandex.ru}}
\footnotetext[2]{Факультет вычислительной математики и~кибернетики
Московского государственного университета им.\ М.\,В.~Ломоносова;
Институт проб\-лем информатики Федерального исследовательского
центра <<Информатика и~управ\-ле\-ние>> Российской академии наук,
\mbox{vgushakov@mail.ru}}

\vspace*{-10pt}




\Abst{Изучена одноканальная система
массового обслуживания с~двумя типами требований, бесконечным
числом мест для ожидания, гиперэкспоненциальным входящим потоком 
и~профилактиками обслуживающего прибора при освобождении системы.
Тип  требования определяется случайно с~заданными вероятностями 
в~момент его поступления в~систему обслуживания. Требования первого
типа имеют относительный приоритет перед требованиями второго
типа. Найдено нестационарное совместное распределение числа
требований каждого типа в~системе. Профилактики прибора
заключаются в~том, что в~момент освобождения системы от требований
прибор на случайное время с~заданным распределением становится
недоступным для обслуживания. Если за время профилактики поступает
хотя бы одно требование, то начинается нормальное функционирование
системы. Если требования не поступают, то прибор отправляется на
новую профилактику. Такие системы хорошо описывают
функционирование большого числа реальных вычислительных и~информационных систем.}

\KW{гиперэкспоненциальный поток; профилактики
обслуживающего прибора; одноканальная система; относительный
приоритет; длина очереди}

\DOI{10.14357/19922264180405}
  
%\vspace*{4pt}


\vskip 10pt plus 9pt minus 6pt

\thispagestyle{headings}

\begin{multicols}{2}

\label{st\stat}

\section{Введение}

В классической системе массового обслуживания ожидание требований
в очереди связано только с~занятостью обслуживающего прибора. В~то
же время в~реальных системах сам  прибор может пребывать как 
в~активном, так и~в~неактивном состоянии. Такое неактивное
состояние прибора (в~литературе на английском языке используется
термин vacation, а~на русском~--- профилактика или прогулка) может
быть связано со многими причинами. В~част\-ности, сис\-те\-мы
обслуживания с~профилактиками прибора хорошо описывают
функционирование  реальных вычислительных и~информационных систем,
в которых наряду с~основными требованиями имеются второстепенные.
Второстепенные требования всегда присутствуют в~сис\-те\-ме, а~их
обслуживание может проводиться только тогда, когда нет основных,
т.\,е.\ в~фоновом режиме.

С точки зрения самого процесса профилактики прибора существует
несколько ее разновидностей. Во-пер\-вых, могут быть разными
правила, задающие условия начала профилактики: прибор может брать
перерыв только при  полном исчерпании требований в~очереди
(exhaustive service) либо при наличии определенного их числа
(nonexhaustive service). Во-вто\-рых, могут быть разными правила
возвращения прибора в~работу. С~этой точки зрения различают случаи
однократного (single vacation) и~многократного (multiple vacation)
перерыва в~работе. В~первом случае ушедший на профилактику прибор
после ее окончания находится в~рабочем состоянии независимо от
наличия требований в~системе. Во втором случае прибор, не
обнаружив новых требований в~очереди, уходит на новую
профилактику.


В работах~[1--4] можно найти обзор известных результатов, большое
число постановок задач, описание различных приложений и~обширную
библиографию по анализу систем с~профилактиками обслуживающего
прибора.


В настоящей работе исследуется совместное распределение длин
очередей в~нестационарном режиме в~однолинейной системе 
с~ожиданием, гиперэкспоненциальным входящим потоком, двумя типами
требований и~относительным приоритетом. Аналогичная неприоритетная
система обслуживания исследована в~[5].

\vspace*{-6pt}

\section{Описание модели}

Рассматривается однолинейная система массового обслуживания 
с~двумя приоритетными классами требований. Входящий поток~---
гиперэкспоненциальный с~функцией распределения интервалов между
поступлениями требований вида:
\begin{multline*}
A(t)=\sum\limits_{i=1}^kc_i\left(1-e^{-a_it}\right),\enskip t>0,\enskip
a_i>0,\enskip c_i>0,\\
a_i\ne a_j\,,\enskip i\ne j\,,\enskip  \sum\limits_{i=1}^k c_i=1\,.
\end{multline*}

Каждое поступившее требование направляется в~первый класс 
с~вероятностью~$p$ и~во второй класс с~вероятностью $1\hm-p$
независимо от остальных требований. Требования первого класса
обладают относительным приоритетом перед требованиями второго
класса. Длительности обслуживания требований $i$-го приоритетного
класса~--- независимые в~совокупности и~не зависящие от входящего
потока случайные величины с~функцией распределения~$B_i(x)$,
$i\hm=1,2.$
 Если в~некоторый момент времени система освободилась от требований, 
 то обслуживающий прибор
 отправляется на профилактику, которая длится случайное время с~функцией 
 распределения~$C(x).$
 Не ограничивая общности, будем считать, что $B_i(x)\hm<1$
 и~$C(x)\hm<1$  для любого~$x$ 
 и~существуют плотности
 распределения~$b_i(x)$ и~$c(x).$
  Обозначим:
$$
 \beta_i(s)=\int\limits_0^{\infty}e^{-sx}b_i(x)\,dx\,;\enskip 
  \gamma(s)=\int\limits_0^{\infty}e^{-sx}c(x)\,dx\,.
$$
Пока прибор находится на профилактике, он не доступен для
обслуживания. Если за время профилактики поступают требования,
после ее завершения начинается их обслуживание. Если ни одно
требование не поступает, то прибор отправляется на новую
профилактику. Длительности различных профилактик являются
независимыми случайными величинами 
и~не зависят от входящего потока и~времен обслуживания.

\section{Вспомогательные результаты}

  Рассмотрим многочлен по $\mu$ степени $k$ вида:
\begin{multline}
\label{1}
\prod\limits_{i=1}^k\left(\mu+a_i\right)-{}\\
{}-
\left(pz_1+(1-p)z_2\right)\sum\limits_{j=1}^kc_ja_j\prod\limits_{i\ne
j}\left(\mu+a_i\right)\,.
\end{multline}
Занумеруем его корни $\mu_1(z_1,z_2),\ldots,\mu_k(z_1,z_2)$ таким образом,
чтобы они были непрерывными функциями и~$\mu_1(1,1)\hm=0.$ Тогда
$\mathrm{Re}\, \mu_j\left(z_1,z_2\right)\hm<0$, $|z_1|\hm<1$, 
$|z_2|\hm<1,$ $\mu_i(z_1,z_2)\hm\ne \mu_j(z_1,z_2),$ $ i\hm\ne j$,
$j\hm=1,\ldots,k.$ Обозначим:
$$
\alpha_m(z_1,z_2)=\prod\limits_{j\ne m}\left(\mu_m\left(z_1,z_2\right)-
\mu_j\left(z_1,z_2\right)\right)\,.
$$
Справедливы следующие леммы.

\smallskip

\noindent
\textbf{Лемма~1.}\
\textit{Для любого $l=1,\ldots,\:k$ система уравнений}
$$
z_j=\beta_j(s-\mu_l(z_1,z_2)),\ \ j=1,2,
$$
\textit{имеет единственное решение $z_i=z_{il}(s)$ такое, 
что $|z_{il}(s)|\hm<1$ при $l\hm=2,\ldots, k,$ $\mathrm{Re}\, s\hm\geqslant 0,$ 
а~$z_{i1}(0)\hm=1$, $|z_{i1}(s)|\hm<1$ при} $\mathrm{Re}\, s\hm> 0$, $i\hm=1,2.$

\smallskip

\noindent
\textbf{Лемма~2.}\
\textit{При каждом $l\hm=1,\ldots,k$ уравнение}
$$
z_1=\beta_1\left(s-\mu_l(z_1,z_2)\right)
$$
\textit{имеет единственное решение $z_1\hm=z_{1l}(z_2,s),$ 
аналитическое в~области $\mathrm{Re}\, s\hm>0$, $|z_2|\hm<1.$
}

\smallskip

Положим
$$
\lambda_l(s)=\mu_l\left(z_{1l}(s),z_{2l}(s)\right)\,.
$$




\section{Распределение длины очереди}

  Гиперэкспоненциальный поток можно рас\-смат\-ри\-вать как
пуассоновский поток со случайной интен\-сив\-ностью~$a,$ которая
принимает $k$ различных значений $a_1,\ldots,a_k$  с~вероятностями
$c_1,\ldots,c_k.$ Текущее значение~$a$ разыгрывается в~момент
поступления требования и~не меняется между двумя соседними
поступлениями. Введем случайный процесс~$j(t)$ такой, что если
$a\hm=a_j$ в~момент времени $t,$ то $j(t)\hm=j.$

Целью работы является нахождение распределения случайного процесса
$\left(L_1(t),L_2(t)\right),$ где $L_i(t)$~--- число требований из
$i$-го приоритетного класса, находящихся в~системе в~момент
времени~$t.$

При сделанных предположениях относительно параметров изучаемой
системы обслуживания\linebreak процесс $\left(L_1(t),L_2(t)\right)$ не
является, вообще говоря, марковским. Пусть $i(t)=i$, $i\hm=1,2,$ если
в~момент времени~$t$ обслуживается требование из $i$-го
приоритетного класса, и~$i(t)\hm=0,$ если в~момент времени~$t$ прибор
находится на профилактике. Случайный процесс~$x(t)$ определим
следующим образом. Если $i(t)\hm\ne 0,$ то $x(t)$ есть
время, прошедшее с~начала обслуживания требования, находящегося на
приборе, до момента~$t.$ Если $i(t)\hm=0,$ то $x(t)$ есть время,
прошедшее с~начала профилактики прибора до момента~$t.$ Случайный
процесс $\left(L_1(t),L_2(t),i(t),j(t),x(t)\right)$ является
однородным марковским процессом. Положим
\begin{multline*}
P_{ij}(n_1,n_2,x,t)=\fr{\partial}{\partial x}
\mathbf{P}\left(L_1(t)=n_1,L_2(t)=n_2,\right.\\
\left. i(t)=i,j(t)=j,x(t)<x
\vphantom{L_1}\right)\,,\enskip 
 x\geqslant 0,\\ 
 j=1,\ldots,k,\enskip i=0,1,2;
\end{multline*}
\begin{gather*}
\eta_i(x)=\fr{b_i(x)}{1-B_i(x)},\ i=1,2;\enskip 
\eta_0(x)=\fr{c(x)}{1-C(x)}\,;\\
\delta_{i,j}=\begin{cases}
1,&\ i=j;\\ 
0,&\ i\ne j\,.
\end{cases}
\end{gather*}
Функции $P_{ij}(n_1,n_2,x,t)$  удовлетворяют при $x\hm>0$
системам дифференциальных уравнений:
\begin{multline}
\label{3}
\fr{\partial P_{ij}(n_1,n_2,x,t)}{\partial t}+\fr{\partial
P_{ij}(n_1,n_2,x,t)}{\partial
x}={}\\
{}=-(a_j+\eta_i(x))P_{ij}(n_1,n_2,x,t)+ {}\\
{}+
c_j\sum\limits_{l=1}^ka_l\left(p\:P_{il}(n_1-1,n_2,x,t)+{}\right.\\
\left.{}+
(1-p)P_{il}(n_1,n_2-1,x,t)\right)
\end{multline}
и краевым условиям при $x\hm=0$:
\begin{multline}
\label{5}
P_{0j}(n_1,n_2,0,t)=0,\ n_1+n_2>0;\\
P_{0j}(0,0,0,t)=\int\limits_0^{\infty}P_{0j}(0,0,x,t)\eta_0(x)\,dx+{}\\
 {}+\int\limits_0^{\infty}P_{1j}(1,0,x,t)\eta_1(x)dx+{}\\
 {}+
\int\limits_0^{\infty}P_{2j}(0,1,x,t)\eta_2(x)\,dx\,;
\end{multline}

\vspace*{-12pt}

\noindent
\begin{multline}
\label{6}
P_{1j}(n_1,n_2,0,t)+P_{2j}(n_1,n_2,0,t)={}\\
{}=\int\limits_0^{\infty}P_{1j}(n_1+1,n_2,x,t)\eta_1(x)\,dx+{}\\
{}+
\int\limits_0^{\infty}P_{2j}(n_1,n_2+1,x,t)\eta_2(x)\,dx+{}\\
{}+\int\limits_0^{\infty}P_{0j}(n_1,n_2,0,t)\eta_0(x)\,dx\,.
\end{multline}

Будем предполагать, что в~начальный момент времени $t\hm=0$ система
свободна от требований, а~с~начала профилактики прибора прошло
случайное время с~заданным распределением с~плотностью $d(x).$
Таким образом,
\begin{align*}
P_{ij}\left(n_1,n_2,x,0\right)&=0,\ i=1,2;
\\
P_{0j}\left(n_1,n_2,x,0\right)&=c_jd(x)\delta_{n_1+n_2,0},\ \
j=1,\ldots,k\,.
\end{align*}
Положим
\begin{multline*}
p_{ij}\left(z_1,z_2,x,s\right)={}\\
{}=\sum\limits_{n_1=0}^{\infty}
\sum\limits_{n_2=0}^{\infty}z_1^{n_1}z_2^{n_2}\!
\int\limits_0^{\infty}e^{-st}P_{ij}(n_1,n_2,x,t)\,dt\,;
\end{multline*}
$$
  \psi(s)=\int\limits_0^{\infty}e^{-sx}\,dx
  \int\limits_0^{\infty}\fr{c(u+x)d(u)}{1-C(u)}\,du\,.
$$
Тогда, учитывая начальные условия,  из \eqref{3}
получаем:
\begin{multline}
\label{7} 
\fr{\partial p_{ij}(z_1,z_2,x,s)}{\partial x}={}\\
{}=-\left(s+a_j+\eta_i(x)\right)p_{ij}
\left(z_1,z_2,x,s\right)+{}\\
{}+c_j\left(pz_1+(1-p)z_2\right)
\sum\limits_{l=1}^ka_lp_{il}\left(z_1,z_2,x,s\right),\\ 
i=1,2;
\end{multline}

\vspace*{-12pt}

\noindent
\begin{multline}
\label{8} 
\fr{\partial p_{0j}(z_1,z_2,x,s)}{\partial x}={}\\
{}=-\left(s+a_j+\eta_0(x)\right)p_{0j}\left(z_1,z_2,x,s\right)+{}\\
{}+c_j\left(pz_1+(1-p)z_2\right)\sum\limits_{l=1}^ka_lp_{0l}\left(z_1,z_2,x,s\right)+{}\\
{}+ c_jd(x).
\end{multline}
Решения \eqref{7} и~\eqref{8} имеют вид:
\begin{multline}
\label{9}
p_{ij}\left(z_1,z_2,x,s\right)=\left(1-B_i(x)\right)c_j\times{}\\
{}\times \sum\limits_{m=1}^k\fr{\gamma_i^{(m)}(z_1,z_2,s)}{\mu_m(z_1,z_2)+a_j}\,
e^{-(s-\mu_m(z_1,z_2))x}\,,\\
 i=1,2\,,
\end{multline}
\vspace*{-12pt}

\noindent
\begin{multline}
\label{10}
p_{0j}\left(z_1,z_2,x,s\right)={}\\
{}=\left(1-C(x)\right)
c_j\!\!\sum\limits_{m=1}^k\!\! e^{-(s-\mu_m(z_1,z_2))x}\!
\!\left(\!
\vphantom{\int\limits_{l=1}^k}
\delta^{(m)}\left(z_1,z_2,s\right)+{}\right.\\
%\left.
{}+\alpha_m^{-1}\left(z_1,z_2\right)
\prod\limits_{l=1}^k
\left(\mu_m\left(z_1,z_2\right)+a_l\right)\times{}\\
\left.{}\times \int\limits_0^x\!
e^{(s-\mu_m(z_1,z_2))u}
\fr{d(u)}{1-C(u)}\,du
\right)
\!\Bigg/ \!\left(\mu_m\left(z_1,z_2\right)+{}\right.\\
\left.{}+a_j\right)\,,
\end{multline}
где функции $\gamma_i^{(m)}(z_1,z_2,s)$  и~$\delta^{(m)}(z_1,z_2,s)$ являются
произвольными функциями указанных переменных и~определяются из
краевых условий. Из~\eqref{5} и~\eqref{6} получаем:
\begin{multline}
\label{11}
p_{1j}\left(z_1,z_2,0,s\right)+p_{2j}\left(z_1,z_2,0,s\right)={}\\
{}=z_1^{-1}\int\limits_0^{\infty}p_{1j}\left(z_1,z_2,x,s\right)\eta_1(x)\,dx+{}
\\
+z_2^{-1}\int\limits_0^{\infty}p_{2j}\left(z_1,z_2,x,s\right)\eta_2(x)\,dx+{}\\
{}+
\int\limits_0^{\infty}p_{0j}\left(z_1,z_2,x,s\right)\eta_0(x)\,dx
-p_{0j}\left(z_1,z_2,0,s\right)\,.
\end{multline}
Заметим, что $p_{0j}(z_1,z_2,0,s)$ не зависит от $z_1$ и~$z_2,$ т.\,е.\
$p_{0j}(z_1,z_2,0,s)\hm=q_j(s).$ 
Подставляя~\eqref{9} и~\eqref{10} в~\eqref{11}, получаем:
\begin{multline}
\label{12}
\gamma_1^{(m)}\left(z_1,z_2,s\right)\left(1-z_1^{-1}\beta_1(s-\mu_m(z_1,z_2))\right)+{}\\
{}+
\gamma_2^{(m)}(z_1,z_2,s)\left(1-z_2^{-1}\beta_2(s-\mu_m(z_1,z_2))\right)={}\\
{} =
\delta^{(m)}\left(z_1,z_2,s\right)\left(\gamma\left(s-\mu_m\left(z_1,z_2\right)\right)-1\right)+{}\\
{}+
\alpha_m^{-1}\left(z_1,z_2\right)\prod\limits_{l=1}^k
\left(\mu_m\left(z_1,z_2\right)+a_l\right)\psi\left(s-\mu_m(z_1,z_2)\right),\\
j=1,\ldots,k.
\end{multline}
В силу леммы~1 левая часть~\eqref{12} обращается в~0 при
$z_1\hm=z_{1m}(s)$ и~$z_2\hm=z_{2m}(s)$, $m\hm=1,\ldots,k.$ Следовательно,
\begin{multline}
\label{13}
\delta^{(m)}\left(z_{1m}(s),z_{2m}(s),s\right)={}\\
{}=\fr{\psi(s-\lambda_m(s))}{\alpha_m(z_{1m}(s),z_{2m}(s))
(1-\gamma(s-\lambda_m(s)))}\times{}\\
{}\times \prod\limits_{l=1}^k\left(\lambda_m(s)+a_l\right).
\end{multline}
Из \eqref{10} следует, что
$$
q_j(s)=c_j\sum\limits_{m=1}^k\fr{\delta^{(m)}(z_1,z_2,s)}{\mu_m(z_1,z_2)+a_j},\
j=1,\ldots,k .
$$
Решая эту систему уравнений относительно
$\delta^{(m)}(z_1,z_2,s),$ получаем:
\begin{multline}
\label{n1}
\delta^{(m)}(z_1,z_2,s)=\left(pz_1+(1-p)z_2\right)\times{}\\
{}\times
\fr{\prod\nolimits_{j=1}^k(\mu_m(z_1,z_2)+a_j)}
{\alpha_m(z_1,z_2)}\sum\limits_{l=1}^k\frac{a_lq_l(s)}{\mu_m(z_1,z_2)+a_l}.
\end{multline}
Подставляя в~\eqref{n1} $z_1\hm=z_{1m}(s)$ и~$z_2\hm=z_{2m}(s),$ имеем:
\begin{multline}
\label{14}
\delta^{(m)}\left(z_{1m}(s),z_{1m}(s),s\right)={}\\
{}=
\left(pz_{1m}(s)+(1-p)z_{2m}(s)\right)\times{}\\
{}\times
\fr{\prod\nolimits_{j=1}^k
(\lambda_m(s)+a_j)}{\alpha_m(z_{1m}(s),z_{1m}(s))}
\sum\limits_{l=1}^k\fr{a_lq_l(s)}{\lambda_m(s)+a_l}\,.
\end{multline}
Сравнивая два представления~\eqref{13} в~\eqref{14} для
$\delta^{(m)}(z_m(s),s),$ получаем систему уравнений для~$q_l(s)$:
\begin{multline*}
\sum\limits_{l=1}^k\fr{a_lq_l(s)}{\lambda_m(s)+a_l}={}\\
{}=\fr{\psi(s-\lambda_m(s))}{(pz_{1m}(s)+(1-p)z_{2m}(s))
(1-\gamma(s-\lambda_m(s)))},\\
m=1,\ldots,k\,,
\end{multline*}
из которой находим
\begin{multline}
\hspace*{-3pt}q_l(s)=c_l\prod\limits_{j=1}^k
\left(\lambda_l(s)+a_j\right) 
\sum\limits_{m=1}^k
%\fr
\psi(s-\lambda_m(s))\!\Bigg/ \!
\Bigg(\left(1-{}\right.\\
\left.
{}-\gamma\left(s-\lambda_m(s)\right)\right)(\lambda_m(s)+a_l)\times{}\\
{}\times \prod\limits_{n\ne m}(\lambda_m(s)-\lambda_n(s))\!\Bigg).
\label{15}
\end{multline}
Подставляя \eqref{15} в~\eqref{n1} и~учитывая~\eqref{1}, получаем:
\begin{multline*}
\delta^{(m)}(z_1,z_2,s)=\fr{(pz_1+(1-p)z_2)}{\alpha_m(z_1,z_2)}\times
\\
\times\sum\limits_{j=1}^k
\fr{\psi(s-\lambda_j(s))\prod\nolimits_{l=1}^k(\lambda_j(s)+a_l)}
{(pz_{1j}(s)+(1-p)z_{2j}(s))(1-\gamma(s-\lambda_j(s)))}\times{}\\
{}\times\prod\limits_{\nu\ne j}
\fr{\mu_m(z_1,z_2)-\lambda_{\nu}(s)}{\lambda_j(s)-\lambda_{\nu}(s)}\,.
\end{multline*}
Положим
$$
\lambda_m(z_2,s)=\mu_m\left(z_{1m}(z_2,s),z_2\right),\enskip m=1,\ldots,k\,.
$$
Подставляя в~\eqref{12} $z_1\hm=z_{1m}(z_2,s)$, имеем:
\begin{multline}
\label{1q}
\gamma_2^{(m)}\left(z_{1m}(z_2,s),z_2,s\right)={}\\
{}=\fr{\delta^{(m)}(z_{1m}(z_2,s),z_2,s)(\gamma_m(s-\lambda_m(z_2,s))-1)}
{1-z_2^{-1}\beta_2(s-\lambda_m(z_2,s))}+{}
\\
{}+\alpha_m^{-1}(z_{1m}(z_2,s),z_2)\psi(s-\lambda_m(z_2,s))
\prod\limits_{l=1}^k\left(\lambda_m(z_2,s)+{}\right.\\
\left.{}+a_l\right)\!\Bigg/\!
\left(
1-z_2^{-1}\beta_2(s-\lambda_m(z_2,s))\right).
\end{multline}
Далее, из~\eqref{9} следует:
$$
p_{2j}(z_1,z_2,0,s)=c_j\sum\limits_{m=1}^k
\fr{\gamma_2^{(m)}(z_1,z_2,s)}{\mu_m(z_1,z_2)+a_j}\,.
$$
Отсюда
\begin{multline}
\label{2q}
\gamma_2^{(m)}(z_1,z_2,s)=\fr{pz_1+(1-p)z_2}{\alpha_m(z_1,z_2)}\times{}\\
{}\times
\prod\limits_{j=1}^k(\mu_m(z_1,z_2)+a_j)
\sum\limits_{l=1}^k\fr{a_lp_{2l}(z_1,z_2,0,s)}{\mu_m(z_1,z_2)+a_l}\,.
\end{multline}
Так как $p_{2j}(z_1,z_2,0,s)$ не зависит от $z_1$, то
\begin{multline}
\label{3q}
p_{2j}\left(z_1,z_2,0,s\right)={}\\
{}=c_j
\sum\limits_{m=1}^k\fr{\gamma_2^{(m)}\left(z_{1m}(z_2,s),z_2,s\right)}{\lambda_m(z_2,s)+a_j}\,.
\end{multline}
Таким образом, соотношения~\eqref{1q}--\eqref{3q} полностью
определяют $\gamma_2^{(m)}(z_1,z_2,s)$ при любых $z_1$ и~$z_2$.
Теперь из~\eqref{12} можно найти $\gamma_2^{(m)}(z_1,z_2,s)$.

Все функции, необходимые для вычисления $p_{ij}(z_1,z_2,x,s)$,
$i\hm=0,1,2$, $j\hm=1,\ldots,k,$ найде-\linebreak\vspace*{-12pt}

\columnbreak

\noindent
ны. Искомая производящая функция
процесса $(L_1(t),L_2(t))$ равна:

\noindent
\begin{multline*}
\int\limits_0^{\infty}e^{-st}\mathbf{E}
z_1^{L_1(t)} z_2^{L_2(t)}\,dt={}\\
{}=
\sum\limits_{i=0}^2\sum\limits_{j=1}^k\int\limits_0^{\infty}p_{ij}
\left(z_1,z_2,x,s\right)\,dx\,.
\end{multline*}

\vspace*{-18pt}

{\small\frenchspacing
 {%\baselineskip=10.8pt
 \addcontentsline{toc}{section}{References}
 \begin{thebibliography}{9}
\bibitem{1-u}
\Au{Doshi B.\,T.} Queueing systems with vacations~--- a~survey~// 
Queueing Syst., 1986. Vol.~1.  P.~29--66.
\bibitem{2-u}
\Au{Takagi H.} Time-dependent analysis of $M\vert G\vert 1$ vacation models 
with exhaustive service~// Queueing Syst.,
1990. Vol.~6.  P.~369--390.
\bibitem{3-u}
\Au{Li J., Tian N., Zhang~Z.\,G. , Luh~H.\,P.} 
Analysis of the $M\vert G\vert 1$ queue with exponentially working vacations~--- 
a~matrix analytic approach~// Queueing Syst., 2009. Vol.~61.
P.~139--166.
\bibitem{4-u}
\Au{Bouman N., Borst S.\,C., Boxma~O.\,J., Leeuwaarden~J.\,S.\,H.} 
Queues with random back-offs~// Queueing Syst.,
2014. Vol.~77. P.~33--74.
\bibitem{5-u}
\Au{Ушаков~В.\,Г.} Система обслуживания с~гиперэкспоненциальным входящим потоком 
и~профилактиками прибора~// Информатика и~её применения, 2016. Т.~10. 
Вып.~2. С.~93--98.
 \end{thebibliography}

 }
 }

\end{multicols}

\vspace*{-9pt}

\hfill{\small\textit{Поступила в~редакцию 11.05.18}}

\vspace*{6pt}

%\pagebreak

%\newpage

%\vspace*{-28pt}

\hrule

\vspace*{2pt}

\hrule

%\vspace*{-2pt}

\def\tit{A~HEAD OF~THE~LINE PRIORITY QUEUE\\ WITH~WORKING VACATIONS}

\def\titkol{A head of the line priority queue with working vacations}

\def\aut{E.\,S.~Kondranin$^1$ and~V.\,G.~Ushakov$^{1,2}$}

\def\autkol{E.\,S.~Kondranin and~V.\,G.~Ushakov}

\titel{\tit}{\aut}{\autkol}{\titkol}

\vspace*{-11pt}


\noindent
$^1$Department of 
Mathematical Statistics, Faculty of Computational Mathematics and Cybernetics, 
M.\,V.~Lo\-mo-\linebreak
$\hphantom{^1}$no\-sov Moscow State University, 1-52~Leninskiye Gory, 
Moscow 119991, GSP-1, Russian Federation

\noindent
$^2$Institute of Informatics Problems, Federal Research Center 
``Computer Science and Control'' of the Russian\linebreak
$\hphantom{^1}$Academy of Sciences,  44-2~Vavilov Str., Moscow 119333, Russian Federation

\def\leftfootline{\small{\textbf{\thepage}
\hfill INFORMATIKA I EE PRIMENENIYA~--- INFORMATICS AND
APPLICATIONS\ \ \ 2018\ \ \ volume~12\ \ \ issue\ 4}
}%
 \def\rightfootline{\small{INFORMATIKA I EE PRIMENENIYA~---
INFORMATICS AND APPLICATIONS\ \ \ 2018\ \ \ volume~12\ \ \ issue\ 4
\hfill \textbf{\thepage}}}

\vspace*{3pt}



\Abste{The authors analyze the single-server queueing system with 
two types of customers, head of the line priority, hyperexponential 
input stream, and working vacations. The authors obtain the Laplace 
transform (with respect to an arbitrary point in time) of the joint 
distribution of server state, queue size, and elapsed time in that state. 
The authors restrict themselves to a~system with exhaustive service (the 
queue must be empty when the server starts a vacation) and multiple vacations. 
The queueing systems with vacations have been well studied because of their 
applications in modeling computer networks, communication, and manufacturing 
systems. For example, in many digital systems, the processor is multiplexed 
among a~number of jobs and, hence, is not available all the time to handle one job type. 
Besides such an application, theoretical interest in vacation models 
has been aroused with respect to their relationship with polling models.}

\KWE{hyperexponential input stream; working vacations; single server; 
head of the line priority; queue length}



\DOI{10.14357/19922264180405}

\vspace*{-14pt}

\Ack
\noindent
This work was supported by the Russian Foundation for Basic Research 
(project 18-07-00678).


%\vspace*{6pt}

  \begin{multicols}{2}

\renewcommand{\bibname}{\protect\rmfamily References}
%\renewcommand{\bibname}{\large\protect\rm References}

{\small\frenchspacing
 {%\baselineskip=10.8pt
 \addcontentsline{toc}{section}{References}
 \begin{thebibliography}{9}
\bibitem{1-u-1}
\Aue{Doshi, B.\,T.} 1986. Queueing systems with vacations~--- a~survey. 
\textit{Queueing Syst.} 1:29--66.
\bibitem{2-u-1}
\Aue{Takagi, H.} 1990. Time-dependent analysis of $M\vert G\vert M\vert 1$ 
vacation models with exhaustive service. \textit{Queueing Syst.} 6:369--390.
\bibitem{3-u-1}
\Aue{Li, J., N. Tian, Z.\,G.~Zhang,  and H.\,P.~Luh.} 2009. Analysis of the 
$M\vert G\vert 1$ queue with exponentially working vacations~--- 
a~matrix analytic approach. \textit{Queueing Syst.} 61:139--166.
{\looseness=1

}
\bibitem{4-u-1}
\Aue{Bouman, N., S.\,C.~Borst, O.\,J.~Boxma, and J.\,S.\,H.~Leeuwaarden.} 
2014. Queues with random back-offs. \textit{Queueing Syst.} 77:33--74.
\bibitem{5-u-1}
\Aue{Ushakov, V.\,G.} 2016. Sistema obsluzhivaniya s~gipereksponentsialnym 
vkhodyashchim potokom i~profilaktikami\linebreak pribora [Queueing system with working 
vacations and hyperexponential input stream]. 
\textit{Informatika i~ee Primeneniya~--- Inform. Appl.} 10(2):93--98.
\end{thebibliography}

 }
 }

\end{multicols}

\vspace*{-6pt}

\hfill{\small\textit{Received May 11, 2018}}

%\pagebreak

%\vspace*{-18pt}

\Contr

\noindent
\textbf{Kondranin Egor S.} (b.\ 1995)~---  MSc student, Department of 
Mathematical Statistics, Faculty of Computational Mathematics and Cybernetics, 
M.\,V.~Lomonosov Moscow State University, 1-52~Leninskiye Gory, 
Moscow 119991, GSP-1, Russian Federation; \mbox{ekondranin@yandex.ru}

\vspace*{6pt}

\noindent
\textbf{Ushakov Vladimir G.} (b.\ 1952)~--- 
Doctor of Science in physics and mathematics, professor, Department of Mathematical 
Statistics, Faculty of Computational Mathematics and Cybernetics, 
M.\,V.~Lomonosov Moscow State University, 1-52~Leninskiye Gory, Moscow 119991, 
GSP-1, Russian Federation; 
senior scientist, Institute of Informatics Problems, Federal Research Center 
``Computer Science and Control'' of the Russian Academy of Sciences, 
44-2~Vavilov Str., Moscow 119333, Russian Federation; \mbox{vgushakov@mail.ru}
\label{end\stat}

\renewcommand{\bibname}{\protect\rm Литература}              %8
%\renewcommand{\figurename}{\protect\bf Figure}
\renewcommand{\tablename}{\protect\bf Table}

\def\stat{razum}


\def\tit{COMPARISON OF TWO ACTIVE QUEUE MANAGEMENT SCHEMES THROUGH THE~$M/D/1/N$ 
QUEUE}

\def\titkol{Comparison of two active queue management schemes through the $M/D/1/N$ 
queue}

\def\autkol{M.\,G.~Konovalov and R.\,V.~Razumchik}

\def\aut{M.\,G.~Konovalov$^1$ and R.\,V.~Razumchik$^2$}

\titel{\tit}{\aut}{\autkol}{\titkol}

%{\renewcommand{\thefootnote}{\fnsymbol{footnote}}
%\footnotetext[1] {The 
%research of Yuri Kabanov was done under partial financial support   of the grant 
%of  RSF No.\,14-49-00079.}}

\renewcommand{\thefootnote}{\arabic{footnote}}
\footnotetext[1]{Institute of Informatics Problems, Federal Research Center ``Computer Science and Control'' of the Russian Academy of Sciences,
44/2~Vavilov Str., Moscow 119333, Russian Federation, \mbox{mkonovalov@ipiran.ru}}
\footnotetext[2]{Institute of Informatics Problems, Federal Research Center 
``Computer Science and Control'' of the Russian Academy of Sciences,
44/2~Vavilov Str., Moscow 119333, Russian Federation; 
Peoples' Friendship University of Russia (RUDN University),
6~Miklukho-Maklaya Str., Moscow 117198, Russian 
Federation; 
\mbox{rrazumchik@ipiran.ru} %\mbox{razumchik\_rv@rudn.ru
}


\index{Konovalov M.\,G.}
\index{Razumchik R.\,V.}
\index{Коновалов М.\,Г.}
\index{Разумчик Р.\,В.}

\def\leftfootline{\small{\textbf{\thepage}
\hfill INFORMATIKA I EE PRIMENENIYA~--- INFORMATICS AND
APPLICATIONS\ \ \ 2018\ \ \ volume~12\ \ \ issue\ 4}
}%
 \def\rightfootline{\small{INFORMATIKA I EE PRIMENENIYA~---
INFORMATICS AND APPLICATIONS\ \ \ 2018\ \ \ volume~12\ \ \ issue\ 4
\hfill \textbf{\thepage}}}



\Abste{The paper focuses on giving the first in the literature numerical evidence
that the stationary performance characteristics of single-server queues
with the general renovation mechanism may be as good as of single-server queues
with the RED-type active queue management mechanisms
(AQM). Comparison is made in the queueing
theory context: the basic model is the $M/D/1/N$ queue. 
The characteristics reported are: the loss ratio, average system size, and 
average number of consecutive losses along 
with the standard deviations. Numerical results are based on the 
well-known facts and some new analytic results, presented in the paper.}

\KWE{queueing system; active queue management; RED; renovation}

\DOI{10.14357/19922264180402}


\vspace*{1pt}


\vskip 12pt plus 9pt minus 6pt

      \thispagestyle{myheadings}

      \begin{multicols}{2}

                  \label{st\stat}


\section{Introduction}

\noindent
A large number of AQM mechanisms have been developed
up to nowadays and quite a~lot of efforts have been devoted to the studies of
their efficiency. 
These mechanisms may be applicable in different contexts but historically, 
they are more often related to communication networks
in the context of mitigation of congestion and congestion avoidance.
This problem, as highlighted in the latest RFC~7567~\cite{RFC7567},
still does not have a~satisfying solution. 
An AQM mechanism is an advanced rejection discipline, 
when an arriving customer (packet, job, etc.) is lost randomly with a~probability 
that may depend on the (current, past, average, etc.) system state or performance.
The most popular class of AQM mechanisms seems to be the Random Early Detection (RED) and
its ramifications like GRED (Gentle RED), REM 
(Random Exponential Marking),
etc.\ (a~recent survey on the AQM can be found in~\cite{Adams}).
The goals of AQM are usually diverse and conflicting: 
prevent queues from growing too long, maintain high server (processor) utilization
and low variance of the queue size, ensure fairness among competing flows, 
and others. These are discussed in detail in~\cite{RFC7567} in the context
of communications network but most of the goals are applicable in other contexts as well
(buffer-bloat problems in data-center, etc.).

Besides simulation, analytic performance evaluation of systems with AQM is quite often
carried out in the queueing theory context (see,
for example,~\cite{Bonald,Chyd,Chyd2,oleg,hao,konnew} and references therein). 
Usually, the system with an AQM mechanism is modeled as a~queueing system or network
and then its performance characteristics are studied using known analytic techniques. 
Throughout the paper, we stay within the queueing theory context.

In the series of recent papers~\cite{Kreinin,Zaryadov2010,zarN1,zarN2,Zaryadov2009},
the authors have proposed the new type of AQM mechanism which they call 
\textit{renovation}. 
Roughly speaking, renovation implies that each customer, 
having received service, may remove some additional work from the system
(i.\,e., may renovate it). We will make this definition more precise in the next 
section 
but for now, note 
that queue management \textit{after service completions} is what makes the renovation
 different 
from the most known AQM schemes\footnote[3]{Indeed, renovation and most of the known AQM 
mechanisms
are conceptually different. One of the main goals of AQM mechanisms is to prevent 
queue from growing too large
leaving space for potential new arrivals.
In systems with renovation, the queue can become full (meaning that fewer customers are 
lost)
but after a~service completion, several customers may be removed from it. In this way, 
the content of the queue
can be preserved at a~certain average level but the loss pattern becomes intricate.}, 
in which the decisions are made \textit{upon arrivals}.  
To our best knowledge, there are no studies, 
which tell whether the performance of the systems with renovation is
better/same/worse than that of the same systems but with the implemented AQM mechanisms.
Thus, there is a~lack of bridge between available theoretical results for renovation and 
its practical perspective.

The scope of this paper is to give the first in the literature numerical evidence 
that the stationary performance of 
single-server queueing systems with the implemented renovation mechanism can
be as good as of 
the same single-server queues but the well-known packed dropping procedures like RED.
The emphasis is primarily on the reporting of this finding, complemented 
with some new insights into 
queueing systems with renovation. The relation to other 
AQM mechanisms like CoDel~\cite{RFC8289} is not discussed here. 
Moreover, in the numerical experiments presented here, 
we did not use any benchmarks to generate the traffic profiles 
but used the theoretical distributions instead.

The main stationary performance characteristics reported are: the loss ratio, 
the average number in the system
(average system size), and the average number of consecutive losses along with 
their standard deviations.
After introducing the renovation mechanism and the analytic setting,
in which  renovation mechanism is compared with RED, we give the new analytic results 
for computing system size moments and the loss ratio under the renovation mechanism.
The results presented in the numerical section are based on the analytic results. 
Monte-Carlo simulation is used only for the average (and standard deviation) 
number of consecutive losses in the system with renovation. 


\section{Settings and the Model}

\noindent
We follow the queueing theoretic approach and as the basic model, we use $M/D/1/N$ queue,
i.\,e., queue of finite capacity~$N$ fed at rate~$\lambda$ by a~Poisson flow of customers,
which are served on a~first-come-first-served basis  by a~single server with constant
service time $d>0$.
We assume that the system is in the steady state.
When an arriving customer sees that the queue is full,
it is lost. If no other type of losses occur in the system,
we say that the Tail Drop mechanism is implemented in it.

If an arriving customer is lost with probability~$d_n$
where~$n$ is the total number of customers 
it sees in the system on arrival, then we say that an AQM mechanism 
is implemented in the system. Various dropping functions can be obtained
by specifying the values of~$d_n$
(see, for example, RED dropping function in~\cite[Example 1]{Chyd}).
Important notice should be made here. In practice, RED-type mechanisms 
may use moving averages of the queue size instead of its instantaneous value. 
Thus, the way~$d_n$ introduced above is a~simplified way of thinking.
Yet, this trade-off is important because it allows to keep the mathematical 
models of RED-type AQM tractable.
Luckily, as noticed in~\cite[Section II.C]{Bonald}, such approximation may not
lead to significant bias, when the weight of the moving average scheme is small
(which is claimed to be the case sometimes in practice).

The renovation mechanism, which is implemented in a~system with Tail Drop,
works as follows. Define $N+1$ numbers, say, $q_i\ge 0$,
$0 \le i \le N$, satisfying \mbox{$\sum\nolimits_{i=0}^N q_i=1$}.
If upon service completion there are $i$, $1 \le i \le N$, customers
waiting in the queue, then the served customer leaves the system and
\begin{itemize}
\item with probability $q_0+Q_i$ nothing else happens, where 
$Q_i=q_i+q_{i+1}+\dots+ q_N$; and
\item with probability $q_j$, $0<j<i$, exactly $j$ customers
from the queue leave the system and those customers 
are chosen successively \textit{starting from the head of the queue}.
\end{itemize}
%\noindent
The served customer, which sees the empty queue, leaves the system.
Thus, after the renovation (if it happened), the system never becomes empty.
%what is appealing from the practical point of view. 

In the numerical section, we rank the systems with RED 
and renovation according to the stationary loss ratio, average system size,  
and average number of consecutive losses along with their standard deviations. 
The system with the Tail Drop is the standard $M/D/1/N$ queue, 
for which all these performance characteristics follow
from the classical results in queueing theory (see, for example,~\cite{Riordan1962}).
Analytic results for the systems of $M/G/1/N$ type with relatively 
arbitrary dropping functions are given in~\cite{Chyd}.
Yet, for the system with renovation, we need to derive these 
performance characteristics anew, since 
the available results in~\cite{Zaryadov2010,Zaryadov2009} 
are not valid for the renovation mechanism introduced above.
We briefly sketch the derivations in the next
section and omit most of the details
since they are based on the methodology, 
developed in~\cite{Zaryadov2010,Zaryadov2009},
and reviewed in~\cite{arxivRK}.

%Note that the above mentioned performance characteristics 
%do not depend on the order in which the customers
%are removed from the system; yet in the derivations we assume that the customers
%are chosen successively \textit{starting from the head of the queue}.

%The analytic results and parameters' values for 
%RED and REM are due to \cite{Chyd}.

\section{Performance Characteristics}

\noindent
Consider the $M/D/1/N$ queue with the renovation mechanism
introduced above. Since a~customer is served for constant service time
$d$, then for the cumulative distribution function $B(x)$ 
of its service time, one has: 
$$
B(x)=
\begin{cases}
0 & \mbox{if } x \le d\,;\\
1 & \mbox{if } x>d\,.
\end{cases}
$$
Let $N(t)$ be the total number of customers %\footnote{We assume that the system 
%starts empty, i.\,e., $N(0)=0$.} 
in the system at instant $t$ 
and $E(t)$ be the elapsed service time of the customer in server
(if there is one). 
In order to compute the stationary system size moments, 
one needs to know the stationary distribution:
\begin{equation*}
%\label{pn}
P_n=\lim\limits_{t \rightarrow \infty} \mathbf{P}\{ N(t)=n \},\enskip  0 \le n \le N+1\,.
\end{equation*} 
For the computation of the loss ratio,
due to the \mbox{PASTA} (Poisson Arrivals See Time Averages) 
property, it is sufficient to know
 the stationary probability densities
$p_n(x)=P'_n(x)$ where
\begin{multline*}
%\label{pnx}
P_n(x)=\lim\limits_{t \rightarrow \infty} \mathbf{P}\{ N(t)=n, E(t)<x \}, \\ 
1\le n \le N, \ x \in [0,d]\,.
\end{multline*}
Since we are dealing with the finite-capacity queue 
and work conserving service discipline, the
introduced stationary distributions exist.  
The probabilities~$P_n$ and 
the densities~$p_n(x)$ can be computed as follows. 
Let~$t_n$ denotes the $n$th service completion epoch 
and $N_n=N(t_n+0)$ denotes the total number of customers in the system. 
Clearly, $\{ N_n, \ n \ge 1\}$ is the finite-state Markov chain.
The entries of the transition probability matrix $\mathbb{P}=(p_{ij})$
of this chain have the form:
$$
p_{0j}=p_{1j}=
\begin{cases}
\beta_0, & \hspace*{-20mm}j=0;\\
\beta_j Q_j + \displaystyle\sum\limits_{k=j}^N \beta_k q_{k-j} +  B_N q_{N-j}, &\\
&\hspace*{-20mm} 1 \le j \le N-1\,;\\
(q_0 + q_N) B_{N-1}, & \hspace*{-20mm}j=N\,;
\end{cases}
$$
\begin{multline*}
\!\!p_{ij}=
\begin{cases}
0, & \hspace*{-38mm}j=0;\\
\sum\limits_{k=j-1}^{N-1} \beta_k q_{k-j+1} +  B_{N-1} q_{N-j}, & \\
& \hspace*{-38mm}1 \le j \le i-2\,;\\
\beta_{j-i+1} Q_j + \displaystyle\!\sum\limits_{k=j-1}^{N-1}\!\! \beta_k q_{k-j+1} + 
 B_{N-1} q_{N-j}, &\\
 &\hspace*{-38mm} i-1 \le j \le N-1;\\
(q_{0} +  q_{N})B_{N-i} , &\hspace*{-38mm} j=N\,,
\end{cases}
\\
 2 \le i \le N\,.
\end{multline*}
Here, $B_0=1-\beta_0$; $B_k=B_{k-1}-\beta_k$; and 
$\beta_k=[{(\lambda d)^k / k!}]e^{-\lambda d}$.
The matrix $\mathbb{P}$ does not have any special structure. 
So, the stationary distribution $\{P^+_n, \ 0 \le n\linebreak \le N\}$
may be found in a~straightforward manner by solving the system of linear algebraic 
equations 
$$
{\vec P}^+={\vec P}^+ \mathbb{P};\enskip 
{\vec P}^+ {\vec 1} =1
$$ 
where ${\vec P}^+= (P^+_0,\dots,P^+_N)$ and $\vec 1$ is the vector of ones. 
{\looseness=1

}

Once the probabilities $P^+_n$ are found,
the stationary distribution \mbox{$\{P_n, \ 0 \le n \le N+1\}$} 
is computed from the relation\footnote{This follows
from the well-known results for the Markov regenerative processes (see, for 
example,~\cite[Theorem 9.19]{kulk}).} 
$$
P_n=\sum\limits_{i=0}^N P^+_i \fr{f_{in}}{f^*}
$$
where $f_{in}$ is the average time during which there were $n$ customers in the system
provided that the system started with~$i$ customers in it; 
and~$f^*$ is the mean time between transitions of the embedded
Markov chain $\{ N_n, \ n \ge 1\}$.
{\looseness=1

}


Finally,the stationary probability densities $p_n(x)\linebreak =P'_n(x)$
can be computed using the fact that the relation for~$p_n(x)$ 
coincides with the relation for $p_n(x)$ in 
the standard $M/D/1/N$ queue.
Thus, $p_n(x)$ are given by (see, for example,~\cite[p.~72]{Riordan1962})

\noindent
\begin{multline}
\label{eq3}
p_n(x)=e^{-\lambda x} \left (1-B(x) \right ) 
\sum\limits_{k=0}^{n-1} p_{n-k}(0) 
\fr{(\lambda x)^k}{k!}\,, \\
1 \le n \le N\,,  \ x \in [0,d]\,.
\end{multline}
Even though~(\ref{eq3}) holds,
due to the presence of renovation, the boundary conditions $p_{n}(0)$ 
for the considered queue do not coincide 
with boundary conditions $p_{n}(0)$ for the standard $M/D/1/N$ queue.
By integration~(\ref{eq3}) from~0 to~$d$, one gets 
the following relation between~$p_{n}(0)$ and $P_n=\int\nolimits_0^d p_n(x) dx$: 
\begin{multline}
\label{eq3nn}
p_n(0)
= \fr{1}{B_0} \left (\lambda P_n- \sum\limits_{k=1}^{n-1} B_k p_{n-k}(0)
\right )\,, \\ 
1 \le n \le N\,.
\end{multline}
Since the stationary distribution \mbox{$\{P_n, \ 0 \le n \le N+1\}$} 
is already known, the values of $p_n(0)$ are computed recursively from~(\ref{eq3nn}).
The closed-form expressions for
 the average and the standard deviation of the system size are, in the most cases,
 not available and thus, they can be computed,
respectively, by $\sum\nolimits_{n=0}^{N+1} nP_n$ and  
$\sqrt{\sum\nolimits_{n=0}^{N+1} n^2P_n-(\sum_{n=0}^{N+1} nP_n)^2}$.

The computation of the loss ratio~$\pi$, i.\,e., the probability that the arriving customer is lost, 
is more involved. This is due to the fact that the accepted customer
may be lost either after the first service completion or the second, etc.
and the chance to be lost varies, depending on the number of
new customers that have arrived between successive service completions.
 
Let us introduce two quantities:
\begin{enumerate}[(1)]
\item $\gamma_{ij}$, $1 \le i \le N$, $j \ge 0$,~--- probability that the arriving customer
finds~$i$~customers in the system and until the next
service completion, exactly $j$ new customers arrive 
at the system; and
\item
$r_{ij}$, $0\le j \le N-1$, $0 \le i \le N-j-1$,~--- probability that the 
tagged customer waiting in the queue
\textit{will not} be served (i.\,e., will be lost), if there are~$j$~customers
 in front of it in the queue (excluding the one in server)
and~$i$ behind.
\end{enumerate}

Given that $\gamma_{ij}$ and $r_{ij}$ are known, the loss ratio~$\pi$  
can be computed as
\begin{multline*}
\pi =
P_{N+1}
 + 
\sum\limits_{i=1}^{N} \left [
\sum\limits_{j=0}^{N-i}
\gamma_{ij} \left ( \sum\limits_{k=0}^{i-2} q_k r_{j,i-2-k}
 +{}\right.\right.\\
\left. {}+ \sum\limits_{k=i}^{i+j-1} q_k  + Q_{j+i} r_{j,i-2} \right )
 + {}
\end{multline*}

\noindent
\begin{multline*}
{}+ \sum\limits_{j=N-i+1}^{\infty} \gamma_{ij} \left (
\sum\limits_{k=0}^{i-2} q_k r_{N-i,i-2-k}  + {}\right.\\
\left.\left.{}+\sum\limits_{k=i}^{N-1}
q_k  + Q_{N} r_{N-i,i-2} \right )
\right ]\,.
%\label{ploss2}
\end{multline*}

Due to the PASTA property of Poisson arrivals,
the expression for $\gamma_{ij}$ follows 
from the law of total probability:
\begin{multline*}
\gamma_{ij}
= \int\limits_0^d p_{i}(x) \fr{(\lambda (d- x))^j }{j!}\, e^{-\lambda (d-x)}\, dx\,,
\\
 1 \le i \le N\,, \ j \ge 0\,.
\end{multline*}
Again, by applying the law of total probability,
one gets the relations for the recursive computation of~$r_{ij}$, 
$0\le j \le N-1$, $0 \le i \le N-j-1$:
\begin{align*}
r_{i0}&= \sum\limits_{m=0}^{N-i-1}
\beta_m 
\sum\limits_{k=1}^{m+i} q_k +
B_{N-i-1} \sum\limits_{k=1}^{N-1} q_k\,;\\
r_{ij}&= \sum\limits_{m=i}^{N-1-j} \beta_{m-i} \left (
\vphantom{\sum\limits_{k=j+1}^{m+j} q_k + Q_{j+m+1} r_{m,j-1}}
\sum\limits_{k=0}^{j-1} q_k r_{m,j-1-k} +{}\right.\\
&\left.{}+
\sum\limits_{k=j+1}^{m+j} q_k + Q_{j+m+1} r_{m,j-1}
\right ) +{}
\\
&{}+ B_{N-j-i-1}
\left ( \vphantom{\sum\limits_{k=j+1}^{m+j} q_k + Q_{j+m+1} r_{m,j-1}}
\sum\limits_{k=0}^{j-1} q_k r_{N-j-1,j-1-k} +{}\right.\\
&\hspace*{16mm}\left.{}+\sum\limits_{k=j+1}^{N-1} q_k +Q_{N} r_{N-j-1,j-1}
\right ).
\end{align*}
The expressions above can be further simplified\footnote{There
are no principal difficulties in generalizing
the results for the $\mathrm{BMAP}/G/1/N$ queue.
Yet, this would obscure the goal of the paper and thus, 
we remain with the simple model.} by computing 
the integrals explicitly, but we do not dwell on it here.
For small and moderate values of~$d$, $N$, and~$\lambda$,
they can be directly used for numerical implementation.
In the numerical section, precisely these expressions are used to calculate
the loss ratio. The expressions for the average and the standard
deviation of consecutive losses are much harder to derive
and we leave it for a~separate study. The values of these 
two parameters were taken from the Monte-Carlo simulation. 

\section{Numerical Example}

\noindent
As the reference point, we have chosen the numerical results in~\cite{Chyd}
which are based on the analytic expressions and which show the 
performance characteristics 
of the $M/D/1/N$ queue with four different AQM mechanisms. 
Since RED scheme is one of the best among the four,
our goal here is to rank the $M/D/1/N$ queue with RED from~\cite{Chyd}
and the $M/D/1/N$ queue with renovation. Comparison is made 
according to the stationary loss ratio, average system size,  
and average number of consecutive losses along with 
their standard deviations.

The initial conditions are: 
the maximum queue size is $N=9$ and the service time is $d=1$. 
Thus, the offered load is $\rho=\lambda d$. 
The RED dropping function is given by (see~\cite[Eq. (59)]{Chyd}):
\begin{equation}
\label{df}
d_n=
\begin{cases}
0\,, & n\le 3\,;\\
0.11917n - 0.35752\,, & 4 \le n \le 9\,;\\
1\,, & n=10.
\end{cases}
\end{equation}
The performance of the $M/D/1/N$ queue with this RED dropping function 
is given in~\cite[Tables 1, 3, and~4]{Chyd}.
In order to find out whether there exists a~renovation 
mechanism under which the $M/D/1/N$ queue can perform at least as good as
under RED, one needs to perform exhaustive search over
the possible values of the renovation parameters $\{q_i, \ 0 \le i \le N\}$.
Since we are unaware of any analytic way of choosing these values,
adaptive search algorithms for partially observable
Markov decision processes from~\cite{kono1} were used instead.
Meta-heuristics (like particle swarm optimization), which are also applicable here,
were not used.

In Tables~1 and~2, one can see the numerical results for the four different
cases of the offered load\footnote{For the sake of reproducibility 
of the results 
presented in the paper, we also report the obtained values of the renovation 
probabilities: for $\rho=0.5$,
${\vec q}=(0.2544,0.0037,0.0053,0.0065,0.0122,0.0352,0.1108,0.1898,0.2129,0.1691)$;
for $\rho=1$, $q_0=0.0551$, $q_6=0.051$, $q_7=0.7166$, $q_8=0.0917$, and
$q_9=0.0856$;
for $\rho=2$, $q_0=0.1078$, $q_1=0.6374$, $q_4=0.0042$, $q_6=0.0084$, and
$q_9=0.2422$; and
for $\rho=3$, $q_1=0.4608$, and $q_2=0.5392$.}~$\rho$: $\rho=0.5$~--- underloaded system;
$\rho=1$~--- critically loaded system;
and $\rho=2$ and~3~--- overloaded system. The values displayed are the 
values from the numerical experiments rounded to three decimal digits.

%\noindent and in each case compute the stationary 
%loss ratio, average system content  
%and average number of consecutive losses along with
%their standard deviations. 


\begin{table*}\small
\begin{center}
\parbox{400pt}{\Caption{Performance characteristics of the $M/D/1/9$ system with the RED 
mechanism~(\ref{df}) and the $M/D/1/9$ the renovation mechanism (ren.)\
under the offered load $\rho=0.5$ and $\rho=1$}
}
%\label{my-label}
\vspace*{2ex}

\tabcolsep=8pt
\begin{tabular}{cc|c|c|c||c|c|c|}
\cline{3-8}
                                        &  & \multicolumn{3}{c||}{$\rho=0.5$} & \multicolumn{3}{c|}{$\rho=1$} \\ \cline{3-8} 
                                        &  &   Tail Drop      & RED     &    ren.   &    Tail Drop   &      RED &    ren.   \\ \hline
\multicolumn{2}{|c|}{loss ratio}    &   0    &   0.002        &   0.002  &    0.051   &    0.091        &   0.104    \\ \hline
\multicolumn{1}{|c|}{\textbf{system}} & average &   0.750    &   0.741        &    0.744    &    5.064   &      3.000       &   2.999  \\ \cline{2-8} 
\multicolumn{1}{|c|}{\textbf{size}} & standard deviation &   0.946    &    0.920      &    0.935      &    2.897   &   1.887       &   2.091      \\ \hline
\multicolumn{1}{|c|}{\textbf{consecutive}} &average  &  1.152     &  1.053        &   1.800     &    1.359   &    1.239       &  6.876         \\ \cline{2-8} 
\multicolumn{1}{|c|}{\textbf{losses}} & standard deviation &  0.403     &   0.240       &    1.260     &    0.647   &       0.561        &   0.852     \\ \hline
\end{tabular}
\end{center}
\vspace*{-6pt}
\end{table*}




\begin{table*}\small %tabl2
\begin{center}
\parbox{400pt}{\Caption{Performance characteristics of the $M/D/1/9$ system with the RED 
mechanism~(\ref{df}) and the $M/D/1/9$ the renovation mechanism (ren.)\
under the offered load $\rho=2$ and $\rho=3$}
}
%\label{my-label}
\vspace*{2ex}

\tabcolsep=8pt
\begin{tabular}{cc|c|c|c||c|c|c|}
\cline{3-8}
                                        &  & \multicolumn{3}{c||}{$\rho=2$} & \multicolumn{3}{c|}{$\rho=3$} \\ \cline{3-8} 
                                        &  &   Tail Drop      & RED     &    ren.   &    Tail Drop   &      RED &    ren.   \\ \hline
\multicolumn{2}{|c|}{loss ratio}    &   0.500    &    0.500        &   0.502     &    0.667   &    0.667       &     0.667      \\ \hline
\multicolumn{1}{|c|}{\textbf{system}} & average &   9.372    &    7.146       &   7.142      &  9.646     &      8.390        &    7.114  \\ \cline{2-8} 
\multicolumn{1}{|c|}{\textbf{size}} & standard deviation &   0.744    &     1.436       &  2.387      &    0.523   &     1.090        &    2.246  \\ \hline
\multicolumn{1}{|c|}{\textbf{consecutive}} &average  &  1.884     &   1.996         &  1.592       &     2.542  &     2.876        &   2.141    \\ \cline{2-8} 
\multicolumn{1}{|c|}{\textbf{losses}} & standard deviation &   1.069    &    1.366        &  1.100      &    1.454   &       2.064       &  1.236        \\ \hline
\end{tabular}
\end{center}
\end{table*}


Data is the tables show that with respect to the loss ratio,
renovation can perform as good as RED in the wide range of the offered load~$\rho$.
The only exception is the case $\rho=1$: here, renovation can keep
only the average system size at the same level as RED; other four 
performance characteristics are worse. 

If we fix the loss ratio, then the renovation mechanism
can guarantee at least the same value of the average system size as guaranteed by RED.
It is worth noticing that as the offered load increases, the average system size 
under the renovation mechanism becomes smaller than the average system size under RED.
Yet, renovation keeps the queue less stable than RED:
the standard deviation of system size is always smaller for RED.

If we fix the loss ratio and the average system size,
then the renovation mechanism spreads out the losses 
worse than RED when the system is underloaded and 
better than RED when it is overloaded.
This can be seen from the values of the averages and
standard deviations in the last two rows of Tables~1 and~2.

\vspace*{-6pt}

\section{Concluding Remarks}

\noindent
Even though the idea behind the renovation-type AQMs is completely 
different from the idea behind RED-type AQMs, 
renovation-type AQMs may allow one to achieve in some cases at least 
the same system performance level as guaranteed by RED-type AQMs. 
Although the comparison presented here is based only on a~single RED dropping 
function~(\ref{df}), 
our numerical experiments show that the results remain 
qualitatively the same for RED-type AQMs with other dropping functions.
Being defined by~$N$~parameters, the renovation mechanism is very flexible
and this constitutes its strength and weakness.
By varying the values of the renovation probabilities~$q_i$,
it is possible to carry out conditional optimization,
but good searching procedures are required here.

Implementation of the renovation as a~packet dropping mechanism
requires \textit{a~priori} tuning and/or operational configuration of its parameters.
Thus, whether it is appropriate to use renovation as a~packet dropping mechanism 
or not in practice heavily depends on the use case.
Although the tuning of the renovation parameters~$q_i$ can be made on the 
fly during operation, with respect to the recommendations of the RFC~7567~\cite{RFC7567},
renovation mechanism is not the proper choice for the network congestion control 
unless simple recommendations on how to set up the renovation parameters are given.
We believe this can be done based on more deep and insightful numerical experiments.

There remain a~large number of unresolved issues 
related to the renovation mechanism 
(e.\,g., can renovation ensure fairness among competing flows?
may the average queue size instead of its instantaneous value
increase the efficiency of the renovation mechanism?)
and this motivates its further analysis. 
Furthermore, evaluation of the renovation mechanism with parameters 
adapted to a~realistic router/switch use case
and/or evaluation which includes TCP feedback loops 
of several flows remains an open issue as well.

\vspace*{-6pt}


\Ack
  \noindent
   The reported study was partially funded by the Russian Foundation for 
Basic Research according to the research project No.\,18-07-00692.

The authors would like to thank the anonymous referees for their valuable comments 
which helped to improve the paper.
  
 \renewcommand{\bibname}{\protect\rmfamily References}

%\vspace*{-6pt}

\vspace*{-6pt}

{\small\frenchspacing
{\baselineskip=10.65pt
\begin{thebibliography}{99}
\bibitem{RFC7567} %1
\Aue{Baker, F., and G.~Fairhurst.} 2015.
IETF recommendations regarding active queue management.
Available at: {\sf https://tools.ietf.org/html/7567} (accessed October~4, 2018).



\bibitem{Adams} %2
\Aue{Adams, R.} 2013. 
Active queue management: A~survey. \textit{IEEE Commun. Surv.
Tut.} 15(3):1425--1476.

\bibitem{Bonald} %3
\Aue{Bonald, T., M.~May, and J.\,C.~Bolot.}
2000. Analytic evaluation of RED performance. 
\textit{IEEE Conference on Computer Communications Proceedings}
3:1415--1424.

\bibitem{hao} %4
\Aue{Hao, W., and Y.~Wei.} 2005.
An extended $GI^X/M/1/N$ queueing
model for evaluating the performance of AQM algorithms
with aggregate traffic.
\textit{Networking and mobile computing}.
Eds.\ Xicheng Lu and Wei Zhao.
{Lecture notes in computer science ser.} Springer. 3619:395--404.

\bibitem{Chyd} %5
\Aue{Chydzi$\acute{\mbox{n}}$nski, A., and L.~Chr$\acute{\mbox{o}}$st.} 
2011. Analysis of AQM queues with queue size based packet
dropping. \textit{Int. J.~Appl. Math. Comp.} 21(3):567--577.

\bibitem{Chyd2} %6
\Aue{Chydzi$\acute{\mbox{n}}$nski, A., and P.~Mrozowski.} 2016. 
Queues with dropping functions and general arrival
processes. \textit{PLoS ONE} 11(3):e0150702. Available at: 
{\sf https://\linebreak journals.plos.org/plosone/article?id=10.1371/journal.\linebreak pone.0150702} 
(accessed October~4, 2018).

\bibitem{oleg} %7
\Aue{Tikhonenko, O., and W.~Kempa.} 2016. Performance evaluation of 
an $M/G/n$-type queue
with bounded capacity and packet dropping. \textit{Int. J.~Appl.
Math. Comp.} 26(4):841--854.




\bibitem{konnew} %8
\Aue{Konovalov, M.\,G., and R.\,V.~Razumchik.} 2018. 
Numerical analysis of improved access restriction algorithms in a~$GI/G/1/N$
system. \textit{J.~Commun. Technol. El.} 63(6):616--625. 


\bibitem{Kreinin} %9
\Aue{Kreinin, A.\,Y.} 1997.
Queueing systems with renovation. 
\textit{J.~Appl. Math. Stochastic Analysis} 10(4):431--441.


\bibitem{zarN2} %10
\Aue{Zaryadov, I.\,S.} 2009. Queueing systems with general renovation.
\textit{Conference (International)
on Ultra Modern Telecommunications Proceedings}. 1--4.

\bibitem{Zaryadov2009} %11
\Aue{Zaryadov, I.\,S., and A.\,V.~Pechinkin.} 2009.
Stationary time characteristics of the ${GI/M/n/\infty}$
system with some variants of the generalized renovation discipline. \textit{Automat.
Rem. Contr.} 70(12):2085--2097.

\bibitem{Zaryadov2010} %12
\Aue{Zaryadov, I.\,S.}
2010. The ${GI/M/n/\infty}$ queuing system with generalized renovation.
\textit{Automat. Rem. Contr.} 71(4):663--671.

\bibitem{zarN1} %13
\Aue{Korolkova, A., and I.~Zaryadov.} 2010.
The mathematical model of the traffic transfer process with a~rate adjustable by {RED}.
\textit{Conference (International) on Ultra Modern Telecommunications Proceedings}. 
1046--1050.

\bibitem{RFC8289} %14
\Aue{Nichols, K., V.~Jacobson, A.~McGregor, and J.~Iyengar.} 2018.
Controlled delay active queue management.
Available at: {\sf https://datatracker.ietf.org/doc/rfc8289} (accessed October~4, 2018).

\bibitem{Riordan1962} %15
\Aue{Riordan, J.} 1962. \textit{Stochastic service systems}. 
SIAM ser. in applied mathematics. New York, NY: Wiley. 139~p.

\bibitem{arxivRK}  %16
\Aue{Konovalov, M.,  and R.~Razumchik.} 2017.
Queueing systems with renovation vs.\ queues with RED. Supplementary material. 
\textit{ArXiv e-prints}. Available at: {\sf https://arxiv.\linebreak org/abs/1709.01477/}
(accessed October~4, 2018).

\bibitem{kulk} %17
\Aue{Kulkarni, V.\,G.} 2016. \textit{Modeling and analysis of stochastic systems}. 
3rd ed. Chapman \&~Hall/CRC texts in statistical science ser.
Chapman \&~Hall/CRC. 606~p.

\bibitem{kono1} %18
\Aue{Konovalov, M.\,G.} 2007.
\textit{Metody adaptivnoy obrabotki informatsii i~ikh prilozheniya}
[Methods of adaptive information processing and their applications]. 
Moscow: Institute of Informatics Problems of RAS. 212~p.
\end{thebibliography} } }

\end{multicols}

\vspace*{-6pt}

\hfill{\small\textit{Received October 9, 2018}}

\vspace*{-18pt}
  

 \Contr

\noindent
\textbf{Konovalov Mikhail G.} (b.\ 1950)~--- 
Doctor of Science in technology, principal scientist, Institute of Informatics
Problems, Federal Research Center ``Computer Science and Control'' 
of the Russian Academy of Sciences, 44-2~Vavilov Str., Moscow 119333, 
Russian Federation; \mbox{mkonovalov@ipiran.ru}

\vspace*{3pt}


\noindent
\textbf{Razumchik Rostislav V.} (b.\ 1984)~--- 
Candidate of Science (PhD) in physics and mathematics, leading scientist,
Institute of Informatics Problems, Federal Research Center ``Computer 
Science and Control'' of the Russian
Academy of Sciences, 44-2~Vavilov Str., Moscow 119333, Russian Federation; 
associate professor, Peoples'
Friendship University of Russia (RUDN University), 
6~Miklukho-Maklaya Str., Moscow 117198, Russian
Federation; \mbox{rrazumchik@ipiran.ru} %; \mbox{razumchik\_rv@rudn.ru}

\vspace*{6pt}

\hrule

\vspace*{2pt}

\hrule

\vspace*{-2pt}

%\newpage

%\vspace*{-24pt}

\def\tit{СРАВНЕНИЕ ДВУХ МЕХАНИЗМОВ АКТИВНОГО УПРАВЛЕНИЯ ОЧЕРЕДЬЮ В~СИСТЕМЕ $M/D/1/N$$^*$}

\def\titkol{Сравнение двух механизмов активного управления очередью в~системе $M/D/1/N$}

\def\aut{М.\,Г.~Коновалов$^1$, Р.\,В.~Разумчик$^{1,2}$}

\def\autkol{М.\,Г.~Коновалов, Р.\,В.~Разумчик}

{\renewcommand{\thefootnote}{\fnsymbol{footnote}} \footnotetext[1]
{Исследование выполнено при частичной финансовой поддержке РФФИ (проект 18-07-00692).}}



\titel{\tit}{\aut}{\autkol}{\titkol}

\vspace*{-11pt}

\noindent
$^1$Институт проблем информатики Федерального исследовательского 
центра <<Информатика и управление>>\linebreak
$\hphantom{^1}$Российской академии наук

\noindent
$^2$Российский университет дружбы народов 

\vspace*{5pt}

\def\leftfootline{\small{\textbf{\thepage}
\hfill ИНФОРМАТИКА И ЕЁ ПРИМЕНЕНИЯ\ \ \ том\ 12\ \ \ выпуск\ 4\ \ \ 2018}
}%
 \def\rightfootline{\small{ИНФОРМАТИКА И ЕЁ ПРИМЕНЕНИЯ\ \ \ том\ 12\ \ \ выпуск\ 4\ \ \ 2018
\hfill \textbf{\thepage}}}

\vspace*{-3pt}


\Abst{Представлены некоторые результаты численных экспериментов, подтверждающие
следующее обстоятельство: параметры механизма обобщенного обновления
могут быть подобраны таким образом,\linebreak\vspace*{-12pt}}

\Abstend{что уровень производительности
однолинейных сис\-тем массового обслуживания с обобщенным обновлением
может быть не ниже уровня производительности
систем с RED-по\-доб\-ны\-ми механизмами активного управ\-ле\-ния очередями.
Механизмы сравниваются на примере сис\-те\-мы $M/D/1/N$
по стационарным значениям сле\-ду\-ющих характеристик:
вероятность потери заявки, среднее число заявок в сис\-те\-ме,
среднее чис\-ло последовательных потерь заявок 
и~их средние квадратические отклонения.
Расчеты основаны на известных фактах,
а~также на ряде новых аналитических результатов для систем
с~обобщенным обновлением, полученных в данной работе.}

\KW{система массового обслуживания; 
алгоритмы активного управления очередями; обобщенное обновление}

\DOI{10.14357/19922264180402}



%\vspace*{-3pt}


 \begin{multicols}{2}

\renewcommand{\bibname}{\protect\rmfamily Литература}
%\renewcommand{\bibname}{\large\protect\rm References}

{\small\frenchspacing
{%\baselineskip=10.8pt
\begin{thebibliography}{99}
%\vspace*{-3pt}

\bibitem{RFC7567-1} %1
\Au{Baker F., Fairhurst~G.}
IETF recommendations regarding active queue management, 2015.
{\sf https://tools.\linebreak ietf.org/html/7567}.



\bibitem{Adams-1} %2
\Au{Adams R.}
Active queue management: A~survey~// 
{IEEE Commun. Surv. Tut.}, 2013. Vol.~15. No.\,3. P.~1425--1476.

\bibitem{Bonald-1} %3
\Au{Bonald T., May M., Bolot~J.\,C.} Analytic evaluation of RED performance~//
{IEEE Conference on Computer Communications Proceedings}, 2000. 
Vol.~3. P.~1415--1424.

\bibitem{hao-1} %4
\Au{Hao W., Wei~Y.}
An extended $GI^X/M/1/N$ queueing
model for evaluating the performance of AQM algorithms
with aggregate traffic~// Networking and mobile computing~/
Eds. Xicheng Lu and Wei Zhao.~---
Lecture notes in computer science ser.~--- Springer, 2005. Vol.~3619. P.~395--404.

\bibitem{Chyd-1} %5
\Au{Chydzi$\acute{\mbox{n}}$ski A., Chr$\acute{\mbox{o}}$st~L.} 
Analysis of AQM queues with queue size based packet
dropping~// Int. J.~Appl. Math. Comp., 2011. Vol.~21. No.\,3. P.~567--577.

\bibitem{Chyd2-1} %6
\Au{Chydzi$\acute{\mbox{n}}$ski A.,  Mrozowski~P.}
 Queues with dropping functions and general arrival
processes~// PLoS ONE, 2016. Vol.~11. No.\,3. 
{\sf https://journals.plos.org/plosone/\linebreak article?id=10.1371/journal.pone.0150702}.

\bibitem{oleg-1} %7
\Au{Tikhonenko O., Kempa~W.} Performance evaluation of an $M/G/n$-type queue
with bounded capacity and packet dropping~// {Int. J.~Appl.
Math. Comp.}, 2016. Vol.~26. No.~4. P.~841--854.



\bibitem{konnew-1} %8
\Au{Konovalov M.\,G., Razumchik~R.\,V.}
Numerical analysis of improved access restriction algorithms in a~$GI/G/1/N$
system // {J.~Commun. Technol. El.}, 2018. Vol.~63. No.\,6. P.~616--625.

\bibitem{Kreinin-1} %9
\Au{Kreinin A.\,Y.}
Queueing systems with renovation //
{J.~Appl. Math. Stochastic Analysis}, 1997. Vol.~10. No.~4. P.~431--441.

\bibitem{zarN2-1} %10
\Au{Zaryadov~I.\,S.} Queueing systems with general renovation~//
{Conference (International) on Ultra Modern Telecommunications Proceedings}, 2009.
P.~1--4.

\bibitem{Zaryadov2009-1} %11
\Au{Зарядов И.\,С.,  Печинкин~А.\,В.}
Стационарные временные характеристики системы ${GI/M/n/\infty}$
с~некоторыми вариантами дисциплины обобщенного об\-нов\-ле\-ния~//
{Автоматика и~телемеханика}, 2009. Вып.~12. С.~161--174.

\bibitem{Zaryadov2010-1} %12
\Au{Зарядов И.\,С.} 
Система массового обслуживания $GI/M/n/\infty$ с~обобщенным об\-нов\-ле\-ни\-ем~//
{Автоматика и~телемеханика}, 2010. Вып.~4. С.~130--139.

\bibitem{zarN1-1} %13
\Au{Korolkova A., Zaryadov~I.}
The mathematical model of the traffic transfer process with a~rate adjustable by {RED}~//
{Conference (International) on Ultra Modern Telecommunications Proceedings}, 2010.
P.~1046--1050.

\bibitem{RFC8289-1} %14
\Au{Nichols K., Jacobson V., McGregor A., Iyengar J.}
Controlled delay active queue management, 2018.
{\sf https:// datatracker.ietf.org/doc/rfc8289}.


\bibitem{Riordan1962-1} %15
\Au{Riordan J.} {Stochastic service systems}.~--- 
SIAM ser. in applied mathematics.~--- New York, NY, USA: Wiley, 1962. 139~p.

\bibitem{arxivRK-1} %16
\Au{Konovalov M.,  Razumchik~R.}
Queueing systems with renovation vs.\ queues with RED. Supplementary material~//
{ArXiv e-prints}, 2017. {\sf https://arxiv.\linebreak org/abs/1709.01477/}.

\bibitem{kulk-1} %17
\Au{Kulkarni V.\,G.} Modeling and analysis of stochastic systems. 3rd ed.~--- 
Chapman \&~Hall/CRC texts in statistical science ser.~---
Chapman \& Hall/CRC, 2016. 606~p.

\bibitem{kono1-1}
\Au{Коновалов М.\,Г.} 
{Методы адаптивной обработки информации и~их приложения.}~--- 
М.: ИПИ РАН, 2007. 212~с.
\end{thebibliography}
} }

\end{multicols}

 \label{end\stat}

 \vspace*{-3pt}

\hfill{\small\textit{Поступила в~редакцию  09.10.2018}}


%\renewcommand{\bibname}{\protect\rm Литература}
%\renewcommand{\figurename}{\protect\bf Рис.}
\renewcommand{\tablename}{\protect\bf Таблица}     %9
\def\stat{samouylov}

\def\tit{К МОДЕЛИРОВАНИЮ ЭФФЕКТОВ ОБСЛУЖИВАНИЯ МНОГОАДРЕСНОГО 
ТРАФИКА В~СЕТЯХ 5G NR$^*$}

\def\titkol{К моделированию эффектов обслуживания многоадресного 
трафика в~сетях 5G NR}

\def\aut{А.\,К.~Самуйлов$^1$, А.\,А.~Платонова$^2$, В.\,С.~Шоргин$^3$, 
Ю.\,В.~Гайдамака$^4$}

\def\autkol{А.\,К.~Самуйлов, А.\,А.~Платонова, В.\,С.~Шоргин, Ю.\,В.~Гайдамака}

\titel{\tit}{\aut}{\autkol}{\titkol}

\index{Самуйлов А.\,К.}
\index{Платонова А.\,А.}
\index{Шоргин В.\,С.}
\index{Гайдамака Ю.\,В.}
\index{Samouylov A.\,K.}
\index{Platonova A.\,A.}
\index{Shorgin V.\,S.} 
\index{Gaidamaka Yu.\,V.}


{\renewcommand{\thefootnote}{\fnsymbol{footnote}} \footnotetext[1]
{Исследование выполнено за счет Российского научного фонда (грант №\,21-79-00142).}}


\renewcommand{\thefootnote}{\arabic{footnote}}
\footnotetext[1]{Российский университет дружбы народов, samuylov-ak@rudn.ru}
\footnotetext[2]{Российский университет дружбы народов, platonova-aa@rudn.ru}
\footnotetext[3]{Федеральный исследовательский центр <<Информатика и~управление>> Российской академии наук, 
\mbox{vshorgin@ipiran.ru}}
\footnotetext[4]{Российский университет дружбы народов; Федеральный исследовательский 
центр <<Информатика и~управ\-ле\-ние>> Российской академии наук, 
\mbox{gaydamaka-yuv@rudn.ru}}

\vspace*{-12pt}
 
 
     
  
  \Abst{Многоадресная передача данных в~сетях беспроводного доступа поз\-во\-ля\-ет 
эффективно предостав\-лять услугу группе абонентов и~оказывается полезной для сокращения 
ресурса, необходимого для обслуживания пользователей, за\-пра\-ши\-ва\-ющих одни и~те же 
данные. Поддержка этой функции в~современной технологии 5G~New Radio (NR) и~будущих 
субтерагерцевых сис\-те\-мах~6G сталкивается с~особенностями, связанными с~использованием 
фазированных антенных решеток (ФАР), фор\-ми\-ру\-ющих на\-прав\-лен\-ные лучи. Пред\-став\-лен\-ная 
 модель обслуживания многоадресного и~одноадресного трафика поз\-во\-ля\-ет 
исследовать области значений па\-ра\-мет\-ров сети связи~5G/6G для снижения плот\-ности 
размещения базовых станций (БС) при поддержании качества предостав\-ле\-ния услуг абонентам.}
   
  \KW{5G; 6G; многоадресная передача; мил\-ли\-мет\-ро\-вые волны; терагерцевые час\-то\-ты; 
фазированные антенные решетки; технологии радиодоступа; математическое 
моделирование}

\DOI{10.14357/19922264230210}{SLMGZU} 
  
\vspace*{3pt}


\vskip 10pt plus 9pt minus 6pt

\thispagestyle{headings}

\begin{multicols}{2}

\label{st\stat}
  
\section{Введение}

  Приложения недалекого будущего, такие как дополненная (AR, augmented reality) и~виртуальная 
реальность (VR, virtual reality), телеприсутствие, видео~8/16~K, требуют резкого 
увеличения ско\-рости передачи по радиоинтерфейсу~[1]. Значительное 
увеличение ско\-рости до~20~Гбит/с на БС ожидается 
в~сис\-те\-мах 5G~NR за счет использования мил\-ли\-мет\-ро\-во\-го 
диапазона длин волн (mmWave) (с~час\-то\-та\-ми~24--52,6~ГГц), а~для сотовых 
сис\-тем~6G планируется использовать ниж\-нюю часть диапазона 
мил\-ли\-мет\-ро\-вых волн (с~час\-то\-та\-ми~52,6--100~ГГц) и~даже субтерагерцевый 
диапазон час\-тот (100--300~ГГц)~[2], потенциально повышая ско\-рость доступа 
до~100~Гбит/с на одну БС.

\begin{figure*}[b] %fig1
\vspace*{-6pt}
\begin{center}
   \mbox{%
\epsfxsize=106.848mm 
\epsfbox{sam-1.eps}
}

\end{center}
\vspace*{-9pt}
\Caption{Схема СМО для модели совместного обслуживания многоадресного 
и~одноадресного трафика в~сетях 5G/6G}
\end{figure*}
  
  Многоадресная передача является важ\-ной возможностью сетей 
беспроводного доступа, по\-вы\-ша\-ющей эф\-фек\-тив\-ность использования ресурсов 
при обслуживании абонентов сети, за\-пра\-ши\-ва\-ющих одни и~те же данные~[3]. 
Если для предоставления услуги одноадресной передачи необходимо 
организовать отдельную сессию для каждого абонента, то для одновременного 
предоставления услуги многоадресной передачи нескольким абонентам из 
группы многоадресной до\-став\-ки информации достаточно организовать одну 
многоадресную сессию. Под сессией понимается процесс непрерывной 
передачи данных от БС, ини\-ци\-иро\-ван\-ный запросом абонента на 
предостав\-ле\-ние со\-от\-вет\-ст\-ву\-ющей услуги и~завершающийся освобождением 
выделенного на БС радиоресурса~[4]. По сравнению с~проводными сетями 
в~беспроводных сис\-те\-мах одновременная передача данных нескольким 
абонентским устройствам (АУ) за\-труд\-не\-на не только из-за от\-ли\-ча\-ющих\-ся 
условий распространения радиосигнала до разных АУ, но и~вследствие 
использования ФАР, фор\-ми\-ру\-ющих 
несколько вы\-со\-ко\-на\-прав\-лен\-ных лучей, что приводит к~не\-воз\-мож\-ности 
обслуживания всех АУ из группы многоадресной до\-став\-ки 
одним лучом. Преимущество вы\-со\-ко\-на\-прав\-лен\-ных лучей при этом заключается 
в~увеличении радиуса по\-кры\-ва\-емо\-го лучом сек\-то\-ра, что поз\-во\-ля\-ет размещать 
БС на большем расстоянии друг от друга и~снижает капительные затраты 
оператора сети. Таким образом, возникает задача исследования особенностей 
обслуживания многоадресного трафика в~сетях 5G~NR с~точ\-ки зрения баланса 
для оператора сети между эф\-фек\-тив\-ностью использования радиоресурсов, 
качеством предостав\-ле\-ния услуг и~за\-тра\-та\-ми на развертывание сети.

\vspace*{-6pt}
  
\section{Формализация модели}

  Рассмотрим сценарий с~одной БС, предоставляющей $M\hm= \vert 
\mathcal{M}\vert$ услуг, тре\-бу\-ющих многоадресную, и~$K\hm= 
\vert\mathcal{K}\vert$ услуг, тре\-бу\-ющих одноадресную доставку информации. 
Абонент формирует запрос на услугу многоадресной передачи с~ве\-ро\-ят\-ностью~$p_M$ 
  и~на услугу одноадресной передачи с~ве\-ро\-ят\-ностью~$p_U$, 
$p_M\hm+p_U\hm=1$. Назовем\linebreak $({I}, m)$-сес\-си\-ей многоадресную 
сессию класса~$m$, со\-от\-вет\-ст\-ву\-ющую услуге~$m$, и~будем считать, что 
по\-сту\-пив\-ший от абонента запрос с~ве\-ро\-ят\-ностью~$p_{M,m}$ пред\-став\-ля\-ет 
собой запрос на предостав\-ле\-ние услуги~$m$, $m\hm= 1,2,\ldots , M$. 
Аналогично $({II}, k)$-сес\-сия~--- одноадресная сессия класса~$k$, 
со\-от\-вет\-ст\-ву\-ющая услуге~$k$ и~ве\-ро\-ят\-ности~$p_{U,k}$, $k\hm= 1,2,\ldots, K$. 
При этом $\sum\nolimits^M_{m=1} p_{M,m} \hm= p_M$, 
$\sum\nolimits^K_{k=1} p_{U,k} \hm= p_U$.
  
  Пусть $\lambda_{UE}$~--- интенсивность поступления запросов на 
установление сессии от абонента, запрашивающего одну из~$M$ 
многоадресных или одну из~$K$ одноадресных услуг. Тогда $\Lambda\hm= 
\lambda_{UE}N_{UE}$~--- общая ин\-тен\-сив\-ность по\-ступ\-ле\-ния запросов 
от~$N_{UE}$ абонентов в~зоне покрытия БС. Для оценки числа~$N_{UE}$ при 
заданной плот\-ности абонентов~$\lambda_B$ делаем предположение 
о~равномерном распределении АУ в~зоне покрытия антенны БС, 
об\-слу\-жи\-ва\-ющей сектор 120$^\circ$ с~радиусом~$r$~[5]. В~этом случае 
$N_{UE}\hm= \lambda_B \pi r^2/3$, а~интенсивности по\-ступ\-ле\-ния на БС 
запросов на предостав\-ле\-ние услуги класса~$m$ и~класса~$k$ равны 
$\lambda_m\hm= p_{M,m}\Lambda$ и~$\nu_k\hm= p_{U,k}\Lambda$ 
соответственно. Кроме этого, для всех классов заданы сред\-ние длительности 
предостав\-ле\-ния услуги абоненту~$\mu_m^{-1}$ и~$\kappa_k^{-1}$, а~так\-же 
объем требуемого ресурса~$b_m$ и~$d_k$ в~первичных ресурсных блоках (РБ), причем два 
последних па\-ра\-мет\-ра, а~так\-же общее чис\-ло~$V$ доступных РБ 
на БС зависят от размера блока, спект\-раль\-ной эф\-фек\-тив\-ности, полосы 
пропускания БС и~могут быть вы\-чис\-ле\-ны, как показано в~[6].


  Описанный сценарий моделируется мультисервисной сис\-те\-мой массового 
обслуживания (СМО) с~потерями~[7], схема которой пред\-став\-ле\-на на рис.~1. Сис\-те\-ма 
поз\-во\-ля\-ет моделировать предостав\-ле\-ние двух типов услуг многоадресной 
передачи: услуг передачи хранимых данных (например, мультимедийное 
оповещение для целей общественной безопас\-ности, массовые обновления 
программного обеспечения и~про\-шив\-ки в~интернете вещей) и~услуг \mbox{передачи} 
данных в~реальном времени (например, средства массовой информации, 
развлечения, включая вещание в~дополненной и~виртуальной ре\-аль\-ности, 
создание профессионального \mbox{контента} в~беспроводной студии с~несколькими 
камерами, иммерсивное видеопроизводство в~реальном времени)~[8--10]. 
Ключевое отличие заключается в~дли\-тель\-ности сессии для предостав\-ле\-ния 
многоадресной услуги на стороне БС, т.\,е.\ \mbox{интервала} за\-ня\-тости ресурса, 
выделенного на БС для непрерывного предостав\-ле\-ния услуги абонентам из 
группы многоадресной до\-став\-ки информации. Про\-дол\-жи\-тель\-ность интервала 
за\-ня\-тости отражает\linebreak особенности многоадресной передачи хранимых данных 
и~данных в~реальном времени и~моделируется в~СМО двумя дис\-цип\-ли\-на\-ми 
<<прозрачного>> обслуживания~[11]~--- Tr$_1$ для многоадресной передачи 
хранимых данных и~Tr$_2$ для многоадресной передачи данных в~реальном 
времени. Изменение чис\-ла абонентов $\xi^{(1)}(t)$ для~Tr$_1$ и~$\xi^{(2)}(t)$ 
для~Tr$_2$, об\-слу\-жи\-ва\-емых в~течение одной многоадресной сессии, показано 
на рис.~2.




 
  
  Для обеих дисциплин запрос абонента на предостав\-ле\-ние услуги класса 
$({I}, m)$ принимается к~обслуживанию, когда при по\-ступ\-ле\-нии он не 
находит на обслуживании запросов такого же класса и~свободны не 
менее~$b_m$ РБ. В~этом случае $({I}, m)$-за\-прос ини\-ци\-иру\-ет 
интервал за\-ня\-тости ресурса запросами класса $({I}, m)$ и~занимает~$b_m$ РБ в~течение случайного интервала времени, не зависящего от 
моментов по\-ступ\-ле\-ния и~длительностей обслуживания многоадресных 
и~одноадресных запросов других классов. Все $({I}, m)$-за\-про\-сы, 
по\-сту\-пив\-шие в~течение этого интервала за\-ня\-тости, т.\,е.\ в~период, когда на 
обслуживании находится один или несколько запросов класса $({I}, 
m)$, принимают-\linebreak\vspace*{-12pt}

{ \begin{center}  %fig2
 \vspace*{-1pt}
   \mbox{%
\epsfxsize=79mm 
\epsfbox{sam-2.eps}
}

\end{center}



\noindent
{{\figurename~2}\ \ \small{Интервалы за\-ня\-тости при передаче хранимых данных (Tr$_1$)~(\textit{а}) 
и~при передаче данных в~реальном времени (Tr$_2$)~(\textit{б})
}}}

\vspace*{9pt}

\addtocounter{figure}{1}

\noindent
ся в~сис\-те\-му и~получают обслуживание без выделения 
дополнительных РБ. Для обеих дис\-цип\-лин до\-ступ в~сис\-те\-му  
$({I}, m)$-за\-про\-са блокируется, если в~сис\-те\-ме нет запросов этого 
класса, а~свободного ресурса недостаточно для начала этой сессии.




  
  Существует принципиальное различие в~моделировании момента окончания 
интервала за\-ня\-тости для дис\-цип\-лин~Tr$_1$ и~Tr$_2$. Для дис\-цип\-ли\-ны~Tr$_1$\linebreak 
все об\-слу\-жи\-ва\-ющи\-еся $({I}, m)$-за\-про\-сы покидают\linebreak сис\-те\-му 
одновременно в~момент окончания обслуживания первого запроса, 
ини\-ци\-иро\-вав\-ше\-го период за\-ня\-тости~[12]. Для дис\-цип\-лины~Tr$_2$ 
об\-слу\-жи\-ва\-ющи\-еся $({I}, m)$-за\-про\-сы покидают сис\-те\-му 
в~\mbox{произвольные} моменты, а~интервал за\-ня\-тости заканчивается, когда все 
выделенные РБ осво\-бож\-да\-ют\-ся при окончании обслуживания последнего 
$({I}, m)$-за\-про\-са, не оставившего после себя в~сис\-те\-ме запросов 
того же класса~[13]. Под ниж\-ни\-ми горизонтальными стрелками на схемах 
рис.~2 указаны сред\-ние значения дли\-тель\-ности интервала за\-ня\-тости для 
обеих дис\-цип\-лин. Здесь $\rho_m\hm= (1\hm+ \rho_{UE,m})^{p_{M,m}N_{UE}} 
\hm-1$ со\-от\-вет\-ст\-ву\-ет пред\-ла\-га\-емой нагрузке запросами класса $({I}, m)$ 
от всех АУ в~секторе покрытия антенны, при этом $\rho_{UE,m}\hm= p_{M,m} 
\lambda_{UE}/\mu_m$~--- пред\-ла\-га\-емая на\-груз\-ка запросами класса 
$({I}, m)$ от одного абонента, $m\hm\in \mathcal{M}$.


  Обслуживание запросов абонентов на предо\-став\-ле\-ние одноадресной услуги 
класса $({II}, k)$ моделируется классической СМО с~потерями~[14], для которой задана ин\-тен\-сив\-ность~$\nu_k$ 
входящего пуассоновского потока, экспоненциальное случайное время 
обслуживания запроса $\kappa_k^{-1}$ и~требование к~ресурсу для 
обслуживания запроса~$d_k$ РБ. Пред\-ла\-га\-емая на\-груз\-ка класса $({II}, k)$ 
обозначена $a_k\hm= \nu_k/\kappa_k$, $k\hm\in \mathcal{K}$.
  
\section{Метрики качества предоставления услуг}

  Функционирование СМО описывается марковским процессом (МП) 
$$
\mathbf{Z}(t)= \left(Y_1(t), \ldots , Y_M(t),N_1(t),\ldots , N_K(t)\right),\enskip 
t\geq 0\,. 
$$
Здесь $Y_m(t)$, $t\hm\geq 0$, $m\hm\in \mathcal{M}$,~--- индикатор наличия в~сис\-те\-ме 
$({I},m)$-за\-про\-сов в~момент~$t$: 
$$
Y_m(t)=\begin{cases}
1, & \mbox{если\ в~момент\ $t$\ обслуживается}\\
&\mbox{хотя\ бы\ один\ $({I}, m)$-за\-прос};\\
0 & \mbox{иначе}.
\end{cases}
$$
 Марковский процесс $N_k(t)$, 
$t\hm\geq0$, отражает чис\-ло  
$({II}, k)$-за\-про\-сов в~момент~$t$, $N_k(t)\hm\in \{ 0,1,2,\ldots , \lfloor 
V/d_k\rfloor\}$. Со\-сто\-яние МП $\mathbf{Z}(t)$ имеет вид 
$(\mathbf{y},\mathbf{n})$, а~пространство со\-сто\-яний опре\-де\-ля\-ет\-ся~как 
  \begin{multline*}
  \mathcal{Z}= \left\{
  \vphantom{\sum\limits^M_{m=1}}
  (\mathbf{y},\mathbf{n})\in \{0,1\}^M \times 
\{0,1,2,\ldots\}^K :\right.\\ 
\left.  \sum\limits^M_{m=1} b_m y_m +\sum\limits^K_{k=1} d_k n_k\leq V\right\}.
  \end{multline*}
  
  \begin{figure*}[b] %fig3
\vspace*{-3pt}
\begin{center}
   \mbox{%
\epsfxsize=160.643mm 
\epsfbox{sam-4.eps}
}

\end{center}
\vspace*{-11pt}
\Caption{Вероятность блокировки доступа в~сис\-те\-му многоадресных~(\textit{а}) 
и~одноадресных~(\textit{б}) запросов в~за\-ви\-си\-мости от ISD: \textit{1}~--- 
$\mathrm{ISD}\hm= 321$; \textit{2}~--- 446; \textit{3}~--- 620; \textit{4}~--- 
$\mathrm{ISD}\hm= 863$}
\end{figure*}
  
  В~[15] показано, что стационарное распределение МП $\mathbf{Z}(t)$ имеет 
мультипликативное пред\-став\-ление

\noindent
  \begin{equation}
  p(\mathbf{y},\mathbf{n}) =G^{-1}(\mathcal{Z}) \!\prod\limits^M_{m=1} 
\gamma_m^{y_m} \prod\limits^k_{k=1} \fr{a_k^{n_k}}{n_k!}\,, \\
  (\mathbf{y},\mathbf{n})\in \mathcal{Z}\,,
  \label{e1-sam}\!\!
  \end{equation}
  
    \vspace*{-2pt}
    
    \noindent
с нормировочной константой $G(\mathcal{Z})$, пред\-став\-лен\-ной в~виде

\vspace*{1pt}

\noindent
$$
G(\mathcal{Z}) =\sum\limits_{\mathbf{z}\in\mathcal{Z}} \prod\limits^M_{m=1} 
\gamma_m^{y_m} \prod\limits^K_{k=1} \fr{a_k^{n_k}}{n_k !}\,.
$$

  \vspace*{-1pt}

\noindent
Здесь $\gamma_m$ имеет смысл загрузки БС многоадресными 
сессиями, которая различается для услуг передачи хранимых данных и~услуг 
передачи данных в~реальном времени и~опре\-де\-ля\-ет\-ся~как 
$$
\gamma_m= \begin{cases}
\rho_m & \mbox{для}\ \mathrm{Tr}_1\,;\\
e^{\rho_m}-1 &\mbox{для}\ \mathrm{Tr}_2\,.
\end{cases}
$$

  \vspace*{-2pt}
  
  Основные характеристики качества предостав\-ле\-ния услуг мож\-но найти 
суммированием стационарных вероятностей~(\ref{e1-sam}) по со\-от\-вет\-ст\-ву\-ющим 
подмножествам~$\Omega$ пространства со\-сто\-яний~$\mathcal{Z}$:
  \begin{equation}
  {\sf P}\{ (\mathbf{y},\mathbf{n})\in \Omega\} =\!\!\!\!\sum\limits_{(\mathbf{y},\mathbf{n}) 
\in \Omega} \!\!\! p(\mathbf{y},\mathbf{n})= \fr{G(\Omega)}{G(\mathcal{Z})}\,,\enskip 
\Omega\subseteq \mathcal{Z}\,.\!\!
\label{e2-sam}
  \end{equation}
  
  \vspace*{-2pt}

  Так, множество состояний блокировки доступа в~сис\-те\-му для 
многоадресного запроса класса~$({I},m)$ определяется выражением

\vspace*{-6pt}

\noindent
  \begin{multline*}
  \hspace*{-8pt}\mathcal{B}_m^I= \!\left\{ (\mathbf{y},\mathbf{n})\in\mathcal{Z} : 
\sum\limits^M_{m=1}\! b_my_m +\!\sum\limits^K_{k=1} \!d_k n_k+b_m > V\,,\right.\\  
\left.y_m=0
\vphantom{\sum\limits^M_{m=1}}
\right\},\enskip m\in \mathcal{M}\,,
 % \label{e3-sam}
  \end{multline*}
а~для одноадресного запроса класса $({II},k)$~--- выражением

\vspace*{-2pt}

\noindent
\begin{multline*}
\hspace*{-7pt}\mathcal{B}_k^{II} =\!\left\{ (\mathbf{y},\mathbf{n}) \in\mathcal{Z} : 
\sum\limits^M_{m=1} \!b_m y_m +\sum\limits^K_{k=1}\! d_k n_k +d_k 
\!>\!V\!\right\},\hspace*{-1.89pt}\\
 k\in\mathcal{K}\,.
%\label{e4-sam}
\end{multline*}
  
  Соответствующие вероятности $B_m^I\hm= {\sf P}\{ (\mathbf{y}, 
\mathbf{n})\hm\in \mathcal{B}^I_m\}$ и~$B_k^{II} \hm= {\sf P}\{ 
(\mathbf{y},\mathbf{n})\hm\in \mathcal{B}_k^{II}\}$ можно найти  
с~по\-мощью~(\ref{e2-sam}).

\section{Результаты численного эксперимента}

\vspace*{-3pt}

  Для иллюстрации эффектов обслуживания многоадресного трафика при 
использовании час\-тот мил\-ли\-мет\-ро\-во\-го диапазона длин волн 
и~субтерагерцевого диапазона час\-тот рас\-смот\-ре\-на сота сети 5G~NR 
с~техническими па\-ра\-мет\-ра\-ми из~[16], радиус которой в~за\-ви\-си\-мости от ФАР 
(от $4\times4$ до~$32\times 4$) может варь\-и\-ро\-вать\-ся от~107 до~288~м. 
Абоненты получают одну одноадресную услугу и~одну услугу многоадресной 
передачи данных в~реальном времени, при этом для по\-след\-ней высокая 
на\-прав\-лен\-ность лучей в~технологии 5G~NR может привести к~не\-об\-хо\-ди\-мости 
поддерживать одновременно несколько многоадресных сессий, чис\-ло которых 
ограничено чис\-лом антенных элементов ФАР. 
  
  На рис.~3 приведены графики ве\-ро\-ят\-ности блокировки доступа в~сис\-те\-му 
для многоадресного и~одноадресного трафика при $p_M\hm= 0{,}5$ 
в~за\-ви\-си\-мости от плот\-ности~$\lambda_B$ абонентов в~зоне покрытия БС 
для~4~значений рас\-сто\-яния меж\-ду БС (ISD, inter-site distance). 
  

  Заметим, что увеличение ин\-тен\-сив\-ности по\-ступ\-ле\-ния запросов от 
абонентов из зоны покры-\linebreak тия сначала приводит к~увеличению ве\-ро\-ят\-ности\linebreak 
блокировки доступа в~сис\-те\-му запросов обоих классов. Однако начиная 
с~определенной ин\-тен\-сив\-ности ве\-ро\-ят\-ность блокировки многоадресных 
запросов начинает уменьшаться, как показано на рис.~3,\,\textit{а}. Объяснение 
заключается в~том, что воз\-рас\-та\-ет ве\-ро\-ят\-ность обслуживания в~сис\-те\-ме\linebreak\vspace*{-12pt}

\pagebreak

\noindent
 многоадресной сессии, которая занимает ресурс,
 в~том чис\-ле для обслуживания 
в~рамках текущей сессии всех сле\-ду\-ющих по\-сту\-па\-ющих запросов на эту 
многоадресную услугу. Вышеупомянутый эффект
 приводит к~резкому 
увеличению ве\-ро\-ят\-ности блокировки доступа для одноадресных запросов, как 
показано на рис.~3,\,\textit{б}, вплоть до момента, когда сис\-те\-ма начинает 
обслуживать почти исключительно многоадресные запросы. Подобный эффект 
обычно наблюдается, когда пред\-ла\-га\-емая нагрузка многоадресных запросов 
увеличивается или когда падает пред\-ла\-га\-емая нагрузка одноадресных запросов, 
и~усиливается для ФАР с~большим чис\-лом элементов, со\-от\-вет\-ст\-ву\-ющих 
большим значениям рас\-сто\-яния между БС соседних сот сети.
  
  Проиллюстрированная неявная приоритизация многоадресных сессий не 
всегда может быть предпочтительной для оператора сети, поскольку она может 
блокировать запросы на одноадресные услуги с~более высоким приоритетом. 
Однако фактический баланс между вероятностями блокировки доступа 
одноадресных и~многоадресных запросов существенно зависит от функции 
по\-лез\-ности сетевого оператора. Последний может обеспечить соблюдение 
необходимого баланса,  введя, например, приоритеты на этапе приема запросов 
на обслуживание.
  
\section{Заключение}

  В работе предложена математическая модель обслуживания 
  многоадресного и~одноадресного трафика на БС в~сис\-те\-мах~5G/6G при использовании 
вы\-со\-ко\-на\-прав\-лен\-ных антенных решеток, характерных для сетей 5G~NR. 
Показано, что со\-вмест\-ная передача многоадресного и~одноадресного трафика 
на радиоинтерфейсе приводит к~ряду эффектов, связанных с~использованием 
ресурсов этими классами трафика. Предложенная модель поз\-во\-ля\-ет 
анализировать области значений па\-ра\-мет\-ров технической сис\-те\-мы для 
эффективного применения ФАР при обслуживании многоадресного трафика, 
в~том чис\-ле оценивать ограничения на рас\-сто\-яние между БС
в~таких сетях.
  
{\small\frenchspacing
 {\baselineskip=11.5pt
 %\addcontentsline{toc}{section}{References}
 \begin{thebibliography}{99}
\bibitem{1-sam}
\Au{David K., Berndt~H.} 6G vision and requirements: Is there any need for beyond 5G?~// IEEE 
Veh. Technol. Mag., 2018. Vol.~13. Iss.~3. P.~72--80. doi: 10.1109/MVT.2018. 2848498.
\bibitem{2-sam}
\Au{Petrov V., Kurner~T., Hosako~I.} IEEE 802.15. 3d: First standardization efforts for  
sub-terahertz band communications toward~6G~// IEEE Commun. Mag., 2020. Vol.~58. Iss.~11. 
P.~28--33. doi: 10.1109/MCOM.001.2000273. 
\bibitem{3-sam}
\Au{Kompella V.\,P., Pasquale~J.\,C., Polyzos~G.\,C.} Multicasting for multimedia applications~// 
Conference on Computer Communications.~--- IEEE, 1992. P.~2078--2085. doi: 
10.1109/INFCOM.1992.263480. 
\bibitem{4-sam}
Multimedia Broadcast/Multicast Service (MBMS); Stage~1 (Release~16): Technical Specification 
22.146 V16.0.0.~--- 3GPP, 2020. {\sf  
https://www.3gpp.org/ftp/ Specs/archive/22\_series/22.146/22146-g00.zip}.
\bibitem{5-sam}
\Au{Moltchanov D.} Distance distributions in random networks~// Ad Hoc Netw., 2012. 
Vol.~10. Iss.~6. P.~1146--1166. doi: 10.1016/j.adhoc.2012.02.005.
\bibitem{6-sam}
\Au{Kovalchukov R., Moltchanov~D., Gai\-da\-ma\-ka~Y., Bob\-ri\-ko\-va~E.} An accurate approximation 
of resource request distributions in millimeter wave 3GPP New Radio systems~// Internet of 
things, smart spaces, and next generation networks and systems~/
Eds.\ O.~Galinina, S.~Andreev, S.~Balandin, Y.~Koucheryavy.~--- Lecture notes in computer science ser.~--- 
Springer, 2019. Vol.~11660. P.~572--585. doi: 10.1007/978-3-030-30859-9\_50.
\bibitem{7-sam}
\Au{Basharin G., Gaidamaka~Y., Samouylov~K.} Mathematical theory of teletraffic and its 
application to the analysis of multiservice communication of next generation networks~// Autom. 
Control Comp.~S., 2013. Vol.~47. P.~62--69. doi:  10.3103/S0146411613020028.

\bibitem{9-sam} %8
\Au{Araniti G., Condoluci~M., Scopelliti~P., Mo\-li\-na\-ro~A., Iera~A.} Multicasting over emerging 
5G Networks: Challenges and perspectives~// IEEE Network, 2017. Vol.~31. No.\,2. P.~80--89. 
doi: 10.1109/MNET.2017.1600067NM.

\bibitem{8-sam} %9
Study on architectural enhancements for 5G multicast-broadcast services (Release~17): Technical 
Report 23.757 V1.2.0.~--- 3GPP, 2020. {\sf 
https://www.3gpp.org/ftp/ Specs/archive/23\_series/23.757/23757-120.zip}.

\bibitem{10-sam}
\Au{Tran T., Navr$\acute{\mbox{a}}$til~D., Sanders~P., Hart~J., Odar\-chen\-ko~R., Barjau~C., 
Altman~B., Burdinat~C., Gomez-Barquero~D.} Enabling multicast and broadcast in the 5G core for 
converged fixed and mobile networks~// IEEE T. Broadcast., 2020. Vol.~66. No.\,2.  
P.~428--439. doi: 10.1109/ TBC.2020.2991548.
\bibitem{11-sam}
\Au{Рыков В.\,В., Самуйлов~К.\,Е.} К~анализу вероятностей блокировок ресурсов сети 
с~динамическими многоадресными со\-еди\-не\-ни\-ями~// Электросвязь, 2000. №\,10. С.~27--30.
\bibitem{12-sam}
\Au{Karvo J., Martikainen~O., Virtamo~J., Aalto~S.} Blocking of dynamic multicast 
connections~// Telecommun. Syst., 2001. Vol.~16. P.~467--481. doi: 10.1023/A:1016631431617.
\bibitem{13-sam}
\Au{Boussetta K., Belyot~A.-L.} Multirate resource sharing for unicast and multicast connections~// 
Broadband communications~/ Eds.\ D.\,H.\,K.~Tsang, P.\,J.~K$\ddot{\mbox{u}}$hn.~--- Boston, MA, USA: Springer, 2000. Vol.~30. P.~561--570. doi: 
10.1007/978-0-387-35579-5\_47.

\pagebreak

\bibitem{14-sam}
\Au{Kelly F.\,P.} Loss networks~// Ann. Appl. Probab., 1991. Vol.~1. Iss.~3.  
P.~319-- 378.  doi: 
10.1214/aoap/1177005872.
\bibitem{15-sam}
\Au{Naumov V., Gaidamaka~Y., Yar\-ki\-na~N., Sa\-mouy\-lov~K.} Matrix and analytical methods for 
performance analysis of telecommunication systems.~--- Springer Nature, 2021. 308~p.
\bibitem{16-sam}
\Au{Samuylov A., Moltchanov~D., Ko\-val\-chu\-kov~R., Pir\-ma\-go\-me\-dov~R., Gai\-da\-ma\-ka~Y., 
And\-re\-ev~S., Kou\-che\-rya\-vy~Y., Samouylov~K.} Characterizing resource allocation trade-offs in 5G 
NR serving multicast and unicast traffic~// IEEE T. Wirel. Commun., 2020. Vol.~19. No.\,5. 
P.~3421--3434. doi: 10.1109/TWC.2020.2973375.

\end{thebibliography}

 }
 }

\end{multicols}

\vspace*{-6pt}

\hfill{\small\textit{Поступила в~редакцию 15.04.23}}

\vspace*{8pt}

%\pagebreak

%\newpage

%\vspace*{-28pt}

\hrule

\vspace*{2pt}

\hrule

%\vspace*{-2pt}

\def\tit{ON MODELING THE EFFECTS OF~MULTICAST TRAFFIC SERVICING IN~5G~NR 
NETWORKS}


\def\titkol{On modeling the effects of multicast traffic servicing in 5G NR 
networks}


\def\aut{A.\,K.~Samouylov$^1$, A.\,A.~Platonova$^1$, V.\,S.~Shorgin$^2$, 
and~Yu.\,V.~Gaidamaka$^{1,2}$}

\def\autkol{A.\,K.~Samouylov, A.\,A.~Platonova, V.\,S.~Shorgin, 
and~Yu.\,V.~Gaidamaka}

\titel{\tit}{\aut}{\autkol}{\titkol}

\vspace*{-10pt}


\noindent
    $^1$RUDN University, 6~Miklukho-Maklaya Str., Moscow 117198, Russian Federation
    
    \noindent
    $^2$Federal Research Center ``Computer Science and Control'' of the Russian Academy of 
Sciences; 44-2~Vavilov\linebreak
$\hphantom{^1}$Str., Moscow 119133, Russian Federation

\def\leftfootline{\small{\textbf{\thepage}
\hfill INFORMATIKA I EE PRIMENENIYA~--- INFORMATICS AND
APPLICATIONS\ \ \ 2023\ \ \ volume~17\ \ \ issue\ 2}
}%
 \def\rightfootline{\small{INFORMATIKA I EE PRIMENENIYA~---
INFORMATICS AND APPLICATIONS\ \ \ 2023\ \ \ volume~17\ \ \ issue\ 2
\hfill \textbf{\thepage}}}

\vspace*{3pt}
  
  
    
  \Abste{Multicasting in wireless access networks allows efficient provision of a~service to 
  a~group of subscribers and is useful for reducing the resource required to serve user equipments 
requesting the same data. The support of this feature in current 5G New Radio (NR) technology and 
future subterahertz (sub-THz) 6G systems faces challenges associated with the use of the 
directional beamforming phased array antennas. The presented multicast and unicast traffic service model 
allows one to explore the range of 5G/6G network parameters to reduce the 
density of base stations while maintaining the quality of services provided to subscribers.}
  
  \KWE{5G; 6G; multicasting; millimeter wave; terahertz; multibeam antennas; 
 multi-RAT; numerical simulation}
  
  
  
\DOI{10.14357/19922264230210}{SLMGZU} 

\vspace*{-14pt}

\Ack
  \noindent
  The reported study was funded by the Russian Science Foundation, project No.\,21-79-00142.

\vspace*{4pt}

  \begin{multicols}{2}

\renewcommand{\bibname}{\protect\rmfamily References}
%\renewcommand{\bibname}{\large\protect\rm References}

{\small\frenchspacing
 {%\baselineskip=10.8pt
 \addcontentsline{toc}{section}{References}
 \begin{thebibliography}{99}
  \bibitem{1-sam-1}
  \Aue{David, K., and H.~Berndt.} 2018. 6G vision and requirements: Is there any need for 
beyond 5G? \textit{IEEE Veh. Technol. Mag.} 13(3):72--80. doi: 10.1109/MVT.2018.2848498.
  \bibitem{2-sam-1}
  \Aue{Petrov, V., T.~Kurner, and I.~Hosako.} 2020. IEEE 802.15. 3d: First standardization 
efforts for sub-terahertz band communications toward 6G. \textit{IEEE Commun. Mag.} 
58(11):28--33. doi: 10.1109/MCOM.001.2000273. 
  \bibitem{3-sam-1}
  \Aue{Kompella, V.\,P., J.\,C.~Pasquale, and G.\,C.~Polyzos.} 1992. Multicasting for multimedia 
applications. \textit{Conference on Computer Communications}. IEEE. 2078--2085. doi: 
10.1109/INFCOM.1992.263480. 
  \bibitem{4-sam-1}
Multimedia broadcast/multicast service (MBMS); Stage~1 (Release~16): Technical specification 
22.146 V16.0.0. 3GPP. Available at:  {\sf 
https://www.3gpp.org/ftp/Specs/ archive/22\_series/22.146/22146-g00.zip} (accessed May~20, 2023).
  \bibitem{5-sam-1}
\Aue{Moltchanov, D.} 2012. Distance distributions in random networks. \textit{AD Hoc Netw.} 
10(6):1146--1166. doi: 10.1016/ j.adhoc.2012.02.005.
  \bibitem{6-sam-1}
  \Aue{Kovalchukov, R., D.~Moltchanov, Y.~Gai\-da\-ma\-ka, and E.~Bob\-ri\-ko\-va.} 2019. An accurate 
approximation of resource request distributions in millimeter wave 3GPP New Radio systems. 
\textit{Internet of things, smart spaces, and next generation networks and systems}. Eds.\ O.~Galinina, S.~Andreev, S.~Balandin,
and Y.~Koucheryavy. Lectures notes 
in computer science ser. Springer. 11660:572--585. doi: 10.1007/978-3-030-30859-9\_50.
  \bibitem{7-sam-1}
  \Aue{Basharin, G., Y.~Gai\-da\-ma\-ka, and K.~Sa\-mouy\-lov.} 2013. Mathematical theory of 
teletraffic and its application to the analysis of multiservice communication of next generation 
networks. \textit{Autom. Control Comp.~S.}  47:62--69. doi:  10.3103/S0146411613020028.
  
  \bibitem{9-sam-1} %8
  \Aue{Araniti, G., M.~Condoluci, P.~Scopelliti, A.~Mo\-li\-na\-ro, and A.~Iera.} 2017. Multicasting 
over emerging 5G networks: Challenges and perspectives. \textit{IEEE Network} 31(2):80--89. doi: 
10.1109/MNET.2017.1600067NM.

\bibitem{8-sam-1} %9
  Study on architectural enhancements for 5G multicast-broadcast services (Release 17): Technical 
Report 23.757 V1.2.0. 3GPP. Available at:  {\sf 
https://www.3gpp.org/ ftp/Specs/archive/23\_series/23.757/23757-120.zip} (accessed May~20, 2023).
  \bibitem{10-sam-1}
  \Aue{Tran, T., D.~Nav$\acute{\mbox{a}}$til, P.~Sanders, J.~Hart, R.~Odar\-chen\-ko, C.~Barjau, 
B.~Altman, C.~Burdinat, and D.~Gomez-Barquero.} 2020. Enabling multicast and broadcast in the 
5G core for converged fixed and mobile networks. \textit{IEEE T. Broadcast.} 66(2):428--439. doi: 
10.1109/ TBC.2020.2991548.
  \bibitem{11-sam-1}
  \Aue{Rykov, V.\,V., and K.\,E.~Samuylov.} 2000. K~analizu ve\-ro\-yat\-no\-stey blokirovok 
resursov seti s~dinamicheskimi mno\-go\-ad\-res\-ny\-mi so\-edi\-ne\-ni\-yami [To the analysis of blocking 
probabilities in a~network with dynamic multicast connections]. \textit{Elektrosvyaz'}  
[Electrosvyaz Magazine] 10:\linebreak 27--30.
  \bibitem{12-sam-1}
  \Aue{Karvo, J., O.~Martikainen, J.~Virtamo, and S.~Aalto.} 2001. Blocking of dynamic 
multicast connections. \textit{Telecommun. Syst.} 16:467--481. doi: 10.1023/A:1016631431617.
  \bibitem{13-sam-1}
  \Aue{Boussetta, K., and A.-L.~Belyot.} 2000. Multirate resource sharing for unicast and 
multicast connections. \textit{Broadband communications}. Eds. D.\,H.\,K.~Tsang and P.\,J.~K$\ddot{\mbox{u}}$hn.
Boston, MA: Springer. 30:561--570. doi: 10.1007/978-0-387-35579-5\_47.
  \bibitem{14-sam-1}
  \Aue{Kelly, F.\,P.} 1991. Loss networks. \textit{Ann. Appl. Probab.} 1(3):319--378. doi: 
10.1214/aoap/1177005872.
  \bibitem{15-sam-1}
  \Aue{Naumov, V., Y.~Gaidamaka, N.~Yar\-ki\-na, and K.~Sa\-mouy\-lov.} 2021. \textit{Matrix and 
analytical methods for performance analysis of telecommunication systems}. Springer Nature. 
308~p.
  \bibitem{16-sam-1}
  \Aue{Samuylov, A., D.~Moltchanov, R.~Ko\-val\-chu\-kov, R.~Pir\-ma\-go\-me\-dov, Y.~Gai\-da\-ma\-ka, 
S.~And\-re\-ev, Y.~Kou\-che\-rya\-vy, and K.~Sa\-mouy\-lov.} 2020. Characterizing resource allocation 
trade-offs in 5G NR serving multicast and unicast traffic. \textit{IEEE T. Wirel. Commun.}  
19(5):3421--3434. doi: 10.1109/TWC.2020.2973375.
  \end{thebibliography}

 }
 }

\end{multicols}

\vspace*{-6pt}

\hfill{\small\textit{Received April 15, 2023}} 
  
  \Contr
  
  \noindent
  \textbf{Samuylov Andrey K.} (b.\ 1988)~--- Candidate of Science (PhD) in physics and 
mathematics, associate professor, Department of Applied Probability and Informatics, RUDN 
University, 6~Miklukho-Maklaya Str., Moscow 117198, Russian Federation;  
\mbox{samuylov-ak@rudn.ru}
  
  \vspace*{3pt}
  
  \noindent
  \textbf{Platonova Anna A.} (b.\ 1996)~--- PhD student, Department of Applied Probability and 
Informatics, RUDN University, 6~Miklukho-Maklaya Str., Moscow 117198, Russian Federation; 
\mbox{platonova-aa@rudn.ru}
  
  \vspace*{3pt}
  
  \noindent
  \textbf{Shorgin Vsevolod S.} (b.\ 1978)~--- Candidate of Science (PhD) in technology, senior 
scientist, Institute of Informatics Problems, Federal Research Center ``Computer Science and 
Control'' of the Russian Academy of Sciences, 44-2~Vavilov Str., Moscow 119333, Russian 
Federation; \mbox{vshorgin@ipiran.ru}
  
  \vspace*{3pt}
  
  \noindent
  \textbf{Gaidamaka Yuliya V.} (b.\ 1971)~--- Doctor of Science in physics and mathematics, 
professor, Department of Applied Probability and Informatics, RUDN University,  
6~Miklukho-Maklaya Str., Moscow 117198, Russian Federation; senior scientist, Institute of 
Informatics Problems, Federal Research Center ``Computer Science and Control'' of the Russian 
Academy of Sciences, 44-2~Vavilov Str., Moscow 119333, Russian Federation;  
\mbox{gaydamaka-yuv@rudn.ru}
  
\label{end\stat}

\renewcommand{\bibname}{\protect\rm Литература}       %10
\def\stat{khachumov}

\def\tit{САМООБУЧЕНИЕ АВТОНОМНЫХ ИНТЕЛЛЕКТУАЛЬНЫХ РОБОТОВ В~ПРОЦЕССЕ 
ПОИСКОВО-ИССЛЕДОВАТЕЛЬСКОЙ ДЕЯТЕЛЬНОСТИ$^*$}

\def\titkol{Самообучение автономных интеллектуальных роботов в~процессе 
поисково-исследовательской деятельности}

\def\aut{В.\,Б.~Мелехин$^1$, В.\,М.~Хачумов$^2$, М.\,В.~Хачумов$^3$}

\def\autkol{В.\,Б.~Мелехин, В.\,М.~Хачумов, М.\,В.~Хачумов}

\titel{\tit}{\aut}{\autkol}{\titkol}

\index{Мелехин В.\,Б.}
\index{Хачумов В.\,М.}
\index{Хачумов М.\,В.}
\index{Melekhin V.\,B.}
\index{Khachumov V.\,M.}
\index{Khachumov M.\,V.}


{\renewcommand{\thefootnote}{\fnsymbol{footnote}} \footnotetext[1]
{Исследование выполнено при поддержке Российского научного фонда (проект 21-71-10056).}}


\renewcommand{\thefootnote}{\arabic{footnote}}
\footnotetext[1]{Дагестанский государственный технический университет, \mbox{pashka1602@rambler.ru}}
\footnotetext[2]{Институт программных систем им.\ А.\,К.~Айламазяна Российской академии наук; Федеральный 
исследовательский центр <<Информатика и~управ\-ле\-ние>> Российской академии наук; Российский 
университет друж\-бы народов, \mbox{vmh48@mail.ru}}
\footnotetext[3]{Институт программных систем им.\ А.\,К.~Айламазяна Российской академии наук; Федеральный 
исследовательский центр <<Информатика и~управ\-ле\-ние>> Российской академии наук; Российский 
университет друж\-бы народов, \mbox{khmike@inbox.ru}}

%\vspace*{-12pt}

 
  

\Abst{Рассматривается один из эффективных подходов к~организации целесообразного 
поведения автономных интегральных роботов (АИР) в~процессе по\-иско\-во-ис\-сле\-до\-ва\-тель\-ской 
деятельности в~априори неописанных условиях проб\-лем\-ной среды (ПС). Предлагается в~основе 
целесообразного поведения роботов использовать процедуры на\-гляд\-но-дей\-ст\-вен\-но\-го 
мышления, основанные на формализации рефлекторного поведения высокоорганизованных 
живых сис\-тем. Разработан алгоритм са\-мо\-обуче\-ния в~условиях с~высоким уровнем 
неопределенности, поз\-во\-ля\-ющий автоматически формировать условные программы 
целесообразного поведения, обес\-пе\-чи\-ва\-ющие АИР
возможность достигать заданной цели поведения в~процессе  
по\-иско\-во-ис\-сле\-до\-ва\-тель\-ской деятельности. Найдены граничные оценки 
функциональной слож\-ности предложенного алгоритма самообучения в~условиях 
не\-опре\-де\-лен\-ности, по\-ка\-зы\-ва\-ющие воз\-мож\-ность его реализации на бортовой ЭВМ 
автономных интегральных роботов, име\-ющих, как правило, ограниченные вы\-чис\-ли\-тель\-ные 
ресурсы. Проведено имитационное моделирование процесса са\-мо\-обуче\-ния 
АИР в~априори неописанной ПС, под\-твер\-див\-шее 
эф\-фек\-тив\-ность применения предложенного подхода для организации планирования 
целесообразного поведения в~априори неописанных~ПС.}

\KW{автономный интегральный робот; алгоритм са\-мо\-обуче\-ния; условия 
не\-оп\-ре\-де\-лен\-ности; проб\-лем\-ная среда; услов\-ные сигналы}

\DOI{10.14357/19922264230211}{SOFDKW} 
  
\vspace*{-4pt}


\vskip 10pt plus 9pt minus 6pt

\thispagestyle{headings}

\begin{multicols}{2}

\label{st\stat}
 
\section{Введение}

  Разработка информационных технологий, связанных с~по\-стро\-ен\-ием 
интеллектуального решателя задач АИР, 
способных целесообразно (рационально) функционировать в~априори 
неописанных ПС,~--- актуальная и~слож\-ная проб\-ле\-ма 
искусственного интеллекта. К~одному из эффективных подходов решения 
данной проб\-ле\-мы следует отнести разработку когнитивных инструментов 
на\-гляд\-но-дей\-ст\-вен\-но\-го мышления интеллектуальных сис\-тем различного 
назначения~[1]. В~общем случае на\-гляд\-но-дей\-ст\-вен\-ное мыш\-ле\-ние АИР 
строится на основе формализации рефлекторного поведения живых сис\-тем~[2, 3] и~включает сле\-ду\-ющие три основные со\-став\-ля\-ющие~[4]. 
  \begin{enumerate}[1.]
  \item Самообучение на основе выполнения пробных действий и~механизмов 
избирательности по\-сту\-па\-ющей из ПС информации, 
обес\-пе\-чи\-ва\-ющих воз\-мож\-ность поиска заданных объектов в~априори 
неописанных условиях функционирования. 
  
  Организовать самообучение АИР в~априори неописанных условиях ПС 
можно на основе, например, алгоритмов роевого поведения~[5, 6] или 
генетических алгоритмов~[7, 8]. Однако непосредственная отработка проб\-ных 
действий может привести к~негативным изменениям, не связанным
 с~достижением заданной цели. Обойти этот недостаток можно на основе 
алгоритмов са\-мо\-обуче\-ния, которые имитируют выполнение проб\-ных действий 
на формальном описании текущей ситуации ПС.
  
  Процесс самообучения АИР в~априори неописанной ПС
сводится к~формированию и~закреплению элементарных актов поведения 
в~фор\-ми\-ру\-емых условных программах целесообразной\linebreak де\-я\-тель\-ности (УПЦД) по 
новизне происходящих в~ПС изменений, а~для всей 
автоматически по\-стро\-ен\-ной упорядоченной по\-сле\-до\-ва\-тель\-ности действий 
характеризуется \mbox{дос\-ти\-же\-ни\-ем} заданного безуслов\-но\-го сигнала. 
  
  В формируемых в~процессе самообучения \mbox{УПЦД} запоминаются происходящие в~ПС
изменения в~форме сигналов, которые возникают в~результате от\-ра\-ба\-ты\-ва\-емых 
АИР действий. Различные сигналы ПС в~процессе са\-мо\-обуче\-ния 
приобретают роль услов\-ных сигналов~--- знаков, вы\-зы\-ва\-ющих у~АИР 
определенные реакции, связанные с~отработкой за\-креп\-лен\-ных в~УПЦД 
действий. Таким образом, услов\-ные сигналы после за\-креп\-ле\-ния в~УПЦД 
приобретают роль ориентиров или предвестников, появление которых 
в~ПС сигнализирует АИР о~воз\-мож\-ности достижения в~ней 
соответствующего без\-услов\-но\-го сигнала.
  \item  Целесообразное поведение АИР, связанное с~отработкой в~текущих 
условиях функционирования действий, ранее сформированных УПЦП для 
достижения со\-от\-вет\-ст\-ву\-ющих им без\-услов\-ных сигналов при восприятии 
в~ПС закрепленных в~этих программах услов\-ных сигналов. 
  \item Отработка безусловных реакций для достижения заданной цели при 
появлении в~ПС со\-от\-вет\-ст\-ву\-ющих им без\-услов\-ных сигналов.
  \end{enumerate}
  
  В настоящей статье предлагаются процедуры са\-мо\-обуче\-ния в~процессе  
по\-иско\-во-ис\-сле\-до\-ва\-тель\-ской деятельности, поз\-во\-ля\-ющие АИР 
организовать целесообразное поведение в~априори \mbox{неописанной} 
ПС с~препятствиями для поиска заданных объектов. Например, при 
выполнении различных спасательных работ в~труднодоступных для человека 
условиях функционирования.
  
\section{Постановка задачи}

  Рассмотрим АИР, оснащенный техническим зрением, манипулятором 
и~моторной сис\-те\-мой, поз\-во\-ля\-ющей ему перемещаться в~ПС. 
Проблемная среда пред\-став\-ля\-ет собой пересеченную мест\-ность 
с~расположенными на ее территории препятствиями и~различными объектами 
$O\hm= \{ o_{i_1}(X_{i_1})\}$, $i_1\hm= \overline{1,n_1}$, где $X_{i_1}$~--- 
множество характеристик, по которым робот способен идентифицировать 
воспринимаемые в~ПС препятствия и~объекты. 
  
  Проблемную среду можно охарактеризовать множеством условных сигналов 
  $A\hm= \{a_{i_2}\}$, $i_2\hm=\overline{1,n_2}$, каждый из которых 
пред\-став\-ля\-ет собой проходимый меж\-ду препятствиями участок ПС.
  
  В общем случае АИР способен отрабатывать множество действий 
  $B\hm= \{ b_{i_3}\}$, $i_3\hm= \overline{1,n_3}$, и~распознавать 
проходимые участ\-ки мест\-ности, а~также найденные в~ПС
объекты $o_{i_1}(X_{i_1})\hm\in O$. (Следует отметить, что проб\-ле\-ма 
распознавания проходимых участков и~объектов ПС является 
самостоятельной задачей и~в~на\-сто\-ящей статье не рас\-смат\-ри\-ва\-ется.)
  
  Требуется разработать алгоритм самообучения АИР в~процессе  
по\-иско\-во-ис\-сле\-до\-ва\-тель\-ской деятельности, поз\-во\-ля\-ющий 
автоматически формировать в~априори неописанной проб\-лем\-ной среде УПЦП 
сле\-ду\-юще\-го вида: 
  \begin{equation}
  a^1_{i_2} \& b^1_{i_3} \to a^2_{i_2} \& b^2_{i_3} \to\cdots\to a^k_{i_2} 
  \& b^k_{i_3} \to a^P_{i_2}\,,
  \label{e1-kh}
  \end{equation}
где $a^1_{i_2}\& b^1_{i_3} \to a^2_{i_2}$~--- элементарный акт поведения, 
озна\-ча\-ющий, что если АИР воспринимает в~ПС условный сигнал 
$a^1_{i_2}$, то отрабатываемое им действие~$b^1_{i_3}$ приводит 
к~появлению условного сигнала~$a^2_{i_2}$; $a^1_{i_2}$ и~$a^2_{i_2}$~--- 
условные сигналы ПС, определяющиеся проходимыми для АИР 
участками ПС; $a^P_{i_2}$~--- безуслов\-ный сигнал, 
вызывающий у~АИР соответствующие безусловные реакции, связанные 
с~выполнением определенных действий над найденным объектом. 

  Требуется разработать алгоритм са\-мо\-обуче\-ния, поз\-во\-ля\-ющий АИР 
в~процессе по\-иско\-во-ис\-сле\-до\-ва\-тель\-ской де\-я\-тель\-ности формировать \mbox{УПЦД} 
в~виде прос\-той цепи~(1) в~априори неописанных~ПС.
  
\section{Синтез алгоритма самообучения автономных интегральных роботов}

  В общем случае алгоритм самообучения АИР в~процессе по\-ис\-ко\-во-ис\-сле\-до\-ва\-тель\-ской 
  де\-я\-тель\-ности опирается на за\-креп\-ле\-ние в~формируемой 
УПЦД элементарных актов поведения по новизне условных сигналов, 
воспринимаемых в~ПС. Вся же полученная таким образом цепь 
действий закрепляется достижением в~ПС заданного 
без\-услов\-но\-го сигнала. Роль безуслов\-но\-го сигнала в~этом случае играет 
заданный объект, воспринятый в~ПС после выхода АИР за пределы последнего 
за\-креп\-лен\-но\-го в~УПЦД условного сигнала. Данный алгоритм са\-мо\-обуче\-ния 
АИР имеет сле\-ду\-ющее структурированное описание. 
  
\noindent
  \textbf{Исходные условия:} заданный АИР объект ПС 
$o_{i_1}(X_{i_1})\hm\in O$, выполняющий роль без\-услов\-но\-го 
сигнала~$a_{i_2}^P$; множество действий~$B$, которые способен 
отрабатывать АИР.
  
 \noindent
  \textbf{Входные переменные:} услов\-ные сигналы $a_{i_2}\hm\in A$ 
и~заданный~$a^P_{i_2}$ безусловный сигнал ПС.
  
 \noindent
  \textbf{Выходные переменные:} фор\-ми\-ру\-емые УПЦД в~виде прос\-той цепи. 
  
\smallskip
  
 \noindent
  \textbf{Начало.}\\[-12pt]
  \begin{enumerate}[1.]
\item  Установить $j_1\hm=1$. Определить в~качестве исходного 
сигнала~$a_{i_2}^{j_1}$ в~фор\-ми\-ру\-емой УПЦД непосредственно 
воспринимаемое роботом в~ПС препятствие. 
  \item  Принять в~качестве подцели поведения на текущем шаге самообучения 
появление в~ПС нового условного сигнала~$a_{i_2}^{j_1+1}$, 
опре\-де\-ля\-емо\-го воспринятым после отработки проб\-но\-го действия новым 
проходимым участ\-ком. 
  \item  Определить на текущем шаге самообучения согласно равномерному 
закону распределения вероятностей выбора проб\-ное действие 
$b^{j_1}_{i_3}\hm\in B$. Выполнить выбранное действие~$b^{j_1}_{i_3}$ 
и~сформировать по результатам его отработки элементарный акт поведения 
$a_{i_2}^{j_1} \& b^{j_1}_{i_3}\hm\to a_{i_2}^{j_1+1}$.
  \item  Проверить условие <<услов\-ный сигнал~$a_{i_2}^{j_1+1}$ был ранее 
закреплен в~формируемой УПЦД>>: если условие выполняется, то перейти 
к~п.~5; в~противном случае перейти к~п.~8.
  \item  Удалить все элементарные акты поведения, за\-креп\-лен\-ные в~УПЦД 
после первого восприятия АИР в~ПС услов\-но\-го 
сигнала~$a_{i_2}^{j_1+1}$.
  \item Исключить выбранное пробное действие~$b_{i_3}^{j_1}$ из 
множества~$B$ как нерезультативное на текущем шаге са\-мо\-обуче\-ния. 
  \item Проверить условие <<множество~$B$ является пус\-тым>>: если условие 
выполняется, то перейти к~п.~11; в~противном случае перейти к~п.~3. 
  \item  Сохранить элементарный акт поведения $a_{i_2}^{j_1} \& 
b_{i_3}^{j_1} \hm\to a_{i_2}^{j_1+1}$ в~фор\-ми\-ру\-емой УПЦД.
  \item  Проверить условие <<после выхода за пределы проходимого участка, 
опре\-де\-ля\-емо\-го условным сигналом~$a_{i_2}^{j_1+1}$, АИР воспринимает 
в~ПС заданный без\-услов\-ный сигнал~$a_{i_2}^P$>>: если 
условие выполняется, перейти к~п.~12; в~противном случае перейти к~п.~10.
  \item  Восстановить все исключенные действия из заданного множества~$B$, 
$j_1\hm= j_1\hm+1$, перейти к~п.~2.
  \item  Сформировать тре\-бу\-емую УПЦД в~текущих условиях ПС не 
пред\-став\-ля\-ет\-ся воз\-мож\-ным, перейти к~п.~13.
  \item  Требуемая УПЦД сформирована; выполнить для достижения заданной 
цели безуслов\-ные реакции.\\[-14pt]
  \end{enumerate}
  
 \noindent
  \textbf{Конец.}
  
  \smallskip
  
  Введем понятие функциональной слож\-ности~$\beta$ алгоритма 
са\-мо\-обуче\-ния АИР, зависящей от общего чис\-ла действий $b_{i_3}\hm\in B$, 
апробируемых роботом в~процессе формирования УПЦД. Тогда для данного алгоритма можно доказать 
сле\-ду\-ющее утверж\-де\-ние.
  
  \smallskip
  
  \noindent
  \textbf{Утверждение.} \textit{Функциональная слож\-ность~$\beta$ 
алгоритма са\-мо\-обуче\-ния АИР определяется сле\-ду\-ющи\-ми граничными 
оценками: 
  $$
  n_{10}\leq\beta\leq n_1 n_{10}\,,
  $$
где $n_{10}$~--- общее чис\-ло выполненных АИР шагов самообучения; $n_1$~--- 
общее чис\-ло различного вида действий, которые робот отрабатывает 
в~процессе са\-мо\-обуче\-ния.}

\smallskip

\noindent
  Д\,о\,к\,а\,з\,а\,т\,е\,л\,ь\,с\,т\,в\,о\,.\ \ Справедливость 
сформулированного утверж\-де\-ния вытекает из сле\-ду\-ющих соображений.
  \begin{enumerate}[1.]
  \item Согласно пп.~3--8 \textit{алгоритма са\-мо\-обуче\-ния}, впол\-не вероятно, 
что в~лучшем случае на каж\-дом $j_1$-м шаге самообучения АИР первым 
случайным образом выбирает результативное действие~$b_{i_3}^{j_1}$. 
Следовательно, ниж\-нее граничное значение оцен\-ки слож\-ности~$\beta_1$ 
в~этом случае определяется величиной, равной~$n_{10}$. 
  \item  В худшем случае результативное действие~$b_{i_3}^{j_1}$ на каж\-дом 
$j_1$-м шаге самообучения АИР может быть выбрано случайным образом 
в~последнюю очередь. Отсюда следует, что на каж\-дом шаге са\-мо\-обуче\-ния АИР 
апробирует отработку не более $n_1$ действий $b_k(j_1)\hm\in B$. Таким 
образом, число выполнений проб\-ных действий в~процессе са\-мо\-обуче\-ния АИР 
не может превышать величины, рав\-ной~$n_1 n_{10}$.
  \item Из пп.~1 и~2 проведенного доказательства с~оче\-вид\-ностью следует 
спра\-вед\-ли\-вость сформулированного утверж\-де\-ния. 
  \end{enumerate}
  
 
  
   Рассмотрим гипотетический пример, связанный с~использованием АИР 
алгоритма са\-мо\-обуче\-ния для решения целевой задачи, когда у~робота 
отсутствует формальное описание кар\-ты мест\-ности, а~известны только границы 
участка ПС, на котором требуется найти заданные объекты.

\section{Пример решения задачи в~процессе  
поисково-исследовательской деятельности автономных интеллектуальных роботов}

  Пусть АИР требуется найти заданный объект в~априори неописанной 
ПС, пред\-став\-ля\-ющей собой мест\-ность с~расположенными на ней 
препятствиями, структура которой приведена на рисунке.
   
   
  Таким образом, ПС характеризуется 11~расположенными 
в~ней препятствиями $P\hm= \{p_{i_6}\}$, $i_6\hm= \overline{1,11}$, 
и~16~проходимыми между препятствиями зонами, обозначенными сигналами 
$A\hm= \{ a_{i_7}\}$, $i_7\hm= \overline{1,16}$. В~этой среде АИР требуется 
найти объекты,\linebreak\vspace*{-12pt}

\pagebreak

\end{multicols}

 \begin{figure*} %fig1
   \vspace*{1pt}
\begin{center}
   \mbox{%
\epsfxsize=117.405mm 
\epsfbox{hac-1.eps}
}


\vspace*{12pt}

   {\small Структура ПС с~препятствиями и~обозначенными исходными 
местоположениями АИР и~заданных объектов}
\end{center}
\vspace*{6pt}
   %\end{figure*}
      %\vspace*{6pt}
   %
   %\begin{table*}[b]\small 
  % 
   \vspace*{6pt}
\begin{center}
{\small \begin{tabular}{|c|c|c|c|c|c|c|c|c|c|c|c|c|c|c|c|c|c|c|c|}
\multicolumn{20}{c}{Закономерности перехода АИР из текущего положения в~ПС к~смежной проходимой зоне}\\
\multicolumn{20}{c}{\ }\\[-3pt]
\hline
&$a_{\mathrm{И}}$&$a_1$&$a_2$&$a_3$&$a_4$&$a_5$&$a_1^*$&$a_6$&$a_7$&$a_8$&$a_9$& 
$a_{10}$&$a_{11}$&$a_2^*$&$a_{12}$&$a_{13}$&$a_{14}$&$a_{15}$&$a_{16}$\\
\hline
$b_1$&$a_1$&$a_3$&$a_4$&---&$a_3$&$a_8$&---&---&$a_9$&$a_{10}$&$a_{12}$&$a_{13}$&$a_{10}$&---&---&$a_{12}$&$a_{16}$&---&$a_{\mathrm{Ц}}$\\
$b_2$&$a_2$&$a_4$&$a_5$&$a_4$&$a_5$&$a_1^*$&---
&$a_9$&$a_{10}$&$a_{11}$&$a_{13}$&$a_{14}$&$a_2^*$&---&$a_{15}$&$a_{14}$&---&$a_{\mathrm{Ц}}$
&---\\
$b_3$&---&---&---&---&---&---&$a_5$&---&---&---&---&---&---&$a_{11}$&---&---&---&---&---\\
\hline
\end{tabular}
}
\end{center}
\end{figure*}
%\end{table*}

\begin{multicols}{2}

\noindent
 расположенные за препятствием $p_8\hm\in P$, при его 
исходном местоположении напротив препятствия $p_2\hm\in P$. Восприятие 
данного объекта в~ПС соответствует достижению роботом 
заданной цели. Пусть АИР для решения по\-став\-лен\-ной перед ним задачи 
способен отрабатывать сле\-ду\-ющие три действия: $b_1$~--- поворот влево 
и~движение вперед до выхода за наблюдаемую в~результате этого проходимую 
зону; $b_2$~--- поворот вправо и~движение вперед до выхода за наблюдаемую 
в~результате этого проходимую зону; $b_3$~--- безуслов\-ные реакции 
<<разворот и~выход из тупика в~предыду\-щее исходное текущее 
местоположение>>.
  
  При этом система технического зрения АИР способна распознавать 
и~отличать друг от друга проходимые участки $a_{i_7}\hm\in A$ 
ПС и~заданный ему объект. Для имитации процесса поиска АИР цели 
в~заданной ПС строится конечный автомат со случайными 
реакциями~[4], в~память которого занесена таб\-ли\-ца команд, отражающая 
закономерности перехода АИР от одного проходимого участка ПС к~другому такому участку (см.\ таблицу). 
  

    
  В таблице использованы сле\-ду\-ющие обозначения: $a_{\mathrm{И}}$~--- 
исходное местоположение АИР; $a_1^*$ и~$a_2^*$~--- соответственно первый 
и~второй тупик; $a_{\mathrm{Ц}}$~--- местоположение заданных объектов; 
прочерк означает отсутствие прохода или выхода за пределы заданного участка 
мест\-ности. 
  
  По итогам проведенного на ПЭВМ эксперимента были получены сле\-ду\-ющие 
результаты. Автономный интеллектуальный робот, выполнив~12~пробных 
действий, прошел по сле\-ду\-юще\-му марш\-ру\-ту в~процессе поиска заданных 
объектов (см.\ рисунок):

\vspace*{-2pt}

\noindent
  \begin{multline*}
  a_{\mathrm{И}} \& b_1 \to a_1\& b_2 \to a_4 \& b_2\to a_5 \& b_1\to a_8 \&
  b_2\to{}\\
  {}\to  a_{11} \& b_2\to a_2^*(\mbox{тупик~2}) \& b_3\to a_{11} \& b_1\to{}\\
  {}\to  a_{10} \& b_1\to a_{13} \& b_1\to a_{12} \& b_2\to 
a_{15} \& b_2\to a_{\mathrm{Ц}}.
  \end{multline*}
  %
  
  \vspace*{-2pt}
  
  \noindent
  При этом у АИР после обнуления в~процессе са\-мо\-обуче\-ния цик\-ла 
сформировалась сле\-ду\-ющая \mbox{УПЦД}:

\vspace*{-2pt}

\noindent
  \begin{multline*}
  a_{\mathrm{И}} \& b_1\to a_1 \& b_2\to a_4 \& b_2\to a_5 \& b_1\to a_8 \& 
b_1\to{}\\
  {}\to
  a_{10} \& b_1\to a_{13} \& b_1\to a_{12} \& b_2\to a_{15} \& b_2\to 
a_{\mathrm{Ц}}\,.
  \end{multline*}
  
  Данную УПЦД интеллектуальный робот может использовать, например, для 
перевозки большого чис\-ла заданных объектов на участок ПС, 
опре\-де\-ля\-ющий заданное их мес\-то\-по\-ло\-же\-ние. 

%\vspace*{-6pt}

\section{Заключение}

\noindent
\begin{enumerate}[1.]
  \item Предложенный алгоритм са\-мо\-обуче\-ния поз\-во\-ля\-ет организовать 
целесообразное поведение АИР в~процессе  
по\-иско\-во-ис\-сле\-до\-ва\-тель\-ской\linebreak\vspace*{-9.5pt}
\end{enumerate}

\noindent
\begin{enumerate}[1.]
\setcounter{enumi}{1}
\item[\,]
 деятельности в~априори неописанных 
труднодоступных для человека~ПС.
  \item Найденные граничные оценки и~\mbox{результаты} имитационного 
моделирования алгоритма самообуче\-ния показали эф\-фек\-тив\-ность его 
использования для проведения АИР по\-иско\-во-ис\-сле\-до\-ва\-тель\-ской 
деятельности в~априори неописанной ПС с~препятствиями 
с~\mbox{целью} поиска заданных объектов, например при выполнении различных 
спасательных работ в~труд\-но\-до\-ступ\-ных для человека условиях 
функ\-цио\-ни\-ро\-ва-\linebreak ния. 
  \end{enumerate}
   
{\small\frenchspacing
 {%\baselineskip=10.8pt
 %\addcontentsline{toc}{section}{References}
 \begin{thebibliography}{9}
  \bibitem{1-kh}
  \Au{Мелехин~В.\,Б., Хачумов~М.\,В.} Формы мышления автономных интеллектуальных 
агентов: особенности и~проб\-ле\-мы их организации~// Морские интеллектуальные технологии, 
2020. №\,4-1. С.~224--230. doi: 10.37220/MIT.2020.50.4.031.
  \bibitem{2-kh}
  \Au{Брайнес С.\,Н., Напалков~А.\,Н., Свечинский~В.\,Б.} Нейрокибернетика.~--- М.: 
Госмедиздат, 1962. 172~с.
  \bibitem{3-kh}
  \Au{Шингаров Г.\,Х.} Условные рефлексы и~проб\-ле\-ма знака и~значения.~--- М.: Наука, 
1986. 200~с.
  \bibitem{4-kh}
  \Au{Мелехин В.\,Б., Хачумов~М.\,В.} Инструментальные средства управ\-ле\-ния 
целесообразным поведением са\-мо\-ор\-га\-ни\-зу\-ющих\-ся автономных интеллектуальных агентов~// 
Мехатроника, автоматизация, управ\-ле\-ние, 2021. Т.~22. №\,4. С.~171--180. doi: 
10.17587/mau.22.171-180.
  \bibitem{5-kh}
  \Au{Карпов В.\,Э., Карпова~И.\,П., Кулинич~А.\,А.} Социальные сообщества роботов.~--- 
М.: Ленанд, 2019. 352~с.
  \bibitem{6-kh}
  \Au{Guan B., Xu~T., Zhao~Y., Li~Y., Dong~X.} A~random grouping-based self-regulating 
artificial bee colony algorithm for interactive feature detection~// Knowl.-Based Syst., 2021. 
Vol.~243. P.~1--12. doi: 10.1016/j.knosys.2022.108434.
  
  \bibitem{8-kh}
  \Au{Рутковская Д., Пилиньский~М., Рутковский~Л.} Нейронные сети, генетические 
алгоритмы и~нечеткие сис\-те\-мы.~--- М.: Горячая линия\,--\,Телеком, 2008. 452~с.

\bibitem{7-kh}
  \Au{Саймон Д.} Алгоритмы эволюционной оптимизации~/ Пер. с~англ.~--- М.: ДМК 
Пресс, 2020. 940~с. (\Au{Simon~D.} Evolutionary optimization algorithms.~--- 1st ed.~--- New 
York, NY, USA: Wiley, 2013. 784~p.)

\end{thebibliography}

 }
 }

\end{multicols}

\vspace*{-6pt}

\hfill{\small\textit{Поступила в~редакцию 02.11.22}}

\vspace*{8pt}

%\pagebreak

%\newpage

%\vspace*{-28pt}

\hrule

\vspace*{2pt}

\hrule

%\vspace*{-2pt}

\def\tit{SELF-LEARNING OF AUTONOMOUS INTELLIGENT ROBOTS IN~THE~PROCESS 
OF~SEARCH AND~EXPLORE ACTIVITIES}


\def\titkol{Self-learning of autonomous intelligent robots in~the~process 
of~search and~explore activities}


\def\aut{V.\,B.~Melekhin$^1$, V.\,M.~Khachumov$^{2,3,4}$, and~M.\,V.~Khachumov$^{2,3,4}$}

\def\autkol{V.\,B.~Melekhin, V.\,M.~Khachumov, and~M.\,V.~Khachumov}

\titel{\tit}{\aut}{\autkol}{\titkol}

\vspace*{-10pt}


\noindent
$^1$Dagestan State Technical University, 70A Imam Shamil Ave., Makhachkala 367015, Republic 
of Dagestan 


\noindent
$^2$Ailamazyan Program Systems Institute of the Russian Academy of Sciences, 4A~Petra Pervogo 
Str., Veskovo 152024,\linebreak
$\hphantom{^1}$Yaroslavl Region, Russian Federation

\noindent
$^3$Federal Research Center ``Computer Science and Control'' of the Russian Academy of Sciences, 
44-2~Vavilov\linebreak
$\hphantom{^1}$Str., Moscow 119333, Russian Federation

\noindent
$^4$RUDN University, 6~Miklukho-Maklaya Str., Moscow 117198, Russian Federation


\def\leftfootline{\small{\textbf{\thepage}
\hfill INFORMATIKA I EE PRIMENENIYA~--- INFORMATICS AND
APPLICATIONS\ \ \ 2023\ \ \ volume~17\ \ \ issue\ 2}
}%
 \def\rightfootline{\small{INFORMATIKA I EE PRIMENENIYA~---
INFORMATICS AND APPLICATIONS\ \ \ 2023\ \ \ volume~17\ \ \ issue\ 2
\hfill \textbf{\thepage}}}

\vspace*{3pt}
  



\Abste{One of the effective approaches to organizing the goal-seeking behavior of autonomous 
integral robots in the process of search and explore activities in an a~priori undescribed conditions of 
a~problematic environment is considered. It is proposed to use the procedures of visual-effective 
thinking based on the formalization of the reflex behavior of highly organized living systems as the 
basis for the goal-seeking behavior of robots. A~self-learning algorithm has been developed 
for the conditions with a~high level of uncertainty which allows automatically generating 
conditional programs of expedient behavior that provide autonomous integral robots with the ability 
to achieve a given behavioral goal in the process of search and explore activities. The boundary 
estimates of the functional complexity of the proposed self-learning algorithm under uncertainty 
are found showing the possibility of its implementation on the onboard computer of 
autonomous integral robots which have, as a~rule, limited computing resources. A~modeling of 
self-learning process for an autonomous integral robot in an a~priori undescribed and problematic 
environment was carried out which confirmed the effectiveness of the proposed approach for 
organizing the planning of goal-seeking behavior in an a~priori undescribed and problematic environments.}


\KWE{autonomous integral robot; self-learning algorithm; uncertainty conditions; problematic 
environment; conditional signals}



\DOI{10.14357/19922264230211}{SOFDKW} 

%\vspace*{-11pt}

\Ack
\noindent
This work was supported by the Russian Science Foundation, project No.\,21-71-10056.
  

%\vspace*{4pt}

  \begin{multicols}{2}

\renewcommand{\bibname}{\protect\rmfamily References}
%\renewcommand{\bibname}{\large\protect\rm References}

{\small\frenchspacing
 {%\baselineskip=10.8pt
 \addcontentsline{toc}{section}{References}
 \begin{thebibliography}{9} 
  \bibitem{1-kh-1}
\Aue{Melekhin, V.\,B., and M.\,V.~Khachumov.} 2020. For\-my mysh\-le\-niya av\-to\-nom\-nykh 
in\-tel\-lek\-tu\-al'\-nykh agen\-tov: oso\-ben\-no\-sti i~prob\-le\-my ikh or\-ga\-ni\-za\-tsii [Forms of thinking of 
autonomous intelligent agents: Features and problems of their organization]. \textit{Morskie 
intellektual'nye tekh\-no\-lo\-gii} [Marine Intelligent Technologies] 4-1:224--230. doi: 
10.37220/MIT.2020.50.4.031.
  \bibitem{2-kh-1}
\Aue{Braynes, S.\,N., A.\,N.~Napalkov, and V.\,B.~Svechinskiy}. 1962. \textit{Ney\-ro\-ki\-ber\-ne\-ti\-ka} 
[Neurocybernetics]. Moscow: Gos\-med\-iz\-dat. 172~p.
  \bibitem{3-kh-1}
\Aue{Shingarov, G.\,Kh.} 1986. \textit{Uslov\-nye ref\-lek\-sy i~prob\-le\-ma zna\-ka i~zna\-ch\-eniya} 
[Conditioned reflexes and the problem of sign and meaning]. Moscow: Nauka. 200~p.
  \bibitem{4-kh-1}
\Aue{Melekhin, V.\,B., and M.\,V.~Khachumov.} 2021. Ins\-tru\-men\-tal'\-nye sredst\-va up\-rav\-le\-niya 
tse\-le\-so\-ob\-raz\-nym po\-ve\-de\-ni\-em sa\-mo\-or\-ga\-ni\-zu\-yushchikhsya av\-to\-nom\-nykh in\-tel\-lek\-tu\-al'\-nykh agen\-tov 
[Instrumental means for managing the rational behavior of self-organizing autonomous intelligent 
agents]. \textit{Mekhatronika, avtomatizatsiya, upravlenie} [Mechanatronics, Automation, and Control] 4:171--180. doi: 
10.17587/mau.22.171-180.
  \bibitem{5-kh-1}
\Aue{Karpov, V.\,E., I.\,P.~Karpova, and A.\,A.~Kulinich.} 2019. \textit{So\-tsi\-al'\-nye so\-ob\-shchest\-va 
ro\-bo\-tov} [Social communities of robots]. Moscow: LENAND. 352 p.
  \bibitem{6-kh-1}
\Aue{Guan, B., T.~Xu, Y.~Zhao, Y.~Li, and X.~Dong.} 2021. A~random grouping-based self-
regulating artificial bee colony algorithm for interactive feature detection. \textit{Knowl.-Based 
Syst.} 243:1--12. doi: 10.1016/j.knosys.2022.108434.
  
  \bibitem{8-kh-1}
\Aue{Rutkovskaya, D., M.~Pilin'skiy, and L.~Rutkovskiy.} 2008. \textit{Ney\-ron\-nye se\-ti, 
ge\-ne\-ti\-che\-skie al\-go\-rit\-my i~ne\-chet\-kie sis\-te\-my} [Neural networks, genetic algorithms, and fuzzy 
systems]. Moscow: Goryachaya Liniya\,--\,Telekom. 452~p.

\bibitem{7-kh-1}
\Aue{Simon, D.} 2013. \textit{Evolutionary optimization algorithms}. 1st ed. New York, NY: 
Wiley. 784~p.

\end{thebibliography}

 }
 }

\end{multicols}

\vspace*{-6pt}

\hfill{\small\textit{Received November 2, 2022}} 

\vspace*{-18pt}

\Contr

\noindent
\textbf{Melekhin Vladimir B.} (b.\ 1954)~--- Doctor of Science in technology, professor, 
Department of Software for Computers and Automated Systems, Dagestan State Technical 
University, 70A~Imam Shamil Ave., Makhachkala 367015, Republic of Dagestan; 
\mbox{pashka1602@rambler.ru}

\vspace*{3pt}

\noindent
\textbf{Khachumov Vyacheslav M.} (b.\ 1948)~--- Doctor of Science in technology, 
head of laboratory, Intelligent Control Laboratory, Ailamazyan Program Systems Institute of the
Russian Academy of Sciences, 4A~Petra Pervogo Str., Veskovo 152024, Yaroslavl Region, Russian 
Federation; principal scientist, Federal Research Center ``Computer Science and Control'' of the 
Russian Academy of Sciences, 44-2~Vavilov Str., Moscow 119333, Russian Federation; professor, 
Department of Information Technology, RUDN University, 6~Miklukho-Maklaya Str., Moscow 
117198, Russian Federation; \mbox{vmh48@mail.ru}
\vspace*{3pt}

\noindent
\textbf{Khachumov Mikhail V.} (b.\ 1986)~--- Candidate of Science (PhD) in physics and 
mathematics, senior scientist, Intelligent Control Laboratory, Ailamazyan Program Systems 
Institute of the Russian Academy of Sciences, 4A~Petra Pervogo Atr., Veskovo 152024, Yaroslavl 
Region, Russian Federation; senior scientist, Federal Research Center ``Computer Science and 
Control'' of the Russian Academy of Sciences, 44-2~Vavilov Str., Moscow 119333, Russian 
Federation; associate professor, Department of Information Technology, RUDN University,  
6~Miklukho-Maklaya Str., Moscow 117198, Russian Federation; \mbox{khmike@inbox.ru}

  


\label{end\stat}

\renewcommand{\bibname}{\protect\rm Литература} 
         %11
\def\stat{grusho}

\def\tit{АРХИТЕКТУРНЫЕ РЕШЕНИЯ В~ЗАДАЧЕ ВЫЯВЛЕНИЯ МОШЕННИЧЕСТВА ПРИ~АНАЛИЗЕ 
ИНФОРМАЦИОННЫХ ПОТОКОВ В~ЦИФРОВОЙ ЭКОНОМИКЕ$^*$}

\def\titkol{Архитектурные решения в~задаче выявления мошенничества при~анализе 
информационных потоков в
%~цифровой 
экономике}

\def\aut{А.\,А.~Грушо$^1$, М.\,И.~Забежайло$^2$, Н.\,А.~Грушо$^3$, 
Е.\,Е.~Тимонина$^4$}

\def\autkol{А.\,А.~Грушо, М.\,И.~Забежайло, Н.\,А.~Грушо, 
Е.\,Е.~Тимонина}

\titel{\tit}{\aut}{\autkol}{\titkol}

\index{Грушо А.\,А.}
\index{Забежайло М.\,И.}
\index{Грушо Н.\,А.}
\index{Тимонина Е.\,Е.}
\index{Grusho A.\,A.}
\index{Zabezhailo M.\,I.}
\index{Grusho N.\,A.}
\index{Timonina E.\,E.}


{\renewcommand{\thefootnote}{\fnsymbol{footnote}} \footnotetext[1]
{Работа частично поддержана РФФИ (проекты 18-29-03081 и~18-07-00274).}}


\renewcommand{\thefootnote}{\arabic{footnote}}
\footnotetext[1]{Институт проблем информатики Федерального исследовательского центра <<Информатика и~управление>> 
Российской академии наук, grusho@yandex.ru}
\footnotetext[2]{Институт проблем информатики Федерального исследовательского центра <<Информатика и~управление>> 
Российской академии наук, m.zabezhailo@yandex.ru}
\footnotetext[3]{Институт проблем информатики Федерального исследовательского центра <<Информатика и~управление>> 
Российской академии наук, info@itake.ru}
\footnotetext[4]{Институт проблем информатики Федерального исследовательского центра <<Информатика и~управление>> 
Российской академии наук, eltimon@yandex.ru}

\vspace*{-12pt}
   

 
  
  \Abst{Cформулирован подход к~исследованию некоторых видов мошенничества в~цифровой 
экономике с~использованием причинно-следственных связей. Во всех видах рассматриваемых 
мошенничеств должно наблюдаться несоответствие между целями финансовых транзакций 
и~реальной стоимостью достижения этих целей. Данные о транзакциях можно собирать, 
наблюдая информационные потоки, в~которых отражаются эти транзакции. Архитектура сбора 
данных и~их анализа может быть организована с~помощью распределенных реестров 
с~централизованным консенсусом, что позволяет создать аналог электронной бухгалтерской 
книги, фиксирующей финансово-экономическую деятельность субъектов цифровой экономики в~регионе. 
  Рассматриваемые методы выявления мошенничества основаны на противоречиях 
между действиями, описанными в~транзакциях, и~информацией, содержащейся в~планах, 
стандартах, прецедентах и~др. Рассмотрен метод, основанный на некоторой упрощенной схеме 
реализации абстрактного проекта. Для выявления противоречий необходимо проводить анализ 
от следствия к~причине, т.\,е.\ искать аномалии в~информации, описывающей порождение 
наблюдаемых следствий. 
  Показано, как в~реализации проекта можно выделять простые <<необходимые условия>> 
нарушения при\-чин\-но-след\-ст\-вен\-ных связей, т.\,е.\ множество <<необходимых условий>>, 
нарушение которых свидетельствует о наличии мошенничества. Это множество <<необходимых 
условий>> можно назвать метаданными для контроля проекта на выявление мошенничества.} 
 
 
  \KW{цифровая экономика; информационные потоки; при\-чин\-но-след\-ст\-вен\-ные связи; 
выявление мошеннических схем} 

\DOI{10.14357/19922264190204}
  
\vspace*{-4pt}


\vskip 10pt plus 9pt minus 6pt

\thispagestyle{headings}

\begin{multicols}{2}

\label{st\stat}

\section{Введение}

\vspace*{3pt}

  В работе сформулирован подход к~исследованию некоторых видов 
мошенничества в~цифровой экономике с~использованием  
при\-чин\-но-след\-ст\-вен\-ных связей. Рассматриваются три вида мошенничества, 
а именно:
  \begin{enumerate}[(1)]
\item отмыв денег; 
\item обман при выполнении договорных обязательств при реализации 
технических проектов (строительные проекты и~др.); 
\item незаконный вывод денег. 
\end{enumerate}

  Названные виды мошенничества могут быть сведены к~решению одного типа 
задач. Для отмывания денег источник должен заключать фиктивные контракты, 
в~соответствии с~которыми будут переводиться средства за заведомо ненужную 
работу и~материалы. 
  
  Мошенничество, связанное с~невыполнением договорных обязательств, связано 
со снижением качества услуг, качества и~количества закупаемых 
материалов, выполнением работ с~ненадлежащим качеством. 
  
  Вывод денег связан с~переводом средств фир\-мам-од\-но\-днев\-кам, которые 
заведомо не могут выполнить обязательства по контрактам, за которые им 
переводятся средства. 
  
  Таким образом, во всех трех видах рассматриваемых мошенничеств должно 
наблюдаться несоответствие между целями финансовых транзакций и~реальной 
стоимостью достижения этих целей. Данные о транзакциях можно собирать, 
наблюдая информационные потоки, в~которых отражаются эти транзакции. 
  
  Однако для наблюдения таких информационных потоков необходимо создавать 
архитектуру\linebreak телекоммуникационной системы, позволяющей перехватывать 
и~собирать данные о всех транзакциях. Например, такая архитектура может быть 
организована с~помощью распределенных реестров с~централизованным 
консенсусом, т.\,е.\ все информационные потоки, сформированные в~цифровой 
экономике и~несущие информацию о транзакциях, проходят через некоторый 
центральный узел, запоминающий их в~форме распределенного реестра. Такие 
реестры могут дублироваться в~аналогичных центрах различных регионов, что 
позволяет создать аналог электронной бухгалтерской книги, фиксирующей 
фи\-нан\-со\-во-эко\-но\-ми\-че\-скую деятельность субъектов цифровой экономики. Такой 
подход предложено реализовать на базе системы ситуационных центров, что 
отражено в~работах~[1, 2].
  
  Собранная из информационных потоков информация о~транзакциях, т.\,е.\ 
о~контрактах, договорах, платежах, отчетах, закупленных материалах, 
характеристиках исполнителей работ и~др., собирается в~базе данных в~указанном 
центре. Согласно теории интеллектуальных сис\-тем~[3], эту базу данных можно 
называть базой фактов (БФ). Базу фактов можно представить как бинарную мат\-ри\-цу, 
строки которой описывают характеристики, входящие в~транзакции, а столбцы 
нумеруются характеристиками. Строки матрицы будем называть 
\textit{объектами}~[4, 5]. 
  
  Рассматриваемые в~работе методы выявления мошенничества будут основаны 
на противоречиях между действиями, описанными в~транзакциях, и~информацией, 
содержащейся в~планах, стандартах, прецедентах и~др. Для нахождения 
противоречий в~архитектуре центра предусмотрена другая база данных~--- база 
знаний (БЗ)~\cite{3-gr, 6-gr}, которая устроена так же, как БФ. 
  
  Информация в~БЗ собирается на основе положительного опыта или расчетов. 
Используя БЗ, можно выводить факты нарушения при\-чин\-но-след\-ст\-вен\-ных 
связей. Нарушения при\-чин\-но-след\-ст\-вен\-ных связей будем называть 
\textit{аномалиями}. 
  
  Для упрощения дальнейшее изложение будет вестись в~рамках поиска 
противоречий при выполнении некоторого абстрактного проекта. Выявление 
аномалий будет происходить на основе фактов из БФ с~помощью знаний из БЗ 
методами искусственного интеллекта и~интеллектуального анализа 
данных~\cite{6-gr}. 

\vspace*{-10pt}
  
  \section{Модели}
  
  \vspace*{-3pt}
  
  Наиболее сложная из рассмотренных выше задач~--- выявление противоречий, 
т.\,е.\ использование БЗ для получения новых знаний и~выявление аномалий из 
полученных фактов. 
  
  Все способы выявления противоречий основаны на определении 
  причинно-следственных связей. При этом противоречия в~параметрах транзакций по 
отношению к~требуемым в~БЗ составляют сущность аномалий. 
  
   Далее будет рассмотрен метод, основанный на некоторой упрощенной схеме 
реализации абстрактного проекта. 
  
  Каждый проект имеет цель: например, цель представляет собой построение 
некоторой системы. Воспользуемся структурным подходом, который позволяет 
строить проект на основе разбиения системы на подсистемы и~определения 
взаимодействий подсистем~\cite{7-gr}. При этом каждая подсистема также 
представима структурной моделью. 
  
  Как сама система, так и~каждая ее подсистема имеют свой функционал 
и~спецификацию, па\-ра\-мет\-ры настройки и~домены параметров настройки. Кроме 
этих характеристик существует множество характеристик, связанных 
с~<<жизненным циклом>> создания системы. Сюда входят работы, ресурсы, 
сроки выполнения работ по созданию подсистем и~самой системы, стоимости 
компонентов и~материалов, стоимости работ, схемы поставок, договорные 
обязательства и~др. Все характеристики связаны между собой, поэтому можно 
говорить о стоимости и~времени изготовления структурных компонентов системы. 
  
  Одной из важнейших характеристик является смета (система смет для 
подсистем). Смета сопоставляет каждому компоненту системы стоимость его 
изготовления и~настройки. 
  
  Схема построения системы может быть пред\-став\-ле\-на диаграммой, 
изображенной на рис.~1. 

{ \begin{center}  %fig1
 \vspace*{9pt}
   \mbox{%
 \epsfxsize=79mm 
 \epsfbox{gru-1.eps}
 }


\vspace*{9pt}


\noindent
{{\figurename~1}\ \ \small{Диаграмма достижения цели}}
\end{center}
}

\vspace*{9pt}

\addtocounter{figure}{1}
  
  


  Представленная на рис.~1 диаграмма позволяет описать основные классы 
возможных противоречий при достижении цели. Противоречия возникают, когда 
данные БФ не соответствуют требуемым характеристикам. 
  
  
  \section{Потенциальные классы аномалий при~достижении цели}
  
  Выделим четыре потенциальных класса противоречий, которые показывают, 
каким образом нужно искать эти противоречия.
  
 
  Противоречие цели и~проекта (рис.~2) возникает при отсутствии обоснования 
или в~случае логического противоречия между возможностями проектируемого 
функционала и~целью системы. Отметим, что в~проект входят сроки, перечень 
работ, материалы, настройки, которые описываются соответствующими 
параметрами и~допустимыми значениями этих параметров. Проект формируется 
на основе БЗ и~расчетов, исходя из информации, полученной по аналогии 
с~другими проектами и~решениями, которые считаются апробированными. 
  
  Отметим, что цель порождает проект и~в этом смысле является причиной 
проекта. Однако для анализа противоречий необходимо двигаться по штриховой 
стрелке диаграммы (см.\ рис.~2) от проекта к~цели. В~самом деле, любой компонент 
проекта направлен на теоретическое достижение цели. Цель~--- сложный объект, 
поэтому в~проекте могут возникнуть характеристики, противоречащие хотя бы 
некоторым характеристикам цели. Это делает проект противоречивым, но вывод 
об этом может быть сделан только на уровне описания цели. 
  

  Противоречия между проектом и~его реализацией, исключая настройки 
(рис.~3), могут возникать, например, при закупке исполнителем материалов более 
низкого качества по более низким ценам, при попытках достижения требуемых 
сроков работы за счет снижения качества выполнения работ, за счет нахождения 
<<объективных>> причин для увеличения сроков работы и,~следовательно, 
увеличения цены реализации проекта. 


  Для выявления указанных противоречий необходимо двигаться по диаграмме 
(см.\ рис.~3) в~обратную сторону в~соответствии со~штриховыми стрелками. 
Действительно, выявить противоречия между характеристиками закупленных 
материалов и~требуемыми по проекту можно только при обращении к~проекту 
и~его спецификациям. Манипуляции со сроками работы также можно выявить 
только при обращении к~соответствующим расчетам в~проекте. Задержки в~сроках 
работы, связанные с~поставками материалов, можно определить только на 
предыдущем этапе диаграммы (см.\ рис.~3) в~описании проекта. 


  


  Противоречия между реализацией проекта и~его настройкой (рис.~4) возникает, 
когда не удается добиться требуемых значений параметров функционала, не 
удается обеспечить необходимый уровень\linebreak\vspace*{-12pt}

{ \begin{center}  %fig2
 \vspace*{-6pt}
   \mbox{%
 \epsfxsize=16mm 
 \epsfbox{gru-2.eps}
 }


\vspace*{6pt}


\noindent
{{\figurename~2}\ \ \small{Противоречия цели и~проекта}}
\end{center}
}

%\vspace*{9pt}

\addtocounter{figure}{1}

{ \begin{center}  %fig3
 \vspace*{6pt}
    \mbox{%
 \epsfxsize=79mm 
 \epsfbox{gru-3.eps}
 }


\end{center}

\vspace*{-2pt}


\noindent
{{\figurename~3}\ \ \small{Противоречия проекта и~его реализации (без настройки)}}
}

\vspace*{6pt}

\addtocounter{figure}{1}

{ \begin{center}  %fig4
 \vspace*{1pt}
   \mbox{%
 \epsfxsize=54.5mm 
 \epsfbox{gru-4.eps}
 }


\end{center}


\noindent
{{\figurename~4}\ \ \small{Противоречия реализации проекта и~его на\-стройки}}
}

%\vspace*{9pt}

\addtocounter{figure}{1}

{ \begin{center}  %fig5
 \vspace*{5pt}
    \mbox{%
 \epsfxsize=79mm 
 \epsfbox{gru-5.eps}
 }


\end{center}



\noindent
{{\figurename~5}\ \ \small{Противоречия цели и~достигнутой реализации проекта}}
}

\vspace*{6pt}

\addtocounter{figure}{1}

\noindent
 качества реализации проекта. Для 
определения противоречия в~настройках надо опять же двигаться по диаграмме 
(см.\ рис.~4) в~обратную сторону по штриховым стрелкам, так как для выявления 
характеристик результатов работы, которые не дают возможности реализации 
определенного функционала, необходимо иметь информацию о результатах этой 
работы. 


  



  Противоречие между целью и~достигнутой реализацией проекта (рис.~5) 
возникает, когда реализованная система не позволяет достичь цели. В~этом случае 
опять противоречие нужно искать, двигаясь от цели к~реальному достигнутому 
функционалу по штриховой стрелке (см.\ рис.~5).
  
  Суммируя положения, изложенные в~данном разделе, приходим к~выводу, что 
для выявления противоречий необходимо проводить анализ от следствия 
к~причине, т.\,е.\ искать аномалии в~информации, описывающей порождение 
наблюдаемых следствий. 
  
  
  \section{Связь противоречий и~причин}
  
  Прежде чем построить связь между причинами и~противоречиями, кратко 
опишем простейшую модель связи этих понятий. Причины и~противоречия будут 
сформулированы для представления компонентов системы как объектов, 
обладающих наборами известных характеристик~\cite{4-gr, 5-gr}. 
  
  Пусть $U\hm=\{\alpha, \beta, \ldots\}$~--- совокупность характеристик 
(пространство характеристик). Согласно~\cite{4-gr} \textit{объектом}~$O$ 
называется любое подмножество характеристик $O\hm\subseteq U$. Рассмотрим 
последовательность объектов, возможно в~различных пространствах 
характеристик. 
  
  \smallskip
  
  \noindent
  \textbf{Определение~1.}\ Объект~$P$ с~числом характеристик, большим или 
равным~2, является \textit{причиной} объекта (\textit{свойства})~$B$ в~цепочке 
наблюдаемых объектов тогда и~только тогда, когда выполнены следующие 
условия:
  \begin{enumerate}[(1)]
\item для каждого объекта~$C$, если $P\hm\subseteq C$, то $C\mapsto B$, где 
$C\mapsto B$ означает, что объект~$B$ присутствует в~объекте, следующем за 
объектом~$C$;
\item объект~$P$ является минимальным объектом, удовлетворяющим 
условию~1, а~именно: $\forall \alpha\hm\in P$ объект~$P\backslash \{\alpha\}$ 
не является причиной, т.\,е.\ $\exists C:\ \alpha\not\in C$, $P\backslash 
\{\alpha\}\hm\subseteq C$ и~$C\not\mapsto B$, где $C\not\mapsto B$ означает, 
что~$B$ не может содержаться в~объекте, следующем за объектом~$C$. 
\end{enumerate}

  Приведенное определение причины является упрощением причин, 
возникающих в~реальном мире. Например, реальные причины могут возникать\linebreak 
как совокупность характеристик из разных пространств. Одно следствие может 
порождаться разными причинами или возникать из внешних\linebreak и~ненаблюдаемых 
характеристик. Однако пред\-став\-лен\-ная далее формализация позволяет доступно 
изложить при\-чин\-но-след\-ст\-вен\-ные истоки противоречий, которые 
инициируют в~дальнейшем глубокое исследование рассматриваемых процессов.
  
  Будем считать, что для любого интересующего нас свойства~$B$ существует 
причина. Тогда справедлива следующая теорема.
  
  \smallskip
  
  \noindent
  \textbf{Теорема~1.}\ \textit{Для любого свойства~$B$ существует 
единственная причина}. 
  
  \smallskip
  
  \noindent
  Д\,о\,к\,а\,з\,а\,т\,е\,л\,ь\,с\,т\,в\,о\,.\ \ Доказательство будем вести от противного, 
т.\,е.\ предположим, что существуют две причины свойства~$B$: $P$ 
и~$P^\prime$, $P\hm\not= P^\prime$. Тогда существует $\alpha\hm\in U$, которое 
удовлетворяет одному из двух условий:
  \begin{itemize}
\item[(а)] $\alpha\in P$, $\alpha\notin P^\prime$;
\item[(б)] $\alpha\notin P$, $\alpha \in P^\prime$.
\end{itemize}

  Пусть выполняется условие~(б). Тогда $P^\prime\backslash \{\alpha\}$ не 
является причиной по условию~2 определения~1, т.\,е.\ $\exists C$ такое, что 
$\alpha\notin C$, $P^\prime\backslash \{\alpha\}\hm\subseteq C$ и~$C\not\mapsto B$. 
Но если~$B$ произошло и~$P$ его причина, то $C\mapsto B$, что противоречит 
предположению. Теорема~1 доказана.
  
  \smallskip
  
  \noindent
  \textbf{Лемма.} \textit{Если $P$~--- причина появления свойства~$B$, то 
объект~$B$ определяет существование свойства~$P$ в~объекте, 
предшествующем~$B$. }
  
  \smallskip
  
  \noindent
  Д\,о\,к\,а\,з\,а\,т\,е\,л\,ь\,с\,т\,в\,о\,.\ \ Из предположения, что у~каж\-до\-го 
свойства~$B$ есть причина, и~условия, что~$P$ является причиной~$B$, следует, 
что при появлении в~данных свойства~$B$ объект~$C$, предшествующий 
появлению~$B$, содержит как часть объект~$P$. Это следует из теоремы~1 
и~определения причины. 
  
  Докажем принцип <<необходимого условия>>, который, несмотря на простоту 
доказательства, будет играть в~дальнейшем существенную роль.
  
  \smallskip
  
  \noindent
  \textbf{Теорема~2.} \textit{Если~$P$~--- причина появления свойства~$B$ 
и~$A\hm\subseteq P$, то объект~$B$ определяет наличие свойства~$A$ 
в~объекте, предшествующем~$B$}. 
  
  \smallskip
  
  \noindent
  Д\,о\,к\,а\,з\,а\,т\,е\,л\,ь\,с\,т\,в\,о\,.\ \ Пусть в~данных имеется объект~$B$ 
и~$P\mapsto B$, тогда в~силу существования и~единственности причины~$B$ 
в~данных должен существовать объект~$C$, предшествующий~$B$ 
и~содержащий причину~$P$. Поскольку $A\hm\subseteq P$ и~$B$ содержит 
причину~$P$, то $B\mapsto A$. С~учетом леммы теорема~2 доказана.
  
  \smallskip
  
  Пусть даны пространства $U_1, U_2,\ldots$ и~имеется последовательность 
данных (процесс выполнения этапов проекта в~соответствии с~рис.~1) $A, B, 
\ldots$, где каждый объект является подмножеством некоторого 
пространства~$U_i$, $i\hm=1,\ldots$ Тогда в~объекте~$A$ присутствует 
причина~$P$ появления интересующего нас свойства~$C$ в~объекте~$B$. Пусть 
$P\hm\subseteq A$, тогда по теореме~2 $\forall \alpha\hm\in P$:  
$C\mapsto \{\alpha\}$, т.\,е.\ из появления~$C$ следует появление 
характеристики~$\alpha$ в~предшествующем объекте. Это необходимое условие 
того, что~$C$ удовлетворяет причинно-следственным связям развития процесса 
выполнения проекта. Если для~$C$ нет характеристики~$\alpha$, которую можно 
отнести к~причине~$C$, то можно считать, что нарушена  
при\-чин\-но-след\-ст\-вен\-ная связь и~$C$~--- аномальный объект. 
  
  \smallskip
  
  \noindent
  \textbf{Пример.} Если объект~$C$ состоит в~получении суммы~$a$ 
фирмой~$K$, то согласно теореме~2 в~пред\-шест\-ву\-ющем объекте должна 
существовать причина перевода суммы~$a$ на фирму~$K$. Если эта причина 
в~проекте отсутствует, то это можно считать признаком мошеннической схемы. 
Все проекты по предположению собираются из <<кубиков>>, содержащихся в~БЗ. 
Тогда можно сравнить цену объекта~$C$, породившего получение суммы~$a$, 
и~сумму, присутствующую в~смете проекта. Если разница велика, то это либо 
ошибка проекта, либо признак мошеннической схемы.
  
  \section{Поиск противоречий на~основе~принципа <<необходимых~условий>>}
   
  Как было показано в~разд.~3, нахождение противоречий соответствуют 
движению от следствия к~причине. Для каждого объекта в~наблюдаемых данных 
выявление причин его появления является трудоемкой задачей. Кроме того, при 
реализации контроля соблюдения при\-чин\-но-след\-ст\-вен\-ных связей на 
большом множестве участников экономической деятельности задача анализа 
причин становится трудоемкой. Поэтому процедуру контроля необходимо разбить 
на два этапа, где первый этап состоит в~анализе простых <<необходимых 
условий>> проявления мошенничества, когда используется хотя бы одна 
известная характеристика причины. Второй этап (в~режиме офлайн) состоит 
в~выявлении причин, позволяющих провести анализ источников мошеннических 
схем. 
  
  Один из подходов к~выбору <<необходимых условий>> состоит в~построении 
множества подцелей исходной цели проекта (структурный метод построения 
проекта~\cite{7-gr}). Каждая подцель описывается диаграммой на рис.~1, 
и~реализации подцелей должны образовывать полный функционал цели. Это 
является необходимым, но не достаточным условием достижения цели, так как 
при таком подходе отсутствует компонент согласования всех подцелей в~единую 
систему. Однако такой подход значительно упрощает анализ выполнения проекта 
на предмет поиска мошенничества. Если признаки мошенничества будут 
обнаружены в~реализации хотя бы одной из подцелей, то это значит, что 
мошенничество присутствует в~реализации всего проекта. 
  
  Аналогично в~реализации каждого этапа в~любой из подцелей можно выделять 
простые <<необходимые условия>> нарушения при\-чин\-но-след\-ст\-венн\-ых 
связей. 
  
  Таким образом, получается множество <<необходимых условий>>, нарушение 
которых свидетельствует о наличии мошенничества. Это множество 
<<необходимых условий>> можно назвать метаданными~[8, 9] для контроля 
проекта на выявление мошенничества. 
  
  
  \section{Заключение }
  
  В поиске противоречий необходимо от транзакций, соответствующих 
следствиям при\-чин\-но-след\-ст\-вен\-ных связей, переходить к~анализу причин 
наблюдаемых следствий. Это сложная задача, которая связана с~описанием причин 
определенных свойств. 
  
  В работе представлена модель, позволяющая строить множество необходимых 
условий соответствия наблюдаемого следствия вызвавшей его причине. Этот 
подход делает поиск противоречий вполне вычислимой задачей, но не гарантирует 
успех. 
  
  {\small\frenchspacing
 {%\baselineskip=10.8pt
 \addcontentsline{toc}{section}{References}
 \begin{thebibliography}{9}
\bibitem{1-gr}
\Au{Грушо А.\,А., Зацаринный~А.\,А., Тимонина~Е.\,Е.} Блокчейны цифровой экономики на базе 
системы ситуационных центров и~централизованного консенсуса~// Радиолокация, навигация, 
связь: Мат-лы XXV Междунар. научн.-технич. конф.~---
Воронеж: Издательский дом ВГУ, 2019. Т.~6. С.~183--191. 
\bibitem{2-gr}
\Au{Grusho A., Zatsarinny~A., Timonina~E.} A~system approach to information security in 
distributed ledgers on the situational centers platform.~---
Lecture notes in computer science ser.~--- Springer, 2019 
(in press).
\bibitem{3-gr}
\Au{Финн В.\,К.} Искусственный интеллект: Методология, применения, философия.~--- М.: 
Красанд, 2011. 448~с.

\bibitem{5-gr} %4
\Au{Аншаков~О.\,М., Фабрикантова~Е.\,Ф.} ДСМ-ме\-тод автоматического порождения 
гипотез: Логические и~эпистемологические основания.~--- М.: Либроком, 2009. 432~с.

\bibitem{4-gr} %5
\Au{Poelmans J., Elzinga~P., Viaene~S., Dedene~G.} Formal concept analysis in knowledge 
discovery: A~survey~// Conceptual structures: From information to intelligence~/ Eds.\ M.~Croitoru, 
S.~Ferr$\acute{\mbox{e}}$, and D.~Lukose.~--- Lecture notes in computer science 
ser.~--- Berlin--Heidelberg: Springer, 2010. Vol.~6208.  P.~139--153.

\bibitem{6-gr}
\Au{Панкратова~Е.\,С., Финн~В.\,К.} Автоматическое по\-рож\-де\-ние гипотез в~интеллектуальных 
системах.~--- М.: Либроком, 2009. 528~с. 
\bibitem{7-gr}
\Au{Денисов А.\,А., Колесников~Д.\,Н.} Теория больших систем управления.~--- Л.: Энергоиздат, 1982. 488~с.

\bibitem{9-gr}
\Au{Грушо А.\,А., Грушо Н.\,А., Забежайло~М.\,И., Смирнов~Д.\,В., Тимонина~Е.\,Е.} 
Параметризация в~прикладных задачах поиска эмпирических причин~// Информатика и~её 
применения, 2018. Т.~12. Вып.~3. С.~62--66.

\bibitem{8-gr}
\Au{Грушо А.\,А., Грушо Н.\,А., Левыкин~М.\,В., Тимонина~Е.\,Е.} Методы идентификации 
захвата хоста в~распределенной ин\-фор\-ма\-ци\-он\-но-вы\-чис\-ли\-тель\-ной сис\-те\-ме, 
защищенной с~помощью метаданных~// Информатика и~её применения, 2018. Т.~12. Вып.~4. 
С.~41--45.

 \end{thebibliography}

 }
 }

\end{multicols}

\vspace*{-3pt}

\hfill{\small\textit{Поступила в~редакцию 03.04.19}}

%\vspace*{8pt}

%\pagebreak

\newpage

\vspace*{-28pt}

%\hrule

%\vspace*{2pt}

%\hrule

%\vspace*{-2pt}

\def\tit{ARCHITECTURAL DECISIONS IN~THE~PROBLEM 
OF~IDENTIFICATION OF~FRAUD IN~THE~ANALYSIS 
OF~INFORMATION FLOWS IN~DIGITAL ECONOMY\\[-5pt]}


\def\titkol{Architectural decisions in~the~problem 
of~identification of~fraud in~the~analysis 
of~information flows in~digital economy}

\def\aut{A.\,A.~Grusho, M.\,I.~Zabezhailo, N.\,A.~Grusho, and~E.\,E.~Timonina}

\def\autkol{A.\,A.~Grusho, M.\,I.~Zabezhailo, N.\,A.~Grusho, and~E.\,E.~Timonina}

\titel{\tit}{\aut}{\autkol}{\titkol}

\vspace*{-13pt}


 \noindent
   Institute of Informatics Problems, Federal Research Center ``Computer Sciences and 
Control'' of the Russian Academy of Sciences; 44-2~Vavilov Str., Moscow 119133, 
Russian Federation

\def\leftfootline{\small{\textbf{\thepage}
\hfill INFORMATIKA I EE PRIMENENIYA~--- INFORMATICS AND
APPLICATIONS\ \ \ 2019\ \ \ volume~13\ \ \ issue\ 2}
}%
 \def\rightfootline{\small{INFORMATIKA I EE PRIMENENIYA~---
INFORMATICS AND APPLICATIONS\ \ \ 2019\ \ \ volume~13\ \ \ issue\ 2
\hfill \textbf{\thepage}}}

\vspace*{3pt}


   
     
   \Abste{An approach to a~research of some types of fraud in digital economy with the usage of relationships of 
cause and effect is formulated. In all types of the considered frauds, the discrepancy between the 
purposes of financial transactions and actual cost of achievement of these purposes
has to be observed. Data on 
transactions can be collected by observing information flows in which these transactions are reflected. 
The architecture of data collection and their analysis can be organized by means of the distributed 
ledgers with the centralized consensus that allows creating an analog of the electronic account book 
fixing financial and economic activity of subjects of digital economy in the region. 
   The methods of fraud identification considered are based on the contradictions 
between actions described in transactions and information, which is contained in plans, standards, 
precedents, etc. 
   The method based on a~simplified scheme of implementation of the abstract project is considered. 
For identification of contradictions, it is necessary to carry out the analysis from the effect to the cause, 
i.\,e., to look for anomalies in information describing the generation of the observed effects. 
   It is shown how in implementation of the project it is possible to allocate simple ``necessary 
conditions'' of violation of cause and effect relationships, i.\,e., a~set of ``necessary conditions'' 
violation of which demonstrates fraud existence. It is possible to call this set of "necessary conditions" 
by metadata for control of the project for fraud identification.} 
   
   \KWE{digital economy; information flows; relationships of reason and effect; detection of 
fraudulent schemes}
   
  

 \DOI{10.14357/19922264190204}

\vspace*{-20pt}

 \Ack
   \noindent
   The work was partially supported by the Russian Foundation for Basic Research (projects  
18-29-03081 and 18-07-00274).



%\vspace*{6pt}

  \begin{multicols}{2}

\renewcommand{\bibname}{\protect\rmfamily References}
%\renewcommand{\bibname}{\large\protect\rm References}

{\small\frenchspacing
 {\baselineskip=10.5pt
 \addcontentsline{toc}{section}{References}
 \begin{thebibliography}{9}
\bibitem{1-gr-1}
\Aue{Grusho, A.\,A., A.\,A.~Zatsarinny, and E.\,E.~Timonina.} 2019. Blokcheyny tsifrovoy ekonomiki 
na baze sistemy situatsionnykh tsentrov i~tsentralizovannogo konsensusa [Blockchains of digital 
economy on the basis of the system of the situational centres and the centralized consensus]. 
\textit{25th Scientific and Technical Conference (International) ``Radar-Location, Navigation, 
Communication'' Proceedings}. Voronezh: VSU Publs. 6:183--191.
\bibitem{2-gr-1}
\Aue{Grusho, A., A.~Zatsarinny, and E.~Timonina.} 2019 (in press). 
A~system approach to information security 
in distributed ledgers on the situational centers platform. 
Lecture notes in computer science ser. Springer.
\bibitem{3-gr-1}
\Aue{Finn, V.\,K.} 2011. \textit{Iskusstvennyy intellekt: Metodologiya, primeneniya, filosofiya} 
[Artificial intelligence: Methodology, applications, philosophy]. Moscow: KRASAND. 448~p.

\bibitem{5-gr-1}
\Aue{Anshakov, O.\,M., and E.\,F.~Fabrikantova}. 2009. \textit{DSM-metod avtomaticheskogo porozhdeniya gipotez: Logicheskie 
i~epistemologicheskie osnovaniya} [JSM-method of automatic hypothesis generation: Logical and 
epistemological]. Moscow: KD LIBROKOM. 432~p.
\bibitem{4-gr-1} %5
\Aue{Poelmans, J., P.~Elzinga, S.~Viaene, and G.~Dedene.} 2010. Formal concept analysis in 
knowledge discovery: A~survey. \textit{Conceptual structures: From information to intelligence}. 
Eds.\ M.~Croitoru, S.~Ferr$\acute{\mbox{e}}$, and D.~Lukose. Lecture notes in 
computer science ser. Berlin--Heidelberg: Springer. 6208:139--153.

\bibitem{6-gr-1}
\Aue{Pankratov, E.\,S., and V.\,K.~Finn}. 
2009. \textit{Avtomaticheskoe porozhdenie gipotez v~intellektual'nykh 
sistemakh} [Automatic hypotheses generation in intelligent systems]. Moscow: KD 
\mbox{LIBROKOM}.  528~p. 
\bibitem{7-gr-1}
\Aue{Denisov, A.\,A., and D.\,N.~Kolesnikov.} 1982. \textit{Teoriya bol'shikh 
sistem upravleniya} [Theory of big control systems]. Leningrad: Energoizdat. 488~p.

\bibitem{9-gr-1}
\Aue{Grusho, A.\,A., N.\,A.~Grusho, M.\,I.~Zabezhailo, D.\,V.~Smirnov, and 
E.\,E.~Timonina.} 2018. 
Parametrizatsiya v~prikladnykh zadachakh poiska empiricheskikh prichin 
[Parametrization in applied 
problems of search of the empirical reasons]. 
\textit{Informatika i~ee Primeneniya~--- 
Inform. Appl.} 12(3):62--66.

\bibitem{8-gr-1}
\Aue{Grusho, A.\,A., N.\,A.~Grusho, M.\,V.~Levykin, and E.\,E.~Timonina.} 2018. Metody 
identifikatsii zakhvata khosta v~raspredelennoy informatsionno-vychislitel'noy sisteme, 
zashchishchennoy s~pomoshch'yu metadannykh [Methods of identification of host capture 
in the  distributed information system which is protected on the base of meta data].
\textit{Informatika i~ee 
Primeneniya~--- Inform. Appl.} 12(4):41--45.
{ %\looseness=1

}

\end{thebibliography}

 }
 }

\end{multicols}

\vspace*{-12pt}

\hfill{\small\textit{Received April 3, 2019}}

%\pagebreak

%\vspace*{-18pt}

\Contr

\noindent
\textbf{Grusho Alexander A.} (b.\ 1946)~--- Doctor of Science in physics and 
mathematics, professor, principal scientist, Institute of Informatics Problems, 
Federal Research Center ``Computer Sciences and Control'' of the Russian 
Academy of Sciences; 44-2~Vavilov Str., Moscow 119133, Russian Federation; 
\mbox{grusho@yandex.ru} 

\vspace*{3pt}

\noindent
\textbf{Zabezhailo Michael I.} (b.\ 1956)~--- Doctor of Science in physics and 
mathematics, principal scientist, Institute of Informatics Problems, Federal Research 
Center ``Computer Sciences and Control'' of the Russian Academy of Sciences;  
44-2~Vavilov Str., Moscow 119133, Russian Federation; 
\mbox{m.zabezhailo@yandex.ru} 

\vspace*{3pt}


\noindent
\textbf{Grusho Nikolai A.} (b.\ 1982)~--- Candidate of Science (PhD) in physics 
and mathematics, senior scientist, Institute of Informatics Problems, Federal 
Research Center ``Computer Sciences and Control'' of the Russian Academy of 
Sciences; 44-2~Vavilov Str., Moscow 119133, Russian Federation; 
\mbox{info@itake.ru} 

\vspace*{3pt}


\noindent
\textbf{Timonina Elena E.} (b.\ 1952)~--- Doctor of Science in technology, 
professor, leading scientist, Institute of Informatics Problems, Federal Research 
Center ``Computer Sciences and Control'' of the Russian Academy of Sciences;  
44-2~Vavilov Str., Moscow 119133, Russian Federation; 
\mbox{eltimon@yandex.ru} 

\label{end\stat}

\renewcommand{\bibname}{\protect\rm Литература}          %12
\def\stat{nuriev}

\def\tit{МЕТОДОЛОГИЯ КОРПУСНО-ОРИЕНТИРОВАННОГО ИССЛЕДОВАНИЯ 
В~ОБЛАСТИ КОНТРАСТИВНОЙ ПУНКТУАЦИИ$^*$\\[-5pt]}

\def\titkol{Методология корпусно-ориентированного исследования 
в~области контрастивной пунктуации}

\def\aut{В.\,А.~Нуриев$^1$, В.\,И.~Карпов$^2$}

\def\autkol{В.\,А.~Нуриев, В.\,И.~Карпов}

\titel{\tit}{\aut}{\autkol}{\titkol}

\index{Нуриев В.\,А.}
\index{Карпов В.\,И.}
\index{Nuriev V.\,A.}
\index{Karpov V.\,I.}


{\renewcommand{\thefootnote}{\fnsymbol{footnote}} \footnotetext[1]
{Работа выполнена за счет гранта Российского научного фонда (проект 23-28-00548) с~использованием инфраструктуры 
Центра коллективного пользования <<Высокопроизводительные вычисления и~большие данные>> (ЦКП 
<<Информатика>>) ФИЦ ИУ РАН (г.~Москва).}}


\renewcommand{\thefootnote}{\arabic{footnote}}
\footnotetext[1]{Федеральный исследовательский центр <<Информатика и~управление>> Российской академии наук, 
\mbox{nurieff.v@gmail.com}}
\footnotetext[2]{Институт языкознания Российской академии наук; Федеральный исследовательский центр <<Информатика 
и~управ\-ле\-ние>> Российской академии наук, \mbox{wi.karpow@gmail.com}}

\vspace*{-3pt}

  
  
    
  \Abst{Уточняется  подход к~современным исследованиям 
в~об\-ласти контрастивной пунктуации с~точки зрения методологии. С~учетом новейших достижений информатики, 
компьютерной лингвистики и~теории перевода такие исследования очевидным образом 
должны иметь кор\-пус\-но-ори\-ен\-ти\-ро\-ван\-ный характер. В~данной статье представлена 
методологическая схема подобного исследования, направленного на выявление 
межъязыковой пунктуационной асим\-мет\-рии посредством сравнения функционального 
диапазона одного и~того же знака препинания в~разных языках. Показываются основные 
методологические тенденции, характерные для этой научной об\-ласти. Внимание 
уделяется особенностям корпусной методологии при контрастивном изучении 
пунктуации. В~качестве одного из современных методологических инструментов 
предлагаются надкорпусные базы данных (НБД), раз\-ра\-ба\-ты\-ва\-емые в~ФИЦ ИУ РАН.}

%\vspace*{-6pt}
  
  \KW{контрастивная пунктуация; перевод; корпусное переводоведение; кор\-пус\-но-ори\-ен\-ти\-ро\-ван\-ное 
  исследование; параллельный корпус; надкорпусная база данных; 
межъязыковая асим\-мет\-рия; методология}

%\vspace*{-6pt}

\DOI{10.14357/19922264230213}{VBOZAO} 
  
%\vspace*{-3pt}


\vskip 10pt plus 9pt minus 6pt

\thispagestyle{headings}

\begin{multicols}{2}

\label{st\stat}
    
    \section{Введение}
    
    \vspace*{-3pt}
    
  Важность и~необходимость исследований в~области контрастивной 
пунктуации в~научной литературе отмечалась неоднократно (см., 
например,~[1--7]). Обычно эта необходимость выводится из нужд 
переводческой практики, которая предполагает при обработке письменного 
текста обязательную речемыслительную программу, связанную с~исходным 
пунктуационным компонентом и~его переносом в~сис\-те\-му переводящего 
языка. Так, Ньюмарк в~своем <<Учебнике перевода>> пишет, что 
<<пунктуация может быть мощнейшим инструментом, но ее настолько легко 
упус\-тить из виду, что я~советую переводчикам: специально сравнивайте, где 
у~вас рас\-став\-ле\-ны знаки препинания, а~где они стоят 
в~оригинале>>~\cite[с.~58]{4-nu}. В~работе <<Переводчик в~текс\-те: 
о~чтении русской литературы  
по-анг\-лий\-ски>> значение пунктуации отмечает Мей, критикуя 
англоязычных переводчиков за недостаточное внимание к~межъязыковой 
пунктуационной асим\-мет\-рии~--- за <<игнорирование отличительных 
особенностей, присущих знакам препинания>>~\cite[с.~121]{2-nu}. 
О~пунктуации в~переводе говорит Юдейл, выделяя три аспекта:
%\begin{enumerate}[(1)]
%\item 
(1)~<<знаки препинания~--- важ\-ная часть перевода, но, концентрируясь на 
общем смыс\-ле переводимого, ее час\-то не замечают>>; 
%\item 
(2)~<<изменения 
в~пунктуации при переводе могут значительно по\-вли\-ять на вы\-ра\-зи\-тель\-ность 
текс\-та, его свя\-зан\-ность и~ритм>>; 
%\item 
(3)~<<час\-то возникает впечатление, что 
литературные переводчики наделили себя правом менять границы исходного 
предложения и~пунктуационные знаки, как им 
заблагорассудится>>~\cite[с.~121]{5-nu}.
%\end{enumerate}
 Гораздо реже 
в~специализированной литературе подчеркивается роль, которую 
исследования в~об\-ласти контрастивной пунктуации играют при обучении 
иностранным языкам, в~част\-ности при обуче\-нии иноязычной письменной 
речи~\cite{7-nu}.
  
  Признавая безусловную зна\-чи\-мость данного научного на\-прав\-ле\-ния и~его 
дальнейшего развития, необходимо предметно разрабатывать методологию 
исследования в~об\-ласти контрастивной пунктуации, которая учитывала бы 
новейшие достижения информатики, компьютерной лингвистики 
и~корпусного переводоведения. Пред\-став\-ля\-ет\-ся, что такая методология 
долж\-на основываться на использовании современных информационных 
корпусных инструментов, поз\-во\-ля\-ющих автоматизированным образом 
обрабатывать пред\-ста\-ви\-тель\-ные массивы текс\-то\-вых данных, 
и,~следовательно, носить кор\-пус\-но-ори\-ен\-ти\-ро\-ван\-ный характер 
(о~корпусных данных при контрастивном изуче\-нии пунктуации  
см.~\cite{6-nu}).

%\vspace*{-6pt}
    
    \section{Методологические модели  
корпусно-ориентированного исследования контрастивной 
пунктуации}

\vspace*{-3pt}
  
  В мае 2019~г.\ в~Регенсбурге (Германия) про\-шла научная конференция 
под названием <<Punctuation Seen Internationally. System--Norm--Practice>> 
(<<Пунктуация в~мировом мас\-шта\-бе: 
 сис\-те\-ма--нор\-ма--прак\-ти\-ка>>)~--- первая конференция, пол\-ностью\linebreak 
по\-свя\-щен\-ная проб\-ле\-мам контрастивной пунктуации. Оргкомитет, собирая 
заявки на участие, справедливо отмечал, что до на\-сто\-яще\-го времени 
пунктуации едва ли уделялось внимание в~рамках \mbox{типологии}, контрастивной 
лингвистики, прагмалингвистики, а~так\-же в~исследованиях индивидуальной 
языковой манеры на фоне языкового стандарта. Сейчас появляются 
отдельные работы, где проводится сопоставительное изуче\-ние пунктуации, 
однако по-преж\-не\-му ощущается острая не\-об\-хо\-ди\-мость в~исследованиях по 
контрастивной пунктуации, которые бы учитывали типологические 
(сис\-тем\-ные), социолингвистические (нормативные) и~прагматические 
(речевые) ас\-пекты.
  
  Итогом конференции стала коллективная монография~\cite{8-nu}, 
со\-сто\-ящая из шестнадцати статей, которые пред\-став\-ля\-ют собой пио\-нер\-ские 
исследования, на\-прав\-лен\-ные на формирование целостной па\-ра\-диг\-мы 
контрастивного изучения пунктуации и~борьбу с~маргинализацией важ\-ной 
научной от\-расли. Все статьи услов\-но мож\-но разделить на~4~категории, 
первые две из которых имеют в~большей степени тео\-ре\-ти\-че\-ский характер и~связаны с~сис\-те\-мой и~нормой, а~вторые~--- более практической 
на\-прав\-лен\-ности~--- с~узусом и~освоением пунктуационных навыков. 
В~пред\-став\-лен\-ных работах доминируют два подхода к~исследованию 
конт\-растив\-ной пунк\-ту\-ации:
  \begin{enumerate}[(1)]
\item интралингвистический (контрастивный анализ знаков препинания 
и~кон\-ку\-ри\-ру\-ющих с~ними маркеров синтаксических отношений в~рамках 
одного языка)~\cite[с.~110]{9-nu};
  \item  интерлингвистический (контрастивный анализ знаков препинания 
  и~конкурирующих с~ними средств в~разных языках, конт\-растив\-ная пунктуация 
рас\-смат\-ри\-ва\-ет\-ся в~том чис\-ле как часть методики обуче\-ния неродному языку, 
например при интеграции трудовых мигрантов в~иноязычную 
среду)~\cite[с.~57--73]{10-nu}.
  \end{enumerate}
  
  Интралингвистический подход час\-то носит смешанный характер: если 
речь идет об эволюции пунктуационной сис\-те\-мы отдельно взятого языка на 
фоне развития аналогичных сис\-тем других языков, контрастивный анализ 
со\-про\-вож\-да\-ет\-ся 
 ис\-то\-ри\-ко-эти\-мо\-ло\-ги\-че\-ским~\cite[с.~187--206]{11-nu}. В~рамках 
этого подхода в~указанной монографии имеются психолингвистические 
исследования с~нетривиальным корпусным материалом. Так, 
в~статье~\cite[с.~163--186]{12-nu} корпусные данные привлекаются для 
контрастивного анализа пунктуационных предпочтений двух групп 
ис\-пы\-ту\-емых. Автор использует корпус \mbox{CoPaDocs} (Corpus of Patient 
Documents), основу которого со\-ста\-ви\-ли письма и~другие личные документы 
бывших пациентов психиатрических учреж\-де\-ний Германии на рубеже  
XIX--XX~вв. Корпус поз\-во\-ля\-ет установить, зависит ли языковое оформление 
пись\-ма от лич\-ности адресата~--- происходит ли переключение ре\-гист\-ров 
сознательно. Данный корпус создан с~целью разработки интегративной 
методики анализа языковой ва\-риа\-тивн\-ости, в~том чис\-ле и~в~об\-ласти 
пунктуации. \mbox{Изучив} специфику расстановки~12~знаков препинания, 
Эбер-Хам\-мерль приходит к~выводу, что пациенты, чей род де\-я\-тель\-ности 
прежде не был связан с~письменной сферой, использовали больше 
пунктуационных маркеров (но с~меньшей ва\-риа\-тив\-ностью), чем 
представители второй опытной группы~--- канцелярские служащие. 
В~лич\-ной переписке участники обеих групп к~знакам препинания прибегали 
гораздо реже, чем в~документах, адресованных официальным лицам.
  
 В статье~\cite[с.~57--73]{10-nu} представлено контрастивное исследование, выполненное в~интерлигвистическом 
ключе. Со\-по\-став\-ле\-ние 
пунктуации в~италь\-ян\-ском и~немецком языках здесь проводится на основе 
комплексной методологии, вклю\-ча\-ющей приемы дескриптивного, 
просодического, синтаксического и~ком\-му\-ни\-ка\-тив\-но-текс\-то\-во\-го 
анализа. Примеры приводятся из различных источников, причем 
к~корпусным данным в~статье отсылают не напрямую, а~опосредованно~--- 
через более раннюю работу~\cite{13-nu}. По мнению авторов, 
пунктуирование в~этих языках организовано по-раз\-но\-му, что объясняется 
резкими различиями в~пунктуационном узусе: если в~итальянском знаки 
препинания коммуникативно на\-гру\-же\-ны, то в~немецком они подчинены 
фор\-маль\-но-син\-так\-си\-че\-ско\-му принципу. Иначе говоря, итальянская 
пунктуация выполняет не формальную функцию, а~сигнализиру-\linebreak ет о~тон\-ких 
смыс\-ло\-вых нюансах, которых нельзя\linebreak достичь другими языковыми 
средствами (аргументативный конфликт, полифонические эффекты, 
метатекстовые комментарии). В~этом же духе\linebreak выполнена и~другая 
интерлингвистическая работа~\cite{14-nu}, по\-свя\-щен\-ная контрастивному 
исследованию многоточия и~тире в~италь\-ян\-ском и~анг\-лий\-ском языках 
и~продуктивно ис\-поль\-зу\-ющая \mbox{корпусный} метод сбора и~обработки 
эмпирических данных.
     
     Объединенные в~коллективную монографию рабо\-ты позволяют 
вывести обобщенную ме\-то\-до\-ло\-гическую схему контрастивного изуче\-ния 
пунктуации. Она имеет трехфазную структуру. Первая\linebreak фаза включает 
тео\-ре\-ти\-че\-ское описание пунктуации в~изуча\-емом языке с~привлечением 
исторических и~современных нормативных грамматик и~справочников. 
Вторая фаза на\-прав\-ле\-на на описание трансформаций в~других языках, 
оказавших существенное влияние на статус и~мес\-то пунктуации в~сис\-те\-ме 
конкретного языка. Обе фазы нацелены на создание такого 
исследовательского поля, которое поз\-во\-лит выявить значение пунктуации 
для языковой культуры. Это, в~свою очередь, долж\-но стать задачей треть\-ей 
фазы. Вторая и~\mbox{третья} фазы предполагают межъязыковое сравнение как 
функционального диапазона отдельно взятых знаков препинания, так 
и~пунктуационного репертуара в~целом. На этих стадиях применяется 
корпусный метод. Контрастивный анализ в~за\-ви\-си\-мости от по\-став\-лен\-ных 
целей и~задач наряду со знаками препинания может охватывать 
и~кон\-ку\-ри\-ру\-ющие с~ними языковые средства. На\-прав\-ле\-ние контрастивного 
исследования пунктуации может быть и~синхронным, и~диахроническим.

\vspace*{-6pt}
    
    \section{Методологические особенности  
корпусно-ориентированного исследования в~области 
контрастивной пунктуации}

\vspace*{-3pt}
  
  Особенности методологии при корпусном контрастивном изуче\-нии 
пунктуации, как, впрочем, и~при любом 
 кор\-пус\-но-ори\-ен\-ти\-ро\-ван\-ном исследовании, связаны прежде всего 
со стремлением получить непротиворечивые, валидные и~на\-деж\-ные данные. 
Электронный корпус, будучи методологически новаторским инструментом 
для получения научного знания, поз\-во\-ля\-ет, с~одной стороны, автоматическим 
образом обрабатывать большие массивы данных и~тем самым серьезно 
сокращает временные издержки на поиск эмпирического материала. 
С~другой стороны, электронные корпусные ресурсы имеют свои 
особенности, и~без их над\-ле\-жа\-ще\-го учета пользователь рискует получить 
искаженные результаты.
  
  Например, в~указанной выше работе~\cite[с.~291]{14-nu} авторы, описывая 
методологию своего исследования, отмечают, что итальянские примеры 
заимствованы из корпуса, хранящегося в~Базельском университете 
и~со\-сто\-яще\-го из двух частей~--- 33~современных  
ро\-ма\-на-бест\-сел\-ле\-ра (1~млн словоупотреблений) 
и~нехудожественных текс\-та разной на\-прав\-лен\-ности (1~млн 40~тыс.\ 
словоупотреблений), в~то время как англоязычные примеры извлечены из 
подкорпуса <<Книги и~периодические издания>> Британского 
национального корпуса (80~млн словоупотреблений). Итальянский материал, 
по словам авторов, был проанализирован весь, а~для английского из-за 
гораздо большего объема ограничились анализом случайной выборки, объем 
которой со\-по\-ста\-вим с~выборкой из итальянского корпуса. Очевидным 
образом ва\-лид\-ность выводов по результатам анализа англоязычного 
материала здесь может оказаться под вопросом в~силу методологически 
неоднородных установок применительно к~процедуре обработки данных, 
полученных по двум языкам. Примечательно к~тому же, что базельский 
корпус, в~отличие от британского, за\-крыт для общественного пользования.
  
  О подобных ограничениях рассуждает На\-двор\-ни\-ко\-ва в~своей работе, 
по\-свя\-щен\-ной корпусной методологии контрастивного изучения 
пунктуации~\cite{15-nu}, где анализируется час\-тот\-ность упо\-треб\-ле\-ния шести 
знаков препинания (запятой, точки, двоеточия, точ\-ки с~запятой, 
вопросительного и~восклицательного знака) в~английском, французском 
и~чешском языках. Для сбора данных используются со\-по\-ста\-ви\-мые 
веб-кор\-пу\-сы, моноязычные общие (референтные) и~параллельные корпусы. Цель 
автора~--- определить, какой из трех типов корпусных ресурсов наиболее 
подходит для исследований в~об\-ласти контрастивной пунктуации.
  
  Полученные данные показывают, что при изучении пунктуации показатели 
час\-тот\-ности проявляют высокую чув\-ст\-ви\-тель\-ность к~типу текс\-та; 
следовательно, веб-кор\-пу\-сы, которые, как правило, отличают стихийное 
наполнение, не\-упо\-ря\-до\-чен\-ность и~низ\-кая степень струк\-ту\-ри\-ро\-ван\-ности, не 
могут служить источником до\-сто\-вер\-ной информации об упо\-треб\-ле\-нии 
знаков препинания в~том или ином языке. Моноязычный общий корпус, 
наоборот, содержит специальную раз\-мет\-ку (морфологическую, 
синтаксическую и~т.\,д.)\ и~поз\-во\-ля\-ет гиб\-ко настраивать поиск (в~том чис\-ле 
выбирать соответствующий тип текс\-та) в~за\-ви\-си\-мости от конкретных 
исследовательских задач. Такие корпусы располагают большими массивами 
данных, поскольку призваны пред\-ста\-вить язык во всей его пол\-но\-те 
и~многообразии, что, казалось бы, обеспечивает на\-деж\-ность и~ва\-лид\-ность 
полученных результатов. Меж\-ду тем этот тип корпусов имеет существенный 
недостаток~--- ограниченную межъязыковую со\-по\-ста\-ви\-мость. Как правило, 
моноязычные общие корпусы разных языков разительно отличаются по 
объему данных и~их со\-ста\-ву и~поэтому не подходят в~качестве основного 
инструмента контрастивного исследования, а~могут служить лишь 
референтным (проверочным) источником для дополнительной верификации 
ре\-зуль\-ти\-ру\-ющих данных. Кроме того, со\-по\-ста\-ви\-тель\-ный анализ 
относительной час\-тот\-ности упо\-треб\-ле\-ния знаков препинания в~разных 
языках на основе данных, извлеченных из корпусов этого типа, так\-же имеет 
свои ограничения. Он не применим для изучения пунк\-ту\-а\-ции в~языках 
разного строя, которым для кодирования информации требуется 
количественно больше (аналитические языки типа французского) или 
меньше слов (синтетические языки типа русского). Таким образом, лучше 
всего для контрастивного изуче\-ния пунк\-ту\-а\-ции подходят параллельные 
корпусы, которые, не\-смот\-ря на свой сравнительно небольшой объем, 
пред\-став\-ля\-ют существенно больше воз\-мож\-но\-стей для качественного анализа 
упо\-треб\-ле\-ния знаков препинания и~непосредственного со\-по\-став\-ле\-ния их 
абсолютной час\-тот\-ности в~параллельных текс\-тах~--- оригинале и~переводе. 
Однако и~этот тип информационного ресурса не может служить 
универсальным исследовательским инструментом. При его использовании 
необходимо учитывать, что пунктуационные рас\-хож\-де\-ния в~исходном 
и~переводном текс\-те могут быть не результатом сис\-тем\-ных дифференциаций, 
а~возникнуть под влиянием переводческих предпочтений. Следовательно, 
чтобы избежать искажения ре\-зуль\-ти\-ру\-ющих данных, надо следовать 
некоторым методологическим принципам: %\\[-13pt] 
\begin{enumerate}[(1)]
\item данные собираются в~обоих 
переводных на\-прав\-ле\-ни\-ях; %\\[-13pt] 
\item выявленные тенденции проходят 
обязательную проверку с~по\-мощью референтного моноязычного корпуса; %\\[-13pt] 
\item контрастивное изуче\-ние пунктуации с~применением параллельных 
корпусов требует сис\-тем\-но\-го подхода в~том смыс\-ле, что в~функциональном 
диапазоне разных знаков препинания могут быть общие зоны, ука\-зы\-ва\-ющие 
на их потенциальную внут\-ри\-язы\-ко\-вую и~межъ\-язы\-ко\-вую конкуренцию. %\\[-13pt]
\end{enumerate}
    
 \vspace*{-12pt}
 
    \section{Заключение}
    
    \vspace*{-3pt}
    
  В статье представлена обобщенная методологическая схема 
  кор\-пус\-но-ори\-ен\-ти\-ро\-ван\-но\-го 
  исследования в~об\-ласти контрастивной пунктуации~--- 
от\-расли научного знания, интенсивно \mbox{раз\-ви\-ва\-ющей\-ся} и~при\-вле\-ка\-ющей 
внимание специалистов самого широкого профиля. Несмотря на то что 
появляются работы, где описываются сопоставительные исследования 
пунктуации на примере одного произведения или литературного наследия 
отдельно взятого писателя (см., например,~\cite{16-nu,17-nu}), очевидно, что 
для ка\-ких-ли\-бо существенных, круп\-но\-мас\-штаб\-ных обобщений относительно 
межъязыковой пунктуационной асимметрии и~специфики функционирования 
знаков препинания в~разных языках требуется привлечение корпусного 
материала.
  
  Дальнейшее изучение контрастивной пунктуации видится в~нескольких 
направлениях. Необходимо качественное углубление со\-по\-ста\-ви\-тель\-но\-го 
анализа, чтобы его тонкая нюансировка \mbox{поз\-во\-ли\-ла} установить, в~какой мере 
совпадает и~разнится функциональный диапазон того или иного знака 
препинания в~кон\-так\-ти\-ру\-ющих языках в~за\-ви\-си\-мости от жанровой 
при\-над\-леж\-ности текс\-та. Этот анализ целесообразно проводить комплексно, 
охватывая всю со\-во\-куп\-ность синтаксических изменений, которые влекут за 
собой отказ от исходного пунктуирования при переводе с~одного языка на 
другой. Такая ком\-плекс\-ность поможет выявить и~с~большей пол\-но\-той 
описать су\-щест\-ву\-ющие межъ\-язы\-ко\-вые структурные различия, что 
необходимо и~для переводческой практики, и~для обуче\-ния иностранным 
языкам. Требует дальнейшего уточ\-не\-ния вопрос, как на пунктуационные 
преференции переводчика влияет род\-ная языковая культура, 
пунктуационные уста\-нов\-ки которой могут меняться со временем. По мере 
наращивания опыта и~мастерства могут меняться пунктуационные 
предпочтения и~самого переводчика, и~это так\-же пред\-став\-ля\-ет определенный 
научный интерес.
  
  В заключение следует отметить, что одним из современных 
информационных инструментов корпусного исследования в~об\-ласти 
контрастивной пунктуации могут быть НБД, 
раз\-ра\-ба\-ты\-ва\-емые в~отделе~54 Федерального исследовательского цент\-ра 
<<Информатика и~управ\-ле\-ние>> Российской академии наук (о~возможностях 
НБД см.~\cite{6-nu}). В~данный момент этот методологический инструмент 
проходит апро\-ба\-цию в~контрастивном исследовании двоеточия и~многоточия в~трех языках~--- русском, французском и~немецком.

\vspace*{-9pt}
  
{\small\frenchspacing
 {\baselineskip=11.5pt
 %\addcontentsline{toc}{section}{References}
 \begin{thebibliography}{99}
 
 \vspace*{-3pt}
 
 \bibitem{4-nu} %1
\Au{Newmark P.} A~textbook of translation.~--- New York, London, Toronto, Sydney, Tokyo: Prentice 
Hall, 1988. 402~p.
 

\bibitem{2-nu} %2
\Au{May R.} The translator in the text: On reading Russian literature in English.~--- Evanston, IL, USA: 
Northwestern University Press, 1994. 209 p.
\bibitem{3-nu}
\Au{Munday J.} Systems in translation: A~systemic model for descriptive translation studies~// 
Crosscultural transgressions: Research models in translation studies II~--- historical and 
ideological issues~/ Ed. T.~Hermans.~---  Manchester, U.K.: St.\ Jerome, 2002. P.~76--92.
\bibitem{1-nu} %4
\Au{Baker M.} In other words.~--- 2nd ed.~--- London, New York: Routledge, 2011. 352~p.

\bibitem{7-nu} %5
\Au{Сигал К.\,Я.} Контрастивная пунктуация в~начале XXI века~// Язык. Текст. Дискурс: 
Научный альманах Ставропольского отделения РАЛК.~--- Ставрополь: СКФУ, 
2019.  Вып.~17. С.~69--78.
\bibitem{5-nu} %6
\Au{Youdale R.} Using computers in the translation of literary style: Challenges and 
opportunities.~--- London, New York: Routledge, 2020. 242~p.
\bibitem{6-nu} %7
\Au{Нуриев В.\,А., Кружков~М.\,Г.} Корпусные данные при контрастивном изуче\-нии 
пунктуации~// Сис\-те\-мы и~средства информатики, 2023. Т.~33. №\,1. С.~14--23. doi: 10.14357/08696527230102.

\bibitem{8-nu}
Vergleichende Interpunktion~--- comparative punctuation~/ Eds. P.~R$\ddot{\mbox{o}}$ssler, P.~Besl, A.~Saller.~--- 
Berlin, Boston: De Gruyter, 2021. 454~p.
\bibitem{9-nu}
\Au{Rinas K.} Vom genormten Satzbau zur genormten Interpunktion. Zur Funktion der 
Zeichensetzung in $\ddot{\mbox{a}}$lterer und neuerer Zeit~// Vergleichende Interpunktion~--- comparative 
punctuation~/ Eds. P.~R$\ddot{\mbox{o}}$ssler, P.~Besl, A.~Saller.~---
 Berlin, Boston: De Gruyter, 2021. P.~109--136. doi: 10.1515/9783110756319-006.
\bibitem{10-nu}
\Au{Ferrari~A., Stojmenova Weber R.} Das Komma in kontrastiver Perspektive Italienisch-Deutsch~// Vergleichende Interpunktion~--- 
comparative punctuation / Eds. P.~R$\ddot{\mbox{o}}$ssler, P.~Besl, 
A.~Saller.~--- Berlin, Boston: De Gruyter, 2021. P.~57--73. doi: 10.1515/9783110756319-003.

\columnbreak

\bibitem{11-nu}
\Au{Besch W.} Zur Entwicklung der deutschen Interpunktion seit dem sp$\ddot{\mbox{a}}$ten Mittelalter~// 
Interpretation und Edition deutscher Texte des Mittelalters. Festschrift f$\ddot{\mbox{u}}$r John Asher zum 60. 
Geburtstag~/ Eds. K.~Smits, W.~Besch, V.~Lange.~--- Berlin: Erich Schmidt, 1981. P.~187--206.
\bibitem{12-nu}
\Au{Eber-Hammerl F.} Interpunktion in historischen Patientenbriefen // Vergleichende
Interpunktion~--- comparative punctuation~/ Eds. P.~R$\ddot{\mbox{o}}$ssler, 
P.~Besl, A.~Saller.~--- Berlin, Boston: De Gruyter, 2021. P.~163--186.
\bibitem{13-nu}
\Au{Ferrari A.} Leggere la virgola. Una prima ricognizione~// Chimera Romance Corpora 
Linguistic Studies, 2017. Vol.~4. Iss.~2. P.~145--162. doi: 
10.15366/chimera2017. 4.2.001.
\bibitem{14-nu}
\Au{Pecorari F., Longo~F.} The ellipsis and the dash in Italian and English: A~contrastive 
perspective~// Vergleichende Interpunktion~--- comparative punctuation~/ Eds.
 P.~R$\ddot{\mbox{o}}$ssler, P.~Besl, A.~Saller.~--- Berlin, Boston: De Gruyter, 2021. P.~289--314.
 doi: 10.1515/9783110756319-013.
\bibitem{15-nu}
\Au{N$\acute{\mbox{a}}$dvorn$\acute{{\iota}}$kov$\acute{\mbox{a}}$~O.}
The use of English, Czech and French punctuation marks in reference, 
parallel and comparable web corpora: A~question of methodology~// 
Linguist. Prag.,  2020. Vol.~30. Iss.~2. P.~30--50. doi: 
10.14712/ 18059635.2020.1.2.
\bibitem{16-nu}
\Au{Сигал К.\,Я.} Пунктуация как средство создания эмоционального под\-текс\-та (на 
материале рассказа М.\,А.~Шолохова <<Судьба человека>> и~его переводов на английский 
язык)~// Известия РАН. Серия литературы и~языка, 2014. Т.~73. №\,6. С.~38--50.
\bibitem{17-nu}
\Au{Богданов К.\,А.} Пунктуация как мотив: многоточие и~тире~// НЛО, 2022. №\,2(174). С.~241--253.
doi: 0.53953/ 08696365\_2022\_174\_2\_241.

\end{thebibliography}

 }
 }

\end{multicols}

\vspace*{-8pt}

\hfill{\small\textit{Поступила в~редакцию 15.04.23}}

\vspace*{6pt}

%\pagebreak

%\newpage

%\vspace*{-28pt}

\hrule

\vspace*{2pt}

\hrule

\vspace*{-2pt}

\def\tit{METHODOLOGY OF~THE~CORPUS-BASED STUDIES\\ 
IN~THE~FIELD OF~CONTRASTIVE PUNCTUATION}


\def\titkol{Methodology of~the~corpus-based studies 
in~the~field of~contrastive punctuation}


\def\aut{V.\,A.~Nuriev$^1$ and~V.\,I.~Karpov$^{1,2}$}

\def\autkol{V.\,A.~Nuriev and~V.\,I.~Karpov}

\titel{\tit}{\aut}{\autkol}{\titkol}

\vspace*{-14pt}


\noindent
      $^1$Federal Research Center ``Computer Science and Control'' of the Russian 
Academy of Sciences, 44-2~Vavilov\linebreak
$\hphantom{^1}$Str., Moscow 119333, Russian Federation
      
      \noindent
      $^2$Institute of Linguistics of the Russian Academy of Sciences, 1~bld.~1 
Bolshoy Kislovsky Lane, Moscow 125009,\linebreak
$\hphantom{^1}$Russian Federation

\def\leftfootline{\small{\textbf{\thepage}
\hfill INFORMATIKA I EE PRIMENENIYA~--- INFORMATICS AND
APPLICATIONS\ \ \ 2023\ \ \ volume~17\ \ \ issue\ 2}
}%
 \def\rightfootline{\small{INFORMATIKA I EE PRIMENENIYA~---
INFORMATICS AND APPLICATIONS\ \ \ 2023\ \ \ volume~17\ \ \ issue\ 2
\hfill \textbf{\thepage}}}

\vspace*{3pt}
      
      
    
    \Abste{The paper refines the methodological approach to the contrastive 
studies of punctuation. Given the recent achievements of information science, 
computer linguistics, and translation theory, such studies are most likely to be 
corpus-based. The paper presents a~methodological model of research into 
interlingual punctuation asymmetry, the aim of which is to shed light on the 
functional scope of the same punctuation marks in different languages. It shows 
what methodological trends are characteristic of this research area. The focus is 
also on the specificities of corpus methodology in the contrastive study of 
punctuation. It is argued that one of the methodological tools, tailored specifically 
to the needs of contrastive punctuation research, may be the supracorpora 
databases developed at the Federal Research Center ``Computer Science and 
Control'' of the Russian Academy of Sciences.}
    
    \KWE{contrastive punctuation; translation; corpus-based translation studies; 
corpus-based studies; parallel corpus; supracorpora database; asymmetry between 
languages; methodology}
    
    
    
\DOI{10.14357/19922264230213}{VBOZAO}

%\vspace*{-18pt}

\Ack
    \noindent
    The research was carried out using the infrastructure of the Shared Research 
Facilities ``High Performance Computing and Big Data'' (CKP ``Informatics'') of 
FRC CSC RAS (Moscow). The research was supported by the Russian Science Foundation (project  
No.\,23-28-00548).
 
%\vspace*{4pt}

  \begin{multicols}{2}

\renewcommand{\bibname}{\protect\rmfamily References}
%\renewcommand{\bibname}{\large\protect\rm References}

{\small\frenchspacing
 {%\baselineskip=10.8pt
 \addcontentsline{toc}{section}{References}
 \begin{thebibliography}{99}
 
 \bibitem{4-nu-1} %1
\Aue{Newmark, P.} 1988. \textit{A~textbook of translation}. New York, London, Toronto, Sydney, Tokyo: 
Prentice Hall. 402~p.   

\bibitem{2-nu-1}
\Aue{May, R.} 1994. \textit{The translator in the text: On reading Russian 
literature in English}. Evanston, IL: Northwestern University Press. 209~p.
\bibitem{3-nu-1}
\Aue{Munday, J.} 2002. Systems in translation: A~systemic model for 
descriptive translation studies. \textit{Crosscultural transgressions: Research models in 
translation studies II~--- historical and ideological issues}. Ed. T.~Hermans. 
Manchester, U.K.: St.\ Jerome. 76--92.

\bibitem{1-nu-1} %4
\Aue{Baker, M.} 2011. \textit{In other words}. 2nd ed. London, New York: 
Routledge. 352~p.

\bibitem{7-nu-1} %5
\Aue{Seagal, K.\,Ya.} 2019. Kont\-ras\-tiv\-naya punk\-tu\-a\-tsiya v~na\-cha\-le XXI~veka 
[Contrastive punctuation at the beginning of the XXI century]. \textit{Yazyk. Tekst. 
Diskurs: Nauchnyy al'manakh Stavropol'skogo otdeleniya RALK} [Language. Text. 
Discourse: Scientific almanac of Stavropol Branch of the Russian Cognitive 
Linguists Association].  Stavropol': SKFU. 17:69--78.

\bibitem{5-nu-1} %6
\Aue{Youdale, R.} 2020. \textit{Using computers in the translation of literary style: 
Challenges and opportunities}. London, New York: Routledge. 242~p.
\bibitem{6-nu-1} %7
\Aue{Nuriev, V.\,A., and M.\,G.~Kruzhkov.} 2023. Kor\-pus\-nye dan\-nye pri 
kont\-ras\-tiv\-nom izu\-che\-nii punk\-tu\-a\-tsii [The parallel corpora perspective on studying 
contrastive punctuation]. \textit{Sistemy i~Sredstva Informatiki~--- Systems and Means of 
Informatics} 33(1):14--23. doi: 10.14357/08696527230102.

  \bibitem{8-nu-1}
R$\ddot{\mbox{o}}$ssler, P., P.~Besl, and A.~Saller, eds. 2021. \textit{Vergleichende 
Interpunktion~--- comparative punctuation}. Berlin, Boston: De Gruyter. 454~p.
\bibitem{9-nu-1}
\Aue{Rinas, K.} 2021. Vom genormten satzbau zur genormten interpunktion. 
Zur funktion der zeichensetzung in $\ddot{\mbox{a}}$lterer und neuerer zeit. \textit{Vergleichende 
Interpunktion~--- comparative punctuation}. Eds.\ P.~R$\ddot{\mbox{o}}$ssler, 
P.~Besl, and A.~Saller. 
Berlin, Boston: De Gruyter. 109--136. doi: 10.1515/ 9783110756319-006.
\bibitem{10-nu-1}
\Aue{Ferrari, A., and R.~Stojmenova.} 2021. Weber das komma in kontrastiver 
perspektive Italienisch-Deutsch. \textit{Vergleichende Interpunktion~--- comparative 
punctuation}. Eds. P.~R$\ddot{\mbox{o}}$ssler, P.~Besl, and A.~Saller. Berlin, Boston: De Gruyter.  
57--73. doi: 10.1515/9783110756319-003.
 \bibitem{11-nu-1}
\Aue{Besch, W.} 1981. Zur entwicklung der deutschen interpunktion seit 
dem sp$\ddot{\mbox{a}}$ten mittelalter. \textit{Interpretation und Edition deutscher Texte des Mittelalters. 
Festschrift f$\ddot{\mbox{u}}$r John Asher zum 60. Geburtstag}. Eds. K.~Smits, W.~Besch, and 
V.~Lange. Berlin: Erich Schmidt. 187--206.
 \bibitem{12-nu-1}
\Aue{Eber-Hammerl, F.} 2021. Interpunktion in historischen 
Patientenbriefen. \textit{Vergleichende Interpunktion~--- comparative punctuation}. Eds. 
P.~R$\ddot{\mbox{o}}$ssler, P.~Besl, and A.~Saller. Berlin, Boston: De Gruyter. 163--186.
\bibitem{13-nu-1}
\Aue{Ferrari, A.} 2017. Leggere la virgola. Una prima ricognizione. 
\textit{Chimera Romance Corpora Linguistic Studies} 4(2):145--162. doi: 
10.15366/chimera2017.4.2.001.
\bibitem{14-nu-1}
\Aue{Pecorari, F., and F.~Longo.} 2021. The ellipsis and the dash in Italian 
and English: A~contrastive perspective. \textit{Vergleichende Interpunktion~--- 
comparative punctuation}. Eds. P.~R$\ddot{\mbox{o}}$ssler, P.~Besl, and A.~Saller. Berlin, Boston: 
De Gruyter. 289--314. doi: 10.1515/9783110756319-013.
\bibitem{15-nu-1}
\Aue{N$\acute{\mbox{a}}$dvorn$\!\acute{\mbox{\ptb{\i}}}$kov$\acute{\mbox{a}}$,~O.} 2020. The use of English, Czech and French 
punctuation marks in reference, parallel and comparable web corpora: A~question 
of methodology. \textit{Linguist. Prag.} 30(2):30--50. doi: 
10.14712/18059635.2020.1.2.
\bibitem{16-nu-1}
\Aue{Seagal, K.\,Ya.} 2014. Punk\-tu\-a\-tsiya kak sred\-st\-vo so\-zda\-niya 
emo\-tsi\-o\-nal'\-no\-go pod\-teks\-ta (na ma\-te\-ri\-ale ras\-ska\-za M.\,A.~Sho\-lo\-kho\-va ``Sud'\-ba 
che\-lo\-ve\-ka'' i~ego pe\-re\-vo\-dov na ang\-liy\-skiy yazyk) [Punctuation as a means of 
revealing the emotional subtext (the case of Mikhail Sholokhov's short story ``The 
Fate of a~Man'' and its translations into English)]. \textit{Izvestiya RAN. Seriya literatury i~yazyka}
 [The Bulletin of the Russian Academy of Sciences: Studies in Literature 
and Language]. 73(6):38--50.
\bibitem{17-nu-1}
\Aue{Bogdanov, K.\,A.} 2022. Punk\-tu\-a\-tsiya kak mo\-tiv: mno\-go\-to\-chie i~ti\-re 
[Punctuation as a~motive: The ellipsis and the dash]. \textit{NLO} [New Literary Observer] 
2(174):241--253. doi: 0.53953/08696365\_2022\_174\_2\_241.
\end{thebibliography}

 }
 }

\end{multicols}

\vspace*{-6pt}

\hfill{\small\textit{Received April 15, 2023}} 

\vspace*{-18pt}
    
    
    \Contr
    
    
    \vspace*{-3pt}
    
    \noindent
    \textbf{Nuriev Vitaly A.} (b.\ 1980)~--- Doctor of Science in philology, leading 
scientist, Institute of Informatics Problems, Federal Research Center ``Computer 
Science and Control'' of the Russian Academy of Sciences, 44-2~Vavilov Str., 
Moscow 119333, Russian Federation; \mbox{nurieff.v@gmail.com}
    
    \vspace*{3pt}
    
    \noindent
    \textbf{Karpov Vladimir I.} (b.\ 1971)~--- Candidate of Science (PhD) in 
philology, leading scientist, Institute of Linguistics of the Russian Academy of 
Sciences, 1~bld.~1 Bolshoy Kislovsky lane, Moscow 125009, Russian Federation; 
scientist, Institute of Informatics Problems, Federal Research Center ``Computer 
Science and Control'' of the Russian Academy of Sciences, 44-2~Vavilov Str., 
Moscow 119333, Russian Federation; \mbox{wi.karpow@gmail.com}
     
      
\label{end\stat}

\renewcommand{\bibname}{\protect\rm Литература}         %13
\def\stat{rumovskaya}

\def\tit{ПОДХОДЫ К ПОДБОРУ СПЕЦИАЛИСТОВ ПРИ~ОРГАНИЗАЦИИ КОЛЛЕКТИВНОГО 
РЕШЕНИЯ ПРОБЛЕМ$^*$}

\def\titkol{Подходы к~подбору специалистов при~организации коллективного 
решения проблем}

\def\aut{С.\,Б.~Румовская$^1$}

\def\autkol{С.\,Б.~Румовская}

\titel{\tit}{\aut}{\autkol}{\titkol}

\index{Румовская С.\,Б.}
\index{Rumovskaya S.\,B.}


{\renewcommand{\thefootnote}{\fnsymbol{footnote}} \footnotetext[1]
{Исследование выполнено за счет гранта РНФ №\,23-21-00218.}}


\renewcommand{\thefootnote}{\arabic{footnote}}
\footnotetext[1]{Федеральный исследовательский центр <<Информатика и~управление>> Российской 
академии наук, \mbox{sophiyabr@gmail.com}}

%\vspace*{-12pt}

 

 

  \Abst{Исследование малых групп (коллективов, команд), их особенностей, проблем, 
динамики и~особенностей подбора специалистов стоит на стыке психологии в~управлении 
персоналом и~социальной психологии. Особое место в~широком спектре направлений 
современной науки занимает моделирование взаимодействия людей в~малых коллективах 
специалистов, в~частности в~рамках многоагентного подхода. При этом, разрабатывая 
интеллектуальные сис\-те\-мы (ИС) (искусственные гетерогенные коллективы) для решения 
практических проблем, сейчас требуется объединять в~составе системы модели специалистов 
(агентов), созданных различными командами разработчиков, имеющих несовместимые цели 
и~модели предметной об\-ласти. Отбор специалистов в~естественные и~моделей
 специалистов в~искусственные гетерогенные коллективы~--- важная задача, результаты решения которой 
влияют на дальнейший процесс принятия решений. Представлен анализ методов 
и~подходов к~подбору специалистов и~комплектования малых групп (коллективов, команд), 
измерительные инструменты которых должны подвергаться оценке качества.}
  
  \KW{группа; малый коллектив специалистов; команда; методы подбора специалистов 
  и~формирования малых групп; командообразование}

\DOI{10.14357/19922264230214}{VJWNOE} 
  
%\vspace*{-8pt}


\vskip 10pt plus 9pt minus 6pt

\thispagestyle{headings}

\begin{multicols}{2}

\label{st\stat}
  
\section{Введение}

  Взаимосвязь психологии в~управлении персоналом и~социальной психологии 
состоит в~том, что объект исследования первой~--- как отдельно взятая 
личность, так и~малые группы~--- один из сложнейших феноменов социальной 
психологии~[1]. <<Малая группа>>~\cite{2-r}~--- элементарное звено 
структуры\linebreak социальных отношений, обретающее через непосредственные 
межличностные контакты структурные, динамические и~феноменологические 
характеристики, отражающие признаки группы как \mbox{целостной} системы 
социальных и~психологических отношений. В~организациях руководители 
получают значительный эффект, создавая малые группы с~учетом групповой 
сплоченности, единства и~других социально-психологических феноменов. 

Отбор специалистов~--- важная задача, результаты выполнения которой влияют 
на дальнейшую работу группы (коллектива, команды), ре\-ша\-ющей в~различных 
сферах проб\-ле\-мы, \mbox{осложненные} слабой формализацией, комплексным 
строением, сетевым характером условий и~целей, неопределенностью, 
субъективностью и~ди\-на\-мич\-ностью. Подобный коллектив, который называют 
естественным гетерогенным коллективным интеллектом поддержки принятия 
решений~\cite{3-r},~--- это малая группа экспертов (специалистов), которой\linebreak 
присущи не\-од\-но\-род\-ность, разнообразие, сотрудничество, до\-пол\-ни\-тель\-ность 
и~относительность знаний. 

Основная форма организации малых  
коллективов~--- совещания, построенные по принципу круглого стола. Особое 
место в~современной науке занимает моделирование взаимодействия людей 
в~таких коллективах, в~частности методами многоагентных систем. Сейчас для 
решения практических проблем требуется объединять в~составе 
ИС модели специалистов (агентов), созданных 
различными командами разработчиков и~имеющих несовместимые цели 
и~модели предметной области. В~этой связи для учета субъективности 
и~динамического характера проблем предполагается разработать новый класс 
ИС~--- реф\-лек\-сив\-но-ак\-тив\-ные сис\-те\-мы искусственных гетерогенных 
интеллектуальных агентов (\mbox{РАСИГИА}), в~которых агенты будут 
взаимно моделировать рефлексивные позиции друг друга, динамически 
вырабатывать стратегии своего поведения, по мере необходимости в~процессе 
решения проблем привлекать новых агентов из пула доступных агентов от 
различных разработчиков и~исключать существующих. 

Результат работы 
\mbox{РАСИГИА} зависит от состава выбираемых для включения агентов, 
а~значит, методы подбора специалистов, разрабатываемые в~об\-ласти 
психологии управ\-ле\-ния персоналом, должны быть проанализированы на 
предмет возможности адаптации для \mbox{РАСИГИА}. 
  
  В настоящей работе проанализированы методы подбора специалистов 
и~комплектования малых групп (коллективов и~команд), разработанные 
в~области психологии управления персоналом.

\section{Малые высокоорганизованные группы (коллективы, 
команды)} 
  
  В работах~\cite{4-r, 5-r, 6-r, 7-r} о малых группах специалистов, 
рассматриваемых в~данном исследовании как естественный гетерогенный 
коллектив, говорят как о~коллективах~--- высокоразвитой форме организации 
групповой дея\-тель\-ности, при которой связи и~отношения между индивидами 
опосредованы общественно значимыми целями. Коллективы в~отечественной 
литературе рас\-смат\-ри\-ва\-ют\-ся как высший уровень развития группы, 
ха\-рак\-те\-ри\-зу\-емый\linebreak
 высокой степенью спло\-чен\-ности, единством,  
цен\-ност\-но-нор\-ма\-тив\-ной ориентации, глубокой идентификацией 
индивида с~группой и~от\-вет\-ст\-вен\-ностью за результаты совместной групповой 
\mbox{деятельности}~\cite{5-r}. В~работах~\cite{8-r, 9-r, 10-r} говорится о~стадиях 
зрелости коллектива (команды) в~рамках более широкого процесса жизни 
малой группы, которые можно свес\-ти к~таким, как (1)~притирка 
и~формирование; (2)~конфликтная (образуются подгруппы, появляются 
разногласия); (3)~экспериментирование в~методах и~средствах; (4)~появление 
спло\-чен\-ности (границы подгрупп стираются, успешное решение задач, 
творчество); (5)~высокий уровень спло\-чен\-ности (формируются прочные связи, 
роли и~полномочия динамично согласовываются, личные разногласия быст\-ро 
устраняются). Катценбах и~Смит~\cite{11-r} определяют команду как 
малое чис\-ло людей (от~2 до~25~человек, но обычно не более~10) 
с~взаимодополняющими умениями, связанных единым за\-мыс\-лом, стремящихся 
к~общим целям и~ответственных за их достижение. Команде присущи 
постоянство со\-ста\-ва, жест\-кое распределение ролей, ясная и~формальная цель, 
а~члены команды сыгранны и~действуют одинаково по отношению 
к~окру\-же\-нию~\cite{12-r}. В~\cite{4-r} отмечается, что в~современной трактовке 
команды много общего с~описанием коллектива в~работах отечественных 
авторов прош\-лых лет. Команда как группа высокого уровня развития, 
сравнительно с~пониманием коллектива, более реалистична, прагматична, 
лишена идеологических ярлыков. Таким образом, понятия <<малый коллектив 
специалистов>> и~<<команда>> идентичны, поэтому актуально исследование 
формирования и~коллективов, и~команд. 
  
\section{Методы и~подходы к~отбору~специалистов }

  \textbf{Методы комплектования малых групп и~коллективов}~\cite{13-r}. 
Определяются оптимальные количественные соотношения между работниками 
в~малых группах и~коллективах. Выделяют два взаимодополняющих принципа 
отбора: сра\-бо\-тан\-ность и~со\-вмес\-ти\-мость~\cite{14-r}. Сра\-бо\-тан\-ность 
характеризуется высокой со\-гла\-со\-ван\-ностью у~членов группы~\cite{15-r} и~ее\linebreak 
про\-дук\-тив\-ностью и~базируется на  
про\-фес\-сио\-наль\-но-ква\-ли\-фи\-ка\-ци\-он\-ной до\-пол\-ня\-емости. 
Совместимость~--- оптимальное сочетание свойств участников, 
обес\-пе\-чи\-ва\-ющее их эффективное \mbox{существование} и~спо\-соб\-ность 
оптимизировать свои взаимоотношения и~согласовывать свои 
действия~\cite{15-r}. Выделяются три уровня со\-вмес\-ти\-мости.\\[-14pt]
  \begin{enumerate}[1.]
\item Согласованность функ\-ци\-о\-наль\-но-ро\-ле\-вых ожи\-да\-ний~--- выделяют 
своеобразные роли, которые в~совместной дея\-тель\-ности дополнительно 
к~основным играют люди, например выделяют целевые и~под\-дер\-жи\-ва\-ющие~\cite{16-r}.\\[-14pt]
\item Ценностно-ори\-ен\-та\-ци\-он\-ное единст\-во (ЦОЕ) по 
А.\,В.~Петровскому~\cite{17-r}~--- сход\-ст\-во мнений, позиций членов группы 
по отношению к~объектам, наиболее значимым для группы в~целом. Есть ряд 
процедур оценки ЦОЕ.\\[-14pt]
\item Психофизиологическая со\-вмес\-ти\-мость:\\[-14pt]
\begin{itemize}
\item первичное комплектование малых групп~\cite{18-r}. Применяют метод 
изуче\-ния характерных особенностей индивидуальной ориентации человека по 
отношению к~другим людям, диагностируемых опросником межличностных 
отношений (ОМО) В.~Шультца, который определяет межличностную 
со\-вмес\-ти\-мость как отношения между двумя или более индивидами, при 
которых достигается та или иная степень взаимного удовле\-тво\-ре\-ния 
меж\-лич\-ност\-ных по\-треб\-но\-стей. Так\-же применяется соционический метод, 
например Е.\,С.~Филатовой и~Ю.\,В.~Иванова~\cite{19-r, 20-r}, ба\-зи\-ру\-ющий\-ся 
на том, что отношения между сотрудниками группы можно пред\-ста\-вить в~виде 
парных взаимодействий, а~психологический тип человека проявляется во 
взаимодействии с~другими людьми;\\ [-14pt]
\item перекомплектование~--- предполагает, что члены об\-сле\-ду\-емо\-го 
коллектива достаточно хорошо знают друг друга по со\-вмест\-ной дея\-тель\-ности. 
Тогда мож\-но использовать со\-цио\-мет\-ри\-че\-ский тест Дж.~Морено~\cite{21-r}, 
пред\-став\-ля\-ющий собой процедуру пе\-ре\-крест\-но\-го опроса членов группы друг 
о~друге по вопросам или критериям, которые на\-прав\-ле\-ны на выявление 
особенностей их взаимоотношений, взаимных оценок тех или иных качеств 
лич\-ности и~поведения. Данные ответов кодируются в~специальные мат\-ри\-цы 
и~анализируются ве\-ро\-ят\-ност\-но-ста\-ти\-сти\-че\-ски\-ми методами. 
В~\cite{22-r, 23-r, 24-r} рас\-смот\-рен ряд приемов экспертизы психологической 
со\-вмес\-ти\-мости в~сло\-жив\-ших\-ся малых группах.\\[-14pt]
\end{itemize}
\end{enumerate}
  
  \textbf{Профессиональный отбор}~\cite{7-r} предполагает выбор по 
критериям профессиональной под\-го\-тов\-лен\-ности и~опыта, уровню и~профилю 
образования. 
  \begin{enumerate}[1.]
\item Формирование профиля должности~\cite{6-r}. Каждая должностная 
позиция, в~том чис\-ле и~в~со\-ста\-ве коллектива, ре\-ша\-юще\-го проб\-ле\-му, 
предъявляет специалисту ряд требований, информацию о~которых 
структурируют и~сводят в~единую сис\-те\-му в~профиле долж\-ности. Используют 
несколько видов критериев~\cite{15-r}: 
\begin{itemize}
\item квалификационные; 
\item объективные, кон\-ста\-ти\-ру\-ющие соответствие реальных достижений 
оце\-ни\-ва\-емых субъектов некоторым количественным и~качественным 
показателям; 
\item внеш\-ние, ха\-рак\-те\-ри\-зу\-ющие наличие качеств, поз\-во\-ля\-ющих 
добиваться высоких результатов; 
\item психологические, раз\-ра\-ба\-ты\-ва\-емые на 
осно\-ве профессиограммы со\-от\-вет\-ст\-ву-\linebreak юще\-го вида дея\-тель\-ности, которая 
пред\-став\-ля\-ет собой сис\-те\-му при\-зна\-ков, опи\-сы\-ва\-ющих профессию, перечень 
норм и~\mbox{требований}~\cite{7-r}; 
\item тес\-то\-вые по ин\-ди\-ви\-ду\-аль\-но-пси\-хо\-ло\-ги\-че\-ским 
характеристикам.
\end{itemize}

 В~профиль включают факторы 
приоритетов при принятии решений, основные мотивации и~др. В~профиле 
долж\-ны быть сформулированы предельно конкретно со шкалами~измерений 
компетенции, которые необходимы, желательны или безразличны, в~част\-ности 
для роли специалиста в~коллективе, фор\-ми\-ру\-ющем\-ся для выработки решения 
по некоторой проб\-ле\-ме~\cite{6-r, 25-r}.\\[-14pt]
\item Первичный отбор~\cite{7-r} начинается с~анализа спис\-ка кандидатов 
с~точ\-ки зрения их соответствия общим требованиям по\-сред\-ст\-вом 
анкетирования, тес\-ти\-ро\-ва\-ния или испытания, графологического анализа 
(экспертиза почерка и~стиля изложения), морфологического анализа и~анализа 
по фотографии.\\ [-14pt]
\item На втором этапе отбора в~основном используют~\cite{26-r, 27-r}:
\begin{itemize}
\item комплексные исследования в~цент\-рах оценки персонала~--- 
оценки одних и~тех же критериев в~разных ситуациях и~различными методами, а~также 
ролевые и~имитационные деловые игры и~анализ конкретных ситуаций 
(моделируются существенные моменты дея\-тель\-ности и~оцениваются реальные 
достижения ис\-пы\-ту\-емых или де\-монст\-ри\-ру\-емое поведение)~\cite{27-r};\\[-14pt]
\item тесты: 
\begin{itemize}
\item[(а)] на проф\-при\-год\-ность~--- оценка психофизиологических качеств 
человека, умений выполнять определенную дея\-тель\-ность;
\item[(б)] общие~--- оценка 
общего уров\-ня развития и~особенностей мыш\-ле\-ния, внимания, памяти и~других 
высших психических функций; 
\item[(в)] био\-гра\-фи\-че\-ские тес\-ты и~изучение 
био\-гра\-фии~--- анализируются семейные отношения, характер образования, 
физическое развитие, глав\-ные по\-треб\-но\-сти и~интересы, осо\-бен\-но\-сти 
интеллекта, об\-щи\-тель\-ность; 
\item[(г)] личностные тес\-ты~--- психодиагностические 
тес\-ты на оцен\-ку уровня отдельных качеств и~пред\-рас\-по\-ло\-жен\-ность 
к~определенному типу поведения;\\[-14pt]
\end{itemize}
\item интервью~--- беседа, на\-прав\-лен\-ная на сбор информации об опыте 
и~уров\-не знаний и~на оценку профессионально важ\-ных качеств претендента;
\item также анализируют рекомендации (их источники и~оформление) 
и~используют: полиграф, медицинские тес\-ты, психоанализ.\\[-14pt]
\end{itemize}
\end{enumerate}
  
  \textbf{Методика подбора персонала О.\,С.~Насташевской с~использованием 
психограмм и~профессиограмм}~\cite{28-r}. Методика базируется на выделении 
типов лич\-ности по уровню отклонений от тео\-ре\-ти\-че\-ской психограммы через 
сис\-те\-му отбора. Сначала разрабатывается психограмма на основе 
\mbox{профессиограммы} специалиста, на которого объявлен отбор,~--- пе\-ре\-чис\-ля\-ют\-ся 
психологические профессионально важ\-ные качества специалиста~\cite{7-r}. 
Затем выбираются диагностические методики и~проводится со\-бе\-се\-до\-ва\-ние-тес\-ти\-ро\-ва\-ние, 
по результатам которого определяют степень соответствия 
кандидатов требованиям психограммы и~по ним выделяют типы лич\-ности 
(кластерный анализ, $k$-сред\-них). Кластеризация предполагает неаддитивную 
модель учета соответствия психограмме, что повышает адекват\-ность 
получаемого решения. В~итоге кандидаты, тип лич\-ности которых 
с~максимальной степенью соответствует требованиям, об\-суж\-да\-ют\-ся 
руководством.
  
  \textbf{Психологическая оценка персонала при выдвижении в~кад\-ро\-вый 
резерв}~\cite{29-r}.
  Кадровый резерв составляет группа сотрудников, которая долж\-на быть 
обучена внут\-рен\-ни\-ми и~внеш\-ни\-ми экспертами. Создается для обеспечения 
гиб\-кости в~замещении сотрудников. Для выдвижения в~кадровый резерв 
проводятся: оценка профессиональных компетенций, психологическое 
тес\-ти\-ро\-ва\-ние и~экспертное оценивание в~ходе деловых игр (ассесс\-мент-центр). 
Принцип включения в~кадровый резерв базируется на двух основных 
измерениях: профессиональном и~управ\-лен\-че\-ском потенциале, а~также 
эф\-фек\-тив\-ности в~дея\-тель\-ности. Оценивается уровень сле\-ду\-ющих базовых 
психологических характеристик: интеллектуальный уровень (тест структуры 
интеллекта Р.~Амтхауэра, методика оценки социального интеллекта 
Дж.~Гилфорда); лидерский потенциал (калифорнийский личностный опросник
(КЛО); стандартизированный метод исследования лич\-ности Л.\,Н.~Собчик 
(СМИЛ));\linebreak коммуникабельность (КЛО, СМИЛ); психологическая устой\-чи\-вость 
(опросник эмоционального интеллекта Д.\,В.~Люсина, СМИЛ); этич\-ность, 
по\-ря\-доч\-ность во взаимодействии с~коллегами, \mbox{подчиненными}, руководством 
(экспертная оценка и~сбор информации). Со\-труд\-ни\-ка мож\-но за\-чис\-лять 
в~кадровый резерв уже при одновременном наличии сред\-них показателей по 
эф\-фек\-тив\-ности и~потенциалу.
  
  \textbf{Формирование команд}~\cite{10-r}. Выделяют:
  \begin{enumerate}[(1)]
  \item динамический 
подход, на\-прав\-лен\-ный на развитие социоэмоциональных и~инструментальных 
отношений в~команде по\-сред\-ст\-вом различных тренингов~--- базируются на 
стадиях развития групп (M.~Kelly, W.~Wellins, B.\,W.~Tuckman, 
Т.\,Ю.~Базаров)~--- модели здесь дескриптивные; 
\item специально 
организованные со\-ци\-аль\-но-пси\-хо\-ло\-ги\-че\-ские технологии 
формирования коллективного субъекта дея\-тель\-ности~--- команды, а~именно: 
командные испытания для развития эмоциональных отношений; тренинги 
навыков командной работы; командный коучинг; деловые игры, тренинги по 
разработке общего видения. 
\end{enumerate}
  
  При комплектовании/пе\-ре\-комп\-лек\-то\-ва\-нии команд превалирует 
ориентация на формирование гетерогенных групп~\cite{30-r}: 
\begin{enumerate}[(1)]
\item по полу, 
возрасту и~профессии~--- не вызывает трудностей; 
\item по интеллекту 
и~личностным чертам~--- используются тес\-ты оценки IQ, методики 
диагностики когнитивного стиля и~креативности мышления, учитывающие 
специфику дея\-тель\-ности фор\-ми\-ру\-ющей\-ся команды (например, тесты 
Фланагана и~Векслера), а~так\-же методики, основанные на типологическом 
подходе К.\,Г.~Юнга (модели Май\-ерс--Бриггс и~Кейрси~--- включают 
си\-ту\-а\-ци\-он\-но-по\-ве\-ден\-че\-ское тес\-ти\-ро\-ва\-ние и/или глубинное био\-гра\-фи\-че\-ское 
ин\-тервью), и~методики, по\-стро\-ен\-ные на базе концепции командных ролей 
Р.\,М.~Белбина~--- используют со\-по\-став\-ле\-ние результатов применения 
опросника и~внеш\-них оценок, полученных от коллег,~--- 360-гра\-дус\-ная 
об\-рат\-ная связь.
\end{enumerate}
 Диагностика ценностной ориентации осуществляется, 
например, с~по\-мощью проективных методик, об\-ра\-ща\-ющих\-ся непосредственно 
к~символическому содержанию, в~котором объединены и~образ, и~отношение 
к~организации, и~личный опыт, и~ценности, и~переживания. 
  
  За рубежом выделяют четыре основных подхода к~образованию 
команд~\cite{31-r}:
  \begin{enumerate}[(1)]
  \item  основанный на развитии и~согласовании целей команды (E.\,A.~Locke, 
E.~Weldon и~др.)~--- развитие спо\-соб\-ности группы людей достигать своих 
целей;
  \item интерперсональный, ориентированный на анализ процессов 
  и~улучшение межличностных отношений (C.~Argyris, W.~Schutz и~др.); 
  \item ролевой подход~--- улучшение работы команды за счет увеличения 
яс\-ности ролей и,~как следствие, увеличения организационной эф\-фек\-тив\-ности. 
Пред\-ва\-ри\-тел\-ьно сотрудников тес\-ти\-ру\-ют, определяют типы их поведения, 
а~далее со\-еди\-ня\-ют в~команды по принципу взаи\-мо\-до\-пол\-ня\-емости (модели 
Р.\,М.~Белбина и~Т.\,Ю.~Базарова; взаи\-мо\-до\-пол\-ня\-ющая команда по 
И.\,А.~Адизес)~\cite{3-r, 32-r};
  \item проблемно-ори\-ен\-ти\-ро\-ван\-ный подход (W.\,G.~Dyer, R.~Kilman, 
I.~Kilman и~др.)~--- более общий, может вклю\-чать все предыду\-щие. Считается, 
что команда становится более эффективной в~результате совместного решения 
проб\-лем. 
  \end{enumerate}
  
  Результаты анализа подходов из разд.~3 сведены в~таб\-лицу.
  
  
  В таблице курсивом отмечены подходы, которые могут быть 
в~адаптированном виде использованы при моделировании коллективного 
принятия\linebreak\vspace*{-12pt}

\pagebreak

\end{multicols}

 \begin{table*}\small
  \begin{center}
 \tabcolsep=1pt
  \begin{tabular}{|l|c|c|c|c|c|}
  \multicolumn{6}{c}{Анализ отечественных и~зарубежных подходов к~отбору 
специалистов}\\
   \multicolumn{6}{c}{\ }\\[-6pt]
   \hline
\multicolumn{1}{|c|}{Основные подходы к~отбору 
специалистов}&\tabcolsep=0pt\begin{tabular}{c}Качест-\\венные\end{tabular}&
\tabcolsep=0pt\begin{tabular}{c}Количе-\\ ственные\end{tabular}&
\tabcolsep=0pt\begin{tabular}{c}Индиви-\\дуальные\end{tabular}&\tabcolsep=0pt\begin{tabular}{c}Груп-\\ повые\end{tabular}& 
\tabcolsep=0pt\begin{tabular}{c}Возможность\\ моделирования\\ в~рамках\\ 
РАСИГИА\end{tabular}\\
\hline
\textit{Согласованность функционально-ролевых  
ожиданий}~\cite{15-r, 16-r}&+&&&+&$\pm$\\
\hline
\textit{Ценностно-ориентационное единство по А.\,В.~Петровскому}~\cite{17-r}&&+&&+&$\pm$\\
\hline
Комплектование групп с~использованием ОМО~\cite{18-r}&&+&&+&\\
\hline
Соционический метод~\cite{19-r, 20-r}&&+&&+&\\
\hline
Социометрическая теория Я.~Морено~\cite{21-r} &&+&+&+&\\
\hline
\textit{Метод формирования профиля должности}~\cite{7-r}&&+&+&&$\pm$\\
\hline
\textit{Анкетирование и~тестирование}~\cite{7-r}&+&+&+&&$\pm$\\
\hline
Графологический и~морфологический анализ~\cite{7-r} &+&&+&&\\
\hline
\textit{Комплексные исследования в~центрах оценки персонала}~\cite{27-r}&+&+&+&&$\pm$\\
\hline
Биографические тесты и~изучение биографии~\cite{27-r}&+&&+&&\\
\hline
Интервью~\cite{24-r, 25-r, 26-r, 27-r}&+&+&+&&\\
\hline
Ролевые и~имитационные деловые игры~\cite{27-r}&+&+&+&+&\\
\hline
\textit{Методика подбора персонала О.\,С.~Насташевской}~\cite{28-r}&&+&&&$\pm$\\
\hline
Психологическая оценка персонала для кадрового резерва~\cite{29-r} &+&+&+&&\\
\hline
\textit{Динамический подход к~командообразованию}~\cite{10-r}&+&+&&+&$\pm$\\
\hline
\tabcolsep=0pt\begin{tabular}{l}Специальные организованные социально-психологические\\ технологии формирования 
команд~\cite{10-r}\end{tabular}&+&&&+&\\
\hline
\tabcolsep=0pt\begin{tabular}{l}Тесты и~методики оценки когнитивной сферы для\\
формирования гетерогенных групп~\cite{30-r}\end{tabular} &+&+&+&&\\
\hline
Проективные методики~\cite{30-r}&+&+&+&&\\
\hline
\textit{Развитие и~согласование целей команды}~\cite{31-r}&+&+&&+&$\pm$\\
\hline
Интерперсональный подход к~командообразованию~\cite{31-r}&+&+&&+&\\
\hline
\textit{Ролевой подход объединения в~команды}~\cite{30-r, 32-r}&+&+&&+&$\pm$\\
\hline
\textit{Проблемно-ориентированный подход к~формированию команд}&+&+&&&$\pm$\\
\hline
\multicolumn{6}{p{162mm}}{\footnotesize \hspace*{2mm}\textbf{Обозначения:} $\pm$~---  в~рамках 
соответствующего подхода есть ограничения и/или противоречия по выделенному критерию.}
\end{tabular}
\end{center}
\vspace*{-6pt}
\end{table*}
 

\begin{multicols}{2}

\noindent
 решения в~РАСИГИА для отбора моделей специалистов (агентов).

\section{Качество измерительного инструмента отбора 
специалистов}

  Одна из проблем отбора~--- повышение прог\-но\-стич\-ности, на\-деж\-ности 
процедуры выделения таких характеристик человека, которые укажут на его 
по\-сле\-ду\-ющую успеш\-ную профессиональную дея\-тель\-ность и~соответствие 
запросам организации, т.\,е.\ необходимо обоснование качества вы\-бран\-но\-го 
измерительного инструмента (анкеты, тес\-та и~т.\,д.)~\cite{14-r, 28-r}. При этом 
оцениваются:
  \begin{itemize}
\item эмпирическая валидность инструмента, т.\,е.\ соответствие его 
результатов характеристике, для измерения которой он разработан,~--- 
проводится пилотное исследование, в~ходе которого респонденты оценивают 
ис\-сле\-ду\-емый объект при помощи альтернативных анкет. Связь между 
результатами измерения определяется расчетом коэффициента ранговой 
корреляции Спирмена~\cite{14-r};
\item надежность инструмента: 
\begin{enumerate}[(1)]
\item с позиции со\-гла\-со\-ван\-ности, т.\,е.\ степени 
однородности со\-ста\-ва вопросов (заданий) с~точ\-ки зрения из\-ме\-ря\-емой 
характеристики~--- определяется связью каждого конкретного элемента 
инструмента с~общим результатом~\cite{33-r}; 
\item с позиции устой\-чи\-вости~--- 
проводится несколько измерений с~некоторым промежутком времени одним 
и~тем же инструментом. Инструмент устойчив, если имеет мес\-то 
статистически значимое значение коэффициента корреляции между данными 
измерений и~статистически незначимые различия в~сред\-них значениях, 
полученных при измерениях.
\end{enumerate}
\end{itemize}

\vspace*{-6pt}

\section{Заключение}

  В работе представлены результаты исследования, которое по материалам 
открытой печати выявило большое разнообразие подходов к~отбору 
специалистов на долж\-ность и~в~группу (команду, коллектив)~--- групповые, 
индивидуальные, качественные, количественные и~комбинированные. При этом 
инструменты оценки кандидатов должны подвергаться проверке на ва\-лид\-ность
 и~на\-деж\-ность. На практике часто комбинируют несколько методов, чтобы 
повысить качество отбора. По результатам анализа так\-же были выделены 
несколько подходов (при условии их адаптации), которые могут быть в~той или 
иной степени использованы для отбора моделей специалистов (агентов) 
в~искусственный гетерогенный коллектив \mbox{РАСИГИА}\linebreak из всего пула 
интеллектуальных агентов: со\-гла\-со\-ван\-ность функ\-цио\-наль\-но-ро\-ле\-вых 
ожиданий; цен\-ност\-но-ори\-ен\-та\-ци\-он\-ное единство; формирование\linebreak 
профиля долж\-ности; анкетирование и~тес\-ти\-ро\-ва\-ние; принципы комплексных 
исследований в~цент\-рах; методика О.\,С.~На\-ста\-шев\-ской; развитие 
и~согласование целей команды; динамический, \mbox{ролевой} и~проб\-лем\-но-ори\-ен\-ти\-ро\-ван\-ный подходы.
  
{\small\frenchspacing
 {%\baselineskip=10.8pt
 %\addcontentsline{toc}{section}{References}
 \begin{thebibliography}{99}
\bibitem{1-r}
\Au{Куроедова Е.\,О.} Ин\-тер\-нет-курс по дисциплине <<Психология в~управлении  персоналом>>. 
{\sf   http://www.\linebreak \mbox{e-biblio}.ru/book/bib/04\_pravo/psiholog\_v\_uprav\_\linebreak personalom/sg\_online.html}.
\bibitem{2-r}
\Au{Кричевский Р.\,Л., Ду\-бов\-ская Е.\,М.} Социальная психология малой группы.~--- М.: Аспект Пресс, 2001. 318~с.
\bibitem{3-r}
\Au{Колесников А.\,В.} Гетерогенные естественные и~искусственные сис\-те\-мы~// 
Интегрированные модели и~мягкие вы\-чис\-ле\-ния в~искусственном интеллекте.~--- М.: Физматлит, 2013. Т.~1. 
С.~86--103.
\bibitem{4-r}
\Au{Андреева Г.\,М.} Социальная психология.~--- М.: Аспект Пресс, 2009. 393~с.
\bibitem{5-r}
\Au{Меньшиков А.\,А.} Основы интегрированных коммуникаций.~---  
Ком\-со\-мольск-на-Аму\-ре: КнАГУ, 2012. 101~с.
\bibitem{6-r}
\Au{Коноваленко В.\,А., Коноваленко~М.\,Ю., Соломатин~А.\,А.} Психология управления 
персоналом.~--- М.: Юрайт, 2014. 477~с.
\bibitem{7-r}
Психология управ\-ле\-ния персоналом~/ Под ред. Е.\,И.~Рогова.~--- М.: 
Юрайт, 2023. 350~с.



\bibitem{9-r} %8
Социальная психология в~современном мире~/ Под ред. 
Г.\,М.~Андреевой, А.\,И.~Донцова.~--- М.: Аспект Пресс, 2002. 335~с.
\bibitem{10-r} %9
\Au{Короткина Е.\,Д.} Современные технологии создания команды в~организации~// 
Вестник Санкт-Пе\-тер\-бург\-ско\-го университета. Сер.~12. Психология.  Социология. Педагогика, 2009. Т.~3. №\,2. С.~46--53.

\bibitem{8-r} %10
\Au{Нургалиева А.\,М., Ахметшина~А.\,Р., Сайфудинова~Н.\,З.} Современные методики 
формирования эффективной команды в~организации~// СКИФ. Вопросы студенческой науки, 
2018. Вып.~11(27). С.~221--230.

\bibitem{11-r}
\Au{Katzenbach J.\,R., Smith D.\,K.} The discipline of teams~// Harward Business Review, 1993. 
Vol.~71. Iss.~2. P.~111--120.
\bibitem{12-r}
\Au{Картушина T.\,Н.} Командообразование как потребность в~современном процессе 
управления персоналом~// Cо\-ци\-аль\-но-эко\-но\-ми\-че\-ские явления и~процессы, 2013. 
№\,5(051). С.~99--102. 
\bibitem{13-r}
\Au{Жаглин А.\,В., Ульянов А.\,Д.} Основы управ\-ле\-ния и~делопроизводства в~органах 
внут\-рен\-них дел: Альбом схем.~--- М.: Юни\-ти-Да\-на, 2014. 191~с.
\bibitem{14-r}
\Au{Горленко О.\,А., Ерохин Д.\,В., Можаева~Т.\,П.} Управление персоналом.~--- М.: Юрайт, 2023. 217~с.
\bibitem{15-r}
\Au{Кабаченко Т.\,С.} Психология в~управ\-ле\-нии человеческими ресурсами.~--- СПб.: Питер, 2003. 400~с.
\bibitem{16-r}
\Au{Mescon M.\,H., Albert~M., Khedouri~F.} Management.~--- New York, NY, USA: Harper \& 
Row Publs., 1988. 777~p.
\bibitem{17-r}
Психологическая теория коллектива~/ Под ред. А.\,В.~Пет\-ров\-ско\-го.~--- М.: Педагогика, 1979. 
240~с.
\bibitem{18-r}
Рабочая книга практического психолога~/ Под ред. А.\,А.~Бодалева, А.\,А.~Деркача, Л.\,Г.~Лаптева.~--- М.: Изд-во 
Института психотерапии, 2001. 640~с.
\bibitem{19-r}
\Au{Иванов Ю.\,В.} Деловая соционика.~--- М.: Топ-пер\-со\-нал, 2004. 200~с.
\bibitem{20-r}
\Au{Филатова Е.\,С.} Соционика в~портретах и~примерах.~--- М.: Черная белка, 2009. 443~с.
\bibitem{21-r}
\Au{Миронова Е.\,Е.} Сборник психологических тес\-тов. Часть~I: Пособие.~--- Мн.: Женский 
институт \mbox{ЭНВИЛА}, 2005. 155~с.


\bibitem{23-r} %22
Лучшие психологические тесты для профотбора и~профориентации~/ Отв. ред. 
А.\,Ф.~Кудряшов.~--- Петрозаводск: Петроком, 1992. 318~с.
\bibitem{24-r} %23
Психологические тесты~/ Под ред. А.\,А.~Карелина: в~2~т.~--- М.: Владос, 2002. Т.~1. 312~с. 
Т.~2. 246~с.
\bibitem{22-r} %24
\Au{Елисеев О.\,П.} Практикум по психологии личности.~--- М.: Юрайт, 
2023. 390~с.

\bibitem{25-r}
\Au{Иванова С.\,В.} Искусство подбора персонала: как оценить человека за час.~--- М.: 
Альпина Паблишер, 2012. 269~с.
\bibitem{26-r}
\Au{Мякушкин Д.\,Е.} Отбор и~под\-бор персонала.~--- Челябинск: ЮУрГУ, 2006. 
26~с.
\bibitem{27-r}
\Au{Базаров Т.\,Ю.} Технология цент\-ров оцен\-ки персонала: процессы и~результаты. 
Практическое пособие.~--- М.: КноРус, 2021. 301~с.
\bibitem{28-r}
\Au{Насташевская О.\,С.} Психологические аспекты технологии подбора персонала для 
торговой организации~// Вестник Самарской гуманитарной академии. Сер. Психология, 
2015. №\,1(17). С.~11--29.
\bibitem{29-r}
\Au{Васильева И.\,В.} Психотехники и~пси\-хо\-диа\-гно\-сти\-ка в~управ\-ле\-нии персоналом: 
Практическое пособие.~--- М.: Юрайт, 2023. 122~с.
\bibitem{30-r}
\Au{Жуков Ю.\,М., Журавлев А.\,В., Павлова~Е.\,Н.} Технологии командообразования.~--- М.: 
Аспект Пресс, 2008. 320~с.

\pagebreak

\bibitem{31-r}
\Au{Безрукова Е.\,Ю.} Информационно-ме\-то\-ди\-че\-ское обеспечение процесса 
командообразования:\linebreak Дисс.\ \ldots\  канд. псих. наук.~--- М., 1998. 289~с.
\bibitem{32-r}
\Au{Семина А.\,П.} Анализ моделей и~подходов в~формировании команды компании~// 
Вестник Алтайской академии экономики и~права, 2020. №\,12-2. С.~399--404. doi: 
10.17513/vaael.1526.
\bibitem{33-r}
\Au{Яхонтова Е.\,С.} Стратегическое управ\-ле\-ние персоналом.~--- М.: 
Дело, 2013. 378~с.

\end{thebibliography}

 }
 }

\end{multicols}

\vspace*{-6pt}

\hfill{\small\textit{Поступила в~редакцию 05.04.23}}

\vspace*{8pt}

%\pagebreak

%\newpage

%\vspace*{-28pt}

\hrule

\vspace*{2pt}

\hrule

%\vspace*{-2pt}

\def\tit{SELECTION OF SPECIALISTS IN~THE~ORGANIZATION OF~COLLECTIVE SOLVING 
PROBLEMS}


\def\titkol{Selection of specialists in~the~organization of~collective solving 
problems}


\def\aut{S.\,B.~Rumovskaya}

\def\autkol{S.\,B.~Rumovskaya}

\titel{\tit}{\aut}{\autkol}{\titkol}

\vspace*{-10pt}


\noindent
Federal Research Center ``Computer Science and Control'' of the Russian Academy 
of Sciences, 44-2~Vavilov Str., Moscow 119333, Russian Federation


\def\leftfootline{\small{\textbf{\thepage}
\hfill INFORMATIKA I EE PRIMENENIYA~--- INFORMATICS AND
APPLICATIONS\ \ \ 2023\ \ \ volume~17\ \ \ issue\ 2}
}%
 \def\rightfootline{\small{INFORMATIKA I EE PRIMENENIYA~---
INFORMATICS AND APPLICATIONS\ \ \ 2023\ \ \ volume~17\ \ \ issue\ 2
\hfill \textbf{\thepage}}}

\vspace*{3pt}
   
   
      
   \Abste{The study of small groups (collectives, teams), their characteristics, problems, dynamics, and 
features of selection of specialists stands at the intersection of psychology of personnel management and 
social psychology. A~special place in a~wide range of areas of modern science is occupied by modeling the 
interaction of people in small collectives of specialists, in particular, within the framework of a~multiagent 
approach. At the same time, when developing intelligent systems (artificial heterogeneous collectives) 
to solve practical problems, it is now required to combine in the system the models of specialists (agents) 
with incompatible goals and domain models. These agents are created by different development teams. The 
selection of specialists in natural and models of specialists in artificial heterogeneous teams is an important 
task, the results of which influence the further decision-making process. The paper presents an analysis of 
methods and approaches to the selection of specialists and the acquisition of small groups (collectives, 
teams) whose measuring tools should be exposed to quality assessment.}
   
   \KWE{group; small collective of specialists; team; methods of selecting specialists and forming small 
groups; teambuilding}
   
   
   
\DOI{10.14357/19922264230214}{VJWNOE} 

\vspace*{-11pt}

\Ack
   \noindent
   The research was supported by the Russian Science Foundation (project No.\,23-21-00218).
  

%\vspace*{4pt}

  \begin{multicols}{2}

\renewcommand{\bibname}{\protect\rmfamily References}
%\renewcommand{\bibname}{\large\protect\rm References}

{\small\frenchspacing
 {%\baselineskip=10.8pt
 \addcontentsline{toc}{section}{References}
 \begin{thebibliography}{99} 
\bibitem{1-r-1}
   \Aue{Kuroedova, E.\,O.} Internet-kurs po dis\-tsip\-li\-ne ``Psi\-kho\-lo\-giya v~uprav\-le\-nii per\-so\-na\-lom'' 
[Online course on the discipline ``Psychology in personnel management'']. Available at: {\sf 
http://www.e-biblio.ru/book/bib/04\_pravo/ psiholog\_v\_uprav\_personalom/sg\_online.html} (accessed 
May~11, 2023).
\bibitem{2-r-1}
   \Aue{Krichevskiy, R.\,L., and E.\,M.~Dubovskaya}. 2001. \textit{So\-tsi\-al'\-naya psi\-kho\-lo\-giya 
ma\-loy grup\-py} [Social psychology of a~small group]. Moscow: 
Aspect Press. 318~p.
\bibitem{3-r-1}
   \Aue{Kolesnikov, A.\,V.} 2013. Ge\-te\-ro\-gen\-nye es\-test\-ven\-nye i~is\-kus\-stven\-nye sis\-te\-my [Natural 
and artificial heterogeneous systems]. \textit{In\-teg\-ri\-ro\-van\-nye mo\-de\-li i~myag\-kie vy\-chis\-le\-niya 
v~iskusstvennom intellekte} [Integrated models and oft computing in artificial intelligence]. 
Moscow: Fizmatlit. 1:86--103.
\bibitem{4-r-1}
   \Aue{Andreeva, G.\,M.} 2009. \textit{So\-tsi\-al'\-naya psi\-kho\-lo\-giya} [Social psychology]. Moscow: 
Aspect Press. 393~p.
\bibitem{5-r-1}
\Aue{Men'shikov, A.\,A.} 2012. \textit{Os\-no\-vy in\-teg\-ri\-ro\-van\-nykh kom\-mu\-ni\-ka\-tsiy} 
[Fundamentals of integrated communications]. Komsomolsk-on-Amur: 
KnAGU. 101~p.
\bibitem{6-r-1}
\Aue{Konovalenko, V.\,A., M.\,Yu.~Konovalenko, and A.\,A.~Solomatin.} 2014. 
\textit{Psi\-kho\-lo\-giya uprav\-le\-niya per\-so\-na\-lom} 
[Psychology of human resources management]. 
Moscow: Yurayt. 477 p.
\bibitem{7-r-1}
   Rogov, E.\,I., ed.  2023. \textit{Psi\-kho\-lo\-giya uprav\-le\-niya per\-so\-na\-lom} 
   [Psychology of human resources management]. Moscow: 
Yurayt. 350~p.


\bibitem{9-r-1} %8
   Andreeva, G.\,M., and A.\,I.~Dontsov, eds. 2002. \textit{So\-tsi\-al'\-naya psi\-kho\-lo\-giya 
v~so\-vre\-men\-nom mi\-re} [Social psychology in the modern world]. Moscow: Aspect Press. 335~p.
\bibitem{10-r-1} %9
   \Aue{Korotkina, E.\,D.} 2009. So\-vre\-men\-nye tekh\-no\-lo\-gii so\-zda\-niya ko\-man\-dy v~or\-ga\-ni\-za\-tsii 
[Modern approaches to teambuilding in organization]. \textit{Vestnik Sankt-Peterburgskogo 
universiteta. Ser.~12. Psikhologiya. Sotsiologiya. Pedagogika} [Vestnik of Saint Petersburg University. Ser.~12.
Psychology. Sociology. Pedagogy] 3(2):46--53.

\bibitem{8-r-1} %10
   \Aue{Nurgalieva, A.\,M., A.\,R.~Akhmetshina, and N.\,Z.~Sayfudinova.} 2018. So\-vre\-men\-nye 
me\-to\-di\-ki for\-mi\-ro\-va\-niya ef\-fek\-tiv\-noy ko\-man\-dy v~or\-ga\-ni\-za\-tsii [Modern methods of forming an 
effective team in the organization]. \textit{Skif. Voprosy stu\-den\-che\-skoy na\-u\-ki} [Skif. Issues of 
Student Science] 11(27):221--230.

\bibitem{11-r-1}
   \Aue{Katzenbach, J.\,R., and D.\,K.~Smith.} 1993. The discipline of teams. \textit{Harward 
Business Review} 71(2):111--120.
\bibitem{12-r-1}
   \Aue{Kartushina, T.\,N.} 2013. Ko\-man\-do\-ob\-ra\-zo\-va\-nie kak po\-treb\-nost' v~so\-vre\-men\-nom 
pro\-tses\-se uprav\-le\-niya per\-so\-na\-lom [Teambuilding as need in modern HR management]. 
\textit{Sotsial'no-ekonomicheskie yavleniya i~protsessy} [Social-Economic Phenomena and 
Processes] 5(051):99--102.
\bibitem{13-r-1}
   \Aue{Zhaglin, A.\,V., and A.\,D.~Ul'yanov.} 2014. \textit{Osno\-vy uprav\-le\-niya 
i~de\-lo\-pro\-iz\-vod\-st\-va v~or\-ga\-nakh vnut\-ren\-nikh del: Al'bom skhem} 
[Fundamentals of management and office work in the internal affairs bodies: Album of schemes]. Moscow: Unity-Dana. 191~p.
\bibitem{14-r-1}
   \Aue{Gorlenko, O.\,A., D.\,V.~Erokhin, and T.\,P.~Mozhaeva.} 2023. \textit{Uprav\-le\-nie 
per\-so\-na\-lom} [Human resource 
management]. Moscow: Yurayt. 217~p.
\bibitem{15-r-1}
   \Aue{Kabachenko, T.\,S.} 2003. \textit{Psi\-kho\-lo\-giya v~uprav\-le\-nii che\-lo\-ve\-che\-ski\-mi re\-sur\-sa\-mi} 
   [Psychology in human resource management]. Saint Petersburg: 
Piter Publishing House. 400~p.
\bibitem{16-r-1}
   \Aue{Mescon, M.\,H., M.~Albert, and F.~Khedouri.} 1988. \textit{Management}. New York, 
NY: Harper \& Row Publs. 777~p.
\bibitem{17-r-1}
   Petrovskiy, A.\,V., ed. 1979. \textit{Psi\-kho\-lo\-gi\-che\-skaya teo\-riya kol\-lek\-ti\-va} [Psychological 
theory of the team]. Moscow: Pedagogika. 240~p.
\bibitem{18-r-1}
   Bodalev, A.\,A., A.\,A.~Derkach, and L.\,G.~Laptev, eds. 2001. \textit{Ra\-bo\-chaya kni\-ga 
prak\-ti\-che\-sko\-go psi\-kho\-lo\-ga} [Practical 
psychologist's workbook]. Moscow: Publishing house of the 
Institute of Psychotherapy Publs. 640~p.
\bibitem{19-r-1}
   \Aue{Ivanov, Yu.\,V.} 2004. \textit{De\-lo\-vaya so\-tsi\-o\-ni\-ka} [Business socionics]. Moscow:  
Top-personal. 200~p.
\bibitem{20-r-1}
   \Aue{Filatova, E.\,S.} 2009. \textit{So\-tsi\-o\-ni\-ka v~portre\-takh i~pri\-me\-rakh} [Socionics in portraits 
and examples]. Moscow: Chernaya belka. 443~p.
\bibitem{21-r-1}
   \Aue{Mironova, E.\,E.}  2005. \textit{Sbor\-nik psi\-kho\-lo\-gi\-che\-skikh tes\-tov. Chast'~I: Posobie} 
[Collection of psychological tests. Part~I: Manual].  Minsk: Zhenskiy Institut ENVILA
[ENVIL Women's Institute]. 155~p.

\bibitem{23-r-1} %22
   Kudryashov, A.\,F., ed. 1992. \textit{Luch\-shie psi\-kho\-lo\-gi\-che\-skie tes\-ty dlya prof\-ot\-bo\-ra 
i~prof\-ori\-en\-ta\-tsii} [The best psychological tests for vocational selection and vocational guidance]. 
Petrozavodsk: Petrokom. 318~p.
\bibitem{24-r-1} %23
   Karelin, A.\,A., ed. 2002. \textit{Psi\-kho\-lo\-gi\-che\-skie tes\-ty}: v~2 tomakh [Psychological tests in 
two volumes]. Moscow: Vlados. Vol.~1. 312~p. Vol.2. 246~p.

\bibitem{22-r-1} %24
   \Aue{Eliseev, O.\,P.} 2023. \textit{Prak\-ti\-kum po psi\-kho\-lo\-gii lich\-nosti} 
[Practical work on personality psychology]. Moscow: 
Yurayt. 390~p.

\bibitem{25-r-1}
   \Aue{Ivanova, S.\,V.} 2012. \textit{Is\-kus\-stvo pod\-bo\-ra per\-so\-na\-la: kak otse\-nit' che\-lo\-ve\-ka za 
chas} [The art of recruiting: How to evaluate a~person in an hour]. Moscow: Alpina Publisher. 
269~p.
\bibitem{26-r-1}
   \Aue{Myakushkin, D.\,E.} 2006. \textit{Ot\-bor i~pod\-bor personala} [Selection 
and recruitment of personnel]. Chelyabinsk: Publishing Center of South Ural State 
University. 26~p.
\bibitem{27-r-1}
   \Aue{Bazarov, T.\,Yu.} 2021. \textit{Tekh\-no\-lo\-giya tsent\-rov otsen\-ki per\-so\-na\-la: pro\-tses\-sy 
i~re\-zul'\-ta\-ty. Prakticheskoe posobie} [Technology of personnel assessment centers: Processes and 
results. Practical guide]. Moscow: KnoRus. 301~p.
\bibitem{28-r-1}
   \Aue{Nastashevskaya, O.\,S.} 2015. Psi\-kho\-lo\-gi\-che\-skie as\-pek\-ty tekh\-no\-lo\-gii pod\-bo\-ra per\-so\-na\-la 
dlya tor\-go\-voy or\-ga\-ni\-za\-tsii [Psychological aspects of technology recruitment for a~trade 
organization]. \textit{Vest\-nik Sa\-mar\-skoy gu\-ma\-ni\-tar\-noy aka\-de\-mii. Ser. Psi\-kho\-lo\-giya} [Bulletin of 
Samara Academy for the Humanities. Ser. Psychology] 1(17):11--29.
\bibitem{29-r-1}
   \Aue{Vasil'eva, I.\,V.} 2023. \textit{Psi\-kho\-tekh\-ni\-ki i~psi\-kho\-diag\-no\-sti\-ka v~uprav\-le\-nii 
per\-so\-na\-lom: Prakticheskoe posobie} [Psychotechnics and psychodiagnostics in personnel 
management: A~practical guide]. Moscow: Yurayt. 122~p.
\bibitem{30-r-1}
   \Aue{Zhukov, Yu.\,M., A.\,V.~Zhuravlev, and E.\,N.~Pavlova.} 2008. \textit{Tekh\-no\-lo\-gii 
ko\-man\-do\-ob\-ra\-zo\-va\-niya} [Teambuilding technologies]. Moscow: Aspect Press. 320~p.
\bibitem{31-r-1}
   \Aue{Bezrukova, E.\,Yu.} 1998. Informatsionno-metodicheskoe obes\-pe\-che\-nie pro\-tses\-sa 
ko\-man\-do\-ob\-ra\-zo\-va\-niya [Information and methodological support of the teambuilding process]. 
Moscow. PhD Diss. 289~p.
\bibitem{32-r-1}
   \Aue{Semina, A.\,P.} 2020. Ana\-liz mo\-de\-ley i~pod\-kho\-dov v~for\-mi\-ro\-vanii ko\-man\-dy kom\-pa\-nii 
[Analysis of models and approaches in the formation of team in company]. \textit{Vest\-nik Al\-tay\-skoy 
aka\-de\-mii eko\-no\-mi\-ki i~pra\-va} [Bulletin of the Altai Academy of Economics and Law]  
12-2:399--404. doi: 10.17513/vaael.1526.
\bibitem{33-r-1}
   \Aue{Yakhontova, E.\,S.} 2013. \textit{Stra\-te\-gi\-che\-skoe uprav\-le\-nie per\-so\-na\-lom} 
   [Strategic human resources management]. Moscow: Delo. 378~p.
   \end{thebibliography}

 }
 }

\end{multicols}

\vspace*{-6pt}

\hfill{\small\textit{Received April 5, 2023}} 
      
   
   \Contrl
   
   \noindent
   \textbf{Rumovskaya Sophiya B.} (b.\ 1985)~--- Candidate of Science (PhD) in technology, senior 
scientist, Kaliningrad Branch of the Federal Research Center ``Computer Science and Control'' of the 
Russian Academy of Sciences, 5~Gostinaya Str., Kaliningrad 236000, Russian Federation; 
\mbox{sophiyabr@gmail.com}
    
\label{end\stat}

\renewcommand{\bibname}{\protect\rm Литература} 
           %14








\def\stat{authorsrus}
{%\hrule\par
%\vskip 7pt % 7pt
\raggedleft\Large \bf%\baselineskip=3.2ex
О\,Б\ \ А\,В\,Т\,О\,Р\,А\,Х \vskip 17pt
    \hrule
    \par
\vskip 21pt plus 8pt minus 4pt }


\def\tit{\ }

\def\aut{\ }

\def\auf{\ }

\def\leftkol{\ } % ENGLISH ABSTRACTS}

\def\rightkol{ОБ АВТОРАХ} %ENGLISH ABSTRACTS}

\titele{\tit}{\aut}{\auf}{\leftkol}{\rightkol}
      
            \label{st\stat}



\vspace*{24pt}

\begin{multicols}{2}




\noindent
\textbf{Архипов Олег Петрович} (р.\ 1948)~---
кандидат технических наук, директор Орловского филиала Института проб\-лем информатики
Российской академии наук
%302025, г.Орел, Московское шоссе, д.137

\vspace*{3pt}

\noindent
\textbf{Бирюкова Татьяна Константиновна} (р.\ 1968)~---
кандидат фи\-зи\-ко-ма\-те\-ма\-ти\-че\-ских наук, старший научный сотрудник Института проб\-лем информатики
Российской академии наук

\vspace*{3pt}

\noindent 
\textbf{Бобков  Сергей Геннадьевич} (р.\ 1955)~---
доктор технических наук,  заведующий отделением На\-уч\-но-ис\-сле\-до\-ва\-тель\-ско\-го 
института системных исследований Российской академии наук
%117218, Москва, Нахимовский просп., 36, к.1 

\vspace*{3pt}

\noindent \textbf{Васильев Николай Семенович} (р.\ 1952)~--- доктор 
фи\-зи\-ко-ма\-те\-ма\-ти\-че\-ских наук, профессор, 
МГТУ им.\ Н.\,Э.~Баумана 
%, Москва 105005, 2-я Бауманская ул., д.~5,

\vspace*{3pt}

\noindent
\textbf{Гершкович Максим Михайлович} (р.\ 1968)~---
старший научный сотрудник Института проб\-лем информатики
Российской академии наук

\vspace*{3pt}

\noindent 
\textbf{Дьяченко Юрий Георгиевич} (р.\ 1958)~--- кандидат технических наук, 
старший научный сотрудник Института проб\-лем информатики
Российской академии наук

\vspace*{3pt}

\noindent 
\textbf{Ерошенко Александр Андреевич} (р.\ 1989)~--- аспирант кафедры 
математической статистики факультета вычисли\-тельной математики и кибернетики 
Московского государственного университета им.\ М.\,В.~Ломоносова
%119991, Москва ГСП-1, Ленинские горы, д.\ 1, стр. 52

\vspace*{3pt}
 
\noindent 
\textbf{Захаров Виктор Николаевич} (р.\ 1948)~--- 
доктор технических наук, доцент, ученый секретарь Института проб\-лем информатики
Российской академии наук

\vspace*{3pt}

\noindent
\textbf{Зейфман Александр Израилевич} (р.\ 1954)~---
доктор фи\-зи\-ко-ма\-те\-ма\-ти\-че\-ских наук, профессор, 
заведующий кафедрой Вологодского государственного университета; 
старший научный сотрудник Института проб\-лем информатики
Российской академии наук; главный научный сотрудник ИСЭРТ Российской академии наук

\vspace*{3pt}

\noindent
\textbf{Зыкин Сергей Владимирович} (р.\ 1959)~--- 
доктор технических наук, профессор, заведующий лабораторией Института математики 
им.\ С.\,Л.~Соболева Сибирского отделения Российской академии наук, Новосибирск 
%630090, пр.\ ак.\ Коптюга, 4 

\vspace*{4pt}

\noindent
\textbf{Киреев Владимир Иванович} (р.\ 1938)~---
доктор фи\-зи\-ко-ма\-те\-ма\-ти\-че\-ских наук, профессор Московского 
государственного горного университета
%Адрес: Россия, 119991, г. Москва, Ленинский проспект, д. 6

%\columnbreak

\vspace*{4pt}

\noindent
\textbf{Козеренко Елена Борисовна} (р.\ 1959)~---
кандидат филологических наук, заведующая лабораторией Института проб\-лем информатики
Российской академии наук

\vspace*{4pt}

\noindent
\textbf{Королев Виктор Юрьевич} (р.\ 1954)~--- доктор
фи\-зи\-ко-ма\-те\-ма\-ти\-че\-ских наук, профессор кафедры математической 
статистики факультета вычисли\-тельной математики и кибернетики 
Московского государственного университета; 
ведущий научный сотрудник Института проб\-лем информатики
Российской академии наук

\vspace*{4pt}

\noindent
\textbf{Коротышева Анна Владимировна} (р.\ 1988)~---
старший преподаватель Вологодского государственного университета

\vspace*{4pt}

\noindent 
\textbf{Кун Де Турк} (р.\ 1981)~--- научный сотрудник 
исследовательской группы SMACS факультета телекоммуникаций и обработки информации
Университета Гента, Бельгия
%В-9000 Гент, Бельгия

\vspace*{4pt}

\noindent
\textbf{Лупенцов Олег Сергеевич} (р.\ 1986)~---
аспирант Омского государственного института сервиса
%Омск 644043, ул.\ Певцова 13

\vspace*{4pt}

\noindent
\textbf{Лучко Олег Николаевич} (р.\ 1961)~---
кандидат педагогических наук, профессор, заведующий кафедрой 
Омского государственного института сервиса
%Омск 644043, ул.\ Певцова 13

\vspace*{4pt}

\noindent
\textbf{Малашенко Юрий Евгеньевич} (р.\ 1946)~---
доктор фи\-зи\-ко-ма\-те\-ма\-ти\-че\-ских наук, заведующий сектором 
Вычислительного центра им.\ А.\,А.~Дородницына Российской академии наук
%Адрес: 119333, Москва, ул. Вавилова, 40,

\vspace*{4pt}

\noindent
\textbf{Маньяков Юрий Анатольевич} (р.\ 1984)~---
кандидат технических наук, научный сотрудник Орловского филиала Института проб\-лем информатики
Российской академии наук
%302025, г.Орел, Московское шоссе, д.137

\vspace*{4pt}

\noindent
\textbf{Маренко Валентина Афанасьевна} (р.\ 1951)~---
кандидат технических наук, доцент, старший научный сотрудник 
Института математики им.\ С.\,Л.~Соболева Сибирского отделения Российской академии наук
%Новосибирск 630090, пр. ак. Коптюга, 4 

\vspace*{3pt}

\noindent 
\textbf{Морозов Евсей Викторович} (р.\ 1947)~--- доктор 
фи\-зи\-ко-ма\-те\-ма\-ти\-че\-ских, профессор, ведущий научный сотрудник 
Института прикладных математических исследований Карельского научного центра Российской
академии наук; 
%%185910 Россия, Республика Карелия, г.\ Петрозаводск, ул.\ Пушкинская, 11
профессор Петрозаводского государственного университета, Петрозаводск
%185910 Россия, Республика Карелия, г.\ Петрозаводск, пр.\ Ленина, 33

%\pagebreak

\vspace*{3pt}

\noindent
\textbf{Назарова Ирина Александровна} (р.\ 1966)~---
кандидат фи\-зи\-ко-ма\-те\-ма\-ти\-че\-ских наук, 
научный сотрудник Вычислительного центра им.\ А.\,А.~Дородницына Российской академии наук 
%Адрес: 119333, Москва, ул. Вавилова, 40

\vspace*{3pt}

\noindent
\textbf{Павлов Игорь Валерианович} (р.\ 1945)~--- 
доктор фи\-зи\-ко-ма\-те\-ма\-ти\-че\-ских наук, профессор МГТУ им.\ Н.\,Э.~Баумана 
%Москва 105005, 2-я Бауманская ул., д.~5 

%\pagebreak

\vspace*{3pt}

\noindent 
\textbf{Потахина Любовь Викторовна} (р.\ 1989)~--- аспирантка
Института прикладных математических исследований Карельского научного центра
Российской академии наук; 
%%185910 Россия, Республика Карелия, г.\ Петрозаводск, ул.\ Пушкинская, 11
инженер Петрозаводского государственного университета, Петрозаводск
%185910 Россия, Республика Карелия, г.\ Петрозаводск, пр.\ Ленина, 33

\vspace*{3pt}

\noindent 
\textbf{Рождественский Юрий Владимирович} (р.\ 1952)~--- 
кандидат технических наук, заведующий сектором Института проб\-лем информатики
Российской академии наук

\vspace*{3pt}

\noindent 
\textbf{Синицын Игорь Николаевич} (р.\ 1940)~--- доктор технических наук,
профессор, заслуженный деятель\linebreak\vspace*{-12pt}

\columnbreak

\noindent
 науки РФ, заведующий отделом Института проб\-лем информатики
Российской академии наук

\vspace*{7pt}


\noindent
\textbf{Сиротинин Денис Олегович} (р.\ 1984)~---
кандидат технических наук, научный сотрудник Орловского филиала Института проб\-лем информатики
Российской академии наук
%302025, г.Орел, Московское шоссе, д.137

\vspace*{7pt}

%\columnbreak

\noindent 
\textbf{Соколов  Игорь Анатольевич} (р.\ 1954)~--- академик (действительный член) Российской 
академии наук, доктор технических наук, директор Института проб\-лем информатики
Российской академии наук

\vspace*{7pt}

\noindent
\textbf{Степченков Юрий Афанасьевич} (р.\ 1951)~---
кандидат технических наук, заведующий отделом Института проб\-лем информатики
Российской академии наук

\vspace*{7pt}

\noindent
\textbf{Сурков Алексей Викторович} (р.\ 1978)~--- 
старший научный сотрудник На\-уч\-но-ис\-сле\-до\-ва\-тель\-ско\-го 
института системных исследований Российской академии наук
%117218, Москва, Нахимовский просп., 36, к.1 

\vspace*{7pt}

\noindent 
\textbf{Шестаков Олег Владимирович} (р.\ 1976)~--- доктор 
фи\-зи\-ко-ма\-те\-ма\-ти\-че\-ских, доцент кафедры математической статистики 
факультета вычисли\-тельной математики и кибернетики Московского 
государственного университета им.\ М.\,В.~Ломоносова; 
%119991, Москва ГСП-1, Ленинские горы, д.\ 1, стр. 52
старший научный сотрудник Института проб\-лем информатики
Российской академии наук
%, Москва 119333, ул. Вавилова, д.~44, корп.~2

\vspace*{7pt}

\noindent 
\textbf{Шоргин Сергей Яковлевич} (р.\ 1952.)~--- доктор
фи\-зи\-ко-ма\-те\-ма\-ти\-че\-ских наук, профессор, заместитель директора Института 
проб\-лем информатики Российской академии наук





%%%%%%%%%%%%%%%%%%%%%%%%%%%%%%%%%%%%%%%%%%%%%%%%%%%%%%%%%%%%%%%%%%%%%%%%%%%%%%%




%\def\rightkol{ОБ АВТОРАХ}
%\def\leftkol{ОБ АВТОРАХ}

 \label{end\stat}





%\def\leftfootline{\small{\textbf{\thepage}
%\hfill ИНФОРМАТИКА И ЕЁ ПРИМЕНЕНИЯ\ \ \ том~7\ \ \ выпуск~1\ \ \ 2013}
%}%
% \def\rightfootline{\small{ИНФОРМАТИКА И ЕЁ ПРИМЕНЕНИЯ\ \ \ том~7\ \ \ выпуск~1\ \ \ 2013
%\hfill \textbf{\thepage}}}


%\thispagestyle{myheadings}



\end{multicols}

\newpage  

%\def\stat{cont}
{%\hrule\par
%\vskip 7pt % 7pt
\raggedleft\Large \bf%\baselineskip=3.2ex
А\,В\,Т\,О\,Р\,С\,К\,И\,Й\ \ У\,К\,А\,З\,А\,Т\,Е\,Л\,Ь\ \ З\,А\ \ 2\,0\,0\,7 г. \vskip 17pt
    \hrule
    \par
\vskip 21pt plus 6pt minus 3pt }

\label{st\stat}

\def\tit{\ }

\def\aut{\ }
\def\auf{\ }

\def\leftkol{\ } % ENGLISH ABSTRACTS}

\def\rightkol{\ } %ENGLISH ABSTRACTS}

\titele{\tit}{\aut}{\auf}{\leftkol}{\rightkol}


\contentsline {chapter}{\ }{Выпуск \quad Стр.} 
\contentsline {section}{\textbf{Батракова Д.\,А., Королев В.\,Ю., Шоргин С.\,Я.}\ \ Новый метод вероятностно-ста\-ти\-сти\-че\-ско\-го анализа информационных потоков в\nobreakspace {}телекоммуникационных сетях}{\qquad 1 \qquad 40} 
\contentsline {section}{\textbf{Борисов А.\,В.}\ \ Байесовское оценивание в системах наблюдения с\nobreakspace {}марковскими скачкообразными процессами: игровой подход}{\qquad 2 \qquad 65}
\contentsline {section}{\textbf{Босов А.\,В., Иванов А.\,В.}\ \ Программная инфраструктура информационного Web-пор\-тала}{\qquad 2 \qquad 50}
\contentsline {section}{\textbf{Захаров В.\,Н., Калиниченко Л.\,А., Соколов И.\,А., Ступников С.\,А.}\ \ Конструирование канонических информационных моделей для интегрированных информационных систем}{\qquad 2 \qquad 15}
\contentsline {section}{\textbf{Захаров В.\,Н., Козмидиади В.\,А.}\ \ Средства обеспечения отказоустойчивости при\-ло\-жений}{\qquad 1 \qquad 14} 
\contentsline {section}{\textbf{Иванов А.\,В.}\ \ см. Босов А.\,В.\hfill\hfill\hfill\hfill\hfill\hfill\hfill\hfill\hfill\hfill\hfill\hfill\hfill\hfill\hfill\hfill\hfill\hfill\hfill\hfill\hfill\hfill\hfill\hfill\hfill\hfill\hfill\hfill\hfill\hfill\hfill\hfill\hfill\hfill\hfill}{\ }
\contentsline {section}{\textbf{Ильин В.\,Д., Соколов И.\,А.}\ \ Символьная модель системы знаний информатики в\nobreakspace {}че\-ло\-ве\-ко-автоматной среде}{\qquad 1 \qquad 66} 
\contentsline {section}{\textbf{Калиниченко Л.\,А.}\ \ см. Захаров В.\,Н.\hfill\hfill\hfill\hfill\hfill\hfill\hfill\hfill\hfill\hfill\hfill\hfill\hfill\hfill\hfill\hfill\hfill\hfill\hfill\hfill\hfill\hfill\hfill\hfill\hfill\hfill\hfill\hfill\hfill\hfill\hfill\hfill\hfill\hfill\hfill}{\ }
\contentsline {section}{\textbf{Козеренко Е.\,Б.}\ \ Лингвистическое моделирование для систем машинного перевода и обработки знаний}{\qquad 1 \qquad 54} 
\contentsline {section}{\textbf{Козмидиади В.\,А.}\ \ см. Захаров В.\,Н.\hfill\hfill\hfill\hfill\hfill\hfill\hfill\hfill\hfill\hfill\hfill\hfill\hfill\hfill\hfill\hfill\hfill\hfill\hfill\hfill\hfill\hfill\hfill\hfill\hfill\hfill\hfill\hfill\hfill\hfill\hfill\hfill\hfill\hfill\hfill }{\ } 
\contentsline {section}{\textbf{Королев В.\,Ю.}\ \ см. Батракова Д.\,А.\hfill\hfill\hfill\hfill\hfill\hfill\hfill\hfill\hfill\hfill\hfill\hfill\hfill\hfill\hfill\hfill\hfill\hfill\hfill\hfill\hfill\hfill\hfill\hfill\hfill\hfill\hfill\hfill\hfill\hfill\hfill\hfill\hfill\hfill\hfill}{\ } 
\contentsline {section}{\textbf{Кудрявцев А.\,А., Шоргин С.\,Я.}\ \ Байесовский подход к\nobreakspace {}анализу систем массового обслуживания и\nobreakspace {}показателей надежности}{\qquad 2 \qquad 76}
\contentsline {section}{\textbf{Печинкин А.\,В., Соколов И.\,А., Чаплыгин В.\,В.}\ \ Многолинейная система массового обслуживания с конечным накопителем и ненадежными приборами}{\qquad 1 \qquad 27} 
\contentsline {section}{\textbf{Печинкин А.\,В., Соколов И.\,А., Чаплыгин В.\,В.}\ \ Стационарные характеристики многолинейной\nobreakspace {}системы массового обслуживания с\nobreakspace {}одновременными отказами приборов}{\qquad 2 \qquad 39}
\contentsline {section}{\textbf{Синицын И.\,Н.}\ \ Корреляционные методы построения аналитических информационных моделей флуктуаций полюса Земли по априорным данным}{\qquad 2 \qquad \hphantom{9}2}
\contentsline {section}{\textbf{Синицын И.\,Н.}\ \ Развитие теории фильтров Пугачева для оперативной обработки информации в стохастических системах}{{\qquad 1 \qquad \hphantom{9}3}} 
\contentsline {section}{\textbf{Соколов И.\,А.}\ \ см. Захаров В.\,Н.\hfill\hfill\hfill\hfill\hfill\hfill\hfill\hfill\hfill\hfill\hfill\hfill\hfill\hfill\hfill\hfill\hfill\hfill\hfill\hfill\hfill\hfill\hfill\hfill\hfill\hfill\hfill\hfill\hfill\hfill\hfill\hfill\hfill\hfill\hfill}{\ }
\contentsline {section}{\textbf{Соколов И.\,А.}\ \ см. Ильин В.\,Д.\hfill\hfill\hfill\hfill\hfill\hfill\hfill\hfill\hfill\hfill\hfill\hfill\hfill\hfill\hfill\hfill\hfill\hfill\hfill\hfill\hfill\hfill\hfill\hfill\hfill\hfill\hfill\hfill\hfill\hfill\hfill\hfill\hfill\hfill\hfill}{\ } 
\contentsline {section}{\textbf{Соколов И.\,А.}\ \ см. Печинкин А.\,В.\hfill\hfill\hfill\hfill\hfill\hfill\hfill\hfill\hfill\hfill\hfill\hfill\hfill\hfill\hfill\hfill\hfill\hfill\hfill\hfill\hfill\hfill\hfill\hfill\hfill\hfill\hfill\hfill\hfill\hfill\hfill\hfill\hfill\hfill\hfill}{\ } 
\contentsline {section}{\textbf{Соколов И.\,А.}\ \ см. Печинкин А.\,В.\hfill\hfill\hfill\hfill\hfill\hfill\hfill\hfill\hfill\hfill\hfill\hfill\hfill\hfill\hfill\hfill\hfill\hfill\hfill\hfill\hfill\hfill\hfill\hfill\hfill\hfill\hfill\hfill\hfill\hfill\hfill\hfill\hfill\hfill\hfill}{\ }
\contentsline {section}{\textbf{Ступников С.\,А.}\ \ см. Захаров В.\,Н.\hfill\hfill\hfill\hfill\hfill\hfill\hfill\hfill\hfill\hfill\hfill\hfill\hfill\hfill\hfill\hfill\hfill\hfill\hfill\hfill\hfill\hfill\hfill\hfill\hfill\hfill\hfill\hfill\hfill\hfill\hfill\hfill\hfill\hfill\hfill}{\ }
\contentsline {section}{\textbf{Чаплыгин В.\,В.}\ \ см. Печинкин А.\,В.\hfill\hfill\hfill\hfill\hfill\hfill\hfill\hfill\hfill\hfill\hfill\hfill\hfill\hfill\hfill\hfill\hfill\hfill\hfill\hfill\hfill\hfill\hfill\hfill\hfill\hfill\hfill\hfill\hfill\hfill\hfill\hfill\hfill\hfill\hfill}{\ } 
\contentsline {section}{\textbf{Чаплыгин В.\,В.}\ \ см. Печинкин А.\,В.\hfill\hfill\hfill\hfill\hfill\hfill\hfill\hfill\hfill\hfill\hfill\hfill\hfill\hfill\hfill\hfill\hfill\hfill\hfill\hfill\hfill\hfill\hfill\hfill\hfill\hfill\hfill\hfill\hfill\hfill\hfill\hfill\hfill\hfill\hfill}{\ }
\contentsline {section}{\textbf{Шоргин С.\,Я.}\ \ см. Батракова Д.\,А.\hfill\hfill\hfill\hfill\hfill\hfill\hfill\hfill\hfill\hfill\hfill\hfill\hfill\hfill\hfill\hfill\hfill\hfill\hfill\hfill\hfill\hfill\hfill\hfill\hfill\hfill\hfill\hfill\hfill\hfill\hfill\hfill\hfill\hfill\hfill}{\ } 
\contentsline {section}{\textbf{Шоргин С.\,Я.}\ \ см. Кудрявцев А.\,А.\hfill\hfill\hfill\hfill\hfill\hfill\hfill\hfill\hfill\hfill\hfill\hfill\hfill\hfill\hfill\hfill\hfill\hfill\hfill\hfill\hfill\hfill\hfill\hfill\hfill\hfill\hfill\hfill\hfill\hfill\hfill\hfill\hfill\hfill\hfill}{\ }
%\thispagestyle{myheadings}
\def\leftfootline{\small{\textbf{\thepage}
\hfill ИНФОРМАТИКА И ЕЁ ПРИМЕНЕНИЯ\ \ \ том~1\ \ \ выпуск~2\ \ \ 2007}
}%
 \def\rightfootline{\small{ИНФОРМАТИКА И ЕЁ ПРИМЕНЕНИЯ\ \ \ том~1\ \ \ выпуск~2\ \ \ 2007
 \hfill \textbf{\thepage}}}
 \label{end\stat} 
                     
%\def\stat{cont-e}
{%\hrule\par
%\vskip 7pt % 7pt
\raggedleft\Large \bf%\baselineskip=3.2ex
2\,0\,0\,7\ \ A\,U\,T\,H\,O\,R\ \ I\,N\,D\,E\,X \vskip 17pt
    \hrule
    \par
\vskip 21pt plus 6pt minus 3pt }

\label{st\stat}

\def\tit{\ }

\def\aut{\ }
\def\auf{\ }

\def\leftkol{\ } % ENGLISH ABSTRACTS}

\def\rightkol{\ } %ENGLISH ABSTRACTS}

\titele{\tit}{\aut}{\auf}{\leftkol}{\rightkol}


\contentsline {chapter}{\ }{Issue \quad Page} 
\contentsline {subsection}{\textbf{Batrakova D.\,A., Korolev V.\,Yu., Shorgin S.\,Ya.}\ \ A New Method for the Probabilistic and Statistical Analysis of Information Flows in Telecommunication Networks}{\qquad 1 \qquad 40} 
\contentsline {subsection}{\textbf{Borisov A.\,V.}\ \ Bayesian Estimation in\nobreakspace {}Observation Systems with\nobreakspace {}Markov Jump Processes: Game-Theoretic Approach}{\qquad 2 \qquad 65} 
\contentsline {subsection}{\textbf{Bosov A.\,V., Ivanov A.\,V.}\ \ Linguistic Simulation for Machine Translation and Knowledge Management Systems}{\qquad 2 \qquad 50} 
\contentsline {subsection}{\textbf{Chaplygin V.\,V.} see Pechinkin A.\,V.\hfill\hfill\hfill\hfill\hfill\hfill\hfill\hfill\hfill\hfill\hfill\hfill\hfill\hfill\hfill\hfill\hfill\hfill\hfill\hfill\hfill\hfill\hfill\hfill\hfill\hfill\hfill\hfill\hfill\hfill\hfill\hfill\hfill\hfill\hfill}{\ }
\contentsline {subsection}{\textbf{Chaplygin V.\,V.} see Pechinkin A.\,V.\hfill\hfill\hfill\hfill\hfill\hfill\hfill\hfill\hfill\hfill\hfill\hfill\hfill\hfill\hfill\hfill\hfill\hfill\hfill\hfill\hfill\hfill\hfill\hfill\hfill\hfill\hfill\hfill\hfill\hfill\hfill\hfill\hfill\hfill\hfill}{\ }
\contentsline {subsection}{\textbf{Ilyin V.\,D., Sokolov I.\,A.}\ \ The Symbol Model of Informatics Knowledge System in Human-Automaton Environment}{\qquad 1 \qquad 66} 
\contentsline {subsection}{\textbf{Ivanov A.\,V.} see Bosov A.\,V.\hfill\hfill\hfill\hfill\hfill\hfill\hfill\hfill\hfill\hfill\hfill\hfill\hfill\hfill\hfill\hfill\hfill\hfill\hfill\hfill\hfill\hfill\hfill\hfill\hfill\hfill\hfill\hfill\hfill\hfill\hfill\hfill\hfill\hfill\hfill}{\ }
\contentsline {subsection}{\textbf{Kalinichenko L.\,A.} see Zakharov V.\,N.\hfill\hfill\hfill\hfill\hfill\hfill\hfill\hfill\hfill\hfill\hfill\hfill\hfill\hfill\hfill\hfill\hfill\hfill\hfill\hfill\hfill\hfill\hfill\hfill\hfill\hfill\hfill\hfill\hfill\hfill\hfill\hfill\hfill\hfill\hfill}{\ }
\contentsline {subsection}{\textbf{Korolev V.\,Yu.} see Batrakova D.\,A.\hfill\hfill\hfill\hfill\hfill\hfill\hfill\hfill\hfill\hfill\hfill\hfill\hfill\hfill\hfill\hfill\hfill\hfill\hfill\hfill\hfill\hfill\hfill\hfill\hfill\hfill\hfill\hfill\hfill\hfill\hfill\hfill\hfill\hfill\hfill}{\ }
\contentsline {subsection}{\textbf{Kozerenko E.\,B.}\ \ Linguistic Simulation for Machine Translation and Knowledge Management Systems}{\qquad 1 \qquad 54} 
\contentsline {subsection}{\textbf{Kozmidiady V.\,A.} see Zakharov V.\,N.\hfill\hfill\hfill\hfill\hfill\hfill\hfill\hfill\hfill\hfill\hfill\hfill\hfill\hfill\hfill\hfill\hfill\hfill\hfill\hfill\hfill\hfill\hfill\hfill\hfill\hfill\hfill\hfill\hfill\hfill\hfill\hfill\hfill\hfill\hfill}{\ }
\contentsline {subsection}{\textbf{Kudryavtsev A.\,A., Shorgin S.\,Ya.}\ \ Bayesian Approach to Queueing Systems and Reliability Characteristics}{\qquad 2 \qquad 76} 
\contentsline {subsection}{\textbf{Pechinkin A.\,V., Sokolov I.\,A., Chaplygin V.\,V.}\ \ Multichannel Queuing System with Finite Buffer and Unreliable Servers}{\qquad 1 \qquad 27} 
\contentsline {subsection}{\textbf{Pechinkin A.\,V., Sokolov I.\,A., Chaplygin V.\,V.}\ \ Stationary Characteristics of a Multichannel Queueing System with\nobreakspace {}Simultaneous Refusals of Servers}{\qquad 2 \qquad 39} 
\contentsline {subsection}{\textbf{Shorgin S.\,Ya.} see Batrakova D.\,A.\hfill\hfill\hfill\hfill\hfill\hfill\hfill\hfill\hfill\hfill\hfill\hfill\hfill\hfill\hfill\hfill\hfill\hfill\hfill\hfill\hfill\hfill\hfill\hfill\hfill\hfill\hfill\hfill\hfill\hfill\hfill\hfill\hfill\hfill\hfill}{\ }
\contentsline {subsection}{\textbf{Shorgin S.\,Ya.} see Kudryavtsev A.\,A.\hfill\hfill\hfill\hfill\hfill\hfill\hfill\hfill\hfill\hfill\hfill\hfill\hfill\hfill\hfill\hfill\hfill\hfill\hfill\hfill\hfill\hfill\hfill\hfill\hfill\hfill\hfill\hfill\hfill\hfill\hfill\hfill\hfill\hfill\hfill}{\ }
\contentsline {subsection}{\textbf{Sinitsyn I.\,N.}\ \ Correlational Methods for Analytical Informational Models of the Earth Pole Fluctuations Design Based on a priori Data}{\qquad 2 \qquad \hphantom{9}2}
\contentsline {subsection}{\textbf{Sinitsyn I.\,N.}\ \ Development of Pugachev Filtering for Stochastic Systems}{\qquad 1 \qquad \hphantom{9}3}
\contentsline {subsection}{\textbf{Sokolov I.\,A.} see Ilyin V.\,D.\hfill\hfill\hfill\hfill\hfill\hfill\hfill\hfill\hfill\hfill\hfill\hfill\hfill\hfill\hfill\hfill\hfill\hfill\hfill\hfill\hfill\hfill\hfill\hfill\hfill\hfill\hfill\hfill\hfill\hfill\hfill\hfill\hfill\hfill\hfill}{\ }
\contentsline {subsection}{\textbf{Sokolov I.\,A.} see Pechinkin A.\,V.\hfill\hfill\hfill\hfill\hfill\hfill\hfill\hfill\hfill\hfill\hfill\hfill\hfill\hfill\hfill\hfill\hfill\hfill\hfill\hfill\hfill\hfill\hfill\hfill\hfill\hfill\hfill\hfill\hfill\hfill\hfill\hfill\hfill\hfill\hfill}{\ }
\contentsline {subsection}{\textbf{Sokolov I.\,A.} see Pechinkin A.\,V.\hfill\hfill\hfill\hfill\hfill\hfill\hfill\hfill\hfill\hfill\hfill\hfill\hfill\hfill\hfill\hfill\hfill\hfill\hfill\hfill\hfill\hfill\hfill\hfill\hfill\hfill\hfill\hfill\hfill\hfill\hfill\hfill\hfill\hfill\hfill}{\ }
\contentsline {subsection}{\textbf{Sokolov I.\,A.} see Zakharov V.\,N.\hfill\hfill\hfill\hfill\hfill\hfill\hfill\hfill\hfill\hfill\hfill\hfill\hfill\hfill\hfill\hfill\hfill\hfill\hfill\hfill\hfill\hfill\hfill\hfill\hfill\hfill\hfill\hfill\hfill\hfill\hfill\hfill\hfill\hfill\hfill}{\ }
\contentsline {subsection}{\textbf{Stupnikov S.\,A.} see Zakharov V.\,N.\hfill\hfill\hfill\hfill\hfill\hfill\hfill\hfill\hfill\hfill\hfill\hfill\hfill\hfill\hfill\hfill\hfill\hfill\hfill\hfill\hfill\hfill\hfill\hfill\hfill\hfill\hfill\hfill\hfill\hfill\hfill\hfill\hfill\hfill\hfill}{\ }
\contentsline {subsection}{\textbf{Zakharov V.\,N., Kalinichenko L.\,A., Sokolov I.\,A., Stupnikov S.\,A.}\ \ Development of Canonical Information Models for Integrated Information Systems}{\qquad 2 \qquad 15} 
\contentsline {subsection}{\textbf{Zakharov V.\,N., Kozmidiady V.\,A.}\ \ Means Providing Applications Fault Tolerance}{\qquad 1 \qquad 14} 
\def\leftfootline{\small{\textbf{\thepage}
\hfill ИНФОРМАТИКА И ЕЁ ПРИМЕНЕНИЯ\ \ \ том~1\ \ \ выпуск~2\ \ \ 2007}
}%
 \def\rightfootline{\small{ИНФОРМАТИКА И ЕЁ ПРИМЕНЕНИЯ\ \ \ том~1\ \ \ выпуск~2\ \ \ 2007
 \hfill \textbf{\thepage}}}
 \label{end\stat} 


%\end{document}

%
\def\stat{rekl}
%\label{preobr}

%\def\tit{АКАДЕМИК ПУГАЧЁВ  ВЛАДИМИР СЕМЁНОВИЧ\\
%25.03.1911--25.03.1998}


%   \vspace*{-48pt}
%   \begin{center}\LARGE
%Академик Пугачёв  Владимир Семёнович\\ (25.03.1911--25.03.1998)
%   \end{center}

   %\vspace*{2.5mm}

   \begin{center}

{\prgsh\LARGE
ЮБИЛЕИ}

\end{center}
%\hrule

\vspace*{6pt}


   \vspace*{8mm}

   \thispagestyle{empty}


%\def\stat{emel}


\section*{К 70-летию заместителя директора ИПИ РАН,\\ члена редколлегии журнала
<<Информатика и её применения>>\\ доктора технических наук В.\,И.~Будзко}

\vspace*{18pt}




          \begin{multicols}{2}

%            \label{st\stat}

\begin{center}
\vspace*{1pt}
\mbox{%
\epsfxsize=78mm
\epsfbox{bud-1.eps}
}
\end{center}

\vspace*{12pt}

      14 августа 2014~г.\ исполнилось 70~лет за\-мес\-ти\-те\-лю директора ИПИ РАН по
научной работе доктору технических наук Владимиру Игоревичу Будзко.

      Владимир Игоревич Будзко родился в г.~Москве. Высшее образование получил на факультете
элект\-рон\-но-вы\-чис\-ли\-тель\-ных устройств в Московском
ин\-же\-нер\-но-фи\-зи\-че\-ском институте
(МИФИ), который он окончил в 1968~г., после чего был на\-прав\-лен для прохождения
службы в одну из войс\-ко\-вых частей, где прошел путь от инженера до первого заместителя
командира войсковой части.

      С приходом В.\,И.~Будзко в ИПИ РАН (2001~г.)\ в институте
сформировалось новое научное на\-прав\-ле\-ние теоретических исследований~--- <<Постро\-ение
ин\-фор\-ма\-ци\-он\-но-те\-ле\-ком\-му\-ни\-ка\-ци\-он\-ных\linebreak сис\-тем
высокой до\-ступ\-ности>>. В~рамках этого
направления выполнен широкий круг фундаментальных исследований по поиску подходов и
определению принципов построения средств обеспечения доступности, конфиденциальности
и целостности современных крупномасштабных
ин\-фор\-ма\-ци\-он\-но-те\-ле\-ком\-му\-ни\-ка\-ци\-он\-ных
сис\-тем (ИТС). Разработаны основные сис\-тем\-но-тех\-ни\-че\-ские принципы и базовые
архитектурные решения построения перспективных для условий России ИТС с
централизованной обработкой и хранением информации, сочетающих в себе свойства
высокой доступности, отказо- и катастрофоустойчивости, информационной защищенности.
Определены принципы, методы и математические основы рационального построения и
оптимизации средств восстановления функционирования центров обработки данных (ЦОД)
после возникновения отказов и катастроф, передачи и хранения данных, обеспечения
информационной безопасности при достижении минимальной совокупной стоимости
владения такими системами. Результаты нашли практическое воплощение при реализации
проектов в интересах ряда отечественных государственных и негосударственных
организаций, таких как Банк России (БР), Внешторгбанк, ОАО <<ГМК <<Норильский Никель>>,
<<Газпром>>, Минэкономразвития России, Правительство Москвы, а также ряд силовых
ведомств.

      Под руководством В.\,И.~Будзко начиная с 2001~г.\ выполнен комплекс
      на\-уч\-но-ис\-сле\-до\-ва\-тель\-ских и
      опыт\-но-кон\-ст\-рук\-тор\-ских работ (свыше 100~проектов),
направленных на развитие электронной информационной технологии БР.
Разработаны концепции развития ИТС БР сначала до 2008~г., а затем до 2013~г., которые
были приняты в качестве основы проведения технической политики. За реализацию проекта
<<Катастрофоустойчивая тер\-ри\-то\-ри\-аль\-но-рас\-пре\-де\-лен\-ная
      ин\-фор\-ма\-ци\-он\-но-те\-ле\-ком\-му\-ни\-ка\-ци\-он\-ная сис\-те\-ма централизованной
обработки банковской информации>> В.\,И.~Будзко удостоен Премии Правительства РФ в
области науки и техники за 2010~г.

      В.\,И.~Будзко возглавлял и возглавляет работы по ряду других прикладных проектов,
связанных с созданием, совершенствованием и развитием крупномасштабных ИТС.

      В.\,И.~Будзко~--- генерал-майор, доктор технических наук, член-кор\-рес\-пон\-дент
Академии криптографии РФ, известный ученый в области информатики и применения
информационных технологий при построении территориально распределенных ИТС
различного назначения. Является автором свыше 250~научных работ, опубликованных в
на\-уч\-но-тех\-ни\-че\-ских и специальных изданиях.

    \thispagestyle{empty}

      В.\,И.~Будзко уделяет большое внимание подготовке научных кадров. Под его
руководством защищено 6~диссертаций на соискание ученой степени кандидата
технических наук. Свыше 30~лет он читает лекции в ИКСИ Академии ФСБ, профессор
кафедры НИЯУ МИФИ. Является членом двух диссертационных советов, главным
редактором журнала <<Системы высокой доступности>> и членом редколлегии журнала
<<Информатика и её применения>>.

      \bigskip

      Редакционный совет и Редакционная коллегия журнала <<Информатика и её
применения>> сердечно поздравляют Владимира Игоревича Будзко с 70-ле\-ти\-ем и желают
крепкого здоровья и новых научных достижений.

\end{multicols}


%%Информатика Т 16 Год 2022-1\\
\def\stat{cont}
{%\hrule\par
%\vskip 7pt % 7pt
\raggedleft\Large \bf%\baselineskip=3.2ex
А\,В\,Т\,О\,Р\,С\,К\,И\,Й\ \ У\,К\,А\,З\,А\,Т\,Е\,Л\,Ь\ \ З\,А\ \ 2\,0\,2\,2 г. \vskip 17pt
 \hrule
 \par
\vskip 21pt plus 6pt minus 3pt }

\label{st\stat}

\def\tit{\ }

\def\aut{\ }
\def\auf{\ }

\def\leftkol{\ } % ENGLISH ABSTRACTS}

\def\rightkol{\ } %АВТОРСКИЙ УКАЗАТЕЛЬ ЗА 2021 г.} %ENGLISH ABSTRACTS}

\titele{\tit}{\aut}{\auf}{\leftkol}{\rightkol}
\addcontentsline{toc}{subsection}{\textrm\textbf Авторский указатель за 2022 г.}

\vspace*{-24pt}

\noindent
{\tabcolsep=3pt
\begin{tabular}{p{397pt}cc}
&\textbf{Вып.} & \textbf{Стр.}\\[6pt]
\Avtors{Абгарян~К.\,К., Гаврилов~Е.\,С.} Программный комплекс для 
многомасштабного модели-\linebreak
\\[-12pt]
\hspace*{23pt}рования структурных свойств композиционных 
материалов&1&88--97\\
\Avtors{Аблаев~Ф.\,М.} см.\ Андрианов~С.\,Н.&&\\
\Avtors{Агаларов Я.\,М.} Оптимальное управление подключением резервного прибора 
в~системе\linebreak
\\[-12pt]
\hspace*{23pt}массового обслуживания $G/M/1$&4&34--41\\
\Avtors{Агаларов~Я.\,М.} Оптимизация порогового управления переключением 
скорости обслу-\linebreak
\\[-12pt]
\hspace*{23pt}живания в~системе массового обслуживания $G/M/1$&1&73--81\\
\Avtors{Агасандян~Г.\,А.} Многомерные бинарные рынки и~CC-VaR&2&\hphantom{1}2--10\\
\Avtors{Алию~Б., Мачнев~Е.\,А., Мокров~Е.\,В.} Гистерезисное управление нагрузкой 
в~беспроводных\linebreak
\\[-12pt]
\hspace*{23pt}сенсорных сетях&3&83--89\\
\Avtors{Андрианов~С.\,Н., Андрианова~Н.\,С., Аблаев~Ф.\,М., Кочнева~Ю.\,Ю.} 
Контекстный поиск\linebreak
\\[-12pt]
\hspace*{23pt}на фотонах с~использованием тестов Белла&1&20--24\\
\Avtors{Андрианова~Н.\,С.} см.\ Андрианов~С.\,Н.&&\\
\Avtors{Базилевский М.\,П.} Обобщение метода выпрямления искаженных из-за 
мультиколлинеарности коэффициентов для~регрессионных моделей с~различной 
степенью\linebreak
\\[-12pt]
\hspace*{23pt}корреляции объясняющих переменных&4&20--25\\
\Avtors{Бесчастный~В.\,А., Острикова~Д.\,Ю., Шоргин~С.\,Я., Молчанов~Д.\,А., 
Гайдамака~Ю.\,В.} Анализ плотности базовых станций 5G NR для предоставления услуг 
виртуальной\linebreak
\\[-12pt]
\hspace*{23pt}и~дополненной реальности&2&102--108\\
\Avtors{Бесчастный~В.\,А. } см.\ Мачнев Е.\,А.&&\\
\Avtors{Битюков~Ю.\,И.} см.\ Босов~А.\,В.&&\\
\Avtors{Борисов А.\,В.} Общий порядок аппроксимации оценок фильтрации состояний 
марков-\linebreak
\\[-12pt]
\hspace*{23pt}ских скачкообразных процессов по~дискретизованным наблюдениям&4&8--13\\
\Avtors{Босов~А.\,В.} Применение самоорганизующихся нейронных сетей к~процессу 
формиро-\linebreak
\\[-12pt]
\hspace*{23pt}вания индивидуальной траектории обучения&3&\hphantom{1}7--15\\
\Avtors{Босов~А.\,В.} Управление линейным выходом марковской цепи по квадратичному 
крите-\linebreak
\\[-12pt]
\hspace*{23pt}рию. Случай полной информации&2&19--26\\
\Avtors{Босов~А.\,В., Битюков~Ю.\,И., Денискина~Г.\,Ю.} О~поиске оптимальной 
схемы 3D-печати\linebreak
\\[-12pt]
\hspace*{23pt}конструкций из композиционных материалов&1&10--19\\
\Avtors{Босов А.\,В., Иванов А.\,В.} Технология классификации типов контента 
электронного\linebreak
\\[-12pt]
\hspace*{23pt}учебника&4&63--72\\
\Avtors{Брюхов Д.\,О., Ступников~С.\,А.} Логическая реляционная модель структур 
данных для\linebreak
\\[-12pt]
\hspace*{23pt}решения задач в~предметной области управления 
землепользованием&4&93--98\\
\Avtors{Бурцева~С.\,А.} см.\ Власкина~А.\,С.&&\\
\Avtors{Васильев~Н.\,С.} О~достаточных условиях экстремума в~многомерных 
вариационных\linebreak
\\[-12pt]
\hspace*{23pt}задачах&3&39--44\\
\Avtors{Власкина~А.\,С., Бурцева~С.\,А., Кочеткова~И.\,А., Шоргин~С.\,Я.} Управляемая 
система массового обслуживания с~эластичным трафиком и~сигналами для анализа 
нарезки\linebreak
\\[-12pt]
\hspace*{23pt}ресурсов в~сети радиодоступа&3&90--96\\
\Avtors{Гаврилов~Е.\,С.} см.\ Абгарян~К.\,К.&&\\
\Avtors{Гайдамака~Ю.\,В.} см.\ Бесчастный~В.\,А.&&\\
\Avtors{Гайдамака~Ю.\,В.} см.\ Мачнев Е.\,А.&&\\
\Avtors{Горшенин~А.\,К., Гусейнова~Е.\,И.} Повышение доходности торговли на~FOREX 
с~помощью\linebreak
\\[-12pt]
\hspace*{23pt}LSTM-идентификации свечных паттернов и~индикатора тиковых 
объемов&3&26--38\\
\Avtors{Григорьев~О.\,Г.} см.\ Смирнов~И.\,В.&&\\
\end{tabular}
}

\pagebreak

\def\leftkol{АВТОРСКИЙ УКАЗАТЕЛЬ ЗА 2022 г.} % ENGLISH ABSTRACTS}

\def\rightkol{АВТОРСКИЙ УКАЗАТЕЛЬ ЗА 2022 г.} %ENGLISH ABSTRACTS}

%\thispagestyle{myheadings}
\def\leftfootline{\small{\textbf{\thepage}
\hfill ИНФОРМАТИКА И ЕЁ ПРИМЕНЕНИЯ\ \ \ том~16\ \ \ выпуск~4\ \ \ 2022}
}%
 \def\rightfootline{\small{ИНФОРМАТИКА И ЕЁ ПРИМЕНЕНИЯ\ \ \ том~16\ \ \ выпуск~4\ \ \ 2022
 \hfill \textbf{\thepage}}}


\noindent
{\tabcolsep=3pt
\begin{tabular}{p{394pt}cc}
&\textbf{Вып.} & \textbf{Стр.}\\[3pt]
\Avtors{Грушо~А.\,А., Грушо~Н.\,А., Забежайло~М.\,И., Зацаринный~А.\,А., 
Тимонина~Е.\,Е., Шор-}\linebreak
\\[-12pt]
\hspace*{23pt}\textbf{гин~С.\,Я.} Анализ цепочек причинно-следственных связей&2&68--74\\
\Avtors{Грушо А.\,А., Грушо Н.\,А., Забежайло~М.\,И., Смирнов~Д.\,В., Тимонина~Е.\,Е., 
Шоргин~С.\,Я.}\linebreak
\\[-12pt]
\hspace*{23pt}О~безопасной архитектуре вычислительной системы на основе 
микросервисов&4&87--92\\
\Avtors{Грушо~А.\,А., Грушо~Н.\,А., Тимонина~Е.\,Е.} Метаданные в~защищенном 
электронном\linebreak
\\[-12pt]
\hspace*{23pt}документообороте&3&\hphantom{1}97--102\\
\Avtors{Грушо~Н.\,А.} см.\ Грушо~А.\,А.&&\\
\Avtors{Грушо Н.\,А.} см.\ Грушо А.\,А.&&\\
\Avtors{Грушо~Н.\,А.} см.\ Грушо~А.\,А.&&\\
\Avtors{Гусейнова~Е.\,И.} см.\ Горшенин~А.\,К.&&\\
\Avtors{Денискина~Г.\,Ю.} см.\ Босов~А.\,В.&&\\
\Avtors{Драгунов~Н.\,А., Дюкова~Е.\,В.} О~поиске максимальных частых 
и~минимальных нечастых\linebreak
\\[-12pt]
\hspace*{23pt}наборов произведения частичных порядков&1&82--87\\
\Avtors{Дубанов~А.\,А., Нефедова~В.\,А.} Кинематические модели задач преследования 
на~плос-\linebreak
\\[-12pt]
\hspace*{23pt}кости методами параллельного сближения и~погони&3&103--109\\
\Avtors{Дунсяо Гу} см.\ Зацман И.\,М.&&\\
\Avtors{Дурново~А.\,А., Инькова~О.\,Ю., Попкова~Н.\,А.} Принципы описания 
показателей логико-\linebreak
\\[-12pt]
\hspace*{23pt}семантических отношений и~их иерархии&2&52--59\\
\Avtors{Дьяченко~Ю.\,Г.} см.\ Соколов И.\,А.&&\\
\Avtors{Дюкова А.\,П.} см.\ Дюкова Е.\,В.&&\\
\Avtors{Дюкова Е.\,В., Дюкова А.\,П.} О~сложности обучения логических процедур 
классификации&4&57--62\\
\Avtors{Дюкова~Е.\,В.} см.\ Драгунов~Н.\,А.&&\\
\Avtors{Забежайло~М.\,И.} см.\ Грушо А.\,А.&&\\
\Avtors{Забежайло~М.\,И.} см.\ Грушо~А.\,А.&&\\
\Avtors{Зацаринный~А.\,А.} см.\ Грушо~А.\,А.&&\\
\Avtors{Зацман И.\,М.} О~научной парадигме информатики: верхний уровень 
классификации\linebreak
\\[-12pt]
\hspace*{23pt}объектов ее предметной области&4&73--79\\
\Avtors{Зацман~И.\,М.} Средовые модели информационных технологий: теоретические 
основа-\linebreak
\\[-12pt]
\hspace*{23pt}ния построения&3&59--67\\
\Avtors{Зацман~И.\,М., Золотарев~О.\,В., Хакимова~А.\,Х.} Средовые модели извлечения 
из текста\linebreak
\\[-12pt]
\hspace*{23pt}новых терминов и~индикаторов настроений&2&60--67\\
\Avtors{Зацман И.\,М., Золотарев~О.\,В., Хакимова~А.\,Х., Дунсяо~Гу.} Модель 
и~технология\linebreak
\\[-12pt]
\hspace*{23pt}извлечения новых терминов из~медицинских текстов&4&80--86\\
\Avtors{Зейфман~А.\,И.} см.\ Ковалёв~И.\,А.&&\\
\Avtors{Зейфман~А.\,И.} см.\ Сатин~Я.\,А.&&\\
\Avtors{Золотарев~О.\,В.} см.\ Зацман И.\,М.&&\\
\Avtors{Золотарев~О.\,В.} см.\ Зацман~И.\,М.&&\\
\Avtors{Иванов А.\,В.} см.\ Босов А.\,В.&&\\
\Avtors{Инькова~О.\,Ю.} см.\ Дурново~А.\,А.&&\\
\Avtors{Кириков~И.\,А.} см.\ Листопад~С.\,В.&&\\
\Avtors{Кириков~И.\,А.} см.\ Румовская~С.\,Б.&&\\
\Avtors{Киселёв~Г.\,А.} см.\ Смирнов~И.\,В.&&\\
\Avtors{Ковалёв~И.\,А., Сатин~Я.\,А., Синицина~А.\,В., Зейфман~А.\,И.} Об одном 
подходе к~оцениванию скорости сходимости нестационарных марковских моделей систем 
обслужи-\linebreak
\\[-12pt]
\hspace*{23pt}вания&3&75--82\\
\Avtors{Ковалёв~С.\,П.} Алгебраическая спецификация графовых вычислительных 
структур&1&2--9\\
\Avtors{Коновалов~М.\,Г., Разумчик~Р.\,В.} Синтез управления двумерным случайным 
блужданием\linebreak
\\[-12pt]
\hspace*{23pt}с~эталонным стационарным распределением&2&109--117\\
\Avtors{Кочеткова~И.\,А.} см.\ Власкина~А.\,С.&&\\
\Avtors{Кочнева~Ю.\,Ю.} см.\ Андрианов~С.\,Н.&&\\
\Avtors{Кравцова~О.\,А.} Использование критериев стационарности для настройки 
моделей при\linebreak
\\[-12pt]
\hspace*{23pt}прогнозировании временных рядов&2&11--18\\
\Avtors{Кривенко~М.\,П.} Выбор модели при факторизации матрицы данных 
с~пропусками&3&52--58\\
\Avtors{Крюкова~А.\,Л.} см.\ Сатин~Я.\,А.&&\\
\end{tabular}
}

\pagebreak

\def\leftkol{АВТОРСКИЙ УКАЗАТЕЛЬ ЗА 2022 г.} % ENGLISH ABSTRACTS}

\def\rightkol{АВТОРСКИЙ УКАЗАТЕЛЬ ЗА 2022 г.} %ENGLISH ABSTRACTS}

%\thispagestyle{myheadings}
\def\leftfootline{\small{\textbf{\thepage}
\hfill ИНФОРМАТИКА И ЕЁ ПРИМЕНЕНИЯ\ \ \ том~16\ \ \ выпуск~4\ \ \ 2022}
}%
 \def\rightfootline{\small{ИНФОРМАТИКА И ЕЁ ПРИМЕНЕНИЯ\ \ \ том~16\ \ \ выпуск~4\ \ \ 2022
 \hfill \textbf{\thepage}}}


\noindent
{\tabcolsep=3pt
\begin{tabular}{p{394pt}cc}
&\textbf{Вып.} & \textbf{Стр.}\\[3pt]
\Avtors{Курузов~И.\,А.} см.\ Смирнов~И.\,В.&&\\[0.3pt]
\Avtors{Листопад~С.\,В., Кириков~И.\,А.} Разрешение конфликтов в~гибридных 
интеллектуальных\linebreak
\\[-12pt]
\hspace*{23pt}многоагентных системах&1&54--60\\[0.3pt]
\Avtors{Малашенко~Ю.\,Е.} Метрические оценки угловых точек множества достижимых 
межуз-\linebreak
\\[-12pt]
\hspace*{23pt}ловых потоков многопользовательской сети&1&25--31\\[0.3pt]
\Avtors{Малашенко~Ю.\,Е.} Последовательный анализ и~метрические оценки 
предельных рас-\linebreak
\\[-12pt]
\hspace*{23pt}пределений межузловых потоков в~многопользовательской сети&3&45--51\\[0.3pt]
\Avtors{Мачнев Е.\,А., Бесчастный~В.\,А., Острикова~Д.\,Ю., Гайдамака~Ю.\,В., 
Шоргин~С.\,Я.} Об оптимальном расположении антенн для~V2X-соединений 
в~субтерагерцевом диа-\linebreak
\\[-12pt]
\hspace*{23pt}пазоне&4&42--50\\
\Avtors{Мачнев~Е.\,А.} см.\ Алию~Б.&&\\[0.3pt]
\Avtors{Мигуля~М.\,А.} см.\ Шнурков~П.\,В.&&\\[0.3pt]
\Avtors{Мокров~Е.\,В.} см.\ Алию~Б.&&\\[0.3pt]
\Avtors{Молчанов~Д.\,А.} см.\ Бесчастный~В.\,А.&&\\[0.3pt]
\Avtors{Нефедова~В.\,А.} см.\ Дубанов~А.\,А.&&\\[0.3pt]
\Avtors{Нуриев~В.\,А.} Переводческий анализ текста с~применением информационных 
ресурсов:\linebreak
\\[-12pt]
\hspace*{23pt}редуцирование спектра моделей перевода в~надкорпусных базах 
данных&3&68--74\\[0.3pt]
\Avtors{Острикова~Д.\,Ю.} см.\ Бесчастный~В.\,А.&&\\[0.3pt]
\Avtors{Острикова~Д.\,Ю.} см.\ Мачнев Е.\,А.&&\\[0.3pt]
\Avtors{Ошушкова~В.\,С.} см.\ Сатин~Я.\,А.&&\\[0.3pt]
\Avtors{Палионная~С.\,И., Шестаков~О.\,В.} Использование FDR-метода множественной 
провер-\linebreak
\\[-12pt]
\hspace*{23pt}ки гипотез при обращении линейных однородных операторов&2&44--51\\[0.3pt]
\Avtors{Панов~А.\,И.} см.\ Смирнов~И.\,В.&&\\[0.3pt]
\Avtors{Пешкова И.\,В.} Границы экстремального индекса времени ожидания в~системе 
$M/G/1$\linebreak
\\[-12pt]
\hspace*{23pt}с~распределением времени обслуживания в~виде конечной смеси&4&26--33\\[0.3pt]
\Avtors{Пешкова~И.\,В.} Сравнение экстремальных индексов времен ожидания 
в~системах об-\linebreak
\\[-12pt]
\hspace*{23pt}служивания $M/G/1$&1&61--67\\[0.3pt]
\Avtors{Попкова~Н.\,А.} см.\ Дурново~А.\,А.&&\\[0.3pt]
\Avtors{Разумчик~Р.\,В.} см.\ Коновалов~М.\,Г.&&\\[0.3pt]
\Avtors{Рождественский~Ю.\,В.} см.\ Соколов И.\,А.&&\\[0.3pt]
\Avtors{Румовская~С.\,Б., Кириков~И.\,А.} Метод визуализации снижения интенсивности 
и~разре-\linebreak
\\[-12pt]
\hspace*{23pt}шения конфликтов в~гибридных интеллектуальных многоагентных 
системах&2&\hphantom{1}94--101\\[0.3pt]
\Avtors{Сатин~Я.\,А., Крюкова~А.\,Л., Ошушкова~В.\,С., Зейфман~А.\,И.} 
О~монотонности\linebreak
\\[-12pt]
\hspace*{23pt}некоторых классов марковских цепей&2&27--34\\[0.3pt]
\Avtors{Сатин~Я.\,А.} см.\ Ковалёв~И.\,А.&&\\[0.3pt]
\Avtors{Синицина~А.\,В.} см.\ Ковалёв~И.\,А.&&\\[0.3pt]
\Avtors{Синицын~И.\,Н.} Нормализация систем, стохастически не разрешенных 
относительно\linebreak
\\[-12pt]
\hspace*{23pt}производных&1&32--38\\[0.3pt]
\Avtors{Синицын~И.\,Н.} Совместная фильтрация и~распознавание нормальных 
процессов в~сто-\linebreak
\\[-12pt]
\hspace*{23pt}хастических системах, не разрешенных относительно 
производных&2&85--93\\
\Avtors{Смирнов~Д.\,В.} см.\ Грушо А.\,А.&&\\[0.3pt]
\Avtors{Смирнов~И.\,В., Панов~А.\,И., Чуганская~А.\,А., Суворова~М.\,И., 
Киселёв~Г.\,А., Курузов~И.\,А., Григорьев~О.\,Г.} Персональный когнитивный 
ассистент: планирование поведения\linebreak
\\[-12pt]
\hspace*{23pt}на основе сценариев деятельности&1&46--53\\[0.3pt]
\Avtors{Соколов И.\,А., Степченков~Ю.\,А., Дьяченко~Ю.\,Г., Рождественский~Ю.\,В.} 
Оценка надеж-\linebreak
\\[-12pt]
\hspace*{23pt}ности синхронного и~самосинхронного конвейеров&4&2--7\\[0.3pt]
\Avtors{Степченков~Ю.\,А.} см.\ Соколов И.\,А.&&\\[0.3pt]
\Avtors{Ступников~С.\,А.} см.\ Брюхов Д.\,О.&&\\[0.3pt]
\Avtors{Суворова~М.\,И.} см.\ Смирнов~И.\,В.&&\\[0.3pt]
\Avtors{Сучков А.\,П.} Единая модель государственных данных: сценарии 
развития&4&\hphantom{9}99--105\\[0.3pt]
\Avtors{Тимонина~Е.\,Е.} см.\ Грушо А.\,А.&&\\[0.3pt]
\Avtors{Тимонина~Е.\,Е.} см.\ Грушо~А.\,А.&&\\[0.3pt]
\Avtors{Тимонина~Е.\,Е} см.\ Грушо~А.\,А.&&\\
\end{tabular}
}

\pagebreak

\def\leftkol{АВТОРСКИЙ УКАЗАТЕЛЬ ЗА 2022 г.} % ENGLISH ABSTRACTS}

\def\rightkol{АВТОРСКИЙ УКАЗАТЕЛЬ ЗА 2022 г.} %ENGLISH ABSTRACTS}

%\thispagestyle{myheadings}
\def\leftfootline{\small{\textbf{\thepage}
\hfill ИНФОРМАТИКА И ЕЁ ПРИМЕНЕНИЯ\ \ \ том~16\ \ \ выпуск~4\ \ \ 2022}
}%
 \def\rightfootline{\small{ИНФОРМАТИКА И ЕЁ ПРИМЕНЕНИЯ\ \ \ том~16\ \ \ выпуск~4\ \ \ 2022
 \hfill \textbf{\thepage}}}


\noindent
{\tabcolsep=3pt
\begin{tabular}{p{394pt}cc}
&\textbf{Вып.} & \textbf{Стр.}\\[3pt]
\Avtors{Торшин~И.\,Ю.} О~применении топологического подхода к анализу плохо 
формализуемых задач для построения алгоритмов виртуального скрининга кван\-то\-во-ме\-ха\-ни\-че\-ских\linebreak
\\[-12pt]
\hspace*{23pt}свойств органических молекул I:~Основы проблемно ориентированной 
теории&1&39--45\\
\Avtors{Торшин~И.\,Ю.} О~применении топологического подхода к~анализу плохо 
формализуемых задач для построения алгоритмов виртуального скрининга кван\-то\-во-ме\-ха\-ни\-че\-ских 
свойств органических молекул II:~Сопоставление формализма 
с~конструктами\linebreak
\\[-12pt]
\hspace*{23pt}квантовой механики и экспериментальная апробация предложенных 
алгоритмов&2&35--43\\
\Avtors{Хакимова~А.\,Х.} см.\ Зацман И.\,М.&&\\
\Avtors{Хакимова~А.\,Х.} см.\ Зацман~И.\,М.&&\\
\Avtors{Хацкевич В.\,Л.} Нечеткие усредняющие операторы в~задаче агрегирования 
нечеткой\linebreak
\\[-12pt]
\hspace*{23pt}информации&4&51--56\\
\Avtors{Чуганская~А.\,А.} см.\ Смирнов~И.\,В.&&\\
\Avtors{Шведов~А.\,С.} Критерий непустоты эпсилон-ядер для нечетких игр с~нетрансферабель-\linebreak
\\[-12pt]
\hspace*{23pt}ной полезностью и~вычислительные процедуры&3&2--6\\
\Avtors{Шестаков О.\,В.} Несмещенная оценка риска пороговой обработки с~двумя 
пороговыми\linebreak
\\[-12pt]
\hspace*{23pt}значениями&4&14--19\\
\Avtors{Шестаков~О.\,В.} см.\ Палионная~С.\,И.&&\\
\Avtors{Шихиев~Ф.\,Ш.} см.\ Шихиев~Ш.\,Б.&&\\
\Avtors{Шихиев~Ш.\,Б., Шихиев~Ф.\,Ш.} Упрощенный язык зрительных 
образов&1&68--72\\
\Avtors{Шнурков~П.\,В.} Об аналитической структуре некоторых видов целевых 
функционалов,\linebreak
\\[-12pt]
\hspace*{23pt}связанных с~задачами управления полумарковскими случайными 
процессами&2&75--84\\
\Avtors{Шнурков~П.\,В., Мигуля~М.\,А.} Некоторые результаты анализа процесса 
изменения цены\linebreak
\\[-12pt]
\hspace*{23pt}бивалютной корзины на основе методов статистики случайных 
процессов&3&16--25\\
\Avtors{Шоргин~С.\,Я.} см.\ Бесчастный~В.\,А.&&\\
\Avtors{Шоргин~С.\,Я.} см.\ Власкина~А.\,С.&&\\
\Avtors{Шоргин~С.\,Я.} см.\ Грушо А.\,А.&&\\
\Avtors{Шоргин~С.\,Я.} см.\ Грушо~А.\,А.&&\\
\Avtors{Шоргин~С.\,Я.} см.\ Мачнев Е.\,А.&&\\
\end{tabular}
}

%\thispagestyle{myheadings}
\def\leftfootline{\small{\textbf{\thepage}
\hfill ИНФОРМАТИКА И ЕЁ ПРИМЕНЕНИЯ\ \ \ том~16\ \ \ выпуск~4\ \ \ 2022}
}%
 \def\rightfootline{\small{ИНФОРМАТИКА И ЕЁ ПРИМЕНЕНИЯ\ \ \ том~16\ \ \ выпуск~4\ \ \ 2022
 \hfill \textbf{\thepage}}}

 \label{end\stat}

\newpage

\def\stat{cont-e}
{%\hrule\par
%\vskip 7pt % 7pt
\raggedleft\Large \bf%\baselineskip=3.2ex
2\,0\,2\,2\ \ A\,U\,T\,H\,O\,R\ \ I\,N\,D\,E\,X \vskip 17pt
 \hrule
 \par
\vskip 21pt plus 6pt minus 3pt }

\label{st\stat}

\def\tit{\ }

\def\aut{\ }
\def\auf{\ }

\def\leftkol{\ } %2021 AUTHOR INDEX} % ENGLISH ABSTRACTS}

\def\rightkol{\ } %2021 AUTHOR INDEX} %ENGLISH ABSTRACTS}

\titele{\tit}{\aut}{\auf}{\leftkol}{\rightkol}
\addcontentsline{toc}{subsection}{\textrm\textbf 2022 Author Index}

\def\leftfootline{\small{\textbf{\thepage}
\hfill INFORMATIKA I EE PRIMENENIYA~--- INFORMATICS AND APPLICATIONS\ \ \ 2022\
\ \ volume~16\ \ \ issue\ 4}
}%
 \def\rightfootline{\small{INFORMATIKA I EE PRIMENENIYA~--- INFORMATICS AND APPLICATIONS\ \ \ 2022\ \ \ volume~16\ \ \ issue\ 4
\hfill \textbf{\thepage}}}

\vspace*{-24pt}

\noindent
{\tabcolsep=3pt
\begin{tabular}{p{395.89pt}cc}
&\textbf{Issue} & \textbf{Page}\\[6pt]
\Avtors{Abgaryan~K.\,K.\ and Gavrilov~E.\,S.} Software package for multiscale modeling of 
structural\linebreak
\\[-12pt]
\hspace*{23pt}properties of composite materials&1&88--97\\
\Avtors{Ablaev~F.\,M.} see Andrianov~S.\,N.&&\\
\Avtors{Agalarov Ya.\,M.} Optimal control of~a~queue-length dependent additional server 
in~$\mathrm{GI}/M/1$\linebreak
\\[-12pt]
\hspace*{23pt}queue&4&34--41\\
\Avtors{Agalarov~Ya.\,M.} Optimization of the threshold service speed control in the $G/M/1$ 
queue&1&73--81\\
\Avtors{Agasandyan~G.\,A.} Multidimensional binary markets and CC-VaR&2&\hphantom{1}2--10\\
\Avtors{Aliyu~B., Machnev~E.\,A., and Mokrov~E.\,V.} Hysteretic congestion control in 
wireless cloud\linebreak
\\[-12pt]
\hspace*{23pt}sensor networks&3&83--89\\
\Avtors{Andrianov~S.\,N., Andrianova~N.\,S., Ablaev~F.\,M., and Kochneva~Yu.\,Yu.} 
Context query on\linebreak
\\[-12pt]
\hspace*{23pt}photons with the use of Bell tests&1&20--24\\
\Avtors{Andrianova~N.\,S.} see Andrianov~S.\,N.&&\\
\Avtors{Bazilevskiy M.\,P.} Generalization of~a~method for~straightening coefficients 
distorted due~to~mul-\linebreak
\\[-12pt]
\hspace*{23pt}ticollinearity in~regression models with different degrees of~explanatory 
variables correlation&4&20--25\\
\Avtors{Beschastnyi~V.\,A., Ostrikova~D.\,Yu., Shorgin~S.\,Ya., Moltchanov~D.\,A., and 
Gaidamaka~Yu.\,V.}\linebreak
\\[-12pt]
\hspace*{23pt}Density analysis of mmWave NR deployments for delivering scalable 
AR/VR video services&2&102--108\\
\Avtors{Beschastnyi~V.\,A.} see Machnev E.\,A.&&\\
\Avtors{Bityukov~Yu.\,I.} see Bosov~A.\,V.&&\\
\Avtors{Borisov A.\,V.} Total approximation order for~Markov jump process filtering given 
discretized\linebreak
\\[-12pt]
\hspace*{23pt}observations&4&8--13\\
\Avtors{Bosov~A.\,V.} Application of self-organizing neural networks to the process of forming 
an individual\linebreak
\\[-12pt]
\hspace*{23pt}learning path&3&\hphantom{1}7--15\\
\Avtors{Bosov~A.\,V.} Linear output control of Markov chain by square criterion. Complete 
information\linebreak
\\[-12pt]
\hspace*{23pt}case&2&19--26\\
\Avtors{Bosov~A.\,V., Bityukov~Yu.\,I., and Deniskina~G.\,Yu.} About searching for the 
optimal 3D printing\linebreak
\\[-12pt]
\hspace*{23pt}scheme of structures from composite materials&1&10--19\\
\Avtors{Bosov A.\,V. and Ivanov~A.\,V.} Technology for~classification of~content types of~e-textbooks&4&63--72\\
\Avtors{Briukhov D.\,O. and Stupnikov~S.\,A.} Logical relational model of~data structures 
for~problem\linebreak
\\[-12pt]
\hspace*{23pt}solving in~land use management&4&93--98\\
\Avtors{Burtseva~S.\,A.} see Vlaskina~A.\,S.&&\\
\Avtors{Chuganskaya~A.\,A.} see Smirnov~I.\,V&&\\
\Avtors{Deniskina~G.\,Yu.} see Bosov~A.\,V.&&\\
\Avtors{Diachenko~Yu.\,G.} see Sokolov I.\,A.&&\\
\Avtors{Djukova~A.\,P.} see Djukova E.\,V.&&\\
\Avtors{Djukova E.\,V. and Djukova~A.\,P.} On the~complexity of~logical classification 
learning procedures&4&57--62\\
\Avtors{Djukova~E.\,V.} see Dragunov~N.\,A.&&\\
\Avtors{Dongxiao~Gu} see Zatsman I.\,M.&&\\
\Avtors{Dragunov~N.\,A.\ and Djukova~E.\,V.} Finding maximal frequent and minimal 
infrequent sets\linebreak
\\[-12pt]
\hspace*{23pt}in partially ordered data&1&82--87\\
\Avtors{Dubanov~A.\,A.\ and Nefedova~V.\,A.} Kinematic models of pursuit problems on the 
plane\linebreak
\\[-12pt]
\hspace*{23pt}by the methods of parallel approach and pursuit&3&103--109\\
\Avtors{Durnovo~A.\,A., Inkova~O.\,Yu., and Popkova~N.\,A.} Principles of describing 
markers of logical-\linebreak
\\[-12pt]
\hspace*{23pt}semantic relations and their hierarchy&2&52--59\\
\Avtors{Gaidamaka~Yu.\,V.} see Beschastnyi~V.\,A.&&\\
\Avtors{Gaidamaka~Yu.\,V.} see Machnev E.\,A.&&\\
\Avtors{Gavrilov~E.\,S.} see Abgaryan~K.\,K.&&\\

\end{tabular}
}
\pagebreak

\def\leftfootline{\small{\textbf{\thepage}
\hfill INFORMATIKA I EE PRIMENENIYA~--- INFORMATICS AND APPLICATIONS\ \ \ 2022\
\ \ volume~16\ \ \ issue\ 4}
}%
 \def\rightfootline{\small{INFORMATIKA I EE PRIMENENIYA~---
INFORMATICS AND APPLICATIONS\ \ \ 2022\ \ \ volume~16\ \ \ issue\ 4
\hfill \textbf{\thepage}}}

\def\leftkol{2022 AUTHOR INDEX} % ENGLISH ABSTRACTS}

\def\rightkol{2022 AUTHOR INDEX} %ENGLISH ABSTRACTS}


\noindent
{\tabcolsep=3pt
\begin{tabular}{p{395.5pt}cc}
&\textbf{Issue} & \textbf{Page}\\[6pt]
\Avtors{Gorshenin~A.\,K.\ and Guseynova~E.\,I.} Increasing FOREX trading profitability with 
LSTM\linebreak
\\[-12pt]
\hspace*{23pt}candlestick pattern recognition and tick volume indicator&3&26--38\\
\Avtors{Grigoriev~O.\,G.} see Smirnov~I.\,V&&\\[-0.1pt]
\Avtors{Grusho~A.\,A., Grusho~N.\,A., and Timonina~E.\,E.} Metadata in secure electronic 
document\linebreak
\\[-12pt]
\hspace*{23pt}management&3&\hphantom{1}97--102\\[-0.1pt]
\Avtors{Grusho A.\,A., Grusho~N.\,A., Zabezhailo~M.\,I., Smirnov~D.\,V., Timonina~E.\,E., 
and Shorgin~S.\,Ya.}\linebreak
\\[-12pt]
\hspace*{23pt}About the~secure architecture of~a~microservice-based computing 
system&4&87--92\\[-0.1pt]
\Avtors{Grusho~A.\,A., Grusho~N.\,A., Zabezhailo~M.\,I., Zatsarinny~A.\,A., 
Timonina~E.\,E.,}\linebreak
\\[-12pt]
\hspace*{23pt}\textbf{and Shorgin~S.\,Ya.} Cause-and-effect chain analysis&2&68--74\\
\Avtors{Grusho~N.\,A.} see Grusho A.\,A.&&\\[-0.1pt]
\Avtors{Grusho~N.\,A.} see Grusho~A.\,A.&&\\[-0.1pt]
\Avtors{Grusho~N.\,A.} see Grusho~A.\,A.&&\\[-0.1pt]
\Avtors{Guseynova~E.\,I.} see Gorshenin~A.\,K.&&\\
\Avtors{Inkova~O.\,Yu.} see Durnovo~A.\,A.&&\\[-0.1pt]
\Avtors{Ivanov~A.\,V.} see Bosov A.\,V.&&\\[-0.1pt]
\Avtors{Khakimova~A.\,K.} see Zatsman I.\,M.&&\\[-0.1pt]
\Avtors{Khakimova~A.\,K.} see Zatsman~I.\,M.&&\\[-0.1pt]
\Avtors{Khatskevich V.\,L.} Fuzzy averaging operators in~the~problem of~aggregating fuzzy 
information&4&51--56\\[-0.1pt]
\Avtors{Kirikov~I.\,A.} see Listopad~S.\,V.&&\\[-0.1pt]
\Avtors{Kirikov~I.\,A.} see Rumovskaya~S.\,B.&&\\[-0.1pt]
\Avtors{Kiselev~G.\,A.} see Smirnov~I.\,V&&\\[-0.1pt]
\Avtors{Kochetkova~I.\,A.} see Vlaskina~A.\,S.&&\\[-0.1pt]
\Avtors{Kochneva~Yu.\,Yu.} see Andrianov~S.\,N.&&\\[-0.1pt]
\Avtors{Konovalov~M.\,G.\ and Razumchik~R.\,V.} Controlling a bounded two-dimensional 
Markov chain\linebreak
\\[-12pt]
\hspace*{23pt}with a~given invariant measure&2&109--117\\[-0.1pt]
\Avtors{Kovalev~I.\,A., Satin~Y.\,A., Sinitcina~A.\,V., and Zeifman~A.\,I.} On an approach 
for estimating\linebreak
\\[-12pt]
\hspace*{23pt}the rate of convergence for nonstationary Markov models of queueing 
systems&3&75--82\\[-0.1pt]
\Avtors{Kovalyov~S.\,P.} Algebraic specification of graph computational structures&1&2--9\\
\Avtors{Kravtsova~O.\,A.} Model setting using stationarity criteria for time series 
forecasting&2&11--18\\[-0.1pt]
\Avtors{Krivenko~M.\,P.} Model selection for matrix factorization with missing 
components&3&52--58\\[-0.1pt]
\Avtors{Kryukova~A.\,L.} see Satin~Y.\,A.&&\\[-0.1pt]
\Avtors{Kuruzov~I.\,A.} see Smirnov~I.\,V&&\\[-0.1pt]
\Avtors{Listopad~S.\,V.\ and Kirikov~I.\,A.} Conflict resolution in hybrid intelligent multiagent 
systems&1&54--60\\[-0.1pt]
\Avtors{Machnev E.\,A., Beschastnyi~V.\,A., Ostrikova~D.\,Yu., Gaidamaka~Yu.\,V., and 
Shorgin~S.\,Ya.} On\linebreak
\\[-12pt]
\hspace*{23pt}the optimal antenna deployment for~subterahertz V2X 
communications&4&42--50\\[-0.1pt]
\Avtors{Machnev~E.\,A.} see Aliyu~B.&&\\[-0.1pt]
\Avtors{Malashenko~Yu.\,E.} Metric evaluations of the angular points of the set of attainable 
internodal\linebreak
\\[-12pt]
\hspace*{23pt}flows of multiuser network&1&25--31\\[-0.1pt]
\Avtors{Malashenko~Yu.\,E.} Sequential analysis and metric estimates of peak load flows in 
the multiuser\linebreak
\\[-12pt]
\hspace*{23pt}network&3&45--51\\[-0.1pt]
\Avtors{Migulya~M.\,A.} see Shnurkov~P.\,V.&&\\[-0.1pt]
\Avtors{Mokrov~E.\,V.} see Aliyu~B.&&\\[-0.1pt]
\Avtors{Moltchanov~D.\,A.} see Beschastnyi~V.\,A.&&\\[-0.1pt]
\Avtors{Nefedova~V.\,A.} see Dubanov~A.\,A.&&\\[-0.1pt]
\Avtors{Nuriev~V.\,A.} Computer-assisted textual analysis in translation: Reducing the 
spectrum of\linebreak
\\[-12pt]
\hspace*{23pt}translation models in supracorpora databases&3&68--74\\[-0.1pt]
\Avtors{Oshushkova~V.\,S.} see Satin~Y.\,A.&&\\[-0.1pt]
\Avtors{Ostrikova~D.\,Yu.} see Beschastnyi~V.\,A.&&\\[-0.1pt]
\Avtors{Ostrikova~D.\,Yu.} see Machnev E.\,A.&&\\[-0.1pt]
\Avtors{Palionnaya~S.\,I.\ and Shestakov~O.\,V.} The use of the FDR method of multiple 
hypothesis testing\linebreak
\\[-12pt]
\hspace*{23pt}when inverting linear homogeneous operators&2&44--51\\[-0.1pt]
\Avtors{Panov~A.\,I.} see Smirnov~I.\,V&&\\[-0.1pt]
\Avtors{Peshkova I.\,V.} On bounds of~the~stationary waiting time extremal index 
in~$M/G/1$ system\linebreak
\\[-12pt]
\hspace*{23pt}with mixture service times&4&26--33\\[-0.1pt]
\end{tabular}
}
\pagebreak

\def\leftfootline{\small{\textbf{\thepage}
\hfill INFORMATIKA I EE PRIMENENIYA~--- INFORMATICS AND APPLICATIONS\ \ \ 2022\
\ \ volume~16\ \ \ issue\ 4}
}%
 \def\rightfootline{\small{INFORMATIKA I EE PRIMENENIYA~---
INFORMATICS AND APPLICATIONS\ \ \ 2022\ \ \ volume~16\ \ \ issue\ 4
\hfill \textbf{\thepage}}}

\def\leftkol{2022 AUTHOR INDEX} % ENGLISH ABSTRACTS}

\def\rightkol{2022 AUTHOR INDEX} %ENGLISH ABSTRACTS}


\noindent
{\tabcolsep=3pt
\begin{tabular}{p{395.5pt}cc}
&\textbf{Issue} & \textbf{Page}\\[6pt]
\Avtors{Peshkova~I.\,V.} The comparison of waiting time extremal indexes in $M/G/1$ 
queueing systems&1&61--67\\[-0.1pt]
\Avtors{Popkova~N.\,A.} see Durnovo~A.\,A.&&\\[-0.1pt]
\Avtors{Razumchik~R.\,V.} see Konovalov~M.\,G.&&\\[-0.1pt]
\Avtors{Rogdestvenski~Yu.\,V.} see Sokolov I.\,A.&&\\[-0.1pt]
\Avtors{Rumovskaya~S.\,B.\ and Kirikov~I.\,A.} Visual representation of the decrease in 
conflict intensity\linebreak
\\[-12pt]
\hspace*{23pt}and its resolution in hybrid intelligent multiagent 
systems&2&\hphantom{1}94--101\\[-0.1pt]
\Avtors{Satin~Y.\,A., Kryukova~A.\,L., Oshushkova~V.\,S., and Zeifman~A.\,I.} On 
monotonicity of some\linebreak
\\[-12pt]
\hspace*{23pt}classes of Markov chains&2&27--34\\[-0.1pt]
\Avtors{Satin~Y.\,A.} see Kovalev~I.\,A.&&\\[-0.1pt]
\Avtors{Shestakov O.\,V.} Unbiased thresholding risk estimate with two threshold 
values&4&14--19\\[-0.1pt]
\Avtors{Shestakov~O.\,V.} see Palionnaya~S.\,I.&&\\[-0.1pt]
\Avtors{Shihiev~F.\,Sh.} see Shihiev~Sh.\,B.&&\\[-0.1pt]
\Avtors{Shihiev~Sh.\,B.\ and Shihiev~F.\,Sh.} Simplified language for visual images&1&68--72\\[-0.1pt]
\Avtors{Shnurkov~P.\,V.} On the analytical structure of some kinds of target functionals 
associated with\linebreak
\\[-12pt]
\hspace*{23pt}the control problems of semi-Markov stoсhastic processes&2&75--84\\[-0.1pt]
\Avtors{Shnurkov~P.\,V.\ and Migulya~M.\,A.} Some results of the analysis of the process of 
changing\linebreak
\\[-12pt]
\hspace*{23pt}the price of a dual currency basket based on random process statistics 
methods&3&16--25\\[-0.1pt]
\Avtors{Shorgin~S.\,Ya.} see Beschastnyi~V.\,A.&&\\[-0.1pt]
\Avtors{Shorgin~S.\,Ya.} see Grusho A.\,A.&&\\[-0.1pt]
\Avtors{Shorgin~S.\,Ya.} see Grusho~A.\,A.&&\\[-0.1pt]
\Avtors{Shorgin~S.\,Ya.} see Machnev E.\,A.&&\\[-0.1pt]
\Avtors{Shorgin~S.\,Ya.} see Vlaskina~A.\,S.&&\\[-0.1pt]
\Avtors{Shvedov~A.\,S.} A~condition for non-emptiness of the epsilon-core of 
a~nontransferable utility\linebreak
\\[-12pt]
\hspace*{23pt}fuzzy game and computational schemes&3&2--6\\[-0.1pt]
\Avtors{Sinitcina~A.\,V.} see Kovalev~I.\,A.&&\\[-0.1pt]
\Avtors{Sinitsyn~I.\,N.} Joint filtration and recognition of normal proсesses in stochastic 
systems with\linebreak
\\[-12pt]
\hspace*{23pt}unsolved derivatives&2&85--93\\[-0.1pt]
\Avtors{Sinitsyn~I.\,N.} Normalization of systems with stochastically unsolved 
derivatives&1&32--38\\[-0.1pt]
\Avtors{Smirnov~D.\,V.} see Grusho A.\,A.&&\\[-0.1pt]
\Avtors{Smirnov~I.\,V., Panov~A.\,I., Chuganskaya~A.\,A., Suvorova~M.\,I., Kiselev~G.\,A., 
Kuruzov~I.\,A., and}\linebreak
\\[-12pt]
\hspace*{23pt}\textbf{Grigoriev~O.\,G.} Personal cognitive assistant: Planning activity with 
scripts&1&46--53\\[-0.1pt]
\Avtors{Sokolov I.\,A., Stepchenkov Yu.\,A., Diachenko~Yu.\,G., 
and~Rogdestvenski~Yu.\,V.} Synchronous and\linebreak
\\[-12pt]
\hspace*{23pt}self-timed pipeline's reliability 
estimation&4&2--7\\[-0.1pt]
\Avtors{Stepchenkov Yu.\,A.} see Sokolov I.\,A.&&\\[-0.1pt]
\Avtors{Stupnikov~S.\,A.} see Briukhov D.\,O.&&\\[-0.1pt]
\Avtors{Suchkov A.\,P.} Unified model of national data: Development scenarios&4&\hphantom{9}99--105\\
\Avtors{Suvorova~M.\,I.} see Smirnov~I.\,V&&\\[-0.1pt]
\Avtors{Timonina~E.\,E.} see Grusho A.\,A.&&\\[-0.1pt]
\Avtors{Timonina~E.\,E.} see Grusho~A.\,A.&&\\[-0.1pt]
\Avtors{Timonina~E.\,E.} see Grusho~A.\,A.&&\\[-0.1pt]
\Avtors{Torshin~I.\,Yu.} On the application of a~topological approach to analysis of poorly 
formalized problems for constructing algorithms for virtual screening of quantum-mechanical 
properties\linebreak
\\[-12pt]
\hspace*{23pt}of organic molecules I:~The basics of the problem-oriented theory&1&39--45\\[-0.1pt]
\Avtors{Torshin~I.\,Yu.} On the application of a topological approach to analysis of poorly 
formalized problems for constructing algorithms for virtual screening of quantum-mechanical 
properties\linebreak
\\[-12pt]
\hspace*{23pt}of organic molecules II:~Comparison of formalism with constructions of quantum mechan-\linebreak
\\[-12pt]
\hspace*{23pt}ics and experimental approbation of the proposed algorithms&2&35--43\\[-0.1pt]
\Avtors{Vasilyev~N.\,S.} On extremum sufficient conditions in multidimensional variation 
calculus\linebreak
\\[-12pt]
\hspace*{23pt}problems&3&39--44\\[-0.1pt]
\Avtors{Vlaskina~A.\,S., Burtseva~S.\,A., Kochetkova~I.\,A., and Shorgin~S.\,Ya.} 
Controllable queuing system\linebreak
\\[-12pt]
\hspace*{23pt}with elastic traffic and signals for analyzing network 
slicing&3&90--96\\[-0.1pt]
\Avtors{Zabezhailo~M.\,I.} see Grusho A.\,A.&&\\[-0.1pt]
\Avtors{Zabezhailo~M.\,I.} see Grusho~A.\,A.&&\\[-0.1pt]
\Avtors{Zatsarinny~A.\,A.} see Grusho~A.\,A.&&\\[-0.1pt]
\end{tabular}
}
\pagebreak

\def\leftfootline{\small{\textbf{\thepage}
\hfill INFORMATIKA I EE PRIMENENIYA~--- INFORMATICS AND APPLICATIONS\ \ \ 2022\
\ \ volume~16\ \ \ issue\ 4}
}%
 \def\rightfootline{\small{INFORMATIKA I EE PRIMENENIYA~---
INFORMATICS AND APPLICATIONS\ \ \ 2022\ \ \ volume~16\ \ \ issue\ 4
\hfill \textbf{\thepage}}}

\def\leftkol{2022 AUTHOR INDEX} % ENGLISH ABSTRACTS}

\def\rightkol{2022 AUTHOR INDEX} %ENGLISH ABSTRACTS}


\noindent
{\tabcolsep=3pt
\begin{tabular}{p{395.5pt}cc}
&\textbf{Issue} & \textbf{Page}\\[6pt]
\Avtors{Zatsman~I.\,M.} Informatics' medium models of information technology: Theoretical 
foundations\linebreak
\\[-12pt]
\hspace*{23pt}for their creating&3&59--67\\
\Avtors{Zatsman I.\,M.} On the~scientific paradigm of~informatics: The~classification high 
level of~its~objects&4&73--79\\
\Avtors{Zatsman~I.\,M., Zolotarev~O.\,V., and Khakimova~A.\,K.} Medium models for 
discovering novel\linebreak
\\[-12pt]
\hspace*{23pt}terms and sentiments from texts&2&60--67\\
\Avtors{Zatsman I.\,M., Zolotarev~O.\,V., Khakimova~A.\,K., and~Dongxiao~Gu.} Model and 
technology\linebreak
\\[-12pt]
\hspace*{23pt}for discovering new terms in medical texts&4&80--86\\
\Avtors{Zeifman~A.\,I.} see Kovalev~I.\,A.&&\\
\Avtors{Zeifman~A.\,I.} see Satin~Y.\,A.&&\\
\Avtors{Zolotarev~O.\,V.} see Zatsman I.\,M.&&\\
\Avtors{Zolotarev~O.\,V.} see Zatsman~I.\,M.&&\\
\end{tabular}
}

%\thispagestyle{myheadings}
\def\leftfootline{\small{\textbf{\thepage}
\hfill INFORMATIKA I EE PRIMENENIYA~--- INFORMATICS AND APPLICATIONS\ \ \ 2022\
\ \ volume~16\ \ \ issue\ 4}
}%
 \def\rightfootline{\small{INFORMATIKA I EE PRIMENENIYA~---
INFORMATICS AND APPLICATIONS\ \ \ 2022\ \ \ volume~16\ \ \ issue\ 4
\hfill \textbf{\thepage}}}

 \label{end\stat}

\newpage

%
   \vspace*{-46pt}

\begin{center}
\vspace*{4pt}
\mbox{%

\epsfxsize=55mm %112.705
\epsfbox{zhur-2.eps}
}
%\end{center}

\vspace*{10pt} 


%   \begin{center}
\fbox{\large\textbf{Академик Юрий Иванович Журавлёв}}\\[10pt]
\textbf{\large 14.01.1935--14.01.2022}
   \end{center}


   %\vspace*{2.5mm}

   \vspace*{5mm}

   \thispagestyle{empty}

%\

%\vspace*{-12pt}
       


В январе этого года ушел из жизни главный научный сотрудник Федерального исследовательского 
центра <<Информатика и управление>> РАН, председатель Редакционного совета журнала 
<<Информатика и~её применения>> академик Юрий Иванович Журавлёв. В~его лице мировая 
наука потеряла одного из своих ярчайших представителей~--- выдающегося ученого-исследователя 
и~талантливого ученого-организатора.

Юрий Иванович родился в Воронеже в 1935~г.\ в семье ученого и врача. Среднее образование 
получил в школе №\,6 г.~Фрунзе (ныне Бишкек) Киргизской ССР. В~1952~г.\ поступил на 
ме\-ха\-ни\-ко-ма\-те\-ма\-ти\-че\-ский факультет МГУ им.\ М.\,В.~Ломоносова. В~1957~г.\ Юрий Иванович 
защищает диплом и продолжает обучение в аспирантуре Московского университета на кафедре 
вычислительной математики (возглавляемой тогда академиком С.\,Л.~Соболевым). После 
успешной защиты кандидатской диссертации (к.ф.-м.н., 1959 г., научный руководитель~--- 
А.\,А.~Ляпунов, оппоненты~--- чл.-корр.\ А.\,А.~Марков, к.ф.-м.н.\ О.\,Б.~Лупанов) и~до 
окончательного переезда в Москву в 1969~г.\ работал в Институте математики Сибирского 
отделения АН СССР, занимая в нем последовательно должности младшего научного сотрудника, 
заведующего отделом, заведующего отделением, заместителя директора по научной работе. 
В~этот период (1954--1966~гг.)\ им был опубликован цикл работ по решению задач алгебры и 
математической логики, причем полученные результаты применялись для создания эффективных 
программ для ЭВМ, конструирования схем и сетей для обработки информации. Наиболее значимый 
результат этого периода научной работы~--- обоснование нового направления исследований, 
общей теории локальных алгоритмов. В~ней были окончательно объединены топологические 
принципы и теория алгоритмов. Эта теория и легла в основу докторской диссертации Юрия 
Ивановича (д.ф.-м.н., 1965~г.)\ по еще тогда новой научной специальности <<Математическая 
кибернетика>>. Оппонировали ему как специалисты по кибернетике~--- академик 
В.\,М.~Глушков, член-корреспондент А.\,А.~Ляпунов и О.\,Б.~Лупанов, так и про\-фес\-сор-ал\-геб\-раист А.\,Д.~Тайманов. 

В 1969~г.\ Юрий Иванович переезжает в Москву и возглавляет в Вычислительном центре АН 
СССР лабораторию проблем распознавания. Впоследствии он~--- заместитель директора по 
научной работе. Научные интересы этого периода связаны с проблемами классификации или 
распознавания образов. В~1976--1978~гг.\ Юрий Иванович публикует цикл работ по ставшему 
вскоре знаменитым алгебраическому подходу к проблеме синтеза корректных алгоритмов. Эти 
работы определили современное состояние всей проблематики распознавания и многих смежных 
областей прикладной математики и информатики. В~своих основополагающих работах Юрий 
Иванович показал, что можно в явном виде строить экстремальные по качеству алгоритмы для 
решения очень широких классов плохо формализованных задач. 
{\looseness=-1

}





Научные заслуги Юрия Ивановича получили широкое признание. В~1966~г.\ он совместно с 
О.\,Б.~Лупановым и чле\-ном-кор\-рес\-пон\-ден\-том АН СССР С.\,В.~Яблонским были удостоены 
звания лауреата Ленинской премии в~об\-ласти науки и техники. В~1984~г.\ Юрий Иванович 
был избран членом-корреспондентом АН СССР (по специальности <<Информатика>>), 
а~в~1992~г.~--- академиком РАН (по той же специальности).\linebreak\vspace*{-12pt}

\pagebreak

\

\vspace*{-46pt}

\noindent
\begin{floatingfigure}{48mm}
\begin{center}
%\vspace*{6pt}
\mbox{%

\epsfxsize=46mm %112.705
\epsfbox{zhur-3.eps}
}
\end{center}
\vspace*{6pt}
\end{floatingfigure}

 \thispagestyle{empty}

\noindent
В~1986~г.\ за цикл прикладных 
работ ему и ряду его учеников была при\-суж\-де\-на премия Совета Министров СССР. Он являлся 
членом иностранных академий наук, председателем секции <<Прикладная математика
 и~информатика>> Отделения математических наук РАН, председателем экспертного совета ВАК 
России по управ\-ле\-нию и информатике, заслуженным профессором нескольких университетов, 
председателем Российской ассоциации <<Распознавание образов и обработка изображений>>, 
членом исполкома Международной ассоциации IAPR (распознавание образов и обработка 
изображений). Был награжден 8-ю орденами и медалями СССР и России.

Юрий Иванович проводил большую научно-литературную работу, являясь, в том числе, главным 
редактором международных научных журналов и членом редколлегий ряда рецензируемых 
научных журналов. 


Параллельно с активной научной деятельностью Юрий Иванович вел и преподавательскую 
работу. С~1961 по~1969~гг.~--- в Новосибирском государственном университете на кафедре 
алгебры и математической логики, которую возглавлял в то время академик А.\,И.~Мальцев. 
С~1970~г., будучи уже профессором (1967~г.),~--- в Московском физико-техническом институте 
на кафедре академика Н.\,Н.~Моисеева. В~1997~г.\ по предложению ректора МГУ им.\ 
М.\,В.~Ломоносова академика В.\,А.~Садовничего Юрий Иванович организовал на факультете 
Вычислительной математики и кибернетики новую кафедру <<Математические методы 
прогнозирования>>, которой и руководил до конца жизни. В~2008~г.\ ему была присуждена 
премия Совета Министров РФ в области образования. С~1965~г.\ Юрий Иванович периодически 
читал курсы лекций за рубежом, в университетах США, Франции, Финляндии, Швеции, Австрии, 
Польши, Болгарии, ГДР и других стран. Эта работа в существенной степени обеспечила широкое 
международное признание советской и российской науки в области дискретной математики и~распознавания образов. 

%\begin{floatingfigure}{60mm}
\begin{figure}[b]
\begin{center}
\vspace*{-6pt}
\mbox{%

\epsfxsize=112mm %90mm %112.705
\epsfbox{zhur-1.eps}
}
\end{center}
\end{figure}
%\end{floatingfigure}

Понимая важность вопроса воспитания подрастающего поколения для развития науки в стране, 
Юрий Иванович вскоре после защиты первой диссертации включился в работу по подготовке 
научных кадров. Им создана большая научная школа: под руководством Юрия Ивановича 
защищены более 100~кандидатских диссертаций по всевозможным разделам естествознания 
(математике, информатике, медицине, технике, экономике, геологии), не один десяток докторов 
наук. Он воспитал академиков и членов-корреспондентов РАН и академий государств СНГ. 
С~большим вниманием и участием Юрий Иванович относился к развитию научных школ страны 
в~об\-ласти обработки изображений, распознавания образов и компьютерной оптики. 

Для всех коллег и учеников Юрия Ивановича он останется примером замечательного человека, 
та\-лант\-ли\-во\-го педагога и выдающегося, преданного служению науке ученого. 


%\def\stat{cont}
{%\hrule\par
%\vskip 7pt % 7pt
\raggedleft\Large \bf%\baselineskip=3.2ex
А\,В\,Т\,О\,Р\,С\,К\,И\,Й\ \ У\,К\,А\,З\,А\,Т\,Е\,Л\,Ь\ \ З\,А\ \ 2\,0\,1\,0 г. \vskip 17pt
    \hrule
    \par
\vskip 21pt plus 6pt minus 3pt }

\label{st\stat}

\def\tit{\ }

\def\aut{\ }
\def\auf{\ }

\def\leftkol{\ } % ENGLISH ABSTRACTS}

\def\rightkol{\ } %АВТОРСКИЙ УКАЗАТЕЛЬ ЗА 2010 г.} %ENGLISH ABSTRACTS}

\titele{\tit}{\aut}{\auf}{\leftkol}{\rightkol}

\vspace*{-12pt}

{\tabcolsep=3pt
\begin{tabular}{p{388pt}rr}
&\textbf{Выпуск} & \textbf{Стр.}\\[6pt]
\hangindent=23pt\noindent\textbf{Арутюнян~А.\,Р.} Моделирование влияния деформаций отпечатков пальцев на 
точность\linebreak
\vspace*{-12pt}\\
\hspace*{23pt}дактилоскопической идентификации$\dotfill$&1&51\\
\hangindent=23pt\noindent\textbf{Архипов~О.\,П., Зыкова~З.\,П.} Интеграция гетерогенной информации о цветных 
пикселях\linebreak
\vspace*{-12pt}\\
\hspace*{23pt}и их цветовосприятии$\dotfill$&4&15\\
\hangindent=23pt\noindent\textbf{Баранов~С.\,И., Френкель~С.\,Л., Захаров~В.\,Н.} Полуформальная верификация 
цифрового устройства с конвейером, основанная на использовании алгоритмических машин\linebreak
\vspace*{-12pt}\\
\hspace*{23pt}состояния$\dotfill$&4&49\\
\textbf{Бекетова~И.\,В.} см.~Каратеев~С.\,Л.&&\\
\textbf{Белоусов~В.\,В.} см.~Синицын~И.\,Н.&&\\
\hangindent=23pt\noindent\textbf{Бенинг~В.\,Е., Королев~Р.\,А.} О предельном поведении мощностей критериев в 
случае\linebreak
\vspace*{-12pt}\\
\hspace*{23pt}распределения Лапласа$\dotfill$&2&63\\
\hangindent=23pt\noindent\textbf{Бенинг~В.\,Е., Сипина~А.\,В.} Асимптотическое разложение для мощности 
критерия,\linebreak
\vspace*{-12pt}\\
\hspace*{23pt}основанного на выборочной медиане, в случае распределения Лапласа$\dotfill$&1&18\\
\textbf{Бондаренко~А.\,В.} см.~Каратеев~С.\,Л.&&\\
\hangindent=23pt\noindent\textbf{Бородина~А.\,В., Морозов~Е.\,В.} Об оценивании асимптотики вероятности 
большого\linebreak
\vspace*{-12pt}\\
\hspace*{23pt}уклонения стационарной регенеративной очереди с одним прибором$\dotfill$&3&29\\
\hangindent=23pt\noindent\textbf{Бунтман~Н.\,В., Минель~Ж.-Л., Ле~Пезан~Д., Зацман~И.\,М.} Типология и 
компьютерное\linebreak
\vspace*{-12pt}\\
\hspace*{23pt}моделирование трудностей перевода$\dotfill$&3&77\\
\textbf{Визильтер~Ю.\,В.} см.~Каратеев~С.\,Л.&&\\
\hangindent=23pt\noindent\textbf{Гавриленко~С.\,В.} Оценки скорости сходимости распределений случайных сумм с 
безгранично делимыми индексами к нормальному закону$\dotfill$&4&81\\
\hangindent=23pt\noindent\textbf{Григорьева~М.\,Е., Шевцова~И.\,Г.} Уточнение неравенства 
Каца--Берри--Эссеена$\dotfill$&2&75\\
\hangindent=23pt\noindent\textbf{Грушо~А.\,А., Грушо~Н.\,А., Тимонина~Е.\,Е.} Поиск конфликтов в политиках 
безопасности: модель случайных графов$\dotfill$&3&38\\
\textbf{Грушо~Н.\,А.} см.~Грушо~А.\,А.&&\\
\hangindent=23pt\noindent\textbf{Гудков~В.\,Ю.} Математические модели изображения отпечатка пальца на основе 
описания линий$\dotfill$&1&58\\
\textbf{Гуртов~А.\,В.} см.~Лукьяненко~А.\,С.&&\\
\textbf{Желтов~С.\,Ю.} см.~Каратеев~С.\,Л.&&\\
\hangindent=23pt\noindent\textbf{Захаров~А.\,А., Серебряков~В.\,А.} Система управления электронной библиотекой 
LibMeta$\dotfill$&4&2\\
\textbf{Захаров~В.\,Н.} см.~Баранов~С.\,И.&&\\
\textbf{Захарова~Т.\,В.} см.~Матвеева~С.\,С.&&\\
\hangindent=23pt\noindent\textbf{Зацаринный~А.\,А., Чупраков~К.\,Г.} Некоторые аспекты выбора технологии для 
постро-\linebreak
\vspace*{-12pt}\\
\hspace*{23pt}ения систем отображения информации ситуационного центра$\dotfill$&3&59\\
\textbf{Зацман~И.\,М.} см.~Бунтман~Н.\,В.&&\\
\hangindent=23pt\noindent\textbf{Зейфман~А.\,И., Коротышева~А.\,В., Сатин~Я.\,А., Шоргин~С.\,Я.} Об 
устойчивости нестаци-\linebreak
\vspace*{-12pt}\\
\hspace*{23pt}онарных систем обслуживания с катастрофами$\dotfill$&3&9\\
\textbf{Зыкова~З.\,П.} см.~Архипов~О.\,П.&&\\
\hangindent=23pt\noindent\textbf{Илюшин~Г.\,Я., Соколов~И.\,А.} Организация управляемого доступа пользователей 
к\linebreak
\vspace*{-12pt}\\
\hspace*{23pt}разнородным ведомственным информационным ресурсам$\dotfill$&1&24\\
\hangindent=23pt\noindent\textbf{Кавагучи~Ю., Ульянов~В.\,В., Фуджикоши~Я.} Приближения для статистик, 
описывающих\linebreak
\vspace*{-12pt}\\
\hspace*{23pt}геометрические свойства данных большой размерности, с оценками 
ошибок$\dotfill$&1&12\\
\hangindent=23pt\noindent\textbf{Каратеев~С.\,Л., Бекетова~И.\,В., Ососков~М.\,В., Князь~В.\,А., 
Визильтер~Ю.\,В., Бондаренко~А.\,В., Желтов~С.\,Ю.} Автоматизированный контроль 
качества цифровых\linebreak
\vspace*{-12pt}\\
\hspace*{23pt}изображений для персональных документов$\dotfill$&1&65\\
\end{tabular}
}

\pagebreak

\def\leftkol{АВТОРСКИЙ УКАЗАТЕЛЬ ЗА 2010 г.} % ENGLISH ABSTRACTS}

\def\rightkol{АВТОРСКИЙ УКАЗАТЕЛЬ ЗА 2010 г.} %ENGLISH ABSTRACTS}

{\tabcolsep=3pt
\begin{tabular}{p{388pt}rr}
&\textbf{Выпуск} & \textbf{Стр.}\\[3pt]
\hangindent=23pt\noindent\textbf{Козеренко~Е.\,Б.} Лингвистические фильтры в статистических моделях машинного\linebreak
\vspace*{-12pt}\\
\hspace*{23pt}перевода$\dotfill$&2&83\\
\hangindent=23pt\noindent\textbf{Козеренко~Е.\,Б., Кузнецов~И.\,П.} Когнитивно-лингвистические представления в 
систе-\linebreak
\vspace*{-12pt}\\
\hspace*{23pt}мах обработки текстов$\dotfill$&3&69\\
\textbf{Князь~В.\,А.} см.~Каратеев~С.\,Л.&&\\
\hangindent=23pt\noindent\textbf{Колесников~А.\,В., Солдатов~С.\,А.} Алгоритм координации для гибридной 
интеллектуальной системы решения сложной задачи оперативно-производственного\linebreak
\vspace*{-12pt}\\
\hspace*{23pt}планирования$\dotfill$&4&61\\
\hangindent=23pt\noindent\textbf{Коновалов~М.\,Г.} О планировании потоков в системах вычислительных 
ресурсов$\dotfill$&2&3\\
\textbf{Конушин~А.\,С.} см.~Конушин~В.\,С.&&\\
\hangindent=23pt\noindent\textbf{Конушин~В.\,С., Кривовязь~Г.\,Р., Конушин~А.\,С.} Алгоритм распознавания людей 
в видео-\linebreak
\vspace*{-12pt}\\
\hspace*{23pt}последовательности по одежде$\dotfill$&1&74\\
\textbf{Корепанов~Э.\, Р.} см.~Синицын~И.\,Н.&&\\
\textbf{Королев~В.\,Ю.} см.~Соколов~И.\,А.&&\\
\textbf{Королев~Р.\,А.} см.~Бенинг~В.\,Е.&&\\
\textbf{Коротышева~А.\,В.} см.~Зейфман~А.\,И.&&\\
\hangindent=23pt\noindent\textbf{Кривенко~М.\,П.} Непараметрическое оценивание элементов байесовского 
клас\-си-\linebreak
\vspace*{-12pt}\\
\hspace*{23pt}фикатора$\dotfill$&2&13\\
\textbf{Кривовязь~Г.\,Р.} см.~Конушин~В.\,С.&&\\
\textbf{Крылов~А.\,С.} см.~Павельева~Е.\,А.&&\\
\hangindent=23pt\noindent\textbf{Крылов~В.\,А.} Моделирование и классификация многоканальных дистанционных\linebreak
\vspace*{-12pt}\\
\hspace*{23pt}изображений с использованием копул$\dotfill$&4&34\\
\hangindent=23pt\noindent\textbf{Крючин~О.\,В.} Разработка параллельных эвристических алгоритмов подбора 
весовых\linebreak
\vspace*{-12pt}\\
\hspace*{23pt}коэффициентов искусственной нейтронной сети$\dotfill$&2&53\\
\hangindent=23pt\noindent\textbf{Кудрявцев~А.\,А., Шоргин~С.\,Я.} Байесовские модели массового обслуживания и 
надеж-\linebreak
\vspace*{-12pt}\\
\hspace*{23pt}ности: характеристики среднего числа заявок в системе $M\vert M \vert 1\vert 
\infty$$\dotfill$&3&16\\
\hangindent=23pt\noindent\textbf{Кузнецов~А.\,А.} Связь между временными и структурно-топологическими 
характери-\linebreak
\vspace*{-12pt}\\
\hspace*{23pt}стиками диаграмм ритма сердца здоровых людей$\dotfill$&4&39\\
\textbf{Кузнецов~И.\,П.} см.~Козеренко~Е.\,Б.&&\\
\textbf{Ле~Пезан~Д.} см.~Бунтман~Н.\,В.&&\\
\hangindent=23pt\noindent\textbf{Лукьяненко~А.\,С., Морозов~Е.\,В., Гуртов~А.\,В.} Анализ сетевого протокола с общей 
функ-\linebreak
\vspace*{-12pt}\\
\hspace*{23pt}цией расширения окна передачи сообщения при конфликтах$\dotfill$&2&46\\
\hangindent=23pt\noindent\textbf{Лямин~О.\,О.} О предельном поведении мощностей критериев в случае обобщенного\linebreak
\vspace*{-12pt}\\
\hspace*{23pt}распределения Лапласа$\dotfill$&3&47\\
\hangindent=23pt\noindent\textbf{Маркин~А.\,В., Шестаков~О.\,В.} Асимптотики оценки риска при пороговой 
обработке\linebreak
\vspace*{-12pt}\\
\hspace*{23pt}вейвлет-вейглет коэффициентов в задаче томографии$\dotfill$&2&36\\
\hangindent=23pt\noindent\textbf{Матвеева~С.\,С., Захарова~Т.\,В.} Сети массового обслуживания с наименьшей 
длиной\linebreak
\vspace*{-12pt}\\
\hspace*{23pt}очереди$\dotfill$&3&22\\
\hangindent=23pt\noindent\textbf{Матюшенко~С.\,И.} Стационарные характеристики двухканальной системы 
обслужива-\linebreak
\vspace*{-12pt}\\
\hspace*{23pt}ния с переупорядочиванием заявок и распределениями фазового типа$\dotfill$&4&68\\
\textbf{Минель~Ж.-Л.} см.~Бунтман~Н.\,В.&&\\
\textbf{Морозов~Е.\,В.} см.~Бородина~А.\,В.&&\\
\textbf{Морозов~Е.\,В.} см.~Лукьяненко~А.\,С.&&\\
\textbf{Ососков~М.\,В.} см.~Каратеев~С.\,Л.&&\\
\hangindent=23pt\noindent\textbf{Павельева~Е.\,А., Крылов~А.\,С.} Поиск и анализ ключевых точек радужной 
оболочки\linebreak
\vspace*{-12pt}\\
\hspace*{23pt}глаза методом преобразования Эрмита$\dotfill$&1&79\\
\textbf{Печинкин~А.\,В.} см.~Френкель~С.\,Л.,&&\\
\hangindent=23pt\noindent\textbf{Протасов~В.\,И.} Составление субъективного портрета с использованием 
эволюционно-\linebreak
\vspace*{-12pt}\\
\hspace*{23pt}го морфинга и квалиметрия метода$\dotfill$&1&83\\
\hangindent=23pt\noindent\textbf{Рудаков~К.\,В., Торшин~И.\,Ю.} Вопросы разрешимости задачи распознавания 
вторичной\linebreak
\vspace*{-12pt}\\
\hspace*{23pt}структуры белка$\dotfill$&2&25\\
\textbf{Сатин~Я.\,А.} см.~Зейфман~А.\,И.&&\\
\hangindent=23pt\noindent\textbf{Сейфуль-Мулюков~Р.\,Б.} Нефть как носитель информации о своем 
происхождении,\linebreak
\vspace*{-12pt}\\
\hspace*{23pt}структуре и эволюции$\dotfill$&1&41\\
\end{tabular}
}

{\tabcolsep=3pt
\begin{tabular}{p{388pt}rr}
&\textbf{Выпуск} & \textbf{Стр.}\\[6pt]
\textbf{Семендяев~Н.\,Н.} см.~Синицын~И.\,Н.&&\\
\textbf{Серебряков~В.\,А.} см.~Захаров~А.\,А.&&\\
\textbf{Синицын~В.\,И.} см.~Синицын~И.\,Н.&&\\
\hangindent=23pt\noindent\textbf{Синицын~И.\,Н., Синицын~В.\,И., Корепанов~Э.\, Р., Белоусов~В.\,В., 
Семендяев~Н.\,Н.} Оперативное построение информационных моделей движения полюса 
Земли\linebreak
\vspace*{-12pt}\\
\hspace*{23pt}методами линейных и линеаризованных фильтров$\dotfill$&1&2\\
\textbf{Сипина~А.\,В.} см.~Бенинг~В.\,Е.&&\\
\hangindent=23pt\noindent\textbf{Соколов~И.\,А.} О работах заслуженного деятеля науки Российской Федерации 
И.\,Н.~Синицына в области информационных технологий и автоматизации (к 70-летию\linebreak
\vspace*{-12pt}\\
\hspace*{23pt}со дня рождения)$\dotfill$&3&84\\
\textbf{Соколов~И.\,А.} см.~Илюшин~Г.\,Я.&&\\
\hangindent=23pt\noindent\textbf{Соколов~И.\,А., Королев~В.\,Ю.} Предисловие$\dotfill$&2&2\\
\textbf{Солдатов~С.\,А.} см.~Колесников~А.\,В.&&\\
\hangindent=23pt\noindent\textbf{Степанов~С.\,Ю.} Использование координатного метода фрагментации 
коммутаторной\linebreak
\vspace*{-12pt}\\
\hspace*{23pt}нейронной сети для сокращения трафика$\dotfill$&2&57\\
\textbf{Тимонина~Е.\,Е.} см.~Грушо~А.\,А.&&\\
\textbf{Торшин~И.\,Ю.} см.~Рудаков~К.\,В.&&\\
\textbf{Ульянов~В.\,В.} см.~Кавагучи~Ю.&&\\
\textbf{Фазекаш~И.} см.~Чупрунов~А.\,Н.&&\\
\textbf{Френкель~С.\,Л.} см.~Баранов~С.\,И.&&\\
\hangindent=23pt\noindent\textbf{Френкель~С.\,Л., Печинкин~А.\,В.} Оценка времени самовосстановления в 
цифровых\linebreak
\vspace*{-12pt}\\
\hspace*{23pt}системах после сбоев, вызываемых переходными помехами$\dotfill$&3&2\\
\textbf{Фуджикоши~Я.} см.~Кавагучи~Ю.&&\\
\hangindent=23pt\noindent\textbf{Цискаридзе~А.\,К.} Математическая модель и метод восстановления позы человека 
по\linebreak
\vspace*{-12pt}\\
\hspace*{23pt}стереопаре силуэтных изображений$\dotfill$&4&27\\
\hangindent=23pt\noindent\textbf{Чупраков~К.\,Г.} К вопросу о размещении коллективных средств отображения в 
ситуа-\linebreak
\vspace*{-12pt}\\
\hspace*{23pt}ционном зале с заданными параметрами$\dotfill$&4&89\\
\textbf{Чупраков~К.\,Г.} см.~Зацаринный~А.\,А.&&\\
\hangindent=23pt\noindent\textbf{Чупрунов~А.\,Н., Фазекаш~И.} Законы повторного логарифма для числа 
безошибочных\linebreak
\vspace*{-12pt}\\
\hspace*{23pt}блоков при помехоустойчивом кодировании$\dotfill$&3&42\\
\textbf{Шевцова~И.\,Г.} см.~Григорьева~М.\,Е.&&\\
\hangindent=23pt\noindent\textbf{Шестаков~О.\,В.} Аппроксимация распределения оценки риска пороговой 
обработки вейвлет-коэффициентов нормальным распределением при использовании 
выбо-\linebreak
\vspace*{-12pt}\\
\hspace*{23pt}рочной дисперсии$\dotfill$&4&73\\
\textbf{Шестаков~О.\,В.} см.~Маркин~А.\,В.&&\\
\textbf{Шоргин~С.\,Я.} см.~Зейфман~А.\,И.&&\\
\textbf{Шоргин~С.\,Я.} см.~Кудрявцев~А.\,А.&&\\
\end{tabular}
}

%\thispagestyle{myheadings}
\def\leftfootline{\small{\textbf{\thepage}
\hfill ИНФОРМАТИКА И ЕЁ ПРИМЕНЕНИЯ\ \ \ том~4\ \ \ выпуск~4\ \ \ 2010}
}%
 \def\rightfootline{\small{ИНФОРМАТИКА И ЕЁ ПРИМЕНЕНИЯ\ \ \ том~4\ \ \ выпуск~4\ \ \ 2010
 \hfill \textbf{\thepage}}}
 \label{end\stat}
%
%Том 10 Выпуск 1-4 Год 2016

\def\stat{cont-e}
{%\hrule\par
%\vskip 7pt % 7pt
\raggedleft\Large \bf%\baselineskip=3.2ex
2\,0\,1\,6\ \ A\,U\,T\,H\,O\,R\ \ I\,N\,D\,E\,X \vskip 17pt
 \hrule
 \par
\vskip 21pt plus 6pt minus 3pt }

\label{st\stat}

\def\tit{\ }

\def\aut{\ }
\def\auf{\ }

\def\leftkol{\ } %2016 AUTHOR INDEX} % ENGLISH ABSTRACTS}

\def\rightkol{\ } %2016 AUTHOR INDEX} %ENGLISH ABSTRACTS}

\titele{\tit}{\aut}{\auf}{\leftkol}{\rightkol}

\def\leftfootline{\small{\textbf{\thepage}
\hfill INFORMATIKA I EE PRIMENENIYA~--- INFORMATICS AND APPLICATIONS\ \ \ 2016\
\ \ volume~10\ \ \ issue\ 4}
}%
 \def\rightfootline{\small{INFORMATIKA I EE PRIMENENIYA~--- INFORMATICS AND APPLICATIONS\ \ \ 2016\ \ \ volume~10\ \ \ issue\ 4
\hfill \textbf{\thepage}}}

\vspace*{-12pt}
\vspace*{-18pt}

{\tabcolsep=2.8pt
\begin{tabular}{p{382pt}cc}
&\textbf{Issue} & \textbf{Page}\\[6pt]
\Avtors{Agalarov~M.\,Ya.} see~Agalarov~Ya.\,M.&&\\
\Avtors{Agalarov~Ya.\,M., Agalarov~M.\,Ya., and
Shorgin~V.\,S.} About the optimal threshold of queue\linebreak
\\[-12pt]
\hspace*{23pt}length in a~particular problem of profit maximization
in the $M/G/1$ queuing system&2&70--79\\
\Avtors{Alexeyevsky~D.\,A.} BioNLP ontology extraction from 
a~restricted language corpus with\linebreak
\\[-12pt]
\hspace*{23pt}context-free grammars&1&119--128\\
\Avtors{Andreev~S.\,D.} see~Gaidamaka~Yu.\,V.&&\\
\Avtors{Andreev~S.\,D.} see~Ometov~A.\,Ya.&&\\
\Avtors{Arkhipov~O.\,P., Arkhipov~P.\,O., and Sidorkin~I.\,I.} The
option to create a~local coordinate\linebreak
\\[-12pt]
\hspace*{23pt}system for synchronization of selected images&3&91--97\\
\Avtors{Arkhipov~P.\,O.} see~Arkhipov~O.\,P.&&\\
\Avtors{Belousov~V.\,V.} see~Shnurkov~P.\,V.&&\\
\Avtors{Belousov~V.\,V.} see~Shnurkov~P.\,V.&&\\
\Avtors{Bening~V.\,E.} Calculation of~the~asymptotic deficiency
of~some statistical procedures based\linebreak
\\[-12pt]
\hspace*{23pt}on~samples with~random sizes&4&34--45\\
\Avtors{Borisov~A.\,V., Bosov~A.\,V., and Miller~G.\,B.} Modeling and
monitoring of VoIP connection&2&\hphantom{1}2--13\\
\Avtors{Bosov~A.\,V.} see~Borisov~A.\,V.&&\\
\Avtors{Briukhov~D.\,O.} see~Stupnikov~S.\,A.&&\\
\Avtors{Callaos~N.\,K.\ and Seyful-Mulyukov~R.\,B.} Complexity and
its information content&1&129--139\\
\Avtors{Chertok~A.\,V., Kadaner~A.\,I., Khazeeva~G.\,T., and
Sokolov~I.\,A.} Regime switching detection\linebreak
\\[-12pt]
\hspace*{23pt}for~the~Levy driven
Ornstein--Uhlenbeck process using CUSUM methods&4&46--56\\
\Avtors{Chichagov~V.\,V.} Asymptotic expansions of mean absolute
error of uniformly minimum variance unbiased and maximum likelihood
estimators on the one-parameter exponential\linebreak
\\[-12pt]
\hspace*{23pt}family model of lattice distributions&3&66--76\\
\Avtors{Danishevsky~V.\,I.} see~Kolesnikov A.\,V.&&\\
\Avtors{Fazliev~A.\,Z.} see~Kalinichenko~L.\,A.&&\\
\Avtors{Fedoseev~A.\,A.} What is behind the concept of ``knowledge in
small packages''&3&105--110\\
\Avtors{Gaidamaka~Yu.\,V., Andreev~S.\,D., Sopin~E.\,S.,
Samouylov~K.\,E., and Shorgin~S.\,Ya.} Interference analysis
of~the~device-to-device communications model with~regard to~a~signal\linebreak
\\[-12pt]
\hspace*{23pt}propagation environment&4&\hphantom{1}2--10\\
\Avtors{Gasilov~A.\,V.} see~Yakovlev~O.\,A.&&\\
\Avtors{Goncharov~A.\,V.\ and Strijov~V.\,V.} Metric time series
classification using weighted dynamic\linebreak
\\[-12pt]
\hspace*{23pt}warping relative to centroids of classes&2&36--47\\
\Avtors{Gordov~E.\,P.} see~Kalinichenko~L.\,A.&&\\
\Avtors{Gorshenin~A.\,K.} Concept of online service for stochastic
modeling of real processes&1&72--81\\
\Avtors{Gorshenin~A.\,K.} see~Shnurkov~P.\,V.&&\\
\Avtors{Gorshenin~A.\,K.} see~Shnurkov~P.\,V.&&\\
\Avtors{Grusho~A.\,A., Grusho~N.\,A., Zabezhailo~M.\,I., and
Timonina~E.\,E.} Integration of statistical and\linebreak
\\[-12pt]
\hspace*{23pt}deterministic methods for
analysis of information security&3&2--8\\
\Avtors{Grusho~A.\,A., Zabezhailo~M.\,I., and Zatsarinny~A.\,A.} On
the advanced procedure to reduce\linebreak
\\[-12pt]
\hspace*{23pt}calculation of Galois closures&4&\hphantom{1}96--104\\
\Avtors{Grusho~N.\,A.} see~Grusho~A.\,A.&&\\
\Avtors{Havanskov~V.\,A.} see~Minin~V.\,A.&&\\
\Avtors{Inkova~O.\,Yu.} see~Zatsman~I.\,M.&&\\
\Avtors{Isachenko~R.\,V.\ and Strijov~V.\,V.} Metric learning in
multiclass time series classification\linebreak
\\[-12pt]
\hspace*{23pt}problem&2&48--57\\
\end{tabular}
}
\pagebreak

\def\leftfootline{\small{\textbf{\thepage}
\hfill INFORMATIKA I EE PRIMENENIYA~--- INFORMATICS AND APPLICATIONS\ \ \ 2016\
\ \ volume~10\ \ \ issue\ 4}
}%
 \def\rightfootline{\small{INFORMATIKA I EE PRIMENENIYA~---
INFORMATICS AND APPLICATIONS\ \ \ 2016\ \ \ volume~10\ \ \ issue\ 4
\hfill \textbf{\thepage}}}

\def\leftkol{2016 AUTHOR INDEX} % ENGLISH ABSTRACTS}

\def\rightkol{2016 AUTHOR INDEX} %ENGLISH ABSTRACTS}


{\tabcolsep=2.83pt
\begin{tabular}{p{382pt}cc}
&\textbf{Issue} & \textbf{Page}\\[6pt]
\Avtors{Kadaner~A.\,I.} see~Chertok~A.\,V.&&\\[.255pt]
\Avtors{Kalinichenko~L.\,A., Volnova~A.\,A., Gordov~E.\,P.,
Kiselyova~N.\,N., Kovaleva~D.\,A., Malkov~O.\,Yu., Okladnikov~I.\,G.,
Podkolodnyy~N.\,L., Pozanenko~A.\,S., Ponomareva~N.\,V.,
Stupnikov~S.\,A.,} \textbf{and Fazliev~A.\,Z.} Data access challenges for data
intensive\linebreak
\\[-12pt]
\hspace*{23pt}research in Russia&1& 2--22\\[.255pt]
\Avtors{Karasikov~M.\,E.\ and Strijov~V.\,V.} Feature-based
time-series classification&4&121--131\\[.255pt]
\Avtors{Khazeeva~G.\,T.} see~Chertok~A.\,V.&&\\[.255pt]
\Avtors{Khokhlov~Yu.\,S.} Multivariate fractional Levy motion and its
applications&2&\hphantom{1}98--106\\[.255pt]
\Avtors{Kirikov~I.\,A., Kolesnikov~A.\,V., Listopad~S.\,V., and
Rumovskaya~S.\,B.} Fine-grained hybrid\linebreak
\\[-12pt]
\hspace*{23pt}intelligent systems. Part 2:
Bidirectional hybridization&1&\hphantom{1}96--105\\[.255pt]
\Avtors{Kirikov~I.\,A., Kolesnikov~A.\,V., Listopad~S.\,V., and
Rumovskaya~S.\,B.} ``Virtual council''~---\linebreak
\\[-12pt]
\hspace*{23pt}source environment
supporting complex diagnostic decision making&3&81--90\\[.255pt]
\Avtors{Kiselyova~N.\,N.} see~Kalinichenko~L.\,A.&&\\[.255pt]
\Avtors{Kolesnikov A.\,V., Listopad~S.\,V., Rumovskaya~S.\,B., and
Danishevsky~V.\,I.} Informal axiomatic\linebreak
\\[-12pt]
\hspace*{23pt}theory of~the~role visual models&4&114--120\\[.255pt]
\Avtors{Kolesnikov~A.\,V.} see~Kirikov~I.\,A.&&\\[.255pt]
\Avtors{Kolesnikov~A.\,V.} see~Kirikov~I.\,A.&&\\[.255pt]
\Avtors{Kolin~K.\,K.} Humanitarian aspects of information
security&3&111--121\\[.255pt]
\Avtors{Konovalov~M.\,G.\ and Razumchik~R.\,V.} Dispatching
to~two parallel nonobservable queues using\linebreak
\\[-12pt]
\hspace*{23pt}only static
information&4&57--67\\[.255pt]
\Avtors{Korchagin~A.\,Yu.} see~Korolev~V.\,Yu.&&\\[.255pt]
\Avtors{Korchagin~A.\,Yu.} see~Korolev~V.\,Yu.&&\\[.255pt]
\Avtors{Korepanov~E.\,R.} see~Sinitsyn~I.\,N.&&\\[.255pt]
\Avtors{Korepanov~E.\,R.} see~Sinitsyn~I.\,N.&&\\[.255pt]
\Avtors{Korolev~V.\,Yu., Korchagin~A.\,Yu., and Zeifman~A.\,I.} The
Poisson theorem for Bernoulli trials\linebreak
\\[-12pt]
\hspace*{23pt}with~a~random probability
of~success and~a~discrete analog of~the~Weibull distribution&4&11--20\\[.255pt]
\Avtors{Korolev~V.\,Yu., Zeifman~A.\,I., and Korchagin~A.\,Yu.}
Asymmetric Linnik distributions as~limit\linebreak
\\[-12pt]
\hspace*{23pt}laws for~random sums
of~independent random variables with~finite variances&4&21--33\\[.255pt]
\Avtors{Koucheryavy~E.\,A.} see~Ometov~A.\,Ya.&&\\[.255pt]
\Avtors{Kovaleva~D.\,A.} see~Kalinichenko~L.\,A.&&\\[.255pt]
\Avtors{Kovalyov~S.\,P.} Metaprogramming to increase
manufacturability of large-scale software-\linebreak
\\[-12pt]
\hspace*{23pt}intensive systems&1&56--66\\[.255pt]
\Avtors{Krivenko~M.\,P.} Significance tests of feature selection for
classification&3&32--40\\[.255pt]
\Avtors{Kruzhkov~M.\,G.} see~Zalizniak~Anna~A.&&\\[.255pt]
\Avtors{Kruzhkov~M.\,G.} see~Zatsman~I.\,M.&&\\[.255pt]
\Avtors{Kudryavtsev~A.\,A.} Bayesian queueing and reliability models:
\textit{A~priori} distributions with\linebreak
\\[-12pt]
\hspace*{23pt}compact support&1&67--71\\[.255pt]
\Avtors{Kudryavtsev~A.\,A.} Characteristics dependent on the balance
coefficient in Bayesian models\linebreak
\\[-12pt]
\hspace*{23pt}with compact support of \textit{a priori}
distributions&3&77--80\\[.255pt]
\Avtors{Kudryavtsev~A.\,A.\ and Palionnaia~S.\,I.} Bayesian recurrent
model of reliability growth:\linebreak
\\[-12pt]
\hspace*{23pt}Parabolic distribution of parameters&2&80--83\\[.255pt]
\Avtors{Kudryavtsev~A.\,A.\ and Titova~A.\,I.} Bayesian queuing
and~reliability models: Degenerate-\linebreak
\\[-12pt]
\hspace*{23pt}Weibull case&4&68--71\\[.255pt]
\Avtors{Leontyev~N.\,D.\ and Ushakov~V.\,G.} Analysis of a queueing
system with autoregressive arrivals\linebreak
\\[-12pt]
\hspace*{23pt}and nonpreemptive priority&3&15--22\\[.255pt]
\Avtors{Listopad~S.\,V.} see~Kirikov~I.\,A.&&\\[.255pt]
\Avtors{Listopad~S.\,V.} see~Kirikov~I.\,A.&&\\[.255pt]
\Avtors{Listopad~S.\,V.} see~Kolesnikov A.\,V.&&\\[.255pt]
\Avtors{Malkov~O.\,Yu.} see~Kalinichenko~L.\,A.&&\\[.255pt]
\Avtors{Markov~A.\,S., Monakhov~M.\,M., and
Ulyanov~V.\,V.} Generalized Cornish--Fisher expansions\linebreak
\\[-12pt]
\hspace*{23pt}for distributions of statistics based on samples
of random size&2&84--91\\[.255pt]
\Avtors{Melnikov~A.\,K.\ and Ronzhin~A.\,F.} Generalized statistical
method of~text analysis based\linebreak
\\[-12pt]
\hspace*{23pt}on~calculation of~probability distributions
of~statistical values&4&89--95\\
\end{tabular}
}
\pagebreak

\def\leftfootline{\small{\textbf{\thepage}
\hfill INFORMATIKA I EE PRIMENENIYA~--- INFORMATICS AND APPLICATIONS\ \ \ 2016\
\ \ volume~10\ \ \ issue\ 4}
}%
 \def\rightfootline{\small{INFORMATIKA I EE PRIMENENIYA~---
INFORMATICS AND APPLICATIONS\ \ \ 2016\ \ \ volume~10\ \ \ issue\ 4
\hfill \textbf{\thepage}}}

\def\leftkol{2016 AUTHOR INDEX} % ENGLISH ABSTRACTS}

\def\rightkol{2016 AUTHOR INDEX} %ENGLISH ABSTRACTS}


{\tabcolsep=3pt
\begin{tabular}{p{381pt}cc}
&\textbf{Issue} & \textbf{Page}\\[6pt]
\Avtors{Meykhanadzhyan~L.\,A.} Stationary characteristics of the finite
capacity queueing system with\linebreak
\\[-12pt]
\hspace*{23pt}inverse service order and generalized
probabilistic priority&2&123--131\\[.23pt]
\Avtors{Miller~G.\,B.} see~Borisov~A.\,V.&&\\[.23pt]
\Avtors{Minin~V.\,A., Zatsman~I.\,M., Havanskov~V.\,A., and
Shubnikov~S.\,K.} Intensity of citation of scientific publications in
inventions on information and computer technologies patented\linebreak
\\[-12pt]
\hspace*{23pt}in Russia by domestic and foreign applicants&2&107--122\\[.23pt]
\Avtors{Monakhov~M.\,M.} see~Markov~A.\,S.&&\\[.23pt]
\Avtors{Naumov~V.\,A.\ and Samouylov~K.\,E.} On relationship
between queuing systems with resources\linebreak
\\[-12pt]
\hspace*{23pt}and Erlang networks&3&\hphantom{1}9--14\\[.23pt]
\Avtors{Okladnikov~I.\,G.} see~Kalinichenko~L.\,A.&&\\[.23pt]
\Avtors{Ometov~A.\,Ya., Andreev~S.\,D., Turlikov~A.\,M., and
Koucheryavy~E.\,A.} Performance analysis of\linebreak
\\[-12pt]
\hspace*{23pt}a wireless data
aggregation system with contention for contemporary sensor
networks&3&23--31\\[.23pt]
\Avtors{Palionnaia~S.\,I.} see~Kudryavtsev~A.\,A.&&\\[.23pt]
\Avtors{Podkolodnyy~N.\,L.} see~Kalinichenko~L.\,A.&&\\[.23pt]
\Avtors{Ponomareva~N.\,V.} see~Kalinichenko~L.\,A.&&\\[.23pt]
\Avtors{Popkova~N.\,A.} see~Zatsman~I.\,M.&&\\[.23pt]
\Avtors{Pozanenko~A.\,S.} see~Kalinichenko~L.\,A.&&\\[.23pt]
\Avtors{Razumchik~R.\,V.} see~Konovalov~M.\,G.&&\\[.23pt]
\Avtors{Ronzhin~A.\,F.} see~Melnikov~A.\,K.&&\\[.23pt]
\Avtors{Rumovskaya~S.\,B.} see~Kirikov~I.\,A.&&\\[.23pt]
\Avtors{Rumovskaya~S.\,B.} see~Kirikov~I.\,A.&&\\[.23pt]
\Avtors{Rumovskaya~S.\,B.} see~Kolesnikov A.\,V.&&\\[.23pt]
\Avtors{Samouylov~K.\,E.} see~Gaidamaka~Yu.\,V.&&\\[.23pt]
\Avtors{Samouylov~K.\,E.} see~Naumov~V.\,A.&&\\[.23pt]
\Avtors{Serebryanskii~S.\,M.} see~Tyrsin~A.\,N.&&\\[.23pt]
\Avtors{Seyful-Mulyukov~R.\,B.} see~Callaos~N.\,K.&&\\[.23pt]
\Avtors{Shestakov~O.\,V.} Statistical properties of the denoising method
based on the stabilized hard\linebreak
\\[-12pt]
\hspace*{23pt}thresholding&2&65--69\\[.23pt]
\Avtors{Shestakov~O.\,V.} The strong law of large numbers for the risk
estimate in the problem of\linebreak
\\[-12pt]
\hspace*{23pt}tomographic image reconstruction from
projections with a correlated noise&3&41--45\\[.23pt]
\Avtors{Shestakov~O.\,V.} see~Zakharova~T.\,V.&&\\[.23pt]
\Avtors{Shnurkov~P.\,V., Gorshenin~A.\,K., and Belousov~V.\,V.}
Analytical solution of~the~optimal control\linebreak
\\[-12pt]
\hspace*{23pt}task of~a~semi-Markov
process with~finite set of~states&4&72--88\\[.23pt]
\Avtors{Shnurkov~P.\,V., Zasypko~V.\,V., Belousov~V.\,V., and
Gorshenin~A.\,K.} Development of the algorithm of numerical solution
of the optimal investment control problem\linebreak
\\[-12pt]
\hspace*{23pt}in the closed dynamical model of three-sector economy&1&82--95\\[.23pt]
\Avtors{Shorgin~S.\,Ya.} see~Gaidamaka~Yu.\,V.&&\\[.23pt]
\Avtors{Shorgin~V.\,S.} see~Agalarov~Ya.\,M.&&\\[.23pt]
\Avtors{Shubnikov~S.\,K.} see~Minin~V.\,A.&&\\[.23pt]
\Avtors{Sidorkin~I.\,I.} see~Arkhipov~O.\,P.&&\\[.23pt]
\Avtors{Sinitsyn~I.\,N.} Analytical modeling of processes in stochastic
systems with complex fractional\linebreak
\\[-12pt]
\hspace*{23pt}order Bessel nonlinearities&3&55--65\\[.23pt]
\Avtors{Sinitsyn~I.\,N.} Orthogonal supoptimal filters for nonlinear
stochastic systems on manifolds&1&34--44\\[.23pt]
\Avtors{Sinitsyn~I.\,N.\ and Korepanov~E.\,R.} Normal Pugachev
conditionally-optimal filters and extra-\linebreak
\\[-12pt]
\hspace*{23pt}polators for state linear stochastic systems&2&14--23\\[.23pt]
\Avtors{Sinitsyn~I.\,N.\ and Sinitsyn~V.\,I.} Analytical modeling of
distributions in stochastic systems on\linebreak
\\[-12pt]
\hspace*{23pt}manifolds based on ellipsoidal approximation&1&45--55\\[.23pt]
\Avtors{Sinitsyn~I.\,N., Sinitsyn~V.\,I., and
Korepanov~E.\,R.} Ellipsoidal suboptimal filters for nonlinear\linebreak
\\[-12pt]
\hspace*{23pt}stochastic systems on manifolds&2&24--35\\[.23pt]
\Avtors{Sinitsyn~V.\,I.} see~Sinitsyn~I.\,N.&&\\[.23pt]
\Avtors{Sinitsyn~V.\,I.} see~Sinitsyn~I.\,N.&&\\[.23pt]
\Avtors{Skvortsov~N.\,A.} see~Stupnikov~S.\,A.&&\\[.23pt]
\Avtors{Sokolov~I.\,A.} see~Chertok~A.\,V.&&\\
\end{tabular}
}
\pagebreak

\def\leftfootline{\small{\textbf{\thepage}
\hfill INFORMATIKA I EE PRIMENENIYA~--- INFORMATICS AND APPLICATIONS\ \ \ 2016\
\ \ volume~10\ \ \ issue\ 4}
}%
 \def\rightfootline{\small{INFORMATIKA I EE PRIMENENIYA~---
INFORMATICS AND APPLICATIONS\ \ \ 2016\ \ \ volume~10\ \ \ issue\ 4
\hfill \textbf{\thepage}}}

\def\leftkol{2016 AUTHOR INDEX} % ENGLISH ABSTRACTS}

\def\rightkol{2016 AUTHOR INDEX} %ENGLISH ABSTRACTS}


{\tabcolsep=3pt
\begin{tabular}{p{382pt}cc}
&\textbf{Issue} & \textbf{Page}\\[6pt]
\Avtors{Sopin~E.\,S.} see~Gaidamaka~Yu.\,V.&&\\
\Avtors{Strijov~V.\,V.} see~Goncharov~A.\,V.&&\\
\Avtors{Strijov~V.\,V.} see~Isachenko~R.\,V.&&\\
\Avtors{Strijov~V.\,V.} see~Karasikov~M.\,E.&&\\
\Avtors{Stupnikov~S.\,A., Briukhov~D.\,O., and Skvortsov~N.\,A.}
Co-lending systemic risk analysis over\linebreak
\\[-12pt]
\hspace*{23pt}heterogeneous data collections&1&23--33\\
\Avtors{Stupnikov~S.\,A.} see~Kalinichenko~L.\,A.&&\\
\Avtors{Suchkov~A.\,P.} see~Zatsarinny~A.\,A.&&\\
\Avtors{Timonina~E.\,E.} see~Grusho~A.\,A.&&\\
\Avtors{Titova~A.\,I.} see~Kudryavtsev~A.\,A.&&\\
\Avtors{Turlikov~A.\,M.} see~Ometov~A.\,Ya.&&\\
\Avtors{Tyrsin~A.\,N.\ and Serebryanskii~S.\,M.} Recognition of
dependences on the basis of inverse\linebreak
\\[-12pt]
\hspace*{23pt}mapping&2&58--64\\
\Avtors{Ulyanov~V.\,V.} see~Markov~A.\,S.&&\\
\Avtors{Ushakov~V.\,G.} Queueing system with working vacations and
hyperexponential input stream&2&92--97\\
\Avtors{Ushakov~V.\,G.} see~Leontyev~N.\,D.&&\\
\Avtors{Volnova~A.\,A.} see~Kalinichenko~L.\,A.&&\\
\Avtors{Yakovlev~O.\,A.\ and Gasilov~A.\,V.} Speeded-up stereo
matching using geodesic support weights&3&\hphantom{1}98--104\\
\Avtors{Zabezhailo~M.\,I.} see~Grusho~A.\,A.&&\\
\Avtors{Zabezhailo~M.\,I.} see~Grusho~A.\,A.&&\\
\Avtors{Zakharova~T.\,V.\ and Shestakov~O.\,V.} Precision analysis of
wavelet processing of aerodynamic\linebreak
\\[-12pt]
\hspace*{23pt}flow patterns&3&46--54\\
\Avtors{Zalizniak~Anna~A.\ and Kruzhkov~M.\,G.} Database
of~Russian impersonal verbal constructions&4&132--141\\
\Avtors{Zasypko~V.\,V.} see~Shnurkov~P.\,V.&&\\
\Avtors{Zatsarinny~A.\,A.\ and Suchkov~A.\,P.} Systems engineering
approaches to~the~establishment of\linebreak
\\[-12pt]
\hspace*{23pt}a~system for~decision support based
on~situational analysis&4&105--113\\
\Avtors{Zatsarinny~A.\,A.} see~Grusho~A.\,A.&&\\
\Avtors{Zatsman~I.\,M., Inkova~O.\,Yu., Kruzhkov~M.\,G., and
Popkova~N.\,A.} Representation of cross-\linebreak
\\[-12pt]
\hspace*{23pt}lingual knowledge about
connectors in supracorpora databases&1&106--118\\
\Avtors{Zatsman~I.\,M.} see~Minin~V.\,A.&&\\
\Avtors{Zeifman~A.\,I.} see~Korolev~V.\,Yu.&&\\
\Avtors{Zeifman~A.\,I.} see~Korolev~V.\,Yu.&&\\
\end{tabular}
}

%\thispagestyle{myheadings}
\def\leftfootline{\small{\textbf{\thepage}
\hfill INFORMATIKA I EE PRIMENENIYA~--- INFORMATICS AND APPLICATIONS\ \ \ 2016\
\ \ volume~10\ \ \ issue\ 4}
}%
 \def\rightfootline{\small{INFORMATIKA I EE PRIMENENIYA~---
INFORMATICS AND APPLICATIONS\ \ \ 2016\ \ \ volume~10\ \ \ issue\ 4
\hfill \textbf{\thepage}}}

 \label{end\stat}

\newpage

%\def\stat{rekl}
%\label{preobr}

%\def\tit{АКАДЕМИК ПУГАЧЁВ  ВЛАДИМИР СЕМЁНОВИЧ\\
%25.03.1911--25.03.1998}


%   \vspace*{-48pt}
%   \begin{center}\LARGE
%Академик Пугачёв  Владимир Семёнович\\ (25.03.1911--25.03.1998)
%   \end{center}
   
   %\vspace*{2.5mm}
   
   \begin{center}

{\prgsh\LARGE
ОБЪЯВЛЕНИЯ О КОНФЕРЕНЦИЯХ}

\end{center}
%\hrule

\vspace*{6pt}

   
   \vspace*{10mm}
   
   \thispagestyle{empty}

\noindent
\begin{tabular}{cc}
%\begin{center}
\multicolumn{1}{c}{\raisebox{-40pt}[0pt][0pt]{\mbox{%
\epsfxsize=33mm
\epsfbox{vspu.eps}
}}}
%\end{center}
&
\tabcolsep=0pt\begin{tabular}{c}
{\prg{\Large\textbf{XII Всероссийское совещание}}}\\[6pt]
{\prg{\Large\textbf{по проблемам управления}}}\\[12pt]
{\prg{\large 16--19 июня 2014~г.}}\\[6pt] 
{\prg{\large Институт проблем управления имени В.\,А.~Трапезникова РАН}}\\[6pt]
{\prg{\large Москва, Россия}}
\end{tabular}
\end{tabular}

\vspace*{60pt}

     
 { %\large    
 XII Всероссийское совещание по проблемам управления (ВСПУ XII), посвященное 75-летию 
Института проблем управления (ИПУ) имени В.\,А.~Трапезникова РАН, проводится 16--19~июня 
2014~г.\ 
в ИПУ РАН (г.~Москва, Россия). ВСПУ XII организуется ИПУ РАН при поддержке РФФИ, Отделения 
энергетики, машиностроения, механики и процессов управления Российской академии наук, 
Российского 
национального комитета по автоматическому управлению, Академии навигации и управ\-ле\-ния 
движением, 
Научного совета РАН по комплексным проблемам управления и автоматизации, Совета по 
мехатронике и робототехнике РАН. Официальный язык Совещания~--- русский.

\vspace*{24pt}
     
     \textbf{Направления работы}
     \begin{enumerate}[1.]
\item Теория систем управления
\item Управление подвижными объектами и навигация
\item Интеллектуальные системы управления
\item Управление в промышленности, транспортом и логистикой
\item Управление системами междисциплинарной природы
\item Средства измерения, вычислений и контроля в управлении
\item Системный анализ и принятие решений в задачах управления
\item Информационные технологии в управлении
\item Проблемы образования в области управления: современное содержание и технологии обучения
\end{enumerate}

\vspace*{24pt}

     Подробная информация о Совещании находится на сайте {\sf http://vspu2014.ipu.ru}. Срок 
окончательной подачи докладов через систему подачи докладов на сайте~--- \textbf{30~ноября} 
2013~г.
}


%\end{document}

%\include{nekrolog-rb}


%\end{document}

%\include{IPPM-25}

\def\stat{cont-rus}
{%\hrule\par
%\vskip 7pt % 7pt
\vspace*{-24pt}
\raggedleft\Large \bf%\baselineskip=3.2ex
Правила подготовки рукописей  для публикации в журнале
<<Информатика~и~её~применения>> \vskip 8pt
    \hrule
    \par
\vskip 14pt plus 6pt minus 3pt }

\label{st\stat}

\def\tit{\ }

\def\aut{\ }
\def\auf{\ }

\def\leftkol{\ }
% Правила подготовки рукописей  для публикации в журнале
%<<Информатика и её применения>>

\def\rightkol{\ }
%Правила подготовки рукописей  для публикации в журнале
%<<Информатика и её применения>>}


\titele{\tit}{\aut}{\auf}{\leftkol}{\rightkol}


\vspace*{-60pt}
{ %\small

Журнал <<Информатика и её применения>>
публикует теоретические, обзорные и дискуссионные статьи,
посвященные научным исследованиям и разработкам в области
информатики и ее приложений.

Журнал издается на русском языке. По специальному решению
редколлегии отдельные статьи могут печататься на английском языке.

Тематика журнала охватывает следующие направления:
\begin{itemize}
\item теоретические основы информатики;\\[-15pt]
      \item
математические методы исследования сложных систем и процессов;\\[-15pt]
           \item
информационные системы и сети;\\[-15pt]
                \item
информационные технологии;\\[-15pt]
                     \item
архитектура и программное обеспечение вычислительных комплексов и сетей.\\[-15pt]
\end{itemize}


\noindent
\begin{enumerate}[1.]
\item В журнале печатаются статьи, содержащие результаты, ранее не опубликованные и
не предназначенные к одновременной публикации в других изданиях.

%Публикация не должна нарушать закон об авторских правах.
Публикация предоставленной автором(ами) рукописи не должна нарушать 
положений глав~69, 70 раздела~VII части~IV Гражданского кодекса, 
которые определяют права на результаты интеллектуальной деятельности 
и~средства индивидуализации, в~том числе авторские права, в~РФ.

Ответственность за нарушение авторских прав, в~случае предъявления претензий к~редакции журнала,  
несут авторы статей.



Направляя рукопись в редакцию, авторы сохраняют свои права на данную
рукопись и при этом передают учредителям и редколлегии журнала неисключительные права на
издание статьи на русском языке 
(или на языке статьи, если он отличен от рус\-ско\-го) и~на перевод ее на английский
язык, а~также на
ее распространение в России и за рубежом. 
Каждый автор должен представить в~редакцию подписанный 
с~его стороны <<Лицензионный договор о~передаче неисключительных прав 
на использование произведения>>, текст которого размещен по адресу 
{\sf http://www.ipiran.ru/publications/licence.doc}. 
Этот договор может быть пред\-став\-лен в~бумажном (в~2-х экз.)\ 
или в~электронном виде (отсканированная копия заполненного и~подписанного документа).




Редколлегия вправе запросить у авторов экспертное заключение о возможности
пуб\-ли\-ка\-ции пред\-став\-лен\-ной статьи в открытой печати.\\[-13.5pt]

\item К статье прилагаются данные автора (авторов) (см.\ п.~8). При наличии нескольких
авторов указывается фамилия автора, ответственного за переписку с редакцией.\\[-13.5pt]

\item Редакция журнала осуществляет экспертизу присланных статей в соответствии с
принятой в журнале процедурой рецензирования.

Возвращение рукописи на доработку не означает ее принятия к печати.

Доработанный вариант с ответом на замечания рецензента необходимо прислать в
редакцию.\\[-13.5pt]

\item Решение редколлегии о публикации статьи или ее отклонении сообщается авторам.

Редколлегия может также направить авторам текст рецензии на их статью. Дискуссия по
поводу отклоненных статей не ведется.\\[-13.5pt]

%\pagebreak

\item Редактура статей высылается авторам для просмотра. Замечания к редактуре должны
быть присланы авторами в кратчайшие сроки.\\[-13.5pt]

\item Рукопись предоставляется в электронном виде в форматах MS WORD (.doc или
.docx) или \LaTeX\  (.tex), дополнительно~--- в формате .pdf, на дискете, лазерном диске
или электронной почтой. Предоставление бумажной рукописи необязательно.\\[-13.5pt]

\item При подготовке рукописи в MS Word рекомендуется использовать следующие
настройки.

Параметры страницы:
формат~--- А4; ориентация~--- книжная; поля (см): внутри~--- 2,5, снаружи~--- 1,5,
сверху~--- 2, снизу~--- 2, от края до нижнего колонтитула~--- 1,3.

Основной текст: стиль~--- <<Обычный>>, шрифт~--- Times New Roman, размер~---
14~пунк\-тов, абзацный отступ~--- 0,5~см, 1,5~интервала, выравнивание~--- по ширине.

\pagebreak

\def\leftkol{Правила подготовки рукописей  для публикации в журнале
<<Информатика и её применения>>}

\def\rightkol{Правила подготовки рукописей  для публикации в журнале
<<Информатика и её применения>>}



Рекомендуемый объем рукописи~--- не свыше 10~страниц указанного формата.
При превышении указанного объема редколлегия вправе потребовать от 
автора сокращения объема рукописи.


Сокращения слов, помимо стандартных, не допускаются. Допускается минимальное
количество аббревиатур.


Все страницы рукописи нумеруются.

Шаблоны оформления представлены в интернете:

\noindent
 {\sf
http://www.ipiran.ru/journal/template\_iiep\_ssi\_2024.zip}\\[-14pt]

\item Статья должна содержать следующую информацию на {\bfseries\textit{русском и
английском языках}}:\\[-16pt]

\begin{itemize}
\item название статьи;\\[-15pt]
\item Ф.И.О.\ авторов, на английском можно только имя и фамилию;\\[-15pt]
\item место работы, с указанием почтового адреса организации и электронного адреса каждого
автора;\\[-15pt]
\item сведения об авторах, в соответствии с форматом, образцы которого
представлены на страницах:



\def\leftfootline{\small{\textbf{\thepage}
\hfill ИНФОРМАТИКА И ЕЁ ПРИМЕНЕНИЯ\ \ \ том\ 18\ \ \ выпуск\ 3\ \ \ 2024}
}%
 \def\rightfootline{\small{ИНФОРМАТИКА И ЕЁ ПРИМЕНЕНИЯ\ \ \ том\ 18\ \ \ выпуск\ 3\ \ \ 2024
\hfill \textbf{\thepage}}}



{\sf http://www.ipiran.ru/journal/issues/2013\_07\_01/authors.asp} и

{\sf http://www.ipiran.ru/journal/issues/2013\_07\_01\_eng/authors.asp};
\item аннотация (не менее 100~слов на каждом из языков). Аннотация~--- это краткое
резюме работы, которое может публиковаться отдельно. Она является основным
источником информации в~ин\-фор\-ма\-ци\-он\-ных системах и базах данных. Английская
аннотация должна быть оригинальной, может не быть дословным переводом русского
текста и должна быть написана хорошим английским языком. В~аннотации не должно
быть ссылок на литературу и, по возможности, формул;\\[-15pt]
\item ключевые слова~--- желательно из принятых в мировой
на\-уч\-но-тех\-ни\-че\-ской литературе тематических тезаурусов. Предложения не
могут быть ключевыми словами;\\[-15pt]
\item источники финансирования работы (ссылки на гранты, проекты,
поддерживающие организации и~т.\,п.).
\end{itemize}



%\pagebreak

\item  Требования к спискам литературы.\\[-14pt]

Ссылки на литературу в тексте статьи нумеруются (в квадратных скобках) и
располагаются в каждом из списков литературы в порядке  первых упоминаний. Если источник имеет DOI и/или EDN,
то их необходимо указывать.

Списки литературы представляются в двух вариантах:\\[-14pt]


\noindent
\begin{enumerate}[(1)]
\item \textbf{Список литературы к русскоязычной части}. Русские и английские
работы~---  на языке и в алфавите оригинала;\\[-14.5pt]
\item  \textbf{References}. Русские работы и работы на других языках~--- в латинской
транслитерации с переводом на английский язык; английские работы и работы на других
языках~--- на языке оригинала.
\end{enumerate}

Необходимо для составления списка ``References'' пользоваться размещенной на сайте
{\sf http://www. translit.net/ru/bgn/} бесплатной программой транслитерации русского
 текста в~латиницу. %, при этом в~за\-клад\-ке <<варианты\ldots>> следует выбратьопцию BGN.

Список литературы ``References'' приводится полностью отдельным блоком, повторяя все
позиции из списка литературы к русскоязычной части, независимо от того, имеются или
нет в нем иностранные источники. Если в списке литературы к русскоязычной части есть
ссылки на иностранные публикации, набранные латиницей, они полностью повторяются в
списке ``References''.

Ниже приведены примеры ссылок на различные виды публикаций в списке ``References''.

\def\leftfootline{\small{\textbf{\thepage}
\hfill ИНФОРМАТИКА И ЕЁ ПРИМЕНЕНИЯ\ \ \ том\ 18\ \ \ выпуск\ 3\ \ \ 2024}
}%
 \def\rightfootline{\small{ИНФОРМАТИКА И ЕЁ ПРИМЕНЕНИЯ\ \ \ том\ 18\ \ \ выпуск\ 3\ \ \ 2024
\hfill \textbf{\thepage}}}

{\small

\noindent
\textbf{Описание статьи из журнала:}

\Aue{Zagurenko, A.\,G., V.\,A.~Korotovskikh, A.\,A.~Kolesnikov, A.\,V.~Timonov, and D.\,V.~Kardymon}. 2008.
Tekhniko-ekonomicheskaya optimizatsiya dizayna gidrorazryva plasta [Technical and
economic optimization of the design
of hydraulic fracturing]. \textit{Neftyanoe hozyaystvo} [\textit{Oil Industry}] 11:54--57.

\Aue{Zhang, Z., and D.~Zhu}. 2008. Experimental research on the localized
electrochemical micromachining. \textit{Russ. J.~Electrochem.}  44(8):926--930.
{\sf doi:10.1134/S1023193508080077}.

\noindent
\textbf{Описание статьи из электронного журнала:}

\Aue{Swaminathan, V., E.~Lepkoswka-White, and B.\,P.~Rao}. 1999. Browsers or buyers in cyberspace? An
investigation of electronic factors influencing electronic exchange. \textit{JCMC}
5(2). Available at: {\sf http://www.ascusc.org/jcmc/vol5/issue2/} (accessed April~28, 2011).

\def\leftkol{Правила подготовки рукописей  для публикации в журнале
<<Информатика и её применения>>}

\def\rightkol{Правила подготовки рукописей  для публикации в журнале
<<Информатика и её применения>>}


\noindent
\textbf{Описание статьи из продолжающегося издания (сборника трудов):}

\Aue{Astakhov, M.\,V., and T.\,V.~Tagantsev}. 2006. Eksperimental'noe
issledovanie prochnosti soedineniy ``stal'--kompozit'' [Experimental study of
the strength of joints ``steel--composite'']. \textit{Trudy MGTU
``Matematicheskoe modelirovanie slozhnykh tekh\-ni\-che\-skikh sistem''}
[\textit{Bauman MSTU ``Mathematical Modeling of Complex Technical
Systems'' Proceedings}]. 593:125--130.


\pagebreak



\noindent
\textbf{Описание материалов конференций:}

\Aue{Usmanov, T.\,S., A.\,A.~Gusmanov, I.\,Z.~Mullagalin, R.\,Ju.~Muhametshina, A.\,N.~Chervyakova, and
A.\,V.~Sveshnikov}. 2007. Osobennosti proektirovaniya razrabotki mestorozhdeniy
s primeneniem gidrorazryva
plasta [Features of the design of field development with the use of hydraulic fracturing].
\textit{Trudy 6-go
Mezhdu\-na\-rod\-no\-go Simpoziuma ``Novye resursosberegayushchie tekhnologii nedropol'zovaniya i povysheniya
neftegazootdachi''} [\textit{6th  Symposium (International) ``New Energy Saving Subsoil Technologies and
the Increasing of the Oil and Gas Impact'' Proceedings}]. Moscow. 267--272.



\def\leftfootline{\small{\textbf{\thepage}
\hfill ИНФОРМАТИКА И ЕЁ ПРИМЕНЕНИЯ\ \ \ том\ 18\ \ \ выпуск\ 3\ \ \ 2024}
}%
 \def\rightfootline{\small{ИНФОРМАТИКА И ЕЁ ПРИМЕНЕНИЯ\ \ \ том\ 18\ \ \ выпуск\ 3\ \ \ 2024
\hfill \textbf{\thepage}}}



\noindent
\textbf{Описание книги (монографии, сборники):}



Lindorf, L.\,S., and L.\,G.~Mamikoniants, eds. 1972.
\textit{Ekspluatatsiya turbogeneratorov s neposredstvennym
okhlazhdeniem} [\textit{Operation of turbine generators with direct cooling}].
Moscow: Energy Publs. 352~p.


\Aue{Latyshev, V.\,N.} 2009. \textit{Tribologiya rezaniya. Kn.~1: Friktsionnye protsessy
pri rezanii metallov}
[\textit{Tribology of cutting. Vol.~1: Frictional processes in metal cutting}]. Ivanovo: Ivanovskii
State Univ. 108~p.

\def\leftkol{Правила подготовки рукописей  для публикации в журнале
<<Информатика и её применения>>}

\def\rightkol{Правила подготовки рукописей  для публикации в журнале
<<Информатика и её применения>>}

\noindent
\textbf{Описание переводной книги}
(в списке литературы к русскоязычной части необходимо указать:~/ Пер.\ с англ.~---
после названия книги, а в конце ссылки указать оригинал книги в круглых скобках):
\begin{enumerate}[1.]
\item  В русскоязычной части:

\def\leftfootline{\small{\textbf{\thepage}
\hfill ИНФОРМАТИКА И ЕЁ ПРИМЕНЕНИЯ\ \ \ том\ 18\ \ \ выпуск\ 3\ \ \ 2024}
}%
 \def\rightfootline{\small{ИНФОРМАТИКА И ЕЁ ПРИМЕНЕНИЯ\ \ \ том\ 18\ \ \ выпуск\ 3\ \ \ 2024
\hfill \textbf{\thepage}}}

\Au{Тимошенко С.\,П., Янг Д.\,Х., Уивер~У.}
Колебания в инженерном деле~/ Пер.\ с англ.~--- М.: Машиностроение, 1985. 472~с.
(\Au{Timoshenko~S.\,P., Young~D.\,H., Weaver~W.}
Vibration problems in engineering.~--- 4th ed.~--- New York, NY, USA: Wiley, 1974. 521~p.)\\[-13.5pt]
\item  В англоязычной части:

\Aue{Timoshenko, S.\,P., D.\,H.~Young, and W.~Weaver}.
1974. \textit{Vibration problems in engineering}. 4th ed. New York: 
Wiley. 521~p.
\end{enumerate}

\vspace*{-3pt}


\noindent
\textbf{Описание неопубликованного документа:}


\Aue{Latypov, A.\,R., M.\,M.~Khasanov, and V.\,A.~Baikov}.
2004 (unpubl.). Geologiya i~dobycha (NGT GiD) [Geology and production (NGT GiD)]. Certificate on official registration of the computer program
No.\,2004611198. 

\noindent
\textbf{Описание интернет-ресурса:}


Pravila tsitirovaniya istochnikov [Rules for the citing of sources]. Available at: {\sf
http://www.scribd.com/doc/1034528/} (accessed February~7, 2011).

%\pagebreak

\noindent
\textbf{Описание диссертации или автореферата диссертации:}

\Aue{Semenov, V.\,I.}
2003. Matematicheskoe modelirovanie plazmy v sisteme kompaktnyy tor [Mathematical
modeling of the plasma in the compact torus].  Moscow.  D.Sc.\ Diss. 272~p.

\Aue{Kozhunova, O.\,S.} 2009. Tekhnologiya razrabotki semanticheskogo
slovarya informatsionnogo monitoringa [Technology of development of
semantic dictionary of information monitoring system].  Moscow: IPI RAN. PhD Thesis. 23~p.


\noindent
\textbf{Описание ГОСТа:}

GOST 8.586.5-2005. 2007. Metodika vypolneniya izmereniy. Izmerenie raskhoda i~kolichestva zhidkostey i~gazov
s~pomoshch'yu standartnykh suzhayushchikh ustroystv [Method of measurement.
Measurement of flow rate and volume of liquids and gases by means of orifice devices]. Moscow:
Standardinform  Publs. 10~p.

\noindent
\textbf{Описание патента:}

\Aue{Bolshakov, M.\,V., A.\,V.~Kulakov, A.\,N.~Lavrenov, and M.\,V.~Palkin}.
2006. Sposob orientirovaniya po krenu letatel'nogo
apparata s opti\-che\-skoy golovkoy
samonavedeniya [The way to orient on the roll of aircraft with optical homing head].
Patent RF No.\,2280590.
}

\item Присланные в редакцию материалы авторам не возвращаются.\\[-13.5pt]

\item При отправке файлов по электронной почте просим придерживаться следующих
правил:
\begin{itemize}
\item указывать в поле subject (тема) название журнала и фамилию автора;\\[-13.5pt]
\item указывать в тексте письма название статьи, авторов и~журнал, в~который направляется статья;\\[-13.5pt]
\item использовать attach (присоединение);\\[-13.5pt]
\item в состав электронной версии статьи должны входить: файл, содержащий текст
статьи, и файл(ы), содержащий(е) иллюстрации.\\[-13.5pt]
\end{itemize}

\item Журнал <<Информатика и её применения>> является некоммерческим изданием.
Плата за публикацию не взимается, гонорар авторам не выплачивается.
\end{enumerate}



\def\leftfootline{\small{\textbf{\thepage}
\hfill ИНФОРМАТИКА И ЕЁ ПРИМЕНЕНИЯ\ \ \ том\ 18\ \ \ выпуск\ 3\ \ \ 2024}
}%
 \def\rightfootline{\small{ИНФОРМАТИКА И ЕЁ ПРИМЕНЕНИЯ\ \ \ том\ 18\ \ \ выпуск\ 3\ \ \ 2024
\hfill \textbf{\thepage}}}


\vspace*{-1mm}

\begin{center}

\textbf{Адрес редакции журнала <<Информатика и её применения>>:} \\




Москва 119333, ул.~Вавилова, д.~44, корп.~2, ФИЦ ИУ РАН\\[-10pt]

\

Тел.: +7\,(499)\,135-86-92\ \ Факс:  +7\,(495)\,930-45-05\\[-10pt]

 \

e-mail:   {\sf iiep@frccsc.ru} (Стригина Светлана Николаевна)\\[-10pt]

\

{\sf http://www.ipiran.ru/journal/issues/}
\end{center}
}


\def\leftkol{Правила подготовки рукописей  для публикации в журнале
<<Информатика и её применения>>}

\def\rightkol{Правила подготовки рукописей  для публикации в журнале
<<Информатика и её применения>>}


\def\leftfootline{\small{\textbf{\thepage}
\hfill ИНФОРМАТИКА И ЕЁ ПРИМЕНЕНИЯ\ \ \ том\ 18\ \ \ выпуск\ 3\ \ \ 2024}
}%
 \def\rightfootline{\small{ИНФОРМАТИКА И ЕЁ ПРИМЕНЕНИЯ\ \ \ том\ 18\ \ \ выпуск\ 3\ \ \ 2024
\hfill \textbf{\thepage}}} 
\def\stat{podg-e}
{%\hrule\par
%\vskip 7pt % 7pt
\vspace*{-24pt}
\raggedleft\Large \bf%\baselineskip=3.2ex
Requirements for manuscripts submitted to Journal
``Informatics~and~Applications'' \vskip 8pt
    \hrule
    \par
\vskip 21pt plus 6pt minus 3pt }

\label{st\stat}

\def\tit{\ }

\def\aut{\ }
\def\auf{\ }

\def\leftkol{\ }

\def\rightkol{\ }
%Requirements for manuscripts submitted to Journal
%``Informatics~and~Applications''}

\titele{\tit}{\aut}{\auf}{\leftkol}{\rightkol}

\def\leftfootline{\small{\textbf{\thepage}
\hfill INFORMATIKA I EE PRIMENENIYA~--- INFORMATICS AND APPLICATIONS\ \ \ 2019\
\ \ volume~13\ \ \ issue\ 4}
}%
 \def\rightfootline{\small{INFORMATIKA I EE PRIMENENIYA~--- INFORMATICS AND APPLICATIONS\ \ \ 2019\ \ \ volume~13\ \ \ issue\ 4
\hfill \textbf{\thepage}}}

\vspace*{-60pt}

{\small

\noindent
Journal ``Informatics and Applications'' (Inform.\ Appl.)
publishes theoretical, review, and discussion
articles on the research and development in the
field of informatics and its applications.

The journal is published in Russian.
By a special decision of the editorial
board, some articles can be published in English.


The topics covered include the following areas:
\begin{itemize}
               \item
     theoretical fundamentals of informatics; \\[-14pt]
\item
mathematical methods for studying complex systems and processes; \\[-14pt]
\item
information systems and networks;\\[-14pt]
\item
information technologies; and \\[-14pt]
\item
architecture and software of computational complexes and networks. \\[-14pt]
\end{itemize}

\noindent
\begin{enumerate}[1.]
\item The Journal publishes original articles which have not been published before and are not
intended for simultaneous publication in other editions. An article submitted to the Journal must not violate the
Copyright law. Sending the manuscript to the Editorial Board, the authors retain all rights of the
owners of the manuscript and transfer the nonexclusive rights to publish the article in Russian
(or the language of the article, if not Russian) and its distribution in Russia and abroad to the
Founders and the Editorial Board. Authors should submit a letter to the Editorial Board in the
following form:

{\bfseries\textit{Agreement on the transfer of rights to publish:}}

``\textit{We, the undersigned authors of the manuscript ``\ldots'', pass to the
Founder and the Editorial Board of the Journal ``Informatics and Applications''
the nonexclusive right to publish the manuscript of the article in Russian (or
in English) in both print and electronic versions of the Journal. We affirm
that this publication does not violate the Copyright of other persons or
organizations.}

\textit{Author(s) signature(s): (name(s), address(es), date).}

This agreement should be submitted in paper form or in the form of a scanned copy (signed by
the authors).


%The Editorial Board has the right to request from the authors an official expert conclusion that
%the submitted article has no secret data prohibited for publication. \\[-13.5pt]
\item
A submitted article should be attached with \textbf{the data on the author(s)} (see item~8). If
there are several authors, the contact person should be indicated who is responsible for
correspondence with the Editorial Board and other authors about revisions and final approval
of the proofs.\\[-13.5pt]

\item The Editorial Board of the Journal examines the article according to the established
reviewing procedure. If the authors receive their article for correction after reviewing, it does not
mean that the article is approved for publication. The corrected article should be sent to the
Editorial Board for the subsequent review and approval.\\[-13.5pt]

\item The decision on the article publication or its rejection is communicated to the authors. The
Editorial Board may also send the reviews on the submitted articles to the authors. Any
discussion upon the rejected articles is not possible.\\[-13.5pt]

\item The edited articles will be sent to the authors for proofread. The comments of the authors
to the edited text of the article should be sent to the Editorial Board as soon as possible.\\[-13.5pt]

\item The manuscript of the article should be presented electronically in the MS WORD (.doc or
.docx) or \LaTeX\ (.tex) formats, and additionally in the .pdf format. All documents
 may be sent
by e-mail or provided on a CD or diskette. A~hard copy submission is not necessary.\\[-13.5pt]

\item The recommended typesetting instructions for manuscript.

Pages parameters: format A4, portrait orientation, document margins (cm): left~--- 2.5, right~---
1.5, above~--- 2.0, below~--- 2.0, footer 1.3.

Text: font~---Times New Roman, font size~--- 14, paragraph indent~--- 0.5, line spacing~--- 1.5,
justified alignment.

The recommended manuscript size: not more than 15~pages of the specified format.
If the specified size exceeded, the editorial board is entitled to require the author
to reduce the manuscript.

Use only standard abbreviations. Avoid  abbreviations in the title and
abstract. The full term for which an abbreviation stands should precede
its first use in the text unless it is a standard unit of measurement.

All pages of the manuscript should be numbered.

The templates for the manuscript typesetting are presented on site: {\sf
http://www.ipiran.ru/journal/template.doc}.\\[-13.5pt]


%\def\leftkol{Requirements for manuscripts submitted to Journal
%``Informatics~and~Applications''}

\item The articles should enclose data both in \textbf{Russian and English}:
\begin{itemize}
\item title;\\[-13.5pt]
\item author's name and surname;\\[-13.5pt]
\item affiliation~--- organization, its address with ZIP code, city, country, and
official e-mail address;\\[-13.5pt]
\item data on authors according to the format: (see site)

{\sf http://www.ipiran.ru/journal/issues/2013\_07\_01/authors.asp}  and

{\sf  http://www.ipiran.ru/journal/issues/2013\_07\_01\_eng/authors.asp};\\[-13.5pt]

\pagebreak

\def\leftfootline{\small{\textbf{\thepage}
\hfill INFORMATIKA I EE PRIMENENIYA~--- INFORMATICS AND APPLICATIONS\ \ \ 2019\
\ \ volume~13\ \ \ issue\ 4}
}%
 \def\rightfootline{\small{INFORMATIKA I EE PRIMENENIYA~--- INFORMATICS AND APPLICATIONS\ \ \ 2019\ \ \ volume~13\ \ \ issue\ 4
\hfill \textbf{\thepage}}}


%\def\leftkol{Requirements for manuscripts submitted to Journal
%``Informatics~and~Applications''}

%\def\rightkol{Requirements for manuscripts submitted to Journal
%``Informatics~and~Applications''}



\item abstract (not less than 100 words) both in Russian and in English. Abstract is a short
summary of the article that can be published separately. The abstract is the
main source of information on the article and it could be included in leading information
systems and data bases. The abstract in English has to be an original text and should
not be an exact translation of the Russian one. Good English is required.
In abstracts, avoid references and formulae;\\[-13.5pt]
\item indexing is performed on the basis of keywords. The use of keywords from the
internationally accepted thematic Thesauri is recommended.

%\def\leftkol{Requirements for manuscripts submitted to Journal
%``Informatics~and~Applications''}

%\def\rightkol{Requirements for manuscripts submitted to Journal
%``Informatics~and~Applications''}

Important! Keywords must not be sentences;
\item Acknowledgments.
\end{itemize}

\item References. Russian references have to be presented both in English translation and Latin
transliteration (refer {\sf http://www.translit.net/ru/bgn/}).

Please take into account the following examples of Russian references appearance:

\noindent
\textbf{Article in journal:}

\Aue{Zhang, Z., and D.~Zhu}. 2008. Experimental research on the localized electrochemical
micromachining.
\textit{Rus. J.~Electrochem.}  44(8):926--930. {\sf doi:10.1134/S1023193508080077}.


\noindent
\textbf{Journal article in electronic format:}

\Aue{Swaminathan, V., E.~Lepkoswka-White, and B.\,P.~Rao}. 1999. Browsers or buyers in
cyberspace? An
investigation of electronic factors influencing electronic exchange. \textit{JCMC}
5(2). Available at: {\sf http://www.ascusc.org/jcmc/vol5/issue2/} (accessed April~28, 2011).




\noindent
\textbf{Article from the continuing publication (collection of works, proceedings):}

\Aue{Astakhov, M.\,V., and T.\,V.~Tagantsev}. 2006. Eksperimental'noe
issledovanie prochnosti soedineniy ``stal'--kompozit'' [Experimental study of
the strength of joints ``steel--composite'']. \textit{Trudy MGTU
``Matematicheskoe modelirovanie slozhnykh tekh\-ni\-che\-skikh sistem''}
[\textit{Bauman MSTU ``Mathematical Modeling of Complex Technical
Systems'' Proceedings}]. 593:125--130.

\def\leftfootline{\small{\textbf{\thepage}
\hfill INFORMATIKA I EE PRIMENENIYA~--- INFORMATICS AND APPLICATIONS\ \ \ 2019\
\ \ volume~13\ \ \ issue\ 4}
}%
 \def\rightfootline{\small{INFORMATIKA I EE PRIMENENIYA~--- INFORMATICS AND APPLICATIONS\ \ \ 2019\ \ \ volume~13\ \ \ issue\ 4
\hfill \textbf{\thepage}}}

\def\leftkol{Requirements for manuscripts submitted to Journal
``Informatics~and~Applications''}

\def\rightkol{Requirements for manuscripts submitted to Journal
``Informatics~and~Applications''}

\noindent
\textbf{Conference proceedings:}

\Aue{Usmanov, T.\,S., A.\,A.~Gusmanov, I.\,Z.~Mullagalin, R.\,Ju.~Muhametshina,
A.\,N.~Chervyakova, and
A.\,V.~Sveshnikov}. 2007. Osobennosti proektirovaniya razrabotki mestorozhdeniy
s primeneniem gidrorazryva
plasta [Features of the design of field development with the use of hydraulic fracturing].
\textit{Trudy 6-go
Mezhdu\-na\-rod\-no\-go Simpoziuma ``Novye resursosberegayushchie tekhnologii
nedropol'zovaniya i povysheniya
neftegazootdachi''} [\textit{6th  Symposium (International) ``New Energy Saving Subsoil
Technologies and
the Increasing of the Oil and Gas Impact'' Proceedings}]. Moscow. 267--272.


\noindent
\textbf{Books and other monographs:}




Lindorf, L.\,S., and L.\,G.~Mamikoniants, eds. 1972.
\textit{Ekspluatatsiya turbogeneratorov s neposredstvennym
okhlazhdeniem} [\textit{Operation of turbine generators with direct cooling}].
Moscow: Energy Publs. 352~p.


%\Aue{Latyshev, V.\,N.} 2009. \textit{Tribologiya rezaniya. Kn.~1: Frikcionnye prosessy
%pri rezanii metallov}
%[\textit{Tribology of cutting. Vol.~1: Frictional processes in metal cutting}]. Ivanovo: Ivanovskii
%State Univ. 108~p.


%\noindent
%\textbf{Unpublished material:}

%\Aue{Latypov, A.\,R., M.\,M.~Khasanov, and V.\,A.~Baikov}.
%2004. Geology and production (NGT GiD). Certificate on official registration of the computer
%program
%No.\,2004611198. (In Russian, unpubl.)

%\noindent
%\textbf{Internet-source:}

%APA Style. 2011. Available at: {\sf http://www.apastyle.org/apa-style-help.aspx} (accessed
%February~5, 2011).

%Pravila citirovaniya istochnikov [Rules for the citing of sources]. Available at: {\sf
%http://www.scribd.com/doc/1034528/} (accessed February~7, 2011).


\noindent
\textbf{Dissertation and Thesis:}

%\Aue{Semenov, V.\,I.}
%2003. Matematicheskoe modelirovanie plazmy v sisteme kompaktnyy tor. [Mathematical
%modeling of the plasma in the compact torus]. D.Sc.\ Diss. Moscow. 272~p.

\Aue{Kozhunova, O.\,S.} 2009. Tekhnologiya razrabotki semanticheskogo
slovarya informatsionnogo monitoringa [Technology of development of
semantic dictionary of information monitoring system]. PhD Thesis. Moscow: IPI RAN. 23~p.


\noindent
\textbf{State standards and patents:}

GOST 8.586.5-2005. 2007. Metodika vypolneniya izmereniy. Izmerenie raskhoda i~kolichestva
zhidkostey i gazov 
s~pomoshch'yu standartnykh suzhayushchikh ustroystv [Method of measurement.
Measurement of flow rate and volume of liquids and gases by means of orifice devices]. M.:
Standardinform
Publs. 10~p.

%\noindent
%\textbf{Patent:}

\Aue{Bolshakov, M.\,V., A.\,V.~Kulakov, A.\,N.~Lavrenov, and M.\,V.~Palkin}.
2006. Sposob orientirovaniya po krenu letatel'nogo
apparata s opti\-che\-skoy golovkoy
samonavedeniya [The way to orient on the roll of aircraft with optical homing head].
Patent RF No.\,2280590.

References in Latin transcription are presented in the original language.

References in the text are numbered according to the order of their
first appearance; the number is
placed in square brackets. All items from the reference list should be
cited.\\[-13.5pt]

\item Manuscripts and additional materials are not returned to Authors by the Editorial Board.\\[-13.5pt]

\item Submissions of files by e-mail must include:\\[-13.5pt]
\begin{itemize}
\item   the journal title and author's name in the ``Subject'' field; \\[-13.5pt]
\item   an article and additional materials have to be attached using the ``attach'' function;\\[-13.5pt]
\item   an electronic version of the article should contain the file with the text and a separate file
with figures.\\[-13.5pt]
\end{itemize}

\item ``Informatics and Applications'' journal is not a profit publication. There are no
charges for the authors as well as there are no royalties.\\[-13.5pt]
\end{enumerate}

\def\leftfootline{\small{\textbf{\thepage}
\hfill INFORMATIKA I EE PRIMENENIYA~--- INFORMATICS AND APPLICATIONS\ \ \ 2019\
\ \ volume~13\ \ \ issue\ 4}
}%
 \def\rightfootline{\small{INFORMATIKA I EE PRIMENENIYA~--- INFORMATICS AND APPLICATIONS\ \ \ 2019\ \ \ volume~13\ \ \ issue\ 4
\hfill \textbf{\thepage}}}

\def\leftkol{Requirements for manuscripts submitted to Journal
``Informatics~and~Applications''}

\def\rightkol{Requirements for manuscripts submitted to Journal
``Informatics~and~Applications''}


%\vspace*{5mm}


\begin{center}
\textbf{Editorial Board address:} \\

%ABOUT AUTHORS



FRC CSC RAS, 44, block~2, Vavilov Str., Moscow 119333, Russia\\[-10pt]

\

Ph.: +7\,(499)\,135\,86\,92,\ \ Fax: +7\,(495)\,930\,45\,05\\[-10pt]

\

 e-mail: {\sf rust@ipiran.ru} (to Prof.\ Rustem Seyful-Mulyukov)\\[-10pt]

\

 {\sf http://www.ipiran.ru/english/journal.asp}
\end{center}
 }
%\thispagestyle{myheadings}

\def\leftkol{Requirements for manuscripts submitted to Journal
``Informatics~and~Applications''}

\def\rightkol{Requirements for manuscripts submitted to Journal
``Informatics~and~Applications''}

\def\leftfootline{\small{\textbf{\thepage}
\hfill INFORMATIKA I EE PRIMENENIYA~--- INFORMATICS AND APPLICATIONS\ \ \ 2019\
\ \ volume~13\ \ \ issue\ 4}
}%
 \def\rightfootline{\small{INFORMATIKA I EE PRIMENENIYA~--- INFORMATICS AND APPLICATIONS\ \ \ 2019\ \ \ volume~13\ \ \ issue\ 4
\hfill \textbf{\thepage}}}

 \label{end\stat}

\newpage



%\include{ipi-ind}

%\tableofcontents

\end{document}





%%%%%%%%%%%%%%%%%%%%%%

%\newcommand{\Ack}{\subsection*{\protect\large\bf Acknowledgments}}

%\vphantom*{\int\limits_0^T}

{ \begin{center}  %fig1
 \vspace*{6pt}
    \mbox{%
 \epsfxsize=79mm 
 \epsfbox{gru-1.eps}
 }

\end{center}



\noindent
{{\figurename~1}\ \ \small{
}}}

%\vspace*{6pt}

\addtocounter{figure}{1}