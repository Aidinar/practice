\def\stat{nuriev}

\def\tit{МЕТОДОЛОГИЯ КОРПУСНО-ОРИЕНТИРОВАННОГО ИССЛЕДОВАНИЯ 
В~ОБЛАСТИ КОНТРАСТИВНОЙ ПУНКТУАЦИИ$^*$\\[-5pt]}

\def\titkol{Методология корпусно-ориентированного исследования 
в~области контрастивной пунктуации}

\def\aut{В.\,А.~Нуриев$^1$, В.\,И.~Карпов$^2$}

\def\autkol{В.\,А.~Нуриев, В.\,И.~Карпов}

\titel{\tit}{\aut}{\autkol}{\titkol}

\index{Нуриев В.\,А.}
\index{Карпов В.\,И.}
\index{Nuriev V.\,A.}
\index{Karpov V.\,I.}


{\renewcommand{\thefootnote}{\fnsymbol{footnote}} \footnotetext[1]
{Работа выполнена за счет гранта Российского научного фонда (проект 23-28-00548) с~использованием инфраструктуры 
Центра коллективного пользования <<Высокопроизводительные вычисления и~большие данные>> (ЦКП 
<<Информатика>>) ФИЦ ИУ РАН (г.~Москва).}}


\renewcommand{\thefootnote}{\arabic{footnote}}
\footnotetext[1]{Федеральный исследовательский центр <<Информатика и~управление>> Российской академии наук, 
\mbox{nurieff.v@gmail.com}}
\footnotetext[2]{Институт языкознания Российской академии наук; Федеральный исследовательский центр <<Информатика 
и~управ\-ле\-ние>> Российской академии наук, \mbox{wi.karpow@gmail.com}}

\vspace*{-3pt}

  
  
    
  \Abst{Уточняется  подход к~современным исследованиям 
в~об\-ласти контрастивной пунктуации с~точки зрения методологии. С~учетом новейших достижений информатики, 
компьютерной лингвистики и~теории перевода такие исследования очевидным образом 
должны иметь кор\-пус\-но-ори\-ен\-ти\-ро\-ван\-ный характер. В~данной статье представлена 
методологическая схема подобного исследования, направленного на выявление 
межъязыковой пунктуационной асим\-мет\-рии посредством сравнения функционального 
диапазона одного и~того же знака препинания в~разных языках. Показываются основные 
методологические тенденции, характерные для этой научной об\-ласти. Внимание 
уделяется особенностям корпусной методологии при контрастивном изучении 
пунктуации. В~качестве одного из современных методологических инструментов 
предлагаются надкорпусные базы данных (НБД), раз\-ра\-ба\-ты\-ва\-емые в~ФИЦ ИУ РАН.}

%\vspace*{-6pt}
  
  \KW{контрастивная пунктуация; перевод; корпусное переводоведение; кор\-пус\-но-ори\-ен\-ти\-ро\-ван\-ное 
  исследование; параллельный корпус; надкорпусная база данных; 
межъязыковая асим\-мет\-рия; методология}

%\vspace*{-6pt}

\DOI{10.14357/19922264230213}{VBOZAO} 
  
%\vspace*{-3pt}


\vskip 10pt plus 9pt minus 6pt

\thispagestyle{headings}

\begin{multicols}{2}

\label{st\stat}
    
    \section{Введение}
    
    \vspace*{-3pt}
    
  Важность и~необходимость исследований в~области контрастивной 
пунктуации в~научной литературе отмечалась неоднократно (см., 
например,~[1--7]). Обычно эта необходимость выводится из нужд 
переводческой практики, которая предполагает при обработке письменного 
текста обязательную речемыслительную программу, связанную с~исходным 
пунктуационным компонентом и~его переносом в~сис\-те\-му переводящего 
языка. Так, Ньюмарк в~своем <<Учебнике перевода>> пишет, что 
<<пунктуация может быть мощнейшим инструментом, но ее настолько легко 
упус\-тить из виду, что я~советую переводчикам: специально сравнивайте, где 
у~вас рас\-став\-ле\-ны знаки препинания, а~где они стоят 
в~оригинале>>~\cite[с.~58]{4-nu}. В~работе <<Переводчик в~текс\-те: 
о~чтении русской литературы  
по-анг\-лий\-ски>> значение пунктуации отмечает Мей, критикуя 
англоязычных переводчиков за недостаточное внимание к~межъязыковой 
пунктуационной асим\-мет\-рии~--- за <<игнорирование отличительных 
особенностей, присущих знакам препинания>>~\cite[с.~121]{2-nu}. 
О~пунктуации в~переводе говорит Юдейл, выделяя три аспекта:
%\begin{enumerate}[(1)]
%\item 
(1)~<<знаки препинания~--- важ\-ная часть перевода, но, концентрируясь на 
общем смыс\-ле переводимого, ее час\-то не замечают>>; 
%\item 
(2)~<<изменения 
в~пунктуации при переводе могут значительно по\-вли\-ять на вы\-ра\-зи\-тель\-ность 
текс\-та, его свя\-зан\-ность и~ритм>>; 
%\item 
(3)~<<час\-то возникает впечатление, что 
литературные переводчики наделили себя правом менять границы исходного 
предложения и~пунктуационные знаки, как им 
заблагорассудится>>~\cite[с.~121]{5-nu}.
%\end{enumerate}
 Гораздо реже 
в~специализированной литературе подчеркивается роль, которую 
исследования в~об\-ласти контрастивной пунктуации играют при обучении 
иностранным языкам, в~част\-ности при обуче\-нии иноязычной письменной 
речи~\cite{7-nu}.
  
  Признавая безусловную зна\-чи\-мость данного научного на\-прав\-ле\-ния и~его 
дальнейшего развития, необходимо предметно разрабатывать методологию 
исследования в~об\-ласти контрастивной пунктуации, которая учитывала бы 
новейшие достижения информатики, компьютерной лингвистики 
и~корпусного переводоведения. Пред\-став\-ля\-ет\-ся, что такая методология 
долж\-на основываться на использовании современных информационных 
корпусных инструментов, поз\-во\-ля\-ющих автоматизированным образом 
обрабатывать пред\-ста\-ви\-тель\-ные массивы текс\-то\-вых данных, 
и,~следовательно, носить кор\-пус\-но-ори\-ен\-ти\-ро\-ван\-ный характер 
(о~корпусных данных при контрастивном изуче\-нии пунктуации  
см.~\cite{6-nu}).

%\vspace*{-6pt}
    
    \section{Методологические модели  
корпусно-ориентированного исследования контрастивной 
пунктуации}

\vspace*{-3pt}
  
  В мае 2019~г.\ в~Регенсбурге (Германия) про\-шла научная конференция 
под названием <<Punctuation Seen Internationally. System--Norm--Practice>> 
(<<Пунктуация в~мировом мас\-шта\-бе: 
 сис\-те\-ма--нор\-ма--прак\-ти\-ка>>)~--- первая конференция, пол\-ностью\linebreak 
по\-свя\-щен\-ная проб\-ле\-мам контрастивной пунктуации. Оргкомитет, собирая 
заявки на участие, справедливо отмечал, что до на\-сто\-яще\-го времени 
пунктуации едва ли уделялось внимание в~рамках \mbox{типологии}, контрастивной 
лингвистики, прагмалингвистики, а~так\-же в~исследованиях индивидуальной 
языковой манеры на фоне языкового стандарта. Сейчас появляются 
отдельные работы, где проводится сопоставительное изуче\-ние пунктуации, 
однако по-преж\-не\-му ощущается острая не\-об\-хо\-ди\-мость в~исследованиях по 
контрастивной пунктуации, которые бы учитывали типологические 
(сис\-тем\-ные), социолингвистические (нормативные) и~прагматические 
(речевые) ас\-пекты.
  
  Итогом конференции стала коллективная монография~\cite{8-nu}, 
со\-сто\-ящая из шестнадцати статей, которые пред\-став\-ля\-ют собой пио\-нер\-ские 
исследования, на\-прав\-лен\-ные на формирование целостной па\-ра\-диг\-мы 
контрастивного изучения пунктуации и~борьбу с~маргинализацией важ\-ной 
научной от\-расли. Все статьи услов\-но мож\-но разделить на~4~категории, 
первые две из которых имеют в~большей степени тео\-ре\-ти\-че\-ский характер и~связаны с~сис\-те\-мой и~нормой, а~вторые~--- более практической 
на\-прав\-лен\-ности~--- с~узусом и~освоением пунктуационных навыков. 
В~пред\-став\-лен\-ных работах доминируют два подхода к~исследованию 
конт\-растив\-ной пунк\-ту\-ации:
  \begin{enumerate}[(1)]
\item интралингвистический (контрастивный анализ знаков препинания 
и~кон\-ку\-ри\-ру\-ющих с~ними маркеров синтаксических отношений в~рамках 
одного языка)~\cite[с.~110]{9-nu};
  \item  интерлингвистический (контрастивный анализ знаков препинания 
  и~конкурирующих с~ними средств в~разных языках, конт\-растив\-ная пунктуация 
рас\-смат\-ри\-ва\-ет\-ся в~том чис\-ле как часть методики обуче\-ния неродному языку, 
например при интеграции трудовых мигрантов в~иноязычную 
среду)~\cite[с.~57--73]{10-nu}.
  \end{enumerate}
  
  Интралингвистический подход час\-то носит смешанный характер: если 
речь идет об эволюции пунктуационной сис\-те\-мы отдельно взятого языка на 
фоне развития аналогичных сис\-тем других языков, контрастивный анализ 
со\-про\-вож\-да\-ет\-ся 
 ис\-то\-ри\-ко-эти\-мо\-ло\-ги\-че\-ским~\cite[с.~187--206]{11-nu}. В~рамках 
этого подхода в~указанной монографии имеются психолингвистические 
исследования с~нетривиальным корпусным материалом. Так, 
в~статье~\cite[с.~163--186]{12-nu} корпусные данные привлекаются для 
контрастивного анализа пунктуационных предпочтений двух групп 
ис\-пы\-ту\-емых. Автор использует корпус \mbox{CoPaDocs} (Corpus of Patient 
Documents), основу которого со\-ста\-ви\-ли письма и~другие личные документы 
бывших пациентов психиатрических учреж\-де\-ний Германии на рубеже  
XIX--XX~вв. Корпус поз\-во\-ля\-ет установить, зависит ли языковое оформление 
пись\-ма от лич\-ности адресата~--- происходит ли переключение ре\-гист\-ров 
сознательно. Данный корпус создан с~целью разработки интегративной 
методики анализа языковой ва\-риа\-тивн\-ости, в~том чис\-ле и~в~об\-ласти 
пунктуации. \mbox{Изучив} специфику расстановки~12~знаков препинания, 
Эбер-Хам\-мерль приходит к~выводу, что пациенты, чей род де\-я\-тель\-ности 
прежде не был связан с~письменной сферой, использовали больше 
пунктуационных маркеров (но с~меньшей ва\-риа\-тив\-ностью), чем 
представители второй опытной группы~--- канцелярские служащие. 
В~лич\-ной переписке участники обеих групп к~знакам препинания прибегали 
гораздо реже, чем в~документах, адресованных официальным лицам.
  
 В статье~\cite[с.~57--73]{10-nu} представлено контрастивное исследование, выполненное в~интерлигвистическом 
ключе. Со\-по\-став\-ле\-ние 
пунктуации в~италь\-ян\-ском и~немецком языках здесь проводится на основе 
комплексной методологии, вклю\-ча\-ющей приемы дескриптивного, 
просодического, синтаксического и~ком\-му\-ни\-ка\-тив\-но-текс\-то\-во\-го 
анализа. Примеры приводятся из различных источников, причем 
к~корпусным данным в~статье отсылают не напрямую, а~опосредованно~--- 
через более раннюю работу~\cite{13-nu}. По мнению авторов, 
пунктуирование в~этих языках организовано по-раз\-но\-му, что объясняется 
резкими различиями в~пунктуационном узусе: если в~итальянском знаки 
препинания коммуникативно на\-гру\-же\-ны, то в~немецком они подчинены 
фор\-маль\-но-син\-так\-си\-че\-ско\-му принципу. Иначе говоря, итальянская 
пунктуация выполняет не формальную функцию, а~сигнализиру-\linebreak ет о~тон\-ких 
смыс\-ло\-вых нюансах, которых нельзя\linebreak достичь другими языковыми 
средствами (аргументативный конфликт, полифонические эффекты, 
метатекстовые комментарии). В~этом же духе\linebreak выполнена и~другая 
интерлингвистическая работа~\cite{14-nu}, по\-свя\-щен\-ная контрастивному 
исследованию многоточия и~тире в~италь\-ян\-ском и~анг\-лий\-ском языках 
и~продуктивно ис\-поль\-зу\-ющая \mbox{корпусный} метод сбора и~обработки 
эмпирических данных.
     
     Объединенные в~коллективную монографию рабо\-ты позволяют 
вывести обобщенную ме\-то\-до\-ло\-гическую схему контрастивного изуче\-ния 
пунктуации. Она имеет трехфазную структуру. Первая\linebreak фаза включает 
тео\-ре\-ти\-че\-ское описание пунктуации в~изуча\-емом языке с~привлечением 
исторических и~современных нормативных грамматик и~справочников. 
Вторая фаза на\-прав\-ле\-на на описание трансформаций в~других языках, 
оказавших существенное влияние на статус и~мес\-то пунктуации в~сис\-те\-ме 
конкретного языка. Обе фазы нацелены на создание такого 
исследовательского поля, которое поз\-во\-лит выявить значение пунктуации 
для языковой культуры. Это, в~свою очередь, долж\-но стать задачей треть\-ей 
фазы. Вторая и~\mbox{третья} фазы предполагают межъязыковое сравнение как 
функционального диапазона отдельно взятых знаков препинания, так 
и~пунктуационного репертуара в~целом. На этих стадиях применяется 
корпусный метод. Контрастивный анализ в~за\-ви\-си\-мости от по\-став\-лен\-ных 
целей и~задач наряду со знаками препинания может охватывать 
и~кон\-ку\-ри\-ру\-ющие с~ними языковые средства. На\-прав\-ле\-ние контрастивного 
исследования пунктуации может быть и~синхронным, и~диахроническим.

\vspace*{-6pt}
    
    \section{Методологические особенности  
корпусно-ориентированного исследования в~области 
контрастивной пунктуации}

\vspace*{-3pt}
  
  Особенности методологии при корпусном контрастивном изуче\-нии 
пунктуации, как, впрочем, и~при любом 
 кор\-пус\-но-ори\-ен\-ти\-ро\-ван\-ном исследовании, связаны прежде всего 
со стремлением получить непротиворечивые, валидные и~на\-деж\-ные данные. 
Электронный корпус, будучи методологически новаторским инструментом 
для получения научного знания, поз\-во\-ля\-ет, с~одной стороны, автоматическим 
образом обрабатывать большие массивы данных и~тем самым серьезно 
сокращает временные издержки на поиск эмпирического материала. 
С~другой стороны, электронные корпусные ресурсы имеют свои 
особенности, и~без их над\-ле\-жа\-ще\-го учета пользователь рискует получить 
искаженные результаты.
  
  Например, в~указанной выше работе~\cite[с.~291]{14-nu} авторы, описывая 
методологию своего исследования, отмечают, что итальянские примеры 
заимствованы из корпуса, хранящегося в~Базельском университете 
и~со\-сто\-яще\-го из двух частей~--- 33~современных  
ро\-ма\-на-бест\-сел\-ле\-ра (1~млн словоупотреблений) 
и~нехудожественных текс\-та разной на\-прав\-лен\-ности (1~млн 40~тыс.\ 
словоупотреблений), в~то время как англоязычные примеры извлечены из 
подкорпуса <<Книги и~периодические издания>> Британского 
национального корпуса (80~млн словоупотреблений). Итальянский материал, 
по словам авторов, был проанализирован весь, а~для английского из-за 
гораздо большего объема ограничились анализом случайной выборки, объем 
которой со\-по\-ста\-вим с~выборкой из итальянского корпуса. Очевидным 
образом ва\-лид\-ность выводов по результатам анализа англоязычного 
материала здесь может оказаться под вопросом в~силу методологически 
неоднородных установок применительно к~процедуре обработки данных, 
полученных по двум языкам. Примечательно к~тому же, что базельский 
корпус, в~отличие от британского, за\-крыт для общественного пользования.
  
  О подобных ограничениях рассуждает На\-двор\-ни\-ко\-ва в~своей работе, 
по\-свя\-щен\-ной корпусной методологии контрастивного изучения 
пунктуации~\cite{15-nu}, где анализируется час\-тот\-ность упо\-треб\-ле\-ния шести 
знаков препинания (запятой, точки, двоеточия, точ\-ки с~запятой, 
вопросительного и~восклицательного знака) в~английском, французском 
и~чешском языках. Для сбора данных используются со\-по\-ста\-ви\-мые 
веб-кор\-пу\-сы, моноязычные общие (референтные) и~параллельные корпусы. Цель 
автора~--- определить, какой из трех типов корпусных ресурсов наиболее 
подходит для исследований в~об\-ласти контрастивной пунктуации.
  
  Полученные данные показывают, что при изучении пунктуации показатели 
час\-тот\-ности проявляют высокую чув\-ст\-ви\-тель\-ность к~типу текс\-та; 
следовательно, веб-кор\-пу\-сы, которые, как правило, отличают стихийное 
наполнение, не\-упо\-ря\-до\-чен\-ность и~низ\-кая степень струк\-ту\-ри\-ро\-ван\-ности, не 
могут служить источником до\-сто\-вер\-ной информации об упо\-треб\-ле\-нии 
знаков препинания в~том или ином языке. Моноязычный общий корпус, 
наоборот, содержит специальную раз\-мет\-ку (морфологическую, 
синтаксическую и~т.\,д.)\ и~поз\-во\-ля\-ет гиб\-ко настраивать поиск (в~том чис\-ле 
выбирать соответствующий тип текс\-та) в~за\-ви\-си\-мости от конкретных 
исследовательских задач. Такие корпусы располагают большими массивами 
данных, поскольку призваны пред\-ста\-вить язык во всей его пол\-но\-те 
и~многообразии, что, казалось бы, обеспечивает на\-деж\-ность и~ва\-лид\-ность 
полученных результатов. Меж\-ду тем этот тип корпусов имеет существенный 
недостаток~--- ограниченную межъязыковую со\-по\-ста\-ви\-мость. Как правило, 
моноязычные общие корпусы разных языков разительно отличаются по 
объему данных и~их со\-ста\-ву и~поэтому не подходят в~качестве основного 
инструмента контрастивного исследования, а~могут служить лишь 
референтным (проверочным) источником для дополнительной верификации 
ре\-зуль\-ти\-ру\-ющих данных. Кроме того, со\-по\-ста\-ви\-тель\-ный анализ 
относительной час\-тот\-ности упо\-треб\-ле\-ния знаков препинания в~разных 
языках на основе данных, извлеченных из корпусов этого типа, так\-же имеет 
свои ограничения. Он не применим для изучения пунк\-ту\-а\-ции в~языках 
разного строя, которым для кодирования информации требуется 
количественно больше (аналитические языки типа французского) или 
меньше слов (синтетические языки типа русского). Таким образом, лучше 
всего для контрастивного изуче\-ния пунк\-ту\-а\-ции подходят параллельные 
корпусы, которые, не\-смот\-ря на свой сравнительно небольшой объем, 
пред\-став\-ля\-ют существенно больше воз\-мож\-но\-стей для качественного анализа 
упо\-треб\-ле\-ния знаков препинания и~непосредственного со\-по\-став\-ле\-ния их 
абсолютной час\-тот\-ности в~параллельных текс\-тах~--- оригинале и~переводе. 
Однако и~этот тип информационного ресурса не может служить 
универсальным исследовательским инструментом. При его использовании 
необходимо учитывать, что пунктуационные рас\-хож\-де\-ния в~исходном 
и~переводном текс\-те могут быть не результатом сис\-тем\-ных дифференциаций, 
а~возникнуть под влиянием переводческих предпочтений. Следовательно, 
чтобы избежать искажения ре\-зуль\-ти\-ру\-ющих данных, надо следовать 
некоторым методологическим принципам: %\\[-13pt] 
\begin{enumerate}[(1)]
\item данные собираются в~обоих 
переводных на\-прав\-ле\-ни\-ях; %\\[-13pt] 
\item выявленные тенденции проходят 
обязательную проверку с~по\-мощью референтного моноязычного корпуса; %\\[-13pt] 
\item контрастивное изуче\-ние пунктуации с~применением параллельных 
корпусов требует сис\-тем\-но\-го подхода в~том смыс\-ле, что в~функциональном 
диапазоне разных знаков препинания могут быть общие зоны, ука\-зы\-ва\-ющие 
на их потенциальную внут\-ри\-язы\-ко\-вую и~межъ\-язы\-ко\-вую конкуренцию. %\\[-13pt]
\end{enumerate}
    
 \vspace*{-12pt}
 
    \section{Заключение}
    
    \vspace*{-3pt}
    
  В статье представлена обобщенная методологическая схема 
  кор\-пус\-но-ори\-ен\-ти\-ро\-ван\-но\-го 
  исследования в~об\-ласти контрастивной пунктуации~--- 
от\-расли научного знания, интенсивно \mbox{раз\-ви\-ва\-ющей\-ся} и~при\-вле\-ка\-ющей 
внимание специалистов самого широкого профиля. Несмотря на то что 
появляются работы, где описываются сопоставительные исследования 
пунктуации на примере одного произведения или литературного наследия 
отдельно взятого писателя (см., например,~\cite{16-nu,17-nu}), очевидно, что 
для ка\-ких-ли\-бо существенных, круп\-но\-мас\-штаб\-ных обобщений относительно 
межъязыковой пунктуационной асимметрии и~специфики функционирования 
знаков препинания в~разных языках требуется привлечение корпусного 
материала.
  
  Дальнейшее изучение контрастивной пунктуации видится в~нескольких 
направлениях. Необходимо качественное углубление со\-по\-ста\-ви\-тель\-но\-го 
анализа, чтобы его тонкая нюансировка \mbox{поз\-во\-ли\-ла} установить, в~какой мере 
совпадает и~разнится функциональный диапазон того или иного знака 
препинания в~кон\-так\-ти\-ру\-ющих языках в~за\-ви\-си\-мости от жанровой 
при\-над\-леж\-ности текс\-та. Этот анализ целесообразно проводить комплексно, 
охватывая всю со\-во\-куп\-ность синтаксических изменений, которые влекут за 
собой отказ от исходного пунктуирования при переводе с~одного языка на 
другой. Такая ком\-плекс\-ность поможет выявить и~с~большей пол\-но\-той 
описать су\-щест\-ву\-ющие межъ\-язы\-ко\-вые структурные различия, что 
необходимо и~для переводческой практики, и~для обуче\-ния иностранным 
языкам. Требует дальнейшего уточ\-не\-ния вопрос, как на пунктуационные 
преференции переводчика влияет род\-ная языковая культура, 
пунктуационные уста\-нов\-ки которой могут меняться со временем. По мере 
наращивания опыта и~мастерства могут меняться пунктуационные 
предпочтения и~самого переводчика, и~это так\-же пред\-став\-ля\-ет определенный 
научный интерес.
  
  В заключение следует отметить, что одним из современных 
информационных инструментов корпусного исследования в~об\-ласти 
контрастивной пунктуации могут быть НБД, 
раз\-ра\-ба\-ты\-ва\-емые в~отделе~54 Федерального исследовательского цент\-ра 
<<Информатика и~управ\-ле\-ние>> Российской академии наук (о~возможностях 
НБД см.~\cite{6-nu}). В~данный момент этот методологический инструмент 
проходит апро\-ба\-цию в~контрастивном исследовании двоеточия и~многоточия в~трех языках~--- русском, французском и~немецком.

\vspace*{-9pt}
  
{\small\frenchspacing
 {\baselineskip=11.5pt
 %\addcontentsline{toc}{section}{References}
 \begin{thebibliography}{99}
 
 \vspace*{-3pt}
 
 \bibitem{4-nu} %1
\Au{Newmark P.} A~textbook of translation.~--- New York, London, Toronto, Sydney, Tokyo: Prentice 
Hall, 1988. 402~p.
 

\bibitem{2-nu} %2
\Au{May R.} The translator in the text: On reading Russian literature in English.~--- Evanston, IL, USA: 
Northwestern University Press, 1994. 209 p.
\bibitem{3-nu}
\Au{Munday J.} Systems in translation: A~systemic model for descriptive translation studies~// 
Crosscultural transgressions: Research models in translation studies II~--- historical and 
ideological issues~/ Ed. T.~Hermans.~---  Manchester, U.K.: St.\ Jerome, 2002. P.~76--92.
\bibitem{1-nu} %4
\Au{Baker M.} In other words.~--- 2nd ed.~--- London, New York: Routledge, 2011. 352~p.

\bibitem{7-nu} %5
\Au{Сигал К.\,Я.} Контрастивная пунктуация в~начале XXI века~// Язык. Текст. Дискурс: 
Научный альманах Ставропольского отделения РАЛК.~--- Ставрополь: СКФУ, 
2019.  Вып.~17. С.~69--78.
\bibitem{5-nu} %6
\Au{Youdale R.} Using computers in the translation of literary style: Challenges and 
opportunities.~--- London, New York: Routledge, 2020. 242~p.
\bibitem{6-nu} %7
\Au{Нуриев В.\,А., Кружков~М.\,Г.} Корпусные данные при контрастивном изуче\-нии 
пунктуации~// Сис\-те\-мы и~средства информатики, 2023. Т.~33. №\,1. С.~14--23. doi: 10.14357/08696527230102.

\bibitem{8-nu}
Vergleichende Interpunktion~--- comparative punctuation~/ Eds. P.~R$\ddot{\mbox{o}}$ssler, P.~Besl, A.~Saller.~--- 
Berlin, Boston: De Gruyter, 2021. 454~p.
\bibitem{9-nu}
\Au{Rinas K.} Vom genormten Satzbau zur genormten Interpunktion. Zur Funktion der 
Zeichensetzung in $\ddot{\mbox{a}}$lterer und neuerer Zeit~// Vergleichende Interpunktion~--- comparative 
punctuation~/ Eds. P.~R$\ddot{\mbox{o}}$ssler, P.~Besl, A.~Saller.~---
 Berlin, Boston: De Gruyter, 2021. P.~109--136. doi: 10.1515/9783110756319-006.
\bibitem{10-nu}
\Au{Ferrari~A., Stojmenova Weber R.} Das Komma in kontrastiver Perspektive Italienisch-Deutsch~// Vergleichende Interpunktion~--- 
comparative punctuation / Eds. P.~R$\ddot{\mbox{o}}$ssler, P.~Besl, 
A.~Saller.~--- Berlin, Boston: De Gruyter, 2021. P.~57--73. doi: 10.1515/9783110756319-003.

\columnbreak

\bibitem{11-nu}
\Au{Besch W.} Zur Entwicklung der deutschen Interpunktion seit dem sp$\ddot{\mbox{a}}$ten Mittelalter~// 
Interpretation und Edition deutscher Texte des Mittelalters. Festschrift f$\ddot{\mbox{u}}$r John Asher zum 60. 
Geburtstag~/ Eds. K.~Smits, W.~Besch, V.~Lange.~--- Berlin: Erich Schmidt, 1981. P.~187--206.
\bibitem{12-nu}
\Au{Eber-Hammerl F.} Interpunktion in historischen Patientenbriefen // Vergleichende
Interpunktion~--- comparative punctuation~/ Eds. P.~R$\ddot{\mbox{o}}$ssler, 
P.~Besl, A.~Saller.~--- Berlin, Boston: De Gruyter, 2021. P.~163--186.
\bibitem{13-nu}
\Au{Ferrari A.} Leggere la virgola. Una prima ricognizione~// Chimera Romance Corpora 
Linguistic Studies, 2017. Vol.~4. Iss.~2. P.~145--162. doi: 
10.15366/chimera2017. 4.2.001.
\bibitem{14-nu}
\Au{Pecorari F., Longo~F.} The ellipsis and the dash in Italian and English: A~contrastive 
perspective~// Vergleichende Interpunktion~--- comparative punctuation~/ Eds.
 P.~R$\ddot{\mbox{o}}$ssler, P.~Besl, A.~Saller.~--- Berlin, Boston: De Gruyter, 2021. P.~289--314.
 doi: 10.1515/9783110756319-013.
\bibitem{15-nu}
\Au{N$\acute{\mbox{a}}$dvorn$\acute{{\iota}}$kov$\acute{\mbox{a}}$~O.}
The use of English, Czech and French punctuation marks in reference, 
parallel and comparable web corpora: A~question of methodology~// 
Linguist. Prag.,  2020. Vol.~30. Iss.~2. P.~30--50. doi: 
10.14712/ 18059635.2020.1.2.
\bibitem{16-nu}
\Au{Сигал К.\,Я.} Пунктуация как средство создания эмоционального под\-текс\-та (на 
материале рассказа М.\,А.~Шолохова <<Судьба человека>> и~его переводов на английский 
язык)~// Известия РАН. Серия литературы и~языка, 2014. Т.~73. №\,6. С.~38--50.
\bibitem{17-nu}
\Au{Богданов К.\,А.} Пунктуация как мотив: многоточие и~тире~// НЛО, 2022. №\,2(174). С.~241--253.
doi: 0.53953/ 08696365\_2022\_174\_2\_241.

\end{thebibliography}

 }
 }

\end{multicols}

\vspace*{-8pt}

\hfill{\small\textit{Поступила в~редакцию 15.04.23}}

\vspace*{6pt}

%\pagebreak

%\newpage

%\vspace*{-28pt}

\hrule

\vspace*{2pt}

\hrule

\vspace*{-2pt}

\def\tit{METHODOLOGY OF~THE~CORPUS-BASED STUDIES\\ 
IN~THE~FIELD OF~CONTRASTIVE PUNCTUATION}


\def\titkol{Methodology of~the~corpus-based studies 
in~the~field of~contrastive punctuation}


\def\aut{V.\,A.~Nuriev$^1$ and~V.\,I.~Karpov$^{1,2}$}

\def\autkol{V.\,A.~Nuriev and~V.\,I.~Karpov}

\titel{\tit}{\aut}{\autkol}{\titkol}

\vspace*{-14pt}


\noindent
      $^1$Federal Research Center ``Computer Science and Control'' of the Russian 
Academy of Sciences, 44-2~Vavilov\linebreak
$\hphantom{^1}$Str., Moscow 119333, Russian Federation
      
      \noindent
      $^2$Institute of Linguistics of the Russian Academy of Sciences, 1~bld.~1 
Bolshoy Kislovsky Lane, Moscow 125009,\linebreak
$\hphantom{^1}$Russian Federation

\def\leftfootline{\small{\textbf{\thepage}
\hfill INFORMATIKA I EE PRIMENENIYA~--- INFORMATICS AND
APPLICATIONS\ \ \ 2023\ \ \ volume~17\ \ \ issue\ 2}
}%
 \def\rightfootline{\small{INFORMATIKA I EE PRIMENENIYA~---
INFORMATICS AND APPLICATIONS\ \ \ 2023\ \ \ volume~17\ \ \ issue\ 2
\hfill \textbf{\thepage}}}

\vspace*{3pt}
      
      
    
    \Abste{The paper refines the methodological approach to the contrastive 
studies of punctuation. Given the recent achievements of information science, 
computer linguistics, and translation theory, such studies are most likely to be 
corpus-based. The paper presents a~methodological model of research into 
interlingual punctuation asymmetry, the aim of which is to shed light on the 
functional scope of the same punctuation marks in different languages. It shows 
what methodological trends are characteristic of this research area. The focus is 
also on the specificities of corpus methodology in the contrastive study of 
punctuation. It is argued that one of the methodological tools, tailored specifically 
to the needs of contrastive punctuation research, may be the supracorpora 
databases developed at the Federal Research Center ``Computer Science and 
Control'' of the Russian Academy of Sciences.}
    
    \KWE{contrastive punctuation; translation; corpus-based translation studies; 
corpus-based studies; parallel corpus; supracorpora database; asymmetry between 
languages; methodology}
    
    
    
\DOI{10.14357/19922264230213}{VBOZAO}

%\vspace*{-18pt}

\Ack
    \noindent
    The research was carried out using the infrastructure of the Shared Research 
Facilities ``High Performance Computing and Big Data'' (CKP ``Informatics'') of 
FRC CSC RAS (Moscow). The research was supported by the Russian Science Foundation (project  
No.\,23-28-00548).
 
%\vspace*{4pt}

  \begin{multicols}{2}

\renewcommand{\bibname}{\protect\rmfamily References}
%\renewcommand{\bibname}{\large\protect\rm References}

{\small\frenchspacing
 {%\baselineskip=10.8pt
 \addcontentsline{toc}{section}{References}
 \begin{thebibliography}{99}
 
 \bibitem{4-nu-1} %1
\Aue{Newmark, P.} 1988. \textit{A~textbook of translation}. New York, London, Toronto, Sydney, Tokyo: 
Prentice Hall. 402~p.   

\bibitem{2-nu-1}
\Aue{May, R.} 1994. \textit{The translator in the text: On reading Russian 
literature in English}. Evanston, IL: Northwestern University Press. 209~p.
\bibitem{3-nu-1}
\Aue{Munday, J.} 2002. Systems in translation: A~systemic model for 
descriptive translation studies. \textit{Crosscultural transgressions: Research models in 
translation studies II~--- historical and ideological issues}. Ed. T.~Hermans. 
Manchester, U.K.: St.\ Jerome. 76--92.

\bibitem{1-nu-1} %4
\Aue{Baker, M.} 2011. \textit{In other words}. 2nd ed. London, New York: 
Routledge. 352~p.

\bibitem{7-nu-1} %5
\Aue{Seagal, K.\,Ya.} 2019. Kont\-ras\-tiv\-naya punk\-tu\-a\-tsiya v~na\-cha\-le XXI~veka 
[Contrastive punctuation at the beginning of the XXI century]. \textit{Yazyk. Tekst. 
Diskurs: Nauchnyy al'manakh Stavropol'skogo otdeleniya RALK} [Language. Text. 
Discourse: Scientific almanac of Stavropol Branch of the Russian Cognitive 
Linguists Association].  Stavropol': SKFU. 17:69--78.

\bibitem{5-nu-1} %6
\Aue{Youdale, R.} 2020. \textit{Using computers in the translation of literary style: 
Challenges and opportunities}. London, New York: Routledge. 242~p.
\bibitem{6-nu-1} %7
\Aue{Nuriev, V.\,A., and M.\,G.~Kruzhkov.} 2023. Kor\-pus\-nye dan\-nye pri 
kont\-ras\-tiv\-nom izu\-che\-nii punk\-tu\-a\-tsii [The parallel corpora perspective on studying 
contrastive punctuation]. \textit{Sistemy i~Sredstva Informatiki~--- Systems and Means of 
Informatics} 33(1):14--23. doi: 10.14357/08696527230102.

  \bibitem{8-nu-1}
R$\ddot{\mbox{o}}$ssler, P., P.~Besl, and A.~Saller, eds. 2021. \textit{Vergleichende 
Interpunktion~--- comparative punctuation}. Berlin, Boston: De Gruyter. 454~p.
\bibitem{9-nu-1}
\Aue{Rinas, K.} 2021. Vom genormten satzbau zur genormten interpunktion. 
Zur funktion der zeichensetzung in $\ddot{\mbox{a}}$lterer und neuerer zeit. \textit{Vergleichende 
Interpunktion~--- comparative punctuation}. Eds.\ P.~R$\ddot{\mbox{o}}$ssler, 
P.~Besl, and A.~Saller. 
Berlin, Boston: De Gruyter. 109--136. doi: 10.1515/ 9783110756319-006.
\bibitem{10-nu-1}
\Aue{Ferrari, A., and R.~Stojmenova.} 2021. Weber das komma in kontrastiver 
perspektive Italienisch-Deutsch. \textit{Vergleichende Interpunktion~--- comparative 
punctuation}. Eds. P.~R$\ddot{\mbox{o}}$ssler, P.~Besl, and A.~Saller. Berlin, Boston: De Gruyter.  
57--73. doi: 10.1515/9783110756319-003.
 \bibitem{11-nu-1}
\Aue{Besch, W.} 1981. Zur entwicklung der deutschen interpunktion seit 
dem sp$\ddot{\mbox{a}}$ten mittelalter. \textit{Interpretation und Edition deutscher Texte des Mittelalters. 
Festschrift f$\ddot{\mbox{u}}$r John Asher zum 60. Geburtstag}. Eds. K.~Smits, W.~Besch, and 
V.~Lange. Berlin: Erich Schmidt. 187--206.
 \bibitem{12-nu-1}
\Aue{Eber-Hammerl, F.} 2021. Interpunktion in historischen 
Patientenbriefen. \textit{Vergleichende Interpunktion~--- comparative punctuation}. Eds. 
P.~R$\ddot{\mbox{o}}$ssler, P.~Besl, and A.~Saller. Berlin, Boston: De Gruyter. 163--186.
\bibitem{13-nu-1}
\Aue{Ferrari, A.} 2017. Leggere la virgola. Una prima ricognizione. 
\textit{Chimera Romance Corpora Linguistic Studies} 4(2):145--162. doi: 
10.15366/chimera2017.4.2.001.
\bibitem{14-nu-1}
\Aue{Pecorari, F., and F.~Longo.} 2021. The ellipsis and the dash in Italian 
and English: A~contrastive perspective. \textit{Vergleichende Interpunktion~--- 
comparative punctuation}. Eds. P.~R$\ddot{\mbox{o}}$ssler, P.~Besl, and A.~Saller. Berlin, Boston: 
De Gruyter. 289--314. doi: 10.1515/9783110756319-013.
\bibitem{15-nu-1}
\Aue{N$\acute{\mbox{a}}$dvorn$\!\acute{\mbox{\ptb{\i}}}$kov$\acute{\mbox{a}}$,~O.} 2020. The use of English, Czech and French 
punctuation marks in reference, parallel and comparable web corpora: A~question 
of methodology. \textit{Linguist. Prag.} 30(2):30--50. doi: 
10.14712/18059635.2020.1.2.
\bibitem{16-nu-1}
\Aue{Seagal, K.\,Ya.} 2014. Punk\-tu\-a\-tsiya kak sred\-st\-vo so\-zda\-niya 
emo\-tsi\-o\-nal'\-no\-go pod\-teks\-ta (na ma\-te\-ri\-ale ras\-ska\-za M.\,A.~Sho\-lo\-kho\-va ``Sud'\-ba 
che\-lo\-ve\-ka'' i~ego pe\-re\-vo\-dov na ang\-liy\-skiy yazyk) [Punctuation as a means of 
revealing the emotional subtext (the case of Mikhail Sholokhov's short story ``The 
Fate of a~Man'' and its translations into English)]. \textit{Izvestiya RAN. Seriya literatury i~yazyka}
 [The Bulletin of the Russian Academy of Sciences: Studies in Literature 
and Language]. 73(6):38--50.
\bibitem{17-nu-1}
\Aue{Bogdanov, K.\,A.} 2022. Punk\-tu\-a\-tsiya kak mo\-tiv: mno\-go\-to\-chie i~ti\-re 
[Punctuation as a~motive: The ellipsis and the dash]. \textit{NLO} [New Literary Observer] 
2(174):241--253. doi: 0.53953/08696365\_2022\_174\_2\_241.
\end{thebibliography}

 }
 }

\end{multicols}

\vspace*{-6pt}

\hfill{\small\textit{Received April 15, 2023}} 

\vspace*{-18pt}
    
    
    \Contr
    
    
    \vspace*{-3pt}
    
    \noindent
    \textbf{Nuriev Vitaly A.} (b.\ 1980)~--- Doctor of Science in philology, leading 
scientist, Institute of Informatics Problems, Federal Research Center ``Computer 
Science and Control'' of the Russian Academy of Sciences, 44-2~Vavilov Str., 
Moscow 119333, Russian Federation; \mbox{nurieff.v@gmail.com}
    
    \vspace*{3pt}
    
    \noindent
    \textbf{Karpov Vladimir I.} (b.\ 1971)~--- Candidate of Science (PhD) in 
philology, leading scientist, Institute of Linguistics of the Russian Academy of 
Sciences, 1~bld.~1 Bolshoy Kislovsky lane, Moscow 125009, Russian Federation; 
scientist, Institute of Informatics Problems, Federal Research Center ``Computer 
Science and Control'' of the Russian Academy of Sciences, 44-2~Vavilov Str., 
Moscow 119333, Russian Federation; \mbox{wi.karpow@gmail.com}
     
      
\label{end\stat}

\renewcommand{\bibname}{\protect\rm Литература} 