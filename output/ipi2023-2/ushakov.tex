\def\stat{ushakov}

\def\tit{ИССЛЕДОВАНИЕ СИСТЕМ ОБСЛУЖИВАНИЯ СО~СМЕШАННЫМИ ПРИОРИТЕТАМИ$^*$}

\def\titkol{Исследование систем обслуживания со~смешанными приоритетами}

\def\aut{А.\,К.~Берговин$^1$,  В.\,Г.~Ушаков$^2$}

\def\autkol{А.\,К.~Берговин,  В.\,Г.~Ушаков}

\titel{\tit}{\aut}{\autkol}{\titkol}

\index{Берговин А.\,К.}
\index{Ушаков В.\,Г.}
\index{Bergovin A.\,K.}
\index{Ushakov V.\,G.}


{\renewcommand{\thefootnote}{\fnsymbol{footnote}} \footnotetext[1]
{Работа выполнена при поддержке Министерства науки и~высшего образования
Российской Федерации, грант №\,075-15-2019-1621.}}


\renewcommand{\thefootnote}{\arabic{footnote}}
\footnotetext[1]{Факультет вычислительной математики и~кибернетики Московского государственного университета имени М.\,В.~Ломоносова, 
\mbox{alexey.bergovin@gmail.com}}
\footnotetext[2]{Факультет вычислительной математики и~кибернетики Московского государственного университета имени М.\,В.~Ломоносова; 
Федеральный исследовательский центр <<Информатика и~управ\-ле\-ние>>  Российской академии наук, \mbox{vgushakov@mail.ru}}

\vspace*{-8pt}




\Abst{Изучена однолинейная система массового обслуживания с~бесконечным числом мест для ожидания, 
произвольным распределением времени обслуживания и~пуассоновскими входящими потоками требований. 
Рассматриваются две модели смешанных приоритетов. В~первой модели между требованиями разных потоков действует либо дисциплина 
абсолютного приоритета с~обслуживанием заново прерванного требования, либо дисциплина относительного приоритета. Во второй модели~--- 
дисциплина абсолютного приоритета либо с~потерей, либо с~обслуживанием заново прерванного требования. 
Методом дополнительных компонент исследуется многомерный случайный процесс, компоненты которого~--- число требований каждого приоритета в~системе и~время, 
прошедшее с~начала обслуживания требования, находящегося на приборе в~момент времени~$t$. Найдено распределение указанного процесса в~нестационарном 
режиме работы сис\-темы.}

\KW{относительный приоритет; абсолютный приоритет с~обслуживанием заново; абсолютный приоритет с~потерей; одноканальная сис\-те\-ма; длина очереди}

\DOI{10.14357/19922264230208}{JULPWS} 
  
\vspace*{-6pt}


\vskip 10pt plus 9pt minus 6pt

\thispagestyle{headings}

\begin{multicols}{2}

\label{st\stat}

\section{Введение}

Системы обслуживания с~несколькими типами\linebreak требований и~различными приоритетными дис\-цип\-ли\-на\-ми 
часто рассматриваются в~качестве математических моделей различных инфокоммуникационных сис\-тем. 
Применение приоритетных \mbox{дисциплин} позволяет как учесть  различную степень\linebreak важности требований разных типов, так и~обеспечить 
наиболее эффективную работу сис\-те\-мы.\linebreak К~настоящему времени достаточно полно развита теория однолинейных сис\-тем обслуживания с~приоритетами~[1--3]. 
Практически во всех работах предполагается, что взаимоотношение требований разных типов регулируется одной и~той же 
приоритетной дисциплиной. Возможность выбора разновидности приоритетного правила для каждой пары потоков позволяет, с~одной стороны, 
адекватно описать работу значительно более широкого класса реальных сис\-тем и,~с~другой стороны, организовать более эффективную работу сис\-те\-мы 
в~целом без значительного ухудшения качества обслуживания наиболее приоритетных требований.

\vspace*{-12pt}

\section{Обозначения и~определения}

\vspace*{-3pt}

Рассматривается система обслуживания типа $M_r|G_r|1|\infty,$ в~которую поступают $r\hm\geqslant 2$ пуассоновских потоков требований с~интенсивностями 
$a_1,\ldots,a_r.$ Длительности обслуживания~--- независимые в~совокупности случайные величины с~функциями распределения $B_1(x),\ldots,B_r(x)$ 
и~плотностями распределения $b_1(x),\ldots,b_r(x).$ Требования $i$-го потока (приоритета~$i$) имеют приоритет перед требованиями $j$-го потока при $i\hm<j.$ 
Рассматриваются две модели: 
\begin{enumerate}[(1)]
\item приоритет может быть или относительным, или абсолютным с~обслуживанием заново прерванного требования;
\item приоритет является абсолютным с~потерей или обслуживанием заново прерванного требования.
\end{enumerate}

Положим в~первой модели $I_i$ и~$J_i$~--- соответственно множества номеров потоков, которые имеют абсолютный приоритет и~перед 
которыми имеют абсолютный приоритет требования $i$-го потока ($I_1\hm=J_r\hm=\emptyset$), а $K_i$ и~$M_i$ во второй модели~--- соответственно 
множества номеров потоков, которые имеют абсолютный приоритет с~потерей и~перед которыми имеют абсолютный приоритет с~потерей требования $i$-го потока.

Пусть
$L_i(t)$~--- число требований $i$-го потока в~системе в~момент времени~$t$; $i(t)$ и~$x(t)$~--- 
соответственно номер потока, требование из которого обслуживается в~момент времени~$t$, и~время, прошедшее с~начала его обслуживания (если 
в~момент~$t$ система свободна, то~$i(t)$ и~$x(t)$ можно доопределить произвольным образом, например положить
$i(t)\hm=x(t)\hm=0$), $\mathbf{L}(t)\hm=(L_1(t),\ldots,L_r(t))$;


\noindent
\begin{gather*}
  \beta_i(s)=\int\limits_0^{\infty}e^{-sx}dB(x);\quad \beta_{ij}=\int\limits_0^{\infty}x^jdB_i(x);\\
  \eta_i(x)=\fr{b_i(x)}{1-B_i(x)}\,;
\end{gather*}

\vspace*{-12pt}

\begin{multline*}
   \sigma_k=a_1+\cdots+a_k,\ k=1,\ldots,r,\\[4pt]
    \sigma_0=0,\ \sigma=\sigma_r,\ {\bf 0}=(0,\ldots,0);
\end{multline*}
$$
   {\bf 1}_i=(0,\ldots,0,1,0,\ldots,0), \mbox{где\ 1\ стоит\ на\ $i$-м\ месте};
$$

\vspace*{-6pt}

\noindent
\begin{gather*}
{\sf P}(\mathbf{n},t)=\mathbf{P}(\mathbf{L}(t)=\mathbf{n});\\
 p(\mathbf{z},s)=\int\limits_0^{\infty}e^{-st}
\sum\limits_{n_1=0}^{\infty}\cdots\sum\limits_{n_r=0}^{\infty}
z_1^{n_1}\cdots z_r^{n_r}{\sf P}(\mathbf{n},t)\,dt;
\end{gather*}

\vspace*{-9pt}

\begin{multline*}
{\sf P}_i(\mathbf{n},x,t)=\fr{\partial}{\partial x}\mathbf{P}(\mathbf{L}(t)=\mathbf{n},i(t)=i,x(t)<x),\\
 \mathbf{n}=(n_1,\ldots,n_r);
\end{multline*}

\vspace*{-12pt}

\begin{multline*}
\hspace*{-2mm}p_i(\mathbf{z},x,s)\!=\!\!\int\limits_0^{\infty}\!e^{-st}\!\sum\limits_{n_1=0}^{\infty}\!\!\cdots\!\!\sum\limits_{n_r=0}^{\infty}
z_1^{n_1}\cdots z_r^{n_r}{\sf P}_i(\mathbf{n},x,t)\,dt,\\ 
\mathbf{z}=(z_1,\ldots,z_r);
\end{multline*}

\vspace*{-12pt}

\begin{multline*}
d_i(\mathbf{z},s)=1-z_i^{-1}\beta_i\left(s+\sigma-\sum\limits_{j\notin I_i}a_jz_j\right)-{}\\[4pt]
{}- \sum\limits_{j\in I_i}a_j z_j \fr{1-\beta_i\left(s+\sigma-\sum\nolimits_{j\notin I_i}a_jz_j\right)}{s+\sigma-\sum\nolimits_{j\notin I_i}a_jz_j};
\end{multline*}

\vspace*{-12pt}

\begin{multline*}
c_i(\mathbf{z},s)=1-z_i^{-1}\beta_i\left(s+\sigma-\sum\limits_{j=i}^ra_jz_j\right)-{}\\[4pt]
{}-\left(
\sum\limits_{j\in K_i}a_jz_jz_i^{-1}+\sum\limits_{j=1, j\notin K_i}^{i-1}a_jz_j\right)\times{}\\[4pt]
{}\times
\fr{1-\beta_i\left(s+\sigma-\sum\nolimits_{j=i}^ra_jz_j\right)}{s+\sigma-\sum\nolimits_{j=i}^ra_jz_j}\,.
\end{multline*}
Будем предполагать, что в~начальный момент времени $t\hm=0$ сис\-те\-ма свободна от требований.

\vspace*{-9pt}

\section{Распределение длины очереди}

\vspace*{-3pt}

В первых двух теоремах содержатся результаты для первой модели, а~в~3-й и~4-й~--- для второй.

\smallskip

\noindent
\textbf{Теорема~1.} \textit{При каждом $k\hm=1,\ldots,r$ сис\-те\-ма урав\-нений}
$$
d_i(\mathbf{z},s)=0,\ i=1,\ldots,k,
$$

\vspace*{-3pt}

\noindent
\textit{имеет единственное решение $z_i\hm=\pi_{ik}(z_{k+1},\ldots,z_r,s),$ аналитическое в~об\-ласти}
$|z_{k+1}|<1,\ldots,|z_r|<1,$ $\mathrm{Re}\,s\hm>0,$ \textit{в~которой} $\left|\pi_{ik}(z_{k+1},\ldots,z_r,s)\right|<1$, $i\hm=1,\ldots,k,$
$k\hm=1,\ldots,r.$


\smallskip

\noindent
Д\,о\,к\,а\,з\,а\,т\,е\,л\,ь\,с\,т\,в\,о\ \ теоремы аналогично доказательству леммы~3 в~[2, с.~122].


\smallskip

\noindent
\textbf{Теорема~2.} \textit{Функция $p(\mathbf{z},s)$ определяется по формуле}:

\vspace*{-3pt}

\noindent
\begin{multline}
\label{t1}
p(\mathbf{z},s)=p_0(s)+{}\\
{}+\sum\limits_{i=1}^r \fr{1-\beta_i\left(s+\sigma-\sum\nolimits_{j\notin I_i}a_jz_j\right)}
{s+\sigma-\sum\nolimits_{j\notin I_i}a_jz_j}\,
p_i(\mathbf{z},0,s),
\end{multline}

\vspace*{-3pt}

\noindent
\textit{где}
\begin{equation}
\label{t2}
p_0(s)=\left(s+\sigma-\sum\limits_{j=1}^ra_j\pi_{jr}(s)\right)^{-1},
\end{equation}
\textit{а $p_i(\mathbf{z},0,s)$ определяются из рекуррентных соотношений}

\vspace*{-3pt}

\noindent
\begin{multline}
\label{t3}
\sum\limits_{i=k+1}^rd_i\left(\pi_{1k}(z_{k+1},\ldots,z_r,s),\ldots\right.\\
\left.\ldots ,\pi_{kk}(z_{k+1},\ldots,z_r,s),
z_{k+1},\ldots,z_r,s\right)p_i(\mathbf{z},0,s)={}\\
{}=1-\left(\!s+\sigma-\!\!\sum\limits_{j=1}^k \!a_j \pi_{jk}(z_{k+1},\ldots,z_r,s)-{}\right.\\
\left.{}-
\sum\limits_{j=k+1}^r \! a_j z_j\!\right)p_0(s),\enskip 
k=0,\ldots,r-1.
\end{multline}

\noindent
Д\,о\,к\,а\,з\,а\,т\,е\,л\,ь\,с\,т\,в\,о\,.\ \ 
Рассматривая изменения состояний процесса $(\mathbf{L}(t),x(t),i(t))$ в~интервале времени $(t,t+\Delta)$ 
и~устремляя $\Delta\hm\rightarrow 0$, имеем

\vspace*{-3pt}

\noindent
\begin{multline}
\label{n1}
\hspace*{-2mm}\fr{\partial {\sf P}_i(\mathbf{n},x,t)}{\partial t}+\fr{\partial {\sf P}_i(\mathbf{n},x,t)}{\partial x}=-(\sigma+\eta_i(x)){\sf P}_i(\mathbf{n},x,t)+{}\\
{}+\sum\limits_{j\notin I_i,\ j\neq i}\left(1-\delta_{n_j,0}\right)a_j{\sf P}_i(\mathbf{n}-{\bf 1}_j,x,t)+{}\\
{}+
\left(1-\delta_{n_i,1}\right)a_i{\sf P}_i(\mathbf{n}-{\bf 1}_i,x,t).
\end{multline}

\noindent
Переходя в~\eqref{n1}  к~производящим функциям и~преобразованиям Лапласа по~$t,$ получаем

\vspace*{-2pt}

\noindent
\begin{multline*}
%\label{n2}
\fr{\partial p_i(\mathbf{z},x,s)}{\partial x}=-(s+\sigma+\eta_i(x))p_i(\mathbf{z},x,s)+{}\\
{}+
\sum\limits_{j\notin I_i} a_j z_j p_i(\mathbf{z},x,s).
\end{multline*}


\noindent
Для вероятности свободного состояния ${\sf P}_0(t)$ имеем
$$
\fr{\partial {\sf P}_0(t)}{\partial t}=-\sigma {\sf P}_0(t)+\sum\limits_{j=1}^r \int\limits_0^{\infty}
{\sf P}_j(\mathbf{1}_j,x,t)\eta_j(x)\,dx
$$
и

\vspace*{-2pt}

\noindent
\begin{multline}
\label{n3}
(s+\sigma)p_0(s)-1={}\\
{}=\int\limits_0^{\infty}e^{-st}\sum\limits_{j=1}^r\int\limits_0^{\infty}
{\sf P}_j({\bf 1}_j,x,t)\eta_j(x)\:dx\:dt.
\end{multline}

Найдем теперь краевые условия в~точке $x=0.$ Имеем:
$$
{\sf P}_i(\mathbf{n},0,t)=0,\ \mbox{если}\ n_1+\cdots+n_{i-1}\neq 0\ \mbox{или}\ n_i=0\,.
$$

\vspace*{-2pt}

\noindent
Для остальных $n_1,\ldots,n_r$ справедливы равенства:

\vspace*{-2pt}

\noindent
\begin{multline}
\label{n4}
{\sf P}_i(\mathbf{n},0,t)=\sum\limits_{j=1, j\notin J_i}^{r}\int\limits_0^{\infty}{\sf P}_j(\mathbf{n}+{\bf 1}_j,x,t)\eta_j(x)\,dx+{}\\
{}+\delta_{n_i,1}\sum\limits_{j\in J_i}\int\limits_0^{\infty}{\sf P}_j(\mathbf{n}-{\bf 1}_i,x,t)\,dx\:a_i+{}\\
{}+\delta_{n_i,1}\prod\limits_{j\neq i}\delta_{n_j,0}
a_i{\sf P}_0(t).
\end{multline}

\vspace*{-2pt}

\noindent
Из \eqref{n3} и~\eqref{n4} получаем
\begin{multline}
\label{n5}
\sum\limits_{i=1}^rp_i(\mathbf{z},0,s)=\sum\limits_{i=1}^r\int\limits_0^{\infty}p_i(\mathbf{z},x,s)\eta_i(x)\,dx+{}\\
{}+\sum\limits_{i=1}^r\left(\sum\limits_{j\in I_i}a_jz_j\right)\int\limits_0^{\infty}p_i(\mathbf{z},x,s)\,dx+
1-{}\\
{}-\left(s+\sigma-\sum\limits_{j=1}^r a_j z_j \right)p_0(s).
\end{multline}

\vspace*{-2pt}

\noindent
Решение системы уравнений \eqref{n3} имеет вид:

\vspace*{-2pt}

\noindent
\begin{multline*}
p_i(\mathbf{z},x,s)={}\\
{}=(1-B_i(x))e^{-\left(s+\sigma-\sum\nolimits_{j\notin I_i}a_jz_j\right)x}p_i(\mathbf{z},0,s).
\end{multline*}
Подставляя его в~\eqref{n5}, имеем
\begin{equation*}
%\label{n6}
\sum\limits_{i=1}^r d_i(\mathbf{z},s)p_i(\mathbf{z},0,s)=1-\left(s+\sigma-\sum\limits_{j=1}^r a_j z_j\right)p_0(s).
\end{equation*}

%\vspace*{-2pt}

\noindent
Так как ${\sf P}_i(\mathbf{n},0,t)\hm=0$ при $n_1+\cdots+n_{i-1}\hm\neq 0,$ то $p_i(\mathbf{z},0,s)$ не зависит от $z_1,\ldots,z_{i-1}.$
Отсюда и~из теоремы~1 следуют~\eqref{t2} и~\eqref{t3}, а~из равенства 

%\columnbreak

\noindent
$$
p(\mathbf{z},s)=p_0(s)+\sum\limits_{i=1}^r \int\limits_0^{\infty}p_i(\mathbf{z},x,s)\,dx
$$ 

\vspace*{-3pt}

\noindent
получаем~\eqref{t1}.

\smallskip

\noindent
\textbf{Теорема~3.} \textit{При каждом $k=1,\ldots,r$ система урав\-нений}
$$
c_i(\mathbf{z},s)=0,\enskip i=1,\ldots,k,
$$

\vspace*{-3pt}

\noindent
\textit{имеет единственное решение $z_i\hm=\tau_{ik}(z_{k+1},\ldots,z_r,s),$ аналитическое в~об\-ласти
$|z_{k+1}|\hm<1,\ldots,|z_r|\hm<1,$ $\mathrm{Re}\,s\hm>0,$ в~которой} $\left|\tau_{ik}(z_{k+1},\ldots,z_r,s)\right|\hm<1$, $i\hm=1,\ldots,k,$
$k\hm=1,\ldots,r.$


\smallskip

\noindent
\textbf{Теорема~4.} \textit{Функция $p(\mathbf{z},s)$ определяется по формуле}

\vspace*{-3pt}

\noindent
\begin{multline*}
%\label{t1-1}
p(\mathbf{z},s)=p_0(s)+{}\\
{}+\sum\limits_{i=1}^r \fr{1-\beta_i\left(s+\sigma-\sum\nolimits_{j=i}^r a_j z_j\right)}
{s+\sigma-\sum\nolimits_{j=i}^ra_jz_j}\,
p_i(\mathbf{z},0,s),
\end{multline*}

\vspace*{-3pt}

\noindent
где
\begin{equation*}
p_0(s)=\left(s+\sigma-\sum\limits_{j=1}^r a_j \tau_{jr}(s)\right)^{-1},
\end{equation*}

\vspace*{-3pt}

\noindent
а $p_i(\mathbf{z},0,s)$ определяются из рекуррентных соотношений

\vspace*{-3pt}

\noindent
\begin{multline*}
\!\!\sum\limits_{i=k+1}^r \! c_i\left(\tau_{1k}(z_{k+1},\ldots,z_r,s),\ldots\right.\\
\left.\ldots,\tau_{kk}(z_{k+1},\ldots,z_r,s),
z_{k+1},\ldots,z_r,s\right)p_i(\mathbf{z},0,s)={}\\
{}=1-\left(s+\sigma-\sum\limits_{j=1}^k a_j \tau_{jk}(z_{k+1},\ldots,z_r,s)-{}\right.\\
\left.{}-
\sum\limits_{j=k+1}^r a_j z_j\right)p_0(s),\ k=0,\ldots,r-1.
\end{multline*}


\noindent
Д\,о\,к\,а\,з\,а\,т\,е\,л\,ь\,с\,т\,в\,о\,.\ \  Для второй модели функции ${\sf P}_i(\mathbf{n},x,t)$ удовлетворяют системе дифференциальных уравнений

\vspace*{-3pt}

\noindent
\begin{multline*}
\hspace*{-2mm}\fr{\partial {\sf P}_i(\mathbf{n},x,t)}{\partial t}+\fr{\partial {\sf P}_i(\mathbf{n},x,t)}{\partial x}=-(\sigma+\eta_i(x)){\sf P}_i(\mathbf{n},x,t)+{}\\
{}+\sum\limits_{j=i+1}^r\left(1-\delta_{n_j,0}\right)a_j {\sf P}_i(\mathbf{n}-\mathbf{1}_j,x,t)+{}\\
{}+
\left(1-\delta_{n_i,1}\right) a_i {\sf P}_i \left(\mathbf{n}-\mathbf{1}_i,x,t\right).
\end{multline*}

\vspace*{-3pt}

\noindent
Отсюда

\vspace*{-3pt}

\noindent
\begin{multline}
\label{m2}
\fr{\partial p_i(\mathbf{z},x,s)}{\partial x}={}\\
{}=-\left(s+\sigma-\sum\limits_{j=i}^ra_jz_j+\eta_i(x)\right)p_i(\mathbf{z},x,s).
\end{multline}

%\vspace*{-3pt}

\noindent
Для вероятности свободного состояния~${\sf P}_0(t)$ и~ее преобразования Лапласа~$p_0(s)$  справедливы те же соотношения, что и~для модели~1.
При $x\hm=0$ имеем:
$$
{\sf P}_i(\mathbf{n},0,t)=0,\ \mbox{если}\ n_1+\cdots+n_{i-1}\neq 0\ \mbox{или}\ n_i=0.
$$
Для остальных $n_1,\ldots,n_r$ справедливы равенства:
\begin{multline}
\label{m4}
{\sf P}_i(\mathbf{n},0,t)=\sum\limits_{j=1}^{i}\int\limits_0^{\infty} {\sf P}_j (\mathbf{n}+\mathbf{1}_j,x,t)\eta_j(x) \,dx+{}\\
{}+\delta_{n_i,1}\prod\limits_{j\neq i}\delta_{n_j,0} a_i {\sf P}_0(t)+{}\\
{}+\delta_{n_i,1}a_i\left(\sum\limits_{j=i+1,j\in M_i}^r\int\limits_0^{\infty}{\sf P}_j(\mathbf{n}+\mathbf{1}_j-\mathbf{1}_i,x,t)\,dx+{}\right.\\
\left.{}+
\sum\limits_{j=i+1,j\notin M_i}^r\int\limits_0^{\infty}{\sf P}_j(\mathbf{n}-\mathbf{1}_i,x,t)\,dx
\right).
\end{multline}
Из \eqref{m2} и~\eqref{m4} находим
$$
p_i(\mathbf{z},x,s)=(1-B_i(x))e^{-\left(s+\sigma-\sum\nolimits_{j=i}^ra_jz_j\right)x}p_i(\mathbf{z},0,s);
$$
\begin{equation*}
\sum\limits_{i=1}^r c_i(\mathbf{z},s)p_i(\mathbf{z},0,s)=1-\left(s+\sigma-\sum\limits_{j=1}^r a_j z_j \right)p_0(s).
\end{equation*}
Окончание доказательства повторяет рассуждения при доказательстве теоремы~2.

\vspace*{-3pt}

{\small\frenchspacing
 {%\baselineskip=10.8pt
 %\addcontentsline{toc}{section}{References}
 \begin{thebibliography}{9}
\bibitem{1-ush}
\Au{Jaiswal N.\,K.} Priority queues.~--- New York; London: Academic press, 1968.  240~p.
\bibitem{2-ush}
\Au{Матвеев В.\,Ф., Ушаков~В.\,Г.} Сис\-те\-мы массового
обслуживания.~--- М.: Изд-во Московского ун-та, 1984. 240~с.
\bibitem{3-ush}
\Au{Takagi H.} Queueing analysis: A~foundation of performance evaluation.~--- 
Amsterdam: North-Holland Elsevier, 1991. Vol.~1. Part~1. 487~p.
\end{thebibliography}

 }
 }

\end{multicols}

\vspace*{-6pt}

\hfill{\small\textit{Поступила в~редакцию 28.03.22}}

\vspace*{8pt}

%\pagebreak

%\newpage

%\vspace*{-28pt}

\hrule

\vspace*{2pt}

\hrule

%\vspace*{-2pt}

\def\tit{ANALYSIS OF~THE~QUEUEING SYSTEMS WITH~MIXED PRIORITIES}


\def\titkol{Analysis of~the~queueing systems with~mixed priorities}


\def\aut{A.\,K.~Bergovin$^1$ and~V.\,G.~Ushakov$^{1,2}$}

\def\autkol{A.\,K.~Bergovin and~V.\,G.~Ushakov}

\titel{\tit}{\aut}{\autkol}{\titkol}

\vspace*{-10pt}


\noindent
$^1$M.\,V.~Lomonosov Moscow State University, 1-52~Leninskie Gory, GSP-1, Moscow 119991, Russian Federation

\noindent
$^2$Federal Research Center ``Computer Science and Control'' of the Russian Academy of Sciences, 44-2~Vavilov\linebreak
$\hphantom{^1}$Str., Moscow 119333, Russian Federation

\def\leftfootline{\small{\textbf{\thepage}
\hfill INFORMATIKA I EE PRIMENENIYA~--- INFORMATICS AND
APPLICATIONS\ \ \ 2023\ \ \ volume~17\ \ \ issue\ 2}
}%
 \def\rightfootline{\small{INFORMATIKA I EE PRIMENENIYA~---
INFORMATICS AND APPLICATIONS\ \ \ 2023\ \ \ volume~17\ \ \ issue\ 2
\hfill \textbf{\thepage}}}

\vspace*{3pt}



\Abste{A one-line queuing system with an infinite number of waiting places, an arbitrary distribution of service time, and 
Poisson incoming flows of customers is studied. Two models of mixed priorities are considered. In the first model, there is 
either preemptive repeat priority discipline between the customers of different flows or a~head of the line priority discipline.
 In the second model~--- preemptive priority discipline either with loss or with repeat of a~newly interrupted customer. 
 By the method of additional components, a~multidimensional random process is investigated, the components of which are 
 the number of customers of each priority in the system and the time elapsed since the start of servicing the customer located on the device at time~$t$. 
 The distribution of the specified process in the nonstationary mode of the system is found.}

\KWE{head of the line; preemptive repeat; preemptive loss; one-line; queue length}



\DOI{10.14357/19922264230208}{JULPWS}

\vspace*{-18pt}

\Ack

\vspace*{-3pt}

\noindent
 The research was supported by the Ministry of Science and Higher Education of the Russian Federation, project No.\,075-15-2019-1621.

%\vspace*{4pt}

  \begin{multicols}{2}

\renewcommand{\bibname}{\protect\rmfamily References}
%\renewcommand{\bibname}{\large\protect\rm References}

{\small\frenchspacing
 {%\baselineskip=10.8pt
 \addcontentsline{toc}{section}{References}
 \begin{thebibliography}{9} 
\bibitem{1-ush-1}
\Aue{Jaiswal, N.\,K.} 1968. \textit{Priority queues}.  New York; London: Academic press.  240~p.
\bibitem{2-ush-1}
\Aue{Matveev, V.\,F., and V.\,G.~Ushakov.} 1984. 
\textit{Sis\-te\-my mas\-so\-vo\-go obslu\-zhi\-va\-niya} [Queueing systems]. Moscow: M.\,V.~Lomonosov Moscow State University Publs. 240~p.
{\looseness=1

} 
\bibitem{3-ush-1}
\Aue{Takagi, H.} 1991. \textit{Queueing analysis: A~foundation of performance evaluation}. 
 Amsterdam: North-Holland Elsevier.  Vol.~1. Part~1. 487~p.
\end{thebibliography}

 }
 }

\end{multicols}

\vspace*{-8pt}

\hfill{\small\textit{Received March 28, 2022}} 

\pagebreak
 
\Contr

\noindent
\textbf{Bergovin Alexey K.} (b.\ 1995)~---  
PhD student, Department of Mathematical Statistics, Faculty of Computational Mathematics and Cybernetics, 
M.\,V.~Lomonosov Moscow State University, 1-52~Leninskie Gory, GSP-1, Moscow 119991, Russian Federation; \mbox{alexey.bergovin@gmail.com}

\vspace*{4pt}

\noindent
\textbf{Ushakov Vladimir G.} (b.\ 1952)~--- Doctor of Science in physics and mathematics, professor, 
Department of Mathematical Statistics, Faculty of Computational Mathematics and Cybernetics, M.\,V.~Lomonosov Moscow State University, 
1-52~Leninskie Gory, GSP-1, Moscow 119991, Russian Federation; senior scientist, Institute of Informatics Problems, 
Federal Research Center ``Computer Science and Control'' of the Russian Academy of Sciences, 44-2~Vavilov Str., Moscow 119333, Russian Federation; 
\mbox{vgushakov@mail.ru}



   
\label{end\stat}

\renewcommand{\bibname}{\protect\rm Литература} 
      