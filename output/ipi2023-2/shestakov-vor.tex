\def\stat{shest}

\def\tit{СРЕДНЕКВАДРАТИЧНЫЙ РИСК FDR-ПРОЦЕДУРЫ\\ В~УСЛОВИЯХ
СЛАБОЙ ЗАВИСИМОСТИ}

\def\titkol{Среднеквадратичный риск FDR-процедуры в~условиях
слабой зависимости}

\def\aut{М.\,О.~Воронцов$^1$,  О.\,В.~Шестаков$^2$}

\def\autkol{М.\,О.~Воронцов,  О.\,В.~Шестаков}

\titel{\tit}{\aut}{\autkol}{\titkol}

\index{Воронцов М.\,О.}
\index{Шестаков О.\,В.}
\index{Vorontsov M.\,O.}
\index{Shestakov O.\,V.}


%{\renewcommand{\thefootnote}{\fnsymbol{footnote}} \footnotetext[1]
%{Исследование выполнено за счет гранта Российского научного фонда (проект 22-28-00588). Работа 
%РАН, Москва).}}


\renewcommand{\thefootnote}{\arabic{footnote}}
\footnotetext[1]{Факультет вычислительной математики и кибернетики,  Московский государственный университет 
имени М.\,В.~Ломоносова;
Московский центр фундаментальной и прикладной математики, \mbox{m.vtsov@mail.ru}}
\footnotetext[2]{Факультет вычислительной математики и кибернетики, Московский государственный университет 
имени М.\,В.~Ломоносова; 
Федеральный исследовательский центр <<Информатика и~управ\-ле\-ние>> Российской 
академии наук; Московский центр фундаментальной и прикладной математики, 
\mbox{oshestakov@cs.msu.ru}}

%\vspace*{-12pt}





\Abst{Во многих прикладных областях возникает задача обработки больших массивов 
данных. При этом часто перед обработкой массив данных подвергается некоторому 
преобразованию, приводящему к <<разреженному>>, или <<экономному>>, 
представлению, при котором абсолютное значение большинства элементов массива 
равно нулю (или достаточно мало).
Кроме того, в~результате помех при получении и~передаче данных в них попадает 
шум, который при дальнейшей обработке желательно некоторым образом удалить. 
Возникающая при этом задача математически эквивалентна некоторым задачам 
множественной проверки гипотез.
Ранее для решения указанной задачи в~условиях нормальности, независимости 
и~разреженности данных была предложена процедура, основанная на методе контроля средней доли ложных 
отклонений (False Discovery Rate, FDR) гипотез.
В~настоящей работе исследуется асимптотика риска указанной процедуры в~случае 
наличия слабой за\-ви\-си\-мости в~данных.}

\KW{пороговая обработка; множественная проверка гипотез; 
среднеквадратичный риск}


\DOI{10.14357/19922264230205}{AVJZDX}
  
%\vspace*{-8pt}


\vskip 10pt plus 9pt minus 6pt

\thispagestyle{headings}

\begin{multicols}{2}

\label{st\stat}


\section{Введение}


В современных приложениях статистики за\-час\-тую требуется обрабатывать большие 
массивы зашумленных данных~-- источниками шума могут выступать помехи 
и~несовершенство оборудования. Примерами служат исследования в~об\-ласти генетики 
с~возникающими в них задачами множественной проверки гипотез~\cite{MultipleTesting}, задачи обработки изображений 
с~высоким разрешением~\cite{FDRImage} и другие прикладные проб\-ле\-мы. В~связи с этим рассмотрим задачу 
нахождения оценки неизвестного вектора~$\mu$ как функции~$x$ в~модели данных
$$
x_i = \mu_i + z_i, \enskip i=1,\ldots,n,
$$
где $\mu_i\in\mathbb{R}$, $z_i \sim N(0,\sigma^2)$ для всех~$i$.

Приведенная задача может рассматриваться как частный случай задачи множественной 
проверки гипотез, а~именно: пусть построено~$n$ статистик~$x_i$ для проверки 
нулевых гипотез~$H_{0,i}$ против альтернатив~$H_{1,i\,}$, причем при 
верной гипотезе~$H_{0, i}$ (соответственно~$H_{1, i}$) распределение~$x_i$ 
известно и~равно $N(0,\sigma^2)$ (соответственно $N(\mu_i,\sigma^2)$, 
$\mu_i\hm\neq0$ и неизвестно). Принятие гипотезы $H_{0,i}$ в такой постановке 
равносильно заключению $\mu_i\hm=0$.

В работе~\cite{AdaptingFDR} для решения рассматриваемой задачи в условиях 
независимости компонент вектора~$x$ и~разреженности вектора~$\mu$ была 
предложена процедура построения оценки~$\hat{\mu}_F$ вектора~$\mu$, основанная 
на методе контроля средней доли ложных отклонений (FDR) 
гипотез при помощи алгоритма Бен\-жа\-ми\-ни--Хоч\-бер\-га,
и~было проведено исследование асимптотики риска построенной оценки.

В то же время в определенных приложениях, например
при анализе полученных в результате использования ДНК-мик\-ро\-чи\-пов 
данных~\cite{ResultsOnFDRUnderDependence}, исследовании геофизических процессов 
и~анализе помех в телекоммуникационных каналах, условие независимости компонент 
вектора~$x$ может не выполняться. В~данной работе исследуется асимптотика риска 
предложенной в~\cite{AdaptingFDR} оценки~$\hat{\mu}_F$ в случае, когда 
компоненты вектора~$x$ слабо зависимы, а~$\mu$ принадлежит классу разреженности
$$
l_0[\eta] = \left\{\mu\in\mathbb{R}^n\,:\, ||\mu||_0 \leq \eta n\right\}, \enskip
\eta\in(0;1).
$$


\section{Обработка вектора данных с~помощью FDR-процедуры}

Предложенная в~\cite{AdaptingFDR} процедура заключается в~жесткой пороговой 
обработке компонент вектора~$x$ порогом $\hat{t}_F = \hat{t}_F(x)$, и~ее 
результат~--- оценка~$\hat{\mu}_F$ вектора~$\mu$ с компонентами

\noindent
\begin{equation*}
(\hat{\mu}_F)_i  = p(x_i,\hat{t}_F) \equiv
 \begin{cases}
   x_i, & |x_i| > \hat{t}_F;\\
   0, & |x_i| \leq \hat{t}_F,
 \end{cases}
\end{equation*}
где
$$
\hat{t}_F = \sigma z\left(\fr{q \hat{k}_F}{2n}\right).
$$
Здесь
$z(\alpha)$~--- квантиль уровня $(1-\alpha)$ стандартного нормального 
распределения; $q\in(0;1)$~--- управ\-ля\-ющий па\-ра\-метр FDR-ме\-то\-да;
$$
\hat{k}_F \hm= \max \left\{k \, :\, |x|_{(k)} \geq t_k \right\}, 
$$
где
$|x|_{(k)}$~--- $k$-й элемент вектора, получаемого в~результате 
упорядочения вектора~$|x|$ по невозрастанию:
$$
|x|_{(1)} \geq |x|_{(2)} \geq \cdots \geq |x|_{(n)},
$$
$$
 t_k = \sigma z\left(\fr{q  k}{2n}\right).
$$
 Далее полагается, что $q\hm\equiv q_n$ зависит от~$n$.

В~\cite{AdaptingFDR} для среднеквадратичного риска
$$
\rho(\hat{\mu}_F, \mu) = \e ||\hat{\mu}_F-\mu||^2 = \e\sum\limits_{i=1}^n 
\left(p(x_i,\hat{t}_F)-\mu_i\right)^2
$$
оценки $\hat{\mu}_F$ получен сле\-ду\-ющий результат.


\smallskip

\noindent
\textbf{Теорема~1.}
\textit{Пусть $x_i$, $i=1,\ldots,n$, независимы, $\liminf q_n \ln n > 0$, $\limsup q_n 
<1$, а также $\eta_n$ лежит в интервале $[n^{-1}\ln^5 n; n^{-\delta}]$, 
$\delta>0$. Тогда при $n\to\infty$}
\begin{multline*}
\sup\limits_{\mu\in l_0[\eta_n]}\rho(\hat{\mu}_F, \mu)  \leq {}\\
{}\leq
R_n(l_0[\eta_n])\left(1+2 q_n(1-q_n)^{-1}+o(1)\right),
\end{multline*}
\textit{где}
\begin{equation*}
R_n(l_0[\eta_n]) = \inf\limits_{\hat{\mu}=\hat{\mu}(x)} \sup\limits_{\mu\in 
l_0[\eta_n]}\rho(\hat{\mu}, \mu).
\end{equation*}



В работе также приведена асимптотика
$$
R_n(l_0[\eta_n])\sim c n\eta_n \ln \eta_n^{-1},
$$
где $c=c(\sigma)$.

При пороговой обработке иногда также используется так называемый универсальный 
порог $T_U \hm= \sigma \sqrt{2\ln n}$, предложенный в работе~\cite{spatialAdaptation}. Исследования в~\cite{AdaptingSURE, ExactRisk} 
показали, что порог~$T_U$ в определенном смысле максимальный и~рассматривать 
пороги выше него не имеет смысла. Более того, нетрудно показать, что $t_k\hm < T_U$ 
для всех~$k$ и всех достаточно больших~$n$.  В~связи с~этим всюду далее 
полагаем, что порог~$\hat{t}_F$ выбирается на отрезке $[0; T_U]$.

\section{Асимптотика среднеквадратичного риска FDR-процедуры в~условиях слабой 
зависимости}

Перейдем к исследованию асимптотики риска оценки~$\hat{\mu}_F$ в случае, когда 
компоненты вектора~$x$ слабо зависимы, а~именно: имеют достаточно быст\-ро 
убывающий коэффициент сильного перемешивания~\cite{Bosq}
\begin{multline*}
\alpha(k) = \sup\limits_{1\leq m\leq n}\alpha\left(\sigma(x_i, i\leq m), 
\sigma(x_i, i\geq m+k)\right), \\ k=1,\ldots,n-1,
\end{multline*}
где
$$
\alpha(\mathcal{B},\mathcal{C}) = \sup\limits_{B\in\mathcal{B}, 
C\in\mathcal{C}} \left|\p(BC)-\p(B)\p(C)\right|.
$$
Отметим, что для любой измеримой функции~$f(\cdot)$ коэффициент сильного 
перемешивания набора $f(x_1),\ldots,f(x_n)$ не больше коэффициента сильного 
перемешивания набора $x_1,\ldots,x_n$.



Покажем справедливость следующего вспомогательного утверж\-де\-ния.

\smallskip

\noindent
\textbf{Утверждение~1.}
\textit{Пусть для набора действительных случайных величин $X_1,\dots,X_n$ с 
коэффициентом сильного перемешивания $\alpha(\cdot)$ выполняется $\e X_i \hm= 0$, 
$|X_i| \leq b$, $i = 1,\dots,n$. Тогда для любого целого числа $m\in[1;n/2]$ и 
любого $\eps > 0$ справедливо}
\begin{multline}
\p\left(\left|\sum\limits_{i=1}^n X_i\right| > \eps\right) \leq 2 \exp\left\{-
\fr{\eps^2}{32 v_0}\, B\left(\fr{nb\eps}{8m v_0}\right)\right\}+{}\\
{}+ 2 \exp\left\{-\fr{\eps^2}{32 v_1} B\left(\fr{nb\eps}{8m  v_1}\right)\right\} +{}\\
{}+ 22\left(1+\fr{4bn}{\eps}\right)^{1/2} m \alpha\left(\left[\fr{n}{2m}\right]\right), 
\label{statement1}
\end{multline}
\textit{где}
\begin{multline*}
v_0 = \sum\limits_{j=1}^m \e\left(([2(j-1)p]+1-2(j-1)p)\right. \times{}\\
{}\times X_{[2(j-1)p]+1}+X_{[2(j-1)p]+2}+\cdots + X_{[(2j-1)p]} +{}\\
\left.{}+ ((2j-1)p-[(2j-1)p])X_{[(2j-1)p]+1}\right)^2 ;
\end{multline*}

\vspace*{-12pt}

\noindent
\begin{multline*}
v_1 = \sum\limits_{j=1}^m \e\left(([(2j-1)p]+1-(2j-1)p)\times{} \right.\\
{}\times X_{[(2j- 1)p]+1}+X_{[(2j-1)p]+2}+\cdots + X_{[2jp]} +{}\\
\left.{}+ (2jp-[2jp])X_{[2jp]+1}\right)^2;
\end{multline*}

\noindent
$$
p=\fr{n}{2m}\,; \enskip B(\lambda) = 2\lambda^{-2}((1+\lambda)\ln(1+\lambda)-
\lambda), \enskip \lambda>0\,.
$$


\noindent
Д\,о\,к\,а\,з\,а\,т\,е\,л\,ь\,с\,т\,в\,о\,.\ \ При доказательстве теоремы~1.3 из~\cite{Bosq} 
показано, что
\begin{multline*}
\p \left(\left|\sum\limits_{j=1}^m V_j\right| > \fr{\eps}{2} \right) \leq 
\p \left(\left|\sum\limits_{j=1}^m W_j\right| > \fr{\eps}{4}\right) + {}\\
{}+
11\left(1+\fr{4bn}{\eps}\right)^{1/2} m 
\alpha\left(\left[\fr{n}{2m}\right]\right),
\end{multline*}
где
\begin{multline*}
V_j = ([2(j-1)p]+1-2(j-1)p)\times{}\\
{}\times X_{[2(j-1)p]+1}+X_{[2(j-1)p]+2}+\cdots + X_{[(2j-1)p]} + {}\\
{}+((2j-1)p-[(2j-1)p])X_{[(2j-1)p]+1},\\
  W_j {\overset{d}{=}} V_j, \enskip j = 1,\ldots,m\,,
  \end{multline*}
а случайные величины $W_1,\ldots,W_m$ независимы. Применяя для случайных величин 
$W_1,\ldots,W_m$ неравенство Беннета~\cite{Pollard}, получим
$$
\p \left(\left|\sum\limits_{j=1}^m W_j\right| > \fr{\eps}{4}\right) \leq 2 
\exp\left\{-\fr{\eps^2}{32 v_0}\, 
B\left(\fr{pb\eps}{4v_0}\right)\right\}.
$$
Проводя аналогичные рассуждения для случайных величин
\begin{multline*}
V'_j = ([(2j-1)p]+1-(2j-1)p) \times{}\\
{}\times X_{[(2j-1)p]+1}+X_{[(2j-1)p]+2}+\cdots + X_{[2jp]} +{}\\
{}+ (2jp-[2jp])X_{[2jp]+1}, \enskip j = 1,\ldots,m,
\end{multline*}
и объединяя результаты, с учетом
\begin{multline*}
\p \left(\left|\sum\limits_{i=1}^n X_i \right| > \eps \right) \leq {}\\
{}\leq 
\p \left(\left|\sum\limits_{i=1}^m V_i \right| > \fr{\eps}{2} \right) + 
\p \left(\left|\sum\limits_{i=1}^m V'_i \right| > \fr{\eps}{2} \right)
\end{multline*}
получаем требуемое.\hfill$\square$


\smallskip

\noindent
\textbf{Замечание.}
Из непрерывности правой части неравенства~(\ref{statement1}) по~$\eps$ следует, 
что в левой части можно заменить знак~$>$ на~$\geq$.


\smallskip

Введем следующие обозначения:

\vspace*{-6pt}

$$
k_n = [\eta_n n]; \enskip \gamma_n = (\ln\ln n)^{-1}; \enskip \kappa_n = (1 - q_n - \gamma_n)^{-1} k_n;
$$

\vspace*{-12pt}

\noindent
\begin{multline*}
p_i = \p(|x_i|\geq t_k), \enskip X_i = \Ik(|x_i|\geq t_k) - p_i, \\ i=1,\ldots,n;
\end{multline*}

\vspace*{-3pt}

\noindent
$$
N(t_k) = \#\{i : |x_i| \geq t_k\};\enskip M = \e N(t_k) = \sum\limits_{i=1}^n  p_i.
$$
Заметим, что $\e X_i \hm= 0$, $|X_i| \hm< 1$, $\D X_i \hm= p_i(1\hm-p_i)$ для всех~$i, k$.

\smallskip

\noindent
\textbf{Лемма~1.}
\textit{Пусть $\eta_n \leq b<1$, $m\in[1;n/2]\cap\mathbb{N}$, а~$\alpha(\cdot)$~--- 
коэффициент сильного перемешивания компонент вектора~$x$. Для некоторого 
$N\hm\in\mathbb{N}$ при $n \hm\geq N$ справедливо}
\begin{multline*}
\hspace*{-7.9pt}\sup\limits_{\mu\in l_0[\eta_n]} \p \left(\hat{k}_F \!\geq\! \kappa_n \right) \leq 
4 n \exp\left\{ \!-\fr{(1-b)m}{64n}  \kappa_n q_n \gamma_n^2   \! \right\}+{}\\[3pt]
{}+ 
22\left(1+\fr{4n}{(1-b)\kappa_n q_n \gamma_n}\right)^{1/2} n m  \alpha\left(\left[\fr{n}{2m}\right]\right).
\end{multline*}

\noindent
Д\,о\,к\,а\,з\,а\,т\,е\,л\,ь\,с\,т\,в\,о\,.\ \ Фиксируем $\mu\hm\in l_0[\eta_n]$. Имеем
\begin{equation}
\label{lem1eq1}
\p(\hat{k}_F\geq \kappa_n) \leq \sum\limits_{k\geq\kappa_n} \p(N(t_k)\geq k).
\end{equation}
Фиксируем $k \geq \kappa_n$; задача~-- ограничить вероятность $\p(N(t_k)\hm\geq k)$ 
сверху. Ниже показано, что $M\hm<k$ для всех $k \hm\geq \kappa_n$ и всех достаточно 
больших~$n$. По утверждению~1 имеем:
\begin{multline}
\p(N(t_k)\geq k) = \p\left(\sum\limits_{i=1}^n X_i \geq k-M\right) \leq {}\\[3pt]
{}\leq
\p\left(\left|\sum\limits_{i=1}^n X_i \right| \geq k-M\right) \leq{}\\[3pt]
{}\leq 2 \exp\left\{-\fr{(k-M)^2}{32 v_0}\, B\left(\fr{n(k-M)}{8m  v_0}\right)\right\}+{}\\[3pt]
{}+2 \exp\left\{-\fr{(k-M)^2}{32 v_1}\, B\left(
\fr{n(k-M)}{8m v_1}\right)\right\} +{}\\[3pt]
{}+ 22\left(1+\fr{4n}{(k-M)}\right)^{1/2} m  \alpha\left(\left[\fr{n}{2m}\right]\right).
\label{lem1eq2}
\end{multline}
Для произвольного набора центрированных случайных величин $\xi_1, \ldots, \xi_l$ 
с~конечными дисперсиями справедливо
$$
\e(\xi_1 + \cdots + \xi_l)^2 \leq l\sum\limits_{i=1}^l \D \xi_i,
$$
откуда
$$
v_{0,\,1} \leq \left(\left[\fr{n}{2m}\right]+1\right) \sum\limits_{i=1}^n  p_i(1-p_i) \leq \fr{nM}{m}\,.
$$

Рассмотрим первое слагаемое в~(\ref{lem1eq2}). Пусть сначала $n(k-M)/(8mv_0) 
\hm\leq 1$. Так как функция~$B(\lambda)$ убывает по~$\lambda$ и $v_0 \hm\leq nM/m$, то

\noindent
\begin{multline*}
\fr{(k-M)^2}{32 v_0}\, B\left(\fr{n(k-M)}{8m v_0}\right) \geq \fr{(k-M)^2 m}{32nM}\, B(1) ={}\\
{}=
\fr{mM}{32n} \left(\fr{k}{M}-1\right)^2 B(1).
\end{multline*}
Если же $n(k-M)/(8mv_0)\hm > 1$, то, поскольку $\lambda B(\lambda)$ возрастает по~$\lambda$ при $\lambda\hm\geq1$,
\begin{multline*}
\fr{(k-M)^2}{32 v_0}\, B\left(\fr{n(k-M)}{8m v_0}\right) \geq  \fr{(k-M) m}{4n}\, B(1) ={}\\
{}=  \fr{mM}{4n} \left(\fr{k}{M}-1\right) B(1).
\end{multline*}
Объединяя данные результаты, с учетом $B(1)\hm>1/2$ получим

\noindent
\begin{multline*}
2 \exp\left\{-\fr{(k-M)^2}{32 v_0}\, B\left(\fr{n(k-M)}{8m  v_0}\right)\right\} \leq{}\\
{}\leq 2 \exp\left\{-\fr{mM}{64n}
\min\left\{\left(\fr{k}{M}-1\right)^2,\left(\fr{k}{M}- 1\right)\right\}\right\}
\end{multline*}
и~аналогично для слагаемого с~$v_1$.

Перейдем к поиску границ возможных значений~$M$. Вспомним, что в векторе $\mu\hm\in 
l_0[\eta_n]$ не более $k_n \hm= [\eta_n n] \leq bn$ ненулевых, а~следовательно, и 
не менее $n\hm-k_n$ нулевых компонент, откуда при $k\hm\geq \kappa_n$ для~$M$ получим 
оценку снизу:
\begin{multline*}
M = \sum\limits_{i=1}^n p_i \geq k_n \cdot 0 + (n-k_n)  \fr{kq_n}{n} 
\geq   \fr{(n-k_n)}{n}\, kq_n \geq{}\\
{}\geq (1-b)\kappa_n q_n.
\end{multline*}
С другой стороны,
$$
M \leq k_n \cdot 1 + (n-k_n)  \fr{kq_n}{n} =  k_n + \left(1-\fr{k_n}{n}\right) k q_n.
$$
Рассмотрим функцию 
$$
g(x) = \fr{x}{k_n \hm+ (1\hm-k_n/n) x q_n}\,.
$$
 Тогда $k/M \hm\geq g(k)$ для 
любого~$k$. Заметим, что функция~$g(x)$ возрастает по~$x$, поэтому при 
$k\hm\geq\kappa_n$ имеем
\begin{multline*}
g(k) \geq g(\kappa_n) =\fr{\kappa_n}{k_n+ (1-k_n/n) \kappa_n q_n} ={}\\
{}= \left(1- \gamma_n- \fr{k_n q_n}{n}\right)^{-1} > 1 + \gamma_n,
\end{multline*}
откуда
$$
\left(\fr{k}{M}-1\right) > \gamma_n.
$$
Также здесь показано, что $k>M$.

\columnbreak

Наконец, заметим, что неравенство 
$$
\left(\fr{k}{M}-1\right)^2 > \fr{k}{M}-1
$$
выполняется лишь в случае 
$$
\fr{k}{M}-1 > 1\,.
$$
 Но тогда тем более 
 $$
 \fr{k}{M}- 1>\gamma_n^2,
 $$
  откуда
$$
\min\left\{\left(\fr{k}{M}-1\right)^2,\left(\fr{k}{M}-1\right)\right\}  > \gamma_n^2.
$$

Объединяя выписанные неравенства, получим
\begin{multline*}
\p(N(t_k)\geq k) \leq 4 \exp\left\{-\fr{(1-b)m}{64n}\,  \kappa_n q_n  \gamma_n^2    \right\}+ {}\\
{}+ 22\left(1+\fr{4n}{(1-b)\kappa_n q_n 
\gamma_n}\right)^{1/2} m \alpha\left(\left[\fr{n}{2m}\right]\right),
\end{multline*}
что вместе с~(\ref{lem1eq1}) дает утверждение леммы.\hfill$\square$

\smallskip


Обозначим $T_1 = \sigma\sqrt{2\ln \eta_n^{-1}}$.

\smallskip

\noindent
\textbf{Лемма~2.}
\textit{Пусть $\eta_n\hm\geq n^{-\delta_1},$ $\delta_1\hm<1;$ $\mathrm{lim}\,\eta_n\hm=0;$ $q_n\hm\leq 1/2;$ 
$\liminf q_n \ln n \hm\geq 2 c_3 > 0;$ а~также существуют такие константы $c_1, 
c_2\hm>0,$ что $\alpha(k) \hm\leq c_1 k^{-1-(9/2)\delta_1/(1\hm-\delta_1)\hm-c_2}$ для 
любого $k\hm\in\mathbb{N}$. Тогда при} $n\hm\to\infty$
$$
\sup\limits_{\mu\in l_0 [\eta_n]} \p\left(\hat{t}_F \leq T_1\right) = o\left(\eta_n^2\right).
$$


\noindent
Д\,о\,к\,а\,з\,а\,т\,е\,л\,ь\,с\,т\,в\,о\,.\ \ Используя требование $q_n\hm\leq1/2$ и свойства 
квантилей нормального распределения~\cite{AdaptingFDR}, можно показать, что при 
всех достаточно больших~$n$ справедливо 
$$
t_{\kappa_n}\equiv \sigma z\left( \fr{q_n \kappa_n}{2n}\right) > T_1,
$$
 откуда
$$
\p\left(\hat{t}_F \leq T_1\right) \leq \p\left(\hat{t}_F \leq 
t_{\kappa_n}\right) = \p \left(\hat{k}_F \geq \kappa_n \right).
$$
Заметив, что $\gamma_n \hm> \ln^{-1} n$, $\kappa_n \hm> \eta_n n/2$, $q_n \hm> c_3 \ln^{-1} n$ 
для всех достаточно больших~$n$, и применив лемму~1 с $m\hm= n^{\delta_1} \ln^5 n$, получим требуемое.\hfill$\square$

\smallskip

Следующее утверждение доказано в~\cite{AdaptingFDR} для $\sigma\hm=1$ и $T_1 \hm\geq 
3^{1/4}$, приведенное ниже обобщение элементарно.


\smallskip

\noindent
\textbf{Лемма~3.}
\textit{
Пусть $\hat{t}$~--- произвольный случайный порог, $\eta_n\hm\leq b\hm<1$, $x_i\sim 
N(\mu_i, \sigma^2)$, $(\hat{\mu})_i \hm= p(x_i, \hat{t})$, $i\hm=1,\ldots,n$. Тогда с 
некоторой константой} $c \hm\equiv c(\sigma, b)$
$$
\e \left|\left| \hat{\mu} - \mu\right|\right|^2 \Ik(\hat{t}\leq T_1) \leq c 
 T_1^2  n \left(\p (\hat{t}\leq T_1)\right)^{1/2}.
$$



Перейдем, наконец, к основному утверждению работы.


\smallskip

\noindent
\textbf{Теорема~2.}
\textit{Пусть выполнены требования леммы}~2. \textit{При} $n\hm\to\infty$
$$
\sup\limits_{\mu\in l_0[\eta_n]}\rho(\hat{\mu}_F, \mu)  \leq n \eta_n T_U^2 (1+o(1)).
$$


\noindent
Д\,о\,к\,а\,з\,а\,т\,е\,л\,ь\,с\,т\,в\,о\,.\ \ Пусть $\mu \hm\in l_0[\eta_n]$. Имеем
\begin{multline}
\label{risk1}
\rho(\hat{\mu}_F, \mu) = \e \left|\left| \hat{\mu}_F - \mu\right|\right|^2 
\Ik(\hat{t}_F\leq T_1) + {}\\
{}+\e \left|\left| \hat{\mu}_F - \mu\right|\right|^2 
\Ik(\hat{t}_F > T_1).
\end{multline}
Используя леммы~2 и~3, для первого слагаемого в~(\ref{risk1}) 
получим
\begin{equation}
\label{proof1}
\e \left|\left| \hat{\mu}_F - \mu\right|\right|^2 \Ik(\hat{t}_F\leq T_1) \leq n 
\, o(\eta_n) \ln \eta_n^{-1}.
\end{equation}
Заметим, что
\begin{equation*}
\left(p(x_i, t)-\mu_i\right)^2 =
 \begin{cases}
   (x_i-\mu_i)^2, & |x_i| > t;\\
   \mu_i^2, & |x_i| \leq t.
 \end{cases}
\end{equation*}
Отсюда для второго слагаемого в~(\ref{risk1})
\begin{multline*}
\e \left|\left| \hat{\mu}_F - \mu\right|\right|^2 \Ik(\hat{t}_F > T_1) = {}\\
{}=
\e  \sum\limits_{i=1}^n \left(p(x_i, \hat{t}_F)-\mu_i\right)^2  \Ik(T_1 < \hat{t}_F  
\leq T_U) ={}\\
{}= \e  \sum\limits_{i=1}^n \left((x_i-\mu_i)^2 \Ik(|x_i| >\hat{t}_F) + {}\right.\\
\left.{}+\mu_i^2 
\Ik(|x_i| \leq \hat{t}_F)\right) \Ik(T_1 < \hat{t}_F  \leq T_U) \leq{}\\
\!{}\leq \e  \sum\limits_{i=1}^n \left((x_i-\mu_i)^2 \Ik(|x_i| >T_1) + \mu_i^2 
\Ik(|x_i| \leq T_U)\right) \equiv{}\\
{}\equiv E_1+E_2,
\end{multline*}
где
\begin{multline*}
E_1 ={}\\
\!{}= \e \!\! \sum\limits_{i : |\mu_i| > 0}\!\!\! \left((x_i-\mu_i)^2 \Ik(|x_i| >T_1) + 
\mu_i^2 \Ik(|x_i| \leq T_U)\right);\hspace*{-6.19644pt}
\end{multline*}
$$
E_2 = \e  \sum\limits_{i : |\mu_i| =0} x_i^2 \Ik(|x_i| >T_1).
$$
Пусть $\xi \sim N(0,1)$, $x\hm>0$, тогда
$$
\e \xi^2 \Ik
(|\xi|>x)  \leq 2\left(x + \fr{1}{x}\right)\varphi(x),
$$
где использовано неравенство $1\hm-\Phi(x)\hm\leq\varphi(x)/x$, $x\hm>0$ ($\Phi(x)$ 
и~$\varphi(x)$~--- соответственно функция распределения и плот\-ность~$N(0,1)$). 
Отсюда

\columnbreak

\noindent
\begin{multline}
\label{proof2}
E_2 \leq \sqrt{\fr{2}{\pi}} \, n \fr{T_1}{\sigma}\, e^{-T_1^2/(2\sigma^2)}(1 + o(1)) = {}\\
{}=
O\left(n \eta_n \sqrt{\ln \eta_n^{-1}}\right). 
\end{multline}
Пусть далее $\xi \sim N(\mu,\sigma)$, тогда если $|\mu| \hm\leq T_U$, то $\mu^2 \p 
(|\xi|\hm\leq T_U) \hm\leq T_U^2$.
Если же $\mu \hm> T_U$ (для $\mu\hm < -T_U$ аналогично), используя $2(1\hm-\Phi(x))\hm \leq 
e^{-x^2/2}$ для $x\hm\geq 0$, получим
\begin{multline*}
\mu^2 \p (|\xi|\leq T_U) < \mu^2 \left(1-\Phi\left(\fr{\mu- T_U}{\sigma}\right)\right) \leq{}\\
{}\leq \fr{\mu^2}{2} e^{-(\mu-T_U)^2/(2\sigma^2)}  \leq \fr{T_U^2}{2}+O(T_U),
\end{multline*}
где последнее неравенство можно получить, исследуя выражение в левой части на 
экстремум по~$\mu$.
Из приведенных соотношений следует, что
\begin{multline}
\label{proof3}
E_1 \leq n\eta_n \sigma^2 + \sum\limits_{i : |\mu_i| > 0} \mu_i^2 \p (|x_i|\leq 
T_U)\leq{}\\
{}\leq  n \eta_n T_U^2 (1+o(1)).
\end{multline}
Объединяя~(\ref{proof1})--(\ref{proof3}), получаем утверждение тео\-ре\-мы.\hfill$\square$

\smallskip

От степени разреженности вектора~$\mu$ (ско\-рости убывания~$\eta_n$) зависит 
асимптотический порядок верхней границы риска, полученной в тео\-ре\-ме~2. Например, при $\eta_n\hm = n^{-\delta}$, $\delta\hm\in(0,1)$, получим
$$
\sup\limits_{\mu\in l_0[\eta_n]}\rho(\hat{\mu}_F, \mu)  \leq 2\sigma^2 \, 
n^{1-\delta} \ln n \, (1+o(1));
$$
если же $\eta_n\hm = (\ln n)^{-r}$, $r\hm>0$, то
$$
\sup\limits_{\mu\in l_0[\eta_n]}\rho(\hat{\mu}_F, \mu)  \leq 2\sigma^2 \, n 
(\ln n)^{1-r} \, (1+o(1)).
$$


{\small\frenchspacing
 {%\baselineskip=10.8pt
 %\addcontentsline{toc}{section}{References}
 \begin{thebibliography}{9}
\bibitem{MultipleTesting}
\Au{Menyhart~O., Weltz~B., Gyorffy~B.}
MultipleTesting.com: A~tool for life science researchers for multiple hypothesis 
testing correction~// PLoS One, 2021. Vol.~16. No.\,6. Art.~0245824.

\bibitem{FDRImage}
\Au{Krylov V.\,A., Moser~G., Serpico~S.\,B., Zerubia~J.}
False discovery rate approach to unsupervised image change detection~// IEEE 
T. Image Process., 2016. Vol.~25. No.~10. P.~4704--4718.

\bibitem{AdaptingFDR}
\Au{Abramovich~F., Benjamini~Y., Donoho~D., Johnstone~I.}
Adapting to unknown sparsity by controlling the false discovery rate~//  
Ann. Stat., 2006. Vol.~34. No.\,2. P.~584--653.

\bibitem{ResultsOnFDRUnderDependence}
\Au{Farcomeni~A.\/}
Some results on the control of the false discovery rate under dependence~// 
Scand. J. Stat., 2007. Vol.~34. No.\,2. P.~275--297.

\bibitem{spatialAdaptation}
\Au{Donoho~D., Johnstone~I.}
Ideal spatial adaptation via wavelet shrinkage~// Biometrika, 1994. Vol.~81. 
No.\,3. P.~425--455.

\bibitem{AdaptingSURE}
\Au{Donoho D., Johnstone~I.\,M.}
Adapting to unknown smoothness via wavelet shrinkage~// J.~Am. Stat. Assoc., 
1995. Vol.~90. P.~1200--1224.

\bibitem{ExactRisk}
\Au{Marron J.\,S., Adak S., Johnstone~I.\,M., Neumann~M.\,H., Patil~ P.}
Exact risk analysis of wavelet regression~// J.~Comput. Graph. Stat., 1998. 
Vol.~7. P.~278--309.

\bibitem{Bosq}
\Au{Bosq~D.}
Nonparametric statistics for stochastic processes: Estimation and prediction.~--- 
Lecture notes in statistics ser.~--- New York, NY, USA: Springer, 1996. Vol.~110. 
188~p.

\bibitem{Pollard}
\Au{Pollard~D.}
Convergence of stochastic processes.~--- Springer ser. in statistics.~--- New 
York, NY, USA: Springer, 1984. 215~p.
\end{thebibliography}

 }
 }

\end{multicols}

\vspace*{-6pt}

\hfill{\small\textit{Поступила в~редакцию 05.12.22}}

\vspace*{8pt}

%\pagebreak

%\newpage

%\vspace*{-28pt}

\hrule

\vspace*{2pt}

\hrule

%\vspace*{-2pt}

\def\tit{MEAN-SQUARE RISK OF~THE~FDR PROCEDURE\\ UNDER~WEAK DEPENDENCE}


\def\titkol{Mean-square risk of~the~FDR procedure under~weak dependence}


\def\aut{M.\,O.~Vorontsov$^{1,2}$ and~O.\,V.~Shestakov$^{1,2,3}$}

\def\autkol{M.\,O.~Vorontsov and~O.\,V.~Shestakov}

\titel{\tit}{\aut}{\autkol}{\titkol}

\vspace*{-10pt}


\noindent
$^{1}$M.\,V.~Lomonosov Moscow State University, 1-52~Leninskie Gory, GSP-1, Moscow 119991, Russian Federation

\noindent
$^{2}$Moscow Center for Fundamental and Applied Mathematics, M.\,V.~Lomonosov Moscow State University,\linebreak
$\hphantom{^1}$1~Leninskie Gory, GSP-1, Moscow 119991, Russian Federation

\noindent
$^{3}$Federal Research Center ``Computer Science and Control'' of the Russian Academy of Sciences, 44-2~Vavilov\linebreak
$\hphantom{^1}$Str., Moscow 119333, Russian Federation


\def\leftfootline{\small{\textbf{\thepage}
\hfill INFORMATIKA I EE PRIMENENIYA~--- INFORMATICS AND
APPLICATIONS\ \ \ 2023\ \ \ volume~17\ \ \ issue\ 2}
}%
 \def\rightfootline{\small{INFORMATIKA I EE PRIMENENIYA~---
INFORMATICS AND APPLICATIONS\ \ \ 2023\ \ \ volume~17\ \ \ issue\ 2
\hfill \textbf{\thepage}}}

\vspace*{3pt}



\Abste{In many application areas, the problem of processing large amounts of data arises. 
In this case, before processing, the data array is often subjected to some transformation leading to a~``sparse'' or ``economical'' 
representation in which the absolute value of most elements of the array is equal to zero (or sufficiently small). 
In addition, as a~result of interference when receiving and transmitting data, they become corrupted with noise and it is desirable 
to remove this noise during further processing. The resulting task is mathematically equivalent to some multiple hypothesis testing problems. 
Previously, to solve this problem under conditions of normality, independence, and sparsity of data, a procedure based on the method 
of controlling the average proportion of erroneously rejected hypotheses was proposed (False Discovery Rate, FDR). 
In this paper, the authors study the asymptotics of the mean-square risk of this procedure in the case of a~weak dependence in the data.}

\KWE{thresholding; multiple hypothesis testing; mean-square risk}



\DOI{10.14357/19922264230205}{AVJZDX}

%\vspace*{-11pt}

%\Ack
%\noindent

  

%\vspace*{4pt}

  \begin{multicols}{2}

\renewcommand{\bibname}{\protect\rmfamily References}
%\renewcommand{\bibname}{\large\protect\rm References}

{\small\frenchspacing
 {%\baselineskip=10.8pt
 \addcontentsline{toc}{section}{References}
 \begin{thebibliography}{9} 
\bibitem{1-she-1}
\Aue{Menyhart, O., B.~Weltz, and B.~Gyorffy.} 2021.
MultipleTesting.com: A~tool for life science researchers for multiple hypothesis testing correction. \textit{PLoS One} 16(6):0245824. doi: 10.1371/journal.pone.0245824.

\bibitem{2-she-1}
\Aue{Krylov, V.\,A., G.~Moser, S.\,B.~Serpico, and J.~Zerubia.} 2016.
False discovery rate approach to unsupervised image change detection. \textit{IEEE T. Image Process.} 25(10):4704--4718. doi: 10.1109/TIP.2016.2593340.
\bibitem{3-she-1}
\Aue{Abramovich, F., Y.~Benjamini, D.~Donoho, and I.\,M.~Johnstone.} 2006. 
Adapting to unknown sparsity by controlling the false discovery rate. \textit{Ann. Stat.} 34(2):584--653. doi: 10.1214/009053606000000074.

\bibitem{4-she-1}
\Aue{Farcomeni, A.} 2007. 
Some results on the control of the false discovery rate under dependence. \textit{Scand. J. Stat.} 34(2):275--297.
doi: 10.1111/j.1467-9469.2006.00530.x.

\bibitem{5-she-1}
\Aue{Donoho, D., and I.\,M.~Johnstone.} 1994. 
Ideal spatial adaptation via wavelet shrinkage. \textit{Biometrika} 81(3):425--455. doi: 10.1093/biomet/81.3.425.

\bibitem{6-she-1}
\Aue{Donoho, D., and I.\,M.~Johnstone.} 1995.
Adapting to unknown smoothness via wavelet shrinkage. \textit{J.~Am. Stat. Assoc.} 90(432):1200--1224. doi: 10.1080/01621459. 1995.10476626.

\bibitem{7-she-1}
\Aue{Marron, J.\,S., S.~Adak, I.\,M.~Johnstone, M.\,H.~Neumann, and P.~Patil.} 1998. 
Exact risk analysis of wavelet regression. \textit{J.~Comput. Graph. Stat.} 7(3):278--309. doi: 10.1080/ 10618600.1998.10474777.

\bibitem{8-she-1}
\Aue{Bosq, D.} 1996.
\textit{Nonparametric statistics for stochastic processes: Estimation and prediction}. Lecture notes in statistics ser. New York, NY: Springer Verlag. 188~p.


\bibitem{9-she-1}
\Aue{Pollard, D.} 1984. 
\textit{Convergence of stochastic processes.} Springer ser. in statistics. New York, NY: Springer. 215~p.
\end{thebibliography}

 }
 }

\end{multicols}

\vspace*{-6pt}

\hfill{\small\textit{Received December 5, 2022}} 
\Contr

\noindent
\textbf{Vorontsov Mikhail O.} (b.\ 1996)~--- PhD student, Department of Mathematical Statistics, Faculty of Computational Mathematics and Cybernetics, 
M.\,V.~Lomonosov Moscow State University, 1-52~Leninskie Gory, GSP-1, Moscow 119991, Russian Federation;
mathematician,  Moscow Center for Fundamental and Applied Mathematics, 
M.\,V.~Lomonosov Moscow State University, 1~Leninskie Gory, GSP-1, Moscow 119991, Russian Federation;
\mbox{m.vtsov@mail.ru}

\vspace*{5pt}

\noindent
\textbf{Shestakov Oleg V.} (b.\ 1976)~--- Doctor of Science in physics and mathematics, professor, Department of Mathematical Statistics, 
Faculty of Computational Mathematics and Cybernetics, M.\,V.~Lomonosov Moscow State University, 1-52~Leninskie Gory, GSP-1, Moscow 119991, 
Russian Federation; senior scientist, Institute of Informatics Problems, Federal Research Center ``Computer Science and Control'' of 
the Russian Academy of Sciences, 44-2~Vavilov Str., Moscow 119333, Russian Federation; leading scientist, Moscow Center for Fundamental and Applied Mathematics, 
M.\,V.~Lomonosov Moscow State University, 1~Leninskie Gory, GSP-1, Moscow 119991, Russian Federation; \mbox{oshestakov@cs.msu.su}
\label{end\stat}

\renewcommand{\bibname}{\protect\rm Литература} 