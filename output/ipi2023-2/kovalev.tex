\def\stat{kovalev}

\def\tit{МОНАДА ДИАГРАММ КАК МАТЕМАТИЧЕСКАЯ МЕТАМОДЕЛЬ СИСТЕМНОЙ 
ИНЖЕНЕРИИ}

\def\titkol{Монада диаграмм как математическая метамодель системной 
инженерии}

\def\aut{С.\,П.~Ковалёв$^1$}

\def\autkol{С.\,П.~Ковалёв}

\titel{\tit}{\aut}{\autkol}{\titkol}

\index{Ковалёв С.\,П.}
\index{Kovalyov S.\,P.}


%{\renewcommand{\thefootnote}{\fnsymbol{footnote}} \footnotetext[1]
%{Исследование выполнено за счет гранта Российского научного фонда (проект 22-28-00588). Работа 
%РАН, Москва).}}


\renewcommand{\thefootnote}{\arabic{footnote}}
\footnotetext[1]{Институт проблем управления им.\ В.\,А.~Трапезникова Российской академии наук, 
\mbox{kovalyov@sibnet.ru}}

%\vspace*{-12pt}





  
  \Abst{Рассматриваются вопросы разработки перспективных математических методов 
системной инженерии, способных лечь в~основу компьютерных инструментов 
автоматического синтеза и~анализа систем и~процессов. В~русле современных тенденций 
в~качестве аппарата для этих методов выбрана теория категорий. Ее применение 
отталкивается от представления структуры систем, процессов, требований и~других 
результатов системного проектирования диаграммами в~категориях, объектами которых 
служат алгебраические модели составных частей, а~морфизмы описывают взаимосвязи 
между частями. При помощи фундаментальной уплощающей конструкции Гротендика 
описано явное построение категорий диаграмм, монады диаграмм, монады и~комонады 
диаграмм с~отмеченной точкой. Указаны области приложения этих конструкций 
в~процедурах системной инженерии. Предложен подход к~реализации 
высокоавтоматизированных технологий типа порождающего проектирования для сложных 
многоуровневых сис\-тем.}
  
  \KW{теория категорий; монада диаграмм; конструкция Гротендика; копредел; системная 
инженерия; система систем; порождающее проектирование}

\DOI{10.14357/19922264230202}{FDUVDU}
  
\vspace*{-3pt}


\vskip 10pt plus 9pt minus 6pt

\thispagestyle{headings}

\begin{multicols}{2}

\label{st\stat}
  
\section{Введение}

%\vspace*{-3pt}

     Эффективность традиционных узкопрофильных инженерных дисциплин 
(механика, гидравлика, электроника и~др.), в~особенности в~условиях 
циф\-ро\-ви\-за\-ции, обусловлена широким применением математического аппарата, 
включающего аналитическую геометрию, дифференциальные уравнения, 
математическое программирование и~др. Однако для решения задач системной 
инженерии математические методы развиты слабо, они проигрывают 
<<практикам>> и~<<техникам>>~[1]. Поэтому успешные системные проекты 
плохо поддаются тиражированию и~сборке в~мегапроекты: не хватает 
формальных абстрактных схем выполненных процедур и~правил расчета 
влияния контекста на достигнутые результаты. Тем более трудно передать 
проектирование систем компьютеру: приходится ограничиваться средствами 
редактирования и~несложного анализа <<наглядных>> цифровых схем, 
изображающих придуманные инженером структуры систем и~процессов.
     
     В ряде публикаций (см., например,~[2--4]) предлагается решить эту 
проблему путем привлечения теории категорий~--- раздела высшей алгебры, 
направленного на унифицированное представление объектов различной 
природы и~отношений между ними. В~основе приложения теории категорий 
лежит представление структуры систем, процессов, требований и~др.\ 
диаграммами в~категориях типа <<каталогов>>, объектами которых служат 
ал\-геб\-ра\-и\-че\-ские модели составных частей, а морфизмы описывают те или иные 
взаимосвязи между частями. Известны ка\-те\-го\-рии-ка\-та\-ло\-ги 
геометрических форм, сценариев поведения частей, энергетических ресурсов 
и~т.\,д.~[5].
     
     Однако диаграммы чаще всего рассматриваются фрагментарно 
и~специфично для конкретного прикладного контекста. Отрывочность 
и~разнобой наблюдаются даже на уровне метамодели~--- языка, на котором 
описываются и~верифицируются свойства и~преобразования диаграмм в~ходе 
моделирования объектов и~процедур системной инженерии. При этом 
терминологически выверенные глубокие исследования диаграммных 
конструкций, написанные <<чистыми>> математиками, совершенно 
непостижимы для инженеров, да и~не очень им интересны.

\begin{figure*}[b]     %fig1-2
\vspace*{-6pt}
\begin{minipage}[b]{80mm}
     \begin{center}
   \mbox{%
\epsfxsize=23.362mm 
\epsfbox{kov-1.eps}
}
\end{center}
\vspace*{-12pt}
\Caption{Уплощающая конструкция Гротендика}
\end{minipage}
\hfill
\begin{minipage}[b]{80mm}
 \begin{center}
   \mbox{%
\epsfxsize=25.998mm 
\epsfbox{kov-2.eps}
}

\end{center}
\vspace*{-12pt}
\Caption{Канонический функтор формы диаграмм}
\end{minipage}
\end{figure*}
     
     Настоящая работа нацелена на преодоление этого недостатка, вводя 
универсальный (в~строгом тео\-ре\-ти\-ко-ка\-те\-го\-ри\-аль\-ном смысле) язык, 
основанный на известной конструкции монады диаграмм~[6, 7]. Основным 
рабочим инструментом служат элементарные универсальные конструкции 
в~<<категории>> \textbf{CAT}, состоящей из всех категорий и~всех функторов. 
Такие конструкции (в~том числе компоненты рассматриваемых в~работе 
монад) определены с~точ\-ностью до изоморфизма, но для ясности изложения 
предполагается, что у~них задан некоторый канонический вид. Описано явное 
построение и~об\-ласти приложения категорий диаграмм, монады диаграмм, 
монады и~комонады диаграмм с~отмеченной точкой.

\section{Категории структур систем}

     Предполагается, что читатель знаком с~основами теории категорий. 
Используются элементарные теоретико-категориальные конструкции 
и~обозначения, введенные в~работах~\cite{7-kov, 8-kov}. Начнем с~конструкции 
универсального расслоения (universal bundle) для категорий~--- это 
канонический функтор $\mathbf{B}: \dot{\mathbf{C}}\mathbf{AT} \hm\to 
\mathbf{CAT}$, где через $\dot{\mathbf{C}}\mathbf{AT}$ обозначена 
<<категория>> всех категорий с~отмеченной точкой, в~которой объектом 
служит любая пара $(A, C)$, $C \hm\in \mathrm{Ob}\,\mathbf{CAT}$, $A \hm\in 
\mathrm{Ob}\,C$, а~морфизмом $(A, C)$ в~$(A^\prime, C^\prime)$ служит любая 
пара ($f : GA \hm\to  A^\prime, G : C \hm\to C^\prime$); функтор~$\mathbf{B}$ 
<<забывает>> отмеченную точку. Для произвольного функтора 
$F : D \hm\to\mathbf{CAT}$ декартов квадрат с~универсальным расслоением 
из\-вес\-тен как уплощающая конструкция Гротендика $\int F$ (Grothendieck 
flattening construction)~\cite{9-kov} (рис.~1).





     В явном виде объектом категории $\int F$ служит любая пара $(A, X)$, $X 
\hm\in \mathrm{Ob} \,D$, $A \hm\in \mathrm{Ob}\, FX$, а морфизмом пары $(A, X)$ 
в~$(A^\prime, X^\prime)$ служит любая пара ($f : (Fg)A \hm\to  A^\prime, g : X 
\hm\to  X^\prime$) (с~законом композиции вида $(f, g) \circ (h, q) \hm= (f \circ 
(Fg)h, g \circ q)$). Можно принять такое описание конструкции Гротендика за 
ее определение~\cite[п.~12.2.10]{10-kov} (и~построить универсальное 
расслоение как $\int 1_{\mathbf{CAT}}$).
     
     Отображение $(A, X) \hm\mapsto X$ задает канонический 
<<забывающий>> функтор из $\int F$ в~$D$, представленный левой 
вертикальной стрелкой в~вышеприведенном декартовом квадрате. Полный 
прообраз (декартов квадрат) этого функтора относительно любого элемента 
$\ulcorner X\urcorner: \bm{1}\hm\to D$ (где~$\bm{1}$~--- сингулярная 
категория, терминальный объект в~$\mathbf{CAT}$) задает каноническое 
вложение $\bm{i}_X : FX \hm\hookrightarrow \int F : A \hm\mapsto (A, X)$, $f 
\hm\mapsto  (f, 1_X)$. В~сумме получается вложение $[\bm{i}_X]_{X \in \mathrm{Ob}\,D} : 
\coprod_{X \in \mathrm{Ob}\,D} FX \hm\hookrightarrow \int F$, биективное на объектах. 
Если категория~$D$ дискретна, то это суммарное вложение является 
изоморфизмом.
     
     Для контравариантного функтора $F : D^{\mathrm{op}}\hm\to \mathbf{CAT}$ 
в~качестве правой вертикальной стрелки вышеприведенного декартового 
квадрата используется универсальное <<op-рас\-сло\-ение>>~$\mathbf{B} : 
\dot{\mathbf{C}}\mathbf{AT}^{\mathrm{op}} \hm\to 
\mathbf{CAT}^{\mathrm{op}}$ (а~функтор~$F$ рассматривается как 
действующий из $D$ в~$\mathbf{CAT}^{\mathrm{op}}$). Применим такую\linebreak \mbox{контравариантную} конструкцию 
Гротендика к~функтору $C^-: \mathbf{Cat}^{\mathrm{op}} \hm\to 
\mathbf{CAT} : X \hm\mapsto C^X$ для некоторой категории~$C$. 
(Универсальный характер этого функтора вытекает из того, что 
экспонента~$C^X$ вычисляется посредством эндофунктора, правого 
сопряженного к~эндофунктору произведения $C\hm \mapsto X \times 
C$~\cite[\S\,IV.6]{8-kov}, которое, в~свою очередь, сводится к~декартову 
квадрату функторов $!_X : X \hm\to \bm{1}$ и~$!_C : C\hm\to \bm{1}$, каждый из 
которых определен единственным образом.) В~результате получается 
категория диаграмм, которая обозначается через $\mathbf{D}C$ и~служит 
<<нестрогим ко-пополнением>> (lax cocompletion)~[6] категории~$C$. Класс 
объектов категории $\mathbf{D}C$ состоит из всех $C$-диа\-грамм, 
а~морфизмом диаграммы $\Delta  : X \hm\to  C$ в~$\Delta^\prime : X^\prime\hm\to  
C$ служит любая пара вида $\langle \gamma, \mathfrak{f}\rangle$, состоящая из 
функтора $\mathfrak{f} : X \hm\to  X^\prime$ и~естественного 
преобразования~$\gamma : \Delta \hm\to \Delta^\prime \mathfrak{f}$; закон 
композиции имеет вид $\langle \gamma, \mathfrak{f}\rangle  \circ \langle \varphi, 
\mathfrak{g}\rangle \hm= \langle \gamma\mathfrak{g}\circ \varphi, 
\mathfrak{fg}\rangle$. Имеется каноническое вложение $[\bm{i}_I]_{I \in \mathrm{Ob}\, 
\mathbf{Cat}} : \coprod_{I \in \mathrm{Ob}\,\mathbf{Cat}} C^I \hm\hookrightarrow \mathbf{D}C$, 
биективное на объектах.
     
     Процедура построения категории $\mathbf{D}C $ функториальна по $C$: 
любой функтор~$F : C\hm\to  C^\prime$ индуцирует функтор
     $$
     F- : \mathbf{D}C  \to \mathbf{D}C^\prime : \Delta \mapsto F\Delta, \langle 
\gamma, \mathfrak{f}\rangle \mapsto \langle F\gamma, \mathfrak{f}\rangle.
     $$
Тем самым определен эндофунктор
$$
     \mathbf{D} : \mathbf{CAT}\to \mathbf{CAT} : C \mapsto  
\mathbf{D}C ,\ F \mapsto F-
     $$
(так что можно построить категорию \textit{всех} диаграмм $\int \mathbf{D}$). 
Отметим, что $\mathbf{D1}\hm\cong \mathbf{Cat}$ и~левая вертикальная 
стрелка в~декартовом квадрате конструкции Гротендика для $\mathbf{D}C$ 
представляет собой канонический функтор формы диаграмм (рис.~2):
$$
     \mathbf{D}!_C : \mathbf{D}C  \to \mathbf{Cat} : \Delta 
\mapsto \mathrm{dom}\,\Delta, \langle \gamma, 
\mathfrak{f}\rangle \mapsto \mathfrak{f}.
     $$
     
     \begin{figure*} %fig3
    \vspace*{1pt}
      \begin{center}
   \mbox{%
\epsfxsize=106.769mm 
\epsfbox{kov-3.eps}
}
\end{center}
\vspace*{-12pt}
\Caption{Отрисовка~--- умножение в монаде диаграмм}
\vspace*{-6pt}
\end{figure*}
     
     В приложениях в~области системной инженерии диаграммы 
представляют структуры систем,\linebreak а~их морфизмы описывают структурные 
преобразования систем на алгебраическом языке~[5]. Функтор~$\mathbf{D}$ 
описывает естественный переход от каталогов объектов к~каталогам структур 
сис\-тем, \mbox{которые} можно составить из объектов. Подходящие подкатегории 
в~$\mathbf{D}C $ могут служить пространствами проектирования (design 
space) для автоматического поиска суб- и~Па\-ре\-то-оп\-ти\-маль\-ных структур 
сис\-тем. Для такой оптимизации целевые функции, изначально заданные 
заинтересованными сторонами систем, преобразуются в~функторы, 
действующие из пространств проектирования в~линейно упорядоченные 
множества значений, рассматриваемые как категории~\cite{11-kov}. Для 
эффективной навигации в~пространствах проектирования вдоль морфизмов 
диаграмм могут привлекаться средства компьютерной алгебры. Открывается 
возможность реализации высокоавтоматизированных технологий типа 
порождающего проектирования (generative design)~[12] для сложных систем.

\vspace*{-6pt}

\section{Монада диаграмм}

     Как известно~[6, 7], функтор $\mathbf{D}$ определяет в~$\mathbf{CAT}$ 
монаду, которая по ряду свойств аналогична монаде степени $\langle 2^-, \{-\}, 
\cup\rangle$ в~категории множеств $\mathbf{Set}$. Компонента единицы 
монады~$\mathbf{ }$D, соответствующая категории~$C$~--- это полное 
вложение~$C$ в~$\mathbf{D}C$, которое переводит объект~$A$  
в~диа\-грам\-му-точ\-ку~$\ulcorner A\urcorner$. Умножение задает 
<<отрисовку>> $\mathbf{D}C $-диа\-грам\-мы в~виде $C$-диа\-грам\-мы: 
отрисовка $\mathbf{K}\Xi$ диаграммы~$\Xi : I \hm \to \mathbf{D}C $ 
порождается заменой каж\-дой точки $i \hm\in \mathbf{Ob}\, I$  
$C$-диа\-грам\-мой $\Xi i$ и~разделением каждого морфизма 
диаграмм~$\Xi k$, $k\hm\in \mathrm{Mor}\,I$, на составляющие его 
морфизмы из категории~$C$. Таким образом, точка диаграммы 
$\mathbf{K}\Xi$~--- это пара $(l, i)$, $i \hm\in \mathrm{Ob} \,I$, $l \hm\in 
\mathrm{Ob}\,\mathrm{dom}\, \Xi i$, помеченная объектом $(\Xi i)l$, 
а~стрелка из нее в~$(l^\prime, i^\prime)$~--- это пара $(q : \mathfrak{h}l \hm\to  
l^\prime, k : i \hm\to  i^\prime)$, помеченная морфизмом $(\Xi i^\prime)q \circ 
\theta_l$, где $\langle\theta, \mathfrak{h}\rangle \hm= \Xi k$ (рис.~3).
    

     Формально отрисовка вычисляется при помощи ковариантной 
конструкции Гротендика для композиции~$\Xi$ с~функтором формы 
$\mathbf{D}!_C$ и~каноническим вложением $\mathbf{Cat}$ в~$\mathbf{CAT}$. 
Приведем композицию соответствующих декартовых квадратов (рис.~4).
Здесь правый декартов квадрат задает универсальное расслоение для 
малых категорий. Центральный декартов квадрат приводит к~категории 
диаграмм с~отмеченной точкой $\dot{\mathbf{D}}C$~--- это конструкция 
Гротендика для конструкции Гротендика $\mathbf{D}C$. Ее объектом служит 
любая пара $(x, \Delta)$, состоящая из диаграммы $\Delta : X \hm\to C$ и~точки 
$x \hm\in \mathrm{Ob} X$, а~морфизмом такой пары в~$(x^\prime, \Delta^\prime : 
X^\prime \hm\to  C)$ служит любая тройка $\langle q, \gamma, 
\mathfrak{f}\rangle$, где $\langle \gamma, \mathfrak{f}\rangle : \Delta \hm\to  
\Delta^\prime$~--- морфизм диаграмм и~$q : \mathfrak{f}x \hm\to  x^\prime$~--- 
стрелка в~$X^\prime$; закон композиции морфизмов имеет вид:
$$
\langle q, 
\gamma, \mathfrak{f}\rangle \circ \langle r, \varphi, \mathfrak{g}\rangle\hm = 
\langle q \circ \mathfrak{f}r, \gamma \mathfrak{g} \circ \varphi, 
\mathfrak{fg}\rangle.
$$ 

\vspace*{-3pt}

\noindent
Имеется канонический функтор $\bm{b}_C : 
\dot{\mathbf{D}}C\hm\to \mathbf{D}C$, забывающий отмеченную точку. Кроме 
того, имеется функтор означения $\bm{e}_C : \dot{\mathbf{D}}C \hm\to  C$, 
переводящий пару $(x, \Delta)$ в~$\Delta x$ и~морфизм $\langle q, \gamma, 
\mathfrak{f}\rangle : (x, \Delta) \hm\to  (x^\prime, \Delta^\prime)$ 
в~$\Delta^\prime q \circ \gamma_x : \Delta x \hm\to  \Delta^\prime x^\prime$. 
Отложим более подробное рассмотрение конструкции $\dot{\mathbf{D}}C$ до 
следующего раздела; приведенных здесь сведений о~ней достаточно для 
проверки того, что левый декартов квадрат задает искомую отрисовку 
диаграммы~$\Xi$.


     
     Имеется вложение диаграмм $[\langle 1_{\Xi i}, \bm{i}_i : 
\mathrm{dom}\, \Xi i \hm\hookrightarrow 
\mathrm{dom}\,\mathbf{K}\Xi\rangle]_{i \in \mathrm{Ob}\, I} : \coprod_{i \in \mathrm{Ob}\, I} 
\Xi i \hm\hookrightarrow \mathbf{K}\Xi$, биективное на точках. Если 
диаграмма~$I$ дискретна, то это вложение является изоморфизмом, 
а~отрисовка~--- вершиной копредела диаграммы~$\Xi$.
     
     Определим отрисовку произвольного морфизма  
$\mathbf{D}C$-диа\-грамм~$\phi  : \Xi \hm\to \Xi^\prime$. Рассмотрим 
$\mathbf{DD}C$-диа\-грам\-му со схемой~$\bm{2}$ (это граф $0\hm\to  1$), 
единственная нетождественная стрелка которой помечена морфизмом~$\phi$. 
Двукратная отрисовка этой диаграммы дает C-диа\-грам\-му со схемой, 
снабженной забывающим функтором в~$\bm{2}$, прообраз нетождественной 
стрелки относительно которого позволяет восстановить морфизм 
$\mathbf{K}\phi : \mathbf{K}\Xi \hm\to \mathbf{K}\Xi^\prime$. 
В~явной\linebreak\vspace*{-12pt}

 { \begin{center}  %fig4
 \vspace*{6pt}
    \mbox{%
\epsfxsize=66.827mm 
\epsfbox{kov-4.eps}
}


\vspace*{6pt}


\noindent
{{\figurename~4}\ \ \small{Вычисление отрисовки
}}
\end{center}
}

%\vspace*{6pt}

\addtocounter{figure}{1}




 %
\noindent
форме пусть $\phi\hm = \langle \varphi, \mathfrak{g}\rangle$, где 
естественное преобразование $\varphi : \Xi\hm\to \Xi^\prime 
\mathfrak{g}$ составлено из компонентов $\varphi_i \hm= \langle \gamma^i, 
\mathfrak{f}^ i\rangle : \Xi i \hm\to  \Xi^\prime \mathfrak{g}i$, $i 
\hm\in\mathrm{Ob}\, I$. Рассмотрим функтор~$\mathfrak{e} : 
\mathrm{dom}\,\mathbf{K}\Xi \hm\to 
\mathrm{dom}\,\mathbf{K}\Xi^\prime$, сопоставляющий точке $(l, i)$ точку 
$(\mathfrak{f}^il, \mathfrak{g}i)$, а~стрелке $(q, k) : (l, i) \hm\to  (l^\prime, 
i^\prime)$~--- пару $(\mathfrak{f}^{i^\prime}q, \mathfrak{g}k)$, которая задает 
стрелку в~диаграмме $\mathbf{K}\Xi^\prime$, направленную из 
$\mathfrak{e}(l, i)$ в~$\mathfrak{e}(l^\prime, i^\prime)$, ввиду соотношения 
$\langle \gamma^{i^\prime}, \mathfrak{f}^{i^\prime}\rangle \circ \Xi k \hm= 
\Xi^\prime \mathfrak{g}k \circ \langle \gamma^i, \mathfrak{f}^i\rangle$. 
Семейство морфизмов $\gamma^i_l : (\Xi i)l \hm\to  (\Xi^\prime 
\mathfrak{g}i)(\mathfrak{f}^il)$, $(l, i) \hm\in 
\mathrm{Ob}\,\mathrm{dom}\,\mathbf{K}\Xi$, составляет естественное 
преобразование $\mathbf{K}\Xi$ 
в~$(\mathbf{K}\Xi^\prime)\mathfrak{e}$, которое в~паре 
с~функтором~$\mathfrak{e}$ и~образует морфизм~$\mathbf{K}\phi$. 
Непосредственно проверяется, что таким путем действительно задается 
функтор~$\mathbf{K}$ и~все такие функторы образуют естественное 
преобразование $\mathbf{DD}$ в~$\mathbf{D}$, удовле\-тво\-ря\-ющее аксиомам 
монады.
     
     Рассмотрим алгебры монады диаграмм. Свободные алгебры порождаются 
компонентами отрисовки~$\mathbf{K}$. Единица также вносит свой вклад: 
если ее компонента, соответствующая категории~$C$, имеет левый 
сопряженный функтор (а~правого сопряженного она иметь не может), то 
единица этого сопряжения состоит из копределов $C$-диа\-грамм, так что 
категория $C$ кополна и, более того, указанный левый сопряженный $\mathrm{colim}\,: 
\mathbf{D}C \hm\to  C$ задает алгебру~\cite{7-kov}. Поскольку функтор, 
сопряженный к~заданному, определяется однозначно с~точностью до 
изоморфизма~\cite[\S\,~IV.1]{8-kov}, получается, что конструкция копредела 
<<закодирована>> в~тривиальной процедуре построения одноточечных 
диаграмм.
     
     Другим примером алгебры монады~$\mathbf{D}$ служит $\int : 
\mathbf{D}(\mathbf{CAT}) \hm\to  \mathbf{CAT} : \Delta \hm\mapsto \int \Delta$, 
устроенная аналогично свободной алгебре над~$\bm{1}$. При помощи 
отрисовки строятся и~другие алгебры, неизоморфные ни копределам, ни 
свободным алгебрам. Например, обозначим через $\mathbf{PrCAT}$ полную 
подкатегорию в~$\mathbf{CAT}$, состоящую из всех тонких категорий 
(предпорядков). Она рефлективна: ее вложение в~$\mathbf{CAT}$ имеет левый 
сопряженный $\mathbf{P} : \mathbf{CAT} \hm\to \mathbf{PrCAT}$ 
с~тождественной коединицей (для произвольной категории~$C$,  
$\bm{P}C$~--- это класс $\mathrm{Ob}\,C$, предупорядоченный бинарным 
предикатом $\mathrm{Mor}(-, -) \not= \varnothing$). Далее, пусть 
$\Check{\mathbf{D}} : \mathbf{CAT}\hm\to \mathbf{ CAT}$~--- подфунктор 
в~$\mathbf{D}$, сопоставляющий категории~$C$ полную подкатегорию 
в~$\mathbf{D}C$, состоящую из всех тонких $C$-диа\-грамм, и~действующий 
на функторы так же, как~$\mathbf{D}$. Если~$C$~--- тонкая категория, то 
имеется алгебра $\bm{P}(\mathbf{K}-) : \mathbf{D}\Check{\mathbf{D}}C \hm\to 
\Check{\mathbf{D}}C$.
     
     Алгебры строятся и~при помощи морфизмов монады~$\mathbf{D}$ в~другие монады~\cite[упражнение~6.2.3(б)]{8-kov}. Так, $\mathbf{D}$ имеет две 
под\-мо\-на\-ды-ре\-трак\-та: одна из них сопоставляет произвольной категории 
$C$ категорию всех дискретных $C$-диа\-грамм, а другая~--- категорию всех  
$C$-диа\-грамм в~форме группоидов. Следовательно, если в~$C$ есть суммы, то 
существуют две в~общем случае неизоморфные друг другу алгебры над~$C$: 
одна сопоставляет произвольной \mbox{$C$-диа}\-грам\-ме сумму всех ее вершин, 
а~вторая~--- сумму всех вершин ее скелета.
     
     Что касается приложений монады диаграмм, то компонента ее единицы 
задает представление любого объекта как бесструктурной сингулярной 
системы, а левый сопряженный к~ней (при его наличии) представляет сборку 
систем в~цельные объекты. Функтор отрисовки строит детальную <<плоскую>> 
структуру многоуровневых систем, состоящих из систем (systems of systems, 
SoS), путем разрешения взаимосвязей верхнего уровня, причем 
ассоциативность умножения монады гарантирует независимость итоговой 
структуры от порядка анализа промежуточных уровней. Алгебры задают 
шаблоны различных процедур <<упаковки>> систем в~сложные объекты 
с~соблюдением условий естественности относительно  
модификаций~\cite{7-kov}. Из <<материала>> монады диаграмм строятся и~другие конструкции.

\section{Монада и~комонада систем систем}

     Существует сопряжение функторов, свя\-зы\-ва\-ющее конструкцию 
Гротендика с~функтором формы диаграмм~\cite{6-kov}. Рассмотрим категорию 
$\mathbf{CAT}/\mathbf{Cat}$, в~которой объектами служат все функторы вида 
$F : D \hm \to \mathbf{Cat}$, а морфизмом такого функтора в~$F^\prime 
 : D^\prime \hm\to \mathbf{Cat}$ служит любой функтор $G : D \hm\to D^\prime$, 
такой что $F^\prime G \hm= F$ (эту категорию можно построить посредством 
декартова квадрата в~$\mathbf{CAT}$~\cite{13-kov}). Поскольку $!_{C^\prime}H 
\hm= !_C$ для произвольного функтора $H : C \hm\to C^\prime$, функтор формы 
диаграмм индуцирует функтор
     $$
\mathbf{D}_! : \mathbf{CAT}\to \mathbf{CAT}/\mathbf{Cat} : 
C \mapsto \mathbf{D}!_C,\enskip  H \mapsto \mathbf{D}H.
$$

\setcounter{figure}{5}
\begin{figure*}[b] %fig6
\vspace*{1pt}
 \begin{center}
   \mbox{%
\epsfxsize=103.315mm 
\epsfbox{kov-6.eps}
}

\end{center}
\vspace*{-9pt}
\Caption{Коумножение в~комонаде диаграмм с~отмеченной точкой
}
\end{figure*}
     

     В свою очередь, любой функтор $F : D \hm\to \mathbf{Cat}$ в~композиции с~каноническим вложением категории $\mathbf{Cat}$ в~$\mathbf{CAT}$ дает 
функтор, обозначаемый через ${\mbox{\ptb{\^{\,}}}}\!F : D \hm\to 
\mathbf{CAT}$, для которого определена конструкция Гротендика $\int 
{\mbox{\!\!\!\ptb{\^{\,}}}}\!F$. Отображение $F \hm\mapsto 
\int{\mbox{\!\!\!\ptb{\^{\,}}}}\!F$ служит функцией объектов функтора $\int 
{\mbox{\!\!\!\ptb{\^{\,}}}} : \mathbf{CAT}/\mathbf{Cat}\hm\to  
\mathbf{CAT}$, который переводит морфизм~$G$ в~единственную стрелку из 
$\int {\mbox{\!\!\!\ptb{\^{\,}}}} \!F$ в~$\int {\mbox{\!\!\!\ptb{\^{\,}}}} \!F^\prime$, делающую коммутативной диаграмму, в~которой 
все пунктирные стрелки изображают ребра декартовых квадратов (рис.~5).



     В явном виде имеем 
\begin{multline*}
     \int {\mbox{\!\!\!\ptb{\^{\,}}}}\! G : \int{\mbox{\!\!\!\ptb{\^{\,}}}}\!F \to 
\int {\mbox{\!\!\!\ptb{\^{\,}}}} \!F^\prime : (A, X) \mapsto (A, GX),\\
 (f, g) \mapsto  (f, Gg).
   \end{multline*}
   
   { \begin{center}  %fig5
 \vspace*{-1pt}
     \mbox{%
\epsfxsize=49.793mm 
\epsfbox{kov-5.eps}
}

\vspace*{6pt}



\noindent
{{\figurename~5}\ \ \small{Конструкция Гротендика как функтор
}}
\end{center}
}

\vspace*{6pt}

\addtocounter{figure}{1}
  
     
     Как показывает диаграмма (см.\ рис.~4), описывающая 
построение отрисовки $\mathbf{K}\Xi$ (см.\ композицию правого 
и~центрального декартовых квадратов), категория $\int 
{\mbox{\!\!\!\ptb{\^{\,}}}}\! (\mathbf{D}!_C)$ совпадает с~категорией диаграмм с~отмеченной точкой $\dot{\mathbf{D}}C$. Тем самым определен эндофунктор
     $$
     \dot{\mathbf{D}} = \int 
{\mbox{\!\!\!\ptb{\^{\,}}}}\!\mathbf{D}_! : \mathbf{CAT}\to 
\mathbf{CAT} : C \mapsto \dot{\mathbf{D}}C\,,
     $$
действующий на функторы так же, как~$\mathbf{D}$, не затрагивая 
отмеченные точки диаграмм. Более того, функтор $\int 
{\mbox{\!\!\!\ptb{\^{\,}}}}$ сопряжен слева к~функтору~$\mathbf{D}_!$, 
причем коединица этого сопряжения~$\bm{e} : \int 
{\mbox{\!\!\!\ptb{\^{\,}}}} \!\mathbf{D}_! \hm= \dot{\mathbf{D}}\hm\to 
1_{\mathbf{CAT}}$ состоит из функторов означения $\bm{e}_C$, $C \hm\in 
\mathrm{Ob}\,\mathbf{CAT}$ (так что верхний треугольник на 
вышеупомянутой диаграмме~--- это частный случай треугольного тож\-дест\-ва 
сопряжения, что свидетельствует об универсальности конструкции отрисовки). 
Компонента единицы $\bm{t}: 1_{\mathbf{CAT}/\mathbf{Cat}} \hm\to  \mathbf{D}_! \int{\mbox{\!\!\!\ptb{\^{\,}}}}$, 
со\-от\-вет\-ст\-ву\-ющая функтору $F : D \hm\to \mathbf{Cat}$, 
представляет собой функтор $\bm{t}_F : D \hm\to \mathbf{D}\int 
{\mbox{\!\!\!\ptb{\^{\,}}}} \!F$, сопоставляющий объекту~$X$\linebreak диаграмму  
$\bm{i}_X : FX \hookrightarrow \int {\mbox{\!\!\!\ptb{\^{\,}}}}\! F$, а морфизму 
$g : X \hm\to  X^\prime$~--- морфизм $\langle\rho, Fg\rangle : \bm{i}_X \hm\to  
\bm{i}_{X^\prime}$, где естественное преобразование $\rho : \bm{i}_X \hm\to 
\bm{i}_{X^\prime}(Fg)$\linebreak со\-став\-ле\-но из компонентов $\rho_A\hm = (1_{(Fg)A}, g) : (A, 
X) \hm\to  ((Fg)A, X^\prime)$, $A \hm\in \mathrm{Ob}\,FX$; ясно, что 
$(\mathbf{D}!_{\int {\mbox{\!\!\!\ptb{\^{\,}}}} \!F})\bm{t}_F \hm= F$.

     Это сопряжение стандартным способом~\cite[\S\,VI.1]{8-kov} задает 
комонаду~$\dot{\mathbf{D}}$ в~$\mathbf{CAT}$, коединица которой состоит 
из функторов~$\bm{e}_C$, а~коумножение состоит из <<размножителей 
диаграмм>> вида

\noindent
\begin{multline*}
\mathbf{M} = \int {\mbox{\!\!\!\ptb{\^{\,}}}} \!\bm{t}_{\mathbf{D}!_C} : 
\dot{\mathbf{D}}C \to \dot{\mathbf{D}} \dot{\mathbf{D}}C : (x, \Delta : X \to C) 
\mapsto {}\\
{}\mapsto (x, \bm{i}_\Delta : X \hookrightarrow  \dot{\mathbf{D}}C), \langle q, 
\gamma, \mathfrak{f}\rangle \mapsto  \langle q, \gamma^{\mathbf{M}}, 
\mathfrak{f}\rangle,
\end{multline*}
где $\gamma^{\mathbf{M}} : \bm{i}_\Delta \hm\to  
\bm{i}_{\Delta^\prime}\mathfrak{f}$~--- естественное преобразование, 
сопоставляющее каждому $x \hm\in \mathrm{Ob}\,X$ морфизм $\langle 
1_{\mathfrak{f}x}, \gamma, \mathfrak{f}\rangle : (x, \Delta) \hm\to  (\mathfrak{f}x, 
\Delta^\prime)$. Тем самым~$\mathbf{M}$ размещает в~каждой точке~$x$ 
диаграммы~$\Delta$ копию~$\Delta$, у~которой точка~$x$ отмечена (рис.~6).


     Рассмотрим коалгебры комонады $\dot{\mathbf{D}}$. Воспользуемся 
конструкцией сравнивающего функтора~\cite[\S\,~VI.3]{8-kov}: таковым 
служит функтор~$K$, действующий из $\mathbf{CAT}/\mathbf{Cat}$ 
в~категорию коалгебр, сопоставляя функтору $F : D \hm\to \mathbf{Cat}$ 
коалгебру $\int {\mbox{\!\!\!\ptb{\^{\,}}}}\! \bm{t}_F: \int 
{\mbox{\!\!\!\ptb{\^{\,}}}} \!F \hm\to  \dot{\mathbf{D}} 
\int{\mbox{\!\!\!\ptb{\^{\,}}}} \!F$. Любая коалгебра 
комонады~$\dot{\mathbf{D}}$ изоморфна коалгебре такого вида~\cite{6-kov}. 
В~частности, $K(\mathbf{D}!_C) \hm= \mathbf{M} : \dot{\mathbf{D}}C \hm\to 
\dot{\mathbf{D}}\dot{\mathbf{D}}C$ (это косвободная коалгебра). А~если $F$ 
переводит любой объект в~$\bm{1}$, т.\,е.\ $F \hm= \ulcorner \bm{1}\urcorner 
!_D$, то конструкция Гротендика дает~$D$ и~коалгебра $K(\ulcorner 
\bm{1}\urcorner !_D) : D \hm\to  \dot{\mathbf{D}}D$ переводит объект~$A$ 
в~диа\-грам\-му-точ\-ку $\ulcorner A\urcorner$ с~единственно возможной 
отмеченной точкой. В~свою очередь, если $F$~--- это некоторая  
диа\-грам\-ма-точ\-ка $\ulcorner I\urcorner : \bm{1}\hm\to \mathbf{Cat}$, то $\int 
{\mbox{\!\!\!\ptb{\^{\,}}}} \ulcorner I\urcorner \hm = I$ и~$K(\ulcorner 
I\urcorner) : I \hm\to \dot{\mathbf{D}}I$: $i \mapsto (i, 1_I)$, $k \hm\mapsto \langle 
k, 1_{1I}, 1_I\rangle$.
     
     Кроме комонады, эндофунктор $\dot{\mathbf{D}}$ определяет в~$\mathbf{CAT}$ и~монаду, устроенную аналогично монаде диаграмм: 
компонента единицы совпадает с~$K(\ulcorner \bm{1}\urcorner !_C)$, 
а~умножение переводит $\dot{\mathbf{D}}C$-диа\-грам\-му~$\Xi$ 
с~отмеченной точкой~$i$ в~отрисовку $\mathbf{K}(\bm{b}_C\Xi)$ 
с~отмеченной точкой $(l, i)$, где $l$~--- отмеченная точка $\Xi i$ (т.\,е.\ 
$\Xi i \hm= (l, \bm{b}_C(\Xi i))$). Тем самым семейство функторов 
$\bm{b}_C$, $C \hm\in\mathrm{Ob}\,\mathbf{CAT}$, образует морфизм монады 
$\dot{\mathbf{D}}$ в~$\mathbf{D}$. В~частности, любая алгебра $a : 
\mathbf{D}C \hm\to  C$ монады~$\mathbf{D}$ порождает алгебру $a\bm{b}_C : 
\dot{\mathbf{D}}C \hm\to  C$ монады~$\dot{\mathbf{D}}$. Легко проверить, 
что алгебру задает также любой функтор означения $\bm{e}_C : 
\dot{\mathbf{D}}C \hm\to C$.
     
     С прикладной точки зрения конструкция $\dot{\mathbf{D}}$ формально 
выражает первый принцип системной инженерии: целевая система (system of 
interest, SoI) рассматривается в~контексте как элемент некоторой известной 
объемлющей системы~[14]. Именно из такого рассмотрения возникают 
требования к~целевой системе и~начинается ее жизненный цикл. А~структурная 
схема объемлющей системы с~со\-став\-ля\-ющи\-ми из каталога~$C$, снабженная 
указанием места целевой сис\-те\-мы в~ней~--- это и~есть объект категории 
$\dot{\mathbf{D}}C$. Разумеется, целевая сис\-те\-ма, в~свою очередь, служит 
объемлющей для своих под\-сис\-тем, так что возникают приложения конструкций 
вида $\dot{\mathbf{D}}^nC$, где $n \hm> 1$~--- чис\-ло рассматриваемых 
системных уровней, в~инженерии систем систем (SoS).
     
     Монада $\dot{\mathbf{D}}$ показывает, как корректно учесть\linebreak вхождение 
целевой системы при проведении с~объемлющими сис\-те\-ма\-ми процедур 
структурного анализа и~синтеза из разд.~3. 
А~комонада~$\dot{\mathbf{D}}$ предоставляет специфические средства для 
работы\linebreak с~системой в~контексте: коединица извлекает из объемлющей систему 
целевую, <<забывая>> все остальное, а~коумножение порождает самоподобные 
многоуровневые системы, в~которых каж\-дый элемент воспроизводит структуру 
предыдущего уровня с~сохранением своей иден\-тич\-ности.\linebreak Самоподобие 
относится к~ключевым свойствам сис\-тем~--- как природных, вплоть до 
Вселенной как целого~[15], так и~технических, например в~сфере 
телекоммуникаций~[16]. Генерация самоподобных структур расширяет 
возможности автоматического синтеза сис\-тем, в~том чис\-ле бионического типа.

\section{Заключение}
     
     При помощи универсальных конструкций в~$\mathbf{CAT}$ (в~основном 
декартовых квадратов, в~том чис\-ле конструкции Гротендика, и~сопряжений) 
в~настоящей работе построены математические\linebreak объекты, в~которых 
<<закодированы>> ключевые процедуры системной инженерии. Показана 
основополагающая роль монады диаграмм. Целесообразно рекомендовать ее и~производные \mbox{конструкции}
 к~реализации в~<<умных>> инструментах цифровой 
системной инженерии, способных автоматически генерировать оптимальные 
структуры систем и~процессов.
     
{\small\frenchspacing
 {%\baselineskip=10.8pt
 %\addcontentsline{toc}{section}{References}
 \begin{thebibliography}{99}
\bibitem{1-kov}
\Au{Левенчук А.\,И.} Системноинженерное мышление.~--- М.: TechInvestLab, 2015. 305~с.
\bibitem{2-kov}
\Au{Mabrok M.\,A., Ryan~M.\,J.} Category theory as a formal mathematical foundation for  
model-based systems engineering~// Appl. Math. Inform. Sci., 2017. Vol.~11. No.\,1. P.~43--51. 
doi: 10.18576/amis/110106.
\bibitem{3-kov}
   \Au{Breiner S., Subrahmanian~E., Jones~A.} Categorical\linebreak foundations for system engineering~// 
Disciplinary convergence in systems engineering research~/ Eds. A.~Madni, B.~Boehm, 
R.~Ghanem, D.~Erwin, D.~Wheaton.~--- Springer, 2018. P.~449--463. doi:  
10.1007/978-3-319-62217-0\_32.

\bibitem{4-kov}
   \Au{Watson M.\,D.} Future of systems engineering~// INCOSE INSIGHT, 2019. Vol.~22. 
Iss.~1. P.~8--12. doi: 10.1002/\linebreak inst.12231.

\bibitem{5-kov}
   \Au{Ковалёв С.\,П.} Методы теории категорий в~модельно-ориентированной системной 
инженерии~// Информатика и~её применения, 2017. Т.~11. Вып.~3. С.~42--50. doi: 
10.14357/1992226417030.

\bibitem{6-kov}
\Au{Guitart R., van den Bril~L.} D$\acute{\mbox{e}}$compositions et lax-
compl$\acute{\mbox{e}}$tions~// Cahiers Topologie 
G$\acute{\mbox{e}}$om$\acute{\mbox{e}}$trie Diff$\acute{\mbox{e}}$rentielle 
Cat$\acute{\mbox{e}}$goriques, 1977. Vol.~18. No.\,4. P.~333--407.

\bibitem{7-kov}
   \Au{Ковалёв С.\,П.} Теория категорий как математическая прагматика  
мо\-дель\-но-ори\-ен\-ти\-ро\-ван\-ной сис\-тем\-ной инженерии~// Информатика и~её 
применения, 2018. Т.~12. Вып.~1. С.~95--104. doi: 10.14357/19922264180112.

\bibitem{8-kov}
\Au{Маклейн С.} Категории для работающего математика~/ Пер. с~англ.~--- М.: Физматлит, 
2004. 352~с. (\Au{Mac Lane~S.} Categories for the working mathematician.~--- Springer, 1978. 
317~p.)
\bibitem{9-kov}
Grothendieck construction.~--- nLab, 2020. {\sf 
https://\linebreak ncatlab.org/nlab/show/Grothendieck+construction}.
\bibitem{10-kov}
\Au{Barr M., Wells~C.} Category theory for computing science.~--- London: Prentice Hall, 1990. 
538~p.
\bibitem{11-kov}
\Au{Ковалёв С.\,П.} Алгебраические методы порождающего проектирования 
крупномасштабных технических сис\-тем~// Управ\-ле\-ние развитием крупномасштабных 
систем: Мат-лы XII Междунар. конф.~--- М.: ИПУ РАН, 2019. С.~384--386.
\bibitem{12-kov}
\Au{Kowalski J.} CAD is a lie: Generative design to the rescue.~--- San Rafael, CA, USA: 
Autodesk, 2016. {\sf https://www.autodesk.com/redshift/generative-design}.
\bibitem{13-kov}
Comma category.~--- nLab, 2019. {\sf https://ncatlab.org/ nlab/show/comma+category}.
\bibitem{14-kov}
   \Au{Hitchins D.} What are the general principles applicable to systems?~// Incose Insight, 
2009. Vol.~12. Iss.~4. P.~59--64. doi: 10.1002/INST.200912459.

\bibitem{15-kov}
\Au{Iovane G., Laserra~E., Tortoriello~F.\,S.} Stochastic self-similar and fractal universe~// 
Chaos Soliton. Fract., 2004. Vol.~20. Iss.~3. P.~415--426. doi: 10.1016/j.chaos. 2003.08.004.
\bibitem{16-kov}
   \Au{Larijani H.} Local area networks and self-similar traffic~// Network performance 
engineering~/ Ed. D.\,D.~Kouvatsos.~--- Lecture notes in computer science ser.~--- Springer, 2011. 
Vol.~5233. P.~174--190. doi: 10.1007/978-3-642-02742-0\_8.
\end{thebibliography}

 }
 }

\end{multicols}

\vspace*{-9pt}

\hfill{\small\textit{Поступила в~редакцию 25.02.21}}

%\vspace*{8pt}

%\pagebreak

\newpage

\vspace*{-28pt}

%\hrule

%\vspace*{2pt}

%\hrule

%\vspace*{-2pt}

\def\tit{THE MONAD OF DIAGRAMS AS~A~MATHEMATICAL METAMODEL OF~SYSTEMS 
ENGINEERING\\[-5pt]}


\def\titkol{The monad of diagrams as~a~mathematical metamodel of~systems 
engineering}


\def\aut{S.\,P.~Kovalyov}

\def\autkol{S.\,P.~Kovalyov}

\titel{\tit}{\aut}{\autkol}{\titkol}

\vspace*{-15pt}


\noindent
   V.\,A.~Trapeznikov Institute of Control Sciences of the Russian Academy of Sciences, 
65~Profsoyuznaya Str., Moscow 117997, Russian Federation


\def\leftfootline{\small{\textbf{\thepage}
\hfill INFORMATIKA I EE PRIMENENIYA~--- INFORMATICS AND
APPLICATIONS\ \ \ 2023\ \ \ volume~17\ \ \ issue\ 2}
}%
 \def\rightfootline{\small{INFORMATIKA I EE PRIMENENIYA~---
INFORMATICS AND APPLICATIONS\ \ \ 2023\ \ \ volume~17\ \ \ issue\ 2
\hfill \textbf{\thepage}}}

\vspace*{1pt}
   
   
   
   
   \Abste{The paper addresses issues associated with the development of advanced 
mathematical methods for systems engineering suitable as the basis of computer tools for automatic 
synthesis and analysis of systems and processes. Following recent trends, category theory is 
employed as the framework for the methods. Its application is based on representing the structure of 
systems, processes, requirements, and other system design results as diagrams in categories whose 
objects are the algebraic models of parts and morphisms describe relationships between parts. 
Applying the fundamental Grothendieck flattening construction, the following constructions are 
described explicitly: categories of diagrams, the monad of diagrams, and the monad and the comonad of 
pointed diagrams. Application areas of these constructions in systems engineering procedures are 
identified. An approach is proposed to implement highly automated technologies of the generative 
design kind for complex multilevel systems.}
   
   \KWE{category theory; monad of diagrams; Grothendieck construction; colimit; systems 
engineering; system of systems; generative design}
   
\DOI{10.14357/19922264230202}{FDUVDU}

\vspace*{-6pt}

%\Ack
%\noindent

%\vspace*{4pt}

  \begin{multicols}{2}

\renewcommand{\bibname}{\protect\rmfamily References}
%\renewcommand{\bibname}{\large\protect\rm References}

{\small\frenchspacing
 {\baselineskip=10.8pt
 \addcontentsline{toc}{section}{References}
 \begin{thebibliography}{99}
 
 \vspace*{-3pt} 
\bibitem{1-kov-1}
   \Aue{Levenchuk, A.\,I.} 2015. \textit{Sistemnoinzhenernoe myshlenie} [Systems engineering 
thinking]. Moscow: TechInvestLab. 305~p.
\bibitem{2-kov-1}
   \Aue{Mabrok, M.\,A., and M.\,J.~Ryan.} 2017. Category theory as a formal mathematical 
foundation for model-based systems engineering. \textit{Appl. Math. Inform. Sci.} 11(1):43--51. 
doi: 10.18576/amis/110106.
\bibitem{3-kov-1}
   \Aue{Breiner, S., E.~Subrahmanian, and A.~Jones.} 2018. Categorical foundations for system 
engineering.  \textit{Disciplinary convergence in systems engineering research.} Eds. A.~Madni, 
B.~Boehm, R.~Ghanem, D.~Erwin, and D.~Wheaton. Springer. 449--463. doi:  
10.1007/978-3-319-62217-0\_32.
\bibitem{4-kov-1}
   \Aue{Watson, M.\,D.} 2019. Future of systems engineering. \textit{INCOSE INSIGHT}  
22(1):8--12. doi: 10.1002/inst.12231.
\bibitem{5-kov-1}
   \Aue{Kovalyov, S.\,P.} 2017. Metody teorii kategoriy v~model'no-orientirovannoy sistemnoy 
inzhenerii [Methods of category theory in model-based systems engineering]. \textit{Informatika 
i~ee Primeneniya~--- Inform. Appl.} 11(3):42--50. doi: 10.14357/1992226417030.
\bibitem{6-kov-1}
   \Aue{Guitart, R., and L.~van den Bril.} 1977. Decompositions et lax-completions. \textit{Cahiers 
Topologie Geometrie Differentielle Categoriques} 18(4):333--407.
\bibitem{7-kov-1}
   \Aue{Kovalyov, S.\,P.} 2018. Teoriya kategoriy kak ma\-te\-ma\-ti\-che\-skaya pragmatika  
model'no-orientirovannoy sistemnoy inzhenerii [Category theory as a~mathematical\linebreak pragmatics of 
model-based systems engineering]. \textit{Informatika i~ee Primeneniya~--- Inform. Appl.}  
12(1):95--104. doi: 10.14357/19922264180112.
\bibitem{8-kov-1}
   \Aue{Mac Lane, S.} 1978. \textit{Categories for the working mathematician}. New York, NY: 
Springer. 317~p.
\bibitem{9-kov-1}
   nLab. 2020. Grothendieck construction. Available at: {\sf 
https://ncatlab.org/nlab/show/Grothendieck+\linebreak construction/} (accessed May~29, 2023).
\bibitem{10-kov-1}
   \Aue{Barr, M., and C.~Wells.} 1990. \textit{Category theory for computing science}. London: 
Prentice Hall. 538~p.
\bibitem{11-kov-1}
   \Aue{Kovalyov, S.\,P.} 2019. Algebraicheskie metody porozhdayushchego proektirovaniya 
krupnomasshtabnykh tekhnicheskikh sistem [Algebraic methods of generative design of large-scale 
technical systems]. \textit{12th Conference (International) ``Management of Large-Scale System 
Development'' Proceedings}. Moscow: IPU RAN. 384--386.
\bibitem{12-kov-1}
   \Aue{Kowalski, J.} 2015. CAD is a lie: Generative design to the rescue. San Rafael, CA: Autodesk. Available at: {\sf 
https://www.themanufacturer.com/articles/cad-is-a-lie-generative-design-to-the-rescue/} (accessed May~29, 
2023).
\bibitem{13-kov-1}
   nLab. 2019. Comma category. Available at: {\sf  
https://\linebreak ncatlab.org/nlab/show/comma+category/} (accessed May~29, 2023).
\bibitem{14-kov-1}
   \Aue{Hitchins, D.} 2009. What are the general principles applicable to systems? \textit{Incose 
Insight} 12(4):59--64. doi: 10.1002/ INST.200912459.
\bibitem{15-kov-1}
   \Aue{Iovane, G., E.~Laserra, and F.\,S.~Tortoriello.} 2004. Stochastic self-similar and fractal 
universe.  \textit{Chaos Soliton. Fract.} 20(3):415--426. doi: 10.1016/j.chaos.2003.08.004.
\bibitem{16-kov-1}
   \Aue{Larijani, H.} 2011. Local area networks and self-similar traffic. \textit{Network 
performance engineering.} Ed. D.\,D.~Kouvatsos. Lecture notes in computer science ser. Springer. 
5233:174--190. doi: 10.1007/978-3-642-02742-0\_8.
   \end{thebibliography}

 }
 }

\end{multicols}

\vspace*{-9pt}

\hfill{\small\textit{Received February 25, 2021}} 
   
   
   \vspace*{-22pt}
   
\Contrl

\vspace*{-4pt}

\noindent
\textbf{Kovalyov Sergey P.} (b.\ 1972)~--- Doctor of Science in physics and mathematics, leading scientist, 
V.\,A.~Trapeznikov Institute of Control Sciences of the Russian Academy of Sciences, 65~Profsoyuznaya Str., Moscow 
117997, Russian Federation; \mbox{kovalyov@sibnet.ru}
   
   

   
\label{end\stat}

\renewcommand{\bibname}{\protect\rm Литература} 
   