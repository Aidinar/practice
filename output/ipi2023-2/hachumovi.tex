\def\stat{khachumov}

\def\tit{САМООБУЧЕНИЕ АВТОНОМНЫХ ИНТЕЛЛЕКТУАЛЬНЫХ РОБОТОВ В~ПРОЦЕССЕ 
ПОИСКОВО-ИССЛЕДОВАТЕЛЬСКОЙ ДЕЯТЕЛЬНОСТИ$^*$}

\def\titkol{Самообучение автономных интеллектуальных роботов в~процессе 
поисково-исследовательской деятельности}

\def\aut{В.\,Б.~Мелехин$^1$, В.\,М.~Хачумов$^2$, М.\,В.~Хачумов$^3$}

\def\autkol{В.\,Б.~Мелехин, В.\,М.~Хачумов, М.\,В.~Хачумов}

\titel{\tit}{\aut}{\autkol}{\titkol}

\index{Мелехин В.\,Б.}
\index{Хачумов В.\,М.}
\index{Хачумов М.\,В.}
\index{Melekhin V.\,B.}
\index{Khachumov V.\,M.}
\index{Khachumov M.\,V.}


{\renewcommand{\thefootnote}{\fnsymbol{footnote}} \footnotetext[1]
{Исследование выполнено при поддержке Российского научного фонда (проект 21-71-10056).}}


\renewcommand{\thefootnote}{\arabic{footnote}}
\footnotetext[1]{Дагестанский государственный технический университет, \mbox{pashka1602@rambler.ru}}
\footnotetext[2]{Институт программных систем им.\ А.\,К.~Айламазяна Российской академии наук; Федеральный 
исследовательский центр <<Информатика и~управ\-ле\-ние>> Российской академии наук; Российский 
университет друж\-бы народов, \mbox{vmh48@mail.ru}}
\footnotetext[3]{Институт программных систем им.\ А.\,К.~Айламазяна Российской академии наук; Федеральный 
исследовательский центр <<Информатика и~управ\-ле\-ние>> Российской академии наук; Российский 
университет друж\-бы народов, \mbox{khmike@inbox.ru}}

%\vspace*{-12pt}

 
  

\Abst{Рассматривается один из эффективных подходов к~организации целесообразного 
поведения автономных интегральных роботов (АИР) в~процессе по\-иско\-во-ис\-сле\-до\-ва\-тель\-ской 
деятельности в~априори неописанных условиях проб\-лем\-ной среды (ПС). Предлагается в~основе 
целесообразного поведения роботов использовать процедуры на\-гляд\-но-дей\-ст\-вен\-но\-го 
мышления, основанные на формализации рефлекторного поведения высокоорганизованных 
живых сис\-тем. Разработан алгоритм са\-мо\-обуче\-ния в~условиях с~высоким уровнем 
неопределенности, поз\-во\-ля\-ющий автоматически формировать условные программы 
целесообразного поведения, обес\-пе\-чи\-ва\-ющие АИР
возможность достигать заданной цели поведения в~процессе  
по\-иско\-во-ис\-сле\-до\-ва\-тель\-ской деятельности. Найдены граничные оценки 
функциональной слож\-ности предложенного алгоритма самообучения в~условиях 
не\-опре\-де\-лен\-ности, по\-ка\-зы\-ва\-ющие воз\-мож\-ность его реализации на бортовой ЭВМ 
автономных интегральных роботов, име\-ющих, как правило, ограниченные вы\-чис\-ли\-тель\-ные 
ресурсы. Проведено имитационное моделирование процесса са\-мо\-обуче\-ния 
АИР в~априори неописанной ПС, под\-твер\-див\-шее 
эф\-фек\-тив\-ность применения предложенного подхода для организации планирования 
целесообразного поведения в~априори неописанных~ПС.}

\KW{автономный интегральный робот; алгоритм са\-мо\-обуче\-ния; условия 
не\-оп\-ре\-де\-лен\-ности; проб\-лем\-ная среда; услов\-ные сигналы}

\DOI{10.14357/19922264230211}{SOFDKW} 
  
\vspace*{-4pt}


\vskip 10pt plus 9pt minus 6pt

\thispagestyle{headings}

\begin{multicols}{2}

\label{st\stat}
 
\section{Введение}

  Разработка информационных технологий, связанных с~по\-стро\-ен\-ием 
интеллектуального решателя задач АИР, 
способных целесообразно (рационально) функционировать в~априори 
неописанных ПС,~--- актуальная и~слож\-ная проб\-ле\-ма 
искусственного интеллекта. К~одному из эффективных подходов решения 
данной проб\-ле\-мы следует отнести разработку когнитивных инструментов 
на\-гляд\-но-дей\-ст\-вен\-но\-го мышления интеллектуальных сис\-тем различного 
назначения~[1]. В~общем случае на\-гляд\-но-дей\-ст\-вен\-ное мыш\-ле\-ние АИР 
строится на основе формализации рефлекторного поведения живых сис\-тем~[2, 3] и~включает сле\-ду\-ющие три основные со\-став\-ля\-ющие~[4]. 
  \begin{enumerate}[1.]
  \item Самообучение на основе выполнения пробных действий и~механизмов 
избирательности по\-сту\-па\-ющей из ПС информации, 
обес\-пе\-чи\-ва\-ющих воз\-мож\-ность поиска заданных объектов в~априори 
неописанных условиях функционирования. 
  
  Организовать самообучение АИР в~априори неописанных условиях ПС 
можно на основе, например, алгоритмов роевого поведения~[5, 6] или 
генетических алгоритмов~[7, 8]. Однако непосредственная отработка проб\-ных 
действий может привести к~негативным изменениям, не связанным
 с~достижением заданной цели. Обойти этот недостаток можно на основе 
алгоритмов са\-мо\-обуче\-ния, которые имитируют выполнение проб\-ных действий 
на формальном описании текущей ситуации ПС.
  
  Процесс самообучения АИР в~априори неописанной ПС
сводится к~формированию и~закреплению элементарных актов поведения 
в~фор\-ми\-ру\-емых условных программах целесообразной\linebreak де\-я\-тель\-ности (УПЦД) по 
новизне происходящих в~ПС изменений, а~для всей 
автоматически по\-стро\-ен\-ной упорядоченной по\-сле\-до\-ва\-тель\-ности действий 
характеризуется \mbox{дос\-ти\-же\-ни\-ем} заданного безуслов\-но\-го сигнала. 
  
  В формируемых в~процессе самообучения \mbox{УПЦД} запоминаются происходящие в~ПС
изменения в~форме сигналов, которые возникают в~результате от\-ра\-ба\-ты\-ва\-емых 
АИР действий. Различные сигналы ПС в~процессе са\-мо\-обуче\-ния 
приобретают роль услов\-ных сигналов~--- знаков, вы\-зы\-ва\-ющих у~АИР 
определенные реакции, связанные с~отработкой за\-креп\-лен\-ных в~УПЦД 
действий. Таким образом, услов\-ные сигналы после за\-креп\-ле\-ния в~УПЦД 
приобретают роль ориентиров или предвестников, появление которых 
в~ПС сигнализирует АИР о~воз\-мож\-ности достижения в~ней 
соответствующего без\-услов\-но\-го сигнала.
  \item  Целесообразное поведение АИР, связанное с~отработкой в~текущих 
условиях функционирования действий, ранее сформированных УПЦП для 
достижения со\-от\-вет\-ст\-ву\-ющих им без\-услов\-ных сигналов при восприятии 
в~ПС закрепленных в~этих программах услов\-ных сигналов. 
  \item Отработка безусловных реакций для достижения заданной цели при 
появлении в~ПС со\-от\-вет\-ст\-ву\-ющих им без\-услов\-ных сигналов.
  \end{enumerate}
  
  В настоящей статье предлагаются процедуры са\-мо\-обуче\-ния в~процессе  
по\-иско\-во-ис\-сле\-до\-ва\-тель\-ской деятельности, поз\-во\-ля\-ющие АИР 
организовать целесообразное поведение в~априори \mbox{неописанной} 
ПС с~препятствиями для поиска заданных объектов. Например, при 
выполнении различных спасательных работ в~труднодоступных для человека 
условиях функционирования.
  
\section{Постановка задачи}

  Рассмотрим АИР, оснащенный техническим зрением, манипулятором 
и~моторной сис\-те\-мой, поз\-во\-ля\-ющей ему перемещаться в~ПС. 
Проблемная среда пред\-став\-ля\-ет собой пересеченную мест\-ность 
с~расположенными на ее территории препятствиями и~различными объектами 
$O\hm= \{ o_{i_1}(X_{i_1})\}$, $i_1\hm= \overline{1,n_1}$, где $X_{i_1}$~--- 
множество характеристик, по которым робот способен идентифицировать 
воспринимаемые в~ПС препятствия и~объекты. 
  
  Проблемную среду можно охарактеризовать множеством условных сигналов 
  $A\hm= \{a_{i_2}\}$, $i_2\hm=\overline{1,n_2}$, каждый из которых 
пред\-став\-ля\-ет собой проходимый меж\-ду препятствиями участок ПС.
  
  В общем случае АИР способен отрабатывать множество действий 
  $B\hm= \{ b_{i_3}\}$, $i_3\hm= \overline{1,n_3}$, и~распознавать 
проходимые участ\-ки мест\-ности, а~также найденные в~ПС
объекты $o_{i_1}(X_{i_1})\hm\in O$. (Следует отметить, что проб\-ле\-ма 
распознавания проходимых участков и~объектов ПС является 
самостоятельной задачей и~в~на\-сто\-ящей статье не рас\-смат\-ри\-ва\-ется.)
  
  Требуется разработать алгоритм самообучения АИР в~процессе  
по\-иско\-во-ис\-сле\-до\-ва\-тель\-ской деятельности, поз\-во\-ля\-ющий 
автоматически формировать в~априори неописанной проб\-лем\-ной среде УПЦП 
сле\-ду\-юще\-го вида: 
  \begin{equation}
  a^1_{i_2} \& b^1_{i_3} \to a^2_{i_2} \& b^2_{i_3} \to\cdots\to a^k_{i_2} 
  \& b^k_{i_3} \to a^P_{i_2}\,,
  \label{e1-kh}
  \end{equation}
где $a^1_{i_2}\& b^1_{i_3} \to a^2_{i_2}$~--- элементарный акт поведения, 
озна\-ча\-ющий, что если АИР воспринимает в~ПС условный сигнал 
$a^1_{i_2}$, то отрабатываемое им действие~$b^1_{i_3}$ приводит 
к~появлению условного сигнала~$a^2_{i_2}$; $a^1_{i_2}$ и~$a^2_{i_2}$~--- 
условные сигналы ПС, определяющиеся проходимыми для АИР 
участками ПС; $a^P_{i_2}$~--- безуслов\-ный сигнал, 
вызывающий у~АИР соответствующие безусловные реакции, связанные 
с~выполнением определенных действий над найденным объектом. 

  Требуется разработать алгоритм са\-мо\-обуче\-ния, поз\-во\-ля\-ющий АИР 
в~процессе по\-иско\-во-ис\-сле\-до\-ва\-тель\-ской де\-я\-тель\-ности формировать \mbox{УПЦД} 
в~виде прос\-той цепи~(1) в~априори неописанных~ПС.
  
\section{Синтез алгоритма самообучения автономных интегральных роботов}

  В общем случае алгоритм самообучения АИР в~процессе по\-ис\-ко\-во-ис\-сле\-до\-ва\-тель\-ской 
  де\-я\-тель\-ности опирается на за\-креп\-ле\-ние в~формируемой 
УПЦД элементарных актов поведения по новизне условных сигналов, 
воспринимаемых в~ПС. Вся же полученная таким образом цепь 
действий закрепляется достижением в~ПС заданного 
без\-услов\-но\-го сигнала. Роль безуслов\-но\-го сигнала в~этом случае играет 
заданный объект, воспринятый в~ПС после выхода АИР за пределы последнего 
за\-креп\-лен\-но\-го в~УПЦД условного сигнала. Данный алгоритм са\-мо\-обуче\-ния 
АИР имеет сле\-ду\-ющее структурированное описание. 
  
\noindent
  \textbf{Исходные условия:} заданный АИР объект ПС 
$o_{i_1}(X_{i_1})\hm\in O$, выполняющий роль без\-услов\-но\-го 
сигнала~$a_{i_2}^P$; множество действий~$B$, которые способен 
отрабатывать АИР.
  
 \noindent
  \textbf{Входные переменные:} услов\-ные сигналы $a_{i_2}\hm\in A$ 
и~заданный~$a^P_{i_2}$ безусловный сигнал ПС.
  
 \noindent
  \textbf{Выходные переменные:} фор\-ми\-ру\-емые УПЦД в~виде прос\-той цепи. 
  
\smallskip
  
 \noindent
  \textbf{Начало.}\\[-12pt]
  \begin{enumerate}[1.]
\item  Установить $j_1\hm=1$. Определить в~качестве исходного 
сигнала~$a_{i_2}^{j_1}$ в~фор\-ми\-ру\-емой УПЦД непосредственно 
воспринимаемое роботом в~ПС препятствие. 
  \item  Принять в~качестве подцели поведения на текущем шаге самообучения 
появление в~ПС нового условного сигнала~$a_{i_2}^{j_1+1}$, 
опре\-де\-ля\-емо\-го воспринятым после отработки проб\-но\-го действия новым 
проходимым участ\-ком. 
  \item  Определить на текущем шаге самообучения согласно равномерному 
закону распределения вероятностей выбора проб\-ное действие 
$b^{j_1}_{i_3}\hm\in B$. Выполнить выбранное действие~$b^{j_1}_{i_3}$ 
и~сформировать по результатам его отработки элементарный акт поведения 
$a_{i_2}^{j_1} \& b^{j_1}_{i_3}\hm\to a_{i_2}^{j_1+1}$.
  \item  Проверить условие <<услов\-ный сигнал~$a_{i_2}^{j_1+1}$ был ранее 
закреплен в~формируемой УПЦД>>: если условие выполняется, то перейти 
к~п.~5; в~противном случае перейти к~п.~8.
  \item  Удалить все элементарные акты поведения, за\-креп\-лен\-ные в~УПЦД 
после первого восприятия АИР в~ПС услов\-но\-го 
сигнала~$a_{i_2}^{j_1+1}$.
  \item Исключить выбранное пробное действие~$b_{i_3}^{j_1}$ из 
множества~$B$ как нерезультативное на текущем шаге са\-мо\-обуче\-ния. 
  \item Проверить условие <<множество~$B$ является пус\-тым>>: если условие 
выполняется, то перейти к~п.~11; в~противном случае перейти к~п.~3. 
  \item  Сохранить элементарный акт поведения $a_{i_2}^{j_1} \& 
b_{i_3}^{j_1} \hm\to a_{i_2}^{j_1+1}$ в~фор\-ми\-ру\-емой УПЦД.
  \item  Проверить условие <<после выхода за пределы проходимого участка, 
опре\-де\-ля\-емо\-го условным сигналом~$a_{i_2}^{j_1+1}$, АИР воспринимает 
в~ПС заданный без\-услов\-ный сигнал~$a_{i_2}^P$>>: если 
условие выполняется, перейти к~п.~12; в~противном случае перейти к~п.~10.
  \item  Восстановить все исключенные действия из заданного множества~$B$, 
$j_1\hm= j_1\hm+1$, перейти к~п.~2.
  \item  Сформировать тре\-бу\-емую УПЦД в~текущих условиях ПС не 
пред\-став\-ля\-ет\-ся воз\-мож\-ным, перейти к~п.~13.
  \item  Требуемая УПЦД сформирована; выполнить для достижения заданной 
цели безуслов\-ные реакции.\\[-14pt]
  \end{enumerate}
  
 \noindent
  \textbf{Конец.}
  
  \smallskip
  
  Введем понятие функциональной слож\-ности~$\beta$ алгоритма 
са\-мо\-обуче\-ния АИР, зависящей от общего чис\-ла действий $b_{i_3}\hm\in B$, 
апробируемых роботом в~процессе формирования УПЦД. Тогда для данного алгоритма можно доказать 
сле\-ду\-ющее утверж\-де\-ние.
  
  \smallskip
  
  \noindent
  \textbf{Утверждение.} \textit{Функциональная слож\-ность~$\beta$ 
алгоритма са\-мо\-обуче\-ния АИР определяется сле\-ду\-ющи\-ми граничными 
оценками: 
  $$
  n_{10}\leq\beta\leq n_1 n_{10}\,,
  $$
где $n_{10}$~--- общее чис\-ло выполненных АИР шагов самообучения; $n_1$~--- 
общее чис\-ло различного вида действий, которые робот отрабатывает 
в~процессе са\-мо\-обуче\-ния.}

\smallskip

\noindent
  Д\,о\,к\,а\,з\,а\,т\,е\,л\,ь\,с\,т\,в\,о\,.\ \ Справедливость 
сформулированного утверж\-де\-ния вытекает из сле\-ду\-ющих соображений.
  \begin{enumerate}[1.]
  \item Согласно пп.~3--8 \textit{алгоритма са\-мо\-обуче\-ния}, впол\-не вероятно, 
что в~лучшем случае на каж\-дом $j_1$-м шаге самообучения АИР первым 
случайным образом выбирает результативное действие~$b_{i_3}^{j_1}$. 
Следовательно, ниж\-нее граничное значение оцен\-ки слож\-ности~$\beta_1$ 
в~этом случае определяется величиной, равной~$n_{10}$. 
  \item  В худшем случае результативное действие~$b_{i_3}^{j_1}$ на каж\-дом 
$j_1$-м шаге самообучения АИР может быть выбрано случайным образом 
в~последнюю очередь. Отсюда следует, что на каж\-дом шаге са\-мо\-обуче\-ния АИР 
апробирует отработку не более $n_1$ действий $b_k(j_1)\hm\in B$. Таким 
образом, число выполнений проб\-ных действий в~процессе са\-мо\-обуче\-ния АИР 
не может превышать величины, рав\-ной~$n_1 n_{10}$.
  \item Из пп.~1 и~2 проведенного доказательства с~оче\-вид\-ностью следует 
спра\-вед\-ли\-вость сформулированного утверж\-де\-ния. 
  \end{enumerate}
  
 
  
   Рассмотрим гипотетический пример, связанный с~использованием АИР 
алгоритма са\-мо\-обуче\-ния для решения целевой задачи, когда у~робота 
отсутствует формальное описание кар\-ты мест\-ности, а~известны только границы 
участка ПС, на котором требуется найти заданные объекты.

\section{Пример решения задачи в~процессе  
поисково-исследовательской деятельности автономных интеллектуальных роботов}

  Пусть АИР требуется найти заданный объект в~априори неописанной 
ПС, пред\-став\-ля\-ющей собой мест\-ность с~расположенными на ней 
препятствиями, структура которой приведена на рисунке.
   
   
  Таким образом, ПС характеризуется 11~расположенными 
в~ней препятствиями $P\hm= \{p_{i_6}\}$, $i_6\hm= \overline{1,11}$, 
и~16~проходимыми между препятствиями зонами, обозначенными сигналами 
$A\hm= \{ a_{i_7}\}$, $i_7\hm= \overline{1,16}$. В~этой среде АИР требуется 
найти объекты,\linebreak\vspace*{-12pt}

\pagebreak

\end{multicols}

 \begin{figure*} %fig1
   \vspace*{1pt}
\begin{center}
   \mbox{%
\epsfxsize=117.405mm 
\epsfbox{hac-1.eps}
}


\vspace*{12pt}

   {\small Структура ПС с~препятствиями и~обозначенными исходными 
местоположениями АИР и~заданных объектов}
\end{center}
\vspace*{6pt}
   %\end{figure*}
      %\vspace*{6pt}
   %
   %\begin{table*}[b]\small 
  % 
   \vspace*{6pt}
\begin{center}
{\small \begin{tabular}{|c|c|c|c|c|c|c|c|c|c|c|c|c|c|c|c|c|c|c|c|}
\multicolumn{20}{c}{Закономерности перехода АИР из текущего положения в~ПС к~смежной проходимой зоне}\\
\multicolumn{20}{c}{\ }\\[-3pt]
\hline
&$a_{\mathrm{И}}$&$a_1$&$a_2$&$a_3$&$a_4$&$a_5$&$a_1^*$&$a_6$&$a_7$&$a_8$&$a_9$& 
$a_{10}$&$a_{11}$&$a_2^*$&$a_{12}$&$a_{13}$&$a_{14}$&$a_{15}$&$a_{16}$\\
\hline
$b_1$&$a_1$&$a_3$&$a_4$&---&$a_3$&$a_8$&---&---&$a_9$&$a_{10}$&$a_{12}$&$a_{13}$&$a_{10}$&---&---&$a_{12}$&$a_{16}$&---&$a_{\mathrm{Ц}}$\\
$b_2$&$a_2$&$a_4$&$a_5$&$a_4$&$a_5$&$a_1^*$&---
&$a_9$&$a_{10}$&$a_{11}$&$a_{13}$&$a_{14}$&$a_2^*$&---&$a_{15}$&$a_{14}$&---&$a_{\mathrm{Ц}}$
&---\\
$b_3$&---&---&---&---&---&---&$a_5$&---&---&---&---&---&---&$a_{11}$&---&---&---&---&---\\
\hline
\end{tabular}
}
\end{center}
\end{figure*}
%\end{table*}

\begin{multicols}{2}

\noindent
 расположенные за препятствием $p_8\hm\in P$, при его 
исходном местоположении напротив препятствия $p_2\hm\in P$. Восприятие 
данного объекта в~ПС соответствует достижению роботом 
заданной цели. Пусть АИР для решения по\-став\-лен\-ной перед ним задачи 
способен отрабатывать сле\-ду\-ющие три действия: $b_1$~--- поворот влево 
и~движение вперед до выхода за наблюдаемую в~результате этого проходимую 
зону; $b_2$~--- поворот вправо и~движение вперед до выхода за наблюдаемую 
в~результате этого проходимую зону; $b_3$~--- безуслов\-ные реакции 
<<разворот и~выход из тупика в~предыду\-щее исходное текущее 
местоположение>>.
  
  При этом система технического зрения АИР способна распознавать 
и~отличать друг от друга проходимые участки $a_{i_7}\hm\in A$ 
ПС и~заданный ему объект. Для имитации процесса поиска АИР цели 
в~заданной ПС строится конечный автомат со случайными 
реакциями~[4], в~память которого занесена таб\-ли\-ца команд, отражающая 
закономерности перехода АИР от одного проходимого участка ПС к~другому такому участку (см.\ таблицу). 
  

    
  В таблице использованы сле\-ду\-ющие обозначения: $a_{\mathrm{И}}$~--- 
исходное местоположение АИР; $a_1^*$ и~$a_2^*$~--- соответственно первый 
и~второй тупик; $a_{\mathrm{Ц}}$~--- местоположение заданных объектов; 
прочерк означает отсутствие прохода или выхода за пределы заданного участка 
мест\-ности. 
  
  По итогам проведенного на ПЭВМ эксперимента были получены сле\-ду\-ющие 
результаты. Автономный интеллектуальный робот, выполнив~12~пробных 
действий, прошел по сле\-ду\-юще\-му марш\-ру\-ту в~процессе поиска заданных 
объектов (см.\ рисунок):

\vspace*{-2pt}

\noindent
  \begin{multline*}
  a_{\mathrm{И}} \& b_1 \to a_1\& b_2 \to a_4 \& b_2\to a_5 \& b_1\to a_8 \&
  b_2\to{}\\
  {}\to  a_{11} \& b_2\to a_2^*(\mbox{тупик~2}) \& b_3\to a_{11} \& b_1\to{}\\
  {}\to  a_{10} \& b_1\to a_{13} \& b_1\to a_{12} \& b_2\to 
a_{15} \& b_2\to a_{\mathrm{Ц}}.
  \end{multline*}
  %
  
  \vspace*{-2pt}
  
  \noindent
  При этом у АИР после обнуления в~процессе са\-мо\-обуче\-ния цик\-ла 
сформировалась сле\-ду\-ющая \mbox{УПЦД}:

\vspace*{-2pt}

\noindent
  \begin{multline*}
  a_{\mathrm{И}} \& b_1\to a_1 \& b_2\to a_4 \& b_2\to a_5 \& b_1\to a_8 \& 
b_1\to{}\\
  {}\to
  a_{10} \& b_1\to a_{13} \& b_1\to a_{12} \& b_2\to a_{15} \& b_2\to 
a_{\mathrm{Ц}}\,.
  \end{multline*}
  
  Данную УПЦД интеллектуальный робот может использовать, например, для 
перевозки большого чис\-ла заданных объектов на участок ПС, 
опре\-де\-ля\-ющий заданное их мес\-то\-по\-ло\-же\-ние. 

%\vspace*{-6pt}

\section{Заключение}

\noindent
\begin{enumerate}[1.]
  \item Предложенный алгоритм са\-мо\-обуче\-ния поз\-во\-ля\-ет организовать 
целесообразное поведение АИР в~процессе  
по\-иско\-во-ис\-сле\-до\-ва\-тель\-ской\linebreak\vspace*{-9.5pt}
\end{enumerate}

\noindent
\begin{enumerate}[1.]
\setcounter{enumi}{1}
\item[\,]
 деятельности в~априори неописанных 
труднодоступных для человека~ПС.
  \item Найденные граничные оценки и~\mbox{результаты} имитационного 
моделирования алгоритма самообуче\-ния показали эф\-фек\-тив\-ность его 
использования для проведения АИР по\-иско\-во-ис\-сле\-до\-ва\-тель\-ской 
деятельности в~априори неописанной ПС с~препятствиями 
с~\mbox{целью} поиска заданных объектов, например при выполнении различных 
спасательных работ в~труд\-но\-до\-ступ\-ных для человека условиях 
функ\-цио\-ни\-ро\-ва-\linebreak ния. 
  \end{enumerate}
   
{\small\frenchspacing
 {%\baselineskip=10.8pt
 %\addcontentsline{toc}{section}{References}
 \begin{thebibliography}{9}
  \bibitem{1-kh}
  \Au{Мелехин~В.\,Б., Хачумов~М.\,В.} Формы мышления автономных интеллектуальных 
агентов: особенности и~проб\-ле\-мы их организации~// Морские интеллектуальные технологии, 
2020. №\,4-1. С.~224--230. doi: 10.37220/MIT.2020.50.4.031.
  \bibitem{2-kh}
  \Au{Брайнес С.\,Н., Напалков~А.\,Н., Свечинский~В.\,Б.} Нейрокибернетика.~--- М.: 
Госмедиздат, 1962. 172~с.
  \bibitem{3-kh}
  \Au{Шингаров Г.\,Х.} Условные рефлексы и~проб\-ле\-ма знака и~значения.~--- М.: Наука, 
1986. 200~с.
  \bibitem{4-kh}
  \Au{Мелехин В.\,Б., Хачумов~М.\,В.} Инструментальные средства управ\-ле\-ния 
целесообразным поведением са\-мо\-ор\-га\-ни\-зу\-ющих\-ся автономных интеллектуальных агентов~// 
Мехатроника, автоматизация, управ\-ле\-ние, 2021. Т.~22. №\,4. С.~171--180. doi: 
10.17587/mau.22.171-180.
  \bibitem{5-kh}
  \Au{Карпов В.\,Э., Карпова~И.\,П., Кулинич~А.\,А.} Социальные сообщества роботов.~--- 
М.: Ленанд, 2019. 352~с.
  \bibitem{6-kh}
  \Au{Guan B., Xu~T., Zhao~Y., Li~Y., Dong~X.} A~random grouping-based self-regulating 
artificial bee colony algorithm for interactive feature detection~// Knowl.-Based Syst., 2021. 
Vol.~243. P.~1--12. doi: 10.1016/j.knosys.2022.108434.
  
  \bibitem{8-kh}
  \Au{Рутковская Д., Пилиньский~М., Рутковский~Л.} Нейронные сети, генетические 
алгоритмы и~нечеткие сис\-те\-мы.~--- М.: Горячая линия\,--\,Телеком, 2008. 452~с.

\bibitem{7-kh}
  \Au{Саймон Д.} Алгоритмы эволюционной оптимизации~/ Пер. с~англ.~--- М.: ДМК 
Пресс, 2020. 940~с. (\Au{Simon~D.} Evolutionary optimization algorithms.~--- 1st ed.~--- New 
York, NY, USA: Wiley, 2013. 784~p.)

\end{thebibliography}

 }
 }

\end{multicols}

\vspace*{-6pt}

\hfill{\small\textit{Поступила в~редакцию 02.11.22}}

\vspace*{8pt}

%\pagebreak

%\newpage

%\vspace*{-28pt}

\hrule

\vspace*{2pt}

\hrule

%\vspace*{-2pt}

\def\tit{SELF-LEARNING OF AUTONOMOUS INTELLIGENT ROBOTS IN~THE~PROCESS 
OF~SEARCH AND~EXPLORE ACTIVITIES}


\def\titkol{Self-learning of autonomous intelligent robots in~the~process 
of~search and~explore activities}


\def\aut{V.\,B.~Melekhin$^1$, V.\,M.~Khachumov$^{2,3,4}$, and~M.\,V.~Khachumov$^{2,3,4}$}

\def\autkol{V.\,B.~Melekhin, V.\,M.~Khachumov, and~M.\,V.~Khachumov}

\titel{\tit}{\aut}{\autkol}{\titkol}

\vspace*{-10pt}


\noindent
$^1$Dagestan State Technical University, 70A Imam Shamil Ave., Makhachkala 367015, Republic 
of Dagestan 


\noindent
$^2$Ailamazyan Program Systems Institute of the Russian Academy of Sciences, 4A~Petra Pervogo 
Str., Veskovo 152024,\linebreak
$\hphantom{^1}$Yaroslavl Region, Russian Federation

\noindent
$^3$Federal Research Center ``Computer Science and Control'' of the Russian Academy of Sciences, 
44-2~Vavilov\linebreak
$\hphantom{^1}$Str., Moscow 119333, Russian Federation

\noindent
$^4$RUDN University, 6~Miklukho-Maklaya Str., Moscow 117198, Russian Federation


\def\leftfootline{\small{\textbf{\thepage}
\hfill INFORMATIKA I EE PRIMENENIYA~--- INFORMATICS AND
APPLICATIONS\ \ \ 2023\ \ \ volume~17\ \ \ issue\ 2}
}%
 \def\rightfootline{\small{INFORMATIKA I EE PRIMENENIYA~---
INFORMATICS AND APPLICATIONS\ \ \ 2023\ \ \ volume~17\ \ \ issue\ 2
\hfill \textbf{\thepage}}}

\vspace*{3pt}
  



\Abste{One of the effective approaches to organizing the goal-seeking behavior of autonomous 
integral robots in the process of search and explore activities in an a~priori undescribed conditions of 
a~problematic environment is considered. It is proposed to use the procedures of visual-effective 
thinking based on the formalization of the reflex behavior of highly organized living systems as the 
basis for the goal-seeking behavior of robots. A~self-learning algorithm has been developed 
for the conditions with a~high level of uncertainty which allows automatically generating 
conditional programs of expedient behavior that provide autonomous integral robots with the ability 
to achieve a given behavioral goal in the process of search and explore activities. The boundary 
estimates of the functional complexity of the proposed self-learning algorithm under uncertainty 
are found showing the possibility of its implementation on the onboard computer of 
autonomous integral robots which have, as a~rule, limited computing resources. A~modeling of 
self-learning process for an autonomous integral robot in an a~priori undescribed and problematic 
environment was carried out which confirmed the effectiveness of the proposed approach for 
organizing the planning of goal-seeking behavior in an a~priori undescribed and problematic environments.}


\KWE{autonomous integral robot; self-learning algorithm; uncertainty conditions; problematic 
environment; conditional signals}



\DOI{10.14357/19922264230211}{SOFDKW} 

%\vspace*{-11pt}

\Ack
\noindent
This work was supported by the Russian Science Foundation, project No.\,21-71-10056.
  

%\vspace*{4pt}

  \begin{multicols}{2}

\renewcommand{\bibname}{\protect\rmfamily References}
%\renewcommand{\bibname}{\large\protect\rm References}

{\small\frenchspacing
 {%\baselineskip=10.8pt
 \addcontentsline{toc}{section}{References}
 \begin{thebibliography}{9} 
  \bibitem{1-kh-1}
\Aue{Melekhin, V.\,B., and M.\,V.~Khachumov.} 2020. For\-my mysh\-le\-niya av\-to\-nom\-nykh 
in\-tel\-lek\-tu\-al'\-nykh agen\-tov: oso\-ben\-no\-sti i~prob\-le\-my ikh or\-ga\-ni\-za\-tsii [Forms of thinking of 
autonomous intelligent agents: Features and problems of their organization]. \textit{Morskie 
intellektual'nye tekh\-no\-lo\-gii} [Marine Intelligent Technologies] 4-1:224--230. doi: 
10.37220/MIT.2020.50.4.031.
  \bibitem{2-kh-1}
\Aue{Braynes, S.\,N., A.\,N.~Napalkov, and V.\,B.~Svechinskiy}. 1962. \textit{Ney\-ro\-ki\-ber\-ne\-ti\-ka} 
[Neurocybernetics]. Moscow: Gos\-med\-iz\-dat. 172~p.
  \bibitem{3-kh-1}
\Aue{Shingarov, G.\,Kh.} 1986. \textit{Uslov\-nye ref\-lek\-sy i~prob\-le\-ma zna\-ka i~zna\-ch\-eniya} 
[Conditioned reflexes and the problem of sign and meaning]. Moscow: Nauka. 200~p.
  \bibitem{4-kh-1}
\Aue{Melekhin, V.\,B., and M.\,V.~Khachumov.} 2021. Ins\-tru\-men\-tal'\-nye sredst\-va up\-rav\-le\-niya 
tse\-le\-so\-ob\-raz\-nym po\-ve\-de\-ni\-em sa\-mo\-or\-ga\-ni\-zu\-yushchikhsya av\-to\-nom\-nykh in\-tel\-lek\-tu\-al'\-nykh agen\-tov 
[Instrumental means for managing the rational behavior of self-organizing autonomous intelligent 
agents]. \textit{Mekhatronika, avtomatizatsiya, upravlenie} [Mechanatronics, Automation, and Control] 4:171--180. doi: 
10.17587/mau.22.171-180.
  \bibitem{5-kh-1}
\Aue{Karpov, V.\,E., I.\,P.~Karpova, and A.\,A.~Kulinich.} 2019. \textit{So\-tsi\-al'\-nye so\-ob\-shchest\-va 
ro\-bo\-tov} [Social communities of robots]. Moscow: LENAND. 352 p.
  \bibitem{6-kh-1}
\Aue{Guan, B., T.~Xu, Y.~Zhao, Y.~Li, and X.~Dong.} 2021. A~random grouping-based self-
regulating artificial bee colony algorithm for interactive feature detection. \textit{Knowl.-Based 
Syst.} 243:1--12. doi: 10.1016/j.knosys.2022.108434.
  
  \bibitem{8-kh-1}
\Aue{Rutkovskaya, D., M.~Pilin'skiy, and L.~Rutkovskiy.} 2008. \textit{Ney\-ron\-nye se\-ti, 
ge\-ne\-ti\-che\-skie al\-go\-rit\-my i~ne\-chet\-kie sis\-te\-my} [Neural networks, genetic algorithms, and fuzzy 
systems]. Moscow: Goryachaya Liniya\,--\,Telekom. 452~p.

\bibitem{7-kh-1}
\Aue{Simon, D.} 2013. \textit{Evolutionary optimization algorithms}. 1st ed. New York, NY: 
Wiley. 784~p.

\end{thebibliography}

 }
 }

\end{multicols}

\vspace*{-6pt}

\hfill{\small\textit{Received November 2, 2022}} 

\vspace*{-18pt}

\Contr

\noindent
\textbf{Melekhin Vladimir B.} (b.\ 1954)~--- Doctor of Science in technology, professor, 
Department of Software for Computers and Automated Systems, Dagestan State Technical 
University, 70A~Imam Shamil Ave., Makhachkala 367015, Republic of Dagestan; 
\mbox{pashka1602@rambler.ru}

\vspace*{3pt}

\noindent
\textbf{Khachumov Vyacheslav M.} (b.\ 1948)~--- Doctor of Science in technology, 
head of laboratory, Intelligent Control Laboratory, Ailamazyan Program Systems Institute of the
Russian Academy of Sciences, 4A~Petra Pervogo Str., Veskovo 152024, Yaroslavl Region, Russian 
Federation; principal scientist, Federal Research Center ``Computer Science and Control'' of the 
Russian Academy of Sciences, 44-2~Vavilov Str., Moscow 119333, Russian Federation; professor, 
Department of Information Technology, RUDN University, 6~Miklukho-Maklaya Str., Moscow 
117198, Russian Federation; \mbox{vmh48@mail.ru}
\vspace*{3pt}

\noindent
\textbf{Khachumov Mikhail V.} (b.\ 1986)~--- Candidate of Science (PhD) in physics and 
mathematics, senior scientist, Intelligent Control Laboratory, Ailamazyan Program Systems 
Institute of the Russian Academy of Sciences, 4A~Petra Pervogo Atr., Veskovo 152024, Yaroslavl 
Region, Russian Federation; senior scientist, Federal Research Center ``Computer Science and 
Control'' of the Russian Academy of Sciences, 44-2~Vavilov Str., Moscow 119333, Russian 
Federation; associate professor, Department of Information Technology, RUDN University,  
6~Miklukho-Maklaya Str., Moscow 117198, Russian Federation; \mbox{khmike@inbox.ru}

  


\label{end\stat}

\renewcommand{\bibname}{\protect\rm Литература} 
    