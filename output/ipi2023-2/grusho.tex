\def\stat{grusho}

\def\tit{СЛОЖНЫЕ ПРИЧИННО-СЛЕДСТВЕННЫЕ СВЯЗИ}

\def\titkol{Сложные причинно-следственные связи}

\def\aut{А.\,А.~Грушо$^1$, Н.\,А.~Грушо$^2$, М.\,И.~Забежайло$^3$, 
Е.\,Е.~Тимонина$^4$, С.\,Я.~Шоргин$^5$}

\def\autkol{А.\,А.~Грушо, Н.\,А.~Грушо, М.\,И.~Забежайло и~др.}
%$^3$,  Е.\,Е.~Тимонина$^4$, С.\,Я.~Шоргин$^5$}

\titel{\tit}{\aut}{\autkol}{\titkol}

\index{А.\,А.~Грушо$^1$, Н.\,А.~Грушо$^2$, М.\,И.~Забежайло$^3$, 
Е.\,Е.~Тимонина$^4$, С.\,Я.~Шоргин$^5$}
\index{Razumchik R.\,V.}
\index{Rumyantsev A.\,S.}
\index{Garimella R.\,M.}


%{\renewcommand{\thefootnote}{\fnsymbol{footnote}} \footnotetext[1]
%{Исследование выполнено за счет гранта Российского 
%научного фонда №\,21-71-10135. Работа проводилась с~использованием 
%инфраструктуры Центра коллективного пользования <<Высокопроизводительные 
%вычисления и~большие данные>> (ЦКП <<Информатика>>) ФИЦ ИУ РАН (г.~Москва).}}


\renewcommand{\thefootnote}{\arabic{footnote}}
\footnotetext[1]{Федеральный исследовательский центр <<Информатика и~управление>> Российской 
академии наук, \mbox{grusho@yandex.ru}}
\footnotetext[2]{Федеральный исследовательский центр <<Информатика и~управление>> Российской 
академии наук, \mbox{info@itake.ru}}
\footnotetext[3]{Федеральный исследовательский центр <<Информатика и~управление>> Российской 
академии наук, \mbox{m.zabezhailo@yandex.ru}}
\footnotetext[4]{Федеральный исследовательский центр <<Информатика и~управление>> Российской 
академии наук, \mbox{eltimon@yandex.ru}}
\footnotetext[5]{Федеральный исследовательский центр <<Информатика и~управление>> Российской 
академии наук, \mbox{sshorgin@ipiran.ru}}

\vspace*{-6pt}



  
  \Abst{Рассматривается задача построения логического вывода конкретного свойства из 
данных, выбираемых из множества наборов исходных данных. При решении задачи 
учитываются как возможности нарушения при\-чин\-но-след\-ст\-вен\-ной схемы из-за шума, так 
и~возможности недостижимости получения необходимого вывода. Построена при\-чин\-но-след\-ст\-вен\-ная 
схема приближенного вывода, состоящая из объектов покрытий причин 
и~следствий, которая начинается из исходных предпосылок и~заканчивается выводом объекта, 
содержащего требуемое свойство.
  Для описания процесса порождения логического вывода объекта с~интересующим свойством 
из исходных данных введено понятие активации объектов. Это свойство позволяет представить 
схему вывода в~форме DAG  (Directed Acyclic 
Graph). Построен простой алгоритм конструирования при\-чин\-но-след\-ст\-вен\-ной схемы 
из исходных данных к~объекту, содержащему искомое свойство. Этот 
алгоритм также определяет условия существования возможностей вывода из исходных условий 
объекта с~требуемым свойством.}
  
  \KW{причинно-следственные связи; приближенный логический вывод; вероятность 
правильного вывода в~условиях шума}

\DOI{10.14357/19922264230212}{TGXQIW} 
  
\vspace*{-6pt}


\vskip 10pt plus 9pt minus 6pt

\thispagestyle{headings}

\begin{multicols}{2}

\label{st\stat}
  
  
  \section{Введение }
  
  Исследованию при\-чин\-но-след\-ст\-вен\-ных связей посвящено много 
научных работ~[1--3]. С~по\-мощью при\-чин\-но-след\-ст\-вен\-ных связей 
решаются задачи поиска первопричины сбоев (Root Cause Analyses~--- RCA) 
и~аномалий~[4--6] в~компьютерных системах и~сетях. Методы искусственного 
интеллекта в~медицине~\cite{3-gr, 7-gr} основаны на поиске причин заболевания. 
Причинно-следственные связи используются при анализе атак на компьютерные 
системы~[8] и~др. 
{\looseness=1

}
  
  Причинно-следственные связи обычно моделируются графами. Причины 
  и~следствия~--- узлы\linebreak графа, если~$A$~--- причина, а~$B$~--- следствие этой 
причины, то графически это изображается ориентированным ребром (дугой) от 
узла~$A$ к~узлу~$B$. Сложные при\-чин\-но-след\-ст\-вен\-ные связи часто 
представляются ориентированными ациклическими графами (DAG)~[9, 10]. 
  
  В причинно-следственные (каузальные) модели~\cite{3-gr, 11-gr} часто 
вносятся элементы случайности. В~[11] рассматривается функциональная 
зависимость при\-чи\-на\;$\to$\;след\-ст\-вие, в~которую\linebreak в~качестве 
дополнительного аргумента внесена\linebreak случайная величина. Такой подход 
значительно затрудняет исследование каузальных отношений, так как требует 
внесения дополнительных ограничений\linebreak в~мо\-дель. 

  
  Схема внесения случайности в~[11] может быть изменена, как это было 
сделано в~работе~[12], а~именно: при\-чин\-но-след\-ст\-вен\-ная связь 
рас\-смат\-ри\-ва\-ет\-ся как детерминированное следствие причины, но в~следствие 
вносится случайный шум, который может не позволить получить (<<увидеть>>) 
данное следствие. В~такой модели допущение\linebreak о~независимом шуме для каждого 
следствия рас\-смат\-ри\-ва\-емой причины позволяет получать точные оценки 
вероятностных распределений воз\-мож\-ности правильного распознавания наличия 
\mbox{причины} по на\-блюда\-емым следствиям~[13]. Более того, используя 
детерминированность отношения при\-чи\-на\;$\to$\;след\-ст\-вие в~такой модели, 
удалось получить оценки вероятностей до\-сти\-жи\-мости правильного результата 
классификации для сложных схем при\-чин\-но-след\-ст\-вен\-ных связей. 
  
  В данной работе рассматривается задача построения логического вывода 
конкретного свойства из данных, выбираемых из множества наборов исходных 
данных. При решении задачи учитываются как возможности нарушения  
при\-чин\-но-след\-ст\-вен\-ной схемы из-за шума, так и~возможности 
недостижимости получения необходимого вывода. По\-стро\-ена при\-чин\-но-след\-ст\-вен\-ная 
схема приближенного вывода, состоящая из объектов покрытий 
причин и~следствий, которая начинается из исходных предпосылок 
и~заканчивается выводом объекта, содержащего требуемое свойство.
  
  \section{Математическая модель причинно-следственных связей 
в~задачах классификации}
  
  В данной работе будем опираться на модель из публикации~[14]. Пусть задано 
некоторое пространство характеристик $U\hm= \{\alpha_1,\alpha_2, \ldots, 
\alpha_n\}$ и~множество объектов $\bm{O}\hm= \{O_1, O_2, \dots, O_m\}$. 
Каждый объект из множества~$\bm{O}$ есть подмножество пространства~$U$, 
но не всякое подмножество пространства~$U$ представляет собой объект. Будем 
считать, что множества характеристик в~объектах неизвестны. Положим по 
определению, что $A\hm\subseteq U$ служит причиной появления следствия 
$B\hm\subseteq U$, если характеристики множеств~$A$ и~$B$ могут 
взаимодействовать между собой, и~в~этом случае при появлении причины~$A$ 
детерминированно возникает следствие~$B$. При этом будем считать, что 
появившееся следствие~$B$ выступает носителем свойства~$B$.
{\looseness=1

}
  
  Для удобства можно представить механизм появления следствия~$B$  
сле\-ду\-ющим образом. Появление причины~$A$ в~ка\-ком-то смысле означает 
активацию элементов множества~$A$. Тогда, используя все существующие 
взаимодействия с~потенциальными следствиями, $A$ рассылает <<сигнал>> от 
своих характеристик по этим взаимодействиям. Если~$B$ является 
следствием~$A$, то принятые <<сигналы>> активируют элементы 
множества~$B$, которое может в~свою очередь рассылать <<сигналы>> о~своей 
активации. 
  
  Положим, что взаимодействие~$A$ может происходить, не обязательно 
<<сигнализируя>> о~своей активации другому подмножеству пространства~$U$, 
но также некоторому взаимодействующему с~$A$ множеству характеристик 
в~другом пространстве характеристик~$U^*$. 
  
  Более того, активация~$B$ происходит только тогда, когда приходят <<сигналы>> 
от всех характеристик множества~$A$, т.\,е.\ причина~$A$ является 
минимальным множеством, порождающим следствие~$B$. Следует обратить 
внимание на то, что множества~$A$ и~$B$ не обязательно представляют собой 
объекты, хотя понятие активации характеристик объекта может относиться не 
только к~характеристикам~$A$. В~этом случае в~пространстве~$U^*$ селективно 
выбираются <<сигналы>> от характеристик из~$A$ и~активируются все объекты, 
определенные на пространстве~$U^*$ и~содержащие~$B$. 
  
  Далее будем считать, что понятие активации в~множестве объектов~$\bm{O}$ 
относится только к~объектам. Активация причины~$A$ позволяет активировать 
все следствия~$A$ при наличии взаимодействия с~соответствующими 
пространствами \mbox{характеристик.}

  
  Будем называть любой объект, содержащий причину~$A$, \textit{покрытием 
причины}, а~любой объект, содержащий следствие~$B$ причины~$A$, 
\textit{покрытием следствия}~$B$. Будем предполагать, что в~отношении 
<<$\mbox{причина}\ A\hm\to \mbox{следствие}\ B$>> существует единственная 
причина~$A$ у~данного следствия~$B$. 
  
  Расширим понятие покрытия причины или следствия следующим образом. 
Пусть часть характеристик множества~$A$ содержится в~объекте~$O_1$, 
а~остальные характеристики~$A$ содержатся в~объекте~$O_2$. Тогда при 
активации обоих объектов~$O_1$ и~$O_2$ и~наличии взаимодействия обоих 
объектов с~пространством~$U^*$ возникает множество <<сигналов>> для 
активации следствия~$B$, т.\,е.\ активируется объект, содержащий~$B$. В~этом 
случае причина~$A$ распределена между объектами~$O_1$ и~$O_2$. 
  
   Если причина~$A$, состоящая из характеристик пространства~$U$, порождает 
следствие~$B$ в~пространстве характеристик~$U^*$, а~$B$ как причина 
порождает следствие~$C$ в~пространстве характеристик~$U^{**}$, то из условия 
детерминированности отношения при\-чи\-на\;$\to$\;след\-ст\-вие следует 
транзитивность при\-чин\-но-след\-ст\-вен\-ных связей. Это означает, что~$A$ 
служит \textit{транзитивной причиной} появления свойства~$C$. При этом 
введение понятия транзитивной причины не нарушает свойство единственности 
причины, так как у~$C$ есть единственная причина~--- это~$B$ из пространства 
характеристик~$U^*$, а~у~$B$ есть единственная причина~--- это~$A$ из 
пространства характеристик~$U$. 
  
  Если $A\to B$ и~$A$ может быть покрыто распределенной сис\-те\-мой 
объектов, то~$B$ должно быть покрыто хотя бы одним объектом $O(B)$. Это 
свойство на графе при\-чин\-но-след\-ст\-вен\-ных отношений будет 
соответствовать совокупности дуг, входящих в~$O(B)$ и~выходящих из объектов 
распределенного покрытия~$A$. Заметим, что активация нужна для выделения 
объекта, содержащего следствие. Поскольку возможно участие $O(B)$ 
в~транзитивной причине, то активация сохраняется. 
  
  Предположим, что нас интересует свойство~$C$, которое является 
следствием~$B$, но неизвестно ни одного его покрытия. Пусть известно, что~$B$ 
может быть активировано из строгих подмножеств множества~$A$, которые 
образуют причину~$B$. Если возможно активировать объекты~$O_1$ и~$O_2$, 
содержащие подмножества множества~$A$ и~образующие причину~$B$, то по 
свойству транзитивной причины будет активировано следствие~$C$ вместе 
с~некоторым объектом $O(C)$, содержащим~$C$. Объекты~$O_1$ и~$O_2$ могут 
быть активированы исходными (входными) данными или другими 
активированными объек\-тами.
{\looseness=1

} 
  
  Нетрудно видеть, что в~построенном по этим правилам графе с~любым числом 
узлов не могут образоваться ориентированные циклы. Ориентированный цикл 
требует повторной активации хотя бы одного объекта, который ранее был 
активирован. Но повторная активация невозможна по соглашению выше. Тогда 
цикл не может быть замкнут. 

  
  Если в~объекте, предполагающем зацикливание, нужны другие характеристики, 
не использованные раньше, то (по соглашению) активируются\linebreak все объекты, 
содержащие активированное следствие. Отсюда следует, что для получения 
в~качестве следствия нужного свойства граф многих  
при\-чин\-но-след\-ст\-вен\-ных связей объектов, содержащих используемые  
при\-чин\-но-след\-ст\-вен\-ные связи, должен быть ориентированным связным 
графом без ориентированных циклов, т.\,е.\ DAG. При этом, чтобы не оставлять 
висячие вершины, среди всех объектов, содержащих нужное следствие, можно 
выбрать одно для дальнейшего построения  
при\-чин\-но-след\-ст\-вен\-но\-го графа, но можно оставить и~другие объекты для 
последующего контроля правильности порожденного графа. Отметим, что по 
соглашению выше наличие свойства или существование нужного следствия 
в~объекте известно, но может быть неизвестен состав его характеристик.
{\looseness=1

}
  
  \section{Вероятностная модель шума }
  
  Построим модель шума, который может влиять на характеристики 
 при\-чин\-но-след\-ст\-вен\-ных связей. Пусть дано  
при\-чин\-но-след\-ст\-вен\-ное отношение $A\hm\to B$ и~в пространстве 
характеристик~$U^*$ следствия~~$B$ любая его характеристика независимо от 
других может из-за воздействия шума перестать распознаваться 
с~вероятностью~$p$. Одинаковые вероятности нераспознавания характеристик 
взяты для удобства, а~свойство независимости допустимо, так как дискретный 
шум в~условиях неупорядоченности характеристик в~множестве~$B$ трудно 
описать иначе. 
  
  Так как следствие, если в~нем присутствуют все характеристики, валидно, то 
нераспознавание хотя бы одной из них означает непоявление всего следствия 
и~всех связанных со следствием активаций. Таким образом, граф  
при\-чин\-но-след\-ст\-вен\-ных связей (при\-чин\-но-след\-ст\-вен\-ная схема), 
состоящий из~$M$~узлов, расположенных в~пространствах характеристик $U_1, 
U_2,\ldots , U_M$, является ориентированным, связным и~ациклическим, т.\,е.\ 
DAG. В~условиях наличия шума могут получаться частичные подграфы, которые 
заведомо не порождают интересующее свойство.
  
  Для простоты предположим, что все причины и~следствия имеют одинаковое 
число характеристик~$s$. Тогда вероятность непоявления данного следствия~$B$ 
(далее~--- событие~$\overline{B}$) равна
  $$
  {\sf P}(\overline{B} =1-(1-p)^s.
  $$
  
  Вероятность того, что при\-чин\-но-след\-ст\-вен\-ная схема от исходных 
данных до порождения тре\-бу\-емо\-го следствия позволяет правильно решать задачу 
логического вывода, равна
  $$
  {\sf P}=(1-p)^{sM}\,,
  $$
а вероятность того, что причинно-следственная схема не позволяет правильно 
решать данную задачу (сбой), соответственно, равна $1\hm-{\sf P}$. 

  Для контроля решения задачи построения логического вывода конкретного 
свойства из исходных данных в~схеме при\-чин\-но-след\-ст\-вен\-ных связей 
можно оставлять контрольные объекты, содержащие соответствующие следствия. 
Отсутствие контрольного объекта означает сбой. Это позволяет удаленно 
контролировать работоспособность схемы до сбоя.
  
  Отметим, что для логического вывода определенного свойства могут 
существовать несколько DAG. Это возможно, например, за счет различного 
представления распределенной причины и~построения других вариантов 
исходного графа при\-чин\-но-след\-ст\-вен\-ной схемы.
  
  \section{Алгоритмы построения DAG причинно-следственных 
связей}
  
  В данном разделе построим алгоритм получения искомого свойства~$C$ 
с~помощью схемы при\-чин\-но-след\-ст\-вен\-ных связей. Пусть  
по-преж\-не\-му~$\bm{O}$~--- ограниченное множество объектов в~семействе 
различных пространств характеристик $U_1, U_2, \ldots , U_M$, которые можно 
использовать при построении  
при\-чин\-но-след\-ст\-вен\-ной схемы.
  
  По определению свойство~$C$ принадлежит объекту $O(C)$ в~одном из 
рассматриваемых пространств характеристик. В~противном случае свойство~$C$ 
недостижимо. Свойство~$C$ должно быть получено из исходных данных, 
которые активируют $O(C)$. Если такие исходные данные сразу активируют 
$O(C)$, то задача решена. Если это не так, то надо построить 
 при\-чин\-но-след\-ст\-вен\-ную схему, в~которой последовательность активаций 
порождает в~качестве одного из следствий $O(C)$. Для этого выберем из $U_1, 
U_2, \ldots , U_M$ те пространства, которые взаимодействуют с~пространством, 
содержащим $O(C)$. Последовательным перебором опробуем все объекты этих 
пространств на предмет поиска объекта, содержащего причину следствия~$C$. 
Если такого объекта не найдено, то опробуем все пары объектов в~объединении 
этих пространств. Если не найдена распределенная причина для следствия~$C$, 
то перейдем к~опробованию троек объектов, и~т.\,д. 
{\looseness=1

}
  
  Ограничениями этого алгоритма могут стать допустимая сложность 
вычислений или ограниченность множества возможных наборов объектов. Если 
найдена прямая или распределенная причина для~$C$, тогда проверяется 
возможность активации соответствующих объектов из исходных данных. Если 
найдены несколько распределенных причин следствия~$C$, то далее независимо 
строятся несколько допустимых DAG для каждой распределенной причины. Если 
найден хотя бы один\linebreak объект, который может быть активирован из исходных 
данных, то в~дальнейших шагах алгоритма построения данного DAG этот объект 
не участ\-вует. 
{\looseness=1

}
  
  Если в~найденной прямой или распределенной причине следствия~$C$ 
участвует причина или часть причины~$B$, то алгоритм работает с~$B$ так же, 
как происходила работа с~$C$. 
  
  Алгоритм прекращает свою работу, когда все объекты в~порожденной 
причинно-следственной схеме могут быть активированы из исходных данных. 
При этом совсем не обязательно, что все исходные данные должны быть 
использованы. Следует также учитывать, что на каждом цикле алгоритма при 
нахождении искомой части необходимой причины нужно активировать все 
объекты соответствующего пространства характеристик, содержащих искомую 
часть причины. 
  
  Если при выполнении перечисленных выше ограничений невозможно вывести 
или получить из исходных данных все узлы графа, то необходимо перейти 
к~следующему допустимому DAG. 
  
  Если все возможности исчерпаны без положительного результата, то считаем, 
что~$C$ не может быть получено из данного набора исходных данных. 
{\looseness=1

}
  
  При построении шагов алгоритма для объектов, порождающих распределенные 
причины, возникает необходимость порождения причины объекта с~неизвестным 
свойством в~отличие от прямых причин, которые могут быть идентифицированы. 
В этом случае используется содержащий эту часть причины объект и~для него 
ищется другой объект, активирующий данный.
  
  Несмотря на сложность переборного процесса, алгоритм корректно определен. 
В~самом деле, отсутствие прямой или распределенной причины на каждом шаге 
алгоритма определяется ограничением сложности или исчерпанностью перебора 
либо возможностью остановки данной ветви алгоритма за счет исходных данных. 
Как было показано ранее, ориентированные циклы получить невозможно, других 
противоречий объектов в~узлах в~каж\-дом DAG не существуют. Наборы 
характеристик в~причинах и~следствиях неизвестны, но активация происходит на 
объектах, содержащих причину или следствие. Поэтому один объект может 
участвовать несколько раз с~разными причинами или следствиями.

  
  Можно предусмотреть некоторые сокращения времени перебора за счет 
распараллеливания. Укажем точки возможного выхода на параллельные пути 
выполнения алгоритма. Прежде всего, построение продолжений DAG из 
различных узлов их возможного возникновения. Другой вариант 
распараллеливания состоит в~том, что шаг алгоритма, состоящий в~поиске 
причины для данного следствия, универсален и~может быть масштабирован 
в~произвольном числе экземпляров при использовании на различных ветках 
поиска промежуточных причин.

%\vspace*{-12pt}
  
  \section{Заключение }
  
  Идея покрытия причин и~следствий объектами из определенного множества 
лежит в~основе приближенного логического вывода. Например, объектами могут 
быть программно-технические устройства в~реализуемой информационной 
технологии, а~искомым свойством служить сбой в~этой технологии.

  
   В работе построена логическая схема приближенного вывода, состоящая из 
объектов покрытий причин и~следствий, которая начинается из исходных 
предпосылок и~заканчивается выводом объекта, содержащего требуемое свойство. 
В условиях появления случайного шума оценена вероятность сбоя и~правильного 
срабатывания логического вывода объекта, содержащего требуемое свойство. 
В~отличие от большинства схем при\-чин\-но-след\-ст\-вен\-но\-го вывода шум 
может присутствовать только в~следствиях простейших схем  
при\-чи\-на\;$\to$\;след\-ст\-вие. 

  
  Для описания процесса порождения логического вывода объекта 
с~интересующим свойством из исходных данных введено понятие активации 
объектов. Это свойство позволяет представить схему вывода в~форме DAG. 
Построен простой алгоритм конструирования при\-чин\-но-след\-ст\-вен\-ной 
схемы из исходных данных к~объекту, содержащему искомое свойство. Этот 
алгоритм также определяет условия существования возможностей вывода из 
исходных условий объекта с~требуемым свой\-ст\-вом. 
{\looseness=1

}

%\vspace*{-24pt}
  
{\small\frenchspacing
 {\baselineskip=11.5pt
 %\addcontentsline{toc}{section}{References}
 \begin{thebibliography}{99}
\bibitem{1-gr}
\Au{Halpern J.\,Y., Pearl~J.} Causes and explanations: A~structural-model approach. 
Part~I: Causes~// Brit. J. Philos. Sci., 2005. Vol.~56. No.\,4. P.~843--887. 
\bibitem{2-gr}
\Au{Pearl J.} Causal inference~// Causality: Objectives and assessment~/ Eds. I.~Guyon, D.~Janzing, B.~Scholkopf.~--- Proceedings of machine learning research 
ser.~--- Whistler, Canada, 2010. Vol.~6. P.~39--58. 
\bibitem{3-gr}
\Au{Pearl J.} The mathematics of causal inference~// Joint Statistical Meetings Proceedings.~--- ASA, 
2013. P.~2515--2529. 

\bibitem{6-gr} %4
\Au{Jurn J., Kim~T., Kim~H.} A~survey of automated root cause analysis of software vulnerability~// 
Innovative mobile and internet services in ubiquitous computing~/ Eds. L.~Barolli, F.~Xhafa, 
N.~Javaid, T.~Enokido.~--- Advances in intelligent systems and computing ser.~--- Cham: Springer, 
2019. Vol.~773. P.~756--761. doi: 10.1007/978-3-319-93554-6\_74.

\bibitem{4-gr} %5
\Au{Grusho A., Grusho~N., Zabezhailo~M., Timonina~E., Senchilo~V.} Metadata for root cause 
analysis~// Communications ECMS, 2021. Vol.~35. Iss.~1. P.~267--271. doi: 10.7148/ 2021-0267.
\bibitem{5-gr} %6
\Au{Grusho A.\,A., Grusho~N.\,A., Zabezhailo~M.\,I., Timonina~E.\,E.} Localization of the root cause 
of the anomaly~// Autom. Control Comp.~S., 2021. Vol.~55. No.\,8. P.~978--983. doi: 10.3103/s0146411621080137.

\bibitem{7-gr}
\Au{Грушо А.\,А., Грушо Н.\,А., Забежайло~М.\,И., Тимонина~Е.\,Е.} Поддержка решения задач 
диагностического типа~// Сис\-те\-мы и~средства информатики, 2021. Т.~31. №\,1. С.~69--81. doi:
10.14357/08696527210106.
\bibitem{8-gr}
\Au{Грушо А.\,А., Забежайло~М.\,И., Зацаринный~А.\,А., Тимонина~Е.\,Е.} О~некоторых 
возможностях управ\-ле\-ния ресурсами при организации проактивного противодействия 
компьютерным атакам~// Информатика и~её применения, 2018. Т.~12. Вып.~1. С.~62--70. doi: 
10.14357/19922264180108.


\bibitem{9-gr}
\Au{Williams T.\,C., Bach~C.\,C., Matthiesen~N.\,B., Henriksen~T.\,B., Gagliardi~L.} Directed acyclic 
graphs: a~tool for causal studies in pediatrics~// Pediatr. Res., 2018. Vol.~84. P.~487--493.
doi: 10.1038/s41390-018-0071-3.
\bibitem{10-gr}
\Au{Grusho A., Grusho~N., Zabezhailo~M., Timonina~E.} Generation of metadata for network 
control~// Distributed computer and communication networks~/ Eds. V.\,M.~Vishnevskiy, 
K.\,E.~Samouylov, D.\,V.~Kozyrev.~--- Lecture notes in computer science ser.~--- Cham: Springer, 
2020. Vol.~12563. P.~723--735. doi: 10.1007/978-3-030-66471-8\_55.
\bibitem{11-gr}
\Au{Sch$\ddot{\mbox{o}}$lkopf~B.} Causality for machine learning.~--- Cornell University,  
2019. arXiv:1911.10500v2 [cs.LG]. 20~p.
\bibitem{12-gr}
\Au{Грушо А.\,А., Грушо~Н.\,А., Забежайло~М.\,И., Кульченков~В.\,В., Тимонина~Е.\,Е., 
Шоргин~С.\,Я.} При\-чин\-но-след\-ст\-вен\-ные связи в~задачах классификации~// Информатика 
и~её применения, 2023. Т.~17. Вып.~1. С.~43--49. doi: 10.14357/19922264230106.

\bibitem{13-gr}
\Au{Грушо А.\,А., Забежайло~М.\,И., Кульченков~В.\,В., Смирнов~Д.\,В., Тимонина~Е.\,Е., 
Шоргин~С.\,Я.} При\-чин\-но-след\-ст\-вен\-ные связи в~задачах анализа не\-наблюда\-емых 
свойств процессов~// Системы и~средства информатики, 2023. Т.~33. №\,2. С.~71--78.
\bibitem{14-gr}
\Au{Аншаков О.\,М.} Об одной интерпретации ДСМ-метода автоматического порождения 
гипотез~// Автоматическое порождение гипотез в~интеллектуальных сис\-те\-мах~/
Под ред.\ В.\,К.~Финна.~--- М.: 
Либроком, 2009. С.~81--95.

\end{thebibliography}

 }
 }

\end{multicols}

\vspace*{-8pt}

\hfill{\small\textit{Поступила в~редакцию 11.04.23}}

\vspace*{6pt}

%\pagebreak

%\newpage

%\vspace*{-28pt}

\hrule

\vspace*{2pt}

\hrule

\vspace*{-2pt}

\def\tit{COMPLEX CAUSE-AND-EFFECT RELATIONSHIPS}


\def\titkol{Complex cause-and-effect relationships}


\def\aut{A.\,A.~Grusho, N.\,A.~Grusho, M.\,I.~Zabezhailo, E.\,E.~Timonina, and~S.\,Ya.~Shorgin}

\def\autkol{A.\,A.~Grusho, N.\,A.~Grusho, M.\,I.~Zabezhailo, et al.}
%E.\,E.~Timonina, and~S.\,Ya.~Shorgin}

\titel{\tit}{\aut}{\autkol}{\titkol}

\vspace*{-13pt}


\noindent
Federal Research Center ``Computer Science and Control'' of the Russian Academy of Sciences, 
 44-2~Vavilov Str., Moscow 119133, Russian Federation

\def\leftfootline{\small{\textbf{\thepage}
\hfill INFORMATIKA I EE PRIMENENIYA~--- INFORMATICS AND
APPLICATIONS\ \ \ 2023\ \ \ volume~17\ \ \ issue\ 2}
}%
 \def\rightfootline{\small{INFORMATIKA I EE PRIMENENIYA~---
INFORMATICS AND APPLICATIONS\ \ \ 2023\ \ \ volume~17\ \ \ issue\ 2
\hfill \textbf{\thepage}}}

\vspace*{2pt}





\Abste{The paper discusses the task of constructing a logical inference of the specific property 
from data selected from a plurality of sets of source data. When solving the problem, it should be taken 
into account both the possibility of violating the cause-and-effect scheme due to noise and the 
possibility of not achieving the necessary conclusion. The cause-and-effect scheme of approximate 
inference has been built, consisting of objects of covering causes and consequences, which begins 
from the initial prerequisites and ends with the output of the object containing the required property.
To describe the process of generating logical inference of the object with the property of interest from 
the source data, the concept of activating objects is introduced. This property allows one to represent 
the inference scheme in the form of a~DAG (Directed Acyclic 
Graph). The simple algorithm for constructing the cause-and-effect 
scheme was built from the source data up to the object containing the desired property. This algorithm 
also determines the conditions for the existence of the ability to inference from the original conditions 
up to the object with the required property.}

\KWE{cause-and-effect relationships; approximate logical inference; probability of correct inference 
under noise conditions}


\DOI{10.14357/19922264230212}{TGXQIW}

%\vspace*{-18pt}

%\Ack
%\noindent

%\vspace*{4pt}

  \begin{multicols}{2}

\renewcommand{\bibname}{\protect\rmfamily References}
%\renewcommand{\bibname}{\large\protect\rm References}

{\small\frenchspacing
 {%\baselineskip=10.8pt
 \addcontentsline{toc}{section}{References}
 \begin{thebibliography}{99} 
\bibitem{1-gr-1}
\Aue{Halpern, J.\,Y., and J.~Pearl.} 2005. Causes and explanations: A~structural-model approach. 
Part~I: Causes. \textit{Brit. J. Philos. Sci.} 56(4):843--887.
\bibitem{2-gr-1}
\Aue{Pearl, J.} 2010. Causal inference. \textit{Causality: Objectives and 
assessment}. Eds. I.~Guyon, D.~Janzing, and B.~Scholkopf. Proceedings of machine learning 
research ser. Whistler, Canada. 6:39--58.
\bibitem{3-gr-1}
\Aue{Pearl, J.} 2013. The mathematics of causal inference. \textit{Joint Statistical Meetings 
Proceedings}. ASA. 2515--2529.

\bibitem{6-gr-1} %4
\Aue{Jurn J., T.~Kim, and H.~Kim.} 2019. A~survey of automated root cause analysis of software 
vulnerability. \textit{Innovative mobile and internet services in ubiquitous computing}. Eds. 
L.~Barolli, F.~Xhafa, N.~Javaid, and T.~Enokido. Advances in intelligent systems and computing ser. 
Cham: Springer. 773:756--761. doi: 10.1007/978-3-319-93554-6\_74.

\bibitem{4-gr-1} %5
\Aue{Grusho, A., N.~Grusho, M.~Zabezhailo, E.~Timonina, and V.~Senchilo.} 2021. Metadata for 
root cause analysis. \textit{Communications ECMS} 35(1):267--271. doi: 10.7148/2021-0267.
\bibitem{5-gr-1} %6
\Aue{Grusho, A.\,A., N.\,A.~Grusho, M.\,I.~Zabezhailo, and E.\,E.~Timonina.} 2021. Localization of 
the root cause of the anomaly. \textit{Autom. Control Comp.~S.} 55(8):978--983. doi: 
10.3103/s0146411621080137.

\bibitem{7-gr-1}
\Aue{Grusho, A.\,A., N.\,A.~Grusho, M.\,I.~Zabezhailo, and E.\,E.~Timonina.} 2021.  Podderzhka 
resheniya zadach diagnosticheskogo tipa [Support for solving diagnostic type problems]. 
\textit{Sistemy i~Sredstva Informatiki~--- Systems and Means of Informatics} 31(1):69--81. doi: 
10.14357/08696527210106.
\bibitem{8-gr-1}
\Aue{Grusho, A.\,A., M.\,I.~Zabezhailo, A.\,A.~Zatsarinny, and E.\,E.~Timonina.} 2018. 
O~nekotorykh vozmozhnostyakh upravleniya resursami pri organizatsii proaktivnogo protivodeystviya 
komp'yuternym atakam [On some possibilities of resource management for organizing active 
counteraction to computer attacks]. \textit{Informatika i~ee Primeneniya~--- Inform. Appl.}  
12(1):62--70. doi: 10.14357/19922264180108.
\bibitem{9-gr-1}
\Aue{Williams, T.\,C., C.\,C.~Bach, N.\,B.~Matthiesen, T.\,B.~Henriksen, and L.~Gagliardi.} 2018. 
Directed acyclic graphs: A~tool for causal studies in pediatrics. \textit{Pediatr. Res.} 84(4):487--493. 
doi: 10.1038/s41390-018-0071-3.
\bibitem{10-gr-1}
\Aue{Grusho, A., N.~Grusho, M.~Zabezhailo, and E.~Timonina.} 2020. Generation of metadata for 
network control. \textit{Distributed computer and communication networks.} Eds. V.\,M.~Vishnevskiy, 
K.\,E.~Samouylov, and D.\,V.~Kozyrev. Lecture notes in computer science ser.  Cham: Springer. 
12563:723--735. doi: 10.1007/978-3-030-66471-8\_55.
\bibitem{11-gr-1}
\Aue{Sch$\ddot{\mbox{o}}$lkopf, B.} 2019. Causality for machine learning. Cornell University. \textit{arXiv.org}. 20~p. 
Available at: {\sf https://\linebreak arxiv.org/abs/1911.10500v2} (accessed June~12, 2023).
\bibitem{12-gr-1}
\Aue{Grusho, A.\,A., N.\,A.~Grusho, M.\,I.~Zabezhailo, V.\,V.~Kul\-chen\-kov, E.\,E.~Timonina, and 
S.\,Ya.~Shorgin.} 2023. Prichinno-sledstvennye svyazi v~zadachakh klassifikatsii [Causal 
relationships in classification problems]. \textit{Informatika i~ee Primeneniya~--- Inform. Appl.}  
17(1):43--49. doi: 10.14357/19922264230106.
\bibitem{13-gr-1}
\Aue{Grusho, A.\,A., M.\,I.~Zabezhailo, V.\,V.~Kul'chenkov, D.\,V.~Smirnov, E.\,E.~Timonina, and 
S.\,Ya.~Shorgin.} 2023. Prichinno-sledstvennye svyazi v~zadachakh analiza ne\-nablyu\-da\-emykh 
svoystv protsessov [Cause-and-effect relationships in analysis of unobservable process properties]. 
\textit{Sistemy i~Sredstva Informatiki~--- Systems and Means of Informatics} 33(2):71--78.
\bibitem{14-gr-1}
\Aue{Anshakov, O.\,M.} 2009. Ob odnoy interpretatsii DSM-metoda avtomaticheskogo porozhdeniya 
gipotez [On one interpretation of the JSM-method of automatic generation of hypotheses]. 
\textit{Avtomaticheskoe porozhdenie gipotez v~intellektual'nykh sistemakh} [Automatic hypotheses 
generation in intelligent systems]. Ed.\ V.\,K.~Finn. Moscow: Librokom. 81--95.
\end{thebibliography}

 }
 }

\end{multicols}

\vspace*{-6pt}

\hfill{\small\textit{Received April 11, 2023}} 

\vspace*{-12pt}

\Contr


\noindent
\textbf{Grusho Alexander A.} (b.\ 1946)~--- Doctor of Science in physics and mathematics, professor, 
principal scientist, Institute of Informatics Problems, Federal Research Center ``Computer Science and 
Control'' of the Russian Academy of Sciences, 44-2~Vavilov Str., Moscow 119333, Russian 
Federation; \mbox{grusho@yandex.ru}

\vspace*{3pt}

\noindent
\textbf{Grusho Nikolai A.} (b.\ 1982)~--- Candidate of Science (PhD) in physics and mathematics, 
senior scientist, Institute of Informatics Problems, Federal Research Center ``Computer Science and 
Control'' of the Russian Academy of Sciences, 44-2~Vavilov Str., Moscow 119133, Russian 
Federation; \mbox{info@itake.ru}

\vspace*{3pt}

\noindent
\textbf{Zabezhailo Michael I.} (b.\ 1956)~--- Doctor of Science in physics and mathematics, professor, 
principal scientist, A.\,A.~Dorodnicyn Computing Center, Federal Research Center ``Computer 
Science and Control'' of the Russian Academy of Sciences, 40~Vavilov Str., Moscow 119333, Russian 
Federation; \mbox{m.zabezhailo@yandex.ru}

\vspace*{3pt}

\noindent
\textbf{Timonina Elena E.} (b.\ 1952)~--- Doctor of Science in technology, professor, leading scientist, 
Institute of Informatics Problems, Federal Research Center ``Computer Science and Control'' of the 
Russian Academy of Sciences, 44-2~Vavilov Str., Moscow 119133, Russian Federation; 
\mbox{eltimon@yandex.ru}

\vspace*{3pt}

\noindent
\textbf{Shorgin Sergey Ya.} (b.\ 1952)~--- Doctor of Science in physics and mathematics, professor, 
principal scientist, Institute of Informatics Problems, Federal Research Center ``Computer Science and 
Control'' of the Russian Academy of Sciences, 44-2~Vavilov Str., Moscow 119133, Russian 
Federation; \mbox{sshorgin@ipiran.ru}

\label{end\stat}

\renewcommand{\bibname}{\protect\rm Литература} 