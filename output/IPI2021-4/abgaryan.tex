\def\stat{abgaryan}

\def\tit{РАСПРЕДЕЛЕННАЯ ИНФОРМАЦИОННАЯ СИСТЕМА ДЛЯ~РАСЧЕТА СТРУКТУРНЫХ 
СВОЙСТВ КОМПОЗИЦИОННЫХ~МАТЕРИАЛОВ$^*$}

\def\titkol{Распределенная информационная система для расчета структурных 
свойств композиционных материалов}

\def\aut{К.\,К.~Абгарян$^1$, Е.\,С.~Гаврилов$^2$}

\def\autkol{К.\,К.~Абгарян, Е.\,С.~Гаврилов}

\titel{\tit}{\aut}{\autkol}{\titkol}

\index{Абгарян К.\,К.}
\index{Гаврилов Е.\,С.}
\index{Abgaryan K.\,K.}
\index{Gavrilov E.\,S.}


{\renewcommand{\thefootnote}{\fnsymbol{footnote}} \footnotetext[1]
{Работа выполнена при поддержке Министерства науки и~высшего образования Российской Федерации (проект 
075-15-2020-799).}}


\renewcommand{\thefootnote}{\arabic{footnote}}
\footnotetext[1]{Федеральный исследовательский центр <<Информатика и~управление>> Российской академии наук; Московский 
авиационный институт (национальный исследовательский университет), \mbox{kristal83@mail.ru}}
\footnotetext[2]{Федеральный исследовательский центр <<Информатика и~управление>> Российской академии наук; 
Московский авиационный институт (национальный исследовательский университет), \mbox{eugavrilov@gmail.com}}

\vspace*{-6pt}

    
           
      
      \Abst{Использование композиционных материалов нашло широкое применение 
в~различных отраслях инженерной деятельности, что обуслов\-ле\-но их преимуществами перед 
металлами при равных механических и~эксплуатационных свойствах. Для решения задач, 
возникающих в~об\-ласти создания композиционных материалов с~набором заданных свойств, 
сегодня широко применяются новые подходы к~разработке математических моделей 
и~информационных сис\-тем на их основе. В~данной работе пред\-став\-ле\-на оригинальная 
многомасштабная математическая модель, которая позволяет рассчитывать структурные 
характеристики композиционных материалов и~может быть использована для численного 
исследования усталостного разрушения композиционных материалов при случайных ударных 
по\-вреж\-де\-ни\-ях. На базе данной многомасштабной модели была создана распределенная 
информационная сис\-те\-ма для проведения широкомасштабных исследований в~об\-ласти 
моделирования композиционных материалов с~заданными свойствами. Развитие данного 
подхода в~дальнейшем поможет обеспечить формирование информации для обоснованного 
выбора композиционных материалов с~заданными свойствами для авиа\-ци\-он\-но-кос\-ми\-че\-ской 
и~других областей промышленности.}
      
      \KW{многомасштабное моделирование; композиционные материалы; интеграционная 
платформа; программный комплекс; распределенная сис\-тема}

\DOI{10.14357/19922264210407}
  
\vspace*{-4pt}


\vskip 10pt plus 9pt minus 6pt

\thispagestyle{headings}

\begin{multicols}{2}

\label{st\stat}
      
\section{Введение}

     На сегодня одной из важнейших и~актуальных\linebreak проблем, решаемых 
в~процессе создания летательных аппаратов, стала задача обеспечения их 
прочности при минимальных весовых затратах и~отсутствии коррозии, что 
обусловливает все более \mbox{широкое} применение композиционных материалов для 
производства основных силовых элементов. Однако с~расширением 
использования композиционных материалов, которые имеют высокие 
характеристики усталостной долговечности и~не подвержены коррозии, 
проявились проб\-ле\-мы, связанные, например, с~необходимостью учета случайных 
ударных воздействий. На сегодня они пред\-став\-ля\-ют\-ся одними из основных 
факторов риска при создании\linebreak авиационных конструкций из композиционных 
материалов. Для решения задач, возникающих в~об\-ласти создания 
композиционных материалов с~набором заданных свойств, широко применяются\linebreak 
новые подходы к~построению математических моделей и~информационных 
сис\-тем на их основе. В~ведущих международных и~российских цент\-рах 
проводятся широкомасштабные исследования в~данной об\-ласти. Так, во ФГУП 
ВНИИ ВИАМ сформулирована необходимость сис\-те\-ма\-ти\-за\-ции накопленных 
данных по проведенным исследованиям зависимостей <<со\-став--свой\-ст\-ва>> 
для создания математических моделей и~прогнозирования свойств новых 
материалов. Планируется создание экс\-пе\-ри\-мен\-таль\-но-рас\-чет\-ной базы для 
квалификации отечественных материалов в~Европе (EASA, European Union Aviation Safety Agency), а~так\-же по 
американским требованиям и~нормам (FAA, Federal Aviation Administration) ({\sf https://viam.ru/news/2013}).
     
     К программным комплексам, наиболее широко ис\-поль\-зу\-емым 
в~авиастроении при построении виртуальных моделей композитных конструкций 
на стадиях проектирования, анализа и~производства, можно отнести: 
     \begin{itemize}
     \item комплекс мирового уровня в~области ко\-неч\-но-эле\-мент\-ных расчетов на 
прочность ABAQUS; 
     \item современное программное обеспечение для инженерного анализа 
и~численного моделирования Ansys; 
     \item программный комплекс MSC Nastran для анализа методом конечных 
элементов (FEA, Finite Element Analysis), первоначально разработанный для NASA, исходный код 
NASTRAN в~настоящее время интегрирован в~ряд различных программных 
пакетов.
     \end{itemize}
     
     Программные комплексы позволяют учитывать свойства материалов, из 
которых состоит композитный пакет, такие как изотропность, ортотропность, 
анизотропность и~др. Важно, что большинство из них, отражая философию 
открытых сис\-тем, позволяют встраивать разработанные пользователем модули 
в~программный комплекс. Например, можно включить метод собственной 
разработки процесса имитации производства композиционых материалов. 
     
     Необходимо отметить, что при проектировании композитных конструкций 
очень важ\-ны прочностные расчеты. Все ведущие пакеты автоматизации 
инженерных задач (CAE, Computer Aided Engineering) предоставляют 
возможность проводить линейные и~нелинейные статические прочностные 
расчеты, моделирование пластичности и~пол\-зу\-чес\-ти, усталости и~др.
     
     Также доступно моделирование разрушения конструкций~--- как на 
макроуровне, так и~на мик\-ро\-уров\-не. Все пред\-став\-лен\-ные CAE-сис\-те\-мы дают 
возможность оценивать проч\-ность композита с~помощью различных критериев 
разрушения. Обычно ведущие сис\-те\-мы охватывают весь процесс 
технологического производства композитных конструкций. Программные 
комплексы включают в~себя базы данных со свойствами различных материалов. 
Для композитов есть возможность выбрать тип композита со стандартными 
свойствами (уг\-ле-, стек\-ло-, органопластики на основе 
эпоксифенолформальдегидных, кремнийорганических смол; эпоксидные 
боропластики и~т.\,д.). Имеется\linebreak возможность коррекции данных свойств 
и~внесения материала с~новыми свойствами в~базы данных. Технологический 
процесс производства композитов имитируется в~сис\-те\-мах в~отдельных%\linebreak 
модулях.
{ %\looseness=1

} 
     
     В связи с~такой организацией систем (когда базы знаний входят в~сис\-те\-му 
CAE) не имеет смысла изобретать какое-то отдельное решение с~отдельно 
стоящей базой знаний. Необходимо включить ее в~программный комплекс, как 
это сейчас делается у~ведущих производителей.
     
     Однако следует отметить, что, несмотря на такие большие возможности 
существующих про\-грам\-мных сис\-тем, моделей, позволяющих с~высокой степенью 
достоверности прогнозировать инициацию трещины, ее рост и~остаточную 
прочность, пока нет. Поэтому для нивелирования данного пробела используются 
конструктивные решения, предотвращающие возникновение недопустимо 
высоких нормальных напряжений. При таком подходе на первое место при 
проектировании выходят не свойства материала, а~естественные кон\-цент\-ра\-торы.
     
     Задачи, связанные с~изучением структурных особенностей современных 
композиционных материалов с~заданными свойствами, их изменениями 
в~процессе эксплуатации, относятся к~слож\-ным многомасштабным проб\-ле\-мам. 
Вследствие своей\linebreak огромной ресурсоемкости они требуют постоянного 
увеличения вычислительных возможностей аппаратных средств и~обновления 
высокоэффективных программных решений с~применением \mbox{распределенных} 
вычислений.

%\vspace*{-6pt}
     
\section{Многомасштабная модель для~расчета структурных свойств 
композиционных материалов}

%\vspace*{-2pt}

      При изучении многомасштабных научных проб\-лем, включающих в~себя 
явления несопоставимых пространственных и/или временн$\acute{\mbox{ы}}$х масштабов 
необходимо учитывать все факторы, иг\-ра\-ющие в~них ключевые роли. Для 
решения таких проблем необходимо в~рамках одной вычислительной задачи 
рассчитать несколько различных сопряженных между собой физических 
процессов с~разных масштабных уровней.
     
     Для расчета структурных характеристик композиционных материалов была 
построена оригинальная многомасштабная математическая модель,\linebreak которая 
может быть применена не только для изуче\-ния свойств композиционных 
материалов с~заданной структурой и~составом, но и~для анализа динамических 
процессов, в~том чис\-ле для \mbox{численного} исследования усталостных разрушений 
при случайных ударных по\-вреж\-де\-ни\-ях. Для ее описания используется  
тео\-ре\-ти\-ко-мно\-жест\-вен\-ный аппарат, пред\-став\-лен\-ный в~[1, 2], 
определяется необходимое в~соответствии с~поставленной задачей число 
масштабных уровней. Далее фи\-зи\-ко-ма\-те\-ма\-ти\-че\-ским моделям, 
распределенным по таким уровням, ставятся в~соответствие базовые  
мо\-де\-ли-ком\-по\-зи\-ции (БК)~--- цифровые аналоги, пе\-ре\-да\-ющие сущ\-ность 
вычислительных процессов, которые реализуются. Структура БК передается 
с~помощью таблицы данных~[1, 2]. 

      \begin{figure*}[b] %fig1
      \vspace*{12pt}
  \begin{center}  
    \mbox{%
\epsfxsize=126.825mm
\epsfbox{abg-1.eps}
}

\end{center}
\vspace*{-6pt}
      \Caption{Схема многомасштабной модели для расчета структурных характеристик  
новых композиционных материалов}
      \end{figure*}
      
     
     Такое пред\-став\-ле\-ние пол\-ностью описывает структуру БК 
     и~задает шаб\-лон, который заполняется конкретными данными при 
создании реальных экземпляров БК для решения практических задач 
математического моделирования композиционных материалов.
{\looseness=-1

}     

     Согласно терминологии из~[1, 2], под многомасштабной композицией (MC) 
понимается однопараметрическое семейство, полученное из эк\-зем\-п\-ля\-ров БК 
с~разных масштабных уровней за счет объединения в~общем вы\-чис\-ли\-тель\-ном 
процессе их основных множеств разного структурного типа, включая данные 
(входные и~выходные) и~методы их обработки. Например: 
\begin{itemize}
\item
$\mathbf{MC}_{12}^A: 
\{ V_{12}^A, X_{12}^A, \mathrm{MA}_{12}^A\}$~--- обозначение БК для проведения 
кван\-то\-во-ме\-ха\-ни\-че\-ских расчетов;  
\item
$\mathbf{MC}_{22}^A: \{ V_{22}^A, 
X_{22}^A, \mathrm{MA}_{22}^A\}$~--- обо\-зна\-че\-ние БК для проведения  
мо\-ле\-ку\-ляр\-но-ди\-на\-ми\-че\-ских расчетов.
\end{itemize}
     
     Такой подход позволяет оперативно создавать и~модифицировать с~учетом 
дополнительных требований набор многомасштабных композиций для решения 
связанных задач, таких как
     \begin{itemize}
\item многомасштабное моделирование процессов зарождения, формирования 
и~трансформации во времени различных видов и~концентраций дефектов 
в~композиционных материалах, а~также их макроразрушение;
\item численное исследование усталостных разрушений в~процессе случайных 
ударных повреждений;
\item построение математических моделей <<со\-став--свой\-ст\-во>> для 
прогнозирования характеристик, влияющих на эксплуатационные свойства 
материала.
\end{itemize}

     Разработанные многомасштабные композиции со\-став\-ля\-ют основу 
многомасштабной модели для расчета структурных характеристик новых 
композиционных материалов.
     
     
     \begin{figure*}[b] %fig2
% \vspace*{-6pt}
  \begin{center}  
    \mbox{%
\epsfxsize=162.103mm
\epsfbox{abg-2.eps}
}

\end{center}
\vspace*{-9pt}
\Caption{Архитектура распределенной информационной системы}
\end{figure*}

     На рис.~1 пред\-став\-ле\-на общая схема многомасштабной композиции для 
расчета структурных характеристик композиционных материалов.

%\pagebreak




     
     
%\vspace*{-6pt}

    \section{Распределенная информационная система}
    
   % \vspace*{-2pt}
    
    
   Для решения практических задач многомасштабного моделирования в~об\-ласти 
композиционных материалов актуальна разработка программных\linebreak комплексов, 
развернутых на {высокопроизводительных} кластерах, которые предоставляют 
исследователям возможность проводить многоуровневые расчеты с~применением 
распределенных\linebreak ресурсов, используя при этом как собственные программные 
разработки, так и~пакеты прикладных программ, например для проведения 
кван\-то\-во-ме\-ха\-ни\-че\-ских расчетов~[3], такие как VASP ({\sf https:// www.vasp.at}) 
и~Quantum Espresso ({\sf https://www. quantum-espresso.org}). 
   
   В данной работе пред\-став\-ле\-на распределенная информационная система 
   и~средства ее разработки, базирующиеся на многомасштабном подходе. 
   
   Необходимо отметить, что при построении программного комплекса для 
решения конкретных задач используются как программные модули собственной 
разработки, в~которых реализованы\linebreak оригинальные численные решения созданных 
фи\-зи\-ко-ма\-те\-ма\-ти\-че\-ских моделей, так и~готовые\linebreak про\-грам\-мные модули. В~его 
основе лежит интеграционная платформа для многомасштабного моделирования, 
которая объединяет информационные потоки на разных масштабных уровнях. 
При решении конкретной задачи, такой как расчет структурных особенностей 
композиционного материала или изучение процессов его деградации 
и~\mbox{разрушения}, выделяются конкретные уровни моделирования, которые 
необходимо задействовать,\linebreak и~строится многомасштабная композиция~--- 
информационный аналог многомасштабной фи\-зи\-ко-ма\-те\-ма\-ти\-че\-ской 
модели. Для про\-грам\-мной реализации на базе интеграционной платформы из\linebreak 
имеющихся про\-грам\-мных модулей формируется вычислительный комплекс.


   
   Перечислим пользовательские роли в~интеграционной платформе 
многомасштабного моделирования:
   \begin{itemize}
\item разработчик вычислительных модулей реализует расчетный модуль 
или осуществляет конфигурирование прикладного пакета;
\item системный разработчик создает веб-сер\-ви\-сы для вычислительного 
модуля и~интегрирует его в~платформу;
\item разработчик расчетных сценариев создает сценарии в~среде 
моделирования;
\item ученый-исследователь прикладной области запускает расчетные 
сценарии с~различными параметрами и~анализирует результаты.
\end{itemize}

      Рассмотрим общую архитектуру распределенной информационной 
системы, пред\-став\-лен\-ной на рис.~2.



Информационная система состоит из типовых сервисов вычислительных модулей, 
общей интеграционной оболочки и~модулей сценариев. Типовой сервис 
вычислительного модуля включает сам вычислительный модуль, развернутый на 
высокопроизводительном кластере или сервере, и~соответствующий сервис 
с~программным интерфейсом и~(опционально) веб-интерфейсом пользователя.

Вычислительный модуль пред\-став\-ля\-ет собой исполняемый файл (консольное 
приложение) на Linux или Windows, работающий в~пакетном режиме 
и~выполняющий расчеты по заданным па\-ра\-мет\-рам. Примеры: пакет VASP, 
LAMMPS, Quantum Espresso, собственные реализации на С++, Fortran. Сервис 
вычислительного модуля пред\-став\-ля\-ет собой приложение, обеспечивающее 
универсальный программный интерфейс в~виде HTTP REST сервисов для ввода 
параметров, запуска и~захвата результатов вычислений.

       Если вычислительному модулю требуются справочные данные, адаптер 
предоставляет их из базы данных либо из своих фай\-лов-ре\-сур\-сов. 
Метаданные модуля содержат информацию о~названиях, типах входных 
параметров и~результатов для дальнейшего использования в~сценариях и~передачи 
данных между модулями.
       
       В дальнейшем планируется наращивание встраиваемых 
в~интеграционную платформу вычислительных модулей, что позволит 
существенно расширить класс решаемых задач и~воз\-мож\-ности\linebreak реализации 
программных решений в~об\-ласти вы\-чис\-ли\-тель\-но\-го материаловедения~[4--6].
       
    Для реализации возможностей ролей разработчика расчетных сценариев 
и~уче\-но\-го-ис\-сле\-до\-ва\-те\-ля были составлены требования к~движку 
сценариев, на основе которых были сформулированы критерии выбора основы 
для его реализации.
\begin{enumerate}[1.]
\item \textit{Интерфейс для пользователей и~его функциональность}:
\begin{itemize}
\item наличие средства моделирования процессов;
\item возможность визуализации исполнения процессов;
\item просмотр результата расчета для выполненных и~вы\-пол\-ня\-ющих\-ся 
сценариев;
\item возможность указать па\-ра\-мет\-ры запуска сценария;
\item возможность перезапуска сценария с~теми же параметрами.
\end{itemize}
\item \textit{Поддержка платформы}:
\begin{itemize}
\item возможность отладки и~диагностики выполнения сценариев 
программистом;
\item легкость решения и~его инфраструктуры для разработчика; 
\item легкость решения и~его инфраструктуры для администратора 
в~промышленной эксплуатации;
\item лицензия, наличие открытого исходного \mbox{кода};
\item доступность специалистов, знающих язык, продукт, стандарт, 
технологию;
\item скорость обучения специалиста.
\end{itemize}

\item \textit{Простота для конечных пользователей}:
\begin{itemize}
\item простота развертывания нового сценария;
\item простота отладки и~тестирования сценариев для  
уче\-ных-ис\-сле\-до\-ва\-те\-лей.
\end{itemize}

\item \textit{Расширение платформы}:
\begin{itemize}
\item простота синхронной и~асинхронной интеграции с~расчетными 
модулями (сервисами); 
\item удобство ручного и~автоматического тестирования интеграции 
с~расчетными сер\-ви\-сами;
\item наличие развитого программного интерфейса.
\end{itemize}
\end{enumerate}

     По данным критериям были оценены су\-ще\-ст\-ву\-ющие программные 
продукты сле\-ду\-ющих категорий.
     \begin{enumerate}[1.]
     \item \textbf{Автоматизация биз\-нес-про\-цес\-сов.} Категория\linebreak продуктов, 
реализующих стандарт BPMN (Business Process Model and Notation) ({\sf https://\linebreak www.bpmn.org}), широко 
ис\-поль\-зу\-ющий\-ся для описания и~моделирования процессов, пред\-став\-ля\-ющих 
граф выполнения ручных (задач для пользователей) и~автоматических действий. 
В~автоматизации биз\-нес-про\-цес\-сов ключевым\linebreak аспектом вы\-сту\-па\-ет 
по\-сле\-до\-ва\-тель\-ность выполнения задач, со\-став\-ля\-ющих процесс, а~передача 
данных меж\-ду задачами не регламентируется. Среди лидеров рас\-смат\-ри\-ва\-лись 
\mbox{сле\-ду\-ющие} продукты:
     \begin{itemize}
     \item[(а)] Activiti BPM ({\sf https://www.activiti.org});
     \item[(б)] 
Camunda BPM ({\sf https://camunda.com/\linebreak products/camunda-platform});
     \item[(в)] 
Flowable BPM ({\sf https://www.flowable.com}) и~CMMN;
     \item[(г)] 
jBPM ({\sf https://www.jbpm.org}).
\end{itemize}
        
        Из этой группы по предъявляемым критериям отбора лучше всего 
подходит Camunda BPM, в~том чис\-ле за счет наличия простого, но достаточного 
по функциональности приложения для разработки сценариев без написания кода.
\item \textbf{Автоматизация работы с~потоками данных.} Для этой категории 
продуктов ключевым элементом пред\-став\-ля\-ют\-ся по\-сту\-па\-ющие на вход данные, 
для которых определяется асинхронный на\-прав\-лен\-ный граф (DAG, Directed Acyclic Graph) операций для 
их трансформации. Продукты этой категории используются для сбора, 
трансформации и~консолидации данных в~хранилищах или облаках данных 
и~в~основном ориентированы на аналитиков данных или разработчиков баз 
данных. Рассматривались сле\-ду\-ющие реализации:
     \begin{itemize}
\item[(а)] Apache Airflow~--- одно из самых популярных решений, 
особенность которого заключается в~описании DAG на Python, но 
практически отсутствует адекватная визуализация как на этапе создания, так и~на этапе
выполнения сценария;
\item[(б)] Apache Beam;
\item[(в)] Nextflow.io;
\item[(г)] Temporal.io;
\item[(д)] Prefect.
\end{itemize}
        
        Хотя по критериям отбора Apache Airflow набрал примерно такое же 
число баллов, что и~Camunda BPM, предпочтение было отдано последнему за счет 
лучшего средства визуального проектирования. С~одной стороны, движки Data 
Flow рассчитаны на работу в~пакетном режиме и~содержат встроенные средства 
распараллеливания, автоматического перезапуска и~другие полезные функции для 
сценариев научных расчетов. С~другой стороны, они могут привнести 
ограничения при управ\-ле\-нии данными расчетов, не дав возможности реализовать 
специфику многомасштабного моделирования.
\item \textbf{Автоматизация научных расчетов.} Наиболее близкие по тематике 
продукты, к~сожалению, имеют достаточно ограниченную функ\-ци\-о\-наль\-ность, 
недостаточную гиб\-кость и~неподходящие лицензии:
     \begin{itemize}
\item[(а)] Knime Analytics Platform~--- реализует достаточно 
большую часть необходимых функций, больше 
ориентирован на работу с~потоками данных для data science. К~\mbox{сожалению}, 
открытая и~бесплатная версия доступна только в~варианте настольного 
приложения, которое не масштабируется под задачи вы\-чис\-ли\-тель\-но\-го 
моделирования материалов;
\item[(б)] Everest ({\sf http://everest.distcomp.org}, разработка ИППИ РАН) 
предоставляет инфраструктуру для запуска вы\-чис\-ли\-тель\-ных задач 
в~гетерогенных вы\-чис\-ли\-тель\-ных \mbox{клас\-те\-рах}. Присутствуют элементы 
оркестрации задач~--- задание варьи\-ру\-емо\-го па\-ра\-мет\-ра, более слож\-ная 
ор\-кест\-ров\-ка задач в~виде скрип\-тов пишется программистом на Python API. 
К~сожалению, имеет закрытые исходные коды и~не очень активно 
развивается;
\item[(в)] Modelica, OpenModelica~--- язык и~реализация для описания 
математических моделей. Содержит множество интересных идей,\linebreak но все 
модули долж\-ны быть описаны на языке Modelica, в~то время как в~об\-ласти\linebreak 
вычислительного материаловедения используются пакеты VASP ({\sf 
https://www.vasp.\linebreak at}), QuantumEspresso  
({\sf https://www.quantum-espresso.org}) и~др., оформленные в~виде 
отдельных приложений на Fortran, C++.
\end{itemize}
\end{enumerate}

        В результате отбора по исходным критериям был выбран подход 
к~моделированию сценариев как биз\-нес-про\-цес\-сов и~его реализация в~виде 
Camunda BPM.
        
   Рассмотрим предлагаемую архитектуру комплекса с~акцентом на модули, 
от\-ве\-ча\-ющие за исполнение сценариев (рис.~3).
   
\begin{figure*} %fig3
 \vspace*{1pt}
  \begin{center}  
    \mbox{%
\epsfxsize=161.899mm
\epsfbox{abg-3.eps}
}

\end{center}
\vspace*{-9pt}
\Caption{Компоненты архитектуры модуля сценариев}
\end{figure*}

   В составе подсистемы сценариев реализуются следующие модули.
   \begin{enumerate}[1.]
\item \textit{Сервис выполнения сценариев}~--- ключевой компонент, 
отвечающий за логику работы сценариев (используется движок Camunda 
BPM), а~также за хранение и~обмен данными.
\item \textit{База данных сценариев}~--- хранилище данных на сис\-те\-ме управ\-ле\-ния базами данных 
MongoDB, в~котором находятся оперативные данные исполняющихся и~уже 
завершенных экземпляров сценариев.
\item \textit{Camunda BPM Engine}~--- готовый продукт, отвечающий за 
хранение BPM-диа\-грамм сценариев, их запуск и~отслеживание статуса 
выполнения.
\item \textit{Camunda Modeler}~--- настольное приложение для создания новых 
сценариев (дизайнер BPMN-про\-цес\-сов). Готовый продукт, который будет 
расширяться по необходимости.
\item \textit{База данных Camunda BPMN}~--- хранит диаграммы, запущенные 
сценарии и~их статусы.
\item \textit{Веб-интерфейс выполнения сценариев}~--- интерфейс для 
пользователей, позволяющий про\-смат\-ри\-вать, запускать, отслеживать 
выполнение и~про\-смат\-ри\-вать/ска\-чи\-вать результаты \mbox{работы} сценариев.
   \end{enumerate}
   
   Сервис сценариев работает с~сервисами вы\-чис\-ли\-тель\-ных модулей по 
стандартизованному протоколу на основе <<словаря данных>>, а~в~BPMN 
вызовы модулей пред\-став\-ля\-ют\-ся в~виде <<внешних задач>> (external tasks), 
контроль за исполнением которых лежит на сервисе сценариев.
   
   Рассмотрим пред\-став\-ле\-ние примера расчетного сценария в~нотации BPMN 
(рис.~4).

    \begin{figure*} %fig4
     \vspace*{1pt}
  \begin{center}  
    \mbox{%
\epsfxsize=147.834mm
\epsfbox{abg-4.eps}
}

\end{center}
\vspace*{-10pt}
    \Caption{Пример расчетного сценария}
    \vspace*{-7pt}
    \end{figure*}
    
     В начале процесса находится элемент, в~свойствах которого задается список 
ожидаемых входных параметров от пользователя. Так как весь процесс работает 
в~пакетном режиме, то все необходимые для работы всех модулей входные 
параметры (которые невозможно вы\-чис\-лить в~самом сценарии) долж\-ны либо 
задаваться пользователем перед началом выполнения, либо быть 
зафиксированными автором сценария в~самой диаграмме.
     
   Далее обозначены следующие типы элементов:
   \begin{itemize}
\item \textit{Запуск вычислительного модуля} (<<VASP>>, <<Фиттинг>>, 
<<Молекулярная динамика>>), который пред\-став\-ля\-ет\-ся элементом типа 
<<сервисная задача>> (service task), что означает ее автоматическое 
выполнение без взаимодействия с~человеком. В~данном случае все задачи 
имеют подтип <<внеш\-няя сервисная задача>>, где в~качестве <<темы>> (topic) 
указывается код вычислительного модуля (например, <<vasp>>, <<qe>>, 
<<fitting>>). В~за\-ви\-си\-мости от этого значения сервисом сценария будет 
запускаться со\-от\-вет\-ст\-ву\-ющий вы\-чис\-ли\-тель\-ный модуль;
\item \textit{Пользовательский скрипт} (<<вычисление типов  
потенциала>>)~--- фрагмент кода на одном из\linebreak поддерживаемых скриптовых 
языков программирования (JavaScript, Groovy, Python), позволяющий авторам 
сценария добавлять произвольную логику обработки данных или \mbox{вы\-чис\-ле\-ния} 
па\-ра\-мет\-ров, специфичную для конкретного сценария;
\item \textit{Подпроцесс-цикл} (блок <<Фиттинг>>~--- <<молекулярная 
динамика>>)~--- запуск подпроцесса в~итерации по указанному па\-ра\-мет\-ру. 
Возможно указание уровня параллелизма (<<loop cardinality>>)~--- 
максимальное число итераций, которое можно запустить параллельно;
\item \textit{Элемент завершения сценария}~--- формальный элемент, 
переводящий экземпляр сценария в~со\-сто\-яние <<завершен>>.
\end{itemize}
  
  Таким образом, у~пользователей есть воз\-мож\-ность создавать 
и~редактировать сценарии выполнения расчетов с~ветвлениями, слияниями, 
цик\-ла\-ми и~другими элементами, предусмот\-рен\-ны\-ми стандартом BPMN. Из 
ограничений следует отметить, что в~настоящий момент не предусмот\-ре\-но 
использование пользовательских задач внут\-ри процесса, т.\,е.\ остановка 
сценария для ожидания ввода данных пользователем.

\vspace*{-10pt}

\section{Выводы}

\vspace*{-2pt}

   Разработанные подходы составляют основу многомасштабной модели, 
сформированной для расчета структурных характеристик новых композиционных 
материалов. Для компьютерной реализации многомасштабной модели была 
построена распределенная информационная сис\-те\-ма, которая позволяет 
интегрировать разные программные модули в~общий вы\-чис\-ли\-тель\-ный процесс 
и~проводить при не\-об\-хо\-ди\-мости их запуск по заданным сценариям 
в~параллельном режиме на разных вычислительных средствах соответственно 
ре\-ша\-емой\linebreak
 задаче. Показано, что создание кроссплатформенной, расширяемой 
интеграционной сис\-те\-мы, предназначенной для решения задач многомасштабного 
моделирования на высокопроизводительных\linebreak
 вычислительных кластерах, 
позволяет оперативно создавать распределенные информационные сис\-те\-мы для 
решения конкретных задач в~области материаловедения композиционных 
ма\-те\-ри\-алов.


   
   Данный подход существенно расширяет возможности проведения 
исследований в~об\-ласти многомасштабного моделирования композиционных 
материалов и~позволяет во многих случаях существенно ускорить процесс 
разработки программных решений с~использованием высокопроизводительных 
программных средств для практических задач.

\vspace*{-6pt}
   
{\small\frenchspacing
 {%\baselineskip=10.8pt
 %\addcontentsline{toc}{section}{References}
 \begin{thebibliography}{9}

\bibitem{1-ab}
\Au{Абгарян К.\,К.} Многомасштабное моделирование в~задачах структурного 
материаловедения.~--- М.: МАКС Пресс, 2017. 284~с.
\bibitem{2-ab}
\Au{Абгарян К.\,К.} Информационная технология по\-стро\-ения многомасштабных моделей 
в~задачах вычислительного материаловедения~// Сис\-те\-мы высокой до\-ступ\-ности, 2018. Т.~14. 
№\,2. С.~9--15.
\bibitem{3-ab}
\Au{Kohn W., Sham L.\,J.} Self-consistent equations including exchange and correlation effects~// 
Phys. Rev.~A, 1965. Vol.~140. No.\,4. P.~1133--1138.
\bibitem{4-ab}
\Au{Абгарян К.\,К., Гаврилов~Е.\,С., Марасанов~А.\,М.} Информационная поддержка задач 
многомасштабного моделирования композиционных материалов~// Int. J.~Open Information 
Technologies, 2017. Vol.~5. Iss.~12. P.~24--29.
\bibitem{5-ab}
\Au{Абгарян К.\,К., Гаврилов~Е.\,С.} Информационная поддержка интеграционной 
платформы многомасштабного моделирования~// Системы и~средства информатики, 2019. 
Т.~29. №\,1. С.~53--62.
\bibitem{6-ab}
\Au{Абгарян К.\,К., Гаврилов~Е.\,С.} Интеграционная платформа для многомасштабного 
моделирования нейроморфных сис\-тем~// Информатика и~её применения, 2020. Т.~14. 
Вып.~2. С.~104--111. 
\end{thebibliography}

 }
 }

\end{multicols}

\vspace*{-6pt}

\hfill{\small\textit{Поступила в~редакцию 28.10.21}}

%\vspace*{8pt}

%\pagebreak

\newpage

\vspace*{-28pt}

%\hrule

%\vspace*{2pt}

%\hrule

%\vspace*{-2pt}

\def\tit{DISTRIBUTED INFORMATION SYSTEM FOR~CALCULATING THE~STRUCTURAL 
PROPERTIES OF~COMPOSITE MATERIALS}


\def\titkol{Distributed information system for calculating the structural 
properties of~composite materials}


\def\aut{K.\,K.~Abgaryan$^{1,2}$ and~E.\,S.~Gavrilov$^{1,2}$}

\def\autkol{K.\,K.~Abgaryan and~E.\,S.~Gavrilov}

\titel{\tit}{\aut}{\autkol}{\titkol}

\vspace*{-11pt}


\noindent
$^1$Federal Research Center ``Computer Science and Control'' of the Russian Academy of Sciences, 
44-2~Vavilov\linebreak
$\hphantom{^1}$Str., Moscow 119333, Russian Federation

\noindent
$^2$Moscow Aviation Institute (National Research University), 4~Volokolamskoe Shosse, Moscow 
125080, Russian\linebreak
$\hphantom{^1}$Federation

\def\leftfootline{\small{\textbf{\thepage}
\hfill INFORMATIKA I EE PRIMENENIYA~--- INFORMATICS AND
APPLICATIONS\ \ \ 2021\ \ \ volume~15\ \ \ issue\ 4}
}%
 \def\rightfootline{\small{INFORMATIKA I EE PRIMENENIYA~---
INFORMATICS AND APPLICATIONS\ \ \ 2021\ \ \ volume~15\ \ \ issue\ 4
\hfill \textbf{\thepage}}}

\vspace*{3pt} 

  

\Abste{The use of composite materials has found wide application in various branches of engineering 
due to their advantages over metals with equal mechanical and operational properties. To solve the
problems arising in the field of creating composite materials with a set of specified properties, new 
approaches to the development of mathematical models and information systems based on them are 
widely used today. The paper presents an original multiscale mathematical model that allows 
calculating the structural characteristics of composite materials and can be used to numerically study 
fatigue fracture of composite materials in case of accidental impact damage. On the basis of this 
multiscale model, a distributed information system was created for conducting large-scale research in 
the field of modeling composite materials with specified properties. The development of this approach 
in the future will help to ensure the formation of information for a reasonable choice of composite 
materials with desired properties for aerospace and other industries.}

\KWE{multiscale modeling; composite materials; integration platform; software package; distributed 
system}


\DOI{10.14357/19922264210407}

\vspace*{-16pt}

\Ack

\vspace*{-3pt}

\noindent
The research was supported by the Ministry of Science and Higher Education of the Russian 
Federation (project No.\,075-15-2020-799).



%\vspace*{6pt}

  \begin{multicols}{2}

\renewcommand{\bibname}{\protect\rmfamily References}
%\renewcommand{\bibname}{\large\protect\rm References}

{\small\frenchspacing
 {%\baselineskip=10.8pt
 \addcontentsline{toc}{section}{References}
 \begin{thebibliography}{9}
\bibitem{1-ab-1}
   \Aue{Abgaryan, K.\,K.} 2017. \textit{Mnogomasshtabnoe modelirovanie v~zadachakh strukturnogo 
materialovedeniya} [Multiscale modeling for structural materials science applications]. Moscow: MAKS 
Press. 284~p.
\bibitem{2-ab-1}
   \Aue{Abgaryan, K.\,K.} 2018. Informatsionnaya tekhnologiya postroeniya mnogomasshtabnykh 
modeley v~zadachakh vychislitel'nogo materialovedeniya [Information technology is the construction of 
multi-scale models in problems of computational materials science]. \textit{Sistemy vysokoy dostupnosti} 
[Highly Available Systems] 14(2):9--15.
\bibitem{3-ab-1}
   \Aue{Kohn, W., and L.\,J.~Sham.} 1965. Self-consistent equations including exchange and correlation 
effects. \textit{Phys. Rev.~A} 140(4):1133--1138.
\bibitem{4-ab-1}
   \Aue{Abgaryan, K.\,K., E.\,S.~Gavrilov, and A.\,M.~Marasanov.} 2017. Informatsionnaya podderzhka 
zadach mnogomasshtabnogo modelirovaniya kompozitsionnykh materialov [Multiscale modeling for 
composite materials computer simulation support]. \textit{Int. J.~Open Information Technologies} 
12:24--29.
\bibitem{5-ab-1}
   \Aue{Abgaryan, K.\,K., and E.\,S.~Gavrilov}. 2019. In\-for\-ma\-tsi\-on\-naya podderzhka integratsionnoy 
platformy mno\-go\-massh\-tab\-no\-go modelirovaniya [Informational support of the multiscale modeling 
integration platform]. \textit{Sistemy i~Sredstva Informatiki~--- Systems and Means of Informatics} 
29(1):53--62.
\bibitem{6-ab-1}
   \Aue{Abgaryan, K.\,K., and E.\,S.~Gavrilov.}  2020. Integratsionnaya platforma dlya 
mnogomasshtabnogo modelirovaniya neyromorfnykh sistem [Integration platform for multiscale 
modeling of neuromorphic systems]. \textit{Informatika i~ee Primeneniya~--- Inform. Appl.} 
14(2):104--111.
\end{thebibliography}

 }
 }

\end{multicols}

\vspace*{-6pt}

\hfill{\small\textit{Received October 28, 2021}}

%\pagebreak

\vspace*{-18pt}

  \Contr
  
  \vspace*{-3pt}
  
  \noindent
  \textbf{Abgaryan Karine K.} (b.\ 1963)~--- Doctor of Science in physics and mathematics, principal 
scientist, A.\,A.~Dorodnicyn Computing Center, Federal Research Center ``Computer Science and 
Control'' of the Russian Academy of Sciences, 40~Vavilov Str., Moscow 119333, Russian Federation; 
Head of Department, Moscow Aviation Institute (National Research University), 4~Volokolamskoe 
Shosse, Moscow 125080, Russian Federation; \mbox{kristal83@mail.ru}
   
   \vspace*{3pt}
   
   \noindent
   \textbf{Gavrilov Evgeny S.} (b.\ 1982)~--- scientist, 
A.\,A.~Dorodnicyn Computing Center, Federal Research Center ``Computer Science and Control'' of the 
Russian Academy of Sciences, 40~Vavilov Str., Moscow 119333, Russian Federation; senior lecturer, 
Moscow Aviation Institute (National Research University), 4~Volokolamskoe Shosse, Moscow 125080, 
Russian Federation; \mbox{eugavrilov@gmail.com}
  
 


\label{end\stat}

\renewcommand{\bibname}{\protect\rm Литература} 