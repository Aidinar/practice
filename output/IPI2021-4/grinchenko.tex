\def\stat{grinchenko}

\def\tit{АНТРОПОГЕННАЯ <<ТРЕТЬЯ>> ПРИРОДА: ОТНОСИТЕЛЬНО АВТОНОМНЫЙ 
СТАТУС ЕЕ ИСКУССТВЕННЫХ ИНТЕЛЛЕКТУАЛЬНЫХ СУБЪЕКТОВ}

\def\titkol{Антропогенная <<третья>> природа: относительно автономный 
статус ее искусственных интеллектуальных субъектов}

\def\aut{С.\,Н.~Гринченко$^1$}

\def\autkol{С.\,Н.~Гринченко}

\titel{\tit}{\aut}{\autkol}{\titkol}

\index{Гринченко С.\,Н.}
\index{Grinchenko S.\,N.}


%{\renewcommand{\thefootnote}{\fnsymbol{footnote}} \footnotetext[1]
%{Работа выполнена при поддержке Министерства науки и~высшего образования Российской Федерации (проект 
%075-15-2020-799).}}


\renewcommand{\thefootnote}{\arabic{footnote}}
\footnotetext[1]{Федеральный исследовательский центр <<Информатика и~управление>> Российской 
академии наук, sgrinchenko@ipiran.ru}

%\vspace*{-12pt}




  \Abst{С позиций информатико-кибернетического моделирования процесса развития 
самоуправляющейся ие\-рар\-хо\-се\-те\-вой сис\-те\-мы Человечества рассмотрена 
антропогенная <<вторая>> природа, в~которой с~середины XX~в.\ начала формироваться 
важная часть с~элементами час\-ти <<первой>> (неживой) природы, получившими 
возможность относительно самостоятельно принимать решения (иерархический 
искусственный интеллект (ИИИ)) и~активно действовать (интеллектуальные роботы), т.\,е.\ 
функционировать в~некоторой степени независимо от создавшего их человека. Для этой 
части второй природы предлагается наименование <<третья>> природа. Ее формирование 
стало возможным с~возникновением базисных информационных технологий (БИТ)~--- локальных 
компьютеров с~$\sim1946$~г.\ и~телекоммуникаций/сетей с~$\sim1979$~г. Приводятся 
типичные пространственные и~временные параметры ие\-рар\-хо\-се\-те\-вой сис\-те\-мы 
Человечества (результаты модельного расчета). Отмечается, что в~активной деятельности 
элементов третьей природы~--- существенно автономных со\-став\-ля\-ющих ИИИ
и~интеллектуальных роботов~--- заключается экзистенциальная 
(гуманитарная!) опас\-ность для Человечества.}
  
  \KW{вторая природа; третья природа; информатико-кибернетическая модель;  
са\-мо\-управ\-ля\-юща\-яся ие\-рар\-хо\-се\-те\-вая сис\-те\-ма Человечества; 
информационные технологии; иерархический искусственный интеллект; интеллектуальные 
роботы}

\DOI{10.14357/19922264210415}
  
%\vspace*{9pt}


\vskip 10pt plus 9pt minus 6pt

\thispagestyle{headings}

\begin{multicols}{2}

\label{st\stat}
  
\section{Введение. Уточнение терминологии}

  Общепринято под \textit{природой} подразумевать совокупность всего 
сущего, мир в~целом. При этом в~природе выделяют две основные 
составляющие: объективно существующий и~независимый от человека мир и~ту 
часть окружающего мира, которая создается человеческой деятельностью,~--- 
техникой, искусством, культурой и~т.\,п.~[1]. Вторую часть так порой 
и~называют~--- <<\textit{второй}>> природой (см., например,~[2]). Исходя из 
этого, и~первую часть целесообразно называть <<первой>> природой. 
  
  В свою очередь, в~последние десятилетия с~возникновением компьютеров 
и~телекоммуникаций в~недрах антропогенной второй природы начала 
формироваться важная ее часть, элементы которой получили возможность 
относительно самостоятельно принимать решения (искусственный интеллект) 
и~активно действовать (интеллектуальные роботы)~--- в~той или иной степени 
независимо от создавшего их человека. При этом аппаратная часть этих 
элементов явно относится к~неживой~--- первой~--- природе. Следовательно, 
для этой суперпозиции первой и~второй природы~--- и~при этом продукта 
жизнедеятельности человека~--- напрашивается наименование 
<<\textit{третья}>> природа.
  
\section{Взаимодействие человека, общества и~природы~--- первой,~второй~и~третьей}

  Таким образом, вторая природа разделяется на две составляющие: 
собственно <<вторую>>, или \textit{ант\-ро\-по\-ген\-ную пассивную}, природу 
и~действующую относительно автономно от человека <<третью>>, 
\textit{ант\-ро\-по\-ген\-ную активную}, природу. Процессы этого\linebreak активного 
функционирования третьей природы (происходящие на наших глазах) 
совершаются либо по задаваемым человеком априори регулярным 
(детерминированным) компьютерным алгоритмам, либо по 
слу\-чай\-но-ре\-гу\-ляр\-ным алгоритмам, которые искусственный интеллект может изменять 
и~развивать автоматически.

  
  Формирование на фоне первой природы второй и~далее~--- третьей природы 
описывается в~терминах этапов гло\-баль\-но-кос\-ми\-че\-ской эволюции 
самоуправляющейся ие\-рар\-хо\-се\-те\-вой сис\-те\-мы Человечества, т.\,е.\ 
\textit{коэволюции} таких его со\-став\-ля\-ющих,\linebreak\vspace*{-12pt}

\pagebreak

\end{multicols}

\begin{figure*} %fig1
\vspace*{1pt}
\begin{center}  
\mbox{%
\epsfxsize=163mm
\epsfbox{gri-1-a.eps}
}
\end{center}
\vspace*{-9pt}
\Caption{Подсистема формирующейся самоуправляющейся ие\-рар\-хо\-се\-те\-вой сис\-те\-мы 
Человечества (период с~28,2~млн до~8,1~тыс.\ лет назад)
}
\vspace*{-6pt}
\end{figure*}


\begin{multicols}{2}

\noindent
 как антропосфера, психосфера, 
когнитосфера, социосфера, техносфера, ноосфера и~т.\,п., в~коэволюции 
с~иерархической системой биогеосферы. Согласно  
ин\-фор\-ма\-ти\-ко-ки\-бер\-не\-ти\-че\-ской модели (ИКМ) сис\-те\-мы 
Человечества, критическими точками этих процессов являются системные 
перевороты в~истории Человечества~[3--8]. Их маркерами выступают 
соответствующие информационные перевороты, т.\,е.\ старты новых 
\textit{базисных информационных технологий}, с~модельными 
расчетными датировками $\sim28\,230$--1860--123--8,1~тыс.\ лет  
назад--1446--1946--1979--1981--1981~гг.--$\ldots$\footnote{Временн$\acute{\mbox{ы}}$е 
и~пространственные количественные параметры ИКМ базируются на геометрической 
прогрессии со знаменателем $e^e\hm= 15{,}15426\ldots$, выявленной А.\,В.~Жирмунским 
и~В.\,И.~Кузьминым при исследовании критических уровней в~развитии  
био\-сис\-тем~\cite{9-gri}.}
  
  При этом каждый системный переворот и~маркирующая его БИТ 
инициируют появление двойки совокупностей иерархических  
уров\-ней/яру\-сов в~составе: (1)~восходящей, выше яруса личности, иерархии 
\textit{со\-ци\-аль\-но-про\-из\-вод\-ст\-вен\-но-инфра\-струк\-тур\-ных} 
технологий (формирования со\-об\-ществ/со\-ци\-умов людей и~инфраструктурных 
образований второй\linebreak и~третьей природы на все больших тер\-ри\-то\-ри\-ях) 
и~(2)~нисходящей, ниже яруса личности,\linebreak иерархии  
\textit{про\-из\-вод\-ст\-вен\-но-мик\-ро\-струк\-тур\-ных} технологий 
(создания все более прецизионных \mbox{инструментариев} и~формирования с~их 
помощью все более <<тонких>> объектов второй и~третьей природы) 
(рис.~1 и~2).

\begin{figure*} %fig2
\vspace*{1pt}
\begin{center}  
\mbox{%
\epsfxsize=162.938mm
\epsfbox{gri-1-b.eps}
}
\end{center}
%\vspace*{-9pt}
\end{figure*}

На рис.~1 и~2  \textit{восходящие стрелки}, име\-ющие структуру  
<<мно\-гие\,--\,к~од\-но\-му>>, отражают поисковую ак\-тив\-ность представителей 
соответствующих ярусов в~иерархии; \textit{нисходящие сплош\-ные стрелки}, име\-ющие 
структуру <<один\,--\,ко мно\-гим>>, отражают целевые критерии поисковой оптимизации 
системной энергетики~--- экстремальные, при ограничениях типа равенств и~неравенств; 
\textit{нисходящие пунктирные стрелки}, име\-ющие структуру <<один\,--\,ко\linebreak мно\-гим>>, 
отражают сис\-тем\-ную память лич\-ност\-но-про\-из\-вод\-ствен\-но-со\-ци\-аль\-но\-го~--- 
результат адап\-тив\-ных влияний представителей вышележащих иерархических ярусов на 
структуру и~поведение вложенных в~них нижележащих; для каждой из них приведены 
характерные времена изменения (средние периоды колебания или релаксации);
 ИСТ~--- инфраструктурные технологии.
  
  Согласно ИКМ, начиная с~середины XX~в.\ (расчетная модельная дата~--- 
1946~г.)\ в~составе антропогенной второй природы и~начала выделяться 
антропогенная третья природа, т.\,е.\ в~сис\-те\-ме\linebreak \mbox{Человечества} начали 
создаваться относительно\linebreak автономные (в~соответствующем смысле) от 
человека ал\-го\-рит\-мы/ме\-ха\-низ\-мы (<<субъекты>>) ИИИ~--- искусственного интеллекта, 
ориентированного на решение задач со\-от\-вет\-ст\-ву\-юще\-го уров\-ня/яру\-са 
в~сис\-тем\-ной иерархии~\cite{5-gri, 6-gri}~--- и~интеллектуальные роботы. 

  
  Активность, присущая этим представителям третьей природы, отражена на 
схеме рис.~2 посредством перехода структуры ярусов, лежащих 
в~иерархии ниже яруса индивида, от указания на \mbox{пассивный}~---  
в~управ\-лен\-че\-ском смысле~--- субстрат  
про\-из\-вод\-ст\-вен\-но-мик\-ро\-струк\-тур\-ных технологий к~активным 
составляющим полноценных контуров иерархической  
оп\-ти\-ми\-за\-ции/са\-мо\-управ\-ле\-ния в~системе Человечества 
соответственно.
  




  Предтечами системообразующих начал третьей природы были БИТ  
пись\-мен\-ности/чте\-ния и~БИТ тиражирования  
текс\-тов/кни\-го\-пе\-ча\-та\-ния, на базе которых рукописи и~книги уже стали 
независимыми от памяти отдельного человека носителями информации 
и~знаний, но еще не получили воз\-мож\-ность их автономно генерировать. 
Перспективная же на\-но-БИТ обещает новый кардинальный скачок 
в~поведенческой и~структурной слож\-ности <<циф\-ро\-вой эпохи>> развития 
третьей природы.
  
\section{Заключение}

  Позиционирование БИТ локальных компьютеров и~БИТ 
телекоммуникаций/сетей в~качестве\linebreak базисных факторов формирования третьей 
природы~--- что, собственно, и~определяется как <<циф\-ро\-вая эпоха>> 
в~сис\-тем\-ном развитии Человечества~--- демонстрирует их важнейшую роль 
в~широком\linebreak \mbox{контексте} единой исторической ретроспективы и~перспективы~--- 
со всеми возможными положительными и~отрицательными проявлениями 
результатов их вклада в~такое развитие.
  
  В частности, в~некоторой перспективе третья\linebreak природа может утратить свою 
изначальную и~непременную <<человеческую>> компоненту~--- прак\-тически 
полностью автоматические заводы по\linebreak производству <<циф\-ро\-вой>> техники 
уже существуют!~--- и~как пойдет ее дальнейшая эволюция, неясно (эта  
проб\-ле\-ма многократно обсуждалась в~мировой научной фантастике: <<бунт 
машин>>, <<Голем~XIV>>, <<массачусетская машина>> и~т.\,д. и~т.\,п.).
  
  Таким образом, пользуясь соответствующими достижениями третьей 
природы, не следует забывать, что в~активности автономной \mbox{деятельности} ее 
элементов~--- тем более <<нечеловеческих>>~--- заключается 
экзистенциальная (гуманитарная!) опасность для Человечества.
  
{\small\frenchspacing
 {%\baselineskip=10.8pt
 %\addcontentsline{toc}{section}{References}
 \begin{thebibliography}{9}
\bibitem{1-gri}
\Au{Месяц С.\,В.} Природа~// Большая российская энциклопедия.~--- М.: Большая 
российская энциклопедия, 2015. Т.~27. С.~519--520.
\bibitem{2-gri}
\Au{Королев В.\,К.} Вторая природа~// Культурология: Краткий тематический словарь.~--- 
Ростов н/Д: Феникс, 2001. 192~с.
\bibitem{3-gri}
\Au{Гринченко С.\,Н.} Системная память живого (как основа его метаэволюции 
и~периодической структуры).~--- М.: ИПИ РАН, Мир, 2004. 512~с.
\bibitem{4-gri}
\Au{Гринченко С.\,Н.} Метаэволюция (сис\-тем неживой, живой  
и~со\-ци\-аль\-но-тех\-но\-ло\-ги\-че\-ской природы).~--- М.: ИПИ РАН, 2007. 456~с.

\bibitem{7-gri} %5
\Au{Щапова Ю.\,Л., Гринченко~С.\,Н.} Введение в~теорию археологической эпохи: числовое 
моделирование и~логарифмические шкалы про\-стран\-ст\-вен\-но-вре\-мен\-ных 
координат.~--- М.: Ист. фак. Моск. ун-та, ФИЦ ИУ РАН, 2017. 236~с.
\bibitem{5-gri} %6
\Au{Гринченко С.\,Н.} О~пространственном структурировании феномена <<искусственный 
интеллект>>~// ИТНОУ: Информационные технологии в~науке, образовании  
и~управ\-ле\-нии, 2019. №\,4(14). С.~10--16.

\bibitem{8-gri} %7
\Au{Щапова Ю.\,Л., Гринченко~С.\,Н., Кокорина~Ю.\,Г.}  
Ин\-фор\-ма\-ти\-ко-ки\-бер\-не\-ти\-че\-ское и~математическое моделирование 
археологической эпохи: ло\-ги\-ко-по\-ня\-тий\-ный аппарат.~--- М.: ФИЦ ИУ РАН, 2019. 
136~с.
\bibitem{6-gri} %8
\Au{Гринченко С.\,Н.} О~сис\-тем\-ной иерархии искусственного интеллекта~// 
Информатика и~её применения, 2021. Вып.~1. С.~111--115.


\bibitem{9-gri}
\Au{Жирмунский А.\,В., Кузьмин~В.\,И.} Критические уровни в~процессах развития 
биологических систем.~--- М.: Наука, 1982. 179~с.
\end{thebibliography}

 }
 }

\end{multicols}

\vspace*{-3pt}

\hfill{\small\textit{Поступила в~редакцию 23.12.20}}

%\vspace*{8pt}

%\pagebreak

\newpage

\vspace*{-28pt}

%\hrule

%\vspace*{2pt}

%\hrule

%\vspace*{-2pt}

%\vspace*{6pt}

\def\tit{ANTHROPOGENIC ``THIRD'' NATURE:\\ THE~RELATIVELY AUTONOMOUS STATUS\\ 
OF~ITS~ARTIFICIAL INTELLECTUAL SUBJECTS}


\def\titkol{Anthropogenic ``third'' nature: The~relatively autonomous status 
of~its~artificial intellectual subjects}


\def\aut{S.\,N.~Grinchenko}

\def\autkol{S.\,N.~Grinchenko}

\titel{\tit}{\aut}{\autkol}{\titkol}

\vspace*{-11pt}


\noindent
Federal Research Center ``Computer Science and Control'' of the Russian Academy of Sciences, 
44-2~Vavilov Str., Moscow 119333, Russian Federation



\def\leftfootline{\small{\textbf{\thepage}
\hfill INFORMATIKA I EE PRIMENENIYA~--- INFORMATICS AND
APPLICATIONS\ \ \ 2021\ \ \ volume~15\ \ \ issue\ 4}
}%
 \def\rightfootline{\small{INFORMATIKA I EE PRIMENENIYA~---
INFORMATICS AND APPLICATIONS\ \ \ 2021\ \ \ volume~15\ \ \ issue\ 4
\hfill \textbf{\thepage}}}

\vspace*{3pt} 


\Abste{From the standpoint of informatics-cybernetic modeling of the development process of the 
self-controlling hierarchical-network system of Humankind, an anthropogenic ``second'' nature is 
considered in which an important part began to form from the middle of the 20th century with 
elements of the part of the ``first'' (inanimate) nature that were able relatively independently to 
make decisions (hierarchical artificial intelligence) and actively act (intelligent robots), i.\,e., to 
function somewhat independently of the person who created them. For this part of the second 
nature, the term ``third'' nature is proposed. Its formation became possible with the emergence of 
basic information technologies~--- local computers since $\sim1946$ and 
telecommunications/networks since $\sim1979$. Typical spatial and temporal parameters of the 
hierarchical-network system of Humankind (the results of a model calculation) are given. It is noted 
that the activity of the actions of the third nature elements~--- essentially autonomous components 
of hierarchical artificial intelligence and intelligent robots~--- is an existential (humanitarian!) 
danger for Humankind.}


\KWE{second nature; third nature; informatics-cybernetic model; self-controlling hierarchical-network 
system of Humankind; information technology; hierarchical artificial intelligence; 
intelligent robots}

\DOI{10.14357/19922264210415}

%\vspace*{-20pt}

%\Ack
%\noindent


%\vspace*{6pt}

  \begin{multicols}{2}

\renewcommand{\bibname}{\protect\rmfamily References}
%\renewcommand{\bibname}{\large\protect\rm References}

{\small\frenchspacing
 {%\baselineskip=10.8pt
 \addcontentsline{toc}{section}{References}
 \begin{thebibliography}{9}
\bibitem{1-gri-1}
\Aue{Mesyats, S.\,V.} 2015. Priroda [Nature]. \textit{Bol'shaya rossiyskaya entsiklopediya} [Great 
Russian Encyclopedia]. Moscow: BRE. 27:519--520.
\bibitem{2-gri-1}
\Aue{Korolev, V.\,K.} 2001. Vtoraya priroda [The second nature]. \textit{Kul'turologiya: Kratkiy 
tematicheskiy slovar'} [Cultural studies: A~short subject dictionary]. Rostov-on-Don: Feniks. 192~p.
\bibitem{3-gri-1}
\Aue{Grinchenko, S.\,N.} 2004. 
\textit{Sistemnaya pamyat' zhivogo (kak osnova ego metaevolyutsii i~periodicheskoy struktury)} 
[System memory of the life (as the basis of its metaevolution and periodic structure)]. Moscow: 
IPIRAN, Mir. 512~p.
\bibitem{4-gri-1}
\Aue{Grinchenko, S.\,N.} 2007. \textit{Metaevolyutsiya (sistem nezhivoy, zhivoy  
i~sotsial'no-tekhnologicheskoy prirody)} [Metaevolution (of inanimate, animate, and 
sociotechnological nature systems)]. Moscow: IPIRAN. 456~p.

\bibitem{7-gri-1} %5
\Aue{Shchapova, Y.\,L., and S.\,N.~Grinchenko.} 2017. \textit{Vvedenie v~teoriyu 
arkheologicheskoy epokhi: chislovoe modelirovanie i~logarifmicheskie shkaly prostranstvenno-vremennykh 
koordinat} [Introduction to the theory of the archaeological epoch: Numerical 
modeling and logarithmic scales of space--time coordinates]. Moscow: Faculty of History MSU, 
FRC CSCRAS. 236~p.
\bibitem{5-gri-1} %6
\Aue{Grinchenko, S.\,N.} 2019. O~prostranstvennom struk\-tu\-ri\-ro\-va\-nii fenomena ``iskusstvennyy 
intellekt'' [On the spatial structuring of the phenomenon of ``artificial in-telligence'']. 
\textit{ITNOU: Informatsionnye tekhnologii v~nauke, ob\-ra\-zo\-va\-nii i~upravlenii} [ITNOU: 
Information Technologies in Science, Education and Conrol] 14(4):10--16.


\bibitem{8-gri-1} %7
\Aue{Shchapova, Yu.\,L., S.\,N.~Grinchenko, and Yu.\,G.~Ko\-ko\-ri\-na.} 2019.
 \textit{Informatiko-kiberneticheskoe i~ma\-te\-ma\-ti\-che\-skoe\linebreak modelirovanie arkheologicheskoy epokhi: 
logiko-ponyatiynyy apparat} [Informatics-cybernetic and {mathematical} modeling of the 
archaeological epoch: Logical-conceptual apparatus]. Moscow: FRC CSC RAS. 136~p.

\bibitem{6-gri-1} %8
\Aue{Grinchenko, S.\,N.} 2021. O~sistemnoy iyerarkhii is\-kus\-st\-ven\-no\-go intellekta [On the system 
hierarchy of artificial intelligence]. \textit{Informatika i~ee Primeneniya~--- Inform. Appl.} 
 15(1):111--115.
\bibitem{9-gri-1}
\Aue{Zhirmunskiy, A.\,V., and V.\,I.~Kuz'min.} 1982. \textit{Kriticheskie urovni v~protsessakh 
razvitiya biologicheskikh sis\-tem} [Critical levels in the development of biological systems]. 
Moscow: Nauka. 179~p.

\end{thebibliography}

 }
 }

\end{multicols}

\vspace*{-3pt}

\hfill{\small\textit{Received December 23, 2020}}

%\pagebreak

%\vspace*{-24pt}


\Contrl

\noindent
\textbf{Grinchenko Sergey N.} (b.\ 1946)~--- Doctor of Science in technology, professor, principal 
scientist, Institute of Informatics Problems, Federal Research Center ``Computer Science and 
Control'' of the Russian Academy of Sciences, 44-2~Vavilov Str., Moscow 119333, Russian 
Federation; \mbox{sgrinchenko@ipiran.ru}

\label{end\stat}

\renewcommand{\bibname}{\protect\rm Литература} 