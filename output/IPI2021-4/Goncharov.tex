\def\stat{goncharov}

\def\tit{ОТРАЖЕНИЕ ЭВОЛЮЦИИ ЛЕКСИКОГРАФИЧЕСКИХ 
ЗНАНИЙ В~ДИНАМИЧЕСКИХ КЛАССИФИКАЦИОННЫХ 
СИСТЕМАХ$^*$}

\def\titkol{Отражение эволюции лексикографических знаний 
в~динамических классификационных системах}

\def\aut{А.\,А.~Гончаров$^1$, И.\,М.~Зацман$^2$, М.\,Г.~Кружков$^3$, 
Е.\,Ю.~Лощилова$^4$}

\def\autkol{А.\,А.~Гончаров, И.\,М.~Зацман, М.\,Г.~Кружков, 
Е.\,Ю.~Лощилова}

\titel{\tit}{\aut}{\autkol}{\titkol}

\index{Гончаров А.\,А.}
\index{Зацман И.\,М.}
\index{Кружков М.\,Г.}
\index{Лощилова Е.\,Ю.}
\index{Goncharov A.\,A.}
\index{Zatsman I.\,M.}
\index{Kruzhkov M.\,G.}
\index{Loshchilova E.\,Yu.}


{\renewcommand{\thefootnote}{\fnsymbol{footnote}} \footnotetext[1]
{Работа выполнена %в~Институте проблем информатики ФИЦ ИУ РАН 
при поддержке РФФИ (проект 20-012-00166) с~использованием ЦКП <<Информатика>> ФИЦ ИУ РАН.}}


\renewcommand{\thefootnote}{\arabic{footnote}}
\footnotetext[1]{Федеральный исследовательский центр <<Информатика 
и~управление>> Российской академии наук, \mbox{a.gonch48@gmail.com}}
\footnotetext[2]{Федеральный исследовательский центр <<Информатика 
и~управление>> Российской академии наук, \mbox{izatsman@yandex.ru}}
\footnotetext[3]{Федеральный исследовательский центр <<Информатика 
и~управление>> Российской академии наук, \mbox{magnit75@yandex.ru}}
\footnotetext[4]{Федеральный исследовательский центр <<Информатика 
и~управление>> Российской академии наук, \mbox{lena0911@mail.ru}}

%\vspace*{-7pt}

     \Abst{Рассматриваются две проблемы, решение которых необходимо для отражения 
эволюции лексикографических знаний. Описываемые в~\mbox{статье} знания представлены 
в~виде классификаций, чьи рубрики используются для снабжения лингвистической 
разметкой текстовых данных в~информационных системах. Эволюция 
лексикографических знаний обсуждается на примере надкорпусной базы данных (НБД). 
Первая проблема касается того, как интегрировать в~структуру НБД средства отражения 
изменений смыслового содержания рубрик. Предлагаемое решение заключается 
в~до\-бав\-ле\-нии в~структуру НБД двух таблиц, обеспечивающих хранение данных о 
состояниях руб\-рик и~об операциях по изменению этих рубрик. Показано, как эти таблицы 
встроены в~структуру НБД. Вторая проблема связана с~тем, как обеспечить внесение 
в~НБД изменений с~помощью пользовательского интерфейса. Представлен интерфейс 
НБД, позволяющий вносить сведения об этих изменениях. Предлагаемые решения могут 
быть масштабированы и~использоваться для отражения эволюции не только 
лексикографического, но и, шире, научного знания, если это знание представлено в~форме 
динамической классификационной системы.}
     
     \KW{эволюция лексикографических знаний; динамическая классификационная 
система; версионные онтологии; лингвистическое аннотирование; реклассификация 
аннотаций}
     
\DOI{10.14357/19922264210406}
  
%\vspace*{-4pt}


\vskip 10pt plus 9pt minus 6pt

\thispagestyle{headings}

\begin{multicols}{2}

\label{st\stat}
     
     
\section{Введение}

\vspace*{-2pt}

  Статья продолжает серию работ по эволюции лексикографических 
знаний, представленных в~виде классификаций, чьи рубрики используются 
для снабжения лингвистической разметкой (=\;для лингвистического 
аннотирования) текстовых данных в~информационных системах. Эволюция 
лексикографических знаний обсуждается на примере НБД.
  
  В работе~[1]~--- первой из этой серии~--- была описана структура 
словарной статьи и~те изменения, которые лексикографы могут вносить 
в~\mbox{статью} в~ходе подготовки словаря. Был предложен подход 
к~проектированию информационной системы, позволяющей сохранять 
информацию об этих изменениях. Затем в~работе~[2] подход был 
детализирован: сопоставлены статические и~динамические 
классификационные системы (далее~--- СКС и~ДКС соответственно), 
рассмотрена проблема реклассификации, возникающая при использовании 
ДКС. Были определены 7~операций по изменению классификационных 
рубрик (на примере рубрик, которые используются в~НБД, чтобы 
распределять на классы употребления немецких модальных глаголов (НМГ) 
на основании их семантики). В~работе~[3] были сопоставлены подходы 
к~классификации сущностей с~использованием изменяемых во времени 
классификационных систем, в~частности ДКС, и~онтологий других 
видов~[4], а~также дано описание того, как обеспечивается поддержка 
ДКС, интегрированной в~НБД НМГ.
  
  Цель настоящей \mbox{статьи} состоит в~том, чтобы описать: (1)~как система 
отражения изменений классификационных рубрик встроена в~структуру 
НБД и~(2)~как обеспечена возможность внесения этих изменений в~ДКС 
через пользовательский интерфейс.

\vspace*{-9pt}
  
\section{Лингвистическое аннотирование и~эволюция 
классификаций}

\vspace*{-2pt}

  Сегодня лексикографическая деятельность все теснее сближается 
  с~корпусной лингвистикой: словари составляются с~опорой на анализ 
корпусного материала\footnote[5]{К примеру, словари, созданные на основе Национального 
корпуса русского языка (НКРЯ), перечислены по адресу {\sf http://dict.ruslang.ru}.}. Это 
интегрированное направление получило название <<корпусная 
лексикография>>~[5].

\pagebreak

\end{multicols}

\begin{table*}\small
\begin{center}
\begin{tabular}{|p{42mm}|p{23mm}|p{55mm}|p{21mm}|}
\multicolumn{4}{c}{Пример аннотации ПС, сформированной в~НБД НМГ}\\
\multicolumn{4}{c}{\ }\\[-6pt]
\hline
\multicolumn{1}{|c|}{\tabcolsep=0pt\begin{tabular}{c}Контекст НМГ\\ в~оригинале\end{tabular}}
&\multicolumn{1}{c|}{\tabcolsep=0pt\begin{tabular}{c}Коды 
рубрик\\ оригинала\end{tabular}}&
\multicolumn{1}{c|}{\tabcolsep=0pt\begin{tabular}{c}Перевод контекста\\ на 
русский язык\end{tabular}}&
\multicolumn{1}{c|}{\tabcolsep=0pt\begin{tabular}{c}Коды рубрик\\ 
перевода\end{tabular}}\\
\hline
Dort \textbf{m$\ddot{\mbox{o}}$chte} ich mit dir \textit{sein}.\newline
\newline
[C.~Funke. Tintenblut (2005)]&\textbf{m$\ddot{\mbox{o}}$gen}\newline
$\langle$1sg$\rangle$\newline
$\langle$PraetConj$\rangle$\newline
$\langle$+Inf$\rangle$\newline
$\langle$m$\ddot{\mbox{o}}$gen-01$\rangle$&
Вот где я \textbf{бы хотела} с~тобой \textit{оказаться}.\newline
\newline
[К.~Функе. Чернильная смерть\newline (пер. М.~Сокольская, 2007)]&
\textbf{хотеть}\newline
$\langle$1sg$\rangle$\newline
$\langle$Past+бы$\rangle$\newline
$\langle$+Inf$\rangle$\\
\hline
\multicolumn{4}{p{156mm}}{\footnotesize \textbf{Примечания}.\newline
Расшифровка кодов рубрик оригинала
\begin{itemize}
   \item[\,] Основная рубрика:\vspace*{-4pt}
   \begin{itemize}
\item 
[$\bullet$] \textbf{m$\ddot{\mbox{o}}$gen}~--- в~данном контексте 
исследуемым НМГ выступает <<m$\ddot{\mbox{o}}$gen>> (слово 
<<m$\ddot{\mbox{o}}$chte>>, выделенное полужирным шрифтом в~столбце~1, 
является формой этого НМГ).\vspace*{-4pt}
\end{itemize}
\item[\,] Дополнительные рубрики:\vspace*{-4pt}
   \begin{itemize}
\item[$\bullet$] $\langle$1sg$\rangle$~---  исследуемый НМГ употреблен в~форме 
первого лица единственного числа;\vspace*{-4pt}
\item[$\bullet$] $\langle$PraetConj$\rangle$~---  исследуемый НМГ употреблен 
в~форме прошедшего времени сослагательного наклонения (от лат.\ 
\textit{praeteritum conjunctivi});\vspace*{-4pt}
\item[$\bullet$] $\langle$+Inf$\rangle$~---  исследуемому НМГ подчинен инфинитив 
(слово <<sein>>, выделенное курсивом в~столб\-це~1);\vspace*{-4pt}
\item[$\bullet$] $\langle$m$\ddot{\mbox{o}}$gen-01$\rangle$~--- исследуемый НМГ 
употреблен в~первом значении согласно порядку описания значений в~словарной 
статье из~\cite{6-go}.\vspace*{-4pt}
\end{itemize}
\end{itemize}
Расшифровка кодов рубрик перевода %\vspace*{-3pt}
\begin{itemize}
   \item[\,] Основная рубрика:\vspace*{-4pt}
   \begin{itemize}
\item[$\bullet$] \textbf{хотеть}~--- в~данном контексте семантику исследуемого 
НМГ в~переводе передает глагол <<хотеть>> (слова <<бы хотела>>, выделенные 
полужирным шрифтом в~столб\-це~3, являются неоднословной формой данного 
глагола).\vspace*{-4pt}
\end{itemize}
\item[\,] Дополнительные рубрики:\vspace*{-4pt}
\begin{itemize}
\item[$\bullet$] $\langle$1sg$\rangle$~---  глагол употреблен в~форме первого лица 
единственного числа;\vspace*{-4pt}
\item[$\bullet$] $\langle$Past+бы$\rangle$~---  глагол употреблен в~форме 
сослагательного наклонения (сочетание формы прошедшего времени и~частицы 
<<бы>>);\vspace*{-4pt}
\item[$\bullet$] $\langle$+Inf$\rangle$~---  глаголу подчинен инфинитив (слово 
<<оказаться>>, выделенное курсивом в~столб\-це~3).\vspace*{-4pt}
\end{itemize}
\end{itemize}
Коды рубрик ПС~--- в~данном примере отсутствуют.}
\end{tabular}
\end{center}
\vspace*{-9pt}
\end{table*}


\begin{multicols}{2}
  
Для фиксации результатов анализа могут использоваться базы данных. Так, 
в~целях интегра-\linebreak ции корпусного материала и~словарных статей
не\-мец\-ко-рус\-ско\-го словаря~\cite{6-go} для  
6~НМГ~--- \textit{d$\ddot{\mbox{u}}$rfen}, \textit{k$\ddot{\mbox{o}}$nnen}, 
\textit{m$\ddot{\mbox{o}}$gen}, \textit{m$\ddot{\mbox{u}}$ssen}, \textit{sollen} 
и~\textit{wollen}~--- была\linebreak создана НБД НМГ\footnote{Описание лингвистических целей 
создания НБД НМГ дано в~\cite{7-go}; некоторые полученные с~ее помощью результаты 
описаны в~\cite{8-go, 9-go}.}. В~нее были загружены тексты\linebreak параллельного немецкого 
подкорпуса НКРЯ  
({\sf https:// ruscorpora.ru/new/search-para-de.html}) для направления перевода  
<<не\-мец\-кий--рус\-ский>> общим объемом более 15,6~млн 
словоупотреблений (на 15.09.2021).

  Чтобы с~использованием НБД интегрировать исходные версии 
словарных статей, в~которых описываются НМГ, с~примерами из корпуса 
и~получить эмпирический материал, необходимый для уточнения их 
лексикографического описания, следует:
  \begin{enumerate}[(1)]
  \item осуществить поиск пар фрагментов параллельных текстов, где 
фрагмент на немецком языке содержит словоупотребление искомого\linebreak НМГ;
\item провести семантический анализ этого словоупотребления, 
выделив контекст оригинала, достаточный для определения значения 
словоупотребления согласно~\cite{6-go}, и~соответству\-ющий 
оригиналу контекст перевода;\\[-14.5pt]
\item снабдить эти контексты лингвистической разметкой 
(=\;осуществить их лингвистическое аннотирование; о его видах см.\ 
%\cite{10-go, 11-go, 12-go, 13-go}; \cite[с.~484--776]{14-go}; \cite[с.~6--7]{15-go});
[10, c.~6-7; 11, c.~484--776; 12--15];\\[-14.5pt]
\item повторять пп.~2 и~3 до получения необходимого объема 
материала.\\[-14.5pt]
\end{enumerate}

  Контекстам оригинала и~перевода, а также переводному соответствию  
(ПС) в~ходе лингвистического аннотирования при помощи НБД 
приписывается информация: о НМГ, который употреблен в~оригинале; 
грамматической форме этого НМГ; характеристиках контекста оригинала; 
семантике употребления НМГ; языковой единице, пе\-ре\-да\-ющей семантику 
НМГ в~переводе; грамматической форме этой единицы; характеристиках 
контекста перевода; характеристиках ПС. Эта информация фиксируется при 
помощи рубрик фасетной классификации~\cite{16-go, 17-go}. Результат 
аннотирования одной пары контекстов (оригинала и~перевода) называется 
аннотацией ПС. Аннотация ПС может быть представлена в~виде таблицы 
(см.\ пример в~таб\-лице).
  

В процессе анализа корпусного материала знание лексикографов о 
семантике языковых единиц нередко меняется\footnote[1]{Так, в~статье 
<<Лексикография>>, приведенной в~<<Глоссарии по корпусной лингвистике>>, отмечается: <<Основанные 
на корпусах подходы могут помочь создателям словарей выявить намного больше значений и~способов 
употребления лексических единиц. К~примеру, второе издание словаря ``Longman Dictionary of Contemporary 
English'' было написано без использования корпусных данных, тогда как при написании третьего издания 
корпусные данные использовались. Если во втором издании для слова ``know'' описывались 20~значений, то 
в~третьем~--- более 40>> (см.~\cite[с.~107--108]{18-go}; перевод авторов статьи).}. Если же семантика 
этих единиц описывается при помощи классификационных рубрик, как 
в~НБД НМГ, где каждому контексту с~НМГ должна быть присвоена одна 
рубрика фасета <<Значения НМГ>>\footnote{Возможность присвоить более одной 
рубрики, описывающей семантику употребления НМГ, существует, но на практике применяется крайне 
редко.}, то эти рубрики также могут меняться вслед за изменением знания. 
Так, при анализе материала параллельного немецко-русского корпуса 
встречаются контексты, содержащие такие употребления НМГ, которые не 
удается описать с~помощью текущей версии классификации из-за 
отсутствия в~ней релевантных рубрик\footnote{Например, была найдена следующая пара 
контекстов: <<,,\mbox{Gro{\hspace*{-1pt}\!\ptb{\ss}}\hspace*{1pt}papa}``, \textbf{konnte} 
der kleine Hans Castorp im Kabinett wohl sagen [$\ldots$], 
,,zeig mir doch, bitte, die Taufschale!''>> [Th.~Mann. Der Zauberberg (1924)]; <<Дедушка,~--- обычно говорил 
маленький Ганс Касторп, войдя в~кабинет [$\ldots$],~--- покажи мне, пожалуйста, купель!>> [Т.~Манн. Волшебная гора (пер. 
В.~Станевич, 1959)]. Для описания значения глагола <<k$\ddot{\mbox{o}}$nnen>> в~этом контексте 
(выделенная полужирным форма <<konnte>>) 23.04.2020 в~классификацию была до\-бав\-ле\-на руб\-ри\-ка с~кодом 
<<k$\ddot{\mbox{o}}$nnen-09>> (Id~953). Она используется, когда <<k$\ddot{\mbox{o}}$nnen>> употреблен 
в~форме прошедшего времени изъявительного наклонения (\textit{praeteritum indicativi}) в~повествовании 
о~прошлом и~может быть переведен как <<бывало>> или <<случалось>>.}. В~таких случаях 
составители словаря пересматривают пред\-став\-ле\-ния о~сегментации спектра 
значений НМГ, что приводит к~изменениям словарной статьи, для 
описания которых нужна \textit{динамическая} классификационная система 
с~изменяемыми рубриками. По этой причине в~НБД НМГ было 
решено использовать для аннотирования именно ДКС. В~отличие от 
версионных классификационных систем с~заданным периодом времени их 
обновления\footnote{Так, рубрики Международной патентной классификации могут меняться не 
чаще, чем раз в~квартал~\cite{19-go}.} до\-бав\-ле\-ние и~изменение рубрик ДКС возможно в~любой 
момент времени. Это позволяет оперативно отражать в~НБД этапы 
эволюции лексикографических знаний и~использовать до\-бав\-лен\-ные 
и~измененные рубрики в~процессе лингвистического аннотирования сразу 
после внесения изменений.

\section{Эволюция фасетной динамической 
классификационной системы}

  Как было сказано выше, совокупность руб\-рик, используемых для 
аннотирования контекстов, содержащих упо\-треб\-ле\-ния НМГ, пред\-став\-ля\-ет 
собой фасетную ДКС. Для удобства представления данных об 
аннотируемом контексте оригинала один фасет~--- <<Немецкий модальный 
 глагол>>~--- считается основным, а~все остальные~--- дополнительными. 
Так, в~таблице (столбец~2) руб\-ри\-ка, входящая в~основной фасет (далее~--- 
основная руб\-ри\-ка) и~использованная для аннотирования кон\-текс\-та 
оригинала, выделена полужирным \mbox{шриф\-том}, а~руб\-ри\-ки, входящие в~другие 
фасеты (далее~--- дополнительные руб\-ри\-ки), перечислены под ней. Сис\-те\-ма 
отражения изменений в~руб\-ри\-ках ДКС на данный момент реализована для 
фасета <<Значения НМГ>>.
{\looseness=1

}
  
  В структуре НБД можно выделить две со\-став\-ля\-ющие: корпусную 
(обеспечивающую хранение аннотируемых текс\-тов) и~надкорпусную 
(обеспечивающую хранение аннотаций, в~част\-ности аннотаций ПС с~НМГ). 
С~описанием логических схем обеих со\-став\-ля\-ющих НБД можно подробнее 
ознакомиться в~\cite{20-go}. На рис.~1 пред\-став\-лен фрагмент логической 
схемы надкорпусной со\-став\-ля\-ющей НБД НМГ, включающий 4~таб\-ли\-цы, 
используемые во всех НБД, разработанных на сегодняшний день (<<Доп. 
рубрика>>, <<Фасет доп.\ руб\-рик>>, <<Фрагмент>>, <<Основная 
рубрика>>), а~так\-же две таб\-ли\-цы, до\-бав\-лен\-ные в~НБД НМГ, чтобы 
обеспечить сохранение ретроспективы поэтапных изменений, вносимых 
в~дефиниции рубрик НБД (это таб\-ли\-цы <<Со\-сто\-яние>> и~<<Операция>>). 
История изменений руб\-рик фасета <<Значения НМГ>> полностью 
сохраняется в~таб\-ли\-цах <<Со\-сто\-яние>> и~<<Операция>>, при этом 
в~таб\-ли\-це <<Со\-сто\-яние>> хранятся все со\-сто\-яния дефиниций руб\-рик этого 
фасета (как актуальные, так и~устаревшие), а~в~таб\-ли\-це <<Операция>>~--- 
описания всех операций, с~по\-мощью которых в~эти рубрики вносились 
изменения. Поля этих таб\-лиц и~реализация связи между ними подробно 
описаны в~работе~\cite{3-go}.
  
\begin{figure*} %fig1
\vspace*{1pt}
  \begin{center}  
    \mbox{%
\epsfxsize=161.76mm
\epsfbox{gon-1.eps}
}

\end{center}
\vspace*{-9pt}
\Caption{Фрагмент логической схемы НБД, включающий таблицы, отвечающие за 
сохранение ретроспективы поэтапных изменений классификационных рубрик}
\vspace*{-6pt}
\end{figure*}

  Рисунок~1 также иллюстрирует связь этих двух таб\-лиц с~другими 
таб\-ли\-ца\-ми надкорпусной со\-став\-ля\-ющей НБД. Следует отметить, что 
в~результате выполнения операций DELETE или MERGE (подробнее об 
этих и~других операциях см.~\cite{2-go, 3-go}) записи вместе 
с~соответствующими уникальными идентификаторами могут пол\-ностью 
удаляться из таблицы <<Доп.\ руб\-ри\-ка>>, однако данные об истории 
изменений этих удаленных записей будут по-преж\-не\-му до\-ступ\-ны в~таб\-ли\-це 
<<Состояние>>. В~связи с~этим\linebreak\vspace*{-12pt}

\pagebreak

\noindent
 связь между таб\-ли\-ца\-ми <<Доп.\ рубрика>> 
и~<<Со\-сто\-яние>> (показанная пунк\-ти\-ром) реализуется лишь
 на логическом, но не на физическом уровне НБД, поскольку в~противном 
случае удаление записей из таб\-ли\-цы <<Доп.\ руб\-ри\-ка>> приводило бы 
к~ошибке, связанной с~нарушением ссылочной це\-лост\-ности данных: записи 
в~таб\-ли\-це <<Со\-сто\-яние>> через внеш\-ний ключ ссылались бы на удаленные 
записи из таб\-ли\-цы <<Доп.\ руб\-ри\-ка>>. При использовании представленной 
выше схемы в~таб\-ли\-це <<Со\-сто\-яние>> может сохраняться история 
изменений не только всех руб\-рик фасета <<Значения НМГ>>, 
присутствующих в~таб\-ли\-це <<Доп.\ руб\-ри\-ка>>, но и~тех руб\-рик, которые 
уже были удалены из этой таб\-лицы.

\section{Внесение изменений в~динамическую 
классификационную систему}

  Чтобы обеспечить возможность внесения изменений в~руб\-ри\-ки ДКС 
лингвистами, был разработан пользовательский интерфейс.

  \begin{figure*} %fig2
  \vspace*{1pt}
  \begin{center}  
    \mbox{%
\epsfxsize=163mm
\epsfbox{gon-2.eps}
}

\end{center}
\vspace*{-9pt}
  \Caption{Форма редактирования рубрик фасета <<Значения НМГ>>: независимое 
изменение дефиниции (операция 10159, выполнена 27.06.2021)}
  \end{figure*}
  
  На рис.~2 представлен пример заполнения формы редактирования руб\-рик 
фасета <<Значения НМГ>>. Для внесения изменений необходимо преж\-де 
всего выбрать из вы\-па\-да\-юще\-го списка одну из 7~операций, что определит 
внешний вид формы редактирования. Поскольку в~примере с~рис.~2 
выбрана операция <<REVISE>>~--- изменение дефиниции рубрики, при 
котором смыс\-ло\-вое содержание этой дефиниции не сужается и~не 
затрагиваются дефиниции других руб\-рик~--- можно выбрать из 
выпадающего списка только одну руб\-ри\-ку (в~данном случае  
<<m$\ddot{\mbox{o}}$gen-01>>). После выбора руб\-ри\-ки в~поле 
<<Описание значения>> отобразится ее текущая дефиниция и~станет 
до\-ступ\-ным текс\-то\-вое поле <<Новое описание значения>>, куда вводится 
новая дефиниция. Изменения вносятся по клику на со\-от\-вет\-ст\-ву\-ющую 
кнопку\footnote{Для отмены изменений, внесенных ошибочно, предусмот\-ре\-на кнопка 
<<Отменить>>.}, после чего во всех аннотациях с~руб\-ри\-кой 
<<m$\ddot{\mbox{o}}$gen-01>> (как, например, аннотация из таблицы) эта 
руб\-ри\-ка будет ссылаться на новую версию дефиниции.

  \begin{figure*} %fig3
  \vspace*{1pt}
  \begin{center}  
    \mbox{%
\epsfxsize=161.76mm
\epsfbox{gon-3.eps}
}

\end{center}
\vspace*{-9pt}
  \Caption{Форма редактирования руб\-рик фасета <<Значения НМГ>>: слияние двух 
дефиниций (операция 10161, выполнена 27.06.2021)}
  \end{figure*}
  
  На рис.~3 представлен пример заполнения фор-\linebreak  мы редактирования руб\-рик 
фасета <<Значения НМГ>> для выполнения операции <<MERGE>>~--- 
слияния дефиниций двух руб\-рик. Пользователь выбирает две руб\-ри\-ки, 
дефиниции которых объединяются в~одну, и~вводит в~текс\-то\-вое поле 
дефиницию результирующей руб\-ри\-ки. После внесения изменений руб\-ри\-ка 
с~кодом <<m$\ddot{\mbox{o}}$gen-06>> получит дефиницию, введенную 
в~поле <<Описание значения после слияния>>, руб\-ри\-ка 
<<m$\ddot{\mbox{o}}$gen-07>> будет удалена, а~во\linebreak\vspace*{-12pt}

\pagebreak

\end{multicols}

\begin{figure*} %fig4
\vspace*{1pt}
  \begin{center}  
    \mbox{%
\epsfxsize=161.76mm
\epsfbox{gon-4.eps}
}

\end{center}
\vspace*{-6pt}
\Caption{Форма редактирования рубрик фасета <<Значения НМГ>>: перераспределение 
смыслового содержания двух дефиниций (операция 10162, выполнена 28.06.2021)}
%\vspace*{-6pt}
\end{figure*}

\begin{multicols}{2}

\noindent
 всех аннотациях, где 
она была использована ра\-нее, вмес\-то кода <<m$\ddot{\mbox{o}}$gen-07>> 
будет автоматически про\-став\-лен код <<m$\ddot{\mbox{o}}$gen-06>>.






  
  На рис.~4 представлен еще один пример заполнения формы 
редактирования руб\-рик фасета <<Значения НМГ>>: для выполнения 
операции \mbox{<<REDISTR>>}. Эта операция описывает такое изменение 
дефиниций двух руб\-рик, при котором компоненты смыс\-ло\-вого содержания 
дефиниций этих руб\-рик перераспределяются между ними. 

В~примере 
с~рис.~4 изменяются дефиниции двух руб\-рик: <<m$\ddot{\mbox{u}}$ssen-02>> 
и~<<m$\ddot{\mbox{u}}$ssen-04>>. Как видно, после выполнения 
этого изменения из дефиниции руб\-ри\-ки <<m$\ddot{\mbox{u}}$ssen-02>> 
фрагмент <<в~во\-про\-си\-тель\-но-вос\-кли\-ца\-тель\-ных конструкциях 
с~оттенком неуверенного предположения>> переносится в~дефиницию 
руб\-ри\-ки <<m$\ddot{\mbox{u}}$ssen-04>>. Следовательно, объем значения 
дефиниции руб\-ри\-ки <<m$\ddot{\mbox{u}}$ssen-02>> сужается, поэтому 
в~форме необходимо по\-ста\-вить соответствующий флажок. Все аннотации, 
содержавшие эту руб\-ри\-ку, после выполнения операции подлежат 
\textit{экспертной реклассификации}, нацеленной на выявление тех из них, 
которые должны быть отнесены к~рубрике <<m$\ddot{\mbox{u}}$ssen-04>>. 
Чтобы обеспечить воз\-мож\-ность поиска этих аннотаций, все они 
программно помечаются со\-от\-вет\-ст\-ву\-ющим тегом.

\vspace*{-6pt}

  
\section{Заключение}

\vspace*{-2pt}

  Описанный в~работах~\cite{2-go, 3-go} и~в~настоящей статье подход 
к~разработке баз данных, обеспечивающих поддержку фасетных ДКС, на 
сегодняшний день реализован для одного фасета классификации, однако 
опыт, полученный в~ходе выполнения этой работы, показывает, что такой 
подход может быть адаптирован ко всей классификации. Сис\-те\-ма 
отражения изменений классификационных рубрик, основанная на 
использовании баз данных, обеспечивает широкий спектр возможностей 
анализа вносимых изменений.
  
  Постоянная эволюция современного \mbox{научного} знания, пред\-став\-ля\-емо\-го 
в~информационных сис\-те\-мах, в~том чис\-ле при помощи 
классификаций \cite{21-go, 22-go, 23-go, 24-go, 25-go, 26-go}, 
свидетельствует о~рас\-ту\-щей ак\-ту\-аль\-ности исследования фасетных ДКС 
и~не\-об\-хо\-ди\-мости продолжать изуче\-ние проб\-лем пред\-став\-ле\-ния 
изменяемого знания в~информационных сис\-темах.

\vspace*{-4pt}
  
{\small\frenchspacing
 {\baselineskip=10.75pt
 %\addcontentsline{toc}{section}{References}
 \begin{thebibliography}{99}

 \vspace*{-2pt}

\bibitem{1-go}
\Au{Гончаров А.\,А., Зацман~И.\,М., Кружков~М.\,Г.} Темпоральные данные 
в~лексикографических базах знаний~// Информатика и~её применения, 2019. Т.~13. 
Вып.~4. С.~90--96.
\bibitem{2-go}
\Au{Гончаров А.\,А., Зацман~И.\,М., Кружков~М.\,Г.} Эволюция классификаций 
в~надкорпусных базах данных~// Информатика и~её применения, 2020. Т.~14. 
Вып.~4. С.~108--116.
\bibitem{3-go}
\Au{Гончаров А.\,А., Зацман~И.\,М., Кружков~М.\,Г.} Пред\-став\-ле\-ние новых 
лексикографических знаний в~динамических классификационных сис\-те\-мах~// 
Информатика и~её применения, 2021. Т.~15. Вып.~1. 
С.~86--93.
\bibitem{4-go}
\Au{McGuinness~D.\,L.} Ontologies come of age~// Spinning the Semantic Web: Bringing the 
World Wide Web to its full potential~/ Eds. D.~Fensel, J.~Hendler, H.~Lieberman, 
W.~Wahlster.~--- Cambridge, MA, USA: MIT Press, 2003. P.~171--194.
\bibitem{5-go}
\Au{Ooi V.\,B.\,Y.} Computer corpus lexicography.~--- Edinburgh, U.K.: Edinburgh University 
Press, 1998. 255~p.
\bibitem{6-go}
Немецко-русский словарь актуальной лексики~/ Под ред.\ Д.\,О.~Добровольского.~--- 
М.: Лексрус, 2021 (в~печати).
\bibitem{7-go}
\Au{Добровольский Д.\,О., Зализняк Анна~А.} Немецкие конструкции с~модальными 
глаголами и~их русские соответствия: проект надкорпусной базы данных~// 
Компьютерная лингвистика и~интеллектуальные \mbox{технологии}: По мат-лам Междунар. 
конф. <<Диалог>>.~--- М.: РГГУ, 2018. Вып.~17(24). С.~172--184.
\bibitem{8-go}
\Au{Добровольский Д.\,О.} Немецкие модальные глаголы в~параллельном корпусе 
и~задачи двуязычной лексикографии~// Германские языки: текст, корпус, перевод.~--- 
М.: Институт языкознания РАН, 2020. С.~103--116.
\bibitem{9-go}
\Au{Добровольский Д.\,О., Зализняк Анна~А.} Русские конструкции с~потенциально 
модальным значением по данным параллельных корпусов~// Труды Института 
русского языка им.\ В.\,В.~Виноградова, 2020. №\,3. С.~35--49.
\bibitem{15-go} %10
\Au{Захаров В.\,П.} Корпусная лингвистика.~--- СПб.: \mbox{СПбГУ}, 2005. 48~с.
\bibitem{14-go} %11
Corpus linguistics: An international handbook~/
Eds. A.~L$\ddot{\mbox{u}}$deling, M.~Kyt$\ddot{\mbox{o}}$.~--- Berlin\,/\,New York: 
Walter de Gruyter, 2008. Vol.~1. 794~p.
\bibitem{10-go} %12
Corpus annotation: Linguistic information from computer text corpora~/
Eds. R.~Garside, G.~Leech, T.~McEnery.~--- London\,/\,New York: Routledge, 2013. 291~p.
\bibitem{12-go} %13
\Au{Pustejovsky J., Stubbs~A.} Natural language annotation for machine learning.~--- 
Beijing/Cambridge/Farnham/ K$\ddot{\mbox{o}}$ln/Sebastopol/Tokyo: 
O'Reilly Media, 2013. 340~p.
\bibitem{11-go} %14
\Au{K$\ddot{\mbox{u}}$bler S., Zinsmeister~H.} Corpus linguistics and linguistically annotated 
corpora.~--- London\,/\,New York: Bloomsbury, 2015. 320~p.
\bibitem{13-go} %15
Handbook of linguistic annotation~/ Eds. N.~Ide, J.~Pustejovsky.~--- Dordrecht, The 
Netherlands: Springer Science\;+\;Business Media, 2017. 1468~p.
\bibitem{16-go}
\Au{Зацман И.\,М., Инькова~О.\,Ю., Кружков~М.\,Г., Попкова~Н.\,А.} Представление 
кроссязыковых знаний о~коннекторах в~надкорпусных базах данных~// Информатика 
и~её применения, 2016. Т.~10. Вып.~1. 
С.~106--118.
\bibitem{17-go}
\Au{Зацман И.\,М., Инькова~О.\,Ю., Нуриев~В.\,А.} Построение классификационных 
схем: методы и~технологии экспертного формирования~// На\-уч\-но-тех\-ни\-че\-ская 
информация. Сер.~2: Информационные процессы и~сис\-те\-мы, 2017. №\,1. С.~8--22.
\bibitem{18-go}
\Au{Baker P., Hardie~A., McEnery~T.} A~glossary of corpus linguistics.~--- Edinburgh, 
U.K.: Edinburgh University Press, 2006. 187~p.
\bibitem{19-go}
\Au{Зацман И.\,М., Косарик~В.\,В., Курчавова~О.\,А.} Задачи представления 
личностных и~коллективных концептов в~циф\-ро\-вой среде~// Информатика и~её 
применения, 2008. Т.~2. Вып.~3. С.~54--69.
\bibitem{20-go}
\Au{Зацман И.\,М., Кружков~М.\,Г., Лощилова~Е.\,Ю.} Методы и~средства 
информатики для описания структуры неоднословных коннекторов~// Структура 
коннекторов и~методы ее описания~/ Под ред. О.\,Ю.~Иньковой.~--- М.: ТОРУС 
ПРЕСС, 2019. С.~205--230.
\bibitem{21-go}
\Au{Зацман И.\,М.} Целенаправленное развитие систем лингвистических знаний: 
выявление и~заполнение лакун~// Информатика и~её применения, 2019. Т.~13. 
Вып.~1. С.~91--98.
\bibitem{22-go}
\Au{Zatsman I.} Finding and filling lacunas in linguistic typologies~// 15th  Forum (International) 
on Knowledge Asset Dynamics Proceedings.~--- Matera, Italy: Institute of Knowledge Asset 
Management, 2020. P.~780--793.
\bibitem{23-go}
\Au{Zatsman I.} Three-dimensional encoding of emerging meanings in AI-systems~// 21st 
European Conference on Knowledge Management Proceedings.~--- Reading, MA, USA: 
Academic Publishing International Ltd., 2020. P.~878--887.
\bibitem{24-go}
\Au{Zatsman I., Khakimova~A.} New knowledge discovery for creating terminological profiles 
of diseases~// 22nd European Conference on Knowledge Management Proceedings.~--- 
Reading, MA, USA: Academic Publishing International Ltd., 2021. P.~837--846.
\bibitem{25-go}
\Au{Zatsman I.} A~model of goal-oriented knowledge discovery based on human--computer 
symbiosis~// 16th Forum (International) on Knowledge Asset Dynamics Proceedings.~--- Rome, 
Italy: Arts for Business Institute, 2021. P.~297--312.
\bibitem{26-go}
\Au{Зацман И.\,М.} Формы представления нового знания, извлеченного из текс\-тов~// 
Информатика и~её применения, 2021. Т.~15. Вып.~3. С.~83--90.
\end{thebibliography}

 }
 }

\end{multicols}

\vspace*{-9pt}

\hfill{\small\textit{Поступила в~редакцию 15.10.21}}

%\vspace*{8pt}

%\pagebreak

\newpage

\vspace*{-28pt}

%\hrule

%\vspace*{2pt}

%\hrule

%\vspace*{-2pt}

\def\tit{CAPTURING EVOLUTION OF~LEXICOGRAPHIC KNOWLEDGE IN~DYNAMIC 
CLASSIFICATION SYSTEMS}


\def\titkol{Capturing evolution of lexicographic knowledge in dynamic classification systems}


\def\aut{A.\,A.~Goncharov, I.\,M.~Zatsman, M.\,G.~Kruzhkov, and~E.\,Yu.~Loshchilova}

\def\autkol{A.\,A.~Goncharov, I.\,M.~Zatsman, M.\,G.~Kruzhkov, and~E.\,Yu.~Loshchilova}

\titel{\tit}{\aut}{\autkol}{\titkol}

\vspace*{-11pt}


\noindent
Federal Research Center ``Computer Science and Control'' of the Russian Academy of Sciences,  
44-2 Vavilov Str., Moscow 119333, Russian Federation

\def\leftfootline{\small{\textbf{\thepage}
\hfill INFORMATIKA I EE PRIMENENIYA~--- INFORMATICS AND
APPLICATIONS\ \ \ 2021\ \ \ volume~15\ \ \ issue\ 4}
}%
 \def\rightfootline{\small{INFORMATIKA I EE PRIMENENIYA~---
INFORMATICS AND APPLICATIONS\ \ \ 2021\ \ \ volume~15\ \ \ issue\ 4
\hfill \textbf{\thepage}}}

\vspace*{3pt} 



\Abste{The paper examines two problems that have to be addressed in order to capture the 
evolution of lexicographic knowledge. The lexicographic knowledge discussed in the paper is 
presented in the form of classifications categories that are used to provide linguistic markup for 
textual data in information systems. Evolution of lexicographic data is considered based on the 
example of a supracorpora database (SCDB). The first problem deals with integration of the 
framework for capturing changes of semantic content of classification categories into the SCDB. 
The proposed solution to this problem involves integration of two new tables that capture 
information about temporal states of the classification categories and about change operations 
applied to those categories. The paper describes how these tables are integrated into the SCDB 
structure. The second problem deals with providing the user interface for application of changes to 
the classification categories. The interface implemented in the SCDB is described in detail. The 
proposed solutions can be scaled so that they would make it possible to capture evolution not only 
of lexicographic knowledge but of scientific knowledge in general if this knowledge can be 
represented in the form of a dynamic classification system.}

\KWE{evolution of lexicographic knowledge; dynamic classification system; ontology versioning; 
linguistic annotation; reclassification of annotations}

\DOI{10.14357/19922264210406}

\vspace*{-9pt}

\Ack

\vspace*{-4pt}

\noindent
The reported study was funded by RFBR, project number 20-012-00166. The research was 
carried out using the infrastructure of the Shared Research Facilities ``High Performance 
Computing and Big Data'' (CKP ``Informatics'') of FRC CSC RAS (Moscow).


%\vspace*{6pt}

  \begin{multicols}{2}

\renewcommand{\bibname}{\protect\rmfamily References}
%\renewcommand{\bibname}{\large\protect\rm References}

{\small\frenchspacing
 {%\baselineskip=10.8pt
 \addcontentsline{toc}{section}{References}
 \begin{thebibliography}{99}

\bibitem{1-go-1}
\Aue{Goncharov, A.\,A., I.\,M.~Zatsman, and M.\,G.~Kruzhkov}. 2019. Temporal'nye dannye 
v~leksikograficheskikh bazakh znaniy [Temporal data in lexicographic databases]. 
\textit{Informatika i~ee Primeneniya~--- Inform. Appl.}
 13(4):90--96.
\bibitem{2-go-1}
\Aue{Goncharov, A.\,A., I.\,M.~Zatsman, and M.\,G.~Kruzhkov.} 2020. Evolyutsiya klassifikatsiy 
v~nadkorpusnykh ba\-zakh dannykh [Evolution of classifications in supracorpora databases]. 
\textit{Informatika i~ee Primeneniya~--- Inform. Appl.} 14(4):108--116.
\bibitem{3-go-1}
\Aue{Goncharov, A.\,A., I.\,M.~Zatsman, and M.\,G.~Kruzhkov.} 2021. Predstavlenie novykh 
leksikograficheskikh znaniy v~dinamicheskikh klassifikatsionnykh sistemakh [Representing New 
Lexicographical Knowledge in Dynamic Classification Systems]. \textit{Informatika i~ee 
Primeneniya~--- Inform. Appl.} 15(1):86--93.
\bibitem{4-go-1}
\Aue{McGuinness, D.\,L.} 2003. Ontologies come of age. \textit{Spinning the Semantic Web: 
Bringing the World Wide Web to its full potential}. Eds. D.~Fensel, J.~Hendler, 
H.~Lieberman, and W.~Wahlster. Cambridge, MA: MIT Press. 171--194.
\bibitem{5-go-1}
\Aue{Ooi, V.\,B.\,Y.} 1998. \textit{Computer corpus lexicography}. Edinburgh, U.K.: 
Edinburgh University Press. 255~p.
\bibitem{6-go-1}
Dobrovol'skiy, D.\,O., ed. 2021 (in press). \textit{Nemetsko-russkiy slovar' aktual'noy 
leksiki} [German--Russian dictionary: Actual vocabulary]. Moscow: Leksrus.
\bibitem{7-go-1}
\Aue{Dobrovol'skiy, D.\,O., and Anna A.~Zalizniak}. 2018. Nemetskie konstruktsii s~modal'nymi 
glagolami i~ikh russkie sootvetstviya: proekt nadkorpusnoy bazy dannykh [German constructions 
with modal verbs and their Russian correlates: A~supracorpora database project]. 
\textit{Komp'yuternaya lingvistika i~intellektual'nye tekhnologii: po  
mat-lam Mezhdunar. konf. ``Dialog''} [Computer Linguistic and Intellectual Technologies: 
Conference (International) ``Dialog'' Proceedings]. Moscow. 17(24):172--184.
{\looseness=1

}
\bibitem{8-go-1}
\Aue{Dobrovol'skiy, D.\,O.} 2020. Nemetskie modal'nye glagoly v~parallel'nom korpuse 
i~zadachi dvuyazychnoy leksikografii [German modal verbs in a parallel corpus and bilingual 
lexicography tasks]. \textit{Germanskie yazyki: Tekst, korpus, and perevod} [German languages: 
Text, corpus, and translation]. Moscow: Institut yazykoznaniya RAS. 103--116.
{\looseness=1

}
\bibitem{9-go-1}
\Aue{Dobrovol'skiy, D.\,O., and Anna A.~Zaliznyak.} 2020. Russkie konstruktsii s~potentsial'no 
modal'nym znacheniem po dannym parallel'nykh korpusov [Russian con-structions with potentially 
modal meanings: An analysis based on parallel corpus data]. \textit{Trudy Instituta russkogo 
yazyka im. V.\,V.~Vinogradova} [V.\,V.~Vinogradov Russian Language Institute Proceedings]. 
35--49.
\bibitem{15-go-1} %10
\Aue{Zakharov, V.\,P.} 2005. \textit{Korpusnaya lingvistika} [Textbook of corpus linguistics]. 
St.\ Petersburg: SPbGU. 48~p.

\pagebreak

\bibitem{14-go-1} %11
L$\ddot{\mbox{u}}$deling, A., and M.~Kyt$\ddot{\mbox{o}}$, eds. 2008. \textit{Corpus 
linguistics: An international handbook}. Berlin\,/\,New York: Walter de Gruyter. Vol.~1. 794~p.
\bibitem{10-go-1} %12
Garside, R., G.~Leech, and T.~McEnery, eds. 2013. \textit{Corpus annotation: Linguistic 
information from computer text corpora}. London\,/\,New York: Routledge. 291~p.
\bibitem{12-go-1} %13
\Aue{Pustejovsky, J., and A.~Stubbs}. 2013. \textit{Natural language annotation for machine 
learning}. Beijing/Cambridge/ Farnham/K$\ddot{\mbox{o}}$ln/Sebastopol/Tokyo: O'Reilly Media. 340~p.

\bibitem{11-go-1} %14
\Aue{K$\ddot{\mbox{u}}$bler, S., and H.~Zinsmeister.} 2015. \textit{Corpus linguistics and 
linguistically annotated corpora}. London\,/\,New York: Bloomsbury. 320~p.
\bibitem{13-go-1} %15
Ide, N., and J. Pustejovsky, eds. 2017. \textit{Handbook of linguistic annotation}. Dordrecht, 
The Netherlands: Springer Science\;+\;Business Media. 1468~p.
\bibitem{16-go-1}
\Aue{Zatsman, I.\,M., O.\,Yu.~Inkova, M.\,G.~Kruzhkov, and N.\,A.~Pop\-ko\-va.} 2016. 
Predstavlenie kross-yazykovykh znaniy o~konnektorakh v~nadkorpusnykh bazakh dannykh 
[Representation of cross-lingual knowledge about connectors in suprocorpora databases]. 
\textit{Informatika i~ee Primeneniya~--- Inform. Appl.} 10(1):106--118.
\bibitem{17-go-1}
\Aue{Zatsman, I., O.~Inkova, and V.~Nuriev.} 2017. The construction of classification schemes: 
Methods and technologies of expert formation. \textit{Automatic Documentation Mathematical Linguistics} 
51(1):27--41.
\bibitem{18-go-1}
\Aue{Baker, P., A. Hardie, and T.~McEnery.} 2006. \textit{A~glossary of corpus linguistics}. 
Edinburgh, U.K.: Edinburgh University Press. 187~p.
\bibitem{19-go-1}
\Aue{Zatsman, I.\,M., V.\,V.~Kosarik, and O.\,A.~Kurchavova.} 2008. Zadachi predstavleniya 
lichnostnykh i~kollektivnykh kontseptov v~tsifrovoy srede [Representation of individual and 
collective concepts in digital medium]. \textit{Informatika i~ee Primeneniya~--- Inform. Appl.} 
2(3):54--69.
\bibitem{20-go-1}
\Aue{Zatsman, I., M.~Kruzhkov, and E.~Loshchilova.} 2019. Metody i~sredstva informatiki dlya 
opisaniya struktury neodnoslovnykh konnektorov [Methods and means of informatics for 
multiword connectives structure description]. \textit{Struktura konnektorov i~metody ee 
opisaniya} [Connectives structure and methods of its description]. Ed. O.\,Yu.~Inkova. Moscow: 
TORUS PRESS. 205--230.
\bibitem{21-go-1}
\Aue{Zatsman, I.\,M.} 2019. Tselenapravlennoe razvitie sistem lingvisticheskikh znaniy: 
vyyavlenie i~zapolnenie lakun [Goal-oriented development of linguistic knowledge systems: 
Identifying and filling of lacunae]. \textit{Informatika i~ee Primeneniya~--- Inform. Appl.} 
13(1):91--98.
\bibitem{22-go-1}
\Aue{Zatsman, I.} 2020. Finding and filling lacunas in linguistic typologies. \textit{15th Forum 
(International) on Knowledge Asset Dynamics Proceedings}. Matera, Italy: Institute of Knowledge 
Asset Management. 780--793.
\bibitem{23-go-1}
\Aue{Zatsman, I.} 2020. Three-dimensional encoding of emerging meanings in AI-systems. 
\textit{21st European Conference on Knowledge Management Proceedings}. Reading, MA: 
Academic Publishing International Ltd. 878--887.
\bibitem{24-go-1}
\Aue{Zatsman, I., and A.~Khakimova.} 2021. New knowledge discovery for creating 
terminological profiles of diseases. \textit{22nd European Conference on Knowledge 
Management Proceedings}. Reading, MA: Academic Publishing International Ltd. 837--846.
\bibitem{25-go-1}
\Aue{Zatsman, I}. 2021. A~model of goal-oriented knowledge discovery based on 
human--computer symbiosis. \textit{16th Forum (International) on Knowledge Asset Dynamics 
Proceedings}. Rome: Arts for Business Institute.
 297--312.
\bibitem{26-go-1}
\Aue{Zatsman, I.\,M.} 2021. Formy predstavleniya novogo znaniya, izvlechennogo iz tekstov 
[Forms representing new knowledge discovered in texts]. \textit{Informatika i~ee 
Primeneniya~--- Inform. Appl.} 15(3):83--90.
\end{thebibliography}

 }
 }

\end{multicols}

\vspace*{-3pt}

\hfill{\small\textit{Received October 15, 2021}}

%\pagebreak

%\vspace*{-24pt}



\Contr

\noindent
\textbf{Goncharov Alexander A.} (b.\ 1994)~--- junior scientist, Institute of Informatics Problems, 
Federal Research Center ``Computer Science and Control'' of the Russian Academy of Sciences, 
44-2~Vavilov Str., Moscow 119333, Russian Federation; \mbox{a.gonch48@gmail.com}

\vspace*{3pt}

\noindent
\textbf{Zatsman Igor M.} (b.\ 1952)~--- Doctor of Science in technology, Head of Department, 
Institute of Informatics Problems, Federal Research Center ``Computer Science and Control'' of 
the Russian Academy of Sciences, 44-2~Vavilov Str., Moscow 119333, Russian Federation; 
\mbox{izatsman@yandex.ru}

\vspace*{3pt}

\noindent
\textbf{Kruzhkov Mikhail G.} (b.\ 1975)~--- senior scientist, Institute of Informatics Problems, 
Federal Research Center "Computer Science and Control" of the Russian Academy of Sciences, 
44-2 Vavilov Str., Moscow 119333, Russian Federation; \mbox{magnit75@yandex.ru}

\vspace*{3pt}

\noindent
\textbf{Loshchilova Elena Yu.} (b.\ 1960)~--- principal specialist, Institute of Informatics Problems, 
Federal Research Center ``Computer Science and Control'' of the Russian Academy of Sciences, 
44-2~Vavilov Str., Moscow 119333, Russian Federation; \mbox{lena0911@mail.ru}


\label{end\stat}

\renewcommand{\bibname}{\protect\rm Литература} 