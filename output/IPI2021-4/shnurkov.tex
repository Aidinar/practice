\def\stat{shnurkov}

\def\tit{СОЗДАНИЕ СТОХАСТИЧЕСКОЙ ДИНАМИЧЕСКОЙ ОДНОСЕКТОРНОЙ ЭКОНОМИЧЕСКОЙ 
МОДЕЛИ С~ДИСКРЕТНЫМ ВРЕМЕНЕМ И~АНАЛИЗ СООТВЕТСТВУЮЩЕЙ~ЗАДАЧИ ОПТИМАЛЬНОГО 
УПРАВЛЕНИЯ}

\def\titkol{Создание стохастической динамической односекторной экономической модели с~дискретным 
временем} % и~анализ соответствующей задачи оптимального управления}

\def\aut{П.\,В.~Шнурков$^1$}

\def\autkol{П.\,В.~Шнурков}

\titel{\tit}{\aut}{\autkol}{\titkol}

\index{Шнурков П.\,В.}
\index{Shnurkov P.\,V.}


%{\renewcommand{\thefootnote}{\fnsymbol{footnote}} \footnotetext[1]
%{Работа выполнена при частичной поддержке РФФИ (проект 19-07-00187-A).}}


\renewcommand{\thefootnote}{\arabic{footnote}}
\footnotetext[1]{Национальный исследовательский университет <<Высшая школа экономики>>, 
\mbox{pshnurkov@hse.ru}}

\vspace*{6pt}


      \Abst{Работа посвящена созданию стохастической динамической модели оптимального 
управ\-ле\-ния с~дискретным временем в~рамках односекторной экономической сис\-те\-мы. За основу 
принята классическая детерминированная динамическая модель экономической сис\-те\-мы, 
в~которой производится один универсальный продукт. Этот продукт делится на инвестиционную 
и~потребительскую со\-став\-ля\-ющие. Управ\-ле\-ние сис\-те\-мой заключается в~определении 
соотношения между этими 
      со\-став\-ля\-ющи\-ми. В~на\-сто\-ящей работе предполагается, что основные параметры 
сис\-те\-мы зависят от некоторого случайного фактора, который характеризует влияние внешней 
среды. Указанный фактор описывается однородной цепью Маркова с~конечным множеством 
состояний и~заданной мат\-ри\-цей вероятностей перехода. В~работе построена стохастическая 
модель эволюции рассматриваемой системы, которая представляет собой двумерный марковский 
процесс с~дискретным временем. По своему экономическому содержанию первая компонента 
этого процесса представляет собой удельный капитал, а~вторая~--- состояние внешнего 
случайного фактора. Параметр управления, или решение, в~каждый момент времени представляет 
собой долю произведенного удельного продукта, направляемую на инвестирование. Описано 
рекуррентное задание стоимостного аддитивного показателя эффективности управления. 
Теоретическую основу решения поставленной задачи оптимального управления составляет метод 
динамического программирования. Получена система функциональных уравнений 
Беллмана, решением которой является оптимальная стратегия управ\-ле\-ния.} 
   
   \KW{задача оптимального управления с~дискретным временем; стохастическая динамическая 
односекторная экономическая модель; управ\-ля\-емая двумерная цепь Маркова; метод динамического 
программирования для задачи управ\-ле\-ния с~дискретным временем; уравнения Беллмана}
   
\DOI{10.14357/19922264210405}
  
%\vspace*{9pt}


\vskip 10pt plus 9pt minus 6pt

\thispagestyle{headings}

\begin{multicols}{2}

\label{st\stat}

\section{Введение}

В литературе по математической экономике хорошо известна односекторная 
динамическая детерминированная модель~[1, 2]. В~рамках этой модели 
предполагается, что в~соответствующей экономической системе производится 
один универсальный продукт, из\-ме\-ря\-емый в~условных денежных единицах. 
Произведенный продукт делится на две части: инвестиции и~по\-треб\-ле\-ние. 
В~качестве основных параметров модели выступают общий объем 
произведенного продукта, объем производственных фондов (капитал), объем 
трудовых ресурсов (рабочая сила), объем инвестиций и~объем потребления. Все 
эти параметры имеют динамический характер, т.\,е.\ зависят от времени. Обычно 
рассматриваются удельные варианты указанных параметров, определяемые как 
отношение их общего объема к объему трудовых ресурсов.

     В стандартной модели предполагается, что состоянием исследуемой системы 
является удельный объем производственных фондов, который часто называют 
фондовооруженностью, или удельным капиталом. Роль параметра управления 
играет удельное по\-треб\-ле\-ние. Удельный объем производства, или 
производительность труда, определяется как заданная функция от удельного 
капитала. Динамика со\-сто\-яния сис\-те\-мы описывается дифференциальным 
уравнением, в~котором отражается выбытие основных фондов за счет их 
физического или морального устаревания и~их обновление за счет 
инвестирования в~сис\-те\-му час\-ти произведенного продукта. Таким образом, 
возникает динамический управ\-ля\-емый процесс в~смыс\-ле классической тео\-рии 
управ\-ле\-ния~[3, 4].
     
     Для постановки задачи оптимального управления вводится целевой 
функционал, или показатель эффективности. В~стандартном варианте этот 
функционал имеет интегральный характер и~по своему экономическому 
содержанию пред\-став\-ля\-ет собой некую накопленную по\-лез\-ность, которая является 
заданной функцией от удельного по\-треб\-ле\-ния. 
{\looseness=1

}
     
     Полученная задача оптимального управления с~непрерывным временем 
хорошо изучена. Подробное изложение этой теории приведено в~уже упомянутых 
классических изданиях~[1, 2], а~также в~современных монографиях~[5--7].
     
     Гораздо менее известна соответствующая задача оптимального управления 
с~дискретным временем. Основное динамическое соотношение для со\-сто\-яния 
системы (удельного капитала) приведено в~\cite{1-sh}, по математической форме 
оно пред\-став\-ля\-ет собой раз\-ност\-ное урав\-не\-ние. Там же отдельно приведено 
пред\-став\-ле\-ние для целевого функционала. Отметим, что в~обоих случаях это 
сделано только в~подстрочных примечаниях и~никакого математического анализа 
со\-от\-вет\-ст\-ву\-ющей задачи оптимального управ\-ле\-ния не проводилось.
     
     В работе~\cite{8-sh} была рассмотрена задача оптимального управления 
в~детерминированной динамической односекторной экономической модели 
с~дискретным временем. Для решения поставленной задачи был использован 
метод динамического программирования Беллмана. По аналогии с~исследованием 
классической общей задачи оптимального управ\-ле\-ния с~дискретным 
временем~\cite[гл.~6]{3-sh} получено уравнение Беллмана. Разработан и~подробно 
описан численный алгоритм решения этого уравнения, который позволяет 
определить оптимальную стратегию управления.
     
     Настоящее исследование посвящено созданию стохастической динамической 
односекторной экономической модели с~дискретным временем. В~качестве 
исходной конструкции взята упомянутая\linebreak выше детерминированная динамическая 
односекторная экономическая модель. Предполагается, что в~новой модели 
основные па\-ра\-мет\-ры сис\-те\-мы зависят от некоторого внешнего случайного 
фактора, который описывается однородной цепью \mbox{Маркова}. Получены основные 
динамические соотношения, опи\-сы\-ва\-ющие поведение сис\-те\-мы. Определено 
понятие ад\-ди\-тив\-но\-го стоимостного функционала, связанного с~по\-стро\-ен\-ным 
случайным процессом. По\-став\-ле\-на задача оптимального управ\-ле\-ния. Выведена 
сис\-те\-ма уравнений Беллмана, решением которой является оптимальная стратегия 
управ\-ле\-ния.
{\looseness=1

}
     
     Начнем с~краткого описания исходной детерминированной динамической 
односекторной модели экономической сис\-те\-мы. При этом описании будем 
следовать классической монографии~\cite{1-sh} с~некоторыми изменениями, 
принятыми в~\cite{8-sh}.
     
\section{Описание детерминированной односекторной модели 
в~экономике}

\vspace*{-4pt}

     Рассмотрим динамическую экономическую сис\-те\-му, при описании которой 
параметр времени считается дискретным и~принимает конечное чис\-ло значений: 
$n\hm\in \{0,1,2,\ldots , N\}$, $N\hm<\infty$. Таким образом, эволюция сис\-те\-мы 
происходит на конечном интервале времени.
     
     Введем обозначения для основных характеристик модели: $y_n$~--- 
удельный объем произведенного продукта (производительность труда);  $k_n$~--- 
удельный объем производственных фондов (фондовооруженность, или удельный 
капитал); $c_n$~--- удельный объем потребления (потребительский фонд), $n\hm\in 
\{0,1,2,\ldots , N\}$.
     
     Будем предполагать, что $y_h\hm= f(k_n)$, $n\hm\in \{ 0,1,2,\ldots , N\}$, где 
$f(\cdot)$~--- заданная функция. В~стандартной детерминированной модели 
дополнительно предполагается, что $f(\cdot)$~--- неотрицательная, неубывающая 
и~выпуклая вверх функция, $f(0)\hm=0$. Эти аналитические свойства имеют 
реальное экономическое содержание~[1, 2].
     
     Поскольку произведенный продукт делится только на инвестиции 
и~потребление, величина $y_n\hm- c_n\hm= f(k_n)\hm - c_n$ представляет собой 
удельный объем инвестиций, $n\hm\in \{0,1,2,\ldots , N\}$.
     
     Обозначим через $\rho_n\hm= (f(k_n)\hm- c_n)/f(k_n)$ долю удельного 
продукта, направляемую на инвестирование. Соответственно, $1\hm- \rho_n\hm= 
c_n/f(k_n)$ есть доля удельного продукта, направляемого на потребление, $n\hm\in 
\{0,1,2,\ldots , N\}$. По определению, $\rho_n \hm\in U\hm= [0,1]$, $n\hm\in 
\{0,1,2,\ldots , N\}$.
     
     В детерминированной модели, рас\-смат\-ри\-ва\-емой в~работе~\cite{8-sh}, 
предполагается, что по\-сле\-до\-ва\-тель\-ность значений удельного капитала $k\hm= (k_0,$\linebreak 
$k_1, \ldots , k_{N-1}, k_N)$ описывает состояние сис\-те\-мы, а~по\-сле\-до\-ва\-тель\-ность 
значений па\-ра\-мет\-ра $\rho \hm= (\rho_0, \rho_1, \ldots , \rho_{N-1})$~--- управ\-ле\-ния, 
или решения, при\-ни\-ма\-емые в~соответствующие моменты вре\-мени.
{\looseness=-1

}
     
     Динамика модели, т.\,е.\ изменение состояний под воздействием 
принимаемых решений, описывается сле\-ду\-ющим соотношением:

\vspace*{-6pt}

\noindent
     \begin{multline}
     k_{n+1}=\varphi_n(k_n, \rho_n) = (1-\lambda_n) k_n +\rho_n f(k_n)\,,\\ 
     n\in \{0,1,2,\ldots, N-1\}.
     \label{e1-sh}
     \end{multline}
     
     \vspace*{-2pt}
     
     Коэффициент~$\lambda_n$, $0\hm< \lambda_n \hm <1$, в~правой 
     части~(\ref{e1-sh}) характеризует выбывание основных фондов системы из 
процесса производства за счет физического и~морального старения, все значения 
коэффици-\linebreak\vspace*{-12pt}

\pagebreak

\noindent
ента~$\lambda_n$, $n\hm\in \{0,1,2,\ldots , N\hm-1\}$, предполагаются 
заданными.
     
     Аналитическое представление для показателя эффективности управления 
сформировано по аналогии с~соответствующей моделью~\cite{1-sh}. Подробные 
пояснения к его экономическому содержанию приведены в~\cite{8-sh}.
     
     Теперь можно сформулировать задачу оптимального управления в~исходной 
детерминированной модели как экстремальную задачу с~ограничениями:
     \begin{equation}
     \left.
     \begin{array}{rl}
          \!\!\!\!I(k,\rho)&=\sum\limits_{n=0}^{N-1} \left( \fr{1}{1+\gamma}\right)^n V\left( (1-
\rho_n)f(k_n)\right) +{}\\[3pt]
&\hspace*{31.5mm}{}+\psi(k_N)\to  \max\,;\\[3pt]
           \!\!  \!\!  k_{n+1}&= \varphi_n(k_n,\rho_n)=(1-\lambda_n)k_n +\rho_n f(k_n)\,,\\[3pt]
     &\hspace*{28.5mm}n=0,1,\ldots , N-1\,;\\[3pt]
              \!\! \!\!k_0&=a_0\,;\\[3pt]
              \!\! \!\!\rho_n&\in U =[0,1]\,,\ n=0,1,\ldots , N-1\,.
     \end{array}
\!     \right\}\!
\label{e2-sh}
     \end{equation}
     
     Экстремальная задача с~дискретным временем~(\ref{e2-sh}) 
известна в~теории оптимального управ\-ле\-ния. Она представляет собой частный 
случай общей задачи, определенной на заданном конечном интервале времени, со 
смешанным целевым функционалом и~закрепленным левым концом 
траектории~\cite{3-sh}. Для решения такой задачи можно использовать метод 
динамического программирования Беллмана. В~работе~\cite{8-sh} получена 
система уравнений Беллмана для этой задачи и~разработан алгоритм численного 
решения указанной сис\-те\-мы и~определения оптимальной стратегии управ\-ле\-ния.

\vspace*{-4pt}

\section{Разработка стохастической динамической модели 
односекторной экономической~системы}

\vspace*{-2pt}

     Возьмем за основу динамическую односекторную экономическую модель 
с~дискретным временем, описанную в~предыдущем разделе. Предположим, что 
некоторые характеристики модели\linebreak \mbox{зависят} от случайного фактора, 
ха\-рак\-те\-ри\-зу\-юще\-го со\-сто\-яние внеш\-ней среды (или экономической конъюнктуры 
рынка). Случайный фактор\linebreak описывается \mbox{цепью} Маркова $\left\{ 
\eta_n\right\}^N_{n=0}\hm= \left\{\eta_0, \eta_1, \ldots\right.$\linebreak $\left.\ldots , \eta_n, \ldots , \eta_N\right\}$, 
заданной на вероятностном пространстве  $(\Omega, \mathcal{A}, {\sf P})$ 
и~конечном периоде времени $\{ 0,1,\ldots , N\}$, $N\hm< \infty$. В~дальнейшем 
будем для крат\-кости использовать обозначение $\left\{ \eta_n\right\}$. Классическая 
тео\-рия марковских цепей изложена, например, в~фундаментальном 
     издании~\cite{9-sh}.
     
     Предположим, что цепь Маркова $\left\{ \eta_n\right\}$ принимает значения 
в~конечном множестве состояний $D\hm= \{1,2,\ldots , r\}$, $r\hm<\infty$, и~заданы 
вероятности перехода $p_{ij}\hm= {\sf P}\left( \eta_{n+1}=j \vert \eta_n=i\right)$, 
$i,j\hm= 1,2,\ldots , r$.
     
     Каждое значение $i\hm\in D$ соответствует определенному состоянию 
внешней среды или всей макроэкономической системы, с~которой связана данная 
односекторная динамическая система.
     
     Введем следующие измеримые вещественные функции: 
     \begin{gather*}
     \lambda_i=\lambda(i): D\to [0,1],\quad i\in D\,;\\
     f(x,i): X\times D\to R\,, \quad x\in X\,,\quad i\in D\,,
     \end{gather*}
где $X\subseteq [0,\infty)$~--- множество возможных значений удельного капитала. 
Функции $\lambda(i)$ и~$f(x,i)$ предполагаются заданными. Данные функции 
описывают зависимость основных характеристик от случайного фактора~$\eta_n$ 
и~удельного капитала~$k_n$. Конкретное экономическое содержание этих 
функций заключается в~следующем: $\lambda\hm= \lambda(\eta_n)$~--- коэффициент 
выбывания основных фондов односекторной динамической системы; $f(k_n, 
\eta_n)$~--- удельный объем производства (производительность труда) 
односекторной динамической сис\-те\-мы, определяемый при любых фиксированных 
значениях $k_n$ и~$\eta_n$ в~момент времени $n\hm\in \{0,1,\ldots , N\}$.
     
     В частности, можно предполагать, что функция $f(x,i)$, $x\hm\in X$, $i\hm\in 
D$, имеет стандартную степенную структуру, известную в~теории математической 
экономики: 
     $$
     f(x,i)=A(i) x^{\alpha(i)}\,,
     $$
     где $A(i)$, $0\hm< A(i)\hm <\infty$, и~$\alpha(i)$, $0\hm< \alpha(i)\hm< 1$, $i\hm\in D$,~--- 
заданные функции. Тогда для произвольных значений $k_n$ и~$\eta_n$ величина 
$y_n\hm= A(\eta_n) k_n^{a(\eta_n)}$ определяет случайное значение удельного 
объема производства в~момент времени~$n$.
      
     В дальнейшем будем предполагать, что пара стохастических объектов 
($k_n,\eta_n$) описывает со\-сто\-яние ис\-сле\-ду\-емой сис\-те\-мы в~момент времени~$n$,  
$n\hm\in \{0,1,\ldots , N\hm-1\}$.
     
     Перейдем к описанию управления в~системе. Будем предполагать, что 
в~любой момент времени~$n$, $n\hm\in \{0,1,\ldots , N\hm-1\}$, принимается 
некоторое решение~$\rho^{(n)}(\eta_n)$, также зависящее от значения случайного 
фактора~$\eta_n$. Обозначим $\rho_i^{(n)} \hm= \rho^{(n)}(i)$, $i\hm\in D$, $D\hm= 
\{1,2,\ldots , r\}$. Таким образом, если в~момент времени~$n$ состояние 
случайного фактора равно~$i$, то параметр управ\-ле\-ния равен~$\rho_i^{(n)}$. 
Рассмотрим $\rho^{(n)} \hm= \left( \rho_1^{(n)}, \rho_2^{(n)}, \ldots , 
\rho_r^{(n)}\right)$~--- векторный параметр управ\-ле\-ния в~момент~$n$. По своему 
экономическому содержанию параметр~$\rho_i^{(n)}$ будет соответствовать 
исходной детерминированной модели, т.\,е.\ задавать распределение 
произведенного продукта на инвестиционную и~потребительскую час\-ти. Таким 
образом, в~каждый момент времени~$n$ и~при фиксированном состоянии 
внешней среды $\eta_n\hm= i$ параметр~$\rho_i^{(n)}$ определяет принимаемое 
решение по разделению произведенного продукта на инвестиционную 
и~потребительскую со\-став\-ля\-ющие. В~соответствии с~содержанием этого 
па\-ра\-мет\-ра будем предполагать, что $\rho_i^{(n)}\hm\in U_i\hm= [0,1]$, $i\hm\in D$, 
$n\hm\in \{0,1,\ldots , N\hm-1\}$.
     
     Обозначим через~$i_n$ удельные инвестиции, а через $c_n$~--- удельное 
потребление в~момент~$n$. Тогда в~соответствии с~принятыми 
предположениями
     \begin{align*}
     i_n&=\rho^{(n)}\left( \eta_n\right) f(k_n, \eta_n)\,;\\
     c_n&=\left(1-\rho^{(n)}\left( \eta_n\right)\right) f(k_n,\eta_n)\,;\\
     &c_n+i_n=f(k_n,\eta_n).
     \end{align*}
     
     По аналогии с~детерминированной моделью будем предполагать, что 
основное динамическое соотношение, описывающее поведение экономической 
системы, имеет вид:

\vspace*{-6pt}

\noindent
     \begin{multline}
     k_{n+1}=\left[ 1-\lambda(\eta_n)\right] k_n+\rho^{(n)}(\eta_n) f(k_n,\eta_n)\,,\\
     n=0,1,\ldots , N-1\,,
     \label{e6-sh}
     \end{multline}
     
     \vspace*{-2pt}
     
     \noindent
где $[1-\lambda(\eta_n)]k_n$~--- часть удельного капитала, оставшаяся в~сис\-те\-ме 
после очередного выбывания в~момент времени~$n$; $\rho^{(n)}(\eta_n) 
f(k_n,\eta_n)$~--- дополнительные удельные инвестиции в~производство, 
\mbox{поступающие} в~момент времени~$n$.
     
     Уточним динамическую сущность соотношения~(\ref{e6-sh}). Со\-глас\-но 
сделанному выше предположению, со\-сто\-яние рас\-смат\-ри\-ва\-емой сис\-те\-мы 
\mbox{описывается} парой переменных ($k_n,\eta_n$), где~$k_n$~--- удельный капитал; 
$\eta_n$~--- со\-сто\-яние внеш\-ней среды. Если зафиксировать $\eta_n\hm= i$, 
$k_n\hm= x$, то из~(\ref{e6-sh}) получаем:

\noindent
     \begin{equation}
     k_{n+1}=\varphi\left( x,i,\rho_i^{(n)}\right) =[1-\lambda(i)]x+\rho_i^{(n)} f(x,i).
     \label{e7-sh}
     \end{equation}
     
     Таким образом, при фиксированном состоянии в~момент времени~$n$ 
и~заданном решении $\rho_i^{(n)}\hm= \rho^{(n)}(i)$ однозначно определяется 
значение~$k_{n+1}$.
     
     Значение фактора, характеризующего внешнюю среду~$\eta_{n+1}$ 
определяется как реализация случайной величины с~заданным условным 
распределением

\vspace*{-6pt}

\noindent
     \begin{multline}
     {\sf P}\left( \eta_{n+1}=j\vert \eta_n=i\right) =p_{ij}\,,\\
      i,j=1,2,\ldots , r\,,\quad
     n=0,1,2,\ldots , N-1\,.
     \label{e8-sh}
     \end{multline}
     
     \vspace*{-2pt}
     
     
   \noindent
     Отсюда следует, что соотношения~(\ref{e6-sh})--(\ref{e8-sh}) пол\-ностью 
определяют стохастическую динамику \mbox{изменения} со\-сто\-яния сис\-те\-мы при 
заданных значениях па\-ра\-мет\-ров управ\-ле\-ния~$\rho_i^{(n)}$, $n\hm\in \{0,1,2,\ldots$\linebreak $\ldots , 
N\hm-1\}$. По\-сле\-до\-ва\-тель\-ность случайных величин $(k_n,\eta_n)$, $n\hm= 
0,1,2,\ldots , N\hm-1, N$, образует двумерную управ\-ля\-емую цепь Маркова.
     
     В дальнейшем будем предполагать, что параметр управления~$\rho_i^{(n)}$ 
не зависит от~$n$, $\rho_i^{(n)}\hm=\rho_i$, $n\hm=0,1,2,\ldots , N\hm-1$, $i\hm=1,2,\ldots , r$. Тогда 
решение задачи оптимального управления представляет собой вектор $\rho\hm= 
(\rho_1, \rho_2, \ldots , \rho_r)$, доставляющий максимальное значение некоторому 
заданному стоимостному показателю эффективности.

\section{Определение аддитивного стоимостного показателя 
эффективности и~постановка задачи оптимального управления}

     Зададим набор конечных измеримых функций~$V_i(y): [0,\infty)\hm\to R$, 
$i\hm\in D$, $D\hm= \{1,2,\ldots , r\}$. По аналогии с~детерминированной моделью 
эти функции будем называть функциями по\-лез\-ности. Каждая функция~$V_i(y)$  
из этого набора соответствует состоянию~$i$ цепи Маркова $\left\{ \eta_n\right\}$, 
т\,е.\ со\-сто\-янию случайного фактора. Если в~момент времени~$n$ выполняется 
условие $\eta_n\hm=i$ и~удельное по\-треб\-ле\-ние принимает значение~$y: c_n=y$, то 
величина~$V_i(y)$ определяет некоторый услов\-ный доход, связанный с~данным моментом времени. 
Зададим также конечную измеримую функцию $\psi(x): [0,\infty)\hm\to R$. Эта 
функция будет характеризовать вклад в~суммарный доход, полученный за все 
время эволюции сис\-те\-мы $\{0,1,\ldots , N\}$, который вносит достигнутый 
в~конечный момент времени $N$ уровень технологического развития 
сис\-те\-мы~$k_N$. Конкретно, если удельный капитал~$k_N$ принимает 
значение~$x$, то величина~$\psi(x)$ определяет этот дополнительный вклад 
в~суммарный доход.
     
     Обозначим через $\tilde{S}_n(k_n,\eta_n)$ случайный суммарный доход, 
полученный на интервале времени $\{n, n+1, \ldots , N\}$, $n\hm= 0,1,2,\ldots , N\hm-
1$. Этот доход зависит от со\-сто\-яния сис\-те\-мы в~момент~$n$, т.\,е.\ от пары 
$(k_n,\eta_n)$. Теперь зафиксируем со\-сто\-яние сис\-те\-мы $(k_n\hm=x,\ \eta_n\hm=i)$. 
Тогда через~$S_n(x,i)$ будем обозначать со\-от\-вет\-ст\-ву\-ющее услов\-ное 
математическое ожидание случайного дохода: 
     \begin{multline*}
     S_n(x,i)= {\sf E}\left[ \tilde{S}_n(k_n,\eta_n)\vert k_n=x, \eta_n=i\right]\,,\\
     n=0,1,2,\ldots , N-1\,.
     \end{multline*}
     
     Согласно сделанному выше предположению, величина~$V_i(y)$ 
представляет собой условное математическое ожидание дохода, полученного 
в~произвольный момент времени $n\hm\in \{0,1,\ldots , N\hm-1\}$ при условиях 
$c_n\hm=y$, $\eta_n\hm=i$. Поскольку при фиксированном управ\-ле\-нии~$\rho_i$ 
величина удельного по\-треб\-ле\-ния выражается формулой $c_n\hm= (1\hm- 
\rho_i)f(k_n,i)$, то мож\-но пред\-ста\-вить условное математическое ожидание дохода, 
полученного \mbox{в~произвольный} момент времени~$n$ при условиях $k_n\hm= x$, 
$\eta_n\hm=i$ и~принятом решении~$\rho_i$, в~виде $V_i((1\hm- \rho_i)f(x.i))$. 
Отсюда следует рекуррентная формула для условных математических ожиданий 
дохода:
     \begin{multline}
     {\sf E}\left[ \tilde{S}_n(k_n,\eta_n)\vert k_n=x, \eta_n=i\right]= V_i ((1-\rho_i) 
f(x,i))+{}\\
     {}+ {\sf E} \left[ \tilde{S}_{n+1}(k_{n+1},\eta_{n+1})\vert 
k_n=x,\eta_n=i\right],\\
 n=0,1,\ldots , N-1\,.
     \label{e9-sh}
     \end{multline}
     
     Воспользовавшись свойствами условных математических ожиданий 
и~соотношениями~(\ref{e6-sh})--(\ref{e8-sh}), получим:
     \begin{multline}
     {\sf E} \left[ \tilde{S}_{n+1} \left( k_{n+1},\eta_{n+1}\right )\vert 
k_n=x,\eta_n=i\right]={}\\
     {}=
     \sum\limits^r_{j=1} {\sf E} \left[ \tilde{S}_{n+1} \left( k_{n+1},\eta_{n+1}\right) 
\vert k_n=x,\ \eta_n=i,\right.\\
\left. \eta_{n+1}=j
     \vphantom{\tilde{S}_{n+1}}
     \right] {\sf P} \left( \eta_{n+1}=j\vert \eta_n=i\right) 
={}\\
     {}=\sum\limits^r_{j=1} {\sf E} \left[ \tilde{S}_{n+1} 
     \left( k_{n+1},\eta_{n+1}\right) \vert k_n=x,\right.\\ 
k_{n+1}=\varphi(x,i,\rho_i)=(1-\lambda_i)x+\rho_i f(x,i),\ \eta_n=i,\\
\left.  \eta_{n+1}=j
     \vphantom{\tilde{S}_{n+1}}
     \right] p_{ij}=\sum\limits^r_{j=1} S_{n+1}(\varphi(x,i,\rho_i),j) p_{ij}\,.
     \label{e10-sh}
     \end{multline}
     
     Из (\ref{e9-sh}) и~(\ref{e10-sh}) следует рекуррентное соотношение для 
условных математических ожиданий доходов:
     \begin{multline}
     S_n(x,i)={}\\
     {}=V_i\left( (1-\rho_i)f(x,i)\right)+\sum\limits^r_{j=1} p_{ij} S_{n+1} \left( 
\varphi\left(x,i,\rho_i\right),j\right)\,,\\
     n=0,1,\ldots , N-1\,.
     \label{e11-sh}
     \end{multline}
     
     В конечный момент времени $n\hm=N$ решение об управлении не 
принимается. Вклад в~суммарный доход определяется функцией~$\psi(k_n)$. 
Таким образом,
     \begin{equation}
     S_N(x,i)=S_N(x)=\psi(x)\,,\enskip x\in X\,.
     \label{e12-sh}
     \end{equation}
     
     Аддитивный стоимостный функционал $S_n(x,i)$, $x\hm\in X$, $i\hm\in D\hm= 
\{1,2,\ldots , r\}$, $n\hm=0,1,2,\ldots , N$,\linebreak пол\-ностью определяется рекуррентными 
соотношениями~(\ref{e11-sh}) и~(\ref{e12-sh}). Поскольку основная задача 
оптимального управ\-ле\-ния связана 
со всем периодом эволюции сис\-те\-мы $\{0,1,\ldots , N\}$, естественно выбрать для нее
в~качестве показателя эф\-фек\-тив\-ности управ\-ле\-ния суммарный средний 
\mbox{доход}~$S_0(x_0,i_0)$, за\-ви\-ся\-щий от начальных условий $k_0\hm= x_0$ 
и~$\eta_0\hm= i_0$. Оптимальная стратегия управ\-ле\-ния $\rho^*\hm= \left( \rho_1^*, 
\rho_2^*, \ldots , \rho^*_r\right)$ также будет зависеть от указанных начальных 
условий. Такая осо\-бен\-ность представляется естественной, поскольку 
в~марковской стохастической модели любые вероятностные характеристики на 
конечном интервале времени зависят от начального со\-сто\-яния про\-цесса.


\section{Уравнения Беллмана и~теоретическое решение задачи 
оптимального управления}

     Применим для решения сформулированной задачи метод динамического 
программирования. Согласно общей схеме этого метода~\cite{10-sh, 11-sh}, 
рассмотрим семейство задач оптимизации, связанных\linebreak с~интервалами времени 
$\{n, n+1, \ldots , N-1, N\}$, $n\hm= 0,1,\ldots , N\hm-1$. Рассмотрим произвольные 
фиксированные значения состояний сис\-те\-мы $k_n\hm= x_n\hm\in X$, $\eta_n\hm= 
i_n\hm\in D$, $n\hm= 0,1,2,\ldots , N\hm-1, N$, и~\mbox{управ\-ле\-ний} $\rho\hm= (\rho_1,\rho_2, 
\ldots , \rho_r)$. Роль целевого показателя в~задаче с~номером~$n$ будет играть 
определенный в~предыду\-щем разделе аддитивный стоимостный функционал 
$S_n(x,i)\hm= S_n(x,i;\rho)$, в~качестве ограничений рас\-смат\-ри\-ва\-ют\-ся 
соотношения, которым удовле\-тво\-ря\-ют основные па\-ра\-мет\-ры модели:
     \begin{equation}
     \left.
     \begin{array}{l}
                  \!\! \!\!S_n\left( x_n,i_n; \rho\right)\to \max\,;
    \\[6pt]
              \!\! \!\!x_{m+1} = \varphi\left( x_m,i_m;\rho\right)={}\\[6pt]
\hspace*{13mm}\left( 1-\lambda_m\right) x_m+\rho_m 
f(x_m,i_m)\,,\\[6pt] 
             \hspace*{15mm}i_m\in D\,,\enskip m=n,n+1, \ldots , N-1\,;
\\[3pt]
               \!\!\!\!(x_n,i_n) \mbox{---\ фиксированное\ состояние}\\[3pt]  
              \!\! \!\!\mbox{в\ очередной\ <<начальный>>\ момент\ времени~$n$;}
\\[6pt]
              \!\! \!\!\rho_i\in U=[0,1]\,,\quad i\in D=\{1,2,\ldots , r\}.
\end{array}\!\!\right\}\!\!\!
\label{e13-sh}
\end{equation}
     
     Параметр~$n$ пробегает значения $\{0,1,\ldots , N-1\}$. Заметим, что при 
$n\hm=0$ задача~(\ref{e13-sh}) совпадает с~основной задачей 
оптимального управ\-ле\-ния в~рас\-смат\-ри\-ва\-емой модели. Как 
     известно~\cite{10-sh, 11-sh}, функция Беллма\-на совпадает со значениями 
целевой функции на оптимальной стратегии управ\-ле\-ния. Обозначим эту функцию, 
со\-от\-вет\-ст\-ву\-ющую задаче~(\ref{e13-sh}), через $B_n(x,i)$, $x\hm\in X$, 
$i\hm\in D$, $n\hm\in \{0,1,\ldots , N\hm-1\}$ Сформулируем основное утверж\-де\-ние, 
связанное с~по\-став\-лен\-ной задачей управ\-ле\-ния.
     
     \smallskip
     
     \noindent
     \textbf{Теорема 1.}\ \textit{В~рассматриваемой модели управления функции 
Беллмана $B_n (x,i)$, $n\hm\in \{ 0,1,\ldots , N\hm-1, N\}$, удовлетворяют 
следующей системе функциональных уравнений, называемых уравнениями 
Беллмана}:

\noindent
     \begin{multline}
     B_n(x,i) =\max\limits_{\rho_i\in U} \Bigg[ 
     \vphantom{\sum\limits^r_{j=1}}
     V_i\left( (1-\rho_i)f(x,i)\right) + {}\\ %\right.\\
%\left.   
  {}+
\sum\limits^r_{j=1} p_{ij} B_{n+1} (\varphi(x,i,\rho_i),j)\Bigg]\,,\\ % right]\,,\\
     n=0,1,2,\ldots , N-1\,;
     \label{e17-sh}
     \end{multline}
     
     \vspace*{-12pt}
     
     \noindent
     \begin{multline}
     B_N(x,i) =B_N(x)=\psi(x)\,,\\ 
     i\in D=\{1,2,\ldots , r\}.
     \label{e18-sh}
     \end{multline}
     
     \textit{Если вектор управлений $\rho^*\hm= \left\{ \rho_1^*, \rho_2^*, \ldots , 
\rho_r^*\right\}$ удов\-ле\-тво\-ря\-ет всем ограничениям в~основной задаче 
и~уравнениям Беллмана}~(\ref{e17-sh}), (\ref{e18-sh}), \textit{то он является 
оптимальным в~этой задаче}.
     
     
     
     \subsection*{Замечания к теореме~1}
     
     \noindent
     \begin{enumerate}[1.]
\item Поставленная задача оптимального стохастического управления по форме 
аналогична задачам управления в~марковских моделях, рассмотренных 
в~работах~\cite{12-sh, 13-sh}. Отличие данной задачи заключается в~том, что 
в~ней основной случайный процесс представляет собой двумерную цепь Маркова 
$\{ k_n, \eta_n \}$, изменения состояний которой описываются 
соотношениями~(\ref{e6-sh})--(\ref{e8-sh}). Кроме того, этот основной процесс 
реализуется на конечном интервале времени. Однако общий принцип Беллмана для 
управляемого процесса, сформулированный в~классических 
работах~\cite{10-sh, 11-sh}, остается справедливым. 
%
Применим этот принцип 
к~задаче управления, в~которой показатель эф\-фек\-тив\-ности определяется сис\-те\-мой 
рекуррентных соотношений~(\ref{e11-sh}), (\ref{e12-sh}). Если зафиксировать 
произвольное со\-сто\-яние\linebreak  ($k_n\hm= x$, $\eta_n\hm=i$) и~решение~$\rho_i$, то, 
согласно принципу Белл\-ма\-на, дальнейшее поведение\linebreak сис\-те\-мы должно быть 
оптимальным по отношению к~состоянию, полученному в~результате этого 
решения. Новое состояние \mbox{сис\-те\-мы} определяется в~соответствии 
с~динамическими
 соотношениями~(\ref{e6-sh})--(\ref{e8-sh}). После перехода в~это состояние 
дальнейшее поведение должно быть оптимально, и~целевая функция задачи 
управ\-ле\-ния на интервале времени $\{ n+1, n+2, \ldots , N\}$ %\linebreak 
при новом начальном 
со\-сто\-янии должна сов\-па\-дать с~функцией Беллма\-на. Теперь для оп\-ти\-маль\-ности 
процесса на всем интервале времени $\{ n, n+1, \ldots , N\}$ достаточно вы\-брать 
\mbox{оптимальное} решение~$\rho_i$, до\-став\-ля\-ющее максимум выражению в~правой 
час\-ти соотношения~(\ref{e17-sh}). Таким образом, соотношение~(\ref{e17-sh})\linebreak 
непосредственно следует из принципа Белл\-мана.

\item Предположим, что некоторый набор управ\-ле\-ний $\rho^*\hm= \left\{ 
\rho_1^*, \rho_2^*, \ldots , \rho_r^*\right\}$ удовлетворяет всем ограничениям  
основной задачи~(\ref{e13-sh}) при значении параметра $n\hm=0$
и~уравнениям Белл\-ма\-на. Тогда 
в~соотношениях~(\ref{e17-sh}), (\ref{e18-sh}) достигаются равенства. При 
этом функции Белл\-ма\-на будут определяться теми же самыми рекуррентными 
соотношениями, что и~целевые показатели~(\ref{e11-sh}), 
(\ref{e12-sh}). В~част\-ности, целевая функция в~основной задаче при 
$n\hm=0$ будет совпадать с~функцией Беллма\-на
\begin{equation}
S_0\left( x_0, i_0, \rho^*\right) =B_0\left( x_0, i_0\right)
\label{e19-sh}
\end{equation}
при фиксированных начальных условиях\linebreak ($k_0\hm=x_0$, $\eta_0\hm= i_0$). 
С~учетом определения функции Беллма\-на из равенства~(\ref{e19-sh}) следует 
оп\-ти\-маль\-ность вектора управ\-ле\-ний $\rho^*\hm= \left\{ \rho_1^*, \rho_2^*, \ldots , 
\rho_r^*\right\}$.
\end{enumerate}

\vspace*{-10pt}

\section{Заключение}

\vspace*{-2pt}

     Вывод системы уравнений Беллмана завершает теоретическую часть 
исследования поставленной задачи оптимального управления. Дальнейшее 
исследование должно быть связано с~численным решением системы уравнений 
Беллмана и~определением вектора оптимальных управлений. Для 
детерминированной динамической односекторной экономической модели 
в~работе~\cite{8-sh} был построен численный алгоритм решения системы 
уравнений Беллмана. Создание такого алгоритма для новой стохастической 
динамической односекторной экономической модели представляет собой 
самостоятельную задачу и~должно быть реализовано в~отдельной публикации.

\vspace*{-4pt}

{\small\frenchspacing
 {%\baselineskip=10.8pt
 %\addcontentsline{toc}{section}{References}
 \begin{thebibliography}{99}
 
 \vspace*{-2pt}


\bibitem{2-sh}
\Au{Ашманов С.\,А.} Математические модели и~методы в~экономике.~--- М.: Изд-во Московского
 ун-та, 1980. 199~с. 
 
 \bibitem{1-sh}
\Au{Интрилигатор М.} Математические методы оптимизации и~экономическая теория~/ Пер. 
с~англ.~--- М.: Айрис-Пресс, 2002. 553~с. (\Au{Intriligator~M.} Mathematical methods of optimization 
and economic theory.~--- Philadelphia, PA, USA: SIAM, 2002. 508~p.)
\bibitem{3-sh}
\Au{Иоффе А.\,Д., Тихомиров~В.\,М.} Теория экстремальных задач.~--- М.: Наука, 1974. 480~с.
\bibitem{4-sh}
\Au{Лившиц К.\,И., Параев Ю.\,И.} Оптимальное управление.~--- СПб.: Лань, 2020. 232~с.

\pagebreak

\bibitem{5-sh}
\Au{Kamien M., Schwartz N.} Dynamic optimization.~--- New York, NY, USA: Elsevier North Holland, 
1981. 331~p.
\bibitem{6-sh}
\Au{Barro R., Sala-i-Martin~X.} Economic growth.~--- 2nd ed.~--- Cambridge, MA, USA: MIT Press, 
2004. 654~p.
\bibitem{7-sh}
\Au{Параев Ю.\,И.} Оптимальное управление в~динамической экономике.~--- Томск: НТЛ, 2015. 
104~с.
\bibitem{8-sh}
\Au{Шнурков П.\,В., Рудак~А.\,О.} Алгоритмическое решение проблемы оптимального управления 
в~динамической односекторной экономической модели на основе метода динамического 
программирования~// Системы и~средства информатики, 2019. Т.~29. №\,1. С.~128--139.

\bibitem{9-sh}
\Au{Ширяев А.\,Н.} Вероятность-2.~--- 4-е изд.~--- М.: \mbox{МЦНМО}, 2007. 416~с.
\bibitem{10-sh}
\Au{Беллман Р.} Динамическое программирование~/ Пер. с~англ.~--- М.: ИЛ, 
1960. 400~с. (\Au{Bellman~R.} Dynamic programming.~--- Princeton, NJ, USA: Princeton University 
Press, 1957. 339~p.)
\bibitem{11-sh}
\Au{Беллман Р., Дрейфус~С.} Прикладные задачи динамического программирования~/ Пер. 
с~англ.~--- М.: Наука, 1965. 459~с. (\Au{Bellman~R., Dreyfus~S.} Applied dynamic programming.~--- 
London: Oxford University Press, 1962. 363~p.)
\bibitem{12-sh}
\Au{Ховард Р.\,А.} Динамическое программирование и~марковские процессы~/ Пер. с~англ.~--- 
М.: Советское радио, 1964. 194~с. (\Au{Howard~R.\,A.} Dynamic programming and Markov processes.~--- 
Cambridge, MA, USA: MIT Press, 1960. 136~p.) 
\bibitem{13-sh}
\Au{Майн Х., Осаки С.} Марковские процессы принятия решений~/ Пер. с~англ.~--- М.: Наука, 
1977. 176~с. (\Au{Mine~H., Osaki~S.} Markovian decision processes.~--- New York, NY, USA:  
Elsevier, 1970. 142~p.)
\end{thebibliography}

 }
 }

\end{multicols}

\vspace*{-3pt}

\hfill{\small\textit{Поступила в~редакцию 18.10.21}}

\vspace*{9pt}

%\pagebreak

%\newpage

%\vspace*{-28pt}

\hrule

\vspace*{2pt}

\hrule

%\vspace*{8pt}

\def\tit{CREATION OF A STOCHASTIC DYNAMIC ONE-SECTOR ECONOMIC MODEL WITH~DISCRETE 
TIME AND~ANALYSIS OF~THE~CORRESPONDING OPTIMAL CONTROL PROBLEM}


\def\titkol{Creation of a stochastic dynamic one-sector economic model with~discrete time
 and~analysis of the %corresponding 
 optimal control problem}


\def\aut{P.\,V.~Shnurkov}

\def\autkol{P.\,V.~Shnurkov}

\titel{\tit}{\aut}{\autkol}{\titkol}

\vspace*{-9pt}


\noindent
National Research University Higher School of Economics, 34~Tallinskaya Str., Moscow 123458, Russian Federation

\def\leftfootline{\small{\textbf{\thepage}
\hfill INFORMATIKA I EE PRIMENENIYA~--- INFORMATICS AND
APPLICATIONS\ \ \ 2021\ \ \ volume~15\ \ \ issue\ 4}
}%
 \def\rightfootline{\small{INFORMATIKA I EE PRIMENENIYA~---
INFORMATICS AND APPLICATIONS\ \ \ 2021\ \ \ volume~15\ \ \ issue\ 4
\hfill \textbf{\thepage}}}

\vspace*{6pt} 


\Abste{The work is devoted to the creation of a~stochastic dynamic model of 
optimal control with discrete time within the framework of a~one-sector economic system. The basis is 
a~classical deterministic dynamic model of the economic system in which one universal product is produced. 
This product is divided into investment and consumer components. 
System management consists in determining the relationship between these components.
 In this work, it is assumed that the main parameters of the system depend on some 
 random factor that characterizes the influence of the external environment. 
 This factor is described by a~homogeneous Markov chain with a finite set of states and 
 a~given transition probability matrix. In this work, 
 a~stochastic model of the evolution of the system under consideration is constructed 
 which is a~two-dimensional Markov process with discrete time. In terms of its economic content, 
 the first component of this process is specific capital and the second is the state of 
 an external random factor. The control parameter or decision at each moment of time represents 
 the share of the specific product produced directed to investment. The recurrent setting of the 
 cost additive indicator of management efficiency is described. The theoretical basis for solving 
 the problem of optimal control is the method of dynamic programming. In this work, 
 a~system of Bellman functional equations is obtained, the solution of which is the optimal control strategy.}

\KWE{optimal control problem with discrete time; stochastic dynamic one-sector economic model; 
controlled two-dimensional Markov chain; dynamic programming method for a~discrete-time control 
problem; Bellman equations}

\DOI{10.14357/19922264210405}

%\vspace*{-20pt}

%\Ack
%\noindent


\vspace*{-4pt}

  \begin{multicols}{2}

\renewcommand{\bibname}{\protect\rmfamily References}
%\renewcommand{\bibname}{\large\protect\rm References}

{\small\frenchspacing
 {%\baselineskip=10.8pt
 \addcontentsline{toc}{section}{References}
 \begin{thebibliography}{99}
 
 \vspace*{-2pt}


\bibitem{2-sh-1}
\Aue{Ashmanov, S.\,A.} 1980. \textit{Matematicheskie modeli i~metody v~ekonomike} [Mathematical models 
and methods in economics]. Moscow: Moscow University Press. 199~p. 

%\vspace*{-2pt}

\bibitem{1-sh-1}
\Aue{Intriligator, M.} 2002. \textit{Mathematical methods of optimization and economic theory}. Philadelphia, 
PA: SIAM. 508~p.
{\looseness=1

}

\columnbreak 


\bibitem{3-sh-1}
\Aue{Ioffe, A.\,D., and V.\,M.~Tikhomirov.} 1974. \textit{Teoriya ekst\-re\-mal'\-nykh zadach} [Extremal problems 
theory]. Moscow: Nauka. 480~p.
\bibitem{4-sh-1}
\Aue{Livshits, K.\,I., and Yu.\,I.~Paraev.} 2020. \textit{Optimal'noe upravlenie} [Optimal control]. St.\ 
Petersburg: Lan'.  232~p.
{ %\looseness=1

}



\bibitem{5-sh-1}
\Aue{Kamien, M., and N.~Schwartz.} 1981. \textit{Dynamic optimization}. New York, NY: Elsevier North 
Holland. 331~p.

\pagebreak

\bibitem{6-sh-1}
\Aue{Barro, R., and X.~Sala-i-Martin.} 2004. \textit{Economic growth}. 2nd ed. Cambridge, MA: MIT Press. 
654~p.



\bibitem{7-sh-1}
\Aue{Paraev, Yu.\,I.} 2015. \textit{Optimal'noe upravlenie v~di\-na\-mi\-che\-skoy economike} [Optimal control in 
a~dynamic economy]. \mbox{Tomsk}: NTL. 104~p.
\bibitem{8-sh-1}
\Aue{Shnurkov, P.\,V., and A.\,O.~Rudak.} 2019. Al\-go\-rit\-mi\-che\-skoe reshenie problemy optimal'nogo upravleniya 
v~di\-na\-mi\-che\-skoy odnosektornoy ekonomicheskoy modeli na osnove metoda di\-na\-mi\-che\-sko\-go programmirovaniya 
[Algorithmic solution of the problem of optimal control in \mbox{a~dynamic} one-sector economic model based on the 
method of dynamic programming]. \textit{Sistemy i~Sredstva Informatiki~--- Systems and Means of Informatics} 
29(1):128--139.
{ %\looseness=1

}

\columnbreak

\bibitem{9-sh-1}
\Aue{Shiryaev, A.\,N.} 2019. \textit{Probability-2}. Graduate texts in mathematics ser. 
3rd ed. New York, NY: Springer. 
Vol.~95. 358~p. 

%\columnbreak


\bibitem{10-sh-1}
\Aue{Bellman, R.} 1972. \textit{Dynamic programming}. 6th ed. Princeton, NJ: Princeton University Press. 
402~p.
\bibitem{11-sh-1}
\Aue{Bellman, R., and S.~Dreyfus.} 1962. \textit{Applied dynamic programming}. London: Oxford University 
Press. 363~p.
\bibitem{12-sh-1}
\Aue{Howard, R.\,A.} 1960. \textit{Dynamic programming and Markov processes}. Cambridge, MA: MIT 
Press. 136~p.
\bibitem{13-sh-1}
\Aue{Mine, H., and S.~Osaki.} 1970. \textit{Markovian decision processes}. New York, NY: Elsevier. 142~p.
\end{thebibliography}

 }
 }

\end{multicols}

\vspace*{-3pt}

\hfill{\small\textit{Received October 18, 2021}}

%\pagebreak

%\vspace*{-24pt}


\Contrl

\noindent
\textbf{Shnurkov Peter V.} (b.\ 1953)~--- Candidate of Science (PhD) in physics and mathematics, associate 
professor, National Research University Higher School of Economics, 34~Tallinskaya Str., Moscow 123458, 
Russian Federation; \mbox{pshnurkov@hse.ru}


\label{end\stat}

\renewcommand{\bibname}{\protect\rm Литература} 
      