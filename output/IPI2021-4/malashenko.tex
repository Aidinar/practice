
%\def\de{\defeq}
%\def\l{\ldots}
%\def\v{\vert}
%\def\V{\Vert}
%\def\dm{d^{max}}
%\def\d{{\rm d}}
%\def\t{\theta}
%\def\c{\cdot}
%\def\r(#1){({\ref{#1}})}
%\def\L{{\rm I}\!{\rm L}}
%\def\G{{\cal G}}
%\def\L{{\cal L}}
%\def\R{{\cal R}}
%\def\D{{\cal D}}
%\def\H{{\cal H}}
%\def\T{{\cal T}}
%\def\Q{{\cal Q}}
%\def\X{{\cal X}}
%\def\Y{{\cal Y}}
%\def\C{{\bf C}}


\def\stat{malash+naz}

\def\tit{АНАЛИЗ РАСПРЕДЕЛЕНИЯ ПРЕДЕЛЬНЫХ НАГРУЗОК В~МНОГОПОЛЬЗОВАТЕЛЬСКОЙ 
СЕТИ}

\def\titkol{Анализ распределения предельных нагрузок в~многопользовательской 
сети}

\def\aut{Ю.\,Е.~Малашенко$^1$, И.\,А.~Назарова$^2$}

\def\autkol{Ю.\,Е.~Малашенко, И.\,А.~Назарова}

\titel{\tit}{\aut}{\autkol}{\titkol}

\index{Малашенко Ю.\,Е.}
\index{Назарова И.\,А.}
\index{Malashenko Yu.\,E.}
\index{Nazarova I.\,A.}


%{\renewcommand{\thefootnote}{\fnsymbol{footnote}} \footnotetext[1]
%{Работа выполнена при частичной поддержке РФФИ (проект 19-07-00187-A).}}


\renewcommand{\thefootnote}{\arabic{footnote}}
\footnotetext[1]{Федеральный исследовательский центр <<Информатика 
и~управление>> Российской академии наук, \mbox{malash09@ccas.ru}}
\footnotetext[2]{Федеральный исследовательский центр <<Информатика 
и~управление>> Российской академии наук, \mbox{irina-nazar@yandex.ru}}

%\vspace*{-6pt}


\Abst{В рамках многопродуктовой  сетевой  модели анализируются 
недискриминирующие  распределения ресурсов для передачи межузловых потоков 
различных видов  между всеми парами узлов одновременно. При постановке 
и~решении  оптимизационных задач  под ресурсом понимается суммарное значение 
пропускных способностей, выделяемых на всех  ребрах, расположенных  на всех 
маршрутах прохождения межузлового потока для данной пары корреспондентов. 
Сумма соответствующих реберных потоков  трактуется  как полная нагрузка  на 
сеть,  возникающая  при передаче конкретного межузлового потока. При 
проведении вычислительных экспериментов последовательно решается   цепочка  
лексикографически упорядоченных задач  поиска  маршрутов с~равными   
нагрузками    для    равноправных  пар узлов.  На каждой итерации для 
многокомпонентной оценки функциональных возможностей системы  используется 
вектор  предельных значений  совместно допустимых   межузловых потоков.  
Метод   позволяет за конечное число шагов найти финальное недискриминирующее 
максиминное    распределение  ресурсов,   при котором достигается предельная 
загрузка  всех ребер сети.}

\KW{многопользовательская сеть;   уравнительное максиминное распределение 
ресурсов; предельная загрузка сети; функциональные характеристики}

\DOI{10.14357/19922264210403}
  
\vspace*{-4pt}


\vskip 10pt plus 9pt minus 6pt

\thispagestyle{headings}

\begin{multicols}{2}

\label{st\stat}


\section{Введение}

Данная работа продолжает исследование проб\-ле\-мы распределения ресурсов 
и~межузловых потоков в~территориально распределенных многопользовательских 
системах связи~\cite{Mal20-5}.
Рассматривается\linebreak  модель сети, в~которой межузловые потоки различных видов 
передаются между всеми парами узлов одновременно. Множества векторов 
совместно допустимых межузловых потоков \mbox{предлагается} использовать как 
многокритериальные оценки предельных функциональных возможностей сети. 
В~рамках вычислительных экспериментов анализируется многокритериальная 
стратегия  уравнительного распределения пропускной способности ребер. Цель 
управления при уравнительной стратегии состоит в~получении равных  
результирующих значений суммарных реберных потоков для всех пар узлов-кор\-рес\-пон\-ден\-тов. 
Для вычислительных экспериментов используется итерационная 
процедура решения последовательности однопродуктовых задач поиска 
максимального потока для выделенной пары узлов при фиксированных нулевых 
значениях для всех остальных. На каждом шаге  часть  ресурса распределяется 
поровну между всеми корреспондентами, для которых предельное значение потока 
строго больше нуля. Разработанный метод позволяет за конечное число шагов 
найти финальное распределение совместно допустимых дуговых потоков, при 
котором достигается предельная загрузка всех ребер сети~\cite{Mal20-5}.

В многопользовательских сетях при маршрутизации потоков требуется обеспечить 
несколько \mbox{путей} передачи для  соединения корреспондентов~[2--4]. Предлагаемый  
в~настоящей работе метод позволяет на каждой итерации  находить  различные 
пути соединения, проходящие  через минимальные разрезы при текущих значениях 
пропускной способности ребер. В~[5--7] методы поиска лексикографического 
максимина используются для получения справедливых распределений при заданных 
требованиях пользователей. В~настоящей работе\linebreak  на каждой итерации  часть 
имеющегося ресурса\linebreak распределяется строго поровну  между парами, для которых 
удается найти хотя бы один путь передачи потока при заданных пропускных 
способностях. Предлагаемую  в~разд.~3 процедуру можно рас\-смат\-ри\-вать как 
вариант  метода возможных направлений~\cite{Zoyt}   для  поиска  максиминного  
предельного распределения   ресурсов между равноправными \mbox{пользователями}.  

\vspace*{-6pt}

\section{Математическая модель}

Для описания многопользовательской сетевой системы связи  используется 
следующая математическая запись модели передачи многопродуктового потока.
Сеть $G(\mathbf{d})$ задается мно-\linebreak жествами  $\langle V, R,  U, P \rangle$:
узлов (вершин) сети  $V \hm= \{ v_1, v_2, \ldots, v_n, \ldots, v_N \}$;
неориентированных \mbox{ребер} $R \hm= \{ r_1, r_2, \ldots, r_k, \ldots, r_E \}$.
Ребро~$r_k$ соединяет концевые вершины $v_{n_k}$ и~$v_{j_k}$. Предполагается, 
что в~сети отсутствуют петли и~сдвоенные ребра.
Каж\-до\-му ребру~$r_k$ ставятся в~соответствие две ориентированные дуги $u_k$ 
и~$u_{k+E}$ из множества ориентированных дуг  $U \hm= \{ u_1, u_2, \ldots, u_k, 
\ldots, u_{2E}\}$.
Дуги $\{u_k, u_{k+E}\}$ определяют прямое и~обратное на\-прав\-ле\-ние передачи 
потока по  ребру~$r_k$ между концевыми вершинами $v_{n_k}$ и~$v_{j_k}$.

В сети $G(\mathbf{d})$ рассматривается $M \hm= N (N\hm-1)$ независимых, 
невзаимозаменяемых и~равноправных межузловых потоков различных видов.
По определению, каждой паре узлов-корреспондентов $p_m$ из множества $P \hm= 
\{ p_1, p_2, \ldots, p_M\}$ ставятся в~соответствие вершина-источник 
с~номером $s_m$,  из которой  входной поток $m$-го вида поступает в~сеть, 
и~вер\-ши\-на-при\-ем\-ник с~номером ${t_m}$, из которой поток $m$-го вида покидает 
сеть.

Обозначим через $z_m$ величину \textit{межузлового} потока $m$-го вида, 
который поступает в~сеть из узла с~номером $s_m$ и~покидает ее из узла с~номером~$t_m$;
$x_{mk}$ и~$x_{m(k + E)}$~--- величину потока $m$-го вида, который передается 
по дугам $u_k$ и~$u_{k + E}$ согласно на\-прав\-ле\-нию передачи, $x_{mk} \hm\geq 
0$, $x_{m(k + E)}\hm\geq 0$, $m \hm=  \overline{1, M}$, $k =  \overline {1, 
E}$;
$S(v_n)$~--- множество номеров исходящих дуг, по которым поток покидает 
узел~$v_n$;
$T(v_n)$~--- множество номеров входящих дуг, по которым поток поступает 
в~узел~$v_n$.

Во всех узлах сети $v_n \hm\in V$, $n \hm=  \overline{1,N}$,  для каж\-до\-го 
вида потока должны выполняться условия сохранения потоков:
\begin{multline}
\sum\limits_{i \in S(v_n)}{x_{mi}} - \sum\limits_{i \in T(v_n)}{x_{mi}} ={}\\
{}=
\begin{cases}
z_m, & \mbox{если } v_n = v_{s_m}\,; \\
- z_m, & \mbox{если } v_n = v_{t_m}\,; \\
 0 & \mbox{в остальных случаях},
\end{cases}\\
n =  \overline {1, N}, \ m =  \overline {1, M}, \ x_{mi} \geq 0, \ z_m \geq 0\,.
\label{e1-m}
\end{multline}
Величина $z_m$ равна входному межузловому потоку $m$-го вида, который 
пропускается от источника к~приемнику пары~$p_m$ при распределении  
потоков~$x_{mi}$ по дугам сети.

Каждому ребру $r_k\hm \in R$ приписывается неотрицательное число~$d_k$, 
определяющее суммарный предельно допустимый поток, который можно передать по 
ребру~$r_k$ в~обоих направлениях. В~исходной сети компоненты вектора 
пропускных способностей $\mathbf{d} \hm= (d_1, d_2, \ldots, d_k, \ldots, 
d_E)$~--- наперед заданные положительные числа $d_k \hm> 0$. 
Вектором~$\mathbf{d}$ определяются ограничения на сумму потоков всех видов, 
передаваемых по ребру~$r_k$ одновременно:

\vspace*{-6pt}

\noindent
\begin{multline}
\sum\limits_{m=1}^{M} {(x_{mk}+ x_{m(k+E)})} \le d_k,  \\
 x_{mk} \geq 0\,, \enskip  
x_{m(k+E)} \geq 0\,, \enskip  k = \overline{1, E}\,. 
\label{e2-m}
\end{multline}

В рамках данной модели пропускная способность ребер сети~--- 
вектор~$\mathbf{d}$~--- трактуется как \textit{ресурсное ограничение}, а сумма  
дуговых потоков рас\-смат\-ри\-ва\-ет\-ся как показатель использования 
\mbox{\textit{ресурсов}}\linebreak сети при передаче межузловых потоков.
Для  $z_m$ и~$x_{mi}$, удовле\-тво\-ря\-ющих условиям~(1) и~(2), вы\-чис\-ля\-ют\-ся 
суммарные потоки по ребрам сети:
\begin{equation}
y_m = \sum\limits_{i=1}^{2E} x_{mi}, \enskip m =  \overline{1, M}. 
\label{e3-m}
\end{equation}

Поток $y_m$ характеризует \textit{нагрузку} на ребра сети  при передаче 
заданного межузлового потока величины~$z_m$ из узла-источника~$s_m$  в~узел-
приемник~$t_m$. Величина~$y_m$ показывает, какая суммарная пропускная 
способность сети используется для передачи межузлового потока~$z_m$, при этом 
отношение
$$
 w_m = \fr{y_m}{z_m}\,,  \enskip m =  \overline{1, M},
$$
можно трактовать как удельные затраты ресурсов для передачи единичного   
потока $m$-го вида между узлами~$s_m$ и~$t_m$ при    дуговых 
потоках~$x_{mi}$.

Ограничения (1) и~(2) задают множество  до\-пус\-ти\-мых значений компонент вектора 
межузловых потоков
$\mathbf{z}\hm = (z_1, z_2, \ldots, z_m, \ldots, z_M)$:
\begin{multline*}
 \mathcal{Z}(\mathbf{d}) ={}\\
 {}= \{\mathbf{z} \geq 0 \ |\  \exists \ \mathbf{x} 
\geq 0: \ (\mathbf{z}, \mathbf{x})  \mbox{ удовлетворяют } (1), (2)\}. 
%\label{e4-m}
\end{multline*}
Допустимые распределения реберных потоков принадлежат подмножеству
\begin{equation*} 
\hspace*{-2mm}Y(\mathbf{d}) = \{\mathbf{y} \geq 0 | (\mathbf{z}, \mathbf{x}, \mathbf{y})  
\mbox{\,удовлетворяют\,(1)--(3)}\}.\!\! 
%\label{e5-m}
\end{equation*}

В рамках модели при проведении вы\-чис\-ли\-тельных экспериментов определяется 
допустимое распределение межузловых потоков, для передачи которых всем парам 
уз\-лов-кор\-рес\-пон\-ден\-тов \mbox{выделяются} одинаковые ресурсы и~полностью используется 
пропускная способность всех ребер сети. Уравнительное распределение ресурсов 
находится в~результате решения цепочки задач поиска сбалансированных нагрузок 
для всех узлов. Фактически решается последовательность лексикографически 
упорядоченных максиминных задач для распределения пропускной способности сети 
и~получения близких значений нагрузки для разных пар. 

\section{Вычислительный эксперимент}

Для поиска уравнительного распределения ресурсов и~балансировки на\-груз\-ки  
использовался метод, подробно описанный в~\cite{Mal20-5}. При реализации  
вы\-чис\-ли\-тель\-ной процедуры  выполнение каж\-до\-го \mbox{шага}~$t$ разбивается на 
несколько этапов. На пред\-ва\-ри\-тель\-ном этапе при заданных значениях про\-пуск\-ной 
спо\-соб\-ности ребер для каж\-дой пары узлов\linebreak  \mbox{определяются}  максимальный 
однопродуктовый межузловой поток $z_a^0 (t)$,  соответствующие дуговые потоки  
$(x_{ai}^0(t)$, $x_{a(i+E)}^0(t))$, $i \hm=  \overline{1, E}$, и~нагрузки 
$y_m^0(t)$, $m \hm=  \overline {1,  M}$, согласно (3). На следующем этапе на 
основе найденных значений $y_m^0 (t)$ вычисляются  коэффициенты
$$ 
\omega_m^0(t) = \fr{1}{y_m^0(t)} \mbox{ для всех}\  y_m^0 (t) > 0, \ m =  
\overline {1,  M}.
$$
Определяется величина
$\alpha^*(t) \hm= \max_{\alpha} \alpha$
при условиях: 
\begin{multline*}
 \alpha \sum\limits_{m = 1}^M \omega_m^0(t) [x_{mi}^0(t)+  x_{m(i + E)}^0(t)]  
\le d_i^*(t-1), \\
 \alpha \geq 0\,,  \enskip   i = \overline{1, E}\,.
\end{multline*}

При найденном $\alpha^*(t)$ вычисляются текущие значения нагрузки на шаге~$t$ 
для всех пар $p_m \hm\in P$
\begin{multline*} 
y_m^{*}(t)= \alpha^*(t)\omega_m^0(t) \sum_{i=1}^{E} [x_{mi}^{0}(t) + x_{m(i + 
E)}^{0}(t)] = {}\\
{}=\alpha^*(t) \sum\limits_{i=1}^{E} \left(\fr{x_{mi}^{0}(t)}{y_m^0(t)} + 
\fr{x_{m(i + E)}^{0}(t)}{y_m^0(t)}\right) = \alpha^*(t), \\  
m =   \overline {1, M}.
\end{multline*}
На каждой текущей итерации нагрузки для всех пар-корреспондентов совпадают, 
т.\,е.\ для всех пар $p_m \hm\in P$, таких что $y_m^0(t)\hm > 0$, выделяются 
одинаковые ресурсы.

На заключительном этапе текущего шага~$t$ для всех ребер сети вычисляется 
остаточная пропускная способность
\begin{multline*} 
d_i^*(t) =
d_i^*(t-1) - {}\\
{}-\alpha^*(t) \sum\limits_{m=1}^{M} \omega_m^0(t) 
\left[x_{mi}^0(t) + x_{m(i+E)}^0(t)\right],\enskip  i = \overline{1, E},
\end{multline*}
и осуществляется проверка условий:
\begin{itemize}
\item если после завершения очередного шага $t$ окажется, что хотя бы для 
одного ребра $r_i \hm\in R$ величина остаточной пропускной способности 
$d_i^{*}(t) > 0$, то происходит переход к~следующему шагу~$t+1$;
\item
 если $d_i^{*}(t) = 0$ для всех  $i = \overline{1, E}$, то происходит 
останов, поскольку все пропускные способности ребер исчерпаны и~сеть 
полностью загружена.
\end{itemize}
Таким образом на каждом шаге определенная часть имеющегося ресурса сети 
делится строго поровну среди всех корреспондентов, для которых существует 
маршрут передачи при текущих значениях пропускной способности.

Вычислительные эксперименты проводились на моделях сетевых сис\-тем, 
пред\-став\-лен\-ных на рис.~1. 
В~каж\-дой сети 69~узлов. Пропускные способности ребер~--- значения $d_k$~--- 
выбирались случайным образом из отрезка $[900, 999]$ и~совпадали для ребер, 
при\-сут\-ст\-ву\-ющих в~обеих сетях. В~кольцевой сети пропускная спо\-соб\-ность каж\-до\-го 
из добавленных ребер со\-став\-ля\-ла~900.







На рис.~2 и~3 представлены финальные значения нагрузок в~базовой и~кольцевой 
сети. При построении диаграмм множество~$P$ было разделено на два 
непересекающихся подмножества: $P(R_{(+)})$~--- пар <<ис\-точ\-ник--при\-ем\-ник>>, 
соединенных ребром (смежных), и~$P(R_{(-)}) \hm= P \backslash 
P(R_{(+)})$~--- всех остальных. На рис.~2,\,\textit{б} и~3,\,\textit{б}
указаны распределения 
нагрузок $y_{(+)}(T)$ для пар из $P(R_{(+)})$, а~на рис.~2,\,\textit{а} и~3,\,\textit{а}~--- для 
$P(R_{(-)})$.
Все значения~$y_{(\cdot)}(T)$ упорядочиваются по величине от большего 
к~меньшему (по невозрастанию).  По горизонтальной оси\linebreak\vspace*{-12pt}

{ \begin{center}  %fig1
 \vspace*{6pt}
   \mbox{%
\epsfxsize=69.408mm
\epsfbox{mal-1.eps}
}

\end{center}

\noindent
{{\figurename~1}\ \ \small{
Модели сетевых систем: (\textit{а})~базовая сеть; 
(\textit{б})~кольцевая сеть
}}}

%\vspace*{6pt}

\setcounter{figure}{1}


\pagebreak

\end{multicols}

\begin{figure*} %fig2
\vspace*{1pt}
\begin{center}  
\mbox{%
\epsfxsize=163mm
\epsfbox{mal-2.eps}
}
\end{center}
\vspace*{-9pt}
\Caption{Базовая сеть. Финальные распределения  ресурсов: 
(\textit{а})~несмежные пары; (\textit{б})~смежные пары}
%\end{figure*}
%\begin{figure*} %fig3
\vspace*{9pt}
\begin{center}  
\mbox{%
\epsfxsize=163mm
\epsfbox{mal-3.eps}
}
\end{center}
\vspace*{-9pt}
\Caption{Кольцевая сеть. Финальные распределения  ресурсов: 
(\textit{а})~несмежные пары;  (\textit{б})~смежные пары
}
\end{figure*}


\begin{multicols}{2}

\noindent
 указываются относительные 
порядковые номера корреспондентов в~упорядоченных последовательностях:
$\pi_-(m) \hm= {m}/{M_-}$ для всех  $p_m \hm\in P(R_{(-)})$, 
$\pi_+(k) \hm= {k}/{M_+}$ для всех $p_k\hm \in P(R_{(+)})$,
где $M_- \hm= |P(R_{(-)})|$ и~$M_+ \hm= |P(R_{(+)})|$~---
число элементов в~подмножествах $P(R_{(-)})$ и~$P(R_{(+)})$ со\-от\-вет\-ст\-венно.

Диаграммы на рис.~2,\,\textit{а} и~3,\,\textit{а} свидетельствуют, что нагрузку   для несмежных 
пар-кор\-рес\-пон\-ден\-тов удается распределить почти равномерно как в~базовой, так 
и~в кольцевой сети. Однако для~ 100~смежных пар финальные нагрузки отличаются 
почти на два порядка от средних значений. Суммарная пропускная спо\-соб\-ность 
в~кольцевой сети на 11\% больше, чем в~базовой, а~чис\-ло смежных пар 
со\-став\-ля\-ет~170. Чис\-лен\-ные значения на диаграммах различаются на 8\%--10\%, 
однако все графики на\-гляд\-но демонстрируют указанные   особенности  
в~распределении   нагрузок при предельной загрузке всех ребер сети.

Диаграммы на рис.~4 и~5  позволяют проследить, как на каждой  итерации в~ходе 
вычислительного эксперимента меняются суммарные  нагрузки и~соответствующие 
межузловые потоки.  

На рис.~4 и~5 данные для нагрузок и~потоков указаны 
отдельно:
$y_{(+)}$ и~$z_{(+)}$~--- для  смежных пар корреспондентов;
$y_{(-)}$ и~$z_{(-)}$~--- для всех  остальных  пар узлов.
По горизонтальным осям отложены суммарные значения для смежных пар узлов 
$y_{(+)}$\linebreak и~$z_{(+)}$, а~по вертикальной~--- для  всех остальных 
корреспондентов $y_{(-)}$ 
и~$z_{(-)}$. Жирные точки на графиках соответствуют суммарным значениям, 
достигнутым после завершения очередной итерации,\linebreak следуют снизу вверх и~слева 
направо согласно порядку их выполнения. Финальным значениям нагрузок 
$y_{(+)}(T)$ и~$y_{(-)}(T)$, изображенным на рис.~2 и~3, соответствуют 
крайние точки в~се\-ве\-ро-вос\-точ\-ной части рис.~4 и~5.  На диаграмме рис.~4 
справа можно проследить динамику изменения суммарных реберных нагрузок 
в~базовой сети.  Из рис.~4\linebreak\vspace*{-12pt}

\pagebreak

\end{multicols}

\begin{figure*}%fig4
\vspace*{1pt}
\begin{center}  
\mbox{%
\epsfxsize=159.088mm
\epsfbox{mal-4.eps}
}
\end{center}
\vspace*{-9pt}
\Caption{Базовая сеть. Уравнительное распределение потоков~(\textit{а}) 
и~ресурсов (\textit{б})}
%\end{figure*}
%\begin{figure*} %fig5
\vspace*{6pt}
\begin{center}  
\mbox{%
\epsfxsize=159.088mm
\epsfbox{mal-5.eps}
}
\end{center}
\vspace*{-9pt}
\Caption{Кольцевая сеть. Уравнительное распределение потоков~(\textit{а}) 
и~ресурсов~(\textit{б})}
\end{figure*}


\begin{multicols}{2}

\noindent
 следует,
 что  при пошаговом уравнительном 
распределении  начиная с~некоторой итерации резко меняются нагрузки, 
а~следовательно, и~доли ресурсов, которые выделяются корреспондентам из разных 
групп: $y_{(-)}$ и~$y_{(+)}$.



В исходной базовой сети 150 смежных корреспондентов, а~всех остальных~--- 
4\,500, соотношение\linebreak  числа  различных межузловых потоков, пе\-ре\-да\-ва\-емых 
одновременно,~--- $1:30$.
На начальных итерациях расчетные агрегированные значения соответствуют как  
числу корреспондентов в~каждой \mbox{группе}, так и~уравнительным правилам 
распределения, которые используются на каждом шаге.  На последующих шагах  
остаточная пропускная спо\-соб\-ность многих  ребер становится равной нулю, сеть 
распадается на отдельные фрагменты и~в~результате для не\-смеж\-ных  
корреспондентов не остается путей со\-еди\-не\-ния и~значения  дуговых потоков 
фик\-си\-ру\-ют\-ся на достигнутом уровне. На последних итерациях  около 
20\%--25\% остаточной суммарной про\-пуск\-ной спо\-соб\-ности распределяются только 
между смеж\-ны\-ми узлами.

Для примера рассмотрим диаграмму на рис.~4, которая относится к~базовой сети. 
Суммарная пропускная спо\-соб\-ность~$D^*$ и~соотношение чис\-ла пар-кор\-рес\-пон\-ден\-тов 
в~подмножествах $P(R_{(-)})$ и~$P(R_{(+)})$ со\-став\-ляют

\noindent
$$ 
D^* = \sum\limits_{k = 1}^{E} d_k = 68\,250; \  |P(R_{(-)})| : 
|P(R_{(+)})| = 30:1.
$$

\vspace*{-2pt}

\noindent
В точке излома на промежуточной итерации распределения ресурсов и~текущих 
значений межузловых потоков:

\noindent
$$ 
y_{(-)}(t) : y_{(+)}(t) = 25:1;   \enskip \ z_{(-)}(t) : z_{(+)}(t) = 4:1.
$$

\vspace*{-2pt}

\noindent
При этом используется почти 75\% суммарной пропускной способности всех ребер 
сети. Однако для финальных  суммарных значений межузловых потоков  получено 
соотношение $z_{(-)} : z_{(+)} \hm= 9:5$. Таким образом, для 150 смежных пар 
финальные межузловые потоки почти в~2~раза превышают суммарный поток для 
4\,500 оставшихся корреспондентов. Финальные нагрузки, которые показывают, 
как используются  ресурсы сети, для 3\% смежных\linebreak\vspace*{-12pt}

\pagebreak

\noindent
 пар-корреспондентов 
составляют~20\% от пропускной способности. 

\vspace*{-12pt}

\section{Заключение}

\vspace*{-2pt}

Описанные в~разд.~3 вычислительные эксперименты проводились для оценки 
предельных возможностей многопользовательской сети и~сравнения различных 
принципов балансировки нагрузок и~распределения ресурсов. Для исследования 
требовалось многократное решение оптимизационных задач с~векторным 
функционалом. При разработке вычислительной процедуры использовался метод 
последовательного  лексикографического максимина для поиска  
недискриминирующего  распределения  ресурсов  и~сбалансированной  нагрузки  
для  всех равноправных  пар-корреспондентов.

При реализации итерационной схемы  остаточная пропускная способность 
распределялась поровну среди  корреспондентов на всех доступных маршрутах 
соединения, проходящих через текущие  минимальные  разрезы на каждом шаге. 
Проведенные эксперименты показали, что только для~96\% корреспондентов 
удается поддерживать распределение в~соответствии 
с~заложенными принципами, а~для~4\%  выделенные ресурсы  и~межузловые потоки 
могут отличаться на два порядка от средних  значений.  

{\small\frenchspacing
 {%\baselineskip=10.8pt
 %\addcontentsline{toc}{section}{References}
 \begin{thebibliography}{9}
  
\bibitem{Mal20-5} 
\Au{Малашенко~Ю.\,Е., Назарова~И.\,А.} Анализ равнодолевого и~уравнительного  
распределения потоков при максимальной загрузке многопользовательской сети~// 
Изв. РАН. ТиСУ, 2021. №\,5. С.~85--93.



\bibitem{Baier2005} 
\Au{Baier G., Kohler~E.,  Skutella~M.}  The k-splittable flow problem~// 
Algorithmica, 2005. Vol.~42. Iss.~3-4. P.~231--248.

\bibitem{Kabadurmus2016} 
\Au{Kabadurmus~O., Smith~A.\,E.}  Multicommodity k-splittable survivable 
network design problems with relays~// Telecommun. Syst., 2016. Vol.~62. 
Iss.~1. P.~123--133.

\bibitem{Bialon2017} 
\Au{Bialon~P.} A randomized rounding approach to a k-splittable 
multicommodity flow problem with lower path flow bounds affording solution 
quality guarantees~// Telecommun. Syst., 2017. Vol.~64. Iss.~3. P.~525--542.

\bibitem{Georgiadis2001} 
\Au{Georgiadis~L., Georgatsos~P., Floros~K., Sartzetakis~S.}  
Lexicographically optimal balanced networks~// IEEE \mbox{INFOCOM} Ser., 2001. 
Vol.~2.  P.~689--698.

\bibitem{Radunovic2007} 
\Au{Radunovic\;B., Le\;Boudec\;J.-Y.}  A~unified 
framework for max-min and min-max fairness with applications~// IEEE ACM 
T.~Network., 2007. Vol.~15. Iss.~5. P.~1073--1083.

\bibitem{Nace2008} 
\Au{Nace~D., Doan~L.\,N., Klopfenstein~O., Bashllari~A.}   Max-min fairness 
in multicommodity flows~// Comput. Oper. Res., 2008. Vol.~35. Iss.~2. P.~557--573.


\bibitem{Zoyt} 
\Au{Зойтендейк~Г.} Методы возможных на\-прав\-ле\-ний~/ Пер. с~англ.~--- М.: 
ИЛ, 1963. 175~c. (\Au{Zoutendijk~G.} Methods of feasible 
directions.~--- Amsterdam: Elsevier Publishing Co., 1960. 126~p.)

\end{thebibliography}

 }
 }

\end{multicols}

\vspace*{-9pt}

\hfill{\small\textit{Поступила в~редакцию 29.07.21}}

\vspace*{4pt}

%\pagebreak

%\newpage

%\vspace*{-28pt}

\hrule

\vspace*{2pt}

\hrule

\vspace*{-4pt}

\def\tit{ANALYSIS OF PEAK LOAD DISTRIBUTION\\ IN~THE~MULTIUSER 
NETWORK}


\def\titkol{Analysis of peak load distribution in the multiuser network}


\def\aut{Yu.\,E.~Malashenko and I.\,A.~Nazarova}

\def\autkol{Yu.\,E.~Malashenko and I.\,A.~Nazarova}

\titel{\tit}{\aut}{\autkol}{\titkol}

\vspace*{-13pt}


\noindent
Federal Research Center ``Computer Science and Control'' of the Russian Academy of Sciences,  
44-2 Vavilov Str., Moscow 119333, Russian Federation

\def\leftfootline{\small{\textbf{\thepage}
\hfill INFORMATIKA I EE PRIMENENIYA~--- INFORMATICS AND
APPLICATIONS\ \ \ 2021\ \ \ volume~15\ \ \ issue\ 4}
}%
 \def\rightfootline{\small{INFORMATIKA I EE PRIMENENIYA~---
INFORMATICS AND APPLICATIONS\ \ \ 2021\ \ \ volume~15\ \ \ issue\ 4
\hfill \textbf{\thepage}}}

\vspace*{3pt} 


\Abste{Within the framework of a~multicommodity network model, nondiscriminatory distribution of the 
tantamount flows of various types transmitted between all pairs of nodes simultaneously is analyzed. 
When setting and solving optimization problems, the resource required by 
a~certain source--receiver pair is treated as the sum of the capacity values of all 
edges located on all routes of this source--receiver flow. The sum of the corresponding 
edge flows is interpreted as the total load on the network occurring during a~transmission 
of this internode flow. A~nuclear-chain of lexicographically ordered problems of searching 
for routes with equal loads for source--receiver pairs is solved in computational experiments. 
At each iteration, a~vector of peak values of jointly permissible internode flows is used 
for assessment of the system's functionality. The method allows for a finite number of 
steps to find the final nondiscriminating maximin distribution of resources providing 
the peak load of all network edges.}

\KWE{multiuser network; equalizing maximum peak load distribution; network peak load; 
functional characteristics}

\DOI{10.14357/19922264210403}

%\vspace*{-20pt}

%\Ack
%\noindent


\vspace*{-6pt}

  \begin{multicols}{2}

\renewcommand{\bibname}{\protect\rmfamily References}
%\renewcommand{\bibname}{\large\protect\rm References}

{\small\frenchspacing
 {%\baselineskip=10.8pt
 \addcontentsline{toc}{section}{References}
 \begin{thebibliography}{9}

\bibitem{1-m}
\Aue{Malashenko, Yu.\,E., and I.\,A.~Nazarova.} 2021. Analysis of the equal share and equalitarian 
flow distributions under the  peak load of the multi-user network. \textit{J.~Comput. Sys. Sc. Int.}  
60(5):889--897.
\bibitem{3-m} %2
\Aue{Baier G., Kohler E., and M.~Skutella.} 2005. The k-splittable flow problem. 
\textit{Algorithmica}
 42(3-4):231--248.
 
 \bibitem{2-m} %3
\Aue{Kabadurmus, O., and A.\,E.~Smith.} 2016. Multicommodity k-splittable survivable network 
design problems with relays. \textit{Telecommun. Syst.} 62(1):123--133.

\bibitem{4-m}
\Aue{Bialon, P.} 2017. A~randomized rounding approach to a~\mbox{k-splittable} multicommodity flow 
problem with lower path flow bounds affording solution quality guarantees. \textit{Telecommun. 
Syst.}   64(3):525--542.
{ %\looseness=1

}
\bibitem{5-m}
\Aue{Georgiadis, L., P.~Georgatsos, K.~Floros, and S.~Sartzetakis.} 2001. Lexicographically 
optimal balanced networks. \textit{IEEE INFOCOM Ser.} 2:689--698. 
\bibitem{6-m}
\Aue{Radunovic, B., and J.-Y.~~Le Boudec.} 2007. A~unified framework for max-min and min-max 
fairness with applications. \textit{IEEE ACM T.~Network.} 15(5):1073--1083.
{\looseness=1

} 
\bibitem{7-m}
\Aue{Nace, D., L.\,N. Doan, O.~Klopfenstein, and A.~Bashllari.} 2008. Max-min fairness in 
multicommodity flows. \textit{Comput. Oper. Res.}  35(2):557--573.
\bibitem{8-m}
\Aue{Zoutendijk, G.} 1960. \textit{Methods of feasible directions}. Amsterdam: Elsevier Publishing 
Co. 126~p.
\end{thebibliography}

 }
 }

\end{multicols}

\vspace*{-3pt}

\hfill{\small\textit{Received July 29, 2021}}

%\pagebreak

%\vspace*{-24pt}



\Contr

\noindent
\textbf{Malashenko Yuri E.} (b.\ 1946)~--- Doctor of Science in physics and mathematics, principal 
scientist, Federal Research Center ``Computer Science and Control'' of the Russian Academy of 
Sciences, 
44-2~Vavilov Str., Moscow 119333, Russian Federation; \mbox{malash09@ccas.ru}

\vspace*{3pt} 

\noindent
\textbf{Nazarova Irina A.} (b.\ 1966)~---  Candidate of Science (PhD) in physics and mathematics, 
scientist, Federal Research Center ``Computer Science and Control'' of the Russian Academy of 
Sciences, 
44-2~Vavilov Str., Moscow 119333, Russian Federation; \mbox{irina-nazar@yandex.ru}



\label{end\stat}

\renewcommand{\bibname}{\protect\rm Литература} 
      