\def\stat{sokolov}

\def\tit{БАЗИС РЕАЛИЗАЦИИ СБОЕУСТОЙЧИВЫХ ЭЛЕКТРОННЫХ СХЕМ}

\def\titkol{Базис реализации сбоеустойчивых электронных схем}

\def\aut{И.\,А.~Соколов$^1$, Ю.\,А.~Степченков$^2$, Ю.\,Г.~Дьяченко$^3$,  
Ю.\,В.~Рождественский$^4$,\\ А.\,Н.~Каменских$^5$}

\def\autkol{И.\,А.~Соколов, Ю.\,А.~Степченков, Ю.\,Г.~Дьяченко и~др.}  
%Ю.\,В.~Рождественский$^4$, А.\,Н.~Каменских$^5$}

\titel{\tit}{\aut}{\autkol}{\titkol}

\index{Соколов И.\,А.}
\index{Степченков Ю.\,А.}
\index{Дьяченко Ю.\,Г.}  
\index{Рождественский Ю.\,В.}
\index{Каменских А.\,Н.}
\index{Sokolov I.\,A.}
\index{Stepchenkov Yu.\,A.}
\index{Diachenko Yu.\,G.} 
\index{Rogdestvenski Yu.\,V.}
\index{Kamenskih A.\,N.}


%{\renewcommand{\thefootnote}{\fnsymbol{footnote}} \footnotetext[1]
%{Работа частично финансирована РФФИ (проект 19-01-00430).}}


\renewcommand{\thefootnote}{\arabic{footnote}}
\footnotetext[1]{Федеральный исследовательский центр <<Информатика и~управление>> Российской академии наук, 
\mbox{ISokolov@ipiran.ru}}
\footnotetext[2]{Федеральный исследовательский центр <<Информатика и~управление>> Российской академии наук, 
\mbox{YStepchenkov@ipiran.ru}}
\footnotetext[3]{Федеральный исследовательский центр <<Информатика и~управление>> Российской академии наук, 
\mbox{diaura@mail.ru}}
\footnotetext[4]{Федеральный исследовательский центр <<Информатика и~управление>> Российской академии наук, 
\mbox{YRogdest@ipiran.ru}}
\footnotetext[5]{Пермский национальный исследовательский политехнический университет, 
\mbox{antoshkinoinfo@yandex.ru}}

\vspace*{6pt}

 


\Abst{Исследуется устойчивость самосинхронных (СС) и~синхронных схем к~логическим 
сбоям (ЛС), которые могут вызвать нарушения режима работы системы управления 
сложным техническим устройством. Предлагается использование сбоеустойчивого 
СС-ко\-ди\-ро\-ва\-ния, рас\-смат\-ри\-ва\-юще\-го ан\-ти\-спей\-сер\-ное со\-сто\-яние как 
второе состояние спейсера, что позволяет повысить уровень сбоеустойчивости 
СС-схем. Количественные оценки в~первом приближении показывают 
явное (в~2,0--4,7~раза) преимущество СС-схе\-мы в~сравнении 
с~синхронным аналогом по уровню сбоеустойчивости. Использование 
модифицированного С-эле\-мен\-та Маллера для реализации разряда регистра ступени 
конвейера увеличивает это преимущество до уровня 2,2--5,4~раза. Благодаря этому 
СС-схе\-мы становятся предпочтительным базисом для реализации 
сбоеустойчивых электронных схем для систем управления сложными техническими 
устройствами.}

\KW{синхронные схемы; самосинхронные схемы; логический сбой; 
сбоеустойчивость; конвейер; индикация завершения переключения; вероятностная 
оценка}

\DOI{10.14357/19922264210409}
  
%\vspace*{9pt}


\vskip 10pt plus 9pt minus 6pt

\thispagestyle{headings}

\begin{multicols}{2}

\label{st\stat}

\section{Введение}

  Неблагоприятные факторы окружающей среды (ядерные частицы, радиация, 
электромагнитное излучение, шумы) вызывают сбои в~работе интегральных 
микросхем. Количественные и~качественные оценки характера сбоев 
показывают, что среди них преобладают кратковременные однократные 
логические сбои (soft errors)~[1]. С~точки зрения жизнеспособности 
микросхемы они менее опасны, чем отказы, поскольку приводят к~временному 
изменению состояния логического элемента схемы.

  
  Характеристики ЛС существенно зависят от типа источника сбоя и~его 
мощности (энергии). Например, ЛС, вызванный высокоэнергичной ядерной 
частицей или космическими лучами, может длиться до нескольких 
наносекунд~[2], в~то время как длительность ЛС из-за помех на трассах 
межсоединений, как правило, не превышает десятков пикосекунд. 
  
  Современная микроэлектроника в~основном реализует синхронный принцип 
обработки данных, базирующийся на глобальном синхросигнале. 
С~повышением частоты синхронизации растет и~вероятность того, что ЛС, 
даже кратковременный, запишется в~регистр и~станет критическим.
  
  Альтернативой синхронным схемам выступают СС-схе\-мы~[3]. 
  Они имеют ряд свойств, отличающих их от синхронных схем:
  \begin{itemize}
\item избыточное кодирование данных. В~комбинационных СС-схе\-мах, как 
правило, используется парафазное кодирование информационных сигналов;
\item двухфазную дисциплину функционирования: рабочая фаза, в~которой 
схема формирует новое рабочее состояние, и~спейсерная фаза, в~которой 
каждый информационный сигнал схемы переключается в~одно состояние, не 
входящее в~множество разрешенных рабочих состояний;
\item индикацию всех элементов схемы, подтверждающую завершение 
переключения схемы в~текущую фазу.
\end{itemize}

  Эти свойства обеспечивают маскирование большей части ЛС. Благодаря им, 
СС-схемы обладают большей устойчивостью к~ЛС, чем их синхронные 
аналоги~[4--6]. Предложенные в~работах~\cite{4-sok, 5-sok} методы 
схемотехнического и~топологического проектирования СС-схем дополнительно 
повышают устойчивость СС-схем к~ЛС.
  
  Известен ряд способов~[7, 8] повышения сбоеустойчивости асинхронных 
схем. Однако в~них не делается попыток количественного анализа 
естественного уровня стойкости синхронных и~самосинхронных схем к~ЛС.
  
  В статье проводится анализ особенности появления и~распространения ЛС 
в~синхронных и~самосинхронных схе\-мах. Даются вероятностные оценки их устойчивости 
к~ЛС, подтверждающие целесообразность использования СС-схем в~качестве 
базиса для реализации сбоеустойчивых систем управления.
  
\section{Типы логических сбоев}

  Типы ЛС зависят от используемого кодирования обрабатываемых данных. 
Синхронные схемы обычно используют неизбыточное двоичное кодирование. 
Логический сбой в~них выражается в~изменении значения сигнала на противоположное.  
Самосинхронные схе\-мы всегда используют избыточное, чаще всего парафазное, 
кодирование~[3] и~строгое чередование спейсера и~рабочих со\-сто\-яний. 
  
  В каждый момент времени парафазный сигнал пребывает в~спейсере 
(нулевом, <<00>>, или единичном, <<11>>) или в~одном из рабочих состояний 
(<<01>> или <<10>>). Состояние, противоположное спейсеру данного 
парафазного сигнала,~--- антиспейсер~--- в~корректно спроектированной  
СС-схе\-ме может появиться только в~результате ЛС. Типовая индикация  
СС-схе\-мы воспринимает антиспейсер как рабочее состояние, что приводит 
к~ошибке. Но индикация его как спейсера с~помощью ячейки 
<<неравнозначность>> (XOR) существенно повышает устойчивость СС-схем 
к~этому типу ЛС~\cite{4-sok}. Таблица~1 демонстрирует сбоеустойчивое 
кодирование парафазного сигнала. Все возможные случаи однократного 
кратковременного ЛС в~СС-схе\-мах для парафазного сигнала с~нулевым 
спейсером показаны в~табл.~2.



  
  Отметим, что не каждый ЛС в~схеме становится критическим, т.\,е.\ приводит к~искажению обрабатываемых данных. Он может замаскироваться значениями 
других сигналов в~соответствии с~выполняемой функцией или не успеть 
записаться в~регистр. Оценим вероятность того, что наблюдаемый\linebreak\vspace*{-12pt}

\begin{center}
{{\tablename~1}\ \ \small{
Сбоеустойчивое кодирование СС-схем
}}

\vspace*{6pt}

{\small
\begin{tabular}{|c|c|c|c|}
\hline
\multicolumn{2}{|c|}{Парафазный сигнал}&
\multicolumn{1}{c|}{\raisebox{-6pt}[0pt][0pt]{Состояние}}&Индикатор \\
\cline{1-2}
\ \ \ \ \ \ \ X\ \ \ \ \ \ \ &XB&&XOR\\
\hline
0&0&спейсер&0\\
0&1&бит <<0>>&1\\
1&0&бит <<1>>&1\\
1&1&спейсер&0\\
\hline
\end{tabular}}
\end{center}
%\begin{table*}
%\end{table*}
%\begin{table*}\small %tabl2
\begin{center}
{{\tablename~2}\ \ \small{
Возможные ЛС
}}


\vspace*{6pt}

{\small
\begin{tabular}{|c|c|c|c|c|c|c|c|c|c|}
\hline
\tabcolsep=0pt\begin{tabular}{c}Состояние\\ парафазного\\ сигнала\end{tabular}& До ЛС& После ЛС\\
\hline
1&00&01\\
2&00&10\\
3&00& 11\\
4&01& 00\\
5&01& 11\\
6&01& 10\\
7&10& 00\\
8&10&11\\
9&10&01\\
\hline
\end{tabular}}
\end{center}
%\end{table*}

\vspace*{6pt}



\noindent
 в~схеме 
кратковременный ЛС окажется критическим.
  
\section{Методика расчета вероятности~сбоя}

  Для оценки вероятности появления критического сбоя используем 
следующие предположения:


\noindent
  \begin{enumerate}[(1)]
\item плотность потока событий, приводящих к~появлению ЛС, 
одинакова по всей площади крис\-тал\-ла микросхемы;
\item разные исходы каждого рассматриваемого события равновероятны;
\item длительность ЛС превышает тактовый период в~синхронной схеме или 
суммарную длительность рабочей и~спейсерной фазы в~СС-схеме. 
\end{enumerate}

  Первое предположение означает, что вероятность ЛС в~какой-то части схемы 
прямо пропорциональна площади, занимаемой этой частью на кристалле.
  
  Второе предположение упрощает анализ со\-сто\-яний при возникновении сбоя. 
Здесь не учитываются некоторые факторы, вли\-я\-ющие на ве\-ро\-ят\-ность разных 
исходов. Например, ве\-ро\-ят\-ность маскирования сбоя в~сигнале другими 
сигналами зависит от вы\-пол\-ня\-емой функции. В~каж\-дом конкретном случае она 
будет иметь разное значение, но в~пер\-вом при\-бли\-же\-нии считаем ее равной~0,5.
  
  Высокопроизводительные синхронные и~самосинхронные схемы имеют конвейерную 
структуру. Отличие СС-кон\-вей\-ера от синхронного аналога состоит 
в~отсутствии глобального сигнала синхронизации и~наличии подсхемы 
управления взаимодействием соседних ступеней. Рисунок~1 иллюстрирует 
конвейер СС-схе\-мы. Здесь Ack и~Req~--- сигналы за\-прос-от\-вет\-но\-го 
взаимодействия ступеней СС-кон\-вей\-ера.
  
\begin{figure*} %fig1
\vspace*{1pt}
\begin{center}  
\mbox{%
\epsfxsize=115.462mm
\epsfbox{sok-1.eps}
}
\end{center}
\vspace*{-9pt}
\Caption{Структура типового СС-конвейера}
\end{figure*}

  Каждая ступень СС-кон\-вей\-ера включает комбинационную часть (КЧ) 
и~выходной регистр (ВР), аналогично синхронному конвейеру, а~также 
индикаторные подсхемы комбинационной час\-ти (ИКЧ) и~регистра (ИВР).  
C-эле\-мен\-ты Маллера~\cite{7-sok} обеспечивают за\-прос-от\-вет\-ное взаимодействие 
соседних ступеней СС-кон\-вей\-ера на основе индикаторных выходов КЧ и~ВР.
  
  Устойчивость конвейера к~кратковременным ЛС определяется 
устойчивостью его составных частей. Число однократных критических ЛС, 
наблюдаемых в~КЧ ступени синхронного конвейера в~течение времени~$T$:
  $$
N_{\mathrm{CPS}} = \lambda T A_{\mathrm{CPS}} P_{\mathrm{CPS},1} P_{\mathrm{CPS},2}\,,
$$
где $\lambda$~--- плотность потока сбоев (число сбоев в~единицу времени на 
единицу площади); $A_{\mathrm{CPS}}$~--- площадь топологии КЧ; $P_{\mathrm{CPS},1}$~--- 
вероятность того, что сбой не замаскируется и~попадет на вход ВР ступени; 
$P_{\mathrm{CPS},2}$~--- вероятность того, что сбой успеет записаться в~ВР. Поскольку 
здесь предполагается равная вероятность исхода для каждого события 
и~считается, что длительность сбоя превышает период тактового сигнала, 
$P_{\mathrm{CPS},1}\hm = 0{,}5$ и~$P_{\mathrm{CPS},2}\hm = 1$. Тогда
$$
N_{\mathrm{CPS}} = 0{,}5\lambda TA_{\mathrm{CPS}}\,.
$$
  
  Аналогичное число однократных критических ЛС, наблюдаемых в~ВР 
ступени синхронного конвейера:
  $$
N_{\mathrm{ORS}} = \lambda TA_{\mathrm{ORS}}P_{\mathrm{ORS},1} P_{\mathrm{ORS},2}\,,
$$
где $A_{\mathrm{ORS}}$~--- площадь топологии ВР; $P_{\mathrm{ORS},1}$~--- вероятность того, что сбой 
не замаскируется КЧ следующей ступени и~попадет на вход ее ВР; $P_{\mathrm{ORS},2}$~--- 
вероятность того, что сбой успеет записаться в~ВР следующей ступени. 
В~соответствии со сделанными предположениями $P_{\mathrm{ORS},1}\hm= 0{,}5$ и~$P_{\mathrm{ORS},2}\hm = 1$. Тогда
$$
N_{\mathrm{ORS}} = 0{,}5\lambda TA_{\mathrm{ORS}}\,.
$$
%
  Поэтому  число критических сбоев в~одной ступени конвейера в~течение 
времени~$T$ равна 
$$
N_S = 0{,}5\lambda T\left(A_{\mathrm{CPS}} + A_{\mathrm{ORS}}\right).
$$


{ \begin{center}  %fig2
 \vspace*{-1pt}
    \mbox{%
\epsfxsize=78.658mm
\epsfbox{sok-2.eps}
}

\vspace*{10pt}

\noindent
{{\figurename~2}\ \ \small{
Индикаторная подсхема
}}
\end{center}}

\vspace*{6pt}

\setcounter{figure}{1}

  

  Аналогичным образом оценивается  число  критических ЛС из табл.~2 
в~ступени СС-кон\-вей\-ера в~течение времени~$T$. 
  
  Наименее чувствительными к~ЛС оказываются ИКЧ и~ИВР ступени 
конвейера. Индикаторная подсхема строится в~виде пирамиды. Рисунок~2 
иллюстрирует структуру индикаторной подсхемы для $N$ парафазных 
сигналов. Ее нижний ярус реализуется на ячейках XOR. С-эле\-мен\-ты на 
остальных ярусах пирамиды объединяют частичные индикаторные сигналы 
в~один индикаторный выход.
  

   
  С-элемент переключается в~то состояние, в~котором находятся все его входы 
одновременно~\cite{7-sok}. В~остальных случаях он хранит свое состояние. 
Однократный ЛС изменяет выход одного элемента в~индикаторной пирамиде. 
Критическим такой ЛС может стать только в~том случае, если он изменил 
состояние выхода С-эле\-мен\-та на вершине пирамиды. В~прочих случаях он 
маскируется остальными элементами индикаторной подсхемы.
  
  Анализ возможных случаев сбоя в~СС-кон\-вей\-ере~[4--6] позволяет 
получить следующие оценки  числа однократных критических сбоев за 
время~$T$:

\noindent
  \begin{equation}
  \left.
  \begin{array}{rl}
     N_{\mathrm{CPST}} & = 0{,}1\lambda T A_{\mathrm{CPST}};\\[6pt]
     N_{\mathrm{CIPST}} &= 0{,}25\lambda TP_{\mathrm{CIP},\mathrm{O}} A_{\mathrm{CIPST}};\\[6pt]
 N_{\mathrm{ORST}} & = 0{,}24\lambda TA_{\mathrm{ORST}};\\[6pt]
N_{\mathrm{RIPST}}  &= 0{,}25\lambda TP_{\mathrm{RIP,O}}A_{\mathrm{RIPST}},
\end{array}
\right\}
\label{e1-sok}
\end{equation}
где $N_{\mathrm{CPST}}$, $N_{\mathrm{CIPST}}$, $N_{\mathrm{ORST}}$ и~$N_{\mathrm{RIPST}}$~--- числа критических ЛС на 
интервале времени~$T$ в~КЧ, ИКЧ, ВР и~ИВР соответственно; $A_{\mathrm{CPST}}$, 
$A_{\mathrm{CIPST}}$, $A_{\mathrm{ORST}}$ и~$A_{\mathrm{RIPST}}$~--- площади их топологической реализации; 
$P_{\mathrm{CIP,O}}$ и~$P_{\mathrm{RIP,O}}$~--- вероятности наблюдения сбоя, появившегося 
в~индикаторной подсхеме, в~ее выходном  
С-эле\-мен\-те для ИКЧ и~ИВР соответственно. 

  Следующие рассуждения позволяют оценить значение вероятности 
$P_{\mathrm{CIP,O}}$ и~$P_{\mathrm{RIP,O}}$. Индикаторная подсхема $N$ парафазных сигналов 
в~комбинационной СС-схе\-ме включает $N$ ячеек XOR и~около $0{,}5N$ 
трехвходовых С-эле\-мен\-тов. Площадь XOR примерно вдвое меньше площади  
С-эле\-мен\-та. Разряд регистра содержит два двухвходовых С-эле\-мен\-та 
и~ячейку XOR в~качестве индикатора~\cite{5-sok}. Разрядность регистра 
ступени конвейера~$M$ в~общем случае отличается от числа парафазных 
сигналов~$N$ в~комбинационной части.
  
  Тогда площади выходных С-эле\-мен\-тов в~ИКЧ ($A_{\mathrm{CIP,O}}$) и~ИВР 
($A_{\mathrm{RIP,O}}$) равны 
  $$
  A_{\mathrm{CIP,O}}\approx \fr{A_{\mathrm{CIPST}}}{4N}\,;\enskip
  A_{\mathrm{RIP,O}}\approx \fr{A_{\mathrm{RIPST}}}{4M}
  $$
и вероятности наблюдения сбоя именно в~них
\begin{equation}
P_{\mathrm{CIP,O}}\approx\fr{1}{4N}\,;\enskip P_{\mathrm{RIP,O}}\approx \fr{1}{4M}\,.
\label{e2-sok}
\end{equation}
  
  Подставляя формулы~(2) в~(1), получаем оценки  числа однократных 
критических ЛС в~течение времени~$T$ в~индикаторных подсхемах 
комбинационной части и~регистра конвейера:
  \begin{align*}
  N_{\mathrm{CIPST}} &=\fr{1}{16N}\,\lambda TA_{\mathrm{CIPST}}\,;\\
  N_{\mathrm{RIPST}}&= \fr{1}{16M}\,\lambda T A_{\mathrm{RIPST}}\,.
  % \label{e3-sok}
  \end{align*}
  
  Пусть площадь топологии ВР ступени синхронного конвейера в~$K$~раз 
меньше площади КЧ. Отношение площадей синхронной и~самосинхронной схем 
соответствует отношению их аппаратных затрат. Аппаратная сложность 
комбинационных СС-схем вместе с~индикаторной подсхемой в~среднем 
в~2,7~раза больше, чем у синхронных аналогов. Регистр СС-кон\-вей\-ера 
сложнее своего синхронного аналога примерно в~1,5~раза.
  
  Для практических величин $M$ и~$N$ вероятностями $PT_{\mathrm{CIPST}}$ 
и~$PT_{\mathrm{RIPST}}$ можно пренебречь. Тогда оценки суммарного  числа 
однократных критических ЛС на интервале времени~$T$ в~ступени 
синхронного ($N_S$) и~самосинхронного кон\-вей\-ера ($N_{\mathrm{ST}}$):
  $$
N_S\approx 0{,}5\lambda T (K + 1)A_{\mathrm{ORS}};\enskip
N_{\mathrm{ST}}\approx0{,}32\lambda TK A_{\mathrm{ORS}},
$$
а их отношение:
$$
K_P= \fr{N_S}{N_{\mathrm{ST}}} \approx \fr{0{,}5(K+1)}{0{,}32 K}\,.
$$
При реальном значении $K$ от 0,5 до~4 СС-кон\-вей\-ер в~2,0--4,7~раз более 
устойчив к~ЛС, чем его синхронный аналог.
  
  Наиболее чувствительной к~ЛС частью цифровой схемы оказывается память, 
включая триггеры и~регистры. В СС-ре\-гист\-ре хранения на  
C-эле\-мен\-тах~\cite{6-sok} сбоеустойчивость может быть повышена путем 
обеспечения возможности восстановления корректного рабочего состояния 
в~сбойном разряде регистра, хранящем антиспейсер, в~рабочей фазе после 
окончания ЛС за счет незначительного усложнения схемы С-эле\-мен\-та 
(рис.~3) и~разряда регистра на его основе (рис.~4)~\cite{5-sok}. 

\setcounter{figure}{2}
\begin{figure*} %fig3
\vspace*{1pt}
\begin{center}  
\mbox{%
\epsfxsize=123.952mm
\epsfbox{sok-3.eps}
}
\end{center}
\vspace*{-9pt}
\Caption{Полустатический C-элемент, защищенный от <<залипания>> в~антиспейсере}
\end{figure*}




  Тогда число однократных критических ЛС на интервале времени~$T$ 
в~регистре ступени СС-кон\-вейера
$$
N_{\mathrm{ORST}} = 0{,}15\lambda TA_{\mathrm{ORST}}
$$
и число однократных критических ЛС в~ступени СС-кон\-вей\-ера уменьшается 
до 
$$
N_{\mathrm{ST}}\approx 0{,}28\lambda TKA_{\mathrm{ORS}}\,,
$$
а отношение $K_P$ числа однократных критических ЛС в~синхронной  
и~самосинхронной схе\-ме улучшается до~2,2--5,4. 

  Самым критичным узлом ступени СС-кон\-вей\-ера является индикаторный 
элемент, фор\-ми\-ру\-ющий сигнал управления регистром предыдущей ступени  
(С-эле\-мен\-ты на рис.~1). Его преждевременное переключение из-за ЛС 
способно вызвать <<зависание>> конвейера. Но его площадь пренебрежимо 
мала по сравнению с~площадью остальных частей СС-кон\-вей\-ера. Поэтому 
его влиянием можно пренебречь. К~тому же его реализация  
DICE-по\-доб\-ным~\cite{7-sok} С-эле\-мен\-том с~синфазными входами 
и~выходами~\cite{5-sok} позволяет кардинально решить проб\-ле\-му 
сбоеустойчивости за счет незначительного увеличения аппаратных затрат.
  
  Таким образом, СС-кон\-вей\-ер, индицирующий антиспейсерное состояние 
как спейсер и~использующий C-эле\-мен\-ты, защищенные от <<залипания>>\linebreak\vspace*{-12pt}

{ \begin{center}  %fig4
 \vspace*{-1pt}
    \mbox{%
\epsfxsize=68.952mm
\epsfbox{sok-4.eps}
}

\end{center}

\noindent
{{\figurename~4}\ \ \small{
Разряд регистра СС-конвейера, защищенный от <<залипания>> в~антиспейсере
}}}

\vspace*{9pt}

\setcounter{figure}{1}



\noindent 
в~антиспейсере, в~разряде регистра ступени, обес\-печивает в~2,2--5,4~раза 
б$\acute{\mbox{о}}$льшую устойчивость к~кратковременным однократным сбоям, чем его 
синхронный аналог. Это предопределяет целесообразность использования  
СС-схем в~качестве базиса для реализации сбоеустойчивых схем управления.
  
\section{Заключение}

  Благодаря двухфазной дисциплине работы и~избыточному кодированию 
сигналов СС-схе\-мы маскируют многие типы ЛС и~вследствие этого становятся 
предпочтительным базисом для реализации сбоеустойчивых систем 
управ\-ления.
  
  Комбинационная часть, регистр и~индикаторная подсхема ступени  
СС-кон\-вей\-ера в~разной степени влияют на уровень сбоеустойчивости 
конвейера. Ступень СС-кон\-вей\-ера устойчива к~90\%~ЛС, возникающих в~ее 
комбинационной части и~только к~76\%~ЛС в~выходном регистре. 
  
  Индикация антиспейсерного состояния как спейсера и~использование  
C-эле\-мен\-та, защищенного от <<залипания>> в~антиспейсере, в~разряде 
регистра ступени СС-кон\-вей\-ера обеспечивают сбоеустойчивость  
СС-кон\-вей\-ера в~2,2--5,4~раза выше уровня сбоеустойчивости синхронного 
конвейера.
  
  Дальнейшая работа будет посвящена исследованию методов и~средств 
обеспечения устойчивости СС-кон\-вей\-ера к~другим типам сбоев и~отказов.
  
{\small\frenchspacing
 {%\baselineskip=10.8pt
 %\addcontentsline{toc}{section}{References}
 \begin{thebibliography}{9}
  \bibitem{1-sok}
  \Au{Викторова В.\,С., Лубков~Н.\,В., Степанянц~А.\,С.} Анализ надежности 
отказоустойчивых управ\-ля\-ющих вы\-чис\-ли\-тель\-ных сис\-тем.~--- М.: ИПУ РАН, 2016. 120~с. 
%{\sf https://www.ipu.ru/sites/default/files/card\_file/VLS. pdf}.
  \bibitem{2-sok}
  \Au{Eaton P., Benedetto~J., Mavis~D., Avery~K., Sibley~M., Gadlage~M., Turflinger~T.} 
Single event transient pulse width measurements using a~variable temporal latch technique~// IEEE 
T.~Nucl. Sci., 2004. Vol.~51. No.\,6. P.~3365--3368. doi: 10.1109/TNS.2004.840020.
  \bibitem{3-sok}
  \Au{Varshavsky V., Kishinevsky~M., Marakhovsky~V., \textit{et. al.}} Self-timed control of 
concurrent processes: The design of aperiodic logical circuits in computers and discrete  
systems.~--- Dordrecht, Netherlands: Kluwer Academic Publs., 1990. 245~p.
  \bibitem{4-sok}
  \Au{Stepchenkov Y.\,A., Kamenskih~A.\,N., Diachenko~Y.\,G., Rogdestvenski~Y.\,V., 
Diachenko~D.\,Y.} Fault-tolerance of self-timed circuits~// 10th Conference (International) on 
Dependable Systems, Services and Technologies.~--- Piscataway, NJ, USA: IEEE, 
2019. P.~41--44. doi: 10.1109/ DESSERT.2019.8770047.
  \bibitem{5-sok}
  \Au{Stepchenkov Y.\,A., Kamenskih~A.\,N., Diachenko~Y.\,G., Rogdestvenski~Y.\,V., 
Diachenko~D.\,Y.} Improvement of the natural self-timed circuit tolerance to short-term soft 
errors~// Advances Science Technology Engineering Systems~J., 2020. Vol.~5. No.\,2. P.~44--56.
  \bibitem{6-sok}
  \Au{Stepchenkov Y., Rogdestvenski~Y., Kamenskih~A., Diachenko~Y., Diachenko~D.} 
Improvement of the quasi delay-insensitive pipeline noise immunity~// 11th Conference 
(International)  on Dependable Systems, Services and Technologies.~--- Piscataway, 
NJ, USA: IEEE, 2020. P.~47--51.

 
  \bibitem{8-sok} %7
  \Au{Monnet Y., Renaudin~M., Leveugle~R.} Hardening techniques against transient faults for 
asynchronous circuits~// 11th IEEE On-Line Testing Symposium (International).~--- Piscataway, 
NJ, USA: IEEE, 2005. P.~129--134. doi: 10.1109/IOLTS.2005.30.

 \bibitem{7-sok} %8
\Au{Danilov I.\,A., Gorbunov~M.\,S., Shnaider~A.\,I., \textit{et al.}} On board electronic devices safety provided by DICE-based 
Muller C-elements~// Acta Astronaut., 2018. Vol.~150. P.~28--32.
{\looseness=1

}
\end{thebibliography}

 }
 }

\end{multicols}

\vspace*{-10pt}

\hfill{\small\textit{Поступила в~редакцию 19.09.21}}

\vspace*{6pt}

%\pagebreak

%\newpage

%\vspace*{-28pt}

\hrule

\vspace*{2pt}

\hrule

%\vspace*{-2pt}

\def\tit{THE ELECTRONIC COMPONENT BASE\\ OF~FAILURE RESILIENCE DIGITAL CIRCUITS}


\def\titkol{The electronic component base of~failure resilience digital circuits}


\def\aut{I.\,A.~Sokolov$^1$, Yu.\,A.~Stepchenkov$^1$, Yu.\,G.~Diachenko$^1$, 
Yu.\,V.~Rogdestvenski$^1$,\\ and~A.\,N.~Kamenskih$^2$}

\def\autkol{I.\,A.~Sokolov, Yu.\,A.~Stepchenkov, Yu.\,G.~Diachenko, et al.} 
%Yu.\,V.~Rogdestvenski$^1$, and~A.\,N.~Kamenskih}

\titel{\tit}{\aut}{\autkol}{\titkol}

\vspace*{-15pt}


\noindent
$^1$Federal Research Center ``Computer Science and Control'' of the Russian 
Academy of Sciences, 44-2~Vavilov\linebreak
\hphantom{$^1$}Str., Moscow 119333, Russian Federation

\noindent
$^2$Perm National Research Polytechnic University, 29~Komsomol Prosp., Perm 
614990, Russian Federation


\def\leftfootline{\small{\textbf{\thepage}
\hfill INFORMATIKA I EE PRIMENENIYA~--- INFORMATICS AND
APPLICATIONS\ \ \ 2021\ \ \ volume~15\ \ \ issue\ 4}
}%
 \def\rightfootline{\small{INFORMATIKA I EE PRIMENENIYA~---
INFORMATICS AND APPLICATIONS\ \ \ 2021\ \ \ volume~15\ \ \ issue\ 4
\hfill \textbf{\thepage}}}

\vspace*{2pt} 
  



\Abste{The article presents the research of self-timed and synchronous 
circuits in terms of resilience to soft errors 
which can cause disruptions in the control system's 
operation of complex technical device. The use of a~fail-resilient 
self-timed code is proposed, which considers the antispacer state as the second spacer state. This approach increases 
the self-timed circuit's failure resilience level. In the first approximation, quantitative estimates show that the self-timed 
pipeline has a better failure resilience than the synchronous counterparts by 2.0--4.7 times. The use of modified  
C-element to implement the pipeline register bit increases this advantage to 2.2--5.4~times. Due to this, self-timed 
circuits are the preferred basis of failure resilient control systems implementation for complex technical equipment.}

\KWE{synchronous circuits; self-timed circuits; soft error; failure resilience; pipeline; transition completion indication; 
probability evaluation}

\DOI{10.14357/19922264210409}

%\vspace*{-20pt}

%\Ack
%\noindent




%\vspace*{6pt}

  \begin{multicols}{2}

\renewcommand{\bibname}{\protect\rmfamily References}
%\renewcommand{\bibname}{\large\protect\rm References}

{\small\frenchspacing
 {%\baselineskip=10.8pt
 \addcontentsline{toc}{section}{References}
 \begin{thebibliography}{9}
  \bibitem{1-sok-1}
\Aue{Viktorova, V.\,S., N.\,V.~Lubkov, and A.\,S.~Stepanyants.} 2016. \textit{Analiz 
nadezhnosti otkazoustoychivykh upravlyayushchikh vychislitel'nykh sistem}  
[Fault-tolerant control computer systems' reliability analysis]. Moscow: IPU RAN. 
120~p.
  \bibitem{2-sok-1}
\Aue{Eaton, P., J.~Benedetto, D.~Mavis, K.~Avery, M.~Sibley, M.~Gadlage, and 
T.~Turflinger.} 2004. Single event transient pulse width measurements using 
a~variable temporal latch technique. \textit{IEEE T.~Nucl. Sci.} 51(6):3365--3368. 
doi: 10.1109/TNS.2004.840020.
  \bibitem{3-sok-1}
\Aue{Varshavsky, V., M.~Kishinevsky, V.~Marakhovsky, \textit{et al.}} 1990.  
\textit{Self-timed control of concurrent processes:
The design of aperiodic logical circuits in computers and discrete  
systems}. Dordrecht, The Netherlands: 
Kluwer Academic Publs. 245~p.
  \bibitem{4-sok-1}
\Aue{Stepchenkov, Y.\,A., A.\,N.~Kamenskih, Y.\,G.~Diachenko, 
Y.\,V.~Rogdestvenski, and D.\,Y.~Diachenko.} 2019. Fault-tolerance of self-timed 
circuits. \textit{10th Conference (International) on Dependable Systems, Services, 
and Technologies Proceedings}. Piscataway, NJ: IEEE. 41--44. doi: 
10.1109/ DESSERT.2019.8770047.
  \bibitem{5-sok-1}
\Aue{Stepchenkov, Y.\,A., A.\,N.~Kamenskih, Y.\,G.~Diachenko, 
Y.\,V.~Rogdestvenski, and D.\,Y.~Diachenko.} 2020. Improvement of the natural 
self-timed circuit tolerance to short-term soft errors. \textit{Advances Science 
Technology Engineering Systems~J.} 5(2):44--56.
  \bibitem{6-sok-1}
\Aue{Stepchenkov, Y., Y.~Rogdestvenski, A.~Kamenskih, Y.~Diachenko, and 
D.~Diachenko.} 2020. Improvement of the quasi delay-insensitive pipeline noise 
immunity. \textit{11th Conference (International) on Dependable Systems, Services, 
and Technologies Proceedings}. Piscataway, NJ: IEEE. 47--51.

\bibitem{8-sok-1}
\Aue{Monnet, Y., M.~Renaudin, and R.~Leveugle.} 2005. Hardening techniques 
against transient faults for asynchronous circuits. \textit{11th International On-Line Testing Symposium 
Proceedings}. Piscataway, NJ: IEEE. 129--134. doi: 10.1109/ IOLTS.2005.30.
  \bibitem{7-sok-1}
\Aue{Danilov, I.\,A., M.\,S.~Gorbunov, A.\,I.~Shnaider, \textit{et al.}} 2018. On board electronic devices safety 
provided by DICE-based Muller C-elements. \textit{Acta Astronaut.} 150:28--32.

\end{thebibliography}

 }
 }

\end{multicols}

\vspace*{-12pt}

\hfill{\small\textit{Received September 19, 2021}}

\pagebreak

\Contr

\noindent
\textbf{Sokolov Igor A.} (b.\ 1954)~--- Doctor of Science in technology, Academician of RAS, 
Director, Federal Research Center ``Computer Science and Control'' of the Russian Academy of 
Sciences, 44-2~Vavilov Str., Moscow 119133, Russian Federation; \mbox{isokolov@ipiran.ru}

\vspace*{3pt}

\noindent
\textbf{Stepchenkov Yuri A.} (b.\ 1951)~--- Candidate of Science (PhD) in technology, leading 
scientist, Federal Research Center ``Computer Science and Control'' of the Russian Academy of 
Sciences, 44-2~Vavilov Str., Moscow 119133, Russian Federation; 
\mbox{YStepchenkov@ipiran.ru}

\vspace*{3pt}

\noindent
\textbf{Diachenko Yuri G.} (b.\ 1958)~--- Candidate of Science (PhD) in technology, senior 
scientist, Federal Research Center ``Computer Science and Control'' of the Russian Academy of 
Sciences, 44-2~Vavilov Str., Moscow 119133, Russian Federation; \mbox{diaura@mail.ru}

\vspace*{3pt}

\noindent
\textbf{Rogdestvenski Yuri V.} (b.\ 1952)~--- Candidate of Science (PhD) in technology, leading 
scientist, Federal Research Center ``Computer Science and Control'' of the Russian Academy of 
Sciences, 44-2~Vavilov Str., Moscow 119333, Russian Federation;  \mbox{YRogdest@ipiran.ru}

\vspace*{3pt}

\noindent
\textbf{Kamenskih Anton N.} (b.\ 1991)~--- Candidate of Science (PhD) in technology, associated 
professor, Perm National Research Polytechnic University, 29~Komsomol Prosp., Perm 614990, 
Russian Federation; \mbox{antoshkinoinfo@yandex.ru}


  
\label{end\stat}

\renewcommand{\bibname}{\protect\rm Литература} 