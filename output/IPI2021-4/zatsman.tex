\def\stat{Zatsman}

\def\tit{КОНЦЕПЦИЯ СОЗДАНИЯ ВОЗ-ЦЕНТРА КОМПЕТЕНЦИЙ ПО~ПАНДЕМИЯМ 
И~ЭПИДЕМИЯМ: КЛЮЧЕВЫЕ ПОНЯТИЯ И~ИХ~ТЕРМИНОЛОГИЧЕСКИЙ АНАЛИЗ}

\def\titkol{Концепция создания ВОЗ-центра компетенций по пандемиям 
и~эпидемиям: ключевые понятия и~их %терминологический 
анализ}

\def\aut{И.\,М.~Зацман$^1$}

\def\autkol{И.\,М.~Зацман}

\titel{\tit}{\aut}{\autkol}{\titkol}

\index{Зацман И.\,М.}
\index{Zatsman I.\,M.}


%{\renewcommand{\thefootnote}{\fnsymbol{footnote}} \footnotetext[1]
%{Работа выполнена при частичной поддержке РФФИ (проект 20-07-00046~А).}}

\renewcommand{\thefootnote}{\arabic{footnote}}
\footnotetext[1]{Федеральный исследовательский центр <<Информатика и~управление>> 
Российской академии наук, \mbox{izatsman@yandex.ru}}

%\vspace*{11pt}

% \vspace*{-2pt}

\Abst{Рассматривается концепция автоматизированной системы, создаваемой по инициативе 
Всемирной организации здравоохранения (ВОЗ) и~получившей название {ВОЗ-центр 
компетенций по пандемиям и~эпидемиям} (далее~--- ВОЗ-центр). В~описании концепции 
используются такие англоязычные термины информатики, как data, information и~knowledge 
(в~словосочетаниях knowledge sharing, knowledge representation, knowledge exchange 
и~knowledge generation). Понимание этой концепции как основы создания ВОЗ-центра во 
многом будет определяться их трактовкой, соответствующей ее общему смыс\-лу. 
Необходимость создания подобных систем на национальном, региональном и~глобальном 
уровнях обоснована в~мае 2021~г.\ в~отчете Международной комиссии экспертов, учрежденной 
ВОЗ, что придает актуальность анализу системы терминов концепции не только для создания 
ВОЗ-центра. Основная цель статьи состоит в~том, чтобы провести анализ основных 
концептуальных положений его создания и~предложить трактовку отдельных ключевых 
терминов концепции и~их русскоязычные переводные эквиваленты, соответствующие ее 
общему смыслу. При этом показано, что для понимания этой концепции необходимо также 
выяснить значения и~таких англоязычных терминов, как intelligence, context и~insight. В~случае 
перевода концепции на русский язык нужно также найти их русскоязычные переводные 
эквиваленты согласно контекстам их использования.}

\KW{компетенции; данные; информация; знание; концепт; контекст; представление; 
терминологический анализ; среды предметной области информатики}

\DOI{10.14357/19922264210414}

\vspace*{-1pt}


\vskip 10pt plus 9pt minus 6pt

\thispagestyle{headings}

\begin{multicols}{2}

\label{st\stat}

\section{Введение}

Первого сентября 2021~г.\ в~Берлине был основан WHO Hub for Pandemic and Epidemic 
Intelligence~--- \textit{ВОЗ-центр компетенций по пандемиям 
и~эпидемиям}\footnote[2]{Приводится авторский перевод названия.}~[1]. Концепция его 
создания опубликована на сайте ВОЗ~[2]. При этом отмечено, что она будет 
регулярно пересматриваться по мере развития ВОЗ-центра. В~русскоязычных 
СМИ, освещавших это событие, дано несколько переводов с~английского языка 
названия этого центра, которые отличаются от предлагаемого в~статье. 
Использовались, например, следующие варианты перевода:
\begin{itemize}
\item Центр \textit{исследований} пандемий~[3];
\item Центр по \textit{предотвращению} будущих пандемий~[4];
\item Центр \textit{информации} о~пандемиях и~эпидемиях~[5];
\item Центр по пандемической и~эпидемической \textit{разведке}~[6];
\item Центр раннего \textit{предупреждения} пандемии (Глобальный центр сбора 
информации о~пандемиях и~эпидемиях) [7].
\end{itemize}

Приведенный спектр русскоязычных названий, вероятно, говорит о~трудностях 
перевода на русский язык термина intelligence в~названии WHO Hub for Pandemic 
and Epidemic Intelligence. Это и~дало несколько столь разных переводов со 
словами \textit{исследования, предотвращение, информация, разведка} 
и~\textit{предупреждение}. Приведенный ниже терминологический анализ 
концепции позволяет предложить еще один вариант перевода~--- ВОЗ-центр 
\textit{компетенций} по пандемиям и~эпидемиям.

Цель статьи состоит в~том, чтобы (1)~провести анализ основных концептуальных 
положений создания ВОЗ-центра и~(2)~предложить трактовку отдельных 
ключевых терминов концепции и~их русскоязычные переводные эквиваленты, 
соответствующие ее общему смыслу.

\vspace*{-7pt}

\section{Терминологический анализ} %2

\vspace*{-2pt}

Термин intelligence на 16~страницах концепции встречается 74~раза, включая 
словосочетания artificial intelligence, intelligence approaches, intelligence solutions, 
intelligence ecosystem и~collaborative in-\linebreak\vspace*{-12pt}

\pagebreak

%\begin{table*}
{\small %tabl1


\begin{center}
\vspace*{-6pt}

%\Caption{Число словоупотреблений ключевых терминов концепции}
%\vspace*{2ex}

%\tabcolsep=4pt
\begin{tabular}{|l|c|}
\multicolumn{2}{p{70mm}}{Количество словоупотреблений ключевых терминов концепции}\\
\multicolumn{2}{c}{\ }\\[-6pt]
\hline
\multicolumn{1}{|c|}{Термин концепции}&
\tabcolsep=0pt\begin{tabular}{c}Количество\\ словоупотреблений\end{tabular}\\
\hline
intelligence&74\\
\hline
data&56\\
\hline
information&20\\
\hline
knowledge&\hphantom{1}5\\
\hline
\tabcolsep=0pt\begin{tabular}{l}context (contextual,\\ contextualized)\end{tabular}&\hphantom{1}7\\
\hline
insight&17\\
\hline
\end{tabular}
\end{center}
}
%\end{table*}

\vspace*{9pt}

\noindent
telligence, что почти в~3~раза чаще, чем 
слова information и~knowledge вместе взятые. Понимание основных положений 
концепции во многом будет обусловлено смысловым содержанием понятий, 
обозначенных словом intelligence, другими ее ключевыми словами, перечисленными 
в~таблице, а~также словосочетаниями с~ними.

Согласно таблице, при описании концепции используются такие традиционные 
англоязычные термины информатики, как data, information и~knowledge 
в~словосочетаниях knowledge sharing, knowledge representation, knowledge 
exchange\linebreak и~knowledge generation. Для понимания этой концепции как основы 
создания ВОЗ-цент\-ра, ана\-ло\-гич\-ных на\-цио\-наль\-ных и~ре\-гио\-наль\-ных 
информационных цент\-ров необходимо определить \mbox{кон\-текст\-ные} значения еще 
и~таких анг\-ло\-языч\-ных \mbox{терминов}, как intelligence, context и~insight.

Одно из широко распространенных значений термина <<context>> 
в~гуманитарных науках достаточно полно передано дефиницией 
в~Лингвистическом энциклопедическом словаре: контекст~--- это <<фрагмент 
текста, включающий избранную для анализа единицу, необходимый 
и~достаточный для определения значения этой единицы, являющегося 
непротиворечивым по отношению к~\textit{общему смыслу текста} (курсив  
мой.~--- И.\,З.)>>~[8]. Слово <<context>> используется в~концепции в~другом 
значении: \textit{обстановка, окружение, обстоятельства, условия или процессы}, 
которые оказали влияние на формирование данных и/или информации. Для 
передачи этого значения в~переводах положений концепции слово <<контекст>> 
не будет использоваться. Оно резервируется только для описания значений 
ключевых ее терминов, которые будут определяться на основе результатов 
терминологического анализа их \textit{контекстов} и~общего смысла концепции.

Начнем с~термина intelligence, который встречается в~74~контекстах. Приведем 
переводы двух положений концепции\footnote{Приводится авторский перевод на русский 
язык этого и~последующих положений.}, указывая страницы согласно~[2].

%\smallskip

I.~С.~2: ``We need the capability to analyse and interpret data so that they become 
useful information, and we need to understand the context of that information to turn it 
into the intelligence that policy- and decision-makers need for action.''

`Мы должны быть способны анализировать и~интерпретировать \textit{данные} 
таким образом, чтобы они становились полезной \textit{информацией}, и~нам надо 
понимать \textit{условия} формирования этой \textit{информации}, чтобы превратить 
ее в~ \textit{компетенции}, необходимые для действий политикам и~лицам, 
принимающим решения'.

В~этом положении при переводе термина intelligence можно было бы 
использовать и~слово <<знания>>. Однако в~следующем положении концепции 
термины intelligence и~knowledge будут присутствовать в~разных значениях. 
Поэтому его контекст не дает возможности использовать здесь слово <<знания>> 
при переводе intelligence. Отметим, что термин <<context>> здесь переведен как 
\textit{условия} формирования.

%\smallskip

II.~С.~7: ``$\ldots$the WHO Hub will develop knowledge-exchange programmes that 
model its inclusive and multidisciplinary approach to collaborative intelligence and are 
specifically tailored to bridge the gap between knowledge generation and pandemic and 
epidemic intelligence decision-making.''

`$\ldots$ВОЗ-центр разработает программы обмена \textit{знаниями}, которые 
станут основой всеобъемлющего и~междисциплинарного подхода 
к~формированию коллективных \textit{компетенций} и~преодоления разрыва 
между накоплением \textit{знаний} и~принятием решений на основе 
\textit{компетенций} в~области пандемий и~эпидемий'.

Приведенное положение говорит, с~одной стороны, об отличающихся значениях 
терминов <<knowledge>> и~<<intelligence>> в~концепции, что требует двух 
разных русских переводных эквивалентов (в~статье предлагается перевести 
<<intelligence>> как \textit{компетенции}), с~другой стороны~--- о~наличии разрыва 
между \textit{накоплением знаний} и~\textit{принятием решений на основе 
компетенций}, а~также о~необходимости разработки специального подхода для 
его преодоления. Один из возможных подходов~--- это накопление знания 
в~процессе его генерации, ориентированного на достижение заданной цели, 
получивший название \textit{целенаправленная генерация знаний}~[9--12].

Термин insight встречается в~17~контекстах. Приведем переводы двух 
положений, указывая страницы согласно~[2].

%\smallskip

III.~С.~2: ``Existing disease surveillance systems are mostly limited to health data. 
Those data are increasingly fragmented, making it difficult to connect them using 
existing systems and approaches. Surveillance data provide limited insight into the 
context from which the data are derived, which limits our understanding and ability to 
take effective action.''

`Существующие системы мониторинга заболеваний в~основном ограничены 
\textit{данными} о~состоянии здоровья. Эти \textit{данные} становятся все более 
фрагментированными и~ненормализованными, что затрудняет их интеграцию 
с~использованием существующих систем и~подходов. \textit{Данные}, полученные 
в~результате мониторинга, дают ограниченное \textit{представление} о~тех 
\textit{условиях}, в~которых они были получены, что ограничивает наше 
понимание и~способность принимать эффективные меры'.

В~приведенном положении предлагается использовать слово <<представление>> 
как переводной эквивалент для <<insight>> в~этом контексте.

%\smallskip

IV.~С.~3: ``Pandemic and epidemic intelligence is a~core function of public health to 
generate actionable insights for decision-making to protect and improve the health of 
populations.''

`Основная миссия системы здравоохранения в~области пандемий и~эпидемий 
состоит в~развитии \textit{компетенций} в~интересах создания 
\textit{представлений}, имеющих большое практическое значение для принятия 
решений по защите и~улучшению здоровья населения'.

Из положений~I, II и~IV можно извлечь следующую последовательность 
технологических трансформаций\footnote{Используемые в~работах~[13; 14; 15, с.~8; 16] 
словосочетания <<информационные трансформации>>, <<процессы информационных 
трансформаций>>, <<трансформации представлений>> и~<<трансформации контента>> при описании 
основного объекта предметной области информатики как компьютинга считаются синонимами термина 
<<технологические трансформации>>, используемого \mbox{в~статье}.} в~концепции ВОЗ-центра: 
в~результате анализа и~интерпретации \textit{данных} формируется полезная 
\textit{информация}, а~на ее основе и~в~процессе обмена \textit{знаниями} 
создаются коллективные \textit{компетенции} в~интересах создания 
\textit{представлений}, необходимых лицам, принимающим решения.

\vspace*{-7pt}

\section{Технологические трансформации} %3

\vspace*{-2pt}

Технологические трансформации как основной объект предметной области 
информатики согласно~[13; 14; 15, с.~8; 16] рассмотрим в~парадигме ее сред, 
изложенной в~работах [17--21]. Это даст возможность предложить распределение 
пяти базовых сущностей концепции (\textit{данных, информации, знания, 
компетенций и~представлений}) по ментальной, информационной, цифровой 
средам информатики\footnote{Из пяти сред информатики (ментальной, информационной, 
цифровой, ДНК- и~нейросреды) технологические трансформации рассматриваются в~статье только 
в~первых трех средах. Их рассмотрение во всех пяти средах не ставится целью данной статьи.} 
и,~соответственно, позиционировать технологические трансформации этих 
сущностей в~средах или на границах между этими тремя средами (см.\ рисунок).

Предлагается следующее распределение пяти базовых сущностей и~их 
компьютерных кодов по трем средам:
\begin{enumerate}[(1)]
\item ментальная среда (\textit{знание} экспертов, их \textit{компетенции} 
и~\textit{представления}; ментальные образы данных);
\item информационная среда (\textit{информация} как от\-чуж\-ден\-ная форма 
представления знания или смыс\-ла, компетенций и~представлений;  
сен\-сор\-но-вос\-при\-ни\-ма\-емые человеком \textit{данные});
\item цифровая среда (компьютерные коды информации, знания, компетенций 
и~представлений; цифровая информация и~цифровые данные).
\end{enumerate}

Объекты каждой из трех сред, приведенные в~скобках, распадаются на два класса, 
что графически показано точкой с~запятой. Такое деление основано на дихотомии 
источников происхождения (генерации) этих объектов~[22]:
\begin{itemize}
\item человек как генератор семантической (знаковой) \textit{информации};
\item материальная сфера как источник сен\-сор\-но-вос\-при\-ни\-ма\-емых 
человеком \textit{данных} и~технические системы как генераторы \textit{цифровых 
данных}.
\end{itemize}

Существенная особенность используемой парадигмы информатики в~том, что 
каждая из трех сред представляется как замкнутая, т.\,е.\ ее сущности вне 
зависимости от класса не могут пересекать границу с~соседней средой и~не 
может быть сущностей, принадлежащих одновременно двум или более средам. 
В~этой парадигме технологические трансформации пяти базовых сущностей 
могут иметь место как на границах между средами (обозначены тремя кругами 
и~двумя овалами), так и~внутри них (обозначены одним треугольником 
с~цифрой~<<4>> и~тремя стрелками с~двойным контуром и~цифрами~<<1>>, 
<<2>> и~<<3>>) (см.\ рисунок).



Данные мониторинга служат одним из основных источников генерации остальных 
четырех базовых сущностей. С~одной стороны, они оцифровываются для 
хранения в~цифровой среде (эта их трансформация обозначена правым кругом 
<<8~Кодовые таблицы>>). С~другой стороны, они подлежат
 концептуализации, 
т.\,е.\ извлечению из них смысла
(обозначено стрелкой с~двойным контуром 
и~циф\-рой <<1>>). Перед началом концептуализации данные
порождают их 
ментальные образы в~головах экс-\linebreak\vspace*{-12pt}

\pagebreak

\end{multicols}

\begin{figure*}
\vspace*{3pt}
\begin{center}  
\mbox{%
\epsfxsize=163mm
\epsfbox{zac-1.eps}
}
\vspace*{9pt}

{\small Технологические трансформации пяти базовых сущностей концепции}
\end{center}
\vspace*{-12pt}
\end{figure*}

\begin{multicols}{2}

\noindent
  пертов (обозначено овалом <<7~Сенсорное 
восприятие>>).

На основе смыслового содержания данных генерируется информация 
о~результатах мониторинга (обозначено кругом <<6~Знаки (семиотические)>>), 
которая оциф\-ро\-вы\-ва\-ет\-ся (левый круг <<8~Кодовые таблицы>>) и~объединяется 
в~циф\-ро\-вой среде с~информацией, извлеченной из других источников (статьи, 
патенты, отчеты о~результатах исследований и~др., которые на рисунке не 
показаны).

Одновременно смысловое содержание данных интегрируется с~уже имеющимся 
знанием в~предметной области ВОЗ-центра (стрелка с~двойным контуром 
и~циф\-рой~<<2>>). Накопление знаний и~обмен ими служат основой 
формирования коллективных компетенций (стрелка с~двойным контуром 
и~циф\-рой~<<3>>). Интеграция накопленных знаний и~коллективных 
компетенций служит источником генерации представлений, необходимых лицам, 
принимающим решения (ЛПР) в~предметной области ВОЗ-центра (треугольник 
с~циф\-рой~4). Накопленное знание, коллективные компетенции и~представления 
ЛПР с~учетом их изменений во времени служат основой создания динамических 
классификационных схем~[23--25], тезаурусов, таксономий и~других видов 
онтологий~[26]. Последняя трансформация обозначена овалом 
<<5~Онтологическое представление>>. Отметим, что в~концепции заявлена 
необходимость использования целе\-на\-прав\-лен\-но формируемых таксономий 
и~онтологий~[2, с.~10].

В~заключение этого раздела зафиксируем, что позиционирование пяти базовых 
сущностей концепции на основе парадигмы сред предметной области 
информатики позволяет выделить восемь видов трансформаций в~трех средах 
и~на границах между ними, из которых только две последние генерируют коды 
циф\-ро\-вой среды:
\begin{enumerate}[(1)]
\item порождение ментальных образов данных в~процессе их сенсорного 
восприятия;
\item концептуализация, т.\,е.\ извлечение смысла из данных;
\item интеграция извлеченного из данных смыслового содержания с~уже 
имеющимся знанием;
\item формирование коллективных компетенций (intelligence);
\item генерация представлений, необходимых ЛПР (insights);
\item знаковое представление смыслового содержания данных в~информационной 
среде;
\item онтологическое представление в~\textit{цифровой среде} накопленного 
знания, коллективных компетенций и~представлений ЛПР с~учетом их изменений 
во времени;
\item кодирование данных и~знаковой информации в~\textit{цифровой среде}.
\end{enumerate}

Приведенный перечень трансформаций говорит о~том, что реализация этой 
концепции, \mbox{скорее} всего, предполагает создание автоматизированной системы  
ис\-кусст\-вен\-но-ес\-тест\-вен\-но\-го интеллекта~[27], так как в~концепции 
говорится, что обмен знанием будет способствовать эффективному гибридному 
принятию решений на основе симбиоза искусственного и~естественного 
интеллекта~[2, с.~5].

\vspace*{-9pt}

\section{Заключение} %4

\vspace*{-3pt}

Приведенный терминологический анализ концепции является авторским и, 
конечно, не претендует на полноту. Например, за рамками \mbox{статьи}\linebreak остался целый 
ряд словосочетаний со словами data, information, knowledge, intelligence, context 
и~insight, значения которых в~концепции заслуживают отдельного рассмотрения. 
Кроме того, из \mbox{пяти} сред информатики (ментальной, информационной, циф\-ро\-вой, 
ДНК- и~нейросреды) технологические трансформации были рас\-смот\-ре\-ны только 
в~первых трех средах. Включение в~рассмотрение ДНК-сре\-ды расширило бы 
список технологических трансформаций, добавив в~базовые сущности 
\textit{биоинформационные последовательности}, полученные в~процессе 
секвенирования генома вирусов. Это дало бы возможность включить данные 
о~мутациях вирусов и~информацию об их последствиях в~общую схему 
технологических трансформаций базовых сущностей концепции.

  \vspace*{-9pt}
  
  
{\small\frenchspacing
 {\baselineskip=10.75pt
 %\addcontentsline{toc}{section}{References}
 \begin{thebibliography}{99}
 
 \vspace*{-3pt}

\bibitem{1-zac}
\Au{Хачатрян К.} ВОЗ открыла в~Берлине Центр предупреждения пандемий. {\sf 
https://ru.euronews.com/ 2021/09/01/hub-for-pandemic}.
\bibitem{2-zac}
WHO Hub for Pandemic and Epidemic Intelligence. Strategy Paper. 16~p. {\sf 
https://cdn.who.int/media/docs/\linebreak default-source/2021-dha-docs/who\_hub.pdf?sfvrsn=8d\linebreak c28ab6\_5}.
\bibitem{3-zac}
Всемирная организация здравоохранения основала центр исследований пандемий~// BNN, 2021. 
{\sf https://\linebreak bnn-news.ru/vsemirnaya-organizacziya-zdravoohraneniya-osnovala-czentr-issledovanij-pandemij-252987}.
\bibitem{4-zac}
Центр по предотвращению будущих пандемий открылся в~Берлине~// НСН, 2021. {\sf https://nsn.fm/in-the-world/tsentr-po-predotvrascheniu-buduschih-pandemii-otkrylsya-v-berline}.
\bibitem{5-zac}
ВОЗ открыла в~Берлине Центр по предотвращению будущих пандемий~// Интерфакс, 
 2021. {\sf https:// www.interfax.ru/world/788001}.
\bibitem{6-zac}
ВОЗ открыла в~Берлине Центр по пандемической и~эпидемической разведке~// ИА Красная 
Весна, 2021. {\sf https://rossaprimavera.ru/news/ cdf75c9f}.
\bibitem{7-zac}
Берлин становится глобальным центром борьбы с~пандемиями~// Aussiedlerbote, 2021. 
{\sf https://\linebreak aussiedlerbote.de/2021/09/berlin-stanovitsya-globalnym-centrom-borby-s-pandemiyami}.
\bibitem{8-zac}
Лингвистический энциклопедический словарь~/ Под ред. В.\,Н.~Ярцевой.~--- М.: Советская 
энциклопедия, 1990. 685~с. 
\bibitem{9-zac}
\Au{Зацман И.\,М.} Стадии целенаправленного извлечения знаний, имплицированных 
в~параллельных текстах~// Системы и~средства информатики, 2018. Т.~28. №\,3. С.~175--188.
\bibitem{10-zac}
\Au{Зацман И.\,М.} Целенаправленное развитие систем лингвистических знаний: выявление 
и~заполнение лакун~// Информатика и~её применения, 2019. Т.~13. Вып.~1. С.~91--98.
\bibitem{11-zac}
\Au{Zatsman I.} Finding and filling lacunas in linguistic typologies~// 15th  Forum  (International) on 
Knowledge Asset Dynamics Proceedings.~--- Matera, Italy: Institute of Knowledge Asset 
Management, 2020. P.~780--793.
\bibitem{12-zac}
\Au{Zatsman I.} A model of goal-oriented knowledge discovery based on human--computer 
symbiosis~// 16th Forum (International) on Knowledge Asset Dynamics Proceedings.~--- Rome, Italy: 
Arts for Business Institute, 2021. P.~297--312.
\bibitem{14-zac} %13
\Au{Denning P., Rosenbloom~P.} Computing: The fourth great domain of science~// 
Commun. ACM, 2009. Vol.~52. No.\,9. P.~27--29.
\bibitem{15-zac} %14
\Au{Denning P., Freeman~P.} Computing's paradigm~// Commun. ACM, 2009. Vol.~52. No.\,12. P.~28--30.
\bibitem{13-zac} %15
\Au{Rosenbloom P.} On computing: The fourth great scientific domain.~--- Cambridge, MA, USA: 
MIT Press, 2013. 307~p.
\bibitem{16-zac}
\Au{Зацман И.\,М.} Методология обратимой генерализации в~контексте классификации 
информационных трансформаций~// Системы и~средства информатики, 2018. Т.~28. №\,2. 
С.~128--144.
\bibitem{17-zac}
\Au{Зацман И.\,М.} Построение системы терминов ин\-фор\-ма\-ци\-он\-но-компьютерной науки:  
проб\-лем\-но-ори\-ен\-ти\-ро\-ван\-ный подход~// Тео\-рия и~практика общественной научной 
информации, %Сб. науч. тр.~--- М.: ИНИОН РАН, 
2013. Вып.~21. С.~120--159.
\bibitem{18-zac}
\Au{Зацман И.\,М.} Таблица интерфейсов информатики как  
ин\-фор\-ма\-ци\-он\-но-ком\-пью\-тер\-ной науки~// На\-уч\-но-тех\-ни\-че\-ская информация. 
Сер.~1: Организация и~методика информационной работы, 2014. №\,11. С.~1--15.
\bibitem{19-zac}
\Au{Зацман И.\,М.} Процессы целенаправленной генерации и~развития кросс-язы\-ко\-вых 
экспертных знаний: семиотические основания моделирования~// Информатика и~её 
применения, 2015. Т.~9. Вып.~3. С.~106--123.

\pagebreak

\bibitem{20-zac}
\Au{Зацман И.\,М.} Интерфейсы третьего порядка в~информатике~// Информатика и~её 
применения, 2019. Т.~13. Вып.~3. С.~82--89.
\bibitem{21-zac}
\Au{Зацман И.\,М.} Кодирование концептов в~цифровой среде~// Информатика и~её 
применения, 2019. Т.~13. Вып.~4. С.~97--106.
\bibitem{22-zac}
\Au{Шемакин Ю.\,И., Романов~А.\,А.} Компьютерная семантика.~--- М.: Школа 
Китайгородской, 1995. 344~с.
\bibitem{23-zac}
\Au{Гончаров А.\,А., Зацман~И.\,М., Кружков~М.\,Г.} Темпоральные данные 
в~лексикографических базах знаний~// Информатика и~её применения, 2019. Т.~13. Вып.~4.  
С.~90--96.

%\columnbreak

\bibitem{24-zac}
\Au{Гончаров А.\,А., Зацман~И.\,М., Кружков~М.\,Г.} Эволюция классификаций в~надкорпусных 
базах данных~// Информатика и~её применения, 2020. Т.~14. Вып.~4. С.~108--116.
\bibitem{25-zac}
\Au{Гончаров~А.\,А., Зацман~И.\,М., Кружков~М.\,Г.} Пред\-став\-ле\-ние новых 
лексикографических знаний в~ди\-на\-ми\-че\-ских классификационных системах~// Информатика 
и~её применения, 2021. Т.~15. Вып.~1.\linebreak С.~86--93.
{ %\looseness=1

}
\bibitem{26-zac}
\Au{McGuinness~D.\,L.} Ontologies come of age~// Spinning the Semantic Web: Bringing the World 
Wide Web to its full potential~/ Eds. D.~Fensel, J.~Hendler, H.~Lieberman, W.~Wahlster.~--- 
Cambridge, MA, USA: MIT Press, 2003. P.~171--194.
\bibitem{27-zac}
\Au{Зацман И.\,М.} Проб\-лем\-но-ори\-ен\-ти\-ро\-ван\-ная актуализация словарных статей 
двуязычных словарей и~медицинской терминологии: сопоставительный анализ~// Информатика 
и~её применения, 2021. Т.~15. Вып.~1. С.~94--101.

\end{thebibliography}

 }
 }

\end{multicols}

\vspace*{-9pt}

\hfill{\small\textit{Поступила в~редакцию 15.10.21}}

\vspace*{6pt}

%\pagebreak

%\newpage

%\vspace*{-28pt}

\hrule

\vspace*{2pt}

\hrule

\vspace*{-2pt}

\vspace*{4pt}

\def\tit{THE CONCEPTION OF~CREATING WHO HUB FOR PANDEMIC AND~EPIDEMIC 
INTELLIGENCE:\\ KEYWORDS AND~THEIR TERMINOLOGICAL ANALYSIS}

\def\titkol{The conception of creating WHO Hub for Pandemic and Epidemic 
Intelligence: Keywords and their terminological analysis}

\def\aut{I.\,M.~Zatsman}

\def\autkol{I.\,M.~Zatsman}

\titel{\tit}{\aut}{\autkol}{\titkol}

\vspace*{-17pt}

\noindent
Federal Research Center ``Computer Science and Control'' of the Russian Academy of Sciences,  
44-2~Vavilov Str., Moscow 119333, Russian Federation

\def\leftfootline{\small{\textbf{\thepage}
\hfill INFORMATIKA I EE PRIMENENIYA~--- INFORMATICS AND
APPLICATIONS\ \ \ 2021\ \ \ volume~15\ \ \ issue\ 4}
}%
 \def\rightfootline{\small{INFORMATIKA I EE PRIMENENIYA~---
INFORMATICS AND APPLICATIONS\ \ \ 2021\ \ \ volume~15\ \ \ issue\ 4
\hfill \textbf{\thepage}}}

\vspace*{2pt} 

\Abste{The conception of the computerized system which is being created on the initiative of the 
World Health Organization (WHO) and called WHO Hub for Pandemic and Epidemic Intelligence is 
considered. In the description of the conception, words ``data,'' ``information,'' and ``knowledge'' are 
used (in keywords ``knowledge sharing,'' ``knowledge representation,'' ``knowledge exchange,'' and 
``knowledge generation''). The understanding of this conception as the basis for the creation of the 
WHO Hub will largely be determined by their interpretation corresponding to its contextual meaning. 
The need to create such systems at the national, regional, and global levels was justified in May 2021 in 
the report of the International Commission of Experts established by WHO, which gives relevance to 
the analysis of the system of terms of the conception not only for the creation of the WHO Hub. The 
main aim of the paper is to analyze the principal conceptual statements of WHO Hub creation and to 
propose an interpretation of their keywords and their Russian-language translation equivalents 
corresponding to its contextual meaning. At the same time, it is shown that in order to understand this 
conception, it is also necessary to find out the meanings of such English-language terms as 
``intelligence,'' ``context,'' and ``insight.'' In the case of translation of the conception into Russian, it is 
also necessary to find their Russian-language translation equivalents according to the contexts of their 
use.}

\KWE{intelligence; data; information; knowledge; concept; context; insight; terminological analysis; 
informatics media}

\DOI{10.14357/19922264210414}

%\vspace*{-20pt}

%\Ack
%\noindent


\vspace*{-4pt}

  \begin{multicols}{2}

\renewcommand{\bibname}{\protect\rmfamily References}
%\renewcommand{\bibname}{\large\protect\rm References}

{\small\frenchspacing
 {\baselineskip=10.65pt
 \addcontentsline{toc}{section}{References}
 \begin{thebibliography}{99}
 
 \vspace*{-3pt}

\bibitem{1-zac-1}
\Aue{Khachatryan, K.} VOZ otkryla v Berline \mbox{Tsentr} pre\-du\-prezh\-de\-niya pandemiy [WHO has opened 
the WHO Hub for Pandemic and Epidemic Intelligence in Berlin]. Available at: {\sf 
https://ru.euronews.com/2021/09/01/hub-for-\linebreak pandemic} (accessed November~18, 2021).
\bibitem{2-zac-1}
WHO Hub for Pandemic and Epidemic Intelligence. Strategy Paper. 16~p. Available at: {\sf 
https://cdn.who.int/\linebreak
 media/docs/default-source/2021-dha-docs/who\_hub. pdf?sfvrsn=8dc28ab6\_5} 
(accessed November~18, 2021).

\columnbreak

\bibitem{3-zac-1}
Vsemirnaya organizatsiya zdravookhraneniya osnovala tsentr issledovaniy pandemiy [The World 
Health Organization has established the WHO Hub for Pandemic and Epidemic Intelligence]. 2021. 
Available at: {\sf  
https://bnn-news.ru/vsemirnaya-organizacziya-zdravoohraneniya-osnovala-czentr-issledovanij-pandemij-252987} (accessed 
November~18, 2021).
\bibitem{4-zac-1}
Tsentr po predotvrashcheniyu budushchikh pandemiy otkrylsya v Berline [The WHO Hub for Pandemic 
and Epidemic Intelligence opened in Berlin]. 2021. \textit{NSN}.
Available at: {\sf  
https://nsn.fm/in-the-world/tsentr-po-predotvrascheniu-buduschih-pandemii-otkrylsya-v-berline} 
(accessed November~18, 2021).
\bibitem{5-zac-1}
VOZ otkryla v~Berline Tsentr po predotvrashcheniyu budushchikh pandemiy [WHO has opened the 
WHO Hub for Pandemic and Epidemic Intelligence in Berlin]. 2021. \textit{Interfax}. Available at: 
{\sf https://www.interfax.ru/\linebreak world/788001} (accessed November~18, 2021).
\bibitem{6-zac-1}
VOZ otkryla v Berline Tsentr po pandemicheskoy i~epidemicheskoy razvedke [WHO has opened the 
WHO Hub for Pandemic and Epidemic Intelligence in Berlin]. 2021. \textit{Red Spring News Agency}.
Available at: {\sf 
https://\linebreak rossaprimavera.ru/news/cdf75c9f} (accessed November~18, 2021).
\bibitem{7-zac-1}
Berlin stanovitsya global'nym tsentrom bor'by s pandemiyami [Berlin is becoming a~global center for 
combating pandemics]. 2021. \textit{Aussiedlerbote}. Available at: {\sf  
https://aussiedlerbote.de/2021/09/berlin-\linebreak stanovitsya-globalnym-centrom-borby-s-pandemiyami/} 
(accessed November~18, 2021).
\bibitem{8-zac-1}
Yartseva, V.\,N., ed. 1990. \textit{Lingvisticheskiy entsiklopedicheskiy slovar'} [Linguistic encyclopedic 
dictionary]. Moscow: Soviet Encyclopedia. 685~p.
\bibitem{9-zac-1}
\Aue{Zatsman, I.} 2018. Stadii tselenapravlennogo izvlecheniya znaniy, implitsirovannykh 
v~parallel'nykh tekstakh [Stages of goal-oriented discovery of knowledge implied in parallel texts]. 
\textit{Sistemy i~Sredstva Informatiki~--- Systems and Means of Informatics} 28(3):175--188.
\bibitem{10-zac-1}
\Aue{Zatsman, I.} 2019. Tselenapravlennoe razvitie sistem lingvisticheskikh znaniy: vyyavlenie 
i~zapolnenie lakun [Goal-oriented development of linguistic knowledge systems: Identifying and 
filling of lacunae]. \textit{Informatika i~ee Primeneniya~--- Inform. Appl.} 13(1):91--98.
\bibitem{11-zac-1}
\Aue{Zatsman, I.} 2020. Finding and filling lacunas in linguistic typologies. \textit{15th Forum 
(International) on Knowledge Asset Dynamics Proceedings}. Matera, Italy: Institute of Knowledge Asset 
Management. 780--793.
\bibitem{12-zac-1}
\Aue{Zatsman, I.} 2021. A~model of goal-oriented knowledge discovery based on human--computer 
symbiosis. \textit{16th Forum (International) on Knowledge Asset Dynamics Proceedings}. Rome, 
Italy: Arts for Business Institute. 297--312.
\bibitem{14-zac-1} %13
\Aue{Denning, P., and P.~Rosenbloom}. 2009. Computing: The fourth great domain of science. 
\textit{Commun. ACM} 52(9):27--29.
\bibitem{15-zac-1} %14
\Aue{Denning, P., and P.~Freeman.} 2009. Computing's paradigm. \textit{Commun. ACM} 52(12):28--30.
\bibitem{13-zac-1} %15
\Aue{Rosenbloom, P.\,S.} 2013. \textit{On computing: The fourth great scientific domain}. 
Cambridge, MA: MIT Press. 307~p.
\bibitem{16-zac-1}
\Aue{Zatsman, I.} 2018. Metodologiya obratimoy generalizatsii v~kontekste klassifikatsii 
informatsionnykh trans\-for\-ma\-tsiy [Methodology of reversible generalization in context of classification 
of information transformations]. \textit{Sistemy i~Sredstva Informatiki~--- Systems and Means of 
Informatics} 28(2):128--144.
\bibitem{17-zac-1}
\Aue{Zatsman, I.} 2013. Postroenie sistemy terminov informatsionno-komp'yuternoy nauki: 
problemno-orientirovannyy podkhod [Construction of the system of terms of information and 
computer science: A~problem- oriented approach]. \textit{Teoriya i~praktika obshchestvennoy 
nauchnoy informatsii} [Theory and practice of scientific information for social sciences] 
21:120--159.
\bibitem{18-zac-1}
\Aue{Zatsman, I.} 2014. 
Table of interfaces of informatics as computer and information science
\textit{Scientific Technical Information Processing} 41(4):233--246.
\bibitem{19-zac-1}
\Aue{Zatsman, I.} 2015. Protsessy tselenapravlennoy generatsii i~razvitiya kross-yazykovykh 
ekspertnykh znaniy: semioticheskie osnovaniya modelirovaniya [Goal-oriented processes of  
cross-lingual expert knowledge creation: Semiotic foundations for modeling]. \textit{Informatika i~ee 
Primeneniya~--- Inform. Appl.} 9(3):106--123.
\bibitem{20-zac-1}
\Aue{Zatsman, I.\,M.} 2019. Interfeysy tret'ego poryadka v~informatike [Third-order interfaces in 
informatics]. \textit{Informatika i~ee Primeneniya~--- Inform. Appl.} 13(3):82--89.
\bibitem{21-zac-1}
\Aue{Zatsman, I.\,M.} 2019. Kodirovanie kontseptov v~tsifrovoy srede [Digital encoding of 
concepts]. \textit{Informatika i~ee Primeneniya~--- Inform. Appl.} 13(4):97--106.
\bibitem{22-zac-1}
\Aue{Shemakin, Yu.\,I., and A.\,A.~Romanov.} 1995. \textit{Komp'yu\-ter\-naya semantika} [Computer 
semantics]. Moscow: Shkola Kitaygorodskoy. 344~p.
\bibitem{23-zac-1}
\Aue{Goncharov, A.\,A., I.\,M.~Zatsman, and M.\,G.~Kruzhkov.} 2019. Tem\-po\-ral'\-nye 
dannye 
v~leksikograficheskikh bazakh znaniy [Temporal data in lexicographic databases]. \textit{Informatika 
i~ee Primeneniya~--- Inform. Appl.} 13(4):90--96.
\bibitem{24-zac-1}
\Aue{Goncharov, A.\,A., I.\,M.~Zatsman, and M.\,G.~Kruzhkov.} 2020. Evolyutsiya klassifikatsiy 
v~nadkorpusnykh ba\-zakh dannykh [Evolution of classifications in supracorpora databases]. 
\textit{Informatika i~ee Primeneniya~--- Inform. \mbox{Appl}.} 14(4):108--116.
\bibitem{25-zac-1}
\Aue{Goncharov, A.\,A., I.\,M.~Zatsman, and M.\,G.~Kruzhkov.} 2021. Predstavlenie novykh 
leksikograficheskikh znaniy v~dinamicheskikh klassifikatsionnykh sistemakh [Representation of new 
lexicographical knowledge in dynamic classification systems]. \textit{Informatika i~ee 
 Primeneniya~--- Inform. Appl.}  15(1):86--93.
\bibitem{26-zac-1}
\Aue{McGuinness, D.\,L.} 2003. Ontologies come of age. \textit{Spinning the Semantic Web: Bringing 
the World Wide Web to its full potential}. Eds. D.~Fensel, J.~Hendler, H.~Lieberman, and 
W.~Wahlster. Cambridge, MA: MIT Press. 171--194.
\bibitem{27-zac-1}
\Aue{Zatsman, I.} 2021. Problemno-orientirovannaya ak\-tu\-a\-li\-za\-tsiya slovarnykh statey dvuyazychnykh 
slovarey i~me\-di\-tsin\-skoy terminologii: sopostavitel'nyy analiz [Problem- oriented updating of dictionary 
entries of bilingual\linebreak dictionaries and medical terminology: Comparative analysis]. \textit{Informatika 
i~ee Primeneniya~--- Inform. Appl.} 15(1):94--101.
\end{thebibliography}

 }
 }

\end{multicols}

\vspace*{-11pt}

\hfill{\small\textit{Received October 15, 2021}}

%\pagebreak

\vspace*{-12pt}





\Contrl

\vspace*{-4pt}

\noindent
\textbf{Zatsman Igor M.} (b.\ 1952)~--- Doctor of Science in technology, Head of Department, 
Institute of Informatics Problems, Federal Research Center ``Computer Science and Control'' of the 
Russian Academy of Sciences, 44-2~Vavilov Str., Moscow 119333, Russian Federation; 
\mbox{izatsman@yandex.ru}

\label{end\stat}

\renewcommand{\bibname}{\protect\rm Литература}