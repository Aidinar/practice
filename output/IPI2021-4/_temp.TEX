\documentclass[10pt]{book}
\usepackage[utf8]{inputenc}

\usepackage{latexsym,amssymb,amsfonts,amsmath,amsxtra,dsfont,
indentfirst,shapepar,%fleqn,%
picinpar,shadow,floatflt,enumerate,multicol,colortbl,moreverb,cite,ipi}

\usepackage{rotating}
\usepackage{mathrsfs}
\usepackage[noend]{algorithmic}
\usepackage{ulem}
\usepackage{graphicx}
%\usepackage{algorithm2e}
\usepackage[linesnumbered,boxed,ruled]{algorithm2e}
%\usepackage{xypic}
\usepackage{oldgerm}
\usepackage{epic}
\usepackage{eepic}

\SetAlgorithmName{Algorithm}{алгоритм}{Список алгоритмов}

%из Дюковой

\newcommand{\algKeyword}[1]{{\bf #1}}
\newcommand{\Proc}[1]{\text{\tt #1}}
\def\CALL{\algKeyword{call}~}

\newenvironment{AlgProcedure}[1]
{
\small
\medskip
%    \hrule
\medskip
\algKeyword{PROCEDURE} #1
\begin{algorithmic}[1]}
{\end{algorithmic}
%    \hrule
\bigskip
}

\def\CALL{\algKeyword{call}~}

%конец для Дюковой

%\RequirePackage[ruled]{algorithm}


\input{epsf}

%\nofiles

%\includeonly{avtor}                          %
%\includeonly{podgot-rus-site,podgot-eng-site}  
%\includeonly{podgot-rus,podgot-eng}  
%\includeonly{ipi-ind} 
%\includeonly{index-15i}
%\includeonly{toc-rus, toc-en}
%\includeonly{toc-rus}
%\includeonly{toc-en} 
%\includeonly{popravka}




%\includeonly{bosov}                 %pdf+авт
%\includeonly{kudr}                  %pdf+авт+
%\includeonly{malashenko}            %pdf+авт+
%\includeonly{inkova-kruzhkov}       %pdf+авт+
%\includeonly{shnurkov}   %pdf
%\includeonly{goncharov}  %pdf
%\includeonly{shetinin}              %pdf+авт+
%\includeonly{abgaryan}              %pdf+авт+
%\includeonly{grusho}     %pdf
%\includeonly{grinchenko}            %pdf+авт+
%\includeonly{kochetkova} %pdf
%\includeonly{dukova}                %pdf+авт+
\includeonly{flerov}     %pdf
%\includeonly{zatsman}               %pdf+авт
%\includeonly{sokolov}               %pdf+авт+


%\includeonly{rekl}




\usepackage{acad}
%\usepackage{courier}
\usepackage{decor}
\usepackage{newton}
\usepackage{pragmatica}
\usepackage{zapfchan}
\usepackage{petrotex}
\usepackage{bm}                     % полужирные греческие буквы
\usepackage{upgreek}                % прямые греческие буквы \upalpha
\usepackage{eufrak}
\usepackage{verbatim}

\renewcommand{\bottomfraction}{0.99}
\renewcommand{\topfraction}{0.99}
\renewcommand{\textfraction}{0.01}

\setcounter{secnumdepth}{1} %здесь - 3 + chapter = 4

\arraycolsep=1.5pt

%\usepackage[pdftex]{graphicx}

%\usepackage{oz}

%NEW COMMANDS


\renewcommand*{\hm}[1]{#1\nobreak\discretionary{}%
            {\hbox{$\mathsurround=0pt #1$}}{}} %% Дублирует знаки операций
                               %при переносе в формуле (перед знаком, который
                               %надо продублировать ставится команда \hm)

%\newcommand{\endproof}{\hfill$\Box$}
\renewcommand{\r}{\mathbb{R}}
%\newcommand{\I}{{\rm I\hspace{-0.7mm}I}}
%\newcommand{\Ikl}{{\tt{1}}\hspace*{-1.44mm}\mathtt{1}}
\newcommand{\Ik}{\mbox{{\small \tt {1}}\hspace{-1.3mm}{\tt 1}}}
\newcommand{\argmin}{\mathop{\mathrm{arg}\,\mathrm{min}}}
\newcommand{\argmax}{\mathop{\mathrm{arg}\,\mathrm{max}}}
%\newcommand{\capr}{\mathop{\cap\,}}
%\newcommand{\cupr}{\mathop{\cup\,}}
%\def\argmin{\mathop{arg\,min}}

\def\vrp{\varphi}
\def\prt{\partial}
\def\mm{{\sf M}}
\def\modnop#1{\mathop{#1}\limits_{n}}
\def\eam{\mathbin{{\mathop{=}\limits^{\mathrm{def}}}}}
\def\dey#1#2{#1 (#2)}
\def\deyc#1#2{#1 \cdot  #2}
\def\ra#1{\;\mathop{\to}\limits^{#1}\;}
\def\raz#1{\;\mathop{\longrightarrow}\limits^{\!\!\!#1}\;}
\def\ral#1{\;\mathop{\longrightarrow}\limits^{#1}\;}

\newcommand{\Nor}{\mathcal{N}}
\newcommand{\T}{\mathbb{T}}
\newcommand{\Z}{\mathbb{Z}}



\newcommand{\il}[2]{\int\limits_{#1}^{#2}}%интеграл с пределами #1 и #2

\def\sm2{\mathop {\sum\limits^{n^\Theta}\sum\limits^{n^\Theta}}}
\def\sss{\sum\limits}
\def\tr{,\,\ldots\,,\,}
\def\rk{\right]}
\def\lk{\left[}
\def\rf{\right\}}
\def\lf{\left\{}
\def\lv{\,\left\vert}
\def\rv{\right\vert\,}
\def\iii{\int\limits}
\def\iin{\int\limits_{-\infty}^\infty}
\def\rrv{\right\vert}


\def\ee{{\cal E}}
\def\ww{{\cal W}}
\def\yy{{\cal Y}}
\def\vv{{\cal V}}

\newcommand{\R}{\mathbb R}
\newcommand{\E}{\mathbb E}
\newcommand{\N}{\mathbb N}

\renewcommand{\P}{\mathbb{P}}

\newcommand{\h}{{\bf H}}
\newcommand{\p}{{\sf P}}  % вероятность

\newcommand{\e}{{\sf E}}  % мат. ожидание
\newcommand{\D}{{\sf D}}  % дисперсия
\newcommand{\eps}{\varepsilon}
\newcommand{\vp}{{\mathbf p}}
\newcommand{\vz}{{\mathbf z}}
\newcommand{\vx}{{\mathbf x}}
\newcommand{\vf}{{\mathbf f}}
\newcommand{\F}{{\mathcal F}}
\def\ap{{\mathrm{ЭР}}}
\newcommand{\ud}{\Delta_n} %uniform ditance
\newcommand{\nud}{\Delta_n(x)}
%\renewcommand{\Re}{\mathrm{Re}\,}

\newcommand{\abs}[1]{\left\vert#1\right\vert}

\newcommand{\norm}[1]{\left\Vert#1\right\Vert}
\def\da{(\Delta_t,A)}

\newcommand{\corr}{\mathrm{corr}}

\newcommand{\cov}{\mathrm{cov}}
\newcommand{\Expect}{\mathbb{E}}

\def\w{\omega}
\def\W{\Omega}

\def\inh{\int\limits_{nh}^{(n+1)h}}

\def\sumin{\sum_{i=1}^N}


\def\bxt{(Y,t)}
\def\xt{(y,t)}

\def\ovth{{\fr{\tau-nh}{h}}}
\def\ov{\overline}
\def\tm{\tilde m}
\def\tl{\tilde\lambda}
\def\tB{\widetilde B}
\def\tb{\tilde b}
\def\ld{\ldots}
\def\cd{\cdots}


\DeclareMathOperator{\sign}{sign}

%\newcommand{\gr}{{\geqslant}}


\newcommand{\g}{\mbox{\textit{g}}}

\renewcommand{\la}{\lambda}
\newcommand{\si}{\sigma}
\newcommand{\alp}{\alpha}

\newcommand{\pto}{\stackrel{P}{\longrightarrow}} % сходимость по веpоятности

\newcommand{\eqd}{\stackrel{\mathrm{d}}{=}} % равенство по pаспpеделению
\newcommand{\eqdelta}{\stackrel{\triangle}{=}} % равенство по pаспpеделению

\def\be#1{\begin{equation}\label{#1}}
\def\ee{\end{equation}}
\def\re#1{(\ref{#1})}

\def\bn{\begin{enumerate}}
\def\en{\end{enumerate}}
\def\bi{\begin{itemize}}
\def\ei{\end{itemize}}
%\def\i{\item}

%\newcommand{\kp}{\kappa}
%\def\Q{{\cal Q}} \def\H{{\cal H}}
%\newcommand{\bet}{\beta_{2+\delta}}


%\newtheorem{definition}{Определение}
%\renewcommand{\thedefinition}{\arabic{definition}.}
%END NEW COMMANDS

%\renewcommand{\baselinestretch}{1.2}

%\pagestyle{myheadings}

\setlength{\textwidth}{167mm}      % 122mm
\setlength{\textheight}{658pt}
%\setlength{\textheight}{635.6pt}
\setlength{\columnsep}{4.5mm}

\setcounter{secnumdepth}{4}

%\addtolength{\headheight}{2pt}
%\addtolength{\headsep}{-2mm}

\addtolength{\topmargin}{-7mm}  % for printing


%\hoffset=-30mm  % From Yap
\hoffset=-23mm  % From Acrobat

%\voffset=0mm % From Yap
\voffset=-5mm   % From Acrobat

%\addtolength{\evensidemargin}{-2.5mm} % for printing
%\addtolength{\oddsidemargin}{2.5mm}  % for printing

\addtolength{\evensidemargin}{-12mm} % for printing
\addtolength{\oddsidemargin}{8mm}  % for printing

%\renewcommand{\thefootnote}{\fnsymbol{footnote}}
%\renewcommand{\thefootnote}{\arabic{footnote}}
\renewcommand{\figurename}{\protect\bf Рис.}
\renewcommand{\tablename}{\protect\bf Таблица}

\newcommand{\Caption}[1]{\caption{\protect\small %\baselineskip=2.5ex
#1}}

\renewcommand{\thefigure}{\arabic{figure}}
\renewcommand{\thetable}{\arabic{table}}
\renewcommand{\theequation}{\arabic{equation}}
\renewcommand{\thesection}{\arabic{section}}

\renewcommand{\contentsname}{СОДЕРЖАНИЕ}
\newcommand{\fr}[2]{\displaystyle\frac{\displaystyle #1\mathstrut}{\displaystyle #2\mathstrut}}

%\renewcommand{\thefootnote}{\fnsymbol{footnote}}
%\newcommand{\g}{\mbox{\textit{g}}}

%\newcommand{\Caption}[1]{\caption{\protect\small\baselineskip=2ex #1}}
\newcounter{razdel}
\setcounter{razdel}{0}

\def\god{2021}
\def\tom{15}
\def\vyp{4}


\newcommand{\titel}[4]{%
\

\vspace*{5pt}

\ifodd\therazdel {\raggedright\noindent\Large\textrm\textbf
 \lineskip .75em
  \baselineskip=3.2ex #1 \par}
\vskip 1em {\noindent\large\textrm\textbf #2 \par}
\addcontentsline{toc}{subsection}{{\textrm\textbf #1}\protect\newline #2}
\def\rightheadline{\underline{\noindent\hbox to \textwidth{\hfill\small\textrm{#4}
%\hfill \large\bf\thepage
}}}
\def\leftheadline{\underline{\noindent\parbox{\textwidth}{
%\raggedleft\large\bf\thepage \hfill
\small\textit{#3}\hfill}}}
\def\leftfootline{\small{\textbf{\thepage}
\hfill ИНФОРМАТИКА И ЕЁ ПРИМЕНЕНИЯ\ \ \ том~\tom\ \ \ выпуск~\vyp\ \ \ \god}
}%
 \def\rightfootline{\small{ИНФОРМАТИКА И ЕЁ ПРИМЕНЕНИЯ\ \ \ том~\tom\ \ \ выпуск~\vyp\ \ \ \god
\hfill \textbf{\thepage}}}
\vskip 2em \setcounter{figure}{0}
\setcounter{table}{0}
\setcounter{equation}{0}
\setcounter{section}{0}
\setcounter{subsection}{0}
\setcounter{subsubsection}{0}
\setcounter{footnote}{0}
\setcounter{razdel}{0}
%\end{flushleft}
\else {
 \raggedright\noindent\Large\textrm\textbf
 \lineskip .75em
\baselineskip=3.2ex #1 \par} \vskip 1em
%\begin{flushleft}
{\noindent\large\textrm\textbf #2 \par}
\addcontentsline{toc}{subsection}{{\textrm\textbf #1}\protect\newline #2}
\def\rightheadline{\underline{\noindent\hbox to \textwidth{\hfill\small\textrm{#4}
%\hfill \large\bf\thepage
}}}
\def\leftheadline{\underline{\noindent\parbox{\textwidth}{%\raggedleft\large\bf\thepage \hfill
\small\textit{#3}\hfill}}}
\def\leftfootline{\small{\textbf{\thepage}
\hfill ИНФОРМАТИКА И ЕЁ ПРИМЕНЕНИЯ\ \ \ том~\tom\ \ \ выпуск~\vyp\ \ \ \god}
}%
 \def\rightfootline{\small{ИНФОРМАТИКА И ЕЁ ПРИМЕНЕНИЯ\ \ \ том~15\ \ \ выпуск~\vyp\ \ \ 2021
\hfill \textbf{\thepage}}} \vskip 2em \setcounter{figure}{0}
\setcounter{table}{0} \setcounter{equation}{0} \setcounter{section}{0}
\setcounter{subsection}{0} \setcounter{subsubsection}{0}
\setcounter{footnote}{0}
%\end{flushleft}
\fi}

\newcommand{\titelr}[2]{%
\

\vspace*{5pt}

\ifodd\therazdel {\raggedright\noindent%\Large\textrm\textbf
 \lineskip .75em
  \baselineskip=3.2ex #1 \par}
\vskip 1em {\noindent\normalsize\textrm\textbf #2 \par}
\else {
 \raggedright\noindent\Large\textrm\textbf
 \lineskip .75em
\baselineskip=3.2ex #1 \par} \vskip 1em
%\begin{flushleft}
{\noindent\large\textrm\textbf #2 \par
%\noindent\normalsize\textrm\textbf #2 \par
} \fi}

\newcommand{\titele}[5]{%
\

%\vspace*{5pt}

\ifodd\therazdel {\raggedright\noindent\large
\textrm\textbf
 \lineskip .75em
%  \baselineskip=3.2ex
#1 \par}
\vskip .5em {\noindent\large\textrm\textbf #2 \par}
\vskip .5em
 {\noindent\textrm #3 \par}
\addcontentsline{toc}{subsection}{{\textrm\textbf #1}\protect\newline #2}
\def\rightheadline{\underline{\noindent\hbox to \textwidth{\hfill\small\textrm{#4}
%\hfill \large\bf\thepage
}}}
\def\leftheadline{\underline{\noindent\parbox{\textwidth}{
%\raggedleft\large\bf\thepage \hfill
\small\textrm{#5}\hfill}}}
\def\leftfootline{\small{\textbf{\thepage}
\hfill ИНФОРМАТИКА И ЕЁ ПРИМЕНЕНИЯ\ \ \ том~15\ \ \ выпуск~4\ \ \ 2021}
}%
 \def\rightfootline{\small{ИНФОРМАТИКА И ЕЁ ПРИМЕНЕНИЯ\ \ \ том~15\ \ \ выпуск~4\ \ \ 2021
\hfill \textbf{\thepage}}} \vskip 1em \setcounter{figure}{0}
\setcounter{table}{0} \setcounter{equation}{0} \setcounter{section}{0}
\setcounter{subsection}{0} \setcounter{subsubsection}{0}
\setcounter{footnote}{0} \setcounter{razdel}{0}
%\end{flushleft}
\else {
 \raggedright\noindent\large
 \textrm\textbf
 \lineskip .75em
%\baselineskip=3.2ex
#1 \par} \vskip .5em
%\begin{flushleft}
{\noindent\large\textrm\textbf #2 \par} \vskip .5em
 {\noindent\textrm #3 \par}
\addcontentsline{toc}{subsection}{{\textrm\textbf #1}\protect\newline #2}
\def\rightheadline{\underline{\noindent\hbox to \textwidth{\hfill\small\textrm{#4}
%\hfill \large\bf\thepage
}}}
\def\leftheadline{\underline{\noindent\parbox{\textwidth}{%\raggedleft\large\bf\thepage \hfill
\small\textrm{#5}\hfill}}}
\def\leftfootline{\small{\textbf{\thepage}
\hfill ИНФОРМАТИКА И ЕЁ ПРИМЕНЕНИЯ\ \ \ том~15\ \ \ выпуск~4\ \ \ 2021}
}%
 \def\rightfootline{\small{ИНФОРМАТИКА И ЕЁ ПРИМЕНЕНИЯ\ \ \ том~15\ \ \ выпуск~4\ \ \ 2021
\hfill \textbf{\thepage}}} \vskip 1em \setcounter{figure}{0}
\setcounter{table}{0} \setcounter{equation}{0} \setcounter{section}{0}
\setcounter{subsection}{0} \setcounter{subsubsection}{0}
\setcounter{footnote}{0}
%\end{flushleft}
\fi}

\def\Abst#1{
\begin{center}\small\nwt
\parbox{150mm}{%\baselineskip=2.5ex
\textbf{Аннотация:}\ \
%\hspace*{\parindent}
#1}
\end{center}}
\def\Abste#1{
\begin{center}\small\nwt
\parbox{150mm}{%\baselineskip=2.5ex
\textbf{Abstract:}\ \
%\hspace*{\parindent}
#1}
\end{center}}

\def\DOI#1{
\begin{center}\small\nwt
\parbox{150mm}{%\baselineskip=2.5ex
\textbf{DOI:}\ \
%\hspace*{\parindent}
#1}
\end{center}}

\def\Abstend#1{
\begin{center}\small\nwt
\parbox{150mm}{%\baselineskip=2.5ex
%\hspace*{\parindent}
#1}
\end{center}}


\def\KW#1{
\begin{center}\small\nwt
\parbox{150mm}{%\baselineskip=2.5ex
\textbf{Ключевые слова:}\ \ #1}
\end{center}}

\def\KWE#1{
\begin{center}\small\nwt
\parbox{150mm}{%\baselineskip=2.5ex
\textbf{Keywords:}\ \ #1}
\end{center}}


\def\KWN#1{
%\begin{center}
%\small
%\parbox{150mm}\end{center}
}

\newcommand{\Avtors}[1]{%\smallskip
%\vspace*{.5pt}
\hangindent=23pt\noindent
%\nwt
{\bfseries#1}\
}


\renewcommand{\thesubsection}{\thesection.\arabic{subsection}\hspace*{-5pt}}
\renewcommand{\thesubsubsection}{\thesubsection\hspace*{5pt}.\arabic{subsubsection}\hspace*{-3pt}}

\newcommand{\Ack}{\section*{\protect\rmfamily Acknowledgments}\noindent}
\newcommand{\Contr}{\section*{\protect\rmfamily Contributors}\noindent}
\newcommand{\Contrl}{\section*{\protect\rmfamily Contributor}\noindent}

\makeindex


\begin{document}
\Rus

\nwt
%\ptb


%\renewcommand{\contentsname}{\protect\Large\bf Содержание}

\setcounter{tocdepth}{2}

%\tableofcontents

\renewcommand{\bibname}{\protect\rmfamily Литература}
  \def\Au#1{{\it #1}}
    \def\Aue#1{{#1}}

%\newcommand{\No}{№}
  \newcommand{\tg}{\,\mathrm{tg}\,}
    \newcommand{\ctg}{\,\mathrm{ctg}\,}
  \newcommand{\arctg}{\,\mathrm{arctg}\,}

\def\forallb{\mathop{\forall}}
\def\cupb{\mathop{\cup}}
\def\existsb{\mathop{\exists}}


\newpage
\addtocounter{razdel}{1}
%\def\razd{РЕГУЛИРУЕМЫЙ ЭЛЕКТРОПРИВОД ДЛЯ ЭЛЕКТРОЭНЕРГЕТИКИ}


\setcounter{page}{3}

%   { %\Large  
   { %\baselineskip=16.6pt
   
   \vspace*{-48pt}
   \begin{center}\LARGE
   \textit{Предисловие}
   \end{center}
   
   %\vspace*{2.5mm}
   
   \vspace*{25mm}
   
   \thispagestyle{empty}
   
   { %\small 

    
Вниманию читателей журнала <<Информатика и её применения>> предлагается 
очередной тематический выпуск <<Вероятностно-статистические методы и 
задачи информатики и информационных технологий>>. Предыдущие тематические 
выпуски журнала по данному направлению вышли в 2008~г.\ (т.~2, вып.~2), 
в 2009~г.\ (т.~3, вып.~3) и в 2010~г.\ (т.~4, вып.~2). 

Статьи, собранные в данном журнале, посвящены разработке новых вероятностно-статистических 
методов, ориентированных на применение к решению конкретных задач информатики и информационных 
технологий, а также~--- в ряде случаев~--- и других прикладных задач. Проблематика, охватываемая 
публикуемыми работами, развивается в рамках научного сотрудничества между Институтом проблем 
информатики Российской академии наук (ИПИ РАН) и Факультетом вычислительной математики и 
кибернетики Московского государственного университета им.\ М.\,В.~Ломоносова в ходе работ 
над совместными научными проектами (в том числе в рамках функционирования 
Научно-образовательного центра <<Вероятностно-статистические методы анализа рисков>>). 
Многие из авторов статей, включенных в данный номер журнала, являются активными участниками 
традиционного международного семинара по проблемам устойчивости стохастических моделей, 
руководимого В.\,М.~Золотаревым и В.\,Ю.~Королевым; регулярные сессии этого семинара 
проводятся под эгидой МГУ и ИПИ РАН (в 2011~г.\ указанный семинар проводится в октябре 
в Калининградской области РФ). 

Наряду с представителями ИПИ РАН и МГУ в число авторов данного выпуска журнала входят 
ученые из Научно-исследовательского института системных исследований РАН, Института 
проблем технологии микроэлектроники и особочистых материалов РАН, Института 
прикладных математических исследований Карельского НЦ РАН, Московского 
авиационного института, Вологодского государственного педагогического университета, 
НИИММ им.\ Н.\,Г.~Чеботарева, Казанского государственного университета, Дебреценского 
университета (Венгрия).

Несколько статей выпуска посвящено разработке и применению стохастических методов и 
информационных технологий для решения различных прикладных задач. В~работе В.\,Г.~Ушакова 
и О.\,В.~Шестакова рассмотрена задача определения вероятностных характеристик случайных 
функций по распределениям интегральных преобразований, возникающих в задачах эмиссионной 
томографии. В~статье Д.\,О.~Яковенко и М.\,А.~Целищева рассмотрены некоторые вопросы 
математической теории риска и предложен новый подход к диверсификации инвестиционных 
портфелей. Работа И.\,А.~Кудрявцевой и А.\,В.~Пантелеева посвящена построению и 
исследованию математической модели, описывающей динамику сильноионизованной плазмы. 
В~статье П.\,П.~Кольцова изучается качество работы ряда алгоритмов сегментации изображений. 
Статья А.\,Н.~Чупрунова и И.~Фазекаша посвящена вероятностному анализу числа без\-оши\-бочных 
блоков при помехоустойчивом кодировании; получены усиленные законы больших чисел для указанных 
величин.

В данном выпуске традиционно присутствует тематика, весьма активно разрабатываемая в течение 
многих лет специалистами ИПИ РАН и МГУ,~--- методы моделирования и управления для 
информационно-телекоммуникационных и вычислительных систем, в частности методы 
теории массового обслуживания. В~статье А.\,И.~Зейфмана с соавторами рассматриваются 
модели обслуживания, описываемые марковскими цепями с непрерывным временем в случае 
наличия катастроф. В~работе М.\,М.~Лери и И.\,А.~Чеплюковой рассматриваются случайные 
графы Интернет-типа, т.\,е.\ графы, степени вершин которых имеют степенные распределения; 
такие задачи находят применение при исследовании глобальных сетей передачи данных. 
Работа Р.\,В.~Разумчика посвящена исследованию систем массового обслуживания специального 
вида~--- с отрицательными заявками и хранением вытесненных заявок.

Ряд статей посвящен развитию перспективных теоретических 
вероятностно-статистических методов, которые находят широкое применение в различных 
задачах информатики и информационных технологий. В~работе В.\,Е.~Бенинга, А.\,К.~Горшенина 
и В.\,Ю.~Королева рассмотрена задача статистической проверки гипотез о числе компонент 
смеси вероятностных распределений, приводится конструкция асимптотически наиболее мощного 
критерия. Результаты этой работы найдут применение в ряде прикладных задач, использующих 
математическую модель смеси вероятностных распределений (в информатике, моделировании 
финансовых рынков, физике турбулентной плазмы и~т.\,д.). В~статье В.\,Ю.~Королева, 
И.\,Г.~Шевцовой и С.\,Я.~Шоргина строится новая, улучшенная оценка точности нормальной 
аппроксимации для пуассоновских случайных сумм; как известно, указанные случайные суммы 
широко используются в качестве моделей многих реальных объектов, в том числе в информатике, 
физике и других прикладных областях. Работа В.\,Г.~Ушакова и Н.\,Г.~Ушакова посвящена 
исследованию ядерной оценки плотности распределения; эти результаты могут применяться, 
в част\-ности, при анализе трафика в телекоммуникационных системах. Серьезные приложения 
в статистике могут получить результаты работы О.\,В.~Шестакова, в которой доказаны оценки 
скорости сходимости распределения выборочного абсолютного медианного отклонения к нормальному 
закону. 

\smallskip

Редакционная коллегия журнала выражает надежду, что данный тематический  выпуск 
будет интересен специалистам в области теории вероятностей и математической статистики 
и их применения к решению задач информатики и информационных технологий.
     
     %\vfill 
     \vspace*{20mm}
     \noindent
     Заместитель главного редактора журнала <<Информатика и её 
применения>>,\\
     директор ИПИ РАН, академик  \hfill
     \textit{И.\,А.~Соколов}\\
     
     \noindent
     Редактор-составитель тематического выпуска,\\
     профессор кафедры математической статистики факультета\\
      вычислительной математики и кибернетики МГУ им.\ М.\,В.~Ломоносова,\\
     ведущий научный сотрудник ИПИ РАН,\\ 
доктор физико-математических наук \hfill
      \textit{В.\,Ю.~Королев}
     
     } }
     }


   
\def\stat{bosov+stef}

\def\tit{УПРАВЛЕНИЕ ВЫХОДОМ СТОХАСТИЧЕСКОЙ ДИФФЕРЕНЦИАЛЬНОЙ СИСТЕМЫ 
ПО~КВАДРАТИЧНОМУ КРИТЕРИЮ. I.~ОПТИМАЛЬНОЕ РЕШЕНИЕ МЕТОДОМ 
ДИНАМИЧЕСКОГО ПРОГРАММИРОВАНИЯ$^*$}

\def\titkol{Управление выходом стохастической дифференциальной системы 
по~квадратичному критерию. I}
%.~Оптимальное решение методом 
%динамического программирования}

\def\aut{А.\,В.~Босов$^1$, А.\,И.~Стефанович$^2$}

\def\autkol{А.\,В.~Босов, А.\,И.~Стефанович}

\titel{\tit}{\aut}{\autkol}{\titkol}

\index{Босов А.\,В.}
\index{Стефанович А.\,И.}
\index{Bosov A.\,V.}
\index{Stefanovich A.\,I.}




{\renewcommand{\thefootnote}{\fnsymbol{footnote}} \footnotetext[1]
{Работа выполнена при частичной поддержке РФФИ (проект 16-07-00677).}}


\renewcommand{\thefootnote}{\arabic{footnote}}
\footnotetext[1]{Институт проблем информатики Федерального исследовательского центра <<Информатика 
и~управление>> Российской академии наук, \mbox{AVBosov@ipiran.ru}}
\footnotetext[2]{Институт проблем информатики Федерального исследовательского центра <<Информатика 
и~управление>> Российской академии наук, \mbox{AStefanovich@frccsc.ru}}

%\vspace*{8pt}



  
  \Abst{Решается задача оптимального управления для диффузионного процесса 
Ито и~линейного управ\-ля\-емо\-го выхода. Рассматриваемая постановка близка 
к~классической ли\-ней\-но-квад\-ра\-тич\-ной гауссовской задаче управления 
(linear-quadratic Gaussian (LQG) control). Отличия состоят в~том, что состояние описывается нелинейным 
дифференциальным уравнение Ито $dy_t\hm= A_t(y_t) \,dt\hm+ \Sigma_t(y_t)\,dv_t$ 
и~не зависит от управ\-ле\-ния~$u_t$, оптимизации подлежит управ\-ля\-емый 
линейный выход $dz_t\hm= a_t y_t\,dt\hm+ b_t z_t \,dt\hm+ c_t u_t \,dt\hm+ \sigma_t\, 
dw_t$. Дополнительные обобщения внесены в~квад\-ра\-тич\-ный критерий качества 
с~целью воз\-мож\-ности постановки таких задач, как отслеживание выходом 
состояния или управ\-ле\-ни\-ем~--- линейной комбинации состояния и~выхода. Для 
решения используется метод динамического программирования. Функцию 
Беллмана позволяет найти предположение о~ее структуре вида $V_t(y,z)\hm= 
\alpha_t z^2\hm+ \beta_t(y)z \hm+\gamma_t(y)$. Решение дают три 
дифференциальных уравнения для коэффициентов~$\alpha_t$, $\beta_t(y)$ 
и~$\gamma_t(y)$. Эти уравнения со\-став\-ля\-ют оптимальное решение 
рас\-смат\-ри\-ва\-емой задачи.}
  
  \KW{стохастическое дифференциальное уравнение; оптимальное управ\-ле\-ние; 
динамическое программирование; функция Беллмана; уравнение Риккати; 
линейные уравнения параболического типа}

\DOI{10.14357/19922264180314}
  
%\vspace*{4pt}


\vskip 10pt plus 9pt minus 6pt

\thispagestyle{headings}

\begin{multicols}{2}

\label{st\stat}

\section{Введение}

     Ключевые результаты в~области оптимизации стохастических 
динамических систем, со\-став\-ля\-ющие классическую теорию управления, 
получены более~40~лет назад (такова работа~[1] в~отношении задачи 
управ\-ле\-ния ли\-ней\-но-гаус\-сов\-ски\-ми стохастическими сис\-те\-ма\-ми по 
квад\-ра\-тич\-но\-му критерию). К~классической тео\-рии следует относить 
линейные модели стохастических сис\-тем и~квадратичный критерий качества. 
Это исходный базис, на котором основано множество успешно 
исследованных и~решенных задач стохастического управ\-ле\-ния 
и~оптимизации. 

Дальнейшее развитие~--- это новые модели и~критерии, но 
прежде всего это новые методы: от тео\-рии линейных регуляторов, метода 
динамического программирования и~принципа максимума к~адаптивному 
и~минимаксному подходу, импульсному управ\-ле\-нию и~т.\,д. Множество 
инноваций как в~час\-ти моделей, так и~в~час\-ти математического аппарата, 
имевших мес\-то в~по\-сле\-ду\-ющие годы, существенно обогатили тео\-рию 
управ\-ле\-ния. Но и~до настоящего времени линейные модели и~квадратичный 
критерий, несмотря на всю справедливую критику в~отношении их 
аде\-кват\-ности и~гиб\-кости, сохраняют исследовательский интерес и~находят 
современные области приложения.
     
     Не претендуя на сколь\-ко-ни\-будь полное обосно\-ва\-ние последнего 
тезиса, приведем несколько примеров, показавшихся наиболее ин\-те\-рес\-ными. 

Так, в~[2] решается ли\-ней\-но-квад\-ра\-тич\-ная за\-да\-ча в~игровой 
постановке с~запаздыванием. В~близ\-кой по модели работе~[3] задача 
управ\-ле\-ния ставится в~терминах $H_\infty$-ро\-баст\-ности. Точнее \mbox{называть} 
эту тематику $H_2/H_\infty$-управ\-ле\-ни\-ем, и~работ по этой теме очень 
много. Аккуратности ради следует уточнить, что под линейными 
понимаются модели с~мультипликативными по состоянию воз\-му\-ще\-ниями. 

Совсем другой класс моделей, особо популярных в~по\-след\-ние годы, 
составляют скачкообразные процессы. Например, линейные уравнения 
в~сочетании с~пуассоновскими скачками в~[4] используются в~моделях, 
описывающих различные показатели функционирования сетевых протоколов 
передачи данных транспортного уровня. Телекоммуникации представляют 
в~последние годы самый популярный прикладной материал для 
исследований, работ по этой проб\-ле\-ма\-ти\-ке множество, математические 
техники привлекаются самые разные и~самые современные, но и~линейным 
моделям место находится. Еще один любопытный пример исследования 
скачкообразного процесса и~оптимизации на основе квад\-ра\-тич\-но\-го критерия 
можно найти в~[5] применительно к~задаче инвестирования на финансовом 
рынке. Наконец, упомянем еще работу~[6], подводящую итог исследований 
в~отношении классической детерминированной  
ли\-ней\-но-квад\-ра\-тич\-ной задачи с~использованием техники матричных 
неравенств.
     
     В данной работе также эксплуатируются привлекательные свойства 
линейных моделей и~квад\-ра\-тич\-но\-го критерия, причем в~стохастической 
постановке. На\-прав\-ле\-ни\-ем для обобщения \mbox{выбрана} модель динамики 
сис\-те\-мы: основные усилия на\-прав\-ле\-ны на то, чтобы сделать ее нелинейной. 
Кроме того, пред\-став\-лен\-ная постановка может рас\-смат\-ри\-вать\-ся и~как 
обобщение ранее решенной задачи в~дискретном времени~[7, 8] на время 
непрерывное. В~упомянутых работах помимо собственно модельной 
постановки важна еще и~привлекаемая прикладная об\-ласть~--- 
функционирование сложных программных сис\-тем. Результатов, 
ориентированных непосредственно на такие приложения, к~настоящему 
времени пренебрежимо мало, поэтому~[7, 8]~--- это еще и~прикладное 
обоснование рас\-смат\-ри\-ва\-емой далее задачи.
     
     Оптимизируемая динамическая сис\-те\-ма описывается двумя 
уравнениями. Состояние задается нелинейным стохастическим 
дифференциальным уравнением Ито, не содержащим управ\-ля\-емой 
переменной. Возмущение здесь описывается стандартным винеровским 
процессом, накладываются простые условия существования 
и~един\-ст\-вен\-ности решения. Поскольку состояние не управ\-ля\-ет\-ся, то уместно 
его интерпретировать как слож\-ное внешнее возмущение. Вторая 
переменная~--- управ\-ля\-емый выход~--- задается линейным стохастическим 
дифференциальным уравнением. Цель оптимизации выхода формируется 
квадратичным критерием общего вида. Формальная постановка задачи 
приведена в~сле\-ду\-ющем разделе.
     
     Для решения задачи используется метод динамического 
программирования, решается уравнение Беллмана~[9]. Соответственно, 
в~результате получаются аналитические выражения и~для оптимального 
управ\-ле\-ния, и~для значения функционала качества. Технически 
традиционный, стандартный подход к~задаче обременен, пожалуй, 
единственной проблемой~--- поиском верного пред\-став\-ле\-ния структуры 
функции Беллмана. Справиться с~этой проблемой в~большей степени удается 
за счет результата, полученного при решении дискретного по времени 
аналога рассматриваемой постановки~\cite{8-bos}. Конечные соотношения 
для оптимального решения, как и~во всех подобных задачах, включая 
классическую ли\-ней\-но-квад\-ра\-тич\-ную, содержат решения 
определенных дифференциальных уравнений (обыкновенных и~в~частных 
производных). Вывод этих уравнений и~со\-став\-ля\-ет содержание первой час\-ти 
данной работы. Во второй части будет обсуждаться их приближенное 
чис\-лен\-ное решение и~компьютерные эксперименты.
     
     Кратко обозначим основные положения, при\-вле\-ка\-емые далее 
к~решению задачи, следуя в~основном обозначениям 
и~терминологии~\cite{9-bos}, а~именно: будем рассматривать задачу 
оптимального управления в~стохастической динамической сис\-те\-ме по полной 
информации, применяя метод динамического программирования. В~качестве 
целевого функционала, опре\-де\-ля\-юще\-го качество управ\-ле\-ния $U_0^T\hm= \{ 
u_t,\ 0\leq t\leq T\}$, выступает
     \begin{equation}
     J\left(U_0^T\right)={\sf E}\left\{ \int\limits_0^T L_t \left(x_t, u_t\right)\,dt+ 
l\left(x_T\right)\right\}\,.
     \label{e1-bos}
     \end{equation}
Здесь ${\sf E}\{\cdot\}$~--- оператор математического ожидания; $x_t$~--- 
случайный процесс, описываемый стохастическим дифференциальным 
уравнением Ито
     \begin{equation}
     dx_t=m_t\left( x_t, u_t\right) dt+ \sigma_t\left( x_t\right)dW_t\,,\enskip 
x_0=X\,,
     \label{e2-bos}
     \end{equation}
где $W_t$~--- стандартный винеровский процесс подходящей раз\-мер\-ности; 
$X$~--- случайный вектор.

     $U_0^T$ будем выбирать из класса допустимых неупреждающих (по 
отношению к~$W_t$) управлений~\cite{9-bos}. Соответственно, 
относительно функций сноса и~диффузии~$m_t$ и~$\sigma_t$  
в~(\ref{e2-bos}) будем предполагать выполненными ка\-кие-ли\-бо условия 
существования сильного решения для заданного до\-пус\-ти\-мо\-го управ\-ле\-ния. 
Например, для управ\-ле\-ния с~обратной связью $u_t\hm= u_t(x_t)$ будем 
считать, что $m_t(x,u_t(x))$ и~$\sigma_t(x)$ удовлетворяют условию 
линейного рос\-та и~локальному условию Липшица по~$x$ равномерно 
по~$t$ (т.\,е.\ условиям Ито).
     
     Для поиска оптимального управления, минимизирующего $J(U_0^T)$, 
рас\-смат\-ри\-ва\-ет\-ся функция Беллмана
     \begin{equation}
     V_t(x)=\left.\mathop{\mathrm{inf}}\limits_{U_t^T} {\sf E} \left\{ \int\limits_t^T 
L_t \left( x_t, u_t\right)\,dt+l\left( x_T\right) \right\vert \mathcal{F}_t^x\right\}\,,
     \label{e3-bos}
     \end{equation}
где $\mathcal{F}_t^x$~--- $\sigma$-ал\-геб\-ра, по\-рож\-ден\-ная~$x_\tau$, 
$0\hm\leq \tau\hm\leq t$, ${\sf E}\{\cdot\vert \mathcal{F}\}$~--- оператор условного 
математического ожидания относительно~$\mathcal{F}$. Соответственно, 
в~качестве достаточного условия оп\-ти\-маль\-ности воспользуемся уравнением 
динамического программирования
\begin{multline}
\fr{\partial V_t(x)}{\partial t} +\fr{1}{2}\sum\limits^n_{i,j=1} \sigma^2_{t_{ij}}
\fr{\partial^2 V_t(x)}{\partial x_i \partial x_j}+{}\\
{}+\min\limits_u\left[  
\sum\limits^n_{i=1} m_{t_i} \fr{\partial V_t(x)}{\partial x_i} + L_t(x,u)\right] 
=0\,,\\
V_T(x)=l(x)\,,
\label{e4-bos}
\end{multline}
где $m_{t_i}$~--- $i$-й элемент век\-тор-функ\-ции~$m_t(x,u)$; 
$\sigma^2_{t_{ij}} \hm= \sum\nolimits^m_{k=1} 
\sigma_{t_{ik}}\sigma_{t_{ki}}$, $\sigma_{t_{ij}}$~--- $i$-й по строке, $j$-й 
по столб\-цу элемент мат\-рич\-ной функции~$\sigma_t(x)$; $n$ и~$m$~--- 
размерности~$x_t$ и~$W_t$ соответственно.

     Традиционно в~рамках применения метода динамического 
программирования будем предполагать, что функции~$L_t$, $l$, $m_t$ 
и~$\sigma_t$ обеспечивают существование хотя бы одного решения 
уравнения~(\ref{e4-bos}), а~следовательно, и~оптимального 
управления~$u_t^*$, $0\hm\leq t\hm\leq T$, до\-став\-ля\-юще\-го минимум 
целевому функционалу~(\ref{e1-bos}). Задача оптимизации далее получается 
путем указания конкретных выражений для~$L_t$, $l$, $m_t$ и~$\sigma_t$.

\section{Постановка задачи управления выходом}

     Рассматриваемые далее случайные функции будут предполагаться 
скалярными. Такое упрощение позволит разгрузить выкладки и~итоговые 
выражения от не самых существенных деталей.
     
     Рассмотрим стохастическую дифференциальную сис\-те\-му, со\-сто\-яние 
которой представляет диффузи\-он\-ный процесс~$y_t$, описываемый 
нелинейным стохастическим дифференциальным уравнением Ито
     \begin{equation}
     dy_t=A_t\left( y_t\right) dt +\Sigma_t \left( y_t\right) dv_t\,,\enskip 
y_0=Y\,,
     \label{e5-bos}
     \end{equation}
где $v_t$~--- стандартный (одномерный) винеровский процесс; $Y$~--- 
случайная величина с~конечным вторым моментом; функции~$A_t$ 
и~$\Sigma_t$ удовлетворяют условиям Ито:
\begin{equation*}
\left\vert A_t(y)\right\vert +\left\vert \Sigma_t(y)\right\vert \leq C(1+\vert y\vert )\ 
\mbox{для\ всех } 0\leq t\leq T\,;
\end{equation*}

\vspace*{-12pt}

\noindent
\begin{multline*}
\hspace*{-2.10051pt}\left\vert A_t\left(y_1\right) -A_t \left( y_2\right) \right\vert +\left\vert 
\Sigma_t\left( y_1\right) -\Sigma_t \left(y_2\right)\right\vert \leq
C\left\vert y_1-y_2\right\vert\\
 \mbox{для\ всех\ } 0\leq t\leq T\ \mbox{и } 
y_1,y_2\in \mathbb{R}^1\,,
\end{multline*}
обеспечивающим существование единственного сильного (потраекторного) 
решения уравнения.
     
     Будем считать, что~$y_t$ описывает состояние некоторой 
динамической системы. Соответственно, поведение этой сис\-те\-мы опишем 
выходом, линейно связанным с~со\-сто\-янием:
     \begin{equation}
     dz_t=a_t y_t \,dt+ b_t z_t \,dt+ c_t u_t \,dt+\sigma_t \,dw_t\,,\enskip
     z_0=Z\,.
     \label{e6-bos}
     \end{equation}
Здесь $w_t$~--- не зависящий от~$v_t$, $Y$ и~$Z$ стандартный (одномерный) 
винеровский процесс; $Z$~--- случайная величина с~конечным вторым 
моментом; $u_t$~--- допустимое неупреждающее управ\-ле\-ние, качество 
которого определяется целевым функционалом следующего вида:
\begin{multline}
\!\hspace*{-3.98538pt}J\left( U_0^T\right) ={\sf E}\left\{ \int\limits_0^T \!\left( S_t\left( s_ty_t-g_t z_t -h_t 
u_t\right)^2 +G_t z_t^2+{}\right.\right.\\
\left.\left.{}+ H_t u_t^2
\vphantom{S_t\left( s_ty_t-g_t z_t -h_t 
u_t\right)^2}
\right) dt+S_T\left( s_T y_T -g_T 
z_T\right)^2+G_T z_T^2
\vphantom{\int\limits_0^T}\right\}\,,
\label{e7-bos}
\end{multline}
где $S_t$, $G_t$ и~$H_t$~--- неотрицательные функции\linebreak
$0\hm\leq t\hm\leq T$. 
Такой критерий отражает физический смысл задачи распределения ресурсов 
со\-глас\-но аналогичной~(\ref{e5-bos})--(\ref{e7-bos}) задаче для дис\-крет\-но\-го 
времени, рас\-смот\-рен\-ной в~\cite{7-bos}. В~част\-ности,  
функци\-онал~(\ref{e7-bos}) поз\-во\-ля\-ет ставить задачи отслеживания
 выходом 
со\-сто\-яния сис\-те\-мы, используя сла\-га\-емое $(y_t\hm- z_t)^2$, или 
управлением~--- линейной комбинации со\-сто\-яния и~выхода, сла\-га\-емое типа\linebreak 
$(y_t\hm+ z_t\hm- u_t)^2$. Поскольку задача формулируется 
в~предположении наличия пол\-ной информации о~со\-сто\-янии~$y_t$ 
и~выходе~$z_t$ (соответствующую $\sigma$-ал\-геб\-ру 
обозначим~$\mathcal{F}_t^{y,z}$), то допустимое управ\-ле\-ние ищется 
в~классе~$\mathcal{F}_t^{y,z}$-из\-ме\-ри\-мых неупреждающих функций 
(и,~как будет показано далее, оказывается управ\-ле\-ни\-ем с~обратной связью).

     Функции~$a_t$, $b_t$, $c_t$ и~$\sigma_t$ будем предполагать 
ограниченными: $\vert a_t\vert \hm+ \vert b_t\vert \hm+\vert c_t\vert \hm+ \vert 
\sigma_t \vert \hm\leq C$ для всех $0\hm\leq t\hm\leq T$, процесс  
управления~--- допустимым не\-упреж\-да\-ющим~\cite{9-bos}, обеспечивая, 
таким образом, существование сильного решения урав\-не\-ния~(\ref{e6-bos}) 
для любого допустимого управ\-ления.
     
     Задачу составляет поиск~$u_t^*$~--- допустимого управ\-ле\-ния, 
доставляющего минимум квад\-ра\-тич\-но\-му функционалу~$J(U_0^T)$.
      
     Поставленная задача очевидным образом формулируется в~терминах 
введенных выше в~(\ref{e1-bos})--(\ref{e3-bos}) обозначений, а~именно: 
     требуется обозначить
     \begin{gather*}
      x_t=\begin{pmatrix}
     y_t\\ z_t\end{pmatrix};\quad  m_t(x_t, u_t)=\begin{pmatrix}
     A_t(y_t)\\ a_t y_t +b_t z_t +c_t u_t\end{pmatrix};\\
     \sigma_t(x_t)= \begin{pmatrix}
     \Sigma_t(y_t)& 0\\
     0& \sigma_t\end{pmatrix};\quad W_t=\begin{pmatrix}
     v_t \\ w_t\end{pmatrix}
     %     \label{e8-bos}
     \end{gather*}
для записи уравнения со\-сто\-яния типа~(\ref{e2-bos}) и
\begin{align*}
L_t(x,u)&= L_t(y,z,u) ={}\\
&\hspace*{3mm}{}=S_t\left( s_t y-g_t z -h_t u\right)^2 +G_t z^2 +H_t  u^2\,;\\
l(x)&= l(y,z) =S_T \left( S_T y-g_T z\right)^2 +G_T z^2
%\label{e9-bos}
\end{align*}
для записи целевого функционала в~виде~(\ref{e1-bos}).

     Функция Беллмана~(\ref{e3-bos}) принимает вид 
     $V_t(x)\hm= V_t(y,z)$. Для записи со\-от\-вет\-ст\-ву\-юще\-го~(\ref{e4-bos}) 
уравнения Беллмана для~$V_t(y,z)$ заметим, что
     $$
     \left( \sigma^2_{t_{ij}}\right)_{i,j=1,2}= \begin{pmatrix}
     \Sigma_t^2(y) & 0\\
     0 & \sigma_t^2\end{pmatrix}\,.
     $$
     
     С~учетом перечисленных обозначений урав\-не\-ние динамического 
программирования~(\ref{e4-bos}) принимает вид:
     \begin{multline}
     \fr{\partial V_t(y,z)}{\partial t} +\fr{1}{2}\left( \Sigma_t^2(y) \fr{\partial^2 
V_t(y,z)} {\partial y^2}+\sigma_t^2\fr{\partial^2 V_t(y,z)} {\partial 
z^2}\right)+{}\\
    {}+\min\limits_u\! \left[ A_t(y) \fr{\partial V_t(y,z)}{\partial y}+\left( a_t 
y+b_t z+c_t u\right) \fr{\partial V_t(y,z)}{\partial z} +{}\right.\hspace*{-3pt}\\
\left.{}+ S_t\left( s_t y-g_t z-h_t 
u\right)^2+G_t z^2+H_t u^2
     \vphantom{\fr{\partial V_t(y,z)}{\partial y}}\right] =0\,,\\
     V_T(y,z)=S_T\left( s_T y-g_T z\right)^2+G_T z^2\,.
     \label{e10-bos}
     \end{multline}
     Это и~есть то самое уравнение, которое требуется решить: 
существование решения данного урав\-не\-ния суть достаточное условие 
оптимальности; оптимальное управ\-ле\-ние при этом~--- точ\-ка минимума 
со\-от\-вет\-ст\-ву\-юще\-го сла\-га\-емого.
     
\section{Динамическое программирование и~оптимальное 
управление}

     В рассматриваемой постановке линейность\linebreak выхода и~квадратичность 
критерия дают те же преимущества, что и~в~классической  
ли\-ней\-но-квад\-ра\-тич\-ной задаче управ\-ле\-ния~\cite{1-bos}, а~именно: 
позволяют сразу определить вид оптимального управ\-ле\-ния и~фактические 
условия его существования. Действительно, со\-хра\-няя в~(\ref{e10-bos}) под 
знаком $\min\nolimits_u$ только члены, зависящие от~$u$, получаем
     \begin{multline*}
     \fr{\partial V_t(y,z)}{\partial t} +\fr{1}{2}\left( \Sigma_t^2(y) \fr{\partial^2 
V_t(y,z)} {\partial y^2}+\sigma_t^2\fr{\partial^2 V_t(y,z)} {\partial 
z^2}\right)+{}\\
     {}+A_t(y)\fr{\partial V_t(y,z)}{\partial y}+\left( a_t y+b_t z\right) 
\fr{\partial V_t(y,z)}{\partial z}+{}\\
{}+S_t\left( s_t y-g_t z\right)^2 +G_t z^2+{}
\end{multline*}

\noindent
\begin{multline*}
     {}+\min\limits_u \left[ \left( c_t \fr{\partial V_t(y,z)}{\partial z}-2S_t \left( 
s_t y-g_t z\right) h_t\right)u +{}\right.\\
\left.{}+\left( S_t h_t^2+H_t\right) u^2
\vphantom{\fr{\partial V_t(y,z)}{\partial z}}
\right]=0\,,
     %\label{e11-bos}
     \end{multline*}
откуда в~предположении $S_t h_t^2\hm+ H_t\hm>0$ следует, что существует 
оптимальное управ\-ле\-ние, которое определяется равенством
\begin{multline}
u_t^* = u_t^*(y,z)=-\fr{1}{2}\left( S_t h_t^2 +H_t\right)^{-1} \left( c_t 
\fr{\partial V_t(y,z)}{\partial z}-{}\right.\\
\left.{}-2S_t\left( s_t y-g_t z\right) h_t
\vphantom{\fr{\partial V_t(y,z)}{\partial z}}
\right)
\label{e12-bos}
\end{multline}
и доставляет минимум соответствующему сла\-га\-емо\-му в~урав\-не\-нии Беллмана, 
равный
$-\left( S_t h_t^2\hm+\right.$\linebreak
$\left.{}+H_t\right)^{-1} \left( c_t 
{\partial V_t(y,z)}/{\partial 
z}\hm-2S_t\left( s_t y \hm-g_t z\right) h_t \right)^2/4.
$ 
     
     Отметим, что, как и~в~классической ли\-ней\-но-квад\-ра\-тич\-ной 
задаче, управ\-ле\-ние из класса до\-пус\-ти\-мых не\-упреж\-да\-ющих получилось 
управ\-ле\-ни\-ем с~обратной связью.
     
     Таким образом, функция Беллмана описывается сле\-ду\-ющим 
дифференциальным уравнением:
     \begin{multline}
     \fr{\partial V_t(y,z)}{\partial t} +\fr{1}{2}\left( \Sigma_t^2(y) \fr{\partial^2 
V_t(y,z)} {\partial y^2}+\sigma_t^2\fr{\partial^2 V_t(y,z)} {\partial 
z^2}\right)+{}\\
     {}+ A_t(y) \fr{\partial V_t(y,z)}{\partial y}+\left( a_t y+b_t z\right) 
\fr{\partial V_t(y,z)}{\partial z}+{}\\
{}+ S_t \left( s_t y- g_t z\right)^2 +G_t z^2-
 \fr{1}{4}\left( S_t h_t^2+H_t\right)^{-1}\times{}\\
 {}\times \left( c_t \fr{\partial V_t(y,z)} 
{\partial z}-2S_t\left( s_t y -g_t z\right) h_t \right)^2=0\,.
     \label{e13-bos}
     \end{multline}
     
     Возводя в~квадрат по\-след\-нее сла\-га\-емое в~(\ref{e13-bos}), перепишем 
его в~виде:
     \begin{multline}
     \fr{\partial V_t(y,z)}{\partial t} +\fr{1}{2}\left( \Sigma_t^2(y) \fr{\partial^2 
V_t(y,z)} {\partial y^2}+\sigma_t^2\fr{\partial^2 V_t(y,z)} {\partial 
z^2}\!\right)+{}\\
{}+A_t(y) \fr{\partial V_t(y,z)}{\partial y}
+ \left( 
\vphantom{\left( S_t h_t^2 +H_t\right)^{-1}}
a_t y+b_t z+{}\right.\\
\left.{}+\left( S_t h_t^2 +H_t\right)^{-1}
 c_t S_t \left( s_t y-g_t z\right) h_t
\right) 
     \fr{\partial V_t(y,z)}{\partial z}+{}\\
     {}+\left( S_t-\left( S_t h_t^2 +H_t\right)^{-1} S_t^2 h_t^2\right)\left( s_t y -
g_t z\right)^2+{}\\
     \!\!{}+
     G_t z^2 -\fr{1}{4}\left( S_t h_t^2+H_t\right)^{-1}\! c_t^2
     \left(\fr{\partial V_t(y,z)}{\partial z}\right)^{\!2}=0\,.\!\!
     \label{e14-bos}
     \end{multline}
     
     Рассматривая полученное уравнение, заметим, что его решение может 
быть пред\-став\-ле\-но в~виде:
   \begin{equation}
     V_t(y,z)= \alpha_t z^2+\beta_t(y) z +\gamma_t(y)\,,
     \label{e15-bos}
     \end{equation}
т.\,е.\ будем искать решение при дополнительном предположении 
о~квад\-ра\-тич\-ности функции Белл\-ма\-на по переменной~$z$, и~сведем, таким 
образом, поиск оптимального решения к~уравнениям относительно функций 
$\alpha_t$, $\beta_t(y)$ и~$\gamma_t(y)$. Отметим сразу, что явный вид 
функции~$\gamma_t(y)$ для реализации оптимального управ\-ле\-ния не 
требуется, однако далее будет предложен вариант вы\-чис\-ле\-ния и~этой 
функции, что пред\-став\-ля\-ет\-ся небесполезным, поскольку позволит выполнять 
расчет минимума целевого функционала. Источником для 
предложения~(\ref{e15-bos}) является уже упоминавшаяся аналогичная 
задача для случая дис\-крет\-но\-го времени~\cite{7-bos, 8-bos}. В~той задаче 
выражение для функции Беллмана получается формально без 
дополнительных усилий. При этом форма~(\ref{e15-bos}) обнаруживается 
как свойство оптимального решения. В~рассматриваемом случае 
непрерывного времени~(\ref{e15-bos}) постулируется, а~пра\-виль\-ность 
постулата под\-тверж\-да\-ет\-ся далее ре\-зуль\-ти\-ру\-ющи\-ми уравнениями 
для~$\alpha_t$, $\beta_t(y)$ и~$\gamma_t(y)$ Кроме того, данное 
предположение пред\-став\-ля\-ет\-ся вы\-те\-ка\-ющим из линейной структуры задачи 
в~отношении переменной~$z$, в~част\-ности, тем фактом, что такой вид 
функции Беллмана обеспечивает линейность оптимального 
управ\-ле\-ния~(\ref{e12-bos}) по~$z$.

     Граничное условие при выбранном предположении~(\ref{e15-bos}) 
принимает вид:

\noindent
     \begin{multline*}
     V_T(y,z)= S_T\left( s_T y- g_T z\right)^2+G_T z^2 ={}\\[-0.5pt]
     {}=\alpha_T z^2 
+\beta_T(y) z +\gamma_T(y)\,,
    \end{multline*}
т.\,е.

\noindent
\begin{align*}
\alpha_T&= S_T g_T^2 +G_T\,;\\[-0.5pt]
\beta_T(y)&=-2S_T s_T g_T y\,;\\[-0.5pt]
\gamma_T(y)&=S_T s_T^2 y^2\,.
%\label{e16-bos}
\end{align*}
          При этом само оптимальное управ\-ле\-ние, определенное 
выражением~(\ref{e12-bos}), оказывается управ\-ле\-ни\-ем с~обратной связью 
по~$y_t$ и~$z_t$:

\noindent
     \begin{multline}
     u_t^*=u_t^*(y,z) ={}\\[-0.5pt]
     {}=
     -\fr{1}{2}\left( S_t h_t^2 +H_t\right)^{-1}
     \left( c_t \left( 2\alpha_t z +\beta_t(y)\right) +{}\right.\\[-0.5pt]
    \left. {}+2S_t\left( s_t y-g_t z\right) 
h_t\right)\,.
     \label{e17-bos}
     \end{multline}
          Подставляем $V_t(y,z)\hm= \alpha_t z^2 \hm+ \beta_t(y) 
z\hm+\gamma_t(y)$ в~(\ref{e14-bos}):

\noindent
     \begin{multline*}
     \fr{\partial \alpha_t}{\partial t}\, z^2 +
     \fr{\partial \beta_t(y)}{\partial t}\,z +
     \fr{\partial \gamma_t(y)}{\partial t}+{}\\[-0.5pt]
     {}+\fr{1}{2}\left( \Sigma_t^2(y) \left( 
\fr{\partial^2\beta_t(y)}{\partial y^2}\,z +\fr{\partial^2 \gamma_t(y)}{\partial 
y^2}\right) +2\sigma_t^2\alpha_t\right)+{}\\[-0.5pt]
 {}+A_t(y)\left(\fr{\partial \beta_t(y)}{\partial y}\,z + \fr{\partial 
\gamma_t(y)}{\partial y}\right) +{}\\[-0.5pt]
\hspace*{-0.22987pt}{}+\left( a_t y+b_t z+\left( S_t h_t^2 +H_t\right)^{-1} c_t S_t \left( s_t y-
g_t z\right) h_t\right)\times{}
\end{multline*}

\noindent
\begin{multline*}
         {}\times \left( 2\alpha_t z+\beta_t(y)\right)+{}\\
     {}+\left( S_t-\left( S_t h_t^2 +H_t\right)^{-1} S_t^2 h_t^2\right)\left( s_t y-
g_t z\right)^2+{}\\
     {}+ G_t z^2 -\fr{1}{4}\left( S_t h_t^2 +H_t\right)^{-1} c_t^2 \left( 
2\alpha_t z+\beta_t(y)\right)^2=0\,.
     \end{multline*}
          Далее выделяем слагаемые при~$z^2$, $z$ и~$z^0$
          
          \noindent
     \begin{multline*}
     \fr{\partial \alpha_t}{\partial t}\, z^2 +\fr{\partial \beta_t(y)}{\partial t}\,z +
     \fr{\partial \gamma_t(y)}{\partial 
t}+\fr{1}{2}\,\Sigma_t^2(y)\fr{\partial^2\beta_t(y)}{\partial y^2}\,z+ {}\\
{}+
\fr{1}{2}\,\Sigma_t^2(y)\fr{\partial^2\gamma_t(y)}{\partial 
y^2}+\sigma_t^2\alpha_t+A_t(y)\fr{\partial \beta_t(y)}{\partial y}\,z +{}\\
{}+A_t(y) \fr{\partial 
\gamma_t(y)}{\partial y}+{}\\
{}+ 2\alpha_t \left( b_t -\left( S_t h_t^2+H_t\right)^{-1} c_t 
S_t h_t g_t \right)z^2+{}\\
     {}+
     \left( 2\alpha_t\left( \alpha_t+\left( S_t h_t^2+H_t\right)^{-1} c_t S_t h_t 
s_t\right)y +{}\right.\\
\left.{}+\beta_t(y) \left( b_t-\left( S_t h_t^2+H_t\right)^{-1} c_t S_t h_t 
g_t\right) \right) z+{}\\
     {}+\beta_t(y)\left( a_t +\left( S_t h_t^2+H_t\right)^{-1} c_t S_t h_t s_t\right) 
y+{}\\
{}+ \left( S_t -\left( S_t h_t^2+H_t\right)^{-1} S_t^2 h_t^2\right) g_t^2 z^2-{}\\
     {}- 2\left( S_t -\left( S_t h_t^2+H_t\right)^{-1} S_t^2 h_t^2\right) s_t g_t yz 
+{}\\
{}+
     \left( S_t-\left( S_t h_t^2+H_t\right)^{-1} S_t^2 h_t^2\right) s_t^2 y^2+{}\\
     {}+G_t z^2 -\left( S_t h_t^2 +H_t\right)^{-1} c_t^2 \alpha_t^2 z^2 -{}\\
     {}-\left( 
S_t h_t^2+H_t\right)^{-1} c_t^2 \alpha_t \beta_t(y) z-{}\\
{}-
\fr{1}{4}\left( S_t h_t^2+H_t\right)^{-1}  c_t^2 \beta_t^2(y)=0\,,
     \end{multline*}
группируем их и~получаем сле\-ду\-ющие уравнения:
\begin{itemize}
\item  для~$\alpha_t$:

\noindent
\begin{multline}
\fr{\partial\alpha_t}{\partial t}+2\alpha_t\left( b_t-\left( S_t h_t^2+H_t\right)^{-1} c_t 
S_t h_t g_t\right)+{}\\
{}+ \left( S_t- \left( S_t h_t^2+H_t\right)^{-1} S_t^2 h_t^2\right) 
g_t^2+G_t-{}\\
\hspace*{-8mm}{}-\left( S_t h_t^2+H_t\right)^{-1} c_t^2 \alpha_t^2 =0\,,\enskip \alpha_T=S_T 
g_t^2+G_T\,;\!\!
\label{e18-bos}
\end{multline}
\item для $\beta_t$:

\noindent
\begin{multline}
\fr{\partial\beta_t(y)}{\partial 
t}+\fr{1}{2}\,\Sigma_t^2(y)\fr{\partial^2\beta_t(y)}{\partial y^2} 
+A_t(y)\fr{\partial \beta_t(y)}{\partial y}+{}\\
{}+ 2\alpha_t\left( a_t +\left( S_t h_t^2+H_t\right)^{-1} c_t S_t h_t s_t\right) y+{}\\
{}+
\beta_t(y)\left( b_t -\left( S_t h_t^2 +H_t\right)^{-1} c_t S_t h_t g_t\right)-{}\\
{}-2\left( S_t-\left( S_t h_t^2+H_t\right)^{-1} S_t^2 h_t^2\right) s_t g_t y-{}
\\
{}-
\left( S_t h_t^2+H_t\right)^{-1} c_t^2 \alpha_t \beta_t(y)=0\,,\\
\beta_T(y)=-2S_T s_T g_T y\,;
\label{e19-bos}
\end{multline}
\item  для $\gamma_t$:
\begin{multline}
\hspace*{-0.8pt}\fr{\partial \gamma_t(y)}{\partial t}+\fr{1}{2}\,\Sigma_t^2(y)
\fr{\partial^2 \gamma_t(y)}{\partial y^2} +\sigma_t^2 \alpha_t +A_t(y)
\fr{\partial \gamma_t(y)}{\partial y}+{}\\
{}+ \beta_t(y)\left( a_t +\left( S_t h_t^2+H_t\right)^{-1} c_t S_t h_t s_t\right) y+{}\\
{}+
\left( S_t-\left( S_t h_t^2+H_t\right)^{-1} S_t^2 h_t^2\right)  s_t^2 y^2-{}\\
{}-\fr{1}{4}\left( S_t h_t^2+H_t\right)^{-1} c_t^2 \beta_t^2(y) =0\,,\\
\gamma_T(y)=S_T s_T^2 y^2\,.
\label{e20-bos}
\end{multline}
\end{itemize}
     
     Уравнение~(\ref{e18-bos}), легко заметить, является уравнением 
Риккати, которое в~силу сформулированного выше условия   
имеет единственное неотрицательное решение для всех $0\hm\leq t\hm\leq T$. 
Этот факт требует дополнительного комментария. Для получения 
уравнения~(\ref{e18-bos}) рас\-смот\-рим исходную задачу при дополнительных 
условиях $a_t\hm=0$ и~$s_t\hm=0$ для всех $0\hm\leq t\hm\leq T$. Нетрудно 
видеть, что эти условия рассматриваемую по\-ста\-нов\-ку сводят фактически 
к~классической ли\-ней\-но-квад\-ра\-тич\-ной задаче. Имеющуюся 
в~рассматриваемой формулировке чуть более общую форму целевой 
функции (принципиального значения это обобщение, конечно, не имеет) 
сведем к~классической еще одним предположением: $S_t\hm=0$ для всех 
$0\hm\leq t\hm\leq T$. Теперь уравнение для~$\alpha_t$ принимает хорошо 
известный вид:
     \begin{equation}
     \fr{\partial \alpha_t}{\partial t}+2\alpha_t b_t +G_t- H_t^{-1} c_t^2 
\alpha_t^2=0\,,\enskip \alpha_T=G_T\,.
     \label{e21-bos}
     \end{equation}

     В таком случае, как известно~\cite{10-bos}, существует единственное 
оптимальное управление~--- линейное с~обратной связью по выходу~$z_t$, 
с~коэффициентом усиления, опи\-сы\-ва\-емым уравнением  
Риккати~(\ref{e21-bos}). Именно этот результат дают  
уравнения~(\ref{e18-bos})--(\ref{e20-bos}) и~описываемая ими функция 
Беллмана~(\ref{e15-bos}), так как из $a_t\hm=0$ и~$s_t\hm=0$ немедленно 
следует, что $\beta_t(y)\hm=0$, откуда, в~свою очередь, с~учетом 
не\-за\-ви\-си\-мости решения от~$y_t$ следует, что $\gamma_t(y)\hm=\gamma_t$, 
т.\,е.\ не зависит от~$y$ и~задается уравнением: 
     $$
     \fr{\partial \gamma_t(y)}{\partial t} +\sigma^2_t \alpha_t=0\,,\enskip 
\gamma_T=0\,.
     $$ 
     Оптимальное управ\-ле\-ние при этом 
     $$
     u_t^*= -H_t^{-1} c_t \alpha_t z_t\,,
     $$
      т.\,е.\ все полностью совпадает с~известным классическим решением.
     
     С уравнениями~(\ref{e19-bos}) и~(\ref{e20-bos}) ситуация, естественно, 
обстоит сложнее. Это линейные уравнения второго порядка параболического 
типа, поскольку\linebreak
 $\Sigma_t^2(y)\hm>0$. Фактически отсутствуют 
конструктивные условия, гарантирующие существование их\linebreak
 решений 
(требовать, чтобы все фигурирующие в~уравнениях коэффициенты были 
представлены аналитическими функциями на всем пространстве значений, 
вряд ли целесообразно), поэтому далее будем предполагать, что данные 
уравнения имеют на рас\-смат\-ри\-ва\-емом интервале $0\hm\leq t\hm\leq T$ хотя 
бы одно ограниченное решение и~именно эти условия будем рас\-смат\-ри\-вать 
как достаточные условия существования оптимального решения 
рассматриваемой задачи.
     
     Таким образом, доказана следующая тео\-рема.
     
     \smallskip
     
     \noindent
     \textbf{Теорема.}\ \textit{Пусть для диффузионного 
процесса}~(\ref{e5-bos}) \textit{выполнены условия Ито, для 
     процесса}~(\ref{e6-bos})~--- \textit{ограничены коэффициенты, 
уравнения}~(\ref{e18-bos})--(\ref{e20-bos}) \textit{имеют ограниченные 
решения для $0\hm\leq t\hm\leq T$. Тогда минимум  
функционалу}~(\ref{e7-bos}) \textit{доставляет оптимальное 
управ\-ле\-ние}~(\ref{e17-bos}), \textit{где} $y\hm= y_t$; $z\hm=z_t$.
     
\section{Заключение}

     Рассмотренная задача оптимизации в~целом близка и~по модели, и~по 
критерию к~классической ли\-ней\-но-квад\-ра\-тич\-ной постановке. 
Принципиальным отличием является нелинейная модель для описания 
со\-сто\-яния динамической сис\-те\-мы, в~которой отсутствует управ\-ля\-ющее 
воздействие.\linebreak
 Такую модель наряду с~традиционной интер\-пре\-тацией  
<<со\-сто\-яние--вы\-ход>> мож\-но понимать как\linebreak модель неконтролируемого 
слож\-но\-го внешнего воздействия. Небольшое дополнительное отличие дает 
предложенная форма квад\-ра\-тич\-но\-го критерия, поз\-во\-ля\-ющая, в~част\-ности, 
ставить такие задачи, как отслеживание выходом или управ\-ле\-ни\-ем со\-сто\-яния 
сис\-те\-мы или ее выхода.
     
     Поскольку обсуждать возможности точного решения уравнений, 
определяющих оптимальное управ\-ле\-ние, не имеет смыс\-ла, наиболее 
актуальной далее является задача их приближенного чис\-лен\-но\-го решения 
и~анализа воз\-мож\-ности практической реализации. Этому посвящена вторая 
часть данной работы, пла\-ни\-ру\-емая к~выходу в~ближайшее время.

{\small\frenchspacing
 {%\baselineskip=10.8pt
 \addcontentsline{toc}{section}{References}
 \begin{thebibliography}{99}
\bibitem{1-bos}
\Au{Athans M.} Editorial on the LQG problem~// IEEE~T. Automat. Contr., 1971. Vol.~16. 
No.\,6. P.~528--552. doi: 10.1109/TAC.1971.1099845.
\bibitem{2-bos}
\Au{Wu Z.} Forward-backward stochastic differential equations, linear quadratic stochastic 
optimal control and nonzero sum differential games~// J.~Syst. Sci. Complex., 2005. Vol.~18. 
No.\,2. P.~179--192.
\bibitem{3-bos}
\Au{Chen B.\,S., Zhang~W.} Stochastic H2/H1 control with state-dependent noise~// IEEE 
T.~Automat. Contr., 2004. Vol.~49. No.\,1. P.~45--56. doi: 10.1109/TAC.2003.821400.
\bibitem{4-bos}
\Au{Bohacek S.} A~stochastic model of TCP and fair video transmission~// IEEE 
INFOCOM, 2003. Vol.~2. P.~1134--1144. doi: 10.1109/INFCOM.2003.1208950.
\bibitem{5-bos}
\Au{Домбровский В.\,В., Объедко~Т.\,Ю.} Управление с~прогнозированием системами 
с~марковскими скачками при ограничениях и~применение к~оптимизации 
инвестиционного портфеля~// Автомат. телемех., 2011. №\,5. С.~96--112. doi: 
10.1134/S0005117911050079.
\bibitem{6-bos}
\Au{Баландин Д.\,В., Коган~М.\,М.} Оптимальное линейно-квад\-ра\-тич\-ное управление: от 
матричных уравнений к~линейным матричным неравенствам~// Автомат. телемех., 2011. 
№\,11. С.~60--69. doi: 10.1134/ S0005117911110038.
\bibitem{7-bos}
\Au{Босов А.\,В.} Обобщенная задача распределения ресурсов программной системы~// 
Информатика и~её применения, 2014. Т.~8. Вып.~2. С.~39--47. doi: 
10.14357/19922264140204.
\bibitem{8-bos}
\Au{Босов А.\,В.} Управление линейным выходом дискретной стохастической системы по 
квадратичному критерию~// Изв. РАН. Теория и~системы управления, 2016. №\,3.  
С.~19--35. doi: 10.1134/S1064230716030060.
\bibitem{9-bos}
\Au{Флеминг У., Ришел~Р.} Оптимальное управление детерминированными 
и~стохастическими системами~/ Пер. с~англ.~--- М.: Мир, 1978. 316~с. 
(\Au{Fleming~W.\,H., Rishel~R.\,W.} Deterministic and stochastic optimal control.~--- New 
York, NY, USA: Springer-Verlag, 1975. 222~p.)
\bibitem{10-bos}
\Au{Девис М.\,Х.\,А.} Линейное оценивание и~стохастическое управление~/ Пер. с~англ.~--- 
М.: Наука, 1984. 206~с. (\Au{Davis~M.\,H.\,A.} Linear estimation and stochastic control.~--- 
London: Chapman and Hall, 1977. 224~p.)

 \end{thebibliography}

 }
 }

\end{multicols}

\vspace*{-6pt}

\hfill{\small\textit{Поступила в~редакцию 30.03.18}}

\vspace*{4pt}

%\newpage

%\vspace*{-24pt}

\hrule

\vspace*{2pt}

\hrule

\vspace*{-2pt}


\def\tit{STOCHASTIC DIFFERENTIAL SYSTEM OUTPUT CONTROL 
BY~THE~QUADRATIC CRITERION.~I.~DYNAMIC\\ PROGRAMMING 
OPTIMAL SOLUTION}


\def\titkol{Stochastic differential system output control 
by~the~quadratic criterion. I.~Dynamic programming 
optimal solution}

\def\aut{A.\,V.~Bosov and~A.\,I.~Stefanovich}

\def\autkol{A.\,V.~Bosov and~A.\,I.~Stefanovich}

\titel{\tit}{\aut}{\autkol}{\titkol}

\vspace*{-11pt}


\noindent
Institute of Informatics Problems, Federal Research Center ``Computer Science 
and Control'' of the Russian Academy of Sciences, 44-2~Vavilov Str., Moscow 
119333, Russian Federation


\def\leftfootline{\small{\textbf{\thepage}
\hfill INFORMATIKA I EE PRIMENENIYA~--- INFORMATICS AND
APPLICATIONS\ \ \ 2018\ \ \ volume~12\ \ \ issue\ 3}
}%
 \def\rightfootline{\small{INFORMATIKA I EE PRIMENENIYA~---
INFORMATICS AND APPLICATIONS\ \ \ 2018\ \ \ volume~12\ \ \ issue\ 3
\hfill \textbf{\thepage}}}

\vspace*{3pt}



\Abste{The problem of optimal control for the Ito diffusion 
process and a~controlled linear output is solved. The considered 
statement is close to the classical linear-quadratic Gaussian 
control  (LQG control) problem. Differences consist in the fact 
that the state is described by the nonlinear differential Ito equation  $dy_y = A_t(y_t) 
\,dt+\Sigma_t(y_t)\,dv_t$ and does not depend on the control~$u_t$, 
optimization subject is controlled linear output 
 $dz_t=a_ty_t\,dt +b_tz_t\,dt +c_t u_t\,dt +\sigma_t \,dw_t$. 
Additional generalizations are included in the quadratic 
quality criterion for the purpose of statement such problems 
as state tracking by output or a linear combination of state 
and output tracking by control. The method of dynamic programming 
is used for the solution. 
The assumption about Bellman function in the form  $V_t(y,z)= \alpha_t 
z^2+\beta_t(y) z+\gamma_t(y)$ allows one to find it. 
Three differential equations for the coefficients $\alpha_t$,  $\beta_t(y)$,
and $\gamma_t(y)$ give the solution. 
These equations constitute the optimal solution of the problem under consideration.}

\KWE{stochastic differential equation; optimal control; dynamic programming; 
Bellman function; Riccati equation; linear differential equations of parabolic type}


\DOI{10.14357/19922264180314}

\vspace*{-12pt}

\Ack
\noindent
This work was partially supported by the Russian Science Foundation (grant  
16-07-00677).



%\vspace*{6pt}

  \begin{multicols}{2}

\renewcommand{\bibname}{\protect\rmfamily References}
%\renewcommand{\bibname}{\large\protect\rm References}

{\small\frenchspacing
 {%\baselineskip=10.8pt
 \addcontentsline{toc}{section}{References}
 \begin{thebibliography}{99}
\bibitem{1-bos-1}
\Aue{Athans, M.} 1971. Editorial on the LQG problem. \textit{IEEE~T. 
Automat. Contr.} 16(6):528--552. doi: 10.1109/ TAC.1971.1099845.
\bibitem{2-bos-1}
\Aue{Wu, Z.} 2005. Forward-backward stochastic differential equations, linear 
quadratic stochastic optimal control and\linebreak\vspace*{-12pt}

\columnbreak

\noindent
 nonzero sum differential games. 
\textit{J.~Syst. Sci. Complex.} 18(2):179--192.
\bibitem{3-bos-1}
\Aue{Chen, B.\,S. and W.~Zhang.} 2004. Stochastic H2/H1 control with  
state-dependent noise. \textit{IEEE~T. Automat. Contr.} 49(1):45--56.
doi: 10.1109/TAC.2003.821400.
\bibitem{4-bos-1}
\Aue{Bohacek, S.} 2003. A~stochastic model of TCP and fair video 
transmission. \textit{IEEE INFOCOM}. 2:1134--1144.
doi: 10.1109/INFCOM.2003.1208950.
\bibitem{5-bos-1}
\Aue{Dombrovskii, V.\,V., and T.\,Yu.~Ob''edko.} 2011. Predictive control of 
systems with Markovian jumps under constraints and its application to the 
investment portfolio optimization. \textit{Automat. Rem. Contr.}  
72(5):989--1003.
\bibitem{6-bos-1}
\Aue{Balandin, D.\,V., and M.\,M.~Kogan.} 2011. Optimal linear-quadratic 
control: From matrix equations to linear matrix inequalities. \textit{Automat. 
Rem. Contr.} 72(11):2276--2284.
\bibitem{7-bos-1}
\Aue{Bosov, A.\,V.} 2014. Obobshchennaya zadacha raspredeleniya resursov 
programmnoy sistemy [The generalized problem of software system resources 
distribution]. \textit{Informatika i~ee Primeneniya~--- Inform. Appl.}  
8(2):39--47. doi: 
10.14357/19922264140204.
\bibitem{8-bos-1}
\Aue{Bosov, A.\,V.} 2016. Discrete stochastic system linear output control 
with respect to a quadratic criterion. \textit{J.~Comput. Syst. Sc. 
Int.} 55(3):349--364.
\bibitem{9-bos-1}
\Aue{Fleming, W.\,H., and R.\,W.~Rishel.} 1975. \textit{Deterministic and 
stochastic optimal control.} New York, NY: Springer-Verlag. 222~p.
\bibitem{10-bos-1}
\Aue{Davis, M.\,H.\,A.} 1977. \textit{Linear estimation and stochastic 
control.} London: Chapman and Hall. 224~p.
\end{thebibliography}

 }
 }

\end{multicols}

\vspace*{-6pt}

\hfill{\small\textit{Received March 30, 2018}}

%\pagebreak

%\vspace*{-18pt}
     
     \Contr
     
       \noindent
       \textbf{Bosov Alexey V.} (b.\ 1969)~--- Doctor of Science in technology, 
principal scientist, Institute of Informatics Problems, Federal Research 
Center ``Computer Science and Control'' of the Russian Academy of Sciences, 
44-2~Vavilov Str., Moscow 119333, Russian Federation; 
\mbox{AVBosov@ipiran.ru}
       
       \vspace*{3pt}
       
       \noindent
       \textbf{Stefanovich Alexey I.} (b.\ 1983)~--- principal specialist, 
Institute of Informatics Problems, Federal Research Center ``Computer Science 
and Control'' of the Russian Academy of Sciences, 44-2~Vavilov Str., Moscow 
119333, Russian Federation; \mbox{AStefanovich@frccsc.ru}
\label{end\stat}

\renewcommand{\bibname}{\protect\rm Литература}       

          %1
\def\stat{kudr}

\def\tit{ПРИБЛИЖЕННЫЕ МЕТОДЫ РЕШЕНИЯ ЗАДАЧИ ДИАГНОСТИКИ ПЛОСКИМ 
ЗОНДОМ СИЛЬНОИОНИЗОВАННОЙ ПЛАЗМЫ С~УЧЕТОМ КУЛОНОВСКИХ 
СТОЛКНОВЕНИЙ}

\def\titkol{Приближенные методы решения задачи диагностики плоским 
зондом сильноионизованной плазмы} %с~учетом Кулоновских  столкновений}

\def\autkol{И.\,А.~Кудрявцева, А.\,В.~Пантелеев}
\def\aut{И.\,А.~Кудрявцева$^1$, А.\,В.~Пантелеев$^2$}

\titel{\tit}{\aut}{\autkol}{\titkol}

%{\renewcommand{\thefootnote}{\fnsymbol{footnote}}\footnotetext[1]
%{Работа поддержана Российским фондом фундаментальных исследований
%(проекты 11-01-00515а и 11-07-00112а), а также Министерством
%образования и науки РФ в рамках ФЦП <<Научные и
%научно-педагогические кадры инновационной России на 2009--2013~годы>>.}}


\renewcommand{\thefootnote}{\arabic{footnote}}
\footnotetext[1]{Московский авиационный институт, irina.home.mail@mail.ru}
\footnotetext[2]{Московский авиационный институт, avpanteleev@inbox.ru}

\vspace*{-2pt}

\Abst{Сформирована математическая модель, описывающая динамику сильноионизованной 
плазмы с учетом столкновений заряженных частиц вблизи плоского зонда. Модель включает уравнение 
Фоккера--Планка и уравнение Пуассона. Предложено два подхода к решению задачи: на основе метода 
статистических испытаний Мон\-те-Кар\-ло и на основе композиции метода крупных частиц и метода 
расщепления.} 

\vspace*{-2pt}

\KW{телекоммуникационные системы; метод Монте-Карло; метод крупных частиц; метод 
расщепления; зонд; уравнение Фоккера--Планка; уравнение Пуассона} 

\vspace*{-4pt}

 \vskip 8pt plus 9pt minus 6pt

      \thispagestyle{headings}

      \begin{multicols}{2}
      
            \label{st\stat}

\section{Введение}

В настоящее время в области телекоммуникаций все более востребованными становятся 
информационные технологии, основанные на использовании математических моделей и численных 
методов физики плазмы. Поэтому особенно актуальным является решение разнообразных задач анализа 
поведения плазмы, включающих в себя формирование новых моделей и методов их исследования. 
Помимо этого, в разработке телекоммуникационного оборудования эффективно используются 
собственно физические свойства плазмы. В~частности, изготовлена антенна, работа которой основана 
на газовом разряде низкотемпературной плазмы~[1], интенсивно ведутся разработки по созданию и 
усовершенствованию источников бесперебойного питания на основе плазменных элементов~[2, 3]. 
      
      Одним из наиболее перспективных направлений для построения систем оптической 
беспроводной связи является использование лазеров~\cite{4-k, 5-k}. В~этой связи большое внимание 
уделяется использованию плазмы при разработке импульсных сильноточных коммутаторов~\cite{6-k}, 
так как практическое применение подобных разработок требует повышения уровня надежности и 
быстродействия лазерных систем.
      
      Исследования низкотемпературной плазмы также связаны с разработками в области дальней 
космической связи, так как моделирование процессов взаимодействия заряженного тела с верхними 
слоями атмосферы позволяет предлагать способы улучшения существующих систем радиосвязи с 
космическими летательными аппаратами~\cite{7-k}. 
      
      Наряду с этим актуальными также являются задачи диагностики плазмы, поскольку перспективы 
ее использования в области телекоммуникаций после более полного изучения физических свойств 
могут значительно расшириться. 

Для диагностики плазмы применяют зондовые методы исследования~[8--11]. Эти методы относятся к 
классу контактных методов; как следствие, возникает сложность в исследовании пристеночной области 
вблизи зонда, которая характеризуется достаточно сложным распределением потенциала и функциями 
распределения, отличными от максвелловских. 

Данная работа посвящена исследованию переходного режима обтекания заряженного тела плазмой. Для 
переходного режима выполняется следующее условие: длина свободного пробега иона до столкновения 
с нейтральным атомом или другим ионом невелика по сравнению с характерными размерами тела. 
В~этом случае возникает необходимость учета столкновений заряженных частиц с нейтральными 
атомами и кулоновских столкновений. В~работах~\cite{10-k, 11-k} подробно рассмотрена модель с 
учетом столкновений заряженных частиц с нейтральными атомами. В~настоящей статье представлена 
теоретическая модель, описывающая влияния ион-ионных и ион-элек\-т\-рон\-ных столкновений на 
измеряемые характеристики плазмы, что ранее детально не исследовалось.
      
      В~рамках данной работы предлагается модель, описывающая динамику сильноионизованной 
плазмы с учетом кулоновских столкновений. Эта модель учитывает такие процессы взаимодействия, 
как перенос частиц и столкновения между заряженными частицами типа <<ион--ион>> и 
      <<ион--электрон>> под влиянием макроскопического электрического поля. Перечисленные 
процессы описываются самосогласованной системой уравнений, включающей уравнение 
      Фок\-ке\-ра--План\-ка и уравнение Пуассона~[12].
      
      Вычислительная модель задачи строится на основе двух методов: метода статистических 
испытаний Мон\-те-Кар\-ло и композиции метода крупных частиц и метода расщепления. Приведены 
результаты численного моделирования, полученные с использованием вышеперечисленных методов.

\vspace*{-4pt}

\section{Постановка задачи}

\vspace*{-2pt}

Рассматривается следующая физическая постановка зондовой задачи~[11]. В~невозмущенную 
бесконечно протяженную плазму, состоящую из электронов и однозарядных ионов, внесена большая\linebreak 
заряженная до потенциала $\varphi_p$ плоскость. Плоскость, расположенная поперек потока плазмы, 
является идеально поглощающей для электронов. Ионы при ударе о плоскость нейтрализуются. 
Предполагается, что частицы в плазме движутся под действием внешнего электрического поля, 
магнитное поле отсутствует. Концентрации ионов $n_{i\infty}$ и электронов $n_{e\infty}$, а также 
температуры данных час\-тиц~$T_{i\infty}$ 
и~$T_{e\infty}$ в невозмущенной плазме заданы. За начальные 
функции распределения обоих типов час\-тиц принимаются функции распределения Максвелла. 
      
      Требуется с учетом столкновений между заряженными частицами найти напряженность 
самосогласованного электрического поля $\vec{E}(\vec{r},t)$, функции распределения однозарядных 
ионов $f_i(\vec{r}, \vec{v}, t)$ и электронов $f_e(\vec{r}, \vec{v}, t)$, 
а также их моменты (плотности 
токов ионов и электронов  $j_i(\vec{r},t)\hm
=q\int f_i(\vec{r}, \vec{v}, t)\vec{v}\,d\vec{v}$, $j_e(\vec{r},t) 
\hm={\sf e}\int f_e(\vec{r},\vec{v},t)\vec{v}\,d\vec{v}$, где $q=Z_i{\sf e}$, $Z_i=1$~--- заряд иона, ${\sf 
e}$~--- заряд электрона; концентрации ионов и электронов $n_i(\vec{r},t)\hm=\int 
f_i(\vec{r},\vec{v},t)\,d\vec{v}$, $n_e(\vec{r},t)\hm=\int f_e(\vec{r},\vec{v}, t)\,d\vec{v}$). 
Поведение частиц во 
времени~$t$ характеризуется ра\-ди\-ус-век\-то\-ром~$\vec{r}$ и вектором скорости~$\vec{v}$.
      
      Математическая модель, соответствующая данной физической постановке задачи, имеет 
вид~\cite{11-k, 13-k}:

\noindent
      \begin{equation}
      \left.
      \begin{array}{c}
      \fr{\partial f_\alpha (\vec{r},\vec{v},t)}{\partial t}+
      \vec{v}\fr{\partial f_\alpha (\vec{r},\vec{v},t)}{ 
\partial \vec{r}}+
\fr{\vec{F}_\alpha(\vec{r},t)}{m_\alpha}\times{}\\[4pt]
{}\times\fr{\partial f_\alpha(\vec{r},\vec{v},t)}{ \partial 
\vec{v}}=
\left(\fr{\partial f_\alpha(\vec{r},\vec{v},t)}{ \partial t}\right)_{\mathrm{с}}+S_\alpha 
(\vec{r},\vec{v},t)\,;\\[6pt]
      \Delta\varphi(\vec{r},t)=-\fr{{\sf e}}{\varepsilon_0}\left( n_i(\vec{r},t)-n_e(\vec{r},t)\right)\,;\\[6pt]
      \vec{E}(\vec{r},t)=-\nabla \varphi(\vec{r},t)\,.
      \end{array}\!\!
      \right\}\!\!
      \label{e1-k}
      \end{equation}
Здесь первое уравнение~--- уравнение Фок\-ке\-ра--План\-ка для частиц сорта~$\alpha$ ($\alpha=i,e$), 
второе~--- уравнение Пуассона для самосогласованного электрического поля; 
$f_\alpha(\vec{r},\vec{v},t)$~--- функция\linebreak
распределения час\-тиц сорта~$\alpha$; $(\partial 
f_\alpha(\vec{r},\vec{v},t)/\partial t)_{\mathrm{с}}$~--- 
оператор столкновений Фок\-ке\-ра--План\-ка; 
функция~$S_\alpha(\vec{r},\vec{v},t)$ описывает источники или стоки\linebreak
 час\-тиц; 
$\vec{F}_\alpha(\vec{r},t)=q_\alpha\vec{E}(\vec{r},t)$, где $\vec{E}(\vec{r},t)$~--- напряженность 
самосогласованного электрического поля, 
$$
q_\alpha =
\begin{cases}
-{\sf e}\,, & \alpha=e\,,\\
{\sf e}\,, & \alpha=i\,;
\end{cases}
$$
$\varphi(\vec{r},t)$~--- потенциал самосогласованного электрического поля; $n_\alpha(\vec{r},t)$ ($\alpha 
\hm=i,e$)~--- концентрация частиц сорта~$\alpha$; $m_\alpha$~--- масса частицы сорта~$\alpha$; 
$\varepsilon_0$~--- электрическая постоянная. 

Оператор столкновений Фок\-ке\-ра--План\-ка имеет вид~\cite{13-k, 14-k}
\begin{multline*}
\fr{1}{\Gamma_\alpha}\left( \fr{\partial f_\alpha}{\partial t}\right)_{\mathrm{с}} 
=\fr{1}{2}\,\nabla_v\nabla_v:\left(f_\alpha\nabla_v\nabla_vg_\alpha(\vec{r},\vec{v},t)\right)-{}\\
{}-
\nabla_v\cdot\left(f_\alpha\nabla_v h_\alpha\right)\,,
\end{multline*}
где $\nabla_v\nabla_v g_\alpha(\vec{r},\vec{v},t)$~--- ковариантная тензорная производная второго ранга, 
знак двоеточия ($:$) обозначает операцию двойного суммирования:
\begin{gather*}
\Gamma_\alpha=\fr{Z_\alpha^4 {\sf e}^4}{4\pi \varepsilon_0^2 m^2_\alpha}\,\ln D_\alpha\,;
\\
D_\alpha =\fr{12\pi\varepsilon_0 kT_{\alpha\infty}}{Z_\alpha^2 {\sf e}^2}\left( \fr{\varepsilon_0 k 
T_{e\infty}}{n_{e\infty} {\sf e}^2}\right)^{1/2}\,;\\
g_\alpha (\vec{r},\vec{v},t)=\sum\limits_{b=i,e}\left( \fr{Z_b}{Z_\alpha}\right) \int f_b 
(\vec{r},{\vec{v}}^{\,\prime},t)\left\vert \vec{v}-{\vec{v}}^{\,\prime}\right\vert\,d\vec{v}^{\,\prime}\,;\\
h_\alpha (\vec{r},\vec{v},t)=\sum\limits_{b=i,e} \fr{m_\alpha+m_b}{m_b} 
\left(\fr{Z_b}{Z_\alpha}\right)
\int
\fr{f_b(\vec{r},{\vec{v}}^{\,\prime}, t)}{\vert \vec{v}-{\vec{v}}^{\,\prime}\vert}
\,d{\vec{v}}^{\,\prime}\,;\\
Z_\alpha =1\,, \quad \alpha=i,e\,.
\end{gather*}
 
К системе уравнений~(\ref{e1-k}) необходимо добавить начальные и краевые условия:
\begin{equation}
\!\left.
\begin{array}{rrl}
t=0:\ & f_\alpha(\vec{r},\vec{v},0)&=f_\alpha^{\mathrm{maksv}}\,,\enskip \alpha=i,e;\\[9pt]
\vec{r}\in \Omega_p:\ & f_\alpha(\vec{r},\vec{v},t)\big\vert_{\vec{r}\in\Omega_p}&=0\,,\enskip \alpha=i,e\,;\\[9pt]
&\varphi(\vec{r},t)\big\vert_{\vec{r}\in\Omega_p}&=\varphi_p\,;\\[9pt]
\vec{r}\in\Omega_\infty:\ & 
f_\alpha(\vec{r},\vec{v},t)\big\vert_{\vec{r}\in\Omega_\infty}&= %{}\\[9pt]
f_\alpha^{\mathrm{maksv}}\,,\enskip \alpha=i,e\,;\\[9pt]
&\varphi(\vec{r},t)\big\vert_{\vec{r}\in\Omega_\infty}&=0\,,
\end{array}\!\!
\right\}\!\!\!\!
\label{e2-k}
\end{equation}
    где 
    
    \noindent
    \begin{multline*}
    f_\alpha^{\mathrm{maksv}}=n_{\alpha\infty}\left(\fr{m_\alpha}{2k\pi T_{\alpha\infty}}\right)^{3/2}\times{}\\
    {}\times
    \exp\left( -
\fr{m_\alpha}{2kT_{\alpha\infty}}\left\vert\vec{v}-\vec{v}_\infty\right\vert^2\right)\,,
\enskip \alpha=i, e\,;
\end{multline*} 
$\Omega_p$ и $\Omega_\infty$~--- множество радиус-векторов час\-тиц, концы которых принадлежат плоскости зонда и 
границе возмущенной зоны соответственно.

Для решения поставленной задачи введем декартову систему координат таким образом, чтобы 
заряженная плоскость совпала с плоскостью~$0xz$. Тогда положение частицы в пространстве будет 
определяться координатами $x,y,z$, а скорость~--- координатами $v_x, v_y, v_z$. В~силу того что 
плоскость является бесконечно большой в сравнении с характерным размером задачи, функции 
распределения частиц будут зависеть только от переменных $y, v_y, t$.

Поставленную задачу предлагается решать независимо двумя методами. Первый метод основывается на 
методе статистических испытаний Мон\-те-Кар\-ло, второй метод является композицией метода 
расщепления и метода крупных частиц.

\section{Применение метода Монте-Карло}

Запишем самосогласованную систему уравнений~(\ref{e1-k}) и~(\ref{e2-k}) в декартовой системе 
координат с учетом сделанных предположений:
\begin{equation}
\left.
\begin{array}{l}
\fr{\partial f_\alpha}{\partial t}+
v_y\fr{\partial f_\alpha}{\partial y}+\fr{F_y^\alpha}{m_\alpha}\,\fr{\partial 
f_\alpha}{\partial v_y}=\fr{1}{2}\,\fr{\partial^2 }{\partial [v_y]^2}\times{}\\
{}\times \left( 
f_\alpha\fr{\partial^2 g_\alpha  }{\partial [v_y]^2}\right) -
\fr{\partial}{\partial v_y}\left( f_\alpha\fr{\partial h_\alpha}{\partial v_y}\right)\,,
\enskip \alpha=i,e\,;\\[6pt]
    \fr{\partial^2\varphi}{\partial y^2} =-\fr{{\sf e}}{\varepsilon_0}\left(n_i-n_e\right)\,;
    \enskip E_y=-
\fr{\partial\varphi}{\partial y}\,;\\[6pt]
\hspace*{3.1mm}    t=0:\  \hspace*{2.6mm}f_\alpha(y,v_y,0)=f_\alpha^{\mathrm{maksv}}\,,\ \alpha=i,e\,;\\[9pt]
\hspace*{2.9mm} y=0:\ \hspace*{2.8mm}f_\alpha(0,v_y,t)=0\,,\ \alpha=i,e\,;\\[9pt]
\hspace*{24.3mm}\varphi(0,t)=\varphi_p\,;\\[9pt]
y=y_\infty:\ f_\alpha(y_\infty, v_y, t)=f_\alpha^{\mathrm{maksv}}\,,\ \alpha=i,e\,;\\[9pt]
\hspace*{21.5mm}\varphi(y_\infty, t)=0\,.
\end{array}
\right \}
\label{e3-k}
\end{equation}

В полученной системе уравнений~(\ref{e3-k}) перейдем к безразмерным величинам, применив 
соотношение $X=M_X \hat{X}$, где $M_X$~--- масштаб размерной величины~$X$, $\hat{X}$~--- 
безразмерная величина~$X$. В~качестве используемых масштабов были взяты следующие: радиус 
Дебая, скорость теплового движения частиц, концентрация частиц в невозмущенной плазме, потенциал, 
возникающий при разделении зарядов в дебаевской сфере, и производные от них величины.

Система безразмерных уравнений имеет следующий вид:
%\noindent
\begin{equation}
\left.
\begin{array}{l}
\fr{\partial 
\hat{f}_\alpha}{\partial\hat{t}}+A_\alpha\fr{\partial\hat{f}_\alpha}{\partial\hat{y}}+
B_\alpha\hat{E}_y\fr{\partial\hat{f}_\alpha}{\partial \hat{v}_y}={}\\
\!{}=
\fr{\partial^2}{\partial[\hat{v}_y]^2}\left(D_\alpha 
\hat{f}_\alpha\right)-\fr{\partial}{\partial\hat{v}_y}\left(K_\alpha \hat{f}_\alpha\right),\enskip 
\alpha=i,e;\\[9pt]
\fr{\partial^2\hat{\varphi}}{\partial\hat{y}^2}=-\left(\hat{n}_i-\hat{n}_e\right)\,;\enskip \hat{e}_y=-
\fr{\partial\hat\varphi}{\partial\hat{y}}\,;\\[9pt]
\hspace*{3.1mm}\hat{t}=0:\ \hspace*{2.6mm}\hat{f}_\alpha(\hat{y},\hat{v}_y,0)=\hat{f}_\alpha^{\mathrm{maksv}}\,,\enskip \alpha-i,e\,;\\[9pt]
\hspace*{2.9mm}\hat{y}=0:\ \hspace*{2.8mm}\hat{f}_\alpha(0,\hat{v}_y,\hat{t})=0\,,\enskip \alpha=i,e\,;\\[9pt]
\hspace*{24.3mm}\hat\varphi(0,\hat{t})=\hat{\varphi}_p\,;\\[9pt]
\hat{y}=\hat{y}_\infty:\ \hat{f}_\alpha(\hat{y}_\infty, \hat{v}_y, \hat{t})=\hat{f}^{\mathrm{maksv}}_\alpha\,,\enskip 
\alpha=i,e\,;\\[9pt]
\hspace*{21.5mm}\hat\varphi(\hat{y}_\infty,\hat{t})=0\,.
\end{array}
\right\}
\label{e4-k}
\end{equation}
Здесь 

\vspace*{-2pt}

\noindent
\begin{gather*}
A_\alpha=\sqrt{\delta_\alpha }\,\hat{v}_y\,;\enskip 
B_\alpha=\sqrt{\delta_\alpha}\,\fr{z_\alpha}{2\varepsilon_\alpha}\,;\\
\delta_\alpha=\fr{\varepsilon_\alpha}{\mu_\alpha}\,;\enskip 
\varepsilon_\alpha=\fr{T_{\alpha\infty}}{T_{i\infty}}\,;\\
\mu_\alpha=\fr{m_\alpha}{m_i}\,;\enskip 
D_\alpha=A_g^\alpha\fr{\partial^2\hat{g}_\alpha}{\partial  [\hat{v}_y]^2}\,;\\
K_\alpha=A_h^\alpha \fr{\partial \hat{h}_\alpha}{\partial \hat{v}_y}\,,\enskip \alpha=i,e\,,
\end{gather*}
где $A_g^\alpha$ и $A_h^\alpha$~--- коэффициенты, определяемые характерными параметрами 
задачи~\cite{15-k}.

Поиск решения самосогласованной системы уравнений~(\ref{e4-k}) осуществляется по следующей 
схе-\linebreak ме. Вначале находятся значения напряженности\linebreak
 электрического поля по значениям потенциала, 
полученным из граничной задачи для уравнения Пуассона. Далее, используя найденные значения 
напряженности, решается уравнение Фок\-ке\-ра--План\-ка путем перехода к стохастическому 
дифференциальному уравнению (СДУ) Ито:

\noindent
\begin{multline*}
d\Theta_\alpha(\hat{t}) = a_\alpha \left(\hat{t},\Theta_\alpha(\hat{t})\right)+{}\\
{}+\sigma\left(
\hat{t},\Theta_\alpha(\hat{t})\right)\,dW(\hat{t})\,,\quad \alpha=i,e\,,
%\label{e5-k}
\end{multline*}
где 

\noindent
\begin{align*}
\Theta_\alpha(\hat{t})&=\begin{bmatrix}
\hat{y}(\hat{t})\\ \hat{v}_y(\hat{t})
\end{bmatrix}\,;\\
a_\alpha\left(\hat{t},\Theta_\alpha(\hat{t})\right)&=\begin{bmatrix}
-A_\alpha\\ -K_\alpha -B_\alpha \hat{E}_y
\end{bmatrix}\,;\\
\sigma_\alpha\left(\hat{t},\Theta_\alpha(\hat{t})\right)\sigma_\alpha^{\mathrm{T}}\left( 
\hat{t},\Theta_\alpha(\hat{t})\right)&=D_\alpha\,,\enskip \alpha=i,e\,;
\end{align*} 
$W(\hat{t})$~--- стандартный винеровский случайный процесс.
\pagebreak

Для нахождения значений вектора состояния~$\Theta_\alpha(\hat{t})$ применим явную разностную 
схему стохастического метода Эйлера~\cite{16-k}:
\begin{multline*}
\Theta_\alpha^{n+1}=\Theta_\alpha^n +h_\tau a_\alpha \left( \hat{t}_n, \Theta_\alpha^n\right)+\sigma_\alpha 
\left( \hat{t}_n, \Theta_\alpha^n\right)\Delta W_n\,,\\ 
n=0,\ldots , N\,,\ \alpha=i,e\,,
%\label{e6-k}
\end{multline*}
где $\Theta_\alpha^n$, $n=0,\ldots , N$,~--- приближенное значение вектора 
состояния~$\Theta_\alpha(\hat{t})$, $\alpha=i,e$, в момент времени $\hat{t}\hm=\hat{t}_n$, 
$\hat{t}_n\hm=n h_\tau$, $n=0,\ldots , N$; $h_\tau$~--- достаточно малый шаг интегрирования; $\Delta 
W_n$, $n=0,\ldots ,N$,~--- величина приращения винеровского процесса~$W(\hat{t})$ на отрезке $\left[ 
\hat{t}_n,\,\hat{t}_{n+1}\right]$, по определению независимая от~$\Theta_\alpha^0$, 
$\Delta W_0,\ldots , 
\Delta W_{n-1}$: $\Delta W_n\hm=W(\hat{t}_{n-1})\hm-W(\hat{t}_n)$; $\Delta W_n\hm\sim N(0,\,h_\tau)$, 
т.\,е.\ $\Delta W_n$ представляют собой гауссовские случайные величины с нулевыми математическими 
ожиданиями и дисперсиями, равными шагу интегрирования; $\Theta_\alpha^0$~--- значение вектора 
состояния $\Theta_\alpha(\hat{t})$, $\alpha\hm=i,e$, в момент времени $\hat{t}=0$, 
$\Theta_\alpha^0\hm\sim \hat{f}_\alpha^{\mathrm{maksv}}$. 

Частные производные $\partial^2\hat{g}_\alpha/\partial[\hat{v}_y]^2$ и $\partial \hat{h}_\alpha/\partial 
\hat{v}_y$, являющиеся составляющими матрицы $\sigma_\alpha (\hat{t}_n, 
\Theta_\alpha^n)\sigma_\alpha^{\mathrm{T}}(\hat{t}_n,\Theta_\alpha^n)$ и вектора $a_\alpha(\hat{t}_n, 
\Theta_\alpha^n)$ соответственно, аппроксимируются со вторым порядком точности на трехточечном 
шаблоне на основе значений~$\hat{g}_\alpha$ и~$\hat{h}_\alpha$~\cite{17-k}.
      
      В выражения для функций~$\hat{g}_\alpha$ и~$\hat{h}_\alpha$ входят интегралы, которые 
вычисляются методом Мон\-те-Кар\-ло с использованием набора значений скоростной компоненты 
вектора состояния~$\hat{v}_y$, полученных из решения СДУ Ито:
      \begin{equation*}
      \int \hat{f}_\alpha \left\vert \hat{v}_y-
\hat{v}_y^\prime\right\vert\,dv_y^\prime=M\left(\zeta\left(\hat{V}_y\right)\right)\,,
\end{equation*}
где
$$
      \zeta\left(\hat{V}_y\right)=\left\vert \hat{v}_y-\hat{V}_y\right\vert\,,\enskip \hat{V}_y\sim 
\hat{f}_\alpha\,.
  $$
      
      Для вычисления напряженности самосогласованного электрического поля $\hat{E}_y=-
\partial\hat{\varphi}/\partial\hat{y}$, входящей в вектор $a_\alpha(\hat{t}_n, \Theta_\alpha^n)$, необходимо 
аналогично аппроксимировать со вторым порядком точности производную 
$\partial\hat{\varphi}/\partial\hat{y}$ на трехточечном шаблоне с использованием значений 
потенциала~$\hat{\varphi}$~\cite{17-k}. Значения потенциала~$\hat\varphi$ находятся из решения 
уравнения Пуассона. 
      
      Граничную задачу для уравнения Пуассона 
      \begin{align*}
      \fr{\partial^2 \hat\varphi}{\partial \hat{y}^2} & = -\left(\hat{n}_i-\hat{n}_e\right)\,;\\
      \hat{\varphi}\big|_{\hat{y}=0} &=\hat{\varphi}_p\,;\\
      \hat{\varphi}\big|_{\hat{y}_\infty=0} &=0
      \end{align*}
    предлагается решать путем перехода к конечно-разностной системе с последующим ее решением 
методом прогонки~\cite{17-k}:

\noindent
\begin{gather*}
\hat{\varphi}^n_{l-1}+2\hat{\varphi}_l^n+\hat{\varphi}^n_{l+1}=
h_y\hat{\delta}_l^n\,,\enskip l=1,\ldots , 
N_y\,;\\
\hat{\delta}_l^n=-\left( \hat{n}^n_{i,l}-\hat{n}^n_{e,l}\right)\,;\enskip 
\hat{\varphi}_0=\hat{\varphi}_p\,;\enskip \hat{\varphi}_{N_y}=0\,,
\end{gather*}
где $N_y$~--- число шагов по переменной~$\hat{y}$, $h_y$~--- величина шагов разбиения по~$\hat{y}$. 
      
      Концентрации $\hat{n}_\alpha$, $\alpha=i,e$, и плотности токов частиц на зонд~$\hat{f}_\alpha$, 
$\alpha=i,e$, вычисляются согласно описанному выше методу Мон\-те-Карло.

\section{Применение метода расщепления и~метода крупных~частиц}

Решение задачи в данном случае предлагается начать с записи правой части уравнения 
Фок\-ке\-ра--План\-ка в декартовой системе координат в виде:
$$
\mathbf{Q} f_\alpha = \fr{1}{2}\,\fr{\partial^2 f_\alpha}{\partial [v_y]^2}\,\fr{\partial^2 g_\alpha}{\partial 
[v_y]^2}+\fr{\partial f_\alpha}{\partial v_y}\,\fr{\partial C_\alpha}{\partial v_y}+H_\alpha\,,\enskip 
\alpha=i,e\,,
$$  
где 
\begin{align*}
C_\alpha(\vec{r},\vec{v},t)&=
\begin{cases}
\fr{1-\gamma}{Z_i^2}\int\fr{f_e(\vec{r},{\vec{v}}^{\,\prime},t)}{|\vec{v}-{\vec{v}}^{\,\prime} |}\,d{\vec{v}}^{\,\prime}\,, 
&\alpha=i\,;\\[9pt]
\fr{Z_i^2(\gamma-1)}{\gamma}\int \fr{f_i(\vec{r},{\vec{v}}^{\,\prime}, t)}
{|\vec{v}-{\vec{v}}^{\,\prime} 
|}\,d{\vec{v}}^{\,\prime}\,, &\alpha=e\,;
\end{cases} 
\\
H_\alpha&=
\begin{cases}
4\pi \left( \fr{\gamma f_e}{Z_i^2}+f_i\right)f_i\,, & \alpha=i\,;\\[9pt]
4\pi\left(\fr{Z_i^2 f_i}{\gamma}+f_e\right)f_e\,, &\alpha=e\,.
\end{cases}
\end{align*}
Тогда при переходе к безразмерным величинам (см.\ разд.~3) система~(\ref{e1-k}) запишется 
следующим образом:
      \begin{equation}
      \left.
\!\!\begin{array}{l}
      \fr{\partial 
\hat{f}_\alpha}{\partial\hat{t}}+A_\alpha\fr{\partial\hat{f}_\alpha}{\partial\hat{y}}+
B_\alpha  \hat{E}_y
\fr{\partial\hat{f}_\alpha}{\partial\hat{v}_\alpha}=\tilde{\mathbf{Q}}\hat{f}_\alpha\,,\enskip 
\alpha=i,e;\\[9pt]
      \fr{\partial^2\hat{\varphi}}{\partial\hat{y}^2}=-\left( \hat{n}_i-\hat{n}_e\right)\,,\enskip \hat{E}_y=-
\fr{\partial\hat\varphi}{\partial\hat{y}}\,,\\[9pt]
\hspace*{3.1mm}\hat{t}=0:\ \hspace*{2.6mm}\hat{f}_\alpha(\hat{y},\hat{v}_y, 0)=\hat{f}_\alpha^{\mathrm{maksv}}\,,\enskip \alpha=i,e\,,\\[9pt]
\hspace*{2.9mm} \hat{y}=0:\ \hspace*{2.8mm}\hat{f}_\alpha(0,\hat{v}_y,\hat{t})=0\,,\enskip \alpha=i,e\,;\\[9pt]
\hspace*{24.3mm}\hat\varphi(0,\hat{t})=\hat{\varphi}_p\,;\\[9pt]
      \hat{y}=\hat{y}_\infty:\ \hat{f}_\alpha(\hat{y}_\infty, 
\hat{v}_y,\hat{t})=\hat{f}_\alpha^{\mathrm{maksv}}\,,\enskip \alpha=i,e\,;\\[9pt]
\hspace*{21.5mm}\hat{\varphi}(\hat{y}_\infty,\hat{t})=0\,,\\[9pt]
    \end{array}
\right\}\!\!
\label{e7-k}
\end{equation}
где 
\begin{gather*}
\tilde{\mathbf{Q}} \hat{f}_\alpha=D_\alpha\fr{\partial^2\hat{f}_\alpha}{\partial 
[\hat{v}_y]^2}+K_\alpha\fr{\partial\hat{f}_\alpha}{\partial\hat{v}_y}+H_\alpha\,;\\
D_\alpha=A_g^\alpha\fr{\partial^2\hat{g}_\alpha}{\partial [\hat{v}_y]^2}\,;\enskip 
K_\alpha=A_h^\alpha \fr{\partial \hat{h}_\alpha}{\partial\hat{v}_y}\,,\ \alpha=i,e\,.
\end{gather*}

Для решения системы уравнений~(\ref{e7-k}) применяется модификация метода 
расщепления~\cite{17-k}, согласно которой исходная задача разбивается на две вспомогательные. Такое 
разбиение можно осуществить, переписав уравнение Фок\-ке\-ра--План\-ка в следующем виде:
$$
\fr{\partial\hat{f}_\alpha}{\partial\hat{t}} =
\tilde{\mathbf{Q}}_1\hat{f}_\alpha+\tilde{\mathbf{Q}}_2\hat{f}_\alpha\,,
$$
где 
\begin{align*}
\tilde{\mathbf{Q}}_1\hat{f}_\alpha &=-
\left(A_\alpha\fr{\partial\hat{f}_\alpha}{\partial\hat{y}}+
B_\alpha\fr{\partial\hat{f}_\alpha}{\partial\hat{y}}
\right)\,;\\
\tilde{\mathbf{Q}}_2\hat{f}_\alpha 
&=\left(D_\alpha\fr{\partial^2\hat{f}_\alpha}{\partial[\hat{v}_y]^2}+K_\alpha\fr{\partial 
\hat{f}_\alpha}{\partial\hat{v}_y}+H_\alpha\right)\,.
\end{align*}

      Правая часть уравнения Фок\-ке\-ра--План\-ка представляет собой сумму двух операторов, 
первый из которых отвечает за перенос частиц, второй~--- за столкновения заряженных частиц. 
В~результате образуются следующие задачи, которые решаются последовательно:
      \begin{itemize}
\item первая задача:
\begin{align*}
&\fr{\partial w_\alpha(\hat{y},\hat{v}_y,\hat{t})}{\partial\hat{t}} =\mathbf{Q}_1 
w_\alpha(\hat{y},\hat{v}_y,\hat{t})\,,\enskip \alpha=i,e\,;\\[9pt]
&\fr{\partial^2\hat\varphi}{\partial\hat{y}^2}=-\left(\hat{n}_i-\hat{n}_e\right)\,;\enskip
\hat{E}_y=-
\fr{\partial\hat\varphi}{\partial\hat{y}}\,;\\[9pt]
&w_\alpha(\hat{y},\hat{v}_y,\hat{t}^n)=\hat{f}_\alpha(\hat{y},\hat{v}_y,\hat{t}^n)\,,\enskip n=0,\ldots ,N-
1\,;\\[9pt]
&\hspace{2.9mm}\hat{y}=0:\ \hspace*{2.9mm}w_\alpha(0,\hat{v}_y,\hat{t})=0\,,\enskip \alpha=i,e\,;\\[9pt]
&\hspace*{25.1mm}\hat\varphi(0,\hat{t})=\hat{\varphi}_p\,;\\[9pt]
&\hat{y}=\hat{y}_\infty:\ w_\alpha(\hat{y}_\infty, \hat{v}_y, \hat{t})=
\hat{f}_\alpha^{\mathrm{maksv}}\,,\enskip 
\alpha=i,e\,;\\[9pt]
&\hspace*{22.5mm}\hat\varphi(\hat{y}_\infty,\hat{t})=0\,;
\end{align*}
\item вторая задача:
\begin{align*}
\!\!\!\!\!\!\!\fr{\partial s_\alpha(\hat{y},\hat{v}_y,\hat{t})}{\partial \hat{t}} &=\mathbf{Q}_2 
s_\alpha(\hat{y},\hat{v}_y,\hat{t})\,, & \alpha&=i,e\,;\\
\!\!\!\!\!\!\!s_\alpha (\hat{y},\hat{v}_y,\hat{t}^n) &=w_\alpha (\hat{y},\hat{v}_y, \hat{t}^{n+1}),& n&=0,\ldots ,N-
1.
\end{align*}
\end{itemize}

Первая задача представляет собой систему безразмерных уравнений Вла\-со\-ва--Пуас\-со\-на. Для ее 
решения применяется метод крупных частиц~\cite{18-k}. Согласно этому методу решение задачи 
осуществляется путем расщепления на два этапа: на первом этапе не учитываются конвективные члены 
и решение получается обычным интегрированием на неподвижной эйлеровой сетке, а на втором этапе 
рассматривается система, которая описывает перенос частиц в лагранжевой системе координат. Кроме 
того, на первом этапе необходимо решить уравнение Пуассона для получения значений потенциала 
самосогласованного электрического поля. Для этого применяется метод, описанный в разд.~3. 

Вторая задача решается путем перехода к ко\-неч\-но-раз\-ност\-ной сис\-те\-ме. При этом частные 
производные $\partial^2\hat{g}_\alpha/\partial[\hat{v}_y]^2$ и $\partial\hat{h}_\alpha/\partial\hat{v}_y$ 
аппроксимируются со вторым порядком точности с использованием трехточечного шаблона, а 
производная $\partial s_\alpha/\partial\hat{t}$ аппроксимируется на двухточечном шаблоне с первым 
порядком точности~\cite{16-k}. К~полученной системе разностных уравнений предлагается применить 
один из классических методов решения систем линейных уравнений, например метод 
Гаусса~\cite{19-k}.
      
      Решением первой задачи является функция $w_\alpha(\hat{y}, \hat{v}_y, \hat{t}^n)$, 
$n\hm=0,\ldots ,N$, , которая дает начальное условие для второй задачи. Решая вторую задачу, находим 
функцию $s_\alpha(\hat{y},\hat{v}_y,\hat{t}^n)\hm=\hat{f}_\alpha(\hat{y},\hat{v}_y,\hat{t}^n)$, 
$n=1,\ldots ,N$, $\alpha=i,e$, которая определяет решение $\hat{f}_\alpha(\hat{y},\hat{v}_y,\hat{t}^n)$, 
$\alpha=i,e$, исходной системы~(\ref{e7-k}) для рассматриваемых моментов времени $n=1,\ldots ,N$.

Моменты функций распределения $\hat{f}_\alpha$, $\alpha=i,e$, находятся с помощью методов 
численного интегрирования, например метода трапеций~\cite{19-k}.

\section{Результаты численного моделирования}

Для двух описанных выше методов реализованы две отдельные программы в среде {Matlab~7.0}. 
Эти программы позволяют по заданным значениям концентраций и температур частиц $n_{i\infty}$, 
$n_{e\infty}$, $T_{i\infty}$ и~$T_{e\infty}$ в невозмущенной плазме, а также потенциала~$\varphi_p$, 
подаваемого на зонд, изучить эволюцию во времени плотностей тока частиц~$j_i$ и~$j_e$, концентраций 
частиц~$n_i$  и~$n_e$ в произвольной точке пространства в возмущенной зоне, а также динамику 
изменения напряженности~$E_y$ самосогласованного электрического поля во времени и пространстве.

С использованием разработанных программ проведены серии расчетных экспериментов, в которых 
значение концентраций варьировалось в пределах $n_{i\infty} \hm = n_{e\infty}\hm =10^{18}\div 
10^{22}$~м$^{-3}$. Значение температур было выбрано неизменным и равным $T_{i\infty}\hm = 
T_{e\infty}\hm=3000$~K, а значения потенциала, подаваемого на зонд, изменялись в пределах 
$\varphi_p\hm=0\div 2{,}6$~В.

На рис.~1  и~2 приведены графики изменения напряженности самосогласованного электрического
 поля (см.\ рис.~1) и плотности токов ионов (см.\linebreak\vspace*{-12pt}

\pagebreak

\end{multicols}

\begin{figure} %fig1
\vspace*{1pt}
\begin{center}
\mbox{%
\epsfxsize=162.594mm
\epsfbox{kud-1.eps}
}
\end{center}
\vspace*{-9pt}
\Caption{Динамика изменения плотности тока ионов во времени в фиксированной точке возмущенной 
зоны для значений потенциала: \textit{1}~--- $\varphi_p=-6$; 
\textit{2}~--- $\varphi_p=-16$; \textit{3}~--- $\varphi_p=- 30$ 
в случае применения методов Монте-Карло~(\textit{а}) 
и крупных частиц~(\textit{б})}
\end{figure}

\begin{figure} %fig2
\vspace*{1pt}
\begin{center}
\mbox{%
\epsfxsize=162.713mm
\epsfbox{kud-2.eps}
}
\end{center}
\vspace*{-9pt}
\Caption{Динамика изменения напряженности электрического поля во времени в фиксированной точке 
возмущенной зоны для значений потенциала: 
\textit{1}~--- $\varphi_p=-6$; \textit{2}~--- $\varphi_p=-16$; 
\textit{3}~--- $\varphi_p=-30$ в случае применения методов Монте-Карло~(\textit{а}) и
крупных частиц~(\textit{б})
}
\end{figure}

\begin{multicols}{2}

\noindent
 рис.~2) во времени в фиксированной точке пространства 
возмущенной зоны в случае применения обоих разработанных алгоритмов.


На основании полученных результатов можно отметить похожее поведение зависимостей 
напряженности электрического поля и плотности тока от времени в двух рассматриваемых случаях. 
Графики кривых сначала убывают, затем начинают возрастать, выходя в некоторый момент 
времени~$t^\prime$ (момент установления) на стационарные значения. 

Одинаковое поведение 
напряженности и плот\-ности тока можно объяснить из следующих соображений: плотность тока ионов в 
данной области пространства равна произведению концентрации ионов на их направленную скорость и 
на заряд иона. Скорость ионов, в свою очередь, зависит от заряда, массы и напряженности 
электрического поля. 
%\columnbreak

При внесении в плазму отрицательно заряженного зонда возникает электрическое поле, которое 
нарушает квазинейтральность плазмы. Для того чтобы компенсировать действие внешнего 
электрического поля, ионы устремляются к зонду, а электроны~--- от зонда. Это приводит к дисбалансу 
концентраций вблизи зонда и, как следствие, к увеличению разности потенциалов; график 
напряженности электрического поля убывает. Вскоре разделение зарядов компенсирует внешнее 
электрическое поле; график выходит на стационарное значение. 

Также можно отметить, что значения 
напряженности электрического поля и плотности тока частиц на зонд в момент установления для двух 
методов совпадают. 

Момент установления~$t^\prime$ зависит от при\-ме\-ня\-емо\-го метода решения. В~случае метода 
Мон\-те-Кар\-ло $t^\prime=3{,}5\div 4$~ед., а для метода крупных частиц совместно с методом 
расщепления $t^\prime\hm=5\div 5{,}5$~ед. Используя ко\-неч\-но-раз\-ност\-ный метод, можно 
получить динамику изменения функций распределения частиц~$f_\alpha$, $\alpha=i,e$, во времени и 
пространстве. Функции распределения позволяют наглядно представить влияние на картину 
распределения частиц вблизи зонда самой поверхности зонда и электрического поля.

\section{Заключение}
      
      В работе найдено решение задачи диагностики плоским зондом сильноионизованной плазмы с 
учетом столкновений заряженных частиц. Разработана математическая модель исследуемого явления, 
описываемая уравнениями Фок\-ке\-ра--План\-ка и Пуассона. Решение получено двумя методами:\linebreak 
статистическим и ко\-неч\-но-раз\-ност\-ным на основе\linebreak сформированных алгоритмов. Приведены 
резуль-\linebreak таты численного моделирования при различных\linebreak характерных параметрах задачи.
 Из  проведенных 
вычислительных экспериментов вытекает, что искомые величины: напряженность 
электрического поля, плотности токов частиц на зонд, концентрации частиц вблизи зонда~--- как по 
характеру зависимости, так и по числовым значениям совпадают. При применении метода 
      Мон\-те-Кар\-ло момент установления наступает быстрее по сравнению с конечно-разностным 
методом, однако конечно-разностный метод позволяет получить более наглядные результаты.

{\small\frenchspacing
{%\baselineskip=10.8pt
\addcontentsline{toc}{section}{Литература}
\begin{thebibliography}{99}

\bibitem{1-k}
\Au{Alexeff I., Anderson T.}
Experimental and theoretical results with plasma antenna~// IEEE Trans. Plasma Sci., 2006. Vol.~34. 
No.\,2. P.~166--172.

\bibitem{2-k}
\Au{Сысун В.\,И.}
Сильноионизованная низкотемпературная плазма в приборах электронной техники: Методы 
исследования, свойства, применение. Дисс. \ldots д-ра физ.-мат. наук в форме науч. докл.: 
01.04.08.~--- Пет\-ро\-за\-водск, 1996.

\bibitem{3-k}
\Au{Тухас В.\,А.}
Методология создания средств измерений и испытаний на устойчивость к кондуктивным помехам~// 
Мат-лы VI Междунар. симп. по электромагнитной совместимости и 
электромагнитной экологии.~--- СПб., 2005. С.~231--234.

\bibitem{4-k}
\Au{Гудзенко Л.\,И., Яковленко С.\,И.}
Плазменные лазеры.~--- М.: Атомиздат, 1978.  256~с.

\bibitem{5-k}
\Au{Звелто О.}
Принципы лазеров.~--- М.: Мир, 1990.  560~с.

\bibitem{6-k}
\Au{Сысун В.\,И., Хромой Ю.\,Д.}
Расширение канала мощного импульсного разряда в парах ртути~// Электронная техника, 1974. 
Сер.~4. Вып.~10. С.~80--85. 

\bibitem{7-k}
\Au{Винклер Дж.\,Р.}
Искусственные пучки частиц в космической плазме.~--- М.: Мир, 1985.  451~с.

\bibitem{8-k}
\Au{Bernstein I.\,B., Rabinowitz I.\,N.}
Theory of electrostatic probes in low-density plasma~// Phys. Fluids, 1959. Vol.~2. No.\,2. P.~112--121. 

\bibitem{9-k}
\Au{Альперт Я.\,Л., Гуревич А.\,В., Питаевский~Л.\,П.}
Искусственные спутники в разреженной плазме.~--- М.: Наука, 1964.  282~с.

\bibitem{10-k}
\Au{Чан П., Тэлбот Л., Турян~К.}
Электрические зонды в неподвижной и движущейся плазме.~--- М.: Мир, 1978.  202~с.

\bibitem{11-k}
\Au{Алексеев Б.\,В., Котельников В.\,А.}
Зондовый метод диагностики плазмы.~--- М.: Энергоатомиздат, 1989.  240~с.

\bibitem{12-k}
\Au{Пантелеев А.\,В., Кудрявцева И.\,А.}
Формирование математической модели двухкомпонентной плазмы с учетом столкновений 
заряженных частиц в случае плоского зонда~// Теоретические вопросы вычислительной техники и 
программного обеспечения: Межвузовский сб. научн. тр.~--- М.: МИРЭА, 2006. С.~11--21.

\bibitem{13-k}
\Au{Олдер Б.}
Вычислительные методы в физике плазмы.~--- М.: Мир, 1974.  111~с.

\bibitem{14-k}
\Au{Montgomery D.\,C., Tidman D.\,A.}
Plasma kinetic theory.~--- New York, 1964. 

\bibitem{15-k}
\Au{Кудрявцева И.\,А., Пантелеев А.\,В.}
Применение метода Мон\-те-Кар\-ло для анализа поведения двухкомпонентной плазмы с учетом 
столкновений между заряженными частицами~// Теоретические вопросы\linebreak
вычислительной техники и 
программного обеспечения: Межвузовский сб. научн. тр.~--- М.: МИРЭА, 2008. С.~122--128. 

\bibitem{16-k}
\Au{Семенов В.\,В., Пантелеев А.\,В., Руденко~Е.\,А., Бор\-та\-ков\-ский~А.\,С.}
Методы описания, анализа и синтеза нелинейных систем управления.~--- М.: МАИ, 1993.  312~с.

\bibitem{17-k}
\Au{Киреев В.\,И., Пантелеев А.\,В.}
Численные методы в примерах и задачах.~--- М.: Высшая школа, 2006.  480~с.

\bibitem{18-k}
\Au{Белоцерковский О.\,М., Давыдов~Ю.\,М.}
Метод крупных частиц в газовой динамике. Вычислительный эксперимент.~--- М.: Наука, 
Физматгиз, 1982.

\label{end\stat}

\bibitem{19-k}
\Au{Вержбицкий В.\,М.}
Основы численных методов.~--- М.: Высшая школа, 2002.  840~с.
 \end{thebibliography}
}
}


\end{multicols}         %2
\def\stat{malashenko}

\def\tit{ПОСЛЕДОВАТЕЛЬНЫЙ АНАЛИЗ И~МЕТРИЧЕСКИЕ ОЦЕНКИ ПРЕДЕЛЬНЫХ
РАСПРЕДЕЛЕНИЙ МЕЖУЗЛОВЫХ ПОТОКОВ В~МНОГОПОЛЬЗОВАТЕЛЬСКОЙ СЕТИ}

\def\titkol{Последовательный анализ и~метрические оценки предельных
распределений межузловых потоков в %~многопользовательской 
сети}

\def\aut{Ю.\,Е. Малашенко$^1$}

\def\autkol{Ю.\,Е. Малашенко}

\titel{\tit}{\aut}{\autkol}{\titkol}

\index{Малашенко Ю.\,Е.}
\index{Malashenko Yu.\,E.}


%{\renewcommand{\thefootnote}{\fnsymbol{footnote}} \footnotetext[1]
%{Исследование выполнено при финансовой поддержке Российского научного фонда (проект 
%<<Информатика>> ФИЦ ИУ РАН, Москва).}}


\renewcommand{\thefootnote}{\arabic{footnote}}
\footnotetext[1]{Федеральный исследовательский центр <<Информатика и~управление>> Российской академии 
\mbox{mala-yur@yandex.ru}}


%\vspace*{-6pt}



\Abst{Для оценки функциональных возможностей
многопользовательской сети связи аналилизируется множество векторов межузловых потоков при предельных распределениях ресурсов
сети. В~рамках многопродуктовой модели про\-пуск\-ные спо\-соб\-ности ребер рас\-смат\-ри\-ва\-ют\-ся 
как компоненты вектора ресурсов различных
типов, которые требуются для передачи потоков различных видов.
При проведении вычислительных экспериментов на каждой итерации вычисляются нормы векторов совместно допустимых межузловых
потоков, при передаче которых полностью используется пропускная спо\-соб\-ность всех ребер сети. Полученные последовательности
метрических оценок позволяют анализировать особенности множества до\-сти\-жи\-мости и~эф\-фек\-тив\-ность использования ресурсов сети при
уравнительном распределении про\-пуск\-ной спо\-соб\-ности между корреспондентами.}

\KW{многопродуктовая потоковая сетевая
модель; множество достижимых межузловых потоков; предельные
распределения пропускной способности}

\DOI{10.14357/19922264220306} 
  
%\vspace*{-3pt}


\vskip 10pt plus 9pt minus 6pt

\thispagestyle{headings}

\begin{multicols}{2}

\label{st\stat}

\section{Введение}

Данная работа продолжает исследования функциональных характеристик
сетевых сис\-тем связи~[1]. В~настоящее время математические модели
передачи многопродуктового потока применяются для поиска
распределений потоков и~ресурсов в~многопользовательских
телекоммуникационных\linebreak сетях~[2--4]. Разрабатываются методы анализа
с~учетом вектора требований всех \textit{равноправных} 
и~невзаимозаменяемых корреспондентов~[5]. С~позиций\linebreak методологии
исследования операций изучаются справедливые распределения потоков
и~ресурсов~[6].

Соответствующие \textit{недискриминирующие} правила управления
потоками являются решениями задач на максмин и/или получаются 
в~результате использования процедур последовательной
лексикографически упорядоченной оптимизации~[7].

В~настоящей работе пути соединения корреспондентов прокладываются
через со\-от\-вет\-ст\-ву\-ющие минимальные разрезы. Указанный метод\linebreak \mbox{можно}
рассматривать как возможный вариант решения задачи о~построении
SPLIT-марш\-ру\-тов~[8,~9]. В~рамках вычислительных экспериментов\linebreak на
многопродуктовой модели анализируются распределения межузловых
потоков  и~пропускной способ\-ности сети.  Для оценки функциональных
возможностей многопользовательской сети используется вектор
совместно допустимых межузловых потоков. Под ресурсом, выделяемым
некоторой паре узлов-кор\-рес\-пон\-ден\-тов,  понимается суммарное
значение тре\-бу\-емых пропускных способностей на всех ребрах,
расположенных на всех маршрутах при прохождении межузлового\linebreak потока
данного вида.  Сумма соответствующих реберных потоков трактуется
как полная нагрузка на сеть, возникающая при передаче заданного
межузлового потока. Рас\-смат\-ри\-ва\-ют\-ся распределения пропускной
способности и~межузловых потоков при предельной загрузке сети.
При проведении вычислительных экспериментов на каждой  итерации
вычисляется норма  вектора совместно допустимых межузловых
потоков.   Для оценки величины требуемых ресурсов при соединении
корреспондентов по различным путям для каж\-дой пары узлов
определяется максимальный однопродуктовый поток. Марш\-ру\-ты передачи
всех допустимых межузловых потоков  проходят по ребрам
соответствующих минимальных разрезов. Вычислительные эксперименты
проводились  для получения последовательности  мет\-ри\-че\-ских оценок
векторов межузловых потоков, принадлежащих множеству до\-сти\-жи\-мости
многопользовательской сети.

\section{Математическая модель}

В качестве математической модели для описания
многопользовательской сетевой системы используется следующая
формальная запись условий и~ограничений, которые должны
выполняться при одновременной передаче потоков различных видов
между всеми парами улов-корреспондентов:

Сеть $G(\mathbf{d})$ задается множествами $\langle V,
R,U,P\rangle$:
\begin{itemize}
\item  узлов (вершин) сети 
$$
V=\left \{v_{1}, v_{2},\dots,v_{n},\dots,v_{N}\right\};
$$
\item  неориентированных ребер 
$$
R=\left\{r_{1}, r_{2}, \dots, r_{k}, \dots,
r_{E}\right\}.
$$
\end{itemize}

Ребро $r_{k}$ \textit{соединяет} концевые вершины~$v_{n_k}$ и~$v_{j_k}$. 
Реб\-ру~$r_{k}$ ставятся в~соответствие две
ориентированные дуги $\{u_{k},u_{k+E}\}$ из множества
ориентированных дуг $U\hm=\{u_{1}, u_{2}, \dots, u_{k}, \dots,
u_{2E}\}$. Дуги $\{u_{k}, u_{k+E}\}$ определяют прямое и~обратное
на\-прав\-ле\-ние передачи потока по реб\-ру~$r_{k}$ между концевыми
вершинами $\{v_{n_k}, v_{j_k}\}$.

В многопользовательской сети~$G(\mathbf{d})$ рассматривается
$M\hm=N(N\hm-1)$ независимых, невзаимозаменяемых и~равноправных потоков
различных видов, которые передаются между уз\-ла\-ми-кор\-рес\-пон\-ден\-та\-ми
из множества 
$$
P=\left\{p_{1}, p_{2}, \dots, p_{M}\right\}.
$$

По определению, каждой паре уз\-лов-кор\-рес\-пон\-ден\-тов~$p_{m}$
соответствуют:
\begin{itemize}
\item вершина-ис\-точ\-ник с~номером~$s_{m}$, через которую входной поток
$m$-го вида~$z_{m}$ поступает в~сеть;
\item  вершина-при\-ем\-ник с~номером~$t_{m}$, из которой поток $m$-го
вида~$z_{m}$ покидает сеть.
\end{itemize}

В множестве~$P$ выделяется подмножество $P(R^{+})$ пар
уз\-лов-кор\-рес\-пон\-ден\-тов, расположенных в~концевых вершинах
ребра~$r_{k}$, $k\hm=\overline{1,E}$. Вводятся сле\-ду\-ющие обозначения:
пусть ребро~$r_{k}$  с~номером~$k$ соединяет вершины с~номерами~$n$ и~$j$ такими, что $n\hm< j$. Для со\-от\-вет\-ст\-ву\-ющей пары
уз\-лов-кор\-рес\-пон\-ден\-тов~$p_{k}$, расположенных в~узлах $\{v_{n},
v_{j}\}$, узел~$v_{n}$ считается источником, а узел~$v_{j}$~---
приемником потока $z_{k}$ $k$-го вида, который передается из узла
c номером~$n$ в~узел с~номером~$j$ для пары~$p_{k}$ с~номером~$k$.
Для пары $p^{}_{k+E} \Longleftrightarrow \{v_{j},v_{n}\}$ узел~$v_{j}$ 
считается источником~$s_{k+E}$, а~узел $v_m$~--- приемником~$t_{k+E}$ для пары с~номером~$p_{k+E}$. Формируется
$R^+\hm=\{1,2,3,\dots,E,E+1,\dots,2E\}$~--- список номеров смежных
пар.

Пары $p_{k}$ из подмножества~$P(R^{+})$ называются
\textit{смежными} уз\-ла\-ми-кор\-рес\-пон\-ден\-та\-ми. Все остальные
\textit{несмежные} пары уз\-лов-кор\-рес\-пон\-ден\-тов относятся к~множеству~$P(R^{-})$:
\begin{equation*}
P=P(R^{+})\cup P(R^{-});\quad
P(R^{+}) \cap P(R^{-}) = \emptyset.
\end{equation*}

Введем обозначения:
\begin{description}
\item[\,]
$z_{m}$~--- величина \textit{межузлового} потока $m$-го вида,
который поступает в~сеть из узла с~номером~$s_{m }$ и~покидает из
узла с~номером~$t_{m}$;
\item[\,]
$S(v_{n})$~--- множество номеров исходящих дуг, по которым поток
покидает узел~$v_{n}$;
\item[\,]
$T(v_{n})$~--- множество номеров входящих дуг, по которым поток
поступает в~узел~$v_{n}$.
\end{description}

Во всех узлах $v_{n}\in V$, $n\hm=\overline{1,N}$, для всех видов
потоков должны выполняться условия сохранения потоков:
\begin{multline}
\label{eq1} 
\sum\limits_{i\in S(v_n )} x_{mi}-\sum\limits_{i\in T(v_n )} x_{mi}
={}\\
{}=\begin{cases}
z_m, & \mbox{если } v=v^{}_{S_m}; \\
-z_m,&\mbox{если } v=v_{t_m}; \\
0&\mbox{в остальных случаях}, \\
\end{cases}
\end{multline}
$n=\overline{1,N}$, $m\hm=\overline{1,M}$, $x_{mi}\hm\ge 0$,
$z_{m}\hm\ge0$.

Величина {z}$_{m}$ равна входному потоку $m$-го вида, который
пропускается от источника к~приемнику пары $p_{m}$ при
распределении потоков $x_{mi}$ по дугам сети.

Каждому ребру $r_{k}\hm\in R$ приписывается неотрицательное число~$d_{k}$, 
определяющее суммарный предельно допустимый поток,
который можно передать по реб\-ру~$r_{k}$ в~обоих на\-прав\-ле\-ни\-ях. 
В~исходной сети компоненты вектора про\-пуск\-ных способностей
$\mathbf{d}\hm=(d_{1}, d_{2},\dots, d_{k}, \dots, d_{E})$~--- наперед
заданные положительные числа $d_{k}
\hm> 0$. Вектором $\mathbf{d}$ определяются сле\-ду\-ющие ограничения на сумму
дуговых потоков всех видов, пе\-ре\-да\-ва\-емых по реб\-ру~$r_{k}$:
\begin{multline}
\sum\limits_{m=1}^M (x_{mk}+x_{m(k+E)}) \le d_k,\\
 x_{mk}\ge 0\,,\enskip
 x_{m(k+E)}\ge 0\,, \enskip k=\overline {1,E}\,.
 \label{eq2} 
\end{multline}
В рамках данной модели пропускная спо\-соб\-ность ребер сети~--- вектор~$\mathbf{d}$~--- трактуется как <<\textit{ресурсное ограничение}>>,
а~сумма дуговых
 потоков рас\-смат\-ри\-ва\-ет\-ся как показатель использования
<<\textit{ресурсов}>> сети при передаче межузловых потоков
различных видов.

Для всех $z_{m}$ и~$x_{mi}$, удовлетворяющих
условиям~\eqref{eq1} и~\eqref{eq2}, вычисляются суммарные потоки:
\begin{equation}
 y_{m }=\sum\limits_{i=1}^{2E} {x}_{mi},\quad
m=\overline{1,M}\,.
\label{eq3}
\end{equation}

Суммарный реберный поток~$y_{m}$ характеризует
<<\textit{нагрузку}>> на сеть при передаче межузлового потока
величины $z_{m}$ из уз\-ла-ис\-точ\-ни\-ка~$s_{m}$ в~узел-при\-ем\-ник~$t_{m}$. 
Величина~$y_{m}$ показывает, какой суммарный
\textit{ресурс}~-- пропускная спо\-соб\-ность сети~-- требуется для
передачи межузлового потока~$z_{m}$, а~отношение
$w_{m}\hm={y_m}/{z_m}$,  $m\hm=\overline{1,M},$
показывает, какие \textit{ресурсы} необходимы для передачи
единичного потока $m$-го вида между узлами~$s_{m}$ и~$t_{m}$.

Ограничения~\eqref{eq1}--\eqref{eq3} задают подмножество
допустимых значений компонент вектора межузловых потоков
$\mathbf{z}\hm=\left(z_{1}, z_{2},\dots,z_{m},\dots,z_{M}\right)$:
\begin{equation*}
{Z}(\mathbf{d})=\left\{\mathbf{z} \ge 0 \mid
(\mathbf{z},\mathbf{x},\mathbf{y}) \ \mbox{удовлетворяют~\eqref{eq1}--\eqref{eq3}}
\right\}\!,
\!\!
%\label{eq4} 
\end{equation*}
а все допустимые распределения ресурсов принадлежат подмножеству
\begin{equation*}
{Y}(\mathbf{d})=\left\{\mathbf{y} \ge 0 \mid
(\mathbf{z},\mathbf{x},\mathbf{y}) \ \mbox{удовлетворяют~\eqref{eq1}--\eqref{eq3}}\right\}\!.
%\!\!\!\label{eq5}
\end{equation*}


\section{Метрические оценки предельных распределений}

Для оценки функциональных возможностей сис\-те\-мы рассматриваются
допустимые распределения межузловых потоков при предельных
загрузках ребер сети.

В рамках данного модельного описания монопольным режимом
называется способ управления, при котором все ресурсы сети
используются для передачи потока одной выделенной пары
уз\-лов-кор\-рес\-пон\-ден\-тов $p_{a}\hm\in P(R^-)$, а для всех
остальных потоки полагаются равными нулю.

Предельно допустимый поток, который можно передать между
фиксированной парой уз\-лов-кор\-рес\-пон\-ден\-тов $p_{a}$ в~монопольном
режиме, является решением стандартной, в~данном случае
однопродуктовой, задачи о~максимальном потоке.

\smallskip

\noindent
\textbf{Задача 1.} Найти
$z_a^0\hm=\max\limits_{\langle z,x\rangle \in Z(d)} z_a
$
при условии $z_{i}=0$, $i\hm=\overline{1,M}$, $i\hm\ne a$.

При решении задачи~1 для пары $p_{a}$ вы\-чис\-ля\-ют\-ся: межузловой
поток~$z_a^0$; дуговые потоки $\{x^{0}_{ak};x^{0}_{a(k+E)}\}$,
$k\hm=\overline{1,E}$; суммарное значение реберного
потока~$y_{a}^{0}\hm=\sum\nolimits_{i=1}^{2E} {x}_{ai}^{0}$.

Поток величины $z_a^0$ является \textit{максимальным потоком},
пе\-ре\-да\-ва\-емым в~\textit{монопольном режиме} для пары
уз\-лов-кор\-рес\-пон\-ден\-тов~$p_{a}$, $p_{a}\hm\in P(R^-)$, в~сети~$G(d)$.

Задача~1 решается последовательно для всех $p_{m}\in P(R^-)$,
вы\-чис\-ля\-ют\-ся значения $z_{m}^{0}(t)$.

При проведении вычислительных экспериментов использовалась
итерационная процедура для оценки функциональных возможностей
сис\-те\-мы при передаче межузловых потоков по нескольким маршрутам.
На предварительном этапе шага~$t$ в~сети~$G(t)$ при заданных
значениях пропускной спо\-соб\-ности ребер~$d_k(t)$ для каждой \mbox{пары}
уз\-лов-кор\-рес\-пон\-ден\-тов $p_m\hm\in P(R^-)$ определяется максимально
допустимый однопродуктовый поток~$z^0_m(t)$, со\-от\-вет\-ст\-ву\-ющие
дуговые потоки $(x_{mk}^0(t),x_{m(k+E)}^0(t))$, $p_m\hm\in P(R^-)$, и~коэффициенты нормировки
$\xi_m^0(t)\hm={1}/{z_m^0(t)}$ для всех  $p_m\hm \in P(R^-)$,
таких что $z^0_m(t)\hm>0$, $y_m^0(t)\hm>0$.
Коэффициенты~$\xi_m^0(t)$ используются для поиска текущих
совместно допустимых квот на передачу потоков одновременно между
всеми парами $p_m\in P(R^-)$.

\smallskip

\noindent
\textbf{Задача 2.} Найти $\alpha^*(t)=\max\limits_\alpha \alpha$
при условиях
$$
\alpha\!\!\sum\limits_{m\in R^-}\! \xi_m^0\left(x_{mk}^0(t)+x_{m(k+E)}^0(t)\right)\le d_k(t),\enskip
k=\overline{1,E}\,.
$$

На основании $\alpha^*(t)$ вычисляются совместно допустимые
дуговые потоки:
\begin{multline*}
x_{mk}^*(t)=\alpha^*(t)\xi^0_m(t)x^0_{mk}(t),\\
x^*_{m(k+E)}(t)=\alpha^*(t)\xi^0_m(t)x^0_{m(k+E)}(t),
\\
m=\overline{1,M}\,,\enskip k=\overline{1,E}\,,
\end{multline*}
и остаточная пропускная способность ребер в~сети $G(t+1)$:
\begin{multline*}
d_k(t+1)=d_k(t)-\sum_{m\in R^-} (x_{mk}^*(t)+x_{m(k+E)}(t)),\\
k=\overline{1,E}\,,\enskip p_m\in P(R^-).
\end{multline*}
Формируется вектор допустимых межузловых потоков:
\begin{align*}
z_k^+(t)&=d_k(t+1),\enskip p_k\in P(R^+),\enskip k=\overline{1,E}\,;
\\
z_m^-(t)&=\sum\limits_{\tau=1}^t\alpha^*(\tau)\xi_m^0(\tau) z_m^0(\tau), \enskip p_m\in P(R^-).
\end{align*}

По построению, на шаге~$t$ при передаче вектора межузлового потока
$\mathbf{z}(t)=\{\mathbf{z}^+(t), \mathbf{z}^-(t)\}$ достигается
предельная загрузка, и~пропускная способность всех ребер  сети
используется полностью.

Точка с~координатами $\mathbf{z}(t)$ принадлежит множеству~$Z(d)$.

Расстояние точки от начала координат определяется как норма
соответствующего вектора:
\begin{align*}
\rho^+(t)&=\|\mathbf{z}^+(t)\|=
\left[\,\sum\limits_{k=1}(\mathbf{z}^+(t))^2\right]^{1/2};
\\
\rho^-(t)&=\|\mathbf{z}^-(t)\|= \left[\sum\limits_{p_m\in P(R^-)}(\mathbf{z}_m^-(t))^2\right]^{1/2}.
\end{align*}

Если при выполнении шага $(t+1)$ окажется, что $z_m^0(t+1)=0$ для
всех $p_m\in P(R^-)$, то про\-изойдет останов и~сформируются
массивы финальных данных:
\begin{align*}
z_m^-(T)&=\sum\limits_{\tau=1}^t \alpha^*(\tau)\xi_m^0(\tau) z_m^0(\tau),\enskip 
p_m\in P(R^-),\\
z_k^+(T)&=d_k(t+1),\enskip p_k\in P(R^+),\enskip k=\overline{1,E}\,.
\end{align*}

Вышеописанная вычислительная процедура далее обозначается как
MFPL-про\-це\-ду\-ра (от англ.\ \textit{max-flow-peak-load}).

При проведении второй серии вычислительных экспериментов
MFPL-про\-це\-ду\-ра использовалась для оценки функциональных
характеристик сис\-те\-мы при \textit{уравнительном} поэтапном
распределении пропускной способности между всеми
па\-ра\-ми-кор\-рес\-пон\-ден\-тами.

При реализации MFPL-процедуры выполнение каждого шага разбивается
на несколько этапов. На предварительном этапе шага~$t$ 
в~сети~$G(t)$ при заданных значениях пропускной способности ребер~$d_k(t)$ 
для каждой пары уз\-лов-кор\-рес\-пон\-ден\-тов $p_m\hm\in P(R^-)$
определяется максимально допустимый однопродуктовый
поток~$z_m^0(t)$, соответствующие дуговые потоки
$\left(x_{mk}^0(t),x_{m(k+E)}^0(t)\right)$, $p_m\hm\in P(R^-)$, и~суммарная
реберная нагрузка
$$
y_m^0(t)=\sum\limits_{k=1}^E (x_{mk}^0(t),x_{m(k+E)}^0(t)),\enskip p_m\in P(R^-).
$$

Для каждой пары $p_m\hm\in P(R^-)$ вычисляются коэффициенты
нормировки
$\theta_m^0(t)\hm={1}/{y_m^0(t)}$ для всех  
$p_m\hm\in P(R^-)$, таких что  $z^0_m(t)\hm>0$,
$y_m^0(t)\hm>0$.
Коэффициенты~$\theta_m^0(t)$ используются для поиска совместно
допустимых дуговых потоков для всех $p_m\hm\in P(R^-)$.

\smallskip

\noindent
\textbf{Задача 3.} Найти $\beta^*(t)=\max\nolimits_\beta \beta$ при
условиях
$$
\beta\!\!\!\!\sum\limits_{p_m\in P(R^-)}\!\!
\theta_m^0(x_{mk}^0(t)+x_{m(k+E)}^0(t))\le d_k(t),\enskip
k=\overline{1,E}\,.
$$

 С помощью $\beta^*(t)$ (решения задачи~3) вычисляются текущие допустимые значения дуговых потоков:
\begin{multline*}
x_{mk}^*(t)=\beta^*(t)\theta^0_m(t)x^0_{mk}(t),\\
x^*_{m(k+E)}(t)=\beta^*(t)\theta^0_m(t)x^0_{m(k+E)}(t), \enskip
k=\overline{1,E},
\end{multline*}
и реберных нагрузок при одновременной передаче межузловых потоков:

\noindent
\begin{multline*}
y_m^*(t)=\sum\limits_{i=1}^E
\left[x_{mi}^*(t)+x^*_{m(i+E)}(t)\right]={}\\
{}= \fr{\beta^*(t)}{y_m^0(t)} \sum\limits_{i=1}^E
\left[x_{mi}^0(t)+x^0_{m(i+E)}(t)\right]=\beta^*(t), \\
 p_m\in P(R^-).
\end{multline*}
Таким образом на каждом шаге определенная часть имеющегося ресурса
(пропускной спо\-соб\-ности) делится строго по\-ров\-ну меж\-ду всеми
корреспондентами $p_m\in P(R^-)$, для которых существует путь
передачи в~$G(t)$.

Формируется вектор допустимых межузловых потоков:
\begin{gather*}
\hspace*{-30mm}z_k^{++}(t)=d_k(t+1)={}\hspace*{10mm}\\
{}=d_k(t)-\!\!\! \sum\limits_{p_m\in P(R^-)}\!\!\!
\left(x_{mk}^*(t)+x_{m(k+E)}(t)\right),\\
\hspace*{35mm}k=\overline{1,E}, \enskip
p_k\in P(R^+);\\
z_m^{(=)}(t)\overset{\Delta}{=}\sum\limits_{\tau=1}^t\beta^*(\tau)
\theta_m^0(\tau) z_m^0(\tau), \enskip p_m\in P(R^-).
\end{gather*}

\noindent
Определяются расстояния:
\begin{align*}
\rho^{++}(t)&=\|\mathbf{z}^{++}(t)\|\overset{\Delta}{=}
\left[\sum\limits_{k=1}^E\left(d_k(t+1)\right)^2\right]^{1/2};\\
\rho^{(=)}(t)&=\|\mathbf{z}^{=}(t)\|= \left[\sum\limits_{p_m\in
P(R^-)}\left(z_m^{(=)}(t)\right)^2\right]^{1/2}.
\end{align*}

Если на предварительном этапе на шаге $(t+1)$ окажется, что в~сети~$G(t+1)$ для всех $p_m\hm\in P(R^-)$ все значения
$z_m^0(t+1)\hm=0$, то произойдет останов и~сформируются финальные
массивы:
\begin{align*}
z_k^{(++)}(T)&=d_k(t+1), \enskip
p_k\in P(R^+), \enskip k=\overline{1,E};
\\
z_m^{(=)}(t)&=\sum\limits_{\tau=1}^{t+1}\beta^*(\tau)
\theta_m^0(\tau) z_m^0(\tau), \enskip p_m\in P(R^-).
\end{align*}



\section{Вычислительный эксперимент}

Результаты вычислительных экспериментов, описанные ниже, служат
продолжением исследований, начатых в~[1]. Вычислительные
эксперименты проводились на моделях сетевых сис\-тем, пред\-став\-лен\-ных
на рис.~1 и~2. В~каждой сети~69~узлов. Пропускные спо\-соб\-но\-сти
ребер~-- значения $d_k$~-- выбирались случайным образом из отрезка
$[900,999]$ и~совпадали для ребер, при\-сут\-ст\-ву\-ющих в~обеих сетях.
В~кольцевой сети пропускная спо\-соб\-ность каждого из добавленных
ребер равнялась~900.

\begin{figure*} %fig1
\vspace*{1pt}
\begin{minipage}[t]{80mm}
  \begin{center}  
    \mbox{%
\epsfxsize=69.408mm
\epsfbox{mal-1.eps}
}

\end{center}
\vspace*{-6pt}
\Caption{Базовая сеть}
\end{minipage}
%\end{figure*}
\hfill
%\begin{figure*} %fig2
\vspace*{1pt}
\begin{minipage}[t]{80mm}
  \begin{center}  
    \mbox{%
\epsfxsize=69.408mm
\epsfbox{mal-2.eps}
}

\end{center}
\vspace*{-6pt}
\Caption{Кольцевая сеть}
\end{minipage}
\end{figure*}

\begin{table*}[b]\small %tabl1
\vspace*{-12pt}
\begin{center}

%\renewcommand{\arraystretch}{1.1}
\Caption{Базовая сеть}
\vspace*{2ex}

\begin{tabular}{|c||c|c|c||c|c|c|} 
\hline
&&&&&&\\[-9pt]
$t$  & $\rho^{-}(t)$ & $\rho^{+}(t)$ & $d^{+}(t+1)$ &
$\rho^{=}(t)$ & $\rho^{++}(t)$&  $d^{++}(t+1)$ \\ 
\hline
\hphantom{99}0  & \hphantom{99}0   & 8048&  68256&  \hphantom{9}0   &  8048&   68256\\
1  & \hphantom{9}63  & 4182&  26544&  \hphantom{9}95  &  3881&   24476\\
$\cdots$  & $\cdots$   & $\cdots$   &  $\cdots$    &  $\cdots$   &  $\cdots$   &   $\cdots$\\
11 & \hphantom{9}70  & 3975&  21469&  \hphantom{9}101\hphantom{9} &  3707&   20155\\
$\cdots$& $\cdots$   & $\cdots$   &  $\cdots$    & $\cdots$   &  $\cdots$   &  $\cdots$\\
22 & \hphantom{9}83  & 3861&  19623&  \hphantom{9}122\hphantom{9} &  3586&   18260\\
$\cdots$ & $\cdots$  & $\cdots$   &  $\cdots$   &  $\cdots$   &  $\cdots$  &   $\cdots$\\
33 & \hphantom{9}103\hphantom{9} & 3778&  18827&  \hphantom{9}139\hphantom{9} &  3522&   17601\\
$\cdots$ &$\cdots$  &$\cdots$  & $\cdots$  & $\cdots$   &  $\cdots$  &  $\cdots$\\
44 & \hphantom{9}\bf 190\hphantom{9} & \bf3553&  \bf17503&  \hphantom{9}\bf203\hphantom{9} &  \bf3285&   \bf16201\\
45 & \hphantom{9}\bf1452\hphantom{99}& \bf2166&  \hphantom{9}\bf7069 &  \hphantom{9}\bf1376\hphantom{99}&  \bf2020&   \hphantom{9}\bf6584\\
46 & \hphantom{9}\bf1498\hphantom{99}& \bf2158&  \hphantom{9}\bf6707 &  \hphantom{9}\bf1388\hphantom{99}&  \bf2017&   \hphantom{9}\bf6483\\
$\cdots$ & $\cdots$   & $\cdots$   &  $\cdots$    & $\cdots$   &  $\cdots$   &  $\cdots$\\
52 & \hphantom{9}1535\hphantom{99}& 2155&  \hphantom{9}6413 & \hphantom{9}1442\hphantom{99} &  2011&   \hphantom{9}6059\\
\hline
\end{tabular}
\end{center}
 %\end{table*}
% \begin{table*}\small %tabl2
\begin{center}
\Caption{Кольцевая сеть}
\vspace*{2ex}


\begin{tabular}{|c||c|c|c||c|c|c|} 
\hline
&&&&&&\\[-9pt]
$t$  & $\rho^{-}(t)$ & $\rho^{+}(t)$ & $d^{+}(t+1)$ &
$\rho^{=}(t)$ & $\rho^{++}(t)$&  $d^{++}(t+1)$ \\
 \hline
\hphantom{9}0  &\hphantom{99}0    & 8440  & 75456   &\hphantom{9}0      &8440   &75456\\
\hphantom{9}1  &\hphantom{9}68   & 5317  & 43038   &92     &5045   &40716 \\ 
$\cdots$ &$\cdots$    & $\cdots$     & $\cdots$   &$\cdots$      &$\cdots$      &$\cdots$      \\
11 &\hphantom{9}95   & 3608  & 20459   &124    &3397   &19080  \\
$\cdots$ &$\cdots$   & $\cdots$    & $\cdots$      &$\cdots$     &$\cdots$     &$\cdots$   \\
22 &\hphantom{9}101\hphantom{9}  & 3540  & 19530   &130    &3350   &18338 \\
$\cdots$ &$\cdots$  & $\cdots$   &$\cdots$      &$\cdots$     &$\cdots$   &$\cdots$    \\
33 &\hphantom{9}135\hphantom{9}  & 3346  & 17561   &154    &3220   &17003 \\
$\cdots$  &$\cdots$   & $\cdots$    & $\cdots$      &$\cdots$     &$\cdots$    &$\cdots$    \\
44 &\hphantom{9}234\hphantom{9}  & 3094  & 14881   &269    &2918   &13848 \\
$\cdots$ &$\cdots$   & $\cdots$    &$\cdots$      &$\cdots$     &$\cdots$     &$\cdots$    \\
50 &\hphantom{9}\bf 413\hphantom{9}  & \bf2770  & \bf12901   &\bf329    &\bf2792   &\bf13079 \\
51 &\hphantom{9}\bf1040\hphantom{99} & \bf2299  & \hphantom{9}\bf8801    &\bf334    &\bf2784   &\bf13034 \\
52 &\hphantom{9}\bf1062\hphantom{99} & \bf2297  & \hphantom{9}\bf8672    &\bf974    &\bf2262   &\hphantom{9}\bf8768  \\
$\cdots$ &$\cdots$   &$\cdots$    & $\cdots$      &$\cdots$      &$\cdots$     &$\cdots$    \\
55 &\hphantom{9}1069\hphantom{99} & 2297  & \hphantom{9}8630    &1010\hphantom{9}   &2259   &\hphantom{9}8553  \\
\hline
 \end{tabular}
\end{center}
 \end{table*}




Для базовой сети исходная сумма пропускных способностей:
$D^+(0)\hm=68\,256$, а~для кольцевой сети $D^{++}(0)=75\,456$.
Соответствующие значения $\rho^+(0)$ и~$\rho^{++}(0)$ указаны в~<<нулевой>> строке 
в~табл.~1 и~2, где собраны результаты
вычислительных экспериментов. В~ходе эксперимента при
уравнительном распределении остаточных ресурсов соблюдается
\textit{равномерное} убывание остаточной пропускной спо\-соб\-ности и~<<\textit{длины}>> вектора~$\rho^+(t)$. 
Однако между 44--46
итерациями для базовой и~50--52 для кольцевой сети наблюдается
резкий скачок величин~$\rho^-(t)$, $\rho^{=}(t)$ и~$d^+(t)$,
$d^{++}(t)$.

На указанных шагах полностью используется пропускная способность
ребер в~центральной час\-ти сети. Сеть \textit{распадается} на
несвязные компоненты, и~для $80\%$ корреспондентов пропадают пути
соединения, а~остаточный ресурс распределяется поровну между
оставшимися парами узлов.

Анализ результатов показал, что почти равные значения потоков
достигаются для~80\% корреспондентов и~требуют 60\%--70\%
ресурсов. Однако для~2\% смежных  пар узлов межузловые потоки на
два порядка выше медианных значений, а~затраты пропускной
способности  со\-став\-ля\-ют~20\%--30\%.








\section{Заключение}

Предложенный метод и~проведенные вычислительные эксперименты
показали, что уравнительное поэтапное распределение   приводит 
к~неравномерному  распределению   потоков  для разных групп\linebreak
корреспондентов.    Метрические оценки, полученные  в~ходе
экспериментов, продемонстрировали\linebreak \textit{деформацию} множества
достижимых потоков. В~рамках модели   предполагалось, что  все
корреспонденты  равноправны, а~потоки невзаимозаменяемы,  однако
при уравнительном предельном  распределении  смежные  пары узлов
оказывались в~привилегированном положении при использовании
остаточной пропускной способности. Пропускные способности  ребер
рассматривались  как вектор   ресурсов  различных типов,  которые
распределяются между корреспондентами   при передаче  потоков
различных видов.  По построению, на каж\-дом шаге норма вектора
смежных   межузловых    потоков численно равна   модулю вектора
остаточных  пропускных способностей.   Полученные мет\-ри\-че\-ские
значения  можно использовать  для   оценки функциональных
возможностей сети  в~режиме  предельной загрузки.

{\small\frenchspacing
 {%\baselineskip=10.8pt
 %\addcontentsline{toc}{section}{References}
 \begin{thebibliography}{9}

\bibitem{1-mal}
\Au{Малашенко Ю.\,Е., Назарова И.\,А.} Неоднородность
распределения   потоков при предельной  загрузке
многопользовательской сети~//  Известия РАН. Теория и~сис\-те\-мы
управления,  2022. №\,3. С.~81--96.

\bibitem{4-mal} %2
\Au{Luss H.} Equitable resource allocation: Models,
algorithms, and applications.~--- Hoboken, NJ, USA: John Wiley \& Sons, 2012.
420~p.

\bibitem{2-mal} %3
\Au{Ogryczak W., Luss~H., Pioro~M., Nace~D., Tomaszewski~A.}   Fair
optimization and networks: A~aurvey~// J.~Appl. Math., 2014. Vol.~2014. Art.~ID~612018. 25~p. doi: 10.1155/ 2014/612018.

\bibitem{3-mal} %4
\Au{Salimifard K., Bigharaz~S.} The multicommodity network
flow problem: State of the art classification, applications, and
solution methods~// J.~Oper. Res., 2020. Vol.~18. Iss.~3. P.~1--47.



\bibitem{5-mal}
\Au{Balakrishnan A., Li~G., Mirchandani~P.}  Optimal
network design with end-to-end service requirements~// Oper. Res.,
2017. Vol.~65. Iss.~3. P.~729--750.

\bibitem{6-mal}
\Au{Nace D., Doan~L.\,N., Klopfenstein~O., Bashllari~A.} Max-min
fairness in multicommodity flows~// Comput. Oper. Res., 2008.
Vol.~35. Iss.~2. P.~557--573.

\bibitem{7-mal}
\Au{Ros-Giralt J., Tsai~W.\,K.} A~lexicographic optimization
framework to the flow control problem~// IEEE T.
Inform. Theory, 2010. Vol.~56. Iss.~6. P.~2875--2886.

\bibitem{8-mal}
\Au{Baier G., Kohler~E., Skutella~M.}  The \mbox{k-splittable}
flow problem~//  Algorithmica, 2005. Vol.~42. Iss.~3-4.
P.~231--248.

\bibitem{9-mal}
\Au{Bialon P.\,A.} Randomized rounding approach to 
a~\mbox{k-splittable} multicommodity flow problem with lower path flow
bounds affording solution quality guarantees~// Telecommun. Syst.,
2017. Vol.~64. Iss.~3. P.~525--542.
\end{thebibliography}

 }
 }

\end{multicols}

\vspace*{-6pt}

\hfill{\small\textit{Поступила в~редакцию 10.06.22}}

\vspace*{8pt}

%\pagebreak

%\newpage

%\vspace*{-28pt}

\hrule

\vspace*{2pt}

\hrule

%\vspace*{-2pt}

\def\tit{SEQUENTIAL ANALYSIS AND METRIC ESTIMATES\\ OF~PEAK LOAD FLOWS IN~THE~MULTIUSER NETWORK}


\def\titkol{Sequential analysis and metric estimates of~peak load flows in~the~multiuser network}


\def\aut{Yu.\,E.~Malashenko}

\def\autkol{Yu.\,E.~Malashenko}

\titel{\tit}{\aut}{\autkol}{\titkol}

\vspace*{-8pt}


\noindent
Federal Research Center ``Computer Science and Control'' of the Russian Academy of Sciences, 
44-2~Vavilov Str., Moscow 119333, Russian Federation



\def\leftfootline{\small{\textbf{\thepage}
\hfill INFORMATIKA I EE PRIMENENIYA~--- INFORMATICS AND
APPLICATIONS\ \ \ 2022\ \ \ volume~16\ \ \ issue\ 3}
}%
 \def\rightfootline{\small{INFORMATIKA I EE PRIMENENIYA~---
INFORMATICS AND APPLICATIONS\ \ \ 2022\ \ \ volume~16\ \ \ issue\ 3
\hfill \textbf{\thepage}}}

\vspace*{3pt} 



\Abste{The set of vectors of internodal flows in a~multiuser communication network under peak load is analyzed. Within the framework of
 the multicommodity model, the throughput capacities of edges are considered as the components of a~vector of resources of various types that 
 are required for the transmission of various kinds of
 flows. When conducting computational experiments, at each iteration, the
  norms of vectors of jointly permissible internodal flows are calculated, during the transmission of which the capacity of 
  all network edges is fully used.\linebreak\vspace*{-12pt}}
 
 \Abstend{The proposed method and computational experiments have shown that the equalizing phased 
  distribution leads to an uneven distribution of flows for different groups of correspondents. Metric values obtained during experiments 
  indicate deformation of the sets of accessible flows. Within the framework of the model, all correspondents are tantamount 
  and the flows are noninterchangeable; however, in the case of an equalizing peak load distribution, adjacent pairs 
  of nodes are in a privileged position when using residual capacity. The obtained metric values can be used to 
  evaluate the functional characteristics of the transmission network in the finite capacity loading mode.}

\KWE{multicommodity flow network model; set of achievable internodal flows; peak load distribution}


\DOI{10.14357/19922264220306} 

%\vspace*{-16pt}

%\Ack
%\noindent



%\vspace*{4pt}

  \begin{multicols}{2}

\renewcommand{\bibname}{\protect\rmfamily References}
%\renewcommand{\bibname}{\large\protect\rm References}

{\small\frenchspacing
 {%\baselineskip=10.8pt
 \addcontentsline{toc}{section}{References}
 \begin{thebibliography}{9}
\bibitem{1-mal-1}
\Aue{Malashenko, Yu.\,E., and I.\,A.~Nazarova.}
2022. Heterogeneous flow distribution at the peak load in the multiuser network. \textit{J.~Comput. Sys. Sc. Int.} 61:372--387.

\bibitem{4-mal-1} %2
\Aue{Luss, H.} 2012. \textit{Equitable resource allocation: Models, algorithms, and applications}.
Hoboken, NJ: John Wiley \& Sons. 420~p.

\bibitem{2-mal-1} %3
\Aue{Ogryczak, W., H.~Luss, M.~Pioro, D.~Nace, and A.~Tomaszewski.}
 2014. Fair optimization and networks: A~survey. \textit{J.~Appl. Math.} 2014:612018. 25~p. doi: 10.1155/ 2014/612018.
\bibitem{3-mal-1} %4
\Aue{Salimifard, K., and S.~Bigharaz.}
 2020. The multicommodity network flow problem: State of the art classification, applications, and solution methods. 
 \textit{J.~Oper. Res.} 18(3):\linebreak 1--47.

\bibitem{5-mal-1}
\Aue{Balakrishnan, A., G.~Li, and P.~Mirchandani.} 2017. Optimal network design with end-to-end service requirements. 
\textit{Oper. Res.} 65(3):729--750.
\bibitem{6-mal-1}
\Aue{Nace, D., L.\,N.~Doan, O.~Klopfenstein, and A.~Bashllari.} 2008. Max-min fairness in multicommodity flows. 
\textit{Comput. Oper. Res.} 35(2):557--573.
\bibitem{7-mal-1}
\Aue{Ros-Giralt, J., and W.\,K.~Tsai.} 2010. A~lexicographic optimization framework to the flow control problem. 
\textit{IEEE T.~Inform. Theory} 56(6):2875--2886.
\bibitem{8-mal-1}
\Aue{Baier, G., E.~Kohler, and M.~Skutella.}
 2005. The k-splittable flow problem. \textit{Algorithmica} 42(3-4):231--248.
\bibitem{9-mal-1}
\Aue{Bialon, P.} 2017. A~randomized rounding approach to a~\mbox{k-splittable} multicommodity flow problem with lower path flow bounds affording solution quality guarantees. 
\textit{Telecommun. Syst.} 64(3):525--542.
 \end{thebibliography}

 }
 }

\end{multicols}

\vspace*{-6pt}

\hfill{\small\textit{Received June 10, 2022}}

\Contrl

\noindent
\textbf{Malashenko Yuri E.} (b.\ 1946)~--- 
Doctor of Science in physics and mathematics, principal scientist, Federal Research Center ``Computer Science and Control'' 
of the Russian Academy of Sciences, 44-2~Vavilov Str., Moscow 119333, Russian Federation; \mbox{malash09@ccas.ru} 


\label{end\stat}

\renewcommand{\bibname}{\protect\rm Литература}   %3
\include{inkova-kruzhkov} %4

\def\stat{shnurkov}

\def\tit{АНАЛИТИЧЕСКОЕ РЕШЕНИЕ ЗАДАЧИ ОПТИМАЛЬНОГО УПРАВЛЕНИЯ ПОЛУМАРКОВСКИМ ПРОЦЕССОМ\\ 
С~КОНЕЧНЫМ МНОЖЕСТВОМ СОСТОЯНИЙ$^*$}

\def\titkol{Аналитическое решение задачи оптимального управления полумарковским 
процессом} %с~конечным множеством состояний}

\def\aut{П.\,В.~Шнурков$^1$, А.\,К.~Горшенин$^2$, В.\,В.~Белоусов$^3$}

\def\autkol{П.\,В.~Шнурков, А.\,К.~Горшенин, В.\,В.~Белоусов}

\titel{\tit}{\aut}{\autkol}{\titkol}

\index{Шнурков П.\,В.}
\index{Горшенин А.\,К.}
\index{Белоусов В.\,В.}
\index{Shnurkov P.\,V.}
\index{Gorshenin A.\,K.}
\index{Belousov V.\,V.}


{\renewcommand{\thefootnote}{\fnsymbol{footnote}} \footnotetext[1]
{Работа выполнена при частичной поддержке РФФИ (проект 15-07-05316).}}


\renewcommand{\thefootnote}{\arabic{footnote}}
\footnotetext[1]{Национальный исследовательский университет <<Высшая школа экономики>>, 
\mbox{pshnurkov@hse.ru}}
\footnotetext[2]{Институт проблем информатики Федерального исследовательского центра <<Информатика 
и~управ\-ле\-ние>> Российской академии наук, \mbox{agorshenin@frccsc.ru}}
\footnotetext[3]{Институт проблем информатики Федерального исследовательского центра <<Информатика 
и~управление>> Российской академии наук, \mbox{vbelousov@ipiran.ru}}

%\vspace*{-6pt}

\Abst{Настоящее исследование посвящено теоретическому обоснованию нового метода 
нахождения оптимальной стратегии управления полумарковским процессом с~конечным 
множеством состояний. Рассматриваются марковские рандомизированные стратегии 
управления, определяемые конечным набором вероятностных мер, соответствующих 
каждому состоянию. Характеристикой качества управления служит стационарный 
стоимостной показатель. Данный показатель представляет собой дроб\-но-ли\-ней\-ный 
интегральный функционал от набора вероятностных мер, задающих стратегию управления. 
Для этого функционала известны явные аналитические представления подынтегральных 
функций числителя и~знаменателя. Дальнейшие результаты основываются на новой 
усиленной и~обобщенной форме теоремы об экстремуме дроб\-но-ли\-ней\-но\-го интегрального 
функционала. Доказывается, что проблемы существования оптимальной стратегии управления 
полумарковским процессом и~ее нахождения сводятся к~задаче численного исследования 
на глобальный экстремум заданной функции от конечного числа вещественных переменных.}

\KW{оптимальное управление полумарковским процессом; стационарный стоимостной 
показатель качества управления; дроб\-но-ли\-ней\-ный интегральный функционал}

\DOI{10.14357/19922264160408} 

\vspace*{9pt}


\vskip 10pt plus 9pt minus 6pt

\thispagestyle{headings}

\begin{multicols}{2}

\label{st\stat}

\section{Введение}

Теория оптимального управления марковскими и~полумарковскими случайными 
процессами интенсивно развивается с~начала 1960-х~гг. Еще в~первых 
основополагающих исследованиях рассматривались не только проблемы существования 
оптимальных стратегий управления, но и~способы нахождения этих стратегий. 

Для решения таких проблем, имеющих алгоритмическое содержание, использовались 
открытые незадолго до этого мощные методы прикладной математики: линейное 
программирование и~динамическое программирование. Отметим, прежде всего, 
классическую работу Р.~Ховарда~\cite{1}, в~которой метод динамического 
программирования был применен для решения проблемы оптимального управления 
марковским процессом с~непрерывным временем. В~дальнейшем В.\,В.~Рыков~\cite{2} 
доказал, что для аналогичной модели управления марковским процессом с~учетом 
переоценки оптимальной стратегией также является стационарная.

Важную роль в~развитии теории управления случайными процессами сыграла работа 
В.~Джевелла~\cite{3}, в~которой были впервые рассмотрены полумарковские модели 
управления для вариантов с~переоценкой и~без переоценки. Данная работа была 
переведена на русский язык и~послужила основой для многих последующих работ 
отечественных и~зарубежных специалистов. В~частности, Б.~Фокс показал~\cite{4}, 
что оптимальной стратегией управления полумарковским процессом в~варианте без 
переоценки является стационарная; аналогичные результаты были получены Э.~Денардо 
и~для варианта с~переоценкой~\cite{5}.

Среди последующих исследований алгоритмической направленности отметим работы 
Р.~Ховарда~\cite{6}, Б.~Фокса~\cite{4}, а также С.~Осаки и~Х.~Майна~\cite{7}. 
В~этих работах для нахождения оптимальных стратегий управления полумарковскими 
процессами использовался метод линейного программирования.

В 1970~г.\ была опубликована фундаментальная монография Х.~Майна и~С.~Осаки~\cite{8}, 
переведенная на русский язык в~1977~г., в~которой были систе\-ма\-ти\-зи\-ро\-ва\-ны и~изложены 
основные результаты по теории оптимального управления марковскими и~полумарковскими 
случайными процессами. Фактически данная книга стала итогом исследований по проблемам 
стохастического управления\linebreak
 за~10~лет. Отметим, что в~этой монографии рас\-смат\-ри\-ва\-лись 
марковские и~полумарковские модели управления с~конечными множествами состояний 
и~допустимых решений, принимаемых \mbox{в~каждом} состоянии. Были получены принципиальные 
тео\-ре\-ти\-че\-ские результаты, заключающиеся в~том, что оптимальные стратегии управ\-ле\-ния 
для основных видов рас\-смат\-ри\-ва\-емых моделей с~переоценкой и~без переоценки являются 
детерминированными и~стационарными. Были разработаны и~обоснованы процедуры нахождения 
оптимальных стратегий управления. В~частности, для модели управления полумарковским 
процессом без переоценки, когда множество со\-сто\-яний образует один эргодический класс, 
а~показатель качества управления пред\-став\-ля\-ет собой стационарный средний удельный 
доход (см.~[8, гл.~5, п.~5.5]), процедура поиска оптимальной рандомизированной 
стратегии осуществлялась методом линейного программирования. Обратим особое внимание 
на данный результат, поскольку аналогичная модель управления полумарковским 
процессом будет рассмотрена в~настоящей работе.

Принципиальную роль в~развитии теории стохастического управления сыграла 
монография И.\,И.~Гихмана и~А.\,В.~Скорохода~\cite{9}. В~этой книге были впервые 
систематически изложены основы теории оптимального управления случайными процессами 
с~дискретным и~непрерывным временем, включая теорию управления процессами, которые 
описываются стохастическими дифференциальными уравнениями. Отдельно были рас\-смот\-ре\-ны 
проблемы управления марковскими процессами с~дискретным временем и~скачкообразными 
марковскими процессами с~непрерывным временем. Роли множеств состояний и~допустимых 
управ\-ле\-ний играли пространства весьма общей структуры. Для широких классов функционалов 
качества управ\-ле\-ния (так называемых эволюционных функционалов в~марковских моделях 
с~дискретным временем и~интегральных функционалов накопления в~марковских моделях 
с~непрерывным временем) были доказаны теоремы о~существовании и~формах пред\-став\-ле\-ния 
оптимальных стратегий управ\-ле\-ния. Было установлено, что для однородных марковских 
моделей оптимальные стратегии управ\-ле\-ния существуют, являются стационарными 
и~детерминированными. Иначе говоря, такие стратегии задаются детерминированными 
функциями, аргументом которых является со\-сто\-яние сис\-те\-мы в~момент принятия решения, 
и~не зависящими от самого момента принятия решения. Что же касается важного вопроса 
о~формах представления этих функций, то их можно охарактеризовать следующим образом. 
Были найдены функциональные уравнения, осложненные условием экстремума, которым 
удовле\-тво\-ря\-ют упомянутые функции. По существу эти соотношения пред\-став\-ля\-ют собой 
уравнения Беллмана для соответствующих динамических стохастических моделей.

Особо отметим, что в~монографии~\cite{9} не рас\-смат\-ри\-ва\-лись проблемы управления 
полумарковскими процессами. Однако дальнейшее развитие общей теории управления 
такими процессами шло по пути, идейно намеченному в~указанной книге.

В последующие годы развитие теории управ\-ле\-ния полумарковскими процессами 
осуществля-\linebreak лось по направлению усложнения моделей и~обобщения исходных предположений. 
Например,\linebreak в~работах~\cite{10, 11} рассмотрены управляемые по\-лумарковские процессы при 
весьма общих предположениях относительно характера пространств состояний и~управлений. 
Проблемы управления исследовались по отношению к~различным видам целевых показателей, 
обобщающих упомянутый выше стационарный показатель средней удельной прибыли. В~этих 
работах доказывается, что оптимальная стратегия управления по отношению к~каж\-до\-му из 
показателей существует и~является одной и~той же стационарной детерминированной 
стратегией, определяемой некоторой функцией, заданной на множестве со\-сто\-яний процесса. 
Об этой функции известно лишь то, что она удовлетворяет некоторому интегральному 
уравнению, которое по содержанию пред\-став\-ля\-ет собой уравнение Бел\-лма\-на для 
соответствующей задачи управ\-ления.

Среди исследований, предшествовавших настоящему, отметим работу 
В.\,А.~Каштанова~[12, гл. 13]. В этом разделе коллективной монографии~\cite{12} 
автором была рассмотрена проблема оптимального управления полумарковским 
процессом с~конечным множеством состояний и~множеством возможных решений, 
которое представляет собой произвольный интервал множества вещественных чисел. 
Модель относится к~виду моделей без переоценки, показателем качества управления 
служит стационарное значение среднего удельного дохода, определяемое аналогично 
классическим работам~\cite{3, 8}. Рандомизированное управление в~каждом состоянии 
определяется в~соответствии с~вероятностным распределением, совокупность которых 
задает\linebreak
 стратегию управления. В.\,А.~Каш\-та\-но\-вым было\linebreak сформулировано утверждение о том, 
что стацио\-нарное значение среднего удельного дохода представляет собой 
дроб\-но-ли\-ней\-ный интегральный функционал от набора вероятностных распределений, 
образующих стратегию управления. При этом\linebreak ранее~[12, гл.~10; 13] было уста\-нов\-ле\-но, 
что дроб\-но-ли\-ней\-ный функционал достигает экстремума на вырожденных распределениях. 
Отсюда естест-\linebreak венно следует, что оптимальная стратегия управ\-ле-ния является 
детерминированной и~должна\linebreak определяться точкой экстремума функции, представляющей 
собой отношение подынтегральных функций чис\-ли\-те\-ля и~знаменателя данного 
дроб\-но-ли\-ней\-но\-го функционала. Однако в~\cite{12} не были получены явные 
представления для указан-\linebreak ных функций. Кроме того, приведенный в~гл.~10 
монографии~\cite{12} вариант теоремы об экстремуме дроб\-но-ли\-ней\-но\-го 
интегрального функционала требовал проверки выполнения условия существования 
этого экстремума. Такие условия указаны не были. В~связи с~этими обстоятельствами 
использовать полученные в~\cite{12} результаты для доказательства существования 
оптимальной детерминированной стратегии управ\-ле\-ния полумарковским процессом и~для 
строгого обоснования способа нахождения такой стратегии оказалось невозможным.

Настоящее исследование посвящено теоретическому обоснованию нового метода 
нахождения\linebreak оптимальной стратегии управления полумарковским процессом с~конечным 
множеством со\-сто\-яний. Рассматриваются марковские рандомизи\-рованные стратегии 
управления, определяемые конеч\-ным набором вероятностных мер, соответствующих 
каждому состоянию. Показателем качества управления служит уже упоминавшийся 
классический  показатель~--- стационарное значение средней удельной прибыли. 
Доказано, что этот показатель представляет собой дроб\-но-ли\-ней\-ный интегральный 
функционал от набора вероятностных мер, задающих стратегию управления. При этом, 
в~отличие от~\cite{12}, получены явные аналитические представления для подынтегральных 
функций числителя и~знаменателя этого дроб\-но-ли\-ней\-но\-го\linebreak
 функционала. Дальнейшие 
результаты основываются на новой усиленной и~обобщенной форме\linebreak
 теоремы об экстремуме 
дроб\-но-ли\-ней\-но\-го интегрального функционала, впервые опубликованной 
в~работе П.\,В.~Шнуркова~\cite{14}. Согласно\linebreak
 утверж\-де\-нию этой теоремы, если 
существует глобальный экстремум так называемой основной функции дроб\-но-ли\-ней\-но\-го 
функционала, которая пред\-став\-ля\-ет собой отношение подынтегральных функций чис\-ли\-те\-ля 
и~знаменателя, то существует безусловный экстремум самого дроб\-но-ли\-ней\-но\-го 
функционала, который достигается на наборе вырожденных вероятностных распределений, 
сосредоточенных в~точке глобального экстремума. В~этом случае оптимальная стратегия 
управ\-ле\-ния существует, является стационарной и~детерминированной и~определяется точкой, 
в~которой основная\linebreak функция достигает глобального экстремума. Таким\linebreak образом, проблемы 
существования оптимальной стратегии управ\-ле\-ния полумарковским процессом и~ее 
нахождения сводятся к~задаче чис\-лен\-но\-го исследования на глобальный экстремум 
заданной функции от конечного чис\-ла вещественных переменных.

\section{Общее описание модели управления полумарковским случайным процессом}

Построим модель управления полумарковским случайным процессом, следуя общему 
подходу, принятому в~классических работах~\cite{3, 8}. Пусть $\xi(t)$~--- 
случайный полумарковский процесс с~конечным множеством состояний
$X\hm=\{1,2,\ldots, N\}$, $N\hm< \infty$. Обозначим через~$t_n$, $n=0,1,2,\ldots$, 
$t_0\hm=0$, случайные моменты изменения состояний данного процесса, 
$\theta_n\hm=t_{n+1}-t_n$, $n\hm=0,1,2,\ldots$, $\xi_n\hm=\xi(t_n)\hm=\xi(t_n+0)$, 
$n\hm=0,1,2,\ldots$ (предполагается, что траектории процесса~$\xi(t)$ 
непрерывны справа). Случайная последовательность~$\{\xi_n\}$
образует цепь Маркова, вложенную в~полумарковский процесс~$\xi(t)$.
Зададим набор измеримых пространств\linebreak $(U_1, \mathscr{B}_1), 
(U_2, \mathscr{B}_2), \ldots, (U_N, \mathscr{B}_N)$, где $U_i$~--- 
множество возможных допустимых управ\-ле\-ний, $\mathscr{B}_i$~--- $\sigma$-ал\-геб\-ра 
подмножеств множества~$U_i$, вклю\-ча\-ющая в~себя все одноточечные подмножества\linebreak  
множества~$U_i$, т.\,е.\ если $u_i \hm\in U_i$, то $\{u_i\} \hm\in \mathscr{B}_i$, 
$i\hm=1,2,\ldots, N$.
Пусть $\Gamma_i$~--- некоторое множество всевозможных вероятностных мер $\Psi_i 
\hm \in \Gamma_i$, заданных на $\sigma$-ал\-геб\-ре~$\mathscr{B}_i$, $i\hm=1,2,\ldots,N$.

Поскольку идейное содержание и~свойства вероятностных мер существенно используются 
в~данной работе, укажем на некоторые фундаментальные издания, в~которых 
изложена соответствующая тео\-рия. Понятие и~основные свойства вероятностной 
меры определены и~подробно проанализированы в~книге А.\,Н.~Ширяева~\cite[гл.~II]{15}. 
Глубокое изложение основ теории вероятностных мер имеется также в~книге 
А.\,А.~Боровкова~\cite{16}. Заметим попутно, что в~книге~\cite{16} имеются разделы, 
посвященные изложению основ теории полумарковских и~регенерирующих случайных процессов. 
Из зарубежных изданий отметим фундаментальную работу П.~Хеннекена и~А.~Тортра~\cite{17}, 
основная часть которой посвящена изложению математических основ теории вероятностей.

Введем специальное понятие вырожденной вероятностной меры, которое будет часто 
использоваться в~дальнейшем. Пусть $(U_0, \mathscr{B}_0)$~--- некоторое измеримое 
пространство, $\mathscr{B}_0$~--- $\sigma$-ал\-геб\-ра подмножеств множества~$U_0$, 
включающая в~себя все одноточечные подмножества этого множества.

\medskip

\noindent
\textbf{Определение 1.}\ Вероятностная мера~$\Psi^*$, заданная 
на $\sigma$-ал\-геб\-рe~$\mathscr{B}_0$, называется вырожденной, если существует 
такой элемент $u^* \hm\in U_0$, для которого выполняются условия $\Psi^*(\{u^*\})\hm=
1$, $\Psi^*(U_0 \setminus \{u^*\})\hm=0$, где $\{u^*\}=u^*$~--- 
множество, состоящее из единственной точки $u^* \hm\in U_0$. Соответствующая 
точка $u^* \hm\in U_0$ будет называться точкой сосредоточения вырожденной 
вероятностной меры~$\Psi^*$.
Таким образом, всякая вырожденная вероятностная мера~$\Psi^*$ определяется 
своей точкой сосредоточения~$u^*$. В~дальнейшем будем использовать 
обозначение~$\Psi_{u^*}^{*}$, имея в~виду, что вырожденная вероятностная мера~$\Psi^*$ 
сосредоточена в~точке~$u^*$.
Отметим также, что вырожденная вероятностная мера~$\Psi_{u^*}^{*}$ соответствует 
детерминированной величине, которая принимает фиксированное значение $u\hm=u^*$ 
с~вероятностью, равной единице.

\medskip

Обозначим через $\Gamma_0$ множество всех  вероятностных мер, заданных 
на измеримом пространстве ($U_0, \mathscr{B}_0$), а через~$\Gamma_0^*$~--- 
множество всех вырожденных вероятностных мер, заданных на этом пространстве, 
$\Gamma_0^*\hm\in \Gamma_0$. Аналогичные обозначения будут использоваться 
и~в~дальнейшем. Заметим, что множество~$\Gamma_0^*$ находится во взаимно
 однозначном соответствии с~множеством точек сосредоточения вырожденных 
 вероятностных мер, т.\,е.\ с~множеством~$U_0$.

Пусть $\Gamma_i^{*}$~--- множество всех вырожденных мер, заданных на 
$\sigma$-ал\-геб\-ре~$\mathscr{B}_i$, $\Gamma_i^{*}\hm\subset \Gamma_i$.
Произвольная вероятностная мера~$\Psi_i$ описывает случайную величину, 
принимающую значения в~$U_i$, а вырожденная мера~$\Psi_i^*$, сосредоточенная 
в~точке~$u_i^*$, соответствует детерминированной величине $u_i^*\hm\in U_i$.
Предполагается, что соответствующие конструкции определены на всех измеримых 
пространствах управлений $(U_1, \mathscr{B}_1), (U_2, \mathscr{B}_2), \ldots, 
(U_N,\mathscr{B}_N)$.

Предположим, что управления случайным полумарковским процессом~$\xi(t)$ 
осуществляются в~моменты времени~$t_n,$ $n\hm=0,1,2,\ldots,$
непосредственно после изменения состояния процесса. Если\linebreak 
$\xi_n\hm=\xi(t_n)\hm=i \hm\in X$, то значение управления представляет 
собой случайную величину~$u_n$, принимающую значения в~множестве допустимых 
управ\-ле\-ний~$U_i$ и~описываемую вероятностной мерой (распределе\-ни\-ем 
вероятностей) $\Psi_i \hm\in \Gamma_i$.
Будем предполагать, что при фиксированном условии $\xi_n\hm=\xi(t_n)=i$ 
управ\-ле\-ние определяется независимо от прошлого поведения процесса~$\xi(t)$ 
и~вероятностная мера~$\Psi_i$,
описывающая стохастическое управление~$u_n$, зависит только от состояния $i\hm\in X$.
Тогда выбор управ\-ле\-ний в~моменты изменения состояний $\{t_n, n\hm=0,1,2,\ldots \}$ 
описывается набором вероятностных мер (распределений вероятностей) 
$(\Psi_1, \Psi_2,\ldots, \Psi_N)$, 
$\Psi_i \hm\in \Gamma_i$, $i\hm=1,2,\ldots,N$.
Назовем любой такой набор стратегией управ\-ле\-ния полумарковским процессом~$\xi(t)$. 
По своим свойствам такая стратегия является марковской, однородной 
и~рандомизированной.

Следуя классической монографии П.~Халмоша~\cite[гл.~VII]{18}, 
рассмотрим декартово произведение пространств $U\hm=U_1\times U_2\times \cdots\times U_N$ 
и~соответствующих $\sigma$-ал\-гебр $\mathscr{B}\hm=\mathscr{B}_1\times \mathscr{B}_2
\times \cdots \times\mathscr{B}_N$. Обозначим через $\Psi\hm=\Psi_1\times \Psi_2\times \cdots
\times \Psi_N$ вероятностную меру на~$(U,\mathscr{B})$, определяемую как 
произведение мер $\Psi_1,\Psi_2, \ldots , \Psi_N$, где $\Psi_i \hm\in \Gamma_i$, 
$i\hm=1,2,\ldots,N$. Обозначим также через~$\Gamma$ множество вероятностных мер~$\Psi$, 
заданных на~$(U,\mathscr{B})$, которые пред\-став\-ля\-ют собой произведение 
мер $\Psi_1,\Psi_2, \ldots , \Psi_N$, где $\Psi_i \hm\in \Gamma_i$, $i\hm=1,2,\ldots,N$.
Множество~$\Gamma$ можно отож\-де\-ст\-вить с~множеством всех стратегий управ\-ле\-ния 
полумарковским процессом~$\xi(t)$.

Проблема оптимального управления полумар\-ковским процессом~$\xi(t)$ будет в~дальнейшем 
сформулирована в~виде задачи безусловного экстремума некоторого функционала 
$I(\Psi)\hm=I(\Psi_1,\Psi_2, \ldots , \Psi_N)$, заданного на множестве 
допустимых стратегий управления. Содержание показателя качества управления~$I(\Psi)$, 
аналитическое представление для него, а~также описание множества допустимых 
стратегий управления будут приведены в~последующих разделах данной работы.

Для получения дальнейших результатов потребуются различные вероятностные 
характеристики управляемого полумарковского процесса~$\xi(t)$. Как известно из
 общей теории полумарковских процессов~\cite{19, 20}, 
 основной вероятностной характеристикой такого процесса является так называемая 
 полумарковская функция. Определим эту функцию для процесса с~управлением 
 (см.~\cite[гл.~5]{8}):
\begin{multline}
Q_{ij}(t,u)=
{\sf P}\left(\xi_{n+1}=j,\theta_n<t \mid \xi_n=i, u_n=u\right)\,,\\
t\in [0,\infty)\,,\ u\in U_i\,;\ i,j\in X=\{1,2,\ldots,N\}\,. \label{e1}
\end{multline}
Используя полумарковские функции, можно получить вероятности перехода 
управляемой цепи Маркова~$\{\xi_n\}$:
\begin{multline}
p_{ij}(u)={\sf P}\left(\xi_{n+1}=j \mid \xi_n=i, u_n=u\right)= {}\\
{}=
\lim\limits_{t\rightarrow\infty}Q_{ij}(t,u)\,,\enskip
u\in U_i\,;\enskip i,j\in X\,, 
\label{e2}
\end{multline}
а также функции распределения длительностей пребывания полумарковского 
процесса~$\xi(t)$ в~соответствующих состояниях:

\noindent
\begin{multline}
H_{i}(t,u)={\sf P}\left(\theta_n<t \mid \xi_n=i, u_n=u\right)={}\\
{}=
\sum\limits_{j\in X}Q_{ij}(t,u)\,,\enskip
t\in [0,\infty)\,,\  u\in U_i\,; \  i\in X\,. 
\label{e3}
\end{multline}

Обозначим через
\begin{multline}
T_{i}(u)=\mathbf{E}\left[\theta_n \mid \xi_n=i, u_n=u\right]={}\\
{}=
\int\limits_0^{\infty}\left[1-H_i(t,u)\right]\,dt\,,\enskip
u\in U_i\,,\ i\in X\,, 
\label{e4}
\end{multline}
математические ожидания длительностей пребывания полумарковского процесса~$\xi(t)$ 
в~каждом из состояний.

Введенные выше характеристики~(1)--(4) определены для случая, когда 
в~момент изменения состояния~$t_n$ процесс оказывается в~состоянии~$i$ 
и~принимается решение $u\hm\in U_i$. При заданной стратегии управления 
$\Psi\hm=\left(\Psi_1,\Psi_2, \ldots , \Psi_N\right)$ можно записать 
соответствующие вероятностные характеристики без условия на управление, а~именно:
\begin{multline*}
Q_{ij}(t)={\sf P}\left(\xi_{n+1}=j,\theta_n<t \mid \xi_n=i\right)={}\\
{}=
\int\limits_{U_i}Q_{ij}(t,u) \,d\Psi_i(u)\,,\enskip 
t\in [0,\infty)\,,\ i,j\in X\,; %\label{e5}
\end{multline*}

\vspace*{-12pt}

\noindent
\begin{multline}
p_{ij}={\sf P}\left(\xi_{n+1}=j \mid \xi_n=i\right)=
\int\limits_{U_i}p_{ij}(u)\, d\Psi_i(u)\,,\\  
i,j\in X\,; 
\label{e6}
\end{multline}

\vspace*{-9pt}

\noindent
\begin{equation}
T_{i}=\mathbf{E}\left[\theta_n \mid \xi_n=i\right]=
\int\limits_{U_i}T_{i}(u)\,d\Psi_i(u)\,,\enskip i\in X\,. 
\label{e7}
\end{equation}
В дальнейшем будем предполагать, что для рас\-смат\-ри\-ва\-емой 
полумарковской модели заданы вероятностные характеристики 
$p_{ij}(u)$, $u\hm\in U_i$, $i,j\hm\in X$, и~$T_i(u)$, $u\hm\in U_i$, $i\hm\in X$, 
определяемые соотношениями~(\ref{e2}) и~(\ref{e4}). 
Для фиксированной стратегии управления $\Psi\hm=(\Psi_1, \Psi_2,\ldots, \Psi_N)$ 
соответствующие вероятностные характеристики~$p_{ij}$ и~ $T_i$, $i,j\hm\in X,$ 
определены равенствами~(\ref{e6}) и~(\ref{e7}) без условий на управление.

\section{Стационарный стоимостной показатель качества управления}

Определим некоторый стоимостной аддитивный функционал, связанный 
с~рассматриваемым полумарковским процессом~$\xi(t)$. По содержанию этот функционал 
представляет собой случайный\linebreak доход или прибыль, накопленную за период времени $[0,t]$. 
Определения такого функционала приведены в~основополагающих работах~[3; 8, гл.~5].\linebreak 
Обозначим через $\widetilde{v}(t)$, $t\hm\geq 0$, значение этого аддитивного 
функционала в~момент времени~$t$; $\widetilde{v}_n\hm=\widetilde{v}(t_n\hm+0)$~--- 
соответствующее значение непосредственно после очередного момента изменения\linebreak 
состояния~$t_n$, $n\hm=0,1,2,\ldots$; $\widetilde{v}_0\hm=v_0$~--- 
заданное начальное значение в~момент $t\hm=0$. Рассмотрим величину
\begin{multline}
d_i(u)=\mathbf{E}\left[\widetilde{v}_{n+1}-\widetilde{v}_n \mid \xi_n=i\,, 
u_n=u\right]\,,\\
u\in U_i\,, \enskip i\in X\,, \label{e8}
\end{multline}
представляющую собой математическое ожидание приращения стоимостного 
аддитивного функционала за период времени между последовательными 
изменениями состояния полумарковского процесса~$\xi(t)$. Тогда соответствующее 
математическое ожидание, вычисляемое без условия на решение, 
принимаемое в~момент времени~$t_n$, представляется в~виде:
\begin{equation*}
d_i=\mathbf{E}\left[\widetilde{v}_{n+1}-\widetilde{v}_n \mid \xi_n=i\right]=
\!\int\limits_{U_i}\!d_i(u)\,d\Psi_i(u)\,,\ i\in X \,. %\label{e9}
\end{equation*}

Предположим, что для заданной стратегии управ\-ле\-ния 
$\Psi\hm=(\Psi_1,\Psi_2,\ldots,\Psi_N)$ вложенная цепь Маркова~$\{\xi_n\}$ 
имеет ровно один класс возвратных положительных состояний (по терминологии, 
принятой в~\cite{8}, такое множество состояний называется эргодическим классом). 
Как известно~\cite[гл.~VIII]{15}, данное условие является необходимым 
и~достаточным для существования единственного\linebreak стационарного распределения. 
Обозначим это стационарное распределение цепи Маркова~$\{\xi_n\}$ через 
$\pi\hm=(\pi_1, \pi_2,\ldots, \pi_N)$. Заметим, что данное\linebreak распределение зависит  
от стратегии управления $\Psi\hm=(\Psi_1,\Psi_2,\ldots,\Psi_N)$. При указанном 
условии имеет место следующий результат, называемый эргодической теоремой 
для аддитивного стоимостного функционала:
\begin{equation}
I=\lim\limits_{t\rightarrow\infty}\fr{\mathbf{E}\widetilde{v}(t)}{t}=
\fr{\sum\nolimits_{i=1}^N d_i\pi_i}{\sum\nolimits_{i=1}^N T_i\pi_i}\,. 
\label{e10}
\end{equation}

Соотношение~(\ref{e10}) доказано в~работе~\cite[гл.~5]{8}. Заметим, что аналогичные 
результаты имеют мес\-то для гораздо более общих полумарковских моделей~\cite{10, 11}.

По своему прикладному содержанию величина, определяемая соотношением~(\ref{e10}), 
представляет собой
среднюю удельную прибыль, связанную с~эволюцией системы в~стационарном
режиме. Кроме того, величина~$I$ представляет собой функционал от
набора вероятностных распределений~$\Psi_{i}$, $i\hm\in\lbrace 1,\ldots
,N\rbrace $, определяющих стратегию управле-\linebreak\vspace*{-12pt}

\pagebreak

\noindent
ния системой. 
В~дальнейшем будем рассматривать стационарный стоимостной функционал 
$I\hm=I(\Psi_{1},\Psi_{2},\ldots , \Psi_{N})$ как
показатель качества управ\-ле\-ния системой и~построенным полумарковским
процессом~$\xi (t)$.

\section{Представление стационарного показателя в~форме
дробно-линейного интегрального функционала}

В данном разделе будет приведено утверждение об аналитическом
представлении стационарного стоимостного функционала~(\ref{e10}), 
служащего критерием качества управления в~рассматриваемой задаче управления 
полумарковским процессом.

\smallskip

\noindent
\textbf{Теорема 1.} \textit{Стационарный стоимостной показатель, 
определяемый равенством}~(\ref{e10}), \textit{представляет собой дроб\-но-ли\-ней\-ный
функционал от вероятностных распределений~$\Psi_{i}(u_{i})$,
$i\hm\in\{1,\dots,N\}$. Данный функционал задается
аналитически следующей формулой:}
\begin{multline}
I=I(\Psi_{1},\ldots, \Psi_{N})={}\\
\hspace*{-2mm}{}=\!
\fr{\int\nolimits_{U_1}\!{\cdots\! 
\int\nolimits_{U_N}\!{A(u_{1},\ldots ,u_{N})d\Psi_{1}(u_{1})\cdots
\,d\Psi_{N}(u_{N})}}}{\int\nolimits_{U_1}{\!\cdots\! \int\nolimits_{U_N}\!{B(u_{1},\ldots ,u_{N})
\,d\Psi_{1}(u_{1})\ldots
d\Psi_{N}(u_{N})}}},\!\!\! \label{e11}
\end{multline}
\textit{где подынтегральные функции числителя и~знаменателя выражаются
соотношениями}:
\begin{align}
A(u_{1},\ldots
,u_{N})&={}\notag\\
&\hspace*{-20mm}{}=\sum\limits_{i=1}^{N}{d_{i}(u_{i})}{\widehat{D}}^{(i)}(u_{1}, \ldots
,u_{i-1},u_{i+1}, \ldots , u_{N})\,;  \label{e12}\\
 B(u_{1},\ldots
,u_{N})&={}\notag\\
&\hspace*{-20mm}{}=\sum\limits_{i=1}^{N}{T_{i}(u_{i})}{\widehat{D}}^{(i)}(u_{1}, \ldots
,u_{i-1},u_{i+1}, \ldots , u_{N})\,.  \label{e13}
\end{align}
\textit{В свою очередь, функции} ${\widehat{D}}^{(i)}(u_{1}, \ldots
,u_{i-1},u_{i+1}, \ldots$\linebreak $\ldots , u_{N})$, $i\hm\in\{1,\dots,N\}$, 
\textit{входящие в~правые части формул}~(\ref{e12}) и~(\ref{e13}), 
\textit{определяются следующим образом:}

\noindent
\begin{multline}
{\widehat{D}}^{(i)}(u_{1}, \ldots ,u_{i-1},u_{i+1}, \ldots , u_{N})={}
\\
{}=(-1)^{N+i+2}\sum\limits_{\alpha ^{(N),i}}{(-1)}^{\delta (\alpha
^{(N),i}) }{\widehat{D}}_{0}^{(i)}\left(\alpha ^{(N),i},u_{1}, \ldots\right.\\
\left.\ldots , u_{i-1},u_{i+1}, \ldots , u_{N}\right)\,. \label{e14}
\end{multline}
\textit{Здесь} $\alpha ^{(N),i}=(\alpha _{1}, \ldots , \alpha _{i-1},\alpha_{i+1}, \ldots , 
\alpha _{N})$~\textit{--- произвольная
перестановка чисел }$(1, \ldots , i-1, i+1, \ldots , N)$;
$\delta
(\alpha ^{(N),i})$~\textit{--- число инверсий в~перестановке} 
$\alpha ^{(N),i}$;

\noindent
\begin{multline}
{\widehat{D}}_{0}^{(i)}(\alpha ^{(N),i},u_{1}, \ldots ,u_{i-1},u_{i+1},
\ldots , u_{N})={}\\
{} ={\widetilde{p}}_{1,\alpha _{1}}\left(u_{1}\right)\cdots {\widetilde{p}}_{i-1,\alpha
_{i-1}}\left(u_{i-1}\right){\widetilde{p}}_{i+1,\alpha _{i+1}}\left(u_{i+1}\right)\cdots\\
\cdots
{\widetilde{p}}_{N,\alpha _{N}}\left(u_{N}\right)\,, 
\label{e15}
\end{multline}
где
\begin{multline}
 {\widetilde{p}}_{k,\alpha _{k}}(u_{k})=
\begin{cases}
p_{kk}(u_{k})-1,\  & \alpha _{k}=k\,; \\
p_{k,\alpha _{k}}(u_{k}),\  & \alpha _{k}\ne k, \\
\end{cases}\\
 k=1, \ldots , i-1, i+1, \ldots ,N\,. \label{e16}
 \end{multline}
\textit{Функции $p_{ij}(u_i)$, $T_{i}(u_{i})$ и~$d_{i}(u_{i})$,
$u_i\hm\in U_i$, $i,j\hm\in \{1,2,\ldots,N\}$, 
входящие в~соотношения}~(\ref{e12})--(\ref{e16}), 
\textit{определяются равенствами}~(\ref{e2}), (\ref{e4}) \textit{и}~(\ref{e8}) \textit{соответственно.}

\smallskip

\noindent
Д\,о\,к\,а\,з\,а\,т\,е\,л\,ь\,с\,т\,в\,о\ теоремы~1 
в~весьма сжатой форме приведено в~работе~\cite{21}. Читателю, интересующемуся 
более подробным обоснованием данного результата, порекомендуем обратиться к~тексту 
кандидатской диссертации А.\,В.~Иванова~\cite[гл.~3]{22}.

\smallskip

Итак, теорема~1 позволяет получить явное аналитическое представление 
для стационарного стоимостного показателя вида~(\ref{e10}) в~форме 
дроб\-но-ли\-ней\-но\-го интегрального функционала от набора\linebreak вероятностных мер 
$\Psi\hm=(\Psi_{1},\Psi_{2},\ldots , \Psi_{N})$, за\-да\-ющих стратегию управления 
полумарковским процессом~$\xi(t)$. При этом подынтегральные функции числителя 
и~знаменателя задаются формулами~(\ref{e12}), (\ref{e13}) 
и~вспомогательными равенствами~(\ref{e14})--(\ref{e16}). Таким образом, функция
\begin{equation}
C\left(u_1, u_2,\ldots, u_N\right)=\fr{A(u_1, u_2,\ldots, u_N)}{B(u_1, u_2,\ldots, u_N)}\,,
\label{e17}
\end{equation}
которая в~дальнейшем будет называться основной функцией дроб\-но-ли\-ней\-но\-го 
интегрального функционала~(\ref{e11}) и~которая будет играть важную роль 
в~дальнейшем исследовании, также явно определяется формулами~(\ref{e17}), 
(\ref{e12}), (\ref{e13}).

\section{Формальная постановка оптимизационной задачи 
и~условия существования оптимальной стратегии управления полумарковским процессом}

Будем рассматривать проблему управления полумарковским процессом~$\xi(t)$ в~форме 
экстремальной задачи
\begin{multline}
I(\Psi)=I\left(\Psi_1, \Psi_2,\ldots,\Psi_N\right)\rightarrow \mathrm{extr}\,,
\\
\Psi=\left(\Psi_1, \Psi_2,\ldots,\Psi_N\right)\in\Gamma\,. \label{e18}
\end{multline}
При этом показатель качества управления~$I(\Psi)$ представляет собой 
дроб\-но-ли\-ней\-ный интегральный функционал вида~(\ref{e11}).

Для решения экстремальной задачи~(\ref{e18}) воспользуемся некоторым утверждением 
об экстремуме дроб\-но-ли\-ней\-но\-го интегрального функционала. Прежде 
чем сформулировать данное утверждение, отметим, что в~теории оптимизации 
хорошо известны задачи, в~которых целевая функция представляет собой 
отношение двух линейных отображений, а имеющиеся ограничения также линейны. 
Такой раздел называется дроб\-но-ли\-ней\-ным программированием. Основные
 теоретические результаты данного направления изложены в~работе~\cite{23},
  там же приведена подробная библиография. В~дальнейшем потребуется некоторый 
  специальный результат о безусловном экстремуме дроб\-но-ли\-ней\-но\-го 
  интегрального функционала вида~(\ref{e11}), который был впервые сформулирован 
  в~работе~\cite{14}. Заметим, что для использования этого результата необходимо, 
  чтобы выполнялись некоторые предварительные условия, которые в~данном случае 
  можно сформулировать следующим образом:
\begin{enumerate}[1.]
\item Интегральные выражения
\begin{align*}
I_1(\Psi)&=I_1\left(\Psi_1,\Psi_2,\ldots,\Psi_N\right)={}&\\
&\hspace*{-13mm}{}=\int\limits_{U_1}\!\cdots\!
\int\limits_{U_N}\!\!A\left(u_1,\ldots ,u_N\right)\,
d\Psi_1\left(u_1\right) %d\Psi_2\left(u_2\right)
\cdots
 d\Psi_N\left(u_N\right)\,;
\\
I_2(\Psi)&=I_2\left(\Psi_1,\Psi_2,\ldots,\Psi_N\right)={}&\\
&\hspace*{-13mm}{}=\int\limits_{U_1}\!\cdots\!\int\limits_{U_N}\!\!
B\left(u_1,\ldots,u_N\right)\,
d\Psi_1\left(u_1\right)% d\Psi_2\left(u_2\right)\cdots\\
\cdots d\Psi_N\left(u_N\right)
\end{align*}
определены для всех стратегий управления $\Psi\hm=(\Psi_1, \ldots,\Psi_N)
\hm\in \Gamma$.

\item Функционал $I_2(\Psi)=I_2(\Psi_1, \ldots,\Psi_N)\hm\neq 0$ 
для всех $\Psi\hm=(\Psi_1, \ldots,\Psi_N)\hm\in \Gamma$.

\item Множество $\Gamma$ включает в~себя множество всех вырожденных 
вероятностных мер: $\Gamma^* \hm\subset \Gamma$.
\end{enumerate}

Сделаем несколько важных замечаний по поводу введенных предварительных условий.

\smallskip

\noindent
\textbf{Замечание~1.}\ Из условия~2 следует, что функция $B(u_1, u_2,\ldots, u_N)$ 
не может принимать значения разных знаков. С~учетом условия~3 
получаем, что указанная функция должна обладать \mbox{свойством} строгой 
знакопостоянности на всем множестве~$U$. С~другой стороны, если выполняется 
условие строгой знакопостоянности функции $B(u_1, u_2,\ldots, u_N), 
(u_1, u_2,\ldots, u_N)\hm\in U$, то условие~2 выполняется автоматически.

\smallskip

\noindent
\textbf{Замечание~2.}\ Если рассматривать в~качестве целевого функционала 
$I(\Psi_1, \Psi_2,\ldots,\Psi_N)$ экстремальной задачи~(\ref{e18}) 
стационарный стоимостной пока\-затель~(\ref{e10}), то функция $B(u_1,u_2,\ldots,u_N)$ 
имеет\linebreak следующее теоретическое содержание. Данная функция представляет собой условное 
математическое ожидание длительности периода времени между соседними моментами 
изменения со\-сто\-яния полумарковского процесса~$\xi(t)$ при условии, что стратегия 
его управ\-ле\-ния является детерминированной и~задается набором значений аргументов 
$(u_1,u_2,\ldots,u_N)$. Тогда условие строгой положительности функции 
$B(u_1,u_2,\ldots,u_N)$ при всех $(u_1,u_2,\ldots,u_N)\hm\in U$ является естественным 
и~фактически означает, что при любой заданной детерминированной стратегии 
управ\-ле\-ния процесс~$\xi(t)$ не имеет мгновенных со\-сто\-яний, длительность пребывания 
в~которых равна нулю.

\smallskip

\noindent
\textbf{Замечание~3.}\ Сделаем некоторые замечания, связан\-ные с~подынтегральной 
функцией числителя дроб\-но-ли\-ней\-но\-го интегрального функционала~(\ref{e11}). 
Как и~ранее, будем рассматривать в~качестве целевого функционала $I(\Psi_1, \Psi_2,\ldots,\Psi_N)$\linebreak 
экстремальной задачи~(\ref{e18}) стационарный стоимостной показатель~(\ref{e10}). 
Тогда для любого фиксированного набора значений аргументов $(u_1,u_2,\ldots,u_N)\hm\in U$ 
значение функции $A(u_1,u_2,\ldots\linebreak \ldots,u_N)$ представляет собой условное математическое
 ожидание приращения рассматриваемого стоимостного функционала, 
 происшедшее за время пребывания полумарковского процесса~$\xi(t)$ в~некотором 
 фиксированном  состоянии при условии, что стратегия управления является 
 детерминированной и~задается указанным набором $(u_1,u_2,\ldots,u_N)\hm\in U$. 
 Отметим, что в~теореме об экстремуме дроб\-но-ли\-ней\-но\-го интегрального 
 функционала, доказанной в~работе~\cite[гл.~10]{12}, 
 на подынтегральную функцию числителя накладываются условия ограниченности на 
 всем множестве значений аргумента. Для многих математических моделей и~связанных 
 с~ними задач оптимального управления такое условие является излишне ограничительным. 
 В~качестве примера можно привести модели оптимального управления запасом непрерывного 
 продукта, рассмотренные в~работах~\cite{27, 28}. 
 В~настоящем исследовании на функцию $A(u_1,u_2,\ldots,u_N)$ накладывается только 
 условие интегрируемости по любому заданному набору вероятностных мер 
 $\Psi\hm=(\Psi_1, \Psi_2,\ldots,\Psi_N)$, образующему стратегию управления 
 полумарковским процессом~$\xi(t)$ (условие~1 системы предварительных условий).

\smallskip

\noindent
\textbf{Замечание~4.} Условия~1--3 являются необходимыми для корректной 
постановки задачи безусловного экстремума дроб\-но-ли\-ней\-но\-го интегрального 
функционала. Если этот функционал служит показателем качества в~задаче оптимального 
управления случайным процессом, то необходимо добавить к~этим условиям дополнительное, 
связанное с~некоторой регулярностью самого управляемого процесса, а~именно: некоторый 
содержательный показатель, связанный с~поведением этого процесса, должен существовать 
и~быть представимым в~виде дроб\-но-ли\-ней\-но\-го интегрального функционала. 
Если потребовать, чтобы выполнялось эргодическое соотношение~(\ref{e10}), 
то можно использовать\linebreak теорему~1 и~сформулировать задачу оптимального управ\-ле\-ния 
в~виде~(\ref{e18}) для дроб\-но-ли\-ней\-но\-го\linebreak интегрального функционала~(\ref{e11}). 
Таким образом, необходимо ввести условие, обеспечивающее существование единственного 
стационарного распределения вложенной цепи Маркова и~выполнение\linebreak соотношения~(\ref{e10}). 
По аналогии с~[8, гл.~5] сформулируем это дополнительное условие в~следующем виде:
\begin{enumerate}
\setcounter{enumi}{3}
\item Для любой рассматриваемой стратегии управ\-ле\-ния $\Psi\hm=
(\Psi_1, \Psi_2,\ldots,\Psi_N)\hm\in \Gamma$ вложенная цепь Маркова 
полумарковского процесса $\xi(t)$ имеет ровно один класс возвратных 
положительных состояний.
\end{enumerate}

Теперь определим понятие допустимой стратегии управления полумарковским процессом 
с~конечным множеством состояний.

\smallskip

\noindent
\textbf{Определение~2.}\ Назовем стратегию управления 
$\Psi\hm=(\Psi_1, \Psi_2,\ldots,\Psi_N)$ 
допустимой в~данной задаче, если она удовлетворяет условиям~1--4.


\smallskip

\noindent
\textbf{Замечание~5.}\ Как следует из замечания~1, если потребовать, 
чтобы функция $B(u_1, u_2,\ldots,u_N)$ являлась строго знакопостоянной при 
всех $(u_1, u_2,\ldots,u_N)\hm\in U$, то можно считать допустимыми стратегии 
$(\Psi_1, \Psi_2,\ldots,\Psi_N)$, удовлетворяющие условиям~1, 3, 4. С~учетом замечания~2 
о~естественном характере условия строгой знакопостоянности функции $B(u_1,u_2,\ldots,u_N)$ 
при всех значениях аргументов $(u_1, u_2,\ldots,u_N)\hm\in U$ будем требовать 
выполнения этого условия в~формулировке приводимой в~дальнейшем основной 
теоремы об оптимальной стратегии управления полумарковским процессом.

\smallskip

\noindent
\textbf{Замечание~6.}\ Ниже будет сформулирована и~доказана основная 
теорема об оптимальной стра\-тегии управления полумарковским процессом с~конеч\-ным 
множеством состояний. Будем формулировать эту теорему по отношению к~экстремальной 
задаче~(\ref{e18}), в~которой целевой функционал $I(\Psi_1, \Psi_2,\ldots,\Psi_N)$ 
имеет вид дроб\-но-ли\-ней\-но\-го интегрального функционала. 
Это обстоятельство связано с~тем, что целевой функционал в~задаче 
оптимального управления необязательно должен иметь характер стационарного 
стоимостного показателя вида~(\ref{e10}). В~частности, еще в~1983~г.\ П.\,В.~Шнурковым 
было установлено~\cite{24}, что ряд показателей, связанных 
с~временем пребывания управляемого полумарковского процесса в~заданном конечном 
подмножестве состояний, имеет структуру дроб\-но-ли\-ней\-но\-го интегрального 
функционала от набора вероятностных мер, определяющих стратегию управления. 
Таким образом, рассматриваемая задача управления имеет более общий характер, 
чем задача, в~которой целевой функционал представляет собой стационарный 
стоимостной показатель вида~(\ref{e10}).






\smallskip

\noindent
\textbf{Замечание~7.}\ Если рассматривать задачу оптимального управления 
полумарковским процессом, в~кото\-рой целевой функционал не совпадает 
со стационарным стоимостным показателем~(\ref{e10}), то возможно, что могут 
потребоваться другие дополнительные условия, обеспечивающие существование этого 
показателя и~его представление в~форме~(\ref{e11}). В~связи с~этим в~формулировке 
основной теоремы будем использовать термин допустимые стратегии в~широком смысле, 
имея в~виду выполнение всех необходимых условий для каждого рассмат\-ри\-ва\-емо\-го 
показателя качества управления.

\smallskip


\noindent
\textbf{Замечание 8.} Множество допустимых стратегий может 
не совпадать с~множеством всех возможных стратегий управления. 
В~частности, допустимые стратегии могут состоять только из дискретных вероятностных 
мер $\Psi_1, \Psi_2,\ldots,\Psi_N$, т.\,е.\ таких, которые сосредоточены на дискретных 
множествах точек пространств $U_1, U_2,\ldots,U_N$.

\section{Теоретическое решение задачи оптимального управления}

Перейдем к~формулировке и~доказательству тео\-ре\-мы об 
оптимальной стратегии управ\-ле\-ния полумарковским процессом с~конечным 
множеством состояний.

\smallskip

\noindent
\textbf{Теорема~2.} \textit{Рассмотрим проблему оптимального управ\-ле\-ния 
полумарковским процессом~$\xi(t)$ в~виде экстремальной задачи}~(\ref{e18}), 
\textit{определенной на множестве допустимых стратегий $\Gamma$, 
для дроб\-но-ли\-ней\-но\-го 
функционала}~(\ref{e11}). \textit{Пусть функция $B(u_1,u_2,\ldots,u_N)$, 
входящая в~определение функционала}~(\ref{e11}),
\textit{является строго знакопостоянной (строго положительной или строго отрицательной) 
при всех значениях аргументов $(u_1,u_2,\ldots,u_N)\hm\in U$.
Тогда справедливы сле\-ду\-ющие утверждения}:
\begin{enumerate}[1.]
\item \textit{Если функция} $C(u_1,u_2,\ldots,u_N)\hm=A(u_1,u_2,\ldots$\linebreak
$\ldots,u_N)/{B(u_1,u_2,\ldots,u_N)}$ 
\textit{ограничена сверху или снизу и~достигает глобального экст\-ре\-му\-ма на множестве
$U\hm=U_1\times U_2\times \cdots \times U_N$ (максимума или минимума), 
то оптимальная стратегия управления полумарковским процессом~$\xi(t)$ существует, 
является детерминированной и~определяется
вырожденной вероятностной мерой $\Psi^*\hm\in \Gamma^*$, сосредоточенной в~точке, 
в~которой достига\-ет соответствующего экстремума функция $C(u_1,u_2,\ldots,u_N)$,
и~при этом выполняются соотношения}:
\begin{multline}  %{\substack{{i=\overline{1,n}}\\ {j=\overline{1,l}}}}
\max\limits_{\Psi \in \Gamma} I(\Psi)=
\max\limits_{\substack{{\Psi_i \in \Gamma_i\,,}\\ 
{i=\overline{1,N}}}}
I\left(\Psi_1,\Psi_2,\ldots,\Psi_N\right)={}\\
{}=
\max\limits_{\substack{{\Psi_i^* \in \Gamma_i^*,}\\ 
{i=\overline{1,N}}}}
 I\left(\Psi_1^*,\Psi_2^*,\ldots,\Psi_N^*\right)={}\\
{}=\max\limits_{(u_1,u_2,\ldots,u_N)\in U}\fr{A(u_1,u_2,\ldots,u_N)}
{B(u_1,u_2,\ldots,u_N)}\,; \label{e19}
\end{multline}

\vspace*{-12pt}

\noindent
\begin{multline*}
\min\limits_{\Psi \in \Gamma} I(\Psi)=
\min\limits_{\substack{{\Psi_i \in \Gamma_i\,,}\\ 
{i=\overline{1,N}}}} I\left(\Psi_1,\Psi_2,\ldots,\Psi_N\right)={}\\
{}=
\min\limits_{\substack{{\Psi_i^* \in \Gamma_i^*,}\\ 
{i=\overline{1,N}}}}
I\left(\Psi_1^*,\Psi_2^*,\ldots,\Psi_N^*\right)={}\\
{}=\min\limits_{(u_1,u_2,\ldots,u_N)\in U}\fr{A(u_1,u_2,\ldots,u_N)}
{B(u_1,u_2,\ldots,u_N)}\,. %\label{e20}
\end{multline*}
\item \textit{Если функция $C(u_1,u_2,\ldots,u_N)\hm=
{A(u_1,u_2,\ldots,u_N)}/{B(u_1,u_2,\ldots,u_N)}$ ограничена сверху или снизу, 
но не достигает глобального экстремума на множестве $U\hm=U_1\times U_2\times\cdots
\times U_N$,
то для любого $\varepsilon\hm > 0$ можно выбрать $\varepsilon$-оп\-ти\-маль\-ную 
детерминированную стратегию управления полумарковским процессом~$\xi(t)$, 
которая определяется вырожденной
вероятностной мерой $\Psi^{*(+)}(\varepsilon)\hm\in \Gamma^*$ или вырожденной
вероятностной мерой $\Psi^{*(-)}(\varepsilon)\hm\in \Gamma^*$, в~зависимости от 
вида экстремума (максимума или минимума) в~задаче}~(\ref{e18}). 
\textit{При этом вероятностная мера $\Psi^{*(+)}(\varepsilon)\hm\in \Gamma^*$ может быть 
сосредоточена в~любой точке $\left(u_1^{(+)}(\varepsilon),u_2^{(+)}(\varepsilon),\ldots,
u_N^{(+)}(\varepsilon)\right)$, удовлетворяющей соотношению}:
\begin{multline}
\sup\limits_{(u_1,u_2,\ldots,u_N) \in U}
\fr{A(u_1,u_2,\ldots,u_N)}{B(u_1,u_2,\ldots,u_N)}-\varepsilon <{}\\
{}<
\fr{A\left(u_1^{(+)}(\varepsilon),u_2^{(+)}(\varepsilon),\ldots,u_N^{(+)}
(\varepsilon)\right)}
{B\left(u_1^{(+)}(\varepsilon),u_2^{(+)}(\varepsilon),\ldots,u_N^{(+)}
(\varepsilon)\right)}<{}\\
{}<\sup\limits_{(u_1,u_2,\ldots,u_N) \in U}
\fr{A(u_1,u_2,\ldots,u_N)}{B(u_1,u_2,\ldots,u_N)}<\infty\,, 
\label{e21}
\end{multline}
\textit{если функция $C(u_1,u_2,\ldots,u_N)$ ограничена сверху 
и~экстремальная задача}~(\ref{e18}) 
\textit{представляет собой задачу на максимум. Аналогично вероятностная мера 
$\Psi^{*(-)}(\varepsilon)\hm\in \Gamma^*$ может быть сосредоточена в~любой точке 
$\left(u_1^{(-)}(\varepsilon),u_2^{(-)}(\varepsilon),\ldots,u_N^{(-)}(\varepsilon)
\right)$, удовлетворяющей соотношению}:

\noindent
\begin{multline*}
-\infty<\inf\limits_{(u_1,u_2,\ldots,u_N) \in U}\fr{A(u_1,u_2,\ldots,u_N)}
{B(u_1,u_2,\ldots,u_N)} <{}\\
{}<
\fr{A\left(u_1^{(-)}(\varepsilon),u_2^{(-)}
(\varepsilon),\ldots,u_N^{(-)}(\varepsilon)\right)}
{B\left(u_1^{(-)}(\varepsilon),u_2^{(-)}(\varepsilon),\ldots,
u_N^{(-)}(\varepsilon)\right)}<{}\\
{}<\inf\limits_{(u_1,u_2,\ldots,u_N) \in U}
\fr{A(u_1,u_2,\ldots,u_N)}{B(u_1,u_2,\ldots,u_N)}+\varepsilon\,, 
%\label{e22}
\end{multline*}
\textit{если функция $C(u_1,u_2,\ldots,u_N)$ ограничена снизу и~экстремальная 
задача}~(\ref{e18})  \textit{представляет собой задачу на минимум}.
\item \textit{Если функция $C(u_1,u_2,\ldots,u_N)\hm=
{A(u_1,u_2,\ldots,u_N)}/{B(u_1,u_2,\ldots,u_N)}$ не ограничена сверху 
или снизу, то оптимальной стратегии управления в~смысле
соответствующей экстремальной задачи не существует. 
При этом найдется такая последовательность вырожденных вероятностных
мер~$\Psi^{*(+)}(n)$, сосредоточенных в~точках 
$\left(u_1^{(+)}(n),u_2^{(+)}(n),\ldots,u_N^{(+)}(n)\right)$, $n\hm=1,2,\dots $, 
для которых выполняется соотношение}:
\begin{multline*}
I\left(\Psi^*(n)\right)={}\\
{}=
I\left(\Psi_1^{*(+)}(n),\Psi_2^{*(+)}(n),\ldots,\Psi_N^{*(+)}(n)\right)={}\\
{}=\fr{A\left(u_1^{(+)}(n),u_2^{(+)}(n),\ldots,u_N^{(+)}(n)\right)}
{B\left(u_1^{(+)}(n),u_2^{(+)}(n),\ldots,u_N^{(+)}(n)\right)}\to 
\infty\\
\mbox{при}\ n\to\infty\,, 
%\label{e23}
\end{multline*}
\textit{если функция $C(u_1,u_2,\ldots,u_N)$ не ограничена сверху. 
Аналогично найдется такая последовательность вырожденных вероятностных
мер~$\Psi^{*(-)}(n)$, сосредоточенных в~точках 
$\left(u_1^{(-)}(n),u_2^{(-)}(n),\ldots,u_N^{(-)}(n)\right)$, 
$n\hm=1,2,\dots $, для которых выполняется соотношение}:
\begin{multline*}
I\left(\Psi^{*(-)}(n)\right)={}\\
{}= I
\left(\Psi_1^{*(-)}(n),\Psi_2^{*(-)}(n),\ldots,\Psi_N^{*(-)}(n)\right)={}\\
{}=\fr{A\left(u_1^{(-)}(n),u_2^{(-)}(n),\ldots,u_N^{(-)}(n)\right)}
{B\left(u_1^{(-)}(n),u_2^{(-)}(n),\ldots,u_N^{(-)}(n)\right)}\to 
-\infty\\
\mbox{при}~~n\to\infty\,,  
%\label{e24}
\end{multline*}
\textit{если функция $C(u_1,u_2,\ldots,u_N)$ не ограничена \mbox{снизу}}.
\end{enumerate}
\textit{При этом сформулированные утверждения каждого пункта теоремы~$2$ 
могут выполняться как по отдельности, для одного из двух
видов экстремума, так и~совместно, для обоих видов экстремума.}

\smallskip

Прежде чем непосредственно доказывать теорему~2, докажем некоторые 
вспомогательные утверждения.

\smallskip

\noindent
\textbf{Лемма~1.}\ 
\textit{Рассмотрим дроб\-но-ли\-ней\-ный интегральный функционал 
$I(\Psi_1, \Psi_2,\ldots, \Psi_N)$ вида}~(\ref{e11}), 
\textit{заданный на некотором множестве наборов вероятностных мер 
$\Psi\hm=(\Psi_1, \Psi_2,\ldots, \Psi_N)\hm \in \Gamma$. Предположим, что на 
множестве~$\Gamma$ выполняется условие~$1$ из набора предварительных условий 
и~функция $B(u_1, u_2,\ldots, u_N)$  обладает свойством строгой знакопостоянности 
при всех $(u_1, u_2,\ldots, u_N) \hm\in U$. Тогда справедливы следующие утверждения}:
\begin{enumerate}[1.]
\item \textit{Если основная функция 
$C(u_1, u_2,\ldots, u_N)\hm={A(u_1, u_2,\ldots, u_N)}/{B(u_1, u_2,\ldots, u_N)}$ 
ограничена сверху, т.\,е.\ выполняется условие}
\begin{multline}
C\left(u_1, u_2,\ldots, u_N\right)=
\fr{A(u_1, u_2,\ldots, u_N)}{B(u_1, u_2,\ldots, u_N)}\leq {}\\
{}\leq
c_0^{(+)}<\infty \,, \enskip \left(u_1, u_2,\ldots, u_N\right) \in U\,, \label{e25}
\end{multline}
\textit{то имеет место неравенство}:
\begin{equation}
I\left(\Psi_1, \Psi_2,\ldots, \Psi_N\right)\leq c_0^{(+)} 
\label{e26}
\end{equation}
\textit{для всех} $(\Psi_1, \Psi_2,\ldots, \Psi_N) \in \Gamma$.
\item \textit{Если основная функция 
$C(u_1, u_2,\ldots, u_N)\hm={A(u_1, u_2,\ldots, u_N)}/{B(u_1, u_2,\ldots, u_N)}$ 
ограничена снизу, т.\,е.\ выполняется условие}
\begin{multline*}
C\left(u_1, u_2,\ldots, u_N\right)=\fr{A(u_1, u_2,\ldots, u_N)}{B(u_1, u_2,\ldots, 
u_N)}\geq{}\\
{}\geq c_0^{(-)}>-\infty \,, 
\left(u_1, u_2,\ldots, u_N\right) \in U\,, 
%\label{e27}
\end{multline*}
\textit{то имеет место неравенство}:
\begin{equation*}
I\left(\Psi_1, \Psi_2,\ldots, \Psi_N\right)\geq c_0^{(-)} 
%\label{e28}
\end{equation*}
\textit{для всех} $(\Psi_1, \Psi_2,\ldots, \Psi_N) \hm\in \Gamma$.
\end{enumerate}

\noindent
Д\,о\,к\,а\,з\,а\,т\,е\,л\,ь\,с\,т\,в\,о\ \ леммы~1.\ 
Докажем первое утверждение леммы. Предположим сначала, 
что функция $B(u_1, u_2,\ldots,  u_N)$ строго положительна:
\begin{equation}
B\left(u_1, u_2,\ldots, u_N\right)>0\,,\enskip
\left(u_1, u_2,\ldots, u_N\right)\in U\,. \label{e29}
\end{equation}
Заметим, что в~таком случае по свойству интеграла~\cite[гл.~V]{18}
\begin{multline}
\hspace*{-2mm}\int\limits_{U_1}\!\!\cdots\! \!\int\limits_{U_N}\!\!B(u_1, \ldots,u_N) \,
d\Psi_1(u_1)%d\Psi_2(u_2)\cdots\\
\cdots d\Psi_N(u_N)>0 \!\!\!\!\label{e30}
\end{multline}
для любого фиксированного набора $\Psi\hm=(\Psi_1, \ldots, \Psi_N)\hm\in \Gamma$.
Из неравенства~(\ref{e25}) с~уче\-том~(\ref{e29}) получаем:
\begin{multline}
\hspace*{-4mm}A\left(u_1,\ldots, u_N\right)\leq{}\\
\hspace*{-4mm}{}\leq c_0^{(+)} B\left(u_1, \ldots, u_N\right)\,, 
\left(u_1, \ldots, u_N\right)\in U\,. \label{e31}
\end{multline}
В свою очередь, из неравенства~(\ref{e31}) и~свойств интеграла следует:
\begin{multline}
\int\limits_{U_1}\!\!\cdots\! \!\int\limits_{U_N}\!\!A(u_1,\ldots, u_N) \,
d\Psi_1\left(u_1\right)%d\Psi_2\left(u_2\right)\cdots\\
\cdots d\Psi_N\left(u_N\right)\leq\\
\hspace*{-24pt}\leq 
c_0^{(+)}\!\!\int\limits_{U_1}\!\!\cdots\!\! \int\limits_{U_N}\!\!\!B\!\left(u_1,\ldots, u_N\right)
 d\Psi_1\!\left(u_1\right)\!%d\Psi_2\left(u_2\right)\cdots\\
 \cdots d\Psi_N\!\left(u_N\right)\!\! 
 \label{e32}
\end{multline}
для любого фиксированного набора $\Psi\hm=(\Psi_1, \ldots, \Psi_N)\hm\in \Gamma$. 
Но тогда из~(\ref{e32}) с~учетом~(\ref{e30}) получаем:
\begin{multline}
I(\Psi_1, \ldots, \Psi_N)={}\\
{}=
\fr{\int\nolimits_{U_1}\!\cdots\! \int\nolimits_{U_N}\!\!A\left(u_1, \ldots, u_N\right)\,
 d\Psi_1(u_1)\cdots d\Psi_N(u_N)}{
\int\nolimits_{U_1}\!\cdots\! \int\nolimits_{U_N}\!\!B\left(u_1, \ldots, u_N\right)\,
 d\Psi_1(u_1)
 \cdots d\Psi_N(u_N)}\leq{}\\
 {}\leq c_0^{(+)} 
 \label{e33}
\end{multline}
для любого фиксированного набора $(\Psi_1, \ldots\linebreak\ldots, \Psi_N)\hm\in \Gamma$.

Предположим теперь, что функция $B(u_1,\ldots, u_N)$ строго отрицательна:
\begin{equation}
B(u_1,\ldots, u_N)<0 \quad \left(u_1, \ldots, u_N\right)\in U\,. 
\label{e34}
\end{equation}
Тогда
\begin{multline}
\hspace*{-6pt}\int\limits_{U_1}\!\!\cdots\!\! \int\limits_{U_N}\!\!B\!\left(u_1,\ldots, u_N\right)\!
 d\Psi_1(u_1) \cdots d\Psi_N(u_N)<0 \!\!\!
 \label{e35}
\end{multline}
для любого фиксированного набора $(\Psi_1, \ldots\linebreak \ldots, \Psi_N)\hm\in \Gamma$.

Как и~ранее, будем исходить из неравенства~(\ref{e25}). 
При выполнении условий~(\ref{e34}) и~(\ref{e35}) характер неравенств~(\ref{e31}) 
и~(\ref{e32}) меняется на противоположный, но характер неравенства~(\ref{e33}) 
остается неизменным. Таким образом, для любой функции 
$B(u_1, u_2,\ldots, u_N)$, обладающей свойством строгой знакопостоянности, 
из условия~(\ref{e25}) следует выполнение неравенства~(\ref{e33}), 
которое совпадает с~(\ref{e26}). Первое утверждение леммы~1 доказано. 
Второе утверждение доказывается аналогично. Лемма~1 доказана.

\smallskip

\noindent
\textbf{Лемма 2.} \textit{Рассмотрим дроб\-но-ли\-ней\-ный интегральный функционал 
$I(\Psi_1, \Psi_2,\ldots, \Psi_N)$ вида}~(\ref{e11}), 
\textit{заданный на некотором множестве наборов вероятностных мер 
$\Psi\hm=(\Psi_1, \Psi_2,\ldots, \Psi_N)\hm\in \Gamma$. Предпо\-ложим, что на 
множестве~$\Gamma$ выполняется условие~$1$ из набора предварительных условий 
и~функция $B(u_1, u_2,\ldots, u_N)$ обладает свойством строгой знакопостоянности 
при всех $(u_1, u_2,\ldots, u_N)\hm\in U$. Тогда справедливы следующие утверждения}:
\begin{enumerate}[1.]
\item \textit{Если основная функция $C(u_1, u_2,\ldots, u_N)\hm=
{A(u_1, u_2,\ldots, u_N)}/{B(u_1, u_2,\ldots, u_N)}$ ограничена сверху, 
но не достигает своего максимального 
значения, то имеет место неравенство}:
\begin{multline}
I\left(\Psi_1, \Psi_2,\ldots, \Psi_N\right)<{}\\
{}< \sup\limits_{(u_1, u_2,\ldots, u_N)\in U}
 C\left(u_1, u_2,\ldots, u_N\right)<\infty \label{e36}
\end{multline}
\textit{для всех} $(\Psi_1, \Psi_2,\ldots, \Psi_N)\in \Gamma$.
\item \textit{Если основная функция $C(u_1, u_2,\ldots, u_N)\hm=
{A(u_1, u_2,\ldots, u_N)}/{B(u_1, u_2,\ldots, u_N)}$ ограничена снизу, 
но не достигает своего минимального значения, то имеет место неравенство}:
\begin{multline*}
I\left(\Psi_1, \Psi_2,\ldots, \Psi_N\right)>{}\\
{}> \inf\limits_{(u_1, u_2,\ldots, u_N)\in U} 
C\left(u_1, u_2,\ldots, u_N\right)>-\infty 
%\label{e37}
\end{multline*}
\textit{для всех} $(\Psi_1, \Psi_2,\ldots, \Psi_N)\hm\in \Gamma$.
\end{enumerate}

\noindent
Д\,о\,к\,а\,з\,а\,т\,е\,л\,ь\,с\,т\,в\,о\ \ леммы~2. 
Докажем первое утверждение леммы. Поскольку множество значений 
основной функции $C(u_1, u_2,\ldots, u_N)$ ограничено сверху, оно имеет конечную 
верхнюю грань:
$$
\exists \sup\limits_{(u_1, u_2,\ldots, u_N)\in U} 
C\left(u_1, u_2,\ldots, u_N\right)<\infty
$$
(см.~\cite[гл.~1, \S3, п.~3.4, теорема~1]{25}).

По условию функция $C(u_1, u_2,\ldots, u_N)$ не достигает своего максимального 
значения. Следовательно, выполняется неравенство:
\begin{multline}
C(u_1, u_2,\ldots, u_N)=\fr{A(u_1, u_2,\ldots, u_N)}{B(u_1, u_2,\ldots, u_N)}<{}\\
{}< 
\sup\limits_{(u_1, u_2,\ldots, u_N)\in U} C(u_1, u_2,\ldots, u_N)<\infty\,, 
\\
\left(u_1, u_2,\ldots, u_N\right)\in U\,.
\label{e38}
\end{multline}
Взяв за основу строгое неравенство~(\ref{e38}), проведем рассуждения, аналогичные тем, 
которые были проведены в~лемме~1 по отношению к~неравенству~(\ref{e25}). 
В~результате получим строгое неравенство~(\ref{e36}).

Второе утверждение леммы~2 доказывается аналогично. Лемма~2 доказана.

\noindent
Д\,о\,к\,а\,з\,а\,т\,е\,л\,ь\,с\,т\,в\,о\ 
\ теоремы~2.
Начнем с~доказательства утверждения~1. Предположим сначала, что основная 
функция $C(u_1, u_2,\ldots, u_N)={A(u_1, u_2,\ldots, u_N)}/{B(u_1, u_2,\ldots, u_N)}$ 
ограничена сверху и~достигает глобального максимума на множестве~$U$ 
в~некоторой точке $u^{(+)}\hm=\left(u^{(+)}_1,u^{(+)}_2,\ldots,u^{(+)}_N\right)\hm\in U$,
а~именно:
\begin{multline*}
\max\limits_{(u_1, u_2,\ldots, u_N)\in U} C\left(u_1, u_2,\ldots, u_N\right) = {}\\
{}=
C\left(u^{(+)}_1,u^{(+)}_2,\ldots,u^{(+)}_N\right)<\infty\,.
\end{multline*}
Тогда выполняется соотношение:
\begin{multline}
C(u_1, u_2,\ldots, u_N)=\fr{A(u_1, u_2,\ldots, u_N)}{B(u_1, u_2,\ldots, u_N)}
\leq{}\\
{}\leq C\left(u^{(+)}_1,u^{(+)}_2,\ldots,u^{(+)}_N\right)<\infty\,, 
\\
\left(u_1, u_2,\ldots, u_N\right)\in U\,.
\label{e39}
\end{multline}
Условия леммы~1 выполнены, и~можно воспользоваться ее утверждениями. 
Согласно первому из них, если выполняется неравенство~(\ref{e39}), 
то имеет место соотношение:
\begin{equation*}
I(\Psi_1, \Psi_2,\ldots, \Psi_N)\leq 
C\left(u^{(+)}_1,u^{(+)}_2,\ldots,u^{(+)}_N\right)<\infty 
%\label{e40}
\end{equation*}
для всех стратегий управления $\Psi\hm=(\Psi_1, \Psi_2,\ldots\linebreak
\ldots, \Psi_N)\hm\in \Gamma$.

Таким образом, множество значений дроб\-но-ли\-ней\-но\-го интегрального 
функционала $I(\Psi_1, \Psi_2,\ldots, \Psi_N)$ ограничено сверху при всех 
$\Psi\hm=(\Psi_1, \Psi_2,\ldots, \Psi_N)\hm\in \Gamma$. Тогда существует верхняя 
грань этого множества и~выполняется неравенство:
\begin{multline}
\sup\limits_{(\Psi_1, \Psi_2,\ldots, \Psi_N)\in \Gamma} 
I\left(\Psi_1, \Psi_2,\ldots, \Psi_N\right)\leq {}\\
{}\leq
C\left(u^{(+)}_1,u^{(+)}_2,\ldots,u^{(+)}_N\right). \label{e41}
\end{multline}
Рассмотрим детерминированную стратегию управ\-ле\-ния 
$\Psi^{*(+)}\hm=\left(\Psi_1^{*(+)}, \Psi_2^{*(+)},\ldots, \Psi_N^{*(+)}\right)$, 
в~которой каждая вероятностная мера~$\Psi_i^{*(+)}$ является вы\-рож\-ден\-ной 
и~сосредоточена в~точке $u_i^{(+)}$, $i\hm=\overline{1, N}$.
По свойству интеграла
\begin{multline}
I\left(\Psi_1^{*(+)}, \Psi_2^{*(+)},\ldots ,\Psi_N^{*(+)}\right)={}\\
{}=
C\left(u^{(+)}_1,u^{(+)}_2,\ldots,u^{(+)}_N\right). \label{e42}
\end{multline}
Из соотношений~(\ref{e41}) и~(\ref{e42}) получаем:
\begin{multline}
\sup\limits_{(\Psi_1, \Psi_2,\ldots, \Psi_N)\in \Gamma} 
I\left(\Psi_1, \Psi_2,\ldots, \Psi_N\right)\leq{}\\
{}\leq
 I\left(\Psi_1^{*(+)}, 
\Psi_2^{*(+)},\ldots, \Psi_N^{*(+)}\right). \label{e43}
\end{multline}
Заметим дополнительно, что выполняются отношения принадлежности:
\begin{equation}
\Psi^{*(+)}=\left(\Psi_1^{*(+)}, \Psi_2^{*(+)},\ldots, \Psi_N^{*(+)}\right) 
\in \Gamma^* \subset \Gamma\,. \label{e44}
\end{equation}
Из~(\ref{e44}) и~свойства верхней грани следует:
\begin{multline}
\sup\limits_{\left(\Psi_1^{*}, \Psi_2^{*},\ldots, \Psi_N^{*}\right) \in \Gamma^*} 
I\left(\Psi_1^{*}, \Psi_2^{*},\ldots, \Psi_N^{*}\right)\leq {}\\
{}\leq
\sup\limits_{\left(\Psi_1, \Psi_2,\ldots, \Psi_N\right) 
\in \Gamma} I\left(\Psi_1, \Psi_2,\ldots, \Psi_N\right)\,. 
\label{e45}
\end{multline}
Объединяя~(\ref{e42}), (\ref{e43}) и~(\ref{e45}), получаем соотношение:
\begin{multline}
\sup\limits_{\left(\Psi_1^{*}, \Psi_2^{*},\ldots, \Psi_N^{*}\right) 
\in \Gamma^*} I\left(\Psi_1^{*}, \Psi_2^{*},\ldots, 
\Psi_N^{*}\right)\leq{}\\
{}\leq \sup\limits_{\left(\Psi_1, \Psi_2,\ldots, \Psi_N\right) 
\in \Gamma} I\left(\Psi_1, \Psi_2,\ldots, \Psi_N\right)\leq{}\\
{}\leq I\left(\Psi_1^{*(+)}, \Psi_2^{*(+)},\ldots, \Psi_N^{*(+)}\right)={}\\
{}=
\fr{A\left(u^{(+)}_1,u^{(+)}_2,\ldots,u^{(+)}_N\right)}{B\left(u^{(+)}_1,u^{(+)}_2,
\ldots,u^{(+)}_N\right)}\,.
 \label{e46}
\end{multline}
Из соотношения~(\ref{e46}) с~учетом~(\ref{e44}) получаем, что максимум 
функционала $I(\Psi_1, \Psi_2,\ldots, \Psi_N)$ на множестве допустимых стратегий 
$\Psi\hm=(\Psi_1, \Psi_2,\ldots, \Psi_N)\hm\in \Gamma$ существует и~достигается 
на детерминированной стратегии $\left(\Psi_1^{*(+)}, \Psi_2^{*(+)},\ldots, 
\Psi_N^{*(+)}\right)$.

Кроме того, выполняются соотношения~(\ref{e19}). Таким образом, утверждение~1 
в~случае, когда основная функция $C(u_1, u_2,\ldots, u_N)$ достигает глобального 
максимума, доказано. Соответствующее утверждение в~случае, когда основная функция 
$C(u_1, u_2,\ldots, u_N)$ достигает глобального минимума, доказывается аналогично. 
При этом используется второе утверждение леммы~1.

\smallskip

Перейдем к~доказательству второго утверждения теоремы~2. Предположим, что основная 
функция $C(u_1, u_2,\ldots, u_N)\hm=A(u_1, u_2,\ldots$\linebreak
$\ldots, u_N)/{B(u_1, u_2,\ldots, u_N)}$ 
ограничена сверху, но не достигает глобального максимума на множестве 
$U \hm= U_1 \times U_2 \times \cdots \times U_N$. Тогда множество значений 
основной функции имеет конечную верхнюю грань:

\noindent
\begin{multline*}
C\left(u_1, u_2,\ldots, u_N\right)=\fr{A(u_1, u_2,\ldots, u_N)}
{B(u_1, u_2,\ldots, u_N)}<{}\\
{}<
\sup\limits_{(u_1, u_2,\ldots, u_N)\in U} \fr{A(u_1, u_2,\ldots, u_N)}
{B(u_1, u_2,\ldots, u_N)}<\infty\,, 
\\
\left(u_1, u_2,\ldots, u_N\right)\in U\,.
%\label{e47}
\end{multline*}
По определению верхней грани для любого фиксированного $\varepsilon \hm>0$ 
существует точка $(u_1^{(+)}(\varepsilon), u_2^{(+)}(\varepsilon),\ldots, 
u_N^{(+)}(\varepsilon))$ такая, что выполняется двойное неравенство~(\ref{e21}) 
(см.~\cite[гл.~1, \S\,3, п.~3.4]{25}). Иначе говоря, значение основной функции 
в~указанной точке лежит в~левой \mbox{$\varepsilon$-окрест}\-ности верхней грани. 
Рассмотрим детерминированную стратегию управления 
$\Psi^{*(+)}(\varepsilon)\hm=\!\left(\Psi_1^{*(+)}(\varepsilon), 
\Psi_2^{*(+)}(\varepsilon),\ldots, \Psi_N^{*(+)}(\varepsilon)\!\right)$, компонентами\linebreak 
которой являются вырожденные вероятностные меры $\Psi_1^{*(+)}(\varepsilon), 
\Psi_2^{*(+)}(\varepsilon),\ldots, \Psi_N^{*(+)}(\varepsilon)$, причем вырожденная 
мера~$\Psi_i^{*(+)}(\varepsilon)$ сосредоточена в~точке~$u_i^{(+)}(\varepsilon)$,
$i\hm=1,2,\ldots,N$.

По свойству интеграла
\begin{multline}
I\left(\Psi_1^{*(+)}(\varepsilon), \Psi_2^{*(+)}(\varepsilon),\ldots,
 \Psi_N^{*(+)}(\varepsilon)\right)={}\\
 {}=
 C\left(u_1^{(+)}(\varepsilon), u_2^{(+)}(\varepsilon),\ldots, 
 u_N^{(+)}(\varepsilon)\right)\,. 
 \label{e48}
\end{multline}
Из соотношения~(\ref{e48}) с~учетом указанного свойства основной функции получаем:
\begin{multline}
\sup\limits_{(u_1, u_2,\ldots, u_N)\in U} \fr{A(u_1, u_2,\ldots, u_N)}
{B(u_1, u_2,\ldots, u_N)}-\varepsilon<{}\\
{}< I\left(\Psi_1^{*(+)}(\varepsilon), 
\Psi_2^{*(+)}(\varepsilon),\ldots, \Psi_N^{*(+)}(\varepsilon)\right)<{}
\\
{}< \sup\limits_{(u_1, u_2,\ldots, u_N)\in U} \fr{A(u_1, u_2,\ldots, u_N)}
{B(u_1, u_2,\ldots, u_N)}<\infty\,. 
\label{e49}
\end{multline}
Заметим также, что в~рассматриваемом случае выполнены условия леммы~2. 
Воспользуемся первым утверждением этой леммы, а~именно соотношением~(\ref{e36}):
\begin{multline}
I(\Psi_1, \Psi_2,\ldots, \Psi_N)< {}\\
{}<\sup\limits_{(u_1, u_2,\ldots, u_N)
\in U} \fr{A(u_1, u_2,\ldots, u_N)}{B(u_1, u_2,\ldots, u_N)}<\infty 
\label{e50}
\end{multline}
для всех $(\Psi_1, \Psi_2,\ldots, \Psi_N)\in\Gamma$.

Из соотношений~(\ref{e49}) и~(\ref{e50}) следует, что детерминированная стратегия 
$\Psi^{*(+)}(\varepsilon)\hm=\left(\Psi_1^{*(+)}(\varepsilon), \Psi_2^{*(+)}(\varepsilon),
\ldots, \Psi_N^{*(+)}(\varepsilon)\right)$, опре\-де\-ля\-емая набором вырожденных 
вероятностных мер, сосредоточенных в~соответствующих точках 
$\left(u_1^{(+)}(\varepsilon), u_2^{(+)}(\varepsilon),\ldots, 
u_N^{(+)}(\varepsilon)\right)$, является $\varepsilon$-оп\-ти\-маль\-ной. 
Вторая часть утверждения~2 теоремы~2, связанная со свойствами нижней грани, 
доказывается аналогично.

Докажем третье утверждение теоремы~2. Предположим, что множество значений 
основной функции $C(u_1, u_2,\ldots, u_N)\hm=
A(u_1, u_2,\ldots$\linebreak $\ldots, u_N)/{B(u_1, u_2,\ldots, u_N)}$
не является ограниченным сверху на множестве $U\hm=U_1\times U_2 \times \cdots $\linebreak
$\cdots \times U_N$.
Тогда существует последовательность\linebreak точек $\left(u_1^{(+)}(n), u_2^{(+)}(n),
\ldots,u_N^{(+)}(n)\right)\hm\in U$, $n\hm=1,2,\ldots$, для которой
\begin{multline}
C\left(u_1^{(+)}(n), u_2^{(+)}(n),\ldots,u_N^{(+)}(n)\right)={}\\
{}=
\fr{A\left(u_1^{(+)}(n), u_2^{(+)}(n),\ldots,u_N^{(+)}(n)\right)}
{B\left(u_1^{(+)}(n), u_2^{(+)}(n),\ldots,u_N^{(+)}(n)\right)}
\longrightarrow \infty \,,\\
n\rightarrow \infty\,.
\label{e51}
\end{multline}
Зафиксируем некоторую последовательность точек $\left(u_1^{(+)}(n), u_2^{(+)}(n),
\ldots,u_N^{(+)}(n)\right)\hm\in U$, $n\hm=1,2,\ldots$, обладающих указанным свойством, 
и~рассмотрим последовательность детерминированных  стратегий управления 
$\Psi^{*(+)}(n)\hm=\left(\Psi_1^{*(+)}(n), \Psi_2^{*(+)}(n),\ldots, 
\Psi_N^{*(+)}(n)\right)$, $n\hm=1,2,\ldots$, определяемых набором вырожденных 
вероятностных мер, сосредоточенных в~соответствующих точках 
$\left(u_1^{(+)}(n), u_2^{(+)}(n),\ldots,u_N^{(+)}(n)\right)$, $n\hm=1,2,\ldots$ 
По свойству интеграла для любого фиксированного значения $n=1,2,\ldots$ 
выполняется равенство:
\begin{multline}
I \left(\Psi^{*(+)}(n)\right)={}\\
{}=I\left(\Psi_1^{*(+)}(n), \Psi_2^{*(+)}(n),\ldots,
 \Psi_N^{*(+)}(n)\right)={}\\
{}=\fr{A\left(u_1^{(+)}(n), u_2^{(+)}(n),\ldots,u_N^{(+)}(n)\right)}
{B\left(u_1^{(+)}(n), u_2^{(+)}(n),\ldots,u_N^{(+)}(n)\right)}\,. 
\label{e52}
\end{multline}
Из соотношений~(\ref{e51}) и~(\ref{e52}) следует, что
\begin{multline}
I\left(\Psi^{*(+)}(n)\right)={}\\
{}=I\left(\Psi_1^{*(+)}(n), \Psi_2^{*(+)}(n),\ldots, 
\Psi_N^{*(+)}(n)\right)\longrightarrow\infty\,,\\ 
n \rightarrow\infty\,.
 \label{e53}
\end{multline}
Соотношение~(\ref{e53}) означает, что множество значе\-ний дроб\-но-ли\-ней\-но\-го 
интегрального функциона\-ла $I(\Psi_1, \Psi_2,\ldots, \Psi_N)$ вида~(\ref{e11}) 
не ограничено сверху\linebreak на множестве наборов вырожденных вероятностных мер 
$\left(\Psi_1^{*(+)}(n), \Psi_2^{*(+)}(n),\ldots, \Psi_N^{*(+)}(n)\right)\hm\in\Gamma^*$, 
а~следовательно, и~на более широком\linebreak множестве наборов вероятностных 
мер $(\Psi_1, \Psi_2,\ldots$\linebreak $\ldots, \Psi_N)\hm\in\Gamma$. В~таком случае решения экстремальной 
задачи~(\ref{e18}) в~форме задачи на максимум не существует. Соответствующее утвержде\-ние 
для варианта, когда множество значений основной функции $C(u_1, u_2,\ldots,u_N)
\hm=A(u_1, u_2,\ldots$\linebreak $\ldots,u_N)/{B(u_1, u_2,\ldots,u_N)}$ 
не является ограниченным снизу, доказывается аналогично. Третье утверж\-де\-ние теоремы~2 
доказано. Тем самым тео\-ре\-ма~2 доказана полностью.

\smallskip

Применим теорему~2 для решения поставленной задачи оптимального управления. 
Из утверждения этой теоремы следует, что для доказательства су-\linebreak ществования 
оптимального управ\-ле\-ния и~его нахождения необходимо исследовать на 
глобальный экстремум основную функцию дроб\-но-ли\-ней\-но\-го интегрального 
функционала $C(u_1,u_2,\ldots,u_N)$, определяемую формулой~(\ref{e17}) с~учетом 
равенств~(\ref{e12})--(\ref{e16}). В~некоторых случаях, например когда основной 
процесс~$\xi(t)$ является регенерирующим, а~стоимостные характеристики 
модели задаются линейными функциями, такое исследование можно провести 
аналитически. Однако для подавляющего большинства полумарковских моделей 
для этого необходимо использовать численные методы.

\section{Заключение}

В заключительной части работы приведем \mbox{краткое} описание теоретической 
основы метода решения задачи оптимального управления полумарковским 
процессом с~конечным множеством состояний.

\begin{enumerate}[1.]
\item Исходная проблема оптимального управления формулируется в~виде 
экстремальной задачи~(\ref{e18}). Целевым показателем качества управ\-ле\-ния в~данной задаче 
служит величина~(\ref{e10}), которая имеет характер средней удельной прибыли.
\item Доказывается, что стационарный показатель~(\ref{e10}) представим в~виде 
дроб\-но-ли\-ней\-но\-го интегрального функционала~(\ref{e11}), для которого явно 
определяются подынтегральные функции числителя и~знаменателя, а~следовательно, 
и~основная функция данного функционала.
\item Используется теорема об экстремуме дроб\-но-ли\-ней\-но\-го интегрального 
функционала. На основании утверждений этой теоремы уста\-нав\-ли\-ва\-ет\-ся, что 
исходная задача оптимального управления сводится к~исследованию на глобальный 
экстремум основной функции этого функционала, для которой получено явное 
аналитическое представление.
\end{enumerate}

Заметим, что такое исследование задач оптимального управления 
стохастическими системами фактически уже было проведено в~ряде работ П.\,В.~Шнуркова 
и~его соавторов. В~частности, в~работе~\cite{26} была рассмотрена модель 
управления для обрывающегося процесса восстановления, описывающего функционирование 
некоторой технической системы. Задача управления решалась для различных показателей 
эффективности и~надежности этой системы, имеющих структуру дроб\-но-ли\-ней\-но\-го 
интегрального функционала.

В работах~\cite{27, 28} рассматривались модели регенерирующих процессов 
для исследования сис\-тем управления запасами. Различные показатели качества 
управления были представлены в~форме дроб\-но-ли\-ней\-ных интегральных функционалов. 
Основные функции этих функционалов были найде\-ны в~явной форме и~исследовались 
на глобальный экстремум. В~работах~\cite{21,29} рассматривалась достаточно 
сложная полумарковская модель с~конечным множеством состояний, описывающая 
сис\-те\-му управления запасом непрерывного продукта. Показатели качества управления в~этой 
модели также имели структуру дроб\-но-ли\-ней\-ных интегральных функционалов, 
для основных функций которых были найдены явные аналитические представления. 
Упомянем также работы~\cite{30, 31}, в~которых была исследована полумарковская 
модель с~дис\-крет\-но-не\-пре\-рыв\-ным фазовым пространством. Показатели 
качества управления в~этой  модели были найдены в~явной форме как функции от 
двух непрерывных параметров управления.

Фактически во всех упомянутых работах уже был использован метод решения задачи 
оптимального управления регенерирующим или полумарковским случайным процессом, 
основанный на исследовании экстремальных свойств основной функции соответствующего 
дроб\-но-ли\-ней\-но\-го интегрального функционала. Из соображений, изложенных 
во\linebreak введении, следует, что в~период написания и~пуб\-ли\-кации этих работ данный метод 
не имел стро\-гого обоснования. Однако после публикации\linebreak работы~\cite{14} и~настоящего 
исследования можно утверж\-дать, что полученные в~них результаты полностью теоретически 
обоснованы.

Таким образом, изложенный выше метод решения проблемы оптимального управления 
полумарковскими процессами с~конечными множествами состояний может быть успешно 
реализован для многих задач, рассматриваемых в~различных областях прикладной 
теории вероятностей.

Практическая реализация численной процедуры поиска оптимального решения на примере\linebreak 
полумарковской модели управления запасом непрерывного продукта (подробнее 
см.~\cite{21, 29}), ба\-зи\-ру\-юща\-яся на изложенных выше результатах (в~частности, 
теореме~1), была осуществлена А.\,К.~Горшениным и~соавторами 
в~статье~\cite{Gorshenin2015}. Коротко опишем наиболее важные аспекты этой работы.

Для решения поставленной задачи опти\-мального управления была создана 
специальная программа \verb"Inventory" на встроенном языке программирования 
пакета \verb"MATLAB", ее возможности\linebreak кратко представле\-ны в~упомянутой ранее 
\mbox{статье}~\cite{Gorshenin2015}. В~программе \verb"Inventory" реализованы функции 
для оценивания через заданные исходные параметры вероятностных и~стоимостных 
характеристик модели, которые в~дальнейшем используются для поиска значений 
основной функции дроб\-но-ли\-ней\-но\-го функционала~(\ref{e17}). Точка глобального 
экстремума этой функции и~определяет оптимальное управление.

В качестве начальных данных необходимо задание следующих параметров:
\begin{itemize}
\item спрос и~вместимость склада;
\item разбиение множества значений объема запаса;
\item вероятностные характеристики, описывающие модель пополнения запаса;
\item условные математические ожидания длительностей задержек пополнения запаса;
\item функции для характеризации затрат и~доходов.
\end{itemize}

По итогам работы программы \verb"Inventory" ряд вспомогательных функций 
представляется в~аналитической форме (в частности, с~использованием аппарата 
символьных вычислений  \verb"Symbolic Toolbox"\linebreak пакета \verb"MATLAB"), выводится 
точка глобального экстремума функции нескольких вещественных переменных~(\ref{e17}), 
найденная с~помощью применения численных и~при\-бли\-жен\-но-ана\-ли\-ти\-че\-ских\linebreak 
аппроксимаций. 
Также формируются графики оценок значений ве\-ро\-ят\-ност\-но-сто\-и\-мост\-ных 
характеристик 
и~основной функции дроб\-но-ли\-ней\-но\-го функционала~(\ref{e17}), либо трехмерных 
сечений в~случае наличия более трех параметров управления (переменных).

Функциональность пакета \verb"Inventory" может быть расширена для практической 
реализации метода решения задачи поиска оптимального управ\-ле\-ния полумарковскими 
процессами с~конечными множествами состояний, рассмотренного в~данной статье.


 {\small\frenchspacing
 {%\baselineskip=10.8pt
 \addcontentsline{toc}{section}{References}
 \begin{thebibliography}{99}
 \bibitem{1}
\Au{Ховард Р.} Динамическое программирование и~марковские процессы~/ 
Пер. с~англ.~--- М.: Сов. радио, 1964. 189~с.
(\Au{Howard~R.\,A.} Dynamic programming and Markov processes.~--- 
Cambridge, MA, USA: MIT Press, 1960. 136~p.)
\bibitem{2} 
\Au{Рыков В.\,В.} Управляемые марковские процессы с~конечными пространствами 
состояний и~управлений~// Теория вероятностей и~ее применения, 1966. Т.~11. 
Вып.~2. С.~343--351.
\bibitem{3} 
\Au{Джевелл В.} Управляемые полумарковские процессы~// Кибернетич. сборник.~--- 
М.: Мир, 1967. Вып.~4. С.~97--134.
%{\em Jewell W.\,S.} Markov-renewal programming~// Operations Research, 1963. Vol.~11. P.~938--971.
\bibitem{4} 
\Au{Fox B.} Markov renewal programming by linear fractional programming~// 
SIAM J.~Appl. Math., 1966. Vol.~14. P.~1418--1432.
\bibitem{5} 
\Au{Denardo E.\,V.} Contraction mappings in the theory underlying dinamic programming~// 
SIAM Rev., 1967. Vol.~9. P.~165--177.

\bibitem{6} 
\Au{Howard R.\,A.} Research in semi-Markovian decision structures~// 
J.~Oper. Res. Soc. Japan, 1963. Vol.~6. P.~163--199.
\bibitem{7} 
\Au{Osaki S., Mine H.} Linear programming algorithms for Markovian decision processes~//
 J.~Math. Anal. Appl., 1968. Vol.~22. P.~356--381.
\bibitem{8} 
\Au{Майн Х., Осаки С.} Марковские процессы принятия решений~/ Пер. с~англ.~--- 
М.: Наука, 1977. 176~с.
(\Au{Mine~H., Osaki~S.} 
Markovian decision processes.~--- New York, NY, USA: 
American Elsevier Publishing Co., 1970. 142~p.)
\bibitem{9} 
\Au{Гихман И.\,И., Скороход А.\,В.} Управляемые случайные процессы.~--- 
Киев: Наукова думка, 1977. 251~с.
\bibitem{10} 
\Au{Luque-Vasquez F., Herndndez-Lerma~О.} Semi-Markov control models with average costs~// 
Appl. Math., 1999. Vol.~26. No.\,3. P.~315--331.
\bibitem{11} 
\Au{Vega-Amaya O., Luque-Vasquez~F.} Sample-path average cost optimality for 
semi-Markov control processes on Borel spaces: Unbounded costs and mean holding times~// 
Appl. Math., 2000. Vol.~27. No.\,3. P.~343--367.
\bibitem{12} 
Вопросы математической теории надежности~/ Под ред. Б.\,В. Гнеденко.~--- 
М.: Радио и~связь, 1983. 376~с.
\bibitem{13} 
\Au{Барзилович Е.\,Ю., Каштанов~В.\,А.} Некоторые математические вопросы теории 
обслуживания сложных систем.~---  М.: Сов. радио, 1971. 272~с.
\bibitem{14} 
\Au{Шнурков П.\,В.} О~решении проблемы безусловного экстремума для 
дроб\-но-ли\-ней\-но\-го интегрального функционала на множестве вероятностных мер~// 
Докл. РАН. Сер. Математика, 2016. Т.~470. №\,4. C.~387--392.
\bibitem{15} 
\Au{Ширяев А.\,Н.}  Вероятность.~--- М.:~МЦНМО, 2011. Кн.~1. 552~с. Кн.~2. 968~с.
\bibitem{16} 
\Au{Боровков А.\,А.} Теория вероятностей.~--- М.: Либроком, 2009. 656~c.
\bibitem{17} 
\Au{Хеннекен П.\,Л., Тортра А.} Теория вероятностей 
и~некоторые ее приложения.~--- М.: Наука, 1974. 472~c.
\bibitem{18} 
\Au{Халмош П.} Теория меры~/ Пер. с~англ.~--- М.: ИЛ, 1953. 282~c.
(\Au{Halmos~P.} Measure theory.~--- Litton Educational Publishing, Inc. 1950. 304~p.)
\bibitem{19} 
\Au{Королюк В.\,С., Турбин~А.\,Ф.} Полумарковские процессы и~их приложения.~--- 
Киев:~Наукова думка, 1976. 184~с.
\bibitem{20} 
\Au{Janssen J., Manca R.} Applied semi-Markov processes.~--- New York,
NY, USA: Springer, 2006. 309~p.
\bibitem{21} 
\Au{Шнурков П.\,В., Иванов~А.\,В.} Анализ дискретной полумарковской модели
 управления запасом непрерывного продукта при периодическом прекращении потребления~// 
 Дискретная математика, 2014. Т.~26. Вып.~1. С.~143--154.
\bibitem{22} 
\Au{Иванов~А.\,В.} Анализ дискретной полумарковской модели
 управления запасом непрерывного продукта при периодическом прекращении 
 потребления.~--- М.: НИУ ВШЭ, 2014.  Дисс.\ \ldots\ канд. физ.-мат. наук. 120~с.
\bibitem{23}  %23
\Au{Bajalinov~E.\,B.} Linear-fractional programming. 
Theory, methods, applications and software.~--- 
Boston/\linebreak Dordrecht/London: Kluwer Academic Publs., 2003. 423~p.

\bibitem{27} %27
\Au{Шнурков П.\,В., Мельников~Р.\,В.} Оптимальное управление запасом 
непрерывного продукта в~модели регенерации~// Обозрение прикладной 
и~промышленной математики, 2006. Т.~13. Вып.~3. С.~434--452.
\bibitem{28} 
\Au{Шнурков П.\,В., Мельников~Р.\,В.} 
Исследование проб\-ле\-мы управления запасом непрерывного продукта при детерминированной 
задержке поставки~// Автоматика и~телемеханика, 2008. Т.~10. С.~93--113.


\bibitem{24}  %26
\Au{Шнурков П.\,В.} Методы исследования задач оптимального обслуживания 
в~математической теории надежности.~--- 
М.: МИЭМ, 1983.  Дисс.\ \ldots\ канд. физ.-мат. наук.

 \bibitem{25}  %25
\Au{Кудрявцев Л.\,Д.} Курс математического анализа. Т.~1.~--- 
М.: Дрофа, 2006. 704~с.

\bibitem{26} %24
\Au{Шнурков П.\,В.} Оптимальное обслуживание на периоде 
до первого отказа системы~// Применение аналитических методов в~вероятностных
 задачах.~--- Киев: Институт математики АН УССР, 1986. С.~121--129.

\bibitem{29} 
\Au{Шнурков П.\,В., Иванов~А.\,В.} Исследование задачи оптимизации в~дискретной 
полумарковской модели управления непрерывным запасом~// Вестник МГТУ им.\ 
Н.\,Э. Баумана. Сер.\ Естественные науки, 2013. Т.~3. Вып.~50. С.~62--87.
\bibitem{30} 
\Au{Shnourkoff P.\,V.} The two-element system with one 
restoring device optimum maintenance~// Stoch. Anal. Appl., 1997. 
Vol.~15. No.\,5. P.~823--837.
\bibitem{31} 
\Au{Shnourkoff P.\,V.} The two-element system optimum maintenance tills the first fail~// 
Stoch. Anal. Appl., 2001. Vol.~19. No.\,6. P.~1005--1024.
\bibitem{Gorshenin2015} 
\Au{Gorshenin~A.\,K., Belousov~V.\,V., Shnourkoff~P.\,V.,
Ivanov~A.\,V.} Numerical research of the optimal control problem in the semi-Markov 
inventory model~// AIP Conference Proceedings, 2015. Vol.~1648. {250007}. 4~p.
%\bibitem{33} {\em Горшенин А.\,К., Белоусов В.\,В., Шнурков П.\,В.} 2016. Система управления запасами на основе стохастических полумарковских моделей. Свидетельство о государственной регистрации программы для ЭВМ \textnumero 2016619021.
 \end{thebibliography}

 }
 }

\end{multicols}

\vspace*{-6pt}

\hfill{\small\textit{Поступила в~редакцию 15.07.16}}

%\vspace*{8pt}

\newpage

\vspace*{-24pt}

%\hrule

%\vspace*{2pt}

%\hrule

%\vspace*{8pt}


\def\tit{ANALYTICAL SOLUTION OF~THE~OPTIMAL CONTROL TASK OF~A~SEMI-MARKOV 
PROCESS WITH~FINITE SET OF~STATES}

\def\titkol{Analytical solution of~the~optimal control task of~a~semi-Markov 
process with~finite set of~states}

\def\aut{P.\,V.~Shnurkov$^{1}$, A.\,K.~Gorshenin$^{2}$, and~V.\,V.~Belousov$^{2}$}

\def\autkol{P.\,V.~Shnurkov, A.\,K.~Gorshenin, and~V.\,V.~Belousov}

\titel{\tit}{\aut}{\autkol}{\titkol}

\vspace*{-9pt}


    
\noindent
$^1$National Research University Higher School of Economics, 34~Tallinskaya Str., 
Moscow, 123458, Russian\linebreak
$\hphantom{^9}$Federation

\noindent
$^2$Institute of Informatics Problems, Federal Research Center 
``Computer Science and Control'' of the Russian\linebreak
$\hphantom{^9}$Academy of Sciences, 44-2~Vavilova Str., 
Moscow 119333, Russian Federation



\def\leftfootline{\small{\textbf{\thepage}
\hfill INFORMATIKA I EE PRIMENENIYA~--- INFORMATICS AND
APPLICATIONS\ \ \ 2016\ \ \ volume~10\ \ \ issue\ 4}
}%
 \def\rightfootline{\small{INFORMATIKA I EE PRIMENENIYA~---
INFORMATICS AND APPLICATIONS\ \ \ 2016\ \ \ volume~10\ \ \ issue\ 4
\hfill \textbf{\thepage}}}

\vspace*{3pt}


\Abste{The theoretical verification of the new method of finding 
the optimal strategy of control of a~semi-Markov process with finite set of states is 
presented. The paper considers Markov randomized strategies of control, determined by 
a~finite collection of probability measures, corresponding to each state. The quality 
characteristic is the stationary cost index. This index is a~linear-fractional integral 
functional, depending on collection of probability measures, giving the strategy of control. 
Explicit analytical forms of integrands of numerator and denominator of this 
linear-fractional integral functional are known. The basis of consequent results is 
the new generalized and strengthened form of the theorem about an extremum of 
a~linear-fractional integral functional. It is proved that problems of existence 
of an optimal control strategy of a~semi-Markov process and finding this strategy 
can be reduced to the task of numerical analysis of global extremum for 
the given function, depending on finite number of real arguments.}

\KWE{optimal control of a~semi-Markov process; stationary cost index of quality control; 
linear-fractional integral functional}




\DOI{10.14357/19922264160408} 

\vspace*{-16pt}

\Ack
\noindent
The research was partially supported by the Russian Foundation 
for Basic Research (project 15-07-05316).



%\vspace*{3pt}

  \begin{multicols}{2}

\renewcommand{\bibname}{\protect\rmfamily References}
%\renewcommand{\bibname}{\large\protect\rm References}

{\small\frenchspacing
 {%\baselineskip=10.8pt
 \addcontentsline{toc}{section}{References}
 \begin{thebibliography}{99}
\bibitem{1-1}
\Aue{Howard,~R.\,A.} 1960. \textit{Dynamic programming and Markov processes}. 
Cambridge, MA: MIT Press. 136~p.
\bibitem{2-1}
\Aue{Rykov,~V.\,V.} 1966. Upravlyaemye markovskie protsessy 
s~konechnymi prostranstvami sostoyaniy i~upravleniy 
[Controlled Markov processes with finite spaces of states and controls ]. 
\textit{Teoriya veroyatnostey i~ee primeneniya} 
[Theory of Probability and Its Applications] 11(2):343--351.
\bibitem{3-1}
\Aue{Jewell,~W.\,S.} 1963. Markov-renewal programming. 
\textit{Oper. Res.} 11:938--971.
\bibitem{4-1}
\Aue{Fox,~B.} 1966. Markov renewal programming by linear fractional programming. 
\textit{SIAM J.~Appl. Math.} 14:1418--1432.
\bibitem{5-1}
\Aue{Denardo, E.\,V.} 1967. Contraction mappings in the theory underlying dinamic 
programming. \textit{SIAM Rev.} 9:165--177.
\bibitem{6-1}
\Aue{Howard,~R.\,A.} 1963. Research in semi-Markovian decision structures. 
\textit{J.~Oper. Res. Soc. Japan} 6:163--199.
\bibitem{7-1}
\Aue{Osaki,~S., and H.~Mine.} 1968. Linear programming algorithms 
for Markovian decision processes. \textit{J.~Math. Anal. Appl.} 22:356--381.
\bibitem{8-1}
\Aue{Mine,~H., and S.~Osaki.} 1970. 
\textit{Markovian decision processes}. New York, NY: Elsevier. 142~p.
\bibitem{9-1}
\Aue{Gikhman,~I.\,I., and A.\,V.~Skorokhod.} 1977. 
\textit{Upravlyaemye sluchaynye protsessy} 
[Controlled random processes]. Kiev: Naukova Dumka. 251~p.
\bibitem{10-1}
\Aue{Luque-Vasquez,~F., and О.~Herndndez-Lerma.} 1999. 
Semi-Markov control models with average costs. \textit{Appl. Math.} 26(3):315--331.
\bibitem{11-1}
\Aue{Vega-Amaya,~O., and  F.~Luque-Vasquez.} 2000.  
Sample-path average cost optimality for semi-Markov control processes on Borel spaces: 
Unbounded costs and mean holding times. \textit{Appl. Math.} 27(3):343--367.
\bibitem{12-1}
Gnedenko,~B.~V., ed. 1983. 
\textit{Voprosy matematicheskoy teorii nadezhnosti} 
[Problems of the mathematical theory of reliability].  Moscow: Radio i~svyaz'. 376~p.
\bibitem{13-1}
\Aue{Barzilovich,~E.\,Yu., and V.\,A.~Kashtanov.} 1971. 
\textit{Nekotorye matematicheskie voprosy teorii obsluzhivaniya slozhnykh sistem}  
[Some mathematical questions in theory of complex systems maintenance]. 
Moscow: Sovetskoe radio. 272~p.
\bibitem{14-1}
\Aue{Shnurkov,~P.\,V.} 2016. Solution of the unconditional extremum problem for 
a~linear-fractional 
integral functional on a~set of probability measures. 
\textit{Dokl. Math.} 94(2):550--554.
\bibitem{15-1} %14
\Aue{Shiryaev,~A.\,N.} 2016. 
\textit{Probability-1}. Graduate texts in mathematics ser.
New York, NY: Springer. Vol.~95. 503~p.;
2017. \textit{Probability-2.} Vol.~900. 500~p.
\bibitem{16-1}
\Aue{Borovkov,~А.\,А.} 2009. 
\textit{Teoriya veroyatnostey} [Probability theory]. Moscow: Librokom. 656~p.
\bibitem{17-1}
\Aue{Khenneken,~P.\,L., and A.~Tortra.} 1974. 
\textit{Teoriya veroyatnostey i~nekotorye ee prilozheniya} 
[Probability theory and some of its applications]. Moscow: Nauka. 472~p.
\bibitem{18-1}
\Aue{Halmos,~P.} 1950. \textit{Measure theory}. Litton Educational Publishing. 304~p.
\bibitem{19-1}
\Aue{Korolyuk, V.\,S., and A.\,F.~Turbin.} 1976. 
\textit{Polumarkovskie protsessy i~ikh prilozheniya} 
[Semi-Markov processes and their applications]. Kiev: Naukova Dumka. 184~p.
\bibitem{20-1}
\Aue{Janssen,~J., and R.~Manca.} 2006. 
\textit{Applied semi-Markov processes}. New York, NY: Springer. 309~p.
\bibitem{21-1}
\Aue{Shnurkov,~P.\,V, and A.\,V~Ivanov.} 2015. Analysis of a~discrete semi-Markov model of continuous inventory 
control with periodic interruptions of consumption. 
\textit{Discrete Math. \mbox{Appl}.} 25(1):59--67.
\bibitem{22-1} %21
\Aue{Ivanov,~A.\,V.} 2014. Analiz diskretnoy polumarkovskoy modeli upravleniya 
zapasom nepreryvnogo produkta pri periodicheskom prekrashchenii potrebleniya 
[Analysis of a~discrete semi-Markov control model of continuous product inventory 
in a~periodic cessation of consumption].  
Moscow: Natsional'nyy Issledovatel'skiy Universitet ``Vysshaya Shkola Ekonomiki.'' 
PhD Thesis. 120~p.
\bibitem{23-1} %22
\Aue{Bajalinov,~E.\,B.} 2003. 
\textit{Linear-fractional programming. Theory, methods, applications and software}. 
Boston/\linebreak Dordrecht/London: Kluwer Academic Publs. 423~p.
\bibitem{26-1} %24
\Aue{Shnurkov,~P.\,V., and R.\,V.~Mel'nikov.} 2006. Optimal'noe upravlenie 
zapasom nepreryvnogo produkta v modeli regeneratsii [Optimal control of 
a~continuous product inventory in the regeneration model]. 
\textit{Obozrenie prikladnoy i~promyshlennoy matematiki} [Rev. Appl. Ind. Math.]
13(3):434--452.

\bibitem{25-1} %25
\Aue{Shnurkov,~P.\,V., and R.\,V.~Mel'nikov.} 2008. 
Analysis of the problem of continuous-product inventory control under deterministic 
lead time. \textit{Automat. Rem. Contr.} 69(10):1734--1751.

\columnbreak

\bibitem{24-1} %26
\Aue{Shnurkov,~P.\,V.} 1983. Metody issledovaniya zadach optimal'nogo obsluzhivaniya 
v~matematicheskoy teorii nadezhnosti [Research methods of optimal service problems 
in the mathematical theory of reliability].  
Moscow: Moskovskiy Institut Elektronnogo Mashinostroeniya.  PhD Thesis. 


\bibitem{27-1} %27
\Aue{Kudryavtsev,~L.\,D.} 2006. 
\textit{Kurs matematicheskogo analiza} 
[A~course of mathematical analysis]. Vol.~1. Moscow: Drofa. 704~p.

\bibitem{28-1}
\Aue{Shnurkov,~P.\,V.} 1986. Optimal'noe obsluzhivanie na periode do 
pervogo otkaza sistemy [The optimum service period until the first system failure]. 
\textit{Primenenie analiticheskikh metodov v~veroyatnostnykh zadachakh} 
[The application of analytical methods in probabilistic tasks]. Kiev:
Institute of Mathematics of the Academy of Sciences of the USSR. 121--129.

\bibitem{29-1}
\Aue{Shnurkov,~P.\,V., and A.\,V.~Ivanov.} 2013. Issledovanie zadachi optimizatsii 
v~diskretnoy polumarkovskoy modeli upravleniya nepreryvnym zapasom 
[Study of the optimization problem in discrete semi-Markov model of continuous 
inventory control]. \textit{Vestnik MGTU im.\ N.\,E.~Baumana. Ser. 
Estestvennye nauki} [Vestnik of MSTU named after N.\,E.~Bauman. Ser. Natural sciences] 
3(50):62--87.
\bibitem{30-1}
\Aue{Shnourkoff,~P.\,V.} 1997. The two-element system with one restoring device 
optimum maintenance.  \textit{Stoch. Anal. Appl.} 15(5):823--837.
\bibitem{31-1}
\Aue{Shnourkoff,~P.\,V.} 2001. The two-element system optimum maintenance tills 
the first fail. \textit{Stoch. Anal. Appl.} 19(6):1005--1024.
\bibitem{32-1}
\Aue{Gorshenin,~A.\,K., V.\,V.~Belousov, P.\,V.~Shnourkoff, and A.\,V.~Ivanov.}
2015. Numerical research of the optimal control problem in the semi-Markov 
inventory model. \textit{AIP Conference Proceedings} 1648:250007.
\end{thebibliography}

 }
 }

\end{multicols}

\vspace*{-3pt}

\hfill{\small\textit{Received July 15, 2016}}

\Contr

\noindent
\textbf{Shnurkov Peter V.} (b.\ 1953)~---
 Candidate of Science (PhD) in physics and mathematics, 
 associate professor, National Research University Higher School of Economics, 
 34~Tallinskaya Str., Moscow 123458, Russian Federation; \mbox{pshnurkov@hse.ru} 
 
 \vspace*{3pt}
 
 \noindent
\textbf{Gorshenin Andrey K.}  (b.\ 1986)~---
Candidate of Science (PhD) in physics and mathematics, leading scientist, 
Institute of Informatics Problems, Federal Research Center ``Computer Science 
and Control'' of the Russian Academy of Sciences, 44-2~Vavilov Str., Moscow 119333, 
Russian Federation; associate professor, Federal State Budget Educational 
Institution of Higher Education ``Moscow Technological University,'' 
78~Vernadskogo Ave., Moscow 119454, Russian Federation;
\mbox{agorshenin@frccsc.ru}

\vspace*{3pt}

\noindent
\textbf{Belousov Vasiliy V.} (b.\ 1977)~---
Candidate of Science (PhD) in technology, senior scientist, Institute of 
Informatics Problems, Federal Research Center ``Computer Science and Control'' 
of the Russian Academy of Sciences, 44-2~Vavilov Str., Moscow 119333, Russian 
Federation; \mbox{VBelousov@ipiran.ru}
\label{end\stat}


\renewcommand{\bibname}{\protect\rm Литература}  %5
\def\stat{goncharov}

\def\tit{ВЫРАВНИВАНИЕ ДЕКАРТОВЫХ ПРОИЗВЕДЕНИЙ УПОРЯДОЧЕННЫХ МНОЖЕСТВ$^*$}

\def\titkol{Выравнивание декартовых произведений упорядоченных множеств}

\def\aut{А.\,В.~Гончаров$^1$, В.\,В.~Стрижов$^2$}

\def\autkol{А.\,В.~Гончаров, В.\,В.~Стрижов}

\titel{\tit}{\aut}{\autkol}{\titkol}

\index{Гончаров А.\,В.}
\index{Стрижов В.\,В.}
\index{Goncharov A.\,V.}
\index{Strijov V.\,V.}


{\renewcommand{\thefootnote}{\fnsymbol{footnote}} \footnotetext[1]
{Работа выполнена при частичной финансовой поддержке РФФИ 
(проекты 19-07-1155 и~19-07-00885). Настоящая статья содержит 
результаты проекта <<Статистические методы машинного обучения>>, 
выполняемого в~рамках реализации Программы Центра компетенций 
Национальной технологической инициативы <<Центр хранения 
и~анализа больших данных>>, поддерживаемого Министерством науки 
и~высшего образования Российской Федерации по договору МГУ им.\ 
М.\,В.~Ломоносова  с~Фондом поддержки проектов Национальной 
технологической инициативы от 11.12.2018 №\,13/1251/2018.}}


\renewcommand{\thefootnote}{\arabic{footnote}}
\footnotetext[1]{Московский физико-технический институт, alex.goncharov@phystech.edu}
\footnotetext[2]{Вычислительный центр им.\ А.\,А.~Дородницына Федерального исследовательского 
центра <<Информатика и~управ\-ле\-ние>> Российской академии наук; 
Московский фи\-зи\-ко-тех\-ни\-че\-ский институт, \mbox{strijov@ccas.ru}}

%\vspace*{-12pt}



\Abst{Работа посвящена исследованию метрических методов анализа 
объектов сложной структуры. Предлагается обобщить метод динамического 
выравнивания двух временных рядов на случай объектов, определенных на 
двух и~более осях времени. В~дискретном представлении такие объекты 
являются матрицами. Метод динамического выравнивания временных рядов 
обобщается как метод динамического выравнивания матриц. Предложена 
функция расстояния, устойчивая к~монотонным нелинейным деформациям 
декартова произведения двух и~более временных шкал. Определен выравнивающий 
путь между объектами. В~дальнейшем объектом называется матрица, 
в~которой строки и~столбцы соответствуют осям времени. Исследованы 
свойства предложенной функции расстояния. Для иллюстрации метода 
решаются задачи метрической классификации объектов на модельных 
данных и~данных из набора MNIST.}

\KW{функция расстояния; динамическое выравнивание; расстояние между матрицами; 
нелинейные деформации времени; про\-стран\-ст\-вен\-но-вре\-мен\-ные ряды}

\DOI{10.14357/19922264200105} 
  
\vspace*{-3pt}


\vskip 10pt plus 9pt minus 6pt

\thispagestyle{headings}

\begin{multicols}{2}

\label{st\stat}


\section{Введение}

Временн$\acute{\mbox{ы}}$е ряды представляют собой набор измерений, упорядоченных 
по оси времени. Анализ временн$\acute{\mbox{ы}}$х рядов производится при решении задач, 
связанных с~классификацией активности человека по измерениям акселерометра 
телефона, поиском паттернов в~EEG-сиг\-на\-лах (электроэнцефалограмма), 
кластеризации набора ECoG (электрокортикограмма) данных и~во многих других 
задачах~\cite{0}. Рассматриваются объекты, для которых время между измерениями 
фиксированно. В~данной работе для построения адекватной функции 
расстояния между объектами требуется учесть нелинейные деформации 
относительно оси времени: глобальные и~локальные сдвиги, растяжения 
и~сжатия~\cite{1}.

В~\cite{2} приводятся различные методы решения задач анализа 
временн$\acute{\mbox{ы}}$х рядов: классификации, детектирования паттернов, 
кластеризации и~др. В~\cite{3} описание временных рядов 
строится с~по\-мощью анализа параметров моделей, в~\cite{4} 
используется их признаковое описание, в~\cite{5} анализируется их форма. 
Комбинации этих подходов описаны в~\cite{2}.

Метрические методы находят схожие объекты в~наборе. Используются 
функции расстояния над временн$\acute{\mbox{ы}}$ми рядами: расстояние Хаусдорфа~\cite{10}, 
MODH~\cite{11}, расстояние, основанное на HMM
(hiden Markov model)~\cite{6}, евклидово расстояние 
в~исходном пространстве или в~пространстве сниженной размерности~\cite{5}, 
\mbox{LCSS} (longest common\linebreak subsequence)~\cite{7}. Показано~\cite{8}, что в~случае локальных или глобальных 
деформаций времени при решении задач, требующих анализа исходной формы 
временн$\acute{\mbox{о}}$го ряда, метод динамического выравнивания оси времени 
DTW (Dynamic Time Warping) 
превосходит другие функции расстояния~\cite{9} по качеству итогового 
решения задачи, так как при наличии смещений двух объектов относительно 
друг друга требуется выравнивать их оптимальным образом для вычисления 
расстояния между ними.

В данной работе предлагается перейти от рас\-смот\-ре\-ния объекта~$\textbf{s}(t)$, 
временн$\acute{\mbox{о}}$го ряда, к~более общему случаю $\textbf{s}(\textbf{t})$, 
в~котором компоненты вектора~$\textbf{t}$~--- оси времени. Из-за 
существенного рос\-та вы\-чис\-ли\-тель\-ной слож\-ности при увеличении чис\-ла 
осей времени предлагается рас\-смот\-реть объекты $\textbf{s}(t_1, t_2)$, 
определенные на двух осях времени. Оси времени считаются независимыми. 
В~случае единственной дискретной и~ограниченной сверху шкалы времени 
объект представим вектором фиксированной размерности. 
Аналогично объект настоящего исследования представим мат\-ри\-цей.

Вводятся ограничения на зависимости осей времени в~декартовом 
произведении для таких объектов. Определена гипотеза порождения данных: 
объекты одного класса эквивалентности получены при помощи допустимых 
преобразований, а~именно: локальных деформаций (растяжений и~сжатий) 
каждой из осей времени по отдельности. В~дискретном случае преобразование 
представимо дуп\-ли\-ци\-ро\-ва\-ни\-ем строк и~столбцов матриц. 
В~число допустимых преобразований попадают и~глобальные деформации: 
сдвиги по осям времени, представимые добавлением и~удалением крайних 
строк и~столбцов исходных матриц. Для каждой из осей времени выполняются 
свойства времени: монотонность и~непрерывность. Похожими на описанные 
свойствами обладает, например, частотный спектр сигнала, где одна ось 
определяет время, а другая~--- частоту, величину, обратную времени.


Между двумя объектами, матрицами, в~случае допустимых преобразований 
требуется определить инвариантную к~преобразованиям осей времени функцию 
расстояния, которая сможет выделить классы эквивалентности множества 
преобразованных объектов. Работа посвящена определению такой функции 
расстояния, как обобщения метода динамического выравнивания временных рядов 
DTW для матриц.

Цель данной работы~--- построение метода, основанного на динамическом 
выравнивании осей времени для матриц. Метод динамического выравнивания 
временн$\acute{\mbox{ы}}$х рядов~\cite{33} определен только для объектов с~одной осью времени, 
что делает его неприменимым для описанного случая. Однако концепции, 
используемые на каждой стадии вы\-чис\-ле\-ния оптимального выравнивания, обобщены 
на рассматриваемый случай. Работа исследует свойства предложенного 
метода и~сравнивает результаты применения метода к~задачам классификации 
изображений~\cite{12} с~результатами функции расстояния~$L_2$.

Для иллюстрации и~анализа результатов решается задача метрической 
классификации объектов (матриц низкой размерности). Используются наборы данных: 
модельные данные, которые согласуются с~выдвинутой гипотезой порождения 
данных для временн$\acute{\mbox{ы}}$х рядов, подмножество набора MNIST сниженной 
размерности и~частотный спектр сигнала.

\vspace*{-10pt}

\section{Постановка задачи построения функции расстояния}

\vspace*{-2pt}

Рассмотрим задачу построения функции расстояния между объектами. 
Функция расстояния инвариантна к~допустимым преобразованиям осей времени: 
глобальным и~локальным линейным и~нелинейным деформациям временн$\acute{\mbox{о}}$й шкалы. 
Ниже приведены две постановки задачи, с~помощью которых определены свойства 
предложенной функции расстояния, оценено ее качество и~проведено сравнение 
нескольких функций расстояния: предложенной и~$L_2$.

Первая постановка задачи использует общее свойство функций расстояния: 
объединение схожих объектов и~разделение непохожих объектов. 
Вводится определение свойства инвариантности функции расстояния к~допустимым 
преобразованиям осей времени.
Вторая постановка задачи уточняет первую и~заключается в~проведении метрической 
классификации методом ближайшего соседа.

\textbf{Постановка задачи выбора функции расстояния между двумя объектами.}
На двух временн$\acute{\mbox{ы}}$х осях заданы объекты вида 
$\textbf{A}(t_1,t_2)\hm \in \mathbb{R}^{n \times n}$. 
Функция $G_w(\textbf{A}):\mathbb{R}^{n \times n} \hm\rightarrow 
\mathbb{R}^{\hat{n} \times \hat{n}}$ задает допустимые преобразования 
исходного объекта~$\textbf{A}$: глобальные сдвиги, локальные линейные 
и~нелинейные деформации, а~именно: растяжения и~сжатия оси времени, 
сдвиги значений по оси времени. Скалярный параметр $w \hm\in \mathbb{R}^+$
 функции~$G$ фиксирует набор этих преобразований.

Допустимым элементарным преобразованием матрицы~$\textbf{A}$ назовем 
дуплицирование случайных строк и~столбцов исходной матрицы, добавление 
или удаление крайних строк и~столбцов. Допустимым преобразованием 
примем некоторую последовательность допустимых элементарных 
преобразований матрицы~$\textbf{A}$ и~обозначим как~$G_w(\textbf{A})$.

Будем называть объект~$\textbf{B} \hm\in \mathbb{R}^{\hat{n} \times \hat{n}}$ 
полученным из объекта~$\textbf{A}$ при помощи допустимых 
преобразований~$G_{\hat{w}}$, если существует $\hat{w}\hm\in \mathbb{R}^+ : 
\textbf{B} \hm= G_{\hat{w}}(\textbf{A})$.

Функцию расстояния между двумя объектами $\rho: 
\mathbb{R}^{{n} \times {n}} \times \mathbb{R}^{\hat{n} \times \hat{n}} 
\hm\rightarrow  \mathbb{R}^+$ оценим на выборке $\mathfrak{D } \hm= 
\{ \textbf{A}_i \}_{i=1}^m$ объектов вида $\textbf{A}_i \hm\in 
\mathbb{R}^{n \times n}$.

Для каждого объекта выборки~$\textbf{A}_i$ и~объекта~$\textbf{B}_j$ его 
класса эквивалентности $\{\textbf{B}_j\}_i \hm= \{  \textbf{B} 
\hm\in \mathfrak{D} | \exists w_i,w_j: G_{w_i}(\textbf{A}_i) \hm= G_{w_j}
(\textbf{B}_j)   \}$ заданы допустимые трансформации с~параметрами~$w_i$ 
и~$w_j$, такие что $G_{w_i}(\textbf{A}_i)\hm = G_{w_j}(\textbf{B}_j)$. 
Для каждого объекта выборки~$\textbf{A}_i$ и~объекта~$\textbf{C}_j$ 
из других классов эквивалентности $\{ \textbf{C}_k\}_i \hm= 
\{  \textbf{C} \hm\in \mathfrak{D} | \nexists w_i,w_k: G_{w_i}(\textbf{A}_i)
\hm = G_{w_k}(\textbf{C})   \}$ не существует таких $ w_i, w_k : G_{w_i}
(\textbf{A}_i) \hm= G_{w_k}(\textbf{C}_k)$.

Решается задача поиска функции расстояния~$\rho$, значение
 которой на паре объектов одного класса эквивалентности меньше, 
 чем на любой паре объектов из разных: для любых $i,j,k \hm\in 
 \{1,\dots,m\}$ $\quad \rho(\textbf{A}_i,\textbf{B}_j) \hm< 
 \rho(\textbf{A},\textbf{C}_k)$. Функцию расстояния, обладающую 
 таким свойством, назовем инвариантной на классах эквивалентности.

Критерием качества для функции расстояния~$\rho$ на выборке~$\mathfrak{D}$ 
примем долю объектов, для которых указанное неравенство выполняется:
$$
S_{\rho}(\mathfrak{D}) = \fr{1}{m} \sum\limits_{i=1}^m 
\prod\limits_{\{ \textbf{B}_j\}_i} 
\prod\limits_{\{ \textbf{C}_k\}_i}  
\left[  \rho(\textbf{A}_i,\textbf{B}_j) < \rho(\textbf{A}_i,\textbf{C}_k)  
 \right].
 $$
Постановка задачи выбора функции расстояния~$\rho$ 
сводится к~задаче максимизации критерия качества.

\textbf{Прикладное использование функции расстояния.}
Задана выборка $\mathfrak{D}\hm = \{(\textbf{A}_i,y_i)\}^m_{i=1}$, 
состоящая из пар объ\-ект--от\-вет. Объектами служат объекты сложной 
структуры: $\textbf{A}_i\hm \in \mathbb{R}^{n\times n}$, 
а~ответами выступают метки класса~---~$y_i\hm \in Y \hm= \{1,\ldots,E\}$, 
где $E \hm\ll m$. Выборка разделена на обучение $\mathfrak{D}_l \hm= 
\{(\textbf{A}_i,y_i)\}^{m_1}_{i=1}$ и~контроль $\mathfrak{D}_t \hm= 
\{(\textbf{A}_i,y_i)\}_{m_1}^{m_1+m_2}$.

Модель классификации~$f$ принадлежит множеству моделей метрической 
классификации 1NN, которые классифицируемому объекту ставят 
в~соответствие метку класса ближайшего объекта из обучающей 
выборки по заданной функции расстояния~$\rho$:
$$ 
\hat{y} = f(\textbf{B} | \rho) = y \argmin\limits_{i = 1,\dots, m_1} 
\rho\left(B,A_i\right)\,.
$$
Критерий качества $S$ модели~$f$ для задачи классификации~--- 
доля правильно проставленного класса на контрольной выборке:
 $$ 
 S(f | \rho) = \fr{1}{m_2}\sum\limits_{i=m_1}^{m_1+m_2} 
 \left[f(\textbf{A}_i | \rho) = y_i\right].
 $$

Требуется выбрать функцию расстояния~$\rho$ для модели 
классификации~$f:~\mathbb{R}^{n\times n} \hm\rightarrow~Y$, 
максимизируюшую критерий качества~$S$ на контрольной выборке:
\begin{equation*}
f =  \argmax\limits_{\rho \in \{\mathrm{mDTW}, L_2\}}\left(S(f | \rho)\right).
\end{equation*}

\section{Вычисление матричного расстояния mDTW}

Предлагается использовать функцию расстояния DTW, 
модифицированную для случая выравнивания двойной шкалы времени.

\smallskip

\noindent
\textbf{Определение~1.} {Даны два объекта~$\textbf{A},\textbf{B}\hm \in 
\mathbb{R}^{n\times n}$. Тензор 
невязок~$\boldsymbol{\Omega}^{n \times n \times n \times n}$~--- 
такой тензор, что его элемент~$\boldsymbol{\Omega}(i,j,k,l)$ 
равен квадрату разности между элементами~$\textbf{A}(i,j)$ и~$\textbf{B}(k,l)$:}
\begin{equation*}
\boldsymbol{\Omega}(i,j,k,l)=(\textbf{A}(i,j) - \textbf{B}(k,l))^2.
\end{equation*}

\noindent
\textbf{Определение 2.} {Путем~$\boldsymbol{\pi}$ между двумя 
объектами $\textbf{A},\textbf{B} \hm\in \mathbb{R}^{n\times n}$ 
назовем множество индексов тензора~$\boldsymbol{\Omega}$: }
$$
\boldsymbol{\pi} = \{(i,j,k,l)\},\quad i,j,k,l \in \{1,\ldots,n\} ,
$$
\textit{удовлетворяющее следующим условиям:}

{\bfseries\textit{Частичный порядок.}}
Для элементов пути~$\boldsymbol{\pi}$ с~фиксированными значениями~$i,k$ 
задан порядок: выравнивающий путь для фиксированных строк двух 
матриц упорядочен~--- $\{(i,j_r,k,l_r))\}_{r=1}^{R} \hm\subset 
\boldsymbol{\pi}$ мощностью~$R$. Аналогично для фиксированных столбцов 
с~индексами~$j,l$.

{\bfseries\textit{Граничные условия.}}
 Пусть $(i,j,k,l) \in \boldsymbol{\pi}$, тогда $(1,j,1,l) \hm\in 
 \boldsymbol{\pi}$ и~$(i,1,k,1) \hm\in \boldsymbol{\pi}$.
Путь $\boldsymbol{\pi}$ содержит элементы тензора~$\boldsymbol{\Omega}$: 
$(1,1,1,1) \hm\in \boldsymbol{\pi}$ и~$(n,n,n,n) \hm\in \boldsymbol{\pi}$.

{\bfseries\textit{Непрерывность по направлению.}}
Для упорядоченного подмножества пути $\{(i,j_r,k,l_r)\}_{r=1}^{R}
\hm\subset\boldsymbol{\pi}$ выполняется условие непрерывности:
$$
j_{r}-j_{r-1}\leq1\,,\quad l_r-l_{r-1}\leq1\,, \quad r = 2,\ldots,R\,.
$$
На~шаге пути~$\boldsymbol{\pi}$ по фиксированному направлению времени~$i,k$ 
встречаются только соседние элементы матрицы (включая соседние по диагонали). 
Аналогично для фиксированных~$j,l$.

{\bfseries\textit{Монотонность по направлению.}}
Для упорядоченного подмножества пути  $\{(i,j_r,k,l_r)\}_{r=1}^{R}
\hm\subset\boldsymbol{\pi}$ выполняется хотя бы одно из условий 
монотонности функции выравнивания времени: 
$$
j_{r}-j_{r-1}\geq1\,,\quad l_r-l_{r-1}\geq1\,, \quad r = 2,\ldots,R\,.
$$

Свойства пути между матрицами обобщают свойства пути между двумя 
временными рядами.

\smallskip

\noindent
\textbf{Определение~3.}\ {Стоимость 
$\mathrm{Cost}\,(\textbf{A},\textbf{B},{\boldsymbol{\pi}})$ пути $\boldsymbol{\pi}$ 
между объектами $\textbf{A}, \textbf{B}$:
\begin{equation*}
\mathrm{Cost}\,(\textbf{A},\textbf{B},{\boldsymbol{\pi}}) = 
\sum\limits_{(i,j,k,l) \in \boldsymbol{\pi}}{\boldsymbol{\Omega}}(i,j,k,l).
\end{equation*}}

\noindent
\textbf{Определение~4.}\ 
{Выравнивающий путь~$\hat{\boldsymbol{\pi}}$ между 
объектами $\textbf{A},\textbf{B}$~--- путь наименьшей стоимости 
среди всех возможных путей между объектами:
\begin{equation*}
\hat{\boldsymbol{\pi}} = 
\argmin\limits_{{\boldsymbol{\pi}}} \mathrm{Cost}
\left(\textbf{A},\textbf{B},{\boldsymbol{\pi}}\right).
\end{equation*}}
Функция расстояния~$\rho (\textbf{A},\textbf{B})\hm = \mathrm{mDTW}\,
(\textbf{A},\textbf{B})$ между объектами~$\textbf{A}$ и~$\textbf{B}$ 
рассчитывается как стоимость выравнивающего пути~$\hat{\boldsymbol{\pi}}$:
\begin{equation}
\mathrm{mDTW}(\textbf{A},\textbf{B}) = \mathrm{Cost}\left(\textbf{A},
\textbf{B},\hat{\boldsymbol{\pi}}\right).
\end{equation}

\setcounter{figure}{1}
\begin{figure*}[b] %fig2
{\small 
\begin{center}
\begin{tabular}{l}
\hline
DTW(\textbf{s},\textbf{c}):\\
\hspace*{3mm}$\boldsymbol{D}$(1:n+1,1:m+1) = inf;\\
\hspace*{3mm}$\boldsymbol{D}$(1,1) = 0;\\
\hspace*{3mm}for $i = 2$: $n+1$\\
\hspace*{6mm}for $j = 2$ : $m+1$\\
\hspace*{9mm}$d = (\textbf{s}(i-1)-\textbf{c}(j-1))^2$;\\
\hspace*{9mm}$\boldsymbol{D}(i,j) = d + \min( 
[ \boldsymbol{D}(i-1,j), \boldsymbol{D}(i,j-1), \boldsymbol{D}(i-1,j-1) ])$;\\
return\ sqrt$(\boldsymbol{D}(n+1,m+1))$\\
\hline
\end{tabular}
\end{center}}
\vspace*{-9pt}

\Caption{Алгоритм вычисления DTW для временных рядов
\label{ris:dtwts}}
%\end{figure*}
%\begin{figure*} %fig3
\vspace*{6pt}
{\small 
\begin{center}
\begin{tabular}{l}
\hline
\\[-9pt]
Correction $(\overline{i,j,k,l}, \boldsymbol{\pi}(\overline{i,j,k,l})):$\\
\hspace*{3mm}if $\overline{i,j,k,l} \in \{ (i-1, j, k,l)  ;  
(i, j, k-1, l)  ;  (i-1, j, k-1, l) \}$:\\
\hspace*{6mm}$ \widehat{\pi} = \{ (\overline{i}, r, \overline{k}, f) \in 
\boldsymbol{\pi}(\overline{i, j, k, l}) \vert r, f \in \mathbb{N} \}$\\
\hspace*{3mm}elif $\overline{i,j,k,l}\in \{  
(i, j-1, k, l); (i, j, k, l-1); (i, j-1, k, l-1) \}$:\\
\hspace*{6mm}$\widehat{\pi} = \{ (r, \overline{j}, f, \overline{l}) 
\in \boldsymbol{\pi}(\overline{i, j, k, l}) \vert r, f \in \mathbb{N} \}$\\
\hspace*{3mm}elif $\overline{i,j,k,l} =  i-1,j-1,k-1,l-1:$\\
\hspace*{6mm}$\widehat{\pi} = \{ (\overline{i}, r, \overline{k}, f) 
\in \boldsymbol{\pi}(\overline{i, j, k, l}) \vert r,f \in \mathbb{N} \} \cup$\\
\hspace*{6mm}$\cup \{ (r, \overline{j}, f, \overline{l}) \in \boldsymbol{\pi}
(\overline{i, j, k, l}) \vert r,f \in \mathbb{N} \}$\\
\hspace*{3mm}$\boldsymbol{d\pi} = \{ \mathrm{element} \in \widehat{\pi}: 
\mbox{произведены\ замены\ индексов } 
\overline{i} = i,\ \overline{j} = j,\ \overline{k} = k,\ \overline{l} = l \}$\\
return $\boldsymbol{d\pi}$\\
\hline
\end{tabular}
\end{center}
}
\vspace*{-9pt}

\Caption{Алгоритм вычисления поправки $\boldsymbol{d\pi}$ 
пути $\boldsymbol{\pi}$
\label{ris:codedpi}}
\end{figure*}


\textbf{Алгоритм вычисления значения расстояния~(4).}
Построение алгоритма вычисления значения функции расстояния 
между матрицами основан на алгоритме расчета функции расстояния 
между временн$\acute{\mbox{ы}}$ми рядами. В~случае выравнивания одной\linebreak\vspace*{-12pt}

{ \begin{center}  %fig1
 \vspace*{-3pt}
    \mbox{%
 \epsfxsize=79mm 
 \epsfbox{gon-1.eps}
 }


\end{center}


\noindent
{{\figurename~1}\ \ \small{Матрица стоимости оптимального выравнивания, по обеим 
осям отложены временные отсчеты}}
}

\vspace*{12pt}


\noindent 
временн$\acute{\mbox{о}}$й шкалы
 итоговая матрица расстояний~$\boldsymbol{D}$ (рис.~1) в~каждом 
 элементе~$\boldsymbol{D}(i,j)$ содержит рас\-сто\-яние между подрядом 
 первого временн$\acute{\mbox{о}}$го ряда и~подрядом второго временн$\acute{\mbox{о}}$го ряда. 
 Рас\-смот\-рим алгоритм динамического выравнивания двух временн$\acute{\mbox{ы}}$х 
 рядов $\textbf{s} \hm\in R^n$ и~$\textbf{c} \hm\in R^m$ на рис.~2.
 
 

Элемент $\boldsymbol{D}(i,j)$ матрицы~$\boldsymbol{D}$ соответствует 
стоимости выравнивающего пути между подпоследовательностями 
исходных временн$\acute{\mbox{ы}}$х рядов: $\textbf{s}(1:i) \hm= \textbf{s}(t)$, 
$t \hm= 1,\ldots,i,$ и~$\textbf{c}(1:j) \hm= \textbf{c}(t)$, $t \hm= 1,\ldots,j$. 
Алгоритм построения наилучшего выравнивания времени 
подразумевает, что выравнивающий путь между этими 
подпоследовательностями получен одним из трех способов~--- 
если стоимость выравнивающего пути между 
подпоследовательностями~$\textbf{s}(1:\overline{i}) $ 
и~$\textbf{c}(1:\overline{j})$ минимальна для~$\overline{i,j}$ из множества
$$
\overline{i,j} \in \left\{ \{i-1,j\},\{i,j-1\},\{i-1,j-1\} \right\},$$
тогда выравнивающий путь между $\textbf{s}(1:i)$ и~$\textbf{c}(1:j)$ получен добавлением пары~$(i,j)$ к~выбранному 
выравнивающему пути с~минимальной стоимостью из трех.



Предложенный алгоритм переносит эти рас\-суж\-де\-ния на случай 
выравнивания двух матриц~$\textbf{A}$ и~$\textbf{B}$. 
Элемент~$\boldsymbol{D}(i,j,k,l)$ четырехиндексного
 тензора расстояний~$\boldsymbol{D}$ соответствует стоимости выравнивающего 
 пути между $\textbf{A}(1:i,1:j) \hm= \textbf{A}(t_1,t_2)$, 
 $t_1 \hm= 1,\ldots, i$, $t_2 \hm= 1,\ldots, j,$ 
 и~$\textbf{B}(1:k,1:l) \hm= \textbf{B}(t_1,t_2)$, $t_1 \hm= 1,\ldots, k$,
 $t_2 \hm= 1,\ldots, l$. Выравнивающий путь между этими 
 подматрицами получен одним из семи способов~--- 
 если стоимость выравнивающего пути между 
 подматрицами $\textbf{A}(1:\overline{i},1:\overline{j})$ 
 и~$\textbf{B}(1:\overline{k},1:\overline{l})$ 
 минимальна для~$\overline{i,j,k,l}$ из множества
\begin{multline*} 
\overline{i,j,k,l} \in 
\left\{ \{i-1,j,k,l\},\{i,j-1,k,l\},\right.\\
\{i,j,k-1,l\},
\{i,j,k,l-1\}, \{i-1,j,k-1,l\},\\
\left.
\{i,j-1,k,l-1\},\{i-1,j-1,k-1,l-1\}\right\},
\end{multline*}

\setcounter{figure}{3}
\begin{figure*} %fig4
{\small 
\begin{center}
\begin{tabular}{l}
\hline
$\mathrm{mDTW}\left(\textbf{A},\textbf{B}\right):$\\
\hspace*{3mm}$\textbf{D}(1:n+1,1:n+1, 1:n+1, 1:n+1) = inf$;\\
\hspace*{3mm}$\textbf{D}(1,1,1,1) = 0;$\\
\hspace*{3mm}$\boldsymbol{\pi}(1,1,1,1) = ((1,1),(1,1))$\\
\hspace*{3mm}$for\ i,j,k,l  \in \mathbb{N}^{2 : n+1} \times 
\mathbb{N}^{2 : n+1} \times \mathbb{N}^{2 : n+1} \times \mathbb{N}^{2 : n+1}:$\\
\hspace*{6mm}$\overline{i,j,k,l} = \argmin($ [ \textbf{D}(i-1, j, k, l), 
\textbf{D}(i, j-1, k, l), \textbf{D}(i, j, k-1, l), 
\textbf{D}(i, j, k, l-1),    \\
\hspace*{9mm}$\textbf{D}(i-1, j, k-1, l), \textbf{D}(i, j-1, k, l-1), 
\textbf{D}(i-1, j-1, k-1, l-1) ])$;\\
\hspace*{3mm}$\boldsymbol{d \pi} = \mathrm{Correction}\,(\overline{i,j,k,l}, 
\boldsymbol{\pi}(\overline{i,j,k,l}))$\\
\hspace*{3mm}$\boldsymbol{\pi}(i, j, k, l) = \boldsymbol{d \pi} \cup 
\{(\overline{i,j,k,l})\}$\\
\hspace*{3mm}$\mathrm{cost} = (\textbf{A}(i, j)-\textbf{B}(k, l))^2 + 
\sum\nolimits_{(r,f,t,g) \in \boldsymbol{d \pi}}
(\textbf{A}(r, f)-\textbf{B}(t, g))^2$;\\
\hspace*{3mm}$\textbf{D}(i,j,k,l) = \mathrm{cost} + \textbf{D}
(\overline{i,j,k,l})$\\
return  sqrt$(\textbf{D}(n+1,n+1,n+1,n+1))$\\
\hline
\end{tabular}
\end{center}
}
\vspace*{-9pt}

\Caption{Алгоритм вычисления расстояния между матрицами
\label{ris:matrixdtw}}
\end{figure*}

\begin{table*}[b]\small
\begin{center}
\begin{tabular}{|l|c|c|c|c|}
\multicolumn{5}{c}{Снижение расстояний при выполнении преобразований 
для различных наборов данных}\\
\multicolumn{5}{c}{\ }\\[-6pt]
\hline
 &\multicolumn{4}{c|}{Метод}\\
 \cline{2-5}
\multicolumn{1}{|c|}{Данные}  & \multicolumn{2}{c|}{$L_2$} & \multicolumn{2}{c|}{MatrixDTW} \\
\cline{2-5}
& $S(f|p)$  &  $S_{\rho}(\mathfrak{D})$ &  $S(f|p)$ & $S_{\rho}(\mathfrak{D})$ \\
\hline
Модельные данные без преобразований& 92\% & 78\% & 100\%\hphantom{9} & 85\% \\
Модельные данные с~преобразованиями & 86\% & 65\% &  100\%\hphantom{9} & 82\% \\
Модельные данные с~преобразованиями и~шумом& 69\% & 61\% &  92\% & 78\% \\
MNIST без преобразований& 95\% & --- & 95\% & --- \\
MNIST с~преобразованиями & 53\% & --- & 92\% & --- \\
Спектр сигнала& 83\% & --- & 96\% & --- \\
\hline
\end{tabular}
\end{center}
\end{table*}

\noindent
то к~выравнивающему пути между этими под\-мат\-ри\-ца\-ми 
добавляется элемент пути $(i,j,k,l)$ и~поправка~$\boldsymbol{d\pi} $ 
пути~$\boldsymbol{\pi}$, алгоритм вычисления которой приведен ниже.

Обозначим выравнивающий путь между $\textbf{A}(1:i,\linebreak 1:j)$
 и~$\textbf{B}(1:k,1:l)$ как~$\boldsymbol{\pi}(i,j,k,l)$, тогда 
 поправка~$\boldsymbol{d\pi} $ пути~$\boldsymbol{\pi}(i,j,k,l)$ 
 при фиксированных~$\overline{i,j,k,l}$ вычисляется приведенным на рис.~3 
 образом.





Алгоритм динамического выравнивания двух матриц и~вычисления 
расстояния $\mathrm{mDTW}$ между ними с~учетом приведенного выше 
алгоритма примет вид, представленный на рис.~4.





\begin{figure*} %fig5
\vspace*{1pt}
    \begin{center}  
  \mbox{%
 \epsfxsize=161.412mm 
 \epsfbox{gon-5.eps}
 }
\end{center}
\vspace*{-12.5pt}
\Caption{Выравнивание модельных данных: (\textit{а})~один класс без шума; 
(\textit{б})~разные классы без шума; 
(\textit{в})~один класс с~шумом; (\textit{г})~разные классы с~шумом
\label{ris:random}}
%\end{figure*}
%\begin{figure*} %fig6
\vspace*{1pt}
    \begin{center}  
  \mbox{%
 \epsfxsize=163mm 
 \epsfbox{gon-6.eps}
 }
\end{center}
\vspace*{-12.5pt}
\Caption{Выравнивание данных MNIST: левый столбец~--- один класс; 
правый столбец~--- разные 
классы;
(\textit{а})~$\mathrm{mDTW}\hm=720{,}1$; 
(\textit{б})~948,6;
(\textit{в})~2017,0;
(\textit{г})~$\mathrm{mDTW}\hm=2071{,}4$
\label{ris:mnist}}
\end{figure*}


Следует отметить, что алгоритм~\cite{15} имеет\linebreak высокую сложность 
вычисления~--- $O(n^4)$. Предполагается ускорение метода 
с~использованием ограниче\-ния Sakoe-Chiba band, что сократит 
вычислительную сложность алгоритма до $O(n^2k^2)$, где~$k$~--- 
параметр ограничения.


\section{Вычислительный эксперимент}

Вычислительный эксперимент проведен на модельных данных с~допустимыми 
преобразованиями и~на реальных данных: объектах коллекции MNIST с~допустимыми 
преобразованиями и~на спектрограммах зашумленных сигналов.





Решается задача метрической классификации методом ближайшего соседа. В~таблице 
приведены значения критерия качества функции расстояния 
$S_{\rho}(\mathfrak{D})$ и~критерия качества метрической классификации $S(f|p)$ 
при использовании двух функций расстояния: предложенной в~работе $\mathrm{mDTW}$ 
и~$L_2$.

Модельные данные~--- это нулевые матрицы со случайными ненулевыми 
строками, столбцами, подпрямоугольниками с~наложенным шумом. 
К~ним применены допустимые преобразования, согласованные с~гипотезой 
наличия локальных и~глобальных искажений. На рис.~\ref{ris:random} 
показан пример оптимального выравнивания двух объектов. 
Линиями показаны элементы пути~$\boldsymbol{\pi}$.

Подготовлена подвыборка набора данных MNIST. Она 
состоит из~100 объектов классов 0 и~1 сниженной размерности
 с~допустимыми преобразованиями. На рис.~\ref{ris:mnist} 
 показан пример оптимального выравнивания объектов.


Аналогичный эксперимент проведен для решения задачи метрической 
классификации спектров различных сигналов, пример которых приведен на 
рис.~\ref{ris:spectr}. На рисунке показаны примеры Фурье-спект\-ров 
этих сигналов. Спектр получен путем применения быстрого преобразования 
Фурье к~исходному сигналу для различных окон с~фиксированным размером и~сдвигом. 
Исходные временн$\acute{\mbox{ы}}$е ряды обладали свойством периодичности, период выбирался 
случайным образом.



Тестирование проведено на разного рода данных: исходных 
модельных данных без наложения\linebreak\vspace*{-12pt}

\pagebreak

\end{multicols}

\begin{figure*} %fig7
\vspace*{1pt}
    \begin{center}  
  \mbox{%
 \epsfxsize=149.062mm 
 \epsfbox{gon-7.eps}
 }
\end{center}
\vspace*{-8pt}
\Caption{Данные спектров сигнала: (\textit{а})~класс~1; (\textit{б})~спектр 
класса~1; (\textit{в})~класс~2; (\textit{г})~спектр класса~2; 
(\textit{д})~класс~3; (\textit{е})~спектр класса~3
\label{ris:spectr}}
\vspace*{9pt}
\end{figure*}

\begin{multicols}{2}

\noindent допустимых преобразований, с~ними, а~также 
на модельных данных с~наложенным поверх объектов случайным шумом.



В каждом из проведенных экспериментов была продемонстрирована 
устойчивость предложенного подхода к~допустимым преобразованиям. 
Наилучшее значение критерия качества задачи классификации было 
достигнуто при использовании предложенной функции расстояния.

\vspace*{-5pt}

\section{Заключение}

В работе предложено обобщение метода динамического выравнивания
 временн$\acute{\mbox{ы}}$х рядов для случая объектов, определенных на двух осях времени. 
 Существует теоретическое обобщение предлагаемых методов на случай 
 конечного множества осей времени. Вычислительный эксперимент позволил 
 проанализировать свойства подхода: устойчивость к~допустимым 
 преобразованиям и~разделяющая способность функции расстояния как 
 на реальных, так и~на модельных данных. Качество решения задачи 
 метрической классификации выше решения, основанного на евклидовом 
 расстоянии. Вычислительная сложность метода высокая, что ограничивает 
 его применимость на объектах высокой размерности.

\vspace*{-2pt}

{\small\frenchspacing
 {%\baselineskip=10.8pt
 \addcontentsline{toc}{section}{References}
 \begin{thebibliography}{99}
%\bibitem{Karasikov2016}
%\Au{Карасиков~М.\,Е., Стрижов~В.\,В.} Классификация временных рядов 
%в~пространстве параметров по\-рож\-да\-ющих моделей~// Информатика и~её 
%применения,~2016. T.~10. Вып.~4. С.~121--131.

\bibitem{0}
\Au{Hill~N.\,J., Lal~T.\,N., Schroder~M., Hinterberger~T., 
Wilhelm~B., Nijboer~F., Mochty~U., Widman~G., Elger~C., 
Scholkopf~B., Kubler~A., Birbaumer~N.} Classifying EEG and 
ECoG signals without subject training for fast BCI implementation: 
Comparison of nonparalyzed and completely paralyzed subjects~//  
IEEE~T. Neur. Sys. Reh., 2006. Vol.~14. 
Iss.~2. P.~183--186.

\bibitem{1}
\Au{Sakoe~H., Chiba~S.} 
A~dynamic programming approach to continuous speech recognition~// 
7th  Congress (International) on Acoustics Proceedings, 1971. Vol.~3. P.~65--69.

\bibitem{2} %3
\Au{Aghabozorgi~S., Ali~S.\,S., Wah~T.\,Y.} 
Time-series clustering~--- a~decade review~// Inform. Syst., 
2015. Vol.~53. P.~16--38.

\bibitem{3} %4
\Au{Warrenliao~T.} Clustering of time series data~--- a~survey~// 
Pattern Recogn., 2005. Vol.~38. Iss.~11. P.~1857--1874.



\bibitem{4} %5
\Au{Hautamaki~V., Nykanen~P., Franti~P.} 
Time-series clustering by approximate prototypes~// 
19th  Conference (International) on Pattern Recognition Proceedings, 2008. No.\,D. 
P.~1--4.

\bibitem{5} %6
\Au{Faloutsos~C., Ranganathan~M., Manolopoulos~Y.} 
Fast subsequence matching in time-series databases~// \mbox{SIGMOD} Rec., 1994. 
Vol.~23. Iss.~2. P.~419--429.

\bibitem{10} %7
\Au{Basalto~N., Bellotti~R., Carlo~F.\,D., Facchi~P., 
Pascazio~S.} Hausdorff clustering of financial time series~// 
Physica~A, 2007. Vol.~379. Iss.~2. P.~635--644.

\bibitem{11} %8
\Au{Gorelick~L., Blank~M., Shechtman~E., Irani~M., Basri~R.} 
Actions as space-time shapes~// IEEE~T. Pattern Anal., 
2007. Vol.~29. Iss.~12. P.~2247--2253.

\bibitem{6} %9
\Au{Smyth~P.} Clustering sequences with hidden Markov models~// 
Adv. Neural In., 1997. Vol.~9. P.~648--654.

\bibitem{7} %10
\Au{Banerjee~A., Ghosh~J.} Clickstream clustering using weighted 
longest common subsequences~// 
Workshop on Web Mining, SIAM Conference on Data Mining
Proceedings, 2001. P.~33--40.

\bibitem{8} %11
\Au{Aach~J., Church~G.M.} Aligning gene expression time series
 with time warping algorithms~// Bioinformatics, 2001. Vol.~17. Iss.~6. P.~495--508.

\bibitem{9} %12
\Au{Yi~B.\,K., Faloutsos~C.} Fast time sequence indexing 
for arbitrary $\mathcal{L}_p$ norms~// 
26th  Conference (International) on Very Large Data Bases Proceedings, 2000. P.~385--394.

\bibitem{33} %13
\Au{Goncharov~A.\,V., Strijov~V.\,V.} 
Analysis of dissimilarity set between time series~// Computational 
Mathematics Modeling, 2018. Vol.~29. Iss.~3. P.~359--366.

\bibitem{12} %14
\Au{Alon~J., Athitsos~V., Sclaroff~S.}
 Online and offline character recognition using alignment to prototypes~// 
 8th  Conference (International) on Document Analysis and Recognition, 2005. 
 Vol.~2. P.~839--843.

\bibitem{15} %15
\Au{Гончаров~А.\,В.} 
Выравнивания декартовых произведений упорядоченных множеств mDTW. 
Про\-грам\-мная реализация алгоритма, 2019. 
{\sf https://github.
com/Intelligent-Systems-Phystech/PhDThesis/tree/\linebreak  master/Goncharov2019/MatrixDTW/code}.
 \end{thebibliography}

 }
 }

\end{multicols}

\vspace*{-9pt}

\hfill{\small\textit{Поступила в~редакцию 24.04.19}}

\vspace*{6pt}

%\pagebreak

%\newpage

%\vspace*{-28pt}

\hrule

\vspace*{2pt}

\hrule

\vspace*{-4pt}

\def\tit{ALIGNMENT OF~ORDERED SET CARTESIAN PRODUCT\\[-5pt]}


\def\titkol{Alignment of~ordered set cartesian product}

\def\aut{A.\,V.~Goncharov$^1$ and~V.\,V.~Strijov$^{1,2}$}

\def\autkol{A.\,V.~Goncharov and~V.\,V.~Strijov}

\titel{\tit}{\aut}{\autkol}{\titkol}

\vspace*{-13pt}


\noindent
$^1$ Moscow Institute of Physics and Technology, 
9~Institutskiy Per., Dolgoprudny, Moscow Region 141700, Russian\linebreak
$\hphantom{^1}$Federation


\noindent
$^2$A.\,A.~Dorodnicyn Computing Center, Federal Research Center 
``Computer Science and Control'' of the Russian\linebreak
$\hphantom{^1}$Academy of Sciences, 
40~Vavilov Str., Moscow 119333, Russian Federation

\def\leftfootline{\small{\textbf{\thepage}
\hfill INFORMATIKA I EE PRIMENENIYA~--- INFORMATICS AND
APPLICATIONS\ \ \ 2020\ \ \ volume~14\ \ \ issue\ 1}
}%
 \def\rightfootline{\small{INFORMATIKA I EE PRIMENENIYA~---
INFORMATICS AND APPLICATIONS\ \ \ 2020\ \ \ volume~14\ \ \ issue\ 1
\hfill \textbf{\thepage}}}

\vspace*{2pt} 



\Abste{The work is devoted to the study of metric methods for analyzing 
objects with complex structure. It proposes to generalize the dynamic 
time warping method of two time series for the case of objects defined 
on two or more time axes. Such objects are matrices in the discrete 
representation. The DTW (Dynamic Time Warping) method of time series is generalized as 
a~method of matrices dynamic alignment. The paper proposes 
a~distance function resistant to monotonic nonlinear deformations of the 
Cartesian product of two time scales. The alignment path between objects is 
defined. An object is called a~matrix in which the rows and columns correspond 
to the axes of time. The properties of the proposed distance function 
are investigated. To illustrate the method, the problems of metric 
classification of objects are solved on model data and data from the 
MNIST dataset.}

\KWE{distance function; dynamic alignment; distance between matrices; 
nonlinear time warping; space--time series}



\DOI{10.14357/19922264200105} 

%\vspace*{-14pt}

\Ack
\noindent
This work was supported by the Russian Foundation for Basic
Research (projects 19-07-1155 and 19-07-00885). 
The paper contains results of the project Statistical 
methods of machine learning, which is carried out within the 
framework of the Program ``Center of Big Data Storage and Analysis'' 
of the National Technology Initiative Competence Center. 
It is supported by the Ministry of Science and Higher Education 
of the Russian Federation according to the agreement between the
 M.\,V.~Lomonosov Moscow State University and the Foundation 
 of project support of the National Technology Initiative from 11.12.2018, 
 No.\,13/1251/2018.
 


%\vspace*{6pt}

  \begin{multicols}{2}

\renewcommand{\bibname}{\protect\rmfamily References}
%\renewcommand{\bibname}{\large\protect\rm References}

{\small\frenchspacing
 {%\baselineskip=10.8pt
 \addcontentsline{toc}{section}{References}
 \begin{thebibliography}{99}

 \bibitem{0-1}   
\Aue{Hill, N.\,J., T.\,N.~Lal, M.~Schroder, T.~Hinterberger, B.~Wilhelm, 
F.~Nijboer, U.~Mochty, G.~Widman, C.~Elger, B.~Scholkopf, A.~Kubler, and 
N.~Birbaumer.} 2006. Classifying EEG and ECoG signals without subject 
training for fast BCI implementation: Comparison of nonparalyzed and completely 
paralyzed subjects. \textit{IEEE~T. Neur. Sys. 
Reh.} 14(2):183--186.

\bibitem{1-1}   
\Aue{Sakoe, H., and S.~Chiba.} 1971. A~dynamic programming approach 
to continuous speech recognition. \textit{7th 
 Congress (International) on Acoustics Proceedings}. 3:65--69.

\bibitem{2-1}    %2
\Aue{Aghabozorgi,~S., S.\,S.~Ali, and T.\,Y.~Wah.} 2015. 
Time-series clustering~--- a~decade review.  \textit{Inform. Syst.} 
53:16--38.

\bibitem{3-1}   %4 
\Aue{Warrenliao,~T.} 2005. Clustering of time series data~--- a~survey. 
\textit{Pattern Recogn.}
38(11):1857--1874.



\bibitem{4-1}    %5
\Aue{Hautamaki,~V., P.~Nykanen, and P.~Franti.} 2008. 
Time-series clustering by approximate prototypes. 
 \textit{19th  Conference (International) on Pattern Recognition Proceedings}. 
 D:1--4.

\bibitem{5-1}    %6
\Aue{Faloutsos,~C., M.~Ranganathan, and Y.~Manolopoulos.} 1994. 
Fast subsequence matching in time-series databases.  \textit{SIGMOD Rec}. 
23(2):419--429.

\bibitem{10-1}    %7
\Aue{Basalto, N., R.~Bellotti, F.\,D.~Carlo, P.~Facchi, and S.~Pascazio.} 
2007. Hausdorff clustering of financial time series. 
\textit{Physica~A} 379(2):635--644.

\bibitem{11-1}   %8
\Aue{Gorelick, L., M.~Blank, E.~Shechtman, M.~Irani, and R.~Basri.} 
2007. Actions as space-time shapes.
\textit{IEEE~T. Pattern Anal.} 29(12):2247--2253.

\bibitem{6-1}    %9
\Aue{Smyth, P.} 1997. 
Clustering sequences with hidden Markov models. \textit{Adv. Neural In.} 9:648--654.

\bibitem{7-1}    %10
\Aue{Banerjee,~A., and J.~Ghosh.} 2001. 
Clickstream clustering using weighted longest common subsequences.  
\textit{Workshop on Web Mining, SIAM Conference 
on Data Mining Proceedings.} 33--40.

\bibitem{8-1}    %11
\Aue{Aach, J., and G.\,M.~Church.} 2001. 
Aligning gene expression time series with time warping algorithms. 
\textit{Bioinformatics} 17(6):495--508.

\bibitem{9-1}   %12
\Aue{Yi, B.\,K., and C.~Faloutsos.} 2000. 
Fast time sequence indexing for arbitrary $\mathcal{L}_p$ norms. 
\textit{26th  Conference (International) 
on Very Large Data Bases Proceedings}. 385--394.

\bibitem{33-1}   %13 
\Aue{Goncharov,~A.\,V., and V.\,V.~Strijov.} 2018. 
Analysis of dissimilarity set between time series. 
\textit{Computational Mathematics Modeling } 29(3):359--366.



\bibitem{12-1}    %14
\Aue{Alon, J., V.~Athitsos, and S.~Sclaroff.} 2005.
 Online and offline character recognition using alignment to prototypes. 
 \textit{8th  Conference (International) on Document Analysis and Recognition}. 
 2:839--843.

\bibitem{15-1}    %15
\Aue{Goncharov, A.\,V.} Alignment of 
Ordered Set Cartesian Product mDTW. Software implementation of the algorithm. 
Available at: {\sf https://github.com/Intelligent-\linebreak 
Systems-Phystech/PhDThesis/tree/master/Goncharov\linebreak 2019/MatrixDTW/code} 
(accessed December~27, 2019).
\end{thebibliography}

 }
 }

\end{multicols}

%\vspace*{-7pt}

\hfill{\small\textit{Received April 24, 2019}}

%\pagebreak

%\vspace*{-22pt}



\Contr

\noindent
\textbf{Goncharov Alexey V.} (b.\ 1995)~--- 
PhD student, Moscow Institute of Physics and Technology, 
9~Institutskiy Per., Dolgoprudny, Moscow Region 141701, 
Russian Federation; \mbox{alex.goncharov@phystech.edu}

\vspace*{3pt}

\noindent
\textbf{Strijov Vadim V.} (b.\ 1967)~--- 
Doctor of Science in physics and mathematics, leading scientist, 
A.\,A.~Dorodnicyn Computing Centre, Federal Research Center 
``Computer Science and Control'' of the Russian Academy of Sciences, 
40~Vavilov Str., Moscow 119333, Russian Federation;
 professor, Moscow Institute of Physics and Technology, 
 9~Institutskiy Per., Dolgoprudny, Moscow Region 141701, Russian Federation; 
 \mbox{strijov@ccas.ru}
\label{end\stat}

\renewcommand{\bibname}{\protect\rm Литература} %6
\def\stat{abgaryan}

\def\tit{ПРОГРАММНЫЙ КОМПЛЕКС ДЛЯ~МНОГОМАСШТАБНОГО МОДЕЛИРОВАНИЯ 
СТРУКТУРНЫХ СВОЙСТВ КОМПОЗИЦИОННЫХ МАТЕРИАЛОВ$^*$}

\def\titkol{Программный комплекс для многомасштабного моделирования 
структурных свойств композиционных материалов}

\def\aut{К.\,К.~Абгарян~$^1$, Е.\,С.~Гаврилов$^2$}

\def\autkol{К.\,К.~Абгарян, Е.\,С.~Гаврилов}

\titel{\tit}{\aut}{\autkol}{\titkol}

\index{Абгарян К.\,К.}
\index{Гаврилов Е.\,С.}
\index{Abgaryan K.\,K.}
\index{Gavrilov E.\,S.}


{\renewcommand{\thefootnote}{\fnsymbol{footnote}} \footnotetext[1]
{Работа выполнена при поддержке Министерства науки и~высшего образования Российской Федерации (проект 
075-15-2020-799).}}


\renewcommand{\thefootnote}{\arabic{footnote}}
\footnotetext[1]{Федеральный исследовательский центр <<Информатика и~управление>> Российской академии наук, 
\mbox{kristal83@mail.ru}}
\footnotetext[2]{Федеральный исследовательский центр <<Информатика и~управление>> Российской академии наук; 
Московский авиационный институт (национальный исследовательский университет), \mbox{eugavrilov@gmail.com}}

%\vspace*{-6pt}
    
      
         
      
      \Abst{Создание новых композиционных материалов (КМ) с~прогнозируемыми свойствами 
      и~разработка способов их конструирования на сегодня стали одними из актуальных и~важнейших 
задач, связанных с~модернизацией промышленного производства в~нашей стране. Для их 
решения активно развиваются технологии многомасштабного компьютерного 
моделирования. Они стали связующим звеном между фундаментальной физикой (химией) 
и~инженерным материаловедением. В~работе представлен программный комплекс по 
моделированию структурных свойств КМ, поз\-во\-ля\-ющий решать ряд 
задач данного класса. Он ориентирован на высокопроизводительные вы\-чис\-ле\-ния. В~основе 
комплекса лежит оригинальная многомасштабная технология, которая позволяет оперативно 
проводить многовариантный анализ различных классов КМ 
и~проводить исследования по проектированию новых с~прогнозируемыми свойствами. 
Разработанные подходы в~сочетании с~экспериментальными данными могут быть использованы 
для лучшего понимания физических основ изменения свойств в~за\-ви\-си\-мости от структуры и,~как 
следствие, для удешевления и~ускорения поиска новых КМ
с~заданными свойствами.}
      
      \KW{многомасштабное моделирование; композиционные материалы; интеграционная 
платформа; программный комплекс; распределенная сис\-тема}

\DOI{10.14357/19922264220113}
  
%\vspace*{-3pt}


\vskip 10pt plus 9pt minus 6pt

\thispagestyle{headings}

\begin{multicols}{2}

\label{st\stat}

\section{Введение}

\vspace*{-3pt}

     Создание новых КМ с~прогнозируемыми 
свойствами и~разработка способов их конструирования на сегодня стали одними 
из актуальных и~важнейших задач по модернизации промышленного 
производства в~нашей стране. Особенно важны такие материалы в~областях, где 
соотношение между проч\-ностью и~массой конструкции определяет ее 
эф\-фек\-тив\-ность. На сегодня процессы создания КМ
непосредственно связаны с~этапом моделирования, включая применение наиболее 
эффективных методов многомасштабного компьютерного моделирования и~анализа данных. 
     
     Для решения данного класса задач разработан\linebreak программный комплекс по 
моделированию структурных свойств КМ. Он 
ориентирован на высокопроизводительные вы\-чис\-ле\-ния. В~осно\-ве комплекса 
лежит оригинальная многомасштабная \mbox{технология}, пред\-став\-лен\-ная в~[1, 2], 
которая позволяет оперативно проводить многовариантный анализ различных 
классов КМ. На базе разработанной технологии была 
создана распределенная информационная сис\-те\-ма для проведения 
многоуровневых исследований в~об\-ласти моделирования~КМ. 

Согласно разработанным подходам в~за\-ви\-си\-мости от типа 
мо\-де\-ли\-ру\-емо\-го КМ строится многомасштабная 
композиция и~ее схематическое представление. На ее основе в~программной среде 
формируется сценарий расчета структурных характеристик и~отдельных свойств 
рас\-смат\-ри\-ва\-емо\-го материала. Созданный программный комплекс позволяет 
автоматизировать уни\-фи\-ци\-ру\-емые этапы моделирования и~помогает 
сформировать на основе анализа полученных результатов более глубокое 
понимание физических процессов. Комплекс построен с~применением 
современных программных средств и~решений и~не уступает международному 
уровню на\-уч\-но-тех\-ни\-че\-ских разработок в~об\-ласти информационной 
поддержки для многомасштабного моделирования новых материалов. 
     
     Разработка такого средства информационной поддержки поз\-во\-ля\-ет 
обеспечить формирование информации для многопараметрического анализа 
структуры и~физических свойств различных классов су\-ще\-ст\-ву\-ющих 
КМ, рассмотреть большое чис\-ло вариантов 
в~на\-прав\-ле\-нии поиска новых материалов и,~таким образом, ускорить и~удешевить 
процесс подбора па\-ра\-мет\-ров получения материалов.  Ис-\linebreak\vspace*{-12pt}

\pagebreak

\noindent
пользование данного 
комплекса позволяет за ограниченное время строить гиб\-рид\-ные модели для 
обоснованного выбора КМ с~заданными свойствами для  
авиа\-ци\-он\-но-кос\-ми\-че\-ской и~других областей промышленности. 
     
     В связи с~тем что традиционные материалы (преимущественно металлы)
      не в~полной мере отвечают высоким фи\-зи\-ко-ме\-ха\-ни\-че\-ским, 
технологическим и~эксплуатационными свойствам, развитие производства 
современных надежных и~экономичных конструкций в~машиностроении 
основано на применении новых КМ. Под 
композиционными понимаются материалы, со\-сто\-ящие из двух или более 
физически различных компонент (фаз), возможные комбинации которых 
приводят к~появлению уникальных свойств, отличных от тех, которыми обладала 
каж\-дая из них отдельно. На сегодня для развития авиа\-ци\-он\-но-кос\-ми\-че\-ской 
отрасли, включая самолетостроение, вертолетостроение, ракетостроение, 
требуется постоянное увеличение доли полимерных КМ
с~набором заданных свойств. Современные летательные аппараты обладают 
слож\-ной конструкцией, со\-сто\-ящей из металлов и~неметаллических материалов. 
Применяются детали из алю\-ми\-ни\-евых и~сталь\-ных сплавов, коррозионностойких 
сталей, титановых сплавов и~полимерных КМ (стек\-ло-, 
угле-, органопластики и~др.). Для снижения веса и~продления срока службы 
летательных аппаратов при производстве деталей все шире применяют 
полимерные~КМ.
     
     Сегодня наиболее востребованные САЕ- (Computer-Aided Engineering) 
сис\-те\-мы, такие как ABAQUS ({\sf https://simulia.com}), \mbox{ANSYS} ({\sf 
https://\linebreak Ansys.com}), LMS Engineering innovation ({\sf https://\linebreak trademarks.justia.com}), 
Femap ({\sf https://www.cad-is.ru/femap}), MSC Software ({\sf 
http://www.mscsoftware.\linebreak ru}) включают в~себя базы данных со свойствами 
материалов. Для КМ мож\-но выбрать тип композита со 
стандартными свойствами (угле-, стекло-, органопластики на основе 
эпоксифенолформальдегидных, кремнийорганических смол, эпоксидные 
боропластики и~т.\,д.). Имеется возможность коррекции данных свойств 
и~внесения материала с~новыми свойствами в~базу данных. Следует также отметить 
российские разработки в~об\-ласти моделирования КМ, 
такие как пакет CAE-Fidesys ({\sf https://cae-fidesys.com}), программный пакет для 
моделирования полимерных материалов Multicomp ({\sf 
https://www.kintechlab.com/products}), Российский исследовательский 
и~ин\-же\-нер\-но-тех\-но\-ло\-ги\-че\-ский проект N1 Composites ({\sf 
http://n1composites.com}) и~др.
{\looseness=-1

}
     
     Программные комплексы позволяют задать\linebreak свойства материалов, из 
которых состоит КМ, такие как изотропность, 
ортотропность, анизотропность. Важная часть проектирования композиционных 
конструкций~--- преобразование модели,\linebreak созданной с~применением CAD 
(Computer-aided design, сис\-те\-мы автоматизированного проектирования) 
в~модель, пригодную для CAE-ана\-ли\-за (нетривиальная задача, тре\-бу\-ющая 
за\-час\-тую создания экспертной сис\-те\-мы). Следует отметить, что функционал всех 
мировых лидеров в~CAE-сег\-мен\-те схож. 
     %
     Так, функционал MSC позволяет встраивать разработанные пользователем 
модули в~программный комплекс (например, можно включить метод имитации 
процесса производства КМ).
     
     Помимо используемых ведущими CAE-сис\-те\-ма\-ми модулями существуют 
коммерческие сис\-те\-мы, позволяющие генерировать КМ на микроуровне, а~затем 
проводить чис\-лен\-ные эксперименты на макроуровне. К~таким сис\-те\-мам 
относятся модуль генерации и~моделирования механических характеристик 
КМ GeoDict ({\sf www.math2market.com}) с~различными типами КМ, 
ге\-не\-ри\-ру\-емы\-ми модулем GeoDict, и~программный комплекс COMSOL ({\sf 
www.comsol.ru}).
     
     В современных ведущих CAE-сис\-те\-мах учет мик\-ро\-струк\-ту\-ры 
КМ проводится после гомогенизации свойств материала 
или определения мак\-ро\-мас\-штаб\-ных свойств КМ. При этом, однако, теряются 
индивидуальные детали микроструктуры КМ~\cite{3-ab}. При определении макромасштабных свойств КМ обычно 
исходят из идеальных условий: оптимального формирования граничной 
поверхности, идеального распределения(отсутствия взаимодействия час\-тиц 
между собой) и~отсутствия влияния компонента на мат\-рицу.
     
     Однако результаты, которые на сегодня могут быть получены 
     с~использованием САЕ-систем для\linebreak воспроизведения характеристик известных 
структур, зачастую могут расходиться с~данными экспериментов~--- например, 
когда речь идет о~полимерных КМ с~на\-но\-вклю\-че\-ни\-ями 
(\mbox{нанотрубками}). \mbox{Известно} влияние до\-бав\-ле\-ния на\-но\-раз\-мер\-ных\linebreak час\-тиц 
наполнителя на изменение механических свойств КМ. 
В~литературе широко описано изменение коэффициента теп\-ло\-про\-вод\-ности 
полимерных\linebreak мат\-риц в~несколько раз при их наполнении 
нанотрубками, пред\-став\-ле\-ны тео\-ре\-ти\-че\-ские исследования с~аналогичными 
результатами~\cite{1-ab}. Использование CAE-сис\-те\-м не позволяет в~полной 
мере \mbox{оценить} фактор влияния на\-но\-час\-тиц на данные свойства. Кроме 
того, применение CAE-сис\-тем в~контексте многомасштабного моделирования 
затруднено жесткими ограничениями пакетных решений. В~настоящее время 
развиваются \mbox{системы} c~программным обеспечением для многомасштабного 
моделирования, такие как Computational Soft Materials (Comsoft) Workbench, 
поз\-во\-ля\-ющий моделировать КМ с~<<мягкой>> 
структурой (полимеры, полимерные композиты), программный пакет LAMMPS 
({\sf https://www.lammps.org}), ис\-поль\-зу\-емый для моделирования в~рамках 
классической молекулярной динамики на атомистическом и~мезомасштабном 
уровнях полимерных, металлических, биологических сис\-тем и~др. Каждый из 
разрабатываемых программных продуктов обладает своими достоинствами 
и~областями применения. В~связи с~большим разнообразием типов 
КМ и~все воз\-рас\-та\-ющи\-ми требованиями к~наборам 
свойств, которыми они должны обладать, пред\-став\-ля\-ет\-ся важ\-ным\linebreak создание 
программных средств, поз\-во\-ля\-ющих оперативно вы\-стра\-и\-вать сис\-тем\-ные 
решения в~об\-ласти\linebreak многомасштабного моделирования с~применением 
высокопроизводительных вычислений, поз\-во\-ля\-ющих проводить моделирование от  
атом\-но-крис\-тал\-ли\-че\-ско\-го до мак\-ро\-уров\-ня. Такие системы \mbox{позволят} 
генерировать и~выполнять в~автоматическом режиме сценарии проведения 
расчетов под конкретную задачу, включать в~вычислительную схему расчеты на 
всех необходимых мас\-штаб\-ных уровнях. Для предсказательного моделирования 
структурных свойств различных классов КМ такой 
подход поз\-во\-ля\-ет создавать вы\-чис\-ли\-тель\-ную среду, в~которой задействованы 
возможности CАE-сис\-тем для верх\-не\-уров\-не\-во\-го (мак\-ро-) моделирования, 
методы анализа экспериментальных и~аналитических данных, а также 
собственные разработки и~пакетные приложения для расчетов на атом\-но-крис\-тал\-ли\-че\-ском и~наноуровне.

\vspace*{-9pt}

\section{Многомасштабная модель для~расчета структурных 
свойств композиционных материалов}

     В работе~\cite{2-ab} представлена общая схема многомасштабной модели 
для расчета структурных характеристик КМ. Для ее 
описания используется тео\-ре\-ти\-ко-мно\-жест\-вен\-ный аппарат, изложенный 
в~\cite{1-ab, 2-ab}. На ее основе формируются схемы для расчета разных классов 
КМ: нанокомпозитов на основе полимерной мат\-ри\-цы, 
КМ с~металлической мат\-ри\-цей, полимерных 
КМ с~углеволокном и~др.

\vspace*{-9pt}
     
     \subsection*{Основные уровни моделирования}
     
     \vspace*{-2pt}
     
     
     \textbf{Квантово-механический}. Рассматриваются отдельные молекулы. 
Решается уравнение Шредингера, определяется атомарная струк\-ту\-ра молекул 
полимера и~наполнителя, строится электронная струк\-ту\-ра и~рас\-счи\-ты\-ва\-ет\-ся 
когезионная энергия, рас\-счи\-ты\-ва\-ют\-ся меж\-атом\-ные и~меж\-мо\-ле\-ку\-ляр\-ные силы, 
определяются отдельные фи\-зи\-ко-хи\-ми\-че\-ские свойства.
     
     \textbf{Молекулярно-динамический}. Изучаются ан\-самб\-ли из молекул. 
Решаются уравнения молекулярной динамики с~использованием потенциалов 
межатомного взаимодействия, рас\-счи\-ты\-ва\-ют\-ся структурные характеристики 
мат\-ри\-цы (полимерной, металлической и~др.), наполнителя (нанотрубки, 
волокна и~др.), физические свойства. 
     
     \textbf{Мезоскопический}. Рас\-смат\-ри\-ва\-ют\-ся крупнозернистые модели. 
Используется упрощенное строение молекул. Цель моделирования на 
мезоуровне~--- получение распределения час\-тиц \mbox{наполнителя} в~мат\-ри\-це 
(полимерной, металлической и~др.)\ с~по\-сле\-ду\-ющим расчетом инженерных 
свойств полученных сис\-тем. 

\begin{figure*}[b] %fig1
\vspace*{8pt}
  \begin{center}  
    \mbox{%
\epsfxsize=133.618mm
\epsfbox{abg-1.eps}
}

\end{center}
\vspace*{-2pt}
\Caption{Схема многомасштабной композиции $\mathbf{MK}_{0,1,2,3,4}^{(\mathrm{Ti/Mo})}$ 
для расчета структурных свойств МКМ}
\end{figure*}
     
     \textbf{Континуальный} (\textbf{макроскопический}). Проводится расчет 
инженерных свойств (механические свойства, теп\-ло\-про\-вод\-ность и~др.). Задачи 
решаются с~применением механики сплош\-ных сред, гид\-ро\-ди\-на\-ми\-ки, тео\-рии 
упру\-гости. Применяются метод конечных элементов, методы решения краевых 
задач для моделирования различных процессов. 
     
     Рассмотрим пример построения многомасштабной композиции для 
тес\-то\-во\-го рас\-че\-та структурных свойств металлического 
КМ (МКМ) на основе Ti (титана), армированного волокнами Mo 
(молибдена). На сегодня Ti и~титановые сплавы стали очень привлекательными 
материалами для перспективных сфер применения благодаря таким свойствам, 
как низкая плот\-ность, высокие механические свойства и~коррозионная стой\-кость. 
Использование данных материалов в~конструкциях самолетов (реактивный 
двигатель и~фюзеляж) и~применение в~автомобильной про\-мыш\-лен\-ности рас\-тут 
быст\-ры\-ми темпами. Одним из способов совершенствования\linebreak титановых сплавов 
стало их применение в~качестве мат\-ри\-цы для КМ, 
армированных волокнами, например из Mo, которые обладают очень \mbox{хорошими} 
механическими свойствами ({\sf http://\linebreak viam-works.ru/ru/articles?art\_id=1103}). 
     
     Задействуем четыре перечисленных выше масштабных уров\-ня (не считая 
нулевого). Используя обозначения из~\cite{1-ab, 2-ab}, для построения 
многомасштабной композиции 
$$
\mathbf{MK}_{0,1,2,3,4}^{(\mathrm{Mo}, \mathrm{Ti}; 
1{,}1; 1{,}2; 2{,}1; 2{,}2; 3{,}1; 4{,}1)}= \mathbf{MK}_{0,1,2,3,4}^{(\mathrm{Ti/Mo})}
$$ 

\vspace*{-3pt}

\noindent
приведем экземпляры базовых мо\-де\-лей-ком\-по\-зи\-ций: 

\vspace*{-9pt}

\noindent
     \begin{align*}
     \mathbf{El}_{01}^{\mathrm{Ti}}:& \left\{ V_{01}^{\mathrm{Ti}}, 
X_{01}^{\mathrm{Ti}}, \mathrm{MA}_{01}^{\mathrm{Ti}}\right\};\\[-3pt]
     \mathbf{El}_{01}^{\mathrm{Mo}}:& \left\{ V_{01}^{\mathrm{Mo}}, 
X_{01}^{\mathrm{Mo}}, \mathrm{MA}_{01}^{\mathrm{Mo}}\right\};\\[-3pt]
\mathbf{MC}_{11}^{\mathrm{Ti}}:& \left\{ V_{11}^{\mathrm{Ti}}, 
X_{11}^{\mathrm{Ti}}, \mathrm{MA}_{11}^{\mathrm{Ti}}\right\};\\[-3pt]
\mathbf{MC}_{11}^{\mathrm{Mo}}:& \left\{ V_{11}^{\mathrm{Mo}}, 
X_{11}^{\mathrm{Mo}}, \mathrm{MA}_{11}^{\mathrm{Mo}}\right\};
\end{align*}

\noindent
\begin{align*}
               \mathbf{MC}_{12}^{\mathrm{Ti}}:& \left\{ V_{12}^{\mathrm{Ti}}, 
X_{12}^{\mathrm{Ti}}, \mathrm{MA}_{12}^{\mathrm{Ti}}\right\};\\
     \mathbf{MC}_{12}^{\mathrm{Mo}}:& \left\{ V_{12}^{\mathrm{Mo}}, 
X_{12}^{\mathrm{Mo}}, \mathrm{MA}_{12}^{\mathrm{Mo}}\right\};\\
     \mathbf{MC}_{21}^{\mathrm{Ti}}:& \left\{ V_{21}^{\mathrm{Ti}}, 
X_{21}^{\mathrm{Ti}}, \mathrm{MA}_{21}^{\mathrm{Ti}}\right\};\\
     \mathbf{MC}_{21}^{\mathrm{Mo}}:& \left\{ V_{21}^{\mathrm{Mo}}, 
X_{21}^{\mathrm{Mo}}, \mathrm{MA}_{21}^{\mathrm{Mo}}\right\};\\
     \mathbf{MC}_{22}^{\mathrm{Ti}}:& \left\{ V_{22}^{\mathrm{Ti}}, 
X_{22}^{\mathrm{Ti}}, \mathrm{MA}_{22}^{\mathrm{Ti}}\right\};\\
     \mathbf{MC}_{22}^{\mathrm{Mo}}:& \left\{ V_{22}^{\mathrm{Mo}}, 
X_{22}^{\mathrm{Mo}}, \mathrm{MA}_{22}^{\mathrm{Mo}}\right\};\\
     \mathbf{MC}_{31}^{\mathrm{Ti}/\mathrm{Mo}}:& \left\{
     V_{31}^{\mathrm{Ti}/\mathrm{Mo}}, X_{31}^{\mathrm{Ti}/\mathrm{Mo}}, 
\mathrm{MA}_{31}^{\mathrm{Ti}/\mathrm{Mo}}\right\};\\
     \mathbf{MC}_{41}^{\mathrm{Ti}/\mathrm{Mo}}:& \left\{
     V_{41}^{\mathrm{Ti}/\mathrm{Mo}}, X_{41}^{\mathrm{Ti}/\mathrm{Mo}}, 
\mathrm{MA}_{41}^{\mathrm{Ti}/\mathrm{Mo}}\right\}.
     \end{align*}
     
     Согласно схематическому пред\-став\-ле\-нию (рис.~1) многомасштабная 
композиция $\mathbf{MK}_{0,1,2,3,4}^{(\mathrm{Ti/Mo})}$ со\-сто\-ит из связанных между 
собой экземпляров базовых моделей композиций, размещенных на 
со\-от\-вет\-ст\-ву\-ющих мас\-штаб\-ных уровнях. На наноуровне проводится  
мо\-ле\-ку\-ляр\-но-ди\-на\-ми\-че\-ское моделирование структурных свойств 
титановой мат\-ри\-цы и~молибденовых волокон. На мезоуровне рас\-смат\-ри\-ва\-ет\-ся 
распределение час\-тиц в~МКМ, на мак\-ро\-уров\-не проводится расчет механических 
свойств МКМ.

\setcounter{figure}{2}
\begin{figure*}[b] %fig3
\vspace*{-6pt}
  \begin{center}  
    \mbox{%
\epsfxsize=120.383mm
\epsfbox{abg-3.eps}
}

\end{center}
\vspace*{-9pt}
\Caption{Пример сценария с~цик\-лом}
\end{figure*}
%\pagebreak
     
\vspace*{-10pt}

\section{Программный комплекс}

\vspace*{-2pt}

   Программный комплекс, интегрированный с~расчетными пакетами 
и~модулями, размещается на высокопроизводительных многоядерных сис\-те\-мах, 
оснащенных мощными графическими процессорами. Это связано с~тем, что 
исполнение вычислительных экспериментов, а~так\-же обработка 
и~анализ результатов вы\-чис\-ли\-тель\-ных  экспериментов
 ориентированы на 
распределенные сис\-те\-мы сбора, хранения и~обработки больших данных. В~основе 
программного комплекса лежит интеграционная платформа для 
многомасштабного моделирования, которая объединяет информационные потоки 
на разных мас\-штаб\-ных уровнях. При решении конкретной задачи, такой как 
расчет структурных особенностей, механических или иных свойств 
КМ, при изучении процессов их де\-гра\-да\-ции 
и~разрушения и~др.\ выделяются конкретные уров\-ни моделирования, которые 
необходимо задействовать. Первоначально строится многомасштабная 
композиция~--- информационный аналог\linebreak мно\-го\-мас\-штаб\-ной  
фи\-зи\-ко-ма\-те\-ма\-ти\-че\-ской модели. Для программной реализации на базе 
интеграционной платформы~\cite{4-ab} из име\-ющих\-ся программных модулей 
формируется вы\-чис\-ли\-тель\-ный \mbox{комплекс}~\cite{5-ab, 6-ab}.
   
   Перечислим пользовательские роли в~интеграционной плат\-фор\-ме 
мно\-го\-мас\-штаб\-но\-го моделирования:
   \begin{itemize}
\item разработчик вычислительных модулей реализует расчетный модуль или 
осуществляет конфигурирование при\-клад\-но\-го па\-кета;\\[-15pt]
\item системный разработчик создает веб-сер\-ви\-сы для вы\-чис\-ли\-тель\-но\-го модуля 
и~интегрирует его в~плат\-форму;\\[-15pt]
\item разработчик расчетных сценариев создает сценарии в~среде моделирования;\\[-15pt]
\item ученый-исследователь прикладной об\-ласти запускает расчетные сценарии 
с~различными па\-ра\-мет\-ра\-ми и~анализирует ре\-зуль\-таты.
\end{itemize}
    
    Как отмечалось в~\cite{5-ab, 6-ab}, программный комплекс предназначен для 
создания и~исполнения сценариев многомасштабных расчетов для моделирования 
структурных свойств композитных материалов.
    
    Сценарий~--- программная реализация мно\-го\-мас\-штаб\-ной композиции~--- 
пред\-став\-ля\-ет собой алгоритм последовательного выполнения расчетов отдельных 
физических характеристик материалов, входящих в~со\-став композита, 
посредством интегрированных с~программным комплексом вы\-чис\-ли\-тель\-ных 
модулей. Среда моделирования сценариев поз\-во\-ля\-ет создавать или 
модифицировать сценарии, учитывая особенности конкретного 
КМ и~тре\-бу\-емые свойства.



 
    
    Среда исполнения сценариев дает возможность осуществить его запуск 
    с~заданными входными па\-ра\-мет\-ра\-ми, отслеживать его выполнение в~целом и~по 
со\-став\-ным задачам, про\-смат\-ри\-вать входные и~выходные данные (результаты 
расчетов). Интеграционная роль среды исполнения заключается\linebreak в~формировании 
входных данных для вычислительных модулей в~со\-от\-вет\-ст\-ву\-ющем формате 
и~единицах измерения, отслеживании работы модулей,\linebreak получении конечного 
результата расчета и~преобразовании его в~формат и~единицы измерения, 
до\-ступ\-ные для других модулей сценария. Таким образом, среда исполнения 
обеспечивает соответствие потока исполнения вы\-чис\-ли\-тель\-ных модулей 
заданному алгоритму в~сценарии и~це\-лост\-ность потока данных между блоками 
сценария. Кроме того,\linebreak среда исполнения предостав\-ля\-ет общие словари для\linebreak 
согласования вход\-ных-вы\-ход\-ных данных вы\-чис\-ли\-тель\-ных экспериментов, 
такие как справочник\linebreak химических элементов и~их свойств, химических формул 
веществ, ис\-поль\-зу\-емых в~композитных материалах, типы крис\-тал\-ли\-че\-ских 
сис\-тем, типы атомных радиусов, пространственные группы.

\setcounter{figure}{3}
\begin{figure*}[b] %fig4
\vspace*{-9pt}
  \begin{center}  
    \mbox{%
\epsfxsize=163mm
\epsfbox{abg-4.eps}
}

\end{center}
\vspace*{-9pt}
\Caption{Сценарий для расчета МКМ}
\end{figure*}

\vspace*{-10pt}
   
    \subsection*{Алгоритм программы}
    
    \vspace*{-2pt}
    
    Алгоритм исполнения сценария основан на стандарте BPMN~2.0 и~со\-сто\-ит из 
сле\-ду\-ющих ключевых элементов (рис.~2).

{ \begin{center}  %fig2
 \vspace*{6pt}
    \mbox{%
\epsfxsize=70.82mm
\epsfbox{abg-2.eps}
}

\vspace*{6pt}

\noindent
{{\figurename~2}\ \ \small{
Пример простого сценария
}}
\end{center}
}

%\vspace*{6pt} 

\noindent
\begin{description}
\item[Э1.]  Точка начала выполнения сценария. В~свойствах этого элемента 
указывается список кодов физических величин, которые пользователь дол\-жен 
будет ввес\-ти перед запуском сценария.

\item[Э2.] Сплошная стрелка определяет строгую по\-сле\-до\-ва\-тель\-ность 
выполнения шагов сценария.

 \begin{figure*}[b] %fig5
  \vspace*{1pt}
  \begin{center}  
    \mbox{%
\epsfxsize=131mm %.834mm
\epsfbox{abg-5.eps}
}

\end{center}
\vspace*{-9pt}
  \Caption{Сценарий для расчета механических свойств полимерного нанокомпозита}
  \end{figure*}

\item[Э3.] Вычислительная задача пред\-став\-ля\-ет\-ся в~BPMN как <<внеш\-няя 
сервисная задача>> (External Service Task). В~поле topic в~настройках 
задачи вводится название очереди задач со\-от\-вет\-ст\-ву\-юще\-го 
вы\-чис\-ли\-тель\-но\-го модуля. Например, для  
кван\-то\-во-ме\-ха\-ни\-че\-ско\-го расчета на пакете VASP вводится 
<<vasp\_topic>>. Список до\-ступ\-ных вы\-чис\-ли\-тель\-ных модулей 
с~названиями очередей хранится в~базе данных в~таб\-ли\-це <<Module>>.\\[-15pt]

\item[Э4.] Точка завершения выполнения сценария. Если в~сценарии существует 
ветвление, точек завершения может быть несколько.



    \item[Э5.] Шаг сценария, в~рамках которого выполняется скрипт, заданный 
пользователем. В~па\-ра\-мет\-рах задачи может быть указан язык скрип\-та и~сам 
скрипт. Доступны языки Groovy и~Jython (реализация языка Python на Java). 
Скрип\-ты могут использоваться для изменения входных и~выходных па\-ра\-мет\-ров, 
небольших вы\-чис\-ле\-ний на основе текущих до\-ступ\-ных данных сценария. 
В~примере на рис.~3 в~цик\-ле определяется список векторов 
крис\-тал\-ли\-че\-ской решетки, по которым будет проводиться кван\-то\-во-ме\-ха\-ни\-че\-ский 
рас\-чет деформированной решетки.\\[-19.5pt]

\begin{figure*}[b] %fig6
\vspace*{1pt}
  \begin{center}  
    \mbox{%
\epsfxsize=163mm
\epsfbox{abg-6.eps}
}

\end{center}
\vspace*{-9pt}
\Caption{Сценарий для расчета КМ с~полимерной мат\-ри\-цей 
и~наполнителем из углеволокна}
\end{figure*}
    
    \item[Э6.] Подпроцесс сценария <<цикл с~параллельным запуском>>  
(Parallel multi-instance) позволяет параллельно запустить выполнение час\-ти 
сцена- %\linebreak\vspace*{-12pt}

\columnbreak

\noindent
рия несколько раз. В~свойствах подпроцесса требуется указать коллекцию 
(Collection), по элементам которой будет проводиться ите\-ри\-ро\-ва\-ние, и~название 
переменной цикла (Element Variable). Весь элемент считается выполненным, когда 
все параллельно выполняющиеся подпроцессы завершат свою работу. Например, 
если требуется запустить кван\-то\-во-ме\-ха\-ни\-че\-ский расчет для некоторого 
множества деформированных решеток (для определения в~дальнейшем констант 
упру\-гости), предварительно в~скрип\-те перед цик\-лом формируется список 
деформированных векторов решетки и~сохраняется в~переменную процесса. 
Далее для каж\-дой деформации параллельно вызывается\linebreak\vspace*{-12pt}

\pagebreak

\noindent  
кван\-то\-во-ме\-ха\-ни\-че\-ский модуль VASP для расчета энергии и~объема 
решетки. Получившаяся таб\-ли\-ца с~данными может использоваться для расчета 
констант элас\-тич\-ности, модуля упру\-гости и~других свойств материала.
\end{description}

\vspace*{-9pt}
  
  \subsection*{Примеры тестовых сценариев для~расчета~структурных~характеристик 
  и~отдельных~свойств различных классов 
композиционных материалов}


     
     \textbf{Пример~1.} Тестовый сценарий для расчета структурных свойств 
КМ с~металлической мат\-ри\-цей (рис.~4).

\smallskip

     
     \textbf{Пример~2.} Тестовый сценарий для расчета механических свойств 
полимерного нанокомпозита (полифениленсульфид с~углеродными нанотрубками). 
На сле\-ду\-ющем этапе проекта планируется расширить сценарий для 
оценки влияния процентного содержания углеродных нанотрубок на изменение 
коэффициента теп\-ло\-про\-вод\-ности полимерного нанокомпозита (рис.~5).
  
 
     
     \textbf{Пример~3.} Тестовый сценарий для расчета механических свойств 
КМ с~полимерной мат\-ри\-цей (эпоксидная смола) и~углеволокном
(рис.~6).

\vspace*{-6pt}

\section{Выводы}

\vspace*{-2pt}

     В работе представлен программный комплекс для расчета структурных 
характеристик КМ с~тре\-бу\-емы\-ми свойствами. В~его 
основе лежит интеграционная плат\-фор\-ма для многомасштабного моделирования, 
которая объединяет информационные потоки на разных мас\-штаб\-ных уровнях. На 
ее основе формируются схемы для рас\-че\-та структурных характеристик разных 
клас\-сов КМ: нанокомпозитов на основе полимерной 
мат\-ри\-цы, КМ с~металлической мат\-ри\-цей, полимерных 
КМ с~углеволокном и~другие. Разработанные подходы 
поз\-во\-ля\-ют моделировать свойства КМ (механические, 
теп\-ло\-вые и~др.), а~так\-же многомасштабные процессы, связанные с~усталостным 
разрушением при случайных по\-вреж\-де\-ни\-ях в~ходе эксплуатации, и~другие 
динамические процессы. Программный комплекс со\-сто\-ит из программных 
модулей и~базируется на типовых сер\-ви\-сах вы\-чис\-ли\-тель\-ных модулей, общей 
интеграционной оболочки и~модулей сценариев. Про\-грам\-мные решения 
сертифицированы. В~дальнейшем планируется раз\-ра\-бо\-тать 
полнофункциональную про\-грам\-мную сис\-те\-му с~целью решения различных 
классов обратных задач в~об\-ласти наук о~материалах. Разработанные подходы 
в~сочетании с~экспериментальными данными могут быть использованы для 
лучшего понимания физических основ изменения свойств в~за\-ви\-си\-мости от 
струк\-ту\-ры и,~как след\-ст\-вие, для уде\-шев\-ле\-ния и~уско\-ре\-ния поиска новых 
КМ с~заданными свойствами.

\vspace*{-6pt}
   
{\small\frenchspacing
 {%\baselineskip=10.8pt
 %\addcontentsline{toc}{section}{References}
 \begin{thebibliography}{9}
 
 \vspace*{-2pt}
   
   \bibitem{1-ab}
   \Au{Абгарян К.\,К.} Многомасштабное моделирование в~задачах структурного 
материаловедения.~--- М.: МАКСПресс, 2017. 284~с.
\bibitem{2-ab}
\Au{Абгарян~К.\,К.} Информационная технология по\-стро\-ения многомасштабных моделей 
в~задачах вы\-чис\-ли\-тель\-но\-го материаловедения~// Сис\-те\-мы высокой до\-ступ\-ности, 2018. Т.~14. 
№\,2. С.~9--15.
\bibitem{3-ab}
\Au{Naffakh M., D$\acute{\!\mbox{{\!\ptb{\i}}}}$ez-Pascuala~A.\,M., Marcoa~C., Ellisa~G.} Morphology and thermal properties of novel poly (phenylene sulfide) 
hybrid nanocomposites based on single-walled carbon nanotubes and 8 inorganic fullerene-like WS~2 
nanoparticles~// J.~Mater. Chem., 2012. Vol.~22. No.\,4. P.~1418--1425.
\bibitem{4-ab}
\Au{Абгарян К.\,К., Гаврилов~Е.\,С.} Распределенная информационная сис\-те\-ма для расчета 
структурных свойств композиционных материалов~// Информатика и~её применения, 2021. 
Т.~15. Вып.~4. С.~50--58. doi: 10.14357/ 19922264210407.
\bibitem{5-ab}
\Au{Гаврилов Е.\,С.} Интегрированный интерфейс к~модулю сплош\-но\-сред\-но\-го взаимодействия. 
Свидетельство о~регистрации программ для ЭВМ №\,2021681058, 2021.
\bibitem{6-ab}
\Au{Гаврилов Е.\,С.} Программные средства для хранения и~обмена данными в~задачах 
моделирования композитных материалов. Свидетельство о~регистрации программ для ЭВМ 
№\,2021681762, 2021.

\end{thebibliography}

 }
 }

\end{multicols}

\vspace*{-8pt}

\hfill{\small\textit{Поступила в~редакцию 22.01.22}}

\vspace*{8pt}

%\pagebreak

%\newpage

%\vspace*{-28pt}

\hrule

\vspace*{2pt}

\hrule

%\vspace*{-2pt}

\def\tit{SOFTWARE PACKAGE FOR MULTISCALE MODELING OF~STRUCTURAL PROPERTIES 
OF~COMPOSITE MATERIALS}


\def\titkol{Software package for multiscale modeling of~structural properties 
of~composite materials}


\def\aut{K.\,K.~Abgaryan$^1$ and~E.\,S.~Gavrilov$^{1,2}$}

\def\autkol{K.\,K.~Abgaryan and~E.\,S.~Gavrilov}

\titel{\tit}{\aut}{\autkol}{\titkol}

\vspace*{-18pt}


\noindent
$^1$Federal Research Center ``Computer Science and Control'' of the Russian Academy of Sciences, 
44-2~Vavilov\linebreak
$\hphantom{^1}$Str., Moscow 119333, Russian Federation

\noindent
$^2$Moscow Aviation Institute (National Research University), 4~Volokolamskoe Shosse, Moscow 
125080, Russian\linebreak
$\hphantom{^1}$Federation

\def\leftfootline{\small{\textbf{\thepage}
\hfill INFORMATIKA I EE PRIMENENIYA~--- INFORMATICS AND
APPLICATIONS\ \ \ 2022\ \ \ volume~16\ \ \ issue\ 1}
}%
 \def\rightfootline{\small{INFORMATIKA I EE PRIMENENIYA~---
INFORMATICS AND APPLICATIONS\ \ \ 2022\ \ \ volume~16\ \ \ issue\ 1
\hfill \textbf{\thepage}}}

\vspace*{3pt} 
      
      
  
\Abste{Today, creation of new composite materials and methods of their construction with predictable 
properties is one of the urgent and most important tasks connected with modernization of 
industrial production in our country. For their solution, technologies of multiscale computer modeling 
are actively developed. They have become a~link between fundamental physics (chemistry) and 
engineering materials science. The paper presents a~software package for modeling structural 
properties of composite materials which allows solving a~number of problems of this class. It is 
focused on high-performance computations. The complex is based on an original multiscale 
technology which allows one to promptly conduct multivariate analysis of different classes of 
composite materials and conduct research on designing the new ones with predictable properties. The 
developed approaches in combination with experimental data can be used for a~better understanding of 
the physical foundations of the change of properties depending on the structure and, as a~consequence, 
for cheaper and faster search of new composite materials with predetermined properties.}

\KWE{multiscale modeling; composite materials; integration platform; software package; distributed 
system}



\DOI{10.14357/19922264220113}

\vspace*{-16pt}

\Ack
\noindent
The research was supported by the Ministry of Science and Higher Education of the Russian 
Federation (project No.\,075-15-2020-799).




%\vspace*{4pt}

  \begin{multicols}{2}

\renewcommand{\bibname}{\protect\rmfamily References}
%\renewcommand{\bibname}{\large\protect\rm References}

{\small\frenchspacing
 {%\baselineskip=10.8pt
 \addcontentsline{toc}{section}{References}
 \begin{thebibliography}{9}
\bibitem{1-ab-1}
\Aue{Abgaryan, K.\,K.} 2017. \textit{Mnogomasshtabnoe modelirovanie v~zadachakh strukturnogo 
materialovedeniya} [Multiscale modeling for structural materials science applications]. Moscow: 
MAKS Press. 284~p.

\vspace*{-2pt}

\bibitem{2-ab-1}
\Aue{Abgaryan, K.\,K.} 2018. In\-for\-ma\-tsi\-on\-naya tekh\-no\-lo\-giya po\-stro\-eniya mno\-go\-mas\-shtab\-nykh 
mo\-de\-ley v~za\-da\-chakh vy\-chis\-li\-tel'\-no\-go ma\-te\-ri\-a\-lo\-ve\-de\-niya 
[Information technology is the construction 
of multi-scale models in problems of computational materials science]. \textit{Sistemy vysokoy 
dostupnosti} [Highly Available Systems] 14(2):9--15.
\bibitem{3-ab-1}
\Aue{Naffakh, M., A.\,M.~D$\acute{\mbox{{\!\ptb{\i}}}}$ez-Pascuala, C.~Marcoa, and G.~Ellisa.} 
2012. Morphology and thermal properties of novel poly (phenylene sulfide) hybrid nanocomposites 
based on single-walled carbon nanotubes and~8~inorganic fullerene-like WS~2 nanoparticles. 
\textit{J.~Mater. Chem.}  
22(4):1418--1425.
{\looseness=1

}
\bibitem{4-ab-1}
  \Aue{Abgaryan, K.\,K., and E.\,S.~Gavrilov.} 2021. 
  Ras\-pre\-de\-len\-naya in\-for\-ma\-tsi\-on\-naya sis\-te\-ma   dlya 
ras\-che\-ta struk\-tur\-nykh svoystv kom\-po\-zi\-tsi\-on\-nykh ma\-te\-ri\-alov 
[Distributed information system for 
calculating the structural properties of composite materials]. \textit{Informatika i~ee Primeneniya~--- 
Inform. Appl.} 15(4):50--58. doi: 10.14357/19922264210407.
\bibitem{5-ab-1}
  \Aue{Gavrilov, E.\,S.} 2021. In\-teg\-ri\-ro\-van\-nyy in\-ter\-feys k~mo\-du\-lyu 
  splosh\-no\-sred\-no\-go 
vza\-imo\-dey\-stviya [Integrated interface to the solid-medium interaction module]. Certificate on official 
registration of the computer program No.\,2021681058.
\bibitem{6-ab-1}
  \Aue{Gavrilov, E.\,S.} 2021. Pro\-gram\-mnye sred\-st\-va dlya khra\-ne\-niya 
  i~ob\-me\-na dan\-ny\-mi  v~za\-da\-chakh mo\-de\-li\-ro\-va\-niya kom\-po\-zit\-nykh ma\-te\-ri\-a\-lov 
  [Software tools for data persistence and data flow in 
composite materials modeling tasks]. Certificate on official registration of the computer program 
No.\,2021681762.
\end{thebibliography}

 }
 }

\end{multicols}

\vspace*{-6pt}

\hfill{\small\textit{Received January 22, 2022}}


\Contr

\noindent
\textbf{Abgaryan Karine K.} (b.\ 1963)~--- Doctor of Science in physics and mathematics, principal 
scientist, A.\,A.~Dorodnicyn Computing Center, Federal Research Center ``Computer Science and 
Control'' of the Russian Academy of Sciences, 40~Vavilov Str., Moscow 119333, Russian Federation; 
head of department, Moscow Aviation Institute (National Research University), 4~Volokolamskoe 
Shosse, Moscow 125080, Russian Federation; \mbox{kristal83@mail.ru}

\vspace*{3pt}

\noindent
\textbf{Gavrilov Evgeny S.} (b.\ 1982)~--- scientist, A.\,A.~Dorodnicyn Computing Center, Federal 
Research Center ``Computer Science and Control'' of the Russian Academy of Sciences, 40~Vavilov 
Str., Moscow 119333, Russian Federation; senior lecturer, Moscow Aviation Institute (National 
Research University), 4~Volokolamskoe Shosse, Moscow 125080, Russian Federation; 
\mbox{eugavrilov@gmail.com}
       

\label{end\stat}

\renewcommand{\bibname}{\protect\rm Литература}  %7
\include{shetinin} %8
\def\stat{sokolov}

\def\tit{БАЗИС РЕАЛИЗАЦИИ СБОЕУСТОЙЧИВЫХ ЭЛЕКТРОННЫХ СХЕМ}

\def\titkol{Базис реализации сбоеустойчивых электронных схем}

\def\aut{И.\,А.~Соколов$^1$, Ю.\,А.~Степченков$^2$, Ю.\,Г.~Дьяченко$^3$,  
Ю.\,В.~Рождественский$^4$,\\ А.\,Н.~Каменских$^5$}

\def\autkol{И.\,А.~Соколов, Ю.\,А.~Степченков, Ю.\,Г.~Дьяченко и~др.}  
%Ю.\,В.~Рождественский$^4$, А.\,Н.~Каменских$^5$}

\titel{\tit}{\aut}{\autkol}{\titkol}

\index{Соколов И.\,А.}
\index{Степченков Ю.\,А.}
\index{Дьяченко Ю.\,Г.}  
\index{Рождественский Ю.\,В.}
\index{Каменских А.\,Н.}
\index{Sokolov I.\,A.}
\index{Stepchenkov Yu.\,A.}
\index{Diachenko Yu.\,G.} 
\index{Rogdestvenski Yu.\,V.}
\index{Kamenskih A.\,N.}


%{\renewcommand{\thefootnote}{\fnsymbol{footnote}} \footnotetext[1]
%{Работа частично финансирована РФФИ (проект 19-01-00430).}}


\renewcommand{\thefootnote}{\arabic{footnote}}
\footnotetext[1]{Федеральный исследовательский центр <<Информатика и~управление>> Российской академии наук, 
\mbox{ISokolov@ipiran.ru}}
\footnotetext[2]{Федеральный исследовательский центр <<Информатика и~управление>> Российской академии наук, 
\mbox{YStepchenkov@ipiran.ru}}
\footnotetext[3]{Федеральный исследовательский центр <<Информатика и~управление>> Российской академии наук, 
\mbox{diaura@mail.ru}}
\footnotetext[4]{Федеральный исследовательский центр <<Информатика и~управление>> Российской академии наук, 
\mbox{YRogdest@ipiran.ru}}
\footnotetext[5]{Пермский национальный исследовательский политехнический университет, 
\mbox{antoshkinoinfo@yandex.ru}}

\vspace*{6pt}

 


\Abst{Исследуется устойчивость самосинхронных (СС) и~синхронных схем к~логическим 
сбоям (ЛС), которые могут вызвать нарушения режима работы системы управления 
сложным техническим устройством. Предлагается использование сбоеустойчивого 
СС-ко\-ди\-ро\-ва\-ния, рас\-смат\-ри\-ва\-юще\-го ан\-ти\-спей\-сер\-ное со\-сто\-яние как 
второе состояние спейсера, что позволяет повысить уровень сбоеустойчивости 
СС-схем. Количественные оценки в~первом приближении показывают 
явное (в~2,0--4,7~раза) преимущество СС-схе\-мы в~сравнении 
с~синхронным аналогом по уровню сбоеустойчивости. Использование 
модифицированного С-эле\-мен\-та Маллера для реализации разряда регистра ступени 
конвейера увеличивает это преимущество до уровня 2,2--5,4~раза. Благодаря этому 
СС-схе\-мы становятся предпочтительным базисом для реализации 
сбоеустойчивых электронных схем для систем управления сложными техническими 
устройствами.}

\KW{синхронные схемы; самосинхронные схемы; логический сбой; 
сбоеустойчивость; конвейер; индикация завершения переключения; вероятностная 
оценка}

\DOI{10.14357/19922264210409}
  
%\vspace*{9pt}


\vskip 10pt plus 9pt minus 6pt

\thispagestyle{headings}

\begin{multicols}{2}

\label{st\stat}

\section{Введение}

  Неблагоприятные факторы окружающей среды (ядерные частицы, радиация, 
электромагнитное излучение, шумы) вызывают сбои в~работе интегральных 
микросхем. Количественные и~качественные оценки характера сбоев 
показывают, что среди них преобладают кратковременные однократные 
логические сбои (soft errors)~[1]. С~точки зрения жизнеспособности 
микросхемы они менее опасны, чем отказы, поскольку приводят к~временному 
изменению состояния логического элемента схемы.

  
  Характеристики ЛС существенно зависят от типа источника сбоя и~его 
мощности (энергии). Например, ЛС, вызванный высокоэнергичной ядерной 
частицей или космическими лучами, может длиться до нескольких 
наносекунд~[2], в~то время как длительность ЛС из-за помех на трассах 
межсоединений, как правило, не превышает десятков пикосекунд. 
  
  Современная микроэлектроника в~основном реализует синхронный принцип 
обработки данных, базирующийся на глобальном синхросигнале. 
С~повышением частоты синхронизации растет и~вероятность того, что ЛС, 
даже кратковременный, запишется в~регистр и~станет критическим.
  
  Альтернативой синхронным схемам выступают СС-схе\-мы~[3]. 
  Они имеют ряд свойств, отличающих их от синхронных схем:
  \begin{itemize}
\item избыточное кодирование данных. В~комбинационных СС-схе\-мах, как 
правило, используется парафазное кодирование информационных сигналов;
\item двухфазную дисциплину функционирования: рабочая фаза, в~которой 
схема формирует новое рабочее состояние, и~спейсерная фаза, в~которой 
каждый информационный сигнал схемы переключается в~одно состояние, не 
входящее в~множество разрешенных рабочих состояний;
\item индикацию всех элементов схемы, подтверждающую завершение 
переключения схемы в~текущую фазу.
\end{itemize}

  Эти свойства обеспечивают маскирование большей части ЛС. Благодаря им, 
СС-схемы обладают большей устойчивостью к~ЛС, чем их синхронные 
аналоги~[4--6]. Предложенные в~работах~\cite{4-sok, 5-sok} методы 
схемотехнического и~топологического проектирования СС-схем дополнительно 
повышают устойчивость СС-схем к~ЛС.
  
  Известен ряд способов~[7, 8] повышения сбоеустойчивости асинхронных 
схем. Однако в~них не делается попыток количественного анализа 
естественного уровня стойкости синхронных и~самосинхронных схем к~ЛС.
  
  В статье проводится анализ особенности появления и~распространения ЛС 
в~синхронных и~самосинхронных схе\-мах. Даются вероятностные оценки их устойчивости 
к~ЛС, подтверждающие целесообразность использования СС-схем в~качестве 
базиса для реализации сбоеустойчивых систем управления.
  
\section{Типы логических сбоев}

  Типы ЛС зависят от используемого кодирования обрабатываемых данных. 
Синхронные схемы обычно используют неизбыточное двоичное кодирование. 
Логический сбой в~них выражается в~изменении значения сигнала на противоположное.  
Самосинхронные схе\-мы всегда используют избыточное, чаще всего парафазное, 
кодирование~[3] и~строгое чередование спейсера и~рабочих со\-сто\-яний. 
  
  В каждый момент времени парафазный сигнал пребывает в~спейсере 
(нулевом, <<00>>, или единичном, <<11>>) или в~одном из рабочих состояний 
(<<01>> или <<10>>). Состояние, противоположное спейсеру данного 
парафазного сигнала,~--- антиспейсер~--- в~корректно спроектированной  
СС-схе\-ме может появиться только в~результате ЛС. Типовая индикация  
СС-схе\-мы воспринимает антиспейсер как рабочее состояние, что приводит 
к~ошибке. Но индикация его как спейсера с~помощью ячейки 
<<неравнозначность>> (XOR) существенно повышает устойчивость СС-схем 
к~этому типу ЛС~\cite{4-sok}. Таблица~1 демонстрирует сбоеустойчивое 
кодирование парафазного сигнала. Все возможные случаи однократного 
кратковременного ЛС в~СС-схе\-мах для парафазного сигнала с~нулевым 
спейсером показаны в~табл.~2.



  
  Отметим, что не каждый ЛС в~схеме становится критическим, т.\,е.\ приводит к~искажению обрабатываемых данных. Он может замаскироваться значениями 
других сигналов в~соответствии с~выполняемой функцией или не успеть 
записаться в~регистр. Оценим вероятность того, что наблюдаемый\linebreak\vspace*{-12pt}

\begin{center}
{{\tablename~1}\ \ \small{
Сбоеустойчивое кодирование СС-схем
}}

\vspace*{6pt}

{\small
\begin{tabular}{|c|c|c|c|}
\hline
\multicolumn{2}{|c|}{Парафазный сигнал}&
\multicolumn{1}{c|}{\raisebox{-6pt}[0pt][0pt]{Состояние}}&Индикатор \\
\cline{1-2}
\ \ \ \ \ \ \ X\ \ \ \ \ \ \ &XB&&XOR\\
\hline
0&0&спейсер&0\\
0&1&бит <<0>>&1\\
1&0&бит <<1>>&1\\
1&1&спейсер&0\\
\hline
\end{tabular}}
\end{center}
%\begin{table*}
%\end{table*}
%\begin{table*}\small %tabl2
\begin{center}
{{\tablename~2}\ \ \small{
Возможные ЛС
}}


\vspace*{6pt}

{\small
\begin{tabular}{|c|c|c|c|c|c|c|c|c|c|}
\hline
\tabcolsep=0pt\begin{tabular}{c}Состояние\\ парафазного\\ сигнала\end{tabular}& До ЛС& После ЛС\\
\hline
1&00&01\\
2&00&10\\
3&00& 11\\
4&01& 00\\
5&01& 11\\
6&01& 10\\
7&10& 00\\
8&10&11\\
9&10&01\\
\hline
\end{tabular}}
\end{center}
%\end{table*}

\vspace*{6pt}



\noindent
 в~схеме 
кратковременный ЛС окажется критическим.
  
\section{Методика расчета вероятности~сбоя}

  Для оценки вероятности появления критического сбоя используем 
следующие предположения:


\noindent
  \begin{enumerate}[(1)]
\item плотность потока событий, приводящих к~появлению ЛС, 
одинакова по всей площади крис\-тал\-ла микросхемы;
\item разные исходы каждого рассматриваемого события равновероятны;
\item длительность ЛС превышает тактовый период в~синхронной схеме или 
суммарную длительность рабочей и~спейсерной фазы в~СС-схеме. 
\end{enumerate}

  Первое предположение означает, что вероятность ЛС в~какой-то части схемы 
прямо пропорциональна площади, занимаемой этой частью на кристалле.
  
  Второе предположение упрощает анализ со\-сто\-яний при возникновении сбоя. 
Здесь не учитываются некоторые факторы, вли\-я\-ющие на ве\-ро\-ят\-ность разных 
исходов. Например, ве\-ро\-ят\-ность маскирования сбоя в~сигнале другими 
сигналами зависит от вы\-пол\-ня\-емой функции. В~каж\-дом конкретном случае она 
будет иметь разное значение, но в~пер\-вом при\-бли\-же\-нии считаем ее равной~0,5.
  
  Высокопроизводительные синхронные и~самосинхронные схемы имеют конвейерную 
структуру. Отличие СС-кон\-вей\-ера от синхронного аналога состоит 
в~отсутствии глобального сигнала синхронизации и~наличии подсхемы 
управления взаимодействием соседних ступеней. Рисунок~1 иллюстрирует 
конвейер СС-схе\-мы. Здесь Ack и~Req~--- сигналы за\-прос-от\-вет\-но\-го 
взаимодействия ступеней СС-кон\-вей\-ера.
  
\begin{figure*} %fig1
\vspace*{1pt}
\begin{center}  
\mbox{%
\epsfxsize=115.462mm
\epsfbox{sok-1.eps}
}
\end{center}
\vspace*{-9pt}
\Caption{Структура типового СС-конвейера}
\end{figure*}

  Каждая ступень СС-кон\-вей\-ера включает комбинационную часть (КЧ) 
и~выходной регистр (ВР), аналогично синхронному конвейеру, а~также 
индикаторные подсхемы комбинационной час\-ти (ИКЧ) и~регистра (ИВР).  
C-эле\-мен\-ты Маллера~\cite{7-sok} обеспечивают за\-прос-от\-вет\-ное взаимодействие 
соседних ступеней СС-кон\-вей\-ера на основе индикаторных выходов КЧ и~ВР.
  
  Устойчивость конвейера к~кратковременным ЛС определяется 
устойчивостью его составных частей. Число однократных критических ЛС, 
наблюдаемых в~КЧ ступени синхронного конвейера в~течение времени~$T$:
  $$
N_{\mathrm{CPS}} = \lambda T A_{\mathrm{CPS}} P_{\mathrm{CPS},1} P_{\mathrm{CPS},2}\,,
$$
где $\lambda$~--- плотность потока сбоев (число сбоев в~единицу времени на 
единицу площади); $A_{\mathrm{CPS}}$~--- площадь топологии КЧ; $P_{\mathrm{CPS},1}$~--- 
вероятность того, что сбой не замаскируется и~попадет на вход ВР ступени; 
$P_{\mathrm{CPS},2}$~--- вероятность того, что сбой успеет записаться в~ВР. Поскольку 
здесь предполагается равная вероятность исхода для каждого события 
и~считается, что длительность сбоя превышает период тактового сигнала, 
$P_{\mathrm{CPS},1}\hm = 0{,}5$ и~$P_{\mathrm{CPS},2}\hm = 1$. Тогда
$$
N_{\mathrm{CPS}} = 0{,}5\lambda TA_{\mathrm{CPS}}\,.
$$
  
  Аналогичное число однократных критических ЛС, наблюдаемых в~ВР 
ступени синхронного конвейера:
  $$
N_{\mathrm{ORS}} = \lambda TA_{\mathrm{ORS}}P_{\mathrm{ORS},1} P_{\mathrm{ORS},2}\,,
$$
где $A_{\mathrm{ORS}}$~--- площадь топологии ВР; $P_{\mathrm{ORS},1}$~--- вероятность того, что сбой 
не замаскируется КЧ следующей ступени и~попадет на вход ее ВР; $P_{\mathrm{ORS},2}$~--- 
вероятность того, что сбой успеет записаться в~ВР следующей ступени. 
В~соответствии со сделанными предположениями $P_{\mathrm{ORS},1}\hm= 0{,}5$ и~$P_{\mathrm{ORS},2}\hm = 1$. Тогда
$$
N_{\mathrm{ORS}} = 0{,}5\lambda TA_{\mathrm{ORS}}\,.
$$
%
  Поэтому  число критических сбоев в~одной ступени конвейера в~течение 
времени~$T$ равна 
$$
N_S = 0{,}5\lambda T\left(A_{\mathrm{CPS}} + A_{\mathrm{ORS}}\right).
$$


{ \begin{center}  %fig2
 \vspace*{-1pt}
    \mbox{%
\epsfxsize=78.658mm
\epsfbox{sok-2.eps}
}

\vspace*{10pt}

\noindent
{{\figurename~2}\ \ \small{
Индикаторная подсхема
}}
\end{center}}

\vspace*{6pt}

\setcounter{figure}{1}

  

  Аналогичным образом оценивается  число  критических ЛС из табл.~2 
в~ступени СС-кон\-вей\-ера в~течение времени~$T$. 
  
  Наименее чувствительными к~ЛС оказываются ИКЧ и~ИВР ступени 
конвейера. Индикаторная подсхема строится в~виде пирамиды. Рисунок~2 
иллюстрирует структуру индикаторной подсхемы для $N$ парафазных 
сигналов. Ее нижний ярус реализуется на ячейках XOR. С-эле\-мен\-ты на 
остальных ярусах пирамиды объединяют частичные индикаторные сигналы 
в~один индикаторный выход.
  

   
  С-элемент переключается в~то состояние, в~котором находятся все его входы 
одновременно~\cite{7-sok}. В~остальных случаях он хранит свое состояние. 
Однократный ЛС изменяет выход одного элемента в~индикаторной пирамиде. 
Критическим такой ЛС может стать только в~том случае, если он изменил 
состояние выхода С-эле\-мен\-та на вершине пирамиды. В~прочих случаях он 
маскируется остальными элементами индикаторной подсхемы.
  
  Анализ возможных случаев сбоя в~СС-кон\-вей\-ере~[4--6] позволяет 
получить следующие оценки  числа однократных критических сбоев за 
время~$T$:

\noindent
  \begin{equation}
  \left.
  \begin{array}{rl}
     N_{\mathrm{CPST}} & = 0{,}1\lambda T A_{\mathrm{CPST}};\\[6pt]
     N_{\mathrm{CIPST}} &= 0{,}25\lambda TP_{\mathrm{CIP},\mathrm{O}} A_{\mathrm{CIPST}};\\[6pt]
 N_{\mathrm{ORST}} & = 0{,}24\lambda TA_{\mathrm{ORST}};\\[6pt]
N_{\mathrm{RIPST}}  &= 0{,}25\lambda TP_{\mathrm{RIP,O}}A_{\mathrm{RIPST}},
\end{array}
\right\}
\label{e1-sok}
\end{equation}
где $N_{\mathrm{CPST}}$, $N_{\mathrm{CIPST}}$, $N_{\mathrm{ORST}}$ и~$N_{\mathrm{RIPST}}$~--- числа критических ЛС на 
интервале времени~$T$ в~КЧ, ИКЧ, ВР и~ИВР соответственно; $A_{\mathrm{CPST}}$, 
$A_{\mathrm{CIPST}}$, $A_{\mathrm{ORST}}$ и~$A_{\mathrm{RIPST}}$~--- площади их топологической реализации; 
$P_{\mathrm{CIP,O}}$ и~$P_{\mathrm{RIP,O}}$~--- вероятности наблюдения сбоя, появившегося 
в~индикаторной подсхеме, в~ее выходном  
С-эле\-мен\-те для ИКЧ и~ИВР соответственно. 

  Следующие рассуждения позволяют оценить значение вероятности 
$P_{\mathrm{CIP,O}}$ и~$P_{\mathrm{RIP,O}}$. Индикаторная подсхема $N$ парафазных сигналов 
в~комбинационной СС-схе\-ме включает $N$ ячеек XOR и~около $0{,}5N$ 
трехвходовых С-эле\-мен\-тов. Площадь XOR примерно вдвое меньше площади  
С-эле\-мен\-та. Разряд регистра содержит два двухвходовых С-эле\-мен\-та 
и~ячейку XOR в~качестве индикатора~\cite{5-sok}. Разрядность регистра 
ступени конвейера~$M$ в~общем случае отличается от числа парафазных 
сигналов~$N$ в~комбинационной части.
  
  Тогда площади выходных С-эле\-мен\-тов в~ИКЧ ($A_{\mathrm{CIP,O}}$) и~ИВР 
($A_{\mathrm{RIP,O}}$) равны 
  $$
  A_{\mathrm{CIP,O}}\approx \fr{A_{\mathrm{CIPST}}}{4N}\,;\enskip
  A_{\mathrm{RIP,O}}\approx \fr{A_{\mathrm{RIPST}}}{4M}
  $$
и вероятности наблюдения сбоя именно в~них
\begin{equation}
P_{\mathrm{CIP,O}}\approx\fr{1}{4N}\,;\enskip P_{\mathrm{RIP,O}}\approx \fr{1}{4M}\,.
\label{e2-sok}
\end{equation}
  
  Подставляя формулы~(2) в~(1), получаем оценки  числа однократных 
критических ЛС в~течение времени~$T$ в~индикаторных подсхемах 
комбинационной части и~регистра конвейера:
  \begin{align*}
  N_{\mathrm{CIPST}} &=\fr{1}{16N}\,\lambda TA_{\mathrm{CIPST}}\,;\\
  N_{\mathrm{RIPST}}&= \fr{1}{16M}\,\lambda T A_{\mathrm{RIPST}}\,.
  % \label{e3-sok}
  \end{align*}
  
  Пусть площадь топологии ВР ступени синхронного конвейера в~$K$~раз 
меньше площади КЧ. Отношение площадей синхронной и~самосинхронной схем 
соответствует отношению их аппаратных затрат. Аппаратная сложность 
комбинационных СС-схем вместе с~индикаторной подсхемой в~среднем 
в~2,7~раза больше, чем у синхронных аналогов. Регистр СС-кон\-вей\-ера 
сложнее своего синхронного аналога примерно в~1,5~раза.
  
  Для практических величин $M$ и~$N$ вероятностями $PT_{\mathrm{CIPST}}$ 
и~$PT_{\mathrm{RIPST}}$ можно пренебречь. Тогда оценки суммарного  числа 
однократных критических ЛС на интервале времени~$T$ в~ступени 
синхронного ($N_S$) и~самосинхронного кон\-вей\-ера ($N_{\mathrm{ST}}$):
  $$
N_S\approx 0{,}5\lambda T (K + 1)A_{\mathrm{ORS}};\enskip
N_{\mathrm{ST}}\approx0{,}32\lambda TK A_{\mathrm{ORS}},
$$
а их отношение:
$$
K_P= \fr{N_S}{N_{\mathrm{ST}}} \approx \fr{0{,}5(K+1)}{0{,}32 K}\,.
$$
При реальном значении $K$ от 0,5 до~4 СС-кон\-вей\-ер в~2,0--4,7~раз более 
устойчив к~ЛС, чем его синхронный аналог.
  
  Наиболее чувствительной к~ЛС частью цифровой схемы оказывается память, 
включая триггеры и~регистры. В СС-ре\-гист\-ре хранения на  
C-эле\-мен\-тах~\cite{6-sok} сбоеустойчивость может быть повышена путем 
обеспечения возможности восстановления корректного рабочего состояния 
в~сбойном разряде регистра, хранящем антиспейсер, в~рабочей фазе после 
окончания ЛС за счет незначительного усложнения схемы С-эле\-мен\-та 
(рис.~3) и~разряда регистра на его основе (рис.~4)~\cite{5-sok}. 

\setcounter{figure}{2}
\begin{figure*} %fig3
\vspace*{1pt}
\begin{center}  
\mbox{%
\epsfxsize=123.952mm
\epsfbox{sok-3.eps}
}
\end{center}
\vspace*{-9pt}
\Caption{Полустатический C-элемент, защищенный от <<залипания>> в~антиспейсере}
\end{figure*}




  Тогда число однократных критических ЛС на интервале времени~$T$ 
в~регистре ступени СС-кон\-вейера
$$
N_{\mathrm{ORST}} = 0{,}15\lambda TA_{\mathrm{ORST}}
$$
и число однократных критических ЛС в~ступени СС-кон\-вей\-ера уменьшается 
до 
$$
N_{\mathrm{ST}}\approx 0{,}28\lambda TKA_{\mathrm{ORS}}\,,
$$
а отношение $K_P$ числа однократных критических ЛС в~синхронной  
и~самосинхронной схе\-ме улучшается до~2,2--5,4. 

  Самым критичным узлом ступени СС-кон\-вей\-ера является индикаторный 
элемент, фор\-ми\-ру\-ющий сигнал управления регистром предыдущей ступени  
(С-эле\-мен\-ты на рис.~1). Его преждевременное переключение из-за ЛС 
способно вызвать <<зависание>> конвейера. Но его площадь пренебрежимо 
мала по сравнению с~площадью остальных частей СС-кон\-вей\-ера. Поэтому 
его влиянием можно пренебречь. К~тому же его реализация  
DICE-по\-доб\-ным~\cite{7-sok} С-эле\-мен\-том с~синфазными входами 
и~выходами~\cite{5-sok} позволяет кардинально решить проб\-ле\-му 
сбоеустойчивости за счет незначительного увеличения аппаратных затрат.
  
  Таким образом, СС-кон\-вей\-ер, индицирующий антиспейсерное состояние 
как спейсер и~использующий C-эле\-мен\-ты, защищенные от <<залипания>>\linebreak\vspace*{-12pt}

{ \begin{center}  %fig4
 \vspace*{-1pt}
    \mbox{%
\epsfxsize=68.952mm
\epsfbox{sok-4.eps}
}

\end{center}

\noindent
{{\figurename~4}\ \ \small{
Разряд регистра СС-конвейера, защищенный от <<залипания>> в~антиспейсере
}}}

\vspace*{9pt}

\setcounter{figure}{1}



\noindent 
в~антиспейсере, в~разряде регистра ступени, обес\-печивает в~2,2--5,4~раза 
б$\acute{\mbox{о}}$льшую устойчивость к~кратковременным однократным сбоям, чем его 
синхронный аналог. Это предопределяет целесообразность использования  
СС-схем в~качестве базиса для реализации сбоеустойчивых схем управления.
  
\section{Заключение}

  Благодаря двухфазной дисциплине работы и~избыточному кодированию 
сигналов СС-схе\-мы маскируют многие типы ЛС и~вследствие этого становятся 
предпочтительным базисом для реализации сбоеустойчивых систем 
управ\-ления.
  
  Комбинационная часть, регистр и~индикаторная подсхема ступени  
СС-кон\-вей\-ера в~разной степени влияют на уровень сбоеустойчивости 
конвейера. Ступень СС-кон\-вей\-ера устойчива к~90\%~ЛС, возникающих в~ее 
комбинационной части и~только к~76\%~ЛС в~выходном регистре. 
  
  Индикация антиспейсерного состояния как спейсера и~использование  
C-эле\-мен\-та, защищенного от <<залипания>> в~антиспейсере, в~разряде 
регистра ступени СС-кон\-вей\-ера обеспечивают сбоеустойчивость  
СС-кон\-вей\-ера в~2,2--5,4~раза выше уровня сбоеустойчивости синхронного 
конвейера.
  
  Дальнейшая работа будет посвящена исследованию методов и~средств 
обеспечения устойчивости СС-кон\-вей\-ера к~другим типам сбоев и~отказов.
  
{\small\frenchspacing
 {%\baselineskip=10.8pt
 %\addcontentsline{toc}{section}{References}
 \begin{thebibliography}{9}
  \bibitem{1-sok}
  \Au{Викторова В.\,С., Лубков~Н.\,В., Степанянц~А.\,С.} Анализ надежности 
отказоустойчивых управ\-ля\-ющих вы\-чис\-ли\-тель\-ных сис\-тем.~--- М.: ИПУ РАН, 2016. 120~с. 
%{\sf https://www.ipu.ru/sites/default/files/card\_file/VLS. pdf}.
  \bibitem{2-sok}
  \Au{Eaton P., Benedetto~J., Mavis~D., Avery~K., Sibley~M., Gadlage~M., Turflinger~T.} 
Single event transient pulse width measurements using a~variable temporal latch technique~// IEEE 
T.~Nucl. Sci., 2004. Vol.~51. No.\,6. P.~3365--3368. doi: 10.1109/TNS.2004.840020.
  \bibitem{3-sok}
  \Au{Varshavsky V., Kishinevsky~M., Marakhovsky~V., \textit{et. al.}} Self-timed control of 
concurrent processes: The design of aperiodic logical circuits in computers and discrete  
systems.~--- Dordrecht, Netherlands: Kluwer Academic Publs., 1990. 245~p.
  \bibitem{4-sok}
  \Au{Stepchenkov Y.\,A., Kamenskih~A.\,N., Diachenko~Y.\,G., Rogdestvenski~Y.\,V., 
Diachenko~D.\,Y.} Fault-tolerance of self-timed circuits~// 10th Conference (International) on 
Dependable Systems, Services and Technologies.~--- Piscataway, NJ, USA: IEEE, 
2019. P.~41--44. doi: 10.1109/ DESSERT.2019.8770047.
  \bibitem{5-sok}
  \Au{Stepchenkov Y.\,A., Kamenskih~A.\,N., Diachenko~Y.\,G., Rogdestvenski~Y.\,V., 
Diachenko~D.\,Y.} Improvement of the natural self-timed circuit tolerance to short-term soft 
errors~// Advances Science Technology Engineering Systems~J., 2020. Vol.~5. No.\,2. P.~44--56.
  \bibitem{6-sok}
  \Au{Stepchenkov Y., Rogdestvenski~Y., Kamenskih~A., Diachenko~Y., Diachenko~D.} 
Improvement of the quasi delay-insensitive pipeline noise immunity~// 11th Conference 
(International)  on Dependable Systems, Services and Technologies.~--- Piscataway, 
NJ, USA: IEEE, 2020. P.~47--51.

 
  \bibitem{8-sok} %7
  \Au{Monnet Y., Renaudin~M., Leveugle~R.} Hardening techniques against transient faults for 
asynchronous circuits~// 11th IEEE On-Line Testing Symposium (International).~--- Piscataway, 
NJ, USA: IEEE, 2005. P.~129--134. doi: 10.1109/IOLTS.2005.30.

 \bibitem{7-sok} %8
\Au{Danilov I.\,A., Gorbunov~M.\,S., Shnaider~A.\,I., \textit{et al.}} On board electronic devices safety provided by DICE-based 
Muller C-elements~// Acta Astronaut., 2018. Vol.~150. P.~28--32.
{\looseness=1

}
\end{thebibliography}

 }
 }

\end{multicols}

\vspace*{-10pt}

\hfill{\small\textit{Поступила в~редакцию 19.09.21}}

\vspace*{6pt}

%\pagebreak

%\newpage

%\vspace*{-28pt}

\hrule

\vspace*{2pt}

\hrule

%\vspace*{-2pt}

\def\tit{THE ELECTRONIC COMPONENT BASE\\ OF~FAILURE RESILIENCE DIGITAL CIRCUITS}


\def\titkol{The electronic component base of~failure resilience digital circuits}


\def\aut{I.\,A.~Sokolov$^1$, Yu.\,A.~Stepchenkov$^1$, Yu.\,G.~Diachenko$^1$, 
Yu.\,V.~Rogdestvenski$^1$,\\ and~A.\,N.~Kamenskih$^2$}

\def\autkol{I.\,A.~Sokolov, Yu.\,A.~Stepchenkov, Yu.\,G.~Diachenko, et al.} 
%Yu.\,V.~Rogdestvenski$^1$, and~A.\,N.~Kamenskih}

\titel{\tit}{\aut}{\autkol}{\titkol}

\vspace*{-15pt}


\noindent
$^1$Federal Research Center ``Computer Science and Control'' of the Russian 
Academy of Sciences, 44-2~Vavilov\linebreak
\hphantom{$^1$}Str., Moscow 119333, Russian Federation

\noindent
$^2$Perm National Research Polytechnic University, 29~Komsomol Prosp., Perm 
614990, Russian Federation


\def\leftfootline{\small{\textbf{\thepage}
\hfill INFORMATIKA I EE PRIMENENIYA~--- INFORMATICS AND
APPLICATIONS\ \ \ 2021\ \ \ volume~15\ \ \ issue\ 4}
}%
 \def\rightfootline{\small{INFORMATIKA I EE PRIMENENIYA~---
INFORMATICS AND APPLICATIONS\ \ \ 2021\ \ \ volume~15\ \ \ issue\ 4
\hfill \textbf{\thepage}}}

\vspace*{2pt} 
  



\Abste{The article presents the research of self-timed and synchronous 
circuits in terms of resilience to soft errors 
which can cause disruptions in the control system's 
operation of complex technical device. The use of a~fail-resilient 
self-timed code is proposed, which considers the antispacer state as the second spacer state. This approach increases 
the self-timed circuit's failure resilience level. In the first approximation, quantitative estimates show that the self-timed 
pipeline has a better failure resilience than the synchronous counterparts by 2.0--4.7 times. The use of modified  
C-element to implement the pipeline register bit increases this advantage to 2.2--5.4~times. Due to this, self-timed 
circuits are the preferred basis of failure resilient control systems implementation for complex technical equipment.}

\KWE{synchronous circuits; self-timed circuits; soft error; failure resilience; pipeline; transition completion indication; 
probability evaluation}

\DOI{10.14357/19922264210409}

%\vspace*{-20pt}

%\Ack
%\noindent




%\vspace*{6pt}

  \begin{multicols}{2}

\renewcommand{\bibname}{\protect\rmfamily References}
%\renewcommand{\bibname}{\large\protect\rm References}

{\small\frenchspacing
 {%\baselineskip=10.8pt
 \addcontentsline{toc}{section}{References}
 \begin{thebibliography}{9}
  \bibitem{1-sok-1}
\Aue{Viktorova, V.\,S., N.\,V.~Lubkov, and A.\,S.~Stepanyants.} 2016. \textit{Analiz 
nadezhnosti otkazoustoychivykh upravlyayushchikh vychislitel'nykh sistem}  
[Fault-tolerant control computer systems' reliability analysis]. Moscow: IPU RAN. 
120~p.
  \bibitem{2-sok-1}
\Aue{Eaton, P., J.~Benedetto, D.~Mavis, K.~Avery, M.~Sibley, M.~Gadlage, and 
T.~Turflinger.} 2004. Single event transient pulse width measurements using 
a~variable temporal latch technique. \textit{IEEE T.~Nucl. Sci.} 51(6):3365--3368. 
doi: 10.1109/TNS.2004.840020.
  \bibitem{3-sok-1}
\Aue{Varshavsky, V., M.~Kishinevsky, V.~Marakhovsky, \textit{et al.}} 1990.  
\textit{Self-timed control of concurrent processes:
The design of aperiodic logical circuits in computers and discrete  
systems}. Dordrecht, The Netherlands: 
Kluwer Academic Publs. 245~p.
  \bibitem{4-sok-1}
\Aue{Stepchenkov, Y.\,A., A.\,N.~Kamenskih, Y.\,G.~Diachenko, 
Y.\,V.~Rogdestvenski, and D.\,Y.~Diachenko.} 2019. Fault-tolerance of self-timed 
circuits. \textit{10th Conference (International) on Dependable Systems, Services, 
and Technologies Proceedings}. Piscataway, NJ: IEEE. 41--44. doi: 
10.1109/ DESSERT.2019.8770047.
  \bibitem{5-sok-1}
\Aue{Stepchenkov, Y.\,A., A.\,N.~Kamenskih, Y.\,G.~Diachenko, 
Y.\,V.~Rogdestvenski, and D.\,Y.~Diachenko.} 2020. Improvement of the natural 
self-timed circuit tolerance to short-term soft errors. \textit{Advances Science 
Technology Engineering Systems~J.} 5(2):44--56.
  \bibitem{6-sok-1}
\Aue{Stepchenkov, Y., Y.~Rogdestvenski, A.~Kamenskih, Y.~Diachenko, and 
D.~Diachenko.} 2020. Improvement of the quasi delay-insensitive pipeline noise 
immunity. \textit{11th Conference (International) on Dependable Systems, Services, 
and Technologies Proceedings}. Piscataway, NJ: IEEE. 47--51.

\bibitem{8-sok-1}
\Aue{Monnet, Y., M.~Renaudin, and R.~Leveugle.} 2005. Hardening techniques 
against transient faults for asynchronous circuits. \textit{11th International On-Line Testing Symposium 
Proceedings}. Piscataway, NJ: IEEE. 129--134. doi: 10.1109/ IOLTS.2005.30.
  \bibitem{7-sok-1}
\Aue{Danilov, I.\,A., M.\,S.~Gorbunov, A.\,I.~Shnaider, \textit{et al.}} 2018. On board electronic devices safety 
provided by DICE-based Muller C-elements. \textit{Acta Astronaut.} 150:28--32.

\end{thebibliography}

 }
 }

\end{multicols}

\vspace*{-12pt}

\hfill{\small\textit{Received September 19, 2021}}

\pagebreak

\Contr

\noindent
\textbf{Sokolov Igor A.} (b.\ 1954)~--- Doctor of Science in technology, Academician of RAS, 
Director, Federal Research Center ``Computer Science and Control'' of the Russian Academy of 
Sciences, 44-2~Vavilov Str., Moscow 119133, Russian Federation; \mbox{isokolov@ipiran.ru}

\vspace*{3pt}

\noindent
\textbf{Stepchenkov Yuri A.} (b.\ 1951)~--- Candidate of Science (PhD) in technology, leading 
scientist, Federal Research Center ``Computer Science and Control'' of the Russian Academy of 
Sciences, 44-2~Vavilov Str., Moscow 119133, Russian Federation; 
\mbox{YStepchenkov@ipiran.ru}

\vspace*{3pt}

\noindent
\textbf{Diachenko Yuri G.} (b.\ 1958)~--- Candidate of Science (PhD) in technology, senior 
scientist, Federal Research Center ``Computer Science and Control'' of the Russian Academy of 
Sciences, 44-2~Vavilov Str., Moscow 119133, Russian Federation; \mbox{diaura@mail.ru}

\vspace*{3pt}

\noindent
\textbf{Rogdestvenski Yuri V.} (b.\ 1952)~--- Candidate of Science (PhD) in technology, leading 
scientist, Federal Research Center ``Computer Science and Control'' of the Russian Academy of 
Sciences, 44-2~Vavilov Str., Moscow 119333, Russian Federation;  \mbox{YRogdest@ipiran.ru}

\vspace*{3pt}

\noindent
\textbf{Kamenskih Anton N.} (b.\ 1991)~--- Candidate of Science (PhD) in technology, associated 
professor, Perm National Research Polytechnic University, 29~Komsomol Prosp., Perm 614990, 
Russian Federation; \mbox{antoshkinoinfo@yandex.ru}


  
\label{end\stat}

\renewcommand{\bibname}{\protect\rm Литература}  %9
\def\stat{dukova}

\def\tit{О ПОИСКЕ МАКСИМАЛЬНЫХ ЧАСТЫХ И~МИНИМАЛЬНЫХ НЕЧАСТЫХ НАБОРОВ ПРОИЗВЕДЕНИЯ ЧАСТИЧНЫХ ПОРЯДКОВ}

\def\titkol{О поиске максимальных частых и~минимальных нечастых наборов произведения частичных порядков}

\def\aut{Н.\,А.~Драгунов$^1$, Е.\,В.~Дюкова$^2$}

\def\autkol{Н.\,А.~Драгунов, Е.\,В.~Дюкова}

\titel{\tit}{\aut}{\autkol}{\titkol}

\index{Драгунов Н.\,А.}
\index{Дюкова Е.\,В.}
\index{Dragunov N.\,A.}
\index{Djukova E.\,V.}


%{\renewcommand{\thefootnote}{\fnsymbol{footnote}} \footnotetext[1]
%{Работа выполнена при поддержке Министерства науки и~высшего образования Российской Федерации (проект 
%075-15-2020-799).}}


\renewcommand{\thefootnote}{\arabic{footnote}}
\footnotetext[1]{Федеральный исследовательский центр <<Информатика 
и~управ\-ле\-ние>> Российской академии наук, \mbox{nikitadragunovjob@gmail.com}}
\footnotetext[2]{Федеральный исследовательский центр <<Информатика и~управ\-ле\-ние>> 
Российской академии наук, \mbox{edjukova@mail.ru}}

\vspace*{-3pt}




\Abst{Исследованы актуальные вопросы снижения временных затрат, возникающие при 
логическом анализе данных с~элементами из декартова произведения конечных час\-тич\-но 
упорядоченных множеств. Для задачи поиска по базе транзакций максимальных час\-тых и~минимальных 
нечастых наборов произведения час\-тич\-ных порядков предложен оригинальный метод, 
основанный на решении слож\-ной дискретной задачи, называемой дуализацией 
над произведением час\-тич\-ных порядков. Метод представляет собой синтез двух других 
известных методов, один из которых достаточно очевиден, а~другой использует идею 
инкрементального пе\-ре\-чис\-ле\-ния искомых наборов и~поэтому пред\-став\-ля\-ет 
в~основном тео\-ре\-ти\-че\-ский интерес. Проведено экспериментальное исследование предложенного 
подхода к~решению рас\-смат\-ри\-ва\-емой задачи в~случае произведения конечных цепей,
 выявлены условия его эф\-фек\-тив\-ности и~для проводимого анализа данных показана 
 це\-ле\-со\-об\-раз\-ность применения асимптотически оптимальных алгоритмов дуализации 
 над произведением час\-тич\-ных порядков.}

\KW{максимальные час\-тые наборы; минимальные не\-час\-тые наборы; дуализация над 
произведением час\-тич\-ных порядков; асимп\-то\-ти\-чески оптимальный алгоритм дуализации}

\DOI{10.14357/19922264220112}
  
%\vspace*{-4pt}


\vskip 10pt plus 9pt minus 6pt

\thispagestyle{headings}

\begin{multicols}{2}

\label{st\stat}

    \section{Введение}
    
    Рас\-смат\-ри\-ва\-емая задача анализа данных занимает важ\-ное мес\-то в~об\-ласти 
    информационного поиска и~в~случае бинарных данных ставится сле\-ду\-ющим образом~\cite{4}.
    
    Дано некоторое множество элементов~$V$. Подмножества $X \hm\subseteq V$ называются наборами. Пусть~$D$~--- 
    база данных, содержащая некоторые, не обязательно различные, наборы. Наборы, 
    содержащиеся в~$D$, называются транз\-ак\-ци\-ями. Под частотой набора~$\nu(X)$ понимается доля транз\-ак\-ций в~$D$, 
    содержащих~$X$. Если $\nu(X) \hm\geq s$, $s \hm\in \left[0, 1\right]$, то набор~$X$ называется $s$-час\-тым, 
    иначе он называется $s$-не\-час\-тым. Если набор частый и~он не содержится ни в~каком другом 
    час\-том наборе, то такой набор называется максимальным час\-тым. Если набор не\-час\-тый 
    и~при этом он не содержит в~себе никакого другого не\-час\-то\-го набора, то такой набор 
    называется минимальным нечастым. Требуется найти все максимальные час\-тые и~минимальные не\-час\-тые 
    наборы при заданном~$s$.
    
    Рас\-смат\-ри\-ва\-емая задача имеет много важных приложений, одним из которых является 
    нахождение ассоциативных правил в~базах данных. В~случае бинарных данных ассоциативное правило~---
     это упорядоченная пара $ \left( X, Y \right)$ непересекающихся подмножеств множества~$V$, обо\-зна\-ча\-емая 
     $X \hm\Rightarrow Y$. Поддержкой правила $X \hm\Rightarrow Y$ называется час\-то\-та набора $Z\hm = X \cup Y$.
      Достоверностью правила $X\hm \Rightarrow Y$ называется доля транзакций, со\-дер\-жа\-щих~$Y$, 
      среди всех транзакций, содержащих~$X$. Требуется \mbox{найти} все ассоциативные правила, 
      удовле\-тво\-ря\-ющие заданным минимальной поддержке $s\hm \in [0, 1]$ и~минимальной 
      достоверности $c \hm\in [0, 1]$.  Впервые задача нахождения ассоциативных правил
       была поставлена в~\cite{1}, где она формулировалась как задача анализа по\-тре\-би\-тель\-ской корзины.

    В случае небинарных данных каждый элемент из~$V$ имеет некоторое множество чис\-ло\-вых значений 
    и~вместо наборов элементов рас\-смат\-ри\-ва\-ют\-ся наборы их значений.

    Поиск ассоциативных правил осуществляется в~два этапа. 
    На первом этапе находятся частые наборы, на втором этапе из найденных час\-тых 
    наборов формируются ассоциативные правила. При формировании правил на втором 
    этапе фактически возникает задача поиска $t$-не\-час\-тых наборов, где $t\hm > s/c$.
    
    С ростом размерности современных баз данных находить все час\-тые и~не\-час\-тые 
    наборы становится неэффективно как по времени, так и~по памяти в~силу 
    экспоненциального рос\-та чис\-ла таких наборов. Одно из решений данной проблемы 
    заключается в~поиске только максимальных час\-тых наборов и~только минимальных 
    нечастых наборов, что позволяет компактно хранить информацию о~всех час\-тых и~не\-час\-тых 
    наборах соответственно. 
    
    
    В~\cite{9} рас\-смот\-ре\-на задача поиска множеств максимальных час\-тых наборов~$X_{\max}$ 
    и~минимальных не\-час\-тых наборов~$Y_{\min}$ в~данных, пред\-став\-лен\-ных в~виде декартова 
    произведения час\-тич\-но упорядоченных множеств. Показано, что в~этом случае 
    при построении тре\-бу\-емых наборов возникают соответственно задача поиска 
    максимальных независимых элементов час\-тич\-ных порядков и~задача поиска минимальных 
    независимых элементов час\-тич\-ных порядков.  Каж\-дая из этих задач называется 
    дуализацией над произведением час\-тич\-ных порядков~\cite{8}. Обе задачи относятся к~одним 
    из цент\-раль\-ных труд\-но\-ре\-ша\-емых пе\-ре\-чис\-ли\-тель\-ных задач дис\-крет\-ной математики.
    
    Существует достаточно очевидный способ поиска максимальных час\-тых и~минимальных
     не\-час\-тых наборов произведения час\-тич\-ных порядков, основанный на по\-сле\-до\-ва\-тель\-ном 
     по\-стро\-ении указанных множеств. Одно из множеств ищется, например, алгоритмом Apriori~\cite{2},
      второе множество получается путем дуализации первого. 
      В~настоящей работе показано, что метод эффективен только в~случае, когда чис\-ло час\-тых 
      наборов существенно меньше или, наоборот, существенно больше чис\-ла не\-час\-тых наборов. 
      В~\cite{9} предложена идея со\-вмест\-но\-го пе\-ре\-чис\-ле\-ния~$X_{\max}$ и~$Y_{\min}$ с~использованием
       инкрементального алгоритма дуализации из~\cite{14}, которая автором экспериментально 
       не исследована.
    
    Основной результат настоящей работы~--- разработка нового подхода к~решению 
    поставленной задачи, который является синтезом последовательного и~совместного подходов. 
    
    Экспериментальные исследования, проведенные в~настоящей работе для случая
     произведения цепей, свидетельствуют о~том, что предложенный по\-сле\-до\-ва\-тель\-но-со\-вмест\-ный 
     метод наиболее эффективен в~случае, когда мощ\-ность множества час\-тых наборов примерно 
     равна мощ\-ности множества не\-час\-тых наборов.
     
     \vspace*{-6pt}
     
    
    \section{Постановка задачи поиска максимальных частых 
    и~минимальных нечастых наборов произведения частичных порядков}
    
         \vspace*{-2pt}
    
    Пусть $\mathcal{P} = \mathcal{P}_1 \times \dots \times \mathcal{P}_n$~--- 
    де\-кар\-то\-во произведение час\-тич\-но упорядоченных множеств. Элементы~$\mathcal{P}$ называются наборами. 
    На множестве~$\mathcal{P}$ определяется отношение частичного порядка~$\preceq$ сле\-ду\-ющим образом: 
    если $p \hm= (p_1, \dots, p_n) \hm\in \mathcal{P}$ и~$q \hm= (q_1, \dots, q_n)\hm \in \mathcal{P}$, 
    то $ p \hm\preceq q$ в~$ \mathcal{P}\hm \Leftrightarrow p_1 \hm\preceq q_1$ 
    в~$\mathcal{P}_1, \dots, p_n \hm\preceq q_n$ в~$ \mathcal{P}_n$.
    
    Пусть $\mathcal{D} (\mathcal{P})$~--- некоторая со\-во\-куп\-ность
     наборов из~$\mathcal{P}$, называемая базой данных. Наборы, на\-хо\-дя\-щи\-еся в~базе 
     данных $\mathcal{D} (\mathcal{P})$, необязательно по\-пар\-но раз\-лич\-ны и~называются транзакциями. 
     
    Введем обозначения: 
    $\vert \mathcal{D} (\mathcal{P}) \vert$~--- чис\-ло транз\-ак\-ций в~$\mathcal{D} (\mathcal{P})$; 
    $\mathcal{S}_\mathcal{D}(p)$~--- число транз\-ак\-ций в~$\mathcal{D} (\mathcal{P})$, 
    сле\-ду\-ющих за $p \hm\in \mathcal{P}$; $s \hm\in [0, 1]$. 
    
    \smallskip
    
    \noindent
    \textbf{Определение~1.}\
     Набор $p \in \mathcal{P}$ называется $s$-час\-тым, 
     если $\mathcal{S}_\mathcal{D}(p) / \vert \mathcal{D} (\mathcal{P}) \vert \hm\geq s$. Иначе набор~$p$ 
     называется $s$-не\-час\-тым.
    
    \smallskip
    
    \noindent
    \textbf{Определение~2.}\
    Набор $p \in \mathcal{P}$ называется максимальным $s$-час\-тым, если 
    он $s$-час\-тый и~никакой сле\-ду\-ющий за ним набор~$z$, $z\hm \neq p$, не является $s$-час\-тым.

    
    \smallskip
    
    \noindent
    \textbf{Определение~3.}\
    Набор $p \in \mathcal{P}$ называется минимальным $s$-не\-час\-тым, если он $s$-не\-час\-тый 
    и~никакой пред\-шест\-ву\-ющий ему набор~$z$, $z \hm\neq p$, не является $s$-не\-час\-тым.


\smallskip
    
    Далее вместо $s$-частый ($s$-не\-час\-тый) набор будем писать час\-тый (не\-час\-тый) набор. 
    Множество всех максимальных час\-тых наборов будем обозначать как $X_{\max}$, 
    а~множество всех минимальных не\-час\-тых наборов как $Y_{\min}$.
    
    Пусть $R \subset \mathcal{P}$, $R^+\hm = \{ x \in \mathcal{P} \vert \exists\, a \hm\in R, a \hm\preceq x \}$, 
    $R^- \hm= \{ x \hm\in \mathcal{P} \vert \exists\, a \hm\in R, x \hm\preceq a \}$.


    \noindent
    \textbf{Определение~4.}\
     Множество $I(R^+)$, со\-сто\-ящее из всех максимальных элементов множества~$\mathcal{P} \setminus R^+$, 
     называется максимальным независимым от~$R$.

\smallskip


   \noindent
    \textbf{Определение~5.}\
     Множество $I(R^-)$, со\-сто\-ящее из всех минимальных элементов множества~$\mathcal{P} \setminus R^-$, 
     называется минимальным независимым от~$R$.

\smallskip
    
    Каждая из задач построения $I(R^+)$ и~$I(R^-)$ 
    при заданном множестве~$R$ называется задачей дуализации над произведением час\-тич\-ных порядков.
    
    \smallskip

    \noindent
    \textbf{Утверждение~1.}\
    Если $X \hm\subset X_{\max}$, а~$y \hm\in I(X^-)$~--- не\-час\-тый набор, 
    то~$y$~--- минимальный не\-час\-тый набор.

\smallskip    
    
    \noindent
    Д\,о\,к\,а\,з\,а\,т\,е\,л\,ь\,с\,т\,в\,о\,.\  \ 
    Пусть $y \hm\notin I(X_{\max}^-)$. Так как~$y$~--- 
    нечастый набор, то в~$\mathcal{P} \setminus X^{-}_{\max}$ найдется минимальный не\-час\-тый набор~$x$ 
    такой, что $x\hm \neq y$ и~$x \hm\preceq y$. Из того, что $\mathcal{P} \setminus X^{-}_{\max} 
    \hm\subseteq \mathcal{P} \setminus X^-$, следует, что $x\hm \in \mathcal{P} \setminus X^-$, 
    что противоречит условию $y \hm\in I(X^-)$.

\smallskip

\noindent
\textbf{Утверждение~2.}\
    Пусть $X \hm\subseteq X_{\max}$, $Y\hm \subseteq Y_{\min}$. 
    Тогда $I(X^-) \hm= Y$ в~том и~только в~том случае, когда $X \hm= X_{\max}$ и~$Y \hm= Y_{\min}$.


\smallskip


  \noindent
    Д\,о\,к\,а\,з\,а\,т\,е\,л\,ь\,с\,т\,в\,о\,.\  \
    Пусть $X\! \subset\! X_{\max}, x \hm\in X_{\max}\!\setminus\!X$.
     Так как множество~$X_{\max}$~--- антицепь, то $x \hm\notin X^-$. 
     Следовательно, $x \hm\in \mathcal{P} \setminus X^{-}$.
      Но тогда существует элемент $ q \hm\in I(X^-) : q \preceq x$, 
      который является час\-тым. Однако во множестве~$Y$ частых наборов нет; следовательно, $I(X^-) \hm\neq Y$. 
      Если же $X \hm= X_{\max}$, то $I(X^-) \hm= Y_{\min}$. Таким образом, $I(X^-) \hm= Y$ тогда и~только
       тогда, когда $X \hm= X_{\max}$ и~$Y\hm = Y_{\min}$.


    
    \section{Методы построения множеств~$X_{\max}$ и~$Y_{\min}$}

    \subsection{Последовательное перечисление $X_{\max}$~и~$Y_{\min}$}

    Достаточно очевиден поиск~$X_{\max}$ и~$Y_{\min}$ при заданной $\mathcal{D} (\mathcal{P})$ 
    путем последовательного по\-стро\-ения множеств~$X_{\max}$ и~$Y_{\min}$. 
    Данный поиск осуществляется в~два этапа. На первом этапе находятся все максимальные частые 
    наборы~$X_{\max}$, например алгоритмом Apriori~\cite{2}. На втором этапе  используется свойство 
    двойственности $I \left(X_{\max}^- \right)\hm = Y_{\min}$. 
    Минимальные нечастые наборы~$Y_{\min}$ находятся путем дуализации найденного на первом этапе 
    множества~$X_{\max}$. Аналогично можно сначала искать~$Y_{\min}$ алгоритмом Apriori, а~затем 
    искать~$X_{\max}$ путем дуализации~$Y_{\min}$.

    Очевидно, что данный подход будет проявлять себя наилучшим образом в~случаях, когда 
    алгоритм Apriori или его модификации могут найти одно из искомых множеств существенно
     быст\-рее, чем другое множество, например когда мощ\-ность~$X_{\max}$ 
     существенно меньше (больше) мощ\-ности~$Y_{\min}$.
    
    \subsection{Совместное перечисление $X_{\max}$ и~$Y_{\min}$}

    В~\cite{9} предложена идея совместного перечисления множеств~$X_{\max}$ и~$Y_{\min}$. 
    На первом шаге рас\-смат\-ри\-ва\-ет\-ся некоторый случайный набор $q \hm\in \mathcal{P}$. Если $q$~--- 
    час\-тый набор, то ищется максимальный час\-тый набор, сле\-ду\-ющий за~$q$, 
    который пополняет множество $X \hm\subseteq X_{\max}$. Если $q$~---
     не\-час\-тый набор, то ищется минимальный не\-час\-тый набор, пред\-шест\-ву\-ющий~$q$, 
     который пополняет множество $Y \hm\subseteq Y_{\min}$. Пусть на шаге~$i$ ($i\hm \geq 1$) 
     построены множества $X \hm\subseteq X_{\max}$ и~$Y \hm\subseteq Y_{\min}$. Если $X \hm\neq \varnothing$, 
     $Y \hm= \varnothing$, то ищется набор~$q$ такой, что $q \hm\npreceq x, \forall x \hm\in X$. Если 
     $X \hm= \varnothing$, $Y \hm\neq \varnothing$, то ищется набор~$q$ такой, что 
     $q \hm\nsucceq y, \forall y \hm\in Y$. Если же и~$X \hm\neq \varnothing$, и~$Y \hm\neq \varnothing$, 
     то ищется набор~$q$ такой, что $q \hm\npreceq x, \forall x \hm\in X, q \hm\nsucceq y, \forall y \hm\in Y$.
      Затем, аналогично первому шагу, находится максимальный частый или минимальный нечастый набор. 
      Однако в~\cite{9} идея совместного перечисления искомых множеств экспериментально 
      не исследована и~не предложены конкретные указания по воз\-мож\-ной ее реализации.
    
    Алгоритм, основанный на совместном пе\-ре\-чис\-ле\-нии множеств~$X_{\max}$ и~$Y_{\min}$,
     реализован в~на\-сто\-ящей работе. Алгоритм строит две последовательности: $X_1 \hm\subset X_2 
     \subset \dots \subset X_{\max}$, $Y_1\hm \subset Y_2 \subset \dots \subset Y_{\min}$. 
     На первом шаге $X_1 \hm= \{x\}$, $Y_1 \hm= \{y\}$, где~$x$ и~$y$ ищутся алгоритмом Apriori.
      На шаге $i \hm+ 1$ ($i\hm \geq 1$) строится либо~$I(X^{-}_{i})$, либо~$I(Y^{+}_{i})$. Пусть на 
      шаге $i \hm+ 1$ ($i \hm\geq 1$) построено множество~$I(X^{-}_{i})$. 
      Согласно утверждениям~1 и~2, множество~$I(X^{-}_{i})$ либо не содержит час\-тых наборов 
      и~совпадает с~множеством~$Y_{\min}$ (в~этом случае $X_i \hm= X_{\max}$ 
      и~алгоритм заканчивает работу), либо~$I(X^{-}_{i})$ содержит как час\-тые, так и~не\-час\-тые наборы. 
      Каждый нечастый набор из~$I(X^{-}_{i})$ является минимальным не\-час\-тым и~пополняет множество~$Y_{i}$, 
      формируя в~результате множество~$Y_{i+1}$. Для каждого час\-то\-го набора находится один содержащий 
      его максимальный час\-тый набор путем последовательного увеличения текущего 
      частого набора в~лексикографическом порядке, который пополняет множество~$X_{i}$, 
      формируя в~результате множество~$X_{i+1}$.
      
    В~экспериментальной части работы (см.\ разд.~4) рас\-смот\-рен случай произведения цепей. 
    Задача дуализации решается с~помощью асимптотически оптимального алгоритма дуализации
     цепей \mbox{RUNC-M}+~\cite{7}. Асимптотически оптимальные алгоритмы дуализации 
     являются лидерами по ско\-рости счета~\cite{6}.

    Очевидно, что время работы совместного алгоритма в~основном зависит от чис\-ла
     минимальных не\-час\-тых и~максимальных час\-тых наборов. На\linebreak каж\-дой новой 
     итерации происходит дуализация\linebreak все б$\acute{\mbox{о}}$льших по мощ\-ности множеств~$X$ или~$Y$.\linebreak 
     Если число итераций становится достаточно\linebreak большим, то ско\-рость работы совместного 
     перечисления существенно снижается, что делает его практически неприменимым для 
     задач большой раз\-мер\-ности.
     { %\looseness=1
     
     }

    \subsection{Последовательно-совместное перечисление~$X_{\max}$ и~$Y_{\min}$}

    Предлагается следующий итеративный метод, который синтезирует идеи последовательного
     и~совместного методов, описанных выше. Положим $X_0 \hm= \varnothing$. 
     Строится одна по\-сле\-до\-ва\-тель\-ность $X_1 \hm\subset X_2 \hm\subset \dots \subset X_{\max}$. 
     На первом шаге $X_1\hm = \{x\}$, где $x$ ищется алгоритмом Apriori. На шаге $i \hm+ 1$ ($i \hm\geq 1$) 
     решается задача дуализации множества $X_{i} \setminus X_{i-1}$.

    
    
   \setcounter{figure}{1}
    \begin{figure*}[b] %fig2
  \vspace*{12pt}
  \begin{center}  
    \mbox{%
\epsfxsize=163mm
\epsfbox{duk-2.eps}
}

\end{center}
\vspace*{-9pt}
    \Caption{Зависимость времени работы алгоритмов от суммы мощностей множеств~$X_{\max}$ и~$Y_{\min}$ 
    для случая~1~(\textit{а}) и~2~(\textit{б}):
    \textit{1}~--- по\-сле\-до\-ва\-тель\-но-со\-вмест\-ный;
    \textit{2}~--- последовательный; \textit{3}~--- совместный; \textit{4}~--- Apriori}
    \label{12}
    \end{figure*}
     
    Пусть множество~$D$ есть результат дуализации $X_{i} \hm\setminus X_{i-1}$. Согласно утверждению~1, 
    множество~$D$ содержит частые наборы. Для каждого час\-то\-го набора из~$D$ 
    находится один содержащий его максимальный час\-тый набор путем последовательного 
    увеличения текущего час\-то\-го набора в~лексикографическом порядке. Все найденные максимальные
     частые наборы, которых нет в~множестве~$X_{i}$, до\-бав\-ля\-ют\-ся к~$X_{i}$, 
     и~таким образом формируется~$X_{i+1}$. Если же все найденные частые наборы уже содержатся в~$X_{i}$, 
     то решается задача дуализации множества~$X_{i}$. Если в~$I(X^{-}_{i})$ нет частых наборов, 
     то $I(X^{-}_{i})\hm = Y_{\min}$, $X_i \hm= X_{\max}$ и~алгоритм завершает работу. 
     Иначе для каждого частого набора из~$I(X^{-}_{i})$ находится один содержащий его максимальный 
     час\-тый набор, который пополняет множество~$X_{i}$, формируя в~результате множество~$X_{i+1}$.

    \section{Экспериментальное исследование}
    
    Рас\-смат\-ри\-вал\-ся случай данных, пред\-став\-лен\-ных в~виде произведения цепей мощ\-ности~5. 
    Для\linebreak таких данных проводился поиск максимальных час\-тых и~минимальных нечастых 
    наборов сле\-ду\-ющи\-ми методами: алгоритмом Apriori, модифицированным для случая 
    цепей; последовательным \mbox{методом}; совместным методом; по\-сле\-до\-ва\-тель\-но-со\-вмест\-ным методом.
    
    Все методы реализованы на языке Python~3. 
    Задача дуализации решалась алгоритмом дуализации цепей RUNC-M+~\cite{7}. 
    Эксперименты проведены на случайных базах данных различной раз\-мер\-ности. 
    Можно выделить два сле\-ду\-ющих случая соотношения мощностей множеств всех час\-тых и~не\-час\-тых наборов.
    \begin{description}
    \item[Случай 1:] мощ\-ность множества частых наборов примерно рав\-на мощ\-ности множества нечастых наборов.
    \item[Случай 2:] мощ\-ность множества частых наборов существенно меньше (больше) мощ\-ности множества 
    не\-час\-тых наборов.
    \end{description}
    
    Описанные случаи схематично изображены на рис.~1. 

    Графики зависимости времени работы тестируемых методов 
    от мощ\-ности множеств~$X_{\max}$ и~$Y_{\min}$ приведены на рис.~2.
    
    

    

    Нетрудно видеть, что в~случае~1 лучше работает по\-сле\-до\-ва\-тель\-но-со\-вмест\-ный алгоритм: 
    множества час\-тых и~не\-час\-тых наборов имеют примерно одинаковую мощ\-ность, 
    поэтому быст\-рее будет обрабатывать их по\-сле\-до\-ва\-тель\-но-со\-вмест\-ным методом. В~случае~2 
    быст\-рее работает последовательный алгоритм: быст\-рее найти множество максимальных час\-тых наборов, 
    обработав множество час\-тых наборов, и~дуализировать результат. Время поиска множеств~$X_{\max}$ 
    и~$Y_{\min}$ совместным методом и~модифицированным алгоритмом Apriori рас\-тет существенно 
    быст\-рее времени поиска по\-сле\-до\-ва\-тель\-но-со\-вмест\-ным методом в~обоих случаях.
    
    { \begin{center}  %fig1
 \vspace*{9pt}
    \mbox{%
\epsfxsize=67.963mm
\epsfbox{duk-1.eps}
}

\end{center}

\noindent
{{\figurename~1}\ \ \small{
Два случая соотношения мощностей множеств час\-тых и~не\-час\-тых наборов
}}}

%\vspace*{6pt}


    \section{Заключение}
    
Рас\-смот\-ре\-на задача поиска максимальных час\-тых и~минимальных не\-час\-тых наборов в~данных, 
представленных в~виде декартова произведения час\-тич\-ных порядков. Актуальны вопросы 
снижения временн$\acute{\mbox{ы}}$х затрат, возникающих при реализации методов нахождения искомых наборов.
 Разработан новый подход к~по\-стро\-ению множества максимальных частых наборов~$X_{\max}$ и~множества 
 минимальных не\-час\-тых наборов~$Y_{\min}$, пред\-став\-ля\-ющий собой синтез двух ранее известных 
 подходов: последовательного и~со\-вмест\-но\-го (первый достаточно очевиден, идея второго предложена в~\cite{9}). 
 Сложность последовательного, совместного и~пред\-ла\-га\-емо\-го по\-сле\-до\-ва\-тель\-но-со\-вмест\-но\-го поиска 
 обуслов\-ле\-на, в~том чис\-ле, не\-об\-хо\-ди\-мостью рас\-смат\-ри\-вать в~процессе поиска 
 труд\-но\-ре\-ша\-емую пе\-ре\-чис\-ли\-тель\-ную задачу дис\-крет\-ной математики, на\-зы\-ва\-емую дуализацией 
 над произведением час\-тич\-ных порядков.

Для случая, когда данные пред\-став\-ле\-ны в~виде произведения конечных цепей, 
приведены результаты экспериментального срав\-не\-ния названных подходов, а~так\-же независимого 
способа \mbox{по\-стро\-ения} множеств~$X_{\max}$ и~$Y_{\min}$, не тре\-бу\-юще\-го решения задачи дуализации. 
Эксперименты проводились на модельных задачах с~применением асимптотически оптимального
 алгоритма дуализации над произведением конечных цепей \mbox{RUNC-M}+~\cite{7}. 
 Результаты исследования свидетельствуют о~том, что по\-сле\-до\-ва\-тель\-но-со\-вмест\-ный 
 метод наиболее эффективен (требует меньших временн$\acute{\mbox{ы}}$х затрат по сравнению с~другими рас\-смот\-рен\-ны\-ми 
 методами) в~случае, когда мощ\-ность множества час\-тых наборов примерно равна мощ\-ности множества
  нечастых наборов. Иначе выигрывает последовательный поиск. Наихудшие показатели 
  у~независимого пе\-ре\-чис\-ле\-ния множеств~$X_{\max}$ и~$Y_{\min}$ с~использованием в~качестве
   базового алгоритма Apriori~\cite{2}, точ\-нее его модификации на тес\-ти\-ру\-емый случай. 
   Таким образом, показана це\-ле\-со\-об\-раз\-ность применения алгоритмов дуализации для 
   по\-стро\-ения множеств~$X_{\max}$ и~$Y_{\min}$.

  
  {\small\frenchspacing
 {%\baselineskip=10.8pt
 %\addcontentsline{toc}{section}{References}
 \begin{thebibliography}{9}  
    \bibitem{4}
    \Au{Aggarwal C.} 
    Frequent pattern mining.~--- Heidelberg: Springer, 2014. 467~p.
    
    \bibitem{1}
    \Au{Agrawal~R., Imielinski~T., Swami~A.} Mining association rules 
    between sets of items in large databases~// \mbox{SIGMOD} Conference (International) on Management of Data
    Proceedings.~--- New York, NY, USA: ACM, 1993. P.~207--216.
    
    \bibitem{9}
    \Au{Elbassioni K.} On finding minimal infrequent elements in multi-dimensional 
    data defined over partially ordered sets~// arXiv.org, 2014. 30~p. arXiv:1411.2275 [cs.DB].
    
    \bibitem{8}
    \Au{Elbassioni K.} Algorithms for dualization over products of partially 
    ordered sets~// SIAM J.~Discrete Math., 2009. Vol.~23. Iss.~1. P.~487--510.
    
    \bibitem{2}
    \Au{Agrawal R., Srikant~R.} 
    Fast algorithms for mining association rules in large databases~// 
    20th Conference (International) on Very Large Data Bases Proceedings.~--- San Francisco, CA, USA: 
    Morgan Kaufmann Publs. Inc., 1994. P.~487--499.
    
    \bibitem{14}
    \Au{Хачиян Л.\,Г.} Избранные труды.~--- М.: МЦНМО, 2009. 520~с.
    
    \bibitem{7}
    \Au{Дюкова Е.\,В., Масляков~Г.\,О., Прокофьев~П.\,А.} 
    О~дуализации над произведением частичных порядков~// Машинное обучение и~анализ данных, 2017. Т.~3. №\,4.  
    C.~239--249.
    
    \bibitem{6}
    \Au{Дюкова Е.\,В., Прокофьев~П.\,А.} Об асимптотически оптимальных алгоритмах дуализации~// 
    Ж.~вычисл. матем. и~матем. физ., 2015. Т.~55. №\,5. С.~895--910.
    \end{thebibliography}

 }
 }

\end{multicols}

\vspace*{-6pt}

\hfill{\small\textit{Поступила в~редакцию 15.01.21}}

\vspace*{8pt}

%\pagebreak

%\newpage

%\vspace*{-28pt}

\hrule

\vspace*{2pt}

\hrule

%\vspace*{-2pt}

\def\tit{FINDING MAXIMAL FREQUENT AND~MINIMAL INFREQUENT SETS IN~PARTIALLY ORDERED DATA}


\def\titkol{Finding maximal frequent and~minimal infrequent sets in~partially ordered data}


\def\aut{N.\,A.~Dragunov and E.\,V.~Djukova}

\def\autkol{N.\,A.~Dragunov and E.\,V.~Djukova}

\titel{\tit}{\aut}{\autkol}{\titkol}

\vspace*{-11pt}


\noindent
Federal Research Center ``Computer Science and Control'' 
of the Russian Academy of Sciences, 44-2~Vavilov Str., Moscow 119333, Russian Federation

\def\leftfootline{\small{\textbf{\thepage}
\hfill INFORMATIKA I EE PRIMENENIYA~--- INFORMATICS AND
APPLICATIONS\ \ \ 2022\ \ \ volume~16\ \ \ issue\ 1}
}%
 \def\rightfootline{\small{INFORMATIKA I EE PRIMENENIYA~---
INFORMATICS AND APPLICATIONS\ \ \ 2022\ \ \ volume~16\ \ \ issue\ 1
\hfill \textbf{\thepage}}}

\vspace*{3pt} 


\Abste{Relevant issues of time costs reducing in the logical analysis of data with elements 
from the Cartesian product of finite partially ordered sets are investigated. 
An original method based on solving a complex discrete problem called dualization
 over the product of partial orders is proposed for the problem of finding maximal 
 frequent and minimal infrequent sets in the transaction database. The proposed method 
 is a~synthesis of two other known methods, one of which is quite obvious and the other uses 
 the idea of an incremental enumeration of target\linebreak\vspace*{-12pt}}
 
 \Abstend{sets and is, therefore, mainly 
 of theoretical interest. An experimental study of the considered approaches in
  the case of the product of finite chains is carried out and conditions for
   their effectiveness are revealed. The expediency of applying 
asymptotically optimal dualization algorithms over the product of partial orders is shown.}

\KWE{maximal frequent sets; minimal infrequent sets; dualization over the product of 
partial orders; asymptotically optimal dualization algorithm}

\DOI{10.14357/19922264220112}

%\vspace*{-16pt}

%\Ack
%\noindent




%\vspace*{6pt}

  \begin{multicols}{2}

\renewcommand{\bibname}{\protect\rmfamily References}
%\renewcommand{\bibname}{\large\protect\rm References}

{\small\frenchspacing
 {%\baselineskip=10.8pt
 \addcontentsline{toc}{section}{References}
 \begin{thebibliography}{9}
\bibitem{1-dr}
\Aue{Aggarwal, C.} 2014. \textit{Frequent pattern mining}. Heidelberg: Springer. 467~p.
\bibitem{2-dr}
\Aue{Agrawal, R., T.~Imielinski, and A.~Swami.}
 1993. Mining association rules between sets of items in large databases. 
 \textit{SIGMOD  Conference (International) on Management of Data Proceedings}. New York, NY:
 ACM. 207--216. 
\bibitem{3-dr}
\Aue{Elbassioni, K.}
 2014. On finding minimal infrequent elements in multidimensional data defined over partially ordered sets. 
 arXiv.org. 30~p. Available at: 
 {\sf https://arxiv.org/\linebreak pdf/1411.2275.pdf} (accessed January~25, 2022).
\bibitem{4-dr}
\Aue{Elbassioni, K.} 2009. Algorithms for dualization over products of partially ordered sets. 
\textit{SIAM J.~Discrete Math.} 23(1):487--510.
\bibitem{5-dr}
\Aue{Agrawal, R., and R.~Srikant.}
 1994. Fast algorithms for mining association rules in large databases. 
 \textit{20th Conference (International) on Very Large Data Bases Proceedings}.
 San Francisco, CA: 
    Morgan Kaufmann Publs. Inc.  487--499.
\bibitem{6-dr}
\Aue{Khachiyan, L.\,G.} 2009. \textit{Izbrannye trudy} [Selected works]. Moscow: MCCME. 520~p.
\bibitem{7-dr}
\Aue{Djukova, E.\,V., G.\,O.~Maslyakov, and P.\,A.~Prokofyev.} 
2017. O~dualizatsii nad proizvedeniem chastichnykh poryadkov [On dualization over the product of 
partial orders]. \textit{Mashinnoe obuchenie i~analiz dannykh} [J.~Machine Learning Data Analysis] 
3(4):239--249.
\bibitem{8-dr}
\Aue{Djukova, E.\,V., and P.\,A.~Prokofyev.}
 2015. Asymptotically optimal dualization algorithms. \textit{Comp. Math.
 Math. Phys.} 55(5):891--905. 
 
 \end{thebibliography}

 }
 }

\end{multicols}

\vspace*{-6pt}

\hfill{\small\textit{Received January 15, 2021}}

%\pagebreak

%\vspace*{-18pt}

\Contr

\noindent
\textbf{Dragunov Nikita A.} (b.\ 1997)~--- 
PhD student, Federal Research Center ``Computer Science and Control'' 
of the Russian Academy of Sciences, 44-2~Vavilov Str., Moscow 119333, Russian Federation; 
\mbox{nikitadragunovjob@gmail.com}

\vspace*{3pt}

\noindent
\textbf{Djukova Elena V.} (b.\ 1945)~--- 
Doctor of Science in physics and mathematics, principal scientist, Federal Research Center
``Computer Science and Control'' of the Russian Academy of Sciences, 44-2~Vavilov Str., Moscow 119333, 
Russian Federation; \mbox{edjukova@mail.ru}




\label{end\stat}

\renewcommand{\bibname}{\protect\rm Литература}  %10
\def\stat{grusho}

\def\tit{АРХИТЕКТУРНЫЕ РЕШЕНИЯ В~ЗАДАЧЕ ВЫЯВЛЕНИЯ МОШЕННИЧЕСТВА ПРИ~АНАЛИЗЕ 
ИНФОРМАЦИОННЫХ ПОТОКОВ В~ЦИФРОВОЙ ЭКОНОМИКЕ$^*$}

\def\titkol{Архитектурные решения в~задаче выявления мошенничества при~анализе 
информационных потоков в
%~цифровой 
экономике}

\def\aut{А.\,А.~Грушо$^1$, М.\,И.~Забежайло$^2$, Н.\,А.~Грушо$^3$, 
Е.\,Е.~Тимонина$^4$}

\def\autkol{А.\,А.~Грушо, М.\,И.~Забежайло, Н.\,А.~Грушо, 
Е.\,Е.~Тимонина}

\titel{\tit}{\aut}{\autkol}{\titkol}

\index{Грушо А.\,А.}
\index{Забежайло М.\,И.}
\index{Грушо Н.\,А.}
\index{Тимонина Е.\,Е.}
\index{Grusho A.\,A.}
\index{Zabezhailo M.\,I.}
\index{Grusho N.\,A.}
\index{Timonina E.\,E.}


{\renewcommand{\thefootnote}{\fnsymbol{footnote}} \footnotetext[1]
{Работа частично поддержана РФФИ (проекты 18-29-03081 и~18-07-00274).}}


\renewcommand{\thefootnote}{\arabic{footnote}}
\footnotetext[1]{Институт проблем информатики Федерального исследовательского центра <<Информатика и~управление>> 
Российской академии наук, grusho@yandex.ru}
\footnotetext[2]{Институт проблем информатики Федерального исследовательского центра <<Информатика и~управление>> 
Российской академии наук, m.zabezhailo@yandex.ru}
\footnotetext[3]{Институт проблем информатики Федерального исследовательского центра <<Информатика и~управление>> 
Российской академии наук, info@itake.ru}
\footnotetext[4]{Институт проблем информатики Федерального исследовательского центра <<Информатика и~управление>> 
Российской академии наук, eltimon@yandex.ru}

\vspace*{-12pt}
   

 
  
  \Abst{Cформулирован подход к~исследованию некоторых видов мошенничества в~цифровой 
экономике с~использованием причинно-следственных связей. Во всех видах рассматриваемых 
мошенничеств должно наблюдаться несоответствие между целями финансовых транзакций 
и~реальной стоимостью достижения этих целей. Данные о транзакциях можно собирать, 
наблюдая информационные потоки, в~которых отражаются эти транзакции. Архитектура сбора 
данных и~их анализа может быть организована с~помощью распределенных реестров 
с~централизованным консенсусом, что позволяет создать аналог электронной бухгалтерской 
книги, фиксирующей финансово-экономическую деятельность субъектов цифровой экономики в~регионе. 
  Рассматриваемые методы выявления мошенничества основаны на противоречиях 
между действиями, описанными в~транзакциях, и~информацией, содержащейся в~планах, 
стандартах, прецедентах и~др. Рассмотрен метод, основанный на некоторой упрощенной схеме 
реализации абстрактного проекта. Для выявления противоречий необходимо проводить анализ 
от следствия к~причине, т.\,е.\ искать аномалии в~информации, описывающей порождение 
наблюдаемых следствий. 
  Показано, как в~реализации проекта можно выделять простые <<необходимые условия>> 
нарушения при\-чин\-но-след\-ст\-вен\-ных связей, т.\,е.\ множество <<необходимых условий>>, 
нарушение которых свидетельствует о наличии мошенничества. Это множество <<необходимых 
условий>> можно назвать метаданными для контроля проекта на выявление мошенничества.} 
 
 
  \KW{цифровая экономика; информационные потоки; при\-чин\-но-след\-ст\-вен\-ные связи; 
выявление мошеннических схем} 

\DOI{10.14357/19922264190204}
  
\vspace*{-4pt}


\vskip 10pt plus 9pt minus 6pt

\thispagestyle{headings}

\begin{multicols}{2}

\label{st\stat}

\section{Введение}

\vspace*{3pt}

  В работе сформулирован подход к~исследованию некоторых видов 
мошенничества в~цифровой экономике с~использованием  
при\-чин\-но-след\-ст\-вен\-ных связей. Рассматриваются три вида мошенничества, 
а именно:
  \begin{enumerate}[(1)]
\item отмыв денег; 
\item обман при выполнении договорных обязательств при реализации 
технических проектов (строительные проекты и~др.); 
\item незаконный вывод денег. 
\end{enumerate}

  Названные виды мошенничества могут быть сведены к~решению одного типа 
задач. Для отмывания денег источник должен заключать фиктивные контракты, 
в~соответствии с~которыми будут переводиться средства за заведомо ненужную 
работу и~материалы. 
  
  Мошенничество, связанное с~невыполнением договорных обязательств, связано 
со снижением качества услуг, качества и~количества закупаемых 
материалов, выполнением работ с~ненадлежащим качеством. 
  
  Вывод денег связан с~переводом средств фир\-мам-од\-но\-днев\-кам, которые 
заведомо не могут выполнить обязательства по контрактам, за которые им 
переводятся средства. 
  
  Таким образом, во всех трех видах рассматриваемых мошенничеств должно 
наблюдаться несоответствие между целями финансовых транзакций и~реальной 
стоимостью достижения этих целей. Данные о транзакциях можно собирать, 
наблюдая информационные потоки, в~которых отражаются эти транзакции. 
  
  Однако для наблюдения таких информационных потоков необходимо создавать 
архитектуру\linebreak телекоммуникационной системы, позволяющей перехватывать 
и~собирать данные о всех транзакциях. Например, такая архитектура может быть 
организована с~помощью распределенных реестров с~централизованным 
консенсусом, т.\,е.\ все информационные потоки, сформированные в~цифровой 
экономике и~несущие информацию о транзакциях, проходят через некоторый 
центральный узел, запоминающий их в~форме распределенного реестра. Такие 
реестры могут дублироваться в~аналогичных центрах различных регионов, что 
позволяет создать аналог электронной бухгалтерской книги, фиксирующей 
фи\-нан\-со\-во-эко\-но\-ми\-че\-скую деятельность субъектов цифровой экономики. Такой 
подход предложено реализовать на базе системы ситуационных центров, что 
отражено в~работах~[1, 2].
  
  Собранная из информационных потоков информация о~транзакциях, т.\,е.\ 
о~контрактах, договорах, платежах, отчетах, закупленных материалах, 
характеристиках исполнителей работ и~др., собирается в~базе данных в~указанном 
центре. Согласно теории интеллектуальных сис\-тем~[3], эту базу данных можно 
называть базой фактов (БФ). Базу фактов можно представить как бинарную мат\-ри\-цу, 
строки которой описывают характеристики, входящие в~транзакции, а столбцы 
нумеруются характеристиками. Строки матрицы будем называть 
\textit{объектами}~[4, 5]. 
  
  Рассматриваемые в~работе методы выявления мошенничества будут основаны 
на противоречиях между действиями, описанными в~транзакциях, и~информацией, 
содержащейся в~планах, стандартах, прецедентах и~др. Для нахождения 
противоречий в~архитектуре центра предусмотрена другая база данных~--- база 
знаний (БЗ)~\cite{3-gr, 6-gr}, которая устроена так же, как БФ. 
  
  Информация в~БЗ собирается на основе положительного опыта или расчетов. 
Используя БЗ, можно выводить факты нарушения при\-чин\-но-след\-ст\-вен\-ных 
связей. Нарушения при\-чин\-но-след\-ст\-вен\-ных связей будем называть 
\textit{аномалиями}. 
  
  Для упрощения дальнейшее изложение будет вестись в~рамках поиска 
противоречий при выполнении некоторого абстрактного проекта. Выявление 
аномалий будет происходить на основе фактов из БФ с~помощью знаний из БЗ 
методами искусственного интеллекта и~интеллектуального анализа 
данных~\cite{6-gr}. 

\vspace*{-10pt}
  
  \section{Модели}
  
  \vspace*{-3pt}
  
  Наиболее сложная из рассмотренных выше задач~--- выявление противоречий, 
т.\,е.\ использование БЗ для получения новых знаний и~выявление аномалий из 
полученных фактов. 
  
  Все способы выявления противоречий основаны на определении 
  причинно-следственных связей. При этом противоречия в~параметрах транзакций по 
отношению к~требуемым в~БЗ составляют сущность аномалий. 
  
   Далее будет рассмотрен метод, основанный на некоторой упрощенной схеме 
реализации абстрактного проекта. 
  
  Каждый проект имеет цель: например, цель представляет собой построение 
некоторой системы. Воспользуемся структурным подходом, который позволяет 
строить проект на основе разбиения системы на подсистемы и~определения 
взаимодействий подсистем~\cite{7-gr}. При этом каждая подсистема также 
представима структурной моделью. 
  
  Как сама система, так и~каждая ее подсистема имеют свой функционал 
и~спецификацию, па\-ра\-мет\-ры настройки и~домены параметров настройки. Кроме 
этих характеристик существует множество характеристик, связанных 
с~<<жизненным циклом>> создания системы. Сюда входят работы, ресурсы, 
сроки выполнения работ по созданию подсистем и~самой системы, стоимости 
компонентов и~материалов, стоимости работ, схемы поставок, договорные 
обязательства и~др. Все характеристики связаны между собой, поэтому можно 
говорить о стоимости и~времени изготовления структурных компонентов системы. 
  
  Одной из важнейших характеристик является смета (система смет для 
подсистем). Смета сопоставляет каждому компоненту системы стоимость его 
изготовления и~настройки. 
  
  Схема построения системы может быть пред\-став\-ле\-на диаграммой, 
изображенной на рис.~1. 

{ \begin{center}  %fig1
 \vspace*{9pt}
   \mbox{%
 \epsfxsize=79mm 
 \epsfbox{gru-1.eps}
 }


\vspace*{9pt}


\noindent
{{\figurename~1}\ \ \small{Диаграмма достижения цели}}
\end{center}
}

\vspace*{9pt}

\addtocounter{figure}{1}
  
  


  Представленная на рис.~1 диаграмма позволяет описать основные классы 
возможных противоречий при достижении цели. Противоречия возникают, когда 
данные БФ не соответствуют требуемым характеристикам. 
  
  
  \section{Потенциальные классы аномалий при~достижении цели}
  
  Выделим четыре потенциальных класса противоречий, которые показывают, 
каким образом нужно искать эти противоречия.
  
 
  Противоречие цели и~проекта (рис.~2) возникает при отсутствии обоснования 
или в~случае логического противоречия между возможностями проектируемого 
функционала и~целью системы. Отметим, что в~проект входят сроки, перечень 
работ, материалы, настройки, которые описываются соответствующими 
параметрами и~допустимыми значениями этих параметров. Проект формируется 
на основе БЗ и~расчетов, исходя из информации, полученной по аналогии 
с~другими проектами и~решениями, которые считаются апробированными. 
  
  Отметим, что цель порождает проект и~в этом смысле является причиной 
проекта. Однако для анализа противоречий необходимо двигаться по штриховой 
стрелке диаграммы (см.\ рис.~2) от проекта к~цели. В~самом деле, любой компонент 
проекта направлен на теоретическое достижение цели. Цель~--- сложный объект, 
поэтому в~проекте могут возникнуть характеристики, противоречащие хотя бы 
некоторым характеристикам цели. Это делает проект противоречивым, но вывод 
об этом может быть сделан только на уровне описания цели. 
  

  Противоречия между проектом и~его реализацией, исключая настройки 
(рис.~3), могут возникать, например, при закупке исполнителем материалов более 
низкого качества по более низким ценам, при попытках достижения требуемых 
сроков работы за счет снижения качества выполнения работ, за счет нахождения 
<<объективных>> причин для увеличения сроков работы и,~следовательно, 
увеличения цены реализации проекта. 


  Для выявления указанных противоречий необходимо двигаться по диаграмме 
(см.\ рис.~3) в~обратную сторону в~соответствии со~штриховыми стрелками. 
Действительно, выявить противоречия между характеристиками закупленных 
материалов и~требуемыми по проекту можно только при обращении к~проекту 
и~его спецификациям. Манипуляции со сроками работы также можно выявить 
только при обращении к~соответствующим расчетам в~проекте. Задержки в~сроках 
работы, связанные с~поставками материалов, можно определить только на 
предыдущем этапе диаграммы (см.\ рис.~3) в~описании проекта. 


  


  Противоречия между реализацией проекта и~его настройкой (рис.~4) возникает, 
когда не удается добиться требуемых значений параметров функционала, не 
удается обеспечить необходимый уровень\linebreak\vspace*{-12pt}

{ \begin{center}  %fig2
 \vspace*{-6pt}
   \mbox{%
 \epsfxsize=16mm 
 \epsfbox{gru-2.eps}
 }


\vspace*{6pt}


\noindent
{{\figurename~2}\ \ \small{Противоречия цели и~проекта}}
\end{center}
}

%\vspace*{9pt}

\addtocounter{figure}{1}

{ \begin{center}  %fig3
 \vspace*{6pt}
    \mbox{%
 \epsfxsize=79mm 
 \epsfbox{gru-3.eps}
 }


\end{center}

\vspace*{-2pt}


\noindent
{{\figurename~3}\ \ \small{Противоречия проекта и~его реализации (без настройки)}}
}

\vspace*{6pt}

\addtocounter{figure}{1}

{ \begin{center}  %fig4
 \vspace*{1pt}
   \mbox{%
 \epsfxsize=54.5mm 
 \epsfbox{gru-4.eps}
 }


\end{center}


\noindent
{{\figurename~4}\ \ \small{Противоречия реализации проекта и~его на\-стройки}}
}

%\vspace*{9pt}

\addtocounter{figure}{1}

{ \begin{center}  %fig5
 \vspace*{5pt}
    \mbox{%
 \epsfxsize=79mm 
 \epsfbox{gru-5.eps}
 }


\end{center}



\noindent
{{\figurename~5}\ \ \small{Противоречия цели и~достигнутой реализации проекта}}
}

\vspace*{6pt}

\addtocounter{figure}{1}

\noindent
 качества реализации проекта. Для 
определения противоречия в~настройках надо опять же двигаться по диаграмме 
(см.\ рис.~4) в~обратную сторону по штриховым стрелкам, так как для выявления 
характеристик результатов работы, которые не дают возможности реализации 
определенного функционала, необходимо иметь информацию о результатах этой 
работы. 


  



  Противоречие между целью и~достигнутой реализацией проекта (рис.~5) 
возникает, когда реализованная система не позволяет достичь цели. В~этом случае 
опять противоречие нужно искать, двигаясь от цели к~реальному достигнутому 
функционалу по штриховой стрелке (см.\ рис.~5).
  
  Суммируя положения, изложенные в~данном разделе, приходим к~выводу, что 
для выявления противоречий необходимо проводить анализ от следствия 
к~причине, т.\,е.\ искать аномалии в~информации, описывающей порождение 
наблюдаемых следствий. 
  
  
  \section{Связь противоречий и~причин}
  
  Прежде чем построить связь между причинами и~противоречиями, кратко 
опишем простейшую модель связи этих понятий. Причины и~противоречия будут 
сформулированы для представления компонентов системы как объектов, 
обладающих наборами известных характеристик~\cite{4-gr, 5-gr}. 
  
  Пусть $U\hm=\{\alpha, \beta, \ldots\}$~--- совокупность характеристик 
(пространство характеристик). Согласно~\cite{4-gr} \textit{объектом}~$O$ 
называется любое подмножество характеристик $O\hm\subseteq U$. Рассмотрим 
последовательность объектов, возможно в~различных пространствах 
характеристик. 
  
  \smallskip
  
  \noindent
  \textbf{Определение~1.}\ Объект~$P$ с~числом характеристик, большим или 
равным~2, является \textit{причиной} объекта (\textit{свойства})~$B$ в~цепочке 
наблюдаемых объектов тогда и~только тогда, когда выполнены следующие 
условия:
  \begin{enumerate}[(1)]
\item для каждого объекта~$C$, если $P\hm\subseteq C$, то $C\mapsto B$, где 
$C\mapsto B$ означает, что объект~$B$ присутствует в~объекте, следующем за 
объектом~$C$;
\item объект~$P$ является минимальным объектом, удовлетворяющим 
условию~1, а~именно: $\forall \alpha\hm\in P$ объект~$P\backslash \{\alpha\}$ 
не является причиной, т.\,е.\ $\exists C:\ \alpha\not\in C$, $P\backslash 
\{\alpha\}\hm\subseteq C$ и~$C\not\mapsto B$, где $C\not\mapsto B$ означает, 
что~$B$ не может содержаться в~объекте, следующем за объектом~$C$. 
\end{enumerate}

  Приведенное определение причины является упрощением причин, 
возникающих в~реальном мире. Например, реальные причины могут возникать\linebreak 
как совокупность характеристик из разных пространств. Одно следствие может 
порождаться разными причинами или возникать из внешних\linebreak и~ненаблюдаемых 
характеристик. Однако пред\-став\-лен\-ная далее формализация позволяет доступно 
изложить при\-чин\-но-след\-ст\-вен\-ные истоки противоречий, которые 
инициируют в~дальнейшем глубокое исследование рассматриваемых процессов.
  
  Будем считать, что для любого интересующего нас свойства~$B$ существует 
причина. Тогда справедлива следующая теорема.
  
  \smallskip
  
  \noindent
  \textbf{Теорема~1.}\ \textit{Для любого свойства~$B$ существует 
единственная причина}. 
  
  \smallskip
  
  \noindent
  Д\,о\,к\,а\,з\,а\,т\,е\,л\,ь\,с\,т\,в\,о\,.\ \ Доказательство будем вести от противного, 
т.\,е.\ предположим, что существуют две причины свойства~$B$: $P$ 
и~$P^\prime$, $P\hm\not= P^\prime$. Тогда существует $\alpha\hm\in U$, которое 
удовлетворяет одному из двух условий:
  \begin{itemize}
\item[(а)] $\alpha\in P$, $\alpha\notin P^\prime$;
\item[(б)] $\alpha\notin P$, $\alpha \in P^\prime$.
\end{itemize}

  Пусть выполняется условие~(б). Тогда $P^\prime\backslash \{\alpha\}$ не 
является причиной по условию~2 определения~1, т.\,е.\ $\exists C$ такое, что 
$\alpha\notin C$, $P^\prime\backslash \{\alpha\}\hm\subseteq C$ и~$C\not\mapsto B$. 
Но если~$B$ произошло и~$P$ его причина, то $C\mapsto B$, что противоречит 
предположению. Теорема~1 доказана.
  
  \smallskip
  
  \noindent
  \textbf{Лемма.} \textit{Если $P$~--- причина появления свойства~$B$, то 
объект~$B$ определяет существование свойства~$P$ в~объекте, 
предшествующем~$B$. }
  
  \smallskip
  
  \noindent
  Д\,о\,к\,а\,з\,а\,т\,е\,л\,ь\,с\,т\,в\,о\,.\ \ Из предположения, что у~каж\-до\-го 
свойства~$B$ есть причина, и~условия, что~$P$ является причиной~$B$, следует, 
что при появлении в~данных свойства~$B$ объект~$C$, предшествующий 
появлению~$B$, содержит как часть объект~$P$. Это следует из теоремы~1 
и~определения причины. 
  
  Докажем принцип <<необходимого условия>>, который, несмотря на простоту 
доказательства, будет играть в~дальнейшем существенную роль.
  
  \smallskip
  
  \noindent
  \textbf{Теорема~2.} \textit{Если~$P$~--- причина появления свойства~$B$ 
и~$A\hm\subseteq P$, то объект~$B$ определяет наличие свойства~$A$ 
в~объекте, предшествующем~$B$}. 
  
  \smallskip
  
  \noindent
  Д\,о\,к\,а\,з\,а\,т\,е\,л\,ь\,с\,т\,в\,о\,.\ \ Пусть в~данных имеется объект~$B$ 
и~$P\mapsto B$, тогда в~силу существования и~единственности причины~$B$ 
в~данных должен существовать объект~$C$, предшествующий~$B$ 
и~содержащий причину~$P$. Поскольку $A\hm\subseteq P$ и~$B$ содержит 
причину~$P$, то $B\mapsto A$. С~учетом леммы теорема~2 доказана.
  
  \smallskip
  
  Пусть даны пространства $U_1, U_2,\ldots$ и~имеется последовательность 
данных (процесс выполнения этапов проекта в~соответствии с~рис.~1) $A, B, 
\ldots$, где каждый объект является подмножеством некоторого 
пространства~$U_i$, $i\hm=1,\ldots$ Тогда в~объекте~$A$ присутствует 
причина~$P$ появления интересующего нас свойства~$C$ в~объекте~$B$. Пусть 
$P\hm\subseteq A$, тогда по теореме~2 $\forall \alpha\hm\in P$:  
$C\mapsto \{\alpha\}$, т.\,е.\ из появления~$C$ следует появление 
характеристики~$\alpha$ в~предшествующем объекте. Это необходимое условие 
того, что~$C$ удовлетворяет причинно-следственным связям развития процесса 
выполнения проекта. Если для~$C$ нет характеристики~$\alpha$, которую можно 
отнести к~причине~$C$, то можно считать, что нарушена  
при\-чин\-но-след\-ст\-вен\-ная связь и~$C$~--- аномальный объект. 
  
  \smallskip
  
  \noindent
  \textbf{Пример.} Если объект~$C$ состоит в~получении суммы~$a$ 
фирмой~$K$, то согласно теореме~2 в~пред\-шест\-ву\-ющем объекте должна 
существовать причина перевода суммы~$a$ на фирму~$K$. Если эта причина 
в~проекте отсутствует, то это можно считать признаком мошеннической схемы. 
Все проекты по предположению собираются из <<кубиков>>, содержащихся в~БЗ. 
Тогда можно сравнить цену объекта~$C$, породившего получение суммы~$a$, 
и~сумму, присутствующую в~смете проекта. Если разница велика, то это либо 
ошибка проекта, либо признак мошеннической схемы.
  
  \section{Поиск противоречий на~основе~принципа <<необходимых~условий>>}
   
  Как было показано в~разд.~3, нахождение противоречий соответствуют 
движению от следствия к~причине. Для каждого объекта в~наблюдаемых данных 
выявление причин его появления является трудоемкой задачей. Кроме того, при 
реализации контроля соблюдения при\-чин\-но-след\-ст\-вен\-ных связей на 
большом множестве участников экономической деятельности задача анализа 
причин становится трудоемкой. Поэтому процедуру контроля необходимо разбить 
на два этапа, где первый этап состоит в~анализе простых <<необходимых 
условий>> проявления мошенничества, когда используется хотя бы одна 
известная характеристика причины. Второй этап (в~режиме офлайн) состоит 
в~выявлении причин, позволяющих провести анализ источников мошеннических 
схем. 
  
  Один из подходов к~выбору <<необходимых условий>> состоит в~построении 
множества подцелей исходной цели проекта (структурный метод построения 
проекта~\cite{7-gr}). Каждая подцель описывается диаграммой на рис.~1, 
и~реализации подцелей должны образовывать полный функционал цели. Это 
является необходимым, но не достаточным условием достижения цели, так как 
при таком подходе отсутствует компонент согласования всех подцелей в~единую 
систему. Однако такой подход значительно упрощает анализ выполнения проекта 
на предмет поиска мошенничества. Если признаки мошенничества будут 
обнаружены в~реализации хотя бы одной из подцелей, то это значит, что 
мошенничество присутствует в~реализации всего проекта. 
  
  Аналогично в~реализации каждого этапа в~любой из подцелей можно выделять 
простые <<необходимые условия>> нарушения при\-чин\-но-след\-ст\-венн\-ых 
связей. 
  
  Таким образом, получается множество <<необходимых условий>>, нарушение 
которых свидетельствует о наличии мошенничества. Это множество 
<<необходимых условий>> можно назвать метаданными~[8, 9] для контроля 
проекта на выявление мошенничества. 
  
  
  \section{Заключение }
  
  В поиске противоречий необходимо от транзакций, соответствующих 
следствиям при\-чин\-но-след\-ст\-вен\-ных связей, переходить к~анализу причин 
наблюдаемых следствий. Это сложная задача, которая связана с~описанием причин 
определенных свойств. 
  
  В работе представлена модель, позволяющая строить множество необходимых 
условий соответствия наблюдаемого следствия вызвавшей его причине. Этот 
подход делает поиск противоречий вполне вычислимой задачей, но не гарантирует 
успех. 
  
  {\small\frenchspacing
 {%\baselineskip=10.8pt
 \addcontentsline{toc}{section}{References}
 \begin{thebibliography}{9}
\bibitem{1-gr}
\Au{Грушо А.\,А., Зацаринный~А.\,А., Тимонина~Е.\,Е.} Блокчейны цифровой экономики на базе 
системы ситуационных центров и~централизованного консенсуса~// Радиолокация, навигация, 
связь: Мат-лы XXV Междунар. научн.-технич. конф.~---
Воронеж: Издательский дом ВГУ, 2019. Т.~6. С.~183--191. 
\bibitem{2-gr}
\Au{Grusho A., Zatsarinny~A., Timonina~E.} A~system approach to information security in 
distributed ledgers on the situational centers platform.~---
Lecture notes in computer science ser.~--- Springer, 2019 
(in press).
\bibitem{3-gr}
\Au{Финн В.\,К.} Искусственный интеллект: Методология, применения, философия.~--- М.: 
Красанд, 2011. 448~с.

\bibitem{5-gr} %4
\Au{Аншаков~О.\,М., Фабрикантова~Е.\,Ф.} ДСМ-ме\-тод автоматического порождения 
гипотез: Логические и~эпистемологические основания.~--- М.: Либроком, 2009. 432~с.

\bibitem{4-gr} %5
\Au{Poelmans J., Elzinga~P., Viaene~S., Dedene~G.} Formal concept analysis in knowledge 
discovery: A~survey~// Conceptual structures: From information to intelligence~/ Eds.\ M.~Croitoru, 
S.~Ferr$\acute{\mbox{e}}$, and D.~Lukose.~--- Lecture notes in computer science 
ser.~--- Berlin--Heidelberg: Springer, 2010. Vol.~6208.  P.~139--153.

\bibitem{6-gr}
\Au{Панкратова~Е.\,С., Финн~В.\,К.} Автоматическое по\-рож\-де\-ние гипотез в~интеллектуальных 
системах.~--- М.: Либроком, 2009. 528~с. 
\bibitem{7-gr}
\Au{Денисов А.\,А., Колесников~Д.\,Н.} Теория больших систем управления.~--- Л.: Энергоиздат, 1982. 488~с.

\bibitem{9-gr}
\Au{Грушо А.\,А., Грушо Н.\,А., Забежайло~М.\,И., Смирнов~Д.\,В., Тимонина~Е.\,Е.} 
Параметризация в~прикладных задачах поиска эмпирических причин~// Информатика и~её 
применения, 2018. Т.~12. Вып.~3. С.~62--66.

\bibitem{8-gr}
\Au{Грушо А.\,А., Грушо Н.\,А., Левыкин~М.\,В., Тимонина~Е.\,Е.} Методы идентификации 
захвата хоста в~распределенной ин\-фор\-ма\-ци\-он\-но-вы\-чис\-ли\-тель\-ной сис\-те\-ме, 
защищенной с~помощью метаданных~// Информатика и~её применения, 2018. Т.~12. Вып.~4. 
С.~41--45.

 \end{thebibliography}

 }
 }

\end{multicols}

\vspace*{-3pt}

\hfill{\small\textit{Поступила в~редакцию 03.04.19}}

%\vspace*{8pt}

%\pagebreak

\newpage

\vspace*{-28pt}

%\hrule

%\vspace*{2pt}

%\hrule

%\vspace*{-2pt}

\def\tit{ARCHITECTURAL DECISIONS IN~THE~PROBLEM 
OF~IDENTIFICATION OF~FRAUD IN~THE~ANALYSIS 
OF~INFORMATION FLOWS IN~DIGITAL ECONOMY\\[-5pt]}


\def\titkol{Architectural decisions in~the~problem 
of~identification of~fraud in~the~analysis 
of~information flows in~digital economy}

\def\aut{A.\,A.~Grusho, M.\,I.~Zabezhailo, N.\,A.~Grusho, and~E.\,E.~Timonina}

\def\autkol{A.\,A.~Grusho, M.\,I.~Zabezhailo, N.\,A.~Grusho, and~E.\,E.~Timonina}

\titel{\tit}{\aut}{\autkol}{\titkol}

\vspace*{-13pt}


 \noindent
   Institute of Informatics Problems, Federal Research Center ``Computer Sciences and 
Control'' of the Russian Academy of Sciences; 44-2~Vavilov Str., Moscow 119133, 
Russian Federation

\def\leftfootline{\small{\textbf{\thepage}
\hfill INFORMATIKA I EE PRIMENENIYA~--- INFORMATICS AND
APPLICATIONS\ \ \ 2019\ \ \ volume~13\ \ \ issue\ 2}
}%
 \def\rightfootline{\small{INFORMATIKA I EE PRIMENENIYA~---
INFORMATICS AND APPLICATIONS\ \ \ 2019\ \ \ volume~13\ \ \ issue\ 2
\hfill \textbf{\thepage}}}

\vspace*{3pt}


   
     
   \Abste{An approach to a~research of some types of fraud in digital economy with the usage of relationships of 
cause and effect is formulated. In all types of the considered frauds, the discrepancy between the 
purposes of financial transactions and actual cost of achievement of these purposes
has to be observed. Data on 
transactions can be collected by observing information flows in which these transactions are reflected. 
The architecture of data collection and their analysis can be organized by means of the distributed 
ledgers with the centralized consensus that allows creating an analog of the electronic account book 
fixing financial and economic activity of subjects of digital economy in the region. 
   The methods of fraud identification considered are based on the contradictions 
between actions described in transactions and information, which is contained in plans, standards, 
precedents, etc. 
   The method based on a~simplified scheme of implementation of the abstract project is considered. 
For identification of contradictions, it is necessary to carry out the analysis from the effect to the cause, 
i.\,e., to look for anomalies in information describing the generation of the observed effects. 
   It is shown how in implementation of the project it is possible to allocate simple ``necessary 
conditions'' of violation of cause and effect relationships, i.\,e., a~set of ``necessary conditions'' 
violation of which demonstrates fraud existence. It is possible to call this set of "necessary conditions" 
by metadata for control of the project for fraud identification.} 
   
   \KWE{digital economy; information flows; relationships of reason and effect; detection of 
fraudulent schemes}
   
  

 \DOI{10.14357/19922264190204}

\vspace*{-20pt}

 \Ack
   \noindent
   The work was partially supported by the Russian Foundation for Basic Research (projects  
18-29-03081 and 18-07-00274).



%\vspace*{6pt}

  \begin{multicols}{2}

\renewcommand{\bibname}{\protect\rmfamily References}
%\renewcommand{\bibname}{\large\protect\rm References}

{\small\frenchspacing
 {\baselineskip=10.5pt
 \addcontentsline{toc}{section}{References}
 \begin{thebibliography}{9}
\bibitem{1-gr-1}
\Aue{Grusho, A.\,A., A.\,A.~Zatsarinny, and E.\,E.~Timonina.} 2019. Blokcheyny tsifrovoy ekonomiki 
na baze sistemy situatsionnykh tsentrov i~tsentralizovannogo konsensusa [Blockchains of digital 
economy on the basis of the system of the situational centres and the centralized consensus]. 
\textit{25th Scientific and Technical Conference (International) ``Radar-Location, Navigation, 
Communication'' Proceedings}. Voronezh: VSU Publs. 6:183--191.
\bibitem{2-gr-1}
\Aue{Grusho, A., A.~Zatsarinny, and E.~Timonina.} 2019 (in press). 
A~system approach to information security 
in distributed ledgers on the situational centers platform. 
Lecture notes in computer science ser. Springer.
\bibitem{3-gr-1}
\Aue{Finn, V.\,K.} 2011. \textit{Iskusstvennyy intellekt: Metodologiya, primeneniya, filosofiya} 
[Artificial intelligence: Methodology, applications, philosophy]. Moscow: KRASAND. 448~p.

\bibitem{5-gr-1}
\Aue{Anshakov, O.\,M., and E.\,F.~Fabrikantova}. 2009. \textit{DSM-metod avtomaticheskogo porozhdeniya gipotez: Logicheskie 
i~epistemologicheskie osnovaniya} [JSM-method of automatic hypothesis generation: Logical and 
epistemological]. Moscow: KD LIBROKOM. 432~p.
\bibitem{4-gr-1} %5
\Aue{Poelmans, J., P.~Elzinga, S.~Viaene, and G.~Dedene.} 2010. Formal concept analysis in 
knowledge discovery: A~survey. \textit{Conceptual structures: From information to intelligence}. 
Eds.\ M.~Croitoru, S.~Ferr$\acute{\mbox{e}}$, and D.~Lukose. Lecture notes in 
computer science ser. Berlin--Heidelberg: Springer. 6208:139--153.

\bibitem{6-gr-1}
\Aue{Pankratov, E.\,S., and V.\,K.~Finn}. 
2009. \textit{Avtomaticheskoe porozhdenie gipotez v~intellektual'nykh 
sistemakh} [Automatic hypotheses generation in intelligent systems]. Moscow: KD 
\mbox{LIBROKOM}.  528~p. 
\bibitem{7-gr-1}
\Aue{Denisov, A.\,A., and D.\,N.~Kolesnikov.} 1982. \textit{Teoriya bol'shikh 
sistem upravleniya} [Theory of big control systems]. Leningrad: Energoizdat. 488~p.

\bibitem{9-gr-1}
\Aue{Grusho, A.\,A., N.\,A.~Grusho, M.\,I.~Zabezhailo, D.\,V.~Smirnov, and 
E.\,E.~Timonina.} 2018. 
Parametrizatsiya v~prikladnykh zadachakh poiska empiricheskikh prichin 
[Parametrization in applied 
problems of search of the empirical reasons]. 
\textit{Informatika i~ee Primeneniya~--- 
Inform. Appl.} 12(3):62--66.

\bibitem{8-gr-1}
\Aue{Grusho, A.\,A., N.\,A.~Grusho, M.\,V.~Levykin, and E.\,E.~Timonina.} 2018. Metody 
identifikatsii zakhvata khosta v~raspredelennoy informatsionno-vychislitel'noy sisteme, 
zashchishchennoy s~pomoshch'yu metadannykh [Methods of identification of host capture 
in the  distributed information system which is protected on the base of meta data].
\textit{Informatika i~ee 
Primeneniya~--- Inform. Appl.} 12(4):41--45.
{ %\looseness=1

}

\end{thebibliography}

 }
 }

\end{multicols}

\vspace*{-12pt}

\hfill{\small\textit{Received April 3, 2019}}

%\pagebreak

%\vspace*{-18pt}

\Contr

\noindent
\textbf{Grusho Alexander A.} (b.\ 1946)~--- Doctor of Science in physics and 
mathematics, professor, principal scientist, Institute of Informatics Problems, 
Federal Research Center ``Computer Sciences and Control'' of the Russian 
Academy of Sciences; 44-2~Vavilov Str., Moscow 119133, Russian Federation; 
\mbox{grusho@yandex.ru} 

\vspace*{3pt}

\noindent
\textbf{Zabezhailo Michael I.} (b.\ 1956)~--- Doctor of Science in physics and 
mathematics, principal scientist, Institute of Informatics Problems, Federal Research 
Center ``Computer Sciences and Control'' of the Russian Academy of Sciences;  
44-2~Vavilov Str., Moscow 119133, Russian Federation; 
\mbox{m.zabezhailo@yandex.ru} 

\vspace*{3pt}


\noindent
\textbf{Grusho Nikolai A.} (b.\ 1982)~--- Candidate of Science (PhD) in physics 
and mathematics, senior scientist, Institute of Informatics Problems, Federal 
Research Center ``Computer Sciences and Control'' of the Russian Academy of 
Sciences; 44-2~Vavilov Str., Moscow 119133, Russian Federation; 
\mbox{info@itake.ru} 

\vspace*{3pt}


\noindent
\textbf{Timonina Elena E.} (b.\ 1952)~--- Doctor of Science in technology, 
professor, leading scientist, Institute of Informatics Problems, Federal Research 
Center ``Computer Sciences and Control'' of the Russian Academy of Sciences;  
44-2~Vavilov Str., Moscow 119133, Russian Federation; 
\mbox{eltimon@yandex.ru} 

\label{end\stat}

\renewcommand{\bibname}{\protect\rm Литература}   %11
\def\stat{kochetkova}

\def\tit{ВЕРОЯТНОСТНАЯ МОДЕЛЬ ЗАТУХАНИЯ МОЩНОСТИ СИГНАЛА В СЦЕНАРИЯХ 3GPP TR 38.901 
РАЗВЕРТЫВАНИЯ СЕТИ 5G$^*$}

\def\titkol{Вероятностная модель затухания мощности сигнала в~сценариях 3GPP TR 38.901 
развертывания сети 5G}

\def\aut{Е.\,Д.~Макеева$^1$, И.\,А.~Кочеткова$^2$, С.\,Я.~Шоргин$^3$}

\def\autkol{Е.\,Д.~Макеева, И.\,А.~Кочеткова, С.\,Я.~Шоргин}

\titel{\tit}{\aut}{\autkol}{\titkol}

\index{Макеева Е.\,Д.}
\index{Кочеткова И.\,А.}
\index{Шоргин С.\,Я.}
\index{Makeeva E.\,D.}
\index{Kochetkova I.\,A.}
\index{Shorgin S.\,Ya.}


{\renewcommand{\thefootnote}{\fnsymbol{footnote}} \footnotetext[1]
{Публикация выполнена в~рамках проекта №\,025319-2-000 Системы грантовой 
поддержки научных проектов РУДН.}}


\renewcommand{\thefootnote}{\arabic{footnote}}
\footnotetext[1]{Российский университет дружбы народов имени Патриса Лумумбы; 
Институт проблем управления имени В.\,А.~Трапезникова Российской академии наук, 
\mbox{elena-makeeva-96@mail.ru}}
\footnotetext[2]{Российский университет дружбы народов имени Патриса Лумумбы; 
Федеральный исследовательский центр <<Информатика и~управ\-ле\-ние>> Российской 
академии наук, \mbox{kochetkova-ia@rudn.ru}}
\footnotetext[3]{Федеральный исследовательский центр <<Информатика 
и~управ\-ле\-ние>> Российской академии наук, \mbox{sshorgin@ipiran.ru}}

\vspace*{2pt}






\Abst{Сети пятого (5G) и~последующих поколений будут использовать терагерцевый диапазон 
радиочастот, что обеспечит сверхвысокую скорость передачи данных. Однако при 
этом возможны потери сигнала при прохождении через препятствия. Поэтому 
становится крайне важным моделирование распространения сигнала с~по\-мощью 
стохастической геометрии и~использование актуальных моделей затухания сигнала. 
Модели для описания затухания сигнала для различных сценариев развертывания сети 
5G в~виде эмпирических формул содержатся в~спецификации 3GPP TR 38.901. Тем не 
менее обычно для построения моделей стохастической геометрии используются 
упрощенные виды формул. В~\mbox{статье} представлена функция распределения (ФР) затухания 
мощности сигнала при случайном расположении пользователей в~соответствии со 
сценариями, описанными в~3GPP TR 38.901. На численных примерах показано, что 
разница значений с~упрощенной формулой значительна и~может привести к~занижению 
оценки пропускной способности сети.}

\KW{беспроводная сеть; 5G; 3GPP TR 38.901; мощ\-ность затухания сигнала; прямая 
видимость; непрямая ви\-ди\-мость; сто\-ха\-сти\-че\-ская гео\-метрия}

\DOI{10.14357/19922264240204}{EKLCAP}
  
%\vspace*{-6pt}


\vskip 10pt plus 9pt minus 6pt

\thispagestyle{headings}

\begin{multicols}{2}

\label{st\stat}



\section{Введение}

Сети пятого и~последующих поколений будут использовать терагерцевый 
диапазон радиочастот, чтобы обеспечить сверхвысокую скорость передачи данных и~пропускную способность. Однако использование миллиметровых волн связано со 
сложностями из-за потери сигнала при про\-хож\-де\-нии препятствий. Таким образом, для 
обеспечения производительности сетей~5G становится крайне важным моделирование 
распространения сигнала. Формула Шен\-но\-на--Харт\-ли с~формулой Фрииса задают 
пропускную способность канала
$$
C=B \log_2 \left(1+\fr{P_t G_t G_r}{(N+I) \mathrm{PL}}\right),
$$
 где $B$~--- полоса 
пропускания канала; $P_t$~--- мощ\-ность передающей антенны; $G_t$~--- коэффициент 
усиления передающей антенны; $G_r$~--- коэффициент усиления приемной антенны; $N$~--- мощ\-ность шума; 
$I$~--- мощ\-ность интерференции; $\mathrm{PL}$~--- мощ\-ность затухания 
сигнала (path loss, PL) на расстоянии от передающей антенны до приемной 
антенны~\cite{Moltchanov2022a}.
Пропускная способность канала уже далее используется в~управ\-ле\-нии занятием 
радиоресурсов базовой станции (БС) для соблюдения необходимого качества обслуживания 
пользователей по требуемой ско\-рости передачи данных.

 Ввиду того что пользователи находятся на разных расстояниях от БС, 
значения мощностей затухания сигнала будут случайными. Как показано в~работе~\cite{Hmamouche2021}, для учета влияния на пропускную\linebreak
 способность канала 
случайного положения пользователей в~соте применяется стохастическая гео\-мет\-рия. 
Рассмотреть совместное занятие радиоресурсов и~случайный характер поведения 
пользователей позволяет модель на основе аппарата \mbox{ресурсных} сис\-тем массового 
обслуживания~\cite{Naumov2016, Gorbunova2018}. Такие модели применяются для 
исследования различных сценариев развертывания сетей 
5G~\cite{Moltchanov2022b, Markova2019}, например при анализе совместного 
обслуживания трафика со сверхнизкой задержкой и~широкополосного трафика~\cite{Kochetkova2021}.

\begin{figure*}[b] %fig1
\vspace*{-6pt}
      \begin{center}
     \mbox{%
\epsfxsize=124.62mm 
\epsfbox{koc-1.eps}
}
\end{center}
\vspace*{-9pt}
\Caption{Схема системной модели}
\label{fig1}
\end{figure*}

Модели для описания мощности $\mathrm{PL}$ затухания сигнала для разных сценариев 
отражены в~спецификации 3GPP TR 38.901~\cite{3GPP38901}. И~если при проведении 
имитационного моделирования исследователи по большей части полностью реализуют 
эти модели~\cite{Bolla2023}, то при построении моделей стохастической гео\-мет\-рии 
зачастую применяется упрощенный вид формул.
В~обзоре~\cite{Hmamouche2021} рассмотрены различные виды функциональной 
зависимости затухания мощ\-ности сигнала от расстояния между пользователем 
и~БС, которые применяют исследователи. Например, для простоты 
расчетов в~работе~\cite{Moltchanov2022b} используются упрощенные формулы без 
учета ку\-соч\-но-за\-дан\-но\-го вида функции для прямой видимости и~максимума нескольких 
величин мощностей PL для непрямой видимости при по\-стро\-ении~ФР.

В данной статье получена ФР затухания мощ\-ности сигнала при случайном 
расположении пользователей в~соответствии со сценариями 3GPP TR~38.901 
развертывания сети 5G. Использованы формулы из этой спецификации, где приведены 
зависимости PL от расстояния между пользователем и~БС. В~данной 
статье закон распределения пользователей в~соте взят произвольный, а~для 
численного анализа~--- в~соответствии с~типовыми рекомендованными значениями 
параметров сценариев.



\section{Затухание сигнала как функция от параметров сценария 3GPP} \label{sec2}

При исследовании распространения сигнала необходимо учитывать множество 
параметров сети, таких как частота, основные характеристики местности, высота 
принимающей и~передающей антенн, конфигурация антенн и~другие факторы. Для 
упрощения расчетов мощности PL затухания сигнала стандартом 3GPP TR~38.901~\cite{3GPP38901} 
были выделены основные сценарии развертывания сети~5G: 
мак\-ро\-со\-та в~городе (urban macro, UMa), микросота в~городе (urban micro, UMi), 
мак\-ро\-со\-та в~сельской местности (rural macro, RMa), точка доступа внут\-ри 
помещения (indoor hotspot, InH) и~крытая фабрика (indoor factory, InF),~--- 
и~путем экспериментов были получены эмпирические модели затухания сигнала для них. 
На основе этих моделей и~в~предположении случайного характера поведения 
пользователей в~данном разделе получена ФР мощности затухания сигнала с~учетом 
особенностей, описанных в~данной спецификации.


Рассмотрим общее описание предлагаемых сценариев (рис.~\ref{fig1}). Пусть 
передающая антенна БС расположена на высоте~$h_{\mathrm{BS}}$, 
использует несущую частоту~$f_c$ и~создает покрытие радиуса~$R$. 
Пользовательские устройства (ПУ) находятся на высоте~$h_{\mathrm{UT}}$, а~проекция 
расстоянии от ПУ до БС со\-став\-ля\-ет $d$.

В зависимости от своего расположения ПУ может находиться в~зоне прямой видимости 
(line-of-sight, LOS) с~устойчивым уровнем сигнала или вне этой зоны (non-line-of-sight, NLOS). 
Если ПУ расположено на расстоянии~$d$, то ве\-ро\-ят\-ность того, что 
ПУ находится в~зоне прямой видимости, пред\-став\-ля\-ет собой кусочно-заданную 
функцию:
\begin{equation}
\label{eq1}
{\mathrm{Pr}}_{\mathrm{LOS}}(d)=
\begin{cases}
{\mathrm{Pr}}_1^{\mathrm{LOS}}(d), & 0=r_0 \leq d < r_1; \\
{\mathrm{Pr}}_2^{\mathrm{LOS}}(d), & r_1 \leq d < r_2; \\
\cdots & \cdots \\
{\mathrm{Pr}}_I^{\mathrm{LOS}}(d), & r_{I-1} \leq d \leq r_I=R,
\end{cases}
\end{equation}
где радиусы $R_i$ определяют границы интервалов. 

Тогда мощ\-ность $\mathrm{PL}(d)$ 
затухания сигнала примет вид:
\begin{multline}
\label{eq2}
\mathrm{PL}\,(d)= \mathrm{PL}_{\mathrm{LOS}}(d)  {\mathrm{Pr}}_{\mathrm{LOS}}(d) + {}\\
{}+\mathrm{PL}_{\mathrm{NLOS}}(d)  \left[1-
{\mathrm{Pr}}_{\mathrm{LOS}}(d)\right].
\end{multline}

Мощность затухания сигнала в~условиях прямой видимости LOS описывается ку\-соч\-но-за\-дан\-ной функцией
\begin{multline}
\label{eq3}
\mathrm{PL}^{\mathrm{LOS}}(d)={}\\
{}=
\begin{cases}
\mathrm{PL}_1^{\mathrm{LOS}}(d), & 0=d_0 \leq d < d_1;\\
\mathrm{PL}_2^{\mathrm{LOS}}(d), & d_1 \leq d < d_2;\\
\cdots & \cdots \\
\mathrm{PL}_J^{\mathrm{LOS}}(d), & d_{J-1} \leq d \leq d_J=R,
\end{cases}
\end{multline}
где $d_j$~--- границы интервалов (break point distance), а~в~условиях непрямой 
видимости NLOS пред\-став\-ля\-ет собой максимум
\begin{multline}
\label{eq4}
\mathrm{PL}_{\mathrm{NLOS}}(d) = {}\\
\!\!{}=\!
\max\left(\mathrm{PL}^{\mathrm{LOS}}(d),\mathrm{PL}^{\mathrm{NLOS}}_1(d), \ldots, \mathrm{PL}^{\mathrm{NLOS}}_K(d)\right)\!.\!\!
\end{multline}

Каждая из компонент функций для случаев LOS и~NLOS имеет схожую структуру:
\begin{multline}
 \mathrm{PL}^{l}_m(d)[\mathrm{dB}] = \alpha_m^{l}[\mathrm{dB}]+\beta_m^{l}[\mathrm{dB}]\log_{10}{D(d)},
 \\
 \mathrm{PL}^{l}_m(d) = \alpha_m^{l} \cdot D^{\beta_m^{l}}(d),
\\
 l=\begin{cases}
 \mbox{``}\mathrm{LOS}\mbox{''}, & m=j=\overline{0,J}\,; \\
 \mbox{``}\mathrm{NLOS}\mbox{''},& m=k=\overline{0,K}\,,
 \end{cases}
\label{eq5}
\end{multline}
где $D(d)=\sqrt{d^2+(h_{\mathrm{BS}}\hm-h_{\mathrm{UT}})^2}$~--- расстояние от ПУ до БС в~трехмерном 
пространстве; $\alpha$ и~$\beta$~--- коэффициенты модели затухания сигнала~--- 
константы для каждого из сценариев 3GPP TR~38.901.



\section{Функция распределения затухания сигнала при~случайном расположении 
пользователей} \label{sec3}

Примем теперь, что расстояние между ПУ и~БС~--- случайная величина (СВ)~$\xi_d$ 
со значениями~$d$ и~ФР~$F_{\xi_d}(x)$. Тогда расстояние от ПУ до БС в~трехмерном 
пространстве~$\xi_D$ будет функцией от СВ~$\xi_d$ с~ФР

\noindent
\begin{multline*}
F_{\xi_D}(x)  =
\mathrm{Pr}\,(\xi_D \leq x) ={}\\
{}=
\mathrm{Pr}\left(\sqrt{\xi_d^2+(h_{\mathrm{BS}}- h_{\mathrm{UT}})^2} \leq x \right) ={} \\
{} = \mathrm{Pr}\left(\xi_d \leq \sqrt{x^2-(h_{\mathrm{BS}}- h_{\mathrm{UT}})^2} \right) ={}\\
{}=
F_{\xi_d}\left(\sqrt{x^2-(h_{\mathrm{BS}}- h_{\mathrm{UT}})^2} \right).
\end{multline*}

Случайная величина $\xi_m^l$~--- компонента функции затухания сигнала~--- зависит от СВ $\xi_D$ и~по формуле~(\ref{eq5}) имеет ФР
\begin{multline*}
F_{\xi_m^l}(x)  =
\mathrm{Pr}\,(\xi_m^l \leq x) =
\mathrm{Pr}\left(\alpha_m^{l}  ({\xi}_D)^{\beta_m^{l}} \leq x \right) = {}\\
{}=
\mathrm{Pr}\left(\xi_D \leq \left(\fr{x}{\alpha_m^{l}}\right)^{{1}/{\beta_m^{l}}} \right) = 
F_{\xi_D}\left( \left( \fr{x}{\alpha_m^{l}}\right)^{{1}/{\beta_m^{l}}} 
\right) ={}\\
{}=  F_{\xi_d}\left( \sqrt{ 
\left(\fr{x}{\alpha_m^{l}}\right)^{{2}/{\beta_m^{l}}} - \left(h_\mathrm{BS}-h_\mathrm{UT}\right)^2 }\right), \\
 l=\begin{cases}
 \mbox{``}\mathrm{LOS}\mbox{''}, &  m=j=\overline{0,J}\,; \\
 \mbox{``}\mathrm{NLOS}\mbox{''}, & m=k=\overline{0,K}\,.
 \end{cases}
\end{multline*}

Для затухания сигнала в~условиях прямой видимости ФР СВ~$\xi_{\mathrm{LOS}}$ по 
формуле~(\ref{eq3}) примет вид:
\begin{multline}
F_{\xi_{\mathrm{LOS}}}(x) =
\mathrm{Pr}\,(\xi_{\mathrm{LOS}} \leq x) = {} \\
{}= \sum\limits_{j=1}^J  \mathrm{Pr}\left(\xi_{\mathrm{LOS}} \leq x \mid d_{j-1} \leq \xi_d < d_j\right) \times{}\\
{}\times \mathrm{Pr}\left(d_{j-1} \leq  \xi_d < d_j\right) = {}\\
{} = \sum\limits_{j=1}^J F_{\xi_j^{\mathrm{LOS}}}(x) \left[ F_{\xi_d}(d_j) - 
F_{\xi_d}(d_{j-1}) \right], 
\label{eq6}
\end{multline}
а для непрямой видимости ФР СВ $\xi_{\mathrm{NLOS}}$ по формуле~(\ref{eq4}) 
и~с~учетом~\cite{Ventzel2018} запишем как
\begin{multline}
F_{\xi_{\mathrm{NLOS}}}(x)  =
\mathrm{Pr}\,(\xi_{\mathrm{NLOS}} \leq x) ={}\\
{}=
\mathrm{Pr}\left(\max{\left(\xi_{\mathrm{LOS}}, \; \xi^{\mathrm{NLOS}}_1, \ldots, \xi^{\mathrm{NLOS}}_K \right)} 
\leq x \right) = {} \\
{} = \mathrm{Pr}\left(\xi_{\mathrm{LOS}} \leq x, \; \xi^{\mathrm{NLOS}}_1 \leq x, \ldots, \xi^{\mathrm{NLOS}}_K 
\leq x \right) = {}\\
{} = \mathrm{Pr}\,(\xi_{\mathrm{LOS}} \leq x)  \prod\limits_{k=1}^K{\mathrm{Pr}\left(\xi^{\mathrm{NLOS}}_k \leq x 
\right)} = {}\\
{}=
F_{\xi_{\mathrm{LOS}}}(x)  \prod\limits_{k=1}^K F_{\xi_k^{\mathrm{NLOS}}}(x). 
\label{eq7}
\end{multline}

Наконец, ФР СВ $\xi_{\mathrm{PL}}$ затухания сигнала по формуле~(\ref{eq2}) запишем 
следующим образом:
\begin{multline*}
F_{\xi_{\mathrm{PL}}}(x) =
\mathrm{Pr}\,(\xi_{\mathrm{PL}} \leq x) ={}\\
{}=
\mathrm{Pr}\left(\xi_{\mathrm{LOS}}  \xi_{\mathrm{Pr}_{\mathrm{LOS}}} + \xi_{\mathrm{NLOS}} \left[1-
\xi_{\mathrm{Pr}_{\mathrm{LOS}}}\right] \leq x \right),
\end{multline*}
где $\xi_{\mathrm{Pr}_{\mathrm{LOS}}}$~--- СВ вероятности расположения ПУ в~зоне прямой 
видимости~(\ref{eq1}).
Функция распределения $F_{\xi_{\mathrm{PL}}}(x)$ будет приближенно представлять собой свертку.



\section{Численный анализ}


\begin{figure*}[b]\small
\begin{center}
\tabcolsep=4pt
\begin{tabular}{|c|l|c|c|}

\multicolumn{4}{c}{Коэффициенты модели затухания сигнала для UMa и~UMi}\\
\multicolumn{4}{c}{\ }\\[-6pt]
\hline
 Зона& \multicolumn{1}{c|}{[dB]} & UMa & UMi \\
\hline
&&&\\[-9pt]
 & $\alpha_1^{\mathrm{LOS}}$ & $28+20 \log_{10}{f_c}$ & $32{,}4+20\log_{10}{f_c}$\\
% \cline{2-4}
 LOS& $\beta_1^{\mathrm{LOS}}$ & $22$ & $21$\\
% \cline{2-4}
 & $\alpha_2^{\mathrm{LOS}}$ & $28+20 \log_{10}{f_c}-9\log_{10}{\left(d_1^2+(h_{\mathrm{BS}}-h_{\mathrm{UT}})^2\right)} $ 
 & $32{,}4+20\log_{10}{f_c}- 9{,}5\log_{10}{\left(d_1^2+(h_{\mathrm{BS}}-h_{\mathrm{UT}})^2\right)}$\\
% \cline{2-4}
 & $\beta_2^{\mathrm{LOS}}$ & $40$ & $40$\\
\hline
&&&\\[-9pt]
& $\alpha_1^{\mathrm{NLOS}}$ & $13{,}54+20\log_{10}{f_c}-0{,}6(h_{\mathrm{UT}}-1,5)$ & 
$22{,}4+21{,}3\log_{10}{f_c} - 0{,}3(h_{\mathrm{UT}}-1{,}5)$ \\
 %\cline{2-4}
NLOS  & $\beta_1^{\mathrm{NLOS}}$ & $39{,}08$ & $35{,}3$\\
 %\cline{2-4}
 & $\alpha_{\mathrm{Opt}}$ & $32{,}4+20\log_{10}{f_c}$ & $32{,}4+20\log_{10}{f_c} $\\
 %\cline{2-4}
 & $\beta_{\mathrm{Opt}}$ & $30$ & $31{,}9$\\
 \hline
\end{tabular}
\end{center}
%\end{table*}
%\begin{figure*}[b] %fig2
\setcounter{figure}{1}
\vspace*{12pt}
      \begin{center}
     \mbox{%
\epsfxsize=163mm 
\epsfbox{koc-2.eps}
}
\end{center}
\vspace*{-10pt}
  \Caption{Функции распределения PL для LOS~(\textit{а}) и ~NLOS~(\textit{б}) для сценариев UMa (черные кривые) и~UMi~(серые кривые):
  \textit{1}~--- расчет по формулам~(6) для LOS и~(7) для NLOS; \textit{2}~--- расчет по упрощенным формулам}
 \label{fig:1}
 \end{figure*}

 В спецификации 3GPP TR 38.901 указаны основные диапазоны значений параметров 
для сценариев развертывания сетей~5G. Рассмотрим сценарии макросоты UMa и~микросоты UMi в~городе со следующим набором исходных данных: радиус действия БС 
$R\hm=5000$~м, центральная частота $f_c\hm=6$~ГГц, высота ПУ
$h_{\mathrm{UT}}\hm=1{,}5$~м, высоты БС для UMa $h_{\mathrm{BS}}\hm=25$ м и~для UMi $h_{\mathrm{BS}}\hm=10$~м. 
Предположим, что пользователи распределены равномерно в~об\-ласти действия БС 
радиусом~$R$.

Согласно упрощенным формулам, подобным тем, что описаны в~работе~\cite{Moltchanov2022b}, ФР мощности PL затухания сигнала в~условиях 
прямой и~непрямой видимости могут быть представлены как 
$$
F_{\xi_\mathrm{LOS}}(x) = 
F_{\xi_1^{\mathrm{LOS}}}(x)\,;
$$

\noindent
\begin{multline*}
F_{\xi_\mathrm{NLOS}}(x)=F_{\mathrm{Opt}}(x)={}\\
{}= F_{\xi_d}\left( \sqrt{ 
\left(\fr {x}{\alpha_{\mathrm{Opt}}}\right)^{{2}/{\beta_{\mathrm{Opt}}}} - 
(h_\mathrm{BS}-h_\mathrm{UT})^2 }\,\right).
\end{multline*}

\noindent
 Коэффициенты модели затухания сигнала $\alpha_m^{l}$ 
[dB] и~$\beta_m^{l}$ [dB] для сценариев UMa и~UMi, согласно~\cite{3GPP38901}, 
представлены в~таб\-лице.


 
 \begin{figure*} %fig3
 \vspace*{1pt}
      \begin{center}
     \mbox{%
\epsfxsize=163.204mm 
\epsfbox{koc-3.eps}
}
\end{center}
\vspace*{-15pt}
 \Caption{Разница значений ФР PL для LOS~(\textit{а}) и~NLOS~(\textit{б}):
 \textit{1}~--- UMa; \textit{2}~--- UMi}
 \label{fig:2}
  \vspace*{-3pt}
 \end{figure*}

 
 

В рамках данного численного анализа покажем графики ФР моделей PL затухания 
сигнала для сценариев UMa и~UMi в~условиях прямой и~непрямой видимости по 
формулам~(\ref{eq6}) и~(\ref{eq7}) и~упрощенным формулам. Графики с~полученными 
результатами представлены на рис.~\ref{fig:1} и~3. Во всех случаях 
график ФР по упрощенным формулам идет выше, что представляет собой 
верхнюю оценку. Однако при дальнейших расчетах с~использованием упрощенных 
формул пропускная спо\-соб\-ность и~максимальное число обслуженных пользователей 
в~соте оказываются занижены. Таким образом, авторы рекомендуют при использовании 
модели затухания сигнала как компоненты, например в~ресурсных сис\-те\-мах массового 
обслуживания для анализа беспроводных сетей, использовать формулы (\ref{eq6}) и~(\ref{eq7}).


\section{Заключение}

В статье исследована модель затухания сигнала по формулам сценариев 3GPP TR 
38.901. Была получена функция распределения (ФР) мощности затухания сигнала при 
случайном (произвольный закон) расположении пользователей в~зоне покрытия 
БС. Она учитывает ку\-соч\-но-за\-дан\-ный вид функции для LOS и~максимум 
нескольких величин для NLOS. Проведен численный анализ для данных из 
спецификации 3GPP TR 38.90 для сценариев макро- и~микросот в~городе для 
сравнения ФР, представленных в~данной работе, и~ФР, полученных с~по\-мощью 
упрощенных формул. Результат анализа показал, что ФР по упрощенным формулам дает 
оценку сверху, что может понижать точность расчетов пропускной способности 
канала. Отметим, что авторы статьи не ставили перед собой задачу аналитического 
сравнения двух ФР, а~хотели бы обратить внимание на несложный вид полученных 
формул, которые рекомендуют для использования как компоненту в~ресурсных 
системах массового обслуживания при моделировании беспроводных сетей 5G/6G. 
Задачей дальнейшего исследования станет разработка ресурсной системы массового 
обслуживания с~учетом случайного расположения пользователей в~соте через 
пред\-став\-лен\-ную ФР для оценки схемы приоритетного обслуживания узкополосного 
трафика и~прерывания обслуживания широкополосного трафика в~се\-ти~5G.


\vspace*{-12pt}

{\small\frenchspacing
 {\baselineskip=10.5pt
 %\addcontentsline{toc}{section}{References}
 \begin{thebibliography}{99}
 
 \vspace*{-2pt}
 
\bibitem{Moltchanov2022a}
\Au{Молчанов~Д.\,А., Бегишев~В.\,О., Самуйлов~К.\,Е., Кучерявый~Е.\,А.}
Сети 5G/6G: архитектура, технологии, методы анализа и~расчета.~--- 
М.: РУДН, 2022. 516~с.

\bibitem{Hmamouche2021}
\textit{Hmamouche~Y., Benjillali~M., Saoudi~S., Yanikomeroglu~H., Renzo~M.\,D.}
New trends in stochastic geometry for wireless networks: A~tutorial and survey~//
P.~IEEE, 2021. Vol.~109. No.\,7. P.~1200--1252.
doi: 10.1109/JPROC.2021.3061778.

\bibitem{Naumov2016}
\Au{Наумов~В.\,А., Самуйлов~К.\,Е.}
О~связи ресурсных сис\-тем массового обслуживания с~сетями Эрланга~//
Информатика и~её применения, 2016. Т.~10. Вып.~3. С.~9--14.
doi: 10.14357/19922264160302.

\bibitem{Gorbunova2018} %4
\Au{Горбунова~А.\,В., Наумов~В.\,А., Гайдамака~Ю.\,В., Самуйлов~К.\,Е.}
Ресурсные системы массового обслуживания как модели беспроводных сис\-тем связи~//
Информатика и~её применения, 2018. Т.~12. Вып.~3. С.~48--55.
doi: 10.14357/19922264180307.

\bibitem{Markova2019} %5
\Au{Маркова~Е.\,В., Гольская~А.\,А., Дзантиев~И.\,Л., Гудкова~И.\,А., 
Шоргин~С.\,Я.}
Сравнительный анализ показателей эффективности модели беспроводной сети 
межмашинного взаимодействия, работающей в~рамках двух политик разделения 
радиоресурсов~//
Информатика и~её применения, 2019. Т.~13. Вып.~1. С.~108--116.
doi: 10.14357/19922264190115.

\bibitem{Moltchanov2022b} %6
\Au{Moltchanov~D.\,A., Sopin~E.\,S., Begishev~V.\,O., Sa\-muylov~A.\,K., Koucheryavy~Y.\,A., 
Sa\-mouylov~K.\,E.}
A~tutorial on mathematical modeling of 5G/6G millimeter wave and terahertz 
cellular systems~//
IEEE Commun. Surv.  Tut., 2022. Vol.~24. No.\,2. P.~1072--1116.
doi: 10.1109/ COMST.2022.3156207.



\bibitem{Kochetkova2021}
\Au{Кочеткова~И.\,А., Кущазли~А.\,И., Харин~П.\,А., Шоргин~С.\,Я.}
Модель схемы приоритетного доступа трафика URLLC и~eMBB в~сети пятого поколения в~виде ресурсной сис\-те\-мы массового обслуживания~//
Информатика и~её применения, 2021. Т.~15. Вып.~4. С.~87--92.
doi: 10.14357/19922264210412.

\bibitem{3GPP38901}
3GPP TR 38.901. Study on channel model for frequencies from~0.5 to~100~GHz, 
2024. Release 17.1.0.

\bibitem{Bolla2023}
\Au{Bolla~R., Bruschi~R., Lombardo~C., Mohammadpour~A., Trivisonno~R., 
Poe~W.\,Y.}
A~5G multi-gNodeB simulator for ultra-reliable 0.5--100~GHz communication in 
indoor Industry~4.0 environments~//
Comput. Netw., 2023. Vol.~237. Art. No.\,110103.
doi: 10.1016/j.comnet.2023.11010.

\bibitem{Ventzel2018}
\textit{Вентцель~Е.\,С., Овчаров~Л.\,А.}
Теория вероятностей и~ее инженерные приложения.~--- М.: Юстиция, 
2018. 480~c.

\end{thebibliography}

 }
 }

\end{multicols}

\vspace*{-10pt}

\hfill{\small\textit{Поступила в~редакцию 15.03.24}}

%\vspace*{10pt}

%\pagebreak

\newpage

\vspace*{-28pt}

%\hrule

%\vspace*{2pt}

%\hrule



\def\tit{STOCHASTIC PATH LOSS MODEL IN~5G~NETWORK DEPLOYMENT SCENARIOS: A~STUDY BASED ON~3GPP~TR~38.901}


\def\titkol{Stochastic Path Loss Model in 5G Network Deployment Scenarios: A~Study Based on 3GPP TR 38.901}


\def\aut{E.\,D.~Makeeva$^{1,2}$, I.\,A.~Kochetkova$^{1,3}$, and~S.\,Ya.~Shorgin$^{3}$}

\def\autkol{E.\,D.~Makeeva, I.\,A.~Kochetkova, and~S.\,Ya.~Shorgin}

\titel{\tit}{\aut}{\autkol}{\titkol}

\vspace*{-8pt}


\noindent
$^1$RUDN University, 6 Miklukho-Maklaya Str., Moscow 117198, Russian Federation

\noindent
$^2$V.\,A.~Trapeznikov Institute of Control Science of the Russian Academy of 
Sciences, 65~Profsoyuznaya Str., Moscow\linebreak
$\hphantom{^1}$117997, Russian Federation

\noindent
$^3$Federal Research Center ``Computer Science and Control'' of the Russian 
Academy of Sciences, 44-2~Vavilov\linebreak
$\hphantom{^1}$Str., Moscow 119333, Russian Federation

\def\leftfootline{\small{\textbf{\thepage}
\hfill INFORMATIKA I EE PRIMENENIYA~--- INFORMATICS AND
APPLICATIONS\ \ \ 2024\ \ \ volume~18\ \ \ issue\ 2}
}%
 \def\rightfootline{\small{INFORMATIKA I EE PRIMENENIYA~---
INFORMATICS AND APPLICATIONS\ \ \ 2024\ \ \ volume~18\ \ \ issue\ 2
\hfill \textbf{\thepage}}}

\vspace*{3pt}




\Abste{The fifth-generation (5G) and beyond networks will utilize radio frequencies in the terahertz 
spectrum, enabling extremely high data transmission rates. However, signal loss may occur when signals 
pass through obstacles, making it crucial to simulate signal propagation using stochastic geometry 
and apply up-to-date models for signal attenuation. The 3GPP TR~38.901 specification includes models that describe signal
 attenuation in various 5G~network scenarios using empirical formulas. Nevertheless, simpler formulas are typically employed 
 to create models based on stochastic geometry. The authors present the cumulative distribution function 
 for path loss at random user locations according to the scenarios described in 3GPP TR~38.901. In numerical examples, it is shown that the difference 
in values with the simplified formula can be significant and lead to underestimation of the network's capacity}

\KWE{wireless network; 5G; 3GPP TR 38.901; path loss; line-of-sight (LOS); non-line-of-sight (NLOS); stochastic geometry}



\DOI{10.14357/19922264240204}{EKLCAP}

\vspace*{-12pt}

\Ack

\vspace*{-3pt}

\noindent
The publication has been supported by the RUDN University Scientific Projects 
Grant System, project No.\,025319-2-000.


  \begin{multicols}{2}

\renewcommand{\bibname}{\protect\rmfamily References}
%\renewcommand{\bibname}{\large\protect\rm References}

{\small\frenchspacing
 {%\baselineskip=10.8pt
 \addcontentsline{toc}{section}{References}
 \begin{thebibliography}{99} 
\bibitem{Moltchanov2022a-1}
\Aue{Moltchanov,~D.\,A., V.\,O.~Begishev, K.\,E.~Samouylov, and Y.\,A.~Koucheryavy.}
2022.
\textit{Seti 5G/6G: arkhitektura, tekhnologii, metody analiza i~rascheta}
[The 5G/6G networks: Architecture, technologies, analysis methods, and calculations].
Moscow: RUDN University. 516~p.

\bibitem{Hmamouche2021-a}
\Aue{Hmamouche,~Y., M.~Benjillali, S.~Saoudi, H.~Yanikomeroglu, and 
M.\,D.~Renzo.}
2021.
New trends in stochastic geometry for wireless networks: A~tutorial and survey.
\textit{P.~IEEE}. 109(7):1200--1252.
doi: 10.1109/ JPROC.2021.3061778.

\bibitem{Naumov2016-1}
\Aue{Naumov,~V.\,A., and K.\,E.~Samouylov.}
2016. O~svyazi resursnykh sistem massovogo obsluzhivaniya s~setyami Erlanga
[On relationship between queuing systems with resources and Erlang networks].
\textit{Informatika i~ee Primeneniya~--- Inform Appl.} 10(3):9--14.
doi: 10.14357/ 19922264160302.

\bibitem{Gorbunova2018-1} %4
\Aue{Gorbunova,~A.\,V., V.\,A.~Naumov, Yu.\,V.~Gaidamaka, and K.\,E.~Samouylov.}
2018. Resursnye sistemy massovogo obsluzhivaniya kak modeli besprovodnykh sistem svyazi
[Resource queuing systems as models of wireless communication systems].
\textit{Informatika i~ee Primeneniya~--- Inform. Appl.} 12(3):48--55.
doi: 10.14357/19922264180307.



\bibitem{Markova2019-1} %5
\Aue{Markova,~E.\,V., A.\,A.~Golskaia, I.\,L.~Dzantiev, I.\,A.~Gudkova, and 
S.\,Ya.~Shorgin.}
2019. Sravnitel'nyy analiz pokazateley effektivnosti modeli besprovodnoy seti mezhmashinnogo 
vzaimodeystviya, rabotayushchey v~ramkakh dvukh politik razdeleniya radioresursov
[Comparative analysis of performance measures for a~wireless machine-to-machine 
network model operating within two radio resource management policies].
\textit{Informatika i~ee Primeneniya~--- Inform. Appl}. 13(1):108--116.
doi: 10.14357/ 19922264190115.


\bibitem{Moltchanov2022b-1} %6
\Aue{Moltchanov,~D.\,A., E.\,S.~Sopin, V.\,O.~Begishev, A.\,K.~Sa\-muy\-lov, 
Y.\,A.~Koucheryavy, and K.\,E.~Samouylov.}
2022.
A tutorial on mathematical modeling of 5G/6G millimeter wave and terahertz 
cellular systems.
\textit{IEEE Commun. Surv. Tut.} 24(2):1072--1116.
doi: 10.1109/COMST. 2022.3156207.

\bibitem{Kochetkova2021-1} %7
\Aue{Kochetkova,~I.\,A., A.\,I.~Kushchazli, P.\,A.~Kharin, and S.\,Ya.~Shorgin.}
2021. Model' skhemy prioritetnogo do\-stu\-pa trafika URLLC i~eMBB v~seti pyatogo pokoleniya v~vide resursnoy sistemy massovogo obsluzhivaniya
[Model for analyzing priority admission control of URLLC and eMBB communications 
in 5G networks as a~resource queuing system].
\textit{Informatika i~ee Primeneniya~--- Inform. Appl}. 15(4):87--92.
doi: 10.14357/19922264210412.

\bibitem{3GPP38901-1}
3GPP TR 38.901. 2023. Study on channel model for frequencies from~0.5 to~100~GHz, 
Release 17.1.0.

\bibitem{Bolla2023-1}
\Aue{Bolla,~R., R.~Bruschi, C.~Lombardo, A.~Mohammadpour, R.~Trivisonno, and 
W.\,Y.~Poe.}
2023.
A 5G multi-gNodeB\linebreak\vspace*{-12pt}

\pagebreak

\noindent
 simulator for ultra-reliable 0.5--100~GHz communication in 
indoor Industry 4.0 environments.
\textit{Comput. Netw.} 37:110103. doi: 10.1016/j.comnet.2023.11010.

\bibitem{Ventzel2018-1}
\Aue{Ventzel,~E.\,S. and L.\,A.~Ovcharov.}
2018.
\textit{Teoriya veroyatnostey i~ee inzhenernye prilozheniya}
[Probability theory and its engineering applications].
Moscow: Justice. 480~p.

\end{thebibliography}

 }
 }

\end{multicols}

\vspace*{-6pt}

\hfill{\small\textit{Received March 15, 2024}} 

\vspace*{-12pt}


\Contr

\vspace*{-3pt}

\noindent
\textbf{Makeeva Elena D.} (b.\ 1996)~--- PhD student, Department of Probability 
Theory and Cyber Security, RUDN University, 6~Miklukho-Maklaya Str., Moscow 
117198, Russian Federation; junior scientist, V.\,A.~Trapeznikov Institute of 
Control Science of the Russian Academy of Sciences, 65~Profsoyuznaya Str., 
Moscow 117997, Russian Federation; \mbox{elena-makeeva-96@mail.ru}

\vspace*{3pt}

\noindent
\textbf{Kochetkova Irina A.} (b.\ 1985)~--- Candidate of Science (PhD) in physics 
and mathematics, associate professor, Department of Probability Theory and Cyber 
Security, RUDN University, 6~Miklukho-Maklaya Str., Moscow 117198, Russian 
Federation; senior scientist, Federal Research Center ``Computer Science and 
Control'' of the Russian Academy of Sciences, 44-2~Vavilov Str., Moscow 119333, 
Russian Federation; \mbox{kochetkova-ia@rudn.ru}

\vspace*{3pt}

\noindent
\textbf{Shorgin Sergey Ya.} (b.\ 1952)~--- Doctor of Science in physics and 
mathematics, professor, principal scientist, Federal Research Center ``Computer Science and Control'' of the Russian Academy of 
Sciences, 44-2~Vavilov Str., Moscow 119133, Russian Federation; 
\mbox{sshorgin@ipiran.ru}





\label{end\stat}

\renewcommand{\bibname}{\protect\rm Литература}  %12
\def\stat{flerov}

\def\tit{АВТОМАТИЗИРОВАННАЯ СИСТЕМА ВЕСОВОГО 
ПРОЕКТИРОВАНИЯ САМОЛЕТОВ}

\def\titkol{Автоматизированная система весового 
проектирования самолетов}

\def\aut{Л.\,Л.~Вышинский$^1$, Ю.\,А.~Флеров$^2$, Н.\,И.~Широков$^1$}

\def\autkol{Л.\,Л.~Вышинский, Ю.\,А.~Флеров, Н.\,И.~Широков}

\titel{\tit}{\aut}{\autkol}{\titkol}

\index{Вышинский Л.\,Л.}
\index{Флеров Ю.\,А.}
\index{Широков Н.\,И.}
\index{Vyshinsky L.\,L.}
\index{Flerov Yu.\,A.}
\index{Shirokov N.\,I.}




%{\renewcommand{\thefootnote}{\fnsymbol{footnote}} \footnotetext[1]
%{Работа выполнена при финансовой поддержке РФФИ (проект 17-01-00816).}}


\renewcommand{\thefootnote}{\arabic{footnote}}
\footnotetext[1]{Вычислительный центр им.\ А.\,А.~Дородницына Федерального исследовательского 
центра <<Информатика и~управ\-ле\-ние>> Российской академии наук, 
\mbox{Wysh@ccas.ru}}
\footnotetext[2]{Вычислительный центр им.\ А.\,А.~Дородницына Федерального исследовательского 
центра <<Информатика и~управ\-ле\-ние>> Российской академии наук, 
fler@ccas.ru}
%\footnotetext[3]{Вычислительный центр им.\ А.\,А.~Дородницына Федерального исследовательского 
%центра <<Информатика и~управ\-ле\-ние>> Российской академии наук, 
%\mbox{Wysh@ccas.ru}}

%\vspace*{-6pt}


 
  \Abst{Статья посвящена вопросам автоматизации задач весового проектирования 
самолетов. Весовые и~мас\-со\-во-инер\-ци\-он\-ные параметры являются одними из основных 
величин, влияющих на эксплуатационные характеристики самолетов. Информационной 
основой системы служит весовая модель самолета. Описывается структура весовой 
модели и~даны характеристики отдельным ее компонентам. Показана программная 
реализация системы, которая выполнена в~рамках архитектуры кли\-ент--сер\-вер. 
Автоматизированная система весового проектирования (АСВП)
реализована с~использованием 
про\-грам\-мно-ин\-стру\-мен\-таль\-но\-го комплекса <<Генератор проектов>> (технология ГП), 
который был разработан в~Вычислительном центре Российской академии наук. Создание 
ин\-фор\-ма\-ци\-он\-но-вы\-чис\-ли\-тель\-ных сис\-тем в~рамках технологии ГП базируется на так 
называемом <<проектном подходе>>, когда по формальному описанию системы автоматически 
генерируются тексты программного кода для клиентских и~серверных компонент системы.}
   
  \KW{математическое моделирование; автоматизация проектирования; самолет; весовое 
проектирование; весовая модель; дерево конструкции; генератор проектов; генерация 
программного кода; архитектура кли\-ент--сер\-вер}

  \DOI{10.14357/19922264180103} 
  
\vspace*{12pt}


\vskip 10pt plus 9pt minus 6pt

\thispagestyle{headings}

\begin{multicols}{2}

\label{st\stat}
   
\section{Введение}

  Развитие и~повсеместное использование информационных технологий за 
последние несколько десятилетий существенно изменили традиционный 
процесс проектирования и~создания различных инженерных систем, 
сооружений, машин. Во многих проектных организациях давно отказались от 
ко\-гда-то привычных инструментов конструктора~--- кульмана 
и~логарифмической линейки. 
%
Сейчас первые эскизы новых проектов 
появляются чаще не на бумаге, как было всегда, а~на экране монитора. Этому 
способствует широкий спектр имеющихся систем автоматизированного 
проектирования. В~российских авиационных конструкторских бюро, например, уже давно 
применяются такие CAD (computer aided design)
сис\-те\-мы, как NX (Unigraphics), CATIA и~др. 
%
Эти развитые системы геометрического трех\-мер\-но\-го (3D) мо\-де\-ли\-ро\-ва\-ния позволяют 
создавать сложные по\-верх\-ности, конструировать любые детали, осуществлять 
сборку узлов, агрегатов и~сложнейших изделий. Однако построение 
геометрических моделей изделий является финальной стадией проектирования, 
за которой следует их реализация <<в~металле>>. Построению электронных 
геометрических макетов предшествует и~сопутствует решение множества 
расчетных задач, а~также задач анализа и~оптимизации в~разных областях инженерных 
знаний. В~авиастроении это аэродинамика, динамика полета, прочность, 
системы управления, двигателестроение и~пр. Все эти задачи 
требуют разработки разноплановых математических моделей и~специальных 
вычислительных программ. 
  
  Одной из важнейших технических характеристик самолета является его вес. 
При решении подавляющего большинства проектных и~конструкторских задач 
весовые параметры в~том или ином виде участвуют в~расчетах. Необходимость 
проведения весовых расчетов возникает на самых ранних шагах 
проектирования и~сопровождает все дальнейшие стадии разработки 
и~эксплуатации. 

В~процессе создания и~эксплуатации самолетов постоянно 
контролируются вес и~другие мас\-со\-во-инер\-ци\-он\-ные характеристики (МИХ)
всех размещаемых на борту систем, агрегатов, узлов и~деталей. Количество 
агрегатов, узлов и~деталей современных самолетов исчисляется 
десятками тысяч, поэтому в~авиастроении весовые расчеты, весовой анализ, 
весовой контроль выливаются в~сложную инженерную проблему и~выделяются 
в~целое направление инженерной деятельности, которое принято называть 
весовым проектированием~[1].
  
  Данная статья посвящена вопросам автоматизации задач весового 
проектирования самолетов. В~разные годы Вычислительным центром РАН\linebreak был 
разработан и~внедрен в~эксплуатацию ряд \mbox{программ}, решающих отдельные 
задачи весовых рас\-че\-тов летательных аппаратов (ЛА)~[2--4]. В~настоящей статье 
представлено описание интегрированной АСВП, предназначенной для использования на всех 
этапах жизненного цикла изделий. Она разработана как интерактивная 
многопользовательская информационная система кли\-ент-сер\-вер\-ной 
архитектуры с~централизованной базой данных. Информационным ядром 
и~основой АСВП является единая струк\-тур\-но-па\-ра\-мет\-ри\-че\-ская весовая модель 
самолета, описание которой дает довольно полное представление о~задачах, 
решаемых с~помощью АСВП.

\section{Структурно-параметрическая весовая модель самолета}

  Самолет является сложным техническим объ\-ектом, состоящим из множества 
различных \mbox{ком\-понентов}, функционально и~конструктивно связанных между 
собой. Под струк\-тур\-но-па\-ра\-мет\-ри\-че\-ской весовой моделью самолета 
здесь понимается база данных, которая содержит всю необходи\-мую 
информацию для проведения комплекса расчетов 
МИХ и~мас\-со\-во-цент\-ро\-воч\-ных данных (МЦД) 
самолета. Весовая модель состоит из нескольких структур, ориентированных на 
определенные группы параметров и~задач весового проектирования. Ниже 
перечислены основные структуры весовой модели, реализованные в~системе 
АСВП:
  \begin{itemize}
\item дерево конструкции самолета;
\item иерархия систем координат, связанных с~самолетом и~его агрегатами;
\item геометрические структуры весовой модели самолета;
\item каталог целевой нагрузки, размещаемой во внут\-рен\-них отсеках и~на 
подвесках;
\item реестр допустимых вариантов загрузки само\-лета;
\item таблицы тарировочных характеристик топливных баков;
\item таблицы характеристик выработки топлива.
\end{itemize}


  \subsection{Дерево конструкции самолета}

  Дерево конструкции самолета является центральной структурой весовой 
модели, которая отражает членение изделия на его составные части~--- 
системы, агрегаты, узлы, детали. В~базе данных весовой модели эта структура 
представлена в~виде многоуровневого корневого дерева $W \hm= (U, V)$, где 
вершинам $U \hm= \{U_i\}$ соответствуют различные\linebreak
 элементы конструкции. 
Ориентированные дуги дере\-ва, идущие из~$U_i$ в~$U_j$, означают вхождение 
конструкции~$U_j$ в~конструкцию~$U_i$ в~качестве ее составной части. 
Терминальными или висячими вершинами дерева конструкции будем называть 
вершины, у которых нет ни одной подчиненной конструкции.
  
  Многолетний опыт самолетостроения выработал устоявшиеся 
конструктивные схемы самолетов различного назначения. Существуют 
отраслевые стандарты и~классификаторы, которые вводят определения 
основных элементов конструкции самолетов. На рис.~1 показан пример 
представления в~АСВП нескольких верхних уровней дерева конструкции 
самолета. 


    

  Существующие классификаторы отражают лишь самые общие принципы 
построения конструкции самолетов. Разумеется, каждый новый проект 
самолета имеет свои конструктивные особенности, которые отражаются на 
структуре весовой модели. Дерево конструкции строится постепенно, сверху 
вниз, в~течение всего процесса проектирования самолета. 

 { \begin{center}  %fig1
 \vspace*{9pt}
\mbox{%
 \epsfxsize=77.216mm 
 \epsfbox{fle-1.eps}
 }

\vspace*{6pt}


\noindent
{{\figurename~1}\ \ \small{Дерево конструкции самолета}}
\end{center}
}

\addtocounter{figure}{1}
  
  Понятие <<конструкции>> в~данном контексте используется и~для 
обозначения любой вершины графа, и~для всего поддерева подчиненных 
конструкций этой вершине. Каждая конструкция дерева имеет уникальное 
в~рамках весовой модели обозначение, которое может быть шифром, кодом, 
идентификатором или чертежным номером конструкции. Разумеется, для более 
полного и~наглядного вербального представления конструкции  
в~струк\-тур\-но-па\-ра\-мет\-ри\-че\-ской модели можно задать ее текстовое 
описание.
  
  \textbf{Масса конструкции.} Основную содержательную и~необходимую 
информацию весовой модели содержит список значений  
МИХ, соответствующих каждой 
вершине дерева конструкций. Центральным параметром является масса. 
  
  На разных стадиях создания самолета, когда неизвестно точное значение 
массы, прибегают к~различным оценкам.  
В~струк\-тур\-но-па\-ра\-мет\-ри\-че\-ской весовой модели фиксируются 
перечисленные ниже оценки массы, которые соответствуют разным этапам 
проектирования:
  \begin{description}
\item[\,]  $M_{\mathrm{теор}}$~--- теоретическая масса~--- оценка массы, вычисленная на 
основании некоторой математической модели конструкции; 
  
\item[\,]  $M_{\mathrm{лим}}$~--- лимитная масса конструкции, уста\-нав\-ли\-ва\-емая на 
основании теоретических оценок и~используемая для весового контроля 
в~процессе детальной разработки конструкции;
  
\item[\,]  $M_{\mathrm{черт}}$~--- чертежная масса конструкции, рассчитанная по чертежу или по 
электронной гео\-мет\-ри\-че\-ской модели конструкции;
  
\item[\,]  $M_{\mathrm{креп}}$~--- масса крепежа конструкции~--- дополнение к~чертежной массе, 
которое учитывает мелкие детали конструкции, предназначенные для 
соединения подчиненных деталей (заклепки, болты, гайки, шайбы и~т.\,п.). 
Введение такой дополнительной массы позволяет избавить дерево конструкции 
от десятков и~сотен тысяч вершин, которые несут относительно небольшую 
нагрузку в~весовых характеристиках, но тем не менее их учет необходим при 
контроле веса. Масса крепежа распределяется по подчиненным конструкциям;  
\item[\,]  $M_{\mathrm{факт}}$~--- фактическая масса изготовленной 
и~взвешенной конструкции. 
Фактическая масса может задаваться не только для изготавливаемых 
конструкций ЛА, но и~для готовых по\-став\-ля\-емых 
изделий при их установке на борту.
\end{description}
  
  Порядок задания оценок массы диктуется логикой развития проекта. 
В~дереве конструкции все оценки массы, кроме $M_{\mathrm{лим}}$ и~$M_{\mathrm{креп}}$, 
суммируются по подчиненным вершинам снизу вверх. Однако если для 
некоторых терминальных значений одна или несколько оценок не определены, 
например некоторые детали конструкции не изготовлены и, стало быть, 
$M_{\mathrm{факт}}$ не определена, то и~для всех вышестоящих конструкций эти оценки не 
определены. При задании $M_{\mathrm{лим}}$ это правило может не соблюдаться. 
  
  На основании оценок массы определяется то расчетное значение массы 
конструкции, которое используется во всех расчетах на текущей стадии 
проекта: 
  $M$~--- текущая масса конструкции. Значение текущей массы \textit{для 
нетерминальных} конструкций определяется суммированием по подчиненным 
конструкциям. \textit{Для терминальных} вершин дерева конструкций 
применяется процедура определения текущей массы по первому известному 
значению из следующего списка в~указанном порядке: $M_{\mathrm{факт}}$, 
$M_{\mathrm{черт}}$\;+\;$M_{\mathrm{креп}}$, $M_{\mathrm{теор}}$, $M_{\mathrm{лим}}$.
  
  \textbf{Геометрия масс конструкции.} Кроме собственно массы в~весовой 
модели задаются или вычисляются значения характеристик, которые принято 
называть характеристиками геометрии масс: 
  \begin{description}
  \item[\,] $X$, $Y$ и $Z$~--- положение центра масс конструкции; 
  \item[\,] $L_x$, $L_y$ и $L_z$~--- габаритные параметры конструкции;
  \item[\,] $I_x$, $I_y$ и $I_z$~--- полные плоскостные моменты инерции;
  \item[\,]  $I_{xy}$, $I_{xz}$ и $I_{yz}$~--- полные центробежные моменты 
инерции;
  \item[\,] $I^c_x$, $I^c_y$ и  $I^c_z$~--- собственные плоскостные моменты 
инерции:
  \begin{align*}
  I^c_x &= I_x - M X^2\,;\\ 
  I^c_y &= I_y - M Y^2\,;\\ 
  I^c_z &= I_z - M Z^2\,;
 \end{align*}
  \item[\,] $I^c_{xy}$, $I^c_{xz}$ и~$I^c_{yz}$~--- собственные центробежные 
моменты инерции:
 \begin{align*}
  I^c_{xy} &= I_{xy}- M X Y\,;\\
   I^c_{xz} &= I_{xz}- M X Z\,;\\
   I^c_{yz} &= I_{yz}- M Y Z\,;
\end{align*}
  \item[\,] $J_x$, $J_y$ и $J_z$~--- собственные осевые моменты инерции 
конструкции:
  \begin{align*}
  J_x &= I^c_y + I^c_z\,;\\ 
  J_y &= I^c_x + I^c_z\,;\\
   J_z &= I^c_y + I^c_x\,;
  \end{align*}
  \item[\,] СК~--- система координат конструкции, в~которой задаются 
характеристики геометрии масс.
  \end{description}
  
  \begin{figure*} %fig2
  \vspace*{1pt}
 \begin{center}
 \mbox{%
 \epsfxsize=162mm 
 \epsfbox{fle-2.eps}
 }
 \end{center}
\vspace*{-9pt}
  \Caption{Основные параметры конструкций весовой модели самолета}
  \end{figure*}
  
  Каждая конструкция привязывается к~одной из систем координат, которые 
описаны в~весовой модели. В~весовой модели изделия для удобства описания 
различных агрегатов может быть описано несколько систем координат. Все 
описанные сис\-те\-мы координат организованы в~иерархическую структуру. 
Считается предописанной глобальная самолетная система координат, в~которой 
могут быть заданы или вычислены координаты всех объектов весовой модели. 
Каждая система координат в~весовой модели задается уникальным именем, 
положением начала координат относительно вышестоящей системы координат 
и~тремя углами поворота относительно вышестоящей. 

Параметр, 
обозначенный как СК,~--- это имя одной из сис\-тем координат весовой модели. 
Если СК не задано, то считается, что характеристики гео\-мет\-рии масс заданы 
в~глобальной системе координат. Каж\-дая сис\-те\-ма координат весовой модели 
содержит матрицу преобразования координат из самолетной (глобальной) 
системы координат в~данную и~обратно. Эта матрица для каждой системы 
координат есть произведение локальных матриц преобразований 
в~соответствии с~положением данной системы в~иерархии систем координат. 
Любое изменение параметров систем координат требует пе\-ре\-вы\-чис\-ле\-ния 
матриц преобразования как измененной сис\-те\-мы, так и~всех подчиненных. На 
рис.~2 показана панель параметрического пред\-став\-ле\-ния конструкций весовой 
модели.
  
  Так же как и~масса, центры тяжести и~моменты инерции вычисляются снизу
вверх от терминальных конструкций к~вышестоящим. При этом осуществляется 
пересчет характеристик по заданной иерархии систем координат от 
нижестоящих к~вышестоящим и~к~самолетной системе координат. Расчет 
МИХ терминальных конструкций 
осуществляется на основании гео\-мет\-ри\-че\-ских моделей. Геометрические модели 
на этапе рабочего проекта строятся в~системах гео\-мет\-ри\-че\-ско\-го 
моделирования. В~процессе их построения автоматически вычисляются 
объемы, массы, положение центра тяжести и~другие характеристики гео\-мет\-рии 
масс. Рассчитанная в~системах гео\-мет\-ри\-че\-ско\-го моделирования масса 
с~по\-мощью соответствующих интерфейсных средств может быть загружена как 
$M_{\mathrm{черт}}$ в~весовую модель. (Раньше документация была представлена в~виде 
чертежей на бумажных носителях и~$M_{\mathrm{черт}}$ вручную вычислялась по этим 
чертежам.) Однако на более ранних этапах проектирования, когда еще не 
проработана гео\-мет\-рия многих элементов конструкции, весовые расчеты 
проводятся на основании эскизов и~наборов гео\-мет\-ри\-че\-ских и~конструктивных 
параметров агрегатов изделия. Для этого в~весовой модели должны быть 
предусмотрены средства параметрического представления гео\-мет\-рии 
конструкций. Геометрическое пред\-став\-ле\-ние конструкций 
в~автоматизированной системе весового проектирования выполняет 
и~немаловажную функцию визуализации конструкций, их компоновки, 
размещения нагрузки и~т.\,д. В~АСВП реализовано несколько форм 
представления гео\-мет\-ри\-че\-ской информации, ориентированных не только на 
расчет МИХ, но и~на визуализацию выполняемых расчетов. Это чертежи 
геометрических проекций изделия, это таб\-лич\-ное задание типовых 
геометрических конструкций, это каркасное представление трехмерных 
геометрических моделей и, наконец, задание объемных конструкций 
триангуляционной (фасеточной) поверхностью. Последний вид представления 
является наиболее перспективным для точного вычисления МИХ. В~АСВП он 
применяется для расчета тарировочных характеристик топливных баков, о~чем 
будет сказано ниже.
  
  \textbf{Классификационные признаки конструкции.} В~весовой модели 
кроме числовых параметров опре\-делен ряд классификационных признаков 
конструкций, по которым проводится весовой анализ.\linebreak
 Таки\-ми маркерами могут 
быть подразделения, ответст\-вен\-ные за разработку конструкции, поставщики 
или изготовители готовых изделий, принадлежность конструкции 
к~определенным функциональным системам, конструкционные материалы 
и~пр.
  
  \textbf{Функциональные подсистемы изделия.} Конст\-рук\-тив\-ное членение 
самолета может не совпадать с~его функциональной структурой. Отдельные\linebreak 
элементы функциональных подсистем самолета удобнее описывать в~составе 
конструкции ка\-ко\-го-ни\-будь агрегата планера. Например, некоторая деталь 
может конструктивно входить в~состав крыла, а принадлежать 
к~функциональной подсистеме гидравлики или электрооборудования. Для того 
чтобы иметь возможность выполнять весовые расчеты, учитывая разные 
подходы к~классификации конструкции самолета, в~АСВП отдельно от дерева 
конструкции ведется реестр подсистем, для которых может быть проведен 
специальный расчет весовых параметров. В~этом реестре ведется полный 
перечень конструкций весовой модели, входящих в~подсистемы реестра, 
независимо от того, в~какой ветви дерева конструкции они находятся. Любая 
конструкция может быть включена только в~одну из подсистем реестра. 
В~зависимости от режима расчетов МИХ
конструкций, входящих в~под\-сис\-те\-му, будут учтены либо в~со\-ста\-ве 
вышестоящих агрегатов дерева конструкции, либо отдельно в~под\-сис\-теме. 
{\looseness=1

}
  
  \textbf{Распределенные характеристики изделия.} Задача вычисления 
распределенных характеристик изделия является родственной задачей 
вычисления характеристик геометрии масс. Основное отличие состоит в~том, 
что в~данной задаче рассчитываются не интегральные характеристики 
распределения материала, а сама функция распределения массы по объему 
конструкции. Такие функции рассчитываются по заданному геометрическому 
разбиению конструкции на пространственные отсеки. Функции распределения 
массы по объему конструкции в~процессе проектирования используются при 
построении динамически подобных моделей для проведения некоторых видов 
испытаний и~продувок, а~также для выполнения прочностных расчетов. 
  
  Каждый отсек разбиения для расчета распределенных характеристик 
представляет собой вы\-пук\-лый многогранник, ограниченный конечным набором 
плоскостей. Задача построения распределенных весовых характеристик состоит 
в~вычислении для каждого отсека массы и~положения центра тяжести той части 
конструкции самолета, которая геометрически расположена внутри этого 
отсека. Эта задача решается путем нахождения геометрического пересечения 
каждой терминальной конструкции с~каждым отсеком разбиения, и~если такое 
пересечение не пусто, то вычисление массы и~центра тяжести той части 
конструкции, которая попадает в~отсек. Некоторые конструкции могут быть 
объявлены сосредоточенными массами. Использование сосредоточенных масс 
позволяет исключить все подчиненные конструкции из распределения по 
отсекам и~рассматривать их отдельно для задания сосредоточенных нагрузок. 
Список сосредоточенных масс с~уникальными именами представляет собой 
отдельную структуру весовой модели. Каждая сосредоточенная масса содержит 
список ссылок на конструкции весовой модели. Любая конструкция может 
быть включена не более чем в~одну сосредоточенную массу.
  
  \textbf{Весовые сводки.} Одной из основных задач \mbox{АСВП} является 
построение так называемых весовых сводок. Весовые сводки являются 
документами, сопровождающими построение весовой модели самолета 
в~процессе его создания. В АСВП реализовано несколько форм весовых 
сводок, которые с~разных сторон отражают дерево конструкции самолета или 
отдельных ветвей этого дерева. Назначение этих сводок и~форма представления 
зависят от ре\-ша\-емых задач. Весовые данные в~сводках могут быть 
представлены либо в~табличном виде, либо в~виде иерархии конструкций. 
Могут содержать информацию в~детализированном или в~укрупненном виде. 
Отдельные виды весовых сводок пред\-став\-ля\-ют распределенные 
характеристики по отсекам. Весовые сводки предназначены для решения задач 
весового контроля и~весового анализа. 
  
  Весовой контроль при проектировании самолетов представляет собой  
ор\-га\-ни\-за\-ци\-он\-но-тех\-ническую сис\-те\-му, нацеленную на создание 
конструк\-ции минимального веса. Для эффективного \mbox{весового} контроля 
необходима оперативная информация о текущей массе изделия и~любой его 
части. Весовая информация для весового контроля в~АСВП представляется 
в~виде оперативных весовых сводок по отдельным подразделениям 
предприятия. В~таких весовых сводках содержится информация о текущей, 
теоретической, лимитной,\linebreak чертежной и~фактической массах конструкций, 
разрабатываемых данным подразделением. Могут также выпускаться 
оперативные сводки по группе подразделений или по всему проекту. Сводки 
весового контроля предназначены для использования руководителями проекта.
  
  Весовой анализ также связан с~выпуском определенного вида весовых 
сводок. Для решения задач весового анализа в~АСВП осуществляется 
сортировка и~выборки конструкций по определенному классификационному 
признаку. Например, могут быть рассчитаны массы силового и~несилового 
набора конструкции, массы продольного и~поперечного набора, массы 
конструкций определенного материала, массы готовых изделий или изделий 
конкретного поставщика и~т.\,д. Весовой контроль и~анализ позволяют 
выявить резервы конструкции, узкие места, тренды в~изменении веса 
кон\-ст\-рук\-ции.
{\looseness=1

}
  
  \subsection{Постоянные и~переменные структуры весовой модели 
самолета}
  
  Дерево конструкции весовой модели готового изделия не является 
статической структурой. Конфигурация самолета зависит от конкретных 
условий его применения. Мас\-со\-во-инер\-ци\-он\-ные характеристики при 
взлете и~посадке отличаются от тех же характеристик в~полете, когда убраны 
стойки шасси. Конфигурация меняется и~в~полете у~самолетов с~изменяемым 
углом стреловидности или с~измененяемым вектором тяги. Текущая 
конфигурация является одним из параметров весовой модели и~параметров 
весовых расчетов. По самому смыс\-лу создания самолета как транспортного 
средства предполагается, что кроме собственно конструкции, которая 
обеспечивает выполнение основных задач, на его  
МИХ существенным образом влияет 
перевозимая нагрузка. Перевозимая нагрузка есть переменная часть структуры 
дерева конструкции. Самолетные весовые классификаторы выделяют 
постоянную часть массы изделия и~переменную, состоящую из снаряжения, 
топлива и~целевой нагрузки:
  \begin{multline*}
{M} = M_{\mathrm{пустого}} + 
M_{\mathrm{снаряжения}} + {}\\
{}+M_{\mathrm{топлива}} + 
M_{\mathrm{целевой\_нагрузки}}\,.
  \end{multline*}
  
  Все переменные и~постоянные компоненты самолета составляют единое 
целое, и~расчет мас\-со\-во-инер\-ци\-он\-ных и~центровочных характеристик 
допусти\-мых конфигураций является одной из главных задач проектирования 
самолетов любого назначения. Переменные структуры в~весовой модели могут 
задаваться альтернативными конструкциями, когда у некоторой вершины 
дерева есть несколько вариантов поддеревьев и~когда любой из вариантов, но 
только один из них, может быть активирован в~конкретный момент времени. 
Существует своя специфика задания переменных структур весовой модели для 
разных содержательных задач. 
  
  \textbf{Пустой самолет}~--- это постоянная часть конструкции самолета, 
которая не меняется в~процессе эксплуатации готового изделия. Компонентами 
пустого самолета являются конструкция планера самолета, силовая установка 
и~ее системы, другие самолетные системы, обеспечивающие управление 
самолетом, а~также специальные системы бортового оборудования, 
предназначенные для решения целевых задач самолета. В~процессе 
проектирования и~при эксплуатации самолетов рассматриваются различные 
варианты отдельных конструкций планера, а~чаще~--- различные варианты 
по\-став\-ля\-емых готовых изделий. В~связи с~этим в~весовой модели АСВП 
рассматриваются возможные комбинации вариантов пустого самолета, 
вариантов снаряжения и~полезной нагрузки. 

\begin{figure*} %fig3
\vspace*{1pt}
 \begin{center}
 \mbox{%
 \epsfxsize=162mm 
 \epsfbox{fle-3.eps}
 }
 \end{center}
\vspace*{-9pt}
\Caption{Тарировочная таблица топливного бака}
\end{figure*}
  
  \textbf{Снаряжение самолета} устанавливается на борту в~процессе 
предполетной подготовки. Снаряжение самолета принято разделять на 
основное и~дополнительное. Основное снаряжение включает несколько 
позиций. Это экипаж и~системы жизнеобеспечения экипажа, системы 
жизнеобеспечения пассажиров, заправляемые компоненты и~расходуемые 
материалы, несливаемый остаток топлива и~другие возможные компоненты. 
Использование различных вариантов экипажа и~другого снаряжения самолета 
связано с~различным характером выполняемых задач. Как правило, существует 
несколько типовых вариантов комплектации экипажа 
и~элементов снаряжения. Весовая модель должна содержать перечень 
альтернативных вариантов снаряжения и~их характеристик. Естественно, что 
этот перечень может модифицироваться. К~дополнительному снаряжению 
относят временное оборудование и~средства, связанные с~установкой на борту 
и~закреплением на подвесках перевозимых грузов. Временно устанавливаемое 
оборудование, как правило, связано со спецификой полетных заданий. Это 
может быть специальная измерительная аппаратура или оборудование, которое 
необходимо проверить в~условиях реального полета. Перечень такого 
оборудования и~его характеристики в~весовой модели должны быть 
пред\-став\-ле\-ны в~специальном реестре, или в~каталоге. Для установки 
оборудования, размещения любой коммерческой нагрузки и~вооружения в~конструкции самолета
должны быть  предусмотрены специальные места 
размещения и~узлы крепления. Точки размещения оборудования и~любых 
элементов целевой нагрузки задаются своими координатами и~установочными 
углами закрепления. 

\begin{figure*} %fig4
  \vspace*{1pt}
 \begin{center}
 \mbox{%
 \epsfxsize=162mm 
 \epsfbox{fle-4.eps}
 }
 \end{center}
\vspace*{-11pt}
\Caption{Варианты размещения целевой нагрузки самолета на подвесках}
\end{figure*}
  
  \textbf{Топливо}~--- величина переменная и~на земле, при подготовке 
самолета к~вылету, и~в~воздухе, при выработке топлива, и, если это 
предусмотрено, при дозаправке в~воздухе. Одной из самых сложных и~важных 
задач построения весовой модели является отражение изменяющихся в~полете  
МИХ топлива, находящегося 
в~топливных баках. Топливные баки современных ЛА
могут иметь довольно сложные геометрические формы. В~процессе выработки 
топлива все характеристики располагаемого запаса топлива меняются. 
Необходимо отслеживать эти изменения в~процессе произвольных допустимых 
эволюций траектории полета. Функции изменения МИХ в~зависимости от 
объема оставшегося топлива задаются тарировочными характеристиками баков. 
Для расчета тарировочных характеристик топливных баков при произвольных 
углах атаки, углах тангажа и~крена в~весовой модели наиболее удобно 
триангуляционное задание баков. В~тарировочной таблице вычисляется масса 
оставшегося топлива в~зависимости от уровня поверхности жидкости 
в~топливном баке. На рис.~3 приведен пример расчета тарировочной таблицы 
крыльевого топливного бака.



  Если МИХ топлива в~конкретном баке по 
мере его выработки определяются тарировочной характеристикой, то 
зависимость МИХ оставшегося топлива определяется последовательностью, 
в~которой осуществляется выработка из разных баков. Топливная система 
самолета состоит из нескольких баков~--- как внутренних, так и~размещенных 
на подвесках, а~также из системы трубопроводов, перекачивающих насосов и~управляющей автоматики. Основой управления расходом топлива является 
программа, определяющая порядок расходования топлива из разных баков. 
Переключение перекачки топлива между разными баками осуществляется для 
обеспечения центровки самолета в~заданных границах. Одним из критериев при 
разработке алгоритмов перекачки является число переключений и~обеспечение 
бесперебойной подачи топлива при любых допустимых параметрах траектории 
полета. Массово-инерционные характеристики топлива в~процессе тарировки 
баков задаются их разбиением плоскопараллельными сечениями на тонкие 
слои. Для каждого слоя указывается масса, координаты центра тяжести 
и~плоскостные моменты инерции. Программа выработки топлива пред\-став\-ля\-ет 
собой последовательность выработки слоев из разных баков в~соответствии 
с~диаграммой переключений. В~весовой модели может быть задано несколько 
вариантов программ расходования топлива. Разумеется, в~процессе выполнения 
полетного задания программа расходования топлива фиксирована. 
Предварительный расчет характеристик для разных вариантов порядка 
выработки топлива необходим для выбора наилучшего, удовле\-тво\-ря\-юще\-го 
всем ограничениям.
  
  \textbf{Целевая нагрузка} зависит от назначения самолета и~от конкретного 
полетного задания. Для пасса\-жирских самолетов целевая нагрузка~--- это 
пассажи\-ры с~багажом, для транспортных са\-мо\-летов~--- это коммерческие 
грузы, для военных~--- подвесное или размещаемое в~специальных \mbox{отсеках} 
вооружение. В~полете возможен сброс и~десантирование целевой нагрузки. 
Комплектация и~установка целевой нагрузки представляет собой довольно 
сложный процесс. Выбор состава грузов и~их размещение могут проходить 
в~несколько этапов. Сложность выбора обусловлена большим количеством 
типов перевозимой нагрузки, наличием большого числа вспомогательных 
специальных устройств закрепления грузов как во внутренних отсеках 
самолета, так и~на внешних подвесках. На рис.~4 приведена панель 
формирования различных расчетных вариантов целевой нагрузки самолета. 
Визуализация этого процесса существенно облегчает решение различных задач 
анализа допустимой нагрузки как на этапе проектирования самолета, так и~при 
эксплуатации во время подготовки полетных заданий.
  
  \begin{figure*} %fig5
\vspace*{1pt}
 \begin{center}
 \mbox{%
 \epsfxsize=162mm 
 \epsfbox{fle-5.eps}
 }
 \end{center}
\vspace*{-9pt}
\Caption{Область допустимых центровок}
\end{figure*}

  Для удобства выбора и~проведения расчетов множества вариантов загрузки 
самолета в~рамках весовой модели реализованы каталоги нагрузки~--- 
специального оборудования, коммерческой нагрузки, вооружения. В~этих 
каталогах ведутся клас\-си\-фи\-ка\-то\-ры, позволяющие в~громадных переч\-нях 
находить нужные позиции и~их характеристики. Кроме  
МИХ размещаемой нагрузки в~каталогах 
даются ссылки на их геометрические модели, задаются габариты, другие 
геометрические па\-ра\-мет\-ры. Эти данные нужны для визуализации размещения 
и~компоновки, для вычисления их МИХ. 
Как правило, существуют довольно жесткие ограничения на 
размещение нагрузки на борту, а~также на внешних узлах крепления. Эти 
ограничения должны указываться в~каталоге и~учитываться в~процессе 
формирования вариантов загрузки самолета. 
  
  Ограничения, которые проверяются при анализе различных вариантов 
снаряжения самолета, программы выработки топлива и~допустимых вариантов 
целевой нагрузки, задают область допустимых центровок самолета. 
  
  \textbf{Область допустимых центровок} является неотъемлемой частью 
весовой модели и~служит одной из основных весовых характеристик самолета, 
особенно важной и~контролируемой в~процессе его эксплуатации. На рис.~5 
проиллюстрированы ограничения, образующие область допустимых центровок, 
и~приведен график изменения центровки самолета при выработке топлива. 



  По оси абсцисс на этом графике откладывается центровка самолета, которая 
определяется как положение центра тяжести самолета на средней 
аэро\-ди\-на\-ми\-че\-ской хорде в~процентах от ее длины. По оси ординат 
откладывается текущая масса самолета с~учетом массы снаряжения, массы 
целевой нагрузки и~текущего запаса топлива. Точки излома на графиках 
центровки соответствуют моментам переключения подачи топлива с~одного 
бака на\linebreak другой, которые определяются программой выработки топлива или 
моментами сброса целевой нагрузки. Двойной график изменения центровки 
соответствует двум полетным конфигурациям~--- с~убранными 
и~выпущенными стойками шасси. Ограничения, которые обеспечивают 
устой\-чи\-вость и~управ\-ля\-емость полета, задаются предельными значениями 
центровки. Предельно передняя и~предельно задняя центровки на графике 
показаны вертикаль\-ными штриховыми линиями. Горизонтальные линии задают 
ограничения на взлетную и~посадочную массы. Ограничения максимальной 
взлетной и~посадочной массы при определенных условиях могут нарушаться, 
но эти нарушения допускаются в~исключительных условиях и~сказываются на 
ресурсных характеристиках самолета.\linebreak Превышение \textbf{предельных} 
значений взлетной и~посадочной массы не допускается. Наклонные штриховые 
линии на графике задают ограничения, связанные с~максимально допустимыми 
нагрузками на переднюю и~главную опоры шасси.  

\begin{figure*} %fig6
\vspace*{1pt}
 \begin{center}
 \mbox{%
 \epsfxsize=165mm 
 \epsfbox{fle-6.eps}
 }
 \end{center}
\vspace*{-9pt}
\Caption{Архитектура программной реализации исполнительных модулей АСВП}
\end{figure*} 

%\vspace*{-12pt}

\section{Программная реализация автоматизированной системы весового
проектирования}

  Представленная здесь струк\-тур\-но-па\-ра\-мет\-ри\-че\-ская весовая модель 
самолета позволяет решать широкий круг задач весового проектирования. 
Весовая модель составляет информационную основу,\linebreak на базе которой могут 
быть построены различные вычислительные программы и~пользовательские 
модули. Рассматриваемая в~данной работе АСВП построена по 
кли\-ент-сер\-вер\-ной архитектуре, где весовая модель служит единым хранилищем 
информации, базой данных системы. Непосредственно с~информацией, 
хранящейся в~этой базе данных, взаимодействуют различные вычислительные, 
расчетные программы~--- серверы, которые кроме расчетных функций 
обеспечивают информационную связь клиентских модулей с~весовой моделью 
самолета. Непосредственными пользователями клиентских модулей являются 
конструкторы и~проектировщики, решающие различные задачи весового 
проектирования.  Построена АСВП как многопользовательская интерактивная 
система. На рис.~6 представлена архитектура АСВП, ее основные программные 
и~информационные компоненты.




  Ниже перечислены основные функции программных модулей АСВП:
 \begin{description}
 \item[\,] 
Сервер ПУСТОЙ ЛА\;+\;Модуль расчета МИХ пус\-то\-го самолета:
\begin{itemize}
\item создание и~модификация дерева конструкции пустого самолета;
\item расчет МИХ пустого изделия, всех его сис\-тем, узлов, агрегатов и~деталей 
на любых уровнях дерева конструкции;
\item весовой анализ и~контроль текущего состояния проекта, выполнения 
лимитных ограничений по весу, осуществление выборок весовой информации 
по различным признакам~--- сис\-те\-мам, агрегатам, типу конструкции 
(си\-ло\-вая/не\-си\-ло\-вая),  материалу конструкции, подразделениям и~т.\,д.;
\item расчет распределения массы самолета по различным разбиениям на 
отсеки; эта информация используется для построения динамически подобных 
моделей и~при прочностных расчетах;
\item расчет МИХ при различных вариантах полетной конфигурации при 
убранных и~выпущенных стойках шасси, при отклонениях консолей крыла для 
самолетов с~из\-ме\-ня\-емой геометрией, при отклонении органов управления.
\end{itemize}
\begin{figure*} %fig7
\vspace*{1pt}
 \begin{center}
 \mbox{%
 \epsfxsize=155.86mm 
 \epsfbox{fle-7.eps}
 }
 \end{center}
\vspace*{-1pt}
\Caption{Проектный подход~--- технология ГП}
\vspace*{6pt}
\end{figure*}
 \item[\,]
Сервер НАГРУЗКА ЛА\;+\;Модуль расчета МИХ самолета с~переменной 
массой:
\begin{itemize}
\item создание и~модификация реестра допустимых вариантов нагрузки 
самолета;
\item расчеты МИХ снаряженного и~загруженного самолета для разных 
вариантов компоновки и~размещения на борту полезной нагрузки;
\item расчет изменения МИХ самолета в~полете при выработке топлива, 
дозаправке в~воздухе, сбросе нагрузки;
\item расчет МИХ самолета в~виде табличных зависимостей для различных 
вариантов снаряжения и~размещения нагрузки;
\item расчет МИХ самолета в~виде графических зависимостей от массы 
самолета и/или от массы топлива;
\item проверка выполнения установленных эксплуатационных ограничений по 
центровке, взлетной и~посадочной массе, нагрузке на опоры шасси для 
различных вариантов снаряжения и~размещения нагрузки; сигнализация 
в~случае нарушения ограничений, а~также для различных вариантов программ 
выработки топлива.
\end{itemize}

\pagebreak

 \item[\,]
Сервер КАТАЛОГ\;+\;Модуль ведения каталога элементов нагрузки:\\[-9pt]
\begin{itemize}
\item создание и~модификация каталога элементов целевой нагрузки самолета;\\[-9pt]
\item создание и~модификация базы данных вариантов размещения 
и~закрепления элементов нагрузки каталога на борту самолета или на подвесках;\\[-9pt]
\item создание и~модификация базы данных вспомогательных элементов 
конструкции установки элементов нагрузки.\\[-9pt]
\end{itemize}
 \item[\,]
Сервер ТОПЛИВО\;+\;Модуль расчета порядка выработки топлива:\\[-9pt]
\begin{itemize}
\item создание и~модификация базы данных различных вариантов программы 
выработки топлива;\\[-9pt]
\item расчет МИХ и~МЦД для различных вариантов переключения выработки 
топлива из внутренних, закладных и~подвесных баков;\\[-9pt]
\item расчет МИХ и~МЦД при различных программах заливки и~дозаправки 
топлива во внутренние, закладные и~подвесные баки.\\[-9pt]
\end{itemize}
 \item[\,]
Сервер БАКИ\;+\;Модуль расчета тарировки топливных баков:\\[-9pt]
\begin{itemize}
\item создание и~модификация базы данных гео\-мет\-рии топливных баков;\\[-9pt]
\item расчет тарировочных характеристик топливных баков при различных 
углах тангажа и~крена.\\[-9pt]
  \end{itemize}
  \end{description}
  
  Программная реализация АСВП велась с~использованием инструментального комплекса 
<<Генератор проектов>> (технология ГП)~\cite{5-fl}. Технология ГП 
обеспечивает возможность разработки приклад\-ных систем многоуровневой  
кли\-ент-сер\-вер\-ной архитектуры с~использованием реляционных и~сетевых 
баз данных со сложным пользовательским и~межпрограммным интерфейсом. 
Создание ин\-фор\-ма\-ци\-он\-но-вы\-чис\-ли\-тель\-ных сис\-тем в~рамках 
технологии ГП базируется на так называемом <<проектном подходе>>. Под 
проектом здесь понимается пакет документов (файлов), содержащий описание 
структуры проекта, описание логической структуры баз данных, спецификации 
пользовательского интерфейса, перечень команд и~сценарии работы 
пользователей, описание функций и~процедур обработки пользовательских 
запросов. Исходное описание проекта подается на вход <<Генератора 
проекта>>, который строит в~памяти модель проекта, осуществляет ее анализ 
на предмет корректности и~целостности, а затем на основании этой модели 
генерирует тексты программного кода для клиентских и~серверных компонент 
системы, а~так\-же ге\-нерирует утилиты, необходимые для сборки, инсталляции 
и~сопровождения системы. 

На рис.~7 показана общая архитектура 
программной конструкции, связанной с~применением технологии ГП.
  


  В приведенной цепочке разработчик прикладной информационной системы 
имеет дело только с~первым ее звеном~--- проектом системы. При этом он 
избавлен от необходимости иметь дело с~системным программным окружением 
вычислительной среды, в~которой должна функционировать разрабатываемая 
прикладная система. Все связи прикладных информационных процессов 
с~конкретной системной вычислительной средой привносит 
в~результирующую рабочую программу <<Генератор проектов>> на стадии 
анализа и~генерации итогового программного кода. Естественно, что при этом 
объем описания проекта оказывается существенно короче программного кода, 
который создается автоматически. Экономия трудозатрат разработчика 
оказывается существенной. В~частности, объем описания проекта АСВП на 
порядок меньше, чем объем сгенерированного программного кода. Даже если 
предположить, что написанный вручную программный код благодаря 
искусству программистов будет весьма экономным, то все равно трудоемкость 
разработки прикладных систем будет в~разы меньше. 

Но главное даже не 
в~числе строк программ, а~прежде всего в~экономии интеллектуальных затрат 
разработчиков прикладных систем и,~в~итоге, автоматически созданные 
программы более надежны и~свободны от нечаянных ошибок и~опечаток.\linebreak 
И~кроме того, разрабатываемые в~рамках технологии ГП прикладные системы 
обеспечивают-\linebreak\vspace*{-12pt}

\pagebreak

\noindent
ся эффективными средствами сопровождения, т.\,е.\linebreak достаточно 
простой процедурой внесения ис\-прав\-ле\-ний и~развития программ в~процессе их 
эксплу\-а\-тации. 

Прикладные программные комплексы в~рамках технологии ГП 
разрабатываются как автономные системы и~не требуют для своей работы 
специальной среды и~дорогостоящих программных продуктов (кроме 
использующихся систем управления базами данных
(СУБД) и~общесистемного обеспечения). Разрабатываемые 
в~рамках технологии ГП прикладные системы допускают масштабирование 
и~портирование на различные вычислительные платформы и~СУБД.
  
  \bigskip
  
  Как уже говорилось, система АСВП разрабатывалась в~течение ряда лет, 
многие ее компоненты и~версии были апробированы и~использовались 
в~реальном проектировании. 
%
Авторы выражают благодарность 
С.\,И.~Скобелеву, М.\,К.~Курьянскому, Д.\,Ю.~Стрель\-цу, П.\,В.~Плунскому 
и~К.\,Н.~Ерасову за плодотворные обсуждения проблем весового проектирования 
самолетов, за постановку многих задач и~за апробацию разработанных 
программ.

%\vspace*{-12pt}

{\small\frenchspacing
 {%\baselineskip=10.8pt
 \addcontentsline{toc}{section}{References}
 \begin{thebibliography}{9}
\bibitem{1-fl}
\Au{Шейнин В.\,М., Козловский~В.\,И.} Весовое проектирование и~эффективность 
пассажирских самолетов.~--- М.: Машиностроение, 1977.   Т.~1. 343~с.

%\columnbreak

\bibitem{2-fl}
\Au{Скобелев С.\,И., Широков~Н.\,И.} Весовой анализ и~контроль в~САПР ЛА~// Задачи 
и~методы автоматизированного проектирования.~--- М.: ВЦ РАН, 1991. С.~92--100.
\bibitem{3-fl}
\Au{Широков Н.\,И.} Автоматизированная система весовых расчетов в~САПР ЛА~// 
Автоматизация проектирования инженерных и~финансовых информационных систем 
средствами Генератора проектов~/ Отв. ред. Ю.\,А.~Флеров.~--- М.: ВЦ РАН, 
2010. С.~55--66.

\vspace*{6pt}

\bibitem{4-fl}
\Au{Вышинский Л.\,Л., Широков~Н.\,И.} Система автоматизации расчетов 
массово-инерционных характеристик ЛА с~переменной массой~// Развитие и~применение 
инструментального комплекса Генератор проектов~/ Отв. ред. Ю.\,А.~Флеров.~--- 
М.: ВЦ РАН, 2014. С.~20--31.
{\looseness=1

}

\vspace*{6pt}

\bibitem{5-fl}
\Au{Вышинский Л.\,Л., Гринев~И.\,Л., Флеров~Ю.\,А., Широков~А.\,Н., Широков~Н.\,И.} 
Генератор проектов~--- инструментальный комплекс для разработки  
<<кли\-ент-сер\-вер\-ных>> сис\-тем~// Информационные технологии и~вычислительные 
системы, 2003. №\,1-2. С.~6--25.
 \end{thebibliography}

 }
 }

\end{multicols}

\vspace*{-6pt}

\hfill{\small\textit{Поступила в~редакцию 24.05.17}}

\vspace*{8pt}

%\newpage

%\vspace*{-24pt}

\hrule

\vspace*{2pt}

\hrule

%\vspace*{8pt}


\def\tit{COMPUTER-AIDED SYSTEM OF~AIRCRAFT WEIGHT DESIGN}

\def\titkol{Computer-aided system of~aircraft weight design}

\def\aut{L.\,L.~Vyshinsky, Yu.\,A.~Flerov, and~N.\,I.~Shirokov}

\def\autkol{L.\,L.~Vyshinsky, Yu.\,A.~Flerov, and~N.\,I.~Shirokov}

\titel{\tit}{\aut}{\autkol}{\titkol}

\vspace*{-9pt}


\noindent
A.\,A.~Dorodnicyn Computing Centre, Federal Research Center ``Computer Science and 
Control'' of the Russian Academy of Sciences,  40~Vavilov Str., Moscow 119333, Russian 
Federation 



\def\leftfootline{\small{\textbf{\thepage}
\hfill INFORMATIKA I EE PRIMENENIYA~--- INFORMATICS AND
APPLICATIONS\ \ \ 2018\ \ \ volume~12\ \ \ issue\ 1}
}%
 \def\rightfootline{\small{INFORMATIKA I EE PRIMENENIYA~---
INFORMATICS AND APPLICATIONS\ \ \ 2018\ \ \ volume~12\ \ \ issue\ 1
\hfill \textbf{\thepage}}}

\vspace*{3pt}
   

\Abste{The article is devoted to the problems of computer-aided weight design of 
aircraft. Weight and mass-inertial parameters are one of the basic values that affect 
the performance characteristics of aircraft. The informational basis of the system is 
the weight model of the aircraft. The paper describes the structure of the weight 
model and its individual components. The program implementation of the system, 
which is executed within the framework of the client-server architecture, is shown. 
The automated system of weight design is implemented using the software tool 
complex ``Project Generator'' (GP technology), which was developed at the 
Computing Centre of the Russian Academy of Sciences. The creation of information 
and computing systems within the framework of the GP technology is based on the 
so-called ``project approach,'' when the formal description of the system 
automatically generates code for the client and server components of the system.}

\KWE{math modeling; design automation; aircraft; weight design; weighting model; 
design tree; project generator; code generation; client-server architecture}

  \DOI{10.14357/19922264180103} 

%\vspace*{-12pt}

%\Ack
%\noindent




%\vspace*{8pt}

  \begin{multicols}{2}

\renewcommand{\bibname}{\protect\rmfamily References}
%\renewcommand{\bibname}{\large\protect\rm References}

{\small\frenchspacing
 {%\baselineskip=10.8pt
 \addcontentsline{toc}{section}{References}
 \begin{thebibliography}{9} 
 
 %\vspace*{-6pt}
 
 \bibitem{1-fl-1}
\Au{Sheynin, V.\,M., and V.\,I.~Kozlovskiy}. 1977. \textit{Vesovoe 
proektirovanie i~effektivnost' passazhirskikh samoletov} [Weight design and 
efficiency of passenger aircraft]. Moscow: Mechanical Engineering. Vol.~1. 343~p.
\bibitem{2-fl-1}
\Aue{Skobelev, S.\,I., and N.\,I.~Shirokov.} 1991. Vesovoy analiz i~kontrol' v~SAPR 
LA [Weight analysis and control in CAD of aircraft]. \textit{Zadachi i~metody 
avtomatizirovannogo proektirovaniya} [Tasks and methods of computer-aided 
design]. Moscow: Computing Centre of the USSR Academy of Sciences.  
92--100.
\bibitem{3-fl-1}
\Aue{Shirokov, N.\,I.} 2010. Avtomatizirovannaya sistema vesovykh raschetov 
v~SAPR LA [Automated system weight calculations in CAD].  
\textit{Avtomatizatsiya proektirovaniya inzhenernykh i~finansovykh 
informatsionnykh system sredsvami Generatora proektov} [Computer 
aided  design of engineering and financial information systems by the means of the 
Project Generator]. Moscow: Computing Centre of RAS. 
55--66.
\bibitem{4-fl-1}
\Aue{Vyshinskiy, L.\,L., and N.\,I.~Shirokov.} 2014. Sistema avtomatizatsii 
raschetov massovo-inertsionnykh kharakteristik LA s~peremennoy massoy [CAD 
system of calculation  aircraft mass-inertial characteristics with variable mass].  
\textit{Razvitie i~primenenie instrumental'nogo kompleksa Generator proektov} 
[The development and application of a tool set Project Generator]. 
Moscow: Computing Centre of RAS. 20--31.
{\looseness=1

}

\bibitem{5-fl-1}
\Aue{Vyshinskiy, L.\,L., I.\,L.~Grinev, Yu.\,A.~Flerov, A.\,N.~Shirokov, and 
N.\,I.~Shirokov.} 2003. Generator proektov~--- instrumental'nyy kompleks dlya 
razrabotki ``klient--servernykh'' sistem [The project generator~--- tool complex for 
development of ``client--server'' systems]. 
\textit{Informatsionnye tekhnologii i~vychislitel'nye sistemy} [Information 
Technologies and Computer Systems] 1-2:6--25.

\end{thebibliography}

 }
 }

\end{multicols}

\vspace*{-6pt}

\hfill{\small\textit{Received May 24, 2017}}

%\vspace*{-10pt}

\Contr

\noindent
\textbf{Vyshinsky Leonid L.} (b.\ 1941)~--- Candidate of Sciences (PhD) in physics and 
mathematics, Head of Laboratory, A.\,A.~Dorodnicyn Computing 
Centre, Federal Research Center ``Computer Science and Control'' of the Russian 
Academy of Sciences, 40~Vavilov Str., Moscow 119333, Russian Federation; 
\mbox{Wysh@ccas.ru} 

\vspace*{3pt}

\noindent
\textbf{Flerov Yuri A.} (b.\ 1942)~--- Corresponding Member of the Russian 
Academy of Science, Doctor of Science in physics and mathematics, professor, 
Deputy Director, A.\,A.~Dorodnicyn Computing Centre, Federal Research Center 
``Computer Science and Control'' of the Russian Academy of Sciences, 40~Vavilov 
Str., Moscow 119333, Russian Federation; \mbox{fler@ccas.ru}

\vspace*{3pt}

\noindent
\textbf{Shirokov Nikolai I.} (b.\ 1963)~--- Candidate of Sciences (PhD) in physics and 
mathematics, senior scientist, A.\,A.~Dorodnicyn Computing Centre, Federal 
Research Center ``Computer Science and Control'' of the Russian Academy of 
Sciences, 40~Vavilov Str., Moscow 119333, Russian Federation; 
\mbox{Wysh@ccas.ru} 



\label{end\stat}


\renewcommand{\bibname}{\protect\rm Литература}  %13
\def\stat{zatsman}

\def\tit{ТРАНСФОРМАЦИИ ОБЪЕКТОВ ПЕРВОГО И~ВТОРОГО ПОРЯДКА 
В~ЛЕКСИКОГРАФИЧЕСКОЙ ИНФОРМАЦИОННОЙ СИСТЕМЕ$^*$}

\def\titkol{Трансформации объектов первого и~второго порядка 
в~лексикографической информационной системе}

\def\aut{И.\,М.~Зацман$^1$}

\def\autkol{И.\,М.~Зацман}

\titel{\tit}{\aut}{\autkol}{\titkol}

\index{Зацман И.\,М.}
\index{Zatsman I.\,M.}


{\renewcommand{\thefootnote}{\fnsymbol{footnote}} \footnotetext[1]
{Исследование выполнено в~ФИЦ ИУ РАН за счет гранта Российского научного фонда №\,24-18-00155, {\sf 
https://rscf.ru/project/24-18-00155}. Работа выполнялась с~использованием инфраструктуры Центра 
коллективного пользования <<Высокопроизводительные вычисления и~большие данные>> (ЦКП 
<<Информатика>>) ФИЦ ИУ РАН (г.\ Москва).}}


\renewcommand{\thefootnote}{\arabic{footnote}}
\footnotetext[1]{ Федеральный исследовательский центр <<Информатика и~управление>> Российской академии наук, 
\mbox{izatsman@yandex.ru}}

\vspace*{-12pt}


  
  \Abst{Рассматриваются теоретические основания проектирования информационных 
технологий (ИТ) интеграции двуязычных словарей и~параллельных корпусов. Дано описание 
первых результатов создания третьего уровня классификации трансформаций объектов 
предметной области информатики, которую предполагается использовать при создании 
концепции лексикографической информационной системы, обеспечивающей интеграцию. 
Все сущности информатики в~статье разделены на два глобальных класса: объекты и~их 
трансформации. Для каждого такого класса конструируется своя классификация. Ранее были 
описаны два верхних уровня классификации трансформаций объектов предметной области. 
В~данной статье рассматривается третий уровень этой классификации. Основанием для 
построения самого верхнего ее уровня служило деление предметной области информатики 
на среды (ментальная, сенсорно воспринимаемая, цифровая и~ряд других сред), каждая из 
которых по определению включает объекты одной природы. Основанием для построения 
второго уровня классификации трансформаций объектов служила типология знаковых  
сис\-тем А.~Соломоника. Цель статьи состоит в~систематизации трансформаций первого 
и~второго порядка объектов предметной области на третьем уровне этой классификации. 
Основанием для систематизации служит средовая версия иерархии Акоффа.}
  
  \KW{объекты предметной области; трансформации объектов; классификация; данные; 
информация; знание; лексикографическая информационная сис\-тема}

\DOI{10.14357/19922264240211}{VZTGVV}
  
\vspace*{3pt}


\vskip 10pt plus 9pt minus 6pt

\thispagestyle{headings}

\begin{multicols}{2}

\label{st\stat}
  
\section{Введение}

\vspace*{-9pt}

  Возникновение параллельных корпусов, в~которых предложениям 
оригинального текста со\-по\-став\-ле\-ны предложения его перевода, обеспечило 
возможность контрастивного лингвистического\linebreak \mbox{анализа} на принципиально 
новом уровне полноты и~точности, недостижимом в~докорпусную эпоху. 
Пионерскими в~этой области стали работы \mbox{1990-х~гг}. Стига Йоханссона  
с~анг\-ло-нор\-веж\-ским корпусом~[1]. В России параллельные корпусы стали 
формироваться в~начале XXI~века в~рамках Национального корпуса русского 
языка~[2].
  
  Создатели двуязычных словарей используют параллельные корпусы для 
сбора материала и~эмпирической проверки своих гипотез, касающихся 
межъязы\-ко\-вой эквивалентности. Ценность параллельных корпусов 
определяется тем, что в~лингвистике этап сбора исходного материала считается 
наиболее трудоемким и~наименее творческим, а~параллельные корпусы 
позволяют значительно сэкономить время и~силы для творческого этапа 
создания словарей~[3].
 % 
  При этом двуязычные словари, создаваемые на основе исходного материала, 
извлеченного из параллельных корпусов, сейчас формируются без связей с~их 
текстами. Другими словами, онлайновые связи созданных словарей 
с~параллельными корпусами, которые служили источниками исходного 
материала, отсутствуют. 

Параллельные корпусы постоянно пополняются 
новыми текстами, в~предложениях которых можно обнаружить новые значения 
слов и~устойчивых словосочетаний. Однако при этом отсутствуют методы 
и~средства оперативного обновления словарей по корпусным данным. 
В~настоящее время проблема установления связей между двуязычными 
словарями и~параллельными корпусами (далее~--- проблема интеграции) 
находится на стадии поиска концептуальных подходов к~их интеграции на 
уровне значений.
  
  Подход к~решению проблемы интеграции, предлагаемый в~статье, учитывает 
  и~появление новых значений слов и~устойчивых словосочетаний, и~динамику 
смысловых значений, которая обусловлена развитием и~пополнением знания 
лингвистов, фиксирующих эти значения в~результате семантического анализа 
пополняемых корпусных данных. Проведенные эксперименты показали, что 
обнаружение нового лингвистического знания обусловливает и~формирование 
дефиниций новых значений, и~пересмотр уже существующих дефиниций~[4, 5].
  
  Например, в~проведенных экспериментах с~использованием ЦКП 
<<Информатика>> ФИЦ ИУ РАН фиксировалась эволюция значений немецких 
модальных глаголов, исходное состояние значений которых было описано 
в~не\-мец\-ко-рус\-ском словаре. В~экспериментальном массиве текстов как 
потенциальных источниках нового знания 16\,268 предложений содержали 
немецкие модальные глаголы и~в~2041 из них встречался глагол sollen. 
В~начале эксперимента в~словаре были описаны~12~значений этого модального 
глагола. По окончании эксперимента лингвисты обнаружили два новых его 
значения, согласовали их дефиниции и~описали эволюцию дефиниций~[6, 7].
  
  Таким образом, для решения проблемы интеграции требуется фиксировать 
новое знание, обнаруженное лингвистами в~текстовых данных параллельных 
корпусов, отслеживать эволюцию знания, представленного в~виде дефиниций 
значений слов и~устойчивых словосочетаний, и,~соответственно, 
актуализировать электронные двуязычные словари. Предлагаемый 
концептуальный подход к~интеграции, который планируется реализовать 
в~процессе проектирования лексикографической информационной сис\-те\-мы, 
фиксирующей эволюцию лингвистического знания, основан на решении 
следующих задач:\\[-14pt]
  \begin{itemize}
  \item категоризация трех базовых понятий информатики, включенных 
  в~иерархию Акоффа~[8] (данные, информация, знание), на объекты 
проектируемой сис\-те\-мы, которая необходима, чтобы фиксировать 
<<кванты>> нового знания и~отслеживать его эволюцию в~этой сис\-теме;\\[-15pt]
  \item  систематизация трансформаций объектов этой сис\-темы.\\[-14pt]
  \end{itemize}
  
  Цель статьи и~состоит в~решении двух задач: категоризации трех базовых 
понятий информатики на объекты лексикографической информационной  
сис\-те\-мы и~сис\-те\-ма\-ти\-за\-ции трансформаций первого и~второго порядка 
ее объектов.
  
  Трансформациями первого порядка, о которых сказано в~формулировке цели 
статьи, называются взаимные преобразования между двумя объектами  
сис\-те\-мы одной природы. Например, перевод в~сис\-те\-ме текста с~русского 
языка на английский относится к~ним. Трансформациями второго порядка 
и~выше называются взаимные преобразования между двумя и~более объектами 
разной природы. Например, кодирование символов текс\-та компьютерными 
кодами и~их декодирование относятся по определению к~трансформациям 
второго порядка.

%\vspace*{-9pt}
  
\section{Процессы трансформаций в~информатике}

%\vspace*{-3pt}

Процессы трансформаций, рассматриваемые в~статье, относятся к~теоретическому ядру информатики, а~не 
только к~проектированию лексикографической информационной сис\-те\-мы. Например, из трех основных 
подходов к~описанию предметной об\-ласти информатики\footnote{В статье предметная область информатики 
трактуется согласно концепции полиадического компьютинга Пола Розенблума~\cite{9-zac}.} (объектный, 
трансформационный и~синтетический) сис\-те\-ма\-ти\-за\-ция трансформаций ближе всего ко второму 
подходу. Примерами первого подхода, в~рамках которого основное внимание уделяется объектам предметной 
области информатики и~в~меньшей степени отношениям\linebreak между ними, могут служить  
работы~\cite{8-zac, 10-zac, 11-zac}; \mbox{примерами} второго подхода, в~рамках которого основное внимание 
уделяется трансформациям и~в~меньшей степени трансформируемым объектам,~---  
работы~\cite{12-zac, 13-zac}; примерами третьего, синтетического подхода, в~котором уделяется внимание 
и~объектам предметной об\-ласти информатики, и~отношениям между ними, могут служить работы~\cite{14-zac, 
15-zac, 16-zac, 17-zac, 18-zac}.

  Таким образом, для описания трансформаций объектов лексикографической 
информационной\linebreak системы предпочтительнее всего трансформационный 
подход, который упоминается и~в определениях информатики. Например, 
в~2009~г.\ П.~Деннинг и~П.~Розенблум сформулировали суть \mbox{информатики} как 
компьютинга следующим образом: <<$\ldots$информатика~--- это не просто 
алгоритмы и~структуры данных; это преобразования [трансформации] 
представлений>>~\cite{12-zac}. Чуть позже, в~контексте краткого описания 
парадигмы информатики как компьютинга, П.~Деннинг и~П.~Фриман изменили 
эту формулировку на такую: <<Центральный объект внимания в~информатике 
можно определить как информационные процессы~--- \textit{естественные или 
искусственные процессы, преобразующие информацию} (курсив мой~--- 
И.\,З.)>>~\cite{13-zac}. Согласно парадигме, предлагаемой авторами этой 
статьи, на начальном этапе проектирования автоматизированных систем 
базовыми элементами моделей их функционирования служат 
\textit{информационные про\-цессы}.
  
  Однако если 15~лет назад в~формулировке из работы~\cite{13-zac} шла речь 
о~процессах, преобразующих информацию, то в~последние~10~лет в~спектр 
процессов трансформаций все чаще стали включать процессы, преобразующие 
не только информацию, но также и~другие объекты автоматизированных 
систем, в~первую очередь данные и~знания~[19--21]. Например, Виктория 
Стодден, позиционируя науку о~данных как одну из дисциплин информатики, 
говорит, что центральный объект исследований в~науке о~данных~--- это 
<<изучение обобщаемого извлечения знания из данных>>~\cite{21-zac}. 
Увеличение и~чис\-ла объектов, и~спект\-ра процессов их трансформаций 
в~автоматизированных сис\-те\-мах обуслов\-ли\-ва\-ет не\-об\-хо\-ди\-мость 
систематизации и~объектов, и~процессов их трансформаций на начальном этапе 
проектирования сис\-тем.
  
  Для создания концепции лексикографической информационной сис\-те\-мы 
и~проектирования ИТ, обеспечивающих интеграцию 
двуязычных словарей и~параллельных корпусов, сначала выполним 
категоризацию на объекты этой сис\-те\-мы трех базовых понятий информатики 
(данные, информация, знание) в~контексте построения классификаций 
сущностей ее предметной об\-ласти.
  
  Необходимость использования классификаций информатики в~процессе 
создания концепции проиллюстрируем, используя иерархию  
Акоффа~\cite{8-zac}. Он использовал принцип их вертикального размещения 
в~иерархии снизу вверх: данные, информация и~знание. Еще в~ней есть термин 
<<мудрость>>, который в~статье не рассматривается. Такое размещение Акофф 
прокомментировал так: <<Каждое из пе\-ре\-чис\-лен\-ных понятий [кроме данных] 
содержит в~себе нижестоящие$\ldots$>>~\cite{8-zac}.
  
  Этому принципу размещения и~комментарию Акоффа свойственны 
недостатки, проанализированные, в~частности, в~работе~\cite{10-zac}. Главный 
вывод, к~которому пришла Роули после изучения иерархии Акоффа, 
заключается в~следующем: <<$\ldots$информация определяется в~терминах 
данных, знание~--- в~терминах информации$\ldots$ но существует меньше 
консенсуса в~описании трансформаций, которые преобразуют сущности, 
расположенные ниже в~иерархии, в~те, которые находятся над ними, что 
приводит к~их терминологической неопределенности>>~\cite{10-zac}. Причина 
этой неопределенности, скорее всего, в~том, что базовые понятия информатики 
включены в~иерархию Акоффа изолированно от общего контекста 
классификаций сущностей ее предметной об\-ласти.

%\vspace*{-9pt}
  
\section{Классификации сущностей информатики}


%\vspace*{-2pt}

  Все сущности предметной области информатики в~работах~[22, 23] 
разделены на два глобальных класса: ее объекты и~их трансформации. Для 
каждого такого класса была предложена своя классификация. 
В~работе~\cite{22-zac} дано описание классификации объектов предметной 
области информатики, первый уровень которой содержит базовые понятия ее 
предметной области (данные, информация, знания и~др.).  
В~работе~\cite{23-zac} дано описание двух верхних уровней классификации 
трансформаций объектов предметной об\-ласти (см.\ рисунок 
в~работе~\cite{23-zac}). Основанием для построения самого верхнего ее уровня послужило деление 
предметной области информатики на среды\footnote{В~работе~\cite{24-zac} дано описание пяти сред 
предметной области информатики (ментальная; сенсорно воспринимаемая, или информационная; 
цифровая; нейро- и~ДНК-среда), каждая из которых по определению включает объекты одной и~той же 
природы.} и~степень разнообразия природы объектов, вовлеченных в~трансформации:
\begin{itemize}
\item  первый класс верхнего уровня классификации включает 
трансформации объектов в~пределах среды только одной природы 
(трансформации первого порядка);
\item  второй класс включает трансформации объектов, относящихся 
к~двум средам разной природы (трансформации второго порядка);
\item третий и~последующие классы включают трансформации объектов, 
относящихся к~трем и~более средам разной природы (трансформации 
третьего и~более высоких порядков).
\end{itemize}

  В работе~\cite{23-zac} были приведены примеры для трех первых классов 
трансформаций, включая пример трансформаций объектов, относящихся 
к~двум средам разной природы (компьютерное кодирование символов текстов 
с~по\-мощью таб\-лиц Unicode).
  
Основанием для построения второго уровня классификации трансформаций объектов послужила типология 
знаковых сис\-тем А.~Соломоника~\cite[c.~131]{25-zac}: естественные знаковые сис\-те\-мы, образные,  
ес\-тест\-вен\-но-язы\-ко\-в$\acute{\mbox{ы}}$е,  
вер\-баль\-но-не\-сло\-вес\-ные сис\-те\-мы записи\footnote{Под системой записи понимается знаковая 
система, сочетающая вербальные знаки с~несловесными (языки нотной записи, карт, таблиц и~др.).} 
и~формализованные знаковые сис\-те\-мы, включая математические. Введем понятие обобщенного текста~--- 
это текст, который может быть создан в~любой из перечисленных знаковых систем. Тогда обобщенные тексты 
могут быть естественными, образными, ес\-тест\-вен\-но-язы\-ко\-в$\acute{\mbox{ы}}$\-ми,  
вер\-баль\-но-не\-сло\-вес\-ны\-ми и~формализованными. Второй уровень классификации трансформаций 
охватывает не все виды объектов предметной  
об\-ласти информатики, а~только перечисленные~5~видов текс\-тов и~их представления, вовлеченные 
в~процессы трансформаций в~одной или более средах вместе с~данными, знанием и~его концептами.

\begin{figure*}[b] %fig1
\vspace*{6pt}
      \begin{center}
     \mbox{%
\epsfxsize=121.191mm 
\epsfbox{zac-1.eps}
}
\end{center}
\vspace*{-6pt}
\Caption{Средовая версия иерархии Акоффа}
\end{figure*}

\section{Классификация трансформаций: построение~третьего 
уровня}

  Основанием для систематизации трансформаций первого и~второго порядка 
на третьем уровне этой классификации служит иерархия Акоффа~\cite{8-zac}, 
на основе которой и~была создана ее средов$\acute{\mbox{а}}$я версия~[26, 
27]. Для создания средов$\acute{\mbox{о}}$й версии была выполнена 
категоризация трех базовых понятий информатики (данные, информация, 
знания) на объекты лексикографической информационной сис\-те\-мы 
в~процессе создания ее концепции\linebreak (рис.~1).
  


  В отличие от классической иерархии Акоффа, в~ее 
средов$\acute{\mbox{о}}$й версии различаются три вида данных: сенсорно 
воспринимаемые, цифровые и~те данные, которые генерируются 
искусственными нейронными сетями (ИНС) в~системах искусственного интеллекта 
(далее~--- ИИ-дан\-ные). Последний вид данных необходим, например, для 
различения входа и~выхода процесса применения обученной 
ИНС в~цифровой модели генерации знания, описанию которой 
посвящена работа~\cite{27-zac}.
  
  Также предлагается различать два вида информации: сенсорно 
воспринимаемая и~цифровая. Кроме знания в~средов$\acute{\mbox{у}}$ю 
версию добавлены концепты и~ментальные образы сенсорно воспринимаемых 
данных. Последние служат промежуточной сущностью между сенсорно 
воспринимаемыми данными и~генерируемым знанием при описании процессов 
извлечения знания из текстовых данных лексикографической информационной 
системы. Описание объектов средов$\acute{\mbox{о}}$й версии иерархии 
Акоффа (см.\ рис.~1) и~отношений между ними дано в~работах~\cite{26-zac, 28-zac}.
  
  В средов$\acute{\mbox{о}}$й версии число объектов равно восьми. Если 
учитывать направления трансформаций, то между восемью объектами на 
рис.~1 она включает~16 их видов (трансформации на границе между сенсорно 
воспринимаемыми данными и~информацией, обозначенные символом~<<?>>, 
в~статье не рас\-смат\-ри\-ва\-ют\-ся). В~будущем число объектов 
в~средов$\acute{\mbox{о}}$й версии, которая выбрана как основание для 
сис\-те\-ма\-ти\-за\-ции трансформаций первого и~второго порядка, может быть 
увеличено. Для построения классификации трансформаций 
важ\-но не возможное увеличение числа объектов 
и~трансформаций между ними, а то, что их виды в~средов$\acute{\mbox{о}}$й 
версии распределены между трансформациями первого и~второго порядка. Из 
16~видов на рис.~1 шесть относятся к~трансформациям первого порядка, это\linebreak 
виды с~номерами~7, 8, 13--16 (далее~--- типология трансформаций первого 
порядка), а~десять~--- к~трансформациям второго порядка, это виды 
с~\mbox{номерами}~1--6 и~9--12 (далее~--- типология трансформаций второго 
порядка). Разместим обе типологии на третьем уровне классификации (см.\ ее 
схему на рис.~2). Перечислим виды трансформаций первой типологии, вводя 
в~скобках их краткие названия, используемые ниже на рис.~3:
  \begin{description}
  \item[\,] 7~--- членение знания на концепты с~помощью одной или нескольких 
знаковых систем (далее~--- членение знания);
  \item[\,] 8~--- формирование знания на основе концептов (формирование 
знания);
  \item[\,] 13~--- обучение ИНС;
  \end{description}
  
  \vspace*{-6pt}
  
  \pagebreak
  
  \end{multicols}
  
  \begin{figure*} %fig2
\vspace*{1pt}
      \begin{center}
     \mbox{%
\epsfxsize=127.513mm 
\epsfbox{zac-2.eps}
}
\end{center}
\vspace*{-9pt}
\Caption{Схема трех верхних уровней классификации трансформаций объектов (объединены 
по три слоя и~для второго, и~для третьего уровней этой классификации)}
\end{figure*}
  
  \begin{multicols}{2}
  
  \noindent
  \begin{description}
  \item[\,] 14~--- восстановление обучающей информации на основе 
содержания обученной ИНС (обращение ИНС);
  \item[\,] 15~--- использование обученной ИНС (использование ИНС);



  \item[\,] 16~--- восстановление исходных данных, соответствующих 
полученным результатам работы обучен\-ной ИНС (восстановление исходных данных 
по результатам ИНС).
  \end{description}
  
  
  Не все виды трансформаций 13--16 поддерживаются в~конкретных системах 
искусственного интеллекта, но с~теоретической точки зрения все их 
предлагается включить в~первую типологию для полноты спектра видов 
трансформаций.
  
  Перечислим виды трансформаций второй типологии:
  \begin{description}
  \item[\,] 1~--- декодирование цифровых данных в~компьютерных системах 
(декодирование данных);
  \item[\,]  2~--- кодирование сенсорно воспринимаемых данных (кодирование 
данных);
  \item[\,] 3~--- ментальное копирование сенсорно воспринимаемых данных 
(ментальное копирование);
  \item[\,] 4~--- восстановление сенсорно воспринимаемых данных по 
ментальным образам (восстановление по образам);
  \item[\,] 5~--- смысловая интерпретация без деления на концепты ментальных 
образов сенсорно воспринимаемых данных (смысловая интерпретация);
  \item[\,] 6~--- восстановление ментальных образов (восстановление образов);
  \item[\,] 9~--- представление концептов в~виде сенсорно воспринимаемой 
информации, например текс\-та\-ми, формулами, таблицами, рисунками и~т.\,д.\ 
(представление концептов);
  \item[\,] 10~--- понимание смысла сенсорно воспринимаемой информации 
(понимание смысла);
  \item[\,] 11~--- кодирование сенсорно воспринимаемой информации 
(кодирование информации);
\end{description}

\vspace*{-6pt}

\pagebreak

\end{multicols}

\begin{figure*} %fig3
\vspace*{1pt}
      \begin{center}
     \mbox{%
\epsfxsize=163mm 
\epsfbox{zac-3.eps}
}
\end{center}
\vspace*{-9pt}
\Caption{Схема частного случая классификации трансформаций объектов (трансформации 
пронумерованы согласно рис.~1)}
\end{figure*}

\begin{multicols}{2}

\noindent
\begin{description}

  \item[\,] 12~--- декодирование цифровой информации (декодирование 
информации).
  \end{description}
  
  Отметим, что в~существующих ИТ
  и~компьютерных системах наиболее часто используются виды 
трансформаций~13 и~15 типологии первого порядка и~1, 2, 11 и~12 типологии 
второго порядка. На рис.~2 в~первом слое третьего уровня классификации 
показаны типологии первого порядка без указания числа трансформаций в~них 
и~без детализации трансформируемых объектов.
  
  Во втором слое третьего уровня классификации условно (без названий) 
показаны типологии второго порядка. Также на рис.~2 в~третьем слое третьего 
уровня классификации условно (также без названий) показаны типологии 
третьего порядка, которые планируется рассмотреть в~отдельной статье. По 
определению они должны включать трансформации между тремя объектами 
разной природы, но средов$\acute{\mbox{а}}$я версия иерархии Акоффа 
включает трансформации только между двумя объектами разной природы. 
Поэтому потребуется другое основание для их систематизации (ранее были 
рассмотрены отдельные примеры трансформаций третьего 
порядка\footnote{Далеко не всегда трансформации третьего и~более высоких порядков можно 
рассматривать как последовательность трансформаций второго порядка. Примером этого могут 
служить трансформации в~процессе обучения пациента пользованию роботизированной рукой, 
охватывающие личностные концепты пациента, релевантные его намерениям, сигналы активности 
мозга как объекты нейросреды и~компьютерные коды~\cite{29-zac}.}~\cite{29-zac}).

\section{Классификация трансформаций: частный~случай}

  Выше было отмечено, что в~будущем число объектов 
в~средов$\acute{\mbox{о}}$й версии иерархии Акоффа может быть увеличено. 
Это означает, что увеличатся и~чис\-ло объектов, и~чис\-ло трансформаций между 
ними в~классификации трансформаций, так как эта средов$\acute{\mbox{а}}$я 
версия служит по определению основанием для систематизации 
трансформаций первого и~второго порядка. Поэтому на третьем уровне рис.~2 
указаны типологии без детализации объектов и~без указания числа 
трансформаций в~каждой из них. С~одной стороны, при таком подходе 
получаем достаточно общий вид этой классификации, так как она не зависит от 
числа объектов в~том или ином варианте средов$\acute{\mbox{о}}$й версии 
(и~это существенно упрощает рис.~2). С~другой стороны, на третьем уровне 
такой общей классификации подразумевается, но не эксплицируется природа 
трансформируемых объектов и~их возможные сочетания в~трансформациях. 

При проектировании лексикографической информационной системы важно 
эксплицировать природу трансформируемых объектов и~их возможные 
сочетания.
  %
  Поэтому в~парадигму информатики~\cite{30-zac} кроме общей 
классификации трансформаций предлагается включать и~ее частные случаи, 
эксплицирующие природу трансформируемых объектов. 

В~этом разделе 
рассмотрим один частный случай, когда используются только естественные 
знаковые сис\-те\-мы из типологии А.~Соломоника~\cite{25-zac} вместе 
с~данными, знанием и~его концептами. Чис\-ло естественных языков при этом не 
ограничено. И~этот частный случай классификации включает только три 
класса природных трансформаций (первого, второго и~третьего порядка, см.\ 
схему классификации на рис.~3).
  
  Первый и~второй уровни схемы общей классификации (см.\ рис.~2) можно 
объединить в~один уровень в~этом частном случае. Ниже этого уровня 
приведено содержание типологий первого и~второго порядка без содержания 
типологий третьего по\-рядка.




  Наполнение типологий первого и~второго порядка соответствует 
средов$\acute{\mbox{о}}$й версии иерархии Акоффа на рис.~1, содержащей 
6~видов трансформаций типологии первого порядка и~10~видов 
трансформаций типологии второго порядка (на рис.~3 стрелки указывают 
направления трансформаций согласно средов$\acute{\mbox{о}}$й версии на рис.~1).
  
  Таким образом, частный случай классификации содержит для этих двух 
типологий 16~теоретически возможных трансформаций, 6 из которых 
в~настоящее время в~существующих ИТ применяются наиболее часто: виды 
трансформаций~1, 2, 11 и~12 типологии второго порядка реализуются 
с~помощью тех или иных методов ко\-ди\-ро\-ва\-ния/де\-ко\-ди\-ро\-ва\-ния 
(например, с~использованием таблиц Unicode), а~виды трансформаций~13 и~15
 в~типологии первого порядка реализуются полностью с~по\-мощью процессов 
цифровой обработки компьютерами.
  
  Остальные виды трансформаций или применяются намного реже (это 
виды~3, 5, 7, 9 и~10), или находятся в~стадии поиска и~разработки (14 и~16) или 
в~настоящее время носят только теоретический характер, обеспечивая полноту 
первой и~второй типологий (4, 6 и~8). Знаком~<<?>> обозначены те виды 
трансформаций, которые по определению не существуют в~используемой 
парадигме информатики~\cite{30-zac}. Однако возможно, что в~других 
будущих подходах к~построению ее парадигмы эти виды трансформаций будут 
существовать.
  
\section{Заключение}

  На сегодняшний день процесс построения классификаций объектов 
предметной области информатики~\cite{22-zac} и~их  
трансформаций~\cite{23-zac} еще не завершен. Однако первые результаты их 
построения уже используются для создания концепции лексикографической 
информационной сис\-те\-мы, обеспечивающей интеграцию двуязычных 
словарей и~параллельных корпусов.
  
  \bigskip
  
  
  Автор признателен рецензентам за помощь в~улучшении статьи.
  
{\small\frenchspacing
 { %\baselineskip=10.6pt
 %\addcontentsline{toc}{section}{References}
 \begin{thebibliography}{99}
\bibitem{1-zac}
\Au{Aijmer K., Altenberg~B.} Advances in corpus-based contrastive linguistics. Studies in honour 
of Stig Johansson.~--- Amsterdam: John Benjamins, 2013. 295~p.  doi: 10.1075/scl.54.
\bibitem{2-zac}
\Au{Добровольский Д.\,О., Кретов~А.\, А., Шаров~С.\,А.} Корпус параллельных текстов~// 
Научная и~техническая информация. Сер.~2: Информационные процессы и~сис\-те\-мы, 2005. 
№\,6. С.~16--27.
\bibitem{3-zac}
\Au{Добровольский Д.\,О.} Корпус параллельных текстов и~сопоставительная 
лексикология~// Труды Института русского языка им.\ В.\,В.~Виноградова, 2015. №\,6. 
С.~413--449. EDN: VJQBHP.
\bibitem{4-zac}
\Au{Гончаров А.\,А., Зацман~И.\,М., Кружков~М.\,Г.} Эволюция классификаций 
в~надкорпусных базах данных~// Информатика и~её применения, 2020. Т.~14. Вып.~4. 
С.~108--116. doi: 10.14357/19922264200415.  
EDN: \mbox{GKWBZT}.
\bibitem{5-zac}
\Au{Гончаров А.\, А., Зацман И. \,М., Кружков~М.\, Г}. Представление новых 
лексикографических знаний в~динамических классификационных сис\-те\-мах~// 
Информатика и~её применения, 2021. Т.~15. Вып.~1. С.~86--93.  doi: 10.14357/19922264210112. EDN: OPEFXW.
\bibitem{6-zac}
\Au{Zatsman I.} Finding and filling lacunas in linguistic typologies~// 15th Forum (International) 
on Knowledge Asset Dynamics Proceedings.~--- Matera, Italy: Institute of Knowledge Asset 
Management, 2020. P.~780--793.
\bibitem{7-zac}
\Au{Zatsman I.} Three-dimensional encoding of emerging meanings in AI-systems~// 21st 
European Conference on Knowledge Management Proceedings.~--- Reading, U.K.: Academic 
Publishing International Ltd., 2020. P.~878--887.
\bibitem{8-zac}
\Au{Ackoff R.} From data to wisdom~// J.~Applied Systems Analysis, 1989. Vol.~16. No.\,1. P.~3--9.
\bibitem{9-zac}
\Au{Rosenbloom P.\,S.} On computing: The fourth great scientific domain.~--- Cambridge, MA, 
USA: MIT Press, 2013. 307~p.
\bibitem{10-zac}
\Au{Rowley J.} The wisdom hierarchy: Representations of the DIKW hierarchy~// J.~Inf. 
Sci., 2007. Vol.~33. Iss.~2. P.~163--180. doi: 10.1177/0165551506070706.
\bibitem{11-zac} 
\Au{Frick$\acute{\mbox{e}}$~M.\,H.} Data--Information--Knowledge--Wisdom (DIKW) pyramid, 
framework, continuum~// Encyclopedia of big data~/ Eds. L.~Schintler, C.~McNeely.~--- Cham: 
Springer, 2018. 4~p. doi: 10.1007/978-3-319-32001-4\_331-1.
\bibitem{12-zac}
\Au{Denning P., Rosenbloom~P.} Computing: The fourth great domain of science~// Commun. 
ACM, 2009. Vol.~52. Iss.~9. P.~27--29.
\bibitem{13-zac}
\Au{Denning P., Freeman~P.} Computing's paradigm~// Commun.  ACM, 2009. Vol.~52. 
Iss.~12. P.~28--30. doi: 10.1145/ 1610252.1610265.
\bibitem{17-zac} %14
\Au{Farradane J.} Knowledge, information, and information science~// J.~Inf. Sci., 
1980. Vol.~2. Iss.~2. P.~75--80. doi: 10.1177/01655515800020020.

\bibitem{15-zac}
\Au{Шрейдер Ю.\,А.} Информация и~знание~// Сис\-тем\-ная концепция информационных 
процессов.~--- М.: ВНИИСИ, 1988. С.~47--52.
\bibitem{16-zac}
\Au{Ingwersen P.} Information and information science~// Enclyclopaedie of library and 
information science~/ Eds. J.\,D.~McDonald, 
M.~Levine-Clark.~--- New York, NY, USA: Marcel Dekker Inc., 1992. Vol.~56. Sup.~19. 
P.~137--174.

\bibitem{14-zac} %17
Информатика как наука об информации: Информационный, документальный, 
технологический, экономический, социальный и~организационный аспекты~/ Под ред. 
Р.\,С.~Гиляревского.~--- М.: Фаир-Пресс, 2006. 592~с.

\bibitem{18-zac}
\Au{Hjorland B.} Library and information science: practice, theory, and philosophical basis~// 
Inform. Process. Manag., 2000. Vol.~36. Iss.~3. P.~501--531. doi:  
10.1016/S0306-\mbox{4573(99)00038-2}.
\bibitem{19-zac}
Deep shift~--- technology tipping points and societal impact.~--- Geneva: WE Forum, 2015. 44~p. 
{\sf http://www3.weforum.org/docs/WEF\_GAC15\_ Technological\_Tipping\_Points\_report\_2015.pdf}.
\bibitem{20-zac}
\Au{Berman F., Rutenbar~R., Hailpern~B., Christensen~H., Davidson~S., Estrin~D., 
Franklin~M., Martonosi~M., Raghavan~P., Stodden~V., Szalay~A.\,S.} Realizing the potential of 
data science~// Commun.  ACM, 2018. Vol.~61. Iss.~4. P.~67--72. doi: 10.1145/3188721.

\bibitem{21-zac}
\Au{Stodden V.} The data science life cycle: A~disciplined approach to advancing data science as 
a~science~// Commun.  ACM, 2020. Vol.~63. Iss.~7. P.~58--66. doi: 10.1145/ 3360646.


\bibitem{23-zac} %22
\Au{Зацман И.\,М.} Научная парадигма информатики: классификация трансформаций 
объектов предметной об\-ласти~// Системы и~средства информатики, 2023. Т.~33. №\,4. 
С.~126--138. doi: 10.14357/08696527230412. EDN: ZIKUWO.

\bibitem{22-zac} %23
\Au{Зацман И.\,М.} Научная парадигма информатики: классификация объектов предметной  
об\-ласти~// Информатика и~её применения, 2023. Т.~17. Вып.~4. С.~96--103. doi: 
10.14357/19922264230413. EDN: FIUQAT.

\bibitem{24-zac}
\Au{Зацман И.\,М.} О~научной парадигме информатики: верхний уровень классификации 
объектов ее предметной об\-ласти~// Информатика и~её применения, 2022. Т.~16. Вып.~4. 
С.~73--79. doi: 10.14357/ 19922264220411. EDN: XZNKVI.

\bibitem{25-zac}
\Au{Соломоник А.\,Б.} Философия знаковых систем и~язык.~--- М.: ЛКИ, 2011. 408~с.
\bibitem{26-zac}
\Au{Зацман И.\,М.} Трансформация иерархии Акоффа в~научной парадигме информатики~// 
Информатика и~её применения, 2023. Т.~17. Вып.~3. С.~107--113. doi: 
10.14357/19922264230315. EDN: UMVRRV.

\bibitem{27-zac}
\Au{Zatsman I.} Building digital spiral models of knowledge generation~// 19th Forum 
(International) on Knowledge Asset Dynamics Proceedings.~--- Matera, Italy: Arts for Business 
Institute, 2024. P.~2185--2196.
\bibitem{28-zac}
\Au{Zatsman I.} Digital spiral model of knowledge creation and encoding its dynamics~// 18th 
Forum (International) on Knowledge Asset Dynamics Proceedings.~--- Matera, Italy: Arts for 
Business Institute, 2023. P.~581--596.
\bibitem{29-zac}
\Au{Зацман И.\,М.} Интерфейсы третьего порядка в~информатике~// Информатика и~её 
применения, 2019. Т.~13. Вып.~3. С.~82--89. doi: 10.14357/19922264190312. EDN: 
EHRQLF.

\bibitem{30-zac}
\Au{Зацман И.\,М.} Научная парадигма информатики как третьей культуры~//  
На\-уч\-но-тех\-ни\-че\-ская информация. Сер.~1: Организация и~методика информационной 
работы, 2023. №\,11. С.~1--14.

\end{thebibliography}

 }
 }

\end{multicols}

\vspace*{-9pt}

\hfill{\small\textit{Поступила в~редакцию 14.04.24}}

\vspace*{4pt}

%\pagebreak

%\newpage

%\vspace*{-28pt}

\hrule

\vspace*{2pt}

\hrule



\def\tit{OBJECT TRANSFORMATIONS OF~THE~FIRST AND~SECOND ORDER
IN~A~LEXICOGRAPHIC INFORMATION SYSTEM\\[-5pt]}


\def\titkol{Object transformations of~the~first and~second order
in~a~lexicographic information system}


\def\aut{I.\,M.~Zatsman}

\def\autkol{I.\,M.~Zatsman}

\titel{\tit}{\aut}{\autkol}{\titkol}

\vspace*{-13pt}


\noindent
Federal Research Center ``Computer Science and Control'' of the Russian Academy of Sciences, 
44-2~Vavilov Str., Moscow 119133, Russian Federation


\def\leftfootline{\small{\textbf{\thepage}
\hfill INFORMATIKA I EE PRIMENENIYA~--- INFORMATICS AND
APPLICATIONS\ \ \ 2024\ \ \ volume~18\ \ \ issue\ 2}
}%
 \def\rightfootline{\small{INFORMATIKA I EE PRIMENENIYA~---
INFORMATICS AND APPLICATIONS\ \ \ 2024\ \ \ volume~18\ \ \ issue\ 2
\hfill \textbf{\thepage}}}

\vspace*{2pt}



\Abste{The theoretical foundations of the design of information technologies used for 
the integration of bilingual dictionaries and parallel corpora are considered. The 
description of the first outcomes of the creation of the third\linebreak\vspace*{-12pt}}

\Abstend{ level of object 
transformations classification in the subject domain of informatics, which is supposed 
to be used
in creating the lexicographic information system providing integration, is 
given. All the entities of informatics are divided into two global classes: objects and 
their transformations. For each such class, its own classification is constructed. 
Previously, the two upper levels of the object transformation classification in the subject 
domain have been described. The present paper discusses the third level of this classification. The 
basis for the construction of its highest level was the division of the subject domain of 
informatics into media (mental, sensory, digital, and a~number of other media), each 
of which by definition includes objects of the same nature. The Solomonick's 
typology of sign systems served as the basis for constructing the second level of the 
object transformation classification. The aim of the paper is to systematize object 
transformations of the first and second orders at the third level of this classification. 
The basis for systematization is the medium version of the Ackoff's hierarchy.}

\KWE{subject domain objects; object transformations; classification; data; 
information; knowledge; lexicographic information system}


\DOI{10.14357/19922264240211}{VZTGVV}

\vspace*{-12pt}

\Ack

\vspace*{-3pt}


\noindent
The reported study was funded by the Russian Science Foundation, project  
No.\,24-18-00155, {\sf 
https://rscf.ru/project/24-18-00155}. The research was carried out using the infrastructure of the Shared 
Research Facilities ``High Performance Computing and Big Data'' (CKP 
``Informatics'') of FRC CSC RAS (Moscow) .
   


  \begin{multicols}{2}

\renewcommand{\bibname}{\protect\rmfamily References}
%\renewcommand{\bibname}{\large\protect\rm References}

{\small\frenchspacing
 {%\baselineskip=10.8pt
 \addcontentsline{toc}{section}{References}
 \begin{thebibliography}{99} 
\bibitem{1-zac-1}
\Aue{Aijmer, K., and B.~Altenberg.} 2013. \textit{Advances in corpus-based 
contrastive linguistics. Studies in honour of Stig Johansson}. Amsterdam: John 
Benjamins. 295~p. doi: 10.1075/scl.54.
\bibitem{2-zac-1}
\Aue{Dobrovolskiy, D.\,O., A.\,A.~Kretov, and S.\,A.~Sharov.} 2005. Korpus 
parallel'nykh tekstov [Corpus of parallel texts]. \textit{Nauchnaya i~tekhnicheskaya 
informatsiya. Ser. 2. Informatsionnye protsessy i~sistemy} [Scientific and Technical 
Information. Ser.~2: Information Processes and Systems] 6:16--27.
\bibitem{3-zac-1}
\Aue{Dobrovolskiy, D.\,O.} 2015. Korpus parallel'nykh tekstov i~sopostavitel'naya 
leksikologiya [The corpus of parallel texts and contrastive lexicology]. \textit{Trudy 
Instituta russkogo yazyka im. V.\,V.~Vinogradova} [Proceedings of the 
V.\,V.~Vinogradov Russian Language Institute] 6:413--449. EDN: VJQBHP.
\bibitem{4-zac-1}
\Aue{Goncharov, A.\,A., I.\,M.~Zatsman, and M.\,G.~Kruzhkov.} 2020. Evolyutsiya 
klassifikatsiy v~nadkorpusnykh ba\-zakh dannykh [Evolution of classifications in 
supracorpora databases]. \textit{Informatika i~ee Primeneniya~--- Inform. \mbox{Appl.}}  
14(4):108--116. doi: 10.14357/19922264200415.  
EDN: GKWBZT.
\bibitem{5-zac-1}
\Aue{Goncharov, A.\,A., I.\,M.~Zatsman, and M.\,G.~Kruzhkov.} 2021. 
Predstavlenie novykh leksikograficheskikh znaniy v~dinamicheskikh 
klassifikatsionnykh sistemakh [Representation of new lexicographical knowledge in 
dynamic classification systems]. \textit{Informatika i~ee Primeneniya~--- Inform. 
Appl.} 15(1):86--93. doi: 10.14357/19922264210112. EDN: OPEFXW.
\bibitem{6-zac-1}
\Aue{Zatsman, I.} 2020. Finding and filling lacunas in linguistic typologies. 
\textit{15th Forum (International) on Knowledge Asset Dynamics Proceedings}. 
Matera, Italy: Institute of Knowledge Asset Management. 780--793.
\bibitem{7-zac-1}
\Aue{Zatsman, I.} 2020. Three-dimensional encoding of emerging meanings in  
AI-systems. \textit{21st European Conference on Knowledge Management 
Proceedings}. Reading, U.K.: Academic Publishing International Ltd. 878--887.
\bibitem{8-zac-1}
\Aue{Ackoff, R.} 1989. From data to wisdom. \textit{J.~Applied Systems Analysis} 
16(1):3--9.
\bibitem{9-zac-1}
\Aue{Rosenbloom, P.\,S.} 2013. \textit{On computing: The fourth great scientific 
domain}. Cambridge, MA: MIT Press. 307~p.
\bibitem{10-zac-1}
\Aue{Rowley, J.} 2007. The wisdom hierarchy: Representations of the DIKW 
hierarchy. \textit{J.~Inf. Sci.} 33(2):163--180. doi: 10.1177/0165551506070706.
\bibitem{11-zac-1}
\Aue{Frick$\acute{\mbox{e}}$, M.\,H.} 2018.  
Data-Information-Knowledge-Wisdom (DIKW) pyramid, framework, continuum. 
\textit{Encyclopedia of big data}. Eds. L.~Schintler and C.~McNeely. Cham: 
Springer. 4~p. doi: 10.1007/978-3-319-32001- 4\_331-1.
\bibitem{12-zac-1}
\Aue{Denning, P., and P.~Rosenbloom.} 2009. Computing: The fourth great domain 
of science. \textit{Commun. ACM} 52(9):27--29.
\bibitem{13-zac-1}
\Aue{Denning, P., and P.~Freeman.} 2009. Computing's paradigm. \textit{Commun. 
ACM} 52(12):28--30. doi: 10.1145/ 1610252.1610265.

\bibitem{17-zac-1} %14
\Aue{Farradane, J.} 1980. Knowledge, information, and information science. 
\textit{J.~Inf. Sci.} 2(2):75--80. doi: 10.1177/ 01655515800020020.

\bibitem{15-zac-1}
\Aue{Shreyder, Yu.\,A.} 1988. Informatsiya i~znanie [Information and knowledge]. 
\textit{Sistemnaya kontseptsiya in\-for\-ma\-tsi\-on\-nykh protsessov} [System concept of 
information processes]. Moscow: VNIISI. 47--52.
\bibitem{16-zac-1}
\Aue{Ingwersen, P.} 1995. Information and information science. 
\textit{Encyclopedia of library and information science}. Eds. J.\,D.~McDonald and 
M.~Levine-Clark. New York, NY: Marcel Dekker Inc. 56(19):137--174.

\bibitem{14-zac-1} %17
Gilyarevskiy, R.\,S., ed. 2006. \textit{Informatika kak nauka ob informatsii: 
informatsionnyy, dokumental'nyy, tekh\-no\-lo\-gi\-che\-skiy, ekonomicheskiy, sotsial'nyy 
i~organizatsionnyy aspekty} [Informatics as information science: Informational, 
documentary, technological, economic, social, and organizational dimensions]. 
Moscow: FAIR-PRESS. 592~p.

\bibitem{18-zac-1}
\Aue{Hjorland, B.} 2000. Library and information science: Practice, theory, and 
philosophical basis. \textit{Inform. Process. Manag.} 36(3):501--531. doi:  
10.1016/S0306-\mbox{4573(99)00038-2}.
\bibitem{19-zac-1}
Deep shift~--- technology tipping points and societal impact. 2015. \textit{World Economic 
Forum}. Geneva. 44~p. Available at: {\sf 
http://www3.weforum.org/docs/WEF\_ GAC15\_Technological\_Tipping\_Points\_report\_2015.pdf} (accessed May~20, 
2024).
\bibitem{20-zac-1}
\Aue{Berman, F., R.~Rutenbar, B.~Hailpern, H.~Christensen, S.~Davidson, 
D.~Estrin, M.~Franklin, M.~Martonosi, P.~Raghavan, V.~Stodden, and 
A.\,S.~Szalay.} 2018. Realizing the potential of data science. \textit{Commun. ACM} 
61(4):67--72. doi: 10.1145/3188721.
\bibitem{21-zac-1}
\Aue{Stodden, V.} 2020. The data science life cycle: A~disciplined approach to 
advancing data science as a~science. \textit{Commun. ACM} 
 63(7):58--66. doi: 10.1145/3360646.

\bibitem{23-zac-1} %22
\Aue{Zatsman, I.\,M.} 2023. Nauchnaya paradigma informatiki: klassifikatsiya 
transformatsiy ob''ektov predmetnoy oblasti [Scientific paradigm of informatics: 
Transformation classification of domain objects]. \textit{Sistemy i~Sredstva 
Informatiki~--- Systems and Means of Informatics} 33(4):126--138. doi: 
10.14357/08696527230412. EDN: ZIKUWO.

\bibitem{22-zac-1} %23
\Aue{Zatsman, I.\,M.} 2023. Nauchnaya paradigma informatiki: klassifikatsiya 
ob''ektov predmetnoy oblasti [Scientific paradigm of informatics: Classification of 
domain objects]. \textit{Informatika i~ee Primeneniya~--- Inform. Appl.} 
 17(4):96--103. doi: 10.14357/19922264230413. EDN: FIUQAT.
 
\bibitem{24-zac-1}
\Aue{   Zatsman, I.\,M.} 2022. O nauchnoy paradigme informatiki: verkhniy uroven' 
klassifikatsii ob''ektov ee predmetnoy oblasti [On the scientific paradigm of 
informatics: The classification high level of its objects]. \textit{Informatika i~ee 
Primeneniya~--- Inform. Appl.} 16(4):73--79. doi: 10.14357/19922264220411. EDN: 
XZNKVI.
\bibitem{25-zac-1}
\Aue{Solomonick, A.\,B.} 2011. \textit{Filosofiya znakovykh system i~yazyk} 
[Philosophy of sign systems and language]. Moscow: LKI. 408~p.
\bibitem{26-zac-1}
\Aue{Zatsman, I.\,M.} 2023. Transformatsiya ierarkhii Akoffa v~nauchnoy 
paradigme informatiki [Transformation of the Ackoff's hierarchy in the scientific 
paradigm of informatics]. \textit{Informatika i~ee Primeneniya~--- Inform. \mbox{Appl.}} 
17(3):107--113. doi: 10.14357/19922264230315. EDN: UMVRRV.
\bibitem{27-zac-1}
\Aue{Zatsman, I.} 2024. Building digital spiral models of knowledge 
generation. \textit{19th Forum (International) on Knowledge Asset Dynamics 
Proceedings}. Matera, Italy: Arts for Business Institute. 2185--2196.
\bibitem{28-zac-1}
\Aue{Zatsman, I.} 2023. Digital spiral model of knowledge creation and encoding its 
dynamics. \textit{18th Forum (International) on Knowledge Asset Dynamics 
Proceedings}. Matera, Italy: Arts for Business Institute. 581--596.
\bibitem{29-zac-1}
\Aue{Zatsman, I.\,M.} 2019. Interfeysy tret'ego poryadka v~informatike 
 [Third-order interfaces in informatics]. \textit{Informatika i~ee Primeneniya~--- 
Inform. Appl.} 13(3):82--89. doi: 10.14357/19922264190312. EDN: EHRQLF.
\bibitem{30-zac-1}
\Aue{Zatsman, I.} 2023. Scientific paradigm of informatics as a~third culture. 
\textit{Scientific Technical Information Processing} 50(4):246--258. doi: 
10.3103/S0147688223040111. EDN: CKHMYS.

\end{thebibliography}

 }
 }

\end{multicols}

\vspace*{-6pt}

\hfill{\small\textit{Received April 14, 2024}} 


\vspace*{-12pt}


\Contrl

\vspace*{-3pt}

\noindent
\textbf{Zatsman Igor M.} (b.\ 1952)~--- Doctor of Science in technology, head of 
department, Federal Research Center ``Computer Science and Control'' of the 
Russian Academy of Sciences, 44-2~Vavilov Str., Moscow 119333, Russian 
Federation; \mbox{izatsman@yandex.ru}





\label{end\stat}

\renewcommand{\bibname}{\protect\rm Литература}  %14
\def\stat{grinchenko}

\def\tit{О ГЕНЕЗИСЕ ИНФОРМАЦИОННОГО ОБЩЕСТВА:  
ИНФОРМАТИКО-КИБЕРНЕТИЧЕСКОЕ МОДЕЛЬНОЕ ПРЕДСТАВЛЕНИЕ}

\def\titkol{О генезисе информационного общества:  
информатико-кибернетическое модельное представление}

\def\aut{С.\,Н.~Гринченко$^1$}

\def\autkol{С.\,Н.~Гринченко}

\titel{\tit}{\aut}{\autkol}{\titkol}

\index{Гринченко С.\,Н.}
\index{Grinchenko S.\,N.}


%{\renewcommand{\thefootnote}{\fnsymbol{footnote}} \footnotetext[1]
%{Работа выполнена при частичной финансовой 
%поддержке РФФИ (проект 17-07-00577).}}


\renewcommand{\thefootnote}{\arabic{footnote}}
\footnotetext[1]{Институт проблем информатики Федерального исследовательского центра <<Информатика и~управление>> 
Российской академии наук, \mbox{sgrinchenko@ipiran.ru}}

\vspace*{-3.5pt}




  \Abst{Вводится понятие <<генезис информационного общества>>, которое рассматривается 
  с~позиций ин\-фор\-ма\-ти\-ко-ки\-бер\-не\-ти\-че\-ско\-го моделирования (ИКМ)
  процесса развития 
Человечества как са\-мо\-управ\-ля\-ющей\-ся иерар\-хо-се\-те\-вой системы. На этой основе 
получены количественные оценки его типовых про\-стран\-ст\-вен\-но-вре\-мен\-ных характеристик, 
представляющих собой геометрические прогрессии со знаменателем 
<<$e$~в~степени~$e$>> (15,15426$\ldots$), а~также скоординированных с~ними во времени 
и~в~пространстве пси\-хи\-ко-ант\-ро\-по\-ло\-ги\-че\-ских, образовательных  
и~ин\-фор\-ма\-ци\-он\-но-ком\-му\-ни\-ка\-ци\-он\-ных параметров и~возможностей 
включенного в~этот процесс усложняющегося человека и~его сообществ различной 
величины. Это позволило раздвинуть рамки существования информационного общества на 
всю историческую и~даже археологическую эпоху такого развития. Результирующая 
последовательность информационных технологий (ИТ) <<сигнальные  
по\-зы/зву\-ки/дви\-же\-ния\,--\,ми\-ми\-ка/жес\-ты\,--\,речь/язык\,--\,пись\-мен\-ность\,--\,ти\-ра\-жи\-ро\-ва\-ние текстов\,--\,компью\-те\-ры\,--\,те\-ле\-ком\-му\-ни\-ка\-ции\,--\,ин\-фор\-ма\-ци\-он\-ная на\-но\-тех\-но\-ло\-гия\,--\,$\ldots$>> 
позволяет рас\-смат\-ри\-вать генезис 
информационного общества в~широком контексте единой исторической ретроспективы 
и~перспективы.}
  
  \KW{информационное общество; информационные технологии;  
ин\-фор\-ма\-ти\-ко-ки\-бер\-не\-ти\-че\-ская модель; самоуправляющаяся 
 иерар\-хо-се\-те\-вая система Человечества; археологическая эпоха}
 
 \DOI{10.14357/19922264190214}
  
%\vspace*{4pt}


\vskip 10pt plus 9pt minus 6pt

\thispagestyle{headings}

\begin{multicols}{2}

\label{st\stat}
  
  В~литературе, даже энциклопедической, распространена трактовка 
<<информационного общества>> как общества <<современного типа>>, 
в~котором общение людей опирается на компьютерные 
и~телекоммуникационные ИТ\footnote[2]{В~[1] дано следующее определение:
<<\textbf{Информационное общество}, одно из понятий, используемых 
в~социологич.\ теории для обозначения обществ.\ систем <<современного типа>>$\ldots$ 
Важнейшие характеристики~И.\,о.: 1)~лавинообразное распространение информац. 
технологий (прежде всего компьютерных и~телекоммуникационных); 2)~превращение 
информации в~важнейший социальный ресурс, необходимую предпосылку управленч. 
деятельности, развития экономики, образования, сферы услуг, домашнего быта, 
рекреационной сферы и~т.\,д.; по некоторым данным, в~наиболее развитых странах проф. 
деятельность более половины занятых связана исключительно с~производством и~обработкой 
информации; 3)~наделение СМИ статусом <<четвертой ветви власти>>; 4)~расширение 
границ и~<<репертуара>> массовой культуры; 5)~увеличение каналов вертикальной 
и~горизонтальной мобильности; 6)~изменение представлений о~социальном пространстве 
(<<глобализация>> пространства, мгновенная доступность даже периферийных его 
сегментов) и~времени (расширение рамок <<современности>>, когда даже отдаленные 
историч. события воспринимаются как происходящие <<здесь>> и~<<сейчас>>); 
7)~возникновение в~процессе коммуникации особой виртуальной реальности, несводимой 
к~результатам технич. визуализации и~выходящей за пределы воображения и~памяти 
индивида; 8)~превращение информац. технологий в~базу для развития высоких технологий 
(Hi-Tech)>>.}. Такая трактовка этого понятия создает иллюзию 
отстраненности информационного общества от его собственного исторического 
прошлого, когда вышеперечисленных ИТ еще не изобрели, но люди в~составе 
сообществ как-то общались между собой, используя иные ИТ. 

Поскольку от 
этой иллюзии недалеко до недооценки полезности соответствующего 
исторического опыта для современности, попытаюсь развеять ее.
  
Результаты ИКМ процесса развития на Земле 
Человечества как самоуправляющейся ие\-рар\-хо-се\-те\-вой\footnote[3]{<<\textbf{Иерархо-сетевая}>> 
структура~--- иерархическая структура типа <<матрешки>>, но с~существенно большим 
единицы числом вложений на каждом ее иерархическом уровне, которые и~образуют 
соответствующие сетевые структуры.} системы~[2--14] (рис.~1) позволяют раздвинуть рамки 
существования информационного\linebreak общества на всю историческую и~даже археологическую эпоху такого 
развития, что дает возможность выделить ту эволюционную линию этого процесса, которую логично 
определить как \textit{генезис информационного общества}. 


\begin{figure*} %fig1
   \vspace*{1pt}
    \begin{center}  
  \mbox{%
 \epsfxsize=130.287mm 
 \epsfbox{gri-1.eps}
 }
\end{center}
%\vspace*{-9pt}
%\Caption{Схема иерархо-сетевой самоуправляющейся (по алгоритмам случайной поисковой 
%оптимизации целевых критериев энергетического характера с~ограничениями типа 
%равенств и~неравенств) системы лич\-ност\-но-про\-из\-вод\-ст\-вен\-но-со\-ци\-аль\-ной природы 
%(Человечества)~\cite{5-grn}}
\end{figure*}


На рис.~1 используются следующие обозначения:
\begin{itemize}
\item восходящие стрелки (имеющие структуру <<мно\-гие\,--\,к~од\-но\-му>>) 
отражают первую из~5~основных со\-став\-ля\-ющих контура поисковой 
оптимизации~--- \textit{поисковую активность} представителей 
соответствующих ярусов в~иерархии; 
\item нисходящие сплошные (имеющие 
структуру <<один\,--\,ко мно\-гим>>) стрелки отражают вторую 
со\-став\-ля\-ющую~--- \textit{целевые критерии} поисковой оптимизации 
энергетики системы Человечества; 
\item нисходящие пунктирные (<<один\,--\,ко 
многим>>) стрелки отражают третью со\-став\-ля\-ющую~--- 
\textit{оптимизационную системную память}  
лич\-ност\-но-про\-из\-вод\-ст\-вен\-но-со\-ци\-аль\-но\-го (результат 
адаптивных влияний представителей вышележащих иерархических ярусов на 
структуру вложенных в~них нижележащих); 
\item полужирными стрелками 
в~левой части схемы условно показана четвертая со\-став\-ля\-ющая~--- 
\textit{антропогенная ак\-тив\-ность} индивидов и~их групп, трак\-ту\-емая как 
<<трудовая деятельность по созданию со\-от\-вет\-ст\-ву\-юще\-го инструментария 
и~результатов его применения>>; 
\item пунктирными полужирными стрелками 
в~правой части схемы условно показана пятая со\-став\-ля\-ющая~--- 
\textit{антропогенная системная\linebreak память}  
лич\-ност\-но-про\-из\-вод\-ст\-вен\-но-со\-ци\-ального (процессы вовлечения 
результатов антропогенной активности в~структуру со\-от\-вет\-ст\-ву\-ющей  
иерар\-хо-се\-те\-вой под\-сис\-те\-мы Человечества).
\end{itemize}

Рассмотрим этот феномен поэтапно, сведя в~общую таблицу расчетные данные 
о~различных его проявлениях. 
       


\begin{table*}\footnotesize
\begin{center}
\Caption{Свод основных характеристик генезиса информационного общества (как 
проявления развития са\-мо\-управ\-ля\-ющей\-ся и~метаэволюционирующей, т.\,е.\ 
наращивающей чис\-ло своих иерархических уров\-ней/яру\-сов, сис\-те\-мы Человечества) от 
прошлого до модельно прогнозируемого будущего}
\vspace*{2ex}

\tabcolsep=1.5pt
\begin{tabular}{|c|c|l|c|c|c|c|}
\hline
&\tabcolsep=0pt\begin{tabular}{c}Характерный\\ ареал (радиус\\
 круга той же\\ площади); точность\\ антропогенного\\ 
воздействия\,/\\
производственных\\ технологий\end{tabular}&
\tabcolsep=0pt\begin{tabular}{c}Характерные\\ времена\\ старта;\\ кульминации\\ 
развития\\ подсистемы\end{tabular}&
\tabcolsep=0pt\begin{tabular}{c}Уровень\\ развития\\ Homo\\  
(и его пред-\\ шествен-\\ ников)\end{tabular}&
\tabcolsep=0pt\begin{tabular}{c}Носитель системной\\ памяти~---\\ субстрат психики\end{tabular}&
\tabcolsep=0pt\begin{tabular}{c}Лидирующая\\ ИТ\end{tabular}&
\tabcolsep=0pt\begin{tabular}{c}Требуемый уровень\\ образованности Homo;\\
аналогия филогенеза\\ и~онтогенеза:\\ примерный возраст\\ гармонично\\ образовываемого\\ 
(сегодня)\end{tabular}\\
\hline
1&2&\multicolumn{1}{c|}{3}&4&5&6&7\\
\hline
0&$\sim4{,}2$~м&\tabcolsep=0pt\begin{tabular}{c} $\sim428$~млн\\ лет назад;\\
$\sim 140{,}1$~млн\\ лет назад\end{tabular}&
\tabcolsep=0pt\begin{tabular}{c}Цефализация\\ позвоночных\end{tabular}&
\tabcolsep=0pt\begin{tabular}{c}Многоклеточный\\организм в~целом\end{tabular}&
\tabcolsep=0pt\begin{tabular}{c}Формирование\\ головного\\ мозга как основы\\
 для реализации\\ 
будущих ИТ\end{tabular}&\tabcolsep=0pt\begin{tabular}{c} ---\\
$\sim0{,}6$--1,0~год\end{tabular}\\
\hline
1&\tabcolsep=0pt\begin{tabular}{c} $\sim64$~м;\\
$\sim28$~см
\end{tabular}&\tabcolsep=0pt\begin{tabular}{c}$\sim28{,}23$~млн\\ лет назад;\\
$\sim9{,}26$~млн\\ лет назад
\end{tabular}&\tabcolsep=0pt\begin{tabular}{c}Пред-пред-\\
люди\\ Hominoidea\end{tabular}&
\tabcolsep=0pt\begin{tabular}{c}Органы многоклеточного\\ организма (его 
нервной\\ системы в~целом)\end{tabular}&
\tabcolsep=0pt\begin{tabular}{c}Сигнальные позы/\\
движения\\ и~неинтонированные\\ звуки (типа 
рычания,\\ ворчания, писка\\ и~т.\,п.)\end{tabular}&
\tabcolsep=0pt\begin{tabular}{c}Выработка\\ 
(младенцами)\\ сигнальных поз;\\
$\sim1{,}0$--1,6~лет \end{tabular}\\
\hline
2&\tabcolsep=0pt\begin{tabular}{c} $\sim1$~км;\\
$\sim1{,}8$~см\end{tabular}&\tabcolsep=0pt\begin{tabular}{c} $\sim1{,}86$~млн\\ лет 
назад;\\
$\sim612$~тыс.\\ лет назад\end{tabular}&
\tabcolsep=0pt\begin{tabular}{c}Пред-люди\\ Homo ergaster\,/\\
Homo erectus\end{tabular}&
\tabcolsep=0pt\begin{tabular}{c}Ткани 
многоклеточного\\ организма\\ (сетей/ансамблей\\ нейронов и~др.)\end{tabular}&
\tabcolsep=0pt\begin{tabular}{c}Мимика/жесты\\ 
и~интонированные\\ звуки\end{tabular}&
\tabcolsep=0pt\begin{tabular}{c}Овладение (ре-\\ бенком) мимикой/\\ 
жестами,\\
начальное\\ понимание речи; \\ $\sim1{,}6$--2,6~лет \end{tabular}\\
\hline
3&\tabcolsep=0pt\begin{tabular}{c} $\sim15$~км; \\
$\sim1{,}2$~мм
\end{tabular}&
\tabcolsep=0pt\begin{tabular}{c} $\sim123$~тыс.\\ лет назад;\\
$\sim40$~тыс.\\ лет назад\end{tabular}&
\tabcolsep=0pt\begin{tabular}{c}Homo\\ sapiens$^\prime$\end{tabular}&
\tabcolsep=0pt\begin{tabular}{c}Эвкариотические\\ клетки\\ 
многоклеточного\\ организма\\ (отдельные нервные\\ и~глиальные клетки\\ и~др.)\end{tabular}&
\tabcolsep=0pt\begin{tabular}{c}Речь/язык\\ 
(артикулированная\\ устная речь)\end{tabular}&
\tabcolsep=0pt\begin{tabular}{c}Овладение (детьми)\\ 
речью/языком\\ (протообразование); \\ $\sim2{,}6$--4,2~лет \end{tabular}\\
\hline
4&\tabcolsep=0pt\begin{tabular}{c} $\sim222$~км;\\
$\sim 80$~мкм
\end{tabular}&\tabcolsep=0pt\begin{tabular}{c}$\sim8{,}1$~тыс.\\ лет назад;\\
$\sim2{,}7$~тыс.\\ лет назад\end{tabular}&
\tabcolsep=0pt\begin{tabular}{c}Homo\\ sapiens$^{\prime\prime}$\end{tabular}&
\tabcolsep=0pt\begin{tabular}{c}Компартменты\\ 
эвкариотической\\ клетки (отдельные\\ рецепторные,\\ или постсинаптические,\\ зоны нейронов и~т.\,п.)\end{tabular}
&Письменность&\tabcolsep=0pt\begin{tabular}{c}Овладение чтением/ \\ письмом 
(дошкольное\\ образование);\\
$\sim4{,}2$--6,9~лет \end{tabular}\\
\hline
5&\tabcolsep=0pt\begin{tabular}{c}$\sim3370$~км;\\
$\sim5$~мкм
\end{tabular}&\tabcolsep=0pt\begin{tabular}{l}$\sim1446$~г.;\\
$\sim1806$~г.\end{tabular}&
\tabcolsep=0pt\begin{tabular}{c}Homo\\ sapiens$^{\prime\prime\prime}$\end{tabular} &
\tabcolsep=0pt\begin{tabular}{c}Субкомпартменты\\ эвкариотической 
клетки\end{tabular}&
\tabcolsep=0pt\begin{tabular}{c}Тиражирование\\ текстов,\\ или книгопечатание\end{tabular}&
\tabcolsep=0pt\begin{tabular}{c}Начальное\\ образование;\\ 
$\sim6{,}9$--11,1~лет \end{tabular}\\
\hline
6&\tabcolsep=0pt\begin{tabular}{c} $\sim51$~тыс.\ км\\ (общепланетарный);\\
$\sim0{,}35$~мкм\end{tabular}&\tabcolsep=0pt\begin{tabular}{l} $\sim1946$~г.;\\
$\sim 1970$~г.\end{tabular}&\tabcolsep=0pt\begin{tabular}{c}Homo \\
sapiens$^{\prime\prime\prime\prime}$\end{tabular}&
\tabcolsep=0pt\begin{tabular}{c}Ультраструктурные\\ (прокариотические)\\ 
внутриклеточные элементы\\ эвкариотической клетки\\ (типа клеточного ядра,\\ деталей 
эндоплазматической\\ сети и~т.\,п.\ образований)\end{tabular}&
Компьютерная ИТ&\tabcolsep=0pt\begin{tabular}{c}Среднее\\ образование;\\
$\sim11{,}1$--18~лет \end{tabular}\\
\hline
7&\tabcolsep=0pt\begin{tabular}{c} $\sim773$~тыс.\ км\\ (ближний\\ космос);\\
$\sim23$~нм\end{tabular}&\tabcolsep=0pt\begin{tabular}{l} $\sim1979$~г.;\\
$\sim2003$~г.\end{tabular}&\tabcolsep=0pt\begin{tabular}{c}Homo\\
 sapiens$^{\prime\prime\prime\prime\prime}$\end{tabular} 
&
\tabcolsep=0pt\begin{tabular}{c}Макромолекулы/гены\\ (компартменты\\ 
ультраструктурных--\\
прокариотических--\\
внутриклеточных\\ элементов)\end{tabular}&
\tabcolsep=0pt\begin{tabular}{c}Телекоммуника-\\ ционная ИТ\end{tabular}&\tabcolsep=0pt\begin{tabular}{c}Высшее обра-\\
зование\;+\;<<аспи-\\ рантура>>; \\
$\sim18$--29,1~лет \end{tabular}\\
\hline
8&\tabcolsep=0pt\begin{tabular}{c}
$\sim11{,}7$~млн км\\ (промежуточный\\ космос);\\
$\sim1{,}5$~нм\end{tabular}&\tabcolsep=0pt\begin{tabular}{l} $\sim1981$~г.;\\ 
$\sim2341$~г.~(?)\end{tabular}&\tabcolsep=0pt\begin{tabular}{c}Homo\\ 
sapiens$^{\prime\prime\prime\prime\prime\prime}$\end{tabular}&
\tabcolsep=0pt\begin{tabular}{c}Органические молекулы \\
(субкомпартменты\\ ультраструктурных--
\\прокариотических--
\\внутриклеточных \\
элементов)\end{tabular}&
\tabcolsep=0pt\begin{tabular}{c}Нано-ИТ (возможно,\\
 <<наноаппаратно\\ поддерживаемая\\ селективная\\ телепатия>>~\cite{16-grn})\end{tabular}&
 \tabcolsep=0pt\begin{tabular}{c}<<Докторантура>>; \\ 
$\sim29{,}1$--47,1~лет \end{tabular}\\
\hline
9&$\cdots$&\multicolumn{1}{c|}{$\cdots$}&$\cdots$&$\cdots$&$\cdots$&$\cdots$\\
\hline
\end{tabular}
\end{center}
\end{table*}




  Промежутки времени между возникновением новых ие\-рар\-хо-се\-те\-вых 
подсистем Человечества (а~следовательно, и~между стартами новых ИТ) 
подчиняются, согласно ИКМ, простой математической за\-ко\-но\-мер\-ности: 
каж\-дый из них в~$e^e\hm= 15{,}15426$\ldots раз короче 
предыдущего\footnote{Эту геометрическую прогрессию~--- как модель критических 
уровней развития биологических сис\-тем~--- выявили А.\,В.~Жирмунский 
и~В.\,И.~Кузьмин~\cite{17-grn}.} (третий\linebreak
 столбец таблицы). В~свою очередь, этой 
же закономерности подчиняются и~размеры ареалов\linebreak
 (радиусы кругов той же 
площади) устойчивых и~эффективно са\-мо\-управ\-ля\-ющих\-ся сообществ 
человека как базисного элемента сис\-те\-мы Человечества, и~точ\-ности 
доступных услож\-ня\-юще\-му\-ся человеку~--- в~конкретный момент 
исторического времени~--- антропогенных воздействий и/или 
производственных технологий (второй столбец таб\-ли\-цы) (рис.~2).
  
  Эмпирические оценки этих времен и~пространств, сделанные 
и~опуб\-ли\-ко\-ван\-ные палео\-ант\-ро\-по\-ло\-га\-ми, археологами и~историками,~--- 
когда они имеются!~--- не противоречат модельным  
результатам~\cite{14-grn}.
  %
Диапазоны примерного возраста <<образовываемых>>, приведенные 
в~седьмом столб\-це таб\-ли\-цы, рассчитаны, исходя из <<золотого сечения>> 
(соотношения смеж\-ных членов чис\-ло\-во\-го ряда, равного 1,618$\ldots$ при 
увеличении ряда, либо 0,618$\ldots$ при его уменьшении, аде\-кват\-ность 
использования которого при выработке количественных оценок в~самых 
различных областях знания хорошо известна\footnote{Применительно 
к~периодизации истории Человечества в~археологическую эпоху это продемонстрировано 
Ю.\,Л.~Щаповой~\cite{18-grn, 19-grn, 20-grn}, согласование подхода к~такой периодизации на 
основе золотого сечения и~пред\-ла\-га\-емо\-го информатико-ки\-бер\-не\-ти\-че\-ско\-го подхода 
подробно показано в~\cite{10-grn, 12-grn, 13-grn, 14-grn, 15-grn, 21-grn}.}), 
опирающегося на ориентировочную оценку завершения человеком среднего 
образования к~18~годам (на сегодня).


  Базируясь на ИКМ, в~качестве нулевого этапа развития будущего 
информационного общества, как пред\-став\-ля\-ет\-ся, можно рас\-смат\-ри\-вать 
процесс \textit{цефализации} позвоночных, т.\,е.\ возникновения 
и~усложнения у~них головного мозга как основного носителя механизмов 
запоминания и~считывания информации о~результатах их адаптивного 
и~социального поведения, начавшейся около 428~млн лет назад 
с~кульминацией около 140,1~млн лет назад (шестой стол\-бец таб\-ли\-цы) на 
<<территории>> порядка 4,2~м~--- т.\,е.\ в~пределах отдельного 
многоклеточного организма.
  

  
  Далее в~качестве первого этапа такого развития будем рассматривать 
начавшуюся около 28,23~млн лет назад, с~кульминацией около 9,26~млн лет 
назад, на территориях порядка 64~м, ИТ сигнальных поз/дви\-же\-ний 
и~неинтонированных звуков (типа рычания, ворчания, писка и~т.\,п.), 
характерную для стад\-ных/стай\-ных животных, в~том числе  
пред-пред-людей {Hominoidea} (четвертый стол\-бец таб\-ли\-цы), 
способных обеспечивать точность своих воздействий на природу порядка~28~см. 
Субстрат их психики относится к~иерархическому уровню органов 
многоклеточного организма (пятый стол\-бец), а~уровень об\-ра\-зо\-ван\-ности 
соответствует современному младенцу возрастом около~1--1,6~лет (седьмой 
столбец).
  
  Следующий, второй этап развития ИТ~--- ми\-ми\-ки/жес\-тов, начавшийся 
около~1,86~млн лет назад, с~кульминацией около~612~тыс.\ лет назад, на 
территориях порядка~1~км, реализовался далекими\linebreak предками современного 
человека~--- пред-людь\-ми {Homo ergaster/Homo erectus}, способными 
обеспечивать точ\-ность своих воздействий на природу\linebreak порядка~1,8~см, 
с~субстратом психики уров\-ня тканей многоклеточного организма и~уровнем 
обра\-зо\-ван\-ности, соответствующим современному ребенку~1,6--2,6~лет.

\pagebreak

\end{multicols}

\setcounter{figure}{1}
\begin{figure*} %fig2
 \vspace*{1pt}
    \begin{center}  
  \mbox{%
 \epsfxsize=163.101mm 
 \epsfbox{gri-2.eps}
 }
\end{center}
\vspace*{-6pt}
\Caption{Пространственно-временн$\acute{\mbox{ы}}$е характеристики и~тренд ИТ в~процессе генезиса 
информационного общества (по ИКМ, в~двойном логарифмическом масштабе; 
иерархическая слож\-ность~--- число уров\-ней/яру\-сов в~системной иерархии)}
\vspace*{1pt}
\end{figure*}

\begin{multicols}{2}



  
  Все последующие этапы развития ИТ~--- речь/язык, пись\-мен\-ность, 
тиражирование текстов (книгопечатание), компьютеры, телекоммуникации, 
на\-но-ИТ~--- реализовались последовательно усложняющимися формами 
{Homo sapiens}, который при этом образовывал относительно 
устойчивые и~относительно эффективно функционирующие 
и~самоуправляющиеся сообщества на все больших ареалах, одновременно 
повышая точность своих (антропогенных) действий при формировании 
вокруг себя <<второй (рукотворной) природы>>.
  
  Так, третий этап развития ИТ~--- речи/языка, начавшийся около 123~тыс.\ 
лет назад, с~кульминацией (верхнепалеолитической революцией) 
около~40~тыс.\ лет назад, на территориях порядка~15~км, реализовался 
{Homo sapiens}$^\prime$, способными обеспечивать точность своих 
производственных технологий порядка~1,2~мм, с~субстратом психики 
уровня эвкариотических клеток многоклеточного организма и~уровнем 
образованности, соответствующим современному ребенку~2,6--4,2~лет.

\begin{figure*}[b] %fig3
%\vspace*{-4pt}
    \begin{center}  
  \mbox{%
 \epsfxsize=162.821mm 
 \epsfbox{gri-3.eps}
 }
\end{center}
\vspace*{-6pt}
\Caption{Тренд изменения времен запаздывания кульминаций развития под\-сис\-тем  
иерар\-хо-се\-те\-вой сис\-те\-мы Человечества относительно их стартов (по ИКМ, в~двойном 
логарифмическом мас\-штабе)}
\end{figure*}
  
  Четвертый этап развития ИТ~--- письменности, начавшийся 
около~8,1~тыс.\ лет назад, с~кульминацией (городской революцией 
<<осевого времени>>) около 2,7~тыс.\ лет назад, на территориях 
порядка~222~км, реализовался {Homo sapiens}$^{\prime\prime}$, 
способными обеспечивать точность своих производственных технологий 
порядка~80~мкм, с~суб\-стра\-том психики уровня компартментов 
эвкариотических клеток многоклеточного организма и~уровнем 
образованности, соответствующим современному ребенку~4,2--6,9~лет 
(дошкольное образование).
  
  Пятый этап развития ИТ~--- тиражирования\linebreak текс\-тов (книгопечатания), 
начавшийся около 1446~г.\ н.\,э., с~кульминацией (промышленной\linebreak 
революцией) около 1806~г., на территориях порядка~3370~км, реализовался 
{Homo sapiens}$^{\prime\prime\prime}$, способными обеспечивать 
точность своих производственных технологий порядка~5~мкм, с~субстратом 
психики уровня субкомпартментов эвкариотических клеток многоклеточного 
организма и~уровнем об\-ра\-зо\-ван\-ности, соответствующим современному 
ребенку~6,9--11,1~лет (начальное образование).
  
  Шестой этап развития ИТ~--- компьютеров (локальных), начавшийся 
около~1946~г., с~кульминацией (изобретением микропроцессоров) 
около~1970~г., на территориях порядка~51~тыс.\ км (т.\,е.\ 
общепланетарного, или глобального размера), реализовался {Homo 
sapiens}$^{\prime\prime\prime\prime}$, способными обеспечивать точ\-ность 
своих производственных технологий порядка~0,35~мкм, с~субстратом 
психики уровня\linebreak
 ультраструктурных (прокариотических) внутриклеточных 
элементов эвкариотической клетки и~уровнем об\-ра\-зо\-ван\-ности, 
соответствующим современному  
под\-рост\-ку-юно\-ше/де\-вуш\-ке~11,1--18~лет\linebreak (среднее образование).
  
  Седьмой этап развития ИТ~--- телекоммуникаций, начавшийся около 
1979~г., с~кульминацией (пиком ско\-рости распространения на планете 
мобильной телефонии, интернета и~т.\,п.) около\linebreak
 2003~г., в~космическом 
объеме радиусом (шара)\linebreak порядка 773~тыс.\ км (т.\,е.\ в~ближнем космосе), 
реализовался {Homo sapiens}$^{\prime\prime\prime\prime\prime}$, 
способными обеспечивать точ\-ность своих производственных технологий 
порядка~23~нм, с~субстратом психики уровня мак\-ро\-мо\-ле\-кул/ге\-нов 
(компартментов\ ульт\-ра\-струк\-тур\-ных--про\-ка\-рио\-ти\-че\-ских--\linebreak
внут\-ри\-кле\-точ\-ных 
элементов эвкариотической клетки) и~уровнем 
об\-ра\-зо\-ван\-ности, со\-от\-вет\-ст\-ву\-ющим современному молодому  
че\-ло\-ве\-ку~18--29,1~лет (высшее обра\-зо\-ва\-ние\;+\;<<ас\-пи\-ран\-ту\-ра, 
с~защитой диссертации кандидата наук>>).
  
  Восьмой этап развития перспективной нано-ИТ (возможно, <<ИТ 
наноаппаратно поддерживаемой селективной телепатии>>~\cite{16-grn}), 
начавшийся около~1981~г., с~кульминацией (пиком скорости ее 
распространения на планете) около~2341~г.\ (расчетная дата), в~космическом 
объеме радиусом шара порядка~11,7~млн км (т.\,е.\ в~промежуточном 
космосе~\cite{5-grn}), реализовался {Homo 
sapiens}$^{\prime\prime\prime\prime\prime\prime }$, способными обеспечивать 
точность своих производственных технологий порядка~1,5~нм (отсюда 
наименование ИТ), с~субстратом психики уровня органических молекул 
(субкомпартментов ульт\-ра\-струк\-тур\-ных--про\-ка\-риоти\-че\-ских--внут\-ри\-кле\-точ\-ных 
элементов эвкариотической клетки) и~уровнем 
об\-ра\-зо\-ван\-ности,\linebreak соответству\-ющим современному зрелому  
человеку~29,1--47,1~лет (<<докторантура>>).
  
  Важно отметить, что процесс появления всех вышеперечисленных 
подсистем подчиняется кумулятивному принципу: возникновение каждой 
новой подсистемы не отменяет существование предыду\-щей: они все активно 
взаимодействуют между собой, коэволюционируют и~т.\,п., но исторически 
более ранние, естественно, постепенно переходят на второй, третий и~т.\,д.\ 
планы исторической сцены.
  
  Точка сходимости этого ряда находится около\linebreak 1981~г., знаменуя собой 
завершение этапа <<детст\-ва--от\-ро\-че\-ст\-ва--юности>> Человечества как 
целого и~начало этапа его <<зрелости>>~--- до\-сти\-же\-ния его максималь\-ной 
иерархической слож\-ности (чис\-ла уров\-ней/яру\-сов в~сис\-тем\-ной 
иерархии)~\cite{5-grn, 7-grn}.
  
  С позиции прогнозирования генезиса информационного общества на 
будущие времена отмечу, что, согласно ИКМ, тренд изменения времен 
запаздывания кульминаций развития под\-сис\-тем относительно их стартов 
сменился прямо на наших глазах. Если во временн$\acute{\mbox{о}}$м диапазоне с~428~млн 
лет назад и~до 1946~г.\ он со\-стоял в~равномерном (в~логарифмическом 
масштабе) укорочении согласно той же за\-ко\-но\-мер\-ности 
(в~$e\hm=15,15426\ldots$~раз), то в~диапазоне от~1946 по 1979~гг.\ это время 
запаздывания не изменилось, а~начиная с~1979~г.\ начало удлиняться 
(рис.~3). 
  

  
  Таким образом, метаэволюция сис\-те\-мы Человечества завершилась около 
1981~г.\ в~том смыс\-ле, что все воз\-мож\-ные ее ие\-рар\-хо-се\-те\-вые под\-сис\-те\-мы 
\textit{в~потенции} уже созданы. Но их \textit{актуализация}, дальнейшее 
услож\-не\-ние, эволюция и~коэволюция с~ранее возникшими аналогичными 
под\-сис\-те\-ма\-ми будет продолжаться неопределенно длительное время.

\vspace*{-10pt}
  
  \section*{Выводы}
  
  \vspace*{-2pt}
  
  \noindent
  \begin{enumerate}[1.]
\item  Изучение \textit{генезиса информационного общества} во всех его 
последовательных формах~--- от древности до современности и~далее~--- на 
базе\linebreak
 ин\-фор\-ма\-ти\-ко-ки\-бер\-не\-ти\-че\-ско\-го модельного подхода 
и~формализации процесса метаэволю\-ционного развития в~соответствующих 
терминах, позволило получить количественные\linebreak оценки его типовых  
про\-стран\-ст\-вен\-но-вре\-менн$\acute{\mbox{ы}}$х характеристик, 
а~также скоординированных с~ними во времени и~в~пространстве  
психико-ант\-ро\-по\-ло\-ги\-че\-ских, образовательных %\linebreak  
и~ин\-фор\-ма\-ци\-он\-но-ком\-му\-ни\-ка\-ци\-он\-ных параметров 
и~возможностей включенного в~этот процесс усложняющегося человека 
и~его сообществ различной величины.
  \item  Позиционирование ИТ локальных компьютеров и~ИТ 
телекоммуникаций в~качестве неотъемлемых составляющих совокупности\linebreak 
монотонно усложняющихся в~ходе цивилизационного развития~--- 
и~информационного общества!~--- ИТ позволяет 
рассматривать их появление и~функционирование в~широком контексте 
единой исторической ретроспективы и~перспективы, давая возможность 
делать не только теоретические, но и~практические выводы.
  \end{enumerate}
  
{\small\frenchspacing
 {%\baselineskip=10.8pt
 \addcontentsline{toc}{section}{References}
 \begin{thebibliography}{99}
\bibitem{1-grn}
\Au{Мелик-Гайгазян И.\,В.} Информационное общество~// Большая российская 
энциклопедия. Т.~11.~--- М.: Большая Российская энциклопедия, 2008. С.~490.
\bibitem{2-grn}
\Au{Гринченко С.\,Н.} Социальная метаэволюция Человечества как последовательность 
шагов формирования механизмов его системной памяти~// Исследовано в~России: 
Электронный журнал, 2001. Т.~145. С.~1652--1681. {\sf  
https://cyberleninka.ru/article/v/sotsialnaya-metaevolyutsiya-chelovechestva-kak-posledovatelnost-shagov-formirovaniya-mehanizmov-ego-sistemnoy-pamyati}.
\bibitem{3-grn}
\Au{Гринченко С.\,Н.} Системная память живого (как основа его метаэволюции
и~периодической структуры).~--- М.: ИПИ РАН, Мир, 2004. 512~с.
\bibitem{4-grn}
\Au{Grinchenko S.\,N.} Meta-evolution of nature system~--- the framework of history~// Social 
Evolution History, 2006. Vol.~5. No.\,1. P.~42--88.
\bibitem{5-grn}
\Au{Гринченко С.\,Н.} Метаэволюция (сис\-тем неживой, живой  
и~со\-ци\-аль\-но-тех\-но\-ло\-ги\-че\-ской природы).~--- М.: ИПИ РАН, 2007. 456~с.
\bibitem{6-grn}
\Au{Гринченко С.\,Н.} Homo eruditus (человек образованный) как элемент сис\-те\-мы 
Человечества~// Открытое образование, 2009. №\,2. С.~48--55.

\bibitem{10-grn} %7
\Au{Гринченко С.\,Н., Щапова~Ю.\,Л.} История Человечества: модели периодизации~// 
Вестник РАН, 2010. №\,12. С.~1076--1084.

%\bibitem{11-grn}  %8
%\Au{Grinchenko S.\,N., Shchapova~Y.\,L.} Human history periodization models~// Herald of the 
%Russian Academy of Sciences, 2010. Vol.~80. No.\,6. P.~498--506.
\bibitem{7-grn} %9
\Au{Grinchenko S.\,N.} The pre- and post-history of Humankind: What is it?~// Problems of 
contemporary world futurology.~--- Newcastle-upon-Tyne: Cambridge Scholars Publishing, 
2011. P.~341--353.
\bibitem{8-grn} %10
\Au{Гринченко С.\,Н.} Об эволюции психики как иерархической сис\-те\-мы 
(кибернетическое пред\-став\-ле\-ние)~// Историческая психология и~социология истории, 
2012. Т.~5. №\,2. С.~60--76.

\bibitem{12-grn} %11
\Au{Гринченко С.\,Н., Щапова~Ю.\,Л.} Информационные технологии в~истории 
Человечества.~--- М.: Новые технологии, 2013. 32~с. (Приложение к~журналу 
<<Информационные технологии>>, 2013. №\,8.)

\bibitem{9-grn} %12
\Au{Гринченко С.\,Н.} Эволюция темпов жизни людей и~развитие человечества~// Человек, 
2014. №\,5. С.~28--36.



\bibitem{13-grn}
\Au{Grinchenko S.\,N., Shchapova~Y.\,L.} Archaeological epoch as the succession of generations 
of evolutive subject-carrier archaeological sub-epoch~// Philosophy of Nature in Cross-Cultural 
Dimensions: The Result of the International Symposium at the University of Vienna~/ 
Komparative Philosophie und Interdisziplin$\ddot{\mbox{a}}$re Bildung (KoPhil). Band~5.~--- 
Hamburg: Verlag Dr.\ Kova$\Check{\mbox{c}}$, 2017. P.~478--499.
\bibitem{14-grn}
\Au{Щапова Ю.\,Л., Гринченко~С.\,Н.} Введение в~теорию археологической эпохи: 
числовое моделирование и~логарифмические шкалы про\-стран\-ст\-вен\-но-вре\-мен\-ных 
координат.~--- М.: Истфак МГУ, ФИЦ ИУ РАН, 2017. 236~с.
\bibitem{15-grn}
\Au{Grinchenko S.\,N., Shchapova~Yu.\,L.} Communications: Model representations about 
historical retrospective and possible perspective~// Communications Media 
Design Electronic~J., 2018. Vol.~3. No.\,2. P.~65--78.
\bibitem{16-grn}
\Au{Гринченко С.\,Н.} Послесловие~// Мат-лы доклада на Совместном научном семинаре 
ИПИ РАН и~\mbox{ИНИОН} РАН <<Методологические проблемы наук об информации>>.~---
М., 2012. С.~5--8. {\sf 
http://legacy.\linebreak inion.ru/files/File/MPNI\_9\_13\_12\_12\_posl.pdf}.
\bibitem{17-grn}
\Au{Жирмунский А.\,В., Кузьмин~В.\,И.} Критические уровни в~процессах развития 
биологических систем.~--- М.: Наука, 1982. 179~с.
\bibitem{18-grn}
\Au{Щапова Ю.\,Л.} Хронология и~периодизации древнейшей истории как числовая 
последовательность (ряд Фибоначчи)~// Информационный бюллетень Ассоциации 
<<История и~компьютер>>, 2000. №\,25.
\bibitem{19-grn}
\Au{Щапова Ю.\,Л.} Археологическая эпоха: хронология, периодизация, теория,  
модель.~--- М.: КомКнига, 2005. 192~с.
\bibitem{20-grn}
\Au{Щапова Ю.\,Л.} Материальное производство в~археологическую эпоху.~--- СПб.: 
Алетейя, 2011. 244~с.
\bibitem{21-grn}
\Au{Гринченко С.\,Н., Щапова~Ю.\,Л.} Пространство и~время в~археологии. Часть~3. 
О~метрике базисной пространственной структуры человечества в~археологическую 
эпоху~// Пространство и~время, 2014. №\,1(15). С.~78--89.
 \end{thebibliography}

 }
 }

\end{multicols}

\vspace*{-8pt}

\hfill{\small\textit{Поступила в~редакцию 17.10.18}}

\vspace*{6pt}

%\pagebreak

%\newpage

%\vspace*{-29pt}

\hrule

\vspace*{2pt}

\hrule

%\vspace*{-2pt}

\def\tit{ON THE GENESIS OF~THE~INFORMATION SOCIETY: INFORMATICS-CYBERNETIC 
MODEL REPRESENTATION}


\def\titkol{On the genesis of~the~information society: Informatics-cybernetic 
model representation}

\def\aut{S.\,N.~Grinchenko}

\def\autkol{S.\,N.~Grinchenko}

\titel{\tit}{\aut}{\autkol}{\titkol}

\vspace*{-11pt}


\noindent
Institute of Informatics Problems of the Federal Research Center ``Informatics and Control'' of 
the Russian Academy of Sciences, 44-2~Vavilov Str., Moscow 119333, Russian Federation

\def\leftfootline{\small{\textbf{\thepage}
\hfill INFORMATIKA I EE PRIMENENIYA~--- INFORMATICS AND
APPLICATIONS\ \ \ 2019\ \ \ volume~13\ \ \ issue\ 2}
}%
 \def\rightfootline{\small{INFORMATIKA I EE PRIMENENIYA~---
INFORMATICS AND APPLICATIONS\ \ \ 2019\ \ \ volume~13\ \ \ issue\ 2
\hfill \textbf{\thepage}}}

\vspace*{6pt}


  
  \Abste{The concept of the information society genesis is introduced, which is 
viewed from the standpoint of informatics-cybernetic modeling of the development 
of Humankind as a self-controlling hierarchical-networking system. On this basis, 
the author obtained quantitative assessments of its typical spatial-temporal 
characteristics, representing geometric progressions with the denominator ``$e$ to the 
degree~$e$'' (15.15426$\ldots$), as well as coordinated with them in time and space 
of the psychoanthropological, educational, and informational communication 
parameters and possibilities of a person who becomes complicated in this process 
and his communities of various sizes. This allowed us to push the framework of 
the information society for the entire historical and even archaeological epoch of 
such development. The resulting sequence of information technologies ``signal 
poses\,/\,sounds/movements\,--\,mimics/gestures\,--\,speech/language\,--\,writing\,--\,replicating 
texts\,--\,computers\,--\,telecommunications\,--\,information 
nanotechnology\,--\,$\ldots$'' allows us to consider the information society genesis 
in the broad context of a unified historical retrospective and perspective.}
  
  \KWE{information society; information technologies; informatics-cybernetic 
model; self-controlling hierarchical-networking system of Humankind; 
archaeological epoch}
  

\DOI{10.14357/19922264190214}

%\vspace*{-14pt}

%\Ack
%\noindent



%\vspace*{6pt}

  \begin{multicols}{2}

\renewcommand{\bibname}{\protect\rmfamily References}
%\renewcommand{\bibname}{\large\protect\rm References}

{\small\frenchspacing
 {%\baselineskip=10.8pt
 \addcontentsline{toc}{section}{References}
 \begin{thebibliography}{99}

\bibitem{1-grn-1}
\Aue{Melik-Gaygazyan, I.\,V.} 2008. Informatsionnoe ob\-shche\-st\-vo [Information 
society]. \textit{Bol'shaya rossiyskaya entsiklopediya} [Great Russian 
Encyclopedia].  Moscow: Great Russian 
Encyclopedia Publs. 11:490.
\bibitem{2-grn-1}
\Aue{Grinchenko, S.\,N.} 2001. Sotsial'naya me\-ta\-evo\-lyu\-tsiya Chelovechestva kak 
posledovatel'nost' shagov for\-mi\-ro\-va\-niya mekhanizmov ego sistemnoy pamyati 
[Social meta-evolution of Mankind as a~sequence of steps for the formation of the 
mechanisms of its system memory]. \textit{Elektronnyy zhurnal <<Issledovano 
v~Rossii>>} [Electronical J. ``Invstigated in Russia'']. 145:1652--1681. Avalable 
at: {\sf  
https://cyberleninka.ru/article/v/sotsialnaya-metaevolyutsiya-chelovechestva-kak-posledovatelnost-shagov-formirovaniya-mehanizmov-ego-sistemnoy-pamyati} (accessed 
October~5, 2018).
\bibitem{3-grn-1}
\Aue{Grinchenko, S.\,N.} 2004. \textit{Sistemnaya pamyat' zhivogo (kak osnova 
ego metaevolyutsii i~periodicheskoy struktury)} [System memory of the life (as the 
basis of its meta-evolution and periodic structure)]. Moscow: IPIRAN, MIR. 
512~p.
\bibitem{4-grn-1}
\Aue{Grinchenko, S.\,N.} 2006. Meta-evolution of nature system~--- the 
framework of history. \textit{Social Evolution History} 5(1):42--88.
\bibitem{5-grn-1}
\Aue{Grinchenko, S.\,N.} 2007. \textit{Metaevolyutsiya (sistem nezhivoy, zhivoy 
i~sotsial'no-tekhnologicheskoy prirody)} [Meta-evolution (of inanimate, animate, 
and socio-technological nature systems)]. Moscow: IPIRAN. 456~p. 
\bibitem{6-grn-1}
\Aue{Grinchenko, S.\,N.} 2009. Homo eruditus (chelovek obrazovannyy) kak 
element sistemy Chelovechestva [Homo eruditus (educated human) as an element 
of the Humakind's system]. \textit{Otkrytoe obrazovanie} [Open Education]  
2:48--55.

\bibitem{10-grn-1} %7
\Aue{Grinchenko, S.\,N., and Yu.\,I.~Shchapova.} 2010. 
Human history periodization models. \textit{Her. Russ. Acad. Sci.} 80(6):498--506.
%\bibitem{11-grn-1} %8
%\Aue{Grinchenko, S.\,N., and Y.\,I.~Shchapova.}  2010. Human history 
%periodization models. \textit{Herald of the Russian Academy of Sciences} 
%80(6):498--506.

\bibitem{7-grn-1} %9
\Aue{Grinchenko, S.\,N.} 2011.The pre- and post-history of Humankind: What is 
it?  \textit{Problems of contemporary world futurology}. 
 Newcastle-upon-Tyne: Cambridge Scholars 
Publishing.  341--353.
\bibitem{8-grn-1} %10
\Aue{Grinchenko, S.\,N.} 2012. Ob evolyutsii psikhiki kak ie\-rar\-khi\-che\-skoy 
sistemy (kiberneticheskoe predstavlenie) [On the evolution of mind as 
a~hierarchical system (a~cybernetic approach)]. \textit{Istoricheskaya 
psikhologiya i~sotsiologiya istorii} [Historical Psychology \& Sociology of 
History] 6(2):\linebreak 60--77.


\bibitem{12-grn-1} %11
\Aue{Grinchenko, S.\,N., and Y.\,I.~Shchapova.} 2013. \textit{In\-for\-ma\-tsi\-on\-nye 
tekhnologii v~istorii Chelovechestva} [Information technology in the history of 
Humankind]. Moscow: Novye tekhnologii. 32~p. (Prilozhenie k zhurnalu 
<<\textit{Informatsionnye tekhnologii}>> [Supplement to J.~Information Technology] 8.

\bibitem{9-grn-1} %12
\Aue{Grinchenko, S.\,N.} 2014. Evolyutsiya tempov zhizni lyudey i~razvitie 
chelovechestva [The evolution of the pace of human life and human development]. 
\textit{Human Being} 5:28--36.

\bibitem{13-grn-1}
\Aue{Grinchenko, S.\,N., and Y.\,I.~Shchapova.} 2017. Archaeological epoch as 
the succession of generations of evolutive subject-carrier archaeological  
sub-epoch. \textit{Philosophy of Nature in Cross-Cultural Dimensions: The Result of 
the International Symposium at the University of Vienna}~/ Komparative 
Philosophie und Interdisziplin$\ddot{\mbox{a}}$re Bildung (KoPhil), Band~5. 
Hamburg: Verlag Dr.\ Kova$\Check{\mbox{c}}$.  478--499.
\bibitem{14-grn-1}
\Aue{Shchapova, Y.\,L., and S.\,N.~Grinchenko.} 2017. \textit{Vvedenie 
v~teoriyu arkheologicheskoy epokhi: chislovoe modelirovanie i~logarifmicheskie 
shkaly prostranstvenno-vremennykh koordinat} [Introduction to the theory of the 
archaeological epoch: Numerical modeling and logarithmic scales of space--time 
coordinates]. Moscow: Faculty 
of History MSU, FRC CSC RAS]. 236~p. 

\vspace*{1pt}

\bibitem{15-grn-1}
\Aue{Grinchenko, S.\,N., and Y.\,I.~Shchapova}. 2018.  Communications: Model 
representations about historical retrospective and possible perspective. 
\textit{Communications Media Design Electronic~J.}  3(2):65--78. 
Available at: {\sf https://elibrary.ru/item.asp?id=36272286} (accessed October~5, 
2018).

\vspace*{1pt}

\bibitem{16-grn-1}
\Aue{Grinchenko, S.\,N.} 2012. Posleslovie [Afterword]. \textit{Mat-ly doklada 
na Sovmestnom nauchnom seminare IPI \mbox{INION} RAN ``Metodologicheskie 
problemy nauk ob informatsii''}  [Report materials at the Joint Scientific 
Seminar of the Institute of Informatics Problems of the Russian Academy of 
Sciences and the Institute of Scientific Information on Social Sciences of the 
Russian Academy of Sciences ``Methodological problems of information 
sciences''].  Moscow. 5--8.  Available at: {\sf 
http://legacy. inion.ru/files/File/MPNI\_9\_13\_12\_12\_posl.pdf} (accessed 
October~5, 2018).

\vspace*{1pt}

\bibitem{17-grn-1}
\Aue{Zhirmunskiy, A.\,V., and V.\,I.~Kuz'min.} 1982. \textit{Kriticheskie urovni 
v~protsessakh razvitiya biologicheskikh sistem} [Critical levels in the development 
of biological systems]. Moscow: Nauka. 179~p.

\vspace*{1pt}

\bibitem{18-grn-1}
\Aue{Shchapova, Y.\,L.} 2000. Khronologiya i~periodizatsii drev\-ney\-shey istorii 
kak chislovaya posledovatel'nost' (ryad Fibonachchi) [Chronology and 
periodization of ancient history as a numerical sequence (Fibonacci's series)]. 
\textit{Informatsionnyy byulleten' Assotsiatsii ``Istoriya i~komp'yuter''} 
[Newsletter of the Association ``History and Computer'']  25.

\vspace*{1pt}

\bibitem{19-grn-1}
\Aue{Shchapova, Y.\,L.} 2005. \textit{Arkheologicheskaya epokha: khro\-no\-lo\-giya, 
periodizatsiya, teoriya, model'} [Archaeological epoch: Chronology, periodization, 
theory, model]. Moscow: KomKniga, 192~p.

\vspace*{1pt}

\bibitem{20-grn-1}
\Aue{Shchapova, Y.\,L.} 2011. \textit{Material'noe proizvodstvo 
v~arkheologicheskuyu epokhu} [Material production in the archaeological epoch]. 
St.\ Petersburg: Aleteyya. 244~p.

\vspace*{1pt}

\bibitem{21-grn-1}
\Aue{Grinchenko, S.\,N., and Yu.\,I.~Shchapova.} 2014. Prostranstvo i~vremya 
v~arheologii. Chast'~3. O~metrike bazisnoy prostranstvennoy struktury 
chelovechestva v~arkheologicheskuyu epokhu [Space and time in archeology. 
Part~3. About the metric of Humankind basic spatial structure  in  
archaeological epoch]. \textit{Space and Time}  
1(15):\linebreak 78--89.
\end{thebibliography}

 }
 }

\end{multicols}

\vspace*{-6pt}

\hfill{\small\textit{Received October 17, 2018}}

%\pagebreak

%\vspace*{-18pt}


  
  \Contrl
  
  \noindent
   \textbf{Grinchenko Sergey N.} (b.\ 1946)~--- Doctor of Science in technology, professor, principal 
scientist, Institute of Informatics Problems, Federal Research Center ``Computer Science and 
Control'' of the Russian Academy of Sciences, 44-2~Vavilov Str., Moscow 119333, Russian 
Federation; \mbox{sgrinchenko@ipiran.ru}
\label{end\stat}

\renewcommand{\bibname}{\protect\rm Литература}   %15


%%%%%%%%%%%%%%%%%%%%%%%%%%%%%%%%%%%%%%%%

%\def\stat{rez}
{%\hrule\par
%\vskip 7pt % 7pt
\raggedleft\Large \bf%\baselineskip=3.2ex
Р\,Е\,Ц\,Е\,Н\,З\,И\,И \vskip 17pt
    \hrule
    \par
\vskip 6pt plus 6pt minus 3pt }

%\thispagestyle{headings} %с верхним колонтитулом
%\thispagestyle{myheadings} %с нижним колонтитулом, но в верхнем РЕЦЕНЗИИ

\def\tit{НОВАЯ КНИГА И.\,Н.~СИНИЦЫНА, А.\,С.~ШАЛАМОВА <<ЛЕКЦИИ ПО ТЕОРИИ 
ИНТЕГРИРОВАННОЙ ЛОГИСТИЧЕСКОЙ ПОДДЕРЖКИ>> (М.: ТОРУС ПРЕСС, 2012. 624~с.)}

%1
\def\aut{Д.ф.-м.н., профессор С.\,Я.~Шоргин}

\def\auf{\ }

\def\leftkol{\ % РЕЦЕНЗИИ
}

\def\rightkol{ \ } 

%\def\leftkol{\ } % ENGLISH ABSTRACTS}

%\def\rightkol{\ } %ENGLISH ABSTRACTS}

%\def\leftkol{РЕЦЕНЗИИ}

%\def\rightkol{РЕЦЕНЗИИ}

\titele{\tit}{\aut}{\auf}{\leftkol}{\rightkol}
\vspace*{-18pt}


     \label{st\stat}

     \begin{multicols}{2}
     {\small
     {\baselineskip=10.1pt
     

      В книге представлено системное изложение теоретических основ одного из новейших 
направлений в \mbox{об\-ласти} экономики послепродажного обслуживания изделий наукоемкой 
продукции (ИНП) длительного пользования~--- интегрированной логистической поддержки
(ИЛП). 
{\looseness=1

}

Приведены также результаты новых работ, выполненных в Институте проблем информатики 
Российской академии наук в рамках научного направления <<Информационные технологии и 
анализ сложных сис\-тем>>.
 {%\looseness=1

}
     
      Излагаемые в книге научные подходы позво\-ляют карди\-наль\-но реформировать 
существующие системы производства и эксплуатации ИНП путем создания и внед\-ре\-ния 
методов рационального и оптимального управ\-ле\-ния процессами расходования 
вре\-мен\-н$\acute{\mbox{ы}}$х, 
мате\-ри\-аль\-ных, трудовых и других ресурсов на всех стадиях жизненного цикла изделий (ЖЦИ) по 
критериям экономической целесообразности и эф\-фек\-тив\-ности.
  {\looseness=1

}
    
      В книге приведен краткий обзор причин возник\-новения и
      развития CALS-методологии как основы 
современных международных стандартов по созданию и функционированию глобальных 
ин\-фор\-ма\-ци\-он\-но-ком\-му\-ни\-ка\-ци\-он\-ных систем, ее ключевых возможностей и эффективности 
результатов ее использования. 
Авторы %\linebreak 
предлагают ряд научных обоснований для разработки 
единой теории проектирования и управления систем ИЛП для полноценного использования 
преимуществ %\linebreak
 суще\-ст\-ву\-ющей методологии, определяют \mbox{общую} структурную схему 
комплексной системы <<ИНП-СППО>> и необходимость разработки для ее описания 
гибридных стохастических моделей.
{%\looseness=1

}

%\columnbreak
      
      Книга состоит из пяти частей, где последовательно излагается материал по каждой из 
следующих тем: <<Интегрированная логистическая поддержка>>, <<Теория гибридных 
стохастических систем и компьютерная поддержка исследований и разработок>>, <<Основы 
математического моделирования, анализа и синтеза систем послепродажного обслуживания>>, 
<<Определение и анализ показателей экспортного потенциала ИНП при проектировании>>, 
<<Задачи управления поддержкой послепродажного обслуживания>>, а также 
<<Моделирование инвестиционных процессов ИЛП в условиях неравновесных финансовых 
рынков>>. 
   
      В конце каждой главы приведены выводы и даны вопросы и задания для 
самоконтроля. В~приложениях содержатся основные определения по программам работ по 
анализу ИЛП, логистическим базам данных и компьютерным решениям, эквивалентной статистической 
линеаризации нелинейных преобразований ИЛП, справочный материал, а также развернутые 
уравнения для вероятностных характеристик.


      \def\leftkol{РЕЦЕНЗИИ}

\def\rightkol{РЕЦЕНЗИИ} 

      
      Книга заинтересует широкий круг специалистов и может быть использована научными 
проектными организациями в сфере промышленного производства ИНП. Большое количество 
иллюстраций, примеров и вопросов, обращенных к читателю, позволяет использовать книгу 
также в качестве учебного пособия для студентов и аспирантов машиностроительных, 
транспортных и~других специальностей, а также для самостоятельного изучения. 
{%\looseness=-1

}

Книга 
представляет несомненный интерес для специалистов и студентов в области прикладной 
математики и информатики.
    

}

}
\end{multicols}

%\newpage

\def\stat{popravka}



\def\tit{ПОПРАВКА К СТАТЬЕ О.\,В.~ШЕСТАКОВА 
<<ПОРОГОВЫЕ ФУНКЦИИ В~МЕТОДАХ ПОДАВЛЕНИЯ ШУМА, ОСНОВАННЫХ~НА~ВЕЙВЛЕТ-РАЗЛОЖЕНИИ СИГНАЛА>>\\
(Информатика и её применения, 2021. Т.\ 15.  Вып.\,3. C.\ 51--56)}



\def\titkol{Поправка к статье О.\,В.~Шестакова\\
<<Пороговые функции в~методах подавления шума, основанных
на~вейвлет-разложении сигнала>>}



  \def\aut{\ }

  \def\autkol{\ } 

\titel{\tit}{\aut}{\autkol}{\titkol}

\def\leftfootline{\small{\textbf{\thepage}
\hfill INFORMATIKA I EE PRIMENENIYA~--- INFORMATICS AND
APPLICATIONS\ \ \ 2021\ \ \ volume~15\ \ \ issue\ 4}
}%
 \def\rightfootline{\small{INFORMATIKA I EE PRIMENENIYA~---
INFORMATICS AND APPLICATIONS\ \ \ 2021\ \ \ volume~15\ \ \ issue\ 4
\hfill \textbf{\thepage}}}


 \label{st\stat}

 \thispagestyle{headings}
 
 \vspace*{-24pt}  

\noindent
{\textbf{DOI:} 10.14357/19922264210307}

\vspace*{20pt}

\def\leftfootline{\small{\textbf{\thepage}
\hfill INFORMATIKA I EE PRIMENENIYA~--- INFORMATICS AND
APPLICATIONS\ \ \ 2021\ \ \ volume~15\ \ \ issue\ 4}
}%
 \def\rightfootline{\small{INFORMATIKA I EE PRIMENENIYA~---
INFORMATICS AND APPLICATIONS\ \ \ 2021\ \ \ volume~15\ \ \ issue\ 4
\hfill \textbf{\thepage}}}


%%%%%%%%%

\medskip

\noindent
С.~55, вместо 

\bigskip

\noindent
{\large ANALYSIS OF THE UNBIASED MEAN-SQUARE RISK ESTIMATE\\[6pt]
 OF~THE~BLOCK THRESHOLDING METHOD}

 



\bigskip

\noindent
должно быть

\bigskip

\noindent
{\large THRESHOLDING FUNCTIONS IN~THE~NOISE SUPPRESSION METHODS\\[6pt] 
BASED ON~THE~WAVELET EXPANSION OF~THE~SIGNAL}

 



 
\vskip 10pt plus 9pt minus 6pt

 \thispagestyle{headings}
 
 %\vspace*{-22pt}
  

\label{end\stat}

\renewcommand{\bibname}{\protect\rm Литература} 


\vspace*{8pt}

\hrule

\vspace*{2pt}

\hrule 

\vspace*{12pt}


\def\stat{popravka-1}



\def\tit{ПОПРАВКА К СТАТЬЕ А.\,А.~КУДРЯВЦЕВА, О.\,В.~ШЕСТАКОВА, С.\,Я.~ШОРГИНА
<<МЕТОД ОЦЕНИВАНИЯ ПАРАМЕТРОВ ИЗГИБА, ФОРМЫ И~МАСШТАБА
ГАММА-ЭКСПОНЕНЦИАЛЬНОГО РАСПРЕДЕЛЕНИЯ>>\\
(Информатика и её применения, 2021. Т.\ 15.  Вып.\,3. C.\ 57--62)}



\def\titkol{Поправка к статье А.\,А.~Кудрявцева, О.\,В.~Шестакова, С.\,Я.~Шоргина
<<Метод оценивания параметров изгиба, формы и масштаба
гамма-экспоненциального распределения>>}



  \def\aut{\ }

  \def\autkol{\ } 

\titel{\tit}{\aut}{\autkol}{\titkol}


 \label{st\stat}

 \thispagestyle{headings}
 
 \vspace*{-24pt}  

\noindent
{\textbf{DOI:} 10.14357/19922264210308}

\vspace*{20pt}




%%%%%%%%%

\medskip

\noindent
С.~61, вместо 

\bigskip

\noindent
{\large PROBABILISTIC CHARACTERISTICS OF~BALANCE INDEX
OF~FACTORS\\[6pt] 
WITH~GENERALIZED GAMMA DISTRIBUTION}



 



\bigskip

\noindent
должно быть

\bigskip

\noindent
{\large A METHOD FOR~ESTIMATING BENT, SHAPE AND~SCALE PARAMETERS\\[6pt] 
OF~THE~GAMMA-EXPONENTIAL DISTRIBUTION} 



 



 
\vskip 10pt plus 9pt minus 6pt

 \thispagestyle{headings}
 
 %\vspace*{-22pt}
  

\label{end\stat}

\renewcommand{\bibname}{\protect\rm Литература}  
%\include{popravka-1}

\def\stat{authorsrus}
{%\hrule\par
%\vskip 7pt % 7pt
\raggedleft\Large \bf%\baselineskip=3.2ex
О\,Б\ \ А\,В\,Т\,О\,Р\,А\,Х \vskip 17pt
    \hrule
    \par
\vskip 21pt plus 8pt minus 4pt }


\def\tit{\ }

\def\aut{\ }

\def\auf{\ }

\def\leftkol{\ } % ENGLISH ABSTRACTS}

\def\rightkol{ОБ АВТОРАХ} %ENGLISH ABSTRACTS}

\titele{\tit}{\aut}{\auf}{\leftkol}{\rightkol}
      
            \label{st\stat}



\vspace*{24pt}

\begin{multicols}{2}




\noindent
\textbf{Архипов Олег Петрович} (р.\ 1948)~---
кандидат технических наук, директор Орловского филиала Института проб\-лем информатики
Российской академии наук
%302025, г.Орел, Московское шоссе, д.137

\vspace*{3pt}

\noindent
\textbf{Бирюкова Татьяна Константиновна} (р.\ 1968)~---
кандидат фи\-зи\-ко-ма\-те\-ма\-ти\-че\-ских наук, старший научный сотрудник Института проб\-лем информатики
Российской академии наук

\vspace*{3pt}

\noindent 
\textbf{Бобков  Сергей Геннадьевич} (р.\ 1955)~---
доктор технических наук,  заведующий отделением На\-уч\-но-ис\-сле\-до\-ва\-тель\-ско\-го 
института системных исследований Российской академии наук
%117218, Москва, Нахимовский просп., 36, к.1 

\vspace*{3pt}

\noindent \textbf{Васильев Николай Семенович} (р.\ 1952)~--- доктор 
фи\-зи\-ко-ма\-те\-ма\-ти\-че\-ских наук, профессор, 
МГТУ им.\ Н.\,Э.~Баумана 
%, Москва 105005, 2-я Бауманская ул., д.~5,

\vspace*{3pt}

\noindent
\textbf{Гершкович Максим Михайлович} (р.\ 1968)~---
старший научный сотрудник Института проб\-лем информатики
Российской академии наук

\vspace*{3pt}

\noindent 
\textbf{Дьяченко Юрий Георгиевич} (р.\ 1958)~--- кандидат технических наук, 
старший научный сотрудник Института проб\-лем информатики
Российской академии наук

\vspace*{3pt}

\noindent 
\textbf{Ерошенко Александр Андреевич} (р.\ 1989)~--- аспирант кафедры 
математической статистики факультета вычисли\-тельной математики и кибернетики 
Московского государственного университета им.\ М.\,В.~Ломоносова
%119991, Москва ГСП-1, Ленинские горы, д.\ 1, стр. 52

\vspace*{3pt}
 
\noindent 
\textbf{Захаров Виктор Николаевич} (р.\ 1948)~--- 
доктор технических наук, доцент, ученый секретарь Института проб\-лем информатики
Российской академии наук

\vspace*{3pt}

\noindent
\textbf{Зейфман Александр Израилевич} (р.\ 1954)~---
доктор фи\-зи\-ко-ма\-те\-ма\-ти\-че\-ских наук, профессор, 
заведующий кафедрой Вологодского государственного университета; 
старший научный сотрудник Института проб\-лем информатики
Российской академии наук; главный научный сотрудник ИСЭРТ Российской академии наук

\vspace*{3pt}

\noindent
\textbf{Зыкин Сергей Владимирович} (р.\ 1959)~--- 
доктор технических наук, профессор, заведующий лабораторией Института математики 
им.\ С.\,Л.~Соболева Сибирского отделения Российской академии наук, Новосибирск 
%630090, пр.\ ак.\ Коптюга, 4 

\vspace*{4pt}

\noindent
\textbf{Киреев Владимир Иванович} (р.\ 1938)~---
доктор фи\-зи\-ко-ма\-те\-ма\-ти\-че\-ских наук, профессор Московского 
государственного горного университета
%Адрес: Россия, 119991, г. Москва, Ленинский проспект, д. 6

%\columnbreak

\vspace*{4pt}

\noindent
\textbf{Козеренко Елена Борисовна} (р.\ 1959)~---
кандидат филологических наук, заведующая лабораторией Института проб\-лем информатики
Российской академии наук

\vspace*{4pt}

\noindent
\textbf{Королев Виктор Юрьевич} (р.\ 1954)~--- доктор
фи\-зи\-ко-ма\-те\-ма\-ти\-че\-ских наук, профессор кафедры математической 
статистики факультета вычисли\-тельной математики и кибернетики 
Московского государственного университета; 
ведущий научный сотрудник Института проб\-лем информатики
Российской академии наук

\vspace*{4pt}

\noindent
\textbf{Коротышева Анна Владимировна} (р.\ 1988)~---
старший преподаватель Вологодского государственного университета

\vspace*{4pt}

\noindent 
\textbf{Кун Де Турк} (р.\ 1981)~--- научный сотрудник 
исследовательской группы SMACS факультета телекоммуникаций и обработки информации
Университета Гента, Бельгия
%В-9000 Гент, Бельгия

\vspace*{4pt}

\noindent
\textbf{Лупенцов Олег Сергеевич} (р.\ 1986)~---
аспирант Омского государственного института сервиса
%Омск 644043, ул.\ Певцова 13

\vspace*{4pt}

\noindent
\textbf{Лучко Олег Николаевич} (р.\ 1961)~---
кандидат педагогических наук, профессор, заведующий кафедрой 
Омского государственного института сервиса
%Омск 644043, ул.\ Певцова 13

\vspace*{4pt}

\noindent
\textbf{Малашенко Юрий Евгеньевич} (р.\ 1946)~---
доктор фи\-зи\-ко-ма\-те\-ма\-ти\-че\-ских наук, заведующий сектором 
Вычислительного центра им.\ А.\,А.~Дородницына Российской академии наук
%Адрес: 119333, Москва, ул. Вавилова, 40,

\vspace*{4pt}

\noindent
\textbf{Маньяков Юрий Анатольевич} (р.\ 1984)~---
кандидат технических наук, научный сотрудник Орловского филиала Института проб\-лем информатики
Российской академии наук
%302025, г.Орел, Московское шоссе, д.137

\vspace*{4pt}

\noindent
\textbf{Маренко Валентина Афанасьевна} (р.\ 1951)~---
кандидат технических наук, доцент, старший научный сотрудник 
Института математики им.\ С.\,Л.~Соболева Сибирского отделения Российской академии наук
%Новосибирск 630090, пр. ак. Коптюга, 4 

\vspace*{3pt}

\noindent 
\textbf{Морозов Евсей Викторович} (р.\ 1947)~--- доктор 
фи\-зи\-ко-ма\-те\-ма\-ти\-че\-ских, профессор, ведущий научный сотрудник 
Института прикладных математических исследований Карельского научного центра Российской
академии наук; 
%%185910 Россия, Республика Карелия, г.\ Петрозаводск, ул.\ Пушкинская, 11
профессор Петрозаводского государственного университета, Петрозаводск
%185910 Россия, Республика Карелия, г.\ Петрозаводск, пр.\ Ленина, 33

%\pagebreak

\vspace*{3pt}

\noindent
\textbf{Назарова Ирина Александровна} (р.\ 1966)~---
кандидат фи\-зи\-ко-ма\-те\-ма\-ти\-че\-ских наук, 
научный сотрудник Вычислительного центра им.\ А.\,А.~Дородницына Российской академии наук 
%Адрес: 119333, Москва, ул. Вавилова, 40

\vspace*{3pt}

\noindent
\textbf{Павлов Игорь Валерианович} (р.\ 1945)~--- 
доктор фи\-зи\-ко-ма\-те\-ма\-ти\-че\-ских наук, профессор МГТУ им.\ Н.\,Э.~Баумана 
%Москва 105005, 2-я Бауманская ул., д.~5 

%\pagebreak

\vspace*{3pt}

\noindent 
\textbf{Потахина Любовь Викторовна} (р.\ 1989)~--- аспирантка
Института прикладных математических исследований Карельского научного центра
Российской академии наук; 
%%185910 Россия, Республика Карелия, г.\ Петрозаводск, ул.\ Пушкинская, 11
инженер Петрозаводского государственного университета, Петрозаводск
%185910 Россия, Республика Карелия, г.\ Петрозаводск, пр.\ Ленина, 33

\vspace*{3pt}

\noindent 
\textbf{Рождественский Юрий Владимирович} (р.\ 1952)~--- 
кандидат технических наук, заведующий сектором Института проб\-лем информатики
Российской академии наук

\vspace*{3pt}

\noindent 
\textbf{Синицын Игорь Николаевич} (р.\ 1940)~--- доктор технических наук,
профессор, заслуженный деятель\linebreak\vspace*{-12pt}

\columnbreak

\noindent
 науки РФ, заведующий отделом Института проб\-лем информатики
Российской академии наук

\vspace*{7pt}


\noindent
\textbf{Сиротинин Денис Олегович} (р.\ 1984)~---
кандидат технических наук, научный сотрудник Орловского филиала Института проб\-лем информатики
Российской академии наук
%302025, г.Орел, Московское шоссе, д.137

\vspace*{7pt}

%\columnbreak

\noindent 
\textbf{Соколов  Игорь Анатольевич} (р.\ 1954)~--- академик (действительный член) Российской 
академии наук, доктор технических наук, директор Института проб\-лем информатики
Российской академии наук

\vspace*{7pt}

\noindent
\textbf{Степченков Юрий Афанасьевич} (р.\ 1951)~---
кандидат технических наук, заведующий отделом Института проб\-лем информатики
Российской академии наук

\vspace*{7pt}

\noindent
\textbf{Сурков Алексей Викторович} (р.\ 1978)~--- 
старший научный сотрудник На\-уч\-но-ис\-сле\-до\-ва\-тель\-ско\-го 
института системных исследований Российской академии наук
%117218, Москва, Нахимовский просп., 36, к.1 

\vspace*{7pt}

\noindent 
\textbf{Шестаков Олег Владимирович} (р.\ 1976)~--- доктор 
фи\-зи\-ко-ма\-те\-ма\-ти\-че\-ских, доцент кафедры математической статистики 
факультета вычисли\-тельной математики и кибернетики Московского 
государственного университета им.\ М.\,В.~Ломоносова; 
%119991, Москва ГСП-1, Ленинские горы, д.\ 1, стр. 52
старший научный сотрудник Института проб\-лем информатики
Российской академии наук
%, Москва 119333, ул. Вавилова, д.~44, корп.~2

\vspace*{7pt}

\noindent 
\textbf{Шоргин Сергей Яковлевич} (р.\ 1952.)~--- доктор
фи\-зи\-ко-ма\-те\-ма\-ти\-че\-ских наук, профессор, заместитель директора Института 
проб\-лем информатики Российской академии наук





%%%%%%%%%%%%%%%%%%%%%%%%%%%%%%%%%%%%%%%%%%%%%%%%%%%%%%%%%%%%%%%%%%%%%%%%%%%%%%%




%\def\rightkol{ОБ АВТОРАХ}
%\def\leftkol{ОБ АВТОРАХ}

 \label{end\stat}





%\def\leftfootline{\small{\textbf{\thepage}
%\hfill ИНФОРМАТИКА И ЕЁ ПРИМЕНЕНИЯ\ \ \ том~7\ \ \ выпуск~1\ \ \ 2013}
%}%
% \def\rightfootline{\small{ИНФОРМАТИКА И ЕЁ ПРИМЕНЕНИЯ\ \ \ том~7\ \ \ выпуск~1\ \ \ 2013
%\hfill \textbf{\thepage}}}


%\thispagestyle{myheadings}



\end{multicols}

\newpage  

%\def\stat{cont}
{%\hrule\par
%\vskip 7pt % 7pt
\raggedleft\Large \bf%\baselineskip=3.2ex
А\,В\,Т\,О\,Р\,С\,К\,И\,Й\ \ У\,К\,А\,З\,А\,Т\,Е\,Л\,Ь\ \ З\,А\ \ 2\,0\,0\,7 г. \vskip 17pt
    \hrule
    \par
\vskip 21pt plus 6pt minus 3pt }

\label{st\stat}

\def\tit{\ }

\def\aut{\ }
\def\auf{\ }

\def\leftkol{\ } % ENGLISH ABSTRACTS}

\def\rightkol{\ } %ENGLISH ABSTRACTS}

\titele{\tit}{\aut}{\auf}{\leftkol}{\rightkol}


\contentsline {chapter}{\ }{Выпуск \quad Стр.} 
\contentsline {section}{\textbf{Батракова Д.\,А., Королев В.\,Ю., Шоргин С.\,Я.}\ \ Новый метод вероятностно-ста\-ти\-сти\-че\-ско\-го анализа информационных потоков в\nobreakspace {}телекоммуникационных сетях}{\qquad 1 \qquad 40} 
\contentsline {section}{\textbf{Борисов А.\,В.}\ \ Байесовское оценивание в системах наблюдения с\nobreakspace {}марковскими скачкообразными процессами: игровой подход}{\qquad 2 \qquad 65}
\contentsline {section}{\textbf{Босов А.\,В., Иванов А.\,В.}\ \ Программная инфраструктура информационного Web-пор\-тала}{\qquad 2 \qquad 50}
\contentsline {section}{\textbf{Захаров В.\,Н., Калиниченко Л.\,А., Соколов И.\,А., Ступников С.\,А.}\ \ Конструирование канонических информационных моделей для интегрированных информационных систем}{\qquad 2 \qquad 15}
\contentsline {section}{\textbf{Захаров В.\,Н., Козмидиади В.\,А.}\ \ Средства обеспечения отказоустойчивости при\-ло\-жений}{\qquad 1 \qquad 14} 
\contentsline {section}{\textbf{Иванов А.\,В.}\ \ см. Босов А.\,В.\hfill\hfill\hfill\hfill\hfill\hfill\hfill\hfill\hfill\hfill\hfill\hfill\hfill\hfill\hfill\hfill\hfill\hfill\hfill\hfill\hfill\hfill\hfill\hfill\hfill\hfill\hfill\hfill\hfill\hfill\hfill\hfill\hfill\hfill\hfill}{\ }
\contentsline {section}{\textbf{Ильин В.\,Д., Соколов И.\,А.}\ \ Символьная модель системы знаний информатики в\nobreakspace {}че\-ло\-ве\-ко-автоматной среде}{\qquad 1 \qquad 66} 
\contentsline {section}{\textbf{Калиниченко Л.\,А.}\ \ см. Захаров В.\,Н.\hfill\hfill\hfill\hfill\hfill\hfill\hfill\hfill\hfill\hfill\hfill\hfill\hfill\hfill\hfill\hfill\hfill\hfill\hfill\hfill\hfill\hfill\hfill\hfill\hfill\hfill\hfill\hfill\hfill\hfill\hfill\hfill\hfill\hfill\hfill}{\ }
\contentsline {section}{\textbf{Козеренко Е.\,Б.}\ \ Лингвистическое моделирование для систем машинного перевода и обработки знаний}{\qquad 1 \qquad 54} 
\contentsline {section}{\textbf{Козмидиади В.\,А.}\ \ см. Захаров В.\,Н.\hfill\hfill\hfill\hfill\hfill\hfill\hfill\hfill\hfill\hfill\hfill\hfill\hfill\hfill\hfill\hfill\hfill\hfill\hfill\hfill\hfill\hfill\hfill\hfill\hfill\hfill\hfill\hfill\hfill\hfill\hfill\hfill\hfill\hfill\hfill }{\ } 
\contentsline {section}{\textbf{Королев В.\,Ю.}\ \ см. Батракова Д.\,А.\hfill\hfill\hfill\hfill\hfill\hfill\hfill\hfill\hfill\hfill\hfill\hfill\hfill\hfill\hfill\hfill\hfill\hfill\hfill\hfill\hfill\hfill\hfill\hfill\hfill\hfill\hfill\hfill\hfill\hfill\hfill\hfill\hfill\hfill\hfill}{\ } 
\contentsline {section}{\textbf{Кудрявцев А.\,А., Шоргин С.\,Я.}\ \ Байесовский подход к\nobreakspace {}анализу систем массового обслуживания и\nobreakspace {}показателей надежности}{\qquad 2 \qquad 76}
\contentsline {section}{\textbf{Печинкин А.\,В., Соколов И.\,А., Чаплыгин В.\,В.}\ \ Многолинейная система массового обслуживания с конечным накопителем и ненадежными приборами}{\qquad 1 \qquad 27} 
\contentsline {section}{\textbf{Печинкин А.\,В., Соколов И.\,А., Чаплыгин В.\,В.}\ \ Стационарные характеристики многолинейной\nobreakspace {}системы массового обслуживания с\nobreakspace {}одновременными отказами приборов}{\qquad 2 \qquad 39}
\contentsline {section}{\textbf{Синицын И.\,Н.}\ \ Корреляционные методы построения аналитических информационных моделей флуктуаций полюса Земли по априорным данным}{\qquad 2 \qquad \hphantom{9}2}
\contentsline {section}{\textbf{Синицын И.\,Н.}\ \ Развитие теории фильтров Пугачева для оперативной обработки информации в стохастических системах}{{\qquad 1 \qquad \hphantom{9}3}} 
\contentsline {section}{\textbf{Соколов И.\,А.}\ \ см. Захаров В.\,Н.\hfill\hfill\hfill\hfill\hfill\hfill\hfill\hfill\hfill\hfill\hfill\hfill\hfill\hfill\hfill\hfill\hfill\hfill\hfill\hfill\hfill\hfill\hfill\hfill\hfill\hfill\hfill\hfill\hfill\hfill\hfill\hfill\hfill\hfill\hfill}{\ }
\contentsline {section}{\textbf{Соколов И.\,А.}\ \ см. Ильин В.\,Д.\hfill\hfill\hfill\hfill\hfill\hfill\hfill\hfill\hfill\hfill\hfill\hfill\hfill\hfill\hfill\hfill\hfill\hfill\hfill\hfill\hfill\hfill\hfill\hfill\hfill\hfill\hfill\hfill\hfill\hfill\hfill\hfill\hfill\hfill\hfill}{\ } 
\contentsline {section}{\textbf{Соколов И.\,А.}\ \ см. Печинкин А.\,В.\hfill\hfill\hfill\hfill\hfill\hfill\hfill\hfill\hfill\hfill\hfill\hfill\hfill\hfill\hfill\hfill\hfill\hfill\hfill\hfill\hfill\hfill\hfill\hfill\hfill\hfill\hfill\hfill\hfill\hfill\hfill\hfill\hfill\hfill\hfill}{\ } 
\contentsline {section}{\textbf{Соколов И.\,А.}\ \ см. Печинкин А.\,В.\hfill\hfill\hfill\hfill\hfill\hfill\hfill\hfill\hfill\hfill\hfill\hfill\hfill\hfill\hfill\hfill\hfill\hfill\hfill\hfill\hfill\hfill\hfill\hfill\hfill\hfill\hfill\hfill\hfill\hfill\hfill\hfill\hfill\hfill\hfill}{\ }
\contentsline {section}{\textbf{Ступников С.\,А.}\ \ см. Захаров В.\,Н.\hfill\hfill\hfill\hfill\hfill\hfill\hfill\hfill\hfill\hfill\hfill\hfill\hfill\hfill\hfill\hfill\hfill\hfill\hfill\hfill\hfill\hfill\hfill\hfill\hfill\hfill\hfill\hfill\hfill\hfill\hfill\hfill\hfill\hfill\hfill}{\ }
\contentsline {section}{\textbf{Чаплыгин В.\,В.}\ \ см. Печинкин А.\,В.\hfill\hfill\hfill\hfill\hfill\hfill\hfill\hfill\hfill\hfill\hfill\hfill\hfill\hfill\hfill\hfill\hfill\hfill\hfill\hfill\hfill\hfill\hfill\hfill\hfill\hfill\hfill\hfill\hfill\hfill\hfill\hfill\hfill\hfill\hfill}{\ } 
\contentsline {section}{\textbf{Чаплыгин В.\,В.}\ \ см. Печинкин А.\,В.\hfill\hfill\hfill\hfill\hfill\hfill\hfill\hfill\hfill\hfill\hfill\hfill\hfill\hfill\hfill\hfill\hfill\hfill\hfill\hfill\hfill\hfill\hfill\hfill\hfill\hfill\hfill\hfill\hfill\hfill\hfill\hfill\hfill\hfill\hfill}{\ }
\contentsline {section}{\textbf{Шоргин С.\,Я.}\ \ см. Батракова Д.\,А.\hfill\hfill\hfill\hfill\hfill\hfill\hfill\hfill\hfill\hfill\hfill\hfill\hfill\hfill\hfill\hfill\hfill\hfill\hfill\hfill\hfill\hfill\hfill\hfill\hfill\hfill\hfill\hfill\hfill\hfill\hfill\hfill\hfill\hfill\hfill}{\ } 
\contentsline {section}{\textbf{Шоргин С.\,Я.}\ \ см. Кудрявцев А.\,А.\hfill\hfill\hfill\hfill\hfill\hfill\hfill\hfill\hfill\hfill\hfill\hfill\hfill\hfill\hfill\hfill\hfill\hfill\hfill\hfill\hfill\hfill\hfill\hfill\hfill\hfill\hfill\hfill\hfill\hfill\hfill\hfill\hfill\hfill\hfill}{\ }
%\thispagestyle{myheadings}
\def\leftfootline{\small{\textbf{\thepage}
\hfill ИНФОРМАТИКА И ЕЁ ПРИМЕНЕНИЯ\ \ \ том~1\ \ \ выпуск~2\ \ \ 2007}
}%
 \def\rightfootline{\small{ИНФОРМАТИКА И ЕЁ ПРИМЕНЕНИЯ\ \ \ том~1\ \ \ выпуск~2\ \ \ 2007
 \hfill \textbf{\thepage}}}
 \label{end\stat} 
                     
%\def\stat{cont-e}
{%\hrule\par
%\vskip 7pt % 7pt
\raggedleft\Large \bf%\baselineskip=3.2ex
2\,0\,0\,7\ \ A\,U\,T\,H\,O\,R\ \ I\,N\,D\,E\,X \vskip 17pt
    \hrule
    \par
\vskip 21pt plus 6pt minus 3pt }

\label{st\stat}

\def\tit{\ }

\def\aut{\ }
\def\auf{\ }

\def\leftkol{\ } % ENGLISH ABSTRACTS}

\def\rightkol{\ } %ENGLISH ABSTRACTS}

\titele{\tit}{\aut}{\auf}{\leftkol}{\rightkol}


\contentsline {chapter}{\ }{Issue \quad Page} 
\contentsline {subsection}{\textbf{Batrakova D.\,A., Korolev V.\,Yu., Shorgin S.\,Ya.}\ \ A New Method for the Probabilistic and Statistical Analysis of Information Flows in Telecommunication Networks}{\qquad 1 \qquad 40} 
\contentsline {subsection}{\textbf{Borisov A.\,V.}\ \ Bayesian Estimation in\nobreakspace {}Observation Systems with\nobreakspace {}Markov Jump Processes: Game-Theoretic Approach}{\qquad 2 \qquad 65} 
\contentsline {subsection}{\textbf{Bosov A.\,V., Ivanov A.\,V.}\ \ Linguistic Simulation for Machine Translation and Knowledge Management Systems}{\qquad 2 \qquad 50} 
\contentsline {subsection}{\textbf{Chaplygin V.\,V.} see Pechinkin A.\,V.\hfill\hfill\hfill\hfill\hfill\hfill\hfill\hfill\hfill\hfill\hfill\hfill\hfill\hfill\hfill\hfill\hfill\hfill\hfill\hfill\hfill\hfill\hfill\hfill\hfill\hfill\hfill\hfill\hfill\hfill\hfill\hfill\hfill\hfill\hfill}{\ }
\contentsline {subsection}{\textbf{Chaplygin V.\,V.} see Pechinkin A.\,V.\hfill\hfill\hfill\hfill\hfill\hfill\hfill\hfill\hfill\hfill\hfill\hfill\hfill\hfill\hfill\hfill\hfill\hfill\hfill\hfill\hfill\hfill\hfill\hfill\hfill\hfill\hfill\hfill\hfill\hfill\hfill\hfill\hfill\hfill\hfill}{\ }
\contentsline {subsection}{\textbf{Ilyin V.\,D., Sokolov I.\,A.}\ \ The Symbol Model of Informatics Knowledge System in Human-Automaton Environment}{\qquad 1 \qquad 66} 
\contentsline {subsection}{\textbf{Ivanov A.\,V.} see Bosov A.\,V.\hfill\hfill\hfill\hfill\hfill\hfill\hfill\hfill\hfill\hfill\hfill\hfill\hfill\hfill\hfill\hfill\hfill\hfill\hfill\hfill\hfill\hfill\hfill\hfill\hfill\hfill\hfill\hfill\hfill\hfill\hfill\hfill\hfill\hfill\hfill}{\ }
\contentsline {subsection}{\textbf{Kalinichenko L.\,A.} see Zakharov V.\,N.\hfill\hfill\hfill\hfill\hfill\hfill\hfill\hfill\hfill\hfill\hfill\hfill\hfill\hfill\hfill\hfill\hfill\hfill\hfill\hfill\hfill\hfill\hfill\hfill\hfill\hfill\hfill\hfill\hfill\hfill\hfill\hfill\hfill\hfill\hfill}{\ }
\contentsline {subsection}{\textbf{Korolev V.\,Yu.} see Batrakova D.\,A.\hfill\hfill\hfill\hfill\hfill\hfill\hfill\hfill\hfill\hfill\hfill\hfill\hfill\hfill\hfill\hfill\hfill\hfill\hfill\hfill\hfill\hfill\hfill\hfill\hfill\hfill\hfill\hfill\hfill\hfill\hfill\hfill\hfill\hfill\hfill}{\ }
\contentsline {subsection}{\textbf{Kozerenko E.\,B.}\ \ Linguistic Simulation for Machine Translation and Knowledge Management Systems}{\qquad 1 \qquad 54} 
\contentsline {subsection}{\textbf{Kozmidiady V.\,A.} see Zakharov V.\,N.\hfill\hfill\hfill\hfill\hfill\hfill\hfill\hfill\hfill\hfill\hfill\hfill\hfill\hfill\hfill\hfill\hfill\hfill\hfill\hfill\hfill\hfill\hfill\hfill\hfill\hfill\hfill\hfill\hfill\hfill\hfill\hfill\hfill\hfill\hfill}{\ }
\contentsline {subsection}{\textbf{Kudryavtsev A.\,A., Shorgin S.\,Ya.}\ \ Bayesian Approach to Queueing Systems and Reliability Characteristics}{\qquad 2 \qquad 76} 
\contentsline {subsection}{\textbf{Pechinkin A.\,V., Sokolov I.\,A., Chaplygin V.\,V.}\ \ Multichannel Queuing System with Finite Buffer and Unreliable Servers}{\qquad 1 \qquad 27} 
\contentsline {subsection}{\textbf{Pechinkin A.\,V., Sokolov I.\,A., Chaplygin V.\,V.}\ \ Stationary Characteristics of a Multichannel Queueing System with\nobreakspace {}Simultaneous Refusals of Servers}{\qquad 2 \qquad 39} 
\contentsline {subsection}{\textbf{Shorgin S.\,Ya.} see Batrakova D.\,A.\hfill\hfill\hfill\hfill\hfill\hfill\hfill\hfill\hfill\hfill\hfill\hfill\hfill\hfill\hfill\hfill\hfill\hfill\hfill\hfill\hfill\hfill\hfill\hfill\hfill\hfill\hfill\hfill\hfill\hfill\hfill\hfill\hfill\hfill\hfill}{\ }
\contentsline {subsection}{\textbf{Shorgin S.\,Ya.} see Kudryavtsev A.\,A.\hfill\hfill\hfill\hfill\hfill\hfill\hfill\hfill\hfill\hfill\hfill\hfill\hfill\hfill\hfill\hfill\hfill\hfill\hfill\hfill\hfill\hfill\hfill\hfill\hfill\hfill\hfill\hfill\hfill\hfill\hfill\hfill\hfill\hfill\hfill}{\ }
\contentsline {subsection}{\textbf{Sinitsyn I.\,N.}\ \ Correlational Methods for Analytical Informational Models of the Earth Pole Fluctuations Design Based on a priori Data}{\qquad 2 \qquad \hphantom{9}2}
\contentsline {subsection}{\textbf{Sinitsyn I.\,N.}\ \ Development of Pugachev Filtering for Stochastic Systems}{\qquad 1 \qquad \hphantom{9}3}
\contentsline {subsection}{\textbf{Sokolov I.\,A.} see Ilyin V.\,D.\hfill\hfill\hfill\hfill\hfill\hfill\hfill\hfill\hfill\hfill\hfill\hfill\hfill\hfill\hfill\hfill\hfill\hfill\hfill\hfill\hfill\hfill\hfill\hfill\hfill\hfill\hfill\hfill\hfill\hfill\hfill\hfill\hfill\hfill\hfill}{\ }
\contentsline {subsection}{\textbf{Sokolov I.\,A.} see Pechinkin A.\,V.\hfill\hfill\hfill\hfill\hfill\hfill\hfill\hfill\hfill\hfill\hfill\hfill\hfill\hfill\hfill\hfill\hfill\hfill\hfill\hfill\hfill\hfill\hfill\hfill\hfill\hfill\hfill\hfill\hfill\hfill\hfill\hfill\hfill\hfill\hfill}{\ }
\contentsline {subsection}{\textbf{Sokolov I.\,A.} see Pechinkin A.\,V.\hfill\hfill\hfill\hfill\hfill\hfill\hfill\hfill\hfill\hfill\hfill\hfill\hfill\hfill\hfill\hfill\hfill\hfill\hfill\hfill\hfill\hfill\hfill\hfill\hfill\hfill\hfill\hfill\hfill\hfill\hfill\hfill\hfill\hfill\hfill}{\ }
\contentsline {subsection}{\textbf{Sokolov I.\,A.} see Zakharov V.\,N.\hfill\hfill\hfill\hfill\hfill\hfill\hfill\hfill\hfill\hfill\hfill\hfill\hfill\hfill\hfill\hfill\hfill\hfill\hfill\hfill\hfill\hfill\hfill\hfill\hfill\hfill\hfill\hfill\hfill\hfill\hfill\hfill\hfill\hfill\hfill}{\ }
\contentsline {subsection}{\textbf{Stupnikov S.\,A.} see Zakharov V.\,N.\hfill\hfill\hfill\hfill\hfill\hfill\hfill\hfill\hfill\hfill\hfill\hfill\hfill\hfill\hfill\hfill\hfill\hfill\hfill\hfill\hfill\hfill\hfill\hfill\hfill\hfill\hfill\hfill\hfill\hfill\hfill\hfill\hfill\hfill\hfill}{\ }
\contentsline {subsection}{\textbf{Zakharov V.\,N., Kalinichenko L.\,A., Sokolov I.\,A., Stupnikov S.\,A.}\ \ Development of Canonical Information Models for Integrated Information Systems}{\qquad 2 \qquad 15} 
\contentsline {subsection}{\textbf{Zakharov V.\,N., Kozmidiady V.\,A.}\ \ Means Providing Applications Fault Tolerance}{\qquad 1 \qquad 14} 
\def\leftfootline{\small{\textbf{\thepage}
\hfill ИНФОРМАТИКА И ЕЁ ПРИМЕНЕНИЯ\ \ \ том~1\ \ \ выпуск~2\ \ \ 2007}
}%
 \def\rightfootline{\small{ИНФОРМАТИКА И ЕЁ ПРИМЕНЕНИЯ\ \ \ том~1\ \ \ выпуск~2\ \ \ 2007
 \hfill \textbf{\thepage}}}
 \label{end\stat} 


%\end{document}

%
\def\stat{rekl}
%\label{preobr}

%\def\tit{АКАДЕМИК ПУГАЧЁВ  ВЛАДИМИР СЕМЁНОВИЧ\\
%25.03.1911--25.03.1998}


%   \vspace*{-48pt}
%   \begin{center}\LARGE
%Академик Пугачёв  Владимир Семёнович\\ (25.03.1911--25.03.1998)
%   \end{center}

   %\vspace*{2.5mm}

   \begin{center}

{\prgsh\LARGE
ЮБИЛЕИ}

\end{center}
%\hrule

\vspace*{6pt}


   \vspace*{8mm}

   \thispagestyle{empty}


%\def\stat{emel}


\section*{К 70-летию заместителя директора ИПИ РАН,\\ члена редколлегии журнала
<<Информатика и её применения>>\\ доктора технических наук В.\,И.~Будзко}

\vspace*{18pt}




          \begin{multicols}{2}

%            \label{st\stat}

\begin{center}
\vspace*{1pt}
\mbox{%
\epsfxsize=78mm
\epsfbox{bud-1.eps}
}
\end{center}

\vspace*{12pt}

      14 августа 2014~г.\ исполнилось 70~лет за\-мес\-ти\-те\-лю директора ИПИ РАН по
научной работе доктору технических наук Владимиру Игоревичу Будзко.

      Владимир Игоревич Будзко родился в г.~Москве. Высшее образование получил на факультете
элект\-рон\-но-вы\-чис\-ли\-тель\-ных устройств в Московском
ин\-же\-нер\-но-фи\-зи\-че\-ском институте
(МИФИ), который он окончил в 1968~г., после чего был на\-прав\-лен для прохождения
службы в одну из войс\-ко\-вых частей, где прошел путь от инженера до первого заместителя
командира войсковой части.

      С приходом В.\,И.~Будзко в ИПИ РАН (2001~г.)\ в институте
сформировалось новое научное на\-прав\-ле\-ние теоретических исследований~--- <<Постро\-ение
ин\-фор\-ма\-ци\-он\-но-те\-ле\-ком\-му\-ни\-ка\-ци\-он\-ных\linebreak сис\-тем
высокой до\-ступ\-ности>>. В~рамках этого
направления выполнен широкий круг фундаментальных исследований по поиску подходов и
определению принципов построения средств обеспечения доступности, конфиденциальности
и целостности современных крупномасштабных
ин\-фор\-ма\-ци\-он\-но-те\-ле\-ком\-му\-ни\-ка\-ци\-он\-ных
сис\-тем (ИТС). Разработаны основные сис\-тем\-но-тех\-ни\-че\-ские принципы и базовые
архитектурные решения построения перспективных для условий России ИТС с
централизованной обработкой и хранением информации, сочетающих в себе свойства
высокой доступности, отказо- и катастрофоустойчивости, информационной защищенности.
Определены принципы, методы и математические основы рационального построения и
оптимизации средств восстановления функционирования центров обработки данных (ЦОД)
после возникновения отказов и катастроф, передачи и хранения данных, обеспечения
информационной безопасности при достижении минимальной совокупной стоимости
владения такими системами. Результаты нашли практическое воплощение при реализации
проектов в интересах ряда отечественных государственных и негосударственных
организаций, таких как Банк России (БР), Внешторгбанк, ОАО <<ГМК <<Норильский Никель>>,
<<Газпром>>, Минэкономразвития России, Правительство Москвы, а также ряд силовых
ведомств.

      Под руководством В.\,И.~Будзко начиная с 2001~г.\ выполнен комплекс
      на\-уч\-но-ис\-сле\-до\-ва\-тель\-ских и
      опыт\-но-кон\-ст\-рук\-тор\-ских работ (свыше 100~проектов),
направленных на развитие электронной информационной технологии БР.
Разработаны концепции развития ИТС БР сначала до 2008~г., а затем до 2013~г., которые
были приняты в качестве основы проведения технической политики. За реализацию проекта
<<Катастрофоустойчивая тер\-ри\-то\-ри\-аль\-но-рас\-пре\-де\-лен\-ная
      ин\-фор\-ма\-ци\-он\-но-те\-ле\-ком\-му\-ни\-ка\-ци\-он\-ная сис\-те\-ма централизованной
обработки банковской информации>> В.\,И.~Будзко удостоен Премии Правительства РФ в
области науки и техники за 2010~г.

      В.\,И.~Будзко возглавлял и возглавляет работы по ряду других прикладных проектов,
связанных с созданием, совершенствованием и развитием крупномасштабных ИТС.

      В.\,И.~Будзко~--- генерал-майор, доктор технических наук, член-кор\-рес\-пон\-дент
Академии криптографии РФ, известный ученый в области информатики и применения
информационных технологий при построении территориально распределенных ИТС
различного назначения. Является автором свыше 250~научных работ, опубликованных в
на\-уч\-но-тех\-ни\-че\-ских и специальных изданиях.

    \thispagestyle{empty}

      В.\,И.~Будзко уделяет большое внимание подготовке научных кадров. Под его
руководством защищено 6~диссертаций на соискание ученой степени кандидата
технических наук. Свыше 30~лет он читает лекции в ИКСИ Академии ФСБ, профессор
кафедры НИЯУ МИФИ. Является членом двух диссертационных советов, главным
редактором журнала <<Системы высокой доступности>> и членом редколлегии журнала
<<Информатика и её применения>>.

      \bigskip

      Редакционный совет и Редакционная коллегия журнала <<Информатика и её
применения>> сердечно поздравляют Владимира Игоревича Будзко с 70-ле\-ти\-ем и желают
крепкого здоровья и новых научных достижений.

\end{multicols}

%%Информатика и её применения
%Том 14 Выпуск 1-4 Год 2020

\def\stat{cont}
{%\hrule\par
%\vskip 7pt % 7pt
\raggedleft\Large \bf%\baselineskip=3.2ex
А\,В\,Т\,О\,Р\,С\,К\,И\,Й\ \ У\,К\,А\,З\,А\,Т\,Е\,Л\,Ь\ \ З\,А\ \ 2\,0\,2\,0 г. \vskip 17pt
 \hrule
 \par
\vskip 21pt plus 6pt minus 3pt }

\label{st\stat}

\def\tit{\ }

\def\aut{\ }
\def\auf{\ }

\def\leftkol{\ } % ENGLISH ABSTRACTS}

\def\rightkol{\ } %АВТОРСКИЙ УКАЗАТЕЛЬ ЗА 2020 г.} %ENGLISH ABSTRACTS}

\titele{\tit}{\aut}{\auf}{\leftkol}{\rightkol}
\addcontentsline{toc}{subsection}{\textrm\textbf Авторский указатель за 2020 г.}

\vspace*{-24pt}

\noindent
{\tabcolsep=3pt
\begin{tabular}{p{397pt}cc}
&\textbf{Вып.} & \textbf{Стр.}\\[6pt]
\Avtors{Абгарян~К.\,К., Гаврилов~Е.\,С.} Интеграционная платформа для многомасштабного моде-\linebreak
\\[-12pt]
\hspace*{23pt}лирования нейроморфных систем&2&104--110\\
\Avtors{Абгарян~К.\,К., Колбин~И.\,С.} Применение многомасштабного подхода и методов анализа\linebreak
\\[-12pt]
\hspace*{23pt}данных для моделирования теплопроводности в слоистых структурах&4&91--99\\
\Avtors{Агаларов~Я.\,М.} Оптимизация емкости основного накопителя в системе массового\linebreak
\\[-12pt]
\hspace*{23pt}обслуживания типа $G/M/1/K$ с дополнительным накопителем&2&72--79\\
\Avtors{Агасандян~Г.\,А.} Вычислительные аспекты применения CC-VaR на совокупности рынков&3&62--70\\
\Avtors{Агеев~К.\,А., Сопин~Э.\,С., Яркина~Н.\,В., Самуйлов~К.\,Е., Шоргин~С.\,Я.} Анализ механизмов\linebreak
\\[-12pt]
\hspace*{23pt}нарезки сети с учетом гарантий для различных типов трафика&3&\hphantom{1}94--100\\
\Avtors{Адамова~К.\,А.} см.\ Шнурков~П.\,В.&&\\
\Avtors{Базилевский~М.\,П.} Многофакторные модели полносвязной линейной регрессии без\linebreak
\\[-12pt]
\hspace*{23pt}ограничений на соотношения дисперсий ошибок переменных&2&92--97\\
\Avtors{Бахтеев~О.\,Ю.} см.\ Грабовой~А.\,В.&&\\
\Avtors{Беленков~В.\,Г.} см.\ Будзко~В.\,И.&&\\
\Avtors{Бетелин~В.\,Б., Кушниренко~А.\,Г., Леонов~А.\,Г.} Основные понятия программирования\linebreak
\\[-12pt]
\hspace*{23pt}в изложении для дошкольников&3&55--61\\
\Avtors{Бетелин~В.\,Б., Кушниренко~А.\,Г., Семенов~А.\,Л., Сопрунов~С.\,Ф.} О цифровой грамотности\linebreak
\\[-12pt]
\hspace*{23pt}и средах ее формирования&4&100--107\\
\Avtors{Борисов~А.\,В.} Численные схемы фильтрации марковских скачкообразных процессов по\linebreak
\\[-12pt]
\hspace*{23pt}дискретизованным наблюдениям II: случай аддитивных шумов&1&17--23\\
\Avtors{Борисов~А.\,В.} Численные схемы фильтрации марковских скачкообразных процессов по\linebreak
\\[-12pt]
\hspace*{23pt}дискретизованным наблюдениям III: случай мультипликативных шумов&2&10--18\\
\Avtors{Босов~А.\,В.} Управление выходом стохастической дифференциальной системы по квад-\linebreak
\\[-12pt]
\hspace*{23pt}ратичному критерию. V. Случай неполной информации о состоянии&2&19--25\\
\Avtors{Босов~А.\,В., Мартюшова~Я.\,Г., Наумов~А.\,В., Сапунова~А.\,П.} Байесовский подход к~по\-стро\-ению индивидуальной траектории пользователя в~системе дистанционного\linebreak
\\[-12pt]
\hspace*{23pt}обучения&3&86--93\\
\Avtors{Босов~А.\,В., Стефанович~А.\,И.} Управление выходом стохастической дифференциальной\linebreak
\\[-12pt]
\hspace*{23pt}системы по квадратичному критерию. IV. Альтернативное численное решение&1&24--30\\
\Avtors{Брюхов~Д.\,О., Ступников~С.\,А., Ковалёв~Д.\,Ю., Шанин~И.\,А.} Нейрофизиология как\linebreak
\\[-12pt]
\hspace*{23pt}предметная область для решения задач с интенсивным использованием данных&1&40--47\\
\Avtors{Будзко~В.\,И., Ядринцев~В.\,В., Соченков~И.\,В., Королёв~В.\,И., Беленков~В.\,Г.} Об одном подходе
 к формированию в условиях высокой неопределенности марке-\linebreak
\\[-12pt]
\hspace*{23pt}ров конфиденциальности в системах интенсивного использования данных&4&69--76\\
\Avtors{Вайсер~К.\,О.} см.\ Потанин~М.\,С.&&\\
\Avtors{Вохминцев~А.\,В., Мельников~А.\,В., Пачганов~C.\,А.} Метод навигации и составления карты в трехмерном пространстве на основе комбинированного решения вариационной\linebreak
\\[-12pt]
\hspace*{23pt}подзадачи точка--точка ICP для аффинных преобразований&1&101--112\\
\Avtors{Гаврилов~Е.\,С.} см.\ Абгарян~К.\,К.&&\\
\Avtors{Гайдамака~Ю.\,В.} см.\  Москалева~Ф.\,А.&&\\
\Avtors{Голембиовский~Д.\,Ю.} см.\ Данилишин~А.\,Р.&&\\
\Avtors{Голембиовский~Д.\,Ю.} см.\ Данилишин~А.\,Р.&&\\
\Avtors{Гончаров~А.\,А., Зацман~И.\,М., Кружков~М.\,Г.} Эволюция классификаций в надкорпусных\linebreak
\\[-12pt]
\hspace*{23pt}базах данных&4&108--116\\
\Avtors{Гончаров~А.\,В., Стрижов~В.\,В.} Выравнивание декартовых произведений упорядоченных\linebreak
\\[-12pt]
\hspace*{23pt}множеств&1&31--39\\
\end{tabular}
}

\pagebreak

\def\leftkol{АВТОРСКИЙ УКАЗАТЕЛЬ ЗА 2020 г.} % ENGLISH ABSTRACTS}

\def\rightkol{АВТОРСКИЙ УКАЗАТЕЛЬ ЗА 2020 г.} %ENGLISH ABSTRACTS}

%\thispagestyle{myheadings}
\def\leftfootline{\small{\textbf{\thepage}
\hfill ИНФОРМАТИКА И ЕЁ ПРИМЕНЕНИЯ\ \ \ том~14\ \ \ выпуск~4\ \ \ 2020}
}%
 \def\rightfootline{\small{ИНФОРМАТИКА И ЕЁ ПРИМЕНЕНИЯ\ \ \ том~14\ \ \ выпуск~4\ \ \ 2020
 \hfill \textbf{\thepage}}}


\noindent
{\tabcolsep=3pt
\begin{tabular}{p{394pt}cc}
&\textbf{Вып.} & \textbf{Стр.}\\[3pt]
\Avtors{Горшенин~А.\,К., Королев~В.\,Ю.} Аппроксимация распределений размеров частиц лунного\linebreak
\\[-12pt]
\hspace*{23pt}реголита на основе метода статистической симуляции выборок&2&50--57\\
\Avtors{Горшенин~А.\,К., Королев~В.\,Ю., Щербинина~А.\,А.} Статистическое оценивание распределений случайных коэффициентов стохастического дифференциального уравнения\linebreak
\\[-12pt]
\hspace*{23pt}Ланжевена&3&\hphantom{1}3--12\\
\Avtors{Горшенин~А.\,К., Кузьмин~В.\,Ю.} Анализ конфигураций LSTM-сетей для построения\linebreak
\\[-12pt]
\hspace*{23pt}среднесрочных векторных прогнозов&1&10--16\\
\Avtors{Грабовой~А.\,В., Бахтеев~О.\,Ю., Стрижов~В.\,В.} Введение отношения порядка на множестве\linebreak
\\[-12pt]
\hspace*{23pt}параметров аппроксимирующих моделей&2&58--65\\
\Avtors{Грушо~А.\,А., Забежайло~М.\,И., Смирнов~Д.\,В., Тимонина~Е.\,Е.} О вероятностных оценках\linebreak
\\[-12pt]
\hspace*{23pt}достоверности эмпирических выводов&4&3--8\\
\Avtors{Грушо~А.\,А., Забежайло~М.\,И., Смирнов~Д.\,В., Тимонина~Е.\,Е., Шоргин~С.\,Я.} Методы\linebreak
\\[-12pt]
\hspace*{23pt}математической статистики в задаче поиска инсайдера&3&71--75\\
\Avtors{Грушо~А.\,А., Забежайло~М.\,И., Тимонина~Е.\,Е.} О каузальной репрезентативности обуча-\linebreak
\\[-12pt]
\hspace*{23pt}ющих выборок прецедентов в задачах диагностического типа&1&80--86\\
\Avtors{Грушо~А.\,А., Тимонина~Е.\,Е., Грушо~Н.\,А., Терехина~И.\,Ю.} Выявление аномалий с по-\linebreak
\\[-12pt]
\hspace*{23pt}мощью метаданных&3&76--80\\
\Avtors{Грушо~А.\,А.} см.\ Грушо~Н.\,А.&&\\
\Avtors{Грушо~Н.\,А., Грушо~А.\,А., Забежайло~М.\,И., Тимонина~Е.\,Е.} Методы нахождения причин\linebreak
\\[-12pt]
\hspace*{23pt}сбоев в информационных технологиях  с помощью метаданных&2&33--39\\
\Avtors{Грушо~Н.\,А.} см.\ Грушо~А.\,А.&&\\
\Avtors{Данилишин~А.\,Р., Голембиовский~Д.\,Ю.} Оценка стоимости опционов на основе моделей\linebreak
\\[-12pt]
\hspace*{23pt}ARIMA--GARCH с ошибками, распределенными по закону $S_u$ Джонсона&4&83--90\\
\Avtors{Данилишин~А.\,Р., Голембиовский~Д.\,Ю.} Риск-нейтральная динамика для модели ARIMA-\linebreak
\\[-12pt]
\hspace*{23pt}GARCH с ошибками, распределенными по закону $S_U$ Джонсона&1&48--55\\
\Avtors{Диментов~А.\,В.} см.\ Краснов~Ф.\,В.&&\\
\Avtors{Донской~В.\,И.} Извлечение оптимизационных моделей из данных&3&109--118\\
\Avtors{Дубнов~Ю.\,А.} см.\ Попков~Ю.\,С.&&\\
\Avtors{Дулин~С.\,К., Дулина~Н.\,Г., Ермаков~П.\,В.} Информационный синтез документов&1&128--135\\
\Avtors{Дулина~Н.\,Г.} см.\ Дулин~С.\,К.&&\\
\Avtors{Дьяченко~Ю.\,Г.} см.\ Соколов~И.\,А.&&\\
\Avtors{Ермаков~П.\,В.} см.\ Дулин~С.\,К.&&\\
\Avtors{Ефросинин~Д.\,В.} см.\ Харин~П.\,А.&&\\
\Avtors{Жолобов~В.\,А.} см.\ Потанин~М.\,С.&&\\
\Avtors{Забежайло~М.\,И.} см.\ Грушо~А.\,А.&&\\
\Avtors{Забежайло~М.\,И.} см.\ Грушо~А.\,А.&&\\
\Avtors{Забежайло~М.\,И.} см.\ Грушо~А.\,А.&&\\
\Avtors{Забежайло~М.\,И.} см.\ Грушо~Н.\,А.&&\\
\Avtors{Захаров В. Н.} см.\ Френкель С. Л.&&\\
\Avtors{Зацман~И.\,М.} Проблемно-ориентированная верификация полноты темпоральных\linebreak
\\[-12pt]
\hspace*{23pt}онтологий и заполнение понятийных лакун&3&119--128\\
\Avtors{Зацман~И.\,М.} см.\ Гончаров~А.\,А.&&\\
\Avtors{Зацман~И.\,М.} см.\ Нуриев~В.\,А.&&\\
\Avtors{Зейфман~А.\,И.} см.\ Сатин~Я.\,А.&&\\
\Avtors{Кириков~И.\,А.} см.\ Румовская~С.\,Б.&&\\
\Avtors{Кирилюк~И.\,Л., Сенько~О.\,В.} Выбор моделей оптимальной сложности методами Монте-Карло (на примере моделей производственных функций регионов Российской\linebreak
\\[-12pt]
\hspace*{23pt}Федерации)&2&111--118\\
\Avtors{Ковалёв~Д.\,Ю.} см.\ Брюхов~Д.\,О.&&\\
\Avtors{Козеренко~Е.\,Б., Михеев~М.\,Ю., Сомин~Н.\,В., Эрлих~Л.\,И., Кузнецов~К.\,И.} Аналити\-че\-ская
текс\-тология в системах интеллектуальной обработки неструктурированных\linebreak
\\[-12pt]
\hspace*{23pt}данных&1&113--120\\
\Avtors{Колбин~И.\,С.} см.\ Абгарян~К.\,К.&&\\
\end{tabular}
}

\pagebreak

\def\leftkol{АВТОРСКИЙ УКАЗАТЕЛЬ ЗА 2020 г.} % ENGLISH ABSTRACTS}

\def\rightkol{АВТОРСКИЙ УКАЗАТЕЛЬ ЗА 2020 г.} %ENGLISH ABSTRACTS}

%\thispagestyle{myheadings}
\def\leftfootline{\small{\textbf{\thepage}
\hfill ИНФОРМАТИКА И ЕЁ ПРИМЕНЕНИЯ\ \ \ том~14\ \ \ выпуск~4\ \ \ 2020}
}%
 \def\rightfootline{\small{ИНФОРМАТИКА И ЕЁ ПРИМЕНЕНИЯ\ \ \ том~14\ \ \ выпуск~4\ \ \ 2020
 \hfill \textbf{\thepage}}}


\noindent
{\tabcolsep=3pt
\begin{tabular}{p{394pt}cc}
&\textbf{Вып.} & \textbf{Стр.}\\[3pt]
\Avtors{Королев~В.\,Ю.} О распределении отношения суммы элементов выборки, превосходящих\linebreak
\\[-12pt]
\hspace*{23pt}некоторый порог, к сумме всех элементов выборки.~I&3&35--43\\
\Avtors{Королев~В.\,Ю.} О распределении отношения суммы элементов выборки, превосходящих\linebreak
\\[-12pt]
\hspace*{23pt}некоторый порог, к сумме всех элементов выборки.~II&4&33--36\\
\Avtors{Королев~В.\,Ю.} см.\ Горшенин~А.\,К&&\\
\Avtors{Королев~В.\,Ю.} см.\ Горшенин~А.\,К.&&\\
\Avtors{Королёв~В.\,И.} см.\ Будзко~В.\,И.&&\\
\Avtors{Костина~А.\,А., Мирин~А.\,Ю., Молдовян~Д.\,Н., Фахрутдинов~Р.\,Ш.} Метод задания конечных некоммутативных ассоциативных алгебр произвольной четной размерности\linebreak
\\[-12pt]
\hspace*{23pt}для построения постквантовых криптосхем&1&\hphantom{1}94--100\\
\Avtors{Кочеткова~И.\,А.} см.\ Харин~П.\,А.&&\\
\Avtors{Краснов~Ф.\,В., Диментов~А.\,В., Шварцман~М.\,Е.} Использование тематических моделей\linebreak
\\[-12pt]
\hspace*{23pt}для парного сравнения  коллекций научных статей&3&129--135\\
\Avtors{Кривенко~М.\,П.} Последовательный анализ серий данных на основе многомерных ре-\linebreak
\\[-12pt]
\hspace*{23pt}фе\-рен\-с\-ных регионов&2&86--91\\
\Avtors{Кружков~М.\,Г.} см.\ Гончаров~А.\,А.&&\\
\Avtors{Кудрявцев~А.\,А., Шестаков~О.\,В.} Метод логарифмических моментов для оценивания\linebreak
\\[-12pt]
\hspace*{23pt}параметров гамма-экспоненциального распределения&3&49--54\\
\Avtors{Кузнецов~К.\,И.} см.\ Козеренко~Е.\,Б.&&\\
\Avtors{Кузьмин~В.\,Ю.} см.\ Горшенин~А.\,К.&&\\
\Avtors{Кушниренко~А.\,Г.} см.\ Бетелин~В.\,Б.&&\\
\Avtors{Кушниренко~А.\,Г.} см.\ Бетелин~В.\,Б.&&\\
\Avtors{Леонов~А.\,Г.} см.\ Бетелин~В.\,Б.&&\\
\Avtors{Макеева~Е.\,Д.} см.\ Харин~П.\,А.&&\\
\Avtors{Малашенко~Ю.\,Е., Назарова~И.\,А.} Аппроксимация множества достижимых потоков\linebreak
\\[-12pt]
\hspace*{23pt}многопользовательской сети&3&81--85\\
\Avtors{Мартюшова~Я.\,Г.} см.\ Босов~А.\,В.&&\\
\Avtors{Матюшенко~С.\,И., Разумчик~Р.\,В.} Стационарные характеристики системы Geo$/G/1/\infty $\linebreak
\\[-12pt]
\hspace*{23pt}с неординарным входящим потоком, управляющим размером очереди&4&25--32\\
\Avtors{Мейханаджян~Л.\,А., Разумчик~Р.\,В.} Стационарные характеристики системы $M/G/2/\infty$ с одним частным случаем дисциплины инверсионного порядка обслуживания\linebreak
\\[-12pt]
\hspace*{23pt}с обобщенным  вероятностным приоритетом&2&66--71\\
\Avtors{Мельников~А.\,В.} см.\ Вохминцев~А.\,В.&&\\
\Avtors{Мельников~С.\,Ю., Самуйлов~К.\,Е.} Статистические свойства двоичных неавтономных\linebreak
\\[-12pt]
\hspace*{23pt}регистров сдвига  с внутренним суммированием&2&80--85\\
\Avtors{Милованова~Т.\,А., Разумчик~Р.\,В.} Однолинейная система массового обслуживания с инверсионным порядком обслуживания с вероятностным приоритетом, групповым\linebreak
\\[-12pt]
\hspace*{23pt}пуассоновским потоком и фоновыми заявками&3&26--34\\
\Avtors{Мирин~А.\,Ю.} см.\ Костина~А.\,А.&&\\
\Avtors{Михеев~М.\,Ю.} см.\ Козеренко~Е.\,Б.&&\\
\Avtors{Молдовян~Д.\,Н.} см.\ Костина~А.\,А.&&\\
\Avtors{Москалева~Ф.\,А., Гайдамака~Ю.\,В., Шоргин~В.\,С.} Влияние параметров изоляции на\linebreak
\\[-12pt]
\hspace*{23pt}разделение ресурсов при нарезке сети&4&\hphantom{1}9--16\\
\Avtors{Назарова~И.\,А.} см.\ Малашенко~Ю.\,Е.&&\\
\Avtors{Наумов~А.\,В.} см.\ Босов~А.\,В.&&\\
\Avtors{Наумов~В.\,А., Самуйлов~К.\,Е.} О марковских и рациональных потоках случайных со-\linebreak
\\[-12pt]
\hspace*{23pt}бытий.~I&3&13--19\\
\Avtors{Наумов~В.\,А., Самуйлов~К.\,Е.} О марковских и рациональных потоках случайных со-\linebreak
\\[-12pt]
\hspace*{23pt}бытий.~II&4&37--46\\
\Avtors{Новиков~Д.\,А.} см.\ Шнурков~П.\,В.&&\\
\Avtors{Нуриев~В.\,А., Зацман~И.\,М.} Редуцирование спектра моделей перевода в надкорпусных\linebreak
\\[-12pt]
\hspace*{23pt}базах данных&2&119--126\\
\Avtors{Пачганов~C.\,А.} см.\ Вохминцев~А.\,В.&&\\
\end{tabular}
}

\pagebreak

\def\leftkol{АВТОРСКИЙ УКАЗАТЕЛЬ ЗА 2020 г.} % ENGLISH ABSTRACTS}

\def\rightkol{АВТОРСКИЙ УКАЗАТЕЛЬ ЗА 2020 г.} %ENGLISH ABSTRACTS}

%\thispagestyle{myheadings}
\def\leftfootline{\small{\textbf{\thepage}
\hfill ИНФОРМАТИКА И ЕЁ ПРИМЕНЕНИЯ\ \ \ том~14\ \ \ выпуск~4\ \ \ 2020}
}%
 \def\rightfootline{\small{ИНФОРМАТИКА И ЕЁ ПРИМЕНЕНИЯ\ \ \ том~14\ \ \ выпуск~4\ \ \ 2020
 \hfill \textbf{\thepage}}}


\noindent
{\tabcolsep=3pt
\begin{tabular}{p{394pt}cc}
&\textbf{Вып.} & \textbf{Стр.}\\[3pt]
\Avtors{Попков~А.\,Ю.} см.\ Попков~Ю.\,С.&&\\
\Avtors{Попков~Ю.\,С., Попков~А.\,Ю., Дубнов~Ю.\,А.} Методы детерминированных и рандомизи-\linebreak
\\[-12pt]
\hspace*{23pt}рованных энтропийных проекций для редукции размерности матрицы данных&4&47--54\\
\Avtors{Попов~Г.\,А., Симаворян~С.\,Ж., Симонян~А.\,Р., Улитина~Е.\,И.} Моделирование процесса мониторинга систем информационной безопасности на основе систем массового\linebreak
\\[-12pt]
\hspace*{23pt}обслуживания&1&71--79\\
\Avtors{Попов~М.\,В., Посыпкин~М.\,А.} Аппроксимация множества решений систем нелинейных\linebreak
\\[-12pt]
\hspace*{23pt}неравенств с использованием графических ускорителей&3&20--25\\
\Avtors{Посыпкин~М.\,А.} см.\ Попов~М.\,В.&&\\
\Avtors{Потанин~М.\,С., Вайсер~К.\,О., Жолобов~В.\,А., Стрижов~В.\,В.} Оптимизация структуры\linebreak
\\[-12pt]
\hspace*{23pt}сетей глубокого обучения&4&55--62\\
\Avtors{Разумчик~Р.\,В.} см.\ Матюшенко~С.\,И.&&\\
\Avtors{Разумчик~Р.\,В.} см.\ Мейханаджян~Л.\,А.&&\\
\Avtors{Разумчик~Р.\,В.} см.\ Милованова~Т.\,А.&&\\
\Avtors{Рождественский~Ю.\,В.} см.\ Соколов~И.\,А.&&\\
\Avtors{Румовская~С.\,Б., Кириков~И.\,А.} Метод визуального представления конфликтов в гибрид-\linebreak
\\[-12pt]
\hspace*{23pt}ных интеллектуальных многоагентных системах&4&77--82\\
\Avtors{Самуйлов~К.\,Е.} см.\ Агеев~К.\,А.&&\\
\Avtors{Самуйлов~К.\,Е.} см.\ Мельников~С.\,Ю.&&\\
\Avtors{Самуйлов~К.\,Е.} см.\ Наумов~В.\,А.&&\\
\Avtors{Самуйлов~К.\,Е.} см.\ Наумов~В.\,А.&&\\
\Avtors{Сапунова~А.\,П.} см.\ Босов~А.\,В.&&\\
\Avtors{Сатин~Я.\,А., Зейфман~А.\,И., Шилова~Г.\,Н.} О подходах к построению предельных режимов\linebreak
\\[-12pt]
\hspace*{23pt}для некоторых моделей массового обслуживания&2&3--9\\
\Avtors{Севастьянов~Л.\,А., Щетинин~Е.\,Ю.} О методах повышения точности многоклассовой\linebreak
\\[-12pt]
\hspace*{23pt}классификации на несбалансированных данных&1&63--70\\
\Avtors{Семенов~А.\,Л.} см.\ Бетелин~В.\,Б.&&\\
\Avtors{Сенько~О.\,В.} см.\ Кирилюк~И.\,Л.&&\\
\Avtors{Серебрянский~С.\,М., Тырсин~А.\,Н.} Повышение точности решения обратных задач за\linebreak
\\[-12pt]
\hspace*{23pt}счет уточнения граничных условий&1&56--62\\
\Avtors{Симаворян~С.\,Ж.} см.\ Попов~Г.\,А.&&\\
\Avtors{Симонян~А.\,Р.} см.\ Попов~Г.\,А.&&\\
\Avtors{Смирнов~Д.\,В.} см.\ Грушо~А.\,А.&&\\
\Avtors{Смирнов~Д.\,В.} см.\ Грушо~А.\,А.&&\\
\Avtors{Соколов~И.\,А., Степченков~Ю.\,А., Дьяченко~Ю.\,Г., Рождественский~Ю.\,В.} Повышение\linebreak
\\[-12pt]
\hspace*{23pt}сбоеустойчивости самосинхронных схем&4&63--68\\
\Avtors{Сомин~Н.\,В.} см.\ Козеренко~Е.\,Б.&&\\
\Avtors{Сопин~Э.\,С.} см.\ Агеев~К.\,А.&&\\
\Avtors{Сопрунов~С.\,Ф.} см.\ Бетелин~В.\,Б.&&\\
\Avtors{Соченков~И.\,В.} см.\ Будзко~В.\,И.&&\\
\Avtors{Степченков~Ю.\,А.} см.\ Соколов~И.\,А.&&\\
\Avtors{Стефанович~А.\,И.} см.\ Босов~А.\,В.&&\\
\Avtors{Стрижов~В.\,В.} см.\ Гончаров~А.\,В.&&\\
\Avtors{Стрижов~В.\,В.} см.\ Грабовой~А.\,В.&&\\
\Avtors{Стрижов~В.\,В.} см.\ Потанин~М.\,С.&&\\
\Avtors{Ступников~С.\,А.} см.\ Брюхов~Д.\,О.&&\\
\Avtors{Терехина~И.\,Ю.} см.\ Грушо~А.\,А.&&\\
\Avtors{Тимонина~Е.\,Е.} см.\  Грушо~А.\,А.&&\\
\Avtors{Тимонина~Е.\,Е.} см.\ Грушо~А.\,А.&&\\
\Avtors{Тимонина~Е.\,Е.} см.\ Грушо~А.\,А.&&\\
\Avtors{Тимонина~Е.\,Е.} см.\ Грушо~А.\,А.&&\\
\Avtors{Тимонина~Е.\,Е.} см.\ Грушо~Н.\,А.&&\\
\Avtors{Тырсин~А.\,Н.} см.\ Серебрянский~С.\,М.&&\\
\Avtors{Улитина~Е.\,И.} см.\ Попов~Г.\,А.&&\\
\end{tabular}
}

\pagebreak

\def\leftkol{АВТОРСКИЙ УКАЗАТЕЛЬ ЗА 2020 г.} % ENGLISH ABSTRACTS}

\def\rightkol{АВТОРСКИЙ УКАЗАТЕЛЬ ЗА 2020 г.} %ENGLISH ABSTRACTS}

%\thispagestyle{myheadings}
\def\leftfootline{\small{\textbf{\thepage}
\hfill ИНФОРМАТИКА И ЕЁ ПРИМЕНЕНИЯ\ \ \ том~14\ \ \ выпуск~4\ \ \ 2020}
}%
 \def\rightfootline{\small{ИНФОРМАТИКА И ЕЁ ПРИМЕНЕНИЯ\ \ \ том~14\ \ \ выпуск~4\ \ \ 2020
 \hfill \textbf{\thepage}}}


\noindent
{\tabcolsep=3pt
\begin{tabular}{p{394pt}cc}
&\textbf{Вып.} & \textbf{Стр.}\\[3pt]
\Avtors{Фахрутдинов~Р.\,Ш.} см.\ Костина~А.\,А.&&\\
\Avtors{Френкель С. Л., Захаров В. Н.} Совместная оценка предсказуемости данных и качества\linebreak
\\[-12pt]
\hspace*{23pt}предикторов&2&40--49\\
\Avtors{Харин~П.\,А., Макеева~Е.\,Д., Кочеткова~И.\,А., Ефросинин~Д.\,В., Шоргин~С.\,Я.} 
Система массового обслуживания с орбитами для анализа совместного обслуживания трафика 
с малыми задержками URLLC и~широкополосного доступа eMBB в~беспроводных\linebreak
\\[-12pt]
\hspace*{23pt}сетях пятого поколения&4&17--24\\
\Avtors{Хусаинов~А.\,А.} Производительность ограниченного конвейера&1&87--93\\
\Avtors{Шанин~И.\,А.} см.\ Брюхов~Д.\,О.&&\\
\Avtors{Шварцман~М.\,Е.} см.\ Краснов~Ф.\,В.&&\\
\Avtors{Шестаков~О.\,В.} Асимптотика оценки среднеквадратичного риска в задаче обращения\linebreak
\\[-12pt]
\hspace*{23pt}преобразования Радона по проекциям, регистрируемым на случайной сетке&2&26--32\\
\Avtors{Шестаков~О.\,В.} Асимптотическая регулярность вейвлет-методов обращения линейных однородных операторов по наблюдениям, регистрируемым в случайные моменты\linebreak
\\[-12pt]
\hspace*{23pt}времени&1&3--9\\
\Avtors{Шестаков~О.\,В.} О статистических свойствах оценки риска в задаче обращения преобра-\linebreak
\\[-12pt]
\hspace*{23pt}зования Радона при случайном объеме проекционных данных&3&44--48\\
\Avtors{Шестаков~О.\,В.} см.\ Кудрявцев~А.\,А.&&\\
\Avtors{Шилова~Г.\,Н.} см.\ Сатин~Я.\,А.&&\\
\Avtors{Шихиев~Ф.\,Ш.} см.\ Шихиев~Ш.\,Б.&&\\
\Avtors{Шихиев~Ш.\,Б., Шихиев~Ф.\,Ш.} Инкапсуляция семантических представлений в элементы\linebreak
\\[-12pt]
\hspace*{23pt}грамматики&1&121--127\\
\Avtors{Шнурков~П.\,В., Адамова~К.\,А.} Решение задачи безусловного экстремума для дробно-\linebreak
\\[-12pt]
\hspace*{23pt}линейного интегрального функционала, зависящего от параметра&2&\hphantom{1}98--103\\
\Avtors{Шнурков~П.\,В., Новиков~Д.\,А.} О концепции стохастической модели с управлением в~моменты выхода процесса на границу заданного подмножества множества\linebreak
\\[-12pt]
\hspace*{23pt}состояний&3&101--108\\
\Avtors{Шоргин~В.\,С.} см.\ Москалева~Ф.\,А.&&\\
\Avtors{Шоргин~С.\,Я.} см.\ Агеев~К.\,А.&&\\
\Avtors{Шоргин~С.\,Я.} см.\ Грушо~А.\,А.&&\\
\Avtors{Шоргин~С.\,Я.} см.\ Харин~П.\,А.&&\\
\Avtors{Щербинина~А.\,А.} см.\ Горшенин~А.\,К.&&\\
\Avtors{Щетинин~Е.\,Ю.} см.\ Севастьянов~Л.\,А.&&\\
\Avtors{Эрлих~Л.\,И.} см.\ Козеренко~Е.\,Б.&&\\
\Avtors{Ядринцев~В.\,В.} см.\ Будзко~В.\,И.&&\\
\Avtors{Яркина~Н.\,В.} см.\ Агеев~К.\,А.&&\\
\end{tabular}
}

%\thispagestyle{myheadings}
\def\leftfootline{\small{\textbf{\thepage}
\hfill ИНФОРМАТИКА И ЕЁ ПРИМЕНЕНИЯ\ \ \ том~14\ \ \ выпуск~4\ \ \ 2020}
}%
 \def\rightfootline{\small{ИНФОРМАТИКА И ЕЁ ПРИМЕНЕНИЯ\ \ \ том~14\ \ \ выпуск~4\ \ \ 2020
 \hfill \textbf{\thepage}}}

 \label{end\stat}

\newpage

\def\stat{cont-e}
{%\hrule\par
%\vskip 7pt % 7pt
\raggedleft\Large \bf%\baselineskip=3.2ex
2\,0\,2\,0\ \ A\,U\,T\,H\,O\,R\ \ I\,N\,D\,E\,X \vskip 17pt
 \hrule
 \par
\vskip 21pt plus 6pt minus 3pt }

\label{st\stat}

\def\tit{\ }

\def\aut{\ }
\def\auf{\ }

\def\leftkol{\ } %2020 AUTHOR INDEX} % ENGLISH ABSTRACTS}

\def\rightkol{\ } %2020 AUTHOR INDEX} %ENGLISH ABSTRACTS}

\titele{\tit}{\aut}{\auf}{\leftkol}{\rightkol}
\addcontentsline{toc}{subsection}{\textrm\textbf 2020 Author Index}

\def\leftfootline{\small{\textbf{\thepage}
\hfill INFORMATIKA I EE PRIMENENIYA~--- INFORMATICS AND APPLICATIONS\ \ \ 2020\
\ \ volume~14\ \ \ issue\ 4}
}%
 \def\rightfootline{\small{INFORMATIKA I EE PRIMENENIYA~--- INFORMATICS AND APPLICATIONS\ \ \ 2020\ \ \ volume~14\ \ \ issue\ 4
\hfill \textbf{\thepage}}}

\vspace*{-24pt}

\noindent
{\tabcolsep=3pt
\begin{tabular}{p{395.89pt}cc}
&\textbf{Issue} & \textbf{Page}\\[6pt]
\Avtors{Abgaryan~K.\,K. and Gavrilov~E.\,S.} Integration platform for multiscale modeling of neuromorphic\linebreak
\\[-12pt]
\hspace*{23pt}systems&2&104--110\\
\Avtors{Abgaryan~K.\,K. and Kolbin~I.\,S.} Application of multiscale approach and data sciences for\linebreak
\\[-12pt]
\hspace*{23pt}modeling thermal conductivity in layered structures&4&91--99\\
\Avtors{Adamova~K.\,A.} see Shnurkov~~P.\,V.&&\\
\Avtors{Agalarov~Ya.\,M.} Optimization of the capacity of the main storage in $G/M/1/K$ queueing system\linebreak
\\[-12pt]
\hspace*{23pt}with an additional storage device&2&72--79\\
\Avtors{Agasandyan~G.\,A.} Computational aspects of optimization on CC-VaR in a complex of markets&3&62--70\\
\Avtors{Ageev~K.\,A., Sopin~E.\,S., Yarkina~N.\,V., Samouylov~K.\,E., and Shorgin~S.\,Ya.} Analysis of the\linebreak
\\[-12pt]
\hspace*{23pt}network slicing mechanisms with guaranteed allocated resources for various traffic types&3&\hphantom{1}94--100\\
\Avtors{Bakhteev~O.\,Yu.} see Grabovoy~A.\,V.&&\\
\Avtors{Bazilevskiy~M.\,P.} Multifactor fully connected linear regression models without constraints to the\linebreak
\\[-12pt]
\hspace*{23pt}ratios of variables errors variances&2&92--97\\
\Avtors{Belenkov~V.\,G.} see Budzko~V.\,I.&&\\
\Avtors{Betelin~V.\,B., Kushnirenko~A.\,G., and Leonov~A.\,G.} Basic concepts of programming expounded\linebreak
\\[-12pt]
\hspace*{23pt}for preschoolers&3&55--61\\
\Avtors{Betelin~V.\,B., Kushnirenko~A.\,G., Semenov~A.\,L., and Soprunov~S.\,F.} About digital literacy and\linebreak
\\[-12pt]
\hspace*{23pt}environments for its development&4&100--107\\
\Avtors{Borisov~A.\,V.} Numerical schemes of Markov jump process filtering given discretized observa-\linebreak
\\[-12pt]
\hspace*{23pt}tions~II: Additive noise case&1&17--23\\
\Avtors{Borisov~A.\,V.} Numerical schemes of Markov jump process filtering given discretized observa-\linebreak
\\[-12pt]
\hspace*{23pt}tions III: Multiplicative noises case&2&10--18\\
\Avtors{Bosov~A.\,V.} Stochastic differential system output control by the quadratic criterion. V. Case of\linebreak
\\[-12pt]
\hspace*{23pt}incomplete state information&2&19--28\\
\Avtors{Bosov~A.\,V., Martyushova~Ya.\,G., Naumov~A.\,V., and Sapunova~A.\,P.} Bayesian approach to the\linebreak
\\[-12pt]
\hspace*{23pt}construction of an individual user trajectory in the system of distance learning&3&86--93\\
\Avtors{Bosov~A.\,V. and Stefanovich~A.\,I.} Stochastic differential system output control by the quadratic\linebreak
\\[-12pt]
\hspace*{23pt}criterion. IV. Alternative numerical decision&1&24--30\\
\Avtors{Briukhov~D.\,O., Stupnikov~S.\,A., Kovalev~D.\,Yu., and Shanin~I.\,A.} Neurophysiology as a subject\linebreak
\\[-12pt]
\hspace*{23pt}domain for~data intensive problem solving&1&40--47\\
\Avtors{Budzko~V.\,I., Yadrintsev~V.\,V., Sochenkov~I.\,V., Korolev~V.\,I., and Belenkov~V.\,G.} Extraction of confidentiality markers from texts under conditions of high uncertainty in systems with\linebreak
\\[-12pt]
\hspace*{23pt}data intensive usage&4&69--76\\
\Avtors{Danilishin~A.\,R. and Golembiovsky~D.\,Yu.} Estimating the fair value of options based on\linebreak
\\[-12pt]
\hspace*{23pt}ARIMA--GARCH models with errors distributed according to the Johnson's $S_u$ law&4&83--90\\
\Avtors{Danilishin~A.\,R. and Golembiovsky~D.\,Yu.} Risk-neutral dynamics for the ARIMA-GARCH\linebreak
\\[-12pt]
\hspace*{23pt}random process with errors distributed according to the Johnson's $S_u$ law&1&48--55\\
\Avtors{Diachenko~Yu.\,G.} see Sokolov~I.\,A.&&\\
\Avtors{Dimentov~A.\,V.} see Krasnov~F.\,V.&&\\
\Avtors{Donskoy~V.\,I.} Optimization models extraction from data&3&109--118\\
\Avtors{Dubnov~Y.\,A.} see Popkov~Y.\,S.&&\\
\Avtors{Dulin~S.\,K., Dulina~N.\,G., and Ermakov~P.\,V.} Information fusion of documents&1&128--135\\
\Avtors{Dulina~N.\,G.} see Dulin~S.\,K.&&\\
\Avtors{Efrosinin~D.\,V.} see Kharin~P.\,A.&&\\
\Avtors{Ehrlich~L.\,I.} see Kozerenko~E.\,B.&&\\
\Avtors{Ermakov~P.\,V.} see Dulin~S.\,K.&&\\
\end{tabular}
}
\pagebreak

\def\leftfootline{\small{\textbf{\thepage}
\hfill INFORMATIKA I EE PRIMENENIYA~--- INFORMATICS AND APPLICATIONS\ \ \ 2020\
\ \ volume~14\ \ \ issue\ 4}
}%
 \def\rightfootline{\small{INFORMATIKA I EE PRIMENENIYA~---
INFORMATICS AND APPLICATIONS\ \ \ 2020\ \ \ volume~14\ \ \ issue\ 4
\hfill \textbf{\thepage}}}

\def\leftkol{2020 AUTHOR INDEX} % ENGLISH ABSTRACTS}

\def\rightkol{2020 AUTHOR INDEX} %ENGLISH ABSTRACTS}


\noindent
{\tabcolsep=3pt
\begin{tabular}{p{395.48108pt}cc}
&\textbf{Issue} & \textbf{Page}\\[6pt]
\Avtors{Fahrutdinov~R.\,Sh.} see Kostina~A.\,A.&&\\
\Avtors{Frenkel~S.\,L. and Zakharov~V.\,N.} Joint assessment of data predictability and quality pre-\linebreak
\\[-12pt]
\hspace*{23pt}dictors&2&40--49\\
\Avtors{Gaidamaka~Yu.\,V.} see Moskaleva~F.\,A.&&\\
\Avtors{Gavrilov~E.\,S.} see Abgaryan~K.\,K.&&\\
\Avtors{Golembiovsky~D.\,Yu.} see Danilishin~A.\,R.&&\\
\Avtors{Golembiovsky~D.\,Yu.} see Danilishin~A.\,R.&&\\
\Avtors{Goncharov~A.\,V. and Strijov~V.\,V.} Alignment of ordered set Cartesian product&1&31--39\\
\Avtors{Goncharov~A.\,A., Zatsman~I.\,M., and Kruzhkov~M.\,G.} Evolution of classifications in supracorpora\linebreak
\\[-12pt]
\hspace*{23pt}databases&4&108--116\\
\Avtors{Gorshenin~A.\,K. and Korolev~V.\,Yu.} Approximation of particle size distributions of lunar regolith\linebreak
\\[-12pt]
\hspace*{23pt}based on the resampling&2&50--57\\
\Avtors{Gorshenin~A.\,K., Korolev~V.\,Yu., and Shcherbinina~A.\,A.} Statistical estimation of distributions\linebreak
\\[-12pt]
\hspace*{23pt}of random coefficients in the Langevin stochastic differential equation&3&\hphantom{1}3--12\\
\Avtors{Gorshenin~A.\,K. and Kuzmin~V.\,Yu.} Analysis of configurations of LSTM networks for medium-\linebreak
\\[-12pt]
\hspace*{23pt}term vector forecasting&1&10--16\\
\Avtors{Grabovoy~A.\,V., Bakhteev~O.\,Yu., and Strijov~V.\,V.} Ordering the set of neural network parameters&2&58--65\\
\Avtors{Grusho~A.\,A., Timonina~E.\,E., Grusho~N.\,A., and Teryokhina~I.\,Yu.} Identifying anomalies using\linebreak
\\[-12pt]
\hspace*{23pt}metadata&3&76--80\\
\Avtors{Grusho~A.\,A., Zabezhailo~M.\,I., Smirnov~D.\,V., and Timonina~E.\,E.} On probabilistic estimates of\linebreak
\\[-12pt]
\hspace*{23pt}the validity of empirical conclusions&4&3--8\\
\Avtors{Grusho~A.\,A., Zabezhailo~M.\,I., and Timonina~E.\,E.} On causal representativeness of training\linebreak
\\[-12pt]
\hspace*{23pt}samples of precedents in diagnostic type tasks&1&80--86\\
\Avtors{Grusho~A.\,A.} see Grusho~N.\,A.&&\\
\Avtors{Grusho~N.\,A., Grusho~A.\,A., Zabezhailo~M.\,I., and Timonina~E.\,E.} Methods of finding the causes\linebreak
\\[-12pt]
\hspace*{23pt}of information technology failures by means of metadata&2&33--39\\
\Avtors{Grusho~N.\,A., Zabezhailo~M.\,I., Smirnov~D.\,V., Timonina~E.\,E., and Shorgin~S.\,Ya.} Mathematical\linebreak
\\[-12pt]
\hspace*{23pt}statistics in the task of identifying hostile insiders&3&71--75\\
\Avtors{Grusho~N.\,A.} see Grusho~A.\,A.&&\\
\Avtors{Kharin~P.\,A., Makeeva~E.\,D., Kochetkova~I.\,A., Efrosinin~D.\,V., and Shorgin~S.\,Ya.} Retrial\linebreak
\\[-12pt]
\hspace*{23pt}queuing model for analyzing joint URLLC and eMBB transmission in 5G networks&4&17--24\\
\Avtors{Khusainov~A.\,A.} Performance of the bounded pipeline&1&87--93\\
\Avtors{Kirikov~I.\,A.} see Rumovskaya~S.\,B.&&\\
\Avtors{Kirilyuk~I.\,L. and Sen'ko~O.\,V.} Selection of optimal complexity models by methods of nonparametric statistics (on the example of production function model of regions of the Russian\linebreak
\\[-12pt]
\hspace*{23pt}Federation)&2&111--118\\
\Avtors{Kochetkova~I.\,A.} see Kharin~P.\,A.&&\\
\Avtors{Kolbin~I.\,S.} see Abgaryan~K.\,K.&&\\
\Avtors{Korolev~V.\,I.} see Budzko~V.\,I.&&\\
\Avtors{Korolev~V.\,Yu.} On the distribution of the ratio of the sum of sample elements exceeding\linebreak
\\[-12pt]
\hspace*{23pt}a threshold to the total sum of sample elements.~I&3&35--43\\
\Avtors{Korolev~V.\,Yu.} On the distribution of the ratio of the sum of sample elements exceeding\linebreak
\\[-12pt]
\hspace*{23pt}a threshold to the total sum of sample elements.~II&4&33--36\\
\Avtors{Korolev~V.\,Yu.} see Gorshenin~A.\,K.&&\\
\Avtors{Korolev~V.\,Yu.} see Gorshenin~A.\,K.&&\\
\Avtors{Kostina~A.\,A., Mirin~A.\,Yu., Moldovyan~D.\,N., and Fahrutdinov~R.\,Sh.} Method for defining finite noncommutative associative algebras of arbitrary even dimension for development of the\linebreak
\\[-12pt]
\hspace*{23pt}postquantum cryptoschemes&1&\hphantom{1}94--100\\
\Avtors{Kovalev~D.\,Yu.} see Briukhov~D.\,O.&&\\
\Avtors{Kozerenko~E.\,B., Mikheev~M.\,Y., Somin~N.\,V., Ehrlich~L.\,I., and Kuznetsov~K.\,I.} Analytical\linebreak
\\[-12pt]
\hspace*{23pt}textology in intelligent processing systems for unstructured data&1&113--120\\
\Avtors{Krasnov~F.\,V., Dimentov~A.\,V., and Shvartsman~M.\,E.} Using topic models for pairwise comparison\linebreak
\\[-12pt]
\hspace*{23pt}of collections of scientific papers&3&129--135\\
\end{tabular}
}
\pagebreak

\def\leftfootline{\small{\textbf{\thepage}
\hfill INFORMATIKA I EE PRIMENENIYA~--- INFORMATICS AND APPLICATIONS\ \ \ 2020\
\ \ volume~14\ \ \ issue\ 4}
}%
 \def\rightfootline{\small{INFORMATIKA I EE PRIMENENIYA~---
INFORMATICS AND APPLICATIONS\ \ \ 2020\ \ \ volume~14\ \ \ issue\ 4
\hfill \textbf{\thepage}}}

\def\leftkol{2020 AUTHOR INDEX} % ENGLISH ABSTRACTS}

\def\rightkol{2020 AUTHOR INDEX} %ENGLISH ABSTRACTS}


\noindent
{\tabcolsep=3pt
\begin{tabular}{p{395.48108pt}cc}
&\textbf{Issue} & \textbf{Page}\\[6pt]
\Avtors{Krivenko~M.\,P.} Sequential analysis of serial measurements based on multivariate reference\linebreak
\\[-12pt]
\hspace*{23pt}regions&2&86--91\\
\Avtors{Kruzhkov~M.\,G.} see Goncharov~A.\,A.&&\\
\Avtors{Kudryavtsev~A.\,A. and Shestakov~O.\,V.} Method of logarithmic moments for estimating the\linebreak
\\[-12pt]
\hspace*{23pt}gamma-exponential distribution parameters&3&49--54\\
\Avtors{Kushnirenko~A.\,G.} see Betelin~V.\,B.&&\\
\Avtors{Kushnirenko~A.\,G.} see Betelin~V.\,B.&&\\
\Avtors{Kuzmin~V.\,Yu.} see Gorshenin~A.\,K.&&\\
\Avtors{Kuznetsov~K.\,I.} see Kozerenko~E.\,B.&&\\
\Avtors{Leonov~A.\,G.} see Betelin~V.\,B.&&\\
\Avtors{Makeeva~E.\,D.} see Kharin~P.\,A.&&\\
\Avtors{Malashenko~Yu.\,E. and Nazarova~I.\,A.} Approximation of the multiuser network feasible\linebreak
\\[-12pt]
\hspace*{23pt}flows set&3&81--85\\
\Avtors{Martyushova~Ya.\,G.} see Bosov~A.\,V.&&\\
\Avtors{Matyushenko~S.\,I. and Razumchik~R.\,V.} Stationary characteristics of discrete-time Geo$/G/1/\infty$\linebreak
\\[-12pt]
\hspace*{23pt}queue with batch arrivals and one queue skipping policy&4&25--32\\
\Avtors{Melnikov~A.\,V.} see Vokhmintcev~A.\,V.&&\\
\Avtors{Melnikov~S.\,Yu. and Samouylov~K.\,E.} Statistical properties of binary nonautonomous shift\linebreak
\\[-12pt]
\hspace*{23pt}registers with internal xor&2&80--85\\
\Avtors{Meykhanadzhyan~L.\,A. and Razumchik~R.\,V.} Stationary characteristics of $M/G/2/\infty$ queue\linebreak
\\[-12pt]
\hspace*{23pt}with identical servers, LIFO service, and resampling policy&2&66--71\\
\Avtors{Mikheev~M.\,Y.} see Kozerenko~E.\,B.&&\\
\Avtors{Milovanova~T.\,A. and Razumchik~R.\,V.} A single-server queueing system with LIFO service,\linebreak
\\[-12pt]
\hspace*{23pt}probabilistic priority, batch Poisson arrivals, and background customers&3&26--34\\
\Avtors{Mirin~A.\,Yu.} see Kostina~A.\,A.&&\\
\Avtors{Moldovyan~D.\,N.} see Kostina~A.\,A.&&\\
\Avtors{Moskaleva~F.\,A., Gaidamaka~Yu.\,V., and Shorgin~V.\,S.} Impact of the isolation parameters on\linebreak
\\[-12pt]
\hspace*{23pt}resource allocation in the network slicing model&4&\hphantom{1}9--16\\
\Avtors{Naumov~A.\,V.} see Bosov~A.\,V.&&\\
\Avtors{Naumov~V.\,A. and Samouylov~К.\,Е.} On Markovian and rational arrival processes.~I&3&13--19\\
\Avtors{Naumov~V.\,A. and Samouylov~K.\,E.} On Markovian and rational arrival processes.~II&4&37--46\\
\Avtors{Nazarova~I.\,A.} see Malashenko~Yu.\,E.&&\\
\Avtors{Novikov~D.\,A.} see Shnurkov~P.\,V.&&\\
\Avtors{Nuriev~V.\,A. and Zatsman~I.\,M.} Reducing the spectrum of translation models in supracorpora\linebreak
\\[-12pt]
\hspace*{23pt}databases&2&119--126\\
\Avtors{Pachganov~S.\,A.} see Vokhmintcev~A.\,V.&&\\
\Avtors{Popkov~A.\,Y.} see Popkov~Y.\,S.&&\\
\Avtors{Popkov~Y.\,S., Popkov~A.\,Y., and Dubnov~Y.\,A.} Deterministic and randomized methods of entropy\linebreak
\\[-12pt]
\hspace*{23pt}projection for dimensionality reduction problems&4&47--54\\
\Avtors{Popov~G.\,A., Simavoryan~S.\,Zh., Simonyan~A.\,R., and Ulitina~E.\,I.} Modeling of monitoring of\linebreak
\\[-12pt]
\hspace*{23pt}information security process on the basis of queuing systems&1&71--79\\
\Avtors{Popov~M.\,V. and Posypkin~M.\,A.} Approximation of the set of solutions of systems of nonlinear\linebreak
\\[-12pt]
\hspace*{23pt}inequalities using graphic accelerators&3&20--25\\
\Avtors{Posypkin~M.\,A.} see Popov~M.\,V.&&\\
\Avtors{Potanin~M.\,S., Vayser~K.\,O., Zholobov~V.\,A., and Strijov~V.\,V.} Deep learning neural network\linebreak
\\[-12pt]
\hspace*{23pt}structure optimization&4&55--62\\
\Avtors{Razumchik~R.\,V.} see Matyushenko~S.\,I.&&\\
\Avtors{Razumchik~R.\,V.} see Meykhanadzhyan~L.\,A.&&\\
\Avtors{Razumchik~R.\,V.} see Milovanova~T.\,A.&&\\
\Avtors{Rogdestvenski~Yu.\,V.} see Sokolov~I.\,A.&&\\
\Avtors{Rumovskaya~S.\,B. and Kirikov~I.\,A.} Conflict visual representation method in hybrid intelligent\linebreak
\\[-12pt]
\hspace*{23pt}multiagent systems&4&77--82\\
\Avtors{Samouylov~K.\,E.} see Ageev~K.\,A.&&\\
\end{tabular}
}
\pagebreak

\def\leftfootline{\small{\textbf{\thepage}
\hfill INFORMATIKA I EE PRIMENENIYA~--- INFORMATICS AND APPLICATIONS\ \ \ 2020\
\ \ volume~14\ \ \ issue\ 4}
}%
 \def\rightfootline{\small{INFORMATIKA I EE PRIMENENIYA~---
INFORMATICS AND APPLICATIONS\ \ \ 2020\ \ \ volume~14\ \ \ issue\ 4
\hfill \textbf{\thepage}}}

\def\leftkol{2020 AUTHOR INDEX} % ENGLISH ABSTRACTS}

\def\rightkol{2020 AUTHOR INDEX} %ENGLISH ABSTRACTS}


\noindent
{\tabcolsep=3pt
\begin{tabular}{p{395.48108pt}cc}
&\textbf{Issue} & \textbf{Page}\\[6pt]
\Avtors{Samouylov~K.\,E.} see Melnikov~S.\,Yu.&&\\
\Avtors{Samouylov~K.\,E.} see Naumov~V.\,A.&&\\
\Avtors{Samouylov~K.\,Е.} see Naumov~V.\,A.&&\\
\Avtors{Sapunova~A.\,P.} see Bosov~A.\,V.&&\\
\Avtors{Satin~Ya.\,A., Zeifman~A.\,I., and Shilova~G.\,N.} On approaches to constructing limiting regimes\linebreak
\\[-12pt]
\hspace*{23pt}for some queuing models&2&3--9\\
\Avtors{Semenov~A.\,L.} see Betelin~V.\,B.&&\\
\Avtors{Sen'ko~O.\,V.} see Kirilyuk~I.\,L.&&\\
\Avtors{Serebryanskii~S.\,M. and Tyrsin~A.\,N.} Improvement of the accuracy of solution of tasks for the\linebreak
\\[-12pt]
\hspace*{23pt}account of the construction of boundary conditions&1&56--62\\
\Avtors{Sevastianov~L.\,A. and Shchetinin~E.\,Yu.} On methods for improving the accuracy of multiclass\linebreak
\\[-12pt]
\hspace*{23pt}classification on imbalanced data&1&63--70\\
\Avtors{Shanin~I.\,A.} see Briukhov~D.\,O.&&\\
\Avtors{Shcherbinina~A.\,A.} see Gorshenin~A.\,K.&&\\
\Avtors{Shchetinin~E.\,Yu.} see Sevastianov~L.\,A.&&\\
\Avtors{Shestakov~O.\,V.} Asymptotic regularity of the wavelet methods of inverting linear homogeneous\linebreak
\\[-12pt]
\hspace*{23pt}operators from observations recorded at random times&1&3--9\\
\Avtors{Shestakov~O.\,V.} Asymptotics of the mean-square risk estimate in the problem of inverting the\linebreak
\\[-12pt]
\hspace*{23pt}Radon transform from projections registered on a random grid&2&29--32\\
\Avtors{Shestakov~O.\,V.} On the statistical properties of risk estimate in the problem of inverting the\linebreak
\\[-12pt]
\hspace*{23pt}Radon transform with a random volume of projection data&3&44--48\\
\Avtors{Shestakov~O.\,V.} see Kudryavtsev~A.\,A.&&\\
\Avtors{Shihiev~F.\,Sh.} see Shihiev~Sh.\,B.&&\\
\Avtors{Shihiev~Sh.\,B. and Shihiev~F.\,Sh.} Incapsulation of semantic representations into elements of\linebreak
\\[-12pt]
\hspace*{23pt}a grammar&1&121--127\\
\Avtors{Shilova~G.\,N.} see Satin~Ya.\,A.&&\\
\Avtors{Shnurkov~~P.\,V. and Adamova~K.\,A.} Solution of the unconditional extremal problem for a~linear-\linebreak
\\[-12pt]
\hspace*{23pt}fractional integral functional dependent on the parameter&2&\hphantom{1}98--103\\
\Avtors{Shnurkov~P.\,V. and Novikov~D.\,A.} On the concept of a stochastic model with control at the\linebreak
\\[-12pt]
\hspace*{23pt}moments of the process at the border of a presented subset of multiple states&3&101--108\\
\Avtors{Shorgin~S.\,Ya.} see Ageev~K.\,A.&&\\
\Avtors{Shorgin~S.\,Ya.} see Grusho~N.\,A.&&\\
\Avtors{Shorgin~S.\,Ya.} see Kharin~P.\,A.&&\\
\Avtors{Shorgin~V.\,S.} see Moskaleva~F.\,A.&&\\
\Avtors{Shvartsman~M.\,E.} see Krasnov~F.\,V.&&\\
\Avtors{Simavoryan~S.\,Zh.} see Popov~G.\,A.&&\\
\Avtors{Simonyan~A.\,R.} see Popov~G.\,A.&&\\
\Avtors{Smirnov~D.\,V.} see Grusho~A.\,A.&&\\
\Avtors{Smirnov~D.\,V.} see Grusho~N.\,A.&&\\
\Avtors{Sochenkov~I.\,V.} see Budzko~V.\,I.&&\\
\Avtors{Sokolov~I.\,A., Stepchenkov~Yu.\,A., Diachenko~Yu.\,G., and Rogdestvenski~Yu.\,V.} Improvement of\linebreak
\\[-12pt]
\hspace*{23pt}self-timed circuit soft error tolerance&4&63--68\\
\Avtors{Somin~N.\,V.} see Kozerenko~E.\,B.&&\\
\Avtors{Sopin~E.\,S.} see Ageev~K.\,A.&&\\
\Avtors{Soprunov~S.\,F.} see Betelin~V.\,B.&&\\
\Avtors{Stefanovich~A.\,I.} see Bosov~A.\,V.&&\\
\Avtors{Stepchenkov~Yu.\,A.} see Sokolov~I.\,A.&&\\
\Avtors{Strijov~V.\,V.} see Goncharov~A.\,V.&&\\
\Avtors{Strijov~V.\,V.} see Grabovoy~A.\,V.&&\\
\Avtors{Strijov~V.\,V.} see Potanin~M.\,S.&&\\
\Avtors{Stupnikov~S.\,A.} see Briukhov~D.\,O.&&\\
\Avtors{Teryokhina~I.\,Yu.} see Grusho~A.\,A.&&\\
\Avtors{Timonina~E.\,E.} see Grusho~A.\,A.&&\\
\end{tabular}
}
\pagebreak

\def\leftfootline{\small{\textbf{\thepage}
\hfill INFORMATIKA I EE PRIMENENIYA~--- INFORMATICS AND APPLICATIONS\ \ \ 2020\
\ \ volume~14\ \ \ issue\ 4}
}%
 \def\rightfootline{\small{INFORMATIKA I EE PRIMENENIYA~---
INFORMATICS AND APPLICATIONS\ \ \ 2020\ \ \ volume~14\ \ \ issue\ 4
\hfill \textbf{\thepage}}}

\def\leftkol{2020 AUTHOR INDEX} % ENGLISH ABSTRACTS}

\def\rightkol{2020 AUTHOR INDEX} %ENGLISH ABSTRACTS}


\noindent
{\tabcolsep=3pt
\begin{tabular}{p{395.48108pt}cc}
&\textbf{Issue} & \textbf{Page}\\[6pt]
\Avtors{Timonina~E.\,E.} see Grusho~A.\,A.&&\\
\Avtors{Timonina~E.\,E.} see Grusho~A.\,A.&&\\
\Avtors{Timonina~E.\,E.} see Grusho~N.\,A.&&\\
\Avtors{Timonina~E.\,E.} see Grusho~N.\,A.&&\\
\Avtors{Tyrsin~A.\,N.} see Serebryanskii~S.\,M.&&\\
\Avtors{Ulitina~E.\,I.} see Popov~G.\,A.&&\\
\Avtors{Vayser~K.\,O.} see Potanin~M.\,S.&&\\
\Avtors{Vokhmintcev~A.\,V., Melnikov~A.\,V., and Pachganov~S.\,A.} Simultaneous localization and mapping method in  three-dimensional space based on the combined solution of the  point--point\linebreak
\\[-12pt]
\hspace*{23pt}variation problem ICP for an affine transformation&1&101--112\\
\Avtors{Yadrintsev~V.\,V.} see Budzko~V.\,I.&&\\
\Avtors{Yarkina~N.\,V.} see Ageev~K.\,A.&&\\
\Avtors{Zabezhailo~M.\,I.} see Grusho~A.\,A.&&\\
\Avtors{Zabezhailo~M.\,I.} see Grusho~A.\,A.&&\\
\Avtors{Zabezhailo~M.\,I.} see Grusho~N.\,A.&&\\
\Avtors{Zabezhailo~M.\,I.} see Grusho~N.\,A.&&\\
\Avtors{Zakharov~V.\,N.} see Frenkel~S.\,L.&&\\
\Avtors{Zatsman~I.\,M.} Problem-oriented verifying the completeness  of~temporal ontologies and\linebreak
\\[-12pt]
\hspace*{23pt}filling~conceptual lacunas&3&119--128\\
\Avtors{Zatsman~I.\,M.} see Goncharov~A.\,A.&&\\
\Avtors{Zatsman~I.\,M.} see Nuriev~V.\,A.&&\\
\Avtors{Zeifman~A.\,I.} see Satin~Ya.\,A.&&\\
\Avtors{Zholobov~V.\,A.} see Potanin~M.\,S.&&\\
\end{tabular}
}

%\thispagestyle{myheadings}
\def\leftfootline{\small{\textbf{\thepage}
\hfill INFORMATIKA I EE PRIMENENIYA~--- INFORMATICS AND APPLICATIONS\ \ \ 2020\
\ \ volume~14\ \ \ issue\ 4}
}%
 \def\rightfootline{\small{INFORMATIKA I EE PRIMENENIYA~---
INFORMATICS AND APPLICATIONS\ \ \ 2020\ \ \ volume~14\ \ \ issue\ 4
\hfill \textbf{\thepage}}}

 \label{end\stat}

\newpage


%\linebreak
%\\[-12pt]
%\hspace*{23pt}
%Информатика и её применения
%Том 15 Выпуск 1-4 Год 2021

\def\stat{cont}
{%\hrule\par
%\vskip 7pt % 7pt
\raggedleft\Large \bf%\baselineskip=3.2ex
А\,В\,Т\,О\,Р\,С\,К\,И\,Й\ \ У\,К\,А\,З\,А\,Т\,Е\,Л\,Ь\ \ З\,А\ \ 2\,0\,2\,1 г. \vskip 17pt
 \hrule
 \par
\vskip 21pt plus 6pt minus 3pt }

\label{st\stat}

\def\tit{\ }

\def\aut{\ }
\def\auf{\ }

\def\leftkol{\ } % ENGLISH ABSTRACTS}

\def\rightkol{\ } %АВТОРСКИЙ УКАЗАТЕЛЬ ЗА 2021 г.} %ENGLISH ABSTRACTS}

\titele{\tit}{\aut}{\auf}{\leftkol}{\rightkol}
\addcontentsline{toc}{subsection}{\textrm\textbf Авторский указатель за 2021 г.}

\vspace*{-24pt}

\noindent
{\tabcolsep=3pt
\begin{tabular}{p{397pt}cc}
&\textbf{Вып.} & \textbf{Стр.}\\[6pt]
\Avtors{Абгарян~К.\,К., Гаврилов~Е.\,С.} Распределенная информационная система для расчета\linebreak
\\[-12pt]
\hspace*{23pt}структурных свойств композиционных материалов&4&50--58\\
\Avtors{Агаларов~Я.\,М.} Оптимальное пороговое управление доступом в системе $M/M/s$ с не-\linebreak
\\[-12pt]
\hspace*{23pt}однородными приборами и общим накопителем&1&57--64\\
\Avtors{Андрианова~Е.\,Г.} см.\ Сигов~А.\,С.&&\\
\Avtors{Арутюнов~Е.\,Н., Кудрявцев~А.\,А., Недоливко~Ю.\,Н.} Вероятностные характеристики\linebreak
\\[-12pt]
\hspace*{23pt}индекса баланса факторов, имеющих обобщенные гамма-распределения&1&65--71\\
\Avtors{Базилевский~М.\,П.} Метод выпрямления искаженных из-за мультиколлинеарности\linebreak
\\[-12pt]
\hspace*{23pt}коэффициентов в регрессионных моделях&2&60--65\\
\Avtors{Бахтеев~О.\,Ю.} см.\ Гребенькова~О.\,С.&&\\
\Avtors{Бахтеев~О.\,Ю.} см.\ Кузнецова~Р.\,В.&&\\
\Avtors{Борисов~А.\,В., Казанчян~Д.\,Х.} Фильтрация состояний марковских скачкообразных про-\linebreak
\\[-12pt]
\hspace*{23pt}цессов по комплексным наблюдениям I: точное решение задачи&2&12--19\\
\Avtors{Борисов~А.\,В., Казанчян~Д.\,Х.} Фильтрация состояний марковских скачкообразных про-\linebreak
\\[-12pt]
\hspace*{23pt}цессов по комплексным наблюдениям II: численный алгоритм&3&\hphantom{1}9--15\\
\Avtors{Босов~А.\,В.} О некоторых частных случаях в задаче управления выходом стохастической\linebreak
\\[-12pt]
\hspace*{23pt}дифференциальной системы по квадратичному критерию&1&11--17\\
\Avtors{Босов~А.\,В.} Управление линейным выходом марковской цепи по квадратичному кри-\linebreak
\\[-12pt]
\hspace*{23pt}терию&2&\hphantom{1}3--11\\
\Avtors{Босов~А.\,В., Жуков~Д.\,В.} Экспертная система для мониторинга и прогнозирования\linebreak
\\[-12pt]
\hspace*{23pt}процессов распределения ресурсов&3&29--40\\
\Avtors{Босов~А.\,В., Игнатов~А.\,Н., Наумов~А.\,В.} Алгоритмы приближенного решения задачи\linebreak
\\[-12pt]
\hspace*{23pt}назначения <<технологического окна>> на участках железнодорожной сети&4&\hphantom{1}3--11\\
\Avtors{Брюхов~Д.\,О., Ступников~С.\,А., Ковалёв~Д.\,Ю., Шанин~И.\,А.} Архитектура распределенного\linebreak
\\[-12pt]
\hspace*{23pt}решения задач анализа данных в области нейрофизиологии&1&78--85\\
\Avtors{Власкина~А.\,С.} см.\ Кочеткова~И.\,А.&&\\
\Avtors{Ву~Н.\,Н.} см.\ Кочеткова~И.\,А.&&\\
\Avtors{Вышинский~Л.\,Л., Флёров~Ю.\,А.} Информационная модель весового облика летательных\linebreak
\\[-12pt]
\hspace*{23pt}аппаратов&1&50--56\\
\Avtors{Вышинский~Л.\,Л., Флёров~Ю.\,А.} Теоретические основы формирования весового облика\linebreak
\\[-12pt]
\hspace*{23pt}самолета&4&\hphantom{1}93--102\\
\Avtors{Гаврилов~Е.\,С.} см.\ Абгарян~К.\,К.&&\\
\Avtors{Гончаренко~М.\,Б., Захарова~Т.\,В.} Некоторые свойства смесей нормальных распределений\linebreak
\\[-12pt]
\hspace*{23pt}и~их приложения к задачам магнитоэнцефалографии&2&44--51\\
\Avtors{Гончаров~А.\,А., Зацман~И.\,М.} Принципы структуризации статей в электронных словарях&2&89--95\\
\Avtors{Гончаров~А.\,А., Зацман~И.\,М., Кружков~М.\,Г.} Представление новых лексикографических\linebreak
\\[-12pt]
\hspace*{23pt}знаний в динамических классификационных системах&1&86--93\\
\Avtors{Гончаров~А.\,А., Зацман~И.\,М., Кружков~М.\,Г., Лощилова~Е.\,Ю.} Отражение эволюции\linebreak
\\[-12pt]
\hspace*{23pt}лексикографических знаний в~динамических классификационных системах&4&41--49\\
\Avtors{Гончаров~А.\,А., Инькова~О.\,Ю.} Извлечение знаний о средствах выражения логико-\linebreak
\\[-12pt]
\hspace*{23pt}се\-ман\-ти\-че\-ских отношений при помощи надкорпусной базы данных&2&\hphantom{1}96--103\\
\Avtors{Горшенин~А.\,К., Кузьмин~В.\,Ю.} Метод повышения точности нейросетевых прогнозов с использованием смешанных вероятностных моделей и его реализация в виде\linebreak
\\[-12pt]
\hspace*{23pt}цифрового сервиса&3&63--74\\
\Avtors{Гребенькова~О.\,С., Бахтеев~О.\,Ю., Стрижов~В.\,В.} Вариационная оптимизация модели\linebreak
\\[-12pt]
\hspace*{23pt}глубокого обучения с контролем сложности&1&42--49\\
\end{tabular}
}

\pagebreak

\def\leftkol{АВТОРСКИЙ УКАЗАТЕЛЬ ЗА 2021 г.} % ENGLISH ABSTRACTS}

\def\rightkol{АВТОРСКИЙ УКАЗАТЕЛЬ ЗА 2021 г.} %ENGLISH ABSTRACTS}

%\thispagestyle{myheadings}
\def\leftfootline{\small{\textbf{\thepage}
\hfill ИНФОРМАТИКА И ЕЁ ПРИМЕНЕНИЯ\ \ \ том~15\ \ \ выпуск~4\ \ \ 2021}
}%
 \def\rightfootline{\small{ИНФОРМАТИКА И ЕЁ ПРИМЕНЕНИЯ\ \ \ том~15\ \ \ выпуск~4\ \ \ 2021
 \hfill \textbf{\thepage}}}


\noindent
{\tabcolsep=3pt
\begin{tabular}{p{394pt}cc}
&\textbf{Вып.} & \textbf{Стр.}\\[3pt]
\Avtors{Гринченко~С.\,Н.} Антропогенная <<третья>> природа: относительно автономный статус\linebreak
\\[-12pt]
\hspace*{23pt}ее искусственных интеллектуальных субъектов&4&110--114\\
\Avtors{Гринченко~С.\,Н.} О системной иерархии искусственного интеллекта&1&111--115\\
\Avtors{Грушо~А.\,А., Грушо~Н.\,А., Забежайло~М.\,И., Смирнов~Д.\,В., Тимонина~Е.\,Е., Шоргин~С.\,Я.} Статистика и кластеры в~поисках аномальных вкраплений в~условиях больших\linebreak
\\[-12pt]
\hspace*{23pt}данных&4&79--86\\
\Avtors{Грушо~А.\,А., Грушо~Н.\,А., Забежайло~М.\,И., Тимонина~Е.\,Е.} Удаленный мониторинг\linebreak
\\[-12pt]
\hspace*{23pt}рабочих процессов&3&2--8\\
\Avtors{Грушо~А.\,А., Забежайло~М.\,И., Смирнов~Д.\,В., Тимонина~Е.\,Е.} Интеллектуальный анализ\linebreak
\\[-12pt]
\hspace*{23pt}пополняемых коллекций Big Data в режиме процессно-реального времени&2&36--40\\
\Avtors{Грушо~Н.\,А.} см.\ Грушо~А.\,А.&&\\
\Avtors{Грушо~Н.\,А.} см.\ Грушо~А.\,А.&&\\
\Avtors{Дараселия~А.\,В., Сопин~Э.\,С., Молчанов~Д.\,А., Самуйлов~К.\,Е.} Анализ стратегии разгрузки\linebreak
\\[-12pt]
\hspace*{23pt}базовых станций 5G NR с помощью технологии NR-U&3&98--111\\
\Avtors{Дорофеева~А.\,В.} см.\ Королев~В.\,Ю.&&\\
\Avtors{Дьяченко~Ю.\,Г.} см.\ Соколов~И.\,А.&&\\
\Avtors{Дюкова~Е.\,В., Масляков~Г.\,О.} О выборе частичных порядков на множествах значений\linebreak
\\[-12pt]
\hspace*{23pt}признаков в~задаче классификации&4&72--78\\
\Avtors{Егорова~А.\,Ю.} см.\ Нуриев~В.\,А.&&\\
\Avtors{Жуков~Д.\,В.} см.\ Босов~А.\,В.&&\\
\Avtors{Жуков~Д.\,О., Хватова~Т.\,Ю., Зальцман~А.\,Д.} Моделирование стохастической динамики изменения состояний узлов и~перколяционных переходов в~социальных сетях\linebreak
\\[-12pt]
\hspace*{23pt}с~учетом самоорганизации и наличия памяти&1&102--110\\
\Avtors{Забежайло~М.\,И.} см.\ Грушо~А.\,А.&&\\
\Avtors{Забежайло~М.\,И.} см.\ Грушо~А.\,А.&&\\
\Avtors{Забежайло~М.\,И.} см.\ Грушо~А.\,А.&&\\
\Avtors{Зальцман~А.\,Д.} см.\ Жуков~Д.\,О.&&\\
\Avtors{Захарова~Т.\,В.} см.\ Гончаренко~М.\,Б.&&\\
\Avtors{Зацман~И.\,М.} Концепция создания ВОЗ-центра компетенций по пандемиям и~эпиде-\linebreak
\\[-12pt]
\hspace*{23pt}миям: ключевые понятия и~их терминологический анализ&4&103--109\\
\Avtors{Зацман~И.\,М.} Проблемно-ориентированная актуализация словарных статей двуязыч-\linebreak
\\[-12pt]
\hspace*{23pt}ных словарей и медицинской терминологии: сопоставительный анализ&1&\hphantom{1}94--101\\
\Avtors{Зацман~И.\,М.} Формы представления нового знания, извлеченного из текстов&3&83--90\\
\Avtors{Зацман~И.\,М.} см.\ Гончаров~А.\,А.&&\\
\Avtors{Зацман~И.\,М.} см.\ Гончаров~А.\,А.&&\\
\Avtors{Зацман~И.\,М.} см.\ Гончаров~А.\,А.&&\\
\Avtors{Зейфман~А.\,И., Сатин~Я.\,А., Ковалёв~И.\,А.} Об одной нестационарной модели обслужи-\linebreak
\\[-12pt]
\hspace*{23pt}вания с катастрофами и тяжелыми хвостами&2&20--25\\
\Avtors{Игнатов~А.\,Н.} см.\ Босов~А.\,В.&&\\
\Avtors{Инькова~О.\,Ю., Кружков~М.\,Г.} Структурированные определения дискурсивных отно-\linebreak
\\[-12pt]
\hspace*{23pt}шений в~надкорпусной базе данных коннекторов&4&27--32\\
\Avtors{Инькова~О.\,Ю.} см.\ Гончаров~А.\,А.&&\\
\Avtors{Истратов~Л.\,А.} см.\ Сигов~А.\,С.&&\\
\Avtors{Казанчян~Д.\,Х.} см.\ Борисов~А.\,В.&&\\
\Avtors{Казанчян~Д.\,Х.} см.\ Борисов~А.\,В.&&\\
\Avtors{Каменских~А.\,Н.} см.\ Соколов~И.\,А.&&\\
\Avtors{Кириков~И.\,А., Листопад~С.\,В.} Согласование целей агентов сплоченных гибридных\linebreak
\\[-12pt]
\hspace*{23pt}интеллектуальных многоагентных систем&2&66--71\\
\Avtors{Кириков~И.\,А.} см.\ Румовская~С.\,Б.&&\\
\Avtors{Ковалёв~Д.\,Ю.}см.\ Брюхов~Д.\,О.&&\\
\Avtors{Ковалёв~И.\,А.} см.\ Зейфман~А.\,И.&&\\
\Avtors{Ковалёв~С.\,П.} Методы теории категорий в цифровом проектировании гетерогенных\linebreak
\\[-12pt]
\hspace*{23pt}киберфизических систем&1&23--29\\
\end{tabular}
}

\pagebreak

\def\leftkol{АВТОРСКИЙ УКАЗАТЕЛЬ ЗА 2021 г.} % ENGLISH ABSTRACTS}

\def\rightkol{АВТОРСКИЙ УКАЗАТЕЛЬ ЗА 2021 г.} %ENGLISH ABSTRACTS}

%\thispagestyle{myheadings}
\def\leftfootline{\small{\textbf{\thepage}
\hfill ИНФОРМАТИКА И ЕЁ ПРИМЕНЕНИЯ\ \ \ том~15\ \ \ выпуск~4\ \ \ 2021}
}%
 \def\rightfootline{\small{ИНФОРМАТИКА И ЕЁ ПРИМЕНЕНИЯ\ \ \ том~15\ \ \ выпуск~4\ \ \ 2021
 \hfill \textbf{\thepage}}}


\noindent
{\tabcolsep=3pt
\begin{tabular}{p{394pt}cc}
&\textbf{Вып.} & \textbf{Стр.}\\[3pt]
\Avtors{Коновалов~М.\,Г., Разумчик~Р.\,В.} Диспетчеризация в системе с параллельным обслужи-\linebreak
\\[-12pt]
\hspace*{23pt}ванием с помощью распределенного градиентного управления марковской цепью&3&41--50\\
\Avtors{Королев~В.\,Ю., Дорофеева~А.\,В.} О точности нормальной аппроксимации при отсутствии\linebreak
\\[-12pt]
\hspace*{23pt}нормальной сходимости&1&116--121\\
\Avtors{Кочеткова~И.\,А., Власкина~А.\,С., Ву~Н.\,Н., Шоргин~В.\,С.} Система массового обслуживания с управляемым по сигналам перераспределением приборов для анализа\linebreak
\\[-12pt]
\hspace*{23pt}нарезки ресурсов сети 5G&3&91--97\\
\Avtors{Кочеткова~И.\,А., Кущазли~А.\,И., Харин~П.\,А., Шоргин~С.\,Я.} Модель схемы приоритетного доступа 
трафика URLLC и~eMBB в~сети пятого поколения в~виде ресурсной\linebreak
\\[-12pt]
\hspace*{23pt}системы массового обслуживания&4&87--92\\
\Avtors{Кривенко~М.\,П.} Мягкие вычисления в задачах медицинской диагностики&2&52--59\\
\Avtors{Кружков~М.\,Г.} см.\ Гончаров~А.\,А.&&\\
\Avtors{Кружков~М.\,Г.} см.\ Гончаров~А.\,А.&&\\
\Avtors{Кружков~М.\,Г.} см.\ Инькова~О.\,Ю.&&\\
\Avtors{Кудрявцев~А.\,А., Шестаков~О.\,В.} Минимаксные оценки функции потерь, основанной на интегральных вероятностях ошибок при пороговой обработке вейвлет-\linebreak
\\[-12pt]
\hspace*{23pt}коэффициентов&4&12--19\\
\Avtors{Кудрявцев~А.\,А., Шестаков~О.\,В., Шоргин~С.\,Я.} Метод оценивания параметров изгиба,\linebreak
\\[-12pt]
\hspace*{23pt}формы и масштаба гамма-экспоненциального распределения&3&57--62\\
\Avtors{Кудрявцев~А.\,А.} см.\ Арутюнов~Е.\,Н.&&\\
\Avtors{Кузнецова~Р.\,В., Бахтеев~О.\,Ю., Чехович~Ю.\,В.} Методы обнаружения переводных\linebreak
\\[-12pt]
\hspace*{23pt}заимствований в больших текстовых коллекциях&1&30--41\\
\Avtors{Кузьмин~В.\,Ю.} см.\ Горшенин~А.\,К.&&\\
\Avtors{Кущазли~А.\,И.} см.\ Кочеткова~И.\,А.&&\\
\Avtors{Липатьев~А.\,А.} Неасимптотический анализ статистики Бартлетта--Нанда--Пилая для\linebreak
\\[-12pt]
\hspace*{23pt}данных большой размерности&1&72--77\\
\Avtors{Листопад~С.\,В.} см.\ Кириков~И.\,А.&&\\
\Avtors{Лощилова~Е.\,Ю.} см.\ Гончаров~А.\,А&&\\
\Avtors{Малашенко~Ю.\,Е.} Максимальные межузловые потоки при предельной загрузке много-\linebreak
\\[-12pt]
\hspace*{23pt}пользовательской сети&3&24--28\\
\Avtors{Малашенко~Ю.\,Е., Назарова~И.\,А.} Анализ распределения предельных нагрузок в~мно-\linebreak
\\[-12pt]
\hspace*{23pt}гопользовательской сети&4&20--26\\
\Avtors{Масляков~Г.\,О.} см.\ Дюкова~Е.\,В.&&\\
\Avtors{Молчанов~Д.\,А.} см.\ Дараселия~А.\,В.&&\\
\Avtors{Монахов~М.\,М.} Разложения Чебышёва--Эджворта для распределений обобщенных\linebreak
\\[-12pt]
\hspace*{23pt}статистик типа Хотеллинга, построенных по выборкам случайного размера&2&72--81\\
\Avtors{Назарова~И.\,А.} см.\ Малашенко~Ю.\,Е.&&\\
\Avtors{Наумов~А.\,В.} см.\ Босов~А.\,В.&&\\
\Avtors{Недоливко~Ю.\,Н.} см.\ Арутюнов~Е.\,Н.&&\\
\Avtors{Нуриев~В.\,А., Егорова~А.\,Ю.} Методы оценки качества машинного перевода: современное\linebreak
\\[-12pt]
\hspace*{23pt}состояние&2&104--111\\
\Avtors{Павлов~Ю.\,Л.} Связность конфигурационных графов в моделях сложных сетей&1&18--22\\
\Avtors{Разумчик~Р.\,В.} см.\ Коновалов~М.\,Г.&&\\
\Avtors{Рождественский~Ю.\,В.} см.\ Соколов~И.\,А.&&\\
\Avtors{Румовская~С.\,Б., Кириков~И.\,А.} Метод визуализации стимуляции конфликтов в гибрид-\linebreak
\\[-12pt]
\hspace*{23pt}ных интеллектуальных многоагентных системах&3&75--82\\
\Avtors{Самуйлов~К.\,Е.} см.\ Дараселия~А.\,В.&&\\
\Avtors{Сатин~Я.\,А.} см.\ Зейфман~А.\,И.&&\\
\Avtors{Севастьянов~Л.\,А.} см.\ Щетинин~Е.\,Ю.&&\\
\Avtors{Сигов~А.\,С., Андрианова~Е.\,Г., Истратов~Л.\,А.} Стохастическая динамика самоорганизу-\linebreak
\\[-12pt]
\hspace*{23pt}ющихся социальных систем с памятью (электоральные процессы)&2&112--121\\
\Avtors{Синицын~И.\,Н.} Нормальные субоптимальные фильтры для дифференциальных стоха-\linebreak
\\[-12pt]
\hspace*{23pt}стических систем, не разрешенных относительно производных&1&\hphantom{1}3--10\\
\end{tabular}
}

\pagebreak

\def\leftkol{АВТОРСКИЙ УКАЗАТЕЛЬ ЗА 2021 г.} % ENGLISH ABSTRACTS}

\def\rightkol{АВТОРСКИЙ УКАЗАТЕЛЬ ЗА 2021 г.} %ENGLISH ABSTRACTS}

%\thispagestyle{myheadings}
\def\leftfootline{\small{\textbf{\thepage}
\hfill ИНФОРМАТИКА И ЕЁ ПРИМЕНЕНИЯ\ \ \ том~15\ \ \ выпуск~4\ \ \ 2021}
}%
 \def\rightfootline{\small{ИНФОРМАТИКА И ЕЁ ПРИМЕНЕНИЯ\ \ \ том~15\ \ \ выпуск~4\ \ \ 2021
 \hfill \textbf{\thepage}}}


\noindent
{\tabcolsep=3pt
\begin{tabular}{p{394pt}cc}
&\textbf{Вып.} & \textbf{Стр.}\\[3pt]
\Avtors{Смирнов~Д.\,В.} см.\ Грушо~А.\,А.&&\\
\Avtors{Смирнов~Д.\,В.} см.\ Грушо~А.\,А.&&\\
\Avtors{Соколов~И.\,А., Степченков~Ю.\,А., Дьяченко~Ю.\,Г., Рождественский~Ю.\,В., Каменских~А.\,Н.}\linebreak
\\[-12pt]
\hspace*{23pt}Базис реализации сбоеустойчивых электронных схем&4&65--71\\
\Avtors{Сопин~Э.\,С.} см.\ Дараселия~А.\,В.&&\\
\Avtors{Степченков~Ю.\,А.} см.\ Соколов~И.\,А.&&\\
\Avtors{Стрижов~В.\,В.} см.\ Гребенькова~О.\,С.&&\\
\Avtors{Ступников~С.\,А.}см.\ Брюхов~Д.\,О.&&\\
\Avtors{Сушко~Д.\,В.} Алгоритмы сжатия данных массивов силовых кривых I: кодирование\linebreak
\\[-12pt]
\hspace*{23pt}ошибок предсказания&2&82--88\\
\Avtors{Сушко~Д.\,В.} Алгоритмы сжатия данных массивов силовых кривых II: кодирование\linebreak
\\[-12pt]
\hspace*{23pt}компонент вейвлет-преобразования&3&16--23\\
\Avtors{Тимонина~Е.\,Е.} см.\ Грушо~А.\,А.&&\\
\Avtors{Тимонина~Е.\,Е.} см.\ Грушо~А.\,А.&&\\
\Avtors{Тимонина~Е.\,Е.} см.\ Грушо~А.\,А.&&\\
\Avtors{Ушаков~В.\,Г., Ушаков~Н.\,Г.} Многомерные распределения выходящих потоков в системе\linebreak
\\[-12pt]
\hspace*{23pt}с абсолютным приоритетом&2&26--29\\
\Avtors{Ушаков~Н.\,Г.} см.\ Ушаков~В.\,Г.&&\\
\Avtors{Флёров~Ю.\,А.} см.\ Вышинский~Л.\,Л.&&\\
\Avtors{Флёров~Ю.\,А.} см.\ Вышинский~Л.\,Л.&&\\
\Avtors{Харин~П.\,А.} см.\ Кочеткова~И.\,А.&&\\
\Avtors{Хватова~Т.\,Ю.} см.\ Жуков~Д.\,О.&&\\
\Avtors{Чехович~Ю.\,В.} см.\ Кузнецова~Р.\,В.&&\\
\Avtors{Шанин~И.\,А.}см.\ Брюхов~Д.\,О.&&\\
\Avtors{Шестаков~О.\,В.} Анализ несмещенной оценки среднеквадратичного риска метода блоч-\linebreak
\\[-12pt]
\hspace*{23pt}ной пороговой обработки&2&30--35\\
\Avtors{Шестаков~О.\,В.} Пороговые функции в методах подавления шума, основанных на\linebreak
\\[-12pt]
\hspace*{23pt}вейвлет-разложении сигнала&3&51--56\\
\Avtors{Шестаков~О.\,В.} см.\ Кудрявцев~А.\,А.&&\\
\Avtors{Шестаков~О.\,В.} см.\ Кудрявцев~А.\,А.&&\\
\Avtors{Шнурков~П.\,В.} Создание стохастической динамической односекторной экономической модели с~дискретным временем и~анализ соответствующей задачи оптимального\linebreak
\\[-12pt]
\hspace*{23pt}управления&4&33--40\\
\Avtors{Шоргин~В.\,С.} см.\ Кочеткова~И.\,А.&&\\
\Avtors{Шоргин~С.\,Я.} см.\ Грушо~А.\,А.&&\\
\Avtors{Шоргин~С.\,Я.} см.\ Кочеткова~И.\,А.&&\\
\Avtors{Шоргин~С.\,Я.} см.\ Кудрявцев~А.\,А.&&\\
\Avtors{Щетинин~Е.\,Ю., Севастьянов~Л.\,А.} О методах переноса глубокого обучения в~задачах\linebreak
\\[-12pt]
\hspace*{23pt}классификации биомедицинских изображений&4&59--64\\
\end{tabular}
}

%\thispagestyle{myheadings}
\def\leftfootline{\small{\textbf{\thepage}
\hfill ИНФОРМАТИКА И ЕЁ ПРИМЕНЕНИЯ\ \ \ том~15\ \ \ выпуск~4\ \ \ 2021}
}%
 \def\rightfootline{\small{ИНФОРМАТИКА И ЕЁ ПРИМЕНЕНИЯ\ \ \ том~15\ \ \ выпуск~4\ \ \ 2021
 \hfill \textbf{\thepage}}}

 \label{end\stat}

\newpage

\def\stat{cont-e}
{%\hrule\par
%\vskip 7pt % 7pt
\raggedleft\Large \bf%\baselineskip=3.2ex
2\,0\,2\,1\ \ A\,U\,T\,H\,O\,R\ \ I\,N\,D\,E\,X \vskip 17pt
 \hrule
 \par
\vskip 21pt plus 6pt minus 3pt }

\label{st\stat}

\def\tit{\ }

\def\aut{\ }
\def\auf{\ }

\def\leftkol{\ } %2021 AUTHOR INDEX} % ENGLISH ABSTRACTS}

\def\rightkol{\ } %2021 AUTHOR INDEX} %ENGLISH ABSTRACTS}

\titele{\tit}{\aut}{\auf}{\leftkol}{\rightkol}
\addcontentsline{toc}{subsection}{\textrm\textbf 2021 Author Index}

\def\leftfootline{\small{\textbf{\thepage}
\hfill INFORMATIKA I EE PRIMENENIYA~--- INFORMATICS AND APPLICATIONS\ \ \ 2021\
\ \ volume~15\ \ \ issue\ 4}
}%
 \def\rightfootline{\small{INFORMATIKA I EE PRIMENENIYA~--- INFORMATICS AND APPLICATIONS\ \ \ 2021\ \ \ volume~15\ \ \ issue\ 4
\hfill \textbf{\thepage}}}

\vspace*{-24pt}

\noindent
{\tabcolsep=3pt
\begin{tabular}{p{395.89pt}cc}
&\textbf{Issue} & \textbf{Page}\\[6pt]
\Avtors{Abgaryan~K.\,K. and Gavrilov~E.\,S.} Distributed information system for calculating the structural\linebreak
\\[-12pt]
\hspace*{23pt}properties of composite materials&4&50--58\\
\Avtors{Agalarov~Ya.\,M.} Optimal threshold-based admission control in the $M/M/s$ system with hetero-\linebreak
\\[-12pt]
\hspace*{23pt}geneous servers and a common queue&1&57--64\\
\Avtors{Andrianova~E.\,G.} see Sigov~A.\,S.&&\\
\Avtors{Arutyunov~E.\,N., Kudryavtsev~A.\,A., and~Nedolivko~Iu.\,N.} Probabilistic characteristics of balance\linebreak
\\[-12pt]
\hspace*{23pt}index of factors with generalized gamma distribution&1&65--71\\
\Avtors{Bakhteev~O.\,Yu.} see Grebenkova~O.\,S.&&\\
\Avtors{Bakhteev~O.\,Yu.} see Kuznetsova~R.\,V.&&\\
\Avtors{Bazilevskiy~M.\,P.} Method of straightening distorted due to multicollinearity coefficients in\linebreak
\\[-12pt]
\hspace*{23pt}regression models&2&60--65\\
\Avtors{Borisov~A.\,V.\ and~Kazanchyan~D.\,Kh.} Filtering of Markov jump processes given composite\linebreak
\\[-12pt]
\hspace*{23pt}ob\-ser\-va\-tions I: Exact solution&2&12--19\\
\Avtors{Borisov~A.\,V.\ and~Kazanchyan~D.\,Kh.} Filtering of Markov jump processes given composite\linebreak
\\[-12pt]
\hspace*{23pt}observations II: Numerical algorithm&3&\hphantom{1}9--15\\
\Avtors{Bosov~A.\,V.} Linear output control of Markov chains by the quadratic criterion&2&\hphantom{1}3--11\\
\Avtors{Bosov~A.\,V.} On some special cases in the problem of stochastic differential system output control\linebreak
\\[-12pt]
\hspace*{23pt}by the quadratic criterion&1&11--17\\
\Avtors{Bosov~A.\,V., Ignatov~A.\,N., and Naumov~A.\,V.} Algorithms for an approximate solution of the\linebreak
\\[-12pt]
\hspace*{23pt}track possession problem on the railway network segment&4&\hphantom{1}3--11\\
\Avtors{Bosov~A.\,V.\ and~Zhukov~D.\,V.} Expert system for monitoring and forecasting of resource allocation\linebreak
\\[-12pt]
\hspace*{23pt}processes&3&29--40\\
\Avtors{Briukhov~D.\,O., Stupnikov~S.\,A., Kovalev~D.\,Yu., and~Shanin~I.\,A.} An architecture for distributed\linebreak
\\[-12pt]
\hspace*{23pt}data analysis problem solving in neurophysiology&1&78--85\\
\Avtors{Chekhovich~Yu.\,V.} see Kuznetsova~R.\,V.&&\\
\Avtors{Daraseliya~А.\,V., Sopin~E.\,S., Moltchanov~D.\,А., and~Samouylov~K.\,E.} Analysis of 5G NR base\linebreak
\\[-12pt]
\hspace*{23pt}stations offloading by means of NR-U technology&3&98--111\\
\Avtors{Diachenko~Yu.\,G.} see Sokolov~I.\,A.&&\\
\Avtors{Djukova~E.\,V. and Masliakov~G.\,O.} On the choice of partial orders on feature values sets in the\linebreak
\\[-12pt]
\hspace*{23pt}supervised classification problem&4&72--78\\
\Avtors{Dorofeeva~A.\,V.} see Korolev~V.\,Yu.&&\\
\Avtors{Egorova~A.\,Yu.} see Nuriev~V.\,A.&&\\
\Avtors{Flerov~Yu.\,A.} see Vyshinsky~L.\,L.&&\\
\Avtors{Flerov~Yu.\,A.} see Vyshinsky~L.\,L.&&\\
\Avtors{Gavrilov~E.\,S.} see Abgaryan~K.\,K.&&\\
\Avtors{Goncharenko~M.\,B.\ and~Zakharova~T.\,V.} Some properties of Gaussian mixtures and applications\linebreak
\\[-12pt]
\hspace*{23pt}to magnetoencephalography problems&2&44--51\\
\Avtors{Goncharov~A.\,A.\ and~Inkova~O.\,Yu.} Extracting knowledge about means of expression of\linebreak
\\[-12pt]
\hspace*{23pt}logical-semantic relations from the supracorpora database&2&\hphantom{1}96--103\\
\Avtors{Goncharov~A.\,A.\ and~Zatsman~I.\,M.} Structuring principles of electronic dictionary's entries&2&89--95\\
\Avtors{Goncharov~A.\,A., Zatsman~I.\,M., and~Kruzhkov~M.\,G.} Representation of new lexicographical\linebreak
\\[-12pt]
\hspace*{23pt}knowledge in dynamic classification systems&1&86--93\\
\Avtors{Goncharov~A.\,A., Zatsman~I.\,M., Kruzhkov~M.\,G., and Loshchilova~E.\,Yu.} Capturing evolution\linebreak
\\[-12pt]
\hspace*{23pt}of lexicographic knowledge in dynamic classification systems&4&41--49\\

\Avtors{Gorshenin~A.\,K.\ and~Kuzmin~V.\,Yu.} Method for improving accuracy of neural network forecasts\linebreak
\\[-12pt]
\hspace*{23pt}based on probability mixture models and its implementation as a digital service&3&63--74\\
\end{tabular}
}
\pagebreak

\def\leftfootline{\small{\textbf{\thepage}
\hfill INFORMATIKA I EE PRIMENENIYA~--- INFORMATICS AND APPLICATIONS\ \ \ 2021\
\ \ volume~15\ \ \ issue\ 4}
}%
 \def\rightfootline{\small{INFORMATIKA I EE PRIMENENIYA~---
INFORMATICS AND APPLICATIONS\ \ \ 2021\ \ \ volume~15\ \ \ issue\ 4
\hfill \textbf{\thepage}}}

\def\leftkol{2021 AUTHOR INDEX} % ENGLISH ABSTRACTS}

\def\rightkol{2021 AUTHOR INDEX} %ENGLISH ABSTRACTS}


\noindent
{\tabcolsep=3pt
\begin{tabular}{p{395.5pt}cc}
&\textbf{Issue} & \textbf{Page}\\[6pt]
\Avtors{Grebenkova~O.\,S., Bakhteev~O.\,Yu., and~Strijov~V.\,V.} Variational deep learning model optimi-\linebreak
\\[-12pt]
\hspace*{23pt}zation with complexity control&1&42--49\\[-.15pt]
\Avtors{Grinchenko~S.\,N.} Anthropogenic ``third'' nature: The relatively autonomous status of its artificial\linebreak
\\[-12pt]
\hspace*{23pt}intellectual subjects&4&110--114\\[-.15pt]
\Avtors{Grinchenko~S.\,N.} On the system hierarchy of artificial intelligence&1&111--115\\[-.15pt]
\Avtors{Grusho~A.\,A., Grusho~N.\,A., Zabezhailo~M.\,I., Smirnov~D.\,V., Timonina~E.\,E., and Shorgin~S.\,Ya.}\linebreak
\\[-12pt]
\hspace*{23pt}Statistics and clusters for detection of anomalous insertions in Big Data
en\-vi\-ron\-ment&4&79--86\\[-.15pt]
\Avtors{Grusho~A.\,A., Grusho~N.\,A., Zabezhailo~M.\,I., and~Timonina~E.\,E.} Remote monitoring of\linebreak
\\[-12pt]
\hspace*{23pt}workflows&3&2--8\\[-.15pt]
\Avtors{Grusho~A.\,A., Zabezhailo~M.\,I., Smirnov~D.\,V., and~Timonina~E.\,E.} Intelligent analysis of Big\linebreak
\\[-12pt]
\hspace*{23pt}Data extendible collections under the limits of process-real time&2&36--43\\[-.15pt]
\Avtors{Grusho~N.\,A.} see Grusho~A.\,A.&&\\[-.15pt]
\Avtors{Grusho~N.\,A.} see Grusho~A.\,A.&&\\[-.15pt]
\Avtors{Ignatov~A.\,N.} see Bosov~A.\,V.&&\\[-.15pt]
\Avtors{Inkova~O.\,Yu.~and Kruzhkov~M.\,G.} Structured definitions of discourse relations in the\linebreak
\\[-12pt]
\hspace*{23pt}Supracorpora Database of Connectives&4&27--32\\[-.15pt]
\Avtors{Inkova~O.\,Yu.} see Goncharov~A.\,A.&&\\[-.15pt]
\Avtors{Istratov~L.\,A.} see Sigov~A.\,S.&&\\[-.15pt]
\Avtors{Kamenskih~A.\,N.} see Sokolov~I.\,A.&&\\[-.15pt]
\Avtors{Kazanchyan~D.\,Kh.} see Borisov~A.\,V.&&\\[-.15pt]
\Avtors{Kazanchyan~D.\,Kh.} see Borisov~A.\,V.&&\\[-.15pt]
\Avtors{Kharin~P.\,A.} see Kochetkova~I.\,A.&&\\[-.15pt]
\Avtors{Khvatova~T.\,Yu.} see Zhukov~D.\,O.&&\\[-.15pt]
\Avtors{Kirikov~I.\,A.\ and~Listopad~S.\,V.} Coordination of agents' goals in cohesive hybrid intelligent\linebreak
\\[-12pt]
\hspace*{23pt}multiagent systems&2&66--71\\[-.15pt]
\Avtors{Kirikov~I.\,A.} see Rumovskaya~S.\,B.&&\\[-.15pt]
\Avtors{Kochetkova~I.\,A., Kushchazli~A.\,I., Kharin~P.\,A., and Shorgin~S.\,Ya.} Model for analyzing priority 
admission control of URLLC and eMBB communications in 5G networks as a~resource\linebreak
\\[-12pt]
\hspace*{23pt}queuing system&4&87--92\\[-.15pt]
\Avtors{Kochetkova~I.\,A., Vlaskina~A.\,S., Vu~N.\,N., and~Shorgin~V.\,S.} Queuing system with signals for\linebreak
\\[-12pt]
\hspace*{23pt}dynamic resource allocation for analyzing network slicing in 5G networks&3&91--97\\[-.15pt]
\Avtors{Konovalov~M.\,G.\ and~Razumchik~R.\,V.} Routing jobs to heterogeneous parallel queues using\linebreak
\\[-12pt]
\hspace*{23pt}distributed policy grandient algorithm&3&41--50\\[-.15pt]
\Avtors{Korolev~V.\,Yu.\ and~Dorofeeva~A.\,V.} On the accuracy of the normal approximation under the\linebreak
\\[-12pt]
\hspace*{23pt}violation of the normal convergence&1&116--121\\[-.15pt]
\Avtors{Kovalev~D.\,Yu.} see Briukhov~D.\,O.&&\\[-.15pt]
\Avtors{Kovalev~I.\,A.} see Zeifman~A.\,I.&&\\[-.15pt]
\Avtors{Kovalyov~S.\,P.} Methods of the category theory in digital design of heterogeneous cyber-physical\linebreak
\\[-12pt]
\hspace*{23pt}systems&1&23--29\\[-.15pt]
\Avtors{Krivenko~M.\,P.} Soft computing in problems of medical diagnostics&2&52--59\\[-.15pt]
\Avtors{Kruzhkov~M.\,G.} see Goncharov~A.\,A&&\\[-.15pt]
\Avtors{Kruzhkov~M.\,G.} see Goncharov~A.\,A.&&\\[-.15pt]
\Avtors{Kruzhkov~M.\,G.} see Inkova~O.\,Yu.&&\\[-.15pt]
\Avtors{Kudryavtsev~A.\,A. and Shestakov~O.\,V.} Minimax estimates of the loss function based on integral\linebreak
\\[-12pt]
\hspace*{23pt}error probabilities during threshold processing of wavelet coefficients&4&12--19\\[-.15pt]
\Avtors{Kudryavtsev~A.\,A., Shestakov~O.\,V., and~Shorgin~S.\,Ya.} A~method for estimating bent, shape and\linebreak
\\[-12pt]
\hspace*{23pt}scale parameters of the gamma-exponential distribution&3&57--62\\[-.15pt]
\Avtors{Kudryavtsev~A.\,A.} see Arutyunov~E.\,N.&&\\[-.15pt]
\Avtors{Kushchazli~A.\,I.} see Kochetkova~I.\,A.&&\\[-.15pt]
\Avtors{Kuzmin~V.\,Yu.} see Gorshenin~A.\,K.&&\\[-.15pt]
\Avtors{Kuznetsova~R.\,V., Bakhteev~O.\,Yu., and~Chekhovich~Yu.\,V.} Methods of cross-lingual text reuse\linebreak
\\[-12pt]
\hspace*{23pt}detection in large textual collections&1&30--41\\[-.15pt]
\Avtors{Lipatiev~A.\,A.} Nonasymptotic analysis of Bartlett--Nanda--Pillai statistic for high-dimensional\linebreak
\\[-12pt]
\hspace*{23pt}data&1&72--77\\
\end{tabular}
}
\pagebreak

\def\leftfootline{\small{\textbf{\thepage}
\hfill INFORMATIKA I EE PRIMENENIYA~--- INFORMATICS AND APPLICATIONS\ \ \ 2021\
\ \ volume~15\ \ \ issue\ 4}
}%
 \def\rightfootline{\small{INFORMATIKA I EE PRIMENENIYA~---
INFORMATICS AND APPLICATIONS\ \ \ 2021\ \ \ volume~15\ \ \ issue\ 4
\hfill \textbf{\thepage}}}

\def\leftkol{2021 AUTHOR INDEX} % ENGLISH ABSTRACTS}

\def\rightkol{2021 AUTHOR INDEX} %ENGLISH ABSTRACTS}


\noindent
{\tabcolsep=3pt
\begin{tabular}{p{395.5pt}cc}
&\textbf{Issue} & \textbf{Page}\\[6pt]
\Avtors{Listopad~S.\,V.} see Kirikov~I.\,A.&&\\
\Avtors{Loshchilova~E.\,Yu.} see Goncharov~A.\,A.&&\\
\Avtors{Malashenko~Yu.\,E.} Maximum internode flows at peak load of a multiuser network&3&24--28\\
\Avtors{Malashenko~Yu.\,E. and Nazarova~I.\,A.} Analysis of peak load distribution in the multiuser\linebreak
\\[-12pt]
\hspace*{23pt}network&4&20--26\\
\Avtors{Masliakov~G.\,O.} see Djukova~E.\,V.&&\\
\Avtors{Moltchanov~D.\,А.} see Daraseliya~А.\,V.&&\\
\Avtors{Monakhov~M.\,M.} Chebyshev--Edgeworth expansions for distributions of generalised Hotelling-\linebreak
\\[-12pt]
\hspace*{23pt}type statistics based on random size samples&2&72--81\\
\Avtors{Naumov~A.\,V.} see Bosov~A.\,V.&&\\
\Avtors{Nazarova~I.\,A.} see Malashenko~Yu.\,E.&&\\
\Avtors{Nedolivko~Iu.\,N.} see Arutyunov~E.\,N.&&\\
\Avtors{Nuriev~V.\,A.\ and~Egorova~A.\,Yu.} Methods of quality estimation for machine translation:\linebreak
\\[-12pt]
\hspace*{23pt}State-of-the-art&2&104--111\\
\Avtors{Pavlov~Yu.\,L.} Connectivity of configuration graphs in complex network models&1&18--22\\
\Avtors{Razumchik~R.\,V.} see Konovalov~M.\,G.&&\\
\Avtors{Rogdestvenski~Yu.\,V.} see Sokolov~I.\,A.&&\\
\Avtors{Rumovskaya~S.\,B.\ and~Kirikov~I.\,A.} Visual representation method for the conflict stimulation in\linebreak
\\[-12pt]
\hspace*{23pt}hybrid intelligent multiagent systems&3&75--82\\
\Avtors{Samouylov~K.\,E.} see Daraseliya~А.\,V.&&\\
\Avtors{Satin~Ya.\,A.} see Zeifman~A.\,I.&&\\
\Avtors{Sevastianov~L.\,A.} see Shchetinin~E.\,Yu.&&\\
\Avtors{Shanin~I.\,A.} see Briukhov~D.\,O.&&\\
\Avtors{Shchetinin~E.\,Yu. and Sevastianov~L.\,A.} On transfer learning methods in biomedical images\linebreak
\\[-12pt]
\hspace*{23pt}classification tasks&4&59--64\\
\Avtors{Shestakov~O.\,V.} Analysis of the unbiased mean-square risk estimate of the block thresholding\linebreak
\\[-12pt]
\hspace*{23pt}method&2&30--35\\
\Avtors{Shestakov~O.\,V.} Thresholding functions in the noise suppression methods based on the wavelet\linebreak
\\[-12pt]
\hspace*{23pt}expansion of the signal&3&51--56\\
\Avtors{Shestakov~O.\,V.} see Kudryavtsev~A.\,A.&&\\
\Avtors{Shestakov~O.\,V.} see Kudryavtsev~A.\,A.&&\\
\Avtors{Shnurkov~P.\,V.} Creation of a stochastic dynamic one-sector economic model with discrete time\linebreak
\\[-12pt]
\hspace*{23pt}and analysis of the corresponding optimal control problem&4&33--40\\
\Avtors{Shorgin~S.\,Ya.} see Grusho~A.\,A.&&\\
\Avtors{Shorgin~S.\,Ya.} see Kochetkova~I.\,A.&&\\
\Avtors{Shorgin~S.\,Ya.} see Kudryavtsev~A.\,A. &&\\
\Avtors{Shorgin~V.\,S.} see Kochetkova~I.\,A.&&\\
\Avtors{Sigov~A.\,S., Andrianova~E.\,G., and~Istratov~L.\,A.} Stochastic dynamics of self-organizing social\linebreak
\\[-12pt]
\hspace*{23pt}systems with memory (electoral processes)&2&112--121\\
\Avtors{Sinitsyn~I.\,N.} Normal suboptimal filtering for differential stochastic systems with unsolved\linebreak
\\[-12pt]
\hspace*{23pt}derivatives&1&\hphantom{1}3--10\\
\Avtors{Smirnov~D.\,V.} see Grusho~A.\,A.&&\\
\Avtors{Smirnov~D.\,V.} see Grusho~A.\,A.&&\\
\Avtors{Sokolov~I.\,A., Stepchenkov~Yu.\,A., Diachenko~Yu.\,G., Rogdestvenski~Yu.\,V., and Kamenskih~A.\,N.}\linebreak
\\[-12pt]
\hspace*{23pt}The electronic component base of failure resilience digital circuits&4&65--71\\
\Avtors{Sopin~E.\,S.} see Daraseliya~А.\,V.&&\\
\Avtors{Stepchenkov~Yu.\,A.} see Sokolov~I.\,A.&&\\
\Avtors{Strijov~V.\,V.} see Grebenkova~O.\,S.&&\\
\Avtors{Stupnikov~S.\,A.} see Briukhov~D.\,O.&&\\
\Avtors{Sushko~D.\,V.} Compression algorithms for force volume data I: Coding of prediction errors&2&82--88\\
\Avtors{Sushko~D.\,V.} Compression algorithms for force volume data II: Coding of wavelet transform\linebreak
\\[-12pt]
\hspace*{23pt}components&3&16--23\\
\Avtors{Timonina~E.\,E.} see Grusho~A.\,A.&&\\
\end{tabular}
}
\pagebreak

\def\leftfootline{\small{\textbf{\thepage}
\hfill INFORMATIKA I EE PRIMENENIYA~--- INFORMATICS AND APPLICATIONS\ \ \ 2021\
\ \ volume~15\ \ \ issue\ 4}
}%
 \def\rightfootline{\small{INFORMATIKA I EE PRIMENENIYA~---
INFORMATICS AND APPLICATIONS\ \ \ 2021\ \ \ volume~15\ \ \ issue\ 4
\hfill \textbf{\thepage}}}

\def\leftkol{2021 AUTHOR INDEX} % ENGLISH ABSTRACTS}

\def\rightkol{2021 AUTHOR INDEX} %ENGLISH ABSTRACTS}


\noindent
{\tabcolsep=3pt
\begin{tabular}{p{395.5pt}cc}
&\textbf{Issue} & \textbf{Page}\\[6pt]
\Avtors{Timonina~E.\,E.} see Grusho~A.\,A.&&\\
\Avtors{Timonina~E.\,E.} see Grusho~A.\,A.&&\\
\Avtors{Ushakov~N.\,G.} see Ushakov~V.\,G.&&\\
\Avtors{Ushakov~V.\,G.\ and~Ushakov~N.\,G.} The multivariate distributions of output streams in a queueing\linebreak
\\[-12pt]
\hspace*{23pt}system with preemptive repeat priority&2&26--29\\
\Avtors{Vlaskina~A.\,S.} see Kochetkova~I.\,A.&&\\
\Avtors{Vu~N.\,N.} see Kochetkova~I.\,A.&&\\
\Avtors{Vyshinsky~L.\,L.\ and~Flerov~Yu.\,A.} Information model of aircraft weight profile&1&50--56\\
\Avtors{Vyshinsky~L.\,L. and Flerov~Yu.\,A.} Theoretical foundation of formation of aircraft weight\linebreak
\\[-12pt]
\hspace*{23pt}appearance&4&\hphantom{1}93--102\\
\Avtors{Zabezhailo~M.\,I.} see Grusho~A.\,A.&&\\
\Avtors{Zabezhailo~M.\,I.} see Grusho~A.\,A.&&\\
\Avtors{Zabezhailo~M.\,I.} see Grusho~A.\,A.&&\\
\Avtors{Zakharova~T.\,V.} see Goncharenko~M.\,B.&&\\
\Avtors{Zaltcman~A.\,D.} see Zhukov~D.\,O.&&\\
\Avtors{Zatsman~I.\,M.} Forms representing new knowledge discovered in texts&3&83--90\\
\Avtors{Zatsman~I.\,M.} Problem-oriented updating of dictionary entries of bilingual dictionaries and\linebreak
\\[-12pt]
\hspace*{23pt}medical terminology: Comparative analysis&1&\hphantom{1}94--101\\
\Avtors{Zatsman~I.\,M.} The conception of creating WHO Hub for Pandemic and Epidemic Intelligence:\linebreak
\\[-12pt]
\hspace*{23pt}Keywords and their terminological analysis&4&103--109\\
\Avtors{Zatsman~I.\,M.} see Goncharov~A.\,A.&&\\
\Avtors{Zatsman~I.\,M.} see Goncharov~A.\,A.&&\\
\Avtors{Zatsman~I.\,M.} see Goncharov~A.\,A.&&\\
\Avtors{Zeifman~A.\,I., Satin~Ya.\,A., and~Kovalev~I.\,A.} On one nonstationary service model with\linebreak
\\[-12pt]
\hspace*{23pt}catastrophes and heavy tails&2&20--25\\
\Avtors{Zhukov~D.\,O., Khvatova~T.\,Yu., and~Zaltcman~A.\,D.} Modeling of the stochastic dynamics of changes in node states and percolation transitions in social networks with self-organization\linebreak
\\[-12pt]
\hspace*{23pt}and memory&1&102--110\\
\Avtors{Zhukov~D.\,V.} see Bosov~A.\,V.&&\\
\end{tabular}
}

%\thispagestyle{myheadings}
\def\leftfootline{\small{\textbf{\thepage}
\hfill INFORMATIKA I EE PRIMENENIYA~--- INFORMATICS AND APPLICATIONS\ \ \ 2021\
\ \ volume~15\ \ \ issue\ 4}
}%
 \def\rightfootline{\small{INFORMATIKA I EE PRIMENENIYA~---
INFORMATICS AND APPLICATIONS\ \ \ 2021\ \ \ volume~15\ \ \ issue\ 4
\hfill \textbf{\thepage}}}

 \label{end\stat}

\newpage


%\linebreak
%\\[-12pt]
%\hspace*{23pt}

%   \vspace*{-48pt}

\begin{center}
\vspace*{6pt}
\mbox{%
%\epsfxsize=50mm %56.519mm  
%\epsfbox{smu-1.eps} 

\epsfxsize=50mm %46.402 mm
\epsfbox{nec-rb.eps}
}
%\end{center}

\vspace*{9pt} %Академик


%   \begin{center}
\fbox{\large\textbf{Рустем Бадриевич Сейфуль-Мулюков}}\\[6pt]
\textbf{\large 1928--2020}
   \end{center}


   %\vspace*{2.5mm}

   \vspace*{5mm}

   \thispagestyle{empty}

%\

%\vspace*{-12pt}

  
      Редакция журнала <<Информатика и~её применения>> с глубоким 
      прискорбием сообщают, что 17~марта 2020~г.\ на 93-м~году жизни 
      скончался заведующий редакцией журнала, главный научный сотрудник Федерального исследовательского центра <<Информатика и~управление>> Российской академии наук
      Рустем Бадриевич Сейфуль-Мулюков.
           
     Всю свою жизнь Рустем Бадриевич посвятил служению науке. Закончив в~1956~г.\ аспирантуру Московского ордена Трудового Красного знамени Нефтяного института им.\ академика
     И.\,М.~Губкина, он прошел путь от заведующего отделом Института геологии зарубежных стран Министерства геологии СССР до заместителя директора ВИНИТИ
     АН СССР, доктора гео\-ло\-го-ми\-не\-ра\-ло\-ги\-че\-ских наук, профессора.
     
     С марта 2002~г.\ Рустем Бадриевич успешно применял свои знания и~организационный талант в ИПИ
     РАН (в~дальнейшем~--- ФИЦ ИУ РАН), в~котором руководил лабораторией и~отделом, занимающимися вопросами технологий информационной технической деятельности. 
Р.\,Б.~Сейфуль-Мулюков, являясь автором значительного количества научных трудов и~монографий по геологии, информационным технологиям и~теоретической информатике, осуществлял организацию издания монографий ИПИ РАН и~ФИЦ ИУ РАН, библиографий научных сотрудников Центра.
     
     Р.\,Б.~Сейфуль-Мулюков являлся заведующим редакцией журналов <<Информатика и~её применения>> и~<<Системы и~средства информатики>>, членом редколлегии журнала <<Системы и~средства информатики>>. Он вложил огромный вклад в становление и~развитие этих журналов, организацию их регистрации, функционирования, редактуры и~издания. Включение этих журналов в ряд отечественных и~зарубежных информационных баз и~систем цитирования во многом является его личной заслугой.
     
     На всех занимаемых должностях Рустем Бадриевич отличался высоким профессионализмом, преданностью делу и~вниманием к коллегам.
     
     \thispagestyle{empty}
     
     Рустема Бадриевича отличали доброта, отзывчивость, неиссякаемый
      оптимизм, простота и~сердечность.
     
     Коллеги Рустема Бадриевича запомнят его как многогранного в~своих увлечениях человека, живописца,
     эрудита и~энциклопедиста, интересующегося историей, литературой и~искусством.
     
     Выражаем глубокое
     соболезнование семье, родственникам, друзьям и~коллегам по работе в~связи с~тяжелой невосполнимой утратой.
     Светлый образ Рустема Бадриевича навсегда сохранится в~нашей памяти.
     

      

%\def\stat{cont}
{%\hrule\par
%\vskip 7pt % 7pt
\raggedleft\Large \bf%\baselineskip=3.2ex
А\,В\,Т\,О\,Р\,С\,К\,И\,Й\ \ У\,К\,А\,З\,А\,Т\,Е\,Л\,Ь\ \ З\,А\ \ 2\,0\,1\,0 г. \vskip 17pt
    \hrule
    \par
\vskip 21pt plus 6pt minus 3pt }

\label{st\stat}

\def\tit{\ }

\def\aut{\ }
\def\auf{\ }

\def\leftkol{\ } % ENGLISH ABSTRACTS}

\def\rightkol{\ } %АВТОРСКИЙ УКАЗАТЕЛЬ ЗА 2010 г.} %ENGLISH ABSTRACTS}

\titele{\tit}{\aut}{\auf}{\leftkol}{\rightkol}

\vspace*{-12pt}

{\tabcolsep=3pt
\begin{tabular}{p{388pt}rr}
&\textbf{Выпуск} & \textbf{Стр.}\\[6pt]
\hangindent=23pt\noindent\textbf{Арутюнян~А.\,Р.} Моделирование влияния деформаций отпечатков пальцев на 
точность\linebreak
\vspace*{-12pt}\\
\hspace*{23pt}дактилоскопической идентификации$\dotfill$&1&51\\
\hangindent=23pt\noindent\textbf{Архипов~О.\,П., Зыкова~З.\,П.} Интеграция гетерогенной информации о цветных 
пикселях\linebreak
\vspace*{-12pt}\\
\hspace*{23pt}и их цветовосприятии$\dotfill$&4&15\\
\hangindent=23pt\noindent\textbf{Баранов~С.\,И., Френкель~С.\,Л., Захаров~В.\,Н.} Полуформальная верификация 
цифрового устройства с конвейером, основанная на использовании алгоритмических машин\linebreak
\vspace*{-12pt}\\
\hspace*{23pt}состояния$\dotfill$&4&49\\
\textbf{Бекетова~И.\,В.} см.~Каратеев~С.\,Л.&&\\
\textbf{Белоусов~В.\,В.} см.~Синицын~И.\,Н.&&\\
\hangindent=23pt\noindent\textbf{Бенинг~В.\,Е., Королев~Р.\,А.} О предельном поведении мощностей критериев в 
случае\linebreak
\vspace*{-12pt}\\
\hspace*{23pt}распределения Лапласа$\dotfill$&2&63\\
\hangindent=23pt\noindent\textbf{Бенинг~В.\,Е., Сипина~А.\,В.} Асимптотическое разложение для мощности 
критерия,\linebreak
\vspace*{-12pt}\\
\hspace*{23pt}основанного на выборочной медиане, в случае распределения Лапласа$\dotfill$&1&18\\
\textbf{Бондаренко~А.\,В.} см.~Каратеев~С.\,Л.&&\\
\hangindent=23pt\noindent\textbf{Бородина~А.\,В., Морозов~Е.\,В.} Об оценивании асимптотики вероятности 
большого\linebreak
\vspace*{-12pt}\\
\hspace*{23pt}уклонения стационарной регенеративной очереди с одним прибором$\dotfill$&3&29\\
\hangindent=23pt\noindent\textbf{Бунтман~Н.\,В., Минель~Ж.-Л., Ле~Пезан~Д., Зацман~И.\,М.} Типология и 
компьютерное\linebreak
\vspace*{-12pt}\\
\hspace*{23pt}моделирование трудностей перевода$\dotfill$&3&77\\
\textbf{Визильтер~Ю.\,В.} см.~Каратеев~С.\,Л.&&\\
\hangindent=23pt\noindent\textbf{Гавриленко~С.\,В.} Оценки скорости сходимости распределений случайных сумм с 
безгранично делимыми индексами к нормальному закону$\dotfill$&4&81\\
\hangindent=23pt\noindent\textbf{Григорьева~М.\,Е., Шевцова~И.\,Г.} Уточнение неравенства 
Каца--Берри--Эссеена$\dotfill$&2&75\\
\hangindent=23pt\noindent\textbf{Грушо~А.\,А., Грушо~Н.\,А., Тимонина~Е.\,Е.} Поиск конфликтов в политиках 
безопасности: модель случайных графов$\dotfill$&3&38\\
\textbf{Грушо~Н.\,А.} см.~Грушо~А.\,А.&&\\
\hangindent=23pt\noindent\textbf{Гудков~В.\,Ю.} Математические модели изображения отпечатка пальца на основе 
описания линий$\dotfill$&1&58\\
\textbf{Гуртов~А.\,В.} см.~Лукьяненко~А.\,С.&&\\
\textbf{Желтов~С.\,Ю.} см.~Каратеев~С.\,Л.&&\\
\hangindent=23pt\noindent\textbf{Захаров~А.\,А., Серебряков~В.\,А.} Система управления электронной библиотекой 
LibMeta$\dotfill$&4&2\\
\textbf{Захаров~В.\,Н.} см.~Баранов~С.\,И.&&\\
\textbf{Захарова~Т.\,В.} см.~Матвеева~С.\,С.&&\\
\hangindent=23pt\noindent\textbf{Зацаринный~А.\,А., Чупраков~К.\,Г.} Некоторые аспекты выбора технологии для 
постро-\linebreak
\vspace*{-12pt}\\
\hspace*{23pt}ения систем отображения информации ситуационного центра$\dotfill$&3&59\\
\textbf{Зацман~И.\,М.} см.~Бунтман~Н.\,В.&&\\
\hangindent=23pt\noindent\textbf{Зейфман~А.\,И., Коротышева~А.\,В., Сатин~Я.\,А., Шоргин~С.\,Я.} Об 
устойчивости нестаци-\linebreak
\vspace*{-12pt}\\
\hspace*{23pt}онарных систем обслуживания с катастрофами$\dotfill$&3&9\\
\textbf{Зыкова~З.\,П.} см.~Архипов~О.\,П.&&\\
\hangindent=23pt\noindent\textbf{Илюшин~Г.\,Я., Соколов~И.\,А.} Организация управляемого доступа пользователей 
к\linebreak
\vspace*{-12pt}\\
\hspace*{23pt}разнородным ведомственным информационным ресурсам$\dotfill$&1&24\\
\hangindent=23pt\noindent\textbf{Кавагучи~Ю., Ульянов~В.\,В., Фуджикоши~Я.} Приближения для статистик, 
описывающих\linebreak
\vspace*{-12pt}\\
\hspace*{23pt}геометрические свойства данных большой размерности, с оценками 
ошибок$\dotfill$&1&12\\
\hangindent=23pt\noindent\textbf{Каратеев~С.\,Л., Бекетова~И.\,В., Ососков~М.\,В., Князь~В.\,А., 
Визильтер~Ю.\,В., Бондаренко~А.\,В., Желтов~С.\,Ю.} Автоматизированный контроль 
качества цифровых\linebreak
\vspace*{-12pt}\\
\hspace*{23pt}изображений для персональных документов$\dotfill$&1&65\\
\end{tabular}
}

\pagebreak

\def\leftkol{АВТОРСКИЙ УКАЗАТЕЛЬ ЗА 2010 г.} % ENGLISH ABSTRACTS}

\def\rightkol{АВТОРСКИЙ УКАЗАТЕЛЬ ЗА 2010 г.} %ENGLISH ABSTRACTS}

{\tabcolsep=3pt
\begin{tabular}{p{388pt}rr}
&\textbf{Выпуск} & \textbf{Стр.}\\[3pt]
\hangindent=23pt\noindent\textbf{Козеренко~Е.\,Б.} Лингвистические фильтры в статистических моделях машинного\linebreak
\vspace*{-12pt}\\
\hspace*{23pt}перевода$\dotfill$&2&83\\
\hangindent=23pt\noindent\textbf{Козеренко~Е.\,Б., Кузнецов~И.\,П.} Когнитивно-лингвистические представления в 
систе-\linebreak
\vspace*{-12pt}\\
\hspace*{23pt}мах обработки текстов$\dotfill$&3&69\\
\textbf{Князь~В.\,А.} см.~Каратеев~С.\,Л.&&\\
\hangindent=23pt\noindent\textbf{Колесников~А.\,В., Солдатов~С.\,А.} Алгоритм координации для гибридной 
интеллектуальной системы решения сложной задачи оперативно-производственного\linebreak
\vspace*{-12pt}\\
\hspace*{23pt}планирования$\dotfill$&4&61\\
\hangindent=23pt\noindent\textbf{Коновалов~М.\,Г.} О планировании потоков в системах вычислительных 
ресурсов$\dotfill$&2&3\\
\textbf{Конушин~А.\,С.} см.~Конушин~В.\,С.&&\\
\hangindent=23pt\noindent\textbf{Конушин~В.\,С., Кривовязь~Г.\,Р., Конушин~А.\,С.} Алгоритм распознавания людей 
в видео-\linebreak
\vspace*{-12pt}\\
\hspace*{23pt}последовательности по одежде$\dotfill$&1&74\\
\textbf{Корепанов~Э.\, Р.} см.~Синицын~И.\,Н.&&\\
\textbf{Королев~В.\,Ю.} см.~Соколов~И.\,А.&&\\
\textbf{Королев~Р.\,А.} см.~Бенинг~В.\,Е.&&\\
\textbf{Коротышева~А.\,В.} см.~Зейфман~А.\,И.&&\\
\hangindent=23pt\noindent\textbf{Кривенко~М.\,П.} Непараметрическое оценивание элементов байесовского 
клас\-си-\linebreak
\vspace*{-12pt}\\
\hspace*{23pt}фикатора$\dotfill$&2&13\\
\textbf{Кривовязь~Г.\,Р.} см.~Конушин~В.\,С.&&\\
\textbf{Крылов~А.\,С.} см.~Павельева~Е.\,А.&&\\
\hangindent=23pt\noindent\textbf{Крылов~В.\,А.} Моделирование и классификация многоканальных дистанционных\linebreak
\vspace*{-12pt}\\
\hspace*{23pt}изображений с использованием копул$\dotfill$&4&34\\
\hangindent=23pt\noindent\textbf{Крючин~О.\,В.} Разработка параллельных эвристических алгоритмов подбора 
весовых\linebreak
\vspace*{-12pt}\\
\hspace*{23pt}коэффициентов искусственной нейтронной сети$\dotfill$&2&53\\
\hangindent=23pt\noindent\textbf{Кудрявцев~А.\,А., Шоргин~С.\,Я.} Байесовские модели массового обслуживания и 
надеж-\linebreak
\vspace*{-12pt}\\
\hspace*{23pt}ности: характеристики среднего числа заявок в системе $M\vert M \vert 1\vert 
\infty$$\dotfill$&3&16\\
\hangindent=23pt\noindent\textbf{Кузнецов~А.\,А.} Связь между временными и структурно-топологическими 
характери-\linebreak
\vspace*{-12pt}\\
\hspace*{23pt}стиками диаграмм ритма сердца здоровых людей$\dotfill$&4&39\\
\textbf{Кузнецов~И.\,П.} см.~Козеренко~Е.\,Б.&&\\
\textbf{Ле~Пезан~Д.} см.~Бунтман~Н.\,В.&&\\
\hangindent=23pt\noindent\textbf{Лукьяненко~А.\,С., Морозов~Е.\,В., Гуртов~А.\,В.} Анализ сетевого протокола с общей 
функ-\linebreak
\vspace*{-12pt}\\
\hspace*{23pt}цией расширения окна передачи сообщения при конфликтах$\dotfill$&2&46\\
\hangindent=23pt\noindent\textbf{Лямин~О.\,О.} О предельном поведении мощностей критериев в случае обобщенного\linebreak
\vspace*{-12pt}\\
\hspace*{23pt}распределения Лапласа$\dotfill$&3&47\\
\hangindent=23pt\noindent\textbf{Маркин~А.\,В., Шестаков~О.\,В.} Асимптотики оценки риска при пороговой 
обработке\linebreak
\vspace*{-12pt}\\
\hspace*{23pt}вейвлет-вейглет коэффициентов в задаче томографии$\dotfill$&2&36\\
\hangindent=23pt\noindent\textbf{Матвеева~С.\,С., Захарова~Т.\,В.} Сети массового обслуживания с наименьшей 
длиной\linebreak
\vspace*{-12pt}\\
\hspace*{23pt}очереди$\dotfill$&3&22\\
\hangindent=23pt\noindent\textbf{Матюшенко~С.\,И.} Стационарные характеристики двухканальной системы 
обслужива-\linebreak
\vspace*{-12pt}\\
\hspace*{23pt}ния с переупорядочиванием заявок и распределениями фазового типа$\dotfill$&4&68\\
\textbf{Минель~Ж.-Л.} см.~Бунтман~Н.\,В.&&\\
\textbf{Морозов~Е.\,В.} см.~Бородина~А.\,В.&&\\
\textbf{Морозов~Е.\,В.} см.~Лукьяненко~А.\,С.&&\\
\textbf{Ососков~М.\,В.} см.~Каратеев~С.\,Л.&&\\
\hangindent=23pt\noindent\textbf{Павельева~Е.\,А., Крылов~А.\,С.} Поиск и анализ ключевых точек радужной 
оболочки\linebreak
\vspace*{-12pt}\\
\hspace*{23pt}глаза методом преобразования Эрмита$\dotfill$&1&79\\
\textbf{Печинкин~А.\,В.} см.~Френкель~С.\,Л.,&&\\
\hangindent=23pt\noindent\textbf{Протасов~В.\,И.} Составление субъективного портрета с использованием 
эволюционно-\linebreak
\vspace*{-12pt}\\
\hspace*{23pt}го морфинга и квалиметрия метода$\dotfill$&1&83\\
\hangindent=23pt\noindent\textbf{Рудаков~К.\,В., Торшин~И.\,Ю.} Вопросы разрешимости задачи распознавания 
вторичной\linebreak
\vspace*{-12pt}\\
\hspace*{23pt}структуры белка$\dotfill$&2&25\\
\textbf{Сатин~Я.\,А.} см.~Зейфман~А.\,И.&&\\
\hangindent=23pt\noindent\textbf{Сейфуль-Мулюков~Р.\,Б.} Нефть как носитель информации о своем 
происхождении,\linebreak
\vspace*{-12pt}\\
\hspace*{23pt}структуре и эволюции$\dotfill$&1&41\\
\end{tabular}
}

{\tabcolsep=3pt
\begin{tabular}{p{388pt}rr}
&\textbf{Выпуск} & \textbf{Стр.}\\[6pt]
\textbf{Семендяев~Н.\,Н.} см.~Синицын~И.\,Н.&&\\
\textbf{Серебряков~В.\,А.} см.~Захаров~А.\,А.&&\\
\textbf{Синицын~В.\,И.} см.~Синицын~И.\,Н.&&\\
\hangindent=23pt\noindent\textbf{Синицын~И.\,Н., Синицын~В.\,И., Корепанов~Э.\, Р., Белоусов~В.\,В., 
Семендяев~Н.\,Н.} Оперативное построение информационных моделей движения полюса 
Земли\linebreak
\vspace*{-12pt}\\
\hspace*{23pt}методами линейных и линеаризованных фильтров$\dotfill$&1&2\\
\textbf{Сипина~А.\,В.} см.~Бенинг~В.\,Е.&&\\
\hangindent=23pt\noindent\textbf{Соколов~И.\,А.} О работах заслуженного деятеля науки Российской Федерации 
И.\,Н.~Синицына в области информационных технологий и автоматизации (к 70-летию\linebreak
\vspace*{-12pt}\\
\hspace*{23pt}со дня рождения)$\dotfill$&3&84\\
\textbf{Соколов~И.\,А.} см.~Илюшин~Г.\,Я.&&\\
\hangindent=23pt\noindent\textbf{Соколов~И.\,А., Королев~В.\,Ю.} Предисловие$\dotfill$&2&2\\
\textbf{Солдатов~С.\,А.} см.~Колесников~А.\,В.&&\\
\hangindent=23pt\noindent\textbf{Степанов~С.\,Ю.} Использование координатного метода фрагментации 
коммутаторной\linebreak
\vspace*{-12pt}\\
\hspace*{23pt}нейронной сети для сокращения трафика$\dotfill$&2&57\\
\textbf{Тимонина~Е.\,Е.} см.~Грушо~А.\,А.&&\\
\textbf{Торшин~И.\,Ю.} см.~Рудаков~К.\,В.&&\\
\textbf{Ульянов~В.\,В.} см.~Кавагучи~Ю.&&\\
\textbf{Фазекаш~И.} см.~Чупрунов~А.\,Н.&&\\
\textbf{Френкель~С.\,Л.} см.~Баранов~С.\,И.&&\\
\hangindent=23pt\noindent\textbf{Френкель~С.\,Л., Печинкин~А.\,В.} Оценка времени самовосстановления в 
цифровых\linebreak
\vspace*{-12pt}\\
\hspace*{23pt}системах после сбоев, вызываемых переходными помехами$\dotfill$&3&2\\
\textbf{Фуджикоши~Я.} см.~Кавагучи~Ю.&&\\
\hangindent=23pt\noindent\textbf{Цискаридзе~А.\,К.} Математическая модель и метод восстановления позы человека 
по\linebreak
\vspace*{-12pt}\\
\hspace*{23pt}стереопаре силуэтных изображений$\dotfill$&4&27\\
\hangindent=23pt\noindent\textbf{Чупраков~К.\,Г.} К вопросу о размещении коллективных средств отображения в 
ситуа-\linebreak
\vspace*{-12pt}\\
\hspace*{23pt}ционном зале с заданными параметрами$\dotfill$&4&89\\
\textbf{Чупраков~К.\,Г.} см.~Зацаринный~А.\,А.&&\\
\hangindent=23pt\noindent\textbf{Чупрунов~А.\,Н., Фазекаш~И.} Законы повторного логарифма для числа 
безошибочных\linebreak
\vspace*{-12pt}\\
\hspace*{23pt}блоков при помехоустойчивом кодировании$\dotfill$&3&42\\
\textbf{Шевцова~И.\,Г.} см.~Григорьева~М.\,Е.&&\\
\hangindent=23pt\noindent\textbf{Шестаков~О.\,В.} Аппроксимация распределения оценки риска пороговой 
обработки вейвлет-коэффициентов нормальным распределением при использовании 
выбо-\linebreak
\vspace*{-12pt}\\
\hspace*{23pt}рочной дисперсии$\dotfill$&4&73\\
\textbf{Шестаков~О.\,В.} см.~Маркин~А.\,В.&&\\
\textbf{Шоргин~С.\,Я.} см.~Зейфман~А.\,И.&&\\
\textbf{Шоргин~С.\,Я.} см.~Кудрявцев~А.\,А.&&\\
\end{tabular}
}

%\thispagestyle{myheadings}
\def\leftfootline{\small{\textbf{\thepage}
\hfill ИНФОРМАТИКА И ЕЁ ПРИМЕНЕНИЯ\ \ \ том~4\ \ \ выпуск~4\ \ \ 2010}
}%
 \def\rightfootline{\small{ИНФОРМАТИКА И ЕЁ ПРИМЕНЕНИЯ\ \ \ том~4\ \ \ выпуск~4\ \ \ 2010
 \hfill \textbf{\thepage}}}
 \label{end\stat}
%
%Том 10 Выпуск 1-4 Год 2016

\def\stat{cont-e}
{%\hrule\par
%\vskip 7pt % 7pt
\raggedleft\Large \bf%\baselineskip=3.2ex
2\,0\,1\,6\ \ A\,U\,T\,H\,O\,R\ \ I\,N\,D\,E\,X \vskip 17pt
 \hrule
 \par
\vskip 21pt plus 6pt minus 3pt }

\label{st\stat}

\def\tit{\ }

\def\aut{\ }
\def\auf{\ }

\def\leftkol{\ } %2016 AUTHOR INDEX} % ENGLISH ABSTRACTS}

\def\rightkol{\ } %2016 AUTHOR INDEX} %ENGLISH ABSTRACTS}

\titele{\tit}{\aut}{\auf}{\leftkol}{\rightkol}

\def\leftfootline{\small{\textbf{\thepage}
\hfill INFORMATIKA I EE PRIMENENIYA~--- INFORMATICS AND APPLICATIONS\ \ \ 2016\
\ \ volume~10\ \ \ issue\ 4}
}%
 \def\rightfootline{\small{INFORMATIKA I EE PRIMENENIYA~--- INFORMATICS AND APPLICATIONS\ \ \ 2016\ \ \ volume~10\ \ \ issue\ 4
\hfill \textbf{\thepage}}}

\vspace*{-12pt}
\vspace*{-18pt}

{\tabcolsep=2.8pt
\begin{tabular}{p{382pt}cc}
&\textbf{Issue} & \textbf{Page}\\[6pt]
\Avtors{Agalarov~M.\,Ya.} see~Agalarov~Ya.\,M.&&\\
\Avtors{Agalarov~Ya.\,M., Agalarov~M.\,Ya., and
Shorgin~V.\,S.} About the optimal threshold of queue\linebreak
\\[-12pt]
\hspace*{23pt}length in a~particular problem of profit maximization
in the $M/G/1$ queuing system&2&70--79\\
\Avtors{Alexeyevsky~D.\,A.} BioNLP ontology extraction from 
a~restricted language corpus with\linebreak
\\[-12pt]
\hspace*{23pt}context-free grammars&1&119--128\\
\Avtors{Andreev~S.\,D.} see~Gaidamaka~Yu.\,V.&&\\
\Avtors{Andreev~S.\,D.} see~Ometov~A.\,Ya.&&\\
\Avtors{Arkhipov~O.\,P., Arkhipov~P.\,O., and Sidorkin~I.\,I.} The
option to create a~local coordinate\linebreak
\\[-12pt]
\hspace*{23pt}system for synchronization of selected images&3&91--97\\
\Avtors{Arkhipov~P.\,O.} see~Arkhipov~O.\,P.&&\\
\Avtors{Belousov~V.\,V.} see~Shnurkov~P.\,V.&&\\
\Avtors{Belousov~V.\,V.} see~Shnurkov~P.\,V.&&\\
\Avtors{Bening~V.\,E.} Calculation of~the~asymptotic deficiency
of~some statistical procedures based\linebreak
\\[-12pt]
\hspace*{23pt}on~samples with~random sizes&4&34--45\\
\Avtors{Borisov~A.\,V., Bosov~A.\,V., and Miller~G.\,B.} Modeling and
monitoring of VoIP connection&2&\hphantom{1}2--13\\
\Avtors{Bosov~A.\,V.} see~Borisov~A.\,V.&&\\
\Avtors{Briukhov~D.\,O.} see~Stupnikov~S.\,A.&&\\
\Avtors{Callaos~N.\,K.\ and Seyful-Mulyukov~R.\,B.} Complexity and
its information content&1&129--139\\
\Avtors{Chertok~A.\,V., Kadaner~A.\,I., Khazeeva~G.\,T., and
Sokolov~I.\,A.} Regime switching detection\linebreak
\\[-12pt]
\hspace*{23pt}for~the~Levy driven
Ornstein--Uhlenbeck process using CUSUM methods&4&46--56\\
\Avtors{Chichagov~V.\,V.} Asymptotic expansions of mean absolute
error of uniformly minimum variance unbiased and maximum likelihood
estimators on the one-parameter exponential\linebreak
\\[-12pt]
\hspace*{23pt}family model of lattice distributions&3&66--76\\
\Avtors{Danishevsky~V.\,I.} see~Kolesnikov A.\,V.&&\\
\Avtors{Fazliev~A.\,Z.} see~Kalinichenko~L.\,A.&&\\
\Avtors{Fedoseev~A.\,A.} What is behind the concept of ``knowledge in
small packages''&3&105--110\\
\Avtors{Gaidamaka~Yu.\,V., Andreev~S.\,D., Sopin~E.\,S.,
Samouylov~K.\,E., and Shorgin~S.\,Ya.} Interference analysis
of~the~device-to-device communications model with~regard to~a~signal\linebreak
\\[-12pt]
\hspace*{23pt}propagation environment&4&\hphantom{1}2--10\\
\Avtors{Gasilov~A.\,V.} see~Yakovlev~O.\,A.&&\\
\Avtors{Goncharov~A.\,V.\ and Strijov~V.\,V.} Metric time series
classification using weighted dynamic\linebreak
\\[-12pt]
\hspace*{23pt}warping relative to centroids of classes&2&36--47\\
\Avtors{Gordov~E.\,P.} see~Kalinichenko~L.\,A.&&\\
\Avtors{Gorshenin~A.\,K.} Concept of online service for stochastic
modeling of real processes&1&72--81\\
\Avtors{Gorshenin~A.\,K.} see~Shnurkov~P.\,V.&&\\
\Avtors{Gorshenin~A.\,K.} see~Shnurkov~P.\,V.&&\\
\Avtors{Grusho~A.\,A., Grusho~N.\,A., Zabezhailo~M.\,I., and
Timonina~E.\,E.} Integration of statistical and\linebreak
\\[-12pt]
\hspace*{23pt}deterministic methods for
analysis of information security&3&2--8\\
\Avtors{Grusho~A.\,A., Zabezhailo~M.\,I., and Zatsarinny~A.\,A.} On
the advanced procedure to reduce\linebreak
\\[-12pt]
\hspace*{23pt}calculation of Galois closures&4&\hphantom{1}96--104\\
\Avtors{Grusho~N.\,A.} see~Grusho~A.\,A.&&\\
\Avtors{Havanskov~V.\,A.} see~Minin~V.\,A.&&\\
\Avtors{Inkova~O.\,Yu.} see~Zatsman~I.\,M.&&\\
\Avtors{Isachenko~R.\,V.\ and Strijov~V.\,V.} Metric learning in
multiclass time series classification\linebreak
\\[-12pt]
\hspace*{23pt}problem&2&48--57\\
\end{tabular}
}
\pagebreak

\def\leftfootline{\small{\textbf{\thepage}
\hfill INFORMATIKA I EE PRIMENENIYA~--- INFORMATICS AND APPLICATIONS\ \ \ 2016\
\ \ volume~10\ \ \ issue\ 4}
}%
 \def\rightfootline{\small{INFORMATIKA I EE PRIMENENIYA~---
INFORMATICS AND APPLICATIONS\ \ \ 2016\ \ \ volume~10\ \ \ issue\ 4
\hfill \textbf{\thepage}}}

\def\leftkol{2016 AUTHOR INDEX} % ENGLISH ABSTRACTS}

\def\rightkol{2016 AUTHOR INDEX} %ENGLISH ABSTRACTS}


{\tabcolsep=2.83pt
\begin{tabular}{p{382pt}cc}
&\textbf{Issue} & \textbf{Page}\\[6pt]
\Avtors{Kadaner~A.\,I.} see~Chertok~A.\,V.&&\\[.255pt]
\Avtors{Kalinichenko~L.\,A., Volnova~A.\,A., Gordov~E.\,P.,
Kiselyova~N.\,N., Kovaleva~D.\,A., Malkov~O.\,Yu., Okladnikov~I.\,G.,
Podkolodnyy~N.\,L., Pozanenko~A.\,S., Ponomareva~N.\,V.,
Stupnikov~S.\,A.,} \textbf{and Fazliev~A.\,Z.} Data access challenges for data
intensive\linebreak
\\[-12pt]
\hspace*{23pt}research in Russia&1& 2--22\\[.255pt]
\Avtors{Karasikov~M.\,E.\ and Strijov~V.\,V.} Feature-based
time-series classification&4&121--131\\[.255pt]
\Avtors{Khazeeva~G.\,T.} see~Chertok~A.\,V.&&\\[.255pt]
\Avtors{Khokhlov~Yu.\,S.} Multivariate fractional Levy motion and its
applications&2&\hphantom{1}98--106\\[.255pt]
\Avtors{Kirikov~I.\,A., Kolesnikov~A.\,V., Listopad~S.\,V., and
Rumovskaya~S.\,B.} Fine-grained hybrid\linebreak
\\[-12pt]
\hspace*{23pt}intelligent systems. Part 2:
Bidirectional hybridization&1&\hphantom{1}96--105\\[.255pt]
\Avtors{Kirikov~I.\,A., Kolesnikov~A.\,V., Listopad~S.\,V., and
Rumovskaya~S.\,B.} ``Virtual council''~---\linebreak
\\[-12pt]
\hspace*{23pt}source environment
supporting complex diagnostic decision making&3&81--90\\[.255pt]
\Avtors{Kiselyova~N.\,N.} see~Kalinichenko~L.\,A.&&\\[.255pt]
\Avtors{Kolesnikov A.\,V., Listopad~S.\,V., Rumovskaya~S.\,B., and
Danishevsky~V.\,I.} Informal axiomatic\linebreak
\\[-12pt]
\hspace*{23pt}theory of~the~role visual models&4&114--120\\[.255pt]
\Avtors{Kolesnikov~A.\,V.} see~Kirikov~I.\,A.&&\\[.255pt]
\Avtors{Kolesnikov~A.\,V.} see~Kirikov~I.\,A.&&\\[.255pt]
\Avtors{Kolin~K.\,K.} Humanitarian aspects of information
security&3&111--121\\[.255pt]
\Avtors{Konovalov~M.\,G.\ and Razumchik~R.\,V.} Dispatching
to~two parallel nonobservable queues using\linebreak
\\[-12pt]
\hspace*{23pt}only static
information&4&57--67\\[.255pt]
\Avtors{Korchagin~A.\,Yu.} see~Korolev~V.\,Yu.&&\\[.255pt]
\Avtors{Korchagin~A.\,Yu.} see~Korolev~V.\,Yu.&&\\[.255pt]
\Avtors{Korepanov~E.\,R.} see~Sinitsyn~I.\,N.&&\\[.255pt]
\Avtors{Korepanov~E.\,R.} see~Sinitsyn~I.\,N.&&\\[.255pt]
\Avtors{Korolev~V.\,Yu., Korchagin~A.\,Yu., and Zeifman~A.\,I.} The
Poisson theorem for Bernoulli trials\linebreak
\\[-12pt]
\hspace*{23pt}with~a~random probability
of~success and~a~discrete analog of~the~Weibull distribution&4&11--20\\[.255pt]
\Avtors{Korolev~V.\,Yu., Zeifman~A.\,I., and Korchagin~A.\,Yu.}
Asymmetric Linnik distributions as~limit\linebreak
\\[-12pt]
\hspace*{23pt}laws for~random sums
of~independent random variables with~finite variances&4&21--33\\[.255pt]
\Avtors{Koucheryavy~E.\,A.} see~Ometov~A.\,Ya.&&\\[.255pt]
\Avtors{Kovaleva~D.\,A.} see~Kalinichenko~L.\,A.&&\\[.255pt]
\Avtors{Kovalyov~S.\,P.} Metaprogramming to increase
manufacturability of large-scale software-\linebreak
\\[-12pt]
\hspace*{23pt}intensive systems&1&56--66\\[.255pt]
\Avtors{Krivenko~M.\,P.} Significance tests of feature selection for
classification&3&32--40\\[.255pt]
\Avtors{Kruzhkov~M.\,G.} see~Zalizniak~Anna~A.&&\\[.255pt]
\Avtors{Kruzhkov~M.\,G.} see~Zatsman~I.\,M.&&\\[.255pt]
\Avtors{Kudryavtsev~A.\,A.} Bayesian queueing and reliability models:
\textit{A~priori} distributions with\linebreak
\\[-12pt]
\hspace*{23pt}compact support&1&67--71\\[.255pt]
\Avtors{Kudryavtsev~A.\,A.} Characteristics dependent on the balance
coefficient in Bayesian models\linebreak
\\[-12pt]
\hspace*{23pt}with compact support of \textit{a priori}
distributions&3&77--80\\[.255pt]
\Avtors{Kudryavtsev~A.\,A.\ and Palionnaia~S.\,I.} Bayesian recurrent
model of reliability growth:\linebreak
\\[-12pt]
\hspace*{23pt}Parabolic distribution of parameters&2&80--83\\[.255pt]
\Avtors{Kudryavtsev~A.\,A.\ and Titova~A.\,I.} Bayesian queuing
and~reliability models: Degenerate-\linebreak
\\[-12pt]
\hspace*{23pt}Weibull case&4&68--71\\[.255pt]
\Avtors{Leontyev~N.\,D.\ and Ushakov~V.\,G.} Analysis of a queueing
system with autoregressive arrivals\linebreak
\\[-12pt]
\hspace*{23pt}and nonpreemptive priority&3&15--22\\[.255pt]
\Avtors{Listopad~S.\,V.} see~Kirikov~I.\,A.&&\\[.255pt]
\Avtors{Listopad~S.\,V.} see~Kirikov~I.\,A.&&\\[.255pt]
\Avtors{Listopad~S.\,V.} see~Kolesnikov A.\,V.&&\\[.255pt]
\Avtors{Malkov~O.\,Yu.} see~Kalinichenko~L.\,A.&&\\[.255pt]
\Avtors{Markov~A.\,S., Monakhov~M.\,M., and
Ulyanov~V.\,V.} Generalized Cornish--Fisher expansions\linebreak
\\[-12pt]
\hspace*{23pt}for distributions of statistics based on samples
of random size&2&84--91\\[.255pt]
\Avtors{Melnikov~A.\,K.\ and Ronzhin~A.\,F.} Generalized statistical
method of~text analysis based\linebreak
\\[-12pt]
\hspace*{23pt}on~calculation of~probability distributions
of~statistical values&4&89--95\\
\end{tabular}
}
\pagebreak

\def\leftfootline{\small{\textbf{\thepage}
\hfill INFORMATIKA I EE PRIMENENIYA~--- INFORMATICS AND APPLICATIONS\ \ \ 2016\
\ \ volume~10\ \ \ issue\ 4}
}%
 \def\rightfootline{\small{INFORMATIKA I EE PRIMENENIYA~---
INFORMATICS AND APPLICATIONS\ \ \ 2016\ \ \ volume~10\ \ \ issue\ 4
\hfill \textbf{\thepage}}}

\def\leftkol{2016 AUTHOR INDEX} % ENGLISH ABSTRACTS}

\def\rightkol{2016 AUTHOR INDEX} %ENGLISH ABSTRACTS}


{\tabcolsep=3pt
\begin{tabular}{p{381pt}cc}
&\textbf{Issue} & \textbf{Page}\\[6pt]
\Avtors{Meykhanadzhyan~L.\,A.} Stationary characteristics of the finite
capacity queueing system with\linebreak
\\[-12pt]
\hspace*{23pt}inverse service order and generalized
probabilistic priority&2&123--131\\[.23pt]
\Avtors{Miller~G.\,B.} see~Borisov~A.\,V.&&\\[.23pt]
\Avtors{Minin~V.\,A., Zatsman~I.\,M., Havanskov~V.\,A., and
Shubnikov~S.\,K.} Intensity of citation of scientific publications in
inventions on information and computer technologies patented\linebreak
\\[-12pt]
\hspace*{23pt}in Russia by domestic and foreign applicants&2&107--122\\[.23pt]
\Avtors{Monakhov~M.\,M.} see~Markov~A.\,S.&&\\[.23pt]
\Avtors{Naumov~V.\,A.\ and Samouylov~K.\,E.} On relationship
between queuing systems with resources\linebreak
\\[-12pt]
\hspace*{23pt}and Erlang networks&3&\hphantom{1}9--14\\[.23pt]
\Avtors{Okladnikov~I.\,G.} see~Kalinichenko~L.\,A.&&\\[.23pt]
\Avtors{Ometov~A.\,Ya., Andreev~S.\,D., Turlikov~A.\,M., and
Koucheryavy~E.\,A.} Performance analysis of\linebreak
\\[-12pt]
\hspace*{23pt}a wireless data
aggregation system with contention for contemporary sensor
networks&3&23--31\\[.23pt]
\Avtors{Palionnaia~S.\,I.} see~Kudryavtsev~A.\,A.&&\\[.23pt]
\Avtors{Podkolodnyy~N.\,L.} see~Kalinichenko~L.\,A.&&\\[.23pt]
\Avtors{Ponomareva~N.\,V.} see~Kalinichenko~L.\,A.&&\\[.23pt]
\Avtors{Popkova~N.\,A.} see~Zatsman~I.\,M.&&\\[.23pt]
\Avtors{Pozanenko~A.\,S.} see~Kalinichenko~L.\,A.&&\\[.23pt]
\Avtors{Razumchik~R.\,V.} see~Konovalov~M.\,G.&&\\[.23pt]
\Avtors{Ronzhin~A.\,F.} see~Melnikov~A.\,K.&&\\[.23pt]
\Avtors{Rumovskaya~S.\,B.} see~Kirikov~I.\,A.&&\\[.23pt]
\Avtors{Rumovskaya~S.\,B.} see~Kirikov~I.\,A.&&\\[.23pt]
\Avtors{Rumovskaya~S.\,B.} see~Kolesnikov A.\,V.&&\\[.23pt]
\Avtors{Samouylov~K.\,E.} see~Gaidamaka~Yu.\,V.&&\\[.23pt]
\Avtors{Samouylov~K.\,E.} see~Naumov~V.\,A.&&\\[.23pt]
\Avtors{Serebryanskii~S.\,M.} see~Tyrsin~A.\,N.&&\\[.23pt]
\Avtors{Seyful-Mulyukov~R.\,B.} see~Callaos~N.\,K.&&\\[.23pt]
\Avtors{Shestakov~O.\,V.} Statistical properties of the denoising method
based on the stabilized hard\linebreak
\\[-12pt]
\hspace*{23pt}thresholding&2&65--69\\[.23pt]
\Avtors{Shestakov~O.\,V.} The strong law of large numbers for the risk
estimate in the problem of\linebreak
\\[-12pt]
\hspace*{23pt}tomographic image reconstruction from
projections with a correlated noise&3&41--45\\[.23pt]
\Avtors{Shestakov~O.\,V.} see~Zakharova~T.\,V.&&\\[.23pt]
\Avtors{Shnurkov~P.\,V., Gorshenin~A.\,K., and Belousov~V.\,V.}
Analytical solution of~the~optimal control\linebreak
\\[-12pt]
\hspace*{23pt}task of~a~semi-Markov
process with~finite set of~states&4&72--88\\[.23pt]
\Avtors{Shnurkov~P.\,V., Zasypko~V.\,V., Belousov~V.\,V., and
Gorshenin~A.\,K.} Development of the algorithm of numerical solution
of the optimal investment control problem\linebreak
\\[-12pt]
\hspace*{23pt}in the closed dynamical model of three-sector economy&1&82--95\\[.23pt]
\Avtors{Shorgin~S.\,Ya.} see~Gaidamaka~Yu.\,V.&&\\[.23pt]
\Avtors{Shorgin~V.\,S.} see~Agalarov~Ya.\,M.&&\\[.23pt]
\Avtors{Shubnikov~S.\,K.} see~Minin~V.\,A.&&\\[.23pt]
\Avtors{Sidorkin~I.\,I.} see~Arkhipov~O.\,P.&&\\[.23pt]
\Avtors{Sinitsyn~I.\,N.} Analytical modeling of processes in stochastic
systems with complex fractional\linebreak
\\[-12pt]
\hspace*{23pt}order Bessel nonlinearities&3&55--65\\[.23pt]
\Avtors{Sinitsyn~I.\,N.} Orthogonal supoptimal filters for nonlinear
stochastic systems on manifolds&1&34--44\\[.23pt]
\Avtors{Sinitsyn~I.\,N.\ and Korepanov~E.\,R.} Normal Pugachev
conditionally-optimal filters and extra-\linebreak
\\[-12pt]
\hspace*{23pt}polators for state linear stochastic systems&2&14--23\\[.23pt]
\Avtors{Sinitsyn~I.\,N.\ and Sinitsyn~V.\,I.} Analytical modeling of
distributions in stochastic systems on\linebreak
\\[-12pt]
\hspace*{23pt}manifolds based on ellipsoidal approximation&1&45--55\\[.23pt]
\Avtors{Sinitsyn~I.\,N., Sinitsyn~V.\,I., and
Korepanov~E.\,R.} Ellipsoidal suboptimal filters for nonlinear\linebreak
\\[-12pt]
\hspace*{23pt}stochastic systems on manifolds&2&24--35\\[.23pt]
\Avtors{Sinitsyn~V.\,I.} see~Sinitsyn~I.\,N.&&\\[.23pt]
\Avtors{Sinitsyn~V.\,I.} see~Sinitsyn~I.\,N.&&\\[.23pt]
\Avtors{Skvortsov~N.\,A.} see~Stupnikov~S.\,A.&&\\[.23pt]
\Avtors{Sokolov~I.\,A.} see~Chertok~A.\,V.&&\\
\end{tabular}
}
\pagebreak

\def\leftfootline{\small{\textbf{\thepage}
\hfill INFORMATIKA I EE PRIMENENIYA~--- INFORMATICS AND APPLICATIONS\ \ \ 2016\
\ \ volume~10\ \ \ issue\ 4}
}%
 \def\rightfootline{\small{INFORMATIKA I EE PRIMENENIYA~---
INFORMATICS AND APPLICATIONS\ \ \ 2016\ \ \ volume~10\ \ \ issue\ 4
\hfill \textbf{\thepage}}}

\def\leftkol{2016 AUTHOR INDEX} % ENGLISH ABSTRACTS}

\def\rightkol{2016 AUTHOR INDEX} %ENGLISH ABSTRACTS}


{\tabcolsep=3pt
\begin{tabular}{p{382pt}cc}
&\textbf{Issue} & \textbf{Page}\\[6pt]
\Avtors{Sopin~E.\,S.} see~Gaidamaka~Yu.\,V.&&\\
\Avtors{Strijov~V.\,V.} see~Goncharov~A.\,V.&&\\
\Avtors{Strijov~V.\,V.} see~Isachenko~R.\,V.&&\\
\Avtors{Strijov~V.\,V.} see~Karasikov~M.\,E.&&\\
\Avtors{Stupnikov~S.\,A., Briukhov~D.\,O., and Skvortsov~N.\,A.}
Co-lending systemic risk analysis over\linebreak
\\[-12pt]
\hspace*{23pt}heterogeneous data collections&1&23--33\\
\Avtors{Stupnikov~S.\,A.} see~Kalinichenko~L.\,A.&&\\
\Avtors{Suchkov~A.\,P.} see~Zatsarinny~A.\,A.&&\\
\Avtors{Timonina~E.\,E.} see~Grusho~A.\,A.&&\\
\Avtors{Titova~A.\,I.} see~Kudryavtsev~A.\,A.&&\\
\Avtors{Turlikov~A.\,M.} see~Ometov~A.\,Ya.&&\\
\Avtors{Tyrsin~A.\,N.\ and Serebryanskii~S.\,M.} Recognition of
dependences on the basis of inverse\linebreak
\\[-12pt]
\hspace*{23pt}mapping&2&58--64\\
\Avtors{Ulyanov~V.\,V.} see~Markov~A.\,S.&&\\
\Avtors{Ushakov~V.\,G.} Queueing system with working vacations and
hyperexponential input stream&2&92--97\\
\Avtors{Ushakov~V.\,G.} see~Leontyev~N.\,D.&&\\
\Avtors{Volnova~A.\,A.} see~Kalinichenko~L.\,A.&&\\
\Avtors{Yakovlev~O.\,A.\ and Gasilov~A.\,V.} Speeded-up stereo
matching using geodesic support weights&3&\hphantom{1}98--104\\
\Avtors{Zabezhailo~M.\,I.} see~Grusho~A.\,A.&&\\
\Avtors{Zabezhailo~M.\,I.} see~Grusho~A.\,A.&&\\
\Avtors{Zakharova~T.\,V.\ and Shestakov~O.\,V.} Precision analysis of
wavelet processing of aerodynamic\linebreak
\\[-12pt]
\hspace*{23pt}flow patterns&3&46--54\\
\Avtors{Zalizniak~Anna~A.\ and Kruzhkov~M.\,G.} Database
of~Russian impersonal verbal constructions&4&132--141\\
\Avtors{Zasypko~V.\,V.} see~Shnurkov~P.\,V.&&\\
\Avtors{Zatsarinny~A.\,A.\ and Suchkov~A.\,P.} Systems engineering
approaches to~the~establishment of\linebreak
\\[-12pt]
\hspace*{23pt}a~system for~decision support based
on~situational analysis&4&105--113\\
\Avtors{Zatsarinny~A.\,A.} see~Grusho~A.\,A.&&\\
\Avtors{Zatsman~I.\,M., Inkova~O.\,Yu., Kruzhkov~M.\,G., and
Popkova~N.\,A.} Representation of cross-\linebreak
\\[-12pt]
\hspace*{23pt}lingual knowledge about
connectors in supracorpora databases&1&106--118\\
\Avtors{Zatsman~I.\,M.} see~Minin~V.\,A.&&\\
\Avtors{Zeifman~A.\,I.} see~Korolev~V.\,Yu.&&\\
\Avtors{Zeifman~A.\,I.} see~Korolev~V.\,Yu.&&\\
\end{tabular}
}

%\thispagestyle{myheadings}
\def\leftfootline{\small{\textbf{\thepage}
\hfill INFORMATIKA I EE PRIMENENIYA~--- INFORMATICS AND APPLICATIONS\ \ \ 2016\
\ \ volume~10\ \ \ issue\ 4}
}%
 \def\rightfootline{\small{INFORMATIKA I EE PRIMENENIYA~---
INFORMATICS AND APPLICATIONS\ \ \ 2016\ \ \ volume~10\ \ \ issue\ 4
\hfill \textbf{\thepage}}}

 \label{end\stat}

\newpage

%\def\stat{rekl}
%\label{preobr}

%\def\tit{АКАДЕМИК ПУГАЧЁВ  ВЛАДИМИР СЕМЁНОВИЧ\\
%25.03.1911--25.03.1998}


%   \vspace*{-48pt}
%   \begin{center}\LARGE
%Академик Пугачёв  Владимир Семёнович\\ (25.03.1911--25.03.1998)
%   \end{center}
   
   %\vspace*{2.5mm}
   
   \begin{center}

{\prgsh\LARGE
ОБЪЯВЛЕНИЯ О КОНФЕРЕНЦИЯХ}

\end{center}
%\hrule

\vspace*{6pt}

   
   \vspace*{10mm}
   
   \thispagestyle{empty}

\noindent
\begin{tabular}{cc}
%\begin{center}
\multicolumn{1}{c}{\raisebox{-40pt}[0pt][0pt]{\mbox{%
\epsfxsize=33mm
\epsfbox{vspu.eps}
}}}
%\end{center}
&
\tabcolsep=0pt\begin{tabular}{c}
{\prg{\Large\textbf{XII Всероссийское совещание}}}\\[6pt]
{\prg{\Large\textbf{по проблемам управления}}}\\[12pt]
{\prg{\large 16--19 июня 2014~г.}}\\[6pt] 
{\prg{\large Институт проблем управления имени В.\,А.~Трапезникова РАН}}\\[6pt]
{\prg{\large Москва, Россия}}
\end{tabular}
\end{tabular}

\vspace*{60pt}

     
 { %\large    
 XII Всероссийское совещание по проблемам управления (ВСПУ XII), посвященное 75-летию 
Института проблем управления (ИПУ) имени В.\,А.~Трапезникова РАН, проводится 16--19~июня 
2014~г.\ 
в ИПУ РАН (г.~Москва, Россия). ВСПУ XII организуется ИПУ РАН при поддержке РФФИ, Отделения 
энергетики, машиностроения, механики и процессов управления Российской академии наук, 
Российского 
национального комитета по автоматическому управлению, Академии навигации и управ\-ле\-ния 
движением, 
Научного совета РАН по комплексным проблемам управления и автоматизации, Совета по 
мехатронике и робототехнике РАН. Официальный язык Совещания~--- русский.

\vspace*{24pt}
     
     \textbf{Направления работы}
     \begin{enumerate}[1.]
\item Теория систем управления
\item Управление подвижными объектами и навигация
\item Интеллектуальные системы управления
\item Управление в промышленности, транспортом и логистикой
\item Управление системами междисциплинарной природы
\item Средства измерения, вычислений и контроля в управлении
\item Системный анализ и принятие решений в задачах управления
\item Информационные технологии в управлении
\item Проблемы образования в области управления: современное содержание и технологии обучения
\end{enumerate}

\vspace*{24pt}

     Подробная информация о Совещании находится на сайте {\sf http://vspu2014.ipu.ru}. Срок 
окончательной подачи докладов через систему подачи докладов на сайте~--- \textbf{30~ноября} 
2013~г.
}

%\include{rekl-1}

%\end{document}

%\include{nekrolog-rb}


%\end{document}

%\include{IPPM-25}

\def\stat{cont-rus}
{%\hrule\par
%\vskip 7pt % 7pt
\vspace*{-24pt}
\raggedleft\Large \bf%\baselineskip=3.2ex
Правила подготовки рукописей  для публикации в журнале
<<Информатика~и~её~применения>> \vskip 8pt
    \hrule
    \par
\vskip 14pt plus 6pt minus 3pt }

\label{st\stat}

\def\tit{\ }

\def\aut{\ }
\def\auf{\ }

\def\leftkol{\ }
% Правила подготовки рукописей  для публикации в журнале
%<<Информатика и её применения>>

\def\rightkol{\ }
%Правила подготовки рукописей  для публикации в журнале
%<<Информатика и её применения>>}


\titele{\tit}{\aut}{\auf}{\leftkol}{\rightkol}


\vspace*{-60pt}
{ %\small

Журнал <<Информатика и её применения>>
публикует теоретические, обзорные и дискуссионные статьи,
посвященные научным исследованиям и разработкам в области
информатики и ее приложений.

Журнал издается на русском языке. По специальному решению
редколлегии отдельные статьи могут печататься на английском языке.

Тематика журнала охватывает следующие направления:
\begin{itemize}
\item теоретические основы информатики;\\[-15pt]
      \item
математические методы исследования сложных систем и процессов;\\[-15pt]
           \item
информационные системы и сети;\\[-15pt]
                \item
информационные технологии;\\[-15pt]
                     \item
архитектура и программное обеспечение вычислительных комплексов и сетей.\\[-15pt]
\end{itemize}


\noindent
\begin{enumerate}[1.]
\item В журнале печатаются статьи, содержащие результаты, ранее не опубликованные и
не предназначенные к одновременной публикации в других изданиях.

%Публикация не должна нарушать закон об авторских правах.
Публикация предоставленной автором(ами) рукописи не должна нарушать 
положений глав~69, 70 раздела~VII части~IV Гражданского кодекса, 
которые определяют права на результаты интеллектуальной деятельности 
и~средства индивидуализации, в~том числе авторские права, в~РФ.

Ответственность за нарушение авторских прав, в~случае предъявления претензий к~редакции журнала,  
несут авторы статей.



Направляя рукопись в редакцию, авторы сохраняют свои права на данную
рукопись и при этом передают учредителям и редколлегии журнала неисключительные права на
издание статьи на русском языке 
(или на языке статьи, если он отличен от рус\-ско\-го) и~на перевод ее на английский
язык, а~также на
ее распространение в России и за рубежом. 
Каждый автор должен представить в~редакцию подписанный 
с~его стороны <<Лицензионный договор о~передаче неисключительных прав 
на использование произведения>>, текст которого размещен по адресу 
{\sf http://www.ipiran.ru/publications/licence.doc}. 
Этот договор может быть пред\-став\-лен в~бумажном (в~2-х экз.)\ 
или в~электронном виде (отсканированная копия заполненного и~подписанного документа).




Редколлегия вправе запросить у авторов экспертное заключение о возможности
пуб\-ли\-ка\-ции пред\-став\-лен\-ной статьи в открытой печати.\\[-13.5pt]

\item К статье прилагаются данные автора (авторов) (см.\ п.~8). При наличии нескольких
авторов указывается фамилия автора, ответственного за переписку с редакцией.\\[-13.5pt]

\item Редакция журнала осуществляет экспертизу присланных статей в соответствии с
принятой в журнале процедурой рецензирования.

Возвращение рукописи на доработку не означает ее принятия к печати.

Доработанный вариант с ответом на замечания рецензента необходимо прислать в
редакцию.\\[-13.5pt]

\item Решение редколлегии о публикации статьи или ее отклонении сообщается авторам.

Редколлегия может также направить авторам текст рецензии на их статью. Дискуссия по
поводу отклоненных статей не ведется.\\[-13.5pt]

%\pagebreak

\item Редактура статей высылается авторам для просмотра. Замечания к редактуре должны
быть присланы авторами в кратчайшие сроки.\\[-13.5pt]

\item Рукопись предоставляется в электронном виде в форматах MS WORD (.doc или
.docx) или \LaTeX\  (.tex), дополнительно~--- в формате .pdf, на дискете, лазерном диске
или электронной почтой. Предоставление бумажной рукописи необязательно.\\[-13.5pt]

\item При подготовке рукописи в MS Word рекомендуется использовать следующие
настройки.

Параметры страницы:
формат~--- А4; ориентация~--- книжная; поля (см): внутри~--- 2,5, снаружи~--- 1,5,
сверху~--- 2, снизу~--- 2, от края до нижнего колонтитула~--- 1,3.

Основной текст: стиль~--- <<Обычный>>, шрифт~--- Times New Roman, размер~---
14~пунк\-тов, абзацный отступ~--- 0,5~см, 1,5~интервала, выравнивание~--- по ширине.

\pagebreak

\def\leftkol{Правила подготовки рукописей  для публикации в журнале
<<Информатика и её применения>>}

\def\rightkol{Правила подготовки рукописей  для публикации в журнале
<<Информатика и её применения>>}



Рекомендуемый объем рукописи~--- не свыше 10~страниц указанного формата.
При превышении указанного объема редколлегия вправе потребовать от 
автора сокращения объема рукописи.


Сокращения слов, помимо стандартных, не допускаются. Допускается минимальное
количество аббревиатур.


Все страницы рукописи нумеруются.

Шаблоны оформления представлены в интернете:

\noindent
 {\sf
http://www.ipiran.ru/journal/template\_iiep\_ssi\_2024.zip}\\[-14pt]

\item Статья должна содержать следующую информацию на {\bfseries\textit{русском и
английском языках}}:\\[-16pt]

\begin{itemize}
\item название статьи;\\[-15pt]
\item Ф.И.О.\ авторов, на английском можно только имя и фамилию;\\[-15pt]
\item место работы, с указанием почтового адреса организации и электронного адреса каждого
автора;\\[-15pt]
\item сведения об авторах, в соответствии с форматом, образцы которого
представлены на страницах:



\def\leftfootline{\small{\textbf{\thepage}
\hfill ИНФОРМАТИКА И ЕЁ ПРИМЕНЕНИЯ\ \ \ том\ 18\ \ \ выпуск\ 3\ \ \ 2024}
}%
 \def\rightfootline{\small{ИНФОРМАТИКА И ЕЁ ПРИМЕНЕНИЯ\ \ \ том\ 18\ \ \ выпуск\ 3\ \ \ 2024
\hfill \textbf{\thepage}}}



{\sf http://www.ipiran.ru/journal/issues/2013\_07\_01/authors.asp} и

{\sf http://www.ipiran.ru/journal/issues/2013\_07\_01\_eng/authors.asp};
\item аннотация (не менее 100~слов на каждом из языков). Аннотация~--- это краткое
резюме работы, которое может публиковаться отдельно. Она является основным
источником информации в~ин\-фор\-ма\-ци\-он\-ных системах и базах данных. Английская
аннотация должна быть оригинальной, может не быть дословным переводом русского
текста и должна быть написана хорошим английским языком. В~аннотации не должно
быть ссылок на литературу и, по возможности, формул;\\[-15pt]
\item ключевые слова~--- желательно из принятых в мировой
на\-уч\-но-тех\-ни\-че\-ской литературе тематических тезаурусов. Предложения не
могут быть ключевыми словами;\\[-15pt]
\item источники финансирования работы (ссылки на гранты, проекты,
поддерживающие организации и~т.\,п.).
\end{itemize}



%\pagebreak

\item  Требования к спискам литературы.\\[-14pt]

Ссылки на литературу в тексте статьи нумеруются (в квадратных скобках) и
располагаются в каждом из списков литературы в порядке  первых упоминаний. Если источник имеет DOI и/или EDN,
то их необходимо указывать.

Списки литературы представляются в двух вариантах:\\[-14pt]


\noindent
\begin{enumerate}[(1)]
\item \textbf{Список литературы к русскоязычной части}. Русские и английские
работы~---  на языке и в алфавите оригинала;\\[-14.5pt]
\item  \textbf{References}. Русские работы и работы на других языках~--- в латинской
транслитерации с переводом на английский язык; английские работы и работы на других
языках~--- на языке оригинала.
\end{enumerate}

Необходимо для составления списка ``References'' пользоваться размещенной на сайте
{\sf http://www. translit.net/ru/bgn/} бесплатной программой транслитерации русского
 текста в~латиницу. %, при этом в~за\-клад\-ке <<варианты\ldots>> следует выбратьопцию BGN.

Список литературы ``References'' приводится полностью отдельным блоком, повторяя все
позиции из списка литературы к русскоязычной части, независимо от того, имеются или
нет в нем иностранные источники. Если в списке литературы к русскоязычной части есть
ссылки на иностранные публикации, набранные латиницей, они полностью повторяются в
списке ``References''.

Ниже приведены примеры ссылок на различные виды публикаций в списке ``References''.

\def\leftfootline{\small{\textbf{\thepage}
\hfill ИНФОРМАТИКА И ЕЁ ПРИМЕНЕНИЯ\ \ \ том\ 18\ \ \ выпуск\ 3\ \ \ 2024}
}%
 \def\rightfootline{\small{ИНФОРМАТИКА И ЕЁ ПРИМЕНЕНИЯ\ \ \ том\ 18\ \ \ выпуск\ 3\ \ \ 2024
\hfill \textbf{\thepage}}}

{\small

\noindent
\textbf{Описание статьи из журнала:}

\Aue{Zagurenko, A.\,G., V.\,A.~Korotovskikh, A.\,A.~Kolesnikov, A.\,V.~Timonov, and D.\,V.~Kardymon}. 2008.
Tekhniko-ekonomicheskaya optimizatsiya dizayna gidrorazryva plasta [Technical and
economic optimization of the design
of hydraulic fracturing]. \textit{Neftyanoe hozyaystvo} [\textit{Oil Industry}] 11:54--57.

\Aue{Zhang, Z., and D.~Zhu}. 2008. Experimental research on the localized
electrochemical micromachining. \textit{Russ. J.~Electrochem.}  44(8):926--930.
{\sf doi:10.1134/S1023193508080077}.

\noindent
\textbf{Описание статьи из электронного журнала:}

\Aue{Swaminathan, V., E.~Lepkoswka-White, and B.\,P.~Rao}. 1999. Browsers or buyers in cyberspace? An
investigation of electronic factors influencing electronic exchange. \textit{JCMC}
5(2). Available at: {\sf http://www.ascusc.org/jcmc/vol5/issue2/} (accessed April~28, 2011).

\def\leftkol{Правила подготовки рукописей  для публикации в журнале
<<Информатика и её применения>>}

\def\rightkol{Правила подготовки рукописей  для публикации в журнале
<<Информатика и её применения>>}


\noindent
\textbf{Описание статьи из продолжающегося издания (сборника трудов):}

\Aue{Astakhov, M.\,V., and T.\,V.~Tagantsev}. 2006. Eksperimental'noe
issledovanie prochnosti soedineniy ``stal'--kompozit'' [Experimental study of
the strength of joints ``steel--composite'']. \textit{Trudy MGTU
``Matematicheskoe modelirovanie slozhnykh tekh\-ni\-che\-skikh sistem''}
[\textit{Bauman MSTU ``Mathematical Modeling of Complex Technical
Systems'' Proceedings}]. 593:125--130.


\pagebreak



\noindent
\textbf{Описание материалов конференций:}

\Aue{Usmanov, T.\,S., A.\,A.~Gusmanov, I.\,Z.~Mullagalin, R.\,Ju.~Muhametshina, A.\,N.~Chervyakova, and
A.\,V.~Sveshnikov}. 2007. Osobennosti proektirovaniya razrabotki mestorozhdeniy
s primeneniem gidrorazryva
plasta [Features of the design of field development with the use of hydraulic fracturing].
\textit{Trudy 6-go
Mezhdu\-na\-rod\-no\-go Simpoziuma ``Novye resursosberegayushchie tekhnologii nedropol'zovaniya i povysheniya
neftegazootdachi''} [\textit{6th  Symposium (International) ``New Energy Saving Subsoil Technologies and
the Increasing of the Oil and Gas Impact'' Proceedings}]. Moscow. 267--272.



\def\leftfootline{\small{\textbf{\thepage}
\hfill ИНФОРМАТИКА И ЕЁ ПРИМЕНЕНИЯ\ \ \ том\ 18\ \ \ выпуск\ 3\ \ \ 2024}
}%
 \def\rightfootline{\small{ИНФОРМАТИКА И ЕЁ ПРИМЕНЕНИЯ\ \ \ том\ 18\ \ \ выпуск\ 3\ \ \ 2024
\hfill \textbf{\thepage}}}



\noindent
\textbf{Описание книги (монографии, сборники):}



Lindorf, L.\,S., and L.\,G.~Mamikoniants, eds. 1972.
\textit{Ekspluatatsiya turbogeneratorov s neposredstvennym
okhlazhdeniem} [\textit{Operation of turbine generators with direct cooling}].
Moscow: Energy Publs. 352~p.


\Aue{Latyshev, V.\,N.} 2009. \textit{Tribologiya rezaniya. Kn.~1: Friktsionnye protsessy
pri rezanii metallov}
[\textit{Tribology of cutting. Vol.~1: Frictional processes in metal cutting}]. Ivanovo: Ivanovskii
State Univ. 108~p.

\def\leftkol{Правила подготовки рукописей  для публикации в журнале
<<Информатика и её применения>>}

\def\rightkol{Правила подготовки рукописей  для публикации в журнале
<<Информатика и её применения>>}

\noindent
\textbf{Описание переводной книги}
(в списке литературы к русскоязычной части необходимо указать:~/ Пер.\ с англ.~---
после названия книги, а в конце ссылки указать оригинал книги в круглых скобках):
\begin{enumerate}[1.]
\item  В русскоязычной части:

\def\leftfootline{\small{\textbf{\thepage}
\hfill ИНФОРМАТИКА И ЕЁ ПРИМЕНЕНИЯ\ \ \ том\ 18\ \ \ выпуск\ 3\ \ \ 2024}
}%
 \def\rightfootline{\small{ИНФОРМАТИКА И ЕЁ ПРИМЕНЕНИЯ\ \ \ том\ 18\ \ \ выпуск\ 3\ \ \ 2024
\hfill \textbf{\thepage}}}

\Au{Тимошенко С.\,П., Янг Д.\,Х., Уивер~У.}
Колебания в инженерном деле~/ Пер.\ с англ.~--- М.: Машиностроение, 1985. 472~с.
(\Au{Timoshenko~S.\,P., Young~D.\,H., Weaver~W.}
Vibration problems in engineering.~--- 4th ed.~--- New York, NY, USA: Wiley, 1974. 521~p.)\\[-13.5pt]
\item  В англоязычной части:

\Aue{Timoshenko, S.\,P., D.\,H.~Young, and W.~Weaver}.
1974. \textit{Vibration problems in engineering}. 4th ed. New York: 
Wiley. 521~p.
\end{enumerate}

\vspace*{-3pt}


\noindent
\textbf{Описание неопубликованного документа:}


\Aue{Latypov, A.\,R., M.\,M.~Khasanov, and V.\,A.~Baikov}.
2004 (unpubl.). Geologiya i~dobycha (NGT GiD) [Geology and production (NGT GiD)]. Certificate on official registration of the computer program
No.\,2004611198. 

\noindent
\textbf{Описание интернет-ресурса:}


Pravila tsitirovaniya istochnikov [Rules for the citing of sources]. Available at: {\sf
http://www.scribd.com/doc/1034528/} (accessed February~7, 2011).

%\pagebreak

\noindent
\textbf{Описание диссертации или автореферата диссертации:}

\Aue{Semenov, V.\,I.}
2003. Matematicheskoe modelirovanie plazmy v sisteme kompaktnyy tor [Mathematical
modeling of the plasma in the compact torus].  Moscow.  D.Sc.\ Diss. 272~p.

\Aue{Kozhunova, O.\,S.} 2009. Tekhnologiya razrabotki semanticheskogo
slovarya informatsionnogo monitoringa [Technology of development of
semantic dictionary of information monitoring system].  Moscow: IPI RAN. PhD Thesis. 23~p.


\noindent
\textbf{Описание ГОСТа:}

GOST 8.586.5-2005. 2007. Metodika vypolneniya izmereniy. Izmerenie raskhoda i~kolichestva zhidkostey i~gazov
s~pomoshch'yu standartnykh suzhayushchikh ustroystv [Method of measurement.
Measurement of flow rate and volume of liquids and gases by means of orifice devices]. Moscow:
Standardinform  Publs. 10~p.

\noindent
\textbf{Описание патента:}

\Aue{Bolshakov, M.\,V., A.\,V.~Kulakov, A.\,N.~Lavrenov, and M.\,V.~Palkin}.
2006. Sposob orientirovaniya po krenu letatel'nogo
apparata s opti\-che\-skoy golovkoy
samonavedeniya [The way to orient on the roll of aircraft with optical homing head].
Patent RF No.\,2280590.
}

\item Присланные в редакцию материалы авторам не возвращаются.\\[-13.5pt]

\item При отправке файлов по электронной почте просим придерживаться следующих
правил:
\begin{itemize}
\item указывать в поле subject (тема) название журнала и фамилию автора;\\[-13.5pt]
\item указывать в тексте письма название статьи, авторов и~журнал, в~который направляется статья;\\[-13.5pt]
\item использовать attach (присоединение);\\[-13.5pt]
\item в состав электронной версии статьи должны входить: файл, содержащий текст
статьи, и файл(ы), содержащий(е) иллюстрации.\\[-13.5pt]
\end{itemize}

\item Журнал <<Информатика и её применения>> является некоммерческим изданием.
Плата за публикацию не взимается, гонорар авторам не выплачивается.
\end{enumerate}



\def\leftfootline{\small{\textbf{\thepage}
\hfill ИНФОРМАТИКА И ЕЁ ПРИМЕНЕНИЯ\ \ \ том\ 18\ \ \ выпуск\ 3\ \ \ 2024}
}%
 \def\rightfootline{\small{ИНФОРМАТИКА И ЕЁ ПРИМЕНЕНИЯ\ \ \ том\ 18\ \ \ выпуск\ 3\ \ \ 2024
\hfill \textbf{\thepage}}}


\vspace*{-1mm}

\begin{center}

\textbf{Адрес редакции журнала <<Информатика и её применения>>:} \\




Москва 119333, ул.~Вавилова, д.~44, корп.~2, ФИЦ ИУ РАН\\[-10pt]

\

Тел.: +7\,(499)\,135-86-92\ \ Факс:  +7\,(495)\,930-45-05\\[-10pt]

 \

e-mail:   {\sf iiep@frccsc.ru} (Стригина Светлана Николаевна)\\[-10pt]

\

{\sf http://www.ipiran.ru/journal/issues/}
\end{center}
}


\def\leftkol{Правила подготовки рукописей  для публикации в журнале
<<Информатика и её применения>>}

\def\rightkol{Правила подготовки рукописей  для публикации в журнале
<<Информатика и её применения>>}


\def\leftfootline{\small{\textbf{\thepage}
\hfill ИНФОРМАТИКА И ЕЁ ПРИМЕНЕНИЯ\ \ \ том\ 18\ \ \ выпуск\ 3\ \ \ 2024}
}%
 \def\rightfootline{\small{ИНФОРМАТИКА И ЕЁ ПРИМЕНЕНИЯ\ \ \ том\ 18\ \ \ выпуск\ 3\ \ \ 2024
\hfill \textbf{\thepage}}} 
\def\stat{podg-e}
{%\hrule\par
%\vskip 7pt % 7pt
\vspace*{-24pt}
\raggedleft\Large \bf%\baselineskip=3.2ex
Requirements for manuscripts submitted to Journal
``Informatics~and~Applications'' \vskip 8pt
    \hrule
    \par
\vskip 21pt plus 6pt minus 3pt }

\label{st\stat}

\def\tit{\ }

\def\aut{\ }
\def\auf{\ }

\def\leftkol{\ }

\def\rightkol{\ }
%Requirements for manuscripts submitted to Journal
%``Informatics~and~Applications''}

\titele{\tit}{\aut}{\auf}{\leftkol}{\rightkol}

\def\leftfootline{\small{\textbf{\thepage}
\hfill INFORMATIKA I EE PRIMENENIYA~--- INFORMATICS AND APPLICATIONS\ \ \ 2019\
\ \ volume~13\ \ \ issue\ 4}
}%
 \def\rightfootline{\small{INFORMATIKA I EE PRIMENENIYA~--- INFORMATICS AND APPLICATIONS\ \ \ 2019\ \ \ volume~13\ \ \ issue\ 4
\hfill \textbf{\thepage}}}

\vspace*{-60pt}

{\small

\noindent
Journal ``Informatics and Applications'' (Inform.\ Appl.)
publishes theoretical, review, and discussion
articles on the research and development in the
field of informatics and its applications.

The journal is published in Russian.
By a special decision of the editorial
board, some articles can be published in English.


The topics covered include the following areas:
\begin{itemize}
               \item
     theoretical fundamentals of informatics; \\[-14pt]
\item
mathematical methods for studying complex systems and processes; \\[-14pt]
\item
information systems and networks;\\[-14pt]
\item
information technologies; and \\[-14pt]
\item
architecture and software of computational complexes and networks. \\[-14pt]
\end{itemize}

\noindent
\begin{enumerate}[1.]
\item The Journal publishes original articles which have not been published before and are not
intended for simultaneous publication in other editions. An article submitted to the Journal must not violate the
Copyright law. Sending the manuscript to the Editorial Board, the authors retain all rights of the
owners of the manuscript and transfer the nonexclusive rights to publish the article in Russian
(or the language of the article, if not Russian) and its distribution in Russia and abroad to the
Founders and the Editorial Board. Authors should submit a letter to the Editorial Board in the
following form:

{\bfseries\textit{Agreement on the transfer of rights to publish:}}

``\textit{We, the undersigned authors of the manuscript ``\ldots'', pass to the
Founder and the Editorial Board of the Journal ``Informatics and Applications''
the nonexclusive right to publish the manuscript of the article in Russian (or
in English) in both print and electronic versions of the Journal. We affirm
that this publication does not violate the Copyright of other persons or
organizations.}

\textit{Author(s) signature(s): (name(s), address(es), date).}

This agreement should be submitted in paper form or in the form of a scanned copy (signed by
the authors).


%The Editorial Board has the right to request from the authors an official expert conclusion that
%the submitted article has no secret data prohibited for publication. \\[-13.5pt]
\item
A submitted article should be attached with \textbf{the data on the author(s)} (see item~8). If
there are several authors, the contact person should be indicated who is responsible for
correspondence with the Editorial Board and other authors about revisions and final approval
of the proofs.\\[-13.5pt]

\item The Editorial Board of the Journal examines the article according to the established
reviewing procedure. If the authors receive their article for correction after reviewing, it does not
mean that the article is approved for publication. The corrected article should be sent to the
Editorial Board for the subsequent review and approval.\\[-13.5pt]

\item The decision on the article publication or its rejection is communicated to the authors. The
Editorial Board may also send the reviews on the submitted articles to the authors. Any
discussion upon the rejected articles is not possible.\\[-13.5pt]

\item The edited articles will be sent to the authors for proofread. The comments of the authors
to the edited text of the article should be sent to the Editorial Board as soon as possible.\\[-13.5pt]

\item The manuscript of the article should be presented electronically in the MS WORD (.doc or
.docx) or \LaTeX\ (.tex) formats, and additionally in the .pdf format. All documents
 may be sent
by e-mail or provided on a CD or diskette. A~hard copy submission is not necessary.\\[-13.5pt]

\item The recommended typesetting instructions for manuscript.

Pages parameters: format A4, portrait orientation, document margins (cm): left~--- 2.5, right~---
1.5, above~--- 2.0, below~--- 2.0, footer 1.3.

Text: font~---Times New Roman, font size~--- 14, paragraph indent~--- 0.5, line spacing~--- 1.5,
justified alignment.

The recommended manuscript size: not more than 15~pages of the specified format.
If the specified size exceeded, the editorial board is entitled to require the author
to reduce the manuscript.

Use only standard abbreviations. Avoid  abbreviations in the title and
abstract. The full term for which an abbreviation stands should precede
its first use in the text unless it is a standard unit of measurement.

All pages of the manuscript should be numbered.

The templates for the manuscript typesetting are presented on site: {\sf
http://www.ipiran.ru/journal/template.doc}.\\[-13.5pt]


%\def\leftkol{Requirements for manuscripts submitted to Journal
%``Informatics~and~Applications''}

\item The articles should enclose data both in \textbf{Russian and English}:
\begin{itemize}
\item title;\\[-13.5pt]
\item author's name and surname;\\[-13.5pt]
\item affiliation~--- organization, its address with ZIP code, city, country, and
official e-mail address;\\[-13.5pt]
\item data on authors according to the format: (see site)

{\sf http://www.ipiran.ru/journal/issues/2013\_07\_01/authors.asp}  and

{\sf  http://www.ipiran.ru/journal/issues/2013\_07\_01\_eng/authors.asp};\\[-13.5pt]

\pagebreak

\def\leftfootline{\small{\textbf{\thepage}
\hfill INFORMATIKA I EE PRIMENENIYA~--- INFORMATICS AND APPLICATIONS\ \ \ 2019\
\ \ volume~13\ \ \ issue\ 4}
}%
 \def\rightfootline{\small{INFORMATIKA I EE PRIMENENIYA~--- INFORMATICS AND APPLICATIONS\ \ \ 2019\ \ \ volume~13\ \ \ issue\ 4
\hfill \textbf{\thepage}}}


%\def\leftkol{Requirements for manuscripts submitted to Journal
%``Informatics~and~Applications''}

%\def\rightkol{Requirements for manuscripts submitted to Journal
%``Informatics~and~Applications''}



\item abstract (not less than 100 words) both in Russian and in English. Abstract is a short
summary of the article that can be published separately. The abstract is the
main source of information on the article and it could be included in leading information
systems and data bases. The abstract in English has to be an original text and should
not be an exact translation of the Russian one. Good English is required.
In abstracts, avoid references and formulae;\\[-13.5pt]
\item indexing is performed on the basis of keywords. The use of keywords from the
internationally accepted thematic Thesauri is recommended.

%\def\leftkol{Requirements for manuscripts submitted to Journal
%``Informatics~and~Applications''}

%\def\rightkol{Requirements for manuscripts submitted to Journal
%``Informatics~and~Applications''}

Important! Keywords must not be sentences;
\item Acknowledgments.
\end{itemize}

\item References. Russian references have to be presented both in English translation and Latin
transliteration (refer {\sf http://www.translit.net/ru/bgn/}).

Please take into account the following examples of Russian references appearance:

\noindent
\textbf{Article in journal:}

\Aue{Zhang, Z., and D.~Zhu}. 2008. Experimental research on the localized electrochemical
micromachining.
\textit{Rus. J.~Electrochem.}  44(8):926--930. {\sf doi:10.1134/S1023193508080077}.


\noindent
\textbf{Journal article in electronic format:}

\Aue{Swaminathan, V., E.~Lepkoswka-White, and B.\,P.~Rao}. 1999. Browsers or buyers in
cyberspace? An
investigation of electronic factors influencing electronic exchange. \textit{JCMC}
5(2). Available at: {\sf http://www.ascusc.org/jcmc/vol5/issue2/} (accessed April~28, 2011).




\noindent
\textbf{Article from the continuing publication (collection of works, proceedings):}

\Aue{Astakhov, M.\,V., and T.\,V.~Tagantsev}. 2006. Eksperimental'noe
issledovanie prochnosti soedineniy ``stal'--kompozit'' [Experimental study of
the strength of joints ``steel--composite'']. \textit{Trudy MGTU
``Matematicheskoe modelirovanie slozhnykh tekh\-ni\-che\-skikh sistem''}
[\textit{Bauman MSTU ``Mathematical Modeling of Complex Technical
Systems'' Proceedings}]. 593:125--130.

\def\leftfootline{\small{\textbf{\thepage}
\hfill INFORMATIKA I EE PRIMENENIYA~--- INFORMATICS AND APPLICATIONS\ \ \ 2019\
\ \ volume~13\ \ \ issue\ 4}
}%
 \def\rightfootline{\small{INFORMATIKA I EE PRIMENENIYA~--- INFORMATICS AND APPLICATIONS\ \ \ 2019\ \ \ volume~13\ \ \ issue\ 4
\hfill \textbf{\thepage}}}

\def\leftkol{Requirements for manuscripts submitted to Journal
``Informatics~and~Applications''}

\def\rightkol{Requirements for manuscripts submitted to Journal
``Informatics~and~Applications''}

\noindent
\textbf{Conference proceedings:}

\Aue{Usmanov, T.\,S., A.\,A.~Gusmanov, I.\,Z.~Mullagalin, R.\,Ju.~Muhametshina,
A.\,N.~Chervyakova, and
A.\,V.~Sveshnikov}. 2007. Osobennosti proektirovaniya razrabotki mestorozhdeniy
s primeneniem gidrorazryva
plasta [Features of the design of field development with the use of hydraulic fracturing].
\textit{Trudy 6-go
Mezhdu\-na\-rod\-no\-go Simpoziuma ``Novye resursosberegayushchie tekhnologii
nedropol'zovaniya i povysheniya
neftegazootdachi''} [\textit{6th  Symposium (International) ``New Energy Saving Subsoil
Technologies and
the Increasing of the Oil and Gas Impact'' Proceedings}]. Moscow. 267--272.


\noindent
\textbf{Books and other monographs:}




Lindorf, L.\,S., and L.\,G.~Mamikoniants, eds. 1972.
\textit{Ekspluatatsiya turbogeneratorov s neposredstvennym
okhlazhdeniem} [\textit{Operation of turbine generators with direct cooling}].
Moscow: Energy Publs. 352~p.


%\Aue{Latyshev, V.\,N.} 2009. \textit{Tribologiya rezaniya. Kn.~1: Frikcionnye prosessy
%pri rezanii metallov}
%[\textit{Tribology of cutting. Vol.~1: Frictional processes in metal cutting}]. Ivanovo: Ivanovskii
%State Univ. 108~p.


%\noindent
%\textbf{Unpublished material:}

%\Aue{Latypov, A.\,R., M.\,M.~Khasanov, and V.\,A.~Baikov}.
%2004. Geology and production (NGT GiD). Certificate on official registration of the computer
%program
%No.\,2004611198. (In Russian, unpubl.)

%\noindent
%\textbf{Internet-source:}

%APA Style. 2011. Available at: {\sf http://www.apastyle.org/apa-style-help.aspx} (accessed
%February~5, 2011).

%Pravila citirovaniya istochnikov [Rules for the citing of sources]. Available at: {\sf
%http://www.scribd.com/doc/1034528/} (accessed February~7, 2011).


\noindent
\textbf{Dissertation and Thesis:}

%\Aue{Semenov, V.\,I.}
%2003. Matematicheskoe modelirovanie plazmy v sisteme kompaktnyy tor. [Mathematical
%modeling of the plasma in the compact torus]. D.Sc.\ Diss. Moscow. 272~p.

\Aue{Kozhunova, O.\,S.} 2009. Tekhnologiya razrabotki semanticheskogo
slovarya informatsionnogo monitoringa [Technology of development of
semantic dictionary of information monitoring system]. PhD Thesis. Moscow: IPI RAN. 23~p.


\noindent
\textbf{State standards and patents:}

GOST 8.586.5-2005. 2007. Metodika vypolneniya izmereniy. Izmerenie raskhoda i~kolichestva
zhidkostey i gazov 
s~pomoshch'yu standartnykh suzhayushchikh ustroystv [Method of measurement.
Measurement of flow rate and volume of liquids and gases by means of orifice devices]. M.:
Standardinform
Publs. 10~p.

%\noindent
%\textbf{Patent:}

\Aue{Bolshakov, M.\,V., A.\,V.~Kulakov, A.\,N.~Lavrenov, and M.\,V.~Palkin}.
2006. Sposob orientirovaniya po krenu letatel'nogo
apparata s opti\-che\-skoy golovkoy
samonavedeniya [The way to orient on the roll of aircraft with optical homing head].
Patent RF No.\,2280590.

References in Latin transcription are presented in the original language.

References in the text are numbered according to the order of their
first appearance; the number is
placed in square brackets. All items from the reference list should be
cited.\\[-13.5pt]

\item Manuscripts and additional materials are not returned to Authors by the Editorial Board.\\[-13.5pt]

\item Submissions of files by e-mail must include:\\[-13.5pt]
\begin{itemize}
\item   the journal title and author's name in the ``Subject'' field; \\[-13.5pt]
\item   an article and additional materials have to be attached using the ``attach'' function;\\[-13.5pt]
\item   an electronic version of the article should contain the file with the text and a separate file
with figures.\\[-13.5pt]
\end{itemize}

\item ``Informatics and Applications'' journal is not a profit publication. There are no
charges for the authors as well as there are no royalties.\\[-13.5pt]
\end{enumerate}

\def\leftfootline{\small{\textbf{\thepage}
\hfill INFORMATIKA I EE PRIMENENIYA~--- INFORMATICS AND APPLICATIONS\ \ \ 2019\
\ \ volume~13\ \ \ issue\ 4}
}%
 \def\rightfootline{\small{INFORMATIKA I EE PRIMENENIYA~--- INFORMATICS AND APPLICATIONS\ \ \ 2019\ \ \ volume~13\ \ \ issue\ 4
\hfill \textbf{\thepage}}}

\def\leftkol{Requirements for manuscripts submitted to Journal
``Informatics~and~Applications''}

\def\rightkol{Requirements for manuscripts submitted to Journal
``Informatics~and~Applications''}


%\vspace*{5mm}


\begin{center}
\textbf{Editorial Board address:} \\

%ABOUT AUTHORS



FRC CSC RAS, 44, block~2, Vavilov Str., Moscow 119333, Russia\\[-10pt]

\

Ph.: +7\,(499)\,135\,86\,92,\ \ Fax: +7\,(495)\,930\,45\,05\\[-10pt]

\

 e-mail: {\sf rust@ipiran.ru} (to Prof.\ Rustem Seyful-Mulyukov)\\[-10pt]

\

 {\sf http://www.ipiran.ru/english/journal.asp}
\end{center}
 }
%\thispagestyle{myheadings}

\def\leftkol{Requirements for manuscripts submitted to Journal
``Informatics~and~Applications''}

\def\rightkol{Requirements for manuscripts submitted to Journal
``Informatics~and~Applications''}

\def\leftfootline{\small{\textbf{\thepage}
\hfill INFORMATIKA I EE PRIMENENIYA~--- INFORMATICS AND APPLICATIONS\ \ \ 2019\
\ \ volume~13\ \ \ issue\ 4}
}%
 \def\rightfootline{\small{INFORMATIKA I EE PRIMENENIYA~--- INFORMATICS AND APPLICATIONS\ \ \ 2019\ \ \ volume~13\ \ \ issue\ 4
\hfill \textbf{\thepage}}}

 \label{end\stat}

\newpage

%\vspace*{-60pt} {\small
{\baselineskip=9.1pt
\section*{Правила подготовки рукописей статей для публикации в журнале
<<Информатика и её применения>>}

\thispagestyle{empty}

 Журнал <<Информатика и её применения>> публикует
теоретические, обзорные и дискуссионные статьи, посвященные научным
исследованиям и разработкам в области информатики и ее приложений. Журнал
издается на русском языке. По специальному решению редколлегии отдельные статьи,
в виде исключения, могут печататься на английском языке.
Тематика журнала охватывает следующие направления:
\begin{itemize}
\item теоретические основы информатики; %\\[-13.5pt]
\item математические методы исследования сложных систем и процессов; %\\[-13.5pt]
\item информационные системы и сети; %\\[-13.5pt]
\item информационные технологии; %\\[-13.5pt]
\item архитектура и программное
обеспечение вычислительных комплексов и сетей.
\end{itemize}
\begin{enumerate}
\item В журнале печатаются результаты, ранее не
опубликованные и не предназначенные к одновременной публикации в других
изданиях. Публикация не должна нарушать закон об авторских правах. Направляя
свою рукопись в редакцию, авторы автоматически передают учредителям и
редколлегии неисключительные права на издание данной статьи на русском языке и
на ее распространение в России и за рубежом. При этом за авторами сохраняются
все права как собственников данной рукописи. В связи с этим авторами должно
быть представлено в редакцию письмо в следующей форме:
Соглашение о передаче права на публикацию:

\textit{<<Мы, нижеподписавшиеся, авторы рукописи <<$\qquad\qquad$>>, передаем
учредителям и редколлегии журнала <<Информатика и её применения>>
неисключительное право опубликовать данную рукопись статьи на русском языке как
в печатной, так и в электронной версиях журнала. Мы подтверждаем, что данная
публикация не нарушает авторского права других лиц или организаций. Подписи
авторов: (ф.\,и.\,о., дата, адрес)>>.}

Указанное соглашение может быть представлено 
как в бумажном виде, так и в виде отсканированной копии (с подписями авторов).


Редколлегия вправе запросить у авторов экспертное заключение о возможности
опубликования представленной статьи в открытой печати. %\\[-13.5pt]
\item Статья
подписывается всеми авторами. На отдельном листе представляются данные автора
(или всех авторов): фамилия, полные имя и отчество, телефон, факс, e-mail,
почтовый адрес. Если работа выполнена несколькими авторами, указывается фамилия
одного из них, ответственного за переписку с редакцией. %\\[-13.5pt]
\item Редакция журнала
осуществляет самостоятельную экспертизу присланных статей. Возвращение рукописи
на доработку не означает, что статья уже принята к печати. Доработанный вариант
с ответом на замечания рецензента необходимо прислать в редакцию. %\\[-13.5pt]
\item Решение
редакционной коллегии о принятии статьи к печати или ее отклонении сообщается
авторам. Редколлегия не обязуется направлять рецензию авторам отклоненной
статьи. %\\[-13.5pt]
\item Корректура статей высылается авторам для просмотра. Редакция
просит авторов присылать свои замечания в кратчайшие сроки. %\\[-13.5pt]
\item При
подготовке рукописи в MS Word рекомендуется использовать следующие настройки.
Параметры страницы: формат~--- А4; ориентация~--- книжная; поля (см): внутри~---
2,5, снаружи~--- 1,5, сверху~--- 2, снизу~--- 2, от края до нижнего
колонтитула~--- 1,3. Основной текст: стиль~--- <<Обычный>>: шрифт Times New
Roman, размер 14~пунктов, абзацный отступ~--- 0,5~см, 1,5 интервала,
выравнивание~--- по ширине. Рекомендуемый объем рукописи~--- не свыше
25~страниц указанного формата. Ознакомиться с шаблонами, содержащими примеры
оформления, можно по адресу в Интернете:
\textsf{http://www.ipiran.ru/journal/template.doc}.
\item К рукописи, предоставляемой в 2-х
экземплярах, обязательно прилагается электронная версия статьи (как правило, в
форматах MS WORD (.doc) или \LaTeX\ (.tex), а также~--- дополнительно~--- в
формате .pdf) на дискете, лазерном диске или по электронной почте. Сокращения
слов, кроме стандартных, не применяются. Все страницы рукописи должны быть
пронумерованы. %\\[-13.5pt]
\item Статья должна содержать следующую информацию на русском и
английском языках: название, Ф.И.О. авторов, места работы авторов и их
электронные адреса, подробные сведения об авторах, оформленные в соответствии с форматом, 
определяемым файлами {\sf http://www.ipiran.ru/journal/issues/2011\_05\_01/authors.asp} и 
{\sf http://www.ipiran.ru/journal/issues/2011\_01\_eng/authors.asp},
аннотация (не более 100~слов), ключевые слова. Ссылки на
литературу в тексте статьи нумеруются (в квадратных скобках) и располагаются в
порядке их первого упоминания. В~списке литературы не должно быть позиций, на которые нет ссылки в тексте статьи.
Все фамилии авторов, заглавия статей, названия
книг, конференций и~т.\,п.\ даются на языке оригинала, если этот язык
использует кириллический или латинский алфавит. %\\[-13.5pt]
\item Присланные в редакцию материалы авторам не возвращаются.
\item При отправке файлов по электронной
почте просим придерживаться следующих правил:
\begin{itemize}
\item указывать в поле subject (тема) название журнала и фамилию автора; %\\[-13.5pt]
\item использовать attach (присоединение); %\\[-13.5pt]
\item в случае больших объемов информации возможно
использование общеизвестных архиваторов (ZIP, RAR); %\\[-13.5pt]
\item в состав электронной версии статьи должны входить: файл, содержащий текст статьи, и файл(ы),
содержащий(е) иллюстрации. %\\[-13.5pt]
\end{itemize}
\item Журнал <<Информатика и её применения>> является некоммерческим изданием. 
Плата за публикацию с авторов не взимается, гонорар авторам не выплачивается.
\end{enumerate}
\thispagestyle{empty}
\textbf{Адрес редакции:} Москва 119333,
ул.~Вавилова, д.~44, корп.~2, ИПИ РАН\\
\hphantom{\textbf{Адрес редакции:} }Тел.: +7 (499) 135-86-92\ \
Факс:  +7 (495) 930-45-05\ \  E-mail:   rust@ipiran.ru }
}

%\include{ipi-ind}

%\tableofcontents

\end{document}

%\tableofcontents

%\end{document}

%\tableofcontents


\end{document}

\newcommand{\Ack}{\subsection*{\protect\large\bf Acknowledgments}}

\vphantom{\int\limits_0^T }

{ \begin{center}  %fig1
 \vspace*{3pt}
    \mbox{%
 \epsfxsize=79mm 
 \epsfbox{gru-1.eps}
 }

\end{center}

\noindent
{{\figurename~1}\ \ \small{
Временные зависимости данные 
}}}

\vspace*{6pt}

\setcounter{figure}{1}

$\acute{\mbox{о}}$

\linebreak