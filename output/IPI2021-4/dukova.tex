\def\stat{dukova}

\def\tit{О ВЫБОРЕ ЧАСТИЧНЫХ ПОРЯДКОВ НА МНОЖЕСТВАХ ЗНАЧЕНИЙ ПРИЗНАКОВ 
В~ЗАДАЧЕ КЛАССИФИКАЦИИ$^*$}

\def\titkol{О выборе частичных порядков на множествах значений признаков 
в~задаче классификации}

\def\aut{Е.\,В.~Дюкова$^1$, Г.\,О.~Масляков$^2$}

\def\autkol{Е.\,В.~Дюкова, Г.\,О.~Масляков}

\titel{\tit}{\aut}{\autkol}{\titkol}

\index{Дюкова Е.\,В.}
\index{Масляков Г.\,О.}
\index{Djukova E.\,V.}
\index{Masliakov G.\,O.}


{\renewcommand{\thefootnote}{\fnsymbol{footnote}} \footnotetext[1]
{Работа частично финансирована РФФИ (проект 19-01-00430).}}


\renewcommand{\thefootnote}{\arabic{footnote}}
\footnotetext[1]{Федеральный исследовательский центр <<Информатика и~управление>> Российской академии наук,
 \mbox{edjukova@mail.ru}}
\footnotetext[2]{Федеральный исследовательский центр <<Информатика и~управление>> Российской академии 
наук, \mbox{gleb-mas@mail.ru}}

\vspace*{-6pt}

  
     \Abst{Рассматривается одна из центральных задач машинного обучения~--- задача 
классификации на основе прецедентов. Приводится описание схемы синтеза логических 
алгоритмов классификации в~предположении, что признаковые описания прецедентов 
являются элементами декартова произведения конечных частичных порядков. В~рамках 
данной схемы формулируется критерий корректности алгоритма голосования по 
представительным элементарным классификаторам классов. При условии, что исходные 
данные не упорядочены (описания прецедентов представляют собой элементы произведения 
антицепей), изучается возможность задания линейных порядков на множествах значений 
признаков, обеспечивающих более качественную классификацию, не обязательно 
корректную. Предлагается процедура <<корректного>> упорядочения допустимых значений 
отдельных признаков, при этом остальные признаки остаются антицепями. Приводятся 
результаты экспериментов на реальных данных, демонстрирующие эффективность 
разработанных в~работе методов.}
     
     \KW{машинное обучение; логические алгоритмы классификации; корректный 
алгоритм классификации; частично упорядоченное множество; декартово произведение 
частичных порядков; линейный порядок; дуализация над произведением частичных 
порядков}

\DOI{10.14357/19922264210410}
  
%\vspace*{9pt}


\vskip 10pt plus 9pt minus 6pt

\thispagestyle{headings}

\begin{multicols}{2}

\label{st\stat}
     
\section{Введение}

\vspace*{-2pt}

  В задаче классификации под прецедентной (обучающей) информацией 
понимается совокупность примеров изучаемых объектов, в~которой каж\-дый 
объект представлен в~виде числового вектора, полученного на основе 
измерения или наблюдения ряда его параметров или характеристик, 
называемых признаками. Каждый пример (обучающий объект, или прецедент) 
приписан к~определенному классу объектов. Требуется по признаковому 
описанию предъявленного объекта, о~котором заранее не известно, какому 
классу он принадлежит, определить (распознать) этот класс.
  
  Основное достоинство логического подхода к~задаче классификации 
(распознавания)~--- воз\-мож\-ность получения результата при отсутствии 
дополнительных предположений вероятностного характера и~при небольшом 
числе прецедентов. \mbox{Предполагается}, что каж\-дый признак принимает 
ограниченное число допустимых целочисленных значений. Анализ 
прецедентной информации сводится к~поиску в~исходных данных 
определенных закономерностей, или элементарных классификаторов, 
различающих объекты из разных классов. При этом большое внимание 
уделяется вопросам синтеза алгоритмов, безошибочно классифицирующих 
материал обучения. Такие алгоритмы называются корректными алгоритмами. 
Наиболее известны в~данной области алгоритмы корректного 
 голосования~[1--4], предложенные впервые в~отечественных работах, а также 
методы Logical Analysis of Data (LAD) и~Formal Concept Analysis (FCA)~[5]. 
Последние два направления логической классификации в~основном нацелены 
на обработку (анализ) бинарных данных. Основополагающие идеи LAD и~FCA 
принадлежат соответственно П.~Хаммеру (1986~г.)\ и~Р.~Вилле (1981~г.).
  
  Стандартная постановка логической классификации не всегда позволяет 
решать прикладные задачи со сложными отношениями на множествах 
допустимых значений признаков. В~\cite{2-duk, 6-duk} на базе обобщения 
классических понятий логической классификации предложены корректные 
процедуры классификации при условии, что признаковые описания объектов 
представляют собой элементы декартова произведения конечных частично 
упорядоченных множеств, и~рассмотрены вопросы применения асимптотически 
оптимальных алгоритмов перечисления элементарных классификаторов общего 
вида. Согласно предложенной общей постановке, в~классическом случае 
признаки являются антицепями. 
  
  В~\cite{7-duk} анонсирован быстрый метод независимого выбора линейных 
порядков на множествах допустимых значений признаков. Метод основан на 
предварительном анализе обучающей выборки и~позволяет существенно 
повысить качество классификации, однако не гарантирует корректность 
классификации.
  
  Основные результаты настоящей работы касаются исследования 
возможности построения час\-тич\-ных порядков на множествах допустимых 
значений признаков, обес\-пе\-чи\-ва\-ющих корректную классификацию обучающей 
выборки. Сформулирован критерий кор\-рект\-ности логической классификации 
над произведением час\-тич\-ных порядков, согласно которому для корректности 
голосования по пред\-ста\-ви\-тель\-ным элементарным классификаторам классов 
описание каж\-до\-го прецедента из каж\-до\-го класса долж\-но быть независимым от 
множества описаний прецедентов из других классов. Показано, что задача 
эффективного упорядочения значений признаков без нарушения корректности 
алгоритма фактически сводится к~труднорешаемой дис\-крет\-ной задаче поиска 
минимального покрытия\linebreak булевой мат\-ри\-цы. Для сокращения 
временн$\acute{\mbox{ы}}$х\linebreak затрат предложена и~исследована процедура 
согласованного линейного упорядочения значений отдельных признаков, 
га\-ран\-ти\-ру\-ющая корректную классификацию только прецедентов одного класса. 
  
\section{Логический анализ данных с~частичными порядками}

  Элементы $x,y$ из частично упорядоченного множества~$P$ называются 
\textit{сравнимыми}, если~$x$ пред\-шест\-ву\-ет~$y$ (запись $x\preceq y$ или 
$y\preceq x$). В~противном случае~$x$ и~$y$ \textit{несравнимые}. Для 
обозначения того, что элемент~$x$ предшествует элементу~$y$ и~$y\not= x$, 
используется обозначение $x\prec y$. Частично упорядоченное множество~$P$ 
называется \textit{цепью} (\textit{антицепью}), если все его элементы попарно 
сравнимые (несравнимые). Если множество~$P$ является цепью (антицепью), 
то частичный порядок $\preceq$ называется \textit{линейным} 
(\textit{антилинейным}) \textit{порядком}, а множество~$P$ \textit{линейно} 
(\textit{антилинейно}) \textit{упорядоченным}. Если~$P$ представляет собой 
декартово произведение $P_1\times \cdots\times P_n$, где $P_1, \ldots , P_n$~--- 
конечные частично упорядоченные множества, то элемент $x\hm= (x_1, \ldots , 
x_n)\hm\in P$ \textit{следует за} $y\hm= (y_1,\ldots , y_n)\hm\in P$, если~$x_i$ 
\textit{следует за} $y_i$ при $i\hm=1,2,\ldots , n$. 
  
  Пусть $R\subset P$, $R^+$~--- множество элементов, следующих за 
элементами из~$R$. Элемент~$x$ множества $P\backslash R^+$ называется 
\textit{независимым} от~$R$ элементом множества~$P$. Элемент~$x$ 
множества $P\backslash R^+$ называется \textit{максимальным независимым} 
от~$R$ элементом множества~$P$, если для любого другого элемента~$y$ 
множества $P\backslash R^+$ отношение $x \prec y$ не выполняется. Задача 
построения для заданного~$R$ множества максимальных независимых от~$R$ 
элементов множества~$P$ называется дуализацией над произведением 
частичных порядков. 
  
  Рассмотрим задачу логической классификации над произведением частичных 
порядков. Пусть~$M$~--- исследуемое множество объектов с~$l$ попарно 
непересекающимися классами $K_1,\ldots , K_l$, и~пусть объекты 
множества~$M$ описываются признаками $x_1,\ldots , x_n$. Предполагается, 
что $M\hm= N_1\times \cdots\times N_n$, где $N_i$, $i\hm\in  
\{1,2,\ldots , n\}$,~--- конечное множество допустимых значений 
признака~$x_i$, на котором установлен частичный порядок. Не ограничивая 
общности, можно считать, что множество~$N_i$, $i\hm\in \{1,2,\ldots , n\}$, 
имеет наибольший элемент~$k_i$. Задан конечный набор $S_1,\ldots , S_m$ 
объектов из множества~$M$, о~которых известно, каким классам они 
принадлежат. Это прецеденты, или обуча\-ющие объекты. Пусть их описания 
имеют вид $S_1\hm= (a_{11}, \ldots , a_{1n}), S_2\hm= (a_{21},\ldots , a_{2n}), 
\ldots , S_m\hm= (a_{m1},\ldots , a_{mn})$, где $a_{ij}$~--- значение 
признака~$x_j$ для объекта~$S_i$. Требуется по предъявленному набору 
значений признаков $(a_1, \ldots , a_n)$, описывающему некоторый объект~$S$ 
из~$M$, о~котором, вообще говоря, неизвестно, какому классу он 
принадлежит, определить этот класс.
  
  Обозначим соответственно через $R(K)$ и~$R(\overline{K})$, $K\hm\in \{ 
K_1, \ldots , K_l\}$, множество прецедентов класса~$K$ и~множество 
прецедентов, не принадлежащих классу~$K$. Будем говорить, что 
алгоритм~$A$ классифицирует объект из $R(K)$ \textit{правильно}, если $A$ 
относит этот объект к~классу~$K$. Алгоритм~$A$ называется 
\textit{корректным на} $M$ алгоритмом, если $A$ правильно классифицирует 
каждый прецедент $S_1,\ldots , S_m$.
  
  Опишем схему работы рассматриваемого в~настоящей работе корректного 
логического классификатора над произведением частичных порядков. Данная 
модель классификатора является обобщением классической модели 
голосования по представительным наборам~\cite{1-duk, 3-duk}.
  
  Пусть $H$~--- набор из~$r$ различных признаков вида $H\hm= \{x_{j_1}, 
\ldots , x_{j_r}\}$, $\sigma\hm= (\sigma_1, \ldots , \sigma_r)$, $\sigma_i\hm \in 
N_{j_i}$, $i\hm=1,2,\ldots , r$. Пара $(\sigma, H)$ называется 
\textit{элементарным классификатором} (\textit{эл.\,кл.})\ \textit{ранга}~$r$. 
Близость объекта $S\hm= (a_1,\ldots , a_n)$ из~$M$ и~эл.\,кл.\ $(\sigma, H)$, 
$H\hm= \{x_{j_1}, \ldots , x_{j_r}\}$, $\sigma\hm= (\sigma_1, \ldots , \sigma_r)$, 
будем оценивать величиной $\hat{B}(\sigma, S,H)$, равной~1, если 
$a_{j_i}\hm\preceq \sigma_i$ при $i\hm=1,2,\ldots , r$, и~равной~0 в~противном 
случае. Будем говорить, что объект~$S$ содержит эл.\,кл.\ $(\sigma, H)$, если 
$\hat{B}(\sigma, S, H)\hm=1$. Эл.\,кл.\ $(\sigma, H)$, $H\hm= \{x_{j_1}, \ldots , 
x_{j_r}\}$, $\sigma\hm= (\sigma_1,\ldots , \sigma_r)$, порождает набор 
$S_{(\sigma,H)} \hm= (\gamma_1, \ldots , \gamma_n)$ из~$M$, в~котором 
$\gamma_{j_i}\hm= \sigma_i$, $i\hm= \{1,2,\ldots , r\}$, и~$\gamma_t\hm= k_t$ 
при $t\not\in \{j_1,\ldots , j_r\}$.
  
  Эл.\,кл.\ $(\sigma, H)$ называется \textit{корректным для класса}~$K$, если 
нельзя указать пару объектов~$S^\prime$ и~$S^{\prime\prime}$ таких, что 
$S^\prime\hm\in R(K)$, $S^{\prime\prime}\hm\in R(\overline{K})$ 
и~$\hat{B}(\sigma, S^\prime, H)\hm=\hat{B}(\sigma, S^{\prime\prime}, H)\hm= 1$. 
Корректный для класса~$K$ эл.\,кл.\ $(\sigma, H)$ называется 
\textit{тупиковым}, если любой эл.\,кл.\ $(\sigma^\prime, H^\prime)$ такой, что 
$S_{(\sigma, H)} \hm \prec S_{(\sigma^\prime, H^\prime)}$, не является 
корректным для класса~$K$. (Тупиковый) корректный эл.\,кл.\ называется 
(\textit{тупиковым}) \textit{представительным} для класса~$K$, если хотя бы 
один прецедент из класса~$K$ содержит данный эл.\,кл. 
  
  Таким образом, если эл.\,кл.\ $(\sigma, H)$~--- представительный для 
класса~$K$, то любой объект из $R(\overline{K})$ не содержит $(\sigma, H)$. 
Нетрудно также видеть, что если эл.\,кл.\ $(\sigma, H)$~--- (тупиковый) 
представительный для класса~$K$ и~объект~$S^\prime$ из $R(K)$ содержит 
$(\sigma, H)$, то $S_{(\sigma, H)}$~--- (максимальный) независимый от 
$R(\overline{K})$ элемент множества~$M$, который следует за~$S^\prime$.
  
  Рассматриваемый классификатор~$A$ на этапе обучения строит для каждого 
класса~$K$, $K\hm\in \{K_1, \ldots , K_l\}$, некоторое множество 
представительных эл.\,кл.\ $C^A(K)$. Далее осуществляется\linebreak процедура 
вычисления оценок, или процедура голосования, которая заключается 
в~сле\-ду\-ющем. На основе суммирования величин $\hat{B}(\sigma, S, H)$, $(\sigma, 
H)\hm\in C^A(K)$, для каждого класса~$K$, $K\hm\in \{K_1, \ldots , K_l\}$, 
вычисляется оценка $\Gamma(S,K)$ принадлежности распознаваемого 
объекта~$S$ классу~$K$, которая либо равна нулю, либо больше нуля. Эта 
оценка имеет вид:
  $$
  \Gamma(S,K) =\fr{1}{\vert C^A(K)\vert} \sum\limits_{(\sigma,H)\in C^A(K)} 
P_{(\sigma, H)} \hat{B}(\sigma, S, H)\,,
  $$
где $\vert C^A(K)\vert $~--- мощность $C^A(K)$; $ P_{(\sigma, H)}$~--- вес 
эл.\,кл.\ $(\sigma, H)$. В~качестве $P_{(\sigma, H)}$ обычно берется число 
объектов из $R(K)$, содержащих $(\sigma, H)$. Оценка $\Gamma(S,K)$ 
полагается равной нулю, если $C^A(K)\hm= \varnothing$. Объект~$S$ относится к~классу с~максимальной положительной оценкой. Если классов 
с~максимальной оценкой несколько, то происходит отказ от распознавания 
объекта~$S$.
  
  Пусть $Q(K, S^\prime)$, $S^\prime\hm\in R(K)$,~--- множество всех 
представительных эл.\,кл.\ из $C^A(K)$, которые объект~$S^\prime$ содержит. 
Очевидно, алгоритм~$A$ классифицирует правильно объект~$S^\prime$ из 
$R(K)$ тогда и~только тогда, когда $Q(K,S^\prime)\not= \varnothing$.
  
  Известно, что эл.\,кл.\ небольшого ранга, в~частности тупиковые 
представительные эл.\,кл., наиболее информативны. Рассмотрим случай, когда 
$C^A(K)$ содержит множество всех тупиковых представительных эл.\,кл.\ 
класса~$K$. В~этом случае справедлива
  
  
  \noindent
  \textbf{Теорема~1.} \textit{Алгоритм~$A$ классифицирует правильно объект 
$S^\prime$ из $R(K)$ тогда и~только тогда, когда~$S^\prime$ является 
независимым от $R(\overline{K})$ элементом множества~$M$}.
  
  \smallskip
  
  \noindent
  Д\,о\,к\,а\,з\,а\,т\,е\,л\,ь\,с\,т\,в\,о\,.\ \ \textit{Необходимость}. Если 
алгоритм~$A$ классифицирует объект~$S^\prime$ из $R(K)$ правильно, то 
$Q(K,S^\prime) \not= \varnothing$. Значит, объект~$S^\prime$ содержит хотя бы 
один тупиковый представительный эл.\,кл.\ $(\sigma, H)$. Имеем $S^\prime 
\preceq S_{(\sigma, H)}$, где~$S_{(\sigma, H)}$~--- максимальный независимый от 
$R(\overline{K})$ элемент множества~$M$. Следовательно, $S^\prime$~--- 
независимый от $R(\overline{K})$ элемент множества~$M$. 
  
  \textit{Достаточность}. Пусть $S^\prime\hm\in R(K)$ и~$S^\prime$~--- 
независимый от $R(\overline{K})$ элемент множества~$M$. Рассмотрим 
эл.\,кл.\ $(\sigma, H)$ такой, что $S^\prime\hm= S_{(\sigma,H)}$. Ни один 
объект~$S^{\prime\prime}$ из $R(\overline{K})$ не содержит эл.\,кл.\ $(\sigma, 
H)$, иначе $S^{\prime\prime}\preceq S^\prime$ и, следовательно, $S^\prime$ не 
является независимым от $R(\overline{K})$ элементом множества~$M$, что 
противоречит условию утверждения теоремы. Таким образом, $S^\prime$ 
содержит представительный эл.\,кл.\ $(\sigma, H)$ класса~$K$. Эл.\,кл.\ 
$(\sigma, H)$ содержит хотя бы один тупиковый представительный эл.\,кл.\ 
класса~$K$. Поэтому $Q(K, S^\prime)\not= \varnothing$. Терема доказана.
  
  Таким образом, если для любого~$K$ любой прецедент из $R(K)$ является 
независимым от $R(\overline{K})$ элементом множества~$M$, то алгоритм, 
ис\-поль\-зу\-ющий в~качестве $C^A(K)$ множество всех тупиковых эл.\,кл., 
корректный. Аналогичный результат имеет место, если в~качестве $C^A(K)$ 
рассматривается множество всех представительных эл.\,кл.\ класса~$K$. 
  
  Пусть $\tilde{M}=\tilde{N}_1\times \cdots \times \tilde{N}_n$. Здесь и~далее 
$\tilde{M}\hm= M$ и~$x\preceq y$ в~$\tilde{M}$, если $y\preceq x$ в~$M$. 
Зададим отображение~$\varphi: M\hm\to M\times \tilde{M}$ сле\-ду\-ющим 
образом. Отображение~$\varphi$ переводит объект $S\hm= (a_1,  \ldots, a_n)$ 
из~$M$  в~объект $\varphi(S)\hm= (a_1, \ldots , a_n, a_{n+1}, \ldots , a_{2n})$ из 
$M\times \tilde{M}$, в~котором $a_{n+i}\hm=a_i$ при $i\hm \in \{1,2,\ldots , 
n\}$. Иными словами, признаковое описание объекта~$S$ дублируется 
с~обратным отношением порядка.
  
  Пусть $\varphi(M^\prime)$, $M^\prime \subset M$,~--- образ~$M^\prime$ при 
отображении~$\varphi$. Имеет \mbox{место}
  
  \smallskip
  
  \noindent
  \textbf{Теорема~2}~\cite{2-duk}. \textit{Если классы множества~$M$ не 
пересекаются, то любой прецедент из класса $\varphi(K)$ содержит 
представительный эл.\,кл.\ класса}~$\varphi(K)$.
  
  \smallskip
  
  Таким образом, при любых частичных порядках $N_1,\ldots , N_n$ 
добавление признаков с обратными порядками делает корректной процедуру 
классификации по (тупиковым) представительным эл.\,кл. Однако увеличение 
числа признаков в~2~раза значительно увеличивает вычислительные затраты.
  
  Отметим, что этап обучения~--- это самый слож\-ный в~вы\-чис\-ли\-тель\-ном плане 
этап логической классификации. В~\cite{2-duk} показано, что по\-стро\-ение 
множества тупиковых пред\-ста\-ви\-тель\-ных наборов класса~$K$ приводит 
к~не\-об\-хо\-ди\-мости рас\-смат\-ри\-вать задачу дуализации над произведением 
час\-тич\-ных порядков, которая относится к~труд\-но\-ре\-ша\-емым пе\-ре\-чис\-ли\-тель\-ным 
задачам дискретной ма\-те\-ма\-тики.
{\looseness=1

} 
  
  На практике порядки на множествах значений признаков могут быть не 
заданы или заданы <<неудачно>>, демонстрируя не очень высокое качество 
классификации и~не обеспечивая корректность классификации на обучающей 
выборке. Поэтому актуальны рассматриваемые ниже вопросы эффективного 
задания частичных порядков $N_1,\ldots , N_n$.

\section{Быстрая процедура независимого линейного упорядочения 
значений признаков}

  В данном разделе описана процедура выбора линейных порядков на 
множествах значений признаков, эффективная по времени вычислений 
и~поз\-во\-ля\-ющая повысить качество классификации, но не обеспечивающая 
корректность классификации.
  
  Предлагаемая процедура основана на оценке информативности отдельных 
значений признаков по методу, идейно близкому к~методике предварительного 
анализа обучающей выборки, предложенной в~\cite{3-duk}. Пусть 
$\mu^{(1)}_{ij}(a)$, $i\hm\in \{1,2,\ldots , l\}$,\linebreak $j\hm\in \{1,2,\ldots , n\}$, $a\hm\in 
N_j$,~--- доля прецедентов класса~$K_i$, у~которых признак~$x_j$ принимает 
значение~$a$; $\mu^{(2)}_{ij}(a)$, $i\hm\in \{1,2,\ldots , l\}$, $j\hm\in \{1,2,\ldots 
, n\}$, $a\hm\in N_j$,~--- доля прецедентов не из класса~$K_i$, у~которых 
признак~$x_j$ принимает значение~$a$. Величину $\mu_{ij}(a)\hm= 
\mu^{(1)}_{ij}(a) \hm- \mu^{(2)}_{ij}(a)$ назовем весом значения~$a$ 
\mbox{признака}~$x_j$ в~классе~$K_i$. Эта величина служит мерой близости 
значения~$a$ признака~$x_j$ классу~$K_i$. Для каждого класса~$K_i$, 
$i\hm\in \{1,2,\ldots , l\}$, и~для каждого признака~$x_j$, $j\hm\in \{1,2,\ldots , 
n\}$, зададим сле\-ду\-ющий линейный порядок: $\forall\ y,z\hm\in N_j$, 
$y\preceq z$ тогда и~только тогда, когда $\mu_{ij}(y)\hm \geq \mu_{ij}(z)$. 
  
  Рассмотрим $S^\prime\hm= (a_1,\ldots , a_n)\hm\in R(K_i)$, $i\hm\in 
\{1,2,\ldots , l\}$. Предположим, что после задания\linebreak порядка данный объект 
классифицирован неправильно. Тогда, согласно теореме~1, существует объект 
$S^{\prime\prime}\hm= (b_1, \ldots , b_n\}\hm\in R(\overline{K}_l)$ такой, что 
$\mu_{ij}(a_j)\hm\leq \mu_{ij}(b_j)$ при $j\hm= 1,2,\ldots , n$. Следовательно,\linebreak 
признаковые значения объекта~$S^\prime$ недостаточно характерны для 
класса~$K_i$, т.\,е.\ объект~$S^\prime$ не является типичным для~$K_i$. 
Таким образом, задание порядка описанным способом делает модель 
голосования по представительным эл.\,кл.\ менее чувствительной к~выбросам. 
Очевидный недостаток процедуры заключается в~том, что порядок на 
множестве значений каждого признака выбирается независимо от выбора 
порядков для других признаков.

\section{Процедура корректного упорядочения значений 
признаков}

\vspace*{-6pt}

  Пусть для любого класса~$K$, $K\hm\in \{K_1, \ldots , K_l\}$, множество 
$C^A(K)$ содержит все тупиковые представительные эл.\,кл.\ класса~$K$. 
Частичный порядок на~$M$ назовем $(A,K)$-кор\-рект\-ным, если 
алгоритм~$A$ правильно классифицирует каждый объект из $R(K)$. Согласно 
тео\-ре\-ме~1, частичный порядок на~$M$ является $(A,K)$-кор\-рект\-ным 
тогда и~только тогда, когда для любого объекта~$S^\prime$, $S^\prime\hm\in 
R(K)$, и~любого объекта $S^{\prime\prime}$, $S^{\prime\prime}\hm\in 
R(\overline{K})$, либо~$S^\prime$ и~$S^{\prime\prime}$ несравнимые, либо 
$S^\prime \prec S^{\prime\prime}$ (запись $S^{\prime\prime}\not\preceq 
S^\prime$). 
  
  Построим булеву матрицу~$B_K$. Каждой строке в~матрице~$B_K$ 
соответствует пара объектов ($S^\prime, S^{\prime\prime}$), где $S^\prime\hm 
\in R(K)$ и~$S^{\prime\prime}\hm\in R(\overline{K})$, а~каждому столб\-цу 
соответствует тройка ($j,a,b$), где $j\hm\in \{1,2,\ldots , n\}$, $a,b,\hm\in N_j$, 
$a\not= b$. Элемент матрицы~$B_K$, расположенный на пересечении строки 
($S^\prime, S^{\prime\prime}$) и~столбца ($j,a,b$), равен~1, если значение 
признака~$x_j$ равно~$a$ и~$b$ у~объектов~$S^\prime$ и~$S^{\prime\prime}$ 
соответственно. Набор столбцов~$H$ матрицы~$B_K$ назовем 
\textit{покрытием}, если каждая строка матрицы~$B_K$ в~пересечении хотя бы 
с одним из столбцов, входящих в~$H$, дает~1 (каж\-дая строка мат\-ри\-цы~$B_K$ 
\textit{покрыта} хотя бы одним столбцом из~$H$). Покрытие мат\-ри\-цы~$B_K$ 
называется \textit{неприводимым}, если любое его собственное подмножество 
покрытием не является. Имеет \mbox{место}
  
  \smallskip
  
  \noindent
  \textbf{Теорема~3.}\ \textit{Частичный порядок, заданный на множестве~$M$, 
является $(A,K)$-кор\-рект\-ным тогда и~только тогда, когда существует 
неприводимое покрытие $H$ матрицы~$B_K$ такое, что $\forall\, j\hm\in 
\{1,2,\ldots , n\}$ и~$\forall a,b\hm\in N_j$, $a\prec b$, столбец $(j,b,a)$ не входит 
в}~$H$.
  
  \smallskip
  
  \noindent
  Д\,о\,к\,а\,з\,а\,т\,е\,л\,ь\,с\,т\,в\,о\,.\ \ \textit{Необходимость}. Предположим 
противное. Пусть существуют $j\hm\in \{1,2,\ldots , n\}$ и~$a,b \hm\in N_j$, 
$a\prec b$, такие, что столбец $(j,b,a)$ входит во все неприводимые покрытия 
матрицы~$B_K$. Тогда найдется строка ($S^\prime, S^{\prime\prime}$) 
матрицы~$B_K$, которая покрыта столбцом $(j,b,a)$ и~не покрыта всеми 
остальными столбцами матрицы~$B_K$. При этом из построения 
матрицы~$B_K$ следует, что объекты~$S^\prime$ и~$S^{\prime\prime}$ 
отличаются только признаком~$x_j$ и~этот признак принимает значения~$b$ 
и~$a$ у~объектов~$S^\prime$ и~$S^{\prime\prime}$ соответственно. Из условия 
$a\prec b$ следует $S^{\prime\prime}\prec S^\prime$, что противоречит  
$(A,K)$-кор\-рект\-ности частичного порядка. 

\begin{table*}[b]\small
\vspace*{2pt}
  \begin{center}
  \begin{tabular}{|c|l|c|c|c|c|}
  \multicolumn{6}{p{130mm}}{Качество классификации алгоритма голосования по тупиковым 
представительным эл.\,кл.\ при различном выборе частичного порядка}\\
  \multicolumn{6}{c}{\ }\\
  \hline
№&\multicolumn{1}{c|}{$(m,n,l,k)$}&\tabcolsep=0pt\begin{tabular}{c}$(A,K)$-корректный\\ порядок\end{tabular}&
\tabcolsep=0pt\begin{tabular}{c}Независимый\\ линейный\\ порядок\end{tabular}&
\tabcolsep=0pt\begin{tabular}{c}Признаки \\ антицепи\end{tabular}&
\tabcolsep=0pt\begin{tabular}{c}Признаки\\ с~обратными\\ порядками\end{tabular}\\
\hline
1&$(1728, 4, 4, 4)$&\textbf{0,90}&0,89&0,73&0,81\\
2&$(302, 13, 2, 151)$&0,85&0,89&0,76&\textbf{0,91}\\
3&$(336, 34, 4, 75)$&0,91&\textbf{0,95}&\textbf{0,95}&\textbf{0,95}\\
4&$(427, 3, 15, 256)$&---&\textbf{0,69}&0,43&0,61\\
5&$(307, 35, 19, 8)$&\textbf{0,78}&0,72&0,58&0,76\\
\hline
\end{tabular}
\end{center}
\end{table*}
  
  \vspace*{3pt}
  
  \textit{Достаточность}. Рассмотрим неприводимое покрытие $H$ 
матрицы~$B_K$, удовлетворяющее условию теоремы. Тогда для любой строки 
($S^\prime, S^{\prime\prime}$) матрицы~$B_K$ в~$H$ найдется столбец вида 
$(j,a,b)$, покрывающий строку ($S^\prime, S^{\prime\prime}$). Это означает, 
что признак~$x_j$ принимает значения~$a$ и~$b$ у объектов~$S^\prime$ 
и~$S^{\prime\prime}$ соответственно и~$b\not\preceq a$. Следовательно, 
$S^{\prime\prime}\not\preceq S^\prime$. Отсюда в~силу произвольности выбора 
объектов $S^\prime$ и~$S^{\prime\prime}$ следует  
$(A,K)$-кор\-рект\-ность частичного порядка. Теорема доказана.
  
  %\smallskip
  
  \noindent
  \textbf{Следствие~1.}\ Произведение линейных порядков
  %Линейный порядок на множестве~$M$ 
  является 
$(A,K)$-кор\-рект\-ным тогда и~только тогда, когда существует 
неприводимое покрытие $H$ мат\-ри\-цы~$B_K$ такое, что $a\prec b$ для 
любого столбца $(j,b,a)$ из покрытия~$H$.
  
  %\smallskip
  
  \noindent
  \textbf{Следствие~2.}\ Произведение антилинейных порядков
  %Антилинейный порядок на множестве~$M$ 
  является  
$(A,K)$-кор\-рект\-ным для любого класса~$K$ из $\{K_1, \ldots , K_l\}$.
  
 % \smallskip
  
  Основная проблема при построении $(A,K)$-кор\-рект\-но\-го частичного 
порядка заключается\linebreak в~вычислительной слож\-ности перечисления 
неприводимых покрытий булевой мат\-ри\-цы~\cite{7-duk}. Для увеличения чис\-ла 
признаков с~линейными порядками можно использовать приближенные 
методы поиска минимального покрытия. 
  
  При проведении экспериментов построение $(A,K)$-кор\-рект\-но\-го 
частичного порядка осуществлялось с применением генетического алгоритма, 
ориентированного на приближенное решение задачи поиска минимального 
покрытия~\cite{8-duk}. После завершения работы генетического алгоритма 
последовательно рассматривался каждый признак.\linebreak С~применением алгоритма 
Тарьяна~\cite{9-duk} осуществлялось упорядочение ориентированного графа, 
вершинами которого служили значения из~$N_i$, а~каж\-до\-му ребру $(a,b)$, 
$a,b\hm\in N_i$, соответствовал \mbox{столбец} ($j,a,b$) (см.\ следствие~1 из 
теоремы~3). При этом обход вершин осуществлялся согласно порядку, 
определяемому процедурой быстрого линейного упорядочения из разд.~3. 
Если в~результате применения алгоритма получался линейный порядок, это 
означало, что~$N_i$~--- цепь. В~противном случае $N_i$~--- антицепь.
  
  Стоит отметить, что из-за сильной разре\-жен\-ности мат\-ри\-цы~$B_K$ 
применение жадного алгоритма для поиска минимального покрытия  
мат\-ри\-цы~$B_K$ оказалось менее эффективным. 

\section{Эксперименты}

%\vspace*{-6pt}

  Эксперименты проводились на наборах данных из {\sf www.kaggle.com} 
  и~{\sf archive.ics.uci.edu}: Car (№\,1); Heart (№\,2); Dermatology (№\,3); Ph (№\,4); 
Soybean (№\,5). Для тестирования применялась методика трехкратного 
скользящего контроля. Качество классификации оценивалось по доле 
правильных ответов.
  
  В таблице представлены результаты классификации алгоритмом голосования 
по тупиковым представительным эл.\,кл.\ для следующих случаев: 
\begin{enumerate}[(1)]
\item частичные порядки выбраны корректной процедурой;\\[-15pt] 
\item для каждого~$N_i$ 
независимо выбран линейный порядок;\\[-15pt]
\item каждое множество~$N_i$ является 
антицепью (классический подход);\\[-15pt] 
\item признаки линейно упорядочены 
экспертом и~добавлены признаки с обратным порядком.
\end{enumerate}

 В~таблице для каждого 
набора данных указано число объектов~$m$, число признаков~$n$, число 
классов~$l$ и~максимальное число значений признака~$k$.
  
  Полученные результаты показывают, что время, затраченное на 
упорядочение значений признаков в~каждой из двух предлагаемых процедур, 
существенно меньше времени обучения логического классификатора. Обе 
процедуры выбора линейных порядков улучшают качество классификации по 
сравнению с~классическим подходом. При этом процедура независимого 
упорядочения показывает лучшее качество и~тратит в~несколько раз меньше 
времени, чем корректная процедура согласованного упорядочения. В~обоих 
случаях суммарное время классификации значительно меньше времени 
классификации с~использованием признаков с~обратными порядками. Таким 
образом, более высокое качество классификации и~меньшее время работы 
делает процедуру независимого выбора линейного порядка предпочтительней 
корректной процедуры. С~другой стороны, подход с~выбором корректного 
частичного порядка предоставляет больше информации для анализа 
классифицируемых данных. Из-за недостатка времени не удалось применить
процедуру выбора корректного порядка к~набору данных Ph.
  
 \vspace*{-3pt} 

\section{Заключение}
  
  Авторами продолжены начатые в~\cite{2-duk, 6-duk, 7-duk} исследования по 
разработке методов логического анализа частично упорядоченных данных 
и~применения\linebreak этих методов в~машинном обучении. Изучены \mbox{вопросы} 
повышения качества логической классификации путем задания на множествах 
значений признаков частичных порядков. Предложены и~экспериментально 
исследованы различные способы выбора <<хороших>> линейных порядков. 
Показано, что предлагаемые методы упорядочения значений признаков 
позволяют улучшить качество классификации по сравнению с~классическим 
подходом, когда отдельные значения признака сравниваются с~использованием 
простого отношения равенства. Представленные результаты имеют 
приоритетный характер.

 \vspace*{-3pt}
  
{\small\frenchspacing
 {%\baselineskip=10.8pt
 %\addcontentsline{toc}{section}{References}
 \begin{thebibliography}{9}
  \bibitem{1-duk}
  \Au{Баскакова Л.\,В., Журавлёв~Ю.\,И.} Модель рас\-по\-зна\-ющих алгоритмов 
с~пред\-ста\-ви\-тель\-ны\-ми наборами и~сис\-те\-ма\-ми опорных множеств~// Ж. вычисл. матем. 
и~матем. физ., 1981. Т.~21. №\,5. С.~1264--1275.

 \bibitem{3-duk} %2
\Au{Дюкова Е.\,В., Песков~Н.\,В.} Поиск информативных фрагментов описаний объектов 
в~дискретных процедурах распознавания~// Ж. вычисл. матем. и~матем. физ., 2002. Т.~42. 
№\,5. С.~711--723.
  \bibitem{4-duk} %3
   \Au{Журавлёв Ю.\,И., Рязанов~В.\,В., Сенько~О.\,В.} <<Распознавание>>. Математические 
методы. Программная сис\-те\-ма. Практические применения.~--- М.: ФАЗИС, 2006. 176~с.
  \bibitem{2-duk} %4
   \Au{Дюкова Е.\,В., Масляков~Г.\,О., Прокофьев~П.\,А.} О~логическом анализе данных 
с~частичными порядками в~задаче классификации по прецедентам~// Ж. вычисл. матем. 
и~матем. физ., 2019. Т.~59. №\,9. С.~1605--1616.
 
  \bibitem{5-duk}
  \Au{Janostik R., Konecny~J., \mbox{Kraj{\!\ptb{\!\v{c}}a}~P.}} Interface between Logical Analysis 
of Data and Formal Concept Analysis~// Eur. J.~Oper. Res., 2020. Vol.~284. 
Iss.~2. P.~792--800.
  \bibitem{6-duk}
   \Au{Дюкова Е.\,В., Масляков~Г.\,О., Прокофьев~П.\,А.} Дуализация над произведением 
цепей: асимп\-то\-ти\-че\-ские оцен\-ки чис\-ла решений~// Докл. Акад. наук, 2018. Т.~483. №\,2.  
С.~130--133.
  \bibitem{7-duk}
   \Au{Бакланова А.\,О., Дюкова Е.\,В., Масляков~Г.\,О.} Исследование зависимости качества 
классификации от выбора час\-тич\-ных порядков на множествах значений признаков~// 
Интеллектуализация обработки информации.~--- М.: Российская академия наук, 2020. С.~21--23.
  \bibitem{8-duk}
  \Au{Sotnezov R.\,M.} Genetic algorithms for problems of logical data analysis in discrete 
optimization and image recognition~// Pattern Recognit. Image Anal., 2009. Vol.~19. Iss.~3.  
P.~469--477.
  \bibitem{9-duk}
  \Au{Tarjan R.\,E.} Edge-disjoint spanning trees and depth-first search~// Acta Inform., 
1976. Vol.~6. Iss.~2. P.~171--185.
\end{thebibliography}

 }
 }

\end{multicols}

\vspace*{-6pt}

\hfill{\small\textit{Поступила в~редакцию 15.01.21}}

\vspace*{8pt}

%\pagebreak

%\newpage

%\vspace*{-28pt}

\hrule

\vspace*{2pt}

\hrule

%\vspace*{-2pt}

\def\tit{ON THE CHOICE OF PARTIAL ORDERS ON~FEATURE VALUES SETS 
IN~THE~SUPERVISED CLASSIFICATION PROBLEM}


\def\titkol{On the choice of partial orders on~feature values sets 
in~the~supervised classification problem}


\def\aut{E.\,V.~Djukova and G.\,O.~Masliakov}

\def\autkol{E.\,V.~Djukova and G.\,O.~Masliakov}

\titel{\tit}{\aut}{\autkol}{\titkol}

\vspace*{-11pt}


\noindent
Federal Research Center ``Computer Science and Control'' of the Russian Academy of 
Sciences, 44-2~Vavilov Str., Moscow 119333, Russian Federation


\def\leftfootline{\small{\textbf{\thepage}
\hfill INFORMATIKA I EE PRIMENENIYA~--- INFORMATICS AND
APPLICATIONS\ \ \ 2021\ \ \ volume~15\ \ \ issue\ 4}
}%
 \def\rightfootline{\small{INFORMATIKA I EE PRIMENENIYA~---
INFORMATICS AND APPLICATIONS\ \ \ 2021\ \ \ volume~15\ \ \ issue\ 4
\hfill \textbf{\thepage}}}

\vspace*{3pt} 
  





\Abste{The authors consider one of the central problems of machine learning~--- the supervised 
classification. A~scheme for the logical classification algorithms synthesis is described under the 
assumption that the features descriptions of precedents are the elements of the finite partial orders 
Cartesian product. A~criterion for the correctness of the voting algorithm of representative 
elementary classifiers is formulated. The authors study the possibility of defining linear orders on 
sets of feature values that provide better classification, which is not necessarily correct, in 
assumption that the source data are not ordered (the precedents descriptions are the elements of the 
antichains product). A~procedure is proposed for ``correct'' consistent ordering of the acceptable 
values of separate features, while the remaining features are antichains. The results of experiments 
on real data are presented demonstrating the effectiveness of the methods developed in the work.}

\KWE{machine learning; logical classification algorithms; correct supervised classification 
algorithm; partially ordered set; Cartesian product of partial orders; linear order; dualization over 
product of partial orders}



\DOI{10.14357/19922264210410}

%\vspace*{-20pt}

\Ack
\noindent
The reported study was funded by RFBR, project number 19-01-00430. 



%\vspace*{6pt}

  \begin{multicols}{2}

\renewcommand{\bibname}{\protect\rmfamily References}
%\renewcommand{\bibname}{\large\protect\rm References}

{\small\frenchspacing
 {%\baselineskip=10.8pt
 \addcontentsline{toc}{section}{References}
 \begin{thebibliography}{9}
  \bibitem{1-duk-1}
\Aue{Baskakova, L., and Yu.~Zhuravlev.} 1981. A model of recognition algorithms with 
representative sampls an systems of supporting sets. \textit{USSR Comp. Math. Math. Phys.} 
21(5):189--199.
\bibitem{3-duk-1} %2
\Aue{Djukova, E., and N.~Peskov.} 2002. Search for informative fragments of object descriptions 
in discrete recognition procedures. \textit{Comp. Math. Math. Phys.} 42(5):711--723.
  \bibitem{4-duk-1} %3
\Aue{Zhuravlev, Yu.\,I., Y.\,V.~Ryazanov, and O.\,Y.~Sen'ko.} 2006. \textit{Raspoznavanie. 
Matematicheskie metody. Programmnaya sistema. Primeneniya} [Recognition. Mathematical 
methods. Software system. Applications]. Moscow: FAZIS. 176~p.

  \bibitem{2-duk-1} %4
\Aue{Djukova, E., G.~Maslyakov, and P.~Prokofyev.} 2019. On the logical analysis of partially 
ordered data in the supervised classification problem. \textit{Comp. Math. Math. Phys.} 
59(9):1542--1552.
  
  \bibitem{5-duk-1}
\Aue{Janostik, R., J.~Konecny, and P.~\mbox{Kraj\!\/{\ptb{\v{c}}}a}.} 2020. Interface between 
logical analysis of data and formal concept analysis. \textit{Eur. J.~Oper. Res.} 
284(2):792--800.
  \bibitem{6-duk-1}
\Aue{Djukova, E., G.~Maslyakov, and P.~Prokofjev.} 2018. Dualization problem over the product 
of chains: Asymptotic estimates for the number of solutions. \textit{Dokl. Math.} 98:564--567.
  \bibitem{7-duk-1}
\Aue{Baklanova, A., E.~Djukova, and G.~Masliakov.} 2020. Investigation of the dependence of 
the supervised classification quality on the choice of partial orders on feature values sets. 
\textit{Intelligent data processing: Theory and applications}. Moscow: 
Russian Academy of Sciences. 24--26.
  \bibitem{8-duk-1}
\Aue{Sotnezov, R.} 2009. Genetic algorithms for problems of logical data analysis in discrete 
optimization and image recognition. \textit{Pattern Recognit. Image Anal.} 19(3):469--477.
  \bibitem{9-duk-1}
\Aue{Tarjan, R.} 1976. Edge-disjoint spanning trees and depth-first search. \textit{Acta 
Inform.} 6(2):171--185.
\end{thebibliography}

 }
 }

\end{multicols}

\vspace*{-3pt}

\hfill{\small\textit{Received January 15, 2021}}

%\pagebreak

%\vspace*{-24pt}

\Contr

\noindent
\textbf{Djukova Elena V.} (b.\ 1945)~--- Doctor of Science in physics and mathematics, principal 
scientist, Federal Research Center ``Computer Science and Control'' of the Russian Academy of 
Sciences, 44-2~Vavilov Str., Moscow 119333, Russian Federation;  \mbox{edjukova@mail.ru}

\vspace*{3pt}

\noindent
\textbf{Masliakov Gleb O.} (b.\ 1996)~---  PhD student, Federal Research Center ``Computer Science 
and Control'' of the Russian Academy of Sciences, 44-2~Vavilov Str., Moscow 119333, Russian 
Federation; \mbox{gleb-mas@mail.ru}

  
\label{end\stat}

\renewcommand{\bibname}{\protect\rm Литература} 