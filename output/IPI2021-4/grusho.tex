\def\stat{grusho}

\def\tit{СТАТИСТИКА И КЛАСТЕРЫ В~ПОИСКАХ АНОМАЛЬНЫХ ВКРАПЛЕНИЙ В~УСЛОВИЯХ 
БОЛЬШИХ ДАННЫХ$^*$}

\def\titkol{Статистика и~кластеры в~поисках аномальных вкраплений в~условиях 
больших данных}

\def\aut{А.\,А.~Грушо$^1$, Н.\,А.~Грушо$^2$, М.\,И.~Забежайло$^3$, 
Д.\,В.~Смирнов$^4$, Е.\,Е.~Тимонина$^5$,\\ С.\,Я.~Шоргин$^6$}

\def\autkol{А.\,А.~Грушо, Н.\,А.~Грушо, М.\,И.~Забежайло и~др.} 
%Д.\,В.~Смирнов$^4$, Е.\,Е.~Тимонина$^5$, С.\,Я.~Шоргин$^6$}

\titel{\tit}{\aut}{\autkol}{\titkol}

\index{Грушо А.\,А.}
\index{Грушо Н.\,А.}
\index{Забежайло М.\,И.} 
\index{Смирнов Д.\,В.}
\index{Тимонина Е.\,Е.}
\index{Шоргин С.\,Я.}
\index{Grusho A.\,A.}
\index{Grusho N.\,A.}
\index{Zabezhailo M.\,I.}
\index{Smirnov D.\,V.} 
\index{Timonina E.\,E.}
\index{Shorgin S.\,Ya.}


{\renewcommand{\thefootnote}{\fnsymbol{footnote}} \footnotetext[1]
{Работа частично поддержана РФФИ (проект 18-29-03081).}}


\renewcommand{\thefootnote}{\arabic{footnote}}
\footnotetext[1]{Федеральный исследовательский центр <<Информатика и~управление>> 
Российской академии наук, \mbox{grusho@yandex.ru}}
\footnotetext[2]{Федеральный исследовательский центр <<Информатика и~управление>>  
Российской академии наук, \mbox{info@itake.ru}}
\footnotetext[3]{Федеральный исследовательский центр <<Информатика и~управление>>  Российской 
академии наук, \mbox{m.zabezhailo@yandex.ru}}
\footnotetext[4]{ПАО Сбербанк России, \mbox{dvlsmirnov@sberbank.ru}}
\footnotetext[5]{Федеральный исследовательский центр <<Информатика и~управление>>  
Российской академии наук, \mbox{eltimon@yandex.ru}}
\footnotetext[6]{Федеральный исследовательский центр <<Информатика и~управление>> 
Российской академии наук, \mbox{sshorgin@ipiran.ru}}

%\vspace*{-6pt}

  \Abst{Построены алгоритмы снижения уровня <<ложных тревог>> (ЛТ) при поиске 
аномалий в~сложных гетерогенных последовательностях объектов (Big Data). Традиционно 
в~математической статистике такое снижение достигается за счет минимизации ошибки 
ЛТ. Однако в~задачах выявления аномалий (редкие вкрапления аномальных 
данных) такой подход ведет к~повышению вероятности потери искомых аномалий.
  Чтобы не потерять искомые аномалии, в~данной работе предлагается, наоборот, в~критериях, 
рассчитанных на наименьшую сложность вычислений, допустить большую ошибку появления 
ЛТ, но использовать тот факт, что выделенных такими критериями объектов 
значительно меньше, чем исходных объектов в~Big Data. Тогда выделенные объекты можно 
объединить в~один кластер и~использовать дополнительную информацию, связанную 
с~объектами этого кластера, для выявления искомых аномалий. Смысл этих действий состоит 
в~том, что более сложно вычислимые характеристики объектов для отсева ЛТ не 
потребуют больших вычислительных ресурсов на меньшем относительно исходных данных 
кластере объектов.
  Показано, что при выполнении некоторых условий порядок использования дополнительной 
информации не влияет на результат ее использования при фильтрации ЛТ.
  Результаты применения алгоритма фильтрации в~последовательности объектов обобщены на 
фильтрацию ЛТ в~форме при\-чин\-но-след\-ст\-вен\-ных схем в~исходных 
данных. При известных схемах показано, как можно фильтровать ЛТ, выявляя 
только фрагменты схем.}
  
  \KW{информационная безопасность; поиск аномалий; алгоритмы фильтрации <<ложных 
тревог>>}

\DOI{10.14357/19922264210411}
  
\vspace*{3pt}


\vskip 10pt plus 9pt minus 6pt

\thispagestyle{headings}

\begin{multicols}{2}

\label{st\stat}
   
  \section{Введение}
  
  \vspace*{-4pt}
  
  В работе рассматривается задача поиска аномалий или небольших вкраплений 
нестандартных данных в~гетерогенные пополняемые большие массивы 
информации (Big Data). Гетерогенность данных связана с~множеством различных 
источников данных и~множеством характеристик, которыми описываются 
свойства накопленной информации. Нестандартность данных часто связана 
с~наличием признаков, по которым осуществляется поиск искомых аномалий. 
Приведем примеры таких задач.
{\looseness=1

}
  
  В работах~[1--3] исследуются задачи обнаружения вторжений 
в~распределенные информационные системы (РИС). В~этом случае 
нестандартность вкрап\-ле\-ний в~данные мониторинга \mbox{заключается} в~появлении 
встречавшихся ранее признаков враждебных атак.
  
  Поиск признаков враждебной деятельности инсайдеров РИС рассматривался 
в~[4, 5]. Опыт сотрудников служб безопасности РИС позволяет выделять 
некоторые нестандартные особенности поведения инсайдеров в~различных 
информационных базах данных, которые в~своем объединении составляют 
гетерогенную информацию. Поскольку инсайдеров мало, то компрометирующие 
их фрагменты данных появляются редко~--- как небольшие вкрап\-ле\-ния.
  
  Неявные сбои компьютерных систем, которые проявляются через некоторое 
время в~рабочих процессах, часто порождают в~местах, связанных 
с~первопричинами сбоев, слабозаметные аномалии~[6,~7].
  
  Мошеннические схемы в~данных финансового мониторинга представляют 
собой небольшие вкрап\-ле\-ния в~информацию об обычных финансовых операциях, 
которые необходимо искать в~гетерогенных Big Data~[8].
  
  Несмотря на различия в~приложениях, в~перечисленных примерах есть общие 
черты. Прежде всего, данные для поиска вкраплений редко представляют собой 
только числовые значения. Чаще всего анализируемые данные~--- это 
комплексные структуры значений (часто нечисловых) нескольких параметров. 
Такие данные будем называть объектами. Вместе с~тем размеры каждого объекта, 
как правило, небольшие. В~статистике такие объекты называются малыми 
выборками. 
  
  На примере множества малых выборок объясним проблемы, возникающие при 
поиске вкрап\-ле\-ний~[9, 10]. Если вкрапления не очевидны, а~множество 
анализируемых объектов можно от\-нес\-ти к~Big\linebreak Data, то по соображениям 
вы\-чис\-ли\-тель\-ной слож\-ности в~объектах выделяют просто вычислимые 
характеристики и~по ним строят критерий аномальности. Такой подход приводит 
к~появлению\linebreak \mbox{большого} числа выбранных объектов, которые обладают свойством 
вкрапления, но к~таковым не относятся. Такие выбранные объекты называются 
<<ложными тревогами>>. Чтобы сократить множество ЛТ, снижают границы 
отбора объектов, которые критерий считает тревогой. Тогда резко возрастает 
возможность потери истинных аномальных вкрап\-ле\-ний. Математический анализ 
этой ситуации для малых выборок рассмотрен в~работах~[9,~10]. 
  
  Чтобы не потерять искомые аномалии, в~данной работе предлагается, 
  наоборот, в~критериях, рассчитанных на наименьшую сложность вычислений, допустить 
большую ошибку появления ЛТ, но использовать тот факт, что выделенных 
такими критериями объектов значительно меньше, чем исходных объектов в~Big 
Data. Тогда выделенные объекты можно объединить в~один кластер 
и~использовать дополнительную информацию, связанную с~объектами этого 
кластера, для выявления искомых аномалий. Смысл этих действий состоит в~том, 
что более сложно вычислимые характеристики объектов для отсева 
ЛТ не потребуют больших вычислительных ресурсов на меньшем 
относительно исходных данных кластере объектов. Отметим, что для различных 
объектов критерии могут отличаться, что дает возможность искать вкрапления 
в~гетерогенных данных. Анализ гетерогенных данных с~помощью различных 
методов анализа использовался в~работах~\cite{4-grusho, 11-grusho}. Близкий 
подход к~использованию дополнительных данных применялся в~поиске схем 
мошенничества в~работе~\cite{8-grusho}. 
  
  В задаче обнаружения вторжений возможен аналогичный подход. В~настоящее 
время вторжение, реализующее атаку, редко представляет собой одноразовое 
действие. Атаки строятся на основе ряда действий над объектами нечисловой 
природы, и~чаще всего некоторые действия берутся из других\linebreak известных атак. 
Такие ранее встречавшиеся действия могут быть найдены в~библиотеках 
уязвимостей и~проявляться в~данных как вкрапления. Динамический поиск таких 
вкраплений позволяет \mbox{своевременно} включать дополнительные меры защиты, 
а~затем, используя графы атак~\cite{12-grusho} как дополнительную 
информацию, последовательно избавляться от ЛТ, сокращая область воздействия 
  атаки~\cite{2-grusho, 3-grusho, 13-grusho}.
  
  При поиске признаков инсайдеров дополнительная информация, сокращающая 
размеры клас\-те\-ров объектов, в~которых присутствуют нестандартные действия, 
может быть взята из баз данных различного рода поведенческой активности 
пользователей~\cite{14-grusho}.
  
  В качестве источников информации при локализации первопричин неявных 
сбоев могут использоваться дополнительные данные  
о~при\-чин\-но-след\-ст\-вен\-ных связях в~схемах информационных 
технологий~\cite{6-grusho}.
  
  \section{Построение вложенной системы кластеров}
  
  В задачах поиска нестандартных вкраплений данные можно представить 
строчкой объектов $X_1, X_2, \ldots , X_n$, где каждый объект имеет сложную 
структуру, зависящую от значений параметров $I_1, I_2, \ldots , I_k$. Сначала 
будем предполагать, что каж\-дый объект выбирается независимо от других 
объектов в~соответствии со своей логикой без учета логики, приводящей к~выбору 
остальных объектов. Искомое вкрапление (аномалия) может быть описано  
с~по\-мощью нестандартных значений параметров по сравнению с~другими 
объектами или с~по\-мощью другой логики выбора объекта, которые имеют 
некоторые отличительные особенности от других объектов. Поиск 
осуществляется с~помощью критерия на наличие таких особенностей. Критерий 
пред\-став\-ля\-ет собой функцию от па\-ра\-мет\-ров рассматриваемого объекта, которая 
равна~1, если использованная в~вычислении функции информация показала 
воз\-мож\-ность наличия отличительных особенностей, присущих искомому 
вкрап\-ле\-нию. В~противном случае функция принимает значение~0. В~результате 
работы критерия для каж\-до\-го из объектов получается последовательность 
значений функции, со\-сто\-ящая из~0 и~1. При этом могут существовать объекты, не 
являющиеся вкрап\-ле\-ни\-ями, но обла\-да\-ющие характеристиками, на которых 
функция критерия принимает значение~1. На настоящем вкраплении функция 
критерия точно принимает значение~1. 
  
  В описании этой схемы не использовалось понятие вероятности, хотя из 
описания поиска вкрапления можно построить вероятностную схему, в~которой 
объекты появляются случайно и~\mbox{независимо} друг от друга. Тогда критерий 
определяется случайной величиной, принимающей значения~1 и~0. 
Распределение критерия на вкраплении отличается от распределения на 
остальных объектах. При этом в~некоторых случаях критерий может ошибаться 
и~принимать значение~1, хотя рас\-смат\-ри\-ва\-емый объект получен в~вероятностной 
схеме, не соответствующей вкраплению. На языке математической статистики 
такие случаи называются <<ложными тревогами>>. Если данные можно 
рас\-смат\-ри\-вать как одинаково распределенные объекты, то последовательность 
данных называется выборкой. В~более общем случае данные являются 
неоднородными, что не меняет схему поиска вкрап\-ле\-ний. 
{\looseness=1

}
  
  Если данных мало, а особенности, выделяющие вкрапления, встречаются редко, 
то первая единица в~последовательности значений функции критерия с~большей 
вероятностью служит признаком искомого вкрапления, чем ЛТ. 
  
  Если данных много, то появление ЛТ происходит с~вероятностью, близкой к~1. 
Для отсева ЛТ необходима дополнительная информация. Легко показать, что такая 
дополнительная информация должна быть получена из других источников 
информации, независимых от использованных в~функции критерия. В~противном 
случае трудно или невозможно доказать, что решение является ЛТ. При этом 
иногда можно доказать противное: что принимаемое решение повторяет ЛТ 
исходного критерия. 
  
  Кроме того, дополнительная информация должна быть получена 
  и~использоваться независимо для каж\-дой~1 в~результатах применения уже 
использованного критерия. Эти дополнительные для каж\-дой~1 данные 
формируют в~исходных данных новую последовательность объектов $X_1^*, 
X_2^*, \ldots , X_m^*$, где $m\hm< n$. Новые данные образуют в~исходных 
данных кластер, в~котором заведомо находится искомое вкрапление. Для 
последовательности данных $X_1^*, X_2^*, \ldots , X_m^*$ надо выбрать новый 
критерий, который не зависит от прежнего, но позволяет подтвердить или 
опровергнуть положительное решение, полученное после применения первого 
критерия. 
  
  В последовательности данных $X_1^*, X_2^*, \ldots , X_m^*$ после 
применения второго критерия также надо искать ЛТ. Для этого необходимо искать 
новую дополнительную информацию, не зависящую от использованной в~первом и~втором критериях. Повторяя алгоритм предыдущего шага, получим в~исходных 
данных кластер меньшего размера, наверняка содержащий искомое вкрапление. 
В~результате нескольких итераций возможны следующие ре\-зуль\-таты:
  \begin{itemize}
\item кластер становится настолько малым, что шансы (вероятность) случайного 
появления ЛТ становятся малы, но искомое вкрапление в~нем присутствует (это 
следует из исходных предположений);
\item дополнительные данные исчерпаны;
\item построенный алгоритм можно повторить;
\item после последней итерации построено покрытие вкрапления в~исходных 
данных, что достаточно для использования на практике.
\end{itemize}
  
  \section{Независимость решения от~порядка использования 
критериев}
   
  Рассмотрим простейший пример вкрапления и~его поиска. Пусть каждый 
объект зависит от трех параметров $I_1$, $I_2$ ит~$I_3$. Предположим, что искомая 
аномалия характеризуется значимым отклонением значений каждого па\-ра\-мет\-ра. 
Сначала строим критерий поиска вкрапления по па\-ра\-мет\-ру~$I_1$. В~результате 
получим кластер~$K_{1.1}$. 

  
  Согласно алгоритму, построенному в~предыдущем разделе, объекты, похожие 
на вкрапления, обрабатываются критерием, основанным на анализе независимых 
от~$I_1$ значений параметра~$I_2$. В~результате получаем кластер~$K_{1.2}$. 
Возникает вопрос, будет ли зависеть сле\-ду\-ющий, третий, этап алгоритма от 
порядка использования критериев, основанных на анализе параметров~$I_1$ 
и~$I_2$. т.\,е.\ сначала после применения критерия, основанного на данных о 
значениях параметра~$I_2$, получаем первый кластер~$K_{2.1}$. К~нему 
применяем алгоритм, основанный на значениях параметра~$I_1$, и~получаем 
второй кластер~$K_{2.2}$. Каждое положительное решение, полученное 
в~результате обработки некоторого объекта~$X$ первым и~вторым критериями, 
основано на использовании одной и~той же информации. При этом в~первом 
случае сначала для анализа~$X$ использована информация~$T_1$, породившая~1 
первого уровня, а затем для анализа~$X$ использована информация~$T_2$, 
породившая~1 второго уровня. Во втором случае, наоборот, сначала 
использовалась информация~$T_2$, а затем использовалась информация~$T_1$. 
Как отмечалось в~предыдущем разделе, дополнительная информация~$T_1$ 
и~$T_2$ должна быть получена из различных, независимых от уже 
использованных в~функциях критериев источников информации. Отсюда следует, 
что порядок применения~$T_1$ и~$T_2$ не влияет на обоснование 
положительного решения в~результате второй итерации алгоритма анализа 
объекта~$X$. Этот вывод можно сформулировать в~форме следующего 
утверж\-де\-ния.
{\looseness=1

}
  
  \smallskip
  
  \noindent
  \textbf{Теорема.}\ \textit{Если данные~$T_1$ и~~$T_2$, используемые для 
анализа и~обоснования вкрапления объекта~$X$ независимы друг от друга 
(независимые источники информации), то порядок их применения не влияет на 
получаемое значение~$1$ после итерации алгоритма.}
  
  \smallskip
  
  \noindent
  \textbf{Следствие~1.}\ Независимо от порядка применения независимых 
данных~$T_1$ и~$T_2$ число единиц в~последнем слое после повторной итерации 
применения алгоритма одно и~то же.
  
  \smallskip
  
  
  \noindent
  Д\,о\,к\,а\,з\,а\,т\,е\,л\,ь\,с\,т\,в\,о\,.\ \ Поскольку информация~$T_1$ и~$T_2$ 
применяется к~$X$ независимо друг от друга, то можно считать, что~1 
в~последнем слое (после двух итераций) получила двойное подтверждение. Если 
хотя бы один из критериев даст~0, то в~силу независимости данный объект~$X$ 
будет отброшен, т.\,е.~$X$ не войдет в~кластер после двух итераций. По 
предположению, вкрапление удовлетворяет всем критериям и, следовательно, оно 
остается после второй итерации. Тогда после второй итерации остаются те 
и~только те объекты, которые на каждом критерии дают~1. Следствие~1 доказано.
  
  \smallskip
  
  \noindent
  \textbf{Следствие~2.}\ Если проведено~$k$ итераций алгоритма кластеризации и~используемые на всех уровнях данные для критериев независимы друг от друга, 
то порядок их применения не влияет на получаемое значение~1 после всех 
итераций алгоритма.
  
  \smallskip
  
  
  \noindent
  Д\,о\,к\,а\,з\,а\,т\,е\,л\,ь\,с\,т\,в\,о\,.\ \ По теореме и~следствию~1 для любой~1 
в~результате $k$ итераций алгоритма все критерии принимают значение~1. Тогда 
любая пара из них перестановочна. Из теории подстановок~\cite{15-grusho} 
следует, что любая подстановка может быть получена произведением 
транспозиций. Отсюда следует утверждение следствия~2.
  
  Таким образом, результат кластеризации при\linebreak условии независимости 
дополнительной информации не зависит от порядка использования информации 
при формировании результирующего \mbox{кластера}. В~случае когда критерии носят 
вероятностный характер и~статистически не зависят друг от друга, итерация 
алгоритма отсева ЛТ по\-рож\-да\-ет множество независимых статистических данных 
на по\-рож\-ден\-ном кластере объектов.
  
  \section{Выявление вкрапления схем}
   
  Предположим, что некоторые данные в~строчке объектов $X_1, X_2, \ldots , 
X_n$ связаны при\-чин\-но-след\-ст\-вен\-ны\-ми связями. Такие сочетания 
событий будем\linebreak
 называть схемами. По логике при\-чин\-но-след\-ст\-венных 
связей схемы могут быть представлены\linebreak ориентированными ациклическими 
графами (DAG~--- Directed Acyclic Graph). Вершины таких DAG~--- это объекты 
из исходных данных, а~дуги направлены от объектов, составляющих причины, 
к~объ\-ек\-там-след\-ст\-ви\-ям. Пусть объекты исходных данных $X_1, X_2, \ldots 
, X_n$, не входящие в~схемы, построены независимо друг от друга, как было 
описано ранее. Размер схемы~--- это число вершин в~ней, он описывается 
параметром~$d$, множество значений которого много меньше~$n$. Признаки 
схемы~--- это фрагменты описывающего ее DAG. 
  
  Две схемы называются \textit{эквивалентными}, если они отличаются именами 
вершин, т.\,е.\ объекты могут отличаться, но при\-чин\-но-след\-ст\-вен\-ные 
связи между объектами соответствуют одной схеме. Эквивалентные схемы имеют 
изоморфные DAG. Кроме того, признаки эквивалентных схем также могут быть 
изоморфны. Если при этом часть значений параметров изоморфных признаков 
эквивалентных схем совпадает, то будем говорить о сходных признаках двух схем. 
При совпадении части признаков в~эквивалентных схемах будем называть эти 
схемы подозрительными.
  
  Рассмотрим задачу поиска вкраплений в~виде эквивалентных схем. Поскольку 
эквивалентные схемы могут быть построены только при переборе 
подпоследовательностей исходных данных, то поиск в~Big Data необходимо 
осуществлять по более простым изоморфным признакам. Упрощение алгоритма 
поиска может породить ЛТ. Таким образом, приходим к~задаче анализа 
результатов поиска на предмет сокращения ЛТ. 
  
  Далее будем использовать логику поиска ЛТ из предыдущего раздела. Для этого 
будем использовать дополнительную информацию. В~исходной схеме объекты, 
входящие в~нее, априори не известны. Сначала предположим, что искомой схеме 
соответствует граф~$G$, представленный в~виде DAG с~непомеченными 
объектами, но различающимися вершинами, который известен. Это означает, что 
известна система при\-чин\-но-след\-ст\-вен\-ных связей, порождающих схему, 
т.\,е.\ известны все по\-рож\-да\-ющие функции, по которым из значений некоторых 
определенных параметров исходных объектов (из разных вершин) следует 
значение или ограничение на параметр в~объ\-ек\-те-след\-ст\-вии. На графе~$G$ 
это означает наличие ориентированных дуг к~объ\-ек\-ту-след\-ст\-вию.
  
  Пусть $\alpha, \beta, \ldots$~--- подграфы~$G$, $G^*\hm= \{\alpha, \beta, 
\ldots\}$. Тогда изоморфные признаки двух эквивалентных схем будут иметь 
изоморфные элементы из~$G^*$. Одной частью дополнительной информации 
в~рассматриваемом поиске является возможная повторяемость сходных признаков 
эквивалентных схем. Отметим, что множество эквивалентных схем в~данных 
может состоять из двух и~более экземпляров. <<Лож\-ные тревоги>> могут состоять из 
элементов~$G^*$. Работа по поиску эквивалентных схем в~рассматриваемом 
случае распадается на два направления.
  \begin{enumerate}[{1}.1]
  
  \item  Прежде всего необходимо исследовать объекты, которые могут 
реализовать схему или ее часть с~точки зрения хотя бы приближенного описания  
при\-чин\-но-след\-ст\-вен\-ных связей, которые описываются в~схеме. Эти данные 
содержатся в~дополнительных данных, которые, в~свою очередь, содержатся 
в~описаниях параметров и~их значений. Результатом этих исследований станут 
потенциальные фрагменты графов фрагментов схем с~конкретными объектами 
вместо вершин графов, изоморфные элементам из~$G^*$. Построенные 
логические связи должны вытекать из значений части параметров реальных 
объектов. На основании полученных результатов можем нанести на исходные 
данные выявленные элементы причин и~следствий искомой схемы. Объекты, 
которые не попадают как потенциальные вершины графов, могут не 
рассматриваться. Тем самым происходит существенное сокращение 
рассматриваемых данных и~сужается кластер объектов дальнейшего 
рассмотрения.
  \item  Используя результаты~1.1, можем выделить потенциальные фрагменты 
эквивалентных схем на рас\-смат\-ри\-ва\-емом массиве данных.
  \item  Имея фрагменты и~логические основания построения 
  причинно-следственных связей, максимально достраиваем каждый фрагмент до графа, 
изоморфного графу~$G$ (это сложные алгоритмы, но данных меньше). 
Построенные схемы будут эквивалентными.
  \end{enumerate}
  \begin{enumerate}[{2}.1]
  \item Второе направление связано с~привязкой значений других, не 
задействованных в~формировании дуг, параметров в~объектах восстановленных 
графов или их фрагментов. Это направление связано с~выявлением 
подозрительных схем. Например, речь может идти о~дополнительной 
информации, связанной со свойствами субъектов, к~которым относятся объекты. 
Будем считать, что эта дополнительная информация не зависит от информации, 
порождающей при\-чин\-но-след\-ст\-вен\-ные связи. Однако используем эту 
дополнительную информацию во всех выделенных в~исходных данных 
эквивалентных схемах. Тогда подозрительные схемы будут состоять из 
эквивалентных схем, в~которых дополнительная информация о субъектах 
частично совпадает.
  \item  Полученное множество подозрительных схем не может быть большим 
и~вряд ли со\-дер\-жит~ЛТ.
  \end{enumerate}
  
  Поскольку можно разделять (считать независимой) дополнительную 
информацию о~при\-чин\-но-след\-ст\-вен\-ных связях и~информацию 
о~субъектности в~объектах, то по теореме их можно переставлять, т.\,е.\ сначала 
выделять субъектные значения параметров объектов, а~затем искать фрагменты 
схемы с~заданными значениями субъектных характеристик. С~точки зрения отсева 
ЛТ результаты будут совпадать.
  
  Если информации о схеме нет, но есть возможность определить отдельные 
эквивалентные при\-чин\-но-след\-ст\-вен\-ные связи, то их можно рас\-смат\-ри\-вать 
в~качестве простейших схем и, выделив их, попытаться избавиться от ЛТ на 
кластере меньшего размера данных. Далее, используя подозрительные 
простейшие графы, достраивать их до логически законченных схем. Отметим, что 
иногда схемы могут содержать меняющиеся компоненты. 
  
  Таким образом можно выявлять мошеннические схемы и~их участников.
  
  \section{Заключение}
  
  В статье построены алгоритмы снижения уровня ЛТ в~сложных гетерогенных 
последовательностях объектов (Big Data). Традиционно в~математической 
статистике такое снижение достигается минимизацией ошибки ЛТ. Однако 
в~задачах выявления вкрап\-ле\-ний (аномалий) такой подход ведет к~повышению 
вероятности потери искомых вкрап\-ле\-ний~[9, 10]. Для предотвращения такой 
потери предлагается использовать дополнительную вспомогательную 
информацию. Организация использования этой информации, предложенная 
в~статье, позволяет снижать объем клас\-те\-ра исходных данных, содержащий 
искомые вкрапления в~такой степени, чтобы использовать статистические оценки 
или логические методы, которые уже не по\-рож\-да\-ют ЛТ с~высокой ве\-ро\-ят\-ностью.
  
  Показано, что при выполнении некоторых условий порядок использования 
дополнительной информации не влияет на результат ее использования при 
фильтрации ЛТ.
  
  Результаты применения алгоритма фильтрации ЛТ в~последовательности 
объектов обобщены на фильтрацию ЛТ в~форме при\-чин\-но-след\-ст\-вен\-ных 
схем в~исходных данных. При известных схемах показано, как можно фильтровать 
ЛТ, выявляя только фрагменты схем.
  
  При фильтрации ЛТ в~схемах и~их фрагментах можно использовать различные 
виды дополнительной информации. При выполнении некоторых условий 
независимости в~этом случае также можно менять порядок использования 
дополнительной информации.
  
  Если рассматривать вкрапления как аномальную схему  
при\-чин\-но-след\-ст\-вен\-ных связей объектов исходных данных, то при 
определенных условиях можно выявлять схемы мошенничества и~их участников.
  
{\small\frenchspacing
 {%\baselineskip=10.8pt
 %\addcontentsline{toc}{section}{References}
 \begin{thebibliography}{99} 
\bibitem{1-grusho}
\Au{Axelsson S.} Intrusion detection systems: A~survey and taxonomy, 2000. {\sf 
http://www.cse.msu.edu/$\sim$cse960/\linebreak Papers/security/axelsson00intrusion.pdf}.

\bibitem{3-grusho} %2
\Au{Grusho A., Levykin~M., Timonina~E., Piskovski~V., Timonina~A.} Architecture of consecutive 
identification of attack to information resources~// 7th  Congress (International) on Ultra Modern 
Telecommunications and Control Systems Proceedings.~--- Piscataway, NJ, USA: IEEE, 
2015. P.~265--268. doi: 10.1109/ICUMT.2015.7382440. 

\bibitem{2-grusho} %3
\Au{Грушо А.\,А., Забежайло~М.\,И., Зацаринный~А.\,А., Тимонина~Е.\,Е.} О~некоторых 
возможностях управления ресурсами при организации проактивного противодействия 
компьютерным атакам~// Информатика и~её применения, 2018. Т.~12. Вып.~1. С.~62--70.

\bibitem{4-grusho}
\Au{Грушо А.\,А., Забежайло~М.\,И., Смирнов~Д.\,В., Тимонина~Е.\,Е.} Модель множества 
информационных пространств в~задаче поиска инсайдера~// Информатика и~её применения, 
2017. Т.~11. Вып.~4. C.~65--69.
\bibitem{5-grusho}
\Au{Грушо А.\,А., Забежайло~М.\,И., Смирнов~Д.\,В., Тимонина~Е.\,Е., Шоргин~С.\,Я.} Методы 
математической статистики в~задаче поиска инсайдера~// Информатика и~её применения, 2020. 
Т.~14. Вып.~3. С.~71--75. doi: 10.14357/19922264200310.
\bibitem{6-grusho}
\Au{Грушо Н.\,А., Грушо~А.\,А., Тимонина~Е.\,Е.} Локализация сбоев с~помощью метаданных~// 
Проблемы информационной безопасности. Компьютерные системы, 2020. №\,3. С.~9--15.
\bibitem{7-grusho}
\Au{Грушо А.\,А., Грушо~Н.\,А., Забежайло~М.\,И., Тимонина~Е.\,Е.} Локализация исходной 
причины аномалии~// Проб\-ле\-мы информационной безопас\-ности. Компьютерные системы, 
2020. №\,4. С.~9--16.
\bibitem{8-grusho}
\Au{Vaughan G.} Efficient big data model selection with applications to fraud detection~// Int.  
J.~Forecasting, 2018. Vol.~36. Iss.~3. P.~1116--1127.

\bibitem{10-grusho} %9
\Au{Grusho A., Grusho~N., Timonina~E.} The bans in finite probability spaces and the problem of 
small samples~// Distributed computer and communication networks~/ Eds. V.~Vishnevskiy, 
K.~Samouylov, D.~Kozyrev.~--- Lecture notes in computer science ser.~--- Springer, 2019. 
Vol.~11965. P.~578--590. doi: 10.1007/978-3-030-36614-8\_44. 

\bibitem{9-grusho} %10
\Au{Axelsson S.} The base-rate fallacy and the difficulty of intrusion detection~// ACM T.~Inform. Syst.
Se., 2000. Vol.~3. No.\,3. P.~186--205.
\bibitem{11-grusho}
\Au{Grusho A., Grusho~N., Timonina~E.} Method of several information spaces for identification of 
anomalies~// Intelligent distributed computing XIII~/ Eds. I.~Kotenko, C.~Badica, V.~Desnitsky, 
D.~El~Baz, M.~Ivanovic.~--- Studies in computational intelligence ser.~--- Springer, 2020. Vol.~868. 
P.~515--520. doi: 10.1007/978-3-030-32258-8\_60.
\bibitem{12-grusho}
\Au{Barik M., Sengupta~A., Mazumdar~C.} Attack graph generation and analysis techniques~// 
Defence Sci.~J., 2016. Vol.~66. No.\,6. P.~559--567. doi: 10.14429/dsj.66.10795.
\bibitem{13-grusho}
\Au{Grusho A., Grusho~N., Timonina~E.} Detection of anomalies in non-numerical data~// 8th  
Congress (International) on Ultra Modern Telecommunications and Control Systems Proceedings.~--- 
Piscataway, NJ, USA: IEEE, 2016. P.~273--276. doi: 10.1109/ICUMT.2016.7765370.
\bibitem{14-grusho}
\Au{Смирнов Д.\,В., Грушо~А.\,А., Забежайло~М.\,И., Тимонина~Е.\,Е.} Система сбора и~анализа 
информации из различных источников в~условиях Big Data~// Int. J.~Open Information 
Technologies, 2021. Vol.~9. No.\,4. P.~64--71. 
\bibitem{15-grusho}
\Au{Wielandt H.} Finite permutation groups.~--- New York\,/ London: Academic Press, 1964. 114~p.
\end{thebibliography}

 }
 }

\end{multicols}

\vspace*{-3pt}

\hfill{\small\textit{Поступила в~редакцию 17.09.21}}

%\vspace*{8pt}

%\pagebreak

\newpage

\vspace*{-28pt}

%\hrule

%\vspace*{2pt}

%hrule

%\vspace*{-2pt}

%\vspace*{6pt}

\def\tit{STATISTICS AND CLUSTERS FOR~DETECTION OF~ANOMALOUS INSERTIONS IN~BIG 
DATA ENVIRONMENT}


\def\titkol{Statistics and clusters for~detection of~anomalous insertions in~Big 
Data environment}


\def\aut{A.\,A.~Grusho$^1$, N.\,A.~Grusho$^1$, M.\,I.~Zabezhailo$^1$, D.\,V.~Smirnov$^2$, 
E.\,E.~Timonina$^1$,\\ and~S.\,Ya.~Shorgin$^1$}

\def\autkol{A.\,A.~Grusho, N.\,A.~Grusho, M.\,I.~Zabezhailo, et al.}
%D.\,V.~Smirnov$^2$,  E.\,E.~Timonina$^1$, and~S.\,Ya.~Shorgin$^1$}

\titel{\tit}{\aut}{\autkol}{\titkol}

\vspace*{-11pt}


\noindent
$^1$Federal Research Center ``Computer Science and Control'' of the Russian Academy of Sciences, 
44-2~Vavilov\linebreak
$\hphantom{^1}$Str., Moscow 119133, Russian Federation

\noindent
$^2$Sberbank of Russia, 19~Vavilov Str., Moscow 117999, Russian Federation


\def\leftfootline{\small{\textbf{\thepage}
\hfill INFORMATIKA I EE PRIMENENIYA~--- INFORMATICS AND
APPLICATIONS\ \ \ 2021\ \ \ volume~15\ \ \ issue\ 4}
}%
 \def\rightfootline{\small{INFORMATIKA I EE PRIMENENIYA~---
INFORMATICS AND APPLICATIONS\ \ \ 2021\ \ \ volume~15\ \ \ issue\ 4
\hfill \textbf{\thepage}}}

\vspace*{6pt} 



\Abste{The paper builds algorithms for reducing the level of ``false alarms'' when searching for 
anomalies in complex heterogeneous sequences of objects (Big Data). Traditionally, in mathematical 
statistics, such a decrease is achieved by minimizing the error of ``false alarms.'' However, in the 
problems of detecting anomalies (rare intrusions of anomalous data), this approach leads to an increase 
in the probability of losing the required anomalies. In this 
paper, in order not to lose the required anomalies, 
 on the contrary, in criteria designed for the least complexity of calculations, it is proposed to 
make a large error of the appearance of ``false alarms'' but use the fact that the number of objects 
allocated by such criteria is much smaller than the number of original objects in Big Data. The selected 
objects can then be grouped into a single cluster and additional information related to the objects in the 
cluster can be used to identify the required anomalies. The sense of these actions is that more 
difficult-to-compute characteristics of objects for dropping out ``false alarms'' will not require large 
computational resources on a smaller cluster of objects relative to the original data. It is shown that 
when certain conditions are satisfied, the order of using additional information does not affect the 
result of its use when filtering ``false alarms.'' The results of the filtering algorithm in the sequence of 
objects are generalized to filtering ``false alarms'' in the form of causal schemes in the initial data. 
Known schemes show how ``false alarms'' can be filtered identifying only fragments of schemes.}

\KWE{information security; search for anomalies; algorithms for filtering ``false alarms''}



\DOI{10.14357/19922264210411}

\vspace*{-12pt}

\Ack
\noindent
The paper was partially supported by the Russian Foundation for Basic Research 
(project 18-29-03081).


%\vspace*{6pt}

  \begin{multicols}{2}

\renewcommand{\bibname}{\protect\rmfamily References}
%\renewcommand{\bibname}{\large\protect\rm References}

{\small\frenchspacing
 {%\baselineskip=10.8pt
 \addcontentsline{toc}{section}{References}
 \begin{thebibliography}{99}
\bibitem{1-grusho-1}
\Aue{Axelsson, S.} 2002. Intrusion detection systems: A~survey and taxonomy. Available at: {\sf 
http://www.cse.msu.\linebreak edu/$\sim$cse960/Papers/security/axelsson00intrusion.pdf} (accessed 
November~15, 2021).

\bibitem{3-grusho-1}
\Aue{Grusho, A., M.~Levykin, E.~Timonina, V.~Piskovski, and A.~Timonina.} 2015. Architecture of 
consecutive identification of attack to information resources. \textit{7th Congress (International) on 
Ultra Modern Telecommunications and Control Systems Proceedings}. Piscataway, NJ: IEEE.  
265--268. doi: 10.1109/ICUMT.2015.7382440. 

\bibitem{2-grusho-1} %3
\Aue{Grusho, A.\,A., M.\,I.~Zabezhailo, A.\,A.~Zatsarinny, and E.\,E.~Timonina.} 2018. 
O~nekotorykh vozmozhnostyakh upravleniya resursami pri organizatsii proaktivnogo protivodeystviya 
komp'yuternym atakam [On some pos-sibilities of resource management for organizing active 
counteraction to computer attacks]. \textit{Informatika i~ee Primeneniya~--- Inform. Appl.} 
 12(1):62--70.
 
\bibitem{4-grusho-1}
\Aue{Grusho, A.\,A., M.\,I.~Zabezhailo, D.\,V.~Smirnov, and E.\,E.~Timonina.} 2017. Model' 
mnozhestva informatsionnykh prostranstv v~zadache poiska insaydera [The model of the set of 
information spaces in the problem of insider detection]. \textit{Informatika i~ee Primeneniya~--- 
Inform. Appl.} 11(4):65--69.
\bibitem{5-grusho-1}
\Aue{Grusho, A.\,A., M.\,I.~Zabezhailo, D.\,V.~Smirnov, E.\,E.~Timonina, and S.\,Ya.~Shorgin.} 
2020. Metody ma\-te\-ma\-ti\-che\-skoy statistiki v~zadache poiska insaydera [Mathematical statistics in task 
of identifying hostile insiders]. \textit{Informatika i~ee Primeneniya~--- Inform. Appl.} 14(3):71--75. 
doi: 10.14357/19922264200310.
\bibitem{6-grusho-1}
\Aue{Grusho, N.\,A., A.\,A.~Grusho, and E.\,E.~Timonina.} 2020. Lokalizatsiya sboev s pomoshch'yu 
metadannykh [Localizing failures with metadata]. \textit{Problemy informatsionnoy bezopasnosti. 
Komp'yuternye sistemy} [Problems of Information Security. Computer Systems] 3:9--15.
\bibitem{7-grusho-1}
\Aue{Grusho, A.\,A., N.\,A.~Grusho, M.\,I.~Zabezhailo, and E.\,E.~Timonina.} 2020. Lokalizatsiya 
iskhodnoy prichiny anomalii [Root cause anomaly localization]. \textit{Problemy informatsionnoy 
bezopasnosti. Komp'yuternye sistemy} [Problems of Information Security. Computer Systems] 4:9--16.
{\looseness=1

}

\bibitem{8-grusho-1}
\Aue{Vaughan, G.} 2018. Efficient big data model selection with applications to fraud detection. 
\textit{Int. J.~Forecasting} 36(3):1116--1127.


\bibitem{10-grusho-1}
\Aue{Grusho, A., N. Grusho, and E.~Timonina.} 2019. The bans in finite probability spaces and the 
problem of small samples. \textit{Distributed computer and communication networks}. Eds. 
V.\,M.~Vishnevskiy, K.\,E.~Samouylov, and D.\,V.~Kozyrev. Lecture notes in computer science ser. 
Springer. 11965:578--590. doi: 10.1007/978-3-030-36614-8\_44. 

\bibitem{9-grusho-1}
\Aue{Axelsson, S.} 2000. The base-rate fallacy and its implications for the difficulty of intrusion 
detection. \textit{ACM T.~Inform. Syst. Se.} 3(3):186--205.

\bibitem{11-grusho-1}
\Aue{Grusho, A., N.~Grusho, and E.~Timonina.} 2020. Method of several information spaces for 
identification of anomalies. \textit{Intelligent distributed computing XIII.} Eds. I.~Kotenko, 
C.~Badica, V.~Desnitsky, D.~El~Baz, and M.~Ivanovic. Studies in computational intelligence ser. 
Springer. 868:515--520. doi: 10.1007/978-3-030-32258-8\_60.
\bibitem{12-grusho-1}
\Aue{Barik, M., A.~Sengupta, and C.~Mazumdar.} 2016. Attack graph generation and analysis 
techniques. \textit{Defence Sci.~J.} 66(6):559--567. doi: 10.14429/dsj.66.10795.
\bibitem{13-grusho-1}
\Aue{Grusho, A., N.~Grusho, and E.~Timonina.} 2016. Detection of anomalies in non-numerical data. 
\textit{8th Congress (International) on Ultra Modern Telecommunications and Control Systems and 
Workshops Proceedings}. Piscataway, NJ: IEEE. 273--276. doi: 10.1109/ICUMT.2016.7765370.
\bibitem{14-grusho-1}
\Aue{Smirnov, D.\,V., A.\,A.~Grusho, M.\,I.~Zabezhailo, and E.\,E.~Timonina.} 2021. System for 
collecting and analyzing information from various sources in Big Data conditions. \textit{Int. J.~Open 
Information Technologies} 9(4):64--71.
\bibitem{15-grusho-1}
\Aue{Wielandt, H.} 1964. \textit{Finite permutation groups}. New York\,/ London: Academic Press. 
114~p.
\end{thebibliography}

 }
 }

\end{multicols}

\vspace*{-3pt}

\hfill{\small\textit{Received September 17, 2021}}

%\pagebreak

%\vspace*{-24pt}

\Contr

\noindent
\textbf{Grusho Alexander A.} (b.\ 1946)~--- Doctor of Science in physics and mathematics, professor, 
principal scientist, Institute of Informatics Problems, Federal Research Center ``Computer Science and 
Control'' of the Russian Academy of Sciences, 44-2~Vavilov Str., Moscow 119333, Russian 
Federation; \mbox{grusho@yandex.ru}

\vspace*{3pt}

\noindent
\textbf{Grusho Nikolai A.} (b.\ 1982)~--- Candidate of Science (PhD) in physics and mathematics, 
senior scientist, Institute of Informatics Problems, Federal Research Center ``Computer Science and 
Control'' of the Russian Academy of Sciences, 44-2~Vavilov Str., Moscow 119133, Russian 
\mbox{Federation; info@itake.ru}

\vspace*{3pt}

\noindent
\textbf{Zabezhailo Michael I.} (b.\ 1956)~--- Doctor of Science in physics and mathematics, principal 
scientist, A.\,A.~Dorodnicyn Computing Center, Federal Research Center ``Computer Science and 
Control'' of the Russian Academy of Sciences, 40~Vavilov Str., Moscow 119133, Russian 
Federation; \mbox{m.zabezhailo@yandex.ru}

\vspace*{3pt}



\noindent
\textbf{Smirnov Dmitry V.} (b.\ 1984)~--- business partner for IT security department, Sberbank of 
Russia, 19~Vavilov Str., Moscow 117999, Russian Federation; \mbox{dvlsmirnov@sberbank.ru}

\vspace*{3pt}

\noindent
\textbf{Timonina Elena E.} (b.\ 1952)~--- Doctor of Science in technology, professor, leading scientist, 
Institute of Informatics Problems, Federal Research Center ``Computer Science and Control'' of the 
Russian Academy of Sciences, 44-2~Vavilov Str., Moscow 119133, Russian Federation; 
\mbox{eltimon@yandex.ru}

\vspace*{3pt}

\noindent
\textbf{Shorgin Sergey Ya.} (b.\ 1952)~--- Doctor of Science in physics and mathematics, professor, 
principal scientist, Institute of Informatics Problems, Federal Research Center ``Computer Science and 
Control'' of the Russian Academy of Sciences, 44-2~Vavilov Str., Moscow 119133, Russian 
Federation; \mbox{sshorgin@ipiran.ru}

\label{end\stat}

\renewcommand{\bibname}{\protect\rm Литература}