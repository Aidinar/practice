\def\stat{bosov+ign}

\def\tit{АЛГОРИТМЫ ПРИБЛИЖЕННОГО РЕШЕНИЯ ЗАДАЧИ НАЗНАЧЕНИЯ <<ТЕХНОЛОГИЧЕСКОГО 
ОКНА>> НА~УЧАСТКАХ~ЖЕЛЕЗНОДОРОЖНОЙ СЕТИ$^*$}

\def\titkol{Алгоритмы приближенного решения задачи назначения <<технологического 
окна>> на участках ж/д сети}

\def\aut{А.\,В.~Босов$^1$, А.\,Н.~Игнатов$^2$, А.\,В.~Наумов$^3$}

\def\autkol{А.\,В.~Босов, А.\,Н.~Игнатов, А.\,В.~Наумов}

\titel{\tit}{\aut}{\autkol}{\titkol}

\index{Босов А.\,В.}
\index{Игнатов А.\,Н.}
\index{Наумов А.\,В.}
\index{Bosov A.\,V.}
\index{Ignatov A.\,N.}
\index{Naumov A.\,V.}

{\renewcommand{\thefootnote}{\fnsymbol{footnote}} \footnotetext[1]
{Работа выполнена при частичной поддержке РФФИ (проект 20-07-00046~А).}}


\renewcommand{\thefootnote}{\arabic{footnote}}
\footnotetext[1]{Федеральный исследовательский центр <<Информатика и~управление>> Российской академии наук; 
Московский авиационный институт, \mbox{avbosov@ipiran.ru}}
\footnotetext[2]{Московский авиационный институт, alexei.ignatov1@gmail.com}
\footnotetext[3]{Московский авиационный институт, naumovav@mail.ru}

%\vspace*{11pt}


 \vspace*{-2pt}
      
      \Abst{Развивается алгоритмическое и~инструментальное обеспечение решения задач 
управления железнодорожным (ж/д) транспортом, основанное на представлении прикладных 
задач в~форме оптимизационных постановок, в~которых применяются средства линейного 
программирования. Ранее предложенные модели и~прикладные постановки расширены новой 
задачей поиска <<технологического окна>> (ТО)~--- промежутка времени, в~которое некоторые 
участки железнодорожной сети закрываются для проведения ремонтных работ. Для ее решения 
предложена математическая модель и~оптимизационная постановка для одновременного поиска 
ТО и~расписания движения поездов по некоторому сегменту ж/д 
сети. Исходная прикладная постановка сведена к~задаче смешанного целочисленного линейного 
программирования. Для учета возможных вычислительных сложностей при решении задачи 
предложен способ поиска приближенного решения, который основан на формировании 
базового расписания движения и~последующей его корректировки с~учетом необходимости 
в~ТО. Для поиска приближенного решения реализованы два алгоритма. 
В~первом строится базовое и~скорректированное расписание движения поездов поэтапно по 
группам поездов, объединенных одинаковостью станций отправлений и~станций назначения, во 
втором этапность осуществляется по одному поезду согласно времени готовности 
к~отправлению. Приведены результаты численного эксперимента.}
      
      
      \KW{мультиграф; железнодорожная сеть; расписание; технологическое окно; смешанное 
целочисленное линейное программирование}

\DOI{10.14357/19922264210401}

\vspace*{-1pt}


\vskip 10pt plus 9pt minus 6pt

\thispagestyle{headings}

\begin{multicols}{2}

\label{st\stat}


       
\section{Введение}

  Задача назначения ТО~--- одна из актуальных задач управления ж/д 
транспортом. Это задача поиска промежутка времени, в~которое некоторые 
участки ж/д сети закрываются для проведения ремонтных работ. В~идеальном 
случае следует выбирать ТО так, чтобы оно не влияло на расписание движения, 
т.\,е.\ в~те моменты времени, когда участки, подлежащие ремонту, свободны от 
движения. Однако в~связи с~интенсивностью движения поездов такое не всегда 
возможно. При этом произвольный выбор промежутка времени для ТО может 
при\-вес\-ти к~задержкам и~отмене поездов. В~этой связи возникает задача поиска 
оптимального с~точки зрения различных критериев промежутка времени для 
назначения ТО.
  
  Среди работ, посвященных исследованию данной задачи, выделим~[1--6]. 
В~\cite{1-bos, 2-bos, 3-bos, 4-bos, 5-bos} рассматривалась задача назначения ТО на 
некотором сегменте ж/д сети, а~в~\cite{6-bos}~--- на одной станции.  
В~\cite{1-bos, 3-bos, 5-bos} задача решалась одновременно с~поиском расписания\linebreak 
движения поездов, в~\cite{2-bos, 4-bos, 6-bos} предлагалось корректировать 
некоторое исходное расписание и~встраивать ТО в~него. В~\cite{1-bos, 3-bos} 
движение поездов предлагалось осуществлять по некоторым заранее \mbox{заданным} 
маршрутам движения. Задача назначения ТО в~\cite{1-bos, 5-bos, 6-bos} была 
сформулирована в~виде задачи смешанного целочисленного линейного 
программирования.
  
  Данная статья представляет модификацию результатов~\cite{5-bos, 7-bos} 
и~продолжает развитие алгоритмического и~инструментального обеспечения 
решения задач управления ж/д транспортом, основанного на 
формализации прикладных задач в~математических моделях, оптимизируемых 
средствами линейного программирования. Развивая эту методику, в~работе 
используется предложенная в~\cite{5-bos} и~усовершенствованная в~\cite{7-bos} 
математическая модель движения по мультиграфу ж/д сети, а~также 
представленный в~\cite{5-bos} универсальный критерий оптимальности для 
формирования расписания движения. На основе модели из~\cite{7-bos} и~ряда 
ограничений из~\cite{5-bos}, посвященных учету необходимости в~ТО, поставлена 
новая задача назначения ТО на участках ж/д сети. Ввиду вычислительных 
сложностей при поиске оптимального решения в~полученной задаче предложены 
два алгоритма поиска приближенного решения. Данные алгоритмы сравниваются 
на содержательном примере.

\vspace*{-4pt}

\section{Основные обозначения и~предположения}

\vspace*{-4pt}
  
  Рассмотрим ж/д сеть, пред\-став\-ля\-емую неориентированным мультиграфом 
$G\hm= \langle V, E\rangle$, где $V$~--- множество узлов (станций, где 
происходит ветвление ж/д сети, у~которых число входящих путей не равно  
чис\-лу исходящих путей, сортировочных или конечных); $E$~--- множество 
ребер (путей), соединяющих данные вершины. Пусть число узлов $\vert V\vert \hm 
= m$. Пронумеровав вершины графа~$G$, со\-ста\-вим множество индексов 
$V^\prime\hm= \{1, 2, \ldots , M\}$, каж\-дый элемент которого определяет 
вершину графа~$G$. Предполагается естественное ограничение $M\hm\geq 2$.
  
  Пусть имеется~$I$~поездов. Для каждого $i$-го поезда, $i\hm= \overline{1,I}$, 
заданы: индекс вершины отправления $v_i^{\mathrm{отправ}}\hm\in V^\prime$; 
индекс вершины назначения $v_i^{\mathrm{приб}}\hm\in V^\prime$; время 
готовности к~отправлению~$t_i^{\mathrm{отправ}}$, которое вычисляется как 
число минут от некоторого заданного момента отсчета; максимальное 
время~$d_i$, в~течение которого поезду позволяется находиться в~пункте 
отправления с~момента готовности; время в~пути~$T_i$, т.\,е.\ максимальное 
время, в~течение которого поезду позволяется находиться на ж/д сети, 
вычисляемое в~минутах.
  
  Движение поездов по перегонам (между вершинами) ж/д сети может 
осуществляться только в~определенные промежутки времени. Для описания таких 
промежутков используется понятие множества бесконфликтных <<подниток>>. 
Это некоторое подмножество всех <<подниток>>, обладающее свойством 
бесконфликтности в~том смысле, что в~этом множестве не может быть двух 
<<подниток>>, при использовании которых два поезда столкнутся. Таким 
множеством пользовались, например, в~\cite{8-bos} при исследовании задачи 
назначения локомотивов. Далее это множество обозначается~$Z$, считается 
заданным априорно, т.\,е.\ является входным параметром в~рассматриваемой 
задаче. Пусть число элементов $\vert Z\vert\hm=K$ и~они пронумерованы от~1 
до~$K$. Каждый элемент $z_k\hm\in Z$, $k\hm= \overline{1,K}$, представляет 
собой пятерку $z_k\hm= (v_k^{\mathrm{нач}}, v_k^{\mathrm{кон}}, n_k, 
t_k^{\mathrm{нач}}, t_k^{\mathrm{кон}})$, где $ v_k^{\mathrm{нач}},  
v_k^{\mathrm{кон}} \hm\in V^\prime$~--- индексы смежных вершин в~графе~$G$ 
начала и~конца движения по <<поднитке>>; $n_k$~--- номер пути, со\-еди\-ня\-юще\-го 
эти вершины; $t_k^{\mathrm{нач}}$ и~$t_k^{\mathrm{кон}}$~--- время начала 
и~конца движения.
{\looseness=1

}
  
  На движение поезда также могут быть наложены ограничения, связанные со 
стоянкой на промежуточных станциях. Например, на некоторых станциях должна 
проходить смена локомотивов, от\-цеп\-ка/при\-цеп\-ка вагонов, а~некоторые 
станции, напротив, надо проезжать без остановки. Поэтому введем минимально 
и~максимально возможную длительность стоянки на станции с~индексом 
вершины~$v_k^{\mathrm{кон}}$ после использования <<поднитки>> 
с~номером~$k$ поезда с~номером~$i$: $t_{i,k}^{\mathrm{ст.мин}}$ 
и~$t_{i,k}^{\mathrm{ст.макс}}$. За $T_{\mathrm{макс}}$ обозначим 
длительность периода движения: например, если расписание строится на сутки, то 
$T_{\mathrm{макс}}\hm= 1440$~мин.
  
  Пусть $\Delta$~--- минимальная продолжительность ТО, а $t_1^0$~--- время, не 
ранее которого ТО может начаться, а~$t_2^0$~--- окончиться. Через $Z^\prime 
\hm\subset \{1, \ldots , K\}$ обозначим множество <<подниток>>, содержащих 
перегоны, на которых должны проводиться работы.

\vspace*{-4pt}

\section{Математическая модель движения поездов по~графу железнодорожной сети}

\vspace*{-4pt}

  Поставим задачу отыскания времени следования указанных выше~$I$ поездов 
через ж/д сеть, задаваемую мультиграфом~$G$, на основе множества 
<<подниток>>~$Z$. Маршрутом $i$-го поезда называется последовательность 
пересекаемых им вершин.
  
  Введем переменные $\delta_{i,j,k}$, характеризующие использование $i$-м 
поездом <<поднитки>> с~номером~$k$ на $j$-м этапе маршрута следования (при 
движении от $j$-й до ($j\hm+1$)-й по порядку следования вершины), $i\hm= 
\overline{1, I}$, $j\hm= \overline{1, J+1}$, $k\hm= \overline{1,K}$. Здесь~$J$~--- наперед 
задаваемый параметр~--- чис\-ло <<реальных>> этапов, на которых происходит 
движение от станции к~станции, этап $J\hm+1$~--- технический, движения на нем 
не происходит, он нужен для корректного задания в~модели условия прибытия 
в~назначенный пункт. Переменная $\delta_{i,j,k}$ равна~1, если $i$-м поездом на 
$j$-м этапе задействована <<под\-нит\-ка>> с~номером~$k$, иначе~0. Воспользуемся 
математической моделью движения по мультиграфу ж/д сети из~\cite{7-bos}, 
за\-да\-ющей\-ся сле\-ду\-ющи\-ми ограничениями на бинарные 
переменные~$\delta_{i,j,k}$:
\begin{itemize}
\item 
  для  $i=\overline{1,I}$, $k=\overline{1,K}$, $m=\overline{1,M}$:
    \begin{multline*}
 \sum\limits^I_{i=1} \sum\limits_{j=1}^{J+1} \delta_{i,j,k}\leq 1\,,\\
 \sum\limits_{k=1}^K \delta_{i,1,k}=1\,,\ \sum\limits^K_{k=1} \delta_{i,1,k} 
v_k^{\mathrm{нач}}=v_i^{\mathrm{отправ}}\,,
\end{multline*}

\noindent
\begin{multline}
t_i^{\mathrm{отправ}} \leq \sum\limits_{k=1}^K\delta_{i,1,k} 
t_k^{\mathrm{нач}}\leq t_i^{\mathrm{отправ}}+d_i\,,\\[1pt]
\displaystyle\sum\limits_{j=1}^{J+1}\sum\limits_{k: v_k^{\mathrm{нач}}=m, 1\leq k\leq K} 
\delta_{i,j,k}\leq 1\,;
  \label{e1-bos}
\end{multline}
 \item для $j=\overline{1,J-1}$: 
 \begin{multline}
 \sum\limits_{k=1}^K\delta_{i,j,k}v_k^{\mathrm{кон}}\leq 
\sum\limits^K_{k=1} \delta_{i,j+1,k} v_k^{\mathrm{нач}} +{}\\[1pt]
{}+\left( 1-
\sum\limits^K_{k=1} \delta_{i,j+1,k}\right)M^3\,,\\[1pt]
 \sum\limits^K_{k=1} \delta_{i,j,k} v_k^{\mathrm{кон}}\geq 
\sum\limits_{k=1}^K \delta_{i,j+1,k} v_k^{\mathrm{нач}}-{}\\[1pt]
{}-\left( 1- \sum^K_{k=1} 
\delta_{i,j+1,k}\right)M\,,\\[1pt]
  \sum\limits^K_{k=1} \delta_{i,j,k}\left( 
t_k^{\mathrm{кон}}+t_{i,k}^{\mathrm{ст.мин}}\right) \leq {}\\[1pt]
{}\leq 
\sum\limits^K_{k=1}\delta_{i,j+1,k} t_k^{\mathrm{нач}} +2\!\left(\!1- 
\sum\limits^K_{k=1} \delta_{i,j+1,k}\!\right) T_{\mathrm{макс}},\\[1pt]
  \sum\limits^K_{k=1} \delta_{i,j,k} \left( 
t_k^{\mathrm{кон}}+t_{i,k}^{\mathrm{ст.макс}}\right) \geq \sum\limits_{k=1}^K 
\delta_{i,j+1,k} t_k^{\mathrm{нач}}\,;
\end{multline}
\item для  $j=\overline{1,J}$:\\
\begin{multline}
 \sum\limits^K_{k=1} \delta_{i,j,k} t_k^{\mathrm{кон}} -
\sum\limits_{k=1}^K \delta_{i,1,k}t_k^{\mathrm{нач}} \leq T_i\,,\\[1pt]
 \sum\limits^K_{k=1} \delta_{i,j,k} v_k^{\mathrm{кон}} \geq 
\left(\sum\limits^K_{k=1}
   \left( \delta_{i,j,k}-\delta_{i,j+1,k}\right)\right) v_i^{\mathrm{приб}},\\[1pt]
\sum\limits^I_{i=1} \sum\limits^K_{k=1}
  \delta_{i,J+1,k}=0\,,\\[1pt]
  \sum\limits^K_{k=1}
   \delta_{i,j,k} v_k^{\mathrm{кон}} \!\leq \!\left( \! 1+ \!\sum\limits^K_{k=1}\!
   \left( \delta_{i,j+1,k}-\delta_{i,j,k}\right)\!\right)\! M+{}\\[1pt]
   {}+\left(\sum\limits^K_{k=1}\!
   \left( \delta_{i,j,k} -\delta_{i,j+1,k}\right)\right) v_i^{\mathrm{приб}}\,.
  \end{multline}
  \end{itemize}
  
   
  
  В ограничениях~(\ref{e1-bos})--(3), в~част\-ности, учитывается, что отправление 
поездов может осуществляться только из соответствующих вершин отправления, 
движение возможно только по смежным вершинам мультиграфа~$G$, 
отправление поездов из промежуточных станций на маршруте не может быть 
раньше прибытия, используются маршруты, не содержащие цик\-лов. Детальное 
описание представленных ограничений можно найти в~\cite{7-bos}. 

\section{Дополнительные ограничения для~назначения <<технологического окна>>}

  Для исключения возможности движения по <<подниткам>>, связанным 
с~дугами мультиграфа~$G$, подлежащими ремонту, воспользуемся 
ограничениями из~\cite{5-bos}. Для этого введем вспомогательные бинарные 
переменные~$\gamma_{k^\prime}$ и~$\ae_{k^\prime}$, $k^\prime\hm\in Z^\prime$, 
и~ограничения
  \begin{multline}
  \gamma_{k^\prime}t_{k^\prime}^{\mathrm{кон}} \leq t_1,\ t_2\leq 
\ae_{k^\prime}t_{k^\prime}^{\mathrm{нач}} +\left( 1-\ae_{k^\prime} \right) 
T_{\mathrm{макс}}\,,\\
  \delta_{i,j,k^\prime} \leq \ae_{k^\prime} +\gamma_{k^\prime}\,,\enskip
  i=\overline{1,I}\,,\ j=\overline{1,J}\,,\
  k^\prime\in Z^\prime\,;
  \label{e2-bos}
  \end{multline}
  \begin{equation}
  t_2-t_1\geq \Delta\,,\enskip t_1^0\leq t_1\,,\enskip t_2\leq t_2^0\,.
  \label{e3-bos}
  \end{equation}
  
  Ограничения~(\ref{e2-bos}) позволяют исключить из движения <<поднитки>>, 
попадающие в~ТО. Ограничения~(\ref{e3-bos}) гарантируют, что ТО будет не 
меньше заданной длительности~$\Delta$, оно начнется не раньше~$t_1^0$ 
и~закончится не позднее~$t_2^0$.

\section{Выбор критерия}

  Воспользовавшись критерием из~\cite{7-bos}, получаем задачу оптимизации
  \begin{multline}
  c_1\sum\limits_{i=1}^I \sum\limits_{j=1}^{J+1} \sum\limits_{k=1}^K 
\delta_{i,j,k}\left( t_k^{\mathrm{кон}} -t_k^{\mathrm{нач}} \right) +c_2 
\sum\limits_{i=1}^I \sum\limits_{j=1}^{J+1} \hat{T}_{i,j} +{}\\
  {}+c_3\left(\sum\limits_{i=1}^I \sum\limits_{k=1}^K 
\delta_{i,j,k} t_k^{\mathrm{нач}} -\sum\limits^I_{i=1} 
t_i^{\mathrm{отправ}}\right)\to \min
  \label{e4-bos}
  \end{multline}
при ограничениях~(\ref{e1-bos})--(\ref{e3-bos}) и
\begin{multline}
\hat{T}_{i,j}\geq \sum\limits^K_{k=1} \delta_{i,j+1,k} t_k^{\mathrm{нач}} -
\sum\limits^K_{k=1} \delta_{i,j,k}t_k^{\mathrm{кон}}\,,\\
 i=\overline{1,I}\,,\enskip  j=\overline{1, J}\,.
\label{e5-bos}
\end{multline}
  
  Критерий~(\ref{e4-bos}) представляет собой свертку трех критериев, вклад 
каждого регулируется выбором положительных констант~$c_1$, $c_2$ и~$c_3$. 
Первое слагаемое отвечает за суммарное время в~движении, второе~--- за 
суммарную длительность стоянок на промежуточных станциях, третье~--- за 
суммарную длительность стоянок на станциях отправления. Через~$\hat{T}_{i,j}$ 
обозначено время стоянки $i$-го поезда на ($j\hm+1$)-й по порядку станции 
в~маршруте, $\hat{T}_{i,J+1}\hm=0$. Минимум берется по переменным $t_1, t_2, 
\delta_{i,j,k}, \hat{T}_{i,j}\hm\geq 0$, $\gamma_{k^\prime}, \ae_{k^\prime}$, $i\hm=\overline{1,I}$, 
$j\hm=\overline{1,J+1}$, $k\hm= \overline{1,K}$, $k^\prime\hm\in Z^\prime$.


\section{Алгоритм поиска приближенного/начального решения}
  
  Как было отмечено в~\cite{7-bos}, решение задачи~(\ref{e4-bos}) даже при 
ограничениях~(\ref{e1-bos})--(3) и~(\ref{e5-bos}) весьма трудоемко, 
ограничения~(\ref{e2-bos}) и~(\ref{e3-bos}) делают задачу еще сложнее, поэтому 
предлагается следующая последовательность действий по поиску приближенного 
решения. Вначале необходимо построить расписание движения в~предположении 
отсутствия ТО. Далее ТО подбирается так, чтобы в~него попадало минимальное 
число поездов (аналогичный прием использован для назначения ТО на станции 
в~\cite{6-bos}). Далее нужно исключить <<поднитки>>, попадающие в~ТО, в~том 
числе нарушающие ограничение~(\ref{e2-bos}), и~заново построить расписание 
движения для всех поездов.
  
  Итак, пусть найдено некоторое расписание движения поездов по всей ж/д сети. 
Обозначим через $Z_i\hm\subset \{1, \ldots , K\}$ множество номеров 
<<подниток>>, используемых при движении $i$-м поездом. Далее сформируем 
множество номеров поездов $\overline{\Im}\hm= \{i: Z_i\cap Z^\prime \not= 
\emptyset\}$. Если множество~$\overline{\Im}$ пусто, то можно назначить ТО 
любой длительности в~пределах горизонта планирования. Если же 
множество~$\overline{\Im}$ непусто, то нужно решить задачу по минимизации 
числа поездов, попадающих (по времени и~месту) в~ТО. Для этого введем 
дополнительные переменные   $\tilde{\delta}_p$, равные~0, если маршрут поезда 
с~номером~$p$ попадает в~ТО, и~1 в~обратном случае. Также введем новые 
переменные $\tilde{\delta}_p^{k_p}$, равные~0, если пересечение отрезков $[t_1, 
t_2]$ и~$[t_{k_p}^{\mathrm{нач}}, t_{k_p}^{\mathrm{кон}}]$ состоит максимум 
из одной точки, и~1 иначе, $p\hm\in \overline{\Im}$, $k_p\hm\in Z_p\cap Z^\prime$. 
Равенство~1 переменной $\tilde{\delta}_p^{k_p}$ означает, что используемая $p$-м 
поездом <<поднитка>> c~номером~$k_p$ будет недоступна вследствие ТО; для 
равенства~0 переменной $\tilde{\delta}_p^{k_p}$ нужно выполнение условия 
$t_1\hm\geq t_{k_p}^{\mathrm{кон}}$ либо $t_2\hm\leq 
t_{k_p}^{\mathrm{нач}}$. Поэтому введем дополнительные переменные 
$\alpha_p^{k_p}$ и~$\beta_p^{k_p}$: 
$$
\alpha_p^{k_p}=\begin{cases}
0, & \mbox{если }t_1\geq t_{k_p}^{\mathrm{кон}}\,;\\[-2pt]
1 & \mbox{иначе}\,;
\end{cases}
$$

\vspace*{-4pt}

\noindent
$$
\beta_p^{k_p}=\begin{cases}
0, &\mbox{если } t_2\leq t_{k_p}^{\mathrm{нач}};\\[-2pt] 
1 & \mbox{иначе.}
\end{cases}
$$

\vspace*{-3pt}

\noindent
 Так что 
$\tilde{\delta}_p^{k_p}\hm=1$ только при $\alpha_p^{k_p} \hm= 
\beta_p^{k_p}\hm=1$.
  
  С использованием введенных переменных получаем следующую задачу:
  
  \vspace*{-10pt}
  
  \noindent
    \begin{multline*}
  \sum\limits_p\tilde{\delta}_p\to \min\,,\enskip \tilde{\delta}_p\geq 
\fr{1}{K}\left( \sum\limits_{k_p}\tilde{\delta}_p^{k_p}\right)\,,\\
 p\in \tilde{\Im}\,,\ k_p\in Z_p\cap Z^\prime\,,\hspace*{10mm}
\end{multline*}

\columnbreak

\noindent
  \begin{multline}
  t_1\geq \left(1-\alpha_p^{k_p}\right) t_{k_p}^{\mathrm{кон}}\,,\\
  t_2\leq \left( 1- \beta_p^{k_p}\right) t_{k_p}^{\mathrm{нач}} +\beta_p^{k_p} 
T_{\mathrm{макс}}\,,\\
  \tilde{\delta}_p^{k_p}\geq \alpha_p^{k_p}+\beta_p^{k_p}-1\,,\\
  t_2-t_1\geq \Delta\,, \ t_1^0\leq t_1\,,\ t_2\leq t_2^0\,.
   \label{e6-bos}
  \end{multline}
  
  Минимум в~(\ref{e6-bos}) берется по переменным $t_1$, $t_2$, 
$\tilde{\delta}_p$, $\tilde{\delta}_p^{k_p}$, $\alpha_p^{k_p}$ и~$\beta_p^{k_p}$. 
Заметим, что в~задачу~(\ref{e6-bos}) включено ограничение~(\ref{e3-bos}).
     
     Если оптимальное значение минимизируемого в~(\ref{e6-bos}) критерия 
$\sum\nolimits_{p\in\overline{\Im}} \tilde{\delta}_p$ равно~0, то перестраивать 
имеющееся расписание не нужно. Если же оно больше~0, то нужно перестраивать 
расписание движения поездов с~учетом <<подниток>>, попадающих в~ТО, т.\,е.\ 
не использовать эти <<поднитки>> при поиске расписания. В~итоге получаем 
следующий алгоритм.
   \begin{description}  
\item [\,] \textbf{Шаг 1.} Множество номеров поездов дробится на~$S$~непересекающихся 
подмножеств $\Im_s$, т.\,е.\ $\{1,\ldots , I \}\hm= \cup^S_{s=1} \Im_s$, 
причем $\forall\, s_1\not= s_2$ $\Im_{s_1}\cap \Im_{s_2}\hm=\emptyset$.\\[-15pt]
\item [\,] 
\textbf{Шаг 2.}\ Инициализируется параметр $s\hm=1$. Формируется множество 
$\aleph_0\hm= \emptyset$.\\[-15pt]
\item [\,] 
\textbf{Шаг 3.}\ Решается задача минимизации критерия~(\ref{e4-bos}) 
с~ограничениями~(\ref{e1-bos})--(3) и~(\ref{e5-bos}) применительно к~множеству 
поездов~$\Im_s$, т.\,е.

\vspace*{-3pt}

\noindent
\begin{multline}
\hspace*{-12.5674pt}c_1\sum\limits_{i\in\Im_s} \sum\limits^{J+1}_{j=1} \sum\limits_{k=1}^K 
\delta_{i,j,k}\left( t_k^{\mathrm{кон}}- t_k^{\mathrm{нач}}\right) +c_2 
\sum\limits_{i\in\Im_s} \sum\limits^{J+1}_{j=1} \hat{T}_{i,j}+{}\\
{}+c_3\left(\sum\limits_{i\in\Im_s} \sum\limits_{k=1}^K \delta_{i,1,k} 
t_k^{\mathrm{нач}} -\sum\limits^I_{i=1} t_i^{\mathrm{отправ}}\right) \to \min
\label{e7-bos}
\end{multline}



\vspace*{-3pt}

\noindent
при дополнительном условии

\vspace*{-3pt}

\noindent
\begin{multline}
\delta_{i,j,k}=0\,,\enskip i\in \Im_s\,,\enskip j=\overline{1,J+1}\,,\\
k\in  \mathop{\bigcup}\limits_{p=0}^{s-1} \aleph_p\,.
\label{e8-bos}
\end{multline}

\vspace*{-3pt}

\noindent
Если решение этой задачи существует, то формируется множество номеров 
<<подниток>> $\aleph_s$, занимаемых поездами с~номерами из $\Im_s$, 
и~выполняется переход к~шагу~4. В~противном случае поиск приближенного 
решения завершен неуспешно.\\[-15pt]
\item [\,] 
\textbf{Шаг 4.}\ Если $s\hm=S$, то выполняется переход к~шагу~5. Если $s\hm< S$, 
то параметр~$s$ увеличивается на единицу и~выполняется переход к~шагу~3.\\[-15pt]
\item [\,] 
\textbf{Шаг 5.}\ Формируется множество $\overline{\Im}$ согласно построенному 
на шагах~1--4 алгоритма расписанию. Если множество $\overline{\Im}$ пусто, то 
можно назначить ТО любой длительности в~пределах горизонта планирования 
и~процесс поиска при\-бли\-жен\-но\-го решения завершен успешно. Если 
множество $\overline{\Im}$ непусто, то выполняется переход к~шагу~6.
\item [\,] 
\textbf{Шаг 6.}\ Решается задача~(\ref{e6-bos}). Пусть $t_1^*$ и~$t_2^*$~--- 
оптимальные значения переменных~$t_1$ и~$t_2$ в~этой задаче. Если значение 
критерия в~(\ref{e6-bos}) равно~0, то $t_1^*$ и~$t_2^*$ задают время для 
проведения ТО, расписание поездов перестраивать не нужно, процесс назначения 
ТО завершен успешно. Если значение критерия в~(\ref{e6-bos}) отлично от~0, то 
формируется множество 
$$
\aleph^\prime\hm= Z^\prime \cap \{k\in Z^\prime: 
t_k^{\mathrm{кон}}\hm\leq t_1^*\} \cap \{ k \in Z^\prime: 
t_k^{\mathrm{нач}}\hm\geq t_2^*\}\hspace*{-1.6819pt}
$$ 
и~выполняется переход к~шагу~3.
\item [\,] 
\textbf{Шаг 7.}\ Параметр~$s$ полагается равным~1, множества $\aleph_p$, 
$p\hm=\overline{0,S}$,~--- пустыми.

      
\item [\,] 
\textbf{Шаг 8.}\ Решается задача минимизации~(\ref{e7-bos}) при  
ограничениях~(\ref{e1-bos})--(3), (\ref{e5-bos}), (\ref{e8-bos}) и~дополнительном 
условии:
$$
\delta_{i,j,k}=0\,,\enskip i\in \Im_s\,,\enskip j=\overline{1,J+1}\,, \enskip k\in 
\aleph^\prime\,.
$$
Если решение этой задачи существует, то формируется множество номеров 
<<подниток>> $\aleph_s$, занимаемых поездами с~номерами из $\Im_s$, 
и~выполняется переход к~шагу~9. Если решения нет, то поиск приближенного 
решения завершен неуспешно.
\item [\,] 
\textbf{Шаг 9.}\ Если $s=S$, то $t_1^*$ и~$t_2^*$ задают время для ТО, поиск 
приближенного решения завершен успешно. Если $s\hm< S$, то параметр~$s$ 
увеличивается на единицу и~выполняется переход к~шагу~8.
\end{description}
  
  Шаг~1 алгоритма требует комментария. В~\cite{7-bos} предлагалось дробление 
множества поездов по направлению, т.\,е.\ множества $\Im_s$ формировались по 
принципу нахождения в~них номеров поездов, у~которых были одинаковые 
вершины отправления, а~также одинаковые вершины назначения. Чем меньше 
элементов в~множестве, тем меньше номер этого множества. При таком  
дроб\-ле\-нии будем называть приведенный алгоритм \textit{алгоритмом по 
направлению} (алгоритмом~1).
  
  Можно предложить другой способ дробления, а~именно: по возрастанию 
времени готовности поездов к~отправлению. Так, множество $\Im_1$ будет 
состоять из номера поезда с~самым ранним временем готовности к~отправлению, 
$\Im_2$~--- со вторым и~т.\,д. Таким образом, для данного подхода к~дроблению 
множества поездов окажется $S\hm=I$, а~$\Im_S$~--- со\-сто\-ящим из номера 
поезда с~самым поздним временем готовности к~отправлению. Если у~нескольких поездов 
одинаково время готовности к~отправлению, то для поезда с~б$\acute{\mbox{о}}$льшим номером 
индекс множества, в~который включается этот номер, также должен быть больше. 
При таком дроблении будем говорить об \textit{алгоритме по готовности} 
(алгоритме~2). По сути, в~алгоритме~2 предполагается искать расписание 
последовательно по одному поезду. Подобный подход использовался  
в~\cite{9-bos, 10-bos}, когда расписание движения поездов по станции 
предполагалось строить для каждого поезда отдельно, а расписание движения 
маневровых локомотивов строилось для каждой маневровой операции отдельно.
  
\section{Пример}

  Рассмотрим ж/д сеть, представимую в~виде мультиграфа~$G$ на рисунке. Часть 
ребер выделена пунктиром с~целью показать разноуровневое пересечение двух 
ж/д путей. Нумерация путей на рисунке опущена: если две соседние вершины 
соединяют два ребра, т.\,е.\ два пути, то ребро, представленной прямой линией, 
имеет номер~1, другое же~--- номер~2.
     

  
  Вершины отправления и~назначения поездов задаются теми же начальными 
данными, что использовались в~\cite{7-bos}, т.\,е.\ требуется пропустить $I\hm= 
62$ поезда, а~для выполнения перевозок имеется $K\hm=1249$ <<подниток>>. 
Детальную информацию о~на\-прав\-ле\-нии следования поездов, а~также 
о~направлениях движения по <<подниткам>> можно найти в~\cite{7-bos}. 
Положим $J\hm= 12$ и~$c_1\hm=1$, $c_2\hm=1$, $c_3\hm=1$, $t_1^0\hm=0$, 
$t_2^0\hm=1440$, $d_i\hm= 180$, $t_{i,k}^{\mathrm{ст.мин}}\hm=0$ 
и~$t_{i,k}^{\mathrm{ст.макс}}\hm=120$, $T_{\mathrm{макс}}\hm=1440$. 
Предположим, что необходимо провести ремонтные работы на пути номер~2 
между вершинами с~индексами~1~и~2.
  
  Проанализируем применимость и~качество алгоритмов~1 и~2, указав в~табл.~1:
  \begin{itemize}
\item значение критерия в~(\ref{e4-bos}) на расписании движения при отсутствии 
необходимости назначения ТО, т.\,е.\ значение критерия в~задаче~(\ref{e4-bos}) 
с~ограничениями~(\ref{e1-bos})--(3) и~(\ref{e5-bos}) при 
фиксированных~$\delta_{i,j,k}$, которые задаются на шагах~1--4;
\item промежуток времени для ТО, который получается на шаге~6;
\item число поездов, которые попадает в~ТО, вы\-чис\-лен\-ное на шаге~6;
\item значение критерия в~задаче~(\ref{e4-bos}) с~учетом необходимости 
назначения ТО, т.\,е.\ с~ограничениями~(\ref{e1-bos})--(3) и~(\ref{e5-bos}) при 
фиксированных значениях~$\delta_{i,j,k}$, которые задаются на шагах~7--9 и~при 
различных~$\Delta$.
\end{itemize}

\end{multicols}

\begin{figure*}
\vspace*{-6pt}
\begin{center}  
\mbox{%
\epsfxsize=132.314mm
\epsfbox{bos-1.eps}
}
\vspace*{9pt}

{\small Мультиграф ж/д сети~$G$}
\end{center}
\vspace*{-12pt}
      \end{figure*}
      
      

\begin{table*}\small %tabl1
\begin{center}
\Caption{Результаты работы алгоритмов~1 и~2}
\vspace*{2ex}

\tabcolsep=4pt
\begin{tabular}{|c|c|r|c|c|c|r|c|c|}
\hline
\raisebox{-18pt}[0pt][0pt]{$\Delta$}&\multicolumn{4}{c|}{Алгоритм 1}&\multicolumn{4}{c|}{Алгоритм 2}\\
\cline{2-9} 
&\tabcolsep=0pt\begin{tabular}{c}Значение\\ критерия в~(\ref{e4-bos})\\ без учета ТО\end{tabular}&
\multicolumn{1}{c|}{$[t_1^*, t_2^*]$}&\tabcolsep=0pt\begin{tabular}{c}Значение \\ критерия\\ в~(\ref{e6-bos})\end{tabular}&
\tabcolsep=0pt\begin{tabular}{c}Значение \\ критерия в~(\ref{e4-bos})\\ с~учетом ТО\end{tabular}&
\tabcolsep=0pt\begin{tabular}{c}Значение \\ критерия в~(\ref{e4-bos})\\ без учета ТО\end{tabular}&
\multicolumn{1}{c|}{$[t_1^*, t_2^*]$}&
\tabcolsep=0pt\begin{tabular}{c}Значение \\ критерия\\ в~(\ref{e6-bos})\end{tabular}&
\tabcolsep=0pt\begin{tabular}{c}Значение \\ критерия в~(\ref{e4-bos})\\ с~учетом ТО\end{tabular}\\
\hline
600&&[0,600]&0 &26951&&[0,600]&0 &27755\\
690&&[16,706]&1 &27287&&[16,706]&1 &27841 \\
720&26951&[56,776]&2 &27772&27755&[56,776]&2 &27911 \\
780&&[0,780]&3 &28266&&[0,780]&3 &Нет решения \\
900&&[0,900]&4 &Нет решения&&[0,900]&4 &Нет решения \\
\hline
\end{tabular}
\end{center}
%\vspace*{3pt}
\end{table*}

  \begin{table*}[b]\small %tabl2
\vspace*{-9pt}
\begin{center}
\Caption{Время работы алгоритмов~1 и~2 (в минутах)}
\vspace*{2ex}

\begin{tabular}{|c|c|c|c|c|c|c|c|c|}
\hline
\multicolumn{1}{|c|}{\raisebox{-18pt}[0pt][0pt]{$\Delta$}}&\multicolumn{4}{c|}{Алгоритм 1} &\multicolumn{4}{c|}{Алгоритм 2}\\
\cline{2-9} 
&\tabcolsep=0pt\begin{tabular}{c}Поиск\\ расписания\\ без учета ТО\end{tabular} &
\tabcolsep=0pt\begin{tabular}{c}Поиск ТО\end{tabular}&
\tabcolsep=0pt\begin{tabular}{c}Поиск\\ расписания\\ с~учетом ТО\end{tabular}&Итого &
\tabcolsep=0pt\begin{tabular}{c}Поиск\\ расписания\\ без учета ТО\end{tabular}&
\tabcolsep=0pt\begin{tabular}{c}Поиск\\ ТО\end{tabular} &
\tabcolsep=0pt\begin{tabular}{c}Поиск\\ расписания\\ с~учетом ТО\end{tabular}&Итого\\
\hline
600& &&0\hphantom{9}&19,07& &&0&\hphantom{9}8,07\\
690&&&14&33,07&&&\hphantom{,9}8,1&16,17\\
720&19&0,07&23&42,07&8 &0,07&8&16,07\\
780&&&\hphantom{,99}15,78&34,85&&&---&---\\
900&&&---&---&&&---&---\\
\hline
\end{tabular}
\end{center}
\end{table*}

\begin{multicols}{2}
  
  Как следует из табл.~1, в~терминах значения критерия алгоритм~1 дает лучшие 
результаты, чем алгоритм~2. 

Также алгоритм~1 позволяет найти решение в~случае 
$\Delta\hm=780$, когда алгоритм~2 решение не находит.
  
  Теперь проанализируем время поиска расписания движения и~ТО по 
алгоритмам~1 и~2 при различных~$\Delta$.

\vspace*{1pt}
\begin{table*}\small %tabl3
\begin{center}
\Caption{Маршруты движения поездов по алгоритму~1 без необходимости в~ТО}
\vspace*{2ex}

\begin{tabular}{|c|c|c|}
\hline
\tabcolsep=0pt\begin{tabular}{c}Направление\\ движения\end{tabular}&
\tabcolsep=0pt\begin{tabular}{c}Количество\\ поездов\end{tabular}&Маршрут движения\\
\hline
$\cdots$&$\cdots$&$\cdots$\\
\hline
\raisebox{-6pt}[0pt][0pt]{$10\to 2$}&12&$10\to 9\to8\to7\to6\to5\to4\to3\to1\to2$\\
&\hphantom{9}1&$10\to30\to29\to22\to21\to20\to17\to16\to15\to14\to13\to12\to2$\\
\hline
$\cdots$&$\cdots$&$\cdots$\\
\hline
\end{tabular}
\end{center}
%\end{table*}
%\begin{table*}\small %tabl4
%\vspace*{1pt}
\begin{center}
\Caption{Маршруты движения по алгоритму~1 в~случае необходимости ТО при 
$\Delta\hm=780$}
\vspace*{2ex}

\begin{tabular}{|c|c|c|}
\hline
\tabcolsep=0pt\begin{tabular}{c}Направление\\ движения\end{tabular}&
\tabcolsep=0pt\begin{tabular}{c}Количество\\ поездов\end{tabular}&Маршрут движения\\
\hline
$\cdots$&$\cdots$&$\cdots$\\
\hline
\raisebox{-18pt}[0pt][0pt]{$10\to2$}&10\hphantom{9}&$10\to9\to8\to7\to6\to5\to4\to3\to1\to2$\\
&1&$10\to30\to9\to8\to7\to6\to5\to4\to3\to1\to2$\\
&1&$10\to9\to8\to7\to6\to5\to4\to3\to1\to12\to2$\\
&1&$10\to9\to8\to7\to6\to5\to4\to3\to1\to23\to13\to12\to2$\\
\hline
$\cdots$&$\cdots$&$\cdots$\\
\hline
\end{tabular}
\end{center}
\vspace*{3pt}
\end{table*}

  
  
  Как следует из табл.~2, алгоритм~1 работает существенно дольше алгоритма~2, 
т.\,е.\ однозначно отдать предпочтение алгоритму~1 над алгоритмом~2 нельзя.
  
  Сравним теперь маршруты движения по алгоритму~1 в~случае отсутствия 
(табл.~3) и~при наличии (табл.~4) необходимости назначения ТО (при $\Delta\hm= 
780$).


  
  Детальная информация о маршрутах движения в~случае отсутствия 
необходимости назначения ТО имеется в~\cite{7-bos}.

  
  Маршруты движения по алгоритму~1 в~случае отсутствия и~при необходимости 
назначения ТО при $\Delta\hm=780$ совпадают за исключением движения между 
вершинами с~индексами~10 и~2. В~част\-ности, два поезда объезжают участок 
между вершинами с~индексами~1 и~2, где должны проводиться ремонтные 
работы.
  
  Все результаты были получены с~по\-мощью пакета ILOG CPLEX на 
персональном компьютере (Intel Core i5~4690; 3,5~ГГц; 8~ГБ DDR3 RAM).

\section{Заключение}

  В работе рассмотрена задача по назначению ТО на участках ж/д сети. 
Оптимизационная постановка сформулирована в~виде задачи смешанного 
целочисленного линейного программирования, которая
 позволяет одновременно 
найти расписание движения поездов по ж/д сети и~промежуток времени, 
в~который следует проводить ремонтные работы. Ввиду вычислительных 
сложностей при решении указанной задачи были предложены два алгоритма 
поиска приближенного решения. Представленный содержательный пример 
продемонстрировал практическую применимость предложенных алгоритмов.

\vspace*{-6pt}

{\small\frenchspacing
{%\baselineskip=10.8pt
%\addcontentsline{toc}{section}{References}
\begin{thebibliography}{99}

\vspace*{-2pt}

\bibitem{2-bos} %1
\Au{Albrecht A.\,R., Panton~D.\,M., Lee~D.\,H.} Rescheduling rail networks with maintenance 
disruptions using problem space search~// Comput. Oper. Res., 2013. Vol.~40. No.\,3. P.~703--712.
\bibitem{3-bos} %2
\Au{Forsgren M., Aronsson~M., Gestrelius~S.} Maintaining tracks and traffic flow at the same time~// 
J.~Rail Transport Planning  Management, 2013. Vol.~3. No.\,3. P.~111--123.
\bibitem{6-bos} %3
\Au{Ignatov A.\,N., Naumov~A.\,V.} On time selection for track possession assignment at the railway 
station~// Вестник Юж\-но-Ураль\-ско\-го государственного ун-та.
Сер. Математическое моделирование и~программирование, 2019. Т.~12. №\,3. С.~5--16.
\bibitem{1-bos} %4
\Au{Liden T.} Coordinating maintenance windows and train traffic: A~case study~// Public Transport, 
2020. Vol.~12. P.~261--298.

\bibitem{4-bos} %5
\Au{Зиндер Я., Лазарев~А.\,А., Мусатова~Е.\,Г.}
Корректировка расписания движения на частично заблокированном сегменте железной дороги с~разъездом~// 
 Автоматика и~телемеханика, 2020. %Т.~81. 
 №\,5. C.~91--105.
 doi: 10.31857/S0005231020050062.



\bibitem{5-bos} %6
\Au{Гайнанов Д.\,Н., Игнатов~А.\,Н., Наумов~А.\,В., Рассказова~В.\,А.}
О~задаче назначения <<технологического окна>> на участках железнодорожной сети~// 
 Автоматика и~телемеханика, 2020. %Vol.~81. 
№\,6. C.~3--16. doi: 10.31857/ S0005231020060013.


\bibitem{7-bos} %7
\Au{Ignatov A.\,N.} On the scheduling problem of cargo transportation on a railway network segment 
and algorithms for its solution~// Вестник Юж\-но-Ураль\-ско\-го государственного ун-та.
Сер. Математическое моделирование и~программирование, 2021. 
Т.~14. №\,3. С.~61--76.
\bibitem{8-bos} %8
\Au{Буянов М.\,В., Иванов~С.\,В., Кибзун~А.\,И., Наумов~А.\,В.} Развитие математической 
модели управления грузоперевозками на участке железнодорожной сети с~учетом случайных 
факторов~// Информатика и~её применения, 2017. Т.~11. Вып.~4. С.~85--93.
\bibitem{10-bos} %9
\Au{Босов А.\,В., Игнатов~А.\,Н., Наумов~А.\,В.} Модель передвижения поездов и~маневровых 
локомотивов на 
 железнодорожной станции в~приложении к~оценке и~анализу вероятности 
бокового столкновения~// Информатика и~её применения, 2018. Т.~12. Вып.~3. C.~107--114.
\bibitem{9-bos} %10
\Au{Игнатов~А.\,Н., Наумов~А.\,В.} О~задаче увеличения пропускной спо\-соб\-ности железнодорожной станции~// 
 Автоматика и~телемеханика, 2021. % Т.~82. 
 №\,1. С.~102--114. doi: 10.31857/S0005231021010074.


 \end{thebibliography}

}
}


\end{multicols}

\vspace*{-9pt}

\hfill{\small\textit{Поступила в~редакцию 28.07.2021}}

\vspace*{8pt}

%\pagebreak

%\newpage

%\vspace*{-28pt}

\hrule

\vspace*{2pt}

\hrule

\vspace*{-2pt}

\def\tit{ALGORITHMS FOR AN APPROXIMATE SOLUTION OF~THE~TRACK POSSESSION 
PROBLEM ON~THE RAILWAY NETWORK SEGMENT}


\def\titkol{Algorithms for an approximate solution of~the~track possession 
problem on~the railway network segment}

\def\aut{A.\,V.~Bosov$^{1,2}$, A.\,N.~Ignatov$^2$, and~A.\,V.~Naumov$^2$}

\def\autkol{A.\,V.~Bosov, A.\,N.~Ignatov, and~A.\,V.~Naumov}


\titel{\tit}{\aut}{\autkol}{\titkol}

\vspace*{-15pt}




\noindent
$^1$Federal Research Center ``Computer Science and Control'' of the Russian Academy of Sciences,  
44-2~Vavilov\linebreak
$\hphantom{^1}$Str., Moscow 119333, Russian Federation


\noindent
$^2$Moscow State Aviation Institute (National Research University), 4~Volokolamskoe Shosse, 
Moscow 125933,\linebreak
$\hphantom{^1}$Russian Federation

 
\def\leftfootline{\small{\textbf{\thepage}
\hfill INFORMATIKA I EE PRIMENENIYA~--- INFORMATICS AND
APPLICATIONS\ \ \ 2021\ \ \ volume~15\ \ \ issue\ 4}
}%
\def\rightfootline{\small{INFORMATIKA I EE PRIMENENIYA~---
INFORMATICS AND APPLICATIONS\ \ \ 2021\ \ \ volume~15\ \ \ issue\ 4
\hfill \textbf{\thepage}}}

\vspace*{3pt}




\Abste{Algorithmic and instrumental support for solving problems of railway transport control, based 
on the presentation of applied problems in the form of optimization statements in which linear  
programming tools are used, is being developed. Previously proposed models and applied statements 
are expanded with a~new problem of finding a track possession~--- a~time interval at which some sections 
of the railway network are closed for repair work. To solve it, a~mathematical model and an 
optimization statement are proposed for the simultaneous search for a~track possession and a~train 
schedule for a~certain segment of the railway network. The original setting is reduced to a~mixed 
integer linear programming problem. To take into account possible computational difficulties in 
solving the problem, a~method for finding an approximate solution is proposed which is based on the 
formation of a~basic schedule of movement and its subsequent correction taking into account the need 
for the track possession. To find an approximate solution, two algorithms have been implemented. In 
the first, a~basic and adjusted train timetable is built in stages by groups of trains united by the same 
departure and destination stations, and in the second, stages are carried out one train at a~time according to 
the time of readiness for departure. The results of a numerical experiment are presented.}

\KWE{multigraph; railway network; schedule; track possession; mixed integer linear programming}


\DOI{10.14357/19922264210401}

\vspace*{-18pt}

\Ack

\vspace*{-4pt}


\noindent
This work was partially supported by the Russian Foundation for Basic Research (grant  
20-07-00046~А).


\vspace*{12pt}

  \begin{multicols}{2}

\renewcommand{\bibname}{\protect\rmfamily References}
%\renewcommand{\bibname}{\large\protect\rm References}

{\small\frenchspacing
 {%\baselineskip=10.8pt
 \addcontentsline{toc}{section}{References}
 \begin{thebibliography}{99}
 
 %\vspace*{-2pt}
 
 \bibitem{2-bos-1} %1
\Aue{Albrecht, A.\,R., D.\,M.~Panton, and D.\,H.~Lee.} 2013. Rescheduling rail networks with 
maintenance disruptions using problem space search. \textit{Comput. Oper. Res.} 40(3):703--712.
\bibitem{3-bos-1} %2
\Aue{Forsgren, M., M.~Aronsson, and S.~Gestrelius.} 2013. Maintaining tracks and traffic flow at the 
same time. \textit{J.~Rail Transport Planning Management} 3(3):111--123.
\bibitem{6-bos-1} %3
\Aue{Ignatov, A.\,N., and A.\,V.~Naumov.} 2019. On time selection for track possession assignment 
at the railway station. \textit{Bull. South Ural State University. Ser. Math. Modelling Programming 
Computer Software} 12(3):5--16.

\bibitem{1-bos-1} %4
\Aue{Liden, T.} 2020. Coordinating maintenance windows and train traffic: A~case study. 
\textit{Public Transport} 12:261--298.

\bibitem{4-bos-1} %5
\Aue{Zinder, Y., A.\,A.~Lazarev, and E.\,G.~Musatova.} 2020. Rescheduling traffic on a partially 
blocked segment of railway with a siding. \textit{Automat. Rem. Contr.} 81(6):955--966.
\bibitem{5-bos-1} %6
\Aue{Gainanov, D.\,N., A.\,N.~Ignatov, A.\,V.~Naumov, and V.\,A.~Rasskazova.} 2020. On track 
procession assignment problem at the railway network sections. \textit{Automat. Rem. Contr.} 
81(6):967--977.

\bibitem{7-bos-1}
\Aue{Ignatov, A.\,N.} 2021. On the scheduling problem of cargo transportation on a railway 
network segment and algorithms for its solution. \textit{Bull. South Ural State University. Ser. Math. 
Modelling Programming Computer Software} 14(3):61--76.
\bibitem{8-bos-1}
\Aue{Buyanov, M.\,V., S.\,V.~Ivanov, A.\,I.~Kibzun, and A.\,V.~Naumov.} 2017. Razvitie 
matematicheskoy modeli upravleniya gruzoperevozkami na uchastke zheleznodorozhnoy seti 
s~uchetom sluchaynykh faktorov [Development of the mathematical model of cargo transportation 
control on a~railway network segment taking into account random factor]. \textit{Informatika i~ee 
Primeneniya~--- Inform. \mbox{Appl.}} 11(4):\linebreak 85--93.

\bibitem{10-bos-1}
\Aue{Bosov, A.\,V., A.\,N.~Ignatov, and A.\,V.~Naumov.} 2018. Mo\-del' peredvizheniya po\-ez\-dov 
i~manevrovykh lo\-ko\-mo\-ti\-vov na zheleznodorozhnoy stantsii v~pri\-lo\-zhe\-nii k~otsen\-ke i~ana\-li\-zu 
veroyatnosti bokovogo stolknoveniya [Transportation of trains and shunting locomotives at the railway
station model for evaluating and analysis of side-collision probabilities]. \textit{Informatika i~ee 
Primeneniya~--- Inform. \mbox{Appl.}} 12(3):107--114.

\bibitem{9-bos-1}
\Aue{Ignatov, A.\,N., and A.\,V.~Naumov.} 2021. On the problem of increasing the railway station 
capacity. \textit{Automat. Rem. Contr.} 82(1):102--114.
\end{thebibliography}

 }
 }

\end{multicols}

\vspace*{-3pt}

  \hfill{\small\textit{Received July~28, 2021}}


%\pagebreak

%\vspace*{-12pt}  

\Contr

\noindent
\textbf{Bosov Alexey V.} (b.\ 1969)~--- Doctor of Science in technology, principal scientist, Institute 
of Informatics Problems, Federal Research Center ``Computer Science and Control'' of the Russian 
Academy of Sciences, 44-2~Vavilov Str., Moscow 119333, Russian Federation; professor, Moscow 
State Aviation Institute (National Research University), 4~Volokolamskoe Shosse, Moscow 125933, 
Russian Federation; \mbox{AVBosov@ipiran.ru}

\vspace*{6pt}

\noindent
\textbf{Naumov Andrey V.} (b.\ 1966)~--- Doctor of Science in physics and mathematics, professor, 
Moscow Aviation Institute (National Research University), 4~Volokolamskoe Shosse, Moscow 
125933, Russian Federation; \mbox{naumovav@mail.ru}

\vspace*{6pt}

\noindent
\textbf{Ignatov Aleksei N.} (b.\ 1991)~--- Candidate of Science in physics and mathematics, Moscow 
Aviation Institute (National Research University), 4~Volokolamskoe Shosse, Moscow 125933, Russian 
Federation; \mbox{alexei.ignatov1@gmail.com}

\label{end\stat}

\renewcommand{\bibname}{\protect\rm Литература}