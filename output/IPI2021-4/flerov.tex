\def\stat{flerov}

\def\tit{ТЕОРЕТИЧЕСКИЕ ОСНОВЫ ФОРМИРОВАНИЯ ВЕСОВОГО~ОБЛИКА САМОЛЕТА}

\def\titkol{Теоретические основы формирования весового облика самолета}

\def\aut{Л.\,Л.~Вышинский$^1$, Ю.\,А.~Флёров$^2$}

\def\autkol{Л.\,Л.~Вышинский, Ю.\,А.~Флёров}

\titel{\tit}{\aut}{\autkol}{\titkol}

\index{Вышинский Л.\,Л.}
\index{Флёров Ю.\,А.}
\index{Vyshinsky L.\,L.}
\index{Flerov Yu.\,A.}


%{\renewcommand{\thefootnote}{\fnsymbol{footnote}} \footnotetext[1]
%{Работа выполнена при поддержке Министерства науки и~высшего образования Российской Федерации (проект 
%075-15-2020-799).}}


\renewcommand{\thefootnote}{\arabic{footnote}}
\footnotetext[1]{Федеральный исследовательский центр <<Информатика и~управление>> Российской академии наук, 
\mbox{wyshinsky@mail.ru}}
\footnotetext[2]{Федеральный исследовательский центр <<Информатика и~управление>> Российской 
академии наук, \mbox{fler@ccas.ru}}

%\vspace*{-12pt}

  \Abst{Статья посвящена вопросам автоматизации задач, связанных с~формированием 
весового облика на начальном этапе проектирования самолетов. На этом этапе создаются 
основные структуры информационной весовой модели изделия, которая детализируется, 
совершенствуется и~используется в~течение всего жизненного цикла, в~том числе на этапах 
производства и~эксплуатации. Описан разработанный авторами программный продукт, 
который является инструментом, предназначенным для использования при проектировании 
самолетов разного типа и~назначения.}
  
  \KW{математическое моделирование; автоматизация проектирования; самолет; 
формирование облика; весовое проектирование; весовая модель; дерево конструкции; 
генератор проектов}

\DOI{10.14357/19922264210413}
  
%\vspace*{9pt}

\vskip 10pt plus 9pt minus 6pt

\thispagestyle{headings}

\begin{multicols}{2}

\label{st\stat}

\section{Задача формирования весового облика самолета}

  Фундаментальная научная проблема, которая встает в~самом начале создания 
сложных технических систем, состоит в~решении двух основных 
концептуальных задач: задачи параметрического синтеза математической 
модели будущей системы (генерация множества альтернатив системы и~выбор 
эффективных вариантов) и~задачи анализа конкретных альтернатив системы 
(определение функциональных характеристик данной альтернативы системы 
и~выяснение их соответствия целевому назначению). Решение этой двуединой 
задачи на начальной стадии проектирования летательных аппаратов (ЛА) 
принято называть формированием облика~[1]. 
  
  Под конструктивными параметрами ЛА будем понимать те параметры, 
совокупность которых определяет его компоновочную схему, а~также состав 
и~взаимосвязи основных агрегатов ЛА. Пусть $x\hm= (x_1, x_2, \ldots , x_n)\hm\in 
X$, где $x$~--- вектор конструктивных параметров и~характеристик ЛА; 
$X$~--- множество допустимых вариантов компоновок ЛА. Формирование 
облика ЛА есть задача определения вектора~$x$, обеспечивающего достижение 
поставленных целей. На практике цели проектирования формулируются как 
оптимизация значений\linebreak некоторого набора частных критериев 
(лет\-но-тех\-ни\-че\-ские характеристики, характеристики двигателя, крыла, фюзеляжа, 
весовые показатели, оценки стоимости и~другие величины). Пусть\linebreak $F(x)\hm= 
(F_1(x), F_2(x), \ldots , F_m(x))$~--- вектор частных критериев эффективности, 
размерность которого много меньше размерности пространства 
конструктивных параметров. Можно считать, что все частные критерии 
положительны и~их значения нужно увеличивать. Решением задачи 
многокритериальной оптимизации $y\hm= \max_x F(x)$ является паретовская 
(недоминируемая) граница множества $F(x)$ при допустимых значениях 
$x\hm\in X$. 
  
  Одним из важнейших частных критериев при проектировании самолетов 
любого типа служит его весовая эффективность~--- отношение массы полезной 
нагрузки (целевая нагрузка плюс масса топлива) к~взлетной массе самолета. 
Эти два весовых параметра связаны между собой фундаментальным 
соотношением весового баланса, которое имеет следующий вид: 
\begin{multline*}
  M_{\mathrm{ВЗЛ}} = M_{\mathrm{ПЛАН}} + M_{\mathrm{СУ}} + M_{\mathrm{ОБ}} + 
M_{\mathrm{БРЭО}} + {}\\
{}+M_{\mathrm{ЭК\_СН}} + M_{\mathrm{ЦЕЛ\_НАГР}} + 
M_{\mathrm{ТОП}},
\end{multline*}
  где $ M_{\mathrm{ВЗЛ}}$~--- расчетная взлетная масса \mbox{самолета};
  $M_{\mathrm{ПЛАН}}$~--- масса планера самолета, его несущей конструкции;
  $M_{\mathrm{СУ}}$~--- масса силовой установки, обеспечивающей 
перемещение ЛА;
  $M_{\mathrm{ОБ}}$~--- масса бортового оборудования, обеспечивающего 
управление ЛА;
  $M_{\mathrm{БРЭО}}$~--- масса бортового оборудования различного целевого 
назначения;
  $M_{\mathrm{ЭК\_СН}}$~--- масса экипажа и~полетного снаряжения;
  $M_{\mathrm{ЦЕЛ\_НАГР}}$~--- масса целевой, или коммерческой, нагрузки;
  $M_{\mathrm{ТОП}}$~--- расчетная стартовая масса топлива.
  
  Одна из центральных задач начального этапа проектирования самолетов~--- 
оценка значений перечисленных величин. Такую задачу будем называть 
формированием весового облика самолета.
  
  Кроме основных весовых параметров в~состав весового облика включают 
и~ряд производных величин, которые характеризуют конструктивные, 
эксплуатационные, эконометрические и~другие свойства самолета. К~наиболее 
часто используемым производным весовым параметрам весового облика ЛА 
относятся:
  \begin{description}
  \item[\,]
  $M_{\mathrm{ПУСТ}} = M_{\mathrm{ПЛАН}} \hm+ M_{\mathrm{СУ}}\hm + 
M_{\mathrm{ОБ}}\hm + M_{\mathrm{БРЭО}}$~--- масса пустого самолета;
  \item[\,]
  $M_{\mathrm{СНАР}} = M_{\mathrm{ПУСТ}} \hm+ M_{\mathrm{ЭК\_СН}}$~--- масса 
снаряженного самолета; 
  \item[\,]
  $\overline{m}_{\mathrm{ПЛАН}} = M_{\mathrm{ПЛАН}}/ M_{\mathrm{ВЗЛ}}$~--- 
относительная масса планера; 
  \item[\,]
  $\overline{m}_{\mathrm{СУ}} = M_{\mathrm{СУ}}/ M_{\mathrm{ВЗЛ}}$~--- 
относительная масса силовой установки;
  \item[\,]
  $\overline{m}_{\mathrm{ОБ}}= M_{\mathrm{ОБ}}/ M_{\mathrm{ВЗЛ}}$~--- 
относительная масса самолетного оборудования;
  \item[\,]
  $\overline{m}_{\mathrm{ТОП}} = M_{\mathrm{ТОП}}/ M_{\mathrm{ВЗЛ}}$~--- 
относительная масса топлива;
  \item[\,]
   $F_M=\overline{m}_{\mathrm{ПОЛ\_НАГР}} = (M_{\mathrm{ЦЕЛ\_НАГР}} \hm+ M_{\mathrm{ТОП}})/ 
M_{\mathrm{ВЗЛ}}$~--- весовая эффективность ЛА.
  \end{description}
  
  Формирование весового облика самолета начинается в~условиях 
значительной неопределенности, когда заданы лишь общие требования 
к~проектируемому изделию. Основное требование состоит в~транспортировке 
целевой нагрузки, которое выражается в~задании расчетной и~максимальной 
массы нагрузки~--- $M_{\mathrm{ЦЕЛ\_НАГР}}$. Перечень задач, которые при этом 
должен решать проектируемый ЛА, задается дальностью крейсерского полета 
и~другими лет\-но-тех\-ни\-че\-ски\-ми и~взлет\-но-по\-са\-доч\-ны\-ми характеристиками. Ряд 
задач может потребовать размещения радиоэлектронного оборудования, 
которое задается своей массой~--- $ M_{\mathrm{БРЭО}}$. Заметим, что 
формирование весового облика~--- это только одна из задач в~ряду расчетов 
геометрических, аэродинамических, лет\-но-тех\-ни\-че\-ских 
и~взлет\-но-по\-са\-доч\-ных характеристик самолета. Приближенные оценки 
массы на начальных стадиях проектирования могут быть получены исходя из 
предыдущего опыта проектирования. Это так называемое проектирование <<от 
прототипов>>. Метод проектирования <<от прототипа>> наиболее эффективен, 
когда идет речь о~модификации и~даже глубокой модификации серийных 
изделий. Анализ тенденций развития отдельных аспектов проектирования 
позволяет выявить определенные закономерности и~связи между различными 
конструктивными параметрами самолетов и~их характеристиками. В~весовом 
проектировании ЛА говорят об анализе весовой <<статистики>>,\linebreak имея в~виду 
методы регрессионного анализа, который осуществляется на выборках 
некоторого множества прототипов. При разумном подходе сравнительный 
анализ, например применение \mbox{традиционного} критерия среднеквадратичного 
отклонения по представительной выборке прототипов, вполне может быть 
полезен для оценки взлетной массы, массы пустого ЛА, массы основных его 
агрегатов~--- крыла, фюзеляжа, оперения, самолетного оборудования и~пр. Для 
реализации такого подхода нужна целенаправленная работа по сбору 
и~хранению информации о~существующих тенденциях и~о готовых изделиях. 
  
  Вторым традиционным методом оценки весовых параметров ЛА и~его 
агрегатов на этапе формирования облика стало применение так называемых 
<<весовых формул>>. Весовые формулы~--- это арифметические выражения, 
зависящие от небольшого числа конструктивных параметров компоновочной 
схемы ЛА, которые, по идее, должны отражать их влияние на значение массы 
изделия и~его агрегатов. Эти эмпирически выведенные зависимости 
верифицируются на выборках прототипов и~в ограниченных диапазонах 
значений своих параметров могут давать неплохие результаты. Для реализации 
этого подхода необходимо иметь возможность построения, модификации, 
параметрического анализа и~верификации формул.
  
  Как видно, весовые расчеты на начальном этапе проектирования ЛА 
существенным образом опираются на инженерный опыт предыдущих 
разработок. Поэтому важной и~необходимой задачей автоматизации весового 
проектирования, особенно на начальных этапах, становится накопление 
весовых данных о проектируемых и~готовых изделиях в~специально 
организованной базе данных (БД) потенциальных прототипов ЛА.
  
  В настоящей статье описывается программный модуль формирования 
весового облика самолета, разработанный в~рамках создания системы 
автоматизации весового проектирования ЛА~[2]. 
В~качестве первичной задачи в~программе решается задача создания БД 
весовых моделей самолетов разных типов. 
  
\begin{figure*} %fig1
\vspace*{1pt}
  \begin{center}  
    \mbox{%
\epsfxsize=163mm
\epsfbox{fle-1.eps}
}

\end{center}
\vspace*{-6pt}
\Caption{Реестр весовых моделей ЛА}
\vspace*{3pt}
\end{figure*}

\section{База данных весовых моделей самолета}

  База данных весовых моделей ЛА~--- это основное 
хранилище весовой информации по созда\-ва\-емым проектам ЛА. В~то же время 
такая БД может служить архивом весовых моделей существующих серийных 
самолетов, которые могут быть использованы в~качестве прототипов в~новых 
проектах. Внесение весовой информации о~самолетах разных типов, разных 
производителей, разных лет выпуска в~БД весовых моделей позволит 
создать широкий спектр полезной информации для проектирования новых 
изделий. В~целях систематизации архивной информации в~БД весовых моделей 
используется классификатор типов самолетов по их назначению. 
Классификатор самолетов, как и~другая нормативная информация, создается 
и~модифицируется в~системе весового проектирования, в~рамках которой 
разработан модуль формирования весового облика. Основной структурой 
систематизации и~упорядочения весовых моделей служит реестр, содержащий 
идентификационные атрибуты ЛА, весовые модели 
которых создаются в~БД. На рис.~1 приведена экранная форма, 
в~которой выводится реестр весовых моделей ЛА. 

  Каждой записи в~реестре соответствует информационная структура весовой 
модели изделия. Структура и~состав параметров весовых моделей существенно 
зависят от типа и~назначения ЛА. Синтез и~анализ весовых моделей являются 
основными задачами разработанного модуля. 
  
  Структурно-параметрическая весовая модель самолета наряду с~электронным 
геометрическим макетом считается важнейшим информационным объектом, 
который создается в~процессе проектирования, развивается, модифицируется 
и~используется в~процессе всего жизненного цикла изделия вплоть до серийного 
производства, сдачи в~эксплуатацию, длительного периода эксплуатации, 
списания и~утилизации. Задача создания весовой модели ЛА состоит 
в~описании множества параметров ЛА и~области допустимых конструкций 
в~пространстве этих параметров. Весовая модель ЛА представляет собой  
струк\-тур\-но-па\-ра\-мет\-ри\-че\-скую модель следующего вида:

\noindent
  \begin{align*}
  X&= \left\{ x_1, x_2,\ldots , x_n, \right.\\[-1pt]
  &\left. \mbox{где для\ всех } i\enskip x_i\in X_i\vert P(x_1, \ldots,
x_n)\right\}\,;\\[-1pt]
  X_i&= \left\{ x_{i1}, x_{i2}, \ldots , x_{in_i},\right.\\[-1pt]
  & \left. \mbox{где для\ всех } j\enskip x_{ij}\in 
X_{ij} \vert P_i(x_1,  \ldots , x_n)\right\}\,;\\[-1pt]
&\ldots
  \end{align*}
  
  \noindent
  Здесь $\{x_i\}$~--- конструктивные параметры и~характеристики (векторы 
параметров) ЛА, а множества $X_i, X_{ij},\ldots$ представляют собой модели 
агрегатов самолета и~образуют иерархию, которую принято называть деревом 
конструкции. Предикаты $\{P,P_i,\ldots \}$ определяют множество допустимых 
конструкций в~пространстве конструктивных пара\-мет\-ров и~характеристик ЛА. 

\vspace*{-2pt}

\section{Пространство конструктивных параметров и~характеристик летательного аппарата}

\vspace*{-1pt}

  Как было сказано выше, основой весового облика ЛА служит набор 
абсолютных, относительных и~безразмерных параметров массы основных 
структурных компонентов самолета~--- взлетная масса, масса планера, масса 
силовой установки, масса самолетного и~специального оборудования, масса 
целевой нагрузки и~топлива. Область существования ЛА в~пространстве 
эксплуатационных характеристик задается в~так\-ти\-ко-тех\-ни\-че\-ских 
требованиях к~ЛА~--- это взлетная и~посадочная дистанция; диапазон 
скоростей на разных высотах; практический потолок; дальность полета 
в~крейсерском режиме, или перегоночная дальность; максимально до\-пус\-ти\-мая 
аэродинамическая перегрузка и~другие параметры. Набор таких параметров 
зависит от типа ЛА. Об\-ласть существования вместе с~требуемой максимальной 
перевозимой нагрузкой фактически определяют весовую категорию самолета. 
  
  Второй группой параметров, необходимых при весовых расчетах, выступают 
конструктивные параметры компоновочной схемы. К~таким параметрам 
относятся габаритные размеры ЛА, площадь и~геометрия несущих 
поверхностей, площадь омываемой поверхности, располагаемые внутренние 
объемы и~ряд других параметров. Параметры агрегатов ЛА, положение 
агрегатов относительно друг друга и~другие геометрические характеристики~--- 
это тоже параметры компоновочной схемы.\linebreak Состав основных агрегатов 
планера, силовой установки и~оборудования задается деревом конструкции, 
каж\-дый узел которого соответствует определенному агрегату изделия. Узлы 
дерева \mbox{конструкции} характеризуются своими па\-ра\-мет\-ра\-ми и~рядом 
соотношений между ними. Таким образом, дерево кон\-ст\-рук\-ции пред\-став\-ля\-ет 
собой иерархию связанных между собой моделей агрегатов ЛА. Отношения 
между агрегатами, например отношения вло\-жен\-ности, взаимное расположение, 
аэродинамическое влияние друг на друга и~пр., также могут быть выражены 
через дополнительные па\-ра\-мет\-ры и~соотношения. В~качестве таких пара\-мет\-ров 
используются: стартовая тяговооруженность, удельная нагрузка на крыло, 
аэродинамические коэффициенты лобового сопротивления и~подъемной силы, 
аэродинамическое качество и~другие принятые в~авиастроении величины. 
  
  В рамках разработанного модуля в~весовую модель можно вводить любые 
параметры и~соотношения, способствующие вычислению значений основных 
базовых параметров весового облика. Это в~определенной степени расширяет 
возможности проектировщиков для проведения исследований и~для анализа 
различных вариантов компоновочных схем и~допустимых конструкций. При 
создании новых параметров весовой модели вводятся следующие данные:
  \begin{itemize}
  \item уникальный в~рамках весовой модели идентификатор параметра;
  \item наименование параметра;
  \item тип параметра (число, логический или пере\-чис\-ли\-мый тип);
  \item текущее значение параметра;
  \item формула параметра (арифметическое или логическое выражение);
  \item минимально допустимое значение параметра;
  \item максимально допустимое значение параметра;
  \item  статус параметра в~весовой модели~--- конструктивный параметр, 
функция;
  \item текущее состояние параметра~--- определен, не определен, выходит за 
ограничения допустимого диапазона. 
  \end{itemize}
  
  Операндами арифметических или логических выражений для вычисления 
параметров могут быть числовые или логические константы, идентификаторы 
любых параметров весовой модели, встроенные функции и~функции, 
описанные в~весовой модели.
  
  Полный список параметров весовой модели включает параметры верхнего 
уровня и~па\-ра\-мет\-ры моделей конструктивных компонентов ЛА. Компоненты 
ЛА по своей роли в~весовой модели разделяют\-ся на постоянную конструкцию 
и~переменные составляющие взлетной конфигурации ЛА~--- снаряжение, 
целевая нагрузка и~топливо. Переменные компоненты рассматриваются 
в~весовой модели как размещаемые на борту готовые изделия, весовые 
характеристики которых известны. Постоянные компоненты ЛА объединяются 
в~единую конструкцию, которую называют пустым ЛА. Понятие дерева 
конструкции относится к~пустому ЛА. На рис.~2 приведен пример списка 
параметров верхнего уровня и~дерево конструкции пустого ЛА.
  
\begin{figure*} %fig2
\vspace*{1pt}
  \begin{center}  
    \mbox{%
\epsfxsize=163mm
\epsfbox{fle-2.eps}
}

\end{center}
\vspace*{-6pt}
\Caption{Список параметров верхнего уровня и~дерево конструкции ЛА}
\end{figure*}

   Каждому узлу дерева конструкции соответствует определенный агрегат ЛА, 
сборочная единица или деталь. При формировании облика до деталей дело не 
доходит и~узлам дерева конструкции соответствуют модели основных агрегатов 
и~сис\-тем ЛА, которые в~процессе проектирования будут уточняться 
и~детализироваться. 
  
  Каждый узел дерева конструкции весовой модели имеет фиксированный 
набор атрибутов:
  \begin{itemize}
\item ID~--- идентификатор узла;
\item название агрегата (совпадает с~названием соответствующего раздела 
классификатора);
\item $M$~--- текущая масса агрегата, точнее, текущая оценка массы агрегата, 
которая может использоваться при расчетах характеристик ЛА; 
\item $M_{\mathrm{ТЕОР}}$~--- теоретическая масса~--- целевая характеристика 
массы агрегата на этапе формирования весового облика; 
\item $X_{\mathrm{ЦМ}}, Y_{\mathrm{ЦМ}}$ и~$Z_{\mathrm{ЦМ}}$~--- координаты 
центра масс агрегата в~самолетной системе координат.
  \end{itemize}
  
  Каждому узлу дерева конструкций кроме пе\-ре\-чис\-ленных параметров может 
быть приписано произвольное множество дополнительных па\-ра\-мет\-ров, 
описывающих специфику со\-от\-вет\-ст\-ву\-юще\-го агрегата. 
  
  Таким образом, полное пространство па\-ра\-мет\-ров весовой модели ЛА со\-сто\-ит 
из па\-ра\-мет\-ров верх\-не\-го уровня и~параметров дерева конструкции пус\-то\-го ЛА. 
Основная функция модуля формирования весового облика~--- это вычисление 
всех введенных на этом этапе па\-ра\-мет\-ров и~формирование начального 
со\-сто\-яния дерева конструкции. На по\-сле\-ду\-ющих этапах проектирования со\-став
 и~значения па\-ра\-мет\-ров, а~так\-же дерево конструкции будут уточняться 
и~детализироваться. 

\vspace*{-3pt}
  
\section{Весовые расчеты в~модуле формирования весового 
облика}

  Весовая модель ЛА~--- основной инструмент 
формирования весового облика, в~процессе которого строятся при\-бли\-жен\-ные 
оценки массы ЛА и~его основных агрегатов. Модуль формирования весового 
облика~--- пред\-мет\-но-ори\-ен\-ти\-ро\-ван\-ный калькулятор, который, 
в~отличие, например, от такого популярного вы\-чис\-ли\-тель\-но\-го инструмента, 
как Excel, позволяет сис\-те\-ма\-ти\-зи\-ро\-вать и~структурировать информацию при 
работе с~компоновочной схемой ЛА, с~деревом его конструкции, с~моделями 
агрегатов ЛА, с~БД прототипов про\-ек\-ти\-ру\-емых изделий. Здесь нет 
однозначного алгоритма решения некоторой заранее сформулированной 
задачи. Такую задачу заранее сформулировать невозможно, поскольку 
и~конструкция ЛА, и~со\-став па\-ра\-мет\-ров, и~соотношения между ними могут 
сильно зависеть от конкретных требований и~условий проектирования. 
Основная функция программы со\-сто\-ит в~том, чтобы предоста\-вить инженеру 
возможности для по\-ста\-нов\-ки различных вы\-чис\-ли\-тель\-ных задач, связанных 
с~оценкой массы ЛА и~его агрегатов на начальных этапах проектирования 
самолетов. Важ\-ную роль в~любых инженерных вы\-чис\-ле\-ни\-ях играет 
воз\-мож\-ность проведения па\-ра\-мет\-ри\-че\-ских расчетов. Приведем в~качестве 
примера проведение па\-ра\-мет\-ри\-че\-ских расчетов при по\-стро\-ении первого 
при\-бли\-же\-ния взлетной массы самолета. Для этого будет использовано базовое 
соотношение весовой модели ЛА~--- уравнение весового баланса. 
  
  Перепишем уравнение весового баланса в~сле\-ду\-ющем виде:
$$
  M_{\mathrm{ВЗЛ}} = \fr{M_{\mathrm{ЦЕЛ\_НАГР}} + M_{\mathrm{ЭК\_СН}} + 
M_{\mathrm{БРЭО}}}{1 - \overline{m}_{\mathrm{ПЛАН}} - \overline{m}_{\mathrm{СУ}} - 
\overline{m}_{\mathrm{ОБ}} -\overline{m}_{\mathrm{ТОП}}}.
$$

\vspace*{-3pt}

\pagebreak
  
  Значение $ M_{\mathrm{ЦЕЛ\_НАГР}}$ устанавливается в~основных требованиях 
  к~проектируемому самолету. По массе целевой нагрузки, назначению 
и~перечню основных задач с~достаточной точ\-ностью для начального 
при\-бли\-же\-ния можно определить массу со\-ста\-ва экипажа, массу базового 
и~специального снаряжения $M_{\mathrm{ЭК\_СН}}$, а~также со\-став и~массу 
стандартного и~специального радиоэлектронного обору\-до\-ва\-ния 
$M_{\mathrm{БРЭО}}$. Значения относительных параметров 
$\overline{m}_{\mathrm{ПЛАН}}$, $\overline{m}_{\mathrm{СУ}}$, 
$\overline{m}_{\mathrm{ОБ}}$ и~$\overline{m}_{\mathrm{ТОП}}$ достаточно 
консервативны и~меняются в~небольших пределах, которые зависят от типа 
ЛА и~его конструкции. В~учебнике по проектированию 
самолетов~[3] приведена таб\-ли\-ца диапазонов значений па\-ра\-мет\-ров для 
самолетов различного на\-зна\-че\-ния и~различных весовых категорий. Эта 
информация~--- результат обработки данных по большому чис\-лу самолетов. 
Воспользуемся в~качестве примера данными из таб\-ли\-цы для легких 
истребителей при по\-стро\-ении начального при\-бли\-же\-ния взлетной массы, 
которое обозначим~$M_0$. Средствами модуля формирования весового облика 
мож\-но провести па\-ра\-мет\-ри\-че\-ские исследования за\-ви\-си\-мости $M_0$ от 
$M_{\mathrm{ЦЕЛ\_НАГР}}$, $M_{\mathrm{ЭК\_СН}}$, $M_{\mathrm{БРЭО}}$, а~также от 
безразмерных па\-ра\-мет\-ров в~заданных диапазонах $0{,}25\hm\leq 
\overline{m}_{\mathrm{ПЛАН}}\hm\leq 0{,}32$;\linebreak
 $0{,}18\hm\leq 
\overline{m}_{\mathrm{СУ}}\hm\leq 0{,}22$; $0{,}12\hm\leq 
\overline{m}_{\mathrm{ОБ}}\hm\leq 0{,}14$ и~$0{,}25\hm\leq 
\overline{m}_{\mathrm{ТОП}}\hm\leq 0{,}30$. Приведенные диапазоны относительных 
па\-ра\-мет\-ров, взятые из учеб\-ни\-ка, до\-ста\-точ\-но услов\-ны и~могут быть уточ\-не\-ны по 
БД \mbox{прототипов}. На рис.~3 приведены результаты параметрических расчетов, 
где $M_0$ пред\-став\-ле\-на в~виде функции двух па\-ра\-мет\-ров~--- сверт\-ки величин 
$\overline{m}_{\mathrm{ПЛАН}}$, $\overline{m}_{\mathrm{СУ}}$, 
$\overline{m}_{\mathrm{ОБ}}$ и~$\overline{m}_{\mathrm{ТОП}}$ и~массы целевой 
нагрузки $M_{\mathrm{ЦЕЛ\_НАГР}}$ при фиксированных $M_{\mathrm{ЭК\_СН}}$ 
и~$M_{\mathrm{БРЭО}}$.

{ \begin{center}  %fig3
 \vspace*{14pt}
    \mbox{%
\epsfxsize=76.302mm
\epsfbox{fle-3.eps}
}

\end{center}

\vspace*{3pt}

\noindent
{{\figurename~3}\ \ \small{
Параметрические расчеты начального приближения массы ЛА: \textit{1}~--- $M_{\mathrm{НАГР}}\hm=
0{,}4$~т; \textit{2}~--- 0,5; \textit{3}~--- 0,6; \textit{4}~--- 0,7;
\textit{5}~--- 0,8; \textit{6}~---  0,9; \textit{7}~--- 1,0; \textit{8}~--- 1,1;
\textit{9}~--- 1,2; \textit{10}~--- 1,3; \textit{11}~--- $M_{\mathrm{НАГР}}\hm= 1{,}4$~т
}}}

%\vspace*{6pt}



  При задании некоторого значения сверт\-ки относительных па\-ра\-мет\-ров массы 
(вертикальная линия при значении $\sim 0{,}857$) мож\-но получить 
при\-бли\-жен\-ные оценки взлетной массы при различных вариантах целевой 
нагрузки.
  
  Приведенный пример параметрических расчетов учитывает лишь один 
важный фактор, от которого зависит взлетная масса самолета,~--- целевую 
нагрузку. Но подобные расчеты могут дать лишь некоторое пред\-став\-ле\-ние 
о~той весовой категории, в~которой реально может существовать 
про\-ек\-ти\-ру\-емый самолет. Имеется еще ряд па\-ра\-мет\-ров, от которых во многом 
зависит весовой облик самолета. Такими параметрами могут быть требуемая 
дальность полета, максимальная достижимая ско\-рость, практический потолок, 
дистанция разбега при взлете и~другие величины. Для анализа влияния этих 
параметров на параметры массы самолета, вообще говоря, требуются 
аэродинамические расчеты, расчеты лет\-но-тех\-ни\-че\-ских  
и~взлет\-но-по\-са\-доч\-ных характеристик. На начальных этапах 
проектирования возможно применение упро\-щен\-ных моделей таких расчетов, 
применение эмпирических формул, которые могут быть получены в~результате 
анализа некоторой выборки прототипов, близких к~проектируемому самолету 
по типу и~решаемым задачам. Существенно больше возможностей для анализа 
параметров весового облика появляется у~проектировщика в~процессе 
построения компоновочной схемы самолета, когда можно использовать 
конструктивные параметры основных агрегатов самолета, от которых зависят 
весовые характеристики. Основным инструментом на этапе формирования 
облика служит построение весовых формул. Весовые формулы, как правило, 
представляют собой функции конструктивных параметров, которые содержат 
множество эмпирических коэффициентов. Они выводятся из общих 
физических соображений и~отражают влияние тех или иных параметров на вес 
конструкции ЛА. Естественно, такие эмпирические зависимости нуждаются 
в~верификации. Множество прототипов, на которых ведется проверка формул, 
должно соответствовать задачам про\-ек\-ти\-ру\-емо\-го самолета. 

\setcounter{figure}{3}
    \begin{figure*} %fig4
    \vspace*{1pt}
  \begin{center}  
    \mbox{%
\epsfxsize=163mm
\epsfbox{fle-4.eps}
}

\end{center}
\vspace*{-10pt}
    \Caption{Верификация весовых формул по выборке прототипов}
    \vspace*{-6pt}
    \end{figure*}

    
На рис.~4 приведен 
пример верификации эмпирических формул для расчета $M_0$~--- первого 
приближения взлет\-ной массы (рис.~4,\,\textit{а})~--- и~формулы для расчета 
$M_{\mathrm{снар}}$~--- массы пус\-то\-го снаряженного пассажирского самолета 
(рис.~4,\,\textit{б}). 
 Под верификацией здесь понимается применение стандартной процедуры 
анализа среднеквадратичного отклонения от линии парной линейной регрессии, 
по\-рож\-ден\-ной верифицируемой весовой формулой на выборке прототипов из 
БД. Здесь трудно говорить о~ка\-кой-то математической строгости такой 
верификации по двум причинам. Во-пер\-вых, выборки прототипов не могут 
быть большими в~силу того, что прототипов, схожих по выполняемым задачам, 
не может быть очень много. И,~во-вто\-рых, как правило, самолеты, 
выпускаемые одной компанией (консорциумом), не могут рассматриваться как 
независимые сущности, что во\-об\-ще-то важно для методов регрессионного 
анализа. Тем не менее с~инженерной точки зрения грамотно построенные 
формульные зависимости отражают качественный характер физических 
закономерностей, а~их верификация (в~приведенном понимании) 
предоставляет возможность убедиться в~этом. Формальные критерии оценки 
достоверности результатов расчета по эмпирическим <<весовым формулам>> 
здесь вряд ли уместны, но их сравнительный качественный анализ может быть 
полезным при прогнозировании массы ЛА. 
%
На данном этапе, когда для точных 
математических расчетов не хватает информации, принятие решения остается 
за ин\-же\-не\-ром-про\-ек\-ти\-ров\-щи\-ком. 

Заметим, что если расчет по формуле, 
пред\-став\-лен\-ной на рис.~4,\,\textit{а} ($M_0\hm=k_0L_{\mathrm{КРЕЙС}} 
M_{\mathrm{НАГР\_МАКС}}$), мож\-но рас\-смат\-ри\-вать как первое приближение, то 
в~формуле рис.~4,\,\textit{б} ($M_{\mathrm{СНАР}}\hm=k_1 ((M_0 
L_{\mathrm{КРЕЙС}}/\cos(X_i))^{e_1}\hm+k_2(3{,}14D_{\mathrm{Ф}} 
L_{\mathrm{Ф}})^{e_2}$), зависящей от параметров компоновочной схемы,\linebreak 
используется это приближение.
   
  Таким образом, процесс формирования облика может быть организован как 
итеративный процесс уточнения и~детализации весовых прогнозов.  
Сле\-ду\-ющий шаг в~этом направлении~--- более детальное рассмотрение 
весовых характеристик основных агрегатов дерева конструкции пустого ЛА: 
$$
M_{\mathrm{СНАР}}\hm = M_{\mathrm{ПУСТ}}\hm + M_{\mathrm{ЭК\_СН}}\,,
$$
где
$$
M_{\mathrm{ПУСТ}}\hm = M_{\mathrm{С}} + M_{\mathrm{СУ}}\hm + 
M_{\mathrm{ОБ}}\,,
$$ 


\noindent
и~далее вниз по дереву конструкции планера 
$$
M_{\mathrm{ПЛАН}}\hm = M_{\mathrm{КРЕЙС}}\hm + M_{\mathrm{ФЮЗ}}\hm + 
M_{\mathrm{ГО}}\hm + M_{\mathrm{ВО}}\hm + M_{\mathrm{ВПУ}}\,.
$$
  
  Для каждого из агрегатов планера, силовой установки и~оборудования 
в~модуле формирования весового облика может быть построена своя 
пара\-мет\-ри\-че\-ская модель, а при необходимости и~модели структурных 
компонентов нижнего уровня. 

На рис.~5 представлена параметрическая модель отъемной части крыла. 
В этой модели приведены две весовые формулы для расчета массы крыла, 
взятые из книги~\cite{4-fl}. 

    \begin{figure*}%fig5
\vspace*{1pt}
  \begin{center}  
    \mbox{%
\epsfxsize=163mm
\epsfbox{fle-5.eps}
}

\end{center}
\vspace*{-9pt}
\Caption{Параметрическая модель отъемной части крыла}
\end{figure*}
  
 В модуле формирования весового облика решается еще одна важная задача 
весового проектирования~--- анализ центровки компоновочной схемы самолета 
и~выбор мест размещения на борту целевых грузов и~топлива. Задача 
центровки связана с~обеспечением устойчивости и~управ\-ля\-емости ЛА как 
в~полете, так и~на земле. Эти характеристики зависят от положения центра 
масс по отношению к~центру приложения аэродинамических сил, 
аэродинамическому фокусу, а~также к~положению опор  
взлет\-но-по\-са\-доч\-ных устройств. Положение аэродинамического фокуса 
зависит от формы в~плане несущих поверхностей ЛА и~определяется 
аэродинамиками уже на ранних стадиях проектирования. 
{\looseness=1

}

Вся информация, 
необходимая модулю формирования весового облика для анализа центровки, 
приходит вмес\-те с~компоновочной схемой в~виде диапазона допустимых 
положений цент\-ра масс самолета~--- предельно передней и~предельно задней 
центровок. Центровка может задаваться в~абсолютных величинах, но, как 
правило, она задается в~процентах от средней аэродинамической хорды. 
Средняя аэродинамическая хорда вместе с~пло\-щадью базовой несущей 
трапеции крыла служат общепринятыми характерными величинами, 
участвующими во многих проектировочных расчетах. 
%
Задача анализа 
центровки со\-сто\-ит в~определении координат цент\-ра масс самолета и~проверки 
ограничений заданного диапазона. Центр масс ЛА определяется из основного 
соотношения, которое в~модуле формирования облика рас\-смат\-ри\-ва\-ет\-ся наряду 
с~уравнением весового баланса:
  \begin{multline*}
  X_0 M_0 = X_{\mathrm{ПУСТ}} M_{\mathrm{ПУСТ}} + X_{\mathrm{ЭК\_СН}} 
M_{\mathrm{ЭК\_СН}} + {}\\
{}+X_{\mathrm{ЦЕЛ\_НАГР}} M_{\mathrm{ЦЕЛ\_НАГР}} + X_{\mathrm{ТОП}} 
M_{\mathrm{ТОП}}\,.
   \end{multline*}
  
  В модуле формирования весового облика предусмотрена возможность 
задания упрощенных гео\-мет\-ри\-че\-ских моделей топливных емкостей и~рас-\linebreak\vspace*{-12pt}

{ \begin{center}  %fig6
 \vspace*{9pt}
    \mbox{%
\epsfxsize=79mm
\epsfbox{fle-6.eps}
}

\vspace*{6pt}

\noindent
{{\figurename~6}\ \ \small{
Пример области допустимых центровок
}}
\end{center}
}

%\vspace*{6pt}


\noindent
чета 
их массы и~координат их центра масс. Также в~модуле формирования облика 
реализован механизм задания параметров целевой нагрузки. 



  Расчет $X_{\mathrm{ПУСТ}}$ и~$M_{\mathrm{ПУСТ}}$ в~модуле формирования 
весового облика реализован рекурсивным проходом по дереву конструкции 
пустого ЛА снизу вверх. В каждом узле дерева конструкции выполняется 
единообразная процедура суммирования по подчиненным вершинам:
  $$
  M=\sum\limits_i M_i\,; \quad X=\fr{\sum\nolimits_i  X_i M_i}{N}\,.
  $$
  

  
 В~результате таких расчетов может быть построена область допустимых 
центровок, в~которой заданы положения центровки в~различных 
конфигурациях. 

На рис.~6 приведен пример построения об\-ласти допустимых 
центровок.
 Приведенные на рис.~6 точки показывают положение центра масс 
в~конфигурации пустого, снаряженного, загруженного и~заправленного ЛА. 
Это крайние точки в~процессе эксплуатации. Анализ изменения центровки 
в~воздухе, при выработке топлива в~полете, является более сложной задачей, 
которая решается в~процессе разработки специальной системы управления 
выработкой топлива.



  \section{Заключение}

  По некоторым оценкам на этапе формирования облика самолета принимается 
до~70\% технических проектных решений (см., например, \cite[с.~72]{5-fl}). При 
формировании весового облика закладывается фундамент для достижения 
весового совершенства~ЛА. 
%
Разработанный программный 
продукт служит инструментом для проектирования самолетов разных типов 
и~назначения. С~его по\-мощью проектировщик может создать инфраструктуру 
для сбора и~хранения весовой и~другой полезной архивной информации по 
готовым самолетам, а~затем использовать эту информации в~текущем 
проектировании.
%
 При разработке весовых моделей в~БД может быть создано 
пространство весовых параметров, всесторонне описывающих проектируемые 
изделия. В~рамках этой программы создаются структуры верхнего уровня 
дерева конструкции ЛА, разрабатываются параметрические модели основных 
агрегатов. На этом уровне могут вычисляться теоретические оценки массы 
агрегатов самолета и~приниматься решения о~назначении их лимитных весов.

Данная программа разработана как модуль сис\-те\-мы весового проектирования, 
и~вся информация, созданная в~рамках данного модуля, предназначена для 
использования в~других компонентах системы. Как и~все другие компоненты 
сис\-те\-мы весового проектирования ЛА, программа формирования весового 
облика разработана с~по\-мощью инструментального комплекса <<Генератор 
проектов>>~\cite{6-fl}.


  
{\small\frenchspacing
 {%\baselineskip=10.8pt
 %\addcontentsline{toc}{section}{References}
 \begin{thebibliography}{9}
\bibitem{1-fl}
\Au{Вышинский Л.\,Л., Самойлович~О.\,С., Флёров~Ю.\,А.} Программный комплекс 
формирования облика летательных аппаратов~// Задачи и~методы автоматизированного 
проектирования в~авиастроении.~--- М: ВЦ АН СССР, 1991. С.~24--42.
\bibitem{2-fl}
\Au{Вышинский Л.\,Л., Флёров~Ю.\,А., Широков~Н.\,И.} Автоматизированная система 
весового проектирования самолетов~// Информатика и~её применения, 2018. Т.~12. Вып.~1. 
С.~18--30. doi: 10.14357/19922264180103.
\bibitem{3-fl}
Проектирование самолетов~/ Под ред. М.\,А.~Погосяна.~--- 5-е изд.~--- М.: 
Инновационное машиностроение, 2018. 864~с.
\bibitem{4-fl}
\Au{Шейнин В.\,М., Козловский~В.\,И.} Весовое проектирование и~эффективность 
пассажирских самолетов.~--- М.: Машиностроение, 1977. Т.~1. 343~с.
\bibitem{5-fl}
\Au{Егер С.\,М., Лисейцев~И.\,К., Самойлович~О.\,С.} Основы автоматизированного 
проектирования самолетов.~--- М.: Машиностроение, 1986. 232~c.
\bibitem{6-fl}
\Au{Вышинский Л.\,Л., Гринев~И.\,Л., Флеров~Ю.\,А., Широков~А.\,Н., Широков~Н.\,И.} 
Генератор проектов~--- инструментальный комплекс для разработки  
<<кли\-ент-сер\-вер\-ных>> сис\-тем~// Информационные технологии и~вычислительные 
системы, 2003. №\,1-2. С.~6--25.
\end{thebibliography}

 }
 }

\end{multicols}

%\vspace*{-6pt}

\hfill{\small\textit{Поступила в~редакцию 31.08.21}}

\vspace*{8pt}

%\pagebreak

%\newpage

\vspace*{2pt}

\hrule

\vspace*{2pt}

\hrule

\vspace*{2pt}

\def\tit{THEORETICAL FOUNDATION OF~FORMATION OF~AIRCRAFT~WEIGHT~APPEARANCE}


\def\titkol{Theoretical foundation of formation of aircraft weight appearance}


\def\aut{L.\,L.~Vyshinsky and Yu.\,A.~Flerov}

\def\autkol{L.\,L.~Vyshinsky and Yu.\,A.~Flerov}

\titel{\tit}{\aut}{\autkol}{\titkol}

\vspace*{-11pt}

\noindent
Federal Research Center ``Computer Science and Control'' of the Russian Academy of Sciences, 
44-2~Vavilov Str., Moscow 119333, Russian Federation


\def\leftfootline{\small{\textbf{\thepage}
\hfill INFORMATIKA I EE PRIMENENIYA~--- INFORMATICS AND
APPLICATIONS\ \ \ 2021\ \ \ volume~15\ \ \ issue\ 4}
}%
 \def\rightfootline{\small{INFORMATIKA I EE PRIMENENIYA~---
INFORMATICS AND APPLICATIONS\ \ \ 2021\ \ \ volume~15\ \ \ issue\ 4
\hfill \textbf{\thepage}}}

\vspace*{8pt} 

\Abste{The article is devoted to automation of tasks related to formation of 
weight appearance at the initial stage of aircraft design. At this stage, the main structures 
of the information weight model of the product are created which is detailed, 
improved, and used throughout the life cycle, including the stages of production and operation. 
The article presents not only the theoretical foundations of formation of aircraft weight 
appearance (design ``from prototypes,'' 
weight formulas) but also describes the software product developed by the authors
 which is a~tool intended for use at 
the initial stage of designing aircraft of various types and purposes.}

\KWE{mathematical modeling; computer aided design; aircraft; appearance formation; 
weight design; weight model; structure tree; project generator}

\DOI{10.14357/19922264210413}

%\vspace*{-20pt}

%\Ack
%\noindent


\vspace*{6pt}

  \begin{multicols}{2}

\renewcommand{\bibname}{\protect\rmfamily References}
%\renewcommand{\bibname}{\large\protect\rm References}

{\small\frenchspacing
 {%\baselineskip=10.8pt
 \addcontentsline{toc}{section}{References}
 \begin{thebibliography}{9}
 
 %\vspace*{-3pt}
 
      \bibitem{1-fl-1}
      \Aue{Vyshinsky, L.\,L., O.\,S.~Samoylovich, and Yu.\,A.~Flerov.} 1991. Programmnyy 
kompleks formirovaniya oblika letatel'nykh apparatov [Program complex for forming the 
appearance of aircraft]. \textit{Zadachi i~metody avtomatizirovannogo proektirovaniya 
v~aviastroenii} [Tasks and methods of computer-aided design in aircraft industry]. 
Moscow: CC USSR AS.
24--42.

\vspace*{-2pt}

      \bibitem{2-fl-1}
      \Aue{Vyshinskiy, L.\,L., Yu.\,A.~Flerov, and N.\,I.~Shirokov.} 2018. Avtomatizirovannaya 
systema vesovogo proektirovaniya samoletov [Automated system of weight design of aircraft]. 
\textit{Informatika i~ee Primeneniya~--- Inform Appl.} 12(1):18--30. doi: 
10.14357/19922264180103.

\pagebreak

      \bibitem{3-fl-1}
      Pogosyan, M.\,A. ed. 2018. \textit{Proektirovanie samoletov} [Aircraft design]. 5th ed. 
Moscow: Innovatsionnoe mashi\-no\-stro\-enie. 864~p. 
      \bibitem{4-fl-1}
      \Aue{Sheynin, V.\,M., and V.\,I.~Kozlovskiy.} 1977. \textit{Vesovoe proektirovanie 
i~effektivnost' passazhirskikh samoletov} [Weight design and efficiency of passenger aircraft]. 
Moscow: Mechanical Engineering. Vol.~1. 343~p.

%\columnbreak

      \bibitem{5-fl-1}
      \Aue{Eger, S.\,M., I.\,K.~Lisejcev, and O.\,S.~Samojlovich.} 1986. \textit{Osnovy 
avtomatizirovannogo proektirovaniya samoletov} [Fundamentals of aircraft automated design]. 
Moscow: Mashinostroenie. 232~p.
      \bibitem{6-fl-1}
      \Aue{Vyshinskiy, L.\,L., I.\,L.~Grinev, Yu.\,A.~Flerov, A.\,N.~Shirokov, and 
N.\,I.~Shirokov.} 2003. Generator proektov~--- instrumental'nyy kompleks dlya razrabotki 
      ``klient-servernykh'' sistem [The project generator~--- tool complex for development of 
``client-server'' systems]. \textit{Informatsionnye tekhnologii i~vychislitel'nye sistemy} 
[J.~of Information Technologies and Computering Systems] 1-2:6--25.

 \end{thebibliography}

 }
 }

\end{multicols}

\vspace*{-3pt}

\hfill{\small\textit{Received August 31, 2021}}

%\pagebreak

%\vspace*{-24pt}     
      
      \Contr
      
      
  \noindent
  \textbf{Vyshinsky Leonid L.} (b.\ 1941)~--- Candidate of Science (PhD) in physics and 
mathematics, leading scientist, A.\,A.~Dorodnicyn Computing Center, Federal Research Center 
``Computer Science and Control'' of the Russian Academy of Sciences, 40~Vavilov Str., Moscow 
119333, Russian Federation; \mbox{wysh@ccas.ru} 
  
  \vspace*{6pt}
  
  \noindent
  \textbf{Flerov Yuri A.} (b.\ 1942)~--- Corresponding Member of the Russian Academy of 
Sciences, Doctor of Science in physics and mathematics, professor, principal scientist, 
A.\,A.~Dorodnicyn Computing Center, Federal Research Center ``Computer Science and Control'' of 
the Russian Academy of Sciences, 40~Vavilov Str., Moscow 119333, Russian Federation; 
\mbox{fler@ccas.ru}

\label{end\stat}

\renewcommand{\bibname}{\protect\rm Литература}
      