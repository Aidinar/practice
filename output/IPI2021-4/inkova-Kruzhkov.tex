
\def\stat{inkova+kr}

\def\tit{СТРУКТУРИРОВАННЫЕ ОПРЕДЕЛЕНИЯ ДИСКУРСИВНЫХ 
ОТНОШЕНИЙ В~НАДКОРПУСНОЙ БАЗЕ ДАННЫХ КОННЕКТОРОВ$^*$}

\def\titkol{Структурированные определения дискурсивных 
отношений в~Надкорпусной базе данных коннекторов}

\def\aut{О.\,Ю.~Инькова$^1$, М.\,Г.~Кружков$^2$}

\def\autkol{О.\,Ю.~Инькова, М.\,Г.~Кружков}

\titel{\tit}{\aut}{\autkol}{\titkol}

\index{Инькова О.\,Ю.}
\index{Кружков М.\,Г.}
\index{Inkova O.\,Yu.}
\index{Kruzhkov M.\,G.}


{\renewcommand{\thefootnote}{\fnsymbol{footnote}} \footnotetext[1]
{ Исследование выполнено с~использованием ЦКП <<Информатика>> ФИЦ ИУ РАН.}}


\renewcommand{\thefootnote}{\arabic{footnote}}
\footnotetext[1]{Федеральный исследовательский центр <<Информатика 
и~управление>> Российской академии наук, \mbox{olyainkova@yandex.ru}}
\footnotetext[2]{Федеральный исследовательский центр <<Информатика 
и~управление>> Российской академии наук, \mbox{magnit75@yandex.ru}}

\vspace*{-10pt}



    \Abst{Работа посвящена первым результатам разработки структурированных 
определений дискурсивных отношений на основе их оригинальной классификации, а также 
фиксации этих определений в~Надкорпусной базе данных коннекторов (НБД). Авторы 
предлагают краткий обзор существующих подходов к~определению дискурсивных 
отношений, а затем описывают принципы, на основе которых создаются структурированные 
определения дискурсивных отношений в~НБД. Они включают (1)~базовую семантическую 
операцию, на которую опирается ло\-ги\-ко-се\-ман\-ти\-че\-ское отношение (ЛСО): 
импликация, расположение на шкале времени, сравнение, соотнесение частного и~общего 
или элемента и~множества; (2)~уровень, на котором может быть установлено ЛСО: 
пропозициональный уровень, уровень высказывания (иллокутивный), метаязыковой; 
(3)~полярность, т.\,е.\ устанавливается ли ЛСО непосредственно между положениями 
вещей~$p$ и~$q$, описанными в~связываемых им фрагментах текста, или же при его 
интерпретации должны быть учтены также их отрицательные корреляты; (4)~семантические 
и~прагматические характеристики контекста. Приводятся примеры структурированных 
определений. Структурированные определения ЛСО фиксируются в~структуре НБД 
с~по\-мощью набора взаимосвязанных таб\-лиц, отражающих четыре перечисленных типа 
признаков. Кроме того, в~рамках таблицы Семейство (<<семейства>> различительных 
признаков) фиксируется информация о концептуальной близости некоторых наборов 
признаков. Описанная структура данных позволяет исчислять сходства и~различия ЛСО~--- 
на сегодняшний день такая возможность не обеспечивается ни одним из существующих 
корпусов, аннотированных с~точки зрения дискурсивных отношений.}
    
    
    \KW{надкорпусная база данных; логико-семантические отношения; коннекторы; 
аннотирование; фасетная классификация}
    
\DOI{10.14357/19922264210404}
  
\vspace*{-1pt}


\vskip 10pt plus 9pt minus 6pt

\thispagestyle{headings}

\begin{multicols}{2}

\label{st\stat}
    
\section{Классификации и~определения дискурсивных отношений: 
существующие подходы}

   Дискурсивные отношения и~их роль в~обеспечении связности текста 
занимают умы ученых разных областей знаний уже не один десяток лет. 
Напомним, что первые исследования, посвященные этому вопросу 
и~выполненные с~целью обеспечения автоматической обработки текста, 
появились в~середине 1970-х~гг.~[1--3]. Первая и~наиболее популярная 
сегодня классификация дискурсивных, или, согласно данной теории, 
риторических, отношений опубликована в~1988~г.~[4]. Лингвисты сразу же 
увидели ее недостаток: риторические отношения определялись не на основе 
языкового материала и~четких семантических критериев, а на основе интуиции 
пишущего и~читающего. Это привело к~тому, что количество отношений, 
выделяемых в~рамках данного подхода, может насчитывать от~80 до~170~[5], 
что делает их трудно применимыми для аннотирования текста с~точки зрения 
отношений связности. 
   
   С тех пор были предложены многочисленные классификации 
дискурсивных, или, в~терминологии авторов статьи, логико-семантических, 
отношений (см.\ обзор в~[6]), которые, однако, также имеют ряд недостатков. 
Причин тому несколько. Прежде всего, неясным остается основание 
классификаций, которое позволило бы определить семантическую близость 
отношений и~возможность их объединения на этой основе в~классы; см., 
например, классификацию, используемую в~Пенсильванском аннотированном 
корпусе, которая регулярно пересматривается: последний ее вариант~[7] 
значительным образом отличается от версии~2.0~[8]; подробнее о~различиях 
между двумя вариантами см.~[9].
{\looseness=-2

}
    
   Еще один недостаток существующих классификаций заключается в~том, 
что многие из них определяют отношения на основе семантики показателя 
(коннектора), который считается прототипическим. Однако, с~одной стороны, 
критерии выбора этого прототипического показателя неочевидны (см., 
например, в~английском языке выбор \textit{but} для отношения Контраст 
в~[10, 11] при наличии \textit{by contrast}). С~другой стороны, хорошо 
известно, что коннекторы многозначны и~могут выражать несколько 
отношений (ср.\ коннектор \textit{если}\ldots\ \textit{то}: \textit{Если завтра 
будет тепло, то поедем купаться на озеро}~--- условные отношения; 
\textit{Если цена на нефть осталась прежней, то газ сильно подорожал}~--- 
сопоставительные отношения). Кроме того, дискурсивные отношения могут 
не иметь показателя; ср.\ \textit{Будет тепло, поедем купаться на озеро}.
   
   Наконец, существующие классификации не имеют объяснительной силы. 
В~част\-ности, они не могут объяснить, почему коннектор может выражать 
определенный набор отношений, а~в~сопоставительном ключе~--- почему 
коннектор одного семантического класса может служить переводным 
эквивалентом коннектора другого семантического класса. Например, в~(1) 
показатель отношения переформулирования \textit{то есть} переведен 
показателем альтернативы \textit{ou} `или'.
   \begin{enumerate}[(1)]
   \item И в~эту минуту ему на голову упал сценический противовес. \textit{То 
есть}, попросту говоря, брезентовый мешок килограммов на двенадцать.~--- 
Au m$\hat{\mbox{e}}$me instant un contrepoids de sc$\grave{\mbox{e}}$ne lui 
tomba sur la t$\hat{\mbox{e}}$te. \textit{Ou}, pour parler plus simplement, un sac de 
toile d'une bonne douzaine de kilos. [С.~Довлатов. Иностранка (1986)~--- перевод 
Jacques Michaut-Paterno (2001)]
   \end{enumerate}
   
Чтобы отчасти преодолеть перечисленные недостатки, в~работе~[12] 
предлагается определять отношения на основе набора семантических 
признаков. Это позволяет увидеть общие семантические свойства у 
отношений, которые в~разных классификациях носят разные названия 
(подробнее см.~[13]). Число классифицирующих 
признаков\footnote{Классифицирующих признаков всего четыре: 
(1)~полярность (т.\,е.\ необходимо ли для интерпретации отношения 
привлекать отрицание одного из соединяемых им положений вещей; 
например, интерпретация уступительных отношений опирается на причинную 
цепочку, которая считается нормальным развитием событий: \textit{Хотя 
погода была хорошая, купаться мы не поехали} предполагает, что при 
хорошей погоде \textit{мы едем купаться}, но в~данном случае этого не 
произошло); (2)~базовая операция (аддитивная или причинная; например, как 
было показано, причинная базовая операция лежит в~основе уступительных 
отношений); (3)~источник связности (т.\,е.\ насколько говорящий вовлечен 
в~устанавливаемое отношение; ср.\ два вида отношения альтернативы: 
объективное \textit{При высокой температуре следует вызвать скорую или 
обратиться к~лечащему врачу} и~субъективное \textit{Замолчи, или я~тебя 
выгоню}, при котором говорящий предполагает, что ситуация \textit{я~тебя 
выгоню} является нежелательной для слушающего); (4)~порядок следования 
фрагментов текста (этот признак применяется только к~группе отношений, 
основанных на причинной базовой операции, и~позволяет различать 
следственные отношения, при которых следствие как хронологически, так 
и~в~тексте следует за причиной, и~причинные отношения, при которых в~тексте 
причина следует за следствием; ср.: \textit{Маша заболела, следовательно она 
не придет} и~\textit{Маша не придет, потому что заболела}). Подробнее 
см.~[13].} оказывается, однако, недостаточным для описания многообразия 
дискурсивных отношений, поскольку некоторые отношения имеют 
одинаковый набор признаков даже после введения дополнительных 
классифицирующих признаков~[14]. Так, отношения спецификации (\textit{Он 
любит книги, в~частности детективы}) и~генерализации (\textit{Он любит 
детективы и~вообще любит читать}) имеют одинаковый набор не только 
основных признаков (положительная полярность, аддитивная операция, 
объ\-ек\-тив\-ный/субъ\-ек\-тив\-ный источник связности, признак порядка не 
применяется), но и~дополнительный (specifity~--- `специфичность'). 

\begin{table*}\small  %tabl1
\begin{center}
\Caption{Примеры структурированных определений ЛСО}
\vspace*{2ex}

\begin{tabular}{|l|l|}
\hline
\multicolumn{1}{|c|}{\textbf{Перифрастическое 
переформулирование}}&\multicolumn{1}{c|}{\textbf{Метаязыковая альтернатива}}\\
\hline
\tabcolsep=0pt\begin{tabular}{l}
--~операция сравнения, устанавливающая\\
\hphantom{--~}сходство $p$ и~$q$\\
--~метаязыковой уровень\\
--~$p$~--- описание положения вещей $r$\\
--~$q$~--- описания того же положения вещей $r$\\
--~$p$ и~$q$ имеют одинаковый экстенсионал
\end{tabular}&
\tabcolsep=0pt\begin{tabular}{l}--~операция сравнения, устанавливающая\\ 
\hphantom{--~}несходство $p$ и~$q$\\
--~метаязыковой уровень\\
--~$p$~--- возможное описание положения вещей $r$\\
--~$q$~--- возможное описание того же положения вещей $r$\\
--~говорящий предлагает сделать выбор между $p$ и~$q$
\end{tabular}\\
\hline
\end{tabular}
\end{center}
\vspace*{2pt}
\end{table*}
\begin{figure*}[b] %fig1
\vspace*{1pt}
\begin{center}  
\mbox{%
\epsfxsize=154.85mm
\epsfbox{ink-1.eps}
}

\vspace*{9pt}

{\small Структурированное описание ЛСО (фрагмент схемы НБД)}

\end{center}
\end{figure*}



\section{Классификация дискурсивных отношений, используемая 
в~Надкорпусной базе данных коннекторов}

   Классификация, используемая в~НБД,
разработанной в~ИПИ ФИЦ ИУ РАН\footnote{Подробнее о структуре НБД, 
ее возможностях и~результатах, полученных с~ее использованием, см., например, [15, 16]. 
Представительный фрагмент НБД доступен по адресу: {\sf 
http://a179.frccsc.ru/RFH41002/main.aspx}.}, принципиальным образом отличается от 
существующих. 

Во-пер\-вых, в~ее основу положены четыре базовые 
семантические операции, или механизма, на которые опирается то или иное 
ЛСО: импликация, расположение 
на шкале времени, сравнение, соотнесение частного и~общего или элемента 
и~множества. 
   
   
   Во-вто\-рых, классификация различает уровни, на которых может быть 
установлено ЛСО: пропозициональный уровень, уровень высказывания 
(иллокутивный), метаязыковой; подробнее см.~[13]. 

Каждое ЛСО может, 
следовательно, определяться на основе этих двух критериев, к~которым 
добавляется критерий, характеризующий ЛСО, основанные на импликации 
и~сравнении: полярность,\linebreak т.\,е.\ устанавливается ли ЛСО непосредственно 
между положениями вещей~$p$ и~$q$, описанными в~связываемых им 
фрагментах текста, или же при его интерпретации должны быть учтены также 
их \mbox{отрицательные} корреляты $\neg p$ и~$\neg q$. Наконец, учитываются 
семантические и~прагматические характеристики контекста.
   
   Такая концепция классификации дискурсивных отношений позволяет 
описывать их при помощи структурированных определений, представляющих 
собой набор различительных признаков. Два таких определения~--- 
перифрастического переформулирования и~метаязыковой альтернативы~--- 
даны в~табл.~1 (другие определения см.\ в~\cite{9-in}).

   Из определений видно, что эти два отношения имеют несколько общих 
семантических признаков: оба они установлены в~результате операции 
сравнения, на метаязыковом уровне, т.\,е.\ не между положениями вещей, 
а~между описаниями одного и~того же положения вещей~$r$. Различия между 
ними заключаются в~том, что в~случае переформулирования операция 
сравнения устанавливает сходство описаний, а при альтернативе~--- 
несходство, поэтому говорящий предлагает сделать между ними выбор. 
Сходные семантические признаки объясняют, почему показатели 
переформулирования и~метаязыковой альтернативы могут заменять друг 
друга без нарушения связности текста (ср.\ вариант~(2) русского оригинала 
примера~(1) c~\textit{или}), а~при переводе с~одного языка на другой~--- 
переводить друг друга, как в~примере~(1).
   \begin{enumerate}[(1)]
   \setcounter{enumi}{1}
\item И в~эту минуту ему на голову упал сценический противовес. 
\textit{Или}, попросту говоря, брезентовый мешок килограммов на 
двенадцать.
   \end{enumerate}

\section{Структурированные определения в~программном 
обеспечении Надкорпусной базы данных}

\vspace*{-14pt}
   
   Структурированные определения ЛСО фиксируются в~структуре НБД 
   с~по\-мощью набора взаимосвязанных таблиц (см.\ соответствующий фрагмент 
схемы НБД на рисунке). В~таблице \textbf{ЛСО} фиксируются неформальные 
описания ЛСО с~отсылками к~соответствующим рубрикам фасетной 
классификации, использующимся при аннотации переводных соответствий 
в~НБД коннекторов. В~таблице \textbf{Признак} содержатся различительные 
признаки, из комбинаций которых складываются структурированные 
определения каждого ЛСО. Различительные признаки могут относиться 
к~одному из типов, определения которых содержатся в~таблице \textbf{Тип}. 
В~настоящее время в~этой таблице зафиксированы четыре типа признаков: 
(1)~базовая операция, лежащая в~основе ЛСО; (2)~языковой уровень, на 
котором устанавливается ЛСО; (3)~дополнительные  
се\-ман\-ти\-че\-ские/праг\-ма\-ти\-че\-ские особенности; (4)~полярность 
аргументов ЛСО. При не\-об\-хо\-ди\-мости в~дальнейшем набор типов признаков 
может быть расширен.

\begin{table*}\small %tabl2
\begin{center}
\Caption{Сопоставление структурированных описаний ЛСО в~структуре НБД}
\vspace*{2ex}

\tabcolsep=5pt
\begin{tabular}{|p{48mm}|p{28mm}|p{48mm}|p{28mm}|}
\hline
\multicolumn{2}{|c|}{\textbf{ЛСО <<Пропозициональное 
замещение>>}}&\multicolumn{2}{c|}{\textbf{ЛСО <<Замещение по дескриптивной 
адекватности>>}}\\
\hline
\multicolumn{1}{|c|}{Признак}&\multicolumn{1}{c|}{\textit{Тип}}&
\multicolumn{1}{c|}{Признак}&\multicolumn{1}{c|}{\textit{Тип}}\\
\hline
операция сравнения, уста\-нав\-ли\-ва\-ющая несходство $p$ и~$q$&\textit{базовая операция}&
операция сравнения, уста\-нав\-ли\-ва\-ющая несходство~$p$~и~$q$&\textit{базовая операция}\\
\hline
пропозициональный уровень&\textit{уровень}&
метаязыковой уровень&\textit{уровень}\\
\hline
$p$~--- положение вещей, осу\-ще\-ст\-в\-ле\-ние которого можно было бы ожидать&\textit{се\-ман\-ти\-ка/праг\-ма\-ти\-ка}&
$p$~--- описание положения вещей~$r$, которое можно 
было бы ожидать&\textit{семантика}/\textit{праг\-ма\-ти\-ка}\\
\hline
$q$~--- положение вещей, не со\-от\-вет\-ст\-ву\-ющее ожиданиям &\textit{семантика}/\textit{праг\-ма\-ти\-ка}&
$q$~--- описание того же положения вещей~$r$, не со\-от\-вет\-ст\-ву\-ющее 
ожиданиям&\textit{се\-ман\-ти\-ка/праг\-ма\-ти\-ка}\\
\hline
$p$ отвергается, принимается~$q$&\textit{полярность}&
$p$ отвергается, принимается~$q$&\textit{полярность}\\
\hline
\end{tabular}
\end{center}

\end{table*}
   
Роль таблицы \textbf{Семейство} (<<семейства>> различительных признаков) 
состоит в~том, чтобы фиксировать информацию о~концептуальной бли\-зости 
некоторых наборов признаков. В~то время как \mbox{описания} различительных 
признаков могут не совпадать, они могут вклю\-чать в~себя общие 
концептуальные компоненты, благодаря которым различные ЛСО, 
описывающиеся этими формально не совпадающими признаками, могут 
оказываться семантически или функционально близкими. Например, 
семейство признаков <<Семантика исключения>> включает в~себя 
следующий набор признаков: 
%\begin{enumerate}[(1)]
%\item 
(1)~$Y$ содержит указание на элемент~$q$, 
ис\-клю\-ча\-емый из множества~$P$; 
%\item %
(2)~$Y$ содержит указание на элемент~$q$, 
который надо исключить, чтобы признать истинным~$p$; 
%\item %
(3)~$p$ верно, 
только если исключить осуществление~$q$\footnote{В описаниях признаков 
строчные литеры~$p$ и~$q$ представляют собой пропозициональные компоненты, 
связываемые данным ЛСО, а заглавные литеры $P$, $Q$, $X$ и~$Y$ представляют 
собой отсылки к~множествам, задействованным в~ЛСО.}.
%\end{enumerate}
 Таким образом, сходство ЛСО может 
определяться не только на основе наличия пол\-ностью совпадающих 
признаков, но и~на основе принадлежности признаков различных ЛСО 
к~одному и~тому же <<семейству>> признаков.
   
   Описанная структура данных уже позволяет выделять функциональные 
   и~концептуальные группы ЛСО, а~так\-же в~предварительном при\-бли\-же\-нии 
оценивать степень бли\-зости различных ЛСО. Например, в~табл.~2 
представлены структурированные описания ЛСО <<Пропозициональное 
замещение>> и~<<Замещение по дескриптивной аде\-кват\-ности>>. Каждое из 
этих ЛСО описывается 5~признаками, из которых~2 совпадают, а~еще по 
2~признака каж\-до\-го отношения относятся к~общему семейству <<компонент 
ожидания>> (признаки, относящиеся к~типу \textit{се\-ман\-ти\-ка/праг\-ма\-ти\-ка}). 
Таким образом, 4 из~5 компонентов описания обоих ЛСО 
оказываются общими или концептуально близ\-ки\-ми, что свидетельствует 
о~значительной бли\-зости от\-но\-ше\-ний.
{\looseness=1

}
   
   Кроме того, при ближайшем рассмотрении становится очевидно, что не 
все различительные признаки в~равной мере влияют на близость опи\-сы\-ва\-емых 
ими ЛСО. Например, признаки, относящиеся к~типу \textit{уровень}, вряд ли 
могут считаться определяющими при выяснении близости соответствующих 
отношений. В~связи с~этим в~таблицах \textbf{Признак} и~\textbf{Семейство} 
также предусмотрены поля <<Коэффициент близости>>, в~которых в~будущем 
предполагается фиксировать весовые коэффициенты, позволяющие более 
точно определять степень близости между ЛСО. Значения этих весовых 
коэффициентов у семейств и/или отдельных признаков будут соответствовать 
степени влияния соответствующих признаков (или признаков, входящих 
в~данное семейство) на близость описываемых ими отношений.

\vspace*{-9pt}

\section{Заключение}

\vspace*{-3pt}

   Используемые в~НБД структурированные определения ЛСО имеют, как 
было показано, ряд преимуществ перед существующими. Во-пер\-вых, они 
включают достаточный набор классифицирующих признаков, позволяющих 
описывать ЛСО. Это, в~свою очередь, позволяет описывать семантические 
свойства ЛСО вне зависимости от семантики показателей, которые его могут 
выражать. Во-вто\-рых, как классификация ЛСО, так и~их определения 
обладают объяснительной силой. Они позволяют, в~част\-ности, объяснить, 
почему один и~тот же показатель может выражать несколько ЛСО и~каких 
именно, а на основе семантических и~прагматических особенностей 
фрагментов текста определить тот набор ЛСО, которые могут быть между 
ними установлены. Наконец, фиксация структурированных определений ЛСО 
в~структуре НБД с~помощью набора взаимосвязанных таблиц позволяет 
количественно оценивать сходства и~различия ЛСО. На сегодняшний день 
такая возможность не обеспечивается ни одним из существующих корпусов, 
аннотированных с~точки зрения дискурсивных отношений. 
   
{\small\frenchspacing
 {%\baselineskip=10.8pt
 %\addcontentsline{toc}{section}{References}
 \begin{thebibliography}{99}

\bibitem{1-in}
\Au{Hobbs J. R.} A~computational approach to discourse analysis. ~--- New York, NY, USA: 
Department of Computer Science, City College, City University of New York, 1976. Research Report 76-2.
P.~28--38.
\bibitem{2-in}
\Au{Hobbs J. R.} Why is discourse coherent?~--- Menlo Park, CA, USA: 
SRI International, 1978. SRI Technical Note 176. 44~p.
\bibitem{3-in}
\Au{Hobbs J. R.} Coherence and coreference~// Cognitive Sci., 1979. Vol.~3. No.\,1. P.~67--90. 
\bibitem{4-in}
\Au{Mann W.\,C., Thompson~S.\,A.} Rhetorical structure theory: Towards a functional theory of text 
organization~// Text, 1988. Vol.~8. No.\,3. P.~243--281. 
\bibitem{5-in}
\Au{Knott A., Dale~R.} Using linguistic phenomena to motivate a set of coherence relations~// 
Discourse Process., 1994. Vol.~18. No.\,1. 
P.~35--62.
\bibitem{6-in}
\Au{Гончаров А.\,А.} Классификации внутритекстовых отношений: основания и~принципы 
структурирования~// Вопросы языкознания, 2021. №\,3. С.~97--119. 
\bibitem{7-in}
\Au{Webber B., Prasad~R., Lee~A., Joshi~A.} The Penn Discourse Treebank~3.0 annotation 
manual, 2019. 81~p. {\sf https://catalog.ldc.upenn.edu/docs/LDC2019T05/ PDTB3-Annotation-Manual.pdf}.
\bibitem{8-in}
PDTB Research Group. The Penn Discourse Treebank~2.0 annotation manual.~--- 
Philadelphia, PA, USA: Institute for Research in Cognitive Science, University of 
Pennsylvania, 2008. Technical Report IRCS-08-01. 99~p. 
{\sf https://www.seas.upenn.edu/$\sim$pdtb/PDTBAPI/pdtb-annotation-manual.pdf}.
\bibitem{9-in}
\Au{Инькова О.\,Ю.} Определения дискурсивных отношений: опыт Надкорпусной базы 
данных коннекторов~// Компьютерная лингвистика и~интеллектуальные технологии: По 
мат-лам ежегодной \mbox{Междунар.} конф. <<Диалог>>.~--- М.: РГГУ, 2021. Вып.~20(27). С.~328--338.
\bibitem{10-in}
\Au{Rudolph E.} Contrast: Adversative and concessive expressions on sentence and text level.~--- 
Berlin/Boston: Walter de Gruyter, 1996. 564~p.
\bibitem{11-in}
\Au{Fraser B.} An account of discourse markers~// International Review Pragmatics, 2009. Vol.~1. 
No.\,2. P.~293--320.
\bibitem{12-in}
\Au{Sanders T., Spooren~W., Noordman~L.} Toward a taxonomy of coherence relations~// 
Discourse Process., 1992. Vol.~15. No.\,1. Р.~1--35.
\bibitem{13-in}
\Au{Инькова О.\,Ю.} Логико-семантические отношения: проблемы классификации~// 
Связ\-ность текста: мереологические 
ло\-ги\-ко-се\-ман\-ти\-че\-ские отношения.~--- М.: ЯСК, 2019. С.~11--98.
\bibitem{14-in}
\Au{Sanders T., Demberg~V., Hoek~J., Scholman~M., Asr~F.\,T., Zufferey~S., Evers-Vermeul~J.} 
Unifying dimensions in coherence relations: How various annotation frameworks are 
related~// Corpus Linguist. Ling., 2021. Vol.~17. No.\,1. P.~1--71.
\bibitem{15-in}
Семантика коннекторов: контрастивное исследование~/ Под. ред. О.\,Ю.~Иньковой.~--- 
М.: ТОРУС ПРЕСС, 2018. 368~с.
\bibitem{16-in}
Структура коннекторов и~методы ее описания~/ Под. ред. О.\,Ю.~Иньковой.~--- М.: 
ТОРУС ПРЕСС, 2019. 316~с.

\end{thebibliography}

 }
 }

\end{multicols}

\vspace*{-6pt}

\hfill{\small\textit{Поступила в~редакцию 29.09.21}}

\vspace*{8pt}

%\pagebreak

%\newpage

%\vspace*{-28pt}

\hrule

\vspace*{2pt}

\hrule

%\vspace*{-2pt}

\def\tit{STRUCTURED DEFINITIONS OF~DISCOURSE RELATIONS IN~THE~SUPRACORPORA 
DATABASE OF~CONNECTIVES}


\def\titkol{Structured definitions of~discourse relations in the Supracorpora 
Database of Connectives}


\def\aut{O.\,Yu.~Inkova and M.\,G.~Kruzhkov}

\def\autkol{O.\,Yu.~Inkova and M.\,G.~Kruzhkov}

\titel{\tit}{\aut}{\autkol}{\titkol}

\vspace*{-11pt}


\noindent
Federal Research Center ``Computer Science and Control'' of the Russian Academy of Sciences,  
44-2 Vavilov Str., Moscow 119333, Russian Federation

\def\leftfootline{\small{\textbf{\thepage}
\hfill INFORMATIKA I EE PRIMENENIYA~--- INFORMATICS AND
APPLICATIONS\ \ \ 2021\ \ \ volume~15\ \ \ issue\ 4}
}%
 \def\rightfootline{\small{INFORMATIKA I EE PRIMENENIYA~---
INFORMATICS AND APPLICATIONS\ \ \ 2021\ \ \ volume~15\ \ \ issue\ 4
\hfill \textbf{\thepage}}}

\vspace*{3pt} 

\Abste{The paper presents initial outcomes resulting from development 
of structured definitions of discourse relations based on novel classification 
principles and describes how these definitions are captured in the Supracorpora 
Database of Connectives (SCDB). The authors provide an overview of existing approaches 
to definition of discourse relations and propose novel principles for capturing
structured definitions of discourse relations in the SCDB based on several aspects 
which include ($i$)~the basic semantic operation that the logical-semantic 
relation (LSR) is based on: implication, relative timeline positioning, comparison, and
correlation between general and specific, between an element and a~set; 
($ii$)~the linguistic level that the LSR is established on: propositional level,
utterance (illocutionary) level, and metalinguistic level; 
($iii$)~the polarity, i.\,e., whether the LSR is established directly between the 
provisions~$p$ and~$q$ featuring in the text or whether their negative correlates 
should also be considered in the relation interpretation;
and ($i\nu$)~the semantic and pragmatic features of the context. The paper
provides some examples of such structured definitions. The structured definitions are 
captured within the SCDB by a set of interrelated tables. In addition, the\linebreak\vspace*{-12pt}}
  
\Abstend{``Family'' 
table is introduced to offer information about conceptual closeness of some sets 
of classification features. The proposed structure allows researchers to access similarities 
and distinctions between various LSRs~--- as of today, this functionality is not implemented 
in any of the existing corpora that include annotation of discourse relations.}

\KWE{supracorpora database; logical-semantic relations; connectives; annotation; 
faceted classification}

\DOI{10.14357/19922264210404}

\vspace*{-12pt}

\Ack

\vspace*{-3pt}


\noindent
The research was carried out using the infrastructure of the Shared Research Facilities ``High Performance 
Computing and Big Data'' (CKP ``Informatics'') of FRC CSC RAS (Moscow).

\vspace*{3pt}

  \begin{multicols}{2}

\renewcommand{\bibname}{\protect\rmfamily References}
%\renewcommand{\bibname}{\large\protect\rm References}

{\small\frenchspacing
 {%\baselineskip=10.8pt
 \addcontentsline{toc}{section}{References}
 \begin{thebibliography}{99}

\bibitem{1-in-1}
\Aue{Hobbs, J.\,R.} 1976. A~computational approach to discourse analyses. New York, NY: 
Department of Computer Science, City College, City University of New York. Research Report 76-2. 
28--38.
\bibitem{2-in-1}
\Aue{Hobbs, J.\,R.} 1978. Why is discourse coherent? Menlo Park, CA: SRI International. SRI 
Technical Note 176. 44~p.
\bibitem{3-in-1}
\Aue{Hobbs, J.\,R.} 1979. Coherence and coreference. \textit{Cognitive Sci.} 3(1):67--90.
\bibitem{4-in-1}
\Aue{Mann, W.\,C., and S.\,A.~Thompson.} 1988. Rhetorical structure theory: Towards a functional 
theory of text organization.
\textit{Text} 8(3):243--281.
\bibitem{5-in-1}
\Aue{Knott, A., and R.~Dale.} 1994. Using linguistic phenomena to motivate a set of coherence relations. 
\textit{Discourse Process.}
 18(1):35--62.
\bibitem{6-in-1}
\Aue{Goncharov, A.\,A.} 2021. Klassifikatsii vnutritekstovykh otnosheniy: Osnovaniya i~printsipy 
strukturirovaniya [Classification of intratextual relations: Bases and structural principles]. \textit{Voprosy 
yazykoznaniya} [Topics in the Study of Language] 3:97--119.
\bibitem{7-in-1}
\Aue{Webber, B., R.~Prasad, A.~Lee, and A.~Joshi.} 2019. The Penn Discourse Treebank~3.0 
Annotation Manual. Available at: {\sf https://catalog.ldc.upenn.edu/docs/
LDC2019T05/PDTB3-Annotation-Manual.pdf} (accessed October~29, 2021).
\bibitem{8-in-1}
PDTB Research Group. 2008. The Penn Discourse Treebank~2.0 Annotation Manual. Philadelphia, PA: 
Institute for Research in Cognitive Science, University of Pennsylvania. Technical Report IRCS-08-01. 
99~p. Available at: {\sf   
https://www.seas.upenn.edu/$\sim$pdtb/PDTBAPI/pdtb-annotation-manual.pdf} (accessed October~29, 
2021).
\bibitem{9-in-1}
\Aue{Inkova, O.\,Yu.} 2021. Opredeleniya diskursivnykh otnosheniy: opyt Nadkorpusnoy bazy dannykh 
konnektorov [Definition of discourse relations: The case of the Supracorpora Database of Connectives]. 
\textit{Komp'yuternaya lingvistika i~intellektual'nye tekhnologii: po mat-lam Mezhdunar. konf. 
``Dialog''} [Computational Linguistics and Intellectual Technologies: Papers from the Annual Conference 
(International) ``Dialogue'']. Moscow: RSHI. 20(27):328--338.
\bibitem{10-in-1}
\Aue{Rudolph, E.} 1996. \textit{Contrast: Adversative and concessive expressions on sentence and 
text level.} Berlin/Boston: Walter de Gruyter. 564~p.
\bibitem{11-in-1}
\Aue{Fraser, B.} 2009. An account of discourse markers. \textit{International Review Pragmatics} 
1(2):293--320.
\bibitem{12-in-1}
\Aue{Sanders, T., W.~Spooren, and L.~Noordman.} 1992. Toward a taxonomy of coherence relations. 
\textit{Discourse Process.} 15(1):1--35.
\bibitem{13-in-1}
\Aue{Inkova, O.\,Yu.} 2019. Logiko-semanticheskie otnosheniya: Problemy klassifikatsii [Logical-semantic 
relations: Classification problems]. \textit{Svyaznost' teksta: mereologicheskie logiko-semanticheskie 
otnosheniya} [Text coherence: Mereological logical semantic relations]. Moscow: LRC Publishing House. 
11--98.
\bibitem{14-in-1}
\Aue{Sanders T., V.~Demberg, J.~Hoek, M.~Scholman, F.\,T.~Asr, S.~Zufferey, and J.~Evers-
Vermeul.} 2021. Unifying dimensions in coherence relations: How various annotation frameworks are 
related. \textit{Corpus Linguist. Ling.} 17(1):1--71.
\bibitem{15-in-1}
In'kova, O.\,Yu., ed. 2018. \textit{Semantika konnektorov: Kontrastivnoe issledovanie} [Semantics of 
connectives: Contrastive study]. Moscow: TORUS PRESS. 368~p.
\bibitem{16-in-1}
In'kova, O.\,Yu., ed. 2019. \textit{Struktura konnektorov i~metody ee opisaniya} [Structure of 
connectives and methods of its description]. Moscow: TORUS PRESS. 316~p.
\end{thebibliography}

 }
 }

\end{multicols}

\vspace*{-3pt}

\hfill{\small\textit{Received September 29, 2021}}

%\pagebreak

\vspace*{-14pt}

\Contr

\vspace*{-3pt}


\noindent
\textbf{Inkova Olga Yu.} (b.\ 1965)~--- Doctor of Science  in philology, senior scientist, Institute of 
Informatics Problems, Federal Research Center ``Computer Science and Control'' of the Russian 
Academy of Sciences, 44-2~Vavilov Str., Moscow 119333, Russian Federation; 
\mbox{olyainkova@yandex.ru}


\vspace*{3pt}

\noindent
\textbf{Kruzhkov Mikhail G.} (b.\ 1975)~--- senior scientist, Institute of Informatics Problems, Federal 
Research Center ``Computer Science and Control'' of the Russian Academy of Sciences, 44-2~Vavilov 
Str., Moscow 119333, Russian Federation; \mbox{magnit75@yandex.ru}



\label{end\stat}

\renewcommand{\bibname}{\protect\rm Литература} 
      
  