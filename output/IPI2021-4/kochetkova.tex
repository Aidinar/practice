\def\stat{kochetkova}

\def\tit{МОДЕЛЬ СХЕМЫ ПРИОРИТЕТНОГО ДОСТУПА ТРАФИКА URLLC И~eMBB 
В~СЕТИ ПЯТОГО ПОКОЛЕНИЯ В~ВИДЕ РЕСУРСНОЙ СИСТЕМЫ МАССОВОГО ОБСЛУЖИВАНИЯ$^*$}

\def\titkol{Модель схемы приоритетного доступа трафика URLLC и~eMBB 
в~сети пятого поколения в~виде РСМО}
%ресурсной системы массового обслуживания}

\def\aut{И.\,А.~Кочеткова$^1$, А.\,И.~Кущазли$^2$, П.\,А.~Харин$^3$, 
С.\,Я.~Шоргин$^4$}

\def\autkol{И.\,А.~Кочеткова, А.\,И.~Кущазли, П.\,А.~Харин, 
С.\,Я.~Шоргин}

\titel{\tit}{\aut}{\autkol}{\titkol}

\index{Кочеткова И.\,А.}
\index{Кущазли А.\,И.}
\index{Харин П.\,А.} 
\index{Шоргин С.\,Я.}
\index{Kochetkova I.\,A.}
\index{Kushchazli A.\,I.}
\index{Kharin P.\,A.}
\index{Shorgin S.\,Ya.}


{\renewcommand{\thefootnote}{\fnsymbol{footnote}} \footnotetext[1]
{Публикация создана при поддержке Программы стратегического академического лидерства РУДН. 
Исследование выполнено при финансовой поддержке РФФИ (проект 20-37-70079).}}


\renewcommand{\thefootnote}{\arabic{footnote}}
\footnotetext[1]{Российский университет дружбы народов; Федеральный исследовательский центр 
  <<Информатика и~управление>> Российской академии наук, gudkova-ia@rudn.ru}
\footnotetext[2]{Российский университет дружбы народов, aikushch@yandex.ru}
\footnotetext[3]{Российский университет дружбы народов, pxarin@mail.ru}
\footnotetext[4]{Федеральный исследовательский центр <<Информатика и~управление>> 
Российской академии наук, sshorgin@ipiran.ru}

\vspace*{-8pt}

  
  
     
  
  \Abst{Сверхнадежная передача данных с~малой задержкой (URLLC, ultrareliable and low-latency communication) 
и~усовершенствованная подвижная широкополосная связь (eMBB, enhanced mobile broadband) стали ключевыми 
сценариями реализации сетей пятого поколения (5G, fifth generation). В~работе формализована модель 
ресурса сети 5G в~виде тройки~--- полоса пропускания радиочастот, продолжительность 
слота времени и~максимально возможная мощность передаваемого сигнала. Для схемы 
занятия ресурса, предполагающей адаптивное изменение мощности сигнала и~равномерное 
распределение слота времени между устройствами, приведено условное распределение 
вероятностей приема запроса на передачу трафика. Модель с~приоритетным доступом 
URLLC и~прерыванием обслуживания eMBB с~учетом указанной схемы занятия ресурса 
построена в~виде ресурсной системы массового обслуживания (РСМО).}
   
  \KW{5G; URLLC; eMBB; приоритетный доступ; прерывание обслуживания; ресурсная 
система массового обслуживания}

\DOI{10.14357/19922264210412}
  
\vspace*{-2pt}


\vskip 10pt plus 9pt minus 6pt

\thispagestyle{headings}

\begin{multicols}{2}

\label{st\stat}
  
\section{Введение}

  Беспроводные сети пятого поколения содержат крупномасштабные 
системы межмашинной связи (massive machine-type communication, mMTC), 
сверхнадежную передачу данных с~малой задержкой  и~усовершенствованную подвижную широкополосную 
связь. Например, в~[1--3] рас\-смат\-ри\-ва\-ют\-ся 
подходы к~совместному обслуживанию URLLC и~eMBB. Для учета случайного 
расположения устройств в~сети применяется стохастическая геометрия~[4]. 
Если же учитывать не только модель движения пользователей, но и~их 
активность обращения к~тем или иным услугам, то удобным аппаратом 
являются РСМО~[5, 6]. 
Ресурсные сис\-те\-мы массового обслуживания находят применения в~различных областях, например при 
анализе совместного использования радиочастот~[7, 8]. 
  
  Данная работа продолжает исследования, представленные в~[9--11]. 
В~\cite{9-koc} рас\-смот\-ре\-на модель совместного обслуживания URLLC и~eMBB 
с~приостановкой обслуживания eMBB, а~в~\cite{10-koc} рас\-смот\-ре\-на модель 
с~изменением скорости передачи сессии eMBB до ее прерывания.
  
  Статья организована сле\-ду\-ющим образом. В~разд.~2 описаны 
характеристики соты, требования к~ресурсу и~показаны существующие 
планировщики для распределения радиоресурсов. Далее, в~разд.~3, детально 
описана процедура приоритетного доступа узкополосного трафика URLLC 
и~функционирования системы в~целом. В~разд.~4 
выведены ве\-ро\-ят\-ност\-но-вре\-мен\-н$\acute{\mbox{ы}}$е характеристики для предложенной модели.

\vspace*{-7pt}
  
\section{Модель ресурса}

\vspace*{-2pt}

  В сетях скорость передачи данных зависит от таких параметров, как частота, 
мощность и~время. Рассмотрим соту сети беспроводной передачи данных, 
в~которой устройства распределены равномерно, при этом радиус передачи 
данных равен~$R$, так что $0\hm< d\hm<R$, где $d$~--- расстояние от 
пользователя до базовой станции. Предполагается, что ресурс имеет 
трехмерную структуру и~задается как $\mathrm{Y}\hm= [0,\hat{F}]\times 
[0,\hat{T}=1] \times [0,\hat{F}]$, где, в~свою очередь, $\hat{F}$~--- полоса 
пропускания частот; $\hat{T}$~--- временн$\acute{\mbox{а}}$я единица, описывающая 
доступную полосу пропускания частот и~нормированная к~1, т.\,е.\ 
$\hat{T}\hm=1$; $\hat{F}$~--- максимально возможная мощность 
передаваемого сигнала. Для планирования распределения ресурсов существует 
множество схем, к~примеру, таких как Full-Power (FP) и~Round Robin (RR). 
Разница между ними в~том, что для FF все устройства всегда работают на 
максимальной мощности, а временной ресурс распределяется соответственно 
достижимой каждым устройством скорости. А~для RR временной ресурс 
делится на равные части между всеми устройствами пользователей, которые 
для обеспечения гарантированной минимальной скорости сами регулируют 
мощность передачи данных.
  
  Требование к~ресурсу для передачи потокового трафика можно выразить 
вектором $\bm{y}\hm= (y_f, y_t, y_p)$. Тогда скорость передачи~$b$ можно 
представить в~виде системы трансцендентных уравнений
  \begin{align*}
  b&= y_f y_t \log_2\left( 1+\fr{y_p\mathrm{PL}}{N_0}\right)\,;\\
  A_j\left( y_t, y_t, y_p\right) &=0\,,\enskip j=1,\ldots , J\,,
  \end{align*}
где $\mathrm{PL}$~--- коэффициент затухания сигнала; $N_0$~--- коэффициент 
мощности шума; $A_j(y_f, y_t, y_p)$~--- некоторая функция. Политика 
занятия ресурса основана на его разделении по частоте и~времени, т.\,е.\ при 
условии~$n$ сессий потокового трафика множество возможных конфигураций 
занятия ресурса имеет вид: 
\begin{multline*}
\mathbf{Y}(n)= \left\{
\vphantom{\sum\limits^n_{i=1}}
 \left( \bm{y}_1, \ldots, \bm{y}_n\right):\right.\\
0<y_f^i\leq \hat{F}\,,\ 0<y_t^i\leq 1\,,\ 0< y_p^i\leq \hat{F},\\
\left. \sum\limits_{i=1}^n y_f^i\leq \hat{F}\,,\ \sum\limits^n_{i=1} y_t^i\leq 1\,,\ 
y_p^i\leq \hat{F}\,,\ i=1,\ldots , n\right\}\,,\\
 n>0\,,
%\label{e1-koc}
\end{multline*}
а значит, при случайном затухании сигнала~$\mathrm{PL}$  вероятность приема запроса на 
передачу потокового трафика при условии, что уже установлено~$n$~сессий, 
определяется по формулам
\begin{align*}
p_0&= {\sf P} \left\{ \left( \bm{y}_1, \ldots , \bm{y}_n, \bm{y}_{n+1}\right) \in 
\mathbf{Y}(1)\right\}\,;\\
p_n&= \fr{{\sf P} \left\{ \left( \bm{y}_1, \ldots , \bm{y}_n, \bm{y}_{n+1}\right) 
\in \mathbf{Y}(n+1)\right\}}{{\sf P} \left\{ \left( \bm{y}_1,\ldots , \bm{y}_n\right) 
\in \mathbf{Y}(n)\right\}}\,,\ n>0\,,
%\label{e2-koc}
\end{align*}
где ${\sf P} \left\{ \left( \bm{y}_1,\ldots , \bm{y}_n\right) \in 
\mathbf{Y}(n)\right\}$ рассчитывается по политике занятия ресурса с~учетом 
функциональной зависимости между компонентами требования к~ресурсу 
$\bm{y}\hm= \left( y_f, y_t, y_p\right)$.

  В работе описывается политика управления радиоресурсами~RR. Как 
упоминалось ранее, в~этом алгоритме временной ресурс делится поровну между 
всеми имеющимися пользователями. Однако в~то же время мощность передачи 
регулируется индивидуально в~зависимости от требований. Получается, что 
если реализована политика адаптивного изменения мощности без разделения 
по частоте, т.\,е.\ требование $\bm{y}\hm= \left(y_f, y_t, y_p\right)$ к~ресурсу 
и~множество $\mathbf{Y}_{\mathrm{RR}}(n)$ возможных конфигураций 
занятия ресурса имеют вид:
  \begin{align*}
    \bm{y}(n)&= \left( y_f=\hat{F}\,,\ y_t=\fr{1}{n}\,,\right.\\[6pt]
  &\left.\hspace*{10mm}y_p=N_0\left( 
2^{bn/\hat{F}}-1\right)\mathrm{PL}^{-1}
\vphantom{\fr{1}{n}}\right)\,;\\
  \mathbf{Y}_{\mathrm{RR}} (n)&= \left\{ \left( \bm{y}_1, \ldots , 
\bm{y}_n\right):\ \bm{y}_i=\left( \hat{F}, \fr{1}{n}, y_p^i\right),\right.\\
& \left.\hspace*{2mm}0<y_p^i\leq  \hat{F}\,,\ i=1,\ldots , n
\vphantom{\fr{1}{n}}\right\},\ n\geq0\,,
  %  \label{e3-koc}
  \end{align*}
то вероятность приема новой сессии потокового трафика при условии, что уже 
установлено~$n$ сессий, вычисляется по формулам
\begin{align*}
p_0&= 1-F_{\mathrm{PL}} \left( N_0 \left[ 2^{b/\hat{F}} -1\right] \hat{F}^{-
1}\right);\\
p_n&= \fr{\left\{ 1-F_{\mathrm{PL}} \left( N_0\left[ 2^{b(n+1)/\hat{F}}-1\right] 
\hat{F}^{-1}\right)\right\}^{n+1}} {\left\{ 1-F_{\mathrm{PL}} \left( N_0\left[ 
2^{bn/\hat{F}}-1\right] \hat{F}^{-1}\right)\right\}^n}\,,\\
&\hspace*{61mm} n>0\,,
%\label{e4-koc}
\end{align*}
где $F_{\mathrm{PL}}(x)$~--- функция распределения коэффициента затухания 
сигнала. 

  Предположим, что устройства распределены равномерно по кругу~$R$, 
тогда функция распределения $F_d(x)\hm= {\sf P} \left\{ \xi_d\leq x\right\} \hm= 
x^2/R^2$, а~плот\-ность распределения равна $f_d(x)\hm= 2x/R^2$. Случайная 
величина затухания сигнала $\xi_{\mathrm{PL}} \hm= A\xi_d^B$, где $A$ 
и~$B$~--- константы. Тогда появляется возможность выразить 
$F_{\mathrm{PL}}(x)$ через $F_d(x)$ по формуле:
  \begin{multline*}
  F_{\mathrm{PL}} ={\sf P} \left\{ \xi_{\mathrm{PL}}\leq x\right\} ={\sf P} \left\{ 
\xi_d\leq \left( \fr{x}{A}\right)^{-B} \right\} ={}\\
{}=F_d\left( \fr{x}{A}\right)^{-B} = 
\left( \fr{x}{A}\right)^{-2B}.
  %\label{e5-koc}
  \end{multline*}
  
  Вероятность приема вычисляется по сле\-ду\-ющим формулам:
  \begin{equation}
  \left.
  \begin{array}{rl}
  p_0 &= 1-\fr{\left( N_0\left[ 2^{b/\hat{F}}-1\right] \hat{F}  
\right)^{-2B}}{(AR)^2}\,;\\[6pt]
  p_n&={}\\
  &\hspace*{-22pt}{}= \fr{\left\{ 1-N_0 \left[ 2^{b(n+1)/\hat{F}}-1\right]\hat{F}^{-1}/(AR)
  \right\}^{2(n+1)}}
  { \left\{ 1-
   N_0\left[ 2^{bn/\hat{F}}-1\right]\hat{F}^{-1}/(AR)\right\}^{2n}}\,,\\[6pt] 
&\hspace*{50mm}n>0\,.
  \end{array}\!\!
  \right\}\!
  \label{e6-koc}
  \end{equation}

\section{Модель схемы приоритетного доступа}
 
  Перейдем к~построению математической модели для узкополосного 
и~широкополосного трафика в~виде системы массового обслуживания 
с~приоритетным обслуживанием узкополосного трафика~--- прерыванием 
скорости передачи широкополосного трафика. Предположим, что входящие 
потоки запросов на передачу узкополосного и~широкополосного трафика 
пуассоновские с~интенсивностями~$\lambda_u$ и~$\lambda_m$ соответственно, 
а~среднее время передачи распределено по экспоненциальному закону 
с~параметрами~$\mu_u$ и~$\mu_m$. Отметим тот факт, что каж\-дая сессия 
узкополосного трафика занимает одну ресурсную единицу, 
а~широкополосного~--- один ресурсный блок (PRB, physical resourse block), состоящий из~$b$ 
ресурсных единиц. В~таком случае максимальное число сессий URLLC равно 
$C\hm= bN$, где $N$~--- максимальное число eMBB-сес\-сий, такое что 
$N\hm= \lfloor C/b\rfloor$. 
  
  Узкополосный трафик имеет приоритет, который реализуется за счет 
прерывания активных сессий широкополосного eMBB-тра\-фи\-ка. Далее 
представлено описание функционирования системы для разных вариантов 
развития событий. При поступлении запроса на передачу широкополосного 
трафика eMBB возможны сле\-ду\-ющие варианты.
  \begin{enumerate}[1.]
\item Если свободного ресурса для обеспечения скорости передачи 
широкополосного трафика достаточно, то запрос на обслуживание 
широкополосного трафика будет принят.
\item Если свободного ресурса для обеспечения скорости передачи 
широкополосного трафика недостаточно, тогда запрос будет заблокирован.
\end{enumerate}

  После поступления запроса на передачу узкополосного трафика URLLC 
возможны сле\-ду\-ющие варианты.
  \begin{enumerate}
\item Если свободного ресурса для обеспечения скорости передачи 
узкополосного трафика достаточно, то запрос будет принят и~передача 
трафика будет начата.
\item Если свободного ресурса для обеспечения скорости передачи 
узкополосного трафика недостаточно, но обслуживается хотя бы одна сессия 
широкополосного трафика, то запрос \mbox{будет} принят за счет прерывания 
обслуживания случайно выбранной сессии широкополосного трафика 
и~передача трафика будет начата.
\item Если свободного ресурса для обеспечения скорости передачи 
узкополосного трафика недостаточно, а сессии широкополосного трафика 
отсутствуют, то запрос будет заблокирован.
\end{enumerate}

  При завершении передачи узкополосного трафика URLLC 
и~широкополосного трафика eMBB сессия будет завершена, и~выделенный для 
нее ресурс будет освобожден в~обоих случаях.
  
  Опишем функционирование системы при помощи случайного процесса 
$\mathbf{X}(t)$ с~состояниями вида $\bm{x}\hm= (n_m, n_u)$, где $n_m$~--- число 
обслуживаемых сессий широкополосного трафика; $n_u$~--- число 
обслуживаемых сессий узкополосного трафика. Тогда пространство состояний 
$\mathbf{X}(t)$ будет иметь вид:
  $$
  \mathbf{X}=\left\{ \left( n_m, n_u\right):\ n_m\geq0\,,\ n_u\geq 0\,,\ 
bn_m+n_u\leq C\right\}.
  $$
  %
  В этом случае возможные переходы для произвольного состояния 
$\bm{\pi}(\bm{x})$, $\bm{x}\hm\in \mathbf{X}$, имеют вид:
  $$
   \begin{array}{l}
  a\left( n_m,n_u\right) \left( n_m+1,n_u\right)=p_m\left( 
n_m+n_u\right)\lambda_m\,,\\[6pt]
\hspace*{44mm} bn_m+n_u+b\leq C\,;\\[6pt]
  a\left( n_m, n_u\right) \left( n_m, n_u+1\right)=p_u\left( 
n_m+n_u\right)\lambda_u\,,\\[6pt]
\hspace*{44mm}  bn_m+n_u+1\leq C\,;\\[6pt]
  a\left( n_m, n_u\right) \left( n_m-1,n_u+1\right) =p_u\left( 
n_m+n_u\right)\lambda_u\,,\\[6pt]
\hspace*{31mm}  bn_m+n_u+1>C\,,\ n_m\geq0\,;\\[6pt]
  a\left( n_m, n_u\right) \left( n_m-1,n_u\right) =n_m\mu_m\,,\ n_m>0\,;\\[6pt]
  a\left( n_m, n_u\right) \left( n_m+1, n_u\right)= n_u\mu_u\,,\ n_u>0\,,
  \end{array}
  $$
где $p_m(\cdot)$ вычисляется по формулам~(\ref{e6-koc}), а~$p_u(\cdot)$ также 
вычисляется по формулам~(\ref{e6-koc}), но уже при $b\hm=1$.
  
  Исходя из описанных правил, записывается мат\-ри\-ца интенсивностей 
переходов, и~могут быть найдены стационарные вероятности 
$\bm{\pi}(\bm{x})$, $\bm{x}\hm\in \mathrm{X}$.

\section{Вероятностные характеристики}

  После нахождения стационарных вероятностей $\bm{\pi}(\bm{x})$, 
$\bm{x}\hm\in \mathbf{X}$, могут быть вычислены сле\-ду\-ющие показатели 
эффективности:
  \begin{itemize}
\item среднее число сессий широкополосного трафика eMBB
$$
\overline{n_m} =\sum\limits^N_{n_m=1} n_m \sum\limits_{n_u=0}^{C-bn_m} 
\bm{\pi} \left( n_m, n_u\right);
$$
\item среднее число сессий узкополосного трафика URLLC 
$$
\overline{n_u} =\sum\limits^C_{n_u=1} n_u \sum\limits_{n_m=0}^{\lfloor (C-
n_u)/b\rfloor}\bm{\pi} \left(n_m, n_u\right);
$$
\item вероятность блокировки (\textit{англ.}\ blocking probability) 
широкополосного трафика eMBB
$$
B_m= \sum\limits^C_{n_u=0} \bm{\pi} \left( \left\lfloor \fr{C-n_u}{b}\right\rfloor, 
n_u\right);
$$
\item вероятность блокировки узкополосного трафика URLLC
$$
B_u=\bm{\pi} (0,C);
$$
\item вероятность прерывания (\textit{англ.}\ interruption probability) 
обслуживания широкополосного трафика eMBB
\begin{multline*}
I=\sum\limits^{N-1}_{n_m=1} \bm{\pi} \left( n_m, C-bn_m\right) \times{}\\
{}\times
\fr{p_u(n_m+n_u)\lambda_u}{p_u(n_m+n_u)\lambda_u+n_m\mu_m+n_u\mu_u}\, 
\fr{1}{n_m}+{}\\
{}+\bm{\pi} (N, C-bN) 
\fr{p_u(n_m+n_u)\lambda_u}{p_u(n_m+n_u)\lambda_u+n_m\mu_m}\,\fr{1}{n_m}
\,.
\end{multline*}
\end{itemize}

\section{Заключение}
  
  Разработана математическая модель для узкополосного URLLC- 
и~широкополосного eMBB-тра\-фи\-ка в~виде РСМО с~приоритетным обслуживанием узкополосного трафика~--- 
прерыванием скорости передачи. Предложены показатели эффективности 
приоритетного доступа~--- среднее число сессий, вероятность блокировки 
и~вероятность прерывания обслуживания. 
  
{\small\frenchspacing
 {%\baselineskip=10.8pt
 %\addcontentsline{toc}{section}{References}
 \begin{thebibliography}{99}
\bibitem{1-koc}
\Au{Alsenwi M., Tran~N.\,H., Bennis~M., Pandey~S.\,R., Bairagi~A.\,K., Hong~C.\,S.} Intelligent 
resource slicing for eMBB and URLLC coexistence in 5G and beyond: A~deep reinforcement 
learning based approach~// IEEE T.~Wirel. Commun., 2021. Vol.~20. Iss.~7.  
P.~4585--4600. doi: 10.1109/TWC.2021.3060514.
\bibitem{2-koc}
\Au{Zhang W., Derakhshani~M., Lambotharan~S.} Stochastic optimization of URLLC--eMBB joint 
scheduling with queuing mechanism~// IEEE Wirel. Commun. Le., 2021. Vol.~10. 
Iss.~4. P.~844--848. doi: 10.1109/LWC.2020.3046628.
\bibitem{3-koc}
\Au{Yin H., Zhang L., Roy~S.} Multiplexing URLLC traffic within eMBB services in 5G NR: Fair 
scheduling~// IEEE T.~Commun., 2021. Vol.~69. Iss.~2. P.~1080--1093. doi: 
10.1109/TCOMM.2020.3035582.
\bibitem{4-koc}
\Au{Hmamouche Y., Benjillali~M., Saoudi~S., Yanikomeroglu~H., Renzo~M.\,D.}  New trends in 
stochastic geometry for wireless networks: A~tutorial and survey~// P.~IEEE, 2021. 
Vol.~109. Iss.~7. P.~1200--1252. doi: 10.1109/JPROC.2021.3061778.

\bibitem{6-koc} %5
\Au{Petrov V., Solomitckii~D., Samuylov~A., Lema~M.\,A., Gapeyenko~M., Moltchanov~D., 
Andreev~S., Naumov~V., Samouylov~K., Dohler~M., Koucheryavy~Y.} Dynamic  
multi-connectivity performance in ultra-dense urban mmWave deployments~// IEEE J.~Sel. 
Area. Comm., 2017. Vol.~35. Iss.~9. P.~2038--2055. doi: 10.1109/JSAC.2017.2720482.

\bibitem{5-koc} %6
\Au{Naumov V., Samouylov~K.} Product-form Markovian queueing systems with multiple 
resources~// Probab. Eng. Inform. Sc., 2021. Vol.~35. Iss.~1. P.~180--188. 
doi: 10.1017/S026996481900024X.

\bibitem{8-koc} %7
\Au{Markova E., Gudkova~I., Ometov~A., Dzantiev~I., Andreev~S., Koucheryavy~Y., 
Samouylov~K.} Flexible spectrum management in a~smart city within licensed shared access 
framework~// IEEE Access, 2017. Vol.~5. P.~22252--22261. doi: 10.1109/ACCESS.2017.2758840.

\bibitem{7-koc} %8
\Au{Маркова Е.\,В., Гольская~А.\,А., Дзантиев~И.\,Л., Гудкова~И.\,А., Шоргин~С.\,Я.} 
Сравнительный анализ показателей эффективности модели беспроводной сети 
межмашинного взаимодействия, работающей в~рамках двух политик разделения 
радиоресурсов~// Информатика и~её применения, 2019. Т.~13. Вып.~1. С.~108--116. doi: 
10.14357/19922264190115.

\bibitem{9-koc}
\Au{Харин П.\,А., Макеева~Е.\,Д., Кочеткова~И.\,А., Ефросинин~Д.\,В., Шоргин~С.\,Я.} 
Система массового обслуживания с~орбитами для анализа совместного обслуживания 
трафика с~малыми задержками URLLC и~широкополосного доступа eMBB в~беспроводных 
сетях пятого поколения~// Информатика и~её применения, 2020. Т.~14. Вып.~4. С.~17--24. 
doi: 10.14357/19922264200403.
\bibitem{10-koc}
\Au{Kushchazli A., Ageeva~A., Kochetkova~I., Kharin~P., Chursin~A., Shorgin~S.} Model of radio 
admission control for URLLC and adaptive bit rate eMBB in 5G network~// CEUR Workshop 
Procee., 2021. Vol.~2946. P.~74--84. 
\bibitem{11-koc}
\Au{Кочеткова И.\,А., Кущазли~А.\,И., Харин~П.\,А., Шоргин~С.\,Я.} Модель для анализа 
приоритетного доступа трафика URLLC при прерывании обслуживания и~снижении 
скорости передачи сессий eMBB в~сети 5G~// Системы и~средства информатики, 2021. Т.~31. 
№\,3. С.~123--134. 
\end{thebibliography}

 }
 }

\end{multicols}

\vspace*{-3pt}

\hfill{\small\textit{Поступила в~редакцию 21.10.21}}

%\vspace*{8pt}

%\pagebreak

\newpage

\vspace*{-28pt}

%\hrule

%\vspace*{2pt}

%\hrule

%\vspace*{6pt}

\def\tit{MODEL FOR ANALYZING PRIORITY ADMISSION CONTROL OF~URLLC AND~eMBB 
COMMUNICATIONS IN~5G NETWORKS AS~A~RESOURCE QUEUING SYSTEM}


\def\titkol{Model for analyzing priority admission control of~URLLC and~eMBB 
communications in~5G networks as~a~resource
% queuing 
system}


\def\aut{I.\,A.~Kochetkova$^{1,2}$, A.\,I.~Kushchazli$^1$, P.\,A.~Kharin$^1$, and~S.\,Ya.~Shorgin$^2$}

\def\autkol{I.\,A.~Kochetkova, A.\,I.~Kushchazli, P.\,A.~Kharin, and~S.\,Ya.~Shorgin}

\titel{\tit}{\aut}{\autkol}{\titkol}

\vspace*{-17pt}


\noindent
$^1$Peoples' Friendship University of Russia (RUDN University), 6~Miklukho-Maklaya Str., 
Moscow 117198, Russian\linebreak
$\hphantom{^1}$Federation
  
\noindent
$^2$Federal Research Center ``Computer Science and Control'' of the Russian Academy of 
Sciences; 44-2~Vavilov\linebreak
$\hphantom{^1}$Str., Moscow 119133, Russian Federation

\def\leftfootline{\small{\textbf{\thepage}
\hfill INFORMATIKA I EE PRIMENENIYA~--- INFORMATICS AND
APPLICATIONS\ \ \ 2021\ \ \ volume~15\ \ \ issue\ 4}
}%
 \def\rightfootline{\small{INFORMATIKA I EE PRIMENENIYA~---
INFORMATICS AND APPLICATIONS\ \ \ 2021\ \ \ volume~15\ \ \ issue\ 4
\hfill \textbf{\thepage}}}

\vspace*{2pt}


  
  \Abste{Ultrareliable and low-latency communication (URLLC) data transmission and enhanced 
mobile broadband (eMBB) are critical scenarios for the fifth generation (5G) networks.
A~5G 
network resource model was formalized using a~triple~--- radio frequency bandwidth, time slot duration, and the 
maximum possible power of the transmitted signal. For the scheme of resource occupation which 
assumes an adaptive change in the signal power and uniform distribution of the time slot between 
devices, the authors show a conditional distribution of the probabilities of receiving a~request for 
traffic transmission. The model with priority access URLLC and interruption of eMBB service, 
considering the specified model of resource occupation, is built in the form of a~resource queuing 
system.}
  
  \KWE{5G; URLLC; eMBB; priority admission control; interruption; resource queuing system}
  
  


\DOI{10.14357/19922264210412}

\vspace*{-22pt}

\Ack

\vspace*{-6pt}


  \noindent
  The paper was supported by the RUDN University Strategic Academic Leadership Program. 
The reported study was funded by RFBR, project number 20-37-70079.



\vspace*{-4pt}

  \begin{multicols}{2}

\renewcommand{\bibname}{\protect\rmfamily References}
%\renewcommand{\bibname}{\large\protect\rm References}

{\small\frenchspacing
 {%\baselineskip=10.8pt
 \addcontentsline{toc}{section}{References}
 \begin{thebibliography}{99}
 
 \vspace*{-4pt}
 
\bibitem{1-koc-1}
  \Aue{Alsenwi, M., N.\,H.~Tran, M.~Bennis, S.\,R.~Pandey, A.\,K.~Bairagi, and C.\,S.~Hong.} 
2021. Intelligent resource slicing for eMBB and URLLC coexistence in 5G and beyond: A~deep 
reinforcement learning based approach. \textit{IEEE T.~Wirel. Commun.}  
20(7):4585--4600. doi: 10.1109/ TWC.2021.3060514.
 \bibitem{2-koc-1}
  \Aue{Zhang, W., M.~Derakhshani, and S.~Lambotharan.} 2021. Stochastic optimization of 
URLLC-eMBB joint scheduling with queuing mechanism. \textit{IEEE Wirel. Commun. 
Le.} 10(4):844--848. doi: 10.1109/LWC.2020.3046628. 
\bibitem{3-koc-1}
  \Aue{Yin, H., L.~Zhang, and S.~Roy.} 2021. Multiplexing URLLC traffic within eMBB 
services in 5G NR: Fair scheduling. \textit{IEEE T.~Commun.} 69(2):1080--1093.
doi: 
10.1109/ TCOMM.2020.3035582.
\bibitem{4-koc-1}
  \Aue{Hmamouche, Y., M.~Benjillali, S.~Saoudi, H.~Yanikomeroglu, and M.\,D.~Renzo.} 2021. 
New trends in stochastic geometry for wireless networks: A~tutorial and survey. 
\textit{P.~IEEE} 109(7):1200--1252. doi: 10.1109/ JPROC.2021.3061778.

\bibitem{6-koc-1} %5
  \Aue{Petrov, V., D.~Solomitckii, A.~Samuylov, M.\,A.~Lema, M.~Gapeyenko, D.~Moltchanov, 
S.~Andreev, V.~Naumov, K.~Samouylov, M.~Dohler, and Y.~Koucheryavy.} 2017. Dynamic 
multi-connectivity performance in ultra-dense urban mmWave deployments. \textit{IEEE 
J.~Sel. Area. Comm.} 35(9):2038--2055. doi: 10.1109/JSAC.2017.2720482.

\bibitem{5-koc-1} %6
  \Aue{Naumov, V., and K.~Samouylov.} 2021. Product-form Markovian queueing systems with 
multiple resources.
 \textit{Probab. Eng. Inform. Sc.} 35(1):180--188.
doi: 10.1017/ S026996481900024X.


\bibitem{8-koc-1} %7
  \Aue{Markova, E., I.~Gudkova, A.~Ometov, I.~Dzantiev, S.~Andreev, Y.~Koucheryavy, and 
K.~Samouylov.} 2017. Flexible spectrum management in a smart city within licensed shared access 
framework. \textit{IEEE Access} 5:22252--22261. doi: 10.1109/ACCESS.2017.2758840.

\bibitem{7-koc-1} %8
  \Aue{Markova, E.\,V., A.\,A.~Golskaia, I.\,L.~Dzantiev, I.\,A.~Gudkova, and S.\,Ya.~Shorgin.} 
2019. Sravnitel'nyy analiz pokazateley effektivnosti modeli besprovodnoy seti mezhmashinnogo 
vzaimodeystviya, rabotayushchey v~ramkakh dvukh politik razdeleniya radioresursov 
[Comparative analysis of performance measures for a wireless machine-to-machine network model 
operating within two radio resource management policies]. \textit{Informatika i~ee  
Primeneniya~--- Inform. Appl.} 13(1):108--116.
doi: 10.14357/ 19922264190115.

\bibitem{9-koc-1}
  \Aue{Kharin, P.\,A., E.\,D.~Makeeva, I.\,A.~Kochetkova, D.\,V.~Ef\-ro\-si\-nin, and 
S.\,Ya.~Shorgin.} 2020. Sistema massovogo obsluzhivaniya s~orbitami dlya analiza so\-vmest\-no\-go 
obsluzhivaniya trafika s~malymi zaderzhkami URLLC i~shirokopolosnogo dostupa eMBB 
v~besprovodnykh setyakh pyatogo pokoleniya [Retrial queuing model for analyzing joint URLLC 
and eMBB transmission in 5G networks].
 \textit{Informatika i~ee Primeneniya~--- Inform. Appl.} 
14(4):17--24. doi: 10.14357/19922264200403.
\bibitem{10-koc-1}
  \Aue{Kushchazli, A., A.~Ageeva, I.~Kochetkova, P.~Kharin, A.~Chursin, and S.~Shorgin.} 
2021. Model of radio admission control for URLLC and adaptive bit rate eMBB in 5G network. 
\textit{CEUR Workshop Procee.} 2946:74--84. 

\pagebreak

\bibitem{11-koc-1}
  \Aue{Kochetkova, I.\,A., A.\,I.~Kushchazli, P.\,A.~Kharin, and S.\,Ya.~Shorgin.} 2021. Model' 
dlya analiza prioritetnogo dostupa trafika URLLC pri preryvanii obsluzhivaniya i~snizhenii skorosti 
peredachi sessiy eMBB v~seti 5G [Model for analyzing priority URLLC transmission with eMBB 
bit rate degradation and interruptions in 5G networks]. \textit{Sistemy i~Sredstva Informatiki~--- 
Systems and Means of Informatics} 31(3):123--134.
 \end{thebibliography}

 }
 }

\end{multicols}

\vspace*{-3pt}

\hfill{\small\textit{Received October 21, 2021}}

%\pagebreak

%\vspace*{-24pt} 
  
  \Contr
  
  \noindent
  \textbf{Kochetkova Irina A.} (b.\ 1985)~--- Candidate of Science (PhD) in physics and 
mathematics, associate professor, Peoples' Friendship University of Russia (RUDN University), 
6~Miklukho-Maklaya Str., Moscow 117198, Russian Federation; senior scientist, Institute of 
Informatics Problems, Federal Research Center ``Computer Science and Control'' of the Russian 
Academy of Sciences, 44-2~Vavilov Str., Moscow 119133, Russian Federation;  
\mbox{gudkova-ia@rudn.ru}
  
  \vspace*{3pt}
  
  \noindent
  \textbf{Kushchazli Anna I.} (b.\ 1997)~--- PhD student, Peoples' Friendship University of 
Russia (RUDN University), 6~Miklukho-Maklaya Str., Moscow 117198, Russian Federation; 
\mbox{aikushch@yandex.ru}
  
  
  \vspace*{3pt}
  
  \noindent
  \textbf{Kharin Petr A.} (b.\ 1993)~--- PhD student, Peoples' Friendship University of Russia 
(RUDN University), 6~Miklukho-Maklaya Str., Moscow 117198, Russian Federation; 
\mbox{pxarin@mail.ru}
  
  \vspace*{3pt}
  
  \noindent
  \textbf{Shorgin Sergey Ya.} (b.\ 1952)~--- Doctor of Science in physics and mathematics, 
professor, principal scientist, Institute of Informatics Problems, Federal Research Center 
``Computer Science and Control'' of the Russian Academy of Sciences, 44-2~Vavilov Str., Moscow 
119333, Russian Federation; \mbox{sshorgin@ipiran.ru}


\label{end\stat}

\renewcommand{\bibname}{\protect\rm Литература} 