\def\stat{kirikov}

\def\tit{КОМПЬЮТЕРНАЯ МОДЕЛЬ СИНЕРГИИ КОЛЛЕКТИВНОГО ПРИНЯТИЯ РЕШЕНИЙ}

\def\titkol{Компьютерная модель синергии коллективного принятия решений}

\def\aut{И.\,А.~Кириков$^1$, А.\,В.~Колесников$^2$, С.\,В.~Листопад$^3$}

\def\autkol{И.\,А.~Кириков, А.\,В.~Колесников, С.\,В.~Листопад}

\titel{\tit}{\aut}{\autkol}{\titkol}

\index{Кириков И.\,А.}
\index{Колесников А.\,В.}
\index{Листопад С.\,В.}
\index{Kirikov I.\,A.}
\index{Kolesnikov A.\,V.}
\index{Listopad S.\,V.}


%{\renewcommand{\thefootnote}{\fnsymbol{footnote}} \footnotetext[1]
%{Исследование выполнено при финансовой поддержке Российского научного фонда (проект 16-11-10227).}}


\renewcommand{\thefootnote}{\arabic{footnote}}
\footnotetext[1]{Калининградский филиал Федерального исследовательского центра <<Информатика и~управление>> 
Российской академии наук, \mbox{baltbipiran@mail.ru}}
\footnotetext[2]{Балтийский федеральный университет им.\ И.~Канта; Калининградский филиал Федерального 
исследовательского центра <<Информатика и~управление>> Российской академии наук, 
\mbox{avkolesnikov@yandex.ru}}
\footnotetext[3]{Калининградский филиал Федерального исследовательского центра <<Информатика 
и~управление>> Российской академии наук, \mbox{ser-list-post@yandex.ru}}

%\vspace*{-18pt}

     
   
   \Abst{Задачи управления сложными социально-техническими системами 
характеризуются множеством НЕ-факторов (в смысле А.\,С.~Нариньяни), затрудняющих 
решение. Традиционно к~подобным задачам привлекаются коллективы экспертов под 
руководством лица, принимающего решения (ЛПР), что позволяет справиться с~разнородностью 
информации и~динамичностью проблемной ситуации. По этой же причине для 
автоматизированного решения сложных задач актуально моделирование коллективных 
процессов в~системах поддержки принятия решений. В~статье рассматриваются вопросы 
моделирования коллективного решения сложных задач и~возникающего при этом эффекта 
синергии, когда интегрированное решение лучше любого решения экспертов, работающих 
индивидуально.}
   
  \KW{малый коллектив экспертов; синергия; гибридная интеллектуальная многоагентная 
система}

\DOI{10.14357/19922264170304} 


\vskip 10pt plus 9pt minus 6pt

\thispagestyle{headings}

\begin{multicols}{2}

\label{st\stat}
  
\section{Введение}

  Традиция решения сложных задач коллективами экспертов под 
руководством ЛПР, имеет давние корни: это 
военные советы, коллегии министерств, всевозможные совещания, планерки, 
консилиумы, аналитические центры и~т.\,д.~[1]. Актуальность коллективного 
решения сложных задач обусловлена преимуществами перед индивидуальной 
работой управленца: повышение качества принимаемых решений за счет учета 
многообразия мнений и~интеграции знаний различных специалистов; 
повышение доверия всех членов коллектива к~результатам его работы 
и~мотивации к~реализации таких решений; соблюдение этических норм. Они во 
многом определяются процессами и~эффектами взаимодействия экспертов, 
изучаемых социальной психологией и~социологией со второй четверти 
XX~в.~[2--4]. Часть этих эффектов: социальная фасилитация, адаптация, 
самоорганизация, синергия~--- положительно влияют на решения сложных задач, 
другие, например социальная ингибация, эффект Рингельмана, групсинк 
и~конформизм,~--- отрицательно. 

Опытные ЛПР обеспечивают условия 
возникновения положительных групповых эффектов и~минимизируют 
отрицательные, перестраивая состав и~структуру системы управления, 
адаптируясь к~изменениям во внешней среде. 
  
  Проблема в~том, что б$\acute{\mbox{о}}$льшая часть современных компьютерных 
технологий~--- среда реализации методов, а не инструментальное средство их 
синтеза. Отсюда аналогично экспертным системам, рассуждающим <<не 
хуже>> одного человека, актуальны информационные технологи для 
управления в~условиях сложных задач не хуже коллектива специалистов. 

В~работе рассматривается компьютерное моделирование эффекта синергии 
методами гиб\-рид\-ных интеллектуальных многоагентных сис\-тем (\mbox{ГиИМАС})~[5], когда 
коллективное решение лучше любого решения экспертов, работающих 
индивидуально, не взаимодействуя. 

\vspace*{-6pt}
  
\section{Понятие синергии}

\vspace*{-2pt}

  Синергия в~широком смысле относится к~<<кооперативным>>, 
коллективным эффектам~--- в~бук\-валь\-ном смысле это эффекты, вызываемые 
сущностями, которые <<работают вместе>> (части, элементы или отдельные 
лица). Этот термин часто ассоциируется с~принципом <<целое больше, чем 
сумма его частей>> метафизики Аристотеля. Однако это узкое и~вводящее 
в~заблуждение понимание многогранной концепции: эффекты, вызываемые 
целым, отличаются от того, что детали могут производить по отдельности.

\begin{table*}\small
\begin{center}
\Caption{Синергия в~научных дисциплинах}
\vspace*{2ex}

\tabcolsep=2pt
\begin{tabular}{|l|l|l|}
\hline
\multicolumn{1}{|c|}{\tabcolsep=0pt\begin{tabular}{c}Научная\\ дисциплина\end{tabular}}&
\multicolumn{1}{c|}{Характерный пример}&
\multicolumn{1}{c|}{Связанные термины}\\
\hline
&Квантовая когерентность&Холизм, упорядочение\\
Физика&Теория хаоса&Эмерджентность, аттрактор, порядок\\
&Фазовые переходы&Кооперативные эффекты, нарушение симметрии\\
\hline
Термодинамика&Диссипативные структуры&Порядок/хаос, низкая энтропия, отрицательная 
энтропия\\
\hline
Биофизика&Гиперциклы&Сотрудничество, взаимодействие, координация, эмерджентность\\
\hline
Химия&
\tabcolsep=0pt\begin{tabular}{l}Молекулярные\\ макроструктуры\end{tabular}&Симметрия, коллективная стабильность, порядок\\
\hline
Биохимия&Супрамолекулы&Взаимодействие, функциональная интеграция, координация\\
\hline
Нейробиология&Синаптическая передача&Кооперативность, пороговые эффекты, 
эмерджентность\\
\hline
\tabcolsep=0pt\begin{tabular}{l}Экология\\ и~поведенческая\\ биология\end{tabular} &
\tabcolsep=0pt\begin{tabular}{l}Коэволюция\\ Симбиоз \\ Социобиология\end{tabular}&
\tabcolsep=0pt\begin{tabular}{l}Взаимность, паразитизм\\ Взаимность, сотрудничество\\
Взаимность, взаимный альтруизм, эмерджентность, сотрудничество\end{tabular}\\
\hline
Экономика&
\tabcolsep=0pt\begin{tabular}{l}Финансовая синергия\\ Производственная синергия\end{tabular}
&\tabcolsep=0pt\begin{tabular}{l}Конгломерат, поглощение, слияние\\
Взаимодействие, координация, эффект от масштаба\end{tabular}\\
\hline
\end{tabular}
\end{center}
\end{table*}
  
  
  Известно много видов ко\-опе\-ра\-тив\-ных/си\-нер\-гети\-че\-ских 
эффектов. Некоторые возникают из линейных или аддитивных явлений. 
Агрегация большо\-го числа однородных сущностей может обеспечить 
преимущество коллективу. Например, колония хищных миксобактерий (\textit{лат.}\ 
Myxococcus xanthus) способна поглотить гораздо более крупную добычу, чем 
одна или несколько бактерий. Колония способна коллективно вырабатывать 
пищеварительные ферменты в~больших концентрациях, которые рассеивались 
бы в~окружающей среде, если бы вырабатывались одной или несколькими 
бактериями~[6]. В~экономике известна синергия: 
\begin{enumerate}[(1)]
\item финансовая, возникающая 
при объединении организаций разного профиля деятельности в~результате 
снижения рисков и,~соответственно, повышения доступности кредитных 
средств; 
\item предпринимательская, обусловленная лучшими инвестиционными 
возможностями для объединенной организации; 
\item расширения от совместного 
использования ресурсов; 
\item рыночная как следствие перекрестного 
субсидирования; 
\item оперативная как результат обмена знаниями~[7].
\end{enumerate}
  
  В настоящее время нет согласованной теории возникновения эффекта 
синергии в~различных сис\-те\-мах. В~табл.~1 сведены известные и~хорошо 
описанные примеры проявления синергетических эффектов с~указанием 
дисциплин, в~которых они изучаются. 


  Выделяются основные особенности эффекта синергии~[7]: системный 
характер; согласуемость с~теорией гештальта М.~Вертхаймеера, В.~Кёлера 
и~К.~Кофки, согласно которой <<целое больше, чем сумма его частей>>; 
невоспроизводимость результатов работы системы как целого при работе всех 
ее элементов по отдельности.

\vspace*{-6pt}

\section{Эффект синергии в~малых коллективах экспертов, 
решающих сложные задачи}

\vspace*{-2pt}

  Преимущества малого коллектива экспертов (МКЭ) ориентированы на 
реализацию идей, не выполнимых при индивидуальном принятии решений 
из-за того, что у~конкретного ЛПР нет воз\-мож\-ности выйти за рамки его 
непосредственной деятельности. Профессиональные обязанности в~МКЭ 
распределяются в~соответствии со способностями и~компетентностями 
исполнителей в~зависимости от сложности деятельности. Синергетический 
эффект в~МКЭ достигается <<групповой компенсацией индивидуальных 
неспособностей>>. Внутрикомандное взаимодействие, партнерство 
и~сотрудничество при решении задач <<повышают эффективность не менее 
чем на~10\%>>~[8], что и~порождает синергетический эффект в~МКЭ, когда 
неумения одного компенсируются навыками и~способностями другого. 
  
  Отсюда актуальны методы выработки единого решения МКЭ на основе 
частных рекомендаций экспертов~\cite{9-kir} (табл.~2): Дельфи~\cite{10-kir, 
11-kir}, анализа иерархий~\cite{12-kir, 13-kir}, мозгового штурма~\cite{11-kir}, 
синектики~\cite{10-kir}, пула мозговой записи~\cite{11-kir} и~др.

\end{multicols}

\begin{table*}\small
\begin{center}
\Caption{Преимущества и~недостатки методов экспертного оценивания}
\vspace*{2ex}

\begin{tabular}{|l|p{65mm}|p{55mm}|}
\hline
\multicolumn{1}{|c|}{Метод}&\multicolumn{1}{c|}{Преимущества}&\multicolumn{1}{c|}{Недостатки}\\
\hline
\multicolumn{1}{|l|}{\raisebox{-16pt}[0pt][0pt]{Дельфи}}&Анонимное обсуждение, исключение груп\-син\-ка; заочное участие 
экспертов&Длительность принятия решений; мнение большинства не всегда правильное; 
многократный пересмотр мнения экспертом \\
\hline
\multicolumn{1}{|l|}{\raisebox{-16pt}[0pt][0pt]{Анализа 
иерархий}}&Высокая скорость выработки решений; качественный (а~не количественный, как в~методе Дельфи) анализ альтернатив&Возможен групсинк; плохо подходит к~условиям 
неопределенности; не использует коллективное творчество для выработки решений\\
\hline
\multicolumn{1}{|l|}{\raisebox{-16pt}[0pt][0pt]{Мозгового штурма}}&Генерирует инновационные варианты, синтезируя идеи; высокая 
скорость выработки решений; ответственность участников коллектива за принятое 
решение&Высокие интеллектуальные усилия участников; исключает <<управление 
мышлением>> \\
\hline
\multicolumn{1}{|l|}{\raisebox{-22pt}[0pt][0pt]{Синектики}}&Генерирует инновационные варианты, синтезируя идеи, используя 
аналогии, метафоры и~т.\,п.; высокая скорость выработки решений; ответственность 
участников коллектива за принятое решение&Высокие интеллектуальные усилия 
участников; требуется обученный коллектив, иначе возрастает его критичность и~снижается 
продуктивность; коллектив решает аналог задачи \\
\hline
\multicolumn{1}{|l|}{\raisebox{-16pt}[0pt][0pt]{Пула мозговой записи}}&Анонимное обсуждение, исключение групсинка; генерирует 
инновационные варианты, синтезируя идеи; высокая скорость выработки решений&Высокие 
интеллектуальные усилия участников\\
\hline
\end{tabular}
\end{center}
\vspace*{6pt}
\end{table*} 
\begin{figure*} %fig1
 \vspace*{1pt}
\begin{center}
\mbox{%
\epsfxsize=161.799mm
\epsfbox{lis-1.eps}
}
\end{center}
\vspace*{-11pt}
\Caption{Схема работы МКЭ методом пула мозговой записи:
\textit{1}~--- участники коллектива; \textit{2}~--- шкала времени; 
\textit{3}~--- действие;
\textit{4}~--- процесс передачи информации между участниками коллектива; 
$t$~--- модельное время}
\end{figure*}

\begin{multicols}{2}
  
  Анализ релевантности методов из табл.~2 компьютерному моделированию 
работы МКЭ позволяет сделать выбор в~пользу пула мозговой атаки, когда 
функционирование МКЭ имитируется последовательностью действий: 
\begin{figure*}[b] %fig2
%\vspace*{126pt}
\begin{center}
\mbox{%
\epsfxsize=162.625mm
\epsfbox{lis-2.eps}
}
\end{center}
\vspace*{-11pt}
\Caption{Функциональная структура ГиИМАС:
\textit{1}~--- взаимоотношения агентов (запросы информации, передача результатов их 
решения);
\textit{2}~--- взаимоотношения агентов (запросы помощи в~решении подзадач); 
\textit{3}~--- взаимодействие (получение сведений из модели, обновление модели) агентов с~
моделью предметной области}
\end{figure*}
(1)~ЛПР 
передает экспертам условия задачи; 
(2)~каждый эксперт на основе собственной 
модели и~оценок вырабатывает вариант ее решения и~передает его ЛПР; 
(3)~ЛПР полученные варианты конфиденциально сообщает всем экспертам, 
кроме источника, для доработки и~улучшения; 
(4)~эксперты улучшают 
варианты и~возвращают их ЛПР: третий и~четвертый этапы повторяются, пока 
каждый из экспертов не обработает хотя бы один раз каждый вариант решения; 
(5)~ЛПР оценивает все варианты решения, в~том числе и~промежуточные, 
и~выбирает лучший на основе собственной модели задачи. 
%\end{enumerate}   

  
  Схематично функционирование МКЭ из ЛПР и~двух экспертов показано на 
рис.~1, где обмен решениями чередуется с~индивидуальной работой экспертов 
по поиску новых или доработке име\-ющих\-ся решений через призму своей 
модели внешнего \mbox{мира}.


  
  В результате сложная задача редуцируется в~подзадачи со специфическими 
для эксперта оценками, при этом эксперт использует один из множества 
методов ее решения. Процессы обмена мнениями касательно решения задачи 
указывают на случайный характер взаимодействия экспертов. 

\vspace*{-6pt}
  
\section{Моделирование эффекта синергии в~гибридных 
интеллектуальных многоагентных системах}

\vspace*{-2pt}

  Моделирование МКЭ и~возникающего в~них эффекта синергии предлагается 
реализовывать с~использованием ГиИМАС, которые представляют собой 
гибридные интеллектуальные системы (ГиИС), практикующие многоагентый 
подход~\cite{14-kir}. Элементы таких ГиИС реализуются в~виде агентов, 
обладающих свойством автономности~\cite{15-kir}. Как и~многоагентные 
системы (МАС), они моделируют взаимодействия автономных агентов между 
собой и~с внешней средой, в~результате которых архитектура системы может 
динамически перестраиваться в~соответствии с~конкретными функциями 
(ролями) агентов и~установившимися отношениями между ними. В~результате 
ГиИМАС сочетают\linebreak в~себе положительные стороны ГиИС и~МАС:\linebreak благодаря 
сочетанию нескольких методов искусственного интеллекта они релевантны 
задачам с~высокой сложностью моделирования~\cite{14-kir}; за счет имитации 
взаимодействия экспертов и~возни\-ка\-ющих при этом коллективных процессов 
они способны менять свою архитектуру для достижения синергетического 
эффекта. 
  
  Для компьютерной реализации модели МКЭ разработана функциональная 
структура ГиИМАС (рис.~2). Она может применяться при проектировании 
ГиИМАС для широкого круга неоднородных задач, поскольку: 
(1)~использована общая многоагентная модель действительности; (2)~перечень 
аген\-тов-ре\-ша\-те\-лей охватывает пять классов методов из шести, 
используемых в~ГиИС~[1]; (3)~порядок взаимодействия агентов определяется 
моделью предметной области.
  

\begin{table*}[b]\small %tabl3
\vspace*{-6pt}
\begin{center}
\Caption{Количественные параметры тестируемых задач}
\vspace*{2ex}

\begin{tabular}{|c|c|c|c|c|c|}
\hline
Задача&
\tabcolsep=0pt\begin{tabular}{c}Количество\\ клиентов\end{tabular}&
\tabcolsep=0pt\begin{tabular}{c}Количество\\ дорог\end{tabular}&
\tabcolsep=0pt\begin{tabular}{c}Количество\\  водителей\end{tabular}&
\tabcolsep=0pt\begin{tabular}{c}Количество\\ грузчиков\end{tabular}&
\tabcolsep=0pt\begin{tabular}{c}Количество\\ автомобилей\end{tabular}\\
\hline
З\_10&10&\hphantom{9}75&3&3&3\\
З\_15&15&240&5&5&5\\
З\_20&20&420&5&5&5\\
З\_25&25&650&9&9&9\\
З\_30&30&377&6&6&6\\
\hline
\end{tabular}
\end{center}
\end{table*}
  
  Рассмотрим назначение ее агентов: 
  \begin{enumerate}[(1)]
  \item интерфейсный агент запрашивает входные данные и~выдает результат; 
  \item агент, принимающий решения (АПР), рассылает агентам поиска 
решения условия задачи, определяет порядок их взаимодействия. Когда 
последние решили задачу, он выбирает альтернативу и~передает 
интерфейсному агенту или запускает новую итерацию, рассылая решение 
остальным агентам поиска;
  \item агенты поиска решения имеют знания о~предметной области 
и~выполняют генерацию и~оценку решений каждый по своему крите\-рию. Для 
решения подзадач тестовой сложной\linebreak транс\-порт\-но-ло\-ги\-сти\-че\-ской 
задачи (СТЛЗ) эти агенты используют муравьиный алгоритм;
  \item агент-посредник отслеживает имена, модели и~возможности 
зарегистрированных агентов интеллектуальных технологий (решателей). 
Агенты обращаются к~нему, чтобы узнать, какой из решателей может помочь 
в~по\-став\-лен\-ной перед ними подзадаче;
  \item решатели в~верхней части рис.~2 вместе 
  с~аген\-том-пре\-об\-ра\-зо\-ва\-те\-лем реализуют гибридную составляющую 
ГиИМАС, комбинируя разнородные знания, и~предоставляют <<услуги>> 
агентам с~использованием моделей и~алгоритмов: алгебраических уравнений 
для описания при\-чин\-но-след\-ст\-вен\-ных связей концептов предметной 
об\-ласти; метода Монте Карло; продукционной экспертной системы 
с~рас\-суж\-де\-ни\-ями в~прямом направлении; нечеткого вывода Мамдани;
  \item модель предметной области~--- семантическая сеть, основа 
взаимодействия агентов, построена по концептуальной модели решаемой 
задачи. Агенты интерпретируют смысл по\-лу\-ча\-емых сообщений на этой модели. 
  \end{enumerate}
  
  Для оценки влияния эффекта синергии на качество решений ГиИМАС 
проведены серии экспериментов, в~которых требовалось решить СТЛЗ, т.\,е.\ 
найти для нескольких транспортных средств совокупность маршрутов, 
оптимальную по четырем критериям: суммарная стоимость; общая 
длительность поездок для всех транспортных средств; вероятность опоздания 
хотя бы к~одному клиенту; надежность (мерой надежности выбрано 
математическое ожидание увеличения стоимости совокупности маршрута)~[1]. 
Учитывались такие стохастические факторы, как вероятность возникновения 
дорожных пробок и~вероятность опоздания к~клиенту, потери от боя груза 
и~др.
  
  Исходные данные: 
  \begin{enumerate}[(1)]
  \item запросы клиентов на доставку грузов (наименование, 
количество товара, временной интервал его доставки); 
  \item сведения о~дорогах 
к~клиентам (протяженность, загруженность, качество); 
  \item паспортные данные 
транспортных средств (расход го\-рю\-че\-сма\-зоч\-ных материалов, 
грузоподъемность и~т.\,п.); 
  \item сведения о~графиках работы и~заработной плате 
персонала (водителей и~грузчиков); 
  \item информация о~грузе (вес, габариты, 
хрупкость и~т.\,п.). 
\end{enumerate}
  
  Выходные данные: совокупность маршрутов доставки грузов (по одному на 
транспортное средство) и~их параметры: стоимость, длительность, надежность 
и~вероятность опоздания, сводный критерий качества маршрута. Для 
тестирования использованы задачи из табл.~3.
  
  Исследовались три архитектуры ГиИМАС, работающие по схеме, 
представленной на рис.~1: с~нейт\-раль\-ны\-ми, сотрудничающими 
и~конкурирующими агентами. В~ГиИМАС с~нейтральными агентами каждый 
из четырех агентов поиска решения минимизирует значение <<своего>> 
критерия оценки решения. В~ГиИМАС с~сотрудничающими агентами все 
четыре аген\-та-по\-иско\-ви\-ка минимизируют\linebreak все четыре критерия оценки решения 
(аналогично АПР). В~ГиИМАС с~конкурирующими агентами\linebreak один агент 
минимизирует стоимость и~максимизирует длительность, второй~--- 
максимизирует \mbox{стоимость} и~минимизирует длительность, \mbox{третий}~--- 
минимизирует вероятность опоздания и~максимизирует надежность, 
а~четвертый~--- максимизирует вероятность опоздания и~минимизирует 
\mbox{надежность}. 

\begin{figure*} %fig3
\vspace*{1pt}
\begin{center}
\mbox{%
\epsfxsize=76.365mm
\epsfbox{lis-3.eps}
}
\end{center}
\vspace*{2pt}
\begin{center}
{\small \begin{tabular}{|l|c|c|c|c|c|}
\hline
\multicolumn{1}{|c|}{Архитектура} &\multicolumn{5}{c|}{Количество клиентов}\\
\cline{2-6}
\multicolumn{1}{|c|}{ГиИМАС} & 10&15&20&25&30\\
\hline
\textit{1}~--- конкуренция& 0,9449& 0,9528& 0,9348& 0,8135& 0,4687\\
\textit{2}~--- нейтралитет& 0,9669& 0,9802 & 0,9796& 0,9699& 0,9504\\
\textit{3}~--- сотрудничество& 0,9661& 0,9800& 0,9795& 0,9639& 0,8592\\
\textit{4}~--- без взаимодействия& 0,9439& 0,9422& 0,9411& 0,8248& 0,7370\\
\hline
\end{tabular}}
\end{center}
\Caption{Среднее значение сводного критерия качества маршрута}
\vspace*{-3pt}
\end{figure*}


  Качество тестовых решений оценивалось по объективным показателям 
и~субъективно экспертами. Для пяти задач и~каждой архитектуры \mbox{ГиИМАС} 
проведено по~100~вычислительных экспериментов. По всем задачам 
и~архитектурам \mbox{ГиИМАС}, а~также для архитектуры без взаимодействия  
(аген\-ты-по\-иско\-ви\-ки не обмениваются индивидуальными решениями) 
построены графические зависимости чис\-ла ситуаций, когда коллективное 
решение лучше любого индивидуального, среднего значения сводного 
критерия качества маршрута (рис.~3), средних значений сто\-имости, 
длительности, на\-деж\-ности, ве\-ро\-ят\-ности опоздания для маршрутов, от чис\-ла 
клиентов, анализ которых показал высокое качество марш\-ру\-тов, 
рекомендуемых ГиИМАС. 



  Как видно из рис.~3, в~большинстве случаев любая из архитектур \mbox{ГиИМАС}
предлагает более качественные решения, чем \mbox{ГиИМАС} без взаимодействия 
агентов, т.\,е.\ проявляется эффект синергии. Качество принимаемых решений 
\mbox{ГиИМАС} с~нейтральными агентами выше, чем \mbox{ГиИМАС} других архитектур. 
Это прямое следствие того, что в~\mbox{ГиИМАС} с~нейтральными агентами 
вероятность возникновения синергетического эффекта выше, но чем меньше 
размерность задачи, тем меньше его влияние на качество решения.
  
  Для задач с~25 и~30~клиентами эффективность \mbox{ГиИМАС}
с~конкурирующими агентами резко снижается и~она демонстрирует 
результаты хуже, чем  
\mbox{ГиИМАС} без взаимодействия, т.\,е.\ возникает дисергия, когда коллективное 
решение не лучше решений индивидуальных агентов. Очевидно, этот эффект 
обусловлен невозможностью кон\-ку\-ри\-ру\-ющих агентов <<договориться>> на 
задачах с~высокой комбинаторной сложностью. 
  
  Таким образом, при правильной организации взаимодействия в~ГиИМАС 
эффект синергии повы\-шает качество принимаемых решений по сравнению 
с~ГиИМАС, в~которой он не моделируется. По итогам тестовой эксплуатации 
программного продукта ТРАНСМАР, реализующего ГиИМАС, на двух 
объектах средняя суммарная себестоимость доставки грузов в~день сократилась 
на~7,2\%, средняя суммарная длительность доставки в~день~--- на~12,13\%, 
среднее время построения маршрутов в~день уменьшилось на~23,14\%.

\section{Заключение}

  В работе рассмотрено понятие эффекта синергии и~один из подходов для его 
достижения при решении сложных задач МКЭ~--- организация рассуждений 
методом пула мозговой записи. Данный метод положен в~основу компьютерной 
модели МКЭ~--- ГиИМАС, отображающей и~ком\-би\-ни\-ру\-ющей на компьютере 
разнообразие знаний экспертов о проблемной среде, что имитирует 
полиязыковой характер сложных задач, с~одной стороны, и~социальный, 
коллективный характер решений, когда моделируется взаимодействие 
экспертов друг с~другом и~с~ЛПР~--- с~другой стороны. 

Результаты 
лабораторных экспериментов с~сис\-те\-мой, а также практического использования 
программного продукта ТРАНСМАР, реализующего модель \mbox{ГиИМАС}, 
показали, что моделирование эффекта синергии в~\mbox{ГиИМАС} повышает качество 
принимаемых решений по сравнению с~\mbox{ГиИМАС}, в~которой агенты поиска 
решений не обмениваются решениями и~эффект синергии отсутствует.
  
{\small\frenchspacing
 {%\baselineskip=10.8pt
 \addcontentsline{toc}{section}{References}
 \begin{thebibliography}{99}
  \bibitem{1-kir}
  \Au{Трахтенгерц Э.\,А., Степин~Ю.\,П., Андреев~А.\,Ф.} Компьютерные 
методы поддержки принятия управленческих решений в~нефтегазовой 
промышленности.~--- М.: СИНТЕГ, 2005. 592~с.
\bibitem{2-kir}
\Au{Freud S.} Group psychology and the analysis of the ego.~--- The international 
psycho-analytical library ser., 1922. Vol.~6. P.~1--134. 
  \bibitem{3-kir}
  \Au{Lewin K.} Resolving social conflicts: Selected papers on group dynamics.~--- 
New York, NY, USA: Harper\,\&\,Row, 1948. 230~p.
  \bibitem{4-kir}
  \Au{Кириков И.\,А., Колесников~А.\,В., Листопад~С.\,В.} Моделирование 
систем поддержки принятия решений синергетическим искусственным 
интеллектом~// Информатика и~её применения, 2013. Т.~7. Вып.~3. С.~62--69.
  \bibitem{5-kir}
  \Au{Колесников А.\,В.} Гибридные интеллектуальные сис\-те\-мы. Теория 
и~технология разработки.~--- СПб.: \mbox{СПбГТУ}, 2001. 711~с.
  \bibitem{6-kir}
  \Au{Bonner J.\,T.} The evolution of complexity.~--- Princeton, NJ, USA: 
Princeton University Press, 1988. 272~p.
  \bibitem{7-kir}
  \Au{Benecke G., Schurink~W., Roodt~G.} Towards a substantive theory of 
synergy~// SA J.~Human Resource Management, 2007. Vol.~5. No.\,2. P.~9--19.
  \bibitem{8-kir}
  \Au{Зимняя И.\,А.} Педагогическая психология.~--- Ростов-на-Дону: Феникс, 
1997. 480~с.
  \bibitem{9-kir}
  \Au{Орлов А.\,И.} Теория принятия решений.~--- М.: 
Экзамен, 2005. 656~с.
  
  \bibitem{11-kir}
  \Au{Сладкевич В.\,П., Чернявский~А.\,Д.} Современный менеджмент 
(в~схемах).~--- Киев: МАУП, 2003. 
152~с.
\bibitem{10-kir}
  \Au{Колпаков В.\,М.} Теория и~практика принятия управ\-лен\-че\-ских 
решений.~--- Киев: МАУП, 2004. 504~с.
  \bibitem{12-kir}
  \Au{Саати Т.} Принятие решений. Метод анализа иерархий~/ Пер. с~англ.~--- 
  М.: Радио и~связь, 1993. 278~с. (\Au{Saaty~T.\,L.} The analytic hierarchy 
process.~--- New York, NY, USA: McGraw-Hill, 1980. 296~p.)
\bibitem{13-kir}
\Au{Сухарев М.\,Г.} Методы прогнозирования.~--- М.: РГУ 
нефти и~газа, 2009. 208~с.
  \bibitem{14-kir}
  \Au{Колесников А.\,В., Кириков~И.\,А., Листопад~С.\,В., Румовская~С.\,Б., 
Доманицкий~А.\,А.} Решение сложных задач коммивояжера методами 
функциональных гибридных интеллектуальных сис\-тем~/ Под ред. 
А.\,В.~Колесникова.~--- М.: ИПИ РАН, 2011. 295~с.
  \bibitem{15-kir}
  \Au{Тарасов В.\,Б.} От многоагентных систем к~интеллектуальным 
организациям: философия, психология, информатика.~--- М.: Эдиториал УРСС, 
2002. 352~с.

 \end{thebibliography}

 }
 }

\end{multicols}

\vspace*{-3pt}

\hfill{\small\textit{Поступила в~редакцию 16.07.17}}

\vspace*{8pt}

%\newpage

%\vspace*{-24pt}

\hrule

\vspace*{2pt}

\hrule

%\vspace*{8pt}


\def\tit{COMPUTER MODEL OF~SYNERGY OF~TEAM DECISION-MAKING}

\def\titkol{Computer model of~synergy of~team decision-making}

\def\aut{I.\,A.~Kirikov$^1$, A.\,V.~Kolesnikov$^{1,2}$, and~S.\,V.~Listopad$^1$}

\def\autkol{I.\,A.~Kirikov, A.\,V.~Kolesnikov, and~S.\,V.~Listopad}

\titel{\tit}{\aut}{\autkol}{\titkol}

\vspace*{-9pt}


 \noindent
  $^1$Kaliningrad Branch of the Federal Research Center 
  ``Computer Science and Control'' of the 
Russian Academy\linebreak
$\hphantom{^1}$of Sciences, 5 Gostinaya Str., Kaliningrad 236000, Russian Federation
   
   \noindent
   $^2$Immanuel Kant Baltic Federal University, 14~A.~Nevskogo Str., Kaliningrad 
236041, Russian Federation



\def\leftfootline{\small{\textbf{\thepage}
\hfill INFORMATIKA I EE PRIMENENIYA~--- INFORMATICS AND
APPLICATIONS\ \ \ 2017\ \ \ volume~11\ \ \ issue\ 3}
}%
 \def\rightfootline{\small{INFORMATIKA I EE PRIMENENIYA~---
INFORMATICS AND APPLICATIONS\ \ \ 2017\ \ \ volume~11\ \ \ issue\ 3
\hfill \textbf{\thepage}}}

\vspace*{3pt}
  
  
    \Abste{The problems of the practice of complex sociotechnical systems management are characterized 
by a variety of NOT-factors (following the terminology suggested by A.\,S.~Narinyani) that hamper their 
solution. Traditionally, teams of experts under the leadership of a~decision-maker are involved in such 
problems to deal with the heterogeneity of information and the dynamic nature of the problem. For the same 
reason, the modeling of team processes in decision support systems is important for the automated solving of 
complex problems. The article deals with the issues of modeling the process of collective solving of complex 
problems and the resulting synergy effect, when an integrated solution is better than any decision of experts 
working individually.}
   
   \KWE{small team of experts; synergy; hybrid intelligent multiagent system}
   

\DOI{10.14357/19922264170304} 

%\vspace*{-18pt}

%\Ack


%\vspace*{3pt}

  \begin{multicols}{2}

\renewcommand{\bibname}{\protect\rmfamily References}
%\renewcommand{\bibname}{\large\protect\rm References}

{\small\frenchspacing
 {%\baselineskip=10.8pt
 \addcontentsline{toc}{section}{References}
 \begin{thebibliography}{99}
  \bibitem{1-kir-1}
  \Aue{Trakhtengerts, E.\,A, Yu.\,P.~Stepin, and A.\,F.~Andreev.} 2005. 
\textit{Komp'yuternye metody podderzhki prinyatiya uprav\-len\-che\-skikh resheniy 
v~neftegazovoy promyshlennosti} [Computer methods for management decision 
making support in the oil and gas industry]. Moscow: SINTEG. 592~p.
  \bibitem{2-kir-1}
  \Aue{Freud, S.} 1922. \textit{Group psychology and the analysis of the ego}. 
  {The 
international psycho-analytical library ser.} 6:1--134.  
  \bibitem{3-kir-1}
  \Aue{Lewin, K.} 1948. \textit{Resolving social conflicts: Selected papers on 
group dynamics}. New York, NY: Harper\,\&\,Row. 230~p.
  \bibitem{4-kir-1}
  \Aue{Kirikov, I.\,A., A.\,V.~Kolesnikov, and S.\,V.~Listopad.} 2013. 
Modelirovanie sistem podderzhki prinyatiya resheniy si\-ner\-ge\-ti\-che\-skim 
iskusstvennym intellektom [Decision support systems modeling with synergetic 
artificial intelligence]. \textit{Informatika i~ee Primeneniya~--- Inform. Appl.} 
7(3):62--69.
  \bibitem{5-kir-1}
  \Aue{Kolesnikov, A.\,V.} 2001. \textit{Gibridnye intellektual'nye sistemy. Teoriya 
i~tekhnologiya razrabotki} [Hybrid intelligent systems: Theory and technology of 
development]. St.\ Petersburg: SPbGTU Publ. 711~p.
  \bibitem{6-kir-1}
  \Aue{Bonner, J.\,T.} 1988. \textit{The evolution of complexity}. Princeton, NJ: 
Princeton University Press. 272~p.
  \bibitem{7-kir-1}
  \Aue{Benecke, G., W.~Schurink, and G.~Roodt.} 2007. Towards a substantive 
theory of synergy. \textit{SA J.~Human Resource Management} 5(2):9--19.
  \bibitem{8-kir-1}
  \Aue{Zimnyaya, I.\,A.} 1997. \textit{Pedagogicheskaya psikhologiya} 
[Pedagogical psychology]. Rostov-on-Don: Phoenix. 480~p.
  \bibitem{9-kir-1}
  \Aue{Orlov, A.\,I.} 2005. \textit{Teoriya prinyatiya resheniy} 
[Decision theory]. Moscow: Examen Publ. 656~p.
  
  \bibitem{11-kir-1}
  \Aue{Sladkevich, V.\,P., and A.\,D.~Chernyavskiy}. 2003. \textit{Sovremennyy 
menedzhment (v~skhemakh)} [Modern 
management (in diagrams)]. Kiev: 
Interregional Academy of Personnel Management. 152~p.

\bibitem{10-kir-1}
  \Aue{Kolpakov, V.\,M.} 2004. \textit{Teoriya i~praktika prinyatiya 
upravlencheskikh resheniy} [Theory and practice of management decision-making]. 
Kiev: Interregional Academy of Personnel Management. 504~p.
  \bibitem{12-kir-1}
  \Aue{Saaty, T.\,L.} 1980. \textit{The analytic hierarchy process}. New York, NY: 
McGraw-Hill. 296~p.
  \bibitem{13-kir-1}
  \Aue{Sukharev, M.\,G.} 2009. \textit{Metody prognozirovaniya} 
  [Forecasting methods]. Moscow: Russian State University of Oil 
and Gas. 208~p.
  \bibitem{14-kir-1}
  \Aue{Kolesnikov, A.\,V., I.\,A.~Kirikov, S.\,V.~Listopad, S.\,B.~Rumovskaya, 
and A.\,A.~Domanitskiy}. 2011. \textit{Reshenie slozhnykh zadach kommivoyazhera 
metodami funktsional'nykh gibridnykh intellektual'nykh system} [Complex travelling 
salesman problems solving by the methods of the functional hybrid intelligent 
systems]. Moscow: IPI RAN. 295~p.
  \bibitem{15-kir-1}
  \Aue{Tarasov, V.\,B.} 2002. Ot mnogoagentnykh sistem k~intellektual'nym 
organizatsiyam: Filosofiya, psikhologiya, informatika [From multiagent systems to 
intelligent organizations: Philosophy, psychology, and informatics]. Moscow: 
Editorial URSS. 352~p.
\end{thebibliography}

 }
 }

\end{multicols}

\vspace*{-3pt}

\hfill{\small\textit{Received July 16, 2017}}
  
  \Contr
  
  \noindent
  \textbf{Kirikov Igor A.} (b.\ 1955)~--- Candidate of  Science (PhD) in 
technology; director, Kaliningrad Branch of the Federal Research Center ``Computer 
Science and Control'' of the Russian Academy of Sciences, 5~Gostinaya Str., 
Kaliningrad 236000, Russian Federation; \mbox{baltbipiran@mail.ru}
  
  \vspace*{3pt}
  
  \noindent
  \textbf{Kolesnikov Alexander V.} (b.\ 1948)~--- Doctor of Science in 
technology; professor, Department of Telecommunications, Immanuel Kant Baltic 
Federal University, 14~A.~Nevskogo Str., Kaliningrad 236041, Russian Federation; 
senior scientist, Kaliningrad Branch of the Federal Research Center ``Computer 
Science and Control'' of the Russian Academy of Sciences, 5~Gostinaya Str., 
Kaliningrad 236000, Russian Federation; \mbox{avkolesnikov@yandex.ru} 
  
  \vspace*{3pt}
  
  \noindent
  \textbf{Listopad Sergey V.} (b.\ 1984)~--- Candidate of  Science (PhD) in 
technology; senior scientist, Kaliningrad Branch of the Federal Research Center 
``Computer Science and Control'' of the Russian Academy of Sciences, 5~Gostinaya 
Str., Kaliningrad 236000, Russian Federation, \mbox{ser-list-post@yandex.ru}
   
\label{end\stat}


\renewcommand{\bibname}{\protect\rm Литература} 