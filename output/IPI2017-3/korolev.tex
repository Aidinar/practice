\def\stat{korolev}

\def\tit{АНАЛОГИ ТЕОРЕМЫ ГЛЕЗЕРА ДЛЯ ОТРИЦАТЕЛЬНЫХ БИНОМИАЛЬНЫХ
И~ОБОБЩЕННЫХ ГАММА-РАСПРЕДЕЛЕНИЙ И НЕКОТОРЫЕ ИХ
ПРИЛОЖЕНИЯ$^*$}

\def\titkol{Аналоги теоремы Глезера для отрицательных биномиальных
и~обобщенных гамма-распределений} % и~некоторые их приложения}

\def\aut{В.\,Ю.~Королев$^1$}

\def\autkol{В.\,Ю.~Королев}

\titel{\tit}{\aut}{\autkol}{\titkol}

\index{Королев В.\,Ю.}
\index{Korolev V.\,Yu.}


{\renewcommand{\thefootnote}{\fnsymbol{footnote}} \footnotetext[1]
{Работа выполнена при частичной поддержке Программы
Президиума РАН №\,I.33П (проект 063-2016-0015) и~Российского фонда
фундаментальных исследований (проекты 15-07-04040 и~17-07-00717).}}


\renewcommand{\thefootnote}{\arabic{footnote}}
\footnotetext[1]{Факультет вычислительной математики и~кибернетики
Московского государственного университета имени М.\,В.~Ломоносова;
Институт проблем информатики Федерального исследовательского центра
<<Информатика и~управление>> Российской академии наук; Университет
Дианьзи города Ханчжоу, Китай, \mbox{vkorolev@cs.msu.su}}

%\vspace*{-18pt}

\Abst{Доказано, что отрицательные биномиальные
распределения с~параметром формы, меньшим единицы, являются
смешанными геометрическими распределениями. Смешивающее
распределение выписывается в~явном виде. Тем самым на дискретный
случай перенесен аналогичный результат Л.~Глезера, устанавливающий,
что гам\-ма-рас\-пре\-де\-ле\-ния с~параметром формы, меньшим единицы,
являются смешанными показательными законами. Также доказан аналог
теоремы Глезера для обобщенных гам\-ма-рас\-пре\-де\-ле\-ний
(GG-распределений, Generalized Gamma distributions). Для смешанных
биномиальных распределений, связанных с~отрицательными биномиальными
распределениями с~параметром формы, меньшим единицы, рассмотрен
случай малой вероятности успеха и~доказан аналог теоремы Пуассона. 
С~по\-мощью представления отрицательных биномиальных распределений 
в~виде смешанных гео\-мет\-ри\-че\-ских законов доказаны предельные теоремы
для отрицательных биномиальных случайных сумм независимых одинаково
распределенных случайных величин (с.в.), в~част\-ности аналоги закона больших
чисел и~центральной предельной теоремы. Рассмотрены случаи как
легких, так и~тяжелых хвостов. Получены выражения для моментов
предельных распределений. Полученные альтернативные эквивалентные
представления предельных законов в~виде смесей позволяют получить
лучшее понимание механизмов, формирующих смешанные вероятностные
(байесовские) модели.}

\KW{отрицательное биномиальное распределение;
смешанное геометрическое распределение; обобщенное
гам\-ма-рас\-пре\-де\-ле\-ние; устойчивое распределение; распределение
Лапласа; распределение Мит\-таг--Леф\-фле\-ра; распределение Линника;
смешанное биномиальное распределение; теорема Пуассона; случайная
сумма; закон больших чисел; центральная предельная теорема}

\DOI{10.14357/19922264170301} 


\vskip 10pt plus 9pt minus 6pt

\thispagestyle{headings}

\begin{multicols}{2}

\label{st\stat}

\section{Введение}

\subsection{Мотивация}

В большинстве работ, посвященных статистическому анализу
метеорологических данных, используемые математические модели
наблюдаемых статистических закономерностей довольно далеки от того,
чтобы считаться адекватными. В~частности, принято считать, что
продолжительность периода выпадения осадков, измеренная в~сутках (т.\,е.\ 
чис\-ло последовательных дождливых дней), подчиняется
геометрическому распределению вероятностей (см., например,~\cite{Zolina2013}), хотя согласие такой модели с~реальными
данными очень далеко от допусти\-мого. 

Возможно, данный предрассудок
основан на общепринятой интерпретации гео\-мет\-ри\-че\-ско\-го распределения
в терминах испытаний Бернулли как распределения числа
последовательных дождливых дней (<<успехов>>) до первого дня без
осадков (<<неудачи>>). Но схема испытаний Бернулли предполагает, что
испытания независимы, тогда как результаты статистического анализа
метеорологических данных, зарегистрированных в~разных географических
точках, демонстрируют, что последовательность дождливых и~сухих дней
не только не обладает свойством независимости, но даже не является
марковской. 

Таким образом, классическая схема испытаний Бернулли
абсолютно не является адекватной при математическом моделировании
метеорологических явлений.

Оказалось, что статистические закономерности поведения некоторых
характеристик процесса выпадения осадков (в частности,
продолжительность дождливых периодов) очень хорошо описываются
отрицательным биномиальным распределением с~параметром формы,
меньшим единицы. Так, в~работе~\cite{Gulev} на примере данных,
зарегистрированных в~таких разных по своим климатическим па\-ра\-мет\-рам
пунктах, как Потсдам и~Элиста, было показано, что флуктуации
продолжительности дождливых периодов, измеренной в~сутках, с~очень
высокой надежностью описывается отрицательным биномиальным
распределением с~параметром формы $r\hm\approx 0{,}85$. В~той же работе
была предложена схематическая интерпретация этого феномена с~по\-мощью
известного свойства отрицательного биномиального распределения,
которое является смешанным пуассоновским распределением, где
смешивающим служит гам\-ма-рас\-пре\-де\-ле\-ние. Как известно (см., 
например,~\cite{Kingman1993, KorolevBeningShorgin2011}), пуассоновское
распределение является наилучшей вероятностной моделью для
дискретных хаотических стохастических\linebreak
 процессов, адекватность
которой обусловлена универсальным принципом неубывания энтропии\linebreak 
в~за\-мк\-ну\-тых системах. Тогда можно считать, что смешивающее
гам\-ма-рас\-пре\-де\-ле\-ние <<случайного>> парамет\-ра пуассоновского
распределения в~отрицательной биномиальной модели~\cite{Gulev}
описывает статистические закономерности случайных изменений внешних
факторов.

В данной работе предпринимается попытка конкретизировать это
возможное объяснение адек\-ват\-ности отрицательной биномиальной модели.
С~этой целью предлагается использовать понятие смешанного
геометрического распределения, введенное в~работе~\cite{Korolev2016TVP} 
(см.\ также~\cite{KorolevPoisson, Korolev2016}). Ниже будет показано, что 
любое отрицательное
биномиальное распределение с~параметром формы, меньшим единицы,
является смешанным гео\-мет\-ри\-че\-ским распределением. Тем самым будет
доказан <<дискретный>> аналог теоремы Л.~Глезера~\cite{Gleser1989},
устанавливающей возможность пред\-став\-ле\-ния гам\-ма-рас\-пре\-де\-ле\-ния 
с~параметром формы, меньшим\linebreak
 единицы, в~виде смешанного показательного
распределения. Указанное представление отрицательного биномиального
распределения в~виде смешанного гео\-мет\-ри\-че\-ско\-го можно
проинтерпретировать в~терминах испытаний Бернулли со случайной
вероятностью успеха. Сначала в~результате <<предварительного>>
эксперимента определяется значение вероятности успеха, а~потом
рассматриваемая с.в.\ определяется как число успехов до
первой неудачи в~последовательности испытаний Бернулли с~так
определенной случайной ве\-ро\-ят\-ностью успеха. Такая интерпретация
позволяет привести дополнительные аргументы, объясняющие
адекватность отрицательной биномиальной модели для распределения
продолжительности дождливых периодов, а~именно: можно предположить,
что последовательность дождливых и~сухих дней не является
независимой, но является {\it условно независимой} при фиксированном
значении с.в., определяющей значение
ве\-ро\-ят\-ности успеха, которое меняется от одного дождливого периода 
к~другому (например, в~зависимости от времени года) и~определяется
факторами, внешними по отношению к~исследуемой локальной системе.

\subsection{Структура статьи}

Статья организована следующим образом. 

В~подразд.~1.3 введены
понятия, используемые в~дальнейшем, и~приведены необходимые
обозначения. 

Основные результаты сформулированы и~доказаны в~разд.~2. 
Здесь доказано, что отрицательные биномиальные распределения с~параметром формы, 
меньшим единицы, являются смешанными
геометрическими распределениями (теорема~1). Смешивающее
распределение выписывается в~явном виде. Тем самым на дискретный
случай перенесен аналогичный результат Л.~Глезера, устанавливающий,
что гам\-ма-рас\-пре\-де\-ле\-ния с~параметром формы, меньшим единицы,
являются смешанными показательными законами. Изучена связь между
смешивающими распределениями в~теореме Глезера и~теореме~1. Здесь же
теорема Глезера распространена на GG-рас\-пре\-де\-ле\-ния. 

В~разд.~3 для смешанных биномиальных распределений, связанных 
с~отрицательными биномиальными распределениями с~параметром формы,
меньшим единицы, рассмотрен случай малой вероятности успеха 
и~доказан аналог теоремы Пуассона. 

Раздел~4 посвящен предельным
теоремам для отрицательных биномиальных сумм. С~точки зрения
моделирования статистических закономерностей процессов выпадения
осадков полученные здесь предельные распределения могут служить
асимптотической аппроксимацией распределения суммарного объема
осадков, выпавших в~течение одного довольно <<продолжительного>>
дождливого периода. С~формальной точки зрения приведенные здесь
результаты по сути являются теоремами переноса, которые можно
сформулировать и~доказать традиционными способами (см., 
например,~\cite{GnedenkoKorolev1996}), однако приводимые здесь формулировки 
и~доказательства, основанные на представлении отрицательных
биномиальных распределений в~виде смешанных геометрических законов,
могут дать дополнительное\linebreak
 понимание эффектов, возникающих при
исследовании схемы испытаний Бернулли со случайной вероят\-ностью
успехов. Более того, получены альтернативные эквивалентные
представления предельных законов в~виде смесей, позволяющие получить
лучшее понимание механизмов, фор\-ми\-ру\-ющих смешанные вероятностные
модели, и~обеспечить более эффективное применение байесовских
методов статистического анализа реальных данных за счет более
адекватного выбора априорных распределений.

\subsection{Определения и~обозначения}


\textit{Отрицательным биномиальным распределением} с~параметрами $r\hm>0$
и $p\hm\in(0,1)$ называется набор положительных чисел
$$
q_k=\fr{\Gamma(r+k)}{k!\Gamma(r)}\, p^r(1-p)^k\,,\enskip
k=0,1,2,\ldots,
$$
где $\Gamma(r)$~--- эйлерова гам\-ма-функ\-ция,
$$
\Gamma(r)=\int\limits_{0}^{\infty}x^{r-1}e^{-x}\,dx\,,\enskip r>0\,.
$$
Несложно убедиться, что $\sum_{k=0}^{\infty}q_k\hm=1$, так что набор
чисел $\{q_k\}_{k=0}^{\infty}$ задает распределение вероятностей на
$\{0\}\bigcup\mathbb{N}$. Частным случаем отрицательного
биномиального распределения, соответствующим значению $r\hm=1$,
является {\it геометрическое распределение}.

В дальнейшем для удобства изложение будет вестись в~терминах с.в.\ 
с~соответствующими распределениями. При этом будет предполагаться, что
все с.в.\ заданы на одном вероятностном пространстве~$(\Omega,\,\mathfrak{F},\,{\sf P})$.

В статье используются стандартные обозначения. Символы~$\eqd$ 
и~$\Longrightarrow$ обозначают совпадение распределений и~сходимость
по распределению. Целую и~дробную часть вещественного числа~$z$
будем соответственно обозначать~$[z]$ и~$\{z\}$.

Случайная величина, имеющая гам\-ма-рас\-пре\-де\-ле\-ние с~параметром формы $r\hm>0$ 
и~параметром масштаба $\lambda\hm>0$, будет обозначаться~$G_{r,\lambda}$,
$$
{\sf P}(G_{r,\lambda}<x)=\int\limits_{0}^{x}g(z;r,\lambda)\,dz\,,\enskip x\ge0\,,
$$
где
$$
g(z;r,\lambda)=\fr{\lambda^r}{\Gamma(r)}\,z^{r-1}e^{-\lambda z}\,,\enskip
z\ge0\,.
$$
В принятых обозначениях $G_{1,1}$~--- с.в.\ со стандартным
показательным распределением: ${\sf P}
(G_{1,1}\hm<x)=\left[1-e^{-x}\right]{\bf 1}(x\hm\ge0)$ (здесь и~далее ${\bf 1}(A)$~--- 
это индикаторная функция множества~$A$).

Гамма-распределение является частным представителем класса
GG-рас\-пре\-де\-ле\-ний, которые были впервые описаны как единое семейство 
в~1962~г.\ в~работе~\cite{Stacy1962} в~качестве семейства веро\-ятностных
моделей, включающего в~себя одновременно гам\-ма-рас\-пре\-де\-ле\-ния 
и~распределения \mbox{Вейбулла}. Обобщенным гам\-ма-рас\-пре\-де\-ле\-ни\-ем называется
распределение, определяемое плот\-ностью вероятностей вида
$$
g^*(x;r,\gamma,\lambda)=\fr{|\gamma|\lambda^r}{\Gamma(r)}\,x^{\gamma
r-1}e^{-\lambda x^{\gamma}}\,,\enskip x\ge0\,,
$$
где $\gamma\in\mathbb{R}$, $\lambda\hm>0$, $r\hm>0$. Более подробное
описание свойств GG-рас\-пре\-де\-ле\-ний см.\ в~\cite{Stacy1962, KorolevZaks2013}. 
В~дальнейшем, как правило, нас будут интересовать
GG-рас\-пре\-де\-ле\-ния с~$\gamma\hm\in(0,1]$. Случайная величина с~плот\-ностью
$g^*(x;r,\gamma,\lambda)$ будет обозначаться~$G^*_{r,\gamma,\lambda}$.

Для частного случая с.в.\ с~GG-рас\-пре\-де\-ле\-нием~--- распределением
Вей\-бул\-ла--Гне\-ден\-ко, определяемым плотностью $g^*(x;1,\gamma,1)$ 
и~функцией распреде\-ления (ф.р.)\ $\left[1-e^{-x^{\gamma}}\right]{\bf 1}(x\hm\ge0)$, будет
использоваться особое обозначение~$W_{\gamma}$. Таким образом,
$G^*_{1,1,1}\eqd G_{1,1}\eqd W_1$.

Случайная величина со стандартной нормальной ф.р.~$\Phi(x)$ будет обозначаться~$X$,
$$
{\sf P}(X<x)=\Phi(x)=\fr{1}{\sqrt{2\pi}}\int\limits_{-\infty}^{x}e^{-z^2/2}\,dz\,,\enskip
x\in\mathbb{R}\,.
$$
Случайную величину с~ф.р.\ Лапласа~$F^{\Lambda}(x)$, соответствующей
плотности $\ell(x)=(1/2)e^{-|x|}$, $x\hm\in\mathbb{R}$,
будем обозначать~$\Lambda$.

Функция распределения строго устойчивого распределения с~характеристическим
показателем~$\alpha$ и~параметром формы~$\theta$, определяемого
характеристической функцией (х.ф.)
$$
\mathfrak{f}_{\alpha,\theta}(t)=\exp
\left\{-|t|^{\alpha}\exp\left\{-\fr{1}{2}\,i\pi\theta\alpha\mathrm{sign}t\right\}\right\}\,,\enskip
t\in\r\,,
$$
где $0<\alpha\le2$, $|\theta|\hm\le\min\{1,{2}/{\alpha}-1\}$, будет
обозначаться $F_{\alpha,\theta}(x)$ (см., например,~\cite{Zolotarev1983}). 
Любую с.в.\ с~ф.р.\
$F_{\alpha,\theta}(x)$ будем обозначать~$T_{\alpha,\theta}$.
Симметричным строго устойчивым распределениям соответствует значение
$\theta\hm=0$ и~х.ф.~$\mathfrak{f}_{\alpha,0}(t)\hm=e^{-|t|^{\alpha}}$,
$t\hm\in\r$. Отсюда несложно видеть, что $T_{2,0}\eqd\sqrt{2}X$.

Односторонним строго устойчивым законам, сосредоточенным на
неотрицательной полуоси, соответствуют значения $\theta\hm=1$ 
и~$0\hm<\alpha\hm\le1$. Пары $\alpha\hm=1$, $\theta\hm=\pm1$ отвечают
распределениям, вы\-рож\-ден\-ным в~$\pm1$ соответственно. Остальные
устойчивые распределения абсолютно непрерывны. Явные выражения
устойчивых плотностей в~терминах элементарных функций отсутствуют за
четырьмя исключениями (нормальный закон ($\alpha\hm=2$, $\theta\hm=0$),
распределение Коши ($\alpha\hm=1$, $\theta\hm=0$), распределение Леви
($\alpha\hm=1/2$, $\theta\hm=1$) и~распределение, симметричное 
к~распределению Леви ($\alpha\hm=1/2$, $\theta\hm=-1$)). Выражения
устойчивых плотностей в~терминах функций Фокса (обобщенных
$G$-функ\-ций Мейера) можно найти 
в~\cite{Schneider1986, UchaikinZolotarev1999}.

Согласно <<теореме умножения>> (см., например, теорему~3.3.1 
в~\cite{Zolotarev1983}) для любой допустимой пары параметров
$(\alpha,\,\theta)$ и~произвольного $\alpha'\hm\in(0,1]$ справедливо
мультипликативное представление:
$$
T_{\alpha\alpha',\theta}\eqd T_{\alpha,\theta}
T_{\alpha',1}^{1/\alpha}\,,
$$
в котором сомножители в~правой части независимы. В~частности, для
любого $\alpha\hm\in(0,2]$
\begin{equation}
T_{\alpha,0}\eqd X\sqrt{2T_{\alpha/2,1}}\,,
\label{e1-kor}
\end{equation}
т.\,е.\ любое симметричное строго устойчивое распределение является
масштабной смесью нормальных законов.

Хорошо известно, что если $0\hm<\alpha\hm<2$, 
то ${\sf E}|T_{\alpha,\theta}|^{\beta}\hm<\infty$ для любого
$\beta\hm\in(0,\alpha)$, но моменты с.в.\ $T_{\alpha,\theta}$ порядков
$\beta\hm\ge\alpha$ не существуют (см., например,~\cite{Zolotarev1983}). 
Несмотря на отсутствие явных выражений
плотностей устойчивых распределений в~терминах элементарных функций,
можно показать~\cite{KorolevWeibull2016}, что для $0\hm<\beta\hm<\alpha\hm<2$
$$
{\sf E}|T_{\alpha,0}|^{\beta}=\fr{2^{\beta}}{\sqrt{\pi}}\,
\fr{\Gamma\left(({\beta+1})/{2}\right)\Gamma\left(({\alpha-\beta})/{\alpha}\right)}
{\Gamma\left(({2-\beta})/{\beta}\right)}
$$
и для $0<\beta<\alpha\le 1$
$$
{\sf
E}T_{\alpha,1}^{\beta}=\fr{\Gamma\left(({\alpha-\beta})/{\alpha}\right)}
{\Gamma(1-\beta)}\,.
$$

Говорят, что распределение с.в.~$Z$ принадлежит 
к~об\-ласти нормального притяжения строго устойчивого закона
$F_{\alpha,\theta}$, $\mathcal{L}(Z)\hm\in \mathrm{DNA}\,(F_{\alpha,\theta})$,
если существует конечная положительная постоянная~$c$ та\-кая,~что
$$
\fr{c}{n^{1/\alpha}}\sum\limits_{j=1}^nZ_j\Longrightarrow
T_{\alpha,\theta}\enskip (n\to\infty)\,,
$$
где $Z_1,Z_2,\ldots$~--- независимые копии с.в.~$Z$. В~дальнейшем будем
рассматривать случай стандартного масштаба и~полагаем $c\hm=1$. 
В~работе~\cite{Tucker1975} было показано, что если $\mathcal{L}(Z)\hm\in
\mathrm{DNA}\,(F_{\alpha,\theta})$, то ${\sf E}|Z|^{\beta}\hm=\infty$ для любого
$\beta\hm>\alpha$.

Пусть $\alpha\in(0,1]$. Распределения с~преобразованием
Лап\-ла\-са--Стилть\-еса (п.~Л.--С.)
\begin{equation}
\mathfrak{m}_{\alpha}(s)=\fr{1}{1+s^{\alpha}}\,,\enskip s\ge0\,,
\label{e2-kor}
\end{equation}
принято называть {\it распределениями Мит\-таг--Леф\-фле\-ра}.
Происхождение этого названия связано с~тем, что плотность,
соответствующая п.~Л.--С.~(\ref{e2-kor}), имеет вид:
\begin{multline}
f_{\alpha}^{M}(x)=\fr{1}{x^{1-\alpha}}\sum\limits_{n=0}^{\infty}\fr{(-1)^nx^{\alpha
n}}{\Gamma(\alpha n+1)}=-\fr{d}{dx}\,E_{\alpha}\left(-x^{\alpha}\right)\,,\\
x\ge0\,,
\label{e3-kor}
\end{multline}
где $E_{\alpha}(z)$~--- функция Мит\-таг--Леф\-фле\-ра индекса~$\alpha$,
определяемая как степенной ряд
$$
E_{\alpha}(z)=\sum\limits_{n=0}^{\infty}\fr{z^n}{\Gamma(\alpha
n+1)}\,,\enskip \alpha>0\,,\ z\in\mathbb{Z}\,.
$$
Функция распределения, соответствующая плот\-ности~(\ref{e3-kor}), будет обозначаться
$F_{\alpha}^{M}(x)$. Случайная величина с~ф.р.~$F_{\alpha}^{M}(x)$ будет
обозначаться~$M_{\alpha}$. В~\cite{KorolevZeifman2016b} приведено
интегральное представление плотности распределения Мит\-таг--Леф\-фле\-ра:
$$
f_{\alpha}^{M}(x)=\fr{\sin(\pi\alpha)}{\pi}\int\limits_{0}^{\infty}\fr{
z^{\alpha}e^{-zx}\,dz}{1+z^{2\alpha}+2z^{\alpha}\cos(\pi\alpha)}\,,\enskip
x>0\,.
$$
При $\alpha=1$ распределение Мит\-таг--Леф\-фле\-ра превращается 
в~стандартное показательное распределение: $M_1\eqd W_1$. Но при
$\alpha\hm<1$ плотность~(\ref{e3-kor}) имеет хвост, убывающий степенн$\acute{\mbox{ы}}$м
образом: если $0\hm<\alpha\hm<1$, то
$$
f_\alpha^{M}(x)\sim \fr{\sin(\alpha\pi)\Gamma(\alpha+1)}{\pi
x^{\alpha+1}}%\eqno(5)
$$
при $x\to\infty$ (см., например,~\cite{Kilbas2014}).

Хорошо известно, что распределение Мит\-таг--Леф\-фле\-ра устойчиво по
отношению к~геометрическому суммированию (или {\it геометрически
\mbox{устойчиво}}). Это означает, что если $X_1,X_2,\ldots$~--- независимые
одинаково распределенные неотрицательные с.в.\ и~$V_p$~---
независимая от $X_1,X_2,\ldots$ с.в.\ с~геометрическим распределением,
то существуют положительные константы $a_p\hm>0$, гарантирующие
сходимость $a_p\left(X_1+\cdots+X_{V_p}\right)\hm\Longrightarrow
M_{\alpha}$ при $p\hm\to 0$ (см., например,~\cite{Bunge1996} 
или~\cite{KlebanovRachev1996}). Более того, еще 
в~1965~г.\ И.\,Н.~Коваленко~\cite{Kovalenko1965} показал, что распределения 
с~п.~Л.--С.~(\ref{e2-kor}) и~только они являются возможными предельными
распределениями для надлежащим образом нормированных геометрических
сумм вида $a_p\left(X_1+\cdots+X_{V_p}\right)$ независимых
не\-от\-ри\-ца\-тель\-ных с.в.\ при $p\hm\to0$. Доказательства этого результата
были воспроизведены в~книгах~\cite{GnedenkoKorolev1996, GnedenkoKovalenko1968,
GnedenkoKovalenko1989}, где вместо
термина <<распределения Мит\-таг--Леф\-фле\-ра>> класс распределений 
с~ п.~Л.--С.~(\ref{e2-kor}) был назван {\it классом} $\mathcal{K}$ в~честь 
И.\,Н.~Коваленко.

Спустя 25~лет упомянутое предельное свойство
распределений с~п.~Л.--С.~(\ref{e2-kor}) было переоткрыто 
Р.~Пиллаи~\cite{Pillai1989, Pillai1990}, который предложил для них
использовать термин {\it распределения Мит\-таг--Леф\-фле\-ра}, ставший
общепринятым.

Распределения Миттаг-Леф\-фле\-ра используются при описании аномальной
диффузии или эффектов релаксации (см.~\cite{WeronKotulski1996, GorenfloMainardi2006} 
и~дальнейшие ссылки в~этих работах).

Пусть $p\in(0,1)$ и~$V_p$~--- с.в.\ с~геометрическим распределением 
с~параметром~$p$:
\begin{equation}
{\sf P}(V_p=k)=p(1-p)^{k}\,,\enskip k=0,1,2,\ldots
\label{e4-kor}
\end{equation}
Это означает, что
$$
{\sf P}(V_p\ge m)=\sum\limits_{k=m}^{\infty}p(1-p)^{k}=(1-p)^{m}
$$
для любого $m\hm\in\mathbb{N}$.

Пусть $Y$~--- с.в., принимающая значения в~интервале $(0,1)$, причем
при всех $p\hm\in(0,1)$ с.в.~$Y$ и~$V_p$ независимы. Положим
$N\hm=V_{Y}$, т.\,е.\ будем считать, что
$$
{\sf P}(N\ge m)=\int\limits_{0}^{1}(1-y)^{m}\,d{\sf P}(Y<y)
$$
для любого $m\in\mathbb{N}$. Распределение с.в.~$N$ назовем {\it
$Y$-сме\-шан\-ным геометрическим}.

\section{Результаты}

В работе~\cite{Gleser1989} было показано, что гам\-ма-рас\-пре\-де\-ле\-ние 
с~параметром формы, меньшим единицы, является смешанным показательным
распределением. Этот результат для удобства будет сформулирован в~виде леммы.

\smallskip

\noindent
\textbf{Лемма~1}~\cite{Gleser1989}. \textit{Плотность гам\-ма-рас\-пре\-де\-ле\-ния
$g(x;r,\mu)$ при $0\hm<r\hm<1$ может быть представлена в~виде}:
$$
g(x;r,\mu)=\int\limits_{0}^{\infty}ze^{-zx}p(z;r,\mu)\,dz\,,
$$
\textit{где}
\begin{equation}
p(z;r,\mu)=\fr{\mu^r}{\Gamma(1-r)\Gamma(r)}\,
\fr{\mathbf{1}(z\ge\mu)}{(z-\mu)^rz}\,.\label{e5-kor}
\end{equation}
\textit{Более того, если $r\hm>1$, то гам\-ма-рас\-пре\-де\-ле\-ние с~параметром 
формы~$r$ нельзя представить в~виде смешанного показательного
распределения.}

\smallskip

Хорошо известно, что отрицательное биномиальное распределение %(1)
является смешанным пуассоновским распределением, в~котором
смешивание происходит по гам\-ма-рас\-пре\-де\-ле\-нию~\cite{GreenwoodYule1920} 
(также см.~\cite{KorolevBeningShorgin2011}): для любых $r\hm>0$, $p\hm\in(0,1)$ 
и~$k\hm\in\{0\}\bigcup\mathbb{N}$
\begin{equation}
\fr{\Gamma(r+k)}{k!\Gamma(r)}\,
p^r(1-p)^k=\fr{1}{k!}\int\limits_{0}^{\infty}e^{-\lambda}\lambda^kg(\lambda;r,\mu)\,
d\lambda\,,\label{e6-kor}
\end{equation}
где $\mu=p/(1-p)$. При условии $0\hm<r\hm<1$ продолжим~(\ref{e6-kor}) с~учетом леммы~1:
\begin{multline*}
\fr{\Gamma(r+k)}{k!\Gamma(r)}\,
p^r(1-p)^k={}\\
{}=\fr{1}{k!}\int\limits_{0}^{\infty}
e^{-\lambda}\lambda^k\left(\int\limits_{\mu}^{\infty}ze^{-z\lambda}p(z;r,\mu)\,dz\right)\,d\lambda={}
\\
{}=\int\limits_{\mu}^{\infty}\left(\fr{1}{k!}\int\limits_{0}^{\infty}
e^{-\lambda(z+1)}\lambda^kz\,d\lambda\right)p(z;r,\mu)\,dz={}\\
{}=
\int\limits_{\mu}^{\infty}\left(\fr{z}{z+1}\right)
\left(1-\fr{z}{z+1}\right)^kp(z;r,\mu)\,dz\,.
\end{multline*}
Сделав в~последнем интеграле замену переменных $z\longmapsto
y/(1-y)$, получим
$$
\fr{\Gamma(r+k)}{k!\Gamma(r)}\,
p^r(1-p)^k=\int\limits_{p}^{1}y(1-y)^kh(y;r,p)\,dy\,,
$$
где
\begin{multline}
h(y;r,p)={}\\
{}=\fr{p^r}{\Gamma(1-r)\Gamma(r)}\,
\fr{(1-y)^{r-1}\mathbf{1}(p<y<1)}{y(y-p)^r}\,.
\label{e7-kor}
\end{multline}
Тем самым доказана

\smallskip

\noindent
\textbf{Теорема 1.} \textit{Отрицательное биномиальное распределение 
с~параметрами $r\hm\in(0,1)$ и~$p\hm\in(0,1)$ является смешанным
геометрическим распределением$:$ для любого}
$k\in\{0\}\bigcup\mathbb{N}$
\begin{multline*}
\fr{\Gamma(r+k)}{k!\Gamma(r)}\,
p^r(1-p)^k={}\\
{}=\int\limits_{\mu}^{\infty}\left(\fr{z}{z+1}\right)
\left(1-\fr{z}{z+1}\right)^kp(z;r,\mu)\,dz={}\\
{}=\int\limits_{p}^{1}y(1-y)^kh(y;r,p)\,dy\,,
\end{multline*}
\textit{где $\mu=p/(1-p)$, а~плотности $p(z;r,\mu)$ и~$h(y;r,p)$
определены соответственно в}~(\ref{e5-kor}) \textit{и}~(\ref{e7-kor}).

\smallskip

\noindent
\textbf{Следствие~1.} Пусть $m\hm>0$, $p\hm\in(0,1)$,
$V_p^{(0)},V_p^{(1)},\ldots$~--- с.в.\ с~одним и~тем же геометрическим
распределением~(\ref{e4-kor}), $Y_{\{m\},p}$~--- с.в.\ с~плот\-ностью
распределения $h(y;\{m\},p)$. Предположим, что все введенные с.в.\
независимы. Пусть $N_{m,p}$~--- с.в.\ с~отрицательным биномиальным
распределением~(\ref{e1-kor}) при $r\hm=m$. Тогда
$$
N_{m,p}\eqd V^{(0)}_{Y_{\{m\},p}}+\sum\limits_{n=1}^{[m]}V_p^{(n)}
$$
(для определенности считаем, что $\sum_{n=1}^{0}=0$ и~$V^{(0)}_{Y_{0,p}}=0)$.

\smallskip

Утверждение теоремы~1 можно проинтерпретировать в~терминах испытаний
Бернулли со случайной вероятностью успеха. Сначала в~результате
<<предварительного>> эксперимента определяется значение вероятности
успеха, т.\,е.\ значение с.в.~$Y_{r,p}$, имеющей плотность
$h(y;r,p)$. Потом с.в.~$N_{r,p}$ определяется как
число успехов до первой неудачи в~последовательности испытаний
Бернулли с~так определенной вероятностью успеха~$Y_{r,p}$.

Такая интерпретация теоремы~1 позволяет привести дополнительные
аргументы, объясняющие адекватность отрицательной биномиальной
модели для распределения дождливых периодов, а~именно: можно
предположить, что последовательность дождливых и~сухих дней не
является независимой, но является {\it условно независимой} при
фиксированном значении с.в.~$Y_{r,p}$, которое меняется от одного
дождливого периода к~другому (например, в~зависимости от времени
года) и~определяется факторами, внешними по отношению к~исследуемой
локальной системе.

Рассмотрим смешивающие плотности $p(z;r,\mu)$ и~$h(y;r,p)$ более
подробно.

\smallskip

\noindent
\textbf{Теорема~2.} \textit{Для $r\hm\in(0,1)$ пусть $G_{r,\,1}$ 
и~$G_{1-r,\,1}$~--- независимые гам\-ма-рас\-пре\-де\-лен\-ные с.в. Пусть
$\mu\hm>0$, $p\hm\in(0,1)$. Тогда}:
\begin{enumerate}[($i$)]
\item \textit{плотность $p(z;r,\mu)$ соответствует с.в.}
$$
Z_{r,\mu}=\fr{\mu(G_{r,\,1}+G_{1-r,\,1})}{G_{r,\,1}}\,;
$$

\item \textit{плотность $h(y;r,p)$ соответствует с.в.}
$$
Y_{r,p}=\fr{p(G_{r,\,1}+G_{1-r,\,1})}{G_{r,\,1}+pG_{1-r,\,1}}\,.
$$
\end{enumerate}

\smallskip

\noindent
Д\,о\,к\,а\,з\,а\,т\,е\,л\,ь\,с\,т\,в\,о\,.\ \ 
Из доказательства теоремы~1 вытекает, что
$$
Y_{r,p}\eqd\fr{Z_{r,\mu}}{1+Z_{r,\mu}}
$$
с $\mu=p/(1-p)$. Поэтому достаточно убедиться в~справедливости
первого утверждения теоре\-мы. С~по\-мощью замены переменных
$z\longmapsto \mu(1+\linebreak + x)$ и~элементарных выкладок легко проверить, что
плотность $p(z;r,\mu)$ соответствует с.в.\
$\mu(1\hm+(({1-r})/{r})Q_{2(1-r),\,2r})$, где $Q_{\nu_1,\nu_2}$~--- 
с.в., име\-ющая распределение Сне\-де\-ко\-ра--Фи\-ше\-ра, для $\nu_1\hm>0$,
$\nu_2\hm>0$ определяемое лебеговой плот\-ностью
\begin{multline*}
f_{\nu_1,\nu_2}(x)={}\\
{}=\fr{\Gamma(({\nu_1+\nu_2})/{2})}{\Gamma({\nu_1}/{2})
\Gamma({\nu_2}/{2})}\,
\nu_1^{\nu_1/2}\nu_2^{\nu_2/2}\,
\fr{x^{\nu_1/2-1}}{(\nu_2+\nu_1x)^{(\nu_1+\nu_2)/2}}\,,\\
x\ge0
\end{multline*}
(в рассматриваемом случае $\nu_1\hm=2(1\hm-r)$, $\nu_2\hm=2r$, так что
$(\nu_1\hm+\nu_2)/2\hm=1$). Но, как известно,
$$
Q_{\nu_1,\nu_2}\eqd\fr{\nu_2 G_{\nu_1/2,\,1/2}}{\nu_1
G_{\nu_2/2,\,1/2}}\eqd\fr{\nu_2 G_{\nu_1/2,\,1}}{\nu_1
G_{\nu_2/2,\,1}}\,,
$$
где с.в. $G_{\nu_1/2,\,1}$ и~$G_{\nu_2/2,\,1}$ независимы (см.,
например,~\cite[с.~32]{Bolshev}). Это замечание завершает
доказательство теоремы.

\smallskip

\noindent
\textbf{Замечание~1.} Несложно видеть, что в~числителях представлений
с.в.~$Z_{r,\mu}$ и~$Y_{r,p}$, приведенных в~формулировке леммы~2,
сумма $G_{r,\,1}+G_{1-r,\,1}$ имеет стандартное показательное
распределение: $G_{r,\,1}+G_{1-r,\,1}\eqd W_1$. Однако при этом
числители и~знаменатели указанных представлений не являются
независимыми с.в.

\smallskip

Из утверждения ($ii$) теоремы~2 вытекает, что при $p\hm\to0$ с.в.~$Y_{r,p}$ 
является величиной порядка~$p$ в~том смысле, что при
$p\hm\to0$ с.в.~$p^{-1}Y_{r,p}$ по распределению сходится 
к~собственной невырожденной с.в. Придадим этому утверждению строгую
формулировку.

\smallskip

\noindent
\textbf{Следствие~2}. Пусть $r\hm\in(0,1)$, $q\hm\in(0,1)$ и~$\mu\hm>0$
произвольны. Тогда
$$
nY_{r,\min\left\{q,\mu/n\right\}}\Longrightarrow Z_{r,\mu}
$$
при $n\to\infty$.

\smallskip

\noindent
Д\,о\,к\,а\,з\,а\,т\,е\,л\,ь\,с\,т\,в\,о\,.\ \ 
Согласно пункту ($ii$) теоремы~2 при
$n\hm\to\infty$ имеем:
\begin{multline}
nY_{r,\min\left\{q,{\mu}/{n}\right\}}=
\fr{\min\{nq,\mu\}(G_{r,\,1}+G_{1-r,\,1})}
{G_{r,\,1}+\min\{q,{\mu}/{n}\}G_{1-r,\,1}}\Longrightarrow{}\\
{}\Longrightarrow
\fr{\mu(G_{r,\,1}+G_{1-r,\,1})}{G_{r,\,1}}\,.\label{e8-kor}
\end{multline}
Но в~соответствии с~пунктом~($i$) теоремы~2 правая часть~(\ref{e8-kor})
совпадает с~$Z_{r,\mu}$. Следствие доказано.

\smallskip

Докажем результат, распространяющий теорему Глезера (лемму~1) на
обобщенные гам\-ма-рас\-пре\-де\-ле\-ния. Для этой цели понадобится следующее
представление с.в.\ с~распределением Вейбулла с~параметром
$\alpha\hm\in(0,1]$, доказанное в~\cite{KorolevWeibull2016}.

\smallskip

\noindent
\textbf{Лемма~2}~\cite{KorolevWeibull2016}. \textit{Пусть
$\alpha\hm\in(0,1]$. Тогда}
$$
W_{\alpha}\eqd\fr{W_1}{T_{\alpha,1}}\,,
$$
\textit{где с.в.\ в~правой части независимы.}

\pagebreak


\noindent
\textbf{Лемма~3}. \textit{Функция распределения~$F(x)$ с~$F(0)\hm=0$ 
соответствует смешанному
показательному распределению тогда и~только тогда, когда функция
$1\hm-F(x)$ вполне монотонна, т.\,е.\ $F\hm\in C_{\infty}$ 
и~$(-1)^{n+1}F^{(n)}(x)\hm\ge 0$ при всех $x\hm>0$}.

\smallskip

Это утверждение немедленно вытекает из теоремы С.\,Н.~Бернштейна~\cite{Bernstein1928}.

\smallskip

\noindent
\textbf{Теорема~3}.\ \textit{Пусть $\alpha\hm\in(0,1]$, $r\hm\in(0,1)$, $\mu\hm>0$.
Тогда обобщенное гам\-ма-рас\-пре\-де\-ле\-ние с~параметрами~$r$, $\alpha$
и~$\mu$ является смешанным показательным распределением}:
$$
G^*_{r,\alpha,\mu}\eqd\fr{W_1}{T_{\alpha,1}\,Z_{r,\mu}^{1/\alpha}}\,,
$$
\textit{где с.в.\ в~правой части независимы. Более того, обобщенное
гам\-ма-рас\-пре\-де\-ле\-ние с~$\alpha r\hm>1$ не может быть представлено в~виде
смешанного показательного распределения.}

\smallskip

\noindent
Д\,о\,к\,а\,з\,а\,т\,е\,л\,ь\,с\,т\,в\,о\,.\ \ Докажем первое утверждение теоремы. 
Во-пер\-вых,
заметим, что согласно лемме~1 для $x\hm\ge0$ справедливы соотношения:
\begin{multline*}
{\sf P}(G_{r,\mu}^{1/\alpha}>x)={\sf P}\left(G_{r,\mu}>x^{\alpha}\right)={}\\
{}={\sf P}\left(W_1>Z_{r,\mu}x^{\alpha}\right)=\int\limits_{0}^{\infty}
e^{-zx^{\alpha}}p(z;r,\mu)\,dz={}
\\
{}=\int\limits_{0}^{\infty}{\sf P}\left(W_{\alpha}>xz^{1/\alpha}\right)p(z;r,\mu)\,dz\,,
\end{multline*}
т.\,е.
\begin{equation}
G_{r,\mu}^{1/\alpha}\eqd
\fr{W_{\alpha}}{Z_{r,\mu}^{1/\alpha}}\,.\label{e9-kor}
\end{equation}
Теперь воспользуемся леммой~2 и,~продолжив~(\ref{e9-kor}), получим:
\begin{equation}
G_{r,\mu}^{1/\alpha}\eqd
\fr{W_1}{T_{\alpha,1}Z_{r,\mu}^{1/\alpha}}\,.\label{e10-kor}
\end{equation}
Во-вторых, несложно убедиться, что
\begin{equation}
G_{r,\mu}^{1/\alpha}\eqd G^*_{r,\alpha,\mu}\label{e11-kor}
\end{equation}
при любых $r>0$, $\mu\hm>0$ и~$\alpha\hm>0$. Теперь первое утверждение
теоремы вытекает из~(\ref{e10-kor}) и~(\ref{e11-kor}).

Докажем второе утверждение. Пусть $\alpha r\hm>1$. Предположим, что 
с.в.~$G^*_{r,\alpha,\mu}$ имеет смешанное показательное распределение.
По лемме~3 это означает, что функция $\psi(s)\hm={\sf
P}(G^*_{r,\alpha,\mu}>s)$, $s\hm\ge0$, является вполне монотонной. Но
$\psi'(s)\hm=g^*(s;r,\alpha,\mu)\ge0$ при всех $s\hm\ge0$, тогда как
\begin{multline*}
\psi''(s)=\left(g^*\right)'(s;r,\alpha,\mu)={}\\
{}=\fr{\alpha\mu^r}{\Gamma(r)}\,s^{\alpha
r-2}e^{-\mu s^{\alpha}}\left((\alpha r-1)-\mu\alpha
s^{\alpha}\right)\le0\,,
\end{multline*}
только если $(\alpha r-1)\hm-\mu\alpha s^{\alpha}\hm\le0$, т.\,е.
$$
s\ge s_0\equiv \left(\fr{\alpha r-1}{\mu\alpha}\right)^{1/\alpha}>0\,,
$$
и $\psi''(s)\ge0$ при $s\hm\in(0,s_0)\hm\neq\varnothing$, что противоречит
вполне монотонности функции~$\psi(s)$, доказывая второе утверждение
теоремы. Теорема доказана.

\vspace*{-9pt}

\section{Смешанные биномиальные распределения, связанные с~отрицательными
биномиальными законами с~$r<1$, и~их асимп\-то\-ти\-чес\-кое
поведение при $p\hm\to0$}

Рассмотрим еще одну задачу, связанную с~описанной выше схемой
испытаний Бернулли со случай\-ной вероятностью успеха $Y_{r,p}$ при
условии <<малости>> последней. В~рамках этой схемы сначала 
в~результате <<предварительного>> эксперимента определяется значение
с.в.~$Y_{r,p}\hm\in(0,1)$. Это значение принимается в~качестве
вероятности успеха в~испытаниях Бернулли. Затем с.в.~$M$ 
определяется как число успехов в~$m\hm\in\mathbb{N}$ испытаниях
Бернулли с~так определенной вероятностью успеха~$Y_{r,p}$. Чтобы
описать бесконечную малость вероятности успеха~$Y_{r,p}$, снабдим
параметр~$p$ и~(для общности) параметр~$m$, а~также, соответственно,
с.в.~$M$ <<бесконечно большим>> индексом~$n$, позволяющим
проследить сходимость последовательности с.в.~$Y_{r,p_n}$ к~нулю
при $n\hm\to\infty$. В~свою очередь, бесконечная малость~$Y_{r,p_n}$
означает, что успехи являются редкими событиями в~рамках
рассматриваемой последовательности испытаний Бернулли со случайной
вероятностью успеха.

В рамках схемы испытаний Бернулли со случайной вероятностью успеха,
описанной выше, можно сформулировать и~доказать <<случайный>> аналог
классической тео\-ре\-мы Пуассона (так называемого <<закона малых
чисел>>) для {\it смешанных биномиальных распределений} со случайной
вероятностью успеха и~неограниченно возрастающим целочисленным
параметром~$m_n$ (<<числом испытаний>>)~\cite{KorolevPoisson,
Korolev2016}. В~ранее известных вариантах <<случайного>> аналога
тео\-ре\-мы Пуассона (см., к~примеру,~\cite{KorolevBeningShorgin2011}),
наоборот, случайным считалось число испытаний, а~вероятность успеха
оставалась неслучайной.

Пусть для каждого $n\hm\in\mathbb{N}$ $Y_n$~--- с.в.\ такая, что ${\sf
P}(0\hm<Y_n\hm<1)\hm=1$, $m_n\hm\in\mathbb{N}$, $k\hm=1,2,\ldots$ Будем гово-\linebreak\vspace*{-12pt}

\pagebreak

\noindent
рить, что
с.в.~$M_n$ имеет {\it $Y_n$-сме\-шан\-ное биномиальное распределение} 
с~параметром~$m_n$, если
\begin{multline}
 {\sf P}(M_n=j)=C_{m_n}^j\int\limits_{0}^{1}z^k(1-z)^{m_n-j}\,d{\sf P}(Y_n<z)\,,\\
j=0,1,\ldots , m_n\,.
\label{e12-kor}
\end{multline}
Для $x\in\mathbb{R}$ обозначим $B_n(x)\hm={\sf P}(M_n<x)$. Пусть~$Z$~---
положительная с.в. Смешанная пуассоновская ф.р.\ со структурной с.в.~$Z$ 
(по терминологии, принятой в~\cite{Grandell1997}) будет
обозначаться $\Pi^{(Z)}(x)$:
$$
\Pi^{(Z)}(x+0)=\sum\limits_{j=0}^{[x]}\fr{1}{j!}\int\limits_{0}^{\infty}e^{-z}z^j\,
d{\sf P}(Z<z)\,,\enskip x\in\mathbb{R}\,.
$$
В~\cite{Korolev2016} доказана следующая теорема.

\smallskip

\noindent
\textbf{Теорема~4}~\cite{Korolev2016}. \textit{Пусть $\{m_n\}_{n\ge1}$~---
неограниченно возрастающая последовательность натуральных чисел.
Пусть при каждом $n\in\mathbb{N}$ $M_n$~--- с.в., имеющая
$Y_n$-сме\-шан\-ное биномиальное распределение}~(\ref{e12-kor}) 
\textit{с~це\-ло\-чис\-лен\-ным
параметром~$m_n$ и~ф.р.~$B_n(x)$. Предположим, что в}~(\ref{e12-kor}) 
\textit{с.в.~$Y_n$ бесконечно малы в~том смысле, что существует с.в.~$Z$ такая,
что ${\sf P}(0\hm< Z\hm<\infty)\hm=1$ и~выполнено условие}
\begin{equation}
m_nY_n\Longrightarrow Z \label{e13-kor}
\end{equation}
\textit{при $n\hm\to\infty$. Тогда}
$$
B_n(x)\Longrightarrow \Pi^{(Z)}(x) \enskip (n\to\infty)\,.
$$

\smallskip

Пусть числа $r\in(0,1)$, $q\hm\in(0,1)$ и~$\mu\hm>0$ произвольны,
$\{m_n\}_{n\ge1}$~--- неограниченно возрастающая последовательность
натуральных чисел. Положим
$$
Y_n=Y_{r,\min\{q,\mu/m_n\}}\,,\enskip n\in\mathbb{N}\,.
$$
Тогда из следствия~2 вытекает, что
$$
m_nY_n\Longrightarrow Z_{r,\mu}
$$
при $n\to\infty$, т.\,е.\ условие~(\ref{e13-kor}) выполнено с~$Z\hm=Z_{r,\mu}$.
При этом из теоремы~4 немедленно получается следующий аналог теоремы
Пуассона.

\smallskip

\noindent
\textbf{Следствие~3}. {Пусть числа $r\hm\in(0,1)$, $q\hm\in(0,1)$ 
и~$\mu\hm>0$ произвольны, $\{m_n\}_{n\ge1}$~--- неограниченно возрастающая
последовательность натуральных чисел. Пусть при каждом
$n\in\mathbb{N}$ $M_n$~-- с.в., имеющая
$Y_{r,\min\{q,\mu/m_n\}}$-сме\-шан\-ное биномиальное распределение 
с~параметром~$m_n$. Тогда}
$$
{\sf P}(M_n<x)\Longrightarrow \Pi^{(Z_{r,\mu})}(x)\enskip
(n\to\infty)\,.
$$

\smallskip

Если $P_{r,\mu}$~--- с.в.\ с~распределением $\Pi^{(Z_{r,\mu})}(x)$,
то для любого $k\hm=0,1,\ldots$
\begin{multline*}
{\sf P}(P_{r,\mu}=k)=\fr{1}{k!}\int\limits_{0}^{\infty}z^ke^{-z}p(z;r,\mu)\,dz={}\\
{}=
\fr{1}{k!}\,{\sf E}\left(Z_{r,\mu}^k\exp\left\{-Z_{r,\mu}\right\}\right)\,.
\end{multline*}

\section{Предельные теоремы для отрицательных биномиальных случайных сумм 
с~$r<1$}

\subsection{Аналог закона больших чисел для~неотрицательных слагаемых. 
Обобщенная теорема Реньи}

Пусть $X_1,X_2,\ldots$~--- независимые одинаково распределенные
неотрицательные с.в. Пусть при каж\-дом $n\hm\in\mathbb{N}$ с.в.\
$N_{r,p_n}$ имеет отрицательное биномиальное распределение 
с~$r\hm\in(0,1)$ и~$p_n\hm\in(0,1)$. Предположим, что при каждом
$n\hm\in\mathbb{N}$ с.в.~$N_{r,p_n}$ независима от последовательности
$X_1,X_2,\ldots$ Обозначим
$$
S_k=X_1+\cdots+X_k\,,\enskip k\in\mathbb{N}\,.
$$

Начнем с~рассмотрения ситуации, в~которой существует математическое
ожидание ${\sf E}X_1\hm\equiv a$. Соглас\-но усиленному закону больших
чисел Колмогорова это условие в~определенном смысле необходимо 
и~достаточно для того, чтобы с~ве\-ро\-ят\-ностью единица
\begin{equation}
\fr{1}{n}\sum\limits_{j=1}^nX_j\longrightarrow a 
\label{e14-kor}
\end{equation}
при $n\to\infty$. Поставим целью изучить асимптотическое поведение
с.в.~$S_{N_{r,p_n}}$ при $p_n\hm\to 0$ при условии~(\ref{e14-kor}) 
и~получить аналог закона больших чисел для отрицательных биномиальных
случайных сумм при $r\hm<1$.

В работе~\cite{Korolev2016TVP} доказано следующее утверждение.

\smallskip

\noindent
\textbf{Теорема~5}~\cite{Korolev2016TVP}. \textit{Предположим, что
неотрицательные с.в.\ $X_1,X_2,\ldots$ удовлетворяют условию}~(\ref{e14-kor}).
\textit{Пусть при каждом $n\in\mathbb{N}$ с.в.~$N_n$ имеют $Y_n$-сме\-шан\-ное
геометрическое распределение и~независимы от $X_1,X_2,\ldots$
Предположим, что существует с.в.~$Z$ такая, что ${\sf P}
(0\hm<Z\hm<\infty)\hm=1$ и~при $n\hm\to\infty$ выполнено условие}:
\begin{equation}
nY_n\Longrightarrow Z\,.
\label{e15-kor}
\end{equation}
\textit{Тогда}
$$
\mathop{\mathrm{lim}}\limits_{n\to\infty}\sup\limits_{x\ge0}\left\vert {\sf P}
\left(\fr{S_{N_n}}{n}>x\right)-\int\limits_{0}^{\infty}e^{-xz/a}\,d{\sf P}
(Z<z)\right\vert=0\,.
$$


\smallskip

Теорема~5 является развитием классической тео\-ре\-мы Реньи об
асимптотическом поведении прореживаемых процессов восстановления
(см., например,~\cite{GnedenkoKorolev1996}). Классическую теорему
Реньи можно считать законом больших чисел для геометрических
случайных сумм. Эта теорема устанавливает, что однородный точечный
процесс с~конечным математическим ожиданием длин интервалов между
соседними точками, подвергнутый операции прос\-тей\-ше\-го прореживания,
при которой каждая точка удаляется с~вероятностью $1\hm-p$ 
и~оставляется на своем месте с~вероятностью~$p$, сопровождающейся
надлежащей компрессией времени с~целью обеспечить нетривиальность
предельного процесса, сходится к~пуассоновскому процессу. Как
известно, пуассоновский процесс характеризуется в~классе процессов
восстановления тем, что длины интервалов времени между
последовательными восстановлениями имеют показательное
распределение. Теорема~5 обобщает теорему Реньи на отрицательные
биномиальные случайные суммы. При этом обобщение сводится к~тому,
что рассматривается <<дважды стохастическое>> прореживание, при
котором одинаковая для всех точек исходного процесса вероятность~$p$
определяется заранее как результат некоторого предварительного
случайного эксперимента. Это приводит к~тому, что предельный процесс
оказывается смешанным пуассоновским, что хорошо согласуется 
с~утверждением следствия~3.

Из теоремы~5, следствия~2 и~леммы~1 непосредственно вытекает
следующее утверждение.

\smallskip

\noindent
\textbf{Следствие~4.} Предположим, что независимые одинаково
распределенные неотрицательные с.в.\ $X_1,X_2,\ldots$ удовлетворяют
условию~(\ref{e14-kor}). Пусть чис\-ла $r\hm\in(0,1)$, $q\hm\in(0,1)$ и~$\mu\hm>0$
произвольны. Пусть при каждом $n\hm\in\mathbb{N}$ с.в.~$N_{r,p_n}$
имеют отрицательные биномиальные распределения с~параметрами~$r$ 
и~$p_n\hm=\min\{q,\mu/n\}$ и~независимы от $X_1,X_2,\ldots$ \mbox{Тогда}
\begin{equation}
\fr{S_{N_{r,p_n}}}{n}\Longrightarrow
aG_{r,\mu}\eqd\fr{aW_1}{Z_{r,\mu}}
\label{e16-kor}
\end{equation}
при $n\to\infty$.

\smallskip

Чтобы убедиться в~справедливости следствия~4, достаточно заметить,
что в~соответствии с~теоремой~5 и~следствием~2 для любого
$x\hm\in\mathbb{R}$ 
\begin{equation}
{\sf P}\left(\fr{S_{N_{r,p_n}}}{n}<x\right)\Longrightarrow
1-\int\limits_{0}^{\infty}e^{-xz/a}p(z;\,r,\mu)\,dz
 \label{e17-kor}
\end{equation}
при $n\to\infty$, но по лемме~1 смешанная показательная ф.р.\ 
в~правой части~(\ref{e17-kor}) совпадает с~функцией гам\-ма-рас\-пре\-де\-ле\-ния 
с~параметрами~$r$ и~$\mu$ в~точке~$x/a$.

С учетом абсолютной непрерывности предельного гам\-ма-рас\-пре\-де\-ле\-ния 
в~следствии~4 можно заключить, что на самом деле в~условии~(\ref{e16-kor}) речь
идет о~равномерной сходимости ф.р.:
$$
\lim\limits_{n\to\infty}\sup\limits_{x\ge0}\left\vert {\sf
P}\left(\fr{S_{N_{r,p_n}}}{n}>x\right)-\int\limits_{x}^{\infty}g(z/a;\,r,\mu)\,dz
\right\vert=0\,.
$$

\smallskip

В терминах статистических закономерностей процессов выпадения
осадков предельное гам\-ма-рас\-пре\-де\-ле\-ние в~следствии~4 может служить
асимптотической аппроксимацией распределения суммар\-но\-го объема
осадков, выпавших в~течение одного <<продолжительного>> ($p_n\hm\to0$)
дождливого периода, если средние арифметические ежедневных осадков
относительно стабильны.

\smallskip

\noindent
\textbf{Замечание~2.} Сходимость $ n^{-1}S_{N_{r,p_n}}\Longrightarrow
aG_{r,\mu}$ имеет место и~для $r\hm\ge 1$.


\subsection{Суммы неотрицательных слагаемых. Случай тяжелых хвостов}

В этом подразделе будет рассмотрена ситуация, когда условие~(\ref{e14-kor}) не
выполнено, т.\,е.\ хвосты распределений слагаемых $X_1,X_2,\ldots$
столь тяжелы, что математическое ожидание отсутствует. Вместо
условия~(\ref{e14-kor}) здесь будем предполагать, что $\mathcal{L}(X_1)\hm\in
\mathrm{DNA}\,(F_{\alpha,1})$ при некотором $\alpha\hm\in(0,1)$.

В работе~\cite{Korolev2016TVP} доказано следующее утверждение.

\smallskip

\noindent
\textbf{Теорема~6}~\cite{Korolev2016TVP}. \textit{Предположим, что
независимые одинаково распределенные неотрицательные с.в.\
$X_1,X_2,\ldots$ таковы, что $\mathcal{L}(X_1)\hm\in \mathrm{DNA}\,(F_{\alpha,1})$
при некотором $\alpha\hm\in(0,1)$. Пусть при каждом $n\hm\in\mathbb{N}$
с.в.~$N_n$ имеют $Y_n$-сме\-шан\-ное геометрическое распределение 
и~независимы от $X_1,X_2,\ldots$ Предположим, что существует с.в.~$Z$
такая, что ${\sf P}(0\hm<Z\hm<\infty)\hm=1$ и~при $n\hm\to\infty$ выполнено
условие}~(\ref{e15-kor}). \textit{Тогда}
\begin{equation}
\fr{S_{N_n}}{n^{1/\alpha}}\Longrightarrow T_{\alpha,1}
\left(\fr{W_1}{Z}\right)^{1/\alpha}\enskip (n\to\infty)\,,
\label{e18-kor}
\end{equation}
\textit{причем с.в.\ в~правой части}~(\ref{e18-kor}) \textit{независимы.}

\vspace*{2pt}

Из теоремы~6, следствия~2 и~леммы~1 непосредственно вытекает
следующее утверждение.

\vspace*{2pt}

\noindent
\textbf{Следствие~5.} {Предположим, что независимые одинаково
распределенные неотрицательные с.в.\ $X_1,X_2,\ldots$ таковы, что
$\mathcal{L}(X_1)\hm\in \mathrm{DNA}\,(F_{\alpha,1})$ при некотором
$\alpha\hm\in(0,1)$. Пусть числа $r\hm\in(0,1)$, $q\hm\in(0,1)$ и~$\mu\hm>0$
произвольны. Пусть при каждом $n\hm\in\mathbb{N}$ с.в.~$N_{r,p_n}$
имеют отрицательные биномиальные распределения с~параметрами~$r$ 
и~$p_n\hm=\min\{q,\mu/n\}$ и~независимы от~$X_1,X_2,\ldots$ Тогда}

\noindent
\begin{equation}
\fr{S_{N_{r,p_n}}}{n^{1/\alpha}}\Longrightarrow T_{\alpha,1}
G_{r,\mu}^{1/\alpha}
\label{e19-kor}
\end{equation}
при $n\to\infty$.

\pagebreak

%\smallskip

Чтобы убедиться в~справедливости следствия~5, достаточно заметить,
что в~соответствии с~теоремой~6 и~следствием~2 для любого
$x\hm\in\mathbb{R}$
$$
\fr{S_{N_{r,p_n}}}{n^{1/\alpha}}\Longrightarrow T_{\alpha,1}
\left(\fr{W_1}{Z_{r,\mu}}\right)^{1/\alpha}
$$
при $n\to\infty$, но по лемме~1
$$
\fr{W_1}{Z_{r,\mu}}\eqd G_{r,\mu}\,.
$$

С учетом абсолютной непрерывности предельного распределения в~следствии~5 
можно заключить, что на самом деле в~условии~(\ref{e19-kor}) речь
идет о~равномерной сходимости ф.р.:
\begin{multline*}
\lim\limits_{n\to\infty}\sup\limits_{x\ge0}\left\vert 
\vphantom{\int\limits_{0}^{x}}
{\sf P}
\left(\fr{S_{N_{r,p_n}}}{n^{1/\alpha}}<x\right)-{}\right.\\
\left.{}-
\int\limits_{0}^{x}F_{\alpha,1}\left(\fr{x}{z^{1/\alpha}}\right)g(z;\,r,\mu)\,dz
\right\vert=0\,.
\end{multline*}

\smallskip

С учетом теоремы~3 предельное распределение в~следствии~5 можно
записать в~альтернативных эквивалентных формах. Для этого
понадобятся еще два вспомогательных утверждения.

\smallskip

\noindent
\textbf{Лемма~4}~\cite{KorolevZeifman2016b, KotzOstrovskii1996,
KorolevZeifman2016a}. \textit{Пусть $\alpha\hm\in(0,1)$. Предположим, что
неотрицательные с.в.~$T_{\alpha,1}$ и~$T'_{\alpha,1}$ независимы 
и~имеют одно и~то же строго устойчивое распределение. Тогда плот\-ность~$v_{\alpha}(x)$ 
с.в.~$R_{\alpha}\hm=T_{\alpha,1}/T'_{\alpha,1}$ имеет
вид}:
\begin{equation*}
v_{\alpha}(x)=\fr{\sin(\pi\alpha)x^{\alpha-1}}
{\pi[1+x^{2\alpha}+2x^{\alpha}\cos(\pi\alpha)]}\,,\enskip
x>0\,.
%\label{e20-kor}
\end{equation*}

\smallskip

\noindent
\textbf{Лемма~5}~\cite{KorolevZeifman2016b, KotzOstrovskii1996,
KorolevZeifman2016a}. \textit{Пусть $\alpha\hm\in(0,1)$, $M_{\alpha}$~---
с.в.\ с~распределением Мит\-таг--Леф\-фле\-ра с~параметром~$\alpha$. Тогда}
$$
M_{\alpha}\eqd W_1  R_{\alpha}\,,
$$
\textit{где с.в.\ в~правой части независимы.}

\smallskip

Первое альтернативное представление распределения в~правой части~(\ref{e19-kor}) 
довольно очевидно. С~учетом соотношения
$G_{r,\mu}^{1/\alpha}\eqd G^*_{r,\alpha,\mu}$ (см.\ доказательство
теоремы~3) вместо~(\ref{e19-kor}) можно записать:
\begin{equation}
\fr{S_{N_{r,p_n}}}{n^{1/\alpha}}\Longrightarrow T_{\alpha,1}
G^*_{r,\alpha,\mu}\,.\label{e21-kor}
\end{equation}
К сожалению, ни соотношением~(\ref{e19-kor}), ни соотношением~(\ref{e21-kor}) нельзя
пользоваться при статистическом анализе на основе функции
правдоподобия, так как плотность с.в.~$T_{\alpha,1}$ нельзя
выписать в~явном виде в~терминах элементарных функций за исключением
случая $\alpha=1/2$. В~отличие от представлений~(\ref{e19-kor}) и~(\ref{e21-kor}), два
следующих представления вполне пригодны для применения 
в~статистическом анализе.

Чтобы получить второе альтернативное пред\-став\-ле\-ние распределения 
в~правой части~(\ref{e19-kor}), воспользуемся теоремой~3, соотношением~(\ref{e21-kor}) 
и~вмес\-то~(\ref{e19-kor}) получим:
$$
\fr{S_{N_{r,p_n}}}{n^{1/\alpha}}\Longrightarrow
\fr{T_{\alpha,1}}{T'_{\alpha,1}}\,
\fr{W_1}{Z_{r,\mu}^{1/\alpha}}\eqd
R_{\alpha}\,t\fr{W_1}{Z_{r,\mu}^{1/\alpha}},
$$
где в~каждом выражении с.в.\ независимы. При этом явный вид
плотности с.в.~$R_{\alpha}$ указан в~лемме~3, а~плот\-ность
$w_{\alpha}(x;r,\mu)$ с.в.~$W_1Z_{r,\mu}^{-1/\alpha}$ выписывается
легко:
$$
w_{\alpha}(x;r,\mu)=\fr{\alpha\mu^r}{\Gamma(1-r)\Gamma(r)}
\int\limits_{\mu^{1/\alpha}}^{\infty}\fr{e^{-zx}\,dz}{(z^{\alpha}-\mu)^r}\,,\enskip
x\ge0\,.
$$

Наконец, чтобы получить третье альтернативное представление
распределения в~правой части~(\ref{e19-kor}), воспользуемся соотношением~(\ref{e21-kor}) 
и~леммой~5. При этом получаем:
\begin{equation}
\fr{S_{N_{r,p_n}}}{n^{1/\alpha}}\Longrightarrow
\fr{T_{\alpha,1}}{T'_{\alpha,1}}\,
\fr{W_1}{Z_{r,\mu}^{1/\alpha}}\eqd
\fr{M_{\alpha}}{Z_{r,\mu}^{1/\alpha}}\,.\label{e22-kor}
\end{equation}
Из соотношения~(\ref{e22-kor}) вытекает, что предельные распределения для
смешанных геометрических сумм неотрицательных независимых с.в.\
являются масштаб\-ными смесями распределений Мит\-таг--Леф\-фле\-ра (как уже
отмечалось, предельных для <<обычных>> геометрических случайных сумм
не\-от\-ри\-ца\-тель\-ных с.в.), в~которых смешивающей является плотность
$$
p_{\alpha}(x;r,\mu)=\fr{\alpha\mu^r}{\Gamma(1-r)\Gamma(r)}\,
\fr{1}{(x^{\alpha}-\mu)^rx}\,,\enskip
x\ge\mu^{1/\alpha}\,,
$$
с.в.~$Z_{r,\mu}^{1/\alpha}$. Интегральное представление плот\-ности
распределения Мит\-таг--Леф\-фле\-ра приведено во введении. Пре\-об\-ра\-зо\-ва\-ние
Лап\-ла\-са--Стил\-ть\-еса с.в.\
в~правой части~(\ref{e22-kor}) имеет вид:
\begin{multline*}
{\sf E}\exp\{-sM_{\alpha}Z_{r,\mu}^{-1/\alpha}\}=
\int\limits_{0}^{\infty}\frac{z^{1/\alpha}p(z;r,\mu)\,dz}{z^{1/\alpha}+s^{\alpha}}={}\\
{}=
\int\limits_{0}^{\infty}\fr{zp_{\alpha}(z;r,\mu)\,dz}{z+s^{\alpha}}\,,\enskip
s>0\,.
\end{multline*}

В терминах статистических закономерностей процессов выпадения
осадков предельное гам\-ма-рас\-пре\-де\-ле\-ние в~следствии~4 может служить
асимптотической аппроксимацией распределения суммарного объема
осадков, выпавших в~течение одно\-го <<продолжительного>> ($p_n\hm\to0$)
дождливого периода, если средние арифметические ежедневных осадков
относительно нестабильны.

\smallskip

\noindent
\textbf{Замечание~3.} Сходимость
$n^{-1/\alpha}S_{N_{r,p_n}}\Longrightarrow T_{\alpha,1}\cdot
G_{r,\mu}^{1/\alpha}$ имеет место и~для $r\hm\ge 1$.

\smallskip

\noindent
\textbf{Замечание~4.} Использовав формулу для моментов строго
устойчивых распределений, сосредоточенных на неотрицательной полуоси
(см.\ подразд.~1.3), для моментов порядков $\beta\hm<\alpha$ с.в.,
предельных в~следствии~5, получим представление:
$$
{\sf E}\left(T_{\alpha,1}G_{r,\mu}^{1/\alpha}\right)^{\beta}=
\fr{\Gamma\left(({\beta+\alpha r})/{\alpha}\right)\Gamma
\left(({\alpha-\beta})/{\alpha}\right)}
{\mu^{\beta/\alpha}\Gamma(1-\beta)\Gamma(r)}\,.
$$


\subsection{Суммы асимптотически симметричных слагаемых. Случай~тяжелых~хвостов}

В отличие от ситуации, рассмотренной в~предыду\-щем подразделе, здесь
будет предполагаться, что слагаемые в~суммах могут принимать
значения обоих знаков и, более того, являются асимптотически
симметричными в~том смысле, что $\mathcal{L}(X_1)\hm\in
\mathrm{DNA}\,(F_{\alpha,0})$. В~работе~\cite{Korolev2016TVP} доказано
сле\-ду\-ющее утверждение.

\smallskip

\noindent
\textbf{Теорема~7}~\cite{Korolev2016TVP}. \textit{Предположим, что
независимые одинаково распределенные неотрицательные с.в.\
$X_1,X_2,\ldots$ таковы, что $\mathcal{L}(X_1)\hm\in \mathrm{DNA}\,(F_{\alpha,0})$
при некотором $\alpha\hm\in(0,2]$. Пусть при каждом $n\hm\in\mathbb{N}$ с.в.~$N_n$ 
имеют $Y_n$-сме\-шан\-ное геометрическое распределение 
и~независимы от~$X_1,X_2,\ldots$ Предположим, что существует с.в.~$Z$
такая, что ${\sf P}(0\hm<Z\hm<\infty)\hm=1$ и~при $n\hm\to\infty$ выполнено
условие}~(\ref{e15-kor}). \textit{Тогда}
\begin{equation}
\fr{S_{N_n}}{n^{1/\alpha}}\Longrightarrow T_{\alpha,0}
\left(\fr{W_1}{Z}\right)^{1/\alpha}\enskip (n\to\infty)\,,
\label{e23-kor}
\end{equation}
\textit{причем с.в.\ в~правой части}~(\ref{e23-kor}) \textit{независимы.}

\smallskip

Легко видеть, что х.ф.\ с.в.~$T_{\alpha,0}W_1^{1/\alpha}$ имеет вид:
\begin{multline}
{\sf E}\exp\{isT_{\alpha,0}W_1^{1/\alpha}\}={}\\
{}={\sf E}{\sf E}
\left(\exp\{isT_{\alpha,0}W_1^{1/\alpha}\}\big|W_1\right)=
{}\\
{}=\int\limits_{0}^{\infty}e^{i(|s|z^{1/\alpha})^{\alpha}}e^{-z}\,dz=
\int\limits_{0}^{\infty}e^{iz(|s|^{\alpha}+1)}\,dz=\fr{1}{1+|s|^{\alpha}}\,,\\
s\in\mathbb{R}\,.\label{e24-kor}
\end{multline}
Распределения с~х.ф.~(\ref{e24-kor}) и~$0\hm<\alpha\hm\le2$ принято называть {\it
распределениями Линника}. Они были введены Ю.\,В.~Линником в~1953~г.~\cite{Linnik1953}. 
При $\alpha\hm=2$ распределение Линника превращается
в распределение Лапласа. Случайная величина, имеющая распределение Линника 
с~параметром~$\alpha$, ее ф.р.\ и~плот\-ность будут соответственно
обозначаться~$L_{\alpha}$, $F_{\alpha}^{L}$ и~$f_{\alpha}^{L}$. При
этом $F_2^{L}(x)\hm\equiv F^{\Lambda}(x)$, $x\hm\in\mathbb{R}$.

Распределения Линника обладают многими интересными свойствами.
Прежде всего, как и~распределения Мит\-таг--Леф\-фле\-ра, они являются
геометри\-чески устойчивыми, т.\,е.\ если $X_1,X_2,\ldots$~--- независимые
одинаково распределенные с.в., причем $\mathcal{L}(X_1)\hm\in
\mathrm{DNA}\,(G_{\alpha,0})$, то при надлежащем выборе положительных
постоянных~$a_p$ распределения нормированных геометрических
случайных сумм $a_p(X_1+\cdots+X_{V_p})$ сходятся к~распределению
Линника с~параметром~$\alpha$. Распределения Линника унимодальны~\cite{Laha1961}, 
безгранично делимы~\cite{Devroye1990}, имеют
бесконечный пик плотности в~нуле при $\alpha\hm\le1$~\cite{Devroye1990} и~т.\,п. 
Аналитические и~асимптотические свойства
распределения Линника рассмотрены в~\cite{KorolevZeifman2016b,
KorolevZeifman2016a, KotzOstrovskiiHayfavi1995a,
KotzOstrovskiiHayfavi1995b}. В~част\-ности, 
в~работах~\cite{KorolevZeifman2016b, KotzOstrovskii1996} установлена
интересная связь между распределениями Линника, Лап\-ла\-са и~Мит\-таг--Леф\-фле\-ра и~показано, что
\begin{equation}
L_{\alpha}\eqd X\sqrt{2M_{\alpha/2}}\eqd
\Lambda\sqrt{R_{\alpha/2}}\,,\label{e25-kor}
\end{equation}
где все сомножители независимы, а~с.в.~$R_{\alpha/2}$ определена в~лемме~4.

Из теоремы~7, следствия~2 и~леммы~1 непосредственно вытекает
следующее утверждение.

\smallskip

\noindent
\textbf{Следствие~6.} Предположим, что независимые одинаково
распределенные неотрицательные с.в. $X_1,X_2,\ldots$ таковы, что
$\mathcal{L}(X_1)\hm\in \mathrm{DNA}\,(F_{\alpha,0})$ при некотором
$\alpha\hm\in(0,2)$. Пусть числа $r\hm\in(0,1)$, $q\hm\in(0,1)$ и~$\mu\hm>0$
произвольны. Пусть при каждом $n\hm\in\mathbb{N}$ с.в.~$N_{r,p_n}$
имеют отрицательные биномиальные распределения с~параметрами~$r$ 
и~$p_n\hm=\min\{q,\mu/n\}$ и~независимы от $X_1,X_2,\ldots$ Тогда
\begin{equation}
\fr{S_{N_{r,p_n}}}{n^{1/\alpha}}\Longrightarrow T_{\alpha,0}\cdot
G_{r,\mu}^{1/\alpha}\label{e26-kor}
\end{equation}
при $n\to\infty$.

\smallskip

С учетом~(\ref{e1-kor}) и~(\ref{e25-kor}) с.в., предельная в~следствии~6, может быть
записана в~разных эквивалентных формах:
\begin{multline*}
T_{\alpha,0} G_{r,\mu}^{1/\alpha}\eqd T_{\alpha,0}
G^*_{r,\alpha,\mu}\eqd{}\\
{}\eqd X 
\sqrt{2T_{\alpha/2,1}G_{r,\mu}^{2/\alpha}}\eqd
X \sqrt{2T_{\alpha/1,1}G^*_{r,\alpha/2,\mu}}\eqd{}\\
{}\eqd
X \fr{\sqrt{2M_{\alpha/2}}}{Z_{r,\mu}^{1/\alpha}}
\eqd\Lambda \fr{\sqrt{R_{\alpha/2}}}{Z_{r,\mu}^{1/\alpha}}\eqd
L_{\alpha} Z_{r,\mu}^{-1/\alpha},
\end{multline*}
т.\,е.\ распределение, предельное в~следствии~6, допускает
представления в~виде масштабной смеси как симметричного строго
устойчивого распределения, так и~распределения Линника, или
нормального закона, или распределения Лапласа.

С учетом абсолютной непрерывности предельного распределения 
в~следствии~6 можно заключить, что на самом деле в~условии~(\ref{e26-kor}) речь
идет о~равномерной сходимости ф.р.:
\begin{multline*}
\lim\limits_{n\to\infty}\sup\limits_{x\ge0}\left\vert 
\vphantom{\int\limits_{0}^{x}}
{\sf P}
\left(\fr{S_{N_{r,p_n}}}{n^{1/\alpha}}<x\right)-{}\right.\\
\left.{}-\int\limits_{0}^{x}F_{\alpha,0}\left(\fr{x}{z^{1/\alpha}}\right)g(z;\,r,\mu)\,dz
\right\vert=0\,.
\end{multline*}

\smallskip

\noindent
\textbf{Замечание~5.} Сходимость
$$
n^{-1/\alpha}S_{N_{r,p_n}}\Longrightarrow T_{\alpha,0}
G_{r,\mu}^{1/\alpha}\eqd T_{\alpha,0} G^*_{r,\alpha,\mu}
$$ 
имеет
место и~для $r\hm\ge 1$.

\smallskip

\noindent
\textbf{Замечание~6.} Использовав формулу для абсолютных моментов
симметричных строго устойчивых распределений (см.\ подразд.~1.3), для
моментов порядков $\beta\hm<\alpha$ с.в., предельных в~следствии~6,
получим представление:
\begin{multline*}
{\sf E}\left(|T_{\alpha,0}|G_{r,\mu}^{1/\alpha}\right)^{\beta}={}\\
{}=
\fr{2^{\beta}\Gamma\left(({\beta+\alpha r})/{\alpha}\right)\Gamma
\left(({\alpha-\beta})/{\alpha}\right)\Gamma\left(({\beta+1})/{2}\right)}
{\sqrt{\pi}\,\mu^{\beta/\alpha}\Gamma(r)\Gamma\left(({2-\beta})/{\beta}\right)}\,.
\end{multline*}


\subsection{Центральная предельная теорема для~отрицательных биномиальных 
случайных сумм независимых одинаково распределенных
случайных величин}

В приведенных выше рассуждениях особый интерес представляет случай
$\alpha\hm=2$. Хотя соответствующий вариант следствия~6 можно получить
прос\-той заменой~$\alpha$ на~2 в~его формулировке, здесь этот случай
будет рассмотрен особо, поскольку соответствующее утверждение можно
трактовать как центральную предельную теорему для отрицательных
биномиальных случайных сумм~$S_{N_{r,p_n}}$, когда
$p_n\hm\to0$ при $n\hm\to\infty$.

Итак, пусть $X_1,X_2,\ldots$~--- независимые одинаково распределенные с.в.\
 с~${\sf E}X_1\hm=0$ и~${\sf E}X_1^2\hm=1$. Известно, что в~общем случае
в аналогах центральной предельной теоремы для отрицательных
биномиальных случайных сумм в~качестве предельных законов возникают
так называемые VG-рас\-пре\-де\-ле\-ния (Variance Gamma distributions)~---
специальные дис\-пер\-си\-он\-но-сдви\-го\-вые смеси нормальных законов, 
в~которых смешивающими являются гам\-ма-рас\-пре\-де\-ле\-ния (см., 
например,~\cite{CarrMadanChang1998, Korolev2013}). В~рассматриваемом здесь
частном случае $r\hm<1$ можно предложить еще одну версию условий и~еще
одну форму записи предельных VG-рас\-пре\-де\-ле\-ний в~виде смеси
распределений Лапласа, что позволяет при статистическом оценивании
параметров предельных законов использовать медианные версии
ЕМ (expectation-maximization) ал\-го\-рит\-ма, более устойчивые к~исходным данным (см., например,~\cite{KGT2008}).

Из теоремы~7 с~$\alpha\hm=2$, следствия~2 и~леммы~1 непосредственно
вытекает следующее утверждение.

\smallskip

\noindent
\textbf{Следствие~7.} Предположим, что независимые одинаково
распределенные с.в.~$X_1,X_2,\ldots$ таковы, что ${\sf E}X_1\hm=0$ 
и~${\sf E}X_1^2\hm=1$. Пусть числа $r\hm\in(0,1)$, $q\hm\in(0,1)$ и~$\mu\hm>0$
произвольны. Пусть при каждом $n\hm\in\mathbb{N}$ с.в.~$N_{r,p_n}$
имеют отрицательные биномиальные распределения с~параметрами~$r$ 
и~$p_n\hm=\min\{q,\mu/n\}$ и~независимы от $X_1,X_2,\ldots$ Тогда
\begin{equation}
\fr{S_{N_{r,p_n}}}{\sqrt{n}}\Longrightarrow
X \sqrt{G_{r,\mu/2}}\eqd X  G^*_{r,2,\mu/2}\eqd
\fr{\Lambda}{\sqrt{Z_{r,\mu}}}\label{e27-kor}
\end{equation}
при $n\to\infty$. Более того, сходимость ф.р.\ с.в., участвующих в~(\ref{e27-kor}), 
является равномерной:
\begin{multline*}
\lim\limits_{n\to\infty}\sup\limits_x\left\vert
\vphantom{\int\limits_{0}^{\infty}}
 {\sf P}
\left(\fr{S_{N_{r,p_n}}}{\sqrt{n}}<x\right)
-{}\right.\\
\left.{}-\int\limits_{0}^{\infty}\Phi\left(\fr{x}{\sqrt{z}}\right)g(z;r,\mu/2)\,dz\right\vert={}
\\
{}=\lim\limits_{n\to\infty}\sup\limits_x\left\vert
\vphantom{\int\limits_{0}^{\infty}}
{\sf P}
\left(\fr{S_{N_{r,p_n}}}{\sqrt{n}}<x\right)
-{}\right.\\
\left.{}-\int\limits_{0}^{\infty}F^{\Lambda}\left(x\sqrt{z}\right)p(z;r,\mu)\,dz
\right\vert=0\,.
\end{multline*}
Другими словами, предельное VG-рас\-пре\-де\-ле\-ние в~данном случае
является масштабной смесью распределений Лапласа, в~которой
смешивающим служит распределение с.в.~$Z_{r,\mu}$.

\smallskip

\noindent
\textbf{Замечание~7.} Сходимость 
$$
n^{-1/2}S_{N_{r,p_n}}\Longrightarrow
X\sqrt{G_{r,\mu/2}}\eqd X G^*_{r,2,\mu/2}
$$ 
имеет место и~для $r\hm\ge 1$.

{\small\frenchspacing
 {%\baselineskip=10.8pt
 \addcontentsline{toc}{section}{References}
 \begin{thebibliography}{99}


\bibitem{Zolina2013} 
\Au{Zolina O., Simmer~C., Belyaev~K., Gulev~S., Koltermann~P.} 
Changes in the duration of European wet and dry spells during
the last~60~years~// J.~Climate, 2013. Vol.~26. P.~2022--2047.

\bibitem{Gulev} 
\Au{Korolev V.\,Yu., Gorshenin~A.\,K., Gulev~S.\,K., Belyaev~K.\,P.,
Grusho~A.\,A.} Statistical analysis of precipitation events~//
AIP Conf. Proc., 2017.
Vol.~1863. Iss.~1. doi: 10.1063/1.4992276.

\bibitem{Kingman1993} 
\Au{Kingman J.\,F.\,C.} Poisson processes.~--- Oxford: Clarendon Press, 1993.
104~p.

\bibitem{KorolevBeningShorgin2011} 
\Au{Королев В.\,Ю., Бенинг~В.\,Е., Шоргин~С.\,Я.} 
Математические основы теории риска.~--- 2-е изд.~--- М.: Физматлит, 2011.
591~с.

\bibitem{Korolev2016TVP}
\Au{Королев В.\,Ю.} Предельные распределения для
дваж\-ды стохастически прореженных процессов вос\-ста\-нов\-ле\-ния и~их
свойства~// Теория вероятностей и~ее применения, 2016. Т.~61. Вып.~4. С.~753--773.

\bibitem{KorolevPoisson}
\Au{Королев В.\,Ю., Корчагин А.\,Ю., Зейфман~А.\,И.} Теорема Пуассона
для схемы испытаний Бернулли со случайной вероятностью успеха 
и~дискретный аналог распределения Вейбулла~// Информатика и~её
применения, 2016. Т.~10. Вып.~4. С.~11--20.

\bibitem{Korolev2016} 
\Au{Korolev V.\,Yu., Korchagin~A.\,Yu., Zeifman~A.\,I.} On doubly
stochastic rarefaction of renewal processes~//  
AIP Conf. Proc., 2017. Vol.~1863. Iss.~1. doi: 10.1063/1.4992275.

\bibitem{Gleser1989} 
\Au{Gleser L.\,J.} 
The gamma distribution as a mixture of exponential distributions~//
Am. Stat., 1989. Vol.~43. P.~115--117.

\bibitem{GnedenkoKorolev1996} 
\Au{Gnedenko B.\,V., Korolev~V.\,Yu.} Random summation:
Limit theorems and applications.~--- Boca Raton: CRC Press, 1996.
267~p.

\bibitem{Stacy1962} 
\Au{Stacy E.\,W.} A~generalization of the gamma
distribution~// Ann. Math. Stat., 1962. Vol.~33.
P.~1187--1192.

\bibitem{KorolevZaks2013}
\Au{Закс Л.\,М., Королев~В.\,Ю.} Обобщенные дисперсионные
гамма-распределения как предельные для случайных сумм~// 
Информатика и~её применения, 2013. Т.~7. Вып.~1. С.~105--115.

\bibitem{Zolotarev1983} 
\Au{Золотарев В.\,М.} Одномерные устойчивые распределения.~--- М.: Наука, 1983.
304~с.

\bibitem{Schneider1986} 
\Au{Schneider W.\,R.} Stable distributions: Fox
function representationand generalization~// 
Stochastic processes in classical and quantum systems~/ Eds. 
S.~Albeverio, G.~Casati, D.~Merlini.~--- Berlin: Springer, 1986. P.~497--511.

\bibitem{UchaikinZolotarev1999} 
\Au{Uchaikin V.\,V., Zolotarev~V.\,M.}
Chance and stability.~--- Utrecht: VSP, 1999. 570~p.

\bibitem{KorolevWeibull2016} 
\Au{Korolev V.\,M.} Product representations for random variables with 
Weibull distributions and their applications~// J.~Math. Sci., 
2016. Vol.~218. No.\,3. P.~298--313.

\bibitem{Tucker1975} 
\Au{Tucker H.} On moments of distribution functions
attracted to stable laws~// Houston J.~Math., 1975.
Vol.~1. No.\,1. P.~149--152.

\bibitem{KorolevZeifman2016b} 
\Au{Korolev V.\,Yu., Zeifman~A.\,I.} 
Convergence of statistics constructed from samples with
random sizes to the Linnik and Mittag--Leffler distributions and
their generalizations~// J.~Korean Stat. Soc.,
2017. Vol.~46. P.~161--181.

\bibitem{Kilbas2014} 
\Au{Gorenflo R., Kilbas~A.\,A., Mainardi~F., Rogosin~S.\,V.}
Mittag--Leffler functions, related topics and applications.~---
Berlin\,--\,New York: Springer, 2014. 443~p.

\columnbreak

\bibitem{Bunge1996} 
\Au{Bunge J.} Compositions semigroups and random stability~//
Ann. Probab., 1996. Vol.~24. P.~1476--1489.

\bibitem{KlebanovRachev1996} {\it Klebanov L. B., Rachev S. T.} Sums of a random number of random variables and their approximations with $\varepsilon$-accompanying
infinitely divisible laws~// Serdica, 1996. Vol.~22. P.~471--498.

\bibitem{Kovalenko1965} 
\Au{Коваленко И.\,Н.} О~классе предельных распределений для
редеющих потоков однородных событий~// Литовский математический
сборник, 1965. Т.~5. Вып.~4. С.~569--573.

\bibitem{GnedenkoKovalenko1968} 
\Au{Gnedenko B.\,V., Kovalenko~I.\,N.}
 Introduction to queueing theory.~--- Jerusalem: Israel Program for 
 Scientific Translations, 1968. 281~p.

\bibitem{GnedenkoKovalenko1989} 
\Au{Gnedenko B.\,V., Kovalenko~I.\,N.} 
Introduction to queueing theory.~--- 2nd ed.~--- Boston: Birkhauser, 1989.
314~p.

\bibitem{Pillai1989} 
\Au{Pillai R.\,N.} Harmonic mixtures and geometric infinite
divisibility~// J.~Indian Stat. Assoc., 1990.
Vol.~28. P.~87--98.

\bibitem{Pillai1990} 
\Au{Pillai R.\,N.} On Mittag--Leffler functions and related
distributions~// Ann. Stat. Math.,
1990. Vol.~42. P.~157--161.

\bibitem{WeronKotulski1996} 
\Au{Weron K., Kotulski~M.} 
On the Cole-Cole relaxation function and
related Mittag--Leffler distributions~// Physica~A, 1996. Vol.~232.
P.~180--188.

\bibitem{GorenfloMainardi2006} 
\Au{Gorenflo R., Mainardi~F.} 
Continuous time random walk,
Mittag--Leffler waiting time and fractional diffusion: Mathematical
aspects. Ch.~4.~// Anomalous transport: Foundations and applications~/ 
Eds. R.~Klages, G.~Ra\-dons, I.\,M.~Sokolov.~--- Weinheim, Germany: 
Wiley-VCH, 2008.   P.~93--127.
{\sf http://arxiv.org/abs/0705.0797}.

\bibitem{GreenwoodYule1920}
\Au{Greenwood M., Yule~G.\,U.} An inquiry into the nature of
frequency-distributions of multiple happenings, etc.~// J.~R.
Stat. Soc., 1920. Vol.~83. P.~255--279.

\bibitem{Bolshev} 
\Au{Большев Л.\,Н., Смирнов~Н.\,В.} Таблицы
математической статистики.~--- 3-е изд.~--- М.: Наука, 1983.
416~c.

\bibitem{Bernstein1928} 
\Au{Bernstein S.\,N.} Sur les fonctions absolument
monotones~// Acta Math., 1929. Vol.~52. Iss.~1. P.~1--66.

\bibitem{Grandell1997}
\Au{Grandell~J.} Mixed Poisson processes.~--- London: Chapman and
Hall, 1997. 268~p.

\bibitem{KotzOstrovskii1996} 
\Au{Kotz S., Ostrovskii~I.\,V.} 
A~mixture representation of the Linnik distribution~// 
Stat. Probabil. Lett., 1996. Vol.~26. P.~61--64.

\bibitem{KorolevZeifman2016a} 
\Au{Korolev V.\,Yu., Zeifman~A.\,I.} 
A~note on mixture representations for the Linnik and Mittag--Leffler
distributions and their applications~// J.~Math. Sci., 2017. 
Vol.~218. No.\,3. P.~314--327.

\bibitem{Linnik1953}
\Au{Линник Ю.\,В.} Линейные формы и~статистические
критерии. I, II~// Украинский математический журнал, 1953. Т.~5.
Вып.~2. С.~207--243; Вып.~3. С.~247--290.

\bibitem{Laha1961} 
\Au{Laha R.\,G.} On a class of unimodal distributions~// 
P. Am. Math. Soc., 1961. Vol.~12. P.~181--184.

\bibitem{Devroye1990} 
\Au{Devroye L.} A~note on Linnik's distribution~// 
Stat. Probabil. Lett., 1990. Vol.~9. P.~305--306.


\bibitem{KotzOstrovskiiHayfavi1995a}
\Au{Kotz S., Ostrovskii~I.\,V., Hayfavi~A.} Analytic and asymptotic
properties of Linnik's probability densities, I~// 
J.~Mathematical Analysis Appl., 1995. Vol.~193. P.~353--371.

\pagebreak

\bibitem{KotzOstrovskiiHayfavi1995b} 
\Au{Kotz S., Ostrovskii~I.\,V., Hayfavi~A.} 
Analytic and asymptotic
properties of Linnik's probability densities, II~// J.~Math. Anal. 
Appl., 1995. Vol.~193. P.~497--521.

\bibitem{CarrMadanChang1998}
\Au{Carr P.\,P., Madan~D.\,B., Chang~E.\,C.} The variance gamma
process and option pricing~// Eur. Financ. Rev., 1998. Vol.~2.
P.~79--105.

\bibitem{Korolev2013} 
\Au{Королев В.\,Ю.} Обобщенные гиперболические законы как
предельные распределения для случайных сумм~// Теория вероятностей 
и~ее применения, 2013. Т.~58. Вып.~1. С.~117--132.

\bibitem{KGT2008} 
\Au{Горшенин А.\,К., Королев~В.\,Ю., Турсунбаев~А.\,М.} Медианные
модификации EM- и~SEM-ал\-го\-рит\-мов для разделения смесей вероятностных
распределений и~их применение к~декомпозиции волатильности
финансовых индексов~// Информатика и~её применения, 2008. Т.~2. Вып.~4. С.~12--47.
 \end{thebibliography}

 }
 }

\end{multicols}

\vspace*{-3pt}

\hfill{\small\textit{Поступила в~редакцию 11.05.17}}

\vspace*{8pt}

%\newpage

%\vspace*{-24pt}

\hrule

\vspace*{2pt}

\hrule

%\vspace*{8pt}


\def\tit{ANALOGS OF GLESER'S THEOREM FOR~NEGATIVE BINOMIAL 
AND~GENERALIZED GAMMA DISTRIBUTIONS AND~SOME~OF~THEIR~APPLICATIONS}

\def\titkol{Analogs of Gleser's theorem for~negative binomial 
and~generalized gamma distributions and~some of their applications}

\def\aut{V.\,Yu.~Korolev$^{1,2,3}$ }

\def\autkol{V.\,Yu.~Korolev}

\titel{\tit}{\aut}{\autkol}{\titkol}

\vspace*{-9pt}


\noindent
$^1$Faculty of Computational Mathematics and Cybernetics, 
M.\,V.~Lomonosov Moscow State University, 1-52~Lenin-\linebreak
$\hphantom{^1}$skiye Gory, Moscow 119991, 
GSP-1, Russian Federation 

\noindent
$^2$Institute of Informatics Problems, Federal Research Center ``Computer 
Science and Control'' of the Russian\linebreak
$\hphantom{^1}$Academy of Sciences, 44-2~Vavilov Str., 
Moscow 119333, Russian Federation

\noindent
$^3$Hangzhou Dianzi University, Higher Education Zone, Hangzhou~310018, China



\def\leftfootline{\small{\textbf{\thepage}
\hfill INFORMATIKA I EE PRIMENENIYA~--- INFORMATICS AND
APPLICATIONS\ \ \ 2017\ \ \ volume~11\ \ \ issue\ 3}
}%
 \def\rightfootline{\small{INFORMATIKA I EE PRIMENENIYA~---
INFORMATICS AND APPLICATIONS\ \ \ 2017\ \ \ volume~11\ \ \ issue\ 3
\hfill \textbf{\thepage}}}

\vspace*{6pt}


\Abste{It is proved that the negative binomial distributions with the shape 
parameter less than one are mixed geometric distributions. The mixing 
distribution is written out explicitly. Thus, the similar result of L.~Gleser, 
stating that the gamma distributions with the shape parameter less than 
one are mixed exponential distributions, is transferred to the discrete case. 
An analog of Gleser's theorem is also proved for generalized gamma distributions. 
For mixed binomial distributions related to the negative binomial laws with 
the shape parameter less than one, the case of a small probability of success 
is considered and an analog of the Poisson theorem is proved. The representation
 of the negative binomial distributions as mixed geometric laws is used 
 to prove limit theorems for negative binomial random sums of independent 
 identically distributed random variables, in particular, analogs of the 
 law of large numbers and the central limit theorem. Both cases of light 
 and heavy tails are considered. The expressions for the moments of limit 
 distributions are obtained. The obtained alternative equivalent mixture 
 representations of the limit laws provide better understanding of how mixed 
 probability (Bayesian) models are formed.}

\KWE{negative binomial distribution; mixed geometric distribution; 
generalized gamma distribution; stable distribution; Laplace distribution; 
Mittag--Leffler distribution; Linnik distribution; mixed binomial distribution; 
Poisson theorem; random sum; law of large numbers; central limit theorem}




\DOI{10.14357/19922264170301} 

%\vspace*{-14pt}

\Ack
\noindent
The research was partially supported by the RAS Presidium Program No.\,I.33P 
(project 0063-2016-0015) and by the Russian Foundation for Basic Research 
(projects Nos.\,15-07-04040 and~17-07-00717).



\vspace*{12pt}

  \begin{multicols}{2}

\renewcommand{\bibname}{\protect\rmfamily References}
%\renewcommand{\bibname}{\large\protect\rm References}

{\small\frenchspacing
 {%\baselineskip=10.8pt
 \addcontentsline{toc}{section}{References}
 \begin{thebibliography}{99}

\bibitem{1-kor-1}
\Aue{Zolina, O., C.~Simmer, K.~Be\-lya\-ev, S.~Gulev, and P.~Koltermann.} 
2013. Changes in the duration of European wet and dry spells during the last~60~years. 
\textit{J.~Climate} 26:2022--2047.
\bibitem{2-kor-1}
\Aue{Korolev, V.\,Yu., A.\,K.~Gorshenin, S.\,K.~Gulev, K.\,P.~Be\-lya\-ev, and 
A.\,A.~Grusho.} 2017. 
Statistical  analysis\linebreak\vspace*{-12pt}

\columnbreak

\noindent
 of precipitation events. 
\textit{AIP Conf. Proc.} 1863(1). doi: 10.1063/1.4992276.
\bibitem{3-kor-1}
\Aue{Kingman, J.\,F.\,C.} 1993. \textit{Poisson processes}. Oxford: Clarendon Press. 104~p.
\bibitem{4-kor-1}
\Aue{Korolev, V.\,Yu., V.\,E.~Bening, and S.\,Ya.~Shorgin.} 
2011. \textit{Matematicheskie osnovy teorii riska}
[Mathematical fundamentals of risk theory]. 2nd ed. Moscow: Fizmatlit. 591~p.
{\looseness=-1

}

\pagebreak

\bibitem{5-kor-1}
\Aue{Korolev, V.\,Yu.} 2016. 
Predel'nye raspredeleniya dlya dvazhdy stohasticheski prorezhennykh 
protsessov vos\-sta\-nov\-le\-niya i~ikh svoystva [Limit 
distributions for doubly stochastically 
rarefied renewal processes and their properties]. 
\textit{Teoriya veroyatnostey i~ee primeneniya} [Theor. Probab. Appl.] 61(4):753--773.
\bibitem{6-kor-1}
\Aue{Korolev, V.\,Yu., A.\,Yu.~Korchagin, and A.\,I.~Zeifman}. 
2016. Teorema Puassona dlya skhemy ispytaniy Bernulli so sluchaynoy 
veroyatnost'yu uspekha i~diskretnyy analog raspredeleniya Veybulla 
[The Poisson theorem for the scheme of Bernoulli trials with a random probability
of success and a~discrete analog of the Weibull distribution]. 
\textit{Informatika i~ee Primeneniya~--- Inform. Appl.} 10(4):11--20.
\bibitem{7-kor-1}
\Aue{Korolev, V.\,Yu., A.\,Yu.~Korchagin, and A.\,I.~Zeifman.} 
2017. On doubly stochastic
rarefaction of renewal processes. 
\textit{AIP Conf. Proc.} 1863(1). doi: 10.1063/1.4992275.
\bibitem{8-kor-1}
\Aue{Gleser, L.\,J.} 1989. The gamma distribution as 
a~mixture of exponential distributions. Am. Stat. 43:115--117.
\bibitem{9-kor-1}
\Aue{Gnedenko, B.\,V., and V.\,Yu.~Korolev}. 1996. 
\textit{Random summation: Limit theorems and applications}. 
Boca Raton: CRC Press. 267~p.
\bibitem{10-kor-1}
\Aue{Stacy, E.\,W.} 1962. 
A~generalization of the gamma distribution. 
\textit{Ann. Math. Stat.} 33:1187--1192.
\bibitem{11-kor-1}
\Aue{Zaks, L.\,M., and V.\,Yu.~Korolev.} 2013. 
Obobshchennye dispersionnye gam\-ma-ras\-pre\-de\-le\-niya kak predel'nye 
dlya sluchaynykh summ [Generalized variance gamma distributions as limit 
laws for random sums]. \textit{Informatika i~ee Primeneniya~--- Inform. Appl.}
7(1):105--115.
\bibitem{12-kor-1}
\Aue{Zolotarev, V.\,M.} 1983. \textit{Odnomernye ustoychivye raspredeleniya} 
[One-dimensional stable distributions]. Moscow: Nauka. 304~p.
\bibitem{13-kor-1}
\Aue{Schneider, W.\,R.} 1986. 
Stable distributions: Fox function representation and generalization. 
\textit{Stochastic processes in classical and quantum systems}. 
Eds. S.~Albeverio, G.~Casati, and D.~Merlini. Berlin: Springer. P.~497--511.
\bibitem{14-kor-1}
\Aue{Uchaikin, V.\,V., and V.\,M.~Zolotarev.} 
1999. \textit{Chance and stability}. Utrecht: VSP. 570~p.
\bibitem{15-kor-1}
\Aue{Korolev, V.\,Yu.} 2016. Product representations for random variables with 
 Weibull distributions and their applications. 
\textit{J.~Math. Sci.} 218(3):298--313.
\bibitem{16-kor-1}
\Aue{Tucker, H.} 1975. On moments of distribution functions attracted to stable laws. 
\textit{Houston J.~Math.} 1(1):149--152.
\bibitem{17-kor-1}
\Aue{Korolev, V.\,Yu., and A.\,I.~Zeifman.} 2017. 
Convergence of statistics constructed from samples with random sizes to the Linnik 
and Mittag--Leffler distributions and their generalizations. 
\textit{J.~Korean Stat. Soc.} 46:161--181.
\bibitem{18-kor-1}
\Aue{Gorenflo, R., A.\,A.~Kilbas, F.~Mainardi, and S.\,V.~Rogosin}. 2014.
\textit{Mittag--Leffler functions, related topics and applications}. 
Berlin\,--\,New York: Springer. 443~p.
\bibitem{19-kor-1}
\Aue{Bunge, J.} 1996. Compositions semigroups and random stability. 
\textit{Ann. Probab.} 24:1476--1489.
\bibitem{20-kor-1}
\Aue{Klebanov, L.\,B., and S.\,T.~Rachev.} 1996. 
Sums of a random number of random variables and their approximations with 
$\varepsilon$-accompanying infinitely divisible laws. \textit{Serdica} 22:471--498.

%\columnbreak


\bibitem{21-kor-1}
\Aue{Kovalenko, I.\,N.} 1965. 
O~klasse predel'nykh raspredeleniy dlya redeyushchikh potokov odnorodnykh sobytiy 
[On the class of limit distributions for rarefying flows of homogeneous events]. 
\textit{Litovskiy matematicheskiy sbornik} [Lithuanian Math.~J.] 5(4):569--573.
\bibitem{22-kor-1}
\Aue{Gnedenko, B.\,V., and I.\,N.~Kovalenko.} 1968. 
\textit{Introduction to queueing theory}. Jerusalem: 
Israel Program for Scientific Translations. 281~p.
\bibitem{23-kor-1}
\Aue{Gnedenko, B.\,V., and I.\,N.~Kovalenko.} 1989. 
\textit{Introduction to queueing theory}. 2nd ed.
Boston: Birkhauser. 314~p.
\bibitem{24-kor-1}
\Aue{Pillai, R.\,N.} 1990. 
Harmonic mixtures and geometric infinite divisibility. \textit{J.~Indian
Stat. Assoc.} 28:87--98.
\bibitem{25-kor-1}
\Aue{Pillai, R.\,N.} 1990. On Mittag--Leffler functions and
 related distributions. \textit{Ann. Stat. Math.} 42:157--161.
\bibitem{26-kor-1}
\Aue{Weron, K., and M.~Kotulski}. 1996. 
On the Cole-Cole relaxation function and related Mittag--Leffler distributions. 
\textit{Physica A} 232:180--188.
\bibitem{27-kor-1}
\Aue{Gorenflo, R., and F.~Mainardi.} 2008. Continuous time random walk, 
Mittag--Leffler waiting time and fractional diffusion: Mathematical aspects. 
Ch.~4.
\textit{Anomalous transport: Foundations and applications}. 
Eds. R.~Klages, G.~Radons, and I.\,M.~Sokolov. Weinheim, Germany: 
Wiley-VCH.   93--127.
Available at: {\sf http://arxiv.org/abs/0705.0797}
(accessed August~25, 2017).
\bibitem{28-kor-1}
\Aue{Greenwood, M., and G.\,U.~Yule.} 1920. 
An inquiry into the nature of frequency-distributions of multiple happenings, etc. 
\textit{J.~R. Stat. Soc.} 83:255--279.
\bibitem{29-kor-1}
\Aue{Bol'shev, L.\,N., and N.\,V.~Smirnov.} 1983. 
\textit{Tablitsy ma\-te\-ma\-ti\-che\-skoy statistiki} 
[Tables of mathematical statistics].  3rd ed. Moscow: Nauka. 416~p.
\bibitem{30-kor-1}
\Aue{Bernstein, S.\,N.} 1929. Sur les fonctions absolument monotones. 
\textit{Acta Math.} 52(1):1--66.
\bibitem{31-kor-1}
\Aue{Grandell, J.} 1997. \textit{Mixed Poisson processes}. 
London: Chapman and Hall. 268~p.
\bibitem{32-kor-1}
\Aue{Kotz, S., and I.\,V.~Ostrovskii}. 1996. 
A~mixture representation of the Linnik distribution. 
\textit{Stat. Probabil. Lett.} 26:61--64.
\bibitem{33-kor-1}
\Aue{Korolev, V.\,Yu., and A.\,I.~Zeifman.} 2017. 
A~note on mixture representations for the Linnik and Mittag--Leffler distributions 
and their applications. \textit{J.~Math. Sci.} 218(3):314--327.
\bibitem{34-kor-1}
\Aue{Linnik, Yu.\,V.} 1953. Lineynye formy i~statisticheskie kriterii. I, II 
[Linear forms and statistical tests, I, II]. 
\textit{Ukrainskiy matematicheskiy zh.} [Ukrainian Math.~J.] 5(2):207--243; 
5(3):247--290.
\bibitem{35-kor-1}
\Aue{Laha, R.\,G.} 1961. On a class of unimodal distributions. 
\textit{P.~Am. Math. Soc.} 12:181--184.
\bibitem{36-kor-1}
\Aue{Devroye, L.} 1990. A~note on Linnik's distribution. 
\textit{Stat. Probabil. Lett.} 9:305--306.
\bibitem{37-kor-1}
\Aue{Kotz, S., I.\,V.~Ostrovskii, and A.~Hayfavi.} 1995. 
Analytic and asymptotic properties of Linnik's probability densities, I. 
\textit{J.~Math. Anal. Appl.} 193:353--371.

\pagebreak

\bibitem{38-kor-1}
\Aue{Kotz, S., I.\,V.~Ostrovskii, and A.~Hayfavi.} 1995. 
Analytic and asymptotic properties of Linnik's probability densities, II. 
\textit{J.~Math. Anal. Appl.} 193:497--521.



\bibitem{39-kor-1}
\Aue{Carr, P.\,P., D.\,B.~Madan, and E.\,C.~Chang.} 1998. 
The Variance Gamma process and option
pricing. \textit{Eur. Financ. Rev.} 2:79--105.



\bibitem{40-kor-1}
\Aue{Korolev, V.\,Yu.} 2014. 
Generalized hyperbolic laws as limit distributions for random sums. 
\textit{Theor. Probab. Appl.} 58(1):63--75.

\columnbreak 

\bibitem{41-kor-1}
\Aue{Gorshenin, A.\,K., V.\,Yu.~Korolev, and A.\,M.~Tursunbaev.} 2008. 
Mediannye modifikatsii EM- i~SEM-al\-go\-rit\-mov dlya razdeleniya smesey 
veroyatnostnykh raspredeleniy i~ikh primenenie 
k~dekompozitsii volatil'nosti finansovykh indeksov [Median modifications 
of the EM- and SEM-algorithms for the separation of mixtures of probability 
distributions and their application to the decomposition of volatility 
of financial indexes]. \textit{Informatika i~ee Primeneniya~--- Inform. Appl.} 
2(4):12--47.
\end{thebibliography}

 }
 }

\end{multicols}

\vspace*{-3pt}

\hfill{\small\textit{Received May 11, 2017}}


\Contrl

\noindent
\textbf{Korolev Victor Yu.} (b.\ 1954)~--- 
Doctor of Science in physics and mathematics, professor, Head of the Department 
of Mathematical Statistics, Faculty of Computational Mathematics and 
Cybernetics, Faculty of Computational Mathematics and Cybernetics, 
M.\,V.~Lomonosov Moscow State University, 1-52~Leninskiye Gory, GSP-1, Moscow 119991, 
Russian Federation; leading scientist, Institute of Informatics Problems, 
Federal Research Center ``Computer Science and Control''' 
of the Russian Academy of Sciences, 44-2~Vavilov Str., Moscow 119333,  
Russian Federation;  professor, Hangzhou Dianzi University, 
Xiasha Higher Education Zone, Hangzhou 310018, China; \mbox{vkorolev@cs.msu.su}


\label{end\stat}


\renewcommand{\bibname}{\protect\rm Литература} 