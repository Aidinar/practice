
\renewcommand{\figurename}{\protect\bf Figure}
\renewcommand{\tablename}{\protect\bf Table}

\def\stat{inkova}


\def\tit{STATISTICAL DATA AS INFORMATION SOURCE FOR~LINGUISTIC ANALYSIS OF~RUSSIAN 
CONNECTORS}

\def\titkol{Statistical data as information source for~linguistic analysis of~Russian 
connectors}

\def\autkol{O.~Inkova and~N.~Popkova}

\def\aut{O.~Inkova$^1$ and~N.~Popkova$^2$}

\titel{\tit}{\aut}{\autkol}{\titkol}

%{\renewcommand{\thefootnote}{\fnsymbol{footnote}}
%\footnotetext[1] {The 
%research of Yuri Kabanov was done under partial financial support   of the grant 
%of  RSF No.\,14-49-00079.}}

\renewcommand{\thefootnote}{\arabic{footnote}}
\footnotetext[1]{Institute of Informatics Problems, Federal Research Center 
``Computer Science and Control'' of the Russian 
Academy of Sciences,  44-2~Vavilov Str., Moscow 119333, Russian Federation;
\mbox{olyainkova@yandex.ru}}
\footnotetext[2]{Institute of Informatics Problems, Federal Research Center 
``Computer Science and Control'' of the Russian 
Academy of Sciences;  44-2~Vavilov Str., Moscow 119333, Russian Federation;
\mbox{natasha\_\_popkova@mail.ru}}


\index{Inkova O.}
\index{Popkova N.}
\index{Инькова О.\,Ю.}
\index{Попкова Н.\,А.}


\vspace*{-10pt}

\def\leftfootline{\small{\textbf{\thepage}
\hfill INFORMATIKA I EE PRIMENENIYA~--- INFORMATICS AND APPLICATIONS\ \ \ 2017\ \ \ volume~11\ \ \ issue\ 3}
}%
 \def\rightfootline{\small{INFORMATIKA I EE PRIMENENIYA~--- INFORMATICS AND APPLICATIONS\ \ \ 2017\ \ \ volume~11\ \ \ issue\ 3
\hfill \textbf{\thepage}}}




\Abste{The aim of this paper is to describe statistical data gathered from the supracorpora 
database (SCDB) of connectors for further analysis of their formal and functional 
properties. Until now, these properties have usually been described applying semantic 
analysis, while corpus data, if used at all, have not been subject to statistical processing. It is 
automatically generated and verifiable information, collected from texts corpora that 
can be 
one of the most reliable tools in the analysis of linguistic units, including connectors. The 
paper shows what statistics one may obtain from the SCDB and how to use it in the linguistic 
analysis in case of \textit{tol'ko}, a~polyfunctional linguistic unit that can 
be a~part of multicomponent and two-place connectors.}


\KWE{annotation of connectors; corpus linguistics; supracorpora databases; parallel texts; 
statistical data}

\DOI{10.14357/19922264170314} 


\vspace*{-4pt}


\vskip 12pt plus 9pt minus 6pt

      \thispagestyle{myheadings}

      \begin{multicols}{2}

                  \label{st\stat}

\section{Introduction}

  \noindent
  The paper aims to show what statistics the SCDB can 
provide in order to analyze formal and functional properties of 
connectors\footnote[3]{Connectors are the functional words of 
different grammatical classes (coordinate 
and subordinate conjunctions, some adverbs, prepositions, particles,
 and ``discourse markers''~--- 
linguistic units with complex grammatical structure) that serve 
to connect different parts of the text.}. 
Until now, semantic analysis has prevailed in the field, and corpus data, if used 
at all, have not been subject to statistical processing~[1--3]. The opportunity 
to apply corpus and statistical methods of analysis has radically changed the way 
linguistic research is perceived, as such methods allow one to enhance the 
reliability and validity of the results achieved. Since quantitative corpus data are 
generated automatically and, hence, can be easily verified, they may serve as one of 
the most reliable tools in the analysis of linguistic units.
  
  Therefore, electronic corpora evolve to have statistical tools. The most 
substantial corpus project in Russia, the ``Russian National Corpus'' (RNC, {\sf 
www.ruscorpora.ru}), offers a~wide range of data analysis tools that get updated on a~regular basis, with some of them having only recently become available. Being 
designed both for regular users and researchers, the RNC is a~representative corpus 
containing texts of different types and genres that date back from the 18th to the 
21st~century. The RNC uses a~metamarkup language that affords a~possibility of 
``marking up a~text with metatags so as to specify how it was created, its author, 
topic, genre particulars, etc.''~[4]. These tags determine which statistical tools 
could be available for the RNC users. Studying a~particular linguistic unit, 
users not only can  get the information about the number of tokens and texts in which the 
word has been used, but also they can find out what is the topic and the type of 
every single text, the date when the text was created and its author's name. The 
data, both in numbers and percentage, are displayed in tables. When the RNC user 
studies if and how a~word or a~word combination occurs in texts within a~certain 
timeframe, (s)he may get an automatically generated frequency graph created with 
the ``Graphs'' function. Together with graphs, the tables are generated where some 
numbers are hyperlinks. Frequency data are also available for downloading in ZIP 
archive format. Nonetheless, such statistical data provide information about texts 
where the linguistic unit in question occurs, and functional properties of the 
linguistic unit itself are only represented indirectly. It goes without saying that 
information about how frequently the unit occurs in texts of a~specific genre, or 
texts written by an author of a~specific sex does not suffice to describe a~connector, 
especially its formal properties.


  
  Another corpus project, referred to as ``Russkaya kor\-pus\-naya grammatika'' 
[Russian corpus grammar], aims ``to give a~synchronic description of 
a~representative fragment of Russian grammar, which would be grounded in the 
data collected from the RNC and use quantitative methods of 
corpus analysis''~[5]. In ``Conjunction'' ({\sf 
http://rusgram.ru/\%D0\%A1\%D0\%BE\%D1\%8E\%D0 \%B7\#4}), that is 
thought to be the most relevant chapter for the present study~[6], one can find data, both in 
numbers and percentage, on how frequently coordinate and subordinate 
conjunctions occur. The percentage indicates the portion of coordinate and 
subordinate conjunctions found in the RNC and the share of every conjunctive 
class in the overall number of conjunctions. Data on individual conjunctions within 
semantic classes are also available; though for some of them, e.\,g., substitutive 
conjunctions, no statistical data can be provided. 


  This paper describes what statistical functions the SCDB has got and what data 
they help to collect, which is exemplified by~834~annotations of 
\textit{tol'ko}\footnote{As of May~1, 2017, the SCDB contains parallel texts (mostly
 fiction) in 
Russian and French of~3.5~M tokens, 10,562~Russian-French annotations,
 and~909~French-Russian 
annotations.}. This particular word \textit{tol'ko} has been chosen for its 
polyfunctionality, i.\,e., it can act as an adverb, a~particle, or a~conjunction:
  \begin{itemize}
  \item Вы, видно, \textbf{только} проснулись [You seem to have \textbf{just} 
woken up] (adverb);
  \item  Каких \textbf{только} подарков он ей ни делал [\textbf{Just} think 
how many presents he gave her]! (particle);
  \item  \textbf{Только} он вошел, все встали [\textbf{Just} after he entered, 
everybody stood up] (conjunction).
  \end{itemize}
  
  What is more, \textit{tol'ko} can also act as a~part of multicomponent and  
two-place connectors, for example, \textbf{Как} только кончается отношение 
служебное, \textbf{так} кончается всякое другое [\textbf{As soon as} service 
relationship is over, any other relationship is over too] (L.~Tolstoy), where 
\textit{tol'ko} acts as a~part of the two-place connector \textit{kak tol'ko$\ldots$\ 
tak} [as soon as]. It is worth noting that existing electronic linguistic resources do 
not allow to run a~query for two-place linguistic units as one unit. Besides, such 
multicomponent units are characterized by a~high degree of formal variability. For 
instance, \textit{kak tol'ko$\ldots$\ tak} may also occur as \textit{kak 
tol'ko$\ldots$\ totchas zhe}: Он, \textbf{как только} проснулся, \textbf{тотчас же}
вознамерился встать, умыться 
[\textbf{As soon as} he woke up, he wanted to get up and wash 
himself] (I.~Goncharov). All the corpora we know are not designed to gather 
data on how multicomponent linguistic units function, in particular, those whose 
elements, as in the example above, are located separately.

\section{Statistical Analysis and~Electronic Linguistic Resources}

  \noindent
  The SCDB, a~new electronic linguistic resource, is a~superstructure built upon 
the RNC or, more precisely, upon its parallel Russian-French subcorpus. The 
database allows to retrieve correspondences between linguistic units from both 
languages and to annotate them, specifying their characteristics that are relevant 
for linguistic analysis. For more details on the SCDB of connectors, see~[7--10].
  
  In the SCDB, the initial information object to start the annotation process with is 
a~\textit{monoequivalence} (ME), defined as a~two-place tuple. Its structure is as 
follows: A~source text fragment containing the linguistic unit under study 
(Column~1, Fig.~1) and a~corresponding translation fragment (Column~2), 
containing its functionally equivalent fragment (FEF, the term coined in~[11]).
  
  The object annotated in the SCDB of connectors is referred to as 
\textit{discursive realization} (DR) or, in other words, the actual form of the 
linguistic unit in which it occurred in the text. Along with the source DR, its FEF 
also gets annotated, which makes the SCDB different from other electronic 
linguistic resources.
  
  Figure~1 shows the annotated DR \textit{kak tol'ko} and its FEF 
\textit{d$\grave{\mbox{e}}$s que}.

\end{multicols}

\begin{figure*}[h] %[h] %fig1
 \vspace*{14pt}
\begin{center}
\mbox{%
\epsfxsize=162.574mm
\epsfbox{ink-1.eps}
}
\end{center}
\vspace*{-11pt}
\Caption{An example of annotation}
\vspace*{-14pt}
\end{figure*}

    \begin{table*}\small %tabl1
  \begin{center}
  \Caption{The \textit{tol'ko} cluster: DRs and the frequency of their occurrence}
  \vspace*{2ex}
  
  \begin{tabular}{clc}
  \hline
&\multicolumn{1}{c}{DRs with \textit{tol'ko}}&
\tabcolsep=0pt\begin{tabular}{c}Frequency of occurrence\\ of the DR\end{tabular} \\
\hline
\hphantom{9}1&Только [just; once]&134\hphantom{9}\\
\hphantom{9}2&Не только (расстояние) но и  [not only (distance) but also]&113\hphantom{9}\\
\hphantom{9}3&Лишь только [hardly; scarcely; only when]&47\\
\hphantom{9}4&Как только [as soon as; once]&43\\
\hphantom{9}5&Не только (расстояние) но даже [not only (distance) but even]&31\\
\hphantom{9}6&Не только (расстояние) но [not only (distance) but]&21\\
\hphantom{9}7&Если только [if only]&17\\
\hphantom{9}8&Вот только [though; only; just]&16\\
\hphantom{9}9&Не только [not only]&16\\
10&Вот только (расстояние) $\varnothing$$^*$ [only/just (distance) $\varnothing$]&16\\
11&Не только (расстояние) и [not only (distance) and]&10\\
12&Не только (расстояние) даже [not only (distance) even]&\hphantom{9}9\\
13&Не (расстояние) только [only (distance) not; not only]&\hphantom{9}8\\
14&Только что [just]&\hphantom{9}8\\
15&Не только (расстояние) но даже~и [not only (distance) but also]&\hphantom{9}8\\
16&Не только (расстояние) но (расстояние)~и [not merely (distance) but (distance)  even]&\hphantom{9}8\\
17&Не только (расстояние) но и вообще [not only (distance) but generally]&\hphantom{9}6\\
18&Не только (расстояние)~а [not only [distance] but]&\hphantom{9}6\\
19&Как только (расстояние) тотчас же [just (distance) at once]&\hphantom{9}5\\
20 &А не только [and not only; and not merely]&\hphantom{9}5\\
\hline
  \multicolumn{3}{l}{\footnotesize \hspace*{3mm}$^*$The$\varnothing$ symbol means that the second 
component of a~two-place DR is not explicit.}
  \end{tabular}
  \end{center}
  \vspace*{6pt}
  \end{table*}
  

\begin{multicols}{2}

  Professional linguists use the SCDB to annotate connectors in three steps. First, 
they identify a~connector's DR, its FEF, and then build an~ME. At the same stage, %\linebreak\vspace*{-12pt}
%\noindent 
the DR is structurally analyzed and attributed to clusters of its components. For 
instance, the DR \textit{kak tol'ko} is assigned both to the \textit{tol'ko} cluster 
and to the \textit{kak} cluster (how clusters are populated is not shown in Fig.~1; 
however, the SCDB has got this option). Accordingly, these two clusters are 
supposed to include all the DRs that are present in the SCDB and contain 
\textit{tol'ko} (Table~1) and \textit{kak}. This annotation procedure allows 
(a)~to retrieve from the SCDB all the combinations of elements of DRs, (b)~to 
generate a~list of linguistic units in which a~certain element can act as a~component, 
and (c)~to find a~prototypical form for a~connector with a~high degree of formal 
variability based on how frequent its DRs are. For example, in the SCDB, there is 
a~vast variety of DRs with \textit{tol'ko} that have temporal meaning, such as: 
\textit{как только} [as soon as; once]; \textit{как только$\ldots$\ тут же} [as 
soon as]; \textit{как только$\ldots$\  тут~и} [as soon as]; \textit{как только$\ldots$\  
как раз~и} [just$\ldots$\  at once]; \textit{как только$\ldots$\  
так} [as soon as; at once; as soon$\ldots$\  so]; \textit{едва только$\ldots$\  как} 
[as soon as$\ldots$\  then; hardly$\ldots$ before]; \textit{лишь
только} [hardly; 
scarcely; only when]; \textit{лишь только$\ldots$\ как} [no sooner$\ldots$\  
when; as soon as]; \textit{лишь только$\ldots$\ лишь только} [as soon 
as$\ldots$\  as soon as]\footnote{The data from the RNC (namely, the Russian-English 
subcorpus) show that the two-place DRs with \textit{tol'ko} are often not translated in their entirety, only 
the first part of those DRs tends to get translated.}. For all these occurrences, the form 
\textit{kak tol'ko} has been established as prototypical, since more than~60\% of 
DRs are those with \textit{kak tol'ko}. Coupled with semantic analysis, this 
approach would help to tackle a~complicated theoretical question of whether two or 
more DRs with common elements are formal variations of one connector and if so, 
which is the prototypical form of this connector or connectors. The question has 
already been raised in~\cite{12-in, 13-in, 14-in}.
  
  During the second stage of the annotation process, linguists characterize the 
functioning of the DR in a~given context. These characteristics are also grouped 
into six clusters: \textbf{Relations}, \textbf{Structure}, \textbf{Position}, 
\textbf{Order}, \textbf{Status}, and \textbf{Disposition}.
  
The following characteristics have been assigned to the DR with \textit{tol'ko}:
\begin{itemize}
\item $\langle$\textbf{temporary}$\rangle$ means that the connector expresses temporary  
logical-semantic relation between situations \textit{q} (\textit{дело дошло
до <<честного слова>>} [\textit{it comes to ``my word of honour''}]) and 
\textit{p} (\textit{я махаю руками и сажусь за стол} [\textit{I~wave my 
hands and sit down at the table}]);
\item $\langle$\textbf{with predication}$\rangle$ means that the text 
fragment (\textit{дело дошло до <<честного слова>>} [\textit{it comes to 
``my word of honour''}]), marked by the connector \textit{kak tol'ko}, has 
a~predicative structure, i.\,e., subject (\textit{дело} [\textit{it}]) and predicate 
(\textit{дошло} [\textit{comes}]);
\item $\langle$\textbf{initial}$\rangle$ means that the connector \textit{kak 
tol'ko} occupies the initial position in the text fragment~$q$ (\textit{дело 
дошло до <<честного слова>>} [\textit{it comes to ``my word of honour''}]) 
it marks;
\item $\langle$\textbf{CNT q p}$\rangle$ indicates the order in which the 
text fragments marked by the connector come;
\item $\langle$\textbf{CNT}$\rangle$ means that the DR \textit{kak tol'ko} 
fulfils a~connective function in the text fragment, i.\,e., it acts as a~connector. In 
case of polyfunctional units like \textit{tol'ko}, the Status cluster and its 
components allow to register different functions of such units; and
\item $\langle$\textbf{Contact}$\rangle$ indicates that both components of 
the connector \textit{kak tol'ko} [as soon as] go one after another and 
are not separated by other words.
  \end{itemize}
  

  
  The FEF is annotated according to the same scheme. Column~4 in Fig.~1 
contains six characteristics of the FEF \textit{d$\grave{\mbox{e}}$s que} that 
coincide with those chosen for \textit{kak tol'ko}, although this is not always the 
case (see Fig.~3 below where characteristics do differ).
  
  At the third and final stage of the annotation process, characteristics of the ME 
itself are marked in Column~5. The \textit{NB} mark signals that MEs are of 
special interest for experts. \textit{Cngrn} and \textit{Dvrg} marks are used to 
describe translation type. When a~connector is translated by a~connector, congruent 
(\textit{Cngrn}) translation is marked, while divergent (\textit{Dvrg}) translation is 
marked when a~connector is translated by another linguistic unit or syntactic 
construction. Marks \textit{Type1}, \textit{Type2}, and \textit{Type3} are used 
to describe annotation type (see more in~\cite{9-in, 10-in, 15-in}). 
Eight hundred and thirty four Russian-French annotations of \textit{tol'ko} 
mentioned above were marked as 
\textit{Type1} since they have been made for the connector as a~whole and not for 
its separate components. Overall, 1,219~Russian-French annotations of 
\textit{tol'ko} of all three types have been registered. Moreover, 
59~French-Russian  annotations of \textit{tol'ko} have been made in the SCDB, with~50 of 
them being of Type1. Therefore, as of May~1, 2017, 1278 annotations for DRs 
with \textit{tol'ko} have been created in the SCDB.
{\looseness=1

}

%\vspace*{-6pt}
  
\section{Supracorpora Database Statistical Functions}

  \noindent
  The SCDB has been designed to have three functions in order to generate 
statistical data. First, it gives information about how frequent are the DRs, 
annotations of which are stored in the database. Second, it provides statistical data 
on every single occurrence of the DR as the queries of it are executed in the SCDB. 
Third, it allows one to see which patterns are used to translate linguistic units, 
annotated in the SCDB, and how frequently they occur.
  
  The first function extracts the data displayed in Table~1 that shows which DRs 
make up the \textit{tol'ko} cluster, it also shows the frequency of their occurrence. 
Table~1 demonstrates first~20~most frequent DRs, each of which has been 
annotated in the SCDB five times or more. The most frequent among them is DR 
\textit{tol'ko} with~134~MEs registered. Overall, the \textit{tol'ko} cluster 
includes~155~DRs.
  

  
  
  The information about the frequency of occurrence of DRs is automatically 
generated for every cluster of components. During the annotation process, the 
distribution within clusters is recalculated.

\begin{figure*} %fig2
 \vspace*{1pt}
\begin{center}
\mbox{%
\epsfxsize=138.435mm
\epsfbox{ink-2.eps}
}
\end{center}
%\vspace*{-11pt}
%\Caption{A~query template with parameters}
\end{figure*}
  
  The second function returns individual statistical data, received as a~result of queries 
that are executed in the SCDB. A~query template (Fig.~2) is used to set query 
parameters. Running the query, the user gets not only the number of annotations 
that comply with the parameters set, but also the list of these annotations that can 
be used subsequently for further analysis.
  
  In Fig.~2, the field \textbf{Кластер РР в оригинале} [Cluster of DRs in the 
source text] contains \textit{tol'ko}, i.\,e., only annotations of \textit{tol'ko} and 
those of the other~166~Russian DRs that contain it (such as \textit{kak tol'ko}, 
\textit{lish' tol'ko}, etc.)\ will be selected. Figure~2 shows that in the query, some 
filters are specified so as to restrict the set of texts and translations, whose 
fragments appear in the annotations. Moreover, there are three other filters:
  \begin{enumerate}[(1)]
\item DRs with \textit{tol'ko} must only express temporary relation;
\item DRs with \textit{tol'ko} mark a~part of a~sentence with a~predication, 
i.\,e., a~text fragment marked by a~DR must have a~predicative structure; and
\item annotations must be of Type~1.
\end{enumerate}

  There are 34 annotations satisfying such criteria. When the query is completed, 
annotations are stored in a~list. One of the annotations is depicted in Fig.~3.


  The third statistical function returns data on patterns used to translate the linguistic 
units under study in the text fragments that get annotated in the SCDB. With this 
function, one can evaluate the frequency of different translation patterns.


  
  For example, let us consider French translations of the DR \textit{tol'ko}. Overall, 35 
translation patterns have been identified, including the zero pattern, i.\,e., the absence 
of translation equivalent. These patterns appear in~137~annotations. Data on seven 
most frequent translation patterns are displayed in Table~2. Data on the frequency 
of occurrence of the remaining~28~patterns are reported in the penultimate row 
of~Table~2.
  
 
  
  What is of utmost importance is that numbers in Column~3 are hyperlinks 
leading to the lists of corresponding annotations. Hyperlinks help to visualize\linebreak\vspace*{-12pt}

\pagebreak

\end{multicols}

\addtocounter{figure}{1}
  
\begin{figure*} %fig3
 \vspace*{1pt}
\begin{center}
\mbox{%
\epsfxsize=161.557mm
\epsfbox{ink-3.eps}
}
\end{center}
\vspace*{-11pt}
\Caption{An annotation found in the query}
\vspace*{6pt}
\end{figure*}

  
  
   \begin{table*}\small %tabl2
  \begin{center}
  \Caption{\textit{Tol'ko}: Frequency distribution of translation patterns}
  \vspace*{2ex}
  
  \begin{tabular}{clcc}
  \hline
\tabcolsep=0pt\begin{tabular}{c}Russian\\ linguistic unit\end{tabular}&
\multicolumn{1}{c}{French translation pattern}&
\tabcolsep=0pt\begin{tabular}{c}Frequency of occurrence\\ of the translation\\ 
pattern in the SCDB\end{tabular}&
\tabcolsep=0pt\begin{tabular}{c}Frequency of occurrence\\ (percentage)\end{tabular}\\
\hline
&ne $\ldots$\ que&30&21.90\\
&zero&16&11.68\\
&seulement&15&10.95\\
tol'ko&seul&14&10.22\\
&mais&12&\hphantom{9}8.76\\
&sauf que&\hphantom{9}6&\hphantom{9}4.38\\
&juste&\hphantom{9}6&\hphantom{9}4.38\\
&Remaining 28~translation patterns&38&27.73\\
\hline
\multicolumn{2}{l}{Number of annotations} &137\hphantom{9}&100\\
\hline
  \end{tabular}
  \end{center}
  \end{table*}

\begin{multicols}{2}

\noindent 
statistics, make it verifiable, and using them gives the possibility to analyze the 
annotations after they have been selected. This is a~big advantage of the SCDB of 
connectors over other electronic linguistic resources. Our statistical data show that 
most often \textit{tol'ko} is translated by the restrictive negation \textit{ne \ldots 
que}, i.\,e., it is its most frequent FEF and appears in 21.9\% of annotations.

\section{Concluding Remarks}

\noindent
The SCDB of connectors opens up vast opportunities to get statistical data on 
functional and formal properties of connectors depending on the research 
objectives and purposes. The main advantage of such data is that they are 
representative and verifiable because of the hyperlinks going to the annotations one 
can find after the query is completed.

  Furthermore, unlike most data from other electronic linguistic resources, 
statistical data in the SCDB describe first and foremost the linguistic units 
themselves and not the texts where they occur. This results from a~fine elaboration 
of annotation parameters: in the SCDB, every parameter characterizes connectors 
either structurally or functionally. For instance, statistics for the Relations 
parameter, denoting the relation expressed by the connector, are not only important 
for the analysis, but also in case of polysemic connectors (vast majority of 
connectors) help to define which relation they tend to convey. Statistics for the 
Structure parameter allow to define the common syntactical structure of the 
connector. In future, these data will help to answer the question about how 
frequently the connector occurs at the interclausal level (i.\,e., in a~complex 
sentence) and at the intersentential level (i.\,e., between separate utterances). So 
far, the question has only been raised~\cite{16-in}, although it seems to be 
theoretically important for the description of linguistic means with a~connective 
function, including connectors.
  
  In this regard, the Order parameter also gives important information about how 
text fragments, linked by a~connector, are positioned. And such parameters as 
Position and Status are used, for example, to identify polyfunctional linguistic 
units, for which a~connective function is only one of the possible functions. The 
question about how to define functions of a~polyfunctional linguistic unit in an 
utterance remains open. In its turn, the Disposition parameter indicates if there 
could be any distance between elements of two-place and multicomponent 
connectors, which is a~specific functional property of these connectors. Such kind 
of statistics allows, for instance, to answer the so far unanswered question: ``Is 
a~DR an independent linguistic unit or is it a~more or less free combination of 
components?''
  
  Thus, the linguistic analysis of connectors strongly relies upon statistical  
data~--- a~pivotal tool in helping researchers to get new insights and to make 
results more reliable. Statistical data processing opens up new research 
perspectives and suggests solutions of many yet unsolved questions.

\vspace*{-3pt}
  

\Ack  
\noindent
The study has been conducted at the Institute of Informatics 
Problems, Federal Research Center 
``Computer Science and Control'' of the
Russian Academy of Sciences with financial aid from the Russian 
Foundation for Basic Research (grant No.\,16-06-00070).



\renewcommand{\bibname}{\protect\rmfamily References}



%\vspace*{-6pt}

{\small\frenchspacing
{\baselineskip=10.65pt
\begin{thebibliography}{99}
    

\bibitem{2-in} %1
\Aue{In'kova-Manzotti, O.\,Yu.} 2001. \textit{Konnektory pro\-ti\-vo\-po\-stavle\-niya 
vo frantsuzskom i~russkom yazykakh. So\-po\-sta\-vi\-tel'\-noe issledovanie} 
[Connectors of opposition in French and Russian. A~comparative study]. 
Moscow: Informelektro. 434~p.
\bibitem{3-in} %2
\Aue{Chzhon, Kh.\,Kh.} 2003. \textit{Prisoedinitel'nye skrepy v~sovremennom 
rus\-skom yazyke: sintaksis i~semantika} [Conjunctive ties in modern Russian: 
Syntax and semantics]. Moscow: Lomonosov Moscow State University. PhD 
Thesis. 190~p.
\bibitem{1-in} %3
\Aue{Zav'yalov, V.\,N.} 2009. Morfologicheskie i~sin\-tak\-si\-che\-skie aspekty 
opisaniya struktury soyuzov v~so\-vre\-men\-nom russkom yazyke [Morphological 
and syntactical aspects of the conjunctions' structure description in modern 
Russian]. Vladivostok: DGU. D.Sc. Thesis. 393~p.
\bibitem{4-in}
Natsional'nyy korpus russkogo yazyka [Russian National 
Corpus]. Available at: {\sf http://www.ruscorpora.ru} (accessed 
April~23, 2017).
\bibitem{5-in}
Russkaya korpusnaya grammatika [Russian Corpus 
Grammar]. Available at: {\sf http://rusgram.ru/} (accessed April~28, 
2017).
\bibitem{6-in}
\Aue{Apresyan, V.\,Yu., and O.\,E.~Pekelis}. 2011. \textit{Soyuz} 
[Conjunction]. Available at: {\sf http://rusgram.ru/} (accessed April~28, 2017).
\bibitem{7-in}
\Aue{Zaliznyak Anna~A., I.\,M.~Zatsman, O.\,Yu.~In'kova, and 
M.\,G.~Kruzhkov.} 2015. Nadkorpusnye bazy dannykh kak lingvisticheskiy 
resurs [Subcorpora databases as linguistic resource]. \textit{Corpus Linguistics: 7th Conference 
(International) Proceedings}. St. Petersburg: St.\ Petersburg State University.  
211--218.

\bibitem{9-in} %8
\Aue{Zatsman, I.\,M., O.\,Yu.~In'kova, M.\,G.~Kruzhkov, and 
N.\,A.~Popkova}. 2016. Predstavlenie krossyazykovykh zna\-niy 
o~konnektorakh v~nadkorpusnykh bazakh dannykh [Representation of  
cross-lingual knowledge about connectors in supracorpora databases] 
\textit{Informatika i~ee Primeneniya~--- Inform. Appl.} 10(1):106--118.
\bibitem{10-in} %9
\Aue{In'kova, O.\,Yu., and M.\,G.~Kruzhkov.} 2016. {Nad\-kor\-pus\-nye  
russko-frantsuzskie basy dannykh gla\-gol'\-nykh form i~konnektorov} 
[Supracorpora databases of Russian and French verbal forms and connectors]. 
\textit{Lingue slave a~confronto} [Slavic languages in comparison]. 
Eds. O.~In'kova and A.~Trovesi. Bergamo: 
Bergamo University Press. 365--392.
\bibitem{8-in} %10
\Aue{Zaliznyak, Anna~A., I.\,M.~Zatsman, and O.\,Yu.~In'kova.} 2017. 
Nadkorpusnaya baza dannykh konnektorov: po\-stro\-enie sistemy terminov 
[Supracorpora database of connectors: Developing a~terminology]. 
\textit{Informatika i~ee Primeneniya~--- Inform. Appl.} 11(1):100--108.
\bibitem{11-in}
\Aue{Dobrovol'skiy, D.\,O., A.\,A.~Kretov, and S.\,A.~Sharov}. 2005. Korpus 
parallel'nykh tekstov: Arkhitektura i~voz\-mozh\-no\-sti ispol'zovaniya [Corpus of 
parallel texts: Architecture and applications]. \textit{Natsional'nyy korpus 
russkogo yazyka: 2003--2005} [Russian National Corpus: 2003--2005]. 
Moscow: Indrik. 263--296.
\bibitem{12-in}
\Aue{In'kova, O.\,Yu., and N.\,A.~Popkova}. 2016. Struktura dvukhmestnykh 
konnektorov russkogo yazyka v~svete kor\-pus\-nykh dannykh [The structure of 
two-part correlative connectors as an object of corpus analysis]. 
\textit{Computational Linguistics and 
Intellectual Technologies: Conference (International) ``Dialogue'' Proceedings}. 
Moscow: RGGU. 15(22):200--213.
\bibitem{13-in}
\Aue{In'kova, O.\,Yu.} 2016. K~probleme opisaniya mnogokomponentnykh 
konnektorov russkogo yazyka: ne tol'ko$\ldots$\  no~i [Towards the problem of 
the description of multiword connectives of Russian language: Ne 
tol'ko$\ldots$\  no~i (not only$\ldots$\  but also)]. \textit{Voprosy 
yazykoznaniya} [Topics in the Study of Language] 2:37--60.
\bibitem{14-in}
\Aue{Kobozeva, I.\,M.} 2016. Kognitivno-semanticheskiy podkhod 
k~opisaniyu sredstv svyazi predlozheniy (na primere konnektorov so 
znacheniem neposredstvennogo sle\-do\-va\-niya) [Cognitive-semantical approach 
to the description of ways to connect sentences (case of connectors of 
immediate consecution)]. \textit{Tr. Instituta russkogo yazyka im.\ 
V.\,V.~Vinogradova}
[V.\,V.~Vinogradov Russian Language Institute of the Russian Academy of 
Sciences Collections] 11:118--131.
\bibitem{15-in}
\Aue{Popkova, N.\,A., O.\,Yu.~In'kova, I.\,M.~Zatsman, and 
M.\,G.~Kruzhkov}. 2015. Metodika postroeniya monoekvivalentsiy 
v~nadkorpusnoy baze dannykh konnektorov [Methodology of constructing 
monoequivalences in the supracorpora database of connectors]. \textit{Tr. 2-y 
nauchn. konf. ``Zadachi sovremennoy informatiki''}
[2nd Scientific Conference ``Modern Informatics' Problems'' Proceedings]. 
Moscow: FRC CSC RAS. 143--153.
\bibitem{16-in}
\Aue{Uryson, E.\,V.} 2012. Soyuzy, konnektory i~teoriya valentnostey 
[Conjunctions, connectors, and the valence theory]. \textit{Computational Linguistics and Intellectual Technologies: 
Conference (International ) ``Dialogue'' Proceedings}. Moscow: RGGU. 11(1):627--637.
    \end{thebibliography} } }

\end{multicols}

\vspace*{-10pt}

\hfill{\small\textit{Received July 11, 2017}}



\vspace*{-16pt}

\Contr

\noindent
\textbf{Inkova Olga Yu.} (b.\ 1965)~---
 Doctor of Science in philology, senior scientist, Institute of Informatics Problems, 
 Federal Research Center ``Computer Science and Control'' 
 of the Russian Academy of Sciences; 
 44-2~Vavilov Str., Moscow 119333, Russian Federation; \mbox{olyainkova@yandex.ru}

 
\vspace*{3pt}


\noindent
\textbf{Popkova Natalia A.} (b.\ 1992)~--- 
junior scientist, Institute of Informatics Problems, Federal Research Center 
``Computer Science and Control'' of the Russian Academy of Sciences; 44-2~Vavilov 
Str., Moscow 119333, Russian Federation; \mbox{natasha\_\_popkova@mail.ru}

\vspace*{6pt}

\hrule

\vspace*{2pt}

\hrule

%\newpage

%\vspace*{-24pt}

\vspace*{8pt}

\def\tit{СТАТИСТИЧЕСКИЕ ДАННЫЕ КАК~ИНФОРМАЦИОННАЯ ОСНОВА 
ЛИНГВИСТИЧЕСКОГО АНАЛИЗА КОННЕКТОРОВ РУССКОГО ЯЗЫКА$^*$}




\def\titkol{Статистические данные как информационная 
основа лингвистического анализа коннекторов русского языка}

\def\aut{О.\,Ю.~Инькова, Н.\,А.~Попкова}

\def\autkol{О.\,Ю.~Инькова, Н.\,А.~Попкова}


{\renewcommand{\thefootnote}{\fnsymbol{footnote}}
\footnotetext[1]{Исследование выполнено при финансовой поддержке РФФИ 
(проект 16-06-00070).}}


\titel{\tit}{\aut}{\autkol}{\titkol}

\vspace*{-12pt}

\noindent
Институт проблем информатики Федерального исследовательского
центра <<Информатика и~управление>> Российской академии наук


\vspace*{6pt}

\def\leftfootline{\small{\textbf{\thepage}
\hfill ИНФОРМАТИКА И ЕЁ ПРИМЕНЕНИЯ\ \ \ том\ 11\ \ \ выпуск\ 3\ \ \ 2017}
}%
 \def\rightfootline{\small{ИНФОРМАТИКА И ЕЁ ПРИМЕНЕНИЯ\ \ \ том\ 11\ \ \ выпуск\ 3\ \ \ 2017
\hfill \textbf{\thepage}}}

 
\Abst{Целью статьи является описание статистических данных, которые получены 
с~по\-мощью надкорпусной базы данных (НБД) коннекторов для лингвистического анализа 
их формальных и~функциональных свойств. До настоящего времени эти свойства 
описывались, как правило, с~применением семантического анализа; корпусные данные, 
если и~привлекались, то без их статистической обработки. Именно программно 
генерируемая и~верифицируемая статистическая информация, полученная на 
основе корпусов текстов, может служить одним из надежных параметров
 лингвистического анализа языковых единиц, в~том числе коннекторов. 
Описаны виды получаемой с~по\-мощью НБД статистики 
 и~воз\-мож\-ности ее использования для собственно лингвистического анализа на примере 
 языковой единицы \textit{только}, отличающейся своей полифункциональностью, 
 а~также способностью входить в~качестве составляющей в~значительное число 
 многокомпонентных и~двухместных коннекторов.}

\KW{аннотирование коннекторов; корпусная лингвистика; надкорпусные базы данных; 
параллельные тексты; статистические данные}


\DOI{10.14357/19922264170314} 


\vspace*{6pt}


 \begin{multicols}{2}

\renewcommand{\bibname}{\protect\rmfamily Литература}
%\renewcommand{\bibname}{\large\protect\rm References}

{\small\frenchspacing
{%\baselineskip=10.8pt
\begin{thebibliography}{99}


\bibitem{2-in-1} %1
\Au{Инькова-Манзотти О.\,Ю.} 
Коннекторы противопоставления во французском и русском языках. 
Сопоставительное исследование.~--- М.: Информэлектро, 2001. 434~с.
\bibitem{3-in-1} %2
\Au{Чжон Х.\,Х.} Присоединительные скрепы в~современном русском языке: 
синтаксис и~семантика: Дис.\ \ldots\ канд. филол. наук.~--- М.: МГУ им. 
М.\,В.~Ломоносова, 2003. 190~с.
\bibitem{1-in-1} %3
\Au{Завьялов В.\,Н.}
Морфологические и синтаксические аспекты описания структуры союзов 
в~современном русском языке: Дис.\ \ldots\ д-ра филол. наук.~--- 
Владивосток: ДГУ, 2009. 393~с.
\bibitem{4-in-1}
Национальный корпус русского языка. {\sf http://www. ruscorpora.ru}.
\bibitem{5-in-1}
Русская корпусная грамматика. {\sf http://rusgram.ru}.
\bibitem{6-in-1}
\Au{Апресян В.\,Ю., Пекелис~О.\,Е.} Союз.~--- М., 2011. 
{\sf http:// rusgram.ru}. % (На правах рукописи.)
\bibitem{7-in-1}
\Au{Зализняк Анна А., Зацман~И.\,М., Инькова~О.\,Ю., Кружков~М.\,Г.}
Надкорпусные базы данных как лингвистический ресурс~// Корпусная лингвистика-2015: 
Тр. междунар. конф.~--- СПб.: СПбГУ, 2015. С.~211--218.

\bibitem{9-in-1} %8
\Au{Зацман И.\,М., Инькова~О.\,Ю., Кружков~М.\,Г., Попкова~Н.\,А.} 
Представление кроссязыковых знаний о~коннекторах в~надкорпусных базах данных~// 
Информатика и~её применения, 2016. Т.~10. Вып.~1. С.~106--118. 
\bibitem{10-in-1} %9
\Au{Инькова О.\,Ю., Кружков~М.\,Г.} 
Надкорпусные рус\-ско-фран\-цуз\-ские базы данных глагольных форм и~коннекторов~// 
Lingue slave in confronto~/ Eds. O.~Inkova, A.~Trovesi.~--- 
Bergamo: Bergamo University Press, 2016. С.~365--392.
\bibitem{8-in-1} %10
\Au{Зализняк Анна А., Зацман~И.\,М., Инькова~О.\,Ю.}
Надкорпусная база данных коннекторов: построение сис\-те\-мы терминов~// 
Информатика и~её применения, 2017. Т.~11. Вып.~1. С.~100--108.
\bibitem{11-in-1}
\Au{Добровольский Д.\,O., Кретов~A.\,A., Шаров~С.\,A.} 
Корпус параллельных текстов: архитектура и~возможности использования~// 
Национальный корпус русского языка: 2003--2005.~--- М.: Индрик, 2005. С.~263--296.
\bibitem{12-in-1}
\Au{Инькова О.\,Ю., Попкова~Н.\,А.} 
Структура двухместных коннекторов русского языка в~свете корпусных данных~// 
Компьютерная лингвистика и интеллектуальные технологии: По мат-лам 
ежегодной Междунар. конф. <<Диалог>>.~--- М.: РГГУ, 2016.  Вып.~15(22). С.~200--213.
\bibitem{13-in-1}
\Au{Инькова О.\,Ю.} К~проблеме описания многокомпонентных коннекторов русского языка: 
не только$\ldots$\ но~и~// Вопросы языкознания, 2016. №\,2. С.~37--60.
\bibitem{14-in-1}
\Au{Кобозева И.\,М.} Когни\-тив\-но-се\-ман\-ти\-че\-ский 
подход к~описанию средств связи предложений (на примере коннекторов со
 значением непосредственного следования)~// Тр. Института русского языка им.\
  В.\,В.~Виноградова, 2016. Т.~11. С.~118--131.
\bibitem{15-in-1}
\Au{Попкова Н.\,А., Инькова~О.\,Ю., Зацман~И.\,М., Кружков~М.\,Г.} 
Методика построения моноэквиваленций в надкорпусной базе данных коннекторов~// 
Задачи современной информатики: Тр. 2-й научной конф.~--- 
М.: ФИЦ ИУ РАН, 2015. С.~143--153.
\bibitem{16-in-1}
\Au{Урысон Е.\,В.} 
Союзы, коннекторы и теория валентностей~// 
Компьютерная лингвистика и интеллектуальные технологии: По мат-лам 
ежегодной Междунар. конф. <<Диалог>>.~--- М.: РГГУ, 2012. 
 Вып.~11(18). Т.~1. С.~627--637.


\end{thebibliography}
} }

\end{multicols}

 \label{end\stat}

 \vspace*{-3pt}

\hfill{\small\textit{Поступила в редакцию  11.07.2017}}
%\renewcommand{\bibname}{\protect\rm Литература}
\renewcommand{\figurename}{\protect\bf Рис.}
\renewcommand{\tablename}{\protect\bf Таблица}