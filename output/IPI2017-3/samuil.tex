\def\stat{sopin}

\def\tit{СИСТЕМА МАССОВОГО ОБСЛУЖИВАНИЯ 
С~ОГРАНИЧЕННЫМИ РЕСУРСАМИ И~СИГНАЛАМИ ДЛЯ~АНАЛИЗА 
ПОКАЗАТЕЛЕЙ ЭФФЕКТИВНОСТИ БЕСПРОВОДНЫХ СЕТЕЙ$^*$}

\def\titkol{Система массового обслуживания с~ограниченными ресурсами и~сигналами 
для анализа 
показателей эффективности} % беспроводных сетей}

\def\aut{К.\,Е.~Самуйлов$^1$, Э.\,С.~Сопин$^2$, С.\,Я.~Шоргин$^3$}

\def\autkol{К.\,Е.~Самуйлов, Э.\,С.~Сопин, С.\,Я.~Шоргин}

\titel{\tit}{\aut}{\autkol}{\titkol}

\index{Самуйлов К.\,Е.}
\index{Сопин Э.\,С.}
\index{Шоргин С.\,Я.}
\index{Samouylov K.\,E.}
\index{Sopin E.\,S.} 
\index{Shorgin S.\,Ya.}


{\renewcommand{\thefootnote}{\fnsymbol{footnote}} \footnotetext[1]
{Исследование выполнено при финансовой поддержке Российского научного фонда 
в~рамках научного проекта №\,16-11-10227.}}


\renewcommand{\thefootnote}{\arabic{footnote}}
\footnotetext[1]{Российский университет дружбы народов; Институт проб\-лем информатики 
Федерального исследовательского центра <<Информатика и~управ\-ле\-ние>> Российской 
академии наук, \mbox{samouylov\_ke@rudn.university}}
\footnotetext[2]{Российский университет дружбы народов; Институт проб\-лем информатики 
Федерального исследовательского центра <<Информатика и~управ\-ле\-ние>> Российской 
академии наук, \mbox{sopin\_es@rudn.university}}
\footnotetext[3]{Институт проблем информатики Федерального исследовательского центра 
<<Информатика и~управ\-ле\-ние>> Российской академии наук, 
\mbox{sshorgin@ipiran.ru}}

%\vspace*{-18pt}
   
 
      
  
  \Abst{Рассматривается многолинейная система массового обслуживания  (СМО)
с~ресурсами ограниченного объема. Поступающая заявка занимает не только прибор, но 
и~некоторый объем ресурсов на все время обслуживания. Помимо потока заявок на 
систему поступает поток сигналов, при поступлении которых заявки заново разыгрывают 
объем занимаемых ресурсов. Рассматриваемая система массового обслуживания 
позволяет описывать функционирование беспроводной сети с~учетом перемещения 
пользователей в~течение периода жизни пользовательской сессии. Исследуются две 
модели перемещения пользователей. В~первой пользователи перемещаются независимо 
друг от друга; следовательно, в~соответствующей математической модели поступление 
сигнала изменяет занимаемый объем ресурсов только одной заявки. Во второй модели 
пользователи перемещаются совместно, поэтому занимаемый объем ресурсов меняется 
одновременно у всех заявок.}

\KW{ограниченные ресурсы; сигнал; система массового обслуживания; 
беспроводная сеть; сети связи 4-го поколения}

\DOI{10.14357/19922264170311} 


\vskip 10pt plus 9pt minus 6pt

\thispagestyle{headings}

\begin{multicols}{2}

\label{st\stat}
  
\section{Введение}

  Для анализа показателей качества услуг в~современных сетях связи 4-го 
поколения с~объектами в~движении широко используется имитационное 
моделирование~[1, 2] и~простые модели теории массового обслуживания~[3] 
с~фиксированным объемом требований. 

В~работах~[4, 5] предлагается 
анализировать показатели эффективности модели современной беспроводной 
гетерогенной сети связи в~виде СМО
ограниченной емкости с~требованиями случайного объема. В~отличие от 
моделей, представленных в~[6, 7], моделирование беспроводной сети 
в~терминах теории массового обслуживания учитывает процессы 
установления новых сессий и~их завершения, а заданная специальным 
образом функция распределения случайных требований к~радиоресурсам 
позволяет описать функционирование планировщика в~соответствии 
с~выбранной политикой распределения частотных ресурсов и~моделью 
распространения сигнала. 

В~предложенной в~[8] экспоненциальной модели 
каждая сессия занимает выделенный ей объем частотного ресурса на все 
время ее длительности. По завершении сессии предполагается освободить 
некоторый случайный объем ресурсов, отличный от занимаемого, так как 
местоположение и~число пользователей в~сети могут с~течением времени 
измениться. Однако данная модель не учитывает изменения в~сети, которые 
могут произойти до завершения обслуживания сессий. 
  
  В~данной работе исследуются модели, в~которых объем занимаемых 
заявками ресурсов может меняться до завершения обслуживания, при 
поступлении сигнала. Эта особенность позволяет моделировать 
функционирование беспроводной сети, в~которой пользователи, удаляясь или 
приближаясь к~базовой станции, увеличивают или уменьшают требуемый 
объем ресурсов в~течение периода жизни пользовательской сессии. 
  %
  При этом в~связи с~тем, что в~беспроводных сетях задача поддержания уже 
принятых сессий имеет более высокий приоритет по сравнению с~задачей 
принятия на обслуживание новых, диспетчеры ресурсов планируют их таким 
образом, чтобы ни в~коем случае не прерывать текущие сессии~[9]. Поэтому 
в~исследуемых в~данной работе моделях поступление сигнала, изменяющего 
объем занятого заявкой ресурса, не может привести к~потере заявки.
  
  Рассматриваются два сценария перемещения пользователей 
в~беспроводной сети. Согласно первому сценарию пользователи 
перемещаются независимо друг от друга; следовательно, моменты изменения 
объемов занимаемых заявками ресурсов не зависят друг от друга. 
  
  Второй сценарий предполагает одновременное перемещение 
пользователей в~соте, например в~общест\-венном транспорте. В~этом случае 
все пользователи перемещаются относительно базовой станции 
одновременно и~достаточно иметь пред\-став\-ле\-ние о~том, каким образом 
изменяется совокупно занимаемый объем ресурса.
  
\section{Независимое перемещение пользователей}

  Рассмотрим модель соты, которая может обслуживать одновременно не 
более~$N$~сессий связи с~устройствами. Объем доступных  
час\-тот\-но-вре\-мен\-н$\acute{\mbox{ы}}$х ресурсов ограничен и~не 
превышает~$R$~единиц. Для установления новой сессии каждое устройство 
требует выделить ему некоторый случайный объем ресурсов $0\hm\leq 
r\hm\leq R$. Если в~соте одновременно установлено не более~$N$~сессий 
и~требование к~ресурсу новой сессии не превышает объема свободного 
ресурса, то сессия будет установлена, в~противном случае она будет 
отклонена. Пусть в~некоторый момент времени один из активных 
пользователей изменит свое местоположение относительно базовой станции; 
в~этом случае он либо освободит часть занимаемого ресурса, либо базовая 
станция выделит ему дополнительный ресурс, не превышающий объем 
доступного в~этот момент ресурса соты. 
  
  Опишем теперь предложенную модель в~терминах теории массового 
обслуживания. Рассматривается СМО с~$N\hm<\infty$ приборами, 
обладающая некоторым объемом ресурсов $R\hm<\infty$. Введем основные 
предположения.
  \begin{enumerate}[1.]
\item В~систему поступает пуассоновский поток заявок 
с~интенсивностью~$\lambda$, время обслуживания заявок имеет 
экспоненциальное распределение с~параметром~$\mu$. 
\item Для обслуживания $i$-й поступающей заявки требуется~$r_i$ 
ресурса, $r_i\hm\geq 0$, с~вероятностью~$p_{r_i}$. 
\item Если в~момент поступления $i$-й заявки в~сис\-те\-ме находится 
$k\hm<N$ заявок, занимающих $r_\bullet \hm= r_1+\cdots+r_k$ ресурсов, 
и~$r_i\hm\leq R\hm-r_\bullet$, то заявка будет принята к~обслуживанию; 
в~противном случае заявка будет потеряна.
\item Каждая заявка, находящаяся на обслуживании, порождает 
пуассоновский поток сигналов с~ин\-тен\-сив\-ностью~$\gamma$, при 
поступлении которого она освобождает весь занимаемый ею ресурс, чтобы 
занять новый объем ресурса.
  \end{enumerate}
  
  Пусть в~некоторый момент времени $t\hm>0$ в~сис\-те\-ме находится~$\xi(t)$ 
заявок, которые занимают $\eta_1(t),\ldots ,\eta_{\xi(t)}(t)$ ресурсов. 
Функционирование сис\-те\-мы описывает случайный процесс (СП) $X(t)\hm=\left( 
\xi(t), \eta_1(t),\ldots , \eta_{\xi(t)}(t)\right)$, однако для дальнейшего анализа 
модели удобно воспользоваться упрощением, предложенным в~\cite{10-sop} 
для СМО с~ресурсами, позволяющими снизить размер простран\-ства 
состояний системы за счет отслеживания только суммарного объема 
ресурсов $\delta(t)\hm= \eta_1(t)+\cdots+\eta_{\xi(t)}(t)$. В~дальнейшем будем 
исследовать СП $\tilde{X}(t)\hm= \left( \xi(t),\delta(t)\right)$. 
  
  Рассмотрим подробнее возможные переходы между состояниями системы. 
Пусть в~некоторый момент времени система находится в~состоянии $(k,r)$. 
С~вероятностью~$p_j$ в~систему может поступить заявка, которая 
займет~$j$~единиц ресурса, если $j\hm\leq R\hm-r$. Из-за того что 
неизвестно число занимаемых ресурсов каждой заявкой, невозможно точно 
определить объем высвобождаемых ресурсов при завершении обслуживания. 
Поэтому будем считать, что заявка освобождает~$i$~единиц ресурса 
с~вероятностью $p_ip_{r-i}^{(k-1)}/p_r^{(k)}$, где $p_r^{(k)}$ является  
$k$-крат\-ной сверткой распределения~$\{p_i\}$, $i\hm\geq0$. Данную 
вероятность можно интерпретировать как вероятность того, что заявка 
занимает~$i$~единиц ресурса при условии, что~$k$~заявок суммарно 
занимают~$r$~ресурсов. 
  
  В момент поступления сигнала одна из заявок системы сначала 
освобождает занимаемые ею~$i$~единиц ресурса с~вероятностью~$p_i p_{r-
i}^{(k-1)}/p_r^{(k)}$ и~занимает~$j$~единиц ресурса с~нормированной 
вероят\-ностью~$p_j/\sum\nolimits_{s=0}^{R-r+i} p_s$, поскольку потери при 
поступле\-нии сигнала по условию не происходят.
  
  Пространство состояний системы описывается множеством
  
  \noindent
   $$
    \mathsf{X}^{\sptilde} \hm= \mathop{\bigcup}\limits_{k=0}^N 
   {\sf X}_k^{\sptilde}\,,
   $$
   
   \vspace*{-2pt}
   
   \noindent
    где ${\mathsf X}_k^{\sptilde} \hm= \left\{ 
(k,r): 0\hm\leq r\hm\leq R,\ p_r^{(k)}\hm>0\right\}$. Упорядочив состояния 
в~множествах~${\mathsf X}_k^{\sptilde}$, $0\hm\leq k\hm\leq N$, по возрастанию числа 
ресурсов, введем функции $I(k,r)$, значения которых равны порядковому 
номеру состояния $(k,r)$ в~множестве~${\sf X}_k^{\sptilde}$.
  
  Матрица интенсивностей переходов СП $\tilde{X}(t)$ 
  $$
  \mathbf{A}=  [a((i,j),(k,r))]$$ 
  является блочной трехдиагональной матрицей 
c~диагональными блоками $\boldsymbol{\Psi}_0, 
\boldsymbol{\Psi}_1, \ldots , \boldsymbol{\Psi}_N$, наддиа-\linebreak\vspace*{-12pt}

\pagebreak

\noindent
гональными 
блоками $\boldsymbol{\Lambda}_1,\ldots , \boldsymbol{\Lambda}_N$ и~поддиагональными блоками 
$\mathbf{M}_0,\ldots , \mathbf{M}_{N-1}$, где 
\begin{align*}
\Psi_0&= -\lambda 
\sum\limits_{j=0}^R p_j\,;\ 
 \boldsymbol{\Lambda}_1=\left(\lambda p_0,\ldots, \lambda p_r\right)\,;
\\
\mathbf{M}_0&= \left(\mu, \ldots, \mu\right)^{\mathrm{T}}\,,
\end{align*}
 а~остальные матрицы 
$\{\boldsymbol{\Psi}_n\}_{1\leq n\leq N}$, $\{\boldsymbol{\Lambda}_n\}_{2\leq n\leq N}$ 
и~$\{\mathbf{M}_n\}_{1\leq n\leq N-1}$ имеют следующие элементы:
  \begin{multline}
  \psi_n(I(n,i),I(n,j))={}\\
  {}=\begin{cases}
  \displaystyle-\left[ \lambda \sum\limits_{k=0}^{R-i} p_k 
+n\mu+n\gamma\right]\,, &\ i=j\,;\\[5pt]
  \displaystyle n\gamma\sum\limits_{s=0}^i \fr{p_sp_{i-s}^{(n-
1)}}{p_i^{(n)}}\,\fr{p_{j-i+s}}{\sum\nolimits_{k=0}^{R-i+s}p_k}\,, &\ i<j\,;\\[5pt]
  \displaystyle n\gamma\sum\limits^i_{s=i-j} \fr{p_s p_{i-s}^{(n-1)}} 
{p_i^{(n)}}\, \fr{p_{j-i+s}}{\sum\nolimits_{k=0}^{R-i+s} p_k}\,, &\ i>j\,,
  \end{cases}\\
     (n,i), (n,j)\in {\sf X}_n^{\sptilde}\,,\enskip 
  n=\overline{1,N-1}\,;
  \label{e1-sop}
  \end{multline}
  
  \vspace*{-12pt}
  
\noindent
  \begin{multline}
  \lambda_n(I(n-1,i),I(n,j))=\begin{cases}
  \lambda p_{j-i}\,, &\ i\leq j\leq R\,;\\
  0\,, &\ j<i\,,\end{cases}\\[5pt]
  (n-1,i)\in {\sf X}_{n-1}^{\sptilde}\,,\enskip
  (n,j)\in {\sf X}^{\sptilde}_n\,,\ n=\overline{2,N}\,;
  \label{e2-sop}
  \end{multline}
  
  \vspace*{-12pt}
  
  \noindent
  \begin{multline}
  \mu_n(I(n+1,i), I(n,j))={}\\
  {}=\begin{cases}
  \displaystyle (n+1)\mu \fr{p_{i-j}-p_j^{(n)}}{p_i^{(n+1)}}\,, &\ j\leq i\leq R\,;\\
  0\,, &\ j>i\,,
  \end{cases}\\[5pt]
    (n+1,i)\in {\sf X}^{\sptilde}_{n+1}\,,\enskip
  (n,j)\in {\sf X}_n^{\sptilde}\,,\ n=\overline{1,N-1}\,;
  \label{e3-sop}
  \end{multline}
  
  \vspace*{-12pt}
  
  \noindent
  \begin{multline}
  \psi_N(I(N,i), I(N,j))={}\\
  {}=\begin{cases}
  -[N\mu +N\gamma]\,, &\ i=j\,;\\[5pt]
  \displaystyle N\gamma \sum\limits_{s=0}^i  \fr{p_s p_{i-s}^{(N-
1)}}{p_i^{(N)}}\,\fr{p_{j-i+s}} {\sum\nolimits_{k=0}^{R-i+s} p_k}\,, &\ i<j\,;\\[5pt]
  \displaystyle N\gamma \sum\limits^i_{s=i-j} \fr{p_s p_{i-s}^{(N-
1)}}{p_i^{(N)}}\,\fr{p_{j-i+s}}{\sum\nolimits_{k=0}^{R-i+s} p_k}\,, &\ i>j\,,
  \end{cases}\\
  (N,i),(N,j)\in {\sf X}_N^{\sptilde}\,.
  \label{e4-sop}
  \end{multline}
  
  %\columnbreak
  
  Стационарные вероятности 
  \begin{align}
  q_0 &= \lim\limits_{t\to\infty} {\sf P}\{\xi(t)=0\}\,;\label{e5-sop}
\\
 q_k(r) &= \lim\limits_{t\to\infty} {\sf P}\{\xi(t)=k, \ \delta(t)=r\}\,,\ (k,r)\in {\sf 
X}_k^{\sptilde}\!\!
  \label{e6-sop}
  \end{align}
являются единственным решением системы уравнений равновесия (СУР):

\noindent
\begin{equation*}
\lambda q_0\sum\limits^R_{j=0} p_j =\mu \sum\limits_{j:\ (1,j)\in {\sf 
X}_1^{\sptilde}} q_1(j)\,;
%\label{e7-sop}
\end{equation*}

\vspace*{-12pt}

\noindent
\begin{multline*}
\left( \lambda\sum\limits_{j=0}^{R-r} p_j +k\mu +k\gamma\right) q_k(r) 
={}\\
{}=\lambda \sum\limits_{j\geq 0\,,\ (k-1,r-j)\in {\sf X}^{\sptilde}_{k-1}} q_{k-1}(r-j) 
p_j+{}\\
{}+ (k+1) \mu \sum\limits_{j\geq0\,,\ (k+1, r+j)\in {\sf X}_{k+1}^{\sptilde}} \hspace*{-1mm}
q_{k+1}(r+j)\fr{p_j p_r^{(k)}}{p_{j+r}^{(k+1)}}+{}\\
{}+k\gamma \sum\limits_{j:\ (k,j)\in {\sf X}_k^{\sptilde}}\hspace*{-6pt} q_k(j) 
\sum\limits^j_{i=\max(0,j-r)} \fr{p_i p_{j-i}^{(k-1)}}{p_j^{(k)}}\, 
\fr{p_{r-j+i}} 
{\sum\nolimits_{s=0}^{R-j+i} p_s}\,,\\ 1\leq k\leq N-1\,,\ (k,r)\in {\sf 
X}_k^{\sptilde}\,;
%\label{e8-sop}
\end{multline*}

\vspace*{-12pt}

\noindent
\begin{multline*}
(N\mu+k\gamma)q_N(r)=\lambda\hspace*{-4pt}\sum\limits_{j\geq0,\ (N-1,r-j)\in {\sf 
X}^{\sptilde}_{N-1}} \hspace*{-15pt}q_{N-1}(r-j) p_j+{}\\
{}+N\gamma\hspace*{-4pt}\sum\limits_{j:\ (N,j)\in {\sf X}^{\sptilde}_N} \hspace*{-5pt}q_N(j) 
\hspace*{-4pt}\sum\limits^j_{i=\max(0,j-r)}\hspace*{-4pt} \fr{p_i p_{j-i}^{(N-1)}}{p_j^{(N)}}\, 
\fr{p_{r-j+i}}{\sum\nolimits_{s=0}^{R-j+i} p_s}\,,\\
(N,r)\in {\sf X}_N^{\sptilde}\,.
%\label{e9-sop}
\end{multline*}
  
  Стационарные вероятности~(\ref{e5-sop}) и~(\ref{e6-sop}) могут быть 
найдены численно методом $UL$-раз\-ло\-же\-ния СУР в~матричном виде 
$$
\mathbf{q}^{\mathrm{T}}\mathbf{A}=\mathbf{0}^{\mathrm{T}}\,;
\quad
\mathbf{q}^{\mathrm{T}}\cdot \mathbf{1}=1\,.
$$

 Обозначим подвекторы 
стационарных вероятностей $\mathbf{q}_0\hm=\{q_0\}$ и~$\mathbf{q}_k\hm= 
\{q_k(r)\}_{(k,r)\in {\sf X}_k^{\sptilde}}$ для всех $1\hm\leq k \leq N$, тогда СУР 
в~матричном виде с~учетом блоч\-но-трех\-диа\-го\-наль\-но\-го вида матрицы 
интенсивностей переходов~$\mathbf{A}$ примет вид:
  \begin{align}
  \!\!\!\mathbf{q}_0\boldsymbol{\Psi}_0-\mathbf{q}_1\mathbf{M}_0&=\mathbf{0}\,; 
\label{e10-sop}\\
   \!\!\! \mathbf{q}_i\boldsymbol{\Psi}_i -\mathbf{q}_{i+1}\mathbf{M}_i- 
\mathbf{q}_{i-1}\boldsymbol{\Lambda}_i&=\mathbf{0}\,,\ 1\leq i\leq N-1\,;
  \label{e11-sop}\\
    \!\!\!\mathbf{q}_N \boldsymbol{\Psi}_N -\mathbf{q}_{N-1} 
\boldsymbol{\Lambda}_N&= \mathbf{0}\,.\label{e12-sop}
  \end{align}
  
\section{Групповое перемещение пользователей}

  Теперь рассмотрим сценарий, при котором пользователи перемещаются 
относительно базовой  станции совместно. В~этом случае в~момент 
срабатывания сигнала изменяется объем занимаемых ресурсов каждой сессии 
и,~соответственно,\linebreak\vspace*{-12pt}

\pagebreak

\noindent
 изменяется объем занимаемых в~совокупности ресурсов 
всеми активными сессиями в~соте. Важно, что при выделении 
дополнительных ресурсов прерывания сессий не происходит.
  
  Функционирование СМО описывается пп.~1--3 из предыдущего 
раздела и~п.~4*, который сформулируем следующим образом:
  \begin{itemize}
  \item[4*.] В~систему поступает пуассоновский поток сигналов 
с~интенсивностью~$\gamma$, при поступлении которого заново 
разыгрывается объем занимаемых всеми заявками ресурсов. 
\end{itemize}
  
Поведение системы во времени описывает СП $X^*(t)\hm= \left( \xi)t), 
\delta(t)\right)$, где $\xi(t)$~--- число заявок в~сис\-те\-ме; $\delta(t)$~--- объем 
совокупно занятых ресурсов. Пространство состояний СП $X^*(t)$ 
идентично пространству состояний процесса~$\tilde{X}(t)$. Обозначим 
распределение стационарных вероятностей:
\begin{align*}
q_0^* &= \lim\limits_{t\to\infty} {\sf P}\{\xi(t)=0\}\,;\\
q_k(r) &= \lim\limits_{t\to\infty} {\sf P}\{ \xi(t)=k,\ \delta(t)=r\}\,,\enskip (k,r)\in {\sf 
X}_k^{\sptilde}\,.
\end{align*}
  
  Переходы между состояниями системы, соответствующие поступлению 
новых заявок и~завершению облуживания заявок системы, происходят 
аналогично переходам между состояниями модели с~независимым 
перемещением пользователей; различие возникает в~переходах, 
соответствующих поступлению сигналов. В~момент срабатывания сигнала 
система из состояния $(k,j)$ совершает переход в~состояние $(k,r)$ 
с~вероятностью $p_r^{(k)}/\sum\nolimits_{i=0}^R p_i^{(k)}$. Таким образом, 
СУР СП~$X^*(t)$ принимает вид:
  \begin{equation}
  \lambda q_0^*\sum\limits^R_{j=0} p_j =\mu \sum\limits_{j:\ (1,j)\in{\sf 
X}_1^{\sptilde} }\hspace*{-2mm} q_1^*(j)\,;
  \label{e13-sop}
  \end{equation}
  
  \vspace*{-12pt}
  
  \noindent
  \begin{multline}
\!\!\!\! \! \left( \!\lambda\!\sum\limits_{j=0}^{R-r}\hspace*{-0.5pt} 
p_j+k\mu+\gamma\!\right) \!
q_k^*(r)={}\\
{}=\lambda \hspace*{-13pt}\sum\limits_{j\geq 0:\ (k-1;r-j)\in 
{\sf X}^{\sptilde}_{k-1}}
\hspace*{-3pt} p_j 
q^*_{k-1}(r-j)+{}\\
  {}+ (k+1)\mu \sum\limits_{j\geq0:\ (k+1;r+j)\in {\sf X}^{\sptilde}_{k+1}} \fr{p_j 
p_r^{(k)}}{p_{j+r}^{(k+1)}}\, q^*_{k+1}(r+j)+{}\\
  {}+\gamma \sum\limits^R_{i=0} \fr{p_r^{(k)}} {\sum\nolimits^R_{i=0} 
p_i^{(k)}}\,q_k^*(j)\,,\\
  1\leq k\leq N-1\,,\ (k,r)\in {\sf X}_k^{\sptilde}\,;
  \label{e14-sop}
  \end{multline}
  
  \vspace*{-12pt}
  
  \noindent
  \begin{multline}
  (N\mu+\gamma) q_N^*(r) =\lambda \sum\limits_{j\geq0:\ (N-1; r-j)\in {\sf 
X}_{N-1}^{\sptilde}} \hspace*{-5mm}p_j q^*_{N-1}(r-j)+{}\\
\!\!\!  {}+ \gamma\hspace*{-2pt}\sum\limits_{j:\ (N;r)\in{\sf X}^{\sptilde}_N} \fr{p_r^{(N)}} 
{\sum\nolimits^R_{i=0} p_i^{(N)}}\,q_N^*(j)\,,\enskip
(N,r)\in {\sf X}^{\sptilde}_N.\!\!\!
  \label{e15-sop}
  \end{multline}
  

\noindent
\textbf{Теорема~1.}\ \textit{Стационарные вероятности СМО со случайными 
требованиями и~потоком сигналов, изменяющих суммарный объем 
занимаемых ресурсов, не зависят от интенсивности~$\gamma$ поступления 
сигналов и~имеют вид}:

\noindent
\begin{equation}
q_k^*(r) = q_0\fr{\rho^k}{k!}\,p_r^{(k)}\,;\quad
q_0^* = \left( \sum\limits_{k=0}^N \sum\limits_{r=0}^R 
\fr{\rho^k}{k!}\,p_r^{(k)}\right) ^{-1}\,.
\label{e16-sop}
%\label{e17-sop}
\end{equation}
  
  \noindent
  Д\,о\,к\,а\,з\,а\,т\,е\,л\,ь\,с\,т\,в\,о\ \ теоремы выполняется путем подстановки 
стационарных вероятностей~(\ref{e16-sop}) в~СУР~(\ref{e13-sop})--(\ref{e15-sop}).

\section{Численный пример}

  Согласно теореме~1 стационарные вероятности экспоненциальной СМО 
с~групповым перемещением пользователей как частный случай 
экспоненциальной СМО со случайными требованиями из~\cite{8-sop} не 
зависят от поступающего потока сигналов в~систему. Вероятностные 
характеристики системы, такие как вероятность блокировки~$B$ и~средний 
объем занятых ресурсов~$b$, в~этом случае могут быть найдены по 
формулам:

\noindent
  \begin{align*}
  B &= 1-G^{-1}(N,R)\sum\limits_{j=0}^R p_j G(N-1,R-j)\,;
  %\label{e18-sop}
  \\
  b &= R-G^{-1}(N,R) \sum\limits_{j=1}^R G(N,R-j)\,, %\label{e19-sop}
  \end{align*}
полученным по аналогии с~\cite{11-sop} с~по\-мощью рекуррентного 
алгоритма вычисления нормировочной константы 
$$
G(N,R)= 
\sum\limits_{k=0}^N \sum\limits_{r=0}^R \fr{\rho_k}{k!}\,p_r^{(k)}\,. 
$$

  Стационарные вероятности~(\ref{e5-sop})--(\ref{e6-sop}) СМО 
с~независимым перемещением пользователей могут быть найдены численно 
как решения системы матричных уравнений~(\ref{e10-sop})--(\ref{e12-sop}). 
Вероятностные характеристики системы в~этом случае определяются 
формулами: 

\noindent
  \begin{align*}
  B&= 1-\sum\limits_{k=0}^{N-1} \sum
  \limits_{r:\ (k,r)\in{\sf X}^{\sptilde}_k} q_k(r)\sum\limits_{j=0}^{R-r} 
p_j\,; %\label{e20-sop}
\\
  b&= \sum\limits^N_{k=0} \ \sum\limits_{r:\ (k,r)\in{\sf X}^{\sptilde}_k} rq_k(r)\,.
%  \label{e21-sop}
  \end{align*}
  

  
  В качестве примеров распределений требований к~ресурсу 
рассматривались биномиальное распределение $\mathrm{Binom}\,(r,p)$ 
и~геометрическое распределение $\mathrm{Geom}\,(p)$, как и~в~\cite{12-sop}. Для 
анализа зависимости  вероятностных характеристик СМО с~независимым\linebreak\vspace*{-12pt}

  \begin{figure*} %fig1
    \vspace*{1pt}
    \begin{minipage}[t]{80mm}
\begin{center}
\mbox{%
\epsfxsize=77.795mm
\epsfbox{sop-1.eps}
}
\end{center}
\vspace*{-9pt}
\Caption{Зависимость вероятности блокировки от интенсивности поступления сигнала: 
\textit{1}~--- $\mathrm{Binom}$; \textit{2}~--- $\mathrm{Geom}$}
\end{minipage}
\hfill
%\end{figure*}
%\begin{figure*} %fig2
 \vspace*{1pt}
     \begin{minipage}[t]{80mm}
\begin{center}
\mbox{%
\epsfxsize=77.795mm
\epsfbox{sop-2.eps}
}
\end{center}
\vspace*{-9pt}
\Caption{Зависимость среднего объема занятого ресурса от интенсивности поступления 
сигнала: \textit{1}~--- $\mathrm{Binom}$; \textit{2}~--- $\mathrm{Geom}$}
\end{minipage}
\end{figure*}

\pagebreak

\noindent
перемещением пользователей от интенсивности~$\gamma$ потока 
поступающих сигналов в~качестве примера рассматриваются: 
  \begin{enumerate}[(1)]
\item биномиальное распределение $\mathrm{Binom}\,(r,p)$ требований к~ресурсу 
с~параметрами $r\hm\geq0$ и~$0\hm\leq p\hm\leq 1$, где $p_i\hm= 
\begin{pmatrix} r\\ i\end{pmatrix} p^i(1-p)^{r-i}$~--- вероятность того, что 
заявка потребует~$i$~единиц ресурса, $0\hm\leq i\hm\leq r$, $p\hm= 
\overline{m}/r$; 
\item геометрическое распределение $\mathrm{Geom}(p)$ требований к~ресурсу 
c~параметром $0\hm\leq p\hm\leq 1$, где $p_i\hm= p^i(1-p)$~--- вероятность 
того, что заявка потребует~$i$~единиц ресурса, $1\hm\leq i\hm\leq r$, 
$p\hm=1/(\overline{m}+1)$.
\end{enumerate}
  
  Для вычисления элементов матриц~(\ref{e1-sop})--(\ref{e4-sop}), которые 
определяют необходимые компоненты\linebreak реше\-ния 
СУР~(\ref{e10-sop})--(\ref{e12-sop}), найдем все $k$-крат\-ные 
свертки~$p_r^{(k)}$ для каждого из предложенных распределений 
требований к~ресурсу. При условии биномиального распределения 
требований вероятность того, что~$k$~заявок системы 
занимают~$j$~единиц ресурса:
$$
p_j^{(k)}= \begin{pmatrix} kr\\ 
j\end{pmatrix} p^j(1-p)^{kr-j}\,; 
$$
для геометрического закона:
$$
p_j^{(k)}  = \begin{pmatrix} k+j-1\\ k\end{pmatrix} p^j(1-p)^k\,.
$$

 Целочисленный 
параметр~$r$~биномиального распреде\-ления требований к~ресурсу 
и~па\-ра\-мет\-ры~$p$~распределений были подобраны таким образом, чтобы 
математическое ожидание~$\overline{m}$ было одинаковым. Максимальное 
число единиц ресурса, требуемых одной заявке, при биномиальном 
рас-\linebreak\vspace*{-12pt}

\columnbreak

\noindent
пределении, таким образом, оказалось $r\hm=18$, а математическое 
ожидание для биномиального и~геометрического распределений 
$\overline{m}\hm=5{,}4$.
  
  Рассматривается пример соты, которая может обслуживать до~100~сессий 
одновременно, а~ресурс выделяется пользователям в~процентном 
соотношении от~100\% всего доступного соте ресурса, $N\hm=R\hm=100$. 
Средняя продолжительность сессии составляет $\mu\hm=1$~мин, а~среднее 
число запросов на установление сессии $\lambda\hm=16$, как оптимальное 
значение нагрузки. 
  
  На рис.~1 и~2 представлены результаты расчета вероятности блокировки 
системы и~среднего объема занимаемых ресурсов в~зависимости от 
поступления сигналов, моделирующих перемещение пользователей в~соте. 
  


  На рис.~1 можно видеть, что вероятность блокировки в~системе 
с~независимым перемещением пользователей растет с~ростом~$\gamma$, 
несмотря на то что поступление сигнала не может вызвать потери заявки. 
Наблюдаемый эффект связан с~тем, что с~повышением интенсивности 
поступления сигналов заявки интенсивнее используют доступный ресурс 
системы, как видно на рис.~2, и~в~результате в~системе остается меньше 
свободного ресурса для принятия новых заявок.

\vspace*{-9pt}

  \section{Заключение}
  
  В работе проведен анализ ресурсной СМО с~сигналами, при поступлении 
которых изменяется объем занимаемых заявками ресурсов. Модель 
позволяет проводить анализ показателей эффективности беспроводной сети, 
учитывая перемещение пользователей в~радиусе действия. Рассмотрены 
частные случаи независимого и~группового перемещений пользователей. 
В~частности, было доказано, что при групповом перемещении пользователей 
показатели качества сети не зависят от интенсивности изменения положения 
группы относительно базовой станции. 
  
  В дальнейшем планируется разработать эффективный вычислительный 
алгоритм расчета ве\-ро\-ят\-но\-ст\-но-вре\-мен\-н$\acute{\mbox{ы}}$х 
характеристик модели.
  
{\small\frenchspacing
 {%\baselineskip=10.8pt
 \addcontentsline{toc}{section}{References}
 \begin{thebibliography}{99}
\bibitem{1-sop}
\Au{Boban M., Barros~J., Tonguz~O.\,K.} Geometry-based vehicle-to-vehicle 
channel modeling for large-scale simulation~// IEEE T. Veh. 
Technol., 2014. Vol.~63. No.\,9. P.~4146--4164.
\bibitem{2-sop}
\Au{Khan M., Han~K.} An optimized network selection and \mbox{handover} triggering 
scheme for heterogeneous self-organized wireless networks~// Math. 
Probl. Eng., 2014. Vol.~2014. No.\,2. P.~173068-1--173068-11. {\sf 
https://www. hindawi.com/journals/mpe/2014/173068}.
\bibitem{3-sop}
\Au{Fowler S., H$\ddot{\mbox{a}}$ll~C.\,H., Yuan~D., Baravdish~D., 
Mellouk~A.} Analysis of vehicular wireless channel communication via 
queueing theory model~// IEEE Conference (International) on 
Communications.~--- Piscataway, NJ, USA: IEEE, 2014. P.~1736--1741.
\bibitem{4-sop}
\Au{Наумов В.\,А., Самуйлов~К.\,Е.} О~моделировании систем массового 
обслуживания с~множественными ресурсами~// Вестник РУДН. Сер. 
Математика. Информатика. Физика, 2014. №\,3. C.~60--64.
\bibitem{5-sop}
\Au{Naumov V., Samouylov~K., Sopin~E., Andreev~S.} Two approaches to 
analysis of queuing systems with limited resources~// Ultra Modern 
Telecommunications and Control Systems and Workshops  
Proceedings.~--- Piscataway, NJ, USA: IEEE, 2014. P.~485--488. 

\bibitem{7-sop}
\Au{Elshaer H., Boccardi~F., Dohler~M., Irmer~R.} Downlink and uplink 
decoupling: A~disruptive architectural design for 5G networks~// IEEE 
Global Communications Conference Proceedings.~--- 
Piscataway, NJ, USA: IEEE, 2014. P.~1798--1803.

\bibitem{6-sop} %7
\Au{Singh S., Zhang~X., Andrews~J.} Joint rate and SINR coverage analysis for 
decoupled uplink downlink biased cell associations in HetNets~// IEEE T. 
Wirel. Commun., 2015. Vol.~14. No.\,10. P.~5360--5373.
doi: 10.1109/TWC.2015.2437378.

\bibitem{8-sop}
\Au{Наумов В.\,А., Самуйлов~К.\,Е., Самуйлов~А.\,К.} О~суммарном 
объеме ресурсов, занимаемых обслуживаемыми заявками~// Автоматика 
и~телемеханика, 2016. №\,8. C.~125--132.
\bibitem{9-sop}
\Au{Bartolini N., Chlamtac~I.} Call admission control in wireless multimedia 
networks~// 13th IEEE Symposium (International) on Personal, Indoor and 
Mobile Radio Communications Proceedings.~--- Piscataway, NJ, USA: IEEE, 
2002. Vol.~1. P.~285--289. doi: 10.1109/PIMRC.2002.1046706.
\bibitem{10-sop}
\Au{Naumov V., Samouylov~K., Yarkina~N., Sopin~E., And\-re\-ev~S., 
Samuylov~A.} LTE performance analysis using queuing systems with finite 
resources and random requirements~// 7th Congress on Ultra Modern 
Telecommunications and Control Systems Proceedings.~--- 
Piscataway, NJ, USA: IEEE, 2015. P.~100--103.
\bibitem{11-sop}
\Au{Вихрова О.\,Г.} К~вычислению вероятностных характеристик СМО 
ограниченной емкости со случайными требованиями к~ресурсам~// 
Вестник РУДН. Сер. Математика. Информатика. Физика, 2017. Т.~25. 
№\,3. C.~203--210.
\bibitem{12-sop}
\Au{Вихрова О.\,Г., Самуйлов~К.\,Е., Сопин~Э.\,С., Шоргин~С.\,Я.} 
К~анализу показателей качества обслуживания в~современных 
беспроводных сетях~// Информатика и~её применения, 2015. Т.~9. Вып.~4. 
С.~48--55. 
 \end{thebibliography}

 }
 }

\end{multicols}

\vspace*{-3pt}

\hfill{\small\textit{Поступила в~редакцию 29.06.17}}

\vspace*{8pt}

%\newpage

%\vspace*{-24pt}

\hrule

\vspace*{2pt}

\hrule

%\vspace*{8pt}


\def\tit{QUEUING SYSTEMS WITH~RESOURCES AND~SIGNALS 
AND~THEIR~APPLICATION FOR~PERFORMANCE 
EVALUATION OF~WIRELESS NETWORKS}

\def\titkol{Queuing systems with~resources and~signals 
and~their~application for~performance 
evaluation of~wireless networks}

\def\aut{K.\,E.~Samouylov$^{1,2}$, E.\,S.~Sopin$^{1,2}$, 
and~S.\,Ya.~Shorgin$^2$}

\def\autkol{K.\,E.~Samouylov, E.\,S.~Sopin, 
and~S.\,Ya.~Shorgin}

\titel{\tit}{\aut}{\autkol}{\titkol}

\vspace*{-9pt}


\noindent
$^1$Peoples' Friendship University of Russia, 6~Miklukho-Maklaya Str., 
Moscow 117198, Russian Federation

\noindent
$^2$Institute of Informatics Problems, Federal Research Center ``Computer Sciences and Control'' 
of the Russian\linebreak
$\hphantom{^1}$Academy of Sciences, 44-2~Vavilov Str., Moscow 119333, Russian Federation



\def\leftfootline{\small{\textbf{\thepage}
\hfill INFORMATIKA I EE PRIMENENIYA~--- INFORMATICS AND
APPLICATIONS\ \ \ 2017\ \ \ volume~11\ \ \ issue\ 3}
}%
 \def\rightfootline{\small{INFORMATIKA I EE PRIMENENIYA~---
INFORMATICS AND APPLICATIONS\ \ \ 2017\ \ \ volume~11\ \ \ issue\ 3
\hfill \textbf{\thepage}}}

\vspace*{3pt}


 
\Abste{The paper considers a queuing system with limited resources, random requirements, and 
signals. Each customer occupies a server and a random amount of resources for the whole 
service duration. Besides, a Poisson flow of signals arrives to the queue.  Signal arrival triggers 
the resource reallocation process. The model can
describe functioning of a wireless network 
taking into account user movement during a session. Two cases are
considered: independent 
movement of users, when resources are reallocated independently for each session, and joint 
movement, when all resources are reallocated at once.}

\KWE{queuing system; random requirement; signals; limited resources; wireless network;  
LTE-advanced}



\DOI{10.14357/19922264170311} 

\vspace*{-12pt}

\Ack
\noindent
This work was financially supported by the Russian Science Foundation (grant  
No.\,16-11-10227).



%\vspace*{3pt}

  \begin{multicols}{2}

\renewcommand{\bibname}{\protect\rmfamily References}
%\renewcommand{\bibname}{\large\protect\rm References}

{\small\frenchspacing
 {%\baselineskip=10.8pt
 \addcontentsline{toc}{section}{References}
 \begin{thebibliography}{99}
\bibitem{1-sop-1}
\Aue{Boban, M., J.~Barros, and O.~Tonguz.} 2014. Geometry-based vehicle-to-vehicle 
channel modeling for large-scale simulation. \textit{IEEE T. Veh. Technol.} 
63(9):4146--4164.
\bibitem{2-sop-1}
\Aue{Khan, M., and K.~Han.} 2014. An optimized network selection and handover triggering 
scheme for heterogeneous self-organized wireless networks. \textit{Math. Probl. 
Eng.} 2014(2):173068-1--173068-11. 
Available at: {\sf 
https://www.hindawi.com/journals/mpe/2014/173068} (accessed Sptember~11, 2017).
\bibitem{3-sop-1}
\Aue{Fowler, S., S.~H$\ddot{\mbox{a}}$ll, D.~Yuan, D.~Baravdish, and A.~Mellouk.} 
2014. Analysis of vehicular wireless channel communication via queueing theory model. 
\textit{IEEE Conference (International) on Communications Proceedings}.  Piscataway, NJ:
IEEE. 1736--1741.
\bibitem{4-sop-1}
\Aue{Naumov, V., and K.~Samouylov.} 2014. O modelirovanii system massovogo 
obsluzivaniya s mnozhestnennymi resursami [On the modeling of queuing systems with 
multiple resources]. \textit{Vestnik RUDN. Ser. matematika, fizika, informatika} [RUDN~J. 
Mathematics, information science and physics ser.] 22(3):60--64.
\bibitem{5-sop-1}
\Aue{Naumov, V., K.~Samouylov, E.~Sopin, and S.~Andreev}. 2014. Two approaches to 
analysis of queuing systems with limited resources. \textit{Ultra Modern Telecommunications 
and Control Systems and Workshops Proceedings}. Piscataway, NJ: IEEE.  
485--488. 

\bibitem{7-sop-1}
\Aue{Elshaer, H., F.~Boccardi, M.~Dohler, and R.~Irmer.} 2014. Downlink and uplink 
decoupling: A~disruptive architectural design for 5G networks. \textit{Global 
Communications Conference Proceedings}. Piscataway, NJ: IEEE. 1798--1803.

\bibitem{6-sop-1}
\Aue{Singh, S., X.~Zhang, and J.~Andrews.} 2015. Joint rate and SINR coverage analysis for 
decoupled uplink-downlink biased cell associations in HetNets. \textit{IEEE T. 
Wirel. Commun.} 14(10):5360--5373. doi: 10.1109/TWC.2015.2437378.
\bibitem{8-sop-1}
\Aue{Naumov, V., K.~Samuilov, and A.~Samuilov.} 2016. On the total amount of 
resources occupied by serviced customers. \textit{Automat. Remote Control} 77(8):1419--1427. 
doi:10.1134/S0005117916080087. 
\bibitem{9-sop-1}
\Aue{Bartolini, N., and I.~Chlamtac.} 2002. Call admission control in wireless multimedia 
networks. \textit{13th IEEE Symposium (International) on Personal, Indoor and Mobile 
Radio Communications Proceedings}. 1:285--289. doi: 10.1109/PIMRC.2002.1046706.
\bibitem{10-sop-1}
\Aue{Naumov, V., K.~Samouylov, N.~Yarkina, E.~Sopin, S.~And\-re\-ev, and A.~Samuylov.} 
2015. LTE performance analysis using queuing systems with finite resources and random 
requirements. \textit{7th Congress (International) on Ultra Modern Telecommunications and 
Control Systems Proceedings}. Piscataway, NJ: IEEE. 100--103. 
\bibitem{11-sop-1}
\Aue{Vikhrova, О.} 2017. K~vychisleniyu veroyatnostnykh kha\-rak\-te\-ri\-stik sistemy massovogo 
obslyuzivaniya ogra\-ni\-chen\-noy emkosti so sluchaynymi trebovaniyamy k~re\-sur\-su [About 
probability characteristics evaluation in queuing system with limited resources and random 
requirements]. \textit{Vestnik RUDN. Ser. matematika, fizikam informatika} [RUDN~J. 
Mathematics, information science, and physics ser.] 25(3):203--210.
\bibitem{12-sop-1}
\Aue{Vikhrova, О., К.~Samouylov, E.~Sopin, and S.~Shorgin.} 2015. K~analizy pokazateley 
kachestva obsluzhivaniya v~sovremennykh besprovodnykh setyakh [On performance analysis 
of modern wireless networks]. \textit{Informatika i~ee Primeneniya~--- Inform. Appl.} 
9(4):48--55.  

\end{thebibliography}

 }
 }

\end{multicols}

\vspace*{-6pt}

\hfill{\small\textit{Received June 29, 2017}}

\vspace*{-10pt}


\Contr

\noindent
\textbf{Samouylov Konstantin E.} (b.\ 1955)~--- Doctor of Science in technology, professor; 
Head of Department, Peoples' Friendship University of Russia (RUDN University), 
6~Miklukho-Maklaya Str., Moscow 117198, Russian Federation; senior scientist, Institute of 
Informatics Problems, Federal Research Center ``Computer Science and Control'' of the Russian 
Academy of Sciences, 44-2~Vavilov Str., Moscow 119333, Russian Federation;  
\mbox{samouylov\_ke@rudn.university} 

\vspace*{3pt}

\noindent
\textbf{Sopin Eduard S.} (b.\ 1986)~--- Candidate of Science in physics and mathematics; associated 
professor, Peoples' Friendship University of Russia (RUDN University), 6~Miklukho-Maklaya 
Str., Moscow 117198, Russian Federation; senior scientist, Institute of Informatics Problems, 
Federal Research Center ``Computer Science and Control'' of the Russian Academy of Sciences, 
44-2~Vavilov Str., Moscow 119333, Russian Federation; \mbox{sopin\_es@rudn.university} 

\vspace*{3pt}

\noindent
\textbf{Shorgin Sergey Ya.} (b.\ 1952)~--- Doctor of Science in physics and mathematics, 
professor; Deputy Director, Federal Research Center ``Computer Science and Control'' of the 
Russian Academy of Sciences (FRC CSC RAS); principal scientist, Institute of Informatics 
Problems, FRC CSC RAS, 44-2~Vavilov Str., Moscow 119333, Russian Federation; 
\mbox{sshorgin@ipiran.ru}  
\label{end\stat}


\renewcommand{\bibname}{\protect\rm Литература} 