\documentclass[10pt]{book}
\usepackage[utf8]{inputenc}

\usepackage{latexsym,amssymb,amsfonts,amsmath,amsxtra,indentfirst,shapepar,%fleqn,%
picinpar,shadow,floatflt,enumerate,multicol,colortbl,moreverb,cite,ipi}

\usepackage{rotating}
\usepackage{mathrsfs}
\usepackage[noend]{algorithmic}
\usepackage{ulem}
\usepackage{graphicx}
%\usepackage{algorithm2e}
\usepackage[linesnumbered,boxed,ruled]{algorithm2e}
%\usepackage{xypic}

\SetAlgorithmName{Алгоритм}{алгоритм}{Список алгоритмов}

%из Дюковой

\newcommand{\algKeyword}[1]{{\bf #1}}
\newcommand{\Proc}[1]{\text{\tt #1}}
\def\CALL{\algKeyword{call}~}

\newenvironment{AlgProcedure}[1]
{
    \small
    \medskip
    %    \hrule
    \medskip
    \algKeyword{PROCEDURE} #1
    \begin{algorithmic}[1]}
    {\end{algorithmic}
    %    \hrule
    \bigskip
}

\def\CALL{\algKeyword{call}~}

%конец для Дюковой

%\RequirePackage[ruled]{algorithm}


\input{epsf}

%\nofiles

%\includeonly{avtor} %+pdf
%\includeonly{obchak,avtor}
%\includeonly{pred}                 %+
%\includeonly{podgot-rus-site,podgot-eng-site}  
%\includeonly{ocherk} 
%\includeonly{nekrol} 
%\includeonly{ipi-ind} 
%\includeonly{toc-rus}
%\includeonly{toc-en} 


%\includeonly{korolev}   %1+pdf
%\includeonly{zakharova} %2+pdf
%\includeonly{krivenko}  %3pdf
%\includeonly{kirikov}   %4pdf
%\includeonly{kovalev}   %5pdf
%\includeonly{lange}     %6pdf
%\includeonly{strizhov}  %7+algorithms
%\includeonly{safin}     %8pdf
%\includeonly{sigov}     %9+pdf
%\includeonly{gudkova}   %10+pdf
%\includeonly{samuil}    %11+ pdf
%\includeonly{razumchik} %12pdf 
%\includeonly{dukova}    %13algorithms+pdf
%\includeonly{inkova}    %14pdf
%\includeonly{zatsman}   %15+pdf

%\includeonly{toc-rus, toc-en}
%\includeonly{obchak} %,toc-en}
%\includeonly{rekl}
%\includeonly{rekl-1}
%\includeonly{reshal}  %
%\includeonly{eng-index}
%\includeonly{cover3}

\usepackage{acad}
%\usepackage{courier}
\usepackage{decor}
\usepackage{newton}
\usepackage{pragmatica}
\usepackage{zapfchan}
\usepackage{petrotex}
\usepackage{bm}                     % полужирные греческие буквы
\usepackage{upgreek}                % прямые греческие буквы
\usepackage{eufrak}
\usepackage{verbatim}

\renewcommand{\bottomfraction}{0.99}
\renewcommand{\topfraction}{0.99}
\renewcommand{\textfraction}{0.01}

\setcounter{secnumdepth}{1} %здесь - 3 + chapter = 4

\arraycolsep=1.5pt

%\usepackage[pdftex]{graphicx}

%\usepackage{oz}

%NEW COMMANDS


\renewcommand*{\hm}[1]{#1\nobreak\discretionary{}%
            {\hbox{$\mathsurround=0pt #1$}}{}} %% Дублирует знаки операций
                               %при переносе в формуле (перед знаком, который
                               %надо продублировать ставится команда \hm)

%\newcommand{\endproof}{\hfill$\Box$}
\renewcommand{\r}{\mathbb{R}}
\newcommand{\I}{{\rm I\hspace{-0.7mm}I}}
%\newcommand{\Ikl}{{\tt{1}}\hspace*{-1.44mm}\mathtt{1}}
\newcommand{\Ik}{\mbox{{\small \tt {1}}\hspace{-1.3mm}{\tt 1}}}
\newcommand{\argmin}{\mathop{\mathrm{arg}\mathrm{min}}}
\newcommand{\argmax}{\mathop{\mathrm{arg}\mathrm{max}}}
%\newcommand{\capr}{\mathop{\cap\,}}
%\newcommand{\cupr}{\mathop{\cup\,}}
%\def\argmin{\mathop{arg\,min}}

\def\vrp{\varphi}
\def\prt{\partial}
\def\mm{{\sf M}}
\def\modnop#1{\mathop{#1}\limits_{n}}
\def\eam{\mathbin{{\mathop{=}\limits^{\mathrm{def}}}}}
\def\dey#1#2{#1 (#2)}
\def\deyc#1#2{#1 \cdot  #2}
\def\ra#1{\;\mathop{\to}\limits^{#1}\;}
\def\raz#1{\;\mathop{\longrightarrow}\limits^{\!\!\!#1}\;}
\def\ral#1{\;\mathop{\longrightarrow}\limits^{#1}\;}

\newcommand{\Nor}{\mathcal{N}}
\newcommand{\T}{\mathbb{T}}
\newcommand{\Z}{\mathbb{Z}}



\newcommand{\il}[2]{\int\limits_{#1}^{#2}}%интеграл с пределами #1 и #2

\def\sm2{\mathop {\sum\limits^{n^\Theta}\sum\limits^{n^\Theta}}}
\def\sss{\sum\limits}
\def\tr{,\,\ldots\,,\,}
\def\rk{\right]}
\def\lk{\left[}
\def\rf{\right\}}
\def\lf{\left\{}
\def\lv{\,\left\vert}
\def\rv{\right\vert\,}
\def\iii{\int\limits}
\def\iin{\int\limits_{-\infty}^\infty}
\def\rrv{\right\vert}


\def\ee{{\cal E}}
\def\ww{{\cal W}}
\def\yy{{\cal Y}}
\def\vv{{\cal V}}

\newcommand{\R}{\mathbb R}
\newcommand{\E}{\mathbb E}
\newcommand{\N}{\mathbb N}

\renewcommand{\P}{\mathbb{P}}

\newcommand{\h}{{\bf H}}
\newcommand{\p}{{\sf P}}  % вероятность

\newcommand{\e}{{\sf E}}  % мат. ожидание
\newcommand{\D}{{\sf D}}  % дисперсия
\newcommand{\eps}{\varepsilon}
\newcommand{\vp}{{\mathbf p}}
\newcommand{\vz}{{\mathbf z}}
\newcommand{\vx}{{\mathbf x}}
\newcommand{\vf}{{\mathbf f}}
\newcommand{\F}{{\mathcal F}}
\def\ap{{\mathrm{ЭР}}}
\newcommand{\ud}{\Delta_n} %uniform ditance
\newcommand{\nud}{\Delta_n(x)}
\renewcommand{\Re}{\mathrm{Re}\,}

\newcommand{\abs}[1]{\left\vert#1\right\vert}

\newcommand{\norm}[1]{\left\Vert#1\right\Vert}
\def\da{(\Delta_t,A)}

\newcommand{\corr}{\mathrm{corr}}

\newcommand{\cov}{\mathrm{cov}}
\newcommand{\Expect}{\mathbb{E}}

\def\w{\omega}
\def\W{\Omega}

\def\inh{\int\limits_{nh}^{(n+1)h}}

\def\sumin{\sum_{i=1}^N}


\def\bxt{(Y,t)}
\def\xt{(y,t)}

\def\ovth{{\fr{\tau-nh}{h}}}
\def\ov{\overline}
\def\tm{\tilde m}
\def\tl{\tilde\lambda}
\def\tB{\widetilde B}
\def\tb{\tilde b}
\def\ld{\ldots}
\def\cd{\cdots}


\DeclareMathOperator{\sign}{sign}

%\newcommand{\gr}{{\geqslant}}


\newcommand{\g}{\mbox{\textit{g}}}

\renewcommand{\la}{\lambda}
\newcommand{\si}{\sigma}
\newcommand{\alp}{\alpha}

%\newcommand{\pto}{\stackrel{P}{\longrightarrow}} % сходимость по веpоятности

\newcommand{\eqd}{\stackrel{\mathrm{d}}{=}} % равенство по pаспpеделению
\newcommand{\eqdelta}{\stackrel{\triangle}{=}} % равенство по pаспpеделению

\def\be#1{\begin{equation}\label{#1}}
\def\ee{\end{equation}}
\def\re#1{(\ref{#1})}

\def\bn{\begin{enumerate}}
\def\en{\end{enumerate}}
\def\bi{\begin{itemize}}
\def\ei{\end{itemize}}
%\def\i{\item}

%\newcommand{\kp}{\kappa}
%\def\Q{{\cal Q}} \def\H{{\cal H}}
%\newcommand{\bet}{\beta_{2+\delta}}


%\newtheorem{definition}{Определение}
%\renewcommand{\thedefinition}{\arabic{definition}.}
%END NEW COMMANDS

%\renewcommand{\baselinestretch}{1.2}

%\pagestyle{myheadings}

\setlength{\textwidth}{167mm}      % 122mm
\setlength{\textheight}{658pt}
%\setlength{\textheight}{635.6pt}
\setlength{\columnsep}{4.5mm}

\setcounter{secnumdepth}{4}

%\addtolength{\headheight}{2pt}
%\addtolength{\headsep}{-2mm}

\addtolength{\topmargin}{-7mm}  % for printing


%\hoffset=-30mm  % From Yap
\hoffset=-23mm  % From Acrobat

%\voffset=0mm % From Yap
\voffset=-5mm   % From Acrobat

%\addtolength{\evensidemargin}{-2.5mm} % for printing
%\addtolength{\oddsidemargin}{2.5mm}  % for printing

\addtolength{\evensidemargin}{-12mm} % for printing
\addtolength{\oddsidemargin}{8mm}  % for printing

%\renewcommand{\thefootnote}{\fnsymbol{footnote}}
%\renewcommand{\thefootnote}{\arabic{footnote}}
\renewcommand{\figurename}{\protect\bf Рис.}
\renewcommand{\tablename}{\protect\bf Таблица}

\newcommand{\Caption}[1]{\caption{\protect\small %\baselineskip=2.5ex
#1}}

\renewcommand{\thefigure}{\arabic{figure}}
\renewcommand{\thetable}{\arabic{table}}
\renewcommand{\theequation}{\arabic{equation}}
\renewcommand{\thesection}{\arabic{section}}

\renewcommand{\contentsname}{СОДЕРЖАНИЕ}
\newcommand{\fr}[2]{\displaystyle\frac{\displaystyle #1\mathstrut}{\displaystyle #2\mathstrut}}

%\renewcommand{\thefootnote}{\fnsymbol{footnote}}
%\newcommand{\g}{\mbox{\textit{g}}}

%\newcommand{\Caption}[1]{\caption{\protect\small\baselineskip=2ex #1}}
\newcounter{razdel}
\setcounter{razdel}{0}


\newcommand{\titel}[4]{%
\

\vspace*{5pt}

\ifodd\therazdel {\raggedright\noindent\Large\textrm\textbf
 \lineskip .75em
  \baselineskip=3.2ex #1 \par}
\vskip 1em {\noindent\large\textrm\textbf #2 \par}
\addcontentsline{toc}{subsection}{{\textrm\textbf #1}\protect\newline #2}
\def\rightheadline{\underline{\noindent\hbox to \textwidth{\hfill\small\textrm{#4}
%\hfill \large\bf\thepage
}}}
\def\leftheadline{\underline{\noindent\parbox{\textwidth}{
%\raggedleft\large\bf\thepage \hfill
\small\textit{#3}\hfill}}}
\def\leftfootline{\small{\textbf{\thepage}
\hfill ИНФОРМАТИКА И ЕЁ ПРИМЕНЕНИЯ\ \ \ том~11\ \ \ выпуск 3\ \ \ 2017}
}%
 \def\rightfootline{\small{ИНФОРМАТИКА И ЕЁ ПРИМЕНЕНИЯ\ \ \ том~11\ \ \ выпуск~3\ \ \ 2017
\hfill \textbf{\thepage}}}
\vskip 2em \setcounter{figure}{0}
\setcounter{table}{0}
\setcounter{equation}{0}
\setcounter{section}{0}
\setcounter{subsection}{0}
\setcounter{subsubsection}{0}
\setcounter{footnote}{0}
\setcounter{razdel}{0}
%\end{flushleft}
\else {
 \raggedright\noindent\Large\textrm\textbf
 \lineskip .75em
\baselineskip=3.2ex #1 \par} \vskip 1em
%\begin{flushleft}
{\noindent\large\textrm\textbf #2 \par}
\addcontentsline{toc}{subsection}{{\textrm\textbf #1}\protect\newline #2}
\def\rightheadline{\underline{\noindent\hbox to \textwidth{\hfill\small\textrm{#4}
%\hfill \large\bf\thepage
}}}
\def\leftheadline{\underline{\noindent\parbox{\textwidth}{%\raggedleft\large\bf\thepage \hfill
\small\textit{#3}\hfill}}}
\def\leftfootline{\small{\textbf{\thepage}
\hfill ИНФОРМАТИКА И ЕЁ ПРИМЕНЕНИЯ\ \ \ том~11\ \ \ выпуск~3\ \ \ 2017}
}%
 \def\rightfootline{\small{ИНФОРМАТИКА И ЕЁ ПРИМЕНЕНИЯ\ \ \ том~11\ \ \ выпуск~3\ \ \ 2017
\hfill \textbf{\thepage}}} \vskip 2em \setcounter{figure}{0}
\setcounter{table}{0} \setcounter{equation}{0} \setcounter{section}{0}
\setcounter{subsection}{0} \setcounter{subsubsection}{0}
\setcounter{footnote}{0}
%\end{flushleft}
\fi}

\newcommand{\titelr}[2]{%
\

\vspace*{5pt}

\ifodd\therazdel {\raggedright\noindent%\Large\textrm\textbf
 \lineskip .75em
  \baselineskip=3.2ex #1 \par}
\vskip 1em {\noindent\normalsize\textrm\textbf #2 \par}
\else {
 \raggedright\noindent\Large\textrm\textbf
 \lineskip .75em
\baselineskip=3.2ex #1 \par} \vskip 1em
%\begin{flushleft}
{\noindent\large\textrm\textbf #2 \par
%\noindent\normalsize\textrm\textbf #2 \par
} \fi}

\newcommand{\titele}[5]{%
\

%\vspace*{5pt}

\ifodd\therazdel {\raggedright\noindent\large
\textrm\textbf
 \lineskip .75em
%  \baselineskip=3.2ex
#1 \par}
\vskip .5em {\noindent\large\textrm\textbf #2 \par}
\vskip .5em
 {\noindent\textrm #3 \par}
\addcontentsline{toc}{subsection}{{\textrm\textbf #1}\protect\newline #2}
\def\rightheadline{\underline{\noindent\hbox to \textwidth{\hfill\small\textrm{#4}
%\hfill \large\bf\thepage
}}}
\def\leftheadline{\underline{\noindent\parbox{\textwidth}{
%\raggedleft\large\bf\thepage \hfill
\small\textrm{#5}\hfill}}}
\def\leftfootline{\small{\textbf{\thepage}
\hfill ИНФОРМАТИКА И ЕЁ ПРИМЕНЕНИЯ\ \ \ том~11\ \ \ выпуск~3\ \ \ 2017}
}%
 \def\rightfootline{\small{ИНФОРМАТИКА И ЕЁ ПРИМЕНЕНИЯ\ \ \ том~11\ \ \ выпуск~3\ \ \ 2017
\hfill \textbf{\thepage}}} \vskip 1em \setcounter{figure}{0}
\setcounter{table}{0} \setcounter{equation}{0} \setcounter{section}{0}
\setcounter{subsection}{0} \setcounter{subsubsection}{0}
\setcounter{footnote}{0} \setcounter{razdel}{0}
%\end{flushleft}
\else {
 \raggedright\noindent\large
 \textrm\textbf
 \lineskip .75em
%\baselineskip=3.2ex
#1 \par} \vskip .5em
%\begin{flushleft}
{\noindent\large\textrm\textbf #2 \par} \vskip .5em
 {\noindent\textrm #3 \par}
\addcontentsline{toc}{subsection}{{\textrm\textbf #1}\protect\newline #2}
\def\rightheadline{\underline{\noindent\hbox to \textwidth{\hfill\small\textrm{#4}
%\hfill \large\bf\thepage
}}}
\def\leftheadline{\underline{\noindent\parbox{\textwidth}{%\raggedleft\large\bf\thepage \hfill
\small\textrm{#5}\hfill}}}
\def\leftfootline{\small{\textbf{\thepage}
\hfill ИНФОРМАТИКА И ЕЁ ПРИМЕНЕНИЯ\ \ \ том~11\ \ \ выпуск~3\ \ \ 2017}
}%
 \def\rightfootline{\small{ИНФОРМАТИКА И ЕЁ ПРИМЕНЕНИЯ\ \ \ том~11\ \ \ выпуск~3\ \ \ 2017
\hfill \textbf{\thepage}}} \vskip 1em \setcounter{figure}{0}
\setcounter{table}{0} \setcounter{equation}{0} \setcounter{section}{0}
\setcounter{subsection}{0} \setcounter{subsubsection}{0}
\setcounter{footnote}{0}
%\end{flushleft}
\fi}

\def\Abst#1{
\begin{center}\small\nwt
\parbox{150mm}{%\baselineskip=2.5ex
\textbf{Аннотация:}\ \
%\hspace*{\parindent}
#1}
\end{center}}
\def\Abste#1{
\begin{center}\small\nwt
\parbox{150mm}{%\baselineskip=2.5ex
\textbf{Abstract:}\ \
%\hspace*{\parindent}
#1}
\end{center}}

\def\DOI#1{
\begin{center}\small\nwt
\parbox{150mm}{%\baselineskip=2.5ex
\textbf{DOI:}\ \
%\hspace*{\parindent}
#1}
\end{center}}

\def\Abstend#1{
\begin{center}\small\nwt
\parbox{150mm}{%\baselineskip=2.5ex
%\hspace*{\parindent}
#1}
\end{center}}


\def\KW#1{
\begin{center}\small\nwt
\parbox{150mm}{%\baselineskip=2.5ex
\textbf{Ключевые слова:}\ \ #1}
\end{center}}

\def\KWE#1{
\begin{center}\small\nwt
\parbox{150mm}{%\baselineskip=2.5ex
\textbf{Keywords:}\ \ #1}
\end{center}}


\def\KWN#1{
%\begin{center}
%\small
%\parbox{150mm}\end{center}
}

\newcommand{\Avtors}[1]{%\smallskip
%\vspace*{.5pt}
\hangindent=23pt\noindent
%\nwt
{\bfseries#1}\
}


\renewcommand{\thesubsection}{\thesection.\arabic{subsection}\hspace*{-5pt}}
\renewcommand{\thesubsubsection}{\thesubsection\hspace*{5pt}.\arabic{subsubsection}\hspace*{-3pt}}

\newcommand{\Ack}{\section*{\protect\rmfamily Acknowledgments}\noindent}
\newcommand{\Contr}{\section*{\protect\rmfamily Contributors}\noindent}
\newcommand{\Contrl}{\section*{\protect\rmfamily Contributor}\noindent}

\makeindex


\begin{document}
\Rus

\nwt
%\ptb


%\renewcommand{\contentsname}{\protect\Large\bf Содержание}

\setcounter{tocdepth}{2}

%\tableofcontents

\renewcommand{\bibname}{\protect\rmfamily Литература}
  \def\Au#1{{\it #1}}
    \def\Aue#1{{#1}}

%\newcommand{\No}{№}
  \newcommand{\tg}{\,\mathrm{tg}\,}
    \newcommand{\ctg}{\,\mathrm{ctg}\,}
  \newcommand{\arctg}{\,\mathrm{arctg}\,}

\def\forallb{\mathop{\forall}}
\def\cupb{\mathop{\cup}}
\def\existsb{\mathop{\exists}}


\newpage
\addtocounter{razdel}{1}
%\def\razd{РЕГУЛИРУЕМЫЙ ЭЛЕКТРОПРИВОД ДЛЯ ЭЛЕКТРОЭНЕРГЕТИКИ}


\setcounter{page}{2}

%   { %\Large  
   { %\baselineskip=16.6pt
   
   \vspace*{-48pt}
   \begin{center}\LARGE
   \textit{Предисловие}
   \end{center}
   
   %\vspace*{2.5mm}
   
   \vspace*{25mm}
   
   \thispagestyle{empty}
   
   { %\small 

    
Вниманию читателей журнала <<Информатика и её применения>> предлагается 
очередной тематический выпуск <<Вероятностно-статистические методы и 
задачи информатики и информационных технологий>>. Предыдущие тематические 
выпуски журнала по данному направлению вышли в 2008~г.\ (т.~2, вып.~2), 
в 2009~г.\ (т.~3, вып.~3) и в 2010~г.\ (т.~4, вып.~2). 

Статьи, собранные в данном журнале, посвящены разработке новых вероятностно-статистических 
методов, ориентированных на применение к решению конкретных задач информатики и информационных 
технологий, а также~--- в ряде случаев~--- и других прикладных задач. Проблематика, охватываемая 
публикуемыми работами, развивается в рамках научного сотрудничества между Институтом проблем 
информатики Российской академии наук (ИПИ РАН) и Факультетом вычислительной математики и 
кибернетики Московского государственного университета им.\ М.\,В.~Ломоносова в ходе работ 
над совместными научными проектами (в том числе в рамках функционирования 
Научно-образовательного центра <<Вероятностно-статистические методы анализа рисков>>). 
Многие из авторов статей, включенных в данный номер журнала, являются активными участниками 
традиционного международного семинара по проблемам устойчивости стохастических моделей, 
руководимого В.\,М.~Золотаревым и В.\,Ю.~Королевым; регулярные сессии этого семинара 
проводятся под эгидой МГУ и ИПИ РАН (в 2011~г.\ указанный семинар проводится в октябре 
в Калининградской области РФ). 

Наряду с представителями ИПИ РАН и МГУ в число авторов данного выпуска журнала входят 
ученые из Научно-исследовательского института системных исследований РАН, Института 
проблем технологии микроэлектроники и особочистых материалов РАН, Института 
прикладных математических исследований Карельского НЦ РАН, Московского 
авиационного института, Вологодского государственного педагогического университета, 
НИИММ им.\ Н.\,Г.~Чеботарева, Казанского государственного университета, Дебреценского 
университета (Венгрия).

Несколько статей выпуска посвящено разработке и применению стохастических методов и 
информационных технологий для решения различных прикладных задач. В~работе В.\,Г.~Ушакова 
и О.\,В.~Шестакова рассмотрена задача определения вероятностных характеристик случайных 
функций по распределениям интегральных преобразований, возникающих в задачах эмиссионной 
томографии. В~статье Д.\,О.~Яковенко и М.\,А.~Целищева рассмотрены некоторые вопросы 
математической теории риска и предложен новый подход к диверсификации инвестиционных 
портфелей. Работа И.\,А.~Кудрявцевой и А.\,В.~Пантелеева посвящена построению и 
исследованию математической модели, описывающей динамику сильноионизованной плазмы. 
В~статье П.\,П.~Кольцова изучается качество работы ряда алгоритмов сегментации изображений. 
Статья А.\,Н.~Чупрунова и И.~Фазекаша посвящена вероятностному анализу числа без\-оши\-бочных 
блоков при помехоустойчивом кодировании; получены усиленные законы больших чисел для указанных 
величин.

В данном выпуске традиционно присутствует тематика, весьма активно разрабатываемая в течение 
многих лет специалистами ИПИ РАН и МГУ,~--- методы моделирования и управления для 
информационно-телекоммуникационных и вычислительных систем, в частности методы 
теории массового обслуживания. В~статье А.\,И.~Зейфмана с соавторами рассматриваются 
модели обслуживания, описываемые марковскими цепями с непрерывным временем в случае 
наличия катастроф. В~работе М.\,М.~Лери и И.\,А.~Чеплюковой рассматриваются случайные 
графы Интернет-типа, т.\,е.\ графы, степени вершин которых имеют степенные распределения; 
такие задачи находят применение при исследовании глобальных сетей передачи данных. 
Работа Р.\,В.~Разумчика посвящена исследованию систем массового обслуживания специального 
вида~--- с отрицательными заявками и хранением вытесненных заявок.

Ряд статей посвящен развитию перспективных теоретических 
вероятностно-статистических методов, которые находят широкое применение в различных 
задачах информатики и информационных технологий. В~работе В.\,Е.~Бенинга, А.\,К.~Горшенина 
и В.\,Ю.~Королева рассмотрена задача статистической проверки гипотез о числе компонент 
смеси вероятностных распределений, приводится конструкция асимптотически наиболее мощного 
критерия. Результаты этой работы найдут применение в ряде прикладных задач, использующих 
математическую модель смеси вероятностных распределений (в информатике, моделировании 
финансовых рынков, физике турбулентной плазмы и~т.\,д.). В~статье В.\,Ю.~Королева, 
И.\,Г.~Шевцовой и С.\,Я.~Шоргина строится новая, улучшенная оценка точности нормальной 
аппроксимации для пуассоновских случайных сумм; как известно, указанные случайные суммы 
широко используются в качестве моделей многих реальных объектов, в том числе в информатике, 
физике и других прикладных областях. Работа В.\,Г.~Ушакова и Н.\,Г.~Ушакова посвящена 
исследованию ядерной оценки плотности распределения; эти результаты могут применяться, 
в част\-ности, при анализе трафика в телекоммуникационных системах. Серьезные приложения 
в статистике могут получить результаты работы О.\,В.~Шестакова, в которой доказаны оценки 
скорости сходимости распределения выборочного абсолютного медианного отклонения к нормальному 
закону. 

\smallskip

Редакционная коллегия журнала выражает надежду, что данный тематический  выпуск 
будет интересен специалистам в области теории вероятностей и математической статистики 
и их применения к решению задач информатики и информационных технологий.
     
     %\vfill 
     \vspace*{20mm}
     \noindent
     Заместитель главного редактора журнала <<Информатика и её 
применения>>,\\
     директор ИПИ РАН, академик  \hfill
     \textit{И.\,А.~Соколов}\\
     
     \noindent
     Редактор-составитель тематического выпуска,\\
     профессор кафедры математической статистики факультета\\
      вычислительной математики и кибернетики МГУ им.\ М.\,В.~Ломоносова,\\
     ведущий научный сотрудник ИПИ РАН,\\ 
доктор физико-математических наук \hfill
      \textit{В.\,Ю.~Королев}
     
     } }
     }
\def\stat{kor-kor}



\def\tit{МОДИФИЦИРОВАННЫЙ СЕТОЧНЫЙ МЕТОД РАЗДЕЛЕНИЯ ДИСПЕРСИОННО-СДВИГОВЫХ
СМЕСЕЙ НОРМАЛЬНЫХ ЗАКОНОВ$^*$}



\def\titkol{Модифицированный сеточный метод разделения дисперсионно-сдвиговых
смесей нормальных законов}

\def\aut{В.\,Ю.~Королев$^1$,  А.\,Ю.~Корчагин$^2$}

\def\autkol{В.\,Ю.~Королев,  А.\,Ю.~Корчагин}

\titel{\tit}{\aut}{\autkol}{\titkol}

{\renewcommand{\thefootnote}{\fnsymbol{footnote}} \footnotetext[1]
{Работа поддержана Российским научным фондом (проект 14-11-00364).}}


\renewcommand{\thefootnote}{\arabic{footnote}}
\footnotetext[1]{Факультет
вычислительной математики и кибернетики Московского государственного
университета им.\ М.\,В.~Ломоносова; Институт проблем информатики
Российской академии наук; victoryukorolev@yandex.ru}
\footnotetext[2]{Факультет вычислительной математики и кибернетики
Московского государственного университета им.\ М.\,В.~Ломоносова;
sasha.korchagin@gmail.com}

%\vspace*{2pt}



\Abst{Описывается модифицированный двухэтапный
сеточный метод разделения дис\-пер\-си\-он\-но-сдви\-го\-вых смесей нормальных
законов, представляющий собой альтернативу чистому ЕМ (expectation-maximization)
ал\-го\-рит\-му. На
первом этапе этого алгоритма строится дискретная аппроксимация для
смешивающего распределения, на втором этапе подбирается абсолютно
непрерывное распределение из заранее заданного семейства, например,
обобщенных обратных гауссовских законов, ближайшее к~дискретному
распределению, полученному на первом этапе. Обсуждаются вопросы
сходимости этого двухэтапного алгоритма. Доказана монотонность
сеточного итерационного метода, используемого на первом этапе.
Подробно обсуждается вопрос оптимального выбора параметров метода,
прежде всего сетки, накидываемой на носитель смешивающего
распределения. С~этой целью предложены статистические оценки
квантилей смешивающего распределения. Эффективность метода
иллюстрируется примерами конкретных вычислений оценок параметров
обобщенных гиперболических распределений.}

\KW{смесь распределений вероятностей;
дис\-пер\-си\-он\-но-сдви\-го\-вая смесь нормальных законов; обобщенное
гиперболическое распределение; ЕМ-ал\-го\-ритм; сеточный метод
разделения смесей}

\vspace*{1pt}

%\vspace*{2pt}

\DOI{10.14357/19922264140402}


\vskip 12pt plus 9pt minus 6pt

\thispagestyle{headings}

\begin{multicols}{2}

\label{st\stat}

\section{Введение}

При {\it практическом} решении задачи моделирования и исследования
волатильности (изменчивости) хаотических стохастических процессов
ключевым этапом является статистическое разделение смесей
вероятностных распределений. Задача разделения смесей~---
статистического оценивания параметров смесей вероятностных
распределений~--- в~деталях разобрана, например, в~книге~\cite{k2011}.

Для решения задачи разделения смесей вероятностных распределений
традиционно используются итерационные процедуры типа ЕМ-ал\-го\-рит\-ма.
К~сожалению, классический ЕМ-ал\-го\-ритм обладает рядом серьезных
недостатков при его применении к~смесям нормальных законов, а~именно:
он демонстрирует крайнюю неустойчивость по отношению к~исходным
данным и~начальным приближениям.

Для преодоления этих недостатков
предложено много модификаций ЕМ-ал\-го\-рит\-ма (см., например,~\cite{k2011}).
Вместе с тем в~указанной книге предложен и~исследован
принципиально новый~--- сеточный~--- метод приближенного решения
задачи разделения смесей. В~работе~\cite{n2013} подробно исследованы
вопросы сходимости сеточных методов разделения смесей.

В соответствии с подходом к~статистическому анализу хаотических
стохастических процессов, в~частности к~решению задачи декомпозиции
волатильности таких процессов, развитом в~книге~\cite{k2011},
в~общем случае на практике приходится решать задачу разделения
конечных смесей нормальных законов с~произвольно большим числом
неизвестных параметров (параметров компонент и~их весов).
И~хотя в~большинстве приложений возникают смеси не более чем с~пятью--семью
компонентами, даже при использовании таких смесей, скажем, в~задачах
анализа и~прогнозирования финансовых рисков приходится моделировать
траекторию движения точки в~пространствах, размерность которых
соответственно лежит в~пределах от~14 (для пятикомпонентных смесей)
до~20 (для семикомпонентных смесей), что существенно увеличивает
вычислительные и~временн$\acute{\mbox{ы}}$е ресурсы, необходимые для практического
решения указанных задач.

Поскольку во многих ситуациях (например,
при прогнозировании на основе высокочастотных данных) эти задачи
необходимо решать в~режиме, близком к~реальному времени, для
создания эффективных методов статистического анализа на основе
смешанных моделей на первый план выходит проб\-ле\-ма снижения
размерности решаемой задачи, т.\,е.\ параметрического пространства.

Одним из возможных подходов к~снижению размерности является
априорное сужение классов допусти\-мых смесей. К~примеру, при решении
многих задач, связанных с~анализом процессов атмосферной или
плазменной турбулентности, а~так\-же процессов, описывающих эволюцию
различных финансовых индексов, высочайшую адекватность
продемонстрировали модели, основанные на дис\-пер\-си\-он\-но-сдви\-го\-вых
смесях нормальных законов. Класс таких смесей очень обширен
и,~в~част\-ности, включает в~себя обобщенные гиперболические распределения,
которые были введены О.-Е.~Барн\-дорфф-Ниль\-се\-ном в~1977--1978~гг.\ как
класс специальных сдвиг-мас\-штаб\-ных смесей нормальных законов~\cite{BN1977, BN1978}.
Пусть $\alpha\hm\in\r$, $\beta\hm\in\r$. Если
функцию распределения обобщенного гиперболического закона
с~параметрами~$\alpha$, $\beta$, $\nu$, $\mu$, $\lambda$ обозначить
$P_{GH}(x;\alpha,\beta,\nu,\mu,\lambda)$, то по определению
\begin{multline}
P_{GH}(x;\alpha,\beta,\nu,\mu,\lambda)={}\\
{}=
\int\limits_{0}^{\infty}\Phi\left(\fr{x-\beta-\alpha
z}{\sqrt{z}}\right)\,p_{GIG}(z;\nu,\mu,\lambda)\,dz\,,\\
x\in\r\,,
\label{e1-kor}
\end{multline}
где $\Phi(x)$~--- стандартная нормальная функция распределения:
$$
\Phi(x)=\int\limits_{-\infty}^{x}\varphi(z)\,dz\,,\enskip
\varphi(x)=\fr{1}{\sqrt{2\pi}}e^{-x^2/2}\,,\enskip  x\in\mathbb{R}\,;
$$
$p_{GIG}(x;\nu,\mu,\lambda)$~--- плот\-ность обобщенного обратного
гауссовского распределения:
\begin{multline*}
p_{GIG}(x;\nu,\mu,\lambda)={}\\
{}=\fr{\lambda^{\nu/2}}{2\mu^{\nu/2}
K_{\nu}\left(\sqrt{\mu\lambda}\right)}\,
x^{\nu-1}\exp\left\{-\fr{1}{2}\left(\fr{\mu}{x}+\lambda
x\right)\right\}\,,\\ x>0\,.
\end{multline*}
Здесь $\nu\in\r$;
$$
\begin{array}{lll}
\mu>0\,, & \lambda\geqslant0\,, & \mbox{если }\nu<0\,;\\[6pt]
\mu>0\,, & \lambda>0\,, & \mbox{если }\nu=0\,;\\[6pt]
\mu\geqslant0\,, & \lambda>0\,, & \mbox{если }\nu>0\,;
\end{array}
$$
$K_{\nu}(z)$~--- модифицированная бесселева функция третьего рода
порядка~$\nu$:

\noindent
\begin{multline*}
K_{\nu}(z)=\fr{1}{2}\int\limits_{0}^{\infty}y^{\nu-1}\exp
\left\{-\fr{z}{2}\left(y+\fr{1}{y}\right)\right\}\,dy\,,\\
z\in\mathbb{C}\,,\enskip \mathrm{Re}\,z>0\,.
\end{multline*}
Обратим внимание, что в~(1) смешивание происходит одновременно и~по
параметру сдвига, и~по параметру масштаба, но так как эти параметры
в~(1)  связаны жесткой зависимостью, так что параметр сдвига
смешиваемого распределения пропорционален его дисперсии, то
фактически смесь~(1) является {\it однопараметрической} и~поэтому
называется {\it дис\-пер\-си\-он\-но-сдви\-го\-вой} (см., например,~\cite{BN1982}).

Другим примером дис\-пер\-си\-он\-но-сдви\-го\-вых смесей нормальных законов
являются обобщенные дисперсионные гам\-ма-рас\-пре\-де\-ле\-ния, в~которых
смешивающими являются обобщенные гам\-ма-рас\-пре\-де\-ле\-ния~\cite{ks2012, zk2013}.

В указанных семействах смесей число неизвестных параметров равно
пяти или шести (если\linebreak учитывать неслучайный сдвиг). Вместе
с~тем у~подоб\-ных моделей имеются довольно серьезные тео\-ре\-ти\-че\-ские
обоснования: в~работах~\cite{zk2013, k2013} показано, что указанные
модели являются асимптотическими аппроксимациями в~простой
предельной схеме случайного суммирования и~потому могут успешно
применяться для анализа процессов типа остановленных случайных
блужданий. Эти выводы подтверждены статистическим анализом
вы\-со\-ко\-час\-тот\-ных финансовых данных, в~результате которого выявлен
синхронизированный характер изменения интенсивностей потоков заявок
в~сис\-те\-мах электронных торгов, что естественно приводит к~синхронизированному
поведению па\-ра\-мет\-ров сдвига и~диффузии в~соответствующих моделях вида смесей
нормальных законов~\cite{kckg2013}.

\section{Описание моди\-фи\-ци\-ро\-ван\-но\-го
сеточного ме\-то\-да разделения дисперсионно-сдвиговых смесей
нормальных законов и~его свойства}

Оказывается, что сеточные методы разделения смесей довольно
эффективны не только при разделении конечных смесей нормальных
законов, но и~при разделении произвольных дис\-пер\-си\-он\-но-сдви\-го\-вых
смесей нормальных законов. Поясним сказанное на примере задачи
оценивания па\-ра\-мет\-ров обобщенных гиперболических распределений.

Для решения задачи оценивания параметров обобщенных гиперболических
распределений традиционно используется метод, предложенный в~статье~\cite{p2004}
и~по сути являющийся классическим ЕМ-ал\-го\-рит\-мом,
приспособленным к~конкретной задаче, и,~соответственно, наследующий
присущие ЕМ-ал\-го\-рит\-мам недостатки.

Рассмотрим следующий альтернативный двухэтапный метод. На первом
этапе на поло\-жи\-тельной полупрямой выделим основную часть носителя
смешивающего распределения, т.\,е.\ \mbox{ограниченный} интервал,
вероятность которого, вычисленная в~соответствии со смешивающим
распределением, практически равна единице. На этот интервал накинем
конечную сетку, содержащую, возможно, очень много {\it известных}
узлов $u_1,\ldots,u_K$. Считая параметр сдвига~$\beta$ равным нулю,
приблизим искомое обобщенное гиперболическое распределение конечной
смесью нормальных законов:

\noindent
\begin{multline}
P_{GH}(x;\,\alpha,0,\nu,\mu,\lambda)\approx{}\\
{}\approx \sum\limits_{i=1}^K
p_i\Phi\left(\fr{x-\alpha u_i}{\sqrt{u_i}}\right)\,,\enskip
x\in\mathbb{R}\,.\label{e2-kor}
\end{multline}
В смеси, стоящей в~правой части соотношения~(2), неизвестными
являются только параметры $p_1,\ldots,p_{K-1}$ и~$\alpha$. Пусть
$x_1,\ldots,x_n$~--- анализируемая выборка значений случайной
величины с~оцениваемым обобщенным гиперболическим распределением.
Итерационный процесс, определяющий сеточный ЕМ-ал\-го\-ритм для данной
задачи, задается следующим образом. Пусть
$p_1^{(m)},\ldots,p_{K-1}^{(m)}$ и~$\alpha^{(m)}$~--- оценки параметров
$p_1,\ldots,p_{K-1}$ и~$\alpha$ на $m$-й итерации,
$p_K^{(m)}\hm=1\hm-p_1^{(m)}-\cdots-p_{K-1}^{(m)}$. Обозначим

\noindent
\begin{align*}
\varphi_{ij}^{(m)}&=\fr{1}{\sqrt{u_i}}\varphi\left(\fr{x_j-\alpha^{(m)}u_i}{\sqrt{u_i}}\right)\,;
\\
g_{ij}^{(m)}&=\fr{p_i^{(m)}\varphi_{ij}^{(m)}}{\sum\limits_{r=1}^K
p_r^{(m)}\varphi_{rj}^{(m)}}\,,\\
&\hspace*{14mm}i=1,\ldots,K\,;\enskip j=1,\ldots,n\,.
\end{align*}
Тогда, используя стандартные рассуждения, определяющие
вычислительные формулы EM-ал\-го\-рит\-ма для параметров конечной смеси
нормальных законов (см, например,~[1, разд.~5.3.7--5.3.8]),
следует положить

\noindent
\begin{equation}
p_i^{(m+1)}=\fr{1}{n}\sum\limits_{j=1}^n g_{ij}^{(m)}\,, \enskip
i=1,\ldots,K\,.\label{e3-kor}
\end{equation}
Обозначим $\overline{x}=(1/n)\sum\limits_{j=1}^nx_j$. Используя
соотношение~(5.3.24) в~\cite{k2011}, с~учетом очевидного равенства
$\sum\limits_{i=1}^K g_{ij}^{(m)}\hm=1$ можно заметить, что уточненная
оценка параметра~$\alpha$ имеет вид:

\columnbreak

\noindent
\begin{equation}
\alpha^{(m+1)}=\fr{\overline{x}}{\sum\limits_{i=1}^K u_ip_i^{(m+1)}}\,,
\label{e4-kor}
\end{equation}
т.\,е.\ равна отношению генерального выборочного среднего и~текущего
эмпирического среднего смешивающего распределения, что вполне
согласуется с~тем, что в~соответствии с~приводимым ниже соотношением~(\ref{e5-kor})
в~данном случае ${\sf E}X\hm=\alpha{\sf E}U$.

В силу монотонности классического ЕМ-ал\-го\-рит\-ма справедливо следующее
утверждение.

\smallskip

\noindent
\textbf{Теорема~1.} {\it Пусть узлы $u_1,\ldots,u_K$ сетки различны,
неотрицательны и~известны. Тогда итерационный процесс $(3)$--$(4)$
является монотонным, т.\,е.\ каждая его итерация не уменьшает
целевую сеточную функцию правдоподобия}
\begin{multline*}
L(p_1,\ldots,p_K,\alpha;x_1,\ldots,x_n)={}\\
{}=
\prod\nolimits_{j=1}^n\left[\sum\nolimits_{i=1}^K
\fr{p_i}{\sqrt{u_i}}\,\varphi\left(\fr{x_j-\alpha^{(m)}u_i}{\sqrt{u_i}}\right)\right].
\end{multline*}

\smallskip

\noindent
\textbf{Замечание~1.} В~разд.~5.7.4 книги~\cite{k2011} показано, что
при каждом фиксированном значении параметра~$\alpha$ сеточная
функция правдоподобия\linebreak
$L(p_1,\ldots,p_{K-1},\alpha;\,x_1,\ldots,x_n)$ вогнута по
аргументам $p_1,\ldots,p_{K-1}$. Поэтому на каждом шаге
итерационного процесса вместо соотношения~(3) можно\linebreak использо\-вать
любой более быстрый алгоритм максимизации функции
$L(p_1,\ldots,p_{K-1},\alpha^{(m)};\,x_1,\ldots$\linebreak $\ldots,x_n)$ по переменным
$p_1,\ldots,p_{K-1}$. Например, оценки весов $p_1,\ldots,p_K$ можно
искать методом условного градиента~\cite{k2011, kn2010}.

\smallskip

Таким образом, на первом этапе получаются оценки параметра~$\alpha$
и~весов всех узлов~$u_i$ конечной сетки, накинутой на носитель
смешивающего обобщенного обратного гауссовского распределения
$P_{\mathrm{GIG}}(z;\,\nu,\mu,\lambda)$.

На втором этапе остается применить ка\-кой-ли\-бо стандартный метод
подгонки обобщенного обратного гауссовского распределения
$P_{\mathrm{GIG}}(z;\,\nu,\mu,\lambda)$ к~эмпирическим данным типа
гистограммы $(u_1, p_1),\ldots, (u_K, p_K)$. Например, параметры~$\nu$,
$\mu$ и~$\lambda$ можно оценить, минимизируя соответствующую
статистику хи-квад\-рат. Или же, например, можно решить задачу
наименьших квад\-ратов:
\begin{multline*}
(\nu^*,\mu^*,\lambda^*)={}\\
{}=\arg\min\limits_{\nu,\mu,\lambda}\sum\limits_{i=1}^K
\left[p_i- \!\!\!\!\!
\int\limits_{(1/2)\left(u_{i-1}+u_i\right)}^{(1/2)(u_i+u_{i+1})}\!\!\!\!\!\!\!\!\!\!\!\!\!\!\!
p_{GIG}(u;\,\nu,\mu,\lambda)\,du\right]^2,
\end{multline*}
где $u_0=0$; $u_{K+1}\hm=\infty$.

На практике хорошие результаты показал подход с решением задачи
наименьших квадратов. Для поиска параметров использовался алгоритм
ns2sol, описанный в~книге~\cite{DSch1983}. Указанный алгоритм
доступен во многих статистических пакетах, отличается высоким
быстродействием и~возможностью при желании задавать разумные
интервалы для поиска параметров.

%\vspace*{-9pt}

\section{О практическом выборе сетки
на~первом этапе моди\-фи\-ци\-ро\-ван\-но\-го
сеточного метода разделения дисперсионно-сдвиговых смесей нормальных
законов}

Естественно, что при использовании указанного двухэтапного метода
в~динамическом режиме крайне важным становится вопрос о~выборе
наиболее эффективных и~быстродействующих численных процедур и~их
параметров. В~частности, исключительную важность приобретает
правильный выбор сетки на первом этапе. Рассмотрим этот вопрос
подробнее.

Формально рассматриваемая задача выглядит так: по наблюдаемым
значениям $x_1,\ldots,x_n$ требуется построить статистическую оценку
верхней границы квантилей заданного порядка сме\-ши\-ва\-юще\-го закона так,
чтобы как можно точнее оценить носитель смешивающего распределения.

В дальнейшем будем считать, что $x_1,\ldots,x_n$~--- независимые
реализации случайной величины $X\hm=Y\sqrt{U}+\alpha U$, где $Y$~---
случайная величина со стандартным нормальным распределением, а~$U$~---
независимая от нее случайная величина с~обобщенным обратным
гауссовским распределением. Тогда, очевидно, распределение случайной
величины~$X$ имеет вид~(1). Предположим, что у~случайной величины~$U$
существуют моменты первых двух порядков. Тогда, как несложно видеть,
\begin{equation}
{\sf E}X={\sf E}Y\cdot{\sf E}\sqrt{U}+\alpha{\sf E}U=\alpha{\sf
E}U\,.\label{e5-kor}
\end{equation}
При этом по усиленному закону больших чисел с~вероятностью единица
$\overline x\hm\longrightarrow {\sf E}X$ $(n\hm\to\infty)$, так что при
больших~$n$ справедливо приближенное равенство ${\sf E}X\hm\approx\overline x$
и~с учетом~(\ref{e5-kor})
\begin{equation}
{\sf E}U\approx\fr{\overline x}{\alpha}\,.\label{e6-kor}
\end{equation}
Далее, очевидно,

\columnbreak

\noindent
\begin{multline}
{\sf E}X^2={\sf E}Y^2\cdot{\sf E}U+2\alpha{\sf E}X\cdot{\sf E}U^{3/2}+{}\\
{}+
\alpha^2{\sf E}U^2={\sf E}U+\alpha^2{\sf E}U^2\,.
\label{e7-kor}
\end{multline}

\noindent
Поэтому, обозначив
$$
m^2=\fr{1}{n}\sum\limits_{i=1}^nx_i^2\,,
$$
получаем приближенное равенство ${\sf E}X^2\hm\approx m^2$, так что
с~учетом~(\ref{e6-kor}) и~(\ref{e7-kor}) имеем:
\begin{equation}
{\sf E}U^2\approx\fr{1}{\alpha^2}\left(m^2-\fr{\overline
x}{\alpha}\right)\,.\label{e8-kor}
\end{equation}
Если параметр~$\alpha$ известен, то для определения верхней границы~$u^*$
сетки, накидываемой на носитель распределения случайной
величины~$U$, можно задать малое положительное число~$\varepsilon$
и~воспользоваться требованием
\begin{equation}
{\sf P}(U\geqslant u^*)\leqslant\varepsilon\,.\label{e9-kor}
\end{equation}
А~для гарантированного выполнения требования~(\ref{e9-kor}) можно использовать
неравенство Маркова:
$$
{\sf P}(U\geqslant u^*)\leqslant\fr{{\sf E}U^2}{(u^*)^2}\leqslant \varepsilon\,,
$$
откуда с учетом~(\ref{e8-kor})
$$
(u^*)^2\geqslant\fr{{\sf E}U^2}{\varepsilon}\approx
\fr{1}{\alpha^2\varepsilon}\left( m^2-\fr{\overline x}{\alpha}\right)
$$
или
\begin{equation}
u^*\approx\fr{1}{\alpha\sqrt{\varepsilon}}\sqrt{m^2-
\fr{\overline x}{\alpha}}\,.\label{e10-kor}
\end{equation}

\begin{figure*}[b] %fig1
\vspace*{1pt}
 \begin{center}
 \mbox{%
 \epsfxsize=161.718mm
 \epsfbox{kor-1.eps}
 }
 \end{center}
 \vspace*{-9pt}
\Caption{Примеры применения модифицированного двухэтапного сеточного
ЕМ-ал\-го\-рит\-ма для подгонки обобщенного гиперболического распределения
к искусственным данным, $\beta\hm=0$: (\textit{a})~$n\hm=1000$, $\alpha\hm=0{,}3$,
$\nu\hm=1{,}3$, $\mu\hm=1{,}6$, $\lambda\hm=0{,}2$;
(\textit{б})~$n\hm=1000$, $\alpha\hm=0{,}5$, $\nu\hm=1$, $\mu\hm=1$,
$\lambda\hm=3$;
(\textit{в})~$n\hm=1000$, $\alpha\hm=3$,
 $\nu\hm=1{,}3$, $\mu\hm=1{,}6$, $\lambda\hm=2$;
(\textit{г})~$n\hm=10\,000$,
$\alpha\hm=0{,}3$, $\nu\hm=1{,}3$, $\mu\hm=1{,}6$, $\lambda\hm=0{,}2$}
\end{figure*}


Если же параметр~$\alpha$, определяющий асим\-мет\-рию распределения
случайной величины~$X$, неизвестен, то можно воспользоваться
следующими рассуждениями. Обозначим
$$
q_n=\fr{1}{n}\sum\limits_{i=1}^n{\bf 1}(x_i<0)\,,
$$
где ${\bf 1}(A)$~--- индикаторная функция множества (события)~$A$.
При этом по усиленному закону больших чисел с~вероятностью единица
$q_n\hm\longrightarrow {\sf P}(X\hm<0)$ $(n\hm\to\infty)$, так что при
больших~$n$ справедливо приближенное равенство
\begin{equation}
q_n\approx{\sf P}(X<0)\,.\label{e11-kor}
\end{equation}
Но
\begin{multline}
{\sf P}(X<0)=\int\limits_{0}^{\infty}\Phi
\left(-\alpha\sqrt{u}\right) p_{\mathrm{GIG}}(u;\nu,\mu,\lambda)\,du={}\\
{}=
{\sf E}\Phi\left(-\alpha\sqrt{U}\right)\,.\label{e12-kor}
\end{multline}

\pagebreak

\noindent
Предположим сначала, что $q_n\hm<1/2$. Если~$n$ достаточно велико,
то можно с~большой степенью
 уверенности утверж\-дать, что тогда
$\overline x\hm>0$ и~$-\alpha\hm<0$, т.\,е.
 $\alpha\hm>0$ и,~стало быть, на
положительной полуоси значений аргумента~$u$ функция $\Phi(\alpha u)$
вогнута, т.\,е.\ выпукла вверх. Тогда из~(\ref{e11-kor}) и~(\ref{e12-kor}), дважды
применяя неравенство Иенсена, в~силу монотонности функции~$\Phi$
получаем:
\begin{multline}
1-q_n\approx 1-{\sf E}\Phi\left(-\alpha\sqrt{U}\right)=
          {\sf E}\Phi\left(\alpha\sqrt{U}\right)\leqslant{}\\
          {}\leqslant\Phi
          \left(\alpha{\sf E}\sqrt{U}\right)\leqslant
          \Phi\left(\alpha\sqrt{{\sf E}U}\right)\,.\label{e13-kor}
\end{multline}
Если теперь для $t\hm\in(0,1)$ символом~$v_t$ обозначить $t$-кван\-тиль
стандартного нормального закона, то из~(\ref{e13-kor}) и~(\ref{e6-kor}) вытекает
<<приближенное неравенство>>
$$
v_{1-q_n}\hm\leqslant \alpha\sqrt{{\sf E}U}\,,
$$
т.\,е.
$$
\alpha\geqslant\fr{v_{1-q_n}}{\sqrt{{\sf E}U}}\approx
\fr{v_{1-q_n}\sqrt{\alpha}}{\sqrt{\overline x}}\,,
$$
откуда получаем, что при достаточно больших~$n$
\begin{equation}
\alpha\geqslant\fr{v_{1-q_n}^2}{\overline x}\,.\label{e14-kor}
\end{equation}
Если теперь задать малое положительное число~$\varepsilon$, то
для определения верхней границы~$u^*$ сетки, накидываемой на
носитель распределения случайной величины~$U$, можно воспользоваться
требованием~(\ref{e9-kor}), для гарантированного выполнения которого
с~учетом~(\ref{e6-kor}) и~(\ref{e14-kor}) можно использовать неравенство Маркова:
$$
{\sf P}(U\geqslant u^*)\leqslant \fr{{\sf E}U}{u^*}\approx\fr{\overline
x}{\alpha u^*}\leqslant \fr{(\overline x)^2}{v_{1-q_n}^2 u^*}\leqslant
\varepsilon\,,
$$
откуда окончательно вытекает оценка
\begin{equation}
u^*\approx\fr{(\overline x)^2}{v_{1-q_n}^2 \varepsilon}\,.\label{e15-kor}
\end{equation}

\begin{figure*}[b] %fig2
\vspace*{18pt}
 \begin{center}
 \mbox{%
 \epsfxsize=162.433mm
 \epsfbox{kor-3.eps}
 }
 \end{center}
 \vspace*{-9pt}
\Caption{Примеры применения модифицированного двухэтапного
сеточного ЕМ-ал\-го\-рит\-ма для подгонки обобщенного гиперболического
распределения к~искусственным данным, $n=10\,000$, $\beta\hm=0$:
(\textit{а})~$\alpha\hm=0{,}3$,
$\nu\hm=2$, $\mu\hm=2$, $\lambda\hm=2{,}5$;
(\textit{б})~$\alpha\hm=0{,}5$,  $\nu\hm=1$, $\mu\hm=1$, $\lambda\hm=3$;
(\textit{в})~$\alpha\hm=0{,}8$,
$\nu\hm=1{,}3$, $\mu\hm=1{,}6$, $\lambda\hm=2$;
(\textit{г})~$\alpha\hm=1{,}3$, $\nu\hm=2$, $\mu\hm=2$, $\lambda\hm=2{,}5$}
\end{figure*}



В случае $q_n\hm\geqslant1/2$, если $n$ достаточно велико, то можно
с~большой степенью уверенности утверж\-дать, что $\overline x\hm\leqslant 0$
и~$-\alpha\hm\geqslant 0$, т.\,е.\ на положительной\linebreak\vspace*{-12pt}

\pagebreak

%\end{multicols}


%\begin{multicols}{2}

\noindent
 полуоси значений аргумента~$u$
функция $\Phi(-\alpha u)$ вогнута, т.\,е.\ выпукла вверх. Тогда
из~(\ref{e11-kor}) и~(\ref{e12-kor}), дважды применяя неравенство Иенсена, в~силу
монотонности функции~$\Phi$ получаем
$$
q_n\approx {\sf E}\Phi\left(-\alpha\sqrt{U}\right)\leqslant
\Phi\left(-\alpha\sqrt{{\sf E}U}\right)\,,
$$
откуда вытекает <<приближенное неравенство>> $v_{q_n}\hm \leqslant
-\alpha\sqrt{{\sf E}U}$,
т.\,е.
$$
-\alpha\geqslant\fr{v_{q_n}}{\sqrt{{\sf E}U}}\approx
\fr{v_{q_n}\sqrt{|\alpha|}}{\sqrt{|\overline x|}}
$$
и при достаточно больших~$n$
\begin{equation}
|\alpha|\geqslant\fr{v_{q_n}^2}{|\overline x|}\,.\label{e16-kor}
\end{equation}
Для определения верхней границы~$u^*$ сетки, накидываемой на
носитель распределения случайной величины~$U$, снова зададим малое
положительное число~$\varepsilon$ и~потребуем, чтобы было
справедливо условие~(\ref{e9-kor}), для гарантированного выполнения которого
с~учетом~(\ref{e6-kor}) и~(\ref{e16-kor}) используем неравенство Маркова и~тот факт, что
$\mathrm{sign}\, \overline x\hm=\mathrm{sign}\,\alpha$ при достаточно
больших~$n$:
\begin{multline}
{\sf P}(U\geqslant u^*)\leqslant \fr{{\sf E}U}{u^*}\approx
\fr{\overline x}{\alpha u^*}=
\fr{|\overline x|}{|\alpha| u^*} \leqslant{}\\
{}\leqslant
\fr{(\overline x)^2}{v_{q_n}^2 u^*}\leqslant
\varepsilon\,.\label{e17-kor}
\end{multline}
В силу симметричности нормального распределения $v_{t}\hm=-v_{1-t}$ для
любого $t\hm\in(0,1)$, поэтому $v_{q_n}^2\hm=v_{1-q_n}^2$ и~в~случае
$q_n\hm\geqslant1/2$ соотношение~(\ref{e17-kor}) снова приводит к~оценке~(\ref{e15-kor}).

Справедливости ради необходимо отметить, что оценки~(\ref{e10-kor}) и~(\ref{e15-kor})
являются завышенными, но они гарантируют, что
$(1-\varepsilon)$-почти-весь носитель распределения случайной
величины~$U$ будет лежать внутри интервала $[0, u^*]$.

\section{Результаты численных экспериментов}

Приводимые в~данном разделе графики иллюстрируют качество работы
модифицированного сеточного метода разделения дис\-пер\-си\-он\-но-сдви\-го\-вых
смесей нормальных законов на примере его\linebreak применения к~оцениванию
параметров обоб\-щенных гиперболических распределений с~ис\-поль\-зованием
указанного алгоритма выбора сетки\linebreak с~умеренным чис\-лом узлов $K\hm=40$.
Для вы\-чис\-ле\-ний использовались искусственно сгенерированные выборки
объемов $n\hm=1000$ и~$n\hm=10\,000$ с~разными наборами параметров, значения
которых указаны на рисунках. На рис.~1 и~2 изображены гистограммы
(серые столбики) и~графики
истинной плот\-ности (штриховые линии), промежуточной
оценки, полученной сеточным ЕМ-ал\-го\-рит\-мом (пунктирные линии)
и~итоговой оценки (непрерывные линии). На рис.~1 и~2 так\-же указаны
значения полученных оценок параметров. Как видно из приводимых
рисунков, параметры~$\alpha$ оцениваются очень точно. Точность
оценок остальных параметров удовлетворительная и~может быть повышена
за счет использования более частых сеток и~более чувствительных
критериев остановки ЕМ-ал\-го\-рит\-ма на первом этапе. Следует отметить,
что даже в~тех случаях, в~которых наблюдаются заметные расхождения
оценок параметров и~их точных значений, оценки самих плотностей
довольно \mbox{точны}.




{\small\frenchspacing
 {%\baselineskip=10.8pt
 \addcontentsline{toc}{section}{References}
 \begin{thebibliography}{99}
\bibitem{k2011}
\Au{Королев В.\,Ю.} Ве\-ро\-ят\-но\-ст\-но-ста\-ти\-сти\-че\-ские методы
декомпозиции волатильности хаотических процессов.~--- М.: Изд-во
Московского ун-та, 2011.

\bibitem{n2013}
\Au{Назаров А.\,Л.} Приближенные методы разделения смесей
вероятностных распределений: Дисс.\ \ldots\  канд. физ.-мат. наук.~--- М.:
МГУ им.\ М.\,В.~Ломоносова, 2013.

\bibitem{BN1977}
\Au{Barndorff-Nielsen~O.-E.} Exponentially decreasing distributions
for the logarithm of particle size~// Proc. Roy. Soc. Lond.~A,
1977. Vol.~353. P.~401--419.

\bibitem{BN1978}
\Au{Barndorff-Nielsen~O.-E.} Hyperbolic distributions and
distributions of hyperbolae~// Scand. J. Statist., 1978. Vol.~5.
P.~151--157.

\bibitem{BN1982}
\Au{Barndorff-Nielsen~O.-E., Kent~J., S\!{\!\ptb{\!\o}}\,rensen~M.} Normal
variance-mean mixtures and $z$-distributions~// Int. Statist. Rev.,
1982. Vol.~50. No.\,2. P.~145--159.

\bibitem{ks2012}
\Aue{Королев В.\,Ю., Соколов И.\,А.} Скошенные распределения
Стьюдента, дисперсионные гам\-ма-рас\-пре\-де\-ле\-ния и~их обобщения как
асимптотические аппроксимации~// Информатика и~её применения, 2012.
Т.~6. Вып.~1. С.~2--10.

\bibitem{zk2013}
\Au{Закс Л.\,М., Королев В.\,Ю.} Обобщенные дисперсионные
гам\-ма-рас\-пре\-де\-ле\-ния как предельные для случайных сумм~// Информатика
и её применения, 2013. Т.~7. Вып.~1. С.~105--115.

\bibitem{k2013}
\Au{Королев В.\,Ю.} Обобщенные гиперболические
распределения как предельные для случайных сумм~// Тео\-рия
вероятностей и~ее применения, 2013. Т.~58. Вып.~1. С.~117--132.

\bibitem{kckg2013}
\Au{Королев В.\,Ю., Черток А.\,В., Корчагин~А.\,Ю.,
Горшенин~А.\,К.} Ве\-ро\-ят\-но\-ст\-но-ста\-ти\-сти\-че\-ское моделирование
информационных потоков в~сложных финансовых системах на основе
высокочастотных данных~// Информатика и~её применения, 2013. Т.~7.
Вып.~1. С.~12--21.

\bibitem{p2004}
\Au{Protassov R.\,S.} EM-based maximum likelihood parameter
estimation for a~multivariate generalized hyperbolic distribution
with fixed~$\lambda$~// Statistics Computing, 2004. Vol.~14.
P.~67--77.

\bibitem{kn2010}
\Au{Королев В.\,Ю., Назаров А.\,Л.} Разделение смесей
вероятностных распределений при помощи сеточных методов моментов и~максимального правдоподобия~//
Автоматика и~телемеханика, 2010. Вып.~3. С.~98--116.

\bibitem{DSch1983}
\Au{Dennis J.\,E., Schnabel R.\,B.} Numerical methods for
unconstrained optimization and nonlinear equations.~--- Englewood
Cliffs: Prentice-Hall, 1983. 378~p.
 \end{thebibliography}

 }
 }

\end{multicols}

\vspace*{-6pt}

\hfill{\small\textit{Поступила в редакцию 01.10.14}}

\newpage

%\vspace*{12pt}

%\hrule

%\vspace*{2pt}

%\hrule

%\vspace*{12pt}

\def\tit{A MODIFIED GRID METHOD FOR~STATISTICAL SEPARATION
OF~NORMAL VARIANCE-MEAN MIXTURES}

\def\titkol{A modified grid method for statistical separation
of~normal variance-mean mixtures}

\def\aut{V.\,Yu.~Korolev$^{1,2}$ and~A.\,Yu.~Korchagin$^1$}

\def\autkol{V.\,Yu.~Korolev and~A.\,Yu.~Korchagin}

\titel{\tit}{\aut}{\autkol}{\titkol}

\vspace*{-9pt}


\noindent
$^1$Faculty of Computational Mathematics and Cybernetics,
M.\,V.~Lomonosov Moscow State University,\linebreak
$\hphantom{^1}$1-52 Leninskiye Gory, GSP-1, Moscow 119991, Russian Federation


\noindent
$^2$Institute of Informatics Problems, Russian Academy of Sciences,
44-2~Vavilov Str., Moscow 119333, Russian\linebreak
$\hphantom{^1}$Federation

\def\leftfootline{\small{\textbf{\thepage}
\hfill INFORMATIKA I EE PRIMENENIYA~--- INFORMATICS AND
APPLICATIONS\ \ \ 2014\ \ \ volume~8\ \ \ issue\ 4}
}%
 \def\rightfootline{\small{INFORMATIKA I EE PRIMENENIYA~---
INFORMATICS AND APPLICATIONS\ \ \ 2014\ \ \ volume~8\ \ \ issue\ 4
\hfill \textbf{\thepage}}}

\vspace*{3pt}

\Abste{A~modified two-stage grid method for
statistical separation of normal variance-mean mixtures is described
as an alternative to a pure EM (expectation-maximization) algorithm.
At the first stage of this
algorithm, a~discrete approximation is constructed to the mixing
distribution. At the second stage, the obtained discrete
distribution is approximated by an absolutely continuous
distribution from a~predetermined family, say, by a generalized
inverse Gaussian distribution. The convergence of this two-stage
procedure is discussed. The monotonicity of the grid procedure used
at the first stage is proved. The problem of the optimal choice of
the parameters of the method is discussed in detail. First of all,
the problem of the optimal choice of the grid thrown on the support
of the mixing distribution is considered. Statistical estimators are
proposed for the quantiles of the mixing law. The efficiency of the
method is illustrated by examples of its application to the
estimation of the parameters of generalized hyperbolic
distributions.}

\smallskip

\KWE{mixture of probability distributions; normal
variance-mean mixture; generalized hyperbolic distribution;
EM-algorithm; grid method of separation of mixtures}

\DOI{10.14357/19922264140402}

\Ack
\noindent
The research was supported by the Russian Science Foundation (project 14-11-00364).

%\vspace*{3pt}

  \begin{multicols}{2}

\renewcommand{\bibname}{\protect\rmfamily References}
%\renewcommand{\bibname}{\large\protect\rm References}



{\small\frenchspacing
 {%\baselineskip=10.8pt
 \addcontentsline{toc}{section}{References}
 \begin{thebibliography}{99}
 \bibitem{k2011eng}
 \Aue{Korolev, V.\,Yu.} 2011.
\textit{Veroyatnostno-statisticheskie metody dekompozitsii
volatil'nosti khaoticheskikh protsessov}
[Probabilistic and statistical methods for the decomposition of volatility
of chaotic processes].
Moscow: Moscow University Press. 510~p.

\bibitem{n2013eng}
\Aue{Nazarov, A.\,L.} 2013.
{Priblizhennye metody razdeleniya smesey veroyatnostnykh raspredeleniy}
[Approximate methods for the decomposition of volatility of chaotic processes].
Ph.D. Thesis. Moscow: Moscow State University.

\bibitem{BN1977eng}
\Aue{Barndorff-Nielsen, O.\,E.} 1977.
Exponentially decreasing distributions for the logarithm of particle size.
\textit{Proc. Roy. Soc. Lond. A} 353:401--419.

\bibitem{BN1978eng}
\Aue{Barndorff-Nielsen, O.\,E.} 1978.
Hyperbolic distributions and distributions of hyperbolae.
\textit{Scand. J. Statist.} 5:151--157.

\bibitem{BN1982eng}
\Aue{Barndorff-Nielsen, O.\,E., J.~Kent, and M.~S\!{\ptb{\o}}rensen}. 1982.
Normal variance-mean mixtures and $z$-distributions.
\textit{Int. Statist. Rev.} 50(2):145--159.

\bibitem{ks2012eng}
\Aue{Korolev, V.\,Yu., and I.\,A. Sokolov}. 2012.
{Skoshennye raspredeleniya St'yudenta, dispersionnye
gam\-ma-ras\-pre\-de\-le\-niya i~ikh obobshcheniya kak asimptoticheskie
approksimatsii}
[Skewed Student's distributions, variance gamma distributions, and their
generalizations as asymptotic approximations].
\textit{Informatika i ee Primeneniya}~--- \textit{Inform. Appl.} 6(1):2--10.

\bibitem{zk2013eng}
\Aue{Korolev, V.\,Yu., and L.\,M.~Zaks}. 2013.
{Obobshchennye dispersionnye gam\-ma-ras\-pre\-de\-le\-niya kak
predel'nye dlya sluchaynykh summ}
[Generalized variance gamma distributions as limiting for random sums].
\textit{Informatika i ee Primeneniya}~--- \textit{Inform. Appl.} 7(1):105--115.

\bibitem{k2013eng} \Aue{Korolev, V.\,Yu.} 2013.
{Obobshchennye giperbolicheskie raspredeleniya kak predel'nye dlya sluchaynykh summ}
[Generalized hyperbolic distributions as limiting for random sums]
\textit{Theory Probab. Appl.} 58(1):117--132.

\bibitem{kckg2013eng}
\Aue{Korolev, V.\,Yu., A.\,V. Chertok, A.\,Yu.~Korchagin, and A.\,K.~Gorshenin}.
2013. {Ve\-ro\-yat\-no\-st\-no-sta\-ti\-sti\-che\-skoe
mo\-de\-li\-ro\-va\-nie informatsionnykh potokov v~slozhnykh finansovykh sistemakh
na osnove vysokochastotnykh dannykh}
[Probability and statistical modeling of information flows in complex
financial systems from high-frequency data].
\textit{Informatika i~ee Primeneniya}~--- \textit{Inform.  Appl.} 7(1):12--21.

\bibitem{p2004eng-1}
\Aue{Protassov, R.\,S.} 2004.
EM-based maximum likelihood parameter estimation for a multivariate
generalized hyperbolic distribution with fixed~$\lambda$.
\textit{Statistics Computing} 14:67--77.

\bibitem{kn2010eng-1}
\Aue{Korolev, V.\,Yu., and A.\,L.~Nazarov}. 2010.
{Razdelenie smesey veroyatnostnykh raspredeleniy pri pomoshchi
setochnykh metodov momentov i~maksimal'nogo pravdopodobiya}
[Separation of mixtures using grid moment-based methods and maximum likelihood].
\textit{Avtomatika i~Telemekhanika} [Automatics and Telemechanics] 3:98--116.

\bibitem{DSch1983eng}
\Aue{Dennis, J.\,E., and R.\,B.~Schnabel}. 1983.
\textit{Numerical methods for unconstrained optimization and nonlinear equations}.
Englewood Cliffs: Prentice-Hall. 378~p.


\end{thebibliography}

 }
 }

\end{multicols}

\vspace*{-6pt}

\hfill{\small\textit{Received October 01, 2014}}

\vspace*{-18pt}

\Contr

\noindent
\textbf{Korolev Victor Yu.} (b.\ 1954)~---
Doctor of Science in physics and mathematics, professor,
Department of Mathematical Statistics, Faculty of Computational Mathematics
and Cybernetics, M.\,V.~Lomonosov Moscow State University,
1-52 Leninskiye Gory, GSP-1, Moscow 119991, Russian Federation;
leading scientist, Institute of Informatics Problems,
Russian Academy of Sciences, 44-2~Vavilov Str., Moscow 119333, Russian
Federation; victoryukorolev@yandex.ru

\vspace*{3pt}

\noindent
\textbf{Korchagin Alexander Yu.} (b.\ 1989)~---
PhD student, Faculty of Computational Mathematics and Cybernetics,
M.\,V.~Lomonosov Moscow State University,
1-52 Leninskiye Gory, GSP-1, Moscow 119991, Russian Federation;
sasha.korchagin@gmail.com


\label{end\stat}

\renewcommand{\bibname}{\protect\rm Литература}   %1
\def\stat{zakharova}

\def\tit{СЕГМЕНТИРОВАНИЕ НЕСТАЦИОНАРНЫХ СИГНАЛОВ НА~ОСНОВЕ 
ВЕРОЯТНОСТНЫХ СВОЙСТВ ОКОННОЙ ДИСПЕРСИИ}

\def\titkol{Сегментирование нестационарных сигналов на основе вероятностных свойств оконной дисперсии}

\def\aut{М.\,А.~Драницына$^1$, Т.\,В.~Захарова$^2$}

\def\autkol{М.\,А.~Драницына, Т.\,В.~Захарова}

\titel{\tit}{\aut}{\autkol}{\titkol}

\index{Драницына М.\,А.}
\index{Захарова Т.\,В.}
\index{Dranitsyna M.\,A.}
\index{Zakharova T.\,V.}


%{\renewcommand{\thefootnote}{\fnsymbol{footnote}} \footnotetext[1]
%{Исследование выполнено при финансовой поддержке Российского научного фонда (проект 16-11-10227).}}


\renewcommand{\thefootnote}{\arabic{footnote}}
\footnotetext[1]{Московский государственный университет имени М.\,В.~Ломоносова,
факультет вычислительной математики и~кибернетики; \mbox{margarita13april@mail.ru}}
\footnotetext[2]{Московский государственный университет имени М.\,В.~Ломоносова, 
факультет вычислительной математики и~кибернетики; 
Институт проб\-лем информатики Федерального исследовательского центра <<Информатика 
и~управ\-ле\-ние>> Российской академии наук, \mbox{lsa@cs.msu.ru}}

%\vspace*{-18pt}


\Abst{Выделение фрагментов регистрируемого сигнала, т.\,е.\ 
его сегментация, является актуальной задачей, в~частности для биомедицинской отрасли. 
Сегментация как этап обработки сигналов, зачастую обязательный, может способствовать 
интерпретации и~классификации регистрируемых данных. Особенно сложно сегментировать 
нестационарные сигналы с~малым отношением сиг\-нал/шум. В~рамках данной работы 
основное внимание уделяется изучению шумовой компоненты оконной дисперсии как 
случайной величины в~рассматриваемых моделях. Авторами предложены модели для 
представления мультикомпонентных сигналов, а~также исследованы некоторые 
вероятностные характеристики шумовой компоненты оконной дисперсии сигналов 
как случайного процесса в~представленных моделях. Результаты работы 
согласуются с~установленными эмпирически свойствами шумовой компоненты оконной 
дисперсии (для миограммы). Полученные результаты планируется 
использовать в~практических задачах сегментирования сигналов и~выделения
 интервалов с~преобладанием тех или иных компонент процесса, а~также для 
 прогнозирования поведения сигналов.}

\KW{оконная дисперсия; модель сигнала}

\DOI{10.14357/19922264170302} 

\vspace*{2pt}


\vskip 10pt plus 9pt minus 6pt

\thispagestyle{headings}

\begin{multicols}{2}

\label{st\stat}

\section{Введение}


Выделение фрагментов регистрируемого сигнала с~различными характеристиками, т.\,е.\ 
его сегментация, является актуальной задачей, в~частности для биомедицинской отрасли 
(например, при анализе электроэнцефалограмм~\cite{Kos, Aza}, данных различных 
мониторирующих состояние здоровья устройств~\cite{Kal} и~других моно- 
и~мультикомпонентных сигналов~\cite{Z7}). Сегментация сигналов как этап их обработки, 
зачастую обязательный~\cite{Kal, Z7}, может способствовать интерпретации 
и~классификации регистрируемых данных.

Будем рассматривать некоторый нестационарный сигнал. Этот сигнал может быть 
пред\-став\-лен в~виде временного ряда, который образуют результаты измерения 
сигнала в~точках~$\tau_{k}$, $k\hm =  1, 2, \ldots, r$. При этом сигнал образован 
составляющими его процессами $A_{1}, A_{2}, \cdots, A_{m}$, каждый из которых 
может быть преобладающим на том или ином временн$\acute{\mbox{о}}$м интервале регистрации сигнала.

Примером такого сигнала может служить фармакокинетическая кривая, т.\,е.\ 
кривая, отра\-жа\-ющая зависимость концентрации вещества, чаще всего в~крови, от времени. 
Для действующего вещества лекарственного препарата, пред\-став\-ля\-юще\-го собой таблетку 
или капсулу, профиль такой кривой определяется ско\-ростью абсорбции вещества из 
просвета тонкой кишки, являющейся наиболее час\-тым абсорбирующим органом, и~количеством 
уже абсорбированного вещества, ско\-ростью распределения вещества из 
крови в~периферические ткани (с~достижением динамического равновесия) и~выведением 
его из организма как за счет метаболизма, так и~выделения соответствующими органами.

Ассоциированная с~подлежащими процессами шумовая компонента предполагается 
случайной величи\-ной. Изменение вероятностных характеристик шумовой компоненты 
оконной дисперсии будет основанием для сегментирования ре\-гист\-ри\-ру\-емо\-го 
сигнала в~дальнейшем на практике. Оконная дисперсия для выделения определенных 
участков сигналов была использована, например, для анализа 
магнитоэнцефалограмм~\cite{Z7} с~целью определения момента начала движения. 
Для выделения опорных точек на миограмме, регистрируемой параллельно 
с~магнитоэнцефалограммой, в~работе~\cite{Khazi} была предложена методология, 
которая использована и~обобщена для модели с~общей шумовой компонентой в~настоящей 
работе. Кроме того, распределение шумовой компоненты оконной дисперсии миограммы 
было охарактеризовано эмпирически в~работе~\cite{All}.

В рамках данной работы рассматриваются модели мультикомпонентных сигналов, 
при этом основное внимание уделяется исследованию изменения вероятностных 
характеристик оконной дисперсии сигнала на разных временн$\acute{\mbox{ы}}$х интервалах.

\section{Модель с~общей шумовой компонентой}

\subsection{Общее представление модели}
%\label{1.1.1}

Пусть регистрируется некий сигнал. Представим в~каждой точке~$t$ 
наблюдаемое значение~$C(t)$ в~виде суммы истинных значений процессов,
 формирующих результирующий сигнал, $A_{1}(t), A_{2}(t), \ldots, A_{m}(t)$ и~белого 
 гауссовского шума~$\xi(t)$,
характеризующегося нормальным распределением с~нулевым математическим 
ожиданием и~дисперсией~$\sigma^{2}$. Таким образом, наблюдаемый сигнал~$C(t)$ 
представим в~виде:
\begin{equation}
\label{1}
C(t) = \sum\limits_{l=1}^{m}A_{l}(t) + \xi(t)\,.
\end{equation}

Пусть $n$~--- ширина окна, т.\,е.\ число точек, используемых при расчете 
скользящего среднего~$\bar{C}_n$ сигнала~$C(t)$.
Через $C_{i}$ и~$\xi_{i}$ для $i\hm=0,1,\ldots,n-1$ обозначим соответственно 
значение сигнала и~шума в~$i$-й точке окна.

Во введенных  обозначениях для~$\bar{C}_n$ справедливо следующее представление:
\begin{multline}
\label{2}
\bar{C}_n= \fr{1}{n}\sum\limits_{i=0}^{n-1}C_{i} = \fr{1}{n}
\sum\limits_{i=0}^{n-1}\left[ \sum\limits_{l=1}^{m}A_{l,i} +
 \xi_{i} \right] = {}\\
 {}=\sum\limits_{l=1}^{m}\bar{A}_{l,n} + \bar{\xi}_{n}  \,,
\end{multline}
где $\bar{A}_{1,n}, \bar{A}_{2,n}, \ldots, \bar{A}_{m,n},$~--- 
скользящее среднее истинных составляющих регистрируемого сигнала, а~$\bar{\xi}_{n}$~--- 
скользящее среднее шума.

Оконная дисперсия по определению имеет вид:
\begin{equation}
\label{3}
W_n= \fr{1}{n}\sum\limits_{i=0}^{n-1}\left( C_{i} -\bar{C}_n \right)^2 \,.
\end{equation}

Исследуем свойства оконной дисперсии сигнала и~оконной дисперсии 
шума в~рассматриваемой модели.

\noindent
\textbf{Лемма.} \textit{Для оконной дисперсии~$W_n$ 
справедливо следующее представление}:
\begin{multline}
\label{4}
W_n= {}\\
{}=\sum\limits_{l=1}^{m}W_n^{A_l} + 
\sum\limits_{s\neq l; s, l=1}^{m}\!\!W_n^{A_l,s} + W_n^{\xi} + 
\sum\limits_{l=1}^{m}W_n^{A_l\xi}\,,
\end{multline}
где
\begin{align*}
W_n^{A_l} &= \fr{1}{n}\sum\limits_{i=0}^{n-1}A_{l,i}^{2} + \bar{A}_{l,n}^{2}\,,
\enskip l \in \{1, 2, \ldots, m \}\,;
\\
W_n^{A_l,s} &= \fr{2}{n}\sum\limits_{i=0}^{n-1}A_{l,i}\left( A_{s,i} - 
\bar{A}_{s,n}\right)\,,\\
&\hspace*{23mm} l, s \in \{1, 2, \ldots, m \}\,, l \neq s\,;
\\
W_n^{\xi} &= \fr{1}{n}\sum\limits_{i=0}^{n-1}\xi_{i}^{2} + \bar{\xi}_{n}^{2}\,;
\\
W_n^{A_l\xi} &= \fr{2}{n}\sum\limits_{i=0}^{n-1}\xi_{i}\left( A_{l,i} - 
\bar{A}_{l,n}\right)\,,\enskip l \in \{1, 2, \ldots, m \}\,.
\end{align*}

\noindent
Д\,о\,к\,а\,з\,а\,т\,е\,л\,ь\,с\,т\,в\,о\,.\ \ 
Подставим~(\ref{1}) и~(\ref{2}) в~уравнение~(\ref{3}) и~раскроем скобки, тогда
\begin{multline*}
W_n= \fr{1}{n}\sum\limits_{i=0}^{n-1}\left( \sum\limits_{l=1}^{m}\left( A_{l,i} - 
\bar{A}_{l,n} \right) + \left( \xi_{i} - \bar{\xi}_{n}\right) \right)^{2} ={}\\
{}= \fr{1}{n}\sum\limits_{i=0}^{n-1}\left( A_{1,i} - \bar{A}_{1,n} \right)^{2} + \cdots 
+ \fr{1}{n}\sum\limits_{i=0}^{n-1}\left( A_{l,i} - \bar{A}_{l,n} \right)^{2} + {}\\
{}+
\fr{1}{n}\sum\limits_{i=0}^{n-1}\left( \xi_{i} - \bar{\xi}_{n}\right)^{2} +{}\\
{}+ \sum\limits_{s\neq l; s, l=1}^{m}\left[ \fr{2}{n}\sum\limits_{i=0}^{n-1}\left( 
A_{l,i} - \bar{A}_{l,n}\right) \left( A_{s,i} - \bar{A}_{s,n}\right) \right] + {}\\
{}+
\sum\limits_{ l=1}^{m}\left[ \fr{2}{n}\sum\limits_{i=0}^{n-1}\left( A_{l,i} - 
\bar{A}_{l,n}\right) \left( \xi_{i} - \bar{\xi}_{n}\right) \right] =  {}\\
{}= \sum\limits_{l=1}^{m}W_n^{A_l} + 
\sum\limits_{s\neq l; s, l=1}^{m}W_n^{A_l,s} + W_n^{\xi} + 
\sum\limits_{l=1}^{m}W_n^{A_l\xi} \,.
\end{multline*}

Последнее равенство совпадает с~утверждением леммы.

Такое представление~(\ref{4}) оконной дисперсии может быть интерпретировано 
следующим образом:
\begin{itemize}
    \item
    компоненты $W_n^{A_l}, l \in \{ 1, 2, \ldots, m\} $, характеризуют тренд, 
    обусловленный истинными компонентами регистрируемого сигнала в~отсутствие шума;
    \item
    компоненты $W_n^{A_l,s}, s, l \in \{ 1, 2, \ldots, m\}, s\neq l, $ характеризуют 
    суперпозицию истинных компонент регистрируемого сигнала;
    \item
    компонента $W_n^{\xi}$  характеризует дисперсию случайной компоненты 
    регистрируемого сигнала~--- оконная дисперсия шума;
    \item
    компоненты $W_n^{A_l\xi}$ характеризуют суперпозицию истинных компонент и~шума.
\end{itemize}

Таким образом, оконная дисперсия рассматриваемого сигнала состоит 
из суммы компонент, обусловленных изменением истинного сигнала во времени,
 и~компонент, ассоциированных с~шумом.

Раскроем скобки для компонент $W_n^{A_l}$, $l \hm\in \{ 1, 2, \ldots, m\} $, 
и~получим:
\begin{multline*}$$
W_n^{A_l} = \fr{1}{n}\sum\limits_{i=0}^{n-1}\left( A_{l,i} - \bar{A}_{l,n}\right)^{2} 
= {}\\
{}=\fr{1}{n}\sum\limits_{i=0}^{n-1}\left( A_{l,i}^{2} + \bar{A}_{l,n}^{2} - 
2A_{l,i}\bar{A}_{l,n}\right) ={}
\\
{}= \fr{1}{n}\left[ \sum\limits_{i=0}^{n-1}\left( A_{l,i}^{2} - 2A_{l,i}\bar{A}_{l,n} 
\right) + n\bar{A}_{l,n}^{2} \right] = {}\\
{}=
\fr{1}{n} \sum\limits_{i=0}^{n-1}A_{l,i}^{2} - \bar{A}_{l,n}^{2}\,.
\end{multline*}

Для оконной дисперсии шума $W_n^{\xi}$ справедливы аналогичные 
преобразования и~представление:
\begin{multline*}
\hspace*{-7pt}W_n^{\xi} = \fr{1}{n}\sum\limits_{i=0}^{n-1}\left( \xi_{i} - \bar{\xi}_{n}\right)^{2} = \frac{1}{n}\sum\limits_{i=0}^{n-1}
\left( \xi_{i}^{2} + \bar{\xi}_{n}^{2} - 2\xi_{i}\bar{\xi}_{n}\right) ={}
\\
{}= \fr{1}{n}\left[ \sum\limits_{i=0}^{n-1}\left( \xi_{i}^{2} -
 2\xi_{i}\bar{\xi}_{n} \right) + n\bar{\xi}_{n}^{2} \right] = 
 \fr{1}{n} \sum\limits_{i=0}^{n-1}\xi_{i}^{2} - \bar{\xi}_{n}^{2}\,.
\end{multline*}


Рассмотрим компоненты, представляющие собой суперпозиции истинных 
компонент сигнала~$W_n^{A_l,s}$ для $s, l \hm\in \{ 1, 2, \ldots, m\}$, $s\hm\neq l $, 
тогда
\begin{multline*}
W_n^{A_l,s} =  \fr{2}{n}\sum\limits_{i=0}^{n-1}\left( A_{l,i} - 
\bar{A}_{l,n}\right) \left( A_{s,i} - \bar{A}_{s,n}\right)  ={}
\\
{}= \fr{2}{n}\sum\limits_{i=0}^{n-1}\!A_{s,i}\left( A_{l,i} - \bar{A}_{l,n}\right) - 
 \fr{2\bar{A}_{s,n}}{n}\hspace*{-0.6pt}\sum\limits_{i=0}^{n-1}\!\left( A_{l,i} - \bar{A}_{l,n}\right) = {}\\
 {}=
 \fr{2}{n}\sum\limits_{i=0}^{n-1}A_{s,i}\left( A_{l,i} - \bar{A}_{l,n}\right)\,.
\end{multline*}


Проведем аналогичные преобразования для компонент~$W_n^{A_l\xi}$, 
$l \hm\in \{ 1, 2, \ldots, m\}$, и~получим:

\noindent
\begin{multline*}
W_n^{A_l\xi} =  \fr{2}{n}\sum\limits_{i=0}^{n-1}\left( A_{l,i} - 
\bar{A}_{l,n}\right) \left( \xi_{i} - \bar{\xi}_{n}\right)  ={}
\\
{}= \fr{2}{n}\sum\limits_{i=0}^{n-1}\xi_{i}\left( A_{l,i} - \bar{A}_{l,n}\right) -  
\fr{2\bar{\xi}_{n}}{n}\sum\limits_{i=0}^{n-1}\left( A_{l,i} - \bar{A}_{l,n}\right) = {}\\
{}=
\fr{2}{n}\sum\limits_{i=0}^{n-1}\xi_{i}\left( A_{l,i} - \bar{A}_{l,n}\right).
\end{multline*}

Подставив полученные выражения в~уравнение~(\ref{4}), получим утверждение леммы.


\subsection{Свойства шумовой компоненты оконной дисперсии в~модели 
с~общей~шумовой~компонентой}
%\label{1.1.2}

Обозначим через $W_{n}^{\Xi}$ шумовую компоненту оконной дисперсии 
регистрируемого сигнала, которая представляет собой сумму оконной 
дис\-пер\-сии шума и~суперпозицию шума и~истинных компонент:
\begin{equation}
\label{5}
\begin{matrix}
W_{n}^{\Xi}= W_n^{\xi} + \sum\limits_{l=1}^{m}W_n^{A_l\xi}\,.
\end{matrix}
\end{equation}

\noindent
\textbf{Теорема.}
\textit{В каждой точке наблюдения~$\tau_{k}$ шумовая компонента оконной 
дисперсии~$W_n^{\Xi}$ представима в~\mbox{виде}}:
\begin{equation*}
%\label{6}
W_{n}^{\Xi} = W_n^{\xi}\,,\end{equation*}
 если 
 $
\left( A_{l,i} - \bar{A}_{l,n}\right) = 0 \ \forall\
   l  \hm\in \{ 1, 2, \ldots, m\},\ \forall \ i \hm\in \{ 0, 1, \ldots, n-1\},
$
где $W_n^{\xi}$ имеет распределение:
$$
W_n^{\xi} \sim \Gamma\left( \fr{n}{2\sigma^{2}}, \frac{n-1}{2}\right)\,,
$$
иначе
$$
W_{n}^{\Xi} = W_n^{\xi}  + \sum\limits_{l: \left( A_{l,i} - \bar{A}_{l,n}\right) 
\neq 0}^{m}W_n^{A_l\xi}\,,
$$
если $\exists \ 
l, i : \left( A_{l,i} - \bar{A}_{l,n}\right)\hm \neq 0,$
при этом $W_n^{A_l\xi}$ имеет распределение:
$$
W_n^{A_l\xi} \sim N \left( 0, \fr{4\sigma^{2}}{n}W_n^{A_l}\right)\,.
$$

\noindent
Д\,о\,к\,а\,з\,а\,т\,е\,л\,ь\,с\,т\,в\,о\,.\ \
По определению случайные величины~$\xi_{k} $ являются независимыми одинаково 
распределенными с~нулевым математическим ожиданием и~дисперсией~$\sigma^{2}$. 
Поэтому распределение компоненты  ${nW_n^{\xi}}/{\sigma^{2}}$ 
как оконной дисперсии случайной величины со стандартным нормальным распределением
 является хи-квад\-рат-рас\-пре\-де\-ле\-ни\-ем вида:
$$
\fr{nW_n^{\xi}}{\sigma^{2}} \sim \chi^{2}_{n-1} = 
\Gamma\left( \fr{1}{2}, \fr{n-1}{2}\right)\,. 
$$

Теперь по свойствам гам\-ма-рас\-пре\-де\-ле\-ния получаем:
$$
W_{n}^{\xi} \sim \chi^{2}_{n-1} = \Gamma\left( \fr{n}{2\sigma^{2}}, 
\fr{n-1}{2}\right)\,.
$$

Компоненты, характеризующие суперпозицию шума и~истинных компонент 
сигнала,~$W_{n}^{A_l\xi}$, $l \hm\in \{ 1, 2, \ldots, m\}$, представляют 
собой суммы независимых нормально распределенных случайных величин:
$$
W_n^{A_l\xi} = \fr{2}{n}\sum\limits_{i=0}^{n-1}\xi_{i}\left( A_{l,i} - 
\bar{A}_{l,n}\right)\,,
$$
при этом
$$
\fr{2}{n}\xi_{i}\left( A_{l,i} - \bar{A}_{l,n}\right)  
\sim N \left( 0, \left[\fr{2\sigma}{n}\left( A_{l,i} - \bar{A}_{l,n}\right) \right]^{2} \right)\,.
$$

Вследствие усиленной воспроизводимости нормального распределения сумма таких 
величин по~$i$ будет иметь нормальное распределение вида:
\begin{multline*}
W_n^{A_l\xi} \sim N \left( 0, \fr{4\sigma^{2}}{n^{2}}\sum\limits_{i=0}^{n-1}
\left( A_{l,i} - \bar{A}_{l,n}\right)^{2} \right) = {}\\
{}=
N \left( 0, \fr{4\sigma^{2}}{n}W_n^{A_l}\right)\,.
\end{multline*}

Итак, доказано, что в~тех случаях, когда истинные компоненты, формирующие 
сигнал, не изменяются, т.\,е.\ $ A_{l,i} \hm- \bar{A}_{l,n} \hm= 0$, 
оконная дис\-пер\-сия шума характеризуется гам\-ма-рас\-пре\-де\-ле\-ни\-ем с~параметрами 
формы и~масштаба~$({n-1})/{2}$  и~${n}/({2\sigma^{2}})$ 
соответственно. Если же для ка\-ких-ли\-бо из истинных компонент  
$ A_{l,i} \hm- \bar{A}_{l,n} \hm\neq 0$, тогда шумовая компонента 
оконной дисперсии~$W_{n}^{\Xi}$ пред\-став\-ля\-ет собой сумму зависимых случайных 
величин с~гам\-ма-рас\-пре\-де\-ле\-ни\-ем и~нормально распределенных.

\section{Модель с~несколькими различными шумовыми компонентами}

\subsection{Общее представление модели}

%\label{1.2.1}
Обратимся теперь к~несколько иному пред\-став\-ле\-нию модели, а~именно: 
представим для каждой точки~$\tau_{k}$ значение сигнала~$C$ в~виде 
суммы независимых в~физическом смысле истинных значений нескольких процессов, 
формирующих сигнал~$A_{l}$, $l \hm= 1, 2, \ldots, m$, и~соответствующего каждой 
такой истинной компоненте независимого в~статистическом смысле 
шума $\xi_{1}, \xi_{2}, \ldots, \xi_{m}$, при этом случайная величина~$\xi_{l, k} $ 
характеризуется нормальным распреде\-лением с~нулевым математическим 
ожиданием и~дисперсией~$\sigma^{2}_{l, k}$, $l \hm= 1, 2, \ldots , m$, 
причем для всех точек~$\tau_{k}$ для фиксированной истинной компоненты~$A_{l}$ 
изучаемого сигнала случайные величины~$\xi_{k,l}$~--- 
независимые одинаково распределенные случайные величины. Тогда для любого~$\tau_{k}$ 
сигнал~$C$  представим в~виде:
\begin{equation}
\label{7}
C = \sum\limits_{l=1}^{m}C_{l} = \sum\limits_{l=1}^{m}\left( A_{l}+\xi_{l}\right)\,.
\end{equation}

Обозначим
$$
\bar{C}_{n} = \sum\limits_{l=1}^{m} \bar{C}_{l,n}\,; 
\quad W_{n} = \sum\limits_{l=1}^{m} W_{l,n}\,,
$$
где $\bar{C}_{n}$~--- скользящее среднее регистрируемого сигнала~$C$, 
а~$ W_{n}$~--- оконная дисперсия сигнала.

Далее рассмотрим отдельно компоненты этой суммы~(\ref{7}) 
$C_{l}\hm =  A_{l}\hm+\xi_{l}$. Для каждой компоненты~$C_{l}$ 
в~соответствии с~леммой справедливо разложение~(\ref{4}), поэтому
$$
\bar{C}_{l,n}=\fr{1}{n} \sum\limits_{i=0}^{n-1}C_{l,i} = \fr{1}{n} 
\sum\limits_{i=0}^{n-1}\left( A_{l,i}+\xi_{l,i}\right)\,;
$$

\vspace*{-12pt}

\noindent
\begin{multline*}
W_{l,n}=\fr{1}{n} \sum\limits_{i=0}^{n-1}\left( C_{l,i} - \bar{C}_{l,n}\right)^{2} 
=\fr{1}{n} \sum\limits_{i=0}^{n-1} A_{l,i}^{2}-
\bar{A}_{l,n}^{2} +{}\\
{}+ 
\fr{1}{n} \sum\limits_{i=0}^{n-1} \xi_{l,i}^{2}-\bar{\xi}_{l,n}^{2} + 
\fr{2}{n} \sum\limits_{i=0}^{n-1}\xi_{l,i} \left( A_{l,i}-\bar{A}_{l,n}\right)\,.
\end{multline*}

Зафиксируем $l$ и~далее для наглядности изложения опустим 
этот индекс, т.\,е.\ будем рассматривать лишь одну истинную 
компоненту и~соответст\-ву\-ющую шумовую компоненту.

Рассмотрим шумовую компоненту оконной дисперсии и~представим ее в~специальном виде:
\begin{multline*}
W_{n}^{\Xi}= \fr{1}{n} \sum\limits_{i=0}^{n-1} \xi_{i}^{2}-\bar{\xi}_{n}^{2} + 
\fr{2}{n} \sum\limits_{i=0}^{n-1}\xi_{i} \left( A_{i}-\bar{A}_{n}\right) = {}\\
{}=\fr{1}{n} \sum\limits_{i=0}^{n-1}\left( \xi_{i}^{2} + \xi_{i} 
\left( 2A_{i}-2\bar{A}_{n}\right)\right)  - \bar{\xi}_{n}^{2}\,.
\end{multline*}

Случайная величина $\bar{\xi}_{n}^{2}$ сходится по вероят\-ности к~нулю, а~свойства 
слагаемых
$\xi_{i}^{2} \hm+ \xi_{i} \left( 2A_{i}\hm-2\bar{A}_{n}\right)$
будут описаны в~подразделе ниже.

Частный случай $l\hm=1$ соответствует сигналу с~единственной истинной 
компонентой и~соответствующим шумом, пример такого сигнала рас\-смот\-рен 
в~работах~\cite{Z7, Khazi}.

\subsection{Свойства случайной величины вида $\left( \xi^{2} + a\xi\right)$}
%    \label{1.2.2}

Рассмотрим свойства одного слагаемого шумовой компоненты 
$ \xi_{i}^{2} \hm+ \xi_{i} \left(2 A_{i}\hm-2\bar{A}_{n}\right)$.
Зафиксировав~$i$ и~введя обозначение $a\hm = 2\left( A\hm-\bar{A}\right)$, 
получим случайную величину вида $ \xi^{2} \hm+ a\xi$. Ее 
свойства отражены в~следующей лемме.

\smallskip
    
\noindent
\textbf{Лемма.}\ \textit{Пусть случайная величина~$\xi$ распределена по нормальному 
закону $N\left( 0, \sigma^{2}\right)$, тогда  $\xi^{2}\hm +a \xi$ 
имеет распределение, соответствующее характеристической функции вида}:
 $$
 \varphi_{\xi^{2} + a\xi}\left(t \right)  = 
 \fr{1}{\sqrt{1-2it\sigma^2}}\,
 e^{{a^{2}t^{2}}/\left({4\left(it - {1}/\left({2\sigma^{2}}\right)\right) }\right)}.
$$

\noindent
Д\,о\,к\,а\,з\,а\,т\,е\,л\,ь\,с\,т\,в\,о\,.\ \
 Вычислим характеристическую функцию для $\xi^{2} \hm+a \xi$.

Запишем определение:
\begin{multline*}
    \varphi_{\xi^{2} + a\xi} \left(t \right) = 
    Ee^{it\left( \xi^{2} + a\xi \right) } = 
      \int\limits_{-\infty}^{\infty}e^{it\left( x^{2} + ax \right) } dF\left( x\right) ={}\\
{}    = \int\limits_{-\infty}^{\infty}
e^{it\left( x^{2} + ax \right) } 
\fr{1}{\sqrt{2\pi\sigma^2}}\,
e^{-{x^{2}}/\left({2\sigma^{2}}\right)} dx ={}\\
{}=\fr{1}{\sqrt{2\pi\sigma^2}}
\int\limits_{-\infty}^{\infty}e^{it\left( x^{2} + 
ax \right) }e^{-{x^{2}}/\left({2\sigma^{2}}\right)} \,dx\,,\enskip
t\in R\,.
\end{multline*}
    
Сначала подробно рассмотрим частный случай $a\hm=1$, $\sigma^{2}\hm = 1$, 
который затем обобщим для произвольных параметров~$a$ и~$\sigma^{2}$.

Проведем элементарные преобразования, выделяя полный квадрат и~проводя 
замену переменных, получим:
\begin{multline*}
\varphi_{\xi^{2} + \xi} \left(t \right) = 
\fr{1}{\sqrt{2\pi}}\int\limits_{-\infty}^{\infty}e^{it\left( x^{2} + x \right) }
e^{-{x^{2}}/{2}}\, dx  ={}\\
{}= 
\fr{1}{\sqrt{2\pi}}\int\limits_{-\infty}^{\infty}
e^{-\left( {1}/{2}-it\right) \left( x + {it}/\left({2\left( it - {1}/{2}\right) }\right) 
\right)^{2} }\times{}\\
{}\times e^{-\left( it - {1}/{2}\right) \left( 
{it}/({2\left(it - {1}/{2}\right) })\right)^{2} } \, dx ={}\\
{}= 
\fr{1}{\sqrt{2\pi\left( {1}/{2}-it\right) }}\,
e^{-\left( it - {1}/{2}\right) \left( 
{it}/({2\left(it - {1}/{2}\right) })\right)^{2} }\times{}\\
{}\times 
\int\limits_{-\infty}^{\infty}
e^{-\left( \sqrt{{1}/{2}-it} \left( x + 
{it}/({2\left( it - {1}/{2}\right) }) \right)\right)^{2} }\times{}\\
{}\times d
\left( \sqrt{\fr{1}{2}-it}\,
 \left( x + \fr{it}{2\left( it - {1}/{2}\right) } \right)\right) ={}
\end{multline*}

\noindent
\begin{multline*}
{}= 
\fr{\sqrt{\pi}}{\sqrt{2\pi\left( {1}/{2}-it\right) }}\,
e^{-{\left( it\right)^{2}\left( it - {1}/{2}\right) }/
\left({4\left(it - {1}/{2}\right)^2 }\right)} = {}\\
{}=
\fr{1}{\sqrt{1-2it}}\,e^{{t^{2}}/({4\left(it - {1}/{2}\right) })}\,.
\end{multline*}
    
Теперь проведем аналогичные преобразования для любых~$a$ и~$\sigma^{2}$:
   \begin{multline*}
    \varphi_{\xi^{2} + a\xi} \left(t \right) = 
    \fr{1}{\sqrt{2\pi\sigma^2}}\int\limits_{-\infty}^{\infty}
    e^{it\left( x^{2} + a x \right) }
    e^{-{x^{2}}/\left({2\sigma^{2}}\right)}\, dx = {}\\
    {}=
    \fr{1}{\sigma\sqrt{{1}/{\sigma^{2}}-2it}}\,
    e^{{a^{2}t^{2}}/\left({4\left(it - {1}/\left({2\sigma^{2}}\right)\right) }\right)} = {}\\
    {}=
    \fr{1}{\sqrt{1-2it\sigma^2}}\,e^{{a^{2}t^{2}}/
    \left({4\left(it - {1}/\left({2\sigma^{2}}\right)\right) }\right)}.
\end{multline*}
    
    Лемма доказана.
    
    \smallskip
    
Согласно теореме единственности полученная характеристическая функция 
однозначным образом определяет функцию распределения случайной величины. 
Рассчитаем первые моменты рассматриваемой случайной величины. Результаты 
расчета приведены в~нижеследующей лемме.

\smallskip
    
\noindent
\textbf{Лемма.}
    Случайная величина $\xi^{2} \hm+ a\xi$ имеет сле\-ду\-ющие 
    математическое ожидание и~дисперсию:
$$
    {\sf E}\left( \xi^{2} + a\xi \right) = \sigma^2\,; \enskip 
    {\sf D}\left( \xi^{2} + a\xi \right) = \sigma^6 + 3\sigma^5+4 a^2 \sigma^3.
$$
    
\noindent
Д\,о\,к\,а\,з\,а\,т\,е\,л\,ь\,с\,т\,в\,о\,.\ \
Для вывода формул используем следующее свойство характеристических функций:
\begin{equation*}
{\sf E}\left(\left( \xi^{2} + a\xi\right)^n \right)  =
\fr{\varphi^{(n)}(0)}{i^n }\,.
\end{equation*}

Первая и~вторая производные по~$t$ для функции~$\varphi(t)$ имеют вид:
\begin{multline*} 
    \fr{d}{dt}\,\varphi\left( t\right) =
    \fr{d}{dt}\,\fr{e^{{a^2 t^2}/\left({4(-{1}/\left({2 \sigma^2}\right)+i t)}\right)}}
    {\sqrt{1-2 i t\sigma^2}} = {}\\
    {}=
    \fr{i e^{{a^2 t^2}/\left({4 \left(-{1}/\left({2 \sigma^2}\right)+i t\right)}\right)}}
    {\sigma \left({1}/{\sigma^2}-2 i t\right)^{3/2}}+{}
\\
        {}+
       e^{{a^2t^2}/\left({4 \left(-{1}/\left({2 \sigma^2}\right)+i t\right)}\right)} 
       \left(
    \fr{a^2 t}{2 \left(-{1}/\left({2 \sigma^2}\right)+i t\right)}-{}\right.\\
\left.    {}-
   \fr{i a^2 t^2}{4 \left(-{1}/\left(2 \sigma^2\right)+i t\right)^2}
  \right)
  \Bigg / 
    \left(\sigma\sqrt{\fr{1}{\sigma^2}-2 it}\right)\,;
\end{multline*}

\vspace*{-12pt}
    
\noindent
\begin{multline*}  
    \fr{d^2}{dt^2}\,\varphi\left( t\right) = 
    -\fr{3 e^{{a^2 t^2}/\left({4 \left(-{1}/\left({2 \sigma^2}\right)+i t\right)}\right)}}
    {\sigma \left({1}/{\sigma^2}-2 it\right)^{5/2}}+{}\\
    {}+
    e^{{a^2 t^2}/\left({4 \left(-{1}/\left({2\sigma^2}\right)+it\right)}\right)} 
    \left(
    \fr{a^2}{2 \left(-{1}/\left({2 \sigma^2}\right)+i t\right)}-{}\right.
    \end{multline*}

\noindent
\begin{multline*}
    {}-
    \fr{i a^2 t}{\left(-{1}/\left({2
            \sigma^2}\right)+i t\right){}^2}-{}\\
\left.            {}-
            \fr{a^2 t^2}{2 \left(-{1}/\left({2\sigma^2}\right)+it\right)^3}\right)
            \Bigg/
            \left(\sigma\sqrt{\fr{1}{\sigma^2}-2 it}\right)+{}\\
{}+ 2 i e^{{a^2 t^2}/\left({4 \left(-{1}/\left({2\sigma^2}\right)+it\right)}\right)} 
\left(
\fr{a^2 t}{2 \left(-{1}/\left({2 \sigma^2}\right)+i t\right)}-{}\right.\\
\left.{}-
\fr{i a^2 t^2}{4 \left(-{1}/\left({2 \sigma^2}\right)+it \right)^2}\right)
\!\Bigg/\!
\left(\sigma\left(\fr{1}{\sigma^2}-2 i t\right)^{3/2}\right)+{}\\
{}+
e^{{a^2 t^2}/\left({4 \left(-{1}/\left({2\sigma^2}\right)+i t\right)}\right)} 
\left(
\fr{a^2 t}{2 \left(-{1}/\left({2 \sigma^2}\right)+i t\right)}-{}\right.\\
\left.{}-
\fr{ia^2 t^2}{4 \left(-{1}/\left({2 \sigma^2}\right)+i t\right)^2}\right)^2\Bigg/
\left(\sigma\sqrt{\fr{1}{\sigma^2}-2 i t}\right)\,.
\end{multline*}
   
\noindent
Поэтому
$$  
iE\left( \xi^{2} + a\xi\right)  = \fr{i}{\left({1}/{\sigma^2}\right)^{3/2} \sigma }= 
i \sigma^2
$$  
и
\begin{multline*}
i^2 E\left( \xi^{2} + a\xi\right)^2 =  {}\\
{}=
-\fr{3}{\left({1}/{\sigma^2}\right)^{5/2} \sigma}-
\fr{a^2 \sigma}{\sqrt{{1}/{\sigma^2}}} = i^2\left(3\sigma^4 + a^2\sigma^2\right)\,.
\end{multline*} 

С помощью начальных моментов найдем дисперсию:
\begin{multline*}  
    D\left( \xi^{2} + a\xi \right)  =  E\left( \xi^{2} + a\xi\right)^2 - 
    \left( E\left(\xi^{2} + a\xi \right)\right)^2 ={}\\
    {}= 3\sigma^4 + a^2\sigma^2 - \left(\sigma^2\right)^2 = 2\sigma^4 + a^2\sigma^2 \,.
\end{multline*}

Таким образом, утверждения леммы  доказаны.

\subsection{Свойства шумовой компоненты оконной дисперсии в~модели 
с~несколькими шумовыми компонентами}
%    \label{1.2.3}

    Заменим обратно~$a$ на $ 2A\hm-2\bar{A}$ и~сформулируем 
    теорему о~свойствах шумовой компоненты оконной дисперсии.

\smallskip

\noindent
    \textbf{Теорема.}
\textit{Для регистрируемого сигнала $C\left(t \right)$ в~каж\-дой точке~$\tau_{k}$, 
$k\hm= \left\lbrace  1, 2, \ldots, r\right\rbrace$, шумовая компонента 
оконной дисперсии~$W_{n}^{\Xi}$ представляет собой}:
\begin{enumerate}[($i$)]
    \item \textit{случайную величину} 
    $$
    \sum\limits_{l=1}^{m} W_{l,n}^{\xi}\sim 
    \Gamma\left(\fr{n}{2\sigma_{l}^{2}}, \fr{n-1}{2}\right)\,,
    $$ 
    \textit{если} 
   \begin{multline*}
     A_{l,j} \hm- \bar{A}_{l,n}= \emptyset \ \forall\ 
     l  \hm\in \left\lbrace 1, 2, \ldots, m\right\rbrace ,\\
      \forall\ 
     j \hm\in \left\lbrace 0, 1, \ldots, n-1\right\rbrace\,;
     \end{multline*}
    
    \item \textit{сумму случайных величин} 
    $\sum\nolimits_{l=1}^{m}\sum\nolimits_{j=0}^{n-1}({1}/{n})\times$\linebreak
$\times     \left( \xi_{l,j}^{2}\hm+\xi_{l,j} \left( 2A_{l,j}\hm-2\bar{A}_{l,n}\right)\right)  \hm-
       \sum\nolimits_{l=1}^{m}\bar{\xi}_{l,n}^{2}$, 
       \textit{если} $\exists\ l \hm\in \left\lbrace 1, 2, \ldots, m\right\rbrace$
\textit{и}  $j \hm\in \left\lbrace 0, 1, \ldots, n-1\right\rbrace :
    A_{l,j}\hm - \bar{A}_{n} \hm\neq 0$,
    
    \textit{при этом}
    $$
    \bar{\xi}_{l,n}^{2}  \sim  \Gamma\left( \fr{n}{2\sigma_{l}^{2}}, 
    \fr{1}{2}\right)\,,
    $$ 
    
    \textit{a характеристическая функция случайной величины} 
    $\sum\nolimits_{l=1}^{m}(1/n) \!\sum\nolimits_{j=0}^{n-1}\!\left( \xi_{l,j}^{2} + 
    \xi_{l,j} \left( 2A_{l,j}-\right.\right.$\linebreak 
    $\left.\left.-\;2\bar{A}_{l,n}\right)\!\right)$ 
    \textit{имеет вид} : 
    \begin{multline*}
    \varphi\left( t\right) \hm= \prod\limits_{l=1}^{m} 
    \left( 1-\fr{2it\sigma_l^2}{n}\right)^{-{n}/{2}}\times{}\\
    {}\times  
    \prod\limits_{j=0}^{n-1} e^{{\left(A_{l,j}-\bar{A}_{l,n}\right)^{2}t^{2}
    \sigma^{2}_{l}}/\left({itn\sigma^{2}_{l} - {n^{2}}/2}\right)}\,.
    \end{multline*}
    
\end{enumerate}

\noindent
Д\,о\,к\,а\,з\,а\,т\,е\,л\,ь\,с\,т\,в\,о\,.\ \
        Сначала будем рассматривать случай для фиксированного~$l$ или, 
        иными словами, сигнал, сформированный единственной истинной 
        компонентой и~соответствующим шумом.
    
    В этом случае (см.\ подразд.~3.1) шумовая компонента для оконной дисперсии 
    сигнала имеет вид:
    $$
    W_{n}^{\Xi}= \fr{1}{n} \sum\limits_{j=0}^{n-1}\left( \xi_{j}^{2} + \xi_{j} \left( 2A_{j}-2\bar{A}_{n}\right)\right)  - \bar{\xi}_{n}^{2}\,.
$$

   \noindent
    Так как
\begin{multline*}
 \fr{1}{n} \sum\limits_{j=0}^{n-1}\left( \xi_{j}^{2} + \xi_{j} \left( 2A_{j}-
 2\bar{A}_{n}\right)\right) ={}\\
 {}= \sum\limits_{j=0}^{n-1}\frac{1}{n} 
 \left( \xi_{j}^{2} + \xi_{j} \left( 2A_{j}-2\bar{A}_{n}\right)\right)\,, 
 \end{multline*}
то, используя свойства характеристической функции, получим характеристическую 
функцию для случайной величины $({1}/{n}) \left( \xi_{j}^{2} \hm+ 
\xi_{j} \left( 2A_{j}\hm-2\bar{A}_{n}\right)\right)$ при каждом фиксированном~$j$:
\begin{multline*}
\varphi_{({1}/{n}) \left( \xi_{j}^{2} + \xi_{j} \left( 2A_{j}-2\bar{A}_{n}\right)
\right) }\left( t\right)  = {}\\
{}=
\varphi_{  \xi_{j}^{2} + \xi_{j} \left( 2A_{j}-2\bar{A}_{n}\right) } 
\left( \fr{t}{n} \right) = {}\\
{}=
\fr{1}{\sqrt{1-{2it\sigma^2}/{n}}}\,
e^{{\left( 2A_{j}-2\bar{A}_{n}\right)^{2}t^{2}}/
\left({4\left(i\,t\,n - {n^2}/\left({2\sigma^{2}}\right)\right) }\right)}\,.
\end{multline*}
    
   \noindent
    Тогда характеристическая функция для случайной величины  
    $({1}/{n}) \sum\nolimits_{j=0}^{n-1}\left( \xi_{j}^{2}\hm + 
    \xi_{j} \left( 2A_{j}\hm-2\bar{A}_{n}\right)\right)$ может быть записана в~виде:
\begin{multline*}
    \varphi_{({1}/{n}) \sum\nolimits_{j=0}^{n-1}\left( \xi_{j}^{2} + \xi_{j} 
    \left( 2A_{j}-2\bar{A}_{n}\right)\right) }\left( t\right)  = {}\\
    {}=
    \prod\limits_{j=0}^{n-1}\varphi_{({1}/{n})\left( 
    \xi_{j}^{2} + \xi_{j} \left( 2A_{j}-2\bar{A}_{n}\right)\right) } = 
    \fr{1}{\left( 1-{2it\sigma^2}/{n}\right)^{{n}/{2}} } \times{}\hspace*{-6pt}\\
    {}\times
\prod\limits_{j=0}^{n-1} 
e^{{\left( 2A_{j}-2\bar{A}_{n}\right)^{2}t^{2}}/
\left({4\left(i\,t\,n - {n^2}/\left({2\sigma^{2}}\right)\right) }\right)} = {}\\
{}= 
\left( 1-\fr{2it\sigma^2}{n}\right)^{-{n}/{2}} \times{}\\
{}\times \prod\limits_{j=0}^{n-1} 
e^{{\left( 2A_{j}-2\bar{A}_{n}\right)^{2}t^{2}}/
\left({4\left(\,i\,t\,n - {n^2}/\left({2\sigma^{2}}\right)\right) }\right)} = {}\\
{}=
\left( 1-\fr{2it\sigma^2}{n}\right)^{-{n}/{2}}  
\prod\limits_{j=0}^{n-1} e^{{\left( A_{j}-\bar{A}_{n}\right)^{2}t^{2} \sigma^{2} }/
\left({itn\sigma^2 - {n^2}/{2} }\right) }\,.\hspace*{-6pt}
\end{multline*}
    
    Рассмотрим теперь компоненту шумовой со\-став\-ля\-ющей сигнала $\bar{\xi}_{n}^{2}$, 
    которая представляет собой квадрат нормально распределенной случайной 
    величины~$\bar{\xi}_{n}$ с~математическим ожиданием~0 
    и~дис\-пер\-си\-ей~${\sigma^{2}}/{n}$. Поскольку
$$
    \fr{\sqrt{n}}{\sigma}\,\bar{\xi} \sim N \left( 0, 1\right)\,,
$$
то
$$
    \fr{n}{\sigma^{2}}\,\bar{\xi}_{n}^{2}  \sim \chi^{2}_{1} = 
    \Gamma\left( \fr{1}{2}, \fr{1}{2}\right)
    $$
    и
    $$ 
    \bar{\xi}_{n}^{2}  \sim  \Gamma\left( \fr{n}{2\sigma^{2}}, \fr{1}{2}\right).
$$
    
    Отметим, что математическое ожидание~$\bar{\xi}_{n}^{2}$ есть~${\sigma^{2}}/{n}$, 
    а~дисперсия равна ${2\sigma^{4}}/{n^{2}}$ и~эти характеристики зависят от числа 
    точек расчета оконной дисперсии~$n$, убывая  как~$n$ и~$n^{2}$ соответственно. 
    Отсюда следует, что можно выбрать такое~$n$, что среднее 
    и~дисперсия~$\bar{\xi}_{n}^{2}$ будут меньше заданной точности измерений.
    
    Принимая во внимание результаты подразд.~2.2, для каждого фиксированного~$l$ шумовую 
    компоненту сигнала можно представить в~виде:
   \begin{enumerate}[($i$)]
        \item случайной величины
        $$
        W_n^{\xi} = \fr{1}{n} 
        \sum\limits_{j=0}^{n-1}\xi_{j}^{2} - \bar{\xi}_{n}^{2} 
        \sim \Gamma\left( \fr{n}{2\sigma^{2}}, \fr{n-1}{2}\right)\,, 
        $$
        если $ A_{j} \hm- \bar{A}_{n} \hm= 0 \ \forall \ j \hm\in 
        \left\lbrace  0, 1, \ldots, n-1\right\rbrace$;
        \item суммы случайных величин $({1}/{n}) \sum\nolimits_{j=0}^{n-1}
        \left( \xi_{j}^{2} +\right.$\linebreak
        $\left.+\;\xi_{j} \left( 2A_{j}-2\bar{A}_{n}\right)\right)  \hm- 
        \bar{\xi}_{n}^{2}$, если $\exists \ j \hm\in \left\lbrace 0, 1, \ldots, 
        n-1\right\rbrace : \left( A_{j} \hm- \bar{A}_{n}\right)\hm \neq 0$, при этом
        $$
        \bar{\xi}_{n}^{2}  \sim  \Gamma\left(\fr{n}{2\sigma^{2}}, \,
\fr{1}{2}\right)\,;
        $$ 
        
        \vspace*{-12pt}
        
        \noindent
        \begin{multline*}
        \varphi_{({1}/{n}) \sum\nolimits_{j=0}^{n-1}\left( \xi_{j}^{2} + 
        \xi_{j} \left( 2A_{j}-2\bar{A}_{n}\right)\right) }\left( t\right)  
        ={}\\
        \hspace*{-30pt}{}= \left( 1-\fr{2it\sigma^2}{n}\right)^{-{n}/{2}}  
        \prod\limits_{j=0}^{n-1} 
        e^{{\left(A_{j}-\bar{A}_{n}\right)^{2}t^{2}\sigma^{2}}/\left({i  tn\sigma^{2} - 
        {n^{2}}/{2} }\right)}.\hspace*{-3.77995pt}
        \end{multline*}
            \end{enumerate}


Регистрируемый сигнал~$C$ представляет собой сумму независимых истинных компонент 
и~соответствующих независимых шумовых со\-став\-ля\-ющих, поэтому
\begin{multline*}
    W_{l,n}^{\Xi} = {}\\
    {}= \sum\limits_{l=1}^{m}\left[ \fr{1}{n} 
    \sum\limits_{j=0}^{n-1}\left( \xi_{l,j}^{2} + \xi_{l,j} \left( 2A_{l,j}-
    2\bar{A}_{l,n}\right)\right)  - \bar{\xi}_{l,n}^{2}\right] ={}
\\
{}    = \sum\limits_{l=1}^{m}\sum\limits_{j=0}^{n-1}\frac{1}{n} \left( \xi_{l,j}^{2} + \xi_{l,j} \left( 2A_{l,j}-2\bar{A}_{l,n}\right)\right)  -  \sum\limits_{l=1}^{m}\bar{\xi}_{l,n}^{2}.
\end{multline*}
    
    Рассмотрим случай, когда $\left( A_{l,j} \hm- \bar{A}_{l,n}\right)\hm = 0 
    \ \forall \   l  \hm\in \{ 1, 2, \ldots, m\}, \ \forall \ 
    j \hm\in \{ 0, 1, \ldots, n-1\} $, т.\,е.\ 
    истинные компоненты сигнала не меняются на фиксированном окне. Тогда
$$
    W_{l,n}^{\Xi} =  \sum\limits_{l=1}^{m}\sum\limits_{j=0}^{n-1}\fr{1}{n} 
    \xi_{l,j}^{2}  -  \sum\limits_{l=1}^{m}\bar{\xi}_{l,n}^{2}  = 
    \sum\limits_{l=1}^{m} W_{l,n}^{\xi}\,.
$$
    
    Таким образом, оконная дисперсия шумовой составляющей сигнала в~данном 
    случае представляет собой сумму гам\-ма-рас\-пре\-де\-лен\-ных случайных величин
$$
W_{l,n}^{\xi}  \sim  \Gamma\left( \fr{n}{2\sigma_{l}^{2}}, \fr{n-1}{2}\right).
$$

    Рассмотрим случай, когда существуют $l \hm\in 
    \{ 1, 2, \ldots, m\}$  и~$j \hm\in \{ 0, 1, \ldots, n-1\}$, 
    при которых $A_{l,j} \hm- \bar{A}_{l,n} \hm\neq 0$. Тогда
 оконная дисперсия шума представляет разность зависимых сумм случайных величин. 
 При этом
функция распределения случайной величины $\sum\nolimits_{l=1}^{m}\sum\nolimits_{j=0}^{n-1}
({1}/{n}) \left( \xi_{l,j}^{2} \hm+ \xi_{l,j} \left( 2A_{l,j}\hm-
2\bar{A}_{l,n}\right)\right)$ соответствует характеристической функции
\begin{multline*}
    \varphi(t)  = \prod\limits_{l=1}^{m}\varphi_{({1}/{n})
     \sum\nolimits_{j=0}^{n-1}\left( \xi_{l,j}^{2} + \xi_{l,j} \left( 
     2A_{l,j}-2\bar{A}_{l,n}\right)\right)}(t) ={}
\\
  {}  =   \prod\limits_{l=1}^{m} \left( 1-\fr{2it\sigma_l^2}{n}\right)^{-{n}/{2}}\times{}\\
  {}\times  
  \prod\limits_{j=0}^{n-1} 
  e^{{\left(A_{l,j}-\bar{A}_{l,n}\right)^{2}t^{2}\sigma^{2}_{l}}/
  \left({i  t n\sigma^{2}_{l} - {n^{2}}/{2} }\right)}.
\end{multline*}

Очевидно, что случайная величина $\sum\nolimits_{l=1}^{m}\bar{\xi}_{l,n}^{2} $ 
представляет собой сумму независимых гам\-ма-рас\-пре\-де\-лен\-ных величин 
с~параметрами формы~${1}/{2}$ и,~вообще говоря, различными параметрами 
масштаба~${n}/({2\sigma_{l}^{2})}$.

   
Теорема доказана.
    
\smallskip

Отметим также, что так как случайная величи-\linebreak на $\bar{\xi}_{l,n}^{2}$ 
неотрицательна, то
    на практике величина\linebreak $ \sum\nolimits_{l=1}^{m}\sum\nolimits_{i=0}^{n-1}({1}/{n})
     \left( \xi_{l,i}^{2} \hm+ \xi_{l,i} \left( 2A_{l,i}-2\bar{A}_{l,n}\right)\right)$ 
     может служить верхней оценкой шумовой компоненты оконной дисперсии~$W_{l,n}^{\Xi} $ 
     регистрируемого сигнала~$C$.

\section{Заключение}

В рамках работы предложены модели для представления сигналов в~виде суммы нескольких 
подлежащих процессов, а~также исследованы некоторые вероятностные характеристики 
оконной диспер\-сии сигналов как случайных процессов в~представленных моделях. 
Результаты работы согласуются с~установленными эмпирически свойствами шумовой 
компоненты оконной дисперсии миограммы~\cite{All}.  В~работе продемонстрировано, 
что на миограмме в~период покоя, т.\,е.\ в~отсутствие полезных компонент сигнала, 
оконная дисперсия характеризуется гам\-ма-рас\-пре\-де\-ле\-нием.

Полученные результаты планируется использовать в~практических задачах 
сегментирования сигналов и~выделения интервалов с~преобладанием тех или 
иных подлежащих процессов. Кроме того, вероятностные характеристики шумовой 
компоненты могут использоваться для прогнозирования поведения сигнала. 
В~частности, предполагается\linebreak приме\-нить эти теоретические результаты для анализа 
фармакокинетических данных.


   {\small\frenchspacing
 {%\baselineskip=10.8pt
 \addcontentsline{toc}{section}{References}
 \begin{thebibliography}{9}
    \bibitem{Kos}
    \Au{Kosar K., Lhotsk$\acute{\mbox{a}}$ L., Krajca~V.} 
    Classification of long-term EEG recordings~//  Biological
    and medical data analysis.~---
    Lecture notes in computer science ser.~--- Springer, 2004. Vol.~3337. P.~322--332.
    doi: 10.1007/978-3-540-30547-7\_33.
    
    \bibitem{Aza}
    \Au{Azami H., Mohammadi~K., Hassanpour~H.}  
    A~hybrid evolutionary approach to segmentation of nonstationary signals~// 
    Digit. Signal Process., 2013. Vol.~23. No.\,4. P.~1103--1114.
    doi: 10.1016/j.dsp.2013.02.019.
    
    \bibitem{Kal}
    \Au{Kalantarian H., Sarrafzadeh~M.} 
    Probabilistic time-series segmentation~// Pervasive Mob. Comput., 
 2017. doi: 10.1016/j.pmcj.2017.03.005.
    
    \bibitem{Z7}
    \Au{Захарова Т.\,В., Никифоров~С.\,Ю., Гончаренко~М.\,Б., Драницына~М.\,А., 
    Климов~Г.\,А., Хазиахметов~М.\,Ш., Чаянов~Н.\,В.} 
    Методы обработки сигналов для локализации невосполнимых областей головного мозга~// 
    Системы и~средства информатики, 2012. T.~22. №\,2. C.~157--175.
    
    \bibitem{Khazi}
    \Au{Хазиахметов М.\,Ш.} Свойства оконной дисперсии миограммы как 
    случайного процесса~// Системы и~средства информатики, 2014. T.~24. №\,3. C.~110--120.
    
    \bibitem{All}
    \Au{Allakhverdieva V.\,M., Chshenyavskaya~E.\,V.,  Dranitsyna~M.\,A., 
    Karpov~P.\,I., Zakharova~T.\,V.} An 
    approach to the inverse problem of brain functional mapping under the assumption of gamma distributed myogram noise within rest intervals using the independent component analysis~// 
    J.~Math. Sci., 2016. Vol.~214. No.\,1. P.~3--11. doi: 10.1007/s10958-016-2753-x.
   \end{thebibliography}

 }
 }

\end{multicols}

%\vspace*{-3pt}

\hfill{\small\textit{Поступила в~редакцию 19.04.17}}

\vspace*{10pt}

%\newpage

%\vspace*{-24pt}

\hrule

\vspace*{2pt}

\hrule

%\vspace*{8pt}


\def\tit{SEGMENTATION OF NONSTATIONARY SIGNALS USING STOCHASTIC CHARACTERISTICS 
OF~THE~WINDOW VARIANCE}

\def\titkol{Segmentation of nonstationary signals using stochastic characteristics 
of~the~window variance}

\def\aut{M.\,A.~Dranitsyna$^1$ and~T.\,V.~Zakharova$^{1,2}$}

\def\autkol{M.\,A.~Dranitsyna and~T.\,V.~Zakharova}

\titel{\tit}{\aut}{\autkol}{\titkol}

\vspace*{-9pt}


\noindent
$^1$Department of Mathematical Statistics, Faculty of Computational 
Mathematics and Cybernetics, M.\,V.~Lo-\linebreak
$\hphantom{^1}$monosov Moscow State University, 
1-52~Leninskiye Gory, GSP-1, Moscow 119991, Russian Federation

\noindent
$^2$Institute of Informatics Problems, Federal Research Center ``Computer 
Science and Control'' of the Russian\linebreak
$\hphantom{^1}$Academy of Sciences, 44-2~Vavilov Str., 
Moscow 119333, Russian Federation


\def\leftfootline{\small{\textbf{\thepage}
\hfill INFORMATIKA I EE PRIMENENIYA~--- INFORMATICS AND
APPLICATIONS\ \ \ 2017\ \ \ volume~11\ \ \ issue\ 3}
}%
 \def\rightfootline{\small{INFORMATIKA I EE PRIMENENIYA~---
INFORMATICS AND APPLICATIONS\ \ \ 2017\ \ \ volume~11\ \ \ issue\ 3
\hfill \textbf{\thepage}}}

\vspace*{6pt}
    

\Abste{Signal or response partitioning (i.\,e., signal segmentation) 
is of great interest, e.\,g., for biomedical research. 
Signal segmentation, being an essential part of signal processing,
 may serve as a~tool for advanced signal interpretation and data classification. 
 Segmentation of nonstationary signals with a~small signal-to-noise ratio 
 is a~particulary complicated task. 
 The paper is mainly devoted to exploration of the window variance noise component 
 as a~random variable for the proposed signal models. 
 Some stochastic characteristics of the window variance noise\linebreak\vspace*{-12pt}}
 
 \Abstend{components 
 are investigated in accordance with the models. 
 Theoretical findings are consistent with the previously obtained empirical characteristics 
 of the window variance noise component and are supposed to be of potential use for signal segmentation 
 and prediction.}

\KWE{window variance; signal model}

\DOI{10.14357/19922264170302} 

%\vspace*{-18pt}

%\Ack
%\noindent


%\vspace*{3pt}

  \begin{multicols}{2}

\renewcommand{\bibname}{\protect\rmfamily References}
%\renewcommand{\bibname}{\large\protect\rm References}

{\small\frenchspacing
 {%\baselineskip=10.8pt
 \addcontentsline{toc}{section}{References}
 \begin{thebibliography}{9}


\bibitem{1-za}
\Aue{Kosar, K., L.~Lhotska, and V.~Krajca.} 2004. 
Classification of long-term EEG recordings. 
\textit{ Biological
    and medical data analysis}.
{Lecture notes in computer science ser.} 3337:322-332. 
doi: 10.1007/978-3-540-30547-7\_33.

\bibitem{2-za}
\Aue{Azami, H., K.~Mohammadi, and H.~Hassanpour.} 2013. 
A~hybrid evolutionary approach to segmentation of nonstationary signals. 
\textit{Digit. Signal Process.} 23(4):1103-1114. doi: 10.1016/j.dsp.2013.02.019.

\bibitem{3-zh}
\Aue{Kalantarian, H., and M.~Sarrafzadeh.} 2017. 
Probabilistic time-series segmentation. 
\textit{Pervasive Mob. Comput.} doi: 10.1016/j.pmcj.2017.03.005.

\bibitem{4-zh}
\Aue{Zakharova, T.\,V., S.\,Yu.~Nikiforov, M.\,B.~Goncharenko, M.\,A.~Dranitsyna, 
G.\,A.~Klimov, M.\,Sh.~Khaziakhmetov, and N.\,V.~Chayanov.} 
2012. Metody obrabotki signalov dlya lokalizatsii nevospolnimykh oblastey
 golovnogo mozga [Signal processing methods for localization of nonrenewable 
 brain regions]. \textit{Sistemy i~Sredstva Informatiki~--- 
 Systems and Means of Informatics} 22(2):157--175.

\bibitem{5-zh}
\Aue{Khaziakhmetov, M.\,Sh.} 2014. Svoystva okonnoy dispersii miogrammy kak
 sluchaynogo protsessa [Properties of window dispersion of myogram as 
 a~stochastic process]. \textit{Sistemy i~Sredstva Informatiki~---
 Systems and Means of Informatics} 24(3):110--120.

\bibitem{6-zh}
\Aue{Allakhverdieva, V.\,M., E.\,V.~Chshenyavskaya, M.\,A.~Dranitsyna, 
P.\,I.~Karpov, and T.\,V.~Zakharova.} 2016. 
An approach to the inverse problem of brain functional mapping under the assumption of gamma distributed myogram noise within rest intervals using the independent component analysis. 
\textit{J.~Math Sci.} 214(1):3--11. doi: 10.1007/s10958-016-2753-x.
\end{thebibliography}

 }
 }

\end{multicols}

\vspace*{-3pt}

\hfill{\small\textit{Received April 19, 2017}}


\Contr

\noindent
\textbf{Dranitsyna Margarita A.} (b.\ 1983)~---
PhD student, Department of Mathematical Statistics, Faculty of Computational 
Mathematics and Cybernetics, M.\,V.~Lomonosov Moscow State University, 
1-52~Leninskiye Gory, GSP-1, Moscow 119991, Russian Federation; 
\mbox{margarita13april@mail.ru}

\vspace*{3pt}

\noindent
\textbf{Zakharova Tatiana V.} (b.\ 1962)~---
Candidate of Science (PhD) in physics and mathematics, associate professor, 
Department of Mathematical Statistics, Faculty of Computational Mathematics 
and Cybernetics, M.\,V.~Lomonosov Moscow State University, 1-52~Leninskiye 
Gory, GSP-1, Moscow 119991, Russian Federation; senior scientist, 
Institute of Informatics Problems, Federal Research Center ``Computer 
Science and Control'' of the Russian Academy of Sciences, 44-2~Vavilov Str., 
Moscow 119333, Russian Federation; \mbox{lsa@cs.msu.ru}

\label{end\stat}


\renewcommand{\bibname}{\protect\rm Литература}  %2
\def\stat{krivenko}

\def\tit{МНОГОМЕРНЫЙ РЕФЕРЕНСНЫЙ РЕГИОН\\ ВЫСОКОЙ ПЛОТНОСТИ}

\def\titkol{Многомерный референсный регион высокой плотности}

\def\aut{М.\,П.~Кривенко$^1$}

\def\autkol{М.\,П.~Кривенко}

\titel{\tit}{\aut}{\autkol}{\titkol}

\index{Кривенко М.\,П.}
\index{Krivenko M.\,P.}


%{\renewcommand{\thefootnote}{\fnsymbol{footnote}} \footnotetext[1]
%{Работа выполнена при финансовой поддержке РФФИ (проекты 16-07-00677 
%и~15-37-20611-мол\_а\_вед).}}


\renewcommand{\thefootnote}{\arabic{footnote}}
\footnotetext[1]{Институт проблем информатики Федерального исследовательского центра <<Информатика и~управление>> Российской академии наук,
\mbox{mkrivenko@ipiran.ru}}

\vspace*{4pt}



\Abst{Рассматриваются принципы построения многомерных референсных регионов
(MRR~--- multivariate reference region). 
Предложен оригинальный метод построения региона на основе областей с~высокой 
плотностью точек и~аппроксимации распределения данных с~помощью смеси нормальных 
распределений. Для оценки порога для плотности распределения используется  
бут\-стреп-ме\-тод. В~качестве эксперимента рассмотрена задача построения 
и~использования эталонной области для прогнозирования типа мочевого камня. Обработка 
реальных данных продемонстрировала преимущества предлагаемых решений.}

\KW{многомерный референсный регион; область высокой плотности; бут\-стреп-ме\-тод; 
смесь многомерных нормальных распределений}

\vspace*{6pt}

\DOI{10.14357/19922264170207} 


\vskip 10pt plus 9pt minus 6pt

\thispagestyle{headings}

\begin{multicols}{2}

\label{st\stat}

\section{Введение}

     Многомерный референсный регион 
был предложен в~литературе по клинической химии в~начале 1970-х~гг.\ как 
альтернатива одномерным референсным интервалам~[1]. Там излагались 
преимущества предлагаемых множественных тестов, хоть и~имеющих 
упрощенный вид, но снижающих (по отношению к~одномерным вариантам) 
число ложных положительных результатов. Появление MRR оказалось 
особенно привлекательным для интерпретации результатов наборов 
медицинских тестов. Тем не менее возникали трудности в~построении 
и~использовании процедур многомерного анализа (см., например,~[2]), 
связанные, в~частности, с~быстрым увеличением числа параметров, которые 
должны быть оценены. Немногие лаборатории использовали MRR в~своей 
практике, причем в~экспериментальном режиме, и,~как следствие, на 
сегодняшний день имеется относительно малое количество соответствующих 
публикаций. 

\vspace*{-6pt}

\section{Многомерный референсный регион на основе расстояния Махалонобиса}

\vspace*{-2pt}

     Одномерный референсный интервал, полученный статистическим путем, 
использует центральную часть значений анализируемого показателя, обычно 
соответствующую~95\% некоторой популяции~--- совокупности особей 
определенного вида (например, здоровой части населения определенного пола 
из некоторого диапазона возрастов). Одномерные референсные интервалы 
применялись в~течение многих лет в~качестве стандартного приема 
интерпретации лабораторных данных. Они легко формируются, хранятся, 
извлекаются и~передаются в~лабораторных информационных системах, просты 
в~понимании, хорошо воспринимаются медицинским сообществом в~ходе 
длительного использования. Тем не менее одномерные референсные интервалы 
при классификации данных могут дать большое число ложно аномальных 
результатов. Этот далеко не единственный недостаток однофакторного 
референсного интервала может быть полностью или частично устранен 
с~помощью MRR.
     
     Простейшим и~весьма распространенным способом построения MRR 
является использование прямого произведения отдельных референсных 
интервалов в~предположении, что они статистически независимы. Пусть 
$(1\hm-\alpha)$~--- вероятность попадания в~MRR, а~$p_0$~--- вероятность 
попадания в~референсный интервал для любого из~$d$~признаков, тогда 
$p_0\hm= \sqrt[d]{1-\alpha}$. С~ростом размерности~$d$ значения~$p_0$ 
быстро приближаются к~1, что фактически лишает смысла применение MRR.
     
     Как и~в одномерном случае, отправной точкой для построения MRR 
может стать нормальное распределение. Идеи центрального расположения 
референсного региона и~заданной вероятности попадания в~него приводят для 
$d$-мер\-но\-го нормального распределения, имеющего плотность 
распределения
     \begin{multline*}
     \varphi(y,\mu,\Sigma) ={}\\
     {}=(2\pi)^{-d/2}\vert\Sigma\vert^{-1/2}\exp \left( -\fr{\left(y-
\mu\right)^{\mathrm{T}} \Sigma^{-1}(y-\mu)}{2}\right),
   \end{multline*}
где величина $(y-\mu)^{\mathrm{T}} \Sigma^{-1} (y-\mu)$ есть квадрат так 
называемого расстояния Махаланобиса между~$y$ и~$\mu$, к~использованию 
многомерного эллипсоида
\begin{multline*}
(2\pi)^{-d/2}\vert\Sigma\vert^{-1/2}\exp \left( -\fr{\left(y-\mu\right)^{\mathrm{T}}
\Sigma^{-1} 
(y-\mu)}{2}\right) ={}\\
{}=const
\end{multline*}
или, что то же самое, 
$$ 
(y-\mu)^{\mathrm{T}} \Sigma^{-1}(y-\mu)=const\,.
$$
Его называют эллипсоидом равной плотности распределения (или просто 
эллипсоидом равной вероятности). 
     
     Если задаться вероятностью $(1\hm-\alpha)$ попадания в~эллипсоид 
равной вероятности вида $(y\hm-\mu)^{\mathrm{T}}\Sigma^{-1} (y\hm-\mu)\hm= 
\rho$, то параметр~$\rho$ удовлетворяет уравнению $\mathrm{Pr}\left\{ 
\chi_d^2\leq \rho\right\} \hm=1\hm-\alpha$.
     
     Использование эллипсоида в~качестве MRR будет оправдано только 
тогда, когда исходное распределение данных есть многомерное нормаль-\linebreak ное. 
Поэтому становятся актуальными критерии\linebreak подгонки, а~также использование 
процедур норма\-ли\-зации распределения данных в~многомерном\linebreak случае.
 Если 
с~помощью тестов выявляется, что распределение не является нормальным, то 
Международная федерация клинической химии и~лабораторной медицины 
рекомендует, согласно~[3], использовать двухступенчатую процедуру 
нормализации. Следует обратить внимание, что многошаговость здесь 
относится не к~многомерности, а касается лишь покоординатного 
преобразования распределения данных к~нормальному.
     
     Первые же попытки применения MRR на основе расстояния 
Махалонобиса (фактически это означает принятие модели нормального 
распределения референсных значений) выявили ряд недостатков (более 
подробно смотри в~\cite[разд.~6.2]{4-kri}):
     \begin{itemize}
\item проявление <<проклятий>> размерности при механическом 
увеличении~$d$, в~особенности если игнорируется этап анализа состава 
признаков~[1, 5, 6];
\item из-за небольших объемов обучающей выборки невысокая устойчивость 
при применении, в~частности чувствительность к~увеличению неточностей 
измерений после того, как регион был установлен~\cite{5-kri, 7-kri}. 
\item предположение о нормальном распределении и~попытки <<подправить>> 
действительность с~помощью преобразований реальных данных для их 
нормализации при увеличении размерности данных становятся все более 
шаткими~\cite{5-kri};
\item представление и~интерпретация выводов на основе MRR трудно 
понимаемы не только для специалистов в~предметной области~[8].
\end{itemize}

\vspace*{-9pt}

\section{Многомерный референсный регион высокой плотности}

\vspace*{-2pt}

     Заметим, что в~случае нормального распределения референсных значений 
для точек внут\-ри построенного эллипсоида значения плотности\linebreak распределения 
больше, чем на границе, а~вне~--- меньше. Это замечание позволяет 
предложить другой подход к~построению MRR.
     
     \smallskip
     
     \noindent
     \textbf{Определение.}\ Eсли плотность распределения референсных 
значений есть $f(y)$, то MRR есть область $A_t\hm= \left\{ y\in 
\mathcal{R}^d\vert f(y)\hm\geq t\right\}$ для некоторого порогового 
значения~$t$. 
     
     \smallskip
     
     Для нормального распределения это уже упомянутый эллипсоид равной 
вероятности. Если задается вероятность $(1\hm-\alpha)$ попадания в~$A_t$, то 
пороговое значение~$t$ есть решение уравнения $\int\nolimits_{A_t} 
f(u)\,du\hm=1\hm-\alpha$, получить которое аналитически в~случае 
произвольной плотности распределения вряд ли возможно. Здесь присутствуют 
две проблемы: вычисление многомерного интеграла и~зависимость области 
интегрирования от неизвестного значения. Для решения их предлагается 
привлечь метод моделирования.
     
     Сгенерируем выборку из $f(y)$, которую обозначим как $Y^f\hm= \left\{ 
y_1^f, \ldots, y_m^f\right\}$. Для оценки $\int\nolimits_{A_t} f(u)\,du$ 
используем отношение:

\noindent
\begin{multline*}
     \fr{\left\vert \left\{ y_i^f\vert y_i^f\in A_t\right\}\right\vert }{m} =
      \fr{\left\vert\left\{ y_i^f\vert 
f\left(y_i^f\right) \geq t\right\}\right\vert }{m} ={}\\
{}= 1-\fr{\left\vert \left\{ y_i^f\vert f(y_i^f)<t\right\}\right\vert }{m}=1-
F_m(t)\,,
     \end{multline*}
где $F_m(t)$~--- эмпирическая функция распределения случайной 
величины~$f(y)$, т.\,е.\ случайной величины, являющейся результатом 
преобразования с~помощью функции~$f(\cdot)$ случайной величины, име\-ющей 
плотность распределения~$f(u)$.

     Таким образом, искомая оценка~$t^*$ должна удовле\-тво\-рять уравнению 
$F_m(t^*)\hm=\alpha$ и~может быть получена как непараметрическая оценка 
квантиля\linebreak\vspace*{-12pt}

\pagebreak

\noindent
 порядка~$\alpha$ из распределения $F_m(\cdot)$. Если обозначить 
$f_i\hm= f(y_i^f)$, то~$t^*$ есть~$f_{(r)}$, где
     $$
     r= \begin{cases}
     m\alpha, &\ m\alpha~\mbox{---~целое}\,;\\
     \lfloor m\alpha+1\rfloor\,, & m\alpha~\mbox{--- не целое}\,.
     \end{cases}
     $$
     Заметим, что для такой оценки можно указать доверительный интервал.
     
     Для построения MRR необходимо знать распределение данных. При 
реализации принципа точек высокой плотности в~первую очередь следует 
обратиться к~параметрическим моделям, в~част\-ности к~смеси нормальных 
распределений, име\-ющей плотность распределения
     $$
     f(u) =\sum\limits_{j=1}^k p_j \varphi\left (u,\mu_j, \Sigma_j\right)\,.
     $$
Если $\hat{f}(u)$~--- оценка смеси, то~$t^*$ строится сле\-ду\-ющим образом:
\begin{itemize}
\item генерируется выборка $\left\{ y_1^f,\ldots , y_m^f\right\}$ из $\hat{f}(u)$ и~
для каждого ее $i$-го элемента подсчитывается значение $\hat{f}\left( 
y_i^f\right)$;
\item в~качестве~$t^*$ берется непараметрическая оценка квантиля 
порядка~$\alpha$ (в случае необходимости дополнительно находится 
непараметрическая оценка доверительного интервала для~$t^*$, что 
может характеризовать правильность выбранного объема для 
генерируемой выборки).
\end{itemize}

     Пусть для $f(u)$ имеется~$A_t$, а также получена $\hat{f}(u)$ 
и~соответствующий MRR вида~$\hat{A}_t$. Качество аппроксимации~$A_t$ 
с~по\-мощью~$\hat{A}_t$ можно оценить с~по\-мощью вероятности совпадения 
этих областей, т.\,е. 
     $$
     P_c= \int\limits_{\{ u\in A_t\}\cup \{u\in \hat{A}_t\}} \hspace*{-6mm}
f(u)\,du+\int\limits_{\{u\not\in A_t\} \cup\{ u\not\in \hat{A}_t\}}\hspace*{-6mm} f(u)\,du\,.
     $$
     
     Для оценки  $P_c$ можно использовать величину
     \begin{multline*}
     \hat{P}_c= \fr{\left\vert \left\{ 
     y_i^f\vert y_i^f \in \left\{\left\{ y_i^f\in A_t\right\}\cup \left\{y_i^f\in 
\hat{A}_t\right\}\right\}\right\}\right\vert}{m}+{}\\
{}+\fr{\left\vert \left\{ y_i^f\vert y_i^f \in \left\{\left\{ y_i^f\not\in A_t\right\}\cup 
\left\{ y_i^f\not\in \hat{A}_t\right\}\right\}\right\}\right\vert}{m}\,.
     \end{multline*}
     
     Использование MRR высокой плотности для диагностирования сводится 
к~реализации так называемого слабого критерия значимости для наблюденного 
значения~$x$: нулевая гипотеза заключается в~том, что $x\hm\in A_t$, 
статистика критерия есть $\hat{f}(x)$ и~решение о~принадлежности 
критической об\-ласти~$A_t$ принимается при больших значениях~$\hat{f}(x)$.
     
     Для медицинской практики важна возможность использования 
референсного региона при интерпретации результатов обследования 
некоторого пациента с~вектором признаков~$x$. В~подобных случаях 
сложившейся практикой для слабых критериев значимости является 
использование критического уровня~$\alpha_{\mathrm{cr}}$ (более распространенным 
в~медицине является употребление термина $p$-зна\-че\-ние)  $\alpha_{\mathrm{cr}}\hm= 
\mathrm{Pr}\left\{ \hat{f}(y)\hm\leq \hat{f}(x)\right\}$, где $y$~--- случайная 
величина, имеющая плотность распределения~$\hat{f}(u)$, а $\hat{f}(x)$~--- 
значение плотности распределения~$\hat{f}(u)$ в~точке~$x$. Эта 
характеристика дает представление о~том, насколько сильно данное 
наблюденное значение~$x$ противоречит гипотезе (или подкрепляет ее) 
о~принадлежности данных MRR. При выбранном же заранее уровне 
значимости с~помощью~$\alpha_{\mathrm{cr}}$ сразу же можно принять конкретное 
решение. 

\vspace*{-9pt}

\section{Эксперименты}

\vspace*{-2pt}

     Для демонстрации возможностей MRR использовались данные по 
прогнозу химического состава мочевых камней по метаболическим 
показателям мочи и~сыворотки крови, а также антропологическим 
характеристикам пациентов~[9]. В качестве исходной классификации камней 
рассматривалась следующая: чисто оксалатные (далее обозначены как O), чисто 
уратные (U), чисто фосфатные (P), смесь только оксалатных и~уратных (OU), 
смесь только оксалатных и~фосфатных (OP), смесь только уратных 
и~фосфатных (UP), все остальные. Данная классификация была построена 
в~[10] на основе доминирующих частот встречаемости основных компонентов. 
В~качестве референсных значений рассматривались наборы метаболических 
и~антропологических показателей (их всего было~14), соответствующих 
определенному классу камней.

\begin{table*}\small
\begin{center}


\begin{tabular}{|c|c|c|c|c|c|c|}
\multicolumn{7}{c}{Качество классификации с~помощью MRR}\\
\multicolumn{7}{c}{\ }\\[-6pt]
\hline
\multicolumn{1}{|c|}{\raisebox{-6pt}[0pt][0pt]{\tabcolsep=0pt\begin{tabular}{c}Тип\\ камня\end{tabular}}}&
\multicolumn{1}{c|}{\raisebox{-6pt}[0pt][0pt]{$N$}}&$(1-\alpha)$, 
&\multicolumn{2}{c|}{MRR(5)}&\multicolumn{2}{c|}{MRR(1)}\\
\cline{4-7}
&&&&&&\\[-9pt]
&&\%&$(1-\hat{\alpha})$, \%&$\hat{\beta}$, \%&$(1-\hat{\alpha})$, \%&$\hat{\beta}$, \%\\
\hline
\multicolumn{1}{|c|}{\raisebox{-18pt}[0pt][0pt]{O}}&
\multicolumn{1}{c|}{\raisebox{-18pt}[0pt][0pt]{82}}
&95&100\hphantom{9}&71&90&24\\
&&85&96&78&89&36\\
&&75&91&85&77&44\\
&&65&76&88&74&50\\
\hline
\multicolumn{1}{|c|}{\raisebox{-18pt}[0pt][0pt]{U}}&
\multicolumn{1}{c|}{\raisebox{-18pt}[0pt][0pt]{76}}&95&100\hphantom{9}&75&91&24\\
&&85&99&85&80&35\\
&&75&82&89&74&48\\
&&65&71&91&68&56\\
\hline
\multicolumn{1}{|c|}{\raisebox{-18pt}[0pt][0pt]{P}}&
\multicolumn{1}{c|}{\raisebox{-18pt}[0pt][0pt]{83}}&95&100\hphantom{9}&66&87&25\\
&&85&94&78&86&33\\
&&75&86&82&82&41\\
&&65&77&87&75&47\\
\hline
\end{tabular}
\end{center}
\end{table*}
     
     
     Для каждого из основных классов O, U, P, OU, OP и~UP перед построением 
MRR проводилась селекция признаков и~принималось то значение размерности 
признакового пространства~$d$ и~соответствующий набор показателей, 
которые позволяли прогнозировать состав камней без потери качества 
(методика описана в~\cite{9-kri} и~привела к~значению $d\hm=9$). В~качестве 
модели данных в~первую очередь рассматривалась смесь многомерных 
нормальных распределений из пяти элементов (подбор числа элементов смеси 
проводился с~по\-мощью AIC~--- Akaike information criterion), для соответствующего региона было принято 
обозначение MRR(5). Для сравнения также использовалась модель 
нормального распределения, которой соответствовал MRR(1). Полученные 
результаты приводятся час\-тич\-но в~таблице, где $N$~--- объем 
классифицируемых данных; $\hat{\alpha}$~--- оценка для~$\alpha$; 
$\hat{\beta}$~--- оценка мощности критерия при определении типа камня на 
основании MRR.


     Одной из базовых характеристик является вероятность попадания в~MRR 
$(1\hm-\alpha)$ и~ее оценка $(1\hm-\hat{\alpha})$. Сравнение соответствующих 
столбцов с~учетом значений~$N$ и~ориентировочных значений разброса 
(стандартные отклонения на основе биномиального распределения) не 
позволило выявить явных отклонений. Необходимо, правда, отметить, что во 
всех проанализированных случаях для MRR(5) оказалось, что $1\hm-
\hat{\alpha}\hm\geq 1\hm-\alpha$.
     
     Назначение MRR, заключающееся в~сжатом представлении референсных 
значений, в~многомерном случае практически не проявляется. Для задания 
MRR(5) необходимо указать следующие величины: $1\hm-\alpha$, $t$, 
$p_1,\ldots, p_{k-1}$, $\mu_1, \Sigma_1,\ldots , \mu_k,\Sigma_k$, общее 
количество которых равно  $[2\hm+ (k\hm-1)\hm+ k(d\hm+ d(d\hm+1)/2)]$ 
и,~в~частности, в~рассматриваемых экспериментах~--- 276. Для MRR(1) это 
значение меньше и~равно~56. При этом для обрабатываемой обучающей 
выборки в~зависимости от класса камней речь идет о~порядка~10$^2$ векторах 
данных (см.\ столбец со значениями~$N$), что приблизительно 
дает~10$^3$~скалярных величин.
     
     Другое назначение MRR состоит в~его использовании для 
диагностирования (классификации). В~этой связи в~первую очередь 
проводился сравнительный анализ MRR(1) (фактически это означает, что 
построение региона осуществляется на основе расстояния Махаланобиса) 
и~MRR(5) (модель смеси нормальных распределений и~предложенный 
в~данной работе метод оценивания па\-ра\-мет\-ров региона). Показателем 
информативности метода построения многомерного региона выступала 
мощность соответствующего слабого критерия значимости, а~именно: 
вероятность не попасть в~MRR при условии, что данные берутся из дополнения 
к~классу, для которого построена MRR. Сравнение соответствующих столбцов 
говорит о~явном преимуществе двух предложенных моментов: усложнение 
модели данных путем перехода от нормального распределения к~смеси 
нормальных распределений и~построение региона высокой плотности.
     
     Использование критического уровня можно продемонстрировать  
с~по\-мощью зависимости результатов сравнения двух классов от того, какой 
класс взять за основу. Введем для возможных значений $p$-ве\-ли\-чи\-ны три 
интервала: $(-\infty, 1\%)$, $[1\%, 5\%)$, $[5\%, 100\%)$ с~соответствующей 
интерпретацией положения наблюденного набора показателей для пациента 
относительно построенного MRR: уверенное непопадание, неуверенное 
попадание, уверенное попадание. Если MRR построить для оксалатных камней, 
то результаты для анализа пациентов с~фосфатными камнями дадут следующий 
вектор относительных частот попадания $p$-ве\-ли\-чин в~указанные 
интервалы: $(60\%, 18\%, 22\%)$. Если же MRR строить для фосфатных 
камней, то получим $(71\%, 5\%, 24\%)$. Таким образом, для классификации 
указанных камней при приблизительно одинаковых частотах попадания в~MRR 
(22\% или~24\%) уверенный отказ от референсного региона происходит чаще, 
если принять за базовый MRR регион для фосфатных камней. Построение 
шкалы, подобной рассмотренной, является прерогативой специалистов 
в~предметной области, в~данной работе она использовалась только для 
иллюстрации. 

\vspace*{-6pt}

\section{Заключение}

\vspace*{-2pt}

     На настоящий момент имеется относительно мало примеров применения 
MRR в~клинической практике. Тому есть несколько причин. Математическое 
обеспечение, необходимое для получения и~применения MRR, не отвечает 
возможностям большинства клинических лабораторий. Лаборатории слабо 
оснащены программными средствами\linebreak для реализации достаточно сложного 
математического аппарата многомерного анализа, а~еще важнее, что 
отсутствуют методики, инструкции по\linebreak использованию соответствующих 
средств. Лишь немногие клинические применения демонстрируют 
преимущества MRR, хотя свидетельств неудачных попыток больше.
     
     Несмотря на сложности внедрения мно\-го\-мерно\-го анализа референсных 
значений, можно сформулировать некоторые рекомендации по иссле\-до\-ва\-нию 
и~разработке MRR. Во-пер\-вых, эффективная размерность в~MRR должна 
быть как можно меньше, чтобы избежать затенения диагностически полезной 
информации тестами, со\-зда\-ющи\-ми шум. Низкая размерность также должна 
уменьшить неблагоприятные последствия увеличения неточности результатов 
в~связи с~ростом числа анализируемых показателей. Во-вто\-рых, показатели 
(тес\-ты), включенные в~MRR, должны быть физиологически релевантными 
исследуемому кругу расстройств, чтобы максимизировать информацию, 
полученную от MRR. В-треть\-их, чтобы учесть эффекты долгосрочной 
лабораторной из\-мен\-чи\-вости, данные, используемые для получения MRR, 
долж\-ны быть собраны и~проанализированы в~течение достаточно большого 
периода времени (от нескольких недель до нескольких месяцев).  
В-чет\-вер\-тых, представление результатов лабораторных исследований 
следует осуществлять в~графическом виде, чтобы помочь врачам лучше понять 
MRR. Различные подходы к~уменьшению размерности помогут выполнить это 
требование.
     
     Необходима дальнейшая разработка пояснительных инструментов, 
способных воспринять результаты анализа MRR. При этом дополнительно 
необходима информация о~том, какие именно тес\-ты являются важнейшими 
факторами нарушения нормы. Надо признать, что соответствующий 
математический аппарат еще предстоит разработать. Решение перечисленных 
вопросов играет важную роль для обеспечения постоянного клинического 
применения MRR. 

\vspace*{-6pt}
     
{\small\frenchspacing
 {%\baselineskip=10.8pt
 \addcontentsline{toc}{section}{References}
 \begin{thebibliography}{99}
 
 \vspace*{-2pt}
 
\bibitem{1-kri}
\Au{Boyd J.\,C.} Reference regions of two or more dimensions~// Clin. Chem. Lab. 
Med., 2004. Vol.~42. No.\,7. P.~739--746.
\bibitem{2-kri}
\Au{Winkel P.} Patterns and clusters~--- multivariate approach for interpreting 
clinical chemistry results~// Clin. Chem., 1973. Vol.~19. No.\,12. P.~1329--1333.
\bibitem{3-kri}
IFCC. Expert panel on theory of reference values. Approved recommendation on the 
theory of reference values. Part~5. Statistical treatment of collected reference values. 
Determination of reference limits~// J.~Clin. Chem. Clin. Biochem., 1987. Vol.~25. 
No.\,9. P.~645--656.
\bibitem{4-kri}
\Au{Кривенко М.\,П.} Статистические методы представления и~предварительной 
обработки референсных значений.~--- М.: ФИЦ ИУ РАН, 2016. 160~с.
\bibitem{5-kri}
\Au{Boyd J.\,C., Lacher~D.\,A.} The multivariate reference range: An alternative 
interpretation of multi-test profiles~// Clin. Chem., 1982. Vol.~28. No.\,2.  
P.~259--265.
\bibitem{6-kri}
\Au{Albert A., Harris~E.\,K.} Multivariate interpretation of clinical laboratory  
data.~--- New York, NY, USA: CRC Press, 1987. 328~p.
\bibitem{7-kri}
\Au{Linnet K.} Influence of sampling variation and analytical errors on the 
performance of the multivariate reference region~// Meth. Inf. Med., 1988. Vol.~27. 
No.\,1. P.~37--42.
\bibitem{8-kri}
\Au{Durbridge T.\,C.} Clinical acceptance of a multi-test reference region for 
biochemical-panel results~// Clin. Chem., 1983. Vol.~29. No.\,10. P.~1724--1726.
\bibitem{9-kri}
\Au{Кривенко М.\,П.} Критерии значимости отбора признаков классификации~// 
Информатика и~её применения, 2016. Т.~10. Вып.~3. С.~32--40.
\bibitem{10-kri}
\Au{Кривенко М.\,П., Голованов~С.\,А., Сивков~А.\,В.} Анализ однородности 
данных о химическом составе камней при уролитиазе~// Информатика и~её 
применения, 2013. Т.~7. Вып.~4. С.~94--104.
 \end{thebibliography}

 }
 }

\end{multicols}

\vspace*{-10pt}

\hfill{\small\textit{Поступила в~редакцию 5.12.16}}

\vspace*{4pt}

%\newpage

%\vspace*{-24pt}

\hrule

\vspace*{2pt}

\hrule

\vspace*{-3pt}


\def\tit{HIGH-DENSITY MULTIVARIATE REFERENCE REGION\\[-5pt]}

\def\titkol{High-density multivariate reference region}

\def\aut{M.\,P.~Krivenko\\[-7pt]}

\def\autkol{M.\,P.~Krivenko}

\titel{\tit}{\aut}{\autkol}{\titkol}

\vspace*{-16pt}


\noindent
Institute of Informatics Problems, Federal Research Center 
``Computer Science and Control'' of the Russian
Academy of Sciences,  44-2~Vavilov Str., Moscow 119333, Russian Federation



\def\leftfootline{\small{\textbf{\thepage}
\hfill INFORMATIKA I EE PRIMENENIYA~--- INFORMATICS AND
APPLICATIONS\ \ \ 2017\ \ \ volume~11\ \ \ issue\ 2}
}%
 \def\rightfootline{\small{INFORMATIKA I EE PRIMENENIYA~---
INFORMATICS AND APPLICATIONS\ \ \ 2017\ \ \ volume~11\ \ \ issue\ 2
\hfill \textbf{\thepage}}}

\vspace*{2pt}




\Abste{The paper considers the principles of construction of multivariate 
reference regions. An original method of construction of 
a~region on the basis of areas of high density of points and approximation 
of data distribution with a~mixture of normal distributions is suggested. 
To estimate the threshold for the probability density, the bootstrap method is used. 
As an experiment, the paper considers the problem of description and use of 
the reference region for predicting the type of urinary stones. 
Real data treatment demonstrated the benefits of the proposed solutions.}

\KWE{multivariate reference region; high-density region; bootstrap method; 
multivariate normal mixture}

\DOI{10.14357/19922264170207} 

%\vspace*{-18pt}

%\Ack
%\noindent



%\vspace*{3pt}

  \begin{multicols}{2}

\renewcommand{\bibname}{\protect\rmfamily References}
%\renewcommand{\bibname}{\large\protect\rm References}

{\small\frenchspacing
 {%\baselineskip=10.8pt
 \addcontentsline{toc}{section}{References}
 \begin{thebibliography}{99}
\bibitem{1-kri-1}
\Aue{Boyd, J.\,C.} 2004. Reference regions of two or more dimensions. \textit{Clin. 
Chem. Lab. Med.} 42(7):739--746.

\bibitem{2-kri-1}
\Aue{Winkel, P.} 1973. Patterns and clusters~--- multivariate approach for interpreting 
clinical chemistry results. \textit{Clin. Chem.} 19(12):1329--1333.
\bibitem{3-kri-1}
IFCC. 1987. Expert panel on theory of reference values. Approved recommendation on the 
theory of reference values. Part~5. Statistical treatment of collected reference values. 
Determination of reference limits. \textit{J.~Clin. Chem. Clin. Biochem.} 
25(9):645--656.
\bibitem{4-kri-1}
\Aue{Krivenko, M.\,P.} 2016. \textit{Statisticheskie metody predstavleniya 
i~predvaritel'noy obrabotki referensnykh znacheniy}
[Statistical methods for representation and preliminary processing of
reference values]. Moscow: FRC CSC RAS. 160~p.

\bibitem{5-kri-1}
\Aue{Boyd, J.\,C., and D.\,A.~Lacher.} 1982. The multivariate reference range: An 
alternative interpretation of multi-test profiles. \textit{Clin. Chem.}  
28(2):259--265.
\bibitem{6-kri-1}
\Aue{Albert, A., and E.\,K.~Harris.} 1987. \textit{Multivariate interpretation of 
clinical laboratory data}. New York, NY: CRC Press. 328~p.
\bibitem{7-kri-1}
\Aue{Linnet, K.} 1988. Influence of sampling variation and analytical errors on the 
performance of the multivariate reference region. \textit{Meth. Inf. Med.}  
27(1):37--42.
\bibitem{8-kri-1}
\Aue{Durbridge, T.\,C.} 1983. Clinical acceptance of a multi-test reference region 
for biochemical-panel results. \textit{Clin. Chem.} 29(10):1724--1726.
\bibitem{9-kri-1}
\Aue{Krivenko, M.\,P.} 2016. Kriterii znachimosti otbora priznakov klassifikatsii
[Significance tests of feature selection for~classification]. \textit{Informatika i~ee 
Primeneniya~--- Inform. Appl.} 10(3):32--40.
\bibitem{10-kri-1}
\Aue{Krivenko, M.\,P., S.\,A.~Golovanov, and A.\,V.~Sivkov}. 2013. Analiz 
odnorodnosti dannykh o~khimicheskom sostave kamney pri urolitiaze
[Analysis of data homogeneity of~the~chemical compositions 
of~stones in~case of~urolithiasis]. \textit{Informatika i~ee Primeneniya~---
Inform Appl.} 7(4):94--104.
\end{thebibliography}

 }
 }

\end{multicols}

\vspace*{-3pt}

\hfill{\small\textit{Received December 5, 2016}}


\Contrl

\noindent
\textbf{Krivenko Michail P.} (b.\ 1946)~--- Doctor of Science in technology, 
professor, leading scientist, Institute of Informatics Problems, Federal Research 
Center ``Computer Science and Control'' of the Russian Academy of Sciences, 
\mbox{44-2}~Vavilov Str., Moscow 119333, Russian Federation; \mbox{mkrivenko@ipiran.ru}

\label{end\stat}


\renewcommand{\bibname}{\protect\rm Литература}   %3
\def\stat{kirikov}

\def\tit{<<ВИРТУАЛЬНЫЙ КОНСИЛИУМ>>~--- ИНСТРУМЕНТАЛЬНАЯ 
СРЕДА ПОДДЕРЖКИ ПРИНЯТИЯ 
  СЛОЖНЫХ ДИАГНОСТИЧЕСКИХ РЕШЕНИЙ$^*$}

\def\titkol{<<Виртуальный консилиум>>~--- инструментальная 
среда поддержки принятия сложных диагностических решений}

\def\aut{И.\,А.~Кириков$^1$, А.\,В.~Колесников$^2$, С.\,В.~Листопад$^3$, 
С.\,Б.~Румовская$^4$}

\def\autkol{И.\,А.~Кириков, А.\,В.~Колесников, С.\,В.~Листопад, 
С.\,Б.~Румовская}

\titel{\tit}{\aut}{\autkol}{\titkol}

\index{Кириков И.\,А.}
\index{Колесников А.\,В.}
\index{Листопад С.\,В.} 
\index{Румовская С.\,Б.}
\index{Kirikov I.\,А.}
\index{Kolesnikov А.\,V.}
\index{Listopad S.\,V.}
\index{Rumovskaya S.\,B.}


{\renewcommand{\thefootnote}{\fnsymbol{footnote}} \footnotetext[1]
{Работа выполнена при частичной поддержке РФФИ (проект 16-07-00272 А).}}


\renewcommand{\thefootnote}{\arabic{footnote}}
\footnotetext[1]{Калининградский филиал Федерального исследовательского центра <<Информатика и~управление>> 
Российской академии наук, \mbox{baltbipiran@mail.ru}}
\footnotetext[2]{Балтийский Федеральный университет
имени  И.~Канта, Калининградский филиал Федерального 
исследовательского центра <<Информатика и~управление>> Российской академии наук, 
\mbox{avkolesnikov@yandex.ru}}
\footnotetext[3]{Калининградский филиал Федерального исследовательского центра <<Информатика и~управление>> 
Российской академии наук, \mbox{ser-list-post@yandex.ru}}
\footnotetext[4]{Калининградский филиал Федерального исследовательского центра <<Информатика 
и~управление>> Российской академии наук, \mbox{sophiyabr@gmail.com}}
 
 \vspace*{-3pt}
 
  \Abst{Рассматривается проблема принятия индивидуального решения при диагностике 
пациентов в~ам\-бу\-ла\-тор\-но-по\-ли\-кли\-ни\-че\-ских учреждениях на примере 
диагностики артериальной гипертензии (АГ). Предлагается повысить качество принятия 
индивидуального решения за счет консультаций с~системой поддержки принятия  
решения~--- <<Виртуальным консилиумом>>, моделирующим коллективный интеллект 
врачей стационара многопрофильного больничного учреждения. Приведены результаты 
проектирования и~экспериментального исследования лабораторного прототипа 
<<Виртуального консилиума>>.}

  \KW{система поддержки принятия решения; виртуальный консилиум; функциональная 
гибридная интеллектуальная система}

\DOI{10.14357/19922264160311} 


\vskip 10pt plus 9pt minus 6pt

\thispagestyle{headings}

\begin{multicols}{2}

\label{st\stat}
  

\section{Введение}

  Степень исследования, понимания и~качества диагностики проблемных сред и~их 
окружения отражена в~научной картине мира, онтологи\-зи\-ру\-ющей его представления 
и~делающей рассуждения и~целенаправленную деятельность <<зависимыми>> от них. 
В~искусственном интеллекте понятию <<картина мира>> соответствует понятие <<модель 
внешнего мира>> М.\,Г.~Га\-азе-Рап\-по\-пор\-та и~Д.\,А.~Поспелова~[1]. 
  
  Новая картина мира складывается из многочисленных теорий и~взглядов: <<ноосфера>>, 
<<разумный мир>> (В.\,И.~Вернадский, Н.\,Н.~Моисеев, А.\,В.~Поздняков); <<мир 
диалектики>>~--- мир диалога разных логик (Е.\,Л.~Доценко); социальная парадигма 
искусственного интеллекта (<<The society of mind>>) М.~Минского;  
сис\-тем\-но-ор\-га\-ни\-за\-ци\-он\-ный подход в~искусственном интеллекте 
В.\,Б.~Тарасова; теория иерархических многоуровневых систем М.~Месаровича, Д.~Мако 
и~И.~Такахары и~др.~--- и~укладывается в~семь постулатов~[2]: (1)~признание 
гетерогенности мира и~любого объекта, разнообразия жизни; (2)~неопределенность границ 
объектов и~связь <<всего со всем>>; (3)~относительность любой иерархии и~горизонтальные 
связи; (4)~дополнительность и~сотрудничество; (5)~полицентризм; (6)~относительность 
знания; (7)~соответствие управления сложности объекта. 
  
  Сложная задача диагностики АГ (СЗДАГ)~---
  за\-да\-ча-сис\-те\-ма, вклю\-ча\-ющая диагностические и~технологические подзадачи, 
повышающие эффективность обработки симптоматической информации о пациенте. 
Разнообразие подзадач СЗДАГ с~различными характеристическими свойствами требует 
разнообразия соответствующих методов принятия решений, системного анализа, 
искусственного интеллекта и~инженерии знаний. 
  
  Анализ результатов влияния новой картины мира на ментальную составляющую 
врачебной практики и~медицинской информатики~[3] показал, что, несмотря на стремление 
биомедицины к~гетерогенности восприятия организма человека и~процесса его диагностики 
в~рамках семипостулатной картины мира, человек по-преж\-не\-му остается 
<<расчлененным>> объектом познания, что сформировало <<узких>> специалистов, 
поглощенных решением частных задач. Новый тип ученого <<праг\-ма\-ти\-ка-фак\-то\-ло\-га>> 
утратил системное мышление, перестал задумываться над тем, что делается <<вокруг>> 
и~какое значение могут иметь добытые им факты для понимания работы организма в~целом. 
В~этой связи\linebreak\vspace*{-12pt}

\pagebreak

\noindent
 очевидна необходимость перехода от методов <<конкурентной>> диагностики 
к системному мышлению и~методам гетерогенной диагностики.
  
  В~[3--5] представлены результаты системного анализа СЗДАГ, следуя 
  проблемно-структурной (ПС) методологии, этапы~1--5~[6]: идентификация, редукция сложной задачи, 
спецификация диагностических подзадач, выбор методов их решения, а~также проверка 
неоднородности сложной задачи диагностики. Работы~[3--5] подтвердили релевантность 
применения междисциплинарных инструментариев для решения 
СЗДАГ, мо\-де\-ли\-ру\-ющих разнообразие информации, 
сотрудничество, дополнительность и~относительность знаний, сочетающих методы 
и~методики системного анализа диагностической проблемы с~динамическим синтезом 
метода ее решения и~имитацией работы искусственного гетерогенного коллектива~--- 
<<виртуального консилиума>>.
  
  Разнообразие~--- признак, проявление гетерогенности. Следствие закона необходимого 
разнообразия У.\,Р.~Эшби констатирует, что управ\-ле\-ние обеспечивается, если разнообразие 
средств управ\-ля\-юще\-го не меньше разнообразия управ\-ля\-емой им ситуации. Для отображения 
в информатике ситуативного разнообразия в~естественных гетерогенных системах в~[6] 
введены модели <<гетерогенная, неоднородная задача>> и~<<гомогенная, однородная 
задача>>, а~сам закон трактуется так: только разнообразная, скоординированная клиническая 
деятельность, элементы которой в~комбинации решают одну задачу, сделает результат 
диагностики качественно лучше в~обществе с~новой научной картиной мира. Специфике 
такой работы соответствует коллективный труд экспертов в~малых группах за круглым 
столом~--- консилиумы, совещания, естественные гетерогенные системы для решения 
сложных задач~\cite{3-kir}, где на первый план выходят знания и~опыт лица, принимающего 
решения (ЛПР), и~экспертов.
  
  \begin{figure*} %fig1
\vspace*{1pt}
 \begin{center}  
\mbox{%
 \epsfxsize=147.497mm
 \epsfbox{kir-1.eps}
 }
\end{center} 
%\vspace*{-9pt}
%\Caption{Концептуальная модель процесса диагностики артериальной гипертензии: в~многопрофильном 
%стационарном больничном учреждении~(\textit{а}); в~амбулаторно-поликлиническом~(\textit{б})}
  \end{figure*}

  \addtocounter{figure}{1}
  
  Настоящая работа~--- продолжение работ~[3--5,\linebreak 7] и~имеет целью представить: (1)~результаты 
исследования процесса диагностики АГ  
в~ле\-чеб\-но-про\-фи\-лак\-ти\-че\-ских больничных учреждениях (ЛПУ) широкого 
профиля~--- предлагается повысить эффективность и~качество индивидуальных 
диагностических решений в~ЛПУ широкого профиля ам\-бу\-ла\-тор\-но-по\-ли\-кли\-ни\-че\-ско\-го 
характера (рис.~1,\,\textit{а}) за счет внедрения информационной технологии 
<<Виртуальный консилиум>>, моделирующей коллективное обсуждение; 
(2)~архитектуру <<Виртуального консилиума>> и~результаты лабораторных экспериментов с~
его интегрированными моделями (первые результаты лабораторных экспериментов 
приведены в~[7]).

\section{Диагностика артериальной гипертензии в~многопрофильном 
стационарном больничном учреждении и~в~амбулаторно-поликлиническом 
учреждении}

\vspace*{-9pt}


  В~[8, 9] представлены результаты исследования процесса диагностики 
АГ в~Калининградской клинической областной больнице (КОКБ) 
(см.\ рис.~1,\,\textit{б}) и~ее Диагностическом центре (см.\ рис.~1,\,\textit{а}). 

Для формирования 
полного дифференциального диагноза АГ коллективом врачей во главе с~лечащим врачом, 
ЛПР-кар\-дио\-ло\-гом, в~стационаре привлекаются до тринадцати вра\-чей-экс\-пер\-тов~--- носителей 
знаний из различных разделов медицины: невролог, нефролог, сосудистый хирург, уролог, 
психолог, педиатр, аку\-шер-ги\-не\-ко\-лог, онколог, окулист, врачи функциональной 
диагностики, эндокринолог, терапевт, кардиолог. 

Для исследований выбраны шесть 
специалистов (см.\ рис.~1,\,\textit{б}), решающих двенадцать функциональных подзадач 
(рис.~\ref{f2-kir}), возникающих в~90\%~случаев диагностики АГ, 
каждый из которых формирует промежуточные заключения о~состоянии объекта 
диагностики в~своей области медицинских зна\-ний. 
{\looseness=1

}

Полученные исходные данные об объекте 
диагностики разнородны (содержатся в~медицинской карте): количественные,  
ви\-зу\-аль\-но-графиче\-ские параметры (детерминированные переменные),\linebreak 
лингвистические четкие и~нечеткие переменные. Лицо, при\-ни\-ма\-ющее решение, изучает в~медицинской карте 
симптомы и~частные диагностические мнения вра\-чей-экс\-пер\-тов, множество которых 
подбирает сам, и~ставит заключительный диагноз. Вра\-чам-экс\-пер\-там доступны симптомы 
и~мнения других врачей-экспертов из медицинской карты.
\mbox{Лицо}, при\-ни\-ма\-ющее решение, и~вра\-чи-экс\-пер\-ты 
обследуют пациента и~формируют диагностические заключения согласно нормативным 
документам, например~[10]. В~ЛПУ широкого профиля (см.\ рис.~1,\,\textit{а}) ЛПР~--- это врач 
общей практики или терапевт (иногда кардиолог, но зачастую без опыта работы, к~которому 
направляет терапевт сразу же при выявлении повышенного артериального давления), это 
врач <<праг\-ма\-тик-фак\-то\-лог>>~\cite{9-kir}, объединяющий в~себе роли вра\-ча-ЛПР  
и~вра\-чей-экс\-пер\-тов узкой специализации.

\end{multicols}

\begin{figure} %fig2
\vspace*{1pt}
 \begin{center}  
\mbox{%
 \epsfxsize=163.044mm
 \epsfbox{kir-2.eps}
 }
\end{center} 
\vspace*{-9pt}
\Caption{Архитектура ВКДАГ }
\label{f2-kir}
\vspace*{3pt}
\end{figure}

\begin{multicols}{2}
  

  Исследования диагностического процесса на материалах Диагностического центра КОКБ 
по модели на рис.~1,\,\textit{а} показали, что~70\%~пациентов с~АГ 
амбулаторно-поликлинического учреждения не знают о своем заболевании, в~то время как в~стационарных 
медицинских учреждениях (см.\ рис.~1,\,\textit{б}) практически в~100\%~случаев имеет место 
как адекватное проведение, так и~отображение в~медицинских картах симптоматических 
данных обследования с~подтверждением диагноза  
ла\-бо\-ра\-тор\-но-ин\-ст\-ру\-мен\-таль\-ны\-ми методами исследования. 
  
  В этой связи предлагается повысить эффективность и~качество индивидуальных 
диагностических решений в~ЛПУ широкого профиля амбула\-тор\-но-по\-ли\-кли\-ни\-че\-ско\-го 
характера (см.\ рис.~1,\,\textit{а}) за счет внед\-ре\-ния информационной технологии 
<<Виртуальный консилиум>> (см.\ рис.~\ref{f2-kir}), моделирующей коллективное обсуждение, 
обладающего синергией, опытом и~знаниями в~решении подзадач диагностики 
АГ в~стационаре (см.\ рис.~1,\,\textit{б}). 


  

  
\section{Инструментальная среда <<Виртуальный консилиум для~диагностики 
артериальной гипертензии>>}

\vspace*{-18pt}

  Инструментальная среда <<Виртуальный консилиум>>, архитектура которой 
представлена на рис.~\ref{f2-kir}, а~структура в~\cite{7-kir}, ограничена пациентами 
стар\-ше~18~лет, без особых состояний, нет распознавания снимков, не предусматривается 
назначение лечения и~не диагностируется ряд симптоматических артериальных гипертензий. 

Архитектура <<Виртуального консилиума для диагностики артериальной гипертензии>> 
(ВКДАГ) включает межмодульные интерфейсы~$\zeta^u$ для модулей, реализованных 
посредством различных методологий гибридных интеллектуальных сис\-тем (\mbox{ГиИС}) 
(генетические алгоритмы ($g$), нечеткие 
сис-\linebreak\vspace*{-12pt}

\pagebreak

\end{multicols}

\begin{table*}\small
%\vspace*{-12pt}
\begin{center}
\Caption{Описание блоков архитектуры ВКДАГ}
\vspace*{2ex}

\begin{tabular}{|p{30mm}|p{40mm}|p{39mm}|p{39mm}|}
\hline
\multicolumn{1}{|c|}{Наименование блока}&\multicolumn{1}{c|}{Функции}&\multicolumn{1}{c|}{Вход}&\multicolumn{1}{c|} 
{Выход}\\
\hline
Технологический модуль $i$-й&
Организация эффективной обработки данных и~знаний, выбирается для 
включения в~функциональную \mbox{ГиИС}~--- построение информативного набора 
признаков для диагностики&Популяция 
индивидуумов, накладывающихся как маска на $i$-й функциональный модуль&
Наилучшая особь с~оптимальным набором признаков~--- накладывается как 
маска на $i$-й функциональный модуль\\
\hline
Функциональный модуль $i$-й&Классификация состояния здоровья пациента в~рамках 
\mbox{$i$-й} диагностической 
подзадачи, выбирается для включения в~функциональную \mbox{ГиИС} &
Подмножество $i$-е симптомов с~интерфейса 
пользователя&Частное $i$-е заключение о~со\-сто\-янии здоровья пациента\\
\hline
Функциональный модуль {HCCCC}, моделирующий ЛПР&
Формирование заключительного диагноза 
АГ (всегда в~составе <<Виртуального консилиума>>)&Подмножество симптомов 
с~интерфейса пользователя, множество выходов функциональных модулей&
Заключительный диагноз АГ \\
\hline
Функциональный модуль {ИНСРЭКГ}&Классификация патологического состояния пациента по его 
электрокардиограмме&\multicolumn{2}{p{60mm}|}{Рассмотрены подробно в~\cite{4-kir}}\\
\cline{1-2}
Функциональный модуль {ИНССМАД}&Прогноз нормальных зна\-чений суточного мониторирования 
артериального давле\-ния и~вычисление отклонения &\multicolumn{2}{c|}{\ }\\
\hline
Интерфейс модификации структуры {ВКДАГ}&Исключение из диагностики модулей, решающих не 
интересующие пользователя подзадачи &
Выбранные пользователем подзадачи диагностики &
Функциональная ГиИС, 
синтезированная посредством алгоритма из~\cite{4-kir}\\
\hline
Интерфейс пользователя <<Диагноз>>&Визуализация результатов диагностики и~корректировка их 
пользователем &Заключительный диагноз от функционального модуля НСССС&Отчет, содержащий 
множество симптомов и~диагноз\\
\hline
Интерфейс пользователя &Ввод информации о~со\-сто\-янии здоровья пациента &
Множество значений 
показателей состояния здоровья пациента&
Показатели состояния здоровья пациента, распределенные по 
функциональным модулям \\
\hline
Модификация интерфейса пользователя&Деактивация элементов на интерфейсе пользователя для ввода 
значений показателей состояния здоровья&Множество выходов технологических модулей&Частично 
деактивированный интерфейс пользователя \\
\hline
\end{tabular}
\end{center}
\end{table*}

\begin{multicols}{2}

\noindent 
те\-мы ($f$), искусственные нейронные сети ($n$)).
 В~библиотеке модулей диагностики 
и~препро\-цессии хранятся заранее инициализированные\linebreak в~программной среде 
функциональные и~технологические модели. 
По умолчанию все модули включены 
в~структуру <<Виртуального консилиума>>, их описание пред\-став\-ле\-но в~табл.~1. %\\[-15pt]
%
      <<Виртуальный консилиум>> (см.\ рис.~\ref{f2-kir}) запускает интерфейс пользователя, 
ЛПР-вра\-ча~--- <<{Интерфейс модификации структуры ВКДАГ}>>, посредством 
которого включаются функциональные 
 и~технологические модули в~работу сис\-те\-мы: модуль 
<<Анализ СМАД>>, модуль <<Распознавание ЭКГ>>, модули технологических подзадач из 
группы <<Построение информативного набора признаков\linebreak (симптомов) при диагностике 
заболеваний>> и~модули подзадач из группы <<Диагностика критериев оценки 
сер\-деч\-но-со\-су\-ди\-сто\-го риска и~вторичной АГ у~пациента>> ({ДАГ}$_1$, \ldots , {ДАГ}$_9$): 
диагностики\linebreak поражений ор\-га\-нов-ми\-ше\-ней, факторов риска, цереброваскулярных 
болезней, метаболического синд\-ро\-ма и~сахарного диабета, заболеваний периферических 
артерий, ишемической болезни сердца,\linebreak эндокринной АГ, паренхиматозной нефропатии 
и~реноваскулярной АГ соответственно. Все выбранные $i$-е технологические модули 
запускаются, решают соответствующую подзадачу и~передают информацию на блок 
<<{Модификация интерфейса пользователя}>>. Он деактивирует показатели 
со\-сто\-яния здоровья на <<{Интерфейсе пользователя для\linebreak ввода значений показателей 
состояния здоровья пациента}>> и~корректирует работу $i$-го функционального модуля 
подзадач {ДАГ}$_1$, \ldots\linebreak \ldots , {ДАГ}$_9$. Далее активируется откорректированный 
интерфейс, вводятся симптомы, которые передаются функциональным нечетким модулям, 
решающим подзадачи {ДАГ}$_1$, \ldots , {ДАГ}$_9$\linebreak (моделируют принятие 
решения экспертами, врачами смежных специальностей~--- кардиологом как экспертом, 
неврологом, нефрологом, терапевтом, эндокринологом, урологом). Последние в~свою 
очередь передают информацию о~патологиях, выявленных ими у~пациента, 
функциональному модулю {НСССС} (моделирует принятие решения ЛПР~---  
вра\-чом-кар\-дио\-ло\-гом), решающему подзадачу <<Оценка степени и~стадии 
артериальной гипертензии, степени риска сер\-дечно-сосу\-ди\-стых заболеваний>>. 

В~библиотеке ВКДАГ есть еще два функциональных модуля (см.\ табл.~1), вклю\-ча\-ющих\-ся 
в~работу консилиума посредством <<{Интерфейса модификации структуры 
ВКДАГ}>>: 
      \begin{enumerate}[(1)]
      \item {ИНСРЭКГ}, передающий информацию на модули диагностики поражений 
ор\-га\-нов-ми\-ше\-ней (на рис.~\ref{f2-kir}~--- это {НСДАГ}$_1$), цереброваскулярных 
болезней ({НСДАГ}$_3$) и~ишемической болезни сердца ({НСДАГ}$_6$); 
      \item {ИНССМАД}, формирующий информацию о~нормальных значениях 
суточного артериального давления на функциональный модуль {НСССС}.
      \end{enumerate}
      
\section{Экспериментальное лабораторное исследование программной 
реализации прототипа инструментальной среды <<Виртуальный консилиум>>}
  
  Экспериментальное лабораторное исследование программной реализации 
исследовательского прототипа функциональной гибридной интеллектуальной системы 
ВКДАГ для поддержки принятия сложных диагностических решений необходимо для 
подтверждения его релевантности~[3--5, 7] реальной ситуации диагностики АГ. В~[4] 
пред\-став\-ле\-на информация по особенностям функциональных и~технологических моделей 
гетерогенного модельного поля ВКДАГ, а~в~[7]~--- информация по их инициализации 
в~среде MATLAB-Simulink, результаты исследований качества работы каждой модели 
гетерогенного модельного поля <<Виртуального консилиума>> автономно, а~также 
подтверждена их релевантность работе экспертов~--- врачей узкой специализации, что 
предотвращает распространение ошибок работы автономных моделей на работу 
интегрированной модели. 

В~настоящей работе приведены результаты исследования качества 
интегрированных моделей, синтезированных <<Виртуальным консилиумом>>\linebreak 
и~моделирующих дополнительность и~сотрудничество, которые имитируют коллективные 
рас\-суж\-де\-ния специалистов при постановке диагноза. 

В~табл.~2 представлены критерии 
и~результаты тес\-ти\-ро\-ва\-ния интегрированных моделей <<Виртуального консилиума>> 
с~различными комбинациями знаний врачей, классифицирующих патологическое состояние 
пациента. Порядок работы моделей гетерогенного модельного поля \mbox{ВКДАГ}: запускаются 
модели первой очереди~--- модели технологических элементов {ГАППС}$_{1\mbox{--}9}$, 
корректирующие множества входных переменных моделей {НСДАГ}$_{1\mbox{--}9}$ 
и~{НСССС}; обработка информации передается функциональным элементам: модели 
второй очереди <<отправляют>> информацию на модели третьей, пятой, шес\-той и~седьмой 
очередей~--- \mbox{ИНСРЭКГ} (модель, решающая задачу распознавания электрокардиограммы (ЭКГ)), 
{ИНССМАД} (формирует оптимальные множества показателей суточного давления), 
{НСДАГ}$_9$, {НСДАГ}$_2$ и~{НСДАГ}$_6$; третья\linebreak очередь содержит 
модели НСДАГ$_4$ и~НСДАГ$_5$, передающие выходную информацию на вход моделей четвертой 
и~седьмой очередей; четвертая очередь содержит модель {НСДАГ}$_8$, пе\-ре\-да\-ющую 
информацию  модели пятой очереди {НСДАГ}$_1$, которая в~свою очередь передает 
информацию\linebreak {НСДАГ}$_3$ (шес\-тая очередь); от {НСДАГ}$_3$ передается 
информация {НСДАГ}$_7$ (седьмая очередь); последней запускается модель 
{НСССС}, формирующая заключительный диагноз, на вход которой передается 
выходная информация функциональных моделей вто\-рой--седь\-мой очередей.
  
  Таким образом: (1)~без знаний кардиолога, или нефролога, или эндокринолога 
сред\-не\-квад\-ратическая ошибка наибольшая~--- 0,697; 0,448 и~0,211 соответственно, 
и~объясняется это тем, что кардиолог играет ключевую роль в~обработке ин\-формации, 
поступающей от других врачей\linebreak\vspace*{-12pt}


\pagebreak

\end{multicols}

\begin{table}\small
\begin{center}
\Caption{Параметры и~результаты тестирования интегрированных моделей }
\vspace*{2ex}

\begin{tabular}{|p{66mm}|p{88mm}|}
\hline
\multicolumn{1}{|c|}{\tabcolsep=0pt\begin{tabular}{c}Наименование параметров\\ 
и результатов тестирования\end{tabular}}&
\multicolumn{1}{c|}{Значения параметров и~результатов 
тестирования}\\
\hline
Объем тестовой выборки ВКДАГ, интегрирующего знания всех шести врачей&800 наблюдений~--- 500 с~
диагнозами эссенциальной АГ и~300 с~диагнозами вторичной АГ\\
\hline
Объем тестовой выборки ВКДАГ, интегрирующего знания менее шести врачей&400 наблюдений~--- 200 с~
диагнозами эссенциальной АГ и~200 с~диагнозами вторичной АГ\\
\hline
Источник формирования тестовой вы\-борки&Архив медицинских карт пациентов 1-го кардиологического 
отделения КОКБ\\
\hline
Элемент тестирующей последова\-тель\-ности&
Содержит множество нечетких лингвистических переменных и~вектор образа электрокардиограммы (может отсутствовать)\\
\hline
Эталонный диагноз&Результаты деятельности лечащего вра\-ча-кар\-дио\-ло\-га, подводящего общий итог~--- 
дифференциальный диагноз АГ\\
\hline
Критерии тестирования&Среднеквадратическая ошибка $f$ классификации состояния здоровья пациента~[7]\\
\hline
$f$(шесть врачей)&0,0837\\
\hline
$f$(без кардиолога)&0,697\\
\hline
$f$(без нефролога)&0,448 (в остальных 55,2\% случаях диагноз не вызовет доверия)\\
\hline
$f$(без терапевта)&0,151\\
\hline
$f$(без невролога)&0,149\\
\hline
$f$(без эндокринолога)&0,211 (в остальных 78,9\% случаях диагноз не вызовет доверия)\\
\hline
$f$(без сосудистого хирурга)&0,0798\\
\hline
$f$(без знаний терапевта, невролога, неф\-ро\-ло\-га, эндокринолога, сосудистого хирурга)&0,711\\
\hline
$f$(без знаний терапевта, невролога, эндокринолога, сосудистого хирурга)&0,485\\
\hline
$f$(без знаний невролога, эндокринолога, сосудистого хирурга)&0,334\\
\hline
$f$(без знаний невролога, сосудистого хи\-рурга)&0,167\\
\hline
\end{tabular}
\end{center}
\end{table}

\begin{multicols}{2}


\noindent
 и~от ла\-бораторных исследований, и~в~постановке
заключительного диагноза, а~нефролог и~эндокринолог~--- в~исключении вторичной 
АГ; (2)~знания врача~--- сосудистого хирурга не влияют на 
результаты работы <<Виртуального консилиума>>, и~объясняется это тем, что знания 
сосудистого хирурга, касающиеся диагностики АГ, составляют только~20\% базы знаний 
нечеткой системы, распознающей заболевания периферических артерий (ассоциативные 
клинические состояния), встречающихся не более чем у~10\% населения~\cite{11-kir}, 
и~в~тес\-то\-вую выборку не попала ни одна карта с~данными заболеваниями; (3)~чем больше 
численный состав <<Виртуального консилиума>>, тем с~меньшей среднеквадратической 
ошибкой он классифицирует состояние здоровья пациента; (4)~<<Виртуальный консилиум>> 
в~со\-ста\-ве шести врачей диагностирует АГ со среднеквадратической 
ошибкой постановки диагноза $f = 0{,}0837$, т.\,е.\ дает диагноз, верный в~84\% слу\-чаях. 
{\looseness=1

}
  
  Поскольку <<Виртуальный консилиум>> разра\-ботан на основе всероссийских~\cite{9-kir} 
и~между\-народных рекомендаций по диагностике АГ и~со\-пут\-ст\-ву\-ющих заболеваний, 
которых должен придерживать\-ся каж\-дый врач в~своей практике, при переносе \mbox{ВКДАГ} 
в~другое больничное учреж\-де\-ние необходимо пред\-оста\-вить врачам данного учреждения 
протоколы подтверждения диагностических правил всех баз знаний экспериментальными 
данными из архива КОКБ для ознакомления 
и~внесения при необходимости коррективов в~связи с~возможными особенностями их 
контингента пациентов, а~также возможных требований по устранению ограничений 
системы со стороны персонала нового больничного учреждения. Значительной 
корректировки баз знаний не потребуется.
  
  Таким образом, лабораторные эксперименты с~прототипом <<Виртуального 
консилиума>> дали обнадеживающие результаты. 

Верное решение получено в~84\% 
случаев. В~ам\-бу\-ла\-тор\-но-кли\-ни\-че\-ских учреждениях диагноз не 
выявляется у~70\% пациентов в~основном по причине инертности врачей, недостатка опыта 
врачей узкой специализации и~нехватки кадров в~ЛПУ
широкого профиля, что по результатам экспериментов может быть устранено с~по\-мощью 
применения \mbox{ВКДАГ} во время приема пациентов с~подозрением на АГ.

\section{Заключение}

  Лабораторно подтверждена эффективность предлагаемого подхода для проектирования 
диагностических систем как гетерогенных искусственных диагностических систем со 
свойствами дополнительности, сотрудничества и~относительности\linebreak
 знаний, синтезирующих 
интегрированные методы и~модели, разнообразие которых устраняет разнообразие 
диагностической информации об организме человека~--- <<Виртуальных консилиумов>>,\linebreak 
моделиру\-ющих работу коллектива врачей в~многопрофильном стационарном больничном 
учреждении (на примере КОКБ) и~внедрение 
которых повыша\-ет эффективность и~качество индивидуальных диагностических решений 
в~ам\-бу\-ла\-тор\-но-по\-ли\-кли\-ни\-че\-ском учреждении широкого профиля (на примере 
Диагностического центра КОКБ), где заключение о состоянии больного из-за проблемы 
с~кадрами узкой специализации принимает чаще всего один специалист~--- терапевт или 
врач общей практики, иногда кардиолог, но без опыта работы.

{\small\frenchspacing
 {%\baselineskip=10.8pt
 \addcontentsline{toc}{section}{References}
 \begin{thebibliography}{99}
\bibitem{1-kir}
\Au{Гаазе-Раппопорт М.\,Г., Поспелов~Д.\,А.} От амебы до робота: модели поведения.~--- 
М.: Наука, 1987. 288~с.
\bibitem{2-kir}
\Au{Колесников А.\,В., Кириков~И.\,А., Листопад~С.\,В. %Румовская~С.\,Б. 
и~др.} Решение 
сложных задач коммивояжера методами функциональных гибридных интеллектуальных 
сис\-тем.~--- М.: ИПИ РАН, 2011. 295~с.
\bibitem{3-kir}
\Au{Кириков И.\,А., Колесников~А.\,В., Румовская~С.\,Б.} Исследование сложной задачи 
диагностики артериальной гипертензии в~методологии искусственных гетерогенных  
сис\-тем~// Системы и~средства информатики, 2013. Т.~23. №\,2. С.~81--99. doi: 
10.14357/08696527130208.
\bibitem{4-kir}
\Au{Кириков И.\,А., Колесников~А.\,В., Румовская~С.\,Б.} Функциональная гибридная 
интеллектуальная система для поддержки принятия решений при диагностике артериальной 
гипертензии~// Системы и~средства информатики, 2014. Т.~24. №\,1. С.~153--179. doi: 
10.14357/08696527140110.
\bibitem{5-kir}
\Au{Колесников А.\,В., Румовская~С.\,Б., Листопад~С.\,В., Кириков~И.\,А.} Системный 
анализ в~решении сложных диагностических задач~// Системный анализ и~информационные 
технологии (САИТ-2015): Тр. VI~Междунар. конф.~--- М.: 
ИСА РАН, 2015. Т.~1. С.~157--167.
\bibitem{6-kir}
\Au{Колесников А.\,В., Кириков~И.\,А.} Методология и~технология решения сложных задач 
методами функциональных гибридных интеллектуальных систем.~--- М.: ИПИ РАН, 2007. 
387~с.
\bibitem{7-kir}
\Au{Кириков И.\,А., Колесников~А.\,В., Румовская~С.\,Б.} Исследование лабораторного 
прототипа искусственной гетерогенной системы для диагностики артериальной 
гипертензии~// Системы и~средства информатики, 2014. Т.~24. №\,3. С.~131--143. doi: 
10.14357/08696527140309.
\bibitem{8-kir}
\Au{Румовская С.\,Б.} Методы и~средства информатики для диагностики 
артериальной гипертензии в~ле\-чеб\-но-про\-фи\-лак\-ти\-че\-ских учреждениях 
широкого профиля~// Задачи современной информатики (ЗСИ-2015): Тр. 2-й 
молодежной научной конф.~--- М.: ФИЦ ИУ РАН, 2015. 
С.~168--174.
\bibitem{9-kir}
\Au{Кириков~И.\,А., Румовская~С.\,Б.} Гетерогенная диагностика артериальной 
гипертензии~// Информатика, управление и~системный анализ (ИУСА-2016): Тр. 
4-й Всеросс. научной конф. молодых ученых с~международным участием.~--- 
Тверь: ТвГТУ, 2016. Т.~1. С.~180--188.
\bibitem{10-kir}
Комитет экспертов ВНОК. Диагностика и~лечение артериальной гипертензии. 
Российские рекомендации~// Системные гипертензии, 2010. Вып.~3. С.~5--26.
\bibitem{11-kir}
\Au{Галимзянов Ф.\,В.} Заболевания периферических артерий (клиника, 
диагностика, лечение)~// Международный журнал экспериментального образования, 
2014. Вып.~8. С.~113--114. 

\end{thebibliography}

 }
 }

\end{multicols}

\vspace*{-6pt}

\hfill{\small\textit{Поступила в~редакцию 18.06.16}}

\vspace*{8pt}

%\newpage

%\vspace*{-24pt}

\hrule

\vspace*{2pt}

\hrule

%\vspace*{8pt}



\def\tit{``VIRTUAL COUNCIL''~--- SOURCE ENVIRONMENT SUPPORTING 
COMPLEX DIAGNOSTIC DECISION MAKING}

\def\titkol{``Virtual council''~--- source environment supporting 
complex diagnostic decision making}

\def\aut{I.\,А.~Kirikov$^1$, А.\,V.~Kolesnikov$^{1,2}$, S.\,V.~Listopad$^1$, and 
S.\,B.~Rumovskaya$^1$}

\def\autkol{I.\,А.~Kirikov, А.\,V.~Kolesnikov, S.\,V.~Listopad, and 
S.\,B.~Rumovskaya}

\titel{\tit}{\aut}{\autkol}{\titkol}

\vspace*{-9pt}

\noindent
$^1$Kaliningrad Branch of the Federal Research Center ``Computer Science and 
Control'' of the Russian Academy\linebreak
$\hphantom{^1}$of Sciences, 5~Gostinaya Str., Kaliningrad 236000, 
Russian Federation
   
   \noindent
   $^2$Immanuel Kant Baltic Federal University, 14~Nevskogo Str., Kaliningrad 236041, 
Russian Federation


\def\leftfootline{\small{\textbf{\thepage}
\hfill INFORMATIKA I EE PRIMENENIYA~--- INFORMATICS AND
APPLICATIONS\ \ \ 2016\ \ \ volume~10\ \ \ issue\ 3}
}%
 \def\rightfootline{\small{INFORMATIKA I EE PRIMENENIYA~---
INFORMATICS AND APPLICATIONS\ \ \ 2016\ \ \ volume~10\ \ \ issue\ 3
\hfill \textbf{\thepage}}}

\vspace*{3pt}
  
    
  
\Abste{The paper considers the problem of individual decision making during 
diagnostics of 
patients in outpatient clinics by the example of arterial 
hypertension diagnostics. It is proposed to 
raise the quality of individual decision\linebreak\vspace*{-12pt}}

\Abstend{making by means of consultations with the ``Virtual council'' 
decision support system, which models the work of physician councils in inpatient multifield 
clinics. The results of development and experimental research of the 
laboratory prototype of ``Virtual council'' are presented.}

\KWE{decision support system; virtual council; functional hybrid intellectual system}

\DOI{10.14357/19922264160311} 

\vspace*{-9pt}

\Ack
\noindent
The work was performed with partial support of the Russian
Foundation for Basic Research (grant No.\,16-07-00272~А).


%\vspace*{3pt}

  \begin{multicols}{2}

\renewcommand{\bibname}{\protect\rmfamily References}
%\renewcommand{\bibname}{\large\protect\rm References}

{\small\frenchspacing
 {%\baselineskip=10.8pt
 \addcontentsline{toc}{section}{References}
 \begin{thebibliography}{99}
\bibitem{1-kir-1}
\Aue{Gaaze-Rappoport, M.\,G., and D.\,A.~Pospelov}. 1987. \textit{Ot ameby do robota: Modeli 
povedeniya} [From ameba to robotic mashine: Behavior model] Moscow: Nauka. 288~p.
\bibitem{2-kir-1}
\Aue{Kolesnikov,~A.\,V., I.\,A.~Kirikov, S.\,V.~Listopad, \textit{et al.}}. 2011. \textit{Reshenie 
slozhnykh zadach kommivoyazhera metodami funktsional'nykh gibridnykh intellektual'nykh 
sistem} [Solving of the complex traveling salesman problem by means of functional hybrid 
intellectual systems]. Moscow: IPI RAN. 295~p.
\bibitem{3-kir-1}
\Aue{Kirikov, I.\,A., A.\,V.~Kolesnikov, and S.\,B.~Rumovskaya}.\linebreak
 2013. Issledovanie slozhnoy 
zadachi diagnostiki arterial'noy gipertenzii v~metodologii iskusstvennykh geterogennykh sistem 
[Research of the complex problem at\linebreak diagnosing of the arterial hypertension within the 
methodology of artificial heterogeneous systems]. \textit{Sistemy i~Sredstva Informatiki~--- 
Systems and Means of Informatics} 23(2):81--99. doi: 10.14357/08696527130208.
\bibitem{4-kir-1}
\Aue{Kirikov, I.\,A., A.\,V.~Kolesnikov, and S.\,B.~Rumovskaya}.\linebreak
 2014. Funktsional'naya 
gibridnaya intellektual'naya sistema dlya podderzhki prinyatiya resheniya pri diagnostike 
arterial'noy gipertenzii [Functional hybrid intelligent decision support system for diagnosing of the 
\mbox{arterial} hypertension]. \textit{Sistemy i~Sredstva Informatiki~--- Systems and Means of Informatics} 
24(1):153--179. doi: 10.14357/08696527140110. 
\bibitem{5-kir-1}
\Aue{Kolesnikov, A.\,V., I.\,A.~Kirikov, S.\,V.~Listopad, and S.\,B.~Rumovskaya}. 2015. 
Sistemnyy analiz v~reshenii slozhnykh diagnosticheskikh zadach [Systems analysis for solving 
complex diagnostic tasks]. \textit{Tr. 6-y Mezhdunar. konf. ``Sistemnyy analiz i~informatsionnye 
tekhnologii''} [6th Conference (International) ``Systems Analysis and Information Technology'' 
Proceedings]. Moscow.  1:157--167.
\bibitem{6-kir-1}
\Au{Kolesnikov, A.\,V., and I.\,A.~Kirikov}. 2007. \textit{Metodologiya i~tekhnologiya resheniya 
slozhnykh zadach metodami funk\-tsi\-o\-nal'\-nykh gibridnykh intellektual'nykh sistem} [Methodology 
and technology for solving of complex problems using the methodology of functional hybrid 
artificial systems]. Moscow: IPI RAN. 387~p.
\bibitem{7-kir-1}
\Aue{Kirikov, I.\,A., A.\,V.~Kolesnikov, and S.\,B.~Rumovskaya}. 2014. Issledovanie 
laboratornogo prototipa iskusstvennoy geterogennoy sistemy dlya diagnostiki arterial'noy 
gipertenzii [Research of the laboratory prototype of the artificial heterogeneous system for 
diagnosing of the arterial hypertension]. \textit{Sistemy i~Sredstva informatiki~--- Systems and 
Means of Informatics} 24(3):131--143. doi: 10.14357/08696527140309.
\bibitem{8-kir-1}
\Au{Rumovskaya, S.\,B.} 2015. Metody i~sredstva informatiki dlya diagnostiki 
arterial'noy gipertenzii v~lechebno-profilakticheskikh uchrezhdeniyakh shirokogo profilya 
[Methods and tools of informatics for diagnostics of arterial hypertension in multiskilled 
medical preventive institution]. \textit{Tr. 2-y molodezhnoy nauchnoy konf. ``Zadachi 
sovremennoy informatiki''} [2nd Youth Conference ``Tasks of Modern Informatics'' 
Proceedings]. Moscow: FRC ``Computer Science and Control'' RAS. 168--174.
\bibitem{9-kir-1}
\Aue{Kirikov, I.\,A., and S.\,B.~Rumovskaya}. 2016. Geterogennaya diagnostika arterial'noy 
gipertenzii [Heterogeneous diagnostics of arterial hypertension]. \textit{Tr. 4-y Vseross. 
nauchnoy konf. molodykh uchenykh s~mezhdunarodnym uchastiem ``Informatika, 
upravlenie i~sistemnyy analiz''} [4th Youth Conference (International) ``Informatics, Control 
and Systems Analysis'' Proceedings]. Tver: Tver State Technical University. 1:180--188.
\bibitem{10-kir-1}
Komitet ekspertov VNOK [Committee of experts of All-Russia Scientific Society of Сardiologists]. 
2010. Diagnostika i~lechenie arterial'noy gipertenzii. Rossiyskie 
rekomendatsii [Diagnosing and treatment of arterial 
hypertension. Russian recommenation]. 
\textit{Sistemnye gipertenzii} [Systemic Hypertension] 3:5--26. 
\bibitem{11-kir-1}
\Aue{Galimzyanov, F.\,V.} 2014. Zabolevaniya perifericheskikh arteriy (Klinika, 
diagnostika, lechenie) [Peripheral vascular disease (Clinic, diagnostics, treatment]. 
\textit{Mezhdunarodnyy zhurnal eksperimental'nogo obrazovaniya} [Int. J.~Research 
Education] 8:113--114. 
   \end{thebibliography}

 }
 }

\end{multicols}

\vspace*{-9pt}

\hfill{\small\textit{Received June 18, 2016}}

\vspace*{-3pt}
    
  
  \Contr
  
  \noindent
  \textbf{Kirikov Igor A.}\ (b.\ 1955)~---
  Candidate of  Sciences (PhD) in technology; director, Kaliningrad Branch of the 
  Federal Research Center ``Computer Science and Control'' of the Russian Academy 
  of Sciences, 5~Gostinaya Str., Kaliningrad 236000,  Russian Federation; 
baltbipiran@mail.ru
  
  \pagebreak
%  \vspace*{3pt}
  
  \noindent
  \textbf{Kolesnikov Alexander V.}\ (b.\ 1948)~---
  Doctor of Sciences in technology; professor, 
Department of Telecommunications, 
 Immanuel Kant Baltic Federal University, 14~Nevskogo Str., Kaliningrad 236041, Russian Federation; senior scientist, Kaliningrad Branch of 
  the Federal Research Center ``Computer Science and Control'' of the Russian 
  Academy of Sciences, 5~Gostinaya Str., Kaliningrad 236000,  Russian Federation; 
  avkolesnikov@yandex.ru
  
  \vspace*{4pt}
  
  \noindent
  \textbf{Listopad Sergey V.}\ (b.\ 1984)~---
  Candidate of  Sciences (PhD) in technology; scientist, Kaliningrad Branch of the 
  Federal Research Center ``Computer Science and Control'' of the Russian Academy 
  of Sciences, 5~Gostinaya Str., Kaliningrad 236000,  Russian Federation;   
ser-list-post@yandex.ru
  
  \vspace*{4pt}
  
  \noindent
  \textbf{Rumovskaya Sophiya B.}\ (b.\ 1985)~--- programmer~I, Kaliningrad Branch 
  of the Federal Research Center ``Computer Science and Control'' of the Russian 
  Academy of Sciences, 5~Gostinaya Str., Kaliningrad 236000,  Russian Federation; 
  sophiyabr@gmail.com
  \label{end\stat}
  
  
  \renewcommand{\bibname}{\protect\rm Литература}   %4
\def\stat{kovalev}

\def\tit{МЕТОДЫ ТЕОРИИ КАТЕГОРИЙ В~МОДЕЛЬНО-ОРИЕНТИРОВАННОЙ СИСТЕМНОЙ 
ИНЖЕНЕРИИ}

\def\titkol{Методы теории категорий в~модельно-ориентированной системной 
инженерии}

\def\aut{С.\,П.~Ковалёв$^1$}

\def\autkol{С.\,П.~Ковалёв}

\titel{\tit}{\aut}{\autkol}{\titkol}

\index{Ковалёв С.\,П.}
\index{Kovalyov S.\,P.}


%{\renewcommand{\thefootnote}{\fnsymbol{footnote}} \footnotetext[1]
%{Исследование выполнено при финансовой поддержке Российского научного фонда (проект 16-11-10227).}}


\renewcommand{\thefootnote}{\arabic{footnote}}
\footnotetext[1]{Институт проблем управления им.\ В.\,А.~Трапезникова 
Российской академии наук,  \mbox{kovalyov@nm.ru}}

%\vspace*{-18pt}

\Abst{Предложен математический аппарат на базе теории категорий, который позволяет 
формально описывать и~строго исследовать процедуры применения моделей в~инженерной 
деятельности, составляющие сущность мо\-дель\-но-ори\-ен\-ти\-ро\-ван\-ной системной 
инженерии (Model-Based Systems Engineering, MBSE). В~основе аппарата лежит 
математическое представление сборочных чертежей (мегамоделей сис\-тем) диаграммами 
в~категориях, объектами которых служат модели, а~морфизмы представляют действия по 
сборке моделей сис\-тем из моделей компонентов. Адекватность аппарата обоснована исходя 
из требований стандартов, регламентирующих описание структуры систем, в~том числе 
IEC~81346. Предложены и~исследованы тео\-ре\-ти\-ко-ка\-те\-гор\-ные методы решения ряда 
практических задач сборки систем. Приведены примеры решения таких задач в~категориях, 
представляющих две ключевые области применения MBSE: гео\-мет\-ри\-че\-ское моделирование 
изделий сложной формы и~дис\-крет\-но-со\-бы\-тий\-ное имитационное моделирование 
поведения технических систем.}

\KW{модельно-ориентированная системная инженерия; мегамодель; теория категорий; 
копредел}



\DOI{10.14357/19922264170305} 


\vspace*{6pt}

\vskip 10pt plus 9pt minus 6pt

\thispagestyle{headings}

\begin{multicols}{2}

\label{st\stat}

\section{Введение}

   Модельно-ориентированная системная инженерия состоит в~формализованном применении моделирования в~
поддержке жизненного цикла сис\-тем, включая сбор требований, 
проектирование, проверку и~приемку, другие стадии~[1]. Модели, 
разрабатываемые в~ходе процедур MBSE, пригодны к~автоматической 
обработке на компьютерах. Это позволяет сначала задавать, верифицировать 
и~оптимизировать проектные решения на моделях <<в циф\-ре~и только потом 
воплощать <<в железе>>, снижая затраты на организацию жизненного цикла 
изделий и~сокращая сроки выполнения работ~[2].
   
   И все же внедрение технологий MBSE в~инженерную деятельность 
происходит медленно. Это связано во многом с~нехваткой единой 
концептуальной базы инженерного моделирования: предлагается много 
частных языков и~технологий, слабо совместимых друг с~другом и~плохо 
приспособленных для совместной разработки моделей большими 
мультидисциплинарными коллективами~[3]. Тем самым затрудняется переход 
от набора электронных чертежей к~полноценному электронно-цифровому 
макету (digital mock-up) промышленного изделия.
   
   Естественный, хотя и~<<трудный>>, подход к~получению результатов 
общего характера, унифи\-ци\-ру\-ющих разнородные технологии, состоит в~том, 
чтобы как можно более строго формализовать процедуры моделирования. 
Формализация позволит совершенствовать процедуры MBSE и~передавать их 
на исполнение компьютеру без пробелов и~искажений. Самый высокий уровень 
строгости достигается при привлечении математического аппарата, поскольку 
математика позволяет надежно доказывать или опровергать утверждения, 
ха\-рак\-те\-ри\-зу\-ющие корректность и~эффективность процедур.
   
   В настоящей работе предложен аппарат, основанный на математическом 
представлении сборочных чертежей (<<мегамоделей>> систем) 
ориенти-\linebreak рованными графами (диаграммами). Узлы такого\linebreak графа помечаются 
обозначениями моделей час\-тей, а~реб\-ра помечаются обозначениями действий\linebreak 
(activities), посредством которых части собираются в~систему. Представление 
структуры систем графами регламентируется, в~частности, стандартом 
IEC~81346~[4]. Естественным источником математических методов 
конструирования и~анализа мегамоделей служит теория категорий (см., 
например,~[5, 6]). Модели рассматриваются как объекты подходящих 
категорий, а~действия формально описываются морфизмами. Строятся 
и~исследу-\linebreak ются тео\-ре\-ти\-ко-ка\-те\-гор\-ные конструкции, опи\-сы\-ва\-ющие процедуры 
MBSE на абстрактном кон-\linebreak цептуальном уровне. Определенный опыт такого\linebreak 
исследования был накоплен в~инженерии программного обеспечения~[7] 
и~теперь может быть обобщен для системной инженерии в~целом. Например, 
сборке системы согласно некоторой мегамодели отвечает построение 
копредела диаграммы~--- универсальной конструкции~\cite{5-kov}.
   
   Статья построена следующим образом. В~разд.~2 приведен обзор 
принципов описания структуры сис\-тем согласно стандарту IEC~81346. 
Раздел~3 посвящен практическим проб\-ле\-мам мегамоделирования и~сборке 
сис\-тем. В~разд.~4 вводятся конструкции тео\-рии категорий, позволяющие 
формально решать задачи мегамоделирования. В~заключении приводятся 
выводы и~намечаются направления дальнейших исследований.

\section{Структура систем и~стандарт~IEC~81346}

   Важной проблемой MBSE, отмеченной во введении, является слабая 
совместимость языков и~инструмен\-тов моделирования от разных поставщиков. 
Основным подходом к~достижению совместимости является стандартизация~--- 
принятие обязывающих документов, устанавливающих требования и~принципы 
взаимозаменяемости инструментов. Многие стандарты определяют конкретные 
форматы машиночитаемой записи моделей, нейтральные относительно 
разработчиков инструментов MBSE. Примером служит формат описания 
твердотельных геометрических моделей STEP, стандартизованный семейством 
ISO~10303. Однако для формализации MBSE в~целом интерес представляют 
в~первую очередь стандарты более общего плана, унифицирующие принципы 
и~методы применения моделей в~жизненном цикле систем независимо от 
способа записи моделей. С~этой точки зрения внимания заслуживает 
международный стандарт IEC 81346-1:2009 <<Промышленные системы, 
установки и~обору\-до\-ва\-ние~--- принципы структурирования и~ссылочные 
обозначения~--- часть~1: основные правила>> (<<Industrial Systems, 
Installations and Equipment and Industrial Products~--- Structuring Principles and 
Reference Designations~--- Part~1: Basic Rules>>)~\cite{4-kov}. Стандарт не 
принят в~России, однако ряду его положений в~области структуры систем 
соответствует российский ГОСТ~2.053-2013 <<ЕСКД. Электронная структура 
изделия. Общие положения>>.
   
   В стандарте IEC~81346 рассматривается ряд вопросов моделирования 
структуры систем и~идентификации отдельных единиц в~составе систем. 
Системная единица названа в~стандарте объектом, причем принципиально не 
проводится различие между объектами реального мира, составляющими 
реально существующие системы, и~объектами мыслительной деятельности~--- 
моделями единиц, составляющими модели систем. Таким образом, стандарт 
выходит за рамки MBSE и~рассматривает ряд вопросов системной инженерии 
вообще. Иерар\-хи\-че\-ская структура системы (холархия~\cite{3-kov}) 
изображается деревом, узлы которого помечены обозначениями объектов. 
Важным достижением стандарта является выявление того факта, что одна и~та 
же система задается не одной, а несколькими в~общем случае различными 
иерархическими структурами, возникающими в~результате декомпозиции 
согласно различным принципам (аспектам). В~их числе:
   \begin{itemize}
\item функциональная (function-oriented) структура, отвечающая разделению 
системных единиц по выполняемым ими функциям в~составе сис\-темы;
\item продуктовая (product-oriented), или модульная, структура, отражающая 
сборочную (технологическую) конфигурацию сис\-темы;
\item структура размещения (location-oriented), в~соответствии с~которой 
единицы располагаются в~физическом пространстве.
\end{itemize}

   Ясно, что один и~тот же объект может входить в~несколько структур и~при 
этом находиться на различных уровнях. В~то же время в~некоторых аспектах 
объект может никак не проявлять себя и~вследствие этого отсутствовать 
в~соответствующих структурах. Полное идентифицирующее ссылочное 
обозначение объекта (reference designation) конструируется путем 
последовательного перечисления всех объектов, находящихся на пути от корня 
дерева рассматриваемой структуры до дан\-ного объекта включительно. 
Наименование каж\-до\-го объекта в~этом перечислении составляется из 
символьного обозначения аспекта, буквенного обозначения класса (типа), 
к~которому относится  объект, и~порядкового номера объекта среди 
экземпляров своего класса. Таким путем обеспечивается\linebreak  уникальность 
наименования любой единицы\linebreak
 в~пределах системы. Например, функциональная 
структура обозначается символом <<=>>, а~функциональный класс 
переключателей потоков ресурсов обозначается буквами QA, так что первая по 
порядку единица, выполняющая функцию переключения, называется =QA1, 
а~ее полное ссылочное обозначение может выглядеть как =WP1=WC1=QA1. 
Если объект присутствует в~нескольких структурах, то он может иметь 
несколько ссылочных обозначений, как показано на рис.~1~\cite{4-kov}.

\begin{figure*} %fig1
    \vspace*{1pt}
\begin{center}
\mbox{%
\epsfxsize=165mm
\epsfbox{kov-1.eps}
}
\end{center}
\vspace*{-9pt}
\Caption{Пример ссылочных обозначений структурных единиц системы}
\vspace*{9pt}
\end{figure*}

   С~точки зрения практики системной инженерии большой интерес 
представляет описание эволюции структурного представления системы по ходу 
жизненного цикла, приведенное в~приложении~B к~стандарту IEC~81346. 
<<Строительный материал>> для структур имеет вид (виртуального) 
справочника или каталога объектов, из которого выбираются объекты для 
включения в~структуру. 

В~начале жизненного цикла системы на основе 
исходных требований к~ней конструктор строит ее функциональную структуру. 
Затем определяется пространственное положение функциональных объектов, 
в~результате чего создается структура размещения. На следующей стадии 
формируются закупочные спецификации, образующие продуктовую структуру. 
В~ходе последующих стадий жизненного цикла эти структуры могут 
трансформироваться. На каждой стадии могут происходить замена, слияние 
и~расщепление объектов. Таким образом, объекты разных структур системы 
связаны отношением вида <<многие ко многим>>, вдоль которого 
прослеживаются (трассируются) исходные требования.
   
   В то же время стандарт не предусматривает указа\-ние способов, какими 
объекты собраны в~сис\-те\-мы. Поэтому структуру сис\-те\-мы можно рас\-смат\-ри\-вать 
как эскизный проект, в~котором отражены лишь факты вхождения системных 
единиц более низкого уровня иерархии в~единицы более высокого уровня. 


Проект такого рода поступает на вход технологу, который определяет 
конкретные операции сборки каждой единицы каждого уровня иерархии. При 
необходимости технолог вносит изменения в~конструкцию объектов (такие как 
нарезка резьбы) и~добавляет связующие интерфейсные объекты (такие как 
клей, трансформатор и~др.). В~результате для каждого составного объекта 
формируется сборочный чертеж, на котором указаны все со\-став\-ля\-ющие 
объекты и~действия по их соединению в~целях получения сис\-те\-мы. 
Технологическая проработка требуется на всех стадиях жизненного цикла, на 
которых формируется либо изменяется ка\-кая-ли\-бо из структур системы.

%\vspace*{-6pt}

\section{Мегамоделирование и~сборка~систем}

   В MBSE объекты, образующие 
структуры\linebreak
 сис\-тем, описываются формализованными ком\-пьютерными моделями 
различных видов: геометрическими фигурами и~телами, численными 
аппроксимациями дифференциальных уравнений, оснащенными графами и~
т.\,д. При этом, как свидетель\-ст\-ву\-ют стандарты типа IEC~81346, для анализа 
структуры систем и~организации сборки необходимо знать не столько 
внутреннюю структуру моделей, сколько ассортимент их возможностей 
соединяться с~другими моделями в~целях формирования моделей составных 
объектов. Иными словами, модели рассматриваются как <<черные ящики>> 
с~известным поведением по отношению к~другим моделям. Каталог объектов, 
упоминавшийся в~предыду\-щем разделе, в~условиях применения \mbox{MBSE} 
составляется из моделей и~описаний действий по их соединению.
   
   Структуры систем и~сборочные чертежи представляют собой частные 
случаи мегамоделей (mega\-mod\-el)~--- моделей, состоящих из моделей и~связей 
между ними~\cite{8-kov}. Мегамодель, в~которой связи описывают соединение 
моделей, образующих некоторую сис\-те\-му, называется конфигурацией этой 
сис\-те\-мы~\cite{5-kov}. Существуют и~другие виды мегамоделей, 
предназначенные для описания других процедур \mbox{MBSE}, таких как 
формирование модели согласно заданной метамодели  
(instantiating)~\cite{9-kov}. Но в~настоящей работе сосредоточимся на 
конфигурациях и~сборке систем.
   
   Например, в~моделировании механических сис\-тем, состоящих из твердых 
тел, моделями деталей и~сборочных единиц служат геометрические тела, 
которые могут быть представлены для компьютерной обработки различными 
способами: конструктивным, воксельным, граничным~\cite{10-kov}. Объекты, 
составляющие механические системы, т.\,е.\ представления экземпляров тел, 
получаются из моделей путем аффинных изометрий и~растяжений. Так, из 
набора цилиндров разных размеров составляется модель штанги (спортивного 
снаряда). В~функциональной структуре штанги по IEC~81346 цилиндры 
представлены разными объектами, поскольку они выполняют разные функции, 
хотя порождаются одной и~той же геометрической моделью. Соответственно, 
в~каталоге моделей содержится тело в~форме цилиндра, допускающее 
несколько разных действий по включению в~состав штанги.
   
   В качестве еще одного примера рассмотрим дис\-крет\-но-со\-бы\-тий\-ное 
имитационное моделирование, поддержка которого относится к~числу 
важнейших достижений MBSE~\cite{1-kov}. Здесь модель имеет вид 
сценария~--- фрагмента предполагаемой истории поведения моделируемой 
системы, пред\-став\-лен\-но\-го потоком дискретных событий различных видов. 
Некоторые события могут вызывать либо запрещать возникновение других 
событий. Описания действий по сборке сценариев поведения систем отражают 
вклад сценариев поведения составляющих. Так, сценарий работы цеха 
составляется из сценариев работы станков, связанных друг с~другом согласно 
маршрутным картам~\cite{11-kov}.
   
   Сформулируем задачу мегамоделирования сборки систем в~общем виде 
следующим образом. По мегамодели, представляющей конфигура\-цию 
некоторой системы, требуется сконструировать модель системы как целого 
и~рассчитать для нее моделируемые параметры, в~том числе эмерджентные~--- 
не присущие никакой из со\-став\-ля\-ющих единиц в~отдельности. Принцип 
конструирования модели системы легко усмотреть из организации 
структур-\linebreak\vspace*{-12pt}

\columnbreak

 { \begin{center}  %fig1
 \vspace*{1pt}
\mbox{%
\epsfxsize=57.246mm
\epsfbox{kov-2.eps}
}


\vspace*{12pt}


\noindent
{{\figurename~2}\ \ \small{Схема склеивания}}
\end{center}
}

\vspace*{18pt}

\addtocounter{figure}{1}

\noindent
ного представления: система должна находиться на иерархическом 
уровне, располагающемся непосредственно над уровнем со\-став\-ля\-ющих ее 
объектов. Иными словами, модель системы должна включать в~себя модели 
всех составляющих с~учетом их конфигурационных связей и~в~то же время 
включаться в~любые модели, включающие в~себя модели всех составляющих 
конфигурации.
   
   Поясним этот принцип на простом примере. Предположим, что нужно 
объединить в~систему два объекта~$P$ и~$S$ и~что технолог решил сделать это 
с~по\-мощью клея~--- третьего объекта~$G$, который может быть соединен 
и~с~$P$, и~с~$S$. Действие клея описывается конфигурацией следующего 
вида: объекты~$G$ и~$P$ порождают в~результате соединения известный 
промежуточный комплексный объект~$P_G$, содержащий их, а~объекты~$G$ 
и~$S$ порождают объект~$S_G$. Система~$R$, полученная путем 
склеивания~$P$ с~$S$ при помощи~$G$, отбирается среди объектов, 
содержащих~$P_G$ и~$S_G$, по следующему структурному критерию: 
объект~$R$ должен содержаться в~любом объекте~$T$, содержащем~$P_G$ 
и~$S_G$. Схематически этот критерий изображен на рис.~2.


   Если объект $R$, удовлетворяющий указанному структурному критерию, 
существует, то он действительно отвечает системе, которая собрана из~$S$ 
и~$P$ путем склеивания посредством~$G$ (и~не содержит ничего 
<<лишнего>>). Более того, легко видеть, что такой объект~$R$ определяется, 
по существу, однозначно в~том смысле, что любые два объекта~$R$ 
и~$R^\prime$, удовлетворяющие структурному критерию, содержатся друг 
в~друге. Если же нужного объекта~$R$ не существует, то делается вывод, что 
технолог ошибся: клей~$G$ не способен соединить объекты~$P$ и~$S$.
   
   В структурное представление, выполненное по стандарту IEC~81346 либо по 
ГОСТу 2.053-2013, входят только объекты~$P$, $S$ и~$R$ и~две композитные 
стрелки: $P\hm\to R$, проходящая через~$P_G$, и~$S\hm\to R$, проходящая 
через~$S_G$ (так что мегамодель склеивания~--- это часть схемы, ограниченная 
треугольником~$PSR$). Кроме того, стрелки на схеме склеивания, в~отличие от 
структуры, представляют не просто факты включения объектов друг в~друга, 
а~конкретные действия по их соединению. При этом соблюдается следующее 
естественное условие структурной корректности: если из одного объекта 
можно прийти в~другой разными путями по схеме, то эти пути задают одно и~то 
же композитное действие. Например, клей~$G$ включается в~состав 
системы~$R$ единственным способом, несмотря на наличие двух путей $G 
\hm\to  P_G \hm\to R$ и~$G \hm\to S_G \hm\to R$: в~действительности не имеет 
значения, через какой промежуточный объект <<прослеживается>> включение 
клея в~систему. Таким образом, мегамодель сборки содержит больше 
информации, чем иерархическая структура системы.
   
   Если модели содержат значения тех или иных параметров, а описание 
действий по их соединению позволяет выявить правила преобразования 
значений, то по мегамодели сборки можно вы\-чис\-лить значения параметров для 
системы. Известны примеры вычислений такого рода в~области разработки 
новых композиционных материалов~\cite{12-kov}. Осредненные 
(эффективные) физические характеристики композитов, такие как модуль Юнга и~коэффициент Пуассона, сложным образом зависят от характеристик 
компонентов и~способов изготовления композита из них. При помощи методов 
теории упру\-гости эти зависимости задаются в~форме линеаризованных 
матричных соотношений, которые приписываются к~стрелкам мегамоделей, 
пред\-став\-ля\-ющим включение компонентов в~композиты. Появляется 
возможность рассчитывать на компьютере свойства композитов по базе данных 
компонентов, без проведения дорогостоящих физических экспериментов.
   
   В заключение раздела отметим, что хотя прямой расчет системы по 
конфигурации имеет большое значение, в~MBSE он играет вспомогательную 
роль. Согласно стандарту IEC~81346 и~практикам системной инженерии, 
система обычно проектируется сверху вниз~--- от корня структурной иерархии 
к~составляющим~\cite{13-kov}. Это означает, что технолог в~основном решает 
не прямую, а~обратную задачу: модель системы, которую нужно собрать, 
известна, а~нужно построить (восстановить) конфигурацию, из которой такая 
система может быть получена путем сборки, с~учетом различных ограничений. 
Формальные математические постановки и~методы решения обратных задач 
мегамоделирования представляют собой крупную перспективную тему 
исследований, выходящую за рамки настоящей статьи.

\section{Теория категорий в~мегамоделировании}

   Как указывалось во введении, естественным источни\-ком математических 
методов кон\-стру\-ирова\-ния и~анализа мегамоделей служит теория категорий. 
Категорией называется коллекция абстрактных объектов, попарно связанных 
морфизмами (стрелками). Точное определение занимает буквально несколько 
строк~\cite{14-kov}: категория~$C$ состоит из совокупности 
объектов~$\mathrm{Ob}\,C$ и~совокупности морфизмов~$\mathrm{Mor}\,C$, 
на которых заданы следующие операции:
\begin{enumerate}[(1)]
\item каждому морфизму~$f$ 
сопоставляется два объекта: область $\mathrm{dom}\,f$ и~кообласть 
$\mathrm{codom}\,f$ (соотношения вида $\mathrm{dom}\,f \hm= A$ и~
$\mathrm{codom}\,f \hm= B$ наглядно записываются в~форме стрелки~$f$: 
$A\hm\to B$, а множество всех морфизмов, удовлетворяющих этим 
соотношениям, обозначается через $\mathrm{Mor}(A, B))$;
\item для 
любой пары морфизмов~$f, g$, удовлетворяющей условию 
$\mathrm{codom}\,f\hm = \mathrm{dom}\,g$, определена композиция~--- 
морфизм $g \circ f : \mathrm{dom}\,f \hm\to  \mathrm{codom}\,g$, причем она 
ассоциативна: для любой тройки морфизмов~$f, g, h$, удовлетворяющей 
условиям $\mathrm{codom}\,f \hm= \mathrm{dom}\,g$ и~$\mathrm{codom}\,g 
\hm= \mathrm{dom}\,h$, выполняется соотношение $h \circ (g \circ f) \hm= (h 
\circ g) \circ f$;
\item любой объект~$A$ обладает тождественным 
морфизмом~$1_A : A \to A$ таким, что для любого морфизма~$f : A\hm\to B$ 
выполняется соотношение $f \circ 1_A \hm= 1_B \circ  f \hm= f$.
\end{enumerate}

Классическим 
примером категории служит $\mathbf{Set}$, состоящая из всех множеств и~всех 
их отображений: закон композиции отображений задается стандартной 
подстановкой, а тождественным морфизмом произвольного множества служит 
его тождественное отображение на себя.
   
   Вместе с~категорией вводится понятие функтора~--- отображения категорий, 
сохраняющего структуру. Функтор $\mathrm{fun}\,: C \hm\to D$, действующий из 
категории~$C$ в~$D$,~--- это пара одноименных отображений $\mathrm{fun}\,: 
\mathrm{Ob}\,C \hm\to \mathrm{Ob}\,D$, $\mathrm{fun}\,: \mathrm{Mor}\,C \hm\to 
\mathrm{Mor}\,D$, удовлетворяющая следующим условиям (для произвольных 
$C$-мор\-физ\-мов~$f, g$ и~$C$-объ\-ек\-та~$A$): 
\begin{enumerate}[(1)]
\item $\mathrm{fun}\,(\mathrm{dom}\,f) 
\hm= \mathrm{dom}\,\mathrm{fun}\,(f), \mathrm{fun}\,(\mathrm{codom}\,f)\hm = 
\mathrm{codom}\,\mathrm{fun}\,(f)$;  
\item $\mathrm{fun}\,(g \circ f) = \mathrm{fun}\,(g) \circ \mathrm{fun}\,(f)$, 
если композиция $g \circ f$ определена; 
\item $\mathrm{fun}\,(1_A) \hm= 1_{\mathrm{fun}\,(A)}$.
\end{enumerate}
 Все категории и~все функторы образуют 
(формальную) категорию~$\mathbf{CAT}$. Чтобы исследовать взаимосвязь 
между функторами, вводится следующее понятие: естественным 
преобразованием~$\varepsilon$ функтора $\mathrm{fun}\, : C\hm\to D$ в~$\mathrm{fun}^\prime\, : C 
\hm\to D$ называется любое семейство $D$-мор\-физ\-мов~$\varepsilon_A : 
\mathrm{fun}\,(A) \hm\to \mathrm{fun}^\prime (A)$, $A \hm\in \mathrm{Ob}\,C$, 
такое что для любого 
\mbox{$C$-мор}\-физ\-ма $f : A\hm\to B$ выполняется соотношение $\varepsilon_B \circ 
\mathrm{fun}\,(f) \hm= \mathrm{fun}^\prime(f) \circ \varepsilon_A$:

%\begin{figure*} %рис
\vspace*{1pt}
\begin{center}
\mbox{%
\epsfxsize=54.473mm
\epsfbox{kov-3.eps}
}
\end{center}
%\vspace*{-9pt}
%\end{figure*}

   Эффективность применения теории категорий в~качестве математического 
аппарата \mbox{MBSE} обуслов\-ле\-на тем, что любой каталог моделей представляет 
собой не что иное, как категорию. Действительно, любая цепочка действий по 
соединению моделей порождает композитное действие (процесс) и, кроме того, 
любая модель допускает пустое действие над самой собою, не 
подразумевающее никаких изменений (процедура <<ничегонеделания>>). 
Например, в~твердотельном моделировании механических систем объектами 
категории\linebreak моделей выступают тела~--- подмножества в~$\mathbb{R}^3$, 
которые являются ограниченными, регулярными\linebreak
 (совпадают с~замыканием 
своей внутренности) и~полуаналитическими (допускают представление 
конечными булевыми комбинациями множеств вида $\{(x, y, z) \vert  F_i(x, y, 
z)\hm\leq 0\}$, где~$F_i : \mathbb{R}^3\hm\to \mathbb{R}$ является 
вещественной аналитической функцией для всех~$i$)~\cite{10-kov}. Чтобы 
было возможно задавать процедуры типа склеивания участков поверхности тел, в~категорию геометрических моделей добавляются ограниченные регулярные 
полуаналитические подмножества в~$\mathbb{R}^n$, $0 \hm\leq n \hm\leq 2$, 
при помощи стандартного вложения~$\mathbb{R}^n$ в~$\mathbb{R}^3$. Далее 
выполняется факторизация: отождествляются друг с~другом все множества, 
переходящие друг в~друга под действием аффинных изометрий. Морфизмы 
таких классов эквивалентности, описывающие действия по сборке составных 
механических сис\-тем, порождаются изометрическими вложениями множеств 
и~растяжениями. Получается подкатегория в~\textbf{Set}, которую будем обозначать 
через $\mathbf{MBS}$ (от Multibody Systems).
   
   Для многих известных технологий MBSE формальное описание каталогов 
поддерживаемых моделей приводит к~категориям множеств со структурой~--- 
алгебраических систем, топологических пространств, графов и~т.\,д. 
Морфизмами в~таких категориях служат отображения множеств, со\-вмес\-ти\-мые 
со структурой. На любой такой категории действует канонический функтор 
в~$\mathbf{Set}$, <<забывающий>> структуру. 

В~качестве примера приведем  
дис\-крет\-но-со\-бы\-тий\-ное моделирование, в~котором математической 
моделью сценария служит множество событий, час-\linebreak тич\-но упорядоченное  
при\-чин\-но-след\-ст\-вен\-ны\-ми зависимостями и~размеченное видами 
событий~\cite{15-kov}. Действия по сборке сложных сценариев задаются 
монотонными отображениями, сохраняющими разметку, поскольку ни 
события, ни зависимости, ни метки не могут быть <<потеряны>> при 
соединении сценариев поведения компонентов в~сценарии поведения систем. 
Получается категория~$\mathbf{Pomset}$, состоящая из всех помеченных 
частично упорядоченных множеств и~всех их монотонных отображений, 
сохраняющих разметку. Имеется функтор $\vert \mbox{--} \vert : 
\mathbf{Pomset}\hm\to \mathbf{Set} : S \mapsto \vert S\vert$, <<забывающий>> 
порядок и~разметку.
   
   Зафиксируем произвольную категорию~$C$, представляющую некоторый 
каталог моделей. Как и~для любой алгебраической системы, определена 
конструкция подкатегории в~$C$~--- это пара, состоящая из подкласса 
в~$\mathrm{Ob}\,C$ и~подкласса в~$\mathrm{Mor}\,C$, замкнутых 
относительно унаследованных из~$C$ операций. Подкатегория в~$C$ 
называется полной, если любой \mbox{$C$-мор}\-физм, область и~кообласть которого 
содержатся в~ней, сам содержится в~ней. Например, подкатегориями 
описываются различные аспекты структурного представления систем согласно 
стандарту IEC~81346. Действительно, композиция двух морфизмов, 
представляющих действия по формированию некоторого аспекта структуры, 
также должна входить в~этот аспект, поскольку стандарт предписывает строить 
цепочки для идентификации объектов в~структуре системы. Кроме того, если 
объект присутствует в~аспекте, то его тождественный морфизм формально 
должен быть включен в~этот аспект. В~то же время подкатегории, 
опи\-сы\-ва\-ющие все аспекты, не обязаны образовывать в~совокупности разбиение 
категории~$C$: как показывает рис.~1, возможны как действия, входящие 
в~несколько аспектов одновременно, так и~композитные действия с~переходом 
между структурами, не входящие ни в~один аспект. Требуется лишь, чтобы 
объединение классов объектов всех этих подкатегорий совпадало 
с~$\mathrm{Ob}\,C$, поскольку не имеет смысла вводить модели, не входящие 
ни в~одну структуру.
   
   Категории можно получать из графов: любой ориентированный мультиграф 
порождает категорию, объектами в~которой служат все узлы, а морфизмами~--- 
все пути. Областью и~кообластью морфизма являются соответственно начало 
и~конец пути, композиция морфизмов действует как конкатенация путей, 
а~тождественным морфизмом узла~$a$ является пустой путь из~$a$ в~$a$, не 
содержащий ни одного ребра. Отсюда получается фундаментальное понятие  
$C$-диа\-грам\-мы~--- это функтор вида~$\Delta : X \hm\to C$, где~$X$~--- 
категория, порожденная некоторым графом и~называемая схемой диаграммы. 
Все $C$-диа\-грам\-мы образуют категорию~$\mathbf{D}C$ (ковариантная 
категория <<сверхзапятой>>~\cite{14-kov}), в~которой морфизмом 
диаграммы~$\Delta : X \hm\to C$ в~$\Xi : Y \hm\to C$ служит любая пара 
вида $\langle\gamma, fd\rangle$, состоящая из функтора~$fd : X\hm\to Y$ 
и~естественного преобразования~$\gamma : \Delta\hm\to \Xi \circ fd$; закон 
композиции морфизмов диаграмм имеет вид:
$$
\langle \gamma, fd\rangle \circ 
\langle \varphi, gd\rangle \hm = \langle \gamma_{gd(-)} \circ \varphi, fd \circ 
gd\rangle\,.
$$ 
В~тео\-рии категорий накоплен богатый арсенал алгебраических 
методов конструирования и~анализа диаграмм.
   
   Любая мегамодель задается $C$-диа\-грам\-мой, так что категорное 
представление каталогов моделей позволяет формально решать задачи 
мегамоделирования. Морфизмы диаграмм описывают структурные 
преобразования мегамоделей, выполняемые при помощи инструментов MBSE. 
Покажем, как решаются средствами теории категорий прямые задачи 
мегамоделирования. Здесь применяется одна из основных  
тео\-ре\-ти\-ко-ка\-те\-гор\-ных конструкций~--- копредел  
диаграммы~\cite{5-kov}, который строится следующим образом. Обозначим 
через~$\mathbf{1}$ категорию,\linebreak состоящую из одного объекта~0 и~одного 
морфизма~$1_0$. Из любой категории~$X$ имеется в~точ\-ности один 
функтор~$!_X : X \hm\to \mathbf{1}$, сопоставляющий объект~0  
любому~$X$-объ\-ек\-ту (иными словами, $\mathbf{1}$ является терминальным 
$\mathbf{CAT}$-объ\-ек\-том). Имеется вложение (инъективный функтор) 
$\ulcorner \mbox{--}\urcorner : C \hookrightarrow \mathbf{D}C$, сопоставляющее 
произвольному $C$-объ\-ек\-ту $Q$~точку~--- диаграмму $\ulcorner Q\urcorner : 
\mathbf{1}\hm\to  C : 0 \mapsto Q$. Коконусом (cocone) называется 
$\mathbf{D}C$-мор\-физм, имеющий точку в~качестве кообласти. Можно 
изобразить коконус $\langle \sigma, !_X\rangle : \Delta\hm\to \ulcorner 
Q\urcorner$ над диаграммой $\Delta : X\hm\to C$ в~виде диаграммы, 
<<пририсовав>> к~$\Delta$ дополнительную вершину, помеченную 
объектом~$Q$, и~набор ребер~--- стрелок, по одной для каждого узла $I\hm\in 
\mathrm{Ob}\,X$, направленной из~$I$ в~вершину и~помеченной морфизмом 
$\sigma_I : \Delta (I) \hm\to Q$. Копределом (colimit) диаграммы~$\Delta$ 
называется коконус $\mathrm{colim}\,\Delta : \Delta\hm\to \ulcorner R\urcorner$, 
универсальный в~том смысле, что для любых \mbox{$C$-объ}\-ек\-та~$T$ 
и~коконуса~$\delta : \Delta\hm\to\ulcorner T\urcorner$ существует единственный 
$C$-мор\-физм~$w : R \hm\to T$ такой, что $\delta\hm= \ulcorner w\urcorner \circ  
\mathrm{colim}\,\Delta$. Легко видеть, что это условие универсальности 
представляет собой в~точности структурный критерий из разд.~3. Таким 
образом, конструирование копредела конфигурации~$\Delta$ описывает на 
строгом математическом языке сборку системы, которой отвечает 
вершина~$R$. В~категориях типа $\mathbf{MBS}$ и~$\mathbf{Pomset}$ 
построение копредела сводится к~факторизации раздельных объединений 
объектов, представляющих компоненты системы, по отношениям 
эквивалентности, индуцированным моделями клея и~других средств сборки.
   
   Копредел любой диаграммы, если он сущест\-вует, определяется однозначно 
   с~точностью до изомор\-физма. Более того, можно описать сборку сис\-тем из 
конфигураций в~виде функтора. Пусть $Cd$~--- некоторый класс  
$C$-диа\-грамм, имеющих копределы. Он порождает полную подкатегорию 
в~$\mathbf{D}C$, из которой в~$C$ действует функтор копредела $\mathrm{colim}$, 
сопоставляя каждой диаграмме из~$Cd$~вершину некоторого ее копредела, а 
каждому \mbox{$\mathbf{D}C$-мор}\-физ\-му~$\theta : \Delta\hm\to \Xi$, 
где~$\Delta, \Xi\hm\in Cd$~--- стрелку копредела $\mathrm{colim}\,(\theta)$ такую, что 
$\mathrm{colim}\,\Xi \circ \theta \hm= \ulcorner \mathrm{colim}\,(\theta)\urcorner \circ 
\mathrm{colim}\,\Delta$.

%\begin{figure*}
\vspace*{1pt}
\begin{center}
\mbox{%
\epsfxsize=56.127mm
\epsfbox{kov-4.eps}
}
\end{center}
%\vspace*{-9pt}
%\end{figure*}

   Например, в~категории \textbf{Set} любая диаграмма имеет 
копредел~\cite[упражнение~5.1.8]{14-kov}, поэтому имеется функтор $\mathrm{colim}\, : 
\mathbf{D}(\mathbf{Set})\hm\to \mathbf{Set}$. Примечательно, что этот функтор 
является рефлектором: он сопряжен слева с~вложением $\ulcorner \mbox{--}\urcorner : 
\mathbf{Set}\hookrightarrow \mathbf{D}(\mathbf{Set})$, причем 
единица рефлексии состоит из $\mathbf{D}(\mathbf{Set})$-мор\-физ\-мов 
$\mathrm{colim}\,\Delta : \Delta\hm\to \ulcorner\mathrm{colim}\,(\Delta)\urcorner$, 
$\Delta\hm\in \mathrm{Ob}\ \mathbf{D}(\mathbf{Set})$. Напомним, что единица 
рефлексии~--- это естественное преобразование тождественного функтора 
в~композицию рефлектора и~вложения (в~данном случае, естественное 
преобразование функтора $1_{\mathbf{D}(\mathbf{Set})}$ в~$\ulcorner \mathrm{colim}\,(  
\mbox{--})\urcorner)$, состоящее из универсальных  
стрелок~\cite[разд.~4.3]{14-kov}. И~для произвольного класса~$Cd$, 
содержащего достаточное количество одноточечных диаграмм, функтор 
$\mathrm{colim}$ сопряжен слева с~ограничением  
вложения~$\ulcorner \mbox{--}\urcorner$ на подходящую полную подкатегорию 
в~$C$. А~поскольку сопряженный функтор задается однозначно с~точностью 
до изоморфизма~\cite[разд.~4.1]{14-kov}, можно сделать вывод, что сборка 
систем в~некотором смысле <<зашифрована>> в~процедуре построения 
одноточечных диаграмм~--- моделей систем как целого без раскрытия 
струк\-туры. 

Так наглядно проявляется двойственность прямых и~обратных задач 
мегамоделирования.

\section{Заключение}

   Аппарат теории категорий обладает большим потенциалом в~области 
повышения полезной отдачи от MBSE, в~том числе путем математически 
строгого решения задач мегамоделирования. Так, базовая процедура системной 
инженерии~--- сборка\linebreak
 системы из заданной конфигурации взаимо\-свя\-занных 
компонентов~--- формально описывается тео\-ретико-ка\-те\-гор\-ной 
конструкцией копредела диа\-граммы. Более сложные конструкции отвечают\linebreak 
сложным процедурам сборки, таким как связывание (weaving) общесистемных 
функций, рассеянных по всем компонентам (crosscutting concerns), например 
мониторинговых или защитных~\cite{16-kov}. Математического представления 
требуют и~другие процедуры MBSE, в~частности коллективная модификация 
мегамоделей и~составляющих моделей, восстановление конфигурации заданной 
системы, оценка взаимозаменяемости компонентов. 

Актуальны вопросы 
внедрения аппарата теории категорий в~практику, в~том числе путем развития 
программных инструментов моделирования и~мегамоделирования. Здесь 
открывается широкий спектр направлений для дальнейших исследований.
   
{\small\frenchspacing
 {%\baselineskip=10.8pt
 \addcontentsline{toc}{section}{References}
 \begin{thebibliography}{99}
\bibitem{1-kov}
Modeling and simulation-based systems engineering handbook~/
Eds.\ D.~Gianni,  A.~D'Ambrogio, A.~Tolk.~--- London: CRC Press, 2014. 513~p.
\bibitem{2-kov}
\Au{Ковалёв С.\,П., Толок~А.\,В.} Применение модельно-ори\-ен\-ти\-ро\-ван\-но\-го подхода 
в~управ\-ле\-нии жизненным циклом технических изделий~// Информационные технологии 
в~проектировании и~производстве, 2015. №\,2. С.~3--9.
\bibitem{3-kov}
\Au{Левенчук А.\,И.} Системноинженерное мышление.~--- М.: TechInvestLab, 2015. 305~с.
\bibitem{4-kov}
IEC 81346-1:2009. Industrial Systems, Installations and Equipment and Industrial Products~--- 
Structuring Principles and Reference Designations~--- Part~1: Basic Rules.~--- Geneva: ISO, 2009. 
168~p.
\bibitem{5-kov}
\Au{Ginali S., Goguen~J.} A~categorical approach to general systems~// 
 Conference (International) on Applied General Systems 
Research Proceedings~/
Ed. G.\,J.~Klir.~--- NATO conference series.~--- New York, NY, USA: Plenum 
Press, 1978. Vol.~5. P.~257--270.
\bibitem{6-kov}
\Au{Mabrok M.\,A., Ryan M.\,J.} Category theory as a~formal mathematical foundation for  
model-based systems engineering~// Appl. Math. Inform. Sci., 2017. Vol.~11. No.\,1. P.~43--51.
\bibitem{7-kov}
\Au{Ковалёв С.\,П.} Тео\-ре\-ти\-ко-ка\-те\-гор\-ный подход к~проектированию программных 
сис\-тем~// Фундаментальная и~прикладная математика, 2014. Т.~19. Вып.~3. С.~111--170.
\bibitem{8-kov}
\Au{B$\acute{\mbox{e}}$zivin J., Jouault~F., Rosenthal~P., Valduriez~P.} Modeling in the large 
and modeling in the small~// Model Driven Architecture: European MDA Workshops on 
Foundations and Applications Proceedings~/
Eds.\ U.~A{\!\ptb{\ss}}mann, M.~Aksit,  A.~Rensink.~--- 
Lecture notes in computer science ser.~--- Springer, 2005. Vol.~3599. 
P.~33--46.
\bibitem{9-kov}
\Au{Diskin Z., Kokaly~S., Maibaum~T.} Mapping-aware mega\-mod\-eling: Design patterns and 
laws~// Software Language Engineering: 6th Conference (International) Proceedings~/
Eds.\ M.~Erwig, R.\,F.~Paige, E.~Van Wyk.~--- 
Lecture notes  in computer science ser.~--- Springer, 2013. Vol.~8225. P.~322--343.
\bibitem{10-kov}
\Au{Requicha A.\,G.} Representations for rigid solids: Theory, methods, and systems~// 
ACM  Comput. Surv., 1980. Vol.~12. Iss.~4. P.~437--464.
\bibitem{11-kov}
\Au{K$\acute{\mbox{a}}$d$\acute{\mbox{a}}$r B., Pfeiffer~A., Monostori~L.} Discrete event 
simulation for supporting production planning and scheduling decisions in digital
 factories~//  37th 
CIRP Seminar (International) on Manufacturing Systems Proceedings.~--- Budapest, 2004.  
P.~444--448.
\bibitem{12-kov}
\Au{Giesa T., Spivak D.\,I., Buehler~M.\,J.} Category theory based solution for the building block 
replacement problem in materials design~// Adv. Eng. Mater., 2012. Vol.~14. 
Iss.~9. P.~810--817.
\bibitem{13-kov}
\Au{Косяков А., Свит У., Сеймур~С., Бимер~С.} Системная инженерия. Принципы 
и~практика~/ Пер. с~англ.~--- М.: ДМК-Пресс, 2014. 636~с. (\Au{Kossiakoff~A., Sweet~W.\,N., 
Seymour~S., Biemer~S.\,M.} Systems engineering principles and practice.~--- 2nd ed.~--- New 
York, NY, USA: John Wiley, 2011. 560~p.)
\bibitem{14-kov}
\Au{Маклейн С.} Категории для работающего математика~/ Пер. с~англ.~--- М.: Физматлит, 
2004. 352~с. (\Au{Mac Lane~S.} Categories for the working mathematician.~--- New York, NY, 
USA: Springer, 1978. 317~p.)
\bibitem{15-kov}
\Au{Pratt V.\,R.} Modeling concurrency with partial orders~// Int. J.~Parallel 
Prog., 1986. Vol.~15. No.\,1. P.~33--71.
\bibitem{16-kov}
\Au{Ковалёв С.\,П.} Семантика ас\-пект\-но-ори\-ен\-ти\-ро\-ван\-но\-го моделирования 
данных и~процессов~// Информатика и~её применения, 2013. Т.~7. Вып.~3. С.~70--80.
 \end{thebibliography}

 }
 }

\end{multicols}

\vspace*{-3pt}

\hfill{\small\textit{Поступила в~редакцию 16.01.17}}

%\vspace*{8pt}

\newpage

\vspace*{-30pt}

%\hrule

%\vspace*{2pt}

%\hrule

%\vspace*{8pt}


\def\tit{METHODS OF CATEGORY THEORY IN~MODEL-BASED SYSTEMS ENGINEERING\\[-7pt]}

\def\titkol{Methods of category theory in~model-based systems engineering}

\def\aut{S.\,P.~Kovalyov\\[-12pt]}

\def\autkol{S.\,P.~Kovalyov}

\titel{\tit}{\aut}{\autkol}{\titkol}

\vspace*{-14pt}


\noindent
Institute of Control Sciences, Russian Academy of Sciences, 65~Profsoyuznaya Str., 
Moscow 117997, Russian Federation



\def\leftfootline{\small{\textbf{\thepage}
\hfill INFORMATIKA I EE PRIMENENIYA~--- INFORMATICS AND
APPLICATIONS\ \ \ 2017\ \ \ volume~11\ \ \ issue\ 3}
}%
 \def\rightfootline{\small{INFORMATIKA I EE PRIMENENIYA~---
INFORMATICS AND APPLICATIONS\ \ \ 2017\ \ \ volume~11\ \ \ issue\ 3
\hfill \textbf{\thepage}}}

\vspace*{1pt}

 

\Abste{A mathematical device based on the category theory is proposed to formally describe and 
rigorously explore procedures of employing models in engineering that constitute the contents of 
model-based systems engineering (MBSE). The essence of the device consists in mathematical 
representation of assembly drawings (megamodels of systems) as diagrams in categories whose 
objects are models, and morphisms represent actions associated with assembling system models 
from component models. The soundness of the device is justified on the basis of standards that 
govern description of the systems' structure such as IEC~81346. Category-theoretical methods for 
solving a number of practical problems of assembling systems are proposed and explored. 
Examples of solving such problems are provided in categories that represent two key application 
areas for MBSE: geometric modeling of complex shapes and discrete-event simulation of the 
behavior of industrial systems.}

\KWE{ model-based systems engineering; megamodel; category theory; colimit}

\DOI{10.14357/19922264170305} 

%\vspace*{-18pt}

%\Ack
%\noindent




\vspace*{-7pt}

  \begin{multicols}{2}

\renewcommand{\bibname}{\protect\rmfamily References}
%\renewcommand{\bibname}{\large\protect\rm References}

{\small\frenchspacing
 {%\baselineskip=10.8pt
 \addcontentsline{toc}{section}{References}
 \begin{thebibliography}{99}
\bibitem{1-kov-1}
Gianni, D., A.~D'Ambrogio, and A.~Tolk, eds. 2014. \textit{Modeling and simulation-based 
systems engineering handbook}. London: CRC Press. 513~p.
\bibitem{2-kov-1}
\Aue{Kovalyov, S.\,P., and A.\,V.~Tolok.} 2015. Primenenie model'no-orientirovannogo podkhoda 
v~upravlenii zhiznennym tsiklom tekhnicheskikh izdeliy [Applying model-based approach 
to product lifecycle management].\linebreak \textit{Informatsionnye tekhnologii v~proektirovanii 
i~proizvod\-st\-ve} [Information Technologies in Design and Industry] 2(158):3--9.
\bibitem{3-kov-1}
\Aue{Levenchuk A.\,I.} 2015. 
\textit{Sistemnoinzhenernoe myshlenie} [Systems engineering thinking]. 
Moscow: TechInvestLab. 305~p.
\bibitem{4-kov-1}
IEC 81346-1:2009. 2009. 
Industrial Systems, Installations and Equipment and Industrial 
Products~--- Structuring Principles and Reference Designations~--- 
Part~1: Basic Rules. Geneva:  ISO. 168~p.
\bibitem{5-kov-1}
\Aue{Ginali, S., and J.~Goguen.} 1978. 
A~categorical approach to general systems. \textit{Conference 
(International) on Applied General Systems Research Proceedings}. Ed.\
 G.\,J.~Klir. \mbox{NATO}  conference ser. Plenum Press. 5:257--270.
\bibitem{6-kov-1}
\Aue{Mabrok, M.\,A., and M.\,J.~Ryan}. 
2017. Category theory as a~formal mathematical foundation for 
model-based systems engineering. \textit{Appl. Math.  Inform. Sci.} 11(1):43--51.
\bibitem{7-kov-1}
\Aue{Kovalyov, S.\,P.} 2016. 
Category-theoretic approach to software systems design. \textit{J.~Math. Sci.} 
214(6):814--853.
\bibitem{8-kov-1}
\Aue{B$\acute{\mbox{e}}$zivin, J., F.~Jouault, P.~Rosenthal, and P.~Valduriez.}
 2005. Modeling in 
the large and modeling in the small. 
\textit{Model Driven Architecture: European MDA Workshops on 
Foundations and Applications Proceedings.} 
Eds.\ U.~\mbox{A{\!\ptb{\ss}}mann}, M.~Aksit, and A.~Rensink. 
Lecture notes in computer science ser. Springer. 3599:33--46.
\bibitem{9-kov-1}
\Aue{Diskin, Z., S.~Kokaly, and T.~Maibaum.} 2013. 
Mapping-aware megamodeling: Design patterns 
and laws. \textit{6th Conference (International) on Software Language Engineering 
Proceedings}. Eds.\ M.~Erwig, R.\,F.~Paige, and E.~Van Wyk. 
Lecture notes in computer science ser. Springer. 
8225:322--343.
\bibitem{10-kov-1}
\Aue{Requicha, A.\,G.} 1980. Representations for rigid solids: 
Theory, methods, and systems. \textit{ACM 
Comput. Surv.} 12(4):437--464.
\bibitem{11-kov-1}
\Aue{K$\acute{\mbox{a}}$d$\acute{\mbox{a}}$r,~B., A.~Pfeiffer, and L.~Monostori.}
2004. Discrete 
event simulation for supporting production planning and scheduling decisions in 
digital factories. \textit{37th CIRP Seminar (International) on Manufacturing 
Systems Proceedings}. Budapest.  444--448.
\bibitem{12-kov-1}
\Aue{Giesa, T., D.\,I.~Spivak, and M.\,J.~Buehler.} 2012. 
Category theory based solution for the building 
block replacement problem in materials design. 
\textit{Adv. Eng. Mater.} 14(9):810--817.
\bibitem{13-kov-1}
\Aue{Kossiakoff, A., W.\,N.~Sweet, S.~Seymour, and S.\,M.~Bie\-mer.}
2011. \textit{Systems engineering 
principles and practice}. 2nd ed. New York, NY: John Wiley. 560~p.
\bibitem{14-kov-1}
\Aue{Mac Lane, S.} 1978. \textit{Categories for the working mathematician}. 
New York, NY: Springer. 317~p.
\bibitem{15-kov-1}
\Aue{Pratt, V.\,R.} 1986. Modeling concurrency with partial orders. 
\textit{Int. J.~Parallel Prog.} 15(1):33--71.
\bibitem{16-kov-1}
\Aue{Kovalyov, S.\,P.} 2013. 
Semantika aspektno-ori\-en\-ti\-ro\-van\-no\-go modelirovaniya dannykh 
i~protsessov [Semantics of aspect-oriented modeling of data and processes]. 
\textit{Informatika i~ee  Primeneniya~--- Inform. Appl.} 7(3):70--80.
\end{thebibliography}

 }
 }

\end{multicols}

\vspace*{-9pt}

\hfill{\small\textit{Received January 16, 2017}}

\vspace*{-18pt}

\Contrl

\noindent
\textbf{Kovalyov Sergey P.} (b.\ 1972)~--- Doctor of Science in physics and 
mathematics, leading scientist, Institute of Control Problems, Russian 
Academy of Sciences, 65~Profsoyuznaya Str., Moscow 117997, Russian 
Federation Federation; \mbox{kovalyov@nm.ru} 

\label{end\stat}


\renewcommand{\bibname}{\protect\rm Литература}    %5
\renewcommand{\figurename}{\protect\bf Figure}
\renewcommand{\tablename}{\protect\bf Table}

\def\stat{lange}


\def\tit{ON COMPARATIVE EFFICIENCY OF~CLASSIFICATION SCHEMES IN~AN~ENSEMBLE 
OF~DATA SOURCES USING AVERAGE MUTUAL INFORMATION}

\def\titkol{On comparative efficiency of~classification schemes in~an~ensemble 
of~data sources using average mutual information}

\def\autkol{M.\,M.~Lange}

\def\aut{M.\,M.~Lange$^1$}

\titel{\tit}{\aut}{\autkol}{\titkol}

%{\renewcommand{\thefootnote}{\fnsymbol{footnote}}
%\footnotetext[1] {The study was carried out under state order to the Karelian Research 
%Centre of the Russian Academy of Sciences (Institute of Applied Mathematical 
%Research KarRC RAS) and supported by the Russian Foundation for Basic Research, 
%projects 18-07-00187, 18-07-00147, 18-07-00156, 19-07-00303.}}

\renewcommand{\thefootnote}{\arabic{footnote}}
\footnotetext[1]{Federal Research Center ``Computer Science and Control'' of the Russian Academy of Sciences, 
44-2~Vavilov Str., Moscow 119333, Russian Federation; \mbox{lange\_mm@ccas.ru}}


\index{Lange M.\,M.}
\index{Ланге M.\,M.}


\def\leftfootline{\small{\textbf{\thepage}
\hfill INFORMATIKA I EE PRIMENENIYA~--- INFORMATICS AND
APPLICATIONS\ \ \ 2019\ \ \ volume~13\ \ \ issue\ 4}
}%
 \def\rightfootline{\small{INFORMATIKA I EE PRIMENENIYA~---
INFORMATICS AND APPLICATIONS\ \ \ 2019\ \ \ volume~13\ \ \ issue\ 4
\hfill \textbf{\thepage}}}

%\vspace*{-2pt}





%The research is partially supported by the Russian Foundation for Basic Research 
%(grants Nos.\,18-07-01231 and 18-07-01385).




\Abste{Given ensemble of data sources and different fusion schemes, an accuracy of multiclass 
classification of the collections of the source objects is investigated. Using the average mutual 
information between the datasets of the sources and a~set of the classes, a~new approach to 
comparing lower bounds to an error probability in two fusion schemes is developed. The authors 
consider the WMV (Weighted Majority Vote) scheme which uses a~composition of the class 
decisions on the objects of the individual sources and the GDM (General Dissimilarity Measure) 
scheme based on a~composition of metrics in datasets of the sources.  For the above fusion 
schemes, the mean values of the average mutual information per one source are estimated. It is 
proved that the mean in the WMV scheme is less than the similar mean in the GDM scheme. As a~corollary, the lower bound to the error probability in the WMV scheme exceeds the similar 
bound to the error probability in the GDM scheme. This theoretical result is confirmed by 
experimental error rates in face recognition of HSI color images that yield the 
ensemble of H, S, and~I sources.} 

\KWE{multiclass classification; ensemble of sources; fusion scheme; composition of decisions; 
composition of metrics; average mutual information; error probability
}


\DOI{10.14357/19922264190403} 


%\vspace*{8pt}


\vskip 12pt plus 9pt minus 6pt

 \thispagestyle{myheadings}

 \begin{multicols}{2}

 \label{st\stat}

\section{Introduction }

\noindent
There are plenty of multiclass classification schemes that use input data from an 
ensemble of different modality sources. Such ensemble of data  sources produces 
the composite objects as the collections of the same class objects taken by one per 
each source. An example is the ensemble of biometric images such as faces, finger-
prints, signatures, palms, irises, and the like for a~given set of persons or classes. In 
this case, the composite objects are the collections of the same person images taken 
by one per each modality. In any correct classification scheme that makes the 
decisions on the submitted composite objects, an error probability decreases with 
increasing a~number of the sources~[1]. The decisions can be obtained using the 
different fusion schemes and the principal question is: What scheme is better? 

      The classification problem in the ensemble of sources is similar to the source 
coding problem based on quantization~[2].  There are known scalar and vector 
quantization for the continuous values.  The scalar quantization is used for the
individual values while the vector quantization is used for blocks of the values. In 
both cases, the above quantization schemes yield the code vectors for the 
appropriate blocks of the continuous values. 
   
    It should be noted that the optimal vector quantization is constructed with 
covering a~multidimensional space of the values by general spheres whose shape is 
adjusted to a~given dissimilarity measure between any pair of blocks of the 
values~[3]. In scalar quantization, the same multidimensional space is covered by 
cubes whose edge size is an optimal quantization step for any dimension. Thus, the 
code vectors are represented by the centers of the above spheres or cubes. Since
 for the same volume the spheres are more compact than the cubes, the vector 
quantization yields a~smaller error with respect to the scalar quantization. 

Also, for classification in a~given ensemble of the sources, an error probability is 
waited to be smaller in a~scheme of joint classifying each composite object as 
compared to an error probability in the scheme of combining the decisions on the 
objects of the individual sources. The proposed paper is focused on both developing a~theoretical validity of this idea and supporting it by a~computing experiment.  
    
Two fusion schemes that use the different data compositions for 
making the class decisions on the composite objects in the ensemble of the sources
have been investigated. 
They are the traditional WMV scheme by weighted majority voting the decisions 
on the objects of the individual sources~[4] and the original GDM scheme by 
combining the sources with a~general dissimilarity measure between any pair of the 
composite objects~[5]. Notice that WMV scheme is based on a~composition of 
decisions on the objects of individual sources while GDM scheme uses 
a~composition of metrics in datasets of the sources. Thus, ideologically, WMV 
and GDM fusion schemes are similar to the above scalar and vector quantization.

The specified similarity allows one to expect a~smaller error probability in GDM 
scheme as against WMV scheme. Some limits on the majority vote accuracy have 
been obtained in~[6]. Intuitively, it is clear that the minimal error probability of 
any classifier should depend on the average mutual information~[7] between a~set 
of the source objects and a~set of the classes. Moreover, the more average mutual 
information, the less error probability can be attained. So, our goal is to introduce 
the mutual information-based characteristics for WMV and GDM fusion schemes 
and, using these characteristics, to show an advantage of GDM scheme as against 
WMV scheme in the error probability. 

\section{Formalization of~the~Problem}


\subsection{Basic definitions and classification schemes}

\noindent
Let $\Omega=\{\omega_1, \ldots ,\omega_c\}$, $c\hm\geq 2$, be a~set of classes 
of the prior probabilities ${\sf P}(\omega_i)> 0$: $\sum\nolimits^c_{i=1} 
{\sf P}(\omega_i)=1$, and $\mathbf{X}^M= \mathbf{X}_1\cdots \mathbf{X}_M$ be 
an ensemble of sources, where the set $\mathbf{X}_m =\{\mathbf{x}_m= 
(x_{m1}, \ldots , x_{mN_m})\}$, $m=1,\ldots, M$, of $N_m$-dimensional 
vectors gives the $m$th source objects. In the ensemble , the components of 
any vector $\mathbf{x}_m\in \mathbf{X}_m$ take real values in $(-\infty, \infty)$, 
and any composite object $\mathbf{x}^M=(\mathbf{x}_1, \ldots ,
\mathbf{x}_M)\in \mathbf{X}^M$ is produced by a~collection of the vectors by 
one per source belonging to the same class in~$\Omega$.

In each set~$\mathbf{X}_m$, $m=1,\ldots , M$, a~dissimilarity measure between 
any pair of the objects $\mathbf{x}_m\in \mathbf{X}_m$ and 
$\hat{\mathbf{x}}_m\in \mathbf{X}_m$ is defined by
\begin{equation}
d\left( \mathbf{x}_m, \hat{\mathbf{x}}_m\right) =\sum\limits_{n=1}^{N_m} 
\fr{(x_{mn}-\hat{x}_{mn})^2}{\sigma^2_{mn}}
\label{e1-l}
\end{equation}
where $0<\sigma^2_{mn} <\infty$, $n=1,\ldots , N_m$, are unknown parameters. 
Also, for any pair of the composite objects $\mathbf{x}^M\in \mathbf{X}^M$ and 
$\hat{\mathbf{x}}^M\in \mathbf{X}^M$, let us define a~general dissimilarity 
measure as a~weighted composition of the metrics of the form~(1) taken with the 
weights $W=\{w_m>0,\ m=1,\ldots , M\}$ as follows:
\begin{equation}
D\left( \mathbf{x}^M, \hat{\mathbf{x}}^M\right) =\sum\limits^M_{m=1} w_m
d\left( 
\mathbf{x}_m, \hat{\mathbf{x}}_m\right)\,.
\label{e2-l}
\end{equation}

Let
\begin{equation}
\left\{\mathbf{x}_{im},\ i=1,\ldots ,c \right\} \subset \mathbf{X}_m,\enskip 
m=1,\ldots ,M\,,
\label{e3-l}
\end{equation}
be the subsets of the source template objects that represent the classes by one 
object from~$\mathbf{X}_m$ per each class. The subsets~(3) produce the subset 
of the template composite objects 
\begin{equation}
\left\{ \mathbf{x}_i^M=\left( \mathbf{x}_{i1},\ldots , \mathbf{x}_{iM}\right),\ 
i=1,\ldots ,c\right\} \subset \mathbf{X}^M\,.
\label{e4-l}
\end{equation}
Using the dissimilarity measure~(1) and assuming a~compactness of the objects 
in~$\mathbf{X}_m$, $m=1,\ldots , M$, relative to the corresponding template 
objects in~(3), let us define class-conditional densities of the~$m$th source objects 
as follows:
\begin{equation}
p\left(\mathbf{x}_m\vert \omega_i\right) =\fr{e^{-d(\mathbf{x}_m, 
\mathbf{x}_{im})}} {\int\nolimits_{\mathbf{X}_m}\!\!\! e^{-d(\mathbf{x}_m, 
\mathbf{x}_{im})}d\mathbf{x}_m}\,,\enskip i=1, \ldots , c\,.\!\!
\label{e5-l}
\end{equation}
Also, assuming a~compactness of the composite objects in~$\mathbf{X}^M$ 
relative to the corresponding templates in~(\ref{e4-l}) and using the general 
dissimilarity measure~(2), let us define class-conditional densities of the composite 
objects by
\begin{multline}
p_W\left(\mathbf{x}^M\vert \omega_i\right) \fr{e^{-D\left(\mathbf{x}^M, 
\mathbf{x}_i^M\right)}} {\int\nolimits_{\mathbf{X}^M} e^{-D\left(\mathbf{x}^M, 
\mathbf{x}_i^M\right)} d\mathbf{x}^M}\\
{} = \prod\limits^M_{m=1} \fr{e^{-w_m 
d(\mathbf{x}_m, \mathbf{x}_{im})}} {e^{-w_m d(\mathbf{x}_m, 
\mathbf{x}_{im})} d\mathbf{x}_m}\,,\enskip i=1,\ldots , c\,.
\label{e6-l}
\end{multline}
Under the product in~(\ref{e6-l}), there are the weighted class-conditional 
densities 
\begin{multline}
p_{w_m}\left(\mathbf{x}_m\vert \omega_i\right) =\fr{e^{-w_m d(\mathbf{x}_m, 
\mathbf{x}_{im})}} {\int\nolimits_{\mathbf{X}_m} e^{-w_m d(\mathbf{x}_m, 
\mathbf{x}_{im})} d\mathbf{x}_m}\,,\\
 i=1,\ldots ,c\,,
\label{e7-l}
\end{multline}
that give the densities of the form~(\ref{e5-l}) when $w_m=1$. In terms of 
information theory,  the densities~(\ref{e7-l}) define the $m$th source observation 
channel between input set~$\Omega$ and the output set~$\mathbf{X}_m$ as well 
as the  densities~(\ref{e6-l}) yield the observation multichannel 
between~$\Omega$ and~$\mathbf{X}^M$.

\begin{figure*} %fig1
  \vspace*{1pt}
    \begin{center}  
  \mbox{%
 \epsfxsize=107.799mm 
 \epsfbox{lan-1.eps}
 }
\end{center}
\vspace*{-9pt}
\Caption{Schemes of WMV-based~(\textit{a}) and GDM-based~(\textit{b}) classifiers}
\end{figure*}

Let $g_i^d(\mathbf{x}_m)$, $i=1,\ldots , c$, be the discriminant functions that are 
defined in the sets~$\mathbf{X}_m$, $m=1,\ldots , M$, using the dissimilarity 
measure of the form~(1). Then, WMV-based  class label decision on a~composite 
object $\mathbf{x}^M\in \mathbf{X}^M$ is defined by  
\begin{equation}
j^{\mathrm{WMV}}\left(\mathbf{x}^M\right) =\mathrm{arg}\,\max\limits^c_{i=1} 
\sum\limits^M_{m=1} w_m g_i^d\left(\mathbf{x}_m\right)
\label{e8-l}
\end{equation}
where the discriminant functions are independent on the source weights. Similarly, 
using in the ensemble~$\mathbf{X}^M$ the discriminant 
functions~$g_i^D(\mathbf{x}^M)$, $i=1,\ldots , c$, that depend on the 
weights~$W$ of all sources, GDM-based class label decision on the same 
composite object $\mathbf{x}^M\in \mathbf{X}^M$ is the following:
\begin{equation}
j^{\mathrm{GDM}}\left(\mathbf{x}^M\right) =\mathrm{arg}\,\max^c_{i=1} 
g_i^D\left(\mathbf{x}^M\right)\,.
\label{e9-l}
\end{equation}

        The classification schemes by the decision rules~(\ref{e8-l}) and~(\ref{e9-l}) 
are shown in Fig.~1. Here, $\hat{\Omega}=\Omega$ provided that the decisions 
in~$\hat{\Omega}$ can be differed from the real classes in~$\Omega$. The 
appropriate class-conditional densities yield the observation multichannels in 
WMV and GDM fusion schemes, respectively.




\subsection{Information criterion of efficiency for~the~fusion schemes}

\noindent  
Given the prior distribution $\{ P(\omega_i),\ i=1,\ldots\linebreak \ldots ,c\}$ and the weighted 
class-conditional densities $\{ p_{w_m}(\mathbf{x}_m\vert\omega_i), i=1,\ldots 
,c\}$ of the form~(\ref{e7-l}),the average mutual information 
between~$\mathbf{X}_m$ and~$\Omega$ is defined according to~\cite{7-l} by 
\begin{equation}
I_{w_m}\left(\mathbf{X}_m;\Omega\right) =H_{w_m}\left(\mathbf{X}_m\right) 
-H_{w_m} \left(\mathbf{X}_m\vert\Omega\right)\,.
\label{e10-l}
\end{equation}
Here,
\begin{align*}
H_{w_m}\left(\mathbf{X}_m\right) &=-\int\limits_{\mathbf{X}_m} p_{w_m} 
\left(\mathbf{x}_m\right) \ln p_{w_m} 
\left(\mathbf{x}_m\right)\,d\mathbf{x}_m\,;\\
H_{w_m}\left(\mathbf{X}_m\vert \Omega\right) &\\
&\hspace*{-11mm}{}=-\sum\limits^c_{i=1} 
P\left(\omega_i\right) \int\limits_{ \mathbf{X}_m} 
p_{w_m}\left(\mathbf{x}_m\vert \omega_i\right) \ln 
\left(\mathbf{x}_m\vert\omega_i\right)\,d\mathbf{x}_m
\end{align*}
are the differential entropies, and 
$p_{w_m}(\mathbf{x}_m)\linebreak =\sum\nolimits^c_{i=1} P(\omega_i) 
p_{w_m}(\mathbf{x}_m\vert \omega_i)$ is the marginal density 
in~$\mathbf{X}_m$, $m=1,\ldots ,M$. Notice that the average mutual information 
in~(\ref{e10-l}) does not exceed the entropy $H(\Omega)=-\sum\nolimits^c_{i=1} 
P(\omega_i)\ln P(\omega_i)$ of the set of the classes. For $w_m=1$, there is valid 
$p_{w_m}(\mathbf{x}_m\vert \omega_i)=p(\mathbf{x}_m\vert \omega_i)$  that 
yields $I_{w_m}(\mathbf{X}_m;\Omega)=I(\mathbf{X}_m;\Omega)$. 

Taking the means of the values 
$I\left(\mathbf{X}_m; \Omega\right)$  and $I_{w_m}(\mathbf{X}_m;\Omega)$ 
over all $m=1,\ldots , 
M$, one obtains the efficiency characteristics for WMV-based decision~(\ref{e8-l}) 
and GDM-based decision~(\ref{e9-l}), respectively. These means are defined as 
follows: 
\begin{align}
\hspace*{-2mm}I^{\mathrm{WMV}}_{W\_\mathrm{mean}}\left(\mathbf{X}^M;\Omega\right) &= 
\sum\limits^M_{m=1} I\left(\mathbf{X}_m;\Omega\right) 
\fr{w_m}{\sum\nolimits^M_{m=1} w_m}\,;\!\!\label{e11-l}\\
\hspace*{-2mm}I^{\mathrm{GDM}}_{W\_\mathrm{mean}} \left(\mathbf{X}^M;\Omega\right) &= \fr{1}{M} 
\sum\limits^M_{m=1} I_{w_m} \left(\mathbf{X}_m;\Omega\right)\,.
\label{e12-l}
\end{align}
Our goal is to prove the inequality 
\begin{equation}
\max\limits_W I^{\mathrm{WMV}}_{W\_\mathrm{mean}} \left(\mathbf{X}^M;\Omega\right) \leq 
I^{\mathrm{GDM}}_{W^*\_\mathrm{mean}} \left(\mathbf{X}^M;\Omega\right)
\label{e13-l}
\end{equation}
where~$W^*$ is the set of the source weights providing the maximum in the left 
part. 

\begin{figure*} %fig2
 \vspace*{1pt}
    \begin{center}  
  \mbox{%
 \epsfxsize=154.826mm 
 \epsfbox{lan-2.eps}
 }
\end{center}
\vspace*{-9pt}
\Caption{Sketches of the lower bounds to the average mutual information as the function 
of the error probability in WMV and GDM fusion schemes}
\end{figure*}  


\subsection{Average mutual information and~classification error probability}

\noindent
The criterion of the form~(\ref{e13-l}) assumes a~dependence of the average mutual 
information~$I(\mathbf{X}^M;\hat{\Omega})$ between the 
ensemble~$\mathbf{X}^M$ and the set of the class decisions~$\hat{\Omega}$ on 
a~lower bound to the error probability~$\varepsilon$ in the schemes shown in 
Fig.~1. Given observation multichannel, such function has been defined 
in~\cite{8-l} as a~generalization of the rate-distortion function for the source 
coding model with a~noisy observation channel~\cite{9-l}.  According  
to~\cite{8-l}, this function is lower bounded by 
\begin{multline}
R_L(\varepsilon) =I\left(\mathbf{X}^M;\Omega\right) -h\left(\varepsilon-
\varepsilon_{\min} \right)\\
{} -\left( \varepsilon -\varepsilon_{\min} \right) \ln (c-
1)\,,\enskip \varepsilon_{\min}\leq \varepsilon \leq \varepsilon_{\max}\,.
\label{e14-l}
\end{multline}
Here, $h(z) = -z\ln z -(1-z) \ln (1-z)$; 
$R_L(\varepsilon_{\min})\linebreak =I(\mathbf{X}^M;\Omega)$; 
$R_L(\varepsilon_{\max})= 0$; and $I(\mathbf{X}^M;\Omega) 
=H(\mathbf{X}^M)\linebreak - H(\mathbf{X}^M\vert\Omega)$ is the average mutual 
information between the input and the output of the observation multichannel in 
Fig.~1. Function~(\ref{e14-l}) has the largest value 
$I(\mathbf{X}^M;\Omega)$ at the point $\varepsilon=\varepsilon_{\min}$ and 
decreases as $\varepsilon$~increases. It is not difficult to show that the minimal error 
probability~$\varepsilon_{\min}$ is lower estimated by the conditional entropy 
$H(\Omega\vert \mathbf{X}^M)$ and~$\varepsilon_{\min}$ tends to zero 
when $H(\Omega\vert \mathbf{X}^M)$  decreases by increasing the size~$M$ of 
the ensemble. Taking into account the symmetry of the average mutual information 
\begin{multline*}
I(\mathbf{X}^M;\Omega)=H(\mathbf{X}^M)-H(\mathbf{X}^M\vert\Omega)\\
{}= 
H(\Omega)-H(\Omega\vert \mathbf{X}^M),
\end{multline*}
 in case of $\varepsilon_{\min}\to0$,  
function~(\ref{e14-l}) yields the Shannon bound of the form  
$H(\Omega)-h(\varepsilon) -\varepsilon\ln (c-1)$~\cite{7-l}. 

In the bound~(\ref{e14-l}), the average mutual information 
$I(\mathbf{X}^M;\Omega)$ is calculated in the product 
$\Omega*\mathbf{X}^M$ using the prior probabilities of the classes and the 
class-conditional densities of the form~(\ref{e6-l}). According to Fig.~1,  
the class-conditional densities in GDM scheme depend on the source weights and, 
therefore, $I(\mathbf{X}^M;\Omega)=I_W^{\mathrm{GDM}}(\mathbf{X}^M;\Omega)$ is 
the function of~$W$. In WMV scheme, the corresponding average mutual 
information $I(\mathbf{X}^M;\Omega)=I^{\mathrm{WMV}}(\mathbf{X}^M;\Omega)$ is 
equal to $I_W^{\mathrm{GDM}}(\mathbf{X}^M;\Omega)$ taken with the weights 
$w_m=1$, $m=1,\ldots , M$. The values $I^{\mathrm{WMV}}(\mathbf{X}^M;\Omega)$ 
and $I_W^{\mathrm{GDM}}(\mathbf{X}^M;\Omega)$ correspond to the minimal error 
probabilities~$\varepsilon_{\min}^{\mathrm{WMV}}$ and~$\varepsilon_{\min}^{\mathrm{GDM}}$ 
in WMV and GDM fusion schemes, respectively.  These error probabilities are 
achieved by the Bayes decisions of the form~(\ref{e9-l}) when the discriminant 
functions are given by the posterior probabilities of the classes~\cite{10-l}.


In general, the source sets $\mathbf{X}_1,\ldots, \mathbf{X}_M$ are statistically 
dependent on each other and there are valid the relations 
\begin{gather*}
I^{\mathrm{WMV}}_{W\_\mathrm{mean}} 
(\mathbf{X}^M;\Omega) < I^{\mathrm{WMV}}(\mathbf{X}^M;\Omega);\\[6pt]
I_{W\_\mathrm{mean}}^{\mathrm{GDM}} (\mathbf{X}^M;\Omega) 
< I^{\mathrm{GDM}}_{W} 
(\mathbf{X}^M;\Omega).
\end{gather*}

 Thus, for the weights~$W^*$ giving the maximum 
in~(\ref{e13-l}), the means $I^{\mathrm{WMV}}_{W^*\_\mathrm{mean}} 
(\mathbf{X}^M;\Omega)$ 
and $I^{\mathrm{GDM}}_{W^*\_\mathrm{mean}}(\mathbf{X}^M;\Omega)$ yield the error 
probabilities $\varepsilon^{\mathrm{WMV}}\linebreak
>\varepsilon^{\mathrm{WMV}}_{\min}$ and 
$\varepsilon^{\mathrm{GDM}}>\varepsilon^{\mathrm{GDM}}_{\min}$ that belong to the 
corresponding lower bounds of the form~(\ref{e14-l}). Also, taking into account 
that $I^{\mathrm{WMV}}(\mathbf{X}^M;\Omega)\leq 
I^{\mathrm{GDM}}_{W^*}(\mathbf{X}^M;\Omega)$, the inequality~(\ref{e13-l}) 
provides the following relation: $\varepsilon^{\mathrm{WMV}}
\geq \varepsilon^{\mathrm{GDM}}$. 
This fact is illustrated  in Fig.~2. 

\section{Calculation of~the~Average Mutual Information}

\noindent
In this section, an upper estimate of the functional 
$I_{w_m}(\mathbf{X}_m;\Omega)$ given in~(\ref{e10-l}) is obtained as 
a~function of the variable~$w_m^{1/2}$.  At the value $w q_m^{1/2}=1$, this 
function yields the upper estimate for~$I(\mathbf{X}_m;\Omega)$. Using the 
marginal density $p_{w_m}(\mathbf{x}_m)$  and taking into account that $-\ln z$ 
is the convex downwards function of~$z$, it is valid the Jensen 
inequality~\cite{11-l} as follows:
\begin{multline*}
-\ln p_{w_m} \left(\mathbf{x}_m\right) = -\ln \sum\limits^c_{i=1} P(\omega_i) 
p_{w_m} \left(\mathbf{x}_m\vert   \omega_i\right)\\
{} \leq -\sum\limits^c_{i=1} 
P(\omega_i) \ln p_{w_m}\left(\mathbf{x}_m\vert \omega_i\right)\,.
\end{multline*}
Applying this inequality in~(\ref{e10-l}), one obtains the upper estimated 
differential entropy:
\begin{multline}
H_{w_m}\left(\mathbf{X}_m\right) \leq -\sum\limits^c_{i=1} P(\omega_i) 
\sum\limits^c_{j=1} P(\omega_j)\\
{}\times \int\limits_{\mathbf{X}_m} 
p_{w_m}\left(\mathbf{x}_m\vert\omega_i\right) \ln p_{w_m} 
\left(\mathbf{x}_m\vert \omega_j\right)\,d\mathbf{x}_m\,.
\label{e15-l}
\end{multline}

Given the dissimilarity measures~(\ref{e1-l}) and~(\ref{e2-l}), the conditional 
density $p_{w_m}(\mathbf{x}_m\vert\omega_i)$ of the form~(\ref{e7-l}) is the 
Gaussian density of~$N_m$ independent variables that have the 
means~$x_{imn}$ and the variances $\sigma^2_{imn}/(2w_m)$, $n=1,\ldots , 
N_m$, subject to $w_m>0$. It allows us to express the integral in~(\ref{e15-l}) 
over the interval $(-\infty, +\infty)$  as the Euler integral~\cite{12-l}. The 
calculation yields the upper estimated differential entropy:
\begin{multline}
H_{w_m}(\mathbf{X}_m)\leq \fr{1}{2}\ln 
\fr{\pi}{w_m}+\fr{1}{2}\sum\limits^c_{j=1} P(\omega_j)  
\sum\limits_{n=1}^{N_m} \ln \sigma^2_{jmn}\\
{}+w_m \sum\limits^c_{i=1} P(\omega_i) \sum\limits^c_{j=1} P(\omega_j) 
\sum\limits_{n=1}^{N_m} \fr{(x_{imn}-x_{jmn})^2}{\sigma^2_{jmn}}\\
+2\fr{w_m^{1/2}}{\sqrt{\pi}}\sum\limits^c_{i=1} 
P(\omega_i)\sum\limits^c_{j=1} P(\omega_j) \sum\limits_{n=1}^{N_m} 
\fr{\vert x_{imn}-x_{jmn}\vert \sigma_{imn}}{\sigma^2_{jmn}}\\
+\fr{1}{2}\sum\limits^c_{i=1} P(\omega_i) \sum\limits^c_{j=1} P(\omega_j) 
\sum\limits_{n=1}^{N_m} \fr{\sigma^2_{imn}}{\sigma^2_{jmn}}
\label{e16-l}
\end{multline}
and the following conditional differential entropy:
\begin{multline}
H_{w_m}\left(\mathbf{X}_m\vert\Omega\right) \\
{}=\fr{1}{2}\ln \fr{\pi e}{w_m} 
+\fr{1}{2} \sum\limits^c_{i=1} P(\omega_i) \sum\limits_{n=1}^{N_m} \ln 
\sigma^2_{imn}\,.
\label{e17-l}
\end{multline}
The substitutions of the differential entropy and the conditional differential entropy 
in~(\ref{e10-l}) by~(\ref{e16-l}) and~(\ref{e17-l}) yield the upper 
estimated average mutual information:

\noindent
\begin{multline}
I_{w_m}\left(\mathbf{X}_m;\Omega\right) \\
{}\leq w_m \sum\limits^c_{i=1} 
P(\omega_i) \sum\limits^c_{j=1} P(\omega_j) \sum\limits_{n=1}^{N_m} 
\fr{(x_{imn}-x_{jmn})^2}{\sigma^2_{jmn}}\\
+2\fr{w_m^{1/2}}{\sqrt{\pi}} \sum\limits^c_{i=1} P(\omega_i) 
\sum\limits^c_{j=1} P(\omega_j) \sum\limits_{n=1}^{N_m} \fr{\vert x_{imn} -
x_{jmn})^2}{\sigma^2_{jmn}}\\
+\fr{1}{2}\sum\limits^c_{i=1}P(\omega_i) \sum\limits_{j=1}^c P(\omega_j) 
\sum\limits^{N_m}_{n=1} \left( \fr{\sigma^2_{imn}} {\sigma^2_{jmn}}-
1\right)\,.
\label{e18-l}
\end{multline}
The right part in~(\ref{e18-l}) is a~parabolic function 
$a_mw_m\linebreak +b_mw_m^{1/2}+c_m$ of the variable $w_m^{1/2}>0$ for 
$m\linebreak =1,\ldots , M$. Since $a_m>0$, $b_m>0$, and $c_m\geq0$, the parabola 
exceeds the value~$c_m$ and grows when~$w_m^{1/2}$ increases.  For 
$w_m^{1/2}=1$, this function gives the upper estimate $a_m+b_m+c_m$ for 
$I(\mathbf{X}_m;\Omega)$. The weights of interest are defined by the values 
$w_m^{1/2}\geq 1$ that satisfy the condition 
$a_mw_m+b_mw_m^{1/2}+c_m\leq H(\Omega)$, $m=1, \ldots , M$. Setting 
$\delta_m=(a_m+b_m+c_m)/H(\Omega)\leq 1$, we assign the parametric source 
weights 
\begin{equation}
w_m(s)=e^{s\delta_m}\,,\enskip m=1,\ldots , M,
\label{e19-l}
\end{equation}
where $s\geq 0$ is a~free parameter that yields $w_m(s)\geq 1$. In what follows, 
we denote the upper estimates~(\ref{e18-l}) taken with the weights~(\ref{e19-l}) 
by $I_s(\mathbf{X}_m;\Omega)$, $m=1,\ldots , M$.

\section{Main Results}

\noindent
   Using in the right part of the form~(\ref{e11-l}) the estimates 
$I(\mathbf{X}_m;\Omega)\leq a_m+b_m+c_m$, $m=1,\ldots , M$, taken with the 
weights~(\ref{e19-l}), one obtains the upper estimated mean value 
$I^{\mathrm{WMV}}_{s\_\mathrm{mean}} (\mathbf{X}^M;\Omega)$.  Also, the estimates 
$I_{w_m}(\mathbf{X}_m;\Omega)\linebreak \leq a_mw_m+b_m w_m^{1/2}+c_m$, 
$m=1,\ldots , M$, taken with the similar weights in the right part of~(\ref{e12-l}) 
yield the upper estimated mean value $I^{\mathrm{GDM}}_{s\_\mathrm{mean}} 
(\mathbf{X}^M;\Omega)$. Then, for $s\to 0$, we calculate an asymptotic 
maximum $I^{\mathrm{WMV}}_{s^*\_\mathrm{mean}}(\mathbf{X}^M;\Omega)$ at the point~$s^*$ 
and show that this maximum satisfies the inequality 
$I^{\mathrm{WMV}}_{s^*\_\mathrm{mean}}(\mathbf{X}^M;\Omega) \leq 
I^{\mathrm{GDM}}_{s^*\_\mathrm{mean}}(\mathbf{X}^M;\Omega)$.

In subsequent statements, we use the following notations:  
\begin{gather*}
\mu=\fr{1}{M}\sum\limits_{m=1}^M \delta_m\,;\quad 
\Delta_1=\fr{1}{M}\sum\limits^M_{m=1} \delta^2_m-\mu^2\,;\\
\Delta_2=\fr{1}{M}\sum\limits^M_{m=1} \delta_m^3-
\mu\fr{1}{M}\sum\limits^M_{m=1} \delta^2_m\,.
\end{gather*}

\noindent
\textbf{Theorem~1.}\ \textit{For $(2\mu\Delta_1-\Delta_2)>\Delta_1 >0$ and 
$s\to 0$, the value $s^*=\Delta_1/(2\mu \Delta_1-\Delta_2)$ yields}
$$
\max\limits_s I^{\mathrm{WMV}}_{s\_\mathrm{mean}} \left(\mathbf{X}^M;\Omega\right) =
\left( 
\mu+\fr{1}{2}\,\Delta_1 s^*\right) H(\Omega)\,.
$$
\textit{For $\Delta_1=0$, there is valid $I^{\mathrm{WMV}}_{s\_\mathrm{mean}} 
(\mathbf{X}^M;\Omega) =\mu H(\Omega)$ for all $s\geq 0$}.

\smallskip

\noindent
P\,r\,o\,o\,f\,.\ \  Using $q_s(\delta_m) =e^{s\delta_m}/\sum\nolimits^M_{m=1} 
e^{s\delta_m}$,  the upper estimated mean value defined in~(\ref{e11-l}) takes the 
form: 
\begin{equation}
I^{\mathrm{WMV}}_{s\_\mathrm{mean}} (\mathbf{X}^M;\Omega) =H(\Omega) 
\sum\limits^M_{m=1} \delta_m q_s(\delta_m)\,.
\label{e20-l}
\end{equation}
For $s\to 0$, there is valid the asymptotic equation: 
\begin{equation}
\sum\limits^M_{m=1} \delta_m q_s(\delta_m) \approx \mu+ \Delta_1 s-
\fr{1}{2}\left( 2\mu \Delta_1-\Delta_2\right) s^2\,.
\label{e21-l}
\end{equation}
Using the assumption of the theorem, the parabola in the right part of~(\ref{e21-l}) 
takes the maximal value $\mu+\Delta_1 s^*/2$ at the point $s^*=\Delta_1/(2\mu 
\Delta_1-\Delta_2)$. Notice that the same values $\delta_m=\delta$, $m=1,\ldots , 
M$, provide $\Delta_1=0$ and $\Delta_2=0$. In this case, $q_s(\delta_m)=1/M$ 
and the sum in~(\ref{e21-l}) is equal to $\mu=\delta$ for all $s\geq0$. Thus, the 
substitution of the sum in~(\ref{e20-l}) by $\mu+\Delta_1 s^*/2$ in case of 
$\Delta_1>0$ or by~$\mu$ in case of $\Delta_1=0$ completes the proof.


\smallskip

\noindent
\textbf{Theorem~2.}\ \textit{For $\Delta_1>0$ and on condition that $a_m\geq 
c_m$, $m=1,\ldots , M$, there is valid the inequality $I^{\mathrm{WMV}}_{s^*\_\mathrm{mean}} 
(\mathbf{X}^M;\Omega)<I^{\mathrm{GDM}}_{s^*\_\mathrm{mean}} (\mathbf{X}^M;\Omega)$ at 
the optimal point $s^*>0$.  For $\Delta_1=0$ and a~given $s\geq 0$, there is valid 
the inequality $I^{\mathrm{WMV}}_{s\_\mathrm{mean}} (\mathbf{X}^M;\Omega) \leq 
I^{\mathrm{GDM}}_{s\_\mathrm{mean}} (\mathbf{X}^M;\Omega)$ which passes into the equality at 
the point $s=0$.}

\smallskip\

\noindent
P\,r\,o\,o\,f\,.\ The estimates~(\ref{e18-l}) taken with the weights~(\ref{e19-l}) 
give the upper estimated mean value~(\ref{e12-l}) as follows:
\begin{multline}
I^{\mathrm{GDM}}_{s\_\mathrm{mean}} \left(\mathbf{X}^M;\Omega\right)\\ 
{}=\fr{1}{M}\sum\limits^M_{m=1} \left( a_m e^{s\delta_m} +b_m 
s^{s\delta_m/2} +c_m\right)\,.
\label{e22-l}
\end{multline}
Taking the square approximations of the exponential terms in~(\ref{e22-l}), one 
obtains the following inequality: 
\begin{multline}
I^{\mathrm{GDM}}_{s\_\mathrm{mean}} \left(\mathbf{X}^M;\Omega\right) \geq \mu H(\Omega) \\
{}+\left( \fr{1}{M} \sum\limits^M_{m=1} 
a_m\delta_m+\fr{1}{2M}\sum\limits^M_{m=1} b_m\delta_m\right) s\\
{}+ \left( \fr{1}{2M}\sum\limits^M_{m=1} 
a_m\delta_m^2+\fr{1}{4M}\sum\limits^M_{m=1} b_m\delta_m^2\right) s^2\,.
\label{e23-l}
\end{multline}
In case of $\Delta_1>0$, the inequality~(\ref{e23-l}) together with the 
estimates~(\ref{e20-l}) and~(\ref{e21-l}) yield: 
\begin{multline}
I^{\mathrm{GDM}}_{s\_\mathrm{mean}}\left(\mathbf{X}^M;\Omega\right)-
I^{\mathrm{WMV}}_{s\_\mathrm{mean}} 
\left(\mathbf{X}^M;\Omega\right)\\
\geq \left( \fr{1}{M} \sum\limits^M_{m=1}a_m\delta_m 
+\fr{1}{2M}\sum\limits^M_{m=1} b_m\delta_m-\Delta_1H(\Omega)\right)s\\
{}+\left( \fr{1}{2M}\sum\limits^M_{m=1} 
a_m\delta_m^2+\fr{1}{4M}\sum\limits^M_{m=1} 
b_m\delta_m^2\right.\\
\left.{}+\fr{1}{2}\left( 2\mu \Delta_1-\Delta_2\right) 
H(\Omega)
\vphantom{\sum\limits^M_{m=1}}
\right)s^2\,.
\label{e24-l}
\end{multline}
Assuming 
\begin{equation}
\fr{1}{M}\sum\limits^M_{m=1} a_m\delta_m+\fr{1}{2M} 
\sum\limits^M_{m=1} b_m\delta_m\geq \fr{1}{2}\Delta_1 H(\Omega),
\label{e25-l}
\end{equation}
the right part in~(\ref{e24-l}) is lower estimated by the parabola
\begin{multline*}
-\fr{1}{2}\,\Delta_1H(\Omega) s+\left( \fr{1}{2M}\sum\limits^M_{m=1} 
a_m\delta_m^2+\fr{1}{4M} \sum\limits^M_{m=1} b_m 
\delta_m^2\right.\\
\left.{}+\fr{1}{2}\left( 2\mu \Delta_1-\Delta_2\right) H(\Omega) 
\vphantom{\sum\limits^M_{m=1}}
\right) s^2
\end{multline*}
that has a~positive root
\begin{multline*}
s_0=
\Delta_1H(\Omega)\Bigg/
\left(
\vphantom{\sum\limits^M_{m=1}}
(2\mu\Delta_1-\Delta_2)H(\Omega)\right.\\
\left.{} +\fr{1}{M} 
\sum\limits^M_{m=1} a_m\delta_m^2+\fr{1}{2M} \sum\limits^M_{m=1} 
b_m\delta_m^2\right)\\
{}<\fr{\Delta_1}{2\mu\Delta_1-\Delta_2}=s^*\,.
\end{multline*}
Since this parabola is positive for $s>s_0$,  the lower estimate of the right part 
in~(\ref{e24-l}) is positive at the point $s^*>0$ of the maximal value 
$I^{\mathrm{WMV}}_{s^*\_\mathrm{mean}} (\mathbf{X}^M;\Omega)$ that provides the inequality 
$I^{\mathrm{GDM}}_{s^*\_\mathrm{mean}} (\mathbf{X}^M;\Omega)- I^{\mathrm{WMV}}_{s^*\_\mathrm{mean}} 
(\mathbf{X}^M;\Omega) >0$. 

Notice that the assumption of the form~(\ref{e25-l}) is equivalent to the inequality 
$$
\fr{1}{M}\sum\limits^M_{m=1} \left( c_m-a_m\right) \delta_m \leq \mu^2 
H(\Omega)
$$
that is valid under the conditions $a_m\geq c_m$, $m\linebreak =1,\ldots , M$. These 
conditions are held if the templates in different classes are sufficiently distinct 
from each other. Formally, the parameters in~(\ref{e18-l}) should satisfy the 
following relation:
\begin{multline*}
\left( x_{imn}-x_{jmn}\right)^2\geq \fr{1}{2}\left\vert \sigma^2_{imn} -
\sigma^2_{jmn}\right\vert \,,\\
 m=1,\ldots , M\,,\enskip n=1,\ldots , N_m\,.
\end{multline*}
In case of $\Delta_1=0$, one has $I^{\mathrm{WMV}}_{s\_\mathrm{mean}} 
(\mathbf{X}^M;\Omega) =\mu H(\Omega)$ and 
$I^{\mathrm{GDM}}_{s\_\mathrm{mean}}(\mathbf{X}^M;\Omega)\geq \mu H(\Omega)$ for a~given 
$s\geq 0$. So, there is valid the inequality $I^{\mathrm{WMV}}_{s\_\mathrm{mean}} 
(\mathbf{X}^M;\Omega)\linebreak \leq I^{\mathrm{GDM}}_{s\_\mathrm{mean}} (\mathbf{X}^M;\Omega)$ 
which passes into the equality at the point $s=0$. The theorem is proved.

\smallskip


Sketches of the graphics in Fig.~3 interpret the theorems~1 and~2.


\begin{figure*} %fig3
 \vspace*{1pt}
    \begin{center}  
  \mbox{%
 \epsfxsize=162.134mm 
 \epsfbox{lan-3.eps}
 }
\end{center}
\vspace*{-9pt}
\Caption{Graphical interpretation of the results for cases of $\Delta_1>0$~(\textit{a}) and 
$\Delta_1=0$~(\textit{b})}
\end{figure*}

\begin{figure*}[b] %fig4
 \vspace*{1pt}
    \begin{center}  
  \mbox{%
 \epsfxsize=163mm 
 \epsfbox{lan-4.eps}
 }
\end{center}
\vspace*{-9pt}
\Caption{Examples of the 8th level representations for the face HSI images}
\end{figure*}
  



\noindent
\textbf{Corollary.}\ For the optimal value~$s^*$ in the case of $\Delta_1>0$ and any 
$s>0$ in the case of $\Delta_1=0$, the mean values of the average mutual information 
per one source in WMV and GDM fusion schemes provide the lower bounds to the 
error probabilities satisfying the inequality 
$\varepsilon^{\mathrm{WMV}}>\varepsilon^{\mathrm{GDM}}$.

\section{Experimental Results}

\noindent
The efficiency of WMV and GDM fusion schemes is shown by comparative error 
rates for face recognition of HSI color images.The components H, S, and~I produce the 
objects of the individual sources and the ensemble HSI produces the composite 
objects. The color images are taken from~25~persons (classes) per 40~images in 
each class~\cite{13-l}.  The prior probability distribution of the classes is uniform. 
Face recognition has been performed in a~space of multilevel tree-structured 
pattern representations with elliptic primitives~\cite{5-l}. The error rates have 
been obtained for multiclass NN (nearest neighbor) and SVM (support vector 
machine) classifiers that are the collections of elementary ``class-vs-all'' classifiers. 
The experiments have been performed using 100~times, 2~fold cross validation. 

The examples of the tree-structured representations for the face components H, S, and I 
are shown in Fig.~4. The image components correspond to the source numbers 
$m=1, 2, 3$.


Using the above representations, the dissimilarity measure 
$d(\mathbf{x}_m, \hat{\mathbf{x}}_m)\geq 0$  for any pair 
of the objects~$\mathbf{x}_m$ and~$\hat{\mathbf{x}}_m$ has been introduced 
in~\cite{14-l}.
The weighted sum of the above measures taken over the components 
H, S, and I yields the general dissimilarity measure $D(\mathbf{x}^3, 
\hat{\mathbf{x}}^3)$  of the form~(\ref{e2-l}) between the corresponding 
composite objects~$\mathbf{x}^3$ and~$\hat{\mathbf{x}}^3$. 

The dissimilarity 
measures $d(\mathbf{x}_m, \hat{\mathbf{x}}_m)$, $m=1,2,3$, and 
$D(\mathbf{x}^3, \hat{\mathbf{x}}^3)$ have allowed us to construct the 
discriminant functions~$g_i^d(\mathbf{x}_m)$ and~$g_i^D(\mathbf{x}^3)$, 
$i=1, \ldots$\linebreak $\ldots , c$, for making the decisions of the form~(\ref{e8-l}) and~(\ref{e9-l}) 
by the appropriate NN and SVM classifiers.
{\looseness=1

}

\begin{table*}\small
\begin{center}
\tabcolsep=8pt
\begin{tabular}{cccccc}
\multicolumn{6}{c}{Error rates for HSI face recognition by NN and SVM classifiers}\\
\multicolumn{6}{c}{\ }\\[-6pt]
\hline
\multicolumn{1}{c}{\raisebox{-6pt}[0pt][0pt]{Classifier}}&
\multicolumn{3}{c}{Sources} &\multicolumn{2}{c}{Fusion schemes}\\ 
\cline{2-6} 
&H&S&I&\hspace*{2mm}WMV&GDM\\ 
\hline 
NN&0.022&0.017&0.015&\hspace*{2mm}0.009&0.006\\ 
SVM&0.019&0.012&0.011&\hspace*{2mm}0.007&0.003\\ 
\hline 
\end{tabular} 
\end{center} 
\vspace*{-12pt}
\end{table*}
The table summarizes the cross-validation error rates for both the individual sources 
and their ensemble using GDM and WMV fusion schemes. The experimental 
results demonstrate a~decrease of the error rates in the ensemble HSI as against the 
error rates for the sources H, S, and~I. Also, the obtained error rates confirm 
some advantage of GDM scheme as compared with the WMV scheme.
 
\vspace*{-9pt}

\section{Concluding Remarks}

\noindent
To compare the potentially achievable  classification error probabilities for two 
fusion schemes in the ensemble of data sources, the information-based criterion 
has been suggested. The proposed  criterion is based on comparing the mean 
values of the average mutual information between the set of the classes and the 
datasets of the sources. These means  are independent on a~decision algorithm 
and  they are defined in the WMV scheme of fusion of the decisions on the source 
objects and in the GDM scheme of fusion of the metrics in datasets of the sources. 
Taking the above mean values as the points of the appropriate rate distortion 
functions, it has been shown that the  lower bound to GDM-based error probability is 
smaller as compared with the similar WMV-based error probability. The advantage 
in accuracy of the GDM scheme relative to the WMV scheme is confirmed by the error 
rates for NN and SVM decision algorithms in experiments on recognition of HSI 
face images given by the ensemble of the sources Н, S, and I.
In future, we plan to 
extend the ensemble of biometric sources and the set of the decision algorithms.  
For the above fusion schemes and the different decision algorithms, we plan 
to estimate a~redundancy of the error rates relative to the appropriate lower 
bounds. 

\vspace*{-9pt}

\Ack
\noindent
The research is partially supported by the Russian Foundation for Basic Research 
(grants Nos.\,18-07-01231 and 18-07-01385).

\renewcommand{\bibname}{\protect\rmfamily References}


\vspace*{-9pt}

{\small\frenchspacing
{\baselineskip=10.45pt
\begin{thebibliography}{99}
\bibitem{1-l}
\Aue{Kuncheva, L.} 2014. \textit{Combining pattern classifiers, methods and algorithms}. 2nd ed. 
New York, NY: John Wiley and Sons. 384~p.
\bibitem{2-l}
\Aue{Gray, R., and D.~Neuhoff.} 1998. Quantization. 
\textit{IEEE T.~Inform. Theory} 44(6):2325--2383.
\bibitem{3-l}
\Aue{Kolmogorov, A.\,N., and V.\,M.~Tikhomirov.} 1961. 
\mbox{$\varepsilon$-entropy} and  $\varepsilon$-capacity of sets in 
functional spaces. \textit{AMS Transl.} 17(2):277--364.
\bibitem{4-l}
\Aue{Lam, L., and C.~Suen.} 1997. 
Application of majority voting to pattern recognition: An 
analysis of its behavior and performance. \textit{IEEE T.~Syst. Man. Cyb.}
27(5):553--568.
\bibitem{5-l}
\Aue{Lange, M.\,M., and D.\,Y.~Stepanov.} 
2014. Recognition of objects given by collections of 
multichannel images. \textit{Pattern Recogn. Image Anal.} 24(3):431--442.
\bibitem{6-l}
\Aue{Kuncheva, L., C.~Whitaker, C.~Shipp, and R.~Duin.} 2003. Limits on the majority
vote accuracy in classifier fusion. \textit{Pattern Anal. Appl.} 6(1):22--31.
\bibitem{7-l}
\Aue{Gallager, R.} 1968. 
\textit{Information theory and reliable communication}. New York, NY: John Wiley and 
Sons. 608~p.
\bibitem{8-l}
\Aue{Lange, M.\,M., and A.\,M.~Lange.} 2018. 
O~teoretiko-informatsionnoy modeli klassifikatsii
dannykh [On information theoretical model for data classification]. 
\textit{Mashinnoe obuchenie i~analiz dannykh}  [J.~Machine Learning Data Analysis] 
4(3):165--179.
\bibitem{9-l}
\Aue{Dobrushin, R.\,L., and B.\,S.~Tsybakov.} 1962. 
Information transmission with additional noise. 
\textit{IRE T.~Inform. Theor.} 8(5):293--304.
\bibitem{10-l}
\Aue{Duda, R., P. Hart, and D.~Stork.}
 2001. \textit{Pattern classification}. 2nd ed. New York, NY: John Wiley and Sons. 
688~p.
\bibitem{11-l}
\Aue{Beckenbach, E., and R.~Bellman.} 1961. 
\textit{Inequalities}. New York, NY: Springer-Verlag. 55~p.
\bibitem{12-l}
\Aue{Gradshteyn, I.\,S., and I.\,M.~Ryzhik.}
 2007. \textit{Table of integrals, series, and products}. 7th ed. 
Academic Press. 1221~p.
\bibitem{13-l}
Database of face images. Available at:
{\sf http://\linebreak sourceforge.net/projects/colorfaces} (accessed 
October~9, 2019).
\bibitem{14-l}
\Aue{Lange, M.\,M., and S.\,N.~Ganebnykh.} 
2018. On fusion schemes for multiclass object 
classification with reject in a~given ensemble of sources. 
\textit{J.~Phys. Conf. Ser.} 1096:012048. 12~p. Available at: 
{\sf https://\linebreak iopscience.iop.org/article/10.1088/1742-6596/1096/1/ 012048}
 (accessed October~7,  2019).
 \end{thebibliography} } }

\end{multicols}

\vspace*{-9pt}

\hfill{\small\textit{Received July 01, 2019}}

\vspace*{-16pt}

\Contrl

\vspace*{-3pt}

\noindent
\textbf{Lange Mikhail M.} (b.\ 1945)~--- Candidate of Science (PhD) in technology, leading 
scientist, Federal Research Center ``Computer Sciences and Control'' of the Russian Academy of 
Sciences, 44-2~Vavilov Str., Moscow 119333, Russian Federation; 
\mbox{lange\_mm@ccas.ru}

 

\newpage

%\vspace*{8pt}

%\hrule

%\vspace*{2pt}

%\hrule

%\vspace*{-7pt}

%\newpage

\vspace*{-28pt}

\def\tit{О СРАВНИТЕЛЬНОЙ ЭФФЕКТИВНОСТИ СХЕМ КЛАССИФИКАЦИИ ДАННЫХ НА~АНСАМБЛЕ 
ИСТОЧНИКОВ С~ИСПОЛЬЗОВАНИЕМ СРЕДНЕЙ ВЗАИМНОЙ ИНФОРМАЦИИ$^*$}

\def\titkol{О сравнительной эффективности схем классификации данных на~ансамбле 
источников} % с~использованием средней взаимной информации}

\def\aut{M.\,M.~Ланге}

\def\autkol{M.\,M.~Ланге}

{\renewcommand{\thefootnote}{\fnsymbol{footnote}} \footnotetext[1]
{Работа частично поддержана РФФИ (проекты 18-07-01231 и 18-07-01385).}}



\titel{\tit}{\aut}{\autkol}{\titkol}

\vspace*{-11pt}

\noindent
Федеральный исследовательский центр <<Информатика и управление>> Российской академии наук, 
\mbox{lange\_mm@ccas.ru}

\vspace*{1pt}

\def\leftfootline{\small{\textbf{\thepage}
\hfill ИНФОРМАТИКА И ЕЁ ПРИМЕНЕНИЯ\ \ \ том\ 13\ \ \ выпуск\ 4\ \ \ 2019}
}%
 \def\rightfootline{\small{ИНФОРМАТИКА И ЕЁ ПРИМЕНЕНИЯ\ \ \ том\ 13\ \ \ выпуск\ 4\ \ \ 2019
\hfill \textbf{\thepage}}}

\vspace*{-1pt}




\Abst{Исследуется точность многоклассовой классификации наборов объектов от 
ансамбля источников при различных схемах комплексирования данных. Предлагается 
новый подход к~сравнению нижних границ вероятности ошибки для двух схем 
классификации с~использованием средней взаимной информации между данными 
источников и множеством классов. Рассмотрена схема WMV (Weighted Majority Vote) на 
основе композиции решений по объектам источников и~схема GDM (General Dissimilarity 
Measure) на основе композиции метрик на множествах объектов источников. Для 
исследуемых схем получены оценки усредненных значений средней взаимной 
информации на один источник. Доказано, что указанная характеристика схемы WMV не 
превосходит аналогичной характеристики схемы GDM, при этом нижняя граница 
вероятности ошибки в~схеме WMV превосходит нижнюю границу вероятности ошибки 
в~схеме GDM. Полученный теоретический результат подтвержден экспериментальными 
оценками вероятности ошибки распознавания цветных HSI изображений лиц для двух 
схем комплексирования данных от источников H, S и~I.} 

\KW{многоклассовая классификация; ансамбль источников; схема комплексирования; 
композиция решений; композиция метрик; средняя взаимная информация; вероятность 
ошибки}

\DOI{10.14357/19922264190403} 



%\vspace*{-3pt}


 \begin{multicols}{2}

\renewcommand{\bibname}{\protect\rmfamily Литература}
%\renewcommand{\bibname}{\large\protect\rm References}

{\small\frenchspacing
{\baselineskip=10.5pt
\begin{thebibliography}{99}
%\vspace*{-3pt}
\bibitem{1-l-1}
\Au{Kuncheva L.} Combining pattern classifiers, methods and algorithms.~--- 2nd ed.~---
  New York, NY, USA: John Wiley and Sons, 2014. 384~p.
\bibitem{2-l-1}
\Au{Gray R., Neuhoff~D.} Quantization~// IEEE T. Inform. Theory, 1998. 
Vol.~44. Iss.~6. P.~2325--2383.
\bibitem{3-l-1}
\Au{Колмогоров А.\,Н.,  Тихомиров~В.\,М.} 
$\varepsilon$-энтропия и~$\varepsilon$-ем\-кость 
множеств в функциональных пространствах~// УМН, 1959. Т.~14. №\,2(86). С.~3--86.

\bibitem{4-l-1}
\Au{Lam L., Suen~C.} Application of majority voting to pattern recognition: An analysis of its behavior and 
performance~// IEEE T. Syst. Man Cyb., 1997. Vol.~27. Iss.~5. P.~553--568.
\bibitem{5-l-1}
\Au{Lange M.\,M., Stepanov~D.\,Y.}
 Recognition of objects given by collections of multichannel images~// 
Pattern Recogn. Image Anal., 2014. Vol.~24. Iss.~3. P.~431--442.
\bibitem{6-l-1}
\Au{Kuncheva L., Whitaker~C., Shipp~C., Duin~R.}
 Limits on the majority vote accuracy in classifier fusion~// 
Pattern Anal. Appl., 2003. Vol.~6. Iss.~1. P.~22--31.
\bibitem{7-l-1}
\Au{Gallager R.} Information theory and reliable communication.~---
  New York, NY, USA: John Wiley and Sons, 1968. 
608~p.
\bibitem{8-l-1}
\Au{Ланге М.\,М., Ланге~А.\,М.} О~тео\-ре\-ти\-ко-ин\-фор\-ма\-ци\-он\-ной 
модели классификации данных~// 
Машинное обучение и анализ данных, 2018. Т.~4. Вып.~3. С.~165--179.
\bibitem{9-l-1}
\Au{Dobrushin R.\,L., Tsybakov~B.\,S.}
 Information transmission with additional noise~// IRE T. 
Inform. Theor., 1962. Vol.~8. Iss.~5. P.~293--304.
\bibitem{10-l-1}
\Au{Duda R., Hart~P., Stork~D.}
 Pattern classification.~--- 2nd ed.~--- New York, NY, USA: John Wiley and Sons, 2001. 688~p.
\bibitem{11-l-1}
\Au{Beckenbach E., Bellman~R.} Inequalities.~--- New York, NY, USA: Springer-Verlag, 1961. 55~p.
\bibitem{12-l-1}
\Au{Gradshteyn I.\,S., Ryzhik~I.\,M.} Table of integrals, series, and products.~---
7th ed.~--- Academic Press, 
2007. 1221~p.
\bibitem{13-l-1}
Database of face images. {\sf http://sourceforge.net/\linebreak projects/colorfaces}.
\bibitem{14-l-1}
\Au{Lange M.\,M., Ganebnykh~S.\,N.} 
On fusion schemes for multiclass object classification with reject in 
a~given ensemble of sources~// J.~Phys. Conf. Ser., 2018. Vol.~1096.
 Art. ID: 012048.  P.~1--12. 
\end{thebibliography}
} }

\end{multicols}

 \label{end\stat}

 \vspace*{-9pt}

\hfill{\small\textit{Поступила в~редакцию 01.07.2019}}


%\renewcommand{\bibname}{\protect\rm Литература}
\renewcommand{\figurename}{\protect\bf Рис.}
\renewcommand{\tablename}{\protect\bf Таблица}



 
 
%Ланге Михаил Михайлович (р.\ 1945)~--- кандидат технических наук, ведущий научный 
%сотрудник Федерального исследовательского центра <<Информатика и управление>> 
%Российской академии наук

 
 
 
      %6
\newcommand{\TSF}{{\sf T}}

\def\stat{strijov}

\def\tit{ПОВЫШЕНИЕ КАЧЕСТВА КЛАССИФИКАЦИИ В~ЗАДАЧЕ~ОБНАРУЖЕНИЯ ВНУТРЕННЕГО ПЛАГИАТА$^*$}

\def\titkol{Повышение качества классификации в~задаче обнаружения внутреннего плагиата}

\def\aut{И.\,О.~Молибог$^1$, А.\,П.~Мотренко$^2$, В.\,В.~Стрижов$^3$}

\def\autkol{И.\,О.~Молибог, А.\,П.~Мотренко, В.\,В.~Стрижов}

\titel{\tit}{\aut}{\autkol}{\titkol}

\index{Молибог И.\,О.}
\index{Мотренко А.\,П.}
\index{Стрижов В.\,В.}
\index{Molybog I.\,O.}
\index{Motrenko A.\,P.}
\index{Strijov V.\,V.}


{\renewcommand{\thefootnote}{\fnsymbol{footnote}} \footnotetext[1]
{Работа выполнена при финансовой поддержке РФФИ (проект 16-07-01155).}}


\renewcommand{\thefootnote}{\arabic{footnote}}
\footnotetext[1]{Центр энергетических систем, Сколковский институт науки и~технологий; 
Московский фи\-зи\-ко-тех\-ни\-че\-ский институт,  \mbox{i.molybog@skoltech.ru}}
\footnotetext[2]{Московский физико-технический институт, \mbox{anastasiya.motrenko@phystech.edu}}
\footnotetext[3]{Вычислительный центр им.\ А.\,А.~Дородницына Федерального исследовательского 
центра <<Информатика и~управление>> Российской академии наук, \mbox{strijov@phystech.edu}}

%\vspace*{-18pt}


\Abst{Исследуется задача классификации объектов в~многомерных пространствах. 
Для снижения размерности задачи предлагается модификация алгоритма t-SNE
 (\textit{англ}.\ t-distributed Stochastic 
Neighbor Embedding), 
в~которой при обучении используется информация о~разметке, не возникает необходимости 
заново обучать алгоритм при добавлении новых данных, а также предусмотрена 
параллельная реализация. Предлагаемый алгоритм решает задачу внутреннего плагиата, 
в~которой признаками являются частотные словесные профили сегментов текста. 
Показано, что качество классификации после применения алгоритма выше, чем 
без него или с~другими алгоритмами.}

\KW{анализ данных; снижение размерности; нелинейные методы снижения размерности; 
обучение многообразий; обнаружение внутреннего плагиата}

\DOI{10.14357/19922264170307} 

\vspace*{6pt}


\vskip 10pt plus 9pt minus 6pt

\thispagestyle{headings}

\begin{multicols}{2}

\label{st\stat}

\section{Введение}

В работе рассматривается задача классификации объектов в~пространствах большой 
раз\-мер\-ности, признаковое описание которых имеет в~себе скрытые функциональные 
зависимости.
Предполагается, что объекты содержатся вблизи многообразия много меньшей размерности, 
чем размерность исходного пространства. Назовем это предположение гипотезой 
многообразия~\cite{fefferman2016testing}. Данные ряда практических задач, включая 
задачи анализа генома, анализа текста и~распознавания изображений, не противоречат 
этой гипотезе~\cite{maaten2008visualizing}. В~\cite{narayanan2010sample} было 
дано ее формальное определение и~перечислены идеи методов, которыми ее можно 
проверить. Практической задачей, рассматриваемой в~данной работе, является задача 
обнаружения внутреннего плагиата~\cite{zu2006intrinsic, kuznetsov2016methods}.

Задача обнаружения внутреннего плагиата состоит в~поиске заимствованных частей 
документа без использования внешних источников. При решении задачи исследуемый 
текст некоторым образом разбивается на сегменты. Каждому сегменту соответствует 
его вектор признаков. Сегмент считается минимальной единицей заимствования. Он считается 
либо полностью заимствованным, либо полностью оригинальным. Тогда задача обнаружения 
внутреннего плагиата является задачей классификации, где объектами являются векторы 
признаков сегментов, а~классами~--- метки заимствования или оригинальности.

Способы разбиения на сегменты, как и~способы вычисления вектора признаков, являются 
предметом отдельного исследования. 
Подходы~\cite{zu2006intrinsic,muhr2010external,stamatatos2009intrinsic,kestemont2011intrinsic} 
продемонстрировали 
на конкурсе PAN-2011~\cite{potthast2011overview} наилучшее качество решения задачи 
обнаружения внутреннего плагиата. Они включают разбиение документа на абзацы, 
предложения, блоки слов или символов. В~них используются признаки, основанные на 
частотных профилях сегментов. Такие признаки имеют размерность, пропорциональную 
числу слов в~документе, сильно разрежены и~не всегда информативны.

В данной работе предполагается, что объекты с~таким признаковым описанием
 подчиняются гипотезе многообразия. Это означает, что метрически близкие объекты 
 могут быть геодезически далекими, и~дает возможность применить методы снижения 
 размерности для улучшения качества классификации.

В задаче понижения размерности требуется построить гладкое отображение 
множества~$\mathbf{X}$ в~пространстве исходных данных в~некоторое 
мно\-же\-ст\-во~$\mathbf{Z}$ в~пространстве меньшей размерности.\linebreak Будем называть 
элементы~$\mathbf{Z}$ образами элементов~$\mathbf{X}$. Пространство образов 
будем называть результирующим.
В конкретных алгоритмах на это отображение накладывают необходимые ограничения, 
исходя из специфики задачи~\cite{fodor2002survey}. Приведем некоторые из них.

Для снижения размерности широко применяются линейные методы, основанные на 
анализе дисперсии: латентно-семантический 
анализ~\cite{brooke2012paragraph, brooke2012unsupervised}, анализ главных 
компонент~\cite{gorban2008principal}. Одна\-ко они могут не сохранять кластерную 
структуру исходных данных и~потому не применимы для решения задач вложений из 
нелинейных мно\-го\-об\-разий.
{ %\looseness=1

}

Для выполнения вложений из нелинейных многообразий были разработаны алгоритмы, 
использующие изометрические отображения. Алгоритмы ISOMAP
(Isometric Mapping)~\cite{tenenbaum2000global} 
и~Laplacian Eigenmap~\cite{belkin2001laplacian} приближают геодезическое расстояние 
с~по\-мощью графа~$k$ ближайших соседей. Алгоритмы Local Linear 
Embedding (LLE)~\cite{roweis2000nonlinear} и~Hessian-based LLE~\cite{donoho2003hessian} основаны на предположении, что 
многообразие аппроксимируется ку\-соч\-но-ли\-ней\-ной функцией. Для каждого 
объекта исходного пространства строится его линейное приближенное описание 
через соседние объекты, после чего по этим описаниям строятся образы в~ре\-зуль\-ти\-ру\-ющем 
пространстве. Метод~\cite{donoho2003hessian} использует для описания объектов 
специальную квадратичную форму, что гарантирует асимптотическую оптимальность 
метода даже в~случае невыпуклых множеств.

Алгоритм Local Tangent Space Alignment Algorithm~\cite{zhang2004principal} 
также использует ку\-соч\-но-ли\-ней\-ную аппроксимацию. Многообразие приближается 
гиперплоскостью в~окрестности каждой точки, после чего полученные приближения 
сглаживаются между собой. При помощи Semidefinite 
Embedding~\cite{weinberger2006unsupervised} можно получить вложение, в~котором 
сохранены точные расстояния между ближайшими объектами. Для этого метод 
максимизирует след матрицы Грама для образов при ограничениях, накладываемых 
отношением соседства объектов исходного пространства~и их~матрицей Грама.

Все перечисленные методы нацелены на наиболее точное сохранение расстояний между 
объектами при снижении размерности. Это может при\-вес\-ти к~неустойчивости решения, 
связанной с~тем, что изменения расстояния между далекими и~близкими объектами 
штрафуются одинаково. Кроме того, они не приспособлены для решения задачи 
классификации, поскольку не учитывают разметку при выполнении вложения, хотя 
существуют их модификации, обладающие этим свойством. В~\cite{chen2010distance} 
метод аппроксимации расстояний, используемый в~ISOMAP, модифицирован в~методе 
оптимизации целевого функционала. Полученный метод получил название TRIMAP. 
В~нем при обучении используется разметка обучающей выборки.

В данной работе применяется метод t-NSE~\cite{maaten2008visualizing}. Выгодной особенностью метода t-SNE 
является склонность к~локализации изолированных плотных пространственных структур 
произвольной геометрии. Под изолированной плотной структурой подразумевается 
множество точек, имеющих близких соседей из той же структуры, но сравнительно 
удаленных от всех точек не из нее. Такой эффект достигается тем, что близким 
и~далеким объектам назначаются разные приоритеты.

Недостатком метода t-SNE в~отношении задачи классификации является то, что в~нем 
не преду\-смот\-ре\-но функции вложения объектов, не участвовавших в~построении уже 
существующего вложения. В~работе~\cite{van2009learning} описана параметрическая 
модификация t-SNE, которая частично избавлена от этой особенности, однако в~данной 
работе она не использовалась.

Дополнительным ограничением применимости метода t-SNE является высокая по 
сравнению с~другими методами вложений вычислительная сложность. 
Хотя в~\cite{van2014accelerating} предлагаются два способа вычисления градиента, 
при использовании которых сложность непараметрического t-SNE 
со\-став\-ля\-ет~$O(k  m \log(m))$, где~$m$ ~--- размер выборки, а~$k$ ~--- 
размерность результирующего пространства, этого ускорения недостаточно для 
обеспечения комфортной работы даже с~выборками длиной порядка~$10^3$.

Основным вкладом данной статьи в~теорию распознавания образов является предложенная 
модификация метода t-SNE, позволяющая строить классификаторы в~результирующем 
пространстве. Преиму\-ществом предлагаемого метода является то, что он расширяет 
границы применимости оригинального метода t-SNE. Разработанная модификация 
предусматривает вложение тестовых данных без повторного вложения обучающих, 
а~также может учитывать разметку обучающих данных и~имеет параллельную реализацию.

\section{Постановка задачи}

Обозначим~$\mathbb X \subset \mathbb R^n$ множество всех возможных 
векторов~$\mathbf x$ признаков изучаемых объектов.
Предполагается, что объекты~$\mathbb X$ подчиняются
\textit{гипотезе многообразия}: найдется гладкое 
отображение~$\mathbf f: \mathbb {R}^{d} \hm\to \mathbb {R}^{n}$ такое, что
$$
\mbox{для~} \mathbf x~ \in \mathbf{X}~\mbox{существует~} \mathbf z^* \in 
\mathbb{R}^{d}: \mathbf x = \mathbf{f}(\mathbf z^*) + \boldsymbol{\varepsilon}\,,
$$
где~$\boldsymbol{\varepsilon}$~--- случайный вектор с~нулевым 
математическим ожиданием и~конечной матрицей корреляций. Будем называть~$d$ 
эффективной размерностью исходного пространства~$\mathbb X$. 
Она определяется природой признакового пространства. Поскольку~$d$ 
заранее не известно, введем понятие результирующего пространства~$\mathbb {R}^{k}$, в~котором выполняется поиск решения. В~общем случае~$k \hm\ne d$. Процесс поиска 
образов объектов выборки в~результирующем пространстве назовем вложением в~него.

Рассмотрим выборку из~$m$ объектов, заданную матрицей
\begin{equation}
\label{X}\mathbf X = 
\left[\mathbf x_{1} \cdots \mathbf x_{m}\right]^{\TSF}\,, 
\quad \mathbf x_i \in \mathbb X\,, \quad i = 1, \dots, m\,.
\end{equation}
Пусть~$p_{ij} = P(\mathbf x_i, \mathbf x_j)$ и~$q_{ij} 
\hm= Q(\mathbf z_i, \mathbf z_j)$~--- расстояния между объектами 
в~$\mathbb R^n$ и~$\mathbb {R}^{k}$ соответственно:
\begin{multline*}
p_{ij} = \fr{p_{j|i} + p_{i|j}}{2m}, \\
p_{i|j} = \fr
{\exp\left(-
{||\mathbf x_i - \mathbf x_j||^2}/
\left({2\sigma_i^2}\right)\right)}
{\sum\nolimits_{k\ne i}\hspace*{-2pt}\exp\left(-{||\mathbf x_i - \mathbf x_k||^2}/
\left({2\sigma_i^2}\right)\right)}\,;
\end{multline*}

\vspace*{-12pt}

\noindent
\begin{multline*}
q_{ij} = \fr
{\left(1 + ||\mathbf z_i - \mathbf z_j||^2\right)^{-1}}
{\sum\nolimits_{k\ne i}
\left(1 + ||\mathbf z_i - \mathbf z_k||^2\right)^{-1}}\,,\quad q_{ii} = 0\,, 
\\
i, j \in\{1,\dots,m\}\,.
\end{multline*}

Параметр~$\sigma_i$ в~условном распределении~$p_{ij}$ задан для 
каждого~$i$ и~зависит от расположения~$\mathbf{x}_i$ относительно других 
объектов в~исходном пространстве. Eсли он расположен в~области высокой 
концентрации исходных данных, то коэффициент~$\sigma_i$ имеет 
меньшие значения, чем если бы концентрация была низкой.

Расположение 
\begin{equation}
\label{Z} 
\mathbf Z = \left[\mathbf z_{1}\cdots\mathbf z_{m}\right]^{\TSF} \subset 
\mathbb {R}^{k}
\end{equation} 
как образов~$\mathbf{X}$ в~результирующем пространстве~$\mathbb R^k$ 
находится путем минимизации дивергенции Куль\-ба\-ка--Лейб\-ле\-ра:
\begin{equation}
\label{argmin}
{\mathbf Z_{\min} = 
\operatornamewithlimits{argmin}\limits_{\mathbf Z \in \mathbb {R}^{m\times k}}
C(\mathbf X, \mathbf Z)}\,,
\end{equation}
где
\begin{equation}
\label{KL}
C(\mathbf X, \mathbf Z) = \text{KL}(P||Q) = 
\sum\limits_{i \neq j} p_{ij} \log \fr{p_{ij}}{q_{ij}}\,.
\end{equation}

Заметим, что минимизация происходит только по координатам 
объектов~$\mathbf z_{1}, \dots ,\mathbf z_{m}$ как по переменным, 
а~координаты~$\mathbf x_{1}, \dots , \mathbf x_{m}$ считаются известными константами.

Задача решается градиентными методами~\cite{maaten2008visualizing}. 
Для инициализации начальных точек~$\mathbf{Z}^{(0)} \hm= [\mathbf z_1^{(0)} \cdots 
\mathbf z_m^{(0)}]^{\TSF}$ градиентного спуска в~стандартной реализации было
 предложено~\cite{maaten2008visualizing} два метода: инициализировать 
 случайными точками либо использовать для задания начальной инициализации 
 метод Principal Components Analysis. От качества начальной 
 инициализации, в~случае с~невыпуклой задачей оптимизации, зависят не 
 только ско\-рость сходимости к~оптимуму, но и~локальный минимум, к~которому 
 будет сходиться градиентный метод.

\section{Предлагаемая модификация t-SNE}

Рассмотрим задачу классификации с~обуча\-ющей выборкой~$\mathbf{X}$~(\ref{X}) и~тестовой 
выборкой из~$m'$ объектов~$\mathbf{X}'\hm= [\mathbf x_{m+1}\cdots\mathbf x_{m+m'}]^{\TSF} 
\hm\subset \mathbb X$. Соответственно, метки классов~$y_i \hm\in \{0,1\}$, 
$i \hm= 1, \ldots, m,$ известны, а~$\hat y_i$, $i\hm = m\hm+1, \ldots, m\hm+m',$ 
необходимо оценить. Так как на этапе обучения данные~$\mathbf{X}'$ могут быть
 недоступны, метод непараметрического t-SNE не применим для снижения размерности 
 в~задачах классификации. Назовем это проблемой непросмотренных объектов 
 (out-of-sample problem). Для ее решения предлагается минимизировать~(\ref{KL}) 
 независимо по различным подмножествам объектов.
 


Для повышения качества классификатора в~результирующем пространстве 
предлагается перед вложением обучающей выборки добавить в~ней метки 
классов в~качестве признаков и~улучшить таким образом начальное приближение 
градиентного метода. Идея такого подхода заключается в~том, что, поскольку t-SNE 
сохраняет только локальную структуру схожести между объектами, после проведения 
процедуры понижения размерности классифицируемые объекты отображаются в~клас\-те\-ры, 
предварительно разнесенные с~учетом меток. При этом используется предположение, 
что объекты из~$\mathbf{X}'$ больше схожи с~объектами~$\mathbf{X}$ того же 
класса, чем с~объектами противоположного. Таким образом удается увеличить 
расстояние между образами классифицируемых объектов из различных классов, 
что упрощает их классификацию. Ниже % рис. \ref{schema}
показаны основные 
отображения оригинального непараметрического t-SNE
\begin{align*}
\mathbf{X}\in\mathbb{R}^{m \times n} &\longrightarrow   
\mathbf{Z}\in\mathbb{R}^{m \times k}\,; \\
\mathbf{X^\prime}\in\mathbb{R}^{m^\prime \times n} &\longrightarrow
\mathbf{Z^\prime}\in\mathbb{R}^{m^\prime \times k}
%\label{tSNE_original}
\end{align*}
и предложенной модификации


\noindent
\begin{center}
\mbox{%
\epsfxsize=74.974mm
\epsfbox{str-0.eps}
}
\end{center}
%{\small\begin{equation*}
%\xymatrix{
%\mathbf{X}|\mu\mathbf{y}\in\mathbb{R}^{m \times (n+1)}
% \ar@<1ex>[rr]^{\text{Начальная партия}} %{\text{Start batch}}
%\ar[rr]_{\text{Дополнительные партии}}%{\text{Supplimentary batches}}
%\ar[dr] &   &\mathbf{Z}\in\mathbb{R}^{m \times k} \ar[dl]\\
%\mathbf{X^\prime}\in\mathbb{R}^{m^\prime \times n} \ar[r] & \mathbf{Z^\prime}^{(0)}\in\mathbb{R}^{m^\prime \times k} \ar[r] & %\ar[dr] &
%\mathbf{Z^\prime}\in\mathbb{R}^{m^\prime \times k}\\
%%\mathbf{X^\prime}\in\mathbb{R}^{m^\prime \times n} \ar[ur]   &    & \mathbf{Z^\prime}\in\mathbb{R}^{m^\prime \times k}   \\
%}
%\label{tSNE_modified}
%\end{equation*}}

\vspace*{-12pt}

\paragraph*{Использование исходной разметки выборки при вложении 
для обучения классификатора.}
Для учета\linebreak\vspace*{-12pt}

\pagebreak

\noindent
 разметки обучающей выборки признаковая матрица~$\mathbf{X}$ 
расширяется дополнительным столбцом признаков

\noindent
$$
\tilde{\mathbf X} = \left (
\begin{array}{c|c}
\mathbf X & \mu \mathbf y
\end{array}\right ),
$$

\vspace*{-2pt}

\noindent
где~$\mu$~--- вес меток как признаков.
В~модифицированном алгоритме на основе расширенной матрицы~$\tilde X$ выполняется 
поиск образов~$\mathbf{Z}$~(\ref{KL}), на которых обучается классификатор.
Таким образом, при построении вложения обучающей выборки решается задача

\noindent
$$
\mathbf Z_{\min} = \argmin\limits_{\mathbf Z \in \mathbb {R}^{m\times k}}
C\left (\mathbf{\left (
\begin{array}{c|c}
\mathbf X & \mu \mathbf y
\end{array}\right )}, \mathbf Z\right )\,.
$$

\vspace*{-12pt}

\paragraph*{Вложение новых объектов в~пространство со сниженной размерностью 
для классификации.}
Обозначим через~$\mathbf{Z}' \hm= [\mathbf z_{m+1}\cdots \mathbf z_{m+m'}]^{\TSF}$ 
образы~$\mathbf{X}'$ в~результирующем пространстве. Аналогично~(\ref{argmin}) 
сформулируем задачу поиска~$\mathbf{Z}'$ в~виде~$m'$ задач~$k$-мер\-ной минимизации, 
которые могут быть решены независимо:

\noindent
\begin{multline*}
\mathbf z_i^{\min} = \argmin\limits_{\mathbf z_i \in \mathbb {R}^{m'}}C
\left (\left [
\fr{\mathbf X}{\mathbf{x}_i^{\TSF} }\right ],\,
\left [ \fr{\mathbf {Z}}{\mathbf {z}_i^{\TSF}}\right ]
\right )\,,\\
i = m+1, \ldots, m+m'\,,
\end{multline*}
где матрицы $\left [
\fr{\mathbf X}{\mathbf x_i^{\TSF}}\right ]$ и~$\left [ \fr{\mathbf{\mathbf Z}}{\mathbf{z}_i^{\TSF} }\right ]$
получены из~$\mathbf{X}$ и~$\mathbf{Z}$ добавлением строк~$\mathbf x_i^{\TSF}$ 
и~$\mathbf z_i^{\TSF}$ соответственно. При использовании такого подхода предполагается, 
что обучающая выборка~$\mathbf{X}$~(\ref{X}) достаточно репрезентативна.

Для инициализации образов~$\mathbf z_{i'}$ классифицируемых объектов предлагается 
использовать метод взвешенного среднего по образам соседей: 
\begin{multline*}
\mathbf z^{(0)}_{i'}
 = \sum\limits_{i = 1}^m\mathbf z_i w_{ii'}\,,\enskip \sum\limits_{i=1}^mw_{ii'} 
= 1\,,\\
i' = m+1, \ldots, m+m'\,,
\end{multline*}
где~$w_{ii'}$~--- веса образов объектов~$\mathbf x_i$, $i\hm = 1, \ldots, m$.
В~работе рассмотрены два способа задания весов: %\linebreak\vspace*{-12pt} 


\end{multicols}

 \begin{algorithm*} %[!htbp]
\caption{Вложение выборки с~известным вектором ответов классификации~$\mathbf y$} 
\label{start}
    \SetAlgoLined
    \setcounter{AlgoLine}{0}
    \KwData{$\mathbf{X}, \mathbf y, \mu, S_s, S_b$}
    \KwResult{$\mathbf{Z}$}

    $\tilde{\mathbf X} = (\mathbf X | \mu \mathbf y)$

    Инициализировать~$\mathbf{Z}$~(\ref{Z}) случайно или при помощи PCA($\tilde{\mathbf X}$). Положить инициализацию начальной точкой градиентного метода:~$\mathbf{Z}^{(0)}$.

    \eIf{$m$ > $S_s$}
    {
        Разбить~$\mathbf{Z}$ на партии: начальная партия~$\mathbf{Z}_0$ размером~$S_s$ и~$B = \left \lceil ({m - S_s})/{S_b} \right \rceil$ дополнительных партий~$\mathbf{Z}_1, \dots, \mathbf{Z}_B$ размером не больше чем~$S_b$ каждая.
        Оптимизировать~(\ref{KL}) по~$\mathbf{Z}_0$, зафиксировав координаты остальных объектов из~$\mathbf{Z}$, известные из предыдущего шага.
        \For{$\mathbf{Z}_i$ $\in$ $\{\mathbf Z_1, \dots \mathbf Z_B\}$}
        {
            Оптимизировать~(\ref{KL}) по $\mathbf{Z}_i$, зафиксировав координаты остальных объектов из~$\mathbf{Z}$, известные из предыдущего шага.
        }
    }
    {
        Оптимизировать~(\ref{KL})
    }
\end{algorithm*}
\begin{algorithm*} %[!htbp]
    \SetAlgoLined
    \setcounter{AlgoLine}{0}
    \KwData{$\left [
\begin{array}{c}
\mathbf X \\
\hline
\mathbf X' \\
\end{array}\right ], \mathbf Z$}
    \KwResult{$\mathbf{Z}'$}
    Инициализировать~$\mathbf{Z}$~(\ref{Z}) случайно, при помощи PCA($\tilde{\mathbf X}$), либо используя~(\ref{knnstud2}) или~(\ref{knnstud}) для расчета~$\mathbf W$ и~считать~$\mathbf{Z}'^{(0)} = \mathbf Z^{\TSF} \mathbf W$.
    
    \For {$i \in \{m+1, \dots, m+m'\}$}
    {
        Оптимизировать~(\ref{KL}) по~$\mathbf z_{i}$, зафиксировав координаты остальных объектов из $\left [\begin{array}{c}
\mathbf Z \\
\hline
\mathbf Z' \\
\end{array}\right ]$, известные из предыдущего шага.
}
\caption{Вложение выборки без известного вектора ответов классификации} \label{batch}
\end{algorithm*}

\begin{figure*}[b] %fig1
    \vspace*{4pt}
\begin{center}
\mbox{%
\epsfxsize=161.897mm
\epsfbox{str-1.eps}
}
\end{center}
\vspace*{-9pt}
\Caption{Зависимость~$F_1$ от~$k$ при различных эффективных размерностях 
выборки~$d\hm=4$~(\textit{a}), 8~(\textit{б}); 12~(\textit{в})  
и~16~(\textit{г}) при использовании t-SNE (\textit{1}~--- Student; \textit{2}~--- Softmax; \textit{3}~--- Random; 
\textit{4}~--- PCA)
и~других методов снижения
размерности (\textit{5}~--- LLE; \textit{6}~--- PCA; \textit{7}~--- ISOMAP),
а~также без применения снижения размерности~(\textit{8})}\label{dim_z_inform}
\end{figure*}

\begin{multicols}{2}

\noindent
\begin{equation}
\label{knnstud}
w_{ii'}^{\mathrm{softmax}} = \fr{\exp(-\|\mathbf x_i - \mathbf x_{i'} \| )}
{\sum\nolimits_{k = 1}^m\exp(-\|\mathbf x_k - \mathbf x_{i'} \| )}
\end{equation}
%\noindent
или

\noindent
\begin{equation}
\label{knnstud2}
 w_{ii'}^{\mathrm{stud}}= \fr{(1 + \|\mathbf x_k - \mathbf x_{i'}\|^2)^{-1}}
{\sum\nolimits_{k = 1}^m (1 + \|\mathbf x_k - \mathbf x_{i'}\|^2)^{-1}}\,.
\end{equation}

Для ускорения процедуры вложения при работе с~большими данными предлагается 
процедура поэтапного вложения объектов блоками, размер которых~--- $S_s$ для 
первого по очереди и~$S_b$ для всех остальных~--- много меньше размера~$m$ всей 
вы\-борки.

Псевдокод предложенного метода приведен в~алгоритмах~1 и~2.

\vspace*{-5pt}

\section{Вычислительный эксперимент}

\vspace*{-2pt}

Вычислительный эксперимент состоит из двух частей: исследование 
разработанного алгоритма на синтетических данных и~применение разработанного 
алгоритма для решения задачи внутреннего плагиата.

Для инициализации вложения тестовых данных использовались четыре различных подхода: 
случайный~--- инициализация случайным образом; PCA (Principal Component Analysis)~--- 
инициализация образами при 
снижении раз\-мер\-ности методом главных компонент; Softmax и~Student~--- задаваемые 
по формулам~(\ref{knnstud}) и~(\ref{knnstud2}). 

На рис.~1--5 представлены 
результаты экспериментов, проведенных при использовании всех этих способов 
инициализации.

Все инициализированные таким образом объекты далее преобразуются, 
минимизируя~(\ref{KL}) при фиксированных образах объектов обучающей выборки. 
После этого происходит классификация полученных образов.



Методы инициализации~(\ref{knnstud}) и~(\ref{knnstud2}) были предложены так, 
чтобы инициализированные данные обладали свойством сохранения локальной структуры 
исходной выборки. Предполагалось, что это\linebreak\vspace*{-12pt}

\pagebreak

\end{multicols}

\begin{figure*} %fig2
\vspace*{1pt}
\begin{center}
\mbox{%
\epsfxsize=161.897mm
\epsfbox{str-2.eps}
}
\end{center}
\vspace*{-11pt}
\Caption{Зависимость $F_1$ от~$\mu$ при различных размерностях 
выборки $n\hm=6$~(\textit{a}); 60~(\textit{б}); 300~(\textit{в})
и~600~(\textit{г}) при использовании t-SNE 
(\textit{1}~--- Student; \textit{2}~--- KNN ($k$ nearest neighbor); \textit{3}~--- Random; 
\textit{4}~--- PCA)
и~других методов снижения
размерности (\textit{5}~--- LLE; \textit{6}~--- PCA; \textit{7}~--- ISOMAP),
а~также без применения снижения размерности~(\textit{8})} 
%\vspace*{3pt}
\label{weight_x}
\end{figure*}

\begin{multicols}{2}

\noindent
 улучшит сходимость градиентного метода, 
используемого для минимизации~(\ref{KL}), по сравнению с~инициализациями PCA и~random.

\vspace*{-2pt}

\subsection{Исследование свойств алгоритма на~синтетических данных}

\vspace*{-1pt}

В данном подразделе для эмпирического исследования свойств предлагаемого 
алгоритма использовались синтетические выборки~$\mathbf{X} \hm= 
[\mathbf x_1 \cdots \mathbf x_m]^{\TSF}$. Для любого вектора~$\mathbf x_i$ компоненты 
были сгенерированы как стандартные нормальные распределения на гранях гиперкуба. 
При этом эффективная размерность выборки составляла~$d$, а~оставшиеся 
признаки были шумовыми. Далее выборка сворачивалась в~спираль по одной из 
размерностей. Это делалось для того, чтобы реализовать предположение 
о~существовании многообразия меньшей размерности, в~котором содержится выборка. 
Генерировалось одинаковое количество объектов разных классов, 
а~на обучение и~контроль выборка разбивалась в~соотношении~$1:4$.

В этом подразделе описывается исследование качества классификации с~применением 
предлагаемого алгоритма в~зависимости от основных его па\-ра\-мет\-ров и~специфики выборки. 
Для сравнения предлагаемого алгоритма и~его исследования рассматривается классификация 
в~комбинации с~другими методами снижения размерности: 
PCA~\cite{kim2007distance}, LLE~\cite{roweis2000nonlinear}, 
ISOMAP~\cite{tenenbaum2000global}, а~также без применения снижения 
размерности. Для построения классификатора использовался метод логистической 
регрессии на основе Stochastic Gradient Descent~\cite{bottou2012stochastic}.

\begin{figure*} %fig3
\vspace*{1pt}
\begin{center}
\mbox{%
\epsfxsize=164.997mm
\epsfbox{str-3.eps}
}
\end{center}
\vspace*{-9pt}
\Caption{Зависимость~$F_1$ от~${S_s}/{m}$ при различных размерах выборки
$m=1000$~(\textit{а}); 2000~(\textit{б}); 3000~(\textit{в})
и~4000~(\textit{г}) при использовании t-SNE (\textit{1}~--- Student; \textit{2}~--- KKN;
\textit{3}~--- Random) и~других методов снижения размерности
(\textit{4}~--- LLE; \textit{5}~--- PCA; \textit{6}~--- ISOMAP),
а~также без применения снижения размерности~(\textit{7})} 
\label{Ss_n}
\end{figure*}
\begin{figure*} %fig4
\vspace*{1pt}
\begin{center}
\mbox{%
\epsfxsize=162.776mm
\epsfbox{str-4.eps}
}
\end{center}
\vspace*{-9pt}
\Caption{Зависимость~$F_1$ от~$S_b$ при различных размерах выборки
$m\hm=500$~(\textit{а}) и~1000~(\textit{б})
при использовании t-SNE (\textit{1}~--- Student; \textit{2}~--- KKN;
\textit{3}~--- Random) и~других методов снижения размерности
(\textit{4}~--- LLE; \textit{5}~--- PCA; \textit{6}~--- ISOMAP),
а~также без применения снижения размерности~(\textit{7})} 
\label{Sb_n}
\end{figure*}


На рис.~\ref{dim_z_inform} изображена зависимость меры качества~$F_1$ от 
размерности вложения~$k$ при различных значениях эффективной размерности~$d$. 
%Пунктиром отмечено стандартное отклонение, срезанное по уровню единицы. 
На графиках видно, что качество значительно ухудшается при увеличении~$k$ 
независимо от соотношения~$k$ и~$d$. Эксперимент проведен при постоянных~$m \hm= 500$,
$n \hm= 20$, $S_b \hm= 100$, $S_s \hm= 400$ и~$\mu \hm= 150$.


\begin{figure*} %fig5
\vspace*{1pt}
\begin{center}
\mbox{%
\epsfxsize=155.148mm
\epsfbox{str-6.eps}
}
\end{center}
\vspace*{-9pt}
\Caption{Демонстрация вложения, выполненного предлагаемым методом 
((\textit{а})~Student; (\textit{б})~Softmax; (\textit{в})~PCA; (\textit{г})~Random)
 при $\mu \hm= 0$ (левый столбец) и~10 (правый столбец):
 \textit{1}~--- обучающая выборка; \textit{2}~--- тестовая
 выборка; \textit{3}~--- плагиат} 
\label{mu0}
\end{figure*}

На рис.~\ref{weight_x} изображена зависимость~$F_1$ от веса~$\mu$ меток класса~$y$ 
в~стартовой выборке при различных значениях размерности выборки~$n$. 
Из них можно сделать вывод, что качество классификации повышается с~ростом~$\mu$, 
при этом скорость роста падает с~ростом~$n$. Также видно, что разработанный метод 
при достаточно больших значениях~$\mu$ показывает в~среднем лучшие результаты 
среди всех рас\-смот\-рен\-ных методов снижения размерности, а~также превосходит 
по качеству классификацию в~исходном пространстве. Эксперимент проведен 
при постоянных~$m \hm= 500$,
$k \hm= 3$, $S_b \hm= 100$ и~$S_s \hm= 400$. 
В~этом эксперименте все исходные признаки были информативными.


Для исследования зависимости качества классификации от величины 
отношения размера стартовой части к~размеру выборки~$S_s/m$ был поставлен 
эксперимент, где при постоянных $n \hm= 6$, $k \hm= d \hm= 3$ 
и~$\mu \hm= 150$ исследовалась зависимость меры качест\-ва~$F_1$ от размера выборки~$m$ 
и~размера начального вложения~$S_s$. При этом размер дополнительно вкладываемых 
блоков~$S_b$ принимался заведомо б$\acute{\mbox{о}}$льшим размера выборки~$m$, 
так что дополняющая часть не разбивалась на блоки. На рис.~\ref{Ss_n} выведены 
результаты. Можно видеть, что зависимость от этих параметров незначительна. 
При этом скорость работы алгоритма увеличивается при наличии разбиений на 
стартовую и~дополняющую части. Таким образом, показано, что предложенная 
модификация алгоритма позволяет значительно ускорить его работу без 
существенного снижения качества.



На графиках рис.~\ref{Ss_n} также видно, что методы инициализации 
с~по\-мощью~(\ref{knnstud2}) и~случайной инициализации дают лучшие результаты, 
в~то время как метод инициализации PCA показал результаты порядка~0,5, 
по причине чего было принято решение не выносить его на рисунок.



Целью эксперимента, результаты которого приведены на рис.~\ref{Sb_n}, было 
исследование зависимости значения функции качества классификации от~$S_b$. 
Он был проведен при постоянных~$n \hm= 6$, $k \hm= d \hm= 3$, $\mu\hm = 150$
и~$S_s \hm= 200$. 
В~результате было обнаружено, что предлагаемый метод устойчив относительно 
параметра~$S_b$.

\subsection{Задача обнаружения внутреннего плагиата}

Целью данной части эксперимента был анализ предложенного метода 
снижения размерности в~применении к~реальным данным задачи внутреннего плагиата. 
Рассматривается набор документов. Каждый документ рассматривается как 
последовательность сегментов~$s_i$, каждый из которых описывается вектором 
признаков~$\mathbf x$. В~данной работе в~качестве сегментов рассматриваются 
предложения. Каждому~$s_i$ поставлена в~соответствие метка класса 
$ y_i \hm\in \{0, 1\}$: $y_i \hm= 1$, если~$s_i$~--- заимствованный сегмент, 
иначе~$y_i \hm= 0$. Задача распознавания внутреннего плагиата ставится 
как задача восстановления меток~$y_i$ по документу.

\vspace*{-7pt}

\paragraph*{Иллюстрация вложения реальных данных.}

Для демонстрации работы алгоритма на реальных данных из 
предоставленного корпуса~\cite{Pan2011Collection}~\verb"part1" выделен один 
из документов. Выделенные из него объекты были разделены на обучающую и~тестовую 
выборки. Каждой из них соответствуют непрерывные части текста. Это разделение 
необходимо для демонстрации работы предложенной модификации и~не учитывается 
при применении оригинального t-SNE. На рис.~6 приведен результат 
применения оригинального непараметрического метода t-SNE к~объектам, выделенным 
из выбранного документа. На нем видно, что объекты, соответствующие заимствованным 
частям текста, имеют очаги концентрации в~исходном пространстве, что 
свидетельствует об информативности выбранных признаков.
Рисунки~5 и~6 имеют безразмерные оси, полученные в~результате нелинейных отображений.
Физического смысла эти оси не несут.

 { \begin{center}  %fig6
 \vspace*{7pt}
 \mbox{%
\epsfxsize=73.201mm
\epsfbox{str-5.eps}
}


\end{center}


\noindent
{{\figurename~6}\ \ \small{Визуализация документа с~использованием оригинального алгоритма t-SNE:
\textit{1}~--- обучающая выборка; \textit{2}~--- тестовая выборка; 
\textit{3}~--- плагиат}}

}

\vspace*{10pt}






На рис.~5
представлены результаты вложения данных выбранного документа с~использованием 
предложенного алгоритма при различных методах начальной инициализации и~при 
различных значениях веса~$\mu$. При выполнении вложений были зафиксированы 
параметры~$S_s \hm= 500$ и~$S_b \hm= 200$. 
В~эксперименте данные из выбранного документа были разделены на обучающую и~тестовую 
выборки. В~тестовую часть попали образы предложений, которые образовывали в~исходном 
тексте непрерывную цепочку. Обучающая часть вкладывалась с~учетом ее 
разметки, а~тестовая~--- без учета.






\vspace*{-7pt}

\paragraph*{Результаты.}
Из полученных графиков можно сделать вывод, что предложенная модификация 
при больших значениях веса~$\mu$ принимает на себя часть ответственности 
за классификацию. Она склонна разделять и~кластеризовать тестовую выборку 
по целевому признаку. Таким образом, исходя из описанных выше свойств t-SNE, 
любой построенный в~результирующем пространстве классификатор получает 
свойство классификатора ближайших соседей с~адаптивной константой, 
подстраиваемой под локальную геометрию выборки.
Следует отметить также, что при больших значениях~$\mu$ минимизация целевой 
функции~(\ref{KL}) требует больше шагов градиентного алгоритма. Таким образом, 
этот параметр следует выбирать с~оглядкой на время работы программы. 
Авторы рекомендуют значение порядка характерной величины координат 
векторов обучающей выборки.

\vspace*{-8pt}

\section{Заключение}

\vspace*{-2pt}

В работе была предложена модификация непараметрического метода 
снижения размерности t-SNE, состоящая в~воплощении возможности\linebreak 
выполнения вложения поэтапно, решении проб-\linebreak лемы непросмотренных объектов и~внедрении 
возмож\-ности учета разметки при выполнении\linebreak вложения для классификации. Был проведен 
вычислительный эксперимент на синтетических данных, показывающий эффективность 
предложенного метода в~применении к~задаче классификации.
 Была определена зависимость 
качества класси-\linebreak фикации с~применением описанного метода от его па\-ра\-мет\-ров,
экспериментально обосновано использование поэтапного обучающего вложения. 
Полученные зна\-че\-ния качества сравнивались с~результатами классификации с~применением 
других методов снижения размерности, а~также без их применения.

Была показана устойчивость алгоритма к~введенным параметрам размера 
начальной части~$S_s$ и~максимального размера блоков~$S_b$, что облегчает его
 использование на практике. Также явно продемонстрирована зависимость свойств 
 метода от параметра веса разметки выборки~$\mu$.

Проанализировано признаковое пространство задачи внутреннего плагиата. 
Проиллюстрированы свойства предложенного алгоритма относительно данных задачи 
внутреннего плагиата. Продемонстрирована эффективность предложенных методов 
инициализации при вложении образов объектов, которые не были использованы 
при выполнении начального вложения.

\vspace*{-8pt}

{\small\frenchspacing
 {%\baselineskip=10.8pt
 \addcontentsline{toc}{section}{References}
 \begin{thebibliography}{99}
 
 \vspace*{-2pt}
 
\bibitem{fefferman2016testing}
\Au{Fefferman~C., Mitter~S., Narayanan~H.} 
Testing the manifold hypothesis~// J.~Am. Math. Soc., 2016. Vol.~29. No.\,4 
P.~983--1049.

\bibitem{maaten2008visualizing} %2
\Au{Van der Maaten~L., Hinton~G.} 
Visualizing data using \mbox{t-SNE}~// J.~Mach. Learn. Res., 2008. Vol.~9. 
P.~2579--2605.

\bibitem{narayanan2010sample} %3
\Au{Narayanan~H., Mitter~S.} 
Sample complexity of testing the manifold hypothesis~// Advances in neural 
information 
processing systems~/ Eds. J.\,D.~Lafferty, C.\,K.\,I.~Williams, J.~Shawe-Taylor, \textit{et al}.~---
Curran Associates, Inc., 2010. Vol.~23. P.~1786--1794.

\bibitem{zu2006intrinsic} %4
 \Au{Zu Eissen~S.\,M., Stein~B.} 
 Intrinsic plagiarism detection~// European Conference on Information Retrieval.~--- 
Springer,  2006. P.~565--569.

\bibitem{kuznetsov2016methods} %5
 \Au{Kuznetsov~M.\,P., Motrenko~A.\,P., Kuznetsova~M.\,V., Strijov~V.\,V.} 
 Methods for intrinsic plagiarism detection and author diarization~// 
 Working Notes  of CLEF~/ Eds.\ K.~Balog, L.~Cappellato, N.~Ferro, C.~Macdonald.~---
 $\acute{\mbox{E}}$vora, Portugal:
 CEUR-WS, 2016. Vol.~1609. P.~912--919.
 
 \bibitem{stamatatos2009intrinsic} %6
\Au{Stamatatos~E.} Intrinsic plagiarism detection using character n-gram profiles~// 
SEPLN Workshop on Uncovering Plagiarism, Authorship, and Social Software Misuse, 
2009. P.~38--46.

\bibitem{muhr2010external} %7
  \Au{Muhr~M., Kern~R., Zechner~M., Granitzer~M.} 
  External and intrinsic plagiarism detection using a~cross-lingual retrieval 
  and segmentation system~// Working Notes for CLEF Conference~/ Eds. M.~Braschler, 
  D.~Harman, E.~Pianta, N.~Ferro.~--- Padua, Italy: CEUR-WS, 2010. 
  Vol.~1176. 
  {\sf http://ceur-ws.org/Vol-1176/CLEF2010wn-PAN-MuhrEt2010.pdf}. 
%  (accessed September~15, 2017).



\bibitem{kestemont2011intrinsic} %8
  \Au{Kestemont~M., Luyckx~K., Daelemans~W.} 
  Intrinsic plagiarism detection using character trigram distance scores~// 
 Working Notes for CLEF Conference~/ Eds. V.~Petras, P.~Forner, P.~Clough, N.~Ferro.~--- 
 Amsterdam, The Netherlands: CEUR-WS, 2011. Vol.~1177. 
 {\sf http://ceur-ws.org/Vol-1177/CLEF2011wn-PAN-KestemontEt2011.pdf}. 
% (accessed September~15, 2017).


\bibitem{potthast2011overview} %9
\Au{Potthast~M., Eiselt~A., Cede$\tilde{\mbox{n}}$o~L.\,A., Stein~B., Rosso~P.}  
Overview of the 3rd international competition on plagiarism detection~// 
Working Notes for CLEF Conference~/ Eds. V.~Petras, P.~Forner, P.~Clough, N.~Ferro.~--- 
Amsterdam, The Netherlands: CEUR-WS, 2011. Vol.~1177. 
{\sf http://ceur-ws.org/Vol-1177/CLEF2011wn-PAN-PotthastEt2011a.pdf}. 



\bibitem{fodor2002survey} %10
 \Au{Fodor~I.\,K.} {A~survey of dimension reduction techniques}. 
 Center for Applied Scientific Computing, Lawrence Livermore National Laboratory, 
 2002. Technical Report. P.~1--18.

\bibitem{brooke2012paragraph} %11
  \Au{Brooke~J., Hirst~G.} Paragraph clustering for intrinsic plagiarism 
  detection using a stylistic vector-space model with extrinsic features~// 
  Working Notes for CLEF Conference~/ Eds. P.~Forner, J.~Karlgren, C.~Womser-Hacker, 
  N.~Ferro.~--- Rome, Italy: CEUR-WS, 2012. Vol.~1178. 
  {\sf http://ceur-ws.org/Vol-1178/CLEF2012wn-PAN-BrookeEt2012.pdf}. 
  %(accessed September~15, 2017).


\bibitem{brooke2012unsupervised} %12
  \Au{Brooke~J., Hammond~A., Hirst~G.} 
  Unsupervised stylistic segmentation of poetry with change curves and extrinsic 
  features~// 1st NAACL-HLT Workshop on Computational Linguistics for Literature
  Proceedings, 2012. Stroudsburg, PA, USA: Association for Computational Linguistics.
  P.~26--35.

\bibitem{gorban2008principal} %13
  \Au{Gorban~A.\,N., K$\acute{\mbox{e}}$gl~B., Wunsch~D.\,C., %Zinovyev~A.\,Y.,
  \textit{et al.}} Principal manifolds for data visualization and dimension 
  reduction.~--- Springer, 2008. 58~p.

\bibitem{tenenbaum2000global} %14
  \Au{Tenenbaum~J.\,B., De Silva~V., Langford~J.\,C.} 
  A~global geometric framework for nonlinear dimensionality reduction~// 
  Science, 2000. Vol.~290. Iss.~5500. P.~2319--2323.

\bibitem{belkin2001laplacian} %15
  \Au{Belkin~M., Niyogi~P.} 
  Laplacian eigenmaps and spectral techniques for embedding and clustering~// 
  Advances in neural information processing systems~/
  Eds.\ T.\,G.~Dietterich, S.~Becker, Z.~Ghahramani.~---
  NIPS Foundation, Inc., 2001. Vol.~14. P.~585--591.

\bibitem{roweis2000nonlinear} %16
  \Au{Roweis~S.\,T., Saul~L.\,K.} 
  Nonlinear dimensionality reduction by locally linear embedding~// 
  Science, 2000. Vol.~290. Iss.~5500. P.~2323--2326.

\bibitem{donoho2003hessian} %17
  \Au{Donoho~D.\,L., Grimes~C.} 
  Hessian eigenmaps: Locally linear embedding techniques for high-dimensional data~// 
  P. Natl. Acad. Sci. USA, 2003. Vol.~100. No.\,10. P.~5591--5596.

\bibitem{zhang2004principal} %18
  \Au{Zhang~Z., Zha~H.} Principal manifolds and nonlinear dimensionality reduction
   via tangent space alignment~// J.~Shanghai University (English Edition), 2004. 
   Vol.~8. No.\,4. P.~406--424.

\bibitem{weinberger2006unsupervised} %19
  \Au{Weinberger~K.\,Q., Saul~L.\,K.} 
  Unsupervised learning of image manifolds by semidefinite programming~// 
  Int. J.~Comput. Vision, 2006. Vol.~70. No.\,1. P.~77--90.

\bibitem{chen2010distance} %20
  \Au{Chen~C., Zhang~J., Fleischer~R.} 
  Distance approximating dimension reduction of Riemannian manifolds~// 
  IEEE T. Syst. Man Cy.~B, 2010. Vol.~40. No.\,1. 
  P.~208--217.

\bibitem{van2009learning} %21
\Au{Van der Maaten~L.} 
Learning a parametric embedding by preserving local structure~// 
RBM, 2009. Vol.~500. P.~26.

\bibitem{van2014accelerating} %22
  \Au{Van der Maaten~L.} Accelerating t-SNE using tree-based algorithms~// 
  J.~Mach. Learn. Res., 2014. Vol.~15. No.\,1. P.~3221--3245.

\bibitem{kim2007distance} %23
  \Au{Kim~H., Park~H., Zha~H.} 
  Distance preserving dimension reduction for manifold learning~// 
  SIAM  Conference (International) on Data Mining Proceedings, 2007. P.~527--532.

\bibitem{bottou2012stochastic} %24
  \Au{Bottou~L.} 
  Stochastic gradient descent tricks~// Neural networks: Tricks of the trade~/ 
Eds. G.~Montavon, G.\,B.~Orr, K.-R.~Muller.~--- 
Lecture notes in computer science ser.~--- 2nd ed.~--- Berlin--Heidelberg: 
Springer, 2012. Vol.~7700. P.~421--436.

 \bibitem{Pan2011Collection} %25
  \Au{Potthast~M., Stein~B., Barr$\acute{\mbox{o}}$n-Cede$\tilde{\mbox{n}}$o~A.,  
  Rosso~P.} 
  An evaluation framework for plagiarism detection~// 
  23rd  Conference (International) on Computational Linguistics Posters, 
  2010. P.~997--1005.
  \end{thebibliography}

 }
 }

\end{multicols}

\vspace*{-3pt}

\hfill{\small\textit{Поступила в~редакцию 20.02.17}}

\vspace*{8pt}

%\newpage

%\vspace*{-24pt}

\hrule

\vspace*{2pt}

\hrule

%\vspace*{8pt}


\def\tit{IMPROVING CLASSIFICATION QUALITY FOR~THE~TASK OF~FINDING INTRINSIC PLAGIARISM}

\def\titkol{Improving classification quality for~the~task of~finding intrinsic plagiarism}

\def\aut{I.\,O.~Molybog$^{1,2}$, A.\,P.~Motrenko$^2$, and~V.\,V.~Strijov$^3$}

\def\autkol{I.\,O.~Molybog, A.\,P.~Motrenko, and~V.\,V.~Strijov}

\titel{\tit}{\aut}{\autkol}{\titkol}

\vspace*{-9pt}


\noindent
$^1$Center for Energy Systems,
Skolkovo Institute of Science and Technology, 
Skolkovo Innovation Center, 3~Nobel\linebreak
$\hphantom{^1}$Str., Moscow 143026, 
Russian Federation

\noindent
$^2$Moscow Institute of Physics and Technology, 
9~Institutskiy Per., Dolgoprudny, Moscow Region 141700, Russian\linebreak
$\hphantom{^1}$Federation 

\noindent
$^3$A.\,A.~Dorodnicyn 
Computing Center, Federal Research Center ``Computer Science and Control'' 
of the Russian\linebreak
$\hphantom{^1}$Academy of Sciences, 40~Vavilov Str., Moscow 119333, 
Russian Federation



\def\leftfootline{\small{\textbf{\thepage}
\hfill INFORMATIKA I EE PRIMENENIYA~--- INFORMATICS AND
APPLICATIONS\ \ \ 2017\ \ \ volume~11\ \ \ issue\ 3}
}%
 \def\rightfootline{\small{INFORMATIKA I EE PRIMENENIYA~---
INFORMATICS AND APPLICATIONS\ \ \ 2017\ \ \ volume~11\ \ \ issue\ 3
\hfill \textbf{\thepage}}}

\vspace*{3pt}

\Abste{The paper addresses the classification problem in multidimensional spaces.
The authors propose a~supervised modification of the t-distributed Stochastic
Neighbor Embedding Algorithm. Additional features of the proposed modification
are that, unlike the original algorithm, it does not require retraining
if new data are added to the training set and can be easily parallelized. The
novel method was applied to detect intrinsic plagiarism in a collection of
documents. The authors also tested the performance of their algorithm using
synthetic data and showed that the quality of classification is higher with the
algorithm than without or with other algorithms for dimension reduction.}

\KWE{data analysis; dimension reduction; nonlinear dimension reduction; manifold learning; intrinsic plagiarism detection}


\DOI{10.14357/19922264170307} 

%\vspace*{-18pt}

\Ack
\noindent
This publication is funded by the Russian Foundation for Basic Research
(project No.\,16-07-01155).



%\vspace*{3pt}

  \begin{multicols}{2}

\renewcommand{\bibname}{\protect\rmfamily References}
%\renewcommand{\bibname}{\large\protect\rm References}

{\small\frenchspacing
 {\baselineskip=11.3pt
 \addcontentsline{toc}{section}{References}
 \begin{thebibliography}{99}


\bibitem{fefferman2016testing-1} %1
\Aue{Fefferman,~C., S.~Mitter, and H.~Narayanan.} 
2016. Testing the manifold hypothesis. \textit{J.~Am. Math. Soc.} 
29(4):983--1049. 

\bibitem{maaten2008visualizing-1} %2
\Aue{Van der Maaten, L., and G.~Hinton.} 
2008. Visualizing data using t-SNE. \textit{J.~Mach. Learn. Res.} 
9(Nov):2579--2605.

\bibitem{narayanan2010sample-1} %3
\Aue{Narayanan, H., and S.~Mitter.} 
2010. Sample complexity of testing the manifold hypothesis. 
\textit{Advances in neural 
information 
processing systems}. Eds. J.\,D.~Lafferty, C.\,K.\,I.~Williams, J.~Shawe-Taylor, 
\textit{et al}. Curran Associates, Inc. 23:1786--1794.

\bibitem{zu2006intrinsic-1} %4
 \Aue{Zu Eissen, S.~M., and B.~Stein.} 
 2006. Intrinsic plagiarism detection. \textit{European Conference on 
 Information Retrieval.} Springer. 565--569.

\bibitem{kuznetsov2016methods-1} %5
  \Aue{Kuznetsov, M.\,P., A.\,P.~Motrenko, M.\,V.~Kuznetsova, and V.\,V.~Strijov}. 
  2016. Methods for intrinsic plagiarism detection and author diarization. 
\textit{Working Notes  of CLEF}. Eds.\ K.~Balog, L.~Cappellato, N.~Ferro,
and C.~Macdonald.  $\acute{\mbox{E}}$vora, Portugal:
 CEUR-WS. 1609:912--919.
 
 \bibitem{stamatatos2009intrinsic-1} %6
 \Aue{Stamatatos, E.} 
 2009. Intrinsic plagiarism detection using character n-gram profiles. 
 \textit{SEPLN Workshop on Uncovering Plagiarism, Authorship, and Social 
 Software Misuse.} 38--46.


\bibitem{muhr2010external-1} %7
  \Aue{Muhr, M., R. Kern, M.~Zechner, and M.~Granitzer.} 
  2010. External and intrinsic plagiarism detection using a~cross-lingual retrieval
   and segmentation system. \textit{Working Notes for CLEF Conference}.
   Eds. M.~Braschler, 
  D.~Harman, E.~Pianta, and N.~Ferro. Padua, Italy: CEUR-WS. 
  Vol.~1176. Available at: 
  {\sf http://ceur-ws.org/Vol-1176/CLEF2010wn-PAN-MuhrEt2010.pdf} 
  (accessed September~15, 2017).


\bibitem{kestemont2011intrinsic-1} %8
  \Aue{Kestemont, M., K.~Luyckx, and W.~Daelemans.} 
  2011. Intrinsic plagiarism detection using character trigram distance scores. 
  \textit{Working Notes for CLEF Conference}. 
  Eds. V.~Petras, P.~Forner, P.~Clough, and N.~Ferro.
 Amsterdam, The Netherlands: CEUR-WS. Vol.~1177. 
Available at:  {\sf http://ceur-ws.org/Vol-1177/CLEF2011wn-PAN-\linebreak KestemontEt2011.pdf} 
 (accessed September~15, 2017).

\bibitem{potthast2011overview-1} %9
\Aue{Potthast, M., A.~Eiselt, L.\,A.~Cede$\tilde{\mbox{o}}$o, B.~Stein, and P.~Rosso.}
2011. Overview of the 3rd international competition on plagiarism detection. 
\textit{Working Notes for CLEF Conference}. Eds. V.~Petras, P.~Forner, P.~Clough, 
and N.~Ferro. Amsterdam, The Netherlands: CEUR-WS. Vol.~1177. 
Available at: {\sf http://ceur-ws.org/Vol-1177/CLEF2011wn-PAN-\linebreak PotthastEt2011a.pdf}
(accessed September~15, 2017).


\bibitem{fodor2002survey-1} %10
\Aue{Fodor, I.\,K.} 2002. A~survey of dimension reduction techniques. 
{Center for Applied Scientific Computing, Lawrence Livermore National 
Laboratory}. Technical Report. 1--18.

\bibitem{brooke2012paragraph-1} %11
  \Aue{Brooke, J., and G.~Hirst.} 2012. 
  Paragraph clustering for intrinsic plagiarism detection using a stylistic 
  vector-space model with extrinsic features. \textit{Working Notes for CLEF Conference}.
  Eds. P.~Forner, J.~Karlgren, C.~Womser-Hacker, 
  and N.~Ferro. Rome, Italy: CEUR-WS. Vol.~1178. 
Available at:   {\sf http://ceur-ws.org/Vol-1178/CLEF2012wn-PAN-BrookeEt2012.pdf}
(accessed September~15, 2017).

\bibitem{brooke2012unsupervised-1} %12
  \Aue{Brooke, J., A.~Hammond, and G.~Hirst.} 2012. Unsupervised stylistic 
  segmentation of poetry with change curves and extrinsic features. 
  \textit{1st NAACL-HLT Workshop on Computational Linguistics for Literature
  Proceedings}.  Stroudsburg, PA: Association for Computational Linguistics. 26--35.

\bibitem{gorban2008principal-1}
\Aue{Gorban, A.\,N., B.~K$\acute{\mbox{e}}$gl, D.\,C.~Wunsch, %A.\,Y.~Zinovyev, 
\textit{et al}.} 2008. \textit{Principal manifolds for data visualization 
and dimension reduction.} Springer. 58~p.

\bibitem{tenenbaum2000global-1} %14
  \Aue{Tenenbaum, J.\,B., V.~De Silva, and J.\,C.~Langford.} 2000. 
  A~global geometric framework for nonlinear dimensionality reduction. 
  \textit{Science} 290(5500):2319--2323.

\bibitem{belkin2001laplacian-1} %15
  \Aue{Belkin, M., and P.~Niyogi.} 
  2001. Laplacian eigenmaps and spectral techniques for embedding and clustering. 
  \textit{Advances in neural information processing systems}.
  Eds.\ T.\,G.~Dietterich, S.~Becker, and Z.~Ghahramani.
  NIPS Foundation, Inc. 14:585--591.

\bibitem{roweis2000nonlinear-1} %16
\Aue{Roweis, S.\,T., and L.\,K.~Saul.} 2000. 
Nonlinear dimensionality reduction by locally linear embedding. 
\textit{Science} 290(5500):2323--2326.

\bibitem{donoho2003hessian-1} %17
\Aue{Donoho, D.\,L., and C.~Grimes.} 2003. 
Hessian eigenmaps: Locally linear embedding techniques for high-dimensional data. 
\textit{P. Natl. Acad. Sci. USA} 100(10):5591--5596.

\bibitem{zhang2004principal-1}
  \Aue{Zhang, Z., and H.~Zha.} 
  2004. Principal manifolds and nonlinear dimensionality reduction via tangent space
   alignment. \textit{J.~Shanghai University (English Edition)} 
8(4):406--424.

\bibitem{weinberger2006unsupervised-1}
  \Aue{Weinberger, K.\,Q., and L.\,K.~Saul.} 
  2006. Unsupervised learning of image manifolds by semidefinite programming. 
  \textit{Int. J.~Comput. Vision} 70(1):77--90.

\bibitem{chen2010distance-1}
 \Aue{Chen, C., J.~Zhang, and R.~Fleischer}. 2010. 
 Distance approximating dimension reduction of Riemannian manifolds. 
 \textit{IEEE T. Syst. Man Cy.~B} 
 40(1):208--217.

\bibitem{van2009learning-1} %21
\Aue{Van der Maaten, L.} 2009. Learning a parametric embedding by preserving 
local structure. \textit{RBM} 500:26.

\bibitem{van2014accelerating-1}
  \Aue{Van der Maaten, L.} 2014. 
  Accelerating t-SNE using tree-based algorithms. 
  \textit{J.~Mach. Learn. Res.} 15(1):3221--3245.

\bibitem{kim2007distance-1} %23
\Aue{Kim, H., H.~Park, and H.~Zha.} 
2007. Distance preserving dimension reduction for manifold learning. 
\textit{SIAM  Conference (International) on Data Mining Proceedings.} 527--532.
{\looseness=1

}

\bibitem{bottou2012stochastic-1} %24
\Aue{Bottou, L.} 2012. Stochastic gradient descent tricks. 
\textit{Neural networks: Tricks of the trade}. 
Eds. G.~Montavon, G.\,B.~Orr, and K.-R.~Muller. 
Lecture notes in computer science ser. 2nd ed. Berlin--Heidelberg: 
Springer. 7700:421--436.

 \bibitem{Pan2011Collection-1}
 \Aue{Potthast, M., B.~Stein, A.~Barr$\acute{\mbox{o}}$n-Cede$\tilde{\mbox{n}}$o, 
 and P.~Rosso.} 2010. An evaluation framework for plagiarism detection. 
 \textit{23rd Conference (International) on Computational Linguistics 
 Posters.} 997--1005.
\end{thebibliography}

 }
 }

\end{multicols}

\vspace*{-3pt}

\hfill{\small\textit{Received February 20, 2017}}

\Contr

\noindent
\textbf{Molybog Igor O.} (b.\ 1995)~--- 
apprentice researcher, Skolkovo Institute of Science and Technology, 
Center for Energy Systems, Skolkovo Innovation Center, 3~Nobel Str., Moscow 143026, 
Russian Federation; student, Moscow Institute of Physics and Technology, 
9~Institutskiy Per., Dolgoprudny, Moscow Region 141700, Russian Federation; 
\mbox{i.molybog@skoltech.ru}

\vspace*{5pt} 

\noindent
\textbf{Motrenko Anastasia P.} (b.\ 1992)~--- 
PhD student, Moscow Institute of Physics and Technology, 
9~Institutskiy Per., Dolgoprudny, Moscow Region 141700, Russian Federation; 
\mbox{anastasiya.motrenko@phystech.edu}

\vspace*{5pt}

\noindent
\textbf{Strijov Vadim V.} (b.\ 1967)~--- 
Doctor of Science in physics and mathematics, leading scientist, A.\,A.~Dorodnicyn 
Computing Centre, Federal Research Center ``Computer Science and Control'' 
of the Russian Academy of Sciences, 40~Vavilov Str., Moscow 119333, 
Russian Federation; \mbox{strijov@ccas.ru}


\label{end\stat}


\renewcommand{\bibname}{\protect\rm Литература}   %7


\def\stat{safin}

\def\tit{ОПРЕДЕЛЕНИЕ ЗАИМСТВОВАНИЙ В ТЕКСТЕ БЕЗ~УКАЗАНИЯ~ИСТОЧНИКА$^*$}

\def\titkol{Определение заимствований в~тексте без указания источника}

\def\aut{К.\,Ф.~Сафин$^1$, М.\,П.~Кузнецов$^2$, М.\,В.~Кузнецова$^3$}

\def\autkol{К.\,Ф.~Сафин, М.\,П.~Кузнецов, М.\,В.~Кузнецова}

\titel{\tit}{\aut}{\autkol}{\titkol}

\index{Сафин К.\,Ф.}
\index{Кузнецов М.\,П.}
\index{Кузнецова М.\,В.}
\index{Safin K.\,F.}
\index{Kuznetsov M.\,P.}
\index{Kuznetsova M.\,V.}


{\renewcommand{\thefootnote}{\fnsymbol{footnote}} \footnotetext[1]
{Работа 
поддержана РФФИ (проект 16-07-01155).}}


\renewcommand{\thefootnote}{\arabic{footnote}}
\footnotetext[1]{Московский физико-технический институт; %(государственный  университет); 
ЗАО <<Анти-плагиат>>, \mbox{kamil.safin@phystech.edu}}
\footnotetext[2]{ООО <<Форексис>>, \mbox{mikhail.kuznecov@phystech.edu}}
\footnotetext[3]{Московский физико-технический институт; %(государственный  университет); 
ЗАО <<Анти-плагиат>>, \mbox{kuznetsova@ap-team.ru}}

%\vspace*{-18pt}



\Abst{Для задачи поиска заимствований в~тексте существуют два подхода: обнаружение 
<<внешних>> и~<<внутренних>> заимствований. При поиске внешних заимствований 
известен корпус, из которого возможны заимствования. При поиске внутренних 
заимствований исследуемый текст анализируется изолированно, т.\,е.\ возможные 
источники заимствований неизвестны. Данная работа посвящена поиску внутренних 
заимствований в~тексте. Предполагается, что большая часть текста написана одним 
автором. Необходимо выделить участки текста, написанные другим автором, если 
таковые имеются. В~работе предлагается алгоритм, строящий статистику сегментов 
текста, по которой определяется факт зависимости. Эксперимент проводится на 
коллекции конкурса  PAN-2011.}

\KW{обработка естественного языка; детектирование внутренних заимствований; поиск 
выбросов в~статистике}

\DOI{10.14357/19922264170308} 


\vskip 10pt plus 9pt minus 6pt

\thispagestyle{headings}

\begin{multicols}{2}

\label{st\stat}



\section{Введение}

Текстовые заимствования являются большой проблемой в~сфере образования и~научных 
исследований~\cite{Stud_plag}. Развитие сети Интернет, в~част\-ности,
и~информационных технологий, в~целом, сделало возможным некорректное заимствование 
информации.

В задаче обнаружения заимствований существуют два глобальных подхода: выявление 
<<внешних>> (external plagiarism detection) и~<<внутренних>> (intrinsic 
plagiarism detection) заимствований. При поиске внешних заимствований 
предполагается, что в~распоряжении исследователя есть некоторый корпус, из 
которого возможны заимствования. Таким образом, задача состоит в~попарном 
сравнении участков подозрительного текста и~текстов из корпуса заимствований.

Задача поиска <<внутренних>> заимствований состоит в~анализе исключительно 
подозрительного текста. Алгоритмы должны анализировать стиль письма и~выделять 
характерные признаки, свойственные данному автору.

Алгоритмы разбивают исходный текст на сегменты и~сравнивают текст сегмента со 
всем текстом. Разбиение проводится по предложениям~\cite{Graz, Innsbruck}, или 
же определяется окно заданной ширины, согласно которому производится 
сегментирование текс\-та~\cite{Aegean, Weimar, Chile, Valencia}. Выбор меры 
схожести сегмента со всем текс\-том или, наоборот, меры различия является ядром 
алгоритма. Работы~\cite{Graz, Innsbruck, Weimar} используют стилистические, 
синтаксические, лексические характеристики: частотность частей речи, порядок 
следования частей речи в~предложении, пунктуацию, среднюю длину предложения 
и~подобные признаки. Возможно использование символьных $n$-грамм (чаще других 
применяют 3-грам\-мы) в~качестве признака, 
а~точнее, час\-тот их использования~\cite{Aegean, Valencia}. 

Метод~\cite{Surrey} использует кластеризацию абзацев по 
час\-то\-те встречаемости существительных. 

В~статье \cite{AP} описан алгоритм 
диаризации текстов, т.\,е.\ классификации сегментов текста по авторству, что 
является обобщением задачи поиска внутренних заимствований. 

В~2011~г.\ был 
проведен конкурс PAN-2011, посвященный поиску заимствований в~текстах. 
Метод 
Oberreuter~\cite{Chile}, ставший победителем в~конкурсе PAN-2011, использует 
функцию, характеризующую письменный стиль автора. Функция строит вектор час\-тот 
встречаемости слов во всем документе и~в~выделенном сегменте. Эти векторы 
используются для определения величины отклонения сегмента от всего текста. 
Данный алгоритм показал результат~0,32 по F1-ме\-ре. Это демонстрирует тот факт, 
что алгоритма, решающего данную задачу в~большинстве случаев, до сих пор нет.

\begin{figure*}[b] %fig1
    \vspace*{-3pt}
\begin{center}
\mbox{%
\epsfxsize=164.328mm
\epsfbox{saf-1.eps}
}
\end{center}
\vspace*{-12pt}
\Caption{Распределение текстов по количеству заимствованных фрагментов~(\textit{а})
и~по доле заимствований~(\textit{б})} 
\end{figure*}
    

Предлагаемый алгоритм строит статистическое описание текста, которое 
используется для на\-хож\-де\-ния заимствованных сегментов текста. Статистика должна 
удовлетворять следующим условиям: на оригинальных сегментах текста иметь 
небольшой разброс значений по сравнению со значением на всем тексте, а на 
заимствованных сегментах иметь значительные отличия.

В работе используются данные конкурса PAN-2011 \cite{PAN}. Для оценки работы 
алгоритма определяются микро- и~макромеры качества precision и~recall (точность 
и~полнота) и~затем вычисляется F1-мера как среднее гармоническое для precision 
и~recall.
Данная мера представляет качество работы алгоритма.

\vspace*{-4pt}

\section{Постановка задачи}

Пусть $D$~--- коллекция текстовых документов, $d$~--- текстовый документ, 
$t_i$~--- сегмент текста $(d \hm= \bigcup t_{i}$, $d \hm\in D)$. Среди сегментов 
текста~$t_i$ необходимо выделить те, значение статистики которых $\sigma 
(\mathbf{t}_i)$ превосходит некоторый заданный порог значений~$\delta_{\mathrm{susp}}$.

\vspace*{-8pt}

\paragraph*{Описание выборки.}
В работе используется блок текстовых документов конкурса PAN-2011~\cite{PAN}. 
В~текстах присутствуют сегменты настоящих, имитированных и~искусственных 
заимствований. Каж\-дый сегмент текста соответственно полностью взят из другого 
источника, либо заимствованный текст переписан человеком другими словами, либо 
специально обученный алгоритм строит текст, стараясь повторить стиль автора.

Выборка состоит из~4753~текс\-тов, разделенных на~10~час\-тей, к~каждому из текстов 
прилагается файл с~экспертной разметкой заимствованных сегментов.

Для анализа корпуса была собрана подвыборка корпуса, состоящая из~30~документов, 
которые были просмотрены вручную. Анализ показал, что большая часть документов 
содержит в~себе заимствования, сильно отличающиеся от остального текста по 
тематике и~набору используемых слов. К~примеру, в~текст по экономике вставляется 
фрагмент, вырезанный из художественного текста.

Также тексты корпуса были исследованы на то, какая доля заимствований содержится 
в~каждом текс\-те (отношение длины заимствованных фрагментов к~длине текс\-та 
в~символах) и~сколько различных фрагментов заимствований присутствует в~текс\-те. На 
рис.~1 приведены гистограммы результатов.

Как видно, тексты в~среднем содержат от~1 до~7~фрагментов заимствований. 
В~большинстве текстов доля заимствований не превышает~4\%--5$\%$, что усложняет 
задачу поиска этих заимствований.

\vspace*{-6pt}

\paragraph*{Критерии качества.}
В экспериментах используют\-ся критерии качества, применявшиеся в~PAN-2011 \cite{PAN}. 
Обозначим за пару $(s, d)$ последовательность символов, помеченную 
экспертом как заимствование в~документе~$d$. $S \hm= \bigcup s_{i}$~--- 
совокупность всех заимствованных сегментов. За пару~$(r, d)$ обозначим 
последовательность, помеченную алгоритмом как заимствованную. Аналогично $R \hm= 
\bigcup r_{i}$~--- совокупность всех сегментов, которые алгоритм 
классифицировал как заимствованные. Рас\-смот\-рим меры качества {Precision} 
и~{Recall}:

\noindent
\begin{gather*}
\text{Prec}(S,R)=\fr{1}{|R|}\sum\limits_{r_j\in 
R}\fr{\left\vert \bigcup\limits_{s_i\in S}(s_i\cap r_j)\right\vert}{|r_j|}\,;\\ 
\text{Rec}(S,R)=\fr{1}{|S|}\sum\limits_{s_i\in S}\fr{\left\vert
\bigcup\limits_{r_j\in 
R}(s_i\cap r_j)\right\vert}{|s_i|}\,.
\end{gather*}

\noindent
Данные величины отражают точность (доля правильного распознавания заимствований 
по отношению ко всем выделенным сегментам) и~полноту (доля правильного 
распознавания заимствований по отношению ко всем заимствованиям в~тексте) работы 
алгоритма.

Вычисляется {F1-мера} как среднее гармоническое между {Precision} 
и~{Recall}:
$$
\mathrm{F}1(S,R) = \fr{\text{Prec} (S, R) \cdot \text{Rec} (S, R)}{\text{Prec} (S, R) + 
\mathrm{Rec}\, (S, R)}\,.
$$
Вычисляется величина гранулярности

\vspace*{2pt}

\noindent
$$
\mathrm{gran}\,(S,R)=\fr{1}{|S_R|}\sum\limits_{s_i\in S_R}|R_{s_i}|\,,
$$
где $S_R$~--- множество заимствованных сегментов, обнаруженных алгоритмом; 
$R_s$~--- сегменты, отмеченные алгоритмом, которые детектируют данный сегмент 
заимствований~$s$:

\vspace*{-2pt}

\noindent
\begin{gather*}
S_R = \lbrace s | s \in S \wedge \exists r \in R: r\ \mathrm{detects}\ s \rbrace\,;
\\
R_S = \lbrace r | r \in R  \wedge r\ \mathrm{detects}\ s\rbrace\,;
\\
r\ \mathrm{detects}\ s: \mathrm{if}\ r\cap s \neq \emptyset\,.
\end{gather*}

\vspace*{-2pt}

\noindent
Таким образом, гранулярность показывает то, насколько мелко алгоритм разбивает 
заимствованные сегменты текста. Если заимствованные сегменты разделяются 
алгоритмом на много мелких, то гранулярность будет иметь высокие значения.

По описанным величинам вычисляется итоговая мера качества {pladget}:
$$
\text{pladget}(S,R)=\fr{F1(S,R)}{\log_2 (1+\text{gran}(S,R))}\,.
$$

\vspace*{-10pt}

\paragraph*{Формальная постановка задачи.}
Для обнаружения заимствований исходный текст~$d$ разбивается на сегменты~$t_i$:
$$
d=\cup t_i\,.
$$
Для каждого сегмента вычисляется вектор признаков~$\mathbf{t}_i$ и~строится 
статистика~$\sigma(\mathbf{t}_i)$.
Затем происходит детектирование выбросов среди значений статистики на основании 
ее отклонения от среднего значения $$
\sigma_{\mathrm{avr}}(d) = 
\fr{1}{N}\sum\limits_{i=1}^N \sigma(\mathbf{t}_i)\,,
$$
 где $N$~--- число 
сегментов в~тексте. Если отклонение превышает заданный порог~$\delta_{\mathrm{susp}}$, то 
сегмент считается заимствованным:
$$
\left\vert \sigma(\mathbf{t}_i))-\sigma_{\mathrm{avr}}(d)\right\vert > \delta_{\mathrm{susp}}\,.
$$

Корпус документов~$D$ разбивается на обуча\-ющую и~тестовую выборки: 
$$
D=D_{\mathrm{test}}\cup D_{\mathrm{learn}}\,.
$$
При обучении параметры алгоритма~$\mathbf{w}$ настраиваются таким образом, чтобы 
улучшить меры качества работы алгоритма. При фиксированном способе разбиения 
текста мера {Granularity} не изменяется, так как она зависит от мелкости 
разбиения. Тогда для увеличения итоговой меры качества {Pladget} достаточно 
улучшить {F1-меру}:

\vspace*{-2pt}

\noindent
$$
\hat{\mathbf{w}}= \underset{\mathbf{w} \in \mathbf{W}}{\arg\max}\ \mathrm{F}1(S,  R)\,, 
$$

\vspace*{-2pt}

\noindent
т.\,е.\ требуется вектор параметров $\mathbf{w} \hm= (l_{\mathrm{segm}}, 
n, \delta_{\mathrm{susp}})$, 
максимизирующий {F1-меру}.
Здесь $l_{\mathrm{segm}}$~--- минимальная длина сегмента; $n$~--- ширина окна 
сглаживания; $\delta_{\mathrm{susp}}$~--- порог выброса.
 Более подробно параметры описаны 
в~вычислительном эксперименте (см.\ разд.~5).



\section{Базовый эксперимент}

Целью базового эксперимента ставилась проверка гипотезы о том, что 
заимствованные сегменты текста имеют отличные от среднего вектора значения 
признаков.

В качестве такого признака была выбрана час\-то\-та встречаемости слов. Каждому 
слову ставится в~соответствие число

\noindent
\begin{equation}
  \label{1-saf}
  \text{fr\_class}_w = \log_2 \fr{n_{\max}}{n_w}\,,
\end{equation}
где $n_{\max}$~--- число вхождений наиболее часто употребимого слова в~тексте; 
$n_w$~--- частота вхождений слова~$w$ в~этом предложении.

В качестве основного признака использовались квантили распределения данной 
величины внутри окна фиксированной ширины.

Обозначим за $m^j \hm=\overline{x^j}$ среднее значение $j$-го признака для 
рассматриваемого документа, за $r^j$~--- среднеквадратичное отклонение. Тогда 
нормализованный признак~$j$ для сегмента~$i$ рассчитывается по формуле:
$$
t_i^j=\fr{x^j-m^j}{r^j}\,. 
$$

За сегменты~$t_i$ были выбраны предложения текс\-та. Для каждого предложения~$t_i$ 
строился вектор признаков~$\mathbf{t}_i$ и~затем подсчитывалось отклонение от 
усредненного по всему текс\-ту вектора~$\mathbf{t}_{\mathrm{avr}}$ в~L1-мет\-рике:

\vspace*{-2pt}  

  \noindent
\begin{equation}
  \sigma(\mathbf{t}_i)=||\mathbf{t}_i-\mathbf{t}_{\mathrm{avr}}|| = 
\sum\limits_{j=1}^l|t_i^j-t_{\mathrm{avr}}^j|\,.
  \label{2-saf}
\end{equation}

\vspace*{-2pt}

Эксперимент проводился на одном из текстов конкурсной коллекции PAN-2011.

На рис.~2 показано отклонение признакового вектора каждого предложения от 
усредненного вектора. Пунктирными линиями выделены предложения, помеченные 
экспертом как заимствованные.

\begin{figure*} %fig2
    \vspace*{1pt}
\begin{center}
\mbox{%
\epsfxsize=160.138mm
\epsfbox{saf-3.eps}
}
\end{center}
\vspace*{-9pt}
\Caption{Отклонение признакового вектора от среднего}
\vspace*{-3pt}
\end{figure*}

Видно, что заимствованные фрагменты имеют характерные выбросы из области средних 
значений отклонения. Однако некоторые предложения, не являющиеся 
заимствованными, также сильно отличаются от усредненного признакового вектора. 
На основании этого можно сделать вывод, что использование только данного 
признака недостаточно для решения поставленной задачи.

\section{Описание алгоритма}

%\vspace*{-8pt}

\paragraph*{Модель.}
Предлагаемый алгоритм работает с~час\-тот\-ны\-ми признаками, предоставляющими 
описание текста. В~качестве такого признака выбран признак частоты встречаемости 
слов, описанный в~формуле~(\ref{1-saf}).

Исходный текст подвергается предобработке: удаляются служебные символы, все 
буквы переводятся в~нижний регистр. Также  из текста удаляются стоп-слова.

\vspace*{-8pt}

\paragraph*{Сегментирование текста.}
Текст разбивается на предложения. Затем формируется разбиение текста на 
сегменты~$t_i$: если длина очередного предложения
 меньше минимальной длины 
сегмента~$l_{\mathrm{segm}}$,\linebreak к~этому предложению добавляется следующее за ним~--- процесс 
повторяется, пока длина сегмента~$t_i$ не превысит заданную минимальную длину. 
Минимальная длина сегмента~$l_{\mathrm{segm}}$ является настраиваемым параметром 
алгоритма.

\vspace*{-8pt}

\paragraph*{Построение статистики и~детектирование аномалий.}
Для каждого сегмента~$t_i$ текста строится  вектор признаков. Затем строится 
статистика~$\sigma (\mathbf{t}_i)$ на основе\linebreak\vspace*{-12pt}

\columnbreak

\noindent
 отклонения вектора признаков от 
усредненного по всему тексту вектора~(\ref{2-saf}).

Полученная статистика сглаживается методом скользящего среднего: новые значения 
статистики~$\sigma'(\mathbf{t}_i)$ вычисляются по формуле:

\noindent
$$
\sigma'(\mathbf{t}_i) = \fr{1}{2n+1}\sum\limits_{k=i-
n}^{i+n}\sigma(\mathbf{t}_i)\,,
$$
где $n$~--- ширина сглаживания, которая также является настраиваемым параметром. 
Значения в~крайних точках вычисляются по формулам ($N$~--- число сегментов):

\noindent
\begin{align*}
\sigma'(\mathbf{t}_i) &= 
\fr{1}{i+n+1}\sum\limits_{k=0}^{i+n}\sigma(\mathbf{t}_i)\,;
\\
\sigma'(\mathbf{t}_i)& = \fr{1}{i+n+1}\sum\limits_{k=i- n}^{N}\sigma(\mathbf{t}_i)\,.
\end{align*}

Полученные значения статистики $\sigma'(\mathbf{t}_i)$ исследуются на выбросы. 
Если в~ряде статистики присутствует аномалия, превышающая заданный 
порог~$\delta_{\mathrm{susp}}$, то сегмент~$t_i$, отвечающий этому выбросу, помечается как 
заимствованный.

Минимальная длина сегмента, ширина окна сглаживания и~порог выброса 
настраиваются на обучающей выборке путем максимизации {F1-меры}.

\vspace*{-6pt}

\section{Вычислительный эксперимент}

Алгоритм настраивался на частях~1--5 корпуса PAN-2011 путем максимизации 
{F1-ме\-ры}. Тестирование проводилось на частях~6--10 корпуса.
Оптимальные параметры после настройки: $\hat{l}_{\mathrm{segm}}\hm=450$; $\hat{n}\hm=8$; 
$\hat{\delta}_{\mathrm{susp}}\hm=0{,}37$.



На рис.~3 и~4 приведены примеры работы алгоритма. Серые участки обозначают 
заимствованные %\linebreak\vspace*{-12pt}
 фрагменты, пунктирными линиями обозначен порог выброса значений стилевой 
функции.

%\pagebreak

\end{multicols}

\begin{figure*} %fig3
    \vspace*{1pt}
\begin{center}
\mbox{%
\epsfxsize=163.257mm
\epsfbox{saf-4.eps}
}
\end{center}
\vspace*{-9pt}
\Caption{Результаты на обучающей выборке}
%\end{figure*}
%\begin{figure*} %fig4
    \vspace*{12pt}
\begin{center}
\mbox{%
\epsfxsize=162.463mm
\epsfbox{saf-5.eps}
}
\end{center}
\vspace*{-9pt}
\Caption{Результаты на тестовой выборке}
\end{figure*}

\begin{table*}\small
\vspace*{6pt}
\begin{center}


\begin{tabular}{|c|c|c|c|c|c|}
\multicolumn{6}{c}{Сравнение качества алгоритмов на корпусе  PAN-2011}\\
\multicolumn{6}{c}{\ }\\[-6pt]
\hline
Алгоритм & Precision & Recall& F1& Granularity& Pladget\\
\hline
 Предлагаемый авторами &  0,27& 0,28& 0,28& 1,04& 0,28\\
 Oberreuter & 0,34& 0,31& 0,33& 1,00& 0,33\\
 Kestemont& 0,11& 0,43& 0,17& 1,03& 0,17\\
\hline
\end{tabular}
\end{center}
\vspace*{6pt}
\end{table*}

\begin{multicols}{2}

%\noindent




Результаты работы и~сравнение с~двумя алгоритмами приведены в~таблице.



Описанный алгоритм использует частоты распределения слов. Сегментирование текста 
происходит по группам предложений. Для определения заимствованных фрагментов 
исследуются значения статистики каждого сегмента.

На корпусе PAN-2011 алгоритм показал сравнимые результаты с~победителем конкурса~--- 
алгоритмом Oberreuter. Также было проведено сравнение с~алгоритмом 
Kestemont'a, занявшим второе место на конкурсе. Качество работы предлагаемого 
алгоритма значительно превышает качество работы алгоритма Kestemont'a.


\section{Анализ ошибок}

Результаты работы предлагаемого алгоритма зави\-сят от длины документа. При 
анализе небольших по объему текстов сглаживание приводит к~существенной потере 
информации об аномальных значениях статистики. При малой ширине сглаживания 
шумовые выбросы вызывают ложное срабатывание алгоритма.

\vspace*{-14pt}

\section{Заключение}

Предлагаемый алгоритм использует распределение частот слов внутри текста для 
нахождения заимствованных сегментов. Сегментирование текста осуществляется по 
группам предложений. Для каждого сегмента строится статистика. Затем ряд 
статистики для всего текста  сглаживается методом скользящего среднего. 
Полученные значения исследуются на отклонение от среднего значения для выявления 
заимствованных сегментов.

Алгоритм был настроен и~протестирован на корпусе PAN-2011. Алгоритм 
Oberreuter~\cite{Chile}, модификацией которого является предлагаемый алгоритм, показал на 
этом же корпусе результаты в~0,32 по {F1-мере}. Таким образом, описанный 
алгоритм показал сравнимые результаты при работе с~тем же корпусом документов.

Дальнейшие исследования могут быть на\-прав\-ле\-ны на более точную настройку 
параметров алгоритма, подбор параметров в~зависимости от длины рассматриваемого 
текста, поиск новых признаков, которые будут точнее выявлять заимствованные\linebreak 
фрагменты, а~также поиск новых способов на\-хож\-де\-ния заимствований.

\bigskip

Авторы выражают свою благодарность доктору фи\-зи\-ко-ма\-те\-ма\-ти\-че\-ских 
наук В.\,В.~Стрижову, а~также 
кандидату фи\-зи\-ко-ма\-те\-ма\-ти\-че\-ских наук Ю.\,В.~Чеховичу 
за ценные советы при планировании исследования и~рекомендации по оформлению статьи.



{\small\frenchspacing
 {%\baselineskip=10.8pt
 \addcontentsline{toc}{section}{References}
 \begin{thebibliography}{99}
\bibitem{Stud_plag}
  \Au{Никитов А.\,В., Орчаков О.\,А., Чехович~Ю.\,В.} Плагиат в~работах 
студентов и~аспирантов: проблема и~методы противодействия~// Университетское 
управление: практика и~анализ, 2012. №\,5. С.~61--68.

\bibitem{Graz} %2
\Au{Zechner M., Muhr~M., Kern~R., Granitzer~M.} External and intrinsic 
plagiarism detection using vector space models~// CEUR Workshop Proceedings, 2009. 
Vol.~502. P.~47--55.

\bibitem{Innsbruck} %3
  \Au{Tschuggnall M., Specht G.} Countering plagiarism by exposing 
irregularities in authors grammars~// European Intelligence and Security 
Informatics Conference.~--- IEEE, 2013. P.~15--22.



\bibitem{Weimar} %4
\Au{Eissen S.\,M., Stein B.} Intrinsic plagiarism detection~// 
Advances in information retrieval~/ Eds.\ M.~Lalmas, A.~MacFarlane, 
S.\,M.~R$\ddot{\mbox{u}}$ger,
\textit{et al.}~--- Lecture 
notes in computer science ser.~--- Springer, 2006. Vol.~3936. 
P.~565--569.

\bibitem{Aegean} %5
\Au{Stamatatos E.} Intrinsic plagiarism detection using character 
$n$-gram profiles~// CEUR Workshop Proceedings, 2009. Vol.~502. P.~38--46.

\bibitem{Chile} %6
\Au{Oberreuter G., L'Huillier~G., R$\acute{\!\!\mbox{{\ptb{\i}}}}$os~S.\,A., Vel$\acute{\mbox{a}}$squez~J.\,D.} 
Outlier-based approaches for intrinsic and external plagiarism detection~// 
Knowlege-based and intelligent 
information and engineering systems~/
Eds. A.~K$\ddot{\mbox{o}}$nig, A.~Dengel, K.~Hinkelmann, \textit{et al.}~--- Lecture 
notes in computer science ser.~--- Springer, 
2011. Vol.~6882.  P.~11--20.

\bibitem{Valencia} %7
\Au{Bensalem I. Rosso~P., Chikhi~S.} Intrinsic plagiarism detection 
using $n$-gram classes~// Conference on Empirical Methods 
in Natural Language Processing Proceedings.~--- 
Stroudsburg, PA, USA: 
Association for Computational Linguistics, 2014. P.~1459--1464.

\bibitem{Surrey}
\Au{Vartapetiance A., Gillam~L.} Quite simple approaches for authorship 
attribution, intrinsic plagiarism detection and sexual predator identification. 
{\sf http:// epubs.surrey.ac.uk/id/eprint/766727}.

\bibitem{AP}
\Au{Kuznetsov M., Motrenko A., Kuznetsova R., Strijov V.} Methods for 
intrinsic plagiarism detection and author diarization. 
{\sf http://ceur-ws.org/Vol-1609/16090912.pdf}.

\bibitem{PAN}
\Au{Potthast M., Stein~B., Barron-Cedeno~A., Rosso~P.} An evaluation 
framework for plagiarism detection~// 23rd  
Conference (International) on Computational Linguistics Proceedings.~--- 
Stroudsburg, PA, USA: Association for Computational 
Linguistics, 2010. P.~997--1005.
 \end{thebibliography}

 }
 }

\end{multicols}

\vspace*{-6pt}

\hfill{\small\textit{Поступила в~редакцию 30.01.17}}

\vspace*{8pt}

%\newpage

%\vspace*{-24pt}

\hrule

\vspace*{2pt}

\hrule

\vspace*{-2pt}


\def\tit{METHODS FOR~INTRINSIC PLAGIARISM DETECTION\\[-5pt]}

\def\titkol{Methods for intrinsic plagiarism detection}

\def\aut{K.\,F.~Safin$^{1,2}$, M.\,P.~Kuznetsov$^3$, and~M.\,V.~Kuznetsova$^{1,2}$\\[-5pt]}

\def\autkol{K.\,F.~Safin, M.\,P.~Kuznetsov, and~M.\,V.~Kuznetsova}

\titel{\tit}{\aut}{\autkol}{\titkol}

\vspace*{-14pt}


\noindent
$^1$Moscow Institute of Physics and Technology, 9~Institutskiy Per., 
Dolgoprudny, Moscow Region 141700, Russian\linebreak
$\hphantom{^1}$Federation

\noindent
$^2$Antiplagiat JSC, 33~Varshavskoe Shosse, 
Moscow 117105, Russian Federation

\noindent
$^3$``Forecsys'' LLC, 42~Vavilov Str., Moscow 119333, Russian Federation



\def\leftfootline{\small{\textbf{\thepage}
\hfill INFORMATIKA I EE PRIMENENIYA~--- INFORMATICS AND
APPLICATIONS\ \ \ 2017\ \ \ volume~11\ \ \ issue\ 3}
}%
 \def\rightfootline{\small{INFORMATIKA I EE PRIMENENIYA~---
INFORMATICS AND APPLICATIONS\ \ \ 2017\ \ \ volume~11\ \ \ issue\ 3
\hfill \textbf{\thepage}}}

\vspace*{2pt}


\Abste{There are two ways to find plagiarism in documents: ``external'' 
and ``intrinsic'' plagiarism detection. External plagiarism detection is the task 
with a known set of possible references. Intrinsic plagiarism detection aims at 
discovering plagiarism by analyzing only the document by itself. The paper 
investigates the  methods of intrinsic plagiarism detection. The authors developed 
a~plagiarism detection method based on constructing statistics from the features 
of the document parts and detecting outliers. The proposed algorithm was tested on 
the PAN-2011 collection for intrinsic plagiarism detection.}

\KWE{natural language processing; intrinsic plagiarism detection; outliers detection}


\DOI{10.14357/19922264170308} 

%\vspace*{-18pt}

\Ack
\noindent
The work was supported by the Russian Foundation for Basic Research (project 16-07-01155).



%\vspace*{3pt}

  \begin{multicols}{2}

\renewcommand{\bibname}{\protect\rmfamily References}
%\renewcommand{\bibname}{\large\protect\rm References}

{\small\frenchspacing
 {%\baselineskip=10.8pt
 \addcontentsline{toc}{section}{References}
 \begin{thebibliography}{99}

\bibitem{1-saf}
\Aue{Nikitov, A.\,V., O.\,A.~Orchakov, and Ju.\,V.~Chehovich.} 
2012. Plagiat v~rabotakh studentov i~aspirantov: 
Problema i~metody protivodeystviya [Plagiarism in works of undergraduate 
and graduate students: Problem and methods of counteraction]. 
\textit{Universitetskoe upravlenie: Praktika i~analiz} 
[University Management: Practice and Analysis] 5:61-68.

\bibitem{2-saf}
\Aue{Zechner, M., M.~Muhr, R.~Kern, and M.~Granitzer.} 2009. 
External and intrinsic plagiarism detection using vector space models. 
\textit{CEUR Workshop Proceedings}.
 502:47--55.

\bibitem{3-saf} %3
\Aue{Tschuggnall, M., and G.~Specht.} 2013. 
Countering pla\-gi\-arism by exposing irregularities in authors grammars. 
\textit{European Intelligence and Security Informatics Conference 
 Proceedings}. IEEE. 15--22.




\bibitem{6-saf} %4
\Aue{Eissen, S.\,M., and B.~Stein.} 2006. 
Intrinsic plagiarism detection. 
\textit{Advances in information retrieval}.
Eds.\ M.~Lalmas, A.~MacFarlane, 
S.\,M.~R$\ddot{\mbox{u}}$ger,
\textit{et al.}
Lecture notes in computer science ser. Springer. 3936:565--569.

\bibitem{4-saf} %5
\Aue{Stamatatos, E.} 2009. 
Intrinsic plagiarism detection using character $n$-gram profiles. 
\textit{CEUR Workshop Proceedings}. 
502:38--46.

\bibitem{7-saf} %6
\Aue{Oberreuter, G., G.~L'Huillier, S.~R$\acute{\mbox{{\!\ptb{\i}}}}$os, 
and~J. Vel$\acute{\mbox{a}}$squez}. 2011. 
Outlier-based approaches for intrinsic and external plagiarism detection. 
\textit{Knowlege-based and intelligent information and engineering systems}.
Eds.\ A.~K$\ddot{\mbox{o}}$nig, A.~Dengel, K.~Hinkelmann, \textit{et al.}
Lecture notes in computer science ser. Springer. 6882:11--20.

\bibitem{8-saf} %7
\Aue{Bensalem, I., P.~Rosso, and S.~Chikhi.} 2014. 
Intrinsic plagiarism detection using $n$-gram classes. 
\textit{Conference on Empirical Methods in Natural Language Processing}.
 Stroudsburg, PA: Association for Computational Linguistics. 1459--1464.

\bibitem{9-saf} %8
\Aue{Vartapetiance, A., and L.~Gillam.}
Quite simple approaches for authorship attribution, intrinsic plagiarism detection
 and sexual predator identification. 
 Avail-\linebreak able at: 
 {\sf http://epubs.surrey.ac.uk/id/eprint/766727}\linebreak (accessed September~23, 2013).

\bibitem{10-saf} %9
\Aue{Kuznetsov, M., A.~Motrenko, R.~Kuznetsova, and V.~Strijov}. 
Methods for intrinsic plagiarism detection and author diarization. 
Available at: 
{\sf http://ceur-ws.org/Vol-1609/16090912.pdf} (accessed September~6, 2016).

\bibitem{11-saf} %10
\Aue{Potthast, M., B.~Stein, A.~Barr$\acute{\mbox{o}}$n-Cede$\tilde{\mbox{n}}$o, 
and P.~Rosso.} 2010.
 An evaluation framework for plagiarism detection. 
 \textit{23rd  Conference (International) on Computational Linguistics 
 Proceedings}. Stroudsburg, PA: 
 Association for Computational Linguistics. 997--1005.
\end{thebibliography}

 }
 }

\end{multicols}

\vspace*{-3pt}

\hfill{\small\textit{Received January 30, 2017}}


\Contr

\noindent
\textbf{Safin Kamil F.} (b.\ 1995)~--- 
student, Moscow Institute of Physics and Technology, 
9~Institutskiy Per., Dolgoprudny, Moscow Region 141700, Russian Federation; 
junior researcher, Antiplagiat JSC,  33~Varshavskoe Shosse, 
Moscow 117105, Russian Federation; \mbox{kamil.safin@phystech.edu}

\noindent
\textbf{Kuznetsov  Mikhail P.} (b.\ 1989)~--- 
Candidate of Science (PhD) in physics and mathematics; 
analyst, ``Forecsys'' LLC, 42~Vavilov Str.,  
Moscow 119333, Russian Federation; \mbox{mikhail.kuznecov@phystech.edu} 


\noindent
\textbf{Kuznetsova Margarita V.} (b.\ 1990)~---
PhD student, Moscow Institute of Physics and Technology, 
9~Institutskiy Per., Dolgoprudny, Moscow Region 141700, Russian Federation; 
Head of Departament, Antiplagiat JSC,  33~Varshavskoe shosse, 
Moscow 117105, Russian Federation; \mbox{kuznetsova@ap-team.ru}

\label{end\stat}


\renewcommand{\bibname}{\protect\rm Литература}      %8
\def\stat{sigov}

\def\tit{ПСИХОЛИНГВИСТИЧЕСКИЙ АНАЛИЗ\\ РУССКОЯЗЫЧНЫХ ТЕКСТОВЫХ СООБЩЕНИЙ\\ 
НА~ОСНОВЕ ИХ ФОНОСЕМАНТИЧЕСКИХ\\ СТАТИСТИЧЕСКИХ ХАРАКТЕРИСТИК$^*$}

\def\titkol{Психолингвистический анализ русскоязычных текстовых сообщений 
на~основе их фоносемантических %статистических 
характеристик}

\def\aut{А.\,С.~Сигов$^1$, Д.\,А.~Акимов$^2$, Д.\,О.~Жуков$^3$, 
Е.\,Г.~Андрианова$^4$, В.\,Е.~Сачков$^5$, В.\,К.~Раев$^6$}

\def\autkol{А.\,С.~Сигов, Д.\,А.~Акимов, Д.\,О.~Жуков и~др.} 
%Е.\,Г.~Андрианова$^4$, В.\,Е.~Сачков$^5$, В.\,К.~Раев$^6$}

\titel{\tit}{\aut}{\autkol}{\titkol}

\index{Сигов А.\,С.}
\index{Акимов Д.\,А.}
\index{Жуков Д.\,О.}
\index{Андрианова Е.\,Г.}
\index{Сачков В.\,Е.}
\index{Раев В.\,К.}
\index{Sigov A.\,S.}
\index{Akimov D.\,A.}
\index{Zhukov D.\,O.}
\index{Andrianova E.\,G.} 
\index{Sachkov V.\,E.}
\index{Raev V.\,K.}


{\renewcommand{\thefootnote}{\fnsymbol{footnote}} \footnotetext[1]
{Работа выполнена за счет финансирования Министерством образования и~науки Российской Федерации 
конкурсной части государственных заданий высшим учебным заведениям и~научным организациям по 
выполнению инициативных научных проектов (№\,28.2635.2017/ПЧ).}}


\renewcommand{\thefootnote}{\arabic{footnote}}
\footnotetext[1]{Московский технологический университет (МИРЭА), \mbox{assigov@yandex.ru}}
\footnotetext[2]{Московский технологический университет (МИРЭА), \mbox{akimov\_d@mirea.ru}}
\footnotetext[3]{Московский технологический университет (МИРЭА), \mbox{zhukovdm@yandex.ru}}
\footnotetext[4]{Московский технологический университет (МИРЭА), \mbox{dtghmflysq@gmail.com}}
\footnotetext[5]{Московский технологический университет (МИРЭА), \mbox{megawatto@mail.ru}}
\footnotetext[6]{Московский технологический университет (МИРЭА), \mbox{raev@mirea.ru}}
  
  %\vspace*{-18pt}

\Abst{Рассматривается проблема идентификации типа акцентуации паттерна 
поведения виртуального субъекта в~сети Интернет и~социальных сетях на основе 
статистического анализа текстов, что позволяет сформулировать гипотезу о~структурных 
свойствах его коммуникаций и~позволяет построить матрицу вероятностей для отношений 
между виртуальными масками субъектов. Тексты пользователей 
рассматриваются как сложные се\-ман\-ти\-ко-син\-так\-си\-че\-ские образования, 
обладающие рядом психолингвистических характеристик. К~их числу относятся цельность, 
а~также смысловая направленность сообщения. Кроме того, в~тексте, рассматриваемом как 
продукт речевой деятельности, обладающий большой степенью семантической 
вариативности, определяемой его темпоральными и~сонарными характеристиками, 
проявляется невербальный характер поведения сетевых субъектов~--- виртуальных масок 
и~роботизированных агентов. Практическая значимость предлагаемого решения для 
психолингвистического анализа строится на возрастающем значении развития системы 
условных знаков, в~данном случае условных языков е-ком\-му\-ни\-ка\-ции, для порождения, 
в~свою очередь, управляющих кластеров, регулирующих социальное поведение 
виртуальных субъектов в~Сети. Это предположение строится на гипотезе Кеннета Айверса, 
в~соответствии с~которой чем лучше развита система условных знаков, тем больше 
возможностей она дает для создания новых алгоритмов.}

\KW{психолингвистические характеристики; невербальное поведение; виртуальные маски; 
процесс мышления; семантический смысл; лингвистический релятивизм}

\DOI{10.14357/19922264170309} 


\vskip 10pt plus 9pt minus 6pt

\thispagestyle{headings}

\begin{multicols}{2}

\label{st\stat}


\section{Введение}

  В настоящее время большой интерес разработчиков информационных сетей 
вызывает анализ социального аспекта информационного массива (потока) 
популярных ин\-тер\-нет-ре\-сур\-сов, воздействие которых изменяет 
семантический смысл и~оказывает управляющее воздействие на виртуальных 
субъектов, равно как и~на их кластеры непосредственного взаимодействия. 
К~такой информации относятся данные, отражающие мнения, тенденции, 
настроения и~интересы, преобладающие среди субъектов е-со\-об\-ще\-ства.
  
  На взгляд авторов, решение таких задач возможно только за счет 
использования междисциплинарных подходов, в~которых методы 
теоретической информатики должны быть дополнены моделями 
математической лингвистики естественных языков.
  
  Теоретическим обоснованием рассматриваемой проблемы является гипотеза 
лингвистической относительности, которая предполагает, что структура языка 
коммуникации влияет на ментальность пользователей социальных сетей~[1] 
и~опосредованно на когнитивные процессы мышления последних. 

Воспользуемся нечеткой трактовкой гипотезы Се\-пи\-ра--Уор\-фа~[2], 
в~соответствии с~которой процессы мышления, а также используемые 
в~письменной/устной речи лингвистические категории определяются при 
е-ком\-му\-ни\-ка\-ции как некая форма неязыкового поведения.
  
  Используемый принцип Уорфа, равно как и~позднее сформулированная 
гипотеза Р.~Брауна и~Э.~Леннеберга~[3] в~отношении цветового восприятия, 
определяющая разницу в~восприятии цветового зрения в~различных языках, 
носит релятивистский характер, равно как и~конструктивистский подход, 
предполагающий, что свойства проявления черт человеческой психики и~общие 
идеи самопроявления в~коммуникации в~значительной степени подвержены 
влиянию категорий, сформированных субъектами в~процессе социализации, 
и~не зависят от биологических ограничений.
  
  В~антологии~[4] 
исследователи лингвистического релятивизма сделали попытку определить 
связи и~границы между мышлением, познанием, языком и~культурой, описать 
степень и~виды взаимосвязанности и~взаимовлияния. Слобин~[5] 
задавал когнитивный процесс <<мышление для речи>> как вид процесса, 
в~котором перцептивные данные и~другие виды долингвистического мышления 
переводятся в~лингвистические категории для коммуникации с~другими 
субъ\-ек\-тами. 
  
  Джон Люси выделил основные направления исследований лингвистического 
релятивизма, и~в том числе %{\sf https://ru.wikipedia.org/wiki/\%D0 \%93\%D0\%B8\%D0\%BF\%D0\%BE\%D1\%82\%D0\%B5\%D0\%B7\%D0\%B0\_\%D0\%BB\%D0\%B8\%D0\%BD\%D0\%B3\%D0\%B2\%D0\%B8\%D1\%81\%D1\%82\%D0\%B8\%D1\%87\%D0\%B5\%D1\%81\%D0\%BA\%D0\%BE\%D0\%B9\_\%D0\%BE\%D1\%82\%D0\%BD\%D0\%BE\%D1\%81\%D0\%B8\%D1\%82\%D0\%B5\%D0\%BB\%D1\%8C\%D0\%BD\%D0\%BE\%D1\%81\%D1\%82\%D0\%B8-cite\_note-50} 
<<областной>> 
подход. При этом подходе выбирается отдельная семантическая область 
и~сравнивается у~различных лингвистических и~культурных групп (в~данном 
случае групп пользователей и~отдельных виртуальных субъектов) 
с~\mbox{целью} обнаружения корреляции между лингвистическими 
средствами, которые используются в~языке для обозначения тех или иных 
понятий, и~характером поведения. С~по\-мощью комбинации вышеуказанных 
подходов и~теоретических положений был проведен расчет квалиметрических 
характеристик процесса коммуникации в~сети Интернет на основе анализа 
отношения (матрицы отношений) виртуальных идентичностей пользователей 
сети Интернет к~тем или иным событиям, явлениям и~персонам (субъектам 
социальной значимости) реального мира. Также учитывалась степень 
взаимовлияния виртуальных идентичностей или групп идентичностей.
  
  В рамках проведенного исследования проверялась гипотеза о~том, что 
виртуальная идентичность формируется на основе совокупности отношений 
пользователя к~тем или иным сетевым событиям и~является формой проявления 
отношений пользователей между собой, а~также доступной информации или 
источниками информации, представленными в~Сети.
  
  Для решения задачи идентификации поведения виртуальной идентичности 
моделировался некий процесс, в~котором пользователь, участвуя во всех 
информационных взаимодействиях, условно проявляет свои личностные 
качества посредством из-\linebreak\vspace*{-12pt}

 { \begin{center}  %fig1
 \vspace*{-3pt}
 \mbox{%
\epsfxsize=77.575mm
\epsfbox{sig-1.eps}
}

\end{center}


\noindent
{{\figurename~1}\ \ \small{Методологические конструкты и~инструментарий анализа сетевых событий}}

}

\vspace*{9pt}

\addtocounter{figure}{1}



\noindent
бранного паттерна и~маски поведения. Исходя из\linebreak
 этого, 
строилась модель <<псевдоличности>> (или\linebreak виртуальный образ) 
с~последующей ее идентификацией в~одном или нескольких кластерах сети 
Ин\-тер\-нет. В~исследовании была использована идея\linebreak дополне\-ния 
лингвистического анализа сетевого поведения субъектов статистическим 
анализом (рис.~1), выбраны методы анализа, определена область 
практического использования результатов, обоснована репрезентативность 
разработанной методики анализа акцентуации виртуальных субъектов.
  
  Классические методы изучения е-ком\-му\-ни\-ка\-ции базируются на 
семантическом анализе получаемого от пользователей Сети текстового образа 
взаимодействия или их поведенческого проявления. Ставится задача 
определения статистически релевантных характеристик языковой среды 
коммуникации и~структурных свойств среды. В~дальнейшем при достаточно 
большом с~точки зрения репрезентативности результата числе измерений 
можно строить матрицу вероятностей для оценки свойств коммуникации между 
виртуальными масками субъектов. 
  
  \begin{figure*} %fig2
  \vspace*{1pt}
\begin{center}
\mbox{%
\epsfxsize=120.335mm
\epsfbox{sig-2.eps}
}
\end{center}
\vspace*{-11pt}
\Caption{Проблемный репертуар практических моделей событий Сети}
\end{figure*}
  

  Областью применения предложенной методики анализа семантического 
контента на основе формирования словарей <<окраски текста>> или 
акцентуации е-ком\-му\-ни\-ка\-ции могут стать аналитические BI (Business Intelligence)
за\-про\-сы 
в~экономических исследованиях, выявление характеристик поведения 
субъектов Сети в~социологии, оценка событийных рядов в~политологии 
и~анализ расстройств поведения в~пси\-хи\-атрии.
  
  Особенности e-ком\-му\-ни\-ка\-ций социальных сетей и~блогов затрудняют 
получение однозначного результата при изучении только лингвистической 
составляющей текстов обмена. В~исследовании предлагается дополнить 
существующие подходы, основанные на использовании лингвистического 
анализа текстов, рядом психолингвистических инструментов (акцентуация) 
анализа, учитывая при
 этом их статистическую репрезентативность. 
Репрезентативность предложенной методики достигается высоким уровнем 
диверсификации типов виртуальной коммуникации (рис.~2) и~ее 
разнородностью.
  

  По грубым оценкам, контент за единицу времени в~1~с пополняется 
на~7820~твитов, 1381~графиче-\linebreak\vspace*{-12pt}

 { \begin{center}  %fig3
 \vspace*{6pt}
 \mbox{%
\epsfxsize=61.242mm
\epsfbox{sig-3.eps}
}

\end{center}


\noindent
{{\figurename~3}\ \ \small{Приблизительные оценки персональной е-ин\-фор\-ма\-ции в~сети Интернет: 
\textit{1}~--- общий поток данных; \textit{2}~---  семантически весомые; \textit{3}~--- 
используются часто}}

}

%\vspace*{9pt}

\addtocounter{figure}{1}



\noindent
ское изображение, 1558~аудиозвонков, 
45\,861~запрос Google+; 2\,340\,000 email, включая~67\% спама~[6]. 
В~Интернете циркулирует колоссальный объем персональной информации, как 
правило, текстового содержания в~знаковой, образной и~звуковой форме 
(статистические оценки потоков которой показаны на рис.~3), при этом 
анализу подлежит не просто текст как некий семантический контент, а характер 
восприятия его физическим объектом либо роботизированным устройством.


  
\section{Методика применения психолингвистического анализа  
е-коммуникаций на основе словарей окраски текста}

  Предметом анализа в~предлагаемой методике выявления акцентуации 
виртуальной коммуникации является вербальное и~невербальное речевое 
поведение, сонарные и~темпоральные характеристики речевого поведения,  
лек\-си\-ко-мор\-фо\-ло\-ги\-че\-ский характер проявления виртуального 
субъ\-екта. 

Для расчета показателей психолингвистических свойств текста 
используем фоносемантический анализ~[7], при этом вычислим процентное 
совпадение со словарем окраски текста.
  
  Так как сообщение является входной информацией, представленной в~виде 
набора слов, то можно выявить процентное совпадение данного сообщения со 
словарем~[8], используя формулу:

\noindent
  \begin{equation}
  S_i=\fr{N_i}{N}\,,
  \label{e1-sig}
  \end{equation}
где $S_i$~--- частота появлений некоторой $i$-й словоформы; $N$~--- общее 
число слов или словосочетаний, встреченных в~исследуемом сообщении; $i$~--- 
данная словоформа; $N_i$~--- число вхождений данной словоформы во 
множество всех встреченных слов из словарей.
  
  Примем во внимание, что каждый звук человеческой речи, или фонема, 
обладает определенным подсознательным значением. Для русского языка эти 
значения в~свое время определил советский ученый, доктор филологических 
наук А.\,П.~Жу\-рав\-лев~\cite{9-sig}, который предложил свой вариант 
фоносемантических значений для каждого звука, или фонемы, русской речи 
по~25~шкалам. Всем фонемам русского языка по этим шкалам сопоставлены 
оценки. Для оценки воздействия на человека слова как набора звуков 
необходимо по со\-от\-вет\-ст\-ву\-ющим расчетам определить общее 
фоносемантическое значение составляющих данное слово звуков по шкалам, 
разбитым на~3~группы (рис.~4).

 { \begin{center}  %fig4
 \vspace*{6pt}
\mbox{%
\epsfxsize=77.449mm
\epsfbox{sig-4.eps}
}

\end{center}


\noindent
{{\figurename~4}\ \ \small{Фоносемантическое соответствие звуков по шкалам}}

}

\vspace*{9pt}

\addtocounter{figure}{1} 
  
 
  
  Данные группы составлены с~учетом их час\-тот\-но\-го использования при 
анализе взаимодействия виртуальных идентичностей~\cite{10-sig} 
относительно внешних и~внутренних критериев оценки.
  
  На основе оценки общего эмоционального состояния виртуальной сущности 
(субъекта) по отношению к~порожденному им тексту выделим следующие 
группы: 
  \begin{enumerate}[(1)]
  \item  психолингвистические показатели эмоциональной напряженности; 
  \item вербальные средства выражения эмоционального напряжения; 
  \item вербальные средства выражения мотивационного напряжения.
  \end{enumerate}
  
  В общем случае анализ тональности и~темпоральности относят к~области 
компьютерной лингви\-стики, т.\,е.\ подразумевается, что можно 
классифицировать тональность и~темпоральность, используя стандартные 
инструменты обработки естественного языка по типам организации обработки: 
(1)~подходы, основанные на правилах; (2)~подходы, основанные на словарях; 
(3)~машинное обучение с~учителем; (4)~машинное обучение без учителя. 

В~данной статье приоритет отдан методам, основанным на использовании 
словарей.
  
\section{Методы анализа текстовых сообщений, основанные 
на~правилах и~словарях}

  Этот метод основан на поиске эмотивной лексики (лексической тональности) 
в~тексте по заранее составленным тональным словарям и~правилам 
с~применением лингвистического анализа~\cite{11-sig}. По совокупности 
найденной эмотивной лексики текст может быть оценен по шкале, 
выражающей объем негативной и~позитивной лексики. Данный метод может 
использовать как списки правил, под\-став\-ля\-емые в~регулярные выражения, так 
и~специальные правила соединения тональной лексики внутри предложения. 
Чтобы проанализировать текст, можно воспользоваться следующим 
алгоритмом: сначала каждому слову в~тексте присвоить его значение 
тональности из словаря (если оно присутствует в~словаре), а затем вычислить 
общую тональность всего текста путем суммирования значения тональностей 
каждого отдельного предложения.


  
  Основной проблемой методов, основанных на словарях и~правилах, 
считается трудоемкость процесса составления словаря. Для того чтобы 
получить метод, классифицирующий документ с~высокой точностью, термины 
словаря должны иметь вес, адекватный предметной области документа. 
Например, слово <<огромный>> по отношению к~объему памяти жесткого 
диска является положительной характеристикой, но отрицательной по 
отношению к~размеру мобильного телефона. Поэтому данный метод требует 
значительных трудозатрат, так как для хорошей работы системы необходимо 
составить большое число правил. Чтобы ускорить процесс составления 
словарей и~правил, данный метод\linebreak\vspace*{-12pt}

\pagebreak

\end{multicols}

\begin{figure*} %fig5
\vspace*{1pt}
\begin{center}
\mbox{%
\epsfxsize=129.924mm
\epsfbox{sig-5.eps}
}
\end{center}
\vspace*{-11pt}
\Caption{Алгоритм формирования тематических сло\-ва\-рей-те\-зау\-ру\-сов}
%\vspace*{-6pt}
\end{figure*}

\begin{table*}\small
  \begin{center}
  
  \begin{tabular}{|p{30mm}|p{67mm}|p{53mm}|}
  \multicolumn{3}{c}{Критерии отбора значимых характеристик~\cite{7-sig}, или маски 
акцентуации, виртуального субъекта по типам}\\
  \multicolumn{3}{c}{\ }\\  [-6pt]
  \hline
\multicolumn{1}{|c|}{\tabcolsep=0pt\begin{tabular}{c}Маркер\\ проявления\end{tabular}}&
\multicolumn{1}{c|}{\tabcolsep=0pt\begin{tabular}{c}Характеристика\\ маски 
виртуального субъекта\end{tabular}}&
\multicolumn{1}{c|}{Алгоритм расчета}\\
\hline
<<Светлые>>, 
или паранойяльная акцентуация  поведения &{Тексты мажорной 
окраски, проявление искусственного оптимизма 
(все идет <<хорошо>>,  <<по плану>>),  <<мессианский  комплекс>>}
&
\\
\hline
<<Темные>>, или эпи\-леп\-то\-ид\-ная  акцентуация  поведения &
{Тексты с~большим 
объемом обсуждения насилия, описания  патологической 
жестокости, выраженным  противостоянием <<мы 
и~они>>, <<добро и~зло>> и~т.\, д.} &
\multicolumn{1}{|c|}{\raisebox{-14pt}[0pt][0pt]{$K_{\mathrm{т}}=\fr{\mbox{число\ 
темных\ 
словосочетаний}}{\mbox{число\ 
слов}}$}}\\
\hline
<<Печальные>>,  или депрессивная  акцентуация 
поведения &{Тексты  меланхолического 
настроения, часто  связанные  с~быстротечностью 
жизни, с~тем, что   жизнь~--- страдания  и~только смерть способна 
положить им конец} &
\\
\hline
<<Веселые>>, или  гипертимическая  акцентуация поведения&
{Тексты,  представляющие собой  описание поведения 
человека, который  сталкивается  с~препятствиями или 
опасностями, но успешно  преодолевает их 
и~достигает успеха} &
\\
\hline
<<Сложные>>,  или шизотимная  акцентуация поведения &
{Тексты, наполненные  философскими понятиями, 
абстракциями  и~усложнениями} &\\
\hline
<<Красивые>>,  или истероидная  акцентуация поведения&
{Тексты с~нарочитым описанием эмоциональных 
аффектов, страстей,  страдания и~эротизма} &
\\
\hline
\end{tabular}
\end{center}
\vspace*{-24pt}
\end{table*}

\pagebreak



\begin{multicols}{2}


\noindent
 используется с~привязкой к~конкретной 
предметной области (например, тематика ресторанов или тематика мобильных 
телефонов).




  Критериями отбора значимых характеристик, или масками акцентуации, 
виртуального субъекта по типам (см.\ таблицу): паранойяльной, 
эпилептоидной, депрессивной, гипертимической и~др.~--- служат агрегации 
выделенных квалиметрических показателей~\cite{12-sig}. Упрощенно 
тональный словарь представляет собой список слов со значением тональности 
для каждого слова. Чтобы проанализировать текст, можно воспользоваться 
следующим алгоритмом (рис.~5): сначала каждому слову в~тексте 
присваивается его значение тональности из словаря (если присутствует), 
а~затем вычисляется общая тональность всего текста путем нахождения 
усредненных величин либо путем <<обучения>> классификатора (например, 
нейронной сети).
  
  

  {\small  \textbf{Примечание.} 
 Под \textit{паранойяльной акцентуацией} 
подразумевается повышенная подозрительность и~болезненная обидчивость, 
стойкость отрицательных аффектов, стремление к~доминированию, непринятие 
мнения другого и,~как следствие, высокая конфликтность, подпадание под 
власть сверхценных идей и~стремление к~навязыванию своего мнения. 
%
\textit{Эпилептоидный тип акцентуации} связан с~такими чертами, как 
склонность к~злоб\-но-тоск\-ли\-во\-му настроению, раздражительности, 
агрессивности, внут\-рен\-ней неудовлетворенности, злости, гнева, ярости, 
жестокости и~конфликтности.
%
 Личности \textit{с депрессивным типом 
акцентуации} проявляют лабильность ко всякого рода неприятностям, 
проявляют неопределенное чувство тяжести, ожидание несчастья. 
%
\textit{Гипертимическая акцентуация} характеризуется специфическим 
поведением, связанным со сменой идей, проявлением словесной ловкости, 
изворотливости, с~направленностью на большое число социальных контактов 
и~отражающим повышенный настрой. 
%
Для \textit{демонстративной, или 
истероидной, акцентуации} характерна поверхностность, наигранность 
переживаний, <<работа на публику>>, стремление вызвать у~аудитории
эмоциональный отклик любой ценой, непродуманность речевого поведения. 
%
\textit{Шизотимность} поведения виртуального субъекта проявляется 
в~направленности высказываний на себя, замкнутостью на узкий круг 
вопросов, \textit{акцентуации} на внутреннем мире.}
  

  
  
\section{Описание методики эксперимента}

  \noindent
  \textbf{Шаг~1.} По открытым публикациям в~социальных сетях определяем 
архитектуру программных средств и~совокупности словарей акцентуации. 

\end{multicols}



\begin{figure*}[h] %fig6
\vspace*{-6pt}
\begin{center}
\mbox{%
\epsfxsize=154.849mm
\epsfbox{sig-6.eps}
}
\end{center}
\vspace*{-11pt}
\Caption{Архитектура экспериментальной платформы программной системы анализа 
акцентуации (тональности) новостных групп сообщений пользователей сети 
<<ВКонтакте>>}
\vspace*{-6pt}
\end{figure*}

\begin{multicols}{2}


 
  Исследование проводилось на примере социальной сети {\sf vk.com} 
с~использованием API IBM Watson Tone Analyzer (библиотеки анализа 
тональности текста IBM), Emotion Analysis (библиотеки анализа эмотивности, 
или эмоциональной\linebreak акцентуации, текста IBM), программной библиотеки 
VK.API (сис\-те\-мы для разработчиков сторонних сайтов, которая предоставляет 
возможность легко авторизовать пользователей <<ВКонтакте>>), шины 
сообщений RabbitMQ (платформы, реализующей систему обмена сообщениями 
между компонентами программной системы, Message Oriented Middleware) 
и~Visual Recognition (библиотеки распознавания изображений). 
  
  Архитектура программного решения проведения экспериментов 
представлена на рис.~6. Для его создания был использован язык 
программирования Python и~язык разработки скриптов ECMAScript.
  
  Эмотивность текста оценивалась с~использованием сервиса Text to Speech 
(TTS). 

\bigskip

  \noindent
  \textbf{Шаг~2.} Для экспериментальных исследований была выбрана группа 
Новости RT социальной сети <<ВКонтакте>>. Исследовалась тональность (как 
доминирующая акцентуация, см.\ таблицу) комментариев пользователей. 
Выбор данного сетевого ресурса обусловлен простым алгоритмом его работы, 
в~том числе и~для неопытного пользователя, что обусловлено использованием 
API социальной сети <<ВКонтакте>>. Пользователь сетевого ресурса\linebreak\vspace*{-12pt}

 { \begin{center}  %fig7
 \vspace*{24pt}
 \mbox{%
\epsfxsize=78mm
\epsfbox{sig-7.eps}
}

\end{center}

\vspace*{3pt}

\noindent
{{\figurename~7}\ \ \small{Оценка тональности (акцентуации) комментариев за день}}

}

%\vspace*{9pt}

\addtocounter{figure}{1}



\noindent
 вводит 
свой идентификационный номер либо короткое доменное имя, после чего 
ресурс получает доступ ко всей текстовой информации со стены пользователя, 
выбирает ключевые слова и~позволяет проанализировать данные на основе 
словарей, маркирующих акцентуацию сообщений, предо\-став\-ля\-емых сервисом 
({\sf http://Indico.io}). 

На следующем шаге рассчитываются суммарные 
значения по\-зи\-тив\-ных/не\-га\-тив\-ных слов в~сообщениях пользователя 
в~процентном соотношении и~ставится в~соответствие эмоциональный маркер 
состояния пользователя в~текущий момент времени (рис.~7).

\bigskip
  
  \noindent
  \textbf{Шаг~3.} В~результате проведенного эксперимента по определению 
тональности текстов виртуальных субъектов социальной сети <<ВКонтакте>> 
на дату обращения были получены следующие результаты: негативных 
комментариев~--- 45,79\%; позитивных комментариев~--- 32,58\%; нейтральных 
комментариев~--- 21,63\%. 
  
  
  \textit{Негативные тексты} объединяют в~себе две категории виртуальных 
субъектов: с~депрессивной и~эпилептоидной акцентуацией поведения, т.\,е.\ 
текс\-ты с~большим объемом обсуждения насилия, описания патологической 
жестокости, выраженным противостоянием, а~так\-же текс\-ты меланхолического 
настроения. 

К~\textit{нейтральным} были отнесены текс\-ты виртуальных 
субъектов с~паранойяльной и~шизотимной акцентуацией~--- это по большей 
час\-ти текс\-ты, наполненные философскими понятиями, абстракциями 
и~услож\-не\-ни\-ями, и~текс\-ты мажорной окраски, проявление искусственного 
оптимизма. 

\textit{Позитивные тексты} получены от виртуальных субъектов 
с~гипертимической акцентуацией речевого поведения, и~отчасти сюда были 
отнесены текс\-ты субъектов с~проявлением истероидной акцентуации, иначе это 
текс\-ты, представляющие собой описание поведения человека, который 
сталкивается с~препятствиями или опасностями, но успешно преодолевает их 
и~достигает успеха, и~час\-тич\-но текс\-ты с~нарочитым описанием эмоциональных 
аффектов~\cite{13-sig}.
  
  \smallskip
  
  Практическое значение разработанной методики заключается в~следующем: 
используя предложенную архитектуру предложенной установки 
вычислительной сети, можно сформировать карту\linebreak поведенческих речевых 
паттернов (см.\ рис.~7) отдельной социальной сети для проведения 
мониторинга напряженности социальных настроений на\linebreak основе тональности 
текстов произвольных виртуальных идентичностей. Следует заметить, что 
методом оценки тональности можно варьировать\linebreak в~зависимости от 
совокупности выбираемых семантических словарей, при этом диапазон 
изменений по результатам проведенного эксперимента демонстрировал 
незначительную выборочную сенситивность, причем без потери значимости 
для одного и~того же текста при использовании иной эталонной коллекции.

\section{Заключение}

  В~исследовании в~рамках практического эксперимента решена задача 
идентификации типа акцентуации паттерна поведения виртуального субъекта 
на основе статистического анализа текста коммуникации и~подтверждена 
гипотеза о~структурных свойствах заданной коммуникации. Актуальность 
предложенной методики определяется еще и~тем, что решение поставленной 
задачи учитывает возрастающее значение развития системы условных знаков, 
в~данном случае условных языков е-ком\-му\-ни\-ка\-ции и~управляющих 
кластеров, позволяющих осуществлять мониторинг и~создавать основу для 
регулирующих воздействий на социальное поведение виртуальных субъектов 
в~Сети.
  
  Результаты экспериментального исследования демонстрируют достаточно 
высокий уровень релевантности характеристик эмоционального анализа 
виртуальных субъектов на основе предложенной методики. Разработанная 
и~апробированная методика оценки среднего фона произвольной социальной 
сети с~привязкой по времени позволяет получить карту поведенческих речевых 
паттернов для проведения мониторинга напряженности социальных настроений 
на основе тональности текстов произвольных виртуальных идентичностей.
  
  Новизна предложенной методики заключается в~идее дополнения 
лингвистического анализа сетевого поведения субъектов статистическим 
анализом и~в выбранных и~адаптированных методах анализа сообщений 
пользователей. В~исследовании определена область практического 
использования результатов и~обоснована репрезентативность разработанной 
методики анализа акцентуации виртуальных субъектов.
  
\section{Тезаурус}

  \textbf{Виртуальная идентичность}~--- коммуникативная репрезентация 
человека в~виртуальной среде, т.\,е.\ то, каким образом он позиционирует себя 
в~сети Интернет (паттерн поведения виртуального объекта). Паттерны 
взаимодействия могут быть представлены в~виде групп речевого 
взаимодействия: подражания, оппонирования, агрессии, конструктивного 
диалога и~т.\,д.
  
  \textbf{Конструктивистский подход} применительно к~виртуальной 
коммуникации строится на предположении, что свойства человеческой психики 
и~мыс\-ле\-фор\-мы, которыми оперирует виртуальный субъект, в~значительной 
степени подвержены влиянию категорий, сформированных социумом в~его 
непосредственном окружении и~усвоенных им в~процессе социализации, 
и,~следовательно, не зависят от биологических ограничений.
  
  \textbf{Лингвистический релятивизм} базируется на гипотезе 
лингвистической относительности и~предполагает, что структура языка  
е-ком\-му\-ни\-ка\-ции влияет на менталитет и~способы идентификации его 
виртуальных агентов, а также на когнитивные процессы реальных субъектов.
  
{\small\frenchspacing
 {%\baselineskip=10.8pt
 \addcontentsline{toc}{section}{References}
 \begin{thebibliography}{99}
\bibitem{1-sig}
\Au{Johansson F., Brynielsson~J., Horling~P., Malm~M., Martenson~C., Truve~S., 
Rosell~M.} Detecting emergent conflicts through Web Mining and Visualization~// 
2011 European Intelligence and Security Informatics Conference 
Proceedings.~--- IEEE, 2011. P.~346--353.
\bibitem{2-sig}
\Au{Kennison S.\,M.} Introduction to language development.~--- Los Angeles, СA, USA: 
SAGE Publications Inc., 2014. 496~p.
\bibitem{3-sig}
\Au{Brown R., Lenneber~E.} A~study in language and cognition~// J.~Abnorm. 
Soc. Psych., 1954. Vol.~49. P.~454--462.
\bibitem{4-sig}
Rethinking linguistic relativity~/ Eds. J.\,J.~Gumperz, S.\,C.~Levinson.~--- 
Studies in the social and cultural foundations of language ser.~--- 
Cambridge: Cambridge University Press, 1999.  No.\,17. 488~p. 
\bibitem{5-sig}
\Au{Slobin D.\,I.} Two ways to travel: Verbs of motion in English and Spanish~// 
Grammatical Constructions: Their form and meaning~/ Eds. M.~Shibatani, 
S.\,A.~Thompson.~--- Oxford: Clarendon Press, 1996. P.~195--220.
\bibitem{6-sig}
\Au{Barbian G.} Detecting hidden friendship in online Social Networks~// 2011 
European Intelligence and Security Informatics Conference  
Proceedings.~--- IEEE, 2011. P.~269--272.
\bibitem{7-sig}
\Au{Горелов И.\,Н., Седов~К.\,Ф.} Основы психолингвистики.~--- М.: Лабиринт, 
2001. 304~с.
\bibitem{8-sig}
\Au{Сидоренко Е.\,В.} Методы математической обработки в~психологии.~--- 
СПб.: Речь, 2002. 350~с.
\bibitem{9-sig}
\Au{Журавлев А.\,П.} Звук и~смысл.~--- М.: Просвещение, 1991. 160~с.
\bibitem{10-sig}
\Au{Vybornova O., Smirnov~I., Sochenkov~I., Kiselyov~A., Tikhomirov~I., 
Chudova~N., Kuznetsova~Y., Osipov~G.} Social tension detection and intention 
recognition using Natural Language Semantic Analysis~// 2011 European 
Intelligence and Security Informatics Conference Proceedings.~--- 
IEEE, 2011. P.~277--281.
\bibitem{11-sig}
\Au{Тихомиров И.\,А., Смирнов~И.\,В.} Интеграция лингвистических 
и~статистических методов поиска в~поисковой машине Exactus~// 
Компьютерная лингвистика и~интеллектуальные технологии: Тр. междунар. 
конф. <<Диалог-2008>>.~--- М.: РГГУ, 2008. С.~485--491.
\bibitem{12-sig}
\Au{Cambria E., Havaci~E., Hussain~A.\,A.} Semantic and affective resource for 
opinion mining and sentiment analysis~// 25th  Florida Artificial 
Intelligence Research Society Conference (International) Proceedings.~--- Palo 
Alto, CA, USA: AAAI Press, 2012. P.~202--207.
\bibitem{13-sig}
\Au{Hoijer H.} The Sapir--Whorf hypothesis~// Conference on the Interrelations of 
Language and Other Aspects of Culture Proceedings: Memoirs of the American 
Anthropological Association, Comparative Studies of Cultures and Civilizations.~--- 
Chicago, IL, USA: University of Chicago Press, 1954. No.\,3. P.~92--105.
 \end{thebibliography}

 }
 }

\end{multicols}

\vspace*{-3pt}

\hfill{\small\textit{Поступила в~редакцию 25.04.17}}

\vspace*{8pt}

%\newpage

%\vspace*{-24pt}

\hrule

\vspace*{2pt}

\hrule

%\vspace*{8pt}


\def\tit{PSYCHOLINGUISTIC ANALYSIS OF~TEXT MESSAGES IN~RUSSIAN 
BASED ON~THEIR PHONOSEMANTIC STATISTICAL 
CHARACTERISTICS}

\def\titkol{Psycholinguistic analysis of~text messages in~Russian 
based on~their phonosemantic statistical 
characteristics}

\def\aut{A.\,S.~Sigov, D.\,A.~Akimov, D.\,O.~Zhukov, E.\,G.~Andrianova, 
V.\,E.~Sachkov, and~V.\,K.~Raev}

\def\autkol{A.\,S.~Sigov, D.\,A.~Akimov, D.\,O.~Zhukov, et al.}
%, E.\,G.~Andrianova,  V.\,E.~Sachkov, and~V.\,K.~Raev}

\titel{\tit}{\aut}{\autkol}{\titkol}

\vspace*{-9pt}


\noindent
Moscow Technological University, 78~Vernadsky Av., Moscow 119454, Russian 
Federation



\def\leftfootline{\small{\textbf{\thepage}
\hfill INFORMATIKA I EE PRIMENENIYA~--- INFORMATICS AND
APPLICATIONS\ \ \ 2017\ \ \ volume~11\ \ \ issue\ 3}
}%
 \def\rightfootline{\small{INFORMATIKA I EE PRIMENENIYA~---
INFORMATICS AND APPLICATIONS\ \ \ 2017\ \ \ volume~11\ \ \ issue\ 3
\hfill \textbf{\thepage}}}

\vspace*{3pt}



\Abste{A text as a complex semantic and syntactic formation has a number of 
psycholinguistic characteristics, which include integrity and semantic orientation. 
A~text can be viewed as a~product of speech activity with a~high degree of semantic 
variation determined by its temporal and sonar characteristics. Nonverbal behavior 
of network entities~--- virtual masks and robotic agents~--- reveals itself in texts.
  The article raises and solves the problem of identifying the type of accentuation of 
pattern of behavior of a virtual entity based on statistical analysis of text 
communication, which allows one to formulate a hypothesis about the structural 
properties of a~given communication and build a matrix of probabilities of 
relationship between virtual masks of subjects.
  The practical significance of the proposed solution is based on the growing 
importance of the development of the system of conditional signs, in this case, the 
conditional languages of e-communication, for the generation of control clusters 
regulating the social behavior of virtual subjects in the network. This assumption is 
based on the hypothesis of Kenneth Ivers, according to which, the better the system 
of conventional signs, the more opportunities to create new algorithms.}
  
\KWE{psycholinguistic characteristics; nonverbal behavior; virtual masks; process 
of thinking; semantic meaning; linguistic relativism}




\DOI{10.14357/19922264170309} 

\vspace*{-9pt}

\Ack
\noindent
The work was carried out within the Ministry of Education and Science of the 
Russian Federation's program of financing the competitive part of public tasks to 
institutions of higher education and scientific organizations to implement initiative 
scientific projects (No.\,28.2635.2017/PP).



%\vspace*{3pt}

  \begin{multicols}{2}

\renewcommand{\bibname}{\protect\rmfamily References}
%\renewcommand{\bibname}{\large\protect\rm References}

{\small\frenchspacing
 {%\baselineskip=10.8pt
 \addcontentsline{toc}{section}{References}
 \begin{thebibliography}{99}
\bibitem{1-sig-1}
\Aue{Johansson, F., J.~Brynielsson, P.~Horling, M.~Malm, C.~Martenson, 
S.~Truve, and M.~Rosell.} 2011. Detecting emergent conflicts through Web Mining 
and Visualization. \textit{European Intelligence and Security Informatics Conference 
 Proceedings}. IEEE. 346--353.
\bibitem{2-sig-1}
\Aue{Kennison, S.\,M.} 2014. \textit{Introduction to language development}. Los 
Angeles, CA: SAGE Publications Inc. 496~p.
\bibitem{3-sig-1}
\Aue{Brown, R., and E.~Lenneber.} 1954. A~study in language and cognition. 
\textit{J.~Abnorm. Soc. Psych.} 49:454--462.
\bibitem{4-sig-1}
Gumperz, J.\,J., and S.\,C.~Levinson, eds. 1999.
\textit{Rethinking linguistic relativity}.  Studies in social and cultural foundations of 
language ser. Cambridge: Cambridge University Press. No.\,17. 488~p.
\bibitem{5-sig-1}
\Aue{Slobin, D.} 1996. Two ways to travel: Verbs of motion in English and Spanish. 
\textit{Grammatical Constructions: Their form and meaning}. Eds. M.~Shibatani and 
S.\,A.~Thomson. Oxford: Clarendon Press. 195--220.
\bibitem{6-sig-1}
\Aue{Barbian, G.} 2011. Detecting hidden friendship in Online Social Networks. 
\textit{European Intelligence and Security Informatics Conference 
Proceedings}. IEEE. 269--272.
\bibitem{7-sig-1}
\Aue{Gorelov, I.\,N., and K.\,F.~Sedov}. 2001. \textit{Osnovy psi\-kho\-ling\-vi\-sti\-ki} 
[Fundamentals of psycholinguistics]. Moscow: Labirint. 304~p.

%\pagebreak

\bibitem{8-sig-1}
\Aue{Sidorenko, E.\,V.} 2002. \textit{Metody matematicheskoy obrabotki 
v~psikhologii} [Methods of mathematical processing in psychology]. St.\ Petersburg: 
Rech. 350~p.
\bibitem{9-sig-1}
\Aue{Zhuravlev, A.\,P.} 1991. \textit{Zvuk i~smysl} [Sound and meaning]. Moscow: 
Prosveshchenie, 1991. 160~p.
\bibitem{10-sig-1}
\Aue{Vybornova, O., I.~Smirnov, I.~Sochenkov, A.~Kiselyov, I.~Tikhomirov, 
N.~Chudova, Y.~Kuznetsova, and G.~Osipov.} 2011. Social tension detection and 
intention recognition using Natural Language Semantic Analysis. 
\textit{European Intelligence and Security Informatics Conference 
Proceedings}. IEEE. 277--281.
\bibitem{11-sig-1}
\Aue{Tihomirov, I.\,A., and I.\,V.~Smirnov.} 2008. In\-te\-gra\-tsiya lingvisticheskikh 
i~statisticheskikh metodov poiska v~poiskovoy mashine Exactus [Integration of 
linguistic and statistical methods of searching in the search engine Exactus]. 
\textit{Conference (International) 
``Dialogue-2008'' Proceedings}. Moscow: RGGU. 485--491.
\bibitem{12-sig-1}
\Aue{Cambria, E., E.~Havaci, and A.\,A.~Hussain.} 2012. Semantic and affective 
resource for opinion mining and sentiment analysis. \textit{25th  
Florida Artificial Intelligence Research Society Conference (International)
Proceedings}. Palo Alto, CA: AAAI Press. 202--207.
\bibitem{13-sig-1}
\Aue{Hoijer, H.} 1954. The Sapir--Whorf hypothesis. \textit{Conference on the 
Interrelations of Language and Other Aspects of Culture Proceedings: Memoirs of the 
American Anthropological Association, Comparative Studies of Cultures and 
Civilizations}. Chicago, IL: University of Chicago Press. 3:92--105.
\end{thebibliography}

 }
 }

\end{multicols}

\vspace*{-3pt}

\hfill{\small\textit{Received April 25, 2017}}
  
\Contr

\noindent
\textbf{Sigov Alexander S.} (b.\ 1945)~--- Academician of the Russian Academy of 
Sciences, President of the Moscow Technological University (MIREA), 
78~Vernadsky Av., Moscow 119454, Russian Federation; 
\mbox{assigov@yandex.ru}

\vspace*{3pt}

\noindent
\textbf{Akimov Dmitry A.} (b.\ 1987)~--- Candidate of Science (PhD) in 
technology, associate professor, Moscow Technological University (MIREA), 
78~Vernadsky Av., Moscow 119454, Russian Federation; 
\mbox{akimov\_d@mirea.ru}

\vspace*{3pt}

\noindent
\textbf{Zhukov Dmitry O.} (b.\ 1965)~--- Doctor of Science in technology, 
professor, Moscow Technological University (MIREA), 78~Vernadsky Av., 
Moscow 119454, Russian Federation; \mbox{zhukovdm@yandex.ru}

\vspace*{3pt}

\noindent
\textbf{Andrianova Elena G.} (b.\ 1963)~--- Candidate of Science (PhD) in 
technology, associate professor, Moscow Technological University (MIREA), 
78~Vernadsky Av., Moscow 119454, Russia n Federation; 
\mbox{dtghmflysq@gmail.com}

\vspace*{3pt}

\noindent
\textbf{Sachkov Valery E.} (b.\ 1989)~--- PhD student, Moscow Technological 
University (MIREA), 78~Vernadsky Av., Moscow 119454, Russian Federation; 
\mbox{megawatto@mail.ru}

\vspace*{3pt}

\noindent
\textbf{Raev Vyacheslav K.} (b.\ 1965)~--- Doctor of Science in technology, 
professor, Moscow Technological University (MIREA), 78~Vernadsky Av., 
Moscow 119454, Russian Federation; \mbox{raev@mirea.ru}

\label{end\stat}


\renewcommand{\bibname}{\protect\rm Литература} 
       %9
\def\stat{gudkova}

\def\tit{ВЕРОЯТНОСТНАЯ МОДЕЛЬ СОВМЕСТНОГО ИСПОЛЬЗОВАНИЯ РЕСУРСОВ 
БЕСПРОВОДНОЙ СЕТИ С~АДАПТИВНЫМ УПРАВЛЕНИЕМ МОЩНОСТЬЮ$^*$}

\def\titkol{Вероятностная модель совместного использования ресурсов 
беспроводной сети с~адаптивным управлением} % мощностью}

\def\aut{И.\,А.~Гудкова$^1$, С.\,Я.~Шоргин$^2$}

\def\autkol{И.\,А.~Гудкова, С.\,Я.~Шоргин}

\titel{\tit}{\aut}{\autkol}{\titkol}

\index{Гудкова И.\,А.}
\index{Шоргин С.\,Я.}
\index{Gudkova I.\,A.}
\index{Shorgin S.\,Ya.}


{\renewcommand{\thefootnote}{\fnsymbol{footnote}} \footnotetext[1]
{Исследование выполнено при финансовой поддержке Российского научного фонда (проект 16-11-10227).}}


\renewcommand{\thefootnote}{\arabic{footnote}}
\footnotetext[1]{Российский университет дружбы народов;
 Институт проблем информатики Федерального исследовательского 
центра <<Информатика и~управление>> Российской академии наук, \mbox{gudkova\_ia@rudn.university}}

\footnotetext[2]{Институт проблем информатики Федерального исследовательского центра <<Информатика 
и~управление>> Российской академии наук, \mbox{sshorgin@ipiran.ru}}

\vspace*{18pt}



\Abst{Развивающиеся беспроводные сети последующего поколения 
(next generation 
network, NGN)
предполагают новые 
приложения и~услуги как для обычных пользователей, так и~для устройств межмашинного 
взаимодействия (machine-to-machine, M2M). Решение проблемы увеличения требований 
к~пропускной способности сети и~недостаточности спектра радиочастот, в~частности 
в~случае умных городов, может быть достигнуто посредством концепции совместного 
использования радиочастот (licensed shared access, LSA). Авторы предлагают 
математическую модель совместного использования ресурсов с~адаптивным управ\-ле\-ни\-ем 
мощностью. Заложенный в~ней алгоритм позволит избежать интерференции M2M-устройств с~владельцем спектра, в~том числе благодаря тому, что учитывает пространственное 
расположение устройств и~их сессионную активность.}
 
\KW{беспроводная сеть; умный город; межмашинное взаимодействие; совместное 
использование радиочастот; адаптивное управление мощностью; случайный процесс; 
рекуррентный алгоритм; вероятность блокировки; вероятность прерывания обслуживания; 
среднее число устройств}

\DOI{10.14357/19922264170310} 

\vspace*{6pt}


\vskip 10pt plus 9pt minus 6pt

\thispagestyle{headings}

\begin{multicols}{2}

\label{st\stat}

\section{Введение}

  Согласно прогнозам развития сетей по\-сле\-ду\-юще\-го поколения, 
  уже в~2025~г.\ беспроводные сети будут перегружены~[1], что 
повлечет за собой необходимость уточнения и~разработки новых стратегий 
использования спектра радиочастот~[2]. Широкое распространение получают 
автономно функционирующие и~взаимодействующие друг с~другом  
(М2М) недорогие устройства, являющиеся неотъемлемой 
частью <<умных городов>> (smart city). Особенностью M2M-устройств 
является их дистанционное управ\-ле\-ние и~высокая плот\-ность расположения. 
Рост числа M2M-устройств существенно сказывается на использовании спектра 
ра\-дио\-час\-тот ввиду того, что сети изначально разрабатывались для 
взаимодействия между людьми (human-to-human, H2H). 
{\looseness=1

}

Один из вариантов 
решения проб\-ле\-мы~--- это динамическое управление спектром в~рамках 
концепции совместного использования радиочастот 
(LSA)~[3--5]. Доступ к~спектру получают две стороны~--- владелец и~временный 
пользователь~[6, 7]. 

  В статье исследуется один из сценариев применения системы LSA~[8--11], 
где владелец запрашивает спектр радиочастот изредка на непродолжительное 
время. В~остальное же время спектр доступен M2M-устрой\-ст\-вам для 
передачи данных. Статья имеет следующую структуру. 

В~разд.~2 описана 
системная модель совместного использования радиоресурсов с~учетом 
расположения устройств на разном расстоянии от базовой станции~[12--15]. 

В~разд.~3 проводится построение математической модели в~виде двух 
случайных процессов (СП), один из которых фиксирует уровень качества 
канала каж\-до\-го из активных устройств, а~второй, укрупненный,~--- только 
суммарное число устройств. Для СП с~укрупненными со\-сто\-яни\-ями представлен 
рекуррентный алгоритм расчета стационарного распределения вероятностей. 

В~разд.~4 предложены формулы для расчета ключевых показателей 
эффективности системы~--- среднего числа устройств и~вероятностей 
блокировки и~прерывания обслуживания, приведен пример численного анализа.

\section{Системная модель совместного использования ресурсов 
с~разноудаленными от базовой станции устройствами}

Рассмотрим одну соту беспроводной сети радиуса~$R$ с~равномерно 
распределенными по зоне покрытия M2M-устройствами (рис.~1). Устройства 
с~интенсивностью~$\lambda$ переходят в~активное состояние и~передают 
данные в~восходящем канале. Время передачи данных одним устройством 
распределено экспоненциально с~параметром~$\mu$. Каждому устройству 
в~зависимости от дальности расположения от базовой станции (БС) 
присваивается один из пятнадцати уровней качества канала (channel quality 
indicator, CQI)~--- $c\hm=1,\ldots ,15$, причем чем больше~$c$, тем ближе 
устройство к~БС и~выше скорость передачи данных. Объединим устройства 
с~одинаковыми уровнями CQI в~логические группы, тогда скорость передачи 
для всех устройств в~группе будет одинаковая. Далее под расстоянием от 
устройства до БС будем понимать максимально возможное расстояние, на 
котором может быть расположено устройство с~таким же уровнем CQI. Введем 
дополнительное обозначение: $\eta\hm=16-c$; уровень CQI~$c$, 
величина~$\eta$ и~расстояние~$\xi_d(\eta)=RL^{-1}\eta$ от устройства до БС 
являются случайными величинами (СВ). Плотность расстояния от устройства 
до БС $$
f_{\xi_d(\eta)}(d)=\fr{2d}{R^2}\,,
$$
 а~функция распределения (ФР)
$$
F_{\xi_d(\eta)}(d)=\left(\fr{d}{R}\right)^2,\enskip 0\leq d\leq R\,.
$$


 { \begin{center}  %fig1
 \vspace*{5pt}
 \mbox{%
\epsfxsize=72mm %.723mm
\epsfbox{gud-1.eps}
}


\vspace*{4pt}


\noindent
{{\figurename~1}\ \ \small{Пример расположения M2M-устройств в~соте}}
\end{center}
}


\addtocounter{figure}{1}

 \noindent
 Ряд распределения для 
параметра~$\eta$:  
$$
q_l=\fr{2L-2l-1}{L^2}\,,\enskip l=1,\ldots ,L\,.
$$

  
  В качестве примера реализации системы LSA рассмотрим случай 
использования владельцем спектра радиочастот для воздушной телеметрии. 
Предположим, что время, в~течение которого владелец (аэропорт) не 
использует спектр, т.\,е.\ время, когда полоса доступна для M2M-устройств, 
и~время пролета самолета над сотой, т.\,е.\ время, когда полоса недоступна для 
устройств, распределены по экспоненциальному закону с~параметрами~$\alpha$ 
и~$\beta$ соответственно.
  
  Управление радиоресурсами предполагает разделение ресурсов по времени, 
т.\,е.\ деление ширины полосы радиочастот~$\omega$ не происходит, 
а~передача данных осуществляется на постоянной мощности. Если полоса не 
требуется аэропорту, то мощность составляет~$p_1^{\max}$, в~противном 
случае для регулирования интерференции мощность снижается до значения 
$p_0^{\max}\hm<p_1^{\max}$. Такое динамическое изменение мощности 
приводит к~изменению достижимой скорости передачи данных $r\left( 
\xi_{d(\eta)},p_s^{\max}\right)$, $s\hm=0,1$, зависящей также от расстояния 
между устройством и~БС. Согласно формуле Шеннона,
  \begin{multline}
  r\left( \xi_{d(\eta)}, p_s^{\max}\right) =\omega \ln\left( 
1+\fr{Gp_s^{\max}}{((R/L)\eta)^\kappa N_0}\right)\,,\\ s=0,1\,,\enskip 
\eta=1,\ldots ,15\,,
  \label{e1-gud}
  \end{multline}
где $N_0$~---  уровень шума; $G$~--- константа затухания сигнала; $\kappa$~--- 
экспонента затухания сигнала.

  Скорость передачи данных каждым активным M2M-устройством не может 
быть ниже порогового (гарантированного) значения~$r_0$. Если устройству не 
может быть обеспечена скорость~$r_0$, то запрос на передачу данных будет 
заблокирован. Если устройство расположено в~непосредственной близости от 
БС, то скорость передачи согласно формуле Шеннона стремится 
к~бесконечности, поэтому определим минимальное расстояние до БС 
$\xi_d(1)\hm=d_0$, ограничив тем самым максимальную скорость передачи 
данных $r_s^{\max}\hm= r\left( d_0, p_s^{\max}\right)$. Таким образом, если 
$\eta\hm=1$, то достижимая скорость передачи данных $r\left( \xi_{d(\eta)}; 
p_s^{\max}\right)$, если $\eta\hm=2,\ldots ,L$, то она вычисляется по 
формуле~(\ref{e1-gud}). Максимальное число устройств в~соте:
$$
K_s= \left \lfloor  
\fr{r( d_0, p_s^{\max})}{r_0}\right\rfloor\,,\enskip s=0,1
\,.
$$

  Сводный перечень основных обозначений приведен в~табл.~1.

\end{multicols}

  \begin{table*}\small
  \begin{center}
  \Caption{Основные обозначения}
  \vspace*{2ex}
  
  \begin{tabular}{|l|p{340pt}|}
  \hline
  \multicolumn{1}{|c|}{Обозначение}&\multicolumn{1}{c|}{Описание}\\
  \hline
  $R$&Радиус соты, м\\
  $\omega$&Ширина полосы радиочастот, МГц\\
  $L$&Число уровней качества канала CQI \\
  $c$&Уровень CQI (СВ)\\
  $\eta=16-c$&Величина, обратная уровню CQI~$c$  (СВ)\\
  $q_l=\fr{2L-2l-1}{L^2}$&Вероятность того, что уровень CQI равен~$l$\\
  $\alpha^{-1}$&Среднее время доступности полосы, с\\
  $\beta^{-1}$&Среднее время недоступности полосы, с\\
  $k$&Число активных устройств\\
  $s$&Состояние полосы: $s=1$, если полоса доступна; $s=0$, если недоступна\\
  $p_0^{\max}$&Максимальное значение мощности сигнала устройства, если полоса 
недоступна, Вт\\
  $p_1^{\max}$&Максимальное значение мощности сигнала устройства, если полоса 
доступна, Вт\\
  $d_0$&Минимальное расстояние от устройства до БС, м\\
  $r\left( \xi_{d(\eta)}, p_s^{\max}\right)$ &Достижимая скорость передачи 
  для устройства с~уровнем CQI $c\hm=16\hm-\eta$, если полоса находится в~состоянии~$s$ (СВ), бит/с\\
  $r_0^{\max}$&Максимально возможная скорость, если полоса недоступна, бит/с\\
  $r_0$&Гарантированная скорость передачи данных от устройств, бит/с\\
  $r_1^{\max}$&Максимально возможная скорость, если полоса доступна, бит/с\\
  $K_0$&Максимальное число устройств, если полоса недоступна\\
  $K_1$&Максимальное число устройств, если полоса доступна\\
  $\xi_d(\eta)$ &Максимальное расстояние от устройства с~уровнем CQI $c\hm=16\hm-\eta$ 
до БС (СВ), м\\
  $\lambda$&Интенсивность суммарного потока данных от всех устройств в~соте, 1/с\\
  $\mu^{-1}$&Среднее время передачи данных от одного устройства, с\\
  $\rho=\fr{\lambda}{\mu}$&Суммарная предложенная нагрузка от всех устройств в~соте, Эрл\\[-9pt]
&\\
  \hline
  \end{tabular}
  \end{center}
  \end{table*}
  
  \begin{multicols}{2}
  

  
\section{Вероятностная модель и~стационарное распределение 
вероятностей состояний беспроводной сети}

  Перейдем к~построению математической модели. Пусть $\xi(t)$~--- число 
активных M2M-устройств; $\eta_i(t)$~--- значение параметра~$\eta$ для 
устройства~$i$; $\zeta(t)$~--- состояние полосы в~момент времени $t\hm\geq 0$. 
Тогда функционирование соты опишем СП $\left\{ \xi(t),\eta_1(t),\ldots 
,\eta_{\xi(t)},\zeta(t), t\geq0\right\}$ над пространством состояний
  \begin{multline*}
  \mathbf{L}\hm= 
  \left\{ 
    \vphantom{\sum\limits_{i=1}^k}
    (0,s), \left(k,l_1,\ldots ,l_k, s\right), \right.\\ 
  s=0,1,\ l_i=1,\ldots 
,L,\ i=1,\ldots , k,\ k=1,2,\ldots: \\
\left.  \sum\limits_{i=1}^k \fr{r_0}{\omega \ln \left( 1+Gp_s^{\max}/((RL^{-
1}l_i)^\kappa N_0)\right)}\leq 1\right\}\,.
  %\label{e2-gud}
  \end{multline*}

Фрагмент пространства состояний показан на рис.~2.
    

  Перейдем к~СП $\{\xi(t),\zeta(t), t\geq0\}$ с~укрупненными состояниями~--- 
суммарным числом устройств и~состоянием полосы. Пространство состояний 
такого процесса будет иметь вид:
  \begin{equation*}
  \mathbf{L}_1 =\left\{ (k,s):\ k=0,1,\ldots ,K_s,\ s=0,1\right\}\,.
  %\label{e3-gud}
  \end{equation*}
Отметим, что при переходе полосы в~недоступное состояние происходит 
снижение мощности передачи данных с~$p_1^{\max}$ до~$p_0^{\max}$ 
и~прерывание обслуживания $k\hm-K_0$ устройств при условии, что число 
устройств $k\hm>K_0$. При переходе из недоступного в~доступное состояние 
мощность снова повышается. На рис.~3 представлена диаграмма 
интенсивностей переходов данного СП.
    
    

  Обозначим через $P_s(k)$, $s\hm=0,1$, условную ве\-ро\-ят\-ность того, что 
$(k+1)$-е M2M-устройство может быть обслужено при условии, что 
активно~$k$~устройств. Можно показать, что ве\-ро\-ят\-но\-сти~$P_s(k)$ 
вы\-чис\-ля\-ют\-ся по формулам:

\noindent
  \begin{align*}
  P_s(0) &={}\\
  &\hspace*{-8mm}{}=F_{\xi_d(\eta)} \left( \min 
  \left\{ R, \left( \fr{Gp_s^{\max}}{\left( 
e^{r_0/\omega} -1\right) N_0}\right)^{1/\kappa}\right\}\right)\,,\\
&\hspace*{50mm}s=0,1\,;
\end{align*}

\end{multicols}

\begin{figure*} %fig2
 \vspace*{1pt}
\begin{center}
\mbox{%
\epsfxsize=107.234mm
\epsfbox{gud-2.eps}
}
\end{center}
\vspace*{-11pt}
\Caption{Фрагмент диаграммы интенсивностей переходов СП с~детальными состояниями}
%\vspace*{-20pt}
\end{figure*}


\begin{multicols}{2}

\noindent
\begin{align*}
  P_s(k) &= \fr{\Phi\left( (1-m_{k+1,s})/\tau_{k+1,s}\right)}{\Phi\left((1-
m_{ks})/\tau_{ks}\right)}\,,\\
& \hspace*{20mm}k=1,\ldots, K_s,\enskip s=0,1\,,
  \end{align*}
где
\begin{gather*}
\Phi(x) =\fr{1}{\sqrt{2\pi}}\int\limits^x_{-\infty} e^{-t^2/2}\,dt\,;\\
m_{ks} =  kr_0E \left[ \fr{1}{r\left(d,p_s^{\max}\right)}\right]\,;
\end{gather*}

\vspace*{-12pt}

\noindent
\begin{multline*}
\tau^2_{ks}=kr_0^2\left( E\left[ \left( 
\fr{1}{r\left( d,p_s^{\max}\right)}\right)^2\right]- {}\right.\\
\left.{}-
\left( E\left[ \fr{1}{r\left(d, p_s^{\max}\right)}\right]\right)^2\right)\,;
\end{multline*}

\vspace*{-12pt}

\noindent
\begin{multline*}
E\left[ \fr{1}{r\left( d,p_s^{\max}\right)}\right] = \fr{1}{r_s^{\max}}\, 
F_{\xi_d(\eta)} (d_0) +{}\\
{}+
\int\limits_{d_0}^R \fr{1}{\omega \ln (1+Gp_s^{\max}/ 
(x^\kappa N_0))}\, f_{\xi_d(\eta)} (x)\,dx\,;
\end{multline*}

\vspace*{-14pt}

\noindent
\begin{multline*}
E\left[ \left(\fr{1}{r\left( d,p_s^{\max}\right)}\right)^2\right] = \left( \fr{1} 
{r_s^{\max}}\right)^2 F_{\xi_d(\eta)}(d_0) + {}\\
{}+\int\limits^R_{d_0} \fr{1} {\omega^2 
\ln^2 (1+Gp_s^{\max}/(x^\kappa N_0))}\,f_{\xi_d(\eta)} (x)\,dx\,.
\end{multline*}

\begin{figure*} %fig3
     \vspace*{1pt}
\begin{center}
\mbox{%
\epsfxsize=146.037mm
\epsfbox{gud-3.eps}
}
\end{center}
\vspace*{-9pt}
\Caption{Диаграмма интенсивностей переходов СП с~укрупненными состояниями}
\end{figure*}
  
  Случайный процесс $\{ \xi(t),\zeta(t), t\geq0\}$ является марковским, и~для расчета его 
стационарного рас-\linebreak\vspace*{-12pt}

\pagebreak

\noindent
пределения вероятностей~$p(k,s)$, $(k,s)\hm\in 
\mathbf{L}_1$ предлагается следующий рекуррентный алгоритм.
  \begin{enumerate}[1.]
  \item Значения ненормированных вероятностей $q(k,s)$ вычисляются по 
формулам:
  \begin{align*}
  q(0,0)&=1\,;\\
  q(0,1) &=x\,;\\
  q(k,s) & =\delta_{ks}+\gamma_{ks} x\,,\ (k,s)\in \mathbf{L}_1:\ k>0\,,
  \end{align*}
где
$$
x=\fr{(K_1\mu +\alpha)\delta_{K_11}-\lambda P_1(K_1-1)\delta_{K_1-1{,}1}} 
{\lambda P_1(K_1-1)\gamma_{K_1-1,1}-(K_1\mu +\alpha)\gamma_{K_11}}\,.
$$
\item Коэффициенты $\delta_{ks}$ и~$\gamma_{ks}$ вычисляются по 
рекуррентным формулам:
\begin{gather*}
\delta_{00}=1\,,\ \gamma_{00}=0\,;\\
\delta_{01}=0\,,\ \gamma_{01}=1\,;\\
\delta_{10}=\fr{\lambda P_0(0)+\beta}{\mu}\,,\ \gamma_{10}=-
\fr{\alpha}{\mu}\,;\\
\delta_{11}=-\fr{\beta}{\mu}\,,\ \gamma_{11}=\fr{\lambda 
P_1(0)+\alpha}{\mu}\,;
\end{gather*}

\vspace*{-12pt}

\noindent
\begin{multline*}
\delta_{k0}=\fr{\lambda P_0(k-1)+(k-1)\mu+\beta}{k\mu}\,\delta_{k-1,0} - {}\\
{}-
\fr{\lambda P_0(k-2)}{k\mu}\,\delta_{k-2,0}-\fr{\alpha}{k\mu}\,\delta_{k-1,1}\,,\ 
k=2,\ldots ,K_0,\hspace*{-0.26485pt}
\end{multline*}

\vspace*{-12pt}

\begin{multline*}
\gamma_{k0} = \fr{\lambda P_0(k-1)+(k-1)\mu+\beta}{k\mu}\,\gamma_{k-1,0}-{}\\
{}-
\fr{\lambda P_0(k-2)}{k\mu}\,\gamma_{k-2,0} -\fr{\alpha}{k\mu}\,\gamma_{k-
1,1}\,,\ k=2,\ldots,K_0;\hspace*{-1.73058pt}
\end{multline*}

%\vspace*{-12pt}

\noindent
\begin{gather*}
\delta_{k1} = \fr{\lambda P_1(k-1)+(k-1)\mu+\alpha}{k\mu}\,\delta_{k-1,1} - {}\\
{}-
\fr{\lambda P_1(k-2)}{k\mu}\,\delta_{k-2,1} -\fr{\beta}{k\mu}\,\delta_{k-1,0}\,,\\ 
k=2,\ldots ,K_0+1\,;\\
\hspace*{-3mm}\gamma_{k1} =\fr{\lambda P_1(k-1)+(k-1)\mu+\alpha}{k\mu}\,\gamma_{k-1,1} - {}\\
{}-
\fr{\lambda P_1(k-2)}{k\mu}\,\gamma_{k-2,1}- \fr{\beta}{k\mu}\,\gamma_{k-
1,0}\,,\\ 
k=2,\ldots ,K_0+1\,;\\
\delta_{k1} = \fr{\lambda P_1(k-1)+(k-1)\mu+\alpha}{k\mu}\,\delta_{k-1,1}- {}\\
{}-
\fr{\lambda P_1(k-2)}{k\mu}\,\delta_{k-2,1}\,,\   k=K_0+2,\ldots , K_1\,,\\
\gamma_{k1} =\fr{\lambda P_1(k-1)+(k-1)\mu+\alpha}{k\mu}\,\gamma_{k-1,1} - {}\\
{}-
\fr{\lambda P_1(k-2)}{k\mu}\,\gamma_{k-2,1}\,,\ k=K_0+2,\ldots ,K_1\,.
\end{gather*}

\item Значения вероятностей $p(k,s)$ вычисляются по формулам:

\noindent
$$
p(k,s)=\fr{q(k,s)}{\sum\nolimits_{(i,j)\in \mathbf{L}} q(i,j)}\,,\ (k,s)\in 
\mathbf{L}_1\,.
$$
\end{enumerate}

\begin{table*}\small
  \begin{center}
  \Caption{Исходные данные для численного анализа}
  \vspace*{2ex}
  
  \begin{tabular}{|l|c|c|c|}
  \hline
\multicolumn{1}{|c|}{Обозначение}&Случай 1  
(рис.~4)&Случай 2  
(рис.~5)&Случай 3  
(рис.~6)\\
\hline
$R$, м&200--400&200; 400&200; 400\\
$\omega$, МГц&10&10&10\\
$L$&15&15&15\\
$\alpha^{-1}$, мин&20; 30&20; 30&30\\
$\beta^{-1}$, с&20&20&20\\
$p_1^{\max}$, дБ$\cdot$м&23; 42&23--42&23; 42\\
$p_0^{\max}$, дБ$\cdot$м&$p_{\max}/2$&$p_{\max}/2$&$p_{\max}/2$\\
$d_0$, м&$R/15$&$R/15$&$R/15$\\
$r_0$, Мбит/с&1&1&1\\
$\lambda$, 1/с&10&10&2--10\\
$\mu^{-1}$, с&0,1&0,1&0,1\\
$N_0$, дБ$\cdot$м&$-60$&$-60$&$-60$\\
$G$&197,43&197,43&197,43\\
$\kappa$&5&5&5\\
\hline
\end{tabular}
\end{center}
%\vspace*{-6pt}
\end{table*}
\begin{figure*}[b] %fig4
% \vspace*{-6pt}
\begin{center}
\mbox{%
\epsfxsize=162.099mm
\epsfbox{gud-4.eps}
}
\end{center}
\vspace*{-9pt}
\Caption{Показатели эффективности в~зависимости от мощности устройств: 
(\textit{а})~вероятность прерывания обслуживания при 
$R=400$ (\textit{1}~--- $\alpha\hm=1200$; 
\textit{2}~--- $\alpha\hm=1800$); 
(\textit{б})~среднее число активных устройств 
при $\alpha\hm=1800$ (\textit{3}~--- $R\hm=200$; 
\textit{4}~---  $R\hm=400$)}
\end{figure*}

\vspace*{-9pt}

\section{Пример численного анализа и~заключение}

\vspace*{-3pt}

  Зная стационарное распределение вероятностей  $p(k,s)$, 
  $(k,s)\hm\in \mathbf{L}_1$, найдем 
основные показатели эффективности модели~--- вероятность~$B$ блокировки, 
вероятность~$\Pi$ прерывания обслуживания и~среднее число~$\overline{K}$ 
устройств по формулам:

\noindent
  \begin{align*}
  B &= \sum\limits_{k=0}^{K_0-1} \left( 1-P_0(k)\right) p(k,0) 
+{}\notag\\
&\hspace*{30mm}{}+\sum\limits_{k=0}^{K_1-1}\left( 1-P_1(k)\right) p(k,1)\,;
  %\label{e4-gud}
  \\
  \Pi &=  \hspace*{-2mm}\hspace*{-0.72604pt}\sum\limits_{k=K_0+1}^{K_1-1}
  \hspace*{-1mm}
   \fr{\alpha}{\alpha+k\mu+\lambda 
P_1(k)}\, \fr{\begin{pmatrix} k-1\\ k-K_0-1\end{pmatrix}}{ \begin{pmatrix} k\\ k-
K_0\end{pmatrix}}\,p(k,1)+{}\notag\\
&\hspace*{10mm}{}+\fr{\alpha}{\alpha+K_1\mu}\,\fr{\begin{pmatrix} K_1-
1\\ K_1-K_0-1\end{pmatrix}} {\begin{pmatrix} K_1\\ K_1-K_0\end{pmatrix}} 
\,p(K_1,1)\,;
  %\label{e5-gud}
  \\
  \overline{K} &= \sum\limits_{k=0}^{K_0} kp (k,0) +\sum\limits_{k=0}^{K_1} 
kp(k,1)\,.
  %\label{e6-gud}
  \end{align*}
  
  Для проведения численного анализа проанализируем передачу данных  
M2M-устройствами небольшими сессиями, составляющими в~среднем~10~с, 
в~высоком качестве на ско\-рости~1~Мбит/с. 
Рассмотрим небольшой аэропорт, 
в~котором самолеты взлетают раз в~20~(30)~мин, среднее время пролета 
самолета над сотой составляет~20~с. 
Исходные данные представлены в~табл.~2.
  
  
  
  На рис.~4 показана зависимость вероятности прерывания обслуживания 
и~среднего числа активных устройств от их мощности. Вероятность\linebreak 
прерывания обслуживания уменьшается пропорционально увеличению 
мощности, так как при более высокой мощности для устройств достижима 
более высокая скорость. При этом вероятность прерывания ниже для более 
низкой интен\-сив\-ности отключения полосы.
Среднее число устройств 
увеличивается пропорционально радиусу соты
 (см.\linebreak

\end{multicols}

\begin{figure*} %fig5
 \vspace*{1pt}
\begin{center}
\mbox{%
\epsfxsize=161.797mm
\epsfbox{gud-5.eps}
}
\end{center}
\vspace*{-9pt}
\Caption{Показатели эффективности в~зависимости от радиуса соты: 
(\textit{а})~вероятность 
прерывания обслуживания при $W\hm=0{,}2$
(\textit{1}~--- $\alpha\hm=1200$;
\textit{2}~--- $\alpha\hm=1800$); (\textit{б})~среднее число активных 
устройств при $\alpha\hm=1200$ (\textit{3}~--- $W\hm=0{,}2$;
\textit{4}~--- $W=15{,}85$)}
%\end{figure*}
%\begin{figure*} %fig6
 \vspace*{6pt}
\begin{center}
\mbox{%
\epsfxsize=159.48mm
\epsfbox{gud-6.eps}
}
\end{center}
\vspace*{-9pt}
\Caption{Показатели эффективности в~зависимости от интенсивности потока пакетов 
данных от устройств: (\textit{а})~вероятность прерывания обслуживания при 
$R=400$ (\textit{1}~--- $\alpha\hm=1200$; 
\textit{2}~---  $\alpha\hm=1800$); (\textit{б})~среднее 
число активных устройств (\textit{3}~---  $R\hm=400$, $\alpha\hm=1200$;
\textit{4}~--- $R\hm=200$, $\alpha\hm=1800$)}
\vspace*{-30pt}
\end{figure*}

\begin{multicols}{2}

\noindent
 рис.~5). Вероятность 
прерывания оказывается ниже при меньшей интенсивности изъятия полосы 
(см.\ рис.~6). 





  В заключение отметим, что в~статье разработана вероятностная модель 
совместного использования радиочастот, при помощи которой проведен анализ 
показателей эффективности применения политики управления мощ\-ностью 
с~учетом разноудаленных от БС M2M-устройств. 

В~дальнейшем 
предполагается учесть случайную высоту, на которой могут находиться 
устройства.

\vspace*{-12pt}

{\small\frenchspacing
 { %\baselineskip=10pt
 \addcontentsline{toc}{section}{References}
 \begin{thebibliography}{99}
\bibitem{1-gud}
Cisco Visual Networking Index: Global Mobile Data Traffic Forecast Update, 2016--2021 White 
Paper. March~28, 2017. {\sf  
http://www.cisco.com/c/en/us/ solutions/collateral/service-provider/visual-networking-index-vni/mobile-white-paper-c11-520862.html}.
\bibitem{2-gud}
\Au{Andrews J., Buzzi~S., Choi~W., Hanly~S.\,V., Lozano~A., Soong~C.\,K., Zhang~J.\,C.} What 
will 5G be?~// IEEE J.~Sel. Area. Comm., 2014. Vol.~32. P.~1065--1082. 

\bibitem{5-gud} %3
ETSI TR 103 113. Electromagnetic compatibility and Radio spectrum Matters 
(ERM); System Reference document (SRdoc); Mobile broadband services in the  
2\,300~MHz\,--\,2\,400~MHz frequency band under Licensed Shared Access regime. 
v1.1.1. July 2013. 
{\sf 
http:// www.etsi.org/deliver/etsi\_tr/103100\_103199/103113/ 01.01.01\_60/tr\_103113v010101p.pdf}.

\bibitem{3-gud} %4
ETSI TR 103 154. Reconfigurable Radio Systems (RRS); System requirements 
for operation of Mobile Broadband Systems in the 2\,300~MHz\,--\,2\,400~MHz band under Licensed 
Shared Access (LSA). v1.1.1. October 2014.
{\sf 
http://www.etsi.org/deliver/etsi\_TS/103100\_103199/ 103154/01.01.01\_60/ts\_103154v010101p.pdf}.

\bibitem{4-gud} %5
ETSI TR 103 235. Reconfigurable Radio Systems (RRS); System architecture 
and high level procedures for operation of Licensed Shared Access (LSA) in  
the 2\,300~MHz\,--\,2\,400~MHz band. v1.1.1. October 2015. {\sf 
http://www.\linebreak
 etsi.org/deliver/etsi\_ts\%5C103200\_103299\%5C103235
 \%5C01.01.01\_60\%5Cts\_103235v010101p.pdf}.

\bibitem{6-gud} %6
\Au{Buckwitz K., Engelberg J., Rausch~G.} Licensed Shared Access (LSA)~--- regulatory 
background and view of Administrations~// 9th Conference (International) on Cognitive Radio 
Oriented Wireless Networks.~--- IEEE, 2014. P.~413--416.
\bibitem{7-gud}
\Au{Ahokangas P., Matinmikko~M., Yrj$\ddot{\mbox{o}}$l$\ddot{\mbox{a}}$~S., 
Mustonen~M., 
Posti~H., Luttinen~E., Kivim$\ddot{\mbox{a}}$ki~A.} Business models for mobile network 
operators in Licensed Shared Access (LSA)~// IEEE Symposium (International) on Dynamic 
Spectrum Access Networks.~--- IEEE, 2014. P.~263--270.


\bibitem{9-gud} %8
\Au{Borodakiy~V.\,Y., Samouylov~K.\,E., Gudkova~I.\,A., Ostrikova~D.\,Y., Ponomarenko~A.\,A., 
Turlikov~A.\,M., Andreev~S.\,D.} Modeling unreliable LSA operation in 3GPP LTE cellular 
networks~// 6th Congress (International) on Ultra Modern Telecommunications and Control 
Systems and Workshops Proceedings.~--- Piscataway, NJ, USA: IEEE, 2015. 
P.~490--496.

\bibitem{8-gud} %9
\Au{Ponomarenko-Timofeev A., Pyattaev~A., Andreev~S., Kou\-che\-rya\-vy~Ye., Mueck~M., Karls~I}. 
Highly dynamic spectrum management within licensed shared access regulatory framework~// 
IEEE Commun. Mag., 2015. Vol.~54. No.\,3. P.~100--109.

\bibitem{10-gud}
\Au{Gudkova I.\,A., Samouylov~K.\,E., Ostrikova~D.\,Y., Mokrov~E.\,V.,  
Ponomarenko-Timofeev~A.\,A., Andreev~S.\,D., Koucheryavy~Y.\,A.} Service failure and 
interruption probability analysis for Licensed Shared Access regulatory framework~// 7th Congress 
(International) on Ultra Modern Telecommunications and Control Systems and Workshops
 Proceedings.~--- Piscataway, NJ, USA: IEEE Computer Society, 2015. P.~123--131.

\bibitem{11-gud}
\Au{Samouylov K., Gudkova~I., Markova~E., Yarkina~N.} Queuing model with unreliable servers 
for limit power policy within Licensed Shared Access framework~// Internet of things, smart
spaces, and next generation networks and systems~/
Eds. O.~Galinina, S.~Balankin, Y.~Koucheryavy.~---
Lecture notes in computer 
science ser.~--- Springer, 2016. Vol.~9870. P.~404--413.
\bibitem{12-gud}
\Au{Galinina O., Andreev~S.\,D., Gerasimenko~M., Kou\-che\-rya\-vy~Y.\,A., Himayat~N., Yeh S.-P., 
Talwar~S.} Capturing spatial randomness of heterogeneous cellular/WLAN deployments with 
dynamic traffic~// IEEE J.~Sel. Area. Comm., 2014. Vol.~32.  
No.\,6. P.~1083--1099.
\bibitem{13-gud}
\Au{Ahmadian A., Galinina~O., Gudkova~I., Andreev~S., Shorgin~S., Samouylov~K.} On capturing 
spatial diversity of joint M2M/H2H dynamic uplink transmissions in 
3GPP LTE cellular system~// Internet of things, smart
spaces, and next generation networks and systems~/
Eds.\ S.~Balandin, S.~Andreev, Y.~Koucheryavy.~---
Lecture notes in computer science ser.~--- Springer, 2014. Vol.~9247. P.~407--421.
\bibitem{14-gud}
\Au{Samouylov K., Gudkova~I., Markova~E., Dzantiev~I.} On analyzing the blocking probability 
of M2M transmissions for a CQI-based RRM scheme model in 3GPP LTE~// Comm. 
Com. Inf. Sci., 2016. Vol.~638. P.~327--340.
\bibitem{15-gud}
\Au{Gudkova I., Markova~E., Masek~P., Andreev~S., Hosek~J., Yarkina~N., Samouylov~K., 
Koucheryavy~Y.} Modeling the utilization of a multi-tenant band in 3GPP LTE system with 
Licensed Shared Access~// 8th Congress (International) on Ultra Modern Telecommunications and 
Control Systems and Workshops Proceedings.~--- Piscataway, NJ, USA: IEEE, 
2016. P.~179--183.
 \end{thebibliography}

 }
 }

\end{multicols}

\vspace*{-6pt}

\hfill{\small\textit{Поступила в~редакцию 20.04.17}}

\vspace*{6pt}

%\newpage

%\vspace*{-24pt}

\hrule

\vspace*{2pt}

\hrule

\vspace*{-4pt}


\def\tit{PROBABILITY MODEL FOR ANALYZING LICENSED SHARED~ACCESS WITH~ADAPTIVE 
POWER CONTROL IN~A~WIRELESS~NETWORK}

\def\titkol{Probability model for analyzing licensed shared access with adaptive 
power control in a wireless network}

\def\aut{I.\,A.~Gudkova$^{1,2}$ and~S.\,Ya.~Shorgin$^2$}

\def\autkol{I.\,A.~Gudkova and  S.\,Ya.~Shorgin}

\titel{\tit}{\aut}{\autkol}{\titkol}

\vspace*{-9pt}


\noindent
$^1$Peoples' Friendship University of Russia, 6~Miklukho-Maklaya Str., Moscow 117198, Russian Federation

\noindent
$^2$Institute of Informatics Problems, Federal Research Center ``Computer Science and Control'' of the 
Russian\linebreak
$\hphantom{^1}$Academy of Sciences, 44-2~Vavilov Str., Moscow 119333, Russian Federation



\def\leftfootline{\small{\textbf{\thepage}
\hfill INFORMATIKA I EE PRIMENENIYA~--- INFORMATICS AND
APPLICATIONS\ \ \ 2017\ \ \ volume~11\ \ \ issue\ 3}
}%
 \def\rightfootline{\small{INFORMATIKA I EE PRIMENENIYA~---
INFORMATICS AND APPLICATIONS\ \ \ 2017\ \ \ volume~11\ \ \ issue\ 3
\hfill \textbf{\thepage}}}

\vspace*{3pt}



\Abste{Emerging next generation wireless networks involve new applications and services for 
human-to-human and machine-to-machine (M2M) devices. The problem of increasing 
requirements for network capacity and lack of radio spectrum arises. The solution could be found in 
the licensed shared access framework, e.\,g., in the case of smart cities. The authors 
propose a mathematical model of shared access to spectrum with adaptive power control. The 
algorithm makes it possible to avoid the interference between M2M devices and the spectrum 
owner due, in part, to the fact that it takes into account the spatial distribution and session activity of 
devices.}

%\pagebreak

\KWE{wireless network; smart city; machine-to-machine (M2M); licensed shared 
access (LSA); 
adaptive power control; stochastic process; recursive algorithm; 
blocking probability; interruption 
probability; average number of M2M devices}




\DOI{10.14357/19922264170310} 

%\vspace*{-18pt}

%\pagebreak

\Ack
\noindent
This work was financially supported by the Russian Science 
Foundation (grant No.\,16-11-10227).



%\vspace*{3pt}

  \begin{multicols}{2}

\renewcommand{\bibname}{\protect\rmfamily References}
%\renewcommand{\bibname}{\large\protect\rm References}

{\small\frenchspacing
 {\baselineskip=10.282pt
 \addcontentsline{toc}{section}{References}
 \begin{thebibliography}{99}
\bibitem{1-gud-1}
Cisco Visual Networking Index: Global Mobile Data Traffic Forecast Update, 2016--2021 White 
Paper. March~28, 2017. Available at: {\sf 
http://www.cisco.com/c/en/us/ solutions/collateral/service-provider/visual-networking-index-vni/mobile-white-paper-c11-520862.html} (accessed June~26, 2017).
\bibitem{2-gud-1}
\Aue{Andrews, J., S.~Buzzi, W.~Choi, S.\,V.~Hanly, A.~Lozano, C.\,K.~Soong, and 
J.\,C.~Zhang.} 2014. What will 5G be? \textit{IEEE J.~Sel. Area. Comm.}  
32:1065--1082. 

\bibitem{5-gud-1} %3
ETSI TR 103 113. July 2013. Electromagnetic compatibility and Radio spectrum 
Matters (ERM); System 
Reference document (SRdoc); Mobile broadband services in the 2300~MHz\,--\,2400~MHz frequency 
band under Licensed Shared Access regime. Available at: {\sf 
http://www.etsi.org/deliver/etsi\_tr/103100\_103199/ 103113/01.01.01\_60/tr\_103113v010101p.pdf}  
(accessed June 26, 2017).

\bibitem{3-gud-1} %4
ETSI TR 103 154. October 2014. Reconfigurable Radio Systems (RRS); System requirements for operation of 
Mobile Broadband Systems in the 2300~MHz\,--\,2400~MHz band under Licensed Shared Access 
(LSA). v1.1.1.  Available at: {\sf 
http://www.etsi.org/deliver/etsi\_TS/103100\_\linebreak  
103199/103154/01.01.01\_60/ts\_103154v010101p.pdf} (accessed June~26, 2017).
\bibitem{4-gud-1} %5
ETSI TR 103 235. October 2015. Reconfigurable Radio Systems (RRS); System architecture and high level 
procedures for operation of Licensed Shared Access (LSA) in the 
2300~MHz\,--\,2400~MHz band.
v1.1.1. Available at: {\sf 
http://www.etsi.org/deliver/etsi\_ts\%5C103200\_103299
\%5C103235\%5C01.01.01\_60\%5Cts\_103235v010101p. pdf} (accessed June~26, 2017).

\bibitem{6-gud-1}
\Aue{Buckwitz, K., J.~Engelberg, and G.~Rausch.} 2014. Licensed Shared Access (LSA)~--- 
regulatory background and view of Administrations. \textit{9th Conference (International) on 
Cognitive Radio Oriented Wireless Networks}. IEEE. 413--416.
\bibitem{7-gud-1}
\Aue{Ahokangas, P., M. Matinmikko, S.~Yrj$\ddot{\mbox{o}}$l$\ddot{\mbox{a}}$, 
M.~Mustonen, H.~Posti, E.~Luttinen, and A.~Kivim$\ddot{\mbox{a}}$ki.} 2014. Business 
models for mobile network operators in Licensed Shared Access (LSA). \textit{IEEE Symposium 
(International) on Dynamic Spectrum Access Networks}. IEEE. 263--270.

\bibitem{9-gud-1} %8
\Aue{Borodakiy, V.\,Y., K.\,E.~Samouylov, I.\,A.~Gudkova, D.\,Y.~Ostrikova, 
A.\,A.~Ponomarenko, A.\,M.~Turlikov, and S.\,D.~Andreev.} 2014. Modeling unreliable LSA 
operation in 3GPP LTE cellular networks. \textit{6th Congress (International) on Ultra Modern 
Telecommunications and Control Systems and Workshops Proceedings}. Piscataway, NJ: IEEE.  
490--496.

\bibitem{8-gud-1} %9
\Aue{Ponomarenko-Timofeev, A., A.~Pyattaev, S.~Andreev, Ye.~Koucheryavy, M.~Mueck, and 
I.~Karls.} 2015. Highly dynamic spectrum management within licensed shared access regulatory 
framework. \textit{IEEE Commun. Mag.} 54(3):100--109.

\bibitem{10-gud-1}
\Aue{Gudkova, I.\,A., K.\,E.~Samouylov, D.\,Y.~Ostrikova, E.\,V.~Mokrov,  
A.\,A.~Ponomarenko-Timofeev, S.\,D.~Andreev, and Y.\,A.~Koucheryavy}. 2015. Service failure 
and interruption probability analysis for Licensed Shared Access regulatory framework. \textit{7th 
Congress (International) on Ultra Modern Telecommunications and Control Systems and Workshops
Proceedings}. Piscataway,  NJ: IEEE. 123--131.
\bibitem{11-gud-1}
\Aue{Samouylov, K., I.~Gudkova, E.~Markova, and N.~Yarkina}. 2016. Queuing model with 
unreliable servers for limit power policy within Licensed Shared Access framework. 
\textit{Internet of things, smart
spaces, and next generation networks and systems}.
Eds. O.~Galinina, S.~Balankin, Y.~Koucheryavy.
{Lecture 
notes in computer science ser.} Springer. 9870:404--413.
\bibitem{12-gid-1}
\Aue{Galinina, O., S.\,D.~Andreev, M.~Gerasimenko, Y.\,A.~Koucheryavy, N.~Himayat,  
S.-P.~Yeh, and S.~Talwar.} 2014. Capturing spatial randomness of heterogeneous cellular/WLAN 
deployments with dynamic traffic. \textit{IEEE J.~Sel. Area. Comm.}  
32(6):1083--1099.
\bibitem{13-gud-1}
\Aue{Ahmadian, A., O.~Galinina, I.~Gudkova, S.~Andreev, S.~Shorgin, and K.~Samouylov.} 
2014. On capturing spatial diversity of joint M2M/H2H dynamic uplink transmissions in 3GPP 
LTE cellular system. 
\textit{Internet of things, smart
spaces, and next generation networks and systems}.
Eds.\ S.~Balandin, S.~Andreev, Y.~Koucheryavy.
{Lecture notes in computer science ser.} Springer. 9247:407--421.
\bibitem{14-gud-1}
\Aue{Samouylov, K., I.~Gudkova, E.~Markova, and I.~Dzantiev.} 2016. On analyzing the 
blocking probability of M2M transmissions for a CQI-based RRM scheme model in 3GPP LTE. 
\textit{Comm. Com. Inf. Sci.} 638:327--340.
\bibitem{15-gud-1}
\Aue{Gudkova, I., E.~Markova, P.~Masek, S.~Andreev, J.~Hosek, N.~Yarkina, K.~Samouylov, 
and Y.~Koucheryavy.} 2016. Modeling the utilization of a multi-tenant band in 3GPP LTE system 
with Licensed Shared Access. \textit{8th Congress (International) on Ultra Modern 
Telecommunications and Control Systems and Workshops Proceedings}.  Piscataway, NJ: IEEE.  
179--183.
\end{thebibliography}

 }
 }

\end{multicols}

\vspace*{-9pt}



\hfill{\small\textit{Received April 20, 2017}}

\vspace*{-24pt}

\Contr

\noindent
\textbf{Gudkova Irina A.}\ (b.\ 1985)~--- Candidate of Sciences (PhD) in physics and 
mathematics; associate professor, Peoples' Friendship University of Russia,  
6~Miklukho-Maklaya Str., Moscow 117198, Russian Federation; senior scientist, Institute 
of Informatics Problems, Federal Research Center ``Computer Science and Control'' of the 
Russian Academy of Sciences, 44-2~Vavilov Str., Moscow 119333, Russian Federation; 
\mbox{ gudkova\_ia@rudn.university }

%\vspace*{1pt}

\noindent
\textbf{Shorgin Sergey Ya.} (b.\ 1952)~--- Doctor of Science in physics and mathematics, professor; Deputy Director, Federal Research Center 
``Computer Science and Control'' of the Russian Academy of Sciences (FRC CSC RAS); principal scientist, Institute of Informatics Problems, FRC 
CSC RAS; 44-2~Vavilov Str., Moscow 119333, Russian Federation; \mbox{sshorgin@ipiran.ru}

\label{end\stat}


\renewcommand{\bibname}{\protect\rm Литература}    %10
\def\stat{sopin}

\def\tit{СИСТЕМА МАССОВОГО ОБСЛУЖИВАНИЯ 
С~ОГРАНИЧЕННЫМИ РЕСУРСАМИ И~СИГНАЛАМИ ДЛЯ~АНАЛИЗА 
ПОКАЗАТЕЛЕЙ ЭФФЕКТИВНОСТИ БЕСПРОВОДНЫХ СЕТЕЙ$^*$}

\def\titkol{Система массового обслуживания с~ограниченными ресурсами и~сигналами 
для анализа 
показателей эффективности} % беспроводных сетей}

\def\aut{К.\,Е.~Самуйлов$^1$, Э.\,С.~Сопин$^2$, С.\,Я.~Шоргин$^3$}

\def\autkol{К.\,Е.~Самуйлов, Э.\,С.~Сопин, С.\,Я.~Шоргин}

\titel{\tit}{\aut}{\autkol}{\titkol}

\index{Самуйлов К.\,Е.}
\index{Сопин Э.\,С.}
\index{Шоргин С.\,Я.}
\index{Samouylov K.\,E.}
\index{Sopin E.\,S.} 
\index{Shorgin S.\,Ya.}


{\renewcommand{\thefootnote}{\fnsymbol{footnote}} \footnotetext[1]
{Исследование выполнено при финансовой поддержке Российского научного фонда 
в~рамках научного проекта №\,16-11-10227.}}


\renewcommand{\thefootnote}{\arabic{footnote}}
\footnotetext[1]{Российский университет дружбы народов; Институт проб\-лем информатики 
Федерального исследовательского центра <<Информатика и~управ\-ле\-ние>> Российской 
академии наук, \mbox{samouylov\_ke@rudn.university}}
\footnotetext[2]{Российский университет дружбы народов; Институт проб\-лем информатики 
Федерального исследовательского центра <<Информатика и~управ\-ле\-ние>> Российской 
академии наук, \mbox{sopin\_es@rudn.university}}
\footnotetext[3]{Институт проблем информатики Федерального исследовательского центра 
<<Информатика и~управ\-ле\-ние>> Российской академии наук, 
\mbox{sshorgin@ipiran.ru}}

%\vspace*{-18pt}
   
 
      
  
  \Abst{Рассматривается многолинейная система массового обслуживания  (СМО)
с~ресурсами ограниченного объема. Поступающая заявка занимает не только прибор, но 
и~некоторый объем ресурсов на все время обслуживания. Помимо потока заявок на 
систему поступает поток сигналов, при поступлении которых заявки заново разыгрывают 
объем занимаемых ресурсов. Рассматриваемая система массового обслуживания 
позволяет описывать функционирование беспроводной сети с~учетом перемещения 
пользователей в~течение периода жизни пользовательской сессии. Исследуются две 
модели перемещения пользователей. В~первой пользователи перемещаются независимо 
друг от друга; следовательно, в~соответствующей математической модели поступление 
сигнала изменяет занимаемый объем ресурсов только одной заявки. Во второй модели 
пользователи перемещаются совместно, поэтому занимаемый объем ресурсов меняется 
одновременно у всех заявок.}

\KW{ограниченные ресурсы; сигнал; система массового обслуживания; 
беспроводная сеть; сети связи 4-го поколения}

\DOI{10.14357/19922264170311} 


\vskip 10pt plus 9pt minus 6pt

\thispagestyle{headings}

\begin{multicols}{2}

\label{st\stat}
  
\section{Введение}

  Для анализа показателей качества услуг в~современных сетях связи 4-го 
поколения с~объектами в~движении широко используется имитационное 
моделирование~[1, 2] и~простые модели теории массового обслуживания~[3] 
с~фиксированным объемом требований. 

В~работах~[4, 5] предлагается 
анализировать показатели эффективности модели современной беспроводной 
гетерогенной сети связи в~виде СМО
ограниченной емкости с~требованиями случайного объема. В~отличие от 
моделей, представленных в~[6, 7], моделирование беспроводной сети 
в~терминах теории массового обслуживания учитывает процессы 
установления новых сессий и~их завершения, а заданная специальным 
образом функция распределения случайных требований к~радиоресурсам 
позволяет описать функционирование планировщика в~соответствии 
с~выбранной политикой распределения частотных ресурсов и~моделью 
распространения сигнала. 

В~предложенной в~[8] экспоненциальной модели 
каждая сессия занимает выделенный ей объем частотного ресурса на все 
время ее длительности. По завершении сессии предполагается освободить 
некоторый случайный объем ресурсов, отличный от занимаемого, так как 
местоположение и~число пользователей в~сети могут с~течением времени 
измениться. Однако данная модель не учитывает изменения в~сети, которые 
могут произойти до завершения обслуживания сессий. 
  
  В~данной работе исследуются модели, в~которых объем занимаемых 
заявками ресурсов может меняться до завершения обслуживания, при 
поступлении сигнала. Эта особенность позволяет моделировать 
функционирование беспроводной сети, в~которой пользователи, удаляясь или 
приближаясь к~базовой станции, увеличивают или уменьшают требуемый 
объем ресурсов в~течение периода жизни пользовательской сессии. 
  %
  При этом в~связи с~тем, что в~беспроводных сетях задача поддержания уже 
принятых сессий имеет более высокий приоритет по сравнению с~задачей 
принятия на обслуживание новых, диспетчеры ресурсов планируют их таким 
образом, чтобы ни в~коем случае не прерывать текущие сессии~[9]. Поэтому 
в~исследуемых в~данной работе моделях поступление сигнала, изменяющего 
объем занятого заявкой ресурса, не может привести к~потере заявки.
  
  Рассматриваются два сценария перемещения пользователей 
в~беспроводной сети. Согласно первому сценарию пользователи 
перемещаются независимо друг от друга; следовательно, моменты изменения 
объемов занимаемых заявками ресурсов не зависят друг от друга. 
  
  Второй сценарий предполагает одновременное перемещение 
пользователей в~соте, например в~общест\-венном транспорте. В~этом случае 
все пользователи перемещаются относительно базовой станции 
одновременно и~достаточно иметь пред\-став\-ле\-ние о~том, каким образом 
изменяется совокупно занимаемый объем ресурса.
  
\section{Независимое перемещение пользователей}

  Рассмотрим модель соты, которая может обслуживать одновременно не 
более~$N$~сессий связи с~устройствами. Объем доступных  
час\-тот\-но-вре\-мен\-н$\acute{\mbox{ы}}$х ресурсов ограничен и~не 
превышает~$R$~единиц. Для установления новой сессии каждое устройство 
требует выделить ему некоторый случайный объем ресурсов $0\hm\leq 
r\hm\leq R$. Если в~соте одновременно установлено не более~$N$~сессий 
и~требование к~ресурсу новой сессии не превышает объема свободного 
ресурса, то сессия будет установлена, в~противном случае она будет 
отклонена. Пусть в~некоторый момент времени один из активных 
пользователей изменит свое местоположение относительно базовой станции; 
в~этом случае он либо освободит часть занимаемого ресурса, либо базовая 
станция выделит ему дополнительный ресурс, не превышающий объем 
доступного в~этот момент ресурса соты. 
  
  Опишем теперь предложенную модель в~терминах теории массового 
обслуживания. Рассматривается СМО с~$N\hm<\infty$ приборами, 
обладающая некоторым объемом ресурсов $R\hm<\infty$. Введем основные 
предположения.
  \begin{enumerate}[1.]
\item В~систему поступает пуассоновский поток заявок 
с~интенсивностью~$\lambda$, время обслуживания заявок имеет 
экспоненциальное распределение с~параметром~$\mu$. 
\item Для обслуживания $i$-й поступающей заявки требуется~$r_i$ 
ресурса, $r_i\hm\geq 0$, с~вероятностью~$p_{r_i}$. 
\item Если в~момент поступления $i$-й заявки в~сис\-те\-ме находится 
$k\hm<N$ заявок, занимающих $r_\bullet \hm= r_1+\cdots+r_k$ ресурсов, 
и~$r_i\hm\leq R\hm-r_\bullet$, то заявка будет принята к~обслуживанию; 
в~противном случае заявка будет потеряна.
\item Каждая заявка, находящаяся на обслуживании, порождает 
пуассоновский поток сигналов с~ин\-тен\-сив\-ностью~$\gamma$, при 
поступлении которого она освобождает весь занимаемый ею ресурс, чтобы 
занять новый объем ресурса.
  \end{enumerate}
  
  Пусть в~некоторый момент времени $t\hm>0$ в~сис\-те\-ме находится~$\xi(t)$ 
заявок, которые занимают $\eta_1(t),\ldots ,\eta_{\xi(t)}(t)$ ресурсов. 
Функционирование сис\-те\-мы описывает случайный процесс (СП) $X(t)\hm=\left( 
\xi(t), \eta_1(t),\ldots , \eta_{\xi(t)}(t)\right)$, однако для дальнейшего анализа 
модели удобно воспользоваться упрощением, предложенным в~\cite{10-sop} 
для СМО с~ресурсами, позволяющими снизить размер простран\-ства 
состояний системы за счет отслеживания только суммарного объема 
ресурсов $\delta(t)\hm= \eta_1(t)+\cdots+\eta_{\xi(t)}(t)$. В~дальнейшем будем 
исследовать СП $\tilde{X}(t)\hm= \left( \xi(t),\delta(t)\right)$. 
  
  Рассмотрим подробнее возможные переходы между состояниями системы. 
Пусть в~некоторый момент времени система находится в~состоянии $(k,r)$. 
С~вероятностью~$p_j$ в~систему может поступить заявка, которая 
займет~$j$~единиц ресурса, если $j\hm\leq R\hm-r$. Из-за того что 
неизвестно число занимаемых ресурсов каждой заявкой, невозможно точно 
определить объем высвобождаемых ресурсов при завершении обслуживания. 
Поэтому будем считать, что заявка освобождает~$i$~единиц ресурса 
с~вероятностью $p_ip_{r-i}^{(k-1)}/p_r^{(k)}$, где $p_r^{(k)}$ является  
$k$-крат\-ной сверткой распределения~$\{p_i\}$, $i\hm\geq0$. Данную 
вероятность можно интерпретировать как вероятность того, что заявка 
занимает~$i$~единиц ресурса при условии, что~$k$~заявок суммарно 
занимают~$r$~ресурсов. 
  
  В момент поступления сигнала одна из заявок системы сначала 
освобождает занимаемые ею~$i$~единиц ресурса с~вероятностью~$p_i p_{r-
i}^{(k-1)}/p_r^{(k)}$ и~занимает~$j$~единиц ресурса с~нормированной 
вероят\-ностью~$p_j/\sum\nolimits_{s=0}^{R-r+i} p_s$, поскольку потери при 
поступле\-нии сигнала по условию не происходят.
  
  Пространство состояний системы описывается множеством
  
  \noindent
   $$
    \mathsf{X}^{\sptilde} \hm= \mathop{\bigcup}\limits_{k=0}^N 
   {\sf X}_k^{\sptilde}\,,
   $$
   
   \vspace*{-2pt}
   
   \noindent
    где ${\mathsf X}_k^{\sptilde} \hm= \left\{ 
(k,r): 0\hm\leq r\hm\leq R,\ p_r^{(k)}\hm>0\right\}$. Упорядочив состояния 
в~множествах~${\mathsf X}_k^{\sptilde}$, $0\hm\leq k\hm\leq N$, по возрастанию числа 
ресурсов, введем функции $I(k,r)$, значения которых равны порядковому 
номеру состояния $(k,r)$ в~множестве~${\sf X}_k^{\sptilde}$.
  
  Матрица интенсивностей переходов СП $\tilde{X}(t)$ 
  $$
  \mathbf{A}=  [a((i,j),(k,r))]$$ 
  является блочной трехдиагональной матрицей 
c~диагональными блоками $\boldsymbol{\Psi}_0, 
\boldsymbol{\Psi}_1, \ldots , \boldsymbol{\Psi}_N$, наддиа-\linebreak\vspace*{-12pt}

\pagebreak

\noindent
гональными 
блоками $\boldsymbol{\Lambda}_1,\ldots , \boldsymbol{\Lambda}_N$ и~поддиагональными блоками 
$\mathbf{M}_0,\ldots , \mathbf{M}_{N-1}$, где 
\begin{align*}
\Psi_0&= -\lambda 
\sum\limits_{j=0}^R p_j\,;\ 
 \boldsymbol{\Lambda}_1=\left(\lambda p_0,\ldots, \lambda p_r\right)\,;
\\
\mathbf{M}_0&= \left(\mu, \ldots, \mu\right)^{\mathrm{T}}\,,
\end{align*}
 а~остальные матрицы 
$\{\boldsymbol{\Psi}_n\}_{1\leq n\leq N}$, $\{\boldsymbol{\Lambda}_n\}_{2\leq n\leq N}$ 
и~$\{\mathbf{M}_n\}_{1\leq n\leq N-1}$ имеют следующие элементы:
  \begin{multline}
  \psi_n(I(n,i),I(n,j))={}\\
  {}=\begin{cases}
  \displaystyle-\left[ \lambda \sum\limits_{k=0}^{R-i} p_k 
+n\mu+n\gamma\right]\,, &\ i=j\,;\\[5pt]
  \displaystyle n\gamma\sum\limits_{s=0}^i \fr{p_sp_{i-s}^{(n-
1)}}{p_i^{(n)}}\,\fr{p_{j-i+s}}{\sum\nolimits_{k=0}^{R-i+s}p_k}\,, &\ i<j\,;\\[5pt]
  \displaystyle n\gamma\sum\limits^i_{s=i-j} \fr{p_s p_{i-s}^{(n-1)}} 
{p_i^{(n)}}\, \fr{p_{j-i+s}}{\sum\nolimits_{k=0}^{R-i+s} p_k}\,, &\ i>j\,,
  \end{cases}\\
     (n,i), (n,j)\in {\sf X}_n^{\sptilde}\,,\enskip 
  n=\overline{1,N-1}\,;
  \label{e1-sop}
  \end{multline}
  
  \vspace*{-12pt}
  
\noindent
  \begin{multline}
  \lambda_n(I(n-1,i),I(n,j))=\begin{cases}
  \lambda p_{j-i}\,, &\ i\leq j\leq R\,;\\
  0\,, &\ j<i\,,\end{cases}\\[5pt]
  (n-1,i)\in {\sf X}_{n-1}^{\sptilde}\,,\enskip
  (n,j)\in {\sf X}^{\sptilde}_n\,,\ n=\overline{2,N}\,;
  \label{e2-sop}
  \end{multline}
  
  \vspace*{-12pt}
  
  \noindent
  \begin{multline}
  \mu_n(I(n+1,i), I(n,j))={}\\
  {}=\begin{cases}
  \displaystyle (n+1)\mu \fr{p_{i-j}-p_j^{(n)}}{p_i^{(n+1)}}\,, &\ j\leq i\leq R\,;\\
  0\,, &\ j>i\,,
  \end{cases}\\[5pt]
    (n+1,i)\in {\sf X}^{\sptilde}_{n+1}\,,\enskip
  (n,j)\in {\sf X}_n^{\sptilde}\,,\ n=\overline{1,N-1}\,;
  \label{e3-sop}
  \end{multline}
  
  \vspace*{-12pt}
  
  \noindent
  \begin{multline}
  \psi_N(I(N,i), I(N,j))={}\\
  {}=\begin{cases}
  -[N\mu +N\gamma]\,, &\ i=j\,;\\[5pt]
  \displaystyle N\gamma \sum\limits_{s=0}^i  \fr{p_s p_{i-s}^{(N-
1)}}{p_i^{(N)}}\,\fr{p_{j-i+s}} {\sum\nolimits_{k=0}^{R-i+s} p_k}\,, &\ i<j\,;\\[5pt]
  \displaystyle N\gamma \sum\limits^i_{s=i-j} \fr{p_s p_{i-s}^{(N-
1)}}{p_i^{(N)}}\,\fr{p_{j-i+s}}{\sum\nolimits_{k=0}^{R-i+s} p_k}\,, &\ i>j\,,
  \end{cases}\\
  (N,i),(N,j)\in {\sf X}_N^{\sptilde}\,.
  \label{e4-sop}
  \end{multline}
  
  %\columnbreak
  
  Стационарные вероятности 
  \begin{align}
  q_0 &= \lim\limits_{t\to\infty} {\sf P}\{\xi(t)=0\}\,;\label{e5-sop}
\\
 q_k(r) &= \lim\limits_{t\to\infty} {\sf P}\{\xi(t)=k, \ \delta(t)=r\}\,,\ (k,r)\in {\sf 
X}_k^{\sptilde}\!\!
  \label{e6-sop}
  \end{align}
являются единственным решением системы уравнений равновесия (СУР):

\noindent
\begin{equation*}
\lambda q_0\sum\limits^R_{j=0} p_j =\mu \sum\limits_{j:\ (1,j)\in {\sf 
X}_1^{\sptilde}} q_1(j)\,;
%\label{e7-sop}
\end{equation*}

\vspace*{-12pt}

\noindent
\begin{multline*}
\left( \lambda\sum\limits_{j=0}^{R-r} p_j +k\mu +k\gamma\right) q_k(r) 
={}\\
{}=\lambda \sum\limits_{j\geq 0\,,\ (k-1,r-j)\in {\sf X}^{\sptilde}_{k-1}} q_{k-1}(r-j) 
p_j+{}\\
{}+ (k+1) \mu \sum\limits_{j\geq0\,,\ (k+1, r+j)\in {\sf X}_{k+1}^{\sptilde}} \hspace*{-1mm}
q_{k+1}(r+j)\fr{p_j p_r^{(k)}}{p_{j+r}^{(k+1)}}+{}\\
{}+k\gamma \sum\limits_{j:\ (k,j)\in {\sf X}_k^{\sptilde}}\hspace*{-6pt} q_k(j) 
\sum\limits^j_{i=\max(0,j-r)} \fr{p_i p_{j-i}^{(k-1)}}{p_j^{(k)}}\, 
\fr{p_{r-j+i}} 
{\sum\nolimits_{s=0}^{R-j+i} p_s}\,,\\ 1\leq k\leq N-1\,,\ (k,r)\in {\sf 
X}_k^{\sptilde}\,;
%\label{e8-sop}
\end{multline*}

\vspace*{-12pt}

\noindent
\begin{multline*}
(N\mu+k\gamma)q_N(r)=\lambda\hspace*{-4pt}\sum\limits_{j\geq0,\ (N-1,r-j)\in {\sf 
X}^{\sptilde}_{N-1}} \hspace*{-15pt}q_{N-1}(r-j) p_j+{}\\
{}+N\gamma\hspace*{-4pt}\sum\limits_{j:\ (N,j)\in {\sf X}^{\sptilde}_N} \hspace*{-5pt}q_N(j) 
\hspace*{-4pt}\sum\limits^j_{i=\max(0,j-r)}\hspace*{-4pt} \fr{p_i p_{j-i}^{(N-1)}}{p_j^{(N)}}\, 
\fr{p_{r-j+i}}{\sum\nolimits_{s=0}^{R-j+i} p_s}\,,\\
(N,r)\in {\sf X}_N^{\sptilde}\,.
%\label{e9-sop}
\end{multline*}
  
  Стационарные вероятности~(\ref{e5-sop}) и~(\ref{e6-sop}) могут быть 
найдены численно методом $UL$-раз\-ло\-же\-ния СУР в~матричном виде 
$$
\mathbf{q}^{\mathrm{T}}\mathbf{A}=\mathbf{0}^{\mathrm{T}}\,;
\quad
\mathbf{q}^{\mathrm{T}}\cdot \mathbf{1}=1\,.
$$

 Обозначим подвекторы 
стационарных вероятностей $\mathbf{q}_0\hm=\{q_0\}$ и~$\mathbf{q}_k\hm= 
\{q_k(r)\}_{(k,r)\in {\sf X}_k^{\sptilde}}$ для всех $1\hm\leq k \leq N$, тогда СУР 
в~матричном виде с~учетом блоч\-но-трех\-диа\-го\-наль\-но\-го вида матрицы 
интенсивностей переходов~$\mathbf{A}$ примет вид:
  \begin{align}
  \!\!\!\mathbf{q}_0\boldsymbol{\Psi}_0-\mathbf{q}_1\mathbf{M}_0&=\mathbf{0}\,; 
\label{e10-sop}\\
   \!\!\! \mathbf{q}_i\boldsymbol{\Psi}_i -\mathbf{q}_{i+1}\mathbf{M}_i- 
\mathbf{q}_{i-1}\boldsymbol{\Lambda}_i&=\mathbf{0}\,,\ 1\leq i\leq N-1\,;
  \label{e11-sop}\\
    \!\!\!\mathbf{q}_N \boldsymbol{\Psi}_N -\mathbf{q}_{N-1} 
\boldsymbol{\Lambda}_N&= \mathbf{0}\,.\label{e12-sop}
  \end{align}
  
\section{Групповое перемещение пользователей}

  Теперь рассмотрим сценарий, при котором пользователи перемещаются 
относительно базовой  станции совместно. В~этом случае в~момент 
срабатывания сигнала изменяется объем занимаемых ресурсов каждой сессии 
и,~соответственно,\linebreak\vspace*{-12pt}

\pagebreak

\noindent
 изменяется объем занимаемых в~совокупности ресурсов 
всеми активными сессиями в~соте. Важно, что при выделении 
дополнительных ресурсов прерывания сессий не происходит.
  
  Функционирование СМО описывается пп.~1--3 из предыдущего 
раздела и~п.~4*, который сформулируем следующим образом:
  \begin{itemize}
  \item[4*.] В~систему поступает пуассоновский поток сигналов 
с~интенсивностью~$\gamma$, при поступлении которого заново 
разыгрывается объем занимаемых всеми заявками ресурсов. 
\end{itemize}
  
Поведение системы во времени описывает СП $X^*(t)\hm= \left( \xi)t), 
\delta(t)\right)$, где $\xi(t)$~--- число заявок в~сис\-те\-ме; $\delta(t)$~--- объем 
совокупно занятых ресурсов. Пространство состояний СП $X^*(t)$ 
идентично пространству состояний процесса~$\tilde{X}(t)$. Обозначим 
распределение стационарных вероятностей:
\begin{align*}
q_0^* &= \lim\limits_{t\to\infty} {\sf P}\{\xi(t)=0\}\,;\\
q_k(r) &= \lim\limits_{t\to\infty} {\sf P}\{ \xi(t)=k,\ \delta(t)=r\}\,,\enskip (k,r)\in {\sf 
X}_k^{\sptilde}\,.
\end{align*}
  
  Переходы между состояниями системы, соответствующие поступлению 
новых заявок и~завершению облуживания заявок системы, происходят 
аналогично переходам между состояниями модели с~независимым 
перемещением пользователей; различие возникает в~переходах, 
соответствующих поступлению сигналов. В~момент срабатывания сигнала 
система из состояния $(k,j)$ совершает переход в~состояние $(k,r)$ 
с~вероятностью $p_r^{(k)}/\sum\nolimits_{i=0}^R p_i^{(k)}$. Таким образом, 
СУР СП~$X^*(t)$ принимает вид:
  \begin{equation}
  \lambda q_0^*\sum\limits^R_{j=0} p_j =\mu \sum\limits_{j:\ (1,j)\in{\sf 
X}_1^{\sptilde} }\hspace*{-2mm} q_1^*(j)\,;
  \label{e13-sop}
  \end{equation}
  
  \vspace*{-12pt}
  
  \noindent
  \begin{multline}
\!\!\!\! \! \left( \!\lambda\!\sum\limits_{j=0}^{R-r}\hspace*{-0.5pt} 
p_j+k\mu+\gamma\!\right) \!
q_k^*(r)={}\\
{}=\lambda \hspace*{-13pt}\sum\limits_{j\geq 0:\ (k-1;r-j)\in 
{\sf X}^{\sptilde}_{k-1}}
\hspace*{-3pt} p_j 
q^*_{k-1}(r-j)+{}\\
  {}+ (k+1)\mu \sum\limits_{j\geq0:\ (k+1;r+j)\in {\sf X}^{\sptilde}_{k+1}} \fr{p_j 
p_r^{(k)}}{p_{j+r}^{(k+1)}}\, q^*_{k+1}(r+j)+{}\\
  {}+\gamma \sum\limits^R_{i=0} \fr{p_r^{(k)}} {\sum\nolimits^R_{i=0} 
p_i^{(k)}}\,q_k^*(j)\,,\\
  1\leq k\leq N-1\,,\ (k,r)\in {\sf X}_k^{\sptilde}\,;
  \label{e14-sop}
  \end{multline}
  
  \vspace*{-12pt}
  
  \noindent
  \begin{multline}
  (N\mu+\gamma) q_N^*(r) =\lambda \sum\limits_{j\geq0:\ (N-1; r-j)\in {\sf 
X}_{N-1}^{\sptilde}} \hspace*{-5mm}p_j q^*_{N-1}(r-j)+{}\\
\!\!\!  {}+ \gamma\hspace*{-2pt}\sum\limits_{j:\ (N;r)\in{\sf X}^{\sptilde}_N} \fr{p_r^{(N)}} 
{\sum\nolimits^R_{i=0} p_i^{(N)}}\,q_N^*(j)\,,\enskip
(N,r)\in {\sf X}^{\sptilde}_N.\!\!\!
  \label{e15-sop}
  \end{multline}
  

\noindent
\textbf{Теорема~1.}\ \textit{Стационарные вероятности СМО со случайными 
требованиями и~потоком сигналов, изменяющих суммарный объем 
занимаемых ресурсов, не зависят от интенсивности~$\gamma$ поступления 
сигналов и~имеют вид}:

\noindent
\begin{equation}
q_k^*(r) = q_0\fr{\rho^k}{k!}\,p_r^{(k)}\,;\quad
q_0^* = \left( \sum\limits_{k=0}^N \sum\limits_{r=0}^R 
\fr{\rho^k}{k!}\,p_r^{(k)}\right) ^{-1}\,.
\label{e16-sop}
%\label{e17-sop}
\end{equation}
  
  \noindent
  Д\,о\,к\,а\,з\,а\,т\,е\,л\,ь\,с\,т\,в\,о\ \ теоремы выполняется путем подстановки 
стационарных вероятностей~(\ref{e16-sop}) в~СУР~(\ref{e13-sop})--(\ref{e15-sop}).

\section{Численный пример}

  Согласно теореме~1 стационарные вероятности экспоненциальной СМО 
с~групповым перемещением пользователей как частный случай 
экспоненциальной СМО со случайными требованиями из~\cite{8-sop} не 
зависят от поступающего потока сигналов в~систему. Вероятностные 
характеристики системы, такие как вероятность блокировки~$B$ и~средний 
объем занятых ресурсов~$b$, в~этом случае могут быть найдены по 
формулам:

\noindent
  \begin{align*}
  B &= 1-G^{-1}(N,R)\sum\limits_{j=0}^R p_j G(N-1,R-j)\,;
  %\label{e18-sop}
  \\
  b &= R-G^{-1}(N,R) \sum\limits_{j=1}^R G(N,R-j)\,, %\label{e19-sop}
  \end{align*}
полученным по аналогии с~\cite{11-sop} с~по\-мощью рекуррентного 
алгоритма вычисления нормировочной константы 
$$
G(N,R)= 
\sum\limits_{k=0}^N \sum\limits_{r=0}^R \fr{\rho_k}{k!}\,p_r^{(k)}\,. 
$$

  Стационарные вероятности~(\ref{e5-sop})--(\ref{e6-sop}) СМО 
с~независимым перемещением пользователей могут быть найдены численно 
как решения системы матричных уравнений~(\ref{e10-sop})--(\ref{e12-sop}). 
Вероятностные характеристики системы в~этом случае определяются 
формулами: 

\noindent
  \begin{align*}
  B&= 1-\sum\limits_{k=0}^{N-1} \sum
  \limits_{r:\ (k,r)\in{\sf X}^{\sptilde}_k} q_k(r)\sum\limits_{j=0}^{R-r} 
p_j\,; %\label{e20-sop}
\\
  b&= \sum\limits^N_{k=0} \ \sum\limits_{r:\ (k,r)\in{\sf X}^{\sptilde}_k} rq_k(r)\,.
%  \label{e21-sop}
  \end{align*}
  

  
  В качестве примеров распределений требований к~ресурсу 
рассматривались биномиальное распределение $\mathrm{Binom}\,(r,p)$ 
и~геометрическое распределение $\mathrm{Geom}\,(p)$, как и~в~\cite{12-sop}. Для 
анализа зависимости  вероятностных характеристик СМО с~независимым\linebreak\vspace*{-12pt}

  \begin{figure*} %fig1
    \vspace*{1pt}
    \begin{minipage}[t]{80mm}
\begin{center}
\mbox{%
\epsfxsize=77.795mm
\epsfbox{sop-1.eps}
}
\end{center}
\vspace*{-9pt}
\Caption{Зависимость вероятности блокировки от интенсивности поступления сигнала: 
\textit{1}~--- $\mathrm{Binom}$; \textit{2}~--- $\mathrm{Geom}$}
\end{minipage}
\hfill
%\end{figure*}
%\begin{figure*} %fig2
 \vspace*{1pt}
     \begin{minipage}[t]{80mm}
\begin{center}
\mbox{%
\epsfxsize=77.795mm
\epsfbox{sop-2.eps}
}
\end{center}
\vspace*{-9pt}
\Caption{Зависимость среднего объема занятого ресурса от интенсивности поступления 
сигнала: \textit{1}~--- $\mathrm{Binom}$; \textit{2}~--- $\mathrm{Geom}$}
\end{minipage}
\end{figure*}

\pagebreak

\noindent
перемещением пользователей от интенсивности~$\gamma$ потока 
поступающих сигналов в~качестве примера рассматриваются: 
  \begin{enumerate}[(1)]
\item биномиальное распределение $\mathrm{Binom}\,(r,p)$ требований к~ресурсу 
с~параметрами $r\hm\geq0$ и~$0\hm\leq p\hm\leq 1$, где $p_i\hm= 
\begin{pmatrix} r\\ i\end{pmatrix} p^i(1-p)^{r-i}$~--- вероятность того, что 
заявка потребует~$i$~единиц ресурса, $0\hm\leq i\hm\leq r$, $p\hm= 
\overline{m}/r$; 
\item геометрическое распределение $\mathrm{Geom}(p)$ требований к~ресурсу 
c~параметром $0\hm\leq p\hm\leq 1$, где $p_i\hm= p^i(1-p)$~--- вероятность 
того, что заявка потребует~$i$~единиц ресурса, $1\hm\leq i\hm\leq r$, 
$p\hm=1/(\overline{m}+1)$.
\end{enumerate}
  
  Для вычисления элементов матриц~(\ref{e1-sop})--(\ref{e4-sop}), которые 
определяют необходимые компоненты\linebreak реше\-ния 
СУР~(\ref{e10-sop})--(\ref{e12-sop}), найдем все $k$-крат\-ные 
свертки~$p_r^{(k)}$ для каждого из предложенных распределений 
требований к~ресурсу. При условии биномиального распределения 
требований вероятность того, что~$k$~заявок системы 
занимают~$j$~единиц ресурса:
$$
p_j^{(k)}= \begin{pmatrix} kr\\ 
j\end{pmatrix} p^j(1-p)^{kr-j}\,; 
$$
для геометрического закона:
$$
p_j^{(k)}  = \begin{pmatrix} k+j-1\\ k\end{pmatrix} p^j(1-p)^k\,.
$$

 Целочисленный 
параметр~$r$~биномиального распреде\-ления требований к~ресурсу 
и~па\-ра\-мет\-ры~$p$~распределений были подобраны таким образом, чтобы 
математическое ожидание~$\overline{m}$ было одинаковым. Максимальное 
число единиц ресурса, требуемых одной заявке, при биномиальном 
рас-\linebreak\vspace*{-12pt}

\columnbreak

\noindent
пределении, таким образом, оказалось $r\hm=18$, а математическое 
ожидание для биномиального и~геометрического распределений 
$\overline{m}\hm=5{,}4$.
  
  Рассматривается пример соты, которая может обслуживать до~100~сессий 
одновременно, а~ресурс выделяется пользователям в~процентном 
соотношении от~100\% всего доступного соте ресурса, $N\hm=R\hm=100$. 
Средняя продолжительность сессии составляет $\mu\hm=1$~мин, а~среднее 
число запросов на установление сессии $\lambda\hm=16$, как оптимальное 
значение нагрузки. 
  
  На рис.~1 и~2 представлены результаты расчета вероятности блокировки 
системы и~среднего объема занимаемых ресурсов в~зависимости от 
поступления сигналов, моделирующих перемещение пользователей в~соте. 
  


  На рис.~1 можно видеть, что вероятность блокировки в~системе 
с~независимым перемещением пользователей растет с~ростом~$\gamma$, 
несмотря на то что поступление сигнала не может вызвать потери заявки. 
Наблюдаемый эффект связан с~тем, что с~повышением интенсивности 
поступления сигналов заявки интенсивнее используют доступный ресурс 
системы, как видно на рис.~2, и~в~результате в~системе остается меньше 
свободного ресурса для принятия новых заявок.

\vspace*{-9pt}

  \section{Заключение}
  
  В работе проведен анализ ресурсной СМО с~сигналами, при поступлении 
которых изменяется объем занимаемых заявками ресурсов. Модель 
позволяет проводить анализ показателей эффективности беспроводной сети, 
учитывая перемещение пользователей в~радиусе действия. Рассмотрены 
частные случаи независимого и~группового перемещений пользователей. 
В~частности, было доказано, что при групповом перемещении пользователей 
показатели качества сети не зависят от интенсивности изменения положения 
группы относительно базовой станции. 
  
  В дальнейшем планируется разработать эффективный вычислительный 
алгоритм расчета ве\-ро\-ят\-но\-ст\-но-вре\-мен\-н$\acute{\mbox{ы}}$х 
характеристик модели.
  
{\small\frenchspacing
 {%\baselineskip=10.8pt
 \addcontentsline{toc}{section}{References}
 \begin{thebibliography}{99}
\bibitem{1-sop}
\Au{Boban M., Barros~J., Tonguz~O.\,K.} Geometry-based vehicle-to-vehicle 
channel modeling for large-scale simulation~// IEEE T. Veh. 
Technol., 2014. Vol.~63. No.\,9. P.~4146--4164.
\bibitem{2-sop}
\Au{Khan M., Han~K.} An optimized network selection and \mbox{handover} triggering 
scheme for heterogeneous self-organized wireless networks~// Math. 
Probl. Eng., 2014. Vol.~2014. No.\,2. P.~173068-1--173068-11. {\sf 
https://www. hindawi.com/journals/mpe/2014/173068}.
\bibitem{3-sop}
\Au{Fowler S., H$\ddot{\mbox{a}}$ll~C.\,H., Yuan~D., Baravdish~D., 
Mellouk~A.} Analysis of vehicular wireless channel communication via 
queueing theory model~// IEEE Conference (International) on 
Communications.~--- Piscataway, NJ, USA: IEEE, 2014. P.~1736--1741.
\bibitem{4-sop}
\Au{Наумов В.\,А., Самуйлов~К.\,Е.} О~моделировании систем массового 
обслуживания с~множественными ресурсами~// Вестник РУДН. Сер. 
Математика. Информатика. Физика, 2014. №\,3. C.~60--64.
\bibitem{5-sop}
\Au{Naumov V., Samouylov~K., Sopin~E., Andreev~S.} Two approaches to 
analysis of queuing systems with limited resources~// Ultra Modern 
Telecommunications and Control Systems and Workshops  
Proceedings.~--- Piscataway, NJ, USA: IEEE, 2014. P.~485--488. 

\bibitem{7-sop}
\Au{Elshaer H., Boccardi~F., Dohler~M., Irmer~R.} Downlink and uplink 
decoupling: A~disruptive architectural design for 5G networks~// IEEE 
Global Communications Conference Proceedings.~--- 
Piscataway, NJ, USA: IEEE, 2014. P.~1798--1803.

\bibitem{6-sop} %7
\Au{Singh S., Zhang~X., Andrews~J.} Joint rate and SINR coverage analysis for 
decoupled uplink downlink biased cell associations in HetNets~// IEEE T. 
Wirel. Commun., 2015. Vol.~14. No.\,10. P.~5360--5373.
doi: 10.1109/TWC.2015.2437378.

\bibitem{8-sop}
\Au{Наумов В.\,А., Самуйлов~К.\,Е., Самуйлов~А.\,К.} О~суммарном 
объеме ресурсов, занимаемых обслуживаемыми заявками~// Автоматика 
и~телемеханика, 2016. №\,8. C.~125--132.
\bibitem{9-sop}
\Au{Bartolini N., Chlamtac~I.} Call admission control in wireless multimedia 
networks~// 13th IEEE Symposium (International) on Personal, Indoor and 
Mobile Radio Communications Proceedings.~--- Piscataway, NJ, USA: IEEE, 
2002. Vol.~1. P.~285--289. doi: 10.1109/PIMRC.2002.1046706.
\bibitem{10-sop}
\Au{Naumov V., Samouylov~K., Yarkina~N., Sopin~E., And\-re\-ev~S., 
Samuylov~A.} LTE performance analysis using queuing systems with finite 
resources and random requirements~// 7th Congress on Ultra Modern 
Telecommunications and Control Systems Proceedings.~--- 
Piscataway, NJ, USA: IEEE, 2015. P.~100--103.
\bibitem{11-sop}
\Au{Вихрова О.\,Г.} К~вычислению вероятностных характеристик СМО 
ограниченной емкости со случайными требованиями к~ресурсам~// 
Вестник РУДН. Сер. Математика. Информатика. Физика, 2017. Т.~25. 
№\,3. C.~203--210.
\bibitem{12-sop}
\Au{Вихрова О.\,Г., Самуйлов~К.\,Е., Сопин~Э.\,С., Шоргин~С.\,Я.} 
К~анализу показателей качества обслуживания в~современных 
беспроводных сетях~// Информатика и~её применения, 2015. Т.~9. Вып.~4. 
С.~48--55. 
 \end{thebibliography}

 }
 }

\end{multicols}

\vspace*{-3pt}

\hfill{\small\textit{Поступила в~редакцию 29.06.17}}

\vspace*{8pt}

%\newpage

%\vspace*{-24pt}

\hrule

\vspace*{2pt}

\hrule

%\vspace*{8pt}


\def\tit{QUEUING SYSTEMS WITH~RESOURCES AND~SIGNALS 
AND~THEIR~APPLICATION FOR~PERFORMANCE 
EVALUATION OF~WIRELESS NETWORKS}

\def\titkol{Queuing systems with~resources and~signals 
and~their~application for~performance 
evaluation of~wireless networks}

\def\aut{K.\,E.~Samouylov$^{1,2}$, E.\,S.~Sopin$^{1,2}$, 
and~S.\,Ya.~Shorgin$^2$}

\def\autkol{K.\,E.~Samouylov, E.\,S.~Sopin, 
and~S.\,Ya.~Shorgin}

\titel{\tit}{\aut}{\autkol}{\titkol}

\vspace*{-9pt}


\noindent
$^1$Peoples' Friendship University of Russia, 6~Miklukho-Maklaya Str., 
Moscow 117198, Russian Federation

\noindent
$^2$Institute of Informatics Problems, Federal Research Center ``Computer Sciences and Control'' 
of the Russian\linebreak
$\hphantom{^1}$Academy of Sciences, 44-2~Vavilov Str., Moscow 119333, Russian Federation



\def\leftfootline{\small{\textbf{\thepage}
\hfill INFORMATIKA I EE PRIMENENIYA~--- INFORMATICS AND
APPLICATIONS\ \ \ 2017\ \ \ volume~11\ \ \ issue\ 3}
}%
 \def\rightfootline{\small{INFORMATIKA I EE PRIMENENIYA~---
INFORMATICS AND APPLICATIONS\ \ \ 2017\ \ \ volume~11\ \ \ issue\ 3
\hfill \textbf{\thepage}}}

\vspace*{3pt}


 
\Abste{The paper considers a queuing system with limited resources, random requirements, and 
signals. Each customer occupies a server and a random amount of resources for the whole 
service duration. Besides, a Poisson flow of signals arrives to the queue.  Signal arrival triggers 
the resource reallocation process. The model can
describe functioning of a wireless network 
taking into account user movement during a session. Two cases are
considered: independent 
movement of users, when resources are reallocated independently for each session, and joint 
movement, when all resources are reallocated at once.}

\KWE{queuing system; random requirement; signals; limited resources; wireless network;  
LTE-advanced}



\DOI{10.14357/19922264170311} 

\vspace*{-12pt}

\Ack
\noindent
This work was financially supported by the Russian Science Foundation (grant  
No.\,16-11-10227).



%\vspace*{3pt}

  \begin{multicols}{2}

\renewcommand{\bibname}{\protect\rmfamily References}
%\renewcommand{\bibname}{\large\protect\rm References}

{\small\frenchspacing
 {%\baselineskip=10.8pt
 \addcontentsline{toc}{section}{References}
 \begin{thebibliography}{99}
\bibitem{1-sop-1}
\Aue{Boban, M., J.~Barros, and O.~Tonguz.} 2014. Geometry-based vehicle-to-vehicle 
channel modeling for large-scale simulation. \textit{IEEE T. Veh. Technol.} 
63(9):4146--4164.
\bibitem{2-sop-1}
\Aue{Khan, M., and K.~Han.} 2014. An optimized network selection and handover triggering 
scheme for heterogeneous self-organized wireless networks. \textit{Math. Probl. 
Eng.} 2014(2):173068-1--173068-11. 
Available at: {\sf 
https://www.hindawi.com/journals/mpe/2014/173068} (accessed Sptember~11, 2017).
\bibitem{3-sop-1}
\Aue{Fowler, S., S.~H$\ddot{\mbox{a}}$ll, D.~Yuan, D.~Baravdish, and A.~Mellouk.} 
2014. Analysis of vehicular wireless channel communication via queueing theory model. 
\textit{IEEE Conference (International) on Communications Proceedings}.  Piscataway, NJ:
IEEE. 1736--1741.
\bibitem{4-sop-1}
\Aue{Naumov, V., and K.~Samouylov.} 2014. O modelirovanii system massovogo 
obsluzivaniya s mnozhestnennymi resursami [On the modeling of queuing systems with 
multiple resources]. \textit{Vestnik RUDN. Ser. matematika, fizika, informatika} [RUDN~J. 
Mathematics, information science and physics ser.] 22(3):60--64.
\bibitem{5-sop-1}
\Aue{Naumov, V., K.~Samouylov, E.~Sopin, and S.~Andreev}. 2014. Two approaches to 
analysis of queuing systems with limited resources. \textit{Ultra Modern Telecommunications 
and Control Systems and Workshops Proceedings}. Piscataway, NJ: IEEE.  
485--488. 

\bibitem{7-sop-1}
\Aue{Elshaer, H., F.~Boccardi, M.~Dohler, and R.~Irmer.} 2014. Downlink and uplink 
decoupling: A~disruptive architectural design for 5G networks. \textit{Global 
Communications Conference Proceedings}. Piscataway, NJ: IEEE. 1798--1803.

\bibitem{6-sop-1}
\Aue{Singh, S., X.~Zhang, and J.~Andrews.} 2015. Joint rate and SINR coverage analysis for 
decoupled uplink-downlink biased cell associations in HetNets. \textit{IEEE T. 
Wirel. Commun.} 14(10):5360--5373. doi: 10.1109/TWC.2015.2437378.
\bibitem{8-sop-1}
\Aue{Naumov, V., K.~Samuilov, and A.~Samuilov.} 2016. On the total amount of 
resources occupied by serviced customers. \textit{Automat. Remote Control} 77(8):1419--1427. 
doi:10.1134/S0005117916080087. 
\bibitem{9-sop-1}
\Aue{Bartolini, N., and I.~Chlamtac.} 2002. Call admission control in wireless multimedia 
networks. \textit{13th IEEE Symposium (International) on Personal, Indoor and Mobile 
Radio Communications Proceedings}. 1:285--289. doi: 10.1109/PIMRC.2002.1046706.
\bibitem{10-sop-1}
\Aue{Naumov, V., K.~Samouylov, N.~Yarkina, E.~Sopin, S.~And\-re\-ev, and A.~Samuylov.} 
2015. LTE performance analysis using queuing systems with finite resources and random 
requirements. \textit{7th Congress (International) on Ultra Modern Telecommunications and 
Control Systems Proceedings}. Piscataway, NJ: IEEE. 100--103. 
\bibitem{11-sop-1}
\Aue{Vikhrova, О.} 2017. K~vychisleniyu veroyatnostnykh kha\-rak\-te\-ri\-stik sistemy massovogo 
obslyuzivaniya ogra\-ni\-chen\-noy emkosti so sluchaynymi trebovaniyamy k~re\-sur\-su [About 
probability characteristics evaluation in queuing system with limited resources and random 
requirements]. \textit{Vestnik RUDN. Ser. matematika, fizikam informatika} [RUDN~J. 
Mathematics, information science, and physics ser.] 25(3):203--210.
\bibitem{12-sop-1}
\Aue{Vikhrova, О., К.~Samouylov, E.~Sopin, and S.~Shorgin.} 2015. K~analizy pokazateley 
kachestva obsluzhivaniya v~sovremennykh besprovodnykh setyakh [On performance analysis 
of modern wireless networks]. \textit{Informatika i~ee Primeneniya~--- Inform. Appl.} 
9(4):48--55.  

\end{thebibliography}

 }
 }

\end{multicols}

\vspace*{-6pt}

\hfill{\small\textit{Received June 29, 2017}}

\vspace*{-10pt}


\Contr

\noindent
\textbf{Samouylov Konstantin E.} (b.\ 1955)~--- Doctor of Science in technology, professor; 
Head of Department, Peoples' Friendship University of Russia (RUDN University), 
6~Miklukho-Maklaya Str., Moscow 117198, Russian Federation; senior scientist, Institute of 
Informatics Problems, Federal Research Center ``Computer Science and Control'' of the Russian 
Academy of Sciences, 44-2~Vavilov Str., Moscow 119333, Russian Federation;  
\mbox{samouylov\_ke@rudn.university} 

\vspace*{3pt}

\noindent
\textbf{Sopin Eduard S.} (b.\ 1986)~--- Candidate of Science in physics and mathematics; associated 
professor, Peoples' Friendship University of Russia (RUDN University), 6~Miklukho-Maklaya 
Str., Moscow 117198, Russian Federation; senior scientist, Institute of Informatics Problems, 
Federal Research Center ``Computer Science and Control'' of the Russian Academy of Sciences, 
44-2~Vavilov Str., Moscow 119333, Russian Federation; \mbox{sopin\_es@rudn.university} 

\vspace*{3pt}

\noindent
\textbf{Shorgin Sergey Ya.} (b.\ 1952)~--- Doctor of Science in physics and mathematics, 
professor; Deputy Director, Federal Research Center ``Computer Science and Control'' of the 
Russian Academy of Sciences (FRC CSC RAS); principal scientist, Institute of Informatics 
Problems, FRC CSC RAS, 44-2~Vavilov Str., Moscow 119333, Russian Federation; 
\mbox{sshorgin@ipiran.ru}  
\label{end\stat}


\renewcommand{\bibname}{\protect\rm Литература}     %11
%\renewcommand{\figurename}{\protect\bf Figure}
\renewcommand{\tablename}{\protect\bf Table}

\def\stat{razum}


\def\tit{COMPARISON OF TWO ACTIVE QUEUE MANAGEMENT SCHEMES THROUGH THE~$M/D/1/N$ 
QUEUE}

\def\titkol{Comparison of two active queue management schemes through the $M/D/1/N$ 
queue}

\def\autkol{M.\,G.~Konovalov and R.\,V.~Razumchik}

\def\aut{M.\,G.~Konovalov$^1$ and R.\,V.~Razumchik$^2$}

\titel{\tit}{\aut}{\autkol}{\titkol}

%{\renewcommand{\thefootnote}{\fnsymbol{footnote}}
%\footnotetext[1] {The 
%research of Yuri Kabanov was done under partial financial support   of the grant 
%of  RSF No.\,14-49-00079.}}

\renewcommand{\thefootnote}{\arabic{footnote}}
\footnotetext[1]{Institute of Informatics Problems, Federal Research Center ``Computer Science and Control'' of the Russian Academy of Sciences,
44/2~Vavilov Str., Moscow 119333, Russian Federation, \mbox{mkonovalov@ipiran.ru}}
\footnotetext[2]{Institute of Informatics Problems, Federal Research Center 
``Computer Science and Control'' of the Russian Academy of Sciences,
44/2~Vavilov Str., Moscow 119333, Russian Federation; 
Peoples' Friendship University of Russia (RUDN University),
6~Miklukho-Maklaya Str., Moscow 117198, Russian 
Federation; 
\mbox{rrazumchik@ipiran.ru} %\mbox{razumchik\_rv@rudn.ru
}


\index{Konovalov M.\,G.}
\index{Razumchik R.\,V.}
\index{Коновалов М.\,Г.}
\index{Разумчик Р.\,В.}

\def\leftfootline{\small{\textbf{\thepage}
\hfill INFORMATIKA I EE PRIMENENIYA~--- INFORMATICS AND
APPLICATIONS\ \ \ 2018\ \ \ volume~12\ \ \ issue\ 4}
}%
 \def\rightfootline{\small{INFORMATIKA I EE PRIMENENIYA~---
INFORMATICS AND APPLICATIONS\ \ \ 2018\ \ \ volume~12\ \ \ issue\ 4
\hfill \textbf{\thepage}}}



\Abste{The paper focuses on giving the first in the literature numerical evidence
that the stationary performance characteristics of single-server queues
with the general renovation mechanism may be as good as of single-server queues
with the RED-type active queue management mechanisms
(AQM). Comparison is made in the queueing
theory context: the basic model is the $M/D/1/N$ queue. 
The characteristics reported are: the loss ratio, average system size, and 
average number of consecutive losses along 
with the standard deviations. Numerical results are based on the 
well-known facts and some new analytic results, presented in the paper.}

\KWE{queueing system; active queue management; RED; renovation}

\DOI{10.14357/19922264180402}


\vspace*{1pt}


\vskip 12pt plus 9pt minus 6pt

      \thispagestyle{myheadings}

      \begin{multicols}{2}

                  \label{st\stat}


\section{Introduction}

\noindent
A large number of AQM mechanisms have been developed
up to nowadays and quite a~lot of efforts have been devoted to the studies of
their efficiency. 
These mechanisms may be applicable in different contexts but historically, 
they are more often related to communication networks
in the context of mitigation of congestion and congestion avoidance.
This problem, as highlighted in the latest RFC~7567~\cite{RFC7567},
still does not have a~satisfying solution. 
An AQM mechanism is an advanced rejection discipline, 
when an arriving customer (packet, job, etc.) is lost randomly with a~probability 
that may depend on the (current, past, average, etc.) system state or performance.
The most popular class of AQM mechanisms seems to be the Random Early Detection (RED) and
its ramifications like GRED (Gentle RED), REM 
(Random Exponential Marking),
etc.\ (a~recent survey on the AQM can be found in~\cite{Adams}).
The goals of AQM are usually diverse and conflicting: 
prevent queues from growing too long, maintain high server (processor) utilization
and low variance of the queue size, ensure fairness among competing flows, 
and others. These are discussed in detail in~\cite{RFC7567} in the context
of communications network but most of the goals are applicable in other contexts as well
(buffer-bloat problems in data-center, etc.).

Besides simulation, analytic performance evaluation of systems with AQM is quite often
carried out in the queueing theory context (see,
for example,~\cite{Bonald,Chyd,Chyd2,oleg,hao,konnew} and references therein). 
Usually, the system with an AQM mechanism is modeled as a~queueing system or network
and then its performance characteristics are studied using known analytic techniques. 
Throughout the paper, we stay within the queueing theory context.

In the series of recent papers~\cite{Kreinin,Zaryadov2010,zarN1,zarN2,Zaryadov2009},
the authors have proposed the new type of AQM mechanism which they call 
\textit{renovation}. 
Roughly speaking, renovation implies that each customer, 
having received service, may remove some additional work from the system
(i.\,e., may renovate it). We will make this definition more precise in the next 
section 
but for now, note 
that queue management \textit{after service completions} is what makes the renovation
 different 
from the most known AQM schemes\footnote[3]{Indeed, renovation and most of the known AQM 
mechanisms
are conceptually different. One of the main goals of AQM mechanisms is to prevent 
queue from growing too large
leaving space for potential new arrivals.
In systems with renovation, the queue can become full (meaning that fewer customers are 
lost)
but after a~service completion, several customers may be removed from it. In this way, 
the content of the queue
can be preserved at a~certain average level but the loss pattern becomes intricate.}, 
in which the decisions are made \textit{upon arrivals}.  
To our best knowledge, there are no studies, 
which tell whether the performance of the systems with renovation is
better/same/worse than that of the same systems but with the implemented AQM mechanisms.
Thus, there is a~lack of bridge between available theoretical results for renovation and 
its practical perspective.

The scope of this paper is to give the first in the literature numerical evidence 
that the stationary performance of 
single-server queueing systems with the implemented renovation mechanism can
be as good as of 
the same single-server queues but the well-known packed dropping procedures like RED.
The emphasis is primarily on the reporting of this finding, complemented 
with some new insights into 
queueing systems with renovation. The relation to other 
AQM mechanisms like CoDel~\cite{RFC8289} is not discussed here. 
Moreover, in the numerical experiments presented here, 
we did not use any benchmarks to generate the traffic profiles 
but used the theoretical distributions instead.

The main stationary performance characteristics reported are: the loss ratio, 
the average number in the system
(average system size), and the average number of consecutive losses along with 
their standard deviations.
After introducing the renovation mechanism and the analytic setting,
in which  renovation mechanism is compared with RED, we give the new analytic results 
for computing system size moments and the loss ratio under the renovation mechanism.
The results presented in the numerical section are based on the analytic results. 
Monte-Carlo simulation is used only for the average (and standard deviation) 
number of consecutive losses in the system with renovation. 


\section{Settings and the Model}

\noindent
We follow the queueing theoretic approach and as the basic model, we use $M/D/1/N$ queue,
i.\,e., queue of finite capacity~$N$ fed at rate~$\lambda$ by a~Poisson flow of customers,
which are served on a~first-come-first-served basis  by a~single server with constant
service time $d>0$.
We assume that the system is in the steady state.
When an arriving customer sees that the queue is full,
it is lost. If no other type of losses occur in the system,
we say that the Tail Drop mechanism is implemented in it.

If an arriving customer is lost with probability~$d_n$
where~$n$ is the total number of customers 
it sees in the system on arrival, then we say that an AQM mechanism 
is implemented in the system. Various dropping functions can be obtained
by specifying the values of~$d_n$
(see, for example, RED dropping function in~\cite[Example 1]{Chyd}).
Important notice should be made here. In practice, RED-type mechanisms 
may use moving averages of the queue size instead of its instantaneous value. 
Thus, the way~$d_n$ introduced above is a~simplified way of thinking.
Yet, this trade-off is important because it allows to keep the mathematical 
models of RED-type AQM tractable.
Luckily, as noticed in~\cite[Section II.C]{Bonald}, such approximation may not
lead to significant bias, when the weight of the moving average scheme is small
(which is claimed to be the case sometimes in practice).

The renovation mechanism, which is implemented in a~system with Tail Drop,
works as follows. Define $N+1$ numbers, say, $q_i\ge 0$,
$0 \le i \le N$, satisfying \mbox{$\sum\nolimits_{i=0}^N q_i=1$}.
If upon service completion there are $i$, $1 \le i \le N$, customers
waiting in the queue, then the served customer leaves the system and
\begin{itemize}
\item with probability $q_0+Q_i$ nothing else happens, where 
$Q_i=q_i+q_{i+1}+\dots+ q_N$; and
\item with probability $q_j$, $0<j<i$, exactly $j$ customers
from the queue leave the system and those customers 
are chosen successively \textit{starting from the head of the queue}.
\end{itemize}
%\noindent
The served customer, which sees the empty queue, leaves the system.
Thus, after the renovation (if it happened), the system never becomes empty.
%what is appealing from the practical point of view. 

In the numerical section, we rank the systems with RED 
and renovation according to the stationary loss ratio, average system size,  
and average number of consecutive losses along with their standard deviations. 
The system with the Tail Drop is the standard $M/D/1/N$ queue, 
for which all these performance characteristics follow
from the classical results in queueing theory (see, for example,~\cite{Riordan1962}).
Analytic results for the systems of $M/G/1/N$ type with relatively 
arbitrary dropping functions are given in~\cite{Chyd}.
Yet, for the system with renovation, we need to derive these 
performance characteristics anew, since 
the available results in~\cite{Zaryadov2010,Zaryadov2009} 
are not valid for the renovation mechanism introduced above.
We briefly sketch the derivations in the next
section and omit most of the details
since they are based on the methodology, 
developed in~\cite{Zaryadov2010,Zaryadov2009},
and reviewed in~\cite{arxivRK}.

%Note that the above mentioned performance characteristics 
%do not depend on the order in which the customers
%are removed from the system; yet in the derivations we assume that the customers
%are chosen successively \textit{starting from the head of the queue}.

%The analytic results and parameters' values for 
%RED and REM are due to \cite{Chyd}.

\section{Performance Characteristics}

\noindent
Consider the $M/D/1/N$ queue with the renovation mechanism
introduced above. Since a~customer is served for constant service time
$d$, then for the cumulative distribution function $B(x)$ 
of its service time, one has: 
$$
B(x)=
\begin{cases}
0 & \mbox{if } x \le d\,;\\
1 & \mbox{if } x>d\,.
\end{cases}
$$
Let $N(t)$ be the total number of customers %\footnote{We assume that the system 
%starts empty, i.\,e., $N(0)=0$.} 
in the system at instant $t$ 
and $E(t)$ be the elapsed service time of the customer in server
(if there is one). 
In order to compute the stationary system size moments, 
one needs to know the stationary distribution:
\begin{equation*}
%\label{pn}
P_n=\lim\limits_{t \rightarrow \infty} \mathbf{P}\{ N(t)=n \},\enskip  0 \le n \le N+1\,.
\end{equation*} 
For the computation of the loss ratio,
due to the \mbox{PASTA} (Poisson Arrivals See Time Averages) 
property, it is sufficient to know
 the stationary probability densities
$p_n(x)=P'_n(x)$ where
\begin{multline*}
%\label{pnx}
P_n(x)=\lim\limits_{t \rightarrow \infty} \mathbf{P}\{ N(t)=n, E(t)<x \}, \\ 
1\le n \le N, \ x \in [0,d]\,.
\end{multline*}
Since we are dealing with the finite-capacity queue 
and work conserving service discipline, the
introduced stationary distributions exist.  
The probabilities~$P_n$ and 
the densities~$p_n(x)$ can be computed as follows. 
Let~$t_n$ denotes the $n$th service completion epoch 
and $N_n=N(t_n+0)$ denotes the total number of customers in the system. 
Clearly, $\{ N_n, \ n \ge 1\}$ is the finite-state Markov chain.
The entries of the transition probability matrix $\mathbb{P}=(p_{ij})$
of this chain have the form:
$$
p_{0j}=p_{1j}=
\begin{cases}
\beta_0, & \hspace*{-20mm}j=0;\\
\beta_j Q_j + \displaystyle\sum\limits_{k=j}^N \beta_k q_{k-j} +  B_N q_{N-j}, &\\
&\hspace*{-20mm} 1 \le j \le N-1\,;\\
(q_0 + q_N) B_{N-1}, & \hspace*{-20mm}j=N\,;
\end{cases}
$$
\begin{multline*}
\!\!p_{ij}=
\begin{cases}
0, & \hspace*{-38mm}j=0;\\
\sum\limits_{k=j-1}^{N-1} \beta_k q_{k-j+1} +  B_{N-1} q_{N-j}, & \\
& \hspace*{-38mm}1 \le j \le i-2\,;\\
\beta_{j-i+1} Q_j + \displaystyle\!\sum\limits_{k=j-1}^{N-1}\!\! \beta_k q_{k-j+1} + 
 B_{N-1} q_{N-j}, &\\
 &\hspace*{-38mm} i-1 \le j \le N-1;\\
(q_{0} +  q_{N})B_{N-i} , &\hspace*{-38mm} j=N\,,
\end{cases}
\\
 2 \le i \le N\,.
\end{multline*}
Here, $B_0=1-\beta_0$; $B_k=B_{k-1}-\beta_k$; and 
$\beta_k=[{(\lambda d)^k / k!}]e^{-\lambda d}$.
The matrix $\mathbb{P}$ does not have any special structure. 
So, the stationary distribution $\{P^+_n, \ 0 \le n\linebreak \le N\}$
may be found in a~straightforward manner by solving the system of linear algebraic 
equations 
$$
{\vec P}^+={\vec P}^+ \mathbb{P};\enskip 
{\vec P}^+ {\vec 1} =1
$$ 
where ${\vec P}^+= (P^+_0,\dots,P^+_N)$ and $\vec 1$ is the vector of ones. 
{\looseness=1

}

Once the probabilities $P^+_n$ are found,
the stationary distribution \mbox{$\{P_n, \ 0 \le n \le N+1\}$} 
is computed from the relation\footnote{This follows
from the well-known results for the Markov regenerative processes (see, for 
example,~\cite[Theorem 9.19]{kulk}).} 
$$
P_n=\sum\limits_{i=0}^N P^+_i \fr{f_{in}}{f^*}
$$
where $f_{in}$ is the average time during which there were $n$ customers in the system
provided that the system started with~$i$ customers in it; 
and~$f^*$ is the mean time between transitions of the embedded
Markov chain $\{ N_n, \ n \ge 1\}$.
{\looseness=1

}


Finally,the stationary probability densities $p_n(x)\linebreak =P'_n(x)$
can be computed using the fact that the relation for~$p_n(x)$ 
coincides with the relation for $p_n(x)$ in 
the standard $M/D/1/N$ queue.
Thus, $p_n(x)$ are given by (see, for example,~\cite[p.~72]{Riordan1962})

\noindent
\begin{multline}
\label{eq3}
p_n(x)=e^{-\lambda x} \left (1-B(x) \right ) 
\sum\limits_{k=0}^{n-1} p_{n-k}(0) 
\fr{(\lambda x)^k}{k!}\,, \\
1 \le n \le N\,,  \ x \in [0,d]\,.
\end{multline}
Even though~(\ref{eq3}) holds,
due to the presence of renovation, the boundary conditions $p_{n}(0)$ 
for the considered queue do not coincide 
with boundary conditions $p_{n}(0)$ for the standard $M/D/1/N$ queue.
By integration~(\ref{eq3}) from~0 to~$d$, one gets 
the following relation between~$p_{n}(0)$ and $P_n=\int\nolimits_0^d p_n(x) dx$: 
\begin{multline}
\label{eq3nn}
p_n(0)
= \fr{1}{B_0} \left (\lambda P_n- \sum\limits_{k=1}^{n-1} B_k p_{n-k}(0)
\right )\,, \\ 
1 \le n \le N\,.
\end{multline}
Since the stationary distribution \mbox{$\{P_n, \ 0 \le n \le N+1\}$} 
is already known, the values of $p_n(0)$ are computed recursively from~(\ref{eq3nn}).
The closed-form expressions for
 the average and the standard deviation of the system size are, in the most cases,
 not available and thus, they can be computed,
respectively, by $\sum\nolimits_{n=0}^{N+1} nP_n$ and  
$\sqrt{\sum\nolimits_{n=0}^{N+1} n^2P_n-(\sum_{n=0}^{N+1} nP_n)^2}$.

The computation of the loss ratio~$\pi$, i.\,e., the probability that the arriving customer is lost, 
is more involved. This is due to the fact that the accepted customer
may be lost either after the first service completion or the second, etc.
and the chance to be lost varies, depending on the number of
new customers that have arrived between successive service completions.
 
Let us introduce two quantities:
\begin{enumerate}[(1)]
\item $\gamma_{ij}$, $1 \le i \le N$, $j \ge 0$,~--- probability that the arriving customer
finds~$i$~customers in the system and until the next
service completion, exactly $j$ new customers arrive 
at the system; and
\item
$r_{ij}$, $0\le j \le N-1$, $0 \le i \le N-j-1$,~--- probability that the 
tagged customer waiting in the queue
\textit{will not} be served (i.\,e., will be lost), if there are~$j$~customers
 in front of it in the queue (excluding the one in server)
and~$i$ behind.
\end{enumerate}

Given that $\gamma_{ij}$ and $r_{ij}$ are known, the loss ratio~$\pi$  
can be computed as
\begin{multline*}
\pi =
P_{N+1}
 + 
\sum\limits_{i=1}^{N} \left [
\sum\limits_{j=0}^{N-i}
\gamma_{ij} \left ( \sum\limits_{k=0}^{i-2} q_k r_{j,i-2-k}
 +{}\right.\right.\\
\left. {}+ \sum\limits_{k=i}^{i+j-1} q_k  + Q_{j+i} r_{j,i-2} \right )
 + {}
\end{multline*}

\noindent
\begin{multline*}
{}+ \sum\limits_{j=N-i+1}^{\infty} \gamma_{ij} \left (
\sum\limits_{k=0}^{i-2} q_k r_{N-i,i-2-k}  + {}\right.\\
\left.\left.{}+\sum\limits_{k=i}^{N-1}
q_k  + Q_{N} r_{N-i,i-2} \right )
\right ]\,.
%\label{ploss2}
\end{multline*}

Due to the PASTA property of Poisson arrivals,
the expression for $\gamma_{ij}$ follows 
from the law of total probability:
\begin{multline*}
\gamma_{ij}
= \int\limits_0^d p_{i}(x) \fr{(\lambda (d- x))^j }{j!}\, e^{-\lambda (d-x)}\, dx\,,
\\
 1 \le i \le N\,, \ j \ge 0\,.
\end{multline*}
Again, by applying the law of total probability,
one gets the relations for the recursive computation of~$r_{ij}$, 
$0\le j \le N-1$, $0 \le i \le N-j-1$:
\begin{align*}
r_{i0}&= \sum\limits_{m=0}^{N-i-1}
\beta_m 
\sum\limits_{k=1}^{m+i} q_k +
B_{N-i-1} \sum\limits_{k=1}^{N-1} q_k\,;\\
r_{ij}&= \sum\limits_{m=i}^{N-1-j} \beta_{m-i} \left (
\vphantom{\sum\limits_{k=j+1}^{m+j} q_k + Q_{j+m+1} r_{m,j-1}}
\sum\limits_{k=0}^{j-1} q_k r_{m,j-1-k} +{}\right.\\
&\left.{}+
\sum\limits_{k=j+1}^{m+j} q_k + Q_{j+m+1} r_{m,j-1}
\right ) +{}
\\
&{}+ B_{N-j-i-1}
\left ( \vphantom{\sum\limits_{k=j+1}^{m+j} q_k + Q_{j+m+1} r_{m,j-1}}
\sum\limits_{k=0}^{j-1} q_k r_{N-j-1,j-1-k} +{}\right.\\
&\hspace*{16mm}\left.{}+\sum\limits_{k=j+1}^{N-1} q_k +Q_{N} r_{N-j-1,j-1}
\right ).
\end{align*}
The expressions above can be further simplified\footnote{There
are no principal difficulties in generalizing
the results for the $\mathrm{BMAP}/G/1/N$ queue.
Yet, this would obscure the goal of the paper and thus, 
we remain with the simple model.} by computing 
the integrals explicitly, but we do not dwell on it here.
For small and moderate values of~$d$, $N$, and~$\lambda$,
they can be directly used for numerical implementation.
In the numerical section, precisely these expressions are used to calculate
the loss ratio. The expressions for the average and the standard
deviation of consecutive losses are much harder to derive
and we leave it for a~separate study. The values of these 
two parameters were taken from the Monte-Carlo simulation. 

\section{Numerical Example}

\noindent
As the reference point, we have chosen the numerical results in~\cite{Chyd}
which are based on the analytic expressions and which show the 
performance characteristics 
of the $M/D/1/N$ queue with four different AQM mechanisms. 
Since RED scheme is one of the best among the four,
our goal here is to rank the $M/D/1/N$ queue with RED from~\cite{Chyd}
and the $M/D/1/N$ queue with renovation. Comparison is made 
according to the stationary loss ratio, average system size,  
and average number of consecutive losses along with 
their standard deviations.

The initial conditions are: 
the maximum queue size is $N=9$ and the service time is $d=1$. 
Thus, the offered load is $\rho=\lambda d$. 
The RED dropping function is given by (see~\cite[Eq. (59)]{Chyd}):
\begin{equation}
\label{df}
d_n=
\begin{cases}
0\,, & n\le 3\,;\\
0.11917n - 0.35752\,, & 4 \le n \le 9\,;\\
1\,, & n=10.
\end{cases}
\end{equation}
The performance of the $M/D/1/N$ queue with this RED dropping function 
is given in~\cite[Tables 1, 3, and~4]{Chyd}.
In order to find out whether there exists a~renovation 
mechanism under which the $M/D/1/N$ queue can perform at least as good as
under RED, one needs to perform exhaustive search over
the possible values of the renovation parameters $\{q_i, \ 0 \le i \le N\}$.
Since we are unaware of any analytic way of choosing these values,
adaptive search algorithms for partially observable
Markov decision processes from~\cite{kono1} were used instead.
Meta-heuristics (like particle swarm optimization), which are also applicable here,
were not used.

In Tables~1 and~2, one can see the numerical results for the four different
cases of the offered load\footnote{For the sake of reproducibility 
of the results 
presented in the paper, we also report the obtained values of the renovation 
probabilities: for $\rho=0.5$,
${\vec q}=(0.2544,0.0037,0.0053,0.0065,0.0122,0.0352,0.1108,0.1898,0.2129,0.1691)$;
for $\rho=1$, $q_0=0.0551$, $q_6=0.051$, $q_7=0.7166$, $q_8=0.0917$, and
$q_9=0.0856$;
for $\rho=2$, $q_0=0.1078$, $q_1=0.6374$, $q_4=0.0042$, $q_6=0.0084$, and
$q_9=0.2422$; and
for $\rho=3$, $q_1=0.4608$, and $q_2=0.5392$.}~$\rho$: $\rho=0.5$~--- underloaded system;
$\rho=1$~--- critically loaded system;
and $\rho=2$ and~3~--- overloaded system. The values displayed are the 
values from the numerical experiments rounded to three decimal digits.

%\noindent and in each case compute the stationary 
%loss ratio, average system content  
%and average number of consecutive losses along with
%their standard deviations. 


\begin{table*}\small
\begin{center}
\parbox{400pt}{\Caption{Performance characteristics of the $M/D/1/9$ system with the RED 
mechanism~(\ref{df}) and the $M/D/1/9$ the renovation mechanism (ren.)\
under the offered load $\rho=0.5$ and $\rho=1$}
}
%\label{my-label}
\vspace*{2ex}

\tabcolsep=8pt
\begin{tabular}{cc|c|c|c||c|c|c|}
\cline{3-8}
                                        &  & \multicolumn{3}{c||}{$\rho=0.5$} & \multicolumn{3}{c|}{$\rho=1$} \\ \cline{3-8} 
                                        &  &   Tail Drop      & RED     &    ren.   &    Tail Drop   &      RED &    ren.   \\ \hline
\multicolumn{2}{|c|}{loss ratio}    &   0    &   0.002        &   0.002  &    0.051   &    0.091        &   0.104    \\ \hline
\multicolumn{1}{|c|}{\textbf{system}} & average &   0.750    &   0.741        &    0.744    &    5.064   &      3.000       &   2.999  \\ \cline{2-8} 
\multicolumn{1}{|c|}{\textbf{size}} & standard deviation &   0.946    &    0.920      &    0.935      &    2.897   &   1.887       &   2.091      \\ \hline
\multicolumn{1}{|c|}{\textbf{consecutive}} &average  &  1.152     &  1.053        &   1.800     &    1.359   &    1.239       &  6.876         \\ \cline{2-8} 
\multicolumn{1}{|c|}{\textbf{losses}} & standard deviation &  0.403     &   0.240       &    1.260     &    0.647   &       0.561        &   0.852     \\ \hline
\end{tabular}
\end{center}
\vspace*{-6pt}
\end{table*}




\begin{table*}\small %tabl2
\begin{center}
\parbox{400pt}{\Caption{Performance characteristics of the $M/D/1/9$ system with the RED 
mechanism~(\ref{df}) and the $M/D/1/9$ the renovation mechanism (ren.)\
under the offered load $\rho=2$ and $\rho=3$}
}
%\label{my-label}
\vspace*{2ex}

\tabcolsep=8pt
\begin{tabular}{cc|c|c|c||c|c|c|}
\cline{3-8}
                                        &  & \multicolumn{3}{c||}{$\rho=2$} & \multicolumn{3}{c|}{$\rho=3$} \\ \cline{3-8} 
                                        &  &   Tail Drop      & RED     &    ren.   &    Tail Drop   &      RED &    ren.   \\ \hline
\multicolumn{2}{|c|}{loss ratio}    &   0.500    &    0.500        &   0.502     &    0.667   &    0.667       &     0.667      \\ \hline
\multicolumn{1}{|c|}{\textbf{system}} & average &   9.372    &    7.146       &   7.142      &  9.646     &      8.390        &    7.114  \\ \cline{2-8} 
\multicolumn{1}{|c|}{\textbf{size}} & standard deviation &   0.744    &     1.436       &  2.387      &    0.523   &     1.090        &    2.246  \\ \hline
\multicolumn{1}{|c|}{\textbf{consecutive}} &average  &  1.884     &   1.996         &  1.592       &     2.542  &     2.876        &   2.141    \\ \cline{2-8} 
\multicolumn{1}{|c|}{\textbf{losses}} & standard deviation &   1.069    &    1.366        &  1.100      &    1.454   &       2.064       &  1.236        \\ \hline
\end{tabular}
\end{center}
\end{table*}


Data is the tables show that with respect to the loss ratio,
renovation can perform as good as RED in the wide range of the offered load~$\rho$.
The only exception is the case $\rho=1$: here, renovation can keep
only the average system size at the same level as RED; other four 
performance characteristics are worse. 

If we fix the loss ratio, then the renovation mechanism
can guarantee at least the same value of the average system size as guaranteed by RED.
It is worth noticing that as the offered load increases, the average system size 
under the renovation mechanism becomes smaller than the average system size under RED.
Yet, renovation keeps the queue less stable than RED:
the standard deviation of system size is always smaller for RED.

If we fix the loss ratio and the average system size,
then the renovation mechanism spreads out the losses 
worse than RED when the system is underloaded and 
better than RED when it is overloaded.
This can be seen from the values of the averages and
standard deviations in the last two rows of Tables~1 and~2.

\vspace*{-6pt}

\section{Concluding Remarks}

\noindent
Even though the idea behind the renovation-type AQMs is completely 
different from the idea behind RED-type AQMs, 
renovation-type AQMs may allow one to achieve in some cases at least 
the same system performance level as guaranteed by RED-type AQMs. 
Although the comparison presented here is based only on a~single RED dropping 
function~(\ref{df}), 
our numerical experiments show that the results remain 
qualitatively the same for RED-type AQMs with other dropping functions.
Being defined by~$N$~parameters, the renovation mechanism is very flexible
and this constitutes its strength and weakness.
By varying the values of the renovation probabilities~$q_i$,
it is possible to carry out conditional optimization,
but good searching procedures are required here.

Implementation of the renovation as a~packet dropping mechanism
requires \textit{a~priori} tuning and/or operational configuration of its parameters.
Thus, whether it is appropriate to use renovation as a~packet dropping mechanism 
or not in practice heavily depends on the use case.
Although the tuning of the renovation parameters~$q_i$ can be made on the 
fly during operation, with respect to the recommendations of the RFC~7567~\cite{RFC7567},
renovation mechanism is not the proper choice for the network congestion control 
unless simple recommendations on how to set up the renovation parameters are given.
We believe this can be done based on more deep and insightful numerical experiments.

There remain a~large number of unresolved issues 
related to the renovation mechanism 
(e.\,g., can renovation ensure fairness among competing flows?
may the average queue size instead of its instantaneous value
increase the efficiency of the renovation mechanism?)
and this motivates its further analysis. 
Furthermore, evaluation of the renovation mechanism with parameters 
adapted to a~realistic router/switch use case
and/or evaluation which includes TCP feedback loops 
of several flows remains an open issue as well.

\vspace*{-6pt}


\Ack
  \noindent
   The reported study was partially funded by the Russian Foundation for 
Basic Research according to the research project No.\,18-07-00692.

The authors would like to thank the anonymous referees for their valuable comments 
which helped to improve the paper.
  
 \renewcommand{\bibname}{\protect\rmfamily References}

%\vspace*{-6pt}

\vspace*{-6pt}

{\small\frenchspacing
{\baselineskip=10.65pt
\begin{thebibliography}{99}
\bibitem{RFC7567} %1
\Aue{Baker, F., and G.~Fairhurst.} 2015.
IETF recommendations regarding active queue management.
Available at: {\sf https://tools.ietf.org/html/7567} (accessed October~4, 2018).



\bibitem{Adams} %2
\Aue{Adams, R.} 2013. 
Active queue management: A~survey. \textit{IEEE Commun. Surv.
Tut.} 15(3):1425--1476.

\bibitem{Bonald} %3
\Aue{Bonald, T., M.~May, and J.\,C.~Bolot.}
2000. Analytic evaluation of RED performance. 
\textit{IEEE Conference on Computer Communications Proceedings}
3:1415--1424.

\bibitem{hao} %4
\Aue{Hao, W., and Y.~Wei.} 2005.
An extended $GI^X/M/1/N$ queueing
model for evaluating the performance of AQM algorithms
with aggregate traffic.
\textit{Networking and mobile computing}.
Eds.\ Xicheng Lu and Wei Zhao.
{Lecture notes in computer science ser.} Springer. 3619:395--404.

\bibitem{Chyd} %5
\Aue{Chydzi$\acute{\mbox{n}}$nski, A., and L.~Chr$\acute{\mbox{o}}$st.} 
2011. Analysis of AQM queues with queue size based packet
dropping. \textit{Int. J.~Appl. Math. Comp.} 21(3):567--577.

\bibitem{Chyd2} %6
\Aue{Chydzi$\acute{\mbox{n}}$nski, A., and P.~Mrozowski.} 2016. 
Queues with dropping functions and general arrival
processes. \textit{PLoS ONE} 11(3):e0150702. Available at: 
{\sf https://\linebreak journals.plos.org/plosone/article?id=10.1371/journal.\linebreak pone.0150702} 
(accessed October~4, 2018).

\bibitem{oleg} %7
\Aue{Tikhonenko, O., and W.~Kempa.} 2016. Performance evaluation of 
an $M/G/n$-type queue
with bounded capacity and packet dropping. \textit{Int. J.~Appl.
Math. Comp.} 26(4):841--854.




\bibitem{konnew} %8
\Aue{Konovalov, M.\,G., and R.\,V.~Razumchik.} 2018. 
Numerical analysis of improved access restriction algorithms in a~$GI/G/1/N$
system. \textit{J.~Commun. Technol. El.} 63(6):616--625. 


\bibitem{Kreinin} %9
\Aue{Kreinin, A.\,Y.} 1997.
Queueing systems with renovation. 
\textit{J.~Appl. Math. Stochastic Analysis} 10(4):431--441.


\bibitem{zarN2} %10
\Aue{Zaryadov, I.\,S.} 2009. Queueing systems with general renovation.
\textit{Conference (International)
on Ultra Modern Telecommunications Proceedings}. 1--4.

\bibitem{Zaryadov2009} %11
\Aue{Zaryadov, I.\,S., and A.\,V.~Pechinkin.} 2009.
Stationary time characteristics of the ${GI/M/n/\infty}$
system with some variants of the generalized renovation discipline. \textit{Automat.
Rem. Contr.} 70(12):2085--2097.

\bibitem{Zaryadov2010} %12
\Aue{Zaryadov, I.\,S.}
2010. The ${GI/M/n/\infty}$ queuing system with generalized renovation.
\textit{Automat. Rem. Contr.} 71(4):663--671.

\bibitem{zarN1} %13
\Aue{Korolkova, A., and I.~Zaryadov.} 2010.
The mathematical model of the traffic transfer process with a~rate adjustable by {RED}.
\textit{Conference (International) on Ultra Modern Telecommunications Proceedings}. 
1046--1050.

\bibitem{RFC8289} %14
\Aue{Nichols, K., V.~Jacobson, A.~McGregor, and J.~Iyengar.} 2018.
Controlled delay active queue management.
Available at: {\sf https://datatracker.ietf.org/doc/rfc8289} (accessed October~4, 2018).

\bibitem{Riordan1962} %15
\Aue{Riordan, J.} 1962. \textit{Stochastic service systems}. 
SIAM ser. in applied mathematics. New York, NY: Wiley. 139~p.

\bibitem{arxivRK}  %16
\Aue{Konovalov, M.,  and R.~Razumchik.} 2017.
Queueing systems with renovation vs.\ queues with RED. Supplementary material. 
\textit{ArXiv e-prints}. Available at: {\sf https://arxiv.\linebreak org/abs/1709.01477/}
(accessed October~4, 2018).

\bibitem{kulk} %17
\Aue{Kulkarni, V.\,G.} 2016. \textit{Modeling and analysis of stochastic systems}. 
3rd ed. Chapman \&~Hall/CRC texts in statistical science ser.
Chapman \&~Hall/CRC. 606~p.

\bibitem{kono1} %18
\Aue{Konovalov, M.\,G.} 2007.
\textit{Metody adaptivnoy obrabotki informatsii i~ikh prilozheniya}
[Methods of adaptive information processing and their applications]. 
Moscow: Institute of Informatics Problems of RAS. 212~p.
\end{thebibliography} } }

\end{multicols}

\vspace*{-6pt}

\hfill{\small\textit{Received October 9, 2018}}

\vspace*{-18pt}
  

 \Contr

\noindent
\textbf{Konovalov Mikhail G.} (b.\ 1950)~--- 
Doctor of Science in technology, principal scientist, Institute of Informatics
Problems, Federal Research Center ``Computer Science and Control'' 
of the Russian Academy of Sciences, 44-2~Vavilov Str., Moscow 119333, 
Russian Federation; \mbox{mkonovalov@ipiran.ru}

\vspace*{3pt}


\noindent
\textbf{Razumchik Rostislav V.} (b.\ 1984)~--- 
Candidate of Science (PhD) in physics and mathematics, leading scientist,
Institute of Informatics Problems, Federal Research Center ``Computer 
Science and Control'' of the Russian
Academy of Sciences, 44-2~Vavilov Str., Moscow 119333, Russian Federation; 
associate professor, Peoples'
Friendship University of Russia (RUDN University), 
6~Miklukho-Maklaya Str., Moscow 117198, Russian
Federation; \mbox{rrazumchik@ipiran.ru} %; \mbox{razumchik\_rv@rudn.ru}

\vspace*{6pt}

\hrule

\vspace*{2pt}

\hrule

\vspace*{-2pt}

%\newpage

%\vspace*{-24pt}

\def\tit{СРАВНЕНИЕ ДВУХ МЕХАНИЗМОВ АКТИВНОГО УПРАВЛЕНИЯ ОЧЕРЕДЬЮ В~СИСТЕМЕ $M/D/1/N$$^*$}

\def\titkol{Сравнение двух механизмов активного управления очередью в~системе $M/D/1/N$}

\def\aut{М.\,Г.~Коновалов$^1$, Р.\,В.~Разумчик$^{1,2}$}

\def\autkol{М.\,Г.~Коновалов, Р.\,В.~Разумчик}

{\renewcommand{\thefootnote}{\fnsymbol{footnote}} \footnotetext[1]
{Исследование выполнено при частичной финансовой поддержке РФФИ (проект 18-07-00692).}}



\titel{\tit}{\aut}{\autkol}{\titkol}

\vspace*{-11pt}

\noindent
$^1$Институт проблем информатики Федерального исследовательского 
центра <<Информатика и управление>>\linebreak
$\hphantom{^1}$Российской академии наук

\noindent
$^2$Российский университет дружбы народов 

\vspace*{5pt}

\def\leftfootline{\small{\textbf{\thepage}
\hfill ИНФОРМАТИКА И ЕЁ ПРИМЕНЕНИЯ\ \ \ том\ 12\ \ \ выпуск\ 4\ \ \ 2018}
}%
 \def\rightfootline{\small{ИНФОРМАТИКА И ЕЁ ПРИМЕНЕНИЯ\ \ \ том\ 12\ \ \ выпуск\ 4\ \ \ 2018
\hfill \textbf{\thepage}}}

\vspace*{-3pt}


\Abst{Представлены некоторые результаты численных экспериментов, подтверждающие
следующее обстоятельство: параметры механизма обобщенного обновления
могут быть подобраны таким образом,\linebreak\vspace*{-12pt}}

\Abstend{что уровень производительности
однолинейных сис\-тем массового обслуживания с обобщенным обновлением
может быть не ниже уровня производительности
систем с RED-по\-доб\-ны\-ми механизмами активного управ\-ле\-ния очередями.
Механизмы сравниваются на примере сис\-те\-мы $M/D/1/N$
по стационарным значениям сле\-ду\-ющих характеристик:
вероятность потери заявки, среднее число заявок в сис\-те\-ме,
среднее чис\-ло последовательных потерь заявок 
и~их средние квадратические отклонения.
Расчеты основаны на известных фактах,
а~также на ряде новых аналитических результатов для систем
с~обобщенным обновлением, полученных в данной работе.}

\KW{система массового обслуживания; 
алгоритмы активного управления очередями; обобщенное обновление}

\DOI{10.14357/19922264180402}



%\vspace*{-3pt}


 \begin{multicols}{2}

\renewcommand{\bibname}{\protect\rmfamily Литература}
%\renewcommand{\bibname}{\large\protect\rm References}

{\small\frenchspacing
{%\baselineskip=10.8pt
\begin{thebibliography}{99}
%\vspace*{-3pt}

\bibitem{RFC7567-1} %1
\Au{Baker F., Fairhurst~G.}
IETF recommendations regarding active queue management, 2015.
{\sf https://tools.\linebreak ietf.org/html/7567}.



\bibitem{Adams-1} %2
\Au{Adams R.}
Active queue management: A~survey~// 
{IEEE Commun. Surv. Tut.}, 2013. Vol.~15. No.\,3. P.~1425--1476.

\bibitem{Bonald-1} %3
\Au{Bonald T., May M., Bolot~J.\,C.} Analytic evaluation of RED performance~//
{IEEE Conference on Computer Communications Proceedings}, 2000. 
Vol.~3. P.~1415--1424.

\bibitem{hao-1} %4
\Au{Hao W., Wei~Y.}
An extended $GI^X/M/1/N$ queueing
model for evaluating the performance of AQM algorithms
with aggregate traffic~// Networking and mobile computing~/
Eds. Xicheng Lu and Wei Zhao.~---
Lecture notes in computer science ser.~--- Springer, 2005. Vol.~3619. P.~395--404.

\bibitem{Chyd-1} %5
\Au{Chydzi$\acute{\mbox{n}}$ski A., Chr$\acute{\mbox{o}}$st~L.} 
Analysis of AQM queues with queue size based packet
dropping~// Int. J.~Appl. Math. Comp., 2011. Vol.~21. No.\,3. P.~567--577.

\bibitem{Chyd2-1} %6
\Au{Chydzi$\acute{\mbox{n}}$ski A.,  Mrozowski~P.}
 Queues with dropping functions and general arrival
processes~// PLoS ONE, 2016. Vol.~11. No.\,3. 
{\sf https://journals.plos.org/plosone/\linebreak article?id=10.1371/journal.pone.0150702}.

\bibitem{oleg-1} %7
\Au{Tikhonenko O., Kempa~W.} Performance evaluation of an $M/G/n$-type queue
with bounded capacity and packet dropping~// {Int. J.~Appl.
Math. Comp.}, 2016. Vol.~26. No.~4. P.~841--854.



\bibitem{konnew-1} %8
\Au{Konovalov M.\,G., Razumchik~R.\,V.}
Numerical analysis of improved access restriction algorithms in a~$GI/G/1/N$
system // {J.~Commun. Technol. El.}, 2018. Vol.~63. No.\,6. P.~616--625.

\bibitem{Kreinin-1} %9
\Au{Kreinin A.\,Y.}
Queueing systems with renovation //
{J.~Appl. Math. Stochastic Analysis}, 1997. Vol.~10. No.~4. P.~431--441.

\bibitem{zarN2-1} %10
\Au{Zaryadov~I.\,S.} Queueing systems with general renovation~//
{Conference (International) on Ultra Modern Telecommunications Proceedings}, 2009.
P.~1--4.

\bibitem{Zaryadov2009-1} %11
\Au{Зарядов И.\,С.,  Печинкин~А.\,В.}
Стационарные временные характеристики системы ${GI/M/n/\infty}$
с~некоторыми вариантами дисциплины обобщенного об\-нов\-ле\-ния~//
{Автоматика и~телемеханика}, 2009. Вып.~12. С.~161--174.

\bibitem{Zaryadov2010-1} %12
\Au{Зарядов И.\,С.} 
Система массового обслуживания $GI/M/n/\infty$ с~обобщенным об\-нов\-ле\-ни\-ем~//
{Автоматика и~телемеханика}, 2010. Вып.~4. С.~130--139.

\bibitem{zarN1-1} %13
\Au{Korolkova A., Zaryadov~I.}
The mathematical model of the traffic transfer process with a~rate adjustable by {RED}~//
{Conference (International) on Ultra Modern Telecommunications Proceedings}, 2010.
P.~1046--1050.

\bibitem{RFC8289-1} %14
\Au{Nichols K., Jacobson V., McGregor A., Iyengar J.}
Controlled delay active queue management, 2018.
{\sf https:// datatracker.ietf.org/doc/rfc8289}.


\bibitem{Riordan1962-1} %15
\Au{Riordan J.} {Stochastic service systems}.~--- 
SIAM ser. in applied mathematics.~--- New York, NY, USA: Wiley, 1962. 139~p.

\bibitem{arxivRK-1} %16
\Au{Konovalov M.,  Razumchik~R.}
Queueing systems with renovation vs.\ queues with RED. Supplementary material~//
{ArXiv e-prints}, 2017. {\sf https://arxiv.\linebreak org/abs/1709.01477/}.

\bibitem{kulk-1} %17
\Au{Kulkarni V.\,G.} Modeling and analysis of stochastic systems. 3rd ed.~--- 
Chapman \&~Hall/CRC texts in statistical science ser.~---
Chapman \& Hall/CRC, 2016. 606~p.

\bibitem{kono1-1}
\Au{Коновалов М.\,Г.} 
{Методы адаптивной обработки информации и~их приложения.}~--- 
М.: ИПИ РАН, 2007. 212~с.
\end{thebibliography}
} }

\end{multicols}

 \label{end\stat}

 \vspace*{-3pt}

\hfill{\small\textit{Поступила в~редакцию  09.10.2018}}


%\renewcommand{\bibname}{\protect\rm Литература}
%\renewcommand{\figurename}{\protect\bf Рис.}
\renewcommand{\tablename}{\protect\bf Таблица} %12 
\def\stat{dukova}

\def\tit{О ПОИСКЕ МАКСИМАЛЬНЫХ ЧАСТЫХ И~МИНИМАЛЬНЫХ НЕЧАСТЫХ НАБОРОВ ПРОИЗВЕДЕНИЯ ЧАСТИЧНЫХ ПОРЯДКОВ}

\def\titkol{О поиске максимальных частых и~минимальных нечастых наборов произведения частичных порядков}

\def\aut{Н.\,А.~Драгунов$^1$, Е.\,В.~Дюкова$^2$}

\def\autkol{Н.\,А.~Драгунов, Е.\,В.~Дюкова}

\titel{\tit}{\aut}{\autkol}{\titkol}

\index{Драгунов Н.\,А.}
\index{Дюкова Е.\,В.}
\index{Dragunov N.\,A.}
\index{Djukova E.\,V.}


%{\renewcommand{\thefootnote}{\fnsymbol{footnote}} \footnotetext[1]
%{Работа выполнена при поддержке Министерства науки и~высшего образования Российской Федерации (проект 
%075-15-2020-799).}}


\renewcommand{\thefootnote}{\arabic{footnote}}
\footnotetext[1]{Федеральный исследовательский центр <<Информатика 
и~управ\-ле\-ние>> Российской академии наук, \mbox{nikitadragunovjob@gmail.com}}
\footnotetext[2]{Федеральный исследовательский центр <<Информатика и~управ\-ле\-ние>> 
Российской академии наук, \mbox{edjukova@mail.ru}}

\vspace*{-3pt}




\Abst{Исследованы актуальные вопросы снижения временных затрат, возникающие при 
логическом анализе данных с~элементами из декартова произведения конечных час\-тич\-но 
упорядоченных множеств. Для задачи поиска по базе транзакций максимальных час\-тых и~минимальных 
нечастых наборов произведения час\-тич\-ных порядков предложен оригинальный метод, 
основанный на решении слож\-ной дискретной задачи, называемой дуализацией 
над произведением час\-тич\-ных порядков. Метод представляет собой синтез двух других 
известных методов, один из которых достаточно очевиден, а~другой использует идею 
инкрементального пе\-ре\-чис\-ле\-ния искомых наборов и~поэтому пред\-став\-ля\-ет 
в~основном тео\-ре\-ти\-че\-ский интерес. Проведено экспериментальное исследование предложенного 
подхода к~решению рас\-смат\-ри\-ва\-емой задачи в~случае произведения конечных цепей,
 выявлены условия его эф\-фек\-тив\-ности и~для проводимого анализа данных показана 
 це\-ле\-со\-об\-раз\-ность применения асимптотически оптимальных алгоритмов дуализации 
 над произведением час\-тич\-ных порядков.}

\KW{максимальные час\-тые наборы; минимальные не\-час\-тые наборы; дуализация над 
произведением час\-тич\-ных порядков; асимп\-то\-ти\-чески оптимальный алгоритм дуализации}

\DOI{10.14357/19922264220112}
  
%\vspace*{-4pt}


\vskip 10pt plus 9pt minus 6pt

\thispagestyle{headings}

\begin{multicols}{2}

\label{st\stat}

    \section{Введение}
    
    Рас\-смат\-ри\-ва\-емая задача анализа данных занимает важ\-ное мес\-то в~об\-ласти 
    информационного поиска и~в~случае бинарных данных ставится сле\-ду\-ющим образом~\cite{4}.
    
    Дано некоторое множество элементов~$V$. Подмножества $X \hm\subseteq V$ называются наборами. Пусть~$D$~--- 
    база данных, содержащая некоторые, не обязательно различные, наборы. Наборы, 
    содержащиеся в~$D$, называются транз\-ак\-ци\-ями. Под частотой набора~$\nu(X)$ понимается доля транз\-ак\-ций в~$D$, 
    содержащих~$X$. Если $\nu(X) \hm\geq s$, $s \hm\in \left[0, 1\right]$, то набор~$X$ называется $s$-час\-тым, 
    иначе он называется $s$-не\-час\-тым. Если набор частый и~он не содержится ни в~каком другом 
    час\-том наборе, то такой набор называется максимальным час\-тым. Если набор не\-час\-тый 
    и~при этом он не содержит в~себе никакого другого не\-час\-то\-го набора, то такой набор 
    называется минимальным нечастым. Требуется найти все максимальные час\-тые и~минимальные не\-час\-тые 
    наборы при заданном~$s$.
    
    Рас\-смат\-ри\-ва\-емая задача имеет много важных приложений, одним из которых является 
    нахождение ассоциативных правил в~базах данных. В~случае бинарных данных ассоциативное правило~---
     это упорядоченная пара $ \left( X, Y \right)$ непересекающихся подмножеств множества~$V$, обо\-зна\-ча\-емая 
     $X \hm\Rightarrow Y$. Поддержкой правила $X \hm\Rightarrow Y$ называется час\-то\-та набора $Z\hm = X \cup Y$.
      Достоверностью правила $X\hm \Rightarrow Y$ называется доля транзакций, со\-дер\-жа\-щих~$Y$, 
      среди всех транзакций, содержащих~$X$. Требуется \mbox{найти} все ассоциативные правила, 
      удовле\-тво\-ря\-ющие заданным минимальной поддержке $s\hm \in [0, 1]$ и~минимальной 
      достоверности $c \hm\in [0, 1]$.  Впервые задача нахождения ассоциативных правил
       была поставлена в~\cite{1}, где она формулировалась как задача анализа по\-тре\-би\-тель\-ской корзины.

    В случае небинарных данных каждый элемент из~$V$ имеет некоторое множество чис\-ло\-вых значений 
    и~вместо наборов элементов рас\-смат\-ри\-ва\-ют\-ся наборы их значений.

    Поиск ассоциативных правил осуществляется в~два этапа. 
    На первом этапе находятся частые наборы, на втором этапе из найденных час\-тых 
    наборов формируются ассоциативные правила. При формировании правил на втором 
    этапе фактически возникает задача поиска $t$-не\-час\-тых наборов, где $t\hm > s/c$.
    
    С ростом размерности современных баз данных находить все час\-тые и~не\-час\-тые 
    наборы становится неэффективно как по времени, так и~по памяти в~силу 
    экспоненциального рос\-та чис\-ла таких наборов. Одно из решений данной проблемы 
    заключается в~поиске только максимальных час\-тых наборов и~только минимальных 
    нечастых наборов, что позволяет компактно хранить информацию о~всех час\-тых и~не\-час\-тых 
    наборах соответственно. 
    
    
    В~\cite{9} рас\-смот\-ре\-на задача поиска множеств максимальных час\-тых наборов~$X_{\max}$ 
    и~минимальных не\-час\-тых наборов~$Y_{\min}$ в~данных, пред\-став\-лен\-ных в~виде декартова 
    произведения час\-тич\-но упорядоченных множеств. Показано, что в~этом случае 
    при построении тре\-бу\-емых наборов возникают соответственно задача поиска 
    максимальных независимых элементов час\-тич\-ных порядков и~задача поиска минимальных 
    независимых элементов час\-тич\-ных порядков.  Каж\-дая из этих задач называется 
    дуализацией над произведением час\-тич\-ных порядков~\cite{8}. Обе задачи относятся к~одним 
    из цент\-раль\-ных труд\-но\-ре\-ша\-емых пе\-ре\-чис\-ли\-тель\-ных задач дис\-крет\-ной математики.
    
    Существует достаточно очевидный способ поиска максимальных час\-тых и~минимальных
     не\-час\-тых наборов произведения час\-тич\-ных порядков, основанный на по\-сле\-до\-ва\-тель\-ном 
     по\-стро\-ении указанных множеств. Одно из множеств ищется, например, алгоритмом Apriori~\cite{2},
      второе множество получается путем дуализации первого. 
      В~настоящей работе показано, что метод эффективен только в~случае, когда чис\-ло час\-тых 
      наборов существенно меньше или, наоборот, существенно больше чис\-ла не\-час\-тых наборов. 
      В~\cite{9} предложена идея со\-вмест\-но\-го пе\-ре\-чис\-ле\-ния~$X_{\max}$ и~$Y_{\min}$ с~использованием
       инкрементального алгоритма дуализации из~\cite{14}, которая автором экспериментально 
       не исследована.
    
    Основной результат настоящей работы~--- разработка нового подхода к~решению 
    поставленной задачи, который является синтезом последовательного и~совместного подходов. 
    
    Экспериментальные исследования, проведенные в~настоящей работе для случая
     произведения цепей, свидетельствуют о~том, что предложенный по\-сле\-до\-ва\-тель\-но-со\-вмест\-ный 
     метод наиболее эффективен в~случае, когда мощ\-ность множества час\-тых наборов примерно 
     равна мощ\-ности множества не\-час\-тых наборов.
     
     \vspace*{-6pt}
     
    
    \section{Постановка задачи поиска максимальных частых 
    и~минимальных нечастых наборов произведения частичных порядков}
    
         \vspace*{-2pt}
    
    Пусть $\mathcal{P} = \mathcal{P}_1 \times \dots \times \mathcal{P}_n$~--- 
    де\-кар\-то\-во произведение час\-тич\-но упорядоченных множеств. Элементы~$\mathcal{P}$ называются наборами. 
    На множестве~$\mathcal{P}$ определяется отношение частичного порядка~$\preceq$ сле\-ду\-ющим образом: 
    если $p \hm= (p_1, \dots, p_n) \hm\in \mathcal{P}$ и~$q \hm= (q_1, \dots, q_n)\hm \in \mathcal{P}$, 
    то $ p \hm\preceq q$ в~$ \mathcal{P}\hm \Leftrightarrow p_1 \hm\preceq q_1$ 
    в~$\mathcal{P}_1, \dots, p_n \hm\preceq q_n$ в~$ \mathcal{P}_n$.
    
    Пусть $\mathcal{D} (\mathcal{P})$~--- некоторая со\-во\-куп\-ность
     наборов из~$\mathcal{P}$, называемая базой данных. Наборы, на\-хо\-дя\-щи\-еся в~базе 
     данных $\mathcal{D} (\mathcal{P})$, необязательно по\-пар\-но раз\-лич\-ны и~называются транзакциями. 
     
    Введем обозначения: 
    $\vert \mathcal{D} (\mathcal{P}) \vert$~--- чис\-ло транз\-ак\-ций в~$\mathcal{D} (\mathcal{P})$; 
    $\mathcal{S}_\mathcal{D}(p)$~--- число транз\-ак\-ций в~$\mathcal{D} (\mathcal{P})$, 
    сле\-ду\-ющих за $p \hm\in \mathcal{P}$; $s \hm\in [0, 1]$. 
    
    \smallskip
    
    \noindent
    \textbf{Определение~1.}\
     Набор $p \in \mathcal{P}$ называется $s$-час\-тым, 
     если $\mathcal{S}_\mathcal{D}(p) / \vert \mathcal{D} (\mathcal{P}) \vert \hm\geq s$. Иначе набор~$p$ 
     называется $s$-не\-час\-тым.
    
    \smallskip
    
    \noindent
    \textbf{Определение~2.}\
    Набор $p \in \mathcal{P}$ называется максимальным $s$-час\-тым, если 
    он $s$-час\-тый и~никакой сле\-ду\-ющий за ним набор~$z$, $z\hm \neq p$, не является $s$-час\-тым.

    
    \smallskip
    
    \noindent
    \textbf{Определение~3.}\
    Набор $p \in \mathcal{P}$ называется минимальным $s$-не\-час\-тым, если он $s$-не\-час\-тый 
    и~никакой пред\-шест\-ву\-ющий ему набор~$z$, $z \hm\neq p$, не является $s$-не\-час\-тым.


\smallskip
    
    Далее вместо $s$-частый ($s$-не\-час\-тый) набор будем писать час\-тый (не\-час\-тый) набор. 
    Множество всех максимальных час\-тых наборов будем обозначать как $X_{\max}$, 
    а~множество всех минимальных не\-час\-тых наборов как $Y_{\min}$.
    
    Пусть $R \subset \mathcal{P}$, $R^+\hm = \{ x \in \mathcal{P} \vert \exists\, a \hm\in R, a \hm\preceq x \}$, 
    $R^- \hm= \{ x \hm\in \mathcal{P} \vert \exists\, a \hm\in R, x \hm\preceq a \}$.


    \noindent
    \textbf{Определение~4.}\
     Множество $I(R^+)$, со\-сто\-ящее из всех максимальных элементов множества~$\mathcal{P} \setminus R^+$, 
     называется максимальным независимым от~$R$.

\smallskip


   \noindent
    \textbf{Определение~5.}\
     Множество $I(R^-)$, со\-сто\-ящее из всех минимальных элементов множества~$\mathcal{P} \setminus R^-$, 
     называется минимальным независимым от~$R$.

\smallskip
    
    Каждая из задач построения $I(R^+)$ и~$I(R^-)$ 
    при заданном множестве~$R$ называется задачей дуализации над произведением час\-тич\-ных порядков.
    
    \smallskip

    \noindent
    \textbf{Утверждение~1.}\
    Если $X \hm\subset X_{\max}$, а~$y \hm\in I(X^-)$~--- не\-час\-тый набор, 
    то~$y$~--- минимальный не\-час\-тый набор.

\smallskip    
    
    \noindent
    Д\,о\,к\,а\,з\,а\,т\,е\,л\,ь\,с\,т\,в\,о\,.\  \ 
    Пусть $y \hm\notin I(X_{\max}^-)$. Так как~$y$~--- 
    нечастый набор, то в~$\mathcal{P} \setminus X^{-}_{\max}$ найдется минимальный не\-час\-тый набор~$x$ 
    такой, что $x\hm \neq y$ и~$x \hm\preceq y$. Из того, что $\mathcal{P} \setminus X^{-}_{\max} 
    \hm\subseteq \mathcal{P} \setminus X^-$, следует, что $x\hm \in \mathcal{P} \setminus X^-$, 
    что противоречит условию $y \hm\in I(X^-)$.

\smallskip

\noindent
\textbf{Утверждение~2.}\
    Пусть $X \hm\subseteq X_{\max}$, $Y\hm \subseteq Y_{\min}$. 
    Тогда $I(X^-) \hm= Y$ в~том и~только в~том случае, когда $X \hm= X_{\max}$ и~$Y \hm= Y_{\min}$.


\smallskip


  \noindent
    Д\,о\,к\,а\,з\,а\,т\,е\,л\,ь\,с\,т\,в\,о\,.\  \
    Пусть $X\! \subset\! X_{\max}, x \hm\in X_{\max}\!\setminus\!X$.
     Так как множество~$X_{\max}$~--- антицепь, то $x \hm\notin X^-$. 
     Следовательно, $x \hm\in \mathcal{P} \setminus X^{-}$.
      Но тогда существует элемент $ q \hm\in I(X^-) : q \preceq x$, 
      который является час\-тым. Однако во множестве~$Y$ частых наборов нет; следовательно, $I(X^-) \hm\neq Y$. 
      Если же $X \hm= X_{\max}$, то $I(X^-) \hm= Y_{\min}$. Таким образом, $I(X^-) \hm= Y$ тогда и~только
       тогда, когда $X \hm= X_{\max}$ и~$Y\hm = Y_{\min}$.


    
    \section{Методы построения множеств~$X_{\max}$ и~$Y_{\min}$}

    \subsection{Последовательное перечисление $X_{\max}$~и~$Y_{\min}$}

    Достаточно очевиден поиск~$X_{\max}$ и~$Y_{\min}$ при заданной $\mathcal{D} (\mathcal{P})$ 
    путем последовательного по\-стро\-ения множеств~$X_{\max}$ и~$Y_{\min}$. 
    Данный поиск осуществляется в~два этапа. На первом этапе находятся все максимальные частые 
    наборы~$X_{\max}$, например алгоритмом Apriori~\cite{2}. На втором этапе  используется свойство 
    двойственности $I \left(X_{\max}^- \right)\hm = Y_{\min}$. 
    Минимальные нечастые наборы~$Y_{\min}$ находятся путем дуализации найденного на первом этапе 
    множества~$X_{\max}$. Аналогично можно сначала искать~$Y_{\min}$ алгоритмом Apriori, а~затем 
    искать~$X_{\max}$ путем дуализации~$Y_{\min}$.

    Очевидно, что данный подход будет проявлять себя наилучшим образом в~случаях, когда 
    алгоритм Apriori или его модификации могут найти одно из искомых множеств существенно
     быст\-рее, чем другое множество, например когда мощ\-ность~$X_{\max}$ 
     существенно меньше (больше) мощ\-ности~$Y_{\min}$.
    
    \subsection{Совместное перечисление $X_{\max}$ и~$Y_{\min}$}

    В~\cite{9} предложена идея совместного перечисления множеств~$X_{\max}$ и~$Y_{\min}$. 
    На первом шаге рас\-смат\-ри\-ва\-ет\-ся некоторый случайный набор $q \hm\in \mathcal{P}$. Если $q$~--- 
    час\-тый набор, то ищется максимальный час\-тый набор, сле\-ду\-ющий за~$q$, 
    который пополняет множество $X \hm\subseteq X_{\max}$. Если $q$~---
     не\-час\-тый набор, то ищется минимальный не\-час\-тый набор, пред\-шест\-ву\-ющий~$q$, 
     который пополняет множество $Y \hm\subseteq Y_{\min}$. Пусть на шаге~$i$ ($i\hm \geq 1$) 
     построены множества $X \hm\subseteq X_{\max}$ и~$Y \hm\subseteq Y_{\min}$. Если $X \hm\neq \varnothing$, 
     $Y \hm= \varnothing$, то ищется набор~$q$ такой, что $q \hm\npreceq x, \forall x \hm\in X$. Если 
     $X \hm= \varnothing$, $Y \hm\neq \varnothing$, то ищется набор~$q$ такой, что 
     $q \hm\nsucceq y, \forall y \hm\in Y$. Если же и~$X \hm\neq \varnothing$, и~$Y \hm\neq \varnothing$, 
     то ищется набор~$q$ такой, что $q \hm\npreceq x, \forall x \hm\in X, q \hm\nsucceq y, \forall y \hm\in Y$.
      Затем, аналогично первому шагу, находится максимальный частый или минимальный нечастый набор. 
      Однако в~\cite{9} идея совместного перечисления искомых множеств экспериментально 
      не исследована и~не предложены конкретные указания по воз\-мож\-ной ее реализации.
    
    Алгоритм, основанный на совместном пе\-ре\-чис\-ле\-нии множеств~$X_{\max}$ и~$Y_{\min}$,
     реализован в~на\-сто\-ящей работе. Алгоритм строит две последовательности: $X_1 \hm\subset X_2 
     \subset \dots \subset X_{\max}$, $Y_1\hm \subset Y_2 \subset \dots \subset Y_{\min}$. 
     На первом шаге $X_1 \hm= \{x\}$, $Y_1 \hm= \{y\}$, где~$x$ и~$y$ ищутся алгоритмом Apriori.
      На шаге $i \hm+ 1$ ($i\hm \geq 1$) строится либо~$I(X^{-}_{i})$, либо~$I(Y^{+}_{i})$. Пусть на 
      шаге $i \hm+ 1$ ($i \hm\geq 1$) построено множество~$I(X^{-}_{i})$. 
      Согласно утверждениям~1 и~2, множество~$I(X^{-}_{i})$ либо не содержит час\-тых наборов 
      и~совпадает с~множеством~$Y_{\min}$ (в~этом случае $X_i \hm= X_{\max}$ 
      и~алгоритм заканчивает работу), либо~$I(X^{-}_{i})$ содержит как час\-тые, так и~не\-час\-тые наборы. 
      Каждый нечастый набор из~$I(X^{-}_{i})$ является минимальным не\-час\-тым и~пополняет множество~$Y_{i}$, 
      формируя в~результате множество~$Y_{i+1}$. Для каждого час\-то\-го набора находится один содержащий 
      его максимальный час\-тый набор путем последовательного увеличения текущего 
      частого набора в~лексикографическом порядке, который пополняет множество~$X_{i}$, 
      формируя в~результате множество~$X_{i+1}$.
      
    В~экспериментальной части работы (см.\ разд.~4) рас\-смот\-рен случай произведения цепей. 
    Задача дуализации решается с~помощью асимптотически оптимального алгоритма дуализации
     цепей \mbox{RUNC-M}+~\cite{7}. Асимптотически оптимальные алгоритмы дуализации 
     являются лидерами по ско\-рости счета~\cite{6}.

    Очевидно, что время работы совместного алгоритма в~основном зависит от чис\-ла
     минимальных не\-час\-тых и~максимальных час\-тых наборов. На\linebreak каж\-дой новой 
     итерации происходит дуализация\linebreak все б$\acute{\mbox{о}}$льших по мощ\-ности множеств~$X$ или~$Y$.\linebreak 
     Если число итераций становится достаточно\linebreak большим, то ско\-рость работы совместного 
     перечисления существенно снижается, что делает его практически неприменимым для 
     задач большой раз\-мер\-ности.
     { %\looseness=1
     
     }

    \subsection{Последовательно-совместное перечисление~$X_{\max}$ и~$Y_{\min}$}

    Предлагается следующий итеративный метод, который синтезирует идеи последовательного
     и~совместного методов, описанных выше. Положим $X_0 \hm= \varnothing$. 
     Строится одна по\-сле\-до\-ва\-тель\-ность $X_1 \hm\subset X_2 \hm\subset \dots \subset X_{\max}$. 
     На первом шаге $X_1\hm = \{x\}$, где $x$ ищется алгоритмом Apriori. На шаге $i \hm+ 1$ ($i \hm\geq 1$) 
     решается задача дуализации множества $X_{i} \setminus X_{i-1}$.

    
    
   \setcounter{figure}{1}
    \begin{figure*}[b] %fig2
  \vspace*{12pt}
  \begin{center}  
    \mbox{%
\epsfxsize=163mm
\epsfbox{duk-2.eps}
}

\end{center}
\vspace*{-9pt}
    \Caption{Зависимость времени работы алгоритмов от суммы мощностей множеств~$X_{\max}$ и~$Y_{\min}$ 
    для случая~1~(\textit{а}) и~2~(\textit{б}):
    \textit{1}~--- по\-сле\-до\-ва\-тель\-но-со\-вмест\-ный;
    \textit{2}~--- последовательный; \textit{3}~--- совместный; \textit{4}~--- Apriori}
    \label{12}
    \end{figure*}
     
    Пусть множество~$D$ есть результат дуализации $X_{i} \hm\setminus X_{i-1}$. Согласно утверждению~1, 
    множество~$D$ содержит частые наборы. Для каждого час\-то\-го набора из~$D$ 
    находится один содержащий его максимальный час\-тый набор путем последовательного 
    увеличения текущего час\-то\-го набора в~лексикографическом порядке. Все найденные максимальные
     частые наборы, которых нет в~множестве~$X_{i}$, до\-бав\-ля\-ют\-ся к~$X_{i}$, 
     и~таким образом формируется~$X_{i+1}$. Если же все найденные частые наборы уже содержатся в~$X_{i}$, 
     то решается задача дуализации множества~$X_{i}$. Если в~$I(X^{-}_{i})$ нет частых наборов, 
     то $I(X^{-}_{i})\hm = Y_{\min}$, $X_i \hm= X_{\max}$ и~алгоритм завершает работу. 
     Иначе для каждого частого набора из~$I(X^{-}_{i})$ находится один содержащий его максимальный 
     час\-тый набор, который пополняет множество~$X_{i}$, формируя в~результате множество~$X_{i+1}$.

    \section{Экспериментальное исследование}
    
    Рас\-смат\-ри\-вал\-ся случай данных, пред\-став\-лен\-ных в~виде произведения цепей мощ\-ности~5. 
    Для\linebreak таких данных проводился поиск максимальных час\-тых и~минимальных нечастых 
    наборов сле\-ду\-ющи\-ми методами: алгоритмом Apriori, модифицированным для случая 
    цепей; последовательным \mbox{методом}; совместным методом; по\-сле\-до\-ва\-тель\-но-со\-вмест\-ным методом.
    
    Все методы реализованы на языке Python~3. 
    Задача дуализации решалась алгоритмом дуализации цепей RUNC-M+~\cite{7}. 
    Эксперименты проведены на случайных базах данных различной раз\-мер\-ности. 
    Можно выделить два сле\-ду\-ющих случая соотношения мощностей множеств всех час\-тых и~не\-час\-тых наборов.
    \begin{description}
    \item[Случай 1:] мощ\-ность множества частых наборов примерно рав\-на мощ\-ности множества нечастых наборов.
    \item[Случай 2:] мощ\-ность множества частых наборов существенно меньше (больше) мощ\-ности множества 
    не\-час\-тых наборов.
    \end{description}
    
    Описанные случаи схематично изображены на рис.~1. 

    Графики зависимости времени работы тестируемых методов 
    от мощ\-ности множеств~$X_{\max}$ и~$Y_{\min}$ приведены на рис.~2.
    
    

    

    Нетрудно видеть, что в~случае~1 лучше работает по\-сле\-до\-ва\-тель\-но-со\-вмест\-ный алгоритм: 
    множества час\-тых и~не\-час\-тых наборов имеют примерно одинаковую мощ\-ность, 
    поэтому быст\-рее будет обрабатывать их по\-сле\-до\-ва\-тель\-но-со\-вмест\-ным методом. В~случае~2 
    быст\-рее работает последовательный алгоритм: быст\-рее найти множество максимальных час\-тых наборов, 
    обработав множество час\-тых наборов, и~дуализировать результат. Время поиска множеств~$X_{\max}$ 
    и~$Y_{\min}$ совместным методом и~модифицированным алгоритмом Apriori рас\-тет существенно 
    быст\-рее времени поиска по\-сле\-до\-ва\-тель\-но-со\-вмест\-ным методом в~обоих случаях.
    
    { \begin{center}  %fig1
 \vspace*{9pt}
    \mbox{%
\epsfxsize=67.963mm
\epsfbox{duk-1.eps}
}

\end{center}

\noindent
{{\figurename~1}\ \ \small{
Два случая соотношения мощностей множеств час\-тых и~не\-час\-тых наборов
}}}

%\vspace*{6pt}


    \section{Заключение}
    
Рас\-смот\-ре\-на задача поиска максимальных час\-тых и~минимальных не\-час\-тых наборов в~данных, 
представленных в~виде декартова произведения час\-тич\-ных порядков. Актуальны вопросы 
снижения временн$\acute{\mbox{ы}}$х затрат, возникающих при реализации методов нахождения искомых наборов.
 Разработан новый подход к~по\-стро\-ению множества максимальных частых наборов~$X_{\max}$ и~множества 
 минимальных не\-час\-тых наборов~$Y_{\min}$, пред\-став\-ля\-ющий собой синтез двух ранее известных 
 подходов: последовательного и~со\-вмест\-но\-го (первый достаточно очевиден, идея второго предложена в~\cite{9}). 
 Сложность последовательного, совместного и~пред\-ла\-га\-емо\-го по\-сле\-до\-ва\-тель\-но-со\-вмест\-но\-го поиска 
 обуслов\-ле\-на, в~том чис\-ле, не\-об\-хо\-ди\-мостью рас\-смат\-ри\-вать в~процессе поиска 
 труд\-но\-ре\-ша\-емую пе\-ре\-чис\-ли\-тель\-ную задачу дис\-крет\-ной математики, на\-зы\-ва\-емую дуализацией 
 над произведением час\-тич\-ных порядков.

Для случая, когда данные пред\-став\-ле\-ны в~виде произведения конечных цепей, 
приведены результаты экспериментального срав\-не\-ния названных подходов, а~так\-же независимого 
способа \mbox{по\-стро\-ения} множеств~$X_{\max}$ и~$Y_{\min}$, не тре\-бу\-юще\-го решения задачи дуализации. 
Эксперименты проводились на модельных задачах с~применением асимптотически оптимального
 алгоритма дуализации над произведением конечных цепей \mbox{RUNC-M}+~\cite{7}. 
 Результаты исследования свидетельствуют о~том, что по\-сле\-до\-ва\-тель\-но-со\-вмест\-ный 
 метод наиболее эффективен (требует меньших временн$\acute{\mbox{ы}}$х затрат по сравнению с~другими рас\-смот\-рен\-ны\-ми 
 методами) в~случае, когда мощ\-ность множества час\-тых наборов примерно равна мощ\-ности множества
  нечастых наборов. Иначе выигрывает последовательный поиск. Наихудшие показатели 
  у~независимого пе\-ре\-чис\-ле\-ния множеств~$X_{\max}$ и~$Y_{\min}$ с~использованием в~качестве
   базового алгоритма Apriori~\cite{2}, точ\-нее его модификации на тес\-ти\-ру\-емый случай. 
   Таким образом, показана це\-ле\-со\-об\-раз\-ность применения алгоритмов дуализации для 
   по\-стро\-ения множеств~$X_{\max}$ и~$Y_{\min}$.

  
  {\small\frenchspacing
 {%\baselineskip=10.8pt
 %\addcontentsline{toc}{section}{References}
 \begin{thebibliography}{9}  
    \bibitem{4}
    \Au{Aggarwal C.} 
    Frequent pattern mining.~--- Heidelberg: Springer, 2014. 467~p.
    
    \bibitem{1}
    \Au{Agrawal~R., Imielinski~T., Swami~A.} Mining association rules 
    between sets of items in large databases~// \mbox{SIGMOD} Conference (International) on Management of Data
    Proceedings.~--- New York, NY, USA: ACM, 1993. P.~207--216.
    
    \bibitem{9}
    \Au{Elbassioni K.} On finding minimal infrequent elements in multi-dimensional 
    data defined over partially ordered sets~// arXiv.org, 2014. 30~p. arXiv:1411.2275 [cs.DB].
    
    \bibitem{8}
    \Au{Elbassioni K.} Algorithms for dualization over products of partially 
    ordered sets~// SIAM J.~Discrete Math., 2009. Vol.~23. Iss.~1. P.~487--510.
    
    \bibitem{2}
    \Au{Agrawal R., Srikant~R.} 
    Fast algorithms for mining association rules in large databases~// 
    20th Conference (International) on Very Large Data Bases Proceedings.~--- San Francisco, CA, USA: 
    Morgan Kaufmann Publs. Inc., 1994. P.~487--499.
    
    \bibitem{14}
    \Au{Хачиян Л.\,Г.} Избранные труды.~--- М.: МЦНМО, 2009. 520~с.
    
    \bibitem{7}
    \Au{Дюкова Е.\,В., Масляков~Г.\,О., Прокофьев~П.\,А.} 
    О~дуализации над произведением частичных порядков~// Машинное обучение и~анализ данных, 2017. Т.~3. №\,4.  
    C.~239--249.
    
    \bibitem{6}
    \Au{Дюкова Е.\,В., Прокофьев~П.\,А.} Об асимптотически оптимальных алгоритмах дуализации~// 
    Ж.~вычисл. матем. и~матем. физ., 2015. Т.~55. №\,5. С.~895--910.
    \end{thebibliography}

 }
 }

\end{multicols}

\vspace*{-6pt}

\hfill{\small\textit{Поступила в~редакцию 15.01.21}}

\vspace*{8pt}

%\pagebreak

%\newpage

%\vspace*{-28pt}

\hrule

\vspace*{2pt}

\hrule

%\vspace*{-2pt}

\def\tit{FINDING MAXIMAL FREQUENT AND~MINIMAL INFREQUENT SETS IN~PARTIALLY ORDERED DATA}


\def\titkol{Finding maximal frequent and~minimal infrequent sets in~partially ordered data}


\def\aut{N.\,A.~Dragunov and E.\,V.~Djukova}

\def\autkol{N.\,A.~Dragunov and E.\,V.~Djukova}

\titel{\tit}{\aut}{\autkol}{\titkol}

\vspace*{-11pt}


\noindent
Federal Research Center ``Computer Science and Control'' 
of the Russian Academy of Sciences, 44-2~Vavilov Str., Moscow 119333, Russian Federation

\def\leftfootline{\small{\textbf{\thepage}
\hfill INFORMATIKA I EE PRIMENENIYA~--- INFORMATICS AND
APPLICATIONS\ \ \ 2022\ \ \ volume~16\ \ \ issue\ 1}
}%
 \def\rightfootline{\small{INFORMATIKA I EE PRIMENENIYA~---
INFORMATICS AND APPLICATIONS\ \ \ 2022\ \ \ volume~16\ \ \ issue\ 1
\hfill \textbf{\thepage}}}

\vspace*{3pt} 


\Abste{Relevant issues of time costs reducing in the logical analysis of data with elements 
from the Cartesian product of finite partially ordered sets are investigated. 
An original method based on solving a complex discrete problem called dualization
 over the product of partial orders is proposed for the problem of finding maximal 
 frequent and minimal infrequent sets in the transaction database. The proposed method 
 is a~synthesis of two other known methods, one of which is quite obvious and the other uses 
 the idea of an incremental enumeration of target\linebreak\vspace*{-12pt}}
 
 \Abstend{sets and is, therefore, mainly 
 of theoretical interest. An experimental study of the considered approaches in
  the case of the product of finite chains is carried out and conditions for
   their effectiveness are revealed. The expediency of applying 
asymptotically optimal dualization algorithms over the product of partial orders is shown.}

\KWE{maximal frequent sets; minimal infrequent sets; dualization over the product of 
partial orders; asymptotically optimal dualization algorithm}

\DOI{10.14357/19922264220112}

%\vspace*{-16pt}

%\Ack
%\noindent




%\vspace*{6pt}

  \begin{multicols}{2}

\renewcommand{\bibname}{\protect\rmfamily References}
%\renewcommand{\bibname}{\large\protect\rm References}

{\small\frenchspacing
 {%\baselineskip=10.8pt
 \addcontentsline{toc}{section}{References}
 \begin{thebibliography}{9}
\bibitem{1-dr}
\Aue{Aggarwal, C.} 2014. \textit{Frequent pattern mining}. Heidelberg: Springer. 467~p.
\bibitem{2-dr}
\Aue{Agrawal, R., T.~Imielinski, and A.~Swami.}
 1993. Mining association rules between sets of items in large databases. 
 \textit{SIGMOD  Conference (International) on Management of Data Proceedings}. New York, NY:
 ACM. 207--216. 
\bibitem{3-dr}
\Aue{Elbassioni, K.}
 2014. On finding minimal infrequent elements in multidimensional data defined over partially ordered sets. 
 arXiv.org. 30~p. Available at: 
 {\sf https://arxiv.org/\linebreak pdf/1411.2275.pdf} (accessed January~25, 2022).
\bibitem{4-dr}
\Aue{Elbassioni, K.} 2009. Algorithms for dualization over products of partially ordered sets. 
\textit{SIAM J.~Discrete Math.} 23(1):487--510.
\bibitem{5-dr}
\Aue{Agrawal, R., and R.~Srikant.}
 1994. Fast algorithms for mining association rules in large databases. 
 \textit{20th Conference (International) on Very Large Data Bases Proceedings}.
 San Francisco, CA: 
    Morgan Kaufmann Publs. Inc.  487--499.
\bibitem{6-dr}
\Aue{Khachiyan, L.\,G.} 2009. \textit{Izbrannye trudy} [Selected works]. Moscow: MCCME. 520~p.
\bibitem{7-dr}
\Aue{Djukova, E.\,V., G.\,O.~Maslyakov, and P.\,A.~Prokofyev.} 
2017. O~dualizatsii nad proizvedeniem chastichnykh poryadkov [On dualization over the product of 
partial orders]. \textit{Mashinnoe obuchenie i~analiz dannykh} [J.~Machine Learning Data Analysis] 
3(4):239--249.
\bibitem{8-dr}
\Aue{Djukova, E.\,V., and P.\,A.~Prokofyev.}
 2015. Asymptotically optimal dualization algorithms. \textit{Comp. Math.
 Math. Phys.} 55(5):891--905. 
 
 \end{thebibliography}

 }
 }

\end{multicols}

\vspace*{-6pt}

\hfill{\small\textit{Received January 15, 2021}}

%\pagebreak

%\vspace*{-18pt}

\Contr

\noindent
\textbf{Dragunov Nikita A.} (b.\ 1997)~--- 
PhD student, Federal Research Center ``Computer Science and Control'' 
of the Russian Academy of Sciences, 44-2~Vavilov Str., Moscow 119333, Russian Federation; 
\mbox{nikitadragunovjob@gmail.com}

\vspace*{3pt}

\noindent
\textbf{Djukova Elena V.} (b.\ 1945)~--- 
Doctor of Science in physics and mathematics, principal scientist, Federal Research Center
``Computer Science and Control'' of the Russian Academy of Sciences, 44-2~Vavilov Str., Moscow 119333, 
Russian Federation; \mbox{edjukova@mail.ru}




\label{end\stat}

\renewcommand{\bibname}{\protect\rm Литература}     %13
\def\stat{inkova}

\def\tit{СТЕПЕНЬ СЕМАНТИЧЕСКОЙ БЛИЗОСТИ ДИСКУРСИВНЫХ ОТНОШЕНИЙ: МЕТОДЫ И~ИНСТРУМЕНТЫ РАСЧЕТА$^*$}

\def\titkol{Степень семантической близости дискурсивных отношений: методы и~инструменты расчета}

\def\aut{О.\,Ю.~Инькова$^1$, М.\,Г.~Кружков$^2$}

\def\autkol{О.\,Ю.~Инькова, М.\,Г.~Кружков}

\titel{\tit}{\aut}{\autkol}{\titkol}

\index{Инькова О.\,Ю.}
\index{Кружков М.\,Г.}
\index{Inkova O.\,Yu.}
\index{Kruzhkov M.\,G.}


{\renewcommand{\thefootnote}{\fnsymbol{footnote}} \footnotetext[1]
{Работа выполнена в~Федеральном исследовательском центре <<Информатика и~управление>> Российской 
академии наук с~использованием ЦКП <<Информатика>> ФИЦ ИУ РАН.}}


\renewcommand{\thefootnote}{\arabic{footnote}}
\footnotetext[1]{Федеральный исследовательский центр <<Информатика и~управление>> Российской академии наук; 
Женевский университет, \mbox{olyainkova@yandex.ru}}
\footnotetext[2]{Федеральный исследовательский центр <<Информатика и~управление>> Российской 
академии наук, \mbox{magnit75@yandex.ru}}

%\vspace*{-14pt}


  
  \Abst{Рассматриваются методы оценки семантической близости дискурсивных 
отношений. Авторы предлагают несколько подходов к~решению этой проблемы с~применением двух информационных ресурсов: коллекции сформированных авторами 
структурированных определений ло\-ги\-ко-се\-ман\-ти\-че\-ских отношений (ЛСО) 
и~Надкорпусной базы данных коннекторов (НБДК), включающей в~себя аннотации переводных 
соответствий текстовых фрагментов с~маркерами ЛСО на русском, французском 
и~итальянском языках. Показано, что при оценке семантической близости ЛСО высокий 
приоритет будут иметь такие факторы, как принадлежность различительных признаков ЛСО к~одному семейству в~структурированных определениях отношений, соответствия между 
показателями различных ЛСО в~оригинальных и~переводных текстах, а также случаи, когда 
различные ЛСО выражаются одинаковыми показателями в~разных контекстах. Менее значим 
фактор сочетаемости различных ЛСО в~рамках одного и~того же контекста. Предполагается, 
что на основе сформулированных методов станет возможным более точно определить, какие 
различительные признаки ЛСО имеют наибольший вес при определении их семантической  
бли\-зости.}
  
  \KW{надкорпусная база данных; логико-семантические отношения; коннекторы; 
аннотирование; фасетная классификация}

  \DOI{10.14357/19922264230412}{FXTSPZ}
  
%\vspace*{-1pt}


\vskip 10pt plus 9pt minus 6pt

\thispagestyle{headings}

\begin{multicols}{2}

\label{st\stat}
  
\section{Степень семантической близости дискурсивных 
отношений}

%\vspace*{-4pt}

  Проблемы классификации дискурсивных отношений, обеспечивающих 
связность текста, занимают лингвистов и~специалистов по автоматической 
обработке текста не один десяток лет: первые исследования начались  
в~1970-х~гг.~[1, 2]. Были предложены их многочисленные классификации (ср.\ 
наиболее известные~[3--7]), однако никто, насколько известно авторам, не 
пытался определить степень семантической близости (ССБ) дискурсивных 
отношений. Это связано прежде всего с~тем, что классификации имеют, за 
редким исключением~\cite{7-in, 8-in, 9-in}, форму списка, и~этот вопрос просто 
не ставился. Однако его решение полезно не только для анализа текста, в~том 
числе автоматического, но и~для когнитивных наук и~переводоведения, 
поскольку позволяет выявить общие закономерности человеческого мышления.
  
  Кроме того, сами дискурсивные отношения определены во многом неточно 
или тавтологично\footnote[3]{См., например, определение отношения альтернативы 
(disjunction) в~теории риторической структуры: (а)~элемент пред\-став\-ля\-ет собой (не 
обязательно исключающую) альтернативу другому; (б)~слу\-ша\-ющий/чи\-та\-тель 
распознает, что связанные элементы альтернативны (см.\ {\sf http://www.sfu.ca/rst}).}, схожие 
или идентичные отношения носят даже в~англоязычных классификациях разные 
названия, а одинаковые названия описывают разную языковую реальность. 
Например, в~теории сегментированного представления дискурса (Segmented 
Discourse Representation Theory, SDRT~[10]) отношение contrast включает как 
отношения <<вопреки ожидаемому>>, так и~уступительные отношения. 
В~классификации Пенсильванского аннотированного корпуса им 
соответствуют два отношения (opposition и~contra-expectation)~\cite{7-in}, 
а~в~теории риторической структуры~--- contrast и~concession~[11] (подробнее 
см.~\cite[с.~37]{9-in}). 

\begin{table*}[b]\small %tabl1
\vspace*{-10pt}
\begin{center}
\Caption{Структурированные определения уступительных ЛСО и~ЛСО <<вопреки 
ожидаемому>>}
\vspace*{2ex}

\tabcolsep=3pt
\begin{tabular}{|l|p{40mm}|p{38mm}|p{57mm}|}
\hline
\multicolumn{1}{|c|}{\textbf{ЛСО}} & \multicolumn{1}{c|}{\tabcolsep=0pt\begin{tabular}{c}\textbf{Базовая семантическая}\\ \textbf{операция}\end{tabular}}&
\multicolumn{1}{c|}{\textbf{Уровень}} &
\multicolumn{1}{c|}{ \tabcolsep=0pt\begin{tabular}{c}\textbf{Дополнительные}\\ \textbf{характеристики}\end{tabular}}\\
\hline
&&&\\[-20pt]
\multicolumn{1}{|l|}{\raisebox{-26pt}[0pt][0pt]{\textbf{Уступительные}}}& 
%\begin{itemize}
\multicolumn{1}{l|}{\raisebox{-26pt}[0pt][0pt]{\ \ \ \  --\ \ операция импликации}}
%\end{itemize} 
& 
%\begin{itemize}
\multicolumn{1}{l|}{\raisebox{-26pt}[0pt][0pt]{\tabcolsep=0pt\begin{tabular}{l}\ \ \ \ --\ \ пропозициональный\\
\hphantom{\ \ \ \ --\ \ }уровень\end{tabular}}}
%\end{itemize}
&
\begin{itemize}
\item $p$ и~$q$~--- положения вещей;\vspace*{-3pt}
\item как правило, если имеет место $q$, то не имеет места~$p$\vspace*{-8pt}
   \end{itemize}
\\
\hline
&&&\\[-20pt]
\multicolumn{1}{|l|}{\raisebox{-48pt}[0pt][0pt]{\tabcolsep=0pt\begin{tabular}{l}\textbf{<<Вопреки}\\ \textbf{ожидаемому>>}\end{tabular} }}& 
%\begin{itemize}
\multicolumn{1}{l|}{\raisebox{-48pt}[0pt][0pt]{\tabcolsep=0pt\begin{tabular}{l}\ \ \ \  --\ \ операция сравнения,\\
 \hphantom{\ \ \ \ --\ \ }уста\-нав\-ли\-ва\-ющая не-\\
 \hphantom{\ \ \ \ --\ \ }сходство $p$ и~$q$\end{tabular}}}
%\end{itemize} 
&
%\begin{itemize}
\multicolumn{1}{l|}{\raisebox{-48pt}[0pt][0pt]{\tabcolsep=0pt\begin{tabular}{l}\ \ \ \  --\ \ пропозициональный\\ 
 \hphantom{\ \ \ \ --\ \ }уровень\end{tabular}}}
%\end{itemize} 
&
 \begin{itemize}
 \item $q$ имеет большую аргументативную\newline силу, чем~$p$;\vspace*{-3pt}
  \item положение вещей $p$ служит аргументом в~пользу ожи\-да\-емо\-го вывода~$r$;\vspace*{-3pt}
  \item положение вещей $q$ служит аргументом в~пользу ожи\-да\-емо\-го вывода не-$r$\vspace*{-8pt}
  \end{itemize}\\
\hline
\end{tabular}
\end{center}
\end{table*}
  
  В~этой связи были сделаны попытки сравнить\linebreak существующие 
классификации, чтобы понять, насколько соотносимы выделяемые в~них 
дискурсивные отношения~[12--14]. В~[14] для этого применяется 
набор различительных признаков. Этих\linebreak признаков, однако, недостаточно, чтобы 
сформулировать уникальное определение отношения, и~некоторые из них 
имеют одинаковый набор признаков. Это касается, например, четырех 
отношений (narration, precondition, background и~parallel) в~SDRT~\cite[с.~38]{14-in}. 
  
  В~работе~[15] были заложены основы для разработки структурированных 
определений дискурсивных, или в~терминологии автора  
ло\-ги\-ко-се\-ман\-ти\-че\-ских, отношений на основе применяемой 
в~НБДК классификации. Каждое 
ЛСО может быть описано набором различительных признаков (см.\ примеры 
в~\cite{16-in} и~\cite{17-in}). Некоторые признаки оказываются общими для 
нескольких ЛСО, другие~--- индивидуальны, т.\,е.\ свойственны только данному 
ЛСО. На момент написания статьи в~НБДК были описаны 26~ЛСО 
с~использованием~52~различительных признаков. Это позволяет дать каждому 
ЛСО уникальное определение (см.\ примеры в~разд.~2), а~также определить 
ССБ ЛСО. 

\vspace*{-6pt}
  
\section{Критерии, лежащие в~основе определения степени 
семантической близости логико-семантических отношений}

\vspace*{-3pt}

  В~предыдущей работе авторов~[17] показано, что не все различительные 
признаки имеют одинаковый вес при определении семантической близости 
ЛСО и~что, предположительно, наибольшее значение имеет принадлежность 
общих признаков к~одному семейству. 
  

  
  В~основе уступительных ЛСО и~ЛСО <<вопреки ожидаемому>> лежат 
разные базовые операции: импликация~--- для первого и~сравнение, 
уста\-нав\-ли\-ва\-ющее несходство $p$ и~$q$,~--- для второго (табл.~1). Это 
значит, что эти два ЛСО находятся в~разных семантических группах. Оба ЛСО 
при этом установлены на пропозициональном уровне, т.\,е.\ непосредственно 
между положениями дел $p$ и~$q$, которые они связывают, и~оба используют 
отрицательный коррелят одного из положений вещей. Иначе говоря, признаки 
<<как правило, если имеет место~$q$, то не имеет места $p$>> и~<<положение 
вещей~$q$ служит аргументом в~пользу ожидаемого вывода не-$r$>> 
принадлежат к~одному семейству. В~примере~(1) с~ЛСО <<вопреки 
ожидаемому>>: \textit{Ему [$\ldots$] очень неприятно было сталкиваться с~народом,} {\bfseries\textit{но}} \textit{он шел именно туда, где виднелось больше 
народу}. [Ф.\,М.~Достоевский. Преступление и~наказание], положение вещей 
$p$\;=\;<<ему очень неприятно было сталкиваться с~народом>> ориентирует в~пользу вывода $r$\;=\;<<он не должен был бы идти к~народу>>. Этот вывод 
опровергается непосредственно в~$q$ (=\;не-$r$)\;=\;<<он шел именно туда, где 
виднелось больше народу>>. Семантический механизм, лежащий в~основе 
уступительных отношений (их прототипическим показателем может считаться 
союз \textit{хотя}), совпадает с~этим семантическим механизмом, но 
в~зеркальном отражении: 
  \begin{gather*}
p\ \mbox{\textit{хотя}}\  q (q \to  \mbox{не-}p)\\
p \to r\ \mbox{но}\  q\ (q = \mbox{не-}r),\ \mbox{т.\,е.}\ p \to \mbox{не-}q\ 
\mbox{\textit{но}}\ q.
\end{gather*}
  %
  Отсюда необходимость при замене \textit{хотя} на \textit{но} и~наоборот 
изменить порядок следования фрагментов текста: \textit{Ему неприятно было 
сталкиваться с~народом}, {\bfseries\textit{но}} \textit{он шел туда, где виднелось 
больше народу} (ЛСО <<вопреки ожидаемому>>); \textit{Он шел туда, где 
виднелось больше народу}, {\bfseries\textit{хотя}} \textit{ему неприятно было 
сталкиваться с~народом} (ЛСО уступки)~\cite{18-in}. Это позволяет говорить 
о~семантической близости двух ЛСО и,~например, в~классификации~\cite{7-in} 
они объединены в~одну группу concession.

\begin{table*}[b]\small %tabl2
\vspace*{-6pt}
\begin{center}
\Caption{Логико-семантические отношения, соответствующие ЛСО <<вопреки ожидаемому>> в~оригинальных и~переводных текстах }
\vspace*{2ex}

\tabcolsep=4.3pt
\begin{tabular}{|c|l|c|c|c|c|c|c|}
\hline
\textbf{ЛСО1}&\multicolumn{1}{c|}{\textbf{ЛСО2}}&\textbf{1}\;+\;\textbf{2}&\textbf{1}&
\textbf{2}&\textbf{1}\;$\to$\;\textbf{2}&\textbf{2}\;$\to$\;\textbf{1}&\textbf{Сумма}\\
\hline
<<вопреки ожидаемому>>&уступительные&237\hphantom{9}&2140&853&11,07\%\hphantom{9}&27,78\%\hphantom{9}&38,86\%\hphantom{9}\\
<<вопреки ожидаемому>>&одновременность&139\hphantom{9}&2140&1268\hphantom{9}&6,50\%&10,96\%\hphantom{9}&17,46\%\hphantom{9}\\
<<вопреки ожидаемому>>&соединительные&149\hphantom{9}&2140&2088\hphantom{9}&6,96\%&7,14\%&14,10\%\hphantom{9}\\
<<вопреки ожидаемому>>&сопоставительные&78&2140&807&3,64\%&9,67\%&13,31\%\hphantom{9}\\
<<вопреки ожидаемому>>&пропозициональное 
сопутствование&39&2140&378&1,82\%&10,32\%\hphantom{9}&12,14\%\hphantom{9}\\
<<вопреки ожидаемому>>&исключение из 
рассмотрения&\hphantom{9}8&2140&\hphantom{9}90&0,37\%&8,89\%&9,26\%\\
<<вопреки ожидаемому>>&иллокутивное 
сопутствование&17&2140&471&0,79\%&3,61\%&4,40\%\\
<<вопреки ожидаемому>>&интенсиональная 
генерализация&\hphantom{9}8&2140&248&0,37\%&3,23\%&3,60\%\\
<<вопреки ожидаемому>>&замещение&\hphantom{9}7&2140&294&0,33\%&2,38\%&2,71\%\\
<<вопреки ожидаемому>>&пропозициональная 
коррекция&\hphantom{9}4&2140&165&0,19\%&2,42\%&2,61\%\\
<<вопреки ожидаемому>>&условные&12&2140&1075\hphantom{9}&0,56\%&1,12\%&1,68\%\\
<<вопреки ожидаемому>>&спецификация&11&2140&1608\hphantom{9}&0,51\%&0,68\%&1,20\%\\
<<вопреки ожидаемому>>&исключение&\hphantom{9}5&2140&615&0,23\%&0,81\%&1,05\%\\
<<вопреки ожидаемому>>&отрицательная 
альтернатива&\hphantom{9}2&2140&271&0,09\%&0,74\%&0,83\%\\
<<вопреки ожидаемому>>&оговорка&\hphantom{9}1&2140&150&0,05\%&0,67\%&0,71\%\\
<<вопреки ожидаемому>>&экстенсиональная 
генерализация&\hphantom{9}2&2140&588&0,09\%&0,34\%&0,43\%\\
<<вопреки ожидаемому>>&переформулирование&\hphantom{9}2&2140&1183\hphantom{9}&0,09\%&0,17\%&0,26\%\\
<<вопреки ожидаемому>>&пропозициональная 
альтернатива&\hphantom{9}1&2140&1238\hphantom{9}&0,05\%&0,08\%&0,13\%\\
\hline
\multicolumn{8}{p{163mm}}{\footnotesize \hspace*{3mm}Расшифровка названий столбцов: 
1\;+\;2~--- число переводных аннотаций, в~которых ЛСО1 в~тексте на одном языке 
соответствует ЛСО2 в~тексте на другом языке; 1~--- число аннотаций, в~которых в~любом из 
текстов проставлено ЛСО1; 2~--- число аннотаций, в~которых в~любом из текстов 
проставлено ЛСО2; 1\;$\to$\;2~--- процент соответствия для ЛСО1 с~ЛСО2; 2\;$\to$\;1~--- 
процент соответствия для ЛСО2 с~ЛСО1; сумма~--- сумма двух предыдущих показателей.}
\end{tabular}
\end{center}
\end{table*}

  
  
  Кроме того, сформулирована гипотеза, согласно которой при определении 
ССБ ЛСО могут учитываться также другие 
факторы:
\begin{enumerate}[(1)] 
\item соответствия ЛСО в~оригинальных и~переводных текстах; 
\item случаи, когда разные ЛСО выражаются одним и~тем же показателем; 
\item сочетаемость показателей ЛСО в~одном фрагменте текста.
\end{enumerate}
 В~НБДК для 
ЛСО, имеющих структурированные определения, были получены 
количественные данные по этим трем критериям.

  
  
\subsection{Соответствие логико-семантических отношений в~оригинальных и~переводных текстах}

  Соответствие ЛСО в~оригинальных и~переводных текстах означает, что 
некоторому ЛСО в~тексте оригинала, точнее, его показателю, соответствует 
показатель иного ЛСО в~тексте перевода. Так, если для перевода на 
французский язык коннектора \textit{но} в~примере~(1) был выбран коннектор 
\textit{mais}, также выражающий ЛСО <<вопреки ожидаемому>>: (2)~\textit{Il 
lui $\acute{\mbox{e}}$tait d$\acute{\mbox{e}}$sagr$\acute{\mbox{e}}$able, 
tr$\grave{\mbox{e}}$s d$\acute{\mbox{e}}$sagr$\acute{\mbox{e}}$able, de 
rencontrer du monde} {\bfseries\textit{mais}} \textit{il allait justement 
l$\grave{\mbox{a}}$ o$\grave{\mbox{u}}$ l'on en voyait le plus} [перевод 
$\acute{\mbox{E}}$lisabeth Guertik], то в~примере~(3) тот же коннектор 
переведен \textit{bien que}~--- показателем уступительных ЛСО: 
\textit{С~такой поправкой смысл телеграммы становился ясен,} 
{\bfseries\textit{но}}\textit{, конечно, трагичен}.~--- \textit{Ainsi 
corrig$\acute{\mbox{e}}$, le t$\acute{\mbox{e}}$l$\acute{\mbox{e}}$gramme 
prenait un sens parfaitement clair,} {\bfseries\textit{bien que}} \textit{tragique, 
naturellement}. [М.~Булгаков. Мастер и~Маргарита, перевод Claude Ligny].
  
  Количественные данные по ЛСО, соответствующим ЛСО <<вопреки 
ожидаемому>> в~оригинальных и~переводных текстах на русском, французском и~итальянском языках, приведены в~табл.~2.
  
  
  Для ЛСО <<вопреки ожидаемому>> в~НБДК сформирована 2141~двуязычная 
аннотация. В~237~случаях ему соответствует уступительное ЛСО. Это 
подтверждает важность критерия принадлежности \mbox{различительных} признаков к~одному семейству. 

Схожую картину можно наблюдать для других отношений 
(табл.~3): для сопоставительных и~соединительных ЛСО (основаны на 
общей базовой операции и~имеют общий различительный признак 
<<сходство~$p$ и~$q$ относительно некоторого ``общего\linebreak знаменателя''>>); для 
ЛСО оговорки и~пропозициональной альтернативы (они имеют общий 
различительный признак~--- <<$p$ и~$q$~--- положения вещей, име\-ющие 
статус гипотезы>>); для ЛСО \mbox{одновременности} и~со\-по\-став\-ле\-ния (их 
различительные при\-зна\-ки <<T$p$ включает в~себя T$q$>> и~<<$p$ и~$q$ 
актуальны для говорящего в~момент речи T$d$>> принадлежат к~семейству 
признаков <<Единство временного интервала>>); для ЛСО одновременности 
и~пропозиционального сопутствования (об\-щий признак <<T$p$ включает 
в~себя T$q$>>). 
  
\begin{table*}\small %tabl3
\begin{center}
\Caption{Соответствия других ЛСО }
\vspace*{2ex}

\begin{tabular}{|l|l|c|c|c|c|c|c|}
\hline
\multicolumn{1}{|c|}{\textbf{ЛСО1}}&\multicolumn{1}{c|}{\textbf{ЛСО2}}&\textbf{1}\;+\;\textbf{2}&\textbf{1}&\textbf{2}&\textbf{1}\;
$\to$\;\textbf{2}&\textbf{2}\;$\to$\;\textbf{1}&\textbf{Сумма}\\
\hline
соединительные&сопоставительные&272\hphantom{9}&2088&807&13,03\%&33,71\%&46,73\%\\
оговорка&пропозициональная альтернатива&40&\hphantom{9}150&1238\hphantom{9}&26,67\%&\hphantom{9}3,23\%&29,90\%\\
одновременность&сопоставление&180\hphantom{9}&1268&807&14,20\%&22,30\%&36,50\%\\
одновременность &пропозициональное 
сопутствование&43&1268&378&\hphantom{9}3,39\%&11,38\%&14,77\%\\
\hline
\end{tabular}
\end{center}
\vspace*{-4pt}
\end{table*}

\begin{table*}[b]\small %tabl4
\vspace*{-12pt}
\begin{center}
\Caption{Количественные данные по ЛСО, выражаемым одним показателем}
\vspace*{2ex}

\begin{tabular}{|c|l|l|c|}
\hline 
\textbf{Язык}&\multicolumn{1}{c|}{\textbf{Коннектор}}&\multicolumn{1}{c|}{\textbf{ЛСО}}&\textbf{Количество аннотаций}\\
\hline
\multicolumn{1}{|c|}{\raisebox{-11pt}[0pt][0pt]{RU}}&\multicolumn{1}{l|}{\raisebox{-11pt}[0pt][0pt]{а то}}&отрицательная альтернатива&125\hphantom{9}\\
&&пропозициональная альтернатива&12\\
&&исключение из рассмотрения&\hphantom{9}6\\
\hline
\multicolumn{1}{|c|}{\raisebox{-6pt}[0pt][0pt]{RU}}&\multicolumn{1}{l|}{\raisebox{-6pt}[0pt][0pt]{если$\|$то}}&условные&183\hphantom{9}\\
&&сопоставительные&13\\
\hline
\multicolumn{1}{|c|}{\raisebox{-6pt}[0pt][0pt]{RU}}&\multicolumn{1}{l|}{\raisebox{-6pt}[0pt][0pt]{когда}}&одновременность&13\\
&&условные&\hphantom{9}1\\
\hline
\multicolumn{1}{|c|}{\raisebox{-6pt}[0pt][0pt]{RU}}&\multicolumn{1}{l|}{\raisebox{-6pt}[0pt][0pt]{когда$\|$то}}&одновременность&38\\
&&условные&\hphantom{9}6\\
\hline
\multicolumn{1}{|c|}{\raisebox{-11pt}[0pt][0pt]{RU}}
&\multicolumn{1}{l|}{\raisebox{-11pt}[0pt][0pt]{между тем}}
&одновременность&126\hphantom{9}\\
&&<<вопреки ожидаемому>>&53\\
&&сопоставительные&11\\
\hline
\multicolumn{1}{|c|}{\raisebox{-6pt}[0pt][0pt]{RU}}&\multicolumn{1}{l|}{\raisebox{-6pt}[0pt][0pt]{между тем как}}&сопоставительные&29\\
&&одновременность&\hphantom{9}6\\
\hline
\multicolumn{1}{|c|}{\raisebox{-18pt}[0pt][0pt]{RU}}
&\multicolumn{1}{l|}{\raisebox{-18pt}[0pt][0pt]{разве}}
&оговорка&20\\
&&исключение&\hphantom{9}5\\
&&исключение из рассмотрения&\hphantom{9}4\\
&&условные&\hphantom{9}2\\
\hline
\multicolumn{1}{|c|}{\raisebox{-6pt}[0pt][0pt]{FR}}&\multicolumn{1}{l|}{\raisebox{-6pt}[0pt][0pt]{cependant}}&<<вопреки ожидаемому>>&100\hphantom{9}\\
&&одновременность&27\\
\hline
\multicolumn{1}{|c|}{\raisebox{-6pt}[0pt][0pt]{FR}}&\multicolumn{1}{l|}{\raisebox{-6pt}[0pt][0pt]{en m$\hat{\mbox{e}}$me temps}}&одновременность&29\\
&&сопоставительные&\hphantom{9}1\\
\hline
\multicolumn{1}{|c|}{\raisebox{-6pt}[0pt][0pt]{FR}}&\multicolumn{1}{l|}{\raisebox{-6pt}[0pt][0pt]{quand}}&одновременность&197\hphantom{9}\\
&&условные&10\\
\hline
\end{tabular}
\end{center}
\end{table*}

  
  Напротив, ЛСО, соответствующие ЛСО <<вопреки ожидаемому>> 
и~представленные менее чем в~1\% аннотаций (см.\ табл.~2), не имеют 
различительных признаков, принадлежащих к~одному семейству, и~выбор их 
показателей для перевода показателя ЛСО <<вопреки ожидаемому>> может 
быть квалифицирован как авторский и~контекстуальный.
  
\subsection{Разные логико-семантические отношения выражаются одним~и~тем~же~показателем}

  Известно, что коннекторы в~значительной своей части относятся 
к~многозначным языковым единицам, т.\,е.\ могут служить показателями более 
чем одного ЛСО. Так, для русского союза \textit{и} принято выделять пять 
значений: сочинительное, временного следования, добавления,  
ре\-зуль\-та\-тив\-но-след\-ст\-вен\-ное и~несоответствия; для союза 
\textit{когда}~--- два: одновременности и~условия; у~союза \textit{но} 
выделяются собственно противительное  
и~про\-ти\-ви\-тель\-но-усту\-пи\-тель\-ное значения, а~у~\textit{хотя}~--- 
уступительное и~усту\-пи\-тель\-но-про\-ти\-ви\-тель\-ное и~т.\,д.~[19--21]. Это 
отражают и~данные НБДК, причем с~указанием на частотность того или иного 
значения коннектора в~сформированных аннотациях. 

В~табл.~4 приведены 
выборочно данные для многозначных коннекторов русского и~французского 
языков.
  

  
  Приведенные данные подтверждают прежде всего положения теории 
грамматикализации, согласно которым семантическая эволюция языковых 
единиц имеет определенные закономерности.\linebreak Так, было показано, что на основе 
значения одновременности может развиваться семантика сопоставления и~противопоставления, а~также импликации~\cite{22-in}. Это хорошо видно на 
примере \mbox{коннекторов} \textit{когда}, \textit{между тем}, а~также французских 
\textit{cependant} `в~то же время, однако', \textit{en m$\hat{\mbox{e}}$me temps} 
`в~то же время' и~\textit{quand} `когда' (см.\ табл.~4). С~другой стороны, эти 
данные подтверждают гипотезу авторов о~том, что набор ЛСО, которые может 
маркировать один показатель, не случаен, а~включает семантически близкие 
ЛСО. Так, коннектор \textit{разве} зафиксирован в~НБДК как показатель ЛСО 
оговорки, исключения, исключения из рассмотрения и~условия. Эти ЛСО имеют 
общие различительные признаки. Ло\-ги\-ко-се\-ман\-ти\-че\-ские отношения оговорки и~условия~--- два признака: 
базовая операция импликации и~признаки из семейства гипотетичность; ЛСО 
условия и~исключения устанавливаются на пропозициональном уровне, а~ЛСО 
оговорки и~исключения из рас\-смот\-ре\-ния~--- на уров\-не вы\-ска\-зы\-ва\-ния; ЛСО 
оговорки, исключения и~исключения из рас\-смот\-ре\-ния обладают общими 
признаками на уровне семейства признаков (семантика исключения), а~ЛСО 
исключения и~исключения из рас\-смот\-ре\-ния осно\-ва\-ны на общей базовой 
операции (соотнесение элемента и~множества).
  
  Таким образом, данный критерий может быть полезен при определении CСБ 
ЛСО и~иметь достаточно высокий приоритет.
  
\subsection{Сочетаемость логико-семантических отношений в~рамках одного фрагмента текста}

  Третий критерий, который можно учитывать при определении ССБ ЛСО,~--- 
сочетаемость ЛСО, точнее их показателей. Здесь, однако, возникает ряд 
сложностей, связанных с~тем, что возможность сочетаемости показателей 
зависит в~первую очередь от морфологической природы показателя ЛСО. Как 
известно, коннекторы относятся к~разнообразным морфологическим классам: 
сочинительные со\-юзы (\textit{и}, \textit{а}, \textit{но}); подчинительные союзы 
(\textit{хотя}, \textit{потому что}, \textit{как}), так называемые 
<<конкретизаторы со\-юзов>>, перешедшие в~класс коннекторов, как правило, из 
наречных выражений (\textit{в~то же время}, \textit{однако}, \textit{впрочем}); 
предлоги (\textit{кроме}, \textit{после}). Союзы, например, как сочинительные, 
так и~подчинительные, не могут сочетаться между собой в~рамках единого 
фрагмента текста, и, наоборот, наибольшей легкостью в~сочетании именно с~союзами обладают <<конкретизаторы>> (\textit{но однако}, \textit{но впрочем}, 
\textit{а~между тем}, \textit{или например}, \textit{и~в~частности}). Если для 
показателей некоторых ЛСО можно выявить закономерности, то другие менее 
избирательны в~своих сочетаниях. Так, показатель ЛСО спецификации 
\textit{например} сочетается со всеми сочинительными союзами, а~показатель 
ЛСО <<вопреки ожидаемому>> \textit{впрочем} только с~союзами~\textit{а} 
и~\textit{но}, т.\,е.\ показателями близких ему (\textit{а}) или тех же (\textit{но}) 
ЛСО. Можно также учитывать двухместные реализации коннекторов, т.\,е.\ 
такие, где компоненты коннектора находятся в~каждом из соединяемых 
фрагментов текста, например \textit{хотя$\ldots$\ но}: \textit{Хотя он меня 
очень уговаривал, но я~не согласился}. Но такие сочетания возможны не для 
всех ЛСО и~сужают круг возможностей для получения адекватных 
количественных данных.
 
  В~связи с~вышесказанным при подсчете ССБ ЛСО этот критерий может 
использоваться лишь как дополнительный.
  
\section{Заключение}

  Из четырех рассмотренных критериев определения ССБ ЛСО: 
(1)~принадлежности различительных признаков ЛСО к~одному семейству, 
(2)~соответствия ЛСО в~оригинальных и~переводных \mbox{текс\-тах}, (3)~возможности 
одного показателя выражать разные ЛСО и~(4)~сочетаемости показателей ЛСО 
в~одном фрагменте текста~--- первые три могут иметь достаточно высокий 
приоритет. Четвертый признак обладает, напротив, наименьшим весом при 
определении ССБ ЛСО. 
  
  Степень детальности разметки, а следовательно, и~определений ЛСО не 
позволяет пока объяснить некоторые явления. Например, семантическую 
близость ЛСО условия и~одновременности, который подтверждается как их 
соответствиями в~оригинальных и~переводных текстах, так и~воз\-мож\-ностью 
выражаться одним показателем (\textit{когда}). Их общий признак <<T$p$ 
включает в~себя T$q$>> не входит в~определение условных ЛСО, так как 
соотношение временн$\acute{\mbox{ы}}$х планов положений вещей~$p$ и~$q$ может быть 
самым различным в~условном периоде. С~другой стороны, при ЛСО 
одновременности различным может быть их семантическое соотношение 
(семантическая независимость, противопоставленность, причина, следствие 
и~т.\,д.). Перевод показателя ЛСО одновременности показателем условных 
ЛСО наблюдается только при одновременной реализации положений 
вещей~$p$ и~$q$ и~при возможности установить между ними отношение 
импликации. Семантическая близость данных двух ЛСО может быть, 
следовательно, установлена на более низком иерархическом уровне, а~именно: 
при определении частных случаев его реализации. В~НБДК такая возможность 
предусмотрена, что позволит в~дальнейшем более детально описывать каждое 
ЛСО и~его виды, а~значит, более точно определить ССБ ЛСО.
{\looseness=1

}
  
{\small\frenchspacing
 {\baselineskip=10.6pt
 %\addcontentsline{toc}{section}{References}
 \begin{thebibliography}{99}
\bibitem{1-in}
\Au{Hobbs J.\,R.} A~computational approach to discourse analysis.~--- 
New York, NY, USA: Department of Computer Science, City College, City University of New 
York, 1976.  Research Report 76-2. P.~28--38.
\bibitem{2-in}
\Au{Hobbs J.\,R.} Why is discourse coherent?~--- Menlo Park, CA, 
USA: SRI International, 1978. SRI Technical Note 176. 44~p.
\bibitem{3-in}
\Au{Halliday M.\,A.\,K., Hasan~R.}  Cohesion in English.~--- London: Longman, 1976. 374~p.


\bibitem{5-in} %4
\Au{Mann W.\,C., Thompson~S.\,A.} Rhetorical structure theory: Towards a functional theory of 
text organization~// Text, 1988. Vol.~8. No.\,3. P.~243--281. doi: 10.1515/text.\linebreak  1.1988.8.3.243.

\bibitem{6-in} %5
\Au{Asher N.} Reference to abstract objects in discourse.~--- Dordrecht: Kluwer, 1993. 455~p.

\bibitem{4-in} %6
\Au{Halliday M.\,A.\,K.} An introduction to functional grammar.~--- 2nd ed.~--- London: 
Edward Arnold, 1994. 434~p.

\bibitem{7-in} %7
PDTB Research Group. The Penn Discourse Treebank 2.0 annotation manual.~--- Philadelphia, PA, USA: Institute for Research in Cognitive Science, University 
of Pennsylvania, 2007.  Technical Report 
IRCS-08-01. 104~p. {\sf https://www.cis.upenn.edu/$\sim$elenimi/\linebreak pdtb-manual.pdf}.
\bibitem{8-in}
\Au{Breindl E., Volodina~A., \mbox{Wa{\!\ptb{\!\ss}}\,ner}~U.\,H.} Handbuch der deutschen 
Konnektoren~2: Semantik der deutschen Satzverkn$\ddot{\mbox{u}}$pfer.~--- Berlin: Walter de Gruyter, 2014. 
1327~p.
\bibitem{9-in}
\Au{Инькова О.\,Ю.} Логико-се\-ман\-ти\-че\-ские отношения: проблемы 
классификации~// Связность текста: мереологические ло\-ги\-ко-се\-ман\-ти\-че\-ские 
отношения.~--- М.: ЯСК, 2019. С.~11--98.
\bibitem{10-in}
\Au{Asher N., Lascarides~A.} Logics of conversation.~--- Cambridge: Cambridge University 
Press, 2003. 526~p.
\bibitem{11-in}
\Au{Carlson L., Marcu D.} Discourse tagging reference manual.~--- Marina del Rey, CA, USA: Information Sciences Institute, University of Southern 
California, 2001.  Technical Report ISI-TR-545. 87~p.



\bibitem{13-in} %12
\Au{Chiarcos Ch.} Towards interoperable discourse annotation: Discourse features in the 
Ontologies of Linguistic Annotation~// 9th Conference (International) on Language Resources 
and Evaluation Proceedings~/ Eds.\ N.~Calzolari, K.~Choukri, T.~Declerck, \textit{et al.}~--- Reykjavik, Iceland: European Language Resources Association 
(ELRA), 2014. P.~4569--4577.

\bibitem{12-in} %13
\Au{Benamara F., Taboada~M.} Mapping different rhetorical relation annotations: A~proposal~// 
4th Joint Conference on Lexical and Computational Semantics  Proceedings~/ Eds.\ M.~Palmer, G.~Boleda, P.~Rosso.~--- Denver, CO, USA: 
Association for Computational Linguistics, 2015. Р.~147--152. doi: 10.18653/v1/S15-1016.

\bibitem{14-in}
\Au{Sanders T., Demberg~V., Hoek~J., Scholman~M., Asr~F.\,T., Zufferey~S., Evers-Vermeul~J.} 
Unifying dimensions in coherence relations: How various annotation frameworks are related~// 
Corpus Linguist. Ling., 2018. Vol.~17. No.\,1. P.~1--71. doi:  
10.1515/cllt-2016-0078.
\bibitem{15-in}
\Au{Инькова О.\,Ю.} Определения дискурсивных отношений: опыт Надкорпусной базы 
данных коннекторов~// Компьютерная лингвистика и~интеллектуальные технологии: По 
мат-лам ежегодной \mbox{Междунар.} конф. <<Диалог>>.~--- М.: РГГУ, 2021. Вып.~20(27). 
С.~328--338.
\bibitem{16-in}
\Au{Инькова О.\,Ю., Кружков М.\,Г.} Структурированные определения дискурсивных 
отношений в~Надкорпусной базе данных коннекторов~// Информатика и~её применения, 
2021. Т.~15. Вып.~4. С.~27--32. doi: 10.14357/19922264210404. EDN: EZJXVI.

\bibitem{17-in}
\Au{Инькова О.\,Ю., Кружков М.\,Г.} Критерии определения семантической близости 
дискурсивных отношений~// Информатика и~её применения, 2023. Т.~17. Вып.~3.  
С.~100--106. doi: 10.14357/19922264230314. EDN: UJZJZI.

\bibitem{18-in}
\Au{Инькова О.\,Ю., Нуриев В.\,А.} Насколько лингвоспецифичен союз \textit{хотя}?~// 
Компьютерная лингвистика и~интеллектуальные технологии: По мат-лам ежегодной 
Междунар. конф. <<Диалог>>.~--- М.: РГГУ, 2018. Вып.~17(24). С.~254--266.

\bibitem{20-in} %19
Словарь современного русского литературного языка: в~17~т.~/ Под ред. 
В.\,И.~Чернышева.~--- М., Л.: Изд-во Академии наук СССР~/ Наука, 1950--1965.

\bibitem{19-in} %20
Русская грамматика~/ Под ред. Н.\,Ю.~Шведовой.~--- М.: Наука, 1980.   Т.~2.
714~с.

\bibitem{21-in}
Словарь русского языка: в~4~т.~/ Под ред. А.\,П.~Ев\-гень\-евой.~--- М.: Русский язык, 
 1981--1984. 
\bibitem{22-in}
\Au{Heine B., Kuteva T.} World lexicon of grammaticalization.~--- Cambridge: Cambridge 
University Press, 2002. 387~p.
\end{thebibliography}

 }
 }

\end{multicols}

\vspace*{-10pt}

\hfill{\small\textit{Поступила в~редакцию 15.10.23}}

\vspace*{8pt}

%\pagebreak

%\newpage

%\vspace*{-28pt}

\hrule

\vspace*{2pt}

\hrule



\def\tit{EVALUATING THE DEGREE OF~DISCOURSE RELATIONS SEMANTIC AFFINITY: 
METHODS AND~INSTRUMENTS}


\def\titkol{Evaluating the degree of~discourse relations semantic affinity: 
Methods and instruments}


\def\aut{O.\,Yu.~Inkova$^{1,2}$ and~M.\,G.~Kruzhkov$^1$}

\def\autkol{O.\,Yu.~Inkova and~M.\,G.~Kruzhkov}

\titel{\tit}{\aut}{\autkol}{\titkol}

\vspace*{-14pt}


\noindent
$^1$Federal Research Center ``Computer Science and Control'' of the Russian Academy of Sciences, 
44-2~Vavilov\linebreak
$\hphantom{^1}$Str., Moscow 119333, Russian Federation

\noindent
$^2$University of Geneva, 22 Bd des Philosophes, CH-1205 Geneva 4, Switzerland


\def\leftfootline{\small{\textbf{\thepage}
\hfill INFORMATIKA I EE PRIMENENIYA~--- INFORMATICS AND
APPLICATIONS\ \ \ 2023\ \ \ volume~17\ \ \ issue\ 4}
}%
 \def\rightfootline{\small{INFORMATIKA I EE PRIMENENIYA~---
INFORMATICS AND APPLICATIONS\ \ \ 2023\ \ \ volume~17\ \ \ issue\ 4
\hfill \textbf{\thepage}}}

\vspace*{3pt}




\Abste{The methods for evaluating semantic affinity of discourse relations are examined. The 
authors propose several approaches to this problem using two information resources: 
a~collection of structured definitions of logical-semantic relations (LSRs) formed by the authors
and the Supracorpora 
Database of Connectives incorporating\linebreak\vspace*{-12pt}}

\Abstend{corpus-based annotations of translation correspondences 
that include text fragments with LSR markers in Russian,
French, and Italian. It is demonstrated that when it comes to 
assessing the semantic affinity of LSRs, the following factors will be of a~higher priority: affiliation of 
distinctive features of LSRs with the same family in the structured definitions of relations; correspondences 
between markers of different LSRs in the source and target texts; and cases when different LSRs are 
regularly expressed by the same markers in different contexts. Of a~lesser importance is the factor of 
compatibility of different LSRs within the same context. It is assumed that based on the proposed 
methods, it will become possible to specify more precisely which distinguishing features of LSRs 
have the greatest impact on their potential semantic affinity.}

\KWE{supracorpora database; logical-semantic relations; connectives; annotation; faceted 
classification}


  \DOI{10.14357/19922264230412}{FXTSPZ}

\vspace*{-16pt}

\Ack

\vspace*{-3pt}

\noindent
The research was carried out using the infrastructure of the Shared Research Facilities ``High 
Performance Computing and Big Data'' (CKP ``Informatics'') of FRC CSC RAS (Moscow).


\vspace*{6pt}

  \begin{multicols}{2}

\renewcommand{\bibname}{\protect\rmfamily References}
%\renewcommand{\bibname}{\large\protect\rm References}

{\small\frenchspacing
 {%\baselineskip=10.8pt
 \addcontentsline{toc}{section}{References}
 \begin{thebibliography}{99}
\bibitem{1-in-1}
\Aue{Hobbs, J.\,R.} 1976. A~computational approach to discourse analyses. New York, NY: 
Department of Computer Science, City College, City University of New York. Research Report  
76-2. 28--38.
\bibitem{2-in-1}
\Aue{Hobbs, J.\,R.} 1978. Why is discourse coherent? Menlo Park, CA: SRI International. SRI 
Technical Note 176. 44~p.
\bibitem{3-in-1}
\Aue{Halliday, M.\,A.\,K., and R.~Hasan.} 1976. \textit{Cohesion in English}. London: Longman. 
374~p.


\bibitem{5-in-1} %4
\Aue{Mann, W.\,C., and S.\,A.~Thompson.} 1988. Rhetorical structure theory: Towards 
a~functional theory of text organization. \textit{Text} 8(3):243--281. doi: 
10.1515/text.1.1988.8.3.243.
\bibitem{6-in-1} %5
\Aue{Asher, N.} 1993. \textit{Reference to abstract objects in discourse}. Dordrecht: Kluwer. 
455~p.
\bibitem{4-in-1} %6
\Aue{Halliday, M.\,A.\,K.} 1994. \textit{An introduction to functional grammar}. 2nd ed. London: 
Edward Arnold. 434~p.

\bibitem{7-in-1}
PDTB Research Group. 2007. The Penn Discourse Treebank 2.0 annotation manual. Philadelphia, 
PA: Institute for Research in Cognitive Science, University of Pennsylvania. Technical Report 
IRCS-08-01. 104~p. Available at: {\sf https://www.cis.upenn.edu/$\sim$elenimi/pdtb-manual.pdf} 
(accessed November~28, 2023).
\bibitem{8-in-1}
\Aue{Breindl, E., A.~Volodina, and U.\,H.~Wa{\!\ptb{\!\ss}}ner.} 2014. \textit{Handbuch der 
deutschen Konnektoren~2: Semantik der deutschen Satzverkn$\ddot{\mbox{u}}$pfer}. Berlin: Walter de Gruyter. 
1327~p.
\bibitem{9-in-1}
\Aue{Inkova, O.\,Yu.} 2019. Logiko-semanticheskie otnosheniya: problemy klassifikatsii  
[Logical-semantic relations: Classification problems]. \textit{Svyaznost' teksta: mereologicheskie 
logiko-semanticheskie otnosheniya} [Text coherence: Mereological logical semantic relations]. 
Moscow: LRC Publishing House. 11--98.
\bibitem{10-in-1}
\Aue{Asher, N., and A.~Lascarides.} 2003. \textit{Logics of conversation}. Cambridge: Cambridge 
University Press. 526~p.
\bibitem{11-in-1}
\Aue{Carlson, L., and D.~Marcu.} 2001. Discourse tagging reference manual.  Marina del Rey, CA: Information Sciences Institute, University of Southern 
California. Technical Report 
ISI-TR-545.  87~p. Available at: {\sf https://www.isi.edu/~marcu/discourse/tagging-ref-manual.pdf} 
(accessed November~28, 2023).

\bibitem{13-in-1} %12
\Aue{Chiarcos, Ch.} 2014. Towards interoperable discourse annotation: Discourse features in the 
Ontologies of Linguistic Annotation. \textit{9th Conference (International) on\linebreak Language Resources 
and Evaluation Proceedings}. Eds. N.~Calzolari, K.~Choukri, T.~Declerck, \textit{et al.} Reykjavik, Iceland: 
European Language Resources Association. 4569--4577.
{ %\looseness=1

}

\bibitem{12-in-1} %13
\Aue{Benamara, F., and M.~Taboada.} 2015. Mapping different rhetorical relation annotations: 
A~proposal. \textit{4th Joint Conference on Lexical and Computational Semantics}. Eds. 
M.~Palmer, G.~Boleda, and P.~Rosso. Denver, CO, USA: Association for Computational 
Linguistics. 147--152. doi: 10.18653/v1/S15-1016.

\bibitem{14-in-1}
\Aue{Sanders, T., V.~Demberg, J.~Hoek, M.~Scholman, F.\,T.~Asr, S.~Zufferey, and  
J.~Evers-Vermeul.} 2018. Unifying dimensions in coherence relations: How various annotation 
frameworks are related. \textit{Corpus Linguist. Ling.} 17(1):1--71. doi: 10.1515/cllt-2016-0078.
\bibitem{15-in-1}
\Aue{Inkova, O.\,Yu.} 2021. Opredeleniya diskursivnykh otnosheniy: opyt Nadkorpusnoy bazy 
dannykh konnektorov [Definition of discursive relations: The experience of the supracorpora 
database of connectors]. \textit{Komp'yuternaya lingvistika i~intellektual'nye Tekhnologii: Po 
mat-lam ezhegodnoy Mezhdunar.  konf. ``Dialog''} [Computational Linguistics 
and Intellectual Technologies: Papers from the Annual Conference (International) ``Dialogue'']. 
Moscow: RGGU. 20(27):328--338.
\bibitem{16-in-1}
\Aue{Inkova, O.\,Yu., and M.\,G.~Kruzhkov.} 2021. Strukturirovannye opredeleniya 
diskursivnykh otnosheniy v~Nadkorpusnoy baze dannykh konnektorov [Structured definitions of 
discourse relations in the Supracorpora Database of Connectives]. \textit{Informatika i~ee 
Primeneniya~--- Inform. Appl.} 15(4):27--32. doi: 10.14357/ 19922264210404. EDN: EZJXVI.
\bibitem{17-in-1}
\Aue{Inkova, O.\,Yu., and M.\,G.~Kruzhkov.} 2023. Kriterii opredeleniya semanticheskoy blizosti 
diskursivnykh otnosheniy [Evaluation criteria for discourse relations semantic affinity]. 
\textit{Informatika i~ee Primeneniya~--- Inform. Appl.} 17(3):100--106. doi: 
10.14357/19922264230314. EDN: UJZJZI.

\pagebreak


\bibitem{18-in-1}
\Aue{Inkova, O.\,Yu., and V.\,A.~Nuriev.} 2018. Naskol'ko lingvospetsifichen soyuz \textit{khotya}? [To 
what extent is the conjunction \textit{khotya} language-specific?]. \textit{Komp'yuternaya lingvistika 
i~intellektual'nye tekhnologii: Po mat-lam ezhegodnoy Mezhdunar. konf. ``Dialog''} 
[Computational Linguistics and Intellectual Technologies: Papers from the Annual Conference 
(International) ``Dialogue'']. Moscow: RGGU. 17(24):254--266. 

\bibitem{20-in-1} %19
Chernyshev, V.\,I., ed. 1950--1965. \textit{Slovar' sovremennogo russkogo literaturnogo yazyka} 
[Dictionary of modern Russian literary language]. In 17~vols. Moscow, Leningrad: USSR Academy 
of Sciences Publishing House/Nauka.

\bibitem{19-in-1} %20
Shvedova, N.\,Yu., ed. 1980. \textit{Russkaya grammatika} [Russian grammar]. Moscow: Nauka. Vol.~2. 714~p.

\bibitem{21-in-1} %21
Evgen'eva, A.\,P., ed. 1981--1984. \textit{Slovar' russkogo yazyka} [Dictionary of the Russian 
language].  Moscow: Russkiy yazyk. 4~vols.


\bibitem{22-in-1}
\Aue{Heine, B., and T.~Kuteva.} 2002. \textit{World lexicon of grammaticalization}. Cambridge: 
Cambridge University Press. 387~p.

\end{thebibliography}

 }
 }

\end{multicols}

\vspace*{-6pt}

\hfill{\small\textit{Received October 5, 2023}} 

%\vspace*{-18pt}

\Contr

\vspace*{-4pt}

\noindent
\textbf{Inkova Olga Yu.} (b.\ 1965)~--- Doctor of Science in philology, senior scientist, Federal 
Research Center ``Computer Science and Control'' of the Russian Academy of Sciences,  
44-2~Vavilov Str., Moscow 119333, Russian Federation; faculty member, University of Geneva, 
22~Bd des Philosophes, CH-1205 Geneva~4, Switzerland; \mbox{olyainkova@yandex.ru}

\vspace*{3pt}

\noindent
\textbf{Kruzhkov Mikhail G.} (b.\ 1975)~--- senior scientist, Federal Research Center ``Computer 
Science and Control'' of the Russian Academy of Sciences, 44-2~Vavilov Str., Moscow 119333, 
Russian Federation; \mbox{magnit75@yandex.ru}


\label{end\stat}

\renewcommand{\bibname}{\protect\rm Литература}     %14
\def\stat{zatsman}

\def\tit{ТРАНСФОРМАЦИИ ОБЪЕКТОВ ПЕРВОГО И~ВТОРОГО ПОРЯДКА 
В~ЛЕКСИКОГРАФИЧЕСКОЙ ИНФОРМАЦИОННОЙ СИСТЕМЕ$^*$}

\def\titkol{Трансформации объектов первого и~второго порядка 
в~лексикографической информационной системе}

\def\aut{И.\,М.~Зацман$^1$}

\def\autkol{И.\,М.~Зацман}

\titel{\tit}{\aut}{\autkol}{\titkol}

\index{Зацман И.\,М.}
\index{Zatsman I.\,M.}


{\renewcommand{\thefootnote}{\fnsymbol{footnote}} \footnotetext[1]
{Исследование выполнено в~ФИЦ ИУ РАН за счет гранта Российского научного фонда №\,24-18-00155, {\sf 
https://rscf.ru/project/24-18-00155}. Работа выполнялась с~использованием инфраструктуры Центра 
коллективного пользования <<Высокопроизводительные вычисления и~большие данные>> (ЦКП 
<<Информатика>>) ФИЦ ИУ РАН (г.\ Москва).}}


\renewcommand{\thefootnote}{\arabic{footnote}}
\footnotetext[1]{ Федеральный исследовательский центр <<Информатика и~управление>> Российской академии наук, 
\mbox{izatsman@yandex.ru}}

\vspace*{-12pt}


  
  \Abst{Рассматриваются теоретические основания проектирования информационных 
технологий (ИТ) интеграции двуязычных словарей и~параллельных корпусов. Дано описание 
первых результатов создания третьего уровня классификации трансформаций объектов 
предметной области информатики, которую предполагается использовать при создании 
концепции лексикографической информационной системы, обеспечивающей интеграцию. 
Все сущности информатики в~статье разделены на два глобальных класса: объекты и~их 
трансформации. Для каждого такого класса конструируется своя классификация. Ранее были 
описаны два верхних уровня классификации трансформаций объектов предметной области. 
В~данной статье рассматривается третий уровень этой классификации. Основанием для 
построения самого верхнего ее уровня служило деление предметной области информатики 
на среды (ментальная, сенсорно воспринимаемая, цифровая и~ряд других сред), каждая из 
которых по определению включает объекты одной природы. Основанием для построения 
второго уровня классификации трансформаций объектов служила типология знаковых  
сис\-тем А.~Соломоника. Цель статьи состоит в~систематизации трансформаций первого 
и~второго порядка объектов предметной области на третьем уровне этой классификации. 
Основанием для систематизации служит средовая версия иерархии Акоффа.}
  
  \KW{объекты предметной области; трансформации объектов; классификация; данные; 
информация; знание; лексикографическая информационная сис\-тема}

\DOI{10.14357/19922264240211}{VZTGVV}
  
\vspace*{3pt}


\vskip 10pt plus 9pt minus 6pt

\thispagestyle{headings}

\begin{multicols}{2}

\label{st\stat}
  
\section{Введение}

\vspace*{-9pt}

  Возникновение параллельных корпусов, в~которых предложениям 
оригинального текста со\-по\-став\-ле\-ны предложения его перевода, обеспечило 
возможность контрастивного лингвистического\linebreak \mbox{анализа} на принципиально 
новом уровне полноты и~точности, недостижимом в~докорпусную эпоху. 
Пионерскими в~этой области стали работы \mbox{1990-х~гг}. Стига Йоханссона  
с~анг\-ло-нор\-веж\-ским корпусом~[1]. В России параллельные корпусы стали 
формироваться в~начале XXI~века в~рамках Национального корпуса русского 
языка~[2].
  
  Создатели двуязычных словарей используют параллельные корпусы для 
сбора материала и~эмпирической проверки своих гипотез, касающихся 
межъязы\-ко\-вой эквивалентности. Ценность параллельных корпусов 
определяется тем, что в~лингвистике этап сбора исходного материала считается 
наиболее трудоемким и~наименее творческим, а~параллельные корпусы 
позволяют значительно сэкономить время и~силы для творческого этапа 
создания словарей~[3].
 % 
  При этом двуязычные словари, создаваемые на основе исходного материала, 
извлеченного из параллельных корпусов, сейчас формируются без связей с~их 
текстами. Другими словами, онлайновые связи созданных словарей 
с~параллельными корпусами, которые служили источниками исходного 
материала, отсутствуют. 

Параллельные корпусы постоянно пополняются 
новыми текстами, в~предложениях которых можно обнаружить новые значения 
слов и~устойчивых словосочетаний. Однако при этом отсутствуют методы 
и~средства оперативного обновления словарей по корпусным данным. 
В~настоящее время проблема установления связей между двуязычными 
словарями и~параллельными корпусами (далее~--- проблема интеграции) 
находится на стадии поиска концептуальных подходов к~их интеграции на 
уровне значений.
  
  Подход к~решению проблемы интеграции, предлагаемый в~статье, учитывает 
  и~появление новых значений слов и~устойчивых словосочетаний, и~динамику 
смысловых значений, которая обусловлена развитием и~пополнением знания 
лингвистов, фиксирующих эти значения в~результате семантического анализа 
пополняемых корпусных данных. Проведенные эксперименты показали, что 
обнаружение нового лингвистического знания обусловливает и~формирование 
дефиниций новых значений, и~пересмотр уже существующих дефиниций~[4, 5].
  
  Например, в~проведенных экспериментах с~использованием ЦКП 
<<Информатика>> ФИЦ ИУ РАН фиксировалась эволюция значений немецких 
модальных глаголов, исходное состояние значений которых было описано 
в~не\-мец\-ко-рус\-ском словаре. В~экспериментальном массиве текстов как 
потенциальных источниках нового знания 16\,268 предложений содержали 
немецкие модальные глаголы и~в~2041 из них встречался глагол sollen. 
В~начале эксперимента в~словаре были описаны~12~значений этого модального 
глагола. По окончании эксперимента лингвисты обнаружили два новых его 
значения, согласовали их дефиниции и~описали эволюцию дефиниций~[6, 7].
  
  Таким образом, для решения проблемы интеграции требуется фиксировать 
новое знание, обнаруженное лингвистами в~текстовых данных параллельных 
корпусов, отслеживать эволюцию знания, представленного в~виде дефиниций 
значений слов и~устойчивых словосочетаний, и,~соответственно, 
актуализировать электронные двуязычные словари. Предлагаемый 
концептуальный подход к~интеграции, который планируется реализовать 
в~процессе проектирования лексикографической информационной сис\-те\-мы, 
фиксирующей эволюцию лингвистического знания, основан на решении 
следующих задач:\\[-14pt]
  \begin{itemize}
  \item категоризация трех базовых понятий информатики, включенных 
  в~иерархию Акоффа~[8] (данные, информация, знание), на объекты 
проектируемой сис\-те\-мы, которая необходима, чтобы фиксировать 
<<кванты>> нового знания и~отслеживать его эволюцию в~этой сис\-теме;\\[-15pt]
  \item  систематизация трансформаций объектов этой сис\-темы.\\[-14pt]
  \end{itemize}
  
  Цель статьи и~состоит в~решении двух задач: категоризации трех базовых 
понятий информатики на объекты лексикографической информационной  
сис\-те\-мы и~сис\-те\-ма\-ти\-за\-ции трансформаций первого и~второго порядка 
ее объектов.
  
  Трансформациями первого порядка, о которых сказано в~формулировке цели 
статьи, называются взаимные преобразования между двумя объектами  
сис\-те\-мы одной природы. Например, перевод в~сис\-те\-ме текста с~русского 
языка на английский относится к~ним. Трансформациями второго порядка 
и~выше называются взаимные преобразования между двумя и~более объектами 
разной природы. Например, кодирование символов текс\-та компьютерными 
кодами и~их декодирование относятся по определению к~трансформациям 
второго порядка.

%\vspace*{-9pt}
  
\section{Процессы трансформаций в~информатике}

%\vspace*{-3pt}

Процессы трансформаций, рассматриваемые в~статье, относятся к~теоретическому ядру информатики, а~не 
только к~проектированию лексикографической информационной сис\-те\-мы. Например, из трех основных 
подходов к~описанию предметной об\-ласти информатики\footnote{В статье предметная область информатики 
трактуется согласно концепции полиадического компьютинга Пола Розенблума~\cite{9-zac}.} (объектный, 
трансформационный и~синтетический) сис\-те\-ма\-ти\-за\-ция трансформаций ближе всего ко второму 
подходу. Примерами первого подхода, в~рамках которого основное внимание уделяется объектам предметной 
области информатики и~в~меньшей степени отношениям\linebreak между ними, могут служить  
работы~\cite{8-zac, 10-zac, 11-zac}; \mbox{примерами} второго подхода, в~рамках которого основное внимание 
уделяется трансформациям и~в~меньшей степени трансформируемым объектам,~---  
работы~\cite{12-zac, 13-zac}; примерами третьего, синтетического подхода, в~котором уделяется внимание 
и~объектам предметной об\-ласти информатики, и~отношениям между ними, могут служить работы~\cite{14-zac, 
15-zac, 16-zac, 17-zac, 18-zac}.

  Таким образом, для описания трансформаций объектов лексикографической 
информационной\linebreak системы предпочтительнее всего трансформационный 
подход, который упоминается и~в определениях информатики. Например, 
в~2009~г.\ П.~Деннинг и~П.~Розенблум сформулировали суть \mbox{информатики} как 
компьютинга следующим образом: <<$\ldots$информатика~--- это не просто 
алгоритмы и~структуры данных; это преобразования [трансформации] 
представлений>>~\cite{12-zac}. Чуть позже, в~контексте краткого описания 
парадигмы информатики как компьютинга, П.~Деннинг и~П.~Фриман изменили 
эту формулировку на такую: <<Центральный объект внимания в~информатике 
можно определить как информационные процессы~--- \textit{естественные или 
искусственные процессы, преобразующие информацию} (курсив мой~--- 
И.\,З.)>>~\cite{13-zac}. Согласно парадигме, предлагаемой авторами этой 
статьи, на начальном этапе проектирования автоматизированных систем 
базовыми элементами моделей их функционирования служат 
\textit{информационные про\-цессы}.
  
  Однако если 15~лет назад в~формулировке из работы~\cite{13-zac} шла речь 
о~процессах, преобразующих информацию, то в~последние~10~лет в~спектр 
процессов трансформаций все чаще стали включать процессы, преобразующие 
не только информацию, но также и~другие объекты автоматизированных 
систем, в~первую очередь данные и~знания~[19--21]. Например, Виктория 
Стодден, позиционируя науку о~данных как одну из дисциплин информатики, 
говорит, что центральный объект исследований в~науке о~данных~--- это 
<<изучение обобщаемого извлечения знания из данных>>~\cite{21-zac}. 
Увеличение и~чис\-ла объектов, и~спект\-ра процессов их трансформаций 
в~автоматизированных сис\-те\-мах обуслов\-ли\-ва\-ет не\-об\-хо\-ди\-мость 
систематизации и~объектов, и~процессов их трансформаций на начальном этапе 
проектирования сис\-тем.
  
  Для создания концепции лексикографической информационной сис\-те\-мы 
и~проектирования ИТ, обеспечивающих интеграцию 
двуязычных словарей и~параллельных корпусов, сначала выполним 
категоризацию на объекты этой сис\-те\-мы трех базовых понятий информатики 
(данные, информация, знание) в~контексте построения классификаций 
сущностей ее предметной об\-ласти.
  
  Необходимость использования классификаций информатики в~процессе 
создания концепции проиллюстрируем, используя иерархию  
Акоффа~\cite{8-zac}. Он использовал принцип их вертикального размещения 
в~иерархии снизу вверх: данные, информация и~знание. Еще в~ней есть термин 
<<мудрость>>, который в~статье не рассматривается. Такое размещение Акофф 
прокомментировал так: <<Каждое из пе\-ре\-чис\-лен\-ных понятий [кроме данных] 
содержит в~себе нижестоящие$\ldots$>>~\cite{8-zac}.
  
  Этому принципу размещения и~комментарию Акоффа свойственны 
недостатки, проанализированные, в~частности, в~работе~\cite{10-zac}. Главный 
вывод, к~которому пришла Роули после изучения иерархии Акоффа, 
заключается в~следующем: <<$\ldots$информация определяется в~терминах 
данных, знание~--- в~терминах информации$\ldots$ но существует меньше 
консенсуса в~описании трансформаций, которые преобразуют сущности, 
расположенные ниже в~иерархии, в~те, которые находятся над ними, что 
приводит к~их терминологической неопределенности>>~\cite{10-zac}. Причина 
этой неопределенности, скорее всего, в~том, что базовые понятия информатики 
включены в~иерархию Акоффа изолированно от общего контекста 
классификаций сущностей ее предметной об\-ласти.

%\vspace*{-9pt}
  
\section{Классификации сущностей информатики}


%\vspace*{-2pt}

  Все сущности предметной области информатики в~работах~[22, 23] 
разделены на два глобальных класса: ее объекты и~их трансформации. Для 
каждого такого класса была предложена своя классификация. 
В~работе~\cite{22-zac} дано описание классификации объектов предметной 
области информатики, первый уровень которой содержит базовые понятия ее 
предметной области (данные, информация, знания и~др.).  
В~работе~\cite{23-zac} дано описание двух верхних уровней классификации 
трансформаций объектов предметной об\-ласти (см.\ рисунок 
в~работе~\cite{23-zac}). Основанием для построения самого верхнего ее уровня послужило деление 
предметной области информатики на среды\footnote{В~работе~\cite{24-zac} дано описание пяти сред 
предметной области информатики (ментальная; сенсорно воспринимаемая, или информационная; 
цифровая; нейро- и~ДНК-среда), каждая из которых по определению включает объекты одной и~той же 
природы.} и~степень разнообразия природы объектов, вовлеченных в~трансформации:
\begin{itemize}
\item  первый класс верхнего уровня классификации включает 
трансформации объектов в~пределах среды только одной природы 
(трансформации первого порядка);
\item  второй класс включает трансформации объектов, относящихся 
к~двум средам разной природы (трансформации второго порядка);
\item третий и~последующие классы включают трансформации объектов, 
относящихся к~трем и~более средам разной природы (трансформации 
третьего и~более высоких порядков).
\end{itemize}

  В работе~\cite{23-zac} были приведены примеры для трех первых классов 
трансформаций, включая пример трансформаций объектов, относящихся 
к~двум средам разной природы (компьютерное кодирование символов текстов 
с~по\-мощью таб\-лиц Unicode).
  
Основанием для построения второго уровня классификации трансформаций объектов послужила типология 
знаковых сис\-тем А.~Соломоника~\cite[c.~131]{25-zac}: естественные знаковые сис\-те\-мы, образные,  
ес\-тест\-вен\-но-язы\-ко\-в$\acute{\mbox{ы}}$е,  
вер\-баль\-но-не\-сло\-вес\-ные сис\-те\-мы записи\footnote{Под системой записи понимается знаковая 
система, сочетающая вербальные знаки с~несловесными (языки нотной записи, карт, таблиц и~др.).} 
и~формализованные знаковые сис\-те\-мы, включая математические. Введем понятие обобщенного текста~--- 
это текст, который может быть создан в~любой из перечисленных знаковых систем. Тогда обобщенные тексты 
могут быть естественными, образными, ес\-тест\-вен\-но-язы\-ко\-в$\acute{\mbox{ы}}$\-ми,  
вер\-баль\-но-не\-сло\-вес\-ны\-ми и~формализованными. Второй уровень классификации трансформаций 
охватывает не все виды объектов предметной  
об\-ласти информатики, а~только перечисленные~5~видов текс\-тов и~их представления, вовлеченные 
в~процессы трансформаций в~одной или более средах вместе с~данными, знанием и~его концептами.

\begin{figure*}[b] %fig1
\vspace*{6pt}
      \begin{center}
     \mbox{%
\epsfxsize=121.191mm 
\epsfbox{zac-1.eps}
}
\end{center}
\vspace*{-6pt}
\Caption{Средовая версия иерархии Акоффа}
\end{figure*}

\section{Классификация трансформаций: построение~третьего 
уровня}

  Основанием для систематизации трансформаций первого и~второго порядка 
на третьем уровне этой классификации служит иерархия Акоффа~\cite{8-zac}, 
на основе которой и~была создана ее средов$\acute{\mbox{а}}$я версия~[26, 
27]. Для создания средов$\acute{\mbox{о}}$й версии была выполнена 
категоризация трех базовых понятий информатики (данные, информация, 
знания) на объекты лексикографической информационной сис\-те\-мы 
в~процессе создания ее концепции\linebreak (рис.~1).
  


  В отличие от классической иерархии Акоффа, в~ее 
средов$\acute{\mbox{о}}$й версии различаются три вида данных: сенсорно 
воспринимаемые, цифровые и~те данные, которые генерируются 
искусственными нейронными сетями (ИНС) в~системах искусственного интеллекта 
(далее~--- ИИ-дан\-ные). Последний вид данных необходим, например, для 
различения входа и~выхода процесса применения обученной 
ИНС в~цифровой модели генерации знания, описанию которой 
посвящена работа~\cite{27-zac}.
  
  Также предлагается различать два вида информации: сенсорно 
воспринимаемая и~цифровая. Кроме знания в~средов$\acute{\mbox{у}}$ю 
версию добавлены концепты и~ментальные образы сенсорно воспринимаемых 
данных. Последние служат промежуточной сущностью между сенсорно 
воспринимаемыми данными и~генерируемым знанием при описании процессов 
извлечения знания из текстовых данных лексикографической информационной 
системы. Описание объектов средов$\acute{\mbox{о}}$й версии иерархии 
Акоффа (см.\ рис.~1) и~отношений между ними дано в~работах~\cite{26-zac, 28-zac}.
  
  В средов$\acute{\mbox{о}}$й версии число объектов равно восьми. Если 
учитывать направления трансформаций, то между восемью объектами на 
рис.~1 она включает~16 их видов (трансформации на границе между сенсорно 
воспринимаемыми данными и~информацией, обозначенные символом~<<?>>, 
в~статье не рас\-смат\-ри\-ва\-ют\-ся). В~будущем число объектов 
в~средов$\acute{\mbox{о}}$й версии, которая выбрана как основание для 
сис\-те\-ма\-ти\-за\-ции трансформаций первого и~второго порядка, может быть 
увеличено. Для построения классификации трансформаций 
важ\-но не возможное увеличение числа объектов 
и~трансформаций между ними, а то, что их виды в~средов$\acute{\mbox{о}}$й 
версии распределены между трансформациями первого и~второго порядка. Из 
16~видов на рис.~1 шесть относятся к~трансформациям первого порядка, это\linebreak 
виды с~номерами~7, 8, 13--16 (далее~--- типология трансформаций первого 
порядка), а~десять~--- к~трансформациям второго порядка, это виды 
с~\mbox{номерами}~1--6 и~9--12 (далее~--- типология трансформаций второго 
порядка). Разместим обе типологии на третьем уровне классификации (см.\ ее 
схему на рис.~2). Перечислим виды трансформаций первой типологии, вводя 
в~скобках их краткие названия, используемые ниже на рис.~3:
  \begin{description}
  \item[\,] 7~--- членение знания на концепты с~помощью одной или нескольких 
знаковых систем (далее~--- членение знания);
  \item[\,] 8~--- формирование знания на основе концептов (формирование 
знания);
  \item[\,] 13~--- обучение ИНС;
  \end{description}
  
  \vspace*{-6pt}
  
  \pagebreak
  
  \end{multicols}
  
  \begin{figure*} %fig2
\vspace*{1pt}
      \begin{center}
     \mbox{%
\epsfxsize=127.513mm 
\epsfbox{zac-2.eps}
}
\end{center}
\vspace*{-9pt}
\Caption{Схема трех верхних уровней классификации трансформаций объектов (объединены 
по три слоя и~для второго, и~для третьего уровней этой классификации)}
\end{figure*}
  
  \begin{multicols}{2}
  
  \noindent
  \begin{description}
  \item[\,] 14~--- восстановление обучающей информации на основе 
содержания обученной ИНС (обращение ИНС);
  \item[\,] 15~--- использование обученной ИНС (использование ИНС);



  \item[\,] 16~--- восстановление исходных данных, соответствующих 
полученным результатам работы обучен\-ной ИНС (восстановление исходных данных 
по результатам ИНС).
  \end{description}
  
  
  Не все виды трансформаций 13--16 поддерживаются в~конкретных системах 
искусственного интеллекта, но с~теоретической точки зрения все их 
предлагается включить в~первую типологию для полноты спектра видов 
трансформаций.
  
  Перечислим виды трансформаций второй типологии:
  \begin{description}
  \item[\,] 1~--- декодирование цифровых данных в~компьютерных системах 
(декодирование данных);
  \item[\,]  2~--- кодирование сенсорно воспринимаемых данных (кодирование 
данных);
  \item[\,] 3~--- ментальное копирование сенсорно воспринимаемых данных 
(ментальное копирование);
  \item[\,] 4~--- восстановление сенсорно воспринимаемых данных по 
ментальным образам (восстановление по образам);
  \item[\,] 5~--- смысловая интерпретация без деления на концепты ментальных 
образов сенсорно воспринимаемых данных (смысловая интерпретация);
  \item[\,] 6~--- восстановление ментальных образов (восстановление образов);
  \item[\,] 9~--- представление концептов в~виде сенсорно воспринимаемой 
информации, например текс\-та\-ми, формулами, таблицами, рисунками и~т.\,д.\ 
(представление концептов);
  \item[\,] 10~--- понимание смысла сенсорно воспринимаемой информации 
(понимание смысла);
  \item[\,] 11~--- кодирование сенсорно воспринимаемой информации 
(кодирование информации);
\end{description}

\vspace*{-6pt}

\pagebreak

\end{multicols}

\begin{figure*} %fig3
\vspace*{1pt}
      \begin{center}
     \mbox{%
\epsfxsize=163mm 
\epsfbox{zac-3.eps}
}
\end{center}
\vspace*{-9pt}
\Caption{Схема частного случая классификации трансформаций объектов (трансформации 
пронумерованы согласно рис.~1)}
\end{figure*}

\begin{multicols}{2}

\noindent
\begin{description}

  \item[\,] 12~--- декодирование цифровой информации (декодирование 
информации).
  \end{description}
  
  Отметим, что в~существующих ИТ
  и~компьютерных системах наиболее часто используются виды 
трансформаций~13 и~15 типологии первого порядка и~1, 2, 11 и~12 типологии 
второго порядка. На рис.~2 в~первом слое третьего уровня классификации 
показаны типологии первого порядка без указания числа трансформаций в~них 
и~без детализации трансформируемых объектов.
  
  Во втором слое третьего уровня классификации условно (без названий) 
показаны типологии второго порядка. Также на рис.~2 в~третьем слое третьего 
уровня классификации условно (также без названий) показаны типологии 
третьего порядка, которые планируется рассмотреть в~отдельной статье. По 
определению они должны включать трансформации между тремя объектами 
разной природы, но средов$\acute{\mbox{а}}$я версия иерархии Акоффа 
включает трансформации только между двумя объектами разной природы. 
Поэтому потребуется другое основание для их систематизации (ранее были 
рассмотрены отдельные примеры трансформаций третьего 
порядка\footnote{Далеко не всегда трансформации третьего и~более высоких порядков можно 
рассматривать как последовательность трансформаций второго порядка. Примером этого могут 
служить трансформации в~процессе обучения пациента пользованию роботизированной рукой, 
охватывающие личностные концепты пациента, релевантные его намерениям, сигналы активности 
мозга как объекты нейросреды и~компьютерные коды~\cite{29-zac}.}~\cite{29-zac}).

\section{Классификация трансформаций: частный~случай}

  Выше было отмечено, что в~будущем число объектов 
в~средов$\acute{\mbox{о}}$й версии иерархии Акоффа может быть увеличено. 
Это означает, что увеличатся и~чис\-ло объектов, и~чис\-ло трансформаций между 
ними в~классификации трансформаций, так как эта средов$\acute{\mbox{а}}$я 
версия служит по определению основанием для систематизации 
трансформаций первого и~второго порядка. Поэтому на третьем уровне рис.~2 
указаны типологии без детализации объектов и~без указания числа 
трансформаций в~каждой из них. С~одной стороны, при таком подходе 
получаем достаточно общий вид этой классификации, так как она не зависит от 
числа объектов в~том или ином варианте средов$\acute{\mbox{о}}$й версии 
(и~это существенно упрощает рис.~2). С~другой стороны, на третьем уровне 
такой общей классификации подразумевается, но не эксплицируется природа 
трансформируемых объектов и~их возможные сочетания в~трансформациях. 

При проектировании лексикографической информационной системы важно 
эксплицировать природу трансформируемых объектов и~их возможные 
сочетания.
  %
  Поэтому в~парадигму информатики~\cite{30-zac} кроме общей 
классификации трансформаций предлагается включать и~ее частные случаи, 
эксплицирующие природу трансформируемых объектов. 

В~этом разделе 
рассмотрим один частный случай, когда используются только естественные 
знаковые сис\-те\-мы из типологии А.~Соломоника~\cite{25-zac} вместе 
с~данными, знанием и~его концептами. Чис\-ло естественных языков при этом не 
ограничено. И~этот частный случай классификации включает только три 
класса природных трансформаций (первого, второго и~третьего порядка, см.\ 
схему классификации на рис.~3).
  
  Первый и~второй уровни схемы общей классификации (см.\ рис.~2) можно 
объединить в~один уровень в~этом частном случае. Ниже этого уровня 
приведено содержание типологий первого и~второго порядка без содержания 
типологий третьего по\-рядка.




  Наполнение типологий первого и~второго порядка соответствует 
средов$\acute{\mbox{о}}$й версии иерархии Акоффа на рис.~1, содержащей 
6~видов трансформаций типологии первого порядка и~10~видов 
трансформаций типологии второго порядка (на рис.~3 стрелки указывают 
направления трансформаций согласно средов$\acute{\mbox{о}}$й версии на рис.~1).
  
  Таким образом, частный случай классификации содержит для этих двух 
типологий 16~теоретически возможных трансформаций, 6 из которых 
в~настоящее время в~существующих ИТ применяются наиболее часто: виды 
трансформаций~1, 2, 11 и~12 типологии второго порядка реализуются 
с~помощью тех или иных методов ко\-ди\-ро\-ва\-ния/де\-ко\-ди\-ро\-ва\-ния 
(например, с~использованием таблиц Unicode), а~виды трансформаций~13 и~15
 в~типологии первого порядка реализуются полностью с~по\-мощью процессов 
цифровой обработки компьютерами.
  
  Остальные виды трансформаций или применяются намного реже (это 
виды~3, 5, 7, 9 и~10), или находятся в~стадии поиска и~разработки (14 и~16) или 
в~настоящее время носят только теоретический характер, обеспечивая полноту 
первой и~второй типологий (4, 6 и~8). Знаком~<<?>> обозначены те виды 
трансформаций, которые по определению не существуют в~используемой 
парадигме информатики~\cite{30-zac}. Однако возможно, что в~других 
будущих подходах к~построению ее парадигмы эти виды трансформаций будут 
существовать.
  
\section{Заключение}

  На сегодняшний день процесс построения классификаций объектов 
предметной области информатики~\cite{22-zac} и~их  
трансформаций~\cite{23-zac} еще не завершен. Однако первые результаты их 
построения уже используются для создания концепции лексикографической 
информационной сис\-те\-мы, обеспечивающей интеграцию двуязычных 
словарей и~параллельных корпусов.
  
  \bigskip
  
  
  Автор признателен рецензентам за помощь в~улучшении статьи.
  
{\small\frenchspacing
 { %\baselineskip=10.6pt
 %\addcontentsline{toc}{section}{References}
 \begin{thebibliography}{99}
\bibitem{1-zac}
\Au{Aijmer K., Altenberg~B.} Advances in corpus-based contrastive linguistics. Studies in honour 
of Stig Johansson.~--- Amsterdam: John Benjamins, 2013. 295~p.  doi: 10.1075/scl.54.
\bibitem{2-zac}
\Au{Добровольский Д.\,О., Кретов~А.\, А., Шаров~С.\,А.} Корпус параллельных текстов~// 
Научная и~техническая информация. Сер.~2: Информационные процессы и~сис\-те\-мы, 2005. 
№\,6. С.~16--27.
\bibitem{3-zac}
\Au{Добровольский Д.\,О.} Корпус параллельных текстов и~сопоставительная 
лексикология~// Труды Института русского языка им.\ В.\,В.~Виноградова, 2015. №\,6. 
С.~413--449. EDN: VJQBHP.
\bibitem{4-zac}
\Au{Гончаров А.\,А., Зацман~И.\,М., Кружков~М.\,Г.} Эволюция классификаций 
в~надкорпусных базах данных~// Информатика и~её применения, 2020. Т.~14. Вып.~4. 
С.~108--116. doi: 10.14357/19922264200415.  
EDN: \mbox{GKWBZT}.
\bibitem{5-zac}
\Au{Гончаров А.\, А., Зацман И. \,М., Кружков~М.\, Г}. Представление новых 
лексикографических знаний в~динамических классификационных сис\-те\-мах~// 
Информатика и~её применения, 2021. Т.~15. Вып.~1. С.~86--93.  doi: 10.14357/19922264210112. EDN: OPEFXW.
\bibitem{6-zac}
\Au{Zatsman I.} Finding and filling lacunas in linguistic typologies~// 15th Forum (International) 
on Knowledge Asset Dynamics Proceedings.~--- Matera, Italy: Institute of Knowledge Asset 
Management, 2020. P.~780--793.
\bibitem{7-zac}
\Au{Zatsman I.} Three-dimensional encoding of emerging meanings in AI-systems~// 21st 
European Conference on Knowledge Management Proceedings.~--- Reading, U.K.: Academic 
Publishing International Ltd., 2020. P.~878--887.
\bibitem{8-zac}
\Au{Ackoff R.} From data to wisdom~// J.~Applied Systems Analysis, 1989. Vol.~16. No.\,1. P.~3--9.
\bibitem{9-zac}
\Au{Rosenbloom P.\,S.} On computing: The fourth great scientific domain.~--- Cambridge, MA, 
USA: MIT Press, 2013. 307~p.
\bibitem{10-zac}
\Au{Rowley J.} The wisdom hierarchy: Representations of the DIKW hierarchy~// J.~Inf. 
Sci., 2007. Vol.~33. Iss.~2. P.~163--180. doi: 10.1177/0165551506070706.
\bibitem{11-zac} 
\Au{Frick$\acute{\mbox{e}}$~M.\,H.} Data--Information--Knowledge--Wisdom (DIKW) pyramid, 
framework, continuum~// Encyclopedia of big data~/ Eds. L.~Schintler, C.~McNeely.~--- Cham: 
Springer, 2018. 4~p. doi: 10.1007/978-3-319-32001-4\_331-1.
\bibitem{12-zac}
\Au{Denning P., Rosenbloom~P.} Computing: The fourth great domain of science~// Commun. 
ACM, 2009. Vol.~52. Iss.~9. P.~27--29.
\bibitem{13-zac}
\Au{Denning P., Freeman~P.} Computing's paradigm~// Commun.  ACM, 2009. Vol.~52. 
Iss.~12. P.~28--30. doi: 10.1145/ 1610252.1610265.
\bibitem{17-zac} %14
\Au{Farradane J.} Knowledge, information, and information science~// J.~Inf. Sci., 
1980. Vol.~2. Iss.~2. P.~75--80. doi: 10.1177/01655515800020020.

\bibitem{15-zac}
\Au{Шрейдер Ю.\,А.} Информация и~знание~// Сис\-тем\-ная концепция информационных 
процессов.~--- М.: ВНИИСИ, 1988. С.~47--52.
\bibitem{16-zac}
\Au{Ingwersen P.} Information and information science~// Enclyclopaedie of library and 
information science~/ Eds. J.\,D.~McDonald, 
M.~Levine-Clark.~--- New York, NY, USA: Marcel Dekker Inc., 1992. Vol.~56. Sup.~19. 
P.~137--174.

\bibitem{14-zac} %17
Информатика как наука об информации: Информационный, документальный, 
технологический, экономический, социальный и~организационный аспекты~/ Под ред. 
Р.\,С.~Гиляревского.~--- М.: Фаир-Пресс, 2006. 592~с.

\bibitem{18-zac}
\Au{Hjorland B.} Library and information science: practice, theory, and philosophical basis~// 
Inform. Process. Manag., 2000. Vol.~36. Iss.~3. P.~501--531. doi:  
10.1016/S0306-\mbox{4573(99)00038-2}.
\bibitem{19-zac}
Deep shift~--- technology tipping points and societal impact.~--- Geneva: WE Forum, 2015. 44~p. 
{\sf http://www3.weforum.org/docs/WEF\_GAC15\_ Technological\_Tipping\_Points\_report\_2015.pdf}.
\bibitem{20-zac}
\Au{Berman F., Rutenbar~R., Hailpern~B., Christensen~H., Davidson~S., Estrin~D., 
Franklin~M., Martonosi~M., Raghavan~P., Stodden~V., Szalay~A.\,S.} Realizing the potential of 
data science~// Commun.  ACM, 2018. Vol.~61. Iss.~4. P.~67--72. doi: 10.1145/3188721.

\bibitem{21-zac}
\Au{Stodden V.} The data science life cycle: A~disciplined approach to advancing data science as 
a~science~// Commun.  ACM, 2020. Vol.~63. Iss.~7. P.~58--66. doi: 10.1145/ 3360646.


\bibitem{23-zac} %22
\Au{Зацман И.\,М.} Научная парадигма информатики: классификация трансформаций 
объектов предметной об\-ласти~// Системы и~средства информатики, 2023. Т.~33. №\,4. 
С.~126--138. doi: 10.14357/08696527230412. EDN: ZIKUWO.

\bibitem{22-zac} %23
\Au{Зацман И.\,М.} Научная парадигма информатики: классификация объектов предметной  
об\-ласти~// Информатика и~её применения, 2023. Т.~17. Вып.~4. С.~96--103. doi: 
10.14357/19922264230413. EDN: FIUQAT.

\bibitem{24-zac}
\Au{Зацман И.\,М.} О~научной парадигме информатики: верхний уровень классификации 
объектов ее предметной об\-ласти~// Информатика и~её применения, 2022. Т.~16. Вып.~4. 
С.~73--79. doi: 10.14357/ 19922264220411. EDN: XZNKVI.

\bibitem{25-zac}
\Au{Соломоник А.\,Б.} Философия знаковых систем и~язык.~--- М.: ЛКИ, 2011. 408~с.
\bibitem{26-zac}
\Au{Зацман И.\,М.} Трансформация иерархии Акоффа в~научной парадигме информатики~// 
Информатика и~её применения, 2023. Т.~17. Вып.~3. С.~107--113. doi: 
10.14357/19922264230315. EDN: UMVRRV.

\bibitem{27-zac}
\Au{Zatsman I.} Building digital spiral models of knowledge generation~// 19th Forum 
(International) on Knowledge Asset Dynamics Proceedings.~--- Matera, Italy: Arts for Business 
Institute, 2024. P.~2185--2196.
\bibitem{28-zac}
\Au{Zatsman I.} Digital spiral model of knowledge creation and encoding its dynamics~// 18th 
Forum (International) on Knowledge Asset Dynamics Proceedings.~--- Matera, Italy: Arts for 
Business Institute, 2023. P.~581--596.
\bibitem{29-zac}
\Au{Зацман И.\,М.} Интерфейсы третьего порядка в~информатике~// Информатика и~её 
применения, 2019. Т.~13. Вып.~3. С.~82--89. doi: 10.14357/19922264190312. EDN: 
EHRQLF.

\bibitem{30-zac}
\Au{Зацман И.\,М.} Научная парадигма информатики как третьей культуры~//  
На\-уч\-но-тех\-ни\-че\-ская информация. Сер.~1: Организация и~методика информационной 
работы, 2023. №\,11. С.~1--14.

\end{thebibliography}

 }
 }

\end{multicols}

\vspace*{-9pt}

\hfill{\small\textit{Поступила в~редакцию 14.04.24}}

\vspace*{4pt}

%\pagebreak

%\newpage

%\vspace*{-28pt}

\hrule

\vspace*{2pt}

\hrule



\def\tit{OBJECT TRANSFORMATIONS OF~THE~FIRST AND~SECOND ORDER
IN~A~LEXICOGRAPHIC INFORMATION SYSTEM\\[-5pt]}


\def\titkol{Object transformations of~the~first and~second order
in~a~lexicographic information system}


\def\aut{I.\,M.~Zatsman}

\def\autkol{I.\,M.~Zatsman}

\titel{\tit}{\aut}{\autkol}{\titkol}

\vspace*{-13pt}


\noindent
Federal Research Center ``Computer Science and Control'' of the Russian Academy of Sciences, 
44-2~Vavilov Str., Moscow 119133, Russian Federation


\def\leftfootline{\small{\textbf{\thepage}
\hfill INFORMATIKA I EE PRIMENENIYA~--- INFORMATICS AND
APPLICATIONS\ \ \ 2024\ \ \ volume~18\ \ \ issue\ 2}
}%
 \def\rightfootline{\small{INFORMATIKA I EE PRIMENENIYA~---
INFORMATICS AND APPLICATIONS\ \ \ 2024\ \ \ volume~18\ \ \ issue\ 2
\hfill \textbf{\thepage}}}

\vspace*{2pt}



\Abste{The theoretical foundations of the design of information technologies used for 
the integration of bilingual dictionaries and parallel corpora are considered. The 
description of the first outcomes of the creation of the third\linebreak\vspace*{-12pt}}

\Abstend{ level of object 
transformations classification in the subject domain of informatics, which is supposed 
to be used
in creating the lexicographic information system providing integration, is 
given. All the entities of informatics are divided into two global classes: objects and 
their transformations. For each such class, its own classification is constructed. 
Previously, the two upper levels of the object transformation classification in the subject 
domain have been described. The present paper discusses the third level of this classification. The 
basis for the construction of its highest level was the division of the subject domain of 
informatics into media (mental, sensory, digital, and a~number of other media), each 
of which by definition includes objects of the same nature. The Solomonick's 
typology of sign systems served as the basis for constructing the second level of the 
object transformation classification. The aim of the paper is to systematize object 
transformations of the first and second orders at the third level of this classification. 
The basis for systematization is the medium version of the Ackoff's hierarchy.}

\KWE{subject domain objects; object transformations; classification; data; 
information; knowledge; lexicographic information system}


\DOI{10.14357/19922264240211}{VZTGVV}

\vspace*{-12pt}

\Ack

\vspace*{-3pt}


\noindent
The reported study was funded by the Russian Science Foundation, project  
No.\,24-18-00155, {\sf 
https://rscf.ru/project/24-18-00155}. The research was carried out using the infrastructure of the Shared 
Research Facilities ``High Performance Computing and Big Data'' (CKP 
``Informatics'') of FRC CSC RAS (Moscow) .
   


  \begin{multicols}{2}

\renewcommand{\bibname}{\protect\rmfamily References}
%\renewcommand{\bibname}{\large\protect\rm References}

{\small\frenchspacing
 {%\baselineskip=10.8pt
 \addcontentsline{toc}{section}{References}
 \begin{thebibliography}{99} 
\bibitem{1-zac-1}
\Aue{Aijmer, K., and B.~Altenberg.} 2013. \textit{Advances in corpus-based 
contrastive linguistics. Studies in honour of Stig Johansson}. Amsterdam: John 
Benjamins. 295~p. doi: 10.1075/scl.54.
\bibitem{2-zac-1}
\Aue{Dobrovolskiy, D.\,O., A.\,A.~Kretov, and S.\,A.~Sharov.} 2005. Korpus 
parallel'nykh tekstov [Corpus of parallel texts]. \textit{Nauchnaya i~tekhnicheskaya 
informatsiya. Ser. 2. Informatsionnye protsessy i~sistemy} [Scientific and Technical 
Information. Ser.~2: Information Processes and Systems] 6:16--27.
\bibitem{3-zac-1}
\Aue{Dobrovolskiy, D.\,O.} 2015. Korpus parallel'nykh tekstov i~sopostavitel'naya 
leksikologiya [The corpus of parallel texts and contrastive lexicology]. \textit{Trudy 
Instituta russkogo yazyka im. V.\,V.~Vinogradova} [Proceedings of the 
V.\,V.~Vinogradov Russian Language Institute] 6:413--449. EDN: VJQBHP.
\bibitem{4-zac-1}
\Aue{Goncharov, A.\,A., I.\,M.~Zatsman, and M.\,G.~Kruzhkov.} 2020. Evolyutsiya 
klassifikatsiy v~nadkorpusnykh ba\-zakh dannykh [Evolution of classifications in 
supracorpora databases]. \textit{Informatika i~ee Primeneniya~--- Inform. \mbox{Appl.}}  
14(4):108--116. doi: 10.14357/19922264200415.  
EDN: GKWBZT.
\bibitem{5-zac-1}
\Aue{Goncharov, A.\,A., I.\,M.~Zatsman, and M.\,G.~Kruzhkov.} 2021. 
Predstavlenie novykh leksikograficheskikh znaniy v~dinamicheskikh 
klassifikatsionnykh sistemakh [Representation of new lexicographical knowledge in 
dynamic classification systems]. \textit{Informatika i~ee Primeneniya~--- Inform. 
Appl.} 15(1):86--93. doi: 10.14357/19922264210112. EDN: OPEFXW.
\bibitem{6-zac-1}
\Aue{Zatsman, I.} 2020. Finding and filling lacunas in linguistic typologies. 
\textit{15th Forum (International) on Knowledge Asset Dynamics Proceedings}. 
Matera, Italy: Institute of Knowledge Asset Management. 780--793.
\bibitem{7-zac-1}
\Aue{Zatsman, I.} 2020. Three-dimensional encoding of emerging meanings in  
AI-systems. \textit{21st European Conference on Knowledge Management 
Proceedings}. Reading, U.K.: Academic Publishing International Ltd. 878--887.
\bibitem{8-zac-1}
\Aue{Ackoff, R.} 1989. From data to wisdom. \textit{J.~Applied Systems Analysis} 
16(1):3--9.
\bibitem{9-zac-1}
\Aue{Rosenbloom, P.\,S.} 2013. \textit{On computing: The fourth great scientific 
domain}. Cambridge, MA: MIT Press. 307~p.
\bibitem{10-zac-1}
\Aue{Rowley, J.} 2007. The wisdom hierarchy: Representations of the DIKW 
hierarchy. \textit{J.~Inf. Sci.} 33(2):163--180. doi: 10.1177/0165551506070706.
\bibitem{11-zac-1}
\Aue{Frick$\acute{\mbox{e}}$, M.\,H.} 2018.  
Data-Information-Knowledge-Wisdom (DIKW) pyramid, framework, continuum. 
\textit{Encyclopedia of big data}. Eds. L.~Schintler and C.~McNeely. Cham: 
Springer. 4~p. doi: 10.1007/978-3-319-32001- 4\_331-1.
\bibitem{12-zac-1}
\Aue{Denning, P., and P.~Rosenbloom.} 2009. Computing: The fourth great domain 
of science. \textit{Commun. ACM} 52(9):27--29.
\bibitem{13-zac-1}
\Aue{Denning, P., and P.~Freeman.} 2009. Computing's paradigm. \textit{Commun. 
ACM} 52(12):28--30. doi: 10.1145/ 1610252.1610265.

\bibitem{17-zac-1} %14
\Aue{Farradane, J.} 1980. Knowledge, information, and information science. 
\textit{J.~Inf. Sci.} 2(2):75--80. doi: 10.1177/ 01655515800020020.

\bibitem{15-zac-1}
\Aue{Shreyder, Yu.\,A.} 1988. Informatsiya i~znanie [Information and knowledge]. 
\textit{Sistemnaya kontseptsiya in\-for\-ma\-tsi\-on\-nykh protsessov} [System concept of 
information processes]. Moscow: VNIISI. 47--52.
\bibitem{16-zac-1}
\Aue{Ingwersen, P.} 1995. Information and information science. 
\textit{Encyclopedia of library and information science}. Eds. J.\,D.~McDonald and 
M.~Levine-Clark. New York, NY: Marcel Dekker Inc. 56(19):137--174.

\bibitem{14-zac-1} %17
Gilyarevskiy, R.\,S., ed. 2006. \textit{Informatika kak nauka ob informatsii: 
informatsionnyy, dokumental'nyy, tekh\-no\-lo\-gi\-che\-skiy, ekonomicheskiy, sotsial'nyy 
i~organizatsionnyy aspekty} [Informatics as information science: Informational, 
documentary, technological, economic, social, and organizational dimensions]. 
Moscow: FAIR-PRESS. 592~p.

\bibitem{18-zac-1}
\Aue{Hjorland, B.} 2000. Library and information science: Practice, theory, and 
philosophical basis. \textit{Inform. Process. Manag.} 36(3):501--531. doi:  
10.1016/S0306-\mbox{4573(99)00038-2}.
\bibitem{19-zac-1}
Deep shift~--- technology tipping points and societal impact. 2015. \textit{World Economic 
Forum}. Geneva. 44~p. Available at: {\sf 
http://www3.weforum.org/docs/WEF\_ GAC15\_Technological\_Tipping\_Points\_report\_2015.pdf} (accessed May~20, 
2024).
\bibitem{20-zac-1}
\Aue{Berman, F., R.~Rutenbar, B.~Hailpern, H.~Christensen, S.~Davidson, 
D.~Estrin, M.~Franklin, M.~Martonosi, P.~Raghavan, V.~Stodden, and 
A.\,S.~Szalay.} 2018. Realizing the potential of data science. \textit{Commun. ACM} 
61(4):67--72. doi: 10.1145/3188721.
\bibitem{21-zac-1}
\Aue{Stodden, V.} 2020. The data science life cycle: A~disciplined approach to 
advancing data science as a~science. \textit{Commun. ACM} 
 63(7):58--66. doi: 10.1145/3360646.

\bibitem{23-zac-1} %22
\Aue{Zatsman, I.\,M.} 2023. Nauchnaya paradigma informatiki: klassifikatsiya 
transformatsiy ob''ektov predmetnoy oblasti [Scientific paradigm of informatics: 
Transformation classification of domain objects]. \textit{Sistemy i~Sredstva 
Informatiki~--- Systems and Means of Informatics} 33(4):126--138. doi: 
10.14357/08696527230412. EDN: ZIKUWO.

\bibitem{22-zac-1} %23
\Aue{Zatsman, I.\,M.} 2023. Nauchnaya paradigma informatiki: klassifikatsiya 
ob''ektov predmetnoy oblasti [Scientific paradigm of informatics: Classification of 
domain objects]. \textit{Informatika i~ee Primeneniya~--- Inform. Appl.} 
 17(4):96--103. doi: 10.14357/19922264230413. EDN: FIUQAT.
 
\bibitem{24-zac-1}
\Aue{   Zatsman, I.\,M.} 2022. O nauchnoy paradigme informatiki: verkhniy uroven' 
klassifikatsii ob''ektov ee predmetnoy oblasti [On the scientific paradigm of 
informatics: The classification high level of its objects]. \textit{Informatika i~ee 
Primeneniya~--- Inform. Appl.} 16(4):73--79. doi: 10.14357/19922264220411. EDN: 
XZNKVI.
\bibitem{25-zac-1}
\Aue{Solomonick, A.\,B.} 2011. \textit{Filosofiya znakovykh system i~yazyk} 
[Philosophy of sign systems and language]. Moscow: LKI. 408~p.
\bibitem{26-zac-1}
\Aue{Zatsman, I.\,M.} 2023. Transformatsiya ierarkhii Akoffa v~nauchnoy 
paradigme informatiki [Transformation of the Ackoff's hierarchy in the scientific 
paradigm of informatics]. \textit{Informatika i~ee Primeneniya~--- Inform. \mbox{Appl.}} 
17(3):107--113. doi: 10.14357/19922264230315. EDN: UMVRRV.
\bibitem{27-zac-1}
\Aue{Zatsman, I.} 2024. Building digital spiral models of knowledge 
generation. \textit{19th Forum (International) on Knowledge Asset Dynamics 
Proceedings}. Matera, Italy: Arts for Business Institute. 2185--2196.
\bibitem{28-zac-1}
\Aue{Zatsman, I.} 2023. Digital spiral model of knowledge creation and encoding its 
dynamics. \textit{18th Forum (International) on Knowledge Asset Dynamics 
Proceedings}. Matera, Italy: Arts for Business Institute. 581--596.
\bibitem{29-zac-1}
\Aue{Zatsman, I.\,M.} 2019. Interfeysy tret'ego poryadka v~informatike 
 [Third-order interfaces in informatics]. \textit{Informatika i~ee Primeneniya~--- 
Inform. Appl.} 13(3):82--89. doi: 10.14357/19922264190312. EDN: EHRQLF.
\bibitem{30-zac-1}
\Aue{Zatsman, I.} 2023. Scientific paradigm of informatics as a~third culture. 
\textit{Scientific Technical Information Processing} 50(4):246--258. doi: 
10.3103/S0147688223040111. EDN: CKHMYS.

\end{thebibliography}

 }
 }

\end{multicols}

\vspace*{-6pt}

\hfill{\small\textit{Received April 14, 2024}} 


\vspace*{-12pt}


\Contrl

\vspace*{-3pt}

\noindent
\textbf{Zatsman Igor M.} (b.\ 1952)~--- Doctor of Science in technology, head of 
department, Federal Research Center ``Computer Science and Control'' of the 
Russian Academy of Sciences, 44-2~Vavilov Str., Moscow 119333, Russian 
Federation; \mbox{izatsman@yandex.ru}





\label{end\stat}

\renewcommand{\bibname}{\protect\rm Литература}    %15


%%%%%%%%%%%%%%%%%%%%%%%%%%%%%%%%%%%%%%%%%%%%%%%

%\def\stat{rez}
{%\hrule\par
%\vskip 7pt % 7pt
\raggedleft\Large \bf%\baselineskip=3.2ex
Р\,Е\,Ц\,Е\,Н\,З\,И\,И \vskip 17pt
    \hrule
    \par
\vskip 6pt plus 6pt minus 3pt }

%\thispagestyle{headings} %с верхним колонтитулом
%\thispagestyle{myheadings} %с нижним колонтитулом, но в верхнем РЕЦЕНЗИИ

\def\tit{НОВАЯ КНИГА И.\,Н.~СИНИЦЫНА, А.\,С.~ШАЛАМОВА <<ЛЕКЦИИ ПО ТЕОРИИ 
ИНТЕГРИРОВАННОЙ ЛОГИСТИЧЕСКОЙ ПОДДЕРЖКИ>> (М.: ТОРУС ПРЕСС, 2012. 624~с.)}

%1
\def\aut{Д.ф.-м.н., профессор С.\,Я.~Шоргин}

\def\auf{\ }

\def\leftkol{\ % РЕЦЕНЗИИ
}

\def\rightkol{ \ } 

%\def\leftkol{\ } % ENGLISH ABSTRACTS}

%\def\rightkol{\ } %ENGLISH ABSTRACTS}

%\def\leftkol{РЕЦЕНЗИИ}

%\def\rightkol{РЕЦЕНЗИИ}

\titele{\tit}{\aut}{\auf}{\leftkol}{\rightkol}
\vspace*{-18pt}


     \label{st\stat}

     \begin{multicols}{2}
     {\small
     {\baselineskip=10.1pt
     

      В книге представлено системное изложение теоретических основ одного из новейших 
направлений в \mbox{об\-ласти} экономики послепродажного обслуживания изделий наукоемкой 
продукции (ИНП) длительного пользования~--- интегрированной логистической поддержки
(ИЛП). 
{\looseness=1

}

Приведены также результаты новых работ, выполненных в Институте проблем информатики 
Российской академии наук в рамках научного направления <<Информационные технологии и 
анализ сложных сис\-тем>>.
 {%\looseness=1

}
     
      Излагаемые в книге научные подходы позво\-ляют карди\-наль\-но реформировать 
существующие системы производства и эксплуатации ИНП путем создания и внед\-ре\-ния 
методов рационального и оптимального управ\-ле\-ния процессами расходования 
вре\-мен\-н$\acute{\mbox{ы}}$х, 
мате\-ри\-аль\-ных, трудовых и других ресурсов на всех стадиях жизненного цикла изделий (ЖЦИ) по 
критериям экономической целесообразности и эф\-фек\-тив\-ности.
  {\looseness=1

}
    
      В книге приведен краткий обзор причин возник\-новения и
      развития CALS-методологии как основы 
современных международных стандартов по созданию и функционированию глобальных 
ин\-фор\-ма\-ци\-он\-но-ком\-му\-ни\-ка\-ци\-он\-ных систем, ее ключевых возможностей и эффективности 
результатов ее использования. 
Авторы %\linebreak 
предлагают ряд научных обоснований для разработки 
единой теории проектирования и управления систем ИЛП для полноценного использования 
преимуществ %\linebreak
 суще\-ст\-ву\-ющей методологии, определяют \mbox{общую} структурную схему 
комплексной системы <<ИНП-СППО>> и необходимость разработки для ее описания 
гибридных стохастических моделей.
{%\looseness=1

}

%\columnbreak
      
      Книга состоит из пяти частей, где последовательно излагается материал по каждой из 
следующих тем: <<Интегрированная логистическая поддержка>>, <<Теория гибридных 
стохастических систем и компьютерная поддержка исследований и разработок>>, <<Основы 
математического моделирования, анализа и синтеза систем послепродажного обслуживания>>, 
<<Определение и анализ показателей экспортного потенциала ИНП при проектировании>>, 
<<Задачи управления поддержкой послепродажного обслуживания>>, а также 
<<Моделирование инвестиционных процессов ИЛП в условиях неравновесных финансовых 
рынков>>. 
   
      В конце каждой главы приведены выводы и даны вопросы и задания для 
самоконтроля. В~приложениях содержатся основные определения по программам работ по 
анализу ИЛП, логистическим базам данных и компьютерным решениям, эквивалентной статистической 
линеаризации нелинейных преобразований ИЛП, справочный материал, а также развернутые 
уравнения для вероятностных характеристик.


      \def\leftkol{РЕЦЕНЗИИ}

\def\rightkol{РЕЦЕНЗИИ} 

      
      Книга заинтересует широкий круг специалистов и может быть использована научными 
проектными организациями в сфере промышленного производства ИНП. Большое количество 
иллюстраций, примеров и вопросов, обращенных к читателю, позволяет использовать книгу 
также в качестве учебного пособия для студентов и аспирантов машиностроительных, 
транспортных и~других специальностей, а также для самостоятельного изучения. 
{%\looseness=-1

}

Книга 
представляет несомненный интерес для специалистов и студентов в области прикладной 
математики и информатики.
    

}

}
\end{multicols}

%\newpage

\def\stat{authorsrus}
{%\hrule\par
%\vskip 7pt % 7pt
\raggedleft\Large \bf%\baselineskip=3.2ex
О\,Б\ \ А\,В\,Т\,О\,Р\,А\,Х \vskip 17pt
    \hrule
    \par
\vskip 21pt plus 8pt minus 4pt }


\def\tit{\ }

\def\aut{\ }

\def\auf{\ }

\def\leftkol{\ } % ENGLISH ABSTRACTS}

\def\rightkol{ОБ АВТОРАХ} %ENGLISH ABSTRACTS}

\titele{\tit}{\aut}{\auf}{\leftkol}{\rightkol}
      
            \label{st\stat}



\vspace*{24pt}

\begin{multicols}{2}




\noindent
\textbf{Архипов Олег Петрович} (р.\ 1948)~---
кандидат технических наук, директор Орловского филиала Института проб\-лем информатики
Российской академии наук
%302025, г.Орел, Московское шоссе, д.137

\vspace*{3pt}

\noindent
\textbf{Бирюкова Татьяна Константиновна} (р.\ 1968)~---
кандидат фи\-зи\-ко-ма\-те\-ма\-ти\-че\-ских наук, старший научный сотрудник Института проб\-лем информатики
Российской академии наук

\vspace*{3pt}

\noindent 
\textbf{Бобков  Сергей Геннадьевич} (р.\ 1955)~---
доктор технических наук,  заведующий отделением На\-уч\-но-ис\-сле\-до\-ва\-тель\-ско\-го 
института системных исследований Российской академии наук
%117218, Москва, Нахимовский просп., 36, к.1 

\vspace*{3pt}

\noindent \textbf{Васильев Николай Семенович} (р.\ 1952)~--- доктор 
фи\-зи\-ко-ма\-те\-ма\-ти\-че\-ских наук, профессор, 
МГТУ им.\ Н.\,Э.~Баумана 
%, Москва 105005, 2-я Бауманская ул., д.~5,

\vspace*{3pt}

\noindent
\textbf{Гершкович Максим Михайлович} (р.\ 1968)~---
старший научный сотрудник Института проб\-лем информатики
Российской академии наук

\vspace*{3pt}

\noindent 
\textbf{Дьяченко Юрий Георгиевич} (р.\ 1958)~--- кандидат технических наук, 
старший научный сотрудник Института проб\-лем информатики
Российской академии наук

\vspace*{3pt}

\noindent 
\textbf{Ерошенко Александр Андреевич} (р.\ 1989)~--- аспирант кафедры 
математической статистики факультета вычисли\-тельной математики и кибернетики 
Московского государственного университета им.\ М.\,В.~Ломоносова
%119991, Москва ГСП-1, Ленинские горы, д.\ 1, стр. 52

\vspace*{3pt}
 
\noindent 
\textbf{Захаров Виктор Николаевич} (р.\ 1948)~--- 
доктор технических наук, доцент, ученый секретарь Института проб\-лем информатики
Российской академии наук

\vspace*{3pt}

\noindent
\textbf{Зейфман Александр Израилевич} (р.\ 1954)~---
доктор фи\-зи\-ко-ма\-те\-ма\-ти\-че\-ских наук, профессор, 
заведующий кафедрой Вологодского государственного университета; 
старший научный сотрудник Института проб\-лем информатики
Российской академии наук; главный научный сотрудник ИСЭРТ Российской академии наук

\vspace*{3pt}

\noindent
\textbf{Зыкин Сергей Владимирович} (р.\ 1959)~--- 
доктор технических наук, профессор, заведующий лабораторией Института математики 
им.\ С.\,Л.~Соболева Сибирского отделения Российской академии наук, Новосибирск 
%630090, пр.\ ак.\ Коптюга, 4 

\vspace*{4pt}

\noindent
\textbf{Киреев Владимир Иванович} (р.\ 1938)~---
доктор фи\-зи\-ко-ма\-те\-ма\-ти\-че\-ских наук, профессор Московского 
государственного горного университета
%Адрес: Россия, 119991, г. Москва, Ленинский проспект, д. 6

%\columnbreak

\vspace*{4pt}

\noindent
\textbf{Козеренко Елена Борисовна} (р.\ 1959)~---
кандидат филологических наук, заведующая лабораторией Института проб\-лем информатики
Российской академии наук

\vspace*{4pt}

\noindent
\textbf{Королев Виктор Юрьевич} (р.\ 1954)~--- доктор
фи\-зи\-ко-ма\-те\-ма\-ти\-че\-ских наук, профессор кафедры математической 
статистики факультета вычисли\-тельной математики и кибернетики 
Московского государственного университета; 
ведущий научный сотрудник Института проб\-лем информатики
Российской академии наук

\vspace*{4pt}

\noindent
\textbf{Коротышева Анна Владимировна} (р.\ 1988)~---
старший преподаватель Вологодского государственного университета

\vspace*{4pt}

\noindent 
\textbf{Кун Де Турк} (р.\ 1981)~--- научный сотрудник 
исследовательской группы SMACS факультета телекоммуникаций и обработки информации
Университета Гента, Бельгия
%В-9000 Гент, Бельгия

\vspace*{4pt}

\noindent
\textbf{Лупенцов Олег Сергеевич} (р.\ 1986)~---
аспирант Омского государственного института сервиса
%Омск 644043, ул.\ Певцова 13

\vspace*{4pt}

\noindent
\textbf{Лучко Олег Николаевич} (р.\ 1961)~---
кандидат педагогических наук, профессор, заведующий кафедрой 
Омского государственного института сервиса
%Омск 644043, ул.\ Певцова 13

\vspace*{4pt}

\noindent
\textbf{Малашенко Юрий Евгеньевич} (р.\ 1946)~---
доктор фи\-зи\-ко-ма\-те\-ма\-ти\-че\-ских наук, заведующий сектором 
Вычислительного центра им.\ А.\,А.~Дородницына Российской академии наук
%Адрес: 119333, Москва, ул. Вавилова, 40,

\vspace*{4pt}

\noindent
\textbf{Маньяков Юрий Анатольевич} (р.\ 1984)~---
кандидат технических наук, научный сотрудник Орловского филиала Института проб\-лем информатики
Российской академии наук
%302025, г.Орел, Московское шоссе, д.137

\vspace*{4pt}

\noindent
\textbf{Маренко Валентина Афанасьевна} (р.\ 1951)~---
кандидат технических наук, доцент, старший научный сотрудник 
Института математики им.\ С.\,Л.~Соболева Сибирского отделения Российской академии наук
%Новосибирск 630090, пр. ак. Коптюга, 4 

\vspace*{3pt}

\noindent 
\textbf{Морозов Евсей Викторович} (р.\ 1947)~--- доктор 
фи\-зи\-ко-ма\-те\-ма\-ти\-че\-ских, профессор, ведущий научный сотрудник 
Института прикладных математических исследований Карельского научного центра Российской
академии наук; 
%%185910 Россия, Республика Карелия, г.\ Петрозаводск, ул.\ Пушкинская, 11
профессор Петрозаводского государственного университета, Петрозаводск
%185910 Россия, Республика Карелия, г.\ Петрозаводск, пр.\ Ленина, 33

%\pagebreak

\vspace*{3pt}

\noindent
\textbf{Назарова Ирина Александровна} (р.\ 1966)~---
кандидат фи\-зи\-ко-ма\-те\-ма\-ти\-че\-ских наук, 
научный сотрудник Вычислительного центра им.\ А.\,А.~Дородницына Российской академии наук 
%Адрес: 119333, Москва, ул. Вавилова, 40

\vspace*{3pt}

\noindent
\textbf{Павлов Игорь Валерианович} (р.\ 1945)~--- 
доктор фи\-зи\-ко-ма\-те\-ма\-ти\-че\-ских наук, профессор МГТУ им.\ Н.\,Э.~Баумана 
%Москва 105005, 2-я Бауманская ул., д.~5 

%\pagebreak

\vspace*{3pt}

\noindent 
\textbf{Потахина Любовь Викторовна} (р.\ 1989)~--- аспирантка
Института прикладных математических исследований Карельского научного центра
Российской академии наук; 
%%185910 Россия, Республика Карелия, г.\ Петрозаводск, ул.\ Пушкинская, 11
инженер Петрозаводского государственного университета, Петрозаводск
%185910 Россия, Республика Карелия, г.\ Петрозаводск, пр.\ Ленина, 33

\vspace*{3pt}

\noindent 
\textbf{Рождественский Юрий Владимирович} (р.\ 1952)~--- 
кандидат технических наук, заведующий сектором Института проб\-лем информатики
Российской академии наук

\vspace*{3pt}

\noindent 
\textbf{Синицын Игорь Николаевич} (р.\ 1940)~--- доктор технических наук,
профессор, заслуженный деятель\linebreak\vspace*{-12pt}

\columnbreak

\noindent
 науки РФ, заведующий отделом Института проб\-лем информатики
Российской академии наук

\vspace*{7pt}


\noindent
\textbf{Сиротинин Денис Олегович} (р.\ 1984)~---
кандидат технических наук, научный сотрудник Орловского филиала Института проб\-лем информатики
Российской академии наук
%302025, г.Орел, Московское шоссе, д.137

\vspace*{7pt}

%\columnbreak

\noindent 
\textbf{Соколов  Игорь Анатольевич} (р.\ 1954)~--- академик (действительный член) Российской 
академии наук, доктор технических наук, директор Института проб\-лем информатики
Российской академии наук

\vspace*{7pt}

\noindent
\textbf{Степченков Юрий Афанасьевич} (р.\ 1951)~---
кандидат технических наук, заведующий отделом Института проб\-лем информатики
Российской академии наук

\vspace*{7pt}

\noindent
\textbf{Сурков Алексей Викторович} (р.\ 1978)~--- 
старший научный сотрудник На\-уч\-но-ис\-сле\-до\-ва\-тель\-ско\-го 
института системных исследований Российской академии наук
%117218, Москва, Нахимовский просп., 36, к.1 

\vspace*{7pt}

\noindent 
\textbf{Шестаков Олег Владимирович} (р.\ 1976)~--- доктор 
фи\-зи\-ко-ма\-те\-ма\-ти\-че\-ских, доцент кафедры математической статистики 
факультета вычисли\-тельной математики и кибернетики Московского 
государственного университета им.\ М.\,В.~Ломоносова; 
%119991, Москва ГСП-1, Ленинские горы, д.\ 1, стр. 52
старший научный сотрудник Института проб\-лем информатики
Российской академии наук
%, Москва 119333, ул. Вавилова, д.~44, корп.~2

\vspace*{7pt}

\noindent 
\textbf{Шоргин Сергей Яковлевич} (р.\ 1952.)~--- доктор
фи\-зи\-ко-ма\-те\-ма\-ти\-че\-ских наук, профессор, заместитель директора Института 
проб\-лем информатики Российской академии наук





%%%%%%%%%%%%%%%%%%%%%%%%%%%%%%%%%%%%%%%%%%%%%%%%%%%%%%%%%%%%%%%%%%%%%%%%%%%%%%%




%\def\rightkol{ОБ АВТОРАХ}
%\def\leftkol{ОБ АВТОРАХ}

 \label{end\stat}





%\def\leftfootline{\small{\textbf{\thepage}
%\hfill ИНФОРМАТИКА И ЕЁ ПРИМЕНЕНИЯ\ \ \ том~7\ \ \ выпуск~1\ \ \ 2013}
%}%
% \def\rightfootline{\small{ИНФОРМАТИКА И ЕЁ ПРИМЕНЕНИЯ\ \ \ том~7\ \ \ выпуск~1\ \ \ 2013
%\hfill \textbf{\thepage}}}


%\thispagestyle{myheadings}



\end{multicols}

\newpage

%\end{document}

%
\def\stat{rekl}
%\label{preobr}

%\def\tit{АКАДЕМИК ПУГАЧЁВ  ВЛАДИМИР СЕМЁНОВИЧ\\
%25.03.1911--25.03.1998}


%   \vspace*{-48pt}
%   \begin{center}\LARGE
%Академик Пугачёв  Владимир Семёнович\\ (25.03.1911--25.03.1998)
%   \end{center}

   %\vspace*{2.5mm}

   \begin{center}

{\prgsh\LARGE
ЮБИЛЕИ}

\end{center}
%\hrule

\vspace*{6pt}


   \vspace*{8mm}

   \thispagestyle{empty}


%\def\stat{emel}


\section*{К 70-летию заместителя директора ИПИ РАН,\\ члена редколлегии журнала
<<Информатика и её применения>>\\ доктора технических наук В.\,И.~Будзко}

\vspace*{18pt}




          \begin{multicols}{2}

%            \label{st\stat}

\begin{center}
\vspace*{1pt}
\mbox{%
\epsfxsize=78mm
\epsfbox{bud-1.eps}
}
\end{center}

\vspace*{12pt}

      14 августа 2014~г.\ исполнилось 70~лет за\-мес\-ти\-те\-лю директора ИПИ РАН по
научной работе доктору технических наук Владимиру Игоревичу Будзко.

      Владимир Игоревич Будзко родился в г.~Москве. Высшее образование получил на факультете
элект\-рон\-но-вы\-чис\-ли\-тель\-ных устройств в Московском
ин\-же\-нер\-но-фи\-зи\-че\-ском институте
(МИФИ), который он окончил в 1968~г., после чего был на\-прав\-лен для прохождения
службы в одну из войс\-ко\-вых частей, где прошел путь от инженера до первого заместителя
командира войсковой части.

      С приходом В.\,И.~Будзко в ИПИ РАН (2001~г.)\ в институте
сформировалось новое научное на\-прав\-ле\-ние теоретических исследований~--- <<Постро\-ение
ин\-фор\-ма\-ци\-он\-но-те\-ле\-ком\-му\-ни\-ка\-ци\-он\-ных\linebreak сис\-тем
высокой до\-ступ\-ности>>. В~рамках этого
направления выполнен широкий круг фундаментальных исследований по поиску подходов и
определению принципов построения средств обеспечения доступности, конфиденциальности
и целостности современных крупномасштабных
ин\-фор\-ма\-ци\-он\-но-те\-ле\-ком\-му\-ни\-ка\-ци\-он\-ных
сис\-тем (ИТС). Разработаны основные сис\-тем\-но-тех\-ни\-че\-ские принципы и базовые
архитектурные решения построения перспективных для условий России ИТС с
централизованной обработкой и хранением информации, сочетающих в себе свойства
высокой доступности, отказо- и катастрофоустойчивости, информационной защищенности.
Определены принципы, методы и математические основы рационального построения и
оптимизации средств восстановления функционирования центров обработки данных (ЦОД)
после возникновения отказов и катастроф, передачи и хранения данных, обеспечения
информационной безопасности при достижении минимальной совокупной стоимости
владения такими системами. Результаты нашли практическое воплощение при реализации
проектов в интересах ряда отечественных государственных и негосударственных
организаций, таких как Банк России (БР), Внешторгбанк, ОАО <<ГМК <<Норильский Никель>>,
<<Газпром>>, Минэкономразвития России, Правительство Москвы, а также ряд силовых
ведомств.

      Под руководством В.\,И.~Будзко начиная с 2001~г.\ выполнен комплекс
      на\-уч\-но-ис\-сле\-до\-ва\-тель\-ских и
      опыт\-но-кон\-ст\-рук\-тор\-ских работ (свыше 100~проектов),
направленных на развитие электронной информационной технологии БР.
Разработаны концепции развития ИТС БР сначала до 2008~г., а затем до 2013~г., которые
были приняты в качестве основы проведения технической политики. За реализацию проекта
<<Катастрофоустойчивая тер\-ри\-то\-ри\-аль\-но-рас\-пре\-де\-лен\-ная
      ин\-фор\-ма\-ци\-он\-но-те\-ле\-ком\-му\-ни\-ка\-ци\-он\-ная сис\-те\-ма централизованной
обработки банковской информации>> В.\,И.~Будзко удостоен Премии Правительства РФ в
области науки и техники за 2010~г.

      В.\,И.~Будзко возглавлял и возглавляет работы по ряду других прикладных проектов,
связанных с созданием, совершенствованием и развитием крупномасштабных ИТС.

      В.\,И.~Будзко~--- генерал-майор, доктор технических наук, член-кор\-рес\-пон\-дент
Академии криптографии РФ, известный ученый в области информатики и применения
информационных технологий при построении территориально распределенных ИТС
различного назначения. Является автором свыше 250~научных работ, опубликованных в
на\-уч\-но-тех\-ни\-че\-ских и специальных изданиях.

    \thispagestyle{empty}

      В.\,И.~Будзко уделяет большое внимание подготовке научных кадров. Под его
руководством защищено 6~диссертаций на соискание ученой степени кандидата
технических наук. Свыше 30~лет он читает лекции в ИКСИ Академии ФСБ, профессор
кафедры НИЯУ МИФИ. Является членом двух диссертационных советов, главным
редактором журнала <<Системы высокой доступности>> и членом редколлегии журнала
<<Информатика и её применения>>.

      \bigskip

      Редакционный совет и Редакционная коллегия журнала <<Информатика и её
применения>> сердечно поздравляют Владимира Игоревича Будзко с 70-ле\-ти\-ем и желают
крепкого здоровья и новых научных достижений.

\end{multicols}

\def\stat{cont}
{%\hrule\par
%\vskip 7pt % 7pt
\raggedleft\Large \bf%\baselineskip=3.2ex
А\,В\,Т\,О\,Р\,С\,К\,И\,Й\ \ У\,К\,А\,З\,А\,Т\,Е\,Л\,Ь\ \ З\,А\ \ 2\,0\,1\,0 г. \vskip 17pt
    \hrule
    \par
\vskip 21pt plus 6pt minus 3pt }

\label{st\stat}

\def\tit{\ }

\def\aut{\ }
\def\auf{\ }

\def\leftkol{\ } % ENGLISH ABSTRACTS}

\def\rightkol{\ } %АВТОРСКИЙ УКАЗАТЕЛЬ ЗА 2010 г.} %ENGLISH ABSTRACTS}

\titele{\tit}{\aut}{\auf}{\leftkol}{\rightkol}

\vspace*{-12pt}

{\tabcolsep=3pt
\begin{tabular}{p{388pt}rr}
&\textbf{Выпуск} & \textbf{Стр.}\\[6pt]
\hangindent=23pt\noindent\textbf{Арутюнян~А.\,Р.} Моделирование влияния деформаций отпечатков пальцев на 
точность\linebreak
\vspace*{-12pt}\\
\hspace*{23pt}дактилоскопической идентификации$\dotfill$&1&51\\
\hangindent=23pt\noindent\textbf{Архипов~О.\,П., Зыкова~З.\,П.} Интеграция гетерогенной информации о цветных 
пикселях\linebreak
\vspace*{-12pt}\\
\hspace*{23pt}и их цветовосприятии$\dotfill$&4&15\\
\hangindent=23pt\noindent\textbf{Баранов~С.\,И., Френкель~С.\,Л., Захаров~В.\,Н.} Полуформальная верификация 
цифрового устройства с конвейером, основанная на использовании алгоритмических машин\linebreak
\vspace*{-12pt}\\
\hspace*{23pt}состояния$\dotfill$&4&49\\
\textbf{Бекетова~И.\,В.} см.~Каратеев~С.\,Л.&&\\
\textbf{Белоусов~В.\,В.} см.~Синицын~И.\,Н.&&\\
\hangindent=23pt\noindent\textbf{Бенинг~В.\,Е., Королев~Р.\,А.} О предельном поведении мощностей критериев в 
случае\linebreak
\vspace*{-12pt}\\
\hspace*{23pt}распределения Лапласа$\dotfill$&2&63\\
\hangindent=23pt\noindent\textbf{Бенинг~В.\,Е., Сипина~А.\,В.} Асимптотическое разложение для мощности 
критерия,\linebreak
\vspace*{-12pt}\\
\hspace*{23pt}основанного на выборочной медиане, в случае распределения Лапласа$\dotfill$&1&18\\
\textbf{Бондаренко~А.\,В.} см.~Каратеев~С.\,Л.&&\\
\hangindent=23pt\noindent\textbf{Бородина~А.\,В., Морозов~Е.\,В.} Об оценивании асимптотики вероятности 
большого\linebreak
\vspace*{-12pt}\\
\hspace*{23pt}уклонения стационарной регенеративной очереди с одним прибором$\dotfill$&3&29\\
\hangindent=23pt\noindent\textbf{Бунтман~Н.\,В., Минель~Ж.-Л., Ле~Пезан~Д., Зацман~И.\,М.} Типология и 
компьютерное\linebreak
\vspace*{-12pt}\\
\hspace*{23pt}моделирование трудностей перевода$\dotfill$&3&77\\
\textbf{Визильтер~Ю.\,В.} см.~Каратеев~С.\,Л.&&\\
\hangindent=23pt\noindent\textbf{Гавриленко~С.\,В.} Оценки скорости сходимости распределений случайных сумм с 
безгранично делимыми индексами к нормальному закону$\dotfill$&4&81\\
\hangindent=23pt\noindent\textbf{Григорьева~М.\,Е., Шевцова~И.\,Г.} Уточнение неравенства 
Каца--Берри--Эссеена$\dotfill$&2&75\\
\hangindent=23pt\noindent\textbf{Грушо~А.\,А., Грушо~Н.\,А., Тимонина~Е.\,Е.} Поиск конфликтов в политиках 
безопасности: модель случайных графов$\dotfill$&3&38\\
\textbf{Грушо~Н.\,А.} см.~Грушо~А.\,А.&&\\
\hangindent=23pt\noindent\textbf{Гудков~В.\,Ю.} Математические модели изображения отпечатка пальца на основе 
описания линий$\dotfill$&1&58\\
\textbf{Гуртов~А.\,В.} см.~Лукьяненко~А.\,С.&&\\
\textbf{Желтов~С.\,Ю.} см.~Каратеев~С.\,Л.&&\\
\hangindent=23pt\noindent\textbf{Захаров~А.\,А., Серебряков~В.\,А.} Система управления электронной библиотекой 
LibMeta$\dotfill$&4&2\\
\textbf{Захаров~В.\,Н.} см.~Баранов~С.\,И.&&\\
\textbf{Захарова~Т.\,В.} см.~Матвеева~С.\,С.&&\\
\hangindent=23pt\noindent\textbf{Зацаринный~А.\,А., Чупраков~К.\,Г.} Некоторые аспекты выбора технологии для 
постро-\linebreak
\vspace*{-12pt}\\
\hspace*{23pt}ения систем отображения информации ситуационного центра$\dotfill$&3&59\\
\textbf{Зацман~И.\,М.} см.~Бунтман~Н.\,В.&&\\
\hangindent=23pt\noindent\textbf{Зейфман~А.\,И., Коротышева~А.\,В., Сатин~Я.\,А., Шоргин~С.\,Я.} Об 
устойчивости нестаци-\linebreak
\vspace*{-12pt}\\
\hspace*{23pt}онарных систем обслуживания с катастрофами$\dotfill$&3&9\\
\textbf{Зыкова~З.\,П.} см.~Архипов~О.\,П.&&\\
\hangindent=23pt\noindent\textbf{Илюшин~Г.\,Я., Соколов~И.\,А.} Организация управляемого доступа пользователей 
к\linebreak
\vspace*{-12pt}\\
\hspace*{23pt}разнородным ведомственным информационным ресурсам$\dotfill$&1&24\\
\hangindent=23pt\noindent\textbf{Кавагучи~Ю., Ульянов~В.\,В., Фуджикоши~Я.} Приближения для статистик, 
описывающих\linebreak
\vspace*{-12pt}\\
\hspace*{23pt}геометрические свойства данных большой размерности, с оценками 
ошибок$\dotfill$&1&12\\
\hangindent=23pt\noindent\textbf{Каратеев~С.\,Л., Бекетова~И.\,В., Ососков~М.\,В., Князь~В.\,А., 
Визильтер~Ю.\,В., Бондаренко~А.\,В., Желтов~С.\,Ю.} Автоматизированный контроль 
качества цифровых\linebreak
\vspace*{-12pt}\\
\hspace*{23pt}изображений для персональных документов$\dotfill$&1&65\\
\end{tabular}
}

\pagebreak

\def\leftkol{АВТОРСКИЙ УКАЗАТЕЛЬ ЗА 2010 г.} % ENGLISH ABSTRACTS}

\def\rightkol{АВТОРСКИЙ УКАЗАТЕЛЬ ЗА 2010 г.} %ENGLISH ABSTRACTS}

{\tabcolsep=3pt
\begin{tabular}{p{388pt}rr}
&\textbf{Выпуск} & \textbf{Стр.}\\[3pt]
\hangindent=23pt\noindent\textbf{Козеренко~Е.\,Б.} Лингвистические фильтры в статистических моделях машинного\linebreak
\vspace*{-12pt}\\
\hspace*{23pt}перевода$\dotfill$&2&83\\
\hangindent=23pt\noindent\textbf{Козеренко~Е.\,Б., Кузнецов~И.\,П.} Когнитивно-лингвистические представления в 
систе-\linebreak
\vspace*{-12pt}\\
\hspace*{23pt}мах обработки текстов$\dotfill$&3&69\\
\textbf{Князь~В.\,А.} см.~Каратеев~С.\,Л.&&\\
\hangindent=23pt\noindent\textbf{Колесников~А.\,В., Солдатов~С.\,А.} Алгоритм координации для гибридной 
интеллектуальной системы решения сложной задачи оперативно-производственного\linebreak
\vspace*{-12pt}\\
\hspace*{23pt}планирования$\dotfill$&4&61\\
\hangindent=23pt\noindent\textbf{Коновалов~М.\,Г.} О планировании потоков в системах вычислительных 
ресурсов$\dotfill$&2&3\\
\textbf{Конушин~А.\,С.} см.~Конушин~В.\,С.&&\\
\hangindent=23pt\noindent\textbf{Конушин~В.\,С., Кривовязь~Г.\,Р., Конушин~А.\,С.} Алгоритм распознавания людей 
в видео-\linebreak
\vspace*{-12pt}\\
\hspace*{23pt}последовательности по одежде$\dotfill$&1&74\\
\textbf{Корепанов~Э.\, Р.} см.~Синицын~И.\,Н.&&\\
\textbf{Королев~В.\,Ю.} см.~Соколов~И.\,А.&&\\
\textbf{Королев~Р.\,А.} см.~Бенинг~В.\,Е.&&\\
\textbf{Коротышева~А.\,В.} см.~Зейфман~А.\,И.&&\\
\hangindent=23pt\noindent\textbf{Кривенко~М.\,П.} Непараметрическое оценивание элементов байесовского 
клас\-си-\linebreak
\vspace*{-12pt}\\
\hspace*{23pt}фикатора$\dotfill$&2&13\\
\textbf{Кривовязь~Г.\,Р.} см.~Конушин~В.\,С.&&\\
\textbf{Крылов~А.\,С.} см.~Павельева~Е.\,А.&&\\
\hangindent=23pt\noindent\textbf{Крылов~В.\,А.} Моделирование и классификация многоканальных дистанционных\linebreak
\vspace*{-12pt}\\
\hspace*{23pt}изображений с использованием копул$\dotfill$&4&34\\
\hangindent=23pt\noindent\textbf{Крючин~О.\,В.} Разработка параллельных эвристических алгоритмов подбора 
весовых\linebreak
\vspace*{-12pt}\\
\hspace*{23pt}коэффициентов искусственной нейтронной сети$\dotfill$&2&53\\
\hangindent=23pt\noindent\textbf{Кудрявцев~А.\,А., Шоргин~С.\,Я.} Байесовские модели массового обслуживания и 
надеж-\linebreak
\vspace*{-12pt}\\
\hspace*{23pt}ности: характеристики среднего числа заявок в системе $M\vert M \vert 1\vert 
\infty$$\dotfill$&3&16\\
\hangindent=23pt\noindent\textbf{Кузнецов~А.\,А.} Связь между временными и структурно-топологическими 
характери-\linebreak
\vspace*{-12pt}\\
\hspace*{23pt}стиками диаграмм ритма сердца здоровых людей$\dotfill$&4&39\\
\textbf{Кузнецов~И.\,П.} см.~Козеренко~Е.\,Б.&&\\
\textbf{Ле~Пезан~Д.} см.~Бунтман~Н.\,В.&&\\
\hangindent=23pt\noindent\textbf{Лукьяненко~А.\,С., Морозов~Е.\,В., Гуртов~А.\,В.} Анализ сетевого протокола с общей 
функ-\linebreak
\vspace*{-12pt}\\
\hspace*{23pt}цией расширения окна передачи сообщения при конфликтах$\dotfill$&2&46\\
\hangindent=23pt\noindent\textbf{Лямин~О.\,О.} О предельном поведении мощностей критериев в случае обобщенного\linebreak
\vspace*{-12pt}\\
\hspace*{23pt}распределения Лапласа$\dotfill$&3&47\\
\hangindent=23pt\noindent\textbf{Маркин~А.\,В., Шестаков~О.\,В.} Асимптотики оценки риска при пороговой 
обработке\linebreak
\vspace*{-12pt}\\
\hspace*{23pt}вейвлет-вейглет коэффициентов в задаче томографии$\dotfill$&2&36\\
\hangindent=23pt\noindent\textbf{Матвеева~С.\,С., Захарова~Т.\,В.} Сети массового обслуживания с наименьшей 
длиной\linebreak
\vspace*{-12pt}\\
\hspace*{23pt}очереди$\dotfill$&3&22\\
\hangindent=23pt\noindent\textbf{Матюшенко~С.\,И.} Стационарные характеристики двухканальной системы 
обслужива-\linebreak
\vspace*{-12pt}\\
\hspace*{23pt}ния с переупорядочиванием заявок и распределениями фазового типа$\dotfill$&4&68\\
\textbf{Минель~Ж.-Л.} см.~Бунтман~Н.\,В.&&\\
\textbf{Морозов~Е.\,В.} см.~Бородина~А.\,В.&&\\
\textbf{Морозов~Е.\,В.} см.~Лукьяненко~А.\,С.&&\\
\textbf{Ососков~М.\,В.} см.~Каратеев~С.\,Л.&&\\
\hangindent=23pt\noindent\textbf{Павельева~Е.\,А., Крылов~А.\,С.} Поиск и анализ ключевых точек радужной 
оболочки\linebreak
\vspace*{-12pt}\\
\hspace*{23pt}глаза методом преобразования Эрмита$\dotfill$&1&79\\
\textbf{Печинкин~А.\,В.} см.~Френкель~С.\,Л.,&&\\
\hangindent=23pt\noindent\textbf{Протасов~В.\,И.} Составление субъективного портрета с использованием 
эволюционно-\linebreak
\vspace*{-12pt}\\
\hspace*{23pt}го морфинга и квалиметрия метода$\dotfill$&1&83\\
\hangindent=23pt\noindent\textbf{Рудаков~К.\,В., Торшин~И.\,Ю.} Вопросы разрешимости задачи распознавания 
вторичной\linebreak
\vspace*{-12pt}\\
\hspace*{23pt}структуры белка$\dotfill$&2&25\\
\textbf{Сатин~Я.\,А.} см.~Зейфман~А.\,И.&&\\
\hangindent=23pt\noindent\textbf{Сейфуль-Мулюков~Р.\,Б.} Нефть как носитель информации о своем 
происхождении,\linebreak
\vspace*{-12pt}\\
\hspace*{23pt}структуре и эволюции$\dotfill$&1&41\\
\end{tabular}
}

{\tabcolsep=3pt
\begin{tabular}{p{388pt}rr}
&\textbf{Выпуск} & \textbf{Стр.}\\[6pt]
\textbf{Семендяев~Н.\,Н.} см.~Синицын~И.\,Н.&&\\
\textbf{Серебряков~В.\,А.} см.~Захаров~А.\,А.&&\\
\textbf{Синицын~В.\,И.} см.~Синицын~И.\,Н.&&\\
\hangindent=23pt\noindent\textbf{Синицын~И.\,Н., Синицын~В.\,И., Корепанов~Э.\, Р., Белоусов~В.\,В., 
Семендяев~Н.\,Н.} Оперативное построение информационных моделей движения полюса 
Земли\linebreak
\vspace*{-12pt}\\
\hspace*{23pt}методами линейных и линеаризованных фильтров$\dotfill$&1&2\\
\textbf{Сипина~А.\,В.} см.~Бенинг~В.\,Е.&&\\
\hangindent=23pt\noindent\textbf{Соколов~И.\,А.} О работах заслуженного деятеля науки Российской Федерации 
И.\,Н.~Синицына в области информационных технологий и автоматизации (к 70-летию\linebreak
\vspace*{-12pt}\\
\hspace*{23pt}со дня рождения)$\dotfill$&3&84\\
\textbf{Соколов~И.\,А.} см.~Илюшин~Г.\,Я.&&\\
\hangindent=23pt\noindent\textbf{Соколов~И.\,А., Королев~В.\,Ю.} Предисловие$\dotfill$&2&2\\
\textbf{Солдатов~С.\,А.} см.~Колесников~А.\,В.&&\\
\hangindent=23pt\noindent\textbf{Степанов~С.\,Ю.} Использование координатного метода фрагментации 
коммутаторной\linebreak
\vspace*{-12pt}\\
\hspace*{23pt}нейронной сети для сокращения трафика$\dotfill$&2&57\\
\textbf{Тимонина~Е.\,Е.} см.~Грушо~А.\,А.&&\\
\textbf{Торшин~И.\,Ю.} см.~Рудаков~К.\,В.&&\\
\textbf{Ульянов~В.\,В.} см.~Кавагучи~Ю.&&\\
\textbf{Фазекаш~И.} см.~Чупрунов~А.\,Н.&&\\
\textbf{Френкель~С.\,Л.} см.~Баранов~С.\,И.&&\\
\hangindent=23pt\noindent\textbf{Френкель~С.\,Л., Печинкин~А.\,В.} Оценка времени самовосстановления в 
цифровых\linebreak
\vspace*{-12pt}\\
\hspace*{23pt}системах после сбоев, вызываемых переходными помехами$\dotfill$&3&2\\
\textbf{Фуджикоши~Я.} см.~Кавагучи~Ю.&&\\
\hangindent=23pt\noindent\textbf{Цискаридзе~А.\,К.} Математическая модель и метод восстановления позы человека 
по\linebreak
\vspace*{-12pt}\\
\hspace*{23pt}стереопаре силуэтных изображений$\dotfill$&4&27\\
\hangindent=23pt\noindent\textbf{Чупраков~К.\,Г.} К вопросу о размещении коллективных средств отображения в 
ситуа-\linebreak
\vspace*{-12pt}\\
\hspace*{23pt}ционном зале с заданными параметрами$\dotfill$&4&89\\
\textbf{Чупраков~К.\,Г.} см.~Зацаринный~А.\,А.&&\\
\hangindent=23pt\noindent\textbf{Чупрунов~А.\,Н., Фазекаш~И.} Законы повторного логарифма для числа 
безошибочных\linebreak
\vspace*{-12pt}\\
\hspace*{23pt}блоков при помехоустойчивом кодировании$\dotfill$&3&42\\
\textbf{Шевцова~И.\,Г.} см.~Григорьева~М.\,Е.&&\\
\hangindent=23pt\noindent\textbf{Шестаков~О.\,В.} Аппроксимация распределения оценки риска пороговой 
обработки вейвлет-коэффициентов нормальным распределением при использовании 
выбо-\linebreak
\vspace*{-12pt}\\
\hspace*{23pt}рочной дисперсии$\dotfill$&4&73\\
\textbf{Шестаков~О.\,В.} см.~Маркин~А.\,В.&&\\
\textbf{Шоргин~С.\,Я.} см.~Зейфман~А.\,И.&&\\
\textbf{Шоргин~С.\,Я.} см.~Кудрявцев~А.\,А.&&\\
\end{tabular}
}

%\thispagestyle{myheadings}
\def\leftfootline{\small{\textbf{\thepage}
\hfill ИНФОРМАТИКА И ЕЁ ПРИМЕНЕНИЯ\ \ \ том~4\ \ \ выпуск~4\ \ \ 2010}
}%
 \def\rightfootline{\small{ИНФОРМАТИКА И ЕЁ ПРИМЕНЕНИЯ\ \ \ том~4\ \ \ выпуск~4\ \ \ 2010
 \hfill \textbf{\thepage}}}
 \label{end\stat}





%Том 10 Выпуск 1-4 Год 2016

\def\stat{cont-e}
{%\hrule\par
%\vskip 7pt % 7pt
\raggedleft\Large \bf%\baselineskip=3.2ex
2\,0\,1\,6\ \ A\,U\,T\,H\,O\,R\ \ I\,N\,D\,E\,X \vskip 17pt
 \hrule
 \par
\vskip 21pt plus 6pt minus 3pt }

\label{st\stat}

\def\tit{\ }

\def\aut{\ }
\def\auf{\ }

\def\leftkol{\ } %2016 AUTHOR INDEX} % ENGLISH ABSTRACTS}

\def\rightkol{\ } %2016 AUTHOR INDEX} %ENGLISH ABSTRACTS}

\titele{\tit}{\aut}{\auf}{\leftkol}{\rightkol}

\def\leftfootline{\small{\textbf{\thepage}
\hfill INFORMATIKA I EE PRIMENENIYA~--- INFORMATICS AND APPLICATIONS\ \ \ 2016\
\ \ volume~10\ \ \ issue\ 4}
}%
 \def\rightfootline{\small{INFORMATIKA I EE PRIMENENIYA~--- INFORMATICS AND APPLICATIONS\ \ \ 2016\ \ \ volume~10\ \ \ issue\ 4
\hfill \textbf{\thepage}}}

\vspace*{-12pt}
\vspace*{-18pt}

{\tabcolsep=2.8pt
\begin{tabular}{p{382pt}cc}
&\textbf{Issue} & \textbf{Page}\\[6pt]
\Avtors{Agalarov~M.\,Ya.} see~Agalarov~Ya.\,M.&&\\
\Avtors{Agalarov~Ya.\,M., Agalarov~M.\,Ya., and
Shorgin~V.\,S.} About the optimal threshold of queue\linebreak
\\[-12pt]
\hspace*{23pt}length in a~particular problem of profit maximization
in the $M/G/1$ queuing system&2&70--79\\
\Avtors{Alexeyevsky~D.\,A.} BioNLP ontology extraction from 
a~restricted language corpus with\linebreak
\\[-12pt]
\hspace*{23pt}context-free grammars&1&119--128\\
\Avtors{Andreev~S.\,D.} see~Gaidamaka~Yu.\,V.&&\\
\Avtors{Andreev~S.\,D.} see~Ometov~A.\,Ya.&&\\
\Avtors{Arkhipov~O.\,P., Arkhipov~P.\,O., and Sidorkin~I.\,I.} The
option to create a~local coordinate\linebreak
\\[-12pt]
\hspace*{23pt}system for synchronization of selected images&3&91--97\\
\Avtors{Arkhipov~P.\,O.} see~Arkhipov~O.\,P.&&\\
\Avtors{Belousov~V.\,V.} see~Shnurkov~P.\,V.&&\\
\Avtors{Belousov~V.\,V.} see~Shnurkov~P.\,V.&&\\
\Avtors{Bening~V.\,E.} Calculation of~the~asymptotic deficiency
of~some statistical procedures based\linebreak
\\[-12pt]
\hspace*{23pt}on~samples with~random sizes&4&34--45\\
\Avtors{Borisov~A.\,V., Bosov~A.\,V., and Miller~G.\,B.} Modeling and
monitoring of VoIP connection&2&\hphantom{1}2--13\\
\Avtors{Bosov~A.\,V.} see~Borisov~A.\,V.&&\\
\Avtors{Briukhov~D.\,O.} see~Stupnikov~S.\,A.&&\\
\Avtors{Callaos~N.\,K.\ and Seyful-Mulyukov~R.\,B.} Complexity and
its information content&1&129--139\\
\Avtors{Chertok~A.\,V., Kadaner~A.\,I., Khazeeva~G.\,T., and
Sokolov~I.\,A.} Regime switching detection\linebreak
\\[-12pt]
\hspace*{23pt}for~the~Levy driven
Ornstein--Uhlenbeck process using CUSUM methods&4&46--56\\
\Avtors{Chichagov~V.\,V.} Asymptotic expansions of mean absolute
error of uniformly minimum variance unbiased and maximum likelihood
estimators on the one-parameter exponential\linebreak
\\[-12pt]
\hspace*{23pt}family model of lattice distributions&3&66--76\\
\Avtors{Danishevsky~V.\,I.} see~Kolesnikov A.\,V.&&\\
\Avtors{Fazliev~A.\,Z.} see~Kalinichenko~L.\,A.&&\\
\Avtors{Fedoseev~A.\,A.} What is behind the concept of ``knowledge in
small packages''&3&105--110\\
\Avtors{Gaidamaka~Yu.\,V., Andreev~S.\,D., Sopin~E.\,S.,
Samouylov~K.\,E., and Shorgin~S.\,Ya.} Interference analysis
of~the~device-to-device communications model with~regard to~a~signal\linebreak
\\[-12pt]
\hspace*{23pt}propagation environment&4&\hphantom{1}2--10\\
\Avtors{Gasilov~A.\,V.} see~Yakovlev~O.\,A.&&\\
\Avtors{Goncharov~A.\,V.\ and Strijov~V.\,V.} Metric time series
classification using weighted dynamic\linebreak
\\[-12pt]
\hspace*{23pt}warping relative to centroids of classes&2&36--47\\
\Avtors{Gordov~E.\,P.} see~Kalinichenko~L.\,A.&&\\
\Avtors{Gorshenin~A.\,K.} Concept of online service for stochastic
modeling of real processes&1&72--81\\
\Avtors{Gorshenin~A.\,K.} see~Shnurkov~P.\,V.&&\\
\Avtors{Gorshenin~A.\,K.} see~Shnurkov~P.\,V.&&\\
\Avtors{Grusho~A.\,A., Grusho~N.\,A., Zabezhailo~M.\,I., and
Timonina~E.\,E.} Integration of statistical and\linebreak
\\[-12pt]
\hspace*{23pt}deterministic methods for
analysis of information security&3&2--8\\
\Avtors{Grusho~A.\,A., Zabezhailo~M.\,I., and Zatsarinny~A.\,A.} On
the advanced procedure to reduce\linebreak
\\[-12pt]
\hspace*{23pt}calculation of Galois closures&4&\hphantom{1}96--104\\
\Avtors{Grusho~N.\,A.} see~Grusho~A.\,A.&&\\
\Avtors{Havanskov~V.\,A.} see~Minin~V.\,A.&&\\
\Avtors{Inkova~O.\,Yu.} see~Zatsman~I.\,M.&&\\
\Avtors{Isachenko~R.\,V.\ and Strijov~V.\,V.} Metric learning in
multiclass time series classification\linebreak
\\[-12pt]
\hspace*{23pt}problem&2&48--57\\
\end{tabular}
}
\pagebreak

\def\leftfootline{\small{\textbf{\thepage}
\hfill INFORMATIKA I EE PRIMENENIYA~--- INFORMATICS AND APPLICATIONS\ \ \ 2016\
\ \ volume~10\ \ \ issue\ 4}
}%
 \def\rightfootline{\small{INFORMATIKA I EE PRIMENENIYA~---
INFORMATICS AND APPLICATIONS\ \ \ 2016\ \ \ volume~10\ \ \ issue\ 4
\hfill \textbf{\thepage}}}

\def\leftkol{2016 AUTHOR INDEX} % ENGLISH ABSTRACTS}

\def\rightkol{2016 AUTHOR INDEX} %ENGLISH ABSTRACTS}


{\tabcolsep=2.83pt
\begin{tabular}{p{382pt}cc}
&\textbf{Issue} & \textbf{Page}\\[6pt]
\Avtors{Kadaner~A.\,I.} see~Chertok~A.\,V.&&\\[.255pt]
\Avtors{Kalinichenko~L.\,A., Volnova~A.\,A., Gordov~E.\,P.,
Kiselyova~N.\,N., Kovaleva~D.\,A., Malkov~O.\,Yu., Okladnikov~I.\,G.,
Podkolodnyy~N.\,L., Pozanenko~A.\,S., Ponomareva~N.\,V.,
Stupnikov~S.\,A.,} \textbf{and Fazliev~A.\,Z.} Data access challenges for data
intensive\linebreak
\\[-12pt]
\hspace*{23pt}research in Russia&1& 2--22\\[.255pt]
\Avtors{Karasikov~M.\,E.\ and Strijov~V.\,V.} Feature-based
time-series classification&4&121--131\\[.255pt]
\Avtors{Khazeeva~G.\,T.} see~Chertok~A.\,V.&&\\[.255pt]
\Avtors{Khokhlov~Yu.\,S.} Multivariate fractional Levy motion and its
applications&2&\hphantom{1}98--106\\[.255pt]
\Avtors{Kirikov~I.\,A., Kolesnikov~A.\,V., Listopad~S.\,V., and
Rumovskaya~S.\,B.} Fine-grained hybrid\linebreak
\\[-12pt]
\hspace*{23pt}intelligent systems. Part 2:
Bidirectional hybridization&1&\hphantom{1}96--105\\[.255pt]
\Avtors{Kirikov~I.\,A., Kolesnikov~A.\,V., Listopad~S.\,V., and
Rumovskaya~S.\,B.} ``Virtual council''~---\linebreak
\\[-12pt]
\hspace*{23pt}source environment
supporting complex diagnostic decision making&3&81--90\\[.255pt]
\Avtors{Kiselyova~N.\,N.} see~Kalinichenko~L.\,A.&&\\[.255pt]
\Avtors{Kolesnikov A.\,V., Listopad~S.\,V., Rumovskaya~S.\,B., and
Danishevsky~V.\,I.} Informal axiomatic\linebreak
\\[-12pt]
\hspace*{23pt}theory of~the~role visual models&4&114--120\\[.255pt]
\Avtors{Kolesnikov~A.\,V.} see~Kirikov~I.\,A.&&\\[.255pt]
\Avtors{Kolesnikov~A.\,V.} see~Kirikov~I.\,A.&&\\[.255pt]
\Avtors{Kolin~K.\,K.} Humanitarian aspects of information
security&3&111--121\\[.255pt]
\Avtors{Konovalov~M.\,G.\ and Razumchik~R.\,V.} Dispatching
to~two parallel nonobservable queues using\linebreak
\\[-12pt]
\hspace*{23pt}only static
information&4&57--67\\[.255pt]
\Avtors{Korchagin~A.\,Yu.} see~Korolev~V.\,Yu.&&\\[.255pt]
\Avtors{Korchagin~A.\,Yu.} see~Korolev~V.\,Yu.&&\\[.255pt]
\Avtors{Korepanov~E.\,R.} see~Sinitsyn~I.\,N.&&\\[.255pt]
\Avtors{Korepanov~E.\,R.} see~Sinitsyn~I.\,N.&&\\[.255pt]
\Avtors{Korolev~V.\,Yu., Korchagin~A.\,Yu., and Zeifman~A.\,I.} The
Poisson theorem for Bernoulli trials\linebreak
\\[-12pt]
\hspace*{23pt}with~a~random probability
of~success and~a~discrete analog of~the~Weibull distribution&4&11--20\\[.255pt]
\Avtors{Korolev~V.\,Yu., Zeifman~A.\,I., and Korchagin~A.\,Yu.}
Asymmetric Linnik distributions as~limit\linebreak
\\[-12pt]
\hspace*{23pt}laws for~random sums
of~independent random variables with~finite variances&4&21--33\\[.255pt]
\Avtors{Koucheryavy~E.\,A.} see~Ometov~A.\,Ya.&&\\[.255pt]
\Avtors{Kovaleva~D.\,A.} see~Kalinichenko~L.\,A.&&\\[.255pt]
\Avtors{Kovalyov~S.\,P.} Metaprogramming to increase
manufacturability of large-scale software-\linebreak
\\[-12pt]
\hspace*{23pt}intensive systems&1&56--66\\[.255pt]
\Avtors{Krivenko~M.\,P.} Significance tests of feature selection for
classification&3&32--40\\[.255pt]
\Avtors{Kruzhkov~M.\,G.} see~Zalizniak~Anna~A.&&\\[.255pt]
\Avtors{Kruzhkov~M.\,G.} see~Zatsman~I.\,M.&&\\[.255pt]
\Avtors{Kudryavtsev~A.\,A.} Bayesian queueing and reliability models:
\textit{A~priori} distributions with\linebreak
\\[-12pt]
\hspace*{23pt}compact support&1&67--71\\[.255pt]
\Avtors{Kudryavtsev~A.\,A.} Characteristics dependent on the balance
coefficient in Bayesian models\linebreak
\\[-12pt]
\hspace*{23pt}with compact support of \textit{a priori}
distributions&3&77--80\\[.255pt]
\Avtors{Kudryavtsev~A.\,A.\ and Palionnaia~S.\,I.} Bayesian recurrent
model of reliability growth:\linebreak
\\[-12pt]
\hspace*{23pt}Parabolic distribution of parameters&2&80--83\\[.255pt]
\Avtors{Kudryavtsev~A.\,A.\ and Titova~A.\,I.} Bayesian queuing
and~reliability models: Degenerate-\linebreak
\\[-12pt]
\hspace*{23pt}Weibull case&4&68--71\\[.255pt]
\Avtors{Leontyev~N.\,D.\ and Ushakov~V.\,G.} Analysis of a queueing
system with autoregressive arrivals\linebreak
\\[-12pt]
\hspace*{23pt}and nonpreemptive priority&3&15--22\\[.255pt]
\Avtors{Listopad~S.\,V.} see~Kirikov~I.\,A.&&\\[.255pt]
\Avtors{Listopad~S.\,V.} see~Kirikov~I.\,A.&&\\[.255pt]
\Avtors{Listopad~S.\,V.} see~Kolesnikov A.\,V.&&\\[.255pt]
\Avtors{Malkov~O.\,Yu.} see~Kalinichenko~L.\,A.&&\\[.255pt]
\Avtors{Markov~A.\,S., Monakhov~M.\,M., and
Ulyanov~V.\,V.} Generalized Cornish--Fisher expansions\linebreak
\\[-12pt]
\hspace*{23pt}for distributions of statistics based on samples
of random size&2&84--91\\[.255pt]
\Avtors{Melnikov~A.\,K.\ and Ronzhin~A.\,F.} Generalized statistical
method of~text analysis based\linebreak
\\[-12pt]
\hspace*{23pt}on~calculation of~probability distributions
of~statistical values&4&89--95\\
\end{tabular}
}
\pagebreak

\def\leftfootline{\small{\textbf{\thepage}
\hfill INFORMATIKA I EE PRIMENENIYA~--- INFORMATICS AND APPLICATIONS\ \ \ 2016\
\ \ volume~10\ \ \ issue\ 4}
}%
 \def\rightfootline{\small{INFORMATIKA I EE PRIMENENIYA~---
INFORMATICS AND APPLICATIONS\ \ \ 2016\ \ \ volume~10\ \ \ issue\ 4
\hfill \textbf{\thepage}}}

\def\leftkol{2016 AUTHOR INDEX} % ENGLISH ABSTRACTS}

\def\rightkol{2016 AUTHOR INDEX} %ENGLISH ABSTRACTS}


{\tabcolsep=3pt
\begin{tabular}{p{381pt}cc}
&\textbf{Issue} & \textbf{Page}\\[6pt]
\Avtors{Meykhanadzhyan~L.\,A.} Stationary characteristics of the finite
capacity queueing system with\linebreak
\\[-12pt]
\hspace*{23pt}inverse service order and generalized
probabilistic priority&2&123--131\\[.23pt]
\Avtors{Miller~G.\,B.} see~Borisov~A.\,V.&&\\[.23pt]
\Avtors{Minin~V.\,A., Zatsman~I.\,M., Havanskov~V.\,A., and
Shubnikov~S.\,K.} Intensity of citation of scientific publications in
inventions on information and computer technologies patented\linebreak
\\[-12pt]
\hspace*{23pt}in Russia by domestic and foreign applicants&2&107--122\\[.23pt]
\Avtors{Monakhov~M.\,M.} see~Markov~A.\,S.&&\\[.23pt]
\Avtors{Naumov~V.\,A.\ and Samouylov~K.\,E.} On relationship
between queuing systems with resources\linebreak
\\[-12pt]
\hspace*{23pt}and Erlang networks&3&\hphantom{1}9--14\\[.23pt]
\Avtors{Okladnikov~I.\,G.} see~Kalinichenko~L.\,A.&&\\[.23pt]
\Avtors{Ometov~A.\,Ya., Andreev~S.\,D., Turlikov~A.\,M., and
Koucheryavy~E.\,A.} Performance analysis of\linebreak
\\[-12pt]
\hspace*{23pt}a wireless data
aggregation system with contention for contemporary sensor
networks&3&23--31\\[.23pt]
\Avtors{Palionnaia~S.\,I.} see~Kudryavtsev~A.\,A.&&\\[.23pt]
\Avtors{Podkolodnyy~N.\,L.} see~Kalinichenko~L.\,A.&&\\[.23pt]
\Avtors{Ponomareva~N.\,V.} see~Kalinichenko~L.\,A.&&\\[.23pt]
\Avtors{Popkova~N.\,A.} see~Zatsman~I.\,M.&&\\[.23pt]
\Avtors{Pozanenko~A.\,S.} see~Kalinichenko~L.\,A.&&\\[.23pt]
\Avtors{Razumchik~R.\,V.} see~Konovalov~M.\,G.&&\\[.23pt]
\Avtors{Ronzhin~A.\,F.} see~Melnikov~A.\,K.&&\\[.23pt]
\Avtors{Rumovskaya~S.\,B.} see~Kirikov~I.\,A.&&\\[.23pt]
\Avtors{Rumovskaya~S.\,B.} see~Kirikov~I.\,A.&&\\[.23pt]
\Avtors{Rumovskaya~S.\,B.} see~Kolesnikov A.\,V.&&\\[.23pt]
\Avtors{Samouylov~K.\,E.} see~Gaidamaka~Yu.\,V.&&\\[.23pt]
\Avtors{Samouylov~K.\,E.} see~Naumov~V.\,A.&&\\[.23pt]
\Avtors{Serebryanskii~S.\,M.} see~Tyrsin~A.\,N.&&\\[.23pt]
\Avtors{Seyful-Mulyukov~R.\,B.} see~Callaos~N.\,K.&&\\[.23pt]
\Avtors{Shestakov~O.\,V.} Statistical properties of the denoising method
based on the stabilized hard\linebreak
\\[-12pt]
\hspace*{23pt}thresholding&2&65--69\\[.23pt]
\Avtors{Shestakov~O.\,V.} The strong law of large numbers for the risk
estimate in the problem of\linebreak
\\[-12pt]
\hspace*{23pt}tomographic image reconstruction from
projections with a correlated noise&3&41--45\\[.23pt]
\Avtors{Shestakov~O.\,V.} see~Zakharova~T.\,V.&&\\[.23pt]
\Avtors{Shnurkov~P.\,V., Gorshenin~A.\,K., and Belousov~V.\,V.}
Analytical solution of~the~optimal control\linebreak
\\[-12pt]
\hspace*{23pt}task of~a~semi-Markov
process with~finite set of~states&4&72--88\\[.23pt]
\Avtors{Shnurkov~P.\,V., Zasypko~V.\,V., Belousov~V.\,V., and
Gorshenin~A.\,K.} Development of the algorithm of numerical solution
of the optimal investment control problem\linebreak
\\[-12pt]
\hspace*{23pt}in the closed dynamical model of three-sector economy&1&82--95\\[.23pt]
\Avtors{Shorgin~S.\,Ya.} see~Gaidamaka~Yu.\,V.&&\\[.23pt]
\Avtors{Shorgin~V.\,S.} see~Agalarov~Ya.\,M.&&\\[.23pt]
\Avtors{Shubnikov~S.\,K.} see~Minin~V.\,A.&&\\[.23pt]
\Avtors{Sidorkin~I.\,I.} see~Arkhipov~O.\,P.&&\\[.23pt]
\Avtors{Sinitsyn~I.\,N.} Analytical modeling of processes in stochastic
systems with complex fractional\linebreak
\\[-12pt]
\hspace*{23pt}order Bessel nonlinearities&3&55--65\\[.23pt]
\Avtors{Sinitsyn~I.\,N.} Orthogonal supoptimal filters for nonlinear
stochastic systems on manifolds&1&34--44\\[.23pt]
\Avtors{Sinitsyn~I.\,N.\ and Korepanov~E.\,R.} Normal Pugachev
conditionally-optimal filters and extra-\linebreak
\\[-12pt]
\hspace*{23pt}polators for state linear stochastic systems&2&14--23\\[.23pt]
\Avtors{Sinitsyn~I.\,N.\ and Sinitsyn~V.\,I.} Analytical modeling of
distributions in stochastic systems on\linebreak
\\[-12pt]
\hspace*{23pt}manifolds based on ellipsoidal approximation&1&45--55\\[.23pt]
\Avtors{Sinitsyn~I.\,N., Sinitsyn~V.\,I., and
Korepanov~E.\,R.} Ellipsoidal suboptimal filters for nonlinear\linebreak
\\[-12pt]
\hspace*{23pt}stochastic systems on manifolds&2&24--35\\[.23pt]
\Avtors{Sinitsyn~V.\,I.} see~Sinitsyn~I.\,N.&&\\[.23pt]
\Avtors{Sinitsyn~V.\,I.} see~Sinitsyn~I.\,N.&&\\[.23pt]
\Avtors{Skvortsov~N.\,A.} see~Stupnikov~S.\,A.&&\\[.23pt]
\Avtors{Sokolov~I.\,A.} see~Chertok~A.\,V.&&\\
\end{tabular}
}
\pagebreak

\def\leftfootline{\small{\textbf{\thepage}
\hfill INFORMATIKA I EE PRIMENENIYA~--- INFORMATICS AND APPLICATIONS\ \ \ 2016\
\ \ volume~10\ \ \ issue\ 4}
}%
 \def\rightfootline{\small{INFORMATIKA I EE PRIMENENIYA~---
INFORMATICS AND APPLICATIONS\ \ \ 2016\ \ \ volume~10\ \ \ issue\ 4
\hfill \textbf{\thepage}}}

\def\leftkol{2016 AUTHOR INDEX} % ENGLISH ABSTRACTS}

\def\rightkol{2016 AUTHOR INDEX} %ENGLISH ABSTRACTS}


{\tabcolsep=3pt
\begin{tabular}{p{382pt}cc}
&\textbf{Issue} & \textbf{Page}\\[6pt]
\Avtors{Sopin~E.\,S.} see~Gaidamaka~Yu.\,V.&&\\
\Avtors{Strijov~V.\,V.} see~Goncharov~A.\,V.&&\\
\Avtors{Strijov~V.\,V.} see~Isachenko~R.\,V.&&\\
\Avtors{Strijov~V.\,V.} see~Karasikov~M.\,E.&&\\
\Avtors{Stupnikov~S.\,A., Briukhov~D.\,O., and Skvortsov~N.\,A.}
Co-lending systemic risk analysis over\linebreak
\\[-12pt]
\hspace*{23pt}heterogeneous data collections&1&23--33\\
\Avtors{Stupnikov~S.\,A.} see~Kalinichenko~L.\,A.&&\\
\Avtors{Suchkov~A.\,P.} see~Zatsarinny~A.\,A.&&\\
\Avtors{Timonina~E.\,E.} see~Grusho~A.\,A.&&\\
\Avtors{Titova~A.\,I.} see~Kudryavtsev~A.\,A.&&\\
\Avtors{Turlikov~A.\,M.} see~Ometov~A.\,Ya.&&\\
\Avtors{Tyrsin~A.\,N.\ and Serebryanskii~S.\,M.} Recognition of
dependences on the basis of inverse\linebreak
\\[-12pt]
\hspace*{23pt}mapping&2&58--64\\
\Avtors{Ulyanov~V.\,V.} see~Markov~A.\,S.&&\\
\Avtors{Ushakov~V.\,G.} Queueing system with working vacations and
hyperexponential input stream&2&92--97\\
\Avtors{Ushakov~V.\,G.} see~Leontyev~N.\,D.&&\\
\Avtors{Volnova~A.\,A.} see~Kalinichenko~L.\,A.&&\\
\Avtors{Yakovlev~O.\,A.\ and Gasilov~A.\,V.} Speeded-up stereo
matching using geodesic support weights&3&\hphantom{1}98--104\\
\Avtors{Zabezhailo~M.\,I.} see~Grusho~A.\,A.&&\\
\Avtors{Zabezhailo~M.\,I.} see~Grusho~A.\,A.&&\\
\Avtors{Zakharova~T.\,V.\ and Shestakov~O.\,V.} Precision analysis of
wavelet processing of aerodynamic\linebreak
\\[-12pt]
\hspace*{23pt}flow patterns&3&46--54\\
\Avtors{Zalizniak~Anna~A.\ and Kruzhkov~M.\,G.} Database
of~Russian impersonal verbal constructions&4&132--141\\
\Avtors{Zasypko~V.\,V.} see~Shnurkov~P.\,V.&&\\
\Avtors{Zatsarinny~A.\,A.\ and Suchkov~A.\,P.} Systems engineering
approaches to~the~establishment of\linebreak
\\[-12pt]
\hspace*{23pt}a~system for~decision support based
on~situational analysis&4&105--113\\
\Avtors{Zatsarinny~A.\,A.} see~Grusho~A.\,A.&&\\
\Avtors{Zatsman~I.\,M., Inkova~O.\,Yu., Kruzhkov~M.\,G., and
Popkova~N.\,A.} Representation of cross-\linebreak
\\[-12pt]
\hspace*{23pt}lingual knowledge about
connectors in supracorpora databases&1&106--118\\
\Avtors{Zatsman~I.\,M.} see~Minin~V.\,A.&&\\
\Avtors{Zeifman~A.\,I.} see~Korolev~V.\,Yu.&&\\
\Avtors{Zeifman~A.\,I.} see~Korolev~V.\,Yu.&&\\
\end{tabular}
}

%\thispagestyle{myheadings}
\def\leftfootline{\small{\textbf{\thepage}
\hfill INFORMATIKA I EE PRIMENENIYA~--- INFORMATICS AND APPLICATIONS\ \ \ 2016\
\ \ volume~10\ \ \ issue\ 4}
}%
 \def\rightfootline{\small{INFORMATIKA I EE PRIMENENIYA~---
INFORMATICS AND APPLICATIONS\ \ \ 2016\ \ \ volume~10\ \ \ issue\ 4
\hfill \textbf{\thepage}}}

 \label{end\stat}

\newpage

%\def\stat{rekl}
%\label{preobr}

%\def\tit{АКАДЕМИК ПУГАЧЁВ  ВЛАДИМИР СЕМЁНОВИЧ\\
%25.03.1911--25.03.1998}


%   \vspace*{-48pt}
%   \begin{center}\LARGE
%Академик Пугачёв  Владимир Семёнович\\ (25.03.1911--25.03.1998)
%   \end{center}
   
   %\vspace*{2.5mm}
   
   \begin{center}

{\prgsh\LARGE
ОБЪЯВЛЕНИЯ О КОНФЕРЕНЦИЯХ}

\end{center}
%\hrule

\vspace*{6pt}

   
   \vspace*{10mm}
   
   \thispagestyle{empty}

\noindent
\begin{tabular}{cc}
%\begin{center}
\multicolumn{1}{c}{\raisebox{-40pt}[0pt][0pt]{\mbox{%
\epsfxsize=33mm
\epsfbox{vspu.eps}
}}}
%\end{center}
&
\tabcolsep=0pt\begin{tabular}{c}
{\prg{\Large\textbf{XII Всероссийское совещание}}}\\[6pt]
{\prg{\Large\textbf{по проблемам управления}}}\\[12pt]
{\prg{\large 16--19 июня 2014~г.}}\\[6pt] 
{\prg{\large Институт проблем управления имени В.\,А.~Трапезникова РАН}}\\[6pt]
{\prg{\large Москва, Россия}}
\end{tabular}
\end{tabular}

\vspace*{60pt}

     
 { %\large    
 XII Всероссийское совещание по проблемам управления (ВСПУ XII), посвященное 75-летию 
Института проблем управления (ИПУ) имени В.\,А.~Трапезникова РАН, проводится 16--19~июня 
2014~г.\ 
в ИПУ РАН (г.~Москва, Россия). ВСПУ XII организуется ИПУ РАН при поддержке РФФИ, Отделения 
энергетики, машиностроения, механики и процессов управления Российской академии наук, 
Российского 
национального комитета по автоматическому управлению, Академии навигации и управ\-ле\-ния 
движением, 
Научного совета РАН по комплексным проблемам управления и автоматизации, Совета по 
мехатронике и робототехнике РАН. Официальный язык Совещания~--- русский.

\vspace*{24pt}
     
     \textbf{Направления работы}
     \begin{enumerate}[1.]
\item Теория систем управления
\item Управление подвижными объектами и навигация
\item Интеллектуальные системы управления
\item Управление в промышленности, транспортом и логистикой
\item Управление системами междисциплинарной природы
\item Средства измерения, вычислений и контроля в управлении
\item Системный анализ и принятие решений в задачах управления
\item Информационные технологии в управлении
\item Проблемы образования в области управления: современное содержание и технологии обучения
\end{enumerate}

\vspace*{24pt}

     Подробная информация о Совещании находится на сайте {\sf http://vspu2014.ipu.ru}. Срок 
окончательной подачи докладов через систему подачи докладов на сайте~--- \textbf{30~ноября} 
2013~г.
}

%\include{rekl-1}

%\end{document}

%   \vspace*{-48pt}

\begin{center}
\vspace*{6pt}
\mbox{%
\epsfxsize=53.502mm
\epsfbox{foto-1.eps}
}
\end{center}

\vspace*{6pt} %Академик


   \begin{center}
\fbox{\Large\textbf{Профессор Игорь Алексеевич Ушаков}}\\[12pt]
\textbf{\large 22.01.1935--27.02.2015}
   \end{center}


   %\vspace*{2.5mm}

   \vspace*{5mm}

   \thispagestyle{empty}

%\

%\vspace*{-12pt}


Редакционный совет и редакционная коллегия журнала <<Информатика и~её применения>> с~глубоким прискорбием извещают, что 27~февраля 2015~г.\ после тяжелой
и~продолжительной болезни скончался Игорь Алексеевич Ушаков~--- доктор технических наук, профессор, член редколлегии журнала <<Информатика и ее применения>>.

Игорь Алексеевич Ушаков окончил Московский авиационный институт, в~1963~г.\ защитил кандидатскую, а~в~1968~г.~--- докторскую диссертацию. С~1958 по 1989~гг.\ работал в~ряде научно-исследовательских организаций СССР, в~том числе руководил отделами в~НИИ АА и~ВЦ АН СССР; с 1969 по 1989 гг. преподавал в~МФТИ (был профессором, а~затем заведующим кафедрой) и~в~МЭИ. С~1989~г.~---- в~США: являлся профессором университета Дж.\ Вашингтона, университета Дж.\ Мэйсона и~Калифорнийского университета, сотрудником компаний MCI, Qualcomm и Hughes.

И.\,А.~Ушаков с момента основания журнала <<Надежность и~контроль качества>> был заместителем ответственного редактора, а~затем на протяжении многих лет членом редколлегии. В~2006~г.\ основал электронный международный журнал ``Reliability: Theory \& Application'', главным редактором которого оставался до конца жизни.

Учебниками и справочниками по теории надежности, написанными И.\,А.~Ушаковым, пользовались и~пользуются несколько поколений ученых и~специалистов в~разных странах мира.

Игорь Алексеевич всегда уделял огромное внимание работе с~молодежью; более~50 его учеников защитили докторские и~кандидатские диссертации.

И.\,А.~Ушаков вел активную научно-про\-све\-ти\-тель\-скую деятельность. В~частности, он был одним из организаторов и~руководителей Московского кабинета качества и~надежности при Политехническом музее (целью этого Кабинета было оказание консультаций работникам промышленных предприятий и~чтение курсов лекций для инженеров, занимающихся проблемой надежности). Находясь в~США, И.\,А.~Ушаков создал международный ин\-тер\-нет-фо\-рум им.\ Б.\,В.~Гнеденко, объединивший около~400~видных специалистов по приложениям теории вероятностей и~математической статистики, преимущественно в~об\-ласти теории надежности и~анализа риска, из десятков стран мира; коллективным членов этого Форума является и~наш журнал. Цели Форума~--- содействие контактам между специалистами из разных стран, организация обмена профессиональными 
новостями и~информацией (новые публикации, предстоящие события и~др.). Также необходимо отметить большое число на\-уч\-но-по\-пу\-ляр\-ных работ, опубликованных И.\,А.~Ушаковым.

И.\,А.~Ушаков обладал большим личным обаянием, имел широкий круг интересов. Все знавшие И.\,А.~Ушакова всегда будут помнить его как замечательного ученого и~прекрасного человека.

\bigskip

Редакционный совет и редакционная коллегия журнала <<Информатика и~её применения>> 
выражают глубокие соболезнования родным и близким покойного, всем, кто его знал и~работал с~ним.



%\end{document}

%\include{IPPM-25}

\def\stat{cont-rus}
{%\hrule\par
%\vskip 7pt % 7pt
\vspace*{-24pt}
\raggedleft\Large \bf%\baselineskip=3.2ex
Правила подготовки рукописей  для публикации в журнале
<<Информатика~и~её~применения>> \vskip 8pt
    \hrule
    \par
\vskip 14pt plus 6pt minus 3pt }

\label{st\stat}

\def\tit{\ }

\def\aut{\ }
\def\auf{\ }

\def\leftkol{\ }
% Правила подготовки рукописей  для публикации в журнале
%<<Информатика и её применения>>

\def\rightkol{\ }
%Правила подготовки рукописей  для публикации в журнале
%<<Информатика и её применения>>}


\titele{\tit}{\aut}{\auf}{\leftkol}{\rightkol}


\vspace*{-60pt}
{ %\small

Журнал <<Информатика и её применения>>
публикует теоретические, обзорные и дискуссионные статьи,
посвященные научным исследованиям и разработкам в области
информатики и ее приложений.

Журнал издается на русском языке. По специальному решению
редколлегии отдельные статьи могут печататься на английском языке.

Тематика журнала охватывает следующие направления:
\begin{itemize}
\item теоретические основы информатики;\\[-15pt]
      \item
математические методы исследования сложных систем и процессов;\\[-15pt]
           \item
информационные системы и сети;\\[-15pt]
                \item
информационные технологии;\\[-15pt]
                     \item
архитектура и программное обеспечение вычислительных комплексов и сетей.\\[-15pt]
\end{itemize}


\noindent
\begin{enumerate}[1.]
\item В журнале печатаются статьи, содержащие результаты, ранее не опубликованные и
не предназначенные к одновременной публикации в других изданиях.

%Публикация не должна нарушать закон об авторских правах.
Публикация предоставленной автором(ами) рукописи не должна нарушать 
положений глав~69, 70 раздела~VII части~IV Гражданского кодекса, 
которые определяют права на результаты интеллектуальной деятельности 
и~средства индивидуализации, в~том числе авторские права, в~РФ.

Ответственность за нарушение авторских прав, в~случае предъявления претензий к~редакции журнала,  
несут авторы статей.



Направляя рукопись в редакцию, авторы сохраняют свои права на данную
рукопись и при этом передают учредителям и редколлегии журнала неисключительные права на
издание статьи на русском языке 
(или на языке статьи, если он отличен от рус\-ско\-го) и~на перевод ее на английский
язык, а~также на
ее распространение в России и за рубежом. 
Каждый автор должен представить в~редакцию подписанный 
с~его стороны <<Лицензионный договор о~передаче неисключительных прав 
на использование произведения>>, текст которого размещен по адресу 
{\sf http://www.ipiran.ru/publications/licence.doc}. 
Этот договор может быть пред\-став\-лен в~бумажном (в~2-х экз.)\ 
или в~электронном виде (отсканированная копия заполненного и~подписанного документа).




Редколлегия вправе запросить у авторов экспертное заключение о возможности
пуб\-ли\-ка\-ции пред\-став\-лен\-ной статьи в открытой печати.\\[-13.5pt]

\item К статье прилагаются данные автора (авторов) (см.\ п.~8). При наличии нескольких
авторов указывается фамилия автора, ответственного за переписку с редакцией.\\[-13.5pt]

\item Редакция журнала осуществляет экспертизу присланных статей в соответствии с
принятой в журнале процедурой рецензирования.

Возвращение рукописи на доработку не означает ее принятия к печати.

Доработанный вариант с ответом на замечания рецензента необходимо прислать в
редакцию.\\[-13.5pt]

\item Решение редколлегии о публикации статьи или ее отклонении сообщается авторам.

Редколлегия может также направить авторам текст рецензии на их статью. Дискуссия по
поводу отклоненных статей не ведется.\\[-13.5pt]

%\pagebreak

\item Редактура статей высылается авторам для просмотра. Замечания к редактуре должны
быть присланы авторами в кратчайшие сроки.\\[-13.5pt]

\item Рукопись предоставляется в электронном виде в форматах MS WORD (.doc или
.docx) или \LaTeX\  (.tex), дополнительно~--- в формате .pdf, на дискете, лазерном диске
или электронной почтой. Предоставление бумажной рукописи необязательно.\\[-13.5pt]

\item При подготовке рукописи в MS Word рекомендуется использовать следующие
настройки.

Параметры страницы:
формат~--- А4; ориентация~--- книжная; поля (см): внутри~--- 2,5, снаружи~--- 1,5,
сверху~--- 2, снизу~--- 2, от края до нижнего колонтитула~--- 1,3.

Основной текст: стиль~--- <<Обычный>>, шрифт~--- Times New Roman, размер~---
14~пунк\-тов, абзацный отступ~--- 0,5~см, 1,5~интервала, выравнивание~--- по ширине.

\pagebreak

\def\leftkol{Правила подготовки рукописей  для публикации в журнале
<<Информатика и её применения>>}

\def\rightkol{Правила подготовки рукописей  для публикации в журнале
<<Информатика и её применения>>}



Рекомендуемый объем рукописи~--- не свыше 10~страниц указанного формата.
При превышении указанного объема редколлегия вправе потребовать от 
автора сокращения объема рукописи.


Сокращения слов, помимо стандартных, не допускаются. Допускается минимальное
количество аббревиатур.


Все страницы рукописи нумеруются.

Шаблоны оформления представлены в интернете:

\noindent
 {\sf
http://www.ipiran.ru/journal/template\_iiep\_ssi\_2024.zip}\\[-14pt]

\item Статья должна содержать следующую информацию на {\bfseries\textit{русском и
английском языках}}:\\[-16pt]

\begin{itemize}
\item название статьи;\\[-15pt]
\item Ф.И.О.\ авторов, на английском можно только имя и фамилию;\\[-15pt]
\item место работы, с указанием почтового адреса организации и электронного адреса каждого
автора;\\[-15pt]
\item сведения об авторах, в соответствии с форматом, образцы которого
представлены на страницах:



\def\leftfootline{\small{\textbf{\thepage}
\hfill ИНФОРМАТИКА И ЕЁ ПРИМЕНЕНИЯ\ \ \ том\ 18\ \ \ выпуск\ 3\ \ \ 2024}
}%
 \def\rightfootline{\small{ИНФОРМАТИКА И ЕЁ ПРИМЕНЕНИЯ\ \ \ том\ 18\ \ \ выпуск\ 3\ \ \ 2024
\hfill \textbf{\thepage}}}



{\sf http://www.ipiran.ru/journal/issues/2013\_07\_01/authors.asp} и

{\sf http://www.ipiran.ru/journal/issues/2013\_07\_01\_eng/authors.asp};
\item аннотация (не менее 100~слов на каждом из языков). Аннотация~--- это краткое
резюме работы, которое может публиковаться отдельно. Она является основным
источником информации в~ин\-фор\-ма\-ци\-он\-ных системах и базах данных. Английская
аннотация должна быть оригинальной, может не быть дословным переводом русского
текста и должна быть написана хорошим английским языком. В~аннотации не должно
быть ссылок на литературу и, по возможности, формул;\\[-15pt]
\item ключевые слова~--- желательно из принятых в мировой
на\-уч\-но-тех\-ни\-че\-ской литературе тематических тезаурусов. Предложения не
могут быть ключевыми словами;\\[-15pt]
\item источники финансирования работы (ссылки на гранты, проекты,
поддерживающие организации и~т.\,п.).
\end{itemize}



%\pagebreak

\item  Требования к спискам литературы.\\[-14pt]

Ссылки на литературу в тексте статьи нумеруются (в квадратных скобках) и
располагаются в каждом из списков литературы в порядке  первых упоминаний. Если источник имеет DOI и/или EDN,
то их необходимо указывать.

Списки литературы представляются в двух вариантах:\\[-14pt]


\noindent
\begin{enumerate}[(1)]
\item \textbf{Список литературы к русскоязычной части}. Русские и английские
работы~---  на языке и в алфавите оригинала;\\[-14.5pt]
\item  \textbf{References}. Русские работы и работы на других языках~--- в латинской
транслитерации с переводом на английский язык; английские работы и работы на других
языках~--- на языке оригинала.
\end{enumerate}

Необходимо для составления списка ``References'' пользоваться размещенной на сайте
{\sf http://www. translit.net/ru/bgn/} бесплатной программой транслитерации русского
 текста в~латиницу. %, при этом в~за\-клад\-ке <<варианты\ldots>> следует выбратьопцию BGN.

Список литературы ``References'' приводится полностью отдельным блоком, повторяя все
позиции из списка литературы к русскоязычной части, независимо от того, имеются или
нет в нем иностранные источники. Если в списке литературы к русскоязычной части есть
ссылки на иностранные публикации, набранные латиницей, они полностью повторяются в
списке ``References''.

Ниже приведены примеры ссылок на различные виды публикаций в списке ``References''.

\def\leftfootline{\small{\textbf{\thepage}
\hfill ИНФОРМАТИКА И ЕЁ ПРИМЕНЕНИЯ\ \ \ том\ 18\ \ \ выпуск\ 3\ \ \ 2024}
}%
 \def\rightfootline{\small{ИНФОРМАТИКА И ЕЁ ПРИМЕНЕНИЯ\ \ \ том\ 18\ \ \ выпуск\ 3\ \ \ 2024
\hfill \textbf{\thepage}}}

{\small

\noindent
\textbf{Описание статьи из журнала:}

\Aue{Zagurenko, A.\,G., V.\,A.~Korotovskikh, A.\,A.~Kolesnikov, A.\,V.~Timonov, and D.\,V.~Kardymon}. 2008.
Tekhniko-ekonomicheskaya optimizatsiya dizayna gidrorazryva plasta [Technical and
economic optimization of the design
of hydraulic fracturing]. \textit{Neftyanoe hozyaystvo} [\textit{Oil Industry}] 11:54--57.

\Aue{Zhang, Z., and D.~Zhu}. 2008. Experimental research on the localized
electrochemical micromachining. \textit{Russ. J.~Electrochem.}  44(8):926--930.
{\sf doi:10.1134/S1023193508080077}.

\noindent
\textbf{Описание статьи из электронного журнала:}

\Aue{Swaminathan, V., E.~Lepkoswka-White, and B.\,P.~Rao}. 1999. Browsers or buyers in cyberspace? An
investigation of electronic factors influencing electronic exchange. \textit{JCMC}
5(2). Available at: {\sf http://www.ascusc.org/jcmc/vol5/issue2/} (accessed April~28, 2011).

\def\leftkol{Правила подготовки рукописей  для публикации в журнале
<<Информатика и её применения>>}

\def\rightkol{Правила подготовки рукописей  для публикации в журнале
<<Информатика и её применения>>}


\noindent
\textbf{Описание статьи из продолжающегося издания (сборника трудов):}

\Aue{Astakhov, M.\,V., and T.\,V.~Tagantsev}. 2006. Eksperimental'noe
issledovanie prochnosti soedineniy ``stal'--kompozit'' [Experimental study of
the strength of joints ``steel--composite'']. \textit{Trudy MGTU
``Matematicheskoe modelirovanie slozhnykh tekh\-ni\-che\-skikh sistem''}
[\textit{Bauman MSTU ``Mathematical Modeling of Complex Technical
Systems'' Proceedings}]. 593:125--130.


\pagebreak



\noindent
\textbf{Описание материалов конференций:}

\Aue{Usmanov, T.\,S., A.\,A.~Gusmanov, I.\,Z.~Mullagalin, R.\,Ju.~Muhametshina, A.\,N.~Chervyakova, and
A.\,V.~Sveshnikov}. 2007. Osobennosti proektirovaniya razrabotki mestorozhdeniy
s primeneniem gidrorazryva
plasta [Features of the design of field development with the use of hydraulic fracturing].
\textit{Trudy 6-go
Mezhdu\-na\-rod\-no\-go Simpoziuma ``Novye resursosberegayushchie tekhnologii nedropol'zovaniya i povysheniya
neftegazootdachi''} [\textit{6th  Symposium (International) ``New Energy Saving Subsoil Technologies and
the Increasing of the Oil and Gas Impact'' Proceedings}]. Moscow. 267--272.



\def\leftfootline{\small{\textbf{\thepage}
\hfill ИНФОРМАТИКА И ЕЁ ПРИМЕНЕНИЯ\ \ \ том\ 18\ \ \ выпуск\ 3\ \ \ 2024}
}%
 \def\rightfootline{\small{ИНФОРМАТИКА И ЕЁ ПРИМЕНЕНИЯ\ \ \ том\ 18\ \ \ выпуск\ 3\ \ \ 2024
\hfill \textbf{\thepage}}}



\noindent
\textbf{Описание книги (монографии, сборники):}



Lindorf, L.\,S., and L.\,G.~Mamikoniants, eds. 1972.
\textit{Ekspluatatsiya turbogeneratorov s neposredstvennym
okhlazhdeniem} [\textit{Operation of turbine generators with direct cooling}].
Moscow: Energy Publs. 352~p.


\Aue{Latyshev, V.\,N.} 2009. \textit{Tribologiya rezaniya. Kn.~1: Friktsionnye protsessy
pri rezanii metallov}
[\textit{Tribology of cutting. Vol.~1: Frictional processes in metal cutting}]. Ivanovo: Ivanovskii
State Univ. 108~p.

\def\leftkol{Правила подготовки рукописей  для публикации в журнале
<<Информатика и её применения>>}

\def\rightkol{Правила подготовки рукописей  для публикации в журнале
<<Информатика и её применения>>}

\noindent
\textbf{Описание переводной книги}
(в списке литературы к русскоязычной части необходимо указать:~/ Пер.\ с англ.~---
после названия книги, а в конце ссылки указать оригинал книги в круглых скобках):
\begin{enumerate}[1.]
\item  В русскоязычной части:

\def\leftfootline{\small{\textbf{\thepage}
\hfill ИНФОРМАТИКА И ЕЁ ПРИМЕНЕНИЯ\ \ \ том\ 18\ \ \ выпуск\ 3\ \ \ 2024}
}%
 \def\rightfootline{\small{ИНФОРМАТИКА И ЕЁ ПРИМЕНЕНИЯ\ \ \ том\ 18\ \ \ выпуск\ 3\ \ \ 2024
\hfill \textbf{\thepage}}}

\Au{Тимошенко С.\,П., Янг Д.\,Х., Уивер~У.}
Колебания в инженерном деле~/ Пер.\ с англ.~--- М.: Машиностроение, 1985. 472~с.
(\Au{Timoshenko~S.\,P., Young~D.\,H., Weaver~W.}
Vibration problems in engineering.~--- 4th ed.~--- New York, NY, USA: Wiley, 1974. 521~p.)\\[-13.5pt]
\item  В англоязычной части:

\Aue{Timoshenko, S.\,P., D.\,H.~Young, and W.~Weaver}.
1974. \textit{Vibration problems in engineering}. 4th ed. New York: 
Wiley. 521~p.
\end{enumerate}

\vspace*{-3pt}


\noindent
\textbf{Описание неопубликованного документа:}


\Aue{Latypov, A.\,R., M.\,M.~Khasanov, and V.\,A.~Baikov}.
2004 (unpubl.). Geologiya i~dobycha (NGT GiD) [Geology and production (NGT GiD)]. Certificate on official registration of the computer program
No.\,2004611198. 

\noindent
\textbf{Описание интернет-ресурса:}


Pravila tsitirovaniya istochnikov [Rules for the citing of sources]. Available at: {\sf
http://www.scribd.com/doc/1034528/} (accessed February~7, 2011).

%\pagebreak

\noindent
\textbf{Описание диссертации или автореферата диссертации:}

\Aue{Semenov, V.\,I.}
2003. Matematicheskoe modelirovanie plazmy v sisteme kompaktnyy tor [Mathematical
modeling of the plasma in the compact torus].  Moscow.  D.Sc.\ Diss. 272~p.

\Aue{Kozhunova, O.\,S.} 2009. Tekhnologiya razrabotki semanticheskogo
slovarya informatsionnogo monitoringa [Technology of development of
semantic dictionary of information monitoring system].  Moscow: IPI RAN. PhD Thesis. 23~p.


\noindent
\textbf{Описание ГОСТа:}

GOST 8.586.5-2005. 2007. Metodika vypolneniya izmereniy. Izmerenie raskhoda i~kolichestva zhidkostey i~gazov
s~pomoshch'yu standartnykh suzhayushchikh ustroystv [Method of measurement.
Measurement of flow rate and volume of liquids and gases by means of orifice devices]. Moscow:
Standardinform  Publs. 10~p.

\noindent
\textbf{Описание патента:}

\Aue{Bolshakov, M.\,V., A.\,V.~Kulakov, A.\,N.~Lavrenov, and M.\,V.~Palkin}.
2006. Sposob orientirovaniya po krenu letatel'nogo
apparata s opti\-che\-skoy golovkoy
samonavedeniya [The way to orient on the roll of aircraft with optical homing head].
Patent RF No.\,2280590.
}

\item Присланные в редакцию материалы авторам не возвращаются.\\[-13.5pt]

\item При отправке файлов по электронной почте просим придерживаться следующих
правил:
\begin{itemize}
\item указывать в поле subject (тема) название журнала и фамилию автора;\\[-13.5pt]
\item указывать в тексте письма название статьи, авторов и~журнал, в~который направляется статья;\\[-13.5pt]
\item использовать attach (присоединение);\\[-13.5pt]
\item в состав электронной версии статьи должны входить: файл, содержащий текст
статьи, и файл(ы), содержащий(е) иллюстрации.\\[-13.5pt]
\end{itemize}

\item Журнал <<Информатика и её применения>> является некоммерческим изданием.
Плата за публикацию не взимается, гонорар авторам не выплачивается.
\end{enumerate}



\def\leftfootline{\small{\textbf{\thepage}
\hfill ИНФОРМАТИКА И ЕЁ ПРИМЕНЕНИЯ\ \ \ том\ 18\ \ \ выпуск\ 3\ \ \ 2024}
}%
 \def\rightfootline{\small{ИНФОРМАТИКА И ЕЁ ПРИМЕНЕНИЯ\ \ \ том\ 18\ \ \ выпуск\ 3\ \ \ 2024
\hfill \textbf{\thepage}}}


\vspace*{-1mm}

\begin{center}

\textbf{Адрес редакции журнала <<Информатика и её применения>>:} \\




Москва 119333, ул.~Вавилова, д.~44, корп.~2, ФИЦ ИУ РАН\\[-10pt]

\

Тел.: +7\,(499)\,135-86-92\ \ Факс:  +7\,(495)\,930-45-05\\[-10pt]

 \

e-mail:   {\sf iiep@frccsc.ru} (Стригина Светлана Николаевна)\\[-10pt]

\

{\sf http://www.ipiran.ru/journal/issues/}
\end{center}
}


\def\leftkol{Правила подготовки рукописей  для публикации в журнале
<<Информатика и её применения>>}

\def\rightkol{Правила подготовки рукописей  для публикации в журнале
<<Информатика и её применения>>}


\def\leftfootline{\small{\textbf{\thepage}
\hfill ИНФОРМАТИКА И ЕЁ ПРИМЕНЕНИЯ\ \ \ том\ 18\ \ \ выпуск\ 3\ \ \ 2024}
}%
 \def\rightfootline{\small{ИНФОРМАТИКА И ЕЁ ПРИМЕНЕНИЯ\ \ \ том\ 18\ \ \ выпуск\ 3\ \ \ 2024
\hfill \textbf{\thepage}}} 
\def\stat{podg-e}
{%\hrule\par
%\vskip 7pt % 7pt
\vspace*{-24pt}
\raggedleft\Large \bf%\baselineskip=3.2ex
Requirements for manuscripts submitted to Journal
``Informatics~and~Applications'' \vskip 8pt
    \hrule
    \par
\vskip 21pt plus 6pt minus 3pt }

\label{st\stat}

\def\tit{\ }

\def\aut{\ }
\def\auf{\ }

\def\leftkol{\ }

\def\rightkol{\ }
%Requirements for manuscripts submitted to Journal
%``Informatics~and~Applications''}

\titele{\tit}{\aut}{\auf}{\leftkol}{\rightkol}

\def\leftfootline{\small{\textbf{\thepage}
\hfill INFORMATIKA I EE PRIMENENIYA~--- INFORMATICS AND APPLICATIONS\ \ \ 2019\
\ \ volume~13\ \ \ issue\ 4}
}%
 \def\rightfootline{\small{INFORMATIKA I EE PRIMENENIYA~--- INFORMATICS AND APPLICATIONS\ \ \ 2019\ \ \ volume~13\ \ \ issue\ 4
\hfill \textbf{\thepage}}}

\vspace*{-60pt}

{\small

\noindent
Journal ``Informatics and Applications'' (Inform.\ Appl.)
publishes theoretical, review, and discussion
articles on the research and development in the
field of informatics and its applications.

The journal is published in Russian.
By a special decision of the editorial
board, some articles can be published in English.


The topics covered include the following areas:
\begin{itemize}
               \item
     theoretical fundamentals of informatics; \\[-14pt]
\item
mathematical methods for studying complex systems and processes; \\[-14pt]
\item
information systems and networks;\\[-14pt]
\item
information technologies; and \\[-14pt]
\item
architecture and software of computational complexes and networks. \\[-14pt]
\end{itemize}

\noindent
\begin{enumerate}[1.]
\item The Journal publishes original articles which have not been published before and are not
intended for simultaneous publication in other editions. An article submitted to the Journal must not violate the
Copyright law. Sending the manuscript to the Editorial Board, the authors retain all rights of the
owners of the manuscript and transfer the nonexclusive rights to publish the article in Russian
(or the language of the article, if not Russian) and its distribution in Russia and abroad to the
Founders and the Editorial Board. Authors should submit a letter to the Editorial Board in the
following form:

{\bfseries\textit{Agreement on the transfer of rights to publish:}}

``\textit{We, the undersigned authors of the manuscript ``\ldots'', pass to the
Founder and the Editorial Board of the Journal ``Informatics and Applications''
the nonexclusive right to publish the manuscript of the article in Russian (or
in English) in both print and electronic versions of the Journal. We affirm
that this publication does not violate the Copyright of other persons or
organizations.}

\textit{Author(s) signature(s): (name(s), address(es), date).}

This agreement should be submitted in paper form or in the form of a scanned copy (signed by
the authors).


%The Editorial Board has the right to request from the authors an official expert conclusion that
%the submitted article has no secret data prohibited for publication. \\[-13.5pt]
\item
A submitted article should be attached with \textbf{the data on the author(s)} (see item~8). If
there are several authors, the contact person should be indicated who is responsible for
correspondence with the Editorial Board and other authors about revisions and final approval
of the proofs.\\[-13.5pt]

\item The Editorial Board of the Journal examines the article according to the established
reviewing procedure. If the authors receive their article for correction after reviewing, it does not
mean that the article is approved for publication. The corrected article should be sent to the
Editorial Board for the subsequent review and approval.\\[-13.5pt]

\item The decision on the article publication or its rejection is communicated to the authors. The
Editorial Board may also send the reviews on the submitted articles to the authors. Any
discussion upon the rejected articles is not possible.\\[-13.5pt]

\item The edited articles will be sent to the authors for proofread. The comments of the authors
to the edited text of the article should be sent to the Editorial Board as soon as possible.\\[-13.5pt]

\item The manuscript of the article should be presented electronically in the MS WORD (.doc or
.docx) or \LaTeX\ (.tex) formats, and additionally in the .pdf format. All documents
 may be sent
by e-mail or provided on a CD or diskette. A~hard copy submission is not necessary.\\[-13.5pt]

\item The recommended typesetting instructions for manuscript.

Pages parameters: format A4, portrait orientation, document margins (cm): left~--- 2.5, right~---
1.5, above~--- 2.0, below~--- 2.0, footer 1.3.

Text: font~---Times New Roman, font size~--- 14, paragraph indent~--- 0.5, line spacing~--- 1.5,
justified alignment.

The recommended manuscript size: not more than 15~pages of the specified format.
If the specified size exceeded, the editorial board is entitled to require the author
to reduce the manuscript.

Use only standard abbreviations. Avoid  abbreviations in the title and
abstract. The full term for which an abbreviation stands should precede
its first use in the text unless it is a standard unit of measurement.

All pages of the manuscript should be numbered.

The templates for the manuscript typesetting are presented on site: {\sf
http://www.ipiran.ru/journal/template.doc}.\\[-13.5pt]


%\def\leftkol{Requirements for manuscripts submitted to Journal
%``Informatics~and~Applications''}

\item The articles should enclose data both in \textbf{Russian and English}:
\begin{itemize}
\item title;\\[-13.5pt]
\item author's name and surname;\\[-13.5pt]
\item affiliation~--- organization, its address with ZIP code, city, country, and
official e-mail address;\\[-13.5pt]
\item data on authors according to the format: (see site)

{\sf http://www.ipiran.ru/journal/issues/2013\_07\_01/authors.asp}  and

{\sf  http://www.ipiran.ru/journal/issues/2013\_07\_01\_eng/authors.asp};\\[-13.5pt]

\pagebreak

\def\leftfootline{\small{\textbf{\thepage}
\hfill INFORMATIKA I EE PRIMENENIYA~--- INFORMATICS AND APPLICATIONS\ \ \ 2019\
\ \ volume~13\ \ \ issue\ 4}
}%
 \def\rightfootline{\small{INFORMATIKA I EE PRIMENENIYA~--- INFORMATICS AND APPLICATIONS\ \ \ 2019\ \ \ volume~13\ \ \ issue\ 4
\hfill \textbf{\thepage}}}


%\def\leftkol{Requirements for manuscripts submitted to Journal
%``Informatics~and~Applications''}

%\def\rightkol{Requirements for manuscripts submitted to Journal
%``Informatics~and~Applications''}



\item abstract (not less than 100 words) both in Russian and in English. Abstract is a short
summary of the article that can be published separately. The abstract is the
main source of information on the article and it could be included in leading information
systems and data bases. The abstract in English has to be an original text and should
not be an exact translation of the Russian one. Good English is required.
In abstracts, avoid references and formulae;\\[-13.5pt]
\item indexing is performed on the basis of keywords. The use of keywords from the
internationally accepted thematic Thesauri is recommended.

%\def\leftkol{Requirements for manuscripts submitted to Journal
%``Informatics~and~Applications''}

%\def\rightkol{Requirements for manuscripts submitted to Journal
%``Informatics~and~Applications''}

Important! Keywords must not be sentences;
\item Acknowledgments.
\end{itemize}

\item References. Russian references have to be presented both in English translation and Latin
transliteration (refer {\sf http://www.translit.net/ru/bgn/}).

Please take into account the following examples of Russian references appearance:

\noindent
\textbf{Article in journal:}

\Aue{Zhang, Z., and D.~Zhu}. 2008. Experimental research on the localized electrochemical
micromachining.
\textit{Rus. J.~Electrochem.}  44(8):926--930. {\sf doi:10.1134/S1023193508080077}.


\noindent
\textbf{Journal article in electronic format:}

\Aue{Swaminathan, V., E.~Lepkoswka-White, and B.\,P.~Rao}. 1999. Browsers or buyers in
cyberspace? An
investigation of electronic factors influencing electronic exchange. \textit{JCMC}
5(2). Available at: {\sf http://www.ascusc.org/jcmc/vol5/issue2/} (accessed April~28, 2011).




\noindent
\textbf{Article from the continuing publication (collection of works, proceedings):}

\Aue{Astakhov, M.\,V., and T.\,V.~Tagantsev}. 2006. Eksperimental'noe
issledovanie prochnosti soedineniy ``stal'--kompozit'' [Experimental study of
the strength of joints ``steel--composite'']. \textit{Trudy MGTU
``Matematicheskoe modelirovanie slozhnykh tekh\-ni\-che\-skikh sistem''}
[\textit{Bauman MSTU ``Mathematical Modeling of Complex Technical
Systems'' Proceedings}]. 593:125--130.

\def\leftfootline{\small{\textbf{\thepage}
\hfill INFORMATIKA I EE PRIMENENIYA~--- INFORMATICS AND APPLICATIONS\ \ \ 2019\
\ \ volume~13\ \ \ issue\ 4}
}%
 \def\rightfootline{\small{INFORMATIKA I EE PRIMENENIYA~--- INFORMATICS AND APPLICATIONS\ \ \ 2019\ \ \ volume~13\ \ \ issue\ 4
\hfill \textbf{\thepage}}}

\def\leftkol{Requirements for manuscripts submitted to Journal
``Informatics~and~Applications''}

\def\rightkol{Requirements for manuscripts submitted to Journal
``Informatics~and~Applications''}

\noindent
\textbf{Conference proceedings:}

\Aue{Usmanov, T.\,S., A.\,A.~Gusmanov, I.\,Z.~Mullagalin, R.\,Ju.~Muhametshina,
A.\,N.~Chervyakova, and
A.\,V.~Sveshnikov}. 2007. Osobennosti proektirovaniya razrabotki mestorozhdeniy
s primeneniem gidrorazryva
plasta [Features of the design of field development with the use of hydraulic fracturing].
\textit{Trudy 6-go
Mezhdu\-na\-rod\-no\-go Simpoziuma ``Novye resursosberegayushchie tekhnologii
nedropol'zovaniya i povysheniya
neftegazootdachi''} [\textit{6th  Symposium (International) ``New Energy Saving Subsoil
Technologies and
the Increasing of the Oil and Gas Impact'' Proceedings}]. Moscow. 267--272.


\noindent
\textbf{Books and other monographs:}




Lindorf, L.\,S., and L.\,G.~Mamikoniants, eds. 1972.
\textit{Ekspluatatsiya turbogeneratorov s neposredstvennym
okhlazhdeniem} [\textit{Operation of turbine generators with direct cooling}].
Moscow: Energy Publs. 352~p.


%\Aue{Latyshev, V.\,N.} 2009. \textit{Tribologiya rezaniya. Kn.~1: Frikcionnye prosessy
%pri rezanii metallov}
%[\textit{Tribology of cutting. Vol.~1: Frictional processes in metal cutting}]. Ivanovo: Ivanovskii
%State Univ. 108~p.


%\noindent
%\textbf{Unpublished material:}

%\Aue{Latypov, A.\,R., M.\,M.~Khasanov, and V.\,A.~Baikov}.
%2004. Geology and production (NGT GiD). Certificate on official registration of the computer
%program
%No.\,2004611198. (In Russian, unpubl.)

%\noindent
%\textbf{Internet-source:}

%APA Style. 2011. Available at: {\sf http://www.apastyle.org/apa-style-help.aspx} (accessed
%February~5, 2011).

%Pravila citirovaniya istochnikov [Rules for the citing of sources]. Available at: {\sf
%http://www.scribd.com/doc/1034528/} (accessed February~7, 2011).


\noindent
\textbf{Dissertation and Thesis:}

%\Aue{Semenov, V.\,I.}
%2003. Matematicheskoe modelirovanie plazmy v sisteme kompaktnyy tor. [Mathematical
%modeling of the plasma in the compact torus]. D.Sc.\ Diss. Moscow. 272~p.

\Aue{Kozhunova, O.\,S.} 2009. Tekhnologiya razrabotki semanticheskogo
slovarya informatsionnogo monitoringa [Technology of development of
semantic dictionary of information monitoring system]. PhD Thesis. Moscow: IPI RAN. 23~p.


\noindent
\textbf{State standards and patents:}

GOST 8.586.5-2005. 2007. Metodika vypolneniya izmereniy. Izmerenie raskhoda i~kolichestva
zhidkostey i gazov 
s~pomoshch'yu standartnykh suzhayushchikh ustroystv [Method of measurement.
Measurement of flow rate and volume of liquids and gases by means of orifice devices]. M.:
Standardinform
Publs. 10~p.

%\noindent
%\textbf{Patent:}

\Aue{Bolshakov, M.\,V., A.\,V.~Kulakov, A.\,N.~Lavrenov, and M.\,V.~Palkin}.
2006. Sposob orientirovaniya po krenu letatel'nogo
apparata s opti\-che\-skoy golovkoy
samonavedeniya [The way to orient on the roll of aircraft with optical homing head].
Patent RF No.\,2280590.

References in Latin transcription are presented in the original language.

References in the text are numbered according to the order of their
first appearance; the number is
placed in square brackets. All items from the reference list should be
cited.\\[-13.5pt]

\item Manuscripts and additional materials are not returned to Authors by the Editorial Board.\\[-13.5pt]

\item Submissions of files by e-mail must include:\\[-13.5pt]
\begin{itemize}
\item   the journal title and author's name in the ``Subject'' field; \\[-13.5pt]
\item   an article and additional materials have to be attached using the ``attach'' function;\\[-13.5pt]
\item   an electronic version of the article should contain the file with the text and a separate file
with figures.\\[-13.5pt]
\end{itemize}

\item ``Informatics and Applications'' journal is not a profit publication. There are no
charges for the authors as well as there are no royalties.\\[-13.5pt]
\end{enumerate}

\def\leftfootline{\small{\textbf{\thepage}
\hfill INFORMATIKA I EE PRIMENENIYA~--- INFORMATICS AND APPLICATIONS\ \ \ 2019\
\ \ volume~13\ \ \ issue\ 4}
}%
 \def\rightfootline{\small{INFORMATIKA I EE PRIMENENIYA~--- INFORMATICS AND APPLICATIONS\ \ \ 2019\ \ \ volume~13\ \ \ issue\ 4
\hfill \textbf{\thepage}}}

\def\leftkol{Requirements for manuscripts submitted to Journal
``Informatics~and~Applications''}

\def\rightkol{Requirements for manuscripts submitted to Journal
``Informatics~and~Applications''}


%\vspace*{5mm}


\begin{center}
\textbf{Editorial Board address:} \\

%ABOUT AUTHORS



FRC CSC RAS, 44, block~2, Vavilov Str., Moscow 119333, Russia\\[-10pt]

\

Ph.: +7\,(499)\,135\,86\,92,\ \ Fax: +7\,(495)\,930\,45\,05\\[-10pt]

\

 e-mail: {\sf rust@ipiran.ru} (to Prof.\ Rustem Seyful-Mulyukov)\\[-10pt]

\

 {\sf http://www.ipiran.ru/english/journal.asp}
\end{center}
 }
%\thispagestyle{myheadings}

\def\leftkol{Requirements for manuscripts submitted to Journal
``Informatics~and~Applications''}

\def\rightkol{Requirements for manuscripts submitted to Journal
``Informatics~and~Applications''}

\def\leftfootline{\small{\textbf{\thepage}
\hfill INFORMATIKA I EE PRIMENENIYA~--- INFORMATICS AND APPLICATIONS\ \ \ 2019\
\ \ volume~13\ \ \ issue\ 4}
}%
 \def\rightfootline{\small{INFORMATIKA I EE PRIMENENIYA~--- INFORMATICS AND APPLICATIONS\ \ \ 2019\ \ \ volume~13\ \ \ issue\ 4
\hfill \textbf{\thepage}}}

 \label{end\stat}

\newpage

%\vspace*{-60pt} {\small
{\baselineskip=9.1pt
\section*{Правила подготовки рукописей статей для публикации в журнале
<<Информатика и её применения>>}

\thispagestyle{empty}

 Журнал <<Информатика и её применения>> публикует
теоретические, обзорные и дискуссионные статьи, посвященные научным
исследованиям и разработкам в области информатики и ее приложений. Журнал
издается на русском языке. По специальному решению редколлегии отдельные статьи,
в виде исключения, могут печататься на английском языке.
Тематика журнала охватывает следующие направления:
\begin{itemize}
\item теоретические основы информатики; %\\[-13.5pt]
\item математические методы исследования сложных систем и процессов; %\\[-13.5pt]
\item информационные системы и сети; %\\[-13.5pt]
\item информационные технологии; %\\[-13.5pt]
\item архитектура и программное
обеспечение вычислительных комплексов и сетей.
\end{itemize}
\begin{enumerate}
\item В журнале печатаются результаты, ранее не
опубликованные и не предназначенные к одновременной публикации в других
изданиях. Публикация не должна нарушать закон об авторских правах. Направляя
свою рукопись в редакцию, авторы автоматически передают учредителям и
редколлегии неисключительные права на издание данной статьи на русском языке и
на ее распространение в России и за рубежом. При этом за авторами сохраняются
все права как собственников данной рукописи. В связи с этим авторами должно
быть представлено в редакцию письмо в следующей форме:
Соглашение о передаче права на публикацию:

\textit{<<Мы, нижеподписавшиеся, авторы рукописи <<$\qquad\qquad$>>, передаем
учредителям и редколлегии журнала <<Информатика и её применения>>
неисключительное право опубликовать данную рукопись статьи на русском языке как
в печатной, так и в электронной версиях журнала. Мы подтверждаем, что данная
публикация не нарушает авторского права других лиц или организаций. Подписи
авторов: (ф.\,и.\,о., дата, адрес)>>.}

Указанное соглашение может быть представлено 
как в бумажном виде, так и в виде отсканированной копии (с подписями авторов).


Редколлегия вправе запросить у авторов экспертное заключение о возможности
опубликования представленной статьи в открытой печати. %\\[-13.5pt]
\item Статья
подписывается всеми авторами. На отдельном листе представляются данные автора
(или всех авторов): фамилия, полные имя и отчество, телефон, факс, e-mail,
почтовый адрес. Если работа выполнена несколькими авторами, указывается фамилия
одного из них, ответственного за переписку с редакцией. %\\[-13.5pt]
\item Редакция журнала
осуществляет самостоятельную экспертизу присланных статей. Возвращение рукописи
на доработку не означает, что статья уже принята к печати. Доработанный вариант
с ответом на замечания рецензента необходимо прислать в редакцию. %\\[-13.5pt]
\item Решение
редакционной коллегии о принятии статьи к печати или ее отклонении сообщается
авторам. Редколлегия не обязуется направлять рецензию авторам отклоненной
статьи. %\\[-13.5pt]
\item Корректура статей высылается авторам для просмотра. Редакция
просит авторов присылать свои замечания в кратчайшие сроки. %\\[-13.5pt]
\item При
подготовке рукописи в MS Word рекомендуется использовать следующие настройки.
Параметры страницы: формат~--- А4; ориентация~--- книжная; поля (см): внутри~---
2,5, снаружи~--- 1,5, сверху~--- 2, снизу~--- 2, от края до нижнего
колонтитула~--- 1,3. Основной текст: стиль~--- <<Обычный>>: шрифт Times New
Roman, размер 14~пунктов, абзацный отступ~--- 0,5~см, 1,5 интервала,
выравнивание~--- по ширине. Рекомендуемый объем рукописи~--- не свыше
25~страниц указанного формата. Ознакомиться с шаблонами, содержащими примеры
оформления, можно по адресу в Интернете:
\textsf{http://www.ipiran.ru/journal/template.doc}.
\item К рукописи, предоставляемой в 2-х
экземплярах, обязательно прилагается электронная версия статьи (как правило, в
форматах MS WORD (.doc) или \LaTeX\ (.tex), а также~--- дополнительно~--- в
формате .pdf) на дискете, лазерном диске или по электронной почте. Сокращения
слов, кроме стандартных, не применяются. Все страницы рукописи должны быть
пронумерованы. %\\[-13.5pt]
\item Статья должна содержать следующую информацию на русском и
английском языках: название, Ф.И.О. авторов, места работы авторов и их
электронные адреса, подробные сведения об авторах, оформленные в соответствии с форматом, 
определяемым файлами {\sf http://www.ipiran.ru/journal/issues/2011\_05\_01/authors.asp} и 
{\sf http://www.ipiran.ru/journal/issues/2011\_01\_eng/authors.asp},
аннотация (не более 100~слов), ключевые слова. Ссылки на
литературу в тексте статьи нумеруются (в квадратных скобках) и располагаются в
порядке их первого упоминания. В~списке литературы не должно быть позиций, на которые нет ссылки в тексте статьи.
Все фамилии авторов, заглавия статей, названия
книг, конференций и~т.\,п.\ даются на языке оригинала, если этот язык
использует кириллический или латинский алфавит. %\\[-13.5pt]
\item Присланные в редакцию материалы авторам не возвращаются.
\item При отправке файлов по электронной
почте просим придерживаться следующих правил:
\begin{itemize}
\item указывать в поле subject (тема) название журнала и фамилию автора; %\\[-13.5pt]
\item использовать attach (присоединение); %\\[-13.5pt]
\item в случае больших объемов информации возможно
использование общеизвестных архиваторов (ZIP, RAR); %\\[-13.5pt]
\item в состав электронной версии статьи должны входить: файл, содержащий текст статьи, и файл(ы),
содержащий(е) иллюстрации. %\\[-13.5pt]
\end{itemize}
\item Журнал <<Информатика и её применения>> является некоммерческим изданием. 
Плата за публикацию с авторов не взимается, гонорар авторам не выплачивается.
\end{enumerate}
\thispagestyle{empty}
\textbf{Адрес редакции:} Москва 119333,
ул.~Вавилова, д.~44, корп.~2, ИПИ РАН\\
\hphantom{\textbf{Адрес редакции:} }Тел.: +7 (499) 135-86-92\ \
Факс:  +7 (495) 930-45-05\ \  E-mail:   rust@ipiran.ru }
}

%\include{ipi-ind}

%\tableofcontents

\end{document}


%\tableofcontents

%\end{document}





%\def\stat{cont}
{%\hrule\par
%\vskip 7pt % 7pt
\raggedleft\Large \bf%\baselineskip=3.2ex
А\,В\,Т\,О\,Р\,С\,К\,И\,Й\ \ У\,К\,А\,З\,А\,Т\,Е\,Л\,Ь\ \ З\,А\ \ 2\,0\,0\,7 г. \vskip 17pt
    \hrule
    \par
\vskip 21pt plus 6pt minus 3pt }

\label{st\stat}

\def\tit{\ }

\def\aut{\ }
\def\auf{\ }

\def\leftkol{\ } % ENGLISH ABSTRACTS}

\def\rightkol{\ } %ENGLISH ABSTRACTS}

\titele{\tit}{\aut}{\auf}{\leftkol}{\rightkol}


\contentsline {chapter}{\ }{Выпуск \quad Стр.} 
\contentsline {section}{\textbf{Батракова Д.\,А., Королев В.\,Ю., Шоргин С.\,Я.}\ \ Новый метод вероятностно-ста\-ти\-сти\-че\-ско\-го анализа информационных потоков в\nobreakspace {}телекоммуникационных сетях}{\qquad 1 \qquad 40} 
\contentsline {section}{\textbf{Борисов А.\,В.}\ \ Байесовское оценивание в системах наблюдения с\nobreakspace {}марковскими скачкообразными процессами: игровой подход}{\qquad 2 \qquad 65}
\contentsline {section}{\textbf{Босов А.\,В., Иванов А.\,В.}\ \ Программная инфраструктура информационного Web-пор\-тала}{\qquad 2 \qquad 50}
\contentsline {section}{\textbf{Захаров В.\,Н., Калиниченко Л.\,А., Соколов И.\,А., Ступников С.\,А.}\ \ Конструирование канонических информационных моделей для интегрированных информационных систем}{\qquad 2 \qquad 15}
\contentsline {section}{\textbf{Захаров В.\,Н., Козмидиади В.\,А.}\ \ Средства обеспечения отказоустойчивости при\-ло\-жений}{\qquad 1 \qquad 14} 
\contentsline {section}{\textbf{Иванов А.\,В.}\ \ см. Босов А.\,В.\hfill\hfill\hfill\hfill\hfill\hfill\hfill\hfill\hfill\hfill\hfill\hfill\hfill\hfill\hfill\hfill\hfill\hfill\hfill\hfill\hfill\hfill\hfill\hfill\hfill\hfill\hfill\hfill\hfill\hfill\hfill\hfill\hfill\hfill\hfill}{\ }
\contentsline {section}{\textbf{Ильин В.\,Д., Соколов И.\,А.}\ \ Символьная модель системы знаний информатики в\nobreakspace {}че\-ло\-ве\-ко-автоматной среде}{\qquad 1 \qquad 66} 
\contentsline {section}{\textbf{Калиниченко Л.\,А.}\ \ см. Захаров В.\,Н.\hfill\hfill\hfill\hfill\hfill\hfill\hfill\hfill\hfill\hfill\hfill\hfill\hfill\hfill\hfill\hfill\hfill\hfill\hfill\hfill\hfill\hfill\hfill\hfill\hfill\hfill\hfill\hfill\hfill\hfill\hfill\hfill\hfill\hfill\hfill}{\ }
\contentsline {section}{\textbf{Козеренко Е.\,Б.}\ \ Лингвистическое моделирование для систем машинного перевода и обработки знаний}{\qquad 1 \qquad 54} 
\contentsline {section}{\textbf{Козмидиади В.\,А.}\ \ см. Захаров В.\,Н.\hfill\hfill\hfill\hfill\hfill\hfill\hfill\hfill\hfill\hfill\hfill\hfill\hfill\hfill\hfill\hfill\hfill\hfill\hfill\hfill\hfill\hfill\hfill\hfill\hfill\hfill\hfill\hfill\hfill\hfill\hfill\hfill\hfill\hfill\hfill }{\ } 
\contentsline {section}{\textbf{Королев В.\,Ю.}\ \ см. Батракова Д.\,А.\hfill\hfill\hfill\hfill\hfill\hfill\hfill\hfill\hfill\hfill\hfill\hfill\hfill\hfill\hfill\hfill\hfill\hfill\hfill\hfill\hfill\hfill\hfill\hfill\hfill\hfill\hfill\hfill\hfill\hfill\hfill\hfill\hfill\hfill\hfill}{\ } 
\contentsline {section}{\textbf{Кудрявцев А.\,А., Шоргин С.\,Я.}\ \ Байесовский подход к\nobreakspace {}анализу систем массового обслуживания и\nobreakspace {}показателей надежности}{\qquad 2 \qquad 76}
\contentsline {section}{\textbf{Печинкин А.\,В., Соколов И.\,А., Чаплыгин В.\,В.}\ \ Многолинейная система массового обслуживания с конечным накопителем и ненадежными приборами}{\qquad 1 \qquad 27} 
\contentsline {section}{\textbf{Печинкин А.\,В., Соколов И.\,А., Чаплыгин В.\,В.}\ \ Стационарные характеристики многолинейной\nobreakspace {}системы массового обслуживания с\nobreakspace {}одновременными отказами приборов}{\qquad 2 \qquad 39}
\contentsline {section}{\textbf{Синицын И.\,Н.}\ \ Корреляционные методы построения аналитических информационных моделей флуктуаций полюса Земли по априорным данным}{\qquad 2 \qquad \hphantom{9}2}
\contentsline {section}{\textbf{Синицын И.\,Н.}\ \ Развитие теории фильтров Пугачева для оперативной обработки информации в стохастических системах}{{\qquad 1 \qquad \hphantom{9}3}} 
\contentsline {section}{\textbf{Соколов И.\,А.}\ \ см. Захаров В.\,Н.\hfill\hfill\hfill\hfill\hfill\hfill\hfill\hfill\hfill\hfill\hfill\hfill\hfill\hfill\hfill\hfill\hfill\hfill\hfill\hfill\hfill\hfill\hfill\hfill\hfill\hfill\hfill\hfill\hfill\hfill\hfill\hfill\hfill\hfill\hfill}{\ }
\contentsline {section}{\textbf{Соколов И.\,А.}\ \ см. Ильин В.\,Д.\hfill\hfill\hfill\hfill\hfill\hfill\hfill\hfill\hfill\hfill\hfill\hfill\hfill\hfill\hfill\hfill\hfill\hfill\hfill\hfill\hfill\hfill\hfill\hfill\hfill\hfill\hfill\hfill\hfill\hfill\hfill\hfill\hfill\hfill\hfill}{\ } 
\contentsline {section}{\textbf{Соколов И.\,А.}\ \ см. Печинкин А.\,В.\hfill\hfill\hfill\hfill\hfill\hfill\hfill\hfill\hfill\hfill\hfill\hfill\hfill\hfill\hfill\hfill\hfill\hfill\hfill\hfill\hfill\hfill\hfill\hfill\hfill\hfill\hfill\hfill\hfill\hfill\hfill\hfill\hfill\hfill\hfill}{\ } 
\contentsline {section}{\textbf{Соколов И.\,А.}\ \ см. Печинкин А.\,В.\hfill\hfill\hfill\hfill\hfill\hfill\hfill\hfill\hfill\hfill\hfill\hfill\hfill\hfill\hfill\hfill\hfill\hfill\hfill\hfill\hfill\hfill\hfill\hfill\hfill\hfill\hfill\hfill\hfill\hfill\hfill\hfill\hfill\hfill\hfill}{\ }
\contentsline {section}{\textbf{Ступников С.\,А.}\ \ см. Захаров В.\,Н.\hfill\hfill\hfill\hfill\hfill\hfill\hfill\hfill\hfill\hfill\hfill\hfill\hfill\hfill\hfill\hfill\hfill\hfill\hfill\hfill\hfill\hfill\hfill\hfill\hfill\hfill\hfill\hfill\hfill\hfill\hfill\hfill\hfill\hfill\hfill}{\ }
\contentsline {section}{\textbf{Чаплыгин В.\,В.}\ \ см. Печинкин А.\,В.\hfill\hfill\hfill\hfill\hfill\hfill\hfill\hfill\hfill\hfill\hfill\hfill\hfill\hfill\hfill\hfill\hfill\hfill\hfill\hfill\hfill\hfill\hfill\hfill\hfill\hfill\hfill\hfill\hfill\hfill\hfill\hfill\hfill\hfill\hfill}{\ } 
\contentsline {section}{\textbf{Чаплыгин В.\,В.}\ \ см. Печинкин А.\,В.\hfill\hfill\hfill\hfill\hfill\hfill\hfill\hfill\hfill\hfill\hfill\hfill\hfill\hfill\hfill\hfill\hfill\hfill\hfill\hfill\hfill\hfill\hfill\hfill\hfill\hfill\hfill\hfill\hfill\hfill\hfill\hfill\hfill\hfill\hfill}{\ }
\contentsline {section}{\textbf{Шоргин С.\,Я.}\ \ см. Батракова Д.\,А.\hfill\hfill\hfill\hfill\hfill\hfill\hfill\hfill\hfill\hfill\hfill\hfill\hfill\hfill\hfill\hfill\hfill\hfill\hfill\hfill\hfill\hfill\hfill\hfill\hfill\hfill\hfill\hfill\hfill\hfill\hfill\hfill\hfill\hfill\hfill}{\ } 
\contentsline {section}{\textbf{Шоргин С.\,Я.}\ \ см. Кудрявцев А.\,А.\hfill\hfill\hfill\hfill\hfill\hfill\hfill\hfill\hfill\hfill\hfill\hfill\hfill\hfill\hfill\hfill\hfill\hfill\hfill\hfill\hfill\hfill\hfill\hfill\hfill\hfill\hfill\hfill\hfill\hfill\hfill\hfill\hfill\hfill\hfill}{\ }
%\thispagestyle{myheadings}
\def\leftfootline{\small{\textbf{\thepage}
\hfill ИНФОРМАТИКА И ЕЁ ПРИМЕНЕНИЯ\ \ \ том~1\ \ \ выпуск~2\ \ \ 2007}
}%
 \def\rightfootline{\small{ИНФОРМАТИКА И ЕЁ ПРИМЕНЕНИЯ\ \ \ том~1\ \ \ выпуск~2\ \ \ 2007
 \hfill \textbf{\thepage}}}
 \label{end\stat}

%\def\stat{cont-e}
{%\hrule\par
%\vskip 7pt % 7pt
\raggedleft\Large \bf%\baselineskip=3.2ex
2\,0\,0\,7\ \ A\,U\,T\,H\,O\,R\ \ I\,N\,D\,E\,X \vskip 17pt
    \hrule
    \par
\vskip 21pt plus 6pt minus 3pt }

\label{st\stat}

\def\tit{\ }

\def\aut{\ }
\def\auf{\ }

\def\leftkol{\ } % ENGLISH ABSTRACTS}

\def\rightkol{\ } %ENGLISH ABSTRACTS}

\titele{\tit}{\aut}{\auf}{\leftkol}{\rightkol}


\contentsline {chapter}{\ }{Issue \quad Page} 
\contentsline {subsection}{\textbf{Batrakova D.\,A., Korolev V.\,Yu., Shorgin S.\,Ya.}\ \ A New Method for the Probabilistic and Statistical Analysis of Information Flows in Telecommunication Networks}{\qquad 1 \qquad 40} 
\contentsline {subsection}{\textbf{Borisov A.\,V.}\ \ Bayesian Estimation in\nobreakspace {}Observation Systems with\nobreakspace {}Markov Jump Processes: Game-Theoretic Approach}{\qquad 2 \qquad 65} 
\contentsline {subsection}{\textbf{Bosov A.\,V., Ivanov A.\,V.}\ \ Linguistic Simulation for Machine Translation and Knowledge Management Systems}{\qquad 2 \qquad 50} 
\contentsline {subsection}{\textbf{Chaplygin V.\,V.} see Pechinkin A.\,V.\hfill\hfill\hfill\hfill\hfill\hfill\hfill\hfill\hfill\hfill\hfill\hfill\hfill\hfill\hfill\hfill\hfill\hfill\hfill\hfill\hfill\hfill\hfill\hfill\hfill\hfill\hfill\hfill\hfill\hfill\hfill\hfill\hfill\hfill\hfill}{\ }
\contentsline {subsection}{\textbf{Chaplygin V.\,V.} see Pechinkin A.\,V.\hfill\hfill\hfill\hfill\hfill\hfill\hfill\hfill\hfill\hfill\hfill\hfill\hfill\hfill\hfill\hfill\hfill\hfill\hfill\hfill\hfill\hfill\hfill\hfill\hfill\hfill\hfill\hfill\hfill\hfill\hfill\hfill\hfill\hfill\hfill}{\ }
\contentsline {subsection}{\textbf{Ilyin V.\,D., Sokolov I.\,A.}\ \ The Symbol Model of Informatics Knowledge System in Human-Automaton Environment}{\qquad 1 \qquad 66} 
\contentsline {subsection}{\textbf{Ivanov A.\,V.} see Bosov A.\,V.\hfill\hfill\hfill\hfill\hfill\hfill\hfill\hfill\hfill\hfill\hfill\hfill\hfill\hfill\hfill\hfill\hfill\hfill\hfill\hfill\hfill\hfill\hfill\hfill\hfill\hfill\hfill\hfill\hfill\hfill\hfill\hfill\hfill\hfill\hfill}{\ }
\contentsline {subsection}{\textbf{Kalinichenko L.\,A.} see Zakharov V.\,N.\hfill\hfill\hfill\hfill\hfill\hfill\hfill\hfill\hfill\hfill\hfill\hfill\hfill\hfill\hfill\hfill\hfill\hfill\hfill\hfill\hfill\hfill\hfill\hfill\hfill\hfill\hfill\hfill\hfill\hfill\hfill\hfill\hfill\hfill\hfill}{\ }
\contentsline {subsection}{\textbf{Korolev V.\,Yu.} see Batrakova D.\,A.\hfill\hfill\hfill\hfill\hfill\hfill\hfill\hfill\hfill\hfill\hfill\hfill\hfill\hfill\hfill\hfill\hfill\hfill\hfill\hfill\hfill\hfill\hfill\hfill\hfill\hfill\hfill\hfill\hfill\hfill\hfill\hfill\hfill\hfill\hfill}{\ }
\contentsline {subsection}{\textbf{Kozerenko E.\,B.}\ \ Linguistic Simulation for Machine Translation and Knowledge Management Systems}{\qquad 1 \qquad 54} 
\contentsline {subsection}{\textbf{Kozmidiady V.\,A.} see Zakharov V.\,N.\hfill\hfill\hfill\hfill\hfill\hfill\hfill\hfill\hfill\hfill\hfill\hfill\hfill\hfill\hfill\hfill\hfill\hfill\hfill\hfill\hfill\hfill\hfill\hfill\hfill\hfill\hfill\hfill\hfill\hfill\hfill\hfill\hfill\hfill\hfill}{\ }
\contentsline {subsection}{\textbf{Kudryavtsev A.\,A., Shorgin S.\,Ya.}\ \ Bayesian Approach to Queueing Systems and Reliability Characteristics}{\qquad 2 \qquad 76} 
\contentsline {subsection}{\textbf{Pechinkin A.\,V., Sokolov I.\,A., Chaplygin V.\,V.}\ \ Multichannel Queuing System with Finite Buffer and Unreliable Servers}{\qquad 1 \qquad 27} 
\contentsline {subsection}{\textbf{Pechinkin A.\,V., Sokolov I.\,A., Chaplygin V.\,V.}\ \ Stationary Characteristics of a Multichannel Queueing System with\nobreakspace {}Simultaneous Refusals of Servers}{\qquad 2 \qquad 39} 
\contentsline {subsection}{\textbf{Shorgin S.\,Ya.} see Batrakova D.\,A.\hfill\hfill\hfill\hfill\hfill\hfill\hfill\hfill\hfill\hfill\hfill\hfill\hfill\hfill\hfill\hfill\hfill\hfill\hfill\hfill\hfill\hfill\hfill\hfill\hfill\hfill\hfill\hfill\hfill\hfill\hfill\hfill\hfill\hfill\hfill}{\ }
\contentsline {subsection}{\textbf{Shorgin S.\,Ya.} see Kudryavtsev A.\,A.\hfill\hfill\hfill\hfill\hfill\hfill\hfill\hfill\hfill\hfill\hfill\hfill\hfill\hfill\hfill\hfill\hfill\hfill\hfill\hfill\hfill\hfill\hfill\hfill\hfill\hfill\hfill\hfill\hfill\hfill\hfill\hfill\hfill\hfill\hfill}{\ }
\contentsline {subsection}{\textbf{Sinitsyn I.\,N.}\ \ Correlational Methods for Analytical Informational Models of the Earth Pole Fluctuations Design Based on a priori Data}{\qquad 2 \qquad \hphantom{9}2}
\contentsline {subsection}{\textbf{Sinitsyn I.\,N.}\ \ Development of Pugachev Filtering for Stochastic Systems}{\qquad 1 \qquad \hphantom{9}3}
\contentsline {subsection}{\textbf{Sokolov I.\,A.} see Ilyin V.\,D.\hfill\hfill\hfill\hfill\hfill\hfill\hfill\hfill\hfill\hfill\hfill\hfill\hfill\hfill\hfill\hfill\hfill\hfill\hfill\hfill\hfill\hfill\hfill\hfill\hfill\hfill\hfill\hfill\hfill\hfill\hfill\hfill\hfill\hfill\hfill}{\ }
\contentsline {subsection}{\textbf{Sokolov I.\,A.} see Pechinkin A.\,V.\hfill\hfill\hfill\hfill\hfill\hfill\hfill\hfill\hfill\hfill\hfill\hfill\hfill\hfill\hfill\hfill\hfill\hfill\hfill\hfill\hfill\hfill\hfill\hfill\hfill\hfill\hfill\hfill\hfill\hfill\hfill\hfill\hfill\hfill\hfill}{\ }
\contentsline {subsection}{\textbf{Sokolov I.\,A.} see Pechinkin A.\,V.\hfill\hfill\hfill\hfill\hfill\hfill\hfill\hfill\hfill\hfill\hfill\hfill\hfill\hfill\hfill\hfill\hfill\hfill\hfill\hfill\hfill\hfill\hfill\hfill\hfill\hfill\hfill\hfill\hfill\hfill\hfill\hfill\hfill\hfill\hfill}{\ }
\contentsline {subsection}{\textbf{Sokolov I.\,A.} see Zakharov V.\,N.\hfill\hfill\hfill\hfill\hfill\hfill\hfill\hfill\hfill\hfill\hfill\hfill\hfill\hfill\hfill\hfill\hfill\hfill\hfill\hfill\hfill\hfill\hfill\hfill\hfill\hfill\hfill\hfill\hfill\hfill\hfill\hfill\hfill\hfill\hfill}{\ }
\contentsline {subsection}{\textbf{Stupnikov S.\,A.} see Zakharov V.\,N.\hfill\hfill\hfill\hfill\hfill\hfill\hfill\hfill\hfill\hfill\hfill\hfill\hfill\hfill\hfill\hfill\hfill\hfill\hfill\hfill\hfill\hfill\hfill\hfill\hfill\hfill\hfill\hfill\hfill\hfill\hfill\hfill\hfill\hfill\hfill}{\ }
\contentsline {subsection}{\textbf{Zakharov V.\,N., Kalinichenko L.\,A., Sokolov I.\,A., Stupnikov S.\,A.}\ \ Development of Canonical Information Models for Integrated Information Systems}{\qquad 2 \qquad 15} 
\contentsline {subsection}{\textbf{Zakharov V.\,N., Kozmidiady V.\,A.}\ \ Means Providing Applications Fault Tolerance}{\qquad 1 \qquad 14} 
\def\leftfootline{\small{\textbf{\thepage}
\hfill ИНФОРМАТИКА И ЕЁ ПРИМЕНЕНИЯ\ \ \ том~1\ \ \ выпуск~2\ \ \ 2007}
}%
 \def\rightfootline{\small{ИНФОРМАТИКА И ЕЁ ПРИМЕНЕНИЯ\ \ \ том~1\ \ \ выпуск~2\ \ \ 2007
 \hfill \textbf{\thepage}}}
 \label{end\stat}


%\tableofcontents


\end{document}

\newcommand{\Ack}{\subsection*{\protect\large\bf Acknowledgments}}