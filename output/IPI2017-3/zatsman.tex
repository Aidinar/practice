\renewcommand{\figurename}{\protect\bf Figure}
\renewcommand{\tablename}{\protect\bf Table}

\def\stat{zatsman}


\def\tit{INDICATOR EVALUATION OF~PROCESSES OF~KNOWLEDGE TRANSFER FROM~SCIENCE 
TO~TECHNOLOGY}

\def\titkol{Indicator evaluation of~processes of~knowledge transfer from~science 
to~technology}

\def\autkol{I.\,M.~Zatsman, G.\,V.~Lukyanov, V.\,A.~Minin, et al.} 
%V.\,A.~Havanskov$^4$, and~S.\,K.~Shubnikov$^5$}

\def\aut{I.\,M.~Zatsman$^1$, G.\,V.~Lukyanov$^2$, V.\,A.~Minin$^3$, 
V.\,A.~Havanskov$^4$, and~S.\,K.~Shubnikov$^5$}

\titel{\tit}{\aut}{\autkol}{\titkol}

%{\renewcommand{\thefootnote}{\fnsymbol{footnote}}
%\footnotetext[1] {The 
%research of Yuri Kabanov was done under partial financial support   of the grant 
%of  RSF No.\,14-49-00079.}}

\renewcommand{\thefootnote}{\arabic{footnote}}
\footnotetext[1]{Institute of Informatics Problems, Federal Research Center 
``Computer Science and Control'' of the Russian 
Academy of Sciences, 44-2~Vavilov Str., Moscow 119333,
Russian Federation; \mbox{izatsman@yandex.ru}}
\footnotetext[2]{Institute of Informatics Problems, Federal Research Center 
``Computer Science and Control'' of the Russian 
Academy of Sciences, 44-2~Vavilov Str., Moscow 119333,
Russian Federation; \mbox{gena-mslu@mail.ru}}
\footnotetext[3]{Institute of Informatics Problems, Federal Research Center 
``Computer Science and Control'' of the Russian 
Academy of Sciences, 44-2~Vavilov Str., Moscow 119333,
Russian Federation; \mbox{aleksisss@ya.ru}}
\footnotetext[4]{Institute of Informatics Problems, Federal Research Center 
``Computer Science and Control'' of the Russian 
Academy of Sciences, 44-2~Vavilov Str., Moscow 119333,
Russian Federation; \mbox{chavanskov@yandex.ru}}
\footnotetext[5]{Institute of Informatics Problems, Federal Research Center 
``Computer Science and Control'' of the Russian 
Academy of Sciences, 44-2~Vavilov Str., Moscow 119333,
Russian Federation; \mbox{sergeysh50@yandex.ru}}


\index{Zatsman I.\,M.}
\index{Lukyanov G.\,V.}
\index{Minin V.\,A.} 
\index{Havanskov V.\,A.}
\index{Shubnikov S.\,K.}
\index{Зацман И.\,М.}
\index{Лукьянов Г.\,В.}
\index{Минин В.\,А.}
\index{Хавансков В.\,А.}
\index{Шубников С.\,К.}


\vspace*{-10pt}

\def\leftfootline{\small{\textbf{\thepage}
\hfill INFORMATIKA I EE PRIMENENIYA~--- INFORMATICS AND APPLICATIONS\ \ \ 2017\ \ \ volume~11\ \ \ issue\ 3}
}%
 \def\rightfootline{\small{INFORMATIKA I EE PRIMENENIYA~--- INFORMATICS AND APPLICATIONS\ \ \ 2017\ \ \ volume~11\ \ \ issue\ 3
\hfill \textbf{\thepage}}}


           
    
    \Abste{The article is dedicated to indicator evaluation of information 
interactions between science and technology. Some of these indicators are defined as 
single numeric values and some as matrices of numeric values that characterize the 
intensity of the knowledge flow from different research areas into specific 
technological branches. The article provides a~description of primary information 
resources, mainly full-text descriptions of patents, which are used to define numerical 
values of these indicators. It also gives a~description of secondary information 
resources generated as the result of patent documentation processing, including 
information on references to scientific publications cited in patents. Primary and 
secondary resources were used to create and test the information model and the 
corresponding indicators of assessment of interaction between science and 
technology. This model was applied as a~foundation for calculation of numerical 
values of integral and thematic indicators of the intensity of scientific knowledge 
flow into the branch of information technologies.}
    
    \KWE{information interaction between science and technology; citation of 
scientific works; intensity of the knowledge flow; indicator assessment; information 
technology}

\DOI{10.14357/19922264170315} 


\vspace*{-4pt}


\vskip 12pt plus 9pt minus 6pt

      \thispagestyle{myheadings}

      \begin{multicols}{2}

                  \label{st\stat}

\section*{Introduction}

\noindent
    Science has always been a~major driver for technological progress and 
sustainable socioeconomic development. As a~matter of fact, just scientific results 
mainly determine the nature, character, pace, and scale of technological progress 
shaping the landscape of a~modern society. Thus, it is of utmost importance for 
strategic planning to reveal functional dependencies between specific scientific (from 
one side) and specific technological (from the other side) areas, branches, and 
disciplines. It is essential to underline that the objective of such an ``investigation'' 
does not lie in the generally acceptable and absolutely clear but totally imprecise 
understanding of the social role of science which actually provides next to nothing 
help as far as management of state scientific and technological resources is 
concerned. Any rational research of this kind is aimed at finding such models of 
interactions between science and technology which could support strategic planning 
with measurable facts and figures more or less strictly reflecting these interactions.
    
    Currently, there is only one intrinsic information resource which comprises 
implicit data on how science is translated into technology; and this resource is patent 
documentation. Historically, it turned out that patents contain perfectly documented 
initial information which makes it possible to extract explicit data on the intangible 
interplay between science and technology with a~subsequent opportunity of finding 
quantitative assessments for scientific knowledge flow into the branch of 
technologies. This initial information in patent documentation is presented in the 
form of references to scientific publications which could be employed 
for calculation of a~comprehensive set of quantitative indicators as essential benchmarks for strategic 
planning. 

It is relevant to mention some favorable conditions to perform such a~sophisticated 
job. Patents are perfectly documented and patent information is generally 
available online and well suited for computerized processing which makes it 
possible to solve the concomitant tasks and get together this tremendous  
science-technology puzzle at a~reasonable time-span and limited labor resources.

\begin{figure*}[b] %fig1
 \vspace*{1pt}
\begin{center}
\mbox{%
\epsfxsize=110.156mm
\epsfbox{zac-1.eps}
}
\end{center}
\vspace*{-9pt}
\Caption{Scenario of information resource processing}
\end{figure*}
    
    Thus, patent documentation containing references to scientific publications is 
the major data resource for the study of information linkages between science and 
technology, reflecting the process of knowledge transfer from different fields of 
scientific research into corresponding technological areas. The conceived approach to 
the study of processes of knowledge transfer, described in the article, is based on the 
analysis of arrays of invention descriptions that contain references to scientific 
publications cited by authors of inventions.
    
    This approach has been actively developed abroad for more than~30~years 
    now~[1--9]. Similar research has been 
conducted for several years by a~dedicated team of experts at the Institute of 
Informatics of the Federal Research Center ``Computer Science and Control'' of the 
Russian Academy of Sciences. The experience acquired by the team to-date has 
allowed to calculate numerical values of a~wide range of indicators by means of 
computer processing of full-text descriptions of inventions and their annotations 
(abstracts)~[10--18]. The study of knowledge transfer processes has been carried out 
as a~part of official research activities approved by the Russian Government for the 
Institute of Informatics in the field of ``Monitoring and Indicator Assessment of 
Scientific Activity''~[19--23].
    
    This paper summarizes the previously obtained results in the form of a~holistic 
model of interplay between science and technology and presents their further 
expansion in the following directions:
    \begin{itemize}
\item consideration of the diversity of scientific document types cited in 
patented inventions, which gives the opportunity to evaluate and compare the 
intensity of scientific knowledge transfer through different channels (journal 
articles, conference proceedings, books, etc.); and
\item modeling the process of calculating the values of indicators of intensity of 
scientific knowledge transfer, which provides the opportunity to clarify the 
meaning of indicators of different types and to consider different aspects of the 
citation in the descriptions of inventions (e.\,g., the citation of scientific 
publications by the inventors and/or experts).
\end{itemize}

\section*{Assessment of~the~Intensity of~Scientific Knowledge Transfer}

\noindent
    The term ``indicator of transfer intensity~--- ITI'' is defined as a~numerical value 
or a~matrix of numerical values which characterizes a~certain aspect of the intensity 
of scientific knowledge transfer. The article covers a~wide range of indicators which 
are divided into two main categories: integral (ITI-I) and thematic (ITI-T).
    
    Integral ITIs yield a~general view on the intensity of knowledge transfer. Thematic 
ITIs are divided into several types and provide perception on thematically 
oriented directions of knowledge transfer. The proposed range of ITI-I types makes 
it possible to characterize the process of knowledge transfer from different points of 
view.

\begin{table*}\small %tabl1
\begin{center}
\Caption{Information resources for ITI calculation}
\vspace*{2ex}

\begin{tabular}{ll}
\hline
\multicolumn{1}{c}{Information resource}&\multicolumn{1}{c}{Category}\\
\hline
Array of invention descriptions&Primary\\
Array of bibliographic data of inventions&Primary\\
International patent classification&Primary\\
Classifier of scientific research areas &Primary\\
Array of reports on patent search&Primary\\
Array of references on scientific publications with assigned CSRA 
items&Secondary\\
Array of interconnections between items of CSRA and IPC indices&Secondary\\
\hline
\end{tabular}
\end{center}
\end{table*}

    
    The scenario of information resource processing aimed at calculating ITIs 
(presented in Fig.~1) embraces the following components, which are determined 
according to the objective of the case study of scientific knowledge transfer process:
    \begin{enumerate}[(1)]
\item description of the technologies under scrutiny using the International 
Patent Classification (IPC);
\item time-span (year start\,--\,year end) for which patent applications were 
submitted or patents on inventions were granted, whose description texts are 
processed under the scenario;
\item classifier of scientific research areas (CSRA) that is used to properly 
assign corresponding items from CSRA to certain references which are referred 
to in the descriptions of inventions;
\item types of cited scientific publications (books, journal articles, conference 
proceedings, etc.); and
\item types of ITI-T and method parameters used in determining the indicator 
values of each type.
\end{enumerate}




    Information resources which are used to determine the values of ITIs are divided 
into two main types:
    \begin{enumerate}[(1)]
    \item  primary information resources that can be obtained from the Russian 
Patent Office (RPO): descriptions of inventions and bibliographic data of inventions; and
    \item secondary information resources generated by processing the primary 
information resources under the scenario presented in Fig.~1.
    \end{enumerate}
    
    Conception of the scenario for processing information resources assumes 
primarily a~description of the very technology sector, where the intensity of scientific 
knowledge transfer is the subject of investigation. It follows from the definition of 
``scenario'' that the technological area under investigation is described using a~set of 
IPC indices. Such a~description of a~specific technology allows to shape an array of 
primary information resources, retrieving them from the information system 
(database) of RPO in the form of full-text descriptions of inventions in patents.
    
    According to the scenario at the first stage, references made by the inventor 
and/or expert to the cited scientific publications are retrieved, and after that, these 
references are distributed among the items of CSRA. According to the retrieval 
results, secondary information resources are created in the form of arrays of 
references to scientific publications. Each reference is assigned an item (or several 
items) from CSRA~\cite{15-zat}.
    
    A list of primary and secondary information resources which are used to define 
values of ITIs is submitted in Table~1. These information resources supplemented by 
a~system of internal relations between their components make up the basis for 
modeling the process of determining numerical values of ITIs.
    

\section*{Principles of~Classification}

\noindent
    The article is dealing with the following three principles (or bases) of ITI 
classification:
    \begin{enumerate}[(1)]
    \item as the first principle of ITI classification, we consider the distribution of 
transferred knowledge among scientific areas, fields, and disciplines in the dichotomy 
``basic (fundamental)\,--\,applied.''
    
    These two categories of indicators reflect how results of basic and applied 
research are translated into inventions. If a~basic science classifier in the scenario is 
replaced by some applied science classifier, it will mean switching to another 
distribution scheme of scientific knowledge according to CSRA and, therefore, to 
a~different category of ITI-Ts\footnote{Values of ITI-Is by definition do not depend 
on such a~replacement.}.
    
    This transition could create an effect when two thematically similar publications 
from the point of view of basic science might be found in absolutely different 
branches of applied science;
    
    \item the second basis for classification of ITIs is based on data published by 
RPO in the register of reports on patent search prepared by experts of RPO during 
the examination of invention applications. These data include symbolic marking with 
letters~X, Y, A, D, and~T under standard~ST.14 of the World Intellectual Property 
Organization to indicate the relationship of a~specific reference to the 
essence of the invention in question~\cite{24-zat}; and
    
    \item to explain the third principle of classification, it is necessary to mention 
that the number of publications cited in the descriptions of inventions and which 
belong to the same thematic category according to CSRA may differ significantly 
from patent to patent.
    \end{enumerate}
    
    However, if in the description of an invention there is at least only one such 
reference, it means a~possible knowledge transfer relating to this category of CSRA. 
If there are several publications that have the same CSRA item within the description 
of an invention, the second\linebreak\vspace*{-12pt}

\pagebreak

\end{multicols}

\begin{table*}\small %tabl2
    \begin{center}
    \Caption{Principles of ITI classification}
    \vspace*{2ex}
    
    \begin{tabular}{lll}
    \hline
\multicolumn{1}{c}{Principle of classification}&\multicolumn{1}{c}{Aspect (base-code)}&
\multicolumn{1}{c}{ITI 
type}\\
\hline
\tabcolsep=0pt\begin{tabular}{l}Basic (fundamental) or applied science
classifier\\ of transferred knowledge (related only to ITI-T)\end{tabular}&K~--- knowledge type&
\tabcolsep=0pt\begin{tabular}{l}Basic (fundamental) research (FR)\\ Applied research (AR)\end{tabular}\\
\hline
\tabcolsep=0pt\begin{tabular}{l}Subject of citation:
author of the invention\\ and/or patent expert\end{tabular}&R~--- relation type&
\tabcolsep=0pt\begin{tabular}{l} Expert (E)\\
Author (A)\\
Expert--author (EA)\end{tabular}\\
\hline
\tabcolsep=0pt\begin{tabular}{l}Method of taking into consideration  of CSRA items\\
of publications cited in the invention  for each 
IPC\\ index of invention\end{tabular}&C~--- computation type &
\tabcolsep=0pt\begin{tabular}{l}Actual (P)\\ Frequency (F)\end{tabular}\\
\hline
\end{tabular}
\end{center}
\vspace*{9pt}
\end{table*}

\begin{figure*} %fig2
\vspace*{1pt}
\begin{center}
\mbox{%
\epsfxsize=165.043mm
\epsfbox{zac-2.eps}
}
\end{center}
\vspace*{-9pt}
\Caption{The ITI types}
\end{figure*}

\begin{multicols}{2}

\noindent
 and subsequent references can be considered as recurrent 
cases of knowledge transfer related to this category of CSRA. Thus, the third 
principle of classification assumes consideration or not consideration of such 
repetitions of CSRA items assigned to cited publications in the descriptions of 
inventions. Then, it is possible to define two types of ITIs: ``actual indicators'' when 
you take into account the presence of a~particular CSRA item without its repetitions 
and ``frequency indicators'' based on the number of repetitions of each CSRA item 
cited within one specific invention.
    
    Thus, the article is dealing with three principles of ITI classification (presented 
in Table~2) which in combination provide an opportunity to offer a~typology of ITIs 
which includes~35~types as presented in Fig.~2.
    
    
    
    Introduction of the above mentioned three principles (or bases) of ITI 
classification entails three levels of typology. Simultaneous consideration of all three 
bases of classification of ITIs can significantly extend their range, namely, from~7 
to~35~types. Seven types of ITIs which have been used previously are filled with 
grey color in Fig.~2~\cite{3-zat}. Strictly speaking, values are assigned only 
to~12~ITI types of the third level for each of the three bases of classification. Values 
for the other~23~IPI types are assigned only for one or two bases. This means that 
each of the indicators is formed as a~combination of IPI types of the third level. For 
example, ITIs of type $I = B(\mathrm{FR})$ of the first level are generated by combination of 
six types of ITIs of types~$B(\mathrm{FR, E, P})$, $B(\mathrm{FR, A, P})$, 
$B(\mathrm{FR, EA, P})$, $B(\mathrm{FR, E, 
F})$, $B(\mathrm{FR, A, F})$, and $B(\mathrm{FR, EA, F})$ which were defined by a~basic science 
classifier.



\section*{Method of~Determining the~Values of~Indicators}

\noindent
    The basic concept of the method, applied to determine the values of ITIs, 
consists in a~matrix of frequency ratios which describes the intensity of knowledge 
transfer. According to the scenario, each technological area under scrutiny is 
described by IPC indices and each scientific area (branch, discipline) is defined by 
CSRA items. The quantitative characteristics of the intensity of knowledge transfer 
are the frequency ratios of pairs $\langle$IPC index; CSRA item$\rangle$. According 
to the proposed conception, the higher the number of a~certain CSRA item relating to 
a~specific IPC index (frequency ratio of the pair $\langle$IPC index; CSRA 
item$\rangle$), the more is significant impact of that CSRA item on the technological 
area with the corresponding IPC index. 
    
    Definition of the values in each cell of this matrix begins with analysis of a~tuple 
of type $\langle$invention; quoted publication$\rangle$. During execution of the 
scenario, such tuples are generated for each of the invention from the list which is 
denoted as $P_s = \{p_1, \ldots , p_N\}$ where~$N$ is the number of descriptions of 
inventions that were selected as a~result of a~search query to the database of RPO. 
The query specifies values for parameters $\{\mathrm{IPC}\}$ and 
$\{\mathrm{Period}\}$. Parameter $\{\mathrm{IPC}\}$ describes every 
technological area as a~set of IPC indexes. Each invention~$p_i$ from the list 
of~$P_s$ is defined by a~nonempty set of IPC indices $Q_i=\{ q_1^i, \ldots , 
q^i_{Z(i)}\}$ where~$Z(i)$ is the number of IPC indices specified in 
invention~$p_i$.
    
    To create the required set of indicators within each type of ITIs, the following 
attributes are retrieved from the field of bibliographic data of inventions (in addition 
to IPC indices): publication date of the initial application;
 date of issue of the patent; and 
country of the applicant. Emphasis on the IPC indices is made due to the fact that 
they are crucial to form the matrix of frequency ratios.
    
    Each description of invention $p_i$ from the list of~$P_s$ is determined by 
a~number of references cited in scientific publications $\Pi_i=\{\pi_1^i,\ldots , 
\pi^i_{Y(i)}\}$ where $Y(i)$ is the number of publications cited in invention~$p_i$. 
It should be noted that some sets of descriptions of inventions might be empty if there 
are no references in the description of an invention.
    
    Each ordered pair of IPC index~$q_m^i$ and reference~$\pi^i_j$, that is, 
a~tuple of type $\langle q_m^i, \pi^i_j\rangle$, is denoted by~$\lambda^i_{m,j}$ 
where $m = 1, \ldots  , Z(i)$; 
$j = 1, \ldots  , Y(i)$; and $i = 1, \ldots , N$. For invention~$p_i$, a~set of tuples $\langle 
q_m^i, \pi_j^i\rangle$ composes a~set of tuples 
$\Lambda_i=\{\lambda^i_{m,j},\ m=1, \ldots  , Z(i),\ j=1, \ldots , Y(i)\}$, generated as a~result of linguistic analysis of the 
description of invention~$p_i$. In the course of distribution of references among 
CSRA items in cited publications, each reference is assigned as a~set of items according 
to the specified in the  CSRA scenario. These items are determined by bibliographic 
data of publication sources where the cited articles are published. At the same time, 
with the set of CSRA items, year, country, and type of a~cited scientific publication 
are determined.

%\begin{table*}
    \begin{center}
    
    {{\tablename~3}\ \ \small{Example of tuple $\langle$IPC index, CSRA item$\rangle$}}
    
    \vspace*{6pt}
     
   {\small  \begin{tabular}{ccccc}
    \hline
    \tabcolsep=0pt\begin{tabular}{c}Patent\\ No.\end{tabular} &
    \tabcolsep=0pt\begin{tabular}{c} Issue\\ year\end{tabular} & 
    \tabcolsep=0pt\begin{tabular}{c}IPC\\ index\end{tabular} & 
    \tabcolsep=0pt\begin{tabular}{c}CSRA\\ item\end{tabular} & 
    \tabcolsep=0pt\begin{tabular}{c}CSRA item \\
repetition\end{tabular}\\
    \hline
    &&&02-202&2\\
    &&&02-205&2\\
    &&&02-206&1\\
    2337396&2008&G06F 1/00&02-410&2\\
    &&&02-440&2\\
    &&&07-820&2\\
    &&&07-450&2\\
    \hline
    &&&02-205&1\\
    &&&02-410&1\\
    2311674& 2007& G06F 1/00& 02-202&1\\
    &&&07-820&1\\
    &&&02-440&1\\
    \hline
    \end{tabular}}
    \end{center}
%    \end{table*}

\vspace*{9pt}

\addtocounter{table}{1}
    
    Each reference can be assigned in several CSRA items, which make up a~set 
$R_j^i=\{ r^i_{j,1}\ldots , r^i_{j,K(j)}\}$ where $K(j)$ is the number of CSRA 
items assigned to a~certain reference~$\pi^i_j$ retrieved from the description of 
invention~$p_i$.
    
    This approach could be explained by data presented in Table~3. There are only 
two patents in Table~3 (just for example) with the same IPC index G06F~1/00. Each 
patent presented in the table is assigned to its specific set of science classifier items. 
Since the two patents have the same IPC index (G06F~1/00), both sets of CSRA 
items with the corresponding numbers of item repetitions are assigned to this index.
    
    
    
    
    
    Each ordered pair of an IPC index~$q_m^i$ and CSRA item~$r^i_{j,k}$, that 
is, a~tuple of type $\langle q^i_m, r^i_{j,k}\rangle$ is denoted as~$\mu^{i,j}_{m,k}$ 
where $m = 1, \ldots , Z(i)$; $k = 1, \ldots , K(j)$; $j = 1, \ldots , Y(i)$; and 
$i = 1, \ldots , 
N$. Then, for the description of invention~$p_i$, one more set $M_i= 
\{\mu^{i,j}_{m,k}\}$ can be defined. It includes tuples of type $\langle$IPC index, CSRA 
item$\rangle$ as an ordered combination of IPC indices of invention~$p_i$ and 
CSRA items assigned to all scientific publications cited in invention~$p_i$. The 
result of constructing any set of tuples~$M_i$ is stipulated by the defined in the 
scenario mode of consideration of reference repetitions retrieved form each invention 
(``actual indicator'' and ``frequency indicator''). Distribution of references in cited 
publications among CSRA items is the final stage of the linguistic analysis of the 
descriptions of inventions in the framework of the developed model.

 \begin{table*}\small %tabl4
    \begin{center}
\Caption{Integral and thematic indicators}
\vspace*{2ex}

\begin{tabular}{lccccccccc}
\hline
\multicolumn{1}{c}{IPC} &\multicolumn{8}{c}{Applied science classifier 
items\,/\,thematic indicator values}&Integral\\
\cline{2-9}
\multicolumn{1}{c}{index}&13.00.00&20.00.00&27.00.00&28.00.00&45.00.00&47.00.00&50.00.00&76.00.00& indicator\\
\hline
\multicolumn{1}{c}{1}&2&3&4&5&6&7&8&9&10\\
\hline
G06E&0&\hphantom{99}0&0&\hphantom{9}0&\hphantom{9}0&16&\hphantom{9}0&0&\hphantom{99}16\\
G06F&2&423&59\hphantom{9}&758\hphantom{9}&469\hphantom{9}&517\hphantom{9}&915\hphantom{9}&0&3143\\
G06G&0&\hphantom{99}0&0&\hphantom{9}2&15&27&14&0&\hphantom{99}58\\
G06K&0&539&4&587\hphantom{9}&550\hphantom{9}&602\hphantom{9}&655\hphantom{9}&2&2939\\
G06N&0&\hphantom{9}19&1&14&30&26&19&1&\hphantom{9}110\\
G06Q&0&\hphantom{99}0&0&\hphantom{9}8&\hphantom{9}0&\hphantom{9}0&\hphantom{9}1&5&\hphantom{99}14\\
G06T&0&\hphantom{99}8&51\hphantom{9}&51&\hphantom{9}8&17&80&0&\hphantom{9}215\\
\hline
\end{tabular}
\end{center}
%\vspace*{6pt}
\end{table*}
    
    
    Unification of all sets of~$M_i$ for all descriptions of inventions~$P_s$ yields 
    a~set of $M_s$ tuples of type $\langle$IPC index, CSRA item$\rangle$. After unification 
of all~$M_i$, there is a~possibility to calculate the frequency ratio of each tuple, i.\,e., 
to find out how many times it is found in set~$M_s$. If one groups together the tuples for 
each pair of IPC--CSRA, then the frequencies of these pairs will reflect the intensity of 
ties of CSRA items with a~specific technological area defined in the scenario with 
the specified CSRA and the time period of the study of these relations. If we arrange\linebreak\vspace*{-12pt}
 { \begin{center}  %fig3
 \vspace*{-4pt}
 \mbox{%
\epsfxsize=77.888mm
\epsfbox{zac-3.eps}
}


\end{center}

\vspace*{2pt}


\noindent
{{\figurename~3}\ \ \small{Data fields of invention and publication:
    $p_i$~--- description of invention $p_i$; $q_m^i$~--- IPC index of invention~$p_i$; $\pi_j^i$~---
    reference to publication cited in~$p_i$;
    $r^i_{j,k}$~--- science classifier item of publication cited in~$p_i$;
    and $\mu_{m,k}^{i,j}$~--- tuple of IPC index and science classifier
    item}}
}

\vspace*{12pt}



\noindent 
IPC indices in rows and  CSRA items in columns and  frequencies of tuples 
$\langle q_m^i, r^i_{j,k}\rangle$ within total set~$M_s$ are placed in cells with addresses 
$(q_m^i, r^i_{j,k})$ as presented in Table~4, then we obtain the desired matrix of 
frequency ratios (columns~2--9 in Table~4).
    
   
     
     The values of frequency ratios presented in Table~4 are calculated for one of the 
indicators of type $I = B(\mathrm{AR, A, F})$ as it is designated in Fig.~3. These values are 
obtained under the following conditions of information resource processing:
     \begin{itemize}
\item applied science classifier is used~\cite{25-zat};
\item only references made by the authors of inventions are taken into 
consideration; and
\item repeated items of science classifier is applied.
\end{itemize}

     The sum of all frequencies in each row (in cells from columns~2--9 in Table~4) 
makes up the value of the integral indicator ITI-I (column~10), reflecting the total 
impact of the identified research areas (branches, disciplines) on the technological 
area appropriate for the IPC index of this row under the three above mentioned 
conditions.
     
     This linguistic-mathematical approach is distinguished by the fact  that each 
tuple $\mu_{m,k}^{i,j}=\langle q^i_m, r^i_{j,k}\rangle$  is accompanied by data 
from several fields of the descriptions of inventions (publication date of the 
application or the patent, country of the applicant, etc.)\ as well as by related 
publications (year, type, country of publication, etc.)\ as presented in Fig.~3.
     
\section*{Concluding Remarks}

\noindent
Investigation of the process of knowledge transfer from different areas of 
scientific research into technology is important from a~scientific as well as 
from practical points of view, because it allows to identify those scientific 
areas, branches, and disciplines which have exercised and thus might with 
substantial probability exercise in the future a~significant influence on 
technological advances of utmost interest.

    The proposed method of calculating matrices of frequency ratios can be used for 
strategic planning to study the extent of expected influence of scientific research on 
the development of technologies of interest if RPO information resources with 
retrospective data for a~sufficiently long period of time and adequate methods of 
statistical forecasting are employed. The indicators devised for studying the influence 
of exploratory research on technologies should be regarded as information objects 
representing various aspects of the process of knowledge transfer and, first of all, the 
intensity to which results of basic and applied science are translated into  
high-technologies. The feasibility of the method was tested on the example of 
calculating indicator values of type $I = B(\mathrm{AR, A, F})$.
    
    
    
    This research entails some concomitant results with significant capabilities of 
contribution to planning, organization, and support of long-term research at federal 
level with definitely specified objectives, structure, outreach, and allocated financial 
and human resources. Such an approach to the management of scientific activities 
actually expedites viable efforts of consolidating financial, organizational, and other 
resources in specific scientific areas which most probably will provide subsequent 
technological advances crucial for economic and social development in the current 
environment.
    
   
    
 \Ack
    \noindent
    The work was done under financial support of the Russian Foundation for Basic 
Research (project No.\,16-07-00075).


\renewcommand{\bibname}{\protect\rmfamily References}

\vspace*{-6pt}

%\vspace*{-6pt}

{\small\frenchspacing
{\baselineskip=10.65pt
\begin{thebibliography}{99}
\bibitem{4-zat} %1
\Aue{Narin, F., and E.~Noma.} 1985. Is technology becoming science? 
\textit{Scientometrics} 7(3-6):369--381.
\bibitem{6-zat} %2
\Aue{Mansfield, E.} 1991. Academic research and innovation. 
\textit{Res. Policy} 20(1):1--12.
\bibitem{1-zat} %3
\Aue{Schmoch, U.} 1993. Tracing the knowledge transfer from science to 
technology as reflected in patent indicators. \textit{Scientometrics} 
26(1):193--211.
\bibitem{7-zat} %4
\Aue{Mansfield, E.} 1995. Academic research underlying industrial 
innovations: Sources, characteristics and financing. \textit{Rev. Econ. 
Statistics} 77(1):55--62.
\bibitem{5-zat} %5
\Aue{Narin, F., and D.~Olivastro}. 1998. Linkage between patents and 
papers: An interim EPO/US comparison. \textit{Scientometrics} 
 41(1-2):51--59.


\bibitem{8-zat} %6
\Aue{Mansfield, E.} 1998. Academic research and industrial innovation: An 
update of empirical findings. \textit{Res. Policy} 26(7-8):773--776.
\bibitem{2-zat} %7
\Aue{Tijssen, R.\,J.\,W., R.\,K.~Buter, and Th.\,N.~Van Leeuwen.} 2000. 
Technological relevance of science: An assessment of citation linkages 
between patents and research papers. \textit{ Scientometrics}  
47(2):389--412.
\bibitem{3-zat} %8
\Aue{Van Looy, B., E.~Zimmermann, R.~Veugelers, A.~Verbeek, J.~Mello, 
and K.~Debackere.} 2003. Do science--technology interactions pay on when 
developing technology? An exploratory investigation of~10~science-intensive 
technology domains. \textit{Scientometrics} 57(3):355--367.


\bibitem{9-zat} %9
European Commission. 2003. {3rd European Report on Science \& Technology Indicators}. 
Luxembourg: Office for Official Publications of the European Communities. 
451~p.
\bibitem{10-zat}
\Aue{Zatsman,~I.\,M., and S.\,K.~Shubnikov.} 2007. Printsipy obrabotki 
informatsionnykh resursov dlya otsenki innovatsionnogo potentsiala 
napravleniy nauchnykh issledovaniy [The principles of processing of 
information resources for estimation of innovative potential of the directions 
of scientific research]. \textit{Tr. IX~Vseross. nauchn. konf. 
``Elektronnye biblioteki: perspektivnye metody i~tekhnologii, elektronnye 
kollektsii} [9th All-Russian Scientific Conference ``Digital 
Libraries: Advanced Methods and Technologies, Digital Collections'' 
Proceedings]. Pereslavl-Zalessky: Pereslavl University.  
35--44.
\bibitem{11-zat}
\Aue{Arkhipova,~M.\,Yu., I.\,M.~Zatsman, and S.\,Yu.~Shul'-\linebreak ga.} 2010. 
Indikatory patentnoy aktivnosti v~sfere\linebreak  
in\-for\-ma\-tsi\-on\-no-kom\-mu\-ni\-ka\-tsi\-on\-nykh tekhnologiy i~me\-to\-di\-ka ikh 
vychisleniya [Indicators of patent activities in the sphere of information and 
communication technologies and a~technique of their computation]. 
\textit{Ekonomika, statistika i~informatika. Vestnik UMO} [Economics, 
statistics, and informatics: UMO Bull]. 4:93--104.
\bibitem{12-zat}
\Aue{Minin, V.\,A., I.\,M.~Zatsman, M.\,G.~Kruzhkov, and 
T.\,P.~Norekyan.} 2013. Metodologicheskie osnovy so\-zda\-niya 
informatsionnykh sistem dlya vychisleniya indikatorov tematicheskikh 
vzaimosvyazey nauki i~tekhnologiy [Methodological bases for creating 
information systems calculating indicators of thematic linkages between 
science and technologies]. \textit{Informatika i~ee Primeneniya~--- Inform. 
Appl.} 7(1):70--81.
\bibitem{13-zat}
\Aue{Minin, V.\,A., I.\,M.~Zatsman, V.\,A.~Havanskov, and 
S.\,K.~Shubnikov.} 2013. Arkhitekturnye resheniya dlya sistem vychisleniya 
indikatorov tematicheskikh vzaimo\-svya\-zey nauki i~tekhnologiy [Information 
system conceptual decisions for assessment of linkages between science and 
technologies]. \textit{Sistemy i~Sredstva Informatiki~--- Systems and Means 
of Informatics} 23(2):260--283.
\bibitem{14-zat}
\Aue{Zatsman, I.\,M., V.\,A.~Havanskov, and S.\,K.~Shubnikov.} 2013. 
Metod izvlecheniya bibliograficheskoy informatsii iz polnotekstovykh 
opisaniy izobreteniy [Method of bibliographic information extraction  from 
full-text descriptions of inventions]. \textit{Informatika i~ee Primeneniya~--- 
Inform. Appl}. 7(4):52--65.
\bibitem{15-zat}
\Aue{Havanskov, V.\,A., and S.\,K.~Shubnikov.} 2014. Poisk 
i~rubritsirovanie ssylok na tsitiruemye publikatsii v~elektronnykh 
bibliotekakh polnotekstovykh opisaniy izobreteniy [Search and classifying 
references to cited publications in electronic libraries of full-text descriptions 
of inventions]. \textit{Tr. XVI Vseross. nauchn. konf.  
``Elektronnye biblioteki: perspektivnye metody i~tekhnologii, elektronnye 
kollektsii} [16th All-Russian Scientific Conference ``Electronic 
Libraries: Perspective Methods, and Technologies, Electronic Collections'' 
Proceedings]. Dubna: JINR. 165--173.
\bibitem{16-zat}
\Aue{Minin, V.\,A., I.\,M.~Zatsman, V.\,A.~Havanskov, and 
S.\,K.~Shubnikov.} 2014. Indikatory tematicheskikh vzaimosvyazey nauki 
i~tekhnologiy: ot teksta k~chislam [Indicators of 
thematic science--technology linkages: From text to numbers]. \textit{Informatika i~ee 
Primeneniya~--- Inform. Appl}. 8(3):114--125.
\bibitem{17-zat}
\Aue{Minin, V.\,A., I.\,M.~Zatsman, V.\,A.~Havanskov, and 
S.\,K.~Shubnikov.} 2015. Indikatory tematicheskikh vzaimosvyazey nauki  
i~informatsionno-komp'yuternykh tekhnologiy v~nachale XXI~veka 
[Indicators for thematic linkages between science and information and
computer technologies at the beginning of the XXI~century]. 
\textit{Informatika i~ee Primeneniya~--- Inform. Appl.} 9(2):111--120.
\bibitem{18-zat} %18
\Aue{Minin, V.\,A., I.\,M.~Zatsman, V.\,A.~Havanskov, and 
S.\,K.~Shubnikov.} 2016. Intensivnost' tsitirovaniya nauchnykh publikatsiy 
v~izobreteniyakh po informatsionno-komp'yuternym tekhnologiyam, 
patentuemykh v~Rossii otechestvennymi i~zarubezhnymi zayavitelyami 
[Intensity of citation of scientific publications in inventions on information and
computer technologies patented 
in Russia by domestic and foreign applicants]. \textit{Informatika i~ee  
Primeneniya~--- Inform. Appl.} 10(2):107--122.


\bibitem{21-zat} %19
\Aue{Zatsman, I.\,M., and O.\,S.~Kozhunova.} 2007. Se\-man\-ti\-che\-skiy slovar' 
sistemy informatsionnogo monitoringa v~sfere nauki: zadachi i~funktsii 
[Semantic dictionary of information monitoring system in science: Tasks and 
functions]. \textit{Sistemy i~Sredstva Informatiki~--- Systems and Means of 
Informatics} 17(1):124--141.
\bibitem{22-zat} %20
\Aue{Zatsman, I., and O.~Kozhunova.} 2008. Evaluating for institutional 
academic activities: Classification scheme for R\&D indicators. \textit{10th 
Conference (International) on Science and Technology Indicators: 
Book of abstracts}. Vienna: ARC GmbH.  
428--431.
\bibitem{23-zat} %21
\Aue{Zatsman, I., and O.~Kozhunova.} 2009. Evaluation system for the 
Russian Academy of Sciences: Objectives-resources-results approach and 
R\&D indicators. \textit{2009 Atlanta Conference on Science and Innovation 
Policy Proceedings}. Eds.\ S.\,E.~Cozzens and P.~Catalаn. Available at: 
{\sf http://smartech.gatech.edu/bitstream/1853/32300/1/ 104-674-1-PB.pdf} 
(accessed July~14, 2017).
\bibitem{20-zat} %22
\Aue{Zatsman, I., and A.~Durnovo.} 2010. Incompleteness problem of 
indicators system of research programme. \textit{11th Conference 
(International) on Science and Technology Indicators: Book of 
abstracts}. Leiden: CWTS. 309--311.
\bibitem{19-zat} %23
\Aue{Zatsman, I.\,M., and A.\,A.~Durnovo.} 2011. Modelirovanie 
protsessov formirovaniya ekspertnykh znaniy dlya monitoringa  
programmno-tselevoy deyatel'nosti [Modeling of processes for creation of 
expert knowledge for monitoring of goal-oriented programme activities]. 
\textit{Informatika i~ee Primeneniya~--- Inform. Appl.} 5(4):84--98.
\bibitem{24-zat} %24
Standart~ST.14. 2016. Rekomendatsii po vklyucheniyu ssylok, tsitiruemykh 
v~patentnykh dokumentakh [Standard ST.14. Recommendations for the 
inclusion of references cited in patent documents]. Available at:  {\sf 
http://www. rupto.ru/docs/interdocs/stand\_wipo/03\_14\_01.pdf} (accessed 
July~14, 2017).
\bibitem{25-zat}
Gosudarstvennyy rubrikator na\-uch\-no-tekh\-ni\-che\-skoy informatsii (GRNTI)
[The State Classifier of Scientific and Technical Information]. Available at:  
{\sf http://grnti.ru} (accessed July~14, 2017).
\end{thebibliography} } }

\end{multicols}

\vspace*{-12pt}

\hfill{\small\textit{Received July 14, 2017}}



%\vspace*{-3pt}

\Contr

\noindent
\textbf{Zatsman Igor M.} (b.\ 1952)~--- 
Doctor of Science in technology, Head of Department, Institute of Informatics 
Problems, Federal Research Center ``Computer Science and Control'' 
of the Russian Academy of Sciences, 44-2~Vavilov Str., Moscow 119333, 
Russian Federation; \mbox{izatsman@yandex.ru}  

\vspace*{3pt}

\noindent
\textbf{Lukyanov Gennady V.} (b.\ 1952)~--- 
Candidate of Military Science (PhD), associate professor, leading scientist, 
Institute of Informatics Problems, Federal Research Center ``Computer Science and 
Control'' of the Russian Academy of Sciences, 44-2~Vavilov Str., Moscow 119333, Russian Federation; 
\mbox{gena-mslu@mail.ru} 

\vspace*{3pt}

\noindent
\textbf{Minin Vladimir A.} (b.\ 1941)~--- 
Doctor of Science in physics and mathematics, consultant, Institute of Informatics 
Problems, Federal Research Center ``Computer Science and Control'' of 
the Russian Academy of Sciences, 44-2~Vavilov Str., Moscow 119333, Russian Federation; 
\mbox{aleksisss@ya.ru}  

\vspace*{3pt}

\noindent
\textbf{Havanskov Valerij A.} (b.\ 1950)~--- 
scientist, Institute of Informatics Problems, Federal Research Center
``Computer Science and Control'' of the Russian Academy of Sciences, 
44-2~Vavilov Str., Moscow 119333, Russian Federation; \mbox{chavanskov@yandex.ru} 

\vspace*{3pt}

\noindent
\textbf{Shubnikov Sergej K.} (b.\ 1955)~--- 
senior scientist, Institute of Informatics Problems, Federal Research Center 
``Computer Science and Control'' of the Russian Academy of Sciences, 
44-2~Vavilov Str., Moscow 119333, Russian Federation; \mbox{sergeysh50@yandex.ru}


\vspace*{12pt}

\hrule

\vspace*{2pt}

\hrule

%\newpage

%\vspace*{-24pt}

\vspace*{8pt}

\def\tit{ИНДИКАТОРНОЕ ОЦЕНИВАНИЕ ПРОЦЕССОВ ПЕРЕНОСА ЗНАНИЙ 
ИЗ~ОБЛАСТИ НАУЧНЫХ ИССЛЕДОВАНИЙ\\ В~СФЕРУ ТЕХНОЛОГИЧЕСКОГО РАЗВИТИЯ$^*$}




\def\titkol{Индикаторное оценивание процессов переноса знаний 
из~области научных исследований} % в~сферу технологического развития}

\def\aut{И.\,М.~Зацман,
Г.\,В.~Лукьянов,
В.\,А.~Минин, В.\,А.~Хавансков, С.\,К.~Шубников}

\def\autkol{И.\,М.~Зацман,
Г.\,В.~Лукьянов, В.\,А.~Минин  и др.} %, В.\,А.~Хавансков, С.\,К.~Шубников}



{\renewcommand{\thefootnote}{\fnsymbol{footnote}}
\footnotetext[1]{Работа выполнена при поддержке РФФИ, проект № 16-07-00075.}}


\titel{\tit}{\aut}{\autkol}{\titkol}

\vspace*{-12pt}


\noindent
Институт проблем информатики Федерального исследовательского центра 
<<Информатика и управление>> Российской академии наук

\vspace*{6pt}

\def\leftfootline{\small{\textbf{\thepage}
\hfill ИНФОРМАТИКА И ЕЁ ПРИМЕНЕНИЯ\ \ \ том\ 11\ \ \ выпуск\ 3\ \ \ 2017}
}%
 \def\rightfootline{\small{ИНФОРМАТИКА И ЕЁ ПРИМЕНЕНИЯ\ \ \ том\ 11\ \ \ выпуск\ 3\ \ \ 2017
\hfill \textbf{\thepage}}}

\Abst{Данная работа посвящена индикаторному оцениванию информационных 
связей науки и~технологий. Индикаторы связей определяются как число или матрица 
числовых значений, которые\linebreak\vspace*{-12pt}}

\Abstend{характеризуют интенсивность и~различные 
аспекты процесса переноса знаний из разных областей исследований в~сферу технологий. 
Дано описание первичных информационных ресурсов, используемых для 
определения значений этих индикаторов, включая полнотекстовые описания 
изобретений. Приводится описание вторичных информационных ресурсов, 
генерируемых в~процессе обработки полнотекстовых описаний, включая 
информацию о~ссылках на научные публикации, цитируемые в~описаниях. 
Исходные и~вторичные ресурсы использовались при создании и~апробации 
информационной модели индикаторного оценивания связей науки и~технологий. 
На ее основе были определены значения интегральных и тематических индикаторов 
интенсивности 
переноса научных знаний в~сферу разработки информационных технологий.}

\KW{информационные взаимосвязи науки и технологий; цитирование научных работ; интенсивность процесса переноса знаний; 
индикаторное оценивание; информационные технологии}
     
   
\DOI{10.14357/19922264170315} 


%\vspace*{6pt}


 \begin{multicols}{2}

\renewcommand{\bibname}{\protect\rmfamily Литература}
%\renewcommand{\bibname}{\large\protect\rm References}

{\small\frenchspacing
{%\baselineskip=10.8pt
\begin{thebibliography}{99}

\bibitem{4-zat-1} %1
\Au{Narin F., Noma~E.} Is technology becoming science?~// 
Scientometrics, 1985. Vol.~7. No.\,3-6. P.~369--381.

\bibitem{6-zat-1} %2
\Aue{Mansfield E.} Academic research and innovation~// 
Res. Policy, 1991. Vol.~20. No.\,1. P.~1--12.

\bibitem{1-zat-1} %3
\Au{Schmoch U.}  
Tracing the knowledge transfer from science to technology as reflected in patent 
indicators~// Scientometrics, 1993. Vol.~26. No.\,1. P.~193--211.

\bibitem{7-zat-1} %4
\Au{Mansfield E.} Academic research underlying industrial innovations: Sources,
 characteristics and financing~// Rev. Econ. Statistics, 1995. Vol.~77. No.\,1. P.~55--62.
 
 \bibitem{5-zat-1} %5
\Au{Narin F., Olivastro~D.} 
Linkage between patents and papers: An interim EPO/US comparison~// 
Scientometrics, 1998. Vol.~41. No.\,1-2. P.~51--59.


\bibitem{8-zat-1} %6
\Au{Mansfield E.} Academic research and industrial innovation: An update of empirical 
findings~// Res. Policy, 1998. Vol.~26. No.\,7-8. P.~773--776.
\bibitem{2-zat-1} %7
\Au{Tijssen R.\,J.\,W., Buter~R.\,K., Van Leeuwen~Th.\,N.}
Technological relevance of science: An assessment of citation linkages between 
patents and research papers~// Scientometrics, 2000. Vol.~47. No.\,2. P.~389--412.
\bibitem{3-zat-1} %8
\Au{Van Looy B., Zimmermann~E., Veugelers~R., Verbeek~A., Mello~J., Debackere~K.}
Do science--technology interactions pay on when developing technology? 
An exploratory investigation of~10~science-intensive technology\linebreak domains~// 
Scientometrics, 2003. Vol.~57. No.\,3. P.~355--367.
{ %\looseness=1

}


\bibitem{9-zat-1} %9
European Commission. 3rd European Report on Science \& Technology Indicators.  
Luxembourg: Office for Official Publications of the European Communities, 2003. 
451~p.
\bibitem{10-zat-1}
\Au{Зацман И.\,М., Шубников~С.\,К.} 
Принципы обработки информационных ресурсов для оценки инновационного потенциала 
направлений научных исследований~// Электронные библиотеки: перспективные методы 
и~технологии, электронные коллекции: Тр. IX~Всеросс. научн. конф.~--- 
Переславль: Университет города Переславля, 2007. С.~35--44.
\bibitem{11-zat-1}
\Au{Архипова М.\,Ю., Зацман~И.\,М., Шульга~С.\,Ю.} 
Индикаторы патентной активности в~сфере 
ин\-фор\-ма\-ци\-он\-но-ком\-му\-ни\-ка\-ци\-он\-ных 
технологий и~методика их вычисления~// Экономика, статистика и~информатика: 
Вестник УМО, 2010. №\,4. С.~93--104.

\columnbreak 

\bibitem{12-zat-1}
\Au{Минин В.\,А., Зацман~И.\,М., Кружков~М.\,Г., Но\-ре\-кян~Т.\,П.}
Методологические основы создания информационных систем для вычисления индикаторов 
тематических взаимосвязей науки и технологий~// Информати\-ка и~её применения, 2013. 
Т.~7. Вып.~1. С.~70--81.
\bibitem{13-zat-1}
\Au{Минин В.\,А., Зацман~И.\,М., Хавансков~В.\,А., Шубников~С.\,К.}
Архитектурные решения для систем вы\-чис\-ле\-ния индикаторов тематических взаимосвязей 
науки и~технологий~// Системы и средства информатики, 2013. Т.~23. №\,2. C.~260--283.
\bibitem{14-zat-1}
\Au{Зацман И.\,М., Хавансков~В.\,А., Шубников~С.\,К.}
Метод извлечения библиографической информации из полнотекстовых описаний изобретений~// 
Информатика и~её применения, 2013. Т.~7. Вып.~4. С.~52--65.
\bibitem{15-zat-1}
\Au{Хавансков В.\,А., Шубников~С.\,К.} Поиск и~рубрицирование ссылок на 
цитируемые публикации в электронных библиотеках полнотекстовых описаний изобретений~// 
Электронные библиотеки: перспективные методы и технологии, электронные коллекции: Тр.\
XVI~Всеросс. научн. конф.~--- Дубна: ОИЯИ, 2014. С.~165--173.
\bibitem{16-zat-1}
\Au{Минин В.\,А., Зацман~И.\,М., Хавансков~В.\,А., Шубников~С.\,К.}
 Индикаторы тематических взаимосвязей науки и~технологий: от текста к~числам~// 
 Информатика и~ёе применения, 2014. Т.~8. Вып.~3. С.~114--125.
\bibitem{17-zat-1}
\Au{Минин В.\,А., Зацман~И.\,М., Хавансков~В.\,А., Шубников~С.\,К.}
Индикаторы тематических взаимосвязей науки 
и~ин\-фор\-ма\-ци\-он\-но-ком\-пью\-тер\-ных технологий в~начале~XXI~века~// 
Информатика и~её применения, 2015. Т.~9. Вып.~2. С.~111--120.
\bibitem{18-zat-1}
\Au{Минин В.\,А., Зацман~И.\,М., Хавансков~В.\,А., Шубников~С.\,К.}
Интенсивность цитирования научных пуб\-ли\-ка\-ций в изобретениях по 
ин\-фор\-ма\-ци\-он\-но-ком\-пью\-тер\-ным технологиям, патентуемых 
в~России отечествен\-ными и~зарубежными заявителями~// Информатика и~её 
применения, 2016. Т.~10. Вып.~2. С.~107--122.

\bibitem{21-zat-1} %19
\Au{Зацман И.\,М., Кожунова~О.\,С.} Семантический словарь системы информационного 
мониторинга в~сфере науки: задачи и~функции~// Системы и~средства информатики, 2007. 
Т.~17. №\,1. С.~124--141.

\bibitem{22-zat-1} %20
\Au{Zatsman I., Kozhunova~O.} Evaluating for institutional academic activities: 
Classification scheme for R\&D indicators~// 10th  Conference 
(International) on Science and Technology Indicators: Book of abstracts.~--- 
Vienna: ARC GmbH, 2008. P.~428--431.
\bibitem{23-zat-1} %21
\Au{Zatsman I., Kozhunova~O.}
Evaluation system for the Russian Academy of Sciences: Objectives-resources-results 
approach and R\&D indicators~// 2009 Atlanta Conference on Science and Innovation 
Policy Proceedings~/ Eds. S.\,E.~Cozzens, P.~Catalаn. 
{\sf http:// smartech.gatech.edu/bitstream/1853/32300/1/104-674-1-PB.pdf}.

\bibitem{20-zat-1} %22
\Au{Zatsman I., Durnovo~A.} Incompleteness 
problem of indicators system of research programme~// 
11th  Conference (International) on Science and Technology Indicators: 
Book of abstracts.~--- Leiden: CWTS, 2010. P.~309--311.

\bibitem{19-zat-1} %23
\Au{Зацман И.\,М., Дурново~А.\,А.}
Моделирование процессов формирования экспертных знаний для мониторинга
 про\-граммно-це\-ле\-вой деятельности~// Информатика и~её применения, 2011. 
 Т.~5. Вып.~4. С.~84--98.


\bibitem{24-zat-1}
Стандарт ST.14. Рекомендации по включению ссылок, цитируемых в патентных документах~// 
Справочник по информации и~документации в~об\-ласти промышленной собственности.~--- 
WIPO, 2016. С.~3-14-1--3-14-12. 
{\sf http://www.rupto.ru/docs/\linebreak interdocs/stand\_wipo/03\_14\_01.pdf}.
\bibitem{25-zat-1}
Государственный рубрикатор на\-уч\-но-тех\-ни\-че\-ской информации (ГРНТИ).
{\sf  http://grnti.ru}. 


\end{thebibliography}
} }

\end{multicols}

 \label{end\stat}

 \vspace*{-3pt}

\hfill{\small\textit{Поступила в редакцию  14.07.2017}}
%\renewcommand{\bibname}{\protect\rm Литература}
\renewcommand{\figurename}{\protect\bf Рис.}
\renewcommand{\tablename}{\protect\bf Таблица}