\def\stat{lange}

\def\tit{ОБ ЭФФЕКТИВНОСТИ ИЕРАРХИЧЕСКОГО АЛГОРИТМА ПОИСКА ПРИБЛИЖЕННОГО 
БЛИЖАЙШЕГО СОСЕДА В~ЗАДАННОМ НАБОРЕ ИЗОБРАЖЕНИЙ$^*$}

\def\titkol{Об эффективности иерархического алгоритма поиска приближенного 
ближайшего соседа в~заданном наборе} % изображений}

\def\aut{М.\,М.~Ланге$^1$, С.\,Н.~Ганебных$^2$, А.\,М.~Ланге$^3$}

\def\autkol{М.\,М.~Ланге, С.\,Н.~Ганебных, А.\,М.~Ланге}

\titel{\tit}{\aut}{\autkol}{\titkol}

\index{Ланге М.\,М.}
\index{Ганебных С.\,Н.}
\index{Ланге А.\,М.}
\index{Lange M.\,M.}
\index{Ganebnykh S.\,N.}
\index{Lange A.\,M.}


{\renewcommand{\thefootnote}{\fnsymbol{footnote}} \footnotetext[1]
{Работа выполнена при поддержке РФФИ (проекты 15-07-07516 и~15-07-09324).}}


\renewcommand{\thefootnote}{\arabic{footnote}}
\footnotetext[1]{Федеральный исследовательский центр <<Информатика и~управ\-ле\-ние>> 
Российской академии наук, 
\mbox{lange\_mm@ccas.ru}}
\footnotetext[2]{Федеральный исследовательский центр <<Информатика и~управ\-ле\-ние>> 
Российской академии наук, 
\mbox{sng@ccas.ru}}
\footnotetext[3]{Федеральный исследовательский центр <<Информатика и~управ\-ле\-ние>> Российской 
академии наук, \mbox{lange\_am@mail.ru}}

%\vspace*{-18pt}

   

\Abst{Исследуется эффективность иерархического алгоритма поиска в~заданном наборе 
изображений близкого представителя к~предъявляемому изображению с~негарантированной 
погрешностью относительно ближайшего соседа. Алгоритм использует пространство 
квадропирамидальных пред\-став\-ле\-ний изображений и~стратегию направленного поиска на 
последовательных уровнях представления с~нарастающим разрешением. Эффективность 
алгоритма исследуется в~терминах эмпирического распределения погрешностей поиска 
и~вычислительной сложности относительно сложности полного перебора. Приводятся 
эмпирические распределения погрешностей и~оценки вычислительной сложности алгоритма 
для двух приложений: поиска в~наборе изображений рукописных цифр из базы данных 
MNIST и~координатной привязки зашумленных изображений к~аэрокосмической карте 
местности из сетевого сервиса Google Maps.} 

\KW{изображение; квадропирамидальное представление; цифровая карта; ближайший сосед; 
приближенный ближайший сосед; погрешность поиска; эмпирическое 
распределение; вычислительная сложность}

\DOI{10.14357/19922264170306} 

\vspace*{4pt}


\vskip 10pt plus 9pt minus 6pt

\thispagestyle{headings}

\begin{multicols}{2}

\label{st\stat}

\section{Введение}

\vspace*{-4pt}

  Существует широкий класс прикладных задач, связанных с~поиском 
в~большом наборе данных объекта, близкого по заданной мере 
к~предъявляемому объекту. К~таким задачам относится поиск в~наборах 
изображений образцов, сходных с~предъявляемыми изображениями, 
координатная привязка наблюдаемого изображения к~цифровой карте 
местности и~др. Перечисленные примеры относятся к~проблеме извлечения 
изображений (Image Retrieval)~[1]. При больших объемах данных решающий 
алгоритм должен удовлетворять заданным требованиям к~допустимой 
по\-греш\-ности (точ\-ности) поиска и~вычислительной слож\-ности 
(быст\-ро\-дей\-ст\-вию). Как правило, эти характеристики находятся в~обратной 
зависимости: с~увеличением быст\-ро\-дей\-ст\-вия уменьшается точ\-ность (растет 
по\-греш\-ность) и~наоборот. Поэтому необходимо обеспечить баланс этих 
требований путем варьирования па\-ра\-мет\-ра\-ми решающего алгоритма.
  
  В качестве решающих алгоритмов могут быть использованы алгоритмы 
поиска приближенного ближайшего соседа~[2--6] или их модификации. При 
фиксированной размерности $d\hm\geq 1$ векторного пространства известные 
алгоритмы с~гарантированной точностью, задаваемой допустимой 
погрешностью $\varepsilon\hm>0$, реализуют поиск в~наборе из   векторов 
представителя на расстоянии $D\hm\leq (1\hm+\varepsilon)D_{\min}$ от 
предъявляемого вектора, где $D_{\min}\hm>0$~--- расстояние до ближайшего 
соседа. Такие алгоритмы используют древовидные структуры данных, которые 
при фиксированных~$d$ и~$\varepsilon$ позволяют уменьшить порядок роста 
вычислительной сложности по~$n$ по сравнению со сложностью переборного 
алгоритма. В~частности, BBD-ал\-го\-ритм, использующий решающее Balance 
Box Decision дерево~\cite{5-lan}, имеет сложность $O(d\lceil 
1\hm+6d/\varepsilon\rceil^d\log n)$, а сложность LSH-ал\-го\-рит\-ма на основе 
локального хеширования Locality Sensitive Hashing~\cite{6-lan} составляет 
$O(dn^{1/(1+\varepsilon)})$. Для сравнения переборный алгоритм поиска 
ближайшего соседа ($\varepsilon\hm=0$) имеет сложность~$\Theta(dn)$. 
Характер зависимости вычислительной сложности указанных алгоритмов от 
размерности~$d$ и~допустимой погрешности   ограничивает их применение 
для поиска изображений размера $N\times N$ из-за чрезмерно высокой 
размерности $d\hm=N^2$ при $N\hm\geq 100$. 
  
  В настоящей работе рассматривается альтернативный алгоритм для быстрого 
поиска в~заданном наборе изображений мощ\-ности~$n$ приближенного 
ближайшего соседа к~предъявляемому изображению. Предлагаемый алгоритм 
является модификацией алгоритма, рассмотренного в~\cite{7-lan}, которая 
использует квадропирамидальное представление\linebreak изобра\-жений 
с~многоуровневым разрешением~[8--10]. Алгоритм базируется на 
параметрической стратегии экспоненциального сужения зоны поиска, 
аналогичной использованной в~работе~\cite{11-lan}. Такая стратегия поиска 
обеспечивает вычислительную сложность $O(n\log N)$, но не гарантирует 
точности приближенного решения.
  
  Эффективность предложенного алгоритма исследована в~терминах 
эмпирического распределения погрешностей поиска и~вычислительного 
выигры\-ша относительно алгоритма полного перебора. При различных 
значениях параметра алгоритма указанные характеристики получены для 
поиска изображений рукописных цифр из базы данных MNIST~\cite{12-lan} 
и~для координатной привязки зашумленных изображений к~цифровой карте 
участка земной поверхности, взятой на ин\-тер\-нет-сер\-ви\-се Google 
Maps~\cite{13-lan}. 
  
  \section{Формальная постановка задачи}
  
  Рассматривается множество изображений $\mathbf{X}$ размера $N\times N$, 
элементы которых принадлежат алфавиту $A\hm=\{0,1,\ldots ,q-1\}$ ($q\hm\geq 
2$). Предполагается, что каждое изображение $\mathbf{x}\hm\in\mathbf{X}$ 
имеет по крайней мере один ненулевой элемент, а размер изображения является 
целочисленной степенью числа~2, так что $N\hm=2^L$, где $L\hm\gg1$. 
В~общем случае мно\-же\-ст\-во~$\mathbf{X}$ содержит набор 
изображений~$\hat{\mathbf{X}}$, в~котором необходимо найти изображение 
$\hat{\mathbf{x}}\hm\in\hat{\mathbf{X}}$, ближайшее или достаточно близкое по 
заданной мере к~заданному изображению $\mathbf{x}\hm\in\mathbf{X}$.
  
  Любое изображение $\mathbf{x}\hm\in\mathbf{X}$ с~указанными размерами 
допускает квадропирамидальное пред\-став\-ление 
  \begin{equation}
  \mathbf{x}_L=\left( x_0, \ldots , x_l, \ldots , x_L\right)
  \label{e1-lan}
  \end{equation}
порядка~$L$, которое содержит последовательность описаний 
изображения~$\mathbf{x}$ с~уровнями разрешения $l\hm=0,\ldots , 
L$~\cite{8-lan}. Пример квадропирамиды порядка $L\hm=2$ дан на рис.~1.



  Описание $l$-го уровня~$x_l$ в~(1) является изоб\-ра\-же\-ни\-ем размера 
$2^l\cdot 2^l$, получаемым из описания~$x_{l+1}$~ путем усреднений по 
непересекающимся группам из четырех смежных элементов. Ненулевое 
значение элемента в~вершине пирамиды~$x_0$ позволяет сформировать 
нормализованное квадропредставление
  \begin{equation}
  \mathbf{y}_L=\left( y_1, \ldots , y_l,\ldots ,y_L\right)
  \label{e2-lan}
  \end{equation}
путем деления всех элементов пирамиды~(1) на значение элемента 
вершины~$x_0$. Нормализация элементов уменьшает зависимость 
представления~(2) от средней яркости изображения по сравнению 
с~представлением~(1). 
  
  Пусть $z(k_{l1}, k_{l2})>0$~--- значение элемента с~индексами $(k_{l1}, 
k_{l2})\hm=1,\ldots , 2^l$ в~нормализованном описании $y_l\hm\in 
\mathbf{y}_L$. Для любой пары изображений $\mathbf{x}\hm\in\mathbf{X}$, 
$\hat{\mathbf{x}}\hm\in\mathbf{X}$, имеющих нормализованные описания 
$\mathbf{y}_L, \hat{\mathbf{y}}_L$, вводится мера их различия $l$-го порядка:
  \begin{multline}
  D_l\left(\mathbf{x}, \hat{\mathbf{x}}\right) = \fr{1}{(2^l\cdot 2^l)} 
\sum\limits_{k_{l1}=1}^{2^l} \sum\limits^{2^l}_{k_{l2}=1} \left\vert z\left( 
k_{l1}, k_{l2}\right) -{}\right.\\
\left.{}-\hat{z}\left(k_{l1}, k_{l2}\right) %-\hat{z}\left( k_{l1}, k_{l2}\right)
\right\vert\,,\enskip l=1,\ldots ,L\,,
  \label{e3-lan}
  \end{multline}
где $ z\left( k_{l1}, k_{l2}\right)$ и~$ \hat{z}\left( k_{l1}, k_{l2}\right)$~--- 
элементы описаний $y_l\hm\in\mathbf{y}_L$ и~$\hat{y}_l\hm\in 
\hat{\mathbf{y}}_L$ соответственно для изображений~$\mathbf{x}$ 
и~$\hat{\mathbf{x}}$. Суммирование мер~$D_t(\mathbf{x}, \hat{\mathbf{x}})$ 
вида~(3) порядка $t\hm=1,\ldots , l$ ($l\hm\leq L$) с~весами 
$w_t=(1/2)\log_2(2^t\cdot 2^t)\hm=t$ порождает взвешенную меру различия 
порядка~$l$:
\begin{equation}
\tilde{D}_l\left( \mathbf{x},\hat{\mathbf{x}}\right)=\sum\limits^l_{t=1} w_t 
D_t\left(\mathbf{x}, \hat{\mathbf{x}}\right)\,,\enskip 1\leq l\leq L\,.
\label{e4-lan}
\end{equation}
  
  Точный или приближенный поиск ближайшего соседа для предъявляемого 
изображения $\mathbf{x}\hm\in\mathbf{X}$ выполняется на подмножестве 
$\hat{\mathbf{X}}\subset \mathbf{X}$, содержащем~$n$~изображений. Алгоритм 
поиска принимает решение $\hat{\mathbf{x}}^*\hm\in \hat{\mathbf{X}}$ по 
мере~(4) наибольшего порядка~$L$ на наборе изображений 
$\hat{\mathbf{X}}^*\hm\subseteq \hat{\mathbf{X}}$ мощности $n^*\hm\leq n$ 
в~соответствии с~решающим правилом
  \begin{equation}
  \hat{\mathbf{x}}^*=\mathrm{arg}\,\min\limits_{\hat{\mathbf{x}}\in \hat{\mathbf{X}}^*} 
\tilde{D}_L\left(\mathbf{x},\hat{\mathbf{x}}\right)\,.
  \label{e5-lan}
  \end{equation}
  
  Набор $\hat{\mathbf{X}}^*$ в~(5) определяется используемой стратегией 
направленного (иерархического) поиска, которая в~случае 
$\hat{\mathbf{X}}^*\hm=\hat{\mathbf{X}}$ обеспечивает точное 
решение~$\hat{\mathbf{x}}^*_{\mathrm{NN}}$ (Nearest Neighbor), 
совпадающее\linebreak\vspace*{-12pt}

 { \begin{center}  %fig1
 \vspace*{1pt}
 \mbox{%
\epsfxsize=70.052mm
\epsfbox{lan-1.eps}
}


\vspace*{9pt}


\noindent
{{\figurename~1}\ \ \small{Квадропирамида $\mathbf{x}_2\hm=\left( x_0, x_1, x_2\right)$ порядка $L\hm=2$}}
\end{center}
}

%\vspace*{6pt}

\addtocounter{figure}{1}



\noindent
 с~ближайшим соседом, а~в~случае $\hat{\mathbf{X}}^*\hm\subset 
\hat{\mathbf{X}}$~--- приближенное 
решение~$\hat{\mathbf{x}}^*_{\mathrm{AN}}$ (Approximate Nearest), которое 
может отличаться от ближайшего соседа или совпадать с~ним. 

Погрешность поиска по правилу~(5) приближенного ближайшего изображения 
в~заданном наборе определяется различием значений $\tilde{D}_L(\mathbf{x}, 
\hat{\mathbf{x}}^*_{\mathrm{AN}})$ и~$\tilde{D}_L(\mathbf{x}, 
\hat{\mathbf{x}}^*_{\mathrm{NN}})$ и~в~случае $\tilde{D}_L(\mathbf{x}, 
\hat{\mathbf{x}}^*_{\mathrm{NN}})\hm >0$ равна:
\begin{equation}
\varepsilon^s_{n^*} \left( \hat{\mathbf{x}}^*_{\mathrm{AN}}, 
\hat{\mathbf{x}}^*_{\mathrm{NN}}\right) = 
\fr{\tilde{D}_L(\mathbf{x},\hat{\mathbf{x}}^*_{\mathrm{AN}})- 
\tilde{D}_L(\mathbf{x},\hat{\mathbf{x}}^*_{\mathrm{NN}})} 
{\tilde{D}_L(\mathbf{x},\hat{\mathbf{x}}^*_{\mathrm{NN}})}\,.
\label{e6-lan}
\end{equation}
  
  В задаче координатной привязки погрешность определяется отклонением 
координат $(a^*, b^*)$ найден\-но\-го по правилу~(5) 
изображения~$\hat{\mathbf{x}}^*_{\mathrm{AN}}$ от координат~$(a,b)$ 
предъявляемого изображения~$\mathbf{x}$ и~в~случае изображений размера 
$N\times N$ равна: 
  \begin{equation}
  \varepsilon^g_{n^*}\left( \hat{\mathbf{x}}^*_{\mathrm{AN}}, \mathbf{x}\right) 
=\fr{1}{2}\left( \fr{|a-a^*|}{N}+\fr{|b-b^*|}{N}\right)\,.
  \label{e7-lan}
  \end{equation}
  
  Качество рассматриваемого алгоритма исследуется в~терминах 
распределений 
  \begin{align}
  &\mathrm{Pr} \left\{ \varepsilon^s_{n^*}\left( 
\hat{\mathbf{x}}^*_{\mathrm{AN}}, \hat{\mathbf{x}}^*_{\mathrm{NN}}\right) \leq 
\varepsilon\right\}\,;\label{e8-lan}\\
  &\mathrm{Pr} \left\{ \varepsilon^g_{n^*}\left( 
\hat{\mathbf{x}}^*_{\mathrm{AN}}, \hat{\mathbf{x}}\right) \leq 
\varepsilon\right\}\label{e9-lan}
  \end{align}
погрешностей~(6) и~(7) с~параметром~$n^*$: $1\hm\leq n^*\hm\leq n$, где 
$\varepsilon\hm\geq 0$~--- допустимая погрешность, при\-ни\-ма\-ющая значения 
с~некоторым шагом. При фиксированном значении~$\varepsilon$ вероятности~(8) 
и~(9) соответствуют надежности выполнения точности поиска и~привязки. 
  
  В разд.~3 дается описание алгоритма поиска и~приводятся оценки его 
вычислительной сложности при больших значениях~$n$ и~$N$, связанных 
соотношением $N^2\hm\geq \log_q n$. В~разд.~4 приводятся параметрическое 
семейство эмпирических распределений~(8) с~параметром~$n^*$ 
и~численные оценки сложности алгоритма, 
полученные на множестве изображений рукописных цифр. Аналогичные 
параметрические семейства эмпирических распределений~(9) и~оценки 
вычислительной сложности, полученные для координатной привязки 
зашумленных изображений к~карте участка земной поверхности, приводятся 
в~разд.~5. 
  
  \section{Алгоритм поиска}
  
  Предполагается, что изображения из набора~$\hat{\mathbf{X}}$, на котором 
производится поиск, заданы нормализованными пирамидальными 
представлениями вида~(2) и~образуют многоуровневую сеть
  \begin{equation}
  \hat{\mathbf{Y}}_1,\ldots , \hat{\mathbf{Y}}_l,\ldots , \hat{\mathbf{Y}}_L\,,
  \label{e10-lan}
  \end{equation}
в которой $\hat{\mathbf{Y}}_l$~--- подмножество представлений всех 
изображений из~$\hat{\mathbf{X}}$, заданных~$l$~уровнями нормализованных 
пирамид. Алгоритм поиска решения~(5) использует стратегию 
последовательного сужения зоны поиска на уровнях $l\hm=1,\ldots , L$ 
сети~(\ref{e10-lan}). Согласно этой стратегии число анализируемых 
изоб\-ра\-же\-ний на $l$-м уровне определяется экспоненциальной функцией 
\begin{equation}
n_l=\left\lfloor n\cdot 4^{-\alpha(l-1)}\right\rfloor\,,\enskip l=1,\ldots , L\,,
\label{e11-lan}
\end{equation}
с~коэффициентом $\alpha\hm= (L\hm-1)^{-1}\log_4(n/n^*)$, где 
$n^*\hm=1,2,\ldots , n$~--- мощность набора $\hat{\mathbf{X}}^*\hm\subseteq 
\hat{\mathbf{X}}$, на котором принимается решение~(5) по 
представлениям~$\hat{\mathbf{Y}}_L$ последнего уровня сети~(\ref{e10-lan}).

  \paragraph*{Алгоритм поиска.} Для предъявляемого изображения 
$\mathbf{x}\hm\in\mathbf{X}$ на последовательных уровнях $l\hm=1,\ldots , 
L\hm-1$ сети~(\ref{e10-lan}) вычисляются значения меры 
различия~$\tilde{D}_l(\mathbf{x}, \hat{\mathbf{x}})$ вида~(4) 
для~$n_l$~изображений из набора~$\hat{\mathbf{X}}$ и~среди них 
отбираются~$n_{l+1}$~изображений с~наименьшими 
значениями~$\tilde{D}_l(\mathbf{x}, \hat{\mathbf{x}})$; на уровне $l\hm=L$ 
среди $n_L\hm=n^*$ изображений отбирается ближайшее с~наименьшим 
значением~$\tilde{D}_L(\mathbf{x}, \hat{\mathbf{x}})$, которое дает 
решение~(5). 
  
  В случае $1\leq n^*<n$ параметр $\alpha\hm>0$ в~(\ref{e11-lan}) 
обеспечивает экспоненциальное сужение зоны иерархического поиска, которое 
приводит к~нахождению приближенного ближайшего соседа на наборе 
изоб\-ра\-же\-ний $\hat{\mathbf{X}}^*\hm\subset \hat{\mathbf{X}}$ мощ\-ности~$n^*$; 
в~случае $n^*\hm=n$ параметр $\alpha\hm=0$ приводит к~переборному поиску 
ближайшего соседа на наборе $\hat{\mathbf{X}}^*\hm\equiv \hat{\mathbf{X}}$ 
мощности~$n$. В~обоих случаях вычисление меры различия изображений 
производится с~использованием рекурсии 
  \begin{multline}
  \tilde{D}_l\left(\mathbf{x},\hat{\mathbf{x}}\right)=\tilde{D}_{l-1} 
\left(\mathbf{x},\hat{\mathbf{x}}\right) +w_l D_l 
\left(\mathbf{x},\hat{\mathbf{x}}\right)\,,\\ l=1,\ldots , L\,,
  \label{e12-lan}
  \end{multline}
при начальном условии $\tilde{D}_0 (\mathbf{x},\hat{\mathbf{x}})\hm=0$.

  Вычислительная сложность алгоритма опре\-де\-ляется числом элементарных 
операций, затрачиваемых на вычисление меры на всех уровнях\linebreak  
сети~(\ref{e10-lan}), и~на сортировку значений меры на последовательных 
уровнях для отбора ближайших изоб\-ра\-же\-ний согласно~(\ref{e11-lan}), включая 
отбор решения на последнем уровне. В~работе~\cite{7-lan} дана 
асимптотическая оценка вычислительной сложности сформулированного 
иерархического алгоритма поиска при больших значениях мощности~$n$ 
набора изображений~$\hat{\mathbf{X}}$, размере изображения~$N$, 
удовлетворяющем условию $N^2\hm\geq \log_q n$ ($q$~--- размер алфавита), 
и~соотношении $n/n^*\hm\geq N^2/4$, обеспечивающем коэффициент сужения 
зоны поиска $\alpha\hm\geq 1$. Асимптотика сложности иерархического 
алгоритма с~указанными параметрами имеет вид $C_{n^*\leq 4n/N^2}\hm= 
O(n\log N)$, вычислительная сложность переборного алгоритма~--- 
$C_{n^*=n}\hm= \Omega(nN^2)$. Из приведенных оценок следует, что доля 
сложности иерархического поиска приближенного решения относительно 
сложности переборного поиска ближайшего соседа убывает с~увеличением 
размера изображения как $O(N^{-2}\log N)$. 
  
  Численные оценки сложности алгоритма получены для суммарного 
количества элементарных операций, затрачиваемых на вычисление меры 
и~сортировку вставками значений меры со сложностью~$m\log_2 m$ на наборе 
из~$m$~элементов~\cite{14-lan}. Поэтому при фиксированных~$n$ 
и~$N\hm=2^L$ вычислительная сложность алгоритма равна 
  \begin{equation}
  C_{n^*}= C_{n^*}^{\mathrm{msr}}+C_{n^*}^{\mathrm{srt}}\,,
  \label{e13-lan}
  \end{equation}
где 
\begin{equation}
C_{n^*}^{\mathrm{msr}} =\sum\limits_{l=1}^L n_l\cdot 4^l \leq n
\sum\limits^L_{l=1} 4^l \left( 
\fr{n}{n^*}\right)^{-(l-1)/(L-1)}\,;\label{e14-lan}
\end{equation}

\vspace*{-12pt}

\noindent
\begin{multline}
C_n^{\mathrm{srt}} =(n-1)\left[ n^*=n\right] +{}\\
{}+\left( 
\vphantom{\sum\limits_{l=1}^{L-1}}
\left(
n^*-1\right) +
\sum\limits_{l=1}^{L-1} n_l\log_2n_l\right)\left[ n^*<n\right]\leq{}\\
{}\leq (n-1) \left[ n^*=n\right] +n\log_2 n\left( 
\sum\limits_{l=1}^{L-1}\!\left(\fr{n}{n^*}\!\right)^{-(l-1)/(L-1)}\!\! - {}\right.\\
\left.{}-
\sum\limits_{l=1}^{L-1} \fr{l-1}{L-1}\left( \fr{n}{n^*}\right)^{-(l-1)/(L-1)}\right)
\left[ n^*<n\right]+{}\\
{}+\left(n^*-1\right) \left(
\vphantom{\sum\limits_{l=1}^{L-1}}
 a+{}\right.\\
\hspace*{-3mm}\left.{}+n\log_2 e\sum\limits_{l=1}^{L-1}\! \fr{l-1}{L-1}\! \left( 
\fr{n}{n^*}\!\right)^{-(l-1)/(L-1)}\right) \left[ n^*<n\right]\!\!\!
\label{e15-lan}
\end{multline}
соответственно затраты на вычисление меры и~на сортировку; $[f]$~--- 
индикатор~$f$. В~случае $n^*\hm=n$ формулы~(\ref{e13-lan})--(\ref{e15-lan}) 
дают оценки вычислительной сложности переборного поиска точного решения, 
а~в~случае $n^*<n$~--- оценки сложности иерархического поиска 
приближенного решения. 

  Формулы~(\ref{e12-lan})--(\ref{e14-lan}) использованы для получения 
численных оценок относительной сложности $C_{n^*\leq n}/C_{n^*=n}$ при 
значениях $\lfloor n^*\hm= n/2^k \rceil$, $k\hm=0,1,\ldots$,  и~параметрах~$n$ 
и~$N\hm=2^L$, с~которыми проведены эксперименты по поиску рукописных 
цифр (см.\ разд.~4) и~по координатной привязке изоб\-ра\-же\-ний к~карте местности 
(см.\ разд.~5). Величина, обратная относительной сложности, соответствует 
вычислительному выигрышу иерархического алгоритма по сравнению 
с~алгоритмом перебора.
  
  \section{Оценки качества и~сложности поиска рукописных цифр}
  
  Экспериментальные характеристики качества алгоритма поиска получены на 
наборе полутоновых изображений рукописных цифр из базы данных 
MNIST~\cite{12-lan}. Вычислительный эксперимент выполнен с~по\-мощью 
кода, написанного на языке MATLAB~\cite{15-lan} . Примеры изображений 
рукописных цифр даны на рис.~2. 

{ \begin{center}  %fig2
 \vspace*{-3pt}
 \mbox{%
\epsfxsize=75mm
\epsfbox{lan-2.eps}
}


\end{center}

\vspace*{-6pt}


\noindent
{{\figurename~2}\ \ \small{Примеры рукописных цифр из базы данных MNIST}}

}

\vspace*{9pt}

\addtocounter{figure}{1}

  \begin{table*}\small
  \begin{center}
  \parbox{260pt}{\Caption{Оценки относительной сложности иерархического алгоритма поиска 
изображений рукописных цифр ($N \hm= 32$, $q \hm= 256$, $n \hm= 60\,000$)}

}

  \vspace*{2ex}
  
\tabcolsep=9pt
  \begin{tabular}{|c|c|c|c|}
  \hline
  &&&\\[-9pt]
$k$& $C^{\mathrm{msr}}_{n^*\leq  n}/C_{n^*=n}$ &
$C^{\mathrm{srt}}_{n^*\leq n}/C_{n^*=n}$ & $C_{n^*\leq  n}/C_{n^*=n}$\\
\ &\ &\ & \ \\[-9pt]
\hline
0& 0,9993& 0,0007& 1,0000\\
1& 0,5325& 0,0356& 0,5681\\
2& 0,2885& 0,0285& 0,3170\\
3& 0,1597& 0,0237& 0,1834\\
4&0,0908& 0,0205& 0,1113\\
5& 0,0535& 0,0182& 0,0717\\
6& 0,0329& 0,0166& 0,0496\\
7&0,0214& 0,0154& 0,0368\\
8& 0,0146& 0,0146& 0,0292\\
9& 0,0107& 0,0139& 0,0246\\
10\hphantom{9} & 0,0082& 0,0134& 0,0216\\
\hline
\end{tabular}
\end{center}
\vspace*{5pt}
\end{table*}
  
  
  
  Цифры на изображениях нормированы по размеру и~центрированы в~поле 
изображения. Параметры изображений: $N\hm=32$; $q\hm=256$; число 
уровней пред\-став\-ле\-ния изоб\-ра\-же\-ний $L\hm=\log_2 N \hm=5$; мощность набора 
данных $\hat{\mathbf{X}}$ равна $n\hm=60\,000$. Параметры~$N$, $q$ и~$n$ 
удовлетворяют необходимому условию $N^2\hm> \log_q n$. 
Набору~$\hat{\mathbf{X}}$ предъявлялось~10\,000 изображений, не входящих 
в~набор данных~$\hat{\mathbf{X}}$, так что $\tilde{D}_L(\mathbf{x}, 
\hat{\mathbf{x}}^*_{\mathrm{AN}})\hm\geq \tilde{D}_L (\mathbf{x}, 
\hat{\mathbf{x}}^*_{\mathrm{NN}})\hm>0$. Для каждого~$\mathbf{x}$ 
и~соответствующей пары $\hat{\mathbf{x}}^*_{\mathrm{AN}}, 
\hat{\mathbf{x}}^*_{\mathrm{NN}}$ вы\-чис\-ля\-лась погрешность поиска вида~(6) 
и~на~10\,000 предъяв\-ля\-емых изображений строилось семейство эмпирических 
распределений вида~(8) с~па\-ра\-мет\-ром $\lfloor n^*\hm= n/2^k \rceil$, 
$k\hm=0,5,\ldots ,10$, 
при значениях допустимой погрешности $0\hm\leq\varepsilon\hm\leq 0{,}1$. 
Графики семейства распределений даны на рис.~3.

\setcounter{figure}{3}
\begin{figure*}[b] %fig4
\vspace*{6pt}
\begin{center}
\mbox{%
\epsfxsize=160mm
\epsfbox{lan-4.eps}
}
\end{center}
\vspace*{-6pt}
\Caption{Аэрокосмический снимок участка земной поверхности: (\textit{а})~без шума; 
(\textit{б})~с гауссовым шумом с~нулевым средним и~дисперсией $\sigma^2\hm=0{,}25$}
\end{figure*}


{ \begin{center}  %fig3
 \vspace*{-3pt}
 \mbox{%
\epsfxsize=77.97mm
\epsfbox{lan-3.eps}
}


\end{center}

\vspace*{-6pt}


\noindent
{{\figurename~3}\ \ \small{Эмпирические распределения 
  погрешностей поиска рукописных цифр на наборе
   изображений с~па\-ра\-мет\-ра\-ми $N \hm= 32$, $q \hm= 256$, $n \hm= 60\,000$,
$k\hm=5$, 6, 7, 8, 9,~10}}

}

%\vspace*{9pt}

\addtocounter{figure}{1}
 
  
 
  Численные оценки сложности алгоритма поиска представлены значениями 
$C_{n^*\leq n}/C_{n^*=n}$, вы\-чис\-лен\-ны\-ми в~точках $\lfloor n^*\hm= n/2^k \rceil$, 
$k=0,1,\ldots ,10$, при $N\hm=32$, $n\hm=60\,000$. Полученные оценки 
сложности алгоритма поиска даны в~табл.~1.
  

  
  Из приведенных распределений и~оценок сложности следует, что при 
значениях $8\hm\leq k\hm\leq 10$ иерархический алгоритм реализует 
нулевую погрешность поиска ближайшего соседа ($\varepsilon^s_{n^*}\hm=0$) 
с~вероятностью~0,980--0,997 и~обеспечивает вы\-чис\-ли\-тель\-ный выигрыш  
в~34--46 раз по сравнению с~переборным алгоритмом. 
С~ростом допустимой 
погрешности $\varepsilon\hm>0$ вероятность нахождения ближайшего соседа 
с~точ\-ностью $\varepsilon^s_{n^*}\hm\leq\varepsilon$ увеличивается при тех же 
значениях вычислительного вы\-иг-\linebreak рыша. 

\columnbreak

{ %\looseness=1

}
  
  \section{Оценки качества и~сложности координатной привязки 
изображений}

\vspace*{-12pt}
  
  Экспериментальные оценки эффективности алгоритма поиска для 
координатной привязки к~карте местности получены на цифровом 
аэрокосмическом снимке участка земной поверхности~\cite{13-lan}. Обработка 
данных выполнена программным кодом~\cite{15-lan}.
  
  

  
  Размеры аэрокосмического снимка $300\times 236$ (в~пикселях), число 
уровней яркости $q\hm=256$. Использованный снимок и~его зашумленная 
версия даны на рис.~4. 



\pagebreak

\end{multicols}

\setcounter{figure}{4}
  
  \begin{figure*} %fig5
  \vspace*{1pt}
\begin{center}
\mbox{%
\epsfxsize=156.651mm
\epsfbox{lan-5.eps}
}
\end{center}
\vspace*{-8pt}
  \Caption{Эмпирические распределения погрешностей координатной привязки при 
различных дисперсиях гауссова шума
((\textit{а})~0,1; (\textit{б})~0,15;
(\textit{в})~0,2; (\textit{г})~0,25) и~параметрах $N \hm= 64$, $q \hm= 256$, $n \hm= 
10\,353$, $k\hm=0$, 3, 6, 9}
\vspace*{9pt}
\end{figure*}



\begin{multicols}{2}


Привязка в~координатах снимка
 проводилась по 
изображению размера $64\times 64$ ($N\hm=64$). Мощность набора 
данных~$\hat{\mathbf{X}}$ определялась количеством всевозможных 
изображений указанного размера, взятых на снимке с~шагом в~два пикселя, 
и~равна $n\hm=10\,353$. Параметры~$N$, $q$ и~$n$ удовлетворяют необходимому 
условию $N^2\hm>\log_q n$. 

Набору~$\hat{\mathbf{X}}$ предъявлялись 
поочередно все~$n$~изоб\-ра\-же\-ний с~аддитивным гауссовым шумом 
с~дис\-пер\-си\-ей $0\hm\leq \sigma^2\hm\leq 0{,}25$. Для любого предъяв\-ля\-емо\-го 
изображения $\mathbf{x}\hm\in \hat{\mathbf{X}}$ с~координатами $(a,b)$ 
вычис-\linebreak лялись координаты $(a^*,b^*)$ найденного алгоритмом 
изображения~$\hat{\mathbf{x}}^*_{\mathrm{AN}}$ и~погрешность привязки 
вида~(\ref{e7-lan}). При различных значениях~$\sigma^2$ строились семейства 
эмпирических распределений~(\ref{e9-lan}) с~параметром~$n^*$ 
в~диапазоне значений погрешности $0\hm\leq \varepsilon\hm\leq 2{,}5$. 
{\looseness=1

}

На 
рис.~5 даны семейства распределений, полученные для четырех значений 
дисперсии шума при $\lfloor n^*\hm= n/2^k\rceil$, $k\hm=0,3,6,9$. 

\begin{table*}\small %tabl2
\begin{center}
\parbox{260pt}{\Caption{Оценки относительной сложности иерархического алгоритма привязки 
изображений к~карте местности ($N = 64$, $q \hm= 256$, $n = 10\,353$)}

}

\vspace*{2ex}

\tabcolsep=9pt
\begin{tabular}{|c|c|c|c|}
\hline
&&&\\[-9pt]
 $k$& $C^{\mathrm{msr}}_{n^*\leq n}/C_{n^*=n}$ &
  $C^{\mathrm{srt}}_{n^*\leq  n}/C_{n^*=n}$&  $C_{n^*\leq n}/C_{n^*=n}$\\
  \ & \ &\ &\ \\[-9pt]
  \hline
 0&0,9998& 0,0002& 1,0000\\
 3&0,1504& 0,0059& 0,1563\\
 4& 0,0824& 0,0050& 0,0874\\
 5&0,0462& 0,0044& 0,0506\\
 6&0,0265& 0,0039& 0,0304\\
 7& 0,0158& 0,0036& 0,0194\\
 8& 0,0098& 0,0034& 0,0132\\
 9&0,0063& 0,0032& 0,0095\\
 10\hphantom{9}& 0,0028&0,0034& 0,0062\\
  \hline
  \end{tabular}
 \end{center}
 \vspace*{6pt}
 \end{table*}

В~табл.~2 приведены 
численные оценки относительной сложности $C_{n^*\leq n}/C_{n^*=n}$ при
значениях $k\hm=0, 3, \ldots , 10$.
  
  Экспериментально установлено, что при отсутствии шума ($\sigma^2\hm=0$) 
иерархический алгоритм\linebreak с~параметром $n^*\hm\geq 1$ с~вероятностью единица 
дает нуле\-вую погрешность привязки $\varepsilon^g_{n^*}\hm=0$ 
и,~следовательно, по качеству эквивалентен алгоритму перебора. В~случае 
шума с~дисперсией $\sigma^2\hm\leq 0{,}10$ погрешность $\varepsilon^g_{n^*} 
\hm=0$ реализуется практически\linebreak с~единичной вероятностью при более 
чем~\mbox{100-крат}\-ном вы\-чис\-ли\-тель\-ном выигрыше иерархического алгоритма 
относительно переборного ал\-го\-ритма. 
{ %\looseness=1

}


Семейства распределений, полученные 
при значениях $\sigma^2\hm=0{,}15$, 0{,}20 и~0{,}25, демонстрируют динамику 
соотношения показателей качества привязки и~быст\-ро\-дей\-ст\-вия иерархического 
алгоритма. При любой фиксированной дисперсии шума~$\sigma^2$ и~заданной 
допустимой погрешности привязки~$\varepsilon$ вероятность реализации 
точности привязки $\varepsilon^g_{n^*}\hm\leq\varepsilon$ уменьшается 
с~ростом вычислительного выигрыша алгоритма. При 
фиксированных значениях~$\varepsilon$ и~$n^*$ вероятность реализации %\linebreak 
точ\-ности $\varepsilon^g_{n^*}\hm\leq \varepsilon$ уменьшается с~увеличением 
дисперсии шума. 

Необходимо отметить, что пред\-став\-ле\-ние циф\-ро\-вой карты 
местности набором всевозможных изображений, взятых с~шагом в~один %\linebreak 
пиксель, должно привести к~увеличению ве\-ро\-ят\-ности тре\-бу\-емой точ\-ности при 
заданных значениях~$\varepsilon$, $n^*$ и~$\sigma^2$.
{ %\looseness=1

} 




  
  \section{Заключение }
  
  Исследована эффективность иерархического алгоритма поиска в~заданном 
наборе изображений приближенного ближайшего соседа к~заданному 
изображению в~терминах эмпирического распределения значений погрешности и~вычислительной сложности алгоритма. Рассматриваемый алгоритм 
предназначен для реализации поиска с~заданным быстродействием 
и~надежностью выполнения требования по точности. 

Эффективность 
алгоритма продемонстрирована на двух источниках изображений: на наборе 
изображений рукописных цифр из базы MNIST и~на наборе пересекающихся 
изображений аэрокосмической карты местности, взятой из  
ин\-тер\-нет-сер\-ви\-са Google Maps. При фиксированной вероят\-ности 
реализации требуемой точности показана возможность <<размена>> точности 
и~вы\-чис\-ли\-тельной слож\-ности путем варьирования па\-ра\-мет\-ром алгоритма. 
{\looseness=1

}

Полученные экспериментальные результаты показали достаточно высокую 
надежность требуемой точности поиска в~наборе изображений рукописных 
цифр и~требуемой точности координатной привязки зашумленного 
изоб\-ра\-же\-ния к~цифровой карте местности. Для уменьшения порядка роста 
вычислительной сложности поиска от мощности набора изображений 
планируется исследовать модификацию иерархического алгоритма 
с~использованием структуры, которая объединяет представление 
с~многоуровневым разрешением и~решающее дерево.

 {\small\frenchspacing
 {%\baselineskip=10.8pt
 \addcontentsline{toc}{section}{References}
 \begin{thebibliography}{99}
\bibitem{1-lan}
\Au{Datta R., Joshi D., Li~J., Wang~J.} Image retrieval: Ideas, influences, and trends of the new 
age~// ACM Comput. Surv., 2008. Vol.~40. No.\,2. P.~1--60.
\bibitem{2-lan}
\Au{Friedman J., Bentley~J., Finkel~R.} An algorithm for finding best matches in logarithmic 
expected time~// ACM T. Math. Software, 1977. Vol.~3. No.\,3. P.~209--226. 
\bibitem{3-lan}
\Au{Cleary J.} Analysis of an algorithm for finding nearest neighbors in Euclidean space~// ACM 
T. Math. Software, 1979. Vol.~5. No.\,2. P.~183--192. 
\bibitem{4-lan}
\Au{Soleymani M., Morgera~S.} An efficient nearest neighbor search method~// IEEE T. 
Commun., 1987. Vol.~35. No.\,6. P.~677--679. 
\bibitem{5-lan}
\Au{Arya S., Mount~D., Netanyahu~N., Silverman~R., Wu~A.} An optimal algorithm for 
approximate nearest neighbor searching in fixed dimensions~// J.~ACM, 1998. Vol.\,45. No.\,6. 
P.~891--923.
\bibitem{6-lan}
\Au{Andoni А., Indyk~P.} Near-optimal hashing algorithms for approximate nearest neighbor in 
high dimensions~// Commun.  ACM, 2008. Vol.~51. No.\,1. P.~117--122.
\bibitem{7-lan}
\Au{Lange M.\,M., Ganebnykh~S.\,N., Lange~A.\,M.} Algorithm of approximate search for the 
nearest digital array in a~hierarchical data set~// Machine Learning Data Analysis, 2016. Vol.~2. 
No.\,1. P.~6--16. 
\bibitem{8-lan}
\Au{Rosenfeld A.} Quadtrees and pyramids for pattern recognition and image analysis~// 5th  
Conference (International) on Pattern Recognition Proceedings, 1980. P.~802--811.
\bibitem{9-lan}
\Au{Jackins C., Tanimoto~S.} Quadtrees, octtrees, and K-trees: A~generalized approach to 
recursive decomposition of Euclidean space~// IEEE T. Pattern Anal., 1983. Vol.~5. No.\,5.  
P.~533--539.
\bibitem{10-lan}
\Au{Samet H.} The quadtree and related hierarchical data structures~// Comput. Surv., 1984. 
Vol.~16. No.\,2. P.~187--260.
\bibitem{11-lan}
\Au{Ланге М.\,М., Новиков~Н.\,А.} Представление данных с~многоуровневым разрешением 
для быстрой координатной привязки изображений~// Техническое зрение в~системах 
управления: Сб. тр. науч.-технич. конф.~---  М.: ИКИ РАН, 2012. С.~242--249.

\pagebreak

\bibitem{12-lan}
MNIST database. {\sf http://yann.lecun.com/exdb/mnist/\linebreak index.html}. 
\bibitem{13-lan}
Network service Google Maps. {\sf http://www.maps.google. com}.
\bibitem{14-lan}
\Au{Cormen T., Leiserson~C., Rivest~R., Stein~C.} Introduction to algorithms.~--- 3rd ed.~--- MIT 
Press, 2009. 1312~p.
\bibitem{15-lan}
Algorithm for searching approximate nearest neighbor. {\sf 
http://sourceforge.net/projects/edivis/files/}.

 \end{thebibliography}

 }
 }

\end{multicols}

\vspace*{-3pt}

\hfill{\small\textit{Поступила в~редакцию 13.12.16}}

\vspace*{8pt}

%\newpage

%\vspace*{-24pt}

\hrule

\vspace*{2pt}

\hrule

%\vspace*{8pt}


\def\tit{ON EFFICIENCY OF~THE~HIERARCHICAL ALGORITHM FOR~SEARCHING APPROXIMATE 
NEAREST NEIGHBOR\\  IN~A~GIVEN SET OF~IMAGES}

\def\titkol{On efficiency of~the~hierarchical algorithm for~searching approximate 
nearest neighbor  in~a~given set of~images}

\def\aut{M.\,M.~Lange, S.\,N.~Ganebnykh, and~A.\,M.~Lange}

\def\autkol{M.\,M.~Lange, S.\,N.~Ganebnykh, and~A.\,M.~Lange}

\titel{\tit}{\aut}{\autkol}{\titkol}

\vspace*{-9pt}


\noindent
Federal Research Center ``Computer Science and Control'' of the Russian Academy of Sciences, 
44-2~Vavilov Str., Moscow 119333, Russian Federation



\def\leftfootline{\small{\textbf{\thepage}
\hfill INFORMATIKA I EE PRIMENENIYA~--- INFORMATICS AND
APPLICATIONS\ \ \ 2017\ \ \ volume~11\ \ \ issue\ 3}
}%
 \def\rightfootline{\small{INFORMATIKA I EE PRIMENENIYA~---
INFORMATICS AND APPLICATIONS\ \ \ 2017\ \ \ volume~11\ \ \ issue\ 3
\hfill \textbf{\thepage}}}

\vspace*{3pt}
  
    
  
  \Abste{The efficiency of the hierarchical algorithm for searching
  approximate nearest neighbor 
in a given set of images subject to an unwarranted error about the nearest image is 
investigated. The algorithm uses a~space of quad pyramidal image representations as well as 
a~guided search strategy in successive representation levels of increasing resolution. The efficiency 
is studied in terms of both an empirical distribution of search errors and computational complexity 
of the hierarchical algorithm relative to the exhaustive search. The above characteristics are 
obtained for two applications, namely, search for approximate nearest image in a~set of hand-written 
digits from the MNIST data base and gridding a given noisy image in an aerospace digital map 
from the Google maps network service.}
  
  \KWE{image; quad pyramidal representation; digital map; nearest neighbor; approximate 
nearest neighbor; search error; empirical distribution; computational complexity}

\DOI{10.14357/19922264170306}
%\vspace*{-18pt}

  \Ack
  \noindent
  The research was supported by the Russian Foundation for Basic Research (projects  
15-07-07516 and~15-07-09324).



%\vspace*{3pt}

  \begin{multicols}{2}

\renewcommand{\bibname}{\protect\rmfamily References}
%\renewcommand{\bibname}{\large\protect\rm References}

{\small\frenchspacing
 {%\baselineskip=10.8pt
 \addcontentsline{toc}{section}{References}
 \begin{thebibliography}{99}
  
\bibitem{1-lan-1}
\Aue{Datta, R., D.~Joshi, J.~Li, and J.~Wang.} 2008. Image retrieval: Ideas, influences, and trends 
of the new age. \textit{ACM Comput. Surv.} 40(2):1--60.
\bibitem{2-lan-1}
\Aue{Friedman, J., J. Bentley, and R.~Finkel.} 1977. An algorithm for finding best matches in 
logarithmic expected time. \textit{ACM T. Math. Software} 3(3):209--226.
\bibitem{3-lan-1}
\Aue{Cleary, J.} Analysis of an algorithm for finding nearest neighbors in Euclidean space. 1979. 
\textit{ACM T. Math. Software} 5(2):183--192.
\bibitem{4-lan-1}
\Aue{Soleymani, M., and S.~Morgera.} 1987. An efficient nearest neighbor search method. 
\textit{IEEE T. Commun.} 35(6):677--679.
\bibitem{5-lan-1}
\Aue{Arya, S., D. Mount, N.~Netanyahu, R.~Silverman, and A.~Wu.} 1998. An optimal algorithm 
for approximate nearest neighbor searching in fixed dimensions. \textit{J.~ACM} 45(6):891--923.
\bibitem{6-lan-1}
\Aue{Andoni, А., and P.~Indyk.} 2008. Near-optimal hashing algorithms for approximate nearest 
neighbor in high dimensions. \textit{Commun. ACM} 51(1):117--122.
\bibitem{7-lan-1}
\Aue{Lange, M.\,M., S.\,N.~Ga\-neb\-nykh, and A.\,M.~Lange.} 2016. Algorithm of approximate 
search for the nearest digital array in a hierarchical data set. \textit{Machine Learning Data 
Analysis} 2(1):6--16. 
\bibitem{8-lan-1}
\Aue{Rosenfeld, A.} 1980. Quadtrees and pyramids for pattern recognition and image analysis. 
\textit{5th Conference (International) on Pattern Recognition Proceedings}. 802--811.
\bibitem{9-lan-1}
\Aue{Jackins, C., and S.~Tanimoto.} 1983. Quadtrees, octtrees, and K-trees: A~generalized 
approach to recursive decomposition of Euclidean space. \textit{IEEE T. Pattern
Anal.} 5(5):533--539.
\bibitem{10-lan-1}
\Aue{Samet, H.} 1984. The quadtree and related hierarchical data structures. 
\textit{Comput. Surv.} 16(2):187--260.
\bibitem{11-lan-1}
\Aue{Lange, M.\,M., and N.\,A.~Novikov.} 2012. Predstavlenie dannykh s mnogourovnevym 
razresheniem dlya bystroy koordinatnoy privyazki izobrazheniy [Multiresolution data 
representation for fast image gridding]. \textit{Sb. tr. nauch.-tekhnich. konf. ``Tekhnicheskoe zrenie 
v~sistemakh upravleniya} [Scientific-Technical Conference ``Computing\linebreak\vspace*{-12pt}

\pagebreak

\noindent
 Vision in Control 
Systems'' Proceedings]. Moscow: IKI RAN. 242--249.
\bibitem{12-lan-1}
MNIST database. Available at: {\sf http://yann.lecun.com/ exdb/mnist/index.html} (accessed 
January~10, 2017). 
\bibitem{13-lan-1}
Network service Google Maps. Available at: {\sf http:// www.maps.google.com} (accessed 
January~10, 2017). 
\bibitem{14-lan-1}
\Aue{Cormen, T., C.~Leiserson, R.~Rivest, and C.~Stein.} 2009. \textit{Introduction to 
algorithms}. 3rd ed. MIT Press. 1312~p.
\bibitem{15-lan-1}
Algorithm for searching approximate nearest neighbor. Available at: {\sf 
http://sourceforge.net/projects/edivis/ files/} (accessed January~10, 2017).
\end{thebibliography}

 }
 }

\end{multicols}

\vspace*{-3pt}

\hfill{\small\textit{Received December 13, 2016}}
\Contr

\noindent
\textbf{Lange Mikhail M.} (b.\ 1945)~--- Candidate of Science (PhD) in technology, leading 
scientist, Federal Research Center ``Computer Sciences and Control'' of the Russian Academy of 
Sciences, 44-2~Vavilov Str., Moscow 119333, Russian Federation; \mbox{lange\_mm@ccas.ru} 

\vspace*{3pt}
  
\noindent
\textbf{Ganebnykh Sergey N.} (b.\ 1968)~--- scientist, Federal Research Center ``Computer 
Sciences and Control'' of the Russian Academy of Sciences, 44-2~Vavilov Str., Moscow 119333, 
Russian Federation; \mbox{sng@ccas.ru}

\vspace*{3pt}

\noindent
\textbf{Lange Andrey M.} (b. 1979) - Candidate of Science (PhD) in physics and mathematics, 
scientist, Federal Research Center ``Computer Sciences and Control'' of the Russian Academy of 
Sciences, 44-2~Vavilov Str., Moscow 119333, Russian Federation; \mbox{lange\_am@mail.ru}
\label{end\stat}


\renewcommand{\bibname}{\protect\rm Литература} 