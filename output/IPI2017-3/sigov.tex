\def\stat{sigov}

\def\tit{ПСИХОЛИНГВИСТИЧЕСКИЙ АНАЛИЗ\\ РУССКОЯЗЫЧНЫХ ТЕКСТОВЫХ СООБЩЕНИЙ\\ 
НА~ОСНОВЕ ИХ ФОНОСЕМАНТИЧЕСКИХ\\ СТАТИСТИЧЕСКИХ ХАРАКТЕРИСТИК$^*$}

\def\titkol{Психолингвистический анализ русскоязычных текстовых сообщений 
на~основе их фоносемантических %статистических 
характеристик}

\def\aut{А.\,С.~Сигов$^1$, Д.\,А.~Акимов$^2$, Д.\,О.~Жуков$^3$, 
Е.\,Г.~Андрианова$^4$, В.\,Е.~Сачков$^5$, В.\,К.~Раев$^6$}

\def\autkol{А.\,С.~Сигов, Д.\,А.~Акимов, Д.\,О.~Жуков и~др.} 
%Е.\,Г.~Андрианова$^4$, В.\,Е.~Сачков$^5$, В.\,К.~Раев$^6$}

\titel{\tit}{\aut}{\autkol}{\titkol}

\index{Сигов А.\,С.}
\index{Акимов Д.\,А.}
\index{Жуков Д.\,О.}
\index{Андрианова Е.\,Г.}
\index{Сачков В.\,Е.}
\index{Раев В.\,К.}
\index{Sigov A.\,S.}
\index{Akimov D.\,A.}
\index{Zhukov D.\,O.}
\index{Andrianova E.\,G.} 
\index{Sachkov V.\,E.}
\index{Raev V.\,K.}


{\renewcommand{\thefootnote}{\fnsymbol{footnote}} \footnotetext[1]
{Работа выполнена за счет финансирования Министерством образования и~науки Российской Федерации 
конкурсной части государственных заданий высшим учебным заведениям и~научным организациям по 
выполнению инициативных научных проектов (№\,28.2635.2017/ПЧ).}}


\renewcommand{\thefootnote}{\arabic{footnote}}
\footnotetext[1]{Московский технологический университет (МИРЭА), \mbox{assigov@yandex.ru}}
\footnotetext[2]{Московский технологический университет (МИРЭА), \mbox{akimov\_d@mirea.ru}}
\footnotetext[3]{Московский технологический университет (МИРЭА), \mbox{zhukovdm@yandex.ru}}
\footnotetext[4]{Московский технологический университет (МИРЭА), \mbox{dtghmflysq@gmail.com}}
\footnotetext[5]{Московский технологический университет (МИРЭА), \mbox{megawatto@mail.ru}}
\footnotetext[6]{Московский технологический университет (МИРЭА), \mbox{raev@mirea.ru}}
  
  %\vspace*{-18pt}

\Abst{Рассматривается проблема идентификации типа акцентуации паттерна 
поведения виртуального субъекта в~сети Интернет и~социальных сетях на основе 
статистического анализа текстов, что позволяет сформулировать гипотезу о~структурных 
свойствах его коммуникаций и~позволяет построить матрицу вероятностей для отношений 
между виртуальными масками субъектов. Тексты пользователей 
рассматриваются как сложные се\-ман\-ти\-ко-син\-так\-си\-че\-ские образования, 
обладающие рядом психолингвистических характеристик. К~их числу относятся цельность, 
а~также смысловая направленность сообщения. Кроме того, в~тексте, рассматриваемом как 
продукт речевой деятельности, обладающий большой степенью семантической 
вариативности, определяемой его темпоральными и~сонарными характеристиками, 
проявляется невербальный характер поведения сетевых субъектов~--- виртуальных масок 
и~роботизированных агентов. Практическая значимость предлагаемого решения для 
психолингвистического анализа строится на возрастающем значении развития системы 
условных знаков, в~данном случае условных языков е-ком\-му\-ни\-ка\-ции, для порождения, 
в~свою очередь, управляющих кластеров, регулирующих социальное поведение 
виртуальных субъектов в~Сети. Это предположение строится на гипотезе Кеннета Айверса, 
в~соответствии с~которой чем лучше развита система условных знаков, тем больше 
возможностей она дает для создания новых алгоритмов.}

\KW{психолингвистические характеристики; невербальное поведение; виртуальные маски; 
процесс мышления; семантический смысл; лингвистический релятивизм}

\DOI{10.14357/19922264170309} 


\vskip 10pt plus 9pt minus 6pt

\thispagestyle{headings}

\begin{multicols}{2}

\label{st\stat}


\section{Введение}

  В настоящее время большой интерес разработчиков информационных сетей 
вызывает анализ социального аспекта информационного массива (потока) 
популярных ин\-тер\-нет-ре\-сур\-сов, воздействие которых изменяет 
семантический смысл и~оказывает управляющее воздействие на виртуальных 
субъектов, равно как и~на их кластеры непосредственного взаимодействия. 
К~такой информации относятся данные, отражающие мнения, тенденции, 
настроения и~интересы, преобладающие среди субъектов е-со\-об\-ще\-ства.
  
  На взгляд авторов, решение таких задач возможно только за счет 
использования междисциплинарных подходов, в~которых методы 
теоретической информатики должны быть дополнены моделями 
математической лингвистики естественных языков.
  
  Теоретическим обоснованием рассматриваемой проблемы является гипотеза 
лингвистической относительности, которая предполагает, что структура языка 
коммуникации влияет на ментальность пользователей социальных сетей~[1] 
и~опосредованно на когнитивные процессы мышления последних. 

Воспользуемся нечеткой трактовкой гипотезы Се\-пи\-ра--Уор\-фа~[2], 
в~соответствии с~которой процессы мышления, а также используемые 
в~письменной/устной речи лингвистические категории определяются при 
е-ком\-му\-ни\-ка\-ции как некая форма неязыкового поведения.
  
  Используемый принцип Уорфа, равно как и~позднее сформулированная 
гипотеза Р.~Брауна и~Э.~Леннеберга~[3] в~отношении цветового восприятия, 
определяющая разницу в~восприятии цветового зрения в~различных языках, 
носит релятивистский характер, равно как и~конструктивистский подход, 
предполагающий, что свойства проявления черт человеческой психики и~общие 
идеи самопроявления в~коммуникации в~значительной степени подвержены 
влиянию категорий, сформированных субъектами в~процессе социализации, 
и~не зависят от биологических ограничений.
  
  В~антологии~[4] 
исследователи лингвистического релятивизма сделали попытку определить 
связи и~границы между мышлением, познанием, языком и~культурой, описать 
степень и~виды взаимосвязанности и~взаимовлияния. Слобин~[5] 
задавал когнитивный процесс <<мышление для речи>> как вид процесса, 
в~котором перцептивные данные и~другие виды долингвистического мышления 
переводятся в~лингвистические категории для коммуникации с~другими 
субъ\-ек\-тами. 
  
  Джон Люси выделил основные направления исследований лингвистического 
релятивизма, и~в том числе %{\sf https://ru.wikipedia.org/wiki/\%D0 \%93\%D0\%B8\%D0\%BF\%D0\%BE\%D1\%82\%D0\%B5\%D0\%B7\%D0\%B0\_\%D0\%BB\%D0\%B8\%D0\%BD\%D0\%B3\%D0\%B2\%D0\%B8\%D1\%81\%D1\%82\%D0\%B8\%D1\%87\%D0\%B5\%D1\%81\%D0\%BA\%D0\%BE\%D0\%B9\_\%D0\%BE\%D1\%82\%D0\%BD\%D0\%BE\%D1\%81\%D0\%B8\%D1\%82\%D0\%B5\%D0\%BB\%D1\%8C\%D0\%BD\%D0\%BE\%D1\%81\%D1\%82\%D0\%B8-cite\_note-50} 
<<областной>> 
подход. При этом подходе выбирается отдельная семантическая область 
и~сравнивается у~различных лингвистических и~культурных групп (в~данном 
случае групп пользователей и~отдельных виртуальных субъектов) 
с~\mbox{целью} обнаружения корреляции между лингвистическими 
средствами, которые используются в~языке для обозначения тех или иных 
понятий, и~характером поведения. С~по\-мощью комбинации вышеуказанных 
подходов и~теоретических положений был проведен расчет квалиметрических 
характеристик процесса коммуникации в~сети Интернет на основе анализа 
отношения (матрицы отношений) виртуальных идентичностей пользователей 
сети Интернет к~тем или иным событиям, явлениям и~персонам (субъектам 
социальной значимости) реального мира. Также учитывалась степень 
взаимовлияния виртуальных идентичностей или групп идентичностей.
  
  В рамках проведенного исследования проверялась гипотеза о~том, что 
виртуальная идентичность формируется на основе совокупности отношений 
пользователя к~тем или иным сетевым событиям и~является формой проявления 
отношений пользователей между собой, а~также доступной информации или 
источниками информации, представленными в~Сети.
  
  Для решения задачи идентификации поведения виртуальной идентичности 
моделировался некий процесс, в~котором пользователь, участвуя во всех 
информационных взаимодействиях, условно проявляет свои личностные 
качества посредством из-\linebreak\vspace*{-12pt}

 { \begin{center}  %fig1
 \vspace*{-3pt}
 \mbox{%
\epsfxsize=77.575mm
\epsfbox{sig-1.eps}
}

\end{center}


\noindent
{{\figurename~1}\ \ \small{Методологические конструкты и~инструментарий анализа сетевых событий}}

}

\vspace*{9pt}

\addtocounter{figure}{1}



\noindent
бранного паттерна и~маски поведения. Исходя из\linebreak
 этого, 
строилась модель <<псевдоличности>> (или\linebreak виртуальный образ) 
с~последующей ее идентификацией в~одном или нескольких кластерах сети 
Ин\-тер\-нет. В~исследовании была использована идея\linebreak дополне\-ния 
лингвистического анализа сетевого поведения субъектов статистическим 
анализом (рис.~1), выбраны методы анализа, определена область 
практического использования результатов, обоснована репрезентативность 
разработанной методики анализа акцентуации виртуальных субъектов.
  
  Классические методы изучения е-ком\-му\-ни\-ка\-ции базируются на 
семантическом анализе получаемого от пользователей Сети текстового образа 
взаимодействия или их поведенческого проявления. Ставится задача 
определения статистически релевантных характеристик языковой среды 
коммуникации и~структурных свойств среды. В~дальнейшем при достаточно 
большом с~точки зрения репрезентативности результата числе измерений 
можно строить матрицу вероятностей для оценки свойств коммуникации между 
виртуальными масками субъектов. 
  
  \begin{figure*} %fig2
  \vspace*{1pt}
\begin{center}
\mbox{%
\epsfxsize=120.335mm
\epsfbox{sig-2.eps}
}
\end{center}
\vspace*{-11pt}
\Caption{Проблемный репертуар практических моделей событий Сети}
\end{figure*}
  

  Областью применения предложенной методики анализа семантического 
контента на основе формирования словарей <<окраски текста>> или 
акцентуации е-ком\-му\-ни\-ка\-ции могут стать аналитические BI (Business Intelligence)
за\-про\-сы 
в~экономических исследованиях, выявление характеристик поведения 
субъектов Сети в~социологии, оценка событийных рядов в~политологии 
и~анализ расстройств поведения в~пси\-хи\-атрии.
  
  Особенности e-ком\-му\-ни\-ка\-ций социальных сетей и~блогов затрудняют 
получение однозначного результата при изучении только лингвистической 
составляющей текстов обмена. В~исследовании предлагается дополнить 
существующие подходы, основанные на использовании лингвистического 
анализа текстов, рядом психолингвистических инструментов (акцентуация) 
анализа, учитывая при
 этом их статистическую репрезентативность. 
Репрезентативность предложенной методики достигается высоким уровнем 
диверсификации типов виртуальной коммуникации (рис.~2) и~ее 
разнородностью.
  

  По грубым оценкам, контент за единицу времени в~1~с пополняется 
на~7820~твитов, 1381~графиче-\linebreak\vspace*{-12pt}

 { \begin{center}  %fig3
 \vspace*{6pt}
 \mbox{%
\epsfxsize=61.242mm
\epsfbox{sig-3.eps}
}

\end{center}


\noindent
{{\figurename~3}\ \ \small{Приблизительные оценки персональной е-ин\-фор\-ма\-ции в~сети Интернет: 
\textit{1}~--- общий поток данных; \textit{2}~---  семантически весомые; \textit{3}~--- 
используются часто}}

}

%\vspace*{9pt}

\addtocounter{figure}{1}



\noindent
ское изображение, 1558~аудиозвонков, 
45\,861~запрос Google+; 2\,340\,000 email, включая~67\% спама~[6]. 
В~Интернете циркулирует колоссальный объем персональной информации, как 
правило, текстового содержания в~знаковой, образной и~звуковой форме 
(статистические оценки потоков которой показаны на рис.~3), при этом 
анализу подлежит не просто текст как некий семантический контент, а характер 
восприятия его физическим объектом либо роботизированным устройством.


  
\section{Методика применения психолингвистического анализа  
е-коммуникаций на основе словарей окраски текста}

  Предметом анализа в~предлагаемой методике выявления акцентуации 
виртуальной коммуникации является вербальное и~невербальное речевое 
поведение, сонарные и~темпоральные характеристики речевого поведения,  
лек\-си\-ко-мор\-фо\-ло\-ги\-че\-ский характер проявления виртуального 
субъ\-екта. 

Для расчета показателей психолингвистических свойств текста 
используем фоносемантический анализ~[7], при этом вычислим процентное 
совпадение со словарем окраски текста.
  
  Так как сообщение является входной информацией, представленной в~виде 
набора слов, то можно выявить процентное совпадение данного сообщения со 
словарем~[8], используя формулу:

\noindent
  \begin{equation}
  S_i=\fr{N_i}{N}\,,
  \label{e1-sig}
  \end{equation}
где $S_i$~--- частота появлений некоторой $i$-й словоформы; $N$~--- общее 
число слов или словосочетаний, встреченных в~исследуемом сообщении; $i$~--- 
данная словоформа; $N_i$~--- число вхождений данной словоформы во 
множество всех встреченных слов из словарей.
  
  Примем во внимание, что каждый звук человеческой речи, или фонема, 
обладает определенным подсознательным значением. Для русского языка эти 
значения в~свое время определил советский ученый, доктор филологических 
наук А.\,П.~Жу\-рав\-лев~\cite{9-sig}, который предложил свой вариант 
фоносемантических значений для каждого звука, или фонемы, русской речи 
по~25~шкалам. Всем фонемам русского языка по этим шкалам сопоставлены 
оценки. Для оценки воздействия на человека слова как набора звуков 
необходимо по со\-от\-вет\-ст\-ву\-ющим расчетам определить общее 
фоносемантическое значение составляющих данное слово звуков по шкалам, 
разбитым на~3~группы (рис.~4).

 { \begin{center}  %fig4
 \vspace*{6pt}
\mbox{%
\epsfxsize=77.449mm
\epsfbox{sig-4.eps}
}

\end{center}


\noindent
{{\figurename~4}\ \ \small{Фоносемантическое соответствие звуков по шкалам}}

}

\vspace*{9pt}

\addtocounter{figure}{1} 
  
 
  
  Данные группы составлены с~учетом их час\-тот\-но\-го использования при 
анализе взаимодействия виртуальных идентичностей~\cite{10-sig} 
относительно внешних и~внутренних критериев оценки.
  
  На основе оценки общего эмоционального состояния виртуальной сущности 
(субъекта) по отношению к~порожденному им тексту выделим следующие 
группы: 
  \begin{enumerate}[(1)]
  \item  психолингвистические показатели эмоциональной напряженности; 
  \item вербальные средства выражения эмоционального напряжения; 
  \item вербальные средства выражения мотивационного напряжения.
  \end{enumerate}
  
  В общем случае анализ тональности и~темпоральности относят к~области 
компьютерной лингви\-стики, т.\,е.\ подразумевается, что можно 
классифицировать тональность и~темпоральность, используя стандартные 
инструменты обработки естественного языка по типам организации обработки: 
(1)~подходы, основанные на правилах; (2)~подходы, основанные на словарях; 
(3)~машинное обучение с~учителем; (4)~машинное обучение без учителя. 

В~данной статье приоритет отдан методам, основанным на использовании 
словарей.
  
\section{Методы анализа текстовых сообщений, основанные 
на~правилах и~словарях}

  Этот метод основан на поиске эмотивной лексики (лексической тональности) 
в~тексте по заранее составленным тональным словарям и~правилам 
с~применением лингвистического анализа~\cite{11-sig}. По совокупности 
найденной эмотивной лексики текст может быть оценен по шкале, 
выражающей объем негативной и~позитивной лексики. Данный метод может 
использовать как списки правил, под\-став\-ля\-емые в~регулярные выражения, так 
и~специальные правила соединения тональной лексики внутри предложения. 
Чтобы проанализировать текст, можно воспользоваться следующим 
алгоритмом: сначала каждому слову в~тексте присвоить его значение 
тональности из словаря (если оно присутствует в~словаре), а затем вычислить 
общую тональность всего текста путем суммирования значения тональностей 
каждого отдельного предложения.


  
  Основной проблемой методов, основанных на словарях и~правилах, 
считается трудоемкость процесса составления словаря. Для того чтобы 
получить метод, классифицирующий документ с~высокой точностью, термины 
словаря должны иметь вес, адекватный предметной области документа. 
Например, слово <<огромный>> по отношению к~объему памяти жесткого 
диска является положительной характеристикой, но отрицательной по 
отношению к~размеру мобильного телефона. Поэтому данный метод требует 
значительных трудозатрат, так как для хорошей работы системы необходимо 
составить большое число правил. Чтобы ускорить процесс составления 
словарей и~правил, данный метод\linebreak\vspace*{-12pt}

\pagebreak

\end{multicols}

\begin{figure*} %fig5
\vspace*{1pt}
\begin{center}
\mbox{%
\epsfxsize=129.924mm
\epsfbox{sig-5.eps}
}
\end{center}
\vspace*{-11pt}
\Caption{Алгоритм формирования тематических сло\-ва\-рей-те\-зау\-ру\-сов}
%\vspace*{-6pt}
\end{figure*}

\begin{table*}\small
  \begin{center}
  
  \begin{tabular}{|p{30mm}|p{67mm}|p{53mm}|}
  \multicolumn{3}{c}{Критерии отбора значимых характеристик~\cite{7-sig}, или маски 
акцентуации, виртуального субъекта по типам}\\
  \multicolumn{3}{c}{\ }\\  [-6pt]
  \hline
\multicolumn{1}{|c|}{\tabcolsep=0pt\begin{tabular}{c}Маркер\\ проявления\end{tabular}}&
\multicolumn{1}{c|}{\tabcolsep=0pt\begin{tabular}{c}Характеристика\\ маски 
виртуального субъекта\end{tabular}}&
\multicolumn{1}{c|}{Алгоритм расчета}\\
\hline
<<Светлые>>, 
или паранойяльная акцентуация  поведения &{Тексты мажорной 
окраски, проявление искусственного оптимизма 
(все идет <<хорошо>>,  <<по плану>>),  <<мессианский  комплекс>>}
&
\\
\hline
<<Темные>>, или эпи\-леп\-то\-ид\-ная  акцентуация  поведения &
{Тексты с~большим 
объемом обсуждения насилия, описания  патологической 
жестокости, выраженным  противостоянием <<мы 
и~они>>, <<добро и~зло>> и~т.\, д.} &
\multicolumn{1}{|c|}{\raisebox{-14pt}[0pt][0pt]{$K_{\mathrm{т}}=\fr{\mbox{число\ 
темных\ 
словосочетаний}}{\mbox{число\ 
слов}}$}}\\
\hline
<<Печальные>>,  или депрессивная  акцентуация 
поведения &{Тексты  меланхолического 
настроения, часто  связанные  с~быстротечностью 
жизни, с~тем, что   жизнь~--- страдания  и~только смерть способна 
положить им конец} &
\\
\hline
<<Веселые>>, или  гипертимическая  акцентуация поведения&
{Тексты,  представляющие собой  описание поведения 
человека, который  сталкивается  с~препятствиями или 
опасностями, но успешно  преодолевает их 
и~достигает успеха} &
\\
\hline
<<Сложные>>,  или шизотимная  акцентуация поведения &
{Тексты, наполненные  философскими понятиями, 
абстракциями  и~усложнениями} &\\
\hline
<<Красивые>>,  или истероидная  акцентуация поведения&
{Тексты с~нарочитым описанием эмоциональных 
аффектов, страстей,  страдания и~эротизма} &
\\
\hline
\end{tabular}
\end{center}
\vspace*{-24pt}
\end{table*}

\pagebreak



\begin{multicols}{2}


\noindent
 используется с~привязкой к~конкретной 
предметной области (например, тематика ресторанов или тематика мобильных 
телефонов).




  Критериями отбора значимых характеристик, или масками акцентуации, 
виртуального субъекта по типам (см.\ таблицу): паранойяльной, 
эпилептоидной, депрессивной, гипертимической и~др.~--- служат агрегации 
выделенных квалиметрических показателей~\cite{12-sig}. Упрощенно 
тональный словарь представляет собой список слов со значением тональности 
для каждого слова. Чтобы проанализировать текст, можно воспользоваться 
следующим алгоритмом (рис.~5): сначала каждому слову в~тексте 
присваивается его значение тональности из словаря (если присутствует), 
а~затем вычисляется общая тональность всего текста путем нахождения 
усредненных величин либо путем <<обучения>> классификатора (например, 
нейронной сети).
  
  

  {\small  \textbf{Примечание.} 
 Под \textit{паранойяльной акцентуацией} 
подразумевается повышенная подозрительность и~болезненная обидчивость, 
стойкость отрицательных аффектов, стремление к~доминированию, непринятие 
мнения другого и,~как следствие, высокая конфликтность, подпадание под 
власть сверхценных идей и~стремление к~навязыванию своего мнения. 
%
\textit{Эпилептоидный тип акцентуации} связан с~такими чертами, как 
склонность к~злоб\-но-тоск\-ли\-во\-му настроению, раздражительности, 
агрессивности, внут\-рен\-ней неудовлетворенности, злости, гнева, ярости, 
жестокости и~конфликтности.
%
 Личности \textit{с депрессивным типом 
акцентуации} проявляют лабильность ко всякого рода неприятностям, 
проявляют неопределенное чувство тяжести, ожидание несчастья. 
%
\textit{Гипертимическая акцентуация} характеризуется специфическим 
поведением, связанным со сменой идей, проявлением словесной ловкости, 
изворотливости, с~направленностью на большое число социальных контактов 
и~отражающим повышенный настрой. 
%
Для \textit{демонстративной, или 
истероидной, акцентуации} характерна поверхностность, наигранность 
переживаний, <<работа на публику>>, стремление вызвать у~аудитории
эмоциональный отклик любой ценой, непродуманность речевого поведения. 
%
\textit{Шизотимность} поведения виртуального субъекта проявляется 
в~направленности высказываний на себя, замкнутостью на узкий круг 
вопросов, \textit{акцентуации} на внутреннем мире.}
  

  
  
\section{Описание методики эксперимента}

  \noindent
  \textbf{Шаг~1.} По открытым публикациям в~социальных сетях определяем 
архитектуру программных средств и~совокупности словарей акцентуации. 

\end{multicols}



\begin{figure*}[h] %fig6
\vspace*{-6pt}
\begin{center}
\mbox{%
\epsfxsize=154.849mm
\epsfbox{sig-6.eps}
}
\end{center}
\vspace*{-11pt}
\Caption{Архитектура экспериментальной платформы программной системы анализа 
акцентуации (тональности) новостных групп сообщений пользователей сети 
<<ВКонтакте>>}
\vspace*{-6pt}
\end{figure*}

\begin{multicols}{2}


 
  Исследование проводилось на примере социальной сети {\sf vk.com} 
с~использованием API IBM Watson Tone Analyzer (библиотеки анализа 
тональности текста IBM), Emotion Analysis (библиотеки анализа эмотивности, 
или эмоциональной\linebreak акцентуации, текста IBM), программной библиотеки 
VK.API (сис\-те\-мы для разработчиков сторонних сайтов, которая предоставляет 
возможность легко авторизовать пользователей <<ВКонтакте>>), шины 
сообщений RabbitMQ (платформы, реализующей систему обмена сообщениями 
между компонентами программной системы, Message Oriented Middleware) 
и~Visual Recognition (библиотеки распознавания изображений). 
  
  Архитектура программного решения проведения экспериментов 
представлена на рис.~6. Для его создания был использован язык 
программирования Python и~язык разработки скриптов ECMAScript.
  
  Эмотивность текста оценивалась с~использованием сервиса Text to Speech 
(TTS). 

\bigskip

  \noindent
  \textbf{Шаг~2.} Для экспериментальных исследований была выбрана группа 
Новости RT социальной сети <<ВКонтакте>>. Исследовалась тональность (как 
доминирующая акцентуация, см.\ таблицу) комментариев пользователей. 
Выбор данного сетевого ресурса обусловлен простым алгоритмом его работы, 
в~том числе и~для неопытного пользователя, что обусловлено использованием 
API социальной сети <<ВКонтакте>>. Пользователь сетевого ресурса\linebreak\vspace*{-12pt}

 { \begin{center}  %fig7
 \vspace*{24pt}
 \mbox{%
\epsfxsize=78mm
\epsfbox{sig-7.eps}
}

\end{center}

\vspace*{3pt}

\noindent
{{\figurename~7}\ \ \small{Оценка тональности (акцентуации) комментариев за день}}

}

%\vspace*{9pt}

\addtocounter{figure}{1}



\noindent
 вводит 
свой идентификационный номер либо короткое доменное имя, после чего 
ресурс получает доступ ко всей текстовой информации со стены пользователя, 
выбирает ключевые слова и~позволяет проанализировать данные на основе 
словарей, маркирующих акцентуацию сообщений, предо\-став\-ля\-емых сервисом 
({\sf http://Indico.io}). 

На следующем шаге рассчитываются суммарные 
значения по\-зи\-тив\-ных/не\-га\-тив\-ных слов в~сообщениях пользователя 
в~процентном соотношении и~ставится в~соответствие эмоциональный маркер 
состояния пользователя в~текущий момент времени (рис.~7).

\bigskip
  
  \noindent
  \textbf{Шаг~3.} В~результате проведенного эксперимента по определению 
тональности текстов виртуальных субъектов социальной сети <<ВКонтакте>> 
на дату обращения были получены следующие результаты: негативных 
комментариев~--- 45,79\%; позитивных комментариев~--- 32,58\%; нейтральных 
комментариев~--- 21,63\%. 
  
  
  \textit{Негативные тексты} объединяют в~себе две категории виртуальных 
субъектов: с~депрессивной и~эпилептоидной акцентуацией поведения, т.\,е.\ 
текс\-ты с~большим объемом обсуждения насилия, описания патологической 
жестокости, выраженным противостоянием, а~так\-же текс\-ты меланхолического 
настроения. 

К~\textit{нейтральным} были отнесены текс\-ты виртуальных 
субъектов с~паранойяльной и~шизотимной акцентуацией~--- это по большей 
час\-ти текс\-ты, наполненные философскими понятиями, абстракциями 
и~услож\-не\-ни\-ями, и~текс\-ты мажорной окраски, проявление искусственного 
оптимизма. 

\textit{Позитивные тексты} получены от виртуальных субъектов 
с~гипертимической акцентуацией речевого поведения, и~отчасти сюда были 
отнесены текс\-ты субъектов с~проявлением истероидной акцентуации, иначе это 
текс\-ты, представляющие собой описание поведения человека, который 
сталкивается с~препятствиями или опасностями, но успешно преодолевает их 
и~достигает успеха, и~час\-тич\-но текс\-ты с~нарочитым описанием эмоциональных 
аффектов~\cite{13-sig}.
  
  \smallskip
  
  Практическое значение разработанной методики заключается в~следующем: 
используя предложенную архитектуру предложенной установки 
вычислительной сети, можно сформировать карту\linebreak поведенческих речевых 
паттернов (см.\ рис.~7) отдельной социальной сети для проведения 
мониторинга напряженности социальных настроений на\linebreak основе тональности 
текстов произвольных виртуальных идентичностей. Следует заметить, что 
методом оценки тональности можно варьировать\linebreak в~зависимости от 
совокупности выбираемых семантических словарей, при этом диапазон 
изменений по результатам проведенного эксперимента демонстрировал 
незначительную выборочную сенситивность, причем без потери значимости 
для одного и~того же текста при использовании иной эталонной коллекции.

\section{Заключение}

  В~исследовании в~рамках практического эксперимента решена задача 
идентификации типа акцентуации паттерна поведения виртуального субъекта 
на основе статистического анализа текста коммуникации и~подтверждена 
гипотеза о~структурных свойствах заданной коммуникации. Актуальность 
предложенной методики определяется еще и~тем, что решение поставленной 
задачи учитывает возрастающее значение развития системы условных знаков, 
в~данном случае условных языков е-ком\-му\-ни\-ка\-ции и~управляющих 
кластеров, позволяющих осуществлять мониторинг и~создавать основу для 
регулирующих воздействий на социальное поведение виртуальных субъектов 
в~Сети.
  
  Результаты экспериментального исследования демонстрируют достаточно 
высокий уровень релевантности характеристик эмоционального анализа 
виртуальных субъектов на основе предложенной методики. Разработанная 
и~апробированная методика оценки среднего фона произвольной социальной 
сети с~привязкой по времени позволяет получить карту поведенческих речевых 
паттернов для проведения мониторинга напряженности социальных настроений 
на основе тональности текстов произвольных виртуальных идентичностей.
  
  Новизна предложенной методики заключается в~идее дополнения 
лингвистического анализа сетевого поведения субъектов статистическим 
анализом и~в выбранных и~адаптированных методах анализа сообщений 
пользователей. В~исследовании определена область практического 
использования результатов и~обоснована репрезентативность разработанной 
методики анализа акцентуации виртуальных субъектов.
  
\section{Тезаурус}

  \textbf{Виртуальная идентичность}~--- коммуникативная репрезентация 
человека в~виртуальной среде, т.\,е.\ то, каким образом он позиционирует себя 
в~сети Интернет (паттерн поведения виртуального объекта). Паттерны 
взаимодействия могут быть представлены в~виде групп речевого 
взаимодействия: подражания, оппонирования, агрессии, конструктивного 
диалога и~т.\,д.
  
  \textbf{Конструктивистский подход} применительно к~виртуальной 
коммуникации строится на предположении, что свойства человеческой психики 
и~мыс\-ле\-фор\-мы, которыми оперирует виртуальный субъект, в~значительной 
степени подвержены влиянию категорий, сформированных социумом в~его 
непосредственном окружении и~усвоенных им в~процессе социализации, 
и,~следовательно, не зависят от биологических ограничений.
  
  \textbf{Лингвистический релятивизм} базируется на гипотезе 
лингвистической относительности и~предполагает, что структура языка  
е-ком\-му\-ни\-ка\-ции влияет на менталитет и~способы идентификации его 
виртуальных агентов, а также на когнитивные процессы реальных субъектов.
  
{\small\frenchspacing
 {%\baselineskip=10.8pt
 \addcontentsline{toc}{section}{References}
 \begin{thebibliography}{99}
\bibitem{1-sig}
\Au{Johansson F., Brynielsson~J., Horling~P., Malm~M., Martenson~C., Truve~S., 
Rosell~M.} Detecting emergent conflicts through Web Mining and Visualization~// 
2011 European Intelligence and Security Informatics Conference 
Proceedings.~--- IEEE, 2011. P.~346--353.
\bibitem{2-sig}
\Au{Kennison S.\,M.} Introduction to language development.~--- Los Angeles, СA, USA: 
SAGE Publications Inc., 2014. 496~p.
\bibitem{3-sig}
\Au{Brown R., Lenneber~E.} A~study in language and cognition~// J.~Abnorm. 
Soc. Psych., 1954. Vol.~49. P.~454--462.
\bibitem{4-sig}
Rethinking linguistic relativity~/ Eds. J.\,J.~Gumperz, S.\,C.~Levinson.~--- 
Studies in the social and cultural foundations of language ser.~--- 
Cambridge: Cambridge University Press, 1999.  No.\,17. 488~p. 
\bibitem{5-sig}
\Au{Slobin D.\,I.} Two ways to travel: Verbs of motion in English and Spanish~// 
Grammatical Constructions: Their form and meaning~/ Eds. M.~Shibatani, 
S.\,A.~Thompson.~--- Oxford: Clarendon Press, 1996. P.~195--220.
\bibitem{6-sig}
\Au{Barbian G.} Detecting hidden friendship in online Social Networks~// 2011 
European Intelligence and Security Informatics Conference  
Proceedings.~--- IEEE, 2011. P.~269--272.
\bibitem{7-sig}
\Au{Горелов И.\,Н., Седов~К.\,Ф.} Основы психолингвистики.~--- М.: Лабиринт, 
2001. 304~с.
\bibitem{8-sig}
\Au{Сидоренко Е.\,В.} Методы математической обработки в~психологии.~--- 
СПб.: Речь, 2002. 350~с.
\bibitem{9-sig}
\Au{Журавлев А.\,П.} Звук и~смысл.~--- М.: Просвещение, 1991. 160~с.
\bibitem{10-sig}
\Au{Vybornova O., Smirnov~I., Sochenkov~I., Kiselyov~A., Tikhomirov~I., 
Chudova~N., Kuznetsova~Y., Osipov~G.} Social tension detection and intention 
recognition using Natural Language Semantic Analysis~// 2011 European 
Intelligence and Security Informatics Conference Proceedings.~--- 
IEEE, 2011. P.~277--281.
\bibitem{11-sig}
\Au{Тихомиров И.\,А., Смирнов~И.\,В.} Интеграция лингвистических 
и~статистических методов поиска в~поисковой машине Exactus~// 
Компьютерная лингвистика и~интеллектуальные технологии: Тр. междунар. 
конф. <<Диалог-2008>>.~--- М.: РГГУ, 2008. С.~485--491.
\bibitem{12-sig}
\Au{Cambria E., Havaci~E., Hussain~A.\,A.} Semantic and affective resource for 
opinion mining and sentiment analysis~// 25th  Florida Artificial 
Intelligence Research Society Conference (International) Proceedings.~--- Palo 
Alto, CA, USA: AAAI Press, 2012. P.~202--207.
\bibitem{13-sig}
\Au{Hoijer H.} The Sapir--Whorf hypothesis~// Conference on the Interrelations of 
Language and Other Aspects of Culture Proceedings: Memoirs of the American 
Anthropological Association, Comparative Studies of Cultures and Civilizations.~--- 
Chicago, IL, USA: University of Chicago Press, 1954. No.\,3. P.~92--105.
 \end{thebibliography}

 }
 }

\end{multicols}

\vspace*{-3pt}

\hfill{\small\textit{Поступила в~редакцию 25.04.17}}

\vspace*{8pt}

%\newpage

%\vspace*{-24pt}

\hrule

\vspace*{2pt}

\hrule

%\vspace*{8pt}


\def\tit{PSYCHOLINGUISTIC ANALYSIS OF~TEXT MESSAGES IN~RUSSIAN 
BASED ON~THEIR PHONOSEMANTIC STATISTICAL 
CHARACTERISTICS}

\def\titkol{Psycholinguistic analysis of~text messages in~Russian 
based on~their phonosemantic statistical 
characteristics}

\def\aut{A.\,S.~Sigov, D.\,A.~Akimov, D.\,O.~Zhukov, E.\,G.~Andrianova, 
V.\,E.~Sachkov, and~V.\,K.~Raev}

\def\autkol{A.\,S.~Sigov, D.\,A.~Akimov, D.\,O.~Zhukov, et al.}
%, E.\,G.~Andrianova,  V.\,E.~Sachkov, and~V.\,K.~Raev}

\titel{\tit}{\aut}{\autkol}{\titkol}

\vspace*{-9pt}


\noindent
Moscow Technological University, 78~Vernadsky Av., Moscow 119454, Russian 
Federation



\def\leftfootline{\small{\textbf{\thepage}
\hfill INFORMATIKA I EE PRIMENENIYA~--- INFORMATICS AND
APPLICATIONS\ \ \ 2017\ \ \ volume~11\ \ \ issue\ 3}
}%
 \def\rightfootline{\small{INFORMATIKA I EE PRIMENENIYA~---
INFORMATICS AND APPLICATIONS\ \ \ 2017\ \ \ volume~11\ \ \ issue\ 3
\hfill \textbf{\thepage}}}

\vspace*{3pt}



\Abste{A text as a complex semantic and syntactic formation has a number of 
psycholinguistic characteristics, which include integrity and semantic orientation. 
A~text can be viewed as a~product of speech activity with a~high degree of semantic 
variation determined by its temporal and sonar characteristics. Nonverbal behavior 
of network entities~--- virtual masks and robotic agents~--- reveals itself in texts.
  The article raises and solves the problem of identifying the type of accentuation of 
pattern of behavior of a virtual entity based on statistical analysis of text 
communication, which allows one to formulate a hypothesis about the structural 
properties of a~given communication and build a matrix of probabilities of 
relationship between virtual masks of subjects.
  The practical significance of the proposed solution is based on the growing 
importance of the development of the system of conditional signs, in this case, the 
conditional languages of e-communication, for the generation of control clusters 
regulating the social behavior of virtual subjects in the network. This assumption is 
based on the hypothesis of Kenneth Ivers, according to which, the better the system 
of conventional signs, the more opportunities to create new algorithms.}
  
\KWE{psycholinguistic characteristics; nonverbal behavior; virtual masks; process 
of thinking; semantic meaning; linguistic relativism}




\DOI{10.14357/19922264170309} 

\vspace*{-9pt}

\Ack
\noindent
The work was carried out within the Ministry of Education and Science of the 
Russian Federation's program of financing the competitive part of public tasks to 
institutions of higher education and scientific organizations to implement initiative 
scientific projects (No.\,28.2635.2017/PP).



%\vspace*{3pt}

  \begin{multicols}{2}

\renewcommand{\bibname}{\protect\rmfamily References}
%\renewcommand{\bibname}{\large\protect\rm References}

{\small\frenchspacing
 {%\baselineskip=10.8pt
 \addcontentsline{toc}{section}{References}
 \begin{thebibliography}{99}
\bibitem{1-sig-1}
\Aue{Johansson, F., J.~Brynielsson, P.~Horling, M.~Malm, C.~Martenson, 
S.~Truve, and M.~Rosell.} 2011. Detecting emergent conflicts through Web Mining 
and Visualization. \textit{European Intelligence and Security Informatics Conference 
 Proceedings}. IEEE. 346--353.
\bibitem{2-sig-1}
\Aue{Kennison, S.\,M.} 2014. \textit{Introduction to language development}. Los 
Angeles, CA: SAGE Publications Inc. 496~p.
\bibitem{3-sig-1}
\Aue{Brown, R., and E.~Lenneber.} 1954. A~study in language and cognition. 
\textit{J.~Abnorm. Soc. Psych.} 49:454--462.
\bibitem{4-sig-1}
Gumperz, J.\,J., and S.\,C.~Levinson, eds. 1999.
\textit{Rethinking linguistic relativity}.  Studies in social and cultural foundations of 
language ser. Cambridge: Cambridge University Press. No.\,17. 488~p.
\bibitem{5-sig-1}
\Aue{Slobin, D.} 1996. Two ways to travel: Verbs of motion in English and Spanish. 
\textit{Grammatical Constructions: Their form and meaning}. Eds. M.~Shibatani and 
S.\,A.~Thomson. Oxford: Clarendon Press. 195--220.
\bibitem{6-sig-1}
\Aue{Barbian, G.} 2011. Detecting hidden friendship in Online Social Networks. 
\textit{European Intelligence and Security Informatics Conference 
Proceedings}. IEEE. 269--272.
\bibitem{7-sig-1}
\Aue{Gorelov, I.\,N., and K.\,F.~Sedov}. 2001. \textit{Osnovy psi\-kho\-ling\-vi\-sti\-ki} 
[Fundamentals of psycholinguistics]. Moscow: Labirint. 304~p.

%\pagebreak

\bibitem{8-sig-1}
\Aue{Sidorenko, E.\,V.} 2002. \textit{Metody matematicheskoy obrabotki 
v~psikhologii} [Methods of mathematical processing in psychology]. St.\ Petersburg: 
Rech. 350~p.
\bibitem{9-sig-1}
\Aue{Zhuravlev, A.\,P.} 1991. \textit{Zvuk i~smysl} [Sound and meaning]. Moscow: 
Prosveshchenie, 1991. 160~p.
\bibitem{10-sig-1}
\Aue{Vybornova, O., I.~Smirnov, I.~Sochenkov, A.~Kiselyov, I.~Tikhomirov, 
N.~Chudova, Y.~Kuznetsova, and G.~Osipov.} 2011. Social tension detection and 
intention recognition using Natural Language Semantic Analysis. 
\textit{European Intelligence and Security Informatics Conference 
Proceedings}. IEEE. 277--281.
\bibitem{11-sig-1}
\Aue{Tihomirov, I.\,A., and I.\,V.~Smirnov.} 2008. In\-te\-gra\-tsiya lingvisticheskikh 
i~statisticheskikh metodov poiska v~poiskovoy mashine Exactus [Integration of 
linguistic and statistical methods of searching in the search engine Exactus]. 
\textit{Conference (International) 
``Dialogue-2008'' Proceedings}. Moscow: RGGU. 485--491.
\bibitem{12-sig-1}
\Aue{Cambria, E., E.~Havaci, and A.\,A.~Hussain.} 2012. Semantic and affective 
resource for opinion mining and sentiment analysis. \textit{25th  
Florida Artificial Intelligence Research Society Conference (International)
Proceedings}. Palo Alto, CA: AAAI Press. 202--207.
\bibitem{13-sig-1}
\Aue{Hoijer, H.} 1954. The Sapir--Whorf hypothesis. \textit{Conference on the 
Interrelations of Language and Other Aspects of Culture Proceedings: Memoirs of the 
American Anthropological Association, Comparative Studies of Cultures and 
Civilizations}. Chicago, IL: University of Chicago Press. 3:92--105.
\end{thebibliography}

 }
 }

\end{multicols}

\vspace*{-3pt}

\hfill{\small\textit{Received April 25, 2017}}
  
\Contr

\noindent
\textbf{Sigov Alexander S.} (b.\ 1945)~--- Academician of the Russian Academy of 
Sciences, President of the Moscow Technological University (MIREA), 
78~Vernadsky Av., Moscow 119454, Russian Federation; 
\mbox{assigov@yandex.ru}

\vspace*{3pt}

\noindent
\textbf{Akimov Dmitry A.} (b.\ 1987)~--- Candidate of Science (PhD) in 
technology, associate professor, Moscow Technological University (MIREA), 
78~Vernadsky Av., Moscow 119454, Russian Federation; 
\mbox{akimov\_d@mirea.ru}

\vspace*{3pt}

\noindent
\textbf{Zhukov Dmitry O.} (b.\ 1965)~--- Doctor of Science in technology, 
professor, Moscow Technological University (MIREA), 78~Vernadsky Av., 
Moscow 119454, Russian Federation; \mbox{zhukovdm@yandex.ru}

\vspace*{3pt}

\noindent
\textbf{Andrianova Elena G.} (b.\ 1963)~--- Candidate of Science (PhD) in 
technology, associate professor, Moscow Technological University (MIREA), 
78~Vernadsky Av., Moscow 119454, Russia n Federation; 
\mbox{dtghmflysq@gmail.com}

\vspace*{3pt}

\noindent
\textbf{Sachkov Valery E.} (b.\ 1989)~--- PhD student, Moscow Technological 
University (MIREA), 78~Vernadsky Av., Moscow 119454, Russian Federation; 
\mbox{megawatto@mail.ru}

\vspace*{3pt}

\noindent
\textbf{Raev Vyacheslav K.} (b.\ 1965)~--- Doctor of Science in technology, 
professor, Moscow Technological University (MIREA), 78~Vernadsky Av., 
Moscow 119454, Russian Federation; \mbox{raev@mirea.ru}

\label{end\stat}


\renewcommand{\bibname}{\protect\rm Литература} 
  