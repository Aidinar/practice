
\def\l{\lambda}


\renewcommand{\figurename}{\protect\bf Algorithm}
\renewcommand{\tablename}{\protect\bf Table}

\def\stat{razumchik}


\def\tit{REVISITING JOINT STATIONARY DISTRIBUTION IN~TWO FINITE CAPACITY QUEUES 
OPERATING IN~PARALLEL}

\def\titkol{Revisiting joint stationary distribution in~two finite capacity queues 
operating in~parallel}

\def\autkol{L.~Meykhanadzhyan, S.~Matyushenko, D.~Pyatkina,
and~R.~Razumchik}

\def\aut{L.~Meykhanadzhyan$^1$, S.~Matyushenko$^2$, D.~Pyatkina$^3$,
and~R.~Razumchik$^4$}

\titel{\tit}{\aut}{\autkol}{\titkol}

%{\renewcommand{\thefootnote}{\fnsymbol{footnote}}
%\footnotetext[1] {The 
%research of Yuri Kabanov was done under partial financial support   of the grant 
%of  RSF No.\,14-49-00079.}}

\renewcommand{\thefootnote}{\arabic{footnote}}
\footnotetext[1]{School No.\,281 of Moscow, 
7~Raduzhnaya Str. Moscow 129344, Russian Federation;
\mbox{lameykhanadzhyan@gmail.com}}
\footnotetext[2]{Peoples' Friendship University of Russia (RUDN University),
6~Miklukho-Maklaya Str., Moscow 117198, Russian Federation;
\mbox{matyushenko\_si@rudn.university}}
\footnotetext[3]{Peoples' Friendship University of Russia (RUDN University),
6~Miklukho-Maklaya Str., Moscow 117198, Russian Federation;
\mbox{pyatkina\_da@rudn.university}}
\footnotetext[4]{Institute of Informatics Problems, Federal Research Center 
``Computer Science and Control'' of the Russian Academy of Sciences; 
44-2~Vavilova Str., Moscow 119333, Russian Federation;
Peoples' Friendship University of Russia,
6~Miklukho-Maklaya Str., Moscow 117198, Russian Federation;
\mbox{rrazumchik@ipiran.ru}}


\index{Meykhanadzhyan L.}
\index{Matyushenko S.}
\index{Pyatkina D.}
\index{Razumchik R.}
\index{Мейханаджян Л.\,А.}
\index{Матюшенко С.\,И.}
\index{Пяткина Д.\,А.}
\index{Разумчик Р.\,В.}


\vspace*{-1pt}

\def\leftfootline{\small{\textbf{\thepage}
\hfill INFORMATIKA I EE PRIMENENIYA~--- INFORMATICS AND APPLICATIONS\ \ \ 2017\ \ \ volume~11\ \ \ issue\ 3}
}%
 \def\rightfootline{\small{INFORMATIKA I EE PRIMENENIYA~--- INFORMATICS AND APPLICATIONS\ \ \ 2017\ \ \ volume~11\ \ \ issue\ 3
\hfill \textbf{\thepage}}}


\Abste{The paper revisits the problem of the computation
of the joint stationary probability distribution~$p_{ij}$
in a~queueing system consisting of 
two single-server queues, each of capacity $N\ge 3$, 
operating in parallel, and a~single Poisson flow. 
Upon each arrival instant, one customer is put simultaneously into each system. 
When a~customer sees a~full system, it is lost.
The service times are exponentially distributed 
with different parameters.
Using the approach based on generating functions,
the authors obtain a~new system of equations of 
a~smaller size than the size of the original
system of equilibrium equations ($3N-2$ compared to $(N+1)^2$). 
Given the solution of the new system, the whole joint stationary 
distribution can be computed recursively.
The new system gives some insights into the interdependence 
of~$p_{ij}$ and~$p_{nm}$. If relations between $p_{i-1,N}$ 
and $p_{i,N}$ for $i=3,5,7,\ldots$
are known, then the blocking probability can be computed recursively.
Using the known results for the asymptotic behavior of~$p_{ij}$ as
$i,j \rightarrow \infty$, the authors illustrate this idea by 
a~simple numerical example.}


\KWE{two queues; generating function; stationary distribution; paired customers}

\DOI{10.14357/19922264170312} 


\vspace*{-4pt}


\vskip 12pt plus 9pt minus 6pt

      \thispagestyle{myheadings}

      \begin{multicols}{2}

                  \label{st\stat}

\section{Introduction}

\noindent
The system with two single-server queues (both limited and unlimited capacity
cases) operating in parallel has received significant attention in the literature
due its potential application in real-life scenarios
(for example, packet switches, packet radio networks, parallel processing systems,
inventory control of database systems, etc.). Further, it is assumed that the system consists of 
two queues (say, queue~1 and queue~2) each with a single server
and there is a single Poisson flow of customers arriving at it. 
Each customer upon arrival is instantly duplicated: one customer goes 
to queue~1 and the other goes to queue~2. Both queues are working independently,
service times follow exponential distribution with different parameters,
and the service discipline in a queue is either FCFS (first-come-first-served), 
LCFS (last-come-first-served), or Random. 
Despite the simplicity of the structure, even under such markovian assumptions, 
the system turned out to be notoriously hard to analyze.

A~big list of publications on the topic is given in~\cite{intro1},
where the authors give an overview of functional equations (and solution approaches), 
which arise in the analysis of such systems with infinite capacity queues. 
References to the application 
related papers are also given. Among the pioneer works in the 
area, papers~\cite{orig1,intro2,orig2-1,orig2-2} are worth noticing.

In this paper, the authors revisit the problem of the computation 
of the joint stationary distribution in the case, when
both queues have finite capacity. 
Under the exponential assumptions (and given additional dedicated Poisson flows
to each queue), the matrix algorithm
has been proposed already in~\cite{orig1}. Some further considerations,
including the study of correlation between the queues' sizes were 
continued in~\cite{orig2}.
In general, the cases, when both of queues are on finite capacity 
or one of the queues is (see, for example,~\cite{NEW1}), have received less attention 
in the literature. This is presumably due to the fact that in
those cases in order to obtain the joint stationary distribution, one can
use widely-adopted general techniques:
folding algorithm, linear level reduction or 
block-gaussian elimination algorithms
(see, for example,~\cite{lat-ram,bocharov}).

Our motivation for revisiting this problem comes
from the papers~\cite{xx1,xx2,xx3,NEW2,NEW3}, where the generating 
function technique (which utilizes some properties 
of special functions (Chebyshev and Gegenbauer
polynomials)) was applied to the systems with 
two finite-capacity queues 
and allowed one to derive 
new relations for the recursive computation of the 
joint stationary distribution. 

      
Having applied the same
approach for the system considered here, 
the present authors found that it does not lead to the recursive solution. 
Yet, it gives an alternative way to compute the joint stationary distribution. 
Specifically, it requires the solution of the system of linear algebraic equations
of the size $(3N-2)$, when the size of both queues
is equal  to $N\ge3$ 
(the exact solutions for $N=1$ and~2
are obtained in~\cite{orig2}) and is immediately suitable for exact 
arithmetics implementation.
If the whole joint stationary distribution is not
of importance, this approach gives the straight way to 
calculate the blocking probability
and new insights into the dependencies between the
joint probabilities~$p_{ij}$, which prevent the recursive solution. 

The paper is structured as follows. In section~2,
the description of the system is given and some known results,
which are necessary in what follows, are repeated.
Section~3 contains the main contribution of the paper. Here,
it is shown how new relations for the joint stationary distribution
can be obtained (see Eqs.~\eqref{n1}--\eqref{n8}). 
The insights into the interdependence between the joint 
stationary probabilities is discussed in section~4.
Section~5 concludes the paper.


\section{System Description}

\noindent
The system under consideration consists of two single-server
finite capacity queues (queue~1 and queue~2), operating in parallel
independently of each other.
By suffering a~little a lack of generality, let us assume 
that the capacities of both queues are equal to $N\ge3$.
There is one incoming Poisson flow of rate~$\lambda$
arriving at the system. Upon arrival,
each customer is split into two customers:
one enters queue~1 and another enters queue~2.
Service time of customers in queue~$i$ follows
exponential distribution with rate~$\mu_i$, $i=1,2$.
Since we are interested here only in the queue size related
characteristics, we allow the service discipline in queues
to be either FCFS, or LCFS, or Random.
We are interested in the case (as in~\cite{orig1}), when 
a customer always occupies the place in the queue 
whenever it is not full. This is much different
from the case, when the customer checks the queues' sizes before 
splitting and leaves the system if at least one queue is full. 

Denote by~$p_{ij}$ stationary probability of the fact that
there are~$i$~customers in queue~1 and~$j$ customers in queue~2.
From~\cite{orig1}, it follows that 
the double generating function for~$p_{ij}$,
$$
P(u,v)
=
\sum\limits_{i=0}^N \sum\limits_{j=0}^N  u^i v^j p_{ij}\,,  \enskip 
0 \le u \le 1\,, \ 0 \le v \le 1\,,
$$
has the form:
\begin{equation}
\label{eq.5.4}
B(u,v) P(u,v)= A(u,v)
 \end{equation}



\noindent
where
$$
B(u,v)=\lambda u^2 v^2 - u(\lambda v+\mu_1 v+ \mu_2 v-\mu_2)+\mu_1 v\,;
$$

\vspace*{-12pt}

\noindent
\begin{multline*}
A(u,v)= \mu_1 v \left( u-1 \right)\sum\limits_{j=0}^N v^j p_{0j}\\
{}+
\mu_2 u \left(v-1\right) \sum\limits_{i=0}^N  u^i p_{i0}
+ \lambda v^2 u^{N+1}(1-u)  \sum\limits_{j=0}^N  v^j p_{Nj}
\\
{}+
\lambda u^2 v^{N+1}(1-v)\sum\limits_{i=0}^N u^i p_{iN} \\
{}+
\lambda u^{N+1}v^{N+1}(1-u)(1-v)p_{NN}\,.
\end{multline*}
The quadratic polynomial $B(u,v)$ has two roots:
\begin{multline*}
%\label{u12}
u_{1,2}=u_{1,2}(v)=\left(
\vphantom{\sqrt{(v(\lambda+\mu_1 + \mu_2)-\mu_2)^2-4 \lambda \mu_1 v^3  }}
v(\lambda+\mu_1 + \mu_2)- \mu_2 \right.\\
\left.{}
\mp \sqrt{(v(\lambda+\mu_1 + \mu_2)-\mu_2)^2-4 \lambda \mu_1 v^3  }\right)\Big / 
(2\lambda v^2)\,.
\end{multline*}
The generating function $P(u,v)$ is the ratio of two polynomial functions.
For each value of~$v$, probability generating function $P(u,v)$ is 
a~continuous function of~$u$
in the interval $[0,1]$.
Then, since the left part in~\eqref{eq.5.4}
vanishes at points $(u_1(v),v)$, and~$(u_2(v),v)$,
then the right part must vanish at these points too.
In the next section, it will be shown that 
from this observation, one can obtain the system 
of linear algebraic equations only for the probabilities 
$\{p_{0j},p_{jN}, 0 \le j \le N\}$ 
and $\{p_{j0}, 0 \le j \le N-3 \}$ which 
can be solved by any standard numerical method. Once these probabilities 
are known, the computation of the rest joint stationary probabilities~$p_{ij}$ 
is performed recursively from the system of equilibrium equations.
{\looseness=1

}


\section{New System of~Equations}

\noindent
Both equations $A(u_1(v),v)=0$ and $A(u_2(v),v)=0$
share the same unknown quantities.
If one expresses term with 
$\sum_{j=0}^N v^j p_{0j}$
from the first equation and put it in the second equation, after collecting
common terms, one obtains:
\begin{multline*}
\mu_2(v-1)\sum\limits_{i=0}^N 
\left(\fr{u_2^{i+1}-u_1^{i+1}}{u_2-u_1}-
u_1u_2 \fr{u_2^{i}-u_1^{i}}{u_2-u_1}\right)
p_{i0}\\
{}+
\lambda v\left(1-u_1-u_2+u_1u_2\right)
\fr{u_2^{N+1}-u_1^{N+1}}{u_2-u_1}\!\sum\limits_{j=0}^N\!\! v^{j+1} p_{Nj}\\
{}+
\lambda v^{N+1}(1-v)\sum\limits_{i=0}^N 
\left ( 
\fr{u_2^{i+2}-u_1^{i+2}}{u_2-u_1}\right.\\
\left.{}-
 u_1u_2\fr{u_2^{i+1}-u_1^{i+1}}{u_2-u_1}
\right)p_{iN}
\end{multline*}

\noindent
\begin{multline}
{}+
\lambda v^{N+1}
{(1-u_1-u_2+u_1u_2)}(1-v)\\
{}\times{}
\fr{u_2^{N+1}-u_1^{N+1}}{u_2-u_1}p_{NN}=0\,.
\label{eq-new-1}
\end{multline}

\noindent
Instead of cancelling $\sum_{j=0}^N v^j p_{0j}$, let us express 
the term with $p_{NN}$ from the {$A(u_1(v),v)=0$}
and put it into ${A(u_2(v),v)=0}$. By doing so, one gets another relation:
\begin{multline}
v\mu_1(1-u_1-u_2+u_1u_2)
\fr{{u_2}^{N+1}-{u_1}^{N+1}}{u_2-u_1} \sum\limits_{j=0}^N v^j p_{0j}
\\
+{}
\mu_2(v-1)\sum\limits_{i=0}^N (u_1u_2)^{i+1}
\left (
\fr{{u_2}^{N-i+1}-u_1^{N-i+1}}{u_2-u_1}\right.\\
\left.{}-\fr{{u_2}^{N-i}-u_1^{N-i}}{u_2-u_1}
\right) p_{i0} 
\\
{}+
\lambda v^{N+1}(1-v)\sum\limits_{i=0}^N (u_1u_2)^{i+2}
\left (
\fr{u_2^{N-i}-{u_1}^{N-i}}{ u_2-u_1}
\right.\\
\left.{}-\fr{u_2^{N-i-1}-{u_1}^{N-i-1}}{u_2-u_1}
\right ) p_{iN}
=0\,.
\label{eq-new-2}
\end{multline}
It is straightforward to see that the roots
$u_{1,2}$ admit the following representation:
\begin{equation*}
\label{eq524}
u_{1}=
\fr{\sqrt{\mu_1}{\lambda v}}
a(x)\,;\enskip
u_{2}=
\fr{\sqrt{\mu_1}{\lambda v}}
b(x)
\end{equation*}
where
\begin{gather*}
x = \fr{v(\lambda+\mu_1 + \mu_2)-\mu_2}{v\sqrt{\mu_1\lambda v}}\,;
\\
a(x)= \fr{ x - \sqrt{x^2 -4}
}{2}\,; \enskip
b(x)= \fr{ x + \sqrt{x^2 -4} }{2}\,.
\end{gather*}
It can be shown that $|x|>2$ for all $v \in (0,1]$.
It is well-known that the fraction
$(b(x)^m -a(x)^m) /( b(x)-a(x))$ is in fact a polynomial in~$v$ for $m\ge 1$.
Thus, $(u_2^m - u_1^m) / (u_2-u_1)$ is a polynomial in~$v$ as well.
After some tedious algebra (derivation is analogous to the one in~\cite{xx3}), 
let us find that for $m\ge1$, the following
representation holds:
\begin{equation}
\hspace*{-0.3mm}\fr{u_2(v)^m - u_1(v)^m }{u_2(v)-u_1(v)}
=
\left (\!
\sqrt{\fr{\mu_1}{\lambda}}\right)^{\!m-1}\!\!
\sum\limits_{n=\left\lfloor{m}/{2}\right\rfloor}^{2(m-1)}\hspace*{-2mm}
v^{-n}a_{m,n}\!
\label{eq_tran}
\end{equation}
where 
\begin{multline*}
a_{m,n}=\sum\limits_{j=\max\left\{0,\left\lceil n-(m-1)\right\rceil\right\}}
^{\lfloor(2n-(m-1))/3\rfloor}
d_{m,2n-(m-1)-2j,j}\,,\\ \fr{m-1}{2}\leq n\leq 2(m-1)\,;
\end{multline*}

%\vspace*{-12pt}

\noindent
\begin{multline*}
d_{i,m,k}\\
{}=C_{i-m-1}^{m+1} (0)\binom{m}{k}\!
\left (
\fr{\lambda +\mu_1 + \mu_2}{\sqrt{\lambda\mu_1}}
\right )^{m-k}\!\!\!
\left(-\fr{\mu_2 }{\sqrt{\lambda\mu_1}}\right)^{\!k}\!\!\!.\hspace*{-2.316pt}
\end{multline*}
Here, $\binom{m}{k}$ is the binomial coefficient
and $C^m_n(0)$ denotes the value of 
Gegenbauer polynomial~$C^m_n(x)$ at point $x=0$ 
(see, for example,~\cite[p.\,175]{erdelyi}).
Since each fraction $(u_2^m - u_1^m) / (u_2-u_1)$
is a polynomial in~$v$ with real coefficients
(defined by~\eqref{eq_tran})
and $u_1u_2={\mu_1}/{\l v}$,  
$u_1+u_2=((\l+\mu_1+\mu_2)v-\mu_2)/(\l v^2)$,
both expressions on the left in~\eqref{eq-new-1} and~\eqref{eq-new-2} 
are polynomials in~$v$ as well
with real coefficients depending on~$\lambda$, $\mu_1$, $\mu_2$
and certain~$p_{ij}$. Due to the lack of space we omit  
detailed derivations and just state the final result.
From the fact that both polynomials~\eqref{eq-new-1} and~\eqref{eq-new-2} 
are equal to zero for $v \in (0,1]$, it follows 
that their coefficients are equal to zero.
This leads to two systems of linear algebraic equations (one from~\eqref{eq-new-1}
and the other from~\eqref{eq-new-2}) for the stationary
probabilities on the boundaries ($p_{0j}$, $p_{i0}$, $p_{iN}$,
and $p_{Nj}$). Careful inspection shows that from these two systems,
one can draw one single system of equations of size $3N\hm-4$ for the probabilities 
$\{p_{0j},p_{jN},\ 0 \le j \le N\}$ and $\{p_{j0}, 0 \le j \le N-3 \}$,
which can be solved numerically. 
Specifically, for odd $N\ge3$,
the new system of equations has the form:
\begin{multline}
\label{n1}
\sum\limits_{j=0}^N p_{0j} a_{N+1,j+(N-1)/2} 
\\
{}+
\sum_{j=0}^{(N-1)/2-1}
p_{jN} \lambda r^{j+2} b_{N-j,N+(N-1)/2-j}
\\
{}- p_{00} \mu_2 r b_{N+1,(N-1)/2}=0\,;
\end{multline}

\vspace*{-12pt}

\noindent
\begin{multline}
\label{n2}
\sum\limits_{j=i}^N
p_{0j} a_{N+1,j+(N-1)/2-i} 
\\
{}+
\sum\limits_{j=0}^{(N-1)/2+(i-1)}
p_{jN} \lambda r^{j+2} b_{N-j,N+(N-1)/2-j-i}=0\,,\\ 
i=\overline{1,\fr{N-1}{2}}\,;
\end{multline}

\vspace*{-12pt}

\noindent
\begin{multline}
\label{n3}
\sum\limits_{j=(N+1)/2}^N
p_{0j} a_{N+1,j-1} +
\sum\limits_{i=0}^{N-1} p_{iN} \lambda r^{i+2} b_{N-i,N-i-1}\\
{}+
p_{NN}\lambda r^{N+1}b_{1,0}=0\,;
\end{multline}

\begin{figure*}
\Caption{Recursive computation of $p_{ij}$}

\

\hrule

\

\noindent
For $0 \le i \le N-2$, $p_{i,N-1} \leftarrow 
p_{i+1,N}\left(1+\rho_1^{-1}+\rho_2^{-1}\right)-p_{iN}-
\rho_1^{-1} p_{i+2,N}$

\noindent 
For $1 \le j \le N-2$, $p_{1,j} \leftarrow 
p_{0,j}\left( \rho_1+{\mu_2/\mu_1}\right)-p_{0,j+1}{\mu_2/\mu_1}$

\noindent 
For $2 \le i \le N-1$
   
   \hspace*{3mm}For $1 \le j \le N-1$

\hspace*{4mm}$p_{i,j} \leftarrow p_{i-1,j}\left(\rho_1+1+{\mu_2/\mu_1}\right)-\rho_1 p_{i-2,j-1}-p_{i-1,j+1}{\mu_2/\mu_1}$

\noindent 
$p_{N-2,0} \leftarrow p_{N-3,0}\left( \rho_1 +1\right)-p_{N-3,1}{\mu_2/\mu_1}$

\noindent
For $1 \le j \le N-1$,
$p_{N,j} \leftarrow p_{N-1,j}\left( \rho_1 +1+{\mu_2/\mu_1}\right)- \rho_1 p_{N-2,j-1}-p_{N-1,j+1}{\mu_2/\mu_1}$

\noindent For $N-1 \le i \le N$, $p_{i,0} \leftarrow p_{i-1,0}\left( \rho_1 +1\right)-p_{i-1,1}{\mu_2/\mu_1}$

\vspace*{3pt}

\hrule
\end{figure*}

\vspace*{-12pt}

\noindent
\begin{multline}
\label{n4}
\sum\limits_{j=N-i}^N
p_{0j} a_{N+1,j+i-(N+1)/2}  \\
{}+ \sum\limits_{j=0}^{2i+1}
p_{jN}  r^{j+2} b_{N-j,(N-1)/2+i-j}=0\,,\\
 i=\overline{1,\fr{N-3}{2}}\,;
\end{multline}
\begin{equation}
\label{n5}
\sum\limits_{j=0}^{2i+1}p_{jN}  r^{j} d_{j+1,i}=0\,,\enskip 
i=\overline{0,\fr{N-3}{2}}\,;
\end{equation}
\begin{equation}
\label{n6}
\sum\limits_{j=1}^{N} p_{jN} \lambda r^{j-1} d_{j+1,(N-1)/2}=0\,;
\end{equation}

\vspace*{-14pt}

\noindent
\begin{multline}
\label{n7}
\sum\limits_{j=0}^N p_{0j} a_{N+1,j+N-2-i} \\[-1pt]
{}+ \sum\limits_{j=0}^i p_{jN} \lambda r^{j+2} b_{n-j,2N-2-i-j} 
\\{}-
\sum\limits_{j=0}^{N-3-2i} p_{j0} \mu_2 r^{j+1} b_{N+1-i,N-2-i-j}=0\,,\\
i=\overline{0,\fr{N-5}{2}}\,;
\end{multline}

\vspace*{-14pt}

\noindent
\begin{multline}
\label{n8}
\sum\limits_{j=0}^i p_{0j} a_{N+1,j+2N-i} 
- \sum\limits_{j=0}^i p_{j0} \mu_2 r^{j+1} b_{N+1-j,2N-i-j}\\
{}=0\,,\enskip
 i=\overline{1,N-3}\,,
\end{multline}
where the following notations are used:
\begin{align*}
r&= \sqrt{\fr{\mu_1}{\lambda}}\,; \\
a_{ij}&=\mu_2\left(\sqrt{\fr{\mu_1}{\lambda}}\right)^2 C_{i,j} (0) -\mu_1C_{i,j+1}(0)\,;
\\
b_{ij}&=\sqrt{\fr{\mu_1}{\lambda}}\, C_{i,j}(0)-C_{i-1,j}(0)\,;\\
d_{ij}&=\sqrt{\fr{\mu_1}{\lambda}}\, C_{i,j}(0)-C_{i+1,j+1}(0)\,.
\end{align*}
System \eqref{n1}--\eqref{n8} consists of $3N-4$
equations in $3N-2$ unknowns. Two additional equations
follow from the fact that each queue, when considered 
independently, operates as the standard $M/M/1/N$ queue
and, thus, 
$$
\sum\limits_{j=0}^N p_{0j} =\fr{1-\rho_1}{1- \rho_1^{N+1}}\,;
\enskip
p_{\cdot,N}=\sum\limits_{i=0}^N
p_{iN}=\rho_2^N \fr{1-\rho_2}{1- \rho_2^{N+1}}.
$$ 
Here and henceforth, $\rho_i=\lambda/\mu_i$, $i=1,2$.
Once the system~\eqref{n1}--\eqref{n8},
supplemented with these two equations, is solved,
all other probabilities~$p_{ij}$ can be found recursively 
(see Algorithm~1).




The relations in Algorithm~1 follow from the system 
of equilibrium equations for~$p_{ij}$ (see, for example,~\cite[p.~435]{orig1}). 
Algorithm~1 is not well suited 
for the computation of the 
whole joint stationary distribution~$p_{ij}$
because the accuracy of the results heavily depends
on the values of initial parameters and sometimes may be low.


\section{Relating {{$p_{i-1,N}$}} and~{{$p_{i,N}$}}}

\noindent
System~\eqref{n1}--\eqref{n8}
gives some insights into the interdependence of~$p_{ij}$ and~$p_{nm}$.
Specifically, Eqs.~\eqref{n5} and~\eqref{n6}
show that it is enough to know the relations between~$p_{2,N}$ 
and~$p_{3,N}$, $p_{4,N}$ 
and~$p_{5,N}$, $p_{6,N}$ and~$p_{7,N}$, etc.\ to compute the value of~$p_{NN}$.
Indeed, let $p_{i,N}= p_{i-1,N} \alpha_i$. From~\eqref{n5} and~\eqref{n6},
for $y_{jN}=p_{jN}/p_{0N}$, one has:
\begin{equation}
\label{app1}
y_{1N} =
-\fr{d_{1,0} }{r d_{2,0}}\,;
\end{equation}

\vspace*{-12pt}

\noindent
\begin{multline}
\label{app2}
y_{2i,N} =-\fr{ \sum\nolimits_{j=0}^{2i-1}
y_{jN} r^{j} d_{j+1,i}}{r^{2i} d_{2i+1,i}+\alpha_{2i+1} r^{2i+1} d_{2i+2,i}}\,, \\
 i=\overline{1,\fr{N-3}{2}}\,;
\end{multline}

\vspace*{-12pt}

\noindent
\begin{multline}
\label{app3}
y_{N-1,N} \\
{}= -\fr{ \sum\nolimits_{j=1}^{N-2}
y_{jN} r^{j-1} d_{j+1,(N-1)/2}}
{r^{N-2} d_{N,(N-1)/2}+\alpha_{N} r^{N-1} d_{N+1,(N-1)/2}}\,.
\end{multline}
From this system and the normalization condition 
$p_{\cdot,N} = \rho_2^N (1-\rho_2)/ (1- \rho_2^{N+1})$,
one finds $p_{0N}\linebreak =p_{\cdot,N}/\sum_{i=0}^N y_{iN}$ and
$p_{NN}=p_{0N} y_{N,N}$.
We are unaware of any general rule for choosing~$\alpha_i$.
Yet, some heuristics can be suggested. 
Without any loss of generality, further on, we
consider $\lambda=1$.
From the results in~\cite{orig2-1}, it
follows that if $\mu_2>\mu_1>1$,
then for large~$N$, one has:
\begin{equation*}
%\label{p25p35}
p_{iN}\approx
p_{i-1,N} \left ( \fr{\Phi(x_{i+1}) - \Phi(x_{i})}
{\Phi(x_{i}) - \Phi(x_{i-1})}
\right )
\end{equation*}
where
\begin{gather*}
x_i = \fr{1}{\sqrt{N}}
\left (i - \fr{\mu_2-\mu_1}{\mu_2-1} N \right )\,;
\\
\Phi(x)=\fr{1}{\sqrt{2 \pi}\, \sigma}
\int\limits_{-\infty}^x e^{-{t^2}/({2 \sigma^2})}\,dt\,;
\\
\sigma^2= 
\fr{\mu_1^2 \mu_2 + \mu_1 \mu_2^2 + \mu_1 + \mu_1^2 + \mu_2 + \mu_2^2 - 
6 \mu_1 \mu_2 }{(\mu_2-1)^3}\,.
\end{gather*}
Thus, the (approximate) value of~$p_{NN}$ can be found from~\eqref{app1}--\eqref{app3}
with 
$$
\alpha_i=\fr{\Phi(x_{i+1}) - \Phi(x_{i})}{\Phi(x_{i}) - \Phi(x_{i-1})}\,.
$$
It is worth noticing that the value $p_{\cdot,N}(1-\Phi(x_{N}))$
gives another approximation for~$p_{NN}$ if $\mu_2>\mu_1>1$ and
can be quite accurate. We can try one's luck and
use the same value of~$\alpha_i$ in the overloaded case as well 
(i.\,e., when the load of at least one queue is~1)
with two minor modifications. Firstly, substitute~$-\sigma^2$ instead 
of~$\sigma^2$ and, secondly, put $\alpha_i \equiv 1$ whenever 
$-\sigma^2<0$ or $\Phi(x_{i}) \approx \Phi(x_{i-1})$.   
With these agreements, by using $p_{i,N}= p_{i-1,N} \alpha_i$
in~\eqref{app1}--\eqref{app3}, the value of~$p_{NN}$ can be approximated with
$2 p_{0N} y_{N,N}$. The data in the table give the idea of the quality 
of the approximation
for the case $N=35$, $\lambda=1$, and $\mu_1=0.01$ ($\rho_1=100$). 

\vspace*{6pt}

%\begin{table*}
{\small
\begin{center}

%\label{tab1}
\begin{tabular}{lcc}
\multicolumn{3}{p{65mm}}{Exact values of $p_{NN}$ (solution of~\eqref{n1}--\eqref{n8})
and approximate values of~$p_{NN}$ (solution of~\eqref{app1}--\eqref{app3}). 
The case of $N=35$,
$\lambda=1$, and $\mu_1=0.01$}\\
\multicolumn{3}{c}{\ }\\[-6pt]
\hline 
\multicolumn{1}{c}{\raisebox{-6pt}[0pt][0pt]{$\mu_2$}} & \multicolumn{2}{c}{$p_{NN}$}\\
\cline{2-3}
& Exact value  & Approximate value\\  
\hline 
&&\\[-9pt]
$2.5$ & $1.1310\cdot 10^{-15}$ & $1.1334\cdot 10^{-15}$  \\ 
%\hline 
$2$ & $3.6258\cdot 10^{-12}$ & $3.6380\cdot 10^{-12}$  \\ 
%\hline 
$1.25$ & $5.1702\cdot 10^{-5}\hphantom{^1}$ & $5.1934\cdot 10^{-5}\hphantom{^1}$  \\ 
%\hline 
$1.1$ & 0.0027 & 0,0027 \\ 
%\hline 
$1.01$ & 0.0217 & 0.0218  \\ 
%\hline 
$0.9$ & 0.1013 & 0.1016  \\ 
%\hline 
$0.8$ & 0.1989 & 0.1972  \\ 
\hline 
\end{tabular} 
\end{center}
}
%\end{table*}


\section{Concluding Remarks}

\noindent
In this paper, it  has been shown that the joint stationary probability distribution
can be computed using the system of equations
of the smaller size (than the original one).
The idea follows from the fact that in the finite-capacity
case, both roots in the denominator of the generating function 
are the roots of its numerator (on the contrast to the infinite-capacity case).
The drawbacks of the utilized method can be seen when computing 
the whole joint distribution~$p_{ij}$. Here, the widely-adopted
Gaussian elimination and matrix-analytic methods are preferable.
Yet, when only the blocking probability is of interest,
the utilized method leads to the new computational procedure and
some insights into the interdependencies between~$p_{ij}$ and~$p_{nm}$. 
Unlike in some other system with two queues of 
finite-capacity, here the values of~$p_{ij}$ do not allow 
recursive computation, which is, as clearly seen, due simultaneously happening arrivals.
Still the utilized method allows further investigations into
the new procedures for the approximate computation of~$p_{ij}$ as suggested 
in~\cite{NEW2}.

\Ack  
\noindent
This work was supported in part by the
Russian Foundation for Basic Research (grants 15-07-03007 and 15-07-03406).


\renewcommand{\bibname}{\protect\rmfamily References}

\vspace*{-6pt}

%\vspace*{-6pt}

{\small\frenchspacing
{\baselineskip=10.65pt
\begin{thebibliography}{99}
\bibitem{intro1} 
\Aue{El-hady, E., J.~Brzdek, and H.~Nassar.} 2017. 
On the structure and solutions of functional 
equations arising from queueing models. \textit{Aequationes Math.} 
91(3):~445--477.

\bibitem{intro2}
\Aue{Kingman, J.\,F.\,C.} 1961. 
Two similar queues in parallel. \textit{Ann. Math. Stat.} 32(4):1314--1323.

\bibitem{orig1} %3
\Aue{Hunter, J.\,J.} 1969. Two queues in parallel. \textit{J.~Roy. 
Stat. Soc. B} 31(3):432--445. 
%The Journal of the Royal Statistical Society, Series B


\bibitem{orig2-2} %4
\Aue{Flatto, L., and S.~Hahn.} 1984. Two parallel queues created by 
arrivals with two demands~I. \textit{SIAM J.~Appl. Math.} 44(5):1041--1053.


\bibitem{orig2-1} %5
\Aue{Flatto, L.} 1985. 
Two parallel queues created by arrivals with two demands~II. 
\textit{SIAM J.~Appl. Math.} 45(5):861--878.

\bibitem{NEW1} %6
\Aue{Rao, B.\,M., and M.\,J.\,M.~Posner}. 
1985. Algorithmic and approzimation analyses of the split and match queue.
\textit{Commun. Stat. Stochastic Models} 1(3):433--456.

\bibitem{lat-ram} %7
\Aue{Latouche, G., and V.~Ramaswami.} 1999. 
\textit{Introduction to matrix analytic methods in stochastic modeling}. 
Philadelphia, PA: SIAM. 334~p.

\bibitem{bocharov} %8
\Aue{Bocharov, P.\,P., C.~D'Apice, A.\,V.~Pechinkin, and S.~Salerno}. 2004.
\textit{Queueing theory}. Utrecht: VSP Publishing. 450~p.

\bibitem{xx2} %9
\Aue{Bavinck, H., G.~Hooghiemstra, and E.~De~Waard.} 1993. 
An application of Gegenbauer polynomials in queueing theory. 
\textit{J.~Comput. Appl. Math.} 49:1--10.


\bibitem{xx1} %10
\Aue{Avrachenkov, K.\,E., N.\,O.~Vilchevsky, and G.\,L.~Shevlyakov}. 
2003. Priority queueing with finite buffer size and randomized push-out mechanism. 
\textit{ACM  Conference (International) on Measurement and Modeling of Computer
Systems Proceedings}. New York, NY: ACM. 31(1):324--335.



\bibitem{xx3} %11
\Aue{Razumchik, R.\,V.} 2014. 
Analysis of finite capacity queue with negative customers and bunker for ousted 
customers using Chebyshev and Gegenbauer polynomials. 
\textit{Asia Pac. J.~Oper. Res.} 31(4).
%1450029 (21 pages)

\bibitem{NEW3} %12
\Aue{Zaryadov, I.\,S., L.\,A.~Meykhanadzhyan, T.\,A.~Milovanova, and R.\,V.~Razumchik}. 2015. 
Metod nakhozhdeniya statsionarnogo raspredeleniya ocheredi v~dvukhkanal'noy sisteme 
s~uporyadochennym vkhodom konechnoy emkosti [On the method of 
calculating the stationary distribution in the finite two-channel system with 
ordered input].
\textit{Sistemy i~Sredstva Informatiki~--- 
Systems and Means of Informatics} 25(3):44--59.

\bibitem{NEW2} %13
\Aue{Razumchik, R.\,V.} 2015. Algebraic method 
for approximating joint stationary distribution in finite capacity queue 
with negative customers and two queues. 
\textit{Informatika i~ee Primeneniya~--- Inform. Appl.} 9(4):68--77.

\bibitem{orig2} %14
\Aue{Hunter, J.\,J.} 1971. 
Further studies on two queues in parallel. 
\textit{Aust. NZ J.~Stat.} 13(2):83--93.


\bibitem{erdelyi} %15
\Aue{Erdelyi, A., and H.~Bateman.} 1985. 
\textit{Higher transcendental functions.} 
Robert E.\ Krieger Publishing Co.  Vol.~II. 414~p.

\end{thebibliography} } }

\end{multicols}

\vspace*{-6pt}

\hfill{\small\textit{Received July 15, 2017}}



\vspace*{-24pt}

\Contr

\vspace*{-2pt}

\noindent
\textbf{Meykhanadzhyan Lusine A.} (b.\ 1990)~---
Candidate of Science (PhD) in physics and mathematics,
 extended education teaching assistant, School No.\,281 of Moscow, 
7~Raduzhnaya Str., Moscow 129344, Russian Federation; 
\mbox{lameykhanadzhyan@gmail.com}

\vspace*{2pt}


\noindent
\textbf{Matyushenko Sergey I.} (b.\ 1963)~--- 
Candidate of Science (PhD) in physics and 
mathematics, associate professor, Peoples' Friendship University of Russia 
(RUDN University), 6~Miklukho-Maklaya Str., Moscow 117198, Russian Federation; 
\mbox{matyushenko\_si@rudn.university}

\vspace*{2pt}


\noindent
\textbf{Pyatkina Daria A.} (b.\ 1968)~--- 
Candidate of Science (PhD) in physics and mathematics, 
associate professor, Peoples' Friendship University of Russia (RUDN University), 
6~Miklukho-Maklaya Str., Moscow 117198, Russian Federation; 
\mbox{pyatkina\_da@rudn.university}

\vspace*{2pt}


\noindent
\textbf{Razumchik Rostislav V.} (b.\ 1984)~--- Candidate of Science (PhD) in physics 
and mathematics, leading scientist, Institute of Informatics Problems, 
Federal Research Center ``Computer Science and Control'' of the Russian 
Academy of Sciences, 44-2~Vavilov Str., Moscow 119333, Russian Federation; 
associate professor, Peoples' Friendship University of Russia (RUDN University), 
6~Miklukho-Maklaya Str., Moscow 117198, Russian Federation; 
\mbox{rrazumchik@ipiran.ru}  %; \mbox{razumchik\_rv@rudn.university} 

\vspace*{6pt}

\hrule

\vspace*{2pt}

\hrule

%\newpage

\vspace*{-2pt}

%\vspace*{8pt}

\def\tit{СОВМЕСТНОЕ СТАЦИОНАРНОЕ РАСПРЕДЕЛЕНИЕ ЧИСЛА ЗАЯВОК В~СИСТЕМЕ С~ДВУМЯ 
ОЧЕРЕДЯМИ КОНЕЧНОЙ ЕМКОСТИ И~ОБЩИМ ВХОДЯЩИМ ПОТОКОМ$^*$}




\def\titkol{Совместное стационарное распределение числа заявок в системе с двумя 
очередями конечной емкости} % и общим входящим потоком}

\def\aut{Л.\,А. Мейханаджян$^{1}$,
С.\,И.~Матюшенко$^2$,
Д.\,А.~Пяткина$^2$, Р.\,В.~Разумчик$^{2,3}$}

\def\autkol{Л.\,А. Мейханаджян, С.\,И.~Матюшенко,
Д.\,А.~Пяткина, Р.\,В.~Разумчик}



{\renewcommand{\thefootnote}{\fnsymbol{footnote}}
\footnotetext[1]{Работа частично поддержана грантами РФФИ
№\,15-07-03007 и~№\,15-07-03406.}}


\titel{\tit}{\aut}{\autkol}{\titkol}

\vspace*{-12pt}

\noindent
$^1$Школа №~281 города Москвы

\noindent
$^2$Российский университет дружбы народов


\noindent
$^3$Институт проблем информатики Федерального исследовательского центра 
<<Информатика и управление>>\linebreak
$\hphantom{^1}$Российской академии наук

\vspace*{6pt}

\def\leftfootline{\small{\textbf{\thepage}
\hfill ИНФОРМАТИКА И ЕЁ ПРИМЕНЕНИЯ\ \ \ том\ 11\ \ \ выпуск\ 3\ \ \ 2017}
}%
 \def\rightfootline{\small{ИНФОРМАТИКА И ЕЁ ПРИМЕНЕНИЯ\ \ \ том\ 11\ \ \ выпуск\ 3\ \ \ 2017
\hfill \textbf{\thepage}}}


\Abst{Рассматривается система массового обслуживания с~входящим пуассоновским 
потоком и~двумя приборами, являющаяся одним из простых вариантов fork-систем. 
Перед каждым прибором имеется накопитель конечной емкости. При поступлении 
в~систему новой заявки создается ее копия и~далее в~каждую 
из очередей поступает по одной заявке. Если в~момент поступления заявки накопитель 
оказывается полностью заполненным, заявка теряется и~в~систему не возвращается. 
Времена обслуживания заявок на приборах имеют экспоненциальное распределение 
с~различными параметрами. Хорошо известно, что подобные системы 
с~трудом поддаются аналитическому анализу. В~работе предлагается метод 
нахождения вероятности блокировки, а~также совместного стационарного 
распределения числа заявок в~накопителях, основанный на методе производящих функций 
и~использующий  некоторые результаты теории специальных функций.}

\KW{система массового обслуживания; fork-система; две очереди; конечная емкость; стационарное распределение}

\DOI{10.14357/19922264170312} 


%\vspace*{6pt}


 \begin{multicols}{2}

\renewcommand{\bibname}{\protect\rmfamily Литература}
%\renewcommand{\bibname}{\large\protect\rm References}

{\small\frenchspacing
{%\baselineskip=10.8pt
\begin{thebibliography}{99}

\bibitem{intro1-1}
\Au{El-hady E., Brzdek~J., Nassar~H.}
On the structure and solutions of functional equations arising from queueing models~// 
{Aequationes Math.}, 2017. Vol.~91. No.\,3. P.~445--477.

\bibitem{intro2-1}
\Au{Kingman J.\,F.\,C.} Two similar queues in parallel~// {Ann. Math. Stat.}, 1961. 
Vol.~32. No.\,4. P.~1314--1323.


\bibitem{orig1-1} %3
\Au{Hunter J.\,J.} Two queues in parallel~// J.~Roy. Stat. Soc.~B, 1969. 
Vol.~31. P.~432--445. %(1969).


\bibitem{orig2-2-1} %4
\Au{Flatto L., Hahn~S.} Two parallel queues created by arrivals with two demands~I~// 
{SIAM J.~Appl. Math.}, 1984. Vol.~44. No.\,5. P.~1041--1053.


\bibitem{orig2-1-2} %5
\Au{Flatto L.} Two parallel queues created by arrivals with two demands~II~// 
{SIAM J.~Appl. Math.}, 1985. Vol.~45. No.\,5. P.~861--878.

\bibitem{NEW1-1} %6
\Au{Rao B.\,M., Posner~M.\,J.\,M.} Algorithmic 
and approzimation analyses of the split and match queue~// 
{Commun. Stat. Stochastic Models}, 1985. Vol.~1. No.\,3. P.~433--\linebreak 456.

\bibitem{lat-ram-1} %7
\Au{Latouche G., Ramaswami~V.} 
{Introduction to matrix analytic methods in stochastic modeling}.~---  
Philadelphia, PA, USA: SIAM, 1999. 334~p.


\bibitem{bocharov-1} %8
\Au{Bocharov P.\,P., D'Apice C., Pechinkin A.\,V., Salerno S.} 
{Queueing theory}.~--- Utrecht: VSP Publishing, 2004.\linebreak 450~p.

\bibitem{xx2-1} %9
\Au{Bavinck H., Hooghiemstra~G., De~Waard~E.} An application of Gegenbauer 
polynomials in queueing theory~// {J.~Comput. Appl. Math.}, 1993. Vol.~49. P.~1--10.

\bibitem{xx1-1} %10
\Au{Avrachenkov K.\,E., Vilchevsky~N.\,O., Shevlyakov~G.\,L.} 
Priority queueing with finite buffer size and randomized push-out mechanism~// 
{ACM  Conference (International) on Measurement and Modeling of Сomputer 
Systems Proceedings}.~--- New York, NY, USA: ACM, 
2003. Vol.~31. No.\,1. P.~324--335.



\bibitem{xx3-1} %11
\Au{Razumchik R.\,V.} Analysis of finite capacity queue with negative customers 
and bunker for ousted customers using Chebyshev and Gegenbauer polynomials~// 
{Asia Pac. J.~Oper. Res.}, 2014. Vol.~31. No.\,4. Id.~1450029.
%1450029 (21 pages)

\bibitem{NEW3-1} %12
\Au{Зарядов И.\,С., Мейханаджян~Л.\,А., Милованова~Т.\,А., Разумчик~Р.\,В.} 
Метод нахождения стационарного распределения очереди в~двухканальной системе 
с~упорядоченным входом конечной емкости~// 
Системы и~средства информатики, 2015. Т.~25. Вып.~3. С.~44--59.

\bibitem{NEW2-1} %13
\Au{Razumchik R.\,V.} Algebraic method for approximating joint stationary 
distribution in finite capacity queue with negative customers and two queues~//
Информатика и~её применения, 2015.  Т.~9. Вып.~4. C.~68--77.

\bibitem{orig2-1-1} %14
\Au{Hunter J.\,J.}
Further studies on two queues in parallel~// 
{Aust. NZ J.~Stat.}, 1971. Vol.~13. No.\,2. P.~83--93.


\bibitem{erdelyi-1} %15
\Au{Erdelyi A., Bateman~H.} {Higher transcendental functions.} 
Robert E.~Krieger Publishing Co., 1985.  Vol.~II. 414~p.




\end{thebibliography}
} }

\end{multicols}

 \label{end\stat}

 \vspace*{-3pt}

\hfill{\small\textit{Поступила в редакцию  15.07.2017}}


%\renewcommand{\bibname}{\protect\rm Литература}
\renewcommand{\figurename}{\protect\bf Рис.}
\renewcommand{\tablename}{\protect\bf Таблица}