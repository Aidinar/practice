\def\stat{zakharova}

\def\tit{СЕГМЕНТИРОВАНИЕ НЕСТАЦИОНАРНЫХ СИГНАЛОВ НА~ОСНОВЕ 
ВЕРОЯТНОСТНЫХ СВОЙСТВ ОКОННОЙ ДИСПЕРСИИ}

\def\titkol{Сегментирование нестационарных сигналов на основе вероятностных свойств оконной дисперсии}

\def\aut{М.\,А.~Драницына$^1$, Т.\,В.~Захарова$^2$}

\def\autkol{М.\,А.~Драницына, Т.\,В.~Захарова}

\titel{\tit}{\aut}{\autkol}{\titkol}

\index{Драницына М.\,А.}
\index{Захарова Т.\,В.}
\index{Dranitsyna M.\,A.}
\index{Zakharova T.\,V.}


%{\renewcommand{\thefootnote}{\fnsymbol{footnote}} \footnotetext[1]
%{Исследование выполнено при финансовой поддержке Российского научного фонда (проект 16-11-10227).}}


\renewcommand{\thefootnote}{\arabic{footnote}}
\footnotetext[1]{Московский государственный университет имени М.\,В.~Ломоносова,
факультет вычислительной математики и~кибернетики; \mbox{margarita13april@mail.ru}}
\footnotetext[2]{Московский государственный университет имени М.\,В.~Ломоносова, 
факультет вычислительной математики и~кибернетики; 
Институт проб\-лем информатики Федерального исследовательского центра <<Информатика 
и~управ\-ле\-ние>> Российской академии наук, \mbox{lsa@cs.msu.ru}}

%\vspace*{-18pt}


\Abst{Выделение фрагментов регистрируемого сигнала, т.\,е.\ 
его сегментация, является актуальной задачей, в~частности для биомедицинской отрасли. 
Сегментация как этап обработки сигналов, зачастую обязательный, может способствовать 
интерпретации и~классификации регистрируемых данных. Особенно сложно сегментировать 
нестационарные сигналы с~малым отношением сиг\-нал/шум. В~рамках данной работы 
основное внимание уделяется изучению шумовой компоненты оконной дисперсии как 
случайной величины в~рассматриваемых моделях. Авторами предложены модели для 
представления мультикомпонентных сигналов, а~также исследованы некоторые 
вероятностные характеристики шумовой компоненты оконной дисперсии сигналов 
как случайного процесса в~представленных моделях. Результаты работы 
согласуются с~установленными эмпирически свойствами шумовой компоненты оконной 
дисперсии (для миограммы). Полученные результаты планируется 
использовать в~практических задачах сегментирования сигналов и~выделения
 интервалов с~преобладанием тех или иных компонент процесса, а~также для 
 прогнозирования поведения сигналов.}

\KW{оконная дисперсия; модель сигнала}

\DOI{10.14357/19922264170302} 

\vspace*{2pt}


\vskip 10pt plus 9pt minus 6pt

\thispagestyle{headings}

\begin{multicols}{2}

\label{st\stat}

\section{Введение}


Выделение фрагментов регистрируемого сигнала с~различными характеристиками, т.\,е.\ 
его сегментация, является актуальной задачей, в~частности для биомедицинской отрасли 
(например, при анализе электроэнцефалограмм~\cite{Kos, Aza}, данных различных 
мониторирующих состояние здоровья устройств~\cite{Kal} и~других моно- 
и~мультикомпонентных сигналов~\cite{Z7}). Сегментация сигналов как этап их обработки, 
зачастую обязательный~\cite{Kal, Z7}, может способствовать интерпретации 
и~классификации регистрируемых данных.

Будем рассматривать некоторый нестационарный сигнал. Этот сигнал может быть 
пред\-став\-лен в~виде временного ряда, который образуют результаты измерения 
сигнала в~точках~$\tau_{k}$, $k\hm =  1, 2, \ldots, r$. При этом сигнал образован 
составляющими его процессами $A_{1}, A_{2}, \cdots, A_{m}$, каждый из которых 
может быть преобладающим на том или ином временн$\acute{\mbox{о}}$м интервале регистрации сигнала.

Примером такого сигнала может служить фармакокинетическая кривая, т.\,е.\ 
кривая, отра\-жа\-ющая зависимость концентрации вещества, чаще всего в~крови, от времени. 
Для действующего вещества лекарственного препарата, пред\-став\-ля\-юще\-го собой таблетку 
или капсулу, профиль такой кривой определяется ско\-ростью абсорбции вещества из 
просвета тонкой кишки, являющейся наиболее час\-тым абсорбирующим органом, и~количеством 
уже абсорбированного вещества, ско\-ростью распределения вещества из 
крови в~периферические ткани (с~достижением динамического равновесия) и~выведением 
его из организма как за счет метаболизма, так и~выделения соответствующими органами.

Ассоциированная с~подлежащими процессами шумовая компонента предполагается 
случайной величи\-ной. Изменение вероятностных характеристик шумовой компоненты 
оконной дисперсии будет основанием для сегментирования ре\-гист\-ри\-ру\-емо\-го 
сигнала в~дальнейшем на практике. Оконная дисперсия для выделения определенных 
участков сигналов была использована, например, для анализа 
магнитоэнцефалограмм~\cite{Z7} с~целью определения момента начала движения. 
Для выделения опорных точек на миограмме, регистрируемой параллельно 
с~магнитоэнцефалограммой, в~работе~\cite{Khazi} была предложена методология, 
которая использована и~обобщена для модели с~общей шумовой компонентой в~настоящей 
работе. Кроме того, распределение шумовой компоненты оконной дисперсии миограммы 
было охарактеризовано эмпирически в~работе~\cite{All}.

В рамках данной работы рассматриваются модели мультикомпонентных сигналов, 
при этом основное внимание уделяется исследованию изменения вероятностных 
характеристик оконной дисперсии сигнала на разных временн$\acute{\mbox{ы}}$х интервалах.

\section{Модель с~общей шумовой компонентой}

\subsection{Общее представление модели}
%\label{1.1.1}

Пусть регистрируется некий сигнал. Представим в~каждой точке~$t$ 
наблюдаемое значение~$C(t)$ в~виде суммы истинных значений процессов,
 формирующих результирующий сигнал, $A_{1}(t), A_{2}(t), \ldots, A_{m}(t)$ и~белого 
 гауссовского шума~$\xi(t)$,
характеризующегося нормальным распределением с~нулевым математическим 
ожиданием и~дисперсией~$\sigma^{2}$. Таким образом, наблюдаемый сигнал~$C(t)$ 
представим в~виде:
\begin{equation}
\label{1}
C(t) = \sum\limits_{l=1}^{m}A_{l}(t) + \xi(t)\,.
\end{equation}

Пусть $n$~--- ширина окна, т.\,е.\ число точек, используемых при расчете 
скользящего среднего~$\bar{C}_n$ сигнала~$C(t)$.
Через $C_{i}$ и~$\xi_{i}$ для $i\hm=0,1,\ldots,n-1$ обозначим соответственно 
значение сигнала и~шума в~$i$-й точке окна.

Во введенных  обозначениях для~$\bar{C}_n$ справедливо следующее представление:
\begin{multline}
\label{2}
\bar{C}_n= \fr{1}{n}\sum\limits_{i=0}^{n-1}C_{i} = \fr{1}{n}
\sum\limits_{i=0}^{n-1}\left[ \sum\limits_{l=1}^{m}A_{l,i} +
 \xi_{i} \right] = {}\\
 {}=\sum\limits_{l=1}^{m}\bar{A}_{l,n} + \bar{\xi}_{n}  \,,
\end{multline}
где $\bar{A}_{1,n}, \bar{A}_{2,n}, \ldots, \bar{A}_{m,n},$~--- 
скользящее среднее истинных составляющих регистрируемого сигнала, а~$\bar{\xi}_{n}$~--- 
скользящее среднее шума.

Оконная дисперсия по определению имеет вид:
\begin{equation}
\label{3}
W_n= \fr{1}{n}\sum\limits_{i=0}^{n-1}\left( C_{i} -\bar{C}_n \right)^2 \,.
\end{equation}

Исследуем свойства оконной дисперсии сигнала и~оконной дисперсии 
шума в~рассматриваемой модели.

\noindent
\textbf{Лемма.} \textit{Для оконной дисперсии~$W_n$ 
справедливо следующее представление}:
\begin{multline}
\label{4}
W_n= {}\\
{}=\sum\limits_{l=1}^{m}W_n^{A_l} + 
\sum\limits_{s\neq l; s, l=1}^{m}\!\!W_n^{A_l,s} + W_n^{\xi} + 
\sum\limits_{l=1}^{m}W_n^{A_l\xi}\,,
\end{multline}
где
\begin{align*}
W_n^{A_l} &= \fr{1}{n}\sum\limits_{i=0}^{n-1}A_{l,i}^{2} + \bar{A}_{l,n}^{2}\,,
\enskip l \in \{1, 2, \ldots, m \}\,;
\\
W_n^{A_l,s} &= \fr{2}{n}\sum\limits_{i=0}^{n-1}A_{l,i}\left( A_{s,i} - 
\bar{A}_{s,n}\right)\,,\\
&\hspace*{23mm} l, s \in \{1, 2, \ldots, m \}\,, l \neq s\,;
\\
W_n^{\xi} &= \fr{1}{n}\sum\limits_{i=0}^{n-1}\xi_{i}^{2} + \bar{\xi}_{n}^{2}\,;
\\
W_n^{A_l\xi} &= \fr{2}{n}\sum\limits_{i=0}^{n-1}\xi_{i}\left( A_{l,i} - 
\bar{A}_{l,n}\right)\,,\enskip l \in \{1, 2, \ldots, m \}\,.
\end{align*}

\noindent
Д\,о\,к\,а\,з\,а\,т\,е\,л\,ь\,с\,т\,в\,о\,.\ \ 
Подставим~(\ref{1}) и~(\ref{2}) в~уравнение~(\ref{3}) и~раскроем скобки, тогда
\begin{multline*}
W_n= \fr{1}{n}\sum\limits_{i=0}^{n-1}\left( \sum\limits_{l=1}^{m}\left( A_{l,i} - 
\bar{A}_{l,n} \right) + \left( \xi_{i} - \bar{\xi}_{n}\right) \right)^{2} ={}\\
{}= \fr{1}{n}\sum\limits_{i=0}^{n-1}\left( A_{1,i} - \bar{A}_{1,n} \right)^{2} + \cdots 
+ \fr{1}{n}\sum\limits_{i=0}^{n-1}\left( A_{l,i} - \bar{A}_{l,n} \right)^{2} + {}\\
{}+
\fr{1}{n}\sum\limits_{i=0}^{n-1}\left( \xi_{i} - \bar{\xi}_{n}\right)^{2} +{}\\
{}+ \sum\limits_{s\neq l; s, l=1}^{m}\left[ \fr{2}{n}\sum\limits_{i=0}^{n-1}\left( 
A_{l,i} - \bar{A}_{l,n}\right) \left( A_{s,i} - \bar{A}_{s,n}\right) \right] + {}\\
{}+
\sum\limits_{ l=1}^{m}\left[ \fr{2}{n}\sum\limits_{i=0}^{n-1}\left( A_{l,i} - 
\bar{A}_{l,n}\right) \left( \xi_{i} - \bar{\xi}_{n}\right) \right] =  {}\\
{}= \sum\limits_{l=1}^{m}W_n^{A_l} + 
\sum\limits_{s\neq l; s, l=1}^{m}W_n^{A_l,s} + W_n^{\xi} + 
\sum\limits_{l=1}^{m}W_n^{A_l\xi} \,.
\end{multline*}

Последнее равенство совпадает с~утверждением леммы.

Такое представление~(\ref{4}) оконной дисперсии может быть интерпретировано 
следующим образом:
\begin{itemize}
    \item
    компоненты $W_n^{A_l}, l \in \{ 1, 2, \ldots, m\} $, характеризуют тренд, 
    обусловленный истинными компонентами регистрируемого сигнала в~отсутствие шума;
    \item
    компоненты $W_n^{A_l,s}, s, l \in \{ 1, 2, \ldots, m\}, s\neq l, $ характеризуют 
    суперпозицию истинных компонент регистрируемого сигнала;
    \item
    компонента $W_n^{\xi}$  характеризует дисперсию случайной компоненты 
    регистрируемого сигнала~--- оконная дисперсия шума;
    \item
    компоненты $W_n^{A_l\xi}$ характеризуют суперпозицию истинных компонент и~шума.
\end{itemize}

Таким образом, оконная дисперсия рассматриваемого сигнала состоит 
из суммы компонент, обусловленных изменением истинного сигнала во времени,
 и~компонент, ассоциированных с~шумом.

Раскроем скобки для компонент $W_n^{A_l}$, $l \hm\in \{ 1, 2, \ldots, m\} $, 
и~получим:
\begin{multline*}$$
W_n^{A_l} = \fr{1}{n}\sum\limits_{i=0}^{n-1}\left( A_{l,i} - \bar{A}_{l,n}\right)^{2} 
= {}\\
{}=\fr{1}{n}\sum\limits_{i=0}^{n-1}\left( A_{l,i}^{2} + \bar{A}_{l,n}^{2} - 
2A_{l,i}\bar{A}_{l,n}\right) ={}
\\
{}= \fr{1}{n}\left[ \sum\limits_{i=0}^{n-1}\left( A_{l,i}^{2} - 2A_{l,i}\bar{A}_{l,n} 
\right) + n\bar{A}_{l,n}^{2} \right] = {}\\
{}=
\fr{1}{n} \sum\limits_{i=0}^{n-1}A_{l,i}^{2} - \bar{A}_{l,n}^{2}\,.
\end{multline*}

Для оконной дисперсии шума $W_n^{\xi}$ справедливы аналогичные 
преобразования и~представление:
\begin{multline*}
\hspace*{-7pt}W_n^{\xi} = \fr{1}{n}\sum\limits_{i=0}^{n-1}\left( \xi_{i} - \bar{\xi}_{n}\right)^{2} = \frac{1}{n}\sum\limits_{i=0}^{n-1}
\left( \xi_{i}^{2} + \bar{\xi}_{n}^{2} - 2\xi_{i}\bar{\xi}_{n}\right) ={}
\\
{}= \fr{1}{n}\left[ \sum\limits_{i=0}^{n-1}\left( \xi_{i}^{2} -
 2\xi_{i}\bar{\xi}_{n} \right) + n\bar{\xi}_{n}^{2} \right] = 
 \fr{1}{n} \sum\limits_{i=0}^{n-1}\xi_{i}^{2} - \bar{\xi}_{n}^{2}\,.
\end{multline*}


Рассмотрим компоненты, представляющие собой суперпозиции истинных 
компонент сигнала~$W_n^{A_l,s}$ для $s, l \hm\in \{ 1, 2, \ldots, m\}$, $s\hm\neq l $, 
тогда
\begin{multline*}
W_n^{A_l,s} =  \fr{2}{n}\sum\limits_{i=0}^{n-1}\left( A_{l,i} - 
\bar{A}_{l,n}\right) \left( A_{s,i} - \bar{A}_{s,n}\right)  ={}
\\
{}= \fr{2}{n}\sum\limits_{i=0}^{n-1}\!A_{s,i}\left( A_{l,i} - \bar{A}_{l,n}\right) - 
 \fr{2\bar{A}_{s,n}}{n}\hspace*{-0.6pt}\sum\limits_{i=0}^{n-1}\!\left( A_{l,i} - \bar{A}_{l,n}\right) = {}\\
 {}=
 \fr{2}{n}\sum\limits_{i=0}^{n-1}A_{s,i}\left( A_{l,i} - \bar{A}_{l,n}\right)\,.
\end{multline*}


Проведем аналогичные преобразования для компонент~$W_n^{A_l\xi}$, 
$l \hm\in \{ 1, 2, \ldots, m\}$, и~получим:

\noindent
\begin{multline*}
W_n^{A_l\xi} =  \fr{2}{n}\sum\limits_{i=0}^{n-1}\left( A_{l,i} - 
\bar{A}_{l,n}\right) \left( \xi_{i} - \bar{\xi}_{n}\right)  ={}
\\
{}= \fr{2}{n}\sum\limits_{i=0}^{n-1}\xi_{i}\left( A_{l,i} - \bar{A}_{l,n}\right) -  
\fr{2\bar{\xi}_{n}}{n}\sum\limits_{i=0}^{n-1}\left( A_{l,i} - \bar{A}_{l,n}\right) = {}\\
{}=
\fr{2}{n}\sum\limits_{i=0}^{n-1}\xi_{i}\left( A_{l,i} - \bar{A}_{l,n}\right).
\end{multline*}

Подставив полученные выражения в~уравнение~(\ref{4}), получим утверждение леммы.


\subsection{Свойства шумовой компоненты оконной дисперсии в~модели 
с~общей~шумовой~компонентой}
%\label{1.1.2}

Обозначим через $W_{n}^{\Xi}$ шумовую компоненту оконной дисперсии 
регистрируемого сигнала, которая представляет собой сумму оконной 
дис\-пер\-сии шума и~суперпозицию шума и~истинных компонент:
\begin{equation}
\label{5}
\begin{matrix}
W_{n}^{\Xi}= W_n^{\xi} + \sum\limits_{l=1}^{m}W_n^{A_l\xi}\,.
\end{matrix}
\end{equation}

\noindent
\textbf{Теорема.}
\textit{В каждой точке наблюдения~$\tau_{k}$ шумовая компонента оконной 
дисперсии~$W_n^{\Xi}$ представима в~\mbox{виде}}:
\begin{equation*}
%\label{6}
W_{n}^{\Xi} = W_n^{\xi}\,,\end{equation*}
 если 
 $
\left( A_{l,i} - \bar{A}_{l,n}\right) = 0 \ \forall\
   l  \hm\in \{ 1, 2, \ldots, m\},\ \forall \ i \hm\in \{ 0, 1, \ldots, n-1\},
$
где $W_n^{\xi}$ имеет распределение:
$$
W_n^{\xi} \sim \Gamma\left( \fr{n}{2\sigma^{2}}, \frac{n-1}{2}\right)\,,
$$
иначе
$$
W_{n}^{\Xi} = W_n^{\xi}  + \sum\limits_{l: \left( A_{l,i} - \bar{A}_{l,n}\right) 
\neq 0}^{m}W_n^{A_l\xi}\,,
$$
если $\exists \ 
l, i : \left( A_{l,i} - \bar{A}_{l,n}\right)\hm \neq 0,$
при этом $W_n^{A_l\xi}$ имеет распределение:
$$
W_n^{A_l\xi} \sim N \left( 0, \fr{4\sigma^{2}}{n}W_n^{A_l}\right)\,.
$$

\noindent
Д\,о\,к\,а\,з\,а\,т\,е\,л\,ь\,с\,т\,в\,о\,.\ \
По определению случайные величины~$\xi_{k} $ являются независимыми одинаково 
распределенными с~нулевым математическим ожиданием и~дисперсией~$\sigma^{2}$. 
Поэтому распределение компоненты  ${nW_n^{\xi}}/{\sigma^{2}}$ 
как оконной дисперсии случайной величины со стандартным нормальным распределением
 является хи-квад\-рат-рас\-пре\-де\-ле\-ни\-ем вида:
$$
\fr{nW_n^{\xi}}{\sigma^{2}} \sim \chi^{2}_{n-1} = 
\Gamma\left( \fr{1}{2}, \fr{n-1}{2}\right)\,. 
$$

Теперь по свойствам гам\-ма-рас\-пре\-де\-ле\-ния получаем:
$$
W_{n}^{\xi} \sim \chi^{2}_{n-1} = \Gamma\left( \fr{n}{2\sigma^{2}}, 
\fr{n-1}{2}\right)\,.
$$

Компоненты, характеризующие суперпозицию шума и~истинных компонент 
сигнала,~$W_{n}^{A_l\xi}$, $l \hm\in \{ 1, 2, \ldots, m\}$, представляют 
собой суммы независимых нормально распределенных случайных величин:
$$
W_n^{A_l\xi} = \fr{2}{n}\sum\limits_{i=0}^{n-1}\xi_{i}\left( A_{l,i} - 
\bar{A}_{l,n}\right)\,,
$$
при этом
$$
\fr{2}{n}\xi_{i}\left( A_{l,i} - \bar{A}_{l,n}\right)  
\sim N \left( 0, \left[\fr{2\sigma}{n}\left( A_{l,i} - \bar{A}_{l,n}\right) \right]^{2} \right)\,.
$$

Вследствие усиленной воспроизводимости нормального распределения сумма таких 
величин по~$i$ будет иметь нормальное распределение вида:
\begin{multline*}
W_n^{A_l\xi} \sim N \left( 0, \fr{4\sigma^{2}}{n^{2}}\sum\limits_{i=0}^{n-1}
\left( A_{l,i} - \bar{A}_{l,n}\right)^{2} \right) = {}\\
{}=
N \left( 0, \fr{4\sigma^{2}}{n}W_n^{A_l}\right)\,.
\end{multline*}

Итак, доказано, что в~тех случаях, когда истинные компоненты, формирующие 
сигнал, не изменяются, т.\,е.\ $ A_{l,i} \hm- \bar{A}_{l,n} \hm= 0$, 
оконная дис\-пер\-сия шума характеризуется гам\-ма-рас\-пре\-де\-ле\-ни\-ем с~параметрами 
формы и~масштаба~$({n-1})/{2}$  и~${n}/({2\sigma^{2}})$ 
соответственно. Если же для ка\-ких-ли\-бо из истинных компонент  
$ A_{l,i} \hm- \bar{A}_{l,n} \hm\neq 0$, тогда шумовая компонента 
оконной дисперсии~$W_{n}^{\Xi}$ пред\-став\-ля\-ет собой сумму зависимых случайных 
величин с~гам\-ма-рас\-пре\-де\-ле\-ни\-ем и~нормально распределенных.

\section{Модель с~несколькими различными шумовыми компонентами}

\subsection{Общее представление модели}

%\label{1.2.1}
Обратимся теперь к~несколько иному пред\-став\-ле\-нию модели, а~именно: 
представим для каждой точки~$\tau_{k}$ значение сигнала~$C$ в~виде 
суммы независимых в~физическом смысле истинных значений нескольких процессов, 
формирующих сигнал~$A_{l}$, $l \hm= 1, 2, \ldots, m$, и~соответствующего каждой 
такой истинной компоненте независимого в~статистическом смысле 
шума $\xi_{1}, \xi_{2}, \ldots, \xi_{m}$, при этом случайная величина~$\xi_{l, k} $ 
характеризуется нормальным распреде\-лением с~нулевым математическим 
ожиданием и~дисперсией~$\sigma^{2}_{l, k}$, $l \hm= 1, 2, \ldots , m$, 
причем для всех точек~$\tau_{k}$ для фиксированной истинной компоненты~$A_{l}$ 
изучаемого сигнала случайные величины~$\xi_{k,l}$~--- 
независимые одинаково распределенные случайные величины. Тогда для любого~$\tau_{k}$ 
сигнал~$C$  представим в~виде:
\begin{equation}
\label{7}
C = \sum\limits_{l=1}^{m}C_{l} = \sum\limits_{l=1}^{m}\left( A_{l}+\xi_{l}\right)\,.
\end{equation}

Обозначим
$$
\bar{C}_{n} = \sum\limits_{l=1}^{m} \bar{C}_{l,n}\,; 
\quad W_{n} = \sum\limits_{l=1}^{m} W_{l,n}\,,
$$
где $\bar{C}_{n}$~--- скользящее среднее регистрируемого сигнала~$C$, 
а~$ W_{n}$~--- оконная дисперсия сигнала.

Далее рассмотрим отдельно компоненты этой суммы~(\ref{7}) 
$C_{l}\hm =  A_{l}\hm+\xi_{l}$. Для каждой компоненты~$C_{l}$ 
в~соответствии с~леммой справедливо разложение~(\ref{4}), поэтому
$$
\bar{C}_{l,n}=\fr{1}{n} \sum\limits_{i=0}^{n-1}C_{l,i} = \fr{1}{n} 
\sum\limits_{i=0}^{n-1}\left( A_{l,i}+\xi_{l,i}\right)\,;
$$

\vspace*{-12pt}

\noindent
\begin{multline*}
W_{l,n}=\fr{1}{n} \sum\limits_{i=0}^{n-1}\left( C_{l,i} - \bar{C}_{l,n}\right)^{2} 
=\fr{1}{n} \sum\limits_{i=0}^{n-1} A_{l,i}^{2}-
\bar{A}_{l,n}^{2} +{}\\
{}+ 
\fr{1}{n} \sum\limits_{i=0}^{n-1} \xi_{l,i}^{2}-\bar{\xi}_{l,n}^{2} + 
\fr{2}{n} \sum\limits_{i=0}^{n-1}\xi_{l,i} \left( A_{l,i}-\bar{A}_{l,n}\right)\,.
\end{multline*}

Зафиксируем $l$ и~далее для наглядности изложения опустим 
этот индекс, т.\,е.\ будем рассматривать лишь одну истинную 
компоненту и~соответст\-ву\-ющую шумовую компоненту.

Рассмотрим шумовую компоненту оконной дисперсии и~представим ее в~специальном виде:
\begin{multline*}
W_{n}^{\Xi}= \fr{1}{n} \sum\limits_{i=0}^{n-1} \xi_{i}^{2}-\bar{\xi}_{n}^{2} + 
\fr{2}{n} \sum\limits_{i=0}^{n-1}\xi_{i} \left( A_{i}-\bar{A}_{n}\right) = {}\\
{}=\fr{1}{n} \sum\limits_{i=0}^{n-1}\left( \xi_{i}^{2} + \xi_{i} 
\left( 2A_{i}-2\bar{A}_{n}\right)\right)  - \bar{\xi}_{n}^{2}\,.
\end{multline*}

Случайная величина $\bar{\xi}_{n}^{2}$ сходится по вероят\-ности к~нулю, а~свойства 
слагаемых
$\xi_{i}^{2} \hm+ \xi_{i} \left( 2A_{i}\hm-2\bar{A}_{n}\right)$
будут описаны в~подразделе ниже.

Частный случай $l\hm=1$ соответствует сигналу с~единственной истинной 
компонентой и~соответствующим шумом, пример такого сигнала рас\-смот\-рен 
в~работах~\cite{Z7, Khazi}.

\subsection{Свойства случайной величины вида $\left( \xi^{2} + a\xi\right)$}
%    \label{1.2.2}

Рассмотрим свойства одного слагаемого шумовой компоненты 
$ \xi_{i}^{2} \hm+ \xi_{i} \left(2 A_{i}\hm-2\bar{A}_{n}\right)$.
Зафиксировав~$i$ и~введя обозначение $a\hm = 2\left( A\hm-\bar{A}\right)$, 
получим случайную величину вида $ \xi^{2} \hm+ a\xi$. Ее 
свойства отражены в~следующей лемме.

\smallskip
    
\noindent
\textbf{Лемма.}\ \textit{Пусть случайная величина~$\xi$ распределена по нормальному 
закону $N\left( 0, \sigma^{2}\right)$, тогда  $\xi^{2}\hm +a \xi$ 
имеет распределение, соответствующее характеристической функции вида}:
 $$
 \varphi_{\xi^{2} + a\xi}\left(t \right)  = 
 \fr{1}{\sqrt{1-2it\sigma^2}}\,
 e^{{a^{2}t^{2}}/\left({4\left(it - {1}/\left({2\sigma^{2}}\right)\right) }\right)}.
$$

\noindent
Д\,о\,к\,а\,з\,а\,т\,е\,л\,ь\,с\,т\,в\,о\,.\ \
 Вычислим характеристическую функцию для $\xi^{2} \hm+a \xi$.

Запишем определение:
\begin{multline*}
    \varphi_{\xi^{2} + a\xi} \left(t \right) = 
    Ee^{it\left( \xi^{2} + a\xi \right) } = 
      \int\limits_{-\infty}^{\infty}e^{it\left( x^{2} + ax \right) } dF\left( x\right) ={}\\
{}    = \int\limits_{-\infty}^{\infty}
e^{it\left( x^{2} + ax \right) } 
\fr{1}{\sqrt{2\pi\sigma^2}}\,
e^{-{x^{2}}/\left({2\sigma^{2}}\right)} dx ={}\\
{}=\fr{1}{\sqrt{2\pi\sigma^2}}
\int\limits_{-\infty}^{\infty}e^{it\left( x^{2} + 
ax \right) }e^{-{x^{2}}/\left({2\sigma^{2}}\right)} \,dx\,,\enskip
t\in R\,.
\end{multline*}
    
Сначала подробно рассмотрим частный случай $a\hm=1$, $\sigma^{2}\hm = 1$, 
который затем обобщим для произвольных параметров~$a$ и~$\sigma^{2}$.

Проведем элементарные преобразования, выделяя полный квадрат и~проводя 
замену переменных, получим:
\begin{multline*}
\varphi_{\xi^{2} + \xi} \left(t \right) = 
\fr{1}{\sqrt{2\pi}}\int\limits_{-\infty}^{\infty}e^{it\left( x^{2} + x \right) }
e^{-{x^{2}}/{2}}\, dx  ={}\\
{}= 
\fr{1}{\sqrt{2\pi}}\int\limits_{-\infty}^{\infty}
e^{-\left( {1}/{2}-it\right) \left( x + {it}/\left({2\left( it - {1}/{2}\right) }\right) 
\right)^{2} }\times{}\\
{}\times e^{-\left( it - {1}/{2}\right) \left( 
{it}/({2\left(it - {1}/{2}\right) })\right)^{2} } \, dx ={}\\
{}= 
\fr{1}{\sqrt{2\pi\left( {1}/{2}-it\right) }}\,
e^{-\left( it - {1}/{2}\right) \left( 
{it}/({2\left(it - {1}/{2}\right) })\right)^{2} }\times{}\\
{}\times 
\int\limits_{-\infty}^{\infty}
e^{-\left( \sqrt{{1}/{2}-it} \left( x + 
{it}/({2\left( it - {1}/{2}\right) }) \right)\right)^{2} }\times{}\\
{}\times d
\left( \sqrt{\fr{1}{2}-it}\,
 \left( x + \fr{it}{2\left( it - {1}/{2}\right) } \right)\right) ={}
\end{multline*}

\noindent
\begin{multline*}
{}= 
\fr{\sqrt{\pi}}{\sqrt{2\pi\left( {1}/{2}-it\right) }}\,
e^{-{\left( it\right)^{2}\left( it - {1}/{2}\right) }/
\left({4\left(it - {1}/{2}\right)^2 }\right)} = {}\\
{}=
\fr{1}{\sqrt{1-2it}}\,e^{{t^{2}}/({4\left(it - {1}/{2}\right) })}\,.
\end{multline*}
    
Теперь проведем аналогичные преобразования для любых~$a$ и~$\sigma^{2}$:
   \begin{multline*}
    \varphi_{\xi^{2} + a\xi} \left(t \right) = 
    \fr{1}{\sqrt{2\pi\sigma^2}}\int\limits_{-\infty}^{\infty}
    e^{it\left( x^{2} + a x \right) }
    e^{-{x^{2}}/\left({2\sigma^{2}}\right)}\, dx = {}\\
    {}=
    \fr{1}{\sigma\sqrt{{1}/{\sigma^{2}}-2it}}\,
    e^{{a^{2}t^{2}}/\left({4\left(it - {1}/\left({2\sigma^{2}}\right)\right) }\right)} = {}\\
    {}=
    \fr{1}{\sqrt{1-2it\sigma^2}}\,e^{{a^{2}t^{2}}/
    \left({4\left(it - {1}/\left({2\sigma^{2}}\right)\right) }\right)}.
\end{multline*}
    
    Лемма доказана.
    
    \smallskip
    
Согласно теореме единственности полученная характеристическая функция 
однозначным образом определяет функцию распределения случайной величины. 
Рассчитаем первые моменты рассматриваемой случайной величины. Результаты 
расчета приведены в~нижеследующей лемме.

\smallskip
    
\noindent
\textbf{Лемма.}
    Случайная величина $\xi^{2} \hm+ a\xi$ имеет сле\-ду\-ющие 
    математическое ожидание и~дисперсию:
$$
    {\sf E}\left( \xi^{2} + a\xi \right) = \sigma^2\,; \enskip 
    {\sf D}\left( \xi^{2} + a\xi \right) = \sigma^6 + 3\sigma^5+4 a^2 \sigma^3.
$$
    
\noindent
Д\,о\,к\,а\,з\,а\,т\,е\,л\,ь\,с\,т\,в\,о\,.\ \
Для вывода формул используем следующее свойство характеристических функций:
\begin{equation*}
{\sf E}\left(\left( \xi^{2} + a\xi\right)^n \right)  =
\fr{\varphi^{(n)}(0)}{i^n }\,.
\end{equation*}

Первая и~вторая производные по~$t$ для функции~$\varphi(t)$ имеют вид:
\begin{multline*} 
    \fr{d}{dt}\,\varphi\left( t\right) =
    \fr{d}{dt}\,\fr{e^{{a^2 t^2}/\left({4(-{1}/\left({2 \sigma^2}\right)+i t)}\right)}}
    {\sqrt{1-2 i t\sigma^2}} = {}\\
    {}=
    \fr{i e^{{a^2 t^2}/\left({4 \left(-{1}/\left({2 \sigma^2}\right)+i t\right)}\right)}}
    {\sigma \left({1}/{\sigma^2}-2 i t\right)^{3/2}}+{}
\\
        {}+
       e^{{a^2t^2}/\left({4 \left(-{1}/\left({2 \sigma^2}\right)+i t\right)}\right)} 
       \left(
    \fr{a^2 t}{2 \left(-{1}/\left({2 \sigma^2}\right)+i t\right)}-{}\right.\\
\left.    {}-
   \fr{i a^2 t^2}{4 \left(-{1}/\left(2 \sigma^2\right)+i t\right)^2}
  \right)
  \Bigg / 
    \left(\sigma\sqrt{\fr{1}{\sigma^2}-2 it}\right)\,;
\end{multline*}

\vspace*{-12pt}
    
\noindent
\begin{multline*}  
    \fr{d^2}{dt^2}\,\varphi\left( t\right) = 
    -\fr{3 e^{{a^2 t^2}/\left({4 \left(-{1}/\left({2 \sigma^2}\right)+i t\right)}\right)}}
    {\sigma \left({1}/{\sigma^2}-2 it\right)^{5/2}}+{}\\
    {}+
    e^{{a^2 t^2}/\left({4 \left(-{1}/\left({2\sigma^2}\right)+it\right)}\right)} 
    \left(
    \fr{a^2}{2 \left(-{1}/\left({2 \sigma^2}\right)+i t\right)}-{}\right.
    \end{multline*}

\noindent
\begin{multline*}
    {}-
    \fr{i a^2 t}{\left(-{1}/\left({2
            \sigma^2}\right)+i t\right){}^2}-{}\\
\left.            {}-
            \fr{a^2 t^2}{2 \left(-{1}/\left({2\sigma^2}\right)+it\right)^3}\right)
            \Bigg/
            \left(\sigma\sqrt{\fr{1}{\sigma^2}-2 it}\right)+{}\\
{}+ 2 i e^{{a^2 t^2}/\left({4 \left(-{1}/\left({2\sigma^2}\right)+it\right)}\right)} 
\left(
\fr{a^2 t}{2 \left(-{1}/\left({2 \sigma^2}\right)+i t\right)}-{}\right.\\
\left.{}-
\fr{i a^2 t^2}{4 \left(-{1}/\left({2 \sigma^2}\right)+it \right)^2}\right)
\!\Bigg/\!
\left(\sigma\left(\fr{1}{\sigma^2}-2 i t\right)^{3/2}\right)+{}\\
{}+
e^{{a^2 t^2}/\left({4 \left(-{1}/\left({2\sigma^2}\right)+i t\right)}\right)} 
\left(
\fr{a^2 t}{2 \left(-{1}/\left({2 \sigma^2}\right)+i t\right)}-{}\right.\\
\left.{}-
\fr{ia^2 t^2}{4 \left(-{1}/\left({2 \sigma^2}\right)+i t\right)^2}\right)^2\Bigg/
\left(\sigma\sqrt{\fr{1}{\sigma^2}-2 i t}\right)\,.
\end{multline*}
   
\noindent
Поэтому
$$  
iE\left( \xi^{2} + a\xi\right)  = \fr{i}{\left({1}/{\sigma^2}\right)^{3/2} \sigma }= 
i \sigma^2
$$  
и
\begin{multline*}
i^2 E\left( \xi^{2} + a\xi\right)^2 =  {}\\
{}=
-\fr{3}{\left({1}/{\sigma^2}\right)^{5/2} \sigma}-
\fr{a^2 \sigma}{\sqrt{{1}/{\sigma^2}}} = i^2\left(3\sigma^4 + a^2\sigma^2\right)\,.
\end{multline*} 

С помощью начальных моментов найдем дисперсию:
\begin{multline*}  
    D\left( \xi^{2} + a\xi \right)  =  E\left( \xi^{2} + a\xi\right)^2 - 
    \left( E\left(\xi^{2} + a\xi \right)\right)^2 ={}\\
    {}= 3\sigma^4 + a^2\sigma^2 - \left(\sigma^2\right)^2 = 2\sigma^4 + a^2\sigma^2 \,.
\end{multline*}

Таким образом, утверждения леммы  доказаны.

\subsection{Свойства шумовой компоненты оконной дисперсии в~модели 
с~несколькими шумовыми компонентами}
%    \label{1.2.3}

    Заменим обратно~$a$ на $ 2A\hm-2\bar{A}$ и~сформулируем 
    теорему о~свойствах шумовой компоненты оконной дисперсии.

\smallskip

\noindent
    \textbf{Теорема.}
\textit{Для регистрируемого сигнала $C\left(t \right)$ в~каж\-дой точке~$\tau_{k}$, 
$k\hm= \left\lbrace  1, 2, \ldots, r\right\rbrace$, шумовая компонента 
оконной дисперсии~$W_{n}^{\Xi}$ представляет собой}:
\begin{enumerate}[($i$)]
    \item \textit{случайную величину} 
    $$
    \sum\limits_{l=1}^{m} W_{l,n}^{\xi}\sim 
    \Gamma\left(\fr{n}{2\sigma_{l}^{2}}, \fr{n-1}{2}\right)\,,
    $$ 
    \textit{если} 
   \begin{multline*}
     A_{l,j} \hm- \bar{A}_{l,n}= \emptyset \ \forall\ 
     l  \hm\in \left\lbrace 1, 2, \ldots, m\right\rbrace ,\\
      \forall\ 
     j \hm\in \left\lbrace 0, 1, \ldots, n-1\right\rbrace\,;
     \end{multline*}
    
    \item \textit{сумму случайных величин} 
    $\sum\nolimits_{l=1}^{m}\sum\nolimits_{j=0}^{n-1}({1}/{n})\times$\linebreak
$\times     \left( \xi_{l,j}^{2}\hm+\xi_{l,j} \left( 2A_{l,j}\hm-2\bar{A}_{l,n}\right)\right)  \hm-
       \sum\nolimits_{l=1}^{m}\bar{\xi}_{l,n}^{2}$, 
       \textit{если} $\exists\ l \hm\in \left\lbrace 1, 2, \ldots, m\right\rbrace$
\textit{и}  $j \hm\in \left\lbrace 0, 1, \ldots, n-1\right\rbrace :
    A_{l,j}\hm - \bar{A}_{n} \hm\neq 0$,
    
    \textit{при этом}
    $$
    \bar{\xi}_{l,n}^{2}  \sim  \Gamma\left( \fr{n}{2\sigma_{l}^{2}}, 
    \fr{1}{2}\right)\,,
    $$ 
    
    \textit{a характеристическая функция случайной величины} 
    $\sum\nolimits_{l=1}^{m}(1/n) \!\sum\nolimits_{j=0}^{n-1}\!\left( \xi_{l,j}^{2} + 
    \xi_{l,j} \left( 2A_{l,j}-\right.\right.$\linebreak 
    $\left.\left.-\;2\bar{A}_{l,n}\right)\!\right)$ 
    \textit{имеет вид} : 
    \begin{multline*}
    \varphi\left( t\right) \hm= \prod\limits_{l=1}^{m} 
    \left( 1-\fr{2it\sigma_l^2}{n}\right)^{-{n}/{2}}\times{}\\
    {}\times  
    \prod\limits_{j=0}^{n-1} e^{{\left(A_{l,j}-\bar{A}_{l,n}\right)^{2}t^{2}
    \sigma^{2}_{l}}/\left({itn\sigma^{2}_{l} - {n^{2}}/2}\right)}\,.
    \end{multline*}
    
\end{enumerate}

\noindent
Д\,о\,к\,а\,з\,а\,т\,е\,л\,ь\,с\,т\,в\,о\,.\ \
        Сначала будем рассматривать случай для фиксированного~$l$ или, 
        иными словами, сигнал, сформированный единственной истинной 
        компонентой и~соответствующим шумом.
    
    В этом случае (см.\ подразд.~3.1) шумовая компонента для оконной дисперсии 
    сигнала имеет вид:
    $$
    W_{n}^{\Xi}= \fr{1}{n} \sum\limits_{j=0}^{n-1}\left( \xi_{j}^{2} + \xi_{j} \left( 2A_{j}-2\bar{A}_{n}\right)\right)  - \bar{\xi}_{n}^{2}\,.
$$

   \noindent
    Так как
\begin{multline*}
 \fr{1}{n} \sum\limits_{j=0}^{n-1}\left( \xi_{j}^{2} + \xi_{j} \left( 2A_{j}-
 2\bar{A}_{n}\right)\right) ={}\\
 {}= \sum\limits_{j=0}^{n-1}\frac{1}{n} 
 \left( \xi_{j}^{2} + \xi_{j} \left( 2A_{j}-2\bar{A}_{n}\right)\right)\,, 
 \end{multline*}
то, используя свойства характеристической функции, получим характеристическую 
функцию для случайной величины $({1}/{n}) \left( \xi_{j}^{2} \hm+ 
\xi_{j} \left( 2A_{j}\hm-2\bar{A}_{n}\right)\right)$ при каждом фиксированном~$j$:
\begin{multline*}
\varphi_{({1}/{n}) \left( \xi_{j}^{2} + \xi_{j} \left( 2A_{j}-2\bar{A}_{n}\right)
\right) }\left( t\right)  = {}\\
{}=
\varphi_{  \xi_{j}^{2} + \xi_{j} \left( 2A_{j}-2\bar{A}_{n}\right) } 
\left( \fr{t}{n} \right) = {}\\
{}=
\fr{1}{\sqrt{1-{2it\sigma^2}/{n}}}\,
e^{{\left( 2A_{j}-2\bar{A}_{n}\right)^{2}t^{2}}/
\left({4\left(i\,t\,n - {n^2}/\left({2\sigma^{2}}\right)\right) }\right)}\,.
\end{multline*}
    
   \noindent
    Тогда характеристическая функция для случайной величины  
    $({1}/{n}) \sum\nolimits_{j=0}^{n-1}\left( \xi_{j}^{2}\hm + 
    \xi_{j} \left( 2A_{j}\hm-2\bar{A}_{n}\right)\right)$ может быть записана в~виде:
\begin{multline*}
    \varphi_{({1}/{n}) \sum\nolimits_{j=0}^{n-1}\left( \xi_{j}^{2} + \xi_{j} 
    \left( 2A_{j}-2\bar{A}_{n}\right)\right) }\left( t\right)  = {}\\
    {}=
    \prod\limits_{j=0}^{n-1}\varphi_{({1}/{n})\left( 
    \xi_{j}^{2} + \xi_{j} \left( 2A_{j}-2\bar{A}_{n}\right)\right) } = 
    \fr{1}{\left( 1-{2it\sigma^2}/{n}\right)^{{n}/{2}} } \times{}\hspace*{-6pt}\\
    {}\times
\prod\limits_{j=0}^{n-1} 
e^{{\left( 2A_{j}-2\bar{A}_{n}\right)^{2}t^{2}}/
\left({4\left(i\,t\,n - {n^2}/\left({2\sigma^{2}}\right)\right) }\right)} = {}\\
{}= 
\left( 1-\fr{2it\sigma^2}{n}\right)^{-{n}/{2}} \times{}\\
{}\times \prod\limits_{j=0}^{n-1} 
e^{{\left( 2A_{j}-2\bar{A}_{n}\right)^{2}t^{2}}/
\left({4\left(\,i\,t\,n - {n^2}/\left({2\sigma^{2}}\right)\right) }\right)} = {}\\
{}=
\left( 1-\fr{2it\sigma^2}{n}\right)^{-{n}/{2}}  
\prod\limits_{j=0}^{n-1} e^{{\left( A_{j}-\bar{A}_{n}\right)^{2}t^{2} \sigma^{2} }/
\left({itn\sigma^2 - {n^2}/{2} }\right) }\,.\hspace*{-6pt}
\end{multline*}
    
    Рассмотрим теперь компоненту шумовой со\-став\-ля\-ющей сигнала $\bar{\xi}_{n}^{2}$, 
    которая представляет собой квадрат нормально распределенной случайной 
    величины~$\bar{\xi}_{n}$ с~математическим ожиданием~0 
    и~дис\-пер\-си\-ей~${\sigma^{2}}/{n}$. Поскольку
$$
    \fr{\sqrt{n}}{\sigma}\,\bar{\xi} \sim N \left( 0, 1\right)\,,
$$
то
$$
    \fr{n}{\sigma^{2}}\,\bar{\xi}_{n}^{2}  \sim \chi^{2}_{1} = 
    \Gamma\left( \fr{1}{2}, \fr{1}{2}\right)
    $$
    и
    $$ 
    \bar{\xi}_{n}^{2}  \sim  \Gamma\left( \fr{n}{2\sigma^{2}}, \fr{1}{2}\right).
$$
    
    Отметим, что математическое ожидание~$\bar{\xi}_{n}^{2}$ есть~${\sigma^{2}}/{n}$, 
    а~дисперсия равна ${2\sigma^{4}}/{n^{2}}$ и~эти характеристики зависят от числа 
    точек расчета оконной дисперсии~$n$, убывая  как~$n$ и~$n^{2}$ соответственно. 
    Отсюда следует, что можно выбрать такое~$n$, что среднее 
    и~дисперсия~$\bar{\xi}_{n}^{2}$ будут меньше заданной точности измерений.
    
    Принимая во внимание результаты подразд.~2.2, для каждого фиксированного~$l$ шумовую 
    компоненту сигнала можно представить в~виде:
   \begin{enumerate}[($i$)]
        \item случайной величины
        $$
        W_n^{\xi} = \fr{1}{n} 
        \sum\limits_{j=0}^{n-1}\xi_{j}^{2} - \bar{\xi}_{n}^{2} 
        \sim \Gamma\left( \fr{n}{2\sigma^{2}}, \fr{n-1}{2}\right)\,, 
        $$
        если $ A_{j} \hm- \bar{A}_{n} \hm= 0 \ \forall \ j \hm\in 
        \left\lbrace  0, 1, \ldots, n-1\right\rbrace$;
        \item суммы случайных величин $({1}/{n}) \sum\nolimits_{j=0}^{n-1}
        \left( \xi_{j}^{2} +\right.$\linebreak
        $\left.+\;\xi_{j} \left( 2A_{j}-2\bar{A}_{n}\right)\right)  \hm- 
        \bar{\xi}_{n}^{2}$, если $\exists \ j \hm\in \left\lbrace 0, 1, \ldots, 
        n-1\right\rbrace : \left( A_{j} \hm- \bar{A}_{n}\right)\hm \neq 0$, при этом
        $$
        \bar{\xi}_{n}^{2}  \sim  \Gamma\left(\fr{n}{2\sigma^{2}}, \,
\fr{1}{2}\right)\,;
        $$ 
        
        \vspace*{-12pt}
        
        \noindent
        \begin{multline*}
        \varphi_{({1}/{n}) \sum\nolimits_{j=0}^{n-1}\left( \xi_{j}^{2} + 
        \xi_{j} \left( 2A_{j}-2\bar{A}_{n}\right)\right) }\left( t\right)  
        ={}\\
        \hspace*{-30pt}{}= \left( 1-\fr{2it\sigma^2}{n}\right)^{-{n}/{2}}  
        \prod\limits_{j=0}^{n-1} 
        e^{{\left(A_{j}-\bar{A}_{n}\right)^{2}t^{2}\sigma^{2}}/\left({i  tn\sigma^{2} - 
        {n^{2}}/{2} }\right)}.\hspace*{-3.77995pt}
        \end{multline*}
            \end{enumerate}


Регистрируемый сигнал~$C$ представляет собой сумму независимых истинных компонент 
и~соответствующих независимых шумовых со\-став\-ля\-ющих, поэтому
\begin{multline*}
    W_{l,n}^{\Xi} = {}\\
    {}= \sum\limits_{l=1}^{m}\left[ \fr{1}{n} 
    \sum\limits_{j=0}^{n-1}\left( \xi_{l,j}^{2} + \xi_{l,j} \left( 2A_{l,j}-
    2\bar{A}_{l,n}\right)\right)  - \bar{\xi}_{l,n}^{2}\right] ={}
\\
{}    = \sum\limits_{l=1}^{m}\sum\limits_{j=0}^{n-1}\frac{1}{n} \left( \xi_{l,j}^{2} + \xi_{l,j} \left( 2A_{l,j}-2\bar{A}_{l,n}\right)\right)  -  \sum\limits_{l=1}^{m}\bar{\xi}_{l,n}^{2}.
\end{multline*}
    
    Рассмотрим случай, когда $\left( A_{l,j} \hm- \bar{A}_{l,n}\right)\hm = 0 
    \ \forall \   l  \hm\in \{ 1, 2, \ldots, m\}, \ \forall \ 
    j \hm\in \{ 0, 1, \ldots, n-1\} $, т.\,е.\ 
    истинные компоненты сигнала не меняются на фиксированном окне. Тогда
$$
    W_{l,n}^{\Xi} =  \sum\limits_{l=1}^{m}\sum\limits_{j=0}^{n-1}\fr{1}{n} 
    \xi_{l,j}^{2}  -  \sum\limits_{l=1}^{m}\bar{\xi}_{l,n}^{2}  = 
    \sum\limits_{l=1}^{m} W_{l,n}^{\xi}\,.
$$
    
    Таким образом, оконная дисперсия шумовой составляющей сигнала в~данном 
    случае представляет собой сумму гам\-ма-рас\-пре\-де\-лен\-ных случайных величин
$$
W_{l,n}^{\xi}  \sim  \Gamma\left( \fr{n}{2\sigma_{l}^{2}}, \fr{n-1}{2}\right).
$$

    Рассмотрим случай, когда существуют $l \hm\in 
    \{ 1, 2, \ldots, m\}$  и~$j \hm\in \{ 0, 1, \ldots, n-1\}$, 
    при которых $A_{l,j} \hm- \bar{A}_{l,n} \hm\neq 0$. Тогда
 оконная дисперсия шума представляет разность зависимых сумм случайных величин. 
 При этом
функция распределения случайной величины $\sum\nolimits_{l=1}^{m}\sum\nolimits_{j=0}^{n-1}
({1}/{n}) \left( \xi_{l,j}^{2} \hm+ \xi_{l,j} \left( 2A_{l,j}\hm-
2\bar{A}_{l,n}\right)\right)$ соответствует характеристической функции
\begin{multline*}
    \varphi(t)  = \prod\limits_{l=1}^{m}\varphi_{({1}/{n})
     \sum\nolimits_{j=0}^{n-1}\left( \xi_{l,j}^{2} + \xi_{l,j} \left( 
     2A_{l,j}-2\bar{A}_{l,n}\right)\right)}(t) ={}
\\
  {}  =   \prod\limits_{l=1}^{m} \left( 1-\fr{2it\sigma_l^2}{n}\right)^{-{n}/{2}}\times{}\\
  {}\times  
  \prod\limits_{j=0}^{n-1} 
  e^{{\left(A_{l,j}-\bar{A}_{l,n}\right)^{2}t^{2}\sigma^{2}_{l}}/
  \left({i  t n\sigma^{2}_{l} - {n^{2}}/{2} }\right)}.
\end{multline*}

Очевидно, что случайная величина $\sum\nolimits_{l=1}^{m}\bar{\xi}_{l,n}^{2} $ 
представляет собой сумму независимых гам\-ма-рас\-пре\-де\-лен\-ных величин 
с~параметрами формы~${1}/{2}$ и,~вообще говоря, различными параметрами 
масштаба~${n}/({2\sigma_{l}^{2})}$.

   
Теорема доказана.
    
\smallskip

Отметим также, что так как случайная величи-\linebreak на $\bar{\xi}_{l,n}^{2}$ 
неотрицательна, то
    на практике величина\linebreak $ \sum\nolimits_{l=1}^{m}\sum\nolimits_{i=0}^{n-1}({1}/{n})
     \left( \xi_{l,i}^{2} \hm+ \xi_{l,i} \left( 2A_{l,i}-2\bar{A}_{l,n}\right)\right)$ 
     может служить верхней оценкой шумовой компоненты оконной дисперсии~$W_{l,n}^{\Xi} $ 
     регистрируемого сигнала~$C$.

\section{Заключение}

В рамках работы предложены модели для представления сигналов в~виде суммы нескольких 
подлежащих процессов, а~также исследованы некоторые вероятностные характеристики 
оконной диспер\-сии сигналов как случайных процессов в~представленных моделях. 
Результаты работы согласуются с~установленными эмпирически свойствами шумовой 
компоненты оконной дисперсии миограммы~\cite{All}.  В~работе продемонстрировано, 
что на миограмме в~период покоя, т.\,е.\ в~отсутствие полезных компонент сигнала, 
оконная дисперсия характеризуется гам\-ма-рас\-пре\-де\-ле\-нием.

Полученные результаты планируется использовать в~практических задачах 
сегментирования сигналов и~выделения интервалов с~преобладанием тех или 
иных подлежащих процессов. Кроме того, вероятностные характеристики шумовой 
компоненты могут использоваться для прогнозирования поведения сигнала. 
В~частности, предполагается\linebreak приме\-нить эти теоретические результаты для анализа 
фармакокинетических данных.


   {\small\frenchspacing
 {%\baselineskip=10.8pt
 \addcontentsline{toc}{section}{References}
 \begin{thebibliography}{9}
    \bibitem{Kos}
    \Au{Kosar K., Lhotsk$\acute{\mbox{a}}$ L., Krajca~V.} 
    Classification of long-term EEG recordings~//  Biological
    and medical data analysis.~---
    Lecture notes in computer science ser.~--- Springer, 2004. Vol.~3337. P.~322--332.
    doi: 10.1007/978-3-540-30547-7\_33.
    
    \bibitem{Aza}
    \Au{Azami H., Mohammadi~K., Hassanpour~H.}  
    A~hybrid evolutionary approach to segmentation of nonstationary signals~// 
    Digit. Signal Process., 2013. Vol.~23. No.\,4. P.~1103--1114.
    doi: 10.1016/j.dsp.2013.02.019.
    
    \bibitem{Kal}
    \Au{Kalantarian H., Sarrafzadeh~M.} 
    Probabilistic time-series segmentation~// Pervasive Mob. Comput., 
 2017. doi: 10.1016/j.pmcj.2017.03.005.
    
    \bibitem{Z7}
    \Au{Захарова Т.\,В., Никифоров~С.\,Ю., Гончаренко~М.\,Б., Драницына~М.\,А., 
    Климов~Г.\,А., Хазиахметов~М.\,Ш., Чаянов~Н.\,В.} 
    Методы обработки сигналов для локализации невосполнимых областей головного мозга~// 
    Системы и~средства информатики, 2012. T.~22. №\,2. C.~157--175.
    
    \bibitem{Khazi}
    \Au{Хазиахметов М.\,Ш.} Свойства оконной дисперсии миограммы как 
    случайного процесса~// Системы и~средства информатики, 2014. T.~24. №\,3. C.~110--120.
    
    \bibitem{All}
    \Au{Allakhverdieva V.\,M., Chshenyavskaya~E.\,V.,  Dranitsyna~M.\,A., 
    Karpov~P.\,I., Zakharova~T.\,V.} An 
    approach to the inverse problem of brain functional mapping under the assumption of gamma distributed myogram noise within rest intervals using the independent component analysis~// 
    J.~Math. Sci., 2016. Vol.~214. No.\,1. P.~3--11. doi: 10.1007/s10958-016-2753-x.
   \end{thebibliography}

 }
 }

\end{multicols}

%\vspace*{-3pt}

\hfill{\small\textit{Поступила в~редакцию 19.04.17}}

\vspace*{10pt}

%\newpage

%\vspace*{-24pt}

\hrule

\vspace*{2pt}

\hrule

%\vspace*{8pt}


\def\tit{SEGMENTATION OF NONSTATIONARY SIGNALS USING STOCHASTIC CHARACTERISTICS 
OF~THE~WINDOW VARIANCE}

\def\titkol{Segmentation of nonstationary signals using stochastic characteristics 
of~the~window variance}

\def\aut{M.\,A.~Dranitsyna$^1$ and~T.\,V.~Zakharova$^{1,2}$}

\def\autkol{M.\,A.~Dranitsyna and~T.\,V.~Zakharova}

\titel{\tit}{\aut}{\autkol}{\titkol}

\vspace*{-9pt}


\noindent
$^1$Department of Mathematical Statistics, Faculty of Computational 
Mathematics and Cybernetics, M.\,V.~Lo-\linebreak
$\hphantom{^1}$monosov Moscow State University, 
1-52~Leninskiye Gory, GSP-1, Moscow 119991, Russian Federation

\noindent
$^2$Institute of Informatics Problems, Federal Research Center ``Computer 
Science and Control'' of the Russian\linebreak
$\hphantom{^1}$Academy of Sciences, 44-2~Vavilov Str., 
Moscow 119333, Russian Federation


\def\leftfootline{\small{\textbf{\thepage}
\hfill INFORMATIKA I EE PRIMENENIYA~--- INFORMATICS AND
APPLICATIONS\ \ \ 2017\ \ \ volume~11\ \ \ issue\ 3}
}%
 \def\rightfootline{\small{INFORMATIKA I EE PRIMENENIYA~---
INFORMATICS AND APPLICATIONS\ \ \ 2017\ \ \ volume~11\ \ \ issue\ 3
\hfill \textbf{\thepage}}}

\vspace*{6pt}
    

\Abste{Signal or response partitioning (i.\,e., signal segmentation) 
is of great interest, e.\,g., for biomedical research. 
Signal segmentation, being an essential part of signal processing,
 may serve as a~tool for advanced signal interpretation and data classification. 
 Segmentation of nonstationary signals with a~small signal-to-noise ratio 
 is a~particulary complicated task. 
 The paper is mainly devoted to exploration of the window variance noise component 
 as a~random variable for the proposed signal models. 
 Some stochastic characteristics of the window variance noise\linebreak\vspace*{-12pt}}
 
 \Abstend{components 
 are investigated in accordance with the models. 
 Theoretical findings are consistent with the previously obtained empirical characteristics 
 of the window variance noise component and are supposed to be of potential use for signal segmentation 
 and prediction.}

\KWE{window variance; signal model}

\DOI{10.14357/19922264170302} 

%\vspace*{-18pt}

%\Ack
%\noindent


%\vspace*{3pt}

  \begin{multicols}{2}

\renewcommand{\bibname}{\protect\rmfamily References}
%\renewcommand{\bibname}{\large\protect\rm References}

{\small\frenchspacing
 {%\baselineskip=10.8pt
 \addcontentsline{toc}{section}{References}
 \begin{thebibliography}{9}


\bibitem{1-za}
\Aue{Kosar, K., L.~Lhotska, and V.~Krajca.} 2004. 
Classification of long-term EEG recordings. 
\textit{ Biological
    and medical data analysis}.
{Lecture notes in computer science ser.} 3337:322-332. 
doi: 10.1007/978-3-540-30547-7\_33.

\bibitem{2-za}
\Aue{Azami, H., K.~Mohammadi, and H.~Hassanpour.} 2013. 
A~hybrid evolutionary approach to segmentation of nonstationary signals. 
\textit{Digit. Signal Process.} 23(4):1103-1114. doi: 10.1016/j.dsp.2013.02.019.

\bibitem{3-zh}
\Aue{Kalantarian, H., and M.~Sarrafzadeh.} 2017. 
Probabilistic time-series segmentation. 
\textit{Pervasive Mob. Comput.} doi: 10.1016/j.pmcj.2017.03.005.

\bibitem{4-zh}
\Aue{Zakharova, T.\,V., S.\,Yu.~Nikiforov, M.\,B.~Goncharenko, M.\,A.~Dranitsyna, 
G.\,A.~Klimov, M.\,Sh.~Khaziakhmetov, and N.\,V.~Chayanov.} 
2012. Metody obrabotki signalov dlya lokalizatsii nevospolnimykh oblastey
 golovnogo mozga [Signal processing methods for localization of nonrenewable 
 brain regions]. \textit{Sistemy i~Sredstva Informatiki~--- 
 Systems and Means of Informatics} 22(2):157--175.

\bibitem{5-zh}
\Aue{Khaziakhmetov, M.\,Sh.} 2014. Svoystva okonnoy dispersii miogrammy kak
 sluchaynogo protsessa [Properties of window dispersion of myogram as 
 a~stochastic process]. \textit{Sistemy i~Sredstva Informatiki~---
 Systems and Means of Informatics} 24(3):110--120.

\bibitem{6-zh}
\Aue{Allakhverdieva, V.\,M., E.\,V.~Chshenyavskaya, M.\,A.~Dranitsyna, 
P.\,I.~Karpov, and T.\,V.~Zakharova.} 2016. 
An approach to the inverse problem of brain functional mapping under the assumption of gamma distributed myogram noise within rest intervals using the independent component analysis. 
\textit{J.~Math Sci.} 214(1):3--11. doi: 10.1007/s10958-016-2753-x.
\end{thebibliography}

 }
 }

\end{multicols}

\vspace*{-3pt}

\hfill{\small\textit{Received April 19, 2017}}


\Contr

\noindent
\textbf{Dranitsyna Margarita A.} (b.\ 1983)~---
PhD student, Department of Mathematical Statistics, Faculty of Computational 
Mathematics and Cybernetics, M.\,V.~Lomonosov Moscow State University, 
1-52~Leninskiye Gory, GSP-1, Moscow 119991, Russian Federation; 
\mbox{margarita13april@mail.ru}

\vspace*{3pt}

\noindent
\textbf{Zakharova Tatiana V.} (b.\ 1962)~---
Candidate of Science (PhD) in physics and mathematics, associate professor, 
Department of Mathematical Statistics, Faculty of Computational Mathematics 
and Cybernetics, M.\,V.~Lomonosov Moscow State University, 1-52~Leninskiye 
Gory, GSP-1, Moscow 119991, Russian Federation; senior scientist, 
Institute of Informatics Problems, Federal Research Center ``Computer 
Science and Control'' of the Russian Academy of Sciences, 44-2~Vavilov Str., 
Moscow 119333, Russian Federation; \mbox{lsa@cs.msu.ru}

\label{end\stat}


\renewcommand{\bibname}{\protect\rm Литература} 