\def\stat{gudkova}

\def\tit{ВЕРОЯТНОСТНАЯ МОДЕЛЬ СОВМЕСТНОГО ИСПОЛЬЗОВАНИЯ РЕСУРСОВ 
БЕСПРОВОДНОЙ СЕТИ С~АДАПТИВНЫМ УПРАВЛЕНИЕМ МОЩНОСТЬЮ$^*$}

\def\titkol{Вероятностная модель совместного использования ресурсов 
беспроводной сети с~адаптивным управлением} % мощностью}

\def\aut{И.\,А.~Гудкова$^1$, С.\,Я.~Шоргин$^2$}

\def\autkol{И.\,А.~Гудкова, С.\,Я.~Шоргин}

\titel{\tit}{\aut}{\autkol}{\titkol}

\index{Гудкова И.\,А.}
\index{Шоргин С.\,Я.}
\index{Gudkova I.\,A.}
\index{Shorgin S.\,Ya.}


{\renewcommand{\thefootnote}{\fnsymbol{footnote}} \footnotetext[1]
{Исследование выполнено при финансовой поддержке Российского научного фонда (проект 16-11-10227).}}


\renewcommand{\thefootnote}{\arabic{footnote}}
\footnotetext[1]{Российский университет дружбы народов;
 Институт проблем информатики Федерального исследовательского 
центра <<Информатика и~управление>> Российской академии наук, \mbox{gudkova\_ia@rudn.university}}

\footnotetext[2]{Институт проблем информатики Федерального исследовательского центра <<Информатика 
и~управление>> Российской академии наук, \mbox{sshorgin@ipiran.ru}}

\vspace*{18pt}



\Abst{Развивающиеся беспроводные сети последующего поколения 
(next generation 
network, NGN)
предполагают новые 
приложения и~услуги как для обычных пользователей, так и~для устройств межмашинного 
взаимодействия (machine-to-machine, M2M). Решение проблемы увеличения требований 
к~пропускной способности сети и~недостаточности спектра радиочастот, в~частности 
в~случае умных городов, может быть достигнуто посредством концепции совместного 
использования радиочастот (licensed shared access, LSA). Авторы предлагают 
математическую модель совместного использования ресурсов с~адаптивным управ\-ле\-ни\-ем 
мощностью. Заложенный в~ней алгоритм позволит избежать интерференции M2M-устройств с~владельцем спектра, в~том числе благодаря тому, что учитывает пространственное 
расположение устройств и~их сессионную активность.}
 
\KW{беспроводная сеть; умный город; межмашинное взаимодействие; совместное 
использование радиочастот; адаптивное управление мощностью; случайный процесс; 
рекуррентный алгоритм; вероятность блокировки; вероятность прерывания обслуживания; 
среднее число устройств}

\DOI{10.14357/19922264170310} 

\vspace*{6pt}


\vskip 10pt plus 9pt minus 6pt

\thispagestyle{headings}

\begin{multicols}{2}

\label{st\stat}

\section{Введение}

  Согласно прогнозам развития сетей по\-сле\-ду\-юще\-го поколения, 
  уже в~2025~г.\ беспроводные сети будут перегружены~[1], что 
повлечет за собой необходимость уточнения и~разработки новых стратегий 
использования спектра радиочастот~[2]. Широкое распространение получают 
автономно функционирующие и~взаимодействующие друг с~другом  
(М2М) недорогие устройства, являющиеся неотъемлемой 
частью <<умных городов>> (smart city). Особенностью M2M-устройств 
является их дистанционное управ\-ле\-ние и~высокая плот\-ность расположения. 
Рост числа M2M-устройств существенно сказывается на использовании спектра 
ра\-дио\-час\-тот ввиду того, что сети изначально разрабатывались для 
взаимодействия между людьми (human-to-human, H2H). 
{\looseness=1

}

Один из вариантов 
решения проб\-ле\-мы~--- это динамическое управление спектром в~рамках 
концепции совместного использования радиочастот 
(LSA)~[3--5]. Доступ к~спектру получают две стороны~--- владелец и~временный 
пользователь~[6, 7]. 

  В статье исследуется один из сценариев применения системы LSA~[8--11], 
где владелец запрашивает спектр радиочастот изредка на непродолжительное 
время. В~остальное же время спектр доступен M2M-устрой\-ст\-вам для 
передачи данных. Статья имеет следующую структуру. 

В~разд.~2 описана 
системная модель совместного использования радиоресурсов с~учетом 
расположения устройств на разном расстоянии от базовой станции~[12--15]. 

В~разд.~3 проводится построение математической модели в~виде двух 
случайных процессов (СП), один из которых фиксирует уровень качества 
канала каж\-до\-го из активных устройств, а~второй, укрупненный,~--- только 
суммарное число устройств. Для СП с~укрупненными со\-сто\-яни\-ями представлен 
рекуррентный алгоритм расчета стационарного распределения вероятностей. 

В~разд.~4 предложены формулы для расчета ключевых показателей 
эффективности системы~--- среднего числа устройств и~вероятностей 
блокировки и~прерывания обслуживания, приведен пример численного анализа.

\section{Системная модель совместного использования ресурсов 
с~разноудаленными от базовой станции устройствами}

Рассмотрим одну соту беспроводной сети радиуса~$R$ с~равномерно 
распределенными по зоне покрытия M2M-устройствами (рис.~1). Устройства 
с~интенсивностью~$\lambda$ переходят в~активное состояние и~передают 
данные в~восходящем канале. Время передачи данных одним устройством 
распределено экспоненциально с~параметром~$\mu$. Каждому устройству 
в~зависимости от дальности расположения от базовой станции (БС) 
присваивается один из пятнадцати уровней качества канала (channel quality 
indicator, CQI)~--- $c\hm=1,\ldots ,15$, причем чем больше~$c$, тем ближе 
устройство к~БС и~выше скорость передачи данных. Объединим устройства 
с~одинаковыми уровнями CQI в~логические группы, тогда скорость передачи 
для всех устройств в~группе будет одинаковая. Далее под расстоянием от 
устройства до БС будем понимать максимально возможное расстояние, на 
котором может быть расположено устройство с~таким же уровнем CQI. Введем 
дополнительное обозначение: $\eta\hm=16-c$; уровень CQI~$c$, 
величина~$\eta$ и~расстояние~$\xi_d(\eta)=RL^{-1}\eta$ от устройства до БС 
являются случайными величинами (СВ). Плотность расстояния от устройства 
до БС $$
f_{\xi_d(\eta)}(d)=\fr{2d}{R^2}\,,
$$
 а~функция распределения (ФР)
$$
F_{\xi_d(\eta)}(d)=\left(\fr{d}{R}\right)^2,\enskip 0\leq d\leq R\,.
$$


 { \begin{center}  %fig1
 \vspace*{5pt}
 \mbox{%
\epsfxsize=72mm %.723mm
\epsfbox{gud-1.eps}
}


\vspace*{4pt}


\noindent
{{\figurename~1}\ \ \small{Пример расположения M2M-устройств в~соте}}
\end{center}
}


\addtocounter{figure}{1}

 \noindent
 Ряд распределения для 
параметра~$\eta$:  
$$
q_l=\fr{2L-2l-1}{L^2}\,,\enskip l=1,\ldots ,L\,.
$$

  
  В качестве примера реализации системы LSA рассмотрим случай 
использования владельцем спектра радиочастот для воздушной телеметрии. 
Предположим, что время, в~течение которого владелец (аэропорт) не 
использует спектр, т.\,е.\ время, когда полоса доступна для M2M-устройств, 
и~время пролета самолета над сотой, т.\,е.\ время, когда полоса недоступна для 
устройств, распределены по экспоненциальному закону с~параметрами~$\alpha$ 
и~$\beta$ соответственно.
  
  Управление радиоресурсами предполагает разделение ресурсов по времени, 
т.\,е.\ деление ширины полосы радиочастот~$\omega$ не происходит, 
а~передача данных осуществляется на постоянной мощности. Если полоса не 
требуется аэропорту, то мощность составляет~$p_1^{\max}$, в~противном 
случае для регулирования интерференции мощность снижается до значения 
$p_0^{\max}\hm<p_1^{\max}$. Такое динамическое изменение мощности 
приводит к~изменению достижимой скорости передачи данных $r\left( 
\xi_{d(\eta)},p_s^{\max}\right)$, $s\hm=0,1$, зависящей также от расстояния 
между устройством и~БС. Согласно формуле Шеннона,
  \begin{multline}
  r\left( \xi_{d(\eta)}, p_s^{\max}\right) =\omega \ln\left( 
1+\fr{Gp_s^{\max}}{((R/L)\eta)^\kappa N_0}\right)\,,\\ s=0,1\,,\enskip 
\eta=1,\ldots ,15\,,
  \label{e1-gud}
  \end{multline}
где $N_0$~---  уровень шума; $G$~--- константа затухания сигнала; $\kappa$~--- 
экспонента затухания сигнала.

  Скорость передачи данных каждым активным M2M-устройством не может 
быть ниже порогового (гарантированного) значения~$r_0$. Если устройству не 
может быть обеспечена скорость~$r_0$, то запрос на передачу данных будет 
заблокирован. Если устройство расположено в~непосредственной близости от 
БС, то скорость передачи согласно формуле Шеннона стремится 
к~бесконечности, поэтому определим минимальное расстояние до БС 
$\xi_d(1)\hm=d_0$, ограничив тем самым максимальную скорость передачи 
данных $r_s^{\max}\hm= r\left( d_0, p_s^{\max}\right)$. Таким образом, если 
$\eta\hm=1$, то достижимая скорость передачи данных $r\left( \xi_{d(\eta)}; 
p_s^{\max}\right)$, если $\eta\hm=2,\ldots ,L$, то она вычисляется по 
формуле~(\ref{e1-gud}). Максимальное число устройств в~соте:
$$
K_s= \left \lfloor  
\fr{r( d_0, p_s^{\max})}{r_0}\right\rfloor\,,\enskip s=0,1
\,.
$$

  Сводный перечень основных обозначений приведен в~табл.~1.

\end{multicols}

  \begin{table*}\small
  \begin{center}
  \Caption{Основные обозначения}
  \vspace*{2ex}
  
  \begin{tabular}{|l|p{340pt}|}
  \hline
  \multicolumn{1}{|c|}{Обозначение}&\multicolumn{1}{c|}{Описание}\\
  \hline
  $R$&Радиус соты, м\\
  $\omega$&Ширина полосы радиочастот, МГц\\
  $L$&Число уровней качества канала CQI \\
  $c$&Уровень CQI (СВ)\\
  $\eta=16-c$&Величина, обратная уровню CQI~$c$  (СВ)\\
  $q_l=\fr{2L-2l-1}{L^2}$&Вероятность того, что уровень CQI равен~$l$\\
  $\alpha^{-1}$&Среднее время доступности полосы, с\\
  $\beta^{-1}$&Среднее время недоступности полосы, с\\
  $k$&Число активных устройств\\
  $s$&Состояние полосы: $s=1$, если полоса доступна; $s=0$, если недоступна\\
  $p_0^{\max}$&Максимальное значение мощности сигнала устройства, если полоса 
недоступна, Вт\\
  $p_1^{\max}$&Максимальное значение мощности сигнала устройства, если полоса 
доступна, Вт\\
  $d_0$&Минимальное расстояние от устройства до БС, м\\
  $r\left( \xi_{d(\eta)}, p_s^{\max}\right)$ &Достижимая скорость передачи 
  для устройства с~уровнем CQI $c\hm=16\hm-\eta$, если полоса находится в~состоянии~$s$ (СВ), бит/с\\
  $r_0^{\max}$&Максимально возможная скорость, если полоса недоступна, бит/с\\
  $r_0$&Гарантированная скорость передачи данных от устройств, бит/с\\
  $r_1^{\max}$&Максимально возможная скорость, если полоса доступна, бит/с\\
  $K_0$&Максимальное число устройств, если полоса недоступна\\
  $K_1$&Максимальное число устройств, если полоса доступна\\
  $\xi_d(\eta)$ &Максимальное расстояние от устройства с~уровнем CQI $c\hm=16\hm-\eta$ 
до БС (СВ), м\\
  $\lambda$&Интенсивность суммарного потока данных от всех устройств в~соте, 1/с\\
  $\mu^{-1}$&Среднее время передачи данных от одного устройства, с\\
  $\rho=\fr{\lambda}{\mu}$&Суммарная предложенная нагрузка от всех устройств в~соте, Эрл\\[-9pt]
&\\
  \hline
  \end{tabular}
  \end{center}
  \end{table*}
  
  \begin{multicols}{2}
  

  
\section{Вероятностная модель и~стационарное распределение 
вероятностей состояний беспроводной сети}

  Перейдем к~построению математической модели. Пусть $\xi(t)$~--- число 
активных M2M-устройств; $\eta_i(t)$~--- значение параметра~$\eta$ для 
устройства~$i$; $\zeta(t)$~--- состояние полосы в~момент времени $t\hm\geq 0$. 
Тогда функционирование соты опишем СП $\left\{ \xi(t),\eta_1(t),\ldots 
,\eta_{\xi(t)},\zeta(t), t\geq0\right\}$ над пространством состояний
  \begin{multline*}
  \mathbf{L}\hm= 
  \left\{ 
    \vphantom{\sum\limits_{i=1}^k}
    (0,s), \left(k,l_1,\ldots ,l_k, s\right), \right.\\ 
  s=0,1,\ l_i=1,\ldots 
,L,\ i=1,\ldots , k,\ k=1,2,\ldots: \\
\left.  \sum\limits_{i=1}^k \fr{r_0}{\omega \ln \left( 1+Gp_s^{\max}/((RL^{-
1}l_i)^\kappa N_0)\right)}\leq 1\right\}\,.
  %\label{e2-gud}
  \end{multline*}

Фрагмент пространства состояний показан на рис.~2.
    

  Перейдем к~СП $\{\xi(t),\zeta(t), t\geq0\}$ с~укрупненными состояниями~--- 
суммарным числом устройств и~состоянием полосы. Пространство состояний 
такого процесса будет иметь вид:
  \begin{equation*}
  \mathbf{L}_1 =\left\{ (k,s):\ k=0,1,\ldots ,K_s,\ s=0,1\right\}\,.
  %\label{e3-gud}
  \end{equation*}
Отметим, что при переходе полосы в~недоступное состояние происходит 
снижение мощности передачи данных с~$p_1^{\max}$ до~$p_0^{\max}$ 
и~прерывание обслуживания $k\hm-K_0$ устройств при условии, что число 
устройств $k\hm>K_0$. При переходе из недоступного в~доступное состояние 
мощность снова повышается. На рис.~3 представлена диаграмма 
интенсивностей переходов данного СП.
    
    

  Обозначим через $P_s(k)$, $s\hm=0,1$, условную ве\-ро\-ят\-ность того, что 
$(k+1)$-е M2M-устройство может быть обслужено при условии, что 
активно~$k$~устройств. Можно показать, что ве\-ро\-ят\-но\-сти~$P_s(k)$ 
вы\-чис\-ля\-ют\-ся по формулам:

\noindent
  \begin{align*}
  P_s(0) &={}\\
  &\hspace*{-8mm}{}=F_{\xi_d(\eta)} \left( \min 
  \left\{ R, \left( \fr{Gp_s^{\max}}{\left( 
e^{r_0/\omega} -1\right) N_0}\right)^{1/\kappa}\right\}\right)\,,\\
&\hspace*{50mm}s=0,1\,;
\end{align*}

\end{multicols}

\begin{figure*} %fig2
 \vspace*{1pt}
\begin{center}
\mbox{%
\epsfxsize=107.234mm
\epsfbox{gud-2.eps}
}
\end{center}
\vspace*{-11pt}
\Caption{Фрагмент диаграммы интенсивностей переходов СП с~детальными состояниями}
%\vspace*{-20pt}
\end{figure*}


\begin{multicols}{2}

\noindent
\begin{align*}
  P_s(k) &= \fr{\Phi\left( (1-m_{k+1,s})/\tau_{k+1,s}\right)}{\Phi\left((1-
m_{ks})/\tau_{ks}\right)}\,,\\
& \hspace*{20mm}k=1,\ldots, K_s,\enskip s=0,1\,,
  \end{align*}
где
\begin{gather*}
\Phi(x) =\fr{1}{\sqrt{2\pi}}\int\limits^x_{-\infty} e^{-t^2/2}\,dt\,;\\
m_{ks} =  kr_0E \left[ \fr{1}{r\left(d,p_s^{\max}\right)}\right]\,;
\end{gather*}

\vspace*{-12pt}

\noindent
\begin{multline*}
\tau^2_{ks}=kr_0^2\left( E\left[ \left( 
\fr{1}{r\left( d,p_s^{\max}\right)}\right)^2\right]- {}\right.\\
\left.{}-
\left( E\left[ \fr{1}{r\left(d, p_s^{\max}\right)}\right]\right)^2\right)\,;
\end{multline*}

\vspace*{-12pt}

\noindent
\begin{multline*}
E\left[ \fr{1}{r\left( d,p_s^{\max}\right)}\right] = \fr{1}{r_s^{\max}}\, 
F_{\xi_d(\eta)} (d_0) +{}\\
{}+
\int\limits_{d_0}^R \fr{1}{\omega \ln (1+Gp_s^{\max}/ 
(x^\kappa N_0))}\, f_{\xi_d(\eta)} (x)\,dx\,;
\end{multline*}

\vspace*{-14pt}

\noindent
\begin{multline*}
E\left[ \left(\fr{1}{r\left( d,p_s^{\max}\right)}\right)^2\right] = \left( \fr{1} 
{r_s^{\max}}\right)^2 F_{\xi_d(\eta)}(d_0) + {}\\
{}+\int\limits^R_{d_0} \fr{1} {\omega^2 
\ln^2 (1+Gp_s^{\max}/(x^\kappa N_0))}\,f_{\xi_d(\eta)} (x)\,dx\,.
\end{multline*}

\begin{figure*} %fig3
     \vspace*{1pt}
\begin{center}
\mbox{%
\epsfxsize=146.037mm
\epsfbox{gud-3.eps}
}
\end{center}
\vspace*{-9pt}
\Caption{Диаграмма интенсивностей переходов СП с~укрупненными состояниями}
\end{figure*}
  
  Случайный процесс $\{ \xi(t),\zeta(t), t\geq0\}$ является марковским, и~для расчета его 
стационарного рас-\linebreak\vspace*{-12pt}

\pagebreak

\noindent
пределения вероятностей~$p(k,s)$, $(k,s)\hm\in 
\mathbf{L}_1$ предлагается следующий рекуррентный алгоритм.
  \begin{enumerate}[1.]
  \item Значения ненормированных вероятностей $q(k,s)$ вычисляются по 
формулам:
  \begin{align*}
  q(0,0)&=1\,;\\
  q(0,1) &=x\,;\\
  q(k,s) & =\delta_{ks}+\gamma_{ks} x\,,\ (k,s)\in \mathbf{L}_1:\ k>0\,,
  \end{align*}
где
$$
x=\fr{(K_1\mu +\alpha)\delta_{K_11}-\lambda P_1(K_1-1)\delta_{K_1-1{,}1}} 
{\lambda P_1(K_1-1)\gamma_{K_1-1,1}-(K_1\mu +\alpha)\gamma_{K_11}}\,.
$$
\item Коэффициенты $\delta_{ks}$ и~$\gamma_{ks}$ вычисляются по 
рекуррентным формулам:
\begin{gather*}
\delta_{00}=1\,,\ \gamma_{00}=0\,;\\
\delta_{01}=0\,,\ \gamma_{01}=1\,;\\
\delta_{10}=\fr{\lambda P_0(0)+\beta}{\mu}\,,\ \gamma_{10}=-
\fr{\alpha}{\mu}\,;\\
\delta_{11}=-\fr{\beta}{\mu}\,,\ \gamma_{11}=\fr{\lambda 
P_1(0)+\alpha}{\mu}\,;
\end{gather*}

\vspace*{-12pt}

\noindent
\begin{multline*}
\delta_{k0}=\fr{\lambda P_0(k-1)+(k-1)\mu+\beta}{k\mu}\,\delta_{k-1,0} - {}\\
{}-
\fr{\lambda P_0(k-2)}{k\mu}\,\delta_{k-2,0}-\fr{\alpha}{k\mu}\,\delta_{k-1,1}\,,\ 
k=2,\ldots ,K_0,\hspace*{-0.26485pt}
\end{multline*}

\vspace*{-12pt}

\begin{multline*}
\gamma_{k0} = \fr{\lambda P_0(k-1)+(k-1)\mu+\beta}{k\mu}\,\gamma_{k-1,0}-{}\\
{}-
\fr{\lambda P_0(k-2)}{k\mu}\,\gamma_{k-2,0} -\fr{\alpha}{k\mu}\,\gamma_{k-
1,1}\,,\ k=2,\ldots,K_0;\hspace*{-1.73058pt}
\end{multline*}

%\vspace*{-12pt}

\noindent
\begin{gather*}
\delta_{k1} = \fr{\lambda P_1(k-1)+(k-1)\mu+\alpha}{k\mu}\,\delta_{k-1,1} - {}\\
{}-
\fr{\lambda P_1(k-2)}{k\mu}\,\delta_{k-2,1} -\fr{\beta}{k\mu}\,\delta_{k-1,0}\,,\\ 
k=2,\ldots ,K_0+1\,;\\
\hspace*{-3mm}\gamma_{k1} =\fr{\lambda P_1(k-1)+(k-1)\mu+\alpha}{k\mu}\,\gamma_{k-1,1} - {}\\
{}-
\fr{\lambda P_1(k-2)}{k\mu}\,\gamma_{k-2,1}- \fr{\beta}{k\mu}\,\gamma_{k-
1,0}\,,\\ 
k=2,\ldots ,K_0+1\,;\\
\delta_{k1} = \fr{\lambda P_1(k-1)+(k-1)\mu+\alpha}{k\mu}\,\delta_{k-1,1}- {}\\
{}-
\fr{\lambda P_1(k-2)}{k\mu}\,\delta_{k-2,1}\,,\   k=K_0+2,\ldots , K_1\,,\\
\gamma_{k1} =\fr{\lambda P_1(k-1)+(k-1)\mu+\alpha}{k\mu}\,\gamma_{k-1,1} - {}\\
{}-
\fr{\lambda P_1(k-2)}{k\mu}\,\gamma_{k-2,1}\,,\ k=K_0+2,\ldots ,K_1\,.
\end{gather*}

\item Значения вероятностей $p(k,s)$ вычисляются по формулам:

\noindent
$$
p(k,s)=\fr{q(k,s)}{\sum\nolimits_{(i,j)\in \mathbf{L}} q(i,j)}\,,\ (k,s)\in 
\mathbf{L}_1\,.
$$
\end{enumerate}

\begin{table*}\small
  \begin{center}
  \Caption{Исходные данные для численного анализа}
  \vspace*{2ex}
  
  \begin{tabular}{|l|c|c|c|}
  \hline
\multicolumn{1}{|c|}{Обозначение}&Случай 1  
(рис.~4)&Случай 2  
(рис.~5)&Случай 3  
(рис.~6)\\
\hline
$R$, м&200--400&200; 400&200; 400\\
$\omega$, МГц&10&10&10\\
$L$&15&15&15\\
$\alpha^{-1}$, мин&20; 30&20; 30&30\\
$\beta^{-1}$, с&20&20&20\\
$p_1^{\max}$, дБ$\cdot$м&23; 42&23--42&23; 42\\
$p_0^{\max}$, дБ$\cdot$м&$p_{\max}/2$&$p_{\max}/2$&$p_{\max}/2$\\
$d_0$, м&$R/15$&$R/15$&$R/15$\\
$r_0$, Мбит/с&1&1&1\\
$\lambda$, 1/с&10&10&2--10\\
$\mu^{-1}$, с&0,1&0,1&0,1\\
$N_0$, дБ$\cdot$м&$-60$&$-60$&$-60$\\
$G$&197,43&197,43&197,43\\
$\kappa$&5&5&5\\
\hline
\end{tabular}
\end{center}
%\vspace*{-6pt}
\end{table*}
\begin{figure*}[b] %fig4
% \vspace*{-6pt}
\begin{center}
\mbox{%
\epsfxsize=162.099mm
\epsfbox{gud-4.eps}
}
\end{center}
\vspace*{-9pt}
\Caption{Показатели эффективности в~зависимости от мощности устройств: 
(\textit{а})~вероятность прерывания обслуживания при 
$R=400$ (\textit{1}~--- $\alpha\hm=1200$; 
\textit{2}~--- $\alpha\hm=1800$); 
(\textit{б})~среднее число активных устройств 
при $\alpha\hm=1800$ (\textit{3}~--- $R\hm=200$; 
\textit{4}~---  $R\hm=400$)}
\end{figure*}

\vspace*{-9pt}

\section{Пример численного анализа и~заключение}

\vspace*{-3pt}

  Зная стационарное распределение вероятностей  $p(k,s)$, 
  $(k,s)\hm\in \mathbf{L}_1$, найдем 
основные показатели эффективности модели~--- вероятность~$B$ блокировки, 
вероятность~$\Pi$ прерывания обслуживания и~среднее число~$\overline{K}$ 
устройств по формулам:

\noindent
  \begin{align*}
  B &= \sum\limits_{k=0}^{K_0-1} \left( 1-P_0(k)\right) p(k,0) 
+{}\notag\\
&\hspace*{30mm}{}+\sum\limits_{k=0}^{K_1-1}\left( 1-P_1(k)\right) p(k,1)\,;
  %\label{e4-gud}
  \\
  \Pi &=  \hspace*{-2mm}\hspace*{-0.72604pt}\sum\limits_{k=K_0+1}^{K_1-1}
  \hspace*{-1mm}
   \fr{\alpha}{\alpha+k\mu+\lambda 
P_1(k)}\, \fr{\begin{pmatrix} k-1\\ k-K_0-1\end{pmatrix}}{ \begin{pmatrix} k\\ k-
K_0\end{pmatrix}}\,p(k,1)+{}\notag\\
&\hspace*{10mm}{}+\fr{\alpha}{\alpha+K_1\mu}\,\fr{\begin{pmatrix} K_1-
1\\ K_1-K_0-1\end{pmatrix}} {\begin{pmatrix} K_1\\ K_1-K_0\end{pmatrix}} 
\,p(K_1,1)\,;
  %\label{e5-gud}
  \\
  \overline{K} &= \sum\limits_{k=0}^{K_0} kp (k,0) +\sum\limits_{k=0}^{K_1} 
kp(k,1)\,.
  %\label{e6-gud}
  \end{align*}
  
  Для проведения численного анализа проанализируем передачу данных  
M2M-устройствами небольшими сессиями, составляющими в~среднем~10~с, 
в~высоком качестве на ско\-рости~1~Мбит/с. 
Рассмотрим небольшой аэропорт, 
в~котором самолеты взлетают раз в~20~(30)~мин, среднее время пролета 
самолета над сотой составляет~20~с. 
Исходные данные представлены в~табл.~2.
  
  
  
  На рис.~4 показана зависимость вероятности прерывания обслуживания 
и~среднего числа активных устройств от их мощности. Вероятность\linebreak 
прерывания обслуживания уменьшается пропорционально увеличению 
мощности, так как при более высокой мощности для устройств достижима 
более высокая скорость. При этом вероятность прерывания ниже для более 
низкой интен\-сив\-ности отключения полосы.
Среднее число устройств 
увеличивается пропорционально радиусу соты
 (см.\linebreak

\end{multicols}

\begin{figure*} %fig5
 \vspace*{1pt}
\begin{center}
\mbox{%
\epsfxsize=161.797mm
\epsfbox{gud-5.eps}
}
\end{center}
\vspace*{-9pt}
\Caption{Показатели эффективности в~зависимости от радиуса соты: 
(\textit{а})~вероятность 
прерывания обслуживания при $W\hm=0{,}2$
(\textit{1}~--- $\alpha\hm=1200$;
\textit{2}~--- $\alpha\hm=1800$); (\textit{б})~среднее число активных 
устройств при $\alpha\hm=1200$ (\textit{3}~--- $W\hm=0{,}2$;
\textit{4}~--- $W=15{,}85$)}
%\end{figure*}
%\begin{figure*} %fig6
 \vspace*{6pt}
\begin{center}
\mbox{%
\epsfxsize=159.48mm
\epsfbox{gud-6.eps}
}
\end{center}
\vspace*{-9pt}
\Caption{Показатели эффективности в~зависимости от интенсивности потока пакетов 
данных от устройств: (\textit{а})~вероятность прерывания обслуживания при 
$R=400$ (\textit{1}~--- $\alpha\hm=1200$; 
\textit{2}~---  $\alpha\hm=1800$); (\textit{б})~среднее 
число активных устройств (\textit{3}~---  $R\hm=400$, $\alpha\hm=1200$;
\textit{4}~--- $R\hm=200$, $\alpha\hm=1800$)}
\vspace*{-30pt}
\end{figure*}

\begin{multicols}{2}

\noindent
 рис.~5). Вероятность 
прерывания оказывается ниже при меньшей интенсивности изъятия полосы 
(см.\ рис.~6). 





  В заключение отметим, что в~статье разработана вероятностная модель 
совместного использования радиочастот, при помощи которой проведен анализ 
показателей эффективности применения политики управления мощ\-ностью 
с~учетом разноудаленных от БС M2M-устройств. 

В~дальнейшем 
предполагается учесть случайную высоту, на которой могут находиться 
устройства.

\vspace*{-12pt}

{\small\frenchspacing
 { %\baselineskip=10pt
 \addcontentsline{toc}{section}{References}
 \begin{thebibliography}{99}
\bibitem{1-gud}
Cisco Visual Networking Index: Global Mobile Data Traffic Forecast Update, 2016--2021 White 
Paper. March~28, 2017. {\sf  
http://www.cisco.com/c/en/us/ solutions/collateral/service-provider/visual-networking-index-vni/mobile-white-paper-c11-520862.html}.
\bibitem{2-gud}
\Au{Andrews J., Buzzi~S., Choi~W., Hanly~S.\,V., Lozano~A., Soong~C.\,K., Zhang~J.\,C.} What 
will 5G be?~// IEEE J.~Sel. Area. Comm., 2014. Vol.~32. P.~1065--1082. 

\bibitem{5-gud} %3
ETSI TR 103 113. Electromagnetic compatibility and Radio spectrum Matters 
(ERM); System Reference document (SRdoc); Mobile broadband services in the  
2\,300~MHz\,--\,2\,400~MHz frequency band under Licensed Shared Access regime. 
v1.1.1. July 2013. 
{\sf 
http:// www.etsi.org/deliver/etsi\_tr/103100\_103199/103113/ 01.01.01\_60/tr\_103113v010101p.pdf}.

\bibitem{3-gud} %4
ETSI TR 103 154. Reconfigurable Radio Systems (RRS); System requirements 
for operation of Mobile Broadband Systems in the 2\,300~MHz\,--\,2\,400~MHz band under Licensed 
Shared Access (LSA). v1.1.1. October 2014.
{\sf 
http://www.etsi.org/deliver/etsi\_TS/103100\_103199/ 103154/01.01.01\_60/ts\_103154v010101p.pdf}.

\bibitem{4-gud} %5
ETSI TR 103 235. Reconfigurable Radio Systems (RRS); System architecture 
and high level procedures for operation of Licensed Shared Access (LSA) in  
the 2\,300~MHz\,--\,2\,400~MHz band. v1.1.1. October 2015. {\sf 
http://www.\linebreak
 etsi.org/deliver/etsi\_ts\%5C103200\_103299\%5C103235
 \%5C01.01.01\_60\%5Cts\_103235v010101p.pdf}.

\bibitem{6-gud} %6
\Au{Buckwitz K., Engelberg J., Rausch~G.} Licensed Shared Access (LSA)~--- regulatory 
background and view of Administrations~// 9th Conference (International) on Cognitive Radio 
Oriented Wireless Networks.~--- IEEE, 2014. P.~413--416.
\bibitem{7-gud}
\Au{Ahokangas P., Matinmikko~M., Yrj$\ddot{\mbox{o}}$l$\ddot{\mbox{a}}$~S., 
Mustonen~M., 
Posti~H., Luttinen~E., Kivim$\ddot{\mbox{a}}$ki~A.} Business models for mobile network 
operators in Licensed Shared Access (LSA)~// IEEE Symposium (International) on Dynamic 
Spectrum Access Networks.~--- IEEE, 2014. P.~263--270.


\bibitem{9-gud} %8
\Au{Borodakiy~V.\,Y., Samouylov~K.\,E., Gudkova~I.\,A., Ostrikova~D.\,Y., Ponomarenko~A.\,A., 
Turlikov~A.\,M., Andreev~S.\,D.} Modeling unreliable LSA operation in 3GPP LTE cellular 
networks~// 6th Congress (International) on Ultra Modern Telecommunications and Control 
Systems and Workshops Proceedings.~--- Piscataway, NJ, USA: IEEE, 2015. 
P.~490--496.

\bibitem{8-gud} %9
\Au{Ponomarenko-Timofeev A., Pyattaev~A., Andreev~S., Kou\-che\-rya\-vy~Ye., Mueck~M., Karls~I}. 
Highly dynamic spectrum management within licensed shared access regulatory framework~// 
IEEE Commun. Mag., 2015. Vol.~54. No.\,3. P.~100--109.

\bibitem{10-gud}
\Au{Gudkova I.\,A., Samouylov~K.\,E., Ostrikova~D.\,Y., Mokrov~E.\,V.,  
Ponomarenko-Timofeev~A.\,A., Andreev~S.\,D., Koucheryavy~Y.\,A.} Service failure and 
interruption probability analysis for Licensed Shared Access regulatory framework~// 7th Congress 
(International) on Ultra Modern Telecommunications and Control Systems and Workshops
 Proceedings.~--- Piscataway, NJ, USA: IEEE Computer Society, 2015. P.~123--131.

\bibitem{11-gud}
\Au{Samouylov K., Gudkova~I., Markova~E., Yarkina~N.} Queuing model with unreliable servers 
for limit power policy within Licensed Shared Access framework~// Internet of things, smart
spaces, and next generation networks and systems~/
Eds. O.~Galinina, S.~Balankin, Y.~Koucheryavy.~---
Lecture notes in computer 
science ser.~--- Springer, 2016. Vol.~9870. P.~404--413.
\bibitem{12-gud}
\Au{Galinina O., Andreev~S.\,D., Gerasimenko~M., Kou\-che\-rya\-vy~Y.\,A., Himayat~N., Yeh S.-P., 
Talwar~S.} Capturing spatial randomness of heterogeneous cellular/WLAN deployments with 
dynamic traffic~// IEEE J.~Sel. Area. Comm., 2014. Vol.~32.  
No.\,6. P.~1083--1099.
\bibitem{13-gud}
\Au{Ahmadian A., Galinina~O., Gudkova~I., Andreev~S., Shorgin~S., Samouylov~K.} On capturing 
spatial diversity of joint M2M/H2H dynamic uplink transmissions in 
3GPP LTE cellular system~// Internet of things, smart
spaces, and next generation networks and systems~/
Eds.\ S.~Balandin, S.~Andreev, Y.~Koucheryavy.~---
Lecture notes in computer science ser.~--- Springer, 2014. Vol.~9247. P.~407--421.
\bibitem{14-gud}
\Au{Samouylov K., Gudkova~I., Markova~E., Dzantiev~I.} On analyzing the blocking probability 
of M2M transmissions for a CQI-based RRM scheme model in 3GPP LTE~// Comm. 
Com. Inf. Sci., 2016. Vol.~638. P.~327--340.
\bibitem{15-gud}
\Au{Gudkova I., Markova~E., Masek~P., Andreev~S., Hosek~J., Yarkina~N., Samouylov~K., 
Koucheryavy~Y.} Modeling the utilization of a multi-tenant band in 3GPP LTE system with 
Licensed Shared Access~// 8th Congress (International) on Ultra Modern Telecommunications and 
Control Systems and Workshops Proceedings.~--- Piscataway, NJ, USA: IEEE, 
2016. P.~179--183.
 \end{thebibliography}

 }
 }

\end{multicols}

\vspace*{-6pt}

\hfill{\small\textit{Поступила в~редакцию 20.04.17}}

\vspace*{6pt}

%\newpage

%\vspace*{-24pt}

\hrule

\vspace*{2pt}

\hrule

\vspace*{-4pt}


\def\tit{PROBABILITY MODEL FOR ANALYZING LICENSED SHARED~ACCESS WITH~ADAPTIVE 
POWER CONTROL IN~A~WIRELESS~NETWORK}

\def\titkol{Probability model for analyzing licensed shared access with adaptive 
power control in a wireless network}

\def\aut{I.\,A.~Gudkova$^{1,2}$ and~S.\,Ya.~Shorgin$^2$}

\def\autkol{I.\,A.~Gudkova and  S.\,Ya.~Shorgin}

\titel{\tit}{\aut}{\autkol}{\titkol}

\vspace*{-9pt}


\noindent
$^1$Peoples' Friendship University of Russia, 6~Miklukho-Maklaya Str., Moscow 117198, Russian Federation

\noindent
$^2$Institute of Informatics Problems, Federal Research Center ``Computer Science and Control'' of the 
Russian\linebreak
$\hphantom{^1}$Academy of Sciences, 44-2~Vavilov Str., Moscow 119333, Russian Federation



\def\leftfootline{\small{\textbf{\thepage}
\hfill INFORMATIKA I EE PRIMENENIYA~--- INFORMATICS AND
APPLICATIONS\ \ \ 2017\ \ \ volume~11\ \ \ issue\ 3}
}%
 \def\rightfootline{\small{INFORMATIKA I EE PRIMENENIYA~---
INFORMATICS AND APPLICATIONS\ \ \ 2017\ \ \ volume~11\ \ \ issue\ 3
\hfill \textbf{\thepage}}}

\vspace*{3pt}



\Abste{Emerging next generation wireless networks involve new applications and services for 
human-to-human and machine-to-machine (M2M) devices. The problem of increasing 
requirements for network capacity and lack of radio spectrum arises. The solution could be found in 
the licensed shared access framework, e.\,g., in the case of smart cities. The authors 
propose a mathematical model of shared access to spectrum with adaptive power control. The 
algorithm makes it possible to avoid the interference between M2M devices and the spectrum 
owner due, in part, to the fact that it takes into account the spatial distribution and session activity of 
devices.}

%\pagebreak

\KWE{wireless network; smart city; machine-to-machine (M2M); licensed shared 
access (LSA); 
adaptive power control; stochastic process; recursive algorithm; 
blocking probability; interruption 
probability; average number of M2M devices}




\DOI{10.14357/19922264170310} 

%\vspace*{-18pt}

%\pagebreak

\Ack
\noindent
This work was financially supported by the Russian Science 
Foundation (grant No.\,16-11-10227).



%\vspace*{3pt}

  \begin{multicols}{2}

\renewcommand{\bibname}{\protect\rmfamily References}
%\renewcommand{\bibname}{\large\protect\rm References}

{\small\frenchspacing
 {\baselineskip=10.282pt
 \addcontentsline{toc}{section}{References}
 \begin{thebibliography}{99}
\bibitem{1-gud-1}
Cisco Visual Networking Index: Global Mobile Data Traffic Forecast Update, 2016--2021 White 
Paper. March~28, 2017. Available at: {\sf 
http://www.cisco.com/c/en/us/ solutions/collateral/service-provider/visual-networking-index-vni/mobile-white-paper-c11-520862.html} (accessed June~26, 2017).
\bibitem{2-gud-1}
\Aue{Andrews, J., S.~Buzzi, W.~Choi, S.\,V.~Hanly, A.~Lozano, C.\,K.~Soong, and 
J.\,C.~Zhang.} 2014. What will 5G be? \textit{IEEE J.~Sel. Area. Comm.}  
32:1065--1082. 

\bibitem{5-gud-1} %3
ETSI TR 103 113. July 2013. Electromagnetic compatibility and Radio spectrum 
Matters (ERM); System 
Reference document (SRdoc); Mobile broadband services in the 2300~MHz\,--\,2400~MHz frequency 
band under Licensed Shared Access regime. Available at: {\sf 
http://www.etsi.org/deliver/etsi\_tr/103100\_103199/ 103113/01.01.01\_60/tr\_103113v010101p.pdf}  
(accessed June 26, 2017).

\bibitem{3-gud-1} %4
ETSI TR 103 154. October 2014. Reconfigurable Radio Systems (RRS); System requirements for operation of 
Mobile Broadband Systems in the 2300~MHz\,--\,2400~MHz band under Licensed Shared Access 
(LSA). v1.1.1.  Available at: {\sf 
http://www.etsi.org/deliver/etsi\_TS/103100\_\linebreak  
103199/103154/01.01.01\_60/ts\_103154v010101p.pdf} (accessed June~26, 2017).
\bibitem{4-gud-1} %5
ETSI TR 103 235. October 2015. Reconfigurable Radio Systems (RRS); System architecture and high level 
procedures for operation of Licensed Shared Access (LSA) in the 
2300~MHz\,--\,2400~MHz band.
v1.1.1. Available at: {\sf 
http://www.etsi.org/deliver/etsi\_ts\%5C103200\_103299
\%5C103235\%5C01.01.01\_60\%5Cts\_103235v010101p. pdf} (accessed June~26, 2017).

\bibitem{6-gud-1}
\Aue{Buckwitz, K., J.~Engelberg, and G.~Rausch.} 2014. Licensed Shared Access (LSA)~--- 
regulatory background and view of Administrations. \textit{9th Conference (International) on 
Cognitive Radio Oriented Wireless Networks}. IEEE. 413--416.
\bibitem{7-gud-1}
\Aue{Ahokangas, P., M. Matinmikko, S.~Yrj$\ddot{\mbox{o}}$l$\ddot{\mbox{a}}$, 
M.~Mustonen, H.~Posti, E.~Luttinen, and A.~Kivim$\ddot{\mbox{a}}$ki.} 2014. Business 
models for mobile network operators in Licensed Shared Access (LSA). \textit{IEEE Symposium 
(International) on Dynamic Spectrum Access Networks}. IEEE. 263--270.

\bibitem{9-gud-1} %8
\Aue{Borodakiy, V.\,Y., K.\,E.~Samouylov, I.\,A.~Gudkova, D.\,Y.~Ostrikova, 
A.\,A.~Ponomarenko, A.\,M.~Turlikov, and S.\,D.~Andreev.} 2014. Modeling unreliable LSA 
operation in 3GPP LTE cellular networks. \textit{6th Congress (International) on Ultra Modern 
Telecommunications and Control Systems and Workshops Proceedings}. Piscataway, NJ: IEEE.  
490--496.

\bibitem{8-gud-1} %9
\Aue{Ponomarenko-Timofeev, A., A.~Pyattaev, S.~Andreev, Ye.~Koucheryavy, M.~Mueck, and 
I.~Karls.} 2015. Highly dynamic spectrum management within licensed shared access regulatory 
framework. \textit{IEEE Commun. Mag.} 54(3):100--109.

\bibitem{10-gud-1}
\Aue{Gudkova, I.\,A., K.\,E.~Samouylov, D.\,Y.~Ostrikova, E.\,V.~Mokrov,  
A.\,A.~Ponomarenko-Timofeev, S.\,D.~Andreev, and Y.\,A.~Koucheryavy}. 2015. Service failure 
and interruption probability analysis for Licensed Shared Access regulatory framework. \textit{7th 
Congress (International) on Ultra Modern Telecommunications and Control Systems and Workshops
Proceedings}. Piscataway,  NJ: IEEE. 123--131.
\bibitem{11-gud-1}
\Aue{Samouylov, K., I.~Gudkova, E.~Markova, and N.~Yarkina}. 2016. Queuing model with 
unreliable servers for limit power policy within Licensed Shared Access framework. 
\textit{Internet of things, smart
spaces, and next generation networks and systems}.
Eds. O.~Galinina, S.~Balankin, Y.~Koucheryavy.
{Lecture 
notes in computer science ser.} Springer. 9870:404--413.
\bibitem{12-gid-1}
\Aue{Galinina, O., S.\,D.~Andreev, M.~Gerasimenko, Y.\,A.~Koucheryavy, N.~Himayat,  
S.-P.~Yeh, and S.~Talwar.} 2014. Capturing spatial randomness of heterogeneous cellular/WLAN 
deployments with dynamic traffic. \textit{IEEE J.~Sel. Area. Comm.}  
32(6):1083--1099.
\bibitem{13-gud-1}
\Aue{Ahmadian, A., O.~Galinina, I.~Gudkova, S.~Andreev, S.~Shorgin, and K.~Samouylov.} 
2014. On capturing spatial diversity of joint M2M/H2H dynamic uplink transmissions in 3GPP 
LTE cellular system. 
\textit{Internet of things, smart
spaces, and next generation networks and systems}.
Eds.\ S.~Balandin, S.~Andreev, Y.~Koucheryavy.
{Lecture notes in computer science ser.} Springer. 9247:407--421.
\bibitem{14-gud-1}
\Aue{Samouylov, K., I.~Gudkova, E.~Markova, and I.~Dzantiev.} 2016. On analyzing the 
blocking probability of M2M transmissions for a CQI-based RRM scheme model in 3GPP LTE. 
\textit{Comm. Com. Inf. Sci.} 638:327--340.
\bibitem{15-gud-1}
\Aue{Gudkova, I., E.~Markova, P.~Masek, S.~Andreev, J.~Hosek, N.~Yarkina, K.~Samouylov, 
and Y.~Koucheryavy.} 2016. Modeling the utilization of a multi-tenant band in 3GPP LTE system 
with Licensed Shared Access. \textit{8th Congress (International) on Ultra Modern 
Telecommunications and Control Systems and Workshops Proceedings}.  Piscataway, NJ: IEEE.  
179--183.
\end{thebibliography}

 }
 }

\end{multicols}

\vspace*{-9pt}



\hfill{\small\textit{Received April 20, 2017}}

\vspace*{-24pt}

\Contr

\noindent
\textbf{Gudkova Irina A.}\ (b.\ 1985)~--- Candidate of Sciences (PhD) in physics and 
mathematics; associate professor, Peoples' Friendship University of Russia,  
6~Miklukho-Maklaya Str., Moscow 117198, Russian Federation; senior scientist, Institute 
of Informatics Problems, Federal Research Center ``Computer Science and Control'' of the 
Russian Academy of Sciences, 44-2~Vavilov Str., Moscow 119333, Russian Federation; 
\mbox{ gudkova\_ia@rudn.university }

%\vspace*{1pt}

\noindent
\textbf{Shorgin Sergey Ya.} (b.\ 1952)~--- Doctor of Science in physics and mathematics, professor; Deputy Director, Federal Research Center 
``Computer Science and Control'' of the Russian Academy of Sciences (FRC CSC RAS); principal scientist, Institute of Informatics Problems, FRC 
CSC RAS; 44-2~Vavilov Str., Moscow 119333, Russian Federation; \mbox{sshorgin@ipiran.ru}

\label{end\stat}


\renewcommand{\bibname}{\protect\rm Литература} 