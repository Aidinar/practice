\def\stat{kovalev}

\def\tit{МЕТОДЫ ТЕОРИИ КАТЕГОРИЙ В~МОДЕЛЬНО-ОРИЕНТИРОВАННОЙ СИСТЕМНОЙ 
ИНЖЕНЕРИИ}

\def\titkol{Методы теории категорий в~модельно-ориентированной системной 
инженерии}

\def\aut{С.\,П.~Ковалёв$^1$}

\def\autkol{С.\,П.~Ковалёв}

\titel{\tit}{\aut}{\autkol}{\titkol}

\index{Ковалёв С.\,П.}
\index{Kovalyov S.\,P.}


%{\renewcommand{\thefootnote}{\fnsymbol{footnote}} \footnotetext[1]
%{Исследование выполнено при финансовой поддержке Российского научного фонда (проект 16-11-10227).}}


\renewcommand{\thefootnote}{\arabic{footnote}}
\footnotetext[1]{Институт проблем управления им.\ В.\,А.~Трапезникова 
Российской академии наук,  \mbox{kovalyov@nm.ru}}

%\vspace*{-18pt}

\Abst{Предложен математический аппарат на базе теории категорий, который позволяет 
формально описывать и~строго исследовать процедуры применения моделей в~инженерной 
деятельности, составляющие сущность мо\-дель\-но-ори\-ен\-ти\-ро\-ван\-ной системной 
инженерии (Model-Based Systems Engineering, MBSE). В~основе аппарата лежит 
математическое представление сборочных чертежей (мегамоделей сис\-тем) диаграммами 
в~категориях, объектами которых служат модели, а~морфизмы представляют действия по 
сборке моделей сис\-тем из моделей компонентов. Адекватность аппарата обоснована исходя 
из требований стандартов, регламентирующих описание структуры систем, в~том числе 
IEC~81346. Предложены и~исследованы тео\-ре\-ти\-ко-ка\-те\-гор\-ные методы решения ряда 
практических задач сборки систем. Приведены примеры решения таких задач в~категориях, 
представляющих две ключевые области применения MBSE: гео\-мет\-ри\-че\-ское моделирование 
изделий сложной формы и~дис\-крет\-но-со\-бы\-тий\-ное имитационное моделирование 
поведения технических систем.}

\KW{модельно-ориентированная системная инженерия; мегамодель; теория категорий; 
копредел}



\DOI{10.14357/19922264170305} 


\vspace*{6pt}

\vskip 10pt plus 9pt minus 6pt

\thispagestyle{headings}

\begin{multicols}{2}

\label{st\stat}

\section{Введение}

   Модельно-ориентированная системная инженерия состоит в~формализованном применении моделирования в~
поддержке жизненного цикла сис\-тем, включая сбор требований, 
проектирование, проверку и~приемку, другие стадии~[1]. Модели, 
разрабатываемые в~ходе процедур MBSE, пригодны к~автоматической 
обработке на компьютерах. Это позволяет сначала задавать, верифицировать 
и~оптимизировать проектные решения на моделях <<в циф\-ре~и только потом 
воплощать <<в железе>>, снижая затраты на организацию жизненного цикла 
изделий и~сокращая сроки выполнения работ~[2].
   
   И все же внедрение технологий MBSE в~инженерную деятельность 
происходит медленно. Это связано во многом с~нехваткой единой 
концептуальной базы инженерного моделирования: предлагается много 
частных языков и~технологий, слабо совместимых друг с~другом и~плохо 
приспособленных для совместной разработки моделей большими 
мультидисциплинарными коллективами~[3]. Тем самым затрудняется переход 
от набора электронных чертежей к~полноценному электронно-цифровому 
макету (digital mock-up) промышленного изделия.
   
   Естественный, хотя и~<<трудный>>, подход к~получению результатов 
общего характера, унифи\-ци\-ру\-ющих разнородные технологии, состоит в~том, 
чтобы как можно более строго формализовать процедуры моделирования. 
Формализация позволит совершенствовать процедуры MBSE и~передавать их 
на исполнение компьютеру без пробелов и~искажений. Самый высокий уровень 
строгости достигается при привлечении математического аппарата, поскольку 
математика позволяет надежно доказывать или опровергать утверждения, 
ха\-рак\-те\-ри\-зу\-ющие корректность и~эффективность процедур.
   
   В настоящей работе предложен аппарат, основанный на математическом 
представлении сборочных чертежей (<<мегамоделей>> систем) 
ориенти-\linebreak рованными графами (диаграммами). Узлы такого\linebreak графа помечаются 
обозначениями моделей час\-тей, а~реб\-ра помечаются обозначениями действий\linebreak 
(activities), посредством которых части собираются в~систему. Представление 
структуры систем графами регламентируется, в~частности, стандартом 
IEC~81346~[4]. Естественным источником математических методов 
конструирования и~анализа мегамоделей служит теория категорий (см., 
например,~[5, 6]). Модели рассматриваются как объекты подходящих 
категорий, а~действия формально описываются морфизмами. Строятся 
и~исследу-\linebreak ются тео\-ре\-ти\-ко-ка\-те\-гор\-ные конструкции, опи\-сы\-ва\-ющие процедуры 
MBSE на абстрактном кон-\linebreak цептуальном уровне. Определенный опыт такого\linebreak 
исследования был накоплен в~инженерии программного обеспечения~[7] 
и~теперь может быть обобщен для системной инженерии в~целом. Например, 
сборке системы согласно некоторой мегамодели отвечает построение 
копредела диаграммы~--- универсальной конструкции~\cite{5-kov}.
   
   Статья построена следующим образом. В~разд.~2 приведен обзор 
принципов описания структуры сис\-тем согласно стандарту IEC~81346. 
Раздел~3 посвящен практическим проб\-ле\-мам мегамоделирования и~сборке 
сис\-тем. В~разд.~4 вводятся конструкции тео\-рии категорий, позволяющие 
формально решать задачи мегамоделирования. В~заключении приводятся 
выводы и~намечаются направления дальнейших исследований.

\section{Структура систем и~стандарт~IEC~81346}

   Важной проблемой MBSE, отмеченной во введении, является слабая 
совместимость языков и~инструмен\-тов моделирования от разных поставщиков. 
Основным подходом к~достижению совместимости является стандартизация~--- 
принятие обязывающих документов, устанавливающих требования и~принципы 
взаимозаменяемости инструментов. Многие стандарты определяют конкретные 
форматы машиночитаемой записи моделей, нейтральные относительно 
разработчиков инструментов MBSE. Примером служит формат описания 
твердотельных геометрических моделей STEP, стандартизованный семейством 
ISO~10303. Однако для формализации MBSE в~целом интерес представляют 
в~первую очередь стандарты более общего плана, унифицирующие принципы 
и~методы применения моделей в~жизненном цикле систем независимо от 
способа записи моделей. С~этой точки зрения внимания заслуживает 
международный стандарт IEC 81346-1:2009 <<Промышленные системы, 
установки и~обору\-до\-ва\-ние~--- принципы структурирования и~ссылочные 
обозначения~--- часть~1: основные правила>> (<<Industrial Systems, 
Installations and Equipment and Industrial Products~--- Structuring Principles and 
Reference Designations~--- Part~1: Basic Rules>>)~\cite{4-kov}. Стандарт не 
принят в~России, однако ряду его положений в~области структуры систем 
соответствует российский ГОСТ~2.053-2013 <<ЕСКД. Электронная структура 
изделия. Общие положения>>.
   
   В стандарте IEC~81346 рассматривается ряд вопросов моделирования 
структуры систем и~идентификации отдельных единиц в~составе систем. 
Системная единица названа в~стандарте объектом, причем принципиально не 
проводится различие между объектами реального мира, составляющими 
реально существующие системы, и~объектами мыслительной деятельности~--- 
моделями единиц, составляющими модели систем. Таким образом, стандарт 
выходит за рамки MBSE и~рассматривает ряд вопросов системной инженерии 
вообще. Иерар\-хи\-че\-ская структура системы (холархия~\cite{3-kov}) 
изображается деревом, узлы которого помечены обозначениями объектов. 
Важным достижением стандарта является выявление того факта, что одна и~та 
же система задается не одной, а несколькими в~общем случае различными 
иерархическими структурами, возникающими в~результате декомпозиции 
согласно различным принципам (аспектам). В~их числе:
   \begin{itemize}
\item функциональная (function-oriented) структура, отвечающая разделению 
системных единиц по выполняемым ими функциям в~составе сис\-темы;
\item продуктовая (product-oriented), или модульная, структура, отражающая 
сборочную (технологическую) конфигурацию сис\-темы;
\item структура размещения (location-oriented), в~соответствии с~которой 
единицы располагаются в~физическом пространстве.
\end{itemize}

   Ясно, что один и~тот же объект может входить в~несколько структур и~при 
этом находиться на различных уровнях. В~то же время в~некоторых аспектах 
объект может никак не проявлять себя и~вследствие этого отсутствовать 
в~соответствующих структурах. Полное идентифицирующее ссылочное 
обозначение объекта (reference designation) конструируется путем 
последовательного перечисления всех объектов, находящихся на пути от корня 
дерева рассматриваемой структуры до дан\-ного объекта включительно. 
Наименование каж\-до\-го объекта в~этом перечислении составляется из 
символьного обозначения аспекта, буквенного обозначения класса (типа), 
к~которому относится  объект, и~порядкового номера объекта среди 
экземпляров своего класса. Таким путем обеспечивается\linebreak  уникальность 
наименования любой единицы\linebreak
 в~пределах системы. Например, функциональная 
структура обозначается символом <<=>>, а~функциональный класс 
переключателей потоков ресурсов обозначается буквами QA, так что первая по 
порядку единица, выполняющая функцию переключения, называется =QA1, 
а~ее полное ссылочное обозначение может выглядеть как =WP1=WC1=QA1. 
Если объект присутствует в~нескольких структурах, то он может иметь 
несколько ссылочных обозначений, как показано на рис.~1~\cite{4-kov}.

\begin{figure*} %fig1
    \vspace*{1pt}
\begin{center}
\mbox{%
\epsfxsize=165mm
\epsfbox{kov-1.eps}
}
\end{center}
\vspace*{-9pt}
\Caption{Пример ссылочных обозначений структурных единиц системы}
\vspace*{9pt}
\end{figure*}

   С~точки зрения практики системной инженерии большой интерес 
представляет описание эволюции структурного представления системы по ходу 
жизненного цикла, приведенное в~приложении~B к~стандарту IEC~81346. 
<<Строительный материал>> для структур имеет вид (виртуального) 
справочника или каталога объектов, из которого выбираются объекты для 
включения в~структуру. 

В~начале жизненного цикла системы на основе 
исходных требований к~ней конструктор строит ее функциональную структуру. 
Затем определяется пространственное положение функциональных объектов, 
в~результате чего создается структура размещения. На следующей стадии 
формируются закупочные спецификации, образующие продуктовую структуру. 
В~ходе последующих стадий жизненного цикла эти структуры могут 
трансформироваться. На каждой стадии могут происходить замена, слияние 
и~расщепление объектов. Таким образом, объекты разных структур системы 
связаны отношением вида <<многие ко многим>>, вдоль которого 
прослеживаются (трассируются) исходные требования.
   
   В то же время стандарт не предусматривает указа\-ние способов, какими 
объекты собраны в~сис\-те\-мы. Поэтому структуру сис\-те\-мы можно рас\-смат\-ри\-вать 
как эскизный проект, в~котором отражены лишь факты вхождения системных 
единиц более низкого уровня иерархии в~единицы более высокого уровня. 


Проект такого рода поступает на вход технологу, который определяет 
конкретные операции сборки каждой единицы каждого уровня иерархии. При 
необходимости технолог вносит изменения в~конструкцию объектов (такие как 
нарезка резьбы) и~добавляет связующие интерфейсные объекты (такие как 
клей, трансформатор и~др.). В~результате для каждого составного объекта 
формируется сборочный чертеж, на котором указаны все со\-став\-ля\-ющие 
объекты и~действия по их соединению в~целях получения сис\-те\-мы. 
Технологическая проработка требуется на всех стадиях жизненного цикла, на 
которых формируется либо изменяется ка\-кая-ли\-бо из структур системы.

%\vspace*{-6pt}

\section{Мегамоделирование и~сборка~систем}

   В MBSE объекты, образующие 
структуры\linebreak
 сис\-тем, описываются формализованными ком\-пьютерными моделями 
различных видов: геометрическими фигурами и~телами, численными 
аппроксимациями дифференциальных уравнений, оснащенными графами и~
т.\,д. При этом, как свидетель\-ст\-ву\-ют стандарты типа IEC~81346, для анализа 
структуры систем и~организации сборки необходимо знать не столько 
внутреннюю структуру моделей, сколько ассортимент их возможностей 
соединяться с~другими моделями в~целях формирования моделей составных 
объектов. Иными словами, модели рассматриваются как <<черные ящики>> 
с~известным поведением по отношению к~другим моделям. Каталог объектов, 
упоминавшийся в~предыду\-щем разделе, в~условиях применения \mbox{MBSE} 
составляется из моделей и~описаний действий по их соединению.
   
   Структуры систем и~сборочные чертежи представляют собой частные 
случаи мегамоделей (mega\-mod\-el)~--- моделей, состоящих из моделей и~связей 
между ними~\cite{8-kov}. Мегамодель, в~которой связи описывают соединение 
моделей, образующих некоторую сис\-те\-му, называется конфигурацией этой 
сис\-те\-мы~\cite{5-kov}. Существуют и~другие виды мегамоделей, 
предназначенные для описания других процедур \mbox{MBSE}, таких как 
формирование модели согласно заданной метамодели  
(instantiating)~\cite{9-kov}. Но в~настоящей работе сосредоточимся на 
конфигурациях и~сборке систем.
   
   Например, в~моделировании механических сис\-тем, состоящих из твердых 
тел, моделями деталей и~сборочных единиц служат геометрические тела, 
которые могут быть представлены для компьютерной обработки различными 
способами: конструктивным, воксельным, граничным~\cite{10-kov}. Объекты, 
составляющие механические системы, т.\,е.\ представления экземпляров тел, 
получаются из моделей путем аффинных изометрий и~растяжений. Так, из 
набора цилиндров разных размеров составляется модель штанги (спортивного 
снаряда). В~функциональной структуре штанги по IEC~81346 цилиндры 
представлены разными объектами, поскольку они выполняют разные функции, 
хотя порождаются одной и~той же геометрической моделью. Соответственно, 
в~каталоге моделей содержится тело в~форме цилиндра, допускающее 
несколько разных действий по включению в~состав штанги.
   
   В качестве еще одного примера рассмотрим дис\-крет\-но-со\-бы\-тий\-ное 
имитационное моделирование, поддержка которого относится к~числу 
важнейших достижений MBSE~\cite{1-kov}. Здесь модель имеет вид 
сценария~--- фрагмента предполагаемой истории поведения моделируемой 
системы, пред\-став\-лен\-но\-го потоком дискретных событий различных видов. 
Некоторые события могут вызывать либо запрещать возникновение других 
событий. Описания действий по сборке сценариев поведения систем отражают 
вклад сценариев поведения составляющих. Так, сценарий работы цеха 
составляется из сценариев работы станков, связанных друг с~другом согласно 
маршрутным картам~\cite{11-kov}.
   
   Сформулируем задачу мегамоделирования сборки систем в~общем виде 
следующим образом. По мегамодели, представляющей конфигура\-цию 
некоторой системы, требуется сконструировать модель системы как целого 
и~рассчитать для нее моделируемые параметры, в~том числе эмерджентные~--- 
не присущие никакой из со\-став\-ля\-ющих единиц в~отдельности. Принцип 
конструирования модели системы легко усмотреть из организации 
структур-\linebreak\vspace*{-12pt}

\columnbreak

 { \begin{center}  %fig1
 \vspace*{1pt}
\mbox{%
\epsfxsize=57.246mm
\epsfbox{kov-2.eps}
}


\vspace*{12pt}


\noindent
{{\figurename~2}\ \ \small{Схема склеивания}}
\end{center}
}

\vspace*{18pt}

\addtocounter{figure}{1}

\noindent
ного представления: система должна находиться на иерархическом 
уровне, располагающемся непосредственно над уровнем со\-став\-ля\-ющих ее 
объектов. Иными словами, модель системы должна включать в~себя модели 
всех составляющих с~учетом их конфигурационных связей и~в~то же время 
включаться в~любые модели, включающие в~себя модели всех составляющих 
конфигурации.
   
   Поясним этот принцип на простом примере. Предположим, что нужно 
объединить в~систему два объекта~$P$ и~$S$ и~что технолог решил сделать это 
с~по\-мощью клея~--- третьего объекта~$G$, который может быть соединен 
и~с~$P$, и~с~$S$. Действие клея описывается конфигурацией следующего 
вида: объекты~$G$ и~$P$ порождают в~результате соединения известный 
промежуточный комплексный объект~$P_G$, содержащий их, а~объекты~$G$ 
и~$S$ порождают объект~$S_G$. Система~$R$, полученная путем 
склеивания~$P$ с~$S$ при помощи~$G$, отбирается среди объектов, 
содержащих~$P_G$ и~$S_G$, по следующему структурному критерию: 
объект~$R$ должен содержаться в~любом объекте~$T$, содержащем~$P_G$ 
и~$S_G$. Схематически этот критерий изображен на рис.~2.


   Если объект $R$, удовлетворяющий указанному структурному критерию, 
существует, то он действительно отвечает системе, которая собрана из~$S$ 
и~$P$ путем склеивания посредством~$G$ (и~не содержит ничего 
<<лишнего>>). Более того, легко видеть, что такой объект~$R$ определяется, 
по существу, однозначно в~том смысле, что любые два объекта~$R$ 
и~$R^\prime$, удовлетворяющие структурному критерию, содержатся друг 
в~друге. Если же нужного объекта~$R$ не существует, то делается вывод, что 
технолог ошибся: клей~$G$ не способен соединить объекты~$P$ и~$S$.
   
   В структурное представление, выполненное по стандарту IEC~81346 либо по 
ГОСТу 2.053-2013, входят только объекты~$P$, $S$ и~$R$ и~две композитные 
стрелки: $P\hm\to R$, проходящая через~$P_G$, и~$S\hm\to R$, проходящая 
через~$S_G$ (так что мегамодель склеивания~--- это часть схемы, ограниченная 
треугольником~$PSR$). Кроме того, стрелки на схеме склеивания, в~отличие от 
структуры, представляют не просто факты включения объектов друг в~друга, 
а~конкретные действия по их соединению. При этом соблюдается следующее 
естественное условие структурной корректности: если из одного объекта 
можно прийти в~другой разными путями по схеме, то эти пути задают одно и~то 
же композитное действие. Например, клей~$G$ включается в~состав 
системы~$R$ единственным способом, несмотря на наличие двух путей $G 
\hm\to  P_G \hm\to R$ и~$G \hm\to S_G \hm\to R$: в~действительности не имеет 
значения, через какой промежуточный объект <<прослеживается>> включение 
клея в~систему. Таким образом, мегамодель сборки содержит больше 
информации, чем иерархическая структура системы.
   
   Если модели содержат значения тех или иных параметров, а описание 
действий по их соединению позволяет выявить правила преобразования 
значений, то по мегамодели сборки можно вы\-чис\-лить значения параметров для 
системы. Известны примеры вычислений такого рода в~области разработки 
новых композиционных материалов~\cite{12-kov}. Осредненные 
(эффективные) физические характеристики композитов, такие как модуль Юнга и~коэффициент Пуассона, сложным образом зависят от характеристик 
компонентов и~способов изготовления композита из них. При помощи методов 
теории упру\-гости эти зависимости задаются в~форме линеаризованных 
матричных соотношений, которые приписываются к~стрелкам мегамоделей, 
пред\-став\-ля\-ющим включение компонентов в~композиты. Появляется 
возможность рассчитывать на компьютере свойства композитов по базе данных 
компонентов, без проведения дорогостоящих физических экспериментов.
   
   В заключение раздела отметим, что хотя прямой расчет системы по 
конфигурации имеет большое значение, в~MBSE он играет вспомогательную 
роль. Согласно стандарту IEC~81346 и~практикам системной инженерии, 
система обычно проектируется сверху вниз~--- от корня структурной иерархии 
к~составляющим~\cite{13-kov}. Это означает, что технолог в~основном решает 
не прямую, а~обратную задачу: модель системы, которую нужно собрать, 
известна, а~нужно построить (восстановить) конфигурацию, из которой такая 
система может быть получена путем сборки, с~учетом различных ограничений. 
Формальные математические постановки и~методы решения обратных задач 
мегамоделирования представляют собой крупную перспективную тему 
исследований, выходящую за рамки настоящей статьи.

\section{Теория категорий в~мегамоделировании}

   Как указывалось во введении, естественным источни\-ком математических 
методов кон\-стру\-ирова\-ния и~анализа мегамоделей служит теория категорий. 
Категорией называется коллекция абстрактных объектов, попарно связанных 
морфизмами (стрелками). Точное определение занимает буквально несколько 
строк~\cite{14-kov}: категория~$C$ состоит из совокупности 
объектов~$\mathrm{Ob}\,C$ и~совокупности морфизмов~$\mathrm{Mor}\,C$, 
на которых заданы следующие операции:
\begin{enumerate}[(1)]
\item каждому морфизму~$f$ 
сопоставляется два объекта: область $\mathrm{dom}\,f$ и~кообласть 
$\mathrm{codom}\,f$ (соотношения вида $\mathrm{dom}\,f \hm= A$ и~
$\mathrm{codom}\,f \hm= B$ наглядно записываются в~форме стрелки~$f$: 
$A\hm\to B$, а множество всех морфизмов, удовлетворяющих этим 
соотношениям, обозначается через $\mathrm{Mor}(A, B))$;
\item для 
любой пары морфизмов~$f, g$, удовлетворяющей условию 
$\mathrm{codom}\,f\hm = \mathrm{dom}\,g$, определена композиция~--- 
морфизм $g \circ f : \mathrm{dom}\,f \hm\to  \mathrm{codom}\,g$, причем она 
ассоциативна: для любой тройки морфизмов~$f, g, h$, удовлетворяющей 
условиям $\mathrm{codom}\,f \hm= \mathrm{dom}\,g$ и~$\mathrm{codom}\,g 
\hm= \mathrm{dom}\,h$, выполняется соотношение $h \circ (g \circ f) \hm= (h 
\circ g) \circ f$;
\item любой объект~$A$ обладает тождественным 
морфизмом~$1_A : A \to A$ таким, что для любого морфизма~$f : A\hm\to B$ 
выполняется соотношение $f \circ 1_A \hm= 1_B \circ  f \hm= f$.
\end{enumerate}

Классическим 
примером категории служит $\mathbf{Set}$, состоящая из всех множеств и~всех 
их отображений: закон композиции отображений задается стандартной 
подстановкой, а тождественным морфизмом произвольного множества служит 
его тождественное отображение на себя.
   
   Вместе с~категорией вводится понятие функтора~--- отображения категорий, 
сохраняющего структуру. Функтор $\mathrm{fun}\,: C \hm\to D$, действующий из 
категории~$C$ в~$D$,~--- это пара одноименных отображений $\mathrm{fun}\,: 
\mathrm{Ob}\,C \hm\to \mathrm{Ob}\,D$, $\mathrm{fun}\,: \mathrm{Mor}\,C \hm\to 
\mathrm{Mor}\,D$, удовлетворяющая следующим условиям (для произвольных 
$C$-мор\-физ\-мов~$f, g$ и~$C$-объ\-ек\-та~$A$): 
\begin{enumerate}[(1)]
\item $\mathrm{fun}\,(\mathrm{dom}\,f) 
\hm= \mathrm{dom}\,\mathrm{fun}\,(f), \mathrm{fun}\,(\mathrm{codom}\,f)\hm = 
\mathrm{codom}\,\mathrm{fun}\,(f)$;  
\item $\mathrm{fun}\,(g \circ f) = \mathrm{fun}\,(g) \circ \mathrm{fun}\,(f)$, 
если композиция $g \circ f$ определена; 
\item $\mathrm{fun}\,(1_A) \hm= 1_{\mathrm{fun}\,(A)}$.
\end{enumerate}
 Все категории и~все функторы образуют 
(формальную) категорию~$\mathbf{CAT}$. Чтобы исследовать взаимосвязь 
между функторами, вводится следующее понятие: естественным 
преобразованием~$\varepsilon$ функтора $\mathrm{fun}\, : C\hm\to D$ в~$\mathrm{fun}^\prime\, : C 
\hm\to D$ называется любое семейство $D$-мор\-физ\-мов~$\varepsilon_A : 
\mathrm{fun}\,(A) \hm\to \mathrm{fun}^\prime (A)$, $A \hm\in \mathrm{Ob}\,C$, 
такое что для любого 
\mbox{$C$-мор}\-физ\-ма $f : A\hm\to B$ выполняется соотношение $\varepsilon_B \circ 
\mathrm{fun}\,(f) \hm= \mathrm{fun}^\prime(f) \circ \varepsilon_A$:

%\begin{figure*} %рис
\vspace*{1pt}
\begin{center}
\mbox{%
\epsfxsize=54.473mm
\epsfbox{kov-3.eps}
}
\end{center}
%\vspace*{-9pt}
%\end{figure*}

   Эффективность применения теории категорий в~качестве математического 
аппарата \mbox{MBSE} обуслов\-ле\-на тем, что любой каталог моделей представляет 
собой не что иное, как категорию. Действительно, любая цепочка действий по 
соединению моделей порождает композитное действие (процесс) и, кроме того, 
любая модель допускает пустое действие над самой собою, не 
подразумевающее никаких изменений (процедура <<ничегонеделания>>). 
Например, в~твердотельном моделировании механических систем объектами 
категории\linebreak моделей выступают тела~--- подмножества в~$\mathbb{R}^3$, 
которые являются ограниченными, регулярными\linebreak
 (совпадают с~замыканием 
своей внутренности) и~полуаналитическими (допускают представление 
конечными булевыми комбинациями множеств вида $\{(x, y, z) \vert  F_i(x, y, 
z)\hm\leq 0\}$, где~$F_i : \mathbb{R}^3\hm\to \mathbb{R}$ является 
вещественной аналитической функцией для всех~$i$)~\cite{10-kov}. Чтобы 
было возможно задавать процедуры типа склеивания участков поверхности тел, в~категорию геометрических моделей добавляются ограниченные регулярные 
полуаналитические подмножества в~$\mathbb{R}^n$, $0 \hm\leq n \hm\leq 2$, 
при помощи стандартного вложения~$\mathbb{R}^n$ в~$\mathbb{R}^3$. Далее 
выполняется факторизация: отождествляются друг с~другом все множества, 
переходящие друг в~друга под действием аффинных изометрий. Морфизмы 
таких классов эквивалентности, описывающие действия по сборке составных 
механических сис\-тем, порождаются изометрическими вложениями множеств 
и~растяжениями. Получается подкатегория в~\textbf{Set}, которую будем обозначать 
через $\mathbf{MBS}$ (от Multibody Systems).
   
   Для многих известных технологий MBSE формальное описание каталогов 
поддерживаемых моделей приводит к~категориям множеств со структурой~--- 
алгебраических систем, топологических пространств, графов и~т.\,д. 
Морфизмами в~таких категориях служат отображения множеств, со\-вмес\-ти\-мые 
со структурой. На любой такой категории действует канонический функтор 
в~$\mathbf{Set}$, <<забывающий>> структуру. 

В~качестве примера приведем  
дис\-крет\-но-со\-бы\-тий\-ное моделирование, в~котором математической 
моделью сценария служит множество событий, час-\linebreak тич\-но упорядоченное  
при\-чин\-но-след\-ст\-вен\-ны\-ми зависимостями и~размеченное видами 
событий~\cite{15-kov}. Действия по сборке сложных сценариев задаются 
монотонными отображениями, сохраняющими разметку, поскольку ни 
события, ни зависимости, ни метки не могут быть <<потеряны>> при 
соединении сценариев поведения компонентов в~сценарии поведения систем. 
Получается категория~$\mathbf{Pomset}$, состоящая из всех помеченных 
частично упорядоченных множеств и~всех их монотонных отображений, 
сохраняющих разметку. Имеется функтор $\vert \mbox{--} \vert : 
\mathbf{Pomset}\hm\to \mathbf{Set} : S \mapsto \vert S\vert$, <<забывающий>> 
порядок и~разметку.
   
   Зафиксируем произвольную категорию~$C$, представляющую некоторый 
каталог моделей. Как и~для любой алгебраической системы, определена 
конструкция подкатегории в~$C$~--- это пара, состоящая из подкласса 
в~$\mathrm{Ob}\,C$ и~подкласса в~$\mathrm{Mor}\,C$, замкнутых 
относительно унаследованных из~$C$ операций. Подкатегория в~$C$ 
называется полной, если любой \mbox{$C$-мор}\-физм, область и~кообласть которого 
содержатся в~ней, сам содержится в~ней. Например, подкатегориями 
описываются различные аспекты структурного представления систем согласно 
стандарту IEC~81346. Действительно, композиция двух морфизмов, 
представляющих действия по формированию некоторого аспекта структуры, 
также должна входить в~этот аспект, поскольку стандарт предписывает строить 
цепочки для идентификации объектов в~структуре системы. Кроме того, если 
объект присутствует в~аспекте, то его тождественный морфизм формально 
должен быть включен в~этот аспект. В~то же время подкатегории, 
опи\-сы\-ва\-ющие все аспекты, не обязаны образовывать в~совокупности разбиение 
категории~$C$: как показывает рис.~1, возможны как действия, входящие 
в~несколько аспектов одновременно, так и~композитные действия с~переходом 
между структурами, не входящие ни в~один аспект. Требуется лишь, чтобы 
объединение классов объектов всех этих подкатегорий совпадало 
с~$\mathrm{Ob}\,C$, поскольку не имеет смысла вводить модели, не входящие 
ни в~одну структуру.
   
   Категории можно получать из графов: любой ориентированный мультиграф 
порождает категорию, объектами в~которой служат все узлы, а морфизмами~--- 
все пути. Областью и~кообластью морфизма являются соответственно начало 
и~конец пути, композиция морфизмов действует как конкатенация путей, 
а~тождественным морфизмом узла~$a$ является пустой путь из~$a$ в~$a$, не 
содержащий ни одного ребра. Отсюда получается фундаментальное понятие  
$C$-диа\-грам\-мы~--- это функтор вида~$\Delta : X \hm\to C$, где~$X$~--- 
категория, порожденная некоторым графом и~называемая схемой диаграммы. 
Все $C$-диа\-грам\-мы образуют категорию~$\mathbf{D}C$ (ковариантная 
категория <<сверхзапятой>>~\cite{14-kov}), в~которой морфизмом 
диаграммы~$\Delta : X \hm\to C$ в~$\Xi : Y \hm\to C$ служит любая пара 
вида $\langle\gamma, fd\rangle$, состоящая из функтора~$fd : X\hm\to Y$ 
и~естественного преобразования~$\gamma : \Delta\hm\to \Xi \circ fd$; закон 
композиции морфизмов диаграмм имеет вид:
$$
\langle \gamma, fd\rangle \circ 
\langle \varphi, gd\rangle \hm = \langle \gamma_{gd(-)} \circ \varphi, fd \circ 
gd\rangle\,.
$$ 
В~тео\-рии категорий накоплен богатый арсенал алгебраических 
методов конструирования и~анализа диаграмм.
   
   Любая мегамодель задается $C$-диа\-грам\-мой, так что категорное 
представление каталогов моделей позволяет формально решать задачи 
мегамоделирования. Морфизмы диаграмм описывают структурные 
преобразования мегамоделей, выполняемые при помощи инструментов MBSE. 
Покажем, как решаются средствами теории категорий прямые задачи 
мегамоделирования. Здесь применяется одна из основных  
тео\-ре\-ти\-ко-ка\-те\-гор\-ных конструкций~--- копредел  
диаграммы~\cite{5-kov}, который строится следующим образом. Обозначим 
через~$\mathbf{1}$ категорию,\linebreak состоящую из одного объекта~0 и~одного 
морфизма~$1_0$. Из любой категории~$X$ имеется в~точ\-ности один 
функтор~$!_X : X \hm\to \mathbf{1}$, сопоставляющий объект~0  
любому~$X$-объ\-ек\-ту (иными словами, $\mathbf{1}$ является терминальным 
$\mathbf{CAT}$-объ\-ек\-том). Имеется вложение (инъективный функтор) 
$\ulcorner \mbox{--}\urcorner : C \hookrightarrow \mathbf{D}C$, сопоставляющее 
произвольному $C$-объ\-ек\-ту $Q$~точку~--- диаграмму $\ulcorner Q\urcorner : 
\mathbf{1}\hm\to  C : 0 \mapsto Q$. Коконусом (cocone) называется 
$\mathbf{D}C$-мор\-физм, имеющий точку в~качестве кообласти. Можно 
изобразить коконус $\langle \sigma, !_X\rangle : \Delta\hm\to \ulcorner 
Q\urcorner$ над диаграммой $\Delta : X\hm\to C$ в~виде диаграммы, 
<<пририсовав>> к~$\Delta$ дополнительную вершину, помеченную 
объектом~$Q$, и~набор ребер~--- стрелок, по одной для каждого узла $I\hm\in 
\mathrm{Ob}\,X$, направленной из~$I$ в~вершину и~помеченной морфизмом 
$\sigma_I : \Delta (I) \hm\to Q$. Копределом (colimit) диаграммы~$\Delta$ 
называется коконус $\mathrm{colim}\,\Delta : \Delta\hm\to \ulcorner R\urcorner$, 
универсальный в~том смысле, что для любых \mbox{$C$-объ}\-ек\-та~$T$ 
и~коконуса~$\delta : \Delta\hm\to\ulcorner T\urcorner$ существует единственный 
$C$-мор\-физм~$w : R \hm\to T$ такой, что $\delta\hm= \ulcorner w\urcorner \circ  
\mathrm{colim}\,\Delta$. Легко видеть, что это условие универсальности 
представляет собой в~точности структурный критерий из разд.~3. Таким 
образом, конструирование копредела конфигурации~$\Delta$ описывает на 
строгом математическом языке сборку системы, которой отвечает 
вершина~$R$. В~категориях типа $\mathbf{MBS}$ и~$\mathbf{Pomset}$ 
построение копредела сводится к~факторизации раздельных объединений 
объектов, представляющих компоненты системы, по отношениям 
эквивалентности, индуцированным моделями клея и~других средств сборки.
   
   Копредел любой диаграммы, если он сущест\-вует, определяется однозначно 
   с~точностью до изомор\-физма. Более того, можно описать сборку сис\-тем из 
конфигураций в~виде функтора. Пусть $Cd$~--- некоторый класс  
$C$-диа\-грамм, имеющих копределы. Он порождает полную подкатегорию 
в~$\mathbf{D}C$, из которой в~$C$ действует функтор копредела $\mathrm{colim}$, 
сопоставляя каждой диаграмме из~$Cd$~вершину некоторого ее копредела, а 
каждому \mbox{$\mathbf{D}C$-мор}\-физ\-му~$\theta : \Delta\hm\to \Xi$, 
где~$\Delta, \Xi\hm\in Cd$~--- стрелку копредела $\mathrm{colim}\,(\theta)$ такую, что 
$\mathrm{colim}\,\Xi \circ \theta \hm= \ulcorner \mathrm{colim}\,(\theta)\urcorner \circ 
\mathrm{colim}\,\Delta$.

%\begin{figure*}
\vspace*{1pt}
\begin{center}
\mbox{%
\epsfxsize=56.127mm
\epsfbox{kov-4.eps}
}
\end{center}
%\vspace*{-9pt}
%\end{figure*}

   Например, в~категории \textbf{Set} любая диаграмма имеет 
копредел~\cite[упражнение~5.1.8]{14-kov}, поэтому имеется функтор $\mathrm{colim}\, : 
\mathbf{D}(\mathbf{Set})\hm\to \mathbf{Set}$. Примечательно, что этот функтор 
является рефлектором: он сопряжен слева с~вложением $\ulcorner \mbox{--}\urcorner : 
\mathbf{Set}\hookrightarrow \mathbf{D}(\mathbf{Set})$, причем 
единица рефлексии состоит из $\mathbf{D}(\mathbf{Set})$-мор\-физ\-мов 
$\mathrm{colim}\,\Delta : \Delta\hm\to \ulcorner\mathrm{colim}\,(\Delta)\urcorner$, 
$\Delta\hm\in \mathrm{Ob}\ \mathbf{D}(\mathbf{Set})$. Напомним, что единица 
рефлексии~--- это естественное преобразование тождественного функтора 
в~композицию рефлектора и~вложения (в~данном случае, естественное 
преобразование функтора $1_{\mathbf{D}(\mathbf{Set})}$ в~$\ulcorner \mathrm{colim}\,(  
\mbox{--})\urcorner)$, состоящее из универсальных  
стрелок~\cite[разд.~4.3]{14-kov}. И~для произвольного класса~$Cd$, 
содержащего достаточное количество одноточечных диаграмм, функтор 
$\mathrm{colim}$ сопряжен слева с~ограничением  
вложения~$\ulcorner \mbox{--}\urcorner$ на подходящую полную подкатегорию 
в~$C$. А~поскольку сопряженный функтор задается однозначно с~точностью 
до изоморфизма~\cite[разд.~4.1]{14-kov}, можно сделать вывод, что сборка 
систем в~некотором смысле <<зашифрована>> в~процедуре построения 
одноточечных диаграмм~--- моделей систем как целого без раскрытия 
струк\-туры. 

Так наглядно проявляется двойственность прямых и~обратных задач 
мегамоделирования.

\section{Заключение}

   Аппарат теории категорий обладает большим потенциалом в~области 
повышения полезной отдачи от MBSE, в~том числе путем математически 
строгого решения задач мегамоделирования. Так, базовая процедура системной 
инженерии~--- сборка\linebreak
 системы из заданной конфигурации взаимо\-свя\-занных 
компонентов~--- формально описывается тео\-ретико-ка\-те\-гор\-ной 
конструкцией копредела диа\-граммы. Более сложные конструкции отвечают\linebreak 
сложным процедурам сборки, таким как связывание (weaving) общесистемных 
функций, рассеянных по всем компонентам (crosscutting concerns), например 
мониторинговых или защитных~\cite{16-kov}. Математического представления 
требуют и~другие процедуры MBSE, в~частности коллективная модификация 
мегамоделей и~составляющих моделей, восстановление конфигурации заданной 
системы, оценка взаимозаменяемости компонентов. 

Актуальны вопросы 
внедрения аппарата теории категорий в~практику, в~том числе путем развития 
программных инструментов моделирования и~мегамоделирования. Здесь 
открывается широкий спектр направлений для дальнейших исследований.
   
{\small\frenchspacing
 {%\baselineskip=10.8pt
 \addcontentsline{toc}{section}{References}
 \begin{thebibliography}{99}
\bibitem{1-kov}
Modeling and simulation-based systems engineering handbook~/
Eds.\ D.~Gianni,  A.~D'Ambrogio, A.~Tolk.~--- London: CRC Press, 2014. 513~p.
\bibitem{2-kov}
\Au{Ковалёв С.\,П., Толок~А.\,В.} Применение модельно-ори\-ен\-ти\-ро\-ван\-но\-го подхода 
в~управ\-ле\-нии жизненным циклом технических изделий~// Информационные технологии 
в~проектировании и~производстве, 2015. №\,2. С.~3--9.
\bibitem{3-kov}
\Au{Левенчук А.\,И.} Системноинженерное мышление.~--- М.: TechInvestLab, 2015. 305~с.
\bibitem{4-kov}
IEC 81346-1:2009. Industrial Systems, Installations and Equipment and Industrial Products~--- 
Structuring Principles and Reference Designations~--- Part~1: Basic Rules.~--- Geneva: ISO, 2009. 
168~p.
\bibitem{5-kov}
\Au{Ginali S., Goguen~J.} A~categorical approach to general systems~// 
 Conference (International) on Applied General Systems 
Research Proceedings~/
Ed. G.\,J.~Klir.~--- NATO conference series.~--- New York, NY, USA: Plenum 
Press, 1978. Vol.~5. P.~257--270.
\bibitem{6-kov}
\Au{Mabrok M.\,A., Ryan M.\,J.} Category theory as a~formal mathematical foundation for  
model-based systems engineering~// Appl. Math. Inform. Sci., 2017. Vol.~11. No.\,1. P.~43--51.
\bibitem{7-kov}
\Au{Ковалёв С.\,П.} Тео\-ре\-ти\-ко-ка\-те\-гор\-ный подход к~проектированию программных 
сис\-тем~// Фундаментальная и~прикладная математика, 2014. Т.~19. Вып.~3. С.~111--170.
\bibitem{8-kov}
\Au{B$\acute{\mbox{e}}$zivin J., Jouault~F., Rosenthal~P., Valduriez~P.} Modeling in the large 
and modeling in the small~// Model Driven Architecture: European MDA Workshops on 
Foundations and Applications Proceedings~/
Eds.\ U.~A{\!\ptb{\ss}}mann, M.~Aksit,  A.~Rensink.~--- 
Lecture notes in computer science ser.~--- Springer, 2005. Vol.~3599. 
P.~33--46.
\bibitem{9-kov}
\Au{Diskin Z., Kokaly~S., Maibaum~T.} Mapping-aware mega\-mod\-eling: Design patterns and 
laws~// Software Language Engineering: 6th Conference (International) Proceedings~/
Eds.\ M.~Erwig, R.\,F.~Paige, E.~Van Wyk.~--- 
Lecture notes  in computer science ser.~--- Springer, 2013. Vol.~8225. P.~322--343.
\bibitem{10-kov}
\Au{Requicha A.\,G.} Representations for rigid solids: Theory, methods, and systems~// 
ACM  Comput. Surv., 1980. Vol.~12. Iss.~4. P.~437--464.
\bibitem{11-kov}
\Au{K$\acute{\mbox{a}}$d$\acute{\mbox{a}}$r B., Pfeiffer~A., Monostori~L.} Discrete event 
simulation for supporting production planning and scheduling decisions in digital
 factories~//  37th 
CIRP Seminar (International) on Manufacturing Systems Proceedings.~--- Budapest, 2004.  
P.~444--448.
\bibitem{12-kov}
\Au{Giesa T., Spivak D.\,I., Buehler~M.\,J.} Category theory based solution for the building block 
replacement problem in materials design~// Adv. Eng. Mater., 2012. Vol.~14. 
Iss.~9. P.~810--817.
\bibitem{13-kov}
\Au{Косяков А., Свит У., Сеймур~С., Бимер~С.} Системная инженерия. Принципы 
и~практика~/ Пер. с~англ.~--- М.: ДМК-Пресс, 2014. 636~с. (\Au{Kossiakoff~A., Sweet~W.\,N., 
Seymour~S., Biemer~S.\,M.} Systems engineering principles and practice.~--- 2nd ed.~--- New 
York, NY, USA: John Wiley, 2011. 560~p.)
\bibitem{14-kov}
\Au{Маклейн С.} Категории для работающего математика~/ Пер. с~англ.~--- М.: Физматлит, 
2004. 352~с. (\Au{Mac Lane~S.} Categories for the working mathematician.~--- New York, NY, 
USA: Springer, 1978. 317~p.)
\bibitem{15-kov}
\Au{Pratt V.\,R.} Modeling concurrency with partial orders~// Int. J.~Parallel 
Prog., 1986. Vol.~15. No.\,1. P.~33--71.
\bibitem{16-kov}
\Au{Ковалёв С.\,П.} Семантика ас\-пект\-но-ори\-ен\-ти\-ро\-ван\-но\-го моделирования 
данных и~процессов~// Информатика и~её применения, 2013. Т.~7. Вып.~3. С.~70--80.
 \end{thebibliography}

 }
 }

\end{multicols}

\vspace*{-3pt}

\hfill{\small\textit{Поступила в~редакцию 16.01.17}}

%\vspace*{8pt}

\newpage

\vspace*{-30pt}

%\hrule

%\vspace*{2pt}

%\hrule

%\vspace*{8pt}


\def\tit{METHODS OF CATEGORY THEORY IN~MODEL-BASED SYSTEMS ENGINEERING\\[-7pt]}

\def\titkol{Methods of category theory in~model-based systems engineering}

\def\aut{S.\,P.~Kovalyov\\[-12pt]}

\def\autkol{S.\,P.~Kovalyov}

\titel{\tit}{\aut}{\autkol}{\titkol}

\vspace*{-14pt}


\noindent
Institute of Control Sciences, Russian Academy of Sciences, 65~Profsoyuznaya Str., 
Moscow 117997, Russian Federation



\def\leftfootline{\small{\textbf{\thepage}
\hfill INFORMATIKA I EE PRIMENENIYA~--- INFORMATICS AND
APPLICATIONS\ \ \ 2017\ \ \ volume~11\ \ \ issue\ 3}
}%
 \def\rightfootline{\small{INFORMATIKA I EE PRIMENENIYA~---
INFORMATICS AND APPLICATIONS\ \ \ 2017\ \ \ volume~11\ \ \ issue\ 3
\hfill \textbf{\thepage}}}

\vspace*{1pt}

 

\Abste{A mathematical device based on the category theory is proposed to formally describe and 
rigorously explore procedures of employing models in engineering that constitute the contents of 
model-based systems engineering (MBSE). The essence of the device consists in mathematical 
representation of assembly drawings (megamodels of systems) as diagrams in categories whose 
objects are models, and morphisms represent actions associated with assembling system models 
from component models. The soundness of the device is justified on the basis of standards that 
govern description of the systems' structure such as IEC~81346. Category-theoretical methods for 
solving a number of practical problems of assembling systems are proposed and explored. 
Examples of solving such problems are provided in categories that represent two key application 
areas for MBSE: geometric modeling of complex shapes and discrete-event simulation of the 
behavior of industrial systems.}

\KWE{ model-based systems engineering; megamodel; category theory; colimit}

\DOI{10.14357/19922264170305} 

%\vspace*{-18pt}

%\Ack
%\noindent




\vspace*{-7pt}

  \begin{multicols}{2}

\renewcommand{\bibname}{\protect\rmfamily References}
%\renewcommand{\bibname}{\large\protect\rm References}

{\small\frenchspacing
 {%\baselineskip=10.8pt
 \addcontentsline{toc}{section}{References}
 \begin{thebibliography}{99}
\bibitem{1-kov-1}
Gianni, D., A.~D'Ambrogio, and A.~Tolk, eds. 2014. \textit{Modeling and simulation-based 
systems engineering handbook}. London: CRC Press. 513~p.
\bibitem{2-kov-1}
\Aue{Kovalyov, S.\,P., and A.\,V.~Tolok.} 2015. Primenenie model'no-orientirovannogo podkhoda 
v~upravlenii zhiznennym tsiklom tekhnicheskikh izdeliy [Applying model-based approach 
to product lifecycle management].\linebreak \textit{Informatsionnye tekhnologii v~proektirovanii 
i~proizvod\-st\-ve} [Information Technologies in Design and Industry] 2(158):3--9.
\bibitem{3-kov-1}
\Aue{Levenchuk A.\,I.} 2015. 
\textit{Sistemnoinzhenernoe myshlenie} [Systems engineering thinking]. 
Moscow: TechInvestLab. 305~p.
\bibitem{4-kov-1}
IEC 81346-1:2009. 2009. 
Industrial Systems, Installations and Equipment and Industrial 
Products~--- Structuring Principles and Reference Designations~--- 
Part~1: Basic Rules. Geneva:  ISO. 168~p.
\bibitem{5-kov-1}
\Aue{Ginali, S., and J.~Goguen.} 1978. 
A~categorical approach to general systems. \textit{Conference 
(International) on Applied General Systems Research Proceedings}. Ed.\
 G.\,J.~Klir. \mbox{NATO}  conference ser. Plenum Press. 5:257--270.
\bibitem{6-kov-1}
\Aue{Mabrok, M.\,A., and M.\,J.~Ryan}. 
2017. Category theory as a~formal mathematical foundation for 
model-based systems engineering. \textit{Appl. Math.  Inform. Sci.} 11(1):43--51.
\bibitem{7-kov-1}
\Aue{Kovalyov, S.\,P.} 2016. 
Category-theoretic approach to software systems design. \textit{J.~Math. Sci.} 
214(6):814--853.
\bibitem{8-kov-1}
\Aue{B$\acute{\mbox{e}}$zivin, J., F.~Jouault, P.~Rosenthal, and P.~Valduriez.}
 2005. Modeling in 
the large and modeling in the small. 
\textit{Model Driven Architecture: European MDA Workshops on 
Foundations and Applications Proceedings.} 
Eds.\ U.~\mbox{A{\!\ptb{\ss}}mann}, M.~Aksit, and A.~Rensink. 
Lecture notes in computer science ser. Springer. 3599:33--46.
\bibitem{9-kov-1}
\Aue{Diskin, Z., S.~Kokaly, and T.~Maibaum.} 2013. 
Mapping-aware megamodeling: Design patterns 
and laws. \textit{6th Conference (International) on Software Language Engineering 
Proceedings}. Eds.\ M.~Erwig, R.\,F.~Paige, and E.~Van Wyk. 
Lecture notes in computer science ser. Springer. 
8225:322--343.
\bibitem{10-kov-1}
\Aue{Requicha, A.\,G.} 1980. Representations for rigid solids: 
Theory, methods, and systems. \textit{ACM 
Comput. Surv.} 12(4):437--464.
\bibitem{11-kov-1}
\Aue{K$\acute{\mbox{a}}$d$\acute{\mbox{a}}$r,~B., A.~Pfeiffer, and L.~Monostori.}
2004. Discrete 
event simulation for supporting production planning and scheduling decisions in 
digital factories. \textit{37th CIRP Seminar (International) on Manufacturing 
Systems Proceedings}. Budapest.  444--448.
\bibitem{12-kov-1}
\Aue{Giesa, T., D.\,I.~Spivak, and M.\,J.~Buehler.} 2012. 
Category theory based solution for the building 
block replacement problem in materials design. 
\textit{Adv. Eng. Mater.} 14(9):810--817.
\bibitem{13-kov-1}
\Aue{Kossiakoff, A., W.\,N.~Sweet, S.~Seymour, and S.\,M.~Bie\-mer.}
2011. \textit{Systems engineering 
principles and practice}. 2nd ed. New York, NY: John Wiley. 560~p.
\bibitem{14-kov-1}
\Aue{Mac Lane, S.} 1978. \textit{Categories for the working mathematician}. 
New York, NY: Springer. 317~p.
\bibitem{15-kov-1}
\Aue{Pratt, V.\,R.} 1986. Modeling concurrency with partial orders. 
\textit{Int. J.~Parallel Prog.} 15(1):33--71.
\bibitem{16-kov-1}
\Aue{Kovalyov, S.\,P.} 2013. 
Semantika aspektno-ori\-en\-ti\-ro\-van\-no\-go modelirovaniya dannykh 
i~protsessov [Semantics of aspect-oriented modeling of data and processes]. 
\textit{Informatika i~ee  Primeneniya~--- Inform. Appl.} 7(3):70--80.
\end{thebibliography}

 }
 }

\end{multicols}

\vspace*{-9pt}

\hfill{\small\textit{Received January 16, 2017}}

\vspace*{-18pt}

\Contrl

\noindent
\textbf{Kovalyov Sergey P.} (b.\ 1972)~--- Doctor of Science in physics and 
mathematics, leading scientist, Institute of Control Problems, Russian 
Academy of Sciences, 65~Profsoyuznaya Str., Moscow 117997, Russian 
Federation Federation; \mbox{kovalyov@nm.ru} 

\label{end\stat}


\renewcommand{\bibname}{\protect\rm Литература} 