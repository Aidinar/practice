\def\stat{leri}

\def\tit{ЛОКАЛЬНАЯ ДРЕВОВИДНОСТЬ В~КОНФИГУРАЦИОННЫХ ГРАФАХ СО~СТЕПЕННЫМ
РАСПРЕДЕЛЕНИЕМ$^*$}

\def\titkol{Локальная древовидность в~конфигурационных графах со~степенным
распределением}

\def\aut{М.\,М.~Лери$^1$, Ю.\,Л.~Павлов$^2$}

\def\autkol{М.\,М.~Лери, Ю.\,Л.~Павлов}

\titel{\tit}{\aut}{\autkol}{\titkol}

\index{Лери М.\,М.}
\index{Павлов Ю.\,Л.}
\index{Leri M.\,M.}
\index{Pavlov Yu.\,L.}


{\renewcommand{\thefootnote}{\fnsymbol{footnote}} \footnotetext[1]
{Финансовое обеспечение исследований осуществлялось из 
средств федерального
бюджета на выполнение государственного задания Карельского научного центра 
Российской академии наук
(Институт прикладных математических исследований КарНЦ РАН).}}


\renewcommand{\thefootnote}{\arabic{footnote}}
\footnotetext[1]{Институт прикладных математических исследований 
Карельского научного центра
Российской академии наук, \mbox{leri@krc.karelia.ru}}
\footnotetext[2]{Институт 
прикладных математических исследований Карельского научного центра Российской 
академии наук,\\ \mbox{pavlov@krc.karelia.ru}}

\vspace*{-12pt}




\Abst{Исследуется локальная древовидность конфигурационных 
графов, предназначенных для моделирования сложных сетей коммуникаций. В~таких 
графах степени вершин независимы и~одинаково распределены по степенному закону. 
В~случае ограниченного числа вершин графа найдены зависимости максимального 
объема подграфа, имеющего вид дерева, от числа вершин графа и~параметра 
распределения степеней вершин. Такая же задача решалась для числа деревьев 
заданного объема. Даны также оценки среднего объема дерева в~графе. Показано, 
что при ограниченном числе вершин конфигурационного графа найденные зависимости 
статистически значимо улучшают описание структуры сетей по сравнению с~известными ранее асимптотическими моделями.}


\KW{конфигурационные графы; степенное распределение; локальная 
древовидность;
объем дерева; имитационное моделирование}


\DOI{10.14357/19922264240107}{ITTZEO}
  
\vspace*{3pt}


\vskip 10pt plus 9pt minus 6pt

\thispagestyle{headings}

\begin{multicols}{2}

\label{st\stat}


\section{Введение}

Случайные графы активно используются для моделирования топологий сложных сетей 
различных видов, таких
как транспортные, компьютерные, биологические, социальные и~др.~\cite{Dur,Hof1,Hof2,Fa}.
При этом в~качестве вершин могут рассматриваться как\linebreak отдельные объекты 
(транспортные средства,
компьютеры, люди и~пр.), так и~некоторые комплекс\-ные сис\-те\-мы (транспортные узлы, 
компьютерные автономные
сис\-те\-мы, социальные сообщества и~др.), что существенно влияет на определение 
степеней вершин соответствующих
случайных графов. Подобное разнообразие касается и~принципов установления связей 
между этими вершинами.
Так, конфигурационные графы, модель которых впервые была предложена 
Б.~Боллобашем~\cite{Bol},
с независимыми одинаково распределенными степенями вершин с~общим дискретным 
законом распределения
часто используются при моделировании телекоммуникационных сетей, среди которых и~сеть Интернет~\cite{Fa,RN}.
Исследованию таких моделей посвящена обширная литература, число опуб\-ли\-ко\-ван\-ных 
монографий и~статей по этой тематике исчисляется сотнями. Наиболее полные обзоры 
современного со\-сто\-яния вопроса, полученных результатов и~ис\-поль\-зу\-емых методов 
можно найти в~фундаментальных трудах~\cite{Hof1, Hof2}. Понятно, что структура и~динамика конфигурационных графов существенно зависят от распределений степеней 
вершин, и~многие работы посвящены исследованию таких распределений. Но 
в~последние годы стало яс\-но, что для построения адекватных моделей недостаточно 
знания законов распределения степеней, важ\-но также иметь пред\-став\-ле\-ние о~различных структурных особенностях изуча\-емых объектов.
Одна из таких особенностей конфигурационных графов~--- локальная дре\-во\-вид\-ность 
их структуры~\cite{Hof1, Pav, PavCh}.

Для изучения структуры конфигурационных графов успешно используются методы 
теории вет\-вя\-щих\-ся процессов (см., например,~\cite{Hof1, Hof2}). Каждой вершине 
графа можно поставить в~соответствие начинающийся с~одной час\-ти\-цы вет\-вя\-щий\-ся 
процесс Гальтона--Ватсона. Связь между такими вет\-вя\-щи\-ми\-ся процессами и~конфигурационными графами подробно изучена в~\cite{Hof2}. С~по\-мощью этой связи 
при стремящемся к~бесконечности числе вершин графа найдены условия возникновения 
гигантской компоненты связности, получены оценки объемов компонент связности, 
расстояний между вершинами и~диаметра. Была выявлена локальная древовидность 
топологии графа, состоящая в~том, что связный подграф, содержащий произвольную 
вершину, асимптотически достоверно сближается с~начинающейся с~соответствующей 
этой вершине частицы частью траектории ветвящегося процесса при некотором 
ограничении общего числа час\-тиц такой час\-ти траектории. Понятно, что эта часть 
траектории представляет собой дерево, в~котором начальная частица служит корнем. 
Таким образом, начиная с~любой вершины графа можно построить дерево, содержащее 
максимально возможное число вершин этого графа. Естественно, возникает вопрос об 
оценке максимального объема дерева среди всех таких деревьев. Другими словами, 
возникает задача поиска максимального по объему дерева среди подграфов 
конфигурационного графа.
До сих пор каждое дерево в~конфигурационном графе как часть траектории 
вет\-вя\-ще\-го\-ся процесса изуча\-лось отдельно. Но если рассматривать деревья 
в~совокупности, то получится случайный лес. Тео\-рия случайных лесов в~значительной 
степени была создана и~изложена в~книге~\cite{Pav2}. В~\cite{Pav,PavCh} впервые 
было предложено использовать тео\-рию случайных лесов для изучения структуры 
конфигурационных графов. Однако в~\cite{Pav2} предполагалось, что леса 
генерируются вет\-вя\-щи\-ми\-ся процессами с~конечной дисперсией чис\-ла потомков каждой 
частицы, а~в~современных моделях сетей это условие нарушено. Поэтому 
в~\cite{Pav,PavCh} были получены не\-до\-ста\-ющие результаты о~лесах и~с~их по\-мощью 
изучалось асимп\-то\-ти\-че\-ское поведение некоторых чис\-ло\-вых характеристик. Были 
доказаны теоремы о~предельных распределениях чис\-ла деревьев заданного 
объема~\cite{PavCh}, а~также предельные тео\-ре\-мы для максимального объема 
дерева~\cite{Pav}. Под объемом дерева понимается чис\-ло его вершин.
Целью исследования на\-сто\-ящей работы было изучение этих же характеристик 
конфигурационных графов, а~так\-же среднего объема дерева в~доасимптотической 
об\-ласти и~по\-стро\-ение моделей их зависимостей от объема графа и~па\-ра\-мет\-ра 
распределения степеней вершин.

\section{Описание модели и~алгоритма}

В работе рассматриваются конфигурационные графы, состоящие из $N$ вершин. 
Построение графа происходит
сле\-ду\-ющим образом. Каждой вершине присваивается степень в~соответствии со 
сле\-ду\-ющим распределением~\cite{RN}:
\begin{equation}
\label{eq1}
\mathbf{P}\{\xi \geqslant k\} = k^{-\tau}, \quad k=1,2,\dots,
\end{equation}
где $\xi$~--- случайная величина, равная степени любой вершины графа, а~параметр 
$\tau\hm>0$ принимает
фиксированные значения. Заметим, что в~известных моделях сетей 
коммуникаций~\cite{Dur, Hof1, Hof2, Fa, RN} значения параметра~$\tau$ не выходят за 
пределы интервала~[2, 4].
Степень вершины равна числу полуребер (см., например,~\cite{RN}) вершины графа.
Все полуребра конфигурационного графа занумерованы в~произвольном порядке. Далее 
все полуребра попарно и~равновероятно соединяются друг с~другом, образуя ребра графа.
Сумма степеней вершин (число полуребер конфигурационного графа) в~данной работе 
считается случайной
величиной. Легко видеть, что для построения графа эта величина должна быть 
четной, поэтому в~противном случае степень равновероятно выбранной вершины 
увеличивается на~1, тем самым добавляя к~ней полуребро. Очевидно, что 
конструкция конфигурационного графа допускает наличие циклов, петель и~кратных 
ребер. Помимо этого, известно~\cite{Dur, Hof1, RN}, что такие графы асимптотически 
почти наверное состоят из более чем одной компоненты связности, а при значениях 
параметра~$\tau$, лежащих в~интервале $(1,2)$, такие графы имеют так называемую 
гигантскую компоненту связности, число вершин которой пропорционально~$N$ при 
$N\hm\rightarrow\infty$, тогда как объемы других компонент бесконечно малы по 
сравнению с~объемом гигантской компоненты.

Пусть степени вершин графа~--- независимые одинаково распределенные случайные 
величины, имеющие распределение~(\ref{eq1}). Тогда
\begin{equation}
\label{eq2}
p_k = {\bf P}\{\xi = k\} = \fr{1}{k^\tau} - \fr{1}{(k+1)^\tau}\,, \quad 
k=1,2,\ldots
\end{equation}
Каждой вершине графа поставим в~соответствие начинающийся с~одной частицы 
ветвящийся процесс Галь\-то\-на--Ват\-со\-на, в~котором распределение числа прямых 
потомков начальной частицы совпадает с~(\ref{eq2}). Обозначим~$\eta$ случайную 
величину, равную числу прямых потомков всех остальных частиц ветвящегося 
процесса. В~\cite{Hof2} показано, что в~этом случае
\begin{equation*}
{\bf P}\{\eta = k\} = \fr{(k+1)p_{k+1}}{{\bf E}\xi}, \quad k=0,1,2,\ldots
\end{equation*}

\setcounter{figure}{1}
\begin{figure*}[b] %fig2
\vspace*{1pt}
      \begin{center}
     \mbox{%
\epsfxsize=83.376mm 
\epsfbox{ler-2.eps}
}
\end{center}
\vspace*{-9pt}
\Caption{Пример нахождения максимального дерева с~корнем в~вершине~1 для графа на рис.~1}
\end{figure*}

Авторами разработан алгоритм нахождения дерева максимального объема с~корнем 
в~произвольно выбранной вершине графа. При построении  дерева\linebreak каждая вершина 
рассматривается вместе с~инцидентными ей ребрами. Вершины, смежные с~корнем, 
считаются вершинами 1-го поколения. Вершины, смежные с~вершинами 1-го поколения и~\mbox{отличные} от корня, образуют 2-е поколение. Вершины 3-го поколения смежны 
с~вершинами 2-го поколения и,~естественно, не принадлежат первым двум поколениям. 
Аналогично находятся и~вершины следующих поколений. Построение 
дерева по алгоритму происходит начиная с~корня от поколения к~поколению. Вершины 
очередного поколения присоединяются к~дереву только в~том случае, если их 
включение не приводит к~появлению кратных ребер и~циклов, включая петли.

Приведенный ниже алгоритм находит максимальное по объему дерево с~корнем в~$i$-й 
вершине и~вычисляет его объем~$v_i$. Обозначим символом~$\Box$ конец работы алгоритма, его 
результатом будет значение $v_i$. В~описании алгоритма~$l$ означает номер 
поколения, значение $l\hm=0$ соответствует корню, а~$k_l$ равно числу вершин 
поколения~$l$.

%\newpage
%\vspace{-2mm}
\noindent 
\begin{center}
{\bf Алгоритм}
\end{center}

\vspace*{-6pt}

\noindent
\begin{enumerate}
\item $i$~--- произвольно выбранная корневая вершина, $l\hm=0$.
\item Проверяем условие: корневая вершина имеет петлю или соединена кратными 
ребрами с~любой другой вершиной графа:
\begin{itemize}
\item если условие~2 выполнено, то $k_0\hm=0$, $v_i\hm=0$~~$\Box$;
\item если условие~2 не выполнено, то $k_0\hm=1$ и~переходим к~шагу~3.
\end{itemize}
\item Полагаем $l=l+1$.
\item Находим все вершины графа, смежные с~вершинами поколения $l\hm-1$, $k_l$~--- 
число вершин поколения~$l$:
\begin{itemize}
\item если $k_l=0$, то $v_i\hm=k_{l-1}$~~$\Box$;
\item если $k_l>0$, переходим к~шагу~5.
\end{itemize}
\item Для всех вершин поколения~$l$ проверяем условия: (1)~вершина имеет петлю; 
(2)~вершина соединена ребром с~другой вершиной поколения~$l$; (3)~вершина 
соединена кратными ребрами с~любой вершиной графа, не являющейся вершиной 
поколения~$l$; (4)~две вершины поколения~$l$ соединены ребрами с~одной любой 
вершиной графа, не являющейся вершиной поколения~$l$ за исключением вершин 
предыдущего поколения $l-1$:
\begin{itemize}
\item если для любой из вершин поколения~$l$ выполнено хотя бы одно из условий~(1)--(4), 
то $v_i\hm=\sum_{j=0}^{l-1}k_j$~~$\Box$;\\[-13.5pt]
\item если для всех вершин поколения~$l$ все условия~(1)--(4) не выполнены, 
переходим к~шагу~3.
\end{itemize}
\end{enumerate}

\begin{table*}[b]\small %tabl1
\vspace*{-12pt}
\begin{minipage}[t]{80mm}
\begin{center}
\parbox{50mm}{\Caption{Значения коэффициентов $a$ и~$b$ зависимостей вида~(\ref{eq3}) для некоторых
значений $\tau$ и~коэффициенты детерминации $R^2$ этих уравнений}
}

\vspace*{2ex}


\begin{tabular}{|l|r|r|r|}
\hline
&&&\\[-9pt]
\multicolumn{1}{|c|}{$\tau$} & \multicolumn{1}{c|}{$a$} & 
\multicolumn{1}{c|}{$b$} & \multicolumn{1}{c|}{$R^2$} \\ 
\hline
1,01 & 0,336 & 1,646 & 0,09 \\
1,05 & 0,466 & 1,097 & 0,13 \\
1,1 & 0,593 & 0,883 & 0,13 \\
1,3 & 1,792 & $-$3,677 & 0,26 \\
1,5 & 4,611 & $-$16,507 & 0,35 \\
1,8 & 13,614 & $-$61,531 & 0,48 \\
2,0 & 24,890 & $-$122,947 & 0,55 \\
2,3 & 53,950 & $-$291,503 & 0,54 \\
2,5 & 65,988 & $-$370,800 & 0,50 \\
3,0 & 17,177 & $-$86,828 & 0,41 \\
 \hline
\end{tabular}
\end{center}
\end{minipage}
%\end{table*}
\hfill
%\begin{table*}\small %tabl2
\begin{minipage}[t]{80mm}
\begin{center}
\parbox{50mm}{\Caption{Значения коэффициентов $a_l$ и~$b_l$ зависимостей вида~(\ref{eq4}) для некоторых
значений $\tau$ и~коэффициенты детерминации $R^2$ этих уравнений}
}

\vspace*{2ex}

\tabcolsep=7.8pt
\begin{tabular}{|l|r|r|r|}
\hline
&&&\\[-9pt]
\multicolumn{1}{|c|}{$\tau$} & \multicolumn{1}{c|}{$a_l$} & 
\multicolumn{1}{c|}{$b_l$} & \multicolumn{1}{c|}{$R^2$} \\ 
\hline
1,01 & 0,095 & 0,653 & 0,12 \\
1,05 & 0,117 & 0,580 & 0,17 \\
1,1 & 0,128 & 0,634 & 0,18 \\
1,3 & 0,211 & 0,580 & 0,32 \\
1,5 & 0,298 & 0,496 & 0,43 \\
1,8 & 0,418 & 0,343 & 0,59 \\
2,0 & 0,498 & 0,139 & 0,67 \\
2,3 & 0,655 & $-$0,603 & 0,75 \\
2,5 & 0,732 & $$1,181 & 0,78 \\
3,0 & 0,499 & $-$0,310 & 0,67 \\
 \hline
\end{tabular}
\end{center}
\end{minipage}
\end{table*}
    
\vspace{-9pt}


В качестве примера рассмотрим конфигурационный граф, состоящий из~20~вершин 
(рис.~1) и~с~по\-мощью приведенного выше алгоритма на этом\linebreak \mbox{графе} найдем 
максимальное дерево с~корнем в~вершине~1 (рис.~2).
В~данном примере первое поколение состоит из вершины с~номером 6.
Вершины второго поколения~--- 5, 14 и~15, третьего~--- 7, 8, 9, 10, 16, 18
 и~20. 
Таким образом, объем максимального дерева с~корнем в~вершине~1 ра\-вен~12~вершинам.

Далее под деревом $i$-й вершины будем понимать дерево максимального объема, 
найденное с~по-\linebreak\vspace*{-12pt}


{ \begin{center}  %fig1
 \vspace*{-6.5pt}
     \mbox{%
\epsfxsize=75mm %8.715mm 
\epsfbox{ler-1.eps}
}

\end{center}

\vspace*{-7pt}

\noindent
{{\figurename~1}\ \ \small{Пример конфигурационного графа, состоящего из 20~вершин
}}}

%\vspace*{6pt}

\addtocounter{figure}{1}

\noindent
мощью приведенного выше алгоритма.
Рас\-смот\-рев каждую $i$-ю вершину графа ($i\hm=\overline{1,N}$) в~качестве корня и~построив 
соответствующие корневые деревья, общее число которых равно числу вершин графа 
$N$, можно оценить следующие структурные характеристики: объем максимального 
дерева, средний объем дерева и~число деревьев заданного объема и~найти 
зависимости этих характеристик\linebreak от числа вершин графа $N$ и~параметра 
распределения степеней вершин~$\tau$.
Исследование проводилось посредством методов имитационного моделирования с~последующей статистической 
\mbox{обработкой} данных с~пом\-ощью 
программного обеспечения Wolfram  Mathematica~9.0.

\vspace*{-6pt}

\section{Результаты}

Рассматривались конфигурационные графы следующих объемов: $100\hm\leqslant N\hm\leqslant 1000$ 
с~шагом~$100$ и~$1500\hm\leqslant N\hm\leqslant 10\,000$ с~шагом~$500$. Значения параметра~$\tau$ изменялись от 
$1{,}01$ до~$3{,}0$
с шагом~$0{,}01$ для $1{,}01\hm\leqslant\tau\hm\leqslant 1{,}1$ и~$0{,}1$ для $1{,}2\hm\leqslant\tau\hm\leqslant 3{,}0$. Для 
каждой пары значений $(N,\tau)$ генерировалось по~$100$~графов (всего 
$81\,200$).

Распределение~(\ref{eq1}) имеет конечное математическое ожидание и~бесконечную 
дисперсию
при $\tau\hm\in(1,2]$, а~при $\tau\hm>2$ дисперсия конечна.
Поэтому при исследовании зависимостей рас\-смат\-ри\-ва\-емых структурных характеристик 
от~$N$ и~$\tau$ будем строить эти зависимости на двух интервалах изменения 
параметра распределения: $\tau\hm\in(1,2]$ и~$\tau\hm\in(2,3]$.

\vspace*{-6pt}

\subsection{Объем максимального дерева}

Объем максимального дерева для каждого графа вычислялся следующим образом:
\begin{equation*}
\nu_{\max}=\max\limits_{i}v_i, \quad i=\overline{1,N}\,.
\end{equation*}

Предельное поведение максимального объема дерева конфигурационного графа при 
$N\rightarrow\infty$ рассматривалось в~\cite{Pav}. Теорема~2 этой работы позволяет предположить, что при достаточно больших~$N$ и~$\tau\hm\in (1,2)$ максимальный объем пропорционален $\ln N$.
Проверим, что и~в~доасимптотической области эта характеристика ведет себя так 
же, как и~при $N\hm\rightarrow\infty$. Для этого построим следующие зависимости 
объема максимального дерева $\nu_{\max}$ от
числа вершин~$N$ при фиксированных значениях параметра~$\tau$:
\begin{equation}
\label{eq3}
\nu_{\max} = a \ln N + b\,.
\end{equation}

\noindent В табл.~1 приведены значения коэффициентов~$a$ и~$b$ регрессионных 
уравнений вида~(\ref{eq3})
для некоторых значений параметра~$\tau$ и~соответствующие коэффициенты 
детерминации~$R^2$ этих уравнений. Для всех уравнений гипотезы о равенстве 
коэффициентов детерминации нулю ($H_0: R^2=0$) отвергаются, и~все коэффициенты~$a$ 
и~$b$ в~табл.~1 значимы на уровне значимости~5\%.






Значения коэффициентов детерминации полученных зависимостей
выше всего при значениях~$\tau$ в~центре интервала $(1,3]$ и~ниже на его концах, 
причем $R^2\hm>0{,}5$ при $1{,}9\hm\leqslant\tau\hm\leqslant 2{,}5$.

Поиск регрессионной зависимости объема максимального дерева~$\nu_{\max}$ от 
числа вершин~$N$ при
фиксированных значениях параметра~$\tau$ с~более высокими коэффициентами 
детерминации привел к~зависимости другого вида:
\begin{equation}
\label{eq4}
\ln\nu_{\max} = a_l \ln N + b_l.
\end{equation}
Для полученных зависимостей вида~(\ref{eq4}) наблюдаем рост значений 
коэффициентов детерминации, и~здесь уже $R^2\hm>0{,}5$ при $1{,}6\hm\leqslant\tau\hm\leqslant 3{,}0$ (табл.~2).



\setcounter{figure}{2}
\begin{figure*} %fig3
\vspace*{1pt}
\begin{minipage}[t]{80mm}
      \begin{center}
     \mbox{%
\epsfxsize=79mm 
\epsfbox{ler-3.eps}
}
\end{center}
\vspace*{-9pt}
\Caption{Регрессионная зависимость~(\ref{eq5}) 
максимального объема дерева~$\nu_{\max}$ от $\tau\hm\in(1,2]$ при фиксированных 
значениях~$N$: \textit{1}~--- $N\hm=100$; \textit{2}~--- $1000$; 
\textit{3}~--- $5000$; \textit{4}~--- $N\hm=10\,000$}
\end{minipage}
%\end{figure*}
\hfill
%\begin{figure*} %fig4
\vspace*{1pt}
\begin{minipage}[t]{80mm}
      \begin{center}
     \mbox{%
\epsfxsize=79mm 
\epsfbox{ler-4.eps}
}
\end{center}
\vspace*{-9pt}
\Caption{Регрессионная зависимость~(\ref{eq6}) 
максимального объема дерева $\nu_{\max}$ от $\tau\hm\in(2,3]$ при фиксированных 
значениях~$N$: \textit{1}~--- $N\hm=100$; \textit{2}~--- $1000$; 
\textit{3}~--- $5000$; \textit{4}~--- $N\hm=10\,000$}
\end{minipage}
\end{figure*}



Оценив значимость различия между коэффициентами множественной корреляции 
регрессионных моделей~(\ref{eq3}) и~(\ref{eq4}) попарно при каждом фиксированном 
$\tau$ на 5\%-ном уровне значимости, получаем, что нулевые гипотезы 
$H_0:r_{(3)}=r_{(4)}$ ($r_{(3)}$ и~$r_{(4)}$~--- коэффициенты множественной 
корреляции зависимостей~(\ref{eq3}) и~(\ref{eq4}) соответственно) для каждой 
пары моделей отвергаются; следовательно, различие между коэффициентами 
корреляции значимо. И~так как значения коэффициентов детерминации моделей вида~(\ref{eq4}) выше, то можно сделать вывод, что эти модели <<лучше>> 
подходят для описания зависимости максимального объема дерева~$\nu_{\max}$ от~$N$ в~доасимптотической об\-ласти.
Таким образом, зависимость объема максимального дерева от объема графа~$N$ 
в~конфигурационных графах, число вершин в~которых меньше~$10\,000$, отличается от 
таковой при $N\hm\rightarrow\infty$.

Далее были найдены зависимости максимального объема дерева~$\nu_{\max}$ от 
размера графа~$N$ и~параметра распределения степеней вершин~$\tau$ для двух 
интервалов изменения~$\tau$:
\begin{itemize}
\item для $\tau\hm\in(1,2]$ (рис.~3):

\vspace*{-2pt}

\noindent
\begin{multline}
\label{eq5}
\ln\nu_{\max} = 0{,}205\ln N + 0{,}941\tau^2 - 1{,}013 \\
\left(R^2=0{,}95\right);
\end{multline}

\item для $\tau\hm\in(2,3]$ (рис.~4):

\vspace*{-2pt}

\noindent
\begin{multline}
\label{eq6}
\ln\nu_{\max} = \left(-0{,}397\tau^2+1{,}921\tau-1{,}728\right)\ln N \\ 
\left(R^2=0{,}97\right).
\end{multline}
\end{itemize}

\vspace*{-2pt}

\noindent
Здесь и~далее все коэффициенты полученных регрессионных уравнений значимы на 
уровне значимости~5\% и~для всех приведенных ниже коэффициентов детерминации~$R^2$ 
полученных моделей гипотезы о~равенстве этих коэффициентов детерминации 
нулю ($H_0: R^2=0$) отвергаются.




На рис.~3 и~4 все кривые зависимостей~(\ref{eq5}) и~(\ref{eq6}) при 
фиксированных значениях $N$ лежат внутри затененных областей и~расположены одна 
над другой по мере роста числа вершин графа в~пределах граничных значений 
$100\hm\leqslant N\hm\leqslant 10\,000$.

Таким образом, согласно полученным результатам, объем максимального дерева 
растет с~увеличением размерности графа и~с ростом значения параметра на интервале 
$\tau\hm\in(1,2]$,
достигая максимальных значений при $\tau\hm=2$.
На интервале $\tau\hm\in(2,3]$ вначале рост~$\nu_{\max}$ продолжается, достигая 
своего максимума
при $\tau\hm\approx 2{,}4$, не превышая, однако, 250~вершин при $N\hm=10\,000$, и~далее 
уменьшается с~приближением к~правой границе интервала.

\vspace*{-4pt}

\subsection{Средний объем дерева}

Следующей рассматривалась такая характеристика, как средний объем дерева, 
который для каж\-до\-го графа вычислялся по формуле
\begin{equation*}
\bar{\nu}=\fr{1}{N}\sum\limits_{i=1}^N v_i\,.
\end{equation*}
Были построены регрессионные зависимости~$\bar{\nu}$ от~$N$ и~$\tau$ при 
$\tau\hm\in(1,2]$ и~$\tau\hm\in(2,3]$:
\begin{alignat*}{2}
\ln\bar{\nu} &= 0{,}039\ln N + 0{,}733 \tau^2 - 1{,}079 &\ (R^2&=0{,}98);
\\
\ln\bar{\nu} &= (-0{,}369 \tau^2+1{,}745 \tau-1{,}745)\ln N &\ (R^2&=0{,}95).
\end{alignat*}
Эти модели показывают, что средний объем дерева ведет себя примерно так же, как и~максимальный объем с~ростом~$N$ и~$\tau$, достигая своих максимальных значений 
также при $\tau\hm\approx 2{,}4$, однако сами значения среднего объема дерева 
значительно меньше и~не превышают 20~вершин при $N\hm=10\,000$.



\subsection{Число деревьев заданного объема}

Пусть $r=0,1,2,\ldots$~--- натуральное число, соответствующее объему дерева 
при произвольной вершине графа $i\hm=\overline{1,N}$. Число деревьев объема~$r$ обозначим 
$n_r$.
Объем дерева $r\hm=0$ соответствует случаю, когда корневая вершина либо имеет 
петлю, либо связана с~любой другой вершиной графа кратными ребрами. Дерево 
объема $r\hm=1$ состоит только из корневой вершины.

В случае $\tau\hm\in(1,2]$ получена следующая регрессионная зависимость~$n_0$ от 
объема графа~$N$ и~параметра распределения степеней вершин $1\hm<\tau\hm\leqslant 2$:
\begin{equation*}
n_0 = 14{,}067+\fr{0{,}186N}{\tau^{5,371}} \quad (R^2=0{,}99).
\end{equation*}

При $\tau\in(2,3]$ для случаев $r\hm=0$ и~$1$ получены следующие регрессионные 
зависимости~$n_0$ и~$n_1$ от объема графа~$N$ и~параметра~$\tau$:
\begin{align*}
n_0 &= \fr{184{,}515\ln N}{\tau^{7{,}279}};
\\
\ln n_1& = 0{,}490\ln N - 8{,}821\ln\tau + 7{,}393
\end{align*}
с коэффициентами детерминации $0{,}93$ и~$0{,}96$ соответственно.

Далее были получены регрессионные зависимости~$n_r$ от~$r$, объема графа~$N$ и~параметра 
распределения
степеней вершин~$\tau$ для двух интервалов изменения параметра:
\begin{equation}
\label{eq7}
n_r = 9{,}754+\fr{0{,}641 N^{1{,}021}}{\tau^{1{,}866} r^{2{,}060}}, \quad r\geqslant 1\,, 
\quad \tau\in(1,2];
\end{equation}

\vspace*{-12pt}

\noindent
\begin{multline}
\label{eq8}
\hspace*{-1mm}n_r = 2{,}959+\fr{1{,}604\,N^{0,994}\,\tau^{1,184}}{r^{3,148}},\hspace*{1mm} \\
 r\geqslant 2,  \enskip \tau\in(2,3],
\end{multline}
с коэффициентами детерминации $0,87$ и~$0,99$ соответственно.
На рис.~5 и~6 представлены графики зависимостей (\ref{eq7}) и~(\ref{eq8}) при 
$N=10\,000$ и~некоторых фиксированных значениях $r$.



Из полученных результатов следует, что число деревьев заданного объема~$n_r$ 
увеличивается с~рос\-том чис\-ла вершин графа и~уменьшается с~увеличением значения~$r$. 
Что касается за\-ви\-си\-мости~$n_r$ от па\-ра\-мет\-ра распределения степеней вершин~$\tau$, 
то чис\-ло деревьев~$n_r$ с~ростом значения па\-ра\-мет\-ра сначала уменьшается 
на интервале $\tau\hm\in(1,2]$, а~затем на интервале $\tau\hm\in(2,3]$ увеличивается.

В конфигурационных графах с~параметром распределения степеней вершин 
$\tau\hm\in(1,2]$ больше всего деревьев объема $r\hm=1$, однако их чис\-ло при 
$N\hm=10\,000$ не превышает~8000~вершин. С~рос\-том значения~$r$ чис\-ло деревьев~$n_r$ 
резко уменьшается.\linebreak\vspace*{-12pt}

{ \begin{center}  %fig5
 \vspace*{-3pt}
    \mbox{%
\epsfxsize=79mm 
\epsfbox{ler-5.eps}
}

\end{center}



\noindent
{{\figurename~5}\ \ \small{Регрессионная зависимость~(\ref{eq7}) $n_r$~--- 
числа деревьев объема $r\hm\geqslant 1$ от $\tau\hm\in(1,2]$ при $N\hm=10\,000$ при 
фиксированных значениях~$r$: \textit{1}~--- $r\hm=1$;  \textit{2}~--- $2$;  \textit{3}~--- 
$3$;  \textit{4}~--- $4$;  \textit{5}~--- $r\hm=10$
}}}

\vspace*{6pt}

{ \begin{center}  %fig6
 \vspace*{1pt}
     \mbox{%
\epsfxsize=79mm 
\epsfbox{ler-6.eps}
}

\end{center}



\noindent
{{\figurename~6}\ \ \small{Регрессионная зависимость~(\ref{eq8}) $n_r$~--- 
числа деревьев объема $r\hm\geqslant 2$ от $\tau\hm\in(2,3]$ при $N\hm=10\,000$ при 
фиксированных значениях~$r$:  \textit{1}~--- $r\hm=2$;  \textit{2}~--- $3$;  \textit{3}~--- 
$4$;  \textit{4}~--- $5$;  \textit{5}~--- $r\hm=10$
}}}

\vspace*{6pt}


\noindent
 Так, для графов объема $N\hm=10\,000$ при $r\hm=2$ оно не будет 
превышать~2000, при $r\hm=3$~--- 1000, при $r\hm=5$~--- 300, а~при $r\hm>10$ число 
деревьев~$n_r$ становится меньше~100.
При $\tau\hm\in(2,3]$ больше всего деревьев объема $r\hm=2$ и~их чис\-ло при $N\hm=10\,000$ 
не превышает~7000. На этом интервале~$n_r$ также уменьшается с~рос\-том значения~$r$. 
Для графов объема $N=10\,000$ при $r\hm=3$ оно не будет превышать~2000, при 
$r\hm=4$~--- 1000, при $r\hm=5$~--- 400, а~при $r\hm>10$ чис\-ло деревьев~$n_r$ становится 
меньше~50.

\vspace*{-6pt}


\section{Заключение}


В результате проведенных исследований локальной древовидности конфигурационных 
графов впервые были по\-стро\-ены модели зависимостей максимального и~среднего 
объемов деревьев от числа вершин графа~$N$ и~па\-ра\-мет\-ра~$\tau$ 
распреде-\linebreak\vspace*{-12pt}

\pagebreak

\noindent
ления~(\ref{eq1}). Также были построены модели зависимостей чис\-ла 
деревьев заданного объема~$r$ от~$N$, $\tau$ и~$r$. Проведение экспериментов, 
необходимых для формирования таких моделей, требует значительных вы\-чис\-ли\-тель\-ных 
ресурсов, поэтому рас\-смат\-ри\-ва\-лись графы с~ограничением $N\hm\leqslant 10\,000$. Для 
максимального объема дерева проведено сравнение по\-стро\-ен\-ных моделей 
с~асимп\-то\-ти\-че\-ски\-ми результатами работы~\cite{Pav}. Оказалось, что асимптотическая 
модель~(\ref{eq3}) применима и~в~до\-асимп\-то\-ти\-че\-ской об\-ласти, но предложенная 
модель~(\ref{eq5}) ста\-ти\-сти\-че\-ски значимо превосходит~(\ref{eq3}). Естественно 
поставить вопрос о~на\-хож\-де\-нии достаточно большого объема графа, начиная 
с~которого асимп\-то\-ти\-че\-ская модель получит преимущество. Возможно, что ответ на 
этот вопрос мож\-но будет получить в~будущем, если использовать разработанную 
авторами имитационную модель и~гораздо более мощную вы\-чис\-ли\-тель\-ную технику.


{\small\frenchspacing
 {\baselineskip=11.5pt
 %\addcontentsline{toc}{section}{References}
 \begin{thebibliography}{9}
 
 \bibitem{Fa} %1
\Au{Faloutsos~C., Faloutsos~P., Faloutsos~M.} On power-law relationships of
the internet topology~// Comput. Commun. Rev., 1999. Vol.~29. P.~251--262.
doi: 10.1145/ 316194.316229.

\bibitem{Dur} %2
\Au{Durrett~R.} Random graph dynamics.~--- Cambridge: Cambridge University
Press, 2007. 221~p. doi: 10.1017/ CBO9780511546594.

\bibitem{Hof1} %3
\Au{Hofstad~R.} Random graphs and complex networks.~--- Cambridge:
Cambridge University Press, 2017. Vol.~1. 337~p. doi: 10.1017/9781316779422.

\bibitem{Hof2} %4
\Au{Hofstad~R.} Random graphs and complex networks~// Notes RGCNII, 2023. 
Vol.~2.
314~p. {\sf https://www.win.tue. nl/$\sim$rhofstad/NotesRGCNII.pdf.}



\bibitem{Bol}
\Au{Bollobas~B.} A~probabilistic proof of an asymptotic formula for the 
number of labelled regular graphs~// Eur. J. Combin., 1980. Vol.~1.
Iss.~4. P.~311--316. doi: 10.1016/S0195-6698(80)80030-8.

\bibitem{RN}
\Au{Reittu~H., Norros~I.} On the power-law random graph model of massive
data networks~// Perform. Evaluation, 2004. Vol.~55. Iss.~1-2. P.~3--23.
doi: 10.1016/S0166-5316(03)00097-X.

\bibitem{Pav}
\Au{Павлов~Ю.\,Л.} Максимальное дерево случайного леса в~конфигурационном 
графе~//
Математический сборник, 2021. Т.~212. Вып.~9. C.~146--163. doi: 10.4213/ sm9481.

\bibitem{PavCh}
\textit{Павлов~Ю.\,Л., Чеплюкова~И.\,А.} Объемы деревьев случайного леса и~конфигурационные графы~//
Труды Математического института им.~В.\,А.~Стеклова, 2022. Т.~216. C.~298--315. 
doi: 10.4213/tm4216.

\bibitem{Pav2}
\Au{Pavlov~Yu.\,L.} Random forests.~--- Utrecht: VSP, 2000. 122~p.

\end{thebibliography}

 }
 }

\end{multicols}

\vspace*{-10pt}

\hfill{\small\textit{Поступила в~редакцию 19.06.23}}

\vspace*{8pt}

%\pagebreak

%\newpage

%\vspace*{-28pt}

\hrule

\vspace*{2pt}

\hrule



\def\tit{LOCAL TREELIKE STRUCTURE IN~THE~POWER-LAW~CONFIGURATION~GRAPHS}


\def\titkol{Local treelike structure in~the~power-law configuration  graphs}


\def\aut{M.\,M.~Leri and~Yu.\,L.~Pavlov}

\def\autkol{M.\,M.~Leri and~Yu.\,L.~Pavlov}

\titel{\tit}{\aut}{\autkol}{\titkol}

\vspace*{-10pt}


\noindent 
Institute of Applied Mathematical Research,
Karelian Research Center of the Russian Academy of Sciences, 11~Pushkinskaya Str.,
Petrozavodsk 185910, Russian Federation

\def\leftfootline{\small{\textbf{\thepage}
\hfill INFORMATIKA I EE PRIMENENIYA~--- INFORMATICS AND
APPLICATIONS\ \ \ 2024\ \ \ volume~18\ \ \ issue\ 1}
}%
 \def\rightfootline{\small{INFORMATIKA I EE PRIMENENIYA~---
INFORMATICS AND APPLICATIONS\ \ \ 2024\ \ \ volume~18\ \ \ issue\ 1
\hfill \textbf{\thepage}}}

\vspace*{3pt}



\Abste{The local treelike structure of 
configuration graphs intended for modeling complex communication networks is stidued. In 
such graphs, the vertex degrees are independent and identically distributed 
according to the power law. In the case of a~limited number of graph vertices, the 
dependences of the maximum volume of a~treelike subgraph on the number of graph 
vertices and the vertex degree distribution parameter are found. The same 
problem was solved for the number of trees of a~given size. Estimates are also 
given for the average size of a~tree in the graph. It is shown that with a limited 
number of vertices of the configuration graph, the found dependences 
statistically significantly improve the description of the network structure in 
comparison with the previously known asymptotic models.}

\KWE{configuration graph; power-law distribution; local 
treelike structure;
tree size; simulations}


\DOI{10.14357/19922264240107}{ITTZEO}

%\vspace*{-12pt}

\Ack

\vspace*{-3pt}


     \noindent
     The study was carried out under state order to the Karelian Research Center of 
the Russian Academy
of Sciences (Institute of Applied Mathematical Research KarRC RAS).

\pagebreak


  \begin{multicols}{2}

\renewcommand{\bibname}{\protect\rmfamily References}
%\renewcommand{\bibname}{\large\protect\rm References}

{\small\frenchspacing
 {%\baselineskip=10.8pt
 \addcontentsline{toc}{section}{References}
 \begin{thebibliography}{9} 
 
 \bibitem{4-ler} %1
\Aue{Faloutsos, C., P.~Faloutsos, and M.~Faloutsos.} 1999. On power-law 
relationships of
the internet topology. \textit{Comput. Commun. Rev.} 29:251--262. 
doi: 10.1145/316194.316229.


\bibitem{1-ler} %2
\Aue{Durrett, R.} 2007. \textit{Random graph dynamics.} Cambridge: Cambridge 
University Press. 221~p. doi: 10.1017/ CBO9780511546594.

\bibitem{2-ler} %3
\Aue{Hofstad, R.} 2017. \textit{Random graphs and complex networks.} 
Cambridge:
Cambridge University Press. Vol.~1. 337~p. doi: 10.1017/9781316779422.

\bibitem{3-ler} %4
\Aue{Hofstad, R.} 2023. Random graphs and complex networks. \textit{Notes 
RGCNII.} Vol.~2. 314~p.
Available at: {\sf https://www.win. tue.nl/$\sim$rhofstad/NotesRGCNII.pdf} 
(accessed April~17, 2023).


\bibitem{5-ler}
\Aue{Bollobas, B.} 1980. A~probabilistic proof of an asymptotic formula for 
the number
of labelled regular graphs. \textit{Eur. J. Combin.} 1(4):311--316. 
doi: 10.1016/S0195-6698(80)80030-8.

\bibitem{6-ler}
\Aue{Reittu, H., and I.~Norros.} 2004. On the power-law random graph model 
of massive data
networks. \textit{Perform. Evaluation.} 55(1-2):3--23. doi: 10.1016/S0166-5316(03)00097-X.

\bibitem{7-ler}
\Aue{Pavlov, Yu.\,L.} 2021. The maximum tree of a~random forest in the 
configuration graph.
\textit{Sb. Math.} 212(9):1329--1346. doi: 10.1070/SM9481.

\bibitem{8-ler}
\Aue{Pavlov, Yu.\,L., and I.\,A.~Cheplyukova.} 2022. Sizes of trees in 
a~random forest and configuration graphs.
\textit{P.~Steklov Inst. Math.} 316:280--297. doi: 10.1134/S0081543822010205.

\bibitem{9-ler}
\Aue{Pavlov,~Yu.\,L.} 2000. \textit{Random forests.} Utrecht: VSP. 122~p.
\end{thebibliography}

 }
 }

\end{multicols}

\vspace*{-6pt}

\hfill{\small\textit{Received June 19, 2023}} 

\vspace*{-18pt}
     
     \Contr
     
     \vspace*{-3pt}

\noindent
\textbf{Leri Marina M.} (b.\ 1969)~--- Candidate of Science (PhD) in technology, 
scientist,
Institute of Applied Mathematical Research of the Karelian Research Center of 
the Russian Academy of Sciences,
11~Pushkinskaya Str., Petrozavodsk 185910, Russian Federation; 
\mbox{leri@krc.karelia.ru}

\vspace*{3pt}

\noindent
\textbf{Pavlov Yuri L.} (b.\ 1949)~--- Doctor of Science in physics and 
mathematics, professor, principal scientist, Institute of Applied Mathematical 
Research of the Karelian Research Center of the Russian Academy of Sciences, 11~Pushkinskaya Str., Petrozavodsk 185910, Russian Federation; 
\mbox{pavlov@krc.karelia.ru}



\label{end\stat}

\renewcommand{\bibname}{\protect\rm Литература} 