\def\stat{torshin}

\def\tit{О ПОРОЖДЕНИИ СИНТЕТИЧЕСКИХ ПРИЗНАКОВ\\ НА~ОСНОВЕ ОПОРНЫХ ЦЕПЕЙ 
И~ПРОИЗВОЛЬНЫХ МЕТРИК\\ В~РАМКАХ ТОПОЛОГИЧЕСКОГО ПОДХОДА 
К~АНАЛИЗУ ДАННЫХ.\\ ЧАСТЬ~1. ВКЛЮЧЕНИЕ В~ФОРМАЛИЗМ\\ ЭМПИРИЧЕСКИХ 
ФУНКЦИЙ РАССТОЯНИЯ$^*$}

\def\titkol{О порождении синтетических признаков на~основе опорных цепей 
и~произвольных метрик. % в~рамках топологического подхода к~анализу данных. 
Часть~1} %. Включение в~формализм эмпирических функций расстояния}

\def\aut{И.\,Ю.~Торшин$^1$}

\def\autkol{И.\,Ю.~Торшин}

\titel{\tit}{\aut}{\autkol}{\titkol}

\index{Торшин И.\,Ю.}
\index{Torshin I.\,Yu.}


{\renewcommand{\thefootnote}{\fnsymbol{footnote}} \footnotetext[1]
{Работа выполнена при поддержке гранта РНФ (проект №\,23-21-00154) с~использованием инфраструктуры 
Центра коллективного пользования <<Высокопроизводительные вычисления и~большие данные>> (ЦКП 
<<Информатика>>) ФИЦ ИУ РАН (г.~Москва).}}


\renewcommand{\thefootnote}{\arabic{footnote}}
\footnotetext[1]{Федеральный исследовательский центр <<Информатика и~управление>> Российской академии наук, 
\mbox{tiy135@yahoo.com}}

%\vspace*{-12pt}



  



\Abst{Анализ формализма топологической теории распознавания на основе 
фундаментальных понятий функционального анализа позволил предложить ранее не 
исследованные подходы к~определению решеточных оценок. В~частности, использование 
опорных цепей для анализа булевых решеток, формируемых над регулярными по Журавлёву 
множествами прецедентов, указало на новое направление исследований, заключающееся 
в~замене оценок элементов решеток на определенного рода функции и/или векторы. Данное 
расширение формализма также позволяет проводить систематическое исследование 
известных в~литературе полуэмпирических функционалов расстояния для решения 
прикладных задач. Обоснованы перспективные направления дальнейшего развития 
формализма, включающие введение функционалов, редуцирующих описания множеств 
булевой решетки к~скалярным оценкам, и~развитие математического аппарата для анализа 
решеток, в~котором вместо оценок фигурируют операции над соответствующими 
функциями. Последнее направление интересно тем, что позволяет вводить расстояния на 
решетке без использования оценок.} 

\KW{топологический анализ данных; теория решеток; алгебраический подход 
Ю.\,И.~Жу\-рав\-лё\-ва; функциональный анализ}

\DOI{10.14357/19922264240110}{RIVOXR}
  
%\vspace*{-6pt}


\vskip 10pt plus 9pt minus 6pt

\thispagestyle{headings}

\begin{multicols}{2}

\label{st\stat}

\section{Введение}

    Топологическая теория распознавания стала\linebreak развитием алгебраического 
подхода к~распознаванию, предложенного научной школой Ю.\,И.~Жу\-равлё\-ва, 
и~предназначена для решения плохо формализованных задач распознавания, 
\mbox{классификации} и~прогнозирования~[1]. Одна\linebreak из основных целей данной тео\-рии~--- 
разработка методов систематического порождения и~отбора \mbox{синтетических} 
признаковых описаний объектов, которые бы характеризовались большей 
информативностью, чем исходные признаки~\cite{2-tr}. В~работе~\cite{3-tr} 
было показано, что веса таких признаков можно эффективно настраивать 
посредством ранговой оптимизации; разработан формализм для по\-рож\-де\-ния 
признаковых описаний на основе па\-ра\-мет\-ри\-зи\-ру\-емых решеточных оценок~[4].
     
     В рамках развиваемого формализма каждый объект~$x$ из множества 
исходных описаний $N_0$ объектов $\mathbf{X}\hm= \{x_1, \ldots , 
x_{N_0}\}$, $\mathbf{X}\hm\subseteq \mathrm{S}$, опи\-сы\-ва\-емый $n$ 
признаками посредством функций $\Gamma_k: \mathrm{S}\hm\to \mathrm{I}_k$ 
(где $\mathrm{I}_k\hm= \{ \lambda_{k_1}, \lambda_{k_2}, \ldots\}$~--- 
множества значений признаковых описаний), представлен множествами 
     $\{ \Gamma_k^{-1}(\Gamma_k(x))\}$, $k\hm=\overline{1, n+l}$, где $l$~--- 
число таргетных (прогнозируемых) переменных. Значение $t$-й целевой 
(таргетной) переменной объекта~$x$, $t\hm = \overline{n+1, n+l}$, 
вычисляется как~$\Gamma_t(x)$. Множество прецедентов над пространством 
допустимых признаковых описаний объектов~$J_{\mathrm{ob}}$ 
опре\-де\-ля\-ет\-ся~как 
$$
Q=\varphi(\mathbf{X})\hm=\{ D(\mathrm{x}_\alpha)\vert 
\mathrm{x}_\alpha \hm\in \mathbf{X}\}
$$ 
посредством $D:\mathrm{S}\hm\to 
J_{\mathrm{ob}}$ и~$\varphi(\mathbf{X})\hm= \{ D(\mathrm{x}_\alpha ) \vert 
\mathrm{x}_\alpha\hm\in \mathbf{X}\}$, $D(\mathrm{x}_\alpha) \hm= 
(\Gamma_1(\mathrm{x}_\alpha \times \cdots\times \Gamma_k(\mathrm{x}_\alpha) 
\times\cdots\times \Gamma_{n+1}(\mathrm{x}_\alpha))_\Delta$. При 
регулярности множеств $\mathbf{X}/Q$ ($\forall\,x\hm\in \mathbf{X}, 
x\hm= {D}^{-1}({D}(\mathrm{x})))\mathbf{X}$ изоморфно~$Q$ 
и~обоим множествам сопоставлены \textit{топология} $T(\mathbf{X})\hm= 
\{\emptyset, \{\mathbf{X}\}, a\hm\cup b, a\hm\cap b: a,b\hm\in U(\mathbf{X})\hm= 
\{\Gamma_k^{-1}(\lambda_{k_b})\}\}$ и~\textit{булева решетка} 
$L(T(\mathbf{X}))\hm= \{a\vee b, a\wedge b: a,b\hm\in  
T(\mathbf{X})\}$~\cite{5-tr}.
     
     Рассмотрим топологический подход к~распознаванию с~точки зрения 
фундаментальных понятий теории функций~--- рефлексивных и~транзитивных 
бинарных отношений между множествами. В~математике применяются два 
таких отношения: симметричное отношение \textit{эквивалентности} множеств 
и~антисимметричное отношение (частичного) \textit{порядка}~\cite{6-tr}. 
     
     Прецедентному соотношению между значениями 
признака~$\Gamma_k(x)$ и~$t$-й таргетной переменной, заданному 
множеством~$Q$ на решетке $L(T(\mathbf{X}))$, соответствует множество пар 
$\{ ( \{ \Gamma_k^{-1}(\Gamma_k(x_i)), k\hm=\overline{1,  n}\},  
\Gamma_t^{-1}(\Gamma_t(x_i))), i\hm= \overline{1, N_0}\}$. На основании 
представленных в~регулярном множестве~$Q$ прецедентов любой алгоритм 
распознавания (<<машинного обучения>>) строит некоторую модель 
соотношения между множествами $\{ \Gamma_k^{-1}(\Gamma_k(x_i))\}$  
и~$\Gamma_t^{-1}(\Gamma_t(x_i))$.
     
     Сфокусируемся на двух произвольных множествах $a\hm= \Gamma_k^{-
1}(\Gamma_k(x_i))$ и~$b\hm= \Gamma_t^{-1}(\Gamma_t(x_i))$. Очевидно, что 
говорить о выполнимости отношения эквивалентности или порядка между 
множествами~$a$ и~$b$ в~общем случае не приходится: ведь эквивалентность 
соответствует идентичности значения $k$-го признака значению $t$-й 
таргетной переменной, а~час\-тич\-ный  
порядок~--- ядерной эквивалентности (т.\,е.\ эквивалентности~$a$ 
подмножеству~$b$ или наоборот). Такие случаи тривиальны и~соответствуют 
<<легко решаемым>> задачам распознавания.
     
     В то же время существующие в~решетке $L(T(\mathbf{X}))$ отношения 
порядка порождают супремум $a\vee b$ и~инфимум $a\wedge b$ множеств~$a$ 
и~$b$. Поэтому в~топологической теории распознавания вводятся более\linebreak 
сложные функционалы над $a$, $b$, $a\vee b$ и~$a\wedge b$, расширенно 
описывающие соотношения между произвольными множествами~$a$ и~$b$ 
в~терминах \textit{расстояний}. При выполнимости четырех аксиом \mbox{мет\-ри\-ки} такие 
функционалы $\rho_L: L^2\hm\to R^+$ формируют \textit{метрическое 
пространство значений признаков} $M_L(L(T(\mathbf{X})), \rho_L)$.
     
     Простейшей метрикой служит функционал 
     $$
     \rho_0(a,b) \hm= \fr{v[a\vee b] 
\hm- v[a\wedge b]}{N_0}\,,
$$
 где $v: L\hm\to R^+$~--- \textit{изотонная оценка} на 
$L(T(\mathbf{X}))$. Возможны и~более сложные варианты 
определения~$\rho_L$ при введении параметрических оценок~\cite{4-tr} или 
использовании известных в~литературе метрик, так что в~общем случае имеется 
ряд метрик~$\rho_m$, $m\hm=\overline{1, m_0}$. Как правило, метрики~$\rho_m$ 
нормируются на интервал значений $[0\ldots 1]$.
     
     Возвращаясь к~рассмотрению фундаментальных отношений теории 
функций, можно сделать вывод о~том, что метрика $\rho_L(a,b)$ служит 
функционалом, численно оценивающим выполнимость отношения 
эквивалентности между~$a$ и~$b$ на основе отношений порядка (заданных 
в~форме $a\vee b$, $a\wedge b$). Действительно, $\rho_L(a,b)\hm=0$ 
соответствует строгому выполнению отношения эквивалентности~$a$ и~$b$, 
а~$\rho_L(a,b)\hm=1$~--- максимально возможному расстоянию между~$a$ 
и~$b$ (например, $\rho_0(a,b)\hm=1$ только для множеств~$\emptyset$ 
и$\{\mathbf{X}\}$, находящихся на концах максимальных цепей решетки 
$L(T(\mathbf{X}))$). 
     
     Таким образом, в~рамках топологической тео\-рии распознавания 
соотношение между множествами $\{\Gamma_k^{-1}(\Gamma_k(x_i))\}$  
и~$\Gamma_t^{-1}(\Gamma_t(x_i))$ моделируется соответствующими 
массивами расстояний, по\-рож\-да\-емы\-ми той или иной метрикой~$\rho_m$. 
Рассмотрим способы вычисления таких расстояний.

\section{О различных способах вычисления метрик на~решетке~$L(T(\mathbf{X}))$}
    
В литературе известны три принципиально различных способа определения 
метрических расстояний:
\begin{enumerate}[(1)]
\item метрики на основе операций над мно\-же\-ст\-вами; 
\item метрики над пространством векторов; 
\item метрики над пространством 
функций. 
\end{enumerate}
Рассмотрим три этих подхода в~применении к~описанным выше 
конструктам топологической теории распознавания.

    
\textbf{Метрики на основе операций над множествами}. Во введении были 
упомянуты метрики оценки расстояния между $a,b \hm\in L(T(\mathbf{X}))$, 
вводимые как функционалы над $a\vee b$ (соответствует $a\cup b$) и~$a\wedge 
b$ ($a\cap b$), оценками высот элементов в~$L(T(\mathbf{X}))$ и~другими  
тео\-ре\-ти\-ко-мно\-жест\-вен\-ны\-ми операциями над множествами~$a$ и~$b$. 
В~работе~\cite{4-tr} для порождения мет\-рик используются взвешенные 
решеточные оценки 
$$
v_\alpha = \sum\limits_{i=0,\vert\bm{\alpha}\vert} 
\omega_i v_{\alpha_i}
$$ 
на основе изотонных оценок~$v_{\alpha_i}$. 
Формирование наборов~$\bm{\alpha}$ может проводиться на основе 
<<информативности>> $\alpha_i\hm\in \bm{\alpha}$ методами метрического 
анализа данных~\cite{7-tr} или на основе различных подцепей таргетных 
числовых переменных.
     
     В то же время известны многочисленные функционалы, носящие 
эмпирический характер и~непосредственно оперирующие множествами: 
расстояния Танимото, Рэнда, Рас\-се\-ла--Рао, Симпсона, Бра\-уна--Блан\-ке,  
Род\-же\-ра--Та\-ни\-мо\-то, Фэйта, дисперсии, образов, $Q_0$, Пирсона; 
различные варианты расстояний Тверского, Со\-ка\-ла--Сни\-са, Гоу\-эра--Ле\-жанд\-ра, Юле и~др.~\cite{8-tr}. Метрические свойства этих функционалов 
могут быть продемонстрированы посредством аналитических выводов или 
комбинаторного анализа на множествах прецедентов.
    
\textbf{Векторные метрики}. Альтернативно методу взвешенных решеточных 
оценок на основании набора~$\bm{\alpha}$ и~оценок~$v_{\alpha_i}$ для 
произвольного множества $a\hm\in L(T(\mathbf{X}))$ может быть вычислен 
вектор 
$$
\vec{v}_{\bm{\alpha}} [a]= (v_{\alpha_1}[a], v_{\alpha_2}[a], \ldots , 
v_{\alpha_i}[a],\ldots),\enskip v_{\alpha_i}[a]\hm\in R^+,
$$
 и~введены метрики на 
пространстве векторов~$\vec{v}_{\bm{\alpha}}$\linebreak   посредством известных подходов: $l_1$-мет\-ри\-ка, 
$l_p$-мет\-ри\-ки Минковского, расстояния Пенроуза, Манхэттена, Лоренца, 
Кларка, Хеллинджера, Уайттеккера, симметрическое~$\chi^2$, Махаланобиса 
(в~том числе с~настраиваемыми весами), расстояния пересечения, Ружечки, 
Робертса, Элленберга, Глисона, Мотыки, Брея--Кур\-ти\-са, Канберры, 
Кульчинского и~корреляционные расстояния (ковариационное, 
корреляционное, косинус, угловое, хордовое, подобности,  
Мо\-ри\-си\-ты--Хор\-на, Спирмана, Кендалла)~\cite{8-tr}.
    
\textbf{Метрики над пространством функций}. Функциональный анализ 
и~теория вероятностей предо\-став\-ля\-ют широкий инструментарий для 
определения расстояний между функциями с~одинаковой областью 
определения: функционалы Колмогорова (в~том числе максимальное 
уклонение~$D$), фон\linebreak Мизеса, Реньи, метрики (интегральная~$L_1$, 
инженерная, разделения, подобности среднего гармонического), расстояния 
Чебышёва, Степанова, варианты расстояний Золотарёва, Круглова,  
Бур\-би--Рао, Бхаттачарья, Чизара (включая вариации расстояний Куль\-ба\-ка--
Лейб\-ле\-ра, $\chi^2$, Хеллинджера)~\cite{8-tr}.
     
     Очевидно, что некоторые из этих функций расстояний заведомо не 
относятся к~метрикам. Например, в~расстоянии Куль\-ба\-ка--Лейб\-ле\-ра 
нарушена аксиома симметричности; оценка \mbox{выполнимости} аксиомы 
треугольника для каждой из этих функций требует отдельного исследования. 
Метриза\-ция рассматриваемых функций расстояния может\linebreak осуществляться 
посредством введения дополнительных конструкций в~определение функции. 
В~част\-ности, при отсутствии симметричности для функции $d(x,y)$ могут быть 
введены конструкты $\min(d(x,y), d(y,x))$, $\max(d(x,y), d(y,x))$ и~др. 
Основной проблемой все же остается <<привязка>> этих подходов 
к~разрабатываемому решеточному формализму. Для этого напомним ряд 
важных понятий.
     
     \smallskip
     
     \noindent
     \textbf{Определение~1.} Решеточный терм, или \textit{изотонная 
оценка}, $v: L\hm\to R^+$ над $L(T(\mathbf{X}))$~--- функция, для которой 
выполнено \textit{условие оценки} (\textbf{уО}:  $\forall_L a,b: 
v[a]\hm+v[b]\hm=v[a\wedge b]\hm+ v[a\vee b]$) и~\textit{условие изотонности} 
(\textbf{уИ}: $\forall_L a,b: a \hm\supseteq b \hm\Rightarrow  v[a]\hm\geq v[b]$). Для изотонной 
$v[\,]$ гарантировано существование метрики $\rho_0(a,b)$.
     \smallskip
     
     \noindent
     \textbf{Определение~2.} \textit{Однородными функциями} будем 
называть произвольные функции с~одинаковыми областями определения и~значений. 
     
     \smallskip
     \noindent
     \textbf{Определение~3.} Пусть задано конечное множество чисел 
$A\hm= \{a_1,a_2, \ldots , a_i, \ldots, a_n\}$, $a_i\hm\in R$. Определим 
\textit{оператор $\hat{\phi}(x)$ для формирования эмпирической функции 
распределения} (э.\,ф.\,р.)\ чисел в~множестве~$A$ как $\hat{\phi}(x) A\hm= 
\mathrm{sup}\vert \{B\hm\subseteq A\vert \forall\,a\hm\in B: a\hm\leq x\}\vert / 
\vert A\vert$, $x\hm\in R$, так что $\hat{\phi}(-\infty) A\hm=0$, $\hat{\phi} 
(+\infty)A\hm=1$. Для краткости $\hat{\phi}(x)A$ будем также записывать 
как~$\hat{\phi}A$. 
     
     \smallskip
     
     \noindent
     \textbf{Определение~4.} $\hat{\mu}$~--- \textit{оператор вычисления 
математического ожидания} значения $x\hm\in A$ по э.\,ф.\,р.~$\hat{\phi}A$~как 
$$
\hat{\mu} \hat{\phi}A= \fr{1}{m} \sum\limits^m_{j=1} x_j \left(\hat{\phi}(x_j)A 
\hat{\phi}(x_{j\,1})A\right),
$$
 где $m\hm= \vert \hat{z}A\vert$; $x_j\hm= 
\hat{\iota}^+(j)\hat{z}A$, а~произвольное $x_0\hm< \mathrm{inf}\,(A)$, 
$x_0\hm\in R$;  $\hat{z}$~--- \textit{оператор формирования множества 
значений} множества~$A$, $\hat{z}A\hm= B\hm\subseteq A\vert \forall\,a\hm\in 
A: a\hm\in B, \forall\,a,b \hm\in B: a\not= b$; $\hat{\iota}^+$~--- \textit{оператор 
упорядочения множества по возрастанию};  
$\hat{\iota}^+(j)A$~--- $j$-й элемент множества~$\hat{\iota}^+A$. 

\section{Анализ решетки~$L(T(\mathbf{X}))$ с~использованием 
функций~$\hat{\phi}A$ на основе опорных цепей}
     
     Рассмотрим подходы к~анализу решетки   посредством э.\,ф.\,р. 
Определения~2--4 существенно расширяют формализм решеточных оценок, 
позволяя (1)~проектировать $L(T(\mathbf{X}))$ в~соответствующую решетку 
э.\,ф.\,р.\ посредством некоторой заранее выбранной (опорной) цепи; 
(2)~измерять расстояния между этими э.\,ф.\,р.; (3)~вводить новые 
разновидности оценок множеств (см.\ определение~1). В~качестве опорной цепи 
может быть выбрана, в~част\-ности, цепь, соответствующая числовой таргетной 
переменной.

\smallskip

\noindent
\textbf{Теорема~1.} \textit{Выберем произвольную максимальную цепь~$A_t$ 
в~качестве <<опорной>> для дальнейших построений. При условии 
регулярности множеств в~$\mathbf{X}/Q$ каждому 
элементу~$L(T(\mathbf{X}))$ сопоставлена э.\,ф.\,р.\ из множества 
однородных э.\,ф.\,р.} 

\smallskip

\noindent
Д\,о\,к\,а\,з\,а\,т\,е\,л\,ь\,с\,т\,в\,о\,.\ \  При выполнении условия ре\-гу\-ляр\-ности 
для $\mathbf{X}/Q$ решетка $L(T(\mathbf{X}))$~--- булева (тео\-ре\-ма~3 
в\cite{9-tr}). Цепи в~$L(T(\mathbf{X}))$ соответствуют тем или иным чис\-ло\-вым 
признаковым описаниям (тео\-ре\-ма~1 в~\cite{4-tr}), так что произвольная 
(максимальная) цепь~$A_t$ в~$L(T(\mathbf{X}))$ пред\-ста\-ви\-ма в~виде

\noindent 
\begin{multline*}
A_t= 
\langle u(\lambda_{t_1}), \ldots, u(\lambda_{t_i}), \ldots u(\lambda_{t_m})\rangle,\\ 
\lambda_{t_i}\in I_t,\enskip u(\lambda_{t_i})= \bigcup\nolimits^i_{\beta=1} 
\Gamma_t^{-1} (\lambda_{t_\beta}),
\end{multline*}
 где $I_t= (\lambda_{t_1},\ldots , 
\lambda_m)$~--- строго монотонная по\-сле\-до\-ва\-тель\-ность чисел. Значение 
функции~$\Gamma_t$, вы\-чис\-ли\-мое для любого объекта в~$\mathbf{X}$, рав\-но 
$\Gamma_t(q)$ для каж\-до\-го решеточного \textit{атома} $\{q\} \hm\in 
L(T(\mathbf{X}))$, высота атома рав\-на~1 ($h[\{q\}] \hm\equiv \vert \{q\}\vert 
\hm\equiv 1$). Поскольку решетка булева, то каж\-дый ее элемент пред\-ста\-вим 
в~виде комбинации атомов, так что любому элементу решетки $u\hm\in 
L(T(\mathbf{X}))$ со\-по\-став\-ле\-но множество значений $t$-го при\-зна\-ка 
$\bm{\Gamma}_t(u)\hm= \{ \Gamma_t(q), q\hm\in u\}$ по всем атомам из~$u$. 
Применяя оператор $\hat{\phi}(x)$ к~множеству $\bm{\Gamma}_t(u)$, получаем 
э.\,ф.\,р.\ $\hat{\phi}\bm{\Gamma}_t(u)$. При выполнении условия ре\-гу\-ляр\-ности 
для $\mathbf{X}/Q$ решетка $L(T(\mathbf{X}))$ однозначно со\-по\-став\-ле\-на 
решетке, образованной чис\-ло\-вы\-ми множествами $\bm{\Gamma}_t(u)$, 
и~вы\-чис\-ли\-ма $\hat{\phi}\bm{\Gamma}_t(u)$. Все эти э.\,ф.\,р.\ однородны по 
по\-стро\-ению~--- ведь они сформированы над одним и~тем же множеством~$I_t$ и~принимают значения в~диапазоне $[0\ldots1]$, а~будучи э.\,ф.\,р., 
характеризуются одинаковой об\-ластью значений.  Тео\-ре\-ма доказана.

\smallskip
     
     Итак, при задании опорной цепи $A_t$ любому элементу решетки 
$u\hm\in L(T(\mathbf{X}))$ со\-по\-став\-ле\-но множество чисел 
$\bm{\Gamma}_t(u)$, чис\-ло\-вая функция $\hat{\phi}\bm{\Gamma}_t(u)$ и~ряд 
функционалов вида $\hat{\mu}\hat{\phi}(x) \bm{\Gamma}_t(u)$, которые могут 
быть использованы для определения оценок в~решетке $L(T(\mathbf{X}))$ 
и/или вы\-чис\-ле\-ния расстояний.

\section{Оценки в~решетке $L(T(\mathbf{X}))$ на~основе 
множеств~$\Gamma_t$ с~использованием понятия~меры}

    С использованием теоремы~1 изотонные оценки на основе множеств   
порождаются функционалами вида $g: 2^{I_t}\hm\to R^+$ так, что при 
произвольных $u,v \hm\in L(T(\mathbf{X}))$ для $g(\bm{\Gamma}_t(u))$ 
и~$g(\bm{\Gamma}_t(v))$ выполнено уО, а~при $u\hm\supseteq v$ выполнено уИ, 
т.\,е.\ $g(\bm{\Gamma}_t(u)) \hm\geq g(\bm{\Gamma}_t(v))$. Одни из наиболее 
очевидных функционалов $g$~--- различные \textit{меры} множеств, 
ис\-поль\-зу\-емые как решеточные оценки~\cite{9-tr}.
     
     Понятие оценки в~теории решеток и~понятие меры в~функциональном 
анализе во многом схожи. Как и~$v[\,]$, мера положительно определена, мера 
пустого множества рав\-на нулю, а~мера пересечения непересекающихся 
множеств рав\-на сумме мер этих множеств. Однако уО выдвигает 
дополнительное требование: если множества пересекаются, то оценка их 
объединения равна сумме оценок множеств минус оценка их пересечения, так 
что любая $v[\,]$~--- мера. Меры могут вводиться различными 
способами~\cite{6-tr}.

\smallskip

\noindent
\textbf{Определение~5.} Пусть точкам действительной оси $I_t\hm= 
\{\lambda_{t_1}, \lambda_{t_2}, \ldots , \lambda_{t_b},\ldots, \lambda_{k_{\vert 
I_t\vert -1}},\Delta\}$, $\lambda_{t_b}\hm\in R$, сопоставлены веса $p_1, p_2, 
\ldots , p_{\vert I_t\vert -1}$. Тогда определена \textit{мера с~дискретным весом} 
$$
\mu_{\mathrm{дв}} (\bm{\Gamma}_t(u))\hm= \sum\limits_{\lambda_{t_b} \in 
\bm{\Gamma}_t(u)} p_b. 
$$
     
     В качестве весов в~определении~5 могут быть выбраны: (1)~длины 
интервалов значений из $I_t$ $(\lambda_{t_b} \hm- \lambda_{t_{b-1}}, 
\lambda_{t_{b+1}} -\lambda_{t_b}$ и~др.); (2)~раз\-но\-сти значений э.\,ф.\,р.\ 
     $(\mathrm{cdf}\,(\lambda_{t_{b+1}}, A_t(\mathbf{X})) \hm- \mathrm{cdf} 
\,( \lambda_{t_b}, A_t(\mathbf{X}))$ и~др.); (3)~веса, на\-стра\-и\-ва\-емые 
в~соответствии с~принципом со\-гла\-со\-ва\-ния мет\-рик, и~т.\,д. Очевидна 

\smallskip

\noindent
\textbf{Теорема~2.} \textit{Мера с~дискретным весом~--- оценка}. 
\smallskip

Утверждение следует из рас\-смот\-ре\-ния пе\-ре\-кры\-ва\-ющих\-ся 
и~не\-пе\-ре\-кры\-ва\-ющих\-ся множеств $\bm{\Gamma}_t(u)$ и~$\bm{\Gamma}_t(v)$ 
и~вы\-пол\-ни\-мости уО из определения~1.

\smallskip

\noindent
\textbf{Следствие~1.} \textit{Интеграл от сум\-ми\-ру\-емой функции~$f$ 
с~использованием~$\mu_{\mathrm{дв}}$ вы\-чис\-ля\-ет\-ся как}  
$$
\int\limits_{-
\infty}^{+\infty} f(\lambda)\,d\mu_{\mathrm{дв}} \hm= \sum\limits_{b=1,\vert 
I_t\vert-1} p_b f(\lambda_{t_b}).
$$

\smallskip

\noindent
     \textbf{Следствие~2.} \textit{Скалярное произведение сум\-ми\-ру\-емых 
функций $f(\lambda)$ и~$g(\lambda)$ на основе~$\mu_{\mathrm{дв}}$ рав\-но} 
     $$
     (f,g)= \sum\limits_{b=1,\vert I_t\vert -1} p_b f(\lambda_{t_b}) 
g(\lambda_{t_b}).
$$
    
\smallskip

\noindent
\textbf{Следствие~3.} \textit{Колмогоровский функционал <<заряда>> 
$\Phi(A)$ на множестве чисел~$A$ с~использованием сум\-ми\-ру\-емой 
функции~$f(\lambda)$, $\Phi(A)\hm= \int\nolimits_A f(\lambda)\,d\mu$, служит 
мерой}. При использовании меры с~дискретными весами 
$$
\Phi(A)= 
\sum\limits_{\lambda_{t_b} \in A} p_b f(\lambda_{t_b})
$$ 
(следствие~1).
    
\smallskip

\noindent
\textbf{Следствие~4.} \textit{Колмогоровский заряд  
$\Phi(\bm{\Gamma}_t(u))$~--- изотонная оценка на решетке $L(T(\mathbf{X}))$ 
при положительной опре\-де\-лен\-ности}~$f(\lambda)$. Перекрывание площади 
под произвольной одномерной~$f$ в~случае множеств $\bm{\Gamma}_t(u)$ 
и~$\bm{\Gamma}_t(v)$ равно $\Phi (\bm{\Gamma}_t(u)\cap 
\bm{\Gamma}_t(v))$, что равно $\Phi (\bm{\Gamma}_t (u\cap v)$ и~равно сумме 
площадей $\Phi(\bm{\Gamma}_t(u))$ и~$\Phi(\bm{\Gamma}_t(v))$ минус площадь 
объединения множеств (что соответствует вы\-пол\-ни\-мости уО в~определении~1). 
Оценка~$\Phi$ \textit{изотонна} при $f(\lambda)\hm\geq 0$. 
     
     Из теоремы~2 со следствиями очевидно, что введение <<заряда>>~$\Phi$ 
позволяет более гибко оценивать вклад каж\-до\-го значения таргетной 
переменной~$\lambda_{t_b}$ в~значение меры: ведь в~$\Phi(A)$ используются 
не только дискретные веса~$p_b$ значений~$\lambda_{t_b}$, но и~весовая 
функция $f(\lambda)$, общая для всех значений.

\section{Перспективы анализа решеток без использования понятия 
оценки}

    Решеточные термы $v: L\hm\to R^+$ дают скалярную оцен\-ку каж\-до\-го 
элемента со\-от\-вет\-ст\-ву\-ющей решетки~$L$, позволяющее сравнивать 
элементы~$L$ меж\-ду собой (при вы\-пол\-ни\-мости уО и~уИ). Очевидно, что 
со\-по\-став\-ле\-ние произвольному элементу $u$ решетки $L(T(\mathbf{X}))$ 
функции~$\hat{\phi}\bm{\Gamma}_t(u)$ представляется гораздо более сложным 
<<оценочным>> описанием множества~$u$, чем скалярная $v[u]$. Эта цепь 
рас\-суж\-де\-ний указывает на два на\-прав\-ле\-ния дальнейшего развития формализма: 
\begin{enumerate}[(1)]
\item введение функционалов, поз\-во\-ля\-ющих редуцировать более слож\-ное 
описание в~виде $\hat{\phi}\bm{\Gamma}_t(u)$ к~скалярным оценкам; 
\item разработку нового математического аппарата для анализа решеток, 
в~котором вмес\-то оценок $v: L\hm\to R^+$ фигурируют операции над 
функциями~$\hat{\phi}\bm{\Gamma}_t(u)$.
\end{enumerate}
     
     Первое направление отчасти покрывается результатами тео\-ре\-мы~2 со 
следствиями. Второе направление интересно тем, что позволяет вводить 
мет\-ри\-че\-ские функции рас\-сто\-яния без использования конструкции $v[x\vee 
y]\hm- v[x\wedge y]$, на основании упо\-ми\-на\-емых выше подходов 
функционального анализа (как это было сделано в~работе~\cite{4-tr} для 
колмогоровского <<максимального уклонения>>). 

\section{Заключение }

    В прикладной математике повсеместно используются функционалы, 
оценивающие расстояния между множествами, векторами или функциями. При 
установлении метрических свойств этих функционалов инструментарий 
формализма топологической тео\-рии распознавания может быть существенно 
обогащен нетривиальными метриками на основе эмпирических  
и~по\-лу\-эм\-пи\-ри\-че\-ских функционалов расстояния. Во второй части 
статьи будут пред\-став\-ле\-ны результаты приложения раз\-ра\-ба\-ты\-ва\-емо\-го 
формализма к~комплексу при\-клад\-ных задач из об\-ласти фармакоинформатики.

{\small\frenchspacing
 { %\baselineskip=10.6pt
 %\addcontentsline{toc}{section}{References}
 \begin{thebibliography}{9}
\bibitem{1-tr}
\Au{Журавлёв Ю.\,И., Рудаков~К.\,В., Торшин~И.\,Ю.} Алгебраические критерии локальной 
разрешимости и~регулярности как инструмент исследования морфологии аминокислотных последовательностей~// Труды 
\mbox{МФТИ}, 2011. Т.~3. №\,4. C.~45--54. EDN: OJYMVJ.
\bibitem{2-tr}
\Au{Рудаков К.\,В., Торшин~И.\,Ю.} Анализ информативности мотивов на основе критерия 
разрешимости в~задаче распознавания вторичной структуры белка~// Информатика и~её 
применения, 2012. Т.~6. Вып.~1. С.~79--90. EDN: OZHDTV.
\bibitem{3-tr}
\Au{Торшин И.\,Ю.} О~задачах оптимизации, воз\-ни\-ка\-ющих при применении 
топологического анализа данных к~поиску алгоритмов прогнозирования с~фиксированными 
корректорами~// Информатика и~её применения, 2023. Т.~17. Вып.~2. С.~2--10. doi: 
10.14357/19922264230201. EDN: IGSPEW.
\bibitem{4-tr}
\Au{Торшин И.\,Ю.} О~формировании множеств прецедентов на основе таблиц разнородных 
признаковых описаний методами топологической теории анализа данных~// Информатика 
и~её применения, 2023. Т.~17. Вып.~3. С.~2--7. doi: 10.14357/19922264230301. EDN: 
AQEUYO.
\bibitem{5-tr}
\Au{Torshin I.\,Y., Rudakov~K.\,V.} On the procedures of generation of numerical features over 
partitions of sets of objects in the problem of predicting numerical target variables~// Pattern 
Recognition Image Analysis, 2019. Vol.~29. No.\,3. P.~654--667. doi: 
10.1134/S1054661819040175.
\bibitem{6-tr}
\Au{Колмогоров A.\,H., Фомин~С.\,В.} Элементы теории функций и~функционального 
анализа.~--- М.: Наука, 1989. 624~с.
\bibitem{7-tr}
\Au{Torshin I.\,Y., Rudakov~K.\,V.} Combinatorial analysis of the solvability properties of the 
problems of recognition and completeness of algorithmic models. Part~2: Metric approach within 
the framework of the theory of classification of feature values~// Pattern Recognition Image Analysis, 
2017. Vol.~27. No.\,2. P.~184--199.  doi: 10.1134/ S1054661817020110.
\bibitem{8-tr}
\Au{Деза Е.\,И., Деза~М.\,М.} Энциклопедический словарь расстояний~/
Пер.\ с~англ.~--- М.: Наука, 2008.  444~с. (\Au{Deza~E., Deza~M.\,M.} {Dictionary of distances}.~---  
North-Holland: Elsevier, 2006. 412~p. doi: 10.1016/B978-0-444-52087-6.X5000-8.)

\bibitem{9-tr}
\Au{Torshin I.\,Y., Rudakov K.\,V.} On the theoretical basis of metric analysis of poorly 
formalized problems of recognition and classification~// Pattern Recognition Image Analysis, 2015. 
Vol.~25. No.\,4. P.~577--587. doi: 10.1134/ S1054661815040252.
\end{thebibliography}

 }
 }

\end{multicols}

\vspace*{-10pt}

\hfill{\small\textit{Поступила в~редакцию 15.01.23}}

%\vspace*{8pt}

%\pagebreak

\newpage

\vspace*{-28pt}

%\hrule

%\vspace*{2pt}

%\hrule



\def\tit{ON THE GENERATION OF~SYNTHETIC FEATURES BASED~ON~SUPPORT 
CHAINS AND~ARBITRARY METRICS\\ WITHIN~A~TOPOLOGICAL APPROACH TO~DATA ANALYSIS.\\ PART~1. 
INCLUSION OF~EMPIRICAL DISTANCE FUNCTIONS 
INTO~THE~FORMALISM}


\def\titkol{On the generation of~synthetic features based on~support 
chains and~arbitrary metrics. Part~1}
% within the~framework  of~a~topological approach to~data analysis. Part~1.  Inclusion of~empirical distance functions  into~the~formalism}


\def\aut{I.\,Yu.~Torshin}

\def\autkol{I.\,Yu.~Torshin}

\titel{\tit}{\aut}{\autkol}{\titkol}

\vspace*{-10pt}


\noindent
Federal Research Center ``Computer Science and Control'' of the Russian Academy of 
Sciences, 44-2~Vavilov Str., Moscow 119333, Russian Federation

\def\leftfootline{\small{\textbf{\thepage}
\hfill INFORMATIKA I EE PRIMENENIYA~--- INFORMATICS AND
APPLICATIONS\ \ \ 2024\ \ \ volume~18\ \ \ issue\ 1}
}%
 \def\rightfootline{\small{INFORMATIKA I EE PRIMENENIYA~---
INFORMATICS AND APPLICATIONS\ \ \ 2024\ \ \ volume~18\ \ \ issue\ 1
\hfill \textbf{\thepage}}}

\vspace*{3pt}



\Abste{The analysis of the formalism of topological recognition theory based on the 
fundamental concepts of functional analysis made it possible to propose previously 
unexplored approaches to determining lattice estimates. In particular, the use of 
support chains for the analysis of Boolean lattices formed over  
Zhuravlev-regular sets of precedents has pointed to a~new direction of research which 
consists in replacing estimates of lattice elements with certain types of functions 
and/or vectors. This extension of the formalism also allows for a~systematic study of 
semiempirical distance functionals known in the literature to solve applied problems. 
Promising directions for further development of the formalism are substantiated 
including the functionals reducing descriptions of sets of a~Boolean lattice to scalars 
and the development of a~mathematical apparatus for the analysis of lattices where 
operations on the corresponding functions are involved. The latter direction is 
interesting as it allows defining lattice metrics without using lattice estimates.}

\KWE{topological data analysis; lattice theory; algebraic approach by 
Yu.\,I.~Zhuravlev; functional analysis}



\DOI{10.14357/19922264240110}{RIVOXR}

\vspace*{-12pt}

\Ack

\vspace*{-3pt}

\noindent
The research was funded by the Russian Science Foundation, project  
No.\,23-21-00154. The research was carried out using the infrastructure of the Shared 
Research Facilities ``High Performance Computing and Big Data'' (CKP 
``Informatics'') of FRC CSC RAS (Moscow).


  \begin{multicols}{2}

\renewcommand{\bibname}{\protect\rmfamily References}
%\renewcommand{\bibname}{\large\protect\rm References}

{\small\frenchspacing
 {%\baselineskip=10.8pt
 \addcontentsline{toc}{section}{References}
 \begin{thebibliography}{9} 
\bibitem{1-tr-1}
\Aue{Zhuravlеv, Yu.\,I., K.\,V.~Rudakov, and I.\,Yu.~Torshin.} 2011. 
Algebraicheskie kriterii lokal'noy razreshimosti i~re\-gu\-lyar\-nosti kak instrument 
issledovaniya morfologii ami\-no\-kis\-lot\-nykh posledovatel'nostey [Algebraic criteria of local solvability and regularity as a~tool
for studying the morphology of amino acid sequences]. \textit{Trudy MFTI} [Proceedings 
of Moscow Institute of Physics and Technology] 3(4):45--54. EDN: OJYMVJ.
\bibitem{2-tr-1}
\Aue{Rudakov, K.\,V., and I.\,Yu.~Torshin.} 2012. Analiz informativnosti motivov na 
osnove kriteriya razreshimosti v~zadache raspoznavaniya vtorichnoy struktury belka 
[Analysis of the informativeness of motives based on the criterion of solvability in 
the problem of recognizing the secondary structure of a~protein]. \textit{Informatika 
i~ee Primeneniya~--- Inform Appl.} 6(1):79--90. EDN: OZHDTV.
\bibitem{3-tr-1}
\Aue{Torshin, I.\,Yu.} 2023. O~zadachakh optimizatsii, voznikayushchikh pri 
primenenii topologicheskogo analiza dannykh k~poisku algoritmov prognozirovaniya 
s~fiksirovannymi korrektorami [On optimization problems arising from the 
application of topological data analysis to the search for forecasting algorithms with 
fixed correctors]. \textit{Informatika i~ee Primeneniya~--- Inform Appl.} 
 17(2):2--10. doi: 10.14357/19922264230201. EDN: IGSPEW.
\bibitem{4-tr-1}
\Aue{Torshin, I.\,Yu.} 2023. O~formirovanii mnozhestv pretsedentov na osnove 
tablits raznorodnykh priznakovykh opisaniy metodami topologicheskoy teorii analiza 
dannykh [On the formation of sets of precedents based on tables of heterogeneous 
feature descriptions by methods of topological theory of data analysis]. 
\textit{Informatika i~ee Primeneniya~--- Inform Appl.} 17(3):2--7. doi: 
10.14357/19922264230301. EDN: AQEUYO.
\bibitem{5-tr-1}
\Aue{Torshin, I.\,Yu., and K.\,V.~Rudakov.} 2019. On the procedures of generation of 
numerical features over partitions of sets of objects in the problem of predicting 
numerical target variables. \textit{Pattern Recognition Image Analysis} 29(4):654--667. 
doi: 10.1134/S1054661819040175.
\bibitem{6-tr-1}
\Aue{Kolmogorov, A.\,N., and S.\,V.~Fomin.} 1989. \textit{Elementy teorii funktsiy 
i~funktsional'nogo analiza} [Elements of theory of functions and functional 
analysis]. Moscow: Nauka. 624~p.
\bibitem{7-tr-1}
\Aue{Torshin, I.\,Yu., and K.\,V.~Rudakov.} 2017. Combinatorial analysis of the 
solvability properties of the problems of recognition and completeness of algorithmic 
models. Part~2: Metric approach within the framework of the theory of classification 
of feature values. \textit{Pattern Recognition Image Analysis} 27(2):184--199. doi: 
10.1134/S1054661817020110.
\bibitem{8-tr-1}
\Aue{Deza, E., and M.\,M.~Deza.} 2006. \textit{Dictionary of distances}.  
North-Holland: Elsevier. 412~p. doi: 10.1016/B978-0-444-52087-6.X5000-8.
\bibitem{9-tr-1}
\Aue{Torshin, I.\,Y., and K.\,V.~Rudakov.} 2015. On the theoretical basis of metric 
analysis of poorly formalized problems of recognition and classification. \textit{Pattern 
Recognition Image Analysis} 25(4):577--587. doi: 10.1134/ S1054661815040252.
\end{thebibliography}

 }
 }

\end{multicols}

\vspace*{-6pt}

\hfill{\small\textit{Received January 15, 2023}} 

%\vspace*{-18pt}
     
     \Contrl
     
 %    \vspace*{-3pt}

\noindent
\textbf{Torshin Ivan Yu.} (b.\ 1972)~--- Candidate of Science (PhD) in physics and 
mathematics, Candidate of Science (PhD) in chemistry, leading scientist, Federal 
Research Center ``Computer Science and Control'' of the Russian Academy of 
Sciences, 44-2~Vavilov Str, Moscow 119333, Russian Federation; 
\mbox{tiy135@yahoo.com}

\label{end\stat}

\renewcommand{\bibname}{\protect\rm Литература} 