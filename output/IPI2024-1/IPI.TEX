\documentclass[10pt]{book}
\usepackage[utf8]{inputenc}

\usepackage{latexsym,amssymb,amsfonts,amsmath,amsxtra,dsfont,
indentfirst,shapepar,%fleqn,%
picinpar,shadow,floatflt,enumerate,multicol,colortbl,moreverb,cite,ipi}

\usepackage{rotating}
\usepackage{mathrsfs}
\usepackage[noend]{algorithmic}
\usepackage{ulem}
\usepackage{graphicx}
%\usepackage{algorithm2e}
\usepackage[linesnumbered,boxed,ruled]{algorithm2e}
%\usepackage{xypic}
\usepackage{oldgerm}
\usepackage{epic}
\usepackage{eepic}

\SetAlgorithmName{Algorithm}{алгоритм}{Список алгоритмов}

%из Дюковой

\newcommand{\algKeyword}[1]{{\bf #1}}
\newcommand{\Proc}[1]{\text{\tt #1}}
\def\CALL{\algKeyword{call}~}

\newenvironment{AlgProcedure}[1]
{
\small
\medskip
%    \hrule
\medskip
\algKeyword{PROCEDURE} #1
\begin{algorithmic}[1]}
{\end{algorithmic}
%    \hrule
\bigskip
}

\def\CALL{\algKeyword{call}~}

%конец для Дюковой

%\RequirePackage[ruled]{algorithm}


\input{epsf}

%\nofiles

%\includeonly{avtor}    %pdf
%\includeonly{podgot-rus-site,podgot-eng-site}  
%\includeonly{podgot-rus,podgot-eng}  
%\includeonly{ipi-ind} 
%\includeonly{index-17i}
%\includeonly{toc-rus, toc-en}
%\includeonly{toc-rus}
%\includeonly{toc-en} 
%\includeonly{popravka}



%\includeonly{sinits}             %+pdf+авт+
%\includeonly{torshin}            %+pdf+авт+
%\includeonly{kudr-shest}         %+pdf+авт+
%\includeonly{grusho}             %+pdf+авт+
%\includeonly{leshin}             %+pdf+авт+
%\includeonly{borisov}            %+pdf+авт+
%\includeonly{bosov}              %+pdf+авт+
%\includeonly{flerov}             %+pdf+авт+
%\includeonly{listopad}    %+pdf+авт повт отпр
%\includeonly{krivenko}    %+pdf
%\includeonly{leri-pavlov}        %+pdf+авт+
%\includeonly{malashenko}  %+pdf+авт???
%\includeonly{kovalev}            %+pdf+авт+




%%%%%%%%%%%%%%%%%%%\includeonly{nekrolog-new}



%\includeonly{rekl}




\usepackage{acad}
%\usepackage{courier}
\usepackage{decor}
\usepackage{newton}
\usepackage{pragmatica}
\usepackage{zapfchan}
\usepackage{petrotex}
\usepackage{bm}                     % полужирные греческие буквы
\usepackage{upgreek}                % прямые греческие буквы \upalpha
\usepackage{eufrak}
\usepackage{verbatim}

\renewcommand{\bottomfraction}{0.99}
\renewcommand{\topfraction}{0.99}
\renewcommand{\textfraction}{0.01}

\setcounter{secnumdepth}{1} %здесь - 3 + chapter = 4

\arraycolsep=1.5pt

%\usepackage[pdftex]{graphicx}

%\usepackage{oz}

%NEW COMMANDS



\renewcommand*{\hm}[1]{#1\nobreak\discretionary{}%
            {\hbox{$\mathsurround=0pt #1$}}{}} %% Дублирует знаки операций
                               %при переносе в формуле (перед знаком, который
                               %надо продублировать ставится команда \hm)
                               
                               \newcommand{\PRB}{\begin{picture}(22.5,11)
      \spline(1,8)(4,10)(7,10.5)(10,11)(13,11)(16,10.5)(19,10)(22,8)
               \put(0,0){$P_{i-1}P_{t_{t-1}}$} \end{picture}}

\newcommand{\prb}{\begin{picture}(15.5,9)
      \spline(1,6)(3,8)(5,8.5)(7,9)(9,9)(11,8.5)(13,8)(15,6)
               \put(0,0){$PP_t$} \end{picture}}
               
                 \newcommand{\PRDN}{\begin{picture}(40,11)
      \spline(4,11.5)(7,10.5)(12,10)(16,9)(20,9)(24,10)(29,10.5)(32,11.5)
               \put(0,0){$P_{i-1}P_{t_{t-1}}$} \end{picture}}

\newcommand{\prdn}{\begin{picture}(18,11)
      \spline(3,10.5)(4,10)(6,9)(8,8.5)(10,8.5)(12,9)(14,10)(15,10.5)
               \put(0,0){$PP_t$} \end{picture}}




%\newcommand{\endproof}{\hfill$\Box$}
%\renewcommand{\r}{\mathbb{R}}
%\newcommand{\I}{{\rm I\hspace{-0.7mm}I}}
%\newcommand{\Ikl}{{\tt{1}}\hspace*{-1.44mm}\mathtt{1}}
%\newcommand{\Ik}{\mbox{{\small \tt {1}}\hspace{-1.3mm}{\tt 1}}}
\newcommand{\Ik}{\mbox{{{\tt 1}}\hspace{-1.3mm}{\sf 1}}}
\newcommand{\argmin}{\mathop{\mathrm{arg}\,\mathrm{min}}}
\newcommand{\argmax}{\mathop{\mathrm{arg}\,\mathrm{max}}}
%\newcommand{\capr}{\mathop{\cap\,}}
%\newcommand{\cupr}{\mathop{\cup\,}}
%\def\argmin{\mathop{arg\,min}}

\def\vrp{\varphi}
\def\prt{\partial}
\def\mm{{\sf M}}
\def\modnop#1{\mathop{#1}\limits_{n}}
\def\eam{\mathbin{{\mathop{=}\limits^{\mathrm{def}}}}}
\def\dey#1#2{#1 (#2)}
\def\deyc#1#2{#1 \cdot  #2}
\def\ra#1{\;\mathop{\to}\limits^{#1}\;}
\def\raz#1{\;\mathop{\longrightarrow}\limits^{\!\!\!#1}\;}
\def\ral#1{\;\mathop{\longrightarrow}\limits^{#1}\;}





\newcommand{\il}[2]{\int\limits_{#1}^{#2}}%интеграл с пределами #1 и #2

\def\sm2{\mathop {\sum\limits^{n^\Theta}\sum\limits^{n^\Theta}}}
\def\sss{\sum\limits}
\def\tr{,\,\ldots\,,\,}
\def\rk{\right]}
\def\lk{\left[}
\def\rf{\right\}}
\def\lf{\left\{}
\def\lv{\,\left\vert}
\def\rv{\right\vert\,}
\def\iii{\int\limits}
\def\iin{\int\limits_{-\infty}^\infty}
\def\rrv{\right\vert}


\def\ee{{\cal E}}
\def\ww{{\cal W}}
\def\yy{{\cal Y}}
\def\vv{{\cal V}}

\newcommand{\R}{\mathbb R}
\newcommand{\E}{\mathbb E}
\newcommand{\N}{\mathbb N}
\newcommand{\T}{\mathbb{T}}
\newcommand{\Z}{\mathbb{Z}}

\renewcommand{\P}{\mathbb{P}}

\newcommand{\Nor}{\mathcal{N}}

\newcommand{\h}{{\bf H}}
\newcommand{\p}{{\sf P}}  % вероятность
\newcommand{\e}{{\sf E}}  % мат. ожидание
\newcommand{\D}{{\sf D}}  % дисперсия



\newcommand{\vw}{{\mathbf w}}
\newcommand{\vp}{{\mathbf p}}
\newcommand{\vz}{{\mathbf z}}
\newcommand{\vx}{{\mathbf x}}
\newcommand{\vf}{{\mathbf f}}
\newcommand{\F}{{\mathcal F}}
\def\ap{{\mathrm{ЭР}}}
\newcommand{\ud}{\Delta_n} %uniform ditance
\newcommand{\nud}{\Delta_n(x)}
%\renewcommand{\Re}{\mathrm{Re}\,}

\newcommand{\abs}[1]{\left\vert#1\right\vert}

\newcommand{\norm}[1]{\left\Vert#1\right\Vert}
\def\da{(\Delta_t,A)}

\newcommand{\corr}{\mathrm{corr}}

\newcommand{\cov}{\mathrm{cov}}
\newcommand{\Expect}{\mathbb{E}}

\def\w{\omega}
\def\W{\Omega}


\def\inh{\int\limits_{nh}^{(n+1)h}}

\def\sumin{\sum_{i=1}^N}


\def\bxt{(Y,t)}
\def\xt{(y,t)}

\def\ovth{{\fr{\tau-nh}{h}}}
\def\ov{\overline}
\def\tm{\tilde m}
\def\tl{\tilde\lambda}
\def\tB{\widetilde B}
\def\tb{\tilde b}
\def\ld{\ldots}
\def\cd{\cdots}


\DeclareMathOperator{\sign}{sign}



\newcommand{\g}{\mbox{\textit{g}}}

\renewcommand{\la}{\lambda}
\newcommand{\si}{\sigma}
\newcommand{\eps}{\varepsilon}
\newcommand{\alp}{\alpha}

\newcommand{\pto}{\stackrel{P}{\longrightarrow}} % сходимость по веpоятности

\newcommand{\eqd}{\stackrel{\mathrm{d}}{=}} % равенство по pаспpеделению
\newcommand{\eqdelta}{\stackrel{\triangle}{=}} % равенство по pаспpеделению

\def\be#1{\begin{equation}\label{#1}}
\def\ee{\end{equation}}
\def\re#1{(\ref{#1})}

\def\bn{\begin{enumerate}}
\def\en{\end{enumerate}}
\def\bi{\begin{itemize}}
\def\ei{\end{itemize}}
%\def\i{\item}

%\newcommand{\kp}{\kappa}
%\def\Q{{\cal Q}} \def\H{{\cal H}}
%\newcommand{\bet}{\beta_{2+\delta}}




%\renewcommand{\baselinestretch}{1.2}

%\pagestyle{myheadings}

\setlength{\textwidth}{167mm}      % 122mm
\setlength{\textheight}{658pt}
%\setlength{\textheight}{635.6pt}
\setlength{\columnsep}{4.5mm}

\setcounter{secnumdepth}{4}

%\addtolength{\headheight}{2pt}
%\addtolength{\headsep}{-2mm}

\addtolength{\topmargin}{-7mm}  % for printing


%\hoffset=-30mm  % From Yap
\hoffset=-23mm  % From Acrobat

%\voffset=0mm % From Yap
\voffset=-5mm   % From Acrobat

%\addtolength{\evensidemargin}{-2.5mm} % for printing
%\addtolength{\oddsidemargin}{2.5mm}  % for printing

\addtolength{\evensidemargin}{-12mm} % for printing
\addtolength{\oddsidemargin}{8mm}  % for printing

%\renewcommand{\thefootnote}{\fnsymbol{footnote}}
%\renewcommand{\thefootnote}{\arabic{footnote}}
\renewcommand{\figurename}{\protect\bf Рис.}
\renewcommand{\tablename}{\protect\bf Таблица}

\newcommand{\Caption}[1]{\caption{\protect\small %\baselineskip=2.5ex
#1}}

\renewcommand{\thefigure}{\arabic{figure}}
\renewcommand{\thetable}{\arabic{table}}
\renewcommand{\theequation}{\arabic{equation}}
\renewcommand{\thesection}{\arabic{section}}

\renewcommand{\contentsname}{СОДЕРЖАНИЕ}
\newcommand{\fr}[2]{\displaystyle\frac{\displaystyle #1\mathstrut}{\displaystyle #2\mathstrut}}

%\renewcommand{\thefootnote}{\fnsymbol{footnote}}
%\newcommand{\g}{\mbox{\textit{g}}}

%\newcommand{\Caption}[1]{\caption{\protect\small\baselineskip=2ex #1}}
\newcounter{razdel}
\setcounter{razdel}{0}

\def\god{2024}
\def\tom{18}
\def\vyp{1}


\newcommand{\titel}[4]{%
\

\vspace*{5pt}

\ifodd\therazdel {\raggedright\noindent\Large\textrm\textbf
 \lineskip .75em
  \baselineskip=3.2ex #1 \par}
\vskip 1em {\noindent\large\textrm\textbf #2 \par}
\addcontentsline{toc}{subsection}{{\textrm\textbf #1}\protect\newline #2}
\def\rightheadline{\underline{\noindent\hbox to \textwidth{\hfill\small\textrm{#4}
%\hfill \large\bf\thepage
}}}
\def\leftheadline{\underline{\noindent\parbox{\textwidth}{
%\raggedleft\large\bf\thepage \hfill
\small\textit{#3}\hfill}}}
\def\leftfootline{\small{\textbf{\thepage}
\hfill ИНФОРМАТИКА И ЕЁ ПРИМЕНЕНИЯ\ \ \ том~\tom\ \ \ выпуск~\vyp\ \ \ \god}
}%
 \def\rightfootline{\small{ИНФОРМАТИКА И ЕЁ ПРИМЕНЕНИЯ\ \ \ том~\tom\ \ \ выпуск~\vyp\ \ \ \god
\hfill \textbf{\thepage}}}
\vskip 2em \setcounter{figure}{0}
\setcounter{table}{0}
\setcounter{equation}{0}
\setcounter{section}{0}
\setcounter{subsection}{0}
\setcounter{subsubsection}{0}
\setcounter{footnote}{0}
\setcounter{razdel}{0}
%\end{flushleft}
\else {
 \raggedright\noindent\Large\textrm\textbf
 \lineskip .75em
\baselineskip=3.2ex #1 \par} \vskip 1em
%\begin{flushleft}
{\noindent\large\textrm\textbf #2 \par}
\addcontentsline{toc}{subsection}{{\textrm\textbf #1}\protect\newline #2}
\def\rightheadline{\underline{\noindent\hbox to \textwidth{\hfill\small\textrm{#4}
%\hfill \large\bf\thepage
}}}
\def\leftheadline{\underline{\noindent\parbox{\textwidth}{%\raggedleft\large\bf\thepage \hfill
\small\textit{#3}\hfill}}}
\def\leftfootline{\small{\textbf{\thepage}
\hfill ИНФОРМАТИКА И ЕЁ ПРИМЕНЕНИЯ\ \ \ том~\tom\ \ \ выпуск~\vyp\ \ \ \god}
}%
 \def\rightfootline{\small{ИНФОРМАТИКА И ЕЁ ПРИМЕНЕНИЯ\ \ \ том~18\ \ \ выпуск~\vyp\ \ \ 2024
\hfill \textbf{\thepage}}} \vskip 2em \setcounter{figure}{0}
\setcounter{table}{0} \setcounter{equation}{0} \setcounter{section}{0}
\setcounter{subsection}{0} \setcounter{subsubsection}{0}
\setcounter{footnote}{0}
%\end{flushleft}
\fi}

\newcommand{\titelr}[2]{%
\

\vspace*{5pt}

\ifodd\therazdel {\raggedright\noindent%\Large\textrm\textbf
 \lineskip .75em
  \baselineskip=3.2ex #1 \par}
\vskip 1em {\noindent\normalsize\textrm\textbf #2 \par}
\else {
 \raggedright\noindent\Large\textrm\textbf
 \lineskip .75em
\baselineskip=3.2ex #1 \par} \vskip 1em
%\begin{flushleft}
{\noindent\large\textrm\textbf #2 \par
%\noindent\normalsize\textrm\textbf #2 \par
} \fi}

\newcommand{\titele}[5]{%
\

%\vspace*{5pt}

\ifodd\therazdel {\raggedright\noindent\large
\textrm\textbf
 \lineskip .75em
%  \baselineskip=3.2ex
#1 \par}
\vskip .5em {\noindent\large\textrm\textbf #2 \par}
\vskip .5em
 {\noindent\textrm #3 \par}
\addcontentsline{toc}{subsection}{{\textrm\textbf #1}\protect\newline #2}
\def\rightheadline{\underline{\noindent\hbox to \textwidth{\hfill\small\textrm{#4}
%\hfill \large\bf\thepage
}}}
\def\leftheadline{\underline{\noindent\parbox{\textwidth}{
%\raggedleft\large\bf\thepage \hfill
\small\textrm{#5}\hfill}}}
\def\leftfootline{\small{\textbf{\thepage}
\hfill ИНФОРМАТИКА И ЕЁ ПРИМЕНЕНИЯ\ \ \ том~18\ \ \ выпуск~1\ \ \ 2024}
}%
 \def\rightfootline{\small{ИНФОРМАТИКА И ЕЁ ПРИМЕНЕНИЯ\ \ \ том~18\ \ \ выпуск~1\ \ \ 2024
\hfill \textbf{\thepage}}} \vskip 1em \setcounter{figure}{0}
\setcounter{table}{0} \setcounter{equation}{0} \setcounter{section}{0}
\setcounter{subsection}{0} \setcounter{subsubsection}{0}
\setcounter{footnote}{0} \setcounter{razdel}{0}
%\end{flushleft}
\else {
 \raggedright\noindent\large
 \textrm\textbf
 \lineskip .75em
%\baselineskip=3.2ex
#1 \par} \vskip .5em
%\begin{flushleft}
{\noindent\large\textrm\textbf #2 \par} \vskip .5em
 {\noindent\textrm #3 \par}
\addcontentsline{toc}{subsection}{{\textrm\textbf #1}\protect\newline #2}
\def\rightheadline{\underline{\noindent\hbox to \textwidth{\hfill\small\textrm{#4}
%\hfill \large\bf\thepage
}}}
\def\leftheadline{\underline{\noindent\parbox{\textwidth}{%\raggedleft\large\bf\thepage \hfill
\small\textrm{#5}\hfill}}}
\def\leftfootline{\small{\textbf{\thepage}
\hfill ИНФОРМАТИКА И ЕЁ ПРИМЕНЕНИЯ\ \ \ том~18\ \ \ выпуск~1\ \ \ 2024}
}%
 \def\rightfootline{\small{ИНФОРМАТИКА И ЕЁ ПРИМЕНЕНИЯ\ \ \ том~18\ \ \ выпуск~1\ \ \ 2024
\hfill \textbf{\thepage}}} \vskip 1em \setcounter{figure}{0}
\setcounter{table}{0} \setcounter{equation}{0} \setcounter{section}{0}
\setcounter{subsection}{0} \setcounter{subsubsection}{0}
\setcounter{footnote}{0}
%\end{flushleft}
\fi}

\def\Abst#1{
\begin{center}\small\nwt
\parbox{150mm}{%\baselineskip=2.5ex
\textbf{Аннотация:}\ \
%\hspace*{\parindent}
#1}
\end{center}}
\def\Abste#1{
\begin{center}\small\nwt
\parbox{150mm}{%\baselineskip=2.5ex
\textbf{Abstract:}\ \
%\hspace*{\parindent}
#1}
\end{center}}

%\def\DOI#1{
%\begin{center}\small\nwt
%\parbox{150mm}{%\baselineskip=2.5ex
%\textbf{DOI:}\ \
%%\hspace*{\parindent}
%#1}
%\end{center}}

\def\Abstend#1{
\begin{center}\small\nwt
\parbox{150mm}{%\baselineskip=2.5ex
%\hspace*{\parindent}
#1}
\end{center}}

\newcommand{\DOI}[2]{\begin{center}\small\nwt
\parbox{150mm}{%\baselineskip=2.5ex
\textbf{DOI:}\ \
%\hspace*{\parindent}
#1 \hfill \textbf{EDN:}\ \
#2}
\end{center}}




\def\KW#1{
\begin{center}\small\nwt
\parbox{150mm}{%\baselineskip=2.5ex
\textbf{Ключевые слова:}\ \ #1}
\end{center}}

\def\KWE#1{
\begin{center}\small\nwt
\parbox{150mm}{%\baselineskip=2.5ex
\textbf{Keywords:}\ \ #1}
\end{center}}


\def\KWN#1{
%\begin{center}
%\small
%\parbox{150mm}\end{center}
}

\newcommand{\Avtors}[1]{%\smallskip
%\vspace*{.5pt}
\hangindent=23pt\noindent
%\nwt
{\bfseries#1}\
}


\renewcommand{\thesubsection}{\thesection.\arabic{subsection}\hspace*{-5pt}}
\renewcommand{\thesubsubsection}{\thesubsection\hspace*{5pt}.\arabic{subsubsection}\hspace*{-3pt}}

\newcommand{\Ack}{\section*{\protect\rmfamily Acknowledgments}\noindent}
\newcommand{\Contr}{\section*{\protect\rmfamily Contributors}\noindent}
\newcommand{\Contrl}{\section*{\protect\rmfamily Contributor}\noindent}

\makeindex


\begin{document}
\Rus

\nwt
%\ptb


%\renewcommand{\contentsname}{\protect\Large\bf Содержание}

\setcounter{tocdepth}{2}

%\tableofcontents

\renewcommand{\bibname}{\protect\rmfamily Литература}
  \def\Au#1{{\it #1}}
    \def\Aue#1{{#1}}

%\newcommand{\No}{№}
  \newcommand{\tg}{\,\mathrm{tg}\,}
    \newcommand{\ctg}{\,\mathrm{ctg}\,}
  \newcommand{\arctg}{\,\mathrm{arctg}\,}

\def\forallb{\mathop{\forall}}
\def\cupb{\mathop{\cup}}
\def\existsb{\mathop{\exists}}


\newpage
\addtocounter{razdel}{1}
%\def\razd{РЕГУЛИРУЕМЫЙ ЭЛЕКТРОПРИВОД ДЛЯ ЭЛЕКТРОЭНЕРГЕТИКИ}


\setcounter{page}{2}

%   { %\Large  
   { %\baselineskip=16.6pt
   
   \vspace*{-48pt}
   \begin{center}\LARGE
   \textit{Предисловие}
   \end{center}
   
   %\vspace*{2.5mm}
   
   \vspace*{25mm}
   
   \thispagestyle{empty}
   
   { %\small 

    
Вниманию читателей журнала <<Информатика и её применения>> предлагается 
очередной тематический выпуск <<Вероятностно-статистические методы и 
задачи информатики и информационных технологий>>. Предыдущие тематические 
выпуски журнала по данному направлению вышли в 2008~г.\ (т.~2, вып.~2), 
в 2009~г.\ (т.~3, вып.~3) и в 2010~г.\ (т.~4, вып.~2). 

Статьи, собранные в данном журнале, посвящены разработке новых вероятностно-статистических 
методов, ориентированных на применение к решению конкретных задач информатики и информационных 
технологий, а также~--- в ряде случаев~--- и других прикладных задач. Проблематика, охватываемая 
публикуемыми работами, развивается в рамках научного сотрудничества между Институтом проблем 
информатики Российской академии наук (ИПИ РАН) и Факультетом вычислительной математики и 
кибернетики Московского государственного университета им.\ М.\,В.~Ломоносова в ходе работ 
над совместными научными проектами (в том числе в рамках функционирования 
Научно-образовательного центра <<Вероятностно-статистические методы анализа рисков>>). 
Многие из авторов статей, включенных в данный номер журнала, являются активными участниками 
традиционного международного семинара по проблемам устойчивости стохастических моделей, 
руководимого В.\,М.~Золотаревым и В.\,Ю.~Королевым; регулярные сессии этого семинара 
проводятся под эгидой МГУ и ИПИ РАН (в 2011~г.\ указанный семинар проводится в октябре 
в Калининградской области РФ). 

Наряду с представителями ИПИ РАН и МГУ в число авторов данного выпуска журнала входят 
ученые из Научно-исследовательского института системных исследований РАН, Института 
проблем технологии микроэлектроники и особочистых материалов РАН, Института 
прикладных математических исследований Карельского НЦ РАН, Московского 
авиационного института, Вологодского государственного педагогического университета, 
НИИММ им.\ Н.\,Г.~Чеботарева, Казанского государственного университета, Дебреценского 
университета (Венгрия).

Несколько статей выпуска посвящено разработке и применению стохастических методов и 
информационных технологий для решения различных прикладных задач. В~работе В.\,Г.~Ушакова 
и О.\,В.~Шестакова рассмотрена задача определения вероятностных характеристик случайных 
функций по распределениям интегральных преобразований, возникающих в задачах эмиссионной 
томографии. В~статье Д.\,О.~Яковенко и М.\,А.~Целищева рассмотрены некоторые вопросы 
математической теории риска и предложен новый подход к диверсификации инвестиционных 
портфелей. Работа И.\,А.~Кудрявцевой и А.\,В.~Пантелеева посвящена построению и 
исследованию математической модели, описывающей динамику сильноионизованной плазмы. 
В~статье П.\,П.~Кольцова изучается качество работы ряда алгоритмов сегментации изображений. 
Статья А.\,Н.~Чупрунова и И.~Фазекаша посвящена вероятностному анализу числа без\-оши\-бочных 
блоков при помехоустойчивом кодировании; получены усиленные законы больших чисел для указанных 
величин.

В данном выпуске традиционно присутствует тематика, весьма активно разрабатываемая в течение 
многих лет специалистами ИПИ РАН и МГУ,~--- методы моделирования и управления для 
информационно-телекоммуникационных и вычислительных систем, в частности методы 
теории массового обслуживания. В~статье А.\,И.~Зейфмана с соавторами рассматриваются 
модели обслуживания, описываемые марковскими цепями с непрерывным временем в случае 
наличия катастроф. В~работе М.\,М.~Лери и И.\,А.~Чеплюковой рассматриваются случайные 
графы Интернет-типа, т.\,е.\ графы, степени вершин которых имеют степенные распределения; 
такие задачи находят применение при исследовании глобальных сетей передачи данных. 
Работа Р.\,В.~Разумчика посвящена исследованию систем массового обслуживания специального 
вида~--- с отрицательными заявками и хранением вытесненных заявок.

Ряд статей посвящен развитию перспективных теоретических 
вероятностно-статистических методов, которые находят широкое применение в различных 
задачах информатики и информационных технологий. В~работе В.\,Е.~Бенинга, А.\,К.~Горшенина 
и В.\,Ю.~Королева рассмотрена задача статистической проверки гипотез о числе компонент 
смеси вероятностных распределений, приводится конструкция асимптотически наиболее мощного 
критерия. Результаты этой работы найдут применение в ряде прикладных задач, использующих 
математическую модель смеси вероятностных распределений (в информатике, моделировании 
финансовых рынков, физике турбулентной плазмы и~т.\,д.). В~статье В.\,Ю.~Королева, 
И.\,Г.~Шевцовой и С.\,Я.~Шоргина строится новая, улучшенная оценка точности нормальной 
аппроксимации для пуассоновских случайных сумм; как известно, указанные случайные суммы 
широко используются в качестве моделей многих реальных объектов, в том числе в информатике, 
физике и других прикладных областях. Работа В.\,Г.~Ушакова и Н.\,Г.~Ушакова посвящена 
исследованию ядерной оценки плотности распределения; эти результаты могут применяться, 
в част\-ности, при анализе трафика в телекоммуникационных системах. Серьезные приложения 
в статистике могут получить результаты работы О.\,В.~Шестакова, в которой доказаны оценки 
скорости сходимости распределения выборочного абсолютного медианного отклонения к нормальному 
закону. 

\smallskip

Редакционная коллегия журнала выражает надежду, что данный тематический  выпуск 
будет интересен специалистам в области теории вероятностей и математической статистики 
и их применения к решению задач информатики и информационных технологий.
     
     %\vfill 
     \vspace*{20mm}
     \noindent
     Заместитель главного редактора журнала <<Информатика и её 
применения>>,\\
     директор ИПИ РАН, академик  \hfill
     \textit{И.\,А.~Соколов}\\
     
     \noindent
     Редактор-составитель тематического выпуска,\\
     профессор кафедры математической статистики факультета\\
      вычислительной математики и кибернетики МГУ им.\ М.\,В.~Ломоносова,\\
     ведущий научный сотрудник ИПИ РАН,\\ 
доктор физико-математических наук \hfill
      \textit{В.\,Ю.~Королев}
     
     } }
     }


\def\stat{sinits}

\def\tit{АНАЛИТИЧЕСКОЕ МОДЕЛИРОВАНИЕ
НОРМАЛЬНЫХ ПРОЦЕССОВ В~СТОХАСТИЧЕСКИХ СИСТЕМАХ СО~СЛОЖНЫМИ~НЕЛИНЕЙНОСТЯМИ}

\def\titkol{Аналитическое моделирование
нормальных процессов в~стохастических системах со~сложными нелинейностями}

\def\aut{И.\,Н.~Синицын$^1$, В.\,И.~Синицын$^2$}

\def\autkol{И.\,Н.~Синицын, В.\,И.~Синицын}

\titel{\tit}{\aut}{\autkol}{\titkol}

\renewcommand{\thefootnote}{\arabic{footnote}}
\footnotetext[1]{Институт проблем
информатики Российской академии наук, sinitsin@dol.ru}
\footnotetext[2]{Институт проблем
информатики Российской академии наук, vsinitsin@ipiran.ru}


\Abst{Рассматриваются конечномерные дифференциальные стохастические системы
(ДСтС) и эредитарные (интегродифференциальные) стохастические системы  (ЭСтС)
с винеровскими и пуассоновскими шумами, приводимые к ДСтС со сложными конечными,
дифференциальными и интегральными нелинейностями. Такие модели функционирования
описывают поведение многих современных нано- и кван\-то\-во-оп\-ти\-че\-ских
технических средств информатики. Приводятся уравнения методов нормальной
аппроксимации (МНА) и статистической линеаризации (МСЛ) для аналитического
моделирования нестационарных и стационарных нормальных (гауссовских) процессов
в нелинейных ДСтС и  нелинейных ЭСтС путем аппроксимации эредитарных ядер
линейными операторными уравнениями для дифференцируемых нелинейностей и
сингулярными ядрами для недифференцируемых нелинейностей. Рассматриваются
методы вычисления типовых интегралов МНА (МСЛ) для сложных (многомерных и
векторного аргумента) конечных и дифференциальных нелинейностей. Особое
внимание уделяется иррациональным и дробно-рациональным нелинейностям
скалярного аргумента. Приводятся примеры вычисления интегралов. Подробно
рассматриваются вопросы вычисления типовых интегралов МНА (МСЛ) для сложных
интегральных нелинейностей.}

\KW{аналитическое моделирование;
дифференциальные стохастические системы с винеровскими и пуассоновскими шумами (ДСтС);
метод нормальной аппроксимации (МНА);
метод статистической линеаризации (МСЛ);
сложные иррациональные нелинейности;
сложные конечные, дифференциальные и интегральные нелинейности;
эредитарные стохастические системы (ЭСтС), приводимые к дифференциальным}

\DOI{10.14357/19922264140302}

\vspace*{9pt}

\vskip 16pt plus 9pt minus 6pt

\thispagestyle{headings}

\begin{multicols}{2}

\label{st\stat}


\section{Введение}


Моделями функционирования многих современных технических сис\-тем информатики
служат стохастические системы (СтС), описываемые дифференциальными, интегральными
и интегродифференциальными уравнениями со сложными дроб\-но-ра\-ци\-о\-наль\-ны\-ми,
иррациональными и интегральными нелинейностями. В~[1] дано систематическое
изложение МНА и МСЛ для ДСтС и ЭСтС, приводимых к дифференциальным.

Обобщая~[2--7], рассмотрим развитие МНА и МСЛ для аналитического моделирования
нормальных стохастических процессов (СтП) на случай СтС со сложными конечными,
дифференциальными и интегральными нелинейностями.

Как показано в~\cite{4-sin}, альтернативным подходом к аналитическому моделированию
СтП в ДСтС и ЭСтС служит подход, основанный на дискретизации стохастических
дифференциальных уравнений на основе использования обобщенной формы Ито и
кратных стохастических интегралов от винеровских и пуассоновских СтП с
последующим применением дискретных версий МНА (МСЛ).

Статья состоит из введения, пяти разделов и заключения.

В~разд.~2 и~3
приводятся уравнения МНА и МСЛ для аналитического моделирования одно- и
двумерных распределений стационарных и нестационарных СтП в ДСтС и ЭСтС,
приводимых к ДСтС.

Типовые интегралы МНА и МСЛ рассматриваются в разд.~4.

Особенности аналитического моделирования в ДСтС со сложными конечными и
дифференциальными нелинейностями обсуждаются в разд.~5.

Раздел~6
посвящен аналитическому моделированию СтП в ДСтС со сложными интегральными
нелинейностями.

Приводятся примеры.


\section{Уравнения методов нормальной~аппроксимации и~статистической
линеаризации для~дифференциальных стохастических систем}

Как известно~\cite{2-sin, 3-sin},  уравнения конечномерных непрерывных нелинейных сис\-тем
со стохастическими возмущениями путем расширения вектора состояния ДСтС
могут быть записаны в виде следующего векторного стохастического
дифференциального уравнения Ито:
    \begin{multline}
    dY_t = a(Y_t, t)\, dt + b (Y_t, t) \,dW_0+{}\\
    {}+ \iii_{R_0} c (Y_t, t, v) P^0
    (dt, dv)\,,\enskip Y(t_0) = Y_0\,.\label{e2.1-sin}
    \end{multline}
Здесь $a=a(Y_t, t)$ и $b\hm=b(y_t, t)$~--- известные
$(p\times 1)$-мер\-ная и  $(p\times m)$-мер\-ная функции~$Y_t$ и~$t$;
$W_0\hm= W_0(t)$~--- $r$-мер\-ный винеровский СтП интенсивности
$\nu_0 \hm= \nu_0(t)$; $c(Y_t, t, v)$~--- $(p\times 1)$-мер\-ная функция  $Y_t, t$
и вспомогательного $(q\times 1)$-мер\-но\-го па\-ра\-мет\-ра~$v$;
$\iii_{\Delta} dP^0 (t, A)$~--- центрированная пуассоновская мера,
определяемая
\begin{equation*}
\iii_{\Delta} dP^0 (t, A) = \iii_{\Delta} dP (t,A) =
\iii_{\Delta} \nu_P (t,A)\, dt\,. %\label{e2.2-sin}
\end{equation*}
В~(\ref{e2.1-sin}) принято: $\iii_{\Delta}$~-- число скачков пуассоновского
СтП в интервале времени  $\Delta \hm= (t_1, t_2]$; $\nu_P (t, A)$~---
интенсивность пуассоновского СтП  $P(t,A)$; $A$~--- некоторое борелевское
множество пространства  $R_0^q$ с выколотым началом.
Начальное значение~$Y_0$ представляет собой случайную величину, не зависящую
от приращений СтП  $W_0(t)$ и $P(t,A)$ на интервалах времени, следующих
за~$t_0$, $t_0 \hm\le t_1\hm\le t_2$ для любого множества~$A$.

В случае аддитивных нормальных (гауссовских) и обобщенных
пуассоновских возмущений уравнение~(\ref{e2.1-sin}) имеет вид:
\begin{equation}
\dot Y_t = a(Y_t,t)+ b_0 (t) V\,, \enskip
V = \dot W\,,\enskip Y(t_0) = Y_0\,.\label{e2.3-sin}
\end{equation}
Здесь $W$~--- СтП с независимыми приращениями, представляющий собой
смесь нормального и обобщенного пуассоновского СтП.

Если предположить существование конечных вероятностных
моментов второго порядка для моментов времени~$t_1$ и~$t_2$, то уравнения
МНА примут следующий вид~\cite{2-sin, 3-sin}:
\begin{itemize}
\item  для характеристических функций
    \begin{equation}
    g_1^N (\la;t) =\exp \lk i\la^{\mathrm{T}} m_t - \fr{1}{2}\, \la^{\mathrm{T}} K_t \la\rk\,;\label{e2.4-sin}
    \end{equation}
\begin{equation}
\hspace*{-7.5mm}g_{t_1, t_2}^N (\la_1, \la_2;t_1, t_2 ) =\exp \lk i\bar \la^{\mathrm{T}} \bar m_2 -
\fr{1}{2}\, \bar \la^{\mathrm{T}} \bar K_2 \la\rk\,,\!\!\label{e2.5-sin}
\end{equation}
где
    \begin{gather*}
    \bar \la =\lk \la_1^{\mathrm{T}}\la_2^{\mathrm{T}}\rk^{\mathrm{T}}\,; \quad
        \bar m_2 = \lk m_{t_1}^{\mathrm{T}} m_{t_2}^{\mathrm{T}}\rk^{\mathrm{T}}\,;\\
        \bar K_2= \begin{bmatrix}
    K(t_1, t_1)& K(t_1, t_2)\\
    K(t_2, t_1)& K(t_2, t_2)
    \end{bmatrix}\,;
    \end{gather*}

\item для математических ожиданий  $m_t$, ковариационной матрицы~$K_t$ и
матрицы ковариационных функций $K(t_1, t_2)$:
    \begin{equation}
    \dot m_t = a_1 (m_t, K_t, t)\,,\enskip m_0 = m(t_0)\,;\label{e2.6-sin}
    \end{equation}
\begin{equation}
\dot K_t = a_2 (m_t, K_t, t)\,,\enskip K_0 = K(t_0)\,;\label{e2.7-sin}
\end{equation}

\vspace*{-12pt}

\noindent
\begin{multline}
\fr{\prt K(t_1, t_2)}{\prt t_2 }= K(t_1, t_2) a_{21} (m_{t_2}, K_{t_2}, t_2)^{\mathrm{T}}\,;\\
K(t_1, t_1) = K_{t_1}\,.
\label{e2.8-sin}
\end{multline}
    \end{itemize}
Здесь приняты следующие обозначения:
\begin{equation}
a_1 = a_1 (m_t, K_t, t) = M_N a (Y_t, t)\,;\label{e2.9-sin}
\end{equation}

\vspace*{-12pt}

\noindent
\begin{multline}
a_2 = a_2 (m_t, K_t, t) = a_{21} (m_t, K_t, t)+{}\\
{}+ a_{21} (m_t, K_t, t)^{\mathrm{T}} +
a_{22}(m_t, K_t, t)\,;\label{e2.10-sin}
\end{multline}

\vspace*{-12pt}

\noindent

\begin{equation}
a_{21} = a_{21}(m_t, K_t, t)=  M_N a(Y_t, t) Y_{t}^{0\mathrm{T}}\,;\label{e2.11-sin}
\end{equation}
\begin{equation*}
a_{22} = a_{22}(m_t, K_t, t)= M_N \sigma (Y_t, t)\,;
%\label{e2.12-sin}
\end{equation*}

\vspace*{-12pt}

\noindent
\begin{multline*}
\sigma (Y_t, t) = b(Y_t, t) \nu_0(t) b(Y_t, t)^{\mathrm{T}} +{}\\
{}+
\iii_{R_0^q} c (Y_t, t, v) c(Y_t, t,v)^{\mathrm{T}}
\nu_P (t, dv)\,; %\label{e2.13-sin}
\end{multline*}

\vspace*{-12pt}

\begin{gather*}
m_t = MY_t\,,\quad Y_t^0 = Y_t - m_t\,,\\
K_t = M_N Y_0^0 Y_t^{0\mathrm{T}}\,,\quad K(t_1, t_2) =
M_N Y_{t_1}^0 Y_{t_2}^0\,; %\label{e2.14-sin}
\end{gather*}
$M_N$~--- символ вычисления математического ожидания для нормальных
распределений~(\ref{e2.4-sin}) и~(\ref{e2.5-sin}).

Для стационарных ДСтС нормальные стационарные СтП~--- если они существуют,
то  $m_t \hm=\bar m$, $ K_t \hm=\bar K$, $K(t_1, t_2) \hm= k(\tau)$
$(\tau \hm= t_1\hm-t_2)$,~--- определяются уравнениями~\cite{2-sin, 3-sin}:
   \begin{equation}
   a_1 (\bar m, \bar K) =0\,;\enskip a_2 (\bar m, \bar K)=0\,;\label{e2.15-sin}
   \end{equation}
   \begin{equation}
   \left.
   \hspace*{-2.8mm}\begin{array}{l}
  \dot k_\tau (\tau) = a_{21} (\bar m, \bar K)\bar K^{-1} k(\tau)\,;\\[9pt]
  k(0) =\bar K \enskip (\forall \tau >0)\,, \
  k(\tau) = k(-\tau)^{\mathrm{T}} \enskip
  (\forall\tau <0)\,.
  \end{array}\!\!
  \right\}\!\!
  \label{e2.16-sin}
  \end{equation}
При этом необходимо, чтобы матрица  $a_{21} (\bar m, \bar K)\hm=\bar a_{21}$
была бы асимптотически устойчивой.

Для ДСтС~(\ref{e2.3-sin}) уравнения МНА переходят в уравнения МСЛ
Казакова~\cite{2-sin, 3-sin}, если принять
\begin{equation}
a(Y_t,t) = a_1 (m_t, K_t) + k_1^a (m_t, K_t) Y_t^0\,;\label{e2.17-sin}
\end{equation}
\begin{equation}\left.
\begin{array}{rl}
b(Y_t,t) &= b_0 (t)\,;\\[9pt]
    \si(Y_t, t)&= b_0(t) \nu(t) b_0(t)^{\mathrm{T}} = \si_0(t)\,,
    \end{array}
    \right\}\label{e2.18-sin}
    \end{equation}
    \begin{equation}
k_1^a (m_t, K_t, t) =\lk \left(\fr{\prt}{\prt m_t} \right)
    a_0 (m_t, K_t, t)^{\mathrm{T}}\rk^{\mathrm{T}}\,;\label{e2.19-sin}
    \end{equation}
    \begin{equation}
\dot m_t = a_1 (m_t, K_t, t) \,,\enskip m_0 = m(t_0)\,,\label{e2.20-sin}
\end{equation}

\vspace*{-12pt}

\noindent
\begin{multline}
\dot K_t = k_1^a (m_t, K_t, t) K_t + K_t k_1^a (m_t, K_t, t)^{\mathrm{T}}
    +\si_0(t)\,;\\
    K_0 = K(t_0)\,;
    \label{e2.21-sin}
    \end{multline}

    \vspace*{-12pt}

    \noindent
\begin{multline}
\fr{\prt K(t_1, t_2)}{\prt t_2} =
    K(t_1, t_2) k_{t_2} k_1^a (m_{t_2}, K_{t_2}, t_2)^{\mathrm{T}}\,;\\
    K(t_1, t_2) = K_{t_1}\,.
    \label{e2.22-sin}
\end{multline}

Для стационарных ДСтС~(\ref{e2.3-sin})
при условии асимптотической устойчивости матрицы $k_1^a (\bar m, \bar K)$
в основе МСЛ лежат уравнения~(\ref{e2.15-sin}), записанные в виде:
    \begin{gather}
    a_1 (\bar m, \bar K) =0\,; \label{e2.23-sin}\\
k_1^a (\bar m, \bar K) \bar K + \bar K k_1^a
(\bar m, \bar K)^{\mathrm{T}} +\bar \si_0 =0\,;\label{e2.24-sin}
\end{gather}

\vspace*{-12pt}

\noindent
\begin{multline}
k_\tau (\tau) = k_1^a (\bar m, \bar K)k(\tau)\,,\enskip
k(0) =\bar K \enskip (\forall \tau >0)\,,\\
k(\tau) = k (-\tau)^{\mathrm{T}} \enskip (\forall \tau <0)\,.
\label{e2.25-sin}
\end{multline}

Уравнения~(\ref{e2.4-sin})--(\ref{e2.8-sin})
лежат в основе МНА для ДСтС~(\ref{e2.1-sin}), а уравнения~(\ref{e2.17-sin})--(\ref{e2.22-sin})~---
в основе МСЛ для ДСтС~(\ref{e2.3-sin}). Для определения стационарных СтП
согласно МНА служат соотношения~(\ref{e2.15-sin}) и~(\ref{e2.16-sin}),
а МСЛ~--- (\ref{e2.17-sin})--(\ref{e2.25-sin}).

\section{Уравнения методов нормальной~аппроксимации и~статистической линеаризации
для~эредитарных стохастических систем, приводимых к~дифференциальным}

Рассмотрим ЭСтС, описываемую интегродифференциальным уравнением Ито
следующего вида~\cite{7-sin}:

\noindent
\begin{multline}
dX_t = \lk a(X_t, t) +\iii_{t_0}^t a_1 (X(\tau) ,\tau, t)\,d\tau\rk dt+{}\\
{}+\lk b(X_t, t) +\iii_{t_0}^t b_1 (X(\tau) ,\tau, t)\,d\tau\rk dW_0+{}\\
\hspace*{-1.5mm}{}+\!\!\iii_{R_0^q}\!\!\lk c(X_t, t,v) +\!\iii_{t_0}^t\! c_1 (X(\tau) ,\tau, t,v)\,d\tau\!\rk\! dP^0 (t, dv)
\!\!\!\!\label{e3.1-sin}
\end{multline}
с начальным условием  $X(t_0) = X_0$. В~(\ref{e3.1-sin})
сохранены обозначения разд.~2.

Функции $a=a(X_t, t)$, $a_1\hm = a_1(X (\tau),\tau, t)$,
$b\hm=b(X_t, t)$, $b_1\hm = b_1(X (\tau),\tau, t)$,
$c\hm=c(X_t,t,v)$ и $c_1\hm = c_1(X (\tau),\tau, t,v)$ имеют
соответственно размерности $p\times 1$, $p\times 1$, $p\times r$,
$p\times r$, $p\times 1$ и $p\times 1$ и допускают представления следующего вида:
\begin{equation}
\left.
\begin{array}{rl}
a_1&=A(t,\tau) \vrp (X(\tau) , \tau)\,;\\[9pt]
b_1&=B(t,\tau) \psi (X(\tau) ,  \tau)\,;\\[9pt]
c_1&=C(t,\tau) \chi (X(\tau) ,  \tau, v)\,.
\end{array}
\right\}
\label{e3.2-sin}
\end{equation}
Здесь эредитарные ядра $A\hm=A(t,\tau)\hm=\lk A_{ij}(t,\tau)\rk$
$(i,j\hm=\overline{1,p})$,
$B\hm=B(t,\tau)=\lk B_{i l}(t,\tau)\rk$ $(i\hm=\overline{1,p}$;
$l\hm=\overline{1,r})$ и $C\hm=C(t,\tau)=\lk C_{ij}(t,\tau)\rk$
$(i,j\hm=\overline{1,p})$ имеют соответственно размерности
$p\times p$, $p\times r$ и $p\times p$. Они удовлетворяют следующим условиям
физической реализуемости и асимптотического затухания:
\begin{multline}
A_{ij}(t,\tau)=0;\enskip B_{i l}(t,\tau)=0;\\[1pt]
C_{ij}(t,\tau)=0\enskip \forall \tau >t;\label{e3.3-sin}
\end{multline}

\vspace*{-12pt}

\begin{equation}
\left.
\hspace*{-3mm}\begin{array}{c}
\displaystyle\iin\! \lv A_{ij} (t,\tau) \rv d\tau <\infty\,;\
\displaystyle\iin\! \lv B_{i l} (t,\tau) \rv d\tau <\infty \,;\\[9pt]
\displaystyle\iin \!\lv C_{ij} (t,\tau) \rv d\tau <\infty\,.
\end{array}\!
\right\}\!
\label{e3.4-sin}
\end{equation}

В дальнейшем ограничимся случаем, когда эредитарные ядра удовлетворяют
линейным операторным уравнениям~\cite{6-sin, 5-sin, 7-sin}.

Нелинейные в общем случае функции $\vrp\hm=\vrp(X(\tau),\tau)$,
$\psi \hm=\psi(X(\tau), \tau)$, $\chi \hm=\chi (X(\tau),  \tau, v)$
отражают нелинейные свойства ЭСтС, зависят от  $X(\tau)$ и имеют размерности
$p\times 1$, $p\times p$, $p\times 1$ соответственно.

Важный класс  эредитарных ядер представляют собой
сингулярные (вырожденные) ядра, когда имеют место представления:
\begin{equation}
\left.
\hspace*{-3mm}\begin{array}{rl}
A_{ij} (t,\tau) &= A_{ij}^+(t) A_{ij}^-(\tau)\,;\\[9pt]
B_{i l} (t,\tau)& = B_{il}^+(t) B_{il}^-(\tau)\,;\\[9pt]
C_{ij} (t,\tau) &= C_{ij}^+ ( t) C_{ij}^- (\tau)\
(i,l= \overline{1,p}, j=\overline{1,r}).
\end{array}\!
\right\}\!\!
\label{e3.5-sin}
\end{equation}

В~\cite{6-sin, 5-sin, 7-sin} показано, что для дифференцируемых нелинейных
функций~$\vrp$, $\psi$, $\chi$ путем расширения вектора состояния за счет
инструментальных переменных, аппроксимируемых линейными операторными уравнениями,
определяющими эредитарные ядра в ЭСтС, (\ref{e3.1-sin})--(\ref{e3.4-sin})
приводятся к ДСтС вида~(\ref{e2.1-sin}) или~(\ref{e2.3-sin}).
В~случае недифференцируемых нелинейных функций~$\vrp$, $\psi$, $\chi$
ЭСтС~(\ref{e3.1-sin})--(\ref{e3.4-sin}) приводятся к~(\ref{e2.1-sin}) или~(\ref{e2.3-sin})
на основе аппроксимации вырожденными (сингулярными) ядрами~\cite{6-sin, 5-sin, 7-sin}.

Таким образом, после приведения ЭСтС~(\ref{e3.1-sin}) к ДСтС~(\ref{e2.1-sin})
или~(\ref{e2.3-sin}) можно воспользоваться уравнениями МНА и МСЛ разд.~2.

\section{Типовые интегралы методов нормальной аппроксимации и~статистической
линеаризации}

Как следует из уравнений~(\ref{e2.9-sin})--(\ref{e2.11-sin}),
для МНА необходимо уметь вычислять следующие интегралы:
\begin{multline}
I_0^a = I_0^a (m_t, K_t, t) = a_1 (m_t, K_t, t)={}\\
{}= M_N a(Y_t, t)\,;
\label{e4.1-sin}
\end{multline}

\vspace*{-12pt}

\noindent
\begin{multline}
I_1^a = I_1^a (m_t, K_t, t)= a_{21}(m_t, K_t, t)= {}\\
{}=M_N a(Y_t , t) Y_t^{0\mathrm{T}}\,;\label{e4.2-sin}
\end{multline}

\vspace*{-12pt}

\noindent
\begin{multline}
I_0^{\bar \si} = I_0^{\bar \si} (m_t, K_t, t) = a_{22}(m_t, K_t, t) ={}\\
{}= M_N \bar \si (Y_t, t)\,.\label{e4.3-sin}
\end{multline}
Для МСЛ достаточно вычислить интеграл~(\ref{e4.1-sin}),
причем интеграл~$I_1^a$ вычисляется по формуле~\cite{2-sin, 3-sin, 4-sin}:
\begin{equation*}
k_1^a = k_1^a (m_t, K_t, t)=\lk \left( \fr{\prt}{\prt m_t}\right)
I_0^a (m_t, K_t, t)^{\mathrm{T}}\rk^{\mathrm{T}}. %\label{e4.4-sin}
\end{equation*}

\medskip

\noindent
\textbf{Пример 1.} В~[1] для типовых степенных, тригоно\-мет\-ри\-че\-ских,
показательных и ку\-соч\-но-по\-сто\-ян\-ных нелинейностей $Z_t \hm=\vrp (Y_t, t)$
скалярного и векторного аргумента приведены формулы для интегралов
$I_0^\vrp \hm= I_0^\vrp (m_t^y, K_t^y, t)$, а также
$k_1^\vrp \hm= k_1^\vrp (m_t^y, K_t^y, t)$.

\medskip

\noindent
\textbf{Замечание.}
 Важно иметь в виду, что уравнения МНА (МСЛ) содержат интегралы
 $I_0^a$, $I_1^a$, $I_0^\si$ в виде соответствующих коэффициентов.
 Поэтому процедура вычисления интегралов должна быть согласована с
 методом численного решения обыкновенных дифференциальных уравнений для
 $m_t$, $K_t$ и $K(t_1, t_2)$. Эти коэффициенты допускают дифференцирование
 по~$m_t$ и~$K_t$, так как под интегралом стоит сглаживающая нормальная плотность.

\section{Сложные конечные и~дифференциальные нелинейности}

Важный класс сложных конечных нелинейностей (многомерных и векторного аргумента)
представляют собой сложные функции вида:
    \begin{equation*}
    \xi =\vrp (X_t, Y_t, t)\,,\enskip X_t =\psi (Y_t, t)\,. %\label{e5.1-sin}
    \end{equation*}
В~этом случае вычисление интегралов (см.\ разд.~4) проводится по совокупности
переменных  $\lk X_t^{\mathrm{T}} Y_t^{\mathrm{T}}\rk^{\mathrm{T}}$.
К таким нелинейностям, например, относятся дроб\-но-ра\-ци\-о\-наль\-ные,
иррациональные  нелинейности, выражаемые специальными функциями, многозначные
нелинейности, зависящие от СтП~$X_t$ и его производных~$\dot X_t$,  $\ddot X_t$
и~др.

\medskip

\noindent
\textbf{Пример 2.}
Рассмотрим вычисление интегралов~(\ref{e4.1-sin}) и~(\ref{e4.2-sin})
для сложных одномерных иррациональных нелинейностей скалярного аргумента
\begin{equation}
\vrp (Y_t, t) =\lv Y_t\rrv^{\alpha-1}\, \mathrm{sgn}\, Y_t
\label{e5.2-sin}
\end{equation}
($\alpha$~--- нецелый показатель).

Пользуясь~(\ref{e2.16-sin}) и~(\ref{e2.19-sin}), представим~(\ref{e5.2-sin}) в виде
\begin{equation*}
\vrp(Y_t, t) = \vrp_0 (m_t, D_t, t) + k_1^\vrp(m_t, D_t, t) Y_t^0. %\label{e5.3-sin}
\end{equation*}
Здесь введены следующие обозначения:
\begin{gather*}
\vrp_0(m_t, D_t, t) =\Gamma(\alpha) D_t^{1/2} e^{-\xi^2/4} D_{-\alpha} (\xi)\,;%\label{e5.4-sin}
\\
k_1^a (m_t, D_t, t) =\fr {\prt \vrp_0(m_t, D_t, t)}{\prt m_t}\,,%\label{e5.5-sin}
\end{gather*}
где  $\Gamma(\alpha)$~--- гамма-функция,  $\xi \hm= m_t/\sqrt{D_t}$~---
отношение <<сиг\-нал--шум>>; $D_{-\alpha} (\xi)$~---
функция параболического цилиндра~\cite{9-sin}.
При вычислении были учтены следующие соотношения~\cite{9-sin, 8-sin}:
\begin{multline}
\iii_0^\infty x^{\alpha-1} e^{-\beta x^2 - \gamma x} \,dx ={}\\
{}=
(2\beta)^{-\alpha/2} \Gamma(\alpha) \exp \left(\fr{\gamma^2}{8\beta}\right)
D_{-\alpha} \left(\fr{\gamma}{\sqrt{2\beta}}\right)\,;\label{e5.6-sin}
\end{multline}

\vspace*{-12pt}

\noindent
\begin{multline}
\fr{dD_\rho(\xi)}{d\xi} =
   -\fr{\xi}{2}\, D_\rho (\xi) -\rho D_{\rho-1} (\xi) =
   \fr{\xi}{2}\, D_\rho (\xi) -{}\\
   {}- D_{\rho+1} (\xi) \enskip
   (\mathrm{Re}\, \beta>0\,,\enskip \mathrm{Re}\,\alpha>0\,,\enskip
   \rho=-\alpha)\,.\label{e5.7-sin}
   \end{multline}

Соотношения~(\ref{e5.6-sin}) и~(\ref{e5.7-sin})
могут быть использованы также для вычисления интегралов~(\ref{e4.3-sin}).

\medskip

\noindent
\textbf{Замечание.}
Для вычисления интегралов $I_0^a$, $I_1^a$ и $I_0^{\bar \si}$
применительно к типовым иррациональным нелинейностям вида
    $\lv Y_t\rrv^{\alp-1} e^{\delta Y_t}$, $\lv Y_t\rrv^{\alp-1}  \cos \w Y_t$,
    $\lv Y_t\rrv^{\alp-1}  \sin \w Y_t$
и более общим нелинейностям \mbox{вида}
    \begin{equation*}
    \vrp (Y_t, t) =\Phi^\vrp \left( \lv Y_t\rrv^{\alpha-1}, t\right) %\label{e5.8-sin}
    \end{equation*}
можно рекомендовать известные численные методы вычисления функций на ЭВМ~\cite{8-sin}.

\smallskip

\noindent
\textbf{Пример 3.}
Для нелинейной дроб\-но-ра\-ци\-о\-наль\-ной функции

\noindent
\begin{equation*}
\vrp (Y_t, t) = \fr{a}{(b+Y_t)^2} %\label{e5.9-sin}
\end{equation*}
имеем

\vspace*{-3pt}

\noindent
\begin{gather*}
\vrp_0 (m_t, D_t, t) =a b^{-2} \lk 1+ \chi (m_t, D_t, t)\rk\,; %\label{e5.10-sin}
\\
k_1^\vrp (m_t, D_t, t) =  a b^{-2}\fr{\prt \chi (m_t, D_t, t)}{\prt m_t}\,. %\label{e5.11-sin}
\end{gather*}
Здесь

\vspace*{-3pt}

\noindent
\begin{multline*}
\chi (m_t, D_t, t) ={}\\
{}=\sss_{n=1}^\infty \sss_{l=0}^{E(n/2)}
\fr{(-1)^n (n+1) n!}{(n-2l)! (2l)!}\, b^{-n} m_t^n \left( \fr{D_t}{ 2 m_t^2}
\right)^l, %\label{e5.12-sin}
\end{multline*}
где  $E(n/2)$~--- целая часть~$n/2$; $a\hm=a(t)$; $b\hm= b(t)$.

\vspace*{-6pt}

\section{Сложные интегральные нелинейности}

\vspace*{-2pt}

Пусть сначала векторно-матричная нелинейность имеет эредитарный характер, т.\,е.\
\begin{equation}
\underline{\vrp} (Y_t, t) =\iii_{t_0}^t A(t,\tau) \vrp (Y(\tau), \tau) \,d\tau\,.
\label{e6.1-sin}
\end{equation}
Тогда, как показано в~\cite{6-sin, 5-sin, 7-sin}, следует соответст\-ву\-ющие
интегродифференциальные соотношения путем введения  инструментальных
переменных привести к дифференциальным соотношениям.  Для
дифференцируемых функций~$\vrp$ и асимптотически устойчивых ядер
$A(t,\tau)$ зависимость~(\ref{e3.5-sin}) имеет следующий дифференциальный вид:
\begin{equation*}
F^A (t, D) \underline{\vrp} (Y_t, t) = H^A (t, D) \vrp (Y_t, t)\,. %\label{e6.2-sin}
\end{equation*}
Здесь $F^A (t, D)$ и  $H^A (t, D)$~--- линейные дифференциальные операторы $(D\hm= d/dt)$.

Для недифференцируемых функций~$\vrp$ и асимптотически устойчивых
сингулярных ядер~(\ref{e3.5-sin}) используются соотношения:
\begin{equation*}
\underline{\vrp} (Y_t, t) = A^+ Z\,,\enskip
\dot Z = A^- \vrp\,,\enskip
Z(t_0)=0\,. %\label{e6.3-sin}
\end{equation*}

Многочисленные примеры аналитического моделирования ЭСтС можно найти
в~[1--3, 5, 7, 10, 11].

Как отмечалось в~\cite{3-sin}, часто наряду с интегральными
нелинейностями~(\ref{e6.1-sin}) рассматривают нелинейности вида:

\columnbreak

\noindent
\begin{equation*}
Z_s =\sss_{\rho=1}^R \mathcal{A}_\rho \vrp_\rho (Y_{t_1}\tr Y_{t_r})\,, %\label{e6.2-sin}
\end{equation*}
где $\mathcal{A}_1 \tr \mathcal{A}_R$~--- произвольные линейные операторы,
действующие над функциями~$r$ переменных  $t_1\tr t_r$; $\vrp_\rho
\hm=\vrp_\rho (Y_{t_1} \tr Y_{t_r})$~--- линейные функции отмеченных
переменных. Такие нелинейности называются приводимыми к линейным.
Важным частным случаем~(\ref{e6.1-sin}) являются интегральные нелинейности вида:

\noindent
\begin{gather}
Z_s =\iii_T \vrp (Y_t, t, s)\, dt\,; \notag%\label{e6.3-sin}
\\
Z_s =\!\iii_T \!\cdots\!\iii_T\! \vrp (Y_{t_1}\tr Y_{t_r}; t_1\tr t_r, s)\,dt_1
\ldots dt_r,\notag %\label{e6.4-sin}
\end{gather}
В этом случае используется МСЛ по совокупности переменных  $Y_{t_1} \tr Y_{t_r}$.

\vspace*{-9pt}

\section{Заключение}

\vspace*{-2pt}

Разработаны методы и алгоритмы МНА и МСЛ для ДСтС и ЭСтС,
приводимых к ДСтС со сложными конечными, дроб\-но-ра\-ци\-о\-наль\-ны\-ми,
иррациональными, а также дифференциальными и интегральными нелинейностями.
Приведены примеры.

Результаты допускают обобщение на случай ДСтС и ЭСтС со
стохастическими нелинейностями, заданными каноническими разложениями и
интегральными каноническими  представлениями~\cite{1-sin, 3-sin, 11-sin}.

\vspace*{-9pt}

{\small\frenchspacing
 {%\baselineskip=10.8pt
 \addcontentsline{toc}{section}{References}
 \begin{thebibliography}{99}

 \vspace*{-2pt}

\bibitem{1-sin}
\Au{Синицын И.\,Н.,  Синицын~В.\,И.}
Лекции по нормальной и эллипсоидальной аппроксимации распределений в
стохастических сис\-те\-мах.~--- М.: ТОРУС ПРЕСС, 2013. 488~с.

\bibitem{6-sin} %2
\Au{Синицын И.\,Н. }
Stochastic hereditary control systems~// Проблемы управления и
теории информации, 1986. Т.~15. №\,4. С.~287--298.

\bibitem{2-sin} %3
\Au{Пугачев В.\,С., Синицын~И.\,Н.}
Стохастические дифференциальные сис\-те\-мы. Анализ и фильтрация.~--- М.:
Наука,  1990.  632~с. [Англ. пер.
 Stochastic differential systems.
Analysis and filtering.~--- Chichester, New York: Jonh Wiley, 1987.
549~p.].

\bibitem{5-sin} %4
\Au{Синицын И.\,Н. }
Конечномерные распределения процессов в стохастических интегральных
и интегродифференциальных системах~// Preprints of the 2nd IFAC
Symposium on Stochastic Control.~--- Vilnius: Pergamon Press,
1987.  Vol.~1. P.~144--153.

\bibitem{3-sin} %5
\Au{Пугачев В.\,С., Синицын~И.\,Н.}
Теория стохастических систем.~--- М.: Логос, 2000; 2004. 1000~с.
[Англ. пер.\linebreak\vspace*{-12pt}

\pagebreak

\noindent Stochastic systems. Theory and  applications.~---
Singapore: World Scientific, 2001. 908~p.].

\bibitem{4-sin} %6
\Au{Синицын И.\,Н.}
Параметрическое статистическое и аналитическое моделирование распределений
в нелинейных стохастических сис\-те\-мах на многообразиях~//
Информатика и её применения, 2013. Т.~7. Вып.~2. С.~4--16.

\bibitem{7-sin} %7
\Au{Синицын И.\,Н. }
Анализ и моделирование распределений в эредитарных стохастических
сис\-те\-мах~// Информатика и её применения, 2014. Т.~8. Вып.~1.\linebreak
С.~2--11.



\bibitem{9-sin} %8
\Au{Градштейн И.\,С., Рыжик~И.\,М.}
Таблицы интегралов, сумм, рядов и произведений.~--- М.: ГИФМЛ, 1963. 1100~с.

\bibitem{8-sin} %9
\Au{Попов Б.\,А., Теслер~Г.\,С. }
Вычисление функций на ЭВМ: Справочник.~--- Киев: Наукова Думка, 1984. 599~с.


\bibitem{11-sin} %10
\Au{Синицын И.\,Н.}
Канонические представления случайных функций и их применение в
задачах компьютерной поддержки научных исследований.~--- М.: ТОРУС
ПРЕСС, 2009. 768~с.

\bibitem{10-sin} %11
\Au{Синицын И.\,Н., Синицын~В.\,И., Корепанов~Э.\,Р., Белоусов~В.\,В.,
Сергеев~И.\,В., Басилашвили~Д.\,А.}
Опыт моделирования эредитарных стохастических сис\-тем~//
Кибернетика и высокие технологии XXI века: Сб. докл.  XIII Междунар.
науч.-технич. конф.~--- Воронеж: Саквоее, 2012. Т.~2. C.~346--357.

 \end{thebibliography}

 }
 }

\end{multicols}

\vspace*{-9pt}

\hfill{\small\textit{Поступила в редакцию 05.05.14}}

%\newpage

\vspace*{12pt}

\hrule

\vspace*{2pt}

\hrule

\vspace*{12pt}

\def\tit{ANALYTICAL MODELING OF NORMAL PROCESSES
 IN~STOCHASTIC SYSTEMS WITH~COMPLEX NONLINEARITIES}

\def\titkol{Analytical modeling of normal processes
 in~stochastic systems with~complex nonlinearities}

\def\aut{I.\,N.~Sinitsyn and V.\,I.~Sinitsyn}

\def\autkol{I.\,N.~Sinitsyn and V.\,I.~Sinitsyn}

\titel{\tit}{\aut}{\autkol}{\titkol}

\vspace*{-9pt}

\noindent
Institute of Informatics Problems, Russian Academy of Sciences,
44-2 Vavilov Str., Moscow 119333, Russian Federation


\def\leftfootline{\small{\textbf{\thepage}
\hfill INFORMATIKA I EE PRIMENENIYA~--- INFORMATICS AND
APPLICATIONS\ \ \ 2014\ \ \ volume~8\ \ \ issue\ 3}
}%
 \def\rightfootline{\small{INFORMATIKA I EE PRIMENENIYA~---
INFORMATICS AND APPLICATIONS\ \ \ 2014\ \ \ volume~8\ \ \ issue\ 3
\hfill \textbf{\thepage}}}

\vspace*{6pt}

\Abste{Differential stochastic systems (DStS) with Wiener and Poisson
noises and complex finite, differential, and  integral nonlinearities and
hereditary StS reducible to DStS are considered. Equations and algorithms
of analytical modeling based on the normal approximation method (NAM) and the
statistical linearization method (SLM) are given. The case of complex
continuous and discontinuous nonlinearities of scalar and vector arguments
is considered. Special attention is paid to NAM (SLM) typical integrals
for finite rational and irrational nonlinear and integral scalar and vector
nonlinear functions. The general case of integral nonlinearities reducible to
linear is considered. Test examples are given.}

\KWE{analytical modeling;
complex finite differential and integral nonlinearities;
complex irrational nonlinerarites
differential stochastic system with Wiener and Poisson noises;
method of normal approximation;
method of statistical linearization;
hereditary stochastic systems reducible to differential}

\DOI{10.14357/19922264140302}

  \begin{multicols}{2}

\renewcommand{\bibname}{\protect\rmfamily References}
%\renewcommand{\bibname}{\large\protect\rm References}

{\small\frenchspacing
 {%\baselineskip=10.8pt
 \addcontentsline{toc}{section}{References}
 \begin{thebibliography}{99}



\bibitem{1-sin-1}
\Aue{Sinitsyn, I.\,N., and  V.\,I.~Sinitsyn}.  2013.
Lektsii po normal'noy i ellipsoidal'noy approksimatsii raspredeleniy
v stokhasticheskikh sistemakh [Lectures on normal and ellipsoidal
approximation of distributions in stochastic systems].
Moscow: TORUS PRESS. 488~p.

\bibitem{6-sin-1} %2
\Aue{Sinitsyn, I.\,N.}  1986.
{Stochastic hereditary control systems}.
\textit{Problems Control Inform. Theory} 15(4):287--298.

\bibitem{2-sin-1} %3
\Aue{Pugachev, V.\,S., and  I.\,N.~Sinitsyn}.  1987.
\textit{Stochastic differential systems. Analysis and filtering.}
Chichester, New York: Jonh Wiley. 549~p.

\bibitem{5-sin-1} %4
\Aue{Sinitsyn, I.\,N.}  1987.
Konechnomernye raspredeleniya protsessov v stokhasticheskikh integral'nykh
i in\-teg\-ro\-dif\-fe\-ren\-tsial'nykh sistemakh [Finite dimensional distributions
of processes in stochastic integral and integrodifferential systems].
\textit{2nd  Symposium (International) IFAC on Stochastic Control
Preprints}. Vilnius: Pergamon Press. 1:144--153.

\bibitem{3-sin-1} %5
\Aue{Pugachev, V.\,S., and I.\,N.~Sinitsyn}. 2001.
\textit{Stochastic systems. Theory and  applications}.
Singapore: World Scientific. 908~p.

\bibitem{4-sin-1} %6
\Aue{Sinitsyn, I.\,N.}  2013.
Parametricheskoe statisticheskoe i analiticheskoe modelirovanie
raspredeleniy v nelineynykh stokhasticheskikh sistemakh na mnogoobraziyakh
[Parametric statistical and analytical modeling of distributions in
stochastic systems on manifolds].
\textit{Informatika i ee Primeneniya}~--- \textit{Inform. Appl.} 7(2):4--16.


\bibitem{7-sin-1} %7
\Aue{Sinitsyn, I.\,N.}  2014.
Analiz i modelirovanie raspredeleniy v ereditarnykh stokhasticheskikh sistemakh
[Analysis and modeling of distributions in hereditary stochastic systems].
\textit{Informatika i ee Primeneniya}~--- \textit{Inform. Appl.} 8(1):2--11.

\bibitem{9-sin-1} %8
\Aue{Gradshteyn, I.\,S., and I.\,M.~Ryzhik}.  1963.
\textit{Tablitsy integralov, summ, ryadov i proizvedeniy}
[Tables of integrals, sums, series, and products]. Moscow:  GIFML.   1100~p.

\pagebreak

\bibitem{8-sin-1} %9
\Aue{Popov, B.\,A., and G.\,S.~Tesler}.  1984.
\textit{Vychislenie funktsiy na EVM}. Spravochnik [Computing of functions].
Kiev: Naukova Dumka.  599~p.


\bibitem{11-sin-1} %10
\Au{Sinitsyn, I.\,N.} 2009.
\textit{Kanonicheskie predstavleniya sluchaynykh funktsiy i ikh primenenie v
zadachakh komp'yuternoy podderzhki nauchnykh issledovaniy}
[Canonical expansions of random functions and its application to
scientific computer-aided support]. Moscow: TORUS PRESS. 768~p.

\bibitem{10-sin-1} %11
\Aue{Sinitsyn, I.\,N., V.\,I.~Sinitsyn, E.\,R.~Korepanov,
V.\,V.~Belousov, I.\,V.~Sergeev, and D.\,A.~Basilashvili}.
2012. Opyt modelirovaniya ereditarnykh stokhasticheskikh sistem
[Experience of modeling in hereditary stochastic systems].
\textit{Kibernetika i Vysokie Tekhnologii XXI~Veka:
Sbornik dokladov  XIII Mezhdunar. nauch.-tekhnich. konf.}
[Cybernatics ans High Technologies of the XXI Century: Materials of
XIII  Scientific and Technological Conference (International)].
Voronezh: Sakvoee. 2:346--357.

\end{thebibliography}

 }
 }

\end{multicols}

\vspace*{-6pt}

\hfill{\small\textit{Received May 05, 2014}}

\vspace*{-18pt}

\Contr

\noindent
\textbf{Sinitsyn Igor N.} (b.\ 1940)~---
Doctor of Science in technology, professor, Honored scientist of RF, Head of Department, Institute of
Informatics Problems, Russian Academy of Sciences,
44-2 Vavilov Str., Moscow 119333, Russian
Federation; sinitsin@dol.ru

\vspace*{3pt}

\noindent
\textbf{Sinitsyn Vladimir I.} (b.\ 1968)~--- Doctor of Science in physics
and mathematics, associate professor, Head of Department, Institute of
Information Problems, Russian Academy of Sciences,
44-2 Vavilov Str., Moscow 119333, Russian Federation; VSinitsin@ipiran.ru




\label{end\stat}

\renewcommand{\bibname}{\protect\rm Литература} %1
\def\stat{kovalev}

\def\tit{МЕТОДЫ ТЕОРИИ КАТЕГОРИЙ В~МОДЕЛЬНО-ОРИЕНТИРОВАННОЙ СИСТЕМНОЙ 
ИНЖЕНЕРИИ}

\def\titkol{Методы теории категорий в~модельно-ориентированной системной 
инженерии}

\def\aut{С.\,П.~Ковалёв$^1$}

\def\autkol{С.\,П.~Ковалёв}

\titel{\tit}{\aut}{\autkol}{\titkol}

\index{Ковалёв С.\,П.}
\index{Kovalyov S.\,P.}


%{\renewcommand{\thefootnote}{\fnsymbol{footnote}} \footnotetext[1]
%{Исследование выполнено при финансовой поддержке Российского научного фонда (проект 16-11-10227).}}


\renewcommand{\thefootnote}{\arabic{footnote}}
\footnotetext[1]{Институт проблем управления им.\ В.\,А.~Трапезникова 
Российской академии наук,  \mbox{kovalyov@nm.ru}}

%\vspace*{-18pt}

\Abst{Предложен математический аппарат на базе теории категорий, который позволяет 
формально описывать и~строго исследовать процедуры применения моделей в~инженерной 
деятельности, составляющие сущность мо\-дель\-но-ори\-ен\-ти\-ро\-ван\-ной системной 
инженерии (Model-Based Systems Engineering, MBSE). В~основе аппарата лежит 
математическое представление сборочных чертежей (мегамоделей сис\-тем) диаграммами 
в~категориях, объектами которых служат модели, а~морфизмы представляют действия по 
сборке моделей сис\-тем из моделей компонентов. Адекватность аппарата обоснована исходя 
из требований стандартов, регламентирующих описание структуры систем, в~том числе 
IEC~81346. Предложены и~исследованы тео\-ре\-ти\-ко-ка\-те\-гор\-ные методы решения ряда 
практических задач сборки систем. Приведены примеры решения таких задач в~категориях, 
представляющих две ключевые области применения MBSE: гео\-мет\-ри\-че\-ское моделирование 
изделий сложной формы и~дис\-крет\-но-со\-бы\-тий\-ное имитационное моделирование 
поведения технических систем.}

\KW{модельно-ориентированная системная инженерия; мегамодель; теория категорий; 
копредел}



\DOI{10.14357/19922264170305} 


\vspace*{6pt}

\vskip 10pt plus 9pt minus 6pt

\thispagestyle{headings}

\begin{multicols}{2}

\label{st\stat}

\section{Введение}

   Модельно-ориентированная системная инженерия состоит в~формализованном применении моделирования в~
поддержке жизненного цикла сис\-тем, включая сбор требований, 
проектирование, проверку и~приемку, другие стадии~[1]. Модели, 
разрабатываемые в~ходе процедур MBSE, пригодны к~автоматической 
обработке на компьютерах. Это позволяет сначала задавать, верифицировать 
и~оптимизировать проектные решения на моделях <<в циф\-ре~и только потом 
воплощать <<в железе>>, снижая затраты на организацию жизненного цикла 
изделий и~сокращая сроки выполнения работ~[2].
   
   И все же внедрение технологий MBSE в~инженерную деятельность 
происходит медленно. Это связано во многом с~нехваткой единой 
концептуальной базы инженерного моделирования: предлагается много 
частных языков и~технологий, слабо совместимых друг с~другом и~плохо 
приспособленных для совместной разработки моделей большими 
мультидисциплинарными коллективами~[3]. Тем самым затрудняется переход 
от набора электронных чертежей к~полноценному электронно-цифровому 
макету (digital mock-up) промышленного изделия.
   
   Естественный, хотя и~<<трудный>>, подход к~получению результатов 
общего характера, унифи\-ци\-ру\-ющих разнородные технологии, состоит в~том, 
чтобы как можно более строго формализовать процедуры моделирования. 
Формализация позволит совершенствовать процедуры MBSE и~передавать их 
на исполнение компьютеру без пробелов и~искажений. Самый высокий уровень 
строгости достигается при привлечении математического аппарата, поскольку 
математика позволяет надежно доказывать или опровергать утверждения, 
ха\-рак\-те\-ри\-зу\-ющие корректность и~эффективность процедур.
   
   В настоящей работе предложен аппарат, основанный на математическом 
представлении сборочных чертежей (<<мегамоделей>> систем) 
ориенти-\linebreak рованными графами (диаграммами). Узлы такого\linebreak графа помечаются 
обозначениями моделей час\-тей, а~реб\-ра помечаются обозначениями действий\linebreak 
(activities), посредством которых части собираются в~систему. Представление 
структуры систем графами регламентируется, в~частности, стандартом 
IEC~81346~[4]. Естественным источником математических методов 
конструирования и~анализа мегамоделей служит теория категорий (см., 
например,~[5, 6]). Модели рассматриваются как объекты подходящих 
категорий, а~действия формально описываются морфизмами. Строятся 
и~исследу-\linebreak ются тео\-ре\-ти\-ко-ка\-те\-гор\-ные конструкции, опи\-сы\-ва\-ющие процедуры 
MBSE на абстрактном кон-\linebreak цептуальном уровне. Определенный опыт такого\linebreak 
исследования был накоплен в~инженерии программного обеспечения~[7] 
и~теперь может быть обобщен для системной инженерии в~целом. Например, 
сборке системы согласно некоторой мегамодели отвечает построение 
копредела диаграммы~--- универсальной конструкции~\cite{5-kov}.
   
   Статья построена следующим образом. В~разд.~2 приведен обзор 
принципов описания структуры сис\-тем согласно стандарту IEC~81346. 
Раздел~3 посвящен практическим проб\-ле\-мам мегамоделирования и~сборке 
сис\-тем. В~разд.~4 вводятся конструкции тео\-рии категорий, позволяющие 
формально решать задачи мегамоделирования. В~заключении приводятся 
выводы и~намечаются направления дальнейших исследований.

\section{Структура систем и~стандарт~IEC~81346}

   Важной проблемой MBSE, отмеченной во введении, является слабая 
совместимость языков и~инструмен\-тов моделирования от разных поставщиков. 
Основным подходом к~достижению совместимости является стандартизация~--- 
принятие обязывающих документов, устанавливающих требования и~принципы 
взаимозаменяемости инструментов. Многие стандарты определяют конкретные 
форматы машиночитаемой записи моделей, нейтральные относительно 
разработчиков инструментов MBSE. Примером служит формат описания 
твердотельных геометрических моделей STEP, стандартизованный семейством 
ISO~10303. Однако для формализации MBSE в~целом интерес представляют 
в~первую очередь стандарты более общего плана, унифицирующие принципы 
и~методы применения моделей в~жизненном цикле систем независимо от 
способа записи моделей. С~этой точки зрения внимания заслуживает 
международный стандарт IEC 81346-1:2009 <<Промышленные системы, 
установки и~обору\-до\-ва\-ние~--- принципы структурирования и~ссылочные 
обозначения~--- часть~1: основные правила>> (<<Industrial Systems, 
Installations and Equipment and Industrial Products~--- Structuring Principles and 
Reference Designations~--- Part~1: Basic Rules>>)~\cite{4-kov}. Стандарт не 
принят в~России, однако ряду его положений в~области структуры систем 
соответствует российский ГОСТ~2.053-2013 <<ЕСКД. Электронная структура 
изделия. Общие положения>>.
   
   В стандарте IEC~81346 рассматривается ряд вопросов моделирования 
структуры систем и~идентификации отдельных единиц в~составе систем. 
Системная единица названа в~стандарте объектом, причем принципиально не 
проводится различие между объектами реального мира, составляющими 
реально существующие системы, и~объектами мыслительной деятельности~--- 
моделями единиц, составляющими модели систем. Таким образом, стандарт 
выходит за рамки MBSE и~рассматривает ряд вопросов системной инженерии 
вообще. Иерар\-хи\-че\-ская структура системы (холархия~\cite{3-kov}) 
изображается деревом, узлы которого помечены обозначениями объектов. 
Важным достижением стандарта является выявление того факта, что одна и~та 
же система задается не одной, а несколькими в~общем случае различными 
иерархическими структурами, возникающими в~результате декомпозиции 
согласно различным принципам (аспектам). В~их числе:
   \begin{itemize}
\item функциональная (function-oriented) структура, отвечающая разделению 
системных единиц по выполняемым ими функциям в~составе сис\-темы;
\item продуктовая (product-oriented), или модульная, структура, отражающая 
сборочную (технологическую) конфигурацию сис\-темы;
\item структура размещения (location-oriented), в~соответствии с~которой 
единицы располагаются в~физическом пространстве.
\end{itemize}

   Ясно, что один и~тот же объект может входить в~несколько структур и~при 
этом находиться на различных уровнях. В~то же время в~некоторых аспектах 
объект может никак не проявлять себя и~вследствие этого отсутствовать 
в~соответствующих структурах. Полное идентифицирующее ссылочное 
обозначение объекта (reference designation) конструируется путем 
последовательного перечисления всех объектов, находящихся на пути от корня 
дерева рассматриваемой структуры до дан\-ного объекта включительно. 
Наименование каж\-до\-го объекта в~этом перечислении составляется из 
символьного обозначения аспекта, буквенного обозначения класса (типа), 
к~которому относится  объект, и~порядкового номера объекта среди 
экземпляров своего класса. Таким путем обеспечивается\linebreak  уникальность 
наименования любой единицы\linebreak
 в~пределах системы. Например, функциональная 
структура обозначается символом <<=>>, а~функциональный класс 
переключателей потоков ресурсов обозначается буквами QA, так что первая по 
порядку единица, выполняющая функцию переключения, называется =QA1, 
а~ее полное ссылочное обозначение может выглядеть как =WP1=WC1=QA1. 
Если объект присутствует в~нескольких структурах, то он может иметь 
несколько ссылочных обозначений, как показано на рис.~1~\cite{4-kov}.

\begin{figure*} %fig1
    \vspace*{1pt}
\begin{center}
\mbox{%
\epsfxsize=165mm
\epsfbox{kov-1.eps}
}
\end{center}
\vspace*{-9pt}
\Caption{Пример ссылочных обозначений структурных единиц системы}
\vspace*{9pt}
\end{figure*}

   С~точки зрения практики системной инженерии большой интерес 
представляет описание эволюции структурного представления системы по ходу 
жизненного цикла, приведенное в~приложении~B к~стандарту IEC~81346. 
<<Строительный материал>> для структур имеет вид (виртуального) 
справочника или каталога объектов, из которого выбираются объекты для 
включения в~структуру. 

В~начале жизненного цикла системы на основе 
исходных требований к~ней конструктор строит ее функциональную структуру. 
Затем определяется пространственное положение функциональных объектов, 
в~результате чего создается структура размещения. На следующей стадии 
формируются закупочные спецификации, образующие продуктовую структуру. 
В~ходе последующих стадий жизненного цикла эти структуры могут 
трансформироваться. На каждой стадии могут происходить замена, слияние 
и~расщепление объектов. Таким образом, объекты разных структур системы 
связаны отношением вида <<многие ко многим>>, вдоль которого 
прослеживаются (трассируются) исходные требования.
   
   В то же время стандарт не предусматривает указа\-ние способов, какими 
объекты собраны в~сис\-те\-мы. Поэтому структуру сис\-те\-мы можно рас\-смат\-ри\-вать 
как эскизный проект, в~котором отражены лишь факты вхождения системных 
единиц более низкого уровня иерархии в~единицы более высокого уровня. 


Проект такого рода поступает на вход технологу, который определяет 
конкретные операции сборки каждой единицы каждого уровня иерархии. При 
необходимости технолог вносит изменения в~конструкцию объектов (такие как 
нарезка резьбы) и~добавляет связующие интерфейсные объекты (такие как 
клей, трансформатор и~др.). В~результате для каждого составного объекта 
формируется сборочный чертеж, на котором указаны все со\-став\-ля\-ющие 
объекты и~действия по их соединению в~целях получения сис\-те\-мы. 
Технологическая проработка требуется на всех стадиях жизненного цикла, на 
которых формируется либо изменяется ка\-кая-ли\-бо из структур системы.

%\vspace*{-6pt}

\section{Мегамоделирование и~сборка~систем}

   В MBSE объекты, образующие 
структуры\linebreak
 сис\-тем, описываются формализованными ком\-пьютерными моделями 
различных видов: геометрическими фигурами и~телами, численными 
аппроксимациями дифференциальных уравнений, оснащенными графами и~
т.\,д. При этом, как свидетель\-ст\-ву\-ют стандарты типа IEC~81346, для анализа 
структуры систем и~организации сборки необходимо знать не столько 
внутреннюю структуру моделей, сколько ассортимент их возможностей 
соединяться с~другими моделями в~целях формирования моделей составных 
объектов. Иными словами, модели рассматриваются как <<черные ящики>> 
с~известным поведением по отношению к~другим моделям. Каталог объектов, 
упоминавшийся в~предыду\-щем разделе, в~условиях применения \mbox{MBSE} 
составляется из моделей и~описаний действий по их соединению.
   
   Структуры систем и~сборочные чертежи представляют собой частные 
случаи мегамоделей (mega\-mod\-el)~--- моделей, состоящих из моделей и~связей 
между ними~\cite{8-kov}. Мегамодель, в~которой связи описывают соединение 
моделей, образующих некоторую сис\-те\-му, называется конфигурацией этой 
сис\-те\-мы~\cite{5-kov}. Существуют и~другие виды мегамоделей, 
предназначенные для описания других процедур \mbox{MBSE}, таких как 
формирование модели согласно заданной метамодели  
(instantiating)~\cite{9-kov}. Но в~настоящей работе сосредоточимся на 
конфигурациях и~сборке систем.
   
   Например, в~моделировании механических сис\-тем, состоящих из твердых 
тел, моделями деталей и~сборочных единиц служат геометрические тела, 
которые могут быть представлены для компьютерной обработки различными 
способами: конструктивным, воксельным, граничным~\cite{10-kov}. Объекты, 
составляющие механические системы, т.\,е.\ представления экземпляров тел, 
получаются из моделей путем аффинных изометрий и~растяжений. Так, из 
набора цилиндров разных размеров составляется модель штанги (спортивного 
снаряда). В~функциональной структуре штанги по IEC~81346 цилиндры 
представлены разными объектами, поскольку они выполняют разные функции, 
хотя порождаются одной и~той же геометрической моделью. Соответственно, 
в~каталоге моделей содержится тело в~форме цилиндра, допускающее 
несколько разных действий по включению в~состав штанги.
   
   В качестве еще одного примера рассмотрим дис\-крет\-но-со\-бы\-тий\-ное 
имитационное моделирование, поддержка которого относится к~числу 
важнейших достижений MBSE~\cite{1-kov}. Здесь модель имеет вид 
сценария~--- фрагмента предполагаемой истории поведения моделируемой 
системы, пред\-став\-лен\-но\-го потоком дискретных событий различных видов. 
Некоторые события могут вызывать либо запрещать возникновение других 
событий. Описания действий по сборке сценариев поведения систем отражают 
вклад сценариев поведения составляющих. Так, сценарий работы цеха 
составляется из сценариев работы станков, связанных друг с~другом согласно 
маршрутным картам~\cite{11-kov}.
   
   Сформулируем задачу мегамоделирования сборки систем в~общем виде 
следующим образом. По мегамодели, представляющей конфигура\-цию 
некоторой системы, требуется сконструировать модель системы как целого 
и~рассчитать для нее моделируемые параметры, в~том числе эмерджентные~--- 
не присущие никакой из со\-став\-ля\-ющих единиц в~отдельности. Принцип 
конструирования модели системы легко усмотреть из организации 
структур-\linebreak\vspace*{-12pt}

\columnbreak

 { \begin{center}  %fig1
 \vspace*{1pt}
\mbox{%
\epsfxsize=57.246mm
\epsfbox{kov-2.eps}
}


\vspace*{12pt}


\noindent
{{\figurename~2}\ \ \small{Схема склеивания}}
\end{center}
}

\vspace*{18pt}

\addtocounter{figure}{1}

\noindent
ного представления: система должна находиться на иерархическом 
уровне, располагающемся непосредственно над уровнем со\-став\-ля\-ющих ее 
объектов. Иными словами, модель системы должна включать в~себя модели 
всех составляющих с~учетом их конфигурационных связей и~в~то же время 
включаться в~любые модели, включающие в~себя модели всех составляющих 
конфигурации.
   
   Поясним этот принцип на простом примере. Предположим, что нужно 
объединить в~систему два объекта~$P$ и~$S$ и~что технолог решил сделать это 
с~по\-мощью клея~--- третьего объекта~$G$, который может быть соединен 
и~с~$P$, и~с~$S$. Действие клея описывается конфигурацией следующего 
вида: объекты~$G$ и~$P$ порождают в~результате соединения известный 
промежуточный комплексный объект~$P_G$, содержащий их, а~объекты~$G$ 
и~$S$ порождают объект~$S_G$. Система~$R$, полученная путем 
склеивания~$P$ с~$S$ при помощи~$G$, отбирается среди объектов, 
содержащих~$P_G$ и~$S_G$, по следующему структурному критерию: 
объект~$R$ должен содержаться в~любом объекте~$T$, содержащем~$P_G$ 
и~$S_G$. Схематически этот критерий изображен на рис.~2.


   Если объект $R$, удовлетворяющий указанному структурному критерию, 
существует, то он действительно отвечает системе, которая собрана из~$S$ 
и~$P$ путем склеивания посредством~$G$ (и~не содержит ничего 
<<лишнего>>). Более того, легко видеть, что такой объект~$R$ определяется, 
по существу, однозначно в~том смысле, что любые два объекта~$R$ 
и~$R^\prime$, удовлетворяющие структурному критерию, содержатся друг 
в~друге. Если же нужного объекта~$R$ не существует, то делается вывод, что 
технолог ошибся: клей~$G$ не способен соединить объекты~$P$ и~$S$.
   
   В структурное представление, выполненное по стандарту IEC~81346 либо по 
ГОСТу 2.053-2013, входят только объекты~$P$, $S$ и~$R$ и~две композитные 
стрелки: $P\hm\to R$, проходящая через~$P_G$, и~$S\hm\to R$, проходящая 
через~$S_G$ (так что мегамодель склеивания~--- это часть схемы, ограниченная 
треугольником~$PSR$). Кроме того, стрелки на схеме склеивания, в~отличие от 
структуры, представляют не просто факты включения объектов друг в~друга, 
а~конкретные действия по их соединению. При этом соблюдается следующее 
естественное условие структурной корректности: если из одного объекта 
можно прийти в~другой разными путями по схеме, то эти пути задают одно и~то 
же композитное действие. Например, клей~$G$ включается в~состав 
системы~$R$ единственным способом, несмотря на наличие двух путей $G 
\hm\to  P_G \hm\to R$ и~$G \hm\to S_G \hm\to R$: в~действительности не имеет 
значения, через какой промежуточный объект <<прослеживается>> включение 
клея в~систему. Таким образом, мегамодель сборки содержит больше 
информации, чем иерархическая структура системы.
   
   Если модели содержат значения тех или иных параметров, а описание 
действий по их соединению позволяет выявить правила преобразования 
значений, то по мегамодели сборки можно вы\-чис\-лить значения параметров для 
системы. Известны примеры вычислений такого рода в~области разработки 
новых композиционных материалов~\cite{12-kov}. Осредненные 
(эффективные) физические характеристики композитов, такие как модуль Юнга и~коэффициент Пуассона, сложным образом зависят от характеристик 
компонентов и~способов изготовления композита из них. При помощи методов 
теории упру\-гости эти зависимости задаются в~форме линеаризованных 
матричных соотношений, которые приписываются к~стрелкам мегамоделей, 
пред\-став\-ля\-ющим включение компонентов в~композиты. Появляется 
возможность рассчитывать на компьютере свойства композитов по базе данных 
компонентов, без проведения дорогостоящих физических экспериментов.
   
   В заключение раздела отметим, что хотя прямой расчет системы по 
конфигурации имеет большое значение, в~MBSE он играет вспомогательную 
роль. Согласно стандарту IEC~81346 и~практикам системной инженерии, 
система обычно проектируется сверху вниз~--- от корня структурной иерархии 
к~составляющим~\cite{13-kov}. Это означает, что технолог в~основном решает 
не прямую, а~обратную задачу: модель системы, которую нужно собрать, 
известна, а~нужно построить (восстановить) конфигурацию, из которой такая 
система может быть получена путем сборки, с~учетом различных ограничений. 
Формальные математические постановки и~методы решения обратных задач 
мегамоделирования представляют собой крупную перспективную тему 
исследований, выходящую за рамки настоящей статьи.

\section{Теория категорий в~мегамоделировании}

   Как указывалось во введении, естественным источни\-ком математических 
методов кон\-стру\-ирова\-ния и~анализа мегамоделей служит теория категорий. 
Категорией называется коллекция абстрактных объектов, попарно связанных 
морфизмами (стрелками). Точное определение занимает буквально несколько 
строк~\cite{14-kov}: категория~$C$ состоит из совокупности 
объектов~$\mathrm{Ob}\,C$ и~совокупности морфизмов~$\mathrm{Mor}\,C$, 
на которых заданы следующие операции:
\begin{enumerate}[(1)]
\item каждому морфизму~$f$ 
сопоставляется два объекта: область $\mathrm{dom}\,f$ и~кообласть 
$\mathrm{codom}\,f$ (соотношения вида $\mathrm{dom}\,f \hm= A$ и~
$\mathrm{codom}\,f \hm= B$ наглядно записываются в~форме стрелки~$f$: 
$A\hm\to B$, а множество всех морфизмов, удовлетворяющих этим 
соотношениям, обозначается через $\mathrm{Mor}(A, B))$;
\item для 
любой пары морфизмов~$f, g$, удовлетворяющей условию 
$\mathrm{codom}\,f\hm = \mathrm{dom}\,g$, определена композиция~--- 
морфизм $g \circ f : \mathrm{dom}\,f \hm\to  \mathrm{codom}\,g$, причем она 
ассоциативна: для любой тройки морфизмов~$f, g, h$, удовлетворяющей 
условиям $\mathrm{codom}\,f \hm= \mathrm{dom}\,g$ и~$\mathrm{codom}\,g 
\hm= \mathrm{dom}\,h$, выполняется соотношение $h \circ (g \circ f) \hm= (h 
\circ g) \circ f$;
\item любой объект~$A$ обладает тождественным 
морфизмом~$1_A : A \to A$ таким, что для любого морфизма~$f : A\hm\to B$ 
выполняется соотношение $f \circ 1_A \hm= 1_B \circ  f \hm= f$.
\end{enumerate}

Классическим 
примером категории служит $\mathbf{Set}$, состоящая из всех множеств и~всех 
их отображений: закон композиции отображений задается стандартной 
подстановкой, а тождественным морфизмом произвольного множества служит 
его тождественное отображение на себя.
   
   Вместе с~категорией вводится понятие функтора~--- отображения категорий, 
сохраняющего структуру. Функтор $\mathrm{fun}\,: C \hm\to D$, действующий из 
категории~$C$ в~$D$,~--- это пара одноименных отображений $\mathrm{fun}\,: 
\mathrm{Ob}\,C \hm\to \mathrm{Ob}\,D$, $\mathrm{fun}\,: \mathrm{Mor}\,C \hm\to 
\mathrm{Mor}\,D$, удовлетворяющая следующим условиям (для произвольных 
$C$-мор\-физ\-мов~$f, g$ и~$C$-объ\-ек\-та~$A$): 
\begin{enumerate}[(1)]
\item $\mathrm{fun}\,(\mathrm{dom}\,f) 
\hm= \mathrm{dom}\,\mathrm{fun}\,(f), \mathrm{fun}\,(\mathrm{codom}\,f)\hm = 
\mathrm{codom}\,\mathrm{fun}\,(f)$;  
\item $\mathrm{fun}\,(g \circ f) = \mathrm{fun}\,(g) \circ \mathrm{fun}\,(f)$, 
если композиция $g \circ f$ определена; 
\item $\mathrm{fun}\,(1_A) \hm= 1_{\mathrm{fun}\,(A)}$.
\end{enumerate}
 Все категории и~все функторы образуют 
(формальную) категорию~$\mathbf{CAT}$. Чтобы исследовать взаимосвязь 
между функторами, вводится следующее понятие: естественным 
преобразованием~$\varepsilon$ функтора $\mathrm{fun}\, : C\hm\to D$ в~$\mathrm{fun}^\prime\, : C 
\hm\to D$ называется любое семейство $D$-мор\-физ\-мов~$\varepsilon_A : 
\mathrm{fun}\,(A) \hm\to \mathrm{fun}^\prime (A)$, $A \hm\in \mathrm{Ob}\,C$, 
такое что для любого 
\mbox{$C$-мор}\-физ\-ма $f : A\hm\to B$ выполняется соотношение $\varepsilon_B \circ 
\mathrm{fun}\,(f) \hm= \mathrm{fun}^\prime(f) \circ \varepsilon_A$:

%\begin{figure*} %рис
\vspace*{1pt}
\begin{center}
\mbox{%
\epsfxsize=54.473mm
\epsfbox{kov-3.eps}
}
\end{center}
%\vspace*{-9pt}
%\end{figure*}

   Эффективность применения теории категорий в~качестве математического 
аппарата \mbox{MBSE} обуслов\-ле\-на тем, что любой каталог моделей представляет 
собой не что иное, как категорию. Действительно, любая цепочка действий по 
соединению моделей порождает композитное действие (процесс) и, кроме того, 
любая модель допускает пустое действие над самой собою, не 
подразумевающее никаких изменений (процедура <<ничегонеделания>>). 
Например, в~твердотельном моделировании механических систем объектами 
категории\linebreak моделей выступают тела~--- подмножества в~$\mathbb{R}^3$, 
которые являются ограниченными, регулярными\linebreak
 (совпадают с~замыканием 
своей внутренности) и~полуаналитическими (допускают представление 
конечными булевыми комбинациями множеств вида $\{(x, y, z) \vert  F_i(x, y, 
z)\hm\leq 0\}$, где~$F_i : \mathbb{R}^3\hm\to \mathbb{R}$ является 
вещественной аналитической функцией для всех~$i$)~\cite{10-kov}. Чтобы 
было возможно задавать процедуры типа склеивания участков поверхности тел, в~категорию геометрических моделей добавляются ограниченные регулярные 
полуаналитические подмножества в~$\mathbb{R}^n$, $0 \hm\leq n \hm\leq 2$, 
при помощи стандартного вложения~$\mathbb{R}^n$ в~$\mathbb{R}^3$. Далее 
выполняется факторизация: отождествляются друг с~другом все множества, 
переходящие друг в~друга под действием аффинных изометрий. Морфизмы 
таких классов эквивалентности, описывающие действия по сборке составных 
механических сис\-тем, порождаются изометрическими вложениями множеств 
и~растяжениями. Получается подкатегория в~\textbf{Set}, которую будем обозначать 
через $\mathbf{MBS}$ (от Multibody Systems).
   
   Для многих известных технологий MBSE формальное описание каталогов 
поддерживаемых моделей приводит к~категориям множеств со структурой~--- 
алгебраических систем, топологических пространств, графов и~т.\,д. 
Морфизмами в~таких категориях служат отображения множеств, со\-вмес\-ти\-мые 
со структурой. На любой такой категории действует канонический функтор 
в~$\mathbf{Set}$, <<забывающий>> структуру. 

В~качестве примера приведем  
дис\-крет\-но-со\-бы\-тий\-ное моделирование, в~котором математической 
моделью сценария служит множество событий, час-\linebreak тич\-но упорядоченное  
при\-чин\-но-след\-ст\-вен\-ны\-ми зависимостями и~размеченное видами 
событий~\cite{15-kov}. Действия по сборке сложных сценариев задаются 
монотонными отображениями, сохраняющими разметку, поскольку ни 
события, ни зависимости, ни метки не могут быть <<потеряны>> при 
соединении сценариев поведения компонентов в~сценарии поведения систем. 
Получается категория~$\mathbf{Pomset}$, состоящая из всех помеченных 
частично упорядоченных множеств и~всех их монотонных отображений, 
сохраняющих разметку. Имеется функтор $\vert \mbox{--} \vert : 
\mathbf{Pomset}\hm\to \mathbf{Set} : S \mapsto \vert S\vert$, <<забывающий>> 
порядок и~разметку.
   
   Зафиксируем произвольную категорию~$C$, представляющую некоторый 
каталог моделей. Как и~для любой алгебраической системы, определена 
конструкция подкатегории в~$C$~--- это пара, состоящая из подкласса 
в~$\mathrm{Ob}\,C$ и~подкласса в~$\mathrm{Mor}\,C$, замкнутых 
относительно унаследованных из~$C$ операций. Подкатегория в~$C$ 
называется полной, если любой \mbox{$C$-мор}\-физм, область и~кообласть которого 
содержатся в~ней, сам содержится в~ней. Например, подкатегориями 
описываются различные аспекты структурного представления систем согласно 
стандарту IEC~81346. Действительно, композиция двух морфизмов, 
представляющих действия по формированию некоторого аспекта структуры, 
также должна входить в~этот аспект, поскольку стандарт предписывает строить 
цепочки для идентификации объектов в~структуре системы. Кроме того, если 
объект присутствует в~аспекте, то его тождественный морфизм формально 
должен быть включен в~этот аспект. В~то же время подкатегории, 
опи\-сы\-ва\-ющие все аспекты, не обязаны образовывать в~совокупности разбиение 
категории~$C$: как показывает рис.~1, возможны как действия, входящие 
в~несколько аспектов одновременно, так и~композитные действия с~переходом 
между структурами, не входящие ни в~один аспект. Требуется лишь, чтобы 
объединение классов объектов всех этих подкатегорий совпадало 
с~$\mathrm{Ob}\,C$, поскольку не имеет смысла вводить модели, не входящие 
ни в~одну структуру.
   
   Категории можно получать из графов: любой ориентированный мультиграф 
порождает категорию, объектами в~которой служат все узлы, а морфизмами~--- 
все пути. Областью и~кообластью морфизма являются соответственно начало 
и~конец пути, композиция морфизмов действует как конкатенация путей, 
а~тождественным морфизмом узла~$a$ является пустой путь из~$a$ в~$a$, не 
содержащий ни одного ребра. Отсюда получается фундаментальное понятие  
$C$-диа\-грам\-мы~--- это функтор вида~$\Delta : X \hm\to C$, где~$X$~--- 
категория, порожденная некоторым графом и~называемая схемой диаграммы. 
Все $C$-диа\-грам\-мы образуют категорию~$\mathbf{D}C$ (ковариантная 
категория <<сверхзапятой>>~\cite{14-kov}), в~которой морфизмом 
диаграммы~$\Delta : X \hm\to C$ в~$\Xi : Y \hm\to C$ служит любая пара 
вида $\langle\gamma, fd\rangle$, состоящая из функтора~$fd : X\hm\to Y$ 
и~естественного преобразования~$\gamma : \Delta\hm\to \Xi \circ fd$; закон 
композиции морфизмов диаграмм имеет вид:
$$
\langle \gamma, fd\rangle \circ 
\langle \varphi, gd\rangle \hm = \langle \gamma_{gd(-)} \circ \varphi, fd \circ 
gd\rangle\,.
$$ 
В~тео\-рии категорий накоплен богатый арсенал алгебраических 
методов конструирования и~анализа диаграмм.
   
   Любая мегамодель задается $C$-диа\-грам\-мой, так что категорное 
представление каталогов моделей позволяет формально решать задачи 
мегамоделирования. Морфизмы диаграмм описывают структурные 
преобразования мегамоделей, выполняемые при помощи инструментов MBSE. 
Покажем, как решаются средствами теории категорий прямые задачи 
мегамоделирования. Здесь применяется одна из основных  
тео\-ре\-ти\-ко-ка\-те\-гор\-ных конструкций~--- копредел  
диаграммы~\cite{5-kov}, который строится следующим образом. Обозначим 
через~$\mathbf{1}$ категорию,\linebreak состоящую из одного объекта~0 и~одного 
морфизма~$1_0$. Из любой категории~$X$ имеется в~точ\-ности один 
функтор~$!_X : X \hm\to \mathbf{1}$, сопоставляющий объект~0  
любому~$X$-объ\-ек\-ту (иными словами, $\mathbf{1}$ является терминальным 
$\mathbf{CAT}$-объ\-ек\-том). Имеется вложение (инъективный функтор) 
$\ulcorner \mbox{--}\urcorner : C \hookrightarrow \mathbf{D}C$, сопоставляющее 
произвольному $C$-объ\-ек\-ту $Q$~точку~--- диаграмму $\ulcorner Q\urcorner : 
\mathbf{1}\hm\to  C : 0 \mapsto Q$. Коконусом (cocone) называется 
$\mathbf{D}C$-мор\-физм, имеющий точку в~качестве кообласти. Можно 
изобразить коконус $\langle \sigma, !_X\rangle : \Delta\hm\to \ulcorner 
Q\urcorner$ над диаграммой $\Delta : X\hm\to C$ в~виде диаграммы, 
<<пририсовав>> к~$\Delta$ дополнительную вершину, помеченную 
объектом~$Q$, и~набор ребер~--- стрелок, по одной для каждого узла $I\hm\in 
\mathrm{Ob}\,X$, направленной из~$I$ в~вершину и~помеченной морфизмом 
$\sigma_I : \Delta (I) \hm\to Q$. Копределом (colimit) диаграммы~$\Delta$ 
называется коконус $\mathrm{colim}\,\Delta : \Delta\hm\to \ulcorner R\urcorner$, 
универсальный в~том смысле, что для любых \mbox{$C$-объ}\-ек\-та~$T$ 
и~коконуса~$\delta : \Delta\hm\to\ulcorner T\urcorner$ существует единственный 
$C$-мор\-физм~$w : R \hm\to T$ такой, что $\delta\hm= \ulcorner w\urcorner \circ  
\mathrm{colim}\,\Delta$. Легко видеть, что это условие универсальности 
представляет собой в~точности структурный критерий из разд.~3. Таким 
образом, конструирование копредела конфигурации~$\Delta$ описывает на 
строгом математическом языке сборку системы, которой отвечает 
вершина~$R$. В~категориях типа $\mathbf{MBS}$ и~$\mathbf{Pomset}$ 
построение копредела сводится к~факторизации раздельных объединений 
объектов, представляющих компоненты системы, по отношениям 
эквивалентности, индуцированным моделями клея и~других средств сборки.
   
   Копредел любой диаграммы, если он сущест\-вует, определяется однозначно 
   с~точностью до изомор\-физма. Более того, можно описать сборку сис\-тем из 
конфигураций в~виде функтора. Пусть $Cd$~--- некоторый класс  
$C$-диа\-грамм, имеющих копределы. Он порождает полную подкатегорию 
в~$\mathbf{D}C$, из которой в~$C$ действует функтор копредела $\mathrm{colim}$, 
сопоставляя каждой диаграмме из~$Cd$~вершину некоторого ее копредела, а 
каждому \mbox{$\mathbf{D}C$-мор}\-физ\-му~$\theta : \Delta\hm\to \Xi$, 
где~$\Delta, \Xi\hm\in Cd$~--- стрелку копредела $\mathrm{colim}\,(\theta)$ такую, что 
$\mathrm{colim}\,\Xi \circ \theta \hm= \ulcorner \mathrm{colim}\,(\theta)\urcorner \circ 
\mathrm{colim}\,\Delta$.

%\begin{figure*}
\vspace*{1pt}
\begin{center}
\mbox{%
\epsfxsize=56.127mm
\epsfbox{kov-4.eps}
}
\end{center}
%\vspace*{-9pt}
%\end{figure*}

   Например, в~категории \textbf{Set} любая диаграмма имеет 
копредел~\cite[упражнение~5.1.8]{14-kov}, поэтому имеется функтор $\mathrm{colim}\, : 
\mathbf{D}(\mathbf{Set})\hm\to \mathbf{Set}$. Примечательно, что этот функтор 
является рефлектором: он сопряжен слева с~вложением $\ulcorner \mbox{--}\urcorner : 
\mathbf{Set}\hookrightarrow \mathbf{D}(\mathbf{Set})$, причем 
единица рефлексии состоит из $\mathbf{D}(\mathbf{Set})$-мор\-физ\-мов 
$\mathrm{colim}\,\Delta : \Delta\hm\to \ulcorner\mathrm{colim}\,(\Delta)\urcorner$, 
$\Delta\hm\in \mathrm{Ob}\ \mathbf{D}(\mathbf{Set})$. Напомним, что единица 
рефлексии~--- это естественное преобразование тождественного функтора 
в~композицию рефлектора и~вложения (в~данном случае, естественное 
преобразование функтора $1_{\mathbf{D}(\mathbf{Set})}$ в~$\ulcorner \mathrm{colim}\,(  
\mbox{--})\urcorner)$, состоящее из универсальных  
стрелок~\cite[разд.~4.3]{14-kov}. И~для произвольного класса~$Cd$, 
содержащего достаточное количество одноточечных диаграмм, функтор 
$\mathrm{colim}$ сопряжен слева с~ограничением  
вложения~$\ulcorner \mbox{--}\urcorner$ на подходящую полную подкатегорию 
в~$C$. А~поскольку сопряженный функтор задается однозначно с~точностью 
до изоморфизма~\cite[разд.~4.1]{14-kov}, можно сделать вывод, что сборка 
систем в~некотором смысле <<зашифрована>> в~процедуре построения 
одноточечных диаграмм~--- моделей систем как целого без раскрытия 
струк\-туры. 

Так наглядно проявляется двойственность прямых и~обратных задач 
мегамоделирования.

\section{Заключение}

   Аппарат теории категорий обладает большим потенциалом в~области 
повышения полезной отдачи от MBSE, в~том числе путем математически 
строгого решения задач мегамоделирования. Так, базовая процедура системной 
инженерии~--- сборка\linebreak
 системы из заданной конфигурации взаимо\-свя\-занных 
компонентов~--- формально описывается тео\-ретико-ка\-те\-гор\-ной 
конструкцией копредела диа\-граммы. Более сложные конструкции отвечают\linebreak 
сложным процедурам сборки, таким как связывание (weaving) общесистемных 
функций, рассеянных по всем компонентам (crosscutting concerns), например 
мониторинговых или защитных~\cite{16-kov}. Математического представления 
требуют и~другие процедуры MBSE, в~частности коллективная модификация 
мегамоделей и~составляющих моделей, восстановление конфигурации заданной 
системы, оценка взаимозаменяемости компонентов. 

Актуальны вопросы 
внедрения аппарата теории категорий в~практику, в~том числе путем развития 
программных инструментов моделирования и~мегамоделирования. Здесь 
открывается широкий спектр направлений для дальнейших исследований.
   
{\small\frenchspacing
 {%\baselineskip=10.8pt
 \addcontentsline{toc}{section}{References}
 \begin{thebibliography}{99}
\bibitem{1-kov}
Modeling and simulation-based systems engineering handbook~/
Eds.\ D.~Gianni,  A.~D'Ambrogio, A.~Tolk.~--- London: CRC Press, 2014. 513~p.
\bibitem{2-kov}
\Au{Ковалёв С.\,П., Толок~А.\,В.} Применение модельно-ори\-ен\-ти\-ро\-ван\-но\-го подхода 
в~управ\-ле\-нии жизненным циклом технических изделий~// Информационные технологии 
в~проектировании и~производстве, 2015. №\,2. С.~3--9.
\bibitem{3-kov}
\Au{Левенчук А.\,И.} Системноинженерное мышление.~--- М.: TechInvestLab, 2015. 305~с.
\bibitem{4-kov}
IEC 81346-1:2009. Industrial Systems, Installations and Equipment and Industrial Products~--- 
Structuring Principles and Reference Designations~--- Part~1: Basic Rules.~--- Geneva: ISO, 2009. 
168~p.
\bibitem{5-kov}
\Au{Ginali S., Goguen~J.} A~categorical approach to general systems~// 
 Conference (International) on Applied General Systems 
Research Proceedings~/
Ed. G.\,J.~Klir.~--- NATO conference series.~--- New York, NY, USA: Plenum 
Press, 1978. Vol.~5. P.~257--270.
\bibitem{6-kov}
\Au{Mabrok M.\,A., Ryan M.\,J.} Category theory as a~formal mathematical foundation for  
model-based systems engineering~// Appl. Math. Inform. Sci., 2017. Vol.~11. No.\,1. P.~43--51.
\bibitem{7-kov}
\Au{Ковалёв С.\,П.} Тео\-ре\-ти\-ко-ка\-те\-гор\-ный подход к~проектированию программных 
сис\-тем~// Фундаментальная и~прикладная математика, 2014. Т.~19. Вып.~3. С.~111--170.
\bibitem{8-kov}
\Au{B$\acute{\mbox{e}}$zivin J., Jouault~F., Rosenthal~P., Valduriez~P.} Modeling in the large 
and modeling in the small~// Model Driven Architecture: European MDA Workshops on 
Foundations and Applications Proceedings~/
Eds.\ U.~A{\!\ptb{\ss}}mann, M.~Aksit,  A.~Rensink.~--- 
Lecture notes in computer science ser.~--- Springer, 2005. Vol.~3599. 
P.~33--46.
\bibitem{9-kov}
\Au{Diskin Z., Kokaly~S., Maibaum~T.} Mapping-aware mega\-mod\-eling: Design patterns and 
laws~// Software Language Engineering: 6th Conference (International) Proceedings~/
Eds.\ M.~Erwig, R.\,F.~Paige, E.~Van Wyk.~--- 
Lecture notes  in computer science ser.~--- Springer, 2013. Vol.~8225. P.~322--343.
\bibitem{10-kov}
\Au{Requicha A.\,G.} Representations for rigid solids: Theory, methods, and systems~// 
ACM  Comput. Surv., 1980. Vol.~12. Iss.~4. P.~437--464.
\bibitem{11-kov}
\Au{K$\acute{\mbox{a}}$d$\acute{\mbox{a}}$r B., Pfeiffer~A., Monostori~L.} Discrete event 
simulation for supporting production planning and scheduling decisions in digital
 factories~//  37th 
CIRP Seminar (International) on Manufacturing Systems Proceedings.~--- Budapest, 2004.  
P.~444--448.
\bibitem{12-kov}
\Au{Giesa T., Spivak D.\,I., Buehler~M.\,J.} Category theory based solution for the building block 
replacement problem in materials design~// Adv. Eng. Mater., 2012. Vol.~14. 
Iss.~9. P.~810--817.
\bibitem{13-kov}
\Au{Косяков А., Свит У., Сеймур~С., Бимер~С.} Системная инженерия. Принципы 
и~практика~/ Пер. с~англ.~--- М.: ДМК-Пресс, 2014. 636~с. (\Au{Kossiakoff~A., Sweet~W.\,N., 
Seymour~S., Biemer~S.\,M.} Systems engineering principles and practice.~--- 2nd ed.~--- New 
York, NY, USA: John Wiley, 2011. 560~p.)
\bibitem{14-kov}
\Au{Маклейн С.} Категории для работающего математика~/ Пер. с~англ.~--- М.: Физматлит, 
2004. 352~с. (\Au{Mac Lane~S.} Categories for the working mathematician.~--- New York, NY, 
USA: Springer, 1978. 317~p.)
\bibitem{15-kov}
\Au{Pratt V.\,R.} Modeling concurrency with partial orders~// Int. J.~Parallel 
Prog., 1986. Vol.~15. No.\,1. P.~33--71.
\bibitem{16-kov}
\Au{Ковалёв С.\,П.} Семантика ас\-пект\-но-ори\-ен\-ти\-ро\-ван\-но\-го моделирования 
данных и~процессов~// Информатика и~её применения, 2013. Т.~7. Вып.~3. С.~70--80.
 \end{thebibliography}

 }
 }

\end{multicols}

\vspace*{-3pt}

\hfill{\small\textit{Поступила в~редакцию 16.01.17}}

%\vspace*{8pt}

\newpage

\vspace*{-30pt}

%\hrule

%\vspace*{2pt}

%\hrule

%\vspace*{8pt}


\def\tit{METHODS OF CATEGORY THEORY IN~MODEL-BASED SYSTEMS ENGINEERING\\[-7pt]}

\def\titkol{Methods of category theory in~model-based systems engineering}

\def\aut{S.\,P.~Kovalyov\\[-12pt]}

\def\autkol{S.\,P.~Kovalyov}

\titel{\tit}{\aut}{\autkol}{\titkol}

\vspace*{-14pt}


\noindent
Institute of Control Sciences, Russian Academy of Sciences, 65~Profsoyuznaya Str., 
Moscow 117997, Russian Federation



\def\leftfootline{\small{\textbf{\thepage}
\hfill INFORMATIKA I EE PRIMENENIYA~--- INFORMATICS AND
APPLICATIONS\ \ \ 2017\ \ \ volume~11\ \ \ issue\ 3}
}%
 \def\rightfootline{\small{INFORMATIKA I EE PRIMENENIYA~---
INFORMATICS AND APPLICATIONS\ \ \ 2017\ \ \ volume~11\ \ \ issue\ 3
\hfill \textbf{\thepage}}}

\vspace*{1pt}

 

\Abste{A mathematical device based on the category theory is proposed to formally describe and 
rigorously explore procedures of employing models in engineering that constitute the contents of 
model-based systems engineering (MBSE). The essence of the device consists in mathematical 
representation of assembly drawings (megamodels of systems) as diagrams in categories whose 
objects are models, and morphisms represent actions associated with assembling system models 
from component models. The soundness of the device is justified on the basis of standards that 
govern description of the systems' structure such as IEC~81346. Category-theoretical methods for 
solving a number of practical problems of assembling systems are proposed and explored. 
Examples of solving such problems are provided in categories that represent two key application 
areas for MBSE: geometric modeling of complex shapes and discrete-event simulation of the 
behavior of industrial systems.}

\KWE{ model-based systems engineering; megamodel; category theory; colimit}

\DOI{10.14357/19922264170305} 

%\vspace*{-18pt}

%\Ack
%\noindent




\vspace*{-7pt}

  \begin{multicols}{2}

\renewcommand{\bibname}{\protect\rmfamily References}
%\renewcommand{\bibname}{\large\protect\rm References}

{\small\frenchspacing
 {%\baselineskip=10.8pt
 \addcontentsline{toc}{section}{References}
 \begin{thebibliography}{99}
\bibitem{1-kov-1}
Gianni, D., A.~D'Ambrogio, and A.~Tolk, eds. 2014. \textit{Modeling and simulation-based 
systems engineering handbook}. London: CRC Press. 513~p.
\bibitem{2-kov-1}
\Aue{Kovalyov, S.\,P., and A.\,V.~Tolok.} 2015. Primenenie model'no-orientirovannogo podkhoda 
v~upravlenii zhiznennym tsiklom tekhnicheskikh izdeliy [Applying model-based approach 
to product lifecycle management].\linebreak \textit{Informatsionnye tekhnologii v~proektirovanii 
i~proizvod\-st\-ve} [Information Technologies in Design and Industry] 2(158):3--9.
\bibitem{3-kov-1}
\Aue{Levenchuk A.\,I.} 2015. 
\textit{Sistemnoinzhenernoe myshlenie} [Systems engineering thinking]. 
Moscow: TechInvestLab. 305~p.
\bibitem{4-kov-1}
IEC 81346-1:2009. 2009. 
Industrial Systems, Installations and Equipment and Industrial 
Products~--- Structuring Principles and Reference Designations~--- 
Part~1: Basic Rules. Geneva:  ISO. 168~p.
\bibitem{5-kov-1}
\Aue{Ginali, S., and J.~Goguen.} 1978. 
A~categorical approach to general systems. \textit{Conference 
(International) on Applied General Systems Research Proceedings}. Ed.\
 G.\,J.~Klir. \mbox{NATO}  conference ser. Plenum Press. 5:257--270.
\bibitem{6-kov-1}
\Aue{Mabrok, M.\,A., and M.\,J.~Ryan}. 
2017. Category theory as a~formal mathematical foundation for 
model-based systems engineering. \textit{Appl. Math.  Inform. Sci.} 11(1):43--51.
\bibitem{7-kov-1}
\Aue{Kovalyov, S.\,P.} 2016. 
Category-theoretic approach to software systems design. \textit{J.~Math. Sci.} 
214(6):814--853.
\bibitem{8-kov-1}
\Aue{B$\acute{\mbox{e}}$zivin, J., F.~Jouault, P.~Rosenthal, and P.~Valduriez.}
 2005. Modeling in 
the large and modeling in the small. 
\textit{Model Driven Architecture: European MDA Workshops on 
Foundations and Applications Proceedings.} 
Eds.\ U.~\mbox{A{\!\ptb{\ss}}mann}, M.~Aksit, and A.~Rensink. 
Lecture notes in computer science ser. Springer. 3599:33--46.
\bibitem{9-kov-1}
\Aue{Diskin, Z., S.~Kokaly, and T.~Maibaum.} 2013. 
Mapping-aware megamodeling: Design patterns 
and laws. \textit{6th Conference (International) on Software Language Engineering 
Proceedings}. Eds.\ M.~Erwig, R.\,F.~Paige, and E.~Van Wyk. 
Lecture notes in computer science ser. Springer. 
8225:322--343.
\bibitem{10-kov-1}
\Aue{Requicha, A.\,G.} 1980. Representations for rigid solids: 
Theory, methods, and systems. \textit{ACM 
Comput. Surv.} 12(4):437--464.
\bibitem{11-kov-1}
\Aue{K$\acute{\mbox{a}}$d$\acute{\mbox{a}}$r,~B., A.~Pfeiffer, and L.~Monostori.}
2004. Discrete 
event simulation for supporting production planning and scheduling decisions in 
digital factories. \textit{37th CIRP Seminar (International) on Manufacturing 
Systems Proceedings}. Budapest.  444--448.
\bibitem{12-kov-1}
\Aue{Giesa, T., D.\,I.~Spivak, and M.\,J.~Buehler.} 2012. 
Category theory based solution for the building 
block replacement problem in materials design. 
\textit{Adv. Eng. Mater.} 14(9):810--817.
\bibitem{13-kov-1}
\Aue{Kossiakoff, A., W.\,N.~Sweet, S.~Seymour, and S.\,M.~Bie\-mer.}
2011. \textit{Systems engineering 
principles and practice}. 2nd ed. New York, NY: John Wiley. 560~p.
\bibitem{14-kov-1}
\Aue{Mac Lane, S.} 1978. \textit{Categories for the working mathematician}. 
New York, NY: Springer. 317~p.
\bibitem{15-kov-1}
\Aue{Pratt, V.\,R.} 1986. Modeling concurrency with partial orders. 
\textit{Int. J.~Parallel Prog.} 15(1):33--71.
\bibitem{16-kov-1}
\Aue{Kovalyov, S.\,P.} 2013. 
Semantika aspektno-ori\-en\-ti\-ro\-van\-no\-go modelirovaniya dannykh 
i~protsessov [Semantics of aspect-oriented modeling of data and processes]. 
\textit{Informatika i~ee  Primeneniya~--- Inform. Appl.} 7(3):70--80.
\end{thebibliography}

 }
 }

\end{multicols}

\vspace*{-9pt}

\hfill{\small\textit{Received January 16, 2017}}

\vspace*{-18pt}

\Contrl

\noindent
\textbf{Kovalyov Sergey P.} (b.\ 1972)~--- Doctor of Science in physics and 
mathematics, leading scientist, Institute of Control Problems, Russian 
Academy of Sciences, 65~Profsoyuznaya Str., Moscow 117997, Russian 
Federation Federation; \mbox{kovalyov@nm.ru} 

\label{end\stat}


\renewcommand{\bibname}{\protect\rm Литература}  %2
\def\stat{bosov+stef}

\def\tit{УПРАВЛЕНИЕ ВЫХОДОМ СТОХАСТИЧЕСКОЙ ДИФФЕРЕНЦИАЛЬНОЙ СИСТЕМЫ 
ПО~КВАДРАТИЧНОМУ КРИТЕРИЮ. I.~ОПТИМАЛЬНОЕ РЕШЕНИЕ МЕТОДОМ 
ДИНАМИЧЕСКОГО ПРОГРАММИРОВАНИЯ$^*$}

\def\titkol{Управление выходом стохастической дифференциальной системы 
по~квадратичному критерию. I}
%.~Оптимальное решение методом 
%динамического программирования}

\def\aut{А.\,В.~Босов$^1$, А.\,И.~Стефанович$^2$}

\def\autkol{А.\,В.~Босов, А.\,И.~Стефанович}

\titel{\tit}{\aut}{\autkol}{\titkol}

\index{Босов А.\,В.}
\index{Стефанович А.\,И.}
\index{Bosov A.\,V.}
\index{Stefanovich A.\,I.}




{\renewcommand{\thefootnote}{\fnsymbol{footnote}} \footnotetext[1]
{Работа выполнена при частичной поддержке РФФИ (проект 16-07-00677).}}


\renewcommand{\thefootnote}{\arabic{footnote}}
\footnotetext[1]{Институт проблем информатики Федерального исследовательского центра <<Информатика 
и~управление>> Российской академии наук, \mbox{AVBosov@ipiran.ru}}
\footnotetext[2]{Институт проблем информатики Федерального исследовательского центра <<Информатика 
и~управление>> Российской академии наук, \mbox{AStefanovich@frccsc.ru}}

%\vspace*{8pt}



  
  \Abst{Решается задача оптимального управления для диффузионного процесса 
Ито и~линейного управ\-ля\-емо\-го выхода. Рассматриваемая постановка близка 
к~классической ли\-ней\-но-квад\-ра\-тич\-ной гауссовской задаче управления 
(linear-quadratic Gaussian (LQG) control). Отличия состоят в~том, что состояние описывается нелинейным 
дифференциальным уравнение Ито $dy_t\hm= A_t(y_t) \,dt\hm+ \Sigma_t(y_t)\,dv_t$ 
и~не зависит от управ\-ле\-ния~$u_t$, оптимизации подлежит управ\-ля\-емый 
линейный выход $dz_t\hm= a_t y_t\,dt\hm+ b_t z_t \,dt\hm+ c_t u_t \,dt\hm+ \sigma_t\, 
dw_t$. Дополнительные обобщения внесены в~квад\-ра\-тич\-ный критерий качества 
с~целью воз\-мож\-ности постановки таких задач, как отслеживание выходом 
состояния или управ\-ле\-ни\-ем~--- линейной комбинации состояния и~выхода. Для 
решения используется метод динамического программирования. Функцию 
Беллмана позволяет найти предположение о~ее структуре вида $V_t(y,z)\hm= 
\alpha_t z^2\hm+ \beta_t(y)z \hm+\gamma_t(y)$. Решение дают три 
дифференциальных уравнения для коэффициентов~$\alpha_t$, $\beta_t(y)$ 
и~$\gamma_t(y)$. Эти уравнения со\-став\-ля\-ют оптимальное решение 
рас\-смат\-ри\-ва\-емой задачи.}
  
  \KW{стохастическое дифференциальное уравнение; оптимальное управ\-ле\-ние; 
динамическое программирование; функция Беллмана; уравнение Риккати; 
линейные уравнения параболического типа}

\DOI{10.14357/19922264180314}
  
%\vspace*{4pt}


\vskip 10pt plus 9pt minus 6pt

\thispagestyle{headings}

\begin{multicols}{2}

\label{st\stat}

\section{Введение}

     Ключевые результаты в~области оптимизации стохастических 
динамических систем, со\-став\-ля\-ющие классическую теорию управления, 
получены более~40~лет назад (такова работа~[1] в~отношении задачи 
управ\-ле\-ния ли\-ней\-но-гаус\-сов\-ски\-ми стохастическими сис\-те\-ма\-ми по 
квад\-ра\-тич\-но\-му критерию). К~классической тео\-рии следует относить 
линейные модели стохастических сис\-тем и~квадратичный критерий качества. 
Это исходный базис, на котором основано множество успешно 
исследованных и~решенных задач стохастического управ\-ле\-ния 
и~оптимизации. 

Дальнейшее развитие~--- это новые модели и~критерии, но 
прежде всего это новые методы: от тео\-рии линейных регуляторов, метода 
динамического программирования и~принципа максимума к~адаптивному 
и~минимаксному подходу, импульсному управ\-ле\-нию и~т.\,д. Множество 
инноваций как в~час\-ти моделей, так и~в~час\-ти математического аппарата, 
имевших мес\-то в~по\-сле\-ду\-ющие годы, существенно обогатили тео\-рию 
управ\-ле\-ния. Но и~до настоящего времени линейные модели и~квадратичный 
критерий, несмотря на всю справедливую критику в~отношении их 
аде\-кват\-ности и~гиб\-кости, сохраняют исследовательский интерес и~находят 
современные области приложения.
     
     Не претендуя на сколь\-ко-ни\-будь полное обосно\-ва\-ние последнего 
тезиса, приведем несколько примеров, показавшихся наиболее ин\-те\-рес\-ными. 

Так, в~[2] решается ли\-ней\-но-квад\-ра\-тич\-ная за\-да\-ча в~игровой 
постановке с~запаздыванием. В~близ\-кой по модели работе~[3] задача 
управ\-ле\-ния ставится в~терминах $H_\infty$-ро\-баст\-ности. Точнее \mbox{называть} 
эту тематику $H_2/H_\infty$-управ\-ле\-ни\-ем, и~работ по этой теме очень 
много. Аккуратности ради следует уточнить, что под линейными 
понимаются модели с~мультипликативными по состоянию воз\-му\-ще\-ниями. 

Совсем другой класс моделей, особо популярных в~по\-след\-ние годы, 
составляют скачкообразные процессы. Например, линейные уравнения 
в~сочетании с~пуассоновскими скачками в~[4] используются в~моделях, 
описывающих различные показатели функционирования сетевых протоколов 
передачи данных транспортного уровня. Телекоммуникации представляют 
в~последние годы самый популярный прикладной материал для 
исследований, работ по этой проб\-ле\-ма\-ти\-ке множество, математические 
техники привлекаются самые разные и~самые современные, но и~линейным 
моделям место находится. Еще один любопытный пример исследования 
скачкообразного процесса и~оптимизации на основе квад\-ра\-тич\-но\-го критерия 
можно найти в~[5] применительно к~задаче инвестирования на финансовом 
рынке. Наконец, упомянем еще работу~[6], подводящую итог исследований 
в~отношении классической детерминированной  
ли\-ней\-но-квад\-ра\-тич\-ной задачи с~использованием техники матричных 
неравенств.
     
     В данной работе также эксплуатируются привлекательные свойства 
линейных моделей и~квад\-ра\-тич\-но\-го критерия, причем в~стохастической 
постановке. На\-прав\-ле\-ни\-ем для обобщения \mbox{выбрана} модель динамики 
сис\-те\-мы: основные усилия на\-прав\-ле\-ны на то, чтобы сделать ее нелинейной. 
Кроме того, пред\-став\-лен\-ная постановка может рас\-смат\-ри\-вать\-ся и~как 
обобщение ранее решенной задачи в~дискретном времени~[7, 8] на время 
непрерывное. В~упомянутых работах помимо собственно модельной 
постановки важна еще и~привлекаемая прикладная об\-ласть~--- 
функционирование сложных программных сис\-тем. Результатов, 
ориентированных непосредственно на такие приложения, к~настоящему 
времени пренебрежимо мало, поэтому~[7, 8]~--- это еще и~прикладное 
обоснование рас\-смат\-ри\-ва\-емой далее задачи.
     
     Оптимизируемая динамическая сис\-те\-ма описывается двумя 
уравнениями. Состояние задается нелинейным стохастическим 
дифференциальным уравнением Ито, не содержащим управ\-ля\-емой 
переменной. Возмущение здесь описывается стандартным винеровским 
процессом, накладываются простые условия существования 
и~един\-ст\-вен\-ности решения. Поскольку состояние не управ\-ля\-ет\-ся, то уместно 
его интерпретировать как слож\-ное внешнее возмущение. Вторая 
переменная~--- управ\-ля\-емый выход~--- задается линейным стохастическим 
дифференциальным уравнением. Цель оптимизации выхода формируется 
квадратичным критерием общего вида. Формальная постановка задачи 
приведена в~сле\-ду\-ющем разделе.
     
     Для решения задачи используется метод динамического 
программирования, решается уравнение Беллмана~[9]. Соответственно, 
в~результате получаются аналитические выражения и~для оптимального 
управ\-ле\-ния, и~для значения функционала качества. Технически 
традиционный, стандартный подход к~задаче обременен, пожалуй, 
единственной проблемой~--- поиском верного пред\-став\-ле\-ния структуры 
функции Беллмана. Справиться с~этой проблемой в~большей степени удается 
за счет результата, полученного при решении дискретного по времени 
аналога рассматриваемой постановки~\cite{8-bos}. Конечные соотношения 
для оптимального решения, как и~во всех подобных задачах, включая 
классическую ли\-ней\-но-квад\-ра\-тич\-ную, содержат решения 
определенных дифференциальных уравнений (обыкновенных и~в~частных 
производных). Вывод этих уравнений и~со\-став\-ля\-ет содержание первой час\-ти 
данной работы. Во второй части будет обсуждаться их приближенное 
чис\-лен\-ное решение и~компьютерные эксперименты.
     
     Кратко обозначим основные положения, при\-вле\-ка\-емые далее 
к~решению задачи, следуя в~основном обозначениям 
и~терминологии~\cite{9-bos}, а~именно: будем рассматривать задачу 
оптимального управления в~стохастической динамической сис\-те\-ме по полной 
информации, применяя метод динамического программирования. В~качестве 
целевого функционала, опре\-де\-ля\-юще\-го качество управ\-ле\-ния $U_0^T\hm= \{ 
u_t,\ 0\leq t\leq T\}$, выступает
     \begin{equation}
     J\left(U_0^T\right)={\sf E}\left\{ \int\limits_0^T L_t \left(x_t, u_t\right)\,dt+ 
l\left(x_T\right)\right\}\,.
     \label{e1-bos}
     \end{equation}
Здесь ${\sf E}\{\cdot\}$~--- оператор математического ожидания; $x_t$~--- 
случайный процесс, описываемый стохастическим дифференциальным 
уравнением Ито
     \begin{equation}
     dx_t=m_t\left( x_t, u_t\right) dt+ \sigma_t\left( x_t\right)dW_t\,,\enskip 
x_0=X\,,
     \label{e2-bos}
     \end{equation}
где $W_t$~--- стандартный винеровский процесс подходящей раз\-мер\-ности; 
$X$~--- случайный вектор.

     $U_0^T$ будем выбирать из класса допустимых неупреждающих (по 
отношению к~$W_t$) управлений~\cite{9-bos}. Соответственно, 
относительно функций сноса и~диффузии~$m_t$ и~$\sigma_t$  
в~(\ref{e2-bos}) будем предполагать выполненными ка\-кие-ли\-бо условия 
существования сильного решения для заданного до\-пус\-ти\-мо\-го управ\-ле\-ния. 
Например, для управ\-ле\-ния с~обратной связью $u_t\hm= u_t(x_t)$ будем 
считать, что $m_t(x,u_t(x))$ и~$\sigma_t(x)$ удовлетворяют условию 
линейного рос\-та и~локальному условию Липшица по~$x$ равномерно 
по~$t$ (т.\,е.\ условиям Ито).
     
     Для поиска оптимального управления, минимизирующего $J(U_0^T)$, 
рас\-смат\-ри\-ва\-ет\-ся функция Беллмана
     \begin{equation}
     V_t(x)=\left.\mathop{\mathrm{inf}}\limits_{U_t^T} {\sf E} \left\{ \int\limits_t^T 
L_t \left( x_t, u_t\right)\,dt+l\left( x_T\right) \right\vert \mathcal{F}_t^x\right\}\,,
     \label{e3-bos}
     \end{equation}
где $\mathcal{F}_t^x$~--- $\sigma$-ал\-геб\-ра, по\-рож\-ден\-ная~$x_\tau$, 
$0\hm\leq \tau\hm\leq t$, ${\sf E}\{\cdot\vert \mathcal{F}\}$~--- оператор условного 
математического ожидания относительно~$\mathcal{F}$. Соответственно, 
в~качестве достаточного условия оп\-ти\-маль\-ности воспользуемся уравнением 
динамического программирования
\begin{multline}
\fr{\partial V_t(x)}{\partial t} +\fr{1}{2}\sum\limits^n_{i,j=1} \sigma^2_{t_{ij}}
\fr{\partial^2 V_t(x)}{\partial x_i \partial x_j}+{}\\
{}+\min\limits_u\left[  
\sum\limits^n_{i=1} m_{t_i} \fr{\partial V_t(x)}{\partial x_i} + L_t(x,u)\right] 
=0\,,\\
V_T(x)=l(x)\,,
\label{e4-bos}
\end{multline}
где $m_{t_i}$~--- $i$-й элемент век\-тор-функ\-ции~$m_t(x,u)$; 
$\sigma^2_{t_{ij}} \hm= \sum\nolimits^m_{k=1} 
\sigma_{t_{ik}}\sigma_{t_{ki}}$, $\sigma_{t_{ij}}$~--- $i$-й по строке, $j$-й 
по столб\-цу элемент мат\-рич\-ной функции~$\sigma_t(x)$; $n$ и~$m$~--- 
размерности~$x_t$ и~$W_t$ соответственно.

     Традиционно в~рамках применения метода динамического 
программирования будем предполагать, что функции~$L_t$, $l$, $m_t$ 
и~$\sigma_t$ обеспечивают существование хотя бы одного решения 
уравнения~(\ref{e4-bos}), а~следовательно, и~оптимального 
управления~$u_t^*$, $0\hm\leq t\hm\leq T$, до\-став\-ля\-юще\-го минимум 
целевому функционалу~(\ref{e1-bos}). Задача оптимизации далее получается 
путем указания конкретных выражений для~$L_t$, $l$, $m_t$ и~$\sigma_t$.

\section{Постановка задачи управления выходом}

     Рассматриваемые далее случайные функции будут предполагаться 
скалярными. Такое упрощение позволит разгрузить выкладки и~итоговые 
выражения от не самых существенных деталей.
     
     Рассмотрим стохастическую дифференциальную сис\-те\-му, со\-сто\-яние 
которой представляет диффузи\-он\-ный процесс~$y_t$, описываемый 
нелинейным стохастическим дифференциальным уравнением Ито
     \begin{equation}
     dy_t=A_t\left( y_t\right) dt +\Sigma_t \left( y_t\right) dv_t\,,\enskip 
y_0=Y\,,
     \label{e5-bos}
     \end{equation}
где $v_t$~--- стандартный (одномерный) винеровский процесс; $Y$~--- 
случайная величина с~конечным вторым моментом; функции~$A_t$ 
и~$\Sigma_t$ удовлетворяют условиям Ито:
\begin{equation*}
\left\vert A_t(y)\right\vert +\left\vert \Sigma_t(y)\right\vert \leq C(1+\vert y\vert )\ 
\mbox{для\ всех } 0\leq t\leq T\,;
\end{equation*}

\vspace*{-12pt}

\noindent
\begin{multline*}
\hspace*{-2.10051pt}\left\vert A_t\left(y_1\right) -A_t \left( y_2\right) \right\vert +\left\vert 
\Sigma_t\left( y_1\right) -\Sigma_t \left(y_2\right)\right\vert \leq
C\left\vert y_1-y_2\right\vert\\
 \mbox{для\ всех\ } 0\leq t\leq T\ \mbox{и } 
y_1,y_2\in \mathbb{R}^1\,,
\end{multline*}
обеспечивающим существование единственного сильного (потраекторного) 
решения уравнения.
     
     Будем считать, что~$y_t$ описывает состояние некоторой 
динамической системы. Соответственно, поведение этой сис\-те\-мы опишем 
выходом, линейно связанным с~со\-сто\-янием:
     \begin{equation}
     dz_t=a_t y_t \,dt+ b_t z_t \,dt+ c_t u_t \,dt+\sigma_t \,dw_t\,,\enskip
     z_0=Z\,.
     \label{e6-bos}
     \end{equation}
Здесь $w_t$~--- не зависящий от~$v_t$, $Y$ и~$Z$ стандартный (одномерный) 
винеровский процесс; $Z$~--- случайная величина с~конечным вторым 
моментом; $u_t$~--- допустимое неупреждающее управ\-ле\-ние, качество 
которого определяется целевым функционалом следующего вида:
\begin{multline}
\!\hspace*{-3.98538pt}J\left( U_0^T\right) ={\sf E}\left\{ \int\limits_0^T \!\left( S_t\left( s_ty_t-g_t z_t -h_t 
u_t\right)^2 +G_t z_t^2+{}\right.\right.\\
\left.\left.{}+ H_t u_t^2
\vphantom{S_t\left( s_ty_t-g_t z_t -h_t 
u_t\right)^2}
\right) dt+S_T\left( s_T y_T -g_T 
z_T\right)^2+G_T z_T^2
\vphantom{\int\limits_0^T}\right\}\,,
\label{e7-bos}
\end{multline}
где $S_t$, $G_t$ и~$H_t$~--- неотрицательные функции\linebreak
$0\hm\leq t\hm\leq T$. 
Такой критерий отражает физический смысл задачи распределения ресурсов 
со\-глас\-но аналогичной~(\ref{e5-bos})--(\ref{e7-bos}) задаче для дис\-крет\-но\-го 
времени, рас\-смот\-рен\-ной в~\cite{7-bos}. В~част\-ности,  
функци\-онал~(\ref{e7-bos}) поз\-во\-ля\-ет ставить задачи отслеживания
 выходом 
со\-сто\-яния сис\-те\-мы, используя сла\-га\-емое $(y_t\hm- z_t)^2$, или 
управлением~--- линейной комбинации со\-сто\-яния и~выхода, сла\-га\-емое типа\linebreak 
$(y_t\hm+ z_t\hm- u_t)^2$. Поскольку задача формулируется 
в~предположении наличия пол\-ной информации о~со\-сто\-янии~$y_t$ 
и~выходе~$z_t$ (соответствующую $\sigma$-ал\-геб\-ру 
обозначим~$\mathcal{F}_t^{y,z}$), то допустимое управ\-ле\-ние ищется 
в~классе~$\mathcal{F}_t^{y,z}$-из\-ме\-ри\-мых неупреждающих функций 
(и,~как будет показано далее, оказывается управ\-ле\-ни\-ем с~обратной связью).

     Функции~$a_t$, $b_t$, $c_t$ и~$\sigma_t$ будем предполагать 
ограниченными: $\vert a_t\vert \hm+ \vert b_t\vert \hm+\vert c_t\vert \hm+ \vert 
\sigma_t \vert \hm\leq C$ для всех $0\hm\leq t\hm\leq T$, процесс  
управления~--- допустимым не\-упреж\-да\-ющим~\cite{9-bos}, обеспечивая, 
таким образом, существование сильного решения урав\-не\-ния~(\ref{e6-bos}) 
для любого допустимого управ\-ления.
     
     Задачу составляет поиск~$u_t^*$~--- допустимого управ\-ле\-ния, 
доставляющего минимум квад\-ра\-тич\-но\-му функционалу~$J(U_0^T)$.
      
     Поставленная задача очевидным образом формулируется в~терминах 
введенных выше в~(\ref{e1-bos})--(\ref{e3-bos}) обозначений, а~именно: 
     требуется обозначить
     \begin{gather*}
      x_t=\begin{pmatrix}
     y_t\\ z_t\end{pmatrix};\quad  m_t(x_t, u_t)=\begin{pmatrix}
     A_t(y_t)\\ a_t y_t +b_t z_t +c_t u_t\end{pmatrix};\\
     \sigma_t(x_t)= \begin{pmatrix}
     \Sigma_t(y_t)& 0\\
     0& \sigma_t\end{pmatrix};\quad W_t=\begin{pmatrix}
     v_t \\ w_t\end{pmatrix}
     %     \label{e8-bos}
     \end{gather*}
для записи уравнения со\-сто\-яния типа~(\ref{e2-bos}) и
\begin{align*}
L_t(x,u)&= L_t(y,z,u) ={}\\
&\hspace*{3mm}{}=S_t\left( s_t y-g_t z -h_t u\right)^2 +G_t z^2 +H_t  u^2\,;\\
l(x)&= l(y,z) =S_T \left( S_T y-g_T z\right)^2 +G_T z^2
%\label{e9-bos}
\end{align*}
для записи целевого функционала в~виде~(\ref{e1-bos}).

     Функция Беллмана~(\ref{e3-bos}) принимает вид 
     $V_t(x)\hm= V_t(y,z)$. Для записи со\-от\-вет\-ст\-ву\-юще\-го~(\ref{e4-bos}) 
уравнения Беллмана для~$V_t(y,z)$ заметим, что
     $$
     \left( \sigma^2_{t_{ij}}\right)_{i,j=1,2}= \begin{pmatrix}
     \Sigma_t^2(y) & 0\\
     0 & \sigma_t^2\end{pmatrix}\,.
     $$
     
     С~учетом перечисленных обозначений урав\-не\-ние динамического 
программирования~(\ref{e4-bos}) принимает вид:
     \begin{multline}
     \fr{\partial V_t(y,z)}{\partial t} +\fr{1}{2}\left( \Sigma_t^2(y) \fr{\partial^2 
V_t(y,z)} {\partial y^2}+\sigma_t^2\fr{\partial^2 V_t(y,z)} {\partial 
z^2}\right)+{}\\
    {}+\min\limits_u\! \left[ A_t(y) \fr{\partial V_t(y,z)}{\partial y}+\left( a_t 
y+b_t z+c_t u\right) \fr{\partial V_t(y,z)}{\partial z} +{}\right.\hspace*{-3pt}\\
\left.{}+ S_t\left( s_t y-g_t z-h_t 
u\right)^2+G_t z^2+H_t u^2
     \vphantom{\fr{\partial V_t(y,z)}{\partial y}}\right] =0\,,\\
     V_T(y,z)=S_T\left( s_T y-g_T z\right)^2+G_T z^2\,.
     \label{e10-bos}
     \end{multline}
     Это и~есть то самое уравнение, которое требуется решить: 
существование решения данного урав\-не\-ния суть достаточное условие 
оптимальности; оптимальное управ\-ле\-ние при этом~--- точ\-ка минимума 
со\-от\-вет\-ст\-ву\-юще\-го сла\-га\-емого.
     
\section{Динамическое программирование и~оптимальное 
управление}

     В рассматриваемой постановке линейность\linebreak выхода и~квадратичность 
критерия дают те же преимущества, что и~в~классической  
ли\-ней\-но-квад\-ра\-тич\-ной задаче управ\-ле\-ния~\cite{1-bos}, а~именно: 
позволяют сразу определить вид оптимального управ\-ле\-ния и~фактические 
условия его существования. Действительно, со\-хра\-няя в~(\ref{e10-bos}) под 
знаком $\min\nolimits_u$ только члены, зависящие от~$u$, получаем
     \begin{multline*}
     \fr{\partial V_t(y,z)}{\partial t} +\fr{1}{2}\left( \Sigma_t^2(y) \fr{\partial^2 
V_t(y,z)} {\partial y^2}+\sigma_t^2\fr{\partial^2 V_t(y,z)} {\partial 
z^2}\right)+{}\\
     {}+A_t(y)\fr{\partial V_t(y,z)}{\partial y}+\left( a_t y+b_t z\right) 
\fr{\partial V_t(y,z)}{\partial z}+{}\\
{}+S_t\left( s_t y-g_t z\right)^2 +G_t z^2+{}
\end{multline*}

\noindent
\begin{multline*}
     {}+\min\limits_u \left[ \left( c_t \fr{\partial V_t(y,z)}{\partial z}-2S_t \left( 
s_t y-g_t z\right) h_t\right)u +{}\right.\\
\left.{}+\left( S_t h_t^2+H_t\right) u^2
\vphantom{\fr{\partial V_t(y,z)}{\partial z}}
\right]=0\,,
     %\label{e11-bos}
     \end{multline*}
откуда в~предположении $S_t h_t^2\hm+ H_t\hm>0$ следует, что существует 
оптимальное управ\-ле\-ние, которое определяется равенством
\begin{multline}
u_t^* = u_t^*(y,z)=-\fr{1}{2}\left( S_t h_t^2 +H_t\right)^{-1} \left( c_t 
\fr{\partial V_t(y,z)}{\partial z}-{}\right.\\
\left.{}-2S_t\left( s_t y-g_t z\right) h_t
\vphantom{\fr{\partial V_t(y,z)}{\partial z}}
\right)
\label{e12-bos}
\end{multline}
и доставляет минимум соответствующему сла\-га\-емо\-му в~урав\-не\-нии Беллмана, 
равный
$-\left( S_t h_t^2\hm+\right.$\linebreak
$\left.{}+H_t\right)^{-1} \left( c_t 
{\partial V_t(y,z)}/{\partial 
z}\hm-2S_t\left( s_t y \hm-g_t z\right) h_t \right)^2/4.
$ 
     
     Отметим, что, как и~в~классической ли\-ней\-но-квад\-ра\-тич\-ной 
задаче, управ\-ле\-ние из класса до\-пус\-ти\-мых не\-упреж\-да\-ющих получилось 
управ\-ле\-ни\-ем с~обратной связью.
     
     Таким образом, функция Беллмана описывается сле\-ду\-ющим 
дифференциальным уравнением:
     \begin{multline}
     \fr{\partial V_t(y,z)}{\partial t} +\fr{1}{2}\left( \Sigma_t^2(y) \fr{\partial^2 
V_t(y,z)} {\partial y^2}+\sigma_t^2\fr{\partial^2 V_t(y,z)} {\partial 
z^2}\right)+{}\\
     {}+ A_t(y) \fr{\partial V_t(y,z)}{\partial y}+\left( a_t y+b_t z\right) 
\fr{\partial V_t(y,z)}{\partial z}+{}\\
{}+ S_t \left( s_t y- g_t z\right)^2 +G_t z^2-
 \fr{1}{4}\left( S_t h_t^2+H_t\right)^{-1}\times{}\\
 {}\times \left( c_t \fr{\partial V_t(y,z)} 
{\partial z}-2S_t\left( s_t y -g_t z\right) h_t \right)^2=0\,.
     \label{e13-bos}
     \end{multline}
     
     Возводя в~квадрат по\-след\-нее сла\-га\-емое в~(\ref{e13-bos}), перепишем 
его в~виде:
     \begin{multline}
     \fr{\partial V_t(y,z)}{\partial t} +\fr{1}{2}\left( \Sigma_t^2(y) \fr{\partial^2 
V_t(y,z)} {\partial y^2}+\sigma_t^2\fr{\partial^2 V_t(y,z)} {\partial 
z^2}\!\right)+{}\\
{}+A_t(y) \fr{\partial V_t(y,z)}{\partial y}
+ \left( 
\vphantom{\left( S_t h_t^2 +H_t\right)^{-1}}
a_t y+b_t z+{}\right.\\
\left.{}+\left( S_t h_t^2 +H_t\right)^{-1}
 c_t S_t \left( s_t y-g_t z\right) h_t
\right) 
     \fr{\partial V_t(y,z)}{\partial z}+{}\\
     {}+\left( S_t-\left( S_t h_t^2 +H_t\right)^{-1} S_t^2 h_t^2\right)\left( s_t y -
g_t z\right)^2+{}\\
     \!\!{}+
     G_t z^2 -\fr{1}{4}\left( S_t h_t^2+H_t\right)^{-1}\! c_t^2
     \left(\fr{\partial V_t(y,z)}{\partial z}\right)^{\!2}=0\,.\!\!
     \label{e14-bos}
     \end{multline}
     
     Рассматривая полученное уравнение, заметим, что его решение может 
быть пред\-став\-ле\-но в~виде:
   \begin{equation}
     V_t(y,z)= \alpha_t z^2+\beta_t(y) z +\gamma_t(y)\,,
     \label{e15-bos}
     \end{equation}
т.\,е.\ будем искать решение при дополнительном предположении 
о~квад\-ра\-тич\-ности функции Белл\-ма\-на по переменной~$z$, и~сведем, таким 
образом, поиск оптимального решения к~уравнениям относительно функций 
$\alpha_t$, $\beta_t(y)$ и~$\gamma_t(y)$. Отметим сразу, что явный вид 
функции~$\gamma_t(y)$ для реализации оптимального управ\-ле\-ния не 
требуется, однако далее будет предложен вариант вы\-чис\-ле\-ния и~этой 
функции, что пред\-став\-ля\-ет\-ся небесполезным, поскольку позволит выполнять 
расчет минимума целевого функционала. Источником для 
предложения~(\ref{e15-bos}) является уже упоминавшаяся аналогичная 
задача для случая дис\-крет\-но\-го времени~\cite{7-bos, 8-bos}. В~той задаче 
выражение для функции Беллмана получается формально без 
дополнительных усилий. При этом форма~(\ref{e15-bos}) обнаруживается 
как свойство оптимального решения. В~рассматриваемом случае 
непрерывного времени~(\ref{e15-bos}) постулируется, а~пра\-виль\-ность 
постулата под\-тверж\-да\-ет\-ся далее ре\-зуль\-ти\-ру\-ющи\-ми уравнениями 
для~$\alpha_t$, $\beta_t(y)$ и~$\gamma_t(y)$ Кроме того, данное 
предположение пред\-став\-ля\-ет\-ся вы\-те\-ка\-ющим из линейной структуры задачи 
в~отношении переменной~$z$, в~част\-ности, тем фактом, что такой вид 
функции Беллмана обеспечивает линейность оптимального 
управ\-ле\-ния~(\ref{e12-bos}) по~$z$.

     Граничное условие при выбранном предположении~(\ref{e15-bos}) 
принимает вид:

\noindent
     \begin{multline*}
     V_T(y,z)= S_T\left( s_T y- g_T z\right)^2+G_T z^2 ={}\\[-0.5pt]
     {}=\alpha_T z^2 
+\beta_T(y) z +\gamma_T(y)\,,
    \end{multline*}
т.\,е.

\noindent
\begin{align*}
\alpha_T&= S_T g_T^2 +G_T\,;\\[-0.5pt]
\beta_T(y)&=-2S_T s_T g_T y\,;\\[-0.5pt]
\gamma_T(y)&=S_T s_T^2 y^2\,.
%\label{e16-bos}
\end{align*}
          При этом само оптимальное управ\-ле\-ние, определенное 
выражением~(\ref{e12-bos}), оказывается управ\-ле\-ни\-ем с~обратной связью 
по~$y_t$ и~$z_t$:

\noindent
     \begin{multline}
     u_t^*=u_t^*(y,z) ={}\\[-0.5pt]
     {}=
     -\fr{1}{2}\left( S_t h_t^2 +H_t\right)^{-1}
     \left( c_t \left( 2\alpha_t z +\beta_t(y)\right) +{}\right.\\[-0.5pt]
    \left. {}+2S_t\left( s_t y-g_t z\right) 
h_t\right)\,.
     \label{e17-bos}
     \end{multline}
          Подставляем $V_t(y,z)\hm= \alpha_t z^2 \hm+ \beta_t(y) 
z\hm+\gamma_t(y)$ в~(\ref{e14-bos}):

\noindent
     \begin{multline*}
     \fr{\partial \alpha_t}{\partial t}\, z^2 +
     \fr{\partial \beta_t(y)}{\partial t}\,z +
     \fr{\partial \gamma_t(y)}{\partial t}+{}\\[-0.5pt]
     {}+\fr{1}{2}\left( \Sigma_t^2(y) \left( 
\fr{\partial^2\beta_t(y)}{\partial y^2}\,z +\fr{\partial^2 \gamma_t(y)}{\partial 
y^2}\right) +2\sigma_t^2\alpha_t\right)+{}\\[-0.5pt]
 {}+A_t(y)\left(\fr{\partial \beta_t(y)}{\partial y}\,z + \fr{\partial 
\gamma_t(y)}{\partial y}\right) +{}\\[-0.5pt]
\hspace*{-0.22987pt}{}+\left( a_t y+b_t z+\left( S_t h_t^2 +H_t\right)^{-1} c_t S_t \left( s_t y-
g_t z\right) h_t\right)\times{}
\end{multline*}

\noindent
\begin{multline*}
         {}\times \left( 2\alpha_t z+\beta_t(y)\right)+{}\\
     {}+\left( S_t-\left( S_t h_t^2 +H_t\right)^{-1} S_t^2 h_t^2\right)\left( s_t y-
g_t z\right)^2+{}\\
     {}+ G_t z^2 -\fr{1}{4}\left( S_t h_t^2 +H_t\right)^{-1} c_t^2 \left( 
2\alpha_t z+\beta_t(y)\right)^2=0\,.
     \end{multline*}
          Далее выделяем слагаемые при~$z^2$, $z$ и~$z^0$
          
          \noindent
     \begin{multline*}
     \fr{\partial \alpha_t}{\partial t}\, z^2 +\fr{\partial \beta_t(y)}{\partial t}\,z +
     \fr{\partial \gamma_t(y)}{\partial 
t}+\fr{1}{2}\,\Sigma_t^2(y)\fr{\partial^2\beta_t(y)}{\partial y^2}\,z+ {}\\
{}+
\fr{1}{2}\,\Sigma_t^2(y)\fr{\partial^2\gamma_t(y)}{\partial 
y^2}+\sigma_t^2\alpha_t+A_t(y)\fr{\partial \beta_t(y)}{\partial y}\,z +{}\\
{}+A_t(y) \fr{\partial 
\gamma_t(y)}{\partial y}+{}\\
{}+ 2\alpha_t \left( b_t -\left( S_t h_t^2+H_t\right)^{-1} c_t 
S_t h_t g_t \right)z^2+{}\\
     {}+
     \left( 2\alpha_t\left( \alpha_t+\left( S_t h_t^2+H_t\right)^{-1} c_t S_t h_t 
s_t\right)y +{}\right.\\
\left.{}+\beta_t(y) \left( b_t-\left( S_t h_t^2+H_t\right)^{-1} c_t S_t h_t 
g_t\right) \right) z+{}\\
     {}+\beta_t(y)\left( a_t +\left( S_t h_t^2+H_t\right)^{-1} c_t S_t h_t s_t\right) 
y+{}\\
{}+ \left( S_t -\left( S_t h_t^2+H_t\right)^{-1} S_t^2 h_t^2\right) g_t^2 z^2-{}\\
     {}- 2\left( S_t -\left( S_t h_t^2+H_t\right)^{-1} S_t^2 h_t^2\right) s_t g_t yz 
+{}\\
{}+
     \left( S_t-\left( S_t h_t^2+H_t\right)^{-1} S_t^2 h_t^2\right) s_t^2 y^2+{}\\
     {}+G_t z^2 -\left( S_t h_t^2 +H_t\right)^{-1} c_t^2 \alpha_t^2 z^2 -{}\\
     {}-\left( 
S_t h_t^2+H_t\right)^{-1} c_t^2 \alpha_t \beta_t(y) z-{}\\
{}-
\fr{1}{4}\left( S_t h_t^2+H_t\right)^{-1}  c_t^2 \beta_t^2(y)=0\,,
     \end{multline*}
группируем их и~получаем сле\-ду\-ющие уравнения:
\begin{itemize}
\item  для~$\alpha_t$:

\noindent
\begin{multline}
\fr{\partial\alpha_t}{\partial t}+2\alpha_t\left( b_t-\left( S_t h_t^2+H_t\right)^{-1} c_t 
S_t h_t g_t\right)+{}\\
{}+ \left( S_t- \left( S_t h_t^2+H_t\right)^{-1} S_t^2 h_t^2\right) 
g_t^2+G_t-{}\\
\hspace*{-8mm}{}-\left( S_t h_t^2+H_t\right)^{-1} c_t^2 \alpha_t^2 =0\,,\enskip \alpha_T=S_T 
g_t^2+G_T\,;\!\!
\label{e18-bos}
\end{multline}
\item для $\beta_t$:

\noindent
\begin{multline}
\fr{\partial\beta_t(y)}{\partial 
t}+\fr{1}{2}\,\Sigma_t^2(y)\fr{\partial^2\beta_t(y)}{\partial y^2} 
+A_t(y)\fr{\partial \beta_t(y)}{\partial y}+{}\\
{}+ 2\alpha_t\left( a_t +\left( S_t h_t^2+H_t\right)^{-1} c_t S_t h_t s_t\right) y+{}\\
{}+
\beta_t(y)\left( b_t -\left( S_t h_t^2 +H_t\right)^{-1} c_t S_t h_t g_t\right)-{}\\
{}-2\left( S_t-\left( S_t h_t^2+H_t\right)^{-1} S_t^2 h_t^2\right) s_t g_t y-{}
\\
{}-
\left( S_t h_t^2+H_t\right)^{-1} c_t^2 \alpha_t \beta_t(y)=0\,,\\
\beta_T(y)=-2S_T s_T g_T y\,;
\label{e19-bos}
\end{multline}
\item  для $\gamma_t$:
\begin{multline}
\hspace*{-0.8pt}\fr{\partial \gamma_t(y)}{\partial t}+\fr{1}{2}\,\Sigma_t^2(y)
\fr{\partial^2 \gamma_t(y)}{\partial y^2} +\sigma_t^2 \alpha_t +A_t(y)
\fr{\partial \gamma_t(y)}{\partial y}+{}\\
{}+ \beta_t(y)\left( a_t +\left( S_t h_t^2+H_t\right)^{-1} c_t S_t h_t s_t\right) y+{}\\
{}+
\left( S_t-\left( S_t h_t^2+H_t\right)^{-1} S_t^2 h_t^2\right)  s_t^2 y^2-{}\\
{}-\fr{1}{4}\left( S_t h_t^2+H_t\right)^{-1} c_t^2 \beta_t^2(y) =0\,,\\
\gamma_T(y)=S_T s_T^2 y^2\,.
\label{e20-bos}
\end{multline}
\end{itemize}
     
     Уравнение~(\ref{e18-bos}), легко заметить, является уравнением 
Риккати, которое в~силу сформулированного выше условия   
имеет единственное неотрицательное решение для всех $0\hm\leq t\hm\leq T$. 
Этот факт требует дополнительного комментария. Для получения 
уравнения~(\ref{e18-bos}) рас\-смот\-рим исходную задачу при дополнительных 
условиях $a_t\hm=0$ и~$s_t\hm=0$ для всех $0\hm\leq t\hm\leq T$. Нетрудно 
видеть, что эти условия рассматриваемую по\-ста\-нов\-ку сводят фактически 
к~классической ли\-ней\-но-квад\-ра\-тич\-ной задаче. Имеющуюся 
в~рассматриваемой формулировке чуть более общую форму целевой 
функции (принципиального значения это обобщение, конечно, не имеет) 
сведем к~классической еще одним предположением: $S_t\hm=0$ для всех 
$0\hm\leq t\hm\leq T$. Теперь уравнение для~$\alpha_t$ принимает хорошо 
известный вид:
     \begin{equation}
     \fr{\partial \alpha_t}{\partial t}+2\alpha_t b_t +G_t- H_t^{-1} c_t^2 
\alpha_t^2=0\,,\enskip \alpha_T=G_T\,.
     \label{e21-bos}
     \end{equation}

     В таком случае, как известно~\cite{10-bos}, существует единственное 
оптимальное управление~--- линейное с~обратной связью по выходу~$z_t$, 
с~коэффициентом усиления, опи\-сы\-ва\-емым уравнением  
Риккати~(\ref{e21-bos}). Именно этот результат дают  
уравнения~(\ref{e18-bos})--(\ref{e20-bos}) и~описываемая ими функция 
Беллмана~(\ref{e15-bos}), так как из $a_t\hm=0$ и~$s_t\hm=0$ немедленно 
следует, что $\beta_t(y)\hm=0$, откуда, в~свою очередь, с~учетом 
не\-за\-ви\-си\-мости решения от~$y_t$ следует, что $\gamma_t(y)\hm=\gamma_t$, 
т.\,е.\ не зависит от~$y$ и~задается уравнением: 
     $$
     \fr{\partial \gamma_t(y)}{\partial t} +\sigma^2_t \alpha_t=0\,,\enskip 
\gamma_T=0\,.
     $$ 
     Оптимальное управ\-ле\-ние при этом 
     $$
     u_t^*= -H_t^{-1} c_t \alpha_t z_t\,,
     $$
      т.\,е.\ все полностью совпадает с~известным классическим решением.
     
     С уравнениями~(\ref{e19-bos}) и~(\ref{e20-bos}) ситуация, естественно, 
обстоит сложнее. Это линейные уравнения второго порядка параболического 
типа, поскольку\linebreak
 $\Sigma_t^2(y)\hm>0$. Фактически отсутствуют 
конструктивные условия, гарантирующие существование их\linebreak
 решений 
(требовать, чтобы все фигурирующие в~уравнениях коэффициенты были 
представлены аналитическими функциями на всем пространстве значений, 
вряд ли целесообразно), поэтому далее будем предполагать, что данные 
уравнения имеют на рас\-смат\-ри\-ва\-емом интервале $0\hm\leq t\hm\leq T$ хотя 
бы одно ограниченное решение и~именно эти условия будем рас\-смат\-ри\-вать 
как достаточные условия существования оптимального решения 
рассматриваемой задачи.
     
     Таким образом, доказана следующая тео\-рема.
     
     \smallskip
     
     \noindent
     \textbf{Теорема.}\ \textit{Пусть для диффузионного 
процесса}~(\ref{e5-bos}) \textit{выполнены условия Ито, для 
     процесса}~(\ref{e6-bos})~--- \textit{ограничены коэффициенты, 
уравнения}~(\ref{e18-bos})--(\ref{e20-bos}) \textit{имеют ограниченные 
решения для $0\hm\leq t\hm\leq T$. Тогда минимум  
функционалу}~(\ref{e7-bos}) \textit{доставляет оптимальное 
управ\-ле\-ние}~(\ref{e17-bos}), \textit{где} $y\hm= y_t$; $z\hm=z_t$.
     
\section{Заключение}

     Рассмотренная задача оптимизации в~целом близка и~по модели, и~по 
критерию к~классической ли\-ней\-но-квад\-ра\-тич\-ной постановке. 
Принципиальным отличием является нелинейная модель для описания 
со\-сто\-яния динамической сис\-те\-мы, в~которой отсутствует управ\-ля\-ющее 
воздействие.\linebreak
 Такую модель наряду с~традиционной интер\-пре\-тацией  
<<со\-сто\-яние--вы\-ход>> мож\-но понимать как\linebreak модель неконтролируемого 
слож\-но\-го внешнего воздействия. Небольшое дополнительное отличие дает 
предложенная форма квад\-ра\-тич\-но\-го критерия, поз\-во\-ля\-ющая, в~част\-ности, 
ставить такие задачи, как отслеживание выходом или управ\-ле\-ни\-ем со\-сто\-яния 
сис\-те\-мы или ее выхода.
     
     Поскольку обсуждать возможности точного решения уравнений, 
определяющих оптимальное управ\-ле\-ние, не имеет смыс\-ла, наиболее 
актуальной далее является задача их приближенного чис\-лен\-но\-го решения 
и~анализа воз\-мож\-ности практической реализации. Этому посвящена вторая 
часть данной работы, пла\-ни\-ру\-емая к~выходу в~ближайшее время.

{\small\frenchspacing
 {%\baselineskip=10.8pt
 \addcontentsline{toc}{section}{References}
 \begin{thebibliography}{99}
\bibitem{1-bos}
\Au{Athans M.} Editorial on the LQG problem~// IEEE~T. Automat. Contr., 1971. Vol.~16. 
No.\,6. P.~528--552. doi: 10.1109/TAC.1971.1099845.
\bibitem{2-bos}
\Au{Wu Z.} Forward-backward stochastic differential equations, linear quadratic stochastic 
optimal control and nonzero sum differential games~// J.~Syst. Sci. Complex., 2005. Vol.~18. 
No.\,2. P.~179--192.
\bibitem{3-bos}
\Au{Chen B.\,S., Zhang~W.} Stochastic H2/H1 control with state-dependent noise~// IEEE 
T.~Automat. Contr., 2004. Vol.~49. No.\,1. P.~45--56. doi: 10.1109/TAC.2003.821400.
\bibitem{4-bos}
\Au{Bohacek S.} A~stochastic model of TCP and fair video transmission~// IEEE 
INFOCOM, 2003. Vol.~2. P.~1134--1144. doi: 10.1109/INFCOM.2003.1208950.
\bibitem{5-bos}
\Au{Домбровский В.\,В., Объедко~Т.\,Ю.} Управление с~прогнозированием системами 
с~марковскими скачками при ограничениях и~применение к~оптимизации 
инвестиционного портфеля~// Автомат. телемех., 2011. №\,5. С.~96--112. doi: 
10.1134/S0005117911050079.
\bibitem{6-bos}
\Au{Баландин Д.\,В., Коган~М.\,М.} Оптимальное линейно-квад\-ра\-тич\-ное управление: от 
матричных уравнений к~линейным матричным неравенствам~// Автомат. телемех., 2011. 
№\,11. С.~60--69. doi: 10.1134/ S0005117911110038.
\bibitem{7-bos}
\Au{Босов А.\,В.} Обобщенная задача распределения ресурсов программной системы~// 
Информатика и~её применения, 2014. Т.~8. Вып.~2. С.~39--47. doi: 
10.14357/19922264140204.
\bibitem{8-bos}
\Au{Босов А.\,В.} Управление линейным выходом дискретной стохастической системы по 
квадратичному критерию~// Изв. РАН. Теория и~системы управления, 2016. №\,3.  
С.~19--35. doi: 10.1134/S1064230716030060.
\bibitem{9-bos}
\Au{Флеминг У., Ришел~Р.} Оптимальное управление детерминированными 
и~стохастическими системами~/ Пер. с~англ.~--- М.: Мир, 1978. 316~с. 
(\Au{Fleming~W.\,H., Rishel~R.\,W.} Deterministic and stochastic optimal control.~--- New 
York, NY, USA: Springer-Verlag, 1975. 222~p.)
\bibitem{10-bos}
\Au{Девис М.\,Х.\,А.} Линейное оценивание и~стохастическое управление~/ Пер. с~англ.~--- 
М.: Наука, 1984. 206~с. (\Au{Davis~M.\,H.\,A.} Linear estimation and stochastic control.~--- 
London: Chapman and Hall, 1977. 224~p.)

 \end{thebibliography}

 }
 }

\end{multicols}

\vspace*{-6pt}

\hfill{\small\textit{Поступила в~редакцию 30.03.18}}

\vspace*{4pt}

%\newpage

%\vspace*{-24pt}

\hrule

\vspace*{2pt}

\hrule

\vspace*{-2pt}


\def\tit{STOCHASTIC DIFFERENTIAL SYSTEM OUTPUT CONTROL 
BY~THE~QUADRATIC CRITERION.~I.~DYNAMIC\\ PROGRAMMING 
OPTIMAL SOLUTION}


\def\titkol{Stochastic differential system output control 
by~the~quadratic criterion. I.~Dynamic programming 
optimal solution}

\def\aut{A.\,V.~Bosov and~A.\,I.~Stefanovich}

\def\autkol{A.\,V.~Bosov and~A.\,I.~Stefanovich}

\titel{\tit}{\aut}{\autkol}{\titkol}

\vspace*{-11pt}


\noindent
Institute of Informatics Problems, Federal Research Center ``Computer Science 
and Control'' of the Russian Academy of Sciences, 44-2~Vavilov Str., Moscow 
119333, Russian Federation


\def\leftfootline{\small{\textbf{\thepage}
\hfill INFORMATIKA I EE PRIMENENIYA~--- INFORMATICS AND
APPLICATIONS\ \ \ 2018\ \ \ volume~12\ \ \ issue\ 3}
}%
 \def\rightfootline{\small{INFORMATIKA I EE PRIMENENIYA~---
INFORMATICS AND APPLICATIONS\ \ \ 2018\ \ \ volume~12\ \ \ issue\ 3
\hfill \textbf{\thepage}}}

\vspace*{3pt}



\Abste{The problem of optimal control for the Ito diffusion 
process and a~controlled linear output is solved. The considered 
statement is close to the classical linear-quadratic Gaussian 
control  (LQG control) problem. Differences consist in the fact 
that the state is described by the nonlinear differential Ito equation  $dy_y = A_t(y_t) 
\,dt+\Sigma_t(y_t)\,dv_t$ and does not depend on the control~$u_t$, 
optimization subject is controlled linear output 
 $dz_t=a_ty_t\,dt +b_tz_t\,dt +c_t u_t\,dt +\sigma_t \,dw_t$. 
Additional generalizations are included in the quadratic 
quality criterion for the purpose of statement such problems 
as state tracking by output or a linear combination of state 
and output tracking by control. The method of dynamic programming 
is used for the solution. 
The assumption about Bellman function in the form  $V_t(y,z)= \alpha_t 
z^2+\beta_t(y) z+\gamma_t(y)$ allows one to find it. 
Three differential equations for the coefficients $\alpha_t$,  $\beta_t(y)$,
and $\gamma_t(y)$ give the solution. 
These equations constitute the optimal solution of the problem under consideration.}

\KWE{stochastic differential equation; optimal control; dynamic programming; 
Bellman function; Riccati equation; linear differential equations of parabolic type}


\DOI{10.14357/19922264180314}

\vspace*{-12pt}

\Ack
\noindent
This work was partially supported by the Russian Science Foundation (grant  
16-07-00677).



%\vspace*{6pt}

  \begin{multicols}{2}

\renewcommand{\bibname}{\protect\rmfamily References}
%\renewcommand{\bibname}{\large\protect\rm References}

{\small\frenchspacing
 {%\baselineskip=10.8pt
 \addcontentsline{toc}{section}{References}
 \begin{thebibliography}{99}
\bibitem{1-bos-1}
\Aue{Athans, M.} 1971. Editorial on the LQG problem. \textit{IEEE~T. 
Automat. Contr.} 16(6):528--552. doi: 10.1109/ TAC.1971.1099845.
\bibitem{2-bos-1}
\Aue{Wu, Z.} 2005. Forward-backward stochastic differential equations, linear 
quadratic stochastic optimal control and\linebreak\vspace*{-12pt}

\columnbreak

\noindent
 nonzero sum differential games. 
\textit{J.~Syst. Sci. Complex.} 18(2):179--192.
\bibitem{3-bos-1}
\Aue{Chen, B.\,S. and W.~Zhang.} 2004. Stochastic H2/H1 control with  
state-dependent noise. \textit{IEEE~T. Automat. Contr.} 49(1):45--56.
doi: 10.1109/TAC.2003.821400.
\bibitem{4-bos-1}
\Aue{Bohacek, S.} 2003. A~stochastic model of TCP and fair video 
transmission. \textit{IEEE INFOCOM}. 2:1134--1144.
doi: 10.1109/INFCOM.2003.1208950.
\bibitem{5-bos-1}
\Aue{Dombrovskii, V.\,V., and T.\,Yu.~Ob''edko.} 2011. Predictive control of 
systems with Markovian jumps under constraints and its application to the 
investment portfolio optimization. \textit{Automat. Rem. Contr.}  
72(5):989--1003.
\bibitem{6-bos-1}
\Aue{Balandin, D.\,V., and M.\,M.~Kogan.} 2011. Optimal linear-quadratic 
control: From matrix equations to linear matrix inequalities. \textit{Automat. 
Rem. Contr.} 72(11):2276--2284.
\bibitem{7-bos-1}
\Aue{Bosov, A.\,V.} 2014. Obobshchennaya zadacha raspredeleniya resursov 
programmnoy sistemy [The generalized problem of software system resources 
distribution]. \textit{Informatika i~ee Primeneniya~--- Inform. Appl.}  
8(2):39--47. doi: 
10.14357/19922264140204.
\bibitem{8-bos-1}
\Aue{Bosov, A.\,V.} 2016. Discrete stochastic system linear output control 
with respect to a quadratic criterion. \textit{J.~Comput. Syst. Sc. 
Int.} 55(3):349--364.
\bibitem{9-bos-1}
\Aue{Fleming, W.\,H., and R.\,W.~Rishel.} 1975. \textit{Deterministic and 
stochastic optimal control.} New York, NY: Springer-Verlag. 222~p.
\bibitem{10-bos-1}
\Aue{Davis, M.\,H.\,A.} 1977. \textit{Linear estimation and stochastic 
control.} London: Chapman and Hall. 224~p.
\end{thebibliography}

 }
 }

\end{multicols}

\vspace*{-6pt}

\hfill{\small\textit{Received March 30, 2018}}

%\pagebreak

%\vspace*{-18pt}
     
     \Contr
     
       \noindent
       \textbf{Bosov Alexey V.} (b.\ 1969)~--- Doctor of Science in technology, 
principal scientist, Institute of Informatics Problems, Federal Research 
Center ``Computer Science and Control'' of the Russian Academy of Sciences, 
44-2~Vavilov Str., Moscow 119333, Russian Federation; 
\mbox{AVBosov@ipiran.ru}
       
       \vspace*{3pt}
       
       \noindent
       \textbf{Stefanovich Alexey I.} (b.\ 1983)~--- principal specialist, 
Institute of Informatics Problems, Federal Research Center ``Computer Science 
and Control'' of the Russian Academy of Sciences, 44-2~Vavilov Str., Moscow 
119333, Russian Federation; \mbox{AStefanovich@frccsc.ru}
\label{end\stat}

\renewcommand{\bibname}{\protect\rm Литература}       

       %3
%\newcommand {\ff}{{\mathcal F}}
\newcommand {\ebd}{\triangleq}
\newcommand{\me}[2]{\mathbf{E}_{ #1 }\left\{ \mathop{#2} \right\} }



\def\stat{borisov}

\def\tit{ФИЛЬТРАЦИЯ СОСТОЯНИЙ МАРКОВСКИХ СКАЧКООБРАЗНЫХ ПРОЦЕССОВ 
ПО~ДИСКРЕТИЗОВАННЫМ НАБЛЮДЕНИЯМ$^*$}

\def\titkol{Фильтрация состояний марковских скачкообразных процессов 
по~дискретизованным наблюдениям}

\def\aut{А.\,В.~Борисов$^1$}

\def\autkol{А.\,В.~Борисов}

\titel{\tit}{\aut}{\autkol}{\titkol}

\index{Борисов А.\,В.}
\index{Borisov A.\,A.}




{\renewcommand{\thefootnote}{\fnsymbol{footnote}} \footnotetext[1]
{Работа выполнена при частичной поддержке РФФИ (проект 16-07-00677).}}


\renewcommand{\thefootnote}{\arabic{footnote}}
\footnotetext[1]{Институт проблем информатики Федерального исследовательского центра <<Информатика 
и~управление>> Российской академии наук,
\mbox{aborisov@frccsc.ru}}

%\vspace*{8pt}



\Abst{Статья посвящена решению задачи оптимальной 
фильтрации состояний однородного марковского скачкообразного процесса (МСП). 
Наблюдения представляют собой приращения случайных процессов~--- интегральных 
преобразований состояний, зашумленные винеровскими процессами, интенсивность 
которых также зависит от оцениваемого состояния. Оптимальная оценка в~моменты 
получения нового наблюдения вычисляется как функция предыдущей оценки и~новых 
наблюдений, а~между моментами наблюдений~--- простейшим прогнозом в~силу системы 
уравнений Колмогорова. Рекуррентная формула пересчета ресурсозатратна, так как 
содержит  интегралы~--- мас\-штаб\-но-сдви\-го\-вые смеси многомерных гауссиан, 
где в~качестве смешивающих выступают распределения времени пребывания 
состояния в~каждом из возможных значений. Предложены более простые аппроксимации, 
основанные на предположении об ограниченности числа скачков состояния за время между 
наблюдениями. Получены универсальные локальная и~глобальная характеристики точности 
аппроксимаций, зависящие от па\-ра\-мет\-ров оцениваемого процесса, величины 
временн$\acute{\mbox{о}}$го шага  между наблюдениями и~максимального числа учитываемых скачков.}

\KW{марковский скачкообразный процесс; оптимальная фильтрация; мультипликативные 
шумы в~наблюдениях; стохастическое дифференциальное уравнение; численная аппроксимация}

\DOI{10.14357/19922264180316}
  
%\vspace*{4pt}


\vskip 10pt plus 9pt minus 6pt

\thispagestyle{headings}

\begin{multicols}{2}

\label{st\stat}



 \section{Введение}
 
 Фильтр Вонэма~\cite{Won_65}~--- один из редких удачных случаев, когда 
 оценка оптимальной фильтрации состо\-яния стохастической системы наблюдения 
 выражается в~виде решения некоторой замк\-ну\-той\linebreak конечномерной сис\-те\-мы 
 стохастических дифференциальных уравнений. 
 
 Алгоритм данного фильт\-ра 
 позволяет вычислить оценку фильт\-ра\-ции со\-сто\-яния \textit{марковского скачкообразного 
 процесса} с~\mbox{конечным} множеством состояний по наблюдениям в~присутствии 
 аддитивных винеровских шумов. Теоретически оптимальная оценка со\-сто\-яния~--- 
 его условное распределение в~текущий момент времени~--- 
 обладает очевидными свойствами неотрицательности и~нормировки. 
 При чис\-лен\-ной реализации данного фильтра классическим методом 
 Эй\-ле\-ра--Ма\-ру\-ямы~\cite{KP_92} данные свойства могут не сохраняться и~процедура 
 вы\-чис\-ле\-ния становится неустойчивой.  В~связи с~этим обстоятельством разрабатывались 
 другие алгоритмы чис\-лен\-но\-го решения уравнения фильтра Вонэма, обладающие 
 требуемыми свойствами устойчивости (см.~\cite{YZL_04, PR_10} и~библиографию в~них). 
 В~час\-ти этих работ доказана лишь слабая сходимость пред\-ла\-га\-емых аппроксимационных 
 схем к~оценке фильт\-ра Вонэма, в~то время как ка\-кая-ли\-бо 
 характеризация точ\-ности этих приближений отсутствует.
 
 В~\cite{B_18} было представлено распространение фильт\-ра Вонэма на случай 
 наблюдений с~мультипликативными шумами. При этом уравнение обобщенного 
 фильт\-ра содержит в~правой части квадратическую характеристику шумов в~наблюдениях. 
 Данный процесс на практике никогда не наблюдается непосредственно, а~является лишь 
 некоторым нелинейным интегральным преобразованием наблюдений. Очевидно, что 
 имеющиеся в~настоящий момент времени алгоритмы приближенного вычисления оценки 
 фильтрации Вонэма для данной системы не подходят. 
 
 Целью предлагаемой работы является ис\-поль\-зование результатов оптимальной 
 фильтрации со\-стояний сис\-тем с~дискретным временем для аппроксимации решения 
 аналогичной задачи для\linebreak стохастических дифференциальных сис\-тем. 
 
 Статья организована следующим образом. Раздел~2 содержит формальную постановку 
 задачи фильт\-ра\-ции со\-сто\-яний однородного МСП с~конечным множеством со\-сто\-яний 
 по наблюдениям, полученным путем временн$\acute{\mbox{о}}$й дискретизации процессов с~непрерывным 
 временем~--- интегральных преобразований со\-сто\-яния сис\-те\-мы в~присутствии 
 мультипликативных винеровских шумов.\linebreak
  В~разд.~3 пред\-став\-ле\-но решение поставленной 
 задачи фильт\-ра\-ции: пересчет оценок со\-сто\-яний в~момент получения новых 
 дискретизованных наблюдений выполняется в~соответствии с~некоторыми\linebreak 
 рекуррентными интегральными соотношениями, в~то время как между 
 моментами наблюдений оценка корректируется в~соответствии с~прогнозом в~силу 
 сис\-те\-мы уравнений Колмогорова. Вы\-чис\-ли\-тель\-ная слож\-ность 
 упомянутых выше интегральных\linebreak 
 соотношений связана с~тем, что в~расчет принимается воз\-мож\-ность того, что между 
 моментами наблюдений оцениваемое со\-сто\-яние может совершить произвольное чис\-ло 
 скачков. В~разд.~4 пред\-став\-лен более простой алгоритм приближенного вы\-чис\-ле\-ния 
 оценки фильт\-ра\-ции, основанный на ограничении возможного числа учитываемых скачков 
 МСП. Доказана тео\-ре\-ма, опре\-де\-ля\-ющая как\linebreak
  локальную (одношаговую), так и~глобальную 
 (многошаговую) характеристики точ\-ности предложенного при\-бли\-же\-ния~--- 
 $\ell_1$-нор\-мы ошибки аппроксимации. Полученные характеристики являются\linebreak 
 универсальными, т.\,е.\ не асимптотическими по шагу дискретизации, и~зависят от характеристик 
 самого МСП, %\linebreak
  шага временн$\acute{\mbox{о}}$й дискретизации и~чис\-ла
  скачков со\-сто\-яния, учи\-ты\-ва\-емых 
 на шаге. Об\-суж\-де\-ние результатов и~заключительные комментарии пред\-став\-ле\-ны 
 в~разд.~5.
 
 \section{Постановка задачи фильтрации}
 
 На полном вероятностном пространстве с~фильт\-ра\-цией 
 $(\Omega,\mathcal{F},\mathcal{P},\{\mathcal{F}_{t}\}_{t \geqslant 0})$ рассматривается система наблюдений
\begin{equation}
 \left.
 \begin{array}{rl}
 \displaystyle X_t &=X_0 +  \displaystyle
 \int\limits_0^t \Lambda^{\top}X_{s}\,ds + \mu_s\,;  \\[6pt]
 \displaystyle Y_k &=  \displaystyle\int\limits_{t_{k-1}}^{t_k}fX_s\,ds+
 \int\limits_{t_{k-1}}^{t_k} 
 \sum\limits_{n=1}^NX_s^ng_n \,dW_s, \\[6pt]
 &\hspace*{10mm}\{t_k\}_{k \geqslant 0}: \; 0 = t_0 < t_1 < t_2\cdots,
 \end{array}
 \right\}
 \label{eq:obsys_1}
 \end{equation}
 где
  \begin{itemize}
  \item
  $X_t \ebd \mathrm{col}\left(X_t^1,\ldots,X_t^N\right) \hm\in \mathbb{S}^N$~--- 
  ненаблюда\-емое состояние системы, являющееся однородным МСП с~конечным 
  множеством состояний $ \mathbb{S}^N \ebd$\linebreak $\ebd \{e_1,\ldots,e_N\}$ ($\mathbb{S}^N$~--- 
  множество единичных векторов евклидова пространства~$\mathbb{R}^N$), 
  матрицей интенсивностей переходов~$\Lambda$ и~начальным распределением~$\pi$;
  \item
  $\mu_t \ebd \mathrm{col}\left(
  \mu_t^1,\ldots,\mu_t^N\right)\hm\in \mathbb{R}^N$~--- 
  ${\mathcal{F}}_t$-со\-гла\-со\-ван\-ный мартингал;
  \item
  $\{Y_k\}_{k \in \mathbb{N}}:\;  Y_k \ebd \mathrm{col}\left(Y_k^1,\ldots,Y_k^M\right) 
  \hm\in \mathbb{R}^M$~--- последовательность дискретизованных наблюдений, 
  доступных в~известные неслучайные  моменты времени~$\{t_k\}_{k \in \mathbb{N}}$,
в~которых $W_t \ebd$\linebreak $\ebd \mathrm{col}\left(W_t^1,\ldots,W_t^M\right) \hm\in \mathbb{R}^M$
 является ${\mathcal{F}}_t$-со\-гла\-со\-ван\-ным стандартным винеровским процессом, 
 определяющим шумы в~наблюдениях,\linebreak  $f$~--- $(M \times N)$-мер\-ная 
 мат\-ри\-ца плана наблюдений, а~набор мат\-риц~$\{g_n\}_{n=\overline{1,N}}$ 
 характеризует интенсивности шумов в~зависимости от текущего состояния~$X_t$.
  \end{itemize}
  
  Введем также в~рассмотрение неубывающие семейства $\sigma$-ал\-гебр 
  $\mathcal{O}_k \ebd \sigma\{ Y_{\ell}: \; 1 \hm\leqslant \ell \hm\leqslant k\}$ 
  и~$\mathcal{O}_t \ebd  \mathcal{O}_{k(t)}$, где 
  $k(t) \ebd \sum\nolimits_{j \in \mathbb{N}}\mathbf{I}(t-t_{j})$; 
  $\mathcal{O}_0 \ebd \{\varnothing,\; \Omega\}$.
  
   \textit{Задача оптимальной фильтрации состояния~$X$ по наблюдениям~$Y$} 
   заключается в~нахождении \textit{условного математического ожидания} (УМО)
  \begin{equation*}
  \widehat{X}_t \ebd {\sf E}\left\{X_t|\mathcal{O}_{t} \right\}\,.
 % \label{eq:fest_1}
  \end{equation*}
  
  Относительно системы~(\ref{eq:obsys_1})  сделаны следующие предположения:
   \begin{itemize}
 \item[(а)]
 ${\mathcal{F}}_t \equiv {\mathcal{F}}_{t}^X \bigvee 
 {\mathcal{F}}_{t}^W $ для любого $t \hm\geqslant 0$;
 \item[(б)]
 шумы в~наблюдениях равномерно невырожденные, т.\,е.\
  $g_ng_n^{\top} \hm\geqslant \alpha I \hm> 0$ для всех $n\hm=\overline{1,N}$ 
  и~некоторого $\alpha\hm>0$.
% \item
 % Верно неравенство
  %\begin{equation}
  %\min_{1\leqslant k \leqslant N}|\lambda_{kk}| > 0.
  %\label{eq:ineq_0}
  % \end{equation}
 %\item
 %Для любого $t \geqslant 0$ все компоненты вектора $p_t \ebd \me{}{X_t}$ строго %положительны. 
 \end{itemize} 

 \section{Уравнения оптимального фильтра} 
 
 Для получения уравнений оптимального фильт\-ра воспользуемся подходом, 
 применяемым для решения аналогичной задачи в~стохастических сис\-те\-мах 
 наблюдения с~дискретным временем~\cite{BSh_85}. 
 Воспользу\-ем\-ся методом математической индукции. 
 
 При $r=0$ 
 \begin{equation}
 \widehat{X}_{t_0}={\sf E}\{X_0|\mathcal{O}_0\}={\sf E}\{X_0\}=\pi\,.
 \label{eq:in_cond}
 \end{equation} 
 
 Пусть для некоторого $ r \hm\geqslant 0$ известна оценка оптимальной 
 фильтрации~$\widehat{X}_{t_r} \hm= {\sf E}{X_{t_r} |\mathcal{O}_r}$. 
 Определим оценку оптимальной фильтрации~$\widehat{X}_{t} $ для $t\hm \in (t_r,t_{r+1}]$. 
 
 Для произвольного момента $t \hm\in (t_r,t_{r+1})$ в~силу мартингального 
 разложения МСП~$X_t$ и~свойств УМО верна следующая цепочка равенств:
 \begin{multline*}
 \widehat{X}_{t} = {\sf E}\left\{X_t | \mathcal{O}_r\right\}={}\\
 {}=
 {\sf E}\left\{{\cal P}^{\top}(t_r,t)X_{t_r}+
 \int\limits_{t_r}^t{\cal P}^{\top}(t_r,s)\,dM_s\big\vert \mathcal{O}_r\right\} = {}
\end{multline*}

\noindent
   \begin{multline}
 \hspace*{-11.66pt}{}=\mathcal{P}^{\top}(t_r,t)\widehat{X}_{t_r} + {\sf E}\hspace*{-2pt}
 \left\{{\sf E}\hspace*{-2pt}\left\{\int\limits_{t_r}^t\hspace*{-2pt}\mathcal{P}^{\top}(t_r,s)\,dM_s |
 {\mathcal{F}}_{t_r}\right\}\!\big\vert 
 \mathcal{O}_r\!\right\} ={}\hspace*{-4.24124pt}\\
 {}=
  \mathcal{P}^{\top}(t_r,t)\widehat{X}_{t_r}\,,
 \label{eq:bw_obs}
 \end{multline}
 где $\mathcal{P}(s,t)$ $(s \hm\leqslant t)$~--- матрица переходной ве\-ро\-ят\-ности МСП 
 на промежутке $[s,t]$, являющаяся решением сис\-те\-мы дифференциальных 
 уравнений Колмогорова
 \begin{equation*}
 \mathcal{P}'_t(s,t) = \mathcal{P}(s,t) \Lambda, \enskip t > s, \enskip \mathcal{P}(s,s) = I.
 \end{equation*}
 В случае однородного МСП $\mathcal{P}(s,t) \hm= e^{(t-s)\Lambda}$.
 
 Далее необходимо определить совместное распределение $(X_{t_{r+1}},Y_{r+1})$ 
 относительно~$ \mathcal{O}_r$. Из модели наблюдений следует, что 
 распределение~$Y_{r+1}$ относительно 
 $\sigma$-ал\-геб\-ры~$\mathcal{F}^X_{t_{r+1}} \vee \mathcal{O}_r$~---
 гауссовское с~параметрами 
 \begin{align*}
{\sf E}\left\{Y_{r+1}|{\mathcal{F}}^X_{t_{r+1}}\right\}& = f \tau_{r+1}\,; \\[6pt]
 \mathrm{cov} \left(Y_{r+1},Y_{r+1}|{\mathcal{F}}^X_{t_{r+1}}\right) &= 
 \displaystyle\sum\limits_{n=1}^N \tau_{r+1}^n g_ng_n^{\top}\,,
% \label{eq:occup_1}
 \end{align*}
 где $\tau_{r+1} \hm= \tau_{r+1}(X(\omega))=
 \mathrm{col}\left(\tau_{r+1}^1,\ldots,\tau_{r+1}^N\right) \ebd$\linebreak
 $\ebd 
 \int\nolimits_{t_r}^{t_{r+1}}X_s\,ds$~--- случайный вектор, $n$-я 
 компонента которого равна времени пребывания процесса~$X$ в~со\-сто\-янии~$e_n$ 
 на  интервале времени $[t_r, t_{r+1}]$. 
 Обозначим через $\mathcal{D}_{r+1} \ebd \{u=\mathrm{col}\,(u^1,\ldots,u^N):\; 
 u_m \hm\geqslant 0,\; \sum\nolimits_{m=1}^Mu_m\hm= t_{r+1}-t_r\}$ $(M-1)$-мер\-ный 
 симплекс в~пространстве~$\mathbb{R}^M$, являющийся носителем распределения 
 вектора~$\tau_{r+1}$. Пусть $\rho^{k,\ell}_{r+1}(du)$~--- 
 распределение вектора $\tau_{r+1} X_{t_{r+1}}^{\ell}$ при условии $X_{t_r}\hm=e_k$, 
 т.\,е.\ 
 для любого $\mathcal{A} \hm\in \mathcal{B}(\mathbb{R}^M)$ верно тождество:
\begin{multline*}
 \mathbf{P}\left\{\omega: \; X_{t_{r+1}}(\omega)=e_{\ell},\right.\\
 \left. 
 \tau_{r+1}(X(\omega)) \in \mathcal{A}\;|\;X_{t_r}=e_k\right\} \equiv
   \rho^{k,\ell}_{r+1}(\mathcal{A})\,.
\end{multline*}
 
Обозначим через
\begin{multline*}
 \mathcal{N}(y,m,K) \ebd (2\pi)^{-M/2} \mathrm{ det}^{-1/2} K\times{}\\
 {}\times\exp
 \left\{ -\fr{1}{2}\left(y-m\right)^{\top}K^{-1}(y-m)\right\}
\end{multline*}
 $M$-мер\-ную плот\-ность гауссовского распределения с~математическим 
 ожиданием~$m$ и~ковариационной матрицей~$K$.
 
 Из марковского свойства  $\{X_{t_{r}},Y_{r})\}_{r \geqslant 0}$ 
 относительно~${\mathcal{F}}_{t_{r}}$~\cite{ZhSh_95} и~теоремы Фубини следует, что 
 для любого  множества $\mathcal{A} \hm\in \mathcal{B}(\mathbb{R}^M)$ 
 верна следующая цепочка равенств:
 \begin{multline*}
 {\sf E}\left\{X_{t_{r+1}}\mathbf{I}_{\mathcal{A}}
 \left(Y_{r+1}\right)\big|\mathcal{O}_r\right\}={}\\
 {}=
{\sf E}\left\{{\sf E}\left\{X_{t_{r+1}}\mathbf{I}_{\mathcal{A}}
\left(Y_{r+1}\right)\big|
\mathcal{F}^X_{t_{r+1}} \vee \mathcal{O}_r\right\}
 \big|\mathcal{O}_r\right\} = {}
\end{multline*}

\noindent
\begin{multline*}
 %{}=
% {\sf E}\left\{{\sf E}\left\{X_{t_{r+1}}\mathbf{I}_{\mathcal{A}}
% \left(Y_{r+1}\right)\vert X_{t_r}\right\}
% \vert\mathcal{O}_r\right\} = {}\\
% {}=
%{\sf E}\left\{\sum\limits_{k=1}^N {\sf E}\left\{X_{t_{r+1}}\mathbf{I}_{\mathcal{A}}
%\left(Y_{r+1}\right)  \big| X_{t_r}=e_k\right\}X_{t_r}^k
% \big|\mathcal{O}_r\right\} = {}\\ 
% {}=
% \sum\limits_{k=1}^N{\sf E}
% \left\{X_{t_{r+1}}\mathbf{I}_{\mathcal{A}}\left(Y_{r+1}\right)\bigl| X_{t_r}=e_k\right\} 
% \widehat{X}_{t_r}^k ={}\\
% {}=\!
% \sum\limits_{k=1}^N{\sf E}
% \left\{{\sf E}\left\{X_{t_{r+1}}\mathbf{I}_{\mathcal{A}}
% \left(Y_{r+1}\right)\!\bigl| {\mathcal{F}}_{t_{r+1}}\right\}\!\bigl| 
% X_{t_r}\!=e_k\right\} \widehat{X}_{t_r}^k ={}\\
% {}=
% \sum\limits_{k=1}^N {\sf E}\left\{
% \vphantom{\int\limits_A\left(\sum\limits_{p=1}^N\right)}
% X_{t_{r+1}} \times{}\right.\\
% {}\times\int\limits_{\mathcal{A}}  
% \mathcal{N}\left(y,f \tau_{r+1}(X),\sum\limits_{p=1}^N \tau_{r+1}^p(X) g_pg_p^{\top}\right)dy
% \Biggl| X_{t_r}={}\\
%\left. {}=e_k
% \vphantom{\int\limits_A\left(\sum\limits_{p=1}^N\right)}
%\right\} \widehat{X}_{t_r}^k = 
% \sum\limits_{k=1}^N \int\limits_{\mathcal{A}}{\sf E}\left\{ 
% \vphantom{\sum\limits_{p=1}^N}
% X_{t_{r+1}} \times{}\right.\\
% {}\times\mathcal{N}\left(y,f \tau_{r+1}(X),\sum\limits_{p=1}^N \tau_{r+1}^p(X) 
% g_p g_p^{\top}\right)
% \Biggl| X_{t_r}={}\\
%\left. {}=e_k
%\vphantom{\sum\limits^N_{p=1}}
%\right\} \widehat{X}_{t_r}^k\, dy
 %={}\\
 {}=
 \sum\limits_{\ell=1}^N e_{\ell} \int\limits_{\mathcal{A}} 
 \left[ \sum\limits_{k=1}^N 
 \int\limits_{\mathcal{D}_{r+1}} 
 \mathcal{N}\left(y,f u,\sum_{p=1}^N u^p g_pg_p^{\top}\right)\times{}\right.\\
\left. {}\times
 \rho^{k,\ell}_{r+1}(du)\widehat{X}_{t_r}^k
 \vphantom{\int\limits_A\sum\limits_{p=1}^N}
 \right] 
 dy,
 \end{multline*}
 из чего следует, что интегранд в~квадратных скобках в~последнем выражении 
 определяет искомое совместное распределение $(X_{t_{r+1}},Y_{r+1})$ 
 относительно~$ \mathcal{O}_r$. Оценка~$\widehat{X}_{t_{r+1}}$ покомпонентно 
 определяется~\cite{BSh_85} с~помощью обобщенного варианта формулы Байеса:
 \begin{multline}
 \widehat{X}_{t_{r+1}}^j = {}\\
 \hspace*{-1mm}{}=
 \fr{\int\nolimits_{\mathcal{D}_{r+1}}\hspace*{-6mm} 
 \mathcal{N}\left(Y_{r+1},f u,\sum\nolimits_{p=1}^N \hspace*{-2mm}
 u^p g_pg_p^{\top}\!\right)\hspace*{-1mm}
 \sum\nolimits_{k=1}^N \hspace*{-2mm}
 \widehat{X}_{t_r}^k
 \rho^{k,j}_{r+1}(du)
 }
 { \int\nolimits_{\mathcal{D}_{r+1}} \hspace*{-6mm}
 \mathcal{N}\left(Y_{r+1},f v,\sum\nolimits_{q=1}^N \hspace*{-2mm}
 v^q g_qg_q^{\top}\!\right)\hspace*{-1mm}
 \sum\nolimits_{i,\ell=1}^N \hspace*{-2mm}
 \widehat{X}_{t_r}^i
 \rho^{i,\ell}_{r+1}(dv)
  },  \\ 
  j = \overline{1,N}\,.
 \label{eq:filt_1}
 \end{multline}
 Таким образом, доказана следующая
 
 %\smallskip
 
 \noindent
 \textbf{Лемма~1.}
\textit{Если для системы наблюдения}~(\ref{eq:obsys_1}) 
\textit{верны условия~(а) и~(б), то оценка~$\widehat{X}_t$ оптимальной фильтрации 
определяется формулой}~(\ref{eq:in_cond}) 
\textit{при $t\hm=0$, рекуррентным соотношением}~(\ref{eq:filt_1})~---
\textit{в~моменты~$t_{r+1}$ получения наблюдений~$Y_{r+1}$ 
и~формулой}~(\ref{eq:bw_obs})~--- 
\textit{в~промежутках времени между моментами получения наблюдений}.


\smallskip
 

 
 Несмотря на компактную запись~(\ref{eq:filt_1}), их прямая численная реализация 
 ресурсозатратна. Во-пер\-вых, в~(\ref{eq:filt_1}) требуется вычислять 
 распределения мас\-штаб\-но-сдви\-го\-вых смесей многомерных нормальных 
 распределений, что является трудоемкой\linebreak процедурой. Во-вто\-рых, 
 распределения~$\rho^{k,j}_{r+1}$ вре-\linebreak мени пребывания представляют собой 
 сумму\linebreak бесконечного ряда, слагаемые которого вычис\-ляются с~помощью 
 некоторой рекуррентной про\-це\-дуры~\cite{S_00}. В-третьих, 
 распределения~$\rho^{k,j}_{r+1}$ не являются абсолютно непрерывными 
 относительно меры Ле\-бега.
 { %\looseness=1
 
 }
 
 Следующий раздел посвящен численной аппроксимации~(\ref{eq:filt_1}) и~исследованию 
 ее точностных характеристик.
 
 \section{Приближенное вычисление оценки фильтрации}
 
 Без ограничения общности будем считать, что сетка~$\{t_r\}_{r \geqslant 0}$ 
 является равномерной с~шагом~$\Delta$, т.\,е.\ $t_r \hm= r \Delta$ 
 и~$\mathcal{D}_r \hm\equiv \mathcal{D}$.
 Обозначим через~$N_{r+1}$ об-\linebreak\vspace*{-12pt}
 
 \pagebreak
 
 \noindent
 щее число скачков процесса~$X_t$, имевших место 
 на промежутке $(t_r,t_{r+1}]$. Тогда из формулы полной вероятности следует, 
 что~(\ref{eq:filt_1}) представима в~виде:
 \begin{multline}
 \widehat{X}_{t_{r+1}}^j =  \left(
 \int\limits_{\mathcal{D}} 
 \mathcal{N}\left(Y_{r+1},f u,\sum\limits_{p=1}^N u^p g_pg_p^{\top}\right)\times{}\right.\\
\left. {}\times
 \sum\limits_{h=0}^{\infty}\sum\limits_{k=1}^N \widehat{X}_{t_r}^k
 \rho^{k,j,h}_{r+1}(du)
 \right)\Bigg/ \\
 \left(
 \vphantom{\sum\limits_{m=0}^{\infty}
 \sum\limits_{i,\ell=1}^N \widehat{X}_{t_r}^i
 \rho^{i,\ell,m}_{r+1}(dv)}
 \int\limits_{\mathcal{D}} 
 \mathcal{N}\left(Y_{r+1},f v,\sum\limits_{q=1}^N v^q g_qg_q^{\top}\right)\times{}\right.\\
\left.{}\times \sum\limits_{m=0}^{\infty}
 \sum\limits_{i,\ell=1}^N \widehat{X}_{t_r}^i
 \rho^{i,\ell,m}_{r+1}(dv)
 \right)
  \,, \enskip j = \overline{1,N}\,,
  \label{eq:filt_1_1}
 \end{multline}
 где 
 $ \rho^{k,j,h}_{r+1}(du)$~--- распределение вектора 
 $\tau_{r+1}X_{t_{r+1}}^{j}\mathbf{I}_{\{h\}}(N_{r+1})$ при 
 условии $X_{t_r}\hm=e_k$, т.\,е.\ 
 для любого $\mathcal{A} \hm\in \mathcal{B}(\mathbb{R}^M)$ верно тождество
\begin{multline*}
 \mathbf{P}\left\{\omega: \; X_{t_{r+1}}(\omega)=e_{j}, \; N_{r+1} = h,\right.\\ 
\left. \tau_{r+1}(X(\omega)) \in \mathcal{A}\;|\;X_{t_r}=e_k\right\} \equiv
  \rho^{k,j,h}_{r+1}(\mathcal{A}).
\end{multline*}
В качестве аппроксимации оценок можно использовать  
 $\overline{X}_{t_{r+1}}^n \ebd 
 \mathrm{col}\,(\overline{X}_{t_{r+1}}^{n,1},\ldots,\overline{X}_{t_{r+1}}^{n,N})$, 
 полученные из~(\ref{eq:filt_1_1}) путем урезания сумм ряда в~числителе и~знаменателе:
 
 \noindent
 \begin{multline}
 \overline{X}_{t_{r+1}}^{n,j} = 
 \left(
 \int\limits_{\mathcal{D}} 
 \mathcal{N}\left(Y_{r+1},f u,\sum\limits_{p=1}^N u^p g_pg_p^{\top}\right)\times{}\right.\\[-1pt]
\left.{}\times \sum\limits_{h=0}^{n}\sum\limits_{k=1}^N \overline{X}_{t_r}^k
 \rho^{k,j,h}_{r+1}(du)
 \right)\Bigg/ \\[-1pt]
 \left(
 \int\limits_{\mathcal{D}} 
 \mathcal{N}\left(Y_{r+1},f v,\sum\limits_{q=1}^N v^q g_qg_q^{\top}\right)\times{}\right.\\[-1pt]
\left. {}\times
 \sum\limits_{m=0}^{n}
 \sum\limits_{i,\ell=1}^N \overline{X}_{t_r}^i
 \rho^{i,\ell,m}_{r+1}(dv)
  \right)\,, \enskip
   j = \overline{1,N}.
  \label{eq:filt_2}
 \end{multline}
 Ниже по формуле полной вероятности получены интегралы из~(\ref{eq:filt_2}) для 
 $h\hm=0,1,2$:
 
\vspace*{-3pt}

 \noindent
  \begin{multline*}
 \int\limits_{\mathcal{D}}  \mathcal{N}
 \left(Y_{r+1},f u,\sum\limits_{p=1}^N u^p g_pg_p^{\top}\right) 
 \rho^{k,j,0}_{r+1}(du) = {}\\[-1pt]
 {}=
 \delta_{kj}\mathcal{N}\left(Y_{r+1},\Delta f^j,\Delta g_jg_j^{\top}\right)
 e^{\lambda_{jj}\Delta};
 %\label{eq:h0}
\\[-1pt]
 \int\limits_{\mathcal{D}}  \mathcal{N}\left(
 Y_{r+1},f u,\sum\limits_{p=1}^N u^p g_pg_p^{\top}\right) 
 \rho^{k,j,1}_{r+1}(du) ={} 
 \end{multline*}
 
 \noindent
 \begin{multline}
 \hspace*{-6.7pt}{}=\left(1-\delta_{kj}\right)\lambda_{kj}e^{\lambda_{jj}\Delta}
\! \int\limits_0^{\Delta}\!
 e^{(\lambda_{kk}-\lambda_{jj})u^k}
 \mathcal{N}\left(Y_{r+1},u^kf^k +{}\right.\hspace*{-0.28818pt}\\[-1pt]
\hspace*{-3mm}\left. {}+ \left(\Delta - u^k\right)f^j, u^k g_kg_k^{\top}+
 \left(\Delta-u^k\right)g_jg_j^{\top}\right)\,du^k;
 \label{eq:h1}
 \end{multline}
 
 \vspace*{-12pt}
 
 \noindent
 \begin{multline}
 \int\limits_D \mathcal{N}\left( 
Y_{r+1},f u,\sum\limits_{p=1}^N u^p g_pg_p^{\top}\right)du ={}\\[-1pt]
{}=
\sum\limits_{\substack{{\ell:\ell \neq k,}\\ {\ell \neq j}}}
 \lambda_{k\ell}\lambda_{\ell j} e^{\lambda_{jj}\Delta}\times {}\\[-1pt] 
 {}\times
 \int\limits_0^{\Delta} \int\limits_0^{\Delta-u^k} \!
e^{(\lambda_{kk}-\lambda_{\ell\ell})u^k+(\lambda_{\ell\ell}-
 \lambda_{jj})u^{\ell}}\times{} \\[-1pt] 
{}  \times
 \mathcal{N}\left(Y_{r+1},u^k f^k+u^{\ell}f^{\ell}+\left(
 \Delta-u^k-u^{\ell} \right)f^j,\right.\\[-1pt]
 \hspace*{-1mm}\left.
 u^k g_kg_k^{\top}+u^{\ell}g_{\ell}g_{\ell}^{\top}+\left(
 \Delta-u^k-u^{\ell} \right)
 g_jg_j^{\top}
 \right) du^{\ell}du^{k}, \!\!
  \label{eq:h2}
 \end{multline} 
 
\vspace*{-2pt}
 
 \noindent
  где  $\delta_{ij}$~--- символ Кронекера. Интегралы для $h\hm>2$ также могут 
  быть получены в~явном виде, однако их сложность резко возрастает.
 

   Так как система~(\ref{eq:obsys_1}) является автономной, то в~качестве локальной 
   характеристики бли\-зости~$\{\overline{X}_{t_r}\}$ 
   к~$\{\widehat{X}_{t_r}\}$ может быть выбрана величина
   
\noindent
 \begin{multline*}
 \overline{\sigma}(\pi) \ebd {\sf E}\left\{
 \|\widehat{X}_{t_{1}}(\pi, Y_{1}) - \overline{X}_{t_{1}}
 \left(\pi,Y_{1}\right)\|_{1}\right\} = {}\\
 {}=
 \sum\limits_{j=1}^N{\sf E}
 \left\{\left\vert \widehat{X}^j_{t_{1}}\left(\pi, Y_{1}\right) - \overline{X}^{n,j}_{t_{1}}
 \left(\pi,Y_{1}\right)\right\vert\right\}.
 %\label{eq:prec_1}
 \end{multline*}
 При этом начальное распределение $\pi \hm\in \mathcal{D}_1 \ebd $\linebreak $\ebd
 \{\mathrm{col}\,(\pi^1,\ldots,\pi^N):\;\pi^j > 0$, 
 $\sum\nolimits_{j=1}^N\pi^j\hm=1\}$ является начальным условием применения 
 одного шага рекурсии~(\ref{eq:filt_1}) или~(\ref{eq:filt_2}) для вычисления 
 оценки~$\widehat{X}_{t_{1}}$
   или~$\overline{X}_{t_{1}}$ соответственно. Фактически, 
 характеристика~$\overline{\sigma}(\pi)$ определяет, насколько сильно 
 рекурсивные схемы~(\ref{eq:filt_1}) и~(\ref{eq:filt_2}) разойдутся за 
 один шаг, стартуя из общей точки~$\pi$.
 
 Рекуррентные схемы~(\ref{eq:filt_1}) и~(\ref{eq:filt_2}), примененные~$r$~раз, 
 позволяют вычислить оценки~$\widehat{X}_{t_r}$ и~$\overline{X}_{t_r}$ 
 в~точке~$t_r$. В~качестве характеристики точности глобальной аппроксимации в~этом 
 случае естественно рассмотреть величину
 
 \vspace*{-2pt}
 
 \noindent
 \begin{equation*}
 \overline{\Sigma}_{t_r}(\pi) \ebd {\sf E}
 \left\{\|\widehat{X}_{t_{r}} - \overline{X}_{t_{r}}\|_{1}\right\} = 
 \!\sum\limits_{j=1}^N\!{\sf E}
 \left\{\left\vert \widehat{X}^j_{t_{r}} - 
 \overline{X}^{n,j}_{t_{r}}\right\vert \right\}.
% \label{eq:prec_2}
 \end{equation*}
 
 Следующее утверждение определяет оценки локальной и~глобальной 
 точности схемы аппроксимации~(\ref{eq:filt_2}).
 
 %\smallskip
 
 \noindent
 \textbf{Теорема~1.}\
\textit{Выполняются неравенства} 

%\vspace*{-2pt}

\noindent
 \begin{equation}
 \sup_{\pi \in \mathcal{D}_1} \overline{\sigma}(\pi) 
 \leqslant 2 \fr{(\overline{\lambda}\Delta)^{n+1}}{(n+1)!}\,;
 \label{eq:prec_loc}
\end{equation}

\noindent
\begin{align}
  \sup\limits_{\pi \in \mathcal{D}_1} \overline{\Sigma}_{t_r}(\pi)
   &\leqslant 2r \fr{(\overline{\lambda}\Delta)^{n+1}}{(n+1)!} +{}\notag\\[-0.5pt]
   &\hspace*{-20mm}{}+
  r(r-1)\left(
  \fr{(\overline{\lambda}\Delta)^{n+1}}{(n+1)!}
  \right)^2
  \left(
  1-\fr{(\overline{\lambda}\Delta)^{n+1}}{(n+1)!}
  \right)^{r-2},
 \label{eq:prec_glob}
 \end{align}
 
 \vspace*{-2pt}
 
 \noindent
 \textit{где} $\overline{\lambda} \ebd \max_{1 \leqslant j \leqslant N}|\lambda_{jj}|$.


%\smallskip

 Доказательство теоремы~1 приведено в~приложении.
 
 Данное утверждение представляет полезные оценки точности. Во-пер\-вых, 
 они являются равномерными по начальному распределению $\pi \hm\in \mathcal{D}_1$. 
 Во-вто\-рых, оценки носят универсальный, а~не асимптотический характер. Это 
 существенно в~практических задачах оценивания по дискретизованным 
 наблюдениям с~физическими или алгоритмическими ограничениями на шаг 
 по времени. Например, в~случае наблюдаемого процесса восстановления в~силу 
 центральной предельной теоремы для процессов восстановления~\cite{B_80} его
  приращения можно рассматривать как гауссовские случайные величины. 
  Однако данная аппроксимация обладает удовлетворительной точностью 
  только в~случае, когда шаг дискретизации по времени достаточно большой. 
 %
 В-третьих, неравенство~(\ref{eq:prec_glob}) позволяет получить порядок 
 аппроксимации при $\Delta \hm\to 0$. Зафиксируем момент времени $t\hm=T$ и~рассмотрим 
 характеристику $\sup\nolimits_{\pi \in \mathcal{D}_1} 
 \overline{\Sigma}_{T}(\pi)$ при $r\hm={T}/{\Delta}$ и~$\Delta \hm\to 0$. 
 Как только~$\Delta$ становится настолько мало, что 
 $\max\left({(\overline{\lambda}\Delta)^{n+1}}/{(n+1)!}, 
 \Delta ({T\lambda^{n+1}}/{(n+1)!})\right)\hm< 1$, из~(\ref{eq:prec_glob}) 
 следует неравенство
  %\begin{equation}
  $\sup\nolimits_{\pi \in \mathcal{D}_1} \overline{\Sigma}_{T}(\pi) 
  \hm\leqslant  ({3\overline{\lambda}^{n+1}}/{(n+1)!}) T\Delta^n.$
 %\label{eq:prec_asympt}
 %\end{equation}
 Это значит, что с~ростом времени~$T$ 
 ошибка аппроксимации копится пропорционально~$T$ и~при этом порядок точности 
 по~$\Delta$ равен~$n$.
 
 %\vspace*{-7pt}
 
  \section{Заключение}
  
  \vspace*{-4pt}
 
  В работе решена задача оценивания состояния однородного МСП по 
  дискретизованным наблюдениям. Получено аналитическое решение и~его 
  чис\-лен\-ные аппроксимации. Локальные и~глобальные показатели точ\-ности этих 
  приближений в~статье так\-же пред\-став\-ле\-ны. Примечательно, что  част\-ный случай 
  аппроксимаций~(\ref{eq:filt_2}) при $n\hm=0$ и~$\Lambda\hm=0$ был ранее 
  пред\-став\-лен в~\cite{B_17_1,B_17_2} для решения задачи байесовской классификации 
  случайного вектора по непрерывным наблюдениям с~мультипликативными шумами. 
 % 
Алгоритм оптимальной фильт\-ра\-ции и~его субоптимальные версии могут 
рас\-смат\-ри\-вать\-ся в~качестве основы чис\-лен\-ной реализации обобщения фильт\-ра 
Вонэма для сис\-тем с~мультипликативными шумами в~наблюдениях. 
Однако для их непосредственного использования необходимо решить 
следующие проб\-ле\-мы. Во-пер\-вых, в~(\ref{eq:h1}) и~(\ref{eq:h2}) присутствуют
 многомерные интегралы. Следует выяснить, какую результирующую погрешность 
 будут вносить ошибки их вы\-чис\-ле\-ния. Во-вто\-рых, представляется интересным 
 определить характеристики точ\-ности оптимальной фильт\-ра\-ции по дискретизованным 
 наблюдениям по отношению к~оптимальной фильт\-ра\-ции по непрерывным наблюдениям: 
 каков порядок точ\-ности по шагу временной дискретизации~$\Delta$? Для случая 
 вы\-чис\-ле\-ния классического фильт\-ра Вонэма с~по\-мощью алгоритма Эй\-ле\-ра--Ма\-ру\-ямы 
 подобный результат известен: порядок глобальной ошибки равен~${1}/{2}$. 
 Перечисленные задачи являются предметом дальнейших исследований.
 
 
  \vspace*{-10pt}
 
{\small
\subsection*{\raggedleft Приложение} 

\vspace*{-2pt}


\noindent
Д\,о\,к\,а\,з\,а\,т\,е\,л\,ь\,с\,т\,в\,о\ \ теоремы~1.\ \ Введем следующие 
обозначения для случайных величин и~мат\-риц, составленных из них:
\begin{align*}
\xi^{ji}(\ell)&\ebd 
\sum\limits_{h=0}^n \int\limits_{\mathcal{D}} 
 \mathcal{N}\left(Y_{\ell},f u,\sum\limits_{p=1}^N u^p g_pg_p^{\top}\right)
 \rho^{j,i,h}_{1}(du)\,; \\
  \theta^{ji}(\ell)&\ebd 
\sum\limits_{h=n+1}^{\infty} \int\limits_{\mathcal{D}} 
 \mathcal{N}\left(Y_{\ell},f u,\sum\limits_{p=1}^N u^p g_pg_p^{\top}\right)
 \rho^{j,i,h}_{1}(du)\,;
\\
 \xi(\ell)&\ebd \|\xi^{ji}(\ell)\|_{j,i=\overline{1,N}}\,,\quad 
 \Xi(r) \ebd \xi(r) \xi(r-1)\cdots \xi(1)\,;
 \\
 \theta(\ell)&\ebd \|\theta^{ji}(\ell)\|_{j,i=\overline{1,N}}\,, \quad 
 \Theta(r) \ebd \theta(r) \theta(r-1)\cdots \theta(1)\,.
%\label{eq:not_1}
\end{align*}
 
 Рекуррентные формулы~(\ref{eq:filt_1}) и~(\ref{eq:filt_2}) можно записать в~явной 
 форме
 
 
\noindent
\begin{align*}
 \widehat{X}_{t_r}& = \left( \mathbf{1}\left(\Xi(r) + 
 \Theta(r)\right)\pi\right)^{-1} \left(\Xi(r) + \Theta(r)\right)\pi\,;
\\
 \overline{X}_{t_r} &= \left( \mathbf{1}\Xi(r)\pi\right)^{-1} \Xi(r) \pi,
\end{align*}

\vspace*{-2pt}

\noindent
где $\mathbf{1} \ebd (1,\ldots,1)$~--- век\-тор-стро\-ка 
подходящей раз\-мер\-ности, составленная из единиц.

%Далее для краткости записи зависимость от~$r$ в~обозначениях~$\Xi(r)$ 
%и~$\Theta(r)$ будет опущена. 
Верна следующая цепочка неравенств:

 \vspace*{-3pt}

\noindent
\begin{multline}
\overline{\Sigma}_{t_r}(\pi)=%
%\me{}{\left\| 
%\widehat{X}_{t_r}(\pi, Y_1,\ldots,Y_r) - \overline{X}_{t_r}(\pi, Y_1,\ldots,Y_r)
%\right\|_1} =\\=
{\sf E}\left\{\left\| 
\fr{1}{\mathbf{1}\left(\Xi(r) + \Theta(r)\right)\pi} \left(\Xi(r) +{}\right.\right.\right.\\[-1pt]
\left.\left.\left.{}+ \Theta(r)\right)\pi
- \fr{1}{\mathbf{1}\Xi(r)\pi}\,\Xi(r) \pi
\right\|_1\right\} ={} \\[-1pt]
{}=
{\sf E}\left\{\fr{1}{\mathbf{1}\left(\Xi(r) + \Theta(r)\right)\pi \mathbf{1}\Xi(r)\pi}
\left\|
 \mathbf{1}\Xi(r) \pi \Theta(r)\pi -{}\right.\right.\\[-1pt]
\left.\left. {}- \mathbf{1}\Theta(r)\pi \Xi(r) \pi
 \right\|_1
 \vphantom{\fr{1}{\mathbf{1}\left(\Xi(r) + \Theta(r)\right)\pi \mathbf{1}\Xi(r)\pi}}
\right\} \leqslant {}\\[-1pt]
{}\leqslant 
{\sf E}\left\{\fr{1}{\mathbf{1}\left(\Xi(r) + \Theta(r)\right)\pi \mathbf{1}\Xi(r)\pi}
\left(
\mathbf{1}\Xi(r)\pi \| \Theta(r)\pi \|_1 +{}\right.\right.\\[-1pt]
\left.\left.{}+ \mathbf{1}\Theta(r)\pi 
\|
\Xi(r) \pi
\|_1
\right)
 \vphantom{\fr{1}{\mathbf{1}\left(\Xi(r) + \Theta(r)\right)\pi \mathbf{1}\Xi(r)\pi}}
\right\} ={}\\[-1pt]
{}=
2\,{\sf E}\left\{\fr{1}{\mathbf{1}\left(\Xi(r) + \Theta(r)\right)\pi}\mathbf{1}\Theta(r)\pi 
\right\}.
\label{eq:ineq_1}
\end{multline}

 
 \noindent
 Рассмотрим случайные события $a_{\ell} \ebd \{\omega \in \Omega: 
 N_{\ell}(\omega) \hm\leqslant n\}$, $\ell \hm= \overline{1,r}$, и~$A_r \ebd \{
 \omega\hm \in \Omega: \max_{1 \leqslant {\ell} \leqslant r}N_{\ell}(\omega) 
 \hm\leqslant n
 \}\hm=\prod\nolimits_{\ell=1}^r a_{\ell}$ и~оценку 
 $
 \widetilde{X}_{t_r}(\pi, Y_1,\ldots,Y_r)\ebd$\linebreak $\ebd
 {\sf E}\left\{X_{t_r}(\omega)\mathbf{I}_{A_r}(\omega)|\mathcal{O}_r\right\}.
 $
 Используя введенные выше обозначе\-ния и~абстрактный вариант формулы Байеса, 
 получаем, что
 
 \noindent
\begin{align}
\widetilde{X}_{t_r}& = \fr{1}{{\mathbf{1}\left(\Xi(r) + 
 \Theta(r)\right)\pi}}\,\Xi(r)\pi\,;\notag
 \\
\widehat{X}_{t_r} - \widetilde{X}_{t_r} &=
{\sf E}\left\{X_{t_r}(\omega)\mathbf{I}_{\overline{A}_r}(\omega)|\mathcal{O}_r\right\} ={}\notag\\[-1pt]
&\hspace*{17mm}{}= 
\fr{1}{\mathbf{1}\left(\Xi(r) + \Theta(r)\right)\pi}\Theta(r)\pi\,. 
\label{eq:eq_2}
 \end{align}
 Из (\ref{eq:ineq_1}) и~(\ref{eq:eq_2}) для $r\hm=1$ следует, что
 
 \vspace*{-4pt}
 
 \noindent
 \begin{multline}
 \overline{\sigma}(\pi) \leqslant 2\,{\sf E}
 \left\{\|{\sf E}\left\{X_{t_1}(\omega)\mathbf{I}_{\overline{a}_1}(\omega)|\mathcal{O}_1
 \right\}\|_1
 \right\} ={}\\[-1.5pt]
 {}=
 2\,{\sf E}\left\{\sum\limits_{n=1}^N {\sf E}
 \left\{X^n_{t_1}(\omega)\mathbf{I}_{\overline{a}_1}
 (\omega)|\mathcal{O}_1\right\}\right\} ={} \\[-2pt] 
 {}=
  2\,{\sf E}\left\{{\sf E}\left\{\mathbf{I}_{\overline{a}_1}(\omega)|\mathcal{O}_1
  \right\}\right\} =
   2 \mathbf{P}\left\{\overline{a}_1(\omega)\right\}.
\label{eq:ineq_3}
\end{multline}

 \vspace*{-2pt}
 
 \noindent
 Процесс $N^X_t$ общего числа скачков состояния~$X_t$ является считающим, и~его
  квадратическая характеристика равна 
  
\vspace*{-2pt}
  
  \noindent
 $$
 \langle N^X, N^X\rangle_t = - \int\limits_0^t \sum\limits_{n=1}^N \lambda_{nn} X_s^n\,ds\,,
 $$
 поэтому искомая вероятность ограничена сверху:
 $$ 
 \mathbf{P}\left\{\overline{a}_1(\omega)\right\} \leqslant 
 e^{-\overline{\lambda}\Delta}\sum\limits_{k=n+1}^{\infty} 
 \fr{(\overline{\lambda}\Delta)^{k}}{k!} <
 \fr{(\overline{\lambda}\Delta)^{n+1}}{(n+1)!}.
 $$
 
  \vspace*{-2pt}
  
  \noindent
 Из последнего неравенства и~(\ref{eq:ineq_3}) следует, что  для любого 
 начального распределения~$\pi$ выполняется неравенство $\overline{\sigma}(\pi)  
 \hm< 2({(\overline{\lambda}\Delta)^{n+1}}/{(n+1)!})$, т.\,е.\ 
 локальная оценка~(\ref{eq:prec_loc}) верна.
 
 С помощью марковского свойства пары $(X_t, N^X_t)$ и~последнего 
 неравенства можно оценить сверху вероятность 
 $\mathbf{P}\left\{\overline{A}_r(\omega)\right\}$:
 
  \vspace*{-2pt}
 
 \noindent
 \begin{multline*}
 \mathbf{P}\left\{\overline{A}_r(\omega)\right\} \leqslant 1 - \left(
 1- \fr{(\overline{\lambda}\Delta)^{n+1}}{(n+1)!}
 \right)^r \leqslant r \fr{(\overline{\lambda}\Delta)^{n+1}}{(n+1)!} + {}\\[-1pt]
 {}+\left|
 \sum\limits_{k=2}^r C_r^k \left(-\fr{(\overline{\lambda}\Delta)^{n+1}}{(n+1)!}
 \right)^k
 \right| \leqslant
 r \fr{(\overline{\lambda}\Delta)^{n+1}}{(n+1)!} +{}\\[-1pt]
 {}+\fr{r(r-1)}{2}
 \left(
 \fr{(\overline{\lambda}\Delta)^{n+1}}{(n+1)!}
 \right)^2
 \left(
 1-\fr{(\overline{\lambda}\Delta)^{n+1}}{(n+1)!}
 \right)^{r-2},
 \end{multline*} 
 из чего следует истинность глобальной оценки~(\ref{eq:prec_glob}).
Теорема~1 доказана.

}

%\vspace*{-12pt}

{\small\frenchspacing
 {%\baselineskip=10.8pt
 \addcontentsline{toc}{section}{References}
 \begin{thebibliography}{99}

\bibitem{Won_65}
\Au{Wonham W.} 
Some applications of stochastic differential equations to optimal
  nonlinear filtering~//
SIAM~J.~Control, 1965. Vol.~2. P.~347--369. 

\bibitem{KP_92}
\Au{Kloeden P., Platen E.} Numerical solution of stochastic
differential equations.~--- Berlin: Springer, 1992.~636~p.

\bibitem{YZL_04}
\Au{Yin G., Zhang Q., Liu Y.} 
Discrete-time approximation of Wonham filters~//
J.~Control Theory Applications, 2004. Iss.~2. P.~1--10.

\bibitem{PR_10}
\Au{Platen E., Rendek R.}
Quasi-exact approximation of hidden Markov chain filters~//
Communicat.~Stoch.~Analys., 2010. Vol.~4. Iss.~1. P.~129--142.

\bibitem{B_18}
\Au{Борисов А.} Фильтрация Вонэма по наблюдениям с~мультипликативными шумами~// 
Автоматика и~телемеханика, 2018.
№~1. C.~52--65. 
 
  \bibitem{BSh_85} %6
\Au{Бертсекас Д., Шрив С.} Стохастическое оптимальное управление. 
Случай дискретного времени~/ Пер. с~англ.~--- М.: Наука, 1985.~280~c.
(\Au{Betsekas~D.\,P., Shreve~S.\,E.} Stochastic optimal control:
The discrete-time case.~--- Orlando, FL, USA:
Academic Press Inc., 1978. 323~p.)

  \bibitem{ZhSh_95} %7
\Au{Жакод Ж., Ширяев А.} Предельные теоремы для случайных процессов,~I.~/
Пер. с~англ.~--- 
М.: Физматлит, 1995.~544~c.
(\Au{Jacod~J., Shiryaev~A.} Limit theorems for stochastic processes.~---
Berlin: Springer, 2003. 664~p.)

\bibitem{S_00}
\Au{Sericola B.} Occupation times in Markov processes~//
Commun. Stat. Stochastic Models, 2000. Vol.~16. Iss.~5. P.~479--510. 

  \bibitem{B_80}
\Au{Боровков А.} Асимптотические методы в~тео\-рии массового обслуживания.~--- 
М.: Физматлит, 1995.~384~c.

  \bibitem{B_17_1}
\Au{Борисов А.} Классификация по непрерывным наблюдениям с~мультипликативными шумами.~I. 
Формулы байесовской оценки~// Информатика и~её применения, 2017. Т.~11. Вып.~1. C.~11--19.
doi: 10.14357/19922264170102.

  \bibitem{B_17_2}
\Au{Борисов А.} Классификация по непрерывным наблюдениям с~мультипликативными 
шумами.~II. Алгоритм численной реализации оценки~// Информатика и~её 
применения, 2017. Т.~11. Вып.~2. C.~33--41.
doi: 10.14357/19922264170204.

 \end{thebibliography}

 }
 }

\end{multicols}

\vspace*{-4pt}

\hfill{\small\textit{Поступила в~редакцию 10.07.18}}

\vspace*{6pt}

%\pagebreak

%\newpage

%\vspace*{-28pt}

\hrule

\vspace*{2pt}

\hrule

%\vspace*{-2pt}

\def\tit{FILTERING OF~MARKOV JUMP PROCESSES\\ BY~DISCRETIZED OBSERVATIONS}

\def\titkol{Filtering of Markov jump processes by discretized observations}

\def\aut{A.\,V.~Borisov}

\def\autkol{A.\,V.~Borisov}

\titel{\tit}{\aut}{\autkol}{\titkol}

\vspace*{-11pt}


\noindent
Institute of Informatics Problems, Federal Research Center ``Computer Science 
and Control'' of the Russian Academy of Sciences, 44-2~Vavilov Str., Moscow 
119333, Russian Federation


\def\leftfootline{\small{\textbf{\thepage}
\hfill INFORMATIKA I EE PRIMENENIYA~--- INFORMATICS AND
APPLICATIONS\ \ \ 2018\ \ \ volume~12\ \ \ issue\ 3}
}%
 \def\rightfootline{\small{INFORMATIKA I EE PRIMENENIYA~---
INFORMATICS AND APPLICATIONS\ \ \ 2018\ \ \ volume~12\ \ \ issue\ 3
\hfill \textbf{\thepage}}}

\vspace*{6pt}



\Abste{The article is devoted to a~solution of the optimal filtering problem 
of a~homogenous Markov
jump process state. The available observations represent 
time increments of the integral transformations of the Markov\linebreak\vspace*{-12pt}}

\Abstend{state corrupted by 
Wiener processes. The noise intensity is also state-dependent. At the instant of 
the consecutive
observation obtaining, the optimal estimate is calculated recursively 
as a~function of previous estimate and the new observation, meanwhile between 
observations the filtering estimate is a simple forecast by virtue of the Kolmogorov 
differential system. The recursion is rather expensive because of  need to calculate 
the integrals, which are the location-scale mixtures of Gaussians. The mixing 
distributions represent the occupation of the state in each of possible values 
during the mid-observation intervals. The paper contains numerically cheaper 
approximations, based on the restriction of the state transitions number between 
the observations. Both the local and global characteristics of approximation 
accuracy are obtained as functions of the dynamics parameters, mid-observation 
interval length, and upper bound of transitions number.}

\KWE{Markov jump process; optimal filtering; multiplicative observation noises; 
stochastic differential equation; numerical approximation}




\DOI{10.14357/19922264180316}

%\vspace*{-14pt}

\Ack
\noindent
The work was supported in part by the Russian Foundation
for Basic Research (Project No.\,16-07-00677).



%\vspace*{6pt}

  \begin{multicols}{2}

\renewcommand{\bibname}{\protect\rmfamily References}
%\renewcommand{\bibname}{\large\protect\rm References}

{\small\frenchspacing
 {%\baselineskip=10.8pt
 \addcontentsline{toc}{section}{References}
 \begin{thebibliography}{99}
\bibitem{Won_65-1}
\Aue{Wonham, W.} 1965.
Some applications of stochastic differential equations to optimal
  nonlinear filtering.
\textit{SIAM~J.~Control} 2:347--369. 

\bibitem{KP_92-1}
\Aue{Kloeden,~P., and E.~Platen.} 1992. \textit{Numerical solution of stochastic
differential equations.} Berlin: Springer. 636~p.

\bibitem{YZL_04-1}
\Aue{Yin,~G., Q.~Zhang, and Y.~Liu.} 2004.
Discrete-time approximation of Wonham filters.
\textit{J.~Control Theory Applications} 2:1--10.

\bibitem{PR_10-1}
\Aue{Platen, E., and R.~Rendek.} 2010.
Quasi-exact approximation of hidden Markov chain filters.
\textit{Communicat. Stoch. Analys.} 4(1):129--142.

\bibitem{B_18-1}
\Aue{Borisov, A.} 2018. Wonham filtering by observations
with multiplicative noises. \textit{Automat.~Rem.~Contr.} 79(1):39--50.  
doi: 10.1134/ S0005117918010046.
 
  \bibitem{BSh_85-1}
\Aue{Bertsekas, D., and S.~Shreve.} 1996.
\textit{Stochastic optimal control: The discrete-time case}.
Nashua, NH: Athena Scientific. 330~p.
  
  \bibitem{ZhSh_95-1}
  \Aue{Jacod,~J., and A.~Shiryaev.} 2003.
\textit{Limit theorems for stochastic processes.}
Berlin: Springer. 664~p.

\bibitem{S_00-1}
\Aue{Sericola, B.}
2000. Occupation times in Markov processes.
\textit{Commun. Stat.} 16(5):479--510. 

  \bibitem{B_80-1}
\Aue{Borovkov, A.} 1984.
 \textit{Asymptotic methods in queueing theory}. 
 Hoboken, NJ: Wiley-Blackwell.~304~p.

  \bibitem{B_17_1-1}
  \Aue{Borisov, A.} 2017. 
  Klassifikatsiya po ne\-pre\-ryv\-nym nablyu\-de\-miyam s~mul'tiplikativnymi shumami. I. 
  Formuly bayesov\-skoy otsenki [Classification by continuous-time observations
in multiplicative noise. I.~Formulae for Bayesian 
estimate]. \textit{Informatika i~ee Primeneniya~--- Inform.~Appl.}
11(1):11--19. doi: 10.14357/19922264170102.

  \bibitem{B_17_2-1}
\Aue{Borisov, A.} 2017. Klassifikatsiya po nepreryvnym nablyudemiyam 
s~mul'tiplikativnymi summami. II.~Formuly bayesovskoy otsenki 
[Classification by continuous-time observations
in multiplicative noise. II.~Numerical algorithm].
\textit{Informatika i~ee Primeneniya~--- Inform.~Appl.}
11(2):33--41. doi: 10.14357/19922264170204.

\end{thebibliography}

 }
 }

\end{multicols}

\vspace*{-6pt}

\hfill{\small\textit{Received July 10, 2018}}

%\pagebreak

%\vspace*{-18pt}

\Contrl

\noindent
\textbf{Borisov Andrey V.} (b.\ 1965)~--- 
Doctor of Science in physics and mathematics, principal scientist, Institute of
Informatics Problems, Federal Research Center ``Computer Science and Control''
 of the Russian Academy of
Sciences, 44-2 Vavilov Str., Moscow 119333, Russian Federation; 
\mbox{aborisov@frccsc.ru}
\label{end\stat}

\renewcommand{\bibname}{\protect\rm Литература}        %4 
\newcommand{\mujk}{\mu_{j,k}}
\newcommand{\psijk}{\psi_{j,k}}
\newcommand{\sumk}{\sum\limits_{k=0}^{2^j-1}}
\newcommand{\betajk}{\beta_{j,k}}



\def\stat{kudr+shest}

\def\tit{МИНИМИЗАЦИЯ ОШИБОК ВЫЧИСЛЕНИЯ ВЕЙВЛЕТ-КОЭФФИЦИЕНТОВ\\
ПРИ~РЕШЕНИИ ОБРАТНЫХ ЗАДАЧ$^*$}

\def\titkol{Минимизация ошибок вычисления вейвлет-коэффициентов
при~решении обратных задач}

\def\aut{А.\,А.~Кудрявцев$^1$, О.\,В.~Шестаков$^2$}

\def\autkol{А.\,А.~Кудрявцев, О.\,В.~Шестаков}

\titel{\tit}{\aut}{\autkol}{\titkol}

\index{Кудрявцев А.\,А.}
\index{Шестаков О.\,В.}
\index{Kudryavtsev A.\,A.}
\index{Shestakov O.\,V.}




{\renewcommand{\thefootnote}{\fnsymbol{footnote}} \footnotetext[1]
{Работа выполнена при частичной финансовой поддержке РФФИ (проект 18-07-00252).}}


\renewcommand{\thefootnote}{\arabic{footnote}}
\footnotetext[1]{Московский государственный университет им.~М.\,В.~Ломоносова, 
факультет вычислительной математики и~кибернетики, \mbox{nubigena@mail.ru}}
\footnotetext[2]{Московский государственный университет им.~М.\,В.~Ломоносова, факультет 
 вычислительной математики и~кибернетики; Институт проб\-лем информатики 
 Федерального исследовательского центра <<Информатика и~управ\-ле\-ние>> 
 Российской академии наук, \mbox{oshestakov@cs.msu.su}}

%\vspace*{-6pt}


\Abst{Статистические обратные задачи возникают во многих прикладных областях, 
вклю\-чая медицину, астрономию, био\-ло\-гию, физику плазмы, химию и~т.\,п. 
При этом в~наблюдаемых данных всегда присутствуют по\-греш\-ности, 
связанные с~несовершенством оборудования, фоновыми шумами, дискретизацией 
данных и~др. Для уменьшения этих погрешностей необходимо 
применять специальные методы регуляризации, поз\-во\-ля\-ющие строить при\-бли\-жен\-ные 
устойчивые решения обратных задач. Классические методы регуляризации базируются 
на использовании оконного сингулярного разложения. Однако при таком подходе 
учитывается лишь вид оператора, участвующего в~формировании наблюдаемых 
данных, и~никак не учитываются свойства самого объекта наблюдения. Для линейных 
однородных операторов эта проб\-ле\-ма решается с~по\-мощью специальных методов 
вейв\-лет-ана\-ли\-за, поз\-во\-ля\-ющих адаптироваться одновременно к~виду 
оператора и~локальным особенностям функции, описывающей объект. 
В~данной работе рас\-смат\-ри\-ва\-ет\-ся задача обращения линейного однородного оператора 
при наличии шума в~наблюдаемых данных с~по\-мощью пороговой обработки коэффициентов 
вейв\-лет-раз\-ло\-же\-ния наблюдаемой функции. Вычисляются асимптотически 
оптимальные пороги и~порядки функции потерь при минимизации усред\-нен\-ной 
ве\-ро\-ят\-ности ошибки вычисления вейв\-лет-ко\-эф\-фи\-ци\-ен\-тов.}

\KW{вейвлеты; пороговая обработка; линейный однородный оператор; функция потерь}

\DOI{10.14357/19922264180203}
  
\vspace*{6pt}


\vskip 10pt plus 9pt minus 6pt

\thispagestyle{headings}

\begin{multicols}{2}

\label{st\stat}

\section{Введение}

Математические модели, лежащие в~основе многих прикладных задач анализа и~обработки 
сигналов, предполагают решение задачи обращения некоторого линейного оператора. 
Например, в~медицинских исследованиях бывает необходимо с~по\-мощью неинвазивных 
методов получить изоб\-ра\-же\-ние структуры ка\-ко\-го-ли\-бо внут\-рен\-не\-го органа. 
В~этом случае возникает задача реконструкции изоб\-ра\-же\-ния путем обращения 
проекционного оператора. Как правило, эта задача является некорректно 
поставленной, и~для ее решения требуется использовать методы регуляризации, 
поскольку в~реальных наблюдениях всегда присутствует шум. Классические методы, 
основанные на сингулярном разложении в~сочетании с~линейной регуляризацией, 
адаптируются к~виду оператора, но не принимают в~расчет свойства функции сигнала, 
которую необходимо оценить. При рассмотрении моделей с~линейными однородными 
операторами с~этими недостатками позволяют справиться методы вейв\-лет-раз\-ло\-же\-ния, 
предложенные в~работах~\cite{Abr98, Don95}. 

В~данной работе рас\-смат\-ри\-ва\-ет\-ся 
метод нелинейной регуляризации при обращении линейных однородных операторов, 
основанный на пороговой обработке, и~вы\-чис\-ля\-ют\-ся па\-ра\-мет\-ры этого метода, 
асимптотически оптимальные в~смыс\-ле функции потерь, основанной на вероятностях 
ошибок вы\-чис\-ле\-ния коэффициентов вейв\-лет-раз\-ло\-же\-ния.

\section{Метод обращения линейных однородных операторов}

Линейный оператор~$K$ называется однородным с~показателем $\alpha\hm>0$, если
$$
K\left[f\left(a\left(x-x_0\right)\right)\right]=a^{-\alpha}(Kf)\left[a\left(x-x_0\right)\right]
$$
для любого $x_0\hm\in\r$ и~любого $a\hm>0$.

Для построения оценки функции сигнала~$f$ в~данной работе рас\-смат\-ри\-ва\-ет\-ся метод 
вейг\-лет-вейв\-лет-раз\-ло\-же\-ния (vaguelette-wavelet decomposition). Функция~$Kf$ 
пред\-став\-ля\-ет\-ся в~виде ряда
\begin{equation*}
%\label{Kf_decomp}
Kf=\sum\limits_{j,k\in {\mathbb {Z}}}\langle Kf,\psi_{j,k}\rangle\psi_{j,k} 
\end{equation*}
по некоторому вейвлет-ба\-зи\-су~$\{\psi_{j,k}\}$. 
Здесь $\psi_{j,k}(x)\hm=2^{j/2}\psi(2^jx-k)$; $\psi(x)$~--- 
заданная материнская вейв\-лет-функ\-ция. Обозначим $\beta_{j,k}\hm=
\norm{K^{-1}\psijk}$. Тогда
в~силу од\-но\-род\-ности оператора $\beta_{j,k}\hm=\beta_{0,0}2^{\alpha j}$~\cite{Abr98}. 
Функция~$f$ пред\-став\-ля\-ет\-ся в~виде ряда:
\begin{equation}
\label{f_decomp}
f=\sum\limits_{j,k\in {\mathbb {Z}}}\beta_{j,k}\langle Kf,\psi_{j,k}\rangle u_{j,k}\,,
\end{equation}
где $u_{j,k}=K^{-1}\psi_{j,k}/\beta_{j,k}$. Функции~$u_{j,k}$ называются 
<<вейглетами>>. По\-сле\-до\-ва\-тель\-ность~$\{u_{j,k}\}$ не является ортонормированной,
однако если выполнены некоторые условия глад\-кости на~$K^*\psi$ и~$K^{-1}\psi$, 
то по\-сле\-до\-ва\-тель\-ность~$\{u_{j,k}\}$ образует
устойчивый базис в~$L^2(\r)$~\cite{Lee97}. Формула~\eqref{f_decomp} 
и~пред\-став\-ля\-ет собой метод обращения, называемый вейг\-лет-вейв\-лет-раз\-ло\-же\-нием.

\section{Модель данных}

Рассмотрим следующую модель данных:
$$
X_i=(Kf)_i+z_i\,,\enskip i=1,\ldots,2^J\,,
$$
где $X_i$~--- наблюдаемые данные; $K$~--- некоторый линейный однородный оператор; 
$f$~--- истинная  функция сигнала; $z_i$~--- случайные по\-греш\-ности
измерения, которые предполагаются независимыми и~име\-ющи\-ми одинаковое гауссово
распределение с~нулевым средним и~дисперсией~$\sigma^2$.

Дискретное вейвлет-пре\-об\-ра\-зо\-ва\-ние пред\-став\-ля\-ет 
собой умножение вектора значений функции~$Kf$ на ортогональную матрицу~$W$, 
опре\-де\-ля\-емую вейв\-лет-функ\-ци\-ей~$\psi$~\cite{Mal99}. 
При этом дискретные вейв\-лет-ко\-эф\-фи\-ци\-ен\-ты 
приближенно рав\-ны~$2^{J/2}\langle Kf,\psi_{jk}\rangle$. Это приближение 
тем точнее, чем больше~$J$. Таким образом, для дискретных вейв\-лет-ко\-эф\-фи\-ци\-ен\-тов 
разложения~$Kf$ принимается сле\-ду\-ющая модель:
$$
Y_{j,k}=\mujk+\zeta_{j,k}\,, \enskip j=0,\ldots,J-1\,,\ k=0,\ldots,2^j-1\,,
$$
где  $\mujk=2^{J/2}\langle Kf,\psi_{j,k}\rangle$, а~случайные величины~$\zeta_{j,k}$ 
в~силу ор\-то\-го\-наль\-ности~$W$ также независимы и~име\-ют 
нормальное распределение с~нулевым средним и~дисперсией~$\sigma^2$.

\section{Функция потерь для~метода пороговой обработки}

Одним из самых популярных методов подавления шума является пороговая обработка, 
смысл которой заключается в~обнулении коэффициентов, чьи абсолютные значения 
не превышают заданного порога.

Через $\hat{Y}_{j,k}$ обозначим оценку вейв\-лет-ко\-эф\-фи\-ци\-ен\-та, которая 
получается 
с~по\-мощью пороговой функции~$\rho_T(x)$ с~порогом~$T$: 
$\hat{Y}_{j,k}\hm=\rho_T(Y_{j,k})$. В~данной работе рас\-смат\-ри\-ва\-ют\-ся функции 
жесткой пороговой обработки $\rho_T^{(h)}(x)\hm=x \mathbf{1} (|x|\hm>T)$ 
и~мягкой пороговой обработки $\rho_T^{(s)}(x)\hm=\mathrm{sign}\,(x)(|x|\hm-T)_+$.

Рассмотрим функцию потерь, основанную на вероятностях ошибок 
вы\-чис\-ле\-ния вейв\-лет-ко\-эф\-фи\-ци\-ен\-тов. Пусть $(\xi,\eta)$~--- 
двумерная случайная величина, не зависящая от всех~$\zeta_{j,k}$ и~име\-ющая 
дискретное равномерное распределение на множестве индексов 
$j\hm=0,\ldots,J-1$, $k\hm=0,\ldots,2^j-1$. Для заданного критического
 уровня  $\varepsilon\hm>0$ определим функцию потерь:
 
 \noindent
\begin{multline*}
r_J(f)={\sf E}{\sf P}\left(\left|\beta_{\xi,\eta}\hat{Y}_{\xi,\eta}-
\beta_{\xi,\eta}\mu_{\xi,\eta}\right|>\varepsilon\ \big|\ \xi,\eta\right)={}\\[-1pt]
{}=\fr{\sum\nolimits_{j=0}^{J-1}\sum\nolimits_{k=0}^{2^j-1} 
{\sf P}\left(\abs{\betajk\hat{Y}_{j,k}-\betajk\mujk}>\varepsilon\right)}{2^J}\,,
\end{multline*}
т.\,е.~$r_J(f)$ представляет собой усредненную вероятность того, что ошибка 
вычисления вейв\-лет-ко\-эф\-фи\-ци\-ен\-та превысит критический уровень~$\varepsilon$. 
Такое определение функции потерь является обобщением определения, 
предложенного в~работе~\cite{SMS14}. В~той же работе показано, что оценки, 
целью которых является минимизация функции потерь, основанной на вероятностях 
ошибок вычисления вейв\-лет-ко\-эф\-фи\-ци\-ен\-тов, дают срав\-ни\-мые, а~иногда и~лучшие 
результаты, чем оценки, минимизирующие сред\-не\-квад\-ра\-тич\-ный риск.

Целью данной работы является поиск асимптотически оптимального 
порога и~оценивание максимального порядка функции потерь~$r_J$ пороговой 
обработки наблюдаемого сигнала~$Kf$ в~классе регулярных по Липшицу 
функций $\mathrm{Lip}(\gamma,L)$, где $\gamma\hm>0$~--- показатель; 
$L\hm>0$~--- константа Липшица~\cite{Mal99}:
\noindent
\begin{equation}
\label{Risk_Definition}
R_J=\sup\limits_{Kf\in\mathrm{Lip}(\gamma,L)}{r_J(f)}\,,
\end{equation}
т.\,е.\ поиск порога, асимптотически оптимального в~минимаксном смысле. 
Подробное исследование поведения асимптотически оптимального порога 
для среднеквадратичного рис\-ка можно найти в~работах~\cite{DJ98, Jan01}. 
Также в~\cite{DJ95} был предложен метод поиска адаптивного оптимального порога,
 с~по\-мощью ко-\linebreak\vspace*{-12pt}
 
 \pagebreak
 
 \noindent
 торого можно оценить среднеквадратичный риск пороговой
  обработки конкретной функции. Этот метод основан на построении несмещенной 
  оценки риска, статистические свойства которой подробно исследованы 
  в~работах~\cite{MSH10, SH12}. Для задачи оценивания функции сигнала по 
  прямым зашумленным наблюдениям асимптотически оптимальные значения порогов, 
  минимизирующие функцию потерь, основанную на вероятностях ошибок, в~том 
  числе для моделей с~негауссовым распределением шума, найдены 
  в~работах~\cite{KS16-1, KS16-2}.

Заметим, что <<разумный>> порог должен воз\-рас\-тать по~$J$ со ско\-ростью, 
не выше чем $C_T\sqrt{\ln2^J}$, где~$C_T$~--- 
положительная константа, зависящая от оператора~$K$ 
(подробнее см.~\cite{Abr98, Jan01}).

Далее символом $\asymp$ обозначается порядок рассматриваемой величины по~$J$, 
т.\,е.\ $a_J\asymp b_J$, если начиная с~некоторого~$J$ выполнено 
$C_1  b_J\hm\leq a_J\hm\leq  C_2 b_J$ для некоторых положительных констант~$C_1$ 
и~$C_2$.

Везде далее дополнительно будем предполагать, что вейв\-лет-функ\-ция имеет~$M$~нулевых 
моментов ($M\hm\geq\gamma$), $M$~раз непрерывно дифференцируема и~достаточно быст\-ро 
убывает на бес\-ко\-неч\-ности вместе со своими производными~\cite{Mal99}.

\vspace*{-6pt}

\section{Жесткая пороговая обработка}

Пусть $\hat{Y}_{j,k}=\rho_T^{(h)}(Y_{j,k})$. Пусть функция $g(J)\hm>0$ сколь 
угодно медленно убывает по~$J$ к~нулю.

\vspace{2pt}

\noindent
\textbf{Теорема~1.}\ \textit{Пусть $\gamma>\alpha$. При выборе асимптотически 
оптимального порога для жесткой пороговой обработки функция 
потерь}~(\ref{Risk_Definition}) \textit{начиная с~некоторого~$J$ 
удовле\-тво\-ря\-ет неравенствам}:

\vspace*{-2pt}

\noindent
\begin{multline*}
C_1^{(h)}\cdot2^{-({2\gamma-2\alpha})J/({2\gamma-2\alpha+1})}
g(J)\leq R_J\leq{}\\[-1pt]
{}\leq
 C_2^{(h)}\cdot2^{-(({2\gamma-2\alpha})/({2\gamma-2\alpha+1}))J}
 J^{1/(2\gamma-2\alpha+1)}\,,
 \end{multline*}
 
 \vspace*{-4pt}
 
 \noindent
\textit{где $C_1^{(h)}$ и~$C_2^{(h)}$~--- некоторые положительные константы.
Для асимптотически оптимального зна\-чения порога, ми\-ни\-ми\-зи\-ру\-юще\-го 
порядок функции\linebreak потерь}~(\ref{Risk_Definition}) \textit{при жест\-кой пороговой обработке, 
справедливо неравенство}:
$$
T_*^{(h)}-T_2^{(h)}\leq T\leq T_*^{(h)}-T_1^{(h)},
$$ 
\textit{где}
%\begin{equation}\label{T_Main_Hard}

\noindent
\begin{align*}
T_*^{(h)}&=\sigma\sqrt{\fr{4\gamma-4\alpha}{2\gamma-2\alpha+1}\,\ln2^J};
\\[-2pt]
T_1^{(h)}&=\sigma\sqrt{\fr{2\gamma-2\alpha+1}{4\gamma-4\alpha}}
\, \fr{\ln\left((\ln 2^J)^{1/2}g(J)\right)}{\sqrt{\ln 2^J}}\,;
\\[-2pt]
T_2^{(h)}&=\sigma\sqrt{\fr{2\gamma-2\alpha+1}{4\gamma-4\alpha}}\,
\fr{\ln\left((\ln 2^J)^{1/2}J^{1/(2\gamma-2\alpha+1)}\right)}{\sqrt{\ln 2^J}}\,.
\end{align*}

\noindent
{Д\,о\,к\,а\,з\,а\,т\,е\,л\,ь\,с\,т\,в\,о\,.}\ \ 
При выполнении приведенных выше ограничений на вейв\-лет-функ\-цию 
справедливо неравенство~\cite{Mal99}:

\noindent
\begin{multline}
\label{Wavelet_CoeffDecacy}
\abs{\mujk}\leq\fr{A2^{J/2}}{2^{j\left(\gamma+1/2\right)}}\,, \\
j=0,\ldots,J-1\,,\ k=0,\ldots,2^j-1\,,
\end{multline}
где $A$~--- некоторая константа, зависящая от~$\gamma$ и~$L$ и~не зависящая от~$J$.


Поскольку $\gamma\hm>\alpha$, неравенство~(\ref{Wavelet_CoeffDecacy}) 
поз\-во\-ля\-ет разбить все множество индексов $\{0,\ldots,J-1\}$ 
на три класса в~за\-ви\-си\-мости от величины~$|\mujk|$. Пусть индексы~$j_1$ и~$j_2$ 
($j_1\hm< j_2$) таковы, что
\begin{equation*}
\left\vert \mujk\right\vert \leq 
\begin{cases}
2^{-\alpha j}(g(J))^{-(\gamma-\alpha+1/2)}\,,&\enskip
j_1\leq j\leq j_2-1\,;\\
 2^{-\alpha j}J^{-1/2}\,, &\enskip
 j_2\leq j \leq J-1\,.
 \end{cases}
 \end{equation*}
При этом в~силу~(\ref{Wavelet_CoeffDecacy})

\noindent
\begin{equation} %\label{Indices_Soft_j1}
\left.
\begin{array}{rl}
\hspace*{-2mm}j_1&=\fr{J}{2\gamma-2\alpha+1}+\log_2 g(J)+ \fr{\log_2 A}{\gamma-\alpha+1/2}\,;
\\[6pt]
\hspace*{-2mm}j_2&=\fr{J}{2\gamma-2\alpha+1}+\fr{\log_2 J}{2\gamma-2\alpha+1}+{}\\[6pt]
&\hspace*{41mm}{}+ \fr{\log_2 A}{\gamma-\alpha+1/2}\,.
\end{array}
\right\}
\label{Indices_Soft_j2}
\end{equation}

Разобьем сумму в~чис\-ли\-те\-ле~(\ref{Risk_Definition}) на три со\-став\-ля\-ющие:

\noindent
\begin{multline}
\sum\limits_{j=0}^{J-1}\sumk{\sf P}
\left(\left|\betajk\hat{Y}_{j,k}-\betajk\mujk\right|>\varepsilon\right)={}\\
{}=
\sum\limits_{j=0}^{j_1-1}\sumk{\sf P}
\left(\left|\betajk\hat{Y}_{j,k}-\betajk\mujk\right|>\varepsilon\right)+{}\\
{}+\sum\limits_{j=j_1}^{j_2-1}\sumk{\sf P}
\left(\left|\betajk\hat{Y}_{j,k}-\betajk\mujk\right|>\varepsilon\right)+{}\\
{}\sum\limits_{j=j_2}^{J-1}\sumk{\sf P}
\left(\left|\betajk\hat{Y}_{j,k}-\betajk\mujk\right|>\varepsilon\right)\equiv{}\\ 
{}\equiv S_1+S_2+S_3\,.
\label{Risk_Decomposition}
\end{multline}

Рассмотрим $S_3$. Заметим, что для любого $\varepsilon\hm>0$ найдется такое 
$J_0\hm=J_0(\varepsilon)$, что 
$2^{-\alpha j}J^{-1/2}\hm\leq\varepsilon/\betajk$ и~$\varepsilon/\betajk\hm\leq cT$, 
$0\hm<c\hm<1$, для всех $J\hm>J_0$, и~найдется такое $J_1\hm=J_1(\varepsilon,c)
\hm\geq J_0$, что для всех $J\hm>J_1$ имеют место соотношения:

\noindent
\begin{alignat*}{2}
\mujk+\fr{\varepsilon}{\betajk}&\geq0\,;&\quad \mujk-\fr{\varepsilon}{\betajk}&\leq0\,;\\
\mujk+\fr{\varepsilon}{\betajk}&\leq T\,;&\quad \mujk-\fr{\varepsilon}{\betajk}&\geq -T
\end{alignat*}
для $j_2\leq j\leq J-1$. Следовательно, для одного сла\-га\-емо\-го из~$S_3$ имеем при 
$J\hm>J_1$:

\pagebreak

\noindent
\begin{multline}
{\sf P}\left(\left|\betajk\hat{Y}_{j,k}-\betajk\mujk\right|>\varepsilon\right)={}\\
{}=
{\sf P}\left(\left|\rho_{T}^{(h)}(Y_{j,k})-\mujk\right|>
\fr{\varepsilon}{\betajk}\right)={}\\
{}={\sf P}\left(\mathbf{1}\left(|Y_{j,k}|>T\right)Y_{j,k}>\mujk+
\fr{\varepsilon}{\betajk}\right)+{}\\
{}+
{\sf P}\left(\mathbf{1}\left(|Y_{j,k}|>T\right)Y_{j,k}<
\mujk-\fr{\varepsilon}{\betajk}\right)={}\\
{}={\sf P}\left(Y_{j,k}>T,Y_{j,k}>\mujk+\fr{\varepsilon}{\betajk}\right)+{}\\
{}+
{\sf P}\left(Y_{j,k}<-T,Y_{j,k}<\mujk-\fr{\varepsilon}{\betajk}\right)={}\\
{}=
{\sf P}\left(\left|Y_{j,k}\right|>T\right)\asymp{}\\
{}\asymp \fr{\exp\{-{(T+\mujk)^2}/({2\sigma^2})\}}{T}+{}\\
{}+
\fr{\exp\{-{(T-\mujk)^2}/({2\sigma^2})\}}{T}\,,
\label{S3_Term}
\end{multline}
поскольку $Y_{j,k}$ имеют нормальное распределение со средним~$\mujk$ 
и~дисперсией~$\sigma^2$, а~для стандартной нормальной функ\-ции распределения 
при до\-ста\-точ\-но больших~$x$ справедливо соотношение:
$$
1-\Phi(x)\asymp\fr{\Phi'(x)}{x}\,.
$$

Заметим, что при $j_2\leq j\hm\leq J-1$ величины~$|\mujk|$ не влияют на 
порядок правой час\-ти~(\ref{S3_Term}). Учитывая, что чис\-ло слагаемых в~$S_3$ 
имеет порядок~$2^J$, получаем:
\begin{multline}
S_3=\sum\limits_{j=j_2}^{J-1}\sumk{\sf P}\left(\left|
\betajk\hat{Y}_{j,k}-\betajk\mujk\right|>\varepsilon\right)\asymp{}\\
{}\asymp
\fr{2^J\exp\{-T^2/(2\sigma^2)\}}{T}\,.
\label{S3_Order}
\end{multline}

Найдем верхнюю оценку для функции потерь~(\ref{Risk_Definition}) при жест\-кой 
пороговой обработке. Для этого предположим, что все сла\-га\-емые в~суммах~$S_1$ и~$S_2$ 
из~(\ref{Risk_Decomposition}) отделены от нуля некоторой константой. Поэтому 
из~(\ref{Indices_Soft_j2}) получаем, что
\begin{multline}
S_1+S_2=\sum\limits_{j=0}^{j_2-1}\sumk{\sf P}\left(\left|
\betajk\hat{Y}_{j,k}-\betajk\mujk\right|>\varepsilon\right)\asymp{}\\
{}\asymp 2^{j_2}\asymp 2^{{J}/({2\gamma-2\alpha+1})}J^{1/(2\gamma-2\alpha+1)}\,.
\label{S1S2_Order}
\end{multline}

Порог $T_*^{(h)}-T_2^{(h)}$
обеспечивает равенство порядков правых час\-тей~(\ref{S3_Order}) и~(\ref{S1S2_Order}) 
и,~таким образом, является нижней границей (с~точ\-ностью до величины 
порядка~$O(1/\sqrt{\ln 2^J})$) для асимптотически оптимального в~смыс\-ле
 функции потерь~$R_J$ порога.

Теперь найдем нижнюю границу для функции потерь~(\ref{Risk_Definition}). 
Заметим, что найдется такая функция $Kf\hm\in\mathrm{Lip}(\gamma,L)$, 
что в~неравенстве~(\ref{Wavelet_CoeffDecacy}) будет достигаться равенство 
для $0\hm\leq j\hm\leq j_1-1$~\cite{Mal99}. Следовательно, существует такое $J_2\hm>0$, 
что для всех $\varepsilon\hm>0$ и~$J\hm>J_2$ выполняется $|\mujk|
\hm>\varepsilon/\betajk$ при $0\hm\leq j\hm\leq j_1-1$. Тогда
\begin{multline*}
{\sf P}\left(\left|\betajk\hat{Y}_{j,k}-\betajk\mujk\right|>\varepsilon\right)\geq{}\\
{}+\geq
{\sf P}\left(\left|Y_{j,k}-\mujk\right|>\fr{\varepsilon}{\betajk}\right)\geq
2-2\Phi\left(\fr{\varepsilon}{\sigma\beta_{0,0}}\right)\,.
\end{multline*}

В этом случае порядок суммы~$S_1$ в~(\ref{Risk_Decomposition}) равен чис\-лу 
сла\-га\-емых, т.\,е.
\begin{multline}
\hspace*{-3.23184pt}S_1=\sum\limits_{j=0}^{j_1-1}\sumk{\sf P}
\left(\left|\betajk\hat{Y}_{j,k}-\betajk\mujk\right|>\varepsilon\right)\asymp
 2^{j_1}\asymp{}\\
 {}\asymp 2^{{J}/({2\gamma-2\alpha+1})}g(J)\,.
\label{S1_Order}
\end{multline}
Приравняем порядки $S_1$ и~$S_3$ из~(\ref{S1_Order}) и~(\ref{S3_Order}). 
В~этом случае порог равен $T_*^{(h)}\hm-T_1^{(h)}.$

Заметим, что сумма~$S_2$ в~данных рассуждениях не присутствует. Это означает, 
что истинное значение~$R_J$ имеет порядок не ниже данного, т.\,е.\
 рас\-смат\-ри\-ва\-емый порядок является нижней оценкой для истинного порядка 
 функции потерь, а~$T_*^{(h)}\hm-T_1^{(h)}$~--- 
 верхней границей для асимптотически оптимального порога~$T$, 
 поскольку, чтобы не удалить важ\-ные компоненты функции сигнала, следует 
 выбирать наименьший порог, не ухудшающий порядок функции потерь.

Теорема доказана.

\smallskip

\noindent
\textbf{Замечание~1.}\ Пороги $T_*^{(h)}\hm-T_2^{(h)}$ и~$T_*^{(h)}\hm-T_1^{(h)}$ 
имеют одинаковую воз\-рас\-та\-ющую по~$J$ компоненту~$T_*^{(h)}$, причем 
$|T_1^{(h)}\hm-T_2^{(h)}|$ стремится к~нулю. Это означает, что истинное значение 
порога, ми\-ни\-ми\-зи\-ру\-юще\-го порядок функции потерь при жест\-кой пороговой обработке, 
так\-же будет иметь главную часть~$T_*^{(h)}$.


\section{Мягкая пороговая обработка}

Пусть $\hat{Y}_{j,k}=\rho_T^{(s)}(Y_{j,k})$.
Пусть функция $g_1(J)\hm>0$ сколь угодно медленно убывает по~$J$ к~нулю,
 а~$g_2(J)\hm>0$ неограниченно возрастает по~$J$, причем
\begin{equation}
\ln  g_2(J)=o\left(\sqrt{\ln 2^J}\right)\,, \enskip  J\to\infty\,.
\label{g_2_Grow}
\end{equation}

\noindent
\textbf{Теорема~2.}\
\textit{Пусть $\gamma\hm>\alpha\hm-1/2$. При выборе асимптотически оптимального 
порога для мяг\-кой пороговой обработки функция потерь}~(\ref{Risk_Definition}), 
\textit{начиная с~некоторого~$J$, удовле\-тво\-ря\-ет неравенствам}:
\begin{multline*}
C_1^{(s)}\cdot2^{-({2\gamma-2\alpha})J/({2\gamma-2\alpha+1})}g_1(J)\leq 
R_J\leq{}\\
{}\leq C_2^{(s)}\cdot2^{-({2\gamma-2\alpha})J/({2\gamma-2\alpha+1})}g_2(J)\,,
\end{multline*}
\textit{где $C_1^{(s)}$ и~$C_2^{(s)}$~--- некоторые положительные константы.
Для асимптотически оптимального значения порога, ми\-ни\-ми\-зи\-ру\-юще\-го 
порядок функции потерь}~(\ref{Risk_Definition}) \textit{при мягкой пороговой 
обработке, справедливо неравенство}:
$$
T_*^{(s)}-T_2^{(s)}\leq T\leq  T_*^{(s)}-T_1^{(s)}\,, 
$$
\textit{где}
\begin{align*}
T_*^{(s)}&=\sigma\sqrt{\fr{4\gamma-4\alpha}{2\gamma-2\alpha+1}\ln2^J}\,;
\\
T_i^{(s)}&=\sigma\sqrt{\fr{2\gamma-2\alpha+1}{4\gamma-4\alpha}}\,
\fr{\ln\left((\ln 2^J)^{1/2}g_i(J)\right)}{\sqrt{\ln 2^J}}\,, \\
&\hspace*{60mm}  i=1,2\,.
\end{align*}


\noindent
{Д\,о\,к\,а\,з\,а\,т\,е\,л\,ь\,с\,т\,в\,о\,.}\ \
 Разобьем, как в~разд.~5, все множество индексов $\{0,\ldots,J-1\}$ 
 на три класса в~за\-ви\-си\-мости от величины~$|\mujk|$. Пусть индексы~$j_1$ 
 и~$j_2$ ($j_1\hm< j_2$) таковы, что
\begin{equation*}
|\mujk|\leq
\begin{cases}
 2^{-\alpha j}(g_1(J))^{-(\gamma-\alpha+1/2)}\,, &\ j_1\leq j\leq j_2-1\,;\\
2^{-\alpha j}(g_2(J))^{-(\gamma-\alpha+1/2)}\,, &\ j_2\leq j \leq J-1\,.
\end{cases}
\end{equation*}
При этом в~силу~(\ref{Wavelet_CoeffDecacy}) для $i\hm=1,2$

\vspace*{2pt}

\noindent
$$
j_i=\fr{J}{2\gamma-2\alpha+1}+\log_2 g_i(J)+ \fr{\log_2 A}{\gamma-\alpha+1/2}\,.
$$
Разобьем сумму в~чис\-ли\-те\-ле~(\ref{Risk_Definition}) на три суммы~(\ref{Risk_Decomposition}).

Рассмотрим $S_3$. Зафиксируем некоторое положительное чис\-ло~$\varepsilon$.
 Заметим, что для любого $\varepsilon\hm>0$ найдется такое $J_0\hm=J_0(\varepsilon)$, 
 что $ 2^{-\alpha j}(g_2(J))^{-(\gamma-\alpha+1/2)}\hm\leq\varepsilon/\betajk$ 
 для всех $J\hm>J_0$. Следовательно, имеет мес\-то соотношение
$|\mujk|\hm\leq\varepsilon/\betajk$ для $j_2\h\leq j\hm\leq J-1$. 
Для одного сла\-га\-емо\-го из~$S_3$ имеем при $J\hm>J_0$:
\begin{multline*}
{\sf P}\left(\left|\betajk\hat{Y}_{j,k}-\betajk\mujk\right|>\varepsilon\right)={}\\
{}=
{\sf P}\left(\left|\rho_{T}^{(s)}(Y_{j,k})-\mujk\right|>\fr{\varepsilon}{\betajk}\right)={}\\
{}={\sf P}\left(
\vphantom{\fr{\varepsilon}{\betajk}}
|\mathbf{1}\left(Y_{j,k}>T\right)(Y_{j,k}-T)+{}\right.\\
\left.{}+
\mathbf{1}(Y_{j,k}<-T)(Y_{j,k}+T)-\mujk|>\fr{\varepsilon}{\betajk}\right)={}\\
{}={\sf P}\left(Y_{j,k}>T,Y_{j,k}>T+\mujk+\fr{\varepsilon}{\betajk}\right)+{}
\end{multline*}

\noindent
\begin{multline}
{}+
{\sf P}\left(Y_{j,k}<-T,Y_{j,k}<-T+\mujk-\fr{\varepsilon}{\betajk}\right)={}\\
{}={\sf P}\left(|Y_{j,k}-\mujk|>T+\fr{\varepsilon}{\betajk}\right).
\label{S3_Term_Soft}
\end{multline}

Поскольку $\betajk$ воз\-рас\-та\-ет как~$2^{cJ}$ и~$Y_{j,k}$ 
имеют нормальное распределение со сред\-ним~$\mujk$ и~дис\-пер\-си\-ей~$\sigma^2$, 
из~(\ref{S3_Term_Soft}) имеем:
\begin{equation}
S_3\asymp \fr{2^J\exp\{-T^2/(2\sigma^2)\}}{T}\,.
\label{S3_Order_Soft}
\end{equation}

Найдем верх\-нюю оценку для функции потерь~(\ref{Risk_Definition}) при мягкой 
пороговой обработке. Для этого предположим, что все сла\-га\-емые в~суммах~$S_1$ и~$S_2$ 
из~(\ref{Risk_Decomposition}) отделены от нуля некоторой константой. 
В~этом случае получаем:
\begin{equation}
S_1+S_2\asymp2^{{J}/({2\gamma-2\alpha+1})}g_2(J)\,.
\label{S1S2_Order_Soft}
\end{equation}

Порог $T_*^{(s)}-T_2^{(s)}$
обеспечивает равенство порядков правых час\-тей~(\ref{S3_Order_Soft}) 
и~(\ref{S1S2_Order_Soft}) и,~таким образом, является ниж\-ней границей 
(с~точ\-ностью до величины порядка~$O(1/\sqrt{\ln 2^J})$) для асимптотически 
оптимального в~смысле функции потерь~$R_J$ порога.

Теперь найдем нижнюю границу для функции потерь~(\ref{Risk_Definition}). 
Пользуясь рас\-суж\-де\-ни\-ями, приведенными в~разд.~5, вви\-ду убывания к~нулю~$g_1(J)$, 
для любого $\varepsilon\hm>0$ и~до\-ста\-точ\-но больших~$J$ найдется такая 
функция $Kf\hm\in\mathrm{Lip}\,(\gamma,L)$, что в~неравенстве~(\ref{Wavelet_CoeffDecacy}) 
будет достигаться равенство и~выполняться $|\mujk|\hm>\varepsilon/\betajk$ при 
$0\hm\leq j\hm\leq j_1\hm-1$. Тогда поскольку

\vspace*{-6pt}

\noindent
\begin{multline*}
{\sf P}\left(\left|\betajk\hat{Y}_{j,k}-\betajk\mujk\right|>\varepsilon\right)={}\\
{}=
{\sf P}\left(|\mujk|>\fr{\varepsilon}{\betajk},\left\vert Y_{j,k}\right\vert\leq T\right)+{}\\
{}+{\sf P}\left(\left\vert Y_{j,k}-T-\mujk\right\vert >\fr{\varepsilon}{\betajk},
Y_{j,k}>T\right)+{}\\
{}+
{\sf P}\left(\left\vert Y_{j,k}+T-\mujk\right\vert >\fr{\varepsilon}{\betajk},
Y_{j,k}<-T\right)\,,
\end{multline*}
учитывая, что~$Y_{j,k}$ имеют нормальное распределение с~максимумом плот\-ности, 
до\-сти\-га\-емым в~точке~$\mujk$, получаем:
$$
{\sf P}\left(\left|\betajk\hat{Y}_{j,k}-\betajk\mujk\right|>\varepsilon\right)\geq
2-2\Phi\left(\fr{\varepsilon}{\sigma\beta_{0,0}}\right).
$$
Отсюда следует оценка, аналогичная~(\ref{S1_Order}):
\begin{equation}
S_1\asymp2^{{J}/({2\gamma-2\alpha+1})}g_1(J)\,.
\label{S1_Order_Soft}
\end{equation}

Приравняем порядки~(\ref{S3_Order_Soft}) и~(\ref{S1_Order_Soft}). 
В~этом случае порог равен $T_*^{(s)}\hm-T_1^{(s)}$.
Воспользовавшись рас\-суж\-де\-ни\-ями, приведенными в~разд.~5, 
получаем, что порог $T_*^{(s)}\hm-T_1^{(s)}$ является верх\-ней оценкой для 
истинного значения асимптотически оптимального порога~$T$.

Теорема доказана.

\smallskip

\noindent
\textbf{Замечание~2.}\ Пороги $T_*^{(s)}\hm-T_1^{(s)}$ и~$T_*^{(s)}\hm-T_2^{(s)}$ 
имеют одинаковую воз\-рас\-та\-ющую по~$J$ компоненту~$T_*^{(s)}$, причем, поскольку 
выполнено~(\ref{g_2_Grow}), $|T_1^{(s)}\hm-T_2^{(s)}|$ стремится к~нулю. 
Это означает, что истинное значение порога, минимизирующего порядок функции 
потерь при мягкой пороговой обработке, также будет иметь глав\-ную часть~$T_*^{(s)}$.

\smallskip

\noindent
\textbf{Замечание~3.}\ В~случае мягкой пороговой обработки функции~$g_i(J)$, $i\hm=1,2$, 
фигурирующие в~верх\-ней и~ниж\-ней оценках, имеют сколь угодно малую ско\-рость 
сходимости по~$J$ (в~отличие от логарифмической функции в~одной из оценок при
 жест\-кой пороговой обработке). Это позволяет сделать вывод о~том, что верх\-няя 
 оценка функции потерь и~ниж\-няя оценка асимптотически оптимального порога 
 при мягкой пороговой обработке точ\-нее, чем при жест\-кой.

\smallskip

\noindent
\textbf{Замечание~4.}\ Если дополнительно предположить, что вейв\-лет-функ\-ция~$\psi$ 
имеет компактный носитель, то требование равномерной ре\-гу\-ляр\-ности по Липшицу 
можно заменить на требование кусочной ре\-гу\-ляр\-ности (подробнее см.~\cite{Jan01}).


{\small\frenchspacing
 {%\baselineskip=10.8pt
 \addcontentsline{toc}{section}{References}
 \begin{thebibliography}{99}




\bibitem{Don95} %1
\Au{Donoho~D.} Nonlinear solution of linear inverse problems by wavelet-vaguelette 
decomposition~// Appl. Comput. Harmon.~A., 1995. Vol.~2. P.~101--126.

\bibitem{Abr98}  %2
\Au{Abramovich F., Silverman~B.\,W.} Wavelet decomposition approaches 
to statistical inverse problems~// Biometrika, 1998. Vol.~85. No.\,1. P.~115--129.

\bibitem{Lee97} 
\Au{Lee N.} Wavelet-vaguelette decompositions and homogenous equations.~---
 West Lafayette, IN, USA: Purdue University, 1997.   PhD Thesis. 103~p.


\bibitem{Mal99} 
\Au{Mallat S.} A~wavelet tour of signal processing.~--- New York, NY, USA:
Academic Press, 1999. 857~p.


\bibitem{SMS14}  %5
\Au{Sadasivan J., Mukherjee~S., Seelamantula~C.\,S.} 
An optimum shrinkage estimator based on minimum-probability-of-error criterion 
and application to signal denoising~// 
    %Proceedings of 39th International Conference on Acoustics, Speech and Signal Processing (ICASSP). 2014. Italy, Florence.
    39th IEEE  Conference (International) on Acoustics, 
    Speech and Signal Processing Proceedings.~--- Piscataway, NJ, USA: IEEE, 2014. 
    P.~4249--4253.

\bibitem{DJ98} 
\Au{Donoho D., Johnstone~I.\,M.} Minimax estimation via wavelet shrinkage~// 
Ann. Stat., 1998. Vol.~26. No.\,3. P.~879--921.



\bibitem{Jan01} 
\Au{Jansen M.} Noise reduction by wavelet thresholding.~--- 
Lecture notes in statistics ser.~---  New York, NY, USA: Springer Verlag, 2001. Vol.~161. 217~p.

\bibitem{DJ95} 
\Au{Donoho D., Johnstone~I.} Adapting to unknown smoothness via wavelet shrinkage~// 
J.~Am. Stat. Assoc., 1995. Vol.~90. P.~1200--1224.

\bibitem{MSH10} 
\Au{Маркин А.\,В., Шестаков~О.\,В.} О~состоятельности оценки риска при 
пороговой обработке вейв\-лет-ко\-эф\-фи\-ци\-ен\-тов~// Вестн. Моск. ун-та. Сер.~15: 
Вычисл. матем. и~киберн., 2010. №\,1. C.~26--34.

\bibitem{SH12} 
\Au{Шестаков О.\,В.} Асимптотическая нор\-маль\-ность оцен\-ки риска пороговой 
обработки вейв\-лет-ко\-эф\-фи\-ци\-ен\-тов при выборе адаптивного порога~// 
Докл. РАН, 2012. Т.~445. №\,5. С.~513--515.

\bibitem{KS16-1} 
\Au{Кудрявцев А.\,А., Шестаков~О.\,В.} Асимптотическое поведение порога, 
минимизирующего усред\-нен\-ную ве\-ро\-ят\-ность ошиб\-ки вы\-чис\-ле\-ния 
вейв\-лет-ко\-эф\-фи\-ци\-ен\-тов~// 
Докл. РАН, 2016. Т.~468. №\,5. С.~487--491.

\bibitem{KS16-2} 
\Au{Кудрявцев~А.\,А., Шестаков~О.\,В.} Асимптотически оптимальная пороговая 
обработка вейв\-лет-ко\-эф\-фи\-ци\-ен\-тов в~моделях с~негауссовым 
рас\-пре\-де\-ле\-ни\-ем шума~// Докл. РАН, 2016. Т.~471. №\,1. С.~11--15.

 \end{thebibliography}

 }
 }

\end{multicols}

\vspace*{-6pt}

\hfill{\small\textit{Поступила в~редакцию 25.02.18}}

\vspace*{8pt}

%\newpage

%\vspace*{-24pt}

\hrule

\vspace*{2pt}

\hrule

%\vspace*{8pt}


\def\tit{MINIMIZATION OF~ERRORS OF~CALCULATING WAVELET COEFFICIENTS 
WHILE~SOLVING INVERSE PROBLEMS}

\def\titkol{Minimization of~errors of~calculating wavelet coefficients 
while~solving inverse problems}

\def\aut{A.\,A.~Kudryavtsev$^1$ and O.\,V.~Shestakov$^{1,2}$}

\def\autkol{A.\,A.~Kudryavtsev and O.\,V.~Shestakov}

\titel{\tit}{\aut}{\autkol}{\titkol}

\vspace*{-9pt}


\noindent
$^1$Department of Mathematical Statistics, Faculty of Computational 
Mathematics and Cybernetics,\linebreak
$\hphantom{^1}$M.\,V.~Lomonosov Moscow State University, 
1-52~Leninskiye Gory, GSP-1, Moscow 119991, Russian Fed-\linebreak
$\hphantom{^1}$eration

\noindent
$^2$Institute of Informatics Problems, Federal Research Center ``Computer 
Science and Control'' of the Russian\linebreak
$\hphantom{^1}$Academy of Sciences, 44-2~Vavilov Str., 
Moscow 119333, Russian Federation


\def\leftfootline{\small{\textbf{\thepage}
\hfill INFORMATIKA I EE PRIMENENIYA~--- INFORMATICS AND
APPLICATIONS\ \ \ 2018\ \ \ volume~12\ \ \ issue\ 2}
}%
 \def\rightfootline{\small{INFORMATIKA I EE PRIMENENIYA~---
INFORMATICS AND APPLICATIONS\ \ \ 2018\ \ \ volume~12\ \ \ issue\ 2
\hfill \textbf{\thepage}}}

\vspace*{3pt}


\Abste{Statistical inverse problems arise in many applied fields, including 
medicine, astronomy, biology, plasma physics, chemistry, etc. At the same time, 
there are always errors in the observed data due to imperfect equipment, 
background noise, data discretization, and other reasons. To reduce these errors, 
it is necessary to apply special regularization methods that allow constructing 
approximate stable solutions of inverse problems. The classical\linebreak\vspace*{-12pt}}

\Abstend{regularization 
methods are based on the use of windowed singular value decomposition. However, 
this approach takes into account only the type of operator involved in the 
formation of observable data and does not take into account the properties 
of the object of observation. For linear homogeneous operators, this problem 
is solved with the help of special methods of wavelet analysis, which allow 
adapting simultaneously to the form of the operator and local features of 
the function describing the object. In this paper, the authors consider the 
problem of inverting a linear homogeneous operator in the presence of noise 
in the observational data by thresholding the wavelet expansion coefficients 
of the observed function. The asymptotically optimal thresholds and orders 
of the loss function are calculated when minimizing the averaged probability 
of error of wavelet coefficient calculation.}

\KWE{wavelets; thresholding; linear homogeneous operator; loss function}



\DOI{10.14357/19922264180203}

\vspace*{-14pt}

\Ack
\noindent
The work was partly supported by the Russian Foundation for Basic Research 
(project 18-07-00252).



%\vspace*{-3pt}

  \begin{multicols}{2}

\renewcommand{\bibname}{\protect\rmfamily References}
%\renewcommand{\bibname}{\large\protect\rm References}

{\small\frenchspacing
 {%\baselineskip=10.8pt
 \addcontentsline{toc}{section}{References}
 \begin{thebibliography}{99}



\bibitem{2-ks}
\Aue{Donoho, D.} 1995. Nonlinear solution of linear inverse problems by 
wavelet-vaguelette decomposition. \textit{Appl. Comput. Harmon.~A.} 2:101--126.

\bibitem{1-ks}
\Aue{Abramovich, F., and B.\,W.~Silverman.} 1998. 
Wavelet decomposition approaches to statistical inverse problems. 
\textit{Biometrika} 85(1):115--129.

\bibitem{3-ks}
\Aue{Lee, N.} 1997. Wavelet-vaguelette decompositions and homogenous equations.
 West Lafayette, IN: Purdue University. PhD Thesis. 103~p.

\bibitem{4-ks}
\Aue{Mallat, S.} 1999. \textit{A~wavelet tour of signal processing}. 
New York, NY: Academic Press. 857~p.

\bibitem{5-ks}
\Aue{Sadasivan, J., S.~Mukherjee, and C.\,S.~Seelamantula}. 2014. 
An optimum shrinkage estimator based on minimum-probability-of-error 
criterion and application to signal denoising. 
\textit{39th IEEE  Conference (International) on Acoustics, 
    Speech and Signal Processing Proceedings}. Piscataway, NJ: IEEE. 4249--4253.

\bibitem{6-ks}
\Aue{Donoho, D., and I.\,M.~Johnstone}. 1998. Minimax estimation via wavelet shrinkage. 
\textit{Ann. Stat.} 26(3):879--921.

\bibitem{7-ks}
\Aue{Jansen, M.} 2001. \textit{Noise reduction by wavelet thresholding}. 
Lecture notes in statistics ser. New York, NY: Springer Verlag. Vol.~161. 217~p.

\bibitem{8-ks}
\Aue{Donoho, D., and I.~Johnstone.} 1995. Adapting to unknown smoothness via wavelet 
shrinkage.  \textit{J.~Am. Stat. Assoc.} 90:1200--1224.

\bibitem{9-ks}
\Aue{Markin, A.\,V.,  and O.\,V.~Shestakov.} 2010. Consistency of risk 
estimation with thresholding of wavelet coefficients. 
\textit{Mosc. Univ. Comput. Math. Cybern.} 34(1):22--30.

\bibitem{10-ks}
\Aue{Shestakov, O.\,V.} 2012. Asymptotic normality of adaptive wavelet
 thresholding risk estimation. \textit{Dokl. Math.} 86(1):556--558.

\bibitem{11-ks}
\Aue{Kudryavtsev, A.\,A., and O.\,V.~Shestakov.} 2016. 
Asymptotic behavior of the threshold minimizing the average 
probability of error in calculation of wavelet coefficients. 
\textit{Dokl. Math.} 93(3):295--299.

\bibitem{12-ks}
\Aue{Kudryavtsev, A.\,A., and O.\,V.~Shestakov.} 2016. 
Asymptotically optimal wavelet thresholding in the models with 
non-Gaussian noise distributions. \textit{Dokl. Math.} 94(3):615--619.
\end{thebibliography}

 }
 }

\end{multicols}

\vspace*{-3pt}

\hfill{\small\textit{Received February 25, 2018}}

%\vspace*{-24pt}



\Contr

\noindent
\textbf{Kudryavtsev Alexey A.} (b.\ 1978)~--- 
Candidate of Science (PhD) in physics and mathematics, associate professor, 
Department of Mathematical Statistics, Faculty of Computational 
Mathematics and Cybernetics, M.\,V.~Lomonosov Moscow State University, 
1-52~Leninskiye Gory, GSP-1, Moscow 119991, Russian Federation; \mbox{nubigena@mail.ru}

\vspace*{3pt}

\noindent
\textbf{Shestakov Oleg V.} (b.\ 1976)~--- 
Doctor of Science in physics and mathematics, associate professor, 
Department of Mathematical Statistics, Faculty of Computational Mathematics 
and Cybernetics, M.\,V.~Lomonosov Moscow State University, 1-52~Leninskiye Gory, 
GSP-1, Moscow 119991, Russian Federation; senior scientist, 
Institute of Informatics Problems, Federal Research Center 
``Computer Science and Control'' of the Russian Academy of Sciences, 
44-2~Vavilov Str., Moscow 119333, Russian Federation; \mbox{oshestakov@cs.msu.su}

\label{end\stat}


\renewcommand{\bibname}{\protect\rm Литература}  %5
\def\stat{malashenko}

\def\tit{ПОСЛЕДОВАТЕЛЬНЫЙ АНАЛИЗ И~МЕТРИЧЕСКИЕ ОЦЕНКИ ПРЕДЕЛЬНЫХ
РАСПРЕДЕЛЕНИЙ МЕЖУЗЛОВЫХ ПОТОКОВ В~МНОГОПОЛЬЗОВАТЕЛЬСКОЙ СЕТИ}

\def\titkol{Последовательный анализ и~метрические оценки предельных
распределений межузловых потоков в %~многопользовательской 
сети}

\def\aut{Ю.\,Е. Малашенко$^1$}

\def\autkol{Ю.\,Е. Малашенко}

\titel{\tit}{\aut}{\autkol}{\titkol}

\index{Малашенко Ю.\,Е.}
\index{Malashenko Yu.\,E.}


%{\renewcommand{\thefootnote}{\fnsymbol{footnote}} \footnotetext[1]
%{Исследование выполнено при финансовой поддержке Российского научного фонда (проект 
%<<Информатика>> ФИЦ ИУ РАН, Москва).}}


\renewcommand{\thefootnote}{\arabic{footnote}}
\footnotetext[1]{Федеральный исследовательский центр <<Информатика и~управление>> Российской академии 
\mbox{mala-yur@yandex.ru}}


%\vspace*{-6pt}



\Abst{Для оценки функциональных возможностей
многопользовательской сети связи аналилизируется множество векторов межузловых потоков при предельных распределениях ресурсов
сети. В~рамках многопродуктовой модели про\-пуск\-ные спо\-соб\-ности ребер рас\-смат\-ри\-ва\-ют\-ся 
как компоненты вектора ресурсов различных
типов, которые требуются для передачи потоков различных видов.
При проведении вычислительных экспериментов на каждой итерации вычисляются нормы векторов совместно допустимых межузловых
потоков, при передаче которых полностью используется пропускная спо\-соб\-ность всех ребер сети. Полученные последовательности
метрических оценок позволяют анализировать особенности множества до\-сти\-жи\-мости и~эф\-фек\-тив\-ность использования ресурсов сети при
уравнительном распределении про\-пуск\-ной спо\-соб\-ности между корреспондентами.}

\KW{многопродуктовая потоковая сетевая
модель; множество достижимых межузловых потоков; предельные
распределения пропускной способности}

\DOI{10.14357/19922264220306} 
  
%\vspace*{-3pt}


\vskip 10pt plus 9pt minus 6pt

\thispagestyle{headings}

\begin{multicols}{2}

\label{st\stat}

\section{Введение}

Данная работа продолжает исследования функциональных характеристик
сетевых сис\-тем связи~[1]. В~настоящее время математические модели
передачи многопродуктового потока применяются для поиска
распределений потоков и~ресурсов в~многопользовательских
телекоммуникационных\linebreak сетях~[2--4]. Разрабатываются методы анализа
с~учетом вектора требований всех \textit{равноправных} 
и~невзаимозаменяемых корреспондентов~[5]. С~позиций\linebreak методологии
исследования операций изучаются справедливые распределения потоков
и~ресурсов~[6].

Соответствующие \textit{недискриминирующие} правила управления
потоками являются решениями задач на максмин и/или получаются 
в~результате использования процедур последовательной
лексикографически упорядоченной оптимизации~[7].

В~настоящей работе пути соединения корреспондентов прокладываются
через со\-от\-вет\-ст\-ву\-ющие минимальные разрезы. Указанный метод\linebreak \mbox{можно}
рассматривать как возможный вариант решения задачи о~построении
SPLIT-марш\-ру\-тов~[8,~9]. В~рамках вычислительных экспериментов\linebreak на
многопродуктовой модели анализируются распределения межузловых
потоков  и~пропускной способ\-ности сети.  Для оценки функциональных
возможностей многопользовательской сети используется вектор
совместно допустимых межузловых потоков. Под ресурсом, выделяемым
некоторой паре узлов-кор\-рес\-пон\-ден\-тов,  понимается суммарное
значение тре\-бу\-емых пропускных способностей на всех ребрах,
расположенных на всех маршрутах при прохождении межузлового\linebreak потока
данного вида.  Сумма соответствующих реберных потоков трактуется
как полная нагрузка на сеть, возникающая при передаче заданного
межузлового потока. Рас\-смат\-ри\-ва\-ют\-ся распределения пропускной
способности и~межузловых потоков при предельной загрузке сети.
При проведении вычислительных экспериментов на каждой  итерации
вычисляется норма  вектора совместно допустимых межузловых
потоков.   Для оценки величины требуемых ресурсов при соединении
корреспондентов по различным путям для каж\-дой пары узлов
определяется максимальный однопродуктовый поток. Марш\-ру\-ты передачи
всех допустимых межузловых потоков  проходят по ребрам
соответствующих минимальных разрезов. Вычислительные эксперименты
проводились  для получения последовательности  мет\-ри\-че\-ских оценок
векторов межузловых потоков, принадлежащих множеству до\-сти\-жи\-мости
многопользовательской сети.

\section{Математическая модель}

В качестве математической модели для описания
многопользовательской сетевой системы используется следующая
формальная запись условий и~ограничений, которые должны
выполняться при одновременной передаче потоков различных видов
между всеми парами улов-корреспондентов:

Сеть $G(\mathbf{d})$ задается множествами $\langle V,
R,U,P\rangle$:
\begin{itemize}
\item  узлов (вершин) сети 
$$
V=\left \{v_{1}, v_{2},\dots,v_{n},\dots,v_{N}\right\};
$$
\item  неориентированных ребер 
$$
R=\left\{r_{1}, r_{2}, \dots, r_{k}, \dots,
r_{E}\right\}.
$$
\end{itemize}

Ребро $r_{k}$ \textit{соединяет} концевые вершины~$v_{n_k}$ и~$v_{j_k}$. 
Реб\-ру~$r_{k}$ ставятся в~соответствие две
ориентированные дуги $\{u_{k},u_{k+E}\}$ из множества
ориентированных дуг $U\hm=\{u_{1}, u_{2}, \dots, u_{k}, \dots,
u_{2E}\}$. Дуги $\{u_{k}, u_{k+E}\}$ определяют прямое и~обратное
на\-прав\-ле\-ние передачи потока по реб\-ру~$r_{k}$ между концевыми
вершинами $\{v_{n_k}, v_{j_k}\}$.

В многопользовательской сети~$G(\mathbf{d})$ рассматривается
$M\hm=N(N\hm-1)$ независимых, невзаимозаменяемых и~равноправных потоков
различных видов, которые передаются между уз\-ла\-ми-кор\-рес\-пон\-ден\-та\-ми
из множества 
$$
P=\left\{p_{1}, p_{2}, \dots, p_{M}\right\}.
$$

По определению, каждой паре уз\-лов-кор\-рес\-пон\-ден\-тов~$p_{m}$
соответствуют:
\begin{itemize}
\item вершина-ис\-точ\-ник с~номером~$s_{m}$, через которую входной поток
$m$-го вида~$z_{m}$ поступает в~сеть;
\item  вершина-при\-ем\-ник с~номером~$t_{m}$, из которой поток $m$-го
вида~$z_{m}$ покидает сеть.
\end{itemize}

В множестве~$P$ выделяется подмножество $P(R^{+})$ пар
уз\-лов-кор\-рес\-пон\-ден\-тов, расположенных в~концевых вершинах
ребра~$r_{k}$, $k\hm=\overline{1,E}$. Вводятся сле\-ду\-ющие обозначения:
пусть ребро~$r_{k}$  с~номером~$k$ соединяет вершины с~номерами~$n$ и~$j$ такими, что $n\hm< j$. Для со\-от\-вет\-ст\-ву\-ющей пары
уз\-лов-кор\-рес\-пон\-ден\-тов~$p_{k}$, расположенных в~узлах $\{v_{n},
v_{j}\}$, узел~$v_{n}$ считается источником, а узел~$v_{j}$~---
приемником потока $z_{k}$ $k$-го вида, который передается из узла
c номером~$n$ в~узел с~номером~$j$ для пары~$p_{k}$ с~номером~$k$.
Для пары $p^{}_{k+E} \Longleftrightarrow \{v_{j},v_{n}\}$ узел~$v_{j}$ 
считается источником~$s_{k+E}$, а~узел $v_m$~--- приемником~$t_{k+E}$ для пары с~номером~$p_{k+E}$. Формируется
$R^+\hm=\{1,2,3,\dots,E,E+1,\dots,2E\}$~--- список номеров смежных
пар.

Пары $p_{k}$ из подмножества~$P(R^{+})$ называются
\textit{смежными} уз\-ла\-ми-кор\-рес\-пон\-ден\-та\-ми. Все остальные
\textit{несмежные} пары уз\-лов-кор\-рес\-пон\-ден\-тов относятся к~множеству~$P(R^{-})$:
\begin{equation*}
P=P(R^{+})\cup P(R^{-});\quad
P(R^{+}) \cap P(R^{-}) = \emptyset.
\end{equation*}

Введем обозначения:
\begin{description}
\item[\,]
$z_{m}$~--- величина \textit{межузлового} потока $m$-го вида,
который поступает в~сеть из узла с~номером~$s_{m }$ и~покидает из
узла с~номером~$t_{m}$;
\item[\,]
$S(v_{n})$~--- множество номеров исходящих дуг, по которым поток
покидает узел~$v_{n}$;
\item[\,]
$T(v_{n})$~--- множество номеров входящих дуг, по которым поток
поступает в~узел~$v_{n}$.
\end{description}

Во всех узлах $v_{n}\in V$, $n\hm=\overline{1,N}$, для всех видов
потоков должны выполняться условия сохранения потоков:
\begin{multline}
\label{eq1} 
\sum\limits_{i\in S(v_n )} x_{mi}-\sum\limits_{i\in T(v_n )} x_{mi}
={}\\
{}=\begin{cases}
z_m, & \mbox{если } v=v^{}_{S_m}; \\
-z_m,&\mbox{если } v=v_{t_m}; \\
0&\mbox{в остальных случаях}, \\
\end{cases}
\end{multline}
$n=\overline{1,N}$, $m\hm=\overline{1,M}$, $x_{mi}\hm\ge 0$,
$z_{m}\hm\ge0$.

Величина {z}$_{m}$ равна входному потоку $m$-го вида, который
пропускается от источника к~приемнику пары $p_{m}$ при
распределении потоков $x_{mi}$ по дугам сети.

Каждому ребру $r_{k}\hm\in R$ приписывается неотрицательное число~$d_{k}$, 
определяющее суммарный предельно допустимый поток,
который можно передать по реб\-ру~$r_{k}$ в~обоих на\-прав\-ле\-ни\-ях. 
В~исходной сети компоненты вектора про\-пуск\-ных способностей
$\mathbf{d}\hm=(d_{1}, d_{2},\dots, d_{k}, \dots, d_{E})$~--- наперед
заданные положительные числа $d_{k}
\hm> 0$. Вектором $\mathbf{d}$ определяются сле\-ду\-ющие ограничения на сумму
дуговых потоков всех видов, пе\-ре\-да\-ва\-емых по реб\-ру~$r_{k}$:
\begin{multline}
\sum\limits_{m=1}^M (x_{mk}+x_{m(k+E)}) \le d_k,\\
 x_{mk}\ge 0\,,\enskip
 x_{m(k+E)}\ge 0\,, \enskip k=\overline {1,E}\,.
 \label{eq2} 
\end{multline}
В рамках данной модели пропускная спо\-соб\-ность ребер сети~--- вектор~$\mathbf{d}$~--- трактуется как <<\textit{ресурсное ограничение}>>,
а~сумма дуговых
 потоков рас\-смат\-ри\-ва\-ет\-ся как показатель использования
<<\textit{ресурсов}>> сети при передаче межузловых потоков
различных видов.

Для всех $z_{m}$ и~$x_{mi}$, удовлетворяющих
условиям~\eqref{eq1} и~\eqref{eq2}, вычисляются суммарные потоки:
\begin{equation}
 y_{m }=\sum\limits_{i=1}^{2E} {x}_{mi},\quad
m=\overline{1,M}\,.
\label{eq3}
\end{equation}

Суммарный реберный поток~$y_{m}$ характеризует
<<\textit{нагрузку}>> на сеть при передаче межузлового потока
величины $z_{m}$ из уз\-ла-ис\-точ\-ни\-ка~$s_{m}$ в~узел-при\-ем\-ник~$t_{m}$. 
Величина~$y_{m}$ показывает, какой суммарный
\textit{ресурс}~-- пропускная спо\-соб\-ность сети~-- требуется для
передачи межузлового потока~$z_{m}$, а~отношение
$w_{m}\hm={y_m}/{z_m}$,  $m\hm=\overline{1,M},$
показывает, какие \textit{ресурсы} необходимы для передачи
единичного потока $m$-го вида между узлами~$s_{m}$ и~$t_{m}$.

Ограничения~\eqref{eq1}--\eqref{eq3} задают подмножество
допустимых значений компонент вектора межузловых потоков
$\mathbf{z}\hm=\left(z_{1}, z_{2},\dots,z_{m},\dots,z_{M}\right)$:
\begin{equation*}
{Z}(\mathbf{d})=\left\{\mathbf{z} \ge 0 \mid
(\mathbf{z},\mathbf{x},\mathbf{y}) \ \mbox{удовлетворяют~\eqref{eq1}--\eqref{eq3}}
\right\}\!,
\!\!
%\label{eq4} 
\end{equation*}
а все допустимые распределения ресурсов принадлежат подмножеству
\begin{equation*}
{Y}(\mathbf{d})=\left\{\mathbf{y} \ge 0 \mid
(\mathbf{z},\mathbf{x},\mathbf{y}) \ \mbox{удовлетворяют~\eqref{eq1}--\eqref{eq3}}\right\}\!.
%\!\!\!\label{eq5}
\end{equation*}


\section{Метрические оценки предельных распределений}

Для оценки функциональных возможностей сис\-те\-мы рассматриваются
допустимые распределения межузловых потоков при предельных
загрузках ребер сети.

В рамках данного модельного описания монопольным режимом
называется способ управления, при котором все ресурсы сети
используются для передачи потока одной выделенной пары
уз\-лов-кор\-рес\-пон\-ден\-тов $p_{a}\hm\in P(R^-)$, а для всех
остальных потоки полагаются равными нулю.

Предельно допустимый поток, который можно передать между
фиксированной парой уз\-лов-кор\-рес\-пон\-ден\-тов $p_{a}$ в~монопольном
режиме, является решением стандартной, в~данном случае
однопродуктовой, задачи о~максимальном потоке.

\smallskip

\noindent
\textbf{Задача 1.} Найти
$z_a^0\hm=\max\limits_{\langle z,x\rangle \in Z(d)} z_a
$
при условии $z_{i}=0$, $i\hm=\overline{1,M}$, $i\hm\ne a$.

При решении задачи~1 для пары $p_{a}$ вы\-чис\-ля\-ют\-ся: межузловой
поток~$z_a^0$; дуговые потоки $\{x^{0}_{ak};x^{0}_{a(k+E)}\}$,
$k\hm=\overline{1,E}$; суммарное значение реберного
потока~$y_{a}^{0}\hm=\sum\nolimits_{i=1}^{2E} {x}_{ai}^{0}$.

Поток величины $z_a^0$ является \textit{максимальным потоком},
пе\-ре\-да\-ва\-емым в~\textit{монопольном режиме} для пары
уз\-лов-кор\-рес\-пон\-ден\-тов~$p_{a}$, $p_{a}\hm\in P(R^-)$, в~сети~$G(d)$.

Задача~1 решается последовательно для всех $p_{m}\in P(R^-)$,
вы\-чис\-ля\-ют\-ся значения $z_{m}^{0}(t)$.

При проведении вычислительных экспериментов использовалась
итерационная процедура для оценки функциональных возможностей
сис\-те\-мы при передаче межузловых потоков по нескольким маршрутам.
На предварительном этапе шага~$t$ в~сети~$G(t)$ при заданных
значениях пропускной спо\-соб\-ности ребер~$d_k(t)$ для каждой \mbox{пары}
уз\-лов-кор\-рес\-пон\-ден\-тов $p_m\hm\in P(R^-)$ определяется максимально
допустимый однопродуктовый поток~$z^0_m(t)$, со\-от\-вет\-ст\-ву\-ющие
дуговые потоки $(x_{mk}^0(t),x_{m(k+E)}^0(t))$, $p_m\hm\in P(R^-)$, и~коэффициенты нормировки
$\xi_m^0(t)\hm={1}/{z_m^0(t)}$ для всех  $p_m\hm \in P(R^-)$,
таких что $z^0_m(t)\hm>0$, $y_m^0(t)\hm>0$.
Коэффициенты~$\xi_m^0(t)$ используются для поиска текущих
совместно допустимых квот на передачу потоков одновременно между
всеми парами $p_m\in P(R^-)$.

\smallskip

\noindent
\textbf{Задача 2.} Найти $\alpha^*(t)=\max\limits_\alpha \alpha$
при условиях
$$
\alpha\!\!\sum\limits_{m\in R^-}\! \xi_m^0\left(x_{mk}^0(t)+x_{m(k+E)}^0(t)\right)\le d_k(t),\enskip
k=\overline{1,E}\,.
$$

На основании $\alpha^*(t)$ вычисляются совместно допустимые
дуговые потоки:
\begin{multline*}
x_{mk}^*(t)=\alpha^*(t)\xi^0_m(t)x^0_{mk}(t),\\
x^*_{m(k+E)}(t)=\alpha^*(t)\xi^0_m(t)x^0_{m(k+E)}(t),
\\
m=\overline{1,M}\,,\enskip k=\overline{1,E}\,,
\end{multline*}
и остаточная пропускная способность ребер в~сети $G(t+1)$:
\begin{multline*}
d_k(t+1)=d_k(t)-\sum_{m\in R^-} (x_{mk}^*(t)+x_{m(k+E)}(t)),\\
k=\overline{1,E}\,,\enskip p_m\in P(R^-).
\end{multline*}
Формируется вектор допустимых межузловых потоков:
\begin{align*}
z_k^+(t)&=d_k(t+1),\enskip p_k\in P(R^+),\enskip k=\overline{1,E}\,;
\\
z_m^-(t)&=\sum\limits_{\tau=1}^t\alpha^*(\tau)\xi_m^0(\tau) z_m^0(\tau), \enskip p_m\in P(R^-).
\end{align*}

По построению, на шаге~$t$ при передаче вектора межузлового потока
$\mathbf{z}(t)=\{\mathbf{z}^+(t), \mathbf{z}^-(t)\}$ достигается
предельная загрузка, и~пропускная способность всех ребер  сети
используется полностью.

Точка с~координатами $\mathbf{z}(t)$ принадлежит множеству~$Z(d)$.

Расстояние точки от начала координат определяется как норма
соответствующего вектора:
\begin{align*}
\rho^+(t)&=\|\mathbf{z}^+(t)\|=
\left[\,\sum\limits_{k=1}(\mathbf{z}^+(t))^2\right]^{1/2};
\\
\rho^-(t)&=\|\mathbf{z}^-(t)\|= \left[\sum\limits_{p_m\in P(R^-)}(\mathbf{z}_m^-(t))^2\right]^{1/2}.
\end{align*}

Если при выполнении шага $(t+1)$ окажется, что $z_m^0(t+1)=0$ для
всех $p_m\in P(R^-)$, то про\-изойдет останов и~сформируются
массивы финальных данных:
\begin{align*}
z_m^-(T)&=\sum\limits_{\tau=1}^t \alpha^*(\tau)\xi_m^0(\tau) z_m^0(\tau),\enskip 
p_m\in P(R^-),\\
z_k^+(T)&=d_k(t+1),\enskip p_k\in P(R^+),\enskip k=\overline{1,E}\,.
\end{align*}

Вышеописанная вычислительная процедура далее обозначается как
MFPL-про\-це\-ду\-ра (от англ.\ \textit{max-flow-peak-load}).

При проведении второй серии вычислительных экспериментов
MFPL-про\-це\-ду\-ра использовалась для оценки функциональных
характеристик сис\-те\-мы при \textit{уравнительном} поэтапном
распределении пропускной способности между всеми
па\-ра\-ми-кор\-рес\-пон\-ден\-тами.

При реализации MFPL-процедуры выполнение каждого шага разбивается
на несколько этапов. На предварительном этапе шага~$t$ 
в~сети~$G(t)$ при заданных значениях пропускной способности ребер~$d_k(t)$ 
для каждой пары уз\-лов-кор\-рес\-пон\-ден\-тов $p_m\hm\in P(R^-)$
определяется максимально допустимый однопродуктовый
поток~$z_m^0(t)$, соответствующие дуговые потоки
$\left(x_{mk}^0(t),x_{m(k+E)}^0(t)\right)$, $p_m\hm\in P(R^-)$, и~суммарная
реберная нагрузка
$$
y_m^0(t)=\sum\limits_{k=1}^E (x_{mk}^0(t),x_{m(k+E)}^0(t)),\enskip p_m\in P(R^-).
$$

Для каждой пары $p_m\hm\in P(R^-)$ вычисляются коэффициенты
нормировки
$\theta_m^0(t)\hm={1}/{y_m^0(t)}$ для всех  
$p_m\hm\in P(R^-)$, таких что  $z^0_m(t)\hm>0$,
$y_m^0(t)\hm>0$.
Коэффициенты~$\theta_m^0(t)$ используются для поиска совместно
допустимых дуговых потоков для всех $p_m\hm\in P(R^-)$.

\smallskip

\noindent
\textbf{Задача 3.} Найти $\beta^*(t)=\max\nolimits_\beta \beta$ при
условиях
$$
\beta\!\!\!\!\sum\limits_{p_m\in P(R^-)}\!\!
\theta_m^0(x_{mk}^0(t)+x_{m(k+E)}^0(t))\le d_k(t),\enskip
k=\overline{1,E}\,.
$$

 С помощью $\beta^*(t)$ (решения задачи~3) вычисляются текущие допустимые значения дуговых потоков:
\begin{multline*}
x_{mk}^*(t)=\beta^*(t)\theta^0_m(t)x^0_{mk}(t),\\
x^*_{m(k+E)}(t)=\beta^*(t)\theta^0_m(t)x^0_{m(k+E)}(t), \enskip
k=\overline{1,E},
\end{multline*}
и реберных нагрузок при одновременной передаче межузловых потоков:

\noindent
\begin{multline*}
y_m^*(t)=\sum\limits_{i=1}^E
\left[x_{mi}^*(t)+x^*_{m(i+E)}(t)\right]={}\\
{}= \fr{\beta^*(t)}{y_m^0(t)} \sum\limits_{i=1}^E
\left[x_{mi}^0(t)+x^0_{m(i+E)}(t)\right]=\beta^*(t), \\
 p_m\in P(R^-).
\end{multline*}
Таким образом на каждом шаге определенная часть имеющегося ресурса
(пропускной спо\-соб\-ности) делится строго по\-ров\-ну меж\-ду всеми
корреспондентами $p_m\in P(R^-)$, для которых существует путь
передачи в~$G(t)$.

Формируется вектор допустимых межузловых потоков:
\begin{gather*}
\hspace*{-30mm}z_k^{++}(t)=d_k(t+1)={}\hspace*{10mm}\\
{}=d_k(t)-\!\!\! \sum\limits_{p_m\in P(R^-)}\!\!\!
\left(x_{mk}^*(t)+x_{m(k+E)}(t)\right),\\
\hspace*{35mm}k=\overline{1,E}, \enskip
p_k\in P(R^+);\\
z_m^{(=)}(t)\overset{\Delta}{=}\sum\limits_{\tau=1}^t\beta^*(\tau)
\theta_m^0(\tau) z_m^0(\tau), \enskip p_m\in P(R^-).
\end{gather*}

\noindent
Определяются расстояния:
\begin{align*}
\rho^{++}(t)&=\|\mathbf{z}^{++}(t)\|\overset{\Delta}{=}
\left[\sum\limits_{k=1}^E\left(d_k(t+1)\right)^2\right]^{1/2};\\
\rho^{(=)}(t)&=\|\mathbf{z}^{=}(t)\|= \left[\sum\limits_{p_m\in
P(R^-)}\left(z_m^{(=)}(t)\right)^2\right]^{1/2}.
\end{align*}

Если на предварительном этапе на шаге $(t+1)$ окажется, что в~сети~$G(t+1)$ для всех $p_m\hm\in P(R^-)$ все значения
$z_m^0(t+1)\hm=0$, то произойдет останов и~сформируются финальные
массивы:
\begin{align*}
z_k^{(++)}(T)&=d_k(t+1), \enskip
p_k\in P(R^+), \enskip k=\overline{1,E};
\\
z_m^{(=)}(t)&=\sum\limits_{\tau=1}^{t+1}\beta^*(\tau)
\theta_m^0(\tau) z_m^0(\tau), \enskip p_m\in P(R^-).
\end{align*}



\section{Вычислительный эксперимент}

Результаты вычислительных экспериментов, описанные ниже, служат
продолжением исследований, начатых в~[1]. Вычислительные
эксперименты проводились на моделях сетевых сис\-тем, пред\-став\-лен\-ных
на рис.~1 и~2. В~каждой сети~69~узлов. Пропускные спо\-соб\-но\-сти
ребер~-- значения $d_k$~-- выбирались случайным образом из отрезка
$[900,999]$ и~совпадали для ребер, при\-сут\-ст\-ву\-ющих в~обеих сетях.
В~кольцевой сети пропускная спо\-соб\-ность каждого из добавленных
ребер равнялась~900.

\begin{figure*} %fig1
\vspace*{1pt}
\begin{minipage}[t]{80mm}
  \begin{center}  
    \mbox{%
\epsfxsize=69.408mm
\epsfbox{mal-1.eps}
}

\end{center}
\vspace*{-6pt}
\Caption{Базовая сеть}
\end{minipage}
%\end{figure*}
\hfill
%\begin{figure*} %fig2
\vspace*{1pt}
\begin{minipage}[t]{80mm}
  \begin{center}  
    \mbox{%
\epsfxsize=69.408mm
\epsfbox{mal-2.eps}
}

\end{center}
\vspace*{-6pt}
\Caption{Кольцевая сеть}
\end{minipage}
\end{figure*}

\begin{table*}[b]\small %tabl1
\vspace*{-12pt}
\begin{center}

%\renewcommand{\arraystretch}{1.1}
\Caption{Базовая сеть}
\vspace*{2ex}

\begin{tabular}{|c||c|c|c||c|c|c|} 
\hline
&&&&&&\\[-9pt]
$t$  & $\rho^{-}(t)$ & $\rho^{+}(t)$ & $d^{+}(t+1)$ &
$\rho^{=}(t)$ & $\rho^{++}(t)$&  $d^{++}(t+1)$ \\ 
\hline
\hphantom{99}0  & \hphantom{99}0   & 8048&  68256&  \hphantom{9}0   &  8048&   68256\\
1  & \hphantom{9}63  & 4182&  26544&  \hphantom{9}95  &  3881&   24476\\
$\cdots$  & $\cdots$   & $\cdots$   &  $\cdots$    &  $\cdots$   &  $\cdots$   &   $\cdots$\\
11 & \hphantom{9}70  & 3975&  21469&  \hphantom{9}101\hphantom{9} &  3707&   20155\\
$\cdots$& $\cdots$   & $\cdots$   &  $\cdots$    & $\cdots$   &  $\cdots$   &  $\cdots$\\
22 & \hphantom{9}83  & 3861&  19623&  \hphantom{9}122\hphantom{9} &  3586&   18260\\
$\cdots$ & $\cdots$  & $\cdots$   &  $\cdots$   &  $\cdots$   &  $\cdots$  &   $\cdots$\\
33 & \hphantom{9}103\hphantom{9} & 3778&  18827&  \hphantom{9}139\hphantom{9} &  3522&   17601\\
$\cdots$ &$\cdots$  &$\cdots$  & $\cdots$  & $\cdots$   &  $\cdots$  &  $\cdots$\\
44 & \hphantom{9}\bf 190\hphantom{9} & \bf3553&  \bf17503&  \hphantom{9}\bf203\hphantom{9} &  \bf3285&   \bf16201\\
45 & \hphantom{9}\bf1452\hphantom{99}& \bf2166&  \hphantom{9}\bf7069 &  \hphantom{9}\bf1376\hphantom{99}&  \bf2020&   \hphantom{9}\bf6584\\
46 & \hphantom{9}\bf1498\hphantom{99}& \bf2158&  \hphantom{9}\bf6707 &  \hphantom{9}\bf1388\hphantom{99}&  \bf2017&   \hphantom{9}\bf6483\\
$\cdots$ & $\cdots$   & $\cdots$   &  $\cdots$    & $\cdots$   &  $\cdots$   &  $\cdots$\\
52 & \hphantom{9}1535\hphantom{99}& 2155&  \hphantom{9}6413 & \hphantom{9}1442\hphantom{99} &  2011&   \hphantom{9}6059\\
\hline
\end{tabular}
\end{center}
 %\end{table*}
% \begin{table*}\small %tabl2
\begin{center}
\Caption{Кольцевая сеть}
\vspace*{2ex}


\begin{tabular}{|c||c|c|c||c|c|c|} 
\hline
&&&&&&\\[-9pt]
$t$  & $\rho^{-}(t)$ & $\rho^{+}(t)$ & $d^{+}(t+1)$ &
$\rho^{=}(t)$ & $\rho^{++}(t)$&  $d^{++}(t+1)$ \\
 \hline
\hphantom{9}0  &\hphantom{99}0    & 8440  & 75456   &\hphantom{9}0      &8440   &75456\\
\hphantom{9}1  &\hphantom{9}68   & 5317  & 43038   &92     &5045   &40716 \\ 
$\cdots$ &$\cdots$    & $\cdots$     & $\cdots$   &$\cdots$      &$\cdots$      &$\cdots$      \\
11 &\hphantom{9}95   & 3608  & 20459   &124    &3397   &19080  \\
$\cdots$ &$\cdots$   & $\cdots$    & $\cdots$      &$\cdots$     &$\cdots$     &$\cdots$   \\
22 &\hphantom{9}101\hphantom{9}  & 3540  & 19530   &130    &3350   &18338 \\
$\cdots$ &$\cdots$  & $\cdots$   &$\cdots$      &$\cdots$     &$\cdots$   &$\cdots$    \\
33 &\hphantom{9}135\hphantom{9}  & 3346  & 17561   &154    &3220   &17003 \\
$\cdots$  &$\cdots$   & $\cdots$    & $\cdots$      &$\cdots$     &$\cdots$    &$\cdots$    \\
44 &\hphantom{9}234\hphantom{9}  & 3094  & 14881   &269    &2918   &13848 \\
$\cdots$ &$\cdots$   & $\cdots$    &$\cdots$      &$\cdots$     &$\cdots$     &$\cdots$    \\
50 &\hphantom{9}\bf 413\hphantom{9}  & \bf2770  & \bf12901   &\bf329    &\bf2792   &\bf13079 \\
51 &\hphantom{9}\bf1040\hphantom{99} & \bf2299  & \hphantom{9}\bf8801    &\bf334    &\bf2784   &\bf13034 \\
52 &\hphantom{9}\bf1062\hphantom{99} & \bf2297  & \hphantom{9}\bf8672    &\bf974    &\bf2262   &\hphantom{9}\bf8768  \\
$\cdots$ &$\cdots$   &$\cdots$    & $\cdots$      &$\cdots$      &$\cdots$     &$\cdots$    \\
55 &\hphantom{9}1069\hphantom{99} & 2297  & \hphantom{9}8630    &1010\hphantom{9}   &2259   &\hphantom{9}8553  \\
\hline
 \end{tabular}
\end{center}
 \end{table*}




Для базовой сети исходная сумма пропускных способностей:
$D^+(0)\hm=68\,256$, а~для кольцевой сети $D^{++}(0)=75\,456$.
Соответствующие значения $\rho^+(0)$ и~$\rho^{++}(0)$ указаны в~<<нулевой>> строке 
в~табл.~1 и~2, где собраны результаты
вычислительных экспериментов. В~ходе эксперимента при
уравнительном распределении остаточных ресурсов соблюдается
\textit{равномерное} убывание остаточной пропускной спо\-соб\-ности и~<<\textit{длины}>> вектора~$\rho^+(t)$. 
Однако между 44--46
итерациями для базовой и~50--52 для кольцевой сети наблюдается
резкий скачок величин~$\rho^-(t)$, $\rho^{=}(t)$ и~$d^+(t)$,
$d^{++}(t)$.

На указанных шагах полностью используется пропускная способность
ребер в~центральной час\-ти сети. Сеть \textit{распадается} на
несвязные компоненты, и~для $80\%$ корреспондентов пропадают пути
соединения, а~остаточный ресурс распределяется поровну между
оставшимися парами узлов.

Анализ результатов показал, что почти равные значения потоков
достигаются для~80\% корреспондентов и~требуют 60\%--70\%
ресурсов. Однако для~2\% смежных  пар узлов межузловые потоки на
два порядка выше медианных значений, а~затраты пропускной
способности  со\-став\-ля\-ют~20\%--30\%.








\section{Заключение}

Предложенный метод и~проведенные вычислительные эксперименты
показали, что уравнительное поэтапное распределение   приводит 
к~неравномерному  распределению   потоков  для разных групп\linebreak
корреспондентов.    Метрические оценки, полученные  в~ходе
экспериментов, продемонстрировали\linebreak \textit{деформацию} множества
достижимых потоков. В~рамках модели   предполагалось, что  все
корреспонденты  равноправны, а~потоки невзаимозаменяемы,  однако
при уравнительном предельном  распределении  смежные  пары узлов
оказывались в~привилегированном положении при использовании
остаточной пропускной способности. Пропускные способности  ребер
рассматривались  как вектор   ресурсов  различных типов,  которые
распределяются между корреспондентами   при передаче  потоков
различных видов.  По построению, на каж\-дом шаге норма вектора
смежных   межузловых    потоков численно равна   модулю вектора
остаточных  пропускных способностей.   Полученные мет\-ри\-че\-ские
значения  можно использовать  для   оценки функциональных
возможностей сети  в~режиме  предельной загрузки.

{\small\frenchspacing
 {%\baselineskip=10.8pt
 %\addcontentsline{toc}{section}{References}
 \begin{thebibliography}{9}

\bibitem{1-mal}
\Au{Малашенко Ю.\,Е., Назарова И.\,А.} Неоднородность
распределения   потоков при предельной  загрузке
многопользовательской сети~//  Известия РАН. Теория и~сис\-те\-мы
управления,  2022. №\,3. С.~81--96.

\bibitem{4-mal} %2
\Au{Luss H.} Equitable resource allocation: Models,
algorithms, and applications.~--- Hoboken, NJ, USA: John Wiley \& Sons, 2012.
420~p.

\bibitem{2-mal} %3
\Au{Ogryczak W., Luss~H., Pioro~M., Nace~D., Tomaszewski~A.}   Fair
optimization and networks: A~aurvey~// J.~Appl. Math., 2014. Vol.~2014. Art.~ID~612018. 25~p. doi: 10.1155/ 2014/612018.

\bibitem{3-mal} %4
\Au{Salimifard K., Bigharaz~S.} The multicommodity network
flow problem: State of the art classification, applications, and
solution methods~// J.~Oper. Res., 2020. Vol.~18. Iss.~3. P.~1--47.



\bibitem{5-mal}
\Au{Balakrishnan A., Li~G., Mirchandani~P.}  Optimal
network design with end-to-end service requirements~// Oper. Res.,
2017. Vol.~65. Iss.~3. P.~729--750.

\bibitem{6-mal}
\Au{Nace D., Doan~L.\,N., Klopfenstein~O., Bashllari~A.} Max-min
fairness in multicommodity flows~// Comput. Oper. Res., 2008.
Vol.~35. Iss.~2. P.~557--573.

\bibitem{7-mal}
\Au{Ros-Giralt J., Tsai~W.\,K.} A~lexicographic optimization
framework to the flow control problem~// IEEE T.
Inform. Theory, 2010. Vol.~56. Iss.~6. P.~2875--2886.

\bibitem{8-mal}
\Au{Baier G., Kohler~E., Skutella~M.}  The \mbox{k-splittable}
flow problem~//  Algorithmica, 2005. Vol.~42. Iss.~3-4.
P.~231--248.

\bibitem{9-mal}
\Au{Bialon P.\,A.} Randomized rounding approach to 
a~\mbox{k-splittable} multicommodity flow problem with lower path flow
bounds affording solution quality guarantees~// Telecommun. Syst.,
2017. Vol.~64. Iss.~3. P.~525--542.
\end{thebibliography}

 }
 }

\end{multicols}

\vspace*{-6pt}

\hfill{\small\textit{Поступила в~редакцию 10.06.22}}

\vspace*{8pt}

%\pagebreak

%\newpage

%\vspace*{-28pt}

\hrule

\vspace*{2pt}

\hrule

%\vspace*{-2pt}

\def\tit{SEQUENTIAL ANALYSIS AND METRIC ESTIMATES\\ OF~PEAK LOAD FLOWS IN~THE~MULTIUSER NETWORK}


\def\titkol{Sequential analysis and metric estimates of~peak load flows in~the~multiuser network}


\def\aut{Yu.\,E.~Malashenko}

\def\autkol{Yu.\,E.~Malashenko}

\titel{\tit}{\aut}{\autkol}{\titkol}

\vspace*{-8pt}


\noindent
Federal Research Center ``Computer Science and Control'' of the Russian Academy of Sciences, 
44-2~Vavilov Str., Moscow 119333, Russian Federation



\def\leftfootline{\small{\textbf{\thepage}
\hfill INFORMATIKA I EE PRIMENENIYA~--- INFORMATICS AND
APPLICATIONS\ \ \ 2022\ \ \ volume~16\ \ \ issue\ 3}
}%
 \def\rightfootline{\small{INFORMATIKA I EE PRIMENENIYA~---
INFORMATICS AND APPLICATIONS\ \ \ 2022\ \ \ volume~16\ \ \ issue\ 3
\hfill \textbf{\thepage}}}

\vspace*{3pt} 



\Abste{The set of vectors of internodal flows in a~multiuser communication network under peak load is analyzed. Within the framework of
 the multicommodity model, the throughput capacities of edges are considered as the components of a~vector of resources of various types that 
 are required for the transmission of various kinds of
 flows. When conducting computational experiments, at each iteration, the
  norms of vectors of jointly permissible internodal flows are calculated, during the transmission of which the capacity of 
  all network edges is fully used.\linebreak\vspace*{-12pt}}
 
 \Abstend{The proposed method and computational experiments have shown that the equalizing phased 
  distribution leads to an uneven distribution of flows for different groups of correspondents. Metric values obtained during experiments 
  indicate deformation of the sets of accessible flows. Within the framework of the model, all correspondents are tantamount 
  and the flows are noninterchangeable; however, in the case of an equalizing peak load distribution, adjacent pairs 
  of nodes are in a privileged position when using residual capacity. The obtained metric values can be used to 
  evaluate the functional characteristics of the transmission network in the finite capacity loading mode.}

\KWE{multicommodity flow network model; set of achievable internodal flows; peak load distribution}


\DOI{10.14357/19922264220306} 

%\vspace*{-16pt}

%\Ack
%\noindent



%\vspace*{4pt}

  \begin{multicols}{2}

\renewcommand{\bibname}{\protect\rmfamily References}
%\renewcommand{\bibname}{\large\protect\rm References}

{\small\frenchspacing
 {%\baselineskip=10.8pt
 \addcontentsline{toc}{section}{References}
 \begin{thebibliography}{9}
\bibitem{1-mal-1}
\Aue{Malashenko, Yu.\,E., and I.\,A.~Nazarova.}
2022. Heterogeneous flow distribution at the peak load in the multiuser network. \textit{J.~Comput. Sys. Sc. Int.} 61:372--387.

\bibitem{4-mal-1} %2
\Aue{Luss, H.} 2012. \textit{Equitable resource allocation: Models, algorithms, and applications}.
Hoboken, NJ: John Wiley \& Sons. 420~p.

\bibitem{2-mal-1} %3
\Aue{Ogryczak, W., H.~Luss, M.~Pioro, D.~Nace, and A.~Tomaszewski.}
 2014. Fair optimization and networks: A~survey. \textit{J.~Appl. Math.} 2014:612018. 25~p. doi: 10.1155/ 2014/612018.
\bibitem{3-mal-1} %4
\Aue{Salimifard, K., and S.~Bigharaz.}
 2020. The multicommodity network flow problem: State of the art classification, applications, and solution methods. 
 \textit{J.~Oper. Res.} 18(3):\linebreak 1--47.

\bibitem{5-mal-1}
\Aue{Balakrishnan, A., G.~Li, and P.~Mirchandani.} 2017. Optimal network design with end-to-end service requirements. 
\textit{Oper. Res.} 65(3):729--750.
\bibitem{6-mal-1}
\Aue{Nace, D., L.\,N.~Doan, O.~Klopfenstein, and A.~Bashllari.} 2008. Max-min fairness in multicommodity flows. 
\textit{Comput. Oper. Res.} 35(2):557--573.
\bibitem{7-mal-1}
\Aue{Ros-Giralt, J., and W.\,K.~Tsai.} 2010. A~lexicographic optimization framework to the flow control problem. 
\textit{IEEE T.~Inform. Theory} 56(6):2875--2886.
\bibitem{8-mal-1}
\Aue{Baier, G., E.~Kohler, and M.~Skutella.}
 2005. The k-splittable flow problem. \textit{Algorithmica} 42(3-4):231--248.
\bibitem{9-mal-1}
\Aue{Bialon, P.} 2017. A~randomized rounding approach to a~\mbox{k-splittable} multicommodity flow problem with lower path flow bounds affording solution quality guarantees. 
\textit{Telecommun. Syst.} 64(3):525--542.
 \end{thebibliography}

 }
 }

\end{multicols}

\vspace*{-6pt}

\hfill{\small\textit{Received June 10, 2022}}

\Contrl

\noindent
\textbf{Malashenko Yuri E.} (b.\ 1946)~--- 
Doctor of Science in physics and mathematics, principal scientist, Federal Research Center ``Computer Science and Control'' 
of the Russian Academy of Sciences, 44-2~Vavilov Str., Moscow 119333, Russian Federation; \mbox{malash09@ccas.ru} 


\label{end\stat}

\renewcommand{\bibname}{\protect\rm Литература}   %6 
\include{leri-pavlov} %7
\def\stat{krivenko}

\def\tit{МНОГОМЕРНЫЙ РЕФЕРЕНСНЫЙ РЕГИОН\\ ВЫСОКОЙ ПЛОТНОСТИ}

\def\titkol{Многомерный референсный регион высокой плотности}

\def\aut{М.\,П.~Кривенко$^1$}

\def\autkol{М.\,П.~Кривенко}

\titel{\tit}{\aut}{\autkol}{\titkol}

\index{Кривенко М.\,П.}
\index{Krivenko M.\,P.}


%{\renewcommand{\thefootnote}{\fnsymbol{footnote}} \footnotetext[1]
%{Работа выполнена при финансовой поддержке РФФИ (проекты 16-07-00677 
%и~15-37-20611-мол\_а\_вед).}}


\renewcommand{\thefootnote}{\arabic{footnote}}
\footnotetext[1]{Институт проблем информатики Федерального исследовательского центра <<Информатика и~управление>> Российской академии наук,
\mbox{mkrivenko@ipiran.ru}}

\vspace*{4pt}



\Abst{Рассматриваются принципы построения многомерных референсных регионов
(MRR~--- multivariate reference region). 
Предложен оригинальный метод построения региона на основе областей с~высокой 
плотностью точек и~аппроксимации распределения данных с~помощью смеси нормальных 
распределений. Для оценки порога для плотности распределения используется  
бут\-стреп-ме\-тод. В~качестве эксперимента рассмотрена задача построения 
и~использования эталонной области для прогнозирования типа мочевого камня. Обработка 
реальных данных продемонстрировала преимущества предлагаемых решений.}

\KW{многомерный референсный регион; область высокой плотности; бут\-стреп-ме\-тод; 
смесь многомерных нормальных распределений}

\vspace*{6pt}

\DOI{10.14357/19922264170207} 


\vskip 10pt plus 9pt minus 6pt

\thispagestyle{headings}

\begin{multicols}{2}

\label{st\stat}

\section{Введение}

     Многомерный референсный регион 
был предложен в~литературе по клинической химии в~начале 1970-х~гг.\ как 
альтернатива одномерным референсным интервалам~[1]. Там излагались 
преимущества предлагаемых множественных тестов, хоть и~имеющих 
упрощенный вид, но снижающих (по отношению к~одномерным вариантам) 
число ложных положительных результатов. Появление MRR оказалось 
особенно привлекательным для интерпретации результатов наборов 
медицинских тестов. Тем не менее возникали трудности в~построении 
и~использовании процедур многомерного анализа (см., например,~[2]), 
связанные, в~частности, с~быстрым увеличением числа параметров, которые 
должны быть оценены. Немногие лаборатории использовали MRR в~своей 
практике, причем в~экспериментальном режиме, и,~как следствие, на 
сегодняшний день имеется относительно малое количество соответствующих 
публикаций. 

\vspace*{-6pt}

\section{Многомерный референсный регион на основе расстояния Махалонобиса}

\vspace*{-2pt}

     Одномерный референсный интервал, полученный статистическим путем, 
использует центральную часть значений анализируемого показателя, обычно 
соответствующую~95\% некоторой популяции~--- совокупности особей 
определенного вида (например, здоровой части населения определенного пола 
из некоторого диапазона возрастов). Одномерные референсные интервалы 
применялись в~течение многих лет в~качестве стандартного приема 
интерпретации лабораторных данных. Они легко формируются, хранятся, 
извлекаются и~передаются в~лабораторных информационных системах, просты 
в~понимании, хорошо воспринимаются медицинским сообществом в~ходе 
длительного использования. Тем не менее одномерные референсные интервалы 
при классификации данных могут дать большое число ложно аномальных 
результатов. Этот далеко не единственный недостаток однофакторного 
референсного интервала может быть полностью или частично устранен 
с~помощью MRR.
     
     Простейшим и~весьма распространенным способом построения MRR 
является использование прямого произведения отдельных референсных 
интервалов в~предположении, что они статистически независимы. Пусть 
$(1\hm-\alpha)$~--- вероятность попадания в~MRR, а~$p_0$~--- вероятность 
попадания в~референсный интервал для любого из~$d$~признаков, тогда 
$p_0\hm= \sqrt[d]{1-\alpha}$. С~ростом размерности~$d$ значения~$p_0$ 
быстро приближаются к~1, что фактически лишает смысла применение MRR.
     
     Как и~в одномерном случае, отправной точкой для построения MRR 
может стать нормальное распределение. Идеи центрального расположения 
референсного региона и~заданной вероятности попадания в~него приводят для 
$d$-мер\-но\-го нормального распределения, имеющего плотность 
распределения
     \begin{multline*}
     \varphi(y,\mu,\Sigma) ={}\\
     {}=(2\pi)^{-d/2}\vert\Sigma\vert^{-1/2}\exp \left( -\fr{\left(y-
\mu\right)^{\mathrm{T}} \Sigma^{-1}(y-\mu)}{2}\right),
   \end{multline*}
где величина $(y-\mu)^{\mathrm{T}} \Sigma^{-1} (y-\mu)$ есть квадрат так 
называемого расстояния Махаланобиса между~$y$ и~$\mu$, к~использованию 
многомерного эллипсоида
\begin{multline*}
(2\pi)^{-d/2}\vert\Sigma\vert^{-1/2}\exp \left( -\fr{\left(y-\mu\right)^{\mathrm{T}}
\Sigma^{-1} 
(y-\mu)}{2}\right) ={}\\
{}=const
\end{multline*}
или, что то же самое, 
$$ 
(y-\mu)^{\mathrm{T}} \Sigma^{-1}(y-\mu)=const\,.
$$
Его называют эллипсоидом равной плотности распределения (или просто 
эллипсоидом равной вероятности). 
     
     Если задаться вероятностью $(1\hm-\alpha)$ попадания в~эллипсоид 
равной вероятности вида $(y\hm-\mu)^{\mathrm{T}}\Sigma^{-1} (y\hm-\mu)\hm= 
\rho$, то параметр~$\rho$ удовлетворяет уравнению $\mathrm{Pr}\left\{ 
\chi_d^2\leq \rho\right\} \hm=1\hm-\alpha$.
     
     Использование эллипсоида в~качестве MRR будет оправдано только 
тогда, когда исходное распределение данных есть многомерное нормаль-\linebreak ное. 
Поэтому становятся актуальными критерии\linebreak подгонки, а~также использование 
процедур норма\-ли\-зации распределения данных в~многомерном\linebreak случае.
 Если 
с~помощью тестов выявляется, что распределение не является нормальным, то 
Международная федерация клинической химии и~лабораторной медицины 
рекомендует, согласно~[3], использовать двухступенчатую процедуру 
нормализации. Следует обратить внимание, что многошаговость здесь 
относится не к~многомерности, а касается лишь покоординатного 
преобразования распределения данных к~нормальному.
     
     Первые же попытки применения MRR на основе расстояния 
Махалонобиса (фактически это означает принятие модели нормального 
распределения референсных значений) выявили ряд недостатков (более 
подробно смотри в~\cite[разд.~6.2]{4-kri}):
     \begin{itemize}
\item проявление <<проклятий>> размерности при механическом 
увеличении~$d$, в~особенности если игнорируется этап анализа состава 
признаков~[1, 5, 6];
\item из-за небольших объемов обучающей выборки невысокая устойчивость 
при применении, в~частности чувствительность к~увеличению неточностей 
измерений после того, как регион был установлен~\cite{5-kri, 7-kri}. 
\item предположение о нормальном распределении и~попытки <<подправить>> 
действительность с~помощью преобразований реальных данных для их 
нормализации при увеличении размерности данных становятся все более 
шаткими~\cite{5-kri};
\item представление и~интерпретация выводов на основе MRR трудно 
понимаемы не только для специалистов в~предметной области~[8].
\end{itemize}

\vspace*{-9pt}

\section{Многомерный референсный регион высокой плотности}

\vspace*{-2pt}

     Заметим, что в~случае нормального распределения референсных значений 
для точек внут\-ри построенного эллипсоида значения плотности\linebreak распределения 
больше, чем на границе, а~вне~--- меньше. Это замечание позволяет 
предложить другой подход к~построению MRR.
     
     \smallskip
     
     \noindent
     \textbf{Определение.}\ Eсли плотность распределения референсных 
значений есть $f(y)$, то MRR есть область $A_t\hm= \left\{ y\in 
\mathcal{R}^d\vert f(y)\hm\geq t\right\}$ для некоторого порогового 
значения~$t$. 
     
     \smallskip
     
     Для нормального распределения это уже упомянутый эллипсоид равной 
вероятности. Если задается вероятность $(1\hm-\alpha)$ попадания в~$A_t$, то 
пороговое значение~$t$ есть решение уравнения $\int\nolimits_{A_t} 
f(u)\,du\hm=1\hm-\alpha$, получить которое аналитически в~случае 
произвольной плотности распределения вряд ли возможно. Здесь присутствуют 
две проблемы: вычисление многомерного интеграла и~зависимость области 
интегрирования от неизвестного значения. Для решения их предлагается 
привлечь метод моделирования.
     
     Сгенерируем выборку из $f(y)$, которую обозначим как $Y^f\hm= \left\{ 
y_1^f, \ldots, y_m^f\right\}$. Для оценки $\int\nolimits_{A_t} f(u)\,du$ 
используем отношение:

\noindent
\begin{multline*}
     \fr{\left\vert \left\{ y_i^f\vert y_i^f\in A_t\right\}\right\vert }{m} =
      \fr{\left\vert\left\{ y_i^f\vert 
f\left(y_i^f\right) \geq t\right\}\right\vert }{m} ={}\\
{}= 1-\fr{\left\vert \left\{ y_i^f\vert f(y_i^f)<t\right\}\right\vert }{m}=1-
F_m(t)\,,
     \end{multline*}
где $F_m(t)$~--- эмпирическая функция распределения случайной 
величины~$f(y)$, т.\,е.\ случайной величины, являющейся результатом 
преобразования с~помощью функции~$f(\cdot)$ случайной величины, име\-ющей 
плотность распределения~$f(u)$.

     Таким образом, искомая оценка~$t^*$ должна удовле\-тво\-рять уравнению 
$F_m(t^*)\hm=\alpha$ и~может быть получена как непараметрическая оценка 
квантиля\linebreak\vspace*{-12pt}

\pagebreak

\noindent
 порядка~$\alpha$ из распределения $F_m(\cdot)$. Если обозначить 
$f_i\hm= f(y_i^f)$, то~$t^*$ есть~$f_{(r)}$, где
     $$
     r= \begin{cases}
     m\alpha, &\ m\alpha~\mbox{---~целое}\,;\\
     \lfloor m\alpha+1\rfloor\,, & m\alpha~\mbox{--- не целое}\,.
     \end{cases}
     $$
     Заметим, что для такой оценки можно указать доверительный интервал.
     
     Для построения MRR необходимо знать распределение данных. При 
реализации принципа точек высокой плотности в~первую очередь следует 
обратиться к~параметрическим моделям, в~част\-ности к~смеси нормальных 
распределений, име\-ющей плотность распределения
     $$
     f(u) =\sum\limits_{j=1}^k p_j \varphi\left (u,\mu_j, \Sigma_j\right)\,.
     $$
Если $\hat{f}(u)$~--- оценка смеси, то~$t^*$ строится сле\-ду\-ющим образом:
\begin{itemize}
\item генерируется выборка $\left\{ y_1^f,\ldots , y_m^f\right\}$ из $\hat{f}(u)$ и~
для каждого ее $i$-го элемента подсчитывается значение $\hat{f}\left( 
y_i^f\right)$;
\item в~качестве~$t^*$ берется непараметрическая оценка квантиля 
порядка~$\alpha$ (в случае необходимости дополнительно находится 
непараметрическая оценка доверительного интервала для~$t^*$, что 
может характеризовать правильность выбранного объема для 
генерируемой выборки).
\end{itemize}

     Пусть для $f(u)$ имеется~$A_t$, а также получена $\hat{f}(u)$ 
и~соответствующий MRR вида~$\hat{A}_t$. Качество аппроксимации~$A_t$ 
с~по\-мощью~$\hat{A}_t$ можно оценить с~по\-мощью вероятности совпадения 
этих областей, т.\,е. 
     $$
     P_c= \int\limits_{\{ u\in A_t\}\cup \{u\in \hat{A}_t\}} \hspace*{-6mm}
f(u)\,du+\int\limits_{\{u\not\in A_t\} \cup\{ u\not\in \hat{A}_t\}}\hspace*{-6mm} f(u)\,du\,.
     $$
     
     Для оценки  $P_c$ можно использовать величину
     \begin{multline*}
     \hat{P}_c= \fr{\left\vert \left\{ 
     y_i^f\vert y_i^f \in \left\{\left\{ y_i^f\in A_t\right\}\cup \left\{y_i^f\in 
\hat{A}_t\right\}\right\}\right\}\right\vert}{m}+{}\\
{}+\fr{\left\vert \left\{ y_i^f\vert y_i^f \in \left\{\left\{ y_i^f\not\in A_t\right\}\cup 
\left\{ y_i^f\not\in \hat{A}_t\right\}\right\}\right\}\right\vert}{m}\,.
     \end{multline*}
     
     Использование MRR высокой плотности для диагностирования сводится 
к~реализации так называемого слабого критерия значимости для наблюденного 
значения~$x$: нулевая гипотеза заключается в~том, что $x\hm\in A_t$, 
статистика критерия есть $\hat{f}(x)$ и~решение о~принадлежности 
критической об\-ласти~$A_t$ принимается при больших значениях~$\hat{f}(x)$.
     
     Для медицинской практики важна возможность использования 
референсного региона при интерпретации результатов обследования 
некоторого пациента с~вектором признаков~$x$. В~подобных случаях 
сложившейся практикой для слабых критериев значимости является 
использование критического уровня~$\alpha_{\mathrm{cr}}$ (более распространенным 
в~медицине является употребление термина $p$-зна\-че\-ние)  $\alpha_{\mathrm{cr}}\hm= 
\mathrm{Pr}\left\{ \hat{f}(y)\hm\leq \hat{f}(x)\right\}$, где $y$~--- случайная 
величина, имеющая плотность распределения~$\hat{f}(u)$, а $\hat{f}(x)$~--- 
значение плотности распределения~$\hat{f}(u)$ в~точке~$x$. Эта 
характеристика дает представление о~том, насколько сильно данное 
наблюденное значение~$x$ противоречит гипотезе (или подкрепляет ее) 
о~принадлежности данных MRR. При выбранном же заранее уровне 
значимости с~помощью~$\alpha_{\mathrm{cr}}$ сразу же можно принять конкретное 
решение. 

\vspace*{-9pt}

\section{Эксперименты}

\vspace*{-2pt}

     Для демонстрации возможностей MRR использовались данные по 
прогнозу химического состава мочевых камней по метаболическим 
показателям мочи и~сыворотки крови, а также антропологическим 
характеристикам пациентов~[9]. В качестве исходной классификации камней 
рассматривалась следующая: чисто оксалатные (далее обозначены как O), чисто 
уратные (U), чисто фосфатные (P), смесь только оксалатных и~уратных (OU), 
смесь только оксалатных и~фосфатных (OP), смесь только уратных 
и~фосфатных (UP), все остальные. Данная классификация была построена 
в~[10] на основе доминирующих частот встречаемости основных компонентов. 
В~качестве референсных значений рассматривались наборы метаболических 
и~антропологических показателей (их всего было~14), соответствующих 
определенному классу камней.

\begin{table*}\small
\begin{center}


\begin{tabular}{|c|c|c|c|c|c|c|}
\multicolumn{7}{c}{Качество классификации с~помощью MRR}\\
\multicolumn{7}{c}{\ }\\[-6pt]
\hline
\multicolumn{1}{|c|}{\raisebox{-6pt}[0pt][0pt]{\tabcolsep=0pt\begin{tabular}{c}Тип\\ камня\end{tabular}}}&
\multicolumn{1}{c|}{\raisebox{-6pt}[0pt][0pt]{$N$}}&$(1-\alpha)$, 
&\multicolumn{2}{c|}{MRR(5)}&\multicolumn{2}{c|}{MRR(1)}\\
\cline{4-7}
&&&&&&\\[-9pt]
&&\%&$(1-\hat{\alpha})$, \%&$\hat{\beta}$, \%&$(1-\hat{\alpha})$, \%&$\hat{\beta}$, \%\\
\hline
\multicolumn{1}{|c|}{\raisebox{-18pt}[0pt][0pt]{O}}&
\multicolumn{1}{c|}{\raisebox{-18pt}[0pt][0pt]{82}}
&95&100\hphantom{9}&71&90&24\\
&&85&96&78&89&36\\
&&75&91&85&77&44\\
&&65&76&88&74&50\\
\hline
\multicolumn{1}{|c|}{\raisebox{-18pt}[0pt][0pt]{U}}&
\multicolumn{1}{c|}{\raisebox{-18pt}[0pt][0pt]{76}}&95&100\hphantom{9}&75&91&24\\
&&85&99&85&80&35\\
&&75&82&89&74&48\\
&&65&71&91&68&56\\
\hline
\multicolumn{1}{|c|}{\raisebox{-18pt}[0pt][0pt]{P}}&
\multicolumn{1}{c|}{\raisebox{-18pt}[0pt][0pt]{83}}&95&100\hphantom{9}&66&87&25\\
&&85&94&78&86&33\\
&&75&86&82&82&41\\
&&65&77&87&75&47\\
\hline
\end{tabular}
\end{center}
\end{table*}
     
     
     Для каждого из основных классов O, U, P, OU, OP и~UP перед построением 
MRR проводилась селекция признаков и~принималось то значение размерности 
признакового пространства~$d$ и~соответствующий набор показателей, 
которые позволяли прогнозировать состав камней без потери качества 
(методика описана в~\cite{9-kri} и~привела к~значению $d\hm=9$). В~качестве 
модели данных в~первую очередь рассматривалась смесь многомерных 
нормальных распределений из пяти элементов (подбор числа элементов смеси 
проводился с~по\-мощью AIC~--- Akaike information criterion), для соответствующего региона было принято 
обозначение MRR(5). Для сравнения также использовалась модель 
нормального распределения, которой соответствовал MRR(1). Полученные 
результаты приводятся час\-тич\-но в~таблице, где $N$~--- объем 
классифицируемых данных; $\hat{\alpha}$~--- оценка для~$\alpha$; 
$\hat{\beta}$~--- оценка мощности критерия при определении типа камня на 
основании MRR.


     Одной из базовых характеристик является вероятность попадания в~MRR 
$(1\hm-\alpha)$ и~ее оценка $(1\hm-\hat{\alpha})$. Сравнение соответствующих 
столбцов с~учетом значений~$N$ и~ориентировочных значений разброса 
(стандартные отклонения на основе биномиального распределения) не 
позволило выявить явных отклонений. Необходимо, правда, отметить, что во 
всех проанализированных случаях для MRR(5) оказалось, что $1\hm-
\hat{\alpha}\hm\geq 1\hm-\alpha$.
     
     Назначение MRR, заключающееся в~сжатом представлении референсных 
значений, в~многомерном случае практически не проявляется. Для задания 
MRR(5) необходимо указать следующие величины: $1\hm-\alpha$, $t$, 
$p_1,\ldots, p_{k-1}$, $\mu_1, \Sigma_1,\ldots , \mu_k,\Sigma_k$, общее 
количество которых равно  $[2\hm+ (k\hm-1)\hm+ k(d\hm+ d(d\hm+1)/2)]$ 
и,~в~частности, в~рассматриваемых экспериментах~--- 276. Для MRR(1) это 
значение меньше и~равно~56. При этом для обрабатываемой обучающей 
выборки в~зависимости от класса камней речь идет о~порядка~10$^2$ векторах 
данных (см.\ столбец со значениями~$N$), что приблизительно 
дает~10$^3$~скалярных величин.
     
     Другое назначение MRR состоит в~его использовании для 
диагностирования (классификации). В~этой связи в~первую очередь 
проводился сравнительный анализ MRR(1) (фактически это означает, что 
построение региона осуществляется на основе расстояния Махаланобиса) 
и~MRR(5) (модель смеси нормальных распределений и~предложенный 
в~данной работе метод оценивания па\-ра\-мет\-ров региона). Показателем 
информативности метода построения многомерного региона выступала 
мощность соответствующего слабого критерия значимости, а~именно: 
вероятность не попасть в~MRR при условии, что данные берутся из дополнения 
к~классу, для которого построена MRR. Сравнение соответствующих столбцов 
говорит о~явном преимуществе двух предложенных моментов: усложнение 
модели данных путем перехода от нормального распределения к~смеси 
нормальных распределений и~построение региона высокой плотности.
     
     Использование критического уровня можно продемонстрировать  
с~по\-мощью зависимости результатов сравнения двух классов от того, какой 
класс взять за основу. Введем для возможных значений $p$-ве\-ли\-чи\-ны три 
интервала: $(-\infty, 1\%)$, $[1\%, 5\%)$, $[5\%, 100\%)$ с~соответствующей 
интерпретацией положения наблюденного набора показателей для пациента 
относительно построенного MRR: уверенное непопадание, неуверенное 
попадание, уверенное попадание. Если MRR построить для оксалатных камней, 
то результаты для анализа пациентов с~фосфатными камнями дадут следующий 
вектор относительных частот попадания $p$-ве\-ли\-чин в~указанные 
интервалы: $(60\%, 18\%, 22\%)$. Если же MRR строить для фосфатных 
камней, то получим $(71\%, 5\%, 24\%)$. Таким образом, для классификации 
указанных камней при приблизительно одинаковых частотах попадания в~MRR 
(22\% или~24\%) уверенный отказ от референсного региона происходит чаще, 
если принять за базовый MRR регион для фосфатных камней. Построение 
шкалы, подобной рассмотренной, является прерогативой специалистов 
в~предметной области, в~данной работе она использовалась только для 
иллюстрации. 

\vspace*{-6pt}

\section{Заключение}

\vspace*{-2pt}

     На настоящий момент имеется относительно мало примеров применения 
MRR в~клинической практике. Тому есть несколько причин. Математическое 
обеспечение, необходимое для получения и~применения MRR, не отвечает 
возможностям большинства клинических лабораторий. Лаборатории слабо 
оснащены программными средствами\linebreak для реализации достаточно сложного 
математического аппарата многомерного анализа, а~еще важнее, что 
отсутствуют методики, инструкции по\linebreak использованию соответствующих 
средств. Лишь немногие клинические применения демонстрируют 
преимущества MRR, хотя свидетельств неудачных попыток больше.
     
     Несмотря на сложности внедрения мно\-го\-мерно\-го анализа референсных 
значений, можно сформулировать некоторые рекомендации по иссле\-до\-ва\-нию 
и~разработке MRR. Во-пер\-вых, эффективная размерность в~MRR должна 
быть как можно меньше, чтобы избежать затенения диагностически полезной 
информации тестами, со\-зда\-ющи\-ми шум. Низкая размерность также должна 
уменьшить неблагоприятные последствия увеличения неточности результатов 
в~связи с~ростом числа анализируемых показателей. Во-вто\-рых, показатели 
(тес\-ты), включенные в~MRR, должны быть физиологически релевантными 
исследуемому кругу расстройств, чтобы максимизировать информацию, 
полученную от MRR. В-треть\-их, чтобы учесть эффекты долгосрочной 
лабораторной из\-мен\-чи\-вости, данные, используемые для получения MRR, 
долж\-ны быть собраны и~проанализированы в~течение достаточно большого 
периода времени (от нескольких недель до нескольких месяцев).  
В-чет\-вер\-тых, представление результатов лабораторных исследований 
следует осуществлять в~графическом виде, чтобы помочь врачам лучше понять 
MRR. Различные подходы к~уменьшению размерности помогут выполнить это 
требование.
     
     Необходима дальнейшая разработка пояснительных инструментов, 
способных воспринять результаты анализа MRR. При этом дополнительно 
необходима информация о~том, какие именно тес\-ты являются важнейшими 
факторами нарушения нормы. Надо признать, что соответствующий 
математический аппарат еще предстоит разработать. Решение перечисленных 
вопросов играет важную роль для обеспечения постоянного клинического 
применения MRR. 

\vspace*{-6pt}
     
{\small\frenchspacing
 {%\baselineskip=10.8pt
 \addcontentsline{toc}{section}{References}
 \begin{thebibliography}{99}
 
 \vspace*{-2pt}
 
\bibitem{1-kri}
\Au{Boyd J.\,C.} Reference regions of two or more dimensions~// Clin. Chem. Lab. 
Med., 2004. Vol.~42. No.\,7. P.~739--746.
\bibitem{2-kri}
\Au{Winkel P.} Patterns and clusters~--- multivariate approach for interpreting 
clinical chemistry results~// Clin. Chem., 1973. Vol.~19. No.\,12. P.~1329--1333.
\bibitem{3-kri}
IFCC. Expert panel on theory of reference values. Approved recommendation on the 
theory of reference values. Part~5. Statistical treatment of collected reference values. 
Determination of reference limits~// J.~Clin. Chem. Clin. Biochem., 1987. Vol.~25. 
No.\,9. P.~645--656.
\bibitem{4-kri}
\Au{Кривенко М.\,П.} Статистические методы представления и~предварительной 
обработки референсных значений.~--- М.: ФИЦ ИУ РАН, 2016. 160~с.
\bibitem{5-kri}
\Au{Boyd J.\,C., Lacher~D.\,A.} The multivariate reference range: An alternative 
interpretation of multi-test profiles~// Clin. Chem., 1982. Vol.~28. No.\,2.  
P.~259--265.
\bibitem{6-kri}
\Au{Albert A., Harris~E.\,K.} Multivariate interpretation of clinical laboratory  
data.~--- New York, NY, USA: CRC Press, 1987. 328~p.
\bibitem{7-kri}
\Au{Linnet K.} Influence of sampling variation and analytical errors on the 
performance of the multivariate reference region~// Meth. Inf. Med., 1988. Vol.~27. 
No.\,1. P.~37--42.
\bibitem{8-kri}
\Au{Durbridge T.\,C.} Clinical acceptance of a multi-test reference region for 
biochemical-panel results~// Clin. Chem., 1983. Vol.~29. No.\,10. P.~1724--1726.
\bibitem{9-kri}
\Au{Кривенко М.\,П.} Критерии значимости отбора признаков классификации~// 
Информатика и~её применения, 2016. Т.~10. Вып.~3. С.~32--40.
\bibitem{10-kri}
\Au{Кривенко М.\,П., Голованов~С.\,А., Сивков~А.\,В.} Анализ однородности 
данных о химическом составе камней при уролитиазе~// Информатика и~её 
применения, 2013. Т.~7. Вып.~4. С.~94--104.
 \end{thebibliography}

 }
 }

\end{multicols}

\vspace*{-10pt}

\hfill{\small\textit{Поступила в~редакцию 5.12.16}}

\vspace*{4pt}

%\newpage

%\vspace*{-24pt}

\hrule

\vspace*{2pt}

\hrule

\vspace*{-3pt}


\def\tit{HIGH-DENSITY MULTIVARIATE REFERENCE REGION\\[-5pt]}

\def\titkol{High-density multivariate reference region}

\def\aut{M.\,P.~Krivenko\\[-7pt]}

\def\autkol{M.\,P.~Krivenko}

\titel{\tit}{\aut}{\autkol}{\titkol}

\vspace*{-16pt}


\noindent
Institute of Informatics Problems, Federal Research Center 
``Computer Science and Control'' of the Russian
Academy of Sciences,  44-2~Vavilov Str., Moscow 119333, Russian Federation



\def\leftfootline{\small{\textbf{\thepage}
\hfill INFORMATIKA I EE PRIMENENIYA~--- INFORMATICS AND
APPLICATIONS\ \ \ 2017\ \ \ volume~11\ \ \ issue\ 2}
}%
 \def\rightfootline{\small{INFORMATIKA I EE PRIMENENIYA~---
INFORMATICS AND APPLICATIONS\ \ \ 2017\ \ \ volume~11\ \ \ issue\ 2
\hfill \textbf{\thepage}}}

\vspace*{2pt}




\Abste{The paper considers the principles of construction of multivariate 
reference regions. An original method of construction of 
a~region on the basis of areas of high density of points and approximation 
of data distribution with a~mixture of normal distributions is suggested. 
To estimate the threshold for the probability density, the bootstrap method is used. 
As an experiment, the paper considers the problem of description and use of 
the reference region for predicting the type of urinary stones. 
Real data treatment demonstrated the benefits of the proposed solutions.}

\KWE{multivariate reference region; high-density region; bootstrap method; 
multivariate normal mixture}

\DOI{10.14357/19922264170207} 

%\vspace*{-18pt}

%\Ack
%\noindent



%\vspace*{3pt}

  \begin{multicols}{2}

\renewcommand{\bibname}{\protect\rmfamily References}
%\renewcommand{\bibname}{\large\protect\rm References}

{\small\frenchspacing
 {%\baselineskip=10.8pt
 \addcontentsline{toc}{section}{References}
 \begin{thebibliography}{99}
\bibitem{1-kri-1}
\Aue{Boyd, J.\,C.} 2004. Reference regions of two or more dimensions. \textit{Clin. 
Chem. Lab. Med.} 42(7):739--746.

\bibitem{2-kri-1}
\Aue{Winkel, P.} 1973. Patterns and clusters~--- multivariate approach for interpreting 
clinical chemistry results. \textit{Clin. Chem.} 19(12):1329--1333.
\bibitem{3-kri-1}
IFCC. 1987. Expert panel on theory of reference values. Approved recommendation on the 
theory of reference values. Part~5. Statistical treatment of collected reference values. 
Determination of reference limits. \textit{J.~Clin. Chem. Clin. Biochem.} 
25(9):645--656.
\bibitem{4-kri-1}
\Aue{Krivenko, M.\,P.} 2016. \textit{Statisticheskie metody predstavleniya 
i~predvaritel'noy obrabotki referensnykh znacheniy}
[Statistical methods for representation and preliminary processing of
reference values]. Moscow: FRC CSC RAS. 160~p.

\bibitem{5-kri-1}
\Aue{Boyd, J.\,C., and D.\,A.~Lacher.} 1982. The multivariate reference range: An 
alternative interpretation of multi-test profiles. \textit{Clin. Chem.}  
28(2):259--265.
\bibitem{6-kri-1}
\Aue{Albert, A., and E.\,K.~Harris.} 1987. \textit{Multivariate interpretation of 
clinical laboratory data}. New York, NY: CRC Press. 328~p.
\bibitem{7-kri-1}
\Aue{Linnet, K.} 1988. Influence of sampling variation and analytical errors on the 
performance of the multivariate reference region. \textit{Meth. Inf. Med.}  
27(1):37--42.
\bibitem{8-kri-1}
\Aue{Durbridge, T.\,C.} 1983. Clinical acceptance of a multi-test reference region 
for biochemical-panel results. \textit{Clin. Chem.} 29(10):1724--1726.
\bibitem{9-kri-1}
\Aue{Krivenko, M.\,P.} 2016. Kriterii znachimosti otbora priznakov klassifikatsii
[Significance tests of feature selection for~classification]. \textit{Informatika i~ee 
Primeneniya~--- Inform. Appl.} 10(3):32--40.
\bibitem{10-kri-1}
\Aue{Krivenko, M.\,P., S.\,A.~Golovanov, and A.\,V.~Sivkov}. 2013. Analiz 
odnorodnosti dannykh o~khimicheskom sostave kamney pri urolitiaze
[Analysis of data homogeneity of~the~chemical compositions 
of~stones in~case of~urolithiasis]. \textit{Informatika i~ee Primeneniya~---
Inform Appl.} 7(4):94--104.
\end{thebibliography}

 }
 }

\end{multicols}

\vspace*{-3pt}

\hfill{\small\textit{Received December 5, 2016}}


\Contrl

\noindent
\textbf{Krivenko Michail P.} (b.\ 1946)~--- Doctor of Science in technology, 
professor, leading scientist, Institute of Informatics Problems, Federal Research 
Center ``Computer Science and Control'' of the Russian Academy of Sciences, 
\mbox{44-2}~Vavilov Str., Moscow 119333, Russian Federation; \mbox{mkrivenko@ipiran.ru}

\label{end\stat}


\renewcommand{\bibname}{\protect\rm Литература}  %8
\def\stat{flerov}

\def\tit{АВТОМАТИЗИРОВАННАЯ СИСТЕМА ВЕСОВОГО 
ПРОЕКТИРОВАНИЯ САМОЛЕТОВ}

\def\titkol{Автоматизированная система весового 
проектирования самолетов}

\def\aut{Л.\,Л.~Вышинский$^1$, Ю.\,А.~Флеров$^2$, Н.\,И.~Широков$^1$}

\def\autkol{Л.\,Л.~Вышинский, Ю.\,А.~Флеров, Н.\,И.~Широков}

\titel{\tit}{\aut}{\autkol}{\titkol}

\index{Вышинский Л.\,Л.}
\index{Флеров Ю.\,А.}
\index{Широков Н.\,И.}
\index{Vyshinsky L.\,L.}
\index{Flerov Yu.\,A.}
\index{Shirokov N.\,I.}




%{\renewcommand{\thefootnote}{\fnsymbol{footnote}} \footnotetext[1]
%{Работа выполнена при финансовой поддержке РФФИ (проект 17-01-00816).}}


\renewcommand{\thefootnote}{\arabic{footnote}}
\footnotetext[1]{Вычислительный центр им.\ А.\,А.~Дородницына Федерального исследовательского 
центра <<Информатика и~управ\-ле\-ние>> Российской академии наук, 
\mbox{Wysh@ccas.ru}}
\footnotetext[2]{Вычислительный центр им.\ А.\,А.~Дородницына Федерального исследовательского 
центра <<Информатика и~управ\-ле\-ние>> Российской академии наук, 
fler@ccas.ru}
%\footnotetext[3]{Вычислительный центр им.\ А.\,А.~Дородницына Федерального исследовательского 
%центра <<Информатика и~управ\-ле\-ние>> Российской академии наук, 
%\mbox{Wysh@ccas.ru}}

%\vspace*{-6pt}


 
  \Abst{Статья посвящена вопросам автоматизации задач весового проектирования 
самолетов. Весовые и~мас\-со\-во-инер\-ци\-он\-ные параметры являются одними из основных 
величин, влияющих на эксплуатационные характеристики самолетов. Информационной 
основой системы служит весовая модель самолета. Описывается структура весовой 
модели и~даны характеристики отдельным ее компонентам. Показана программная 
реализация системы, которая выполнена в~рамках архитектуры кли\-ент--сер\-вер. 
Автоматизированная система весового проектирования (АСВП)
реализована с~использованием 
про\-грам\-мно-ин\-стру\-мен\-таль\-но\-го комплекса <<Генератор проектов>> (технология ГП), 
который был разработан в~Вычислительном центре Российской академии наук. Создание 
ин\-фор\-ма\-ци\-он\-но-вы\-чис\-ли\-тель\-ных сис\-тем в~рамках технологии ГП базируется на так 
называемом <<проектном подходе>>, когда по формальному описанию системы автоматически 
генерируются тексты программного кода для клиентских и~серверных компонент системы.}
   
  \KW{математическое моделирование; автоматизация проектирования; самолет; весовое 
проектирование; весовая модель; дерево конструкции; генератор проектов; генерация 
программного кода; архитектура кли\-ент--сер\-вер}

  \DOI{10.14357/19922264180103} 
  
\vspace*{12pt}


\vskip 10pt plus 9pt minus 6pt

\thispagestyle{headings}

\begin{multicols}{2}

\label{st\stat}
   
\section{Введение}

  Развитие и~повсеместное использование информационных технологий за 
последние несколько десятилетий существенно изменили традиционный 
процесс проектирования и~создания различных инженерных систем, 
сооружений, машин. Во многих проектных организациях давно отказались от 
ко\-гда-то привычных инструментов конструктора~--- кульмана 
и~логарифмической линейки. 
%
Сейчас первые эскизы новых проектов 
появляются чаще не на бумаге, как было всегда, а~на экране монитора. Этому 
способствует широкий спектр имеющихся систем автоматизированного 
проектирования. В~российских авиационных конструкторских бюро, например, уже давно 
применяются такие CAD (computer aided design)
сис\-те\-мы, как NX (Unigraphics), CATIA и~др. 
%
Эти развитые системы геометрического трех\-мер\-но\-го (3D) мо\-де\-ли\-ро\-ва\-ния позволяют 
создавать сложные по\-верх\-ности, конструировать любые детали, осуществлять 
сборку узлов, агрегатов и~сложнейших изделий. Однако построение 
геометрических моделей изделий является финальной стадией проектирования, 
за которой следует их реализация <<в~металле>>. Построению электронных 
геометрических макетов предшествует и~сопутствует решение множества 
расчетных задач, а~также задач анализа и~оптимизации в~разных областях инженерных 
знаний. В~авиастроении это аэродинамика, динамика полета, прочность, 
системы управления, двигателестроение и~пр. Все эти задачи 
требуют разработки разноплановых математических моделей и~специальных 
вычислительных программ. 
  
  Одной из важнейших технических характеристик самолета является его вес. 
При решении подавляющего большинства проектных и~конструкторских задач 
весовые параметры в~том или ином виде участвуют в~расчетах. Необходимость 
проведения весовых расчетов возникает на самых ранних шагах 
проектирования и~сопровождает все дальнейшие стадии разработки 
и~эксплуатации. 

В~процессе создания и~эксплуатации самолетов постоянно 
контролируются вес и~другие мас\-со\-во-инер\-ци\-он\-ные характеристики (МИХ)
всех размещаемых на борту систем, агрегатов, узлов и~деталей. Количество 
агрегатов, узлов и~деталей современных самолетов исчисляется 
десятками тысяч, поэтому в~авиастроении весовые расчеты, весовой анализ, 
весовой контроль выливаются в~сложную инженерную проблему и~выделяются 
в~целое направление инженерной деятельности, которое принято называть 
весовым проектированием~[1].
  
  Данная статья посвящена вопросам автоматизации задач весового 
проектирования самолетов. В~разные годы Вычислительным центром РАН\linebreak был 
разработан и~внедрен в~эксплуатацию ряд \mbox{программ}, решающих отдельные 
задачи весовых рас\-че\-тов летательных аппаратов (ЛА)~[2--4]. В~настоящей статье 
представлено описание интегрированной АСВП, предназначенной для использования на всех 
этапах жизненного цикла изделий. Она разработана как интерактивная 
многопользовательская информационная система кли\-ент-сер\-вер\-ной 
архитектуры с~централизованной базой данных. Информационным ядром 
и~основой АСВП является единая струк\-тур\-но-па\-ра\-мет\-ри\-че\-ская весовая модель 
самолета, описание которой дает довольно полное представление о~задачах, 
решаемых с~помощью АСВП.

\section{Структурно-параметрическая весовая модель самолета}

  Самолет является сложным техническим объ\-ектом, состоящим из множества 
различных \mbox{ком\-понентов}, функционально и~конструктивно связанных между 
собой. Под струк\-тур\-но-па\-ра\-мет\-ри\-че\-ской весовой моделью самолета 
здесь понимается база данных, которая содержит всю необходи\-мую 
информацию для проведения комплекса расчетов 
МИХ и~мас\-со\-во-цент\-ро\-воч\-ных данных (МЦД) 
самолета. Весовая модель состоит из нескольких структур, ориентированных на 
определенные группы параметров и~задач весового проектирования. Ниже 
перечислены основные структуры весовой модели, реализованные в~системе 
АСВП:
  \begin{itemize}
\item дерево конструкции самолета;
\item иерархия систем координат, связанных с~самолетом и~его агрегатами;
\item геометрические структуры весовой модели самолета;
\item каталог целевой нагрузки, размещаемой во внут\-рен\-них отсеках и~на 
подвесках;
\item реестр допустимых вариантов загрузки само\-лета;
\item таблицы тарировочных характеристик топливных баков;
\item таблицы характеристик выработки топлива.
\end{itemize}


  \subsection{Дерево конструкции самолета}

  Дерево конструкции самолета является центральной структурой весовой 
модели, которая отражает членение изделия на его составные части~--- 
системы, агрегаты, узлы, детали. В~базе данных весовой модели эта структура 
представлена в~виде многоуровневого корневого дерева $W \hm= (U, V)$, где 
вершинам $U \hm= \{U_i\}$ соответствуют различные\linebreak
 элементы конструкции. 
Ориентированные дуги дере\-ва, идущие из~$U_i$ в~$U_j$, означают вхождение 
конструкции~$U_j$ в~конструкцию~$U_i$ в~качестве ее составной части. 
Терминальными или висячими вершинами дерева конструкции будем называть 
вершины, у которых нет ни одной подчиненной конструкции.
  
  Многолетний опыт самолетостроения выработал устоявшиеся 
конструктивные схемы самолетов различного назначения. Существуют 
отраслевые стандарты и~классификаторы, которые вводят определения 
основных элементов конструкции самолетов. На рис.~1 показан пример 
представления в~АСВП нескольких верхних уровней дерева конструкции 
самолета. 


    

  Существующие классификаторы отражают лишь самые общие принципы 
построения конструкции самолетов. Разумеется, каждый новый проект 
самолета имеет свои конструктивные особенности, которые отражаются на 
структуре весовой модели. Дерево конструкции строится постепенно, сверху 
вниз, в~течение всего процесса проектирования самолета. 

 { \begin{center}  %fig1
 \vspace*{9pt}
\mbox{%
 \epsfxsize=77.216mm 
 \epsfbox{fle-1.eps}
 }

\vspace*{6pt}


\noindent
{{\figurename~1}\ \ \small{Дерево конструкции самолета}}
\end{center}
}

\addtocounter{figure}{1}
  
  Понятие <<конструкции>> в~данном контексте используется и~для 
обозначения любой вершины графа, и~для всего поддерева подчиненных 
конструкций этой вершине. Каждая конструкция дерева имеет уникальное 
в~рамках весовой модели обозначение, которое может быть шифром, кодом, 
идентификатором или чертежным номером конструкции. Разумеется, для более 
полного и~наглядного вербального представления конструкции  
в~струк\-тур\-но-па\-ра\-мет\-ри\-че\-ской модели можно задать ее текстовое 
описание.
  
  \textbf{Масса конструкции.} Основную содержательную и~необходимую 
информацию весовой модели содержит список значений  
МИХ, соответствующих каждой 
вершине дерева конструкций. Центральным параметром является масса. 
  
  На разных стадиях создания самолета, когда неизвестно точное значение 
массы, прибегают к~различным оценкам.  
В~струк\-тур\-но-па\-ра\-мет\-ри\-че\-ской весовой модели фиксируются 
перечисленные ниже оценки массы, которые соответствуют разным этапам 
проектирования:
  \begin{description}
\item[\,]  $M_{\mathrm{теор}}$~--- теоретическая масса~--- оценка массы, вычисленная на 
основании некоторой математической модели конструкции; 
  
\item[\,]  $M_{\mathrm{лим}}$~--- лимитная масса конструкции, уста\-нав\-ли\-ва\-емая на 
основании теоретических оценок и~используемая для весового контроля 
в~процессе детальной разработки конструкции;
  
\item[\,]  $M_{\mathrm{черт}}$~--- чертежная масса конструкции, рассчитанная по чертежу или по 
электронной гео\-мет\-ри\-че\-ской модели конструкции;
  
\item[\,]  $M_{\mathrm{креп}}$~--- масса крепежа конструкции~--- дополнение к~чертежной массе, 
которое учитывает мелкие детали конструкции, предназначенные для 
соединения подчиненных деталей (заклепки, болты, гайки, шайбы и~т.\,п.). 
Введение такой дополнительной массы позволяет избавить дерево конструкции 
от десятков и~сотен тысяч вершин, которые несут относительно небольшую 
нагрузку в~весовых характеристиках, но тем не менее их учет необходим при 
контроле веса. Масса крепежа распределяется по подчиненным конструкциям;  
\item[\,]  $M_{\mathrm{факт}}$~--- фактическая масса изготовленной 
и~взвешенной конструкции. 
Фактическая масса может задаваться не только для изготавливаемых 
конструкций ЛА, но и~для готовых по\-став\-ля\-емых 
изделий при их установке на борту.
\end{description}
  
  Порядок задания оценок массы диктуется логикой развития проекта. 
В~дереве конструкции все оценки массы, кроме $M_{\mathrm{лим}}$ и~$M_{\mathrm{креп}}$, 
суммируются по подчиненным вершинам снизу вверх. Однако если для 
некоторых терминальных значений одна или несколько оценок не определены, 
например некоторые детали конструкции не изготовлены и, стало быть, 
$M_{\mathrm{факт}}$ не определена, то и~для всех вышестоящих конструкций эти оценки не 
определены. При задании $M_{\mathrm{лим}}$ это правило может не соблюдаться. 
  
  На основании оценок массы определяется то расчетное значение массы 
конструкции, которое используется во всех расчетах на текущей стадии 
проекта: 
  $M$~--- текущая масса конструкции. Значение текущей массы \textit{для 
нетерминальных} конструкций определяется суммированием по подчиненным 
конструкциям. \textit{Для терминальных} вершин дерева конструкций 
применяется процедура определения текущей массы по первому известному 
значению из следующего списка в~указанном порядке: $M_{\mathrm{факт}}$, 
$M_{\mathrm{черт}}$\;+\;$M_{\mathrm{креп}}$, $M_{\mathrm{теор}}$, $M_{\mathrm{лим}}$.
  
  \textbf{Геометрия масс конструкции.} Кроме собственно массы в~весовой 
модели задаются или вычисляются значения характеристик, которые принято 
называть характеристиками геометрии масс: 
  \begin{description}
  \item[\,] $X$, $Y$ и $Z$~--- положение центра масс конструкции; 
  \item[\,] $L_x$, $L_y$ и $L_z$~--- габаритные параметры конструкции;
  \item[\,] $I_x$, $I_y$ и $I_z$~--- полные плоскостные моменты инерции;
  \item[\,]  $I_{xy}$, $I_{xz}$ и $I_{yz}$~--- полные центробежные моменты 
инерции;
  \item[\,] $I^c_x$, $I^c_y$ и  $I^c_z$~--- собственные плоскостные моменты 
инерции:
  \begin{align*}
  I^c_x &= I_x - M X^2\,;\\ 
  I^c_y &= I_y - M Y^2\,;\\ 
  I^c_z &= I_z - M Z^2\,;
 \end{align*}
  \item[\,] $I^c_{xy}$, $I^c_{xz}$ и~$I^c_{yz}$~--- собственные центробежные 
моменты инерции:
 \begin{align*}
  I^c_{xy} &= I_{xy}- M X Y\,;\\
   I^c_{xz} &= I_{xz}- M X Z\,;\\
   I^c_{yz} &= I_{yz}- M Y Z\,;
\end{align*}
  \item[\,] $J_x$, $J_y$ и $J_z$~--- собственные осевые моменты инерции 
конструкции:
  \begin{align*}
  J_x &= I^c_y + I^c_z\,;\\ 
  J_y &= I^c_x + I^c_z\,;\\
   J_z &= I^c_y + I^c_x\,;
  \end{align*}
  \item[\,] СК~--- система координат конструкции, в~которой задаются 
характеристики геометрии масс.
  \end{description}
  
  \begin{figure*} %fig2
  \vspace*{1pt}
 \begin{center}
 \mbox{%
 \epsfxsize=162mm 
 \epsfbox{fle-2.eps}
 }
 \end{center}
\vspace*{-9pt}
  \Caption{Основные параметры конструкций весовой модели самолета}
  \end{figure*}
  
  Каждая конструкция привязывается к~одной из систем координат, которые 
описаны в~весовой модели. В~весовой модели изделия для удобства описания 
различных агрегатов может быть описано несколько систем координат. Все 
описанные сис\-те\-мы координат организованы в~иерархическую структуру. 
Считается предописанной глобальная самолетная система координат, в~которой 
могут быть заданы или вычислены координаты всех объектов весовой модели. 
Каждая система координат в~весовой модели задается уникальным именем, 
положением начала координат относительно вышестоящей системы координат 
и~тремя углами поворота относительно вышестоящей. 

Параметр, 
обозначенный как СК,~--- это имя одной из сис\-тем координат весовой модели. 
Если СК не задано, то считается, что характеристики гео\-мет\-рии масс заданы 
в~глобальной системе координат. Каж\-дая сис\-те\-ма координат весовой модели 
содержит матрицу преобразования координат из самолетной (глобальной) 
системы координат в~данную и~обратно. Эта матрица для каждой системы 
координат есть произведение локальных матриц преобразований 
в~соответствии с~положением данной системы в~иерархии систем координат. 
Любое изменение параметров систем координат требует пе\-ре\-вы\-чис\-ле\-ния 
матриц преобразования как измененной сис\-те\-мы, так и~всех подчиненных. На 
рис.~2 показана панель параметрического пред\-став\-ле\-ния конструкций весовой 
модели.
  
  Так же как и~масса, центры тяжести и~моменты инерции вычисляются снизу
вверх от терминальных конструкций к~вышестоящим. При этом осуществляется 
пересчет характеристик по заданной иерархии систем координат от 
нижестоящих к~вышестоящим и~к~самолетной системе координат. Расчет 
МИХ терминальных конструкций 
осуществляется на основании гео\-мет\-ри\-че\-ских моделей. Геометрические модели 
на этапе рабочего проекта строятся в~системах гео\-мет\-ри\-че\-ско\-го 
моделирования. В~процессе их построения автоматически вычисляются 
объемы, массы, положение центра тяжести и~другие характеристики гео\-мет\-рии 
масс. Рассчитанная в~системах гео\-мет\-ри\-че\-ско\-го моделирования масса 
с~по\-мощью соответствующих интерфейсных средств может быть загружена как 
$M_{\mathrm{черт}}$ в~весовую модель. (Раньше документация была представлена в~виде 
чертежей на бумажных носителях и~$M_{\mathrm{черт}}$ вручную вычислялась по этим 
чертежам.) Однако на более ранних этапах проектирования, когда еще не 
проработана гео\-мет\-рия многих элементов конструкции, весовые расчеты 
проводятся на основании эскизов и~наборов гео\-мет\-ри\-че\-ских и~конструктивных 
параметров агрегатов изделия. Для этого в~весовой модели должны быть 
предусмотрены средства параметрического представления гео\-мет\-рии 
конструкций. Геометрическое пред\-став\-ле\-ние конструкций 
в~автоматизированной системе весового проектирования выполняет 
и~немаловажную функцию визуализации конструкций, их компоновки, 
размещения нагрузки и~т.\,д. В~АСВП реализовано несколько форм 
представления гео\-мет\-ри\-че\-ской информации, ориентированных не только на 
расчет МИХ, но и~на визуализацию выполняемых расчетов. Это чертежи 
геометрических проекций изделия, это таб\-лич\-ное задание типовых 
геометрических конструкций, это каркасное представление трехмерных 
геометрических моделей и, наконец, задание объемных конструкций 
триангуляционной (фасеточной) поверхностью. Последний вид представления 
является наиболее перспективным для точного вычисления МИХ. В~АСВП он 
применяется для расчета тарировочных характеристик топливных баков, о~чем 
будет сказано ниже.
  
  \textbf{Классификационные признаки конструкции.} В~весовой модели 
кроме числовых параметров опре\-делен ряд классификационных признаков 
конструкций, по которым проводится весовой анализ.\linebreak
 Таки\-ми маркерами могут 
быть подразделения, ответст\-вен\-ные за разработку конструкции, поставщики 
или изготовители готовых изделий, принадлежность конструкции 
к~определенным функциональным системам, конструкционные материалы 
и~пр.
  
  \textbf{Функциональные подсистемы изделия.} Конст\-рук\-тив\-ное членение 
самолета может не совпадать с~его функциональной структурой. Отдельные\linebreak 
элементы функциональных подсистем самолета удобнее описывать в~составе 
конструкции ка\-ко\-го-ни\-будь агрегата планера. Например, некоторая деталь 
может конструктивно входить в~состав крыла, а принадлежать 
к~функциональной подсистеме гидравлики или электрооборудования. Для того 
чтобы иметь возможность выполнять весовые расчеты, учитывая разные 
подходы к~классификации конструкции самолета, в~АСВП отдельно от дерева 
конструкции ведется реестр подсистем, для которых может быть проведен 
специальный расчет весовых параметров. В~этом реестре ведется полный 
перечень конструкций весовой модели, входящих в~подсистемы реестра, 
независимо от того, в~какой ветви дерева конструкции они находятся. Любая 
конструкция может быть включена только в~одну из подсистем реестра. 
В~зависимости от режима расчетов МИХ
конструкций, входящих в~под\-сис\-те\-му, будут учтены либо в~со\-ста\-ве 
вышестоящих агрегатов дерева конструкции, либо отдельно в~под\-сис\-теме. 
{\looseness=1

}
  
  \textbf{Распределенные характеристики изделия.} Задача вычисления 
распределенных характеристик изделия является родственной задачей 
вычисления характеристик геометрии масс. Основное отличие состоит в~том, 
что в~данной задаче рассчитываются не интегральные характеристики 
распределения материала, а сама функция распределения массы по объему 
конструкции. Такие функции рассчитываются по заданному геометрическому 
разбиению конструкции на пространственные отсеки. Функции распределения 
массы по объему конструкции в~процессе проектирования используются при 
построении динамически подобных моделей для проведения некоторых видов 
испытаний и~продувок, а~также для выполнения прочностных расчетов. 
  
  Каждый отсек разбиения для расчета распределенных характеристик 
представляет собой вы\-пук\-лый многогранник, ограниченный конечным набором 
плоскостей. Задача построения распределенных весовых характеристик состоит 
в~вычислении для каждого отсека массы и~положения центра тяжести той части 
конструкции самолета, которая геометрически расположена внутри этого 
отсека. Эта задача решается путем нахождения геометрического пересечения 
каждой терминальной конструкции с~каждым отсеком разбиения, и~если такое 
пересечение не пусто, то вычисление массы и~центра тяжести той части 
конструкции, которая попадает в~отсек. Некоторые конструкции могут быть 
объявлены сосредоточенными массами. Использование сосредоточенных масс 
позволяет исключить все подчиненные конструкции из распределения по 
отсекам и~рассматривать их отдельно для задания сосредоточенных нагрузок. 
Список сосредоточенных масс с~уникальными именами представляет собой 
отдельную структуру весовой модели. Каждая сосредоточенная масса содержит 
список ссылок на конструкции весовой модели. Любая конструкция может 
быть включена не более чем в~одну сосредоточенную массу.
  
  \textbf{Весовые сводки.} Одной из основных задач \mbox{АСВП} является 
построение так называемых весовых сводок. Весовые сводки являются 
документами, сопровождающими построение весовой модели самолета 
в~процессе его создания. В АСВП реализовано несколько форм весовых 
сводок, которые с~разных сторон отражают дерево конструкции самолета или 
отдельных ветвей этого дерева. Назначение этих сводок и~форма представления 
зависят от ре\-ша\-емых задач. Весовые данные в~сводках могут быть 
представлены либо в~табличном виде, либо в~виде иерархии конструкций. 
Могут содержать информацию в~детализированном или в~укрупненном виде. 
Отдельные виды весовых сводок пред\-став\-ля\-ют распределенные 
характеристики по отсекам. Весовые сводки предназначены для решения задач 
весового контроля и~весового анализа. 
  
  Весовой контроль при проектировании самолетов представляет собой  
ор\-га\-ни\-за\-ци\-он\-но-тех\-ническую сис\-те\-му, нацеленную на создание 
конструк\-ции минимального веса. Для эффективного \mbox{весового} контроля 
необходима оперативная информация о текущей массе изделия и~любой его 
части. Весовая информация для весового контроля в~АСВП представляется 
в~виде оперативных весовых сводок по отдельным подразделениям 
предприятия. В~таких весовых сводках содержится информация о текущей, 
теоретической, лимитной,\linebreak чертежной и~фактической массах конструкций, 
разрабатываемых данным подразделением. Могут также выпускаться 
оперативные сводки по группе подразделений или по всему проекту. Сводки 
весового контроля предназначены для использования руководителями проекта.
  
  Весовой анализ также связан с~выпуском определенного вида весовых 
сводок. Для решения задач весового анализа в~АСВП осуществляется 
сортировка и~выборки конструкций по определенному классификационному 
признаку. Например, могут быть рассчитаны массы силового и~несилового 
набора конструкции, массы продольного и~поперечного набора, массы 
конструкций определенного материала, массы готовых изделий или изделий 
конкретного поставщика и~т.\,д. Весовой контроль и~анализ позволяют 
выявить резервы конструкции, узкие места, тренды в~изменении веса 
кон\-ст\-рук\-ции.
{\looseness=1

}
  
  \subsection{Постоянные и~переменные структуры весовой модели 
самолета}
  
  Дерево конструкции весовой модели готового изделия не является 
статической структурой. Конфигурация самолета зависит от конкретных 
условий его применения. Мас\-со\-во-инер\-ци\-он\-ные характеристики при 
взлете и~посадке отличаются от тех же характеристик в~полете, когда убраны 
стойки шасси. Конфигурация меняется и~в~полете у~самолетов с~изменяемым 
углом стреловидности или с~измененяемым вектором тяги. Текущая 
конфигурация является одним из параметров весовой модели и~параметров 
весовых расчетов. По самому смыс\-лу создания самолета как транспортного 
средства предполагается, что кроме собственно конструкции, которая 
обеспечивает выполнение основных задач, на его  
МИХ существенным образом влияет 
перевозимая нагрузка. Перевозимая нагрузка есть переменная часть структуры 
дерева конструкции. Самолетные весовые классификаторы выделяют 
постоянную часть массы изделия и~переменную, состоящую из снаряжения, 
топлива и~целевой нагрузки:
  \begin{multline*}
{M} = M_{\mathrm{пустого}} + 
M_{\mathrm{снаряжения}} + {}\\
{}+M_{\mathrm{топлива}} + 
M_{\mathrm{целевой\_нагрузки}}\,.
  \end{multline*}
  
  Все переменные и~постоянные компоненты самолета составляют единое 
целое, и~расчет мас\-со\-во-инер\-ци\-он\-ных и~центровочных характеристик 
допусти\-мых конфигураций является одной из главных задач проектирования 
самолетов любого назначения. Переменные структуры в~весовой модели могут 
задаваться альтернативными конструкциями, когда у некоторой вершины 
дерева есть несколько вариантов поддеревьев и~когда любой из вариантов, но 
только один из них, может быть активирован в~конкретный момент времени. 
Существует своя специфика задания переменных структур весовой модели для 
разных содержательных задач. 
  
  \textbf{Пустой самолет}~--- это постоянная часть конструкции самолета, 
которая не меняется в~процессе эксплуатации готового изделия. Компонентами 
пустого самолета являются конструкция планера самолета, силовая установка 
и~ее системы, другие самолетные системы, обеспечивающие управление 
самолетом, а~также специальные системы бортового оборудования, 
предназначенные для решения целевых задач самолета. В~процессе 
проектирования и~при эксплуатации самолетов рассматриваются различные 
варианты отдельных конструкций планера, а~чаще~--- различные варианты 
по\-став\-ля\-емых готовых изделий. В~связи с~этим в~весовой модели АСВП 
рассматриваются возможные комбинации вариантов пустого самолета, 
вариантов снаряжения и~полезной нагрузки. 

\begin{figure*} %fig3
\vspace*{1pt}
 \begin{center}
 \mbox{%
 \epsfxsize=162mm 
 \epsfbox{fle-3.eps}
 }
 \end{center}
\vspace*{-9pt}
\Caption{Тарировочная таблица топливного бака}
\end{figure*}
  
  \textbf{Снаряжение самолета} устанавливается на борту в~процессе 
предполетной подготовки. Снаряжение самолета принято разделять на 
основное и~дополнительное. Основное снаряжение включает несколько 
позиций. Это экипаж и~системы жизнеобеспечения экипажа, системы 
жизнеобеспечения пассажиров, заправляемые компоненты и~расходуемые 
материалы, несливаемый остаток топлива и~другие возможные компоненты. 
Использование различных вариантов экипажа и~другого снаряжения самолета 
связано с~различным характером выполняемых задач. Как правило, существует 
несколько типовых вариантов комплектации экипажа 
и~элементов снаряжения. Весовая модель должна содержать перечень 
альтернативных вариантов снаряжения и~их характеристик. Естественно, что 
этот перечень может модифицироваться. К~дополнительному снаряжению 
относят временное оборудование и~средства, связанные с~установкой на борту 
и~закреплением на подвесках перевозимых грузов. Временно устанавливаемое 
оборудование, как правило, связано со спецификой полетных заданий. Это 
может быть специальная измерительная аппаратура или оборудование, которое 
необходимо проверить в~условиях реального полета. Перечень такого 
оборудования и~его характеристики в~весовой модели должны быть 
пред\-став\-ле\-ны в~специальном реестре, или в~каталоге. Для установки 
оборудования, размещения любой коммерческой нагрузки и~вооружения в~конструкции самолета
должны быть  предусмотрены специальные места 
размещения и~узлы крепления. Точки размещения оборудования и~любых 
элементов целевой нагрузки задаются своими координатами и~установочными 
углами закрепления. 

\begin{figure*} %fig4
  \vspace*{1pt}
 \begin{center}
 \mbox{%
 \epsfxsize=162mm 
 \epsfbox{fle-4.eps}
 }
 \end{center}
\vspace*{-11pt}
\Caption{Варианты размещения целевой нагрузки самолета на подвесках}
\end{figure*}
  
  \textbf{Топливо}~--- величина переменная и~на земле, при подготовке 
самолета к~вылету, и~в~воздухе, при выработке топлива, и, если это 
предусмотрено, при дозаправке в~воздухе. Одной из самых сложных и~важных 
задач построения весовой модели является отражение изменяющихся в~полете  
МИХ топлива, находящегося 
в~топливных баках. Топливные баки современных ЛА
могут иметь довольно сложные геометрические формы. В~процессе выработки 
топлива все характеристики располагаемого запаса топлива меняются. 
Необходимо отслеживать эти изменения в~процессе произвольных допустимых 
эволюций траектории полета. Функции изменения МИХ в~зависимости от 
объема оставшегося топлива задаются тарировочными характеристиками баков. 
Для расчета тарировочных характеристик топливных баков при произвольных 
углах атаки, углах тангажа и~крена в~весовой модели наиболее удобно 
триангуляционное задание баков. В~тарировочной таблице вычисляется масса 
оставшегося топлива в~зависимости от уровня поверхности жидкости 
в~топливном баке. На рис.~3 приведен пример расчета тарировочной таблицы 
крыльевого топливного бака.



  Если МИХ топлива в~конкретном баке по 
мере его выработки определяются тарировочной характеристикой, то 
зависимость МИХ оставшегося топлива определяется последовательностью, 
в~которой осуществляется выработка из разных баков. Топливная система 
самолета состоит из нескольких баков~--- как внутренних, так и~размещенных 
на подвесках, а~также из системы трубопроводов, перекачивающих насосов и~управляющей автоматики. Основой управления расходом топлива является 
программа, определяющая порядок расходования топлива из разных баков. 
Переключение перекачки топлива между разными баками осуществляется для 
обеспечения центровки самолета в~заданных границах. Одним из критериев при 
разработке алгоритмов перекачки является число переключений и~обеспечение 
бесперебойной подачи топлива при любых допустимых параметрах траектории 
полета. Массово-инерционные характеристики топлива в~процессе тарировки 
баков задаются их разбиением плоскопараллельными сечениями на тонкие 
слои. Для каждого слоя указывается масса, координаты центра тяжести 
и~плоскостные моменты инерции. Программа выработки топлива пред\-став\-ля\-ет 
собой последовательность выработки слоев из разных баков в~соответствии 
с~диаграммой переключений. В~весовой модели может быть задано несколько 
вариантов программ расходования топлива. Разумеется, в~процессе выполнения 
полетного задания программа расходования топлива фиксирована. 
Предварительный расчет характеристик для разных вариантов порядка 
выработки топлива необходим для выбора наилучшего, удовле\-тво\-ря\-юще\-го 
всем ограничениям.
  
  \textbf{Целевая нагрузка} зависит от назначения самолета и~от конкретного 
полетного задания. Для пасса\-жирских самолетов целевая нагрузка~--- это 
пассажи\-ры с~багажом, для транспортных са\-мо\-летов~--- это коммерческие 
грузы, для военных~--- подвесное или размещаемое в~специальных \mbox{отсеках} 
вооружение. В~полете возможен сброс и~десантирование целевой нагрузки. 
Комплектация и~установка целевой нагрузки представляет собой довольно 
сложный процесс. Выбор состава грузов и~их размещение могут проходить 
в~несколько этапов. Сложность выбора обусловлена большим количеством 
типов перевозимой нагрузки, наличием большого числа вспомогательных 
специальных устройств закрепления грузов как во внутренних отсеках 
самолета, так и~на внешних подвесках. На рис.~4 приведена панель 
формирования различных расчетных вариантов целевой нагрузки самолета. 
Визуализация этого процесса существенно облегчает решение различных задач 
анализа допустимой нагрузки как на этапе проектирования самолета, так и~при 
эксплуатации во время подготовки полетных заданий.
  
  \begin{figure*} %fig5
\vspace*{1pt}
 \begin{center}
 \mbox{%
 \epsfxsize=162mm 
 \epsfbox{fle-5.eps}
 }
 \end{center}
\vspace*{-9pt}
\Caption{Область допустимых центровок}
\end{figure*}

  Для удобства выбора и~проведения расчетов множества вариантов загрузки 
самолета в~рамках весовой модели реализованы каталоги нагрузки~--- 
специального оборудования, коммерческой нагрузки, вооружения. В~этих 
каталогах ведутся клас\-си\-фи\-ка\-то\-ры, позволяющие в~громадных переч\-нях 
находить нужные позиции и~их характеристики. Кроме  
МИХ размещаемой нагрузки в~каталогах 
даются ссылки на их геометрические модели, задаются габариты, другие 
геометрические па\-ра\-мет\-ры. Эти данные нужны для визуализации размещения 
и~компоновки, для вычисления их МИХ. 
Как правило, существуют довольно жесткие ограничения на 
размещение нагрузки на борту, а~также на внешних узлах крепления. Эти 
ограничения должны указываться в~каталоге и~учитываться в~процессе 
формирования вариантов загрузки самолета. 
  
  Ограничения, которые проверяются при анализе различных вариантов 
снаряжения самолета, программы выработки топлива и~допустимых вариантов 
целевой нагрузки, задают область допустимых центровок самолета. 
  
  \textbf{Область допустимых центровок} является неотъемлемой частью 
весовой модели и~служит одной из основных весовых характеристик самолета, 
особенно важной и~контролируемой в~процессе его эксплуатации. На рис.~5 
проиллюстрированы ограничения, образующие область допустимых центровок, 
и~приведен график изменения центровки самолета при выработке топлива. 



  По оси абсцисс на этом графике откладывается центровка самолета, которая 
определяется как положение центра тяжести самолета на средней 
аэро\-ди\-на\-ми\-че\-ской хорде в~процентах от ее длины. По оси ординат 
откладывается текущая масса самолета с~учетом массы снаряжения, массы 
целевой нагрузки и~текущего запаса топлива. Точки излома на графиках 
центровки соответствуют моментам переключения подачи топлива с~одного 
бака на\linebreak другой, которые определяются программой выработки топлива или 
моментами сброса целевой нагрузки. Двойной график изменения центровки 
соответствует двум полетным конфигурациям~--- с~убранными 
и~выпущенными стойками шасси. Ограничения, которые обеспечивают 
устой\-чи\-вость и~управ\-ля\-емость полета, задаются предельными значениями 
центровки. Предельно передняя и~предельно задняя центровки на графике 
показаны вертикаль\-ными штриховыми линиями. Горизонтальные линии задают 
ограничения на взлетную и~посадочную массы. Ограничения максимальной 
взлетной и~посадочной массы при определенных условиях могут нарушаться, 
но эти нарушения допускаются в~исключительных условиях и~сказываются на 
ресурсных характеристиках самолета.\linebreak Превышение \textbf{предельных} 
значений взлетной и~посадочной массы не допускается. Наклонные штриховые 
линии на графике задают ограничения, связанные с~максимально допустимыми 
нагрузками на переднюю и~главную опоры шасси.  

\begin{figure*} %fig6
\vspace*{1pt}
 \begin{center}
 \mbox{%
 \epsfxsize=165mm 
 \epsfbox{fle-6.eps}
 }
 \end{center}
\vspace*{-9pt}
\Caption{Архитектура программной реализации исполнительных модулей АСВП}
\end{figure*} 

%\vspace*{-12pt}

\section{Программная реализация автоматизированной системы весового
проектирования}

  Представленная здесь струк\-тур\-но-па\-ра\-мет\-ри\-че\-ская весовая модель 
самолета позволяет решать широкий круг задач весового проектирования. 
Весовая модель составляет информационную основу,\linebreak на базе которой могут 
быть построены различные вычислительные программы и~пользовательские 
модули. Рассматриваемая в~данной работе АСВП построена по 
кли\-ент-сер\-вер\-ной архитектуре, где весовая модель служит единым хранилищем 
информации, базой данных системы. Непосредственно с~информацией, 
хранящейся в~этой базе данных, взаимодействуют различные вычислительные, 
расчетные программы~--- серверы, которые кроме расчетных функций 
обеспечивают информационную связь клиентских модулей с~весовой моделью 
самолета. Непосредственными пользователями клиентских модулей являются 
конструкторы и~проектировщики, решающие различные задачи весового 
проектирования.  Построена АСВП как многопользовательская интерактивная 
система. На рис.~6 представлена архитектура АСВП, ее основные программные 
и~информационные компоненты.




  Ниже перечислены основные функции программных модулей АСВП:
 \begin{description}
 \item[\,] 
Сервер ПУСТОЙ ЛА\;+\;Модуль расчета МИХ пус\-то\-го самолета:
\begin{itemize}
\item создание и~модификация дерева конструкции пустого самолета;
\item расчет МИХ пустого изделия, всех его сис\-тем, узлов, агрегатов и~деталей 
на любых уровнях дерева конструкции;
\item весовой анализ и~контроль текущего состояния проекта, выполнения 
лимитных ограничений по весу, осуществление выборок весовой информации 
по различным признакам~--- сис\-те\-мам, агрегатам, типу конструкции 
(си\-ло\-вая/не\-си\-ло\-вая),  материалу конструкции, подразделениям и~т.\,д.;
\item расчет распределения массы самолета по различным разбиениям на 
отсеки; эта информация используется для построения динамически подобных 
моделей и~при прочностных расчетах;
\item расчет МИХ при различных вариантах полетной конфигурации при 
убранных и~выпущенных стойках шасси, при отклонениях консолей крыла для 
самолетов с~из\-ме\-ня\-емой геометрией, при отклонении органов управления.
\end{itemize}
\begin{figure*} %fig7
\vspace*{1pt}
 \begin{center}
 \mbox{%
 \epsfxsize=155.86mm 
 \epsfbox{fle-7.eps}
 }
 \end{center}
\vspace*{-1pt}
\Caption{Проектный подход~--- технология ГП}
\vspace*{6pt}
\end{figure*}
 \item[\,]
Сервер НАГРУЗКА ЛА\;+\;Модуль расчета МИХ самолета с~переменной 
массой:
\begin{itemize}
\item создание и~модификация реестра допустимых вариантов нагрузки 
самолета;
\item расчеты МИХ снаряженного и~загруженного самолета для разных 
вариантов компоновки и~размещения на борту полезной нагрузки;
\item расчет изменения МИХ самолета в~полете при выработке топлива, 
дозаправке в~воздухе, сбросе нагрузки;
\item расчет МИХ самолета в~виде табличных зависимостей для различных 
вариантов снаряжения и~размещения нагрузки;
\item расчет МИХ самолета в~виде графических зависимостей от массы 
самолета и/или от массы топлива;
\item проверка выполнения установленных эксплуатационных ограничений по 
центровке, взлетной и~посадочной массе, нагрузке на опоры шасси для 
различных вариантов снаряжения и~размещения нагрузки; сигнализация 
в~случае нарушения ограничений, а~также для различных вариантов программ 
выработки топлива.
\end{itemize}

\pagebreak

 \item[\,]
Сервер КАТАЛОГ\;+\;Модуль ведения каталога элементов нагрузки:\\[-9pt]
\begin{itemize}
\item создание и~модификация каталога элементов целевой нагрузки самолета;\\[-9pt]
\item создание и~модификация базы данных вариантов размещения 
и~закрепления элементов нагрузки каталога на борту самолета или на подвесках;\\[-9pt]
\item создание и~модификация базы данных вспомогательных элементов 
конструкции установки элементов нагрузки.\\[-9pt]
\end{itemize}
 \item[\,]
Сервер ТОПЛИВО\;+\;Модуль расчета порядка выработки топлива:\\[-9pt]
\begin{itemize}
\item создание и~модификация базы данных различных вариантов программы 
выработки топлива;\\[-9pt]
\item расчет МИХ и~МЦД для различных вариантов переключения выработки 
топлива из внутренних, закладных и~подвесных баков;\\[-9pt]
\item расчет МИХ и~МЦД при различных программах заливки и~дозаправки 
топлива во внутренние, закладные и~подвесные баки.\\[-9pt]
\end{itemize}
 \item[\,]
Сервер БАКИ\;+\;Модуль расчета тарировки топливных баков:\\[-9pt]
\begin{itemize}
\item создание и~модификация базы данных гео\-мет\-рии топливных баков;\\[-9pt]
\item расчет тарировочных характеристик топливных баков при различных 
углах тангажа и~крена.\\[-9pt]
  \end{itemize}
  \end{description}
  
  Программная реализация АСВП велась с~использованием инструментального комплекса 
<<Генератор проектов>> (технология ГП)~\cite{5-fl}. Технология ГП 
обеспечивает возможность разработки приклад\-ных систем многоуровневой  
кли\-ент-сер\-вер\-ной архитектуры с~использованием реляционных и~сетевых 
баз данных со сложным пользовательским и~межпрограммным интерфейсом. 
Создание ин\-фор\-ма\-ци\-он\-но-вы\-чис\-ли\-тель\-ных сис\-тем в~рамках 
технологии ГП базируется на так называемом <<проектном подходе>>. Под 
проектом здесь понимается пакет документов (файлов), содержащий описание 
структуры проекта, описание логической структуры баз данных, спецификации 
пользовательского интерфейса, перечень команд и~сценарии работы 
пользователей, описание функций и~процедур обработки пользовательских 
запросов. Исходное описание проекта подается на вход <<Генератора 
проекта>>, который строит в~памяти модель проекта, осуществляет ее анализ 
на предмет корректности и~целостности, а затем на основании этой модели 
генерирует тексты программного кода для клиентских и~серверных компонент 
системы, а~так\-же ге\-нерирует утилиты, необходимые для сборки, инсталляции 
и~сопровождения системы. 

На рис.~7 показана общая архитектура 
программной конструкции, связанной с~применением технологии ГП.
  


  В приведенной цепочке разработчик прикладной информационной системы 
имеет дело только с~первым ее звеном~--- проектом системы. При этом он 
избавлен от необходимости иметь дело с~системным программным окружением 
вычислительной среды, в~которой должна функционировать разрабатываемая 
прикладная система. Все связи прикладных информационных процессов 
с~конкретной системной вычислительной средой привносит 
в~результирующую рабочую программу <<Генератор проектов>> на стадии 
анализа и~генерации итогового программного кода. Естественно, что при этом 
объем описания проекта оказывается существенно короче программного кода, 
который создается автоматически. Экономия трудозатрат разработчика 
оказывается существенной. В~частности, объем описания проекта АСВП на 
порядок меньше, чем объем сгенерированного программного кода. Даже если 
предположить, что написанный вручную программный код благодаря 
искусству программистов будет весьма экономным, то все равно трудоемкость 
разработки прикладных систем будет в~разы меньше. 

Но главное даже не 
в~числе строк программ, а~прежде всего в~экономии интеллектуальных затрат 
разработчиков прикладных систем и,~в~итоге, автоматически созданные 
программы более надежны и~свободны от нечаянных ошибок и~опечаток.\linebreak 
И~кроме того, разрабатываемые в~рамках технологии ГП прикладные системы 
обеспечивают-\linebreak\vspace*{-12pt}

\pagebreak

\noindent
ся эффективными средствами сопровождения, т.\,е.\linebreak достаточно 
простой процедурой внесения ис\-прав\-ле\-ний и~развития программ в~процессе их 
эксплу\-а\-тации. 

Прикладные программные комплексы в~рамках технологии ГП 
разрабатываются как автономные системы и~не требуют для своей работы 
специальной среды и~дорогостоящих программных продуктов (кроме 
использующихся систем управления базами данных
(СУБД) и~общесистемного обеспечения). Разрабатываемые 
в~рамках технологии ГП прикладные системы допускают масштабирование 
и~портирование на различные вычислительные платформы и~СУБД.
  
  \bigskip
  
  Как уже говорилось, система АСВП разрабатывалась в~течение ряда лет, 
многие ее компоненты и~версии были апробированы и~использовались 
в~реальном проектировании. 
%
Авторы выражают благодарность 
С.\,И.~Скобелеву, М.\,К.~Курьянскому, Д.\,Ю.~Стрель\-цу, П.\,В.~Плунскому 
и~К.\,Н.~Ерасову за плодотворные обсуждения проблем весового проектирования 
самолетов, за постановку многих задач и~за апробацию разработанных 
программ.

%\vspace*{-12pt}

{\small\frenchspacing
 {%\baselineskip=10.8pt
 \addcontentsline{toc}{section}{References}
 \begin{thebibliography}{9}
\bibitem{1-fl}
\Au{Шейнин В.\,М., Козловский~В.\,И.} Весовое проектирование и~эффективность 
пассажирских самолетов.~--- М.: Машиностроение, 1977.   Т.~1. 343~с.

%\columnbreak

\bibitem{2-fl}
\Au{Скобелев С.\,И., Широков~Н.\,И.} Весовой анализ и~контроль в~САПР ЛА~// Задачи 
и~методы автоматизированного проектирования.~--- М.: ВЦ РАН, 1991. С.~92--100.
\bibitem{3-fl}
\Au{Широков Н.\,И.} Автоматизированная система весовых расчетов в~САПР ЛА~// 
Автоматизация проектирования инженерных и~финансовых информационных систем 
средствами Генератора проектов~/ Отв. ред. Ю.\,А.~Флеров.~--- М.: ВЦ РАН, 
2010. С.~55--66.

\vspace*{6pt}

\bibitem{4-fl}
\Au{Вышинский Л.\,Л., Широков~Н.\,И.} Система автоматизации расчетов 
массово-инерционных характеристик ЛА с~переменной массой~// Развитие и~применение 
инструментального комплекса Генератор проектов~/ Отв. ред. Ю.\,А.~Флеров.~--- 
М.: ВЦ РАН, 2014. С.~20--31.
{\looseness=1

}

\vspace*{6pt}

\bibitem{5-fl}
\Au{Вышинский Л.\,Л., Гринев~И.\,Л., Флеров~Ю.\,А., Широков~А.\,Н., Широков~Н.\,И.} 
Генератор проектов~--- инструментальный комплекс для разработки  
<<кли\-ент-сер\-вер\-ных>> сис\-тем~// Информационные технологии и~вычислительные 
системы, 2003. №\,1-2. С.~6--25.
 \end{thebibliography}

 }
 }

\end{multicols}

\vspace*{-6pt}

\hfill{\small\textit{Поступила в~редакцию 24.05.17}}

\vspace*{8pt}

%\newpage

%\vspace*{-24pt}

\hrule

\vspace*{2pt}

\hrule

%\vspace*{8pt}


\def\tit{COMPUTER-AIDED SYSTEM OF~AIRCRAFT WEIGHT DESIGN}

\def\titkol{Computer-aided system of~aircraft weight design}

\def\aut{L.\,L.~Vyshinsky, Yu.\,A.~Flerov, and~N.\,I.~Shirokov}

\def\autkol{L.\,L.~Vyshinsky, Yu.\,A.~Flerov, and~N.\,I.~Shirokov}

\titel{\tit}{\aut}{\autkol}{\titkol}

\vspace*{-9pt}


\noindent
A.\,A.~Dorodnicyn Computing Centre, Federal Research Center ``Computer Science and 
Control'' of the Russian Academy of Sciences,  40~Vavilov Str., Moscow 119333, Russian 
Federation 



\def\leftfootline{\small{\textbf{\thepage}
\hfill INFORMATIKA I EE PRIMENENIYA~--- INFORMATICS AND
APPLICATIONS\ \ \ 2018\ \ \ volume~12\ \ \ issue\ 1}
}%
 \def\rightfootline{\small{INFORMATIKA I EE PRIMENENIYA~---
INFORMATICS AND APPLICATIONS\ \ \ 2018\ \ \ volume~12\ \ \ issue\ 1
\hfill \textbf{\thepage}}}

\vspace*{3pt}
   

\Abste{The article is devoted to the problems of computer-aided weight design of 
aircraft. Weight and mass-inertial parameters are one of the basic values that affect 
the performance characteristics of aircraft. The informational basis of the system is 
the weight model of the aircraft. The paper describes the structure of the weight 
model and its individual components. The program implementation of the system, 
which is executed within the framework of the client-server architecture, is shown. 
The automated system of weight design is implemented using the software tool 
complex ``Project Generator'' (GP technology), which was developed at the 
Computing Centre of the Russian Academy of Sciences. The creation of information 
and computing systems within the framework of the GP technology is based on the 
so-called ``project approach,'' when the formal description of the system 
automatically generates code for the client and server components of the system.}

\KWE{math modeling; design automation; aircraft; weight design; weighting model; 
design tree; project generator; code generation; client-server architecture}

  \DOI{10.14357/19922264180103} 

%\vspace*{-12pt}

%\Ack
%\noindent




%\vspace*{8pt}

  \begin{multicols}{2}

\renewcommand{\bibname}{\protect\rmfamily References}
%\renewcommand{\bibname}{\large\protect\rm References}

{\small\frenchspacing
 {%\baselineskip=10.8pt
 \addcontentsline{toc}{section}{References}
 \begin{thebibliography}{9} 
 
 %\vspace*{-6pt}
 
 \bibitem{1-fl-1}
\Au{Sheynin, V.\,M., and V.\,I.~Kozlovskiy}. 1977. \textit{Vesovoe 
proektirovanie i~effektivnost' passazhirskikh samoletov} [Weight design and 
efficiency of passenger aircraft]. Moscow: Mechanical Engineering. Vol.~1. 343~p.
\bibitem{2-fl-1}
\Aue{Skobelev, S.\,I., and N.\,I.~Shirokov.} 1991. Vesovoy analiz i~kontrol' v~SAPR 
LA [Weight analysis and control in CAD of aircraft]. \textit{Zadachi i~metody 
avtomatizirovannogo proektirovaniya} [Tasks and methods of computer-aided 
design]. Moscow: Computing Centre of the USSR Academy of Sciences.  
92--100.
\bibitem{3-fl-1}
\Aue{Shirokov, N.\,I.} 2010. Avtomatizirovannaya sistema vesovykh raschetov 
v~SAPR LA [Automated system weight calculations in CAD].  
\textit{Avtomatizatsiya proektirovaniya inzhenernykh i~finansovykh 
informatsionnykh system sredsvami Generatora proektov} [Computer 
aided  design of engineering and financial information systems by the means of the 
Project Generator]. Moscow: Computing Centre of RAS. 
55--66.
\bibitem{4-fl-1}
\Aue{Vyshinskiy, L.\,L., and N.\,I.~Shirokov.} 2014. Sistema avtomatizatsii 
raschetov massovo-inertsionnykh kharakteristik LA s~peremennoy massoy [CAD 
system of calculation  aircraft mass-inertial characteristics with variable mass].  
\textit{Razvitie i~primenenie instrumental'nogo kompleksa Generator proektov} 
[The development and application of a tool set Project Generator]. 
Moscow: Computing Centre of RAS. 20--31.
{\looseness=1

}

\bibitem{5-fl-1}
\Aue{Vyshinskiy, L.\,L., I.\,L.~Grinev, Yu.\,A.~Flerov, A.\,N.~Shirokov, and 
N.\,I.~Shirokov.} 2003. Generator proektov~--- instrumental'nyy kompleks dlya 
razrabotki ``klient--servernykh'' sistem [The project generator~--- tool complex for 
development of ``client--server'' systems]. 
\textit{Informatsionnye tekhnologii i~vychislitel'nye sistemy} [Information 
Technologies and Computer Systems] 1-2:6--25.

\end{thebibliography}

 }
 }

\end{multicols}

\vspace*{-6pt}

\hfill{\small\textit{Received May 24, 2017}}

%\vspace*{-10pt}

\Contr

\noindent
\textbf{Vyshinsky Leonid L.} (b.\ 1941)~--- Candidate of Sciences (PhD) in physics and 
mathematics, Head of Laboratory, A.\,A.~Dorodnicyn Computing 
Centre, Federal Research Center ``Computer Science and Control'' of the Russian 
Academy of Sciences, 40~Vavilov Str., Moscow 119333, Russian Federation; 
\mbox{Wysh@ccas.ru} 

\vspace*{3pt}

\noindent
\textbf{Flerov Yuri A.} (b.\ 1942)~--- Corresponding Member of the Russian 
Academy of Science, Doctor of Science in physics and mathematics, professor, 
Deputy Director, A.\,A.~Dorodnicyn Computing Centre, Federal Research Center 
``Computer Science and Control'' of the Russian Academy of Sciences, 40~Vavilov 
Str., Moscow 119333, Russian Federation; \mbox{fler@ccas.ru}

\vspace*{3pt}

\noindent
\textbf{Shirokov Nikolai I.} (b.\ 1963)~--- Candidate of Sciences (PhD) in physics and 
mathematics, senior scientist, A.\,A.~Dorodnicyn Computing Centre, Federal 
Research Center ``Computer Science and Control'' of the Russian Academy of 
Sciences, 40~Vavilov Str., Moscow 119333, Russian Federation; 
\mbox{Wysh@ccas.ru} 



\label{end\stat}


\renewcommand{\bibname}{\protect\rm Литература}     %9
\def\stat{torshin}

\def\tit{О ПОРОЖДЕНИИ СИНТЕТИЧЕСКИХ ПРИЗНАКОВ НА~ОСНОВЕ~ОПОРНЫХ ЦЕПЕЙ 
И~ПРОИЗВОЛЬНЫХ МЕТРИК В~РАМКАХ~ТОПОЛОГИЧЕСКОГО ПОДХОДА 
К~АНАЛИЗУ ДАННЫХ.\\ ЧАСТЬ~2.~ЭКСПЕРИМЕНТАЛЬНАЯ АПРОБАЦИЯ\\ НА~ЗАДАЧАХ ФАРМАКОИНФОРМАТИКИ$^*$}

\def\titkol{О порождении синтетических признаков на основе опорных цепей 
и~произвольных метрик} % в~рамках топологического подхода  к~анализу данных. Часть~2. Экспериментальная апробация на  задачах фармакоинформатики}

\def\aut{И.\,Ю.~Торшин$^1$}

\def\autkol{И.\,Ю.~Торшин}

\titel{\tit}{\aut}{\autkol}{\titkol}

\index{Торшин И.\,Ю.}
\index{Torshin I.\,Yu.}


{\renewcommand{\thefootnote}{\fnsymbol{footnote}} \footnotetext[1]
{Работа выполнена при поддержке гранта РНФ (проект №\,23-21-00154) с~использованием инфраструктуры 
Центра коллективного пользования <<Высокопроизводительные вычисления и~большие данные>> (ЦКП 
<<Информатика>>) ФИЦ ИУ РАН (г.~Москва).}}


\renewcommand{\thefootnote}{\arabic{footnote}}
\footnotetext[1]{Федеральный исследовательский центр <<Информатика и~управление>> Российской академии наук, 
\mbox{tiy135@yahoo.com}}

\vspace*{-12pt}


\Abst{Рассмотрение прецедентных отношений между признаками и~таргетной переменной в~виде наборов элементов булевой решетки указывает на возможность порождения 
синтетических признаков с~использованием метрических функций расстояния. 
Сформулированы подходы к~(1)~оценке релевантности (<<информативности>>) метрик 
по отношению к~решаемым задачам, (2)~порождению и~(3)~отбору синтетических 
признаков, более информативных, чем исходные признаковые описания. Представленные 
результаты топологического анализа 2400~выборок данных  
<<мо\-ле\-ку\-ла--свойство>> из ProteomicsDB позволили получить достаточно 
эффективные алгоритмы прогнозирования свойств молекул (ранговая корреляция  
в~кросс-ва\-ли\-да\-ции~--- $0{,}90\pm0{,}23$). На данной выборке задач установлены 
метрики, которые наиболее часто порождают информативные синтетические признаки: 
максимальное уклонение Колмогорова, <<косое>> расстояние, метрики Lp, Реньи, фон 
Мизеса. Для решения изученного комплекса задач показано преимущество полиномных 
корректоров по сравнению с~нейросетевыми и~с~корректорами типа <<случайный 
лес>>.}

\KW{топологический анализ данных; теория решеток; алгебраический подход 
Ю.\,И.~Жу\-рав\-лё\-ва; фармакоинформатика}

\DOI{10.14357/19922264240207}{OTXCUD}
  
\vspace*{-1pt}


\vskip 10pt plus 9pt minus 6pt

\thispagestyle{headings}

\begin{multicols}{2}

\label{st\stat}

\section{Введение}

     В первой части работы~[1] принимается, что задано регулярное 
множество прецедентов 
$$
\mathbf{Q}\hm= \{\mathrm{D}(x_i)\vert x_i\in 
\mathbf{X}\}
$$ 
на решетке $L(T(\mathbf{X}))$, по\-рож\-ден\-ное на основе 
множества исходных описаний объектов $\mathbf{X}\hm= \{ x_1, \ldots , 
x_{N_0}\}$. Для индивидуального объекта\linebreak $x_i\hm\in \mathbf{X}$ 
прецедентному соотношению между значениями признаками 
$\Gamma_k(x_i)$ и~\mbox{$t$-й} таргетной переменной соответствует множество пар 
$\{(\{\Gamma_k^{-1}(\Gamma_k(x_i)),\linebreak k\hm=\overline{1, ,n}\}, \Gamma_t^{-1}(\Gamma_t(x_i))), i\hm=\overline{1,N_0},\
 k\hm=\overline{1,n},\linebreak t\hm=\overline{n+1, n+l}\}$, где $l$~--- 
число таргетных переменных. В~рамках топологической теории 
распознавания прецедентное соотношение между множествами $\{ 
\Gamma_k^{-1}(\Gamma_k(x_i))\}$ и~$\Gamma_t^{-1}(\Gamma_t(x_i))$ 
моделируется как со\-от\-вет\-ст\-ву\-ющие массивы расстояний, по\-рож\-да\-емые той 
или иной мет\-ри\-кой~$\rho_m$: $L(T(\mathbf{X}))^2\hm\to [0\ldots 1]$, 
$m\hm= \overline{1, m_0}$. В~[1] предложены способы <<встра\-и\-ва\-ния>> 
в~формализм полуэмирических рас\-сто\-яний на множествах $a\hm\in 
L(T(\mathbf{X}))$, векторах $\vec{v}_\alpha [a] \hm= ( v_{\alpha_1}[a], 
v_{\alpha_2}[a], \ldots , v_{\alpha_i}[a],\ldots)$ и~функциях 
$\hat{\phi}(x)\bm{\Gamma}_t(u)$. 
     
     Здесь для практического приложения формализма сформулированы 
подходы к~исследованию свойств~$\rho_m$, способы оценки релевантности 
функций~$\rho_m$ по отношению к~решаемым задачам, способы 
порождения и~отбора синтетических признаков, основанных на~$\rho_m$. 
Представлены результаты экспериментальной апробации на задачах 
фармакоинформатики.
     
\section{Об исследовании свойств функций расстояния~$\rho_m$}

    Рабочая гипотеза настоящего исследования со\-сто\-ит в~том, что для 
порождения более <<информативных>> признаков могут использоваться 
полуэмпирические функционалы расстояния на \mbox{множествах}, векторах, 
функциях~[2]. Метрические свойства ис\-поль\-зу\-емых функций 
расстояния~$\rho_m$ могут исследоваться аналитически или комбинаторно 
с~использованием аксиом метрики~[3]. Для анализа свойств этих 
функционалов в~топологической теории распознавания вводится следующее 
понятие.

\smallskip

\noindent
\textbf{Определение~1.} Обобщенной оценочной функцией расстояния 
будем называть конструкцию вида 
$$
\rho(a,b) = f(g ( v[a\vee b]) - g(v[a\wedge b])),
$$
 в~которой~$f$ и~$g$~--- функции, монотонные на 
соответствующих участках действительной оси; $v:\ L\hm\to R^+$~--- 
изотонная оценка, для которой выполнено условие оценки (\textbf{уО}: $\forall_L 
a,b: v[a]\hm+v[b]\hm= v[a \wedge b]\hm+ v[a\vee b]$) и~изотонности 
(\textbf{уИ}:  $\forall_L a,b: a\supseteq b \hm\Rightarrow v[a]\hm\geq v[b]$). 

\smallskip

\noindent
\textbf{Теорема~1.} \textit{Функция расстояния~$\rho$ считается 
обобщенной оценочной функцией расстояния тогда и~только тогда, когда 
$\rho(a,b)\hm= \rho(a\vee b, a\wedge b)$, а~термы от $a$ и~$b$ в~формуле для 
$\rho(a,b)$ представляют собой композицию монотонной функции 
и~изотонной оценки}. 

\smallskip

Необходимость следует из  $a\vee b\hm= (a\vee b)\vee (a\wedge b)$ и~$a\wedge b \hm= (a\vee b) \wedge (a\wedge b)$  при 
подстановке $a\vee b$ и~$a\wedge b$ вместо $a$ и~$b$ в~определение~1. 
Эквивалентность $\rho(a,b)$ и~$\rho(a\vee b, a\wedge b)$ указывает на то, что 
в~выражение для вычисления~$\rho$ входят тер\-мы-функ\-ци\-о\-на\-лы, 
содержащие выражения $a\vee b$ и~$a\wedge b$, взаимозаменяемые с~$a$ 
и~$b$, т.\,е.\ термы вида $g^\prime (a\vee b)$ и~$g^\prime(a\wedge b)$. По 
условию теоремы эти термы включают монотонную функцию от изотонной 
оценки, т.\,е.~$g^\prime$ монотонна. Так как $\rho$~--- функция расстояния, 
то $g^\prime$-тер\-мы не могут входить в~выражение для~$\rho$ в~виде 
произведения, суммы, отношения, степени или суммы, а~только в~виде 
разности, т.\,е.\
$$
\rho(a,b) = f\left(g^\prime(a\vee b) \hm- g^\prime (a\wedge b)\right),
$$ 
из чего следует достаточность. Теорема доказана.

\smallskip

\noindent
\textbf{Следствие~1.} Для обобщенной оценочной~$\rho$ 
\begin{multline*}
\forall \ell \subseteq L(T(\mathbf{X})): \Delta_{\vee\wedge}(\ell)\equiv 0,\\ 
\Delta_{\vee\wedge}(\ell)=  \sum\limits_{a,b\in \ell} \vert\rho(a,b)- 
\rho(a\vee b, a\wedge b)\vert \fr{2}{\vert\ell\vert/(\vert\ell\vert -1)}\,.
\end{multline*}

\smallskip

\noindent
\textbf{Следствие~2.} Выберем <<опорное>> множество $a\hm\in 
L(T(\mathbf{X}))$ и~обобщенную оценочную~$\rho$. При $f(x)\hm= g(x)\hm= x$ 
$v_{a,\rho}[b]\hm= \rho(a,b)\hm= \rho(a\vee b, a\wedge b)$~--- изотонная 
оценка. 

Следует из того, что любая линейная комбинация изотонных оценок~--- 
изотонная оценка при условии положительной определенности (теорема~2 
в~[4]). Также проверяется прямой подстановкой $v_{a,\rho}[b]$ в~уО и~уИ. 

\smallskip

\noindent
\textbf{Следствие~3.} Расстояния Фре\-ше--Ни\-ко\-ди\-ма, Амана,  
Рэн\-да/Ще\-ка\-нов\-ско\-го, Со\-ка\-ла--Сни\-са (варианты~1, 2 и~3),  
Рас\-се\-ла--Рао, Род\-же\-ра--Та\-ни\-мо\-то, Фейта, Тверского и~Юле 
могут служить обобщенными оценочными функциями расстояния. 

\smallskip

\noindent
\textbf{Следствие~4.} Расстояния Симпсона, Бра\-у\-на--Блан\-ке, 
Андерберга и~Говера-2  не входят в~число обобщенных оценочных функций 
расстояния.

\smallskip

     Теорема~1 со следствиями предоставляет аналитический 
и~комбинаторный инструментарий для исследования свойств 
полуэмпирических функций расстояния. Если заданная~$\rho$ служит 
обобщенной оценочной функцией расстояния, то могут быть получены 
соответствующие аналитические выражения для функций~$f$ и~$g$. 
Например, расстояние Со\-ка\-ла--Сни\-са-2
$$
\rho(a,b) = 1- \fr{\vert a\cap 
b\vert }{\vert a\cup b\vert + \vert a\Delta b\vert}
$$ 
выступает 
обобщенным оценочным расстоянием с~$f(x)\hm= (e^x\hm-1)/(0{,}5e^x\hm-1)$ и~$g(x)\hm=\ln (x)$. При невозможности аналитической проверки 
свойства~$\rho$ как обобщенной оценочной могут быть изучены на 
подмножествах~$\ell$ решетки $L(T(\mathbf{X}))$ посредством вычисления 
значений функционала $\Delta_{\vee\wedge}(\ell)$ (следствие~1). 

\section{О способах оценки релевантности метрик~$\rho_m$ по~отношению к~задаче клас\-сификации/прогнозирования}

     Биекция между множеством прецедентов~$\mathbf{Q}$ и~множеством 
исходных описаний объектов~$\mathbf{X}$, существующая при выполнении 
условия регулярности по Журавлёву ($\forall \mathrm{x}\hm\in \mathbf{X}, 
\mathrm{x}\hm= D^{-1}(D(\mathrm{x}))$, гарантирует однозначность 
соответствия описаний~$x_i$ и~$q_i$. Это делает возможным рассматривать 
прецедентные соотношения, заданные на~$\mathbf{Q}$, в~терминах 
множеств $\{ \Gamma_k^{-1}(\Gamma_k(x_i))\}$ и~$\Gamma_t^{-1}( 
\Gamma_t(x_i))$ с~использованием расстояний~$\rho_m$ на подмножествах 
множества~$\mathbf{X}$~[1].
     
     Пусть таргетный класс объектов $\mathbf{c}_{\bm{\alpha}}$ задан 
посредством $\alpha$-го значения $t$-й переменной $\lambda_{t\alpha}\hm\in 
\mathrm{I}_t$, $t\hm= \overline{n+1,  n+l}$, как $\mathbf{c}_{{\bm 
\alpha}} \hm= \Gamma_t^{-1}(\lambda_{t\alpha})$. В~случае числовой 
переменной за $\mathbf{c}_{\bm{\alpha}}$ может приниматься каждый из 
элементов $u(\lambda_{t\alpha})$ цепи~$A_t$. Так как 
$L(T(\mathbf{X}))$ булева, то дополнение множества 
$\mathbf{c}_{\bm{\alpha}}$, $\overline{\mathbf{c}}_{\bm{\alpha}} \hm= 
\mathbf{X}\backslash \Gamma_t^{-1}(\lambda_{t\alpha})$, определено 
однозначно. Таким образом, выделение класса $\mathbf{c}_{\bm{\alpha}}$ 
порождает задачу классификации $\mathbf{c}_{\bm{\alpha}}/ 
\overline{\mathbf{c}}_{\bm{\alpha}}$. Любая задача числового 
прогнозирования может быть сведена к~последовательности корректно 
решаемых задач $\mathbf{c}_{\bm{\alpha}}/ 
\overline{\mathbf{c}}_{\bm{\alpha}}$~\cite{5-tor}.
     
     Пусть задано подмножество признаков~$p \hm\subseteq [1\ldots n]$ 
     и~элемент решетки $c\in L(T(\mathbf{X}))$. Определим функцию 
$$
\bm{\rho}_{\mathbf{mc}} (x_i, c, {p}) \hm= \{ \rho_m(c, \Gamma_k^{-1}(\Gamma_k (x_i)),\ k\hm\in {p})\}.
$$
 При заданных~$\rho_m$, $p$, 
$\mathbf{c}_{\bm{\alpha}}$ и~$\overline{\mathbf{c}}_{\bm{\alpha}}$ 
для~$x_i$ вычислимы множества расстояний $\bm{\rho}_{\mathbf{mc}}(x_i, 
\mathbf{c}_{\bm{\alpha}}, {p})$ и~$\bm{\rho}_{\mathbf{mc}}(x_i, 
\overline{\mathbf{c}}_{\bm{\alpha}}, {p})$. Обозначим 
\begin{align*}
\bm{\rho}_{\mathbf{m}\bm{\alpha}}(x_i) &=  \bm{\rho}_{\mathbf{mc}} 
(x_i, \mathbf{c}_{\bm{\alpha}}, [1\ldots n]); \\
\bm{\rho}_{\mathbf{m}\overline{\bm{\alpha}}} (x_i) &= 
\bm{\rho}_{\mathbf{mc}}(x_i, \overline{\mathbf{c}}_{\bm{\alpha}} , [1\ldots n]).
\end{align*}
 Для $x_i\hm\in \mathbf{X}$ 
определено множество 
\begin{multline*}
\bm{\rho}_{\mathbf{m}}(x_i,{p})=\left \{ \rho_{mk_1k_2}(x_i, {p}) = {}\right.\\
{}\rho_m\left(\Gamma^{-1}_{k_1}\left(\Gamma_{k_1}(x_i), \Gamma^{-1}_{k_2}\left(\Gamma_{k_2}(x_i)\right)\right)\right),\\
\left. k_1, k_2\hm \in {p},\  k_1\not= k_2\right\},\ \bm{\rho}_{\mathbf{m}}(x_i)=  \bm{\rho}_{\mathbf{m}}(x_i, [1\ldots n]).
\end{multline*}
     
     На основе $\bm{\rho}_{\mathbf{m}{\bm{\alpha}}}(x_i)$ 
и~$\bm{\rho}_{\mathbf{m}\overline{\bm{\alpha}}}(x_i)$ вводятся оценки 
релевантности~$\rho_m$. По отношению к~задаче $\mathbf{c}_{\bm{\alpha}}/ 
\overline{\mathbf{c}}_{\bm{\alpha}}$ более релевантна или 
<<информативна>> такая мет\-ри\-ка~$\rho_m$, которая для всех $x\hm\in 
\mathbf{c}_{\bm{\alpha}}$ минимизирует расстояния в~списке 
$\bm{\rho}_{\mathbf{m}{\bm{\alpha}}}(x)$ и~максимизирует расстояния 
в~списке $\bm{\rho}_{\mathbf{m}\overline{\bm{\alpha}}}(x)$ (т.\,е.\ 
<<приближает>> объекты к~их классам). Выделены два взаимосвязанных 
направления дальнейших исследований: 
\begin{enumerate}[(1)]
\item нахождение подмножеств $p$ 
признаков, <<более информативных>> для~$\rho_m$;  
\item на\-строй\-ка/вы\-бор~$\rho_m$ при фиксированном~$p$.
\end{enumerate}
     
     Для $c^\prime\hm\in L(T(\mathbf{X}))$ определим 
$\vartheta_{\mathbf{mc}}$, операцию слияния списков 
$\bm{\rho}_{\mathbf{mc}}$:
$$
\vartheta_{\mathbf{mc}}(c^\prime, c, 
{p})\hm= \bigcup\limits_{y\in c^\prime} \bm{\rho}_{\mathbf{mc}} (y,c, 
{p}).
$$
 Обозначим 
 $$
 \vartheta_{\mathbf{m}\bm{\alpha}}(\mathbf{c}, 
{p}) \!=\! \vartheta_{\mathbf{mc}}(\mathbf{c}, 
\mathbf{c}_{\bm{\alpha}}, {p});\ 
\vartheta_{\mathbf{m}\bm{\alpha}}(\mathbf{c},{p})\!=\! 
\vartheta_{\mathrm{mc}}(\mathbf{c}, \overline{\mathbf{c}}_{\bm \alpha}, 
{p}),
$$
 вычислим множества $\vartheta_{\mathbf{m}{\bm \alpha}} 
(\mathbf{c}_{\bm \alpha}, {p})$ и~$\vartheta_{\mathbf{m}{\bm \alpha}} 
(\overline{\mathbf{c}}_{\bm \alpha},{p})$ и~сформируем 
эмпирические функции распределения (э.ф.р.)\ $\hat{\phi}(x) 
\vartheta_{\mathbf{m}{\bm \alpha}} (\mathbf{c}_{\bm \alpha}, {p})$ 
и~$\hat{\phi}(x) \vartheta_{\mathbf{m}{\bm \alpha}} 
(\overline{\mathbf{c}}_{\bm \alpha}, {p})$. На пространстве 
однородных монотонно возрастающих функций 
\begin{multline*}
\mathbf{M}^+_{0\ldots1} ={}\\
{}= 
\{f: [0\ldots 1]\hm\to [0\ldots 1],\ x\geq y\hm\Rightarrow f(x)\geq f(y)\}
\end{multline*}
введем 
функционал расстояния $d_f$: $\mathbf{M}^+_{0..1}\hm\to [0\ldots 1]$ 
(максимальное уклонение Колмогорова $D(f(x), g(x))\hm= \mathrm{sup}_x 
\vert f(x)\hm- g(x)\vert$, метрики фон Мизеса, Реньи и~др.). Выбор~$d_f$ 
делает возможной постановку ряда задач топологического анализа данных:
     \begin{enumerate}[(1)]
\item количественные оценки релевантности~$\rho_m$ как 
$d_f(\hat{\phi}(x)\vartheta_{\mathbf{m}{\bm \alpha}}(\mathbf{c}_{\bm \alpha}, 
{p}), \hat{\phi}(x)\vartheta_{\mathbf{m}{\bm 
\alpha}}(\overline{\mathbf{c}}_{\bm \alpha}, {p}))$ для 
разных~$\mathbf{c}_{\bm \alpha}$, $\lambda_{t\alpha} \hm\in \mathrm{I}_t$, 
$\alpha \hm= \overline{1, \vert \mathrm{I}_t\vert}$;
\item задачи оптимизации для увеличения разделения классов 
$\mathbf{c}_{\bm \alpha}/\overline{\mathbf{c}}_{\bm \alpha}$ 
($\argmax_{\rho_m,{p}} d_f(\hat{\phi}\vartheta_{\mathbf{m}{\bm \alpha}}(\overline{\mathbf{c}}_{\bm \alpha},{p}), 
\hat{\phi}\vartheta_{\mathbf{m}{\bm \alpha}}(\mathbf{c}_{\bm \alpha}, {p}))$,
$\argmax_{\rho_m,{p}} d_f(\hat{\phi}\vartheta_{\mathbf{m}\overline{\bm{\alpha}}}, (\overline{\mathbf{c}}_{\bm \alpha}, {p}), 
\hat{\phi}\vartheta_{\mathbf{m}\overline{\bm \alpha}}
(\mathbf{c}_{\bm \alpha}, {p}))$  и~др.);
\item определение $\rho_q$-мет\-рик на пространстве объектов~[2, с.~184--199] 
(например, в~виде $d_f (\hat{\phi}\bm{\rho}_{\mathbf{m}{\bm \alpha}}(x, 
{p}), \hat{\phi}\bm{\rho}_{\mathbf{m}{\bm \alpha}}(y, {p})), 
d_f (\hat{\phi}\bm{\rho}_{\mathbf{m}}(x,{p})$, 
$\hat{\phi}\bm{\rho}_{\mathbf{m}}(y, {p}))$); 
\item оценка близости метрик~$\rho_q$ к~метрике разреза по классам 
$\mathbf{c}_{\bm{\alpha}}/ \overline{\mathbf{c}}_{\bm{\alpha}}$; 
\item формулировка критериев раз\-ре\-ши\-мости/ре\-гу\-ляр\-ности задачи 
$\mathbf{c}_{\bm{\alpha}}/ \overline{\mathbf{c}}_{\bm{\alpha}}$~[6]; 
\item оценки компактности классов $\mathbf{c}_{\bm{\alpha}}$  
и~$\overline{\mathbf{c}}_{\bm{\alpha}}$~[3]. 
\end{enumerate}

\section{О способах порождения и~отбора синтетических 
признаков на~основании функций расстояния}

     Множества $\bm{\rho}_{\mathbf{m}{\bm \alpha}}(x_i,{p})$, 
$\bm{\rho}_{\mathbf{m}{\overline{\bm \alpha}}}(x_i, {p})$ 
и~$\bm{\rho}_{\mathbf{m}}(x_i)$ и~отдельные $\rho_m(\mathbf{c}_{\bm 
\alpha}, \Gamma_k^{-1}(\Gamma_k(x_i))$ используются для формирования 
синтетических числовых признаков $\Gamma_{k^\prime}(x_i)$ 
объекта~$x_i$, $k^\prime\hm= \overline{n+ l+1, n+l+n_S}$. 
Значение синтетического признака~$\Gamma_{k^\prime}(x_i)$ зависит от 
выбора~$\rho_m$, классов $\mathbf{c}_{\bm{\alpha}}$ 
и~$\overline{\mathbf{c}}_{\bm{\alpha}}$  и~от способа его вы\-чис\-ле\-ния: 
\begin{enumerate}[(1)]
\item $\rho_m(\mathbf{c}_{\bm \alpha}, \Gamma_k^{-1}(\Gamma_k(x_i))$; 
\item $\rho_m(\overline{\mathbf{c}}_{\bm \alpha}, \Gamma_k^{-1}(\Gamma_k(x_i))$; 
\item $\rho_m(\mathbf{c}_{\bm \alpha}, \ldots ) \hm- \rho_m(\overline{\mathbf{c}}_{\bm \alpha}, \ldots)$;
\item $1\hm- \rho_m(\mathbf{c}_{\bm \alpha}, \ldots)$;
\item значения э.ф.р.\ 
$\hat{\phi}(x)\bm{\rho}_{\mathbf{m}{\bm \alpha}}(x_i,{p})$ при 
разных~$x$ (например, соответствующих процентилям 
$\hat{\phi}\bm{\rho}_{\mathbf{m}{\bm \alpha}}(x_i,{p})$); 
\item значения $\hat{\phi}(x)\bm{\rho}_{\mathbf{m}\overline{\bm{\alpha}}} 
(x_i, {p})$ при разных~$x$;
\item $\hat{\phi}(x\hm+ \Delta x) 
\bm{\rho}_{\mathbf{m}{\bm \alpha}}(x_i,p) \hm- 
\hat{\phi}(x)\bm{\rho}_{\mathbf{m}{\bm \alpha}} (x_i, {p})$ 
и~$\hat{\phi}(x\hm+ \Delta x) \bm{\rho}_{\mathbf{m}{\overline{\bm \alpha}}} 
(x_i,{p}) \hm- \hat{\phi}(x)\bm{\rho}_{\mathbf{m}\overline{\bm 
\alpha}} (x_i, {p})$, где $\Delta x$~--- шаг.
\end{enumerate}
     
     Кроме того, $\mathbf{c}_{\bm{\alpha}}$ может определяться как 
$\Gamma_t^{-1}(\lambda_{t\alpha})$ или как $u(\lambda_{t\alpha})$; если 
$\mathbf{c}_{\bm \alpha} \hm= \Gamma_t^{-1}(\lambda_{t\alpha})$, то 
$\overline{\mathbf{c}}_{\bm{\alpha}}$ может быть равно $\Gamma^{-1}_t 
(\lambda_{t\alpha+1})$; классы $\mathbf{c}_{\bm{\alpha}}/ 
\overline{\mathbf{c}}_{\bm{\alpha}}$  
$t$-й переменной могут определяться с~использованием раз\-би\-ений на 
различные процентили (которые определяются как подвыборка значений 
$\lambda_{t\alpha} \hm\in \mathrm{I}_t$) и~т.\,д. 
     
     Таким образом, предлагаемые схемы порождают значительное число 
синтетических признаков $\Gamma_{k^\prime}(x_i)$ ($10n$ и~более при $n$ 
исходных признаках $\Gamma_k$), что делает необходимым введение 
процедур отбора признаков. Таргетная переменная $\Gamma_t(x_i)$~--- 
чис\-ло\-вая, и~по\-рож\-да\-емые признаки $\Gamma_{k^\prime}(x_i)$~--- также 
чис\-ло\-вые. Для данного случая в~прикладной математике имеется несколько 
различных подходов к~оценке взаимосвязи $\Gamma_t(x_i)$ 
и~$\Gamma_{k^\prime}(x_i)$: корреляционные оценки (для линейных 
закономерностей), полиномная аппроксимация с~оценкой качества (для 
нелинейных закономерностей) и~методы теории  
ве\-ро\-ят\-но\-стей\,/\,ма\-те\-ма\-ти\-че\-ской статистики, не зависящие от 
вида закономерности (в~том числе на основе <<взаимной 
информации>>~[7]).
{\looseness=1

}
     
     Наиболее фундаментальным представляется тес\-ти\-ро\-ва\-ние взаимосвязи 
двух переменных на осно\-ве <<нулевой гипотезы>> об их независимости. 
Пусть заданы пары тестируемых значений, $(x_i, y_i)$,\linebreak $i\hm= \overline{1,\mathbf{n}_{(\mathrm{x,y})}}$, э.ф.р.~$F_{xy}(x,y)$ характеризует 
совместное распределение~$x$ и~$y$, а~э.ф.р.~$F_{{x}}(x)$ 
и~$F_{{y}}(y)$~--- индивидуальные распределения переменных. 
Эмпирическая функция распределения нулевой \mbox{гипотезы} (независимость~$x$ и~$y$) определяется как 
$F_{{x}}(x)F_{{y}}(y)$. 
     
     Для оценки отличий между $F_{{xy}}(x,y)$\linebreak 
и~$F_{{x}}(x) F_{{y}}(y)$ необходимо ввести расстояние 
меж-\linebreak ду такими функциями (так называемую <<статисти-\linebreak ку>>) и~оценить 
достоверность различий посред\-ст\-вом \mbox{того} или иного статистического\linebreak \mbox{тес\-та}. 
В~качестве расстояния можно использовать функции~$d_f$, адап\-ти\-ро\-ван\-ные 
для 2-мер\-но\-го случая (например, макси\-маль\-ное уклонение 
     $D(\mathrm{F}_{{xy}}(x,y), \mathrm{F}_{{x}}(x) 
\mathrm{F}_{{y}}(y)) \hm= \max ( \vert 
\mathrm{F}_{{xy}}(x_i,y_i) \hm- \mathrm{F}_{{x}}(x_i) 
\mathrm{F}_{{y}}(y_i)\vert )$) и~статистический тест  
Кол\-мо\-го\-ро\-ва--Смир\-но\-ва 
$P_{\mathrm{КС}}$ $(D 
(\mathrm{F}_{{xy}}(x,y), \mathrm{F}_{{x}}(x) 
\mathrm{F}_{{y}}(y)), n_{(x,y)})$. Тогда $1\hm- 
P_{\mathrm{КС}}$ характеризует <<информативность>>~$x$ 
относительно~$y$. 
     
     Более универсальным подходом к~оценке достоверности различий 
между $\mathrm{F}_{{xy}}(x,y)$ и~$\mathrm{F}_{{x}}(x) 
\mathrm{F}_{{y}}(y)$ считается прямое вычисление выбранной 
статистики~$d_f$ на множествах пар значений $(x_i, y_i)$, полученных 
датчиком случайных чисел. 
     
     Пусть \textit{оператор $\hat{\zeta}$, семплирующий} 
множество~$\mathbf{X}$, формирует набор семплов 
$$
\hat{\zeta}\mathbf{X}\hm= \{a_1, a_2, \ldots , a_k, \ldots , 
a_{\vert\hat{\zeta}X\vert}\vert a_k\hm\subset \mathbf{X}\},
$$
 а~процедура 
random~--- датчик случайных чисел (в~диапазоне $[0\ldots 1]$). Для каждого 
семпла~$a_k$ принимается, что ${n}_{({x,y})} \hm= \vert 
a_k\vert$, и~вычисляется множество значений~$d_f$ для случайных 
выборок, 

\noindent
\begin{multline*}
\mathrm{rnd}\,(\hat{\zeta}\mathbf{X}, d_f)= \left\{ 
\vphantom{i=\overline{1,\left\vert \hat{\zeta} X\right\vert }}
d_f\left(
\vphantom{\overline{1, \vert a_i\vert }}
\mathrm{F}_{{xy}}(x_{ij}, y_{ij}), 
\mathrm{F}_{{x}}(x_{ij}) \mathrm{F}_{{y}}(y_{ij}),\right.\right.\\
\left.\left. x_{ij}, 
y_{ij}= \mathrm{random},\  j=\overline{1, \vert a_i\vert }\right),\ i=\overline{1,\left\vert \hat{\zeta} X\right\vert }\right\}.
\end{multline*}

 Для $a\hm\in \hat{\zeta} \mathbf{X}$ значение 
${P}(d_f, \hat{\zeta}\mathbf{X}, a, k^\prime, t)\hm= 1\hm-
\hat{\phi}(d_f(\mathrm{F}_{k^\prime t}(\Gamma_{k^\prime}(z), \Gamma_t(z)), F_{k^\prime}(\Gamma_{k^\prime}(z)) 
\mathrm{F}_t(\Gamma_t(z)))\vert z\hm\in a) \mathrm{rnd}\,(\hat{\zeta}\mathbf{X}, d_f)$~--- статистическая достоверность 
<<зависимости>> $\Gamma_t(z)$ и~$\Gamma_{k^\prime}(z)$ по 
статистике~$d_f$ на семпле~$a$, а~$1\hm- P(d_f, 
\hat{\zeta}\mathbf{X}, a, k^\prime, t)$ количественно оценивает зависимость.
    

При заданном способе оценки зависимости $1\hm- P(d_f, 
\hat{\zeta}\mathbf{X}, a, k^\prime, t)$ задача отбора информативных 
признаков решается посредством так называемого\linebreak  
В-ал\-го\-рит\-ма, исходно разработанного для построения оптимальных 
словарей финальных ин\-фор\-маций (чему и~соответствует литера~<<В>>)~[8]. 
\mbox{Данный} алгоритм, основанный на критерии раз\-ре\-ши\-мости по Журавлёву, 
позволяет выбирать множества финальных информаций на основе 
максимального час\-тич\-но\-го покрытия при минимуме\linebreak элементов покрытия. 
Замена мощности пересечения множеств на $1\hm- P(d_f, 
\hat{\zeta}\mathbf{X}, a, k^\prime, t)$ приведет к~тому, что  
В-ал\-го\-ритм будет выбирать минимум признаков с~максимальной 
<<информативностью>>\linebreak (наиболее информативные признаки, см.\ 
теоремы~1, 7  и~8 работы~[8]).

    Таким образом, в~рамках развиваемого формализма синтез более 
информативных синтетических~$\Gamma_{k^\prime}(x_i)$ осуществляется 
в~5~стадий: 
\begin{enumerate}[(1)]
\item определяется набор исходных (как правило, 
<<низкоинформативных>>) признаков~$\Gamma_k(x_i)$ и~таргетная 
переменная~$\Gamma_t(x_i)$;
\item вводится набор метрик~$\rho_m$, 
оценивается их релевантность $d_f(\hat{\phi}(x)\vartheta_{\mathbf{m}{\bm 
\alpha}}(\mathbf{c}_{\bm \alpha},{p})$,\linebreak 
$\hat{\phi}(x)\vartheta_{\mathbf{m}{\bm \alpha}}(\overline{\mathbf{c}}_{\bm 
\alpha}, {p}))$ для каждого класса~$\mathbf{c}_{\bm \alpha}$ 
значений $t$-й переменной и~отбираются наиболее релевантные~$\rho_m$; 
\item посредством каждой из отобранных~$\rho_m$ по\-рож\-да\-ют\-ся 
синтетические признаки~$\Gamma_{k^\prime}(x_i)$;
\item посредством 
вычислений $1\hm- P(d_f, \hat{\zeta}\mathbf{X}, a, k^\prime, t)$  
и~В-ал\-го\-рит\-ма отбирается минимальное чис\-ло признаков максимальной 
<<ин\-фор\-ма\-тив\-ности>>;
\item применяется алгоритм прогнозирования 
таргетной переменной (корректор по Жу\-рав\-лё\-ву--Ру\-да\-кову). 
\end{enumerate}

\begin{table*}\small
\begin{center}
\begin{tabular}{|l|c|c|}
\multicolumn{3}{p{140mm}}{Ранговые корреляции между экспериментальными 
и~расчетными значениями $EC_{50}$ и~других величин хемокиномного анализа: $r$~--- 
коэффициент ранговой корреляции на обучении; $r_c$~---  на контроле. Усреднение~$r$ 
и~$r_c$ проводилось по 2400~выборкам хемокиномных данных}\\
\multicolumn{3}{c}{\ }\\[-6pt]
\hline
\multicolumn{1}{|c|}{{Эксперимент}}&$r$&$r_c$\\
\hline
{\boldmath $f_{\theta_k}$}\textbf{-алгоритмы, корректор~--- нейросеть}&\boldmath{$0{,}88\pm 
0{,}15$}&\boldmath{$0{,}86\pm0{,}20$}\\
Синтетические $\Gamma_{k^\prime}(x_i)$, корректор~--- нейросеть (2~слоя)&$0{,}45\pm 
0{,}22$&$0{,}22\pm 0{,}21$\\
Синтетические $\Gamma_{k^\prime}(x_i)$, корректор~--- нейросеть 
(10~слоев)&$0{,}52\pm 0{,}25$&$0{,}21\pm 0{,}20$\\
Синтетические $\Gamma_{k^\prime}(x_i)$, корректор~--- <<случайный лес>>, 
вариант~1&$0{,}98\pm 0{,}15$&$0{,}67\pm 0{,}31$\\
Синтетические $\Gamma_{k^\prime}(x_i)$, корректор~--- <<случайный лес>>, 
вариант~2&$0{,}99\pm 0{,}14$&$0{,}71\pm 0{,}35$\\
\textbf{Синтетические {\boldmath $\Gamma_{k^\prime}(x_i)$}, полиномные корректоры, 
вариант~1}&\boldmath{$0{,}93\pm 0{,}11$}&\boldmath{$0{,}90\pm 0{,}23$}\\
\textbf{Синтетические {\boldmath $\Gamma_{k^\prime}(x_i)$}, полиномные корректоры, 
вариант~2}&\boldmath{$0{,}95\pm0{,}08$}&\boldmath{$0{,}86\pm 0{,}27$}\\
\hline
\end{tabular}
\end{center}
\end{table*}

\section{Экспериментальная апробация }

    Формализм апробирован на комплексе задач\linebreak фармакоинформатики: 
получение количественных оценок ингибирования киназ протеома 
перспективными лекарствами (хемокиномный анализ)~[9]. Использованы 
2400~выборок данных <<\mbox{мо\-ле\-ку\-ла}--свой\-ст\-во>> из ProteomicsDB; 
свойства молекул включили константы $EC_{50}$ и~активности для 
концентраций~$(E_j(C_i))$.

     Исходные признаки $\Gamma_k(x_i)$ определялись как булевы 
инварианты над множествами $\chi$-це\-пей и~$\chi$-уз\-лов 
хемографов~$x_i$, как и~в~[9]. Таргетная $\Gamma_t(x_i)$ определялась как 
числовое значение прогнозируемого свойства. В~качестве~$\rho_m$ 
использовались функции расстояния на множествах, векторах и~э.ф.р.\ (всего 
65~функций из справочника~[2]). Классы~$\mathbf{c}_{\bm{\alpha}}$ 
определялись как квартили значений~$\Gamma_t$. Векторы элементов 
$L(T(\mathbf{X}))$ формировались из оценок $v^+_\alpha$, $v^-_\alpha$ 
и~$d_\alpha$~\cite{4-tor} для каждого~$\mathbf{c}_{\bm{\alpha}}$. 
Релевантность~$\rho_m$ по $d_f(\hat{\phi}(x),\vartheta_{\mathbf{m}{\bm 
\alpha}}(\mathbf{c}_{\bm{\alpha}},{p}), 
\hat{\phi}(x)\vartheta_{\mathbf{m}{\bm \alpha}} 
(\overline{\mathbf{c}}_{\bm{\alpha}}, {p}))$ оценивалась для 
каждого~$\mathbf{c}_{\bm{\alpha}}$, $d_f$~--- максимальное уклонение. 
Синтетические признаки~$\Gamma_{k^\prime}(x_i)$ по\-рож\-да\-лись всеми 
перечисленными выше способами; их отбор проводился В-ал\-го\-рит\-мом 
с~использованием $1\hm- {P}(d_f, \hat{\zeta}\mathbf{X}, a, 
     k^\prime, t)$. 
     
     В качестве корректоров использовались нейронные сети с~несколькими 
слоями (от~2 до~10) с~функцией активации softmax, полиномы различных 
конструкций (более 20~формул, в~том числе квазиполиномные модели 
с~элементарными функциями) и~<<случайные леса>> решающих деревьев. 
Оператор семплирования~$\hat{\zeta}$ был реализован как десятикратная  
кросс-ва\-ли\-да\-ция с~делением каждой выборки объектов на 80\% 
(обучение) и~20\% (конт\-роль). Результаты экспериментов суммированы 
в~таблице.
     

     
     Наилучший результат применения нового <<топологического>> 
формализма с~полиномным корректором ($r_c\hm=0{,}90\hm\pm0{,}23$) 
немного превзошел наилучший результат применения \mbox{метода} опорных 
функций (композиций вида $f_{\theta_k} \hm= g(f_1(\sum \omega_k^j x_k), 
\ldots\linebreak \ldots , f_l(\sum \omega_k^j x_k))$, см.~[9]), для которого 
$r_c\hm=0{,}86\hm\pm0{,}20$. Полиномными формулами, наиболее часто 
показывавшими наилучший результат, оказались полиномы 1-й или 2-й 
степеней с~произведениями переменных первой степени, полиномы 5-й 
степени, квазиполиномы 5-й степени с~сигмоидами и~Фурье-по\-ли\-но\-мы  
3-й степени.
     
     Нейросетевые корректоры всех использованных конфигураций 
отличались крайне низкими показателями ($r\hm=0{,}45\hm\pm0{,}22$, 
$r_c\hm=0{,}22\hm\pm0{,}21$), а~<<случайный лес>> приводил 
к~существенному переобучению (см.\ таб\-ли\-цу). При этом в~290 
из~2400~выборок данных (12\%) <<случайный лес>> приводил к~улучшению 
результатов по сравнению с~наилучшими полиномными корректорами, 
а~в~1670 из 2400~выборок данных (70\%)~--- к~ухудшению.
     
     
     Анализ синтетических признаков $\Gamma_{k^\prime}(x_i)$, 
вошедших в~наилучшие полиномные модели, показал, что среди более 
информативных (по оценке $1\hm- P(d_f, \hat{\zeta}\mathbf{X},  
a, k^\prime, t)$) признаков чаще всего встречались признаки, порождаемые 
с~использованием э.ф.р.\ на основе опорных цепей (теорема~1 в~1-й части 
работы~[1]), среди наименее информативных~--- исходные признаки 
$\Gamma_k(x_i)$ и~признаки на основе отдельных расстояний 
$\rho_m(\mathbf{c}_{\bm{\alpha}} , \Gamma_k^{-1}(\Gamma_k(x_i))$. 
Функциями~$\rho_m$, наиболее часто порождающими информативные 
$\Gamma_{k^\prime}(x_i)$ на пространстве э.ф.р., оказались максимальное 
уклонение Колмогорова, <<косое>> расстояние, метрики $\mathrm{Lp}$, 
Реньи, $\chi2$, фон Мизеса, инженерная~\cite{2-tor}. В~среднем по всем 
выборкам данных эти~7~разновидностей~$\rho_m$ порождали более 50\% 
самых информативных признаков~$\Gamma_{k^\prime}(x_i)$, отобранных  
В-ал\-го\-рит\-мом.

\vspace*{-6pt}

\section{Заключение}

\vspace*{-2pt}

    Предлагаемый подход к~порождению информативных синтетических 
признаков подразумевает последовательные трансформации описаний 
объекта:\\[-13pt]
\begin{enumerate}[(1)]
\item исходное множество значений признаков;\\[-13.5pt]
\item множество 
соответствующих элементов решетки;\\[-13.5pt] 
\item ~множество расстояний 
(измеряемых посредством~$\rho_m$) между элементами решетки, 
соответствующими классам и~признакам;\\[-13.5pt]
\item множество э.ф.р.\ расстояний, 
измеренных заданными~$\rho_m$;\\[-13.5pt] 
\item множество синтетических признаков 
объ-\linebreak екта.
\end{enumerate}

\noindent
 Использование многочисленных метрик на стадии порождения 
признаков позволяет рассматривать развиваемый формализм как вариант 
развития идеологии АВО (алгоритмы вычисления \mbox{оценок}) научной школы 
Ю.\,И.~Журавлёва. Экспериментальная апробация предлагаемого подхода на 
2400~однородных задачах фармакоинформатики позволила повысить 
аккуратность и~обобщающую способность алгоритмов. 


{\small\frenchspacing
 {\baselineskip=10.6pt
 %\addcontentsline{toc}{section}{References}
 \begin{thebibliography}{99}
  
  \bibitem{1-tor}
\Au{Торшин И.\,Ю.} О~порождении синтетических признаков на основе опорных цепей 
и~произвольных метрик в~рамках топологического подхода к~анализу данных. Часть~1. 
Включение в~формализм эмпирических функций расстояния~// Информатика и~её 
применения, 2024. Т.~18. Вып.~1. С.~71--77. doi: 10.14357/19922264240110. EDN: 
RIVOXR.
  \bibitem{2-tor}
  \Au{Деза Е.\,И., Деза~М.\,М.} Энциклопедический словарь расстояний~/ Пер. с~англ.~--- М.: Наука, 
2008. 444~с. (\Au{Deza~E.\,I., Deza~M.\,M.} {Dictionary of distances}.~--- North-Holland: 
Elsevier, 2006. 412~p. doi: 10.1016/B978-0-444-52087-6.X5000-8.)
  \bibitem{3-tor}
  \Au{Torshin I.\,Y., Rudakov~K.\,V.} Combinatorial analysis of the solvability properties of 
the problems of recognition and completeness of algorithmic models. Part~2: Metric approach 
within the framework of the theory of classification of feature values~// Pattern Recognition Image 
Analysis, 2017. Vol.~27. No.\,2. P.~184--199. doi: 10.1134/S1054661817020110.
  \bibitem{4-tor}
\Au{Торшин И.\,Ю.} О~формировании множеств прецедентов на основе таблиц 
разнородных признаковых описаний методами топологической теории анализа данных~// 
Информатика и~её применения, 2023. Т.~17. Вып.~3. С.~2--7. doi: 
10.14357/19922264230301. EDN: AQEUYO.
  \bibitem{5-tor}
  \Au{Torshin I.\,Yu., Rudakov~K.\,V.} On the procedures of generation of numerical features 
over partitions of sets of objects in the problem of predicting numerical target variables~// 
Pattern Recognition Image Analysis, 2019. Vol.~29. No.\,4. P.~654--667. doi: 
10.1134/S1054661819040175. 
  \bibitem{6-tor}
  \Au{Torshin I.\,Y., Rudakov~K.\,V.} Combinatorial analysis of the solvability properties of 
the problems of recognition and completeness of algorithmic models. Part~1: Factorization 
approach~// Pattern Recognition Image Analysis, 2017. Vol.~27. No.\,1. P.~16--28. doi: 
10.1134/S1054661817010151.
  \bibitem{7-tor}
  \Au{Sosa-Cabrera G., G$\acute{\mbox{o}}$mez-Guerrero~S.,  
Garc$\acute{\iota}$a-Torres~M., Schaerer~C.\,E.} Feature selection: A~perspective on inter-attribute 
cooperation~// Int. J. Data Science Analytics, 2024. Vol.~17. P.~139--151. doi:  
10.1007/s41060-023-00439-z.
  \bibitem{8-tor}
  \Au{Torshin I.\,Y.} Optimal dictionaries of the final information on the basis of the solvability 
criterion and their applications in bioinformatics~// Pattern Recognition Image Analysis, 2013. 
Vol.~23. No.\,2. P.~319--327. doi: 10.1134/S1054661813020156.
  \bibitem{9-tor}
\Au{Торшин И.\,Ю.} О~задачах оптимизации, воз\-ни\-ка\-ющих при применении 
топологического анализа данных к~поиску алгоритмов прогнозирования 
с~фиксированными корректорами~// Информатика и~её применения, 2023. Т.~17. Вып.~2. 
С.~2--10. doi: 10.14357/19922264230201. EDN: IGSPEW.

\end{thebibliography}

 }
 }

\end{multicols}

\vspace*{-8pt}

\hfill{\small\textit{Поступила в~редакцию 09.04.24}}

\vspace*{6pt}

%\pagebreak

%\newpage

%\vspace*{-28pt}

\hrule

\vspace*{2pt}

\hrule



\def\tit{ON THE GENERATION OF~SYNTHETIC FEATURES BASED~ON~SUPPORT~CHAINS 
AND~ARBITRARY METRICS\\ WITHIN THE~FRAMEWORK OF~A~TOPOLOGICAL 
APPROACH\\ TO~DATA ANALYSIS. PART~2. EXPERIMENTAL TESTING 
ON~PHARMACOINFORMATICS PROBLEMS}


\def\titkol{On the generation of~synthetic features based on~support chains 
and~arbitrary metrics} % within the~framework of~a~topological  approach to~data analysis. Part~2. Experimental testing  on~pharmacoinformatics problems}


\def\aut{I.\,Yu.~Torshin}

\def\autkol{I.\,Yu.~Torshin}

\titel{\tit}{\aut}{\autkol}{\titkol}

\vspace*{-15pt}


\noindent
Federal Research Center ``Computer Science and Control'' of the Russian Academy of 
Sciences, 44-2~Vavilov Str., Moscow 119333, Russian Federation

\def\leftfootline{\small{\textbf{\thepage}
\hfill INFORMATIKA I EE PRIMENENIYA~--- INFORMATICS AND
APPLICATIONS\ \ \ 2024\ \ \ volume~18\ \ \ issue\ 2}
}%
 \def\rightfootline{\small{INFORMATIKA I EE PRIMENENIYA~---
INFORMATICS AND APPLICATIONS\ \ \ 2024\ \ \ volume~18\ \ \ issue\ 2
\hfill \textbf{\thepage}}}

\vspace*{3pt}
  
  


\Abste{Consideration of precedent relationships between features and a target variable in the 
form of sets of Boolean lattice elements indicates the possibility of generating synthetic features 
using metric distance functions. Approaches to ($i$)~assessing the relevance (``informativeness'') 
of metrics in relation to the problems being solved; ($ii$)~generating; and ($iii$)~selecting synthetic 
features that are more informative than the original feature descriptions are formulated. The 
results of topological analysis of~2400~samples of ``molecule--property'' data
from 
ProteomicsDB made it possible to obtain fairly effective algorithms for 
predicting the properties of molecules (rank correlation in cross-validation is~$0.90\pm 0.23$). 
Using this sample of problems, metrics have been established\linebreak\vspace*{-12pt}}

\Abstend{that most often generate 
informative synthetic features: maximum Kolmogorov deviation, ``oblique'' distance, and Lp, Renyi, 
and von Mises metrics. To solve the studied set of problems, the advantage of polynomial 
correctors compared to neural network and random forest correctors is shown.}

\KWE{topological data analysis; lattice theory; algebraic approach of Yu.\,I.~Zhuravlev; 
pharmacoinformatics}




\DOI{10.14357/19922264240207}{OTXCUD}

%\vspace*{-12pt}

\Ack

\vspace*{-3pt}


\noindent
The research was funded by the Russian Science Foundation, project No.\,23-21-00154. The 
research was carried out using the infrastructure of the Shared Research Facilities ``High 
Performance Computing and Big Data'' (CKP ``Informatics'') of FRC CSC RAS (Moscow).
 


  \begin{multicols}{2}

\renewcommand{\bibname}{\protect\rmfamily References}
%\renewcommand{\bibname}{\large\protect\rm References}

{\small\frenchspacing
 {%\baselineskip=10.8pt
 \addcontentsline{toc}{section}{References}
 \begin{thebibliography}{9} 
 
 %\vspace*{-3pt}
  \bibitem{1-tor-1}
\Aue{Torshin, I.\,Yu.} 2024. O~porozhdenii sinteticheskikh priznakov na osno\-ve opor\-nykh 
tsepey i~proizvol'nykh metrik v~ram\-kakh topologicheskogo podkhoda k~analizu dannykh. 
Chast'~1. Vklyuchenie v~formalizm empiricheskikh funktsiy rasstoyaniya [On the generation 
of synthetic features based on support chains and arbitrary metrics within a~topological approach 
to data analysis. Part~1. Inclusion of empirical distance functions into the formalism]. 
\textit{Informatika i~ee Primeneniya~--- Inform Appl.} 18(1):71--77. doi: 
10.14357/19922264240110. EDN: RIVOXR.
  \bibitem{2-tor-1}
\Aue{Deza, E.\,I., and M.\,M.~Deza.} 2006. \textit{Dictionary of distances}. North-Holland: 
Elsevier. 412~p. doi: 10.1016/B978-0-444-52087-6.X5000-8.
  \bibitem{3-tor-1}
\Aue{Torshin, I.\,Yu., and K.\,V.~Rudakov.} 2017. Combinatorial analysis of the solvability 
properties of the problems of recognition and completeness of algorithmic models. Part~2: 
Metric approach within the framework of the theory of classification of feature values. 
\textit{Pattern Recognition Image Analysis} 27(2):184--199. doi: 10.1134/S1054661817020110.
  \bibitem{4-tor-1}
\Aue{Torshin, I.\,Yu.} 2023. O~formirovanii mnozhestv pretsedentov na osnove tablits 
raznorodnykh priznakovykh opisaniy metodami topologicheskoy teorii analiza dannykh [On the 
formation of sets of precedents based on tables of heterogeneous feature descriptions by methods 
of topological theory of data analysis]. \textit{Informatika i~ee Primeneniya~--- Inform Appl.} 
17(3):2--7. doi: 10.14357/19922264230301. EDN: AQEUYO.
  \bibitem{5-tor-1}
\Aue{Torshin, I.\,Yu., and K.\,V.~Rudakov.} 2019. On the procedures of generation of 
numerical features over partitions of sets of objects in the problem of predicting numerical target 
variables. \textit{Pattern Recognition Image Analysis} 29(4):654--667. doi: 
10.1134/S1054661819040175.
  \bibitem{6-tor-1}
\Aue{Torshin, I.\,Y., and K.\,V.~Rudakov.} 2017. Combinatorial analysis of the solvability of 
the problems of recognition, completeness of algorithmic models. Part~1: Factorization 
approach. \textit{Pattern Recognition Image Analysis} 27(1):16--28. doi: 
10.1134/S1054661817010151.
  \bibitem{7-tor-1}
\Aue{Sosa-Cabrera, G., S.~Gуmez-Guerrero, \mbox{M.~Garc$\acute{\!\mbox{{\ptb{\i}}}}$a}-Torres, 
and C.\,E.~Schaerer.} 2024. Feature selection: A~perspective on inter-attribute cooperation. \textit{Int. J. 
Data Science Analytics} 17:139--151. doi: 10.1007/s41060-023-00439-z.
  \bibitem{8-tor-1}
\Aue{Torshin, I.\,Y.} 2013. Optimal dictionaries of the final information on the basis of the 
solvability criterion and their applications in bioinformatics. \textit{Pattern Recognition Image 
Analysis}  23(2):319--327. doi: 10.1134/ S1054661813020156.
  \bibitem{9-tor-1}
\Aue{Torshin, I.\,Yu.} 2023. O~zadachakh optimizatsii, voznikayushchikh pri primenenii 
topologicheskogo analiza dannykh k~poisku algoritmov prognozirovaniya s~fiksirovannymi 
korrektorami [On optimization problems arising from the application of topological data analysis 
to the search for forecasting algorithms with fixed correctors]. \textit{Informatika i~ee 
Primeneniya~--- Inform Appl.} 17(2):2--10. doi: 10.14357/19922264230201. EDN: IGSPEW.

\end{thebibliography}

 }
 }

\end{multicols}

\vspace*{-6pt}

\hfill{\small\textit{Received April 9, 2024}} 

\vspace*{-12pt}


\Contrl

\vspace*{-3pt}

\noindent
\textbf{Torshin Ivan Y.} (b.\ 1972)~--- Candidate of Science (PhD) in physics and mathematics, 
Candidate of Science (PhD) in chemistry, leading scientist, Federal Research Center ``Computer 
Science and Control'' of the Russian Academy of Sciences, 44-2~Vavilov Str, Moscow 119333, 
Russian Federation; \mbox{tiy135@yahoo.com}
  
  



\label{end\stat}

\renewcommand{\bibname}{\protect\rm Литература}   %10
\def\stat{grusho}

\def\tit{АРХИТЕКТУРНЫЕ РЕШЕНИЯ В~ЗАДАЧЕ ВЫЯВЛЕНИЯ МОШЕННИЧЕСТВА ПРИ~АНАЛИЗЕ 
ИНФОРМАЦИОННЫХ ПОТОКОВ В~ЦИФРОВОЙ ЭКОНОМИКЕ$^*$}

\def\titkol{Архитектурные решения в~задаче выявления мошенничества при~анализе 
информационных потоков в
%~цифровой 
экономике}

\def\aut{А.\,А.~Грушо$^1$, М.\,И.~Забежайло$^2$, Н.\,А.~Грушо$^3$, 
Е.\,Е.~Тимонина$^4$}

\def\autkol{А.\,А.~Грушо, М.\,И.~Забежайло, Н.\,А.~Грушо, 
Е.\,Е.~Тимонина}

\titel{\tit}{\aut}{\autkol}{\titkol}

\index{Грушо А.\,А.}
\index{Забежайло М.\,И.}
\index{Грушо Н.\,А.}
\index{Тимонина Е.\,Е.}
\index{Grusho A.\,A.}
\index{Zabezhailo M.\,I.}
\index{Grusho N.\,A.}
\index{Timonina E.\,E.}


{\renewcommand{\thefootnote}{\fnsymbol{footnote}} \footnotetext[1]
{Работа частично поддержана РФФИ (проекты 18-29-03081 и~18-07-00274).}}


\renewcommand{\thefootnote}{\arabic{footnote}}
\footnotetext[1]{Институт проблем информатики Федерального исследовательского центра <<Информатика и~управление>> 
Российской академии наук, grusho@yandex.ru}
\footnotetext[2]{Институт проблем информатики Федерального исследовательского центра <<Информатика и~управление>> 
Российской академии наук, m.zabezhailo@yandex.ru}
\footnotetext[3]{Институт проблем информатики Федерального исследовательского центра <<Информатика и~управление>> 
Российской академии наук, info@itake.ru}
\footnotetext[4]{Институт проблем информатики Федерального исследовательского центра <<Информатика и~управление>> 
Российской академии наук, eltimon@yandex.ru}

\vspace*{-12pt}
   

 
  
  \Abst{Cформулирован подход к~исследованию некоторых видов мошенничества в~цифровой 
экономике с~использованием причинно-следственных связей. Во всех видах рассматриваемых 
мошенничеств должно наблюдаться несоответствие между целями финансовых транзакций 
и~реальной стоимостью достижения этих целей. Данные о транзакциях можно собирать, 
наблюдая информационные потоки, в~которых отражаются эти транзакции. Архитектура сбора 
данных и~их анализа может быть организована с~помощью распределенных реестров 
с~централизованным консенсусом, что позволяет создать аналог электронной бухгалтерской 
книги, фиксирующей финансово-экономическую деятельность субъектов цифровой экономики в~регионе. 
  Рассматриваемые методы выявления мошенничества основаны на противоречиях 
между действиями, описанными в~транзакциях, и~информацией, содержащейся в~планах, 
стандартах, прецедентах и~др. Рассмотрен метод, основанный на некоторой упрощенной схеме 
реализации абстрактного проекта. Для выявления противоречий необходимо проводить анализ 
от следствия к~причине, т.\,е.\ искать аномалии в~информации, описывающей порождение 
наблюдаемых следствий. 
  Показано, как в~реализации проекта можно выделять простые <<необходимые условия>> 
нарушения при\-чин\-но-след\-ст\-вен\-ных связей, т.\,е.\ множество <<необходимых условий>>, 
нарушение которых свидетельствует о наличии мошенничества. Это множество <<необходимых 
условий>> можно назвать метаданными для контроля проекта на выявление мошенничества.} 
 
 
  \KW{цифровая экономика; информационные потоки; при\-чин\-но-след\-ст\-вен\-ные связи; 
выявление мошеннических схем} 

\DOI{10.14357/19922264190204}
  
\vspace*{-4pt}


\vskip 10pt plus 9pt minus 6pt

\thispagestyle{headings}

\begin{multicols}{2}

\label{st\stat}

\section{Введение}

\vspace*{3pt}

  В работе сформулирован подход к~исследованию некоторых видов 
мошенничества в~цифровой экономике с~использованием  
при\-чин\-но-след\-ст\-вен\-ных связей. Рассматриваются три вида мошенничества, 
а именно:
  \begin{enumerate}[(1)]
\item отмыв денег; 
\item обман при выполнении договорных обязательств при реализации 
технических проектов (строительные проекты и~др.); 
\item незаконный вывод денег. 
\end{enumerate}

  Названные виды мошенничества могут быть сведены к~решению одного типа 
задач. Для отмывания денег источник должен заключать фиктивные контракты, 
в~соответствии с~которыми будут переводиться средства за заведомо ненужную 
работу и~материалы. 
  
  Мошенничество, связанное с~невыполнением договорных обязательств, связано 
со снижением качества услуг, качества и~количества закупаемых 
материалов, выполнением работ с~ненадлежащим качеством. 
  
  Вывод денег связан с~переводом средств фир\-мам-од\-но\-днев\-кам, которые 
заведомо не могут выполнить обязательства по контрактам, за которые им 
переводятся средства. 
  
  Таким образом, во всех трех видах рассматриваемых мошенничеств должно 
наблюдаться несоответствие между целями финансовых транзакций и~реальной 
стоимостью достижения этих целей. Данные о транзакциях можно собирать, 
наблюдая информационные потоки, в~которых отражаются эти транзакции. 
  
  Однако для наблюдения таких информационных потоков необходимо создавать 
архитектуру\linebreak телекоммуникационной системы, позволяющей перехватывать 
и~собирать данные о всех транзакциях. Например, такая архитектура может быть 
организована с~помощью распределенных реестров с~централизованным 
консенсусом, т.\,е.\ все информационные потоки, сформированные в~цифровой 
экономике и~несущие информацию о транзакциях, проходят через некоторый 
центральный узел, запоминающий их в~форме распределенного реестра. Такие 
реестры могут дублироваться в~аналогичных центрах различных регионов, что 
позволяет создать аналог электронной бухгалтерской книги, фиксирующей 
фи\-нан\-со\-во-эко\-но\-ми\-че\-скую деятельность субъектов цифровой экономики. Такой 
подход предложено реализовать на базе системы ситуационных центров, что 
отражено в~работах~[1, 2].
  
  Собранная из информационных потоков информация о~транзакциях, т.\,е.\ 
о~контрактах, договорах, платежах, отчетах, закупленных материалах, 
характеристиках исполнителей работ и~др., собирается в~базе данных в~указанном 
центре. Согласно теории интеллектуальных сис\-тем~[3], эту базу данных можно 
называть базой фактов (БФ). Базу фактов можно представить как бинарную мат\-ри\-цу, 
строки которой описывают характеристики, входящие в~транзакции, а столбцы 
нумеруются характеристиками. Строки матрицы будем называть 
\textit{объектами}~[4, 5]. 
  
  Рассматриваемые в~работе методы выявления мошенничества будут основаны 
на противоречиях между действиями, описанными в~транзакциях, и~информацией, 
содержащейся в~планах, стандартах, прецедентах и~др. Для нахождения 
противоречий в~архитектуре центра предусмотрена другая база данных~--- база 
знаний (БЗ)~\cite{3-gr, 6-gr}, которая устроена так же, как БФ. 
  
  Информация в~БЗ собирается на основе положительного опыта или расчетов. 
Используя БЗ, можно выводить факты нарушения при\-чин\-но-след\-ст\-вен\-ных 
связей. Нарушения при\-чин\-но-след\-ст\-вен\-ных связей будем называть 
\textit{аномалиями}. 
  
  Для упрощения дальнейшее изложение будет вестись в~рамках поиска 
противоречий при выполнении некоторого абстрактного проекта. Выявление 
аномалий будет происходить на основе фактов из БФ с~помощью знаний из БЗ 
методами искусственного интеллекта и~интеллектуального анализа 
данных~\cite{6-gr}. 

\vspace*{-10pt}
  
  \section{Модели}
  
  \vspace*{-3pt}
  
  Наиболее сложная из рассмотренных выше задач~--- выявление противоречий, 
т.\,е.\ использование БЗ для получения новых знаний и~выявление аномалий из 
полученных фактов. 
  
  Все способы выявления противоречий основаны на определении 
  причинно-следственных связей. При этом противоречия в~параметрах транзакций по 
отношению к~требуемым в~БЗ составляют сущность аномалий. 
  
   Далее будет рассмотрен метод, основанный на некоторой упрощенной схеме 
реализации абстрактного проекта. 
  
  Каждый проект имеет цель: например, цель представляет собой построение 
некоторой системы. Воспользуемся структурным подходом, который позволяет 
строить проект на основе разбиения системы на подсистемы и~определения 
взаимодействий подсистем~\cite{7-gr}. При этом каждая подсистема также 
представима структурной моделью. 
  
  Как сама система, так и~каждая ее подсистема имеют свой функционал 
и~спецификацию, па\-ра\-мет\-ры настройки и~домены параметров настройки. Кроме 
этих характеристик существует множество характеристик, связанных 
с~<<жизненным циклом>> создания системы. Сюда входят работы, ресурсы, 
сроки выполнения работ по созданию подсистем и~самой системы, стоимости 
компонентов и~материалов, стоимости работ, схемы поставок, договорные 
обязательства и~др. Все характеристики связаны между собой, поэтому можно 
говорить о стоимости и~времени изготовления структурных компонентов системы. 
  
  Одной из важнейших характеристик является смета (система смет для 
подсистем). Смета сопоставляет каждому компоненту системы стоимость его 
изготовления и~настройки. 
  
  Схема построения системы может быть пред\-став\-ле\-на диаграммой, 
изображенной на рис.~1. 

{ \begin{center}  %fig1
 \vspace*{9pt}
   \mbox{%
 \epsfxsize=79mm 
 \epsfbox{gru-1.eps}
 }


\vspace*{9pt}


\noindent
{{\figurename~1}\ \ \small{Диаграмма достижения цели}}
\end{center}
}

\vspace*{9pt}

\addtocounter{figure}{1}
  
  


  Представленная на рис.~1 диаграмма позволяет описать основные классы 
возможных противоречий при достижении цели. Противоречия возникают, когда 
данные БФ не соответствуют требуемым характеристикам. 
  
  
  \section{Потенциальные классы аномалий при~достижении цели}
  
  Выделим четыре потенциальных класса противоречий, которые показывают, 
каким образом нужно искать эти противоречия.
  
 
  Противоречие цели и~проекта (рис.~2) возникает при отсутствии обоснования 
или в~случае логического противоречия между возможностями проектируемого 
функционала и~целью системы. Отметим, что в~проект входят сроки, перечень 
работ, материалы, настройки, которые описываются соответствующими 
параметрами и~допустимыми значениями этих параметров. Проект формируется 
на основе БЗ и~расчетов, исходя из информации, полученной по аналогии 
с~другими проектами и~решениями, которые считаются апробированными. 
  
  Отметим, что цель порождает проект и~в этом смысле является причиной 
проекта. Однако для анализа противоречий необходимо двигаться по штриховой 
стрелке диаграммы (см.\ рис.~2) от проекта к~цели. В~самом деле, любой компонент 
проекта направлен на теоретическое достижение цели. Цель~--- сложный объект, 
поэтому в~проекте могут возникнуть характеристики, противоречащие хотя бы 
некоторым характеристикам цели. Это делает проект противоречивым, но вывод 
об этом может быть сделан только на уровне описания цели. 
  

  Противоречия между проектом и~его реализацией, исключая настройки 
(рис.~3), могут возникать, например, при закупке исполнителем материалов более 
низкого качества по более низким ценам, при попытках достижения требуемых 
сроков работы за счет снижения качества выполнения работ, за счет нахождения 
<<объективных>> причин для увеличения сроков работы и,~следовательно, 
увеличения цены реализации проекта. 


  Для выявления указанных противоречий необходимо двигаться по диаграмме 
(см.\ рис.~3) в~обратную сторону в~соответствии со~штриховыми стрелками. 
Действительно, выявить противоречия между характеристиками закупленных 
материалов и~требуемыми по проекту можно только при обращении к~проекту 
и~его спецификациям. Манипуляции со сроками работы также можно выявить 
только при обращении к~соответствующим расчетам в~проекте. Задержки в~сроках 
работы, связанные с~поставками материалов, можно определить только на 
предыдущем этапе диаграммы (см.\ рис.~3) в~описании проекта. 


  


  Противоречия между реализацией проекта и~его настройкой (рис.~4) возникает, 
когда не удается добиться требуемых значений параметров функционала, не 
удается обеспечить необходимый уровень\linebreak\vspace*{-12pt}

{ \begin{center}  %fig2
 \vspace*{-6pt}
   \mbox{%
 \epsfxsize=16mm 
 \epsfbox{gru-2.eps}
 }


\vspace*{6pt}


\noindent
{{\figurename~2}\ \ \small{Противоречия цели и~проекта}}
\end{center}
}

%\vspace*{9pt}

\addtocounter{figure}{1}

{ \begin{center}  %fig3
 \vspace*{6pt}
    \mbox{%
 \epsfxsize=79mm 
 \epsfbox{gru-3.eps}
 }


\end{center}

\vspace*{-2pt}


\noindent
{{\figurename~3}\ \ \small{Противоречия проекта и~его реализации (без настройки)}}
}

\vspace*{6pt}

\addtocounter{figure}{1}

{ \begin{center}  %fig4
 \vspace*{1pt}
   \mbox{%
 \epsfxsize=54.5mm 
 \epsfbox{gru-4.eps}
 }


\end{center}


\noindent
{{\figurename~4}\ \ \small{Противоречия реализации проекта и~его на\-стройки}}
}

%\vspace*{9pt}

\addtocounter{figure}{1}

{ \begin{center}  %fig5
 \vspace*{5pt}
    \mbox{%
 \epsfxsize=79mm 
 \epsfbox{gru-5.eps}
 }


\end{center}



\noindent
{{\figurename~5}\ \ \small{Противоречия цели и~достигнутой реализации проекта}}
}

\vspace*{6pt}

\addtocounter{figure}{1}

\noindent
 качества реализации проекта. Для 
определения противоречия в~настройках надо опять же двигаться по диаграмме 
(см.\ рис.~4) в~обратную сторону по штриховым стрелкам, так как для выявления 
характеристик результатов работы, которые не дают возможности реализации 
определенного функционала, необходимо иметь информацию о результатах этой 
работы. 


  



  Противоречие между целью и~достигнутой реализацией проекта (рис.~5) 
возникает, когда реализованная система не позволяет достичь цели. В~этом случае 
опять противоречие нужно искать, двигаясь от цели к~реальному достигнутому 
функционалу по штриховой стрелке (см.\ рис.~5).
  
  Суммируя положения, изложенные в~данном разделе, приходим к~выводу, что 
для выявления противоречий необходимо проводить анализ от следствия 
к~причине, т.\,е.\ искать аномалии в~информации, описывающей порождение 
наблюдаемых следствий. 
  
  
  \section{Связь противоречий и~причин}
  
  Прежде чем построить связь между причинами и~противоречиями, кратко 
опишем простейшую модель связи этих понятий. Причины и~противоречия будут 
сформулированы для представления компонентов системы как объектов, 
обладающих наборами известных характеристик~\cite{4-gr, 5-gr}. 
  
  Пусть $U\hm=\{\alpha, \beta, \ldots\}$~--- совокупность характеристик 
(пространство характеристик). Согласно~\cite{4-gr} \textit{объектом}~$O$ 
называется любое подмножество характеристик $O\hm\subseteq U$. Рассмотрим 
последовательность объектов, возможно в~различных пространствах 
характеристик. 
  
  \smallskip
  
  \noindent
  \textbf{Определение~1.}\ Объект~$P$ с~числом характеристик, большим или 
равным~2, является \textit{причиной} объекта (\textit{свойства})~$B$ в~цепочке 
наблюдаемых объектов тогда и~только тогда, когда выполнены следующие 
условия:
  \begin{enumerate}[(1)]
\item для каждого объекта~$C$, если $P\hm\subseteq C$, то $C\mapsto B$, где 
$C\mapsto B$ означает, что объект~$B$ присутствует в~объекте, следующем за 
объектом~$C$;
\item объект~$P$ является минимальным объектом, удовлетворяющим 
условию~1, а~именно: $\forall \alpha\hm\in P$ объект~$P\backslash \{\alpha\}$ 
не является причиной, т.\,е.\ $\exists C:\ \alpha\not\in C$, $P\backslash 
\{\alpha\}\hm\subseteq C$ и~$C\not\mapsto B$, где $C\not\mapsto B$ означает, 
что~$B$ не может содержаться в~объекте, следующем за объектом~$C$. 
\end{enumerate}

  Приведенное определение причины является упрощением причин, 
возникающих в~реальном мире. Например, реальные причины могут возникать\linebreak 
как совокупность характеристик из разных пространств. Одно следствие может 
порождаться разными причинами или возникать из внешних\linebreak и~ненаблюдаемых 
характеристик. Однако пред\-став\-лен\-ная далее формализация позволяет доступно 
изложить при\-чин\-но-след\-ст\-вен\-ные истоки противоречий, которые 
инициируют в~дальнейшем глубокое исследование рассматриваемых процессов.
  
  Будем считать, что для любого интересующего нас свойства~$B$ существует 
причина. Тогда справедлива следующая теорема.
  
  \smallskip
  
  \noindent
  \textbf{Теорема~1.}\ \textit{Для любого свойства~$B$ существует 
единственная причина}. 
  
  \smallskip
  
  \noindent
  Д\,о\,к\,а\,з\,а\,т\,е\,л\,ь\,с\,т\,в\,о\,.\ \ Доказательство будем вести от противного, 
т.\,е.\ предположим, что существуют две причины свойства~$B$: $P$ 
и~$P^\prime$, $P\hm\not= P^\prime$. Тогда существует $\alpha\hm\in U$, которое 
удовлетворяет одному из двух условий:
  \begin{itemize}
\item[(а)] $\alpha\in P$, $\alpha\notin P^\prime$;
\item[(б)] $\alpha\notin P$, $\alpha \in P^\prime$.
\end{itemize}

  Пусть выполняется условие~(б). Тогда $P^\prime\backslash \{\alpha\}$ не 
является причиной по условию~2 определения~1, т.\,е.\ $\exists C$ такое, что 
$\alpha\notin C$, $P^\prime\backslash \{\alpha\}\hm\subseteq C$ и~$C\not\mapsto B$. 
Но если~$B$ произошло и~$P$ его причина, то $C\mapsto B$, что противоречит 
предположению. Теорема~1 доказана.
  
  \smallskip
  
  \noindent
  \textbf{Лемма.} \textit{Если $P$~--- причина появления свойства~$B$, то 
объект~$B$ определяет существование свойства~$P$ в~объекте, 
предшествующем~$B$. }
  
  \smallskip
  
  \noindent
  Д\,о\,к\,а\,з\,а\,т\,е\,л\,ь\,с\,т\,в\,о\,.\ \ Из предположения, что у~каж\-до\-го 
свойства~$B$ есть причина, и~условия, что~$P$ является причиной~$B$, следует, 
что при появлении в~данных свойства~$B$ объект~$C$, предшествующий 
появлению~$B$, содержит как часть объект~$P$. Это следует из теоремы~1 
и~определения причины. 
  
  Докажем принцип <<необходимого условия>>, который, несмотря на простоту 
доказательства, будет играть в~дальнейшем существенную роль.
  
  \smallskip
  
  \noindent
  \textbf{Теорема~2.} \textit{Если~$P$~--- причина появления свойства~$B$ 
и~$A\hm\subseteq P$, то объект~$B$ определяет наличие свойства~$A$ 
в~объекте, предшествующем~$B$}. 
  
  \smallskip
  
  \noindent
  Д\,о\,к\,а\,з\,а\,т\,е\,л\,ь\,с\,т\,в\,о\,.\ \ Пусть в~данных имеется объект~$B$ 
и~$P\mapsto B$, тогда в~силу существования и~единственности причины~$B$ 
в~данных должен существовать объект~$C$, предшествующий~$B$ 
и~содержащий причину~$P$. Поскольку $A\hm\subseteq P$ и~$B$ содержит 
причину~$P$, то $B\mapsto A$. С~учетом леммы теорема~2 доказана.
  
  \smallskip
  
  Пусть даны пространства $U_1, U_2,\ldots$ и~имеется последовательность 
данных (процесс выполнения этапов проекта в~соответствии с~рис.~1) $A, B, 
\ldots$, где каждый объект является подмножеством некоторого 
пространства~$U_i$, $i\hm=1,\ldots$ Тогда в~объекте~$A$ присутствует 
причина~$P$ появления интересующего нас свойства~$C$ в~объекте~$B$. Пусть 
$P\hm\subseteq A$, тогда по теореме~2 $\forall \alpha\hm\in P$:  
$C\mapsto \{\alpha\}$, т.\,е.\ из появления~$C$ следует появление 
характеристики~$\alpha$ в~предшествующем объекте. Это необходимое условие 
того, что~$C$ удовлетворяет причинно-следственным связям развития процесса 
выполнения проекта. Если для~$C$ нет характеристики~$\alpha$, которую можно 
отнести к~причине~$C$, то можно считать, что нарушена  
при\-чин\-но-след\-ст\-вен\-ная связь и~$C$~--- аномальный объект. 
  
  \smallskip
  
  \noindent
  \textbf{Пример.} Если объект~$C$ состоит в~получении суммы~$a$ 
фирмой~$K$, то согласно теореме~2 в~пред\-шест\-ву\-ющем объекте должна 
существовать причина перевода суммы~$a$ на фирму~$K$. Если эта причина 
в~проекте отсутствует, то это можно считать признаком мошеннической схемы. 
Все проекты по предположению собираются из <<кубиков>>, содержащихся в~БЗ. 
Тогда можно сравнить цену объекта~$C$, породившего получение суммы~$a$, 
и~сумму, присутствующую в~смете проекта. Если разница велика, то это либо 
ошибка проекта, либо признак мошеннической схемы.
  
  \section{Поиск противоречий на~основе~принципа <<необходимых~условий>>}
   
  Как было показано в~разд.~3, нахождение противоречий соответствуют 
движению от следствия к~причине. Для каждого объекта в~наблюдаемых данных 
выявление причин его появления является трудоемкой задачей. Кроме того, при 
реализации контроля соблюдения при\-чин\-но-след\-ст\-вен\-ных связей на 
большом множестве участников экономической деятельности задача анализа 
причин становится трудоемкой. Поэтому процедуру контроля необходимо разбить 
на два этапа, где первый этап состоит в~анализе простых <<необходимых 
условий>> проявления мошенничества, когда используется хотя бы одна 
известная характеристика причины. Второй этап (в~режиме офлайн) состоит 
в~выявлении причин, позволяющих провести анализ источников мошеннических 
схем. 
  
  Один из подходов к~выбору <<необходимых условий>> состоит в~построении 
множества подцелей исходной цели проекта (структурный метод построения 
проекта~\cite{7-gr}). Каждая подцель описывается диаграммой на рис.~1, 
и~реализации подцелей должны образовывать полный функционал цели. Это 
является необходимым, но не достаточным условием достижения цели, так как 
при таком подходе отсутствует компонент согласования всех подцелей в~единую 
систему. Однако такой подход значительно упрощает анализ выполнения проекта 
на предмет поиска мошенничества. Если признаки мошенничества будут 
обнаружены в~реализации хотя бы одной из подцелей, то это значит, что 
мошенничество присутствует в~реализации всего проекта. 
  
  Аналогично в~реализации каждого этапа в~любой из подцелей можно выделять 
простые <<необходимые условия>> нарушения при\-чин\-но-след\-ст\-венн\-ых 
связей. 
  
  Таким образом, получается множество <<необходимых условий>>, нарушение 
которых свидетельствует о наличии мошенничества. Это множество 
<<необходимых условий>> можно назвать метаданными~[8, 9] для контроля 
проекта на выявление мошенничества. 
  
  
  \section{Заключение }
  
  В поиске противоречий необходимо от транзакций, соответствующих 
следствиям при\-чин\-но-след\-ст\-вен\-ных связей, переходить к~анализу причин 
наблюдаемых следствий. Это сложная задача, которая связана с~описанием причин 
определенных свойств. 
  
  В работе представлена модель, позволяющая строить множество необходимых 
условий соответствия наблюдаемого следствия вызвавшей его причине. Этот 
подход делает поиск противоречий вполне вычислимой задачей, но не гарантирует 
успех. 
  
  {\small\frenchspacing
 {%\baselineskip=10.8pt
 \addcontentsline{toc}{section}{References}
 \begin{thebibliography}{9}
\bibitem{1-gr}
\Au{Грушо А.\,А., Зацаринный~А.\,А., Тимонина~Е.\,Е.} Блокчейны цифровой экономики на базе 
системы ситуационных центров и~централизованного консенсуса~// Радиолокация, навигация, 
связь: Мат-лы XXV Междунар. научн.-технич. конф.~---
Воронеж: Издательский дом ВГУ, 2019. Т.~6. С.~183--191. 
\bibitem{2-gr}
\Au{Grusho A., Zatsarinny~A., Timonina~E.} A~system approach to information security in 
distributed ledgers on the situational centers platform.~---
Lecture notes in computer science ser.~--- Springer, 2019 
(in press).
\bibitem{3-gr}
\Au{Финн В.\,К.} Искусственный интеллект: Методология, применения, философия.~--- М.: 
Красанд, 2011. 448~с.

\bibitem{5-gr} %4
\Au{Аншаков~О.\,М., Фабрикантова~Е.\,Ф.} ДСМ-ме\-тод автоматического порождения 
гипотез: Логические и~эпистемологические основания.~--- М.: Либроком, 2009. 432~с.

\bibitem{4-gr} %5
\Au{Poelmans J., Elzinga~P., Viaene~S., Dedene~G.} Formal concept analysis in knowledge 
discovery: A~survey~// Conceptual structures: From information to intelligence~/ Eds.\ M.~Croitoru, 
S.~Ferr$\acute{\mbox{e}}$, and D.~Lukose.~--- Lecture notes in computer science 
ser.~--- Berlin--Heidelberg: Springer, 2010. Vol.~6208.  P.~139--153.

\bibitem{6-gr}
\Au{Панкратова~Е.\,С., Финн~В.\,К.} Автоматическое по\-рож\-де\-ние гипотез в~интеллектуальных 
системах.~--- М.: Либроком, 2009. 528~с. 
\bibitem{7-gr}
\Au{Денисов А.\,А., Колесников~Д.\,Н.} Теория больших систем управления.~--- Л.: Энергоиздат, 1982. 488~с.

\bibitem{9-gr}
\Au{Грушо А.\,А., Грушо Н.\,А., Забежайло~М.\,И., Смирнов~Д.\,В., Тимонина~Е.\,Е.} 
Параметризация в~прикладных задачах поиска эмпирических причин~// Информатика и~её 
применения, 2018. Т.~12. Вып.~3. С.~62--66.

\bibitem{8-gr}
\Au{Грушо А.\,А., Грушо Н.\,А., Левыкин~М.\,В., Тимонина~Е.\,Е.} Методы идентификации 
захвата хоста в~распределенной ин\-фор\-ма\-ци\-он\-но-вы\-чис\-ли\-тель\-ной сис\-те\-ме, 
защищенной с~помощью метаданных~// Информатика и~её применения, 2018. Т.~12. Вып.~4. 
С.~41--45.

 \end{thebibliography}

 }
 }

\end{multicols}

\vspace*{-3pt}

\hfill{\small\textit{Поступила в~редакцию 03.04.19}}

%\vspace*{8pt}

%\pagebreak

\newpage

\vspace*{-28pt}

%\hrule

%\vspace*{2pt}

%\hrule

%\vspace*{-2pt}

\def\tit{ARCHITECTURAL DECISIONS IN~THE~PROBLEM 
OF~IDENTIFICATION OF~FRAUD IN~THE~ANALYSIS 
OF~INFORMATION FLOWS IN~DIGITAL ECONOMY\\[-5pt]}


\def\titkol{Architectural decisions in~the~problem 
of~identification of~fraud in~the~analysis 
of~information flows in~digital economy}

\def\aut{A.\,A.~Grusho, M.\,I.~Zabezhailo, N.\,A.~Grusho, and~E.\,E.~Timonina}

\def\autkol{A.\,A.~Grusho, M.\,I.~Zabezhailo, N.\,A.~Grusho, and~E.\,E.~Timonina}

\titel{\tit}{\aut}{\autkol}{\titkol}

\vspace*{-13pt}


 \noindent
   Institute of Informatics Problems, Federal Research Center ``Computer Sciences and 
Control'' of the Russian Academy of Sciences; 44-2~Vavilov Str., Moscow 119133, 
Russian Federation

\def\leftfootline{\small{\textbf{\thepage}
\hfill INFORMATIKA I EE PRIMENENIYA~--- INFORMATICS AND
APPLICATIONS\ \ \ 2019\ \ \ volume~13\ \ \ issue\ 2}
}%
 \def\rightfootline{\small{INFORMATIKA I EE PRIMENENIYA~---
INFORMATICS AND APPLICATIONS\ \ \ 2019\ \ \ volume~13\ \ \ issue\ 2
\hfill \textbf{\thepage}}}

\vspace*{3pt}


   
     
   \Abste{An approach to a~research of some types of fraud in digital economy with the usage of relationships of 
cause and effect is formulated. In all types of the considered frauds, the discrepancy between the 
purposes of financial transactions and actual cost of achievement of these purposes
has to be observed. Data on 
transactions can be collected by observing information flows in which these transactions are reflected. 
The architecture of data collection and their analysis can be organized by means of the distributed 
ledgers with the centralized consensus that allows creating an analog of the electronic account book 
fixing financial and economic activity of subjects of digital economy in the region. 
   The methods of fraud identification considered are based on the contradictions 
between actions described in transactions and information, which is contained in plans, standards, 
precedents, etc. 
   The method based on a~simplified scheme of implementation of the abstract project is considered. 
For identification of contradictions, it is necessary to carry out the analysis from the effect to the cause, 
i.\,e., to look for anomalies in information describing the generation of the observed effects. 
   It is shown how in implementation of the project it is possible to allocate simple ``necessary 
conditions'' of violation of cause and effect relationships, i.\,e., a~set of ``necessary conditions'' 
violation of which demonstrates fraud existence. It is possible to call this set of "necessary conditions" 
by metadata for control of the project for fraud identification.} 
   
   \KWE{digital economy; information flows; relationships of reason and effect; detection of 
fraudulent schemes}
   
  

 \DOI{10.14357/19922264190204}

\vspace*{-20pt}

 \Ack
   \noindent
   The work was partially supported by the Russian Foundation for Basic Research (projects  
18-29-03081 and 18-07-00274).



%\vspace*{6pt}

  \begin{multicols}{2}

\renewcommand{\bibname}{\protect\rmfamily References}
%\renewcommand{\bibname}{\large\protect\rm References}

{\small\frenchspacing
 {\baselineskip=10.5pt
 \addcontentsline{toc}{section}{References}
 \begin{thebibliography}{9}
\bibitem{1-gr-1}
\Aue{Grusho, A.\,A., A.\,A.~Zatsarinny, and E.\,E.~Timonina.} 2019. Blokcheyny tsifrovoy ekonomiki 
na baze sistemy situatsionnykh tsentrov i~tsentralizovannogo konsensusa [Blockchains of digital 
economy on the basis of the system of the situational centres and the centralized consensus]. 
\textit{25th Scientific and Technical Conference (International) ``Radar-Location, Navigation, 
Communication'' Proceedings}. Voronezh: VSU Publs. 6:183--191.
\bibitem{2-gr-1}
\Aue{Grusho, A., A.~Zatsarinny, and E.~Timonina.} 2019 (in press). 
A~system approach to information security 
in distributed ledgers on the situational centers platform. 
Lecture notes in computer science ser. Springer.
\bibitem{3-gr-1}
\Aue{Finn, V.\,K.} 2011. \textit{Iskusstvennyy intellekt: Metodologiya, primeneniya, filosofiya} 
[Artificial intelligence: Methodology, applications, philosophy]. Moscow: KRASAND. 448~p.

\bibitem{5-gr-1}
\Aue{Anshakov, O.\,M., and E.\,F.~Fabrikantova}. 2009. \textit{DSM-metod avtomaticheskogo porozhdeniya gipotez: Logicheskie 
i~epistemologicheskie osnovaniya} [JSM-method of automatic hypothesis generation: Logical and 
epistemological]. Moscow: KD LIBROKOM. 432~p.
\bibitem{4-gr-1} %5
\Aue{Poelmans, J., P.~Elzinga, S.~Viaene, and G.~Dedene.} 2010. Formal concept analysis in 
knowledge discovery: A~survey. \textit{Conceptual structures: From information to intelligence}. 
Eds.\ M.~Croitoru, S.~Ferr$\acute{\mbox{e}}$, and D.~Lukose. Lecture notes in 
computer science ser. Berlin--Heidelberg: Springer. 6208:139--153.

\bibitem{6-gr-1}
\Aue{Pankratov, E.\,S., and V.\,K.~Finn}. 
2009. \textit{Avtomaticheskoe porozhdenie gipotez v~intellektual'nykh 
sistemakh} [Automatic hypotheses generation in intelligent systems]. Moscow: KD 
\mbox{LIBROKOM}.  528~p. 
\bibitem{7-gr-1}
\Aue{Denisov, A.\,A., and D.\,N.~Kolesnikov.} 1982. \textit{Teoriya bol'shikh 
sistem upravleniya} [Theory of big control systems]. Leningrad: Energoizdat. 488~p.

\bibitem{9-gr-1}
\Aue{Grusho, A.\,A., N.\,A.~Grusho, M.\,I.~Zabezhailo, D.\,V.~Smirnov, and 
E.\,E.~Timonina.} 2018. 
Parametrizatsiya v~prikladnykh zadachakh poiska empiricheskikh prichin 
[Parametrization in applied 
problems of search of the empirical reasons]. 
\textit{Informatika i~ee Primeneniya~--- 
Inform. Appl.} 12(3):62--66.

\bibitem{8-gr-1}
\Aue{Grusho, A.\,A., N.\,A.~Grusho, M.\,V.~Levykin, and E.\,E.~Timonina.} 2018. Metody 
identifikatsii zakhvata khosta v~raspredelennoy informatsionno-vychislitel'noy sisteme, 
zashchishchennoy s~pomoshch'yu metadannykh [Methods of identification of host capture 
in the  distributed information system which is protected on the base of meta data].
\textit{Informatika i~ee 
Primeneniya~--- Inform. Appl.} 12(4):41--45.
{ %\looseness=1

}

\end{thebibliography}

 }
 }

\end{multicols}

\vspace*{-12pt}

\hfill{\small\textit{Received April 3, 2019}}

%\pagebreak

%\vspace*{-18pt}

\Contr

\noindent
\textbf{Grusho Alexander A.} (b.\ 1946)~--- Doctor of Science in physics and 
mathematics, professor, principal scientist, Institute of Informatics Problems, 
Federal Research Center ``Computer Sciences and Control'' of the Russian 
Academy of Sciences; 44-2~Vavilov Str., Moscow 119133, Russian Federation; 
\mbox{grusho@yandex.ru} 

\vspace*{3pt}

\noindent
\textbf{Zabezhailo Michael I.} (b.\ 1956)~--- Doctor of Science in physics and 
mathematics, principal scientist, Institute of Informatics Problems, Federal Research 
Center ``Computer Sciences and Control'' of the Russian Academy of Sciences;  
44-2~Vavilov Str., Moscow 119133, Russian Federation; 
\mbox{m.zabezhailo@yandex.ru} 

\vspace*{3pt}


\noindent
\textbf{Grusho Nikolai A.} (b.\ 1982)~--- Candidate of Science (PhD) in physics 
and mathematics, senior scientist, Institute of Informatics Problems, Federal 
Research Center ``Computer Sciences and Control'' of the Russian Academy of 
Sciences; 44-2~Vavilov Str., Moscow 119133, Russian Federation; 
\mbox{info@itake.ru} 

\vspace*{3pt}


\noindent
\textbf{Timonina Elena E.} (b.\ 1952)~--- Doctor of Science in technology, 
professor, leading scientist, Institute of Informatics Problems, Federal Research 
Center ``Computer Sciences and Control'' of the Russian Academy of Sciences;  
44-2~Vavilov Str., Moscow 119133, Russian Federation; 
\mbox{eltimon@yandex.ru} 

\label{end\stat}

\renewcommand{\bibname}{\protect\rm Литература}   %11
\def\stat{listopad}

\def\tit{ЖИЗНЕННЫЙ ЦИКЛ МЕТОДОЛОГИИ ПОСТРОЕНИЯ РЕФЛЕКСИВНО-АКТИВНЫХ 
СИСТЕМ ИСКУССТВЕННЫХ ГЕТЕРОГЕННЫХ ИНТЕЛЛЕКТУАЛЬНЫХ АГЕНТОВ$^*$}

\def\titkol{Жизненный цикл методологии построения РАСИГИА} %рефлексивно-активных систем искусственных гетерогенных интеллектуальных агентов}

\def\aut{С.\,В.~Листопад$^1$}

\def\autkol{С.\,В.~Листопад}

\titel{\tit}{\aut}{\autkol}{\titkol}

\index{Листопад С.\,В.}
\index{Listopad S.\,V.}


{\renewcommand{\thefootnote}{\fnsymbol{footnote}} \footnotetext[1]
{Исследование выполнено за счет гранта Российского научного фонда №\,23-21-00218, 
{\sf https://rscf.ru/project/23-21-00218/}.}}


\renewcommand{\thefootnote}{\arabic{footnote}}
\footnotetext[1]{Федеральный исследовательский центр <<Информатика и~управ\-ле\-ние>> Российской академии наук, 
\mbox{ser-list-post@yandex.ru}}

%\vspace*{-12pt}

  
  

  \Abst{Представлена темпоральная структура (жизненный цикл) методологии построения 
рефлексивно-активных систем искусственных гетерогенных интеллектуальных агентов (\mbox{РАСИГИА}). 
Такие системы предназначены для компьютерного моделирования процессов и~эффектов, 
возникающих при решении практических проблем коллективами специалистов под 
руководством лица, принимающего решения. Искусственные гетерогенные 
интеллектуальные агенты реф\-лек\-сив\-но-ак\-тив\-ных сис\-тем~--- активные субъекты, 
способные к~рассуждениям, коммуникации и~рефлексии как умению моделировать 
рассуждения других агентов системы и~себя самих. Моделирование рефлексивных процессов 
обеспечивает выработку агентами согласованного представления об объекте управ\-ле\-ния, 
цели коллективной работы и~нормах взаимодействия, позволяя системе в~ходе 
самоорганизации генерировать заново релевантный гибридный интеллектуальный метод 
решения очередной проб\-лемы.} 
  
  \KW{рефлексия; методология; рефлексивно-активная сис\-те\-ма искусственных 
гетерогенных интеллектуальных агентов; гибридная интеллектуальная многоагентная 
система; коллектив специалистов}

\DOI{10.14357/19922264240112}{GUAMVE}
  
%\vspace*{-6pt}


\vskip 10pt plus 9pt minus 6pt

\thispagestyle{headings}

\begin{multicols}{2}

\label{st\stat}

\section{Введение}

  Компьютерное моделирование процессов и~эффектов, возникающих при 
решении практических проблем коллективами специалистов, каждый из 
которых обладает собственным опытом, знаниями и~пониманием предметной 
области,~--- перспективное на\-прав\-ле\-ние научных разработок, которое 
Д.\,А.~Поспелов выделял как одну из десяти горячих точек в~исследованиях по 
искусственному\linebreak интеллекту~[1]. Для компьютерного моделирования 
рас\-суж\-де\-ний коллективов специалистов предлагается создание \mbox{РАСИГИА} 
в~рамках многоагентного подхода~[2] на основе модели 
\mbox{ги\-брид\-ных} интеллектуальных многоагентных сис\-тем~[3]. Агенты 
\mbox{РАСИГИА}~--- активные программные сущности, способные 
рас\-суж\-дать, взаимодействовать и~рефлексировать. Рефлексивное 
моделирование агентами друг друга обеспечивает выработку согласованного 
пред\-став\-ле\-ния об объекте управ\-ле\-ния, \mbox{цели} коллективной работы и~нормах 
взаимодействия, а~также эволюцию \mbox{РАСИГИА} в~ходе 
самоорганизации в~сильном смыс\-ле. В~на\-сто\-ящей работе рас\-смат\-ри\-ва\-ют\-ся 
вопросы создания методологии разработки сис\-тем такого класса, которая 
понимается как учение об организации продуктивной де\-я\-тель\-ности 
в~це\-лост\-ную сис\-те\-му с~чет\-ко определенными характеристиками, логической 
структурой и~процессом ее осуществления (темпоральной структурой)~[4]. 
Характеристики (особенности и~принципы) и~логическая структура (субъект, 
объект, предмет, методы, средства, результат) методологии разработки 
\mbox{РАСИГИА} рас\-смот\-ре\-ны в~[5]. Данная работа по\-свя\-ще\-на разработке 
жизненного цик\-ла (темпоральной структуры) предлагаемой методологии.

\begin{figure*} %fig1
\vspace*{1pt}
      \begin{center}
     \mbox{%
\epsfxsize=148.855mm 
\epsfbox{lis-1.eps}
}
\end{center}
%\vspace*{-9pt}

{\small Темпоральная структура методологии построения РАСИГИА: \textit{1}~--- этап методологии; \textit{2}~--- стадия методологии;
\textit{3}~--- граница фазы методологии; \textit{4}~--- отношение следования при нормальном завершении этапа;  
\textit{5}~--- возврат к~предыдущим этапам при выявлении допущенных на них недочетов}
\end{figure*}

\vspace*{-6pt}
  
\section{Темпоральная структура методологии}

\vspace*{-6pt}

  Укрупненно в~жизненном цикле методологии построения \mbox{РАСИГИА}, 
показанном на рисунке, могут быть выделены проектная, технологическая 
и~рефлексивная фазы, которые со\-сто\-ят из стадий и~этапов. Как видно, 
последовательное выполнение этапов методологии приводит к~же\-ла\-емо\-му 
результату лишь в~идеальном случае, когда проектировщик сразу получает всю 
необходимую достоверную информацию, имеет необходимый арсенал методов, 
не совершает ошибок ни на одном из этапов и,~по сути, заранее знает, какой 
должна быть разрабатываемая \mbox{РАСИГИА}. В~реальности на каждом из 
этапов могут обнаруживаться ранее допущенные недочеты, требующие 
возврата к~соответствующему этапу, их исправления и~повторного выполнения 
проделанной работы с~новыми исходными данными. В~определенном смыс\-ле 
такой подход представляет собой метод проб и~ошибок, и~чем слож\-нее 
проблема, для которой проектируется \mbox{РАСИГИА}, с~точ\-ки зрения 
конкретного коллектива разработчиков, тем больше будет возвратов в~ходе 
проектирования системы~[6]. Рас\-смот\-рим по\-дроб\-нее каждую из фаз 
методологии.



\section{Проектная фаза}

  Проектная фаза включает в~себя стадии концептуального описания проб\-ле\-мы и~моделирования, выполняемые сис\-тем\-ны\-ми аналитиками из коллектива 
разработчиков. В~рамках первой стадии фазы на доформальном, 
содержательном уровне рас\-смат\-ри\-ва\-ет\-ся проб\-ле\-ма как отрицательное 
отношение субъекта к~реальности~[6] и~проблемная ситуация как объективное 
стечение обстоятельств, обуслов\-ли\-ва\-ющее проб\-ле\-му. Данная стадия со\-сто\-ит из 
сле\-ду\-ющих этапов:
  \begin{itemize}
\item формулирование проб\-ле\-мы, ее предварительное описание в~ходе 
интервьюирования лица, при\-ни\-ма\-юще\-го решение, его советников и~активных 
групп на естественном языке с~использованием привычных для них 
определений и~формулировок~[7];
  \item определение проб\-ле\-ма\-ти\-ки, т.\,е.\ комплекса проб\-лем, связанных 
с~рас\-смат\-ри\-ва\-емой~[4], чтобы учесть создаваемые ее решением последствия 
для каж\-дой из них. Необходимо охватить весь круг участников проб\-лем\-ной 
ситуации (стейкхолдеров, заинтересованных лиц): непосредственных 
участников ситуации, пред\-ста\-ви\-те\-лей проб\-ле\-мо\-раз\-ре\-ша\-ющих 
и~проб\-ле\-мо\-со\-дер\-жа\-щих сис\-тем, же\-ла\-емых помощников или союзников, 
субъектов, связанных с~ситуацией юридически, лиц с~возможным негативным 
отношением к~решению проб\-ле\-мы~[6]. Для по\-стро\-ения проб\-ле\-ма\-ти\-ки может 
быть использована, например, технология Дж.~Уор\-фил\-да, подходы 
с~использованием метафор организации, взгляда на проблему стейкхолдером 
с~раз\-ных точек зрения, рас\-смот\-ре\-ния проб\-ле\-мы в~рамках различных парадигм 
(функциональной, объяснительной, освободительной, пост\-мо\-дер\-нист\-ской)~[4, 6]. 
Формируется древовидная или сетевая структура в~виде диаграммы связей, 
концептуальной кар\-ты или аналогичных инструментов;
  \item определение целей проектирования \mbox{РАСИГИА}, 
пред\-по\-ла\-га\-ющее проведение собеседований с~каж\-дым стейк\-хол\-де\-ром, 
выяснение их целей и~пожеланий, формирование и~структурирование 
множества целей в~виде дерева или сетевидной структуры и~его 
визуализация~[4, 6]. Выделяются следующие уровни целей: ожи\-да\-емые 
в~плановом периоде результаты; задачи, которые не будут решены 
в~рас\-смат\-ри\-ва\-емом периоде, но будет достигнут существенный прогресс на 
пути к~ним; не\-до\-сти\-жи\-мые идеалы, к~которым следует стремиться~[8];
  \item выбор критериев, т.\,е.\ до\-ступ\-ных для наблюдения и~измерения 
характеристик, опи\-сы\-ва\-ющих важ\-ные особенности объектов или процессов 
и~поз\-во\-ля\-ющих сравнивать \mbox{пред\-ла\-га\-емые} альтернативы, контролировать 
процесс решения~[6]. Со\-во\-куп\-ность критериев долж\-на быть релевантной 
количественной моделью выделенных ранее качественных целей. Отдельно 
выделяются ограничения, фик\-си\-ру\-ющие условия, которые не могут нарушаться 
при до\-сти\-же\-нии цели;
  \item оценка концептуального описания проб\-ле\-мы в~ходе специально 
спланированного эксперимента. Если существует коллектив специалистов, 
решающий на практике по\-став\-ле\-нную или схожие проб\-ле\-мы, он выступает 
образцом, прототипом создаваемой сис\-те\-мы агентов. В~этом случае 
выполняется наблюдение за работой такого коллектива в~рамках решения 
реальных или тренировочных проб\-лем и~оценка релевантности 
зафиксированной информации сведениям, полученным в~ходе предыду\-щих 
этапов. Если выявлено существенное рас\-хож\-де\-ние, выполняется возврат 
к~этапу, в~рамках которого были получены некорректные сведения. Сведения 
о~составе участников коллектива, вы\-де\-ля\-емых ими под\-проб\-ле\-мах, методах их 
решения используются на по\-сле\-ду\-ющих этапах проектирования 
\mbox{РАСИГИА} при по\-стро\-ении со\-от\-вет\-ст\-ву\-ющих моделей проб\-ле\-мы 
и~сис\-те\-мы <<как есть сейчас>>. Данные о~качестве принятых решений 
и~дли\-тель\-ности их выработки используются в~дальнейшем как показатель 
эффекта от разработки и~внед\-ре\-ния \mbox{РАСИГИА}. Если подобных 
коллективов нет или не\-воз\-мож\-но реализовать со\-от\-вет\-ст\-ву\-ющий эксперимент, 
данный этап отсутствует.
  \end{itemize}
  
  Стадия моделирования предполагает разработку формализованного описания 
проб\-ле\-мы, коллектива специалистов, ре\-ша\-юще\-го проб\-ле\-му на момент 
разработки \mbox{РАСИГИА}, если он существует, и~самой 
\mbox{РАСИГИА}. Модели строятся с~использованием визуального 
метаязыка~[9], что позволяет наглядно их изобразить, а~так\-же поз\-во\-ля\-ет 
с~использованием заранее заданных соответствий однозначно отоб\-ра\-зить 
графическое пред\-став\-ле\-ние моделей в~формальное символьное пред\-став\-ле\-ние, 
пригодное для компьютерной интерпретации. Данная стадия со\-сто\-ит из 
сле\-ду\-ющих этапов:
  \begin{itemize}
  \item моделирование проб\-ле\-мы, которое обеспечивает ее формальное 
пред\-став\-ле\-ние на макро- и~мик\-ро\-уров\-не. Мак\-ро\-уров\-не\-вая модель описывает 
проб\-ле\-му как <<чер\-ный ящик>>, отражая ее место в~ме\-та\-проб\-ле\-ме (проб\-ле\-ме 
более высокого уровня), свойства как целого и~связи с~другими проб\-ле\-ма\-ми 
ме\-та\-проб\-ле\-мы. Атрибуты проблемы на макроуровне~--- цели, критерии 
(включая ограничения), исходные данные и~идентификатор. Мик\-ро\-уров\-не\-вая 
модель раскрывает со\-став и~структуру проб\-ле\-мы, описывает ее под\-проб\-ле\-мы 
и~связи между ними. Для каждой под\-проб\-ле\-мы специфицируются цели, 
критерии, исходные данные и~идентификатор, выполняется поиск релевантных 
методов решения. Если такие методы найдены, дальнейшая декомпозиция 
под\-проб\-ле\-мы не требуется, иначе выполняется по\-стро\-ение ее мик\-ро\-уров\-не\-вой 
модели, т.\,е.\ модели более глубокого уров\-ня иерархии. Таким образом, 
формируется многоуровневая иерархическая структура по\-став\-лен\-ной 
проб\-лемы;
  \item моделирование коллектива, которое отражает ситуацию решения 
проб\-ле\-мы <<как есть сейчас>> со всеми ее преимуществами и~недостатками. 
Модель коллектива~--- основа, образец для проектирования \mbox{РАСИГИА} и~оценки эф\-фек\-тив\-ности 
альтернативных конфигураций \mbox{РАСИГИА}. 
При моделировании коллектива специалистов фиксируется его со\-став в~виде 
множества ролей участников, час\-ти проб\-ле\-мы, ре\-ша\-емые каж\-дым из 
участников с~определенной ролью, знания и~методы, ис\-поль\-зу\-емые 
участниками для решения своей части проб\-ле\-мы, а~так\-же порядок и~нормы 
взаимодействия участников коллектива; 
  \item моделирование \mbox{РАСИГИА}, фор\-ми\-ру\-ющее идеализированное 
пред\-став\-ле\-ние <<как должно стать>> о~коллективе интеллектуальных агентов, 
ре\-ша\-ющих по\-став\-лен\-ную проб\-ле\-му. В~ходе моделирования \mbox{РАСИГИА} 
должны быть специфицированы со\-став и~иерархия ролей агентов, множество 
агентов, ис\-поль\-зу\-емые протоколы взаимодействия, под\-дер\-жи\-ва\-емые языки 
передачи сообщений, базовая онтология как осно\-ва для интерпретации 
семантики пе\-ре\-да\-ва\-емых сообщений, модель окру\-жа\-ющей среды, содержащая 
в~том чис\-ле пул, из которого агенты могут привлекаться сис\-те\-мой по мере 
не\-об\-хо\-ди\-мости и~в~который попадают ис\-клю\-ча\-емые из нее агенты, множество 
моделей архитектур \mbox{РАСИГИА}, множество необходимых моделей 
мак\-ро\-уров\-не\-вых эффектов. В~множестве агентов должны присутствовать 
агенты, пред\-став\-ля\-ющие стейк\-хол\-де\-ров с~их целями, критериями достижения 
цели и~ограничениями. Если на предыду\-щем этапе была по\-стро\-ена модель 
коллектива, то одна из архитектур \mbox{РАСИГИА} долж\-на соответствовать 
данной модели. 
  \end{itemize}
  
\section{Технологическая фаза}

  Технологическая фаза включает в~себя разработку эскизного проекта 
\mbox{РАСИГИА}, ее технического проекта и~программной реализации. 
Стадия разработки эскизного проекта \mbox{РАСИГИА} обеспечивает 
пред\-став\-ле\-ние создаваемой сис\-те\-мы и~ее внеш\-ней среды в~виде 
взаимосвязанных мо\-ду\-лей-бло\-ков в~соответствии с~моделью 
\mbox{РАСИГИА}, по\-стро\-ен\-ной на стадии проектирования. Данная стадия 
со\-сто\-ит из сле\-ду\-ющих этапов:
  \begin{itemize}
  \item разработка функциональной структуры, в~ходе которой строится 
множество взаимосвязанных схем-диа\-грамм, определяющих под\-сис\-те\-мы 
РАСИГИА, распределение агентов по ним, функционал агентов, до\-пус\-ти\-мые 
языки передачи сообщений и~протоколы взаимодействия для каж\-дой пары или 
группы ролей агентов, технологические элементы сис\-те\-мы, потоки 
информации и~управ\-ле\-ния, а~также отношения, воз\-ни\-ка\-ющие между агентами 
в~процессе решения проб\-лем. Для каждой роли указывается множество 
релевантных ей уже существующих (разработанных ранее для других сис\-тем) 
агентов, если таковые имеются. В~случае отсутствия релевантных агентов они 
должны быть разработаны на сле\-ду\-ющих этапах. Кроме того, 
специфицируются функциональные мо\-ду\-ли-бло\-ки, от\-ве\-ча\-ющие за организацию 
макроуровневых эффектов в~\mbox{РАСИГИА};
  \item разработка структуры внешней среды по аналогии с~разработкой 
функциональной структуры \mbox{РАСИГИА} предполагает построение схем-диа\-грамм, описывающих виртуальную внеш\-нюю среду, ее под\-сис\-те\-мы, роли 
агентов и~способы взаимодействия \mbox{РАСИГИА} с~ними, т.\,е.\ языки 
передачи сообщений и~протоколы взаимодействия, отношения, потоки 
информации и~управ\-ле\-ния. Для каж\-дой роли указываются су\-щест\-ву\-ющие 
релевантные ей агенты, если они имеются;
  \item разработка архитектур агентов выполняется для тех ролей 
в~функциональной структуре и~структуре внеш\-ней среды, для которых не 
найдено релевантных реализованных агентов. Архитектура агента~--- схема, 
описывающая со\-став, структуру и~взаимосвязь функ\-ций-бло\-ков, 
ре\-а\-ли\-зу\-емых агентом, обеспечивающая выполнение им своего предназначения. 
Для каждой функ\-ции-бло\-ка указывается метод или алгоритм, с~по\-мощью 
которого она реализуется, в~случае если таковые отсутствуют, они долж\-ны 
быть разработаны в~рамках сле\-ду\-ющей стадии.
  \end{itemize}
  
  Стадия разработки технического проекта \mbox{РАСИГИА} обеспечивает 
создание недостающих блоков для ее агентов или технологических элементов. 
При этом может по\-тре\-бо\-вать\-ся разработка методов решения под\-проб\-лем, 
алгоритмов на основе метода, баз данных, онтологий и~др. Порядок их 
разработки не регламентируется на\-сто\-ящей методологией в~связи 
с~существенным разнообразием и~не\-воз\-мож\-ностью совместного рас\-смот\-ре\-ния. 
На данной стадии должен быть сформирован технический проект, 
опи\-сы\-ва\-ющий для каждого блока со\-став, структуру и~форму пред\-став\-ле\-ния 
входных и~выходных данных, алгоритм его функционирования, спецификацию 
необходимых технических средств~[10].
  
  Стадия программной реализации и~отладки предполагает разработку 
программного кода \mbox{РАСИГИА} и~его тестирование на предмет 
корректной работы с~\mbox{целью} формирования полноценного программного 
продукта, а~так\-же разработку программной документации. Данная стадия 
со\-сто\-ит из сле\-ду\-ющих этапов:
  \begin{itemize}
  \item программная реализация и~разработка документации выполняется 
с~использованием платформы JaCaMo~[11], объединяющей технологию Jason 
для программирования автономных агентов, Cartago для программирования 
элементов внеш\-ней среды и~Moise для программирования многоагентных 
организаций. Кроме того, применяется язык Java для программирования 
отдельных элементов сис\-те\-мы и~тонкой настройки механизмов 
платформы~[12];
  \item тестирование и~отладка обеспечивают выявление и~устранение 
основных дефектов в~сис\-те\-ме. Ввиду того что полное тестирование  
сколь\-ко-ни\-будь слож\-ной программы не\-воз\-мож\-но~[13], выполняется 
выборочное тестирование в~сле\-ду\-ющем порядке: отдельные функ\-ции и~блоки 
из состава аген\-тов и~технологических элементов, межмодульные связи, агенты 
и~технологические элементы в~целом, протоколы взаимодействия агентов, 
\mbox{РАСИГИА} в~целом. В~тес\-ти\-ро\-ва\-нии принимают участие 
представители всех ролей команды разработчиков, так как каж\-дый из них 
выполняет поиск ошибок разного рода~[14]. При этом выделяется отдельная 
роль тестировщика, опре\-де\-ля\-юще\-го стратегию тес\-ти\-ро\-ва\-ния,  
тест-тре\-бо\-ва\-ния и~тест-пла\-ны для каждой из фаз проекта; он выполняет 
тестирование сис\-те\-мы, собирает и~анализирует отчеты о~про\-хож\-де\-нии 
тестирования. 
\end{itemize}

\section{Рефлексивная фаза}

  Рефлексивная фаза предназначена для оценки показателей реализованной 
\mbox{РАСИГИА} и~процесса ее разработки, выявления ее недостатков и~при 
не\-об\-хо\-ди\-мости до\-ра\-бот\-ки как сис\-те\-мы, так и~методологии ее построения. 
Стадия оценки эф\-фек\-тив\-ности \mbox{РАСИГИА} предполагает сбор 
показателей работы сис\-те\-мы и~их сравнение с~целевыми значениями. Если 
выявляется их несоответствие, выполняется анализ причин отклонений, 
переход к~этапу методологии, вызвавшему их, и~повторное выполнение данного и~по\-сле\-ду\-ющих этапов с~учетом тре\-бу\-емых корректировок. Кроме того, на этой 
стадии продолжается отладка сис\-те\-мы. Данная стадия выполняется в~три этапа:
  \begin{enumerate}[(1)]
  \item оценка в~лабораторных условиях командой разработчиков, когда 
система работает в~виртуальной внешней среде, решая тестовые проб\-ле\-мы. На 
данной стадии оценка сис\-те\-мы выполняется вычислительными моделями 
стейкхолдеров, реализованными со\-от\-вет\-ст\-ву\-ющи\-ми агентами виртуальной 
внеш\-ней среды; 
  \item оценка по результатам тестовой эксплуатации, когда \mbox{РАСИГИА} 
функционирует в~реальной внеш\-ней среде параллельно с~традиционным 
методом решения проб\-ле\-мы и~выполняется сравнение их эф\-фек\-тив\-ности 
пользователями и~реальными стейк\-хол\-де\-ра\-ми. Первоначально 
у~\mbox{РАСИГИА} должна быть отключена воз\-мож\-ность оказывать ка\-кое-ли\-бо воздействие на реальную внеш\-нюю среду, а~результатом ее\linebreak
 работы 
долж\-ны стать рекомендации по оказанию таких воздействий. После 
удовлетворительной оценки пользователей и~стейк\-хол\-де\-ров \mbox{РАСИГИА} 
может быть переведена в~\mbox{автоматический} режим взаимодействия со средой, 
а~традиционный метод решения проб\-ле\-мы используется в~качестве резервного 
для проверки ее работы еще в~течение некоторого времени. Длительности 
каждого из этих периодов долж\-ны определяться заказчиками сис\-те\-мы для 
решения конкретной проб\-ле\-мы совместно с~коллективом разработчиков; 
  \item сопровождение после внед\-ре\-ния поз\-во\-ля\-ет собирать жалобы, замечания и~предложения в~процессе эксплуатации \mbox{РАСИГИА}, в~том чис\-ле от 
людей, которые ошибочно не были включены в~со\-став стейкхолдеров.\\[-13pt] 
  \end{enumerate}
  
  Стадия оценки и~корректировки методологии в~определенном смысле длится 
на протяжении всего проекта, так как для ее реализации долж\-ны вес\-тись 
протоколы де\-я\-тель\-ности разработчиков,\linebreak в~которых отмечается дли\-тель\-ность 
реализации каж\-до\-го этапа, выполненные возвраты и~их причины. Однако 
именно по завершении проекта выполняется рефлексия проделанной работы, 
когда разработчики долж\-ны проанализировать удачные и~провальные решения, 
причины рас\-хож\-де\-ния результатов с~планами, возвратов к~предыду\-щим этапам 
и~фазам разработки \mbox{РАСИГИА}, затягивания отдельных этапов 
разработки, из\-бы\-точ\-ность или, наоборот, не\-ин\-фор\-ма\-тив\-ность 
по\-стро\-ений~\cite{4-lis}. По результатам анализа в~методологию вносятся 
изменения в~статусе <<предложение>>, которые после под\-тверж\-де\-ния 
эф\-фек\-тив\-ности в~новых проектах закрепляются в~новой версии методологии.

\vspace*{-9pt}

\section{Заключение}

\vspace*{-3pt}

  В работе представлена темпоральная структура (жизненный цикл) 
разработки \mbox{РАСИГИА}, опи\-сы\-ва\-ющая процессы сис\-тем\-но\-го анализа 
проб\-ле-\linebreak мы, моделирования, эскизного и~технического \mbox{проектирования} сис\-те\-мы, 
ее программной реализации, отладки и~тестирования. 
Основной результат 
организации работ в~соответствии с~предложенной методологией~--- 
программная реализация \mbox{РАСИГИА}, релевантно моделирующая 
коллектив специалистов, со\-вмест\-но ре\-ша\-ющих по\-став\-лен\-ную проб\-ле\-му 
с~учетом ее слабой формализации, не\-од\-но\-род\-ности, сетевого характера условий 
и~целей, не\-опре\-де\-лен\-ности и~ди\-на\-мич\-ности~\cite{5-lis}. Кроме того, в~результате 
рефлексивной стадии методологии формируется ее новая версия или 
подтверждается эф\-фек\-тив\-ность су\-щест\-ву\-ющей, что представляется\linebreak 
дополнительным результатом работ. Таким образом, методология предполагает 
свое развитие, потенциально обеспечивающее ее ре\-ле\-вант\-ность \mbox{актуальным}
подходам к~проектированию и~реализации интеллектуальных информационных 
сис\-тем.

\vspace*{-9pt}
  
{\small\frenchspacing
 { %\baselineskip=10.6pt
 %\addcontentsline{toc}{section}{References}
 \begin{thebibliography}{99}
 
 \vspace*{-3pt}
 
  \bibitem{1-lis}
   \Au{Поспелов Д.\,А.} Десять <<горячих точек>> в~исследованиях по искусственному 
интеллекту~// Искусственный\linebreak\vspace*{-12pt}

\columnbreak

\noindent
 интеллект и~принятие решений, 2019. №\,4. С.~3--9. doi: 
10.14357/20718594190401. EDN: BAUHFV.
  
  \bibitem{2-lis}
\Au{Тарасов В.\,Б.} От многоагентных сис\-тем к~интеллектуальным организациям: 
философия, психология, информатика.~--- М.: Эдиториал УРСС, 2002. 348~с.
  \bibitem{3-lis}
  \Au{Колесников А.\,В., Кириков~И.\,А., Листопад~С.\,В.} Ги\-брид\-ные интеллектуальные 
сис\-те\-мы с~самоорганизацией: координация, со\-гла\-со\-ван\-ность, спор.~--- М.: ИПИ РАН, 2014. 
189~с.
  \bibitem{4-lis}
  \Au{Новиков А.\,М., Новиков~Д.\,А.} Методология.~--- М.: Синтег, 2007. 668~с.
  \bibitem{5-lis}
  \Au{Листопад С.\,В.} Характеристики и~логическая структура методологии по\-стро\-ения  
реф\-лек\-сив\-но-ак\-тив\-ных сис\-тем искусственных гетерогенных интеллектуальных 
агентов~// Сис\-те\-мы и~средства \mbox{информатики}, 2023. Т.~33. №\,4. С.~16--27. doi: 
10.14357/ 08696527230402. EDN: TRTHEI.
  \bibitem{6-lis}
  \Au{Тарасенко Ф.\,П.} Прикладной сис\-те\-мный анализ.~--- М.: 
КНОРУС, 2010. 224~с.
  \bibitem{7-lis}
  \Au{Ларичев О.\,И.} Вербальный анализ решений.~--- М.: Наука, 2006. 181~с.
  \bibitem{8-lis}
  \Au{Акофф Р.} Акофф о менеджменте~/ Пер.\ с~англ.~--- СПб.: Питер, 2002. 448~с.
  (\Au{Akoff~R.\,L.} Ackoff's best: His classic writings on management.~--- New 
York, NY, USA: Wiley, 1999. 368~p.)
  \bibitem{9-lis}
  \Au{Колесников А.\,В., Листопад~С.\,В., Румовская~С.\,Б., Данишевский~В.\,И.} 
Неформальная аксиоматическая тео\-рия ролевых визуальных моделей~// Информатика и~её 
применения, 2016. Т.~10. Вып.~4. С.~114--120.  doi: 10.14357/19922264160412. EDN: XGSIVN.
  \bibitem{10-lis}
  \Au{Черушева Т.\,В.} Проектирование программного обеспечения.~--- Пенза: ПГУ, 2014. 
172~с.
  \bibitem{11-lis}
  \Au{Boissier O., Bordini~R.\,H., Hubnerand~J., Ricci~A.} Multi-agent oriented programming: 
Programming multi-agent systems using JaCaMo.~--- Intelligent robotics and autonomous agents 
series.~--- Cambridge: The MIT Press, 2020. 264~p.
  \bibitem{12-lis}
  \Au{Смирнов С.\,С., Смольянинова~В.\,А.} Введение в~разработку многоагентных сис\-тем 
в~среде Jason. Основы программирования на языке AgentSpeak.~--- М.: \mbox{МИРЭА}, 2009. 136~с.
  \bibitem{13-lis}
  \Au{Канер~С., Фолк~Д., Нгуен~Е.\,К.} Тестирование про\-грам\-мно\-го обеспечения. 
Фундаментальные концепции менеджмента биз\-нес-при\-ло\-же\-ний~/
Пер. с~англ.~--- Киев: ДиаСофт, 
2001. 544~с. (\Au{Kaner~С., Falk~J., Nguyen~H.\,Q.} {Testing computer software}.~--- 
International Thomson Computer Press,  1999. 496~p.)
  \bibitem{14-lis}
  \Au{Романькова Т.\,Л.} Тестирование программного обеспечения. {\sf 
https://elib.gstu.by/bitstream/handle/220612/ 9860/416.pdf}.

\end{thebibliography}

 }
 }

\end{multicols}

\vspace*{-6pt}

\hfill{\small\textit{Поступила в~редакцию 25.11.23}}

%\vspace*{8pt}

%\pagebreak

\newpage

\vspace*{-28pt}

%\hrule

%\vspace*{2pt}

%\hrule



\def\tit{LIFE CYCLE OF METHODOLOGY FOR~CONSTRUCTING REFLEXIVE-ACTIVE SYSTEMS OF~ARTIFICIAL HETEROGENEOUS INTELLIGENT AGENTS}


\def\titkol{Life cycle of methodology for~constructing reflexive-active systems of~artificial heterogeneous intelligent agents}


\def\aut{S.\,V.~Listopad}

\def\autkol{S.\,V.~Listopad}

\titel{\tit}{\aut}{\autkol}{\titkol}

\vspace*{-8pt}


\noindent
Federal Research Center ``Computer Science and Control'' of the Russian Academy of 
Sciences, 44-2~Vavilov Str., Moscow 119333, Russian Federation

\def\leftfootline{\small{\textbf{\thepage}
\hfill INFORMATIKA I EE PRIMENENIYA~--- INFORMATICS AND
APPLICATIONS\ \ \ 2024\ \ \ volume~18\ \ \ issue\ 1}
}%
 \def\rightfootline{\small{INFORMATIKA I EE PRIMENENIYA~---
INFORMATICS AND APPLICATIONS\ \ \ 2024\ \ \ volume~18\ \ \ issue\ 1
\hfill \textbf{\thepage}}}

\vspace*{4pt}
  
  
   
   \Abste{The paper presents the temporal structure (life cycle) of the methodology for 
constructing reflexive-active systems of artificial heterogeneous intelligent agents. These systems 
are designed for computer modeling of processes and effects that arise when solving practical 
problems by teams of specialists under the guidance of a~decision maker. Artificial heterogeneous 
intelligent agents of reflexive-active systems are active subjects capable of reasoning, 
communication, and reflection as the ability to model the reasoning of other agents of the system 
and themselves. Modeling of reflexive processes ensures the development by agents of a~consistent 
understanding of the control object, the purpose of collective work, and the norms of interaction 
allowing the system to self-organize and re-develop a relevant hybrid intelligent method for solving 
the next problem.}
   
   \KWE{reflection; methodology; reflexive-active system of artificial heterogeneous intelligent 
agents; hybrid intelligent multiagent system; team of specialists}
   
 
   
\DOI{10.14357/19922264240112}{GUAMVE}

\vspace*{-8pt}

\Ack

\vspace*{-1pt}


     \noindent
     This work was supported by the Russian Science Foundation, project No.\,23-21-00218.


\vspace*{6pt}

  \begin{multicols}{2}

\renewcommand{\bibname}{\protect\rmfamily References}
%\renewcommand{\bibname}{\large\protect\rm References}

{\small\frenchspacing
 {\baselineskip=11.5pt
 \addcontentsline{toc}{section}{References}
 \begin{thebibliography}{99} 
  \bibitem{1-lis-1}
   \Aue{Pospelov, D.\,A.} 2019. Desyat' ``goryachikh tochek'' v~issledovaniyakh po 
iskusstvennomu intellektu [Ten hot topics in AI research]. \textit{Is\-kus\-stven\-nyy in\-tel\-lekt 
i~pri\-nya\-tie re\-she\-niy} [Artificial Intelligence and Decision Making] 4:3--9. doi: 
10.14357/20718594190401. EDN: BAUHFV.
  \bibitem{2-lis-1}
   \Aue{Tarasov, V.\,B.} 2002. \textit{Ot mnogoagentnykh sis\-tem k~in\-tel\-lek\-tu\-al'\-nym 
or\-ga\-ni\-za\-tsi\-yam: fi\-lo\-so\-fiya, psi\-kho\-lo\-giya, in\-for\-ma\-ti\-ka} [From multiagent systems to intelligent 
organizations: Philosophy, psychology, and computer science]. Moscow: Editorial URSS. 348~p.
  \bibitem{3-lis-1}
   \Aue{Kolesnikov, A.\,V., I.\,A.~Kirikov, and S.\,V.~Listopad.} 2014. \textit{Gib\-rid\-nye 
in\-tel\-lek\-tu\-al'\-nye sis\-te\-my s~sa\-mo\-or\-ga\-ni\-za\-tsiey: ko\-or\-di\-na\-tsiya, so\-gla\-so\-van\-nost', spor} [Hybrid 
intelligent systems with self-organization: Coordination, consistency, and dispute]. Moscow: IPI 
RAN. 189~p.
  \bibitem{4-lis-1}
   \Aue{Novikov, A.\,M., and D.\,A.~Novikov.} 2007. \textit{Me\-to\-do\-lo\-giya} [Methodology]. 
Moscow: SINTEG. 668~p.
  \bibitem{5-lis-1}
   \Aue{Listopad, S.\,V.} 2023. Kharakteristiki i~logicheskaya struk\-tu\-ra me\-to\-do\-lo\-gii po\-stro\-eniya 
refleksivno-aktivnykh sis\-tem is\-kus\-stven\-nykh ge\-te\-ro\-gen\-nykh in\-tel\-lek\-tu\-al'\-nykh agen\-tov 
[Characteristics and logical structure of the methodology for constructing reflexive-active systems 
of artificial heterogeneous intelligent agents]. \textit{Sistemy i~Sredstva Informatiki~--- Systems 
and Means of Informatics} 33(4):16--27. doi: 10.14357/08696527230402. EDN: TRTHEI.
  \bibitem{6-lis-1}
   \Aue{Tarasenko, F.\,P.} 2010. \textit{Pri\-klad\-noy sis\-tem\-nyy ana\-liz} 
[Applied systems analysis]. Moscow: KNORUS. 224~p.
  \bibitem{7-lis-1}
   \Aue{Larichev, O.\,I.} 2006. \textit{Ver\-bal'\-nyy ana\-liz re\-she\-niy} [Verbal analysis of decisions]. 
Moscow: Nauka. 181~p.
  \bibitem{8-lis-1}
   \Aue{Akoff, R.\,L.} 1999. \textit{Ackoff's best: His classic writings on management}. New 
York, NY: Wiley. 368~p.
  \bibitem{9-lis-1}
   \Aue{Kolesnikov, A.\,V., S.\,V.~Listopad, S.\,B.~Rumovskaya, and V.\,I.~Danishevskiy.} 
2016. Ne\-for\-mal'\-naya ak\-sio\-ma\-ti\-che\-skaya teo\-riya ro\-le\-vykh vi\-zu\-al'\-nykh mo\-de\-ley [Informal axiomatic 
theory of the role visual models]. \textit{Informatika i~ee Primeneniya~--- Inform. Appl.} 
10(4):114--120. doi: 10.14357/19922264160412. EDN: XGSIVN.
  \bibitem{10-lis-1}
   \Aue{Cherusheva, T.\,V.} 2014. \textit{Pro\-ek\-ti\-ro\-va\-nie pro\-gram\-mno\-go obes\-pe\-che\-niya} 
[Software design]. Penza: PGU. 172~p.
  \bibitem{11-lis-1}
   \Aue{Boissier, O., R.\,H.~Bordini, J.~Hubnerand, and A.~Ricci}. 2020. \textit{Multi-agent 
oriented programming: Programming multi-agent systems using JaCaMo}. Intelligent robotics and 
autonomous agents ser. Cambridge: The MIT Press. 264~p.
  \bibitem{12-lis-1}
   \Aue{Smirnov, S.\,S., and V.\,A.~Smol'yaninova}. 2009. \textit{Vve\-de\-nie v~raz\-ra\-bot\-ku 
mno\-go\-agent\-nykh sis\-tem v~sre\-de Jason. Osno\-vy pro\-gram\-mi\-ro\-va\-niya na yazy\-ke AgentSpeak} 
[Introduction to the development of multiagent systems in the Jason environment. Fundamentals of 
programming in the AgentSpeak language]. Moscow: MIREA. 136~p.
  \bibitem{13-lis-1}
   \Aue{Kaner, С., J.~Falk, and H.\,Q.~Nguyen}. 1999. \textit{Testing computer software}. 
International Thomson Computer Press. 496~p.
  \bibitem{14-lis-1}
   \Aue{Romankova, T.\,L.} 2014. Tes\-ti\-ro\-va\-nie pro\-gram\-mno\-go obes\-pe\-che\-niya [Software testing]. 
Available at: {\sf https://}\linebreak\vspace*{-12pt}

\columnbreak

\noindent
 {\sf elib.gstu.by/bitstream/handle/220612/9860/416.pdf} (accessed January~16, 
2024).
   
  \end{thebibliography}

 }
 }

\end{multicols}

\vspace*{-6pt}

\hfill{\small\textit{Received November 25, 2023}} 

%\vspace*{-18pt}
     
     \Contrl
     
 %    \vspace*{-3pt}
   
   \noindent
   \textbf{Listopad Sergey V.} (b.\ 1984)~--- Candidate of Science (PhD) in technology, senior 
scientist, Federal Research Center ``Computer Science and Control'' of the Russian Academy of 
Sciences, 44-2~Vavilov Str., Moscow 119133, Russian Federation;  
\mbox{ser-list-post@yandex.ru}
   
    
\label{end\stat}

\renewcommand{\bibname}{\protect\rm Литература}  %12






\def\stat{leshinsk}

\def\tit{АНАЛИЗ ПОДХОДОВ К ОПРЕДЕЛЕНИЮ НЕЧЕТКОЙ~РЕЗОЛЬВЕНТЫ}

\def\titkol{Анализ подходов к~определению нечеткой резольвенты}

\def\aut{Т.\,М.~Леденева$^1$, М.\,В.~Лещинская$^2$}

\def\autkol{Т.\,М.~Леденева, М.\,В.~Лещинская}

\titel{\tit}{\aut}{\autkol}{\titkol}

\index{Леденева Т.\,М.}
\index{Лещинская М.\,В.}
\index{Ledeneva T.\,M.}
\index{Leshchinskaya M.\,V.}


%{\renewcommand{\thefootnote}{\fnsymbol{footnote}} \footnotetext[1]
%{Финансовое обеспечение исследований осуществлялось из 
%(Институт прикладных математических исследований КарНЦ РАН).}}


\renewcommand{\thefootnote}{\arabic{footnote}}
\footnotetext[1]{Воронежский государственный университет, \mbox{ledeneva-tm@yandex.ru}}
\footnotetext[2]{Воронежский государственный университет, \mbox{maria-leshchinskaya@mail.ru}}

%\vspace*{-12pt}

  
  

   
  \Abst{Представлены результаты исследования, касающиеся различных определений 
резольвенты в~нечеткой логике. Определены условия, при выполнении которых резольвента 
Ли становится значимым логическим следствием в~случае классического определения 
нечетких логических связок. Показано, что при использовании для их формализации 
треугольных норм и~конорм логически значимую резольвенту Ли получить невозможно. 
Однако если треугольная конорма задается как операция максимума, то для произвольной 
треугольной нормы резольвента Ли существует. Определены условия, при выполнении 
которых для классических операций минимума и~максимума резольвента Мукайдоно 
становится значимым логическим следствием. При использовании треугольных норм 
и~конорм, отличных от классических, требуется дополнительное исследование. Приведен 
иллюстративный пример, демонстрирующий процесс построения резольвенты Мукайдоно.}
  
  \KW{резольвента; метод резолюций; треугольные нормы и~конормы}
  
  \DOI{10.14357/19922264240113}{MIDEHY}
  
%\vspace*{-6pt}


\vskip 10pt plus 9pt minus 6pt

\thispagestyle{headings}

\begin{multicols}{2}

\label{st\stat}

\section{Введение}

  Принцип резолюций, впервые предложенный в~[1], составляет основу 
методов машинной логики, связанных с~автоматическим доказательством 
теорем~[2, 3], и~применяется в~интерактивных системах различного назначения. 
Актуальной остается задача модификации существующих логических 
инструментов в~соответствии с~тенденциями развития области представления 
и~обработки знаний. Неопределенность аргументации и~выводов, характерная для 
неформальных рассуждений, значительно ограничивает применимость 
классического метода резолюций и~обусловливает использование нечеткой 
логики~[4]. Впервые понятие нечеткой резольвенты было предложено Ли в~[5], 
а~затем развито в~работах Мукайдоно~[6] и~других исследователей. Можно 
выделить несколько направлений в~развитии нечеткого метода резолюций. 
В~[7] предложен подход, учитывающий неопределенные высказывания; в~[8] 
введено понятие нечеткой гиперрезолюции и~доказано свойство полноты;  
[9--13] посвящены методу резолюций в~различных логиках; в~[14, 15] 
предложен обобщенный принцип резолюции на основе обратных 
приближенных рассуждений (inverse approximate reasoning), что позволяет 
использовать для построения нечеткой резольвенты правило вывода modus 
ponens. Как показывают исследования, основная проблема нечеткого метода 
резолюций~--- получение условий, при которых резольвента приводит 
к~значимому логическому заключению. Цель статьи заключается в~анализе 
определений нечеткой резольвенты и~исследовании возможности 
использования для ее по\-стро\-ения обобщений нечетких логических операций~--- 
треугольных норм и~конорм.
  

\section{Теоретическая основа исследования}

  Введем основные понятия, базируясь, например, на~[16]. Переменную~$x$ 
или ее отрицание $\overline{x}$ будем называть \textit{литерой}. 
\textit{Дизъюнкт}, или \textit{элементарная дизъюнкция},~--- это совокупность 
различных литер, связанных символом дизъюнкции. \textit{Пустой 
дизъюнкт}~--- это тождественно ложный дизъюнкт, который будем обозначать 
$\sqcup$. Резолюция представляет собой дедуктивное правило вывода, которое 
основано на \textit{дизъюнктивном силлогизме} $(A\vee B, \overline{A})/B$.
  
  Пусть $D_1=D_1^\prime \vee p$  и~$D_2\hm= D_2^\prime \vee \overline{p}$~--- 
дизъюнкты, содержащие контрарные литеры~$p$ и~$\overline{p}$, 
причем~$D_1^\prime$ и~$D_2^\prime$ данных литер не содержат. 
\textit{Резольвентой} дизъюнктов (по переменной~$p$) $D_1$ и~$D_2$ 
называется дизъюнкт вида $\mathrm{res}\,(D_1, D_2) \hm= D_1^\prime \vee D_2^\prime$. 
Если дизъюнкты не содержат контрарных литер, то резольвент у них не 
существует. Пустой дизъюнкт определяется следующим образом: $\mathrm{res}\,(p,\neg p 
)\hm=\Box$.
  
  Пусть $C=\{D_1, \ldots , D_m\}$~--- множество дизъюнктов. 
\textit{Резолютивным выводом} из~$C$ называется такая конечная 
последовательность дизъюнктов $\varphi_1, \ldots, \varphi_n$, в~которой для 
каждого $\varphi_i$ $\left(i\hm= \overline{1,n}\right)$ выполняется одно из 
условий: (1)~$\varphi_i\hm\in C$; (2)~существуют $j,k\hm<i$ такие, что 
$\varphi_i\hm= \mathrm{res}\, (\varphi_j, \varphi_k)$. Дизъюнкт~$D$ считается 
\textit{резолютивно выводимым} из множества~$C$, если существует 
резолютивный вывод из~$C$, последней формулой которого является~$D$. 
Согласно теореме о полноте резолюций, множество~$C$ противоречиво в~том 
и~только в~том случае, когда существует резолютивный вывод из~$C$, 
заканчивающийся пустым дизъюнктом~$\Box$.
  
  Метод резолюций используется для решения следующих задач, которые 
сводятся к~доказательству противоречивости специальным образом 
построенного множества~$C$: проверка правильности рассуждений; 
доказательство выводимости формулы из множества гипотез; доказательство 
того, что формулу можно считать теоремой; доказательство противоречивости 
заданного множества формул.
  
  Чтобы ввести нечеткую резольвенту, рас\-смот\-рим необходимые понятия из 
нечеткой логики~[4], которую можно определить как алгебру $\langle  [0;1], 
\wedge,\vee,\neg\rangle$, при этом для формализации логических связок~$\vee$ 
и~$\wedge$ используются различные операции. Наиболее распространенный 
подход предполагает, что~$\vee$ реализуется как $\max$, а~$\wedge$~--- как 
$\min$ при использовании стандартного отрицания. Заметим, что для операций 
$\max$ и~$\min$, называемых классическими, не выполняются законы 
комплементарности, поэтому в~нечеткой логике высказывание, в~котором~$a$ 
и~$\overline{a}$ встречаются одновременно, оказывается содержательным 
в~отличие от классической логики. В~общем случае для 
формализации~$\wedge$ и~$\vee$ используются треугольные нормы~$T$ 
и~двойственные им (в~смысле законов де Моргана) треугольные конормы~$S$. 
Различные семейства треугольных норм\linebreak и~конорм представлены в~[17]. Новые 
параметрические семейства треугольных норм и~конорм получены в~[18, 19]. 
Следует отметить, что для классической пары не выполняется свойство 
\mbox{комплементарности}, а~для остальных помимо комплементарности не 
выполняются свойства идемпотентности и~дистрибутивности, что 
обусловливает необходимость проявления аккуратности при выполнении 
всевозможных преобразований логических формул.
  
  Каждому элементарному высказыванию   поставим в~соответствие оценку 
его истинности $t(x)\hm\in [0;1]$, при этом чем ближе значение $t(x)$ к~0, тем 
в~большей степени высказывание считается \textit{ложным}; чем ближе $t(x)$ 
к~1, тем в~большей степени высказывание считается \textit{истинным}; если 
$t(x)\hm=0{,}5$, то высказывание считается \textit{неопределенным} 
(соответствует максимальной неопределенности). В~нечеткой логике 
высказывание~$x$ принимается, если степень его истинности больше степени 
истинности высказывания \textit{не} $x (\overline{x})$, т.\,е.\ $t(x)\hm> 1\hm- 
t(x)$, а~следовательно, $t(x)\hm>0{,}5$.
  
  \textit{Степенью доверия} к~$t(x)$ называется величина $c_x\hm= 2(t(x)\hm-
0{,}5)\hm\in [-1;1]$, а~ее модуль $\vert c_x\vert$ характеризует уровень 
определенности высказывания~$x$: если $t(x)\hm=0{,}5$, то $c_x\hm= 
0$~\cite{6-les}.
  
  Пусть $F$~--- некоторая формула. При заданной\linebreak интерпретации~$I$ 
значение истинности каждой переменной определяется однозначно, поэтому, 
выбрав подходящую формализацию логических связок, можно получить 
значение истинности \mbox{формулы}~$F$. Формула~$F$ называется 
\textit{выполнимой} в~интерпретации~$I$, если $t(F)\hm> 0{,}5$, 
и~\textit{невыполнимой}, если $t(F)\hm< 0{,}5$. Если $t(F)\hm= 0{,}5$, то 
интерпретация~$I$ одновременно удовлетворяет и~опровергает~$F$.
  
  Если $F=\{F_1,\ldots , F_m\}$~--- множество формул, то $t_I(F)\hm= 
t_I(F_1\wedge \cdots\wedge F_m)$.

\vspace*{-6pt}

\section{Нечеткая резольвента и~ее~свойства}

  Будем считать, что
  \begin{align*}
  t(a\vee b) &=\max \left\{ t(a),t(b)\right\};\\
     t(a\wedge b) &=\min \{t(a),t(b)\};\\
  t\left( \overline{a}\right) &=1-t(a)\,.
  \end{align*}
  
  Пусть $D_1=p\vee D_1^\prime$, $D_2\hm= \overline{p}\vee D_2^\prime$, 
$\mathrm{res}\,(D_1,D_2) \hm= D_1^\prime \vee D_2^\prime$.
 В~[5] нечеткая резольвента 
определяется в~виде 
$$
t\left(\mathrm{res}\,(D_1,D_2) \right)= t\left( D_1^\prime \vee 
D_2^\prime\right) = \max \left\{ t(D_1^\prime), t(D_2^\prime)\right\}
$$ 
и~называется \textit{резольвентой Ли} (L-ре\-золь\-вен\-той).
  
  В классической логике действует следующий постулат~[20]: истинность 
множества формул остается неизменной, если к~ним добавить их резольвенту, 
т.\,е.\ $t(D_1\wedge D_2)=t(D_1\wedge D_2\wedge \mathrm{res}\,(D_1, D_2))$, и~поэтому
  \begin{equation}
  d\left( D_1\wedge D_2\right) \leq t\left( \mathrm{res} \left( D_1, D_2\right)\right)\,.
  \end{equation}
  
  Резольвента называется \textit{логически значимой} или \textit{значимым 
логическим следствием}, если она удовлетворяет~(1). Найдем условия, при 
которых выполняется соотношение~(1), а~следовательно, резольвента 
становится логически значимой.
  
  Положим $t(D_1\wedge D_2) \hm= \min \{t(D_1), t(D_2)\} \hm=a$. Без 
ограничения общности будем считать, что
  \begin{align}
  t(D_1) &= t(p\vee D_1^\prime)=\max \left\{ t(p),t(D_1^\prime)\right\}=a\,;
  \label{e2-les}\\
  t(D_2) &= t\left( \overline{p} \vee D_2^\prime\right) =\max \left\{ t(\overline{p}), 
t(D_2^\prime) \right\} ={}\notag\\
&\hspace*{15mm}{}=\max \left\{ 1-t(p), t(D_2^\prime\right\}>a\,.
  \label{e3-les}
  \end{align}
  
  В соответствии с~введенными предположениями возможны ситуации:
  \begin{itemize}
\item  для~(2) имеем:
\begin{itemize}
\item[(а)] $t(p)=a$, $t(D_1^\prime) <a$; 
\item[(б)] $t(p)<a$,  $t(D_1^\prime)\hm=a$;
\end{itemize}
  \item
  для~(3) имеем: 
  \begin{itemize}
  \item[(в)] $\left[    \begin{matrix} 
  (\mathrm{в}1)~a<t(D_2^\prime)<1-t(p);\\
  (\mathrm{в}2)~t(D_2^\prime) <a<1-t(p);
  \end{matrix}
  \right]$
\item[(г)] $\left[   \begin{matrix}
  (\mathrm{г}1)~a<1- t(p)<t/(D_2^\prime);\\
 (\mathrm{г}2)~1-t(p)<a<t(D_2^\prime).
  \end{matrix}
  \right.$
  \end{itemize}
  \end{itemize}
  
  Анализ комбинаций приведенных ситуаций поз\-во\-лил сформулировать 
следующие утверж\-де\-ния.
  
  \smallskip
  
  \noindent
  \textbf{Утверждение~1.} \textit{Пусть $t(p)\hm< 0{,}5$, тогда 
неравенство}~(1) \textit{выполняется, а~следовательно, резольвента 
становится значимым логическим следствием, если имеет место одна из 
следующих ситуаций}: 
\begin{enumerate}[(1)]
\item \textit{для одного из дизъюнктов~$D_1^\prime$ 
или~$D_2^\prime$ степень истинности больше $t(p)$, а~степень истинности 
другого принадлежит промежутку} $\left( \min \left\{ t(D_1^\prime), 
t(D_2^\prime)\right\}, 1\hm- t(p)\right)$; 
\item \textit{для одного из 
дизъюнктов~$D_1^\prime$ или~$D_2^\prime$ степень истинности меньше 
$t(p)$, а~степень истинности другого принадлежит промежутку} 
  $\left( t(p), 1\hm- t(p)\right)$;
  \item $t(D_1^\prime) \hm< t(D_2^\prime)$ 
  \textit{и}~$t(D_2^\prime) \hm>1\hm- t(p)$; 
  \item $t(D_1^\prime)\hm< t(D_2^\prime)$ \textit{и}~$t(D_1^\prime) \hm\in \left( 1\hm- t(p), t(D_2^\prime)\right)$.
\end{enumerate}
  
  %\smallskip
  
  \noindent
  \textbf{Утверждение~2.} \textit{Если $t(p)\hm> 0{,}5$, $t(D_1^\prime) \hm< 
t(D_1^\prime)$ и~$t(D_2^\prime) \hm> t(p)$, то неравенство}~(1) 
\textit{выполняется}.
  
  \smallskip
  
  \noindent
  \textbf{Утверждение~3.} \textit{Если для переменной $p$ и~каждого из 
дизъюнктов~$D_1^\prime$ и~$D_2^\prime$ степень истинности меньше~0,5, 
то неравенство}~(1) \textit{не выполняется}.
  
  \smallskip
  
  \noindent
  \textbf{Утверждение~4.} \textit{L-ре\-золь\-вен\-та становится выполнимой 
формулой, если}
  \begin{enumerate}[(1)]
  \item $t(p)<0{,}5$ \textit{и} $\max \left\{ t(D_1), t(D_2)\right\} \in [0{,}5;1]$;
  \item $t(p)>0{,}5$ \textit{и} $\max \left\{ t(D_1), t(D_2)\right\} \in [t(p);1]$.
  \end{enumerate}
  
  \smallskip
  
  Таким образом, L-ре\-золь\-вен\-та становится значимым логическим 
следствием, если степень ее истинности или степень истинности по крайней 
мере одного из дизъюнктов больше~0,5. На основе приведенных рассуждений 
был построен алгоритм для нахождения логически значимой резольвенты.
  
  Теперь попытаемся обобщить результаты на другие типы логических связок. 
Рассмотрим случай,\linebreak когда конъюнкция формализуется треугольной 
нормой~$T$, дизъюнкция~--- треугольной конормой~$S$, так что пара $(T,S)$ 
образует пару двойственных операций относительно стандартного отрицания 
$n(x)\hm=1\hm-x$. Исследуем условие~(1) для произвольной пары $(T,S)$, 
учитывая, что для нее не выполняются законы дистрибутивности, 
комплементарности и~идемпотентности.
  %
  В этом случае
  \begin{align*}
  t(D_1)&=t\left(D_1^\prime\vee p\right) =S\left\{ t(D_1^\prime), t(p)\right\};\\
  t(D_2)&=t\left(D_2^\prime\vee \overline{p}\right) =S\left\{ t(D_2^\prime), 
t(\overline{p})\right\};\\
  t\left(\overline{D}_1\right) &=t\left( \overline{D_1^\prime\vee p}\right) =T\left( 
t(\overline{D}_1^\prime, t\left(\overline{p}\right)\right);\\
  t\left( \overline{D}_2\right) &=t\left( \overline{D_2^\prime \vee 
\overline{p}}\right) =T\left( t\left(\overline{D}_2^\prime\right),  t(p)\right);\\
  t\left( \mathrm{res}\, (D_1, D_2)\right) &= t(D_1^\prime\vee D_2^\prime) =S\left( 
t(D_1^\prime), t(D_2^\prime)\right).
  \end{align*}
  %
  Тогда, учитывая, что для любых~$T$ и~$S$ имеют мес\-то неравенства 
$T(x,y)\hm\leq \min\{x,y\}$ и~$\max \{ x,y\} \hm\leq S(x,y)$, получим

\vspace*{-4pt}

\noindent
  \begin{multline*}
  t(D_1\wedge D_2) =T(t(D_1), t(D_2))={}\\[1pt]
  {}= 1-S(1-t(D_1), 1-t(D_2)) ={}\\[1pt]
  {}=
  1- S\left(t(\overline{D}_1), t(\overline{D}_2)\right)={}\\[1pt]
  {}= 1-S\left(T\left(t\left(\overline{D}_1^\prime\right), t(\overline{p})\right), T\left( 
t\left(\overline{D}_2^\prime\right), t(p)\right)\right) ={}\\[1pt]
{}=
  1-S\left( 1-S\left( 
  t\left(D_1^\prime\right), t(p)\right), 1-S\left( t\left(D_2^\prime\right), 
t(\overline{p})\right)\right)={}\\[1pt]
  {}=
  T\left( S\left(t\left(D_1^\prime\right), t(p)\right), S\left(t\left(D_2^\prime\right), t\left(\overline{p}\right)\right)\right) \leq{}\\[1pt]
  {}\leq
  \min\left( S(t(D_1^\prime), t(p)), S(t(D_2^\prime), t(\overline{p}))\right)\leq{}\\[1pt]
  {}\leq
  \max (S(t(D_1^\prime), t(p)), S(t(D_2^\prime), t(\overline{p})))\leq{}\\[1pt]
  {}\leq
  S(S(t(D_1^\prime), t(p)), S(t(D_2^\prime), t(\overline{p})))={}\\[1pt]
  {}=
  S \left(S\left(t(D_1^\prime), t(D_2^\prime)\right), S\left( t(p), t(\overline{p}) 
\right)\right).
  \end{multline*}
  
  \vspace*{-4pt}
  
  Заметим, что в~качестве результата данного выражения можно получить 
резольвенту $S\left(t(D_1^\prime), t(D_2^\prime)\right)\hm= t\left(\mathrm{res}\,(D_1, 
D_2)\right)$, если $S\left(t(p), t(\overline{p})\right)\hm=0$, так как для 
треугольных конорм $S(x,0)\hm=x$ для всех $x\hm\in [0,1]$. Согласно 
определению, соотношение $S\left(t(p), t(\overline{p})\right)\hm=0$ 
выполняется, если одновременно $t(p)\hm= 0$ и~$t(\overline{p})\hm=0$, что 
невозможно. Если $S\left( t(p), t(\overline{p})\right)\hm= S\left(t(D_1^\prime), 
t(D_2^\prime)\right)$, то для идемпотентных конорм неравенство~(1) будет 
выполнено. Но единственной идемпотентной парой остается пара $(\min, 
\max)$ ~[17]. Таким образом, доказано

  \smallskip
  
  \noindent
  \textbf{Утверждение~5.} \textit{Для произвольной пары $(T,S)$ 
двойственных треугольных операций, отличных от классических $\min$ 
и~$\max$, L-ре\-золь\-вен\-та не существует}.
  
  \smallskip
  
  Рассмотрим комбинацию произвольной треугольной нормы~$T$ 
и~$S\hm=\max$:
  \begin{align*}
  t\left( \mathrm{res}\left (D_1, D_2\right)\right) &=t\left( D_1^\prime \vee D_2^\prime\right) =\max \left( 
t(D_1^\prime), t(D_2^\prime)\right);\\
  t(D_1) &=t(D_1^\prime\vee p) =\max \left\{ t(D_1^\prime), t(p)\right\};\\
  t(D_2)&= t(D_2^\prime\vee \overline{p})=\max \left\{ t(D_2^\prime), 
t(\overline{p})\right\}.
  \end{align*}
  
  Найдем 
  \begin{multline*}
  t(D_1\wedge D_2) =T\left( t(D_1), t(D_2)\right) \leq {}\\
  \hspace*{-10mm}{}\leq \min\left\{ t(D_1), t(D_2)\right\}={}\hspace*{10mm}
  \end{multline*}

\noindent
  \begin{multline*}
  {}=
  \min\left\{ \max \left\{ t(D_1^\prime), t(p)\right\}, \max \left\{ t(D_2^\prime), 
t\left(\overline{p}\right)\right\}\right\}={}\\[1pt]
  {}=
  \max \left\{ \min\left\{ t(D_1^\prime), t(D_2^\prime)\right\}, 
  \min\left\{ t\left(D_1^\prime\right), t\left(\overline{p}\right)\right\},\right.\\[1pt]
 \left. \min\left\{ t(D_2^\prime), t(p)\right\},
  \min\left\{ t(p), t(\overline{p})\right\}\right\}\leq{}\\[1pt]
  {}\leq
  \max \left\{ \max \left\{ t(D_1^\prime), t(D_2^\prime)\right\}, \min\left\{ 
t(D_1^\prime), t(\overline{p})\right\},\right.\\[1pt]
\left.
  \min\left\{ t(D_2^\prime), t(p)\right\}, \min\left\{ t(p), 
t(\overline{p})\right\}\right\}.
  \end{multline*}
  
%\vspace*{-3pt}
  
  Заметим, что результатом последнего выражения будет 
  $$
  \max \left\{ t\left(D_1^\prime\right), t\left(D_2^\prime\right)\right\}= t\left(\mathrm{res}\left( D_1, D_2\right)\right),
  $$
  
  \vspace*{-3pt}
  
  \noindent
   а~следовательно, 
неравенство~(1) будет выполнено, если одновременно будут иметь место 
следующие неравенства:
  \begin{equation*}
  \left\{
  \begin{array}{l}
  \max\left\{ t(D_1^\prime), t(D_2^\prime)\right\} \geq \min \left\{ t(D_1^\prime), 
t(\overline{p})\right\};\\[6pt]
  \max\left\{ t(D_1^\prime), t(D_2^\prime)\right\} \geq \min \left\{ t(D_2^\prime), 
t(p)\right\};\\[6pt]
  \max\left\{ t(D_1^\prime), t(D_2^\prime)\right\} \geq \min \left\{ t(p),  
t(\overline{p})\right\}.
  \end{array}
  \right.
  \end{equation*}
  
  С учетом того что существуют всего 24~случая различных упорядочений 
значений $t(D_1^\prime)$, $t(D_2^\prime)$, $t(p)$ и~$t(\overline{p})$, установлено, что 
одновременно данные неравенства не выполняются, если 
$$
\max \{ t\left(D_1^\prime\right), t\left(D_2^\prime\right)\} < \min \{ t(p), t\left(\overline{p}\right)\}.
$$
 Так как по 
закону де Моргана
$$
\min \left\{ t(p), t\left(\overline{p}\right)\right\} = 1- \max \left\{t(p),  t\left(\overline{p}\right)\right\},
$$
 то предыдущее неравенство равносильно неравенству 
 $$
 \max  \left\{ t\left(D_1^\prime\right), t\left(D_2^\prime\right)\right\} + \max \left\{ t(p), t\left(\overline{p}\right)\right\}<1\,. 
 $$
Таким образом, доказано следующее
  
  \smallskip
  
  \noindent
  \textbf{Утверждение~6.} \textit{Пусть для формализации конъюнкции 
используется норма~$T$, а~дизъюнкции~--- $\max$. Если $\max \{ 
t(D_1^\prime), t(D_2^\prime)\}\hm+ \max \{ t(p), t(\overline{p})\}\hm<1$, то 
неравенство~$(1)$ не выполняется. В~остальных случаях неравенство~$(1)$ 
выполняется, при этом резольвента становится значимым логическим 
следствием, если} $\max \{ t(D_1^\prime), t(D_2^\prime)\} \hm> 0{,}5$.
  
  \smallskip
  
  Заметим, что L-ре\-золь\-вен\-та может быть или не быть значимым 
логическим следствием в~за\-ви\-си\-мости от степени истинности ее слагаемых 
компонент и~переменной~$p$. В~[6] Мукайдоно получил обобщение  
L-ре\-золь\-вен\-ты, сопряженное со сте\-пенью доверия.
  
  Пусть $D_1=p\vee D_1^\prime$ и~$D_2\hm= \overline{p}\vee D_2^\prime$~--- 
дизъюнкты. \textit{Нечеткая резольвента Му\-кай\-до\-но~---  
\mbox{M-ре}\-золь\-вен\-та} дизъюнктов~$D_1$ и~$D_2$, обозначаемая $\mathrm{res}\,(D_1, 
D_2)_{c_p}$, где $c_p\hm= 2\left(\max \{ t(p), t(\overline{p})\}\hm- 
0{,}5\right)$~--- степень доверия к~переменной~$p$, вычисляется по формуле: 

\columnbreak

\noindent
\begin{multline*}
\mathrm{res}\left (D_1,D_2\right)_{c_p} = \mathrm{res}\left(D_1, D_2\right)\vee \left(p\wedge \overline{p}\right),
\\
\mathrm{res}\,(D_1,D_2)\hm= D_1^\prime \vee D_2^\prime.
\end{multline*}


  
  Множество дизъюнктов~$C$ считается невыполнимым тогда и~только тогда, 
когда существует вывод пустого дизъюнкта~$\Box$ с~ненулевой степенью 
доверия~[6].
  
  Используя пару $(T,S)$ двойственных операций, степень истинности  
М-ре\-золь\-вен\-ты запишем в~\mbox{виде}: 
\begin{multline*}
t\left(\mathrm{res}\left(D_1,D_2\right)_{c_p}\right) ={}\\
{}= S\left\{ t\left( \mathrm{res}\left(D_1, D_2\right)\right), 
T\left\{ t(p), t\left(\overline{p}\right)\right\}\right\}.
\end{multline*}
 Учитывая рассуждения при 
доказательстве утверж\-де\-ния~5, используя закон де Моргана и~свойство 
мо\-но\-тон\-ности~$S$, в~предположении, что $1\hm- T\left( t(\overline{p}), 
t(p)\right) \hm\leq T\left( t(\overline{p}), t(p)\right)$, получим
  \begin{multline*}
  t(D_1\wedge D_2) \leq S\left( S\left(t(D_1^\prime), t(D_2^\prime)\right), S(t(p), 
t(\overline{p}))\right)={}\\
  {}=
  S\left( S(t(D_1^\prime), t(D_2^\prime), 1-T(t(\overline{p}), t(p))\right)\leq{}\\
  {}\leq
  S\left( S(t(D_1^\prime), t(D_2^\prime)), T(t(p), t(\overline{p}))\right)={}\\
  {}=
  S\left( t(\mathrm{res}\,(D_1, D_2)), T(t(p), t(\overline{p}))\right)={}\\
  {}=
  t\left( \mathrm{res}\,(D_1, D_2)_{c_p}\right),
  \end{multline*}
а следовательно, неравенство~(1) выполняется, при этом степень истинности 
переменной~$p$ и~треугольная норма~$T$ должны быть такими, чтобы\linebreak $T\left( 
t(\overline{p}), t(p)\right) \hm\geq 0{,}5$. В~данном случае необходимо 
исследование семейств треугольных норм, поз\-во\-ля\-ющих обеспечить 
выполнение данного неравенства, и~в~общем случае ка\-кой-то вывод \mbox{сделать} 
невозможно. При $T\hm=\min$ неравенство $\min \left\{ t(\overline{p}), 
t(p)\right\} \hm\geq 0{,}5$ выполняется только как равенство при 
$t(\overline{p})\hm= t(p)\hm=0{,}5$, что дает воз\-мож\-ность работать 
с~неопределенными высказываниями. Кроме того, для поиска подходящей 
\mbox{треугольной} нормы можно рассматривать семейства, для которых 
<<действует>> отрицание, не являющееся инволюцией.

  Пусть $T=\min$ и~$S\hm=\max$, тогда резольвента имеет вид:
  \begin{multline*}
  t\left( \mathrm{res}\left(D_1, D_2\right)_{c_p}\right) ={}\\
  {}= \max \left\{ \max \left\{ t(D_1^\prime), 
t(D_2^\prime)\right\}, \min \left\{ t(p), t(\overline{p})\right\}\right\}.
\end{multline*}

\begin{figure*}[b] %fig1
\begin{center}
{\small \begin{tabular}{|p{118mm}|}
  \hline
   \textbf{Input:} $C$~--- исходное множество дизъюнктов\\
   \textbf{Input:} $k=0$, $R_k=\varnothing$ (текущее множество резольвент)\\
   \textbf{Output:} Результат работы алгоритма: $\{$$C$ is inconsistent, $C$ is 
unknown$\}$\\
{\footnotesize // упорядочить литеры, образующие контрарные пары, по убыванию степени 
доверия~$c_p$}\\
  {\sf calculateConfidenceLevel} ($c_p$); {\sf sortConfidenceLevels} ($c_p$);\\
  {\footnotesize // вычисление резольвент}\\
  \textbf{while} ($\exists\,D_i\in C$ and $\exists\,D_j\in C$) \textbf{do}\\
  {\footnotesize // если существует значимая резольвента, то добавить ее в~$R_{k+1}$}\\
        \hspace*{5mm}\textbf{if} ($\exists\,\mathrm{res}\,(D_i, D_j)$ is significant) \textbf{do}\\
        \hspace*{5mm}$\mathsf{searchAndChoose} (D_i, D_j, c_p)$; $R_{k+1}=R_k\cup 
\left\{ \mathrm{res}\,(D_i, D_j)\right\}$\\
        \hspace*{5mm}{\footnotesize // проверка: существует ли в~$R_{k+1}$ пустой 
дизъюнкт~$\Box$}\\
        \hspace*{5mm}\textbf{if} ($\Box\in R_{k+1}$) \textbf{then} $C$ is inconsistent;  
\textbf{end program};\\
        \hspace*{5mm}{\footnotesize // если в~$R_{k+1}$ нет $\Box$, то изменить множество~$C$}\\
        \hspace*{5mm}\textbf{else} $C=C\backslash D_i\cup C\backslash D_j \cup\,\mathrm{res}\,(D_i,D_j)$; $k=k+1$;\\
        \hspace*{5mm}\textbf{end if}\\
  \textbf{end while}\\
  {\footnotesize // если в~$C$ нет дизъюнктов для получения резольвенты, то противоречивость $C$ 
неизвестна}\\
   $C$ is unknown;\\
  \textbf{end program}\\
  \hline
  \end{tabular}
  }
  \end{center}
  \Caption{Алгоритм построения M-резольвенты: $C$~--- множество 
дизъюнктов, невыполнимость (противоречивость) которого исследуется}
  \end{figure*}
  
  Рассмотрим
  \begin{multline*}
  t(D_1,D_2) =\min \left\{ t(D_1), t(D_2)\right\} ={}\\
  {}= \min\left\{ \max \left\{
  t(D_1^\prime), t(p)\right\}, \max \left\{ t(D_2^\prime), 
t(\overline{p})\right\}\right\}={}\\
 {}=
  \max\left\{ \min \left\{ t(D_1^\prime), t(D_2^\prime)\right\}, \min\left\{ 
t(D_1^\prime), t(\overline{p})\right\},\right.\\
\left.
  \min\left\{ t(D_2^\prime), t(p)\right\}, \min \left\{ t(p), 
t(\overline{p})\right\}\right\}\leq{}\\
 {}\leq
    \max\left\{ \max\left\{ t(D_1^\prime), t(D_2^\prime)\right\}, \min\left\{ 
t(D_1^\prime), t(\overline{p})\right\},\right.\\
\left.
  \min\left\{ t(D_2^\prime), t(p)\right\}, \min \left\{ t(p), 
t(\overline{p})\right\}\right\}={}
\end{multline*}

\noindent  
\begin{multline*}
    =\!
  \max\!\Bigg\{\!
  \underbrace{\max\!\left\{ \max\left\{ t(D_1^\prime), t(D_2^\prime)\right\}\!, 
\min\left\{ t(p), t(\overline{p})\right\}\!\right\}}_{t(\mathrm{res}\,(D_1,D_2)_{c_p})}, \\[2pt]
\max\left\{ \min \left\{ t(D_1^\prime), t(\overline{p})\right\}, \min\left\{ 
t(D_2^\prime), t(p)\right\}\right\}\Bigg\}.
  \end{multline*}
  
  Заметим, что
  \begin{multline*}
\max\left\{\min \left\{ t\left(D_1^\prime\right), t\left(\overline{p}\right)\right\}, \min \left\{ t\left( D_2^\prime\right), t(p)\right\}\right\}={}\\[2pt]
{}=  
  \min\Bigg\{ 
  \underbrace{\max\left\{ t(D_1^\prime), t(D_2^\prime)\right\}}_{t(\mathrm{res}\,(D_1,D_2))}, 
\underbrace{\max\left\{ t(D_1^\prime), t(p)\right\}}_{t(D_1)},\\[2pt] 
\underbrace{\max\left\{ t(D_2^\prime), t(\overline{p})\right\}}_{t(D_2)}, 
\max\left\{ t(p), t(\overline{p})\right\}\Bigg\}={}\\[2pt]
  {}= \min\Bigg\{
  \underbrace{\min\left\{ t(D_1), t(D_2)\right\}}_{t(D_1\wedge D_2)},\\[2pt]
   \min\left\{ 
t\left(\mathrm{res}\left(D_1, D_2\right)\right), \max\left\{t(p), t(\overline{p})\right\}\right\}
  \Bigg\}.
  \end{multline*}
  
  Таким образом, получено следующее не\-ра\-вен\-ство:
  \begin{multline*}
  t\left(D_1, D_2\right) \leq \max\left\{ t\left( \mathrm{res}\left(D_1, D_2\right)_{c_p}\right)\right.,\\[2pt]
 \min \left\{ t\left(D_1\wedge D_2\right), t\left(\mathrm{res}\left(D_1,D_2\right)\right),\right.\\[2pt]
\left.   \left. \max \left\{ t(p), 
t\left(\overline{p}\right)\right\}\!\right\}\!
\vphantom{\left( \mathrm{res}\left(D_1, D_2\right)_{c_p}\right)}
\right\}.
 \end{multline*}
 
 \noindent
  Здесь $\mathrm{res}\,(D_1, D_2)$~--- \mbox{L-ре}\-золь\-вен\-та. Тогда если \mbox{L-ре}\-золь\-ве\-нта 
логически значима, т.\,е.\ выполняется неравенство $t(D_1\wedge D_2) \hm\leq 
t\left( \mathrm{res}\,(D_1, D_2)\right)$, то
  \begin{multline*}
  t(D_1\wedge D_2) \leq \max \left\{ t\left( \mathrm{res}\left (D_1,D_2\right)_{c_p}\right),\right.\\
\left.  \min \left\{ 
t(D_1\wedge D_2), \max \left\{ t(p), t(\overline{p})\right\}\right\}\right\}.
\end{multline*}
  
  Если $t(D_1\wedge D_2)\hm\leq \max \left\{ t(p), t(\overline{p})\right\}$, то
  $$
  t(D_1\wedge D_2)\leq \max \left\{ t\left( \mathrm{res}\,(D_1, D_2)_{c_p}\right), 
t(D_1\wedge D_2)\right\}
  $$
и неравенство~(1) выполняется.
  
  Если $t(D_1\wedge D_2) \hm> \max \left\{ t(p), t(\overline{p})\right\}$, то
\begin{multline*}
t(D_1\wedge D_2) \leq {}\\
{}\leq
\max\left\{ t\left( \mathrm{res}\left(D_1, D_2\right)_{c_p}\right), \max \left\{ 
t(p), t (\overline{p})\right\}\right\}
\end{multline*}
и неравенство~(1) выполняется, если 
$$
\max \left\{ t(p),t(\overline{p})\right\} 
\leq t\left( \mathrm{res}\left(D_1, D_2\right)_{c_p}\right).
$$
  
  Таким образом, доказано следующее
  
  \begin{figure*}[b] %fig2
\begin{center}
{\small
 \begin{tabular}{|p{124mm}|}
\hline
$C=\left\{ W(x)\vee O(x), Z(y)\vee \neg W(y), Y(A), \neg Z(A), \neg O(z)\vee Y(z)\right\}$\newline
$c_p(Y)=2(0{,}8-0{,}5) =0{,}6$, $c_p(Z)=2(0{,}7- 0{,}5)=0{,}4$\newline
$c_p(O)=2(0{,}6-0{,}5)=0{,}2$, $c_p(W)= 2(0{,}5-0{,}5)=0$\\
\hline
$k=0$, $R_0=\varnothing$; $R_0$ не содержит пустого дизъюнкта; в~$C$ есть унифицируемые 
литералы\\
\hline
$D_i=Y(A)$, $D_j=\neg O(z)\vee \neg Y(z)$ \\
\hline
$\neg O(A)\vee \neg Y(A)=\neg O(A)$, $R_1=\left\{ \neg O(A)\right\}$\\
\hline
$k=1$; $R_1$ не содержит пустого дизъюнкта; в~$C$ есть унифицируемые литералы\newline
$C=\left\{ W(x)\vee O(x), Z(y)\vee \neg W(y), \neg Z(A), \neg O(A)\right\}$\\
\hline
$D_i=Z(y)\vee \neg W(y)$, $D_j=\neg Z(A)$ \\
\hline
$\neg W(A)\vee Z(A) =\neg W(A)$, $R_2=\{ \neg O(A), \neg W(A)\}$ \\
\hline
$k=2$, $R_2$ не содержит пустого дизъюнкта, в~$C$ есть унифицируемые литералы\newline
$C=\left\{ W(x)\vee O(x), \neg O(A), \neg W(A)\right\}$\\
\hline
$D_i=W(x)\vee O(x)$, $D_j=\neg O(A)$, $R_3=\left\{ \neg O(A), \neg W(A), W(A)\right\}$\\
\hline
$k=3$, $R_3$ не содержит пустого дизъюнкта; в~$C$ есть унифицируемые литералы\newline
$C=\left\{ \neg W(A), W(A)\right\}$\\
\hline
$D_i=\neg W(A)$, $D_j=W(A)$, $R_4=\left\{ \neg O(A), \neg W(A), W(A), \Box\right\}$\\
\hline
$R_4$ содержит пустой дизъюнкт~--- исходное множество~$C$ противоречиво.\\
\hline
\end{tabular}
}
\end{center}
\Caption{Работа алгоритма в~интерпретации Мукайдоно}
\end{figure*}


  \smallskip
  
  \noindent
  \textbf{Утверждение~7.} \textit{Если L-ре\-золь\-вен\-та~--- значимое 
логическое следствие и
$$
\max \left\{ t(p), t\left(\overline{p}\right)\right\} \leq t\left( \mathrm{res}
\left(D_1, D_2\right)_{c_p}\right),
$$
 то М-ре\-золь\-вен\-та $t\left( \mathrm{res}\,(D_1,  D_2)_{c_p}\right)$ удовлетворяет неравенству~$(1)$, а~следовательно, также 
становится значимым логическим следствием}.

\section{Нечеткий метод резолюций}
  
  Нечеткий метод резолюций реализуется согласно классической схеме [16] на 
основе определений \mbox{L-ре}\-золь\-вен\-ты или \mbox{M-ре}\-золь\-вен\-ты, при этом 
в~первом случае вычисляются только значимые \mbox{L-ре}\-золь\-вен\-ты. 

При 
нахождении \mbox{M-ре}\-золь\-вен\-ты используются степени доверия литер: чем 
больше~$c_p$, тем построение резольвенты из дизъюнктов, содержащих 
литеру~$p$, происходит раньше. 

Рассмотрим более подробно резолютивный 
вывод на основе \mbox{M-ре}\-золь\-вен\-ты (рис.~1).  Для 
компактного изложения алгоритма введем следующие функции:
  \begin{enumerate}[(1)]
\item функция $\mathsf{calculateConfidenceLevel} (c_p)$ вычисляет степень 
доверия каждой литеры (результат в~$c_p$);
\item функция $\mathsf{sortConfidenceLevels} (c_p)$ сортирует 
последовательность степеней доверия литер в~порядке убывания;
\item функция $\mathsf{searchAndChoose} (D_i, D_j, c_p)$ осуществляет 
поиск дизъюнктов, содержащих литеру с~максимальным значением степени 
доверия~$c_p$. 
\end{enumerate}


  
  Рассмотрим пример. Докажем или опровергнем противоречивость 
следующего множества дизъюнктов:
$  \left\{
  W(x)\vee O(x), Z(y)\vee 
   \neg W(y), Y(A,\right.$\linebreak $\left. \neg Z(A), \neg O(z)\vee \neg 
Y(z)\right\}$,
где $t(W)=0{,}5$; $t(O)\hm= 0{,}4$; $t(Z)\hm= 0{,}3$; $t(Y)\hm= 0{,}8$. 
Результаты работы алгоритма, основанного на вычислении M-ре\-золь\-вен\-ты, 
представлены на рис.~2.


\section{Заключение}

%\vspace*{-6pt}
  
  Метод резолюций прежде всего известен как инструмент для 
автоматического доказательства теорем. В~моделях обработки знаний, 
основанных на правиле дедукции, проблема формулируется в~виде 
совокупности утверж\-де\-ний-ги\-по\-тез и~целевого утверж\-де\-ния~---  
тео\-ре\-мы, справедливость которой следует установить или опровергнуть на 
основе данных гипотез, аксиом и~правил вывода. Данная схема может быть 
положена в~основу различных автоматизированных процедур принятия 
решений в~медицине, экономике, финансовой сфере, судебной практике. 
Метод резолюций используется для моделирования итерационных вычислений 
в~рамках дедуктивного синтеза программ. Теория рассуждений, основанная на 
методе резолюций, стала активно развивающимся направлением 
искусственного\linebreak
 интеллекта. Классический метод резолюций базируется на 
двузначной логике, но зачастую для многих рассуждений характерна 
неопределенность знаний, обусловленная различными факторами. \mbox{Поэтому} 
многие исследователи пытаются обобщить метод резолюций на случай 
неклассических логик и~сделать его таким же эффективным, как в~классической 
логике. 
%
В~данной статье предложены различные обобщения известных 
определений нечетких резольвент с~использованием треугольных норм 
и~конорм, которые формализуют нечеткие\linebreak логические связки \textit{и} 
и~\textit{или}. 
%
Проведенное иссле\-до\-вание показывает, что для того чтобы 
нечеткая резольвента была значимым логическим следствием, необходимо 
выполнение определенных условий, которые уменьшают неопределенность 
и~делают правило нечеткого вывода, опирающееся на резолюционный 
принцип, осмысленным в~разичных интерпретациях.
  
{\small\frenchspacing
 {\baselineskip=11.4pt
 %\addcontentsline{toc}{section}{References}
 \begin{thebibliography}{99}
\bibitem{1-les}
\Au{Robinson J.\,A.} A~machine-oriented logic based on the resolution principle~// J.~ACM, 1965. 
Vol.~12. Iss.~1. P.~23--41. doi: 10.1145/321250.321253.
\bibitem{2-les}
\Au{Newborn M.} Automated theorem proving: Theory and practice.~--- Berlin, Heidelberg: 
Springer-Verlag, 2001. 231~p.
\bibitem{3-les}
\Au{Ламберов Л.\,Д.} Практика компьютерных доказательств и~человеческое понимание: 
эпистемологическая проблематика~// Вестник Пермского университета. Философия. 
Психология. Социология, 2021. №\,1. С.~5--19. doi: 10.17072/2078-7898/2021-1-5-19. 
EDN: \mbox{GTOYOE}.
\bibitem{4-les}
\Au{Новак В., Перфильева~И., Мочкорж~И.} Математические принципы нечеткой  
логики.~--- М.: Физматлит, 2006. 252~с.
\bibitem{5-les}
\Au{Lee R.\,C.\,T.} Fuzzy logic and the resolution principle~// J.~ACM, 1972. Vol.~19. Iss.~1. 
P.~109--119. doi: 10.1145/ 321679.321688.
\bibitem{6-les}
\Au{Mukaidono M.} Fuzzy inference of resolution style~// Fuzzy set and possibility theory~/ Ed. 
R.\,R.~Yager.~--- New York, NY, USA: Pergamon Press, 1988. P.~224--231.
\bibitem{7-les}
\Au{Dubois D., Prade~H.} Necessity and resolution principle~// IEEE T. Syst. Man  
Cyb., 1987. Vol.~17. Iss.~3. P.~474--478. doi: 10.1109/TSMC.1987.4309063.
\bibitem{8-les}
\Au{Guller D.} Hyperresolution for G$\ddot{\mbox{o}}$del logic with truth constants~// Fuzzy Set. 
Syst., 2019. Vol.~363. P.~1--65. doi: 10.1016/j.fss.2018.09.008.


\bibitem{11-les} %9
\Au{Tammet T.} A~resolution theorem prover for intuitonistic logic~// Automated deduction~--- 
Cade-13~/ Eds. M.\,A.~McRobbie, J.\,K.~Slaney.~--- Lecture notes in computer science ser.~--- 
Berlin, Heidelberg: Springer, 1996. Vol.~1104. P.~2--16. doi: 10.1007/3-540-61511-3\_65.
\bibitem{12-les} %10
\Au{Viedma M.\,A.\,C., Morales~R.\,M., Sanchez~I.\,N.} Fuzzy temporal constraint logic: A~valid resolution 
principle~// Fuzzy Set. Syst., 2001. Vol.~117. Iss.~2. P.~231--250. doi:  
10.1016/S0165-0114(99)00099-8.

\bibitem{10-les} %11
\Au{Habiballa H.} Resolution principle and fuzzy logic~// Fuzzy logic~--- algorithms, techniques, 
and implementations~/ Ed. E.~Dadios.~--- London: IntechOpen, 2012. P.~55--74.

\bibitem{13-les} %12
\Au{Nguyen T.\,M.\,T., Tran D.\,A.\,K.} Resolution method in linguistic propositional logic~// Int. J. 
Advanced Computer Science Applications, 2016. Vol.~7. Iss.~1. P.~672--678. doi: 
10.14569/IJACSA.2016.070191.

\bibitem{9-les} %13
\Au{Samokhvalov Y.} Proof of theorems in fuzzy logic based on structural resolution~// Cybern. 
Syst. Anal., 2019. Vol.~55. P.~207--219. doi: 10.1007/s10559-019-00125-8.

\bibitem{14-les}
\Au{Raha S., Ray~K.\,S.} Approximate reasoning based on generalized disjunctive syllogism~// 
Fuzzy Set. Syst., 1994. Vol.~61. Iss.~2. P.~143--151. doi:  
10.1016/0165-0114(94)90230-5.
\bibitem{15-les}
\Au{Mondal B., Raha~S.} Approximate reasoning in fuzzy resolution~// Int. J. Intelligence Science, 
2013. Vol.~3. Iss.~2. P.~86--98. doi: 10.4236/ijis.2013.32010.
\bibitem{16-les}
\Au{Леденева Т.\,М., Лещинская~М.\,В.} Метод резолюций и~стратегии поиска 
опровержений~// Вестник ВГУ. Сер. Системный анализ и~информационные технологии, 
2021. №\,1. С.~98--111. doi: 10.17308/sait.2021.1/3374. EDN: RBAVMW.
\bibitem{17-les}
\Au{Klement E.\,P., Mesiar~R., Pap~E.} Triangular norms. Position paper~II: General constructions and 
parameterized families~// Fuzzy Set. Syst., 2004. Vol.~145. P.~411--438. doi:  
10.1016/S0165-0114(03)00327-0.


\bibitem{19-les} %18
\Au{Ledeneva T.} Additive generators of fuzzy operation in the form of linear fractional function~// 
Fuzzy Set. Syst., 2020. Vol.~386. P.~1--24. doi: 10.1016/j.fss.2019.03.005.

\bibitem{18-les} %19
\Au{Ledeneva T.} New family of triangular norms for decreasing generators in the form of 
a~logarithm of a~linear fractional function~// Fuzzy Set. Syst., 2022. Vol.~427. P.~37--54. doi: 
10.1016/j.fss.2020.11.020.

\bibitem{20-les}
\Au{Вагин В.\,Н., Головина~Е.\,Ю., Загорянская~А.\,А., Фомина~М.\,В.} Достоверный 
и~правдоподобный вывод в~интеллектуальных системах~/ Под ред. В.\,Н.~Вагина, 
Д.\,А.~Поспелова.~--- М.: Физматлит, 2008. 712~с. EDN: MUWRTJ.
\end{thebibliography}

 }
 }

\end{multicols}

\vspace*{-6pt}

\hfill{\small\textit{Поступила в~редакцию 24.04.23}}

\vspace*{8pt}

%\pagebreak

%\newpage

%\vspace*{-28pt}

\hrule

\vspace*{2pt}

\hrule



\def\tit{ANALYSIS OF APPROACHES TO~DEFINING FUZZY RESOLVENT}


\def\titkol{Analysis of approaches to~defining fuzzy resolvent}


\def\aut{T.\,M.~Ledeneva and~M.\,V.~Leshchinskaya}

\def\autkol{T.\,M.~Ledeneva and~M.\,V.~Leshchinskaya}

\titel{\tit}{\aut}{\autkol}{\titkol}

\vspace*{-8pt}


\noindent
Voronezh State University, 1~Universitetskaya Sq., Voronezh 394010, Russian 
Federation

\def\leftfootline{\small{\textbf{\thepage}
\hfill INFORMATIKA I EE PRIMENENIYA~--- INFORMATICS AND
APPLICATIONS\ \ \ 2024\ \ \ volume~18\ \ \ issue\ 1}
}%
 \def\rightfootline{\small{INFORMATIKA I EE PRIMENENIYA~---
INFORMATICS AND APPLICATIONS\ \ \ 2024\ \ \ volume~18\ \ \ issue\ 1
\hfill \textbf{\thepage}}}

\vspace*{4pt}


\Abste{The article presents the results of a study concerning various definitions of 
the resolvent in fuzzy logic. Conditions are defined under which the Lee resolvent is 
a~significant logical consequence in the case of the classical definition of fuzzy 
logical connectives. It is shown that when using triangular norms and conorms for 
their formalization, it is impossible to obtain a~logically significant Lee resolvent. 
However, if the triangular conorm is defined as max, then the Lee resolvent exists for 
any triangular norm. Conditions are defined under which the Mukaidano resolvent is 
a~significant logical consequence for classical min and max operations. When using\linebreak\vspace*{-12pt}}

\pagebreak

\Abstend{triangular norms and conorms other than classical ones, further research is required. 
An illustrative example is provided demonstrating the process of constructing the 
Mukaidano resolvent.}

\KWE{resolvent; resolution method; triangular norms and conorms}

  \DOI{10.14357/19922264240113}{MIDEHY}

%\vspace*{-12pt}

%\Ack
%\vspace*{-3pt}
%\noindent
 


  \begin{multicols}{2}

\renewcommand{\bibname}{\protect\rmfamily References}
%\renewcommand{\bibname}{\large\protect\rm References}

{\small\frenchspacing
 {%\baselineskip=10.8pt
 \addcontentsline{toc}{section}{References}
 \begin{thebibliography}{99} 
\bibitem{1-les-1}
\Aue{Robinson, J.\,A.} 1965. A~machine-oriented logic based on the resolution 
principle.  \textit{J.~ACM} 12(1):23--41. doi: 10.1145/321250.321253.
\bibitem{2-les-1}
\Aue{Newborn, M.} 2000. \textit{Automated theorem proving: Theory and practice}. 
New York, NY: Springer. 231~p.
\bibitem{3-les-1}
\Aue{Lamberov, L.\,D.} 2021. Praktika komp'yuternykh dokazatel'stv 
i~chelovecheskoe ponimanie: epi\-ste\-mo\-lo\-gi\-che\-skaya problematika [Computer 
proofs practice and human understanding: Epistemological issues]. \textit{Vestnik 
Permskogo universiteta. Filosofiya. Psikhologiya. So\-tsi\-o\-lo\-giya} [Perm University 
Herald. Philosophy. Psychology. Sociology] 1:5--19.  
doi: 10.17072/2078-7898/2021-1-5-19. EDN: GTOYOE.
\bibitem{4-les-1}
\Aue{Novak, V., I.~Perfil'eva, and I.~Mochkorzh}. 2006. \textit{Ma\-te\-ma\-ti\-che\-skie 
prin\-tsi\-py ne\-chet\-koy logiki} [Mathematical principles of fuzzy logic]. Moscow: 
Fizmatlit. 252~p.
\bibitem{5-les-1}
\Aue{Lee, R.\,C.\,T.} 1972. Fuzzy logic and the resolution principle. \textit{J.~ACM} 
19(1):109--119. doi:10.1145/321679.321688.
\bibitem{6-les-1}
\Aue{Mukaidono, M.} 1988. Fuzzy inference of resolution style. \textit{Fuzzy set and 
possibility theory}. Ed. R.\,R.~Yager.  New York, NY: Pergamon Press. 224--231.
\bibitem{7-les-1}
\Aue{Dubois, D., and H.~Prade.} 1987. Necessity and resolution principle. 
\textit{IEEE T. Syst. Man Cyb.} 17(3):474--478. doi: 
10.1109/TSMC.1987.4309063.
\bibitem{8-les-1}
\Aue{Guller, D.} 2019. Hyperresolution for G$\ddot{\mbox{o}}$del logic with truth 
constants. \textit{Fuzzy Set. Syst.} 363:1--65. doi: 10.1016/ j.fss.2018.09.008.


\bibitem{11-les-1} %9
\Aue{Tammet, T.} 1996. A resolution theorem prover for intuitionistic logic. 
\textit{Automated deduction~--- Cade-13}. Eds. M.\,A.~McRobbie and 
J.\,K.~Slaney. Lecture notes in computer science ser. Berlin, Heidelberg: Springer. 
1104:2--16. doi: 10.1007/3-540-61511-3\_65.
\bibitem{12-les-1} %10
\Aue{Viedma, M.\,A.\,C., R.\,M.~Morales, and I.\,N.~Sanchez.} 2001. Fuzzy temporal 
constraint logic: A~valid resolution principle. \textit{Fuzzy Set. Syst.} 
117(2):231--250. doi: 10.1016/ S0165-0114(99)00099-8.
\bibitem{10-les-1} %11
\Aue{Habiballa, H.} 2012. Resolution principle and fuzzy logic. \textit{Fuzzy logic 
algorithms, techniques, and implementations}. Ed. E.~Dadios.  London: Intech.  
55--74.
\bibitem{13-les-1} %12
\Aue{Nguyen, T.\,M.\,T., and D.\,A.\,K.~Tran.} 2016. Resolution method in linguistic 
propositional logic. \textit{Int. J. Advanced Computer Science Applications}  
7(1):672--678.  doi: 10.14569/IJACSA.2016.070191.

\bibitem{9-les-1} %13
\Aue{Samokhvalov, Y.} 2019. Proof of theorems in fuzzy logic based on structural 
resolution. \textit{Cybern. Syst. Anal.} 55:207--219. doi:  
10.1007/s10559-019-00125-8.
\bibitem{14-les-1}
\Aue{Raha, S., and K.\,S.~Ray.} 1994. Approximate reasoning based on generalized 
disjunctive syllogism. \textit{Fuzzy Set. Syst.} 61(2):143--151.  
doi: 10.1016/0165-0114(94)90230-5.
\bibitem{15-les-1}
\Aue{Mondal, B., and S.~Raha.} 2013. Approximate reasoning in fuzzy resolution. 
\textit{Int. J. Intelligence Science} 3(2):86--98. doi: 10.4236/ijis.2013.32010.
\bibitem{16-les-1}
\Aue{Ledeneva, T.\,M., and M.\,V.~Leshchinskaya.} 2021. Metod rezolyutsiy 
i~strategii poiska oproverzheniy [The resolution inference and strategies for finding 
refutation]. \textit{Vestnik VGU. Ser. Sistemnyy analiz i~informatsionnye 
tekhnologii} [Proceedings of Voronezh State University. Ser. Systems Analysis and 
Information Technologies] 1:98--111. doi: 10.17308/sait.2021.1/3374. EDN: 
RBAVMW.
\bibitem{17-les-1}
\Aue{Klement, E.\,P., R.~Mesiar, and E.~Pap.} 2004. Triangular norms. Position 
paper II: General constructions and parameterized families. \textit{Fuzzy Set. Syst.} 
145(3):411--438. doi: 10.1016/S0165-0114(03)00327-0.

\bibitem{19-les-1}
\Aue{Ledeneva, T.} 2020. Additive generators of fuzzy operations in the form of linear 
fractional functions. \textit{Fuzzy Set. Syst.} 386:1--24. doi: 
10.1016/j.fss.2019.03.005.
\bibitem{18-les-1}
\Aue{Ledeneva, T.} 2022. New family of triangular norms for decreasing generators in 
the form of a~logarithm of a~linear fractional function. \textit{Fuzzy Set. Syst.} 
427:37--54. doi: 10.1016/j.fss.2020.11.020.

\bibitem{20-les-1}
\Aue{Vagin, V.\,N., E.\,Yu.~Golovina, A.\,A.~Zagoryanskaya, and M.\,V.~Fomina}. 
2008. \textit{Dostovernyy i~pravdopodobnyy vyvod v~intellektual'nykh sistemakh} 
[Reliable and plausible conclusion in intelligent systems]. Eds. V.\,N.~Vagin and 
D.\,A.~Pospelov. Moscow: Fizmatlit. 712~p. EDN: MUWRTJ.

\end{thebibliography}

 }
 }

\end{multicols}

\vspace*{-6pt}

\hfill{\small\textit{Received April 24, 2023}} 

\vspace*{-18pt}
     
     \Contr
     
     \vspace*{-3pt}

\noindent
\textbf{Ledeneva Tatyana M.} (b.\ 1959)~--- Doctor of Science in technology, 
professor, Department of Computational Mathematics and Applied Information 
Technologies, Faculty of Applied Mathematics, Informatics and Mechanics, Voronezh 
State University, 1~Universitetskaya Sq., Voronezh 394010, Russian Federation; 
\mbox{ledeneva-tm@yandex.ru}

\vspace*{3pt}

\noindent
\textbf{Leshchinskaya Maria V.} (b.\ 1995)~--- PhD student, Faculty of Applied 
Mathematics, Informatics and Mechanics, Voronezh State University, 
1~Universitetskaya Sq., Voronezh 394010, Russian Federation; maria-
\mbox{leshchinskaya@mail.ru}



\label{end\stat}

\renewcommand{\bibname}{\protect\rm Литература} 



\def\stat{authorsrus}
{%\hrule\par
%\vskip 7pt % 7pt
\raggedleft\Large \bf%\baselineskip=3.2ex
О\,Б\ \ А\,В\,Т\,О\,Р\,А\,Х \vskip 17pt
    \hrule
    \par
\vskip 21pt plus 8pt minus 4pt }


\def\tit{\ }

\def\aut{\ }

\def\auf{\ }

\def\leftkol{\ } % ENGLISH ABSTRACTS}

\def\rightkol{ОБ АВТОРАХ} %ENGLISH ABSTRACTS}

\titele{\tit}{\aut}{\auf}{\leftkol}{\rightkol}
      
            \label{st\stat}



\vspace*{24pt}

\begin{multicols}{2}




\noindent
\textbf{Архипов Олег Петрович} (р.\ 1948)~---
кандидат технических наук, директор Орловского филиала Института проб\-лем информатики
Российской академии наук
%302025, г.Орел, Московское шоссе, д.137

\vspace*{3pt}

\noindent
\textbf{Бирюкова Татьяна Константиновна} (р.\ 1968)~---
кандидат фи\-зи\-ко-ма\-те\-ма\-ти\-че\-ских наук, старший научный сотрудник Института проб\-лем информатики
Российской академии наук

\vspace*{3pt}

\noindent 
\textbf{Бобков  Сергей Геннадьевич} (р.\ 1955)~---
доктор технических наук,  заведующий отделением На\-уч\-но-ис\-сле\-до\-ва\-тель\-ско\-го 
института системных исследований Российской академии наук
%117218, Москва, Нахимовский просп., 36, к.1 

\vspace*{3pt}

\noindent \textbf{Васильев Николай Семенович} (р.\ 1952)~--- доктор 
фи\-зи\-ко-ма\-те\-ма\-ти\-че\-ских наук, профессор, 
МГТУ им.\ Н.\,Э.~Баумана 
%, Москва 105005, 2-я Бауманская ул., д.~5,

\vspace*{3pt}

\noindent
\textbf{Гершкович Максим Михайлович} (р.\ 1968)~---
старший научный сотрудник Института проб\-лем информатики
Российской академии наук

\vspace*{3pt}

\noindent 
\textbf{Дьяченко Юрий Георгиевич} (р.\ 1958)~--- кандидат технических наук, 
старший научный сотрудник Института проб\-лем информатики
Российской академии наук

\vspace*{3pt}

\noindent 
\textbf{Ерошенко Александр Андреевич} (р.\ 1989)~--- аспирант кафедры 
математической статистики факультета вычисли\-тельной математики и кибернетики 
Московского государственного университета им.\ М.\,В.~Ломоносова
%119991, Москва ГСП-1, Ленинские горы, д.\ 1, стр. 52

\vspace*{3pt}
 
\noindent 
\textbf{Захаров Виктор Николаевич} (р.\ 1948)~--- 
доктор технических наук, доцент, ученый секретарь Института проб\-лем информатики
Российской академии наук

\vspace*{3pt}

\noindent
\textbf{Зейфман Александр Израилевич} (р.\ 1954)~---
доктор фи\-зи\-ко-ма\-те\-ма\-ти\-че\-ских наук, профессор, 
заведующий кафедрой Вологодского государственного университета; 
старший научный сотрудник Института проб\-лем информатики
Российской академии наук; главный научный сотрудник ИСЭРТ Российской академии наук

\vspace*{3pt}

\noindent
\textbf{Зыкин Сергей Владимирович} (р.\ 1959)~--- 
доктор технических наук, профессор, заведующий лабораторией Института математики 
им.\ С.\,Л.~Соболева Сибирского отделения Российской академии наук, Новосибирск 
%630090, пр.\ ак.\ Коптюга, 4 

\vspace*{4pt}

\noindent
\textbf{Киреев Владимир Иванович} (р.\ 1938)~---
доктор фи\-зи\-ко-ма\-те\-ма\-ти\-че\-ских наук, профессор Московского 
государственного горного университета
%Адрес: Россия, 119991, г. Москва, Ленинский проспект, д. 6

%\columnbreak

\vspace*{4pt}

\noindent
\textbf{Козеренко Елена Борисовна} (р.\ 1959)~---
кандидат филологических наук, заведующая лабораторией Института проб\-лем информатики
Российской академии наук

\vspace*{4pt}

\noindent
\textbf{Королев Виктор Юрьевич} (р.\ 1954)~--- доктор
фи\-зи\-ко-ма\-те\-ма\-ти\-че\-ских наук, профессор кафедры математической 
статистики факультета вычисли\-тельной математики и кибернетики 
Московского государственного университета; 
ведущий научный сотрудник Института проб\-лем информатики
Российской академии наук

\vspace*{4pt}

\noindent
\textbf{Коротышева Анна Владимировна} (р.\ 1988)~---
старший преподаватель Вологодского государственного университета

\vspace*{4pt}

\noindent 
\textbf{Кун Де Турк} (р.\ 1981)~--- научный сотрудник 
исследовательской группы SMACS факультета телекоммуникаций и обработки информации
Университета Гента, Бельгия
%В-9000 Гент, Бельгия

\vspace*{4pt}

\noindent
\textbf{Лупенцов Олег Сергеевич} (р.\ 1986)~---
аспирант Омского государственного института сервиса
%Омск 644043, ул.\ Певцова 13

\vspace*{4pt}

\noindent
\textbf{Лучко Олег Николаевич} (р.\ 1961)~---
кандидат педагогических наук, профессор, заведующий кафедрой 
Омского государственного института сервиса
%Омск 644043, ул.\ Певцова 13

\vspace*{4pt}

\noindent
\textbf{Малашенко Юрий Евгеньевич} (р.\ 1946)~---
доктор фи\-зи\-ко-ма\-те\-ма\-ти\-че\-ских наук, заведующий сектором 
Вычислительного центра им.\ А.\,А.~Дородницына Российской академии наук
%Адрес: 119333, Москва, ул. Вавилова, 40,

\vspace*{4pt}

\noindent
\textbf{Маньяков Юрий Анатольевич} (р.\ 1984)~---
кандидат технических наук, научный сотрудник Орловского филиала Института проб\-лем информатики
Российской академии наук
%302025, г.Орел, Московское шоссе, д.137

\vspace*{4pt}

\noindent
\textbf{Маренко Валентина Афанасьевна} (р.\ 1951)~---
кандидат технических наук, доцент, старший научный сотрудник 
Института математики им.\ С.\,Л.~Соболева Сибирского отделения Российской академии наук
%Новосибирск 630090, пр. ак. Коптюга, 4 

\vspace*{3pt}

\noindent 
\textbf{Морозов Евсей Викторович} (р.\ 1947)~--- доктор 
фи\-зи\-ко-ма\-те\-ма\-ти\-че\-ских, профессор, ведущий научный сотрудник 
Института прикладных математических исследований Карельского научного центра Российской
академии наук; 
%%185910 Россия, Республика Карелия, г.\ Петрозаводск, ул.\ Пушкинская, 11
профессор Петрозаводского государственного университета, Петрозаводск
%185910 Россия, Республика Карелия, г.\ Петрозаводск, пр.\ Ленина, 33

%\pagebreak

\vspace*{3pt}

\noindent
\textbf{Назарова Ирина Александровна} (р.\ 1966)~---
кандидат фи\-зи\-ко-ма\-те\-ма\-ти\-че\-ских наук, 
научный сотрудник Вычислительного центра им.\ А.\,А.~Дородницына Российской академии наук 
%Адрес: 119333, Москва, ул. Вавилова, 40

\vspace*{3pt}

\noindent
\textbf{Павлов Игорь Валерианович} (р.\ 1945)~--- 
доктор фи\-зи\-ко-ма\-те\-ма\-ти\-че\-ских наук, профессор МГТУ им.\ Н.\,Э.~Баумана 
%Москва 105005, 2-я Бауманская ул., д.~5 

%\pagebreak

\vspace*{3pt}

\noindent 
\textbf{Потахина Любовь Викторовна} (р.\ 1989)~--- аспирантка
Института прикладных математических исследований Карельского научного центра
Российской академии наук; 
%%185910 Россия, Республика Карелия, г.\ Петрозаводск, ул.\ Пушкинская, 11
инженер Петрозаводского государственного университета, Петрозаводск
%185910 Россия, Республика Карелия, г.\ Петрозаводск, пр.\ Ленина, 33

\vspace*{3pt}

\noindent 
\textbf{Рождественский Юрий Владимирович} (р.\ 1952)~--- 
кандидат технических наук, заведующий сектором Института проб\-лем информатики
Российской академии наук

\vspace*{3pt}

\noindent 
\textbf{Синицын Игорь Николаевич} (р.\ 1940)~--- доктор технических наук,
профессор, заслуженный деятель\linebreak\vspace*{-12pt}

\columnbreak

\noindent
 науки РФ, заведующий отделом Института проб\-лем информатики
Российской академии наук

\vspace*{7pt}


\noindent
\textbf{Сиротинин Денис Олегович} (р.\ 1984)~---
кандидат технических наук, научный сотрудник Орловского филиала Института проб\-лем информатики
Российской академии наук
%302025, г.Орел, Московское шоссе, д.137

\vspace*{7pt}

%\columnbreak

\noindent 
\textbf{Соколов  Игорь Анатольевич} (р.\ 1954)~--- академик (действительный член) Российской 
академии наук, доктор технических наук, директор Института проб\-лем информатики
Российской академии наук

\vspace*{7pt}

\noindent
\textbf{Степченков Юрий Афанасьевич} (р.\ 1951)~---
кандидат технических наук, заведующий отделом Института проб\-лем информатики
Российской академии наук

\vspace*{7pt}

\noindent
\textbf{Сурков Алексей Викторович} (р.\ 1978)~--- 
старший научный сотрудник На\-уч\-но-ис\-сле\-до\-ва\-тель\-ско\-го 
института системных исследований Российской академии наук
%117218, Москва, Нахимовский просп., 36, к.1 

\vspace*{7pt}

\noindent 
\textbf{Шестаков Олег Владимирович} (р.\ 1976)~--- доктор 
фи\-зи\-ко-ма\-те\-ма\-ти\-че\-ских, доцент кафедры математической статистики 
факультета вычисли\-тельной математики и кибернетики Московского 
государственного университета им.\ М.\,В.~Ломоносова; 
%119991, Москва ГСП-1, Ленинские горы, д.\ 1, стр. 52
старший научный сотрудник Института проб\-лем информатики
Российской академии наук
%, Москва 119333, ул. Вавилова, д.~44, корп.~2

\vspace*{7pt}

\noindent 
\textbf{Шоргин Сергей Яковлевич} (р.\ 1952.)~--- доктор
фи\-зи\-ко-ма\-те\-ма\-ти\-че\-ских наук, профессор, заместитель директора Института 
проб\-лем информатики Российской академии наук





%%%%%%%%%%%%%%%%%%%%%%%%%%%%%%%%%%%%%%%%%%%%%%%%%%%%%%%%%%%%%%%%%%%%%%%%%%%%%%%




%\def\rightkol{ОБ АВТОРАХ}
%\def\leftkol{ОБ АВТОРАХ}

 \label{end\stat}





%\def\leftfootline{\small{\textbf{\thepage}
%\hfill ИНФОРМАТИКА И ЕЁ ПРИМЕНЕНИЯ\ \ \ том~7\ \ \ выпуск~1\ \ \ 2013}
%}%
% \def\rightfootline{\small{ИНФОРМАТИКА И ЕЁ ПРИМЕНЕНИЯ\ \ \ том~7\ \ \ выпуск~1\ \ \ 2013
%\hfill \textbf{\thepage}}}


%\thispagestyle{myheadings}



\end{multicols}

\newpage  

%\def\stat{cont}
{%\hrule\par
%\vskip 7pt % 7pt
\raggedleft\Large \bf%\baselineskip=3.2ex
А\,В\,Т\,О\,Р\,С\,К\,И\,Й\ \ У\,К\,А\,З\,А\,Т\,Е\,Л\,Ь\ \ З\,А\ \ 2\,0\,0\,7 г. \vskip 17pt
    \hrule
    \par
\vskip 21pt plus 6pt minus 3pt }

\label{st\stat}

\def\tit{\ }

\def\aut{\ }
\def\auf{\ }

\def\leftkol{\ } % ENGLISH ABSTRACTS}

\def\rightkol{\ } %ENGLISH ABSTRACTS}

\titele{\tit}{\aut}{\auf}{\leftkol}{\rightkol}


\contentsline {chapter}{\ }{Выпуск \quad Стр.} 
\contentsline {section}{\textbf{Батракова Д.\,А., Королев В.\,Ю., Шоргин С.\,Я.}\ \ Новый метод вероятностно-ста\-ти\-сти\-че\-ско\-го анализа информационных потоков в\nobreakspace {}телекоммуникационных сетях}{\qquad 1 \qquad 40} 
\contentsline {section}{\textbf{Борисов А.\,В.}\ \ Байесовское оценивание в системах наблюдения с\nobreakspace {}марковскими скачкообразными процессами: игровой подход}{\qquad 2 \qquad 65}
\contentsline {section}{\textbf{Босов А.\,В., Иванов А.\,В.}\ \ Программная инфраструктура информационного Web-пор\-тала}{\qquad 2 \qquad 50}
\contentsline {section}{\textbf{Захаров В.\,Н., Калиниченко Л.\,А., Соколов И.\,А., Ступников С.\,А.}\ \ Конструирование канонических информационных моделей для интегрированных информационных систем}{\qquad 2 \qquad 15}
\contentsline {section}{\textbf{Захаров В.\,Н., Козмидиади В.\,А.}\ \ Средства обеспечения отказоустойчивости при\-ло\-жений}{\qquad 1 \qquad 14} 
\contentsline {section}{\textbf{Иванов А.\,В.}\ \ см. Босов А.\,В.\hfill\hfill\hfill\hfill\hfill\hfill\hfill\hfill\hfill\hfill\hfill\hfill\hfill\hfill\hfill\hfill\hfill\hfill\hfill\hfill\hfill\hfill\hfill\hfill\hfill\hfill\hfill\hfill\hfill\hfill\hfill\hfill\hfill\hfill\hfill}{\ }
\contentsline {section}{\textbf{Ильин В.\,Д., Соколов И.\,А.}\ \ Символьная модель системы знаний информатики в\nobreakspace {}че\-ло\-ве\-ко-автоматной среде}{\qquad 1 \qquad 66} 
\contentsline {section}{\textbf{Калиниченко Л.\,А.}\ \ см. Захаров В.\,Н.\hfill\hfill\hfill\hfill\hfill\hfill\hfill\hfill\hfill\hfill\hfill\hfill\hfill\hfill\hfill\hfill\hfill\hfill\hfill\hfill\hfill\hfill\hfill\hfill\hfill\hfill\hfill\hfill\hfill\hfill\hfill\hfill\hfill\hfill\hfill}{\ }
\contentsline {section}{\textbf{Козеренко Е.\,Б.}\ \ Лингвистическое моделирование для систем машинного перевода и обработки знаний}{\qquad 1 \qquad 54} 
\contentsline {section}{\textbf{Козмидиади В.\,А.}\ \ см. Захаров В.\,Н.\hfill\hfill\hfill\hfill\hfill\hfill\hfill\hfill\hfill\hfill\hfill\hfill\hfill\hfill\hfill\hfill\hfill\hfill\hfill\hfill\hfill\hfill\hfill\hfill\hfill\hfill\hfill\hfill\hfill\hfill\hfill\hfill\hfill\hfill\hfill }{\ } 
\contentsline {section}{\textbf{Королев В.\,Ю.}\ \ см. Батракова Д.\,А.\hfill\hfill\hfill\hfill\hfill\hfill\hfill\hfill\hfill\hfill\hfill\hfill\hfill\hfill\hfill\hfill\hfill\hfill\hfill\hfill\hfill\hfill\hfill\hfill\hfill\hfill\hfill\hfill\hfill\hfill\hfill\hfill\hfill\hfill\hfill}{\ } 
\contentsline {section}{\textbf{Кудрявцев А.\,А., Шоргин С.\,Я.}\ \ Байесовский подход к\nobreakspace {}анализу систем массового обслуживания и\nobreakspace {}показателей надежности}{\qquad 2 \qquad 76}
\contentsline {section}{\textbf{Печинкин А.\,В., Соколов И.\,А., Чаплыгин В.\,В.}\ \ Многолинейная система массового обслуживания с конечным накопителем и ненадежными приборами}{\qquad 1 \qquad 27} 
\contentsline {section}{\textbf{Печинкин А.\,В., Соколов И.\,А., Чаплыгин В.\,В.}\ \ Стационарные характеристики многолинейной\nobreakspace {}системы массового обслуживания с\nobreakspace {}одновременными отказами приборов}{\qquad 2 \qquad 39}
\contentsline {section}{\textbf{Синицын И.\,Н.}\ \ Корреляционные методы построения аналитических информационных моделей флуктуаций полюса Земли по априорным данным}{\qquad 2 \qquad \hphantom{9}2}
\contentsline {section}{\textbf{Синицын И.\,Н.}\ \ Развитие теории фильтров Пугачева для оперативной обработки информации в стохастических системах}{{\qquad 1 \qquad \hphantom{9}3}} 
\contentsline {section}{\textbf{Соколов И.\,А.}\ \ см. Захаров В.\,Н.\hfill\hfill\hfill\hfill\hfill\hfill\hfill\hfill\hfill\hfill\hfill\hfill\hfill\hfill\hfill\hfill\hfill\hfill\hfill\hfill\hfill\hfill\hfill\hfill\hfill\hfill\hfill\hfill\hfill\hfill\hfill\hfill\hfill\hfill\hfill}{\ }
\contentsline {section}{\textbf{Соколов И.\,А.}\ \ см. Ильин В.\,Д.\hfill\hfill\hfill\hfill\hfill\hfill\hfill\hfill\hfill\hfill\hfill\hfill\hfill\hfill\hfill\hfill\hfill\hfill\hfill\hfill\hfill\hfill\hfill\hfill\hfill\hfill\hfill\hfill\hfill\hfill\hfill\hfill\hfill\hfill\hfill}{\ } 
\contentsline {section}{\textbf{Соколов И.\,А.}\ \ см. Печинкин А.\,В.\hfill\hfill\hfill\hfill\hfill\hfill\hfill\hfill\hfill\hfill\hfill\hfill\hfill\hfill\hfill\hfill\hfill\hfill\hfill\hfill\hfill\hfill\hfill\hfill\hfill\hfill\hfill\hfill\hfill\hfill\hfill\hfill\hfill\hfill\hfill}{\ } 
\contentsline {section}{\textbf{Соколов И.\,А.}\ \ см. Печинкин А.\,В.\hfill\hfill\hfill\hfill\hfill\hfill\hfill\hfill\hfill\hfill\hfill\hfill\hfill\hfill\hfill\hfill\hfill\hfill\hfill\hfill\hfill\hfill\hfill\hfill\hfill\hfill\hfill\hfill\hfill\hfill\hfill\hfill\hfill\hfill\hfill}{\ }
\contentsline {section}{\textbf{Ступников С.\,А.}\ \ см. Захаров В.\,Н.\hfill\hfill\hfill\hfill\hfill\hfill\hfill\hfill\hfill\hfill\hfill\hfill\hfill\hfill\hfill\hfill\hfill\hfill\hfill\hfill\hfill\hfill\hfill\hfill\hfill\hfill\hfill\hfill\hfill\hfill\hfill\hfill\hfill\hfill\hfill}{\ }
\contentsline {section}{\textbf{Чаплыгин В.\,В.}\ \ см. Печинкин А.\,В.\hfill\hfill\hfill\hfill\hfill\hfill\hfill\hfill\hfill\hfill\hfill\hfill\hfill\hfill\hfill\hfill\hfill\hfill\hfill\hfill\hfill\hfill\hfill\hfill\hfill\hfill\hfill\hfill\hfill\hfill\hfill\hfill\hfill\hfill\hfill}{\ } 
\contentsline {section}{\textbf{Чаплыгин В.\,В.}\ \ см. Печинкин А.\,В.\hfill\hfill\hfill\hfill\hfill\hfill\hfill\hfill\hfill\hfill\hfill\hfill\hfill\hfill\hfill\hfill\hfill\hfill\hfill\hfill\hfill\hfill\hfill\hfill\hfill\hfill\hfill\hfill\hfill\hfill\hfill\hfill\hfill\hfill\hfill}{\ }
\contentsline {section}{\textbf{Шоргин С.\,Я.}\ \ см. Батракова Д.\,А.\hfill\hfill\hfill\hfill\hfill\hfill\hfill\hfill\hfill\hfill\hfill\hfill\hfill\hfill\hfill\hfill\hfill\hfill\hfill\hfill\hfill\hfill\hfill\hfill\hfill\hfill\hfill\hfill\hfill\hfill\hfill\hfill\hfill\hfill\hfill}{\ } 
\contentsline {section}{\textbf{Шоргин С.\,Я.}\ \ см. Кудрявцев А.\,А.\hfill\hfill\hfill\hfill\hfill\hfill\hfill\hfill\hfill\hfill\hfill\hfill\hfill\hfill\hfill\hfill\hfill\hfill\hfill\hfill\hfill\hfill\hfill\hfill\hfill\hfill\hfill\hfill\hfill\hfill\hfill\hfill\hfill\hfill\hfill}{\ }
%\thispagestyle{myheadings}
\def\leftfootline{\small{\textbf{\thepage}
\hfill ИНФОРМАТИКА И ЕЁ ПРИМЕНЕНИЯ\ \ \ том~1\ \ \ выпуск~2\ \ \ 2007}
}%
 \def\rightfootline{\small{ИНФОРМАТИКА И ЕЁ ПРИМЕНЕНИЯ\ \ \ том~1\ \ \ выпуск~2\ \ \ 2007
 \hfill \textbf{\thepage}}}
 \label{end\stat} 
                     
%\def\stat{cont-e}
{%\hrule\par
%\vskip 7pt % 7pt
\raggedleft\Large \bf%\baselineskip=3.2ex
2\,0\,0\,7\ \ A\,U\,T\,H\,O\,R\ \ I\,N\,D\,E\,X \vskip 17pt
    \hrule
    \par
\vskip 21pt plus 6pt minus 3pt }

\label{st\stat}

\def\tit{\ }

\def\aut{\ }
\def\auf{\ }

\def\leftkol{\ } % ENGLISH ABSTRACTS}

\def\rightkol{\ } %ENGLISH ABSTRACTS}

\titele{\tit}{\aut}{\auf}{\leftkol}{\rightkol}


\contentsline {chapter}{\ }{Issue \quad Page} 
\contentsline {subsection}{\textbf{Batrakova D.\,A., Korolev V.\,Yu., Shorgin S.\,Ya.}\ \ A New Method for the Probabilistic and Statistical Analysis of Information Flows in Telecommunication Networks}{\qquad 1 \qquad 40} 
\contentsline {subsection}{\textbf{Borisov A.\,V.}\ \ Bayesian Estimation in\nobreakspace {}Observation Systems with\nobreakspace {}Markov Jump Processes: Game-Theoretic Approach}{\qquad 2 \qquad 65} 
\contentsline {subsection}{\textbf{Bosov A.\,V., Ivanov A.\,V.}\ \ Linguistic Simulation for Machine Translation and Knowledge Management Systems}{\qquad 2 \qquad 50} 
\contentsline {subsection}{\textbf{Chaplygin V.\,V.} see Pechinkin A.\,V.\hfill\hfill\hfill\hfill\hfill\hfill\hfill\hfill\hfill\hfill\hfill\hfill\hfill\hfill\hfill\hfill\hfill\hfill\hfill\hfill\hfill\hfill\hfill\hfill\hfill\hfill\hfill\hfill\hfill\hfill\hfill\hfill\hfill\hfill\hfill}{\ }
\contentsline {subsection}{\textbf{Chaplygin V.\,V.} see Pechinkin A.\,V.\hfill\hfill\hfill\hfill\hfill\hfill\hfill\hfill\hfill\hfill\hfill\hfill\hfill\hfill\hfill\hfill\hfill\hfill\hfill\hfill\hfill\hfill\hfill\hfill\hfill\hfill\hfill\hfill\hfill\hfill\hfill\hfill\hfill\hfill\hfill}{\ }
\contentsline {subsection}{\textbf{Ilyin V.\,D., Sokolov I.\,A.}\ \ The Symbol Model of Informatics Knowledge System in Human-Automaton Environment}{\qquad 1 \qquad 66} 
\contentsline {subsection}{\textbf{Ivanov A.\,V.} see Bosov A.\,V.\hfill\hfill\hfill\hfill\hfill\hfill\hfill\hfill\hfill\hfill\hfill\hfill\hfill\hfill\hfill\hfill\hfill\hfill\hfill\hfill\hfill\hfill\hfill\hfill\hfill\hfill\hfill\hfill\hfill\hfill\hfill\hfill\hfill\hfill\hfill}{\ }
\contentsline {subsection}{\textbf{Kalinichenko L.\,A.} see Zakharov V.\,N.\hfill\hfill\hfill\hfill\hfill\hfill\hfill\hfill\hfill\hfill\hfill\hfill\hfill\hfill\hfill\hfill\hfill\hfill\hfill\hfill\hfill\hfill\hfill\hfill\hfill\hfill\hfill\hfill\hfill\hfill\hfill\hfill\hfill\hfill\hfill}{\ }
\contentsline {subsection}{\textbf{Korolev V.\,Yu.} see Batrakova D.\,A.\hfill\hfill\hfill\hfill\hfill\hfill\hfill\hfill\hfill\hfill\hfill\hfill\hfill\hfill\hfill\hfill\hfill\hfill\hfill\hfill\hfill\hfill\hfill\hfill\hfill\hfill\hfill\hfill\hfill\hfill\hfill\hfill\hfill\hfill\hfill}{\ }
\contentsline {subsection}{\textbf{Kozerenko E.\,B.}\ \ Linguistic Simulation for Machine Translation and Knowledge Management Systems}{\qquad 1 \qquad 54} 
\contentsline {subsection}{\textbf{Kozmidiady V.\,A.} see Zakharov V.\,N.\hfill\hfill\hfill\hfill\hfill\hfill\hfill\hfill\hfill\hfill\hfill\hfill\hfill\hfill\hfill\hfill\hfill\hfill\hfill\hfill\hfill\hfill\hfill\hfill\hfill\hfill\hfill\hfill\hfill\hfill\hfill\hfill\hfill\hfill\hfill}{\ }
\contentsline {subsection}{\textbf{Kudryavtsev A.\,A., Shorgin S.\,Ya.}\ \ Bayesian Approach to Queueing Systems and Reliability Characteristics}{\qquad 2 \qquad 76} 
\contentsline {subsection}{\textbf{Pechinkin A.\,V., Sokolov I.\,A., Chaplygin V.\,V.}\ \ Multichannel Queuing System with Finite Buffer and Unreliable Servers}{\qquad 1 \qquad 27} 
\contentsline {subsection}{\textbf{Pechinkin A.\,V., Sokolov I.\,A., Chaplygin V.\,V.}\ \ Stationary Characteristics of a Multichannel Queueing System with\nobreakspace {}Simultaneous Refusals of Servers}{\qquad 2 \qquad 39} 
\contentsline {subsection}{\textbf{Shorgin S.\,Ya.} see Batrakova D.\,A.\hfill\hfill\hfill\hfill\hfill\hfill\hfill\hfill\hfill\hfill\hfill\hfill\hfill\hfill\hfill\hfill\hfill\hfill\hfill\hfill\hfill\hfill\hfill\hfill\hfill\hfill\hfill\hfill\hfill\hfill\hfill\hfill\hfill\hfill\hfill}{\ }
\contentsline {subsection}{\textbf{Shorgin S.\,Ya.} see Kudryavtsev A.\,A.\hfill\hfill\hfill\hfill\hfill\hfill\hfill\hfill\hfill\hfill\hfill\hfill\hfill\hfill\hfill\hfill\hfill\hfill\hfill\hfill\hfill\hfill\hfill\hfill\hfill\hfill\hfill\hfill\hfill\hfill\hfill\hfill\hfill\hfill\hfill}{\ }
\contentsline {subsection}{\textbf{Sinitsyn I.\,N.}\ \ Correlational Methods for Analytical Informational Models of the Earth Pole Fluctuations Design Based on a priori Data}{\qquad 2 \qquad \hphantom{9}2}
\contentsline {subsection}{\textbf{Sinitsyn I.\,N.}\ \ Development of Pugachev Filtering for Stochastic Systems}{\qquad 1 \qquad \hphantom{9}3}
\contentsline {subsection}{\textbf{Sokolov I.\,A.} see Ilyin V.\,D.\hfill\hfill\hfill\hfill\hfill\hfill\hfill\hfill\hfill\hfill\hfill\hfill\hfill\hfill\hfill\hfill\hfill\hfill\hfill\hfill\hfill\hfill\hfill\hfill\hfill\hfill\hfill\hfill\hfill\hfill\hfill\hfill\hfill\hfill\hfill}{\ }
\contentsline {subsection}{\textbf{Sokolov I.\,A.} see Pechinkin A.\,V.\hfill\hfill\hfill\hfill\hfill\hfill\hfill\hfill\hfill\hfill\hfill\hfill\hfill\hfill\hfill\hfill\hfill\hfill\hfill\hfill\hfill\hfill\hfill\hfill\hfill\hfill\hfill\hfill\hfill\hfill\hfill\hfill\hfill\hfill\hfill}{\ }
\contentsline {subsection}{\textbf{Sokolov I.\,A.} see Pechinkin A.\,V.\hfill\hfill\hfill\hfill\hfill\hfill\hfill\hfill\hfill\hfill\hfill\hfill\hfill\hfill\hfill\hfill\hfill\hfill\hfill\hfill\hfill\hfill\hfill\hfill\hfill\hfill\hfill\hfill\hfill\hfill\hfill\hfill\hfill\hfill\hfill}{\ }
\contentsline {subsection}{\textbf{Sokolov I.\,A.} see Zakharov V.\,N.\hfill\hfill\hfill\hfill\hfill\hfill\hfill\hfill\hfill\hfill\hfill\hfill\hfill\hfill\hfill\hfill\hfill\hfill\hfill\hfill\hfill\hfill\hfill\hfill\hfill\hfill\hfill\hfill\hfill\hfill\hfill\hfill\hfill\hfill\hfill}{\ }
\contentsline {subsection}{\textbf{Stupnikov S.\,A.} see Zakharov V.\,N.\hfill\hfill\hfill\hfill\hfill\hfill\hfill\hfill\hfill\hfill\hfill\hfill\hfill\hfill\hfill\hfill\hfill\hfill\hfill\hfill\hfill\hfill\hfill\hfill\hfill\hfill\hfill\hfill\hfill\hfill\hfill\hfill\hfill\hfill\hfill}{\ }
\contentsline {subsection}{\textbf{Zakharov V.\,N., Kalinichenko L.\,A., Sokolov I.\,A., Stupnikov S.\,A.}\ \ Development of Canonical Information Models for Integrated Information Systems}{\qquad 2 \qquad 15} 
\contentsline {subsection}{\textbf{Zakharov V.\,N., Kozmidiady V.\,A.}\ \ Means Providing Applications Fault Tolerance}{\qquad 1 \qquad 14} 
\def\leftfootline{\small{\textbf{\thepage}
\hfill ИНФОРМАТИКА И ЕЁ ПРИМЕНЕНИЯ\ \ \ том~1\ \ \ выпуск~2\ \ \ 2007}
}%
 \def\rightfootline{\small{ИНФОРМАТИКА И ЕЁ ПРИМЕНЕНИЯ\ \ \ том~1\ \ \ выпуск~2\ \ \ 2007
 \hfill \textbf{\thepage}}}
 \label{end\stat} 


%\end{document}

%
\def\stat{rekl}
%\label{preobr}

%\def\tit{АКАДЕМИК ПУГАЧЁВ  ВЛАДИМИР СЕМЁНОВИЧ\\
%25.03.1911--25.03.1998}


%   \vspace*{-48pt}
%   \begin{center}\LARGE
%Академик Пугачёв  Владимир Семёнович\\ (25.03.1911--25.03.1998)
%   \end{center}

   %\vspace*{2.5mm}

   \begin{center}

{\prgsh\LARGE
ЮБИЛЕИ}

\end{center}
%\hrule

\vspace*{6pt}


   \vspace*{8mm}

   \thispagestyle{empty}


%\def\stat{emel}


\section*{К 70-летию заместителя директора ИПИ РАН,\\ члена редколлегии журнала
<<Информатика и её применения>>\\ доктора технических наук В.\,И.~Будзко}

\vspace*{18pt}




          \begin{multicols}{2}

%            \label{st\stat}

\begin{center}
\vspace*{1pt}
\mbox{%
\epsfxsize=78mm
\epsfbox{bud-1.eps}
}
\end{center}

\vspace*{12pt}

      14 августа 2014~г.\ исполнилось 70~лет за\-мес\-ти\-те\-лю директора ИПИ РАН по
научной работе доктору технических наук Владимиру Игоревичу Будзко.

      Владимир Игоревич Будзко родился в г.~Москве. Высшее образование получил на факультете
элект\-рон\-но-вы\-чис\-ли\-тель\-ных устройств в Московском
ин\-же\-нер\-но-фи\-зи\-че\-ском институте
(МИФИ), который он окончил в 1968~г., после чего был на\-прав\-лен для прохождения
службы в одну из войс\-ко\-вых частей, где прошел путь от инженера до первого заместителя
командира войсковой части.

      С приходом В.\,И.~Будзко в ИПИ РАН (2001~г.)\ в институте
сформировалось новое научное на\-прав\-ле\-ние теоретических исследований~--- <<Постро\-ение
ин\-фор\-ма\-ци\-он\-но-те\-ле\-ком\-му\-ни\-ка\-ци\-он\-ных\linebreak сис\-тем
высокой до\-ступ\-ности>>. В~рамках этого
направления выполнен широкий круг фундаментальных исследований по поиску подходов и
определению принципов построения средств обеспечения доступности, конфиденциальности
и целостности современных крупномасштабных
ин\-фор\-ма\-ци\-он\-но-те\-ле\-ком\-му\-ни\-ка\-ци\-он\-ных
сис\-тем (ИТС). Разработаны основные сис\-тем\-но-тех\-ни\-че\-ские принципы и базовые
архитектурные решения построения перспективных для условий России ИТС с
централизованной обработкой и хранением информации, сочетающих в себе свойства
высокой доступности, отказо- и катастрофоустойчивости, информационной защищенности.
Определены принципы, методы и математические основы рационального построения и
оптимизации средств восстановления функционирования центров обработки данных (ЦОД)
после возникновения отказов и катастроф, передачи и хранения данных, обеспечения
информационной безопасности при достижении минимальной совокупной стоимости
владения такими системами. Результаты нашли практическое воплощение при реализации
проектов в интересах ряда отечественных государственных и негосударственных
организаций, таких как Банк России (БР), Внешторгбанк, ОАО <<ГМК <<Норильский Никель>>,
<<Газпром>>, Минэкономразвития России, Правительство Москвы, а также ряд силовых
ведомств.

      Под руководством В.\,И.~Будзко начиная с 2001~г.\ выполнен комплекс
      на\-уч\-но-ис\-сле\-до\-ва\-тель\-ских и
      опыт\-но-кон\-ст\-рук\-тор\-ских работ (свыше 100~проектов),
направленных на развитие электронной информационной технологии БР.
Разработаны концепции развития ИТС БР сначала до 2008~г., а затем до 2013~г., которые
были приняты в качестве основы проведения технической политики. За реализацию проекта
<<Катастрофоустойчивая тер\-ри\-то\-ри\-аль\-но-рас\-пре\-де\-лен\-ная
      ин\-фор\-ма\-ци\-он\-но-те\-ле\-ком\-му\-ни\-ка\-ци\-он\-ная сис\-те\-ма централизованной
обработки банковской информации>> В.\,И.~Будзко удостоен Премии Правительства РФ в
области науки и техники за 2010~г.

      В.\,И.~Будзко возглавлял и возглавляет работы по ряду других прикладных проектов,
связанных с созданием, совершенствованием и развитием крупномасштабных ИТС.

      В.\,И.~Будзко~--- генерал-майор, доктор технических наук, член-кор\-рес\-пон\-дент
Академии криптографии РФ, известный ученый в области информатики и применения
информационных технологий при построении территориально распределенных ИТС
различного назначения. Является автором свыше 250~научных работ, опубликованных в
на\-уч\-но-тех\-ни\-че\-ских и специальных изданиях.

    \thispagestyle{empty}

      В.\,И.~Будзко уделяет большое внимание подготовке научных кадров. Под его
руководством защищено 6~диссертаций на соискание ученой степени кандидата
технических наук. Свыше 30~лет он читает лекции в ИКСИ Академии ФСБ, профессор
кафедры НИЯУ МИФИ. Является членом двух диссертационных советов, главным
редактором журнала <<Системы высокой доступности>> и членом редколлегии журнала
<<Информатика и её применения>>.

      \bigskip

      Редакционный совет и Редакционная коллегия журнала <<Информатика и её
применения>> сердечно поздравляют Владимира Игоревича Будзко с 70-ле\-ти\-ем и желают
крепкого здоровья и новых научных достижений.

\end{multicols}



\def\stat{cont}
{%\hrule\par
%\vskip 7pt % 7pt
\raggedleft\Large \bf%\baselineskip=3.2ex
А\,В\,Т\,О\,Р\,С\,К\,И\,Й\ \ У\,К\,А\,З\,А\,Т\,Е\,Л\,Ь\ \ З\,А\ \ 2\,0\,2\,3 г. \vskip 17pt
 \hrule
 \par
\vskip 21pt plus 6pt minus 3pt }

\label{st\stat}

\def\tit{\ }

\def\aut{\ }
\def\auf{\ }

\def\leftkol{\ } % ENGLISH ABSTRACTS}

\def\rightkol{\ } %АВТОРСКИЙ УКАЗАТЕЛЬ ЗА 2021 г.} %ENGLISH ABSTRACTS}

\titele{\tit}{\aut}{\auf}{\leftkol}{\rightkol}
\addcontentsline{toc}{subsection}{\textrm\textbf Авторский указатель за 2023 г.}

\vspace*{-24pt}

\noindent
{\tabcolsep=3pt
\begin{tabular}{p{397pt}cc}
&\textbf{Вып.} & \textbf{Стр.}\\[6pt]
\Avtors{Агаларов~Я.\,М.} Об оптимизации работы резервного прибора в~многолинейной 
системе массового обслуживания&\raisebox{-12pt}[0pt][0pt]{1}&\raisebox{-12pt}[0pt][0pt]{89--95}\\
\Avtors{Агаларов~Я.\,М.} Оптимизация схемы распределения буферной памяти узла 
пакетной коммутации&\raisebox{-12pt}[0pt][0pt]{3}&\raisebox{-12pt}[0pt][0pt]{39--48}\\
\Avtors{Агасандян~Г.\,А.} Многомерные баттерфляи в~задачах оптимизации по CC-VaR&1&107--115\\
\Avtors{Аду~К.\,И.\,Б., Маркова~Е.\,В., Гайдамака~Ю.\,В., Шоргин~С.\,Я.} Анализ схемы 
доступа с~прерыванием при нарезке радиоресурсов сети пятого 
поколения&\raisebox{-12pt}[0pt][0pt]{1}&\raisebox{-12pt}[0pt][0pt]{\hphantom{1}96--106}\\
\Avtors{Архипов~П.\,О., Филиппских~С.\,Л., Цуканов~М.\,В.} Разработка новой модели 
ступенчатой сверточной нейронной сети для классификации аномалий на панорамах&\raisebox{-12pt}[0pt][0pt]{1}&\raisebox{-12pt}[0pt][0pt]{50--56}\\
\Avtors{Бегишев~В.\,О.} см.\ Сопин~Э.\,С.&&\\
\Avtors{Берговин~А.\,К., Ушаков~В.\,Г.} Исследование систем обслуживания со 
смешанными приоритетами&\raisebox{-12pt}[0pt][0pt]{2}&\raisebox{-12pt}[0pt][0pt]{57--61}\\
\Avtors{Борисов~А.\,В.} Рынок с~марковской скачкообразной волатильностью 
I:~мониторинг цены риска как задача оптимальной фильтрации&\raisebox{-12pt}[0pt][0pt]{2}&\raisebox{-12pt}[0pt][0pt]{27--33}\\
\Avtors{Борисов~А.\,В.} Рынок с~марковской скачкообразной волатильностью~II: алгоритм 
вы\-чис\-ле\-ния справедливой цены деривативов&\raisebox{-12pt}[0pt][0pt]{3}&\raisebox{-12pt}[0pt][0pt]{18--24}\\
\Avtors{Борисов А.\,В.} Рынок с марковской скачкообразной волатильностью III:  алгоритм 
мониторинга цены риска по дискретным наблюдениям цен активов&\raisebox{-12pt}[0pt][0pt]{4}&\raisebox{-12pt}[0pt][0pt]{\hphantom{9}9--16}\\
\Avtors{Босов~А.\,В.} Исследование робастности численных аппроксимаций фильтра 
Вонэма&2&41--49\\
\Avtors{Босов~А.\,В.} Оптимальная фильтрация состояния нелинейной динамической 
системы по наблюдениям со случайными запаздываниями&\raisebox{-12pt}[0pt][0pt]{3}&\raisebox{-12pt}[0pt][0pt]{\hphantom{1}8--17}\\
\Avtors{Босов~А.\,В., Иванов~А.\,В.} Технология многофакторной классификации 
математического контента электронной системы обучения&\raisebox{-12pt}[0pt][0pt]{4}&\raisebox{-12pt}[0pt][0pt]{32--41}\\
\Avtors{Босов~А.\,В., Игнатов~А.\,Н.} О~задаче оценки и~анализа риска транспортных 
происшествий на рельсовом транспорте&\raisebox{-12pt}[0pt][0pt]{1}&\raisebox{-12pt}[0pt][0pt]{73--82}\\
\Avtors{Вакуленко~В.\,В., Зацман~И.\,М.} Формализованное описание статистической 
обработки информации в~базах данных&\raisebox{-12pt}[0pt][0pt]{3}&\raisebox{-12pt}[0pt][0pt]{93--99}\\
\Avtors{Васильев~Н.\,С.} Композициональное представление структуры игры многих лиц 
в~моноидальной категории бинарных отношений&\raisebox{-12pt}[0pt][0pt]{2}&\raisebox{-12pt}[0pt][0pt]{18--26}\\
\Avtors{Волканов~Д.\,Ю.} см.\ Горшенин~А.\,К.&&\\
\Avtors{Воронцов~М.\,О., Шестаков~О.\,В.} Среднеквадратичный риск FDR-процедуры 
в~условиях слабой зависимости&\raisebox{-12pt}[0pt][0pt]{2}&\raisebox{-12pt}[0pt][0pt]{34--40}\\
\Avtors{Гайдамака~Ю.\,В.} см.\ Аду~К.\,И.\,Б.&&\\
\Avtors{Гайдамака~Ю.\,В.} см.\ Иванова Д.\,В.&&\\
\Avtors{Гайдамака~Ю.\,В.} см.\ Самуйлов~А.\,К.&&\\
\Avtors{Гаримелла~Р.\,М.} см.\ Разумчик~Р.\,В.&&\\
\Avtors{Гончаров~А.\,А.} Аннотирование параллельных корпусов: подходы и направления 
развития&4&81--87\\
\Avtors{Горбунов~С.\,А.} см.\ Горшенин~А.\,К.&&\\
\Avtors{Горшенин~А.\,К., Горбунов~С.\,А., Волканов~Д.\,Ю.} О~кластеризации объектов 
сетевой вы\-чис\-ли\-тель\-ной инфраструктуры на основе анализа статистических аномалий 
в~трафике&\raisebox{-12pt}[0pt][0pt]{3}&\raisebox{-12pt}[0pt][0pt]{76--87}\\
\Avtors{Грушо~А.\,А., Грушо~Н.\,А., Забежайло~М.\,И., Кульченков~В.\,В., 
Тимонина~Е.\,Е., Шоргин~С.\,Я.} Причинно-следственные связи в~задачах 
классификации&\raisebox{-12pt}[0pt][0pt]{1}&\raisebox{-12pt}[0pt][0pt]{43--49}\\
\Avtors{Грушо~А.\,А., Грушо~Н.\,А., Забежайло~М.\,И., Смирнов~Д.\,В., Тимонина~Е.\,Е.} 
Классификация с~помощью причинно-следственных связей&\raisebox{-12pt}[0pt][0pt]{3}&\raisebox{-12pt}[0pt][0pt]{71--75}\\
\Avtors{Грушо~А.\,А., Грушо~Н.\,А., Забежайло~М.\,И., Тимонина~Е.\,Е., Шоргин~С.\,Я.} 
Сложные причинно-следственные связи&\raisebox{-12pt}[0pt][0pt]{2}&\raisebox{-12pt}[0pt][0pt]{84--89}\\
\end{tabular}
}

\pagebreak

\def\leftkol{АВТОРСКИЙ УКАЗАТЕЛЬ ЗА 2023 г.} % ENGLISH ABSTRACTS}

\def\rightkol{АВТОРСКИЙ УКАЗАТЕЛЬ ЗА 2023 г.} %ENGLISH ABSTRACTS}

%\thispagestyle{myheadings}
\def\leftfootline{\small{\textbf{\thepage}
\hfill ИНФОРМАТИКА И ЕЁ ПРИМЕНЕНИЯ\ \ \ том~17\ \ \ выпуск~4\ \ \ 2023}
}%
 \def\rightfootline{\small{ИНФОРМАТИКА И ЕЁ ПРИМЕНЕНИЯ\ \ \ том~17\ \ \ выпуск~4\ \ \ 2023
 \hfill \textbf{\thepage}}}


\noindent
{\tabcolsep=3pt
\begin{tabular}{p{394pt}cc}
&\textbf{Вып.} & \textbf{Стр.}\\[3pt]
\Avtors{Грушо~Н.\,А.} см.\ Грушо~А.\,А.&&\\
\Avtors{Грушо~Н.\,А.} см.\ Грушо~А.\,А.&&\\
\Avtors{Грушо~Н.\,А.} см.\ Грушо~А.\,А.&&\\
\Avtors{Дулин~С.\,К.} см.\ Розенберг~И.\,Н.&&\\
\Avtors{Дулина~Н.\,Г.} см.\ Розенберг~И.\,Н.&&\\
\Avtors{Дюкова~А.\,П.} см.\ Дюкова~Е.\,В.&&\\
\Avtors{Дюкова~Е.\,В., Масляков~Г.\,О., Дюкова~А.\,П.} Логические методы корректной 
классификации данных&\raisebox{-12pt}[0pt][0pt]{3}&\raisebox{-12pt}[0pt][0pt]{64--70}\\
\Avtors{Забежайло~М.\,И.} см.\ Грушо~А.\,А&&\\
\Avtors{Забежайло~М.\,И.} см.\ Грушо~А.\,А.&&\\
\Avtors{Забежайло~М.\,И.} см.\ Грушо~А.\,А.&&\\
\Avtors{Захаров~В.\,Н.} см.\ Сазонтьев В.\,В.&&\\
\Avtors{Захаров В.\,Н.} см.\ Френкель С.\,Л.&&\\
\Avtors{Зацман~И.\,М.} Данные, информация и~знание в~научной парадигме 
информатики&1&116--125\\
\Avtors{Зацман И.\,М.} Научная парадигма информатики: классификация объектов 
предметной области&\raisebox{-12pt}[0pt][0pt]{4}&\raisebox{-12pt}[0pt][0pt]{\hphantom{9}96--103}\\
\Avtors{Зацман~И.\,М.} Трансформация иерархии Акоффа в~научной парадигме 
информатики&3&107--113\\
\Avtors{Зацман~И.\,М.} см.\ Вакуленко~В.\,В.&&\\
\Avtors{Зейфман~А.\,И.} см.\ Усов~И.\,А.&&\\
\Avtors{Иванов~А.\,В.} см.\ Босов~А.\,В.&&\\
\Avtors{Иванова Д.\,В., Маркова Е.\,В., Шоргин~С.\,Я., Гайдамака~Ю.\,В.} Модели 
совместного обслуживания трафика eMBB и URLLC на основе приоритетов в 
промышленных развертываниях 5G NR&\raisebox{-24pt}[0pt][0pt]{4}&\raisebox{-24pt}[0pt][0pt]{64--70}\\
\Avtors{Игнатов~А.\,Н.} см.\ Босов~А.\,В.&&\\
\Avtors{Инькова~О.\,Ю., Кружков~М.\,Г.} Критерии определения семантической близости 
дискурсивных отношений&\raisebox{-12pt}[0pt][0pt]{3}&\raisebox{-12pt}[0pt][0pt]{100--106}\\
\Avtors{Инькова О.\,Ю., Кружков~М.\,Г.} Степень семантической близости дискурсивных 
отношений:  методы и инструменты расчета&\raisebox{-12pt}[0pt][0pt]{4}&\raisebox{-12pt}[0pt][0pt]{88--95}\\
\Avtors{Кабанов~Ю.\,М., Сидоренко~А.\,П.} Аксиоматический взгляд на модели системного 
риска Роджерса--Вераарт и~Судзуки--Эльсингера&\raisebox{-12pt}[0pt][0pt]{1}&\raisebox{-12pt}[0pt][0pt]{11--17}\\
\Avtors{Карпов~В.\,И.} см.\ Нуриев~В.\,А.&&\\
\Avtors{Кириков~И.\,А.} см.\ Листопад~С.\,В.&&\\
\Avtors{Ковалёв~С.\,П.} Монада диаграмм как математическая метамодель системной 
инженерии&2&11--17\\
\Avtors{Королев~Д.\,О., Малеев~О.\,Г.} Исследование эффективности применения бинарных 
нейронных сетей при детектировании объекта на изображении&\raisebox{-12pt}[0pt][0pt]{3}&\raisebox{-12pt}[0pt][0pt]{88--92}\\
\Avtors{Кривенко~М.\,П.} Критерии выбора размерности модели факторизации&2&50--56\\
\Avtors{Кружков~М.\,Г.} см.\ Инькова О.\,Ю.&&\\
\Avtors{Кружков~М.\,Г.} см.\ Инькова~О.\,Ю.&&\\
\Avtors{Кудрявцев~А.\,А., Шестаков~О.\,В.} Метод оценивания параметров 
гамма-экс\-по\-нен\-ци\-аль\-но\-го распределения по выборке со слабо зависимыми компонентами&\raisebox{-12pt}[0pt][0pt]{3}&\raisebox{-12pt}[0pt][0pt]{58--62}\\
\Avtors{Кульченков~В.\,В.} см.\ Грушо~А.\,А.&&\\
\Avtors{Лапко~А.\,В.} см.\ Тубольцев~В.\,П.&&\\
\Avtors{Лапко~В.\,А.} см.\ Тубольцев~В.\,П.&&\\
\Avtors{Лери~М.\,М.} Среднее расстояние в~конфигурационных графах со степенным 
распределением&\raisebox{-12pt}[0pt][0pt]{1}&\raisebox{-12pt}[0pt][0pt]{28--34}\\
\Avtors{Листопад~С.\,В., Кириков~И.\,А.} Метод на основе нечетких правил для 
управления конфликтами агентов в~гибридных интеллектуальных многоагентных 
системах&\raisebox{-12pt}[0pt][0pt]{1}&\raisebox{-12pt}[0pt][0pt]{66--72}\\
\Avtors{Малашенко~Ю.\,Е., Назарова~И.\,А.} Анализ загрузки многопользовательской сети 
при расщеплении потоков по кратчайшим маршрутам&\raisebox{-12pt}[0pt][0pt]{3}&\raisebox{-12pt}[0pt][0pt]{33--38}\\
\Avtors{Малашенко~Ю.\,Е., Назарова~И.\,А.} Оценки распределения ресурсов 
в~многопользовательской сети при равных межузловых нагрузках&\raisebox{-12pt}[0pt][0pt]{1}&\raisebox{-12pt}[0pt][0pt]{83--88}\\
\Avtors{Малеев~О.\,Г.} см.\ Королев~Д.\,О.&&\\
\Avtors{Маркова~Е.\,В.} см.\ Аду~К.\,И.\,Б.&&\\
\Avtors{Маркова Е.\,В.} см.\ Иванова Д.\,В.&&\\
\end{tabular}
}

\pagebreak

\def\leftkol{АВТОРСКИЙ УКАЗАТЕЛЬ ЗА 2023 г.} % ENGLISH ABSTRACTS}

\def\rightkol{АВТОРСКИЙ УКАЗАТЕЛЬ ЗА 2023 г.} %ENGLISH ABSTRACTS}

%\thispagestyle{myheadings}
\def\leftfootline{\small{\textbf{\thepage}
\hfill ИНФОРМАТИКА И ЕЁ ПРИМЕНЕНИЯ\ \ \ том~17\ \ \ выпуск~4\ \ \ 2023}
}%
 \def\rightfootline{\small{ИНФОРМАТИКА И ЕЁ ПРИМЕНЕНИЯ\ \ \ том~17\ \ \ выпуск~4\ \ \ 2023
 \hfill \textbf{\thepage}}}


\noindent
{\tabcolsep=3pt
\begin{tabular}{p{394pt}cc}
&\textbf{Вып.} & \textbf{Стр.}\\[3pt]
\Avtors{Маслов~А.\,Р.} см.\ Сопин~Э.\,С&&\\
\Avtors{Масляков~Г.\,О.} см.\ Дюкова~Е.\,В.&&\\
\Avtors{Мелехин~В.\,Б., Хачумов~В.\,М., Хачумов~М.\,В.} Самообучение автономных 
интеллектуальных роботов в~процессе поисково-исследовательской деятельности&\raisebox{-12pt}[0pt][0pt]{2}&\raisebox{-12pt}[0pt][0pt]{78--83}\\
\Avtors{Назарова~И.\,А.} см.\ Малашенко~Ю.\,Е.&&\\
\Avtors{Назарова~И.\,А.} см.\ Малашенко~Ю.\,Е.&&\\
\Avtors{Нейчев~Р.\,Г., Шибаев~И.\,А., Стрижов~В.\,В.} Восстановление матрицы 
суперпозиции в~задаче символьной регрессии&\raisebox{-12pt}[0pt][0pt]{1}&\raisebox{-12pt}[0pt][0pt]{35--42}\\
\Avtors{Нуриев~В.\,А., Карпов~В.\,И.} Методология корпусно-ориентированного 
исследования в~области контрастивной пунктуации&\raisebox{-12pt}[0pt][0pt]{2}&\raisebox{-12pt}[0pt][0pt]{90--95}\\
\Avtors{Пешкова И.\,В.} Границы незавершенной работы в системе с повторными вызовами 
разных классов и показательным временем обслуживания&\raisebox{-12pt}[0pt][0pt]{4}&\raisebox{-12pt}[0pt][0pt]{57--63}\\
\Avtors{Платонова~А.\,А.} см.\ Самуйлов~А.\,К.&&\\
\Avtors{Рабинович Я.\,И.} Процедура построения множества Парето для дифференцируемых 
критериальных функций&\raisebox{-12pt}[0pt][0pt]{4}&\raisebox{-12pt}[0pt][0pt]{17--22}\\
\Avtors{Разумчик~Р.\,В., Румянцев~А.\,С., Гаримелла~Р.\,М.} Вероятностная модель для 
оценки основных характеристик производительности марковской модели 
суперкомпьютера&\raisebox{-24pt}[0pt][0pt]{2}&\raisebox{-24pt}[0pt][0pt]{62--70}\\
\Avtors{Розенберг~И.\,Н., Дулин~С.\,К., Дулина~Н.\,Г.} Моделирование структуры 
интероперабельности средствами структурной согласованности&\raisebox{-12pt}[0pt][0pt]{1}&\raisebox{-12pt}[0pt][0pt]{57--65}\\
\Avtors{Румовская~С.\,Б.} Подходы к~подбору специалистов при организации 
коллективного решения проблем&\raisebox{-12pt}[0pt][0pt]{2}&\raisebox{-12pt}[0pt][0pt]{\hphantom{1}96--103}\\
\Avtors{Румянцев~А.\,С.} см.\ Разумчик~Р.\,В.&&\\
\Avtors{Сазонтьев В.\,В., Ступников~С.\,А., Захаров~В.\,Н.} Расширяемый подход к слиянию 
данных в распределенных вычислительных средах&\raisebox{-12pt}[0pt][0pt]{4}&\raisebox{-12pt}[0pt][0pt]{42--47}\\
\Avtors{Самуйлов~А.\,К., Платонова~А.\,А., Шоргин~В.\,С., Гайдамака~Ю.\,В.} 
К~моделированию эффектов обслуживания многоадресного трафика в~сетях 5G~NR&\raisebox{-12pt}[0pt][0pt]{2}&\raisebox{-12pt}[0pt][0pt]{71--77}\\
\Avtors{Сатин~Я.\,А.} см.\ Усов~И.\,А.&&\\
\Avtors{Сидоренко~А.\,П.} см.\ Кабанов~Ю.\,М.&&\\
\Avtors{Синицын~И.\,Н.} Аналитическое моделирование распределений с~инвариантной 
мерой в~стохастических системах, не разрешенных относительно 
производных&\raisebox{-12pt}[0pt][0pt]{1}&\raisebox{-12pt}[0pt][0pt]{\hphantom{1}2--10}\\
\Avtors{Смирнов~Д.\,В.} см.\ Грушо~А.\,А.&&\\
\Avtors{Сопин~Э.\,С., Маслов~А.\,Р., Шоргин~В.\,С., Бегишев~В.\,О.} Моделирование 
настойчивого поведения пользователей в~сетях 5G NR с~адаптацией скорости 
и~блокировками&\raisebox{-12pt}[0pt][0pt]{3}&\raisebox{-12pt}[0pt][0pt]{25--32}\\
\Avtors{Степанов~Е.\,П.} см.\ Шестаков~О.\,В.&&\\
\Avtors{Стрижов~В.\,В.} см.\ Нейчев~Р.\,Г.&&\\
\Avtors{Ступников~С.\,А.} см.\ Сазонтьев В.\,В.&&\\
\Avtors{Тимонина~Е.\,Е.} см.\ Грушо~А.\,А.&&\\
\Avtors{Тимонина~Е.\,Е.} см.\ Грушо~А.\,А.&&\\
\Avtors{Тимонина~Е.\,Е.} см.\ Грушо~А.\,А.&&\\
\Avtors{Торшин~И.\,Ю.} О~задачах оптимизации, возникающих при применении 
топологического анализа данных к~поиску алгоритмов прогнозирования с~фиксированными 
корректорами&\raisebox{-24pt}[0pt][0pt]{2}&\raisebox{-24pt}[0pt][0pt]{\hphantom{1}2--10}\\
\Avtors{Торшин~И.\,Ю.} О~формировании множеств прецедентов на основе таблиц 
разнородных признаковых описаний методами топологической теории анализа 
данных&\raisebox{-12pt}[0pt][0pt]{3}&\raisebox{-12pt}[0pt][0pt]{2--7}\\
\Avtors{Тубольцев~В.\,П., Лапко~А.\,В., Лапко~В.\,А.} Непараметрический алгоритм 
автоматической классификации данных дистанционного зондирования&\raisebox{-12pt}[0pt][0pt]{4}&\raisebox{-12pt}[0pt][0pt]{23--31}\\
\Avtors{Усов~И.\,А., Сатин~Я.\,А., Зейфман~А.\,И.} О~скорости сходимости и~предельных 
характеристиках для одного обобщенного процесса рождения и~гибели&\raisebox{-12pt}[0pt][0pt]{3}&\raisebox{-12pt}[0pt][0pt]{49--57}\\
\Avtors{Ушаков~В.\,Г., Ушаков~Н.\,Г.} Критерии нормальности вероятностного 
распределения при округленных данных&\raisebox{-12pt}[0pt][0pt]{1}&\raisebox{-12pt}[0pt][0pt]{18--27}\\
\Avtors{Ушаков~В.\,Г.} см.\ Берговин~А.\,К.&&\\
\Avtors{Ушаков~Н.\,Г.} см.\ Ушаков~В.\,Г.&&\\
\Avtors{Филиппских~С.\,Л.} см.\ Архипов~П.\,О.&&\\
\end{tabular}
}

\pagebreak

\def\leftkol{АВТОРСКИЙ УКАЗАТЕЛЬ ЗА 2023 г.} % ENGLISH ABSTRACTS}

\def\rightkol{АВТОРСКИЙ УКАЗАТЕЛЬ ЗА 2023 г.} %ENGLISH ABSTRACTS}

%\thispagestyle{myheadings}
\def\leftfootline{\small{\textbf{\thepage}
\hfill ИНФОРМАТИКА И ЕЁ ПРИМЕНЕНИЯ\ \ \ том~17\ \ \ выпуск~4\ \ \ 2023}
}%
 \def\rightfootline{\small{ИНФОРМАТИКА И ЕЁ ПРИМЕНЕНИЯ\ \ \ том~17\ \ \ выпуск~4\ \ \ 2023
 \hfill \textbf{\thepage}}}


\noindent
{\tabcolsep=3pt
\begin{tabular}{p{394pt}cc}
&\textbf{Вып.} & \textbf{Стр.}\\[3pt]
\Avtors{Френкель С.\,Л., Захаров В.\,Н.} Модели учета влияния статистических 
характеристик трафика вычислительных сетей на эффективность прогнозирования 
средствами машинного обучения&\raisebox{-24pt}[0pt][0pt]{4}&\raisebox{-24pt}[0pt][0pt]{71--80}\\
\Avtors{Хачумов~В.\,М.} см.\ Мелехин~В.\,Б.&&\\
\Avtors{Хачумов~М.\,В.} см.\ Мелехин~В.\,Б.&&\\
\Avtors{Цуканов~М.\,В.} см.\ Архипов~П.\,О.&&\\
\Avtors{Шестаков~О.\,В., Степанов~Е.\,П.} Нелинейная регуляризация обращения линейных 
однородных операторов с помощью метода блочной пороговой обработки&\raisebox{-12pt}[0pt][0pt]{4}&\raisebox{-12pt}[0pt][0pt]{2--8}\\
\Avtors{Шестаков~О.\,В.} см.\ Воронцов~М.\,О.&&\\
\Avtors{Шестаков~О.\,В.} см.\ Кудрявцев~А.\,А.&&\\
\Avtors{Шибаев~И.\,А.} см.\ Нейчев~Р.\,Г.&&\\
\Avtors{Шнурков П.\,В.} Решение задачи оптимального управления запасом непрерывного 
продукта в стохастической модели регенерации со случайными стоимостными 
характеристиками&\raisebox{-24pt}[0pt][0pt]{4}&\raisebox{-24pt}[0pt][0pt]{48--56}\\
\Avtors{Шоргин~В.\,С.} см.\ Самуйлов~А.\,К.&&\\
\Avtors{Шоргин~В.\,С.} см.\ Сопин~Э.\,С.&&\\
\Avtors{Шоргин~С.\,Я.} см.\ Аду~К.\,И.\,Б.&&\\
\Avtors{Шоргин~С.\,Я.} см.\ Грушо~А.\,А.&&\\
\Avtors{Шоргин~С.\,Я.} см.\ Грушо~А.\,А.&&\\
\Avtors{Шоргин~С.\,Я.} см.\ Иванова Д.\,В.&&\\


\end{tabular}
}

%\thispagestyle{myheadings}
\def\leftfootline{\small{\textbf{\thepage}
\hfill ИНФОРМАТИКА И ЕЁ ПРИМЕНЕНИЯ\ \ \ том~17\ \ \ выпуск~4\ \ \ 2023}
}%
 \def\rightfootline{\small{ИНФОРМАТИКА И ЕЁ ПРИМЕНЕНИЯ\ \ \ том~17\ \ \ выпуск~4\ \ \ 2023
 \hfill \textbf{\thepage}}}

 \label{end\stat}

\newpage

\def\stat{cont-e}
{%\hrule\par
%\vskip 7pt % 7pt
\raggedleft\Large \bf%\baselineskip=3.2ex
2\,0\,2\,3\ \ A\,U\,T\,H\,O\,R\ \ I\,N\,D\,E\,X \vskip 17pt
 \hrule
 \par
\vskip 21pt plus 6pt minus 3pt }

\label{st\stat}

\def\tit{\ }

\def\aut{\ }
\def\auf{\ }

\def\leftkol{\ } %2021 AUTHOR INDEX} % ENGLISH ABSTRACTS}

\def\rightkol{\ } %2021 AUTHOR INDEX} %ENGLISH ABSTRACTS}

\titele{\tit}{\aut}{\auf}{\leftkol}{\rightkol}
\addcontentsline{toc}{subsection}{\textrm\textbf 2023 Author Index}

\def\leftfootline{\small{\textbf{\thepage}
\hfill INFORMATIKA I EE PRIMENENIYA~--- INFORMATICS AND APPLICATIONS\ \ \ 2023\
\ \ volume~17\ \ \ issue\ 4}
}%
 \def\rightfootline{\small{INFORMATIKA I EE PRIMENENIYA~--- INFORMATICS AND APPLICATIONS\ \ \ 2023\ \ \ volume~17\ \ \ issue\ 4
\hfill \textbf{\thepage}}}

\vspace*{-24pt}

\noindent
{\tabcolsep=3pt
\begin{tabular}{p{395.89pt}cc}
&\textbf{Issue} & \textbf{Page}\\[6pt]
\Avtors{Adou~K.\,Y.\,B., Markova~E.\,V., Gaidamaka~Yu.\,V., and~Shorgin~S.\,Ya.} 
Preemption-based prioritization scheme for network resources slicing in 5G 
systems&\raisebox{-12pt}[0pt][0pt]{1}&\raisebox{-12pt}[0pt][0pt]{\hphantom{1}96--106}\\
\Avtors{Agalarov~Ya.\,M.} Optimization of a queue-length dependent additional server in the 
multiserver queue&\raisebox{-12pt}[0pt][0pt]{1}&\raisebox{-12pt}[0pt][0pt]{89--95}\\
\Avtors{Agalarov~Ya.\,M.} Optimization of the buffer memory allocation scheme of the packet 
switching node&\raisebox{-12pt}[0pt][0pt]{3}&\raisebox{-12pt}[0pt][0pt]{39--48}\\
\Avtors{Agasandyan~G.\,A.} Multidimensional butterflies in problems of optimization on CC-VaR&1&107--115\\
\Avtors{Arkhipov~P.\,O., Philippskih~S.\,L., and~Tsukanov~M.\,V.} Development of a~new model 
of a~step convolutional neural network for classification of anomalies on panoramas&\raisebox{-12pt}[0pt][0pt]{1}&\raisebox{-12pt}[0pt][0pt]{50--56}\\
\Avtors{Begishev~V.\,O.} see Sopin~E.\,S.&&\\
\Avtors{Bergovin~A.\,K. and~Ushakov~V.\,G.} Analysis of the queueing systems with mixed 
priorities&2&57--61\\
\Avtors{Borisov~A.\,V.} Market with Markov jump volatility I: Price of risk monitoring as an 
optimal filtering problem&\raisebox{-12pt}[0pt][0pt]{2}&\raisebox{-12pt}[0pt][0pt]{27--33}\\
\Avtors{Borisov~A.\,V.} Market with Markov jump volatility~II: Algorithm of derivative fair price 
calculation&3&18--24\\
\Avtors{Borisov A.\,V.} Market with Markov jump volatility III: Price of risk monitoring algorithm 
given discrete-time observations of asset prices&\raisebox{-12pt}[0pt][0pt]{4}&\raisebox{-12pt}[0pt][0pt]{\hphantom{9}9--16}\\
\Avtors{Bosov~A.\,V.} Nonlinear dynamic system state optimal filtering by observations with 
random delays&\raisebox{-12pt}[0pt][0pt]{3}&\raisebox{-12pt}[0pt][0pt]{\hphantom{1}8--17}\\
\Avtors{Bosov~A.\,V.} Robustness investigation of the numerical approximation of the Wonham 
filter&2&41--49\\
\Avtors{Bosov~A.\,V. and~Ignatov~A.\,N.} On the problem of assessing and analyzing traffic 
accidents risk on the rail transport&\raisebox{-12pt}[0pt][0pt]{1}&\raisebox{-12pt}[0pt][0pt]{73--82}\\
\Avtors{Bosov~A.\,V. and Ivanov~A.\,V.} Multifactor classification technology of mathematical 
content of e-learning system&\raisebox{-12pt}[0pt][0pt]{4}&\raisebox{-12pt}[0pt][0pt]{32--41}\\
\Avtors{Djukova~A.\,P.} see Djukova~E.\,V.&&\\
\Avtors{Djukova~E.\,V., Masliakov~G.\,O., and Djukova~A.\,P.} Logical methods of correct data 
classification&3&64--70\\
\Avtors{Dulin~S.\,K.} see Rozenberg~I.\,N.&&\\
\Avtors{Dulina~N.\,G.} see Rozenberg~I.\,N.&&\\
\Avtors{Frenkel~S.\,L. and Zakharov~V.\,N.} Models for study of the influence of statistical 
characteristics of computer networks traffic on the efficiency of prediction by machine learning 
tools&\raisebox{-12pt}[0pt][0pt]{4}&\raisebox{-12pt}[0pt][0pt]{71--80}\\
\Avtors{Gaidamaka~Yu.\,V.} see Adou~K.\,Y.\,B.&&\\
\Avtors{Gaidamaka~Yu.\,V.} see Ivanova~D.\,V.&&\\
\Avtors{Gaidamaka~Yu.\,V.} see Samouylov~A.\,K.&&\\
\Avtors{Garimella~R.\,M.} see Razumchik~R.\,V.&&\\
\Avtors{Goncharov~A.\,A.} Parallel corpus annotation: Approaches and directions for 
development&4&81--87\\
\Avtors{Gorbunov~S.\,A.} see Gorshenin~A.\,K.&&\\
\Avtors{Gorshenin~A.\,K., Gorbunov~S.\,A., and Volkanov~D.\,Yu.} Toward clustering of 
network computing infrastructure objects based on analysis of statistical anomalies in network 
traffic&\raisebox{-12pt}[0pt][0pt]{3}&\raisebox{-12pt}[0pt][0pt]{76--87}\\
\Avtors{Grusho~A.\,A., Grusho~N.\,A., Zabezhailo~M.\,I., Kulchenkov~V.\,V., Timonina~E.\,E., 
and~Shorgin~S.\,Ya.} Causal relationships in classification problems&\raisebox{-12pt}[0pt][0pt]{1}&\raisebox{-12pt}[0pt][0pt]{43--49}\\
\Avtors{Grusho~A.\,A., Grusho~N.\,A., Zabezhailo~M.\,I., Smirnov~D.\,V., and Timonina~E.\,E.} 
Classification by cause-and-effect relationships&\raisebox{-12pt}[0pt][0pt]{3}&\raisebox{-12pt}[0pt][0pt]{71--75}\\
\Avtors{Grusho~A.\,A., Grusho~N.\,A., Zabezhailo~M.\,I., Timonina~E.\,E., 
and~Shorgin~S.\,Ya.} Complex cause-and-effect relationships&\raisebox{-12pt}[0pt][0pt]{2}&\raisebox{-12pt}[0pt][0pt]{84--89}\\
\Avtors{Grusho~N.\,A.} see Grusho~A.\,A.&&\\
\Avtors{Grusho~N.\,A.} see Grusho~A.\,A.&&\\
\Avtors{Grusho~N.\,A.} see Grusho~A.\,A.&&\\
\end{tabular}
}
\pagebreak

\def\leftfootline{\small{\textbf{\thepage}
\hfill INFORMATIKA I EE PRIMENENIYA~--- INFORMATICS AND APPLICATIONS\ \ \ 2023\
\ \ volume~17\ \ \ issue\ 4}
}%
 \def\rightfootline{\small{INFORMATIKA I EE PRIMENENIYA~---
INFORMATICS AND APPLICATIONS\ \ \ 2023\ \ \ volume~17\ \ \ issue\ 4
\hfill \textbf{\thepage}}}

\def\leftkol{2023 AUTHOR INDEX} % ENGLISH ABSTRACTS}

\def\rightkol{2023 AUTHOR INDEX} %ENGLISH ABSTRACTS}


\noindent
{\tabcolsep=3pt
\begin{tabular}{p{395.5pt}cc}
&\textbf{Issue} & \textbf{Page}\\[6pt]
\Avtors{Ignatov~A.\,N.} see Bosov~A.\,V.&&\\
\Avtors{Inkova O.\,Yu. and Kruzhkov M.\,G.} Evaluating the degree of discourse relations 
semantic affinity: Methods and instruments&\raisebox{-12pt}[0pt][0pt]{4}&\raisebox{-12pt}[0pt][0pt]{88--95}\\
\Avtors{Inkova~O.\,Yu. and Kruzhkov~M.\,G.} Evaluation criteria for discourse relations semantic 
affinity&3&100--106\\
\Avtors{Kruzhkov~M.\,G.} see Inkova~O.\,Yu.&&\\
\Avtors{Ivanov~A.\,V.} see Bosov~A.\,V.&&\\
\Avtors{Ivanova~D.\,V., Markova~E.\,V., Shorgin~S.\,Ya., and~Gaidamaka~Yu.\,V.} Priority-based 
eMBB and URLLC traffic coexistence models in 5G NR industrial deployments&\raisebox{-12pt}[0pt][0pt]{4}&\raisebox{-12pt}[0pt][0pt]{64--70}\\
\Avtors{Kabanov~Yu.\,M. and~Sidorenko~A.\,P.} An axiomatic viewpoint on the Rogers--Veraart 
and Suzuki--Elsinger models of systemic risk&\raisebox{-12pt}[0pt][0pt]{1}&\raisebox{-12pt}[0pt][0pt]{11--17}\\
\Avtors{Karpov~V.\,I.} see Nuriev~V.\,A.&&\\
\Avtors{Khachumov~M.\,V.} see Melekhin~V.\,B.&&\\
\Avtors{Khachumov~V.\,M.} see Melekhin~V.\,B.&&\\
\Avtors{Kirikov~I.\,A.} see Listopad~S.\,V.&&\\
\Avtors{Korolev~D.\,O. and Maleev~O.\,G.} Efficiency of binary neural networks for object 
detection on an image&\raisebox{-12pt}[0pt][0pt]{3}&\raisebox{-12pt}[0pt][0pt]{88--92}\\
\Avtors{Kovalyov~S.\,P.} The monad of diagrams as a mathematical metamodel of systems 
engineering&2&11--17\\
\Avtors{Krivenko~M.\,P.} Criteria for choosing the factorization model dimensionality&2&50--56\\
\Avtors{Kruzhkov M.\,G.} see Inkova O.\,Yu.&&\\
\Avtors{Kudryavtsev~A.\,A. and Shestakov~O.\,V.} A~method for estimating parameters of the 
gamma-exponential distribution from a~sample with weakly dependent components&\raisebox{-12pt}[0pt][0pt]{3}&\raisebox{-12pt}[0pt][0pt]{58--63}\\
\Avtors{Kulchenkov~V.\,V.} see Grusho~A.\,A.&&\\
\Avtors{Lapko~A.\,V.} see Tuboltsev V.\,P.&&\\
\Avtors{Lapko~V.\,A.} see Tuboltsev V.\,P.&&\\
\Avtors{Leri~M.\,M.} An average distance in the power-law configuration graphs&1&28--34\\
\Avtors{Listopad~S.\,V. and~Kirikov~I.\,A.} Fuzzy rules based method for agent conflict 
management in hybrid intelligent multiagent systems&\raisebox{-12pt}[0pt][0pt]{1}&\raisebox{-12pt}[0pt][0pt]{66--72}\\
\Avtors{Malashenko~Yu.\,E. and~Nazarova~I.\,A.} Estimates of the resource distribution in the 
multiuser network with equal internodal loads&\raisebox{-12pt}[0pt][0pt]{1}&\raisebox{-12pt}[0pt][0pt]{83--88}\\
\Avtors{Malashenko~Yu.\,E. and Nazarova~I.\,A.} Multiuser network load analysis by splitting 
flows along the shortest routes&\raisebox{-12pt}[0pt][0pt]{3}&\raisebox{-12pt}[0pt][0pt]{33--38}\\
\Avtors{Maleev~O.\,G.} see Korolev~D.\,O.&&\\
\Avtors{Markova~E.\,V.} see Adou~K.\,Y.\,B.&&\\
\Avtors{Markova~E.\,V.} see Ivanova~D.\,V.&&\\
\Avtors{Masliakov~G.\,O.} see Djukova~E.\,V.&&\\
\Avtors{Maslov~A.\,R.} see Sopin~E.\,S.&&\\
\Avtors{Melekhin~V.\,B., Khachumov~V.\,M., and~Khachumov~M.\,V.} Self-learning of 
autonomous intelligent robots in the process of search and explore activities&\raisebox{-12pt}[0pt][0pt]{2}&\raisebox{-12pt}[0pt][0pt]{78--83}\\
\Avtors{Nazarova~I.\,A.} see Malashenko~Yu.\,E.&&\\
\Avtors{Nazarova~I.\,A.} see Malashenko~Yu.\,E.&&\\
\Avtors{Neychev~R.\,G., Shibaev~I.\,A., and~Strijov~V.\,V.} Optimal spanning tree reconstruction 
in symbolic regression&\raisebox{-12pt}[0pt][0pt]{1}&\raisebox{-12pt}[0pt][0pt]{35--42}\\
\Avtors{Nuriev~V.\,A. and~Karpov~V.\,I.} Methodology of the corpus-based studies in the field of 
contrastive punctuation&\raisebox{-12pt}[0pt][0pt]{2}&\raisebox{-12pt}[0pt][0pt]{90--95}\\
\Avtors{Peshkova~I.\,V.} Bounds of the workload in a~multiclass retrial queue with exponential 
services&\raisebox{-12pt}[0pt][0pt]{4}&\raisebox{-12pt}[0pt][0pt]{57--63}\\
\Avtors{Philippskih~S.\,L.} see Arkhipov~P.\,O.&&\\
\Avtors{Platonova~A.\,A.} see Samouylov~A.\,K.&&\\
\Avtors{Rabinovich Ya.\,I.} Procedure of constructing a~Pareto set for differentiable criteria 
functions&4&17--22\\
\Avtors{Razumchik~R.\,V., Rumyantsev~A.\,S., and~Garimella~R.\,M.} A~queueing system for 
performance evaluation of a~Markovian supercomputer model&\raisebox{-12pt}[0pt][0pt]{2}&\raisebox{-12pt}[0pt][0pt]{62--70}\\
\Avtors{Rozenberg~I.\,N., Dulin~S.\,K., and~Dulina~N.\,G.} Modeling the structure of 
interoperability by means of structural consistency&\raisebox{-12pt}[0pt][0pt]{1}&\raisebox{-12pt}[0pt][0pt]{57--65}\\
\Avtors{Rumovskaya~S.\,B.} Selection of specialists in the organization of collective solving 
problems&2&\hphantom{1}96--103\\
\end{tabular}
}
\pagebreak

\def\leftfootline{\small{\textbf{\thepage}
\hfill INFORMATIKA I EE PRIMENENIYA~--- INFORMATICS AND APPLICATIONS\ \ \ 2023\
\ \ volume~17\ \ \ issue\ 4}
}%
 \def\rightfootline{\small{INFORMATIKA I EE PRIMENENIYA~---
INFORMATICS AND APPLICATIONS\ \ \ 2023\ \ \ volume~17\ \ \ issue\ 4
\hfill \textbf{\thepage}}}

\def\leftkol{2023 AUTHOR INDEX} % ENGLISH ABSTRACTS}

\def\rightkol{2023 AUTHOR INDEX} %ENGLISH ABSTRACTS}


\noindent
{\tabcolsep=3pt
\begin{tabular}{p{395.5pt}cc}
&\textbf{Issue} & \textbf{Page}\\[6pt]
\Avtors{Rumyantsev~A.\,S.} see Razumchik~R.\,V.&&\\
\Avtors{Samouylov~A.\,K., Platonova~A.\,A., Shorgin~V.\,S., and~Gaidamaka~Yu.\,V.} On 
modeling the effects of multicast traffic servicing in 5G NR networks&\raisebox{-12pt}[0pt][0pt]{2}&\raisebox{-12pt}[0pt][0pt]{71--77}\\
\Avtors{Satin~Y.\,A.} see Usov~I.\,A.&&\\
\Avtors{Sazontev V.\,V., Stupnikov~S.\,A., and~Zakharov~V.\,N.} An extensible approach to data 
fusion in~distributed computing environments&\raisebox{-12pt}[0pt][0pt]{4}&\raisebox{-12pt}[0pt][0pt]{42--47}\\
\Avtors{Shestakov~O.\,V. and Stepanov~E.\,P.} Nonlinear regularization of the inversion of linear 
homogeneous operators using the block thresholding method&\raisebox{-12pt}[0pt][0pt]{4}&\raisebox{-12pt}[0pt][0pt]{2--8}\\
\Avtors{Shestakov~O.\,V.} see Kudryavtsev~A.\,A.&&\\
\Avtors{Shestakov~O.\,V.} see Vorontsov~M.\,O.&&\\
\Avtors{Shibaev~I.\,A.} see Neychev~R.\,G.&&\\
\Avtors{Shnurkov P.\,V.} Solution of the problem of~optimal control of~the stock of a~continuous 
product in a~stochastic model of regeneration with random cost characteristics&\raisebox{-12pt}[0pt][0pt]{4}&\raisebox{-12pt}[0pt][0pt]{48--56}\\
\Avtors{Shorgin~S.\,Ya.} see Adou~K.\,Y.\,B.&&\\
\Avtors{Shorgin~S.\,Ya.} see Grusho~A.\,A.&&\\
\Avtors{Shorgin~S.\,Ya.} see Grusho~A.\,A.&&\\
\Avtors{Shorgin~S.\,Ya.} see Ivanova~D.\,V.&&\\
\Avtors{Shorgin~V.\,S.} see Samouylov~A.\,K.&&\\
\Avtors{Shorgin~V.\,S.} see Sopin~E.\,S.&&\\
\Avtors{Sidorenko~A.\,P.} see Kabanov~Yu.\,M.&&\\
\Avtors{Sinitsyn~I.\,N.} Analytical modeling of distributions with invariant measure in stochastic 
systems with unsolved derivatives&\raisebox{-12pt}[0pt][0pt]{1}&\raisebox{-12pt}[0pt][0pt]{\hphantom{1}2--10}\\
\Avtors{Smirnov~D.\,V.} see Grusho~A.\,A.&&\\
\Avtors{Sopin~E.\,S., Maslov~A.\,R., Shorgin~V.\,S., and Begishev~V.\,O.} Modeling insistent 
user behavior in~5G New Radio networks with rate adaptation and blockage&\raisebox{-12pt}[0pt][0pt]{3}&\raisebox{-12pt}[0pt][0pt]{25--32}\\
\Avtors{Stepanov~E.\,P.} see Shestakov~O.\,V.&&\\
\Avtors{Strijov~V.\,V.} see Neychev~R.\,G.&&\\
\Avtors{Stupnikov~S.\,A.} see Sazontev V.\,V.&&\\
\Avtors{Timonina~E.\,E.} see Grusho~A.\,A.&&\\
\Avtors{Timonina~E.\,E.} see Grusho~A.\,A.&&\\
\Avtors{Timonina~E.\,E.} see Grusho~A.\,A.&&\\
\Avtors{Torshin~I.\,Yu.} On optimization problems arising from the application of topological data 
analysis to the search for forecasting algorithms with fixed correctors&\raisebox{-12pt}[0pt][0pt]{2}&\raisebox{-12pt}[0pt][0pt]{\hphantom{1}2--10}\\
\Avtors{Torshin~I.\,Yu.} On the formation of sets of precedents based on tables of heterogeneous 
feature descriptions by methods of topological theory of data analysis&\raisebox{-12pt}[0pt][0pt]{3}&\raisebox{-12pt}[0pt][0pt]{2--7}\\
\Avtors{Tsukanov~M.\,V.} see Arkhipov~P.\,O.&&\\
\Avtors{Tuboltsev V.\,P., Lapko~A.\,V., and~Lapko~V.\,A.} Nonparametric algorithm for 
automatic classification of remote sensing data&\raisebox{-12pt}[0pt][0pt]{4}&\raisebox{-12pt}[0pt][0pt]{23--31}\\
\Avtors{Ushakov~N.\,G.} see Ushakov~V.\,G.&&\\
\Avtors{Ushakov~V.\,G. and~Ushakov~N.\,G.} Tests for normality of the probabilistic distribution 
when data are rounded&\raisebox{-12pt}[0pt][0pt]{1}&\raisebox{-12pt}[0pt][0pt]{18--27}\\
\Avtors{Ushakov~V.\,G.} see Bergovin~A.\,K.&&\\
\Avtors{Usov~I.\,A., Satin~Y.\,A., and Zeifman~A.\,I.} On the rate of convergence and limiting 
characteristics for one quasi-birth--death process&\raisebox{-12pt}[0pt][0pt]{3}&\raisebox{-12pt}[0pt][0pt]{49--57}\\
\Avtors{Vakulenko~V.\,V. and Zatsman~I.\,M.} Formalized description of statistical information 
processing in databases&\raisebox{-12pt}[0pt][0pt]{3}&\raisebox{-12pt}[0pt][0pt]{93--99}\\
\Avtors{Vasilyev~N.\,S.} Multiplayers' games compositional structure in the monoidal category of 
binary relations&\raisebox{-12pt}[0pt][0pt]{2}&\raisebox{-12pt}[0pt][0pt]{18--26}\\
\Avtors{Volkanov~D.\,Yu.} see Gorshenin~A.\,K.&&\\
\Avtors{Vorontsov~M.\,O. and~Shestakov~O.\,V.} Mean-square risk of the FDR procedure under 
weak dependence&\raisebox{-12pt}[0pt][0pt]{2}&\raisebox{-12pt}[0pt][0pt]{34--40}\\
\Avtors{Zabezhailo~M.\,I.} see Grusho~A.\,A.&&\\
\Avtors{Zabezhailo~M.\,I.} see Grusho~A.\,A.&&\\
\end{tabular}
}
\pagebreak

\def\leftfootline{\small{\textbf{\thepage}
\hfill INFORMATIKA I EE PRIMENENIYA~--- INFORMATICS AND APPLICATIONS\ \ \ 2023\
\ \ volume~17\ \ \ issue\ 4}
}%
 \def\rightfootline{\small{INFORMATIKA I EE PRIMENENIYA~---
INFORMATICS AND APPLICATIONS\ \ \ 2023\ \ \ volume~17\ \ \ issue\ 4
\hfill \textbf{\thepage}}}

\def\leftkol{2023 AUTHOR INDEX} % ENGLISH ABSTRACTS}

\def\rightkol{2023 AUTHOR INDEX} %ENGLISH ABSTRACTS}


\noindent
{\tabcolsep=3pt
\begin{tabular}{p{395.5pt}cc}
&\textbf{Issue} & \textbf{Page}\\[6pt]
\Avtors{Zabezhailo~M.\,I.} see Grusho~A.\,A.&&\\
\Avtors{Zakharov~V.\,N.} see Frenkel~S.\,L.&&\\
\Avtors{Zakharov~V.\,N.} see Sazontev V.\,V.&&\\
\Avtors{Zatsman~I.\,M.} On the scientific paradigm of informatics: Data, information, and 
knowledge&1&116--125\\
\Avtors{Zatsman I.\,M.} Scientific paradigm of informatics: Classification of domain 
objects&4&\hphantom{9}96--103\\
\Avtors{Zatsman~I.\,M.} Transformation of the Ackoff's hierarchy in the scientific paradigm of 
informatics&3&107--113\\
\Avtors{Zatsman~I.\,M.} see Vakulenko~V.\,V.&&\\
\Avtors{Zeifman~A.\,I.} see Usov~I.\,A.&&\\
\end{tabular}
}

%\thispagestyle{myheadings}
\def\leftfootline{\small{\textbf{\thepage}
\hfill INFORMATIKA I EE PRIMENENIYA~--- INFORMATICS AND APPLICATIONS\ \ \ 2023\
\ \ volume~17\ \ \ issue\ 4}
}%
 \def\rightfootline{\small{INFORMATIKA I EE PRIMENENIYA~---
INFORMATICS AND APPLICATIONS\ \ \ 2023\ \ \ volume~17\ \ \ issue\ 4
\hfill \textbf{\thepage}}}

 \label{end\stat}

\newpage

%
   \vspace*{-46pt}

\begin{center}
\vspace*{4pt}
\mbox{%

\epsfxsize=55mm %112.705
\epsfbox{zhur-2.eps}
}
%\end{center}

\vspace*{10pt} 


%   \begin{center}
\fbox{\large\textbf{Академик Юрий Иванович Журавлёв}}\\[10pt]
\textbf{\large 14.01.1935--14.01.2022}
   \end{center}


   %\vspace*{2.5mm}

   \vspace*{5mm}

   \thispagestyle{empty}

%\

%\vspace*{-12pt}
       


В январе этого года ушел из жизни главный научный сотрудник Федерального исследовательского 
центра <<Информатика и управление>> РАН, председатель Редакционного совета журнала 
<<Информатика и~её применения>> академик Юрий Иванович Журавлёв. В~его лице мировая 
наука потеряла одного из своих ярчайших представителей~--- выдающегося ученого-исследователя 
и~талантливого ученого-организатора.

Юрий Иванович родился в Воронеже в 1935~г.\ в семье ученого и врача. Среднее образование 
получил в школе №\,6 г.~Фрунзе (ныне Бишкек) Киргизской ССР. В~1952~г.\ поступил на 
ме\-ха\-ни\-ко-ма\-те\-ма\-ти\-че\-ский факультет МГУ им.\ М.\,В.~Ломоносова. В~1957~г.\ Юрий Иванович 
защищает диплом и продолжает обучение в аспирантуре Московского университета на кафедре 
вычислительной математики (возглавляемой тогда академиком С.\,Л.~Соболевым). После 
успешной защиты кандидатской диссертации (к.ф.-м.н., 1959 г., научный руководитель~--- 
А.\,А.~Ляпунов, оппоненты~--- чл.-корр.\ А.\,А.~Марков, к.ф.-м.н.\ О.\,Б.~Лупанов) и~до 
окончательного переезда в Москву в 1969~г.\ работал в Институте математики Сибирского 
отделения АН СССР, занимая в нем последовательно должности младшего научного сотрудника, 
заведующего отделом, заведующего отделением, заместителя директора по научной работе. 
В~этот период (1954--1966~гг.)\ им был опубликован цикл работ по решению задач алгебры и 
математической логики, причем полученные результаты применялись для создания эффективных 
программ для ЭВМ, конструирования схем и сетей для обработки информации. Наиболее значимый 
результат этого периода научной работы~--- обоснование нового направления исследований, 
общей теории локальных алгоритмов. В~ней были окончательно объединены топологические 
принципы и теория алгоритмов. Эта теория и легла в основу докторской диссертации Юрия 
Ивановича (д.ф.-м.н., 1965~г.)\ по еще тогда новой научной специальности <<Математическая 
кибернетика>>. Оппонировали ему как специалисты по кибернетике~--- академик 
В.\,М.~Глушков, член-корреспондент А.\,А.~Ляпунов и О.\,Б.~Лупанов, так и про\-фес\-сор-ал\-геб\-раист А.\,Д.~Тайманов. 

В 1969~г.\ Юрий Иванович переезжает в Москву и возглавляет в Вычислительном центре АН 
СССР лабораторию проблем распознавания. Впоследствии он~--- заместитель директора по 
научной работе. Научные интересы этого периода связаны с проблемами классификации или 
распознавания образов. В~1976--1978~гг.\ Юрий Иванович публикует цикл работ по ставшему 
вскоре знаменитым алгебраическому подходу к проблеме синтеза корректных алгоритмов. Эти 
работы определили современное состояние всей проблематики распознавания и многих смежных 
областей прикладной математики и информатики. В~своих основополагающих работах Юрий 
Иванович показал, что можно в явном виде строить экстремальные по качеству алгоритмы для 
решения очень широких классов плохо формализованных задач. 
{\looseness=-1

}





Научные заслуги Юрия Ивановича получили широкое признание. В~1966~г.\ он совместно с 
О.\,Б.~Лупановым и чле\-ном-кор\-рес\-пон\-ден\-том АН СССР С.\,В.~Яблонским были удостоены 
звания лауреата Ленинской премии в~об\-ласти науки и техники. В~1984~г.\ Юрий Иванович 
был избран членом-корреспондентом АН СССР (по специальности <<Информатика>>), 
а~в~1992~г.~--- академиком РАН (по той же специальности).\linebreak\vspace*{-12pt}

\pagebreak

\

\vspace*{-46pt}

\noindent
\begin{floatingfigure}{48mm}
\begin{center}
%\vspace*{6pt}
\mbox{%

\epsfxsize=46mm %112.705
\epsfbox{zhur-3.eps}
}
\end{center}
\vspace*{6pt}
\end{floatingfigure}

 \thispagestyle{empty}

\noindent
В~1986~г.\ за цикл прикладных 
работ ему и ряду его учеников была при\-суж\-де\-на премия Совета Министров СССР. Он являлся 
членом иностранных академий наук, председателем секции <<Прикладная математика
 и~информатика>> Отделения математических наук РАН, председателем экспертного совета ВАК 
России по управ\-ле\-нию и информатике, заслуженным профессором нескольких университетов, 
председателем Российской ассоциации <<Распознавание образов и обработка изображений>>, 
членом исполкома Международной ассоциации IAPR (распознавание образов и обработка 
изображений). Был награжден 8-ю орденами и медалями СССР и России.

Юрий Иванович проводил большую научно-литературную работу, являясь, в том числе, главным 
редактором международных научных журналов и членом редколлегий ряда рецензируемых 
научных журналов. 


Параллельно с активной научной деятельностью Юрий Иванович вел и преподавательскую 
работу. С~1961 по~1969~гг.~--- в Новосибирском государственном университете на кафедре 
алгебры и математической логики, которую возглавлял в то время академик А.\,И.~Мальцев. 
С~1970~г., будучи уже профессором (1967~г.),~--- в Московском физико-техническом институте 
на кафедре академика Н.\,Н.~Моисеева. В~1997~г.\ по предложению ректора МГУ им.\ 
М.\,В.~Ломоносова академика В.\,А.~Садовничего Юрий Иванович организовал на факультете 
Вычислительной математики и кибернетики новую кафедру <<Математические методы 
прогнозирования>>, которой и руководил до конца жизни. В~2008~г.\ ему была присуждена 
премия Совета Министров РФ в области образования. С~1965~г.\ Юрий Иванович периодически 
читал курсы лекций за рубежом, в университетах США, Франции, Финляндии, Швеции, Австрии, 
Польши, Болгарии, ГДР и других стран. Эта работа в существенной степени обеспечила широкое 
международное признание советской и российской науки в области дискретной математики и~распознавания образов. 

%\begin{floatingfigure}{60mm}
\begin{figure}[b]
\begin{center}
\vspace*{-6pt}
\mbox{%

\epsfxsize=112mm %90mm %112.705
\epsfbox{zhur-1.eps}
}
\end{center}
\end{figure}
%\end{floatingfigure}

Понимая важность вопроса воспитания подрастающего поколения для развития науки в стране, 
Юрий Иванович вскоре после защиты первой диссертации включился в работу по подготовке 
научных кадров. Им создана большая научная школа: под руководством Юрия Ивановича 
защищены более 100~кандидатских диссертаций по всевозможным разделам естествознания 
(математике, информатике, медицине, технике, экономике, геологии), не один десяток докторов 
наук. Он воспитал академиков и членов-корреспондентов РАН и академий государств СНГ. 
С~большим вниманием и участием Юрий Иванович относился к развитию научных школ страны 
в~об\-ласти обработки изображений, распознавания образов и компьютерной оптики. 

Для всех коллег и учеников Юрия Ивановича он останется примером замечательного человека, 
та\-лант\-ли\-во\-го педагога и выдающегося, преданного служению науке ученого. 


%\def\stat{cont}
{%\hrule\par
%\vskip 7pt % 7pt
\raggedleft\Large \bf%\baselineskip=3.2ex
А\,В\,Т\,О\,Р\,С\,К\,И\,Й\ \ У\,К\,А\,З\,А\,Т\,Е\,Л\,Ь\ \ З\,А\ \ 2\,0\,1\,0 г. \vskip 17pt
    \hrule
    \par
\vskip 21pt plus 6pt minus 3pt }

\label{st\stat}

\def\tit{\ }

\def\aut{\ }
\def\auf{\ }

\def\leftkol{\ } % ENGLISH ABSTRACTS}

\def\rightkol{\ } %АВТОРСКИЙ УКАЗАТЕЛЬ ЗА 2010 г.} %ENGLISH ABSTRACTS}

\titele{\tit}{\aut}{\auf}{\leftkol}{\rightkol}

\vspace*{-12pt}

{\tabcolsep=3pt
\begin{tabular}{p{388pt}rr}
&\textbf{Выпуск} & \textbf{Стр.}\\[6pt]
\hangindent=23pt\noindent\textbf{Арутюнян~А.\,Р.} Моделирование влияния деформаций отпечатков пальцев на 
точность\linebreak
\vspace*{-12pt}\\
\hspace*{23pt}дактилоскопической идентификации$\dotfill$&1&51\\
\hangindent=23pt\noindent\textbf{Архипов~О.\,П., Зыкова~З.\,П.} Интеграция гетерогенной информации о цветных 
пикселях\linebreak
\vspace*{-12pt}\\
\hspace*{23pt}и их цветовосприятии$\dotfill$&4&15\\
\hangindent=23pt\noindent\textbf{Баранов~С.\,И., Френкель~С.\,Л., Захаров~В.\,Н.} Полуформальная верификация 
цифрового устройства с конвейером, основанная на использовании алгоритмических машин\linebreak
\vspace*{-12pt}\\
\hspace*{23pt}состояния$\dotfill$&4&49\\
\textbf{Бекетова~И.\,В.} см.~Каратеев~С.\,Л.&&\\
\textbf{Белоусов~В.\,В.} см.~Синицын~И.\,Н.&&\\
\hangindent=23pt\noindent\textbf{Бенинг~В.\,Е., Королев~Р.\,А.} О предельном поведении мощностей критериев в 
случае\linebreak
\vspace*{-12pt}\\
\hspace*{23pt}распределения Лапласа$\dotfill$&2&63\\
\hangindent=23pt\noindent\textbf{Бенинг~В.\,Е., Сипина~А.\,В.} Асимптотическое разложение для мощности 
критерия,\linebreak
\vspace*{-12pt}\\
\hspace*{23pt}основанного на выборочной медиане, в случае распределения Лапласа$\dotfill$&1&18\\
\textbf{Бондаренко~А.\,В.} см.~Каратеев~С.\,Л.&&\\
\hangindent=23pt\noindent\textbf{Бородина~А.\,В., Морозов~Е.\,В.} Об оценивании асимптотики вероятности 
большого\linebreak
\vspace*{-12pt}\\
\hspace*{23pt}уклонения стационарной регенеративной очереди с одним прибором$\dotfill$&3&29\\
\hangindent=23pt\noindent\textbf{Бунтман~Н.\,В., Минель~Ж.-Л., Ле~Пезан~Д., Зацман~И.\,М.} Типология и 
компьютерное\linebreak
\vspace*{-12pt}\\
\hspace*{23pt}моделирование трудностей перевода$\dotfill$&3&77\\
\textbf{Визильтер~Ю.\,В.} см.~Каратеев~С.\,Л.&&\\
\hangindent=23pt\noindent\textbf{Гавриленко~С.\,В.} Оценки скорости сходимости распределений случайных сумм с 
безгранично делимыми индексами к нормальному закону$\dotfill$&4&81\\
\hangindent=23pt\noindent\textbf{Григорьева~М.\,Е., Шевцова~И.\,Г.} Уточнение неравенства 
Каца--Берри--Эссеена$\dotfill$&2&75\\
\hangindent=23pt\noindent\textbf{Грушо~А.\,А., Грушо~Н.\,А., Тимонина~Е.\,Е.} Поиск конфликтов в политиках 
безопасности: модель случайных графов$\dotfill$&3&38\\
\textbf{Грушо~Н.\,А.} см.~Грушо~А.\,А.&&\\
\hangindent=23pt\noindent\textbf{Гудков~В.\,Ю.} Математические модели изображения отпечатка пальца на основе 
описания линий$\dotfill$&1&58\\
\textbf{Гуртов~А.\,В.} см.~Лукьяненко~А.\,С.&&\\
\textbf{Желтов~С.\,Ю.} см.~Каратеев~С.\,Л.&&\\
\hangindent=23pt\noindent\textbf{Захаров~А.\,А., Серебряков~В.\,А.} Система управления электронной библиотекой 
LibMeta$\dotfill$&4&2\\
\textbf{Захаров~В.\,Н.} см.~Баранов~С.\,И.&&\\
\textbf{Захарова~Т.\,В.} см.~Матвеева~С.\,С.&&\\
\hangindent=23pt\noindent\textbf{Зацаринный~А.\,А., Чупраков~К.\,Г.} Некоторые аспекты выбора технологии для 
постро-\linebreak
\vspace*{-12pt}\\
\hspace*{23pt}ения систем отображения информации ситуационного центра$\dotfill$&3&59\\
\textbf{Зацман~И.\,М.} см.~Бунтман~Н.\,В.&&\\
\hangindent=23pt\noindent\textbf{Зейфман~А.\,И., Коротышева~А.\,В., Сатин~Я.\,А., Шоргин~С.\,Я.} Об 
устойчивости нестаци-\linebreak
\vspace*{-12pt}\\
\hspace*{23pt}онарных систем обслуживания с катастрофами$\dotfill$&3&9\\
\textbf{Зыкова~З.\,П.} см.~Архипов~О.\,П.&&\\
\hangindent=23pt\noindent\textbf{Илюшин~Г.\,Я., Соколов~И.\,А.} Организация управляемого доступа пользователей 
к\linebreak
\vspace*{-12pt}\\
\hspace*{23pt}разнородным ведомственным информационным ресурсам$\dotfill$&1&24\\
\hangindent=23pt\noindent\textbf{Кавагучи~Ю., Ульянов~В.\,В., Фуджикоши~Я.} Приближения для статистик, 
описывающих\linebreak
\vspace*{-12pt}\\
\hspace*{23pt}геометрические свойства данных большой размерности, с оценками 
ошибок$\dotfill$&1&12\\
\hangindent=23pt\noindent\textbf{Каратеев~С.\,Л., Бекетова~И.\,В., Ососков~М.\,В., Князь~В.\,А., 
Визильтер~Ю.\,В., Бондаренко~А.\,В., Желтов~С.\,Ю.} Автоматизированный контроль 
качества цифровых\linebreak
\vspace*{-12pt}\\
\hspace*{23pt}изображений для персональных документов$\dotfill$&1&65\\
\end{tabular}
}

\pagebreak

\def\leftkol{АВТОРСКИЙ УКАЗАТЕЛЬ ЗА 2010 г.} % ENGLISH ABSTRACTS}

\def\rightkol{АВТОРСКИЙ УКАЗАТЕЛЬ ЗА 2010 г.} %ENGLISH ABSTRACTS}

{\tabcolsep=3pt
\begin{tabular}{p{388pt}rr}
&\textbf{Выпуск} & \textbf{Стр.}\\[3pt]
\hangindent=23pt\noindent\textbf{Козеренко~Е.\,Б.} Лингвистические фильтры в статистических моделях машинного\linebreak
\vspace*{-12pt}\\
\hspace*{23pt}перевода$\dotfill$&2&83\\
\hangindent=23pt\noindent\textbf{Козеренко~Е.\,Б., Кузнецов~И.\,П.} Когнитивно-лингвистические представления в 
систе-\linebreak
\vspace*{-12pt}\\
\hspace*{23pt}мах обработки текстов$\dotfill$&3&69\\
\textbf{Князь~В.\,А.} см.~Каратеев~С.\,Л.&&\\
\hangindent=23pt\noindent\textbf{Колесников~А.\,В., Солдатов~С.\,А.} Алгоритм координации для гибридной 
интеллектуальной системы решения сложной задачи оперативно-производственного\linebreak
\vspace*{-12pt}\\
\hspace*{23pt}планирования$\dotfill$&4&61\\
\hangindent=23pt\noindent\textbf{Коновалов~М.\,Г.} О планировании потоков в системах вычислительных 
ресурсов$\dotfill$&2&3\\
\textbf{Конушин~А.\,С.} см.~Конушин~В.\,С.&&\\
\hangindent=23pt\noindent\textbf{Конушин~В.\,С., Кривовязь~Г.\,Р., Конушин~А.\,С.} Алгоритм распознавания людей 
в видео-\linebreak
\vspace*{-12pt}\\
\hspace*{23pt}последовательности по одежде$\dotfill$&1&74\\
\textbf{Корепанов~Э.\, Р.} см.~Синицын~И.\,Н.&&\\
\textbf{Королев~В.\,Ю.} см.~Соколов~И.\,А.&&\\
\textbf{Королев~Р.\,А.} см.~Бенинг~В.\,Е.&&\\
\textbf{Коротышева~А.\,В.} см.~Зейфман~А.\,И.&&\\
\hangindent=23pt\noindent\textbf{Кривенко~М.\,П.} Непараметрическое оценивание элементов байесовского 
клас\-си-\linebreak
\vspace*{-12pt}\\
\hspace*{23pt}фикатора$\dotfill$&2&13\\
\textbf{Кривовязь~Г.\,Р.} см.~Конушин~В.\,С.&&\\
\textbf{Крылов~А.\,С.} см.~Павельева~Е.\,А.&&\\
\hangindent=23pt\noindent\textbf{Крылов~В.\,А.} Моделирование и классификация многоканальных дистанционных\linebreak
\vspace*{-12pt}\\
\hspace*{23pt}изображений с использованием копул$\dotfill$&4&34\\
\hangindent=23pt\noindent\textbf{Крючин~О.\,В.} Разработка параллельных эвристических алгоритмов подбора 
весовых\linebreak
\vspace*{-12pt}\\
\hspace*{23pt}коэффициентов искусственной нейтронной сети$\dotfill$&2&53\\
\hangindent=23pt\noindent\textbf{Кудрявцев~А.\,А., Шоргин~С.\,Я.} Байесовские модели массового обслуживания и 
надеж-\linebreak
\vspace*{-12pt}\\
\hspace*{23pt}ности: характеристики среднего числа заявок в системе $M\vert M \vert 1\vert 
\infty$$\dotfill$&3&16\\
\hangindent=23pt\noindent\textbf{Кузнецов~А.\,А.} Связь между временными и структурно-топологическими 
характери-\linebreak
\vspace*{-12pt}\\
\hspace*{23pt}стиками диаграмм ритма сердца здоровых людей$\dotfill$&4&39\\
\textbf{Кузнецов~И.\,П.} см.~Козеренко~Е.\,Б.&&\\
\textbf{Ле~Пезан~Д.} см.~Бунтман~Н.\,В.&&\\
\hangindent=23pt\noindent\textbf{Лукьяненко~А.\,С., Морозов~Е.\,В., Гуртов~А.\,В.} Анализ сетевого протокола с общей 
функ-\linebreak
\vspace*{-12pt}\\
\hspace*{23pt}цией расширения окна передачи сообщения при конфликтах$\dotfill$&2&46\\
\hangindent=23pt\noindent\textbf{Лямин~О.\,О.} О предельном поведении мощностей критериев в случае обобщенного\linebreak
\vspace*{-12pt}\\
\hspace*{23pt}распределения Лапласа$\dotfill$&3&47\\
\hangindent=23pt\noindent\textbf{Маркин~А.\,В., Шестаков~О.\,В.} Асимптотики оценки риска при пороговой 
обработке\linebreak
\vspace*{-12pt}\\
\hspace*{23pt}вейвлет-вейглет коэффициентов в задаче томографии$\dotfill$&2&36\\
\hangindent=23pt\noindent\textbf{Матвеева~С.\,С., Захарова~Т.\,В.} Сети массового обслуживания с наименьшей 
длиной\linebreak
\vspace*{-12pt}\\
\hspace*{23pt}очереди$\dotfill$&3&22\\
\hangindent=23pt\noindent\textbf{Матюшенко~С.\,И.} Стационарные характеристики двухканальной системы 
обслужива-\linebreak
\vspace*{-12pt}\\
\hspace*{23pt}ния с переупорядочиванием заявок и распределениями фазового типа$\dotfill$&4&68\\
\textbf{Минель~Ж.-Л.} см.~Бунтман~Н.\,В.&&\\
\textbf{Морозов~Е.\,В.} см.~Бородина~А.\,В.&&\\
\textbf{Морозов~Е.\,В.} см.~Лукьяненко~А.\,С.&&\\
\textbf{Ососков~М.\,В.} см.~Каратеев~С.\,Л.&&\\
\hangindent=23pt\noindent\textbf{Павельева~Е.\,А., Крылов~А.\,С.} Поиск и анализ ключевых точек радужной 
оболочки\linebreak
\vspace*{-12pt}\\
\hspace*{23pt}глаза методом преобразования Эрмита$\dotfill$&1&79\\
\textbf{Печинкин~А.\,В.} см.~Френкель~С.\,Л.,&&\\
\hangindent=23pt\noindent\textbf{Протасов~В.\,И.} Составление субъективного портрета с использованием 
эволюционно-\linebreak
\vspace*{-12pt}\\
\hspace*{23pt}го морфинга и квалиметрия метода$\dotfill$&1&83\\
\hangindent=23pt\noindent\textbf{Рудаков~К.\,В., Торшин~И.\,Ю.} Вопросы разрешимости задачи распознавания 
вторичной\linebreak
\vspace*{-12pt}\\
\hspace*{23pt}структуры белка$\dotfill$&2&25\\
\textbf{Сатин~Я.\,А.} см.~Зейфман~А.\,И.&&\\
\hangindent=23pt\noindent\textbf{Сейфуль-Мулюков~Р.\,Б.} Нефть как носитель информации о своем 
происхождении,\linebreak
\vspace*{-12pt}\\
\hspace*{23pt}структуре и эволюции$\dotfill$&1&41\\
\end{tabular}
}

{\tabcolsep=3pt
\begin{tabular}{p{388pt}rr}
&\textbf{Выпуск} & \textbf{Стр.}\\[6pt]
\textbf{Семендяев~Н.\,Н.} см.~Синицын~И.\,Н.&&\\
\textbf{Серебряков~В.\,А.} см.~Захаров~А.\,А.&&\\
\textbf{Синицын~В.\,И.} см.~Синицын~И.\,Н.&&\\
\hangindent=23pt\noindent\textbf{Синицын~И.\,Н., Синицын~В.\,И., Корепанов~Э.\, Р., Белоусов~В.\,В., 
Семендяев~Н.\,Н.} Оперативное построение информационных моделей движения полюса 
Земли\linebreak
\vspace*{-12pt}\\
\hspace*{23pt}методами линейных и линеаризованных фильтров$\dotfill$&1&2\\
\textbf{Сипина~А.\,В.} см.~Бенинг~В.\,Е.&&\\
\hangindent=23pt\noindent\textbf{Соколов~И.\,А.} О работах заслуженного деятеля науки Российской Федерации 
И.\,Н.~Синицына в области информационных технологий и автоматизации (к 70-летию\linebreak
\vspace*{-12pt}\\
\hspace*{23pt}со дня рождения)$\dotfill$&3&84\\
\textbf{Соколов~И.\,А.} см.~Илюшин~Г.\,Я.&&\\
\hangindent=23pt\noindent\textbf{Соколов~И.\,А., Королев~В.\,Ю.} Предисловие$\dotfill$&2&2\\
\textbf{Солдатов~С.\,А.} см.~Колесников~А.\,В.&&\\
\hangindent=23pt\noindent\textbf{Степанов~С.\,Ю.} Использование координатного метода фрагментации 
коммутаторной\linebreak
\vspace*{-12pt}\\
\hspace*{23pt}нейронной сети для сокращения трафика$\dotfill$&2&57\\
\textbf{Тимонина~Е.\,Е.} см.~Грушо~А.\,А.&&\\
\textbf{Торшин~И.\,Ю.} см.~Рудаков~К.\,В.&&\\
\textbf{Ульянов~В.\,В.} см.~Кавагучи~Ю.&&\\
\textbf{Фазекаш~И.} см.~Чупрунов~А.\,Н.&&\\
\textbf{Френкель~С.\,Л.} см.~Баранов~С.\,И.&&\\
\hangindent=23pt\noindent\textbf{Френкель~С.\,Л., Печинкин~А.\,В.} Оценка времени самовосстановления в 
цифровых\linebreak
\vspace*{-12pt}\\
\hspace*{23pt}системах после сбоев, вызываемых переходными помехами$\dotfill$&3&2\\
\textbf{Фуджикоши~Я.} см.~Кавагучи~Ю.&&\\
\hangindent=23pt\noindent\textbf{Цискаридзе~А.\,К.} Математическая модель и метод восстановления позы человека 
по\linebreak
\vspace*{-12pt}\\
\hspace*{23pt}стереопаре силуэтных изображений$\dotfill$&4&27\\
\hangindent=23pt\noindent\textbf{Чупраков~К.\,Г.} К вопросу о размещении коллективных средств отображения в 
ситуа-\linebreak
\vspace*{-12pt}\\
\hspace*{23pt}ционном зале с заданными параметрами$\dotfill$&4&89\\
\textbf{Чупраков~К.\,Г.} см.~Зацаринный~А.\,А.&&\\
\hangindent=23pt\noindent\textbf{Чупрунов~А.\,Н., Фазекаш~И.} Законы повторного логарифма для числа 
безошибочных\linebreak
\vspace*{-12pt}\\
\hspace*{23pt}блоков при помехоустойчивом кодировании$\dotfill$&3&42\\
\textbf{Шевцова~И.\,Г.} см.~Григорьева~М.\,Е.&&\\
\hangindent=23pt\noindent\textbf{Шестаков~О.\,В.} Аппроксимация распределения оценки риска пороговой 
обработки вейвлет-коэффициентов нормальным распределением при использовании 
выбо-\linebreak
\vspace*{-12pt}\\
\hspace*{23pt}рочной дисперсии$\dotfill$&4&73\\
\textbf{Шестаков~О.\,В.} см.~Маркин~А.\,В.&&\\
\textbf{Шоргин~С.\,Я.} см.~Зейфман~А.\,И.&&\\
\textbf{Шоргин~С.\,Я.} см.~Кудрявцев~А.\,А.&&\\
\end{tabular}
}

%\thispagestyle{myheadings}
\def\leftfootline{\small{\textbf{\thepage}
\hfill ИНФОРМАТИКА И ЕЁ ПРИМЕНЕНИЯ\ \ \ том~4\ \ \ выпуск~4\ \ \ 2010}
}%
 \def\rightfootline{\small{ИНФОРМАТИКА И ЕЁ ПРИМЕНЕНИЯ\ \ \ том~4\ \ \ выпуск~4\ \ \ 2010
 \hfill \textbf{\thepage}}}
 \label{end\stat}
%
%Том 10 Выпуск 1-4 Год 2016

\def\stat{cont-e}
{%\hrule\par
%\vskip 7pt % 7pt
\raggedleft\Large \bf%\baselineskip=3.2ex
2\,0\,1\,6\ \ A\,U\,T\,H\,O\,R\ \ I\,N\,D\,E\,X \vskip 17pt
 \hrule
 \par
\vskip 21pt plus 6pt minus 3pt }

\label{st\stat}

\def\tit{\ }

\def\aut{\ }
\def\auf{\ }

\def\leftkol{\ } %2016 AUTHOR INDEX} % ENGLISH ABSTRACTS}

\def\rightkol{\ } %2016 AUTHOR INDEX} %ENGLISH ABSTRACTS}

\titele{\tit}{\aut}{\auf}{\leftkol}{\rightkol}

\def\leftfootline{\small{\textbf{\thepage}
\hfill INFORMATIKA I EE PRIMENENIYA~--- INFORMATICS AND APPLICATIONS\ \ \ 2016\
\ \ volume~10\ \ \ issue\ 4}
}%
 \def\rightfootline{\small{INFORMATIKA I EE PRIMENENIYA~--- INFORMATICS AND APPLICATIONS\ \ \ 2016\ \ \ volume~10\ \ \ issue\ 4
\hfill \textbf{\thepage}}}

\vspace*{-12pt}
\vspace*{-18pt}

{\tabcolsep=2.8pt
\begin{tabular}{p{382pt}cc}
&\textbf{Issue} & \textbf{Page}\\[6pt]
\Avtors{Agalarov~M.\,Ya.} see~Agalarov~Ya.\,M.&&\\
\Avtors{Agalarov~Ya.\,M., Agalarov~M.\,Ya., and
Shorgin~V.\,S.} About the optimal threshold of queue\linebreak
\\[-12pt]
\hspace*{23pt}length in a~particular problem of profit maximization
in the $M/G/1$ queuing system&2&70--79\\
\Avtors{Alexeyevsky~D.\,A.} BioNLP ontology extraction from 
a~restricted language corpus with\linebreak
\\[-12pt]
\hspace*{23pt}context-free grammars&1&119--128\\
\Avtors{Andreev~S.\,D.} see~Gaidamaka~Yu.\,V.&&\\
\Avtors{Andreev~S.\,D.} see~Ometov~A.\,Ya.&&\\
\Avtors{Arkhipov~O.\,P., Arkhipov~P.\,O., and Sidorkin~I.\,I.} The
option to create a~local coordinate\linebreak
\\[-12pt]
\hspace*{23pt}system for synchronization of selected images&3&91--97\\
\Avtors{Arkhipov~P.\,O.} see~Arkhipov~O.\,P.&&\\
\Avtors{Belousov~V.\,V.} see~Shnurkov~P.\,V.&&\\
\Avtors{Belousov~V.\,V.} see~Shnurkov~P.\,V.&&\\
\Avtors{Bening~V.\,E.} Calculation of~the~asymptotic deficiency
of~some statistical procedures based\linebreak
\\[-12pt]
\hspace*{23pt}on~samples with~random sizes&4&34--45\\
\Avtors{Borisov~A.\,V., Bosov~A.\,V., and Miller~G.\,B.} Modeling and
monitoring of VoIP connection&2&\hphantom{1}2--13\\
\Avtors{Bosov~A.\,V.} see~Borisov~A.\,V.&&\\
\Avtors{Briukhov~D.\,O.} see~Stupnikov~S.\,A.&&\\
\Avtors{Callaos~N.\,K.\ and Seyful-Mulyukov~R.\,B.} Complexity and
its information content&1&129--139\\
\Avtors{Chertok~A.\,V., Kadaner~A.\,I., Khazeeva~G.\,T., and
Sokolov~I.\,A.} Regime switching detection\linebreak
\\[-12pt]
\hspace*{23pt}for~the~Levy driven
Ornstein--Uhlenbeck process using CUSUM methods&4&46--56\\
\Avtors{Chichagov~V.\,V.} Asymptotic expansions of mean absolute
error of uniformly minimum variance unbiased and maximum likelihood
estimators on the one-parameter exponential\linebreak
\\[-12pt]
\hspace*{23pt}family model of lattice distributions&3&66--76\\
\Avtors{Danishevsky~V.\,I.} see~Kolesnikov A.\,V.&&\\
\Avtors{Fazliev~A.\,Z.} see~Kalinichenko~L.\,A.&&\\
\Avtors{Fedoseev~A.\,A.} What is behind the concept of ``knowledge in
small packages''&3&105--110\\
\Avtors{Gaidamaka~Yu.\,V., Andreev~S.\,D., Sopin~E.\,S.,
Samouylov~K.\,E., and Shorgin~S.\,Ya.} Interference analysis
of~the~device-to-device communications model with~regard to~a~signal\linebreak
\\[-12pt]
\hspace*{23pt}propagation environment&4&\hphantom{1}2--10\\
\Avtors{Gasilov~A.\,V.} see~Yakovlev~O.\,A.&&\\
\Avtors{Goncharov~A.\,V.\ and Strijov~V.\,V.} Metric time series
classification using weighted dynamic\linebreak
\\[-12pt]
\hspace*{23pt}warping relative to centroids of classes&2&36--47\\
\Avtors{Gordov~E.\,P.} see~Kalinichenko~L.\,A.&&\\
\Avtors{Gorshenin~A.\,K.} Concept of online service for stochastic
modeling of real processes&1&72--81\\
\Avtors{Gorshenin~A.\,K.} see~Shnurkov~P.\,V.&&\\
\Avtors{Gorshenin~A.\,K.} see~Shnurkov~P.\,V.&&\\
\Avtors{Grusho~A.\,A., Grusho~N.\,A., Zabezhailo~M.\,I., and
Timonina~E.\,E.} Integration of statistical and\linebreak
\\[-12pt]
\hspace*{23pt}deterministic methods for
analysis of information security&3&2--8\\
\Avtors{Grusho~A.\,A., Zabezhailo~M.\,I., and Zatsarinny~A.\,A.} On
the advanced procedure to reduce\linebreak
\\[-12pt]
\hspace*{23pt}calculation of Galois closures&4&\hphantom{1}96--104\\
\Avtors{Grusho~N.\,A.} see~Grusho~A.\,A.&&\\
\Avtors{Havanskov~V.\,A.} see~Minin~V.\,A.&&\\
\Avtors{Inkova~O.\,Yu.} see~Zatsman~I.\,M.&&\\
\Avtors{Isachenko~R.\,V.\ and Strijov~V.\,V.} Metric learning in
multiclass time series classification\linebreak
\\[-12pt]
\hspace*{23pt}problem&2&48--57\\
\end{tabular}
}
\pagebreak

\def\leftfootline{\small{\textbf{\thepage}
\hfill INFORMATIKA I EE PRIMENENIYA~--- INFORMATICS AND APPLICATIONS\ \ \ 2016\
\ \ volume~10\ \ \ issue\ 4}
}%
 \def\rightfootline{\small{INFORMATIKA I EE PRIMENENIYA~---
INFORMATICS AND APPLICATIONS\ \ \ 2016\ \ \ volume~10\ \ \ issue\ 4
\hfill \textbf{\thepage}}}

\def\leftkol{2016 AUTHOR INDEX} % ENGLISH ABSTRACTS}

\def\rightkol{2016 AUTHOR INDEX} %ENGLISH ABSTRACTS}


{\tabcolsep=2.83pt
\begin{tabular}{p{382pt}cc}
&\textbf{Issue} & \textbf{Page}\\[6pt]
\Avtors{Kadaner~A.\,I.} see~Chertok~A.\,V.&&\\[.255pt]
\Avtors{Kalinichenko~L.\,A., Volnova~A.\,A., Gordov~E.\,P.,
Kiselyova~N.\,N., Kovaleva~D.\,A., Malkov~O.\,Yu., Okladnikov~I.\,G.,
Podkolodnyy~N.\,L., Pozanenko~A.\,S., Ponomareva~N.\,V.,
Stupnikov~S.\,A.,} \textbf{and Fazliev~A.\,Z.} Data access challenges for data
intensive\linebreak
\\[-12pt]
\hspace*{23pt}research in Russia&1& 2--22\\[.255pt]
\Avtors{Karasikov~M.\,E.\ and Strijov~V.\,V.} Feature-based
time-series classification&4&121--131\\[.255pt]
\Avtors{Khazeeva~G.\,T.} see~Chertok~A.\,V.&&\\[.255pt]
\Avtors{Khokhlov~Yu.\,S.} Multivariate fractional Levy motion and its
applications&2&\hphantom{1}98--106\\[.255pt]
\Avtors{Kirikov~I.\,A., Kolesnikov~A.\,V., Listopad~S.\,V., and
Rumovskaya~S.\,B.} Fine-grained hybrid\linebreak
\\[-12pt]
\hspace*{23pt}intelligent systems. Part 2:
Bidirectional hybridization&1&\hphantom{1}96--105\\[.255pt]
\Avtors{Kirikov~I.\,A., Kolesnikov~A.\,V., Listopad~S.\,V., and
Rumovskaya~S.\,B.} ``Virtual council''~---\linebreak
\\[-12pt]
\hspace*{23pt}source environment
supporting complex diagnostic decision making&3&81--90\\[.255pt]
\Avtors{Kiselyova~N.\,N.} see~Kalinichenko~L.\,A.&&\\[.255pt]
\Avtors{Kolesnikov A.\,V., Listopad~S.\,V., Rumovskaya~S.\,B., and
Danishevsky~V.\,I.} Informal axiomatic\linebreak
\\[-12pt]
\hspace*{23pt}theory of~the~role visual models&4&114--120\\[.255pt]
\Avtors{Kolesnikov~A.\,V.} see~Kirikov~I.\,A.&&\\[.255pt]
\Avtors{Kolesnikov~A.\,V.} see~Kirikov~I.\,A.&&\\[.255pt]
\Avtors{Kolin~K.\,K.} Humanitarian aspects of information
security&3&111--121\\[.255pt]
\Avtors{Konovalov~M.\,G.\ and Razumchik~R.\,V.} Dispatching
to~two parallel nonobservable queues using\linebreak
\\[-12pt]
\hspace*{23pt}only static
information&4&57--67\\[.255pt]
\Avtors{Korchagin~A.\,Yu.} see~Korolev~V.\,Yu.&&\\[.255pt]
\Avtors{Korchagin~A.\,Yu.} see~Korolev~V.\,Yu.&&\\[.255pt]
\Avtors{Korepanov~E.\,R.} see~Sinitsyn~I.\,N.&&\\[.255pt]
\Avtors{Korepanov~E.\,R.} see~Sinitsyn~I.\,N.&&\\[.255pt]
\Avtors{Korolev~V.\,Yu., Korchagin~A.\,Yu., and Zeifman~A.\,I.} The
Poisson theorem for Bernoulli trials\linebreak
\\[-12pt]
\hspace*{23pt}with~a~random probability
of~success and~a~discrete analog of~the~Weibull distribution&4&11--20\\[.255pt]
\Avtors{Korolev~V.\,Yu., Zeifman~A.\,I., and Korchagin~A.\,Yu.}
Asymmetric Linnik distributions as~limit\linebreak
\\[-12pt]
\hspace*{23pt}laws for~random sums
of~independent random variables with~finite variances&4&21--33\\[.255pt]
\Avtors{Koucheryavy~E.\,A.} see~Ometov~A.\,Ya.&&\\[.255pt]
\Avtors{Kovaleva~D.\,A.} see~Kalinichenko~L.\,A.&&\\[.255pt]
\Avtors{Kovalyov~S.\,P.} Metaprogramming to increase
manufacturability of large-scale software-\linebreak
\\[-12pt]
\hspace*{23pt}intensive systems&1&56--66\\[.255pt]
\Avtors{Krivenko~M.\,P.} Significance tests of feature selection for
classification&3&32--40\\[.255pt]
\Avtors{Kruzhkov~M.\,G.} see~Zalizniak~Anna~A.&&\\[.255pt]
\Avtors{Kruzhkov~M.\,G.} see~Zatsman~I.\,M.&&\\[.255pt]
\Avtors{Kudryavtsev~A.\,A.} Bayesian queueing and reliability models:
\textit{A~priori} distributions with\linebreak
\\[-12pt]
\hspace*{23pt}compact support&1&67--71\\[.255pt]
\Avtors{Kudryavtsev~A.\,A.} Characteristics dependent on the balance
coefficient in Bayesian models\linebreak
\\[-12pt]
\hspace*{23pt}with compact support of \textit{a priori}
distributions&3&77--80\\[.255pt]
\Avtors{Kudryavtsev~A.\,A.\ and Palionnaia~S.\,I.} Bayesian recurrent
model of reliability growth:\linebreak
\\[-12pt]
\hspace*{23pt}Parabolic distribution of parameters&2&80--83\\[.255pt]
\Avtors{Kudryavtsev~A.\,A.\ and Titova~A.\,I.} Bayesian queuing
and~reliability models: Degenerate-\linebreak
\\[-12pt]
\hspace*{23pt}Weibull case&4&68--71\\[.255pt]
\Avtors{Leontyev~N.\,D.\ and Ushakov~V.\,G.} Analysis of a queueing
system with autoregressive arrivals\linebreak
\\[-12pt]
\hspace*{23pt}and nonpreemptive priority&3&15--22\\[.255pt]
\Avtors{Listopad~S.\,V.} see~Kirikov~I.\,A.&&\\[.255pt]
\Avtors{Listopad~S.\,V.} see~Kirikov~I.\,A.&&\\[.255pt]
\Avtors{Listopad~S.\,V.} see~Kolesnikov A.\,V.&&\\[.255pt]
\Avtors{Malkov~O.\,Yu.} see~Kalinichenko~L.\,A.&&\\[.255pt]
\Avtors{Markov~A.\,S., Monakhov~M.\,M., and
Ulyanov~V.\,V.} Generalized Cornish--Fisher expansions\linebreak
\\[-12pt]
\hspace*{23pt}for distributions of statistics based on samples
of random size&2&84--91\\[.255pt]
\Avtors{Melnikov~A.\,K.\ and Ronzhin~A.\,F.} Generalized statistical
method of~text analysis based\linebreak
\\[-12pt]
\hspace*{23pt}on~calculation of~probability distributions
of~statistical values&4&89--95\\
\end{tabular}
}
\pagebreak

\def\leftfootline{\small{\textbf{\thepage}
\hfill INFORMATIKA I EE PRIMENENIYA~--- INFORMATICS AND APPLICATIONS\ \ \ 2016\
\ \ volume~10\ \ \ issue\ 4}
}%
 \def\rightfootline{\small{INFORMATIKA I EE PRIMENENIYA~---
INFORMATICS AND APPLICATIONS\ \ \ 2016\ \ \ volume~10\ \ \ issue\ 4
\hfill \textbf{\thepage}}}

\def\leftkol{2016 AUTHOR INDEX} % ENGLISH ABSTRACTS}

\def\rightkol{2016 AUTHOR INDEX} %ENGLISH ABSTRACTS}


{\tabcolsep=3pt
\begin{tabular}{p{381pt}cc}
&\textbf{Issue} & \textbf{Page}\\[6pt]
\Avtors{Meykhanadzhyan~L.\,A.} Stationary characteristics of the finite
capacity queueing system with\linebreak
\\[-12pt]
\hspace*{23pt}inverse service order and generalized
probabilistic priority&2&123--131\\[.23pt]
\Avtors{Miller~G.\,B.} see~Borisov~A.\,V.&&\\[.23pt]
\Avtors{Minin~V.\,A., Zatsman~I.\,M., Havanskov~V.\,A., and
Shubnikov~S.\,K.} Intensity of citation of scientific publications in
inventions on information and computer technologies patented\linebreak
\\[-12pt]
\hspace*{23pt}in Russia by domestic and foreign applicants&2&107--122\\[.23pt]
\Avtors{Monakhov~M.\,M.} see~Markov~A.\,S.&&\\[.23pt]
\Avtors{Naumov~V.\,A.\ and Samouylov~K.\,E.} On relationship
between queuing systems with resources\linebreak
\\[-12pt]
\hspace*{23pt}and Erlang networks&3&\hphantom{1}9--14\\[.23pt]
\Avtors{Okladnikov~I.\,G.} see~Kalinichenko~L.\,A.&&\\[.23pt]
\Avtors{Ometov~A.\,Ya., Andreev~S.\,D., Turlikov~A.\,M., and
Koucheryavy~E.\,A.} Performance analysis of\linebreak
\\[-12pt]
\hspace*{23pt}a wireless data
aggregation system with contention for contemporary sensor
networks&3&23--31\\[.23pt]
\Avtors{Palionnaia~S.\,I.} see~Kudryavtsev~A.\,A.&&\\[.23pt]
\Avtors{Podkolodnyy~N.\,L.} see~Kalinichenko~L.\,A.&&\\[.23pt]
\Avtors{Ponomareva~N.\,V.} see~Kalinichenko~L.\,A.&&\\[.23pt]
\Avtors{Popkova~N.\,A.} see~Zatsman~I.\,M.&&\\[.23pt]
\Avtors{Pozanenko~A.\,S.} see~Kalinichenko~L.\,A.&&\\[.23pt]
\Avtors{Razumchik~R.\,V.} see~Konovalov~M.\,G.&&\\[.23pt]
\Avtors{Ronzhin~A.\,F.} see~Melnikov~A.\,K.&&\\[.23pt]
\Avtors{Rumovskaya~S.\,B.} see~Kirikov~I.\,A.&&\\[.23pt]
\Avtors{Rumovskaya~S.\,B.} see~Kirikov~I.\,A.&&\\[.23pt]
\Avtors{Rumovskaya~S.\,B.} see~Kolesnikov A.\,V.&&\\[.23pt]
\Avtors{Samouylov~K.\,E.} see~Gaidamaka~Yu.\,V.&&\\[.23pt]
\Avtors{Samouylov~K.\,E.} see~Naumov~V.\,A.&&\\[.23pt]
\Avtors{Serebryanskii~S.\,M.} see~Tyrsin~A.\,N.&&\\[.23pt]
\Avtors{Seyful-Mulyukov~R.\,B.} see~Callaos~N.\,K.&&\\[.23pt]
\Avtors{Shestakov~O.\,V.} Statistical properties of the denoising method
based on the stabilized hard\linebreak
\\[-12pt]
\hspace*{23pt}thresholding&2&65--69\\[.23pt]
\Avtors{Shestakov~O.\,V.} The strong law of large numbers for the risk
estimate in the problem of\linebreak
\\[-12pt]
\hspace*{23pt}tomographic image reconstruction from
projections with a correlated noise&3&41--45\\[.23pt]
\Avtors{Shestakov~O.\,V.} see~Zakharova~T.\,V.&&\\[.23pt]
\Avtors{Shnurkov~P.\,V., Gorshenin~A.\,K., and Belousov~V.\,V.}
Analytical solution of~the~optimal control\linebreak
\\[-12pt]
\hspace*{23pt}task of~a~semi-Markov
process with~finite set of~states&4&72--88\\[.23pt]
\Avtors{Shnurkov~P.\,V., Zasypko~V.\,V., Belousov~V.\,V., and
Gorshenin~A.\,K.} Development of the algorithm of numerical solution
of the optimal investment control problem\linebreak
\\[-12pt]
\hspace*{23pt}in the closed dynamical model of three-sector economy&1&82--95\\[.23pt]
\Avtors{Shorgin~S.\,Ya.} see~Gaidamaka~Yu.\,V.&&\\[.23pt]
\Avtors{Shorgin~V.\,S.} see~Agalarov~Ya.\,M.&&\\[.23pt]
\Avtors{Shubnikov~S.\,K.} see~Minin~V.\,A.&&\\[.23pt]
\Avtors{Sidorkin~I.\,I.} see~Arkhipov~O.\,P.&&\\[.23pt]
\Avtors{Sinitsyn~I.\,N.} Analytical modeling of processes in stochastic
systems with complex fractional\linebreak
\\[-12pt]
\hspace*{23pt}order Bessel nonlinearities&3&55--65\\[.23pt]
\Avtors{Sinitsyn~I.\,N.} Orthogonal supoptimal filters for nonlinear
stochastic systems on manifolds&1&34--44\\[.23pt]
\Avtors{Sinitsyn~I.\,N.\ and Korepanov~E.\,R.} Normal Pugachev
conditionally-optimal filters and extra-\linebreak
\\[-12pt]
\hspace*{23pt}polators for state linear stochastic systems&2&14--23\\[.23pt]
\Avtors{Sinitsyn~I.\,N.\ and Sinitsyn~V.\,I.} Analytical modeling of
distributions in stochastic systems on\linebreak
\\[-12pt]
\hspace*{23pt}manifolds based on ellipsoidal approximation&1&45--55\\[.23pt]
\Avtors{Sinitsyn~I.\,N., Sinitsyn~V.\,I., and
Korepanov~E.\,R.} Ellipsoidal suboptimal filters for nonlinear\linebreak
\\[-12pt]
\hspace*{23pt}stochastic systems on manifolds&2&24--35\\[.23pt]
\Avtors{Sinitsyn~V.\,I.} see~Sinitsyn~I.\,N.&&\\[.23pt]
\Avtors{Sinitsyn~V.\,I.} see~Sinitsyn~I.\,N.&&\\[.23pt]
\Avtors{Skvortsov~N.\,A.} see~Stupnikov~S.\,A.&&\\[.23pt]
\Avtors{Sokolov~I.\,A.} see~Chertok~A.\,V.&&\\
\end{tabular}
}
\pagebreak

\def\leftfootline{\small{\textbf{\thepage}
\hfill INFORMATIKA I EE PRIMENENIYA~--- INFORMATICS AND APPLICATIONS\ \ \ 2016\
\ \ volume~10\ \ \ issue\ 4}
}%
 \def\rightfootline{\small{INFORMATIKA I EE PRIMENENIYA~---
INFORMATICS AND APPLICATIONS\ \ \ 2016\ \ \ volume~10\ \ \ issue\ 4
\hfill \textbf{\thepage}}}

\def\leftkol{2016 AUTHOR INDEX} % ENGLISH ABSTRACTS}

\def\rightkol{2016 AUTHOR INDEX} %ENGLISH ABSTRACTS}


{\tabcolsep=3pt
\begin{tabular}{p{382pt}cc}
&\textbf{Issue} & \textbf{Page}\\[6pt]
\Avtors{Sopin~E.\,S.} see~Gaidamaka~Yu.\,V.&&\\
\Avtors{Strijov~V.\,V.} see~Goncharov~A.\,V.&&\\
\Avtors{Strijov~V.\,V.} see~Isachenko~R.\,V.&&\\
\Avtors{Strijov~V.\,V.} see~Karasikov~M.\,E.&&\\
\Avtors{Stupnikov~S.\,A., Briukhov~D.\,O., and Skvortsov~N.\,A.}
Co-lending systemic risk analysis over\linebreak
\\[-12pt]
\hspace*{23pt}heterogeneous data collections&1&23--33\\
\Avtors{Stupnikov~S.\,A.} see~Kalinichenko~L.\,A.&&\\
\Avtors{Suchkov~A.\,P.} see~Zatsarinny~A.\,A.&&\\
\Avtors{Timonina~E.\,E.} see~Grusho~A.\,A.&&\\
\Avtors{Titova~A.\,I.} see~Kudryavtsev~A.\,A.&&\\
\Avtors{Turlikov~A.\,M.} see~Ometov~A.\,Ya.&&\\
\Avtors{Tyrsin~A.\,N.\ and Serebryanskii~S.\,M.} Recognition of
dependences on the basis of inverse\linebreak
\\[-12pt]
\hspace*{23pt}mapping&2&58--64\\
\Avtors{Ulyanov~V.\,V.} see~Markov~A.\,S.&&\\
\Avtors{Ushakov~V.\,G.} Queueing system with working vacations and
hyperexponential input stream&2&92--97\\
\Avtors{Ushakov~V.\,G.} see~Leontyev~N.\,D.&&\\
\Avtors{Volnova~A.\,A.} see~Kalinichenko~L.\,A.&&\\
\Avtors{Yakovlev~O.\,A.\ and Gasilov~A.\,V.} Speeded-up stereo
matching using geodesic support weights&3&\hphantom{1}98--104\\
\Avtors{Zabezhailo~M.\,I.} see~Grusho~A.\,A.&&\\
\Avtors{Zabezhailo~M.\,I.} see~Grusho~A.\,A.&&\\
\Avtors{Zakharova~T.\,V.\ and Shestakov~O.\,V.} Precision analysis of
wavelet processing of aerodynamic\linebreak
\\[-12pt]
\hspace*{23pt}flow patterns&3&46--54\\
\Avtors{Zalizniak~Anna~A.\ and Kruzhkov~M.\,G.} Database
of~Russian impersonal verbal constructions&4&132--141\\
\Avtors{Zasypko~V.\,V.} see~Shnurkov~P.\,V.&&\\
\Avtors{Zatsarinny~A.\,A.\ and Suchkov~A.\,P.} Systems engineering
approaches to~the~establishment of\linebreak
\\[-12pt]
\hspace*{23pt}a~system for~decision support based
on~situational analysis&4&105--113\\
\Avtors{Zatsarinny~A.\,A.} see~Grusho~A.\,A.&&\\
\Avtors{Zatsman~I.\,M., Inkova~O.\,Yu., Kruzhkov~M.\,G., and
Popkova~N.\,A.} Representation of cross-\linebreak
\\[-12pt]
\hspace*{23pt}lingual knowledge about
connectors in supracorpora databases&1&106--118\\
\Avtors{Zatsman~I.\,M.} see~Minin~V.\,A.&&\\
\Avtors{Zeifman~A.\,I.} see~Korolev~V.\,Yu.&&\\
\Avtors{Zeifman~A.\,I.} see~Korolev~V.\,Yu.&&\\
\end{tabular}
}

%\thispagestyle{myheadings}
\def\leftfootline{\small{\textbf{\thepage}
\hfill INFORMATIKA I EE PRIMENENIYA~--- INFORMATICS AND APPLICATIONS\ \ \ 2016\
\ \ volume~10\ \ \ issue\ 4}
}%
 \def\rightfootline{\small{INFORMATIKA I EE PRIMENENIYA~---
INFORMATICS AND APPLICATIONS\ \ \ 2016\ \ \ volume~10\ \ \ issue\ 4
\hfill \textbf{\thepage}}}

 \label{end\stat}

\newpage

%\def\stat{rekl}
%\label{preobr}

%\def\tit{АКАДЕМИК ПУГАЧЁВ  ВЛАДИМИР СЕМЁНОВИЧ\\
%25.03.1911--25.03.1998}


%   \vspace*{-48pt}
%   \begin{center}\LARGE
%Академик Пугачёв  Владимир Семёнович\\ (25.03.1911--25.03.1998)
%   \end{center}
   
   %\vspace*{2.5mm}
   
   \begin{center}

{\prgsh\LARGE
ОБЪЯВЛЕНИЯ О КОНФЕРЕНЦИЯХ}

\end{center}
%\hrule

\vspace*{6pt}

   
   \vspace*{10mm}
   
   \thispagestyle{empty}

\noindent
\begin{tabular}{cc}
%\begin{center}
\multicolumn{1}{c}{\raisebox{-40pt}[0pt][0pt]{\mbox{%
\epsfxsize=33mm
\epsfbox{vspu.eps}
}}}
%\end{center}
&
\tabcolsep=0pt\begin{tabular}{c}
{\prg{\Large\textbf{XII Всероссийское совещание}}}\\[6pt]
{\prg{\Large\textbf{по проблемам управления}}}\\[12pt]
{\prg{\large 16--19 июня 2014~г.}}\\[6pt] 
{\prg{\large Институт проблем управления имени В.\,А.~Трапезникова РАН}}\\[6pt]
{\prg{\large Москва, Россия}}
\end{tabular}
\end{tabular}

\vspace*{60pt}

     
 { %\large    
 XII Всероссийское совещание по проблемам управления (ВСПУ XII), посвященное 75-летию 
Института проблем управления (ИПУ) имени В.\,А.~Трапезникова РАН, проводится 16--19~июня 
2014~г.\ 
в ИПУ РАН (г.~Москва, Россия). ВСПУ XII организуется ИПУ РАН при поддержке РФФИ, Отделения 
энергетики, машиностроения, механики и процессов управления Российской академии наук, 
Российского 
национального комитета по автоматическому управлению, Академии навигации и управ\-ле\-ния 
движением, 
Научного совета РАН по комплексным проблемам управления и автоматизации, Совета по 
мехатронике и робототехнике РАН. Официальный язык Совещания~--- русский.

\vspace*{24pt}
     
     \textbf{Направления работы}
     \begin{enumerate}[1.]
\item Теория систем управления
\item Управление подвижными объектами и навигация
\item Интеллектуальные системы управления
\item Управление в промышленности, транспортом и логистикой
\item Управление системами междисциплинарной природы
\item Средства измерения, вычислений и контроля в управлении
\item Системный анализ и принятие решений в задачах управления
\item Информационные технологии в управлении
\item Проблемы образования в области управления: современное содержание и технологии обучения
\end{enumerate}

\vspace*{24pt}

     Подробная информация о Совещании находится на сайте {\sf http://vspu2014.ipu.ru}. Срок 
окончательной подачи докладов через систему подачи докладов на сайте~--- \textbf{30~ноября} 
2013~г.
}


%\end{document}

%\include{nekrolog-rb}


%\end{document}

%\include{IPPM-25}

\def\stat{cont-rus}
{%\hrule\par
%\vskip 7pt % 7pt
\vspace*{-24pt}
\raggedleft\Large \bf%\baselineskip=3.2ex
Правила подготовки рукописей  для публикации в журнале
<<Информатика~и~её~применения>> \vskip 8pt
    \hrule
    \par
\vskip 14pt plus 6pt minus 3pt }

\label{st\stat}

\def\tit{\ }

\def\aut{\ }
\def\auf{\ }

\def\leftkol{\ }
% Правила подготовки рукописей  для публикации в журнале
%<<Информатика и её применения>>

\def\rightkol{\ }
%Правила подготовки рукописей  для публикации в журнале
%<<Информатика и её применения>>}


\titele{\tit}{\aut}{\auf}{\leftkol}{\rightkol}


\vspace*{-60pt}
{ %\small

Журнал <<Информатика и её применения>>
публикует теоретические, обзорные и дискуссионные статьи,
посвященные научным исследованиям и разработкам в области
информатики и ее приложений.

Журнал издается на русском языке. По специальному решению
редколлегии отдельные статьи могут печататься на английском языке.

Тематика журнала охватывает следующие направления:
\begin{itemize}
\item теоретические основы информатики;\\[-15pt]
      \item
математические методы исследования сложных систем и процессов;\\[-15pt]
           \item
информационные системы и сети;\\[-15pt]
                \item
информационные технологии;\\[-15pt]
                     \item
архитектура и программное обеспечение вычислительных комплексов и сетей.\\[-15pt]
\end{itemize}


\noindent
\begin{enumerate}[1.]
\item В журнале печатаются статьи, содержащие результаты, ранее не опубликованные и
не предназначенные к одновременной публикации в других изданиях.

%Публикация не должна нарушать закон об авторских правах.
Публикация предоставленной автором(ами) рукописи не должна нарушать 
положений глав~69, 70 раздела~VII части~IV Гражданского кодекса, 
которые определяют права на результаты интеллектуальной деятельности 
и~средства индивидуализации, в~том числе авторские права, в~РФ.

Ответственность за нарушение авторских прав, в~случае предъявления претензий к~редакции журнала,  
несут авторы статей.



Направляя рукопись в редакцию, авторы сохраняют свои права на данную
рукопись и при этом передают учредителям и редколлегии журнала неисключительные права на
издание статьи на русском языке 
(или на языке статьи, если он отличен от рус\-ско\-го) и~на перевод ее на английский
язык, а~также на
ее распространение в России и за рубежом. 
Каждый автор должен представить в~редакцию подписанный 
с~его стороны <<Лицензионный договор о~передаче неисключительных прав 
на использование произведения>>, текст которого размещен по адресу 
{\sf http://www.ipiran.ru/publications/licence.doc}. 
Этот договор может быть пред\-став\-лен в~бумажном (в~2-х экз.)\ 
или в~электронном виде (отсканированная копия заполненного и~подписанного документа).




Редколлегия вправе запросить у авторов экспертное заключение о возможности
пуб\-ли\-ка\-ции пред\-став\-лен\-ной статьи в открытой печати.\\[-13.5pt]

\item К статье прилагаются данные автора (авторов) (см.\ п.~8). При наличии нескольких
авторов указывается фамилия автора, ответственного за переписку с редакцией.\\[-13.5pt]

\item Редакция журнала осуществляет экспертизу присланных статей в соответствии с
принятой в журнале процедурой рецензирования.

Возвращение рукописи на доработку не означает ее принятия к печати.

Доработанный вариант с ответом на замечания рецензента необходимо прислать в
редакцию.\\[-13.5pt]

\item Решение редколлегии о публикации статьи или ее отклонении сообщается авторам.

Редколлегия может также направить авторам текст рецензии на их статью. Дискуссия по
поводу отклоненных статей не ведется.\\[-13.5pt]

%\pagebreak

\item Редактура статей высылается авторам для просмотра. Замечания к редактуре должны
быть присланы авторами в кратчайшие сроки.\\[-13.5pt]

\item Рукопись предоставляется в электронном виде в форматах MS WORD (.doc или
.docx) или \LaTeX\  (.tex), дополнительно~--- в формате .pdf, на дискете, лазерном диске
или электронной почтой. Предоставление бумажной рукописи необязательно.\\[-13.5pt]

\item При подготовке рукописи в MS Word рекомендуется использовать следующие
настройки.

Параметры страницы:
формат~--- А4; ориентация~--- книжная; поля (см): внутри~--- 2,5, снаружи~--- 1,5,
сверху~--- 2, снизу~--- 2, от края до нижнего колонтитула~--- 1,3.

Основной текст: стиль~--- <<Обычный>>, шрифт~--- Times New Roman, размер~---
14~пунк\-тов, абзацный отступ~--- 0,5~см, 1,5~интервала, выравнивание~--- по ширине.

\pagebreak

\def\leftkol{Правила подготовки рукописей  для публикации в журнале
<<Информатика и её применения>>}

\def\rightkol{Правила подготовки рукописей  для публикации в журнале
<<Информатика и её применения>>}



Рекомендуемый объем рукописи~--- не свыше 10~страниц указанного формата.
При превышении указанного объема редколлегия вправе потребовать от 
автора сокращения объема рукописи.


Сокращения слов, помимо стандартных, не допускаются. Допускается минимальное
количество аббревиатур.


Все страницы рукописи нумеруются.

Шаблоны оформления представлены в интернете:

\noindent
 {\sf
http://www.ipiran.ru/journal/template\_iiep\_ssi\_2024.zip}\\[-14pt]

\item Статья должна содержать следующую информацию на {\bfseries\textit{русском и
английском языках}}:\\[-16pt]

\begin{itemize}
\item название статьи;\\[-15pt]
\item Ф.И.О.\ авторов, на английском можно только имя и фамилию;\\[-15pt]
\item место работы, с указанием почтового адреса организации и электронного адреса каждого
автора;\\[-15pt]
\item сведения об авторах, в соответствии с форматом, образцы которого
представлены на страницах:



\def\leftfootline{\small{\textbf{\thepage}
\hfill ИНФОРМАТИКА И ЕЁ ПРИМЕНЕНИЯ\ \ \ том\ 18\ \ \ выпуск\ 3\ \ \ 2024}
}%
 \def\rightfootline{\small{ИНФОРМАТИКА И ЕЁ ПРИМЕНЕНИЯ\ \ \ том\ 18\ \ \ выпуск\ 3\ \ \ 2024
\hfill \textbf{\thepage}}}



{\sf http://www.ipiran.ru/journal/issues/2013\_07\_01/authors.asp} и

{\sf http://www.ipiran.ru/journal/issues/2013\_07\_01\_eng/authors.asp};
\item аннотация (не менее 100~слов на каждом из языков). Аннотация~--- это краткое
резюме работы, которое может публиковаться отдельно. Она является основным
источником информации в~ин\-фор\-ма\-ци\-он\-ных системах и базах данных. Английская
аннотация должна быть оригинальной, может не быть дословным переводом русского
текста и должна быть написана хорошим английским языком. В~аннотации не должно
быть ссылок на литературу и, по возможности, формул;\\[-15pt]
\item ключевые слова~--- желательно из принятых в мировой
на\-уч\-но-тех\-ни\-че\-ской литературе тематических тезаурусов. Предложения не
могут быть ключевыми словами;\\[-15pt]
\item источники финансирования работы (ссылки на гранты, проекты,
поддерживающие организации и~т.\,п.).
\end{itemize}



%\pagebreak

\item  Требования к спискам литературы.\\[-14pt]

Ссылки на литературу в тексте статьи нумеруются (в квадратных скобках) и
располагаются в каждом из списков литературы в порядке  первых упоминаний. Если источник имеет DOI и/или EDN,
то их необходимо указывать.

Списки литературы представляются в двух вариантах:\\[-14pt]


\noindent
\begin{enumerate}[(1)]
\item \textbf{Список литературы к русскоязычной части}. Русские и английские
работы~---  на языке и в алфавите оригинала;\\[-14.5pt]
\item  \textbf{References}. Русские работы и работы на других языках~--- в латинской
транслитерации с переводом на английский язык; английские работы и работы на других
языках~--- на языке оригинала.
\end{enumerate}

Необходимо для составления списка ``References'' пользоваться размещенной на сайте
{\sf http://www. translit.net/ru/bgn/} бесплатной программой транслитерации русского
 текста в~латиницу. %, при этом в~за\-клад\-ке <<варианты\ldots>> следует выбратьопцию BGN.

Список литературы ``References'' приводится полностью отдельным блоком, повторяя все
позиции из списка литературы к русскоязычной части, независимо от того, имеются или
нет в нем иностранные источники. Если в списке литературы к русскоязычной части есть
ссылки на иностранные публикации, набранные латиницей, они полностью повторяются в
списке ``References''.

Ниже приведены примеры ссылок на различные виды публикаций в списке ``References''.

\def\leftfootline{\small{\textbf{\thepage}
\hfill ИНФОРМАТИКА И ЕЁ ПРИМЕНЕНИЯ\ \ \ том\ 18\ \ \ выпуск\ 3\ \ \ 2024}
}%
 \def\rightfootline{\small{ИНФОРМАТИКА И ЕЁ ПРИМЕНЕНИЯ\ \ \ том\ 18\ \ \ выпуск\ 3\ \ \ 2024
\hfill \textbf{\thepage}}}

{\small

\noindent
\textbf{Описание статьи из журнала:}

\Aue{Zagurenko, A.\,G., V.\,A.~Korotovskikh, A.\,A.~Kolesnikov, A.\,V.~Timonov, and D.\,V.~Kardymon}. 2008.
Tekhniko-ekonomicheskaya optimizatsiya dizayna gidrorazryva plasta [Technical and
economic optimization of the design
of hydraulic fracturing]. \textit{Neftyanoe hozyaystvo} [\textit{Oil Industry}] 11:54--57.

\Aue{Zhang, Z., and D.~Zhu}. 2008. Experimental research on the localized
electrochemical micromachining. \textit{Russ. J.~Electrochem.}  44(8):926--930.
{\sf doi:10.1134/S1023193508080077}.

\noindent
\textbf{Описание статьи из электронного журнала:}

\Aue{Swaminathan, V., E.~Lepkoswka-White, and B.\,P.~Rao}. 1999. Browsers or buyers in cyberspace? An
investigation of electronic factors influencing electronic exchange. \textit{JCMC}
5(2). Available at: {\sf http://www.ascusc.org/jcmc/vol5/issue2/} (accessed April~28, 2011).

\def\leftkol{Правила подготовки рукописей  для публикации в журнале
<<Информатика и её применения>>}

\def\rightkol{Правила подготовки рукописей  для публикации в журнале
<<Информатика и её применения>>}


\noindent
\textbf{Описание статьи из продолжающегося издания (сборника трудов):}

\Aue{Astakhov, M.\,V., and T.\,V.~Tagantsev}. 2006. Eksperimental'noe
issledovanie prochnosti soedineniy ``stal'--kompozit'' [Experimental study of
the strength of joints ``steel--composite'']. \textit{Trudy MGTU
``Matematicheskoe modelirovanie slozhnykh tekh\-ni\-che\-skikh sistem''}
[\textit{Bauman MSTU ``Mathematical Modeling of Complex Technical
Systems'' Proceedings}]. 593:125--130.


\pagebreak



\noindent
\textbf{Описание материалов конференций:}

\Aue{Usmanov, T.\,S., A.\,A.~Gusmanov, I.\,Z.~Mullagalin, R.\,Ju.~Muhametshina, A.\,N.~Chervyakova, and
A.\,V.~Sveshnikov}. 2007. Osobennosti proektirovaniya razrabotki mestorozhdeniy
s primeneniem gidrorazryva
plasta [Features of the design of field development with the use of hydraulic fracturing].
\textit{Trudy 6-go
Mezhdu\-na\-rod\-no\-go Simpoziuma ``Novye resursosberegayushchie tekhnologii nedropol'zovaniya i povysheniya
neftegazootdachi''} [\textit{6th  Symposium (International) ``New Energy Saving Subsoil Technologies and
the Increasing of the Oil and Gas Impact'' Proceedings}]. Moscow. 267--272.



\def\leftfootline{\small{\textbf{\thepage}
\hfill ИНФОРМАТИКА И ЕЁ ПРИМЕНЕНИЯ\ \ \ том\ 18\ \ \ выпуск\ 3\ \ \ 2024}
}%
 \def\rightfootline{\small{ИНФОРМАТИКА И ЕЁ ПРИМЕНЕНИЯ\ \ \ том\ 18\ \ \ выпуск\ 3\ \ \ 2024
\hfill \textbf{\thepage}}}



\noindent
\textbf{Описание книги (монографии, сборники):}



Lindorf, L.\,S., and L.\,G.~Mamikoniants, eds. 1972.
\textit{Ekspluatatsiya turbogeneratorov s neposredstvennym
okhlazhdeniem} [\textit{Operation of turbine generators with direct cooling}].
Moscow: Energy Publs. 352~p.


\Aue{Latyshev, V.\,N.} 2009. \textit{Tribologiya rezaniya. Kn.~1: Friktsionnye protsessy
pri rezanii metallov}
[\textit{Tribology of cutting. Vol.~1: Frictional processes in metal cutting}]. Ivanovo: Ivanovskii
State Univ. 108~p.

\def\leftkol{Правила подготовки рукописей  для публикации в журнале
<<Информатика и её применения>>}

\def\rightkol{Правила подготовки рукописей  для публикации в журнале
<<Информатика и её применения>>}

\noindent
\textbf{Описание переводной книги}
(в списке литературы к русскоязычной части необходимо указать:~/ Пер.\ с англ.~---
после названия книги, а в конце ссылки указать оригинал книги в круглых скобках):
\begin{enumerate}[1.]
\item  В русскоязычной части:

\def\leftfootline{\small{\textbf{\thepage}
\hfill ИНФОРМАТИКА И ЕЁ ПРИМЕНЕНИЯ\ \ \ том\ 18\ \ \ выпуск\ 3\ \ \ 2024}
}%
 \def\rightfootline{\small{ИНФОРМАТИКА И ЕЁ ПРИМЕНЕНИЯ\ \ \ том\ 18\ \ \ выпуск\ 3\ \ \ 2024
\hfill \textbf{\thepage}}}

\Au{Тимошенко С.\,П., Янг Д.\,Х., Уивер~У.}
Колебания в инженерном деле~/ Пер.\ с англ.~--- М.: Машиностроение, 1985. 472~с.
(\Au{Timoshenko~S.\,P., Young~D.\,H., Weaver~W.}
Vibration problems in engineering.~--- 4th ed.~--- New York, NY, USA: Wiley, 1974. 521~p.)\\[-13.5pt]
\item  В англоязычной части:

\Aue{Timoshenko, S.\,P., D.\,H.~Young, and W.~Weaver}.
1974. \textit{Vibration problems in engineering}. 4th ed. New York: 
Wiley. 521~p.
\end{enumerate}

\vspace*{-3pt}


\noindent
\textbf{Описание неопубликованного документа:}


\Aue{Latypov, A.\,R., M.\,M.~Khasanov, and V.\,A.~Baikov}.
2004 (unpubl.). Geologiya i~dobycha (NGT GiD) [Geology and production (NGT GiD)]. Certificate on official registration of the computer program
No.\,2004611198. 

\noindent
\textbf{Описание интернет-ресурса:}


Pravila tsitirovaniya istochnikov [Rules for the citing of sources]. Available at: {\sf
http://www.scribd.com/doc/1034528/} (accessed February~7, 2011).

%\pagebreak

\noindent
\textbf{Описание диссертации или автореферата диссертации:}

\Aue{Semenov, V.\,I.}
2003. Matematicheskoe modelirovanie plazmy v sisteme kompaktnyy tor [Mathematical
modeling of the plasma in the compact torus].  Moscow.  D.Sc.\ Diss. 272~p.

\Aue{Kozhunova, O.\,S.} 2009. Tekhnologiya razrabotki semanticheskogo
slovarya informatsionnogo monitoringa [Technology of development of
semantic dictionary of information monitoring system].  Moscow: IPI RAN. PhD Thesis. 23~p.


\noindent
\textbf{Описание ГОСТа:}

GOST 8.586.5-2005. 2007. Metodika vypolneniya izmereniy. Izmerenie raskhoda i~kolichestva zhidkostey i~gazov
s~pomoshch'yu standartnykh suzhayushchikh ustroystv [Method of measurement.
Measurement of flow rate and volume of liquids and gases by means of orifice devices]. Moscow:
Standardinform  Publs. 10~p.

\noindent
\textbf{Описание патента:}

\Aue{Bolshakov, M.\,V., A.\,V.~Kulakov, A.\,N.~Lavrenov, and M.\,V.~Palkin}.
2006. Sposob orientirovaniya po krenu letatel'nogo
apparata s opti\-che\-skoy golovkoy
samonavedeniya [The way to orient on the roll of aircraft with optical homing head].
Patent RF No.\,2280590.
}

\item Присланные в редакцию материалы авторам не возвращаются.\\[-13.5pt]

\item При отправке файлов по электронной почте просим придерживаться следующих
правил:
\begin{itemize}
\item указывать в поле subject (тема) название журнала и фамилию автора;\\[-13.5pt]
\item указывать в тексте письма название статьи, авторов и~журнал, в~который направляется статья;\\[-13.5pt]
\item использовать attach (присоединение);\\[-13.5pt]
\item в состав электронной версии статьи должны входить: файл, содержащий текст
статьи, и файл(ы), содержащий(е) иллюстрации.\\[-13.5pt]
\end{itemize}

\item Журнал <<Информатика и её применения>> является некоммерческим изданием.
Плата за публикацию не взимается, гонорар авторам не выплачивается.
\end{enumerate}



\def\leftfootline{\small{\textbf{\thepage}
\hfill ИНФОРМАТИКА И ЕЁ ПРИМЕНЕНИЯ\ \ \ том\ 18\ \ \ выпуск\ 3\ \ \ 2024}
}%
 \def\rightfootline{\small{ИНФОРМАТИКА И ЕЁ ПРИМЕНЕНИЯ\ \ \ том\ 18\ \ \ выпуск\ 3\ \ \ 2024
\hfill \textbf{\thepage}}}


\vspace*{-1mm}

\begin{center}

\textbf{Адрес редакции журнала <<Информатика и её применения>>:} \\




Москва 119333, ул.~Вавилова, д.~44, корп.~2, ФИЦ ИУ РАН\\[-10pt]

\

Тел.: +7\,(499)\,135-86-92\ \ Факс:  +7\,(495)\,930-45-05\\[-10pt]

 \

e-mail:   {\sf iiep@frccsc.ru} (Стригина Светлана Николаевна)\\[-10pt]

\

{\sf http://www.ipiran.ru/journal/issues/}
\end{center}
}


\def\leftkol{Правила подготовки рукописей  для публикации в журнале
<<Информатика и её применения>>}

\def\rightkol{Правила подготовки рукописей  для публикации в журнале
<<Информатика и её применения>>}


\def\leftfootline{\small{\textbf{\thepage}
\hfill ИНФОРМАТИКА И ЕЁ ПРИМЕНЕНИЯ\ \ \ том\ 18\ \ \ выпуск\ 3\ \ \ 2024}
}%
 \def\rightfootline{\small{ИНФОРМАТИКА И ЕЁ ПРИМЕНЕНИЯ\ \ \ том\ 18\ \ \ выпуск\ 3\ \ \ 2024
\hfill \textbf{\thepage}}} 
\def\stat{podg-e}
{%\hrule\par
%\vskip 7pt % 7pt
\vspace*{-24pt}
\raggedleft\Large \bf%\baselineskip=3.2ex
Requirements for manuscripts submitted to Journal
``Informatics~and~Applications'' \vskip 8pt
    \hrule
    \par
\vskip 21pt plus 6pt minus 3pt }

\label{st\stat}

\def\tit{\ }

\def\aut{\ }
\def\auf{\ }

\def\leftkol{\ }

\def\rightkol{\ }
%Requirements for manuscripts submitted to Journal
%``Informatics~and~Applications''}

\titele{\tit}{\aut}{\auf}{\leftkol}{\rightkol}

\def\leftfootline{\small{\textbf{\thepage}
\hfill INFORMATIKA I EE PRIMENENIYA~--- INFORMATICS AND APPLICATIONS\ \ \ 2019\
\ \ volume~13\ \ \ issue\ 4}
}%
 \def\rightfootline{\small{INFORMATIKA I EE PRIMENENIYA~--- INFORMATICS AND APPLICATIONS\ \ \ 2019\ \ \ volume~13\ \ \ issue\ 4
\hfill \textbf{\thepage}}}

\vspace*{-60pt}

{\small

\noindent
Journal ``Informatics and Applications'' (Inform.\ Appl.)
publishes theoretical, review, and discussion
articles on the research and development in the
field of informatics and its applications.

The journal is published in Russian.
By a special decision of the editorial
board, some articles can be published in English.


The topics covered include the following areas:
\begin{itemize}
               \item
     theoretical fundamentals of informatics; \\[-14pt]
\item
mathematical methods for studying complex systems and processes; \\[-14pt]
\item
information systems and networks;\\[-14pt]
\item
information technologies; and \\[-14pt]
\item
architecture and software of computational complexes and networks. \\[-14pt]
\end{itemize}

\noindent
\begin{enumerate}[1.]
\item The Journal publishes original articles which have not been published before and are not
intended for simultaneous publication in other editions. An article submitted to the Journal must not violate the
Copyright law. Sending the manuscript to the Editorial Board, the authors retain all rights of the
owners of the manuscript and transfer the nonexclusive rights to publish the article in Russian
(or the language of the article, if not Russian) and its distribution in Russia and abroad to the
Founders and the Editorial Board. Authors should submit a letter to the Editorial Board in the
following form:

{\bfseries\textit{Agreement on the transfer of rights to publish:}}

``\textit{We, the undersigned authors of the manuscript ``\ldots'', pass to the
Founder and the Editorial Board of the Journal ``Informatics and Applications''
the nonexclusive right to publish the manuscript of the article in Russian (or
in English) in both print and electronic versions of the Journal. We affirm
that this publication does not violate the Copyright of other persons or
organizations.}

\textit{Author(s) signature(s): (name(s), address(es), date).}

This agreement should be submitted in paper form or in the form of a scanned copy (signed by
the authors).


%The Editorial Board has the right to request from the authors an official expert conclusion that
%the submitted article has no secret data prohibited for publication. \\[-13.5pt]
\item
A submitted article should be attached with \textbf{the data on the author(s)} (see item~8). If
there are several authors, the contact person should be indicated who is responsible for
correspondence with the Editorial Board and other authors about revisions and final approval
of the proofs.\\[-13.5pt]

\item The Editorial Board of the Journal examines the article according to the established
reviewing procedure. If the authors receive their article for correction after reviewing, it does not
mean that the article is approved for publication. The corrected article should be sent to the
Editorial Board for the subsequent review and approval.\\[-13.5pt]

\item The decision on the article publication or its rejection is communicated to the authors. The
Editorial Board may also send the reviews on the submitted articles to the authors. Any
discussion upon the rejected articles is not possible.\\[-13.5pt]

\item The edited articles will be sent to the authors for proofread. The comments of the authors
to the edited text of the article should be sent to the Editorial Board as soon as possible.\\[-13.5pt]

\item The manuscript of the article should be presented electronically in the MS WORD (.doc or
.docx) or \LaTeX\ (.tex) formats, and additionally in the .pdf format. All documents
 may be sent
by e-mail or provided on a CD or diskette. A~hard copy submission is not necessary.\\[-13.5pt]

\item The recommended typesetting instructions for manuscript.

Pages parameters: format A4, portrait orientation, document margins (cm): left~--- 2.5, right~---
1.5, above~--- 2.0, below~--- 2.0, footer 1.3.

Text: font~---Times New Roman, font size~--- 14, paragraph indent~--- 0.5, line spacing~--- 1.5,
justified alignment.

The recommended manuscript size: not more than 15~pages of the specified format.
If the specified size exceeded, the editorial board is entitled to require the author
to reduce the manuscript.

Use only standard abbreviations. Avoid  abbreviations in the title and
abstract. The full term for which an abbreviation stands should precede
its first use in the text unless it is a standard unit of measurement.

All pages of the manuscript should be numbered.

The templates for the manuscript typesetting are presented on site: {\sf
http://www.ipiran.ru/journal/template.doc}.\\[-13.5pt]


%\def\leftkol{Requirements for manuscripts submitted to Journal
%``Informatics~and~Applications''}

\item The articles should enclose data both in \textbf{Russian and English}:
\begin{itemize}
\item title;\\[-13.5pt]
\item author's name and surname;\\[-13.5pt]
\item affiliation~--- organization, its address with ZIP code, city, country, and
official e-mail address;\\[-13.5pt]
\item data on authors according to the format: (see site)

{\sf http://www.ipiran.ru/journal/issues/2013\_07\_01/authors.asp}  and

{\sf  http://www.ipiran.ru/journal/issues/2013\_07\_01\_eng/authors.asp};\\[-13.5pt]

\pagebreak

\def\leftfootline{\small{\textbf{\thepage}
\hfill INFORMATIKA I EE PRIMENENIYA~--- INFORMATICS AND APPLICATIONS\ \ \ 2019\
\ \ volume~13\ \ \ issue\ 4}
}%
 \def\rightfootline{\small{INFORMATIKA I EE PRIMENENIYA~--- INFORMATICS AND APPLICATIONS\ \ \ 2019\ \ \ volume~13\ \ \ issue\ 4
\hfill \textbf{\thepage}}}


%\def\leftkol{Requirements for manuscripts submitted to Journal
%``Informatics~and~Applications''}

%\def\rightkol{Requirements for manuscripts submitted to Journal
%``Informatics~and~Applications''}



\item abstract (not less than 100 words) both in Russian and in English. Abstract is a short
summary of the article that can be published separately. The abstract is the
main source of information on the article and it could be included in leading information
systems and data bases. The abstract in English has to be an original text and should
not be an exact translation of the Russian one. Good English is required.
In abstracts, avoid references and formulae;\\[-13.5pt]
\item indexing is performed on the basis of keywords. The use of keywords from the
internationally accepted thematic Thesauri is recommended.

%\def\leftkol{Requirements for manuscripts submitted to Journal
%``Informatics~and~Applications''}

%\def\rightkol{Requirements for manuscripts submitted to Journal
%``Informatics~and~Applications''}

Important! Keywords must not be sentences;
\item Acknowledgments.
\end{itemize}

\item References. Russian references have to be presented both in English translation and Latin
transliteration (refer {\sf http://www.translit.net/ru/bgn/}).

Please take into account the following examples of Russian references appearance:

\noindent
\textbf{Article in journal:}

\Aue{Zhang, Z., and D.~Zhu}. 2008. Experimental research on the localized electrochemical
micromachining.
\textit{Rus. J.~Electrochem.}  44(8):926--930. {\sf doi:10.1134/S1023193508080077}.


\noindent
\textbf{Journal article in electronic format:}

\Aue{Swaminathan, V., E.~Lepkoswka-White, and B.\,P.~Rao}. 1999. Browsers or buyers in
cyberspace? An
investigation of electronic factors influencing electronic exchange. \textit{JCMC}
5(2). Available at: {\sf http://www.ascusc.org/jcmc/vol5/issue2/} (accessed April~28, 2011).




\noindent
\textbf{Article from the continuing publication (collection of works, proceedings):}

\Aue{Astakhov, M.\,V., and T.\,V.~Tagantsev}. 2006. Eksperimental'noe
issledovanie prochnosti soedineniy ``stal'--kompozit'' [Experimental study of
the strength of joints ``steel--composite'']. \textit{Trudy MGTU
``Matematicheskoe modelirovanie slozhnykh tekh\-ni\-che\-skikh sistem''}
[\textit{Bauman MSTU ``Mathematical Modeling of Complex Technical
Systems'' Proceedings}]. 593:125--130.

\def\leftfootline{\small{\textbf{\thepage}
\hfill INFORMATIKA I EE PRIMENENIYA~--- INFORMATICS AND APPLICATIONS\ \ \ 2019\
\ \ volume~13\ \ \ issue\ 4}
}%
 \def\rightfootline{\small{INFORMATIKA I EE PRIMENENIYA~--- INFORMATICS AND APPLICATIONS\ \ \ 2019\ \ \ volume~13\ \ \ issue\ 4
\hfill \textbf{\thepage}}}

\def\leftkol{Requirements for manuscripts submitted to Journal
``Informatics~and~Applications''}

\def\rightkol{Requirements for manuscripts submitted to Journal
``Informatics~and~Applications''}

\noindent
\textbf{Conference proceedings:}

\Aue{Usmanov, T.\,S., A.\,A.~Gusmanov, I.\,Z.~Mullagalin, R.\,Ju.~Muhametshina,
A.\,N.~Chervyakova, and
A.\,V.~Sveshnikov}. 2007. Osobennosti proektirovaniya razrabotki mestorozhdeniy
s primeneniem gidrorazryva
plasta [Features of the design of field development with the use of hydraulic fracturing].
\textit{Trudy 6-go
Mezhdu\-na\-rod\-no\-go Simpoziuma ``Novye resursosberegayushchie tekhnologii
nedropol'zovaniya i povysheniya
neftegazootdachi''} [\textit{6th  Symposium (International) ``New Energy Saving Subsoil
Technologies and
the Increasing of the Oil and Gas Impact'' Proceedings}]. Moscow. 267--272.


\noindent
\textbf{Books and other monographs:}




Lindorf, L.\,S., and L.\,G.~Mamikoniants, eds. 1972.
\textit{Ekspluatatsiya turbogeneratorov s neposredstvennym
okhlazhdeniem} [\textit{Operation of turbine generators with direct cooling}].
Moscow: Energy Publs. 352~p.


%\Aue{Latyshev, V.\,N.} 2009. \textit{Tribologiya rezaniya. Kn.~1: Frikcionnye prosessy
%pri rezanii metallov}
%[\textit{Tribology of cutting. Vol.~1: Frictional processes in metal cutting}]. Ivanovo: Ivanovskii
%State Univ. 108~p.


%\noindent
%\textbf{Unpublished material:}

%\Aue{Latypov, A.\,R., M.\,M.~Khasanov, and V.\,A.~Baikov}.
%2004. Geology and production (NGT GiD). Certificate on official registration of the computer
%program
%No.\,2004611198. (In Russian, unpubl.)

%\noindent
%\textbf{Internet-source:}

%APA Style. 2011. Available at: {\sf http://www.apastyle.org/apa-style-help.aspx} (accessed
%February~5, 2011).

%Pravila citirovaniya istochnikov [Rules for the citing of sources]. Available at: {\sf
%http://www.scribd.com/doc/1034528/} (accessed February~7, 2011).


\noindent
\textbf{Dissertation and Thesis:}

%\Aue{Semenov, V.\,I.}
%2003. Matematicheskoe modelirovanie plazmy v sisteme kompaktnyy tor. [Mathematical
%modeling of the plasma in the compact torus]. D.Sc.\ Diss. Moscow. 272~p.

\Aue{Kozhunova, O.\,S.} 2009. Tekhnologiya razrabotki semanticheskogo
slovarya informatsionnogo monitoringa [Technology of development of
semantic dictionary of information monitoring system]. PhD Thesis. Moscow: IPI RAN. 23~p.


\noindent
\textbf{State standards and patents:}

GOST 8.586.5-2005. 2007. Metodika vypolneniya izmereniy. Izmerenie raskhoda i~kolichestva
zhidkostey i gazov 
s~pomoshch'yu standartnykh suzhayushchikh ustroystv [Method of measurement.
Measurement of flow rate and volume of liquids and gases by means of orifice devices]. M.:
Standardinform
Publs. 10~p.

%\noindent
%\textbf{Patent:}

\Aue{Bolshakov, M.\,V., A.\,V.~Kulakov, A.\,N.~Lavrenov, and M.\,V.~Palkin}.
2006. Sposob orientirovaniya po krenu letatel'nogo
apparata s opti\-che\-skoy golovkoy
samonavedeniya [The way to orient on the roll of aircraft with optical homing head].
Patent RF No.\,2280590.

References in Latin transcription are presented in the original language.

References in the text are numbered according to the order of their
first appearance; the number is
placed in square brackets. All items from the reference list should be
cited.\\[-13.5pt]

\item Manuscripts and additional materials are not returned to Authors by the Editorial Board.\\[-13.5pt]

\item Submissions of files by e-mail must include:\\[-13.5pt]
\begin{itemize}
\item   the journal title and author's name in the ``Subject'' field; \\[-13.5pt]
\item   an article and additional materials have to be attached using the ``attach'' function;\\[-13.5pt]
\item   an electronic version of the article should contain the file with the text and a separate file
with figures.\\[-13.5pt]
\end{itemize}

\item ``Informatics and Applications'' journal is not a profit publication. There are no
charges for the authors as well as there are no royalties.\\[-13.5pt]
\end{enumerate}

\def\leftfootline{\small{\textbf{\thepage}
\hfill INFORMATIKA I EE PRIMENENIYA~--- INFORMATICS AND APPLICATIONS\ \ \ 2019\
\ \ volume~13\ \ \ issue\ 4}
}%
 \def\rightfootline{\small{INFORMATIKA I EE PRIMENENIYA~--- INFORMATICS AND APPLICATIONS\ \ \ 2019\ \ \ volume~13\ \ \ issue\ 4
\hfill \textbf{\thepage}}}

\def\leftkol{Requirements for manuscripts submitted to Journal
``Informatics~and~Applications''}

\def\rightkol{Requirements for manuscripts submitted to Journal
``Informatics~and~Applications''}


%\vspace*{5mm}


\begin{center}
\textbf{Editorial Board address:} \\

%ABOUT AUTHORS



FRC CSC RAS, 44, block~2, Vavilov Str., Moscow 119333, Russia\\[-10pt]

\

Ph.: +7\,(499)\,135\,86\,92,\ \ Fax: +7\,(495)\,930\,45\,05\\[-10pt]

\

 e-mail: {\sf rust@ipiran.ru} (to Prof.\ Rustem Seyful-Mulyukov)\\[-10pt]

\

 {\sf http://www.ipiran.ru/english/journal.asp}
\end{center}
 }
%\thispagestyle{myheadings}

\def\leftkol{Requirements for manuscripts submitted to Journal
``Informatics~and~Applications''}

\def\rightkol{Requirements for manuscripts submitted to Journal
``Informatics~and~Applications''}

\def\leftfootline{\small{\textbf{\thepage}
\hfill INFORMATIKA I EE PRIMENENIYA~--- INFORMATICS AND APPLICATIONS\ \ \ 2019\
\ \ volume~13\ \ \ issue\ 4}
}%
 \def\rightfootline{\small{INFORMATIKA I EE PRIMENENIYA~--- INFORMATICS AND APPLICATIONS\ \ \ 2019\ \ \ volume~13\ \ \ issue\ 4
\hfill \textbf{\thepage}}}

 \label{end\stat}

\newpage



%\include{ipi-ind}

%\tableofcontents

\end{document}





%%%%%%%%%%%%%%%%%%%%%%

%\newcommand{\Ack}{\subsection*{\protect\large\bf Acknowledgments}}

%\vphantom*{\int\limits_0^T}

{ \begin{center}  %fig1
 \vspace*{6pt}
    \mbox{%
 \epsfxsize=79mm 
 \epsfbox{gru-1.eps}
 }

\end{center}



\noindent
{{\figurename~1}\ \ \small{
}}}

%\vspace*{6pt}

\addtocounter{figure}{1}