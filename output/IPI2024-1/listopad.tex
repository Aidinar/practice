\def\stat{listopad}

\def\tit{ЖИЗНЕННЫЙ ЦИКЛ МЕТОДОЛОГИИ ПОСТРОЕНИЯ РЕФЛЕКСИВНО-АКТИВНЫХ 
СИСТЕМ ИСКУССТВЕННЫХ ГЕТЕРОГЕННЫХ ИНТЕЛЛЕКТУАЛЬНЫХ АГЕНТОВ$^*$}

\def\titkol{Жизненный цикл методологии построения РАСИГИА} %рефлексивно-активных систем искусственных гетерогенных интеллектуальных агентов}

\def\aut{С.\,В.~Листопад$^1$}

\def\autkol{С.\,В.~Листопад}

\titel{\tit}{\aut}{\autkol}{\titkol}

\index{Листопад С.\,В.}
\index{Listopad S.\,V.}


{\renewcommand{\thefootnote}{\fnsymbol{footnote}} \footnotetext[1]
{Исследование выполнено за счет гранта Российского научного фонда №\,23-21-00218, 
{\sf https://rscf.ru/project/23-21-00218/}.}}


\renewcommand{\thefootnote}{\arabic{footnote}}
\footnotetext[1]{Федеральный исследовательский центр <<Информатика и~управ\-ле\-ние>> Российской академии наук, 
\mbox{ser-list-post@yandex.ru}}

%\vspace*{-12pt}

  
  

  \Abst{Представлена темпоральная структура (жизненный цикл) методологии построения 
рефлексивно-активных систем искусственных гетерогенных интеллектуальных агентов (\mbox{РАСИГИА}). 
Такие системы предназначены для компьютерного моделирования процессов и~эффектов, 
возникающих при решении практических проблем коллективами специалистов под 
руководством лица, принимающего решения. Искусственные гетерогенные 
интеллектуальные агенты реф\-лек\-сив\-но-ак\-тив\-ных сис\-тем~--- активные субъекты, 
способные к~рассуждениям, коммуникации и~рефлексии как умению моделировать 
рассуждения других агентов системы и~себя самих. Моделирование рефлексивных процессов 
обеспечивает выработку агентами согласованного представления об объекте управ\-ле\-ния, 
цели коллективной работы и~нормах взаимодействия, позволяя системе в~ходе 
самоорганизации генерировать заново релевантный гибридный интеллектуальный метод 
решения очередной проб\-лемы.} 
  
  \KW{рефлексия; методология; рефлексивно-активная сис\-те\-ма искусственных 
гетерогенных интеллектуальных агентов; гибридная интеллектуальная многоагентная 
система; коллектив специалистов}

\DOI{10.14357/19922264240112}{GUAMVE}
  
%\vspace*{-6pt}


\vskip 10pt plus 9pt minus 6pt

\thispagestyle{headings}

\begin{multicols}{2}

\label{st\stat}

\section{Введение}

  Компьютерное моделирование процессов и~эффектов, возникающих при 
решении практических проблем коллективами специалистов, каждый из 
которых обладает собственным опытом, знаниями и~пониманием предметной 
области,~--- перспективное на\-прав\-ле\-ние научных разработок, которое 
Д.\,А.~Поспелов выделял как одну из десяти горячих точек в~исследованиях по 
искусственному\linebreak интеллекту~[1]. Для компьютерного моделирования 
рас\-суж\-де\-ний коллективов специалистов предлагается создание \mbox{РАСИГИА} 
в~рамках многоагентного подхода~[2] на основе модели 
\mbox{ги\-брид\-ных} интеллектуальных многоагентных сис\-тем~[3]. Агенты 
\mbox{РАСИГИА}~--- активные программные сущности, способные 
рас\-суж\-дать, взаимодействовать и~рефлексировать. Рефлексивное 
моделирование агентами друг друга обеспечивает выработку согласованного 
пред\-став\-ле\-ния об объекте управ\-ле\-ния, \mbox{цели} коллективной работы и~нормах 
взаимодействия, а~также эволюцию \mbox{РАСИГИА} в~ходе 
самоорганизации в~сильном смыс\-ле. В~на\-сто\-ящей работе рас\-смат\-ри\-ва\-ют\-ся 
вопросы создания методологии разработки сис\-тем такого класса, которая 
понимается как учение об организации продуктивной де\-я\-тель\-ности 
в~це\-лост\-ную сис\-те\-му с~чет\-ко определенными характеристиками, логической 
структурой и~процессом ее осуществления (темпоральной структурой)~[4]. 
Характеристики (особенности и~принципы) и~логическая структура (субъект, 
объект, предмет, методы, средства, результат) методологии разработки 
\mbox{РАСИГИА} рас\-смот\-ре\-ны в~[5]. Данная работа по\-свя\-ще\-на разработке 
жизненного цик\-ла (темпоральной структуры) предлагаемой методологии.

\begin{figure*} %fig1
\vspace*{1pt}
      \begin{center}
     \mbox{%
\epsfxsize=148.855mm 
\epsfbox{lis-1.eps}
}
\end{center}
%\vspace*{-9pt}

{\small Темпоральная структура методологии построения РАСИГИА: \textit{1}~--- этап методологии; \textit{2}~--- стадия методологии;
\textit{3}~--- граница фазы методологии; \textit{4}~--- отношение следования при нормальном завершении этапа;  
\textit{5}~--- возврат к~предыдущим этапам при выявлении допущенных на них недочетов}
\end{figure*}

\vspace*{-6pt}
  
\section{Темпоральная структура методологии}

\vspace*{-6pt}

  Укрупненно в~жизненном цикле методологии построения \mbox{РАСИГИА}, 
показанном на рисунке, могут быть выделены проектная, технологическая 
и~рефлексивная фазы, которые со\-сто\-ят из стадий и~этапов. Как видно, 
последовательное выполнение этапов методологии приводит к~же\-ла\-емо\-му 
результату лишь в~идеальном случае, когда проектировщик сразу получает всю 
необходимую достоверную информацию, имеет необходимый арсенал методов, 
не совершает ошибок ни на одном из этапов и,~по сути, заранее знает, какой 
должна быть разрабатываемая \mbox{РАСИГИА}. В~реальности на каждом из 
этапов могут обнаруживаться ранее допущенные недочеты, требующие 
возврата к~соответствующему этапу, их исправления и~повторного выполнения 
проделанной работы с~новыми исходными данными. В~определенном смыс\-ле 
такой подход представляет собой метод проб и~ошибок, и~чем слож\-нее 
проблема, для которой проектируется \mbox{РАСИГИА}, с~точ\-ки зрения 
конкретного коллектива разработчиков, тем больше будет возвратов в~ходе 
проектирования системы~[6]. Рас\-смот\-рим по\-дроб\-нее каждую из фаз 
методологии.



\section{Проектная фаза}

  Проектная фаза включает в~себя стадии концептуального описания проб\-ле\-мы и~моделирования, выполняемые сис\-тем\-ны\-ми аналитиками из коллектива 
разработчиков. В~рамках первой стадии фазы на доформальном, 
содержательном уровне рас\-смат\-ри\-ва\-ет\-ся проб\-ле\-ма как отрицательное 
отношение субъекта к~реальности~[6] и~проблемная ситуация как объективное 
стечение обстоятельств, обуслов\-ли\-ва\-ющее проб\-ле\-му. Данная стадия со\-сто\-ит из 
сле\-ду\-ющих этапов:
  \begin{itemize}
\item формулирование проб\-ле\-мы, ее предварительное описание в~ходе 
интервьюирования лица, при\-ни\-ма\-юще\-го решение, его советников и~активных 
групп на естественном языке с~использованием привычных для них 
определений и~формулировок~[7];
  \item определение проб\-ле\-ма\-ти\-ки, т.\,е.\ комплекса проб\-лем, связанных 
с~рас\-смат\-ри\-ва\-емой~[4], чтобы учесть создаваемые ее решением последствия 
для каж\-дой из них. Необходимо охватить весь круг участников проб\-лем\-ной 
ситуации (стейкхолдеров, заинтересованных лиц): непосредственных 
участников ситуации, пред\-ста\-ви\-те\-лей проб\-ле\-мо\-раз\-ре\-ша\-ющих 
и~проб\-ле\-мо\-со\-дер\-жа\-щих сис\-тем, же\-ла\-емых помощников или союзников, 
субъектов, связанных с~ситуацией юридически, лиц с~возможным негативным 
отношением к~решению проб\-ле\-мы~[6]. Для по\-стро\-ения проб\-ле\-ма\-ти\-ки может 
быть использована, например, технология Дж.~Уор\-фил\-да, подходы 
с~использованием метафор организации, взгляда на проблему стейкхолдером 
с~раз\-ных точек зрения, рас\-смот\-ре\-ния проб\-ле\-мы в~рамках различных парадигм 
(функциональной, объяснительной, освободительной, пост\-мо\-дер\-нист\-ской)~[4, 6]. 
Формируется древовидная или сетевая структура в~виде диаграммы связей, 
концептуальной кар\-ты или аналогичных инструментов;
  \item определение целей проектирования \mbox{РАСИГИА}, 
пред\-по\-ла\-га\-ющее проведение собеседований с~каж\-дым стейк\-хол\-де\-ром, 
выяснение их целей и~пожеланий, формирование и~структурирование 
множества целей в~виде дерева или сетевидной структуры и~его 
визуализация~[4, 6]. Выделяются следующие уровни целей: ожи\-да\-емые 
в~плановом периоде результаты; задачи, которые не будут решены 
в~рас\-смат\-ри\-ва\-емом периоде, но будет достигнут существенный прогресс на 
пути к~ним; не\-до\-сти\-жи\-мые идеалы, к~которым следует стремиться~[8];
  \item выбор критериев, т.\,е.\ до\-ступ\-ных для наблюдения и~измерения 
характеристик, опи\-сы\-ва\-ющих важ\-ные особенности объектов или процессов 
и~поз\-во\-ля\-ющих сравнивать \mbox{пред\-ла\-га\-емые} альтернативы, контролировать 
процесс решения~[6]. Со\-во\-куп\-ность критериев долж\-на быть релевантной 
количественной моделью выделенных ранее качественных целей. Отдельно 
выделяются ограничения, фик\-си\-ру\-ющие условия, которые не могут нарушаться 
при до\-сти\-же\-нии цели;
  \item оценка концептуального описания проб\-ле\-мы в~ходе специально 
спланированного эксперимента. Если существует коллектив специалистов, 
решающий на практике по\-став\-ле\-нную или схожие проб\-ле\-мы, он выступает 
образцом, прототипом создаваемой сис\-те\-мы агентов. В~этом случае 
выполняется наблюдение за работой такого коллектива в~рамках решения 
реальных или тренировочных проб\-лем и~оценка релевантности 
зафиксированной информации сведениям, полученным в~ходе предыду\-щих 
этапов. Если выявлено существенное рас\-хож\-де\-ние, выполняется возврат 
к~этапу, в~рамках которого были получены некорректные сведения. Сведения 
о~составе участников коллектива, вы\-де\-ля\-емых ими под\-проб\-ле\-мах, методах их 
решения используются на по\-сле\-ду\-ющих этапах проектирования 
\mbox{РАСИГИА} при по\-стро\-ении со\-от\-вет\-ст\-ву\-ющих моделей проб\-ле\-мы 
и~сис\-те\-мы <<как есть сейчас>>. Данные о~качестве принятых решений 
и~дли\-тель\-ности их выработки используются в~дальнейшем как показатель 
эффекта от разработки и~внед\-ре\-ния \mbox{РАСИГИА}. Если подобных 
коллективов нет или не\-воз\-мож\-но реализовать со\-от\-вет\-ст\-ву\-ющий эксперимент, 
данный этап отсутствует.
  \end{itemize}
  
  Стадия моделирования предполагает разработку формализованного описания 
проб\-ле\-мы, коллектива специалистов, ре\-ша\-юще\-го проб\-ле\-му на момент 
разработки \mbox{РАСИГИА}, если он существует, и~самой 
\mbox{РАСИГИА}. Модели строятся с~использованием визуального 
метаязыка~[9], что позволяет наглядно их изобразить, а~так\-же поз\-во\-ля\-ет 
с~использованием заранее заданных соответствий однозначно отоб\-ра\-зить 
графическое пред\-став\-ле\-ние моделей в~формальное символьное пред\-став\-ле\-ние, 
пригодное для компьютерной интерпретации. Данная стадия со\-сто\-ит из 
сле\-ду\-ющих этапов:
  \begin{itemize}
  \item моделирование проб\-ле\-мы, которое обеспечивает ее формальное 
пред\-став\-ле\-ние на макро- и~мик\-ро\-уров\-не. Мак\-ро\-уров\-не\-вая модель описывает 
проб\-ле\-му как <<чер\-ный ящик>>, отражая ее место в~ме\-та\-проб\-ле\-ме (проб\-ле\-ме 
более высокого уровня), свойства как целого и~связи с~другими проб\-ле\-ма\-ми 
ме\-та\-проб\-ле\-мы. Атрибуты проблемы на макроуровне~--- цели, критерии 
(включая ограничения), исходные данные и~идентификатор. Мик\-ро\-уров\-не\-вая 
модель раскрывает со\-став и~структуру проб\-ле\-мы, описывает ее под\-проб\-ле\-мы 
и~связи между ними. Для каждой под\-проб\-ле\-мы специфицируются цели, 
критерии, исходные данные и~идентификатор, выполняется поиск релевантных 
методов решения. Если такие методы найдены, дальнейшая декомпозиция 
под\-проб\-ле\-мы не требуется, иначе выполняется по\-стро\-ение ее мик\-ро\-уров\-не\-вой 
модели, т.\,е.\ модели более глубокого уров\-ня иерархии. Таким образом, 
формируется многоуровневая иерархическая структура по\-став\-лен\-ной 
проб\-лемы;
  \item моделирование коллектива, которое отражает ситуацию решения 
проб\-ле\-мы <<как есть сейчас>> со всеми ее преимуществами и~недостатками. 
Модель коллектива~--- основа, образец для проектирования \mbox{РАСИГИА} и~оценки эф\-фек\-тив\-ности 
альтернативных конфигураций \mbox{РАСИГИА}. 
При моделировании коллектива специалистов фиксируется его со\-став в~виде 
множества ролей участников, час\-ти проб\-ле\-мы, ре\-ша\-емые каж\-дым из 
участников с~определенной ролью, знания и~методы, ис\-поль\-зу\-емые 
участниками для решения своей части проб\-ле\-мы, а~так\-же порядок и~нормы 
взаимодействия участников коллектива; 
  \item моделирование \mbox{РАСИГИА}, фор\-ми\-ру\-ющее идеализированное 
пред\-став\-ле\-ние <<как должно стать>> о~коллективе интеллектуальных агентов, 
ре\-ша\-ющих по\-став\-лен\-ную проб\-ле\-му. В~ходе моделирования \mbox{РАСИГИА} 
должны быть специфицированы со\-став и~иерархия ролей агентов, множество 
агентов, ис\-поль\-зу\-емые протоколы взаимодействия, под\-дер\-жи\-ва\-емые языки 
передачи сообщений, базовая онтология как осно\-ва для интерпретации 
семантики пе\-ре\-да\-ва\-емых сообщений, модель окру\-жа\-ющей среды, содержащая 
в~том чис\-ле пул, из которого агенты могут привлекаться сис\-те\-мой по мере 
не\-об\-хо\-ди\-мости и~в~который попадают ис\-клю\-ча\-емые из нее агенты, множество 
моделей архитектур \mbox{РАСИГИА}, множество необходимых моделей 
мак\-ро\-уров\-не\-вых эффектов. В~множестве агентов должны присутствовать 
агенты, пред\-став\-ля\-ющие стейк\-хол\-де\-ров с~их целями, критериями достижения 
цели и~ограничениями. Если на предыду\-щем этапе была по\-стро\-ена модель 
коллектива, то одна из архитектур \mbox{РАСИГИА} долж\-на соответствовать 
данной модели. 
  \end{itemize}
  
\section{Технологическая фаза}

  Технологическая фаза включает в~себя разработку эскизного проекта 
\mbox{РАСИГИА}, ее технического проекта и~программной реализации. 
Стадия разработки эскизного проекта \mbox{РАСИГИА} обеспечивает 
пред\-став\-ле\-ние создаваемой сис\-те\-мы и~ее внеш\-ней среды в~виде 
взаимосвязанных мо\-ду\-лей-бло\-ков в~соответствии с~моделью 
\mbox{РАСИГИА}, по\-стро\-ен\-ной на стадии проектирования. Данная стадия 
со\-сто\-ит из сле\-ду\-ющих этапов:
  \begin{itemize}
  \item разработка функциональной структуры, в~ходе которой строится 
множество взаимосвязанных схем-диа\-грамм, определяющих под\-сис\-те\-мы 
РАСИГИА, распределение агентов по ним, функционал агентов, до\-пус\-ти\-мые 
языки передачи сообщений и~протоколы взаимодействия для каж\-дой пары или 
группы ролей агентов, технологические элементы сис\-те\-мы, потоки 
информации и~управ\-ле\-ния, а~также отношения, воз\-ни\-ка\-ющие между агентами 
в~процессе решения проб\-лем. Для каждой роли указывается множество 
релевантных ей уже существующих (разработанных ранее для других сис\-тем) 
агентов, если таковые имеются. В~случае отсутствия релевантных агентов они 
должны быть разработаны на сле\-ду\-ющих этапах. Кроме того, 
специфицируются функциональные мо\-ду\-ли-бло\-ки, от\-ве\-ча\-ющие за организацию 
макроуровневых эффектов в~\mbox{РАСИГИА};
  \item разработка структуры внешней среды по аналогии с~разработкой 
функциональной структуры \mbox{РАСИГИА} предполагает построение схем-диа\-грамм, описывающих виртуальную внеш\-нюю среду, ее под\-сис\-те\-мы, роли 
агентов и~способы взаимодействия \mbox{РАСИГИА} с~ними, т.\,е.\ языки 
передачи сообщений и~протоколы взаимодействия, отношения, потоки 
информации и~управ\-ле\-ния. Для каж\-дой роли указываются су\-щест\-ву\-ющие 
релевантные ей агенты, если они имеются;
  \item разработка архитектур агентов выполняется для тех ролей 
в~функциональной структуре и~структуре внеш\-ней среды, для которых не 
найдено релевантных реализованных агентов. Архитектура агента~--- схема, 
описывающая со\-став, структуру и~взаимосвязь функ\-ций-бло\-ков, 
ре\-а\-ли\-зу\-емых агентом, обеспечивающая выполнение им своего предназначения. 
Для каждой функ\-ции-бло\-ка указывается метод или алгоритм, с~по\-мощью 
которого она реализуется, в~случае если таковые отсутствуют, они долж\-ны 
быть разработаны в~рамках сле\-ду\-ющей стадии.
  \end{itemize}
  
  Стадия разработки технического проекта \mbox{РАСИГИА} обеспечивает 
создание недостающих блоков для ее агентов или технологических элементов. 
При этом может по\-тре\-бо\-вать\-ся разработка методов решения под\-проб\-лем, 
алгоритмов на основе метода, баз данных, онтологий и~др. Порядок их 
разработки не регламентируется на\-сто\-ящей методологией в~связи 
с~существенным разнообразием и~не\-воз\-мож\-ностью совместного рас\-смот\-ре\-ния. 
На данной стадии должен быть сформирован технический проект, 
опи\-сы\-ва\-ющий для каждого блока со\-став, структуру и~форму пред\-став\-ле\-ния 
входных и~выходных данных, алгоритм его функционирования, спецификацию 
необходимых технических средств~[10].
  
  Стадия программной реализации и~отладки предполагает разработку 
программного кода \mbox{РАСИГИА} и~его тестирование на предмет 
корректной работы с~\mbox{целью} формирования полноценного программного 
продукта, а~так\-же разработку программной документации. Данная стадия 
со\-сто\-ит из сле\-ду\-ющих этапов:
  \begin{itemize}
  \item программная реализация и~разработка документации выполняется 
с~использованием платформы JaCaMo~[11], объединяющей технологию Jason 
для программирования автономных агентов, Cartago для программирования 
элементов внеш\-ней среды и~Moise для программирования многоагентных 
организаций. Кроме того, применяется язык Java для программирования 
отдельных элементов сис\-те\-мы и~тонкой настройки механизмов 
платформы~[12];
  \item тестирование и~отладка обеспечивают выявление и~устранение 
основных дефектов в~сис\-те\-ме. Ввиду того что полное тестирование  
сколь\-ко-ни\-будь слож\-ной программы не\-воз\-мож\-но~[13], выполняется 
выборочное тестирование в~сле\-ду\-ющем порядке: отдельные функ\-ции и~блоки 
из состава аген\-тов и~технологических элементов, межмодульные связи, агенты 
и~технологические элементы в~целом, протоколы взаимодействия агентов, 
\mbox{РАСИГИА} в~целом. В~тес\-ти\-ро\-ва\-нии принимают участие 
представители всех ролей команды разработчиков, так как каж\-дый из них 
выполняет поиск ошибок разного рода~[14]. При этом выделяется отдельная 
роль тестировщика, опре\-де\-ля\-юще\-го стратегию тес\-ти\-ро\-ва\-ния,  
тест-тре\-бо\-ва\-ния и~тест-пла\-ны для каждой из фаз проекта; он выполняет 
тестирование сис\-те\-мы, собирает и~анализирует отчеты о~про\-хож\-де\-нии 
тестирования. 
\end{itemize}

\section{Рефлексивная фаза}

  Рефлексивная фаза предназначена для оценки показателей реализованной 
\mbox{РАСИГИА} и~процесса ее разработки, выявления ее недостатков и~при 
не\-об\-хо\-ди\-мости до\-ра\-бот\-ки как сис\-те\-мы, так и~методологии ее построения. 
Стадия оценки эф\-фек\-тив\-ности \mbox{РАСИГИА} предполагает сбор 
показателей работы сис\-те\-мы и~их сравнение с~целевыми значениями. Если 
выявляется их несоответствие, выполняется анализ причин отклонений, 
переход к~этапу методологии, вызвавшему их, и~повторное выполнение данного и~по\-сле\-ду\-ющих этапов с~учетом тре\-бу\-емых корректировок. Кроме того, на этой 
стадии продолжается отладка сис\-те\-мы. Данная стадия выполняется в~три этапа:
  \begin{enumerate}[(1)]
  \item оценка в~лабораторных условиях командой разработчиков, когда 
система работает в~виртуальной внешней среде, решая тестовые проб\-ле\-мы. На 
данной стадии оценка сис\-те\-мы выполняется вычислительными моделями 
стейкхолдеров, реализованными со\-от\-вет\-ст\-ву\-ющи\-ми агентами виртуальной 
внеш\-ней среды; 
  \item оценка по результатам тестовой эксплуатации, когда \mbox{РАСИГИА} 
функционирует в~реальной внеш\-ней среде параллельно с~традиционным 
методом решения проб\-ле\-мы и~выполняется сравнение их эф\-фек\-тив\-ности 
пользователями и~реальными стейк\-хол\-де\-ра\-ми. Первоначально 
у~\mbox{РАСИГИА} должна быть отключена воз\-мож\-ность оказывать ка\-кое-ли\-бо воздействие на реальную внеш\-нюю среду, а~результатом ее\linebreak
 работы 
долж\-ны стать рекомендации по оказанию таких воздействий. После 
удовлетворительной оценки пользователей и~стейк\-хол\-де\-ров \mbox{РАСИГИА} 
может быть переведена в~\mbox{автоматический} режим взаимодействия со средой, 
а~традиционный метод решения проб\-ле\-мы используется в~качестве резервного 
для проверки ее работы еще в~течение некоторого времени. Длительности 
каждого из этих периодов долж\-ны определяться заказчиками сис\-те\-мы для 
решения конкретной проб\-ле\-мы совместно с~коллективом разработчиков; 
  \item сопровождение после внед\-ре\-ния поз\-во\-ля\-ет собирать жалобы, замечания и~предложения в~процессе эксплуатации \mbox{РАСИГИА}, в~том чис\-ле от 
людей, которые ошибочно не были включены в~со\-став стейкхолдеров.\\[-13pt] 
  \end{enumerate}
  
  Стадия оценки и~корректировки методологии в~определенном смысле длится 
на протяжении всего проекта, так как для ее реализации долж\-ны вес\-тись 
протоколы де\-я\-тель\-ности разработчиков,\linebreak в~которых отмечается дли\-тель\-ность 
реализации каж\-до\-го этапа, выполненные возвраты и~их причины. Однако 
именно по завершении проекта выполняется рефлексия проделанной работы, 
когда разработчики долж\-ны проанализировать удачные и~провальные решения, 
причины рас\-хож\-де\-ния результатов с~планами, возвратов к~предыду\-щим этапам 
и~фазам разработки \mbox{РАСИГИА}, затягивания отдельных этапов 
разработки, из\-бы\-точ\-ность или, наоборот, не\-ин\-фор\-ма\-тив\-ность 
по\-стро\-ений~\cite{4-lis}. По результатам анализа в~методологию вносятся 
изменения в~статусе <<предложение>>, которые после под\-тверж\-де\-ния 
эф\-фек\-тив\-ности в~новых проектах закрепляются в~новой версии методологии.

\vspace*{-9pt}

\section{Заключение}

\vspace*{-3pt}

  В работе представлена темпоральная структура (жизненный цикл) 
разработки \mbox{РАСИГИА}, опи\-сы\-ва\-ющая процессы сис\-тем\-но\-го анализа 
проб\-ле-\linebreak мы, моделирования, эскизного и~технического \mbox{проектирования} сис\-те\-мы, 
ее программной реализации, отладки и~тестирования. 
Основной результат 
организации работ в~соответствии с~предложенной методологией~--- 
программная реализация \mbox{РАСИГИА}, релевантно моделирующая 
коллектив специалистов, со\-вмест\-но ре\-ша\-ющих по\-став\-лен\-ную проб\-ле\-му 
с~учетом ее слабой формализации, не\-од\-но\-род\-ности, сетевого характера условий 
и~целей, не\-опре\-де\-лен\-ности и~ди\-на\-мич\-ности~\cite{5-lis}. Кроме того, в~результате 
рефлексивной стадии методологии формируется ее новая версия или 
подтверждается эф\-фек\-тив\-ность су\-щест\-ву\-ющей, что представляется\linebreak 
дополнительным результатом работ. Таким образом, методология предполагает 
свое развитие, потенциально обеспечивающее ее ре\-ле\-вант\-ность \mbox{актуальным}
подходам к~проектированию и~реализации интеллектуальных информационных 
сис\-тем.

\vspace*{-9pt}
  
{\small\frenchspacing
 { %\baselineskip=10.6pt
 %\addcontentsline{toc}{section}{References}
 \begin{thebibliography}{99}
 
 \vspace*{-3pt}
 
  \bibitem{1-lis}
   \Au{Поспелов Д.\,А.} Десять <<горячих точек>> в~исследованиях по искусственному 
интеллекту~// Искусственный\linebreak\vspace*{-12pt}

\columnbreak

\noindent
 интеллект и~принятие решений, 2019. №\,4. С.~3--9. doi: 
10.14357/20718594190401. EDN: BAUHFV.
  
  \bibitem{2-lis}
\Au{Тарасов В.\,Б.} От многоагентных сис\-тем к~интеллектуальным организациям: 
философия, психология, информатика.~--- М.: Эдиториал УРСС, 2002. 348~с.
  \bibitem{3-lis}
  \Au{Колесников А.\,В., Кириков~И.\,А., Листопад~С.\,В.} Ги\-брид\-ные интеллектуальные 
сис\-те\-мы с~самоорганизацией: координация, со\-гла\-со\-ван\-ность, спор.~--- М.: ИПИ РАН, 2014. 
189~с.
  \bibitem{4-lis}
  \Au{Новиков А.\,М., Новиков~Д.\,А.} Методология.~--- М.: Синтег, 2007. 668~с.
  \bibitem{5-lis}
  \Au{Листопад С.\,В.} Характеристики и~логическая структура методологии по\-стро\-ения  
реф\-лек\-сив\-но-ак\-тив\-ных сис\-тем искусственных гетерогенных интеллектуальных 
агентов~// Сис\-те\-мы и~средства \mbox{информатики}, 2023. Т.~33. №\,4. С.~16--27. doi: 
10.14357/ 08696527230402. EDN: TRTHEI.
  \bibitem{6-lis}
  \Au{Тарасенко Ф.\,П.} Прикладной сис\-те\-мный анализ.~--- М.: 
КНОРУС, 2010. 224~с.
  \bibitem{7-lis}
  \Au{Ларичев О.\,И.} Вербальный анализ решений.~--- М.: Наука, 2006. 181~с.
  \bibitem{8-lis}
  \Au{Акофф Р.} Акофф о менеджменте~/ Пер.\ с~англ.~--- СПб.: Питер, 2002. 448~с.
  (\Au{Akoff~R.\,L.} Ackoff's best: His classic writings on management.~--- New 
York, NY, USA: Wiley, 1999. 368~p.)
  \bibitem{9-lis}
  \Au{Колесников А.\,В., Листопад~С.\,В., Румовская~С.\,Б., Данишевский~В.\,И.} 
Неформальная аксиоматическая тео\-рия ролевых визуальных моделей~// Информатика и~её 
применения, 2016. Т.~10. Вып.~4. С.~114--120.  doi: 10.14357/19922264160412. EDN: XGSIVN.
  \bibitem{10-lis}
  \Au{Черушева Т.\,В.} Проектирование программного обеспечения.~--- Пенза: ПГУ, 2014. 
172~с.
  \bibitem{11-lis}
  \Au{Boissier O., Bordini~R.\,H., Hubnerand~J., Ricci~A.} Multi-agent oriented programming: 
Programming multi-agent systems using JaCaMo.~--- Intelligent robotics and autonomous agents 
series.~--- Cambridge: The MIT Press, 2020. 264~p.
  \bibitem{12-lis}
  \Au{Смирнов С.\,С., Смольянинова~В.\,А.} Введение в~разработку многоагентных сис\-тем 
в~среде Jason. Основы программирования на языке AgentSpeak.~--- М.: \mbox{МИРЭА}, 2009. 136~с.
  \bibitem{13-lis}
  \Au{Канер~С., Фолк~Д., Нгуен~Е.\,К.} Тестирование про\-грам\-мно\-го обеспечения. 
Фундаментальные концепции менеджмента биз\-нес-при\-ло\-же\-ний~/
Пер. с~англ.~--- Киев: ДиаСофт, 
2001. 544~с. (\Au{Kaner~С., Falk~J., Nguyen~H.\,Q.} {Testing computer software}.~--- 
International Thomson Computer Press,  1999. 496~p.)
  \bibitem{14-lis}
  \Au{Романькова Т.\,Л.} Тестирование программного обеспечения. {\sf 
https://elib.gstu.by/bitstream/handle/220612/ 9860/416.pdf}.

\end{thebibliography}

 }
 }

\end{multicols}

\vspace*{-6pt}

\hfill{\small\textit{Поступила в~редакцию 25.11.23}}

%\vspace*{8pt}

%\pagebreak

\newpage

\vspace*{-28pt}

%\hrule

%\vspace*{2pt}

%\hrule



\def\tit{LIFE CYCLE OF METHODOLOGY FOR~CONSTRUCTING REFLEXIVE-ACTIVE SYSTEMS OF~ARTIFICIAL HETEROGENEOUS INTELLIGENT AGENTS}


\def\titkol{Life cycle of methodology for~constructing reflexive-active systems of~artificial heterogeneous intelligent agents}


\def\aut{S.\,V.~Listopad}

\def\autkol{S.\,V.~Listopad}

\titel{\tit}{\aut}{\autkol}{\titkol}

\vspace*{-8pt}


\noindent
Federal Research Center ``Computer Science and Control'' of the Russian Academy of 
Sciences, 44-2~Vavilov Str., Moscow 119333, Russian Federation

\def\leftfootline{\small{\textbf{\thepage}
\hfill INFORMATIKA I EE PRIMENENIYA~--- INFORMATICS AND
APPLICATIONS\ \ \ 2024\ \ \ volume~18\ \ \ issue\ 1}
}%
 \def\rightfootline{\small{INFORMATIKA I EE PRIMENENIYA~---
INFORMATICS AND APPLICATIONS\ \ \ 2024\ \ \ volume~18\ \ \ issue\ 1
\hfill \textbf{\thepage}}}

\vspace*{4pt}
  
  
   
   \Abste{The paper presents the temporal structure (life cycle) of the methodology for 
constructing reflexive-active systems of artificial heterogeneous intelligent agents. These systems 
are designed for computer modeling of processes and effects that arise when solving practical 
problems by teams of specialists under the guidance of a~decision maker. Artificial heterogeneous 
intelligent agents of reflexive-active systems are active subjects capable of reasoning, 
communication, and reflection as the ability to model the reasoning of other agents of the system 
and themselves. Modeling of reflexive processes ensures the development by agents of a~consistent 
understanding of the control object, the purpose of collective work, and the norms of interaction 
allowing the system to self-organize and re-develop a relevant hybrid intelligent method for solving 
the next problem.}
   
   \KWE{reflection; methodology; reflexive-active system of artificial heterogeneous intelligent 
agents; hybrid intelligent multiagent system; team of specialists}
   
 
   
\DOI{10.14357/19922264240112}{GUAMVE}

\vspace*{-8pt}

\Ack

\vspace*{-1pt}


     \noindent
     This work was supported by the Russian Science Foundation, project No.\,23-21-00218.


\vspace*{6pt}

  \begin{multicols}{2}

\renewcommand{\bibname}{\protect\rmfamily References}
%\renewcommand{\bibname}{\large\protect\rm References}

{\small\frenchspacing
 {\baselineskip=11.5pt
 \addcontentsline{toc}{section}{References}
 \begin{thebibliography}{99} 
  \bibitem{1-lis-1}
   \Aue{Pospelov, D.\,A.} 2019. Desyat' ``goryachikh tochek'' v~issledovaniyakh po 
iskusstvennomu intellektu [Ten hot topics in AI research]. \textit{Is\-kus\-stven\-nyy in\-tel\-lekt 
i~pri\-nya\-tie re\-she\-niy} [Artificial Intelligence and Decision Making] 4:3--9. doi: 
10.14357/20718594190401. EDN: BAUHFV.
  \bibitem{2-lis-1}
   \Aue{Tarasov, V.\,B.} 2002. \textit{Ot mnogoagentnykh sis\-tem k~in\-tel\-lek\-tu\-al'\-nym 
or\-ga\-ni\-za\-tsi\-yam: fi\-lo\-so\-fiya, psi\-kho\-lo\-giya, in\-for\-ma\-ti\-ka} [From multiagent systems to intelligent 
organizations: Philosophy, psychology, and computer science]. Moscow: Editorial URSS. 348~p.
  \bibitem{3-lis-1}
   \Aue{Kolesnikov, A.\,V., I.\,A.~Kirikov, and S.\,V.~Listopad.} 2014. \textit{Gib\-rid\-nye 
in\-tel\-lek\-tu\-al'\-nye sis\-te\-my s~sa\-mo\-or\-ga\-ni\-za\-tsiey: ko\-or\-di\-na\-tsiya, so\-gla\-so\-van\-nost', spor} [Hybrid 
intelligent systems with self-organization: Coordination, consistency, and dispute]. Moscow: IPI 
RAN. 189~p.
  \bibitem{4-lis-1}
   \Aue{Novikov, A.\,M., and D.\,A.~Novikov.} 2007. \textit{Me\-to\-do\-lo\-giya} [Methodology]. 
Moscow: SINTEG. 668~p.
  \bibitem{5-lis-1}
   \Aue{Listopad, S.\,V.} 2023. Kharakteristiki i~logicheskaya struk\-tu\-ra me\-to\-do\-lo\-gii po\-stro\-eniya 
refleksivno-aktivnykh sis\-tem is\-kus\-stven\-nykh ge\-te\-ro\-gen\-nykh in\-tel\-lek\-tu\-al'\-nykh agen\-tov 
[Characteristics and logical structure of the methodology for constructing reflexive-active systems 
of artificial heterogeneous intelligent agents]. \textit{Sistemy i~Sredstva Informatiki~--- Systems 
and Means of Informatics} 33(4):16--27. doi: 10.14357/08696527230402. EDN: TRTHEI.
  \bibitem{6-lis-1}
   \Aue{Tarasenko, F.\,P.} 2010. \textit{Pri\-klad\-noy sis\-tem\-nyy ana\-liz} 
[Applied systems analysis]. Moscow: KNORUS. 224~p.
  \bibitem{7-lis-1}
   \Aue{Larichev, O.\,I.} 2006. \textit{Ver\-bal'\-nyy ana\-liz re\-she\-niy} [Verbal analysis of decisions]. 
Moscow: Nauka. 181~p.
  \bibitem{8-lis-1}
   \Aue{Akoff, R.\,L.} 1999. \textit{Ackoff's best: His classic writings on management}. New 
York, NY: Wiley. 368~p.
  \bibitem{9-lis-1}
   \Aue{Kolesnikov, A.\,V., S.\,V.~Listopad, S.\,B.~Rumovskaya, and V.\,I.~Danishevskiy.} 
2016. Ne\-for\-mal'\-naya ak\-sio\-ma\-ti\-che\-skaya teo\-riya ro\-le\-vykh vi\-zu\-al'\-nykh mo\-de\-ley [Informal axiomatic 
theory of the role visual models]. \textit{Informatika i~ee Primeneniya~--- Inform. Appl.} 
10(4):114--120. doi: 10.14357/19922264160412. EDN: XGSIVN.
  \bibitem{10-lis-1}
   \Aue{Cherusheva, T.\,V.} 2014. \textit{Pro\-ek\-ti\-ro\-va\-nie pro\-gram\-mno\-go obes\-pe\-che\-niya} 
[Software design]. Penza: PGU. 172~p.
  \bibitem{11-lis-1}
   \Aue{Boissier, O., R.\,H.~Bordini, J.~Hubnerand, and A.~Ricci}. 2020. \textit{Multi-agent 
oriented programming: Programming multi-agent systems using JaCaMo}. Intelligent robotics and 
autonomous agents ser. Cambridge: The MIT Press. 264~p.
  \bibitem{12-lis-1}
   \Aue{Smirnov, S.\,S., and V.\,A.~Smol'yaninova}. 2009. \textit{Vve\-de\-nie v~raz\-ra\-bot\-ku 
mno\-go\-agent\-nykh sis\-tem v~sre\-de Jason. Osno\-vy pro\-gram\-mi\-ro\-va\-niya na yazy\-ke AgentSpeak} 
[Introduction to the development of multiagent systems in the Jason environment. Fundamentals of 
programming in the AgentSpeak language]. Moscow: MIREA. 136~p.
  \bibitem{13-lis-1}
   \Aue{Kaner, С., J.~Falk, and H.\,Q.~Nguyen}. 1999. \textit{Testing computer software}. 
International Thomson Computer Press. 496~p.
  \bibitem{14-lis-1}
   \Aue{Romankova, T.\,L.} 2014. Tes\-ti\-ro\-va\-nie pro\-gram\-mno\-go obes\-pe\-che\-niya [Software testing]. 
Available at: {\sf https://}\linebreak\vspace*{-12pt}

\columnbreak

\noindent
 {\sf elib.gstu.by/bitstream/handle/220612/9860/416.pdf} (accessed January~16, 
2024).
   
  \end{thebibliography}

 }
 }

\end{multicols}

\vspace*{-6pt}

\hfill{\small\textit{Received November 25, 2023}} 

%\vspace*{-18pt}
     
     \Contrl
     
 %    \vspace*{-3pt}
   
   \noindent
   \textbf{Listopad Sergey V.} (b.\ 1984)~--- Candidate of Science (PhD) in technology, senior 
scientist, Federal Research Center ``Computer Science and Control'' of the Russian Academy of 
Sciences, 44-2~Vavilov Str., Moscow 119133, Russian Federation;  
\mbox{ser-list-post@yandex.ru}
   
    
\label{end\stat}

\renewcommand{\bibname}{\protect\rm Литература} 