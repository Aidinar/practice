\def\stat{leshinsk}

\def\tit{АНАЛИЗ ПОДХОДОВ К ОПРЕДЕЛЕНИЮ НЕЧЕТКОЙ~РЕЗОЛЬВЕНТЫ}

\def\titkol{Анализ подходов к~определению нечеткой резольвенты}

\def\aut{Т.\,М.~Леденева$^1$, М.\,В.~Лещинская$^2$}

\def\autkol{Т.\,М.~Леденева, М.\,В.~Лещинская}

\titel{\tit}{\aut}{\autkol}{\titkol}

\index{Леденева Т.\,М.}
\index{Лещинская М.\,В.}
\index{Ledeneva T.\,M.}
\index{Leshchinskaya M.\,V.}


%{\renewcommand{\thefootnote}{\fnsymbol{footnote}} \footnotetext[1]
%{Финансовое обеспечение исследований осуществлялось из 
%(Институт прикладных математических исследований КарНЦ РАН).}}


\renewcommand{\thefootnote}{\arabic{footnote}}
\footnotetext[1]{Воронежский государственный университет, \mbox{ledeneva-tm@yandex.ru}}
\footnotetext[2]{Воронежский государственный университет, \mbox{maria-leshchinskaya@mail.ru}}

%\vspace*{-12pt}

  
  

   
  \Abst{Представлены результаты исследования, касающиеся различных определений 
резольвенты в~нечеткой логике. Определены условия, при выполнении которых резольвента 
Ли становится значимым логическим следствием в~случае классического определения 
нечетких логических связок. Показано, что при использовании для их формализации 
треугольных норм и~конорм логически значимую резольвенту Ли получить невозможно. 
Однако если треугольная конорма задается как операция максимума, то для произвольной 
треугольной нормы резольвента Ли существует. Определены условия, при выполнении 
которых для классических операций минимума и~максимума резольвента Мукайдоно 
становится значимым логическим следствием. При использовании треугольных норм 
и~конорм, отличных от классических, требуется дополнительное исследование. Приведен 
иллюстративный пример, демонстрирующий процесс построения резольвенты Мукайдоно.}
  
  \KW{резольвента; метод резолюций; треугольные нормы и~конормы}
  
  \DOI{10.14357/19922264240113}{MIDEHY}
  
%\vspace*{-6pt}


\vskip 10pt plus 9pt minus 6pt

\thispagestyle{headings}

\begin{multicols}{2}

\label{st\stat}

\section{Введение}

  Принцип резолюций, впервые предложенный в~[1], составляет основу 
методов машинной логики, связанных с~автоматическим доказательством 
теорем~[2, 3], и~применяется в~интерактивных системах различного назначения. 
Актуальной остается задача модификации существующих логических 
инструментов в~соответствии с~тенденциями развития области представления 
и~обработки знаний. Неопределенность аргументации и~выводов, характерная для 
неформальных рассуждений, значительно ограничивает применимость 
классического метода резолюций и~обусловливает использование нечеткой 
логики~[4]. Впервые понятие нечеткой резольвенты было предложено Ли в~[5], 
а~затем развито в~работах Мукайдоно~[6] и~других исследователей. Можно 
выделить несколько направлений в~развитии нечеткого метода резолюций. 
В~[7] предложен подход, учитывающий неопределенные высказывания; в~[8] 
введено понятие нечеткой гиперрезолюции и~доказано свойство полноты;  
[9--13] посвящены методу резолюций в~различных логиках; в~[14, 15] 
предложен обобщенный принцип резолюции на основе обратных 
приближенных рассуждений (inverse approximate reasoning), что позволяет 
использовать для построения нечеткой резольвенты правило вывода modus 
ponens. Как показывают исследования, основная проблема нечеткого метода 
резолюций~--- получение условий, при которых резольвента приводит 
к~значимому логическому заключению. Цель статьи заключается в~анализе 
определений нечеткой резольвенты и~исследовании возможности 
использования для ее по\-стро\-ения обобщений нечетких логических операций~--- 
треугольных норм и~конорм.
  

\section{Теоретическая основа исследования}

  Введем основные понятия, базируясь, например, на~[16]. Переменную~$x$ 
или ее отрицание $\overline{x}$ будем называть \textit{литерой}. 
\textit{Дизъюнкт}, или \textit{элементарная дизъюнкция},~--- это совокупность 
различных литер, связанных символом дизъюнкции. \textit{Пустой 
дизъюнкт}~--- это тождественно ложный дизъюнкт, который будем обозначать 
$\sqcup$. Резолюция представляет собой дедуктивное правило вывода, которое 
основано на \textit{дизъюнктивном силлогизме} $(A\vee B, \overline{A})/B$.
  
  Пусть $D_1=D_1^\prime \vee p$  и~$D_2\hm= D_2^\prime \vee \overline{p}$~--- 
дизъюнкты, содержащие контрарные литеры~$p$ и~$\overline{p}$, 
причем~$D_1^\prime$ и~$D_2^\prime$ данных литер не содержат. 
\textit{Резольвентой} дизъюнктов (по переменной~$p$) $D_1$ и~$D_2$ 
называется дизъюнкт вида $\mathrm{res}\,(D_1, D_2) \hm= D_1^\prime \vee D_2^\prime$. 
Если дизъюнкты не содержат контрарных литер, то резольвент у них не 
существует. Пустой дизъюнкт определяется следующим образом: $\mathrm{res}\,(p,\neg p 
)\hm=\Box$.
  
  Пусть $C=\{D_1, \ldots , D_m\}$~--- множество дизъюнктов. 
\textit{Резолютивным выводом} из~$C$ называется такая конечная 
последовательность дизъюнктов $\varphi_1, \ldots, \varphi_n$, в~которой для 
каждого $\varphi_i$ $\left(i\hm= \overline{1,n}\right)$ выполняется одно из 
условий: (1)~$\varphi_i\hm\in C$; (2)~существуют $j,k\hm<i$ такие, что 
$\varphi_i\hm= \mathrm{res}\, (\varphi_j, \varphi_k)$. Дизъюнкт~$D$ считается 
\textit{резолютивно выводимым} из множества~$C$, если существует 
резолютивный вывод из~$C$, последней формулой которого является~$D$. 
Согласно теореме о полноте резолюций, множество~$C$ противоречиво в~том 
и~только в~том случае, когда существует резолютивный вывод из~$C$, 
заканчивающийся пустым дизъюнктом~$\Box$.
  
  Метод резолюций используется для решения следующих задач, которые 
сводятся к~доказательству противоречивости специальным образом 
построенного множества~$C$: проверка правильности рассуждений; 
доказательство выводимости формулы из множества гипотез; доказательство 
того, что формулу можно считать теоремой; доказательство противоречивости 
заданного множества формул.
  
  Чтобы ввести нечеткую резольвенту, рас\-смот\-рим необходимые понятия из 
нечеткой логики~[4], которую можно определить как алгебру $\langle  [0;1], 
\wedge,\vee,\neg\rangle$, при этом для формализации логических связок~$\vee$ 
и~$\wedge$ используются различные операции. Наиболее распространенный 
подход предполагает, что~$\vee$ реализуется как $\max$, а~$\wedge$~--- как 
$\min$ при использовании стандартного отрицания. Заметим, что для операций 
$\max$ и~$\min$, называемых классическими, не выполняются законы 
комплементарности, поэтому в~нечеткой логике высказывание, в~котором~$a$ 
и~$\overline{a}$ встречаются одновременно, оказывается содержательным 
в~отличие от классической логики. В~общем случае для 
формализации~$\wedge$ и~$\vee$ используются треугольные нормы~$T$ 
и~двойственные им (в~смысле законов де Моргана) треугольные конормы~$S$. 
Различные семейства треугольных норм\linebreak и~конорм представлены в~[17]. Новые 
параметрические семейства треугольных норм и~конорм получены в~[18, 19]. 
Следует отметить, что для классической пары не выполняется свойство 
\mbox{комплементарности}, а~для остальных помимо комплементарности не 
выполняются свойства идемпотентности и~дистрибутивности, что 
обусловливает необходимость проявления аккуратности при выполнении 
всевозможных преобразований логических формул.
  
  Каждому элементарному высказыванию   поставим в~соответствие оценку 
его истинности $t(x)\hm\in [0;1]$, при этом чем ближе значение $t(x)$ к~0, тем 
в~большей степени высказывание считается \textit{ложным}; чем ближе $t(x)$ 
к~1, тем в~большей степени высказывание считается \textit{истинным}; если 
$t(x)\hm=0{,}5$, то высказывание считается \textit{неопределенным} 
(соответствует максимальной неопределенности). В~нечеткой логике 
высказывание~$x$ принимается, если степень его истинности больше степени 
истинности высказывания \textit{не} $x (\overline{x})$, т.\,е.\ $t(x)\hm> 1\hm- 
t(x)$, а~следовательно, $t(x)\hm>0{,}5$.
  
  \textit{Степенью доверия} к~$t(x)$ называется величина $c_x\hm= 2(t(x)\hm-
0{,}5)\hm\in [-1;1]$, а~ее модуль $\vert c_x\vert$ характеризует уровень 
определенности высказывания~$x$: если $t(x)\hm=0{,}5$, то $c_x\hm= 
0$~\cite{6-les}.
  
  Пусть $F$~--- некоторая формула. При заданной\linebreak интерпретации~$I$ 
значение истинности каждой переменной определяется однозначно, поэтому, 
выбрав подходящую формализацию логических связок, можно получить 
значение истинности \mbox{формулы}~$F$. Формула~$F$ называется 
\textit{выполнимой} в~интерпретации~$I$, если $t(F)\hm> 0{,}5$, 
и~\textit{невыполнимой}, если $t(F)\hm< 0{,}5$. Если $t(F)\hm= 0{,}5$, то 
интерпретация~$I$ одновременно удовлетворяет и~опровергает~$F$.
  
  Если $F=\{F_1,\ldots , F_m\}$~--- множество формул, то $t_I(F)\hm= 
t_I(F_1\wedge \cdots\wedge F_m)$.

\vspace*{-6pt}

\section{Нечеткая резольвента и~ее~свойства}

  Будем считать, что
  \begin{align*}
  t(a\vee b) &=\max \left\{ t(a),t(b)\right\};\\
     t(a\wedge b) &=\min \{t(a),t(b)\};\\
  t\left( \overline{a}\right) &=1-t(a)\,.
  \end{align*}
  
  Пусть $D_1=p\vee D_1^\prime$, $D_2\hm= \overline{p}\vee D_2^\prime$, 
$\mathrm{res}\,(D_1,D_2) \hm= D_1^\prime \vee D_2^\prime$.
 В~[5] нечеткая резольвента 
определяется в~виде 
$$
t\left(\mathrm{res}\,(D_1,D_2) \right)= t\left( D_1^\prime \vee 
D_2^\prime\right) = \max \left\{ t(D_1^\prime), t(D_2^\prime)\right\}
$$ 
и~называется \textit{резольвентой Ли} (L-ре\-золь\-вен\-той).
  
  В классической логике действует следующий постулат~[20]: истинность 
множества формул остается неизменной, если к~ним добавить их резольвенту, 
т.\,е.\ $t(D_1\wedge D_2)=t(D_1\wedge D_2\wedge \mathrm{res}\,(D_1, D_2))$, и~поэтому
  \begin{equation}
  d\left( D_1\wedge D_2\right) \leq t\left( \mathrm{res} \left( D_1, D_2\right)\right)\,.
  \end{equation}
  
  Резольвента называется \textit{логически значимой} или \textit{значимым 
логическим следствием}, если она удовлетворяет~(1). Найдем условия, при 
которых выполняется соотношение~(1), а~следовательно, резольвента 
становится логически значимой.
  
  Положим $t(D_1\wedge D_2) \hm= \min \{t(D_1), t(D_2)\} \hm=a$. Без 
ограничения общности будем считать, что
  \begin{align}
  t(D_1) &= t(p\vee D_1^\prime)=\max \left\{ t(p),t(D_1^\prime)\right\}=a\,;
  \label{e2-les}\\
  t(D_2) &= t\left( \overline{p} \vee D_2^\prime\right) =\max \left\{ t(\overline{p}), 
t(D_2^\prime) \right\} ={}\notag\\
&\hspace*{15mm}{}=\max \left\{ 1-t(p), t(D_2^\prime\right\}>a\,.
  \label{e3-les}
  \end{align}
  
  В соответствии с~введенными предположениями возможны ситуации:
  \begin{itemize}
\item  для~(2) имеем:
\begin{itemize}
\item[(а)] $t(p)=a$, $t(D_1^\prime) <a$; 
\item[(б)] $t(p)<a$,  $t(D_1^\prime)\hm=a$;
\end{itemize}
  \item
  для~(3) имеем: 
  \begin{itemize}
  \item[(в)] $\left[    \begin{matrix} 
  (\mathrm{в}1)~a<t(D_2^\prime)<1-t(p);\\
  (\mathrm{в}2)~t(D_2^\prime) <a<1-t(p);
  \end{matrix}
  \right]$
\item[(г)] $\left[   \begin{matrix}
  (\mathrm{г}1)~a<1- t(p)<t/(D_2^\prime);\\
 (\mathrm{г}2)~1-t(p)<a<t(D_2^\prime).
  \end{matrix}
  \right.$
  \end{itemize}
  \end{itemize}
  
  Анализ комбинаций приведенных ситуаций поз\-во\-лил сформулировать 
следующие утверж\-де\-ния.
  
  \smallskip
  
  \noindent
  \textbf{Утверждение~1.} \textit{Пусть $t(p)\hm< 0{,}5$, тогда 
неравенство}~(1) \textit{выполняется, а~следовательно, резольвента 
становится значимым логическим следствием, если имеет место одна из 
следующих ситуаций}: 
\begin{enumerate}[(1)]
\item \textit{для одного из дизъюнктов~$D_1^\prime$ 
или~$D_2^\prime$ степень истинности больше $t(p)$, а~степень истинности 
другого принадлежит промежутку} $\left( \min \left\{ t(D_1^\prime), 
t(D_2^\prime)\right\}, 1\hm- t(p)\right)$; 
\item \textit{для одного из 
дизъюнктов~$D_1^\prime$ или~$D_2^\prime$ степень истинности меньше 
$t(p)$, а~степень истинности другого принадлежит промежутку} 
  $\left( t(p), 1\hm- t(p)\right)$;
  \item $t(D_1^\prime) \hm< t(D_2^\prime)$ 
  \textit{и}~$t(D_2^\prime) \hm>1\hm- t(p)$; 
  \item $t(D_1^\prime)\hm< t(D_2^\prime)$ \textit{и}~$t(D_1^\prime) \hm\in \left( 1\hm- t(p), t(D_2^\prime)\right)$.
\end{enumerate}
  
  %\smallskip
  
  \noindent
  \textbf{Утверждение~2.} \textit{Если $t(p)\hm> 0{,}5$, $t(D_1^\prime) \hm< 
t(D_1^\prime)$ и~$t(D_2^\prime) \hm> t(p)$, то неравенство}~(1) 
\textit{выполняется}.
  
  \smallskip
  
  \noindent
  \textbf{Утверждение~3.} \textit{Если для переменной $p$ и~каждого из 
дизъюнктов~$D_1^\prime$ и~$D_2^\prime$ степень истинности меньше~0,5, 
то неравенство}~(1) \textit{не выполняется}.
  
  \smallskip
  
  \noindent
  \textbf{Утверждение~4.} \textit{L-ре\-золь\-вен\-та становится выполнимой 
формулой, если}
  \begin{enumerate}[(1)]
  \item $t(p)<0{,}5$ \textit{и} $\max \left\{ t(D_1), t(D_2)\right\} \in [0{,}5;1]$;
  \item $t(p)>0{,}5$ \textit{и} $\max \left\{ t(D_1), t(D_2)\right\} \in [t(p);1]$.
  \end{enumerate}
  
  \smallskip
  
  Таким образом, L-ре\-золь\-вен\-та становится значимым логическим 
следствием, если степень ее истинности или степень истинности по крайней 
мере одного из дизъюнктов больше~0,5. На основе приведенных рассуждений 
был построен алгоритм для нахождения логически значимой резольвенты.
  
  Теперь попытаемся обобщить результаты на другие типы логических связок. 
Рассмотрим случай,\linebreak когда конъюнкция формализуется треугольной 
нормой~$T$, дизъюнкция~--- треугольной конормой~$S$, так что пара $(T,S)$ 
образует пару двойственных операций относительно стандартного отрицания 
$n(x)\hm=1\hm-x$. Исследуем условие~(1) для произвольной пары $(T,S)$, 
учитывая, что для нее не выполняются законы дистрибутивности, 
комплементарности и~идемпотентности.
  %
  В этом случае
  \begin{align*}
  t(D_1)&=t\left(D_1^\prime\vee p\right) =S\left\{ t(D_1^\prime), t(p)\right\};\\
  t(D_2)&=t\left(D_2^\prime\vee \overline{p}\right) =S\left\{ t(D_2^\prime), 
t(\overline{p})\right\};\\
  t\left(\overline{D}_1\right) &=t\left( \overline{D_1^\prime\vee p}\right) =T\left( 
t(\overline{D}_1^\prime, t\left(\overline{p}\right)\right);\\
  t\left( \overline{D}_2\right) &=t\left( \overline{D_2^\prime \vee 
\overline{p}}\right) =T\left( t\left(\overline{D}_2^\prime\right),  t(p)\right);\\
  t\left( \mathrm{res}\, (D_1, D_2)\right) &= t(D_1^\prime\vee D_2^\prime) =S\left( 
t(D_1^\prime), t(D_2^\prime)\right).
  \end{align*}
  %
  Тогда, учитывая, что для любых~$T$ и~$S$ имеют мес\-то неравенства 
$T(x,y)\hm\leq \min\{x,y\}$ и~$\max \{ x,y\} \hm\leq S(x,y)$, получим

\vspace*{-4pt}

\noindent
  \begin{multline*}
  t(D_1\wedge D_2) =T(t(D_1), t(D_2))={}\\[1pt]
  {}= 1-S(1-t(D_1), 1-t(D_2)) ={}\\[1pt]
  {}=
  1- S\left(t(\overline{D}_1), t(\overline{D}_2)\right)={}\\[1pt]
  {}= 1-S\left(T\left(t\left(\overline{D}_1^\prime\right), t(\overline{p})\right), T\left( 
t\left(\overline{D}_2^\prime\right), t(p)\right)\right) ={}\\[1pt]
{}=
  1-S\left( 1-S\left( 
  t\left(D_1^\prime\right), t(p)\right), 1-S\left( t\left(D_2^\prime\right), 
t(\overline{p})\right)\right)={}\\[1pt]
  {}=
  T\left( S\left(t\left(D_1^\prime\right), t(p)\right), S\left(t\left(D_2^\prime\right), t\left(\overline{p}\right)\right)\right) \leq{}\\[1pt]
  {}\leq
  \min\left( S(t(D_1^\prime), t(p)), S(t(D_2^\prime), t(\overline{p}))\right)\leq{}\\[1pt]
  {}\leq
  \max (S(t(D_1^\prime), t(p)), S(t(D_2^\prime), t(\overline{p})))\leq{}\\[1pt]
  {}\leq
  S(S(t(D_1^\prime), t(p)), S(t(D_2^\prime), t(\overline{p})))={}\\[1pt]
  {}=
  S \left(S\left(t(D_1^\prime), t(D_2^\prime)\right), S\left( t(p), t(\overline{p}) 
\right)\right).
  \end{multline*}
  
  \vspace*{-4pt}
  
  Заметим, что в~качестве результата данного выражения можно получить 
резольвенту $S\left(t(D_1^\prime), t(D_2^\prime)\right)\hm= t\left(\mathrm{res}\,(D_1, 
D_2)\right)$, если $S\left(t(p), t(\overline{p})\right)\hm=0$, так как для 
треугольных конорм $S(x,0)\hm=x$ для всех $x\hm\in [0,1]$. Согласно 
определению, соотношение $S\left(t(p), t(\overline{p})\right)\hm=0$ 
выполняется, если одновременно $t(p)\hm= 0$ и~$t(\overline{p})\hm=0$, что 
невозможно. Если $S\left( t(p), t(\overline{p})\right)\hm= S\left(t(D_1^\prime), 
t(D_2^\prime)\right)$, то для идемпотентных конорм неравенство~(1) будет 
выполнено. Но единственной идемпотентной парой остается пара $(\min, 
\max)$ ~[17]. Таким образом, доказано

  \smallskip
  
  \noindent
  \textbf{Утверждение~5.} \textit{Для произвольной пары $(T,S)$ 
двойственных треугольных операций, отличных от классических $\min$ 
и~$\max$, L-ре\-золь\-вен\-та не существует}.
  
  \smallskip
  
  Рассмотрим комбинацию произвольной треугольной нормы~$T$ 
и~$S\hm=\max$:
  \begin{align*}
  t\left( \mathrm{res}\left (D_1, D_2\right)\right) &=t\left( D_1^\prime \vee D_2^\prime\right) =\max \left( 
t(D_1^\prime), t(D_2^\prime)\right);\\
  t(D_1) &=t(D_1^\prime\vee p) =\max \left\{ t(D_1^\prime), t(p)\right\};\\
  t(D_2)&= t(D_2^\prime\vee \overline{p})=\max \left\{ t(D_2^\prime), 
t(\overline{p})\right\}.
  \end{align*}
  
  Найдем 
  \begin{multline*}
  t(D_1\wedge D_2) =T\left( t(D_1), t(D_2)\right) \leq {}\\
  \hspace*{-10mm}{}\leq \min\left\{ t(D_1), t(D_2)\right\}={}\hspace*{10mm}
  \end{multline*}

\noindent
  \begin{multline*}
  {}=
  \min\left\{ \max \left\{ t(D_1^\prime), t(p)\right\}, \max \left\{ t(D_2^\prime), 
t\left(\overline{p}\right)\right\}\right\}={}\\[1pt]
  {}=
  \max \left\{ \min\left\{ t(D_1^\prime), t(D_2^\prime)\right\}, 
  \min\left\{ t\left(D_1^\prime\right), t\left(\overline{p}\right)\right\},\right.\\[1pt]
 \left. \min\left\{ t(D_2^\prime), t(p)\right\},
  \min\left\{ t(p), t(\overline{p})\right\}\right\}\leq{}\\[1pt]
  {}\leq
  \max \left\{ \max \left\{ t(D_1^\prime), t(D_2^\prime)\right\}, \min\left\{ 
t(D_1^\prime), t(\overline{p})\right\},\right.\\[1pt]
\left.
  \min\left\{ t(D_2^\prime), t(p)\right\}, \min\left\{ t(p), 
t(\overline{p})\right\}\right\}.
  \end{multline*}
  
%\vspace*{-3pt}
  
  Заметим, что результатом последнего выражения будет 
  $$
  \max \left\{ t\left(D_1^\prime\right), t\left(D_2^\prime\right)\right\}= t\left(\mathrm{res}\left( D_1, D_2\right)\right),
  $$
  
  \vspace*{-3pt}
  
  \noindent
   а~следовательно, 
неравенство~(1) будет выполнено, если одновременно будут иметь место 
следующие неравенства:
  \begin{equation*}
  \left\{
  \begin{array}{l}
  \max\left\{ t(D_1^\prime), t(D_2^\prime)\right\} \geq \min \left\{ t(D_1^\prime), 
t(\overline{p})\right\};\\[6pt]
  \max\left\{ t(D_1^\prime), t(D_2^\prime)\right\} \geq \min \left\{ t(D_2^\prime), 
t(p)\right\};\\[6pt]
  \max\left\{ t(D_1^\prime), t(D_2^\prime)\right\} \geq \min \left\{ t(p),  
t(\overline{p})\right\}.
  \end{array}
  \right.
  \end{equation*}
  
  С учетом того что существуют всего 24~случая различных упорядочений 
значений $t(D_1^\prime)$, $t(D_2^\prime)$, $t(p)$ и~$t(\overline{p})$, установлено, что 
одновременно данные неравенства не выполняются, если 
$$
\max \{ t\left(D_1^\prime\right), t\left(D_2^\prime\right)\} < \min \{ t(p), t\left(\overline{p}\right)\}.
$$
 Так как по 
закону де Моргана
$$
\min \left\{ t(p), t\left(\overline{p}\right)\right\} = 1- \max \left\{t(p),  t\left(\overline{p}\right)\right\},
$$
 то предыдущее неравенство равносильно неравенству 
 $$
 \max  \left\{ t\left(D_1^\prime\right), t\left(D_2^\prime\right)\right\} + \max \left\{ t(p), t\left(\overline{p}\right)\right\}<1\,. 
 $$
Таким образом, доказано следующее
  
  \smallskip
  
  \noindent
  \textbf{Утверждение~6.} \textit{Пусть для формализации конъюнкции 
используется норма~$T$, а~дизъюнкции~--- $\max$. Если $\max \{ 
t(D_1^\prime), t(D_2^\prime)\}\hm+ \max \{ t(p), t(\overline{p})\}\hm<1$, то 
неравенство~$(1)$ не выполняется. В~остальных случаях неравенство~$(1)$ 
выполняется, при этом резольвента становится значимым логическим 
следствием, если} $\max \{ t(D_1^\prime), t(D_2^\prime)\} \hm> 0{,}5$.
  
  \smallskip
  
  Заметим, что L-ре\-золь\-вен\-та может быть или не быть значимым 
логическим следствием в~за\-ви\-си\-мости от степени истинности ее слагаемых 
компонент и~переменной~$p$. В~[6] Мукайдоно получил обобщение  
L-ре\-золь\-вен\-ты, сопряженное со сте\-пенью доверия.
  
  Пусть $D_1=p\vee D_1^\prime$ и~$D_2\hm= \overline{p}\vee D_2^\prime$~--- 
дизъюнкты. \textit{Нечеткая резольвента Му\-кай\-до\-но~---  
\mbox{M-ре}\-золь\-вен\-та} дизъюнктов~$D_1$ и~$D_2$, обозначаемая $\mathrm{res}\,(D_1, 
D_2)_{c_p}$, где $c_p\hm= 2\left(\max \{ t(p), t(\overline{p})\}\hm- 
0{,}5\right)$~--- степень доверия к~переменной~$p$, вычисляется по формуле: 

\columnbreak

\noindent
\begin{multline*}
\mathrm{res}\left (D_1,D_2\right)_{c_p} = \mathrm{res}\left(D_1, D_2\right)\vee \left(p\wedge \overline{p}\right),
\\
\mathrm{res}\,(D_1,D_2)\hm= D_1^\prime \vee D_2^\prime.
\end{multline*}


  
  Множество дизъюнктов~$C$ считается невыполнимым тогда и~только тогда, 
когда существует вывод пустого дизъюнкта~$\Box$ с~ненулевой степенью 
доверия~[6].
  
  Используя пару $(T,S)$ двойственных операций, степень истинности  
М-ре\-золь\-вен\-ты запишем в~\mbox{виде}: 
\begin{multline*}
t\left(\mathrm{res}\left(D_1,D_2\right)_{c_p}\right) ={}\\
{}= S\left\{ t\left( \mathrm{res}\left(D_1, D_2\right)\right), 
T\left\{ t(p), t\left(\overline{p}\right)\right\}\right\}.
\end{multline*}
 Учитывая рассуждения при 
доказательстве утверж\-де\-ния~5, используя закон де Моргана и~свойство 
мо\-но\-тон\-ности~$S$, в~предположении, что $1\hm- T\left( t(\overline{p}), 
t(p)\right) \hm\leq T\left( t(\overline{p}), t(p)\right)$, получим
  \begin{multline*}
  t(D_1\wedge D_2) \leq S\left( S\left(t(D_1^\prime), t(D_2^\prime)\right), S(t(p), 
t(\overline{p}))\right)={}\\
  {}=
  S\left( S(t(D_1^\prime), t(D_2^\prime), 1-T(t(\overline{p}), t(p))\right)\leq{}\\
  {}\leq
  S\left( S(t(D_1^\prime), t(D_2^\prime)), T(t(p), t(\overline{p}))\right)={}\\
  {}=
  S\left( t(\mathrm{res}\,(D_1, D_2)), T(t(p), t(\overline{p}))\right)={}\\
  {}=
  t\left( \mathrm{res}\,(D_1, D_2)_{c_p}\right),
  \end{multline*}
а следовательно, неравенство~(1) выполняется, при этом степень истинности 
переменной~$p$ и~треугольная норма~$T$ должны быть такими, чтобы\linebreak $T\left( 
t(\overline{p}), t(p)\right) \hm\geq 0{,}5$. В~данном случае необходимо 
исследование семейств треугольных норм, поз\-во\-ля\-ющих обеспечить 
выполнение данного неравенства, и~в~общем случае ка\-кой-то вывод \mbox{сделать} 
невозможно. При $T\hm=\min$ неравенство $\min \left\{ t(\overline{p}), 
t(p)\right\} \hm\geq 0{,}5$ выполняется только как равенство при 
$t(\overline{p})\hm= t(p)\hm=0{,}5$, что дает воз\-мож\-ность работать 
с~неопределенными высказываниями. Кроме того, для поиска подходящей 
\mbox{треугольной} нормы можно рассматривать семейства, для которых 
<<действует>> отрицание, не являющееся инволюцией.

  Пусть $T=\min$ и~$S\hm=\max$, тогда резольвента имеет вид:
  \begin{multline*}
  t\left( \mathrm{res}\left(D_1, D_2\right)_{c_p}\right) ={}\\
  {}= \max \left\{ \max \left\{ t(D_1^\prime), 
t(D_2^\prime)\right\}, \min \left\{ t(p), t(\overline{p})\right\}\right\}.
\end{multline*}

\begin{figure*}[b] %fig1
\begin{center}
{\small \begin{tabular}{|p{118mm}|}
  \hline
   \textbf{Input:} $C$~--- исходное множество дизъюнктов\\
   \textbf{Input:} $k=0$, $R_k=\varnothing$ (текущее множество резольвент)\\
   \textbf{Output:} Результат работы алгоритма: $\{$$C$ is inconsistent, $C$ is 
unknown$\}$\\
{\footnotesize // упорядочить литеры, образующие контрарные пары, по убыванию степени 
доверия~$c_p$}\\
  {\sf calculateConfidenceLevel} ($c_p$); {\sf sortConfidenceLevels} ($c_p$);\\
  {\footnotesize // вычисление резольвент}\\
  \textbf{while} ($\exists\,D_i\in C$ and $\exists\,D_j\in C$) \textbf{do}\\
  {\footnotesize // если существует значимая резольвента, то добавить ее в~$R_{k+1}$}\\
        \hspace*{5mm}\textbf{if} ($\exists\,\mathrm{res}\,(D_i, D_j)$ is significant) \textbf{do}\\
        \hspace*{5mm}$\mathsf{searchAndChoose} (D_i, D_j, c_p)$; $R_{k+1}=R_k\cup 
\left\{ \mathrm{res}\,(D_i, D_j)\right\}$\\
        \hspace*{5mm}{\footnotesize // проверка: существует ли в~$R_{k+1}$ пустой 
дизъюнкт~$\Box$}\\
        \hspace*{5mm}\textbf{if} ($\Box\in R_{k+1}$) \textbf{then} $C$ is inconsistent;  
\textbf{end program};\\
        \hspace*{5mm}{\footnotesize // если в~$R_{k+1}$ нет $\Box$, то изменить множество~$C$}\\
        \hspace*{5mm}\textbf{else} $C=C\backslash D_i\cup C\backslash D_j \cup\,\mathrm{res}\,(D_i,D_j)$; $k=k+1$;\\
        \hspace*{5mm}\textbf{end if}\\
  \textbf{end while}\\
  {\footnotesize // если в~$C$ нет дизъюнктов для получения резольвенты, то противоречивость $C$ 
неизвестна}\\
   $C$ is unknown;\\
  \textbf{end program}\\
  \hline
  \end{tabular}
  }
  \end{center}
  \Caption{Алгоритм построения M-резольвенты: $C$~--- множество 
дизъюнктов, невыполнимость (противоречивость) которого исследуется}
  \end{figure*}
  
  Рассмотрим
  \begin{multline*}
  t(D_1,D_2) =\min \left\{ t(D_1), t(D_2)\right\} ={}\\
  {}= \min\left\{ \max \left\{
  t(D_1^\prime), t(p)\right\}, \max \left\{ t(D_2^\prime), 
t(\overline{p})\right\}\right\}={}\\
 {}=
  \max\left\{ \min \left\{ t(D_1^\prime), t(D_2^\prime)\right\}, \min\left\{ 
t(D_1^\prime), t(\overline{p})\right\},\right.\\
\left.
  \min\left\{ t(D_2^\prime), t(p)\right\}, \min \left\{ t(p), 
t(\overline{p})\right\}\right\}\leq{}\\
 {}\leq
    \max\left\{ \max\left\{ t(D_1^\prime), t(D_2^\prime)\right\}, \min\left\{ 
t(D_1^\prime), t(\overline{p})\right\},\right.\\
\left.
  \min\left\{ t(D_2^\prime), t(p)\right\}, \min \left\{ t(p), 
t(\overline{p})\right\}\right\}={}
\end{multline*}

\noindent  
\begin{multline*}
    =\!
  \max\!\Bigg\{\!
  \underbrace{\max\!\left\{ \max\left\{ t(D_1^\prime), t(D_2^\prime)\right\}\!, 
\min\left\{ t(p), t(\overline{p})\right\}\!\right\}}_{t(\mathrm{res}\,(D_1,D_2)_{c_p})}, \\[2pt]
\max\left\{ \min \left\{ t(D_1^\prime), t(\overline{p})\right\}, \min\left\{ 
t(D_2^\prime), t(p)\right\}\right\}\Bigg\}.
  \end{multline*}
  
  Заметим, что
  \begin{multline*}
\max\left\{\min \left\{ t\left(D_1^\prime\right), t\left(\overline{p}\right)\right\}, \min \left\{ t\left( D_2^\prime\right), t(p)\right\}\right\}={}\\[2pt]
{}=  
  \min\Bigg\{ 
  \underbrace{\max\left\{ t(D_1^\prime), t(D_2^\prime)\right\}}_{t(\mathrm{res}\,(D_1,D_2))}, 
\underbrace{\max\left\{ t(D_1^\prime), t(p)\right\}}_{t(D_1)},\\[2pt] 
\underbrace{\max\left\{ t(D_2^\prime), t(\overline{p})\right\}}_{t(D_2)}, 
\max\left\{ t(p), t(\overline{p})\right\}\Bigg\}={}\\[2pt]
  {}= \min\Bigg\{
  \underbrace{\min\left\{ t(D_1), t(D_2)\right\}}_{t(D_1\wedge D_2)},\\[2pt]
   \min\left\{ 
t\left(\mathrm{res}\left(D_1, D_2\right)\right), \max\left\{t(p), t(\overline{p})\right\}\right\}
  \Bigg\}.
  \end{multline*}
  
  Таким образом, получено следующее не\-ра\-вен\-ство:
  \begin{multline*}
  t\left(D_1, D_2\right) \leq \max\left\{ t\left( \mathrm{res}\left(D_1, D_2\right)_{c_p}\right)\right.,\\[2pt]
 \min \left\{ t\left(D_1\wedge D_2\right), t\left(\mathrm{res}\left(D_1,D_2\right)\right),\right.\\[2pt]
\left.   \left. \max \left\{ t(p), 
t\left(\overline{p}\right)\right\}\!\right\}\!
\vphantom{\left( \mathrm{res}\left(D_1, D_2\right)_{c_p}\right)}
\right\}.
 \end{multline*}
 
 \noindent
  Здесь $\mathrm{res}\,(D_1, D_2)$~--- \mbox{L-ре}\-золь\-вен\-та. Тогда если \mbox{L-ре}\-золь\-ве\-нта 
логически значима, т.\,е.\ выполняется неравенство $t(D_1\wedge D_2) \hm\leq 
t\left( \mathrm{res}\,(D_1, D_2)\right)$, то
  \begin{multline*}
  t(D_1\wedge D_2) \leq \max \left\{ t\left( \mathrm{res}\left (D_1,D_2\right)_{c_p}\right),\right.\\
\left.  \min \left\{ 
t(D_1\wedge D_2), \max \left\{ t(p), t(\overline{p})\right\}\right\}\right\}.
\end{multline*}
  
  Если $t(D_1\wedge D_2)\hm\leq \max \left\{ t(p), t(\overline{p})\right\}$, то
  $$
  t(D_1\wedge D_2)\leq \max \left\{ t\left( \mathrm{res}\,(D_1, D_2)_{c_p}\right), 
t(D_1\wedge D_2)\right\}
  $$
и неравенство~(1) выполняется.
  
  Если $t(D_1\wedge D_2) \hm> \max \left\{ t(p), t(\overline{p})\right\}$, то
\begin{multline*}
t(D_1\wedge D_2) \leq {}\\
{}\leq
\max\left\{ t\left( \mathrm{res}\left(D_1, D_2\right)_{c_p}\right), \max \left\{ 
t(p), t (\overline{p})\right\}\right\}
\end{multline*}
и неравенство~(1) выполняется, если 
$$
\max \left\{ t(p),t(\overline{p})\right\} 
\leq t\left( \mathrm{res}\left(D_1, D_2\right)_{c_p}\right).
$$
  
  Таким образом, доказано следующее
  
  \begin{figure*}[b] %fig2
\begin{center}
{\small
 \begin{tabular}{|p{124mm}|}
\hline
$C=\left\{ W(x)\vee O(x), Z(y)\vee \neg W(y), Y(A), \neg Z(A), \neg O(z)\vee Y(z)\right\}$\newline
$c_p(Y)=2(0{,}8-0{,}5) =0{,}6$, $c_p(Z)=2(0{,}7- 0{,}5)=0{,}4$\newline
$c_p(O)=2(0{,}6-0{,}5)=0{,}2$, $c_p(W)= 2(0{,}5-0{,}5)=0$\\
\hline
$k=0$, $R_0=\varnothing$; $R_0$ не содержит пустого дизъюнкта; в~$C$ есть унифицируемые 
литералы\\
\hline
$D_i=Y(A)$, $D_j=\neg O(z)\vee \neg Y(z)$ \\
\hline
$\neg O(A)\vee \neg Y(A)=\neg O(A)$, $R_1=\left\{ \neg O(A)\right\}$\\
\hline
$k=1$; $R_1$ не содержит пустого дизъюнкта; в~$C$ есть унифицируемые литералы\newline
$C=\left\{ W(x)\vee O(x), Z(y)\vee \neg W(y), \neg Z(A), \neg O(A)\right\}$\\
\hline
$D_i=Z(y)\vee \neg W(y)$, $D_j=\neg Z(A)$ \\
\hline
$\neg W(A)\vee Z(A) =\neg W(A)$, $R_2=\{ \neg O(A), \neg W(A)\}$ \\
\hline
$k=2$, $R_2$ не содержит пустого дизъюнкта, в~$C$ есть унифицируемые литералы\newline
$C=\left\{ W(x)\vee O(x), \neg O(A), \neg W(A)\right\}$\\
\hline
$D_i=W(x)\vee O(x)$, $D_j=\neg O(A)$, $R_3=\left\{ \neg O(A), \neg W(A), W(A)\right\}$\\
\hline
$k=3$, $R_3$ не содержит пустого дизъюнкта; в~$C$ есть унифицируемые литералы\newline
$C=\left\{ \neg W(A), W(A)\right\}$\\
\hline
$D_i=\neg W(A)$, $D_j=W(A)$, $R_4=\left\{ \neg O(A), \neg W(A), W(A), \Box\right\}$\\
\hline
$R_4$ содержит пустой дизъюнкт~--- исходное множество~$C$ противоречиво.\\
\hline
\end{tabular}
}
\end{center}
\Caption{Работа алгоритма в~интерпретации Мукайдоно}
\end{figure*}


  \smallskip
  
  \noindent
  \textbf{Утверждение~7.} \textit{Если L-ре\-золь\-вен\-та~--- значимое 
логическое следствие и
$$
\max \left\{ t(p), t\left(\overline{p}\right)\right\} \leq t\left( \mathrm{res}
\left(D_1, D_2\right)_{c_p}\right),
$$
 то М-ре\-золь\-вен\-та $t\left( \mathrm{res}\,(D_1,  D_2)_{c_p}\right)$ удовлетворяет неравенству~$(1)$, а~следовательно, также 
становится значимым логическим следствием}.

\section{Нечеткий метод резолюций}
  
  Нечеткий метод резолюций реализуется согласно классической схеме [16] на 
основе определений \mbox{L-ре}\-золь\-вен\-ты или \mbox{M-ре}\-золь\-вен\-ты, при этом 
в~первом случае вычисляются только значимые \mbox{L-ре}\-золь\-вен\-ты. 

При 
нахождении \mbox{M-ре}\-золь\-вен\-ты используются степени доверия литер: чем 
больше~$c_p$, тем построение резольвенты из дизъюнктов, содержащих 
литеру~$p$, происходит раньше. 

Рассмотрим более подробно резолютивный 
вывод на основе \mbox{M-ре}\-золь\-вен\-ты (рис.~1).  Для 
компактного изложения алгоритма введем следующие функции:
  \begin{enumerate}[(1)]
\item функция $\mathsf{calculateConfidenceLevel} (c_p)$ вычисляет степень 
доверия каждой литеры (результат в~$c_p$);
\item функция $\mathsf{sortConfidenceLevels} (c_p)$ сортирует 
последовательность степеней доверия литер в~порядке убывания;
\item функция $\mathsf{searchAndChoose} (D_i, D_j, c_p)$ осуществляет 
поиск дизъюнктов, содержащих литеру с~максимальным значением степени 
доверия~$c_p$. 
\end{enumerate}


  
  Рассмотрим пример. Докажем или опровергнем противоречивость 
следующего множества дизъюнктов:
$  \left\{
  W(x)\vee O(x), Z(y)\vee 
   \neg W(y), Y(A,\right.$\linebreak $\left. \neg Z(A), \neg O(z)\vee \neg 
Y(z)\right\}$,
где $t(W)=0{,}5$; $t(O)\hm= 0{,}4$; $t(Z)\hm= 0{,}3$; $t(Y)\hm= 0{,}8$. 
Результаты работы алгоритма, основанного на вычислении M-ре\-золь\-вен\-ты, 
представлены на рис.~2.


\section{Заключение}

%\vspace*{-6pt}
  
  Метод резолюций прежде всего известен как инструмент для 
автоматического доказательства теорем. В~моделях обработки знаний, 
основанных на правиле дедукции, проблема формулируется в~виде 
совокупности утверж\-де\-ний-ги\-по\-тез и~целевого утверж\-де\-ния~---  
тео\-ре\-мы, справедливость которой следует установить или опровергнуть на 
основе данных гипотез, аксиом и~правил вывода. Данная схема может быть 
положена в~основу различных автоматизированных процедур принятия 
решений в~медицине, экономике, финансовой сфере, судебной практике. 
Метод резолюций используется для моделирования итерационных вычислений 
в~рамках дедуктивного синтеза программ. Теория рассуждений, основанная на 
методе резолюций, стала активно развивающимся направлением 
искусственного\linebreak
 интеллекта. Классический метод резолюций базируется на 
двузначной логике, но зачастую для многих рассуждений характерна 
неопределенность знаний, обусловленная различными факторами. \mbox{Поэтому} 
многие исследователи пытаются обобщить метод резолюций на случай 
неклассических логик и~сделать его таким же эффективным, как в~классической 
логике. 
%
В~данной статье предложены различные обобщения известных 
определений нечетких резольвент с~использованием треугольных норм 
и~конорм, которые формализуют нечеткие\linebreak логические связки \textit{и} 
и~\textit{или}. 
%
Проведенное иссле\-до\-вание показывает, что для того чтобы 
нечеткая резольвента была значимым логическим следствием, необходимо 
выполнение определенных условий, которые уменьшают неопределенность 
и~делают правило нечеткого вывода, опирающееся на резолюционный 
принцип, осмысленным в~разичных интерпретациях.
  
{\small\frenchspacing
 {\baselineskip=11.4pt
 %\addcontentsline{toc}{section}{References}
 \begin{thebibliography}{99}
\bibitem{1-les}
\Au{Robinson J.\,A.} A~machine-oriented logic based on the resolution principle~// J.~ACM, 1965. 
Vol.~12. Iss.~1. P.~23--41. doi: 10.1145/321250.321253.
\bibitem{2-les}
\Au{Newborn M.} Automated theorem proving: Theory and practice.~--- Berlin, Heidelberg: 
Springer-Verlag, 2001. 231~p.
\bibitem{3-les}
\Au{Ламберов Л.\,Д.} Практика компьютерных доказательств и~человеческое понимание: 
эпистемологическая проблематика~// Вестник Пермского университета. Философия. 
Психология. Социология, 2021. №\,1. С.~5--19. doi: 10.17072/2078-7898/2021-1-5-19. 
EDN: \mbox{GTOYOE}.
\bibitem{4-les}
\Au{Новак В., Перфильева~И., Мочкорж~И.} Математические принципы нечеткой  
логики.~--- М.: Физматлит, 2006. 252~с.
\bibitem{5-les}
\Au{Lee R.\,C.\,T.} Fuzzy logic and the resolution principle~// J.~ACM, 1972. Vol.~19. Iss.~1. 
P.~109--119. doi: 10.1145/ 321679.321688.
\bibitem{6-les}
\Au{Mukaidono M.} Fuzzy inference of resolution style~// Fuzzy set and possibility theory~/ Ed. 
R.\,R.~Yager.~--- New York, NY, USA: Pergamon Press, 1988. P.~224--231.
\bibitem{7-les}
\Au{Dubois D., Prade~H.} Necessity and resolution principle~// IEEE T. Syst. Man  
Cyb., 1987. Vol.~17. Iss.~3. P.~474--478. doi: 10.1109/TSMC.1987.4309063.
\bibitem{8-les}
\Au{Guller D.} Hyperresolution for G$\ddot{\mbox{o}}$del logic with truth constants~// Fuzzy Set. 
Syst., 2019. Vol.~363. P.~1--65. doi: 10.1016/j.fss.2018.09.008.


\bibitem{11-les} %9
\Au{Tammet T.} A~resolution theorem prover for intuitonistic logic~// Automated deduction~--- 
Cade-13~/ Eds. M.\,A.~McRobbie, J.\,K.~Slaney.~--- Lecture notes in computer science ser.~--- 
Berlin, Heidelberg: Springer, 1996. Vol.~1104. P.~2--16. doi: 10.1007/3-540-61511-3\_65.
\bibitem{12-les} %10
\Au{Viedma M.\,A.\,C., Morales~R.\,M., Sanchez~I.\,N.} Fuzzy temporal constraint logic: A~valid resolution 
principle~// Fuzzy Set. Syst., 2001. Vol.~117. Iss.~2. P.~231--250. doi:  
10.1016/S0165-0114(99)00099-8.

\bibitem{10-les} %11
\Au{Habiballa H.} Resolution principle and fuzzy logic~// Fuzzy logic~--- algorithms, techniques, 
and implementations~/ Ed. E.~Dadios.~--- London: IntechOpen, 2012. P.~55--74.

\bibitem{13-les} %12
\Au{Nguyen T.\,M.\,T., Tran D.\,A.\,K.} Resolution method in linguistic propositional logic~// Int. J. 
Advanced Computer Science Applications, 2016. Vol.~7. Iss.~1. P.~672--678. doi: 
10.14569/IJACSA.2016.070191.

\bibitem{9-les} %13
\Au{Samokhvalov Y.} Proof of theorems in fuzzy logic based on structural resolution~// Cybern. 
Syst. Anal., 2019. Vol.~55. P.~207--219. doi: 10.1007/s10559-019-00125-8.

\bibitem{14-les}
\Au{Raha S., Ray~K.\,S.} Approximate reasoning based on generalized disjunctive syllogism~// 
Fuzzy Set. Syst., 1994. Vol.~61. Iss.~2. P.~143--151. doi:  
10.1016/0165-0114(94)90230-5.
\bibitem{15-les}
\Au{Mondal B., Raha~S.} Approximate reasoning in fuzzy resolution~// Int. J. Intelligence Science, 
2013. Vol.~3. Iss.~2. P.~86--98. doi: 10.4236/ijis.2013.32010.
\bibitem{16-les}
\Au{Леденева Т.\,М., Лещинская~М.\,В.} Метод резолюций и~стратегии поиска 
опровержений~// Вестник ВГУ. Сер. Системный анализ и~информационные технологии, 
2021. №\,1. С.~98--111. doi: 10.17308/sait.2021.1/3374. EDN: RBAVMW.
\bibitem{17-les}
\Au{Klement E.\,P., Mesiar~R., Pap~E.} Triangular norms. Position paper~II: General constructions and 
parameterized families~// Fuzzy Set. Syst., 2004. Vol.~145. P.~411--438. doi:  
10.1016/S0165-0114(03)00327-0.


\bibitem{19-les} %18
\Au{Ledeneva T.} Additive generators of fuzzy operation in the form of linear fractional function~// 
Fuzzy Set. Syst., 2020. Vol.~386. P.~1--24. doi: 10.1016/j.fss.2019.03.005.

\bibitem{18-les} %19
\Au{Ledeneva T.} New family of triangular norms for decreasing generators in the form of 
a~logarithm of a~linear fractional function~// Fuzzy Set. Syst., 2022. Vol.~427. P.~37--54. doi: 
10.1016/j.fss.2020.11.020.

\bibitem{20-les}
\Au{Вагин В.\,Н., Головина~Е.\,Ю., Загорянская~А.\,А., Фомина~М.\,В.} Достоверный 
и~правдоподобный вывод в~интеллектуальных системах~/ Под ред. В.\,Н.~Вагина, 
Д.\,А.~Поспелова.~--- М.: Физматлит, 2008. 712~с. EDN: MUWRTJ.
\end{thebibliography}

 }
 }

\end{multicols}

\vspace*{-6pt}

\hfill{\small\textit{Поступила в~редакцию 24.04.23}}

\vspace*{8pt}

%\pagebreak

%\newpage

%\vspace*{-28pt}

\hrule

\vspace*{2pt}

\hrule



\def\tit{ANALYSIS OF APPROACHES TO~DEFINING FUZZY RESOLVENT}


\def\titkol{Analysis of approaches to~defining fuzzy resolvent}


\def\aut{T.\,M.~Ledeneva and~M.\,V.~Leshchinskaya}

\def\autkol{T.\,M.~Ledeneva and~M.\,V.~Leshchinskaya}

\titel{\tit}{\aut}{\autkol}{\titkol}

\vspace*{-8pt}


\noindent
Voronezh State University, 1~Universitetskaya Sq., Voronezh 394010, Russian 
Federation

\def\leftfootline{\small{\textbf{\thepage}
\hfill INFORMATIKA I EE PRIMENENIYA~--- INFORMATICS AND
APPLICATIONS\ \ \ 2024\ \ \ volume~18\ \ \ issue\ 1}
}%
 \def\rightfootline{\small{INFORMATIKA I EE PRIMENENIYA~---
INFORMATICS AND APPLICATIONS\ \ \ 2024\ \ \ volume~18\ \ \ issue\ 1
\hfill \textbf{\thepage}}}

\vspace*{4pt}


\Abste{The article presents the results of a study concerning various definitions of 
the resolvent in fuzzy logic. Conditions are defined under which the Lee resolvent is 
a~significant logical consequence in the case of the classical definition of fuzzy 
logical connectives. It is shown that when using triangular norms and conorms for 
their formalization, it is impossible to obtain a~logically significant Lee resolvent. 
However, if the triangular conorm is defined as max, then the Lee resolvent exists for 
any triangular norm. Conditions are defined under which the Mukaidano resolvent is 
a~significant logical consequence for classical min and max operations. When using\linebreak\vspace*{-12pt}}

\pagebreak

\Abstend{triangular norms and conorms other than classical ones, further research is required. 
An illustrative example is provided demonstrating the process of constructing the 
Mukaidano resolvent.}

\KWE{resolvent; resolution method; triangular norms and conorms}

  \DOI{10.14357/19922264240113}{MIDEHY}

%\vspace*{-12pt}

%\Ack
%\vspace*{-3pt}
%\noindent
 


  \begin{multicols}{2}

\renewcommand{\bibname}{\protect\rmfamily References}
%\renewcommand{\bibname}{\large\protect\rm References}

{\small\frenchspacing
 {%\baselineskip=10.8pt
 \addcontentsline{toc}{section}{References}
 \begin{thebibliography}{99} 
\bibitem{1-les-1}
\Aue{Robinson, J.\,A.} 1965. A~machine-oriented logic based on the resolution 
principle.  \textit{J.~ACM} 12(1):23--41. doi: 10.1145/321250.321253.
\bibitem{2-les-1}
\Aue{Newborn, M.} 2000. \textit{Automated theorem proving: Theory and practice}. 
New York, NY: Springer. 231~p.
\bibitem{3-les-1}
\Aue{Lamberov, L.\,D.} 2021. Praktika komp'yuternykh dokazatel'stv 
i~chelovecheskoe ponimanie: epi\-ste\-mo\-lo\-gi\-che\-skaya problematika [Computer 
proofs practice and human understanding: Epistemological issues]. \textit{Vestnik 
Permskogo universiteta. Filosofiya. Psikhologiya. So\-tsi\-o\-lo\-giya} [Perm University 
Herald. Philosophy. Psychology. Sociology] 1:5--19.  
doi: 10.17072/2078-7898/2021-1-5-19. EDN: GTOYOE.
\bibitem{4-les-1}
\Aue{Novak, V., I.~Perfil'eva, and I.~Mochkorzh}. 2006. \textit{Ma\-te\-ma\-ti\-che\-skie 
prin\-tsi\-py ne\-chet\-koy logiki} [Mathematical principles of fuzzy logic]. Moscow: 
Fizmatlit. 252~p.
\bibitem{5-les-1}
\Aue{Lee, R.\,C.\,T.} 1972. Fuzzy logic and the resolution principle. \textit{J.~ACM} 
19(1):109--119. doi:10.1145/321679.321688.
\bibitem{6-les-1}
\Aue{Mukaidono, M.} 1988. Fuzzy inference of resolution style. \textit{Fuzzy set and 
possibility theory}. Ed. R.\,R.~Yager.  New York, NY: Pergamon Press. 224--231.
\bibitem{7-les-1}
\Aue{Dubois, D., and H.~Prade.} 1987. Necessity and resolution principle. 
\textit{IEEE T. Syst. Man Cyb.} 17(3):474--478. doi: 
10.1109/TSMC.1987.4309063.
\bibitem{8-les-1}
\Aue{Guller, D.} 2019. Hyperresolution for G$\ddot{\mbox{o}}$del logic with truth 
constants. \textit{Fuzzy Set. Syst.} 363:1--65. doi: 10.1016/ j.fss.2018.09.008.


\bibitem{11-les-1} %9
\Aue{Tammet, T.} 1996. A resolution theorem prover for intuitionistic logic. 
\textit{Automated deduction~--- Cade-13}. Eds. M.\,A.~McRobbie and 
J.\,K.~Slaney. Lecture notes in computer science ser. Berlin, Heidelberg: Springer. 
1104:2--16. doi: 10.1007/3-540-61511-3\_65.
\bibitem{12-les-1} %10
\Aue{Viedma, M.\,A.\,C., R.\,M.~Morales, and I.\,N.~Sanchez.} 2001. Fuzzy temporal 
constraint logic: A~valid resolution principle. \textit{Fuzzy Set. Syst.} 
117(2):231--250. doi: 10.1016/ S0165-0114(99)00099-8.
\bibitem{10-les-1} %11
\Aue{Habiballa, H.} 2012. Resolution principle and fuzzy logic. \textit{Fuzzy logic 
algorithms, techniques, and implementations}. Ed. E.~Dadios.  London: Intech.  
55--74.
\bibitem{13-les-1} %12
\Aue{Nguyen, T.\,M.\,T., and D.\,A.\,K.~Tran.} 2016. Resolution method in linguistic 
propositional logic. \textit{Int. J. Advanced Computer Science Applications}  
7(1):672--678.  doi: 10.14569/IJACSA.2016.070191.

\bibitem{9-les-1} %13
\Aue{Samokhvalov, Y.} 2019. Proof of theorems in fuzzy logic based on structural 
resolution. \textit{Cybern. Syst. Anal.} 55:207--219. doi:  
10.1007/s10559-019-00125-8.
\bibitem{14-les-1}
\Aue{Raha, S., and K.\,S.~Ray.} 1994. Approximate reasoning based on generalized 
disjunctive syllogism. \textit{Fuzzy Set. Syst.} 61(2):143--151.  
doi: 10.1016/0165-0114(94)90230-5.
\bibitem{15-les-1}
\Aue{Mondal, B., and S.~Raha.} 2013. Approximate reasoning in fuzzy resolution. 
\textit{Int. J. Intelligence Science} 3(2):86--98. doi: 10.4236/ijis.2013.32010.
\bibitem{16-les-1}
\Aue{Ledeneva, T.\,M., and M.\,V.~Leshchinskaya.} 2021. Metod rezolyutsiy 
i~strategii poiska oproverzheniy [The resolution inference and strategies for finding 
refutation]. \textit{Vestnik VGU. Ser. Sistemnyy analiz i~informatsionnye 
tekhnologii} [Proceedings of Voronezh State University. Ser. Systems Analysis and 
Information Technologies] 1:98--111. doi: 10.17308/sait.2021.1/3374. EDN: 
RBAVMW.
\bibitem{17-les-1}
\Aue{Klement, E.\,P., R.~Mesiar, and E.~Pap.} 2004. Triangular norms. Position 
paper II: General constructions and parameterized families. \textit{Fuzzy Set. Syst.} 
145(3):411--438. doi: 10.1016/S0165-0114(03)00327-0.

\bibitem{19-les-1}
\Aue{Ledeneva, T.} 2020. Additive generators of fuzzy operations in the form of linear 
fractional functions. \textit{Fuzzy Set. Syst.} 386:1--24. doi: 
10.1016/j.fss.2019.03.005.
\bibitem{18-les-1}
\Aue{Ledeneva, T.} 2022. New family of triangular norms for decreasing generators in 
the form of a~logarithm of a~linear fractional function. \textit{Fuzzy Set. Syst.} 
427:37--54. doi: 10.1016/j.fss.2020.11.020.

\bibitem{20-les-1}
\Aue{Vagin, V.\,N., E.\,Yu.~Golovina, A.\,A.~Zagoryanskaya, and M.\,V.~Fomina}. 
2008. \textit{Dostovernyy i~pravdopodobnyy vyvod v~intellektual'nykh sistemakh} 
[Reliable and plausible conclusion in intelligent systems]. Eds. V.\,N.~Vagin and 
D.\,A.~Pospelov. Moscow: Fizmatlit. 712~p. EDN: MUWRTJ.

\end{thebibliography}

 }
 }

\end{multicols}

\vspace*{-6pt}

\hfill{\small\textit{Received April 24, 2023}} 

\vspace*{-18pt}
     
     \Contr
     
     \vspace*{-3pt}

\noindent
\textbf{Ledeneva Tatyana M.} (b.\ 1959)~--- Doctor of Science in technology, 
professor, Department of Computational Mathematics and Applied Information 
Technologies, Faculty of Applied Mathematics, Informatics and Mechanics, Voronezh 
State University, 1~Universitetskaya Sq., Voronezh 394010, Russian Federation; 
\mbox{ledeneva-tm@yandex.ru}

\vspace*{3pt}

\noindent
\textbf{Leshchinskaya Maria V.} (b.\ 1995)~--- PhD student, Faculty of Applied 
Mathematics, Informatics and Mechanics, Voronezh State University, 
1~Universitetskaya Sq., Voronezh 394010, Russian Federation; maria-
\mbox{leshchinskaya@mail.ru}



\label{end\stat}

\renewcommand{\bibname}{\protect\rm Литература} 