\def\ora{\overrightarrow}
\def\ol{\overline}


\def\stat{malashenko}

\def\tit{СРАВНИТЕЛЬНЫЙ АНАЛИЗ УЗЛОВЫХ МУЛЬТИПОТОКОВ В~МНОГОПОЛЬЗОВАТЕЛЬСКОЙ 
СЕТЕВОЙ СИСТЕМЕ}

\def\titkol{Сравнительный анализ узловых мультипотоков в~многопользовательской 
сетевой системе}

\def\aut{Ю.\,Е.~Малашенко$^1$, И.\,А.~Назарова$^2$}

\def\autkol{Ю.\,Е.~Малашенко, И.\,А.~Назарова}

\titel{\tit}{\aut}{\autkol}{\titkol}

\index{Малашенко Ю.\,Е.}
\index{Назарова И.\,А.}
\index{Malashenko Yu.\,E.}
\index{Nazarova I.\,A.}


%{\renewcommand{\thefootnote}{\fnsymbol{footnote}} \footnotetext[1]
%{Работа выполнена при финансовой поддержке РНФ (проект 21-71-10135).}}


\renewcommand{\thefootnote}{\arabic{footnote}}
\footnotetext[1]{Федеральный исследовательский центр <<Информатика и~управление>> Российской академии наук, \mbox{malash09@ccas.ru}}
\footnotetext[2]{Федеральный исследовательский центр <<Информатика и~управление>> Российской академии наук, \mbox{irina-nazar@yandex.ru}}

\vspace*{-12pt}


 

\Abst{В рамках вычислительных экспериментов изучаются монопольные 
и~предельные режимы передачи исходящих узловых мультипотоков в~многопользовательской сетевой системе. Анализируются межузловые потоки разных 
видов, которые передаются всеми корреспондентами из всех узлов по всем 
кратчайшим маршрутам. В~рамках алгоритмической схемы вычисляются равные  
межузловые   потоки. Найденные значения выступают в~качестве  компонент  вектора 
мультипотока, исходящего из каждого уз\-ла-ис\-точ\-ни\-ка во все 
уз\-лы-ад\-ре\-са\-ты, и~интерпретируются как гарантированные многокритериальные оценки показателей  
функционирования  многопользовательской  системы. Для оценки минимальных 
удельных затрат для каждого узла используется  монопольный режим передачи 
исходящего мультипотока без учета всех остальных. Для поиска максимально 
достижимого межузлового потока  рассматривается предельный режим передачи, при 
котором полностью  загружены все  ребра  сети. Исследуются сети с~различными 
структурными особенностями и~одинаковой суммарной пропускной способностью.}

\KW{многопродуктовая потоковая модель;  распределение 
узловых исходящих мультипотоков; удельные  затраты при передаче потока}

\DOI{10.14357/19922264240106}{AKCMCQ}
  
%\vspace*{-6pt}


\vskip 10pt plus 9pt minus 6pt

\thispagestyle{headings}

\begin{multicols}{2}

\label{st\stat}


\section{Введение}

Работа продолжает исследование функциональных характеристик сетевых 
телекоммуникационных  систем при различных способах передачи потока, 
распределения ресурсов и~управ\-ле\-ния~[1--6]. В~\cite{Mal23-1, Mal23-3, Mal23-6} в~качестве 
многопараметрической оценки функциональных возможностей многопользовательской 
сети используются векторы исходящих узловых мультипотоков. Компонентами вектора 
выступают все межузловые потоки с~источником в~выбранной вершине и~стоками в~остальных узлах сети. В~ходе вычислительных экспериментов вначале определяются 
распределения  межузловых потоков, которые передаются корреспондентами по 
кратчайшим маршрутам. Затем на основе найденных значений  формируются допустимые 
векторы исходящих мультипотоков,  передаваемых в~монопольном режиме без учета 
потоков для остальных узлов сети. Анализируются  величины  узловых исходящих  
потоков в~ситуациях, когда внешняя нагрузка превышает нормативные ограничения.

В работе изучаются пиковые распределения при полной загрузке сети и~максимальном 
суммарном межузловом потоке. В~рамках  модели  пропускная способность  ребер 
сети рассматривается как вектор ресурсов. Ресурсы распределяются  для 
обеспечения  передачи потоков соответствующих видов.  Для каждого узла 
оцениваются требуемые  затраты ресурсов  при  передаче всех исходящих межузловых 
потоков, для которых  данный узел служит источником.   Предложенный способ  
моделирования дает возможность сравнивать и~выявлять различия в~распределении 
ресурсов при использовании  различных диспетчерских правил стратегий.  
Полученные  оценки можно трактовать как ограничения на величину исходящих 
узловых мультипотоков, которые можно будет передать в~сети одновременно с~потоками из других узлов, если  условия совместной передачи на остальные виды 
потоков не будут нарушены.

Результаты, полученные в~ходе экспериментов,  можно также рассматривать как 
описание  функциональных возможностей сети при предельных внешних входных 
потоках. Появляется возможность  на этапе предпроектного анализа проследить  
взаимосвязь узловых исходящих мультипотоков и~требуемых затрат ресурсов --- 
пропускной способности ребер сети~--- при использовании монопольных и~уравнительных диспетчерских правил.

\vspace*{-9pt}

\section{Математическая модель}

\vspace*{-3pt}

Для описания многопользовательской сетевой системы связи  воспользуемся 
следующей математической записью модели передачи многопродуктового потока.
Сеть~$G$ задается множествами\linebreak  $\langle V,\ R, \ U,\ P \rangle$:
узлов (вершин) сети  $V \hm= \{ v_1, v_2, \ldots\linebreak \ldots, v_n, \ldots, v_N \}$;
неориентированных ребер $R \hm= \{ r_1, r_2, \ldots, r_k, \ldots, r_E \}$;
ориентированных дуг\linebreak\vspace*{-12pt}

\pagebreak

\noindent
  $U \hm= \{ u_1, u_2, \ldots, u_k, \ldots, u_{2E}\}$
и~пар уз\-лов-кор\-рес\-пон\-ден\-тов $P \hm= \{ p_1, p_2, \ldots, p_M\}$.
Предполагается, что в~сети отсутствуют петли и~сдвоенные ребра.


Ребро $r_k \in R$ соединяет \textit{смежные} вершины $v_{n_k}$ и~$v_{j_k}$.
Каждому ребру~$r_k$ ставятся в~соответствие две ориентированные дуги~$u_k$ 
и~$u_{k+E}$ из множества~$U$.\linebreak
Дуги $\{u_k, u_{k+E}\}$ определяют прямое и~обратное направ\-ле\-ние передачи потока 
по  ребру~$r_k$ между концевыми вершинами $v_{n_k}$ и~$v_{j_k}$. Для каждой 
вершины~$v_n$ формируется список~$K(n)$ номеров инцидентных ей ребер: $K(n) \hm= 
\{k_1, k_2, \ldots, k_a\}$, где $a$~--- число инцидентных ребер для~$v_n$.

В многопользовательской сети~$G$ рассматривается $M \hm= N (N\hm-1)$ независимых, 
невзаимозаменяемых и~равноправных межузловых потоков различных видов.
Каждой паре уз\-лов-кор\-рес\-пон\-ден\-тов~$p_m$ из множества~$P$ соответствуют:
вер\-ши\-на-ис\-точ\-ник с~номером~$s_m$,  из~$s_m$  входной поток $m$-го вида поступает в~сеть;
вер\-ши\-на-при\-ем\-ник с~номером ${t_m}$, из  ${t_m}$ поток $m$-го вида покидает сеть.
Для каждой вершины $v_n \hm\in V$, $n \hm= \ol {1, N}$, определяется подмножество~$P(v_n)$ 
всех пар-кор\-рес\-пон\-ден\-тов, для которых вершина~$v_n$ служит уз\-лом-ис\-точ\-ни\-ком:
$$
P(v_n) =\left \{ p_m | s_m = n, t_m \not = n, t_m = \ol{1, N} \right\},
$$
а для каждого $P(v_n)$~--- список номеров~$M(n)$ пар~$p_m$, входящих в~подмножество~$P(v_n)$:
$$
M(n) = \left\{m_1(n), m_2(n), \ldots , m_{N-1}(n)\right\}.
$$

Обозначим через~$z_m$ величину \textit{межузлового} потока $m$-го вида, 
поступающего в~сеть через узел с~номером~$s_m$ и~покидающего сеть из узла с~номером~$t_m$;
$x_{mk}$ и~$x_{m(k + E)}$~---  поток $m$-го вида, который передается по дугам 
$u_k$ и~$u_{k + E}$ согласно направлению передачи, $x_{mk} \hm\ge 0$, $x_{m(k + 
E)}\hm\ge 0$, $m \hm= \ol{1, M}$, $k \hm= \ol {1, E}$;
$S(v_n)$~--- множество номеров исходящих дуг, по ним поток покидает узел~$v_n$;
$T(v_n)$~--- множество номеров входящих дуг, по ним поток поступает в~узел~$v_n$.
Состав множеств~$S(v_n)$ и~$T(v_n)$ однозначно формируется в~ходе выполнения 
следующей процедуры. Пусть некоторое ребро $r_k \hm\in R$ соединяет вершины с~номерами~$n$ и~$j$, такими что $n \hm< j$. Тогда ориентированная дуга $u_k \hm= (v_n, 
v_j)$, направленная из вершины~$v_n$ в~$v_j$, считается \textit{исходящей} из 
вершины~$v_{n}$, и~ее номер~$k$ заносится в~множество~$S(v_n)$, 
а~дуга~$u_{k+E}$, направленная из~$v_j$ в~$v_n$,~--- \textit{входящей} для~$v_{n}$, и~ее номер $k\hm+E$ помещается в~список~$T(v_n)$.
Дуга~$u_k$ является \textit{входящей} для~$v_j$, и~ее номер~$k$ попадает в~$T(v_j)$, а~дуга~$u_{k+E}$~--- \textit{исходящей}, и~номер $k\hm+E$ вносится 
в~список исходящих дуг~$S(v_j)$.

Во всех узлах сети $v_n \hm\in V$, $n = \ol{1,N}$,  для каж\-до\-го вида потока должны 
выполняться условия сохранения потоков:

\noindent
\begin{multline}
\sum\limits_{i \in S(v_n)}{x_{mi}} - \sum\limits_{i \in T(v_n)}{x_{mi}} ={}\\
{}=
\begin{cases}
\ \ z_m, & \mbox{если } v_n = v_{s_m}, \\
- z_m, & \mbox{если } v_n = v_{t_m}, \\
\ \ 0 & \mbox{в\ остальных\ случаях,}
\end{cases}
\\
n = \ol {1, N},\  m \hm= \ol {1, M},\  x_{mi} \hm\ge 0,\  z_m \hm\ge 0\,.
\label{e1-mal}
\end{multline}
Величина $z_m$ равна входному межузловому потоку $m$-го вида, проходящему от 
источника~$s_m$ к~приемнику~$t_m$ пары~$p_m$ при распределении  потоков~$x_{mi}$ 
по дугам сети.

Каждому ребру $r_k \in R$ приписывается неотрицательное число~$d_k$~--- 
суммарный предельно допустимый поток, который можно передать по реб\-ру~$r_k$ в~обоих направлениях. В~исходной сети 
\mbox{компоненты} вектора пропускных способностей   
$\mathbf{d} \hm= (d_1, d_2, \ldots, d_k, \ldots, d_E)$~---  положительные числа $d_k \hm> 
0$.  Вектор~$\mathbf{d}$ определяет следующие ограничения на сумму потоков всех 
видов, передаваемых по ребру~$r_k$ одновременно:
\begin{multline}
\sum\limits_{m=1}^{M} {(x_{mk}+ x_{m(k+E)})} \le d_k,  \\
  x_{mk} \ge 0, \enskip  
x_{m(k+E)} \ge 0, \enskip   k =\ol{1, E}. \label{e2-mal}
\end{multline}

Ограничения~(1) и~(2) задают множество  до\-пус\-ти\-мых значений компонент вектора 
межузловых потоков
$\mathbf{z} \hm= (z_1, z_2, \ldots, z_m, \ldots, z_M)$:
\begin{multline*} 
\mathcal{Z}(\mathbf{d}) = \{\mathbf{z} \ge 0 \ |\  \exists \ \mathbf{x} \ge 0: \ 
(\mathbf{z}, \mathbf{x}) \\
 \mbox{ удовлетворяют}\ (1), (2)\}.
\end{multline*}

В рамках формализма модели для каждой вершины $v_n \hm\in V$, $n \hm= \ol{1, N}$, 
формируется VMF-век\-тор (от \textit{англ.}\ vertex multiflow) исходящего мультипотока 
$\ora{\mathbf{z}}(n)$, $\ora{\mathbf{z}}(n)\hm \in \mathcal{Z}(\mathbf{d})$, 
компонентами которого выступают $z_j(n) \hm= 0$, $j \not \in M(n)$, $z_j(n) \hm= z_m$, 
$j \hm= m$, $m \hm\in M(n)$. Таким образом, ненулевыми компонентами 
$\ora{\mathbf{z}}(n)$ становятся межузловые потоки $z_m\hm > 0$ для всех пар $p_m 
\hm\in P(v_n)$ с~уз\-ла\-ми-ис\-точ\-ни\-ками~$s_m$, расположенными в~вершине~$v_n$.

\vspace*{-9pt}

\section{Уравнительное распределение в~монопольном и~предельном режимах}

\vspace*{-2pt}

При проведении вычислительных экспериментов векторы~$\ora{\mathbf{z}}(n)$ 
формируются  на основе найденных распределений совместно до\-пус\-ти\-мых межузловых 
потоков~$z_m$, $p_m \hm\in P(v_n)$, передаваемых \mbox{одновременно}. Для каждой пары 
уз\-лов-кор\-рес\-пон-\linebreak\vspace*{-12pt}

\pagebreak

\noindent
ден\-тов $p_m \hm\in P$, для некоторого заданного допустимого 
межузлового потока~$\tilde{z}_m$ и~соответствующих значений дуговых потоков~$\tilde{x}_{mk}$,  $k\hm = \ol{1, 2E}$, \mbox{величина}

\vspace*{2pt}

\noindent
$$
\tilde{y}_m = \sum\limits_{i=1}^{2E} \tilde{x}_{mi}, \ m = \ol{1, M},
$$

\vspace*{-2pt}

\noindent
характеризует результирующую межузловую \textit{нагрузку} (далее~--- RI-на\-груз\-ку, 
от \textit{англ.}\ resulting internodal load) на ребра сети  при передаче  межузлового 
потока~$\tilde{z}_m$ из уз\-ла-ис\-точ\-ни\-ка~$s_m$  в~узел-при\-ем\-ник~$t_m$. Величина 
$\tilde{y}_m$ показывает, какая суммарная пропускная способность сети 
потребуется для передачи дуговых потоков~$\tilde{x}_{mk}$. В~рамках модели 
отношение RI-на\-груз\-ки и~межузлового потока

\vspace*{2pt}

\noindent
$$
 \tilde{w}_m = \fr{\tilde{y}_m}{\tilde{z}_m},  \ m = \ol{1, M},
 $$

 \vspace*{-2pt}

\noindent
можно трактовать как удельные \textit{затраты}  ресурсов сети при передаче 
единичного   потока $m$-го вида между узлами~$s_m$ и~$t_m$ при  дуговых потоках~$\tilde{x}_{mi}$.

Для получения гарантированных оценок VMF-век\-то\-ров предполагается, что межузловые 
потоки передаются по маршрутам, содержащим минимальное число ребер (далее~--- 
MER-марш\-ру\-ты, от \textit{англ.}\ mi\-ni\-mum edge route).
При проведении вычислительных экспериментов для оценки величин <<расщепленного>> 
потока на первом этапе для каждой пары узлов $p_m\hm = (s_m, t_m)$ в~сети~$G(1)$ 
формируется набор~$H_m(1)$ всех  путей, которые далее используются как MER-марш\-ру\-ты передачи $m$-го вида потока:

\vspace*{2pt}

\noindent
$$
H_m(1) = \left\{ h_m^1(1), h_m^2(1), \ldots, h_m^j(1), \ldots, h_m^{J_m(1)}(1)\right\},
$$

\vspace*{-2pt}

\noindent
где $h_m^j(1)$~--- список номеров дуг в~$j$-м  пути в~сети $G(1)$ между узлами~$s_m$ и~$t_m$;  $\mu_m(1)$~--- минимальное число ребер в~MER-марш\-ру\-те 
$h_m^j(1)$; $J_m(1)$~--- число MER-марш\-ру\-тов для $m$-й пары.

Для каждой пары $p_m \hm\in P$ по всем MER-марш\-ру\-там из~$H_m(1)$ передается 
единичный межузловой поток~$z_m$ и~подсчитываются значения индикаторной функции:

\noindent
$$ 
\eta_k^j(m) = \begin{cases}
 1, & \mbox{если}\ k \in h_m^j(1); \\
 0 & \mbox{в\ остальных\ случаях.}
\end{cases}
 $$
 
 \vspace*{-2pt}

\noindent
Определяются  дуговые потоки для пары~$p_m$:
\begin{equation}
x_{mk}^0 (1) = \sum\limits_{j=1}^{J_m(1)} \eta_k^j(m), \ \ m = \ol{1, M}, \ \ k = 
\ol{1, 2E}. \label{e3-mal}
\end{equation}

\vspace*{-2pt}


Межузловой поток по MER-марш\-ру\-там (далее~---  MER-по\-ток) $z_{m}^0(1)$ между 
узлами~$s_m$ и~$t_m$ вычисляется  по формулам~(1) и~(3). Рассчитываются  
нормирующий коэффициент

\vspace*{2pt}

\noindent
\begin{equation*}
\omega_m^0(1) = \fr{1}{z_{m}^0(1)}, \ z_{m}^0(1) \not = 0, \ \ m = \ol{1, M}, %\label{e4-mal}
\end{equation*}

\noindent
и дуговые потоки
\begin{equation}
x_{mk}^0 = \omega_m^0 x_{mk}^0(1), \ \ m = \ol{1, M}, \ \ k = \ol{1, 2E}.\label{e5-mal}
\end{equation}
При передаче всех потоков $x_{mk}^0$ по ребрам сети межузловой поток из узла~$s_m$ в~узел~$t_m$ равен единице для всех $p_m \hm\in P$.

Распределение совместно допустимых межузловых потоков последовательно 
определяется на основе MER-марш\-ру\-тов и~дуговых потоков~$x_{mk}^0$. Для каждого 
VMF-вектора вычисляются равные значения исходящих мультипотоков, которые могут 
передаваться одновременно в~сети~$G(1)$ без учета остальных корреспондентов. 
Указанный способ называется далее \textit{монопольным} режимом передачи VMF-по\-тока.

\smallskip

\noindent
\textbf{Задача~1.} Для заданной вершины~$v_n$ найти 

\noindent
$$
\beta^*(n) =  \max\limits_{\beta} \beta
$$

\vspace*{-4pt}

\noindent
при условиях: 
$$
\beta \!\!\!\sum\limits_{m \in M(n)} \left[ x_{mk}^0+  x_{m(k+E)}^0\right] 
\le d_k(1), \ \beta \ge 0, \  \  k =\ol{1, E}.
$$

\begin{figure*}[b] %fig1
%\vspace*{6pt}
      \begin{center}
     \mbox{%
\epsfxsize=153.408mm 
\epsfbox{mal-1.eps}
}
\end{center}
\vspace*{-9pt}
\Caption{Базовая~(\textit{а}) и~кольцевая~(\textit{б}) сети}
%\end{figure*}
%\begin{figure*} %fig2
\vspace*{12pt}
      \begin{center}
     \mbox{%
\epsfxsize=163mm 
\epsfbox{mal-3.eps}
}
\end{center}
\vspace*{-9pt}
 \Caption{Узловые мультипотоки и~ресурсы в~базовой~(\textit{а}) 
 и~в~кольцевой~(\textit{б}) сетях }
\end{figure*}

\vspace*{-2pt}

С помощью решения задачи~1 для всех $p_m \hm\in P$, $m \hm\in M(n)$, вычисляются VMF-век\-то\-ры~$\ora{\mathbf{z}}^*(n)$, 
все компоненты которых равны~$\beta^*(n)$, 
т.\,е.\ $z_m^*(n) \hm= \beta^*(n)$, $m \hm\in M(n)$,  $n \hm=\ol{1, N}$. Для каждого ребра~$r_k$, $k \hm=\ol{1, E}$, определяется
$$ 
\Delta_k^*(n) =  \beta^*(n)\!\!\! \sum\limits_{m \in M(n)} \left[ x_{mk}^0+  x_{m(k+E)}^0\right], \ \  
k =\ol{1, E}.
$$

\vspace*{-3pt}

\noindent
Векторы ресурсов $\ora\Delta^*(n) \hm= \{\Delta_1^*(n), \Delta_2^*(n), \ldots\linebreak$ $\ldots ,
\Delta_E^*(n)\}$ показывают, какие затраты пропускной способности требуются при 
передаче исходящих межузловых потоков из вершины~$v_n$ в~монопольном режиме без 
учета всех остальных корреспондентов. Для каждого VMF-век\-то\-ра подсчитываются

\noindent
\begin{equation*}
\zeta(n) = \!\!\!\!\sum\limits_{m \in M(n)}\!\! \!\!\!\!z_m^*(n) = \beta^*(n) (N - 1);\enskip
\delta(n) 
=\sum\limits_{k = 1}^E \Delta_k^{*}(n);
\end{equation*}

\vspace*{-12pt}

\noindent
\begin{multline*}
\left\|\ora{\mathbf{z}}^{*}(n)\right\| = \beta^*(n)(N - 1)^{1/2}; \\
\left\|\ora\Delta^{*}(n)\right\|  = \left[\sum\limits_{k = 1}^E (\Delta_k^*(n))^2\right]^{1/2}, \enskip n =\ol{1, N}\,.
\end{multline*}

\vspace*{-3pt}

\noindent
Для оценки VMF-векторов при полной загрузке всех ребер сети сформируем вектор  
$\ora{\mathbf{z}}^{**}(n)$, компоненты которого определяются по следующему 
правилу: для каждого узла~$v_n$  и~всех пар $p_m\hm \in P(v_n)$, которые не 
являются смежными узлами и~кратчайший маршрут соединения для них состоит из двух и~более ребер, поток $z^{**}_m(n) \hm= 0$. Для всех пар $p_a \hm\in P(v_n)$, для 
которых кратчайшим маршрутом соединения служит некоторое ребро~$r_k$, поток 
$z^{**}_a(n) \hm= d_k(1)/2$, где $d_k(1)$~--- пропускная способность ребра~$r_k$ в~сети~$G(1)$.

При одновременной передаче всех VMF-по\-то\-ков $\ora{\mathbf{z}}^{**}(n)$ из всех 
узлов~$v_n$, $n \hm= \ol{1, N}$, достигается полная загрузка всех ребер сети~$G(1)$, 
и~максимальное суммарное значение допустимых межузловых потоков~$z^{**}$ 
равно 
$$ 
z^{**} = \sum\limits_{n = 1}^N \sum_{m = 1}^{M(n)} z^{**}_m(n) = \sum\limits_{k = 1}^E 
d_k(1) = D(0). 
$$

Вычисляется норма векторов
\begin{multline*}
||\ora{\mathbf{z}}^{**}(n)|| = \left[\sum\limits_{k \in 
K(n)}\left(\fr{d_k(1)}{2}\right)^2\right]^{1/2} = {}\\[4pt]
{}=  ||\ora\Delta^{**}(n)||,\enskip n  = \ol{1, N},
\end{multline*}
где $\ora\Delta^{**}(n)$~--- вектор ресурсов, необходимых при передаче 
мультипотока~$\ora{\mathbf{z}}^{**}(n)$.
Значения~$x_{mk}^0$ из~(\ref{e5-mal}) позволяют формировать VMF-век\-тор равных межузловых 
потоков $\ora{\mathbf{z}}^{(=)}(n)$, которые могут передаваться одновременно из 
всех узлов сети.

\smallskip

\noindent
\textbf{Задача~2.} Найти 
$$
\alpha^*(n) = \max\limits_{\alpha} \alpha
$$

\noindent
при условиях:

\noindent
$$  \alpha \sum\limits_{m=1}^M [ x_{mk}^0+  x_{m(k+E)}^0] \le 
d_k(1), \ \alpha \ge 0, \  \  k =\ol{1, E}\,.
$$

\smallskip

Из решения задачи~2 определяется $z_m^{(=)} \hm= \alpha^*$ для всех $p_m \hm\in P$ 
и,~соответственно, для всех VMF-век\-то\-ров $\ora{\mathbf{z}}^{(=)}(n) \hm= \alpha^*$, $n \hm= \ol{1, N}$:

\noindent
$$
\left\|\ora{\mathbf{z}}^{(=)}(n)\right\| = \alpha^*(N - 1)^{1/2}. $$

\vspace*{-14pt}

\section{Вычислительный эксперимент }

\vspace*{-2pt}

Вычислительный эксперимент проводился на моделях сетевых систем, представленных 
на рис.~1. В~каждой сети~69~узлов. В~ходе вычислительного эксперимента 
проводилась нормировка, и~суммарная пропускная способность в~обеих сетях была 
одинакова:
$\sum\nolimits_{k=1}^{E} d_k(1)\hm = D(0)\hm= 68\,256$.



\begin{figure*} %fig3
\vspace*{1pt}
      \begin{center}
     \mbox{%
\epsfxsize=163mm 
\epsfbox{mal-5.eps}
}
\end{center}
\vspace*{-9pt}
\Caption{Удельные затраты в~базовой~(\textit{а}) и~в~кольцевой~(\textit{б}) сетях }
\end{figure*}

На диаграммах, представленных на рис.~2, для  VMF-век\-то\-ров 
$\ora{\mathbf{z}}(n)$ в~базовой и~кольцевой сетях по горизонтальной оси 
откладываются суммарные значения мультипотока~$\zeta(n)$, а по вертикальной  
$\delta(n)$ --- со\-от\-вет\-ст\-ву\-ющие значения требуемой  суммарной пропускной 
способности при передаче  в~монопольном режиме. На рис.~2,\,\textit{а} группа точек, для 
которых $\zeta(n)\hm\le 1000$, относится к~висячим вершинам и~суммарный исходящий 
поток ограничен величиной пропускной способности инцидентного ребра. Для большой 
группы узлов суммарный исходящий межузловой поток находится в~диапазоне  $1\,300\hm\le \zeta(n)\hm\le 1\,500$, поскольку в~монопольном режиме используется 
уравнительная стратегия для\linebreak всех исходящих потоков.
В~кольцевой сети <<средняя>> длина кратчайшего пути меньше, чем в~базовой. В~результате, как следует из рис.~2,\,\textit{б}, 
\mbox{затраты} ресурсов~--- пропускной способности~--- 
значительно меньше практически для всех VMF-по\-то\-ков и~$\delta(n) \hm\le 10\,000$.

На рис.~3 представлены диаграммы  распределения удельных затрат $w(n) \hm= 
\delta(n)/\zeta(n)$ передачи единичного потока из узла~$v(n)$.
Значения $w(n)$ откладываются по вертикальной оси, а~относительные номера узлов 
$\nu(n)\hm = n/N$, $n = \ol{1, N}$,  указываются по горизонтальной оси. Удельные 
затраты в~кольцевой сети меньше, чем в~базовой, поскольку добавление ребер 
приводит к~уменьшению \textit{длины} кратчайших путей для большого чис\-ла пар-кор\-рес\-пон\-ден\-тов. Для висячих узлов, соединенных единственным реб\-ром с~сетью, 
удельные затраты выше,
чем для вершин, расположенных в~центре. Самая
правая 
точка на рис.~3,\,\textit{а} со значением $w^*(n)\hm< 6$ соответствует центральному узлу в~базовой сети. 

\vspace*{-9pt}

\section{Заключение }

\vspace*{-3pt}

Построенные априорные оценки VMF-век\-то\-ров позволяют анализировать ситуации, 
когда внешняя (входная) нагрузка на сеть превышает допустимые нормативные 
значения. Исходящие взаимно допустимые мультипотоки  $\ora{\mathbf{z}}(n)$ 
указывают на возможные ограничения, отказы в~обслуживании и/или возникновение 
очередей на входе в~сис\-те\-му. Оценив удельные затраты, полученные на основе VMF-
векторов, можно выделить группы уз\-лов-кор\-рес\-пон\-ден\-тов, требующих б$\acute{\mbox{о}}$льших 
ресурсов для передачи соответствующих потоков. Анализ структурных потоковых 
характеристик, их взаимосвязи и~взаимовлияния может быть использован на 
предпроектном этапе создания сетей различного назначения~[7--9].

%\columnbreak

\vspace*{-9pt}

{\small\frenchspacing
 { %\baselineskip=10.6pt
 %\addcontentsline{toc}{section}{References}
 \begin{thebibliography}{9}
 
\vspace*{-2pt}
 
 \bibitem{Kung05} %1
\Au{Kung~H.\,T., Wu~C.\,H.} Content networks: Taxonomy 
and new approaches~// The Internet as a~large-scale complex system.~--- Oxford: 
Oxford University Press,    2005. P.~203--225. doi: 10.1093/oso/9780195157208.003.0007.

\bibitem{Yang12} %2
\Au{Yang~R., van der Mei~R.\,D., Roubos~D., \textit{et al.}} 
Resource optimization in distributed real-time multimedia applications~// 
Multimed. Tools Appl., 2012. Vol.~59. Р.~941--971. doi: 10.1007/s11042-011-0782-5.

\bibitem{Beben13} %3
\Au{Beben~A., Batalla~J.\,M., Chai~W.\,K., 
Sliwinski~J.} Multi-criteria decision algorithms for efficient content delivery 
in content networks~// Ann. Telecommun., 2013. Vol.~68.  P.~153--165.
doi: 10.1007/s12243-012-0321-z.
    
\bibitem{Mal23-1}  %4
\Au{Малашенко~Ю.\,Е., Назарова~И.\,А.} Оценки 
распределения ресурсов в~многопользовательской сети при равных межузловых 
нагрузках~// Информатика и~её применения, 2023. Т.~17. Вып.~1. С.~83--88.
doi: 10.14357/19922264230111. EDN: BUKVGV.

\bibitem{Mal23-3}  %5
\Au{Малашенко~Ю.\,Е., Назарова~И.\,А.} Анализ загрузки  
многопользовательской сети при расщеплении потоков по кратчайшим маршрутам~// 
Информатика и~её применения, 2023. Т.~17. Вып.~3. С.~33--38.
doi: 10.14357/19922264230305. EDN: NLUSQJ.

%\pagebreak

\bibitem{Mal23-6} %6
\Au{Малашенко~Ю.\,Е., Назарова~И.\,А.} Анализ узловых  
мультипотоков в~многопользовательской сис\-те\-ме при уравнительных стратегиях  
управ\-ле\-ния~// Известия РАН. Теория и~сис\-те\-мы управ\-ле\-ния, 2023.  №\,6. С.~137--149.



\bibitem{Kach11}  %7
\Au{Качанов~С.\,А., Медведев~Н.\,В.} Алгоритм 
выравнивания загрузки узлов мобильной ин\-фор\-ма\-ци\-он\-но-ком\-му\-ни\-ка\-ци\-он\-ной сети~//  
Технологии гражданской безопас\-ности, 2011. Т.~8. №\,1. С.~26--29. EDN: NQUSAR.



\bibitem{Zhang14} %8
\Au{Zhang~H.\,P., Yin~B.\,Q., Lu~X.\,N.} Modeling and 
analysis for streaming service systems // Int. J. Automation 
Computing, 2014.  Vol.~11. P.~449--458.    doi: 10.1007/s11633-014-\mbox{0812-7}. 

\bibitem{Sim2016} %9
\Au{Симаков~Д.\,В.} Управление трафиком в~сети с~высокой динамикой метрик сетевых маршрутов~// Науковедение, 
2016. Т.~8. №\,1. Ст.~60TVN116.

    \end{thebibliography}

 }
 }

\end{multicols}

\vspace*{-10pt}

\hfill{\small\textit{Поступила в~редакцию 04.12.23}}

\vspace*{4pt}

%\pagebreak

%\newpage

%\vspace*{-28pt}

\hrule

\vspace*{2pt}

\hrule

\vspace*{-6pt}

\def\tit{ANALYSIS OF NODE  MULTIFLOWS IN~A~MULTIUSER NETWORK~SYSTEM\\[-7pt]}


\def\titkol{Analysis of node  multiflows in~a~multiuser network system}


\def\aut{Yu.\,E.~Malashenko and~I.\,A.~Nazarova}

\def\autkol{Yu.\,E.~Malashenko and~I.\,A.~Nazarova}

\titel{\tit}{\aut}{\autkol}{\titkol}

\vspace*{-14.9pt}


\noindent
Federal Research Center ``Computer Science and Control'' of the Russian Academy of 
Sciences, 44-2~Vavilov Str., Moscow 119333, Russian Federation

\def\leftfootline{\small{\textbf{\thepage}
\hfill INFORMATIKA I EE PRIMENENIYA~--- INFORMATICS AND
APPLICATIONS\ \ \ 2024\ \ \ volume~18\ \ \ issue\ 1}
}%
 \def\rightfootline{\small{INFORMATIKA I EE PRIMENENIYA~---
INFORMATICS AND APPLICATIONS\ \ \ 2024\ \ \ volume~18\ \ \ issue\ 1
\hfill \textbf{\thepage}}}

\vspace*{2pt}





\Abste{Within the framework of computational experiments, 
exclusive and limit modes of transmission of outgoing node multiflows in 
a~multiuser network system are studied. Internodal flows of different types 
 that are transmitted from all nodes along all shortest routes are 
analyzed. Within 
the framework of the algorithmic scheme,  the transmission of all internodal 
flows is calculated. The found values act as components of the multiflow vector 
originating from each source node to all destination nodes and are interpreted 
as guaranteed multicriteria estimates of the functioning of a multiuser system.  
To estimate the minimum unit cost for each node, the monopole mode of 
transmitting the outgoing multiflow is used without taking into account all the 
others. To search for the maximum achievable internodal flow, the limiting 
transmission mode is considered in which all edges of the network are 
completely loaded. The networks with different structural features and the same 
total capacity are studied.}


\KWE{multicommodity flow model; distribution of node 
multiflows;  unit cost of flow transmission}

\DOI{10.14357/19922264240106}{AKCMCQ}

%\vspace*{1pt}

%\Ack
%\vspace*{-4pt}
%\noindent



  \begin{multicols}{2}

\renewcommand{\bibname}{\protect\rmfamily References}
%\renewcommand{\bibname}{\large\protect\rm References}

{\small\frenchspacing
 {\baselineskip=10.5pt
 \addcontentsline{toc}{section}{References}
 \begin{thebibliography}{9} 
 
% \vspace*{-7pt}
 
 \bibitem{Kung05-a} %1
\Aue{Kung, H.\,T., and C.\,H.~Wu.} 2005. Content networks: Taxonomy and 
new approaches. \textit{The Internet as a~large-scale complex system}. Oxford: 
Oxford University Press. 203--225. doi: 10.1093/oso/9780195157208.003.0007.

%\vspace*{-2pt}

\bibitem{Yang12-a} %2
\Aue{Yang, R., R.\,D.~van der Mei, D.~Roubos,  \textit{et al.}} 2012. 
Resource optimization in distributed real-time multimedia applications. 
\textit{Multimed. Tools Appl.} 59(3):941--971. doi: 10.1007/s11042-011-0782-5.

\bibitem{Beben13-a} %3
\Aue{Beben, A., J.\,M.~Batalla, W.\,K.~Chai, and J.~Sliwinski.} 2013. 
Multi-criteria decision algorithms for efficient content delivery in content 
networks. \textit{Ann. Telecommun.} 68:153--165. doi: 10.1007/s12243-012-0321-z.

\bibitem{Mal23-1a} %4
\Aue{Malashenko, Yu.\,E., and I.\,A.~Nazarova.} 2023. Otsenki 
raspredeleniya resursov v~mnogopol'zovatel'skoy seti pri ravnykh 
mezhuzlovykh nagruzkakh [Estimates of the resource distribution in the multiuser 
network with equal internodal loads]. \textit{Informatika i~ee Primeneniya~--- 
Inform. Appl.} 17(1):83--88. doi: 10.14357/19922264230111. EDN: BUKVGV.

\bibitem{Mal23-3a} %5
\Aue{Malashenko, Yu.\,E., and I.\,A.~Nazarova.} 2023.  Analiz zagruzki  
mnogopol'zovatel'skoy seti pri rasshcheplenii potokov po kratchayshim 
marshrutam [Multiuser network load analysis by splitting flows along the 
shortest routes]. \textit{Informatika i~ee Primeneniya~--- Inform. Appl.} 
17(3):33--38. doi: 10.14357/19922264230305. EDN: NLUSQJ.


\bibitem{Mal23-6a} %6
\Aue{Malashenko, Yu.\,E., and I.\,A.~Nazarova.} 2023. Analysis of nodal 
multi-flows in a multiuser system with equalizing management strategies. 
\textit{J. Comput. Sys. Sc. Int.} 62(6):1022--1033.

\bibitem{Kach11-a} %7
\Aue{Kachanov, S.\,A., and N.\,V.~Medvedev.} 2011. Algoritm 
vyravnivaniya zagruzki uzlov mobil'noy informatsionno-kommunikatsionnoy seti 
[Algorithm of leveling loading of mobile information and communication centers 
net]. \textit{Tekhnologii grazhdanskoy bezopasnosti} [Civil Security 
Technologies] 8(1):26--29. EDN: NQUSAR.

\bibitem{Zhang14-a} %8
\Aue{Zhang, H.\,P., B.\,Q.~Yin, and X.\,N.~Lu.} 2014. Modeling and 
analysis for streaming service systems. \textit{Int. J. 
Automation Computing} 11:449--458. doi: 10.1007/s11633-014-0812-7.

\bibitem{Sim2016-a} %9
\Aue{Simakov, D.\,V.} 2016. Upravlenie trafikom v~seti s~vysokoy dinamikoy metrik setevykh marshrutov [Traffic engineering for networks 
with high dynamics of routing metrics]. \textit{Naukovedenie} [Science Studies] 8(1):60TVN116.


 \end{thebibliography}

 }
 }

\end{multicols}

\vspace*{-8pt}

\hfill{\small\textit{Received December 4, 2023}} 

\vspace*{-22pt}
 
      \Contr

\vspace*{-4pt}

\noindent
\textbf{Malashenko Yuri E.} (b.\ 1946)~--- Doctor of Science in physics and 
mathematics, principal scientist, Federal Research Center ``Computer Science 
and Control'' of the Russian Academy of Sciences, 44-2~Vavilov Str., Moscow 
119333, Russian Federation; \mbox{malash09@ccas.ru }

%\vspace*{3pt}

\noindent
\textbf{Nazarova Irina A.} (b.\ 1966)~--- Candidate of Science (PhD) in physics 
and mathematics, scientist, Federal Research Center ``Computer Science and 
Control'' of the Russian Academy of Sciences, 44-2 Vavilov Str., Moscow 119333, 
Russian Federation; \mbox{irina-nazar@yandex.ru}




        
\label{end\stat}

\renewcommand{\bibname}{\protect\rm Литература} 