\def\stat{bosov}

\def\tit{УПРАВЛЕНИЕ ЛИНЕЙНЫМ ВЫХОДОМ АВТОНОМНОЙ ДИФФЕРЕНЦИАЛЬНОЙ 
СИСТЕМЫ ПО~КВАДРАТИЧНОМУ КРИТЕРИЮ НА~БЕСКОНЕЧНОМ 
ГОРИЗОНТЕ$^*$}

\def\titkol{Управление линейным выходом автономной дифференциальной 
системы по квадратичному критерию} % на бесконечном  горизонте}

\def\aut{А.\,В.~Босов$^1$}

\def\autkol{А.\,В.~Босов}

\titel{\tit}{\aut}{\autkol}{\titkol}

\index{Босов А.\,В.}
\index{Bosov A.\,V.}


{\renewcommand{\thefootnote}{\fnsymbol{footnote}} \footnotetext[1]
{Работа выполнялась с использованием инфраструктуры Центра коллективного пользования 
<<Высокопроизводительные вы\-чис\-ле\-ния и~большие данные>> (ЦКП <<Информатика>>) ФИЦ ИУ 
РАН (г.\ Москва).}}


\renewcommand{\thefootnote}{\arabic{footnote}}
\footnotetext[1]{Федеральный исследовательский центр <<Информатика и управление>> Российской академии наук, 
\mbox{ABosov@frccsc.ru}}

\vspace*{-6pt}


  

\Abst{Решена задача оптимального управления линейным выходом 
стохастической дифференциальной сис\-те\-мы на бесконечном горизонте. Решение 
получено как предельная форма оптимального управ\-ле\-ния в~со\-от\-вет\-ст\-ву\-ющей 
задаче с~конечным горизонтом. Приведены достаточные условия существования 
управления, состоящие из требований стационарности нелинейной динамики, 
конечности квад\-ра\-тич\-но\-го целевого функционала, стабилизируемости линейного 
выхода и~существования предела в~формуле Фейн\-ма\-на--Ка\-ца, опи\-сы\-ва\-ющей 
нелинейную часть управ\-ле\-ния. Условия для линейной час\-ти управления связаны 
с~классическими результатами существования решения автономного урав\-не\-ния 
Риккати. Существование предела в формуле Фейн\-ма\-на--Ка\-ца~--- с~решением 
параболического уравнения, за\-да\-юще\-го коэффициенты для нелинейной части 
управления. Рас\-смот\-рен част\-ный случай линейного сноса, при котором 
сохраняется нелинейный характер задачи, но оптимальное управ\-ле\-ние 
оказывается линейным и по выходу, и~по переменной со\-сто\-яния. Приведены 
результаты численного эксперимента, поз\-во\-ля\-юще\-го проанализировать 
переходный процесс в~задаче с~конечным горизонтом и~эргодическим процессом 
в~динамике. Для коэффициентов управ\-ле\-ния проиллюстрирован предельный 
переход к~оптимальным значениям соответствующего оптимального автономного 
управ\-ле\-ния.}

\KW{стохастическая дифференциальная система Ито; управление по выходу; 
оптимальное управ\-ле\-ние; квад\-ра\-тич\-ный критерий; параболическое уравнение; 
формула Фейн\-ма\-на--Каца}

\DOI{10.14357/19922264240103}{UEESFO}
  
%\vspace*{-6pt}


\vskip 10pt plus 9pt minus 6pt

\thispagestyle{headings}

\begin{multicols}{2}

\label{st\stat}

\section{Введение}

     Типовым вариантом классической задачи управ\-ле\-ния~[1] линейной 
дифференциальной гауссовской системой по квад\-ра\-тич\-но\-му критерию 
качества (LQG, linear-quadratic-Gaussian) стала \mbox{постановка} с~бесконечным горизонтом  
управ\-ле\-ния~[2, 3]. Естественно, что на сис\-те\-му управ\-ле\-ния в этом 
случае накладываются жесткие ограничения, обес\-пе\-чи\-ва\-ющие ее 
существование бесконечное время. Прежде всего это автономность, т.\,е.\ 
независимость от времени всех функций в~модели динамики и~в~целевом 
функционале, устой\-чи\-вость и~ста\-би\-ли\-зи\-ру\-емость. Последние свойства 
обеспечивают принципиальную воз\-мож\-ность управ\-ле\-ния и~ко\-неч\-ность 
квадратичного критерия. Кроме того, вполне <<понятным>> элементом 
постановки варианта LQG с бесконечным горизонтом становится поиск 
оптимального автономного управ\-ле\-ния в~виде линейной функции выхода. 
Линейный класс допустимых управ\-ле\-ний сводит исследование задачи 
управления к~изуче\-нию свойств уравнения Риккати, так что решение задачи 
управления существует, если имеет решение предельное уравнения Риккати. 
Постановка задачи управ\-ле\-ния, ис\-поль\-зу\-емая в~данной статье, похожа на 
модель LQG, т.\,е.\ используется квад\-ра\-тич\-ный целевой функционал 
и~линейный выход, дающий в~оптимальном решении типовое, т.\,е.\ 
линейное по выходу, сла\-га\-емое. Принципиальное отличие заключается 
в~нелинейной не\-управ\-ля\-емой динамике, из-за которой в оптимальном 
управ\-ле\-нии в~варианте с~конечным горизонтом появляется нелинейная 
часть, для вы\-чис\-ле\-ния которой нужно решать параболическое 
дифференциальное урав\-не\-ние в~част\-ных производных. 

Подробно эта задача 
исследована в~[4], а~цель данной статьи~--- пред\-ста\-вить решение 
аналогичной задачи в~постановке с~бесконечным горизонтом. Эта 
постановка сформулирована в~сле\-ду\-ющем разделе статьи, а~в~разд.~3 
сформулирован основной результат. Ровно так же, как управ\-ле\-ние~[4], 
<<похожее>> на LQG, все-та\-ки оказывается нелинейным, полученное 
оптимальное автономное управ\-ле\-ние также остается нелинейным, но 
вместо параболического урав\-не\-ния описывается теперь обыкновенным 
дифференциальным уравнением, хотя и~нелинейным. Чтобы посмотреть, как 
на практике применится автономное управ\-ле\-ние, в~разд.~4 рассмотрен 
част\-ный случай линейного сноса, когда управ\-ле\-ние оказывается линейным, 
несмотря на оста\-ющу\-юся нелинейной задачу. Этот вариант интересен тем, 
что поз\-во\-ля\-ет еще и посмотреть, как чис\-лен\-ные решения параболического 
уравнения традиционным методом сеток~[5] и~методом имитационного 
моделирования формулы Фейн\-ма\-на--Ка\-ца~[6], которые <<не знают>> 
о~наличии точного линейного решения, ведут себя в~связке при переходе 
обычного решения в~автономный режим.

\vspace*{-3pt}

\section{Постановка задачи}

\vspace*{-3pt}

     На каноническом вероятностном пространстве $(\Omega, 
\mathcal{F},\mathcal{P},\mathcal{F}_t)$, $t\hm\in [0,\infty)$, рас\-смот\-рим 
автономную сто\-ха\-сти\-че\-скую динамическую сис\-те\-му, со\-сто\-яние которой 
пред\-став\-ля\-ет диффузионный процесс $y_t\hm\in \mathbb{R}^{n_y}$, 
опи\-сы\-ва\-емый сис\-те\-мой нелинейных сто\-ха\-сти\-че\-ских дифференциальных 
уравнений Ито:
     \begin{equation}
     dy_t= A(y_t)\,dt+\Sigma(y_t)\,dv_t\,,\enskip y_0=Y,\enskip t\in [0,\infty),
     \label{e1-bos}
     \end{equation}
   где $v_t\in \mathbb{R}^{n_v}$~--- стандартный векторный винеровский 
процесс; $Y\hm\in \mathbb{R}^{n_y}$~--- случайная величина с~конечным 
вторым моментом; векторная функция $A\hm= A(y): \mathbb{R}^{n_y}\hm\to 
\mathbb{R}^{n_y}$ и~мат\-рич\-ная функция $\Sigma\hm= \Sigma(y): 
\mathbb{R}^{n_y}\hm\to \mathbb{R}^{n_y\times n_v}$ удовле\-тво\-ря\-ют 
условиям Ито:
\begin{align*}
\vert A(y)\vert +\vert \Sigma(y)\vert &\leq C(1+\vert y\vert);\\
\vert A(y_1)- A(y_2)\vert +\vert \Sigma(y_1) -\Sigma(y_2)\vert&\leq C\vert y_1-
y_2\vert,\\
& \hspace*{5mm}y_1,y_2\in\mathbb{R}^{n_y}\,,
\end{align*}
обеспечивающим существование единственного потраекторного решения 
уравнения~(1) на любом конечном интервале $t\hm\in [0,T]$~\cite{7-bos} 
(здесь и~далее $\vert\cdot\vert$ обозначает евклидову норму вектора или 
мат\-ри\-цы). Чтобы рассматривать решения~(1) на интервале $[0,\infty)$, 
дополнительно по\-тре\-бу\-ем от процесса~$y_t$ стационарности в широком и 
узком смысле~[8]. Это предположение избыточно, пока рассматривается 
задача с~полной информацией, т.\,е.\ состояние~$y_t$ предполагается 
известным. Но в~дальнейшем при переходе к~постановке задачи 
с~косвенными наблюдениями за~$y_t$ ста\-ци\-о\-нар\-ность примет вполне 
содержательный смысл.

     Состоянием $y_t$ формируется управ\-ля\-емый выход, опи\-сы\-ва\-емый 
процессом $z_t\hm\in\mathbb{R}^{n_z}$, линейно связанным с~со\-сто\-я\-нием:

\vspace*{-3pt}

\noindent
     \begin{multline}
     dz_t=ay_t\, dt +bz_t \,dt+c u_t\, dt+\sigma\, dw_t,\\
     z_0=Z,\ t\in [0,\infty),
     \label{e2-bos}
     \end{multline}
     
     \vspace*{-3pt}

\noindent
где $a\in \mathbb{R}^{n_z\times n_y}$, $b\hm\in \mathbb{R}^{n_z\times n_z}$, 
$c\hm\in \mathbb{R}^{n_z\times n_u}$ и~$\sigma\hm\in \mathbb{R}^{n_z\times n_w}$~--- известные матрицы; $w_t\hm\in \mathbb{R}^{n_w}$~--- не 
зависящий от $v_t$, $Y$ и~$Z$ стандартный векторный винеровский процесс; 
$Z\hm\in \mathbb{R}^{n_z}$~---случайная величина с~конечным вторым 
моментом, не зависящая от $v_t$ и~$Y$; $u_t\hm\in \mathbb{R}^{n_u}$~--- 
управление, целью которого ставится стабилизация выхода~$z_t$ около 
некоторой траектории, формируемой со\-сто\-яни\-ем~$y_t$. Цель управ\-ле\-ния 
формализуется ниже квад\-ра\-тич\-ным критерием качества общего вида.

     Задача управления формулируется в предположении наличия полной 
информации о~со\-сто\-янии~$y_t$ и~выходе~$z_t$ (со\-от\-вет\-ст\-ву\-ющая  
$\sigma$-ал\-геб\-ра обозначается $\mathcal{F}_t^{y,z}$ и~выполнено 
$\mathcal{F}_t^{y,z}\hm\subseteq \mathcal{F}_t\hm\subseteq \mathcal{F}$), 
т.\,е.\ управ\-ле\-ние~$u_t$ предполагается  
$\mathcal{F}_t^{y,z}$-из\-ме\-ри\-мым. Класс~$U_0^\infty$ допустимых 
управ\-ле\-ний составляют автономные (не зависящие прямо от времени) 
управ\-ле\-ния с~пол\-ной обратной связью, т.\,е.\ функции вида\linebreak $u_t\hm= 
u(y,z)\hm\in\mathbb{R}^{n_u}$, $y\hm\in \mathbb{R}^{n_y}$ и~$z\hm\in 
\mathbb{R}^{n_z}$, в~предположении, что со\-от\-вет\-ст\-ву\-ющая реализация 
$u_t\hm= u(y_t,z_t)$ обеспечивает выполнение условий существования~$y_t$ 
и~$z_t$ для $u_t\hm\in U_0^\infty$. Поскольку \mbox{со\-сто\-яние}~$y_t$  не зависит
от  управ\-ле\-ния~$u_t$, а~выход~$z_t$ описывается линейным 
автономным уравнением с~винеровским процессом, то данное формальное 
требование ограничивает допустимые управ\-ле\-ния процессами второго 
порядка, что обеспечивает существование решения~(2) на любом конечном 
интервале $t\hm\in [0,T]$. Для управ\-ле\-ния на интервале $[0,\infty)$ 
дополнительно потребуются типовые условия ста\-би\-ли\-зи\-ру\-емости~[2], 
обсуж\-да\-емые далее.
     
     Качество управления~$U_0^\infty$ определяется целевым 
функционалом сле\-ду\-юще\-го вида:
     \begin{multline}
     \!\! J(U_0^\infty) =\lim\limits_{T\to\infty} J\left(U_0^{{T}}\right),\\
      J\left(U_0^{{T}}\right) =\mathbb{E}\left\{ \fr{1}{T}\int\limits_0^{{T}} \left\| Py_t+Qz_t+Ru_t\right\|_S^2 
dt\right\},
     \label{e3-bos}
     \end{multline}
где $P\in \mathbb{R}^{n_J\times n_y}$, $Q\hm\in \mathbb{R}^{n_J\times n_z}$, $R\hm\in \mathbb{R}^{n_J\times k_u}$ 
и~$S\hm\in \mathbb{R}^{n_J\times n_J}$ ($S\hm\geq 0$, $S\hm= S^\prime$)~--- заданные 
матрицы, весовая функция $\| x\|_S^2\hm= x^\prime Sx$, единичной матрице 
$S\hm= E$ соответствует евклидова норма $\| x\|_E^2\hm= \vert x\vert^2$, 
<<${}^\prime$>>~--- операция транспонирования; $\mathbb{E}\{\cdot\}$~--- 
оператор математического ожидания (далее еще используется обозначение 
$\mathbb{E}\{ \cdot\vert\mathcal{F}\}$ для условного математического 
ожидания относительно $\sigma$-ал\-геб\-ры~$\mathcal{F}$). Кроме того, 
предполагается выполненным обычное условие не\-вы\-рож\-ден\-ности, в данных 
обозначениях принимающее вид $R^\prime SR\hm>0$.
     
     Задача состоит, таким образом, в поиске 
     $(U^*)_0^\infty \hm= \{ u^*(y,z), y\hm\in \mathbb{R}^{n_y}, z\hm\in 
\mathbb{R}^{n_z}\}$, допустимого управления с обратной связью, 
реализации $u_t^*\hm= u^*(y_t, z_t^*)$, $t\hm\in [0,\infty)$, которого 
доставляют минимум квад\-ра\-тич\-но\-му функционалу $J(U_0^\infty)$: 
     \begin{equation}
     (U^*)_0^\infty =\argmin\limits_{u_t\in U_0^\infty} J(U_0^\infty).
     \label{e4-bos}
     \end{equation}
     
     Далее через~$z_t^*$ обозначается решение~(2), от\-ве\-ча\-ющее~$u_t^*$, 
и~учитывается, что $y_t$ от~$u_t$ не зависит.
     
     Следует обратить внимание на некоторые отличия данной постановки 
от классической задачи LQG с бесконечным горизонтом (хорошее описание 
которой дано в~[2]). В~классическом варианте вопрос с~определением класса 
допустимых управ\-ле\-ний однозначно решается его описанием линейными 
функциями со\-сто\-яния и~выхода, а~именно: если обозначить через $u_t^\# 
\hm= u_t^\# (y_t, z_t^\#, T)$ решение (неавтономное, поэтому $u_t^\#(y,z,T)$ 
с~нижним индексом~$t$) LQG-за\-да\-чи с конечным горизонтом $t\hm\in 
[0,T]$, оптимальное на классе нелинейных до\-пус\-ти\-мых управ\-ле\-ний, то оно 
получится линейным: 
$$
u_t^\# = L_t^{y\#} y_t + L_t^{z\#} z_t^\# +  l_t^\#.
$$ 
%
Здесь, как и аналогично выше, через~$z_t^\#$  обозначено 
решение~(2), от\-ве\-ча\-ющее~$u_t^\#$. Соответственно, в описании класса 
допустимых управ\-ле\-ний в~задаче с~бесконечным горизонтом будут 
фигурировать только линейные управ\-ления 
$$
u_t= u\left(y_t,z_t\right)= L^y  y_t+ L^z z_t+l\,,
$$
 что естественным образом вытекает из оп\-ти\-маль\-ности 
в задаче с конечным горизонтом именно линейного управ\-ле\-ния. 
В~рас\-смат\-ри\-ва\-емой задаче этого нет, поскольку и~со\-сто\-яние~(1), 
и,~главное, решение задачи с~конечным горизонтом, полученное в~[4], 
нелинейные. Так что, хотя искомое решение и~находится в~виде предельной 
формы оптимального управ\-ле\-ния с~конечным горизонтом, оно не будет 
относиться к~классу линейных.
     
\section{Основной результат}

     Очевидно, что основу для решения задачи~(\ref{e4-bos}) обеспечивает 
решение со\-от\-вет\-ст\-ву\-ющей задачи с~конечным горизонтом, которое получено 
в~[4]. Для корректного предельного перехода от постановки с~конечным 
горизонтом к~бесконечному времени требуется выполнение ряда условий, 
объединенных в~сле\-ду\-ющем утверж\-де\-нии.
     
     \smallskip
     
     \noindent
     \textbf{Теорема.} \textit{Решение задачи}~(\ref{e4-bos}) \textit{может 
быть записано в~виде}

\vspace*{-4pt}

\noindent
     \begin{multline}
     u^*(y,z)=-\fr{1}{2}\left( R^\prime SR\right)^{-1} \left( c^\prime(2\alpha_* z+\beta_*(y))+{}\right.\\
    \left. {}+2R^\prime S(Py+Qz)\right),
     \label{e5-bos}
     \end{multline}
     
     \vspace*{-4pt}
     
     \noindent
\textit{где симметричная неотрицательно определенная мат\-ри\-ца 
$\alpha_*\hm\in \mathbb{R}^{n_z\times n_z}$ и~век\-тор-функ\-ция 
$\beta_*\hm=\beta_*(y)\hm= (\beta_*^{(1)}(y),\ldots , \beta_*^{(n_z)}(y))^\prime\hm\in 
\mathbb{R}^{n_z}$ пред\-став\-ля\-ют собой решения уравнений}

\noindent
\begin{multline}
\left( b^\prime -Q^\prime SR (R^\prime SR)^{-1} c^\prime\right) \alpha_* 
+{}\\[2pt]
{}+\alpha_* \left(b-c(R^\prime SR)^{-1} R^\prime SQ\right)+{}\\[2pt]
{}+
Q^\prime \left(S-SR(R^\prime SR)^{-1} R^\prime S\right)Q -{}\\[2pt]
{}- \alpha_* c\left(R^\prime SR\right)^{-1} 
c^\prime \alpha_*=0\,;
\label{e6-bos}
\end{multline}

\vspace*{-12pt}

\noindent
\begin{multline}
\fr{1}{2}\mathrm{tr}\left\{ \Sigma^{\prime} \fr{\partial^2\beta_*^{(i)}}{\partial y^2}\,\Sigma\right\} +A^\prime \fr{\partial\beta_*^{(i)}}{\partial y} +{}\\
{}+\displaystyle\sum\limits_{j=1}^{n_y} y^{(i)} [M_*]^{(ji)} +\sum\limits_{j=1}^{n_z} 
\beta_*^{(j)} [N_*]^{(ji)} =0\,,
\label{e7-bos}
\end{multline}
где
\begin{align*}
M_*&=2\left( \left( a^\prime -P^\prime SR(R^\prime SR)^{-1} c^\prime 
\right)\alpha_*+{}\right.\\[2pt]
&\hspace*{20mm}\left.{}+ P^\prime (S-SR(R^\prime SR)^{-1} R^\prime S)Q\right);\\[2pt]
N_*&= b-c(R^\prime SR)^{-1}R^\prime SQ -c(R^\prime SR)^{-1} c^\prime \alpha_*,\\[2pt]
&\hspace*{50mm}i=\overline{1, n_z}\,,
\end{align*}
\textit{если для параметров системы}~(1), (2) \textit{и целевого 
функционала}~(3) \textit{выполнены сле\-ду\-ющие условия}:
\begin{enumerate}[(1)]
\item \textit{матрица $b$ устойчива};
\item \textit{матрицы $(K_b, c)$ стабилизируемы}, $K_b\hm= b\hm- 
c(R^\prime SR)^{-1} R^\prime SQ$;
\item \textit{матрицы ($K_b^\prime, K_Q$) стабилизируемы}, $K_Q\hm= Q^\prime S^{1/2}(E\hm- S^{1/2}R(R^\prime SR)^{-1} R^\prime S^{1/2})$;
\item \textit{для любого $y\hm\in \mathbb{R}^{n_y}$ существует и не 
зависит от~$t$ конечный предел}
\begin{equation}
\lim\limits_{T\to\infty} \mathbb{E}\left\{ I^{-1}(t)\int\limits_t^{{T}} I^{-1}(\tau)M^\prime_* y_\tau\,d\tau \vert \mathcal{F}_t^y \right\},
\label{e8-bos}
\end{equation}
\textit{где $I^{-1}(\tau)\hm=\exp \{ N_*^\prime \tau\}$; $y_\tau$~--- 
решение уравнения}~(1) \textit{с~переменной времени $\tau\hm\in 
[t,\infty)$ и~начальным условием} $y_t\hm= y$.
\end{enumerate}

     \textit{Уравнение}~(\ref{e7-bos}) \textit{записано с~использованием 
обозначений $y\hm= (y^{(1)}, \ldots , y^{(n_y)})^\prime$ для элементов 
вектора~$y$, $[A]^{(ji)}$~--- для элемента $j$-й строки $i$-го столб\-ца 
мат\-ри\-цы~$A$, $\mathrm{tr}\{A\}$~--- для следа мат\-ри\-цы~$A$.}
     
     \smallskip
     
     \noindent
     Д\,о\,к\,а\,з\,а\,т\,е\,л\,ь\,с\,т\,в\,о\,.\ \ Перечисленные в теореме 
условия обеспечивают существование предельного решения 
соответствующей~(4) задачи с конечным горизонтом, т.\,е.\ управ\-ле\-ния 
     $(U^\#)_0^{{T}}\hm= \min_{u_t\in U_0^{{T}}} J(U_0^{{T}})$, где $U_0^{{T}}\hm= \{ 
u_t(y,z,T)$,\linebreak $y\hm\in \mathbb{R}^{n_y},\ z\hm\in \mathbb{R}^{n_z},\ t\hm\in 
[0,T]\}$. Оптимальное управ\-ле\-ние $u_t^\#\hm= u_t^\# (y_t, z_t^\#, T)$ 
получено в~[4] в~виде:
     \begin{multline}
     u_t^\#= u_t^\# (y,z,T) =-\fr{1}{2}(R^\prime SR)^{-1} \times{}\\
     {}\times \left( c^\prime 
(2\alpha_t z+\beta_t)+2R^\prime S(Py+Qz)\right),
     \label{e9-bos}
     \end{multline}
где матричный коэффициент $\alpha_t\hm= \alpha_t(T)\hm\in 
\mathbb{R}^{n_z\times n_z}$ представляет собой решение задачи Коши для 
уравнения Риккати

\vspace*{-3pt}

\noindent
\begin{multline}
\fr{d\alpha_t}{dt} +\left( b^\prime -Q^\prime SR(R^\prime SR)^{-1} c^\prime 
\right)\alpha_t +{}\\[-3pt]
{}+\alpha_t \left (b-c(R^\prime SR)^{-1} R^\prime SQ\right)+{}\\
{}+
Q^\prime \left(S-SR(R^\prime SR)^{-1} R^\prime S\right) Q - {}\\
{}- \alpha_t c 
(R^\prime SR)^{-1}c^\prime \alpha_t=0\,,\enskip \alpha_T=Q^\prime SQ\,,
\label{e10-bos}
\end{multline} 

%\vspace*{-3pt}

\noindent
а векторный коэффициент $\beta_t\hm= \beta_t(y,T)\hm= 
(\beta_t^{(1)}(y,T),\ldots , \beta_t^{(n_z)}(y,T))^\prime \hm\in 
\mathbb{R}^{n_z}$~--- задачи Коши для сис\-те\-мы дифференциальных 
уравнений в~част\-ных производных 

\vspace*{-3pt}

\noindent
\begin{multline}
\fr{\partial \beta_t^{(i)}}{\partial t}+\fr{1}{2}\mathrm{tr}\left\{ \Sigma^{\prime} 
\fr{\partial^2\beta_t^{(i)}}{\partial y^2}\, \Sigma\right\}+A^\prime 
\fr{\partial\beta_t^{(i)}}{\partial y} +{}\\
{}+\displaystyle \sum\limits_{j=1}^{n_y} 
y^{(j)} [M_t]^{(ji)} +\sum\limits^{n_z}_{j=1} \beta_t^{(j)} 
[N_t]^{(ji)}=0\,,
\label{e11-bos}
\end{multline}

\vspace*{-3pt}

\noindent
где
\begin{align*}
M_t&=M_t(T)=2\left((a^\prime - P^\prime SR(R^\prime SR)^{-1} c^\prime)\alpha_t+{}\right.\\
&\hspace*{20mm}\left.{}+P^\prime (S-SR(R^\prime SR)^{-1} R^\prime S)Q\right);\\
N_t&=N_t(T) =b-c(R^\prime SR)^{-1} R^\prime SQ -{}\\[3pt]
&\hspace*{40mm}{}-c(R^\prime SR)^{-1} c^\prime \alpha_t\,;\\
\beta_T^{(i)} &=\displaystyle 2\sum\limits_{j=1}^{n_y} y^{(j)} [Q^\prime SP]^{(ji)},\enskip i=\overline{1, n_z}\,.
\end{align*}
     
     Перечисленные в теореме условия вместе с~исходным предположением 
о~ста\-ци\-о\-нар\-ности~$y_t$ обеспечивают, во-пер\-вых, ко\-неч\-ность целевого 
функционала~(3). Во-вто\-рых, из~(\ref{e10-bos}) для коэффициента 
оптимального управ\-ле\-ния~$\alpha_*$ получается уравнение~(\ref{e6-bos}) 
как предел $\alpha_*\hm= \lim_{T\to\infty} \alpha_t(T)$. Предельное решение 
мат\-рич\-но\-го уравнения Риккати~(\ref{e10-bos}) существует, не зависит от 
граничного условия и~является неотрицательно определенным, поскольку 
сформулированные в~тео\-ре\-ме условия ста\-би\-ли\-зи\-ру\-емости приводят 
коэффициенты уравнения Риккати к~формулировкам условий 
тео\-ре\-мы~12.2~[2]. Действительно, ста\-би\-ли\-зи\-ру\-емость пары мат\-риц $(K_b, 
c)$ ~--- это условие ста\-би\-ли\-зи\-ру\-емости наблюдений, ста\-би\-ли\-зи\-ру\-емость 
пары $(K_b^\prime, K_Q)$ ~--- это условие де\-тек\-ти\-ру\-емости. Последнее 
основано на том факте, что мат\-ри\-ца~$K_Q$ обеспечивает пред\-став\-ление 
$$
Q^\prime \left(S- SR\left(R^\prime SR\right)^{-1} R^\prime S\right)Q= K_Q K_Q^\prime,
$$ 
так как

\vspace*{-3pt}

\noindent
     \begin{multline*}
     \left( E-S^{1/2} R(R^\prime SR)^{-1} R^\prime S^{1/2}\right)\times{}\\
     {}\times \left( E-S^{1/2} R(R^\prime SR)^{-1} R^\prime S^{1/2}\right)^\prime={}\hspace*{10mm}
     \end{multline*}
     
     
     \noindent
     \begin{multline*}
     {}=
     \left( E-S^{1/2} R(R^\prime SR)^{-1} R^\prime S^{1/2}\right)^2 ={}\\
     {}=E- S^{1/2} R(R^\prime SR)^{-1} R^\prime S^{1/2}.
     \end{multline*}
     %
     
     \vspace*{-3pt}
     
     \noindent
     Кроме того, эта же теорема гарантирует устой\-чи\-вость мат\-ри\-цы~$N_*$, 
тре\-бу\-емую для выполнения~(\ref{e8-bos}).
     %
     И~последним из~(\ref{e11-bos}) получается коэффициент~$\beta_*$, 
для которого предельный переход\linebreak $\beta_*(y)\hm= \lim_{T\to\infty} 
\beta_t(y,T)$ обеспечивается условием~4 тео\-ре\-мы. Действительно, 
поскольку решение~(\ref{e11-bos}) может быть представлено с~по\-мощью 
формулы Фейн\-ма\-на--Ка\-ца~[9],

\vspace*{-3pt}
     
     \noindent
     \begin{multline}
     \beta_t(y,T) =\mathbb{E}\left\{ 
     \vphantom{\int\limits_y^{{T}}}
     2I^{-1}(t) I^{-1}(T) Q^\prime SPy(T)+{}\right.\\
\left.     {}+ I^{-1}(t)\int\limits_y^{{T}} I^{-1}(\tau) M^\prime(\tau) y(\tau)d\tau\vert 
\mathcal{F}_t^y\right\},\\
     I^{-1}(t)=\exp \{ N_t^\prime t\}.
     \label{e12-bos}
     \end{multline}
     
     \vspace*{-3pt}
     
     
     Поскольку матрица $N_*\hm= \lim_{T\to\infty} N_t(T)$ (предел 
существует и не зависит от~$t$, так как $N_t(T)$ выражается линейно через 
$\alpha_t(T)$) устойчива, первое сла\-га\-емое в~(\ref{e12-bos}) обращается 
в~ноль, что вместе с~условием~4 тео\-ре\-мы обеспечивает существование 
предела при $T\hm\to \infty$ в~(\ref{e11-bos}), не зависящего от~$t$ 
и~граничного условия $\beta_T^{(i)}(y,T)$, что и дает  
уравнения~(\ref{e7-bos}), завершая доказательство.
     
     \smallskip
     
     Отметим, что использованный результат в~[4] включает еще 
соотношения, опи\-сы\-ва\-ющие решение задачи с~конечным горизонтом 
полностью, т.\,е.\ опре\-де\-ля\-ющие функцию Беллмана. Эти соотношения 
можно трансформировать и~для рассматриваемой задачи~(\ref{e4-bos}), но 
ничего практически содержательного они не дадут, поэтому не 
используются.
     
     Принципиально важно в полученном решении то, что оптимальное 
управление~(\ref{e5-bos}) остается нелинейным, как и~в~допредельной 
по\-ста\-нов\-ке, что и~является, как уже упоминалось, главным отличием 
классической задачи LQG от рас\-смат\-ри\-ва\-емой. Другой вопрос, что 
к~условиям существования решения ничего конструктивного, кроме 
традиционных требований к~коэффициентам уравнения Риккати, добавить не 
удается. Это ожидаемо, так как даже конструктивные условия существования 
решения сис\-те\-мы параболических уравнений~(\ref{e11-bos}) представляют 
проб\-ле\-му, тем более затруднительно изучение свойств самих решений. 
С~другой стороны, аналитические исследования можно впол\-не успеш\-но 
заменить практическими расчетами. В~работах~[5, 6] предложены два 
эффективных метода численного решения~(\ref{e11-bos}), а~при наличии 
готовых расчетов анализ схо\-ди\-мости $\lim_{T\to\infty} \beta_t(y,T)$ тру-\linebreak\vspace*{-12pt}

\pagebreak

\noindent
  да не 
составляет. Другой вопрос, что реальные возможности реализации 
нелинейного управ\-ле\-ния для общей модели со\-сто\-яния~(1) пред\-став\-ля\-ют\-ся 
весьма ограниченными. Более практически интересным будет част\-ный 
случай линейного сноса или сводящийся к~нему случай косвенных 
наблюдений за со\-сто\-яни\-ем дискретной цепи Маркова, которому планируется 
по\-свя\-тить будущую работу.
     
\section{Частный случай линейного сноса}

     Итак, практическая реализация управ\-ле\-ния $(U^*)_0^\infty$ сведена 
к~решению алгебраического уравнения~(\ref{e6-bos}) и~сис\-те\-мы 
обыкновенных дифференциальных уравнений~(\ref{e7-bos}) вместо 
уравнения Риккати~(\ref{e10-bos}) и~сис\-те\-мы параболических  
уравнений~(\ref{e11-bos}). При этом если упро\-ще\-ние в~час\-ти отказа от 
решения уравнения Риккати не представляется сколь-ли\-бо значимым, так 
как с~эффективными численными методами для него нет проб\-лем, то замена 
параболических уравнений на обыкновенные~--- это действительно большое 
упрощение. Тем не менее работа с системой~(\ref{e7-bos}) все еще остается 
довольно слож\-ной, поскольку решать требуется хоть и~обыкновенные 
дифференциальные уравнения второго порядка, но для сетки, по\-кры\-ва\-ющей 
об\-ласть воз\-мож\-ных значений вектора $y\hm\in \mathbb{R}^{n_y}$, что даже 
для значений $n_y\hm= 3, 4$ (это минимальные раз\-мер\-но\-сти для моделей, 
име\-ющих практический смысл) становится уже крайне ресурсоемким. 
Вместе с~тем заменять уравнение~(1) линейным неинтересно, так как тогда 
задача сведется к~классической LQG. Интерес, таким образом, пред\-став\-ля\-ет 
случай линейного сноса, т.\,е.\ линейной функции $A(y)$, а~не\-ли\-ней\-ность 
задачи останется за счет диффузии~$\Sigma(y)$.
      
     Случай неавтономной сис\-те\-мы с~линейным сносом $A_t(y)\hm= A_t^a 
y\hm+ A_t^b$, где мат\-ри\-ца $A_t^a\hm\in \mathbb{R}^{n_y\times n_y}$ и~вектор $A_t^b\hm\in \mathbb{R}^{n_y}$ не зависят от~$y$, рас\-смот\-рен в~[4]. 
В~этом случае $\beta_t(y,T)\hm= \beta_t^a(T)y \hm+ \beta_t^b(T)$ и~для 
коэффициентов получены уравнения:
     \begin{equation}
     \left.
     \begin{array}{c}
     \fr{d\beta_t^a}{dt} +\beta_t^a A^a +M^\prime_* +N^\prime_* \beta_t^a 
=0\,;\\[3pt]
     \fr{d\beta_t^b}{dt} +\beta_t^a A^b +N^\prime_* \beta_t^b=0\,;\\[3pt]
     \beta_T^a(T) =2Q^\prime SP\,;\enskip \beta_T^b(T)=0\,.
     \end{array}
     \right\}
     \label{e13-bos}
     \end{equation}
     
     В условиях сформулированной тео\-ре\-мы существуют пределы 
$\beta_*^a\hm= \lim_{T\to\infty} \beta_t^a$ и~$\beta_*^b\hm= \lim_{T\to \infty} 
\beta_t^b$, не зависящие от~$t$ и~от начального условия $\beta_T^a(T), 
\beta_T^b(T)$ в~задаче Коши, опи\-сы\-ва\-емые уравнениями:
     \begin{equation}
     \beta_*^a A^a +M^\prime_* +N^\prime_* \beta_*^a =0\,;\
     \beta_*^a A^b +N^\prime_* \beta_*^b=0\,.
     \label{e14-bos}
     \end{equation}
     
     Соответственно, коэффициент $\beta_*(y)\hm= \beta_*^a y \hm+ 
\beta_*^b$, а~оптимальное управ\-ле\-ние~(\ref{e5-bos}) принимает вид:
     \begin{multline}
     u^*(y,z) =-\fr{1}{2}\left( R^\prime SR\right)^{-1} \left(2(c^\prime \alpha_* 
+R^\prime SQ)z +{}\right.\\
\left.{}+\left(c^\prime \beta_*^a +2R^\prime SP\right)y +c^\prime 
\beta_*^b\right).
     \label{e15-bos}
     \end{multline}
     
     \vspace*{-6pt}

\section{Численный анализ стационарного режима}

     Для иллюстрации перехода неавтономного управ\-ле\-ния~(\ref{e9-bos}) 
$u_t^\#\hm= u_t^\# (y,z,T)$ в~автономный режим~(\ref{e5-bos}) $u^*\hm= 
u^*(y,z)$, который должен иметь мес\-то при достаточно больших~$T$, 
используем пример, детально рас\-смот\-рен\-ный в~[1]. Простая модель 
эволюции процентных ставок Кок\-са--Ин\-гер\-сол\-ла--Рос\-са  
(Cox--Ingersoll--Ross model)~[10], при\-ме\-ня\-емая так\-же для описания 
показателя RTT (Round-Trip Time) сетевого протокола TCP (Transmission 
Control Protocol)~\cite{11-bos}, имеет вид:
     \begin{equation}
     dy_t=(1-y_t)\,dt+2{,}5\sqrt{y_t}\,dv_t,\ y_0=Y\sim \mathbb{N}(15{,}9). \!
     \label{e16-bos}
     \end{equation}
     
     Выход и целевой функционал описываются уравнениями:
     \begin{gather*}
     dz_t = y_t\,dt-z_t\,dt+u_t\,dt+2{,}5\,dw_t, \ z_0=Z\sim \mathbb{N}(9{,}9);\hspace*{-0.48953pt}\\
     J(U_0^{{T}}) = \mathbb{E}\left\{ \int\limits_0^{{T}} \left( (y_t-z_t)^2 
+z_t^2+u_t^2\right)dt +{}\right.\\
\left.{}+(y_T-z_T)^2 +z_T^2
\vphantom{\int\limits_0^{{T}}}
\right\},\ T=5\,.
     \end{gather*}
     
      \begin{figure*} %fig1
     \vspace*{1pt}
     \begin{minipage}[t]{80mm}
      \begin{center}
     \mbox{%
\epsfxsize=79mm 
\epsfbox{bos-1.eps}
}
\end{center}
\vspace*{-9pt}
     \Caption{Коэффициенты оптимального управ\-ле\-ния: \textit{1}~--- $\alpha_t$; 
\textit{2}~--- $\beta_t^a$;
  \textit{3}~--- $\beta_t^b$}
  \vspace*{3pt}
  \end{minipage}
     % \end{figure*}
   \hfill   
      %\begin{figure*} %fig2
\vspace*{1pt}
\begin{minipage}[t]{80mm}
      \begin{center}
     \mbox{%
\epsfxsize=79.055mm 
\epsfbox{bos-2.eps}
}
\end{center}
\vspace*{-9pt}
\Caption{Примеры сечений поверхности $\beta_t(y,T)$ для разных значений~$y$: сверху вниз~--- 
$y\hm=0; 1; \ldots$; $y\hm=15$}
\vspace*{3pt}
  \end{minipage}
\end{figure*}

 
     
     Процессы $y_t$, $z_t$ и~$u_t$ и~возмущения~$v_t$ и~$w_t$~--- 
скалярные, $\mathbb{N}(M,{\sf D})$ обозначает нормальное распределение со 
средним~$M$ и дисперсией~${\sf D}$. Известно, что траектории~$y_t$ 
неотрицательны, т.\,е.\ $y_t\hm>0$, процесс~--- эргодический, известны 
предельное распределение и~переходная ве\-ро\-ят\-ность. При этом начальные 
условия в~(\ref{e15-bos}) выбраны так, что $M$ и~${\sf D}$ значительно отличаются 
от предельных моментов, так что динамика ис\-сле\-ду\-ет\-ся в~переходном 
процессе. Соответственно, в~стационарном режиме процесс удовле\-тво\-ря\-ет 
предположениям модели~(1), а~также имеет линейный снос, т.\,е.\ мож\-но 
аналитически вычислить коэффициенты~(\ref{e13-bos}) и~оптимальное 
автономное управ\-ле\-ние~(\ref{e14-bos}). Также очевидна устой\-чи\-вость~$z_t$, 
так что вопрос остается только к~выполнению условия~4 тео\-ре\-мы, т.\,е.\ 
к~характеру по\-верх\-ности $\beta_t(y,T)$, опре\-де\-ля\-емой  
уравнением~(\ref{e11-bos}) (в~данном скалярном случае~--- од-\linebreak\vspace*{-12pt}

\pagebreak

\noindent
ним 
параболическим уравнением, а~не сис\-те\-мой) или формулой  
Фейн\-ма\-на--Ка\-ца~(\ref{e12-bos}).

     
     Отметим сначала, что в~рас\-смат\-ри\-ва\-емом примере не\-труд\-но найти 
точное решение уравнения Риккати~(\ref{e10-bos}), а~именно: 
    \begin{multline*}
     \alpha_t= \fr{C_\alpha e^{2\sqrt{3}\,t} (1+\sqrt{3})-1+\sqrt{3}}{1-
C_\alpha e^{2\sqrt{3}t}}\,,\\
     C_\alpha =\fr{3-\sqrt{3}}{3+\sqrt{3}}\,e^{-10\sqrt{3}}\,.
    \end{multline*}
     Уравнение~(\ref{e6-bos}) дает $\alpha_*\hm= \sqrt{3}\hm-1\hm\approx 
0{,}73$, что, как нетрудно видеть, совпадает с $\lim_{t\to -\infty} \alpha_t$. 
Для решения~(\ref{e11-bos}) есть три варианта: не учитывать част\-ный случай 
линейного сноса и~решать~(\ref{e11-bos}) традиционным сеточным методом 
(как предлагается в~\cite{5-bos}), или методом имитационного 
моделирования формулы Фейн\-ма\-на--Ка\-ца (как предлагается в~[4]), или 
решать обыкновенные дифференциальные уравнения~(\ref{e7-bos}). Все три 
расчета пред\-став\-ле\-ны в~[4]. Применительно к~рас\-смат\-ри\-ва\-емой автономной 
задаче все три метода под\-тверж\-да\-ют схо\-ди\-мость $\beta_t(y,T)$ и~дают 
$\beta_*(y)\hm= \beta_*^a y\hm+ \beta_*^b$, где $\beta_*^a\hm\approx -
0{,}19$, $\beta_*^b\hm\approx -0{,}11$, т.\,е.\ при любом варианте расчета 
$\beta_*^a y\hm= \lim_{t\to-\infty} \beta_t^a(y,5)$, $\beta_*^b\hm= \lim_{t\to-
\infty} \beta_t^b(5)$. Значения~$y$ определялись в~результате 
предварительного моделирования~(\ref{e16-bos}) и~заданы отрезком $[0,40]$. 
Иллюстрируют эти сходимости рис.~1 и~2.
     
  
\section{Заключение}

     Представленное исследование дополняет ранее\linebreak решенную задачу 
оптимального управления линейным выходом стохастической сис\-те\-мы по 
квад\-ра\-тич\-но\-му критерию традиционным автономным вариантом 
с~бесконечным временем. Основной \mbox{результат} эксплуатирует классические 
методы исследования уравнения Риккати, как и~в~традиционной 
 LQG-за\-да\-че. Важное отличие при этом заключается в том, что решение 
остается в~классе нелинейных управ\-ле\-ний, что было также важным\linebreak 
свойством задачи с~конечным горизонтом. Завершить изучение данной 
задачи должен случай \mbox{косвенных} наблюдений и~практический анализ 
качества автономного управ\-ле\-ния в~сравнении с~оптимальным на 
конечном горизонте. Эти исследования запланированы на будущее.
     
{\small\frenchspacing
 { %\baselineskip=10.6pt
 %\addcontentsline{toc}{section}{References}
 \begin{thebibliography}{99}
\bibitem{1-bos}
      \Au{Athans M.} The role and use of the stochastic linear-quadratic-Gaussian problem in 
control system design~// IEEE T. Automat. Contr., 1971. Vol.~16. No.\,6. P.~529--552. doi: 
10.1109/TAC.1971.1099818.

\bibitem{2-bos}
\Au{Wonham W.\,M.} Linear multivariable control. A~geometric approach.~--- Lecture notes in 
economics and mathematical systems ser.~--- Berlin: Springer-Verlag, 1974. Vol.~101. 347~p.
\bibitem{3-bos}
\Au{Девис М.\,Х.\,А.} Линейное оценивание и~сто\-ха\-сти\-че\-ское управ\-ле\-ние~/ Пер. с англ.~--- 
М.: Наука, 1984. 206~с. (\Au{Davis~M.\,H.\,A.} Linear estimation and stochastic control.~--- 
London: Chapman and Hall, 1977. 224~p.)
\bibitem{4-bos}
\Au{Босов А.\,В.} Задача управления линейным выходом нелинейной не\-управ\-ля\-емой 
стохастической дифференциальной сис\-те\-мы по квад\-ра\-тич\-но\-му критерию~// Известия 
РАН. Теория и~сис\-те\-мы управ\-ле\-ния, 2021. №\,5. C.~52--73. doi: 
10.31857/S000233882104003X.
\bibitem{5-bos}
\Au{Босов А.\,В., Стефанович~А.\,И.} Управ\-ле\-ние выходом стохастической 
дифференциальной сис\-те\-мы по квад\-ра\-тич\-но\-му критерию. II. Численное решение 
уравнений динамического программирования~// Информатика и её применения, 2019. 
Т.~13. Вып.~1. С.~9--15. doi: 10.14357/19922264190102. EDN: ZASZFR.
\bibitem{6-bos}
\Au{Босов А.\,В., Стефанович~А.\,И.} Управление выходом стохастической 
дифференциальной сис\-те\-мы по квад\-ра\-тич\-но\-му критерию. IV. Альтернативное численное 
решение~// Информатика и~её применения, 2020. Т.~14. Вып.~1. С.~24--30. doi: 
10.14357/19922264200104. EDN: XNHVFT.
\bibitem{7-bos}
\Au{Флеминг У., Ришел~Р.} Оптимальное управление детерминированными 
и~стохастическими сис\-те\-ма\-ми~/ Пер. с~англ.~--- М.: Мир, 1978. 316~с. 
(\Au{Fleming~W.\,H., Rishel~R.\,W.} Deterministic and stochastic optimal control.~--- New 
York, NY, USA: Springer-Verlag, 1975. 222~p.)
\bibitem{8-bos}
\Au{Ширяев А.\,Н.} Вероятность.~--- 2-е изд.~--- М.: Наука, 1989. 640~с.
\bibitem{9-bos}
\Au{{\ptb{\O}}\,\,ksendal~B.} Stochastic differential equations. An introduction with 
applications.~--- New York, NY, USA: Springer-Verlag, 2003. 324~p.
\bibitem{10-bos}
      \Au{Cox J.\,C., Ingersoll~J.\,E., Ross~S.\,A.} A~theory of the term structure of interest 
rates~// Econometrica, 1985. Vol.~53. Iss.~2. P.~385--407. doi: 10.2307/1911242.
\bibitem{11-bos}
      \Au{Bohacek S., Rozovskii~B.} A~diffusion model of roundtrip time~// Comput. 
Stat. Data An., 2004. Vol.~45. Iss.~1. P.~25--50. doi:  
10.1016/S0167-9473(03)00114-2.

\end{thebibliography}

 }
 }

\end{multicols}

\vspace*{-10pt}

\hfill{\small\textit{Поступила в~редакцию 07.12.23}}

\vspace*{8pt}

%\pagebreak

%\newpage

%\vspace*{-28pt}

\hrule

\vspace*{2pt}

\hrule



\def\tit{AUTONOMOUS DIFFERENTIAL SYSTEM LINEAR OUTPUT 
CONTROL BY~SQUARE CRITERION ON~AN~INFINITE HORIZON\\[-5pt]}


\def\titkol{Autonomous differential system linear output 
control by~square criterion on~an~infinite horizon}


\def\aut{A.\,V.~Bosov}

\def\autkol{A.\,V.~Bosov}

\titel{\tit}{\aut}{\autkol}{\titkol}

\vspace*{-15pt}


\noindent
Federal Research Center ``Computer Science and Control'' of the Russian Academy of 
Sciences, 44-2~Vavilov Str., Moscow 119333, Russian Federation

\def\leftfootline{\small{\textbf{\thepage}
\hfill INFORMATIKA I EE PRIMENENIYA~--- INFORMATICS AND
APPLICATIONS\ \ \ 2024\ \ \ volume~18\ \ \ issue\ 1}
}%
 \def\rightfootline{\small{INFORMATIKA I EE PRIMENENIYA~---
INFORMATICS AND APPLICATIONS\ \ \ 2024\ \ \ volume~18\ \ \ issue\ 1
\hfill \textbf{\thepage}}}

\vspace*{1pt}

      
      
     
     \Abste{The problem of optimal control of the stochastic differential system 
linear output on an infinite horizon is solved. The solution is considered as the 
limit form of optimal control in the corresponding problem with a~finite horizon. 
Sufficient conditions for the existence of control are given. They consist of the 
requirements of the stationarity of nonlinear dynamics, the finiteness of the 
quadratic target functional, the stabilizability of the linear output, and the existence 
of a limit in the Feynman--Katz formula describing the nonlinear part of control. 
The conditions for the linear part of the control are related to the classical results of 
the existence of a~solution to the autonomous Riccati equation. The existence of 
a~limit in the Feynman--Katz formula is associated with the solution of a~parabolic 
equation that sets the coefficients for the nonlinear part of the control. A~special 
case of linear drift is considered in which the nonlinear nature of the problem is 
preserved but optimal control turns out to be linear both in output and in the state 
variable. The results of a numerical experiment are presented which makes it 
possible to analyze the transient process in a~problem with a~finite horizon and an 
ergodic process in dynamics. For the control coefficients, the limiting transition to 
the optimal values of the corresponding optimal autonomous control is illustrated.}
     
     \KWE{stochastic differential Ito system; output control; optimal control; 
quadratic criterion; parabolic equation; Feynman--Katz formula}
     
     


\DOI{10.14357/19922264240103}{UEESFO}

\vspace*{-22pt}

\Ack

\vspace*{-3pt}


     \noindent
     The research was carried out using the infrastructure of the Shared Research 
Facilities ``High Performance Computing and Big Data'' (CKP ``Informatics'') of 
FRC CSC RAS (Moscow).


  \begin{multicols}{2}

\renewcommand{\bibname}{\protect\rmfamily References}
%\renewcommand{\bibname}{\large\protect\rm References}

{\small\frenchspacing
 {%\baselineskip=10.8pt
 \addcontentsline{toc}{section}{References}
 \begin{thebibliography}{99} 
\bibitem{1-bos-1}
      \Aue{Athans, M.} 1971. The role and use of the stochastic 
 linear-quadratic-Gaussian problem in control system design. \textit{IEEE T. 
Automat. Contr.} 16(6):529--552. doi: 10.1109/TAC.1971.1099818.
\bibitem{2-bos-1}
      \Aue{Wonham, W.\,M.} 1974. \textit{Linear multivariable control. 
A~geometric approach.} Lecture notes in economics and mathematical systems 
ser. Berlin: Springer-Verlag. 347~p.
\bibitem{3-bos-1}
      \Aue{Davis, M.\,H.\,A.} 1977. \textit{Linear estimation and stochastic 
control}. London: Chapman and Hall. 224~p.



\bibitem{4-bos-1}
      \Aue{Bosov, A.\,V.} 2021. The problem of controlling the linear output of 
a~nonlinear uncontrollable stochastic differential system by the square criterion. 
\textit{J.~Comput. Sys. Sc. Int.} 60(5):719--739. doi: 
10.1134/S1064230721040031.

%\vspace*{-1pt}

\bibitem{5-bos-1}
      \Aue{Bosov, A.\,V., and A.\,I.~Stefanovich.} 2019. Upravlenie vykhodom 
stokhasticheskoy differentsial'noy sistemy po kvadratichnomu kriteriyu. II. 
Chislennoe reshenie uravneniy dinamicheskogo programmirovaniya [Stochastic 
differential system output control by the quadratic criterion. II. Dynamic 
programming equations numerical solution]. \textit{Informatika i~ee 
Primeneniya~--- Inform. \mbox{Appl.}} 13(1):9--15. doi: 10.14357/19922264190102. EDN: 
\mbox{ZASZFR}.



%\vspace*{-1pt}

\bibitem{6-bos-1}
      \Aue{Bosov, A.\,V., and A.\,I.~Stefanovich.} 2020. Upravlenie vykhodom 
sto\-kha\-sti\-che\-skoy differentsial'noy sistemy po
kvadratichnomu kriteriyu. IV. 
Al'ternativnoe chislennoe\linebreak\vspace*{-12pt}

\pagebreak

\noindent
  reshenie [Stochastic differential system output control 
by the quadratic criterion. IV. Alternative numerical decision]. \textit{Informatika 
i~ee Primeneniya~--- Inform. \mbox{Appl.}} 14(1):24--30. doi: 
10.14357/19922264200104. EDN: \mbox{XNHVFT}.
\bibitem{7-bos-1}
      \Aue{Fleming, W.\,H., and R.\,W.~Rishel}. 1975. \textit{Deterministic and 
stochastic optimal control}. New York, NY: Springer-Verlag. 222~p.
\bibitem{8-bos-1}
      \Aue{Shiryaev, A.\,N.} 1996. \textit{Probability}. New York, NY: Springer 
Verlag. 624~p.



\bibitem{9-bos-1}
      \Aue{{\ptb{\O}}ksendal,~B.} 2003. \textit{Stochastic differential equations. An 
introduction with applications}. New York, NY: Springer-Verlag. 324~p.
\bibitem{10-bos-1}
      \Aue{Cox, J.\,C., J.\,E.~Ingersoll, and S.\,A.~Ross.} 1985. A~theory of the 
term structure of interest rates. \textit{Econometrica} 53(2):385--407. doi: 
10.2307/1911242.
\bibitem{11-bos-1}
      \Aue{Bohacek, S., and B.~Rozovskii.} 2004. A~diffusion model of roundtrip 
time. \textit{Comput. Stat. Data An.} 45(1):25--50. doi:  
10.1016/S0167-9473(03)00114-2.

\end{thebibliography}

 }
 }

\end{multicols}

\vspace*{-6pt}

\hfill{\small\textit{Received December 7, 2023}} 

%\vspace*{-18pt}
     
     \Contrl
     
     \vspace*{-3pt}
     
     \noindent
     \textbf{Bosov Alexey V.} (b.\ 1969)~--- Doctor of Science in technology, 
principal scientist, Federal Research Center ``Computer Science and Control'' of 
the Russian Academy of Sciences, 44-2~Vavilov Str., Moscow 119333, Russian 
Federation; \mbox{avbosov@ipiran.ru}


\label{end\stat}

\renewcommand{\bibname}{\protect\rm Литература} 