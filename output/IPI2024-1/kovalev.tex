\def\stat{kovalev}

\def\tit{АЛГЕБРАИЧЕСКАЯ СПЕЦИФИКАЦИЯ РАСПРЕДЕЛЕННЫХ СИСТЕМ С~ИЗМЕНЯЮЩЕЙСЯ 
АРХИТЕКТУРОЙ}

\def\titkol{Алгебраическая спецификация распределенных систем с~изменяющейся 
архитектурой}

\def\aut{С.\,П.~Ковалёв$^1$}

\def\autkol{С.\,П.~Ковалёв}

\titel{\tit}{\aut}{\autkol}{\titkol}

\index{Ковалёв С.\,П.}
\index{Kovalyov S.\,P.}


%{\renewcommand{\thefootnote}{\fnsymbol{footnote}} \footnotetext[1]
%{Работа выполнялась с~использованием инфраструктуры Центра коллективного пользования 
%<<Высокопроизводительные вычисления и~большие данные>> (ЦКП <<Информатика>> ФИЦ ИУ 
%РАН, Москва).}}


\renewcommand{\thefootnote}{\arabic{footnote}}
\footnotetext[1]{Институт проблем управления им.\ В.\,А.~Трапезникова Российской академии наук, \mbox{kovalyov@sibnet.ru}}

\vspace*{-4pt}




\Abst{Развивается предложенный ранее метод обобщенной алгебраической спецификации 
распределенных сис\-тем на базе новой тео\-ре\-ти\-ко-ка\-те\-гор\-ной конструкции графалгебры. 
Основой граф\-ал\-геб\-ра\-иче\-ской спецификации служит ори\-ен\-ти\-ро\-ван\-ный мультиграф, ребра 
которого пред\-став\-ля\-ют вы\-чис\-ли\-тель\-ные операции, вы\-пол\-ня\-емые в~узлах сис\-те\-мы, 
а~вершины задают пор\-ты обмена данными между компонентами. Изменение архитектуры 
сис\-те\-мы в~ходе жизненного цикла приводит к~изменению формы графа, алгоритмов 
вы\-чис\-ле\-ний и~содержания обмена данными.  Для формального описания таких изменений 
предложена техника типа трансформации графов для графалгебр. Введена новая 
 тео\-ре\-ти\-ко-ка\-те\-гор\-ная конструкция гиб\-ких графалгебр, тес\-но связанная с~известной 
монадой диаграмм. Описан функтор, представляющий конструирование категории гиб\-ких 
графалгебр по сигнатуре. Тео\-ре\-ти\-че\-ские результаты про\-ил\-люст\-ри\-ро\-ва\-ны примерами из 
об\-ласти автоматического синтеза архитектуры нейронных сетей путем пошаговых 
трансформаций.}

\KW{алгебраическая спецификация; распределенная система; эволюция архитектуры; теория 
категорий; графалгебра; монада диаграмм}

\DOI{10.14357/19922264240102}{MZAHBS}
  
%\vspace*{-6pt}


\vskip 10pt plus 9pt minus 6pt

\thispagestyle{headings}

\begin{multicols}{2}

\label{st\stat}

\section{Введение}

     Для специфицирования программ с~математическим уровнем строгости 
нередко применяются алгебраические методы (см., например,~[1]). Они 
позволяют описывать операции, вы\-пол\-ня\-емые программами, термами языка 
первого порядка (или его расширений) и~доказывать утверждения об 
операциях, в~том числе индукцией по синтаксической структуре термов. Для 
описания программ, изменяющихся во времени (при фиксированной сигнатуре 
операций), используется аппарат эво\-лю\-ци\-о\-ни\-ру\-ющих алгебр~[2].
    
     В то же время классической алгебраической спецификации присущ 
недостаток, состоящий в~невозможности явно описать структуру 
многокомпонентной распределенной программной сис\-те\-мы.\linebreak Все операции 
описываются так, будто выполняются одним универсальным абстрактным 
вы\-чис\-ли\-тель\-ным устройством. Для преодоления этого\linebreak недостатка была 
предложена конструкция граф\-ал\-геб\-ры~[3]~--- структуры алгебраического типа 
над произвольным ориентированным мультиграфом. Реб\-ра этого графа 
представляют вы\-чис\-ли\-тель\-ные операции, выполняемые в~узлах системы, 
а~вершины~--- порты обмена данными между компонентами. Сигнатура 
графалгебры со\-сто\-ит из графа и~правил пред\-став\-ле\-ния данных в~портах.
     
     Однако фиксированная форма сигнатуры ограничивает применение 
категорий графалгебр в~приложениях, поскольку современным распределенным 
сис\-те\-мам присуща гиб\-кость~--- спо\-соб\-ность изменять граф структуры, в~том 
числе в~темпе, близком к~реальному времени. Например, архитектура 
многослойной нейронной сети может быть оптимизирована под класс 
решаемых задач (по чис\-лу\linebreak слоев, мощ\-ности каждого слоя, топологии и~весам\linebreak 
связей) путем це\-ле\-на\-прав\-лен\-ных многошаговых трансформаций некоторой 
небольшой начальной сети~[4], а~распределенные  
про\-грам\-мно-ап\-па\-рат\-ные сис\-те\-мы всегда изменяются в~ходе создания и~модернизации.
     
     Для формального описания и~верификации процедур эволюции систем 
применяется техника типа трансформации графов~[5]. Для графалгебр такая 
техника впервые предложена в~на\-сто\-ящей статье. Введена новая тео\-ре\-ти\-ко-ка\-те\-гор\-ная конструкция гибких графалгебр, тес\-но связанная с~известной 
монадой диаграмм.

\vspace*{-6pt}
     
\section{Графалгебры}

\vspace*{-1pt}

     С точки зрения теории категорий алгебра над некоторой 
категорией~$C$~---  это морфизм~$a: FA \hm\to A$, где $F$~--- эндофунктор 
в~$C$, представляющий сигнатуру; $A$~--- носитель. У~алгебры 
первопорядковой сигнатуры~$\Sigma$ функтор~$F$ имеет 
<<полиномиальный>> вид: 
$$
FA=\coprod_{\varphi\in \Sigma}  
A^{n(\varphi)},
$$
 где $n(\varphi)$~--- мест\-ность (чис\-ло аргументных мест) 
функционального символа $\varphi\hm\in \Sigma$. Алгебру можно 
рас\-смат\-ри\-вать как $C$-диа\-грам\-му в~форме~\textbf{2} (так обозначается граф, 
со\-сто\-ящий из двух вершин 0 и~1 и~одного ребра $0\hm\to 1$), а~гомоморфизм 
алгебр~--- как морфизм носителей, индуцирующий естественное 
преобразование таких диаграмм. Путем обобщения этой точки зрения 
получается графалгебра~--- диаграмма произвольной формы, вершины которой 
помечены результатами действия заданных сигнатурных функторов на 
носитель~[3]. Все сигнатурные функторы имеют одну и~ту же об\-ласть и~одну 
и~ту же ко\-об\-ласть (в~общем случае отличную от об\-ласти). Поэтому сигнатурой 
графалгебры служит любая пара $\langle I, F : D \hm\to C^{\mathrm{Ob}\,I}\rangle$, 
где $I$~--- малая категория, называемая формой граф\-ал\-геб\-ры; категория~$D$~--- 
пространство данных; $C$~--- пространство вы\-чис\-ле\-ний. 
Графалгебра сигнатуры $\langle I, F\rangle$~--- это произвольная пара $(A, 
\Delta)$, состоящая из носителя $A\hm\in \mathrm{Ob}\,D$ и~диаграммы 
структуры $\Delta  : I \hm\to C$, такая что $\Delta i \hm= \mathrm{Pr}_iFA$, где $\mathrm{Pr}_i : 
C^{\mathrm{Ob}\,I} \hm\to C$~--- проекция степени на $i$-й экземп\-ляр основания, $i 
\hm\in \mathrm{Ob}\,I$. Гомоморфизмом графалгебры $(A, \Delta)$ 
в~$(A^\prime, \Delta^\prime)$ называется морфизм носителей $f : A\hm\to  
A^\prime$ такой, что семейство морфизмов $\mathrm{Pr}_iFf : \Delta i \hm\to  
\Delta^\prime i$, $i \hm\in \mathrm{Ob}\,I$, образует естественное 
преобразование~$\Delta$ в~$\Delta^\prime$. Все графалгебры некоторой 
сигнатуры и~все их гомоморфизмы образуют категорию, обозначаемую через 
$\mathsf {A}_I F$, на которой действуют канонические за\-бы\-ва\-ющие 
функторы: функтор носителя $U_I^F : \mathsf {A}_IF \hm\to  D : (A, \Delta) 
\hm\mapsto A$ и~функтор структуры $\mathsf {S}_I^F : \mathsf {A}_IF \hm\to 
C^I : (A, \Delta) \hm\mapsto \Delta$.
     
     Графалгебры выступают спецификациями распределенных сис\-тем: 
стрелки диаграммы структуры помечаются вычислительными операциями 
в~узлах системы, а~вершины описывают передаваемые между узлами данные. 
С~этой точки зрения обычная алгебра действительно представляет автономный 
вычислительный узел, а,~например, многослойная нейросеть прямого 
распространения имеет форму нетривиальной конечной цепочки: это граф 
множеств и~отображений вида
     $$
A^{k_0}\to  A^{k_1}\to \cdots\to  A^{k_l},
$$
где $A$~--- множество значений распознаваемого сигнала (например, $A\hm= 
\mathbb{R}$); $l \hm> 1$~--- глубина сети; $k_i \hm> 0$~--- число нейронов  
в~$i$-м слое, $i \hm= \overline{0,  l - 1}$. Обработка сигнала в~каждом слое $f_i : 
A^{k_i}\hm\to  A^{k_{i+1}}$ состоит из отоб\-ра\-же\-ний-ком\-по\-нен\-тов (например:
\begin{multline*}
     f_{is} : \mathbb{R}^{k_i} \to \mathbb{R} : (x_1, \ldots, x_{k_i}) \mapsto {}\\
     \mapsto
\sum\limits_{j=1}^{k_i}  w_{ijs}\sigma_{ij}(x_j + b_{ij}),\enskip
      s = \overline{1, k_{i+1}}\,,
     \end{multline*}
где $\sigma_{ij} : \mathbb{R}\hm\to \mathbb{R}$~--- функция активации $j$-го нейрона $i$-го 
слоя; $w_{ijs}$ и~$b_{ij}$~--- веса и~смещения, подбираемые при обучении 
нейросети). Сигнатура графалгебраической спецификации такой нейросети 
имеет форму цепочки из $l$ стрелок, каждой вершине которой соответствует 
эндофунктор возведения в~конечную степень в~\textbf{Set}.
     
     Распределенные системы общего вида, в~том числе сложно устроенные 
нейросети, могут иметь графалгебраическую спецификацию практически 
любой конечной формы (с~учетом ресурсных ограничений). В~дополнение 
к~конечным цепочкам, пред\-став\-ля\-ющим последовательную конвейерную 
обработку данных, форма может содержать подграфы, представляющие 
параллельную обработку (например, $\bullet \rightrightarrows 
\bullet$), обратную связь (например, $\bullet 
\rightleftarrows\bullet$), разделяемую память (например, $\hspace*{-4pt}{\raisebox{-2pt}{
\epsfxsize=3mm %6.548mm
\epsfbox{kov-t-1.eps}
}}\! 
\bullet \hspace*{-4pt}{\raisebox{-2pt}{
\epsfxsize=3mm %6.548mm
\epsfbox{kov-t-2.eps}
}})$, экземпляры других шаблонов проектирования. Сигнатурные 
функторы полиномиальны, коль скоро функционирование сис\-те\-мы сводится 
к~традиционным вычислениям и~детерминированному взаимодействию.

\vspace*{-4pt}

\section{Гибкие графалгебры}

\vspace*{-1pt}

     Гомоморфизмы графалгебр фиксированной формы~$I$ индуцируют 
естественные преобразования диаграмм структуры~--- морфизмы  
в~ка\-те\-го\-рии-экс\-по\-нен\-те~$C^I$. Чтобы связать морфизмами 
графалгебры произвольных форм, необходимо объединить и~<<провязать>> 
меж\-ду собой все экспоненты с~фиксированным основанием~$C$ и~малыми 
категориями в~качестве показателя. Для этого существует канонический способ, 
а~именно: уплощающая конструкция Гротендика (Grothendieck flattening 
construction~[6, п.~12.2.10]). Произвольному функтору вида $F : Q \hm\to 
\mathbf{CAT}$ эта конструкция со\-по\-став\-ля\-ет категорию $\int F$ типа 
зависимой суммы (dependent sum), объектом которой служит любая пара $(A, 
X)$, $X \hm\in \mathrm{Ob}\,Q$, $A \hm\in \mathrm{Ob}\,FX$, а~морфизмом 
пары $(A, X)$ в~$(A^\prime, X^\prime)$ служит любая пара $(f : (Fg)A \hm\to  
A^\prime, g : X \hm\to  X^\prime)$ (с законом композиции вида $(f, g) \circ (h, q) 
\hm= (f \circ (Fg)h, g \circ q))$. Имеется забывающий функтор $\bm{U}_{(F)} : \int 
F \hm\to  Q : (A, X) \hm\mapsto X$, $(f, g) \hm\mapsto g$. И~обратно: каждый 
$X\hm\in \mathrm{Ob}\,Q$ по\-рож\-да\-ет вложение $\bm{i}_X : FX \hm\hookrightarrow  
\int F : A \mapsto (A, X)$, $f \hm\mapsto (f, 1_X)$, причем в~сумме из таких 
вложений получается вложение  
$$
[\bm{i}_X]_{X \in\mathrm{Ob}\,D} : \coprod\limits_{X\in\mathrm{Ob}\,D} FX 
\hm\hookrightarrow  \int F,
$$ биективное на объектах.
     
     Для функтора $C^{(-)} : \mathbf{Cat}^{\mathrm{op}} \hm\to \mathbf{CAT} : X \hm\mapsto 
C^X$ конструкция Гротендика дает категорию диа-\linebreak\vspace*{-12pt}

\pagebreak

\noindent
грамм $\mathbf{D}C$, класс 
объектов которой состоит из всех \mbox{$C$-диа}\-грамм, а~морфизмом диаграммы 
$\Delta : X \hm\to C$ в~$\Sigma : Y \hm\to C$ служит любая пара вида 
$\langle\gamma, \mathsf {f}\rangle$, состоящая из функтора $\mathsf {f} : X 
\hm\to Y$ и~естественного преобразования $\gamma : \Delta \hm\to 
\Sigma\mathsf {f}$; закон композиции морфизмов диаграмм имеет вид:
$$
\langle \gamma, \mathsf {f}\rangle \circ \langle\varphi, \mathsf {g}\rangle = \langle 
\gamma_{\mathsf {g}(-)} \circ \varphi, \mathsf {fg}\rangle.
$$
 С~при\-клад\-ной точ\-ки 
зрения такие морфизмы описывают структурные преобразования моделей\linebreak 
сис\-тем в~алгебраическом пред\-став\-ле\-нии~[7]. Имеется забывающий функтор 
формы $\mathbf{D}!_{C} : \mathbf{D}C\hm\to \mathbf{Cat} : \Delta \mapsto  
\mathrm{dom}\,\Delta$, $\langle \gamma, \mathsf {f}\rangle \hm\mapsto 
\mathsf {f}$. Такое обозначение функтора связано с~тем, что соответствие 
$C\hm\mapsto \mathbf{D}C$ функториально: оно расширяется до эндофунктора 
$\mathbf{D} : \mathbf{CAT}\hm\to \mathbf{CAT}$ (а~функтор формы 
получается путем его действия на функтор $!_{C} : C \hm\to \mathbf{1}$, где 
$\mathbf{1}$~--- сингулярная категория). Эндофунктор~$\mathbf{D}$ 
индуцирует 2-мо\-на\-ду в~$\mathbf{CAT}$, известную как метамодель 
системного моделирования, поскольку в~ней <<закодированы>> основ\-ные 
применяемые в~нем категорные конструкции~[8].
     
     Форма графалгебры фигурирует не только в~диаграмме структуры, но 
и~в~ко\-об\-ласти сигнатурного функтора, в~час\-ти множества вершин, которое\linebreak 
выступает показателем экспоненты. Каноническое объединение всех экспонент 
с~фиксированным основанием и~множеством в~качестве показателя так\-же 
получается путем конструкции \mbox{Гротендика}, только для функтора  
$C^{(-)}: \mathbf{Set}^{\mathrm{op}}\hm\to \mathbf{CAT}$. Получается полная подкатегория 
в~$\mathbf{D}C$, со\-сто\-ящая из всех дискретных $C$-диа\-грамм, которую будем 
обозначать через $\mathbf{D}_{\mathbf{Set}}C$. Она корефлективна 
в~$\mathbf{D}C$: корефлектор убирает из произвольной диаграммы $\Delta : 
X\hm\to  C$ все не\-тож\-дест\-вен\-ные морфизмы,  превращая ее в~дискретную 
диаграмму $]\Delta[ : \mathrm{Ob}\,X \hm\to C : x \hm\mapsto \Delta x$. 
Соответствие $C\hm\mapsto \mathbf{D}_{\mathbf{Set}}C$ расширяется до 
эндофунктора в~$\mathbf{CAT}$, ин\-ду\-ци\-ру\-юще\-го подмонаду в~$\mathbf{D}$, 
причем указанные корефлекторы образуют ретракцию монады~$\mathbf{D}$ на 
нее.
     %
     Тем самым из <<материала>> монады диаграмм получается следующее 
обобщение конструкции графалгебры.
     
     \smallskip
     
     \noindent
     \textbf{Определение~1.}\ Гибкой графалгеброй сигнатуры\linebreak $F : D \hm\to  
\mathbf{D}_{\mathbf{Set}}C$ называется любая пара $(A, \Delta : I \hm\to  C), A 
\hm\in \mathrm{Ob}\, D$, $I \hm\in \mathbf{Cat}$, такая что $]\Delta[ \hm= FA$.\linebreak 
Морфизмом такой гиб\-кой графалгебры в~($A^\prime, \Delta^\prime : 
I^\prime\hm\to  C$) называется любая пара ($f : A \hm\to  A^\prime, \mathsf {g} : 
I \hm\to  I^\prime$), такая что существует (обязательно единственное) семейство 
морфизмов $\theta_i : \Delta i \hm\to  \Delta^\prime \mathsf {g}i$, $i \hm\in 
\mathrm{Ob}\,I$, такое что пара $\langle \theta, ]\mathsf{g}[ : 
\mathrm{Ob}\,I\hm\to  \mathrm{Ob}\,I^\prime\rangle$ совпадает с~$Ff : ]\Delta[ 
\hm\to  ]\Delta^\prime[$, а~пара $\langle\theta, \mathsf {g}\rangle$ 
пред\-став\-ля\-ет собой морфизм диаграммы~$\Delta$ в~$\Delta^\prime$ 
в~$\mathbf{D}C$.~\hfill$\square$
     
     \smallskip
     
     \noindent
     \textbf{Теорема~1.}\ \textit{Все гибкие графалгебры сигнатуры\linebreak $F : D 
\hm\to  {\bm D}_{\mathbf{Set}}C$ и~все их морфизмы образуют категорию, изоморфную 
вершине сле\-ду\-юще\-го декартова квадрата в}~$\mathbf{CAT}$:

%\begin{figure*} %fig
\vspace*{1pt}
      \begin{center}
     \mbox{%
\epsfxsize=28.351mm 
\epsfbox{kov-1.eps}
}
\end{center}
%\vspace*{-9pt}
%\end{figure*}

     \noindent
     Д\,о\,к\,а\,з\,а\,т\,е\,л\,ь\,с\,т\,в\,о\,.\ \  Проверяется непосредственно по 
правилам вычисления пределов и~экспонент в~$\mathbf{CAT}$.~\hfill$\square$
     
     \smallskip
     
     Ребра декартова квадрата из теоремы~1 задают канонические 
забывающие функторы, которые, аналогично графалгебрам фиксированной 
формы, выделяют из гиб\-ких графалгебр сигнатуры~$F$ носитель (левая 
вертикальная стрел\-ка $\mathsf {U}^ F : \mathsf {A}F \hm\to D : (A, \Delta) 
\hm\mapsto A$) и~структуру (верх\-няя горизонтальная стрелка $\mathsf {S}^F : 
\mathsf {A}F \hm\to \mathbf{D}C : (A, \Delta) \hm\mapsto \Delta$). Имеет 
значение также функтор выделения формы
     \begin{multline*}
     \mathsf {H}^F = (\mathbf{D}!_C)\mathsf {S}^F : \mathsf {A}F \to 
\mathbf{Cat} : (A, \Delta) \mapsto \mathrm{dom}\,\Delta,\\
 (f, \mathsf {g}) \mapsto  
\mathsf {g}.
\end{multline*}
В частности, функтор $\langle \mathsf {U}^F, \mathsf {H}^F \rangle : 
\mathsf {A}F \hm\to  D \times \mathbf{Cat} : (A, \Delta) \hm\mapsto (A, 
\mathrm{dom}\ \Delta)$ унивалентен, а~функтор носителя, как правило, нет. Зато 
есть возможность строить свободные гиб\-кие графалгебры.

\smallskip

     \noindent
     \textbf{Следствие~1.1.} Функтор носителя гибких графалгебр любой 
сигнатуры представляет собой корефлектор.
     
     \smallskip
     
     \noindent
     Д\,о\,к\,а\,з\,а\,т\,е\,л\,ь\,с\,т\,в\,о\,.\ \ Непосредственно проверяется, что 
класс всех корефлекторов устойчив относительно декартовых квадратов 
в~$\mathbf{CAT}$. В~явной форме для произвольной сигнатуры гибких 
графалгебр $F : D \hm\to \mathbf{D}_{\mathbf{Set}}C$ левым сопряженным 
к~$\mathsf {U}^F$ служит функтор
     $$
     \mathsf {G}^F : D \hookrightarrow  \mathsf {A}F : A 
\mapsto (A, FA), f \mapsto (f, (\mathbf{D}_{\mathbf{Set}}!_C)Ff).
     $$
Единица этого сопряжения тождественна, а~коединица состоит из морфизмов 
гиб\-ких графалгебр ($1_A, \mathrm{Ob}\,\mathrm{dom}\,\Delta   
\hm\hookrightarrow \mathrm{dom}\,\Delta) : (A, FA) \hm\to  (A, \Delta)$, $(A, 
\Delta) \hm\in \mathrm{Ob}\,\mathsf {A}F$.~\hfill$\square$

\smallskip

\noindent
\textbf{Следствие~1.2.}\ Для любой сигнатуры гибких граф\-ал\-гебр $F : D \hm\to 
{\bm D}_{\mathbf{Set}}C$ следующие утверждения эквивалентны:
\begin{itemize}
\item[$(i)$] функтор носителя представляет собой изоморфизм;
\item[$(ii)$] функтор носителя унивалентен;
\item[$(iii)$] $F$ сопоставляет любому носителю пустую диаграмму.
\end{itemize}

     \noindent
     Д\,о\,к\,а\,з\,а\,т\,е\,л\,ь\,с\,т\,в\,о\,.
     
     $(iii)$\;$\Rightarrow$\;$(i)$. Если $]\Delta[$~--- пустая диаграмма, то 
и~$\Delta$ пуста, поэтому при выполнении условия~($iii$) коединица 
сопряжения $\mathsf {G}^F \dashv \mathsf {U}^F$ тождественна, т.\,е.\ 
$\mathsf {G}^F\mathsf {U}^F \hm= 1_{\mathsf {A}F}$.

\vspace*{2pt}
     
     $(i)$\;$\Rightarrow$\;$(ii)$. Тривиально.
     
     \vspace*{2pt}
     
     $(ii)$\;$\Rightarrow$\;$(iii)$. Предположим противное: пусть диаграмма $FA$ 
при некотором $A \hm\in \mathrm{Ob}\, D$ содержит точку~$i$. Обозначим 
через $\Delta$ $C$-диа\-грам\-му, полученную из $FA$ путем добавления 
эндострелки $e : i \hm\to i$ с~законом композиции $e \circ e \hm= 1_i$, 
помеченной морфизмом $1_{(FA)i}$. Гиб\-кая графалгебра $(A, \Delta)$ имеет два 
разных эндоморфизма с~тождественным носителем: один переводит $e$ в~$e$, 
а~другой~--- в~$1_i$.~\hfill$\square$

\smallskip
     
     Чтобы выделить в~$\mathsf {A}F$ графалгебры произвольной 
фиксированной формы~$I$, нужно найти в~$D$\linebreak все носители, которым 
сигнатурный функ\-тор сопоставляет объекты категории~$C^{\mathrm{Ob}\,I}$. Для 
этого\linebreak используется конструкция полного прообраза сигнатурного функтора~--- 
декартова квад\-ра\-та с~со\-от\-вет\-ст\-ву\-ющим вложением. Любая гибкая граф\-ал\-геб\-ра 
сигнатуры~$F$ содержится в~одной и~только одной из категорий графалгебр 
фиксированной формы, полученных таким способом.
     
     \smallskip
     
     \noindent
     \textbf{Теорема~2.} \textit{Пусть $F : D \hm\to {\bm D}_{\mathbf{Set}}C$~--- 
сигнатура гибких графалгебр, $I$~--- произвольная форма, $F\vert_I : D\vert_I 
\hm\to C^{\mathrm{Ob}\,I}$~--- полный прообраз функтора~$F$ относительно 
канонического вложения $\bm{i}_{\mathrm{Ob}\,I} : C^{\mathrm{Ob}\,I} 
\hookrightarrow {\bm D}_{\mathbf{Set}}C$. Полный прообраз функтора структуры 
$\mathsf {S}^F : \mathsf {A}F \hm\to  {\bm D}C$ относительно канонического 
вложения ${\bm i}_I : C^I \hookrightarrow  {\bm D}C$ совпадает (с точностью 
до изоморфизма) с~функтором структуры категории  
$\langle I, F\vert_I\rangle$-ал\-гебр, которая тем самым канонически 
вкладывается в}~$\mathsf {A}F$:

%\begin{figure*} %fig2
\vspace*{1pt}
      \begin{center}
     \mbox{%
\epsfxsize=28.307mm 
\epsfbox{kov-2.eps}
}
\end{center}
%\vspace*{-9pt}
%\end{figure*}

\noindent
     Д\,о\,к\,а\,з\,а\,т\,е\,л\,ь\,с\,т\,в\,о\,.\ \  Рассмотрим сле\-ду\-ющую диаграмму в~форме куба:
     
%\begin{figure*} %fig3
\vspace*{1pt}
      \begin{center}
     \mbox{%
\epsfxsize=38.403mm 
\epsfbox{kov-3.eps}
}
\end{center}
%\vspace*{-9pt}
%\end{figure*}

\noindent
     Здесь стрелки с~закругленным началом обозначают канонические 
вложения (так что, в~част\-ности, правая грань является коммутативным 
квад\-ра\-том); ниж\-няя грань представляет построение функтора $F\vert_I$ как 
полного прообраза (т.\,е.\ является декартовым квад\-ра\-том); передняя грань~--- 
это декартов квад\-рат из тео\-ре\-мы~1; задняя грань (с~правой вертикальной 
стрелкой $C^{(\mathrm{Ob}\,I \hookrightarrow I)} : C^I \hm\to  C^{\mathrm{Ob}\,I}$)~--- это 
декартов квадрат, представляющий построение категории графалгебр $\mathsf{A}_I F\vert_I$ [3, 
теорема~1]; стрелка из $\mathsf {A}_I F\vert_I$ в~$\mathsf {A}F$ изображает 
единственный функтор, де\-ла\-ющий всю диаграмму коммутативной. Ввиду 
композиционных свойств декартовых квад\-ра\-тов~[9, предложение~11.10(1)], 
прямоугольник $\mathsf {A}_I F\vert_I$--$C^I$--$\mathbf{D}_{\mathbf{Set}}C$--$D$ является 
декартовым, а~значит, и~верх\-няя грань куба~--- декартов квад\-рат.~\hfill$\square$
     
     Чтобы описать динамику изменения формы\linebreak структуры распределенной 
системы, введем пространство данных $D$ вида $\check{D}\times T$, 
где~$\check{D}$~--- инвариантное (не зависящее от времени) пред\-став\-ле\-ние 
данных; $T$~--- вполне упорядоченное \mbox{множество}, пред\-став\-ля\-ющее время. 
(Напомним, что любое предупорядоченное множество стандартным образом 
пред\-став\-ля\-ет\-ся в~виде категории~[10, \S\,I.2].) Тогда графалгебраические 
спецификации сис\-тем, которые могут существовать в~некоторый момент 
времени $t \hm\in T$, определяются как объекты подкатегории 
$\mathsf {A}F\vert_t \hm\subseteq \mathsf {A}F$~--- полного прообраза 
функтора носителя относительно сечения 
$$
\langle 1_{\check{D}}, \lceil t\rceil  !_{\check{D}}\rangle  : \check{D} \hm\hookrightarrow D : X \hm\mapsto \langle X, 
t\rangle.
$$

 Сценарий эволюции некоторой сис\-те\-мы $(A, \Delta) \hm\in 
\mathrm{Ob}\,\mathsf {A}F$ может быть задан функтором вида $S : T 
\hookrightarrow \mathsf {A}F$, обратным справа к~функтору проекции 
{носителя} на~$T$ и~при\-ни\-ма\-ющим значение $(A, \Delta)$ в~начальный момент 
времени. Появляется воз\-мож\-ность строго специфицировать и~верифицировать 
структурные требования к~трансформациям архитектуры сис\-те\-мы в~ходе 
эволюции, такие как переход от реализации некоторого шаб\-ло\-на 
проектирования к~реализации другого. Требования такого рода задаются как 
свойства входящих в~об\-ласть значений функтора~$S$ морфизмов гиб\-ких 
графалгебр.
     
     Например, автоматический синтез архитектуры нейросетей путем 
пошаговых трансформаций описывается гибкими графалгебрами сле\-ду\-ющим 
образом. Выберем в~качестве $\check{D}$ подкатегорию в~$\mathbf{Set}$, 
состоящую из единственного объекта~$\mathbb{R}$ и~под\-хо\-дя\-щих 
вещественных функций (<<перекодировок>> в~составе до\-пус\-ти\-мых 
трансформаций), в~качестве~$T$ некоторый начальный отрезок натуральных 
чисел (счетчик шагов), а~в~качестве пространства вы\-чис\-ле\-ний~$C$ категорию 
$\mathbf{Set}$. Сигнатурный функтор~$F$ сопоставляет паре $(\mathbb{R}, 
i)$, $i \hm\in T$, конечную дискретную диаграмму, размеченную конечными 
степенями множества~$\mathbb{R}$. Тем самым сигнатура фиксирует чис\-ло 
и~мощность слоев нейросети на каждом шаге, но не накладывает ограничений 
на связи между\linebreak слоями, что поз\-во\-ля\-ет искать среди объектов подкатегории 
$\mathsf {A}F\vert_i$ наиболее подходящую конфигурацию (топологию и~веса) 
связей. Можно по\-рож\-дать и~параллельно анализировать разные \mbox{варианты} 
сигнатуры, например с~по\-мощью генетического алгоритма. В~итоге любой 
сценарий синтеза сети $S : T \hookrightarrow \mathsf {A}F$ включает морфизм 
результата $i$-го шага в~результат ($i \hm+ 1$)-го, га\-ран\-ти\-ру\-ющий сохранение 
функ\-ци\-о\-наль\-ности при трансформации~--- необходимое условие 
вы\-чис\-ли\-тель\-ной эф\-фек\-тив\-ности алгоритма~[4]. Действительно, требование 
наличия такого морфизма может существенно \mbox{сузить} пространство поиска 
конфигурации связей, ограничивая его кообластями морфизмов результата 
предыду\-ще\-го шага. Тео\-ре\-ти\-ко-ка\-те\-гор\-ное описание открывает путь 
к~дальнейшему повышению эффективности путем привлечения средств 
компьютерной алгебры для навигации вдоль морфизмов графалгебр в~ходе 
поиска.
     
     В еще более общем случае эволюции распределенной сис\-те\-мы от 
времени зависит не только ее форма, но и~пред\-став\-ле\-ние данных. Чтобы 
обеспечивать преемственность данных по ходу эволюции такой сис\-те\-мы, 
указанная за\-ви\-си\-мость задается функтором $D^\circ : T \hm\to  \mathbf{CAT}$ 
таким, что $D^\circ t$~--- представление данных в~момент времени $t \hm\in T$. 
В~качестве носителя гибких графалгебр целесообразно выбрать конструкцию 
Гротендика $\int D^\circ$. Если и~пред\-став\-ле\-ние вы\-чис\-ле\-ний зависит от 
времени, то вводится функ\-тор $C^\circ : T \hm\to \mathbf{CAT}$ и~сигнатура 
приобретает вид: 
$$
F : \int D^\circ \hm\to \mathbf{D}_{\mathbf{Set}} \int C^\circ.
$$
 Чтобы она 
не нарушала при\-чин\-ность, в~любой момент времени~$t$ она долж\-на 
удовле\-тво\-рять соотношению 
\begin{multline*}
\left(\mathbf{D}_{\mathbf{Set}}(\lceil t\rceil !_{\int\! C^\circ})\right)F{\bm i}_t 
={}\\
{}= \left(\mathbf{D}_{\mathbf{Set}}{\bm U}_{(C^\circ)}\right)F{\bm i}_t : D^\circ t \hm\to  
\mathbf{D}_{\mathbf{Set}}T.
\end{multline*}
 Сценарий эволюции сис\-те\-мы с~такой сильно изменчивой 
архитектурой~--- это функтор $S : T \hm\hookrightarrow \mathsf {A}F$ такой, что 
$$
\bm{U}_{(D^\circ)}\mathsf {U}^FS \hm= 1_T.
$$
 Например, в~этой форме можно будет 
описать технологии развития компьютерного пред\-став\-ле\-ния объектов 
управления от грубой рас\-чет\-ной/ин\-фор\-ма\-ци\-он\-ной модели до 
полнофункционального циф\-ро\-во\-го\- двойника. Такие технологии активно 
разрабатываются в~настоящее время~[11].
     
\section{Функтор гибких графалгебр}

     Примечательно, что сигнатуры гиб\-ких граф\-ал\-гебр также строятся из 
<<материала>> монады диаграмм: в~со\-во\-куп\-ности все они образуют класс 
морфизмов категории Клейсли монады $\mathbf{D}_{\mathbf{Set}}$. \mbox{Однако}, чтобы 
выявить функториальную природу конструкции гиб\-кой графалгебры, требуется 
категория, в~которой сигнатуры выступают объектами, а~не морфизмами. 
В~таких случаях применяется другая универсальная конструкция~--- категория 
запятой (comma category) [10, \S\,II.6] (которая, в~свою очередь, пред\-став\-ля\-ет 
собой част\-ный случай графалгебры~[3]).
     
     \smallskip
     
     \noindent
     \textbf{Теорема~3.}\ \textit{Отображение $F\hm\mapsto \mathsf {A}F$ служит функцией 
объектов функтора}
     $$
     \mathsf {A} : (1_{\mathbf{CAT}}\downarrow 
\mathbf{D}_{\mathbf{Set}}) \to \mathbf{CAT}\,.
     $$
     
     
     \noindent
     Д\,о\,к\,а\,з\,а\,т\,е\,л\,ь\,с\,т\,в\,о\,.\ \  По определению категории запятой 
объектом в~($1_{\mathbf{CAT}}\downarrow \mathbf{D}_{\mathbf{Set}}$) служит в~точ\-ности любой 
функтор $F : D \hm\to \mathbf{D}_{\mathbf{Set}}C$, т.\,е.\ сигнатура гибких графалгебр,\linebreak 
а~его морфизмом в~$F^\prime : D^\prime \hm\to \mathbf{D}_{\mathbf{Set}}C^\prime$ 
служит любая пара функторов $(R : D \hm\to D^\prime, S : C \hm\to  C^\prime)$, 
такая что $F^\prime R \hm= (\mathbf{D}_{\mathbf{Set}}S)F$. Эта пара индуцирует 
единственный функтор из $\mathsf {A}F$ в~$\mathsf {A}F^\prime$, который 
делает коммутативным весь нижеприведенный куб, передняя и~задняя грани 
которого представляют собой декартовы квадраты из теоремы~1:

   %  \begin{figure*} %fig4
     \vspace*{1pt}
      \begin{center}
     \mbox{%
\epsfxsize=37.322mm 
\epsfbox{kov-4.eps}
}
\end{center}
%\vspace*{-9pt}
    % \end{figure*}
     


\section{Заключение}

     Конструкция графалгебры открывает новые горизонты в~применении 
алгебраических методов для анализа и~проектирования распределенных сис\-тем. 
В~ее гибком варианте появляются возможности формально специфицировать, 
верифицировать и~в~конечном счете эффективно автоматизировать процедуры 
эволюционного изменения архитектуры систем. Для практической реализации 
этих возможностей предстоит создать удобные языки и~инструменты.
     
С~теоретической точки зрения примечательно, что гибкие графалгебры 
строятся из <<материала>> монады диаграмм: это подчеркивает ключевую 
роль указанной монады в~качестве математического фундамента сис\-тем\-ной 
инженерии. В~част\-ности, намечается <<сис\-тем\-но-ин\-же\-нер\-ный>> 
подход к~конструированию классического понятия ал\-геб\-ры: категорию 
обыч\-ных алгебр можно получить как част\-ный случай комбинации декартовых 
квад\-ра\-тов, приведенных в~тео\-ре\-мах~1 и~2.
     
{\small\frenchspacing
 { %\baselineskip=10.6pt
 %\addcontentsline{toc}{section}{References}
 \begin{thebibliography}{99}
\bibitem{1-kov}
\Au{Ковалёв С.\,П.} Формальный подход к~разработке программных сис\-тем.~--- 
Новосибирск: НГУ, 2004. 180~с.
\bibitem{2-kov}
\Au{Gurevich Y.} Evolving algebras 1993: Lipari guide~// Specification and validation methods.~--- 
Oxford: Oxford University Press, 1995. P.~9--36.
\bibitem{3-kov}
\Au{Ковалёв С.\,П.} Алгебраическая спецификация графовых вычислительных структур~// 
Информатика и~её применения, 2022. Т.~16. Вып.~1. С.~2--9. doi: 10.14357/ 19922264220101. 
EDN: GXHPZI.
\bibitem{4-kov}
\Au{Cai H., Chen~T., Zhang~W., Yu~Y., Wang~J.} Efficient architecture search by network 
transformation~// 32th AAAI Conference  ``Artificial Intelligence'' Proceedings~/ Eds. 
S.\,A.~McIlraith, K.\,Q.~Weinberger.~--- Palo Alto, CA, USA: AAAI Press, 2018. P.~2787--2794.
\bibitem{5-kov}
   \Au{Mens T., Magee~J., Rumpe~B.} Evolving software architecture descriptions of critical 
systems~// Computer,  2010. Vol.~43. No.\,5. P.~42--48. doi: 10.1109/MC.2010.136.

\bibitem{6-kov}
\Au{Barr M., Wells~C.} Category theory for computing science.~--- London: Prentice Hall, 1990. 
538~p.
\bibitem{7-kov}
   \Au{Ковалёв С.\,П.} Методы теории категорий в~модельно-ори\-ен\-ти\-ро\-ван\-ной сис\-тем\-ной 
инженерии~// Информатика и~её применения, 2017. Т.~11. Вып.~3. С.~42--50. doi: 
10.14357/19922264170305. EDN: ZGIGGD.

\bibitem{8-kov}
   \Au{Ковалёв С.\,П.} Теория категорий как математическая прагматика модельно-ориентированной сис\-тем\-ной инженерии~// Информатика и~её применения, 2018. Т.~12. 
Вып.~1. С.~95--104. EDN: YTTRFC.

\bibitem{9-kov}
\Au{Ad$\acute{\mbox{a}}$mek J., Herrlich~H., Strecker~G.\,E.} Abstract and concrete 
categories.~--- New York, NY, USA: John Wiley, 1990. 507~p.
\bibitem{10-kov}
\Au{Маклейн С.} Категории для работающего математика~/ Пер. с~англ.~--- М.: Физматлит, 
2004. 352~с. (\Au{Mac Lane~S.} Categories for the working mathematician.~--- New York, NY, 
USA: Springer, 1978. 317~p.)
\bibitem{11-kov}
   \Au{Семенов П.\,В., Семишкур~Р.\,П., Дяченко~И.\,А.} Концептуальная модель 
реализации технологии <<цифровых двойников>> для предприятий нефтегазового 
комплекса~// Газовая промышленность, 2019. №\,7(787). С.~24--30. EDN: EZXDVJ.

\end{thebibliography}

 }
 }

\end{multicols}

\vspace*{-6pt}

\hfill{\small\textit{Поступила в~редакцию 12.09.22}}

\vspace*{8pt}

%\pagebreak

%\newpage

%\vspace*{-28pt}

\hrule

\vspace*{2pt}

\hrule



\def\tit{ALGEBRAIC SPECIFICATION OF~GRAPH~COMPUTATIONAL~STRUCTURES}


\def\titkol{Algebraic specification of~graph computational structures}


\def\aut{S.\,P.~Kovalyov}

\def\autkol{S.\,P.~Kovalyov}

\titel{\tit}{\aut}{\autkol}{\titkol}

\vspace*{-8pt}


 \noindent
   V.\,A.~Trapeznikov Institute of Control Sciences of the Russian Academy of Sciences, 
65~Profsoyuznaya Str., Moscow 117997, Russian Federation

\def\leftfootline{\small{\textbf{\thepage}
\hfill INFORMATIKA I EE PRIMENENIYA~--- INFORMATICS AND
APPLICATIONS\ \ \ 2024\ \ \ volume~18\ \ \ issue\ 1}
}%
 \def\rightfootline{\small{INFORMATIKA I EE PRIMENENIYA~---
INFORMATICS AND APPLICATIONS\ \ \ 2024\ \ \ volume~18\ \ \ issue\ 1
\hfill \textbf{\thepage}}}

%\vspace*{3pt}


   
   \Abste{The previously proposed generalized approach to algebraic specification 
of distributed systems is developed based on the novel category-theoretic construction called graphalgebra. The 
graphalgebraic specification is based upon a~directed multigraph, the edges of which represent 
computational operations performed in the nodes of the system and the vertices denote the data 
exchange ports between the components. Changing the system architecture during the life cycle 
leads to changes in the graph shape, computation algorithms, and data exchange contents. For 
a~formal description of such changes, a~graph transformation technique for graphalgebras is proposed. 
A~novel category-theoretic construction called flexible graphalgebra is introduced which appeared 
to be closely related to the well-known monad of diagrams. A~functor is presented that produces all 
categories of flexible graphalgebras from their signatures. The theoretical results are illustrated by 
examples from the field of automatic synthesis of neural network architecture by step-by-step 
transformations.}
   
   \KWE{algebraic specification; distributed system; architecture evolution; category theory; 
graphalgebra; monad of diagrams}
   
\DOI{10.14357/19922264240102}{MZAHBS}

%\vspace*{-12pt}

%\Ack
%     \noindent
 

  \begin{multicols}{2}

\renewcommand{\bibname}{\protect\rmfamily References}
%\renewcommand{\bibname}{\large\protect\rm References}

{\small\frenchspacing
 {%\baselineskip=10.8pt
 \addcontentsline{toc}{section}{References}
 \begin{thebibliography}{99} 
\bibitem{1-kov-1}
   \Aue{Kovalyov, S.\,P.} 2004. \textit{Formal'nyy podkhod k~razrabotke programmnykh sistem} 
[Formal approach to software systems development]. Novosibirsk: NSU. 180~p.
\bibitem{2-kov-1}
\Aue{Gurevich, Y.} 1995. Evolving algebras 1993: Lipari guide. \textit{Specification and 
validation methods}. Oxford: Oxford University Press. 9--36.
\bibitem{3-kov-1}
   \Aue{Kovalyov, S.\,P.} 2022. Algebraicheskaya spetsifikatsiya grafovykh vychislitel'nykh 
struktur [Algebraic specification of graph computational structures]. \textit{Informatika i~ee 
Primeneniya~--- Inform. Appl.} 16(1):2--9. doi: 10.14357/ 19922264220101. EDN: GXHPZI.
\bibitem{4-kov-1}
   \Aue{Cai, H., T.~Chen, W.~Zhang, Y.~Yu, and J.~Wang.} 2018. Efficient architecture search 
by network transformation. \textit{32th AAAI Conference ``Artificial Intelligence'' Proceedings.} 
Eds. S.\,A.~McIlraith and K.\,Q.~Weinberger. Palo Alto, CA: AAAI Press. 2787--2794. 
\bibitem{5-kov-1}
   \Aue{Mens, T., J.~Magee, and B.~Rumpe.} 2010. Evolving software architecture descriptions 
of critical systems. \textit{Computer} 43(5):42--48. doi: 10.1109/MC.2010.136.
\bibitem{6-kov-1}
   \Aue{Barr, M., and C.~Wells.} 1990. \textit{Category theory for computing science}. London: 
Prentice Hall. 538~p.
 \bibitem{7-kov-1}
   \Aue{Kovalyov, S.\,P.} 2017. Metody teorii kategoriy v~model'no-orientirovannoy sistemnoy 
inzhenerii [Methods of category theory in model-based systems engineering]. \textit{Informatika 
i~ee Primeneniya~--- Inform. Appl.} 11(3):42--50. doi: 10.14357/19922264170305. EDN: ZGIGGD.
\bibitem{8-kov-1}
   \Aue{Kovalyov, S.\,P.} 2018. Teoriya kategoriy kak ma\-te\-ma\-ti\-che\-skaya pragmatika  
model'no-orientirovannoy sistemnoy inzhenerii [Category theory as a~mathematical pragmatics of 
model-based systems engineering]. \textit{Informatika i~ee Primeneniya~--- Inform. Appl.} 
 12(1):95--104. EDN: \mbox{YTTRFC}.
\bibitem{9-kov-1}
   \Aue{Ad$\acute{\mbox{a}}$mek, J., H.~Herrlich, and G.\,E.~Strecker.} 1990. \textit{Abstract 
and concrete categories}. New York, NY: John Wiley. 507~p.
\bibitem{10-kov-1}
   \Aue{Mac Lane, S.} 1978. \textit{Categories for the working mathematician}. New York, NY: 
Springer. 317~p.
\bibitem{11-kov-1}
   \Aue{Semenov, P.\,V., R.\,P.~Semishkur, and I.\,A.~Dyachenko.} 2019. Kontseptual'naya 
model' realizatsii tekhnologii ``tsifrovykh dvoynikov'' dlya predpriyatiy neftegazovogo kompleksa 
[Conceptual model of digital twin technology implementation for oil and gas industry]. 
\textit{Gazovaya promyshlennost'} [Gas Industry] 7(787):24--30. EDN: EZXDVJ.
   
 \end{thebibliography}

 }
 }

\end{multicols}

\vspace*{-6pt}

\hfill{\small\textit{Received September 12, 2022}} 

\vspace*{-18pt}
     
     \Contrl
     
     \vspace*{-3pt}
   
   \noindent
   \textbf{Kovalyov Sergey P.} (b.\ 1972)~--- Doctor of Science in physics and mathematics, 
leading scientist, V.\,A.~Trapeznikov Institute of Control Sciences of the Russian Academy of Sciences, 
65~Profsoyuznaya Str., Moscow 117997, Russian Federation; \mbox{kovalyov@sibnet.ru}
   


\label{end\stat}

\renewcommand{\bibname}{\protect\rm Литература} 

   