\def\stat{sinitsin}

\def\tit{СУБОПТИМАЛЬНАЯ ФИЛЬТРАЦИЯ В~СТОХАСТИЧЕСКИХ СИСТЕМАХ, НЕ РАЗРЕШЕННЫХ
ОТНОСИТЕЛЬНО ПРОИЗВОДНЫХ, СО СЛУЧАЙНЫМИ ПАРАМЕТРАМИ}

\def\titkol{Субоптимальная фильтрация в~СтС, %стохастических системах, 
не~разрешенных относительно производных, со~случайными параметрами}

\def\aut{И.\,Н.~Синицын$^1$}

\def\autkol{И.\,Н.~Синицын}

\titel{\tit}{\aut}{\autkol}{\titkol}

\index{Синицын И.\,Н.}
\index{Sinitsyn I.\,N.}


%{\renewcommand{\thefootnote}{\fnsymbol{footnote}} \footnotetext[1]
%{Работа выполнялась с~использованием инфраструктуры Центра коллективного пользования 
%<<Высокопроизводительные вычисления и~большие данные>> (ЦКП <<Информатика>> ФИЦ ИУ 
%РАН, Москва).}}


\renewcommand{\thefootnote}{\arabic{footnote}}
\footnotetext[1]{Федеральный исследовательский центр <<Информатика и~управление>> Российской академии наук; Московский авиационный 
институт, \mbox{sinitsin@dol.ru}}

%\mbox{kafedra802@yandex.ru}}

\vspace*{-12pt}





\Abst{Для наблюдаемых гауссовских дифференциальных стохастических систем (СтС), не 
разрешенных относительно производных (НРОП), со случайными параметрами в~виде интегральных канонических представлений (ИКП), приводимых 
к~дифференциальным СтС, разработано методическое обеспечение анализа точности 
субоптимальной фильтрации (СОФ).
Представлен обзор результатов в~об\-ласти аналитического моделирования 
и~СОФ, экстраполяции и~идентификации. Приведены необходимые сведения из корреляционной теории скалярных и~векторных 
многокомпонентных (МК) ИКП. Особое внимание уделено среднеквадратичной 
оптимальной регрессионной линеаризации, в~том числе на основе МК ИКП. 
Представлено методическое обеспечение СОФ 
гауссовской дифференциальной СтС НРОП, приведенной к~дифференциальной, на основе 
метода нормальной аппроксимации (МНА) для условных и~безусловных относительно 
переменных случайных параметров, заданных МК ИКП. Особое внимание уделено 
фильтрационным уравнениям. Иллюстративный пример для одномерной системы, 
нелинейной относительно старшей производной и~линейной относительно параметров 
и~возмущений,  иллюстрирует методику синтеза нормальной (гауссовской) СОФ (НСОФ) на основе 
МК ИКП.}

\KW{метод нормальной аппроксимации (МНА);
регрессионная линеаризация 1-го и~2-го рода; стохастический процесс;
сто\-ха\-сти\-че\-ская сис\-те\-ма, не разрешенная относительно производной (СтС НРОП);
субоптимальная фильт\-ра\-ция (СОФ)} 

\DOI{10.14357/19922264240101}{KUWMKJ}
  
%\vspace*{-6pt}


\vskip 10pt plus 9pt minus 6pt

\thispagestyle{headings}

\begin{multicols}{2}

\label{st\stat}

\section{Введение}


В [1--4] рассмотрены вопросы аналитического моделирования процессов 
в~СтС НРОП. 
Особое внимание в~них уделено нормальным (гауссовским) стохастическим процессам 
(СтП). В~[5] предложены методы нормализации сис\-тем, стохастически НРОП. Теория распределений с~инвариантной мерой в~СтС НРОП 
развита в~[6].

Статья [7] посвящена нелинейным корреляционным методам аналитического 
моделирования (МАМ) процессов в~дифференциальных СтС НРОП. Дан обзор работ 
в~об\-ласти аналитического моделирования СтС НРОП. Приводятся необходимые сведения 
из теории ИКП случайных процессов и~их 
линейных и~нелинейных преобразований.\linebreak Отдельный раздел посвящен СтС НРОП, 
приводимым к~дифференциальным сис\-те\-мам. Представлены основные алгоритмы МАМ 
качества на основе ИКП для типовых СтС НРОП. Приведен \mbox{пример}, иллюстрирующий 
особенности СтС НРОП при нестационарных случайных параметрах, заданных ИКП.

В~[8] рассмотрены вопросы синтеза НСОФ для 
дифференциальных СтС НРОП. Представлены 
уравнения состояния и~наблюдения нелинейных дифференциальных СтС НРОП. 
Синтез НСОФ выполнен при следующих условиях:
\begin{enumerate}[(1)]
\item отсутствуют пуассоновские шумы в~наблюдениях; 
\item коэффициент при гауссовском 
шуме не зависит от состояния.
\end{enumerate}
 Подробно рассмотрен синтез НСОФ при аддитивных 
шумах в~уравнениях состояния и~наблюдения.

В~[9] для нелинейных ин\-тег\-ро-диф\-фе\-рен\-ци\-аль\-ных (ИД) СтС НРОП, 
приводимых к~дифференциальным методом 
сингулярных ядер, разработаны алгоритмы аналитического моделирования\linebreak нормальных 
СтП, при этом нелинейность под интегралом может быть разрывной, а~также синтеза 
НСОФ для он\-лайн-об\-ра\-бот\-ки информации в~ИД СтС. Предложены алгоритмы оценки 
качества НСОФ на основе теории чув\-ст\-ви\-тель\-ности.

В~[3] разработано методическое обеспечение для негладких правых частей уравнений 
СтС НРОП. Рассмотрены вопросы аналитического моделирования нормальных СтП на 
основе нелинейных регрессионных моделей. Особое внимание уделено методам 
гауссовской фильтрации и~экстраполяции. Изучены вопросы условно-оптимальной 
фильтрации и~экстраполяции для СтС НРОП с~параметрическими шумами.

В [10] разработано методическое и~алгоритмическое обеспечение аналитического 
моделирования оценивания и~идентификации для существенно нестационарных 
процессов (например, ударных) в~СтС НРОП. Дан обзор профильных публикаций 
и~изучены основные классы регрессионных уравнений СтС НРОП. Основные результаты: 
\begin{enumerate}[(1)]
\item для общего вида нелинейных СтС НРОП приведены оптимальные алгоритмы 
совместной фильтрации и~распознавания; 
\item для линейных гауссовских СтС НРОП 
получены простые алгоритмы; 
\item для СтС НРОП, линейных относительно состояния~$X_t$ 
и~нелинейных относительно~$Y_t$ наблюдений, получены соответствующие  
алгоритмы; 
\item в~случае~3 методом нормальной аппроксимации получен простой 
алгоритм. Приводится иллюстративный пример скалярной нелинейной гауссовской СтС 
НРОП.
\end{enumerate}

Ставится задача разработки методического обеспечения среднеквадратичной (с.\,к.)\ 
СОФ процессов в~СтС НРОП со случайными переменными 
параметрами, описываемыми ИКП. В~разд.~2 приводятся сведения из теории ИКП, их 
линейных и~нелинейных преобразований, допускающих регрессионную линеаризацию. 
Представлены результаты в~области с.\,к.\ СОФ для СтС НРОП, приводимых 
к~дифференциальным СтС. Приводится иллюстративный пример.


\section{Интегральные канонические представления}

Как известно~[11], для скалярного СтП $X\hm=X(t)$ ИКП называется его и~его 
ковариационной функции интегральное представление через непрерывный скалярный 
белый шум $V\hm=V(\lambda)$ параметра $\lambda \hm\in \Lambda$:
\begin{align}
X(t) &= m_x (t) + \int\limits_\Lambda V(\la) x (t,\la)\,d\la;
\label{e2.1-s}
\\
K_x (t, t') &=\mathrm{M} \lk X(t) \overline{X(t')}\rk = \!\int\limits_\Lambda \!G (\lambda) x 
(t,\lambda) \overline{x(t',\lambda)} \,d\lambda.\notag %\label{e2.2-s}
\end{align}
Здесь  $x(t,\lambda)$~--- координатная функция; $G (\la)$~--- интенсивность $V(\la)$, 
причем $K_v (\la,\la') \hm= G(\la) \delta (\la\hm-\la')$. Формула для координатной 
функции ИКП имеет вид:
\begin{equation*}
x(t,\la) = \fr{1}{G(\la)}\, K_{xv} (t,\la) = \fr{1}{G(\la)} \,\mathrm{M} \lk X^0 
(t) \overline{V(\la)}\rk.
%\label{e2.3-s}
\end{equation*}
Для построения ИКП используются следующие необходимые и~достаточные условия:
\begin{align}
V(\la) &= \int\limits_T\overline{a(t,\la)} X^0(t) \,dt;\notag %\label{e2.4-s}
\\
   x(t,\la)& =  \fr{1}{G(\la)} \int\limits_T a(s,\la)K_x(t,s)\,ds;\label{e2.5-s}\\
   \int\limits_T\overline{a(t,\la)} x(t,\la') \,dt &=\delta (\la-\la');\label{e2.6-s}\\
   \int\limits_\Lambda\overline{a(t',\la)} x(t,\la) \,d\la &=\delta (t-t').\label{e2.7-s}
   \end{align}

\noindent
\textbf{Теорема~2.1.}\ \textit{Для того чтобы скалярный СтП $X\hm=X(t)$ в~области~$T$ допускал ИКП}~(\ref{e2.1-s}), 
\textit{условия}~(\ref{e2.5-s}) \textit{и}~(\ref{e2.6-s}) \textit{необходимы, а~условия}~(\ref{e2.5-s})--(\ref{e2.7-s})~--- \textit{достаточны}.

\smallskip

Аналогично для векторного СтП $X(t) \hm=\lk X_1(t) \cdots X_n(t)\rk^{\mathrm{T}}$ имеем:
  \begin{equation}
  \left.
    \begin{array}{rl}
    X(t) &= m^x (t) + \displaystyle \!\int\limits_\Lambda \!V(\la) x (t,\la) \,d\la;\\[6pt]
    X_h(t) &= m_h^x (t) \displaystyle \!\int\limits_\Lambda \!V(\la) x_h (t,\la) \,d\la\enskip 
\left(h=\overline{1,n}\right).\\
     \end{array}
\!     \right\}\!
     \label{e2.8-s}
\end{equation}
Здесь 
$$
m^x (t) = \lk m_1^x (t) \cdots m_n^x (t)\rk^{\mathrm{T}}.
$$
 При этом ИКП матрицы 
ковариационных функций $K^x (t, t') \hm= \lk K_{hl}^x (t,t')\rk$ имеет вид:
 \begin{multline}
 K_{hl}^x (t,t') = \int\limits_\Lambda G(\la) x_h (t,\la) \overline{x_l(t',\la)} 
\,d\la \\
 \left(h,l=\overline{1,n}\right),\label{e2.9-s}
\end{multline}
где белый шум $V(\la)$  определяется формулой
    $$
    V(\la) = \sum\limits_{h=1}^n \int\limits_T \overline{a_h (t,\la)} X_h^0 (t)\, dt,
    $$
а его интенсивность~$G(\la)$ равна
    $$
    G(\la) = \!\!\sum\limits_{h,l=1}^n \!\int\limits_T \!\int\limits_T\!\int\limits_T \!\overline{a_h (t,\la)} a_l 
(t',\la') K_{hl}^x (t,t') \,dtdt' d\la'.
$$
Для вычисления координатный функций $x_h (t,\la)$ и~функций~$a_h (t,\la)$ 
получим вместо~(\ref{e2.5-s}), (\ref{e2.6-s}) и~(\ref{e2.7-s}) уравнения:

\noindent
\begin{multline}
x_h (t,\la) =\fr{1}{G(\la)} \sum\limits_{l=1}^n \int\limits_T a_l (t',\la) K_{hl}^x 
(t,t') \,dt' \\
 \left(h=\overline{1,n}\right);
\label{e2.10-s}
\end{multline}

\noindent
\begin{equation}
\sum\limits_{h=1}^n \int\limits_T \overline{a_h(t,\la)} x_h (t, \la')\, dt = \delta (\la-\la');
\label{e2.11-s}
\end{equation}
$$
\int\limits_\Lambda \overline{a_l(t',\la)} x_h (t, \la) \,d\la = \delta_{hl} (t-t').
$$

Если параметр $\la$ принимает все возможные значения, принадлежащие нескольким 
областям $\Lambda_1\tr \Lambda_r$, то такое МК ИКП 
векторного СтП $X(t) \hm=\lk X_1(t) \cdots X_n(t)\rk^{\mathrm{T}}$ и~матрицы ковариационных 
функций $K^x (t,t')\hm = \lk K_{hl}^x (t,t')\rk$  будут иметь вид:
   \begin{equation}
   \left.
   \begin{array}{rl}
    X(t) &= m^x (t) +\displaystyle \sum\limits_{\rho =1}^r \int\limits_{\Lambda_\rho} V_\rho (\la) x_\rho 
(t,\la) \,d\la;
\\[6pt]
    X_h(t) &= m_h^x(t) +\displaystyle\sum\limits_{\rho =1}^r \int\limits_{\Lambda_\rho} V_\rho (\la) 
x_{\rho h} (t,\la) \,d\la\\[6pt]
& \hspace*{35mm}\left(h=\overline{1,n}\right);
\end{array}
\right\}
\label{e2.12-s}
\end{equation}

\vspace*{-12pt}

\noindent
\begin{multline}
    K_{hl}(t, t') =\sum\limits_{\rho =1}^r \int\limits_{\Lambda_\rho} G_\rho (\la) x_{\rho 
h} (t,\la)\overline{x_{\rho l} (t',\la)} \,d\la\\ 
\left(h,l=\overline{1,n}\right),\label{e2.13-s}
\end{multline}
где $V_1(\la)\tr V_r(\la)$~--- некоррелированные белые шумы, определяемые 
формулой
\begin{equation*}
V_\rho (\la) = \sum\limits_{h=1}^n  \int\limits_{T} \overline{a_{\rho h} (t,\la)} X_h^0 
(t) dt\enskip (\la\in \Lambda_\rho; \enskip \rho=\overline{1,n}).
%\label{e2.14-s}
\end{equation*}

Интенсивности белых шумов $V_\rho (\la)$ определяются формулой:
  \begin{multline*}
  G_\rho (\la) = {}\\
  {}=
  \sum\limits_{h,l=1}^n   \int\limits_{\Lambda_\rho}\int\limits_{T}\int\limits_T 
\overline{a_{\rho h} (t,\la)}a_{\rho l}(t',\la') K_{hl}^x (t,t') \,dt dt'd\la'\\
\left (\la\in \Lambda_\rho; \enskip \rho=\overline{1,r}\right).
%\label{e2.15-s}
\end{multline*}
Для вычисления координатных функций $x_{\rho h} (t',\la')$ и~функций $a_{\rho h} 
(t,\la)$ используются формулы:
  \begin{multline}
  x_{\rho h} (t,\la) =\fr{1}{G_\rho (\la)} \sum\limits_{l=1}^n \int\limits_T a_{\rho l} 
(t',\la)K_{hl}^x(t,t')\,dt'\\
 \left(\la\in \Lambda_\rho; \enskip 
\rho=\overline{1,r};\enskip h=\overline{1,n}\right);
\label{e2.16-s}
\end{multline}

\vspace*{-12pt}

\noindent
\begin{multline}
\sum\limits_{h=1}^n \int_T \overline{a_{\mu h} (t,\la)}a_{\rho h} 
(t,\la')\,dt=\delta_{\rho \mu} \delta (\la-\la')\\
 \left(\la\in \Lambda_\mu; 
\enskip \la'\in \Lambda_\rho;\enskip \rho,\mu=\overline{1,r}\right);
\label{e2.17-s}
\end{multline}

\noindent
\begin{equation*}
\sum\limits_{\rho=1}^r \!\int\limits_{\Lambda_r} \overline{a_{\rho l} (t',\la)}x_{\rho h} 
(t,\la)\,d\la=\delta_{hl} \delta (t-t')\enskip \left(h,l=\overline{1,n}\right),
%\label{e2.18-s}
\end{equation*}
выражающие необходимые и~достаточные условия представления векторного СтП 
посредством МК ИКП.

\smallskip

\noindent
\textbf{Теорема~2.2.}\ \textit{В условиях}~(\ref{e2.10-s}), (\ref{e2.11-s}) \textit{или}~(\ref{e2.16-s}), (\ref{e2.17-s}) 
\textit{из ИКП}~(\ref{e2.8-s}) \textit{или}~(\ref{e2.12-s}) \textit{вытекает МК 
ИКП матрицы его ковариационных функций}~(\ref{e2.9-s}) \textit{или}~(\ref{e2.13-s}).


\smallskip

\noindent
\textbf{Теорема~2.3.}\ \textit{Если известно ИКП векторного СтП}~(\ref{e2.8-s}), \textit{то вектор математического ожидания 
и~матрица ковариационных функций линейного преобразования $Y_t (t) \hm= \mathrm{A}_t 
X_t$ допускают МК ИКП, определяемые формулами}~\cite{12-s}:
\begin{equation*}
X_p(t) = m_p^x(t) + \sum\limits_{\rho=1}^r \int\limits_{\Lambda_\rho} V_\rho (\la) 
x_{\rho p}(t,\la) \,d\la\enskip \left(p=\overline{1,n}\right);
%\label{e2.19-s}
\end{equation*}

\vspace*{-12pt}

\noindent
\begin{multline*}
Y_p(s) = m_p^y(s) + \sum\limits_{\rho=1}^r \int\limits_{\Lambda_\rho} V_\rho (\la) 
y_{\rho p}(s,\la) \,d\la, 
%\label{e2.20-s}
\\
y_{\rho p}(s,\la)) = \sum\limits_{h=1}^n  A_{ph} x_{\rho h} (t,\la)\enskip 
\left(p=\overline{1,m}\right);
%\label{e2.21-s}
\end{multline*}

\vspace*{-12pt}

\noindent
\begin{multline*}
     K^y (s,s') =\lk K_{pq}^y (s,s')\rk,\\
     K_{pq}^y (s,s') =\displaystyle \sum\limits_{\rho=1}^r \int\limits_{\Lambda_\rho} G_\rho (\la) 
y_{\rho p}(s,\la) \overline{y_{\rho p}(s',\la)}d\la\\ 
\left(p,q=\overline{1,m}\right).
           %\label{e2.22-s}
\end{multline*}

В задачах нелинейной корреляционной теории невырожденные безынерционные 
скалярные и~векторные существенно нелинейные преобразования $Y_t \hm=\varphi_t(X_t)$ 
заменяют оптимальными (в~с.\,к.\ смыс\-ле) линейными регрессионными 
преобразованиями~[12, 13]. Задача эквивалентной регрессионной линеаризации 
детерминированной векторной нелинейной функции $Y\hm=\varphi (X)$ при использовании 
критерия минимума с.\,к.\ ошибки совпадает с~классической задачей 
линейного регрессионного анализа. В этом случае оптимальная линейная 
с.\,к.\ регрессия вектора $Y$ на вектор $X$ определяется формулой:
    \begin{equation}
    m^y(X) =gX+a,
    \label{e2.23-s}
    \end{equation}
    где
    \begin{align}
g&=K^{yx} (K^x)^{-1}; \label{e2.23a-s}\\
a&=m^y-gm^x.\notag
\end{align}
        


Пусть $f(y,x)$~--- совместная плотность случайных векторов~$Y$ и~$X$; $m^x$ и~$K^x$~--- математическое
 ожидание и~ковариационная матрица вектора~$x$,\linebreak\vspace*{-12pt}

\pagebreak

\noindent
$\mathrm{det}\,|K^x |\hm\ne 0$. Формула~(\ref{e2.23a-s}) при этом принимает вид:
    \begin{multline}
    g=K^{yx} (K^x)^{-1} ={}\\
    {}=\!\!\int\limits_{-\infty}^\infty \!\!\lk m^y(x) - m^y\rk \left(x-m^x\right)^T \left(K^x\right)^{-1} f_1 (x) \, dx,
    \!\!\label{e2.24-s}
\end{multline}
где  $f_1(x)$~--- плотность случайного вектора~$X$. Эта формула вместе с~приближенной формулой
   \begin{equation*}
    m^y(X) \approx m^y +g(X-m^x)
    %\label{e2.25-s}
    \end{equation*}
дает статистическую линеаризацию регрессии~$m^y(X)$ по Казакову.

\smallskip

\noindent
\textbf{Теорема~2.4.}\ \textit{Если существуют конечные моменты первого и~второго порядка векторного СтП 
$X\hm=X(t)$, то векторное нелинейное преобразование $Y\hm=\varphi(X)$ допускает линейную 
с.\,к.\ регрессию $Y\hm=Y(t)$ на $X\hm=X(t)$, определяемую по Казакову 
формулами}~(\ref{e2.23-s}) \textit{и}~(\ref{e2.24-s}).


\smallskip

Первый подход к~линеаризации основан на использовании формул для ИКП~$X_t$ 
в~тео\-ре\-ме~2.2 в~случае одной об\-ласти~$\Lambda$, а второй подход~--- для нескольких 
областей~$\Lambda_r$. При этом имеют место сле\-ду\-ющие утверж\-де\-ния.

\smallskip

\noindent
\textbf{Теорема~2.5.}\ \textit{Оптимальная с.\,к.\ линеаризация посредством ИКП $($теорема~$2.2$$)$ 
первого рода определяется следующими формулами}:
    \begin{multline*}
    \varphi_t (X_t) \approx m_t^{(1)y} (X_t) = \varphi_{0t}^{(1)} (m_t^x, K_t^x) + {}\\
    {}+
g_t^{(1)} (m_t^x, K_t^x) \lk \sum\limits_{\rho=1}^r \int\limits_{\Lambda_\rho} V_\rho (\la) 
x_t (\la) \,d\la-m_t^x\rk, %\label{e2.26-s}
\end{multline*}
где
\begin{align*}
\varphi_{0t}^{(1)} \left(m_t^x, K_t^x\right) &= \mathrm{M}_N \lk \varphi (X_t)\rk;\\ 
g_{t}^{(1)} \left(m_t^x, K_t^x\right) &= K_t^{yx}\left(K_t^x\right)^{-1}.
%\label{e2.27-s}
\end{align*}
\textit{Здесь $\mathrm{M}_N$~--- символ математического ожидания для нормального распределения}.

\smallskip

\noindent
\textbf{Теорема~2.6.}\ \textit{Оптимальная с.\,к.\ линеаризация посредством МК ИКП $($теорема~$2.2$$)$ 
второго рода определяется следующими формулами}:
\begin{multline*}
\varphi_t (X_t) \approx m_t^{(2)y} (X_t) = \varphi_{0t}^{(2)} (m_t^x, K_{\rho 
t}^x) +{}\\
{}+\!\sum\limits_{\rho=1}^r \int\limits_{\Lambda_\rho}\! g_{0t}^{(2)}  (m_t^x, K_{\rho t}^x) 
V_\rho (\la) x_{\rho t} \,d\la - g_t^{(2)} \! \left( m_t^x, K_t^x\right) m_t^x,\hspace*{-1.47351pt}
%\label{e2.28-s}
\end{multline*}
где
\begin{align*}
\varphi_{0t}^{(2)}  \left(m_t^x, K_{\rho t}^x\right) &=\mathrm{M}_N \lk \varphi (X_t)\rk;\\ 
g_{0t}^{(2)}  \left(m_t^x, K_{\rho t}^x\right) &=K_{\rho t}^{yx} (K_{\rho t}^x)^{-1}.
%\label{e2.29-s}
\end{align*}


\section{Основные результаты}


В качестве исходной СтС НРОП рассмотрим детерминированную  векторную сис\-те\-му 
уравнений
  \begin{equation}
  \Phi = \Phi \left( t, X_t, \bar X_t, Y_t, \Theta_t, U_t\right) =0\,.\label{e3.1-s}
  \end{equation}
Здесь $X_t$~--- вектор состояния; $\bar X_t\hm= \lk \dot X_t^{\mathrm{T}} \cdots 
(X^{(l)})^{\mathrm{T}}\rk^{\mathrm{T}}$~--- расширенный вектор состояния, состоящий из $l$-го порядка 
производных по времени; $Y_t$~--- вектор наблюдений; $\Theta_t$~--- вектор 
случайных параметров, описываемый МК ИКП (см.\ разд.~2); $\Phi$~--- нелинейная 
функция переменных  $\bar{\bar X}_t \hm= \lk X_t^{\mathrm{T}} \bar X_t^{\mathrm{T}} U_t^{\mathrm{T}}\rk^{\mathrm{T}}$ и~$\Theta_t$, 
допускающая при фиксированном~$\Theta_t$ с.~к.\ регрессионную 
линеаризацию по~$\bar X_t$  и~$U_t$ вида
    \begin{equation}
    \Phi\approx \Phi_0 + \sum\limits_{j=1} k_{\bar X,j}^\Phi X_t^j + k_U^\Phi 
U_t^0.\label{e3.2-s}
\end{equation}
Вектор возмущений~$U_t$ связан с~гауссовским белым шумом~$V_0$ линейным 
уравнением формирующего фильтра:
\begin{equation}
\dot U_t = a_t^U U_t +  a_{0t}^U + b_t^U V_0,
\label{e3.3-s}
\end{equation}
где 
$$
\mathrm{M} V_0 =0\,;\enskip  \mathrm{M} \lk \nu_0 (t) \nu_0(t)^{\mathrm{T}}\rk = \nu_0 \delta (t-\tau).
$$ 
Тогда при фиксированном $\Theta\hm=\Theta_t$ и~при условиях $\mathrm{det}\, k_{Xl}^\Phi 
\hm\ne 0$ и~$ \mathrm{det}\, K_U^\Phi\hm\ne 0$ дифференциальная СтС НРОП~(\ref{e3.1-s}) приводится 
к~дифференциальной СтС следующего вида (\textbf{теорема~3.1}):
\begin{align*}
\dot{\bar X}_{1t} &= \bar X_{2t} \tr {\dot{\bar X}}_{(l-1)t} =\bar 
X_{lt}\,;\\
    \dot{\bar X}_{lt}&= -\left(k_{Xl}^\Phi\right)^{-1} \bar X_t^{(l)}-
\left(k_{Xl}^\Phi\right)^{-1}\left(k_{U}^\Phi\right)^{-1} U_t
%\label{e3.4-s}
\end{align*}
и~(\ref{e3.3-s}). Матрицы коэффициентов $k_{Xl}^\Phi$ и~$k_{U}^\Phi$ неявно зависят от 
первых двух вероятностных моментов $m_t^{\bar X_t}$ и~$K_t^{\bar X_t}$.

Пусть объектовая СтС НРОП допускает приведение к~дифференциальной (теорема 3.1), 
измерительная система вполне наблюдаема, наблюдения влияют на объект, а 
уравнение наблюдения разрешено относительно $Y_t$. Тогда в~качестве исходных 
приведенных дифференциальных уравнений для объекта и~измерительной сис\-те\-мы можно 
принять следующую:
   \begin{multline*}
   \dot X_t =A^{\mbox{п}}\left( X_t, Y_t, m_t^{\bar X_t}, K_t^{\bar X_t},\Theta_t, t\right)= {}\\
   {}=
   a^{\mbox{п}}\left( X_t, Y_t, m_t^{\bar X_t}, K_t^{\bar X_t},\Theta_t, t\right) + {}\\
   {}+
   b^{\mbox{п}}\left( X_t, Y_t,  m_t^{\bar X_t}, K_t^{\bar X_t},\Theta_t, t\right) V_0  (\Theta_t);
%\label{e3.5-s}
\end{multline*}



\noindent
\begin{multline*}
Z_t = \dot Y_t =B\left( X_t, Y_t, \Theta_t,t\right)= a_1 \left( X_t, Y_t, \Theta_t, t\right) + {}\\
{}+ b_1\left( X_t, Y_t, \Theta_t,t\right) V_0 (\Theta_t).
%\label{e3.6-s}
\end{multline*}
Здесь $a^{\mbox{п}}$, $a_1$, $b^{\mbox{п}}$ и~$b_1$~--- известные век\-тор\-но-мат\-рич\-ные 
функции; $V_0$~--- векторный нормальный (гауссовский) белый шум интенсивности 
$\nu_0 \hm= \nu_0 (\Theta_t)$.

В рамках МНА апостериорной плот\-ности 
ве\-ро\-ят\-ности, учитывая, что
гауссовское (нормальное) распределение, ап\-прок\-си\-ми\-ру\-ющее
апостериорное распределение вектора~$\hat{\tilde X}_t$, пол\-ностью определяется
апостериорными математическим ожиданием~$\dot{\hat{\tilde X}}_t$\linebreak
 и~ковариационной мат\-ри\-цей~$R_t$ вектора~$\hat{\tilde X}_t$, при аппроксимации
апостериорного распределения вектора~$\hat{\tilde  X}_t$ нормальным
распределением все математические ожидания будут
пред\-став\-лять собой\linebreak стохастические дифференциальные уравнения,
опре\-де\-ля\-ющие $\dot{\hat{\tilde  X}}_t$ и~$R_t$ (\textbf{теорема~3.2}):

\vspace*{-4pt}

\noindent
\begin{multline}
\dot{\hat{\tilde  X}}_t = f \left(\dot{\hat{\tilde  X}}_t, Y_t,R_t,t\right) +
    h\left(\dot{\hat{\tilde  X}}_t,Y_t, R_t,\Theta_t, t\right)\times{}\\
    {}\times
    \lk\dot Y_t - f^{(1)} 
(\dot{\hat{\tilde  X}}_t,Y_t,
    R_t,\Theta_t,t)\,dt\rk;\label{e3.7-s}
    \end{multline}
    
    \vspace*{-12pt}
    
    \noindent
    \begin{multline}
\hspace*{-5pt}\!\dot R_t=\left\{ 
\vphantom{\left({\dot{\hat{\tilde  X}}}_t,  Y_t,R_t,\Theta,t\right)^{\mathrm{T}}}
f^{(2)}\!\left(\!\dot{\hat{\tilde  X}}_t, Y_t,R_t,\Theta,t\right)\!-\!h\left(\!\dot{\hat{\tilde  X}}_t, Y_t,R_t,\Theta,t\right)\times{}\right.\\
\left.{}\times b_1\nu_0 b_1^{\mathrm{T}} \left(Y_t,\Theta,t\right) 
 h \left({\dot{\hat{\tilde  X}}}_t, 
Y_t,R_t,\Theta,t\right)^{\mathrm{T}}\right\}+{}\\
{}+\sum\limits_{r=1}^{n_y} \rho_r \left( {\dot{\hat{\tilde  
X}}}_t,Y_t, R_t,\Theta,t\right)\times{}\\
{}\times \lk
    \dot Y_r -f_r^{(1)}({\dot{\hat{\tilde  X}}}_t,Y_t, 
R_t,\Theta,t)\rk.\label{e3.8-s}
\end{multline}

\vspace*{-4pt}

\noindent
Здесь

\vspace*{-4pt}

\noindent
\begin{multline*}
f\left(\dot{\hat{\tilde  X}}_t, Y_t,R_t,\Theta,t\right)={}\\
{}= \lk(2\pi)^n \lv     R_t\rv\rk^{-1/2}  \!\int\limits_{-\infty}^\infty\! a(Y_t,x,\Theta,t)\times{}\\
{}\times \exp \left\{ -
    \fr{\left(x^{\mathrm{T}}
    -\dot{\hat{\tilde  X}}_t^{\mathrm{T}}\right) R_t^{-1} \left(x -\dot{\hat{\tilde  X}}_t\right)}{2}\right\}
    dx\,;
    %\label{e3.9-s}
    \end{multline*}
    
    \vspace*{-12pt}
    
    \noindent
  \begin{multline*}
  f^{(1)}\!\left(\!\dot{\hat{\tilde  X}}_t, Y_t,R_t,\Theta,t\!\right)\!=\!\left\{\! f_r^{(1)} \left(\! 
\dot{\hat{\tilde  X}}_t, Y_t, R_t,\Theta, t\right)\right\}={}\\
    {}=\lk (2\pi)^{n_x}\lv
    R_t\rv\rk^{-1/2}\int\limits_{-\infty}^\infty a_1(Y_t,x,\Theta,t)\times{}\\
    {}\times \exp \lf
    -\fr{\left(x^{\mathrm{T}} -\dot{\hat{\tilde  X}}_t^{\mathrm{T}}\right) R_t^{-1} \left(x -\dot{\hat{\tilde  X}}_t\right)}{2}\rf
    dx\,;
    %\label{e3.10-s}
    \end{multline*}
    
    \noindent
\begin{multline*}
h\left(\dot{\hat{\tilde  X}}_t, Y_t,R_t,\Theta,t\right)=\biggl\{ \lk (2\pi)^{n_x}\lv
    R_t\rv\rk^{-1/2}\times{}\\
    {}\times\int\limits_{-\infty}^\infty\!\!
    \lk xa_1\left(Y_t,x,\Theta,t\right)^{\mathrm{T}} + b\nu_0 b_1^{\mathrm{T}} \left(Y_t,x,\Theta,t\right)\rk\times{}\\
{}\times \exp \lf -\fr{\left(x^{\mathrm{T}} -\dot{\hat{\tilde  X}}_t^{\mathrm{T}}\right) R_t^{-1} \left(x -
\dot{\hat{\tilde X}}_t\right)}{2}\rf dx-{}\\
{}-
    \dot{\hat{\tilde  X}}_t f^{(1)}(\dot{\hat{\tilde  X}}_t, 
Y_t,R_t,\Theta,t)^{\mathrm{T}}\biggr\} %\times{}\\
%{}\times 
\left(b_1\nu_0 b_1^{\mathrm{T}}\right)^{-1} (Y_t,\Theta,t);
%\label{e3.11-s}
\end{multline*}

\vspace*{-14pt}

\noindent
\begin{multline*}
f^{(2)}\left(\dot{\hat{\tilde  X}}_t, Y_t,R_t,\Theta,t\right)=\lk (2\pi)^{n_x}\lv
    R_t\rv\rk^{-1/2}\times{}\\
    {}\times 
    \int\limits_{-\infty}^\infty
    \biggl\{  \left(x-\dot{\hat{\tilde  X}}_t\right)a\left(Y_t,x,\Theta,t\right)^{\mathrm{T}} + {}\\
{}+ a \left(Y_t,x,\Theta,t\right) \left(x^{\mathrm{T}}-\dot{\hat{\tilde X}}_t^{\mathrm{T}}\right) +b\nu_0 b_1^{\mathrm{T}} 
\left(Y_t,x,\Theta,t\right)\biggr\}\times{}\\
{}\times \exp \lf -\fr{\left(x^{\mathrm{T}} -\dot{\hat{\tilde  X}}_t^{\mathrm{T}}\right) R_t^{-1}\left(x -
\dot{\hat{\tilde  X}}_t\right)}{2}\rf
    dx;
    %\label{e3.12-s}
    \end{multline*}
    
    \vspace*{-14pt}
    
    \noindent
 \begin{multline}
 \rho_r\left(\dot{\hat{\tilde  X}}_t,Y_t, R_t,\Theta,t\right)=\lk (2\pi)^{n_x}\lv
    R_t\rv\rk^{-1/2}\times{}\\
    {}\times \int\limits_{-\infty}^\infty
    \biggl\{  \left(x-\dot{\hat{\tilde  X}}_t\right)\left(x^{\mathrm{T}}-\dot{\hat{\tilde  X}}_t^{\mathrm{T}}\right) a_r 
(Y_t,x,\Theta,t)+ {}\\
{}+ \left(x-\dot{\hat{\tilde  X}}_t\right) b_r\left(Y_t,x,\Theta,t\right)^{\mathrm{T}} \left(x^{\mathrm{T}}-\dot{\hat{\tilde  
X}}_t^{\mathrm{T}}\right)+{}\\
{}+b_r \left(Y_t,x,\Theta,t\right) \left(x^{\mathrm{T}}-\dot{\hat{\tilde  X}}_t^{\mathrm{T}}\right)\biggr\} \times{}\\
{}\times \exp \lf -
\fr{\left(x^{\mathrm{T}} -\dot{\hat{\tilde  X}}_t^{\mathrm{T}}\right) R_t^{-1} \left(x -
\dot{\hat{\tilde  X}}_t\right)}{2}\rf dx\\
 \left(r=\overline{1,n_y}\right),\label{e3.13-s}
\end{multline}

\vspace*{-4pt}

\noindent
где $a_r$~---  $r$-й элемент мат\-ри\-цы-стро\-ки~$(a_1^{\mathrm{T}}\hm-{\hat a}_1^{\mathrm{T}}) (b_1\nu_0 b_1^{\mathrm{T}})^{-1}$; $b_{kr}$~--- элемент
$k$-й строки и~$r$-го столбца матрицы
$(b_1\nu_0 b_1^{\mathrm{T}})^{-1}$. Тогда, обозначив через
$b_r$ $r$-й столбец матрицы
$b\nu_0 b_1^{\mathrm{T}}(b_1\nu_0 b_1^{\mathrm{T}})^{-1}$, имеем $b_r \hm= [ b_{1r}\cdots
b_{pr}]^{\mathrm{T}}$ $(r\hm=\overline{1, n_1})$.

Число уравнений МНА одномерного апостериорного распределения
определяется по формуле:
    $$
    Q_{\mathrm{МНА}} = n_x + \fr{n_x (n_x+1)}{2} = \fr{n_x(n_x+3)}{2}\,.
    $$
    
    \vspace*{-12pt}
    
    \pagebreak

За начальные значения $\dot{\hat{\tilde X}}_t$ и~$R_t$  при интегрировании
уравнений~(\ref{e3.7-s}) и~(\ref{e3.8-s}), естественно, следует принять
условные математическое ожидание и~ковариационную матрицу величины
$\hat{\tilde X}_0$ относительно~$Y_0$:

\noindent
\begin{align*}
    \dot{\hat{\tilde  X}}_0 &= \mathrm{M}_N\lk \hat{\tilde  X}_0 \mid 
Y_0\rk;\\ 
 R_0 &= \mathrm{M} \lk (\hat{\tilde  X}_0 -\dot{\hat{\tilde  X}}_0) 
\left(\hat{\tilde  X}_0^{\mathrm{T}} -\dot{\hat{\tilde  X}}_0^{\mathrm{T}}\right)\mid
    Y_0\rk.
%    \label{e3.14-s}
    \end{align*}
    
    \vspace*{-3pt}
    
    \noindent
 Если нет
информации об условном распределении $\hat{\tilde X}_0$ относительно~$Y_0$, то
начальные условия можно взять в~виде:

\noindent
\begin{align*}
\dot{\hat{\tilde  X}}_0 &= \mathrm{M}_N 
\hat{\tilde  X}_0;\\
  R_0&= \mathrm{M}_N(\hat{\tilde  X}_0-\mathrm{M}_N \hat{\tilde 
X}_0) \left(\hat{\tilde  X}_0^{\mathrm{T}} - \mathrm{M}_N \hat{\tilde  X}_0^{\mathrm{T}}\right).
\end{align*}

\vspace*{-3pt}

\noindent
 Если
же и~об этих величинах нет никакой информации, то начальные
значения $\dot{\hat{\tilde  X}}_t$ и~$R_t$ придется задавать произвольно.

Из формулы~(\ref{e3.13-s}) видно, что если функция~$a_1$ линейна
относительно~$\hat{\tilde X}_t$, а~функция~$b$ не зависит от~$\hat{\tilde  X}_t$, то при
нормальной аппроксимации апостериорного распределения все матрицы~$\rho_r$ равны нулю, вследствие чего уравнение~(\ref{e3.8-s}) не содержит~$\dot Y_t$.

Наконец, применяя регрессионную линеаризацию к~уравнениям~(\ref{e3.7-s}) и~(\ref{e3.8-s}) 
посредством  МК ИКП, согласно теоремам~2.5 и~2.6, для условных характеристик, 
придем к~следующим уравнениям аналитического моделирования параметров СОФ (\textbf{теорема~3.3}):
\begin{equation*}
\hat{\tilde X}_t = \hat{\tilde{\tilde X}}_t + \delta \hat{\tilde 
X}_t;\enskip \tilde R_t = \tilde{\tilde R}_t + \delta \tilde R_t; %\label{e3.15-s}
\end{equation*}
\begin{equation}
\fr{d}{dt}\,\hat{\tilde{\tilde X}}_t = f_0 + h_0 \dot Y_t -(hf^{(1)})_0;
\label{e3.16-s}
\end{equation}

\vspace*{-12pt}

\noindent
\begin{multline}
\fr{d}{dt}\, \tilde{\tilde R}_t = f_0^{(2)} - h_0 \alp_0 h_0^{\mathrm{T}} + 
\sum\limits_{i=1}^{n_y} \lk \rho_{i0} \dot Y_t - \left(\rho_i f_i^{(1)}\right)_0\rk \\
 \left(\alpha= 
b_1\nu_0 b_1^{\mathrm{T}}\right);
\label{e3.17-s}
\end{multline}

\vspace*{-12pt}

\noindent
\begin{multline}
\fr{d}{dt}\, \delta \hat{\tilde X}_t = f_{1,\delta X} \delta \hat{\tilde 
X}_t + f_{1,\delta R} \delta \tilde R_t + f_{1,\Theta} \Theta_t 
+{}\\
{}+\left(h_{1,\delta X} \delta \hat{\tilde X}_t + h_{1,\delta R} \delta \tilde R_t 
+ h_{1,\Theta} \Theta_t\right) \dot Y_t -{}\\
{}-\left[ \left( hf^{(1)}\right)_{1,\delta X} \delta \hat{\tilde X}_t +\left( 
hf^{(1)}\right)_{1,\delta R} \delta \tilde R_t +{}\right.\\
\left.{}+\left( 
hf^{(1)}\right)_{1,\Theta} \Theta_t\right];
\label{e3.18-s}
\end{multline}

\vspace*{-12pt}

\noindent
\begin{multline*}
\fr{d}{dt} \,\delta \tilde R_t =f_{1,\delta X}^{(2)} \delta \hat{\tilde 
X}_t + f_{1,\delta R}^{(2)} \delta \tilde R_t + f_{1,\Theta}^{(2)} \Theta_t -{}\\
{}-\lk 
\left(h\alpha h\right)_{1,\delta X} \delta \hat{\tilde X}_t + \left(h\alpha h\right)_{1,\delta R} \delta 
\tilde R_t + \left(h\alpha h\right)_{1,\Theta} \Theta_t\rk +{}\hspace*{-2pt}
\end{multline*}

\noindent
\begin{multline}
{}+\sum\limits_{i=1}^{n_y} \left\{
\vphantom{\left( \rho_i f_i^{(1)}\right)_{1,\Theta}}
\left(  \rho_{i,1,\delta X}\delta \hat{\tilde X}_t 
+\rho_{i,1,\delta R} \delta \tilde R_t +\rho_{i,1,\Theta} \Theta_t\right)\dot 
Y_t-{}\right.\\
{}-\left[ \left( \rho_i f_i^{(1)}\right)_{1,\delta X} \delta 
\hat{\tilde X}_t +\left( \rho_i f_i^{(1)}\right)_{1,\delta R} \delta \tilde R_t 
+{}\right.\\
\left.\left.{}+\left( \rho_i f_i^{(1)}\right)_{1,\Theta} \Theta_t\right]\right\}.\label{e3.19-s}
\end{multline}
Здесь при регрессионной линеаризации первого рода аргументами функций $f$, 
$f^{(1)}$, $f^{(2)}$, $h$ и~$f_l^{(1)}$ служат переменные $ \hat{\tilde{\tilde X}}_t$, 
$Y_t$,  $\tilde{\tilde R}_t, m_t^\Theta$ и~$K_t^\Theta$, а для линеаризации второго 
рода -- $ \hat{\tilde{\tilde X}}_t$, $Y_t$,  $\tilde{\tilde R}_t, m_t^{\Theta_r}$ 
и~$K_t^{\Theta_r}$.

Анализ уравнений теоремы~3.3 позволяет сделать следующие выводы:
\begin{enumerate}[(1)]
\item фильтрационные уравнения~(\ref{e3.16-s}) и~(\ref{e3.17-s}) нелинейны относительно безусловных 
характеристик $\hat{\tilde{\tilde X}}_t$ и~$\tilde{\tilde R}_t$, определяемых 
уравнением Риккати, а уравнения точности НСОФ~(\ref{e3.18-s}) и~(\ref{e3.19-s}) линейны 
относительно  $\delta \hat{\tilde X}_t, \delta \tilde R_t$;

\item если приравнять правые части уравнений~(\ref{e3.16-s}) и~(\ref{e3.17-s}) нулю, найдем 
стационарные решения, а~уравнения~(\ref{e3.18-s}) и~(\ref{e3.19-s}) позволят \mbox{найти} как условия 
существования и~устой\-чи\-вости\linebreak стационарных решений, так и~оценить точ\-ность НСОФ. 
Для определения весовых функций физически ре\-а\-ли\-зу\-емых линейных уравнений~(\ref{e3.17-s}) 
и~(\ref{e3.18-s}) следует образовать \mbox{со\-от\-вет\-ст\-ву\-ющие} коэффициенты при переменных 
количествах;

\item в~прикладных задачах при оценке точности НСОФ в~условиях экстремальных 
значений переменных случайных параметров суждение о~точности НСОФ делается на 
основе анализа уравнений Риккати~(\ref{e3.17-s}) и~(\ref{e3.18-s}) путем чис\-лен\-но\-го решения сис\-те\-мы 
взаимосвязанных детерминированных уравнений.
\end{enumerate}

\vspace*{-12pt}

\section{Пример}

 {Рассмотрим наблюдаемую дифференциальную систему вида
\begin{align}
\varphi &=\varphi_1 (\dot X_{1t}) + \gamma X_{1t} + X_{2t} =0;\label{e4.1-s}\\
\dot X_{2t} &= a^U X_{2t} + V_1;\label{e4.2-s}\\
Z_t&=\dot Y_t = c_1 X_{1t} + V_2.\label{e4.3-s}
\end{align}
Здесь $[X_{1t} X_{2t}]^{\mathrm{T}}\hm=X_t$~--- вектор состояния; $V_1$ и~$V_2$~--- независимые 
скалярные белые шумы с~интенсивностями~$\nu_1$ и~$\nu_2$; $\varphi_1 \hm=\varphi_1 (\dot 
X_{1t})$~--- функция, допускающая регрессионную линеаризацию}
 \begin{equation}
 \varphi_1 \approx \varphi_{10} + k_1 \dot X_{1t}^0,\label{e4.4-s}
 \end{equation}
 
 \pagebreak
 
 \noindent
 {где $\varphi_{10} \hm=\varphi_{10} (m_{1t}^{\dot X}, D_{1t}^{\dot X})$; $k_1\hm= 
k_1(m_{1t}^{\dot X}, D_{1t}^{\dot X})$. При  $k_1\hm\ne 0$ уравнение~(\ref{e4.1-s}) с~учетом~(\ref{e4.4-s}) приводится к~\mbox{виду}}:
    \begin{equation}
    \dot X_{1t} = -\left(\varphi_{10} + \gamma X_{1t} - X_{2t}\right) k_1^{-1}. 
    \label{e4.5-s}
    \end{equation}
  {Обозначая}
    \begin{alignat*}{2}
    a_{0t} &= \lk \begin{array}{c}
    -\varphi_{10} k_1^{-1}\\
    0
    \end{array}\rk;&\enskip
    a_{1t} &= \lk \begin{array}{cc}
    \gamma&k_1^{-1}\\
    c_1&0
    \end{array}\rk;
   \\
    b_{1t} &= [c_1\, 0];&\enskip \tilde R &= \left[\begin{array}{cc}
    R_{11}& R_{12}\\
    R_{12}& R_{22}
    \end{array}\right],
 %   \label{e4.6-s}
    \end{alignat*}
 {запишем уравнения~(\ref{e4.5-s}), (\ref{e4.2-s}) и~(\ref{e4.3-s}) в~форме}
     \begin{equation}
     \left.
     \begin{array}{rl}
     \dot X_t &= a_{1t} X_t + a_{0t} + V_1; %\label{e4.7-s}
\\[6pt]
Z_t &=\dot Y_t= b_{1t} X_t + V_2.
\end{array}
\right\}
\label{e4.8-s}
\end{equation}
 {Для~(\ref{e4.8-s}) %и~(\ref{e4.8-s}) 
уравнения НСОФ~[13] имеют сле\-ду\-ющий вид}:
\begin{align}
\dot{\hat{\tilde X}}_t& = a_{1t} {\hat{\tilde X}}_t + a_{0t} +\tilde 
\beta_t \left(Z_t - b_{1t} {\hat{\tilde X}}_t\right);\notag
%\label{e4.9-s}
\\
  \tilde \beta_t &=\tilde R_t b_{1t}^{\mathrm{T}} \nu_{2t}^{-1};\notag
  %\label{e4.10-s}
  \\
  {\dot{\tilde R}}_t &= a_{1t}\tilde R_t + \tilde R_t a_{1t}^{\mathrm{T}} -\tilde R_t 
b_{1t}^{\mathrm{T}} \nu_{2t}^{-1} b_{1t} \tilde R_t +{}\notag\\
&\hspace*{35mm}{}+ (\nu_{10}+\nu_{11} \Theta_t).
\label{e4.11-s}
\end{align}
 {Здесь}
 
 \vspace*{-4pt}
 
 \noindent
  \begin{align*}
  \Theta_t& = m_t^\Theta + \int\limits_\Lambda V(\la) x^\Theta (\lambda,t) 
\,d\lambda\,; %\label{e4.12-s}
\\
K_t^\Theta (t_1, t_2) &=\int\limits_\Lambda G(\la) x^\Theta (\la,t_1) x^\Theta 
(\la,t_2) \,d\la\,, %\label{e4.13-s}
\end{align*}
 {где}
 
 \vspace*{-4pt}
 
 \noindent
    \begin{multline*}
    x(\lambda,t) = q_1 (t) q_2^{-1} (\lambda) {\bf 1} (t-\la),\\
     q_1 (t) = e^{-\alpha t},\enskip q_2(t) = q_1 (t) \int\limits_{t_0}^t q_1^{-2} (\lambda) \,d\lambda\,;
    \end{multline*}
    
    \vspace*{-3pt}
    
    \noindent
 \begin{equation*}
    K^\Theta (t_1, t_2) =\begin{cases}
    q_1(t_2) q_2 (t_1) & \mbox{при}\  t_1<t_2;\\
    q_1(t_1) q_2 (t_2) & \mbox{при}\  t_1>t_2.\\
\end{cases}
   % \label{e4.14-s}
    \end{equation*}

 {При $\nu_{11}^* \ll \nu_{10}^*$, $a_{0t}\hm = a_0^*$, $ a_{1t}\hm=a_1^*, b_{1t}\hm= 
b_1^*$, $\nu_{10} \hm= \nu_{10}^*$, $ m_t^\Theta \hm= m_*^\Theta$ и~$\nu_2 \hm= \nu_2^*$
уравнение~(\ref{e4.11-s}) имеет стационарное решение:
 \begin{equation*}
 a_1^* R^* + R^* a_1^{*\mathrm{T}} - R^* b_1^{*\mathrm{T}} 
 \left(\nu_2^*\right)^{-1} R^* + 
\nu_{10}^*=0.
%\label{e4.15-s}
\end{equation*}
При этом для оценки точности СОФ после линеаризации относительно $\delta R_t \hm= 
\tilde R_t \hm- R^*$ получаем линейное уравнение с~постоянными коэффициентами:
 \begin{multline*}
 \delta \dot R_t =\lk a_1^* + R^* b_1^{*\mathrm{T}} \left(\nu_2^*\right)^{-1} b_1^*\rk \delta 
R_t +{}\\
{}+ \delta R_t \lk b_1^{*\mathrm{T}} \left(\nu_2^*\right)^{-1} b_1^* R^*\rk +\nu_{11}^* 
\Theta_t.
%\label{e4.16-s}
\end{multline*}
Отсюда сразу вытекают уравнения для $m^{\delta R}\hm =\mathrm{M}\, \delta R_t$, а~также 
ковариационной матрицы~$K_t^{\delta R}$.
}


\section{ Заключение}

\vspace*{-12pt}

Для наблюдаемых гауссовских дифферен\-ци\-альных СтС НРОП со случайными па\-ра\-мет\-ра\-ми 
в~виде ИКП, приводимых к~дифференциальным\linebreak СтС, разработано методическое 
обеспечение анализа точ\-ности СОФ, основанное на МНА при фиксированном векторе параметров для условных 
\mbox{вероятностных} характеристик и~регрессионной линеаризации посредством ИКП 
безусловных характеристик.

Результаты могут быть использованы для наблюдаемых негауссовских СтС НРОП, 
приводимых к~негауссовским дифференциальным СтС.
Дальнейшее развитие методического обеспечения для негауссовских СтС НРОП связано с~использованием методов моментов, квазимоментов и~ортогональных разложений 
одно- и~многомерных плотностей для условных вероятностных характеристик.

\vspace*{-12pt}

{\small\frenchspacing
 { %\baselineskip=10.6pt
 %\addcontentsline{toc}{section}{References}
 \begin{thebibliography}{99}
 
 \vspace*{-12pt}
 
\bibitem{1-s}
\Au{Синицын И.\,Н.}
Аналитическое моделирование широкополосных процессов в~стохастических системах, 
не разрешенных относительно производных~// Информатика и~её применения, 2017. 
Т.~11. Вып.~1. С.~3--10. doi: 10.14357/19922264170101. EDN: YOCMVL.

\bibitem{2-s}
\Au{Синицын И.\,Н.}
Параметрическое аналитическое моделирование процессов в~стохастических  
системах, не разрешенных относительно производных~// Системы и~средства 
информатики, 2017. Т.~27. №\,1. С.~21--45. doi: 10.14357/08696527170102. EDN: 
YODCZL.

\bibitem{3-s}
\Au{Sinitsyn I.\,N.}
Analytical modeling and estimation of normal processes defined by stochastic 
differential equations with unsolved derivatives~// J.~Mathematics Statistics 
Research, 2021. Vol.~3. Iss.~1. Art.~139. 7~p. doi: 10.36266/ \mbox{JMSR}/139.

\bibitem{4-s}
\Au{Синицын И.\,Н.}
Аналитическое моделирование и~оценивание нестационарных нормальных процессов 
в~стохастических системах, не разрешенных относительно производных~// Сис\-те\-мы 
и~средства информатики, 2022. Т.~32. №\,2. С.~58--71. doi: 10.14357/ 08696527220206. 
EDN: YMGERJ.


\bibitem{5-s}
\Au{Синицын И.\,Н.}
Нормализация систем, стохастически не разрешенных относительно производных~// 
Информатика и~её применения, 2022. Т.~16. Вып.~1. С.~32--38. doi: 
10.14357/19922264220105. EDN: \mbox{LDFJJB}.

\bibitem{6-s}
\Au{Синицын И.\,Н.}
Аналитическое моделирование распределений с~инвариантной мерой в~стохастических 
системах, не разрешенных относительно производных~// Информатика и~её 
применения, 2023. Т.~17. Вып.~1. С.~2--10. doi: 10.14357/19922264230101. EDN: 
QWXVXC.

\bibitem{7-s}
\Au{Синицын И.\,Н.}
Аналитическое моделирование стохастических  сис\-тем, не разрешенных относительно\linebreak\vspace*{-12pt}

\pagebreak

\noindent 
производных, со случайными параметрами~// Сис\-те\-мы и~средства информатики, 2024  (в печати). 
Т.~34. №\,1.

\bibitem{8-s}
\Au{Синицын И.\,Н.}
Нормальные субоптимальные фильт\-ры для дифференциальных стохастических систем, не 
разрешенных относительно производных~// Информатика и~её применения, 2021. 
Т.~15. Вып.~1. С.~3--10. doi:10.14357/19922264210101. EDN: UPEHRI. 

\bibitem{9-s}
\Au{Синицын И.\,Н.}
Аналитическое моделирование и~фильтрация нормальных процессов 
в~интегродифференциальных стохастических системах, не разрешенных относительно 
производных~// Системы и~средства информатики, 2021. Т.~31. №\,1. С.~37--56. doi: 
10.14357/08696527210104. EDN: PLYOSF.


\bibitem{10-s}
\Au{Синицын И.\,Н.}
Совместная фильтрация и~распознавание нормальных процессов в~стохастических 
сис\-те\-мах, не разрешенных относительно производных~// Информатика и~её 
применения, 2022. Т.~16. Вып.~2. С.~85--93. doi: 10.14357/19922264220211. EDN: 
SMJCBB.

\bibitem{11-s}
\Au{Синицын И.\,Н.}
Канонические представления случайных функций. Теория и~применения.~--- М.: ТОРУС 
ПРЕСС, 2023. 816~с.

\bibitem{12-s}
\Au{Синицын И.\,Н.}
Фильтры Калмана и~Пугачёва.~--- 2-е изд.~--- М.: Логос, 2007. 
776~с.

\bibitem{13-s}
\Au{Пугачёв В.\,С.}
Теория вероятностей и~математическая статистика.~--- 2-е изд.~--- 
М.: Физматлит, 2002. 496~с.
\end{thebibliography}

 }
 }

\end{multicols}

\vspace*{-6pt}

\hfill{\small\textit{Поступила в~редакцию 26.09.23}}

\vspace*{6pt}

%\pagebreak

%\newpage

%\vspace*{-28pt}

\hrule

\vspace*{2pt}

\hrule

\vspace*{-2pt}

\def\tit{SUBOPTIMAL FILTERING IN~STOCHASTIC SYSTEMS WITH~RANDOM PARAMETERS AND~UNSOLVED DERIVATIVES}


\def\titkol{Suboptimal filtering in~stochastic systems with~random parameters and unsolved derivatives}


\def\aut{I.\,N.~Sinitsyn$^{1,2}$}

\def\autkol{I.\,N.~Sinitsyn}

\titel{\tit}{\aut}{\autkol}{\titkol}

\vspace*{-10pt}


\noindent
$^1$Federal Research Center ``Computer Science and Control'' of the Russian Academy of Sciences, 44-2~Vavilov\linebreak
$\hphantom{^1}$Str., Moscow 119333, Russian Federation

\noindent
$^2$Moscow State Aviation Institute (National Research University), 4~Volokolamskoe Shosse, Moscow 125933,\linebreak
$\hphantom{^1}$Russian Federation

\def\leftfootline{\small{\textbf{\thepage}
\hfill INFORMATIKA I EE PRIMENENIYA~--- INFORMATICS AND
APPLICATIONS\ \ \ 2024\ \ \ volume~18\ \ \ issue\ 1}
}%
 \def\rightfootline{\small{INFORMATIKA I EE PRIMENENIYA~---
INFORMATICS AND APPLICATIONS\ \ \ 2024\ \ \ volume~18\ \ \ issue\ 1
\hfill \textbf{\thepage}}}

\vspace*{3pt}





\Abste{For observable Gaussian differential stochastic systems (StS) with random parameters in the form of integral 
canonical expansions (ICE) and StS with unsolved derivatives (USD), methodological 
support for synthesis of suboptimal filters is presented. Survey in fields of analytical modeling
 and suboptimal filtering (SOF), extrapolation, and identification is presented. Necessary information concerning multicomponent (MC) ICE 
 is given. Special attention is paid to mean square regressive linearization including MC ICE. Basic results in normal SOF (NSOF)
  are presented for StS USD reducible to differential StS. Stationary and nonstationary NSOF are considered. 
  An illustrative example for scalar StS USD reducible to differential is given. For future, SOF generalization methods of moments, 
  quasi-moments, and one- and multidimensional densities of orthogonal expansions are recommended.}


\KWE{normal approximation method (NAM); regression linearization; stochastic process; stochastic systems with unsolved derivatives (StS USD);
 suboptimal filtering (SOF)}  
 
\DOI{10.14357/19922264240101}{KUWMKJ}


\vspace*{6pt}

%\Ack
%\vspace*{-3pt}
    % \noindent
    


  \begin{multicols}{2}

\renewcommand{\bibname}{\protect\rmfamily References}
%\renewcommand{\bibname}{\large\protect\rm References}

{\small\frenchspacing
 {%\baselineskip=10.8pt
 \addcontentsline{toc}{section}{References}
 \begin{thebibliography}{99}  

\bibitem{1-s-1} 
\Aue{Sinitsyn, I.\,N.} 2017. 
Analiticheskoe modelirovanie shirokopolosnykh protsessov v~stokhasticheskikh sistemakh, ne razreshennykh otnositel'no proizvodnykh  
[Analytical modeling of wide band processes in stochastic systems with unsolved derivatives]. 
\textit{Informatika i~ee Primeneniya~--- Inform. Appl.} 11(1):3--10.
doi: 10.14357/19922264170101. EDN: YOCMVL.


\bibitem{2-s-1} 
\Aue{Sinitsyn, I.\,N.}  2017.
Parametricheskoe analiticheskoe modelirovanie protsessov v~stokhasticheskikh sistemakh, ne razreshennykh otnositel'no proizvodnykh
 [Parametric analytical modeling of wide band processes in stochastic systems with unsolved derivatives]. \textit{Sistemy i~Sredstva
Informatiki~--- Systems and Means of Informatics} 27(1):21--45. doi: 10.14357/08696527170102. EDN: YODCZL.

\bibitem{3-s-1} 
\Aue{Sinitsyn, I.\,N.} 2021. 
Analytical modeling and estimation of normal processes defined by stochastic differential equations with unsolved derivatives. 
\textit{J. Mathematics Statistics Research} 3(1):139. 7~p. doi: 10.36266/JMSR/139.

\bibitem{4-s-1} 
\Aue{Sinitsyn, I.\,N.} 2022. 
Analiticheskoe modelirovanie i~otsenivanie nestatsionarnykh normal'nykh pro\-tses\-sov v~sto\-kha\-sti\-che\-skikh sis\-te\-makh, ne raz\-re\-shen\-nykh 
ot\-no\-si\-tel'\-no pro\-iz\-vod\-nykh 
[Analytical modeling and estimation\linebreak of nonstationary normal processors with unsolved derivatives].
 \textit{Sistemy i~Sredstva Informatiki~--- Systems and Means of Informatics} 32(2):58--71. doi: 10.14357/ 08696527220206. EDN: YMGERJ.

\bibitem{5-s-1} 
\Aue{Sinitsyn, I.\,N.} 2022.
Normalizatsiya sistem, sto\-kha\-sti\-che\-ski ne razreshennykh otnositel'no proizvodnykh 
[Normalization of systems with stochastically unsolved derivatives]. \textit{Informatika i~ee Primeneniya~--- Inform. Appl.} 16(1):32--38.
doi: 10.14357/19922264220105. EDN: LDFJJB.

\bibitem{6-s-1} 
\Aue{Sinitsyn, I.\,N.} 2023. 
Analiticheskoe modelirovanie raspredeleniy s~invariantnoy meroy v~stokhasticheskikh sistemakh, ne razreshennykh otnositel'no proizvodnykh 
[Analytical modeling of distributions with invariant measure in stochastic systems with unsolved derivatives]. 
\textit{Informatika i~ee Primeneniya~--- Inform. Appl.}  17(1):2--10. doi: 10.14357/19922264230101. EDN: QWXVXC.

\bibitem{7-s-1} 
\Aue{Sinitsyn, I.\,N.} 2024 (in press). 
Analiticheskoe mo\-de\-li\-ro\-va\-nie stokhasticheskikh  sistem, ne razreshennykh ot\-no\-si\-tel'\-no proizvodnykh, so sluchaynymi parametrami 
[Analytical modeling of stochastic systems with random parameters and unsolved derivatives]. \textit{Sistemy i~Sredstva Informatiki~--- 
Systems and Means of Informatics} 34(1). 

\bibitem{8-s-1} 
\Aue{Sinitsyn, I.\,N.} 2021. 
Normal'nye suboptimal'nye fil'try dlya differentsial'nykh stokhasticheskikh sistem, ne raz\-re\-shen\-nykh otnositel'no proizvodnykh 
[Normal suboptimal filtering for differential stochastic systems with unsolved derivatives]. 
\textit{Informatika i~ee Primeneniya~--- Inform. Appl.} 15(1):3--10. doi: 10.14357/19922264210101. EDN: \mbox{UPEHRI}. 

\bibitem{9-s-1} 
\Aue{Sinitsyn, I.\,N.} 2021. 
Analiticheskoe modelirovanie i~fil'tratsiya normal'nykh protsessov v~in\-teg\-ro\-dif\-fe\-ren\-ts\-ial'\-nykh sto\-kha\-sti\-che\-skikh sis\-te\-makh, 
ne raz\-re\-shen\-nykh ot\-no\-si\-tel'\-no pro\-iz\-vod\-nykh [Analytical modeling and filtering for integrodifferential systems with unsolved derivatives]. 
\textit{Sistemy i~Sredstva Informatiki~--- Systems and Means of Informatics} 31(1):37--56. doi: 10.14357/ 08696527210104. EDN: PLYOSF.

\bibitem{10-s-1} 
\Aue{Sinitsyn, I.\,N.} 2022.
Sovmestnaya fil'tratsiya i~ras\-po\-zna\-va\-nie normal'nykh protsessov v~stokhasticheskikh sis\-te\-makh, ne razreshennykh otnositel'no proizvodnykh 
[Joint filtration and recognition of normal prosesses in stochastic systems with unsolved derivatives]. 
\textit{Informatika i~ee Primeneniya~--- Inform. Appl.} 16(2):85--93. doi: 10.14357/ 19922264220211. EDN: SMJCBB.

\bibitem{11-s-1} 
\Aue{Sinitsyn, I.\,N.} 2023.
\textit{Kanonicheskie predstavleniya slu\-chay\-nykh funktsiy. Teoriya i~primeneniya} [Canonical expansion of random functions. Theory and application].
 Moscow: TORUS PRESS. 816~p.

\bibitem{12-s-1} 
\Aue{Sinitsyn, I.\,N.} 2007. 
\textit{Fil'try Kalmana i~Pugacheva} [Kalman and Pugachev filters]. 2nd ed. Moscow: Logos. 776~p.

\bibitem{13-s-1} 
\Aue{Pugachev, V.\,S.} 2002. 
\textit{Teoriya veroyatnostey i~ma\-te\-ma\-ti\-che\-skaya sta\-ti\-sti\-ka} [Probability theory and mathematical statistics]. 2nd ed. Moscow: Fizmatlit. 496~p.

 \end{thebibliography}

 }
 }

\end{multicols}

\vspace*{-6pt}

\hfill{\small\textit{Received September 26, 2023}} 

\vspace*{-18pt}
     
     \Contrl
     
     \vspace*{-3pt}

\noindent
\textbf{Sinitsyn Igor N.} (b.\ 1940)~--- 
Doctor of Science in technology, professor, Honored scientist of RF, principal scientist, Federal Research 
Center ``Computer Science and Control'' of the Russian Academy of Sciences, 44-2~Vavilov Str., Moscow 119333, Russian Federation; 
professor, Moscow State Aviation Institute (National Research University), 4~Volokolamskoe Shosse, Moscow 125933, Russian Federation; 
\mbox{sinitsin@dol.ru}

\label{end\stat}

\renewcommand{\bibname}{\protect\rm Литература} 
      