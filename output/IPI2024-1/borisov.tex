\newcommand {\ebd}{\triangleq}
\newcommand{\me}[2]{\mathsf{E}_{ #1 }\left\{ \mathop{#2} \right\} }


\def\stat{borisov}

\def\tit{РЫНОК С МАРКОВСКОЙ СКАЧКООБРАЗНОЙ ВОЛАТИЛЬНОСТЬЮ IV: АЛГОРИТМ МОНИТОРИНГА 
РЫНОЧНОЙ~ЦЕНЫ РИСКА ПО~ПОТОКУ ВЫСОКОЧАСТОТНЫХ НАБЛЮДЕНИЙ БАЗОВЫХ АКТИВОВ 
И~ДЕРИВАТИВОВ$^*$}

\def\titkol{Рынок с~марковской скачкообразной волатильностью IV: алгоритм мониторинга 
рыночной цены риска} % по потоку высокочастотных наблюдений базовых активов и~деривативов}

\def\aut{А.\,В.~Борисов$^1$}

\def\autkol{А.\,В.~Борисов}

\titel{\tit}{\aut}{\autkol}{\titkol}

\index{Борисов А.\,В.}
\index{Borisov A.\,V.}


{\renewcommand{\thefootnote}{\fnsymbol{footnote}} \footnotetext[1]
{Работа выполнялась с~использованием инфраструктуры Центра 
коллективного пользования <<Высокопроизводительные вы\-чис\-ле\-ния и~большие данные>>
(ЦКП <<Информатика>>) ФИЦ ИУ РАН (г.\ Москва).}}


\renewcommand{\thefootnote}{\arabic{footnote}}
\footnotetext[1]{Федеральный исследовательский центр <<Информатика и~управление>> Российской академии наук, 
\mbox{aborisov@frccsc.ru}}

\vspace*{2pt}




\Abst{Четвертая часть цикла посвящена
субоптимальному алгоритму мониторинга рыночной цены риска по наблюдениям цен 
базовых активов и~деривативов.
Как и~во всем цикле, базовые активы описываются моделью со стохастической 
волатильностью, представленной скрытым марковским скачкообразным процессом (МСП).
В условиях безарбитражности рыночная цена риска представляет собой функцию 
текущего состояния этого процесса. Особенность исследуемой модели рынка как 
стохастической сис\-те\-мы наблюдений заключена в~структуре доступных наблюдений. 
Они представляют собой последовательности цен базовых активов и~деривативов, 
фиксируемых в~случайные моменты времени. Цены базовых активов наблюдаются точно, 
в~то время как цены деривативов содержат случайные шумы. Распределения отрезков 
времени между поступлениями наблюдений, а~также шумов в~наблюдениях зависят от 
состояния оцениваемого процесса. Важнейшее свойство наблюдений заключено 
в~высокой интенсивности их поступления по сравнению с~интенсивностью изменения 
состояния оцениваемого марковского процесса. Это свойство позволяет использовать 
в~предлагаемом фильтре центральную предельную теорему для обобщенных процессов 
восстановления (ЦПТ ОПВ). Качество оценок фильтрации в~за\-ви\-си\-мости от со\-ста\-ва ис\-поль\-зу\-емых 
наблюдений про\-ил\-люст\-ри\-ро\-ва\-но чис\-лен\-ным при\-ме\-ром.}


\KW{рыночная цена риска; марковский скачкообразный процесс; 
высокочастотные наблюдения;
мультивариантный точечный процесс; численный алгоритм}

\DOI{10.14357/19922264240104}{ZRQKIT}
  
%\vspace*{-6pt}


\vskip 10pt plus 9pt minus 6pt

\thispagestyle{headings}

\begin{multicols}{2}

\label{st\stat}


\section{Введение}

Заметка продолжает цикл работ~\cite{B_23_1_IA, B_23_2_IA, B_23_3_IA}, 
посвященных решению прикладных задач сис\-тем\-но\-го анализа для класса моделей 
финансового рынка со стохастической волатильностью~\cite{AndersenBenzoni_09}, 
опи\-сы\-ва\-емой внеш\-ним ненаблюдаемым МСП.
Подобный класс моделей, несмотря на свою простоту, используется достаточно 
активно~\cite{MAMON_05, BoyleDraviam_07, mamon2014hidden} как для определения 
справедливой цены различного рода деривативов, так и~для решения задач 
хеджирования, оптимального инвестирования и~пр.

Одно из основных качеств исследуемого рынка, затрудняющих решение данных задач,~--- 
неполнота рынка, порожденная именно наличием скрытого МСП. Эта статистическая 
неопределенность определяет актуальность задач оценивания этого процесса по 
имеющейся статистической информации.  В~\cite{ B_23_3_IA} наблюдения 
представляли собой временн$\acute{\mbox{у}}$ю дискретизацию непрерывных процессов, 
выполненную с~некоторым неслучайным шагом. Такая структура наблюдений ближе к~реальности, 
однако и~она остается идеализацией. В~действительности биржевая информация 
представляет собой последовательность заявок на покупку/продажу бумаг, а~также 
проведенных сделок. Ключевое свойство этих данных заключается в~том, что все 
события и~цены фиксируются в~\textit{случайные} моменты времени. Очевидно, что 
полезная информация об оце\-ни\-ва\-емом МСП содержится не только в~ценах бумаг, но и~в~случайных отрезках времени между наблюдениями~\cite{Cvitanic_06}.

Цель работы заключается в~представлении субоптимального алгоритма оценивания 
рыночной цены риска (Market Price of Risk, MPR), или, что эквивалентно, скрытого 
состояния рынка по наблюдениям цен базовых активов и~деривативов, поступающих 
в~случайные моменты времени. 

Статья организована следующим образом. Раздел~2 
содержит сведения об ис\-сле\-ду\-емой модели рынка, а~также составе и~свойствах 
доступных наблюдений. Раздел~3 посвящен представлению алгоритма оценивания 
состояния рынка. Он основан\linebreak на важнейшем свойстве име\-ющих\-ся наблюдений~--- 
высокой интенсивности их поступления по сравнению со сменой режимов рынка. 
Данное\linebreak свойство позволяет использовать ЦПТ ОПВ~\cite{Smith_55} применительно 
к~исходным наблюдениям, пред\-об\-ра\-бо\-тан\-ным специальным образом. Использование этого 
утверждения позволяет сконструировать\linebreak достаточно прос\-той и~эффективный 
субоптимальный фильтр. Его качество, а~также влияние на точ\-ность оценивания 
наличия дополнительных наблюдений цен деривативов проиллюстрированы численным 
примером в~разд.~4. Раздел~5 содержит заключительные замечания.



\section{Модель рынка и~постановка задачи мониторинга рыночной цены риска}

На вероятностном базисе с~фильтрацией $(\Omega, \mathcal{F}, \mathcal{P}, \{\mathcal{F}_t\}_{t \in 
[0,T]})$ рассматривается модель финансового рынка, со\-сто\-ящего:
\begin{itemize}
\item
из банковского вклада с~детерминированной ставкой~$r_t$;
\item
$N$ базовых финансовых инструментов $S_t \hm= \mathrm{col}\,(S_t^1,\ldots,S_t^N)$;
\item
$M$ деривативов $F_t \hm= \mathrm{col}\,(F_t^1,\ldots,F_t^M)$ c~единым сроком погашения~$T$, 
опре\-де\-ля\-емых платежным требованием $H(S_T) \hm= \mathrm{col}\,(H^1(S_T),\ldots\linebreak \ldots,H^M(S_T))$.
\end{itemize}

Цена $S_t $  описывается стохастической дифференциальной системой (СДС)
  \begin{multline}
  dS_t = \mathrm{diag} S_t a(t,Z_t)\,dt + \mathrm{diag} S_t \sigma(t,Z_t)\,dw_t,\\
   t \in (0,T],  \enskip S_0 \sim \pi_0(s),
  \label{eq:S_sys_P}
  \end{multline}
  где $w_t \ebd \mathrm{col}\left( w_t^1,\ldots,w_t^K\right)$~--- $K$-мер\-ный стандартный винеровский 
процесс ($K \hm\geqslant N$), а
  случайные функции мгновенной процентной ставки~$a$ и~внут\-рен\-ней волатильности~$\sigma$  имеют вид:
  $$
  a\left(t,Z_t\right) = \sum\limits_{\ell=1}^L Z_t^{\ell} a^{\ell}(t);\enskip \sigma\left(t,Z_t\right) = \sum\limits_{\ell=1}^L Z_t^{\ell}  \sigma^{\ell}(t)
  $$
с набором известных детерминированных функций 
$\{a^{\ell}(t)\}_{\ell=\overline{1,L}}$  и~$\{\sigma^{\ell}(t)\}_{\ell=\overline{1,L}}$. Распределение начальной цены~$S_0$ 
таково, что все ее компоненты положительны с~ве\-ро\-ят\-ностью~1.

Мгновенная процентная ставка $a(t,Z_t)$ и~стохастическая волатильность~$\sigma(t,Z_t)$ 
определяются внешним скрытым МСП $Z_t \ebd \mathrm{col}\,(Z_t^1, \ldots, 
Z_t^L) \hm\in \{e_1,\ldots, e_L\}$  с~матрицей интенсивностей переходов~$\Lambda(\cdot)$ 
и~начальным распределением~$\pi_0^Z$. Марковский скачкообразный процесс~$Z_t$ представляет 
собой решение СДС с~мартингалом~$M_t$ в~правой части
\begin{equation}
  dZ_t = \Lambda^{\top}(t)Z_t\,dt + dM_t,\enskip t \in (0,T], \enskip Z_0 \sim 
\pi_0^Z.
  \label{eq:Z_sys}
  \end{equation}



Согласно~\cite{B_23_1_IA}, MPR~$\theta_t$ в~данной модели определяется формулой:
$$
\theta_t = \sum\limits_{\ell=1}^L Z_t^{\ell} \left(\sigma^{\ell}(t)\right)^+ \left( 
a^{\ell}(t) - r_t \mathbf{1} \right)
$$
(здесь $\mathbf{1}$~--- век\-тор-стол\-бец подходящей раз\-мер\-ности, составленный из 
единиц) и~однозначно определяется значением МСП~$Z_t$. Таким образом, задача 
оценивания MPR очевидным образом сводится к~задаче фильтрации скрытого процесса~$Z_t$.

Процессы $(S_t,Z_t,F_t)$~--- ненаблюдаемые. В~качестве доступной статистической 
информации выступают мультивариантные точечные процессы (МТП)~\cite{JSh_10} 
$\{(\tau_j^S,\mathsf{S}_j)\}_j$ и~$\{(\tau_j^F,\mathsf{F}_j)\}_j$, где 
$(\tau_j^S,\mathsf{S}_j)$~--- моменты наблюдений и~собственно наблюдения цен 
базового актива; $(\tau_j^F,\mathsf{F}_j)$~--- соответствующие значения 
производных ценных бумаг.

\textit{Задача фильтрации} заключается в~построении оценки~$\widehat{Z}_i$ 
состояния~$Z_{t_i}$, $t_i \hm= i \Delta$, по всем наблюдениям 
$\{(\tau_j^S,\mathsf{S}_j),\;(\tau_j^F,\mathsf{F}_j): \; 0 \hm\leqslant 
\tau_j^S,\tau_k^F \hm\leqslant t_i\}$.

Следует отметить, что в~данной задаче структура доступных наблюдений в~точности 
совпадает с~реальностью: таким же образом выглядит часть электронного журнала 
биржевых транзакций, посвященная регистрации успешных сделок.

Поставленная задача может быть решена оптимально в~смысле среднеквадратичного 
критерия, подобно~\cite{BMS_15_ARC}. Однако вычисление оптимальных оценок 
требует достаточно большого объема вы\-чис\-ле\-ний. Помимо этого, оптимальные оценки 
чувствительны к~неточному знанию па\-ра\-мет\-ров сис\-те\-мы наблюдения. В~сле\-ду\-ющем 
разделе представлен субоптимальный фильтр, обеспечивающий высокую точ\-ность 
оценивания и~тре\-бу\-ющий умеренных вы\-чис\-ли\-тель\-ных затрат.

\section{Алгоритм субоптимальной фильтрации}

Прежде всего уточним свойства наблюдений. Без ограничения общности для простоты 
изложения будем считать, что на рынке присутствуют один базовый и~один 
производный актив и~что рынок стационарен, т.\,е.\ функции~$r$, $a$, $\sigma$ 
и~$\Lambda$ не зависят от времени~$t$. Дополнительно предположим, что для модели 
рынка и~комплекса наблюдений выполнены сле\-ду\-ющие предположения.

\noindent
\begin{enumerate}
\item
$\mathsf{S}_j \ebd S_{\tau_j^S}$, т.\,е.\ в~случайные моменты~$\tau_j^S$ 
наблюдаются точные цены базовой бумаги.
\item
При известной траектории~$Z$ интервалы времени $\delta_j^S \ebd \tau_j^S \hm- 
\tau_{j-1}^S$
между наблюдениями цен базового актива независимы в~совокупности. Распределение~$\delta_j^S$ 
зависит от~$Z_{\tau_{j-1}^S}$ и~имеет известные условные моменты
\begin{multline*}
m_{\ell}^S \ebd \me{}{\delta_j^S | Z_{\tau_{j-1}^S} = e_{\ell}}, \\
D_{\ell}^S  \ebd  \me{}{(\delta_j^S - m_{\ell}^S )^2 | Z_{\tau_{j-1}^S} = 
e_{\ell}}, \enskip \ell = \overline{1, L}\,.
\end{multline*}
\item
$\mathsf{F}_k \ebd F_{\tau_k^F}v_k$, т.\,е.\ в~случайные моменты времени 
$\tau_k^F$ цены дериватива~$\{F_{\tau_k^F}\}$ наблюдаются с~муль\-ти\-пли\-ка\-тив\-ны\-ми 
ошибками~$\{v_k\}$, которые при известной траектории $Z$ независимы 
в~со\-во\-куп\-ности.  Распределение~$v_k$ также зависит от текущего состояния~$Z_{\tau_{k-1}^S}$ и~имеет известные условные моменты:
\begin{multline*}
m_{\ell}^v  \ebd  \me{}{\ln(v_k) | Z_{\tau_{k}^F} = e_{\ell}}, \\
D_{\ell}^v  \ebd  \me{}{(\ln(v_k) - m_{\ell}^v )^2 | Z_{\tau_{k}^F} = e_{\ell}}, 
\  \ell = \overline{1, L}\,.
\end{multline*}
\item
При известной траектории $Z$ интервалы времени $\delta_k^F \ebd \tau_k^F \hm- 
\tau_{k-1}^F$ между наблюдениями цен дериватива независимы в~совокупности. 
Распределение $\delta_k^F$ зависит от текущего со\-сто\-яния~$Z_{\tau_{k-1}^F}$~и
\begin{multline*}
m_{\ell}^F \ebd \me{}{\delta_k^F | Z_{\tau_{k-1}^F} = e_{\ell}}, \\
D_{\ell}^F  \ebd  \me{}{(\delta_k^F - m_{\ell}^F )^2 | Z_{\tau_{k-1}^F} = 
e_{\ell}} , \enskip \ell = \overline{1, L}\,.
\end{multline*}
\item
При известной траектории~$Z$ последовательности~$\{\delta_j^S\}_j$, 
$\{\delta_k^F\}_k$ и~$\{v_k\}_k$ независимы в~совокупности.
\item
Среднее время между скачками считающих процессов~$\{\tau_j^S\}_i$ 
и~$\{\tau_k^F\}_i$ много меньше шага дискретизации~$\Delta$, который, в~свою 
очередь, много меньше среднего времени между сменой со\-сто\-яний процесса~$Z$:
$$
\max\limits_{1 \leqslant \ell_1,\ell_2 \leqslant L} \max\left(m_{\ell_1}^S, m_{\ell_2}^L\right) 
\ll \Delta \ll \min\limits_{1 \leqslant \ell \leqslant L} |\Lambda_{\ell \ell}|^{-1}.
$$
\end{enumerate}

Предположения~1--5 гарантируют, что при условии $Z_t \hm\equiv e_{\ell}$,  $t \hm\in 
[t_{i-1},t_i]$, наблюдаемые МТП могут быть преобразованы к~некоторым обобщенным 
процессам восстановления; функция преобразования определена  ниже. Аргументация 
отсутствия на практике непрерывных наблюдений цен бумаг, а~так\-же наличия шумов 
в~наблюдениях деривативов представлена в~\cite{B_23_3_IA}.
Предположение~6~--- ключевое для обеспечения корректного функционирования 
предлагаемого алгоритма фильт\-ра\-ции. С~одной стороны, оно гарантирует малую 
вероятность того, что на отдельном интервале времени  $[t_{i-1},t_i]$ произойдет 
смена состояния~$Z$. С~другой стороны, интервала длины~$\Delta$ должно хватить, 
чтобы на нем начала действовать асимптотика, определенная ЦПТ ОПВ~\cite{Smith_55}.

На каждом отрезке дискретизации $[t_{i-1},t_i]$ преобразуем име\-ющи\-еся наблюдения 
сле\-ду\-ющим образом:
$$
\mathtt{W}_i =
\left[
\mathtt{W}_i^1,\;
\mathtt{W}_i^2,\;
\mathtt{W}_i^3,\;
\mathtt{W}_i^4
\right]
\ebd{}\hspace*{40mm}
$$
{\small
\begin{equation*}
{}\ebd\!
\left[
\sum\limits_{\substack{{j: \; \tau_{j-1}^S,\tau_{j}^S \in }\\{\in [t_{i-1},t_{i}]
}}}  \!\!\!\!\! 1 \
\sum\limits_{\substack{{j: \; \tau_{j-1}^S,\tau_{j}^S \in }\\{\in [t_{i-1},t_{i}]}}}   \!\!\!\!\!  
\ln \fr{\mathsf{S}_j}{\mathsf{S}_{j-1}} \
\sum\limits_{\substack{{k: \; \tau_{k-1}^F,\tau_{k}^F \in }\\ {\in [t_{i-1},t_{i}]}}}  \!\!\!\!\! 1 
\
\sum\limits_{\substack{{k: \;\tau_{k}^F \in }\\ {\in [t_{i-1},t_{i}]}}}  \!\!\!\!\!
\ln \mathsf{F}_k
\!\right]\!.\hspace*{-2.73398pt}
%\label{eq:discr_obs_1}
\end{equation*}
}
Найдем асимптотику распределения вектора~$\mathtt{W}_i$ при достаточно большом 
значении $\Delta$ для каждого из условий $Z_t \hm\equiv e_{\ell}$,  $t \hm\in [t_{i-1}, t_i]$.
Прежде всего при фиксированной траектории~$Z$ подвекторы 
$\mathrm{col}\,(\mathtt{W}_i^1,\mathtt{W}_i^2)$ и~$\mathrm{col}\,(\mathtt{W}_i^3,\mathtt{W}_i^4)$ 
независимы в~совокупности. Из правила Ито следует, что
$$
 \ln \fr{\mathsf{S}_j}{\mathsf{S}_{j-1}} = \left(a^{\ell} - 
\fr{B^{\ell}}{2}\right)\delta_i^S + \sqrt{B^{\ell}\delta_i^S} \mathtt{u}_j,
$$
где последовательность $\{\mathtt{u}_j\}$~--- стандартный гауссовский дискретный 
белый шум; $B^{\ell} \ebd \sigma^{\ell}(\sigma^{\ell})^{\top}$. Каждый вектор~$\mathtt{W}_i$~--- 
случайная сумма слагаемых вида
 \begin{multline*}
\mathtt{w}_{ik}
\ebd{}\\[3pt]
{}\ebd 
\left[
 1\enskip
 \left(a^{\ell} \!-\! \fr{B^{\ell}}{2}\right)\delta_i^S \!+\! \sqrt{B^{\ell}\delta_i^S} \mathtt{u}_j \enskip
1 \enskip
\ln \mathsf{F}_{\tau_k^F}+ \ln(v_k)
\right]\!.\hspace*{-8.5817pt}
\end{multline*}
Для этих слагаемых выполнено равенство:
\begin{multline*}
\me{}{\mathtt{w}_{ik} - \left[
 0\;\;\; 0\;\;\; 0\;\;\; \ln \mathsf{F}_{\tau_k^F}
\right]
} ={}\\[2pt]
{}=
\left[
 1 \;\;\;
 \left(a^{\ell} - \fr{B^{\ell}}{2}\right)m_{\ell}^S \;\;\;
1 \;\;\;
m_{\ell}^S
\right].
\end{multline*}
Векторы
 \begin{multline*}
\overline{\mathtt{w}}_{ik}
\ebd
\mathtt{w}_{ik} -
 \left[
 0 \;\;\; 0\;\;\; 0\;\;\; \ln \mathsf{F}_{\tau_k^F}
\right] - {}\\[2pt]
{}-\me{}{\mathtt{w}_{ik}}^{\top}
\mathrm{diag}
\left(
 \fr{\delta_i^S}{m_{\ell}^S}, \;\;\;
 \fr{\delta_i^S}{m_{\ell}^S}, \;\;\;
 \fr{\delta_k^F}{m_{\ell}^F}, \;\;\;
  \frac{\delta_k^F}{m_{\ell}^F}
\right)
\end{multline*}
имеют нулевое среднее и~ковариационную матрицу
 \begin{multline*}
K_{\ell}^\mathtt{w} \ebd \mathrm{cov} 
\left(\overline{\mathtt{w}_{ik}},\overline{\mathtt{w}_{ik}}\right) =
\mathrm{diag} \left(
 \fr{D_{\ell}^S}{(m_{\ell}^S)^2} , \;\;
 B^{\ell}m_{\ell}^S, \;\;\right.\\
\left. \fr{D_{\ell}^F}{(m_{\ell}^F)^2}, \;\;
 D_{\ell}^v + \fr{(m_{\ell}^v)^2}{m_{\ell}^F}D_{\ell}^F
\right).
\end{multline*}
Согласно ЦПТ ОПВ \cite{Smith_55} при $\Delta \hm\to \infty$ имеет место следующая 
слабая схо\-ди\-мость:
 \begin{multline*}
 \fr{1}{\sqrt{\Delta}}
 \left(
 \mathtt{W}_i -
  \left[
 0 \;\; 0\;\; 0\;\; \sum\limits_{\substack{{k: \tau_{k}^F \in }\\ {\in [t_{i-1},t_{i}]}}} 
\ln \mathsf{F}_{\tau_k^F}
\right]- {}\right.\\
\left.{}-\me{}{\mathtt{w}_{ik}}^{\top}
\mathrm{diag}
 \left(
 \fr{\Delta}{m_{\ell}^S} , \;\;
 \fr{\Delta}{m_{\ell}^S} , \;\;
 \fr{\Delta}{m_{\ell}^F} , \;\;
 \fr{\Delta}{m_{\ell}^F}
\right)
\vphantom{\sum\limits_{\substack{{k: \tau_{k}^F \in }\\ {\in [t_{i-1},t_{i}]}}}}
 \right)
 \overset{\mathrm{Law}}{\longrightarrow}
  \end{multline*}
  \begin{equation*}
  \overset{\mathrm{Law}}{\longrightarrow}
 \mathcal{N}
 \left( \mathbf{0},
 \mathrm{diag}
  \left(
 \fr{1}{m_{\ell}^S}, \;\;
 \fr{1}{m_{\ell}^S}, \;\;
 \fr{1}{m_{\ell}^F}, \;\;
 \fr{1}{m_{\ell}^F}
\right)K_{\ell}^\mathtt{w}
 \right).
 \end{equation*}
Таким образом, можно считать, что распределение семплированных наблюдений 
$({1}/{\sqrt{\Delta}}) \mathtt{W}_i$ близко 
к~гауссовскому $\mathcal{N}(\mathbf{m}_{i,\ell}, \mathbf{K}_{i})$ со средним
\begin{multline*}
\mathbf{m}_{i,\ell} \ebd
\left[
  \vphantom{\sum\limits_{\substack{{k: \,\tau_{k}^F \in }\\ { \in [t_{i-1},t_{i}]}}}}
\fr{\sqrt{\Delta}}{m_{\ell}^S} \qquad
  \left(a^{\ell} - \fr{b^{\ell}}{2}\right)\sqrt{\Delta} \qquad \fr{\sqrt{\Delta}}{m_{\ell}^F} \right.\\
\left. \fr{1}{\sqrt{\Delta}} \!\sum\limits_{\substack{{k: \,\tau_{k}^F \in }\\ 
{ \in [t_{i-1},t_{i}]}}}\!\! \ln \mathsf{F}_{\tau_k^F}+
  \fr{\sqrt{\Delta}m_{\ell}^v}{m_{\ell}^F} 
  \right]
\end{multline*}
и ковариационной матрицей
\begin{multline*}
\mathbf{K}_{i} ={}\\
{}=\mathrm{diag}
\left(
\fr{D_{\ell}^S}{(m_{\ell}^S)^3},\;\;
B^{\ell},\;\;\;
\fr{D_{\ell}^F}{(m_{\ell}^F)^3},\;\;
\fr{D_{\ell}^v}{m_{\ell}^F} +  \fr{(m_{\ell}^v)^2 
D_{\ell}^F}{(m_{\ell}^F)^3}
\right)\!.\hspace*{-7.28433pt}
\end{multline*}
Предлагаемый субоптимальный алгоритм фильтрации имеет следующий вид.
 \begin{itemize}
 \item[\textbf{Шаг 1.}] $i:=0.$
  \item[\textbf{Шаг 2.}]
 Hачальное условие:
 $
 \widehat{Z}_0 = \pi_0^Z.
$
 \item[\textbf{Шаг 3.}] $i:=i+1.$
 \item[\textbf{Шаг 4.}] Прогноз:
$
 \overline{Z}_i = \exp{\left\{\Delta \Lambda^{\top}\right\}}  \widehat{Z}_{i-1}.
 $
\item[\textbf{Шаг 5.}] Коррекция:
 $$
  \displaystyle
 \widehat{Z}_i^{\ell} =
 \displaystyle \fr{\overline{Z}_i^{\ell} 
\mathcal{G}\left( \left({1}/{\sqrt{\Delta}}\right)\mathtt{W}_i,\mathbf{m}_{i,\ell},\mathbf{K}_{\ell}\right)}
 {\sum\nolimits_{k=1}^L \overline{Z}_i^{k} 
\mathcal{G}\left(\left({1}/{\sqrt{\Delta}}\right)\mathtt{W}_i,\mathbf{m}_{i,k},\mathbf{K}_{k}\right)}.
 $$
 \item[\textbf{Шаг 6.}] Перейти к~шагу~3.
  \end{itemize}

\section{Численный пример}

Предлагаемый пример иллюстрирует влияние наблюдений цен дериватива
в~дополнение к~име\-ющим\-ся наблюдениям цены базового актива
на качество оценивания скрытого процесса состояния рынка~$Z$.

Параметры модели исследуемого рынка совпадают с~рассмотренными в~\cite{B_23_3_IA}. На отрезке времени $[0,1]$ (1~год, 250~торговых дней по 
8~ч каждый) моделируется поведение одного базового актива, стохастическая 
волатильность которого пред\-став\-ля\-ет собой МСП с~четырьмя возможными состояниями: 
<<рост\,--\,со\-сто\-яние, предшествующее па\-ни\-ке,\,--\,па\-ни\-ка\,--\,ре\-цес\-сия>>.
Ставка по депозиту $r \hm= 2\%$ годовых, на рынке присутствует один дериватив~--- 
{call}-оп\-ци\-он с~ценой исполнения~1. Для моделирования~(\ref{eq:S_sys_P}) 
и~(\ref{eq:Z_sys}) выбраны следующие параметры:
$
S_0 \hm\equiv 1$; $a \hm= (0{,}04; 0{,}035; 0{,}015; 0{,}02)$; $\sigma \hm= (0{,}1; 0{,}12; 0{,}25; 0{,}15)$;
$$
\Lambda =
\left[
\begin{array}{cccc}
-12{,}5 & 12{,}5 & 0 & 0 \\
100 & -500 & 400 & 0 \\
0 & 0 & -125 & 125 \\
40 & 0 & 10 & -50 \\
\end{array}
\right]; \enskip
\pi_0^Z =
\left[
\begin{array}{c}
0{,}7273  \\
0{,}0182  \\
0{,}0727  \\
0{,}1818  \\
\end{array}
\right].
$$
Интегрирование СДС выполнялось методом Эй\-ле\-ра--Ма\-ру\-ямы с~шагом $\tau \hm= 10^{-6}$.

В МТП $\{(\tau_j^S, \mathsf{S}_j)\}$ интервалы между наблюдениями базового 
актива при фиксированной\linebreak траектории $Z$ представляют собой независимые 
экспоненциально распределенные случайные величины \cite{Scalas_04}. Их параметры 
зависят от текущего состояния~$Z_t$ и~определяются вектором 
$\mathrm{col}\,(150\,000,\linebreak 130\,000,\;110\,000,\;140\,000)$.

В МТП $\{(\tau_j^F, \mathsf{F}_j)\}$ интервалы между наблюдениями цен опциона 
также экспоненциально распределены с~параметрами
$\mathrm{col}\,(60\,000,\;57\,000,\;51\,000,\;54\,000)$. Шумы в~наблюдениях имеют 
логнормальное распределение с~векторами средних $(0;\; 0;\; 0;\; 0)$ 
и~среднеквадратичных отклонений $(0{,}5;\; 0{,}51;\; 0{,}53;\; 0{,}52)$.

Состояние $Z$ необходимо оценить с~шагом  $\Delta\hm=0{,}0001$, который 
соответствует 12~мин операционного времени.

\setcounter{figure}{1}
\begin{figure*}[b] %fig2
\vspace*{1pt}
      \begin{center}
     \mbox{%
\epsfxsize=163mm 
\epsfbox{bor-2.eps}
}
\end{center}
\vspace*{-9pt}
\Caption{Результаты фильтрации: \textit{1}~---  $Z_t$;
\textit{2}~--- $\widehat{Z}_t^S$; \textit{3}~--- $\widehat{Z}_t^{SF}$}
\label{pic:pic_2}
\end{figure*}

Рисунок~1 представляет наблюдаемые МТП $\{(\tau_j^S, \mathsf{S}_j)\}$ 
и
$\{(\tau_j^F, \mathsf{F}_j)\}$, полученные на отдельном интервале~$\Delta$. 
%и~преобразованные затем в~вектор~$\mathtt{W}_i$.



Рисунок~\ref{pic:pic_2} содержит результаты фильтрации: покомпонентно представлено 
истинное значение состояния~$Z_t$ в~сравнении с~оценкой фильтрации~$\widehat{Z}_t^{S}$, 
построенной только по наблюдениям базового актива, и~оценкой~$\widehat{Z}_t^{SF}$, построенной по наблюдениям базового актива 
и~опциона.



По результатам расчетов можно сделать сле\-ду\-ющие заключения.

\pagebreak

{ \begin{center}  %fig1
 \vspace*{-6pt}
  \mbox{%
\epsfxsize=79mm 
\epsfbox{bor-1.eps}
}

\end{center}

\noindent
{{\figurename~1}\ \ \small{Наблюдаемые МТП: \textit{1}~--- $\{(\tau_j^S, \mathsf{S}_j)\}$;
\textit{2}~--- $\{(\tau_j^F, \mathsf{F}_j)\}$
}}}

\vspace*{6pt}




\noindent
\begin{enumerate}
\item
 Число наблюдений, попавших в~интервал длиной~$\Delta$, достаточно для 
применения к~преобразованным наблюдениям ЦПТ ОПВ.
\item
Использование в~процессе фильтрации только наблюдений базовых активов не 
позволяет определять второе состояние рынка (<<состояние, предшествующее 
панике>>) с~приемлемой точностью в~связи со слишком коротким временем пребывания в~этом состоянии и~малым объемом статистической информации, получаемым за это 
время.
\item
Использование в~процессе фильтрации дополнительных зашумленных наблюдений цен 
опциона значительно повышает качество оценивания.
\item
Различия в~характеристиках распределений интервалов между поступлениями 
измерений в~МТП, а также в~характеристиках шумов наблюдений для различных 
значений рынка играют положительную роль.
\end{enumerate}

\section{Промежуточные выводы}

В работе представлен субоптимальный алгоритм фильтрации скрытого состояния рынка 
по доступным наблюдениям рисковых ценных бумаг, поступающих в~случайные моменты 
времени. Оценки состояния рынка однозначным образом позволяют оценить MPR. 
Численный эксперимент проиллюстрировал высокое качество полученных оценок. 
В~процедуре фильтрации использовались не только наблюдения цен бумаг, но 
и~собственно поток моментов наблюдений. Ключевое свойство име\-ющих\-ся наблюдений, 
позволяющее применить этот алгоритм, состоит в~их высокой интенсивности.

Оценивание скрытого состояния рынка, как и~оценивание MPR,~--- лишь один из 
возможных способов снижения статистической неопределенности данной модели рынка, 
связанной с~его неполнотой. Другой подход заключается в~пополнении рынка 
некоторым набором производных финансовых инструментов. Решению этой задачи 
посвящена финальная часть данного цикла заметок.


{\small\frenchspacing
 { %\baselineskip=10.6pt
 %\addcontentsline{toc}{section}{References}
 \begin{thebibliography}{99}

   \bibitem{B_23_1_IA}
\Au{Борисов A.} Рынок с~марковской скачкообразной волатильностью I: мониторинг 
цены риска как задача оптимальной фильтрации~// Информатика и~её 
применения, 2023. Т.~17. Вып.~2. С.~27--33. doi: 10.14357/ 19922264230204. EDN:  GAXCHQ.

   \bibitem{B_23_2_IA}
\Au{Борисов A.} Рынок с~марковской скачкообразной волатильностью II: алгоритм 
вычисления справедливой цены деривативов~// Информатика и~её применения, 2023. 
Т.~17. Вып.~3. С.~18--24. doi: 10.14357/ 19922264230303. EDN: DNXJGB.

   \bibitem{B_23_3_IA}
\Au{Борисов A.} Рынок с~марковской скачкообразной волатильностью III:
алгоритм мониторинга цены риска по дискретным наблюдениям цен активов~// Информатика 
и~её применения, 2023. Т.~17. Вып.~4. С.~9--16. doi: 10.14357/19922264230402. EDN: 
OFYELT.

\bibitem{AndersenBenzoni_09}
\Au{Andersen T.\,G., Benzoni~L.}
Stochastic volatility~//~Complex systems in finance and
econometrics / Ed. R.\,A.~Meyers.~--- New York, NY, USA: Springer, 
2009.  Vol.~1. P.~694--726. doi: 10.1007/978-1-4419-7701-4\_38.

  \bibitem{MAMON_05}
\Au{Mamon R.\,S., Rodrigo~M.\,R.}
Explicit solutions to European options in a~regime-switching economy~//
Oper. Res. Lett., 2005. Vol.~33. P.~581--586.
doi: 10.1016/ j.orl.2004.12.003.

  \bibitem{BoyleDraviam_07}
\Au{Boyle P., Draviam~T.}
Pricing exotic options under regime switching~//~Insur. Math. 
Econ., 2007. Vol.~40. P.~267--282.
doi: 10.1016/j.insmatheco.2006.05.001.

  \bibitem{mamon2014hidden}
\Au{Mamon R., Elliott~R.}
Hidden Markov models in finance: Further developments and applications.~--- New York, NY, USA: Springer,  2014.  Vol.~II. 283~p.

  \bibitem{Cvitanic_06}
\Au{Cvitani$\acute{\mbox{c}}$~J., Liptser~R., Rozovskii~B.}
A filtering approach to tracking volatility from prices observed at
  random times~// Ann. Appl. Probab., 2006. Vol.~16. P.~1633--1652. doi: 
10.1214/105051606000000222.

  \bibitem{Smith_55}
\Au{Smith W.}
Regenerative stochastic processes~//
P.~Roy. Soc.~A~--- Math. Phy., 1955. Vol.~232. P.~6--31. doi: 
10.1098/rspa.1955.0198.

 \bibitem{JSh_10}
\Au{Jacod J., Shiryayev~A.} Limit theorems for stochastic processes.~--- Berlin: 
Springer, 2010. 684~p.

    \bibitem{BMS_15_ARC}
\Au{Борисов A., Миллер~Б., Семенихин~К.} Фильтрация марковского скачкообразного 
процесса по наблюдениям мультивариантного точечного процесса~// Автоматика и~телемеханика, 2015. Вып.~2. С.~34--60.
%doi: 10.1134/S0005117915020034.

\bibitem{Scalas_04}
\Au{Scalas E., Gorenflo R., Luckock~H., Mainardi~F., Mantelli~M., Raberto~M.}
 Anomalous waiting times in high-frequency financial data //
 Quant. Financ., 2004. Vol.~4. P.~695--702.
doi: 10.1080/14697680500040413.
\end{thebibliography}

 }
 }

\end{multicols}

\vspace*{-6pt}

\hfill{\small\textit{Поступила в~редакцию 05.01.24}}

\vspace*{10pt}

%\pagebreak

%\newpage

%\vspace*{-28pt}

\hrule

\vspace*{2pt}

\hrule



\def\tit{MARKET WITH MARKOV JUMP VOLATILITY IV: PRICE~OF~RISK~MONITORING ALGORITHM
GIVEN HIGH~FREQUENCY~OBSERVATION FLOWS OF~ASSETS PRICES}


\def\titkol{Market with Markov jump volatility IV: Price of~risk monitoring algorithm
given high frequency observation flows of~assets prices}


\def\aut{A.\,V.~Borisov}

\def\autkol{A.\,V.~Borisov}

\titel{\tit}{\aut}{\autkol}{\titkol}

\vspace*{-8pt}


\noindent
Federal Research Center ``Computer Science and Control'' of the Russian Academy of 
Sciences, 44-2~Vavilov Str., Moscow 119333, Russian Federation

\def\leftfootline{\small{\textbf{\thepage}
\hfill INFORMATIKA I EE PRIMENENIYA~--- INFORMATICS AND
APPLICATIONS\ \ \ 2024\ \ \ volume~18\ \ \ issue\ 1}
}%
 \def\rightfootline{\small{INFORMATIKA I EE PRIMENENIYA~---
INFORMATICS AND APPLICATIONS\ \ \ 2024\ \ \ volume~18\ \ \ issue\ 1
\hfill \textbf{\thepage}}}

\vspace*{4pt}





\Abste{The fourth part of the series presents a suboptimal algorithm of the market price of risk 
monitoring given the observations of the underlying and derivative asset prices. 
As in the previous papers, the market model contains the stochastic volatility described by 
a~hidden Markov jump process. This market has no arbitrage; so, the market price of risk is 
a~function of the Markov process state. The key feature of the investigated market lies in the structure of the available observations.
 They represent the underlying and derivative prices registered at random instants. The underlying prices are observed accurately, 
 while the derivative prices are corrupted by a~random noise. The distribution of the interarrival times between the observable 
 prices and the observation noises depends on the estimated process. The essential feature of the obtained observations is 
 their high arrival intensity compared with the hidden process transition rate. This property allows one to use 
 the central limit theorem for\linebreak\vspace*{-12pt}}
 
 \pagebreak
 
 \Abstend{generalized regenerative processes for the filter design. 
 The influence of the estimation performance depending on the observation complexes is illustrated with a~numerical example.}



\KWE{market price of risk; Markov jump process; high frequency observations;
multivariate point process; numerical algorithm}



\DOI{10.14357/19922264240104}{ZRQKIT}

\vspace*{-12pt}

\Ack

\vspace*{-3pt}


     \noindent
     The research was carried out using the infrastructure of the Shared Research Facilities ``High Performance Computing and Big Data'' 
(CKP ``Informatics'') of FRC CSC RAS (Moscow).


  \begin{multicols}{2}

\renewcommand{\bibname}{\protect\rmfamily References}
%\renewcommand{\bibname}{\large\protect\rm References}

{\small\frenchspacing
 {%\baselineskip=10.8pt
 \addcontentsline{toc}{section}{References}
 \begin{thebibliography}{99} 

%1
\bibitem{B_23_1_IA-1} 
\Aue{Borisov, A.} 2023. 
Rynok s~markovskoy skachkoobraznoy volatil'nost'yu I: monitoring tseny riska kak zadacha optimal'noy fil'tratsii 
[Market with Markov jump volatility I: Price of risk monitoring as an optimal filtering problem]. 
\textit{Informatika i~ee Primeneniya~--- Inform. Appl.} 17(2):27--33. 
doi: 10.14357/19922264230204. EDN: GAXCHQ.

%2
\bibitem{B_23_2_IA-1}
\Aue{Borisov, A.} 2023. 
Rynok s~markovskoy skachkoobraznoy volatil'nost'yu II: algoritm vychisleniya spravedlivoy tseny derivativov 
[Market with Markov jump volatility II: Algorithm of derivative fair price calculation].
\textit{Informatika i~ee primeneniya~--- Inform. Appl.} 17(3):18--24. 
doi: 10.14357/19922264230303. EDN: DNXJGB.

%3
\bibitem{B_23_3_IA-1}
\Aue{Borisov, A.} 2023. 
Rynok s~markovskoy skachkoobraznoy volatil'nost'yu III: algoritm monitoringa tseny riska po diskretnym nablyudeniyam tsen aktivov 
[Market with Markov jump volatility III: Price of risk monitoring algorithm given discrete-time observations of asset prices]. 
\textit{Informatika i~ee primeneniya~--- Inform. Appl.} 17(4):9--16. 
doi: 10.14357/19922264230402. EDN: OFYELT.

%4
\bibitem{AndersenBenzoni_09-1}
\Aue{Andersen, T.\,G., and L.~Benzoni.} 2009. 
Stochastic volatility. \textit{Complex systems in finance and econometrics}. Ed. R.\,A.~Meyers.  New York, NY: Springer. 1:694--726.
doi: 10.1007/978-1-4419-7701-4\_38.

%5
\bibitem{MAMON_05-1}
\Aue{Mamon, R.\,S., and M.\,R.~Rodrigo.} 2005. 
Explicit solutions to European options in a regime-switching economy. 
\textit{Oper. Res. Lett.} 33(6):581--586.
doi: 10.1016/j.orl. 2004.12.003.

%6
\bibitem{BoyleDraviam_07-1}
\Aue{Boyle, P., and T.~Draviam.} 2007. 
Pricing exotic options under regime switching. \textit{Insur. Math. Econ.} 40(2):267--282.
doi: 10.1016/j.insmatheco.2006.05.001.

%7
\bibitem{mamon2014hidden-1}
\Aue{Mamon, R.\,S., and R.~Elliott.} 2014.
\textit{Hidden Markov models in finance: Further developments and applications.} New York, NY: Springer.  Vol.~II.  283~p.

%8 
\bibitem{Cvitanic_06-1}
\Aue{Cvitani$\acute{\mbox{c}}$, J., R.~Liptser, and B.~Rozovskii.} 2006. 
A~filtering approach to tracking volatility from prices observed at random times. \textit{Ann. Appl. Probab.} 16(3):1633-1652. doi: 10.1214/105051606000000222.

%9
\bibitem{Smith_55-1}
\Aue{Smith, W.} 1955. 
Regenerative stochastic processes. \textit{P. Roy. Soc.~A~--- Math. Phy.} 232(1188):6--31. 
doi: 10.1098/ rspa.1955.0198.

 
%10
\bibitem{JSh_10-1}
\Aue{Jacod, J., and A.\,N.~Shiryaev.} 2010. \textit{Limit theorems for stochastic processes}. Berlin: Springer. 664~p.

 
%11
\bibitem{BMS_15_ARC-1}
\Aue{Borisov, A., B.~Miller, and K.~Semenikhin.} 2015. Filtering of the Markov jump process given the observations of multivariate point process.
\textit{Automat. Rem. Contr.} 76(2):219--240.
doi: 10.1134/S0005117915020034.

%12
\bibitem{Scalas_04-1}
\Aue{Scalas, E., R.~Gorenflo, H.~Luckock, F.~Mainardi, M.~Mantelli, and M.~Raberto.} 2004. 
Anomalous waiting times in high-frequency financial data.  
\textit{Quant. Financ.} 4:695--702.
doi: 10.1080/14697680500040413.
\end{thebibliography}

 }
 }

\end{multicols}

\vspace*{-6pt}

\hfill{\small\textit{Received January 5, 2024}} 

%\vspace*{-18pt}


\Contrl

\vspace*{-3pt}

\noindent
\textbf{Borisov Andrey V.} (b.\ 1965)~--- 
Doctor of Science in physics and mathematics, principal scientist, Federal Research Center ``Computer Science and Control'' 
of the Russian Academy of Sciences, 44-2~Vavilov Str., Moscow 119333, Russian Federation; \mbox{aborisov@frccsc.ru}



\label{end\stat}

\renewcommand{\bibname}{\protect\rm Литература} 