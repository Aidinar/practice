
%\newcommand{\p}{{\sf P}}
%\newcommand{\e}{{\sf E}}
%\newcommand{\D}{{\sf D}}
%\newcommand{\cov}{{\sf cov}}
%\newcommand{\corr}{{\sf corr}}
%\newcommand{\G}{{\sf Ge}}
%\newcommand{\IG}{{\sf Gi}}
%\newcommand{\abs}[1]{\left|#1\right|}
%\renewcommand{\Re}{{\sf Re} }
%\newcommand{\sgn}{{\rm sgn}}

\def\stat{kudr+shest}

\def\tit{РАВНОМЕРНЫЕ ОЦЕНКИ СКОРОСТИ СХОДИМОСТИ ДЛЯ~ИНТЕГРАЛЬНОГО ИНДЕКСА 
БАЛАНСА$^*$}

\def\titkol{Равномерные оценки скорости сходимости для~интегрального индекса 
баланса}

\def\aut{А.\,А.~Кудрявцев$^1$, О.\,В.~Шестаков$^2$}

\def\autkol{А.\,А.~Кудрявцев, О.\,В.~Шестаков}

\titel{\tit}{\aut}{\autkol}{\titkol}

\index{Кудрявцев А.\,А.}
\index{Шестаков О.\,В.}
\index{Kudryavtsev A.\,A.}
\index{Shestakov O.\,V.}


{\renewcommand{\thefootnote}{\fnsymbol{footnote}} \footnotetext[1]
{Работа выполнена при финансовой поддержке РНФ (проект 
№\,22--11--00212).}}


\renewcommand{\thefootnote}{\arabic{footnote}}
\footnotetext[1]{Московский государственный университет 
имени М.\,В.~Ломоносова, факультет вычислительной математики и~кибернетики; 
Московский центр фундаментальной и~прикладной математики, 
\mbox{aakudryavtsev@cs.msu.ru}}
\footnotetext[2]{Московский государственный университет 
имени М.\,В.~Ломоносова, факультет вычислительной математики и~кибернетики; 
Московский центр фундаментальной и~прикладной математики; Федеральный 
исследовательский центр <<Информатика и~управление>> Российской академии наук, 
\mbox{oshestakov@cs.msu.ru}}

\vspace*{-6pt}


\Abst{Рассматривается байесовская модель баланса, в~рамках которой 
исследуется скорость слабой сходимости нормированного интегрального индекса 
баланса к~ди\-гам\-ма-распре\-де\-ле\-нию в терминах равномерной мет\-ри\-ки. 
Рассматриваются интегральные факторы, негативно и~позитивно влияющие на 
функционирование сис\-те\-мы, и~их отношение~--- интегральный индекс баланса сис\-те\-мы. 
Предполагается, что чис\-ло факторов априори не известно и~описывается смешанным 
пуассоновским распределением со структурным обобщенным гам\-ма-рас\-пре\-де\-ле\-ни\-ем. 
Исследуется ско\-рость слабой сходимости в~описанной схеме к ди\-гам\-ма-рас\-пре\-де\-ле\-нию. В~качестве вспомогательного приводится утверж\-де\-ние о~ско\-рости 
слабой сходимости нормированной случайной суммы с~индексом, имеющим обобщенное 
отрицательное биномиальное распределение, к~предельному обобщенному гам\-ма-рас\-пре\-де\-ле\-нию. Результаты работы могут быть востребованы при исследовании 
моделей, использующих для описания процессов распределения с~неограниченным 
неотрицательным носителем.}

\KW{дигамма-распределение; смешанные распределения;  
индекс баланса; случайное суммирование; слабая сходимость; оценки скорости
сходимости}

\DOI{10.14357/19922264240105}{WNUUHY}
  
\vspace*{-4pt}


\vskip 10pt plus 9pt minus 6pt

\thispagestyle{headings}

\begin{multicols}{2}

\label{st\stat}

\section{Введение}


Большинство аспектов современной жизни усложнилось настолько, что определение 
критериев эффективности путем детерминированного анализа стало практически 
невозможным. По этой причине активно применяются разного рода показатели, 
индексы и~рейтинги, дающие возможность оперативного принятия решений.

В основе построения рейтингов и~индексов\linebreak обычно лежит условное разделение 
факторов, влияющих на эффективность функционирования сис\-те\-мы, на два класса. 
Первый класс включает па\-ра\-мет\-ры, способствующие функционированию \mbox{целевого} 
объекта и~позитивно влия\-ющие на ис\-сле\-ду\-емый процесс (\textit{p-фак\-то\-ры}); второй 
класс включает па\-ра\-мет\-ры, препятствующие и~негативно вли\-я\-ющие (\textit{n-фак\-то\-ры}).

Вполне естественно, что функционирование исследуемой системы в итоге зависит не 
столько от значений n- и~p-фак\-то\-ров, сколько от их соотношения. При этом большое 
расхождение между величинами факторов обычно свидетельствует либо о чрезмерных 
затратах на борьбу с негативным влиянием, либо о недооценке негативного 
воздействия. Таким образом, для того чтобы система была сбалансированной, имеет 
смысл стремиться уравновесить n- и~p-фак\-то\-ры. Для этой цели было предложено 
использовать и~исследовать свойства \textit{индекса баланса}~--- 
отношения n-фак\-то\-ра к~p-фак\-то\-ру~\cite{Ku2018}. В~\cite{Ku2018,KuShe2021_1} приводится ряд 
примеров применения моделей баланса в политологии, науке о безопасности 
жизнедеятельности, медицине, экономике, демографии, физике, теории массового 
обслуживания и~надежности.

С течением времени n- и~p-фак\-то\-ры, а~следовательно, и~индексы баланса 
претерпевают изменения. Это связано с неустойчивостью среды, в~которой 
происходит функционирование,~--- изменяется экономическая ситуация, политическая 
сис\-те\-ма, технологии производства, пристрастия населения и~т.~д. Такие изменения 
зачастую невозможно предугадать. Измерения при помощи приборов и~методик нельзя 
считать точными, ввиду неизбежно вносимых погрешностей, име\-ющих в~большинстве 
случаев изменчивый (случайный) характер. Это дает предпосылки для рассмотрения 
факторов и~индексов как случайных величин. При этом стоит учитывать, что 
глобальные изменения окружающей среды происходят достаточно редко, поэтому 
законы, влияющие на значения факторов, можно считать (в~рамках конкретной 
модели) неизменными. Из этого следует, что распределения рассматриваемых 
случайных величин следует полагать заданными априорно.
Приведенные рассуждения обусловливают применение к~моделям баланса байесовского 
метода, а~исследование рандомизированных индексов баланса, таким образом, имеет 
смысл осуществлять при помощи метода анализа масштабных смесей априорных 
распределений негативных и~позитивных факторов.
Кроме того, точное число факторов, оказывающих влияние на\linebreak функционирование 
сис\-те\-мы, невозможно определить ап\-ри\-ори. Это приводит к естественному 
обоб\-ще\-нию~\cite{KuShe2022_1_ru} байесовских моделей баланса, за\-клю\-ча\-юще\-му\-ся 
в~использовании интегральных \mbox{характеристик}, основанных на случайном суммировании.

В статье обсуждается асимптотика интегрального индекса баланса и~приводятся 
равномерные оценки скорости его сходимости к предельному распределению.

\section{Математическое описание моделей баланса}

Рассмотрим некоторую сложную систему, на функционирование которой влияют 
негативные n-фак\-то\-ры $\xi_1,\xi_2,\ldots$ и~позитивные p-фак\-то\-ры 
$\eta_1,\eta_2,\ldots$ Поскольку априори невозможно предугадать чис\-ло влияющих 
на систему в данный момент времени~$t$ негативных и~позитивных факторов, имеет 
смысл говорить о том, что их случайное чис\-ло, которое будем соответственно 
обозначать~$N_\xi(t)$ и~$N_\eta(t)$. Таким образом, интегральные величины n- и~p-фак\-то\-ров сис\-те\-мы 
пред\-став\-ля\-ют собой случайные суммы, а~интегральный индекс 
баланса сис\-те\-мы~--- их отношение.

При моделировании наступления случайного числа событий и~исследовании априори 
неизвестного числа наблюдений широко используются смешанные пуассоновские 
процессы
и~процессы Кокса
с~соответствующими непрерывными структурными распределениями, имеющими смысл 
соответственно рандомизированных интенсивностей и~накопленных интенсивностей.
Для проекции процесса Кокса $N(t)\hm=N_1(\Lambda (t))$ в точке $t\hm>0$ справедливо
\begin{multline*}
%\label{MPP_distrib}
{\sf P}(N(t)=n)={\sf P}\left(N_1(\Lambda (t))=n\right)={}\\
{}=\int\limits_0^\infty e^{-y}\fr{y^n}{n!}\, 
d{\sf P}(\Lambda(t)<y),\enskip n=0,1,\ldots
\end{multline*}
В случае когда проекция независимой от стандартного пуассоновского процесса 
$N_1(t)$ случайной меры $\Lambda(t)$ имеет обобщенное гам\-ма-рас\-пре\-де\-ле\-ние~\cite{KrMe1946,KrMe1948} 
$\mathrm{GG}\,(\nu,p,\delta (t))$ с функцией распределения
\begin{multline*}
%\label{GG_df}
{\cal G}_{\nu,p,\delta(t)}(x)=\int\limits_0^x\fr{|\nu| y^{\nu p -1}e^{-(y/\delta(t))^\nu}}
{(\delta(t))^{\nu p}\Gamma(p)}\,dy\,, \\
\nu\neq0\,, \ \ p>0\,, \ 
\ \delta(t)>0\,, \ \ t>0\,,\ \ x>0\,,
\end{multline*}
говорят, что проекции процесса Кокса~$N(t)$ имеют обобщенное 
отрицательное биномиальное распределение~\cite{Ku2019_2}.
Такие распределения широко применяются в страховании, финансовой математике, 
физике и~других областях~\cite{KoZe2019,WaZhSu2019,KoGo2020,ZhWaYa2020,BhAh2021,MaThMoVePa2020}.

Предположим, что n-фак\-то\-ры $\xi_1,\xi_2,\ldots$ и~\mbox{p-фак}\-то\-ры 
$\eta_1,\eta_2,\ldots$ суть неотрицательные и~одинаково распределенные внутри 
каждой последовательности случайные величины, имеющие \mbox{конечные} положительные 
математические ожидания ${\sf E}\xi_1\hm=a$ и~${\sf E}\eta_1\hm=b$.
Пусть $N_{1,1}(t)$ и~$N_{1,2}(t)$~--- стандартные пуассоновские процессы 
и~$N_{1,1}(t),N_{1,2}(t),\Lambda_1(t),\Lambda_2(t),\xi_1,\xi_2,\ldots,\eta_1,\eta_
2,\ldots$ независимы, а~$\Lambda_1(t)$ и~$\Lambda_2(t)$ имеют соответственно 
обобщенные гам\-ма-рас\-пре\-де\-ле\-ния $\mathrm{GG}\,(\nu_1, p, \delta_1(t))$ и~$\mathrm{GG}\,(\nu_2, q, 
\delta_2(t))$.
Предположим, что число негативных факторов~$N_\xi(t)$ и~число позитивных 
факторов $N_\eta(t)$ случайны и~имеют вид:
$$
N_\xi(t)=N_{1,1}\left(\Lambda_1(t)\right)+1\,;\enskip N_\eta(t)=N_{1,2}\left(\Lambda_2(t)\right)+1\,,
$$
т.\,е.\ в каждой группе присутствует хотя бы один фактор.

Определим интегральные n- и~p-фак\-то\-ры и~интегральный индекс баланса:
$$
\xi(t)=\sum\limits_{k=1}^{N_\xi(t)}\xi_k;\enskip
\eta(t)=\sum\limits_{k=1}^{N_\eta(t)}\eta_k;\enskip
\rho(t)=\fr{\xi(t)}{\eta(t)}.
$$

\section{Вспомогательные утверждения}

Поскольку на систему влияет случайное число случайных факторов, вопрос 
о~сбалансированности ее функционирования следует рассматривать при помощи 
исследования распределения интегрального индекса баланса. Классический подход 
теории вероятностей при исследовании такого рода объектов основан на предельных 
теоремах и~фактической замене анализа исходного распределения на анализ 
предельного.

Основные результаты работы базируются на приведенном в данном разделе аналоге 
результата, доказанного в~\cite{KuShe2022_1_ru}, для формулирования которого 
понадобится следующее распределение \cite{KuNeShe2022}.

\smallskip

\noindent
\textbf{Определение~1.}\ 
Будем говорить, что случайная величина ${\hat \rho}$ имеет ди\-гам\-ма-рас\-пре\-де\-ле\-ние 
$\mathrm{DiG}\,(r,\nu,p,q,\delta)$ с характеристическим показателем $r\hm\in\mathbb{R}$ и
параметрами формы $\nu\hm\neq0$, концентрации $p,q\hm>0$ и~масштаба $\delta\hm>0$, если 
ее преобразование Меллина задается соотношением:
\begin{multline}
\mathcal{M}_{{\hat \rho}}(z) = 
\fr{\delta^z\Gamma\left(p+{z}/{\nu}\right)\Gamma\left(q-{rz}/{\nu}\right)}{\Gamma(p)\Gamma(q)}, \\
 p+\fr{\mathrm{Re}\,(z)}{\nu}>0, \ \ q-\fr{r\mathrm{Re}\,(z)}{\nu}>0.
  \label{DG_Mellin}
\end{multline}


\noindent
\textbf{Замечание~1.}
Случайная величина~${\hat \rho}$, имеющая распределение~(\ref{DG_Mellin}), 
представима в виде масштабной смеси обобщенных гам\-ма-за\-ко\-нов $\mathrm{GG}\,(\nu,p,\delta)$ 
и~$\mathrm{GG}\,(\nu/r,q,1)$:
\begin{equation*}
\label{lambda_over_mu}
{\hat \rho}\stackrel{d}{=}\delta\left(\fr{\lambda}{\mu^r}\right)^{1/\nu},
\end{equation*}
где независимые случайные величины~$\lambda$ и~$\mu$ соответственно имеют гам\-ма-рас\-пре\-де\-ле\-ния $\mathrm{GG}\,(1,p,1)$ и~$\mathrm{GG}\,(1,q,1)$.

\smallskip

Аналогично работе~\cite{KuShe2022_1_ru} можно показать справедливость следующего 
утверж\-де\-ния.

\smallskip

\noindent
\textbf{Лемма~1.}
\textit{Пусть $\delta_i(t){\longrightarrow}\infty$ при $t\hm\to\infty$, $i\hm=1,2$.
При введенных в разд.~$2$ предположениях}
\begin{equation*}
\label{Balace_limit}
\fr{\rho(t)}{\delta_1(t)/\delta_2(t)}\Longrightarrow {\hat \rho},\enskip
t\to\infty,
\end{equation*}
\textit{где} ${\hat \rho}\sim \mathrm{DiG}\,(\nu_1/\nu_2,\nu_1,p,q,a/b)$.


\smallskip

Поскольку интегральный индекс баланса, асимптотическое поведение которого 
описано в лемме~1, представляет собой отношение двух 
независимых случайных сумм, вопрос об оценивании ско\-рости его сходимости 
к~предельному закону можно свес\-ти к~вопросу о~ско\-рости сходимости числителя 
и~знаменателя к~соответствующим предельным распределениям. Для получения 
необходимых оценок понадобится сле\-ду\-ющее утверж\-де\-ние из~\cite{ShTs2021}.

\smallskip

\noindent
\textbf{Лемма~2.}
\textit{Если $\nu\hm\in(0,1]$, $p\hm\in(0,1)$, то обобщенное гам\-ма-рас\-пре\-де\-ле\-ние~--- это 
смешанное показательное распределение}:
\begin{multline*}
{\cal G}_{\nu,p,\delta}(x)={}\\
{}=\int\limits_0^x\int\limits_0^1\fr{y}{1-y}e^{-{y}z/({1-y})} 
h_{\nu,p,\delta}(y)\,dydz,\enskip x>0\,,
\end{multline*}
где
\begin{multline}
h_{\nu,p,\delta}(y)=
\fr{ \delta^2}{\Gamma(1-p)\Gamma(p)}\fr{1}{(1-y)^2}\times{}\\
{}\times
\int\limits_1^\infty\fr{f_{\nu,1}(\delta^\nu y(1-y)^{-1}x^{-1/\nu})\,dx}
{(x-1)^px^{1+2/\nu}},\enskip 0< y<1\,;
\label{h_density}
\end{multline}
$f_{\nu,1}(x)$, $0<\nu<1$,~--- плотность одностороннего строго устойчивого 
закона, сосредоточенного на положительной полуоси, с характеристической функцией
$$
\phi_{\nu,1}(t)=\exp\left\{-|t|^\nu\exp\left\{-\fr{1}{2}i\pi\nu\, \mathrm{sgn}\,t\right\}\right\}.
$$



Воспользуемся леммой~2 для доказательства аналога тео\-ре\-мы~1 из~\cite{ShTs2021}.

\smallskip

\noindent
\textbf{Лемма~3.}
\textit{Предположим, что $\nu_1\hm\in(0,1]$, $p\hm\in(0,1)$.
Тогда для интегрального n-фак\-то\-ра $\xi(t)$ имеет место оценка}

\vspace*{-4pt}

\noindent
\begin{multline*}
\Delta_\xi\equiv\sup\limits_{x>0}\left|{\sf P}\left( \fr{\xi(t)}{\delta_1 (t)}< x\right)-
{\cal G}_{\nu_1,p,a}(x)\right|\le{}
\\
{}
\le
\int\limits_0^1
\min\left\{1,
\fr{y}{1-y}\,\fr{{\sf E} \xi_1^2}{a^2}\max\left\{1-y,\fr{1}{2}\right\}
\right\}\times{}\\
{}\times
h_{\nu_1,p,\delta_1 (t)}(y)\,dy,
\end{multline*}
\textit{где плотность $h_{\nu,p,\delta}(y)$ определена в}~(\ref{h_density}).


\smallskip

\noindent
Д\,о\,к\,а\,з\,а\,т\,е\,л\,ь\,с\,т\,в\,о\,.\ \ 
Имеем

\vspace*{-4pt}

\noindent
\begin{multline*}
\Delta_\xi\equiv\sup\limits_{x>0}\left|\vert
{\sf P}\left( \fr{\xi(t)}{\delta_1 (t)}< x\right)-
{\cal G}_{\nu_1,p,a}(x)\right|={}
\\
{}=\sup\limits_{x>0}\left|{\sf P}\left( \fr{\xi(t)}{\delta_1 (t)}< ax\right)-{\cal  G}_{\nu_1,p,a}(ax)\right|={}
\\
{}=\sup\limits_{x>0}\left|{\sf P}\left(\xi(t)<x\right)-
{\cal G}_{\nu_1,p,\delta_1 (t)}\left(\fr{x}{a}\right)\right|.
\end{multline*}

Рассмотрим геометрическую сумму
$$
S_{p,0}=\sum\limits_{k=1}^{N_{p,0}}\xi_k,
$$
где
$$
{\sf P} (N_{p,0}=n)=p(1-p)^{n},\enskip n=0,1,\ldots, \enskip p\in(0,1)\,.
$$
В работе~\cite{KuShe2023_math_zeta} было показано, что
$$
{\sf P}\left(\xi(t)<x\right)=
\int\limits_0^1 {\sf P}\left(S_{y,0}<x\right)h_{\nu_1,p,\delta_1 (t)}(y)\,dy,
$$
поэтому по  лемме~2
\begin{multline*}
\Delta_\xi=
\sup\limits_{x>0}\Bigg\vert
\int\limits_0^1 {\sf P}\left(S_{y,0}<x\right)h_{\nu_1,p,\delta_1 (t)}(y)\,dy-{}\\
{}-
\int\limits_0^{x/a}
\int\limits_0^1\fr{y}{1-y}e^{-{yz}/({1-y})}  h_{\nu_1,p,\delta_1 (t)}(y)\,dy
\,dz
\Bigg\vert\le{}
\end{multline*}

\noindent
\begin{multline*}
{}\le
\int\limits_0^1\sup\limits_{x>0}\left| 
{\sf P}\left(S_{y,0}<x\right)-
{\cal G}_{1,1,({1-y})/{y}}\left(\fr{x}{a}\right)
\right|\times{}\\
{}\times
h_{\nu_1,p,\delta_1 (t)}(y)\,dy
={}
\\
{}=\int\limits_0^1\sup\limits_{x>0}\left| {\sf P}\left(\fr{yS_{y,0}}{(1-y)a}<x\right)-{}\right.\\
\left.{}-
{\cal G}_{1,1,({1-y})/{y}}\left(\fr{(1-y)x}{y}\right)
\right|
h_{\nu_1,p,\delta_1 (t)}(y)\,dy
={}
\\
{}=
\int\limits_0^1\sup\limits_{x>0}\left|{\sf P}\left(\fr{yS_{y,0}}{(1-y)a}<x\right)-
{\cal G}_{1,1,1}\left(x\right)
\right|\times{}\\
{}\times
h_{\nu_1,p,\delta_1 (t)}(y)\,dy.
\end{multline*}
Используя оценку из~\cite{ShTs2021}, получаем
\begin{multline*}
\Delta_\xi\le
\int\limits_0^1
\min\left\{1, y \fr{{\sf E} X_1^2}{a^2}\max\left\{1,\fr{1}{2(1-y)}\right\}
\right\}\times{}\\
{}\times h_{\nu_1,p,\delta_1 (t)}(y)\,dy
={}
\\
{}=
\int\limits_0^1 \min\left\{1, \fr{y}{1-y}\,\fr{{\sf E} X_1^2}{a^2}\max\left\{1-y,\fr{1}{2}\right\}
\right\}\times{}\\
{}\times
h_{\nu_1,p,\delta_1 (t)}(y)\,dy.
\end{multline*}
Лемма доказана.

\smallskip

\noindent
\textbf{Замечание~2.}
Поскольку интегральный p-фак\-тор $\eta(t)$ имеет ту же структуру, что и~$\xi(t)$, 
справедлива аналогичная оценка
\begin{multline*}
\Delta_\eta\equiv\sup\limits_{x>0}\left|{\sf P}\left( \fr{\eta(t)}{\delta_2 (t)}< 
x\right)-{\cal G}_{\nu_2,q,b}(x)\right|\le{}
\\
{}\le
\int\limits_0^1
\min\left\{1,
\fr{y}{1-y}\,\fr{{\sf E} \eta_1^2}{b^2}\max\left\{1-y,\fr{1}{2}\right\}
\right\}\times{}\\
{}\times
h_{\nu_2,q,\delta_2 (t)}(y)\,dy.
\end{multline*}

\section{Оценивание скорости сходимости нормированного интегрального индекса 
баланса к~дигамма-распределению}

Рассмотрим оценки ско\-рости схо\-ди\-мости в лемме~1 в терминах 
равномерной мет\-ри\-ки. Справедливо сле\-ду\-ющее утверж\-дение.

\smallskip

\noindent
\textbf{Теорема~1.}\
\textit{Предположим, что $\nu_1,\nu_2\hm\in(0,1]$, $p,q\hm\in(0,1)$. Пусть интегральные n- и
p-фак\-то\-ры удовле\-тво\-ря\-ют условиям разд.~$2$. Тогда}

\noindent
\begin{multline}
\Delta\equiv\sup\limits_{x>0}\left|{\sf P}\left(\fr{\rho(t)}{\delta_1(t)/\delta_2(t)}<x\right)-{\sf P}({\hat \rho}<x)\right|\le{}
\\
{}\le
\int\limits_0^1
\min\left\{1,
\fr{y}{1-y}\,\fr{{\sf E} \xi_1^2}{a^2}\max\left\{1-y,\fr{1}{2}\right\}
\right\}\times{}\\
{}\times
h_{\nu_1,p,\delta_1 (t)}(y)\,dy
+{}
\\
{}+
\int\limits_0^1
\min\left\{1,
\fr{y}{1-y}\,\fr{{\sf E} \eta_1^2}{b^2}\max\left\{1-y,\fr{1}{2}\right\}
\right\}\times{}\\
{}\times
h_{\nu_2,q,\delta_2 (t)}(y)\,dy.
\label{delta_speed}
\end{multline}


\noindent
Д\,о\,к\,а\,з\,а\,т\,е\,л\,ь\,с\,т\,в\,о\,.\ \ 
Согласно замечанию~1 предельная случайная величина ${\hat \rho}$ в 
лемме~1 может быть представлена в виде

\noindent
$$
{\hat \rho}\stackrel{d}{=}\fr{\hat\xi}{\hat\eta},
$$

\vspace*{-3pt}

\noindent
где случайные величины ${\hat\xi}$ и~${\hat\eta}$ соответственно имеют 
распределения $\mathrm{GG}\,(\nu_1,p,a)$ и~$\mathrm{GG}\,(\nu_2,q,b)$.

Рассмотрим равномерное расстояние между допредельными и~предельными величинами. 
Имеем
\begin{multline*}
\Delta=\sup\limits_{x>0}\left|{\sf P}\left(\fr{\rho(t)}{\delta_1(t)/\delta_2(t)}<x\right)-
{\sf P}\left(\fr{\hat \xi}{\hat\eta}<x\right)\right|={}
\\
{}=\sup\limits_{x>0}\left|
\vphantom{\left(\ln\fr{\hat \xi}{\hat\eta}<\ln x\right)}
{\sf P}\left(\ln \fr{\rho(t)}{\delta_1(t)/\delta_2(t)}<\ln 
x\right)-{}\right.\\
\left.{}-{\sf P}\left(\ln\fr{\hat \xi}{\hat\eta}<\ln x\right)\right|={}
\\
{}=\sup\limits_{x\in\mathbb{R}}\left|{\sf P}\left(\ln \fr{\xi(t)}{\delta_1 (t)}-\ln 
\fr{\eta(t)}{\delta_2 (t)}< x\right)-{}\right.\\
\left.{}-{\sf P}(\ln{\hat \xi}-\ln{\hat\eta}< 
x)
\vphantom{\left(\ln \fr{\xi(t)}{\delta_1 (t)}-\ln 
\fr{\eta(t)}{\delta_2 (t)}< x\right)}
\right|.
\end{multline*}
Поскольку равномерная метрика обладает свойством регулярности~\cite{Senatov1998}, имеем
\begin{multline*}
\Delta\le\sup\limits_{x\in\mathbb{R}}\left|{\sf P}\left(\ln \fr{\xi(t)}{\delta_1 (t)}< 
x\right)-{\sf P}(\ln{\hat \xi}< x)\right|+{}\\
{}+
\sup\limits_{x\in\mathbb{R}}\left|{\sf P}\left(\ln \fr{\eta(t)}{\delta_2 (t)}< x\right)-
{\sf P}(\ln{\hat\eta}< x)\right|={}
\\
{}=\sup\limits_{x>0}\left|{\sf P}\left( \fr{\xi(t)}{\delta_1 (t)}< x\right)-{\sf P}({\hat \xi}< 
x)\right|+{}\\
{}+
\sup\limits_{x>0}\left|{\sf P}\left( \fr{\eta(t)}{\delta_2 (t)}< x\right)-{\sf P}({\hat\eta}< 
x)\right|=\Delta_\xi+\Delta_\eta.
\end{multline*}

Таким образом, утверждение теоремы следует из леммы~3 и
замечания~2. Теорема доказана.

\smallskip

В качестве примера применения оценки~(\ref{delta_speed}) рассмотрим случай, 
когда  $a\hm=b\hm=1$ и~проекции управ\-ля\-ющих процессов Кокса, отвечающих за число 
негативных и~позитивных факторов, имеют гам\-ма-рас\-пре\-де\-ле\-ния с па\-ра\-мет\-ра\-ми 
масштаба 

\pagebreak

\noindent
$$
\delta_i(t)=\fr{1-\theta_i(t)}{\theta_i(t)}\,,
$$
 где 
$\theta_i(t)\to 0$ при $t\hm\to\infty$, $i\hm=1,2$. Для этого  
воспользуемся соответствующим результатом из работы~\cite{ShTs2021}:
\begin{multline*}
\sup\limits_{x>0}\left|{\sf P}\left( \xi(t)< x\right)-{\cal G}_{1,p,(1-
\theta_1(t))/\theta_1(t)}(x)\right|\le{}\\
{}\le
\fr{(\theta_1(t))^p(2{\sf E}\xi_1^2+p^{-1}+1)}{\Gamma(p)\Gamma(2-p)}.
\end{multline*}
Из последнего неравенства получаем
\begin{multline*}
\sup\limits_{x>0}\left| {\sf P}\left( \fr{\xi(t)}{\delta_1(t)}< x\right)-{\cal  G}_{1,p,1}(x)\right|\le{}\\
{}\le
\fr{(\theta_1(t))^p(2\e\xi_1^2+p^{-1}+1)}{\Gamma(p)\Gamma(2-p)}.
\end{multline*}
Поскольку интегральный p-фактор $\eta(t)$ имеет структуру, аналогичную $\xi(t)$, 
окончательно для интегрального индекса баланса получаем оценку:
\begin{multline*}
\sup\limits_{x>0}\left|{\sf P}\left(\fr{\rho(t)}{\delta_1(t)/\delta_2(t)}<x\right)-
{\sf P}({\hat \rho}<x)\right|\le{}
\\
{}\le \fr{(\theta_1(t))^p(2{\sf E}\xi_1^2+p^{-1}+1)}{\Gamma(p)\Gamma(2-p)}+{}\\
{}+
\fr{(\theta_2(t))^q(2{\sf E}\eta_1^2+q^{-1}+1)}{\Gamma(q)\Gamma(2-q)}.
\end{multline*}

\vspace*{-14pt}

\section{Заключение}

\vspace*{-3pt}


В работе доказана теорема о скорости слабой сходимости нормированного 
интегрального индекса баланса к ди\-гам\-ма-распре\-де\-ле\-нию. Получена верхняя 
оценка равномерного расстояния между функциями распределения допредельного 
и~предельного законов. Предполагается, что интегральный индекс баланса 
представляет собой отношение двух случайных сумм, индексы которых описываются 
обобщенными отрицательными биномиальными распределениями. Результаты работы 
могут найти применение при исследовании масштабных смесей в схеме случайного 
суммирования, в~частности при анализе байесовских моделей баланса.

\vspace*{-12pt}


{\small\frenchspacing
 {\baselineskip=10.51pt
 %\addcontentsline{toc}{section}{References}
 \begin{thebibliography}{99}
\bibitem{Ku2018}
\Au{Кудрявцев~А.\,А.}
Байесовские модели баланса~// Информатика и~её применения, 2018. Т.~12. Вып.~3. С.~18--27.
doi: 10.14357/19922264180303. EDN: YAMDHF.

\bibitem{KuShe2021_1}
\Au{Kudryavtsev~A.\,A., Shestakov~O.\,V.}
Asymptotically normal estimators for the parameters of the gamma-exponential 
distribution~// Mathematics, 2021. Vol.~9. Iss.~3. Art.~273. 13~p. doi: 10.3390/math9030273. 

\bibitem{KuShe2022_1_ru}
\Au{Кудрявцев~А.\,А., Шестаков~О.\,В.}
Ди\-гам\-ма-рас\-пре\-де\-ле\-ние как предельное для интегрального индекса баланса~// Вестн. Моск. ун-та. Сер~15. Вычисл. матем. и~киберн., 2022. №\,3. 
С.~26--32.

\bibitem{KrMe1946}
\Au{Крицкий~С.\,Н., Менкель~М.\,Ф.}
О~приемах исследования случайных колебаний речного стока~// Труды НИУ ГУГМС. Сер.~IV, 1946. Вып.~29. С.~3--32.
%. Труды Гос. гидрологического ин-та, вып. 29, 1946.

\bibitem{KrMe1948}
\Au{Крицкий~С.\,Н., Менкель~М.\,Ф.}
Выбор кривых распределения вероятностей для расчетов речного стока~// Известия АН СССР. Отд. техн. наук, 1948. №\,6. С.~\mbox{15--21}.

\bibitem{Ku2019_2}
\Au{Кудрявцев~А.\,А.}
О~представлении гам\-ма-экс\-по\-нен\-ци\-аль\-но\-го и~обобщенного отрицательного 
биномиального распределений~// Информатика и~её применения, 2019. Т.~13. Вып.~4. С.~76--80.
doi: 10.14357/ 19922264190412. EDN: YTZLNY.

\bibitem{KoZe2019}
\Au{Korolev~V.\,Yu., Zeifman~A.\,I.}
Generalized negative binomial distributions as mixed geometric laws and related 
limit theorems~// Lith. Math. J., 2019. Vol.~59. Iss.~3. P.~366--388.
10.1007/s10986-019-09452-x.

\bibitem{WaZhSu2019} %8
\Au{Wang~X., Zhao~X., Sun~J.}
A compound negative binomial distribution with mutative termination conditions 
based on a change point~// J. Comput. Appl. Math., 2019. Vol.~351. P.~237--249.
doi: 10.1016/j.cam.2018.11.009.

\bibitem{KoGo2020}
\Au{Korolev~V.\,Yu., Gorshenin~A.\,K.\/}
Probability models and statistical tests for extreme precipitation based on 
generalized negative binomial distributions~// Mathematics, 2020. Vol.~8. Iss.~4. Art.~604. 30~p. doi: 10.3390/math8040604.


\bibitem{ZhWaYa2020} %10
\Au{Zhang~J., Wang~D., Yang~K.}
A study of RCINAR(1) process with generalized negative binomial marginals~// Commun. Stat. B~--- Simul., 2020. Vol.~49. Iss.~6. P.~1487--1510.
doi: 10.1080/03610918.2018.1498891.

\bibitem{BhAh2021} %11
\Au{Bhati~D., Ahmed~I.\,S.}
On uniform-negative binomial distribution including Gauss hypergeometric 
function and its application in count regression modeling~// Commun. Stat.~--- Theor.~M., 2021. Vol.~50. Iss.~13. P.~3106--3122.
doi: 10.1080/03610926.2019.1682163.

\bibitem{MaThMoVePa2020} %12
\Au{Mangiola~S., Thomas~E.\,A., Modr$\acute{\mbox{a}}$k~M., Vehtari~A., Papenfuss~A.\,T.}
Probabilistic outlier identification for RNA sequencing generalized linear 
models~// NAR Genomics Bioinformatics, 2021. Vol.~3. Iss.~1. Art.~lqab005. 9~p.
doi: 10.1093/nargab/lqab005.

\bibitem{KuNeShe2022}
\Au{Кудрявцев~А.\,А., Недоливко~Ю.\,Н., Шестаков~О.\,В.}
Оновные вероятностные характеристики ди\-гам\-ма-рас\-пре\-де\-ле\-ния и~метод оценивания 
его параметров~// Вестн. Моск. ун-та. Сер~15. Вычисл. матем. и~киберн., 2022. №\,3. 
С.~22--29.

\bibitem{ShTs2021}
\Au{Shevtsova~I., Tselishchev~M.}
On the accuracy of the generalized gamma approximation to generalized negative 
binomial random sums~// Mathematics, 2021. Vol.~9. Iss.~13. Art.~1571. 8~p. doi: 10.3390/math9131571.

\bibitem{KuShe2023_math_zeta}
\Au{Kudryavtsev~A.\,A., Shestakov~O.\,V.}
Estimates of the convergence rate in the generalized Renyi theorem with 
a~structural digamma distribution using zeta metrics~// Mathematics, 2023. Vol.~11. Iss.~21. Art.~4477. 10~p. doi: 10.3390/math11214477.

\bibitem{Senatov1998}
\Au{Senatov~V.\,V.}
Normal approximation: New results, methods, and problems.~--- Utrecht, The Netherland: VSP, 1998. 363~p.
\end{thebibliography}

 }
 }

\end{multicols}

\vspace*{-10pt}

\hfill{\small\textit{Поступила в~редакцию 15.01.24}}

%\vspace*{8pt}

%\pagebreak

\newpage

\vspace*{-28pt}

%\hrule

%\vspace*{2pt}

%\hrule



\def\tit{UNIFORM CONVERGENCE RATE ESTIMATES\\ FOR~THE~INTEGRAL BALANCE INDEX}


\def\titkol{Uniform convergence rate estimates for~the~integral balance index}


\def\aut{A.\,A.~Kudryavtsev$^{1,2}$ and~O.\,V.~Shestakov$^{1,2,3}$}

\def\autkol{A.\,A.~Kudryavtsev and~O.\,V.~Shestakov}

\titel{\tit}{\aut}{\autkol}{\titkol}

\vspace*{-10pt}


\noindent
$^{1}$Department of Mathematical Statistics, Faculty of Computational Mathematics and Cybernetics, M.\,V.~Lomo-\linebreak
$\hphantom{^1}$nosov Moscow State University, 1-52~Leninskie Gory, GSP-1, Moscow 119991, Russian Federation

\noindent
$^{2}$Moscow Center for Fundamental and Applied Mathematics, M.\,V.~Lomonosov Moscow State University,\linebreak
$\hphantom{^1}$1~Leninskie Gory, GSP-1, Moscow 119991, Russian Federation

\noindent
$^{3}$Federal Research Center ``Computer Science and Control'' of the Russian Academy of Sciences, 44-2~Vavilov\linebreak
$\hphantom{^1}$Str., Moscow 119333, Russian Federation

\def\leftfootline{\small{\textbf{\thepage}
\hfill INFORMATIKA I EE PRIMENENIYA~--- INFORMATICS AND
APPLICATIONS\ \ \ 2024\ \ \ volume~18\ \ \ issue\ 1}
}%
 \def\rightfootline{\small{INFORMATIKA I EE PRIMENENIYA~---
INFORMATICS AND APPLICATIONS\ \ \ 2024\ \ \ volume~18\ \ \ issue\ 1
\hfill \textbf{\thepage}}}

\vspace*{4pt}




\Abste{The paper considers the Bayesian balance model in which the rate of weak convergence of the normalized integral 
balance index to the digamma distribution is studied in terms of a uniform metric. The integral factors negatively 
and positively influencing the functioning of the system and their ratio, the integral balance index of the system, 
are considered. It is assumed that the number of factors is not known \textit{a~priori} and is described by 
a~mixed Poisson distribution with a~structural generalized gamma distribution. The rate of weak convergence to the digamma 
distribution in the described scheme is studied. As an auxiliary statement, the rate of weak convergence of 
a~normalized random sum with an index having a~generalized negative binomial distribution to a limiting generalized gamma distribution is estimated.
 The results of the work may be in demand in the study of models used to describe processes with distributions having an unlimited nonnegative support.}

\KWE{digamma distribution; mixed distributions; balance index; random summation; weak convergence; estimates of convergence rate}




\DOI{10.14357/19922264240105}{WNUUHY}

\vspace*{-12pt}

\Ack


\vspace*{-3pt}

\noindent
The research was supported by the Russian Science Foundation, project No.\,22-11-00212.

\vspace*{6pt}

  \begin{multicols}{2}

\renewcommand{\bibname}{\protect\rmfamily References}
%\renewcommand{\bibname}{\large\protect\rm References}



{\small\frenchspacing
 {%\baselineskip=10.8pt
 \addcontentsline{toc}{section}{References}
 \begin{thebibliography}{99}
\bibitem{Ku2018-1}
\Aue{Kudryavtsev, A.\,A.} 2018. 
Bayesovskie modeli ba\-lan\-sa [Bayesian balance models].
\textit{Informatika i~ee Primeneniya~--- Inform. Appl.} 12(3):18--27.
doi: 10.14357/ 19922264180303. EDN: YAMDHF.


\bibitem{KuShe2021_1-1}
\Aue{Kudryavtsev, A.\,A., and O.\,V.~Shestakov.} 2021. 
Asymptotically normal estimators for the parameters of the gamma-exponential distribution.
\textit{Mathematics} 9(3):273. 13~p. doi: 10.3390/math9030273.

\bibitem{KuShe2022_1_ru-1}
\Aue{Kudryavtsev, A.\,A., and O.\,V.~Shestakov.} 2022.
Digamma distribution as a~limit for the integral balance index. 
\textit{Moscow University Computational Mathematics Cybernetics} 46:133--139.
doi: 10.3103/S0278641922030062.

\bibitem{KrMe1946-1}
\Aue{Kritskiy, S.\,N., and M.\,F.~Menkel'.} 1946. 
O~priemakh issledovaniya sluchaynykh kolebaniy rechnogo stoka [Methods
of investigation of random fluctuations of river flow]. \textit{Trudy
NIU GUGMS Ser.~IV} [Proceedings of GUGMS Research
Institutions Ser.~IV] 29:3--32.

\bibitem{KrMe1948-1}
\Aue{Kritsky, S.\,N., and M.\,F.~Menkel'}. 1948. Vybor krivykh
raspredeleniya veroyatnostey dlya raschetov rechnogo stoka [Selection of probability distribution curves for river
flow calculations]. \textit{Izvestiya AN SSSR. Otd. tekhn. nauk}
[Herald of the Russian Academy of Sciences. Technical
Sciences] 6:15--21.

\bibitem{Ku2019_2-1}
\Aue{Kudryavtsev, A.\,A.} 2019. 
O~predstavlenii gamma-eksponentsial'nogo i~obobshchennogo otritsatel'nogo binomial'nogo raspredeleniy 
[On the representation of gamma-exponential and generalized negative binomial distributions].
\textit{Informatika i~ee Primeneniya~--- Inform. \mbox{Appl}.} 13(4):76--80.
doi: 10.14357/19922264190412. EDN: YTZLNY.


\bibitem{KoZe2019-1}
\Aue{Korolev, V.\,Yu., and A.\,I.~Zeifman.} 2019. 
Generalized negative binomial distributions as mixed geometric laws and related limit theorems.
\textit{Lith. Math. J.} 59(3):366--388. doi: 10.1007/s10986-019-09452-x.

\bibitem{WaZhSu2019-1}
\Aue{Wang, X., X.~Zhao, and J.~Sun.} 2019.  
A compound negative binomial distribution with mutative termination conditions based on a~change point.
\textit{J. Comput. Appl. Math.} 351:237--249. doi: 10.1016/j.cam.2018.11.009.

\bibitem{KoGo2020-1}
\Aue{Korolev, V.\,Yu., and A.\,K.~Gorshenin.} 2020.
Probability models and statistical tests for extreme precipitation based on generalized negative binomial distributions.
\textit{Mathematics} 8(4):604. 30~p. doi: 10.3390/math8040604.

\bibitem{ZhWaYa2020-1}
\Aue{Zhang, J., D.~Wang, and K.~Yang.} 2020. 
A study of RCINAR(1) process with generalized negative binomial marginals.
\textit{Commun. Stat. B~--- Simul.} 49(6):1487--1510.
doi: 10.1080/03610918.2018.1498891.

\bibitem{BhAh2021-1}
\Aue{Bhati, D., and I.\,S.~Ahmed.} 2021. 
On uniform-negative binomial distribution including Gauss hypergeometric function and its application in count regression modeling.
\textit{Commun. Stat.~--- Theor. M.} 50(13):3106--3122. doi: 10.1080/03610926.2019.1682163.

\bibitem{MaThMoVePa2020-1}
\Aue{Mangiola, S., E.\,A.~Thomas, M. Modr$\acute{\mbox{a}}$k, A.~Vehtari, and A.\,T.~Papenfuss.}  2021.
Probabilistic outlier identification for RNA sequencing generalized linear models
\textit{NAR Genomics Bioinformatics} 3(1):lqab005. 9~p. doi: 10.1093/nargab/lqab005.

\bibitem{KuNeShe2022-1}
\Aue{Kudryavtsev, A.\,A., Yu.\,N.~Nedolivko, and O.\,V.~Shestakov.} 2022. 
Main probabilistic characteristics of the digamma distribution and the method of estimating its parameters.
\textit{Moscow University Computational Mathematics Cybernetics} 46:81--88. doi: 10.3103/S0278641922020054.

\bibitem{ShTs2021-1}
\Aue{Shevtsova, I., and M.~Tselishchev.} 2021.
On the accuracy of the generalized gamma approximation to generalized negative binomial random sums.
\textit{Mathematics} 9(13):1571. 8~p. doi: 10.3390/math9131571.

\bibitem{KuShe2023_math_zeta-1}
\Aue{Kudryavtsev, A.\,A., and O.\,V.~Shestakov.} 2023.
Estimates of the convergence rate in the generalized Renyi theorem with a structural digamma distribution using zeta metrics.
\textit{Mathematics} 11(21):4477. 10~p. doi: 10.3390/math11214477.

\bibitem{Senatov1998-1}
\Aue{Senatov, V.\,V.} 1998. 
\textit{Normal approximation: New results, methods, and problems}.
Utrecht, The Netherland: VSP. 363~p.

 \end{thebibliography}

 }
 }

\end{multicols}

\vspace*{-6pt}

\hfill{\small\textit{Received January 15, 2024}} 

\vspace*{-18pt}
 
      \Contr

\vspace*{-3pt}


\noindent
\textbf{Kudryavtsev Alexey A.} (b.\ 1978)~--- Candidate of Science (PhD) in physics and mathematics, associate professor, 
Department of Mathematical Statistics, Faculty of Computational Mathematics and Cybernetics, M.\,V.~Lomonosov Moscow State University, 
1-52~Leninskie Gory, GSP-1, Moscow 119991, Russian Federation; senior scientist, Moscow Center for Fundamental and Applied Mathematics, 
M.\,V.~Lomonosov Moscow State University,
1~Leninskie Gory, GSP-1, Moscow 119991, Russian Federation; \mbox{aakudryavtsev@cs.msu.ru}

\vspace*{3pt}

\noindent
\textbf{Shestakov Oleg V.} (b.\ 1976)~--- Doctor of Science in physics and mathematics, professor, 
Department of Mathematical Statistics, Faculty of Computational Mathematics and Cybernetics, M.\,V.~Lomonosov Moscow State University, 
1-52~Leninskie Gory, GSP-1, Moscow 119991, Russian Federation; senior scientist, Federal Research Center ``Computer Science and Control'' 
of the Russian Academy of Sciences, 44-2~Vavilov Str., Moscow 119333, Russian Federation; leading scientist, 
Moscow Center for Fundamental and Applied Mathematics, M.\,V.~Lomonosov Moscow State University, 
1~Leninskie Gory, GSP-1, Moscow 119991, Russian Federation; \mbox{oshestakov@cs.msu.su}


\label{end\stat}

\renewcommand{\bibname}{\protect\rm Литература}