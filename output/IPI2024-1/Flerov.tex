\def\stat{flerov}

\def\tit{МОДЕЛИ СИНТЕЗА КОМПОНОВОЧНОЙ СХЕМЫ В~ЗАДАЧЕ~ФОРМИРОВАНИЯ 
ОБЛИКА САМОЛЕТА}

\def\titkol{Модели синтеза компоновочной схемы в~задаче формирования 
облика самолета}

\def\aut{Л.\,Л.~Вышинский$^1$, Ю.\,А.~Флёров$^2$}

\def\autkol{Л.\,Л.~Вышинский, Ю.\,А.~Флёров}

\titel{\tit}{\aut}{\autkol}{\titkol}

\index{Вышинский Л.\,Л.}
\index{Флёров Ю.\,А.}
\index{Vyshinsky L.\,L.}
\index{Flerov Yu.\,A.}


%{\renewcommand{\thefootnote}{\fnsymbol{footnote}} \footnotetext[1]
%{Работа выполнена при финансовой поддержке РНФ (проект 21-71-10135).}}


\renewcommand{\thefootnote}{\arabic{footnote}}
\footnotetext[1]{Федеральный исследовательский центр <<Информатика и~управление>> Российской академии наук, 
\mbox{wyshinsky@mail.ru}}
\footnotetext[2]{Федеральный исследовательский центр <<Информатика и~управление>> Российской академии наук, 
\mbox{fler@ccas.ru}}

\vspace*{-12pt}




  \Abst{Рассматриваются задачи синтеза математических моделей самолета на начальной 
стадии проектирования. Центральная задача начальной стадии проектирования состоит 
в~разработке струк\-тур\-но-па\-ра\-мет\-ри\-че\-ской модели (СПМ) изделия и~построении его 
компоновочной схемы. Струк\-тур\-но-па\-ра\-мет\-ри\-че\-ская модель изделия описывает 
пространство основных конструктивных па\-ра\-мет\-ров самолета, а~компоновочная схема 
представляет его структуру, множество основных агрегатов и~сис\-тем летательного аппарата (ЛА), 
их взаимное расположение и~конструктивные связи между агрегатами. Компоновочная 
схема самолета служит основой для дальнейшей детализации проекта. На начальной стадии 
проектирования определяются допустимые диапазоны значений конструктивных параметров 
агрегатов компоновочной схемы, в~которых осуществляется поиск оптимальных 
конструктивных решений. Представлено описание СПМ и~компоновочных схем широкого класса самолетов, которые могут быть 
реализованы средствами автоматизированной системы весового проектирования.}
  
  \KW{математическое моделирование; автоматизация проектирования; самолет; весовое 
проектирование; генератор проектов}

\DOI{10.14357/19922264240109}{DSPGKV}
  
%\vspace*{-6pt}


\vskip 10pt plus 9pt minus 6pt

\thispagestyle{headings}

\begin{multicols}{2}

\label{st\stat}
  
\section{Введение}

  В~работе~[1] была представлена автоматизированная сис\-те\-ма весового 
проектирования самолетов (АСВП). Задачи весового проектирования встают на 
всех этапах жизненного цикла (ЖЦ) \mbox{самолетов}, начиная с~формирования облика, 
эскизного и~рабочего проектирования вплоть до изготовления серийных 
образцов и~их эксплуатации~[2]. Естественно, что на разных этапах 
необходимы разные модели и~разные средства их синтеза и~анализа. 
В~настоящей статье более подробно описаны задачи, связанные с~синтезом 
формирования облика, поскольку именно на этом этапе принимаются основные 
технические и~проектные решения.

  \begin{figure*}[b] %fig1
  \vspace*{10pt}
      \begin{center}
     \mbox{%
\epsfxsize=160mm 
\epsfbox{fle-1.eps}
}
\end{center}
\vspace*{-8.5pt}
  \Caption{База данных СПМ готовых изделий}
  \end{figure*}
  
  Ситуация на начальных стадиях проектирования самолетов может быть 
совершенно различной в~за\-ви\-си\-мости от характера разработки~--- либо это 
модификация некоторого серийного или экспериментального самолета, либо 
новая разработка и~создание самолета принципиально нового поколения. 
Поэтому каждый раз перед проектировщиком встает задача выбора моделей, 
которые бы позволили дать формальное описание конструкции проектируемого 
изделия. Широкое применение автоматизированных сис\-тем проектирования, 
CAD-сис\-тем, сис\-тем геометрического моделирования позволяет создавать 
уже на ранних этапах проектирования слож\-ные гео\-мет\-ри\-че\-ские 3D-мо\-де\-ли 
современных ЛА. Однако по\-стро\-ению 3D-мо\-де\-лей 
обычно пред\-шест\-ву\-ет большой объем расчетов, связанных с~выбором 
структуры про\-ек\-ти\-ру\-емо\-го изделия, со\-ста\-ва его основных агрегатов, 
оптимизацией значений их конструктивных па\-ра\-мет\-ров и~характеристик. 
Такие расчеты должны опираться на адекватное описание про\-ек\-ти\-ру\-емо\-го 
изделия в~виде СПМ, разработка которой и~есть 
основная задача начального этапа проектирования самолета. Классическая 
схема формирования облика самолета предполагает решение сле\-ду\-ющих 
задач~[3, 4]:
  \begin{enumerate}[(1)]
\item  анализ тактико-тех\-ни\-че\-ско\-го задания (ТТЗ) к~проектируемому изделию, 
поиск возможных прототипов и~выбор значений основных конструктивных 
параметров на основании анализа опыта разработки самолетов, близ\-ких по 
назначению и~уровню требований ТТЗ, и,~если это воз\-мож\-но, выбор 
реальных прототипов для будущего проекта;
  \item на основе проведенного анализа и~выбранных прототипов синтез 
компоновочной схемы самолета;
  \item  анализ компоновочной схемы, оценка выполнения требований ТТЗ 
и~оптимизация па\-ра\-мет\-ров компоновочной схемы.
  \end{enumerate}
  
  Анализ ТТЗ и~поиск прототипов проектируемого самолета должен быть 
обеспечен наличием пред\-ста\-ви\-тель\-ной информации о~серийных и~опытных 
самолетах текущего и~прош\-лых поколений. В~\mbox{АСВП} информация для поиска 
прототипов может накапливаться и~храниться в~базе данных  
СПМ  (далее БД) готовых изделий.

%\vspace*{-2pt}
  
\section{База данных структурно-параметрических моделей 
готовых изделий}

%\vspace*{-1pt}

  Ведение БД  готовых изделий~--- это отдельная\linebreak 
специальная функция \mbox{АСВП}. Информация для пополнения этой БД, как 
правило, берется из внеш\-них источников. Опыт работы с~такой информацией 
показывает, что основная проб\-ле\-ма в~\mbox{использовании} публичных данных 
состоит в~не\-об\-хо\-ди\-мости приведения их к~единой, непротиворечивой структуре и~терминологии. 
Для этого в~\mbox{АСВП} реализованы механизмы каталогизации 
и~классификации информационных объектов, которые пользователь может 
на\-стра\-и\-вать в~соответствии с~ре\-ша\-емы\-ми задачами. 

На более высоком уровне 
одним из механизмов упорядочения информации становится создание базовых, 
\textbf{типовых} СПМ, с~тем чтобы унифицировать со\-став, обозначения 
и~определения \mbox{конструктивных} па\-ра\-мет\-ров изделий, раз\-ме\-ща\-емых в~БД. 
Только такие унифицированные данные могут в~дальнейшем эффективно 
использоваться и~при анализе архивированной информации, и~при создании 
математических моделей новых про\-ек\-ти\-ру\-емых изделий.
  
  На рис.~1 представлен вид экрана ведения БД готовых изделий.
  
  \begin{figure*} %fig2
\vspace*{1pt}
      \begin{center}
     \mbox{%
\epsfxsize=100.912mm 
\epsfbox{fle-2.eps}
}
\end{center}
\vspace*{-9pt}
\Caption{Основные компоненты СПМ самолета}
\end{figure*}

    Понятие <<готовые изделия>> здесь толкуется в~расширительном смыс\-ле. 
Наряду с~моделями готовых, серийных изделий в~БД может вноситься 
информация об изделиях, которые находятся в~процессе изготовления. Атрибут 
модели <<Этап ЖЦ>> обозначает статус модели в~БД, который может быть 
обозначен как <<проектирование>>, <<исследования>>, <<испытания>>, 
<<эксплуатация>>, <<снят с~эксплуатации>>, <<типовая модель>>. Модель 
с~обозначением XXX, которая выведена в~первой строке таб\-ли\-цы на рис.~1, 
сформирована как <<типовая модель>>, которая может быть использована 
в~качестве шаблона при по\-стро\-ении новых моделей данного типа изделий. Для 
каждого типа авиационных изделий в~\mbox{АСВП} может быть создана своя типовая 
модель. 
%
Классификатор типов изделий по критерию назначения, который на 
рис.~1 представлен в~левой панели экрана, создается в~отдельном модуле 
\mbox{АСВП} пользователями сис\-те\-мы и~может отвечать конкретным задачам, 
решаемым на предприятии. В~БД могут создаваться модели не только 
самолетов, но и~других изделий, в~част\-ности авиационных двигателей 
и~сис\-тем оборудования ЛА, которые могут быть использованы при 
разработке моделей самолетов.

  


  Включение новой модели в~БД реализуется специальной командой модуля 
в~интерактивном диалоге или путем копирования другой модели, \mbox{чаще}
мо\-де\-ли-шаб\-ло\-на, находящейся в~БД. Содержательная информация об 
изделии, его структуре, конструктивных па\-ра\-мет\-рах и~характеристиках 
создается в~процессе разработки модели. На рис.~2 \mbox{показаны} основные 
компоненты СПМ самолета.
  
\section{Параметрические модели на~этапе формирования облика 
самолета}

  В публичном пространстве~--- в~различных авиационных справочниках 
и~энциклопедиях~--- цифровая информация о~самолетах сводится к~значениям 
основных параметров и~характеристик, которые дают общее пред\-став\-ле\-ние 
о~самолете, его размерах и~эксплуатационных характеристиках. Такие 
параметры можно определить как основные так\-ти\-ко-тех\-ни\-че\-ские 
характеристики (ТТХ), опре\-де\-ля\-ющие па\-ра\-мет\-ри\-че\-ский облик самолета. 
К~этим па\-ра\-мет\-рам относятся, во-пер\-вых, гео\-мет\-ри\-че\-ские размеры 
самолета~--- размах крыла~$L_{\mathrm{кр}}$, площадь крыла~$S$, габаритная 
длина фюзеляжа~$L_{\mathrm{ф}}$. Во-вто\-рых, это основные весовые 
па\-ра\-мет\-ры: расчетная взлетная масса самолета $M_{\mathrm{взл}}$, масса 
пустого изделия~$M_{\mathrm{пуст}}$, масса топлива~$M_{\mathrm{топ}}$, 
масса полезной на\-груз\-ки $M_{\mathrm{пол\_нагр}} \hm= M_{\mathrm{топ}}\hm + 
M_{\mathrm{нагр}}$. И~третья группа па\-ра\-мет\-ров~--- это  
лет\-но-тех\-ни\-че\-ские характеристики самолета: максимальная ско\-рость 
полета~$V_{\mathrm{макс}}$, практический потолок~$H_{\mathrm{макс}}$, 
крейсерская даль\-ность полета~$L_{\mathrm{крейс}}$, дистанция 
взлета~$L_{\mathrm{взл}}$ и~посадки~$L_{\mathrm{пос}}$, стартовая тяга 
двигателей силовой установки~$P_0$ и,~возможно, некоторые другие 
характеристики. Все эти и~другие известные из внешних источников 
па\-ра\-мет\-ры, опре\-де\-ля\-ющие облик самолета, могут быть включены в~СПМ 
готового изделия при создании этой модели в~БД.     
  
  Кроме этих основных параметров в~практике проектирования используются 
различные производные величины, относительные или удельные\linebreak 
конструктивные параметры, такие как безразмерная величина удлинения крыла 
$(L_{\mathrm{кр}}^2)/S$, относительная стартовая масса топлива 
$M_{\mathrm{топ}}/M_{\mathrm{взл}}$,\linebreak взлетная нагрузка на крыло 
$M_{\mathrm{взл}} \g/S$, взлетная тя\-го\-во\-ору\-жен\-ность 
$P_0/(M_{\mathrm{взл}} \g)$ и~другие па\-ра\-мет\-ры, вы\-чис\-ля\-емые по основным 
па\-ра\-мет\-рам ТТХ.\linebreak Удельные и~относительные параметры более консервативны, 
чем абсолютные величины, изменяются в~более узких диапазонах, которые 
в~определенной степени характеризуют тип самолета. Это \mbox{позволяет} более 
эффективно проводить сравнительный анализ при создании новых проектов.
  
  В АСВП пользователи при создании СПМ могут вводить любые па\-ра\-мет\-ры, 
описывающие изделие. При вводе нового па\-ра\-мет\-ра или при его модификации 
необходимо задать несколько атрибутов, которые показаны на рис.~3.
  
\setcounter{figure}{3}
\begin{figure*}[b] %fig4
\vspace*{-3pt}
      \begin{center}
     \mbox{%
\epsfxsize=95.758mm 
\epsfbox{fle-4.eps}
}
\end{center}
\vspace*{-9pt}
\Caption{Требования к~ТТХ самолета}
\end{figure*}

   Множество параметров СПМ может состоять из большого числа записей, 
поэтому с~по\-мощью двухуровневой нумерации мож\-но устанавливать 
удобный для обзора порядок следования па\-ра\-мет\-ров при их выводе в~списке. 
Идентифицирующие реквизиты па\-ра\-мет\-ров вводятся пользователем по 
своему усмот\-ре\-нию (соблюдая общепринятые правила,\linebreak\vspace*{-12pt}

{ \begin{center}  %fig3
 \vspace*{-3pt}
    \mbox{%
\epsfxsize=79mm 
\epsfbox{fle-3.eps}
}


\vspace*{6pt}



\noindent
{{\figurename~3}\ \ \small{Вводимые атрибуты па\-ра\-мет\-ров СПМ
}}
\end{center}
}

\vspace*{9pt}

%\addtocounter{figure}{1}

\noindent
 терминологию 
и~обозначения, что, впрочем, для обозначений, принятых в~литературе, не 
всегда возможно). Под статусом параметра здесь понимается способ 
получения и~использования его значений. Па\-ра\-мет\-ры могут быть 
задаваемыми пользователями, вы\-чис\-ля\-емы\-ми по формулам, варьируемыми 
при параметрических расчетах или трактуемыми как критерии при 
оптимизации СПМ. Для вы\-чис\-ля\-емых па\-ра\-мет\-ров в~поле  
<<Фор\-му\-ла/зна\-че\-ние>> вводится арифметическое выражение, 
ис\-поль\-зу\-емое в~качестве операндов, идентификаторы других па\-ра\-мет\-ров, 
встроенные функции и~константы. Для за\-да\-ва\-емых или варь\-и\-ру\-емых 
параметров задается чис\-ло\-вое значение. В~СПМ па\-ра\-мет\-ры ограничены 
своими минимальными и~максимальными чис\-ло\-вы\-ми значениями. Если были 
выявлены ошиб\-ки в~расчетных формулах или если значения па\-ра\-мет\-ра 
оказались вне диапазона до\-пус\-ти\-мых значений, то в~поле <<Комментарий>> 
выводится со\-от\-вет\-ст\-ву\-ющая диагностика.




  
\section{Анализ требований к~вновь создаваемым самолетам}

  В БД АСВП, как было сказано выше, может храниться не только архивная 
информация о существующих готовых изделиях авиационной техники, но 
и~могут создаваться модели новых про\-ек\-ти\-ру\-емых самолетов. В~отличие от 
готовых изделий при создании СПМ для новых проектов самолетов некоторые 
параметры, опре\-де\-ля\-ющие их ТТХ, которые для готовых изделий брались из 
внеш\-них источников, еще не известны, их еще предстоит вы\-чис\-лить в~процессе 
работы над моделью. Для новых проектов разработка СПМ в~\mbox{АСВП} начинается 
с~ввода требований к~тем ТТХ, 
которые сформулированы в~ТТЗ к~про\-ек\-ти\-ру\-емо\-му изделию. Состав и~уровень 
требований определяются на этапе внеш\-не\-го проектирования~\cite{2-fl}.
  
  На рис.~4 приведен типичный перечень характеристик, требования 
к~которым обычно предъявляются в~ТТЗ. Он отражает основные динамические 
характеристики ЛА и~его характеристики как средства транспортировки 
различных объектов целевой на\-груз\-ки. Однако современные самолеты могут 
выполнять задачи, выходящие за рамки простого ле\-та\-юще\-го транспортного 
средства. В~большинстве случаев специальные задачи, соответствующие 
назначению самолета, обеспечиваются наличием на его борту сис\-тем, 
ориентированных на выполнение этих задач. В~приведенных на рис.~4 
требованиях это учтено в~задании массы специального оборудования, перечень 
которого задается в~ТТЗ.
  


  Остальные параметры ТТХ самолета, такие как площадь несущей 
по\-верх\-ности, габаритные па\-ра\-мет\-ры, а~также взлетная масса, масса пус\-то\-го 
изделия, масса топлива и,~воз\-мож\-но, другие па\-ра\-мет\-ры, опре\-де\-ля\-ющие облик 
проектируемого изделия на момент создания, в~БД его модели неизвестны, если, 
конечно, модель не копирует какой-нибудь прототип. Их значения, а~точнее, 
бо\-лее-ме\-нее до\-сто\-вер\-ные диапазоны их возможных значений должны быть 
вы\-чис\-ле\-ны в~результате анализа требований ТТЗ.

\setcounter{figure}{4}
\begin{figure*} %fig5
\vspace*{1pt}
 \begin{center}  
    \mbox{%
\epsfxsize=110mm 
\epsfbox{fle-5.eps}
}
\end{center}
\vspace*{-9pt}
\Caption{Анализ массы пустого ЛА по выборке из БД готовых изделий
}
\end{figure*}
  
   В процессе анализа требований ТТЗ в~\mbox{АСВП} решаются две основные задачи. 
Первая задача~--- это сравнение требований ТТЗ нового проекта с~уровнем 
ТТХ современных серийных или опытных 
изделий, СПМ которых хранятся в~БД. Вмес\-те с~этой задачей решается задача 
выявления воз\-мож\-ных прототипов для начала проектирования нового самолета. 
Вторая задача, которая решается на этапе анализа требований ТТЗ и~которая 
может использовать результаты решения первой задачи,~--- определение 
диапазонов значений основных конструктивных па\-ра\-мет\-ров СПМ, в~пределах 
которых следует вести дальнейшие исследования по формированию облика 
будущего проекта. Это задача по\-стро\-ения об\-ласти существования ЛА 
в~пространстве па\-ра\-мет\-ров СПМ. В~процессе решения первой задачи в~\mbox{АСВП} 
формируется выборка из БД готовых изделий. Формирование выборок может 
вестись по разным критериям: по типу самолета, по производителю, по  
об\-ласти применения, по бли\-зости характеристик изделий выборки 
к~заявленным в~ТТЗ требованиям. Если ввести ка\-кую-ни\-будь меру такой 
бли\-зости, то она может служить критерием для по\-стро\-ения вы\-бор\-ки. Вы\-бор\-ки 
прототипов естественно использовать для определения диапазонов значений 
основных ТТХ, а~так\-же других конструктивных 
па\-ра\-мет\-ров, которые проектировщик использует при описании СПМ проекта.
  
  В практике проектирования самолетов нашло широкое применение 
использование информации о~готовых изделиях для по\-стро\-ения парных или 
многофакторных зависимостей между конструктивными па\-ра\-мет\-ра\-ми по 
выборкам прототипов проектируемого изделия. Чаще всего такие по\-стро\-ен\-ные 
зависимости используются для весового анализа при расчетах первого 
при\-бли\-же\-ния массы самолета. Приведем пример такого анализа средствами 
\mbox{АСВП}\footnote{В~приведенном и~в~других примерах настоящей статьи используются 
данные из тес\-то\-вой БД готовых изделий \mbox{АСВП}. Все эти данные брались из пуб\-лич\-ных 
источников интернета, при этом авторы допускают воз\-мож\-ность каких-либо неточностей 
в~цифровых значениях. Поэтому эти примеры не могут быть использованы в~реальном 
проектировании и~представлены исключительно в~демонстрационных целях.}. Основными 
параметрами ТТЗ, влияющими на весовые характеристики ЛА, служат масса 
целевой нагрузки~$M_{\mathrm{нагр}}$ и~крейсерская даль\-ность 
полета~$L_{\mathrm{крейс}}$, которая определяет необходимую массу 
за\-прав\-ля\-емо\-го топлива~$M_{\mathrm{топ}}$ и~которая в~первом при\-бли\-же\-нии 
должна быть вы\-чис\-ле\-на в~процессе анализа ТТЗ. На рис.~5 приведен пример 
анализа зависимостей массы пус\-то\-го самолета от массы полезной нагрузки по 
по\-стро\-ен\-ной выборке пассажирских самолетов. Полезная нагрузка самолета, 
определенная выше как сумма массы целевой нагрузки и~за\-прав\-ля\-емо\-го 
топлива, зависит от конкретного полетного задания. Для пассажирских 
самолетов имеет смысл рассматривать полетное задание в~режиме крейсерского 
полета с~пол\-ной за\-груз\-кой требуемого в~ТТЗ числа пассажиров. Для 
определения стартовой полезной на\-груз\-ки для крейсерского полета введем 
новый параметр $\mathrm{LM} \hm= L_{\mathrm{крейс}} M_{\mathrm{нагр}}$, отражающий 
характеристику выполненного задания по транспортировке целевой нагрузки 
по заданному маршруту. Этот параметр легко вычисл$\acute{\mbox{и}}$м по описаниям готовых 
изделий и~по ограничениям ТТЗ. Построив выборку из БД реально 
существующих самолетов, можно проанализировать, существует ли корреляция 
между~$\mathrm{LM}$ и~$M_{\mathrm{пол\_нагр}}$, по крайней мере на этой выборке. 
И~далее: существует ли связь между массой пустого самолета и~массой 
полезной на\-грузки.



  \setcounter{figure}{5}
\begin{figure*} %fig6
\vspace*{1pt}
      \begin{center}
     \mbox{%
\epsfxsize=160mm 
\epsfbox{fle-6.eps}
}
\end{center}
\vspace*{-9pt}
\Caption{Анализ зависимостей габаритов ЛА от массы полезной нагрузки}
\vspace*{3pt}
\end{figure*}
  

  На рис.~5 с~точностью до статистической погрешности такие связи 
прослеживаются, и~исходя из этого мож\-но построить эмпирические модели для 
определения первого при\-бли\-же\-ния \mbox{массы} пус\-то\-го самолета, расчетной 
взлетной массы, расчетной стартовой массы за\-прав\-ля\-емо\-го топ\-ли\-ва и~других 
весовых па\-ра\-мет\-ров самолета. Заметим, что масса раз\-ме\-ща\-емой на борту 
целевой нагрузки и~масса за\-прав\-ля\-емо\-го топлива напрямую влияют и~на 
параметры компоновочной схемы, в~част\-ности на ее габаритные размеры. 
Так, длина фюзеляжа пассажирских самолетов зависит от чис\-ла 
пассажиров (разумеется, и~от компоновки пассажирских салонов), от объема 
фюзеляжных топ\-лив\-ных баков.
  


  На рис.~6 приведены в~качестве примера результаты анализа по той же 
вы\-бор\-ке, что и~для весового анализа, зависимостей основных конструктивных 
па\-ра\-мет\-ров, опре\-де\-ля\-ющих гео\-мет\-ри\-че\-ские размеры пассажирского самолета, 
от того же па\-ра\-мет\-ра~--- массы полезной нагрузки.
  



  На графике явно прослеживается связь между массой полезной нагрузки 
  и~ана\-ли\-зи\-ру\-емы\-ми па\-ра\-мет\-ра\-ми. И~что немаловажно, по этим графикам могут 
быть получены реальные диапазоны па\-ра\-мет\-ров, в~рамках которых должны 
вестись дальнейшие исследования проекта.
  
  Данные примеры приведены для демонстрации того, как на ранних этапах 
проектирования весьма эффективно могут использоваться эмпирические 
модели, основанные на анализе данных по различным выборкам прототипов 
про\-ек\-ти\-ру\-емых изделий. На начальных стадиях проектирования диапазоны 
изменения па\-ра\-мет\-ров про\-ек\-ти\-ру\-емо\-го изделия определяются чаще всего 
исходя из опыта проектирования и~выбора одного или нескольких реально 
су\-щест\-ву\-ющих прототипов текущего проекта.
  
  Эмпирические зависимости, полученные в~результате анализа выбранной 
группы прототипов, подобные тем, которые приведены выше, могут быть 
использованы при синтезе компоновочной схемы.

\section{Параметрическая модель компоновочной схемы}

    Как уже говорилось выше, основная задача этапа формирования облика 
самолета~--- построение его компоновочной схемы. Под компоновочной 
схемой в~проектировании понимается отоб\-ра\-же\-ние общего вида самолета 
в~трех главных проекциях (см.\ рис.~2).

\begin{figure*} %fig7
\vspace*{1pt}
      \begin{center}
     \mbox{%
\epsfxsize=124.142mm 
\epsfbox{fle-7.eps}
}
\end{center}
\vspace*{-7pt}
\Caption{Основные параметры компоновочной схемы самолета}
\vspace*{5pt}
\end{figure*}

  При дальнейшей детализации в~компоновочную схему вносятся отображения 
двигателей силовой установки, модели которых могут создаваться и~храниться 
в~БД \mbox{АСВП}, а~так\-же модели \mbox{других} самолетных сис\-тем и~специального 
оборудования, экипажа, топ\-ли\-ва и~другой целевой на\-груз\-ки. В~\mbox{АСВП} 
реализовано традиционное отображение компоновочной схемы в~тех деталях, 
которые описаны в~струк\-тур\-но-па\-ра\-мет\-ри\-че\-ской модели. Но 
<<рисование>> компоновочной схемы~--- это не только дань традиционным 
взглядам на процесс проектирования самолетов, это важный механизм 
визуального контроля корректности задания чис\-ло\-вых па\-ра\-мет\-ров СПМ. 
В~СПМ, реализованной в~\mbox{АСВП}, 
компоновочная схема пред\-став\-ле\-на деревом конструкции и~набором 
па\-ра\-мет\-ров, опре\-де\-ля\-ющих гео\-мет\-рию, конструкцию и~взаимное расположение 
агрегатов.\linebreak Дерево конструкции~--- это иерархическая структура, за\-да\-ющая 
отношение включения между конструктивными элементами изделия, которые 
служат узлами этой иерархии. Для большинства \mbox{современных} типов самолетов 
верхние уровни дерева конструкции стандартны и~состоят из основных 
агрегатов и~сис\-тем~--- крыла, фюзеляжа, оперения, взлет\-но-по\-са\-доч\-ных 
устройств, бортового оборудования и~силовой установки со своими сис\-те\-ма\-ми, 
в~том числе двигателями и~топливными баками. Именно эти агрегаты, их 
параметры и~их размещение определяют компоновочную схему самолета. 

Принято выделять несколько аспектов конструкции компоновочной схемы, 
опре\-де\-ля\-ющих ее тип. Наиболее существенную роль в~компоновке играет тип 
балансировочной схемы ЛА, который характеризует размещение относительно 
крыла агрегатов оперения, наличие заднего и~переднего горизонтального 
оперения, чис\-ло килей и~другие детали компоновки несущих поверхностей. 
Форма основной несущей по\-верх\-ности (крыла), наличие развитых передних 
наплывов крыла так\-же влияют на положение аэродинамического фокуса 
и,~следовательно, на балансировку ЛА. Традиционное членение фюзеляжа на 
головную, цент\-раль\-ную и~хвос\-то\-вую части имеет свои особенности и~иногда 
встречает за\-труд\-не\-ние в~классификации. Например, для самолетов так 
называемых интегральных схем с~плав\-ным переходом конструкции от 
фюзеляжа к~крылу в~за\-ви\-си\-мости от значений па\-ра\-мет\-ров корневую часть 
несущей по\-верх\-ности мож\-но относить к~фюзеляжу или к~цент\-ро\-пла\-ну. На 
рис.~7 приведен список па\-ра\-мет\-ров компоновочной схемы самолета, 
реализованный средствами \mbox{АСВП} в~рамках одного из воз\-мож\-ных сценариев 
по\-стро\-ения СПМ изделия.



  Сценарий построения компоновочной схемы предполагает в~первую очередь 
выбор балансировочной схемы самолета и~расположение крыла\linebreak относительно 
фюзеляжа по высоте. Под типом балансировочной схемы понимают 
расположение основной несущей по\-верх\-ности относительно оперения 
в~продольной оси самолета. Выбор балансировочной схемы определяет способ 
управ\-ле\-ния\linebreak положением самолета в~продольном канале. Отклонением органов 
управ\-ле\-ния горизонтального оперения создаются каб\-ри\-ру\-ющие или 
пи\-ки\-ру\-ющие моменты, и~самолет изменяет свою траекторию\linebreak в~вертикальной 
плос\-кости. Существуют <<нормальная>> балансировочная схема~--- когда 
горизонтальное оперение располагается в~хвос\-то\-вой час\-ти \mbox{самолета}, позади 
крыла, и~схема <<утка>>, пред\-по\-ла\-га\-ющая размещение горизонтального 
оперения в~зоне носовой час\-ти фюзеляжа со смещенным назад крылом. Иногда 
применяется смешанная схема~--- с~зад\-ним и~передним горизонтальным 
оперением, а~иногда вообще горизонтального оперения нет, и~самолет 
балансируется с~пом\-ощью механизации крыла~--- такую схему называют 
<<бесхвостка>>.

 Выбор типа балансировочной схемы в~определенной степени 
влияет на параметрическое задание основной несущей по\-верх\-ности~--- крыла, 
и~в~первую очередь на так называемую <<базовую трапецию>>. Базовая 
трапеция крыла~--- это не одна, а~две сим\-мет\-рич\-ные относительно оси 
самолета трапеции с~корневой хордой, лежащей на этой оси. 

Для задания 
формы в~плане базовой трапеции достаточно четырех основных па\-ра\-мет\-ров: 
размаха базовой трапеции~$L_{\mathrm{бт}}$, длины корневой хор\-ды~$B_0$, длины концевой 
формы~$B_{\mathrm{к}}$ и~угла стре\-ло\-вид\-ности по передней кромке $\mathrm{Hi}_{\mathrm{бт}}$. Наряду с~этими 
базовыми па\-ра\-мет\-ра\-ми задают площадь базовой трапеции~$S_{\mathrm{бт}}$, 
удлинение~$\mathrm{La}_{\mathrm{бт}}$ и~сужение~$\mathrm{Eta}_{\mathrm{бт}}$ базовой трапеции. Па\-ра\-мет\-ры базовой трапеции связаны между 
собой соотношениями: 
$$
S_{\mathrm{бт}}=\fr{(B_0+B_{\mathrm{к}})L_{\mathrm{бт}}}{2}\,; \enskip
\mathrm{La}_{\mathrm{бт}}m=\fr{L^2_{\mathrm{бт}}}{S_{\mathrm{бт}}}\,;\enskip 
\mathrm{Eta}_{\mathrm{бт}}=\fr{B_0}{B_{\mathrm{к}}}\,.
$$
 Положение базовой трапеции 
задается координатами носка корневой хорды~$X_{\mathrm{бт}}$ 
и~$Y_{\mathrm{бт}}$.




  
  Полная модель крыла существенно сложнее и~включает дополнительно еще 
целый ряд па\-ра\-мет\-ров, которые могут задаваться или вы\-чис\-лять\-ся по 
формулам, за\-да\-ва\-емым при определении этих па\-ра\-мет\-ров. У~современных 
самолетов час\-то передняя кромка крыла имеет слож\-ную нелинейную форму. 
Это характерно для так на\-зы\-ва\-емых <<интегральных>> компоновочных схем 
с~плав\-ным сочленением крыла и~фюзеляжа. В~рамках \mbox{АСВП} такие 
компоновочные схемы аппроксимируются слож\-ным со\-став\-ным крылом  
с~ку\-соч\-но-ли\-ней\-ны\-ми передней и~задней кромками. На рис.~8 показана 
форма в~плане крыла, со\-сто\-яще\-го из отъемной час\-ти крыла (ОЧК) и~корневой 
части.


  
  \setcounter{figure}{8}
\begin{figure*}[b] %fig9
\vspace*{1pt}
      \begin{center}
     \mbox{%
\epsfxsize=160mm 
\epsfbox{fle-9.eps}
}
\end{center}
\vspace*{-9pt}
\Caption{Параметрическая модель крыла}
\end{figure*}

{ \begin{center}  %fig8
 \vspace*{-6pt}
     \mbox{%
\epsfxsize=79mm 
\epsfbox{fle-8.eps}
}

\vspace*{6pt}



\noindent
{{\figurename~8}\ \ \small{Форма в~плане крыла
}}
\end{center}
}

\vspace*{6pt}


Корневую часть крыла иногда называют цент\-ро\-пла\-ном. Однако, чтобы не 
путать ее с~\textit{подфюзеляжным} цент\-ро\-пла\-ном, служащим частью силовой 
конструкции, будем называть корневую часть крыла \textit{околофюзеляжным} 
центропланом.


  Параметры составного крыла кроме па\-ра\-мет\-ров базовой трапеции включают 
углы стре\-ло\-вид\-ности переднего и~заднего на\-плы\-вов ОЧК $(\mathrm{Hi}_{\mathrm{пн}}$ 
и~$\mathrm{Hi}_{\mathrm{зн}})$, координаты точек излома у~передней и~зад\-ней кромки ОЧК 
$(Z_{\mathrm{пн}}$ и~$Z_{\mathrm{зн}})$, а~также па\-ра\-мет\-ры 
\textit{околофюзеляжного} центроплана $(Z_\mathrm{б}$, $Z_{\mathrm{цп}}$, 
$B_{\mathrm{цп}}$, $\mathrm{Hi}_{\mathrm{цп}}$ и~$\mathrm{Hi}_{\mathrm{зцп}})$. Заметим, что на рис.~8 
па\-ра\-мет\-ры $\mathrm{Hi}_{\mathrm{зн}}$ и~$\mathrm{Hi}_{\mathrm{зцп}}$ не обозначены и~имеют 
нулевые значения.
  
  Кроме параметров, опре\-де\-ля\-ющих форму крыла в~плане, для многих 
компоновочных задач и~задач весового проектирования необходимо задать 
па\-ра\-мет\-ры профиля крыла и~положение его в~пространстве. Положение крыла 
по высоте относительно фюзеляжа определяется в~за\-ви\-си\-мости от типа 
компоновки крыла~--- центроплан, высокоплан, низкоплан. Однако такое 
определение в~некоторых случаях требует уточнения, т.\,е.\ задания координаты 
$Y_{\mathrm{кр}}$ и~угла V-образ\-ности крыла~---  угла поворота 
плоскости крыла вокруг продольной оси самолета относительно плос\-кости 
земли в~стояночном положении. Что касается профиля крыла, то он может 
выбираться из биб\-лио\-тек стандартных профилей. Основной параметр профиля, 
ис\-поль\-зу\-емый в~\mbox{АСВП}, от которого зависят многие компоновочные и~весовые 
характеристики крыла,~--- это его относительная толщина. Этот параметр 
может задаваться отдельно для ОЧК и~околофюзеляжного центроплана. 
Приведенные па\-ра\-мет\-ры крыла и~соотношения между ними представляют 
собой в~\mbox{АСВП} па\-ра\-мет\-ри\-че\-скую модель крыла, фрагмент которой приведен на 
рис.~9.
  


    Аналогичным образом строятся модели других агрегатов компоновочной 
схемы~--- фюзеляжа, оперения, шасси, силовой установки. Фюзеляж 
в~реализованном сценарии по\-стро\-ения компоновочной схемы задается своей 
длиной, миделем, а~так\-же удли\-не\-ни\-ями и~габаритными размерами сечений 
головной и~хвос\-то\-вой частей. Вертикальное и~горизонтальное оперение, 
включая переднее горизонтальное оперение, задаются аналогично крылу 
своими базовыми трапециями и~координатами их размещения, привязанными 
к~самолетной сис\-те\-ме координат.

\begin{figure*} %fig10
\vspace*{1pt}
      \begin{center}
     \mbox{%
\epsfxsize=160mm 
\epsfbox{fle-10.eps}
}
\end{center}
\vspace*{-9pt}
\Caption{Варианты компоновочных схем ЛА}
\end{figure*}

  Модели синтезируемых в~\mbox{АСВП} компоновочных схем ЛА позволяют 
перейти к~построению \textbf{па\-ра\-мет\-ри\-че\-ских} трехмерных каркасных 
габаритных моделей про\-ек\-ти\-ру\-емых изделий. Трехмерные каркасные 
габаритные модели ЛА и~его агрегатов на начальных этапах проектирования 
поз\-во\-ля\-ют более точ\-но вы\-чис\-лять характерные площади поверхностей 
и~располагаемые объемы этих агрегатов~--- па\-ра\-мет\-ры, существенным 
образом влияющие на их мас\-со\-во-инер\-ци\-он\-ные характеристики. Кроме\linebreak 
того, по каркасным моделям можно аппроксимировать трехмерные 
поверхности, создавая па\-ра\-мет\-ри\-че\-ское трехмерное пред\-став\-ле\-ние 
про\-ек\-ти\-ру\-емо\-го изделия, перестройка которого не требует больших 
вы\-чис\-ле\-ний. Такое пред\-став\-ле\-ние служит не только эффектной иллюстрацией 
этапа формирования облика, но и~поз\-во\-ля\-ет решать такие задачи, как 
по\-стро\-ение распределенных мас\-со\-во-инер\-ци\-он\-ных характеристик конструкции 
ЛА на начальных этапах проектирования, когда полноценного электронного 
макета изделия еще не существует.
  


  На рис.~10 показаны различные варианты параметрических компоновочных 
схем, по\-стро\-ен\-ных в~\mbox{АСВП}, и~переход от них к~3D-ап\-прок\-си\-ма\-ци\-ям 
облика самолета. Эти модели позволяют решать\linebreak задачи формирования облика 
достаточно широкого %\linebreak 
класса самолетов разного назначения и~предостав\-ля\-ют 
достаточно информации, чтобы на инженерном уровне проводить анализ 
основных \mbox{характеристик} ЛА и~оценивать, может ли данная компоновочная 
схема быть взята за основу для дальнейшей проработки проекта. Задачи 
анализа компоновочных схем требуют построения другого класса моделей, 
в~которых затрагиваются такие области знаний, как аэродинамика, динамика 
полета, тео\-рия двигателей и~др. На начальных этапах проектирования 
необходима разработка инженерных упрощенных моделей для при\-бли\-жен\-ных 
расчетов, которые поз\-во\-ля\-ли бы решать на этих этапах задачи оптимизации 
компоновочных схем са\-мо\-летов.
  
\section{Заключение}

  Описанные в~статье сценарии по\-стро\-ения 
СПМ компоновочных схем самолетов не являются единственно 
воз\-мож\-ны\-ми. Автоматизированная сис\-те\-ма весового проектирования предоставляет гиб\-кий инструмент для по\-стро\-ения моделей 
широкого класса проектов самолетов. И~не только \mbox{самолетов}. Она служит 
своего рода демонстрацией технологии <<Генератора проектов>>~\cite{5-fl}, 
применявшейся при создании многих ин\-фор\-ма\-ци\-он\-но-вы\-чис\-ли\-тель\-ных сис\-тем. 
Разработке этой технологии много лет посвятил ее создатель, коллега 
и~товарищ авторов \mbox{статьи}, Николай Иванович Широков, недавно ушедший из 
жизни.
  
{\small\frenchspacing
 { %\baselineskip=10.6pt
 %\addcontentsline{toc}{section}{References}
 \begin{thebibliography}{9}
\bibitem{1-fl}
\Au{Вышинский Л.\,Л., Флеров~Ю.\,А., Ши\-ро\-ков~Н.\,И.} Автоматизированная 
система весового проектирования самолетов~// Информатика и~её применения, 
2018. Т.~12. Вып.~1. С.~18--30. doi: 10.14357/19922264180103. EDN: YTTRBQ.
\bibitem{2-fl}
\Au{Шейнин В.\,М., Козловский~В.\,И.} Весовое проектирование 
и~эф\-фек\-тив\-ность пассажирских самолетов.~---  М.: Машиностроение, 1977. 
Т.~1. 343~с.

\bibitem{4-fl} %3
\Au{Егер С.\,М., Лисейцев~И.\,К., Са\-мой\-ло\-вич~О.\,С.} Осно\-вы 
автоматизированного проектирования самолетов.~--- М.: Машиностроение, 
1986. 232~c.

\bibitem{3-fl} %4
\Au{Вышинский Л.\,Л., Самойлович~О.\,С., Фле\-ров~Ю.\,А.} Программный 
комплекс формирования облика летательных аппаратов~// Задачи и~методы 
автоматизированного проектирования в~авиа\-стро\-ении.~--- М.: ВЦ АН СССР, 
1991. С.~24--42.

\bibitem{5-fl}
\Au{Вышинский Л.\,Л., Гринев~И.\,Л., Флеров~Ю.\,А., Широков~А.\,Н., 
Широков~Н.\,И.} Генератор проектов~--- инструментальный комплекс для 
разработки <<кли\-ент-сер\-вер\-ных>> сис\-тем~// Информационные 
технологии и~вы\-чис\-ли\-тель\-ные сис\-те\-мы, 2003. №\,1-2. С.~6--25.
\end{thebibliography}

 }
 }

\end{multicols}

\vspace*{-6pt}

\hfill{\small\textit{Поступила в~редакцию 28.09.23}}

%\vspace*{8pt}

%\pagebreak

\newpage

\vspace*{-28pt}

%\hrule

%\vspace*{2pt}

%\hrule



\def\tit{SYNTHESIS MODELS OF~LAYOUT SCHEME IN~THE~TASK~OF~FORMING 
AN~AIRCRAFT~IMAGE}


\def\titkol{Synthesis models of layout scheme in~the~task of~forming 
an~aircraft image}


\def\aut{L.\,L.~Vyshinsky and~Yu.\,A.~Flerov}

\def\autkol{L.\,L.~Vyshinsky and~Yu.\,A.~Flerov}

\titel{\tit}{\aut}{\autkol}{\titkol}

\vspace*{-10pt}


\noindent
Federal Research Center ``Computer Science and Control'' of the Russian Academy of 
Sciences, 44-2~Vavilov Str., Moscow 119333, Russian Federation

\def\leftfootline{\small{\textbf{\thepage}
\hfill INFORMATIKA I EE PRIMENENIYA~--- INFORMATICS AND
APPLICATIONS\ \ \ 2024\ \ \ volume~18\ \ \ issue\ 1}
}%
 \def\rightfootline{\small{INFORMATIKA I EE PRIMENENIYA~---
INFORMATICS AND APPLICATIONS\ \ \ 2024\ \ \ volume~18\ \ \ issue\ 1
\hfill \textbf{\thepage}}}

\vspace*{3pt}
     
  
      
      
      \Abste{The article discusses the problems of synthesizing mathematical models of an aircraft 
at the initial design stage. The central task of the initial design stage is to develop 
a~structural-parametric model of the product and construct its layout diagram. The structural-parametric model 
of the product describes the space of the main design parameters of the aircraft and the layout 
diagram represents its structure, the set of main units and systems of the aircraft, their relative 
arrangement, and structural connections between the units. The layout diagram of the aircraft serves 
as the basis for further detailing of the project. At the initial design stage, the permissible ranges of 
values of the design parameters of the units of the layout diagram are determined in which the 
search for optimal design solutions is carried out. The paper presents a~description of 
structural-parametric models and layout diagrams of a~wide class of aircraft which can be implemented using 
an automated weight design system.}
      
      \KWE{mathematical modeling; design automation; aircraft; weight design; project 
generator}
      
\DOI{10.14357/19922264240109}{DSPGKV}

%\vspace*{-12pt}

%\Ack
%\vspace*{-4pt}
%\noindent



  \begin{multicols}{2}

\renewcommand{\bibname}{\protect\rmfamily References}
%\renewcommand{\bibname}{\large\protect\rm References}

{\small\frenchspacing
 {%\baselineskip=10.8pt
 \addcontentsline{toc}{section}{References}
 \begin{thebibliography}{9} 
\bibitem{1-fl-1}
     \Aue{Vyshinskiy, L.\,L., Yu.\,A.~Flerov, and N.\,I.~Shirokov.} 2018. Av\-to\-ma\-ti\-zi\-ro\-van\-naya 
sis\-te\-ma ve\-so\-vo\-go pro\-ek\-ti\-ro\-va\-niya sa\-mo\-le\-tov [Automated system of weight design of aircraft]. 
\textit{Informatika i~ee Primeneniya~--- Inform Appl}. 12(1):18--30. doi: 
10.14357/19922264180103. EDN: YTTRBQ.
\bibitem{2-fl-1}
     \Aue{Sheynin, V.\,M., and V.\,I.~Kozlovskiy.} 1977. \textit{Ve\-so\-voe pro\-ek\-ti\-ro\-va\-nie 
i~ef\-fek\-tiv\-nost' pas\-sa\-zhir\-skikh sa\-mo\-le\-tov} [Weight design and efficiency of passenger aircraft]. 
Moscow: Mashinostroenie. Vol.~1. 343~p.

\bibitem{4-fl-1} %3
     \Aue{Eger, S.\,M., I.\,K.~Liseytsev, and O.\,S.~Samoylovich.} 1986. \textit{Osnovy 
av\-to\-ma\-ti\-zi\-ro\-van\-no\-go pro\-ek\-ti\-ro\-va\-niya sa\-mo\-le\-tov} [Fundamentals of aircraft automated design]. 
Moscow: Mashinostroenie. 232~p.

\bibitem{3-fl-1} %4
     \Aue{Vyshinsky, L.\,L., O.\,S.~Samoylovich, and Yu.\,A.~Flerov.} 1991. Prog\-ram\-mnyy 
kom\-pleks for\-mi\-ro\-va\-niya ob\-li\-ka le\-ta\-tel'\-nykh ap\-pa\-ra\-tov [Program complex for forming the 
appearance of aircraft]. \textit{Zadachi i~metody avtomatizirovannogo proektirovaniya 
v~aviastroenii} [Tasks and methods of computer-aided design in aircraft industry]. Moscow: Computing Center of the
USSR Academy of Sciences. 24--42.

\bibitem{5-fl-1}
     \Aue{Vyshinskiy, L.\,L., I.\,L.~Grinev, Yu.\,A.~Flerov, A.\,N.~Shirokov, and 
N.\,I.~Shirokov.} 2003. Ge\-ne\-ra\-tor pro\-ek\-tov~--- ins\-tru\-men\-tal'\-nyy komp\-leks dlya raz\-ra\-bot\-ki  
``klient-servernykh'' sis\-tem [The project generator~--- tool complex for development of  
``client--server'' systems]. \textit{In\-for\-ma\-tsi\-on\-nye tekh\-no\-lo\-gii i~vy\-chis\-li\-tel'\-nye sis\-te\-my} [J.~Information 
Technologies Computing Systems] 1-2:6--25.
      
     \end{thebibliography}

 }
 }

\end{multicols}

\vspace*{-6pt}

\hfill{\small\textit{Received September 28, 2023}} 

\vspace*{-18pt}
 
      \Contr

%\vspace*{-3pt}
      
  \noindent
  \textbf{Vyshinsky Leonid L.} (b.\ 1941)~--- Candidate of Science (PhD) in physics and 
mathematics, senior scientist, Federal Research Center ``Computer Science and Control'' of the 
Russian Academy of Sciences, 44-2~Vavilov Str., Moscow 119333, Russian Federation; 
\mbox{wyshinsky@mail.ru} 
  
  \vspace*{3pt}
  
  \noindent
  \textbf{Flerov Yuri A.} (b.\ 1942)~--- Corresponding Member of the Russian Academy of 
Sciences, Doctor of Science in physics and mathematics, professor, principal scientist, Federal 
Research Center ``Computer Science and Control'' of the Russian Academy of Sciences, 44-2~Vavilov 
Str., Moscow 119333, Russian Federation; \mbox{fler@ccas.ru}
     
\label{end\stat}

\renewcommand{\bibname}{\protect\rm Литература} 
      