\def\stat{krivenko}

\def\tit{СТАТИСТИЧЕСКИЙ КРИТЕРИЙ СТАБИЛЬНОСТИ СИСТЕМЫ~МАССОВОГО~ОБСЛУЖИВАНИЯ, ОСНОВАННЫЙ~НА~ВХОДНОМ~И~ВЫХОДНОМ ПОТОКАХ}

\def\titkol{Статистический критерий стабильности СМО, %системы массового обслуживания, 
основанный на~входном и~выходном потоках}

\def\aut{М.\,П.~Кривенко$^1$}

\def\autkol{М.\,П.~Кривенко}

\titel{\tit}{\aut}{\autkol}{\titkol}

\index{Кривенко М.\,П.}
\index{Krivenko M.\,P.}


%{\renewcommand{\thefootnote}{\fnsymbol{footnote}} \footnotetext[1]
%{Работа выполнялась с использованием инфраструктуры Центра 
%коллективного пользования <<Высокопроизводительные вычисления и большие данные>>
%(ЦКП <<Информатика>>) ФИЦ ИУ РАН.}}


\renewcommand{\thefootnote}{\arabic{footnote}}
\footnotetext[1]{Федеральный исследовательский центр <<Информатика и управление>> Российской академии наук, 
\mbox{mkrivenko@ipiran.ru}}

%\vspace*{-12pt}




      \Abst{Одним из базовых свойств системы массового обслуживания (СМО) считается 
стабильность~--- способность системы функционировать, сохраняя во времени неизменной 
свою структуру и~характеристики. Рассматривается проблема статистической проверки 
стабильности СМО на основе характеристик входного $A(t)$ и~выходного $D(t)$ потоков 
заявок. Подтверждение стабильности основывается на установлении равенства интенсивностей 
этих потоков. Таким образом на языке статистического анализа данных получаем классическую 
задачу сравнения интенсивностей потоков. Для её решения период наблюдений делится на 
фрагменты, которые дают отдельные оценки. Вместе они составляют выборку, которая 
участвует в~сравнении. При анализе стабильности необходимо учитывать возможную 
зависимость $A(t)$ и~$D(t)$, поэтому приходится обращаться к~методам обработки так 
называемых согласованных пар (matched pairs) наблюдений. Контроль стабильности вынуждает 
решать ряд вспомогательных задач: подбор объемных параметров оценивания интенсивностей, 
проверку нормальности распределения, анализ корреляций. В~ходе экспериментов с~реальной 
системой выявлен ряд особенностей: эффект подмены допредельного распределения реальным 
при фрагментировании, наличие зависимости оценок интенсивностей анализируемых потоков, 
сходящая на нет для нестабильных систем.}
      
      \KW{СМО (система массового обслуживания); ста\-биль\-ность выборочной траектории; 
критерии для согласованных пар; тес\-ты муль\-ти\-нор\-маль\-ности}
      
      
      \DOI{10.14357/19922264240108}{JNJJMU}
  
%\vspace*{-6pt}


\vskip 10pt plus 9pt minus 6pt

\thispagestyle{headings}

\begin{multicols}{2}

\label{st\stat}

\section{Введение}

     Одним из базовых свойств СМО 
считается ста\-биль\-ность~--- спо\-соб\-ность сис\-те\-мы функционировать, сохраняя во 
времени неизменной свою структуру и~характеристики. 
     
     Статья посвящена постановке задачи статистического контроля 
ста\-биль\-ности функционирования сис\-тем обслуживания как задачи анализа данных 
при предположениях общего характера, когда исчерпываются возможности 
аналитических методов исследования сис\-те\-мы, а~также выбору со\-от\-вет\-ст\-ву\-ющих 
методов вместе с~выявлением возможных проб\-лем их применения.
     
     Термин <<стабильность>> имеет много значений в~различных контекстах. 
Подобная расплывчатость позволяет охватить ряд сходных, но далеко не 
идентичных поведений исследуемых процессов\linebreak и~рассматриваемых моделей, 
а~также решать общие вопросы структуры системы и~устойчивости ее\linebreak 
функционирования в~удивительно широких усло\-ви\-ях. С~практической точки 
зрения сис\-те\-ма, \mbox{которая} стабильна в~соответствии с~одним определением, скорее 
всего, будет стабильной в~соответствии с~другим определением, но не надо 
забывать, что не\-чет\-кость определений ста\-биль\-ности, нюансы описания принятых 
моделей, различия в~способах получения результатов могут приводить к~несхожим 
условиям ста\-биль\-ности (для примера мож\-но сравнить~[1] и~[2]).
     
     Далее обсуждается проблема статистического контроля стабильности СМО 
на основе характеристик входных и~выходных потоков заявок. При этом 
содержание и~организация процесса обслуживания либо неизвестны, либо известны и~допускают воз\-мож\-ность воспроизведения в~виде статистического 
моделирования, но не доступны с~точки зрения теоретических исследований. Для 
каждого неотрицательного времени~$t$ пусть $A(t)$ обозначает чис\-ло 
поступивших заявок (входной процесс), а~$D(t)$~--- покинувших сис\-те\-му 
(выходной процесс). Тогда общее число заявок в~сис\-те\-ме~$N(t)$ рав\-но 
$$
N(t)=  N(0)+A(t)-D(t).
$$
 Предполагается доступность реализации перечисленных 
процессов для заданного интервала времени $[0,T]$. 
     
     В последние годы со стороны исследователей и~специалистов, ра\-бо\-та\-ющих в~области стохастических процессов и~их приложений, наблюдается значительный 
интерес к~ориентированному на выборку анализу. Изучение свойств одной 
выборочной траектории (sample path~--- SP) стохастического процесса час\-то 
приводит к~лучшему и~более глубокому пониманию свойств ис\-сле\-ду\-емой 
сис\-те\-мы. Оно также предоставляет спе\-циа\-ли\-стам-прак\-ти\-кам мощный 
инструмент для определения того, какие свойства данной сис\-те\-мы не зависят от 
обычно навязываемых вероятностных предположений. 

\vspace*{-4pt}

\section{Определение стабильности}

\vspace*{-1pt}
     
     В литературе были предложены различные определения ста\-биль\-ности, 
     в~основном в~контексте сто\-ха\-сти\-че\-ских моделей (см., например, обзор этих 
определений в~\cite[приложение~C]{3-kri}). Особое мес\-то занимают определения, 
которые не зависят от конкретных сто\-ха\-сти\-че\-ских предположений и~формируются 
в~рам\-ках SP-ана\-ли\-за. Следуя разд.~5 работы~\cite{3-kri}, дадим
     
     \smallskip
     
     \noindent
     \textbf{Определение.} Вход\-ной/вы\-ход\-ной процесс $N(t)$, $t\hm\geq 0$, 
называется R-ста\-биль\-ным (Rate-ста\-биль\-ность), если 
     $$
     t^{-1} N(t)\to 0\ \mbox{при}\ t\to \infty\,.
     $$
     
     \vspace*{-3pt}
     
     \noindent
Из определения процесса $N(t)$ следует: при условии, что $t^{-1}A(t)\hm\to 
\lambda_A\hm< \infty$, $t\hm\to \infty$, он R-ста\-би\-лен тогда и~только тогда, 
когда $t^{-1}D(t)\hm\to \lambda_D \hm< \infty$ и~$\lambda_D\hm= \lambda_A$. 

\smallskip
     
     Таким образом, стабильность присутствует, если долгосрочные скорости 
по\-ступ\-ле\-ния и~убытия заявок существуют, конечны и~рав\-ны, т.\,е.\ находятся 
в~равновесии. Можно сформулировать ряд эквивалентных по\-ста\-но\-вок задачи 
под\-тверж\-де\-ния ста\-биль\-ности некоторой СМО, из которых как основной далее 
рас\-смат\-ри\-ва\-ет\-ся вариант анализа $A(t)$ и~$D(t)$. После придания стохастического 
характера задаче установления стабильности появляется воз\-мож\-ность оценить 
обосно\-ван\-ность при\-ни\-ма\-емых решений. При этом на языке статистического 
анализа данных получаем классическую задачу сравнения интенсивностей 
потоков. 
     
     Предполагается, что реализации процессов $A(t)$ и~$D(t)$ доступны 
в~любом требуемом объеме. Но если ограничения на процесс по\-ступ\-ле\-ния заявок 
обычно входят в~условия задачи контроля стабильности и~их можно считать 
заданными, то относительно существования ско\-рости убытия заявок и~ее 
конечности приходится рас\-смат\-ри\-вать два случая: характеристики выходного 
процесса известны или нет. Априорные данные появляются, когда исследуемая 
СМО изучена тео\-ре\-ти\-че\-ски достаточно полно и~факт существования и~конечности 
предела $t^{-1} D(t)$ при $t\hm\to\infty$ установлен. Здесь речь идет 
о~классических результатах тео\-ре\-мы Берка~\cite{4-kri}, о~примерах из обзора 
типа~\cite{5-kri}, а~также о~множестве отдельных публикаций. Более интересным с~прикладной точки зрения представляется случай, когда полной информации об 
интенсивностях выходного потока нет. В~этом случае анализ ста\-биль\-ности 
сис\-те\-мы необходимо предварять исследованием поведения $A(t)$ и~$D(t)$ при 
растущих значениях~$t$. 
     
     Здесь сразу же надо пояснить: если есть возможность проводить 
наблюдения сколь угодно долго, то сделать вывод о~ста\-биль\-ности можно на 
основании того, что обычно асимптотически распределение оценки 
ин\-тен\-сив\-ности~$\tilde{\lambda}_B$ для некоторого процесса~$B$ оказывается 
нормальным и~схо\-ди\-мость относительно чис\-ла событий~$n$ имеет вид:
     $$
     \sqrt{n}\left( \tilde{\lambda}_B-\lambda_B\right) \overset{d}{\to} 
N \!\left(0,\sigma^2_B\right).
     $$
Тогда за счет выбора~$n$ всегда можно однозначно определиться относительно 
значимого отличия интенсивностей входного и~выходного потоков. Данный 
результат остается в~силе по крайне мере для стационарных и~эргодических 
процессов~\cite{6-kri}. Поэтому сравнение интенсивностей потоков пред\-став\-ля\-ет 
интерес с~учетом сокращения объема использованного статистического материала.
     
     Пожалуй, единственное систематическое рассмотрение со\-от\-вет\-ст\-ву\-ющих 
методов проведено в~уже дав\-ниш\-ней работе~\cite{7-kri}. Однако использовать их 
непосредственно не получается. Во-пер\-вых,\linebreak достаточно подробный разбор 
сравнения пуассоновских процессов малоинтересен: если подобная ситуация 
и~возникает при выявлении ста\-биль\-ности, то для тривиальных СМО; поэтому ее 
предназначение~--- верификация раз\-ра\-ба\-ты\-ва\-емо\-го программного обеспечения 
со\-по\-став\-ле\-ния выборок.\linebreak Непуассоновский случай проанализирован кратко, 
причем для процессов восстановления, а~так\-же, что очень важ\-но в~случае анализа 
входных и~выходных потоков, в~рус\-ле задачи проверки од\-но\-род\-ности двух 
независимых выборок, что не обязательно встречается реально. Хотя сами по себе 
методы, опи\-ра\-ющи\-еся на большие выборки, в~совокупности с~понятием  
R-ста\-биль\-ности, носящим асимптотический характер, могут стать 
пло\-до\-твор\-ны\-ми. В~этой связи важны результаты разд.~4.5 и~4.6 работы~[7], 
справедливые для многих стационарных точечных процессов при $t\hm\to \infty$. 
Если некоторый стационарный целочисленный процесс обозначить как $B(t)$, 
а~его характеристики второго порядка через $M_B(t)$ и~$V_B(t)$, то для индекса 
дисперсии $I_B(t)\hm= V_B(t)/M_B(t)$ получим асимптотическое равенство 
$I_B(t)\simeq I_B$. В~результате приходим к~практически важ\-ным следствиям:

\noindent
     \begin{enumerate}[(1)]
\item величину $I_B$ можно оценить по асимптотическому значению угла 
наклона дисперсионной кривой, которую саму по себе можно использовать как 
средство визуального контроля пра\-виль\-ности принятых предположений;
\item если асимптотическое равенство справедливо с~хорошим приближением для 
всех~$t$, б$\acute{\mbox{о}}$льших некоторого~$t^\prime$, то численности 
событий в~не\-пе\-ре\-кры\-ва\-ющих\-ся интервалах длины~$t^\prime$ практически 
независимы;
\item если за продолжительный период наблюдения длины~$t$ 
произошли~$n$~событий, то оценкой ин\-тен\-сив\-ности будет $\tilde{\lambda}_B\hm= 
n/t$, а~$\mathrm{var}\,(\tilde{\lambda}_B)$~--- величина 
$I_B\tilde{\lambda}_B/t\simeq I_B n/t^2$. 
     \end{enumerate}
     
     Если значение $I_B$ известно, например из результатов предварительного 
теоретического или эмпирического исследования СМО с~по\-мощью следствия~(1), 
то согласно следствию~(3) для конкретной последовательности данных 
ин\-тен\-сив\-ность на\-ступ\-ле\-ния событий можно оценить обычным образом, используя 
далее для статистического вывода нормальную аппроксимацию 
с~со\-от\-вет\-ст\-ву\-ющей дисперсией.
     
     Если значение $I_B$ неизвестно, то приходится обращаться к~следствиям~(2) 
и~(3) в~предположении, что тем или иным путем установлено значение~$t^\prime$. 
Далее каж\-дая по\-сле\-до\-ва\-тель\-ность разбивается на смеж\-ные интервалы 
длины~$t^\prime$, для которых подсчитываются численности событий в~каждом 
из них. В~результате приходим к~задаче сравнения средних значений из двух 
выборок с~воз\-мож\-ностью оценивания их точ\-ности~$V(t^\prime)$. 
     
     Обе рассмотренные ситуации со значением~$I_B$ фактически отличаются 
друг от друга тем, что выделяется или нет в~виде отдельной задачи оценивание  
точ\-ности~$\tilde{\lambda}_B$. В~остальном определенный произвол приемов 
формирования распределений при сравнении интенсивностей остается, что не 
придает уве\-рен\-ности исследователю. Не менее важно и~другое: при анализе 
ста\-биль\-ности необходимо учитывать возможную за\-ви\-си\-мость входящего 
и~выходящего потоков. По этой причине приходится обращаться к~методам 
обработки так на\-зы\-ва\-емых согласованных пар (matched pairs) наблюдений, 
естественно воз\-ни\-ка\-ющих при исследованиях входного потока заявок до 
и~выходного потока после обработки в~СМО. Для анализа различий $d_i$  в~каждой 
из $M$ пар мож\-но использовать ряд  
критериев~\cite[с.~4579--4583, 7726--7730, 9150--9153]{8-kri}, но ни один из них 
не кажется безуслов\-но лучшим. Принимая во внимание, что объектами служат 
оценки средних, т.\,е.\ данные распределены, скорее всего, нормально, 
предпочтение из пе\-ре\-чис\-лен\-ных следует отдать $t$-кри\-те\-рию. 
     
     Пусть $(x_{i1}, x_{i2})$~--- $i$-е наблюденное значение сопоставленной 
пары $(\tilde{\lambda}_A, \tilde{\lambda}_D)$, $i\hm= \overline{1, M}$. Тогда $d_i\hm= 
x_{i2}\hm- x_{i1}$ дают оценку математического ожидания~$\Delta$ отличия 
интенсивности выходного потока от интенсивности входного. Нулевая гипотеза 
о~ста\-биль\-ности СМО заключается в~том, что $\Delta\hm=0$, кон\-ку\-ри\-ру\-ющая~--- 
$\Delta\hm <0$.
     
     Статистика $t$-кри\-те\-рия c $M\hm-1$ степенями свободы имеет вид:
     $$
     t_{\mathrm{MP}} =\left( S_d\right)^{-1}\sqrt{M}\,\overline{d}\,,
     $$
     
     \vspace*{-2pt}
     
     \noindent
     где 
     
     \noindent
     $$
     \overline{d}= M^{-1}\sum\limits_{i=1}^M d_i\,;\enskip S_d^2= (M-1)^{-1} 
\sum\limits^M_{i=1} \left( d_i-\overline{d}\right)^2.
     $$ 
     \vspace*{-2pt}
     
     \noindent
Вводя обычным образом выборочный коэффициент корреляции~$r$ между~$x_1$ 
и~$x_2$, получаем тож\-дество
$$
S_d^2 =S^2_{x_1} -2r S_{x_1}S_{x_2}+S^2_{x_2}
$$
и иной эквивалентный вид статистики~$t_{\mathrm{MP}}$, т.\,е.\ описанный критерий 
включает коэффициент корреляции, ха\-рак\-те\-ри\-зу\-ющий связь выходного 
и~входного потоков. Поэтому интерес вызывает его поведение в~зависимости от 
стабильного или нестабильного типа СМО, а~так\-же становится актуальным 
описание выборочных свойств этой статистики. Если за основу принять обычную 
оценку коэффициента корреляции, то для нормализации ее дисперсии 
применяется преобразование Фишера~\cite[с.~1375--1385]{8-kri} вида
$$
z=\fr{1}{2}\ln \fr{1+r}{1-r}\,.
$$
  %   
     Тогда для больших значений $M$ при тео\-ре\-ти\-че\-ском значении 
корреляции~$\rho$ имеем:
     $$
     Z\sim N\left( \fr{1}{2}\ln\fr{1+\rho}{1-\rho}\,, \fr{1}{M-3}\right).
     $$
     
     Если $Q$~--- квантиль стандартного нормального распределения уровня 
$0{,}5(1\hm- \alpha)$, то сразу же мож\-но получить доверительный интервал 
уров\-ня~$\alpha$ для среднего значения~$Z$. 
     
     Традиционно используемое $z$-пре\-обра\-зо\-ва\-ние практично, поскольку 
доверительный интервал может быть вы\-чис\-лен на основе стандартного 
нормального распределения и~он демонстрирует достаточную точ\-ность уже при 
десятке наблюдений. Но не надо забывать о~наличии иных  
результатов~\cite{9-kri}, включающих точные. 
     
     Надо обратить внимание, что ниоткуда не следует, что совместное 
распределение оценок интенсивностей асимптотически также будет нормальным.\linebreak 
На важность осно\-во\-по\-ла\-га\-ющих предположений при анализе корреляции 
нормально распределенных наблюдений обращено внимание в~[10], а~в~случае 
статистического контроля функционирования СМО общего характера вряд ли 
можно без колебаний принять предположение о~совместном нормальном 
распределении оценок интенсивностей\linebreak входного и~выходного потоков, особенно 
если сис\-те\-ма пред\-став\-ля\-ет собой <<черный ящик>>. Поэтому так важны 
процедуры тестирования многомерной нор\-маль\-ности результатов наблюдений 
\mbox{входного} и~выходного потоков СМО одновременно. В~этой связи 
к~фундаментальным работам следует отнести~[11], примером сравнительного 
анализа подходов~служит [12], специфика двухмерного случая отражена в~[13]. 
     
     Множественность методов вызвана сложностью постановки задачи 
(содержательная асим\-мет\-рич\-ность альтернативы, мно\-го\-мер\-ность данных, 
ограничение на маргинальные распределения, опора на выборочные 
характеристики), которая приводит к~их перманентным модификациям 
и~к~не\-об\-хо\-ди\-мости соблюдать осторожность при применении. 
     
     Так, тестирование муль\-ти\-нор\-маль\-ности~\cite[с.~5047--5051]{8-kri} на 
основе мер асим\-мет\-рии и~эксцесса обладает отя\-го\-ща\-ющим фактором: 
асимп\-то\-ти\-че\-ские распределения со\-от\-вет\-ст\-ву\-ющих статистик при нулевой 
гипотезе можно использовать корректно только при достаточно больших объемах 
данных, причем асимптотика специфична для разных показателей формы 
распределения статистик. 
     
     Подобные проблемы обычно решаются методом статистических испытаний, 
что неприемлемо при автоматизированном контроле функционирования СМО 
в~меняющейся обстановке. Более успешной оказывается попытка точной 
аппроксимации предельных распределений вкупе с~объединением частных 
критериев муль\-ти\-нор\-маль\-ности в~виде сводного тес\-та на основе коэффициентов 
асим\-мет\-рии\linebreak и~эксцесса. Но здесь может проявиться так называемый эффект 
услож\-не\-ния текс\-та~[14], за\-труд\-ня\-ющий понимание и~реализацию пред\-ла\-га\-емых 
процедур. В~качестве соответствующего примера \mbox{приведем} один из вариантов 
критерия мультинормальности. Пусть $\tilde{\mathbf{X}}^{\mathrm{T}}$~---  
($p\times N$)-мат\-ри\-ца $N$ центрированных наблюденных значений  
$p$-век\-то\-ра с~выборочной ковариационной матрицей 
$$
\mathbf{C}= \fr{1}{N} \,\tilde{\mathbf{X}}^{\mathrm{T}}\tilde{\mathbf{X}}$$ 
и~корреляционной матрицей 
$$
\mathbf{R}= \mathbf{V}^{-1/2} \mathbf{CV}^{-1/2},
$$
 где $\mathbf{V}\hm= 
\mathrm{diag}\left( \hat{\sigma}^2_1, \ldots , \hat{\sigma}^2_p\right)$.
Согласно~[15] определим ($p\times N$)-мат\-ри\-цу $\mathbf{Y}^{\mathrm{T}}$ 
преобразованных наблюдений 

\noindent
$$
\mathbf{Y}^{\mathrm{T}}= \mathbf{H}\bm{\Lambda}^{-1/2} \mathrm{H}^{\mathrm{T}} \mathbf{V}^{-1/2} 
\tilde{\mathbf{X}}^{\mathrm{T}}\,,
$$
где фигурируют элементы спектрального разложения корреляционной матрицы 
$\mathbf{R}\hm= \mathbf{H}\bm{\Lambda} \mathbf{H}^{\mathrm{T}}$, 
$\mathbf{H}^{\mathrm{T}}\mathbf{H}\hm= \mathbf{I}$; $\bm{\Lambda}$~--- диагональная 
матрица с~со\-от\-вет\-ст\-ву\-ющи\-ми собственными значениями на глав\-ной диа\-го\-на\-ли. 
Тогда в~условиях нулевой гипотезы данные из многомерного нормального 
распределения преобразуются в~со\-во\-куп\-ность независимых величин со 
стандартным нормальным распределением. Для каждой из по\-лу\-чив\-ших\-ся 
переменных можно подсчитать коэффициенты асимметрии и~эксцесса, которые 
в~со\-во\-куп\-ности дадут статистику сводного критерия. Несмотря на то что в~случае 
сравнения потоков речь идет о~$p\hm=2$, актуальным остается уточ\-не\-ние этого 
подхода для случая, когда ранг мат\-ри\-цы~$\mathbf{R}$ может оказаться меньше 
пред\-по\-ла\-га\-емо\-го значения~$p$. 
     
     Искусственность усложнения в~[15] заключается в~том, что нет 
необходимости перехода к~стандартным распределениям. Достаточно обратиться к~первым главным компонентам, а~затем применить набор со\-став\-ных критериев 
значимости. При этом в~ходе перехода совершенно естественно решается задача 
снижения раз\-мер\-ности. 

\begin{table*}\small
\begin{center}
\begin{tabular}{|l|c|c|c|c|c|c|c|c|c|c|c|}
\multicolumn{12}{p{100mm}}{Результаты проверки нулевых гипотез об адекватности модели (строки 
помечены бук\-вой <<M>>) и~о~ста\-биль\-ности (строки помечены бук\-вой~<<S>>): <<$+$>>~--- 
принятие; <<$\pm$>>~--- принятие с~отдельными исключениями в~многокритериальном 
случае;  <<$-$>>~--- отвер\-же\-ние}\\
\multicolumn{12}{c}{\ }\\[-6pt]
\hline
\multicolumn{2}{|c|}{Вариант}&\multicolumn{10}{c|}{Номер фазы}\\
\cline{3-12}
\multicolumn{2}{|c|}{СМО}&1&2&3&4&5&6&7&8&9&10\\
\hline
$C_\lambda=0{,}95$,
&M&$+$&$+$&$+$&$\pm$&$\pm$&$+$&$+$&$+$&$\pm$&$+$\\
стабильна&S&$+$&$+$&$+$&$+$&$+$&$+$&$+$&$+$&$+$&$+$\\
\hline
$C_\lambda=1{,}05$,
&M&$\pm$&$+$&$+$&$+$&$\pm$&$+$&$+$&$\pm$&$\pm$&$+$\\
нестабильна&S&$-$&$-$&$-$&$-$&$-$&$-$&$-$&$-$&$-$&$-$\\
\hline
\end{tabular}
\end{center}
\end{table*}

\section{Эксперименты}

     
     Рассмотрим задачу контроля стабильной работы двухпроцессорной системы 
обработки заданий со случайным выбором числа требуемых  
процессоров~\cite{2-kri}. Функционирование подобной сис\-те\-мы\linebreak $M/M/2$ 
определяется параметрами: $\lambda$~--- интенсивность поступления заданий на 
обработку; $p_1$~--- вероятность того, что для выполнения задания \mbox{требуется} 
один процессор; $\mu$~--- среднее время обработки задания  
про\-цес\-со\-ром/про\-цес\-со\-ра\-ми. \mbox{Согласно} ссылке в~разд.~3 работы~\cite{2-kri}, 
условие ста\-биль\-ности функционирования этой сис\-те\-мы при\-ни\-ма\-ет~вид 
     $$
     \fr{\lambda}{\mu}< \fr{2}{2- p_1^2},
     $$
      с~по\-мощью которого будет 
осуществляться выбор моделируемых сис\-тем: па\-ра\-мет\-ры~$p_1$ и~$\mu$ 
принимаются базовыми со значениями $p_1\hm= 0{,}5$ и~$\mu\hm= 1$,\linebreak 
а~$\lambda$ выбирается как $C_\lambda(2\mu/(2\hm- p_1^2))$, где $C_\lambda$~--- 
индикатор стабильности работы сис\-те\-мы (при $C_\lambda\hm<1$ сис\-те\-ма 
стабильна). 
     
     В качестве объекта статистического анализа выступают входной и~выходной 
процессы, время наблюдения за которыми равно~$T$. Этот период для учета 
особенностей функционирования СМО делится на~$N_s$ последовательных 
(идущих одна за другой) фаз одинаковой дли\-тель\-ности~$T_s$. Каждая фаза 
включает $N_f$  фрагментов одинаковый продолжительности~$T_f$ (то же самое, что 
и~$t^\prime$ в~[7]), которые служат для оценивания интенсивности, их 
насыщенность данными должна обеспечивать воз\-мож\-ность корректной 
аппроксимации распределения оценок. По со\-во\-куп\-ности фрагментов в~фазе 
строится статистический вывод о~свойствах наблю\-да\-емых процессов, рост числа 
фрагментов повышает мощ\-ность критерия, при\-ме\-ня\-емо\-го к~данным фазы. 
     
     Центральным становится задание объемных параметров~$T_f$ и~$N_f$, 
при выборе которых встречается неожиданная на первый взгляд ситуация: при 
одной и~той же длине фрагмента (например, для прос\-тей\-ше\-го потока событий, 
когда во фрагменте в~среднем 50~событий) оценка ве\-ро\-ят\-ности отклонения 
нулевой гипотезы 
 с~пом\-ощью составного критерия при увеличении числа фрагментов ($N_f\hm= 
20, 50, 100$ и~200) растет, начиная с~пред\-по\-ла\-га\-емо\-го значения ошиб\-ки первого 
рода в~5\%, а~не остается постоянной ($\alpha^*\hm= 5{,}1\%, 5{,}8\%, 7{,}0\%$ 
и~10{,}0\% соответственно). Здесь проявляется эффект подмены (замещения, 
подстановки), когда проверка нор\-маль\-ности ориентирована на предельное 
нормальное распределение и~не зависит от объема выборки, а~применяется 
к~данным с~фактически иным распределением. Поэтому пока объемы~$T_f$ 
и~$N_f$ со\-гла\-су\-ют\-ся, ис\-поль\-зу\-емый критерий ведет себя ожидаемо, но с~рос\-том 
чис\-ла наблюдений~$N_f$ проявляется отклонение фактического распределения от 
предельного. Это означает, что подбор~$T_f$ и~$N_f$ надо осуществлять только 
со\-вместно.
     
     Для каждой фазы проводились:
     \begin{itemize}
\item контроль основных характеристик $A(t)$ и~$D(t)$ (ин\-тен\-сив\-ность, среднее 
чис\-ло запросов на обработку во фрагменте); 
\item проверка адекватности модели оценивания интенсивностей $A(t)$ и~$D(t)$, 
включающая корреляционный анализ (наличие взаимной связи и~автокорреляция) 
и~тестирование муль\-ти\-нор\-маль\-ности и~нор\-маль\-ности в~от\-дель\-ности по каж\-до\-му 
процессу; 
\item решение о стабильности ана\-ли\-зи\-ру\-емой СМО.
\end{itemize}


     
     Результаты моделирования, представленные в~таблице, позволяют сделать 
следующие выводы:
     \begin{itemize}
\item принятые предположения (строки таб\-ли\-цы, помеченные буквой <<M>>) 
выполняются за редкими исключениями, чис\-ло которых не противоречит 
статистическим принципам принятия решений; все отклонения от ожи\-да\-емых 
результатов соответствовали критериям муль\-ти\-нор\-маль\-ности; 
\item критерий стабильности (строки <<S>>) дает ожи\-да\-емые результаты для двух 
типов СМО;
\item обнаруживается одна особенность оценок интенсивности, когда для 
входного и~выходного процессов корреляционные связи проявляются в~случае 
стабильных сис\-тем и~исчезают для нестабильных.
\end{itemize}




     
     Для детализации последнего вывода была оценена за\-ви\-си\-мость 
коэффициента корреляции~$r$ между фрагментарными оценками ин\-тен\-сив\-ности 
процессов $A(t)$ и~$D(t)$ от индикатора ста\-биль\-ности\linebreak\vspace*{-12pt}

{ \begin{center}  %fig1
 \vspace*{6pt}
    \mbox{%
\epsfxsize=78.056mm 
\epsfbox{kri-1.eps}
}

\end{center}



\noindent
{{\small Зависимость коэффициента корреляции~$r$ между фрагментарными оценками 
интенсивности процессов $A(t)$ и~$D(t)$ от индикатора ста\-биль\-ности работы~$C_\lambda$ 
в~виде линии, соединяющей отдельные оценки~$r$, и~со\-от\-вет\-ст\-ву\-ющих доверительных 
интервалов в~виде вертикальных отрезков}}
}

%\vspace*{6pt}

\noindent
 работы $C_\lambda$, 
которая в~виде графика отдельных значений~$r$ и~со\-от\-вет\-ст\-ву\-ющих 95\% 
доверительных интервалов дана на рисунке (взята 4-я фаза). Из него вид\-но, что 
для стабильных сис\-тем (значения $C_\lambda\hm<1$) корреляционная связь 
присутствует, но нет вес\-ких оснований предполагать ее наличие для нестабильных 
сис\-тем (при\-над\-леж\-ность нуля со\-от\-вет\-ст\-ву\-ющим доверительным интервалам 
под\-тверж\-да\-ет\-ся результатами проверки нулевой гипотезы о~нулевой корреляции). 
Получается, что для нестабильных сис\-тем, когда нарушается равновесие их 
функционирования, выход <<теряет связь>> со входом и~начинает зависеть только 
от характеристик обработки заданий.

     
      
\section{Заключение}

     Критерий стабильности СМО наиболее востребован при по\-стро\-ении 
и~исследовании малоизученных сис\-тем, для которых известны час\-тич\-но или не 
описаны вообще следующие свойства: тео\-ре\-ти\-че\-ские и~выборочные 
характеристики входного и~выходного потоков, наличие и~характер 
статистических связей между ними, свойства и~возможности применения 
процедур анализа данных, интерпретация по\-лу\-ча\-емых результатов. Решения 
приходится искать среди известных моделей и~методов, что усиливает роль задач 
проверки адек\-ват\-ности принятых предположений, приводит к~смещению акцентов 
при исследованиях на технологии обработки информации со всеми обычными 
проблемами: соблюдение кор\-рект\-ности; преодоление услож\-не\-ний из-за 
комплексного характера воз\-ни\-ка\-ющих задач; обеспечение эф\-фек\-тив\-ности 
обработки. Поэтому такими важными пред\-став\-ля\-ют\-ся дальнейшие исследования 
по порядку применения сформировавшихся методов анализа данных 
и~пополнению набора базовых процедур обработки информации. 

{\small\frenchspacing
 { %\baselineskip=10.6pt
 %\addcontentsline{toc}{section}{References}
 \begin{thebibliography}{99}
\bibitem{1-kri}
\Au{Green L.} A~queueing system in which customers require a random number of servers~// Oper. 
Res., 1980. Vol.~28. No.\,6. P.~1335--1346. doi: 10.1287/opre.28.6.1335.
\bibitem{2-kri}
\Au{Brill P.\,H., Green~L.} Queues in which customers receive simultaneous service from a random 
number of servers: A~system point approach~// Manage. Sci., 1984. Vol.~30. No.\,1.  
P.~51--68. doi: 10.1287/mnsc.30.1.51.
\bibitem{3-kri}
\Au{El-Taha~M., Stidham~S., Jr.} Sample-path analysis of queueing systems.~--- New York, NY, 
USA: Springer Science\;+\;Business Media, 1999. 302~p. doi: 10.1007/978-1-4615-5721-0.
\bibitem{4-kri}
\Au{Burke~P.\,J.} The output of a queuing system~// Oper. Res., 1956. Vol.~4. No.\,6. P.~699--704. 
doi: 10.1287/opre.4.6.699.
\bibitem{5-kri}
\Au{Daley~D.\,J.} Queueing output processes~// Adv. \mbox{Appl.} Probab., 1976. Vol.~8. No.\,2. 
 P.~395--415. doi: 10.2307/ 1425911.
\bibitem{6-kri}
\Au{Karr~A.\,F.} Point processes and their statistical inference.~--- 2nd ed.~--- New York, NY, USA: 
Marcel Dekker, 1991. 512~p.
\bibitem{7-kri}
\Au{Кокс~Д., Льюис~П.} Статистический анализ последовательностей событий~/
Пер. с~англ.~--- М.: Мир, 
1969. 312~с. (\Au{Cox~D.\,R., Lewis~P.\,A.\,W.} {The statistical analysis of series 
of events}.~--- New York, NY, USA: John Wiley, 1966. 285~p.)
\bibitem{8-kri}
\Au{Kotz~S., Read~C.\,B., Balakrishnan~N., Vidakovic~B.} Encyclopedia of statistical sciences: 
16~vol. set.~--- 2nd ed.~--- Hoboken, NJ, USA: Wiley, 2006. 9686~p.
\bibitem{9-kri}
\Au{Taraldsen~G.} The confidence density for correlation~// Sankhya Ser.~A, 2023. 
Vol.~85. Part~1. P.~600--616. doi: 10.1007/s13171-021-00267-y.
\bibitem{10-kri}
\Au{Levy~K.\,J.} Non-normality and testing that a correlation equals zero~// Educ. Psychol. 
Meas., 1977. Vol.~37. Iss.~3. P.~691--694.
\bibitem{11-kri}
\Au{Mardia~K.\,V.} Tests of univariate and multivariate normality~// Handbook of statistics~/ Ed. 
R.~Krishnaiah.~--- Amsterdam: North-Holland Publishing Co., 1980. Vol.~1. P.~279--320.
doi: 10.1016/S0169-7161(80)01011-5.
\bibitem{12-kri}
\Au{Romeu~J.\,L., Ozturk~A.} A~comparative study of goodness-of-fit tests for multivariate 
normality~// J.~Multivariate Anal., 1993. Vol.~46. Iss.~2. P.~309--334. doi: 
10.1006/\linebreak jmva.1993.1063.
\bibitem{13-kri}
\Au{Versluis~C.} Comparison of tests for bivariate normality with unknown parameters by 
transformation to an univariate statistic~// Commun. Stat.~--- Theor. M., 1996. Vol.~25. Iss.~3. 
P.~647--665.  doi: 10.1080/03610929608831719.
\bibitem{14-kri}
\Au{Дьякова~А.\,А.} Усложнение текста: струк\-тур\-но-се\-ман\-ти\-че\-ский аспект~// Известия 
ВГПУ. Филологические науки, 2016. №\,1(105). C.~158--165. EDN: VRWUIV.
\bibitem{15-kri}
\Au{Doornik~J., Hansen~H.} An omnibus test for univariate and multivariate normality~// Oxford B. 
Econ. Stat., 2008. Vol.~70. Iss.~S1. P.~927--939. doi: 10.1111/j.1468-0084.2008.00537.x.
\end{thebibliography}

 }
 }

\end{multicols}

\vspace*{-10pt}

\hfill{\small\textit{Поступила в~редакцию 29.11.23}}

%\vspace*{8pt}

%\pagebreak

\newpage

\vspace*{-28pt}

%\hrule

%\vspace*{2pt}

%\hrule


\def\tit{STATISTICAL CRITERION FOR~QUEUING SYSTEM STABILITY 
BASED~ON~INPUT  AND~OUTPUT FLOWS}


\def\titkol{Statistical criterion for~queuing system stability 
based~on~input  and~output flows}


\def\aut{M.\,P.~Krivenko}

\def\autkol{M.\,P.~Krivenko}

\titel{\tit}{\aut}{\autkol}{\titkol}

\vspace*{-10pt}


\noindent
Federal Research Center ``Computer Science and Control'' of the Russian Academy of 
Sciences, 44-2~Vavilov Str., Moscow 119333, Russian Federation

\def\leftfootline{\small{\textbf{\thepage}
\hfill INFORMATIKA I EE PRIMENENIYA~--- INFORMATICS AND
APPLICATIONS\ \ \ 2024\ \ \ volume~18\ \ \ issue\ 1}
}%
 \def\rightfootline{\small{INFORMATIKA I EE PRIMENENIYA~---
INFORMATICS AND APPLICATIONS\ \ \ 2024\ \ \ volume~18\ \ \ issue\ 1
\hfill \textbf{\thepage}}}

\vspace*{3pt}


\Abste{One of the basic properties of a queuing system is stability~--- the ability of the 
system to function, maintaining its structure and characteristics unchanged over time. 
The problem of statistical verification of the stability of the queuing system based on the 
characteristics of the input $A(t)$ and output $D(t)$ order flows is considered. 
The confirmation of stability is based on establishing the equality of the rates of these flows. 
Thus, in the language of statistical data analysis, one obtains the classic problem of comparing 
rates of occurrence. To solve it, the observation period is divided into fragments that 
give separate estimates. Together, they make up the sample that participates in the 
comparison. When analyzing stability, it is necessary to take into account possible 
dependence of $A(t)$ and $D(t)$; so, it is necessary to turn to methods for processing the 
so-called matched pairs of observations. Stability control makes it necessary to solve 
a~number of auxiliary tasks: selection of volumetric parameters for rate estimation, checking 
the normality of the distribution, and analysis of correlations. In the course of 
experiments with the real system, a~number of features were revealed: the effect of 
substituting the prelimit distribution with the real one during fragmentation as well as 
the presence of dependence of the rate estimates of analyzed flows which comes to 
naught for unstable systems.}

\KWE{queueing system; sample-path stability; matched pairs tests; tests of 
multinormality}

      \DOI{10.14357/19922264240108}{JNJJMU}

\vspace*{-12pt}

%\Ack
%\vspace*{-3pt}
%\noindent
 

  \begin{multicols}{2}

\renewcommand{\bibname}{\protect\rmfamily References}
%\renewcommand{\bibname}{\large\protect\rm References}

{\small\frenchspacing
 {%\baselineskip=10.8pt
 \addcontentsline{toc}{section}{References}
 \begin{thebibliography}{99} 
\bibitem{1-kri-1}
\Aue{Green, L.} 1980. A queueing system in which customers require a~random number 
of servers. \textit{Oper. Res.} 28(6):1335--1346. doi: 10.1287/opre.28.6.1335.
\bibitem{2-kri-1}
\Aue{Brill, P.\,H., and L.~Green.} 1984. Queues in which customers receive 
simultaneous service from a~random number of servers: A~system point approach. 
\textit{Manage. Sci.} 30(1):51--68. doi: 10.1287/mnsc.30.1.51.
\bibitem{3-kri-1}
\Aue{El-Taha, M., and S.~Stidham, Jr.} 1999. \textit{Sample-path analysis of queueing 
systems}. New York, NY: Springer. 302~p. doi: 10.1007/978-1-4615-5721-0.
\bibitem{4-kri-1}
\Aue{Burke, P.\,J.} 1956. The output of a queuing system. \textit{Oper. Res.} 
4(6):699--704. doi: 10.1287/opre.4.6.699.
\bibitem{5-kri-1}
\Aue{Daley, D.\,J.} 1976. Queueing output processes. \textit{Adv. Appl. Probab.} 
8(2):395--415. doi: 10.2307/1425911.
\bibitem{6-kri-1}
\Aue{Karr, A.\,F.} 1991. \textit{Point processes and their statistical inference}. 2nd ed. 
New York, NY: Marcel Dekker. 512~p.
\bibitem{7-kri-1}
\Aue{Cox, D.\,R., and P.\,A.\,W.~Lewis.} 1966. \textit{The statistical analysis of series 
of events}. New York, NY: John Wiley. 285~p.
\bibitem{8-kri-1}
\Aue{Kotz, S., C.\,B.~Read, N.~Balakrishnan, and B.~Vidakovic.} 2006. 
\textit{Encyclopedia of statistical sciences}. 16~vol. set. 2nd ed. Hoboken, NJ: Wiley. 
9686~p.
\bibitem{9-kri-1}
\Aue{Taraldsen, G.} 2023. The confidence density for correlation. \textit{Sankhya Ser.~A} 85(1):600--616. doi: 10.1007/s13171-021-00267-y.
\bibitem{10-kri-1}
\Aue{Levy, K.\,J.} 1977. Non-normality and testing that a correlation equals zero. 
\textit{Educ. Psychol. Meas.} 37(3):691--694. 
\bibitem{11-kri-1}
\Aue{Mardia, K.\,V.} 1980. Tests of univariate and multivariate normality. 
\textit{Handbook of statistics}. Ed. R.~Krishnaiah. Amsterdam: North-Holland 
Publishing Co. 1:279--320. doi: 10.1016/S0169-7161(80)01011-5.
\bibitem{12-kri-1}
\Aue{Romeu, J.\,L., and A.~Ozturk.} 1993. A~comparative study of goodness-of-fit tests 
for multivariate normality. \textit{J.~Multivariate Anal.} 46(2):309--334. doi: 
10.1006/\linebreak jmva.1993.1063.
\bibitem{13-kri-1}
\Aue{Versluis, C.} 1996. Comparison of tests for bivariate normality with unknown 
parameters by transformation to an univariate statistic. \textit{Commun. Stat.~--- 
Theor.~M.} 25(3):647--665. doi: 10.1080/03610929608831719.
\bibitem{14-kri-1}
\Aue{D'yakova, A.\,A.} 2016. Uslozhnenie teksta: strukturno-semanticheskiy aspekt 
[Complication of a~text: Structural and semantic aspect]. \textit{Izvestiya VGPU. 
Filologicheskie nauki} [Ivzestia of the Volgograd State Pedagogical University. 
Philological Sciences] 1(105):158--165. EDN: \mbox{VRWUIV}.
\bibitem{15-kri-1}
\Aue{Doornik, J., and H.~Hansen.} 2008. An omnibus test for univariate and 
multivariate normality. \textit{Oxford B. Econ. Stat.} 70(S1):927--939. doi: 
10.1111/j.1468-0084.2008.00537.x.

\end{thebibliography}

 }
 }

\end{multicols}

\vspace*{-6pt}

\hfill{\small\textit{Received November 29, 2023}} 

\vspace*{-18pt}


\Contrl

\vspace*{-3pt}

\noindent
\textbf{Krivenko Michail P.} (b.\ 1946)~--- Doctor of Science in technology, professor, 
leading scientist, Federal Research Center ``Computer Science and Control'' of the 
Russian Academy of Sciences, 44-2~Vavilov Str., Moscow 119333, Russian Federation; 
\mbox{mkrivenko@ipiran.ru}




\label{end\stat}

\renewcommand{\bibname}{\protect\rm Литература} 