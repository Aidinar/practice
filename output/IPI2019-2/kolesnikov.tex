\def\stat{koles+list}

\def\tit{ПРОТОКОЛ ГЕТЕРОГЕННОГО МЫШЛЕНИЯ ГИБРИДНОЙ ИНТЕЛЛЕКТУАЛЬНОЙ  
МНОГОАГЕНТНОЙ СИСТЕМЫ ДЛЯ~РЕШЕНИЯ ПРОБЛЕМЫ ВОССТАНОВЛЕНИЯ  
РАСПРЕДЕЛИТЕЛЬНОЙ ЭЛЕКТРОСЕТИ$^*$}

\def\titkol{Протокол гетерогенного мышления гибридной интеллектуальной  
многоагентной системы} % для решения проблемы восстановления   распределительной электросети}

\def\aut{А.\,В.~Колесников$^1$, С.\,В.~Листопад$^2$}

\def\autkol{А.\,В.~Колесников, С.\,В.~Листопад}

\titel{\tit}{\aut}{\autkol}{\titkol}

\index{Колесников А.\,В.}
\index{Листопад С.\,В.}
\index{Kolesnikov A.\,V.}
\index{Listopad S.\,V.}


{\renewcommand{\thefootnote}{\fnsymbol{footnote}} \footnotetext[1]
{Работа выполнена при поддержке РФФИ (проект 18-07-00448а).}}


\renewcommand{\thefootnote}{\arabic{footnote}}
\footnotetext[1]{Балтийский федеральный университет им.\ И.~Канта, avkolesnikov@yandex.ru}
\footnotetext[2]{Калининградский филиал Федерального исследовательского центра <<Информатика 
и~управление>> Российской академии наук, \mbox{ser-list-post@yandex.ru}}

\vspace*{-12pt}

   
   \Abst{Для проблемы восстановления электроснабжения в~региональной 
распределительной электросети после масштабных аварий характерны высокая 
комбинаторная слож\-ность, неоднородность, недоопределенность, неточность и~нечеткость. 
Применение механизма коллективного решения проблем для преодоления перечисленных 
НЕ-фак\-то\-ров невозможно из-за временных ограничений. Для решения проблемы 
предлагается новый класс интеллектуальных систем, моделирующих коллективное принятие 
решений под руководством фасилитатора~--- гибридные интеллектуальные многоагентные 
сис\-те\-мы гетерогенного мышления (ГИМСГМ). В~отличие от традиционных гибридных 
интеллектуальных сис\-тем, интегрирующих модели знаний экспертов, они дополнительно 
моделируют взаимодействие в~групповых процессах и~эффектах, адап\-ти\-ру\-ясь к~динамичным 
проблемным ситуациям послеаварийного восстановления региональной распределительной 
электросети. В~работе рас\-смат\-ри\-ва\-ет\-ся один из компонентов таких сис\-тем~--- протокол 
организации коллективного гетерогенного мыш\-ле\-ния агентов.} 
  
  \KW{гетерогенное мышление; гибридная интеллектуальная многоагентная система; 
проблема восстановления распределительной электросети}

\DOI{10.14357/19922264190211}
  
\vspace*{-6pt}


\vskip 10pt plus 9pt minus 6pt

\thispagestyle{headings}

\begin{multicols}{2}

\label{st\stat}
  
\section{Введение}

\vspace*{-2pt}

  При авариях в~энергосистемах, происходящих несмотря на проводимые 
предупредительно-про\-фи\-лак\-ти\-че\-ские мероприятия, принимаются меры, 
снижающие их интенсивность и~продолжительность. Сразу после масштабной 
аварии повышение частоты и~развал энер\-го\-сис\-те\-мы тормозятся автоматически 
изменением и~отключением нагрузки управляемым разделением. В~течение 
по\-сле\-ду\-ющих часов выполняются восстановительные мероприятия, в~ходе 
которых операторы распределительной сети и~электростанций вручную 
поддерживают баланс нагрузки и~генерации. Длительность выработки решений 
и~действий по их реализации намного превышает допустимые ограничения по 
оборудованию, и~возникает опас\-ность, что управ\-ле\-ние электростанциями 
и~энер\-го\-сис\-те\-мой не может обеспечить необходимую координацию~[1]. 

Одна 
из важнейших задач опе\-ра\-тив\-но-дис\-пет\-чер\-ско\-го персонала~--- 
составление плана обеспечения безопасности, сохранности оборудования 
и~быст\-ро\-го восстановления электроснабжения потребителей, не 
противоречащего требованиям энер\-го\-сис\-те\-мы~[2]. 
  
  При разработке плана оперируют большим разнообразием ресурсов, их 
свойств и~отношений, а~так\-же учитывают НЕ-фак\-то\-ры 
А.\,С.~Нариньяни~[3]: недоопределенность места аварии на момент 
планирования; не\-точ\-ность величины мощ\-ности, по\-треб\-ля\-емой клиентами 
и~генерируемой источниками распределенной генерации; не\-чет\-кость времени 
восстановительных операций; не\-кор\-рект\-ность работы датчиков; неполнота 
модели электросети. В~силу временн$\acute{\mbox{ы}}$х ограничений организовать 
всестороннее коллективное обсуж\-де\-ние послеаварийной проб\-лем\-ной ситуации 
не\-воз\-можно. 

\begin{figure*}[b] %fig1
\vspace*{6pt}
    \begin{center}  
  \mbox{%
 \epsfxsize=123.206mm 
 \epsfbox{kol-1.eps}
 }
 \end{center}
\vspace*{-7pt}
\Caption{Модель коллективного решения проблем методами гетерогенного мышления~--- 
ромб группового принятия решений С. Кейнера, К. Толди, С. Фиск, Д. Бергера: \textit{1}~--- 
альтернатива; \textit{2}~--- досрочное несогласованное решение; \textit{3}~--- согласованное 
решение}
\end{figure*}
  
  Для информационной подготовки решений (<<предрешений>>, по 
П.\,К.~Анохину) по восстановлению электроснабжения в~распределительной 
электросети~[4] предлагается разработать новый класс интеллектуальных 
систем~--- \mbox{ГИМСГМ}, комбинирующих гиб\-рид\-ный подход 
А.\,В.~Колесникова~\cite{5-kol}, аппарат многоагентных систем В.\,Б.~Тарасова~\cite{6-kol} 
и~методики гетерогенного мышления~\cite{7-kol, 8-kol, 9-kol}. Такие системы интегрируют 
знания и~взаимодействие экспертов различных специальностей, учитывают 
несколько критериев оптимальности и~множество ограничений в~условиях 
динамиче-\linebreak\vspace*{-12pt}

\pagebreak

\noindent
ских сред и~дефицита времени на принятие ре\-шения.
{ %\looseness=1

} 
  
\section{Формализованная модель гибридной интеллектуальной  
многоагентной системы гетерогенного мышления}

  Формально ГИМСГМ определяется следующим образом~\cite{10-kol}:
\begin{align*}
&\mathrm{himsht}=\langle \mathrm{AG}^*, \mathrm{env}, \mathrm{INT}^*, 
\mathrm{ORG}, \{\mathrm{ht}\}\rangle\,;\\
&\mathrm{act}_{\mathrm{himsht}}={}\\
&= \Bigg( \mathop{\bigcup}\limits_{\mathrm{ag}\in \mathrm{AG}^*} 
\mathrm{act}_{\mathrm{ag}}\Bigg)  \cup \mathrm{act}_{\mathrm{dmsa}} \cup 
\mathrm{act}_{\mathrm{htmc}}\cup{} \mathrm{act}_{\mathrm{col}}\,;\\
&\mathrm{act}_{\mathrm{ag}}= \left( \mathrm{MET}_{\mathrm{ag}}, 
\mathrm{IT}_{\mathrm{ag}}\right), \mathrm{ag}\in \mathrm{AG}^*, \Bigg\vert\! 
\mathop{\bigcup}\limits_{\mathrm{ag}\in \mathrm{AG}^*} \mathrm{IT}_{\mathrm{ag}}\Bigg\vert \geq 2,
\end{align*}
где $\mathrm{AG}^*=\{ \mathrm{ag}_1, \ldots , \mathrm{ag}_n, \mathrm{ag}^{\mathrm{dm}}, 
\mathrm{ag}^{\mathrm{fc}}\}$~--- множество агентов, 
включающее аген\-тов-экс\-пер\-тов (АЭ)~$\mathrm{ag}_i$, $i\hm\in \mathbb{N}$, 
$1\hm\leq i\hm\leq n$, агента, принимающего решения (АПР), $\mathrm{ag}^{\mathrm{dm}}$ 
и~аген\-та-фа\-си\-ли\-та\-то\-ра
 (АФ) $\mathrm{ag}^{\mathrm{fc}}$, $n$~--- число АЭ;  
$\mathrm{env}$~--- концептуальная модель внешней среды \mbox{ГИМСГМ}; 
$\mathrm{INT}^*\hm=  \{\mathrm{prot}_{\mathrm{gm}},$\linebreak
$ \mathrm{lang}, 
\mathrm{ont}, \mathrm{dmscl}\}$~--- элементы структурирования взаимодействий 
агентов: $\mathrm{prot}_{\mathrm{gm}}$~--- протокол взаимодействия агентов, позволяющий 
организовать их коллективное гетерогенное мышление, $\mathrm{lang}$~--- язык 
передачи сообщений, $\mathrm{ont}$~--- модель предметной об\-ласти, $\mathrm{dmscl}$~--- 
классификатор ситуаций коллективного решения проб\-ле\-мы, 
идентифицирующий стадии этого процесса; $\mathrm{ORG}$~--- множество архитектур 
\mbox{ГИМСГМ}; $\{\mathrm{ht}\}$~--- множество концептуальных моделей макроуровневых 
процессов в~\mbox{ГИМСГМ}: $\mathrm{ht}$~--- модель процесса коллективного решения 
проб\-лем методами гетерогенного мышления~--- ромб группового принятия 
решений С.~Кейнера, К.~Толди, С.~Фиск, Д.~Бергера (рис.~1)~\cite{8-kol}; 
$\mathrm{act}_{\mathrm{himsht}}$~--- функция \mbox{ГИМСГМ} в~целом; 
$\mathrm{act}_{\mathrm{ag}}$~--- функция АЭ из\linebreak 
мно\-жества~$\mathrm{AG}^*$; $\mathrm{act}_{\mathrm{dmsa}}$~--- 
функция <<анализ ситуации 
коллективного решения проб\-ле\-мы>> АФ,\linebreak
 обеспечивающая идентификацию 
стадии процесса гетерогенного мышления \mbox{ГИМСГМ} на основе предлагаемых 
АЭ частных решений, на\-пря\-жен\-ности конфликта между АЭ и~предшествующей\linebreak 
стадии процесса решения проблемы; $\mathrm{act}_{\mathrm{htmc}}$~--- 
функция <<выбор метода 
гетерогенного мышления>> АФ, которая реализуется с~использованием 
нечеткой базы знаний об эффективности методов гетерогенного мышления 
в~зависимости от характеристик проб\-ле\-мы, стадии процесса ее решения 
и~текущей ситуации решения в~\mbox{ГИМСГМ}; $\mathrm{act}_{\mathrm{col}}$~--- 
коллективная 
функция \mbox{ГИМСГМ}, конструируемая динамически; 
$\mathrm{met}_{\mathrm{ag}}$~--- метод 
решения задачи; $\mathrm{it}_{\mathrm{ag}}$~--- 
интеллектуальная технология, в~рамках которой\linebreak 
реализован метод~$\mathrm{met}_{\mathrm{ag}}$.


  Согласно модели, пред\-став\-лен\-ной на рис.~1, процесс решения проб\-ле\-мы 
\mbox{ГИМСГМ} трехстадийный: (1)~дивергентное мышление; (2)~бурление; 
(3)~конвергентное мышление. На стадии дивергентного мышления  
АЭ генерируют множество вариантов решения проблемы, 
а~АФ стимулирует их выработку соответствующими 
методами~\cite{11-kol}. В~случае если даже с~применением методов дивергентного 
мышления противоречий не возникает, т.\,е.\ задача имеет очевидное решение, 
процесс завершается. В~противном случае агенты ГИМСГМ конфликтуют по 
поводу знаний, убеждений, мнений, т.\,е.\ участвуют в~своего рода когнитивных 
конфликтах~\cite{11-kol, 12-kol}. Конфликт~--- отличительная черта стадии бурления, 
позволяющая АФ предпринять меры по сближению 
точек зрения агентов. На стадии конвергентного мыш\-ле\-ния агенты совместно 
переформулируют, дорабатывают решения, пока не получат коллективное 
решение, релевантное разнообразию моделей экспертов \mbox{ГИМСГМ}. 
  
  Функциональная структура ГИМСГМ для решения проблемы 
восстановления электроснабжения в~региональной распределительной 
электросети после масштабных аварий представлена в~[12]. Рассмотрим 
протокол взаимодействия агентов ГИМСГМ в~процессе коллективного 
гетерогенного мышления.
  
\section{Протокол гетерогенного мышления гибридной 
интеллектуальной многоагентной системы}
 
  Основная цель протоколирования агентов гиб\-рид\-ной интеллектуальной 
многоагентной системы~---инкапсуляция разрешенных взаимодействий. Протокол 
определяет схемы (распределенный алгоритм) обмена информацией, знаниями, 
координации агентов при решении по\-став\-лен\-ных задач~\cite{13-kol}. С~одной стороны, 
протокол служит для объединения агентов через концептуальный интерфейс 
и~организации их совместной работы, с~другой~--- определяет четкие границы 
компонентов сис\-те\-мы~\cite{14-kol}. При описании протокола должны быть 
специфицированы: роли агентов; типы сообщений между парами ролей; 
семантика каждого типа сообщения декларативно; любые дополнительные 
ограничения на сообщения, такие как порядок их следования или правила 
передачи информации из одного сообщения в~другое. Такая спецификация 
протокола однозначно определяет, удовлетворяет ли конкретная реализация 
взаимодействия агентов указанному протоколу и~совместим ли конкретный 
агент с~\mbox{ГИМСГМ}~\cite{14-kol}.



  
  К настоящему времени разработано множество протоколов многоагентных 
сис\-тем как общего назначения, так и~для решения конкретных задач. Выделим 
наиболее известные классы протоколов~\cite{13-kol, 15-kol}: 
\begin{enumerate}[(1)]
\item на основе контрактной 
сети~\cite{16-kol, 17-kol} для автоматического планирования взаимодействия агентов 
и~минимизации за\-трат посредством метафоры переговоров агентов на 
рыночных торгах; 
\item на основе теории речевых актов~\cite{18-kol}, когда переговоры 
строятся с~использованием небольшого чис\-ла примитивов, по\-сред\-ст\-вом обмена 
которыми агенты об\-суж\-да\-ют некоторую тему, обновляют свои базы знаний, 
обмениваются <<мнениями>> и~приходят к~общему решению~\cite{13-kol}; 
\item переговорные~\cite{19-kol, 20-kol}, пред\-ла\-га\-ющие механизмы разрешения конфликтов 
для повышения суммарной по\-лез\-ности, до\-сти\-га\-емой аген\-тами; 
\item на основе 
доски объявлений, когда выделяется общая область памяти для взаимодействия 
агентов.
\end{enumerate} 
  
  Предлагаемый протокол организации коллективного гетерогенного 
мышления агентов пред\-назначен для моделирования работы малого кол\-лектива 
экспертов за круглым столом, где нет\linebreak необходимости сбивать цену за услуги 
или подбирать экспертов, поэтому он основывается на теории речевых актов. 
Схема работы гибридной интеллектуальной многоагентной сис\-те\-мы по 
протоколу гетерогенного мыш\-ле\-ния представлена на рис.~2.
  



  Как показано на рис.~2, стандартный протокол речевых актов~\cite{18-kol} расширен 
сле\-ду\-ющи\-ми типами сообщений: request-ch-tt, commit-ch-tt, request-start-ps, 
request-stop-ps, request-task, report-decision, используемых для организации 
взаимодействия АФ, АПР и~АЭ. Взаимодействуют АЭ между собой 
и~с~другими агентами по протоколу речевых актов~\cite{18-kol}.
  
  По предлагаемому протоколу процесс коллективного гетерогенного 
мышления начинается\linebreak с~отправки агентом-фа\-си\-ли\-та\-то\-ром АПР и~АЭ\linebreak 
сообщения типа request-ch-tt, в~теле которого указывается метод гетерогенного 
мышления, при\-ме\-ня\-емый на данном этапе работы \mbox{ГИМСГМ}. Агент-фа\-си\-ли\-та\-тор 
приостанавливает свою работу\linebreak
 в~ожидании от\-ве\-тов-под\-тверж\-де\-ний от АЭ 
и~АПР. Получив от АФ сообщение request-ch-tt, АПР и~АЭ выбирают 
соответствующий алгоритм и~переходят в~режим ожидания сигнала на начало 
решения проб\-ле\-мы в~соответствии с~данным алгоритмом. Для подтверждения 
получения сообщения и~го\-тов\-ности к~работе по установленному алгоритму 
АПР и~АЭ отправляют АФ от\-вет-под\-тверж\-де\-ние commit-ch-tt. Дождавшись 
под\-тверж\-де\-ний от АПР и~всех АЭ, АФ отправляет АПР сообщение  
request-start-ps, сиг\-на\-ли\-зи\-ру\-ющее о~том, что все агенты перешли на 
соответствующий метод гетерогенного мышления, а~сис\-те\-ма готова 
к~дальнейшей работе, и~ожидает решений от~АЭ. 
  
  Получив сообщение request-start-ps, АПР формирует и~рассылает задачи АЭ 
с~использованием сообщения типа request-task, в~теле которого так\-же 
описываются исходные данные задачи, после чего он при\-сту\-па\-ет к~сбору 
решений, поступающих от АЭ, и~работает с~ними в~соответствии 
с~установленным ранее алгоритмом гетерогенного мыш\-ле\-ния.\linebreak\vspace*{-12pt}

\pagebreak

\end{multicols}

\begin{figure*} %fig2
\vspace*{1pt}
    \begin{center}  
  \mbox{%
 \epsfxsize=160.791mm 
 \epsfbox{kol-2.eps}
 }
 \end{center}
\vspace*{-4pt}
\Caption{Схема работы гибридной интеллектуальной многоагентной сис\-те\-мы по протоколу  
гетерогенного мыш\-ле\-ния: \textit{1}~--- участники коллектива; \textit{2}~--- 
шкала времени; 
\textit{3}~--- действие; \textit{4}~--- процесс передачи информации между участниками 
коллектива; $t$~--- модельное время}
\vspace*{3pt}
\end{figure*}

\begin{multicols}{2}

\noindent
 Агенты-экс\-пер\-ты после 
получения задания от АПР начинают решать поставленные задачи 
в~соответствии с~установленным алгоритмом гетерогенного мышления. При 
этом в~за\-ви\-си\-мости от алгоритма они могут генерировать несколько решений 
по\-став\-лен\-ной задачи. Все полученные альтернативы АЭ отправляют в~теле 
сообщения типа report-decision одновременно АПР и~АФ. 
  
  При получении очередного решения от АЭ агент-фа\-си\-ли\-та\-тор 
запускает функцию <<анализ ситуации коллективного решения проблемы>> 
$\mathrm{act}_{\mathrm{dmsa}}$, определяет с~ее помощью напряженность конфликта в~каждой 
паре АЭ и~в ГИМСГМ в~целом~\cite{4-kol}. При достижении определенного 
уровня напряженности конфликта в~соответствии с~нечеткой базой знаний АФ 
запускает функцию <<выбор метода гетерогенного мышления>>~$\mathrm{act}_{\mathrm{htmc}}$, 
позволяющую ему выбрать релевантный ситуации коллективного принятия 
решений метод гетерогенного мышления. 

Для реализации данной функции АФ 
имеет\linebreak нечеткую базу знаний о~релевантности <<стилей мышления>> агентов 
различным ситуациям принятия решений в~ГИМСГМ, а~методов~--- 
различным особенностям проблем и~стадиям коллективного принятия решений. 
Для формирования такой базы знаний необходимо провести серию 
вычислительных экспериментов по решению проблем из различных 
классов~\cite{5-kol} и~установить соответствие между классом проблем 
и~релевантными подходами к~организации гетерогенного мышления. Выбрав 
релевантный метод, АФ отправляет АЭ и~АПР сообщения типа request-ch-tt, 
в~теле которых указан метод гетерогенного мышления для АЭ и~АПР. 
Дожидается подтверждений от АПР и~всех АЭ, после чего отправляет АЭ 
сообщение request-start-ps, сигнализирующее о том, что все агенты перешли на 
соответствующий метод гетерогенного мышления и~АЭ могут продолжить 
работу. Агент-фа\-си\-ли\-та\-тор начинает прием решений от АЭ,
 и~процесс анализа ситуации~--- 
выбора метода гетерогенного мышления~--- повторяется до тех пор, пока 
в~\mbox{ГИМСГМ} не завершится стадия конвергентного мышления (см.\ рис.~1). После 
ее завершения АФ отправляет сигнал request-stop-ps к~окончанию процесса 
решения проблемы АПР и~АЭ и~завершает свою работу. Получив такой сигнал, 
АЭ прерывают выполнение заданий и~завершают работу, а~АПР принимает 
окончательное коллективное решение в~соответствии с~уста\-нов\-лен\-ным 
алгоритмом гетерогенного мыш\-ле\-ния, например решение, по которому был 
достигнут консенсус или отдано большинство голосов агентов на стадии 
конвергентного мыш\-ле\-ния. Далее передает это решение интерфейсному 
агенту и~так\-же завершает работу.
  
  Число <<переключений>> методов мышления протоколом заранее не 
определено, так как стадия бурления может отсутствовать, а~на стадиях 
дивергентного и~конвергентного мыш\-ле\-ния последовательно могут 
применяться различные методы. Таким образом, за счет наличия у~АФ 
нечеткой базы знаний, а~также путем репрезентации неоднородной 
функциональной структуры сложной задачи и~гетерогенного коллективного 
мышления интеллектуальных агентов, взаимодействующих в~соответствии 
с~предложенным протоколом, \mbox{ГИМСГМ} вырабатывает для каждой проб\-ле\-мы 
релевантный ей метод решения без упрощения и~идеализации в~условиях 
динамической среды. 

\section{Заключение}

  Рассмотрены особенности проблемы вос\-ста\-нов\-ле\-ния электроснабжения 
в~региональной распределительной электросети после масштабных аварий 
и~предложен новый класс интеллектуальных сис\-тем для ее решения~--- 
\mbox{ГИМСГМ}. 

Представлено
 формализованное описание \mbox{ГИМСГМ}, ее основных 
составных частей. 

Описан протокол организации коллективного гетерогенного 
мышления агентов на основе теории речевых актов. Его применение 
в~гиб\-рид\-ных\linebreak интеллектуальных многоагентных системах, содержащих 
разнородные интеллектуальные самоорганизующиеся агенты, позволяет 
релевантно моделировать эффективные практики коллективного\linebreak решения 
проблем. 

Применение \mbox{ГИМСГМ} позволит  
опе\-ра\-тив\-но-дис\-пет\-чер\-ско\-му персоналу электроснабжающих 
организаций принимать релевантные решения по восстановлению электросети 
в~условиях дефицита времени.
  
{\small\frenchspacing
 {%\baselineskip=10.8pt
 \addcontentsline{toc}{section}{References}
 \begin{thebibliography}{99}
\bibitem{1-kol}
  \Au{Adibi M.\,M., Fink~L.\,H.} Overcoming restoration challenges associated with major 
power system disturbances~// IEEE Power Energy~M., 2006. Vol. 4. Iss. 5. P. 68--77.
\bibitem{2-kol}
  \Au{Красник В.\,В.} Потребители электрической энергии, энергоснабжающие организации 
  и~органы Ростехнадзора. Правовые основы взаимоотношений.~--- М.: НЦ ЭНАС, 2005. 250~с.
\bibitem{3-kol}
  \Au{Нариньяни А.\,С.} Инженерия знаний и~НЕ-факторы: краткий обзор-08~// Вопросы 
искусственного интеллекта, 2008. №\,1. С.~61--77.
\bibitem{4-kol}
  \Au{Kolesnikov A.\,V., Listopad~S.\,V.}
  Hybrid intelligent multiagent system of heterogeneous 
thinking for solving the problem of restoring the distribution
 power grid after failures~// Open 
Semantic Technologies for Intelligent Systems: 
Research Papers Collection.~--- 
Minsk: BGUIR, 2019. P. 133--138.
\bibitem{5-kol}
  \Au{Колесников А.\,В.} Гибридные интеллектуальные системы. Теория и~технология 
разработки.~--- СПб.: \mbox{СПбГТУ}, 2001. 711~с.
\bibitem{6-kol}
  \Au{Тарасов В.\,Б.} От многоагентных систем к~интеллектуальным организациям: 
философия, психология, информатика.~--- М.: Эдиториал УРСС, 2002. 352~с.
\bibitem{7-kol}
  \Au{Gardner H.} Multiple intelligences~--- the theory in practice.~--- New York, NY, USA: 
Basic Books, 1993. 320~p.

\bibitem{9-kol} %8
  \Au{De Bono~E.} Parallel thinking: From Socratic to De Bono thinking.~--- Melbourne: Penguin 
Books, 1994. 228~p.

\bibitem{8-kol} %9
  \Au{Kaner S., Lind~L., Toldi~C., Fisk~S., Beger~D.}
   The facilitator's guide to participatory 
decision-making.~--- San Francisco, CA, USA: Jossey-Bass, 2011. 368~p.

\bibitem{10-kol}
  \Au{Колесников А.\,В., Листопад~С.\,В.} Модель гибридной интеллектуальной 
многоагентной системы гетерогенного мышления для информационной подготовки 
оперативных решений в~региональных электрических сетях~// Системы и~средства 
информатики, 2018. Т.~28. №\,4. С.~31--41.
\bibitem{11-kol}
  \Au{Tang A.\,Y.\,C., Basheer~G.\,S.}
  A~Conflict Resolution Strategy Selection Method (ConfRSSM) 
in multi-agent systems~// Int. J.~Advanced Computer Sci. Appl., 
2017. Vol.~8. Iss.\,5. P.~398--404.
\bibitem{12-kol}
  \Au{Колесников А.\,В., Листопад~С.\,В.} Функциональная структура гибридной 
интеллектуальной многоагентной системы гетерогенного мышления для решения проблемы 
восстановления распределительной электросети~// Системы и~средства информатики, 2019. 
Т.~29. №\,1. С.~41--52.
\bibitem{13-kol}
  \Au{Городецкий В.\,И., Грушинский~М.\,С., Хабалов~А.\,В.}
  Многоагентные системы (обзор)~// 
Новости искусственного интеллекта, 1998. №\,2. С.~64--116.
\bibitem{14-kol}
  \Au{Singh M.\,P., Chopra~A.\,K.}
  Programming multiagent systems without programming agents~// 
Programming multi-agent systems~/ Eds. L.~Braubach, J.-P.~Briot, J.~Thangarajah.~--- 
  Lecture notes in artificial intelligence ser.~--- Springer, 2010.   Vol.~5919. P.~1--14. 
\bibitem{15-kol} 
  \Au{Singh R., Singh~A., Mukherjee~S.} 
  A~critical investigation of agent interaction protocols in 
multiagent systems~// Int. J.~Advancements Technology, 2014. Vol.~5. Iss.\,2. 
P.~72--81.
\bibitem{17-kol}
  \Au{Smith G.} The Contract Net Protocol: High level communication and control in a 
distributed problem solver~// IEEE T.~Comput., 1980. Vol.~29. Iss.\,12.  
P.~1104--1113.

\bibitem{16-kol} %16
  \Au{Smith R.\,G.} A~framework for distributed problem solving.~--- Ann Arbor, MI,
  USA: UMI  Research Press, 1981. 188~p.

\bibitem{18-kol}
  \Au{Weerasooriya D., Rao A.\,S., Ramamohanarao~K.}
  Design of a~concurrent agent-oriented 
language~// Intelligent agents~/ Eds. M.\,J.~Wooldridge, N.\,R.~Jennings.~---
Lecture notes in computer science ser.~---  Springer, 1995.  
Vol.~890. P.~386--401.
\bibitem{19-kol}
  \Au{Zlotkin G., Rosenschtein~J.}
   Mechanisms for automated negotiation in state oriented 
domain~// J.~Artif. Intell. Res., 1996. Vol.~5. P.~163--238.
\bibitem{20-kol}
  \Au{Marzougui B., Barkaoui~K.}
  Interaction protocols in multi-agent systems based on agent 
Petri nets model~// Int. J.~Advanced Computer Sci. Appl., 2013. 
Vol.~4. Iss.\,7. P.~166--173. 

 \end{thebibliography}

 }
 }

\end{multicols}

\vspace*{-3pt}

\hfill{\small\textit{Поступила в~редакцию 31.03.19}}

\vspace*{8pt}

%\pagebreak

%\newpage

%\vspace*{-29pt}

\hrule

\vspace*{2pt}

\hrule

%\vspace*{-2pt}

\def\tit{HETEROGENEOUS THINKING PROTOCOL OF~HYBRID INTELLIGENT MULTIAGENT 
SYSTEM FOR~SOLVING DISTRIBUTIONAL POWER GRID RECOVERY PROBLEM}


\def\titkol{Heterogeneous thinking protocol of~hybrid intelligent multiagent 
system for~solving distributional power grid recovery problem}

\def\aut{A.\,V.~Kolesnikov$^1$ and~S.\,V.~Listopad$^2$}

\def\autkol{A.\,V.~Kolesnikov and~S.\,V.~Listopad}

\titel{\tit}{\aut}{\autkol}{\titkol}

\vspace*{-11pt}


\noindent
   $^1$Immanuel Kant Baltic Federal University, 14~A.~Nevskogo Str., 
   Kaliningrad 236041, Russian Federation
   
   \noindent
   $^2$Kaliningrad Branch of the Federal Research Center ``Computer 
   Science and Control'' of the Russian Academy\linebreak
   $\hphantom{^1}$of Sciences, 5~Gostinaya Str, Kaliningrad 236022, Russian Federation

\def\leftfootline{\small{\textbf{\thepage}
\hfill INFORMATIKA I EE PRIMENENIYA~--- INFORMATICS AND
APPLICATIONS\ \ \ 2019\ \ \ volume~13\ \ \ issue\ 2}
}%
 \def\rightfootline{\small{INFORMATIKA I EE PRIMENENIYA~---
INFORMATICS AND APPLICATIONS\ \ \ 2019\ \ \ volume~13\ \ \ issue\ 2
\hfill \textbf{\thepage}}}

\vspace*{6pt}
  
  
  
   \Abste{The problem of power supply restoration in the regional distributional power grid after large-scale 
accidents is characterized by high combinatorial complexity, heterogeneity, underdetermination, inaccuracy, 
and ambiguity. The use of the collective problem solving mechanism to overcome the listed non-factors in 
the sense of A.\,S.~Narinyani is impossible due to time constraints. To solve this problem, a new class of 
intelligent systems that model collective decision-making under the guidance of a facilitator is proposed, 
namely, hybrid intelligent multiagent systems of heterogeneous thinking. Unlike traditional hybrid 
intelligent systems that integrate models of expert knowledge, they additionally model group processes and 
effects arising from collective problem solving, adapting to the dynamic nature of the problem of restoring 
the regional distribution grid. The paper discusses one of the components of such systems, namely, the 
protocol for organizing collective heterogeneous thinking of agents.}
   
   \KWE{heterogeneous thinking; hybrid intelligent multiagent system; distributional power grid recovery 
problem} 
   
   
\DOI{10.14357/19922264190211}

%\vspace*{-14pt}

 \Ack
   \noindent
   The reported study was funded by the Russian Foundation for Basic Research according to the research project 
No.\,18-07-00448А.



%\vspace*{6pt}

  \begin{multicols}{2}

\renewcommand{\bibname}{\protect\rmfamily References}
%\renewcommand{\bibname}{\large\protect\rm References}

{\small\frenchspacing
 {%\baselineskip=10.8pt
 \addcontentsline{toc}{section}{References}
 \begin{thebibliography}{99}
\bibitem{1-kol-1}
\Aue{Adibi, M.\,M., and L.\,H.~Fink.} 2006. Overcoming restoration challenges 
associated with major power system disturbances. \textit{IEEE Power Energy~M.} 
4(5):68--77.
\bibitem{2-kol-1}
\Aue{Krasnik, V.\,V.} 2005. \textit{Potrebiteli elektricheskoy energii, 
energosnabzhayushchie organizatsii i~organy Rostekhnadzora. Pravovye osnovy 
vzaimootnosheniy} [Consumers of electric energy, energy supplying organizations and 
bodies of Rostechnadzor. Legal basis of relationships]. Moscow: ENAS. 250 p.
\bibitem{3-kol-1}
\Aue{Narinyani, A.\,S.} 2008. Inzheneriya znaniy i~NE-faktory: kratkiy obzor-08 
[Knowledge engineering and non-factors: A~brief overview-08]. \textit{Voprosy 
iskusstvennogo intellekta} [Artificial Intelligence Issues] 1:61--77.
\bibitem{4-kol-1}
\Aue{Kolesnikov, A.\,V., and S.\,V.~Listopad.} 2019. Hybrid intelligent multiagent 
system of heterogeneous thinking for solving the problem of restoring the distribution 
power grid after failures. \textit{Open Semantic Technologies for Intelligent 
Systems: Research Papers Collection}. Minsk: BGUIR. 133--138.
\bibitem{5-kol-1}
\Aue{Kolesnikov, A.\,V.} 2001. \textit{Gibridnye intellektual'nye sistemy. Teoriya 
i~tekhnologiya razrabotki} [Hybrid intelligent systems: Theory and technology of 
development]. St.\ Petersburg: SPbSTU Publs. 711~p.
\bibitem{6-kol-1}
\Aue{Tarasov, V.\,B.} 2002. \textit{Ot mnogoagentnykh sistem k~intellektual'nym 
organizatsiyam: filosofiya, psikhologiya, informatika} [From multiagent systems to 
intelligent organizations: Philosophy, psychology, and informatics]. Moscow: 
Editorial URSS. 352~p.
\bibitem{7-kol-1}
\Aue{Gardner, H.} 1993. \textit{Multiple intelligences~--- 
  the theory in practice.} New York, NY:  Basic Books. 320~p.

\bibitem{9-kol-1}
\Aue{De Bono, E.} 1994. \textit{Parallel thinking: From Socratic to De Bono thinking}. 
Melbourne: Penguin Books. 228~p.

\bibitem{8-kol-1}
\Aue{Kaner, S., L.~Lind, C.~Toldi, S.~Fisk, and D.~Beger.} 2011. \textit{The facilitator's 
guide to participatory decision-making.} San Francisco, CA: Jossey-Bass. 368~p.

\bibitem{10-kol-1}
\Aue{Kolesnikov, A.\,V., and S.\,V.~Listopad.} 2018. Model' gibridnoy intellektual'noy 
mnogoagentnoy sistemy geterogennogo myshleniya dlya informatsionnoy 
podgotovki operativnykh resheniy v~regional'nykh elektricheskikh setyakh [Model of 
a~hybrid intelligent multiagent system of heterogeneous thinking for preparation of 
information about operational decisions in a~regional power system]. \textit{Sistemy 
i~Sredstva Informatiki~--- Systems and Means of Informatics} 28(4):31--41.
\bibitem{11-kol-1}
\Aue{Tang, A.\,Y.\,C., and G.\,S.~Basheer.} 2017. A~Conflict Resolution Strategy 
Selection Method (ConfRSSM) in multi-agent systems. \textit{Int. 
J.~Adv. Computer Sci. Appl.} 8(5):398--404.
\bibitem{12-kol-1}
\Aue{Kolesnikov, A.\,V., and S.\,V.~Listopad.} 2019. Funk\-tsi\-o\-nal'\-naya struktura 
gibridnoy intellektual'noy mno\-go\-agent\-noy sistemy geterogennogo myshleniya dlya 
resheniya problemy vosstanovleniya raspredelitel'noy elektroseti [Functional 
structure of the hybrid intelligent multiagent system of heterogeneous thinking for 
solving the problem of restoring the distribution power grid]. \textit{Sistemy i~Sredstva 
Informatiki~--- Systems and Means of Informatics} 29(1):41--52.
\bibitem{13-kol-1}
\Aue{Gorodetskiy, V.\,I., M.\,S.~Grushinskiy, and A.\,V.~Khabalov.} 1998. 
Mno\-go\-agent\-nye 
sistemy (obzor) [Multiagent systems (review)]. 
\textit{Novosti iskusstvennogo intellekta} [Artificial Intelligence News] 2:64--116.
\bibitem{14-kol-1}
\Aue{Singh, M.\,P., and A.\,K.~Chopra.} 2010. Programming multiagent systems 
without programming agents. \textit{Programming multi-agent systems}.  
Eds. L.~Braubach, J.-P.~Briot, and J.~Thangarajah.
Lecture  notes in artificial intelligence ser. Springer. 5919:1--14. 
\bibitem{15-kol-1}
\Aue{Singh, R., A.~Singh, and S.~Mukherjee.} 2014. A~critical investigation of agent 
interaction protocols in multiagent systems. \textit{Int. J.~Advancements 
Technology} 5(2):72--81.

\bibitem{17-kol-1}
\Aue{Smith, G.} 1980. The Contract Net Protocol: High level communication and 
control in a distributed problem solver. \textit{IEEE T.~Comput.} 29(12):1104--1113.

\bibitem{16-kol-1}
\Aue{Smith, R.\,G.} 1981. \textit{A~framework for distributed problem solving}. 
Ann Arbor, MI: UMI Research Press. 188~p.

\bibitem{18-kol-1}
\Aue{Weerasooriya, D., A.\,S.~Rao, and K.~Ramamohanarao.} 1994. Design of 
a~concurrent agent-oriented language. 
\textit{Intelligent agents}. Eds. M.\,J.~Wooldridge and N.\,R.~Jennings.
  Lecture notes in computer science ser. Springer. 890:386--401.
\bibitem{19-kol-1}
\Aue{Zlotkin, G., and J.~Rosenschtein.} 1996. Mechanisms for automated 
negotiation in state oriented domain. \textit{J.~Artif. Intell. Res.} 
5:163--238.
\bibitem{20-kol-1}
\Aue{Marzougui, B., and K.~Barkaoui.} 2013. Interaction protocols in multi-agent 
systems based on agent Petri nets model. \textit{Int. J.~Advanced 
Computer Sci. Appl.} 4(7):166--173.

\end{thebibliography}

 }
 }

\end{multicols}

\vspace*{-6pt}

\hfill{\small\textit{Received March 31, 2019}}

%\pagebreak

%\vspace*{-18pt}
  
  \Contr
  
  
  \noindent
  \textbf{Kolesnikov Alexander V.} (b.\ 1948)~--- Doctor of Science in 
technology, professor, Institute of Physical and Mathematical Sciences and 
Information Technology, Immanuel Kant Baltic Federal University, 
14~A.~Nevskogo Str., Kaliningrad 236041, Russian Federation; 
\mbox{avkolesnikov@yandex.ru} 
  
  \vspace*{3pt}
  
  \noindent
  \textbf{Listopad Sergey V.} (b.\ 1984)~--- Candidate of  Science (PhD) in 
technology, senior scientist, Kaliningrad Branch of the Federal Research Center 
``Computer Science and Control'' of the Russian Academy of Sciences, 5~Gostinaya 
Str., Kaliningrad 236022, Russian Federation; \mbox{ser-list-post@yandex.ru }
  
  
\label{end\stat}

\renewcommand{\bibname}{\protect\rm Литература}  