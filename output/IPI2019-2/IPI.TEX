\documentclass[10pt]{book}
\usepackage[utf8]{inputenc}

\usepackage{latexsym,amssymb,amsfonts,amsmath,amsxtra,dsfont,
indentfirst,shapepar,%fleqn,%
picinpar,shadow,floatflt,enumerate,multicol,colortbl,moreverb,cite,ipi}

\usepackage{rotating}
\usepackage{mathrsfs}
\usepackage[noend]{algorithmic}
\usepackage{ulem}
\usepackage{graphicx}
%\usepackage{algorithm2e}
\usepackage[linesnumbered,boxed,ruled]{algorithm2e}
%\usepackage{xypic}
\usepackage{oldgerm}
\usepackage{epic}
\usepackage{eepic}


\SetAlgorithmName{Algorithm}{алгоритм}{Список алгоритмов}

%из Дюковой

\newcommand{\algKeyword}[1]{{\bf #1}}
\newcommand{\Proc}[1]{\text{\tt #1}}
\def\CALL{\algKeyword{call}~}

\newenvironment{AlgProcedure}[1]
{
    \small
    \medskip
    %    \hrule
    \medskip
    \algKeyword{PROCEDURE} #1
    \begin{algorithmic}[1]}
    {\end{algorithmic}
    %    \hrule
    \bigskip
}

\def\CALL{\algKeyword{call}~}

%конец для Дюковой

%\RequirePackage[ruled]{algorithm}


\input{epsf}

%\nofiles

%\includeonly{avtor}             %+pdf+
%\includeonly{obchak,avtor}
%\includeonly{pred}                 %+
%\includeonly{podgot-rus-site,podgot-eng-site}  
%\includeonly{ocherk} 
%\includeonly{nekrol} 
%\includeonly{ipi-ind} 
%\includeonly{index12}
%\includeonly{toc-rus, toc-en}
%\includeonly{toc-rus}
%\includeonly{toc-en} 


                  

%\includeonly{agalarov}                   %1+pdf+авт+
%\includeonly{bosov}                      %2+pdf+авт+
%\includeonly{shestakov}                  %3+pdf+авт+
%\includeonly{grusho}                     %4+pdf+авт+
%\includeonly{anashin}                    %5+pdf+авт+
%\includeonly{logachev}                   %6+pdf+авт???
%\includeonly{abgaryan}                   %7+pdf+авт
%\includeonly{shnurkov}                   %8+pdf+авт
%\includeonly{strijov}                    %9+pdf+авт+
%\includeonly{kudr}                       %10+pdf+авт+  
%\includeonly{kolesnikov}                 %11+pdf+авт+
%\includeonly{inkova}                     %12+pdf+авт
%\includeonly{zatsar}                     %13+pdf+авт
%\includeonly{grinchenko}                 %14+pdf+авт+
%\includeonly{lukashenko}                 %15+pdf
%\includeonly{kovalev}                    %16+pdf


 



%\includeonly{nekrol}             %+


%\includeonly{obchak}
%\includeonly{rekl}
%\includeonly{rekl-1}
%\includeonly{reshal}  %
%\includeonly{cover3}

\usepackage{acad}
%\usepackage{courier}
\usepackage{decor}
\usepackage{newton}
\usepackage{pragmatica}
\usepackage{zapfchan}
\usepackage{petrotex}
\usepackage{bm}                     % полужирные греческие буквы
\usepackage{upgreek}                % прямые греческие буквы
\usepackage{eufrak}
\usepackage{verbatim}

\renewcommand{\bottomfraction}{0.99}
\renewcommand{\topfraction}{0.99}
\renewcommand{\textfraction}{0.01}

\setcounter{secnumdepth}{1} %здесь - 3 + chapter = 4

\arraycolsep=1.5pt

%\usepackage[pdftex]{graphicx}

%\usepackage{oz}

%NEW COMMANDS


\renewcommand*{\hm}[1]{#1\nobreak\discretionary{}%
            {\hbox{$\mathsurround=0pt #1$}}{}} %% Дублирует знаки операций
                               %при переносе в формуле (перед знаком, который
                               %надо продублировать ставится команда \hm)

%\newcommand{\endproof}{\hfill$\Box$}
\renewcommand{\r}{\mathbb{R}}
%\newcommand{\I}{{\rm I\hspace{-0.7mm}I}}
%\newcommand{\Ikl}{{\tt{1}}\hspace*{-1.44mm}\mathtt{1}}
\newcommand{\Ik}{\mbox{{\small \tt {1}}\hspace{-1.3mm}{\tt 1}}}
\newcommand{\argmin}{\mathop{\mathrm{arg}\,\mathrm{min}}}
\newcommand{\argmax}{\mathop{\mathrm{arg}\,\mathrm{max}}}
%\newcommand{\capr}{\mathop{\cap\,}}
%\newcommand{\cupr}{\mathop{\cup\,}}
%\def\argmin{\mathop{arg\,min}}

\def\vrp{\varphi}
\def\prt{\partial}
\def\mm{{\sf M}}
\def\modnop#1{\mathop{#1}\limits_{n}}
\def\eam{\mathbin{{\mathop{=}\limits^{\mathrm{def}}}}}
\def\dey#1#2{#1 (#2)}
\def\deyc#1#2{#1 \cdot  #2}
\def\ra#1{\;\mathop{\to}\limits^{#1}\;}
\def\raz#1{\;\mathop{\longrightarrow}\limits^{\!\!\!#1}\;}
\def\ral#1{\;\mathop{\longrightarrow}\limits^{#1}\;}

\newcommand{\Nor}{\mathcal{N}}
\newcommand{\T}{\mathbb{T}}
\newcommand{\Z}{\mathbb{Z}}



\newcommand{\il}[2]{\int\limits_{#1}^{#2}}%интеграл с пределами #1 и #2

\def\sm2{\mathop {\sum\limits^{n^\Theta}\sum\limits^{n^\Theta}}}
\def\sss{\sum\limits}
\def\tr{,\,\ldots\,,\,}
\def\rk{\right]}
\def\lk{\left[}
\def\rf{\right\}}
\def\lf{\left\{}
\def\lv{\,\left\vert}
\def\rv{\right\vert\,}
\def\iii{\int\limits}
\def\iin{\int\limits_{-\infty}^\infty}
\def\rrv{\right\vert}


\def\ee{{\cal E}}
\def\ww{{\cal W}}
\def\yy{{\cal Y}}
\def\vv{{\cal V}}

\newcommand{\R}{\mathbb R}
\newcommand{\E}{\mathbb E}
\newcommand{\N}{\mathbb N}

\renewcommand{\P}{\mathbb{P}}

\newcommand{\h}{{\bf H}}
\newcommand{\p}{{\sf P}}  % вероятность

\newcommand{\e}{{\sf E}}  % мат. ожидание
\newcommand{\D}{{\sf D}}  % дисперсия
\newcommand{\eps}{\varepsilon}
\newcommand{\vp}{{\mathbf p}}
\newcommand{\vz}{{\mathbf z}}
\newcommand{\vx}{{\mathbf x}}
\newcommand{\vf}{{\mathbf f}}
\newcommand{\F}{{\mathcal F}}
\def\ap{{\mathrm{ЭР}}}
\newcommand{\ud}{\Delta_n} %uniform ditance
\newcommand{\nud}{\Delta_n(x)}
%\renewcommand{\Re}{\mathrm{Re}\,}

\newcommand{\abs}[1]{\left\vert#1\right\vert}

\newcommand{\norm}[1]{\left\Vert#1\right\Vert}
\def\da{(\Delta_t,A)}

\newcommand{\corr}{\mathrm{corr}}

\newcommand{\cov}{\mathrm{cov}}
\newcommand{\Expect}{\mathbb{E}}

\def\w{\omega}
\def\W{\Omega}

\def\inh{\int\limits_{nh}^{(n+1)h}}

\def\sumin{\sum_{i=1}^N}


\def\bxt{(Y,t)}
\def\xt{(y,t)}

\def\ovth{{\fr{\tau-nh}{h}}}
\def\ov{\overline}
\def\tm{\tilde m}
\def\tl{\tilde\lambda}
\def\tB{\widetilde B}
\def\tb{\tilde b}
\def\ld{\ldots}
\def\cd{\cdots}


\DeclareMathOperator{\sign}{sign}

%\newcommand{\gr}{{\geqslant}}


\newcommand{\g}{\mbox{\textit{g}}}

\renewcommand{\la}{\lambda}
\newcommand{\si}{\sigma}
\newcommand{\alp}{\alpha}

\newcommand{\pto}{\stackrel{P}{\longrightarrow}} % сходимость по веpоятности

\newcommand{\eqd}{\stackrel{\mathrm{d}}{=}} % равенство по pаспpеделению
\newcommand{\eqdelta}{\stackrel{\triangle}{=}} % равенство по pаспpеделению

\def\be#1{\begin{equation}\label{#1}}
\def\ee{\end{equation}}
\def\re#1{(\ref{#1})}

\def\bn{\begin{enumerate}}
\def\en{\end{enumerate}}
\def\bi{\begin{itemize}}
\def\ei{\end{itemize}}
%\def\i{\item}

%\newcommand{\kp}{\kappa}
%\def\Q{{\cal Q}} \def\H{{\cal H}}
%\newcommand{\bet}{\beta_{2+\delta}}


%\newtheorem{definition}{Определение}
%\renewcommand{\thedefinition}{\arabic{definition}.}
%END NEW COMMANDS

%\renewcommand{\baselinestretch}{1.2}

%\pagestyle{myheadings}

\setlength{\textwidth}{167mm}      % 122mm
\setlength{\textheight}{658pt}
%\setlength{\textheight}{635.6pt}
\setlength{\columnsep}{4.5mm}

\setcounter{secnumdepth}{4}

%\addtolength{\headheight}{2pt}
%\addtolength{\headsep}{-2mm}

\addtolength{\topmargin}{-7mm}  % for printing


%\hoffset=-30mm  % From Yap
\hoffset=-23mm  % From Acrobat

%\voffset=0mm % From Yap
\voffset=-5mm   % From Acrobat

%\addtolength{\evensidemargin}{-2.5mm} % for printing
%\addtolength{\oddsidemargin}{2.5mm}  % for printing

\addtolength{\evensidemargin}{-12mm} % for printing
\addtolength{\oddsidemargin}{8mm}  % for printing

%\renewcommand{\thefootnote}{\fnsymbol{footnote}}
%\renewcommand{\thefootnote}{\arabic{footnote}}
\renewcommand{\figurename}{\protect\bf Рис.}
\renewcommand{\tablename}{\protect\bf Таблица}

\newcommand{\Caption}[1]{\caption{\protect\small %\baselineskip=2.5ex
#1}}

\renewcommand{\thefigure}{\arabic{figure}}
\renewcommand{\thetable}{\arabic{table}}
\renewcommand{\theequation}{\arabic{equation}}
\renewcommand{\thesection}{\arabic{section}}

\renewcommand{\contentsname}{СОДЕРЖАНИЕ}
\newcommand{\fr}[2]{\displaystyle\frac{\displaystyle #1\mathstrut}{\displaystyle #2\mathstrut}}

%\renewcommand{\thefootnote}{\fnsymbol{footnote}}
%\newcommand{\g}{\mbox{\textit{g}}}

%\newcommand{\Caption}[1]{\caption{\protect\small\baselineskip=2ex #1}}
\newcounter{razdel}
\setcounter{razdel}{0}


\newcommand{\titel}[4]{%
\

\vspace*{5pt}

\ifodd\therazdel {\raggedright\noindent\Large\textrm\textbf
 \lineskip .75em
  \baselineskip=3.2ex #1 \par}
\vskip 1em {\noindent\large\textrm\textbf #2 \par}
\addcontentsline{toc}{subsection}{{\textrm\textbf #1}\protect\newline #2}
\def\rightheadline{\underline{\noindent\hbox to \textwidth{\hfill\small\textrm{#4}
%\hfill \large\bf\thepage
}}}
\def\leftheadline{\underline{\noindent\parbox{\textwidth}{
%\raggedleft\large\bf\thepage \hfill
\small\textit{#3}\hfill}}}
\def\leftfootline{\small{\textbf{\thepage}
\hfill ИНФОРМАТИКА И ЕЁ ПРИМЕНЕНИЯ\ \ \ том~13\ \ \ выпуск 2\ \ \ 2019}
}%
 \def\rightfootline{\small{ИНФОРМАТИКА И ЕЁ ПРИМЕНЕНИЯ\ \ \ том~13\ \ \ выпуск~2\ \ \ 2019
\hfill \textbf{\thepage}}}
\vskip 2em \setcounter{figure}{0}
\setcounter{table}{0}
\setcounter{equation}{0}
\setcounter{section}{0}
\setcounter{subsection}{0}
\setcounter{subsubsection}{0}
\setcounter{footnote}{0}
\setcounter{razdel}{0}
%\end{flushleft}
\else {
 \raggedright\noindent\Large\textrm\textbf
 \lineskip .75em
\baselineskip=3.2ex #1 \par} \vskip 1em
%\begin{flushleft}
{\noindent\large\textrm\textbf #2 \par}
\addcontentsline{toc}{subsection}{{\textrm\textbf #1}\protect\newline #2}
\def\rightheadline{\underline{\noindent\hbox to \textwidth{\hfill\small\textrm{#4}
%\hfill \large\bf\thepage
}}}
\def\leftheadline{\underline{\noindent\parbox{\textwidth}{%\raggedleft\large\bf\thepage \hfill
\small\textit{#3}\hfill}}}
\def\leftfootline{\small{\textbf{\thepage}
\hfill ИНФОРМАТИКА И ЕЁ ПРИМЕНЕНИЯ\ \ \ том~13\ \ \ выпуск~2\ \ \ 2019}
}%
 \def\rightfootline{\small{ИНФОРМАТИКА И ЕЁ ПРИМЕНЕНИЯ\ \ \ том~13\ \ \ выпуск~2\ \ \ 2019
\hfill \textbf{\thepage}}} \vskip 2em \setcounter{figure}{0}
\setcounter{table}{0} \setcounter{equation}{0} \setcounter{section}{0}
\setcounter{subsection}{0} \setcounter{subsubsection}{0}
\setcounter{footnote}{0}
%\end{flushleft}
\fi}

\newcommand{\titelr}[2]{%
\

\vspace*{5pt}

\ifodd\therazdel {\raggedright\noindent%\Large\textrm\textbf
 \lineskip .75em
  \baselineskip=3.2ex #1 \par}
\vskip 1em {\noindent\normalsize\textrm\textbf #2 \par}
\else {
 \raggedright\noindent\Large\textrm\textbf
 \lineskip .75em
\baselineskip=3.2ex #1 \par} \vskip 1em
%\begin{flushleft}
{\noindent\large\textrm\textbf #2 \par
%\noindent\normalsize\textrm\textbf #2 \par
} \fi}

\newcommand{\titele}[5]{%
\

%\vspace*{5pt}

\ifodd\therazdel {\raggedright\noindent\large
\textrm\textbf
 \lineskip .75em
%  \baselineskip=3.2ex
#1 \par}
\vskip .5em {\noindent\large\textrm\textbf #2 \par}
\vskip .5em
 {\noindent\textrm #3 \par}
\addcontentsline{toc}{subsection}{{\textrm\textbf #1}\protect\newline #2}
\def\rightheadline{\underline{\noindent\hbox to \textwidth{\hfill\small\textrm{#4}
%\hfill \large\bf\thepage
}}}
\def\leftheadline{\underline{\noindent\parbox{\textwidth}{
%\raggedleft\large\bf\thepage \hfill
\small\textrm{#5}\hfill}}}
\def\leftfootline{\small{\textbf{\thepage}
\hfill ИНФОРМАТИКА И ЕЁ ПРИМЕНЕНИЯ\ \ \ том~13\ \ \ выпуск~2\ \ \ 2019}
}%
 \def\rightfootline{\small{ИНФОРМАТИКА И ЕЁ ПРИМЕНЕНИЯ\ \ \ том~13\ \ \ выпуск~2\ \ \ 2019
\hfill \textbf{\thepage}}} \vskip 1em \setcounter{figure}{0}
\setcounter{table}{0} \setcounter{equation}{0} \setcounter{section}{0}
\setcounter{subsection}{0} \setcounter{subsubsection}{0}
\setcounter{footnote}{0} \setcounter{razdel}{0}
%\end{flushleft}
\else {
 \raggedright\noindent\large
 \textrm\textbf
 \lineskip .75em
%\baselineskip=3.2ex
#1 \par} \vskip .5em
%\begin{flushleft}
{\noindent\large\textrm\textbf #2 \par} \vskip .5em
 {\noindent\textrm #3 \par}
\addcontentsline{toc}{subsection}{{\textrm\textbf #1}\protect\newline #2}
\def\rightheadline{\underline{\noindent\hbox to \textwidth{\hfill\small\textrm{#4}
%\hfill \large\bf\thepage
}}}
\def\leftheadline{\underline{\noindent\parbox{\textwidth}{%\raggedleft\large\bf\thepage \hfill
\small\textrm{#5}\hfill}}}
\def\leftfootline{\small{\textbf{\thepage}
\hfill ИНФОРМАТИКА И ЕЁ ПРИМЕНЕНИЯ\ \ \ том~13\ \ \ выпуск~2\ \ \ 2019}
}%
 \def\rightfootline{\small{ИНФОРМАТИКА И ЕЁ ПРИМЕНЕНИЯ\ \ \ том~13\ \ \ выпуск~2\ \ \ 2019
\hfill \textbf{\thepage}}} \vskip 1em \setcounter{figure}{0}
\setcounter{table}{0} \setcounter{equation}{0} \setcounter{section}{0}
\setcounter{subsection}{0} \setcounter{subsubsection}{0}
\setcounter{footnote}{0}
%\end{flushleft}
\fi}

\def\Abst#1{
\begin{center}\small\nwt
\parbox{150mm}{%\baselineskip=2.5ex
\textbf{Аннотация:}\ \
%\hspace*{\parindent}
#1}
\end{center}}
\def\Abste#1{
\begin{center}\small\nwt
\parbox{150mm}{%\baselineskip=2.5ex
\textbf{Abstract:}\ \
%\hspace*{\parindent}
#1}
\end{center}}

\def\DOI#1{
\begin{center}\small\nwt
\parbox{150mm}{%\baselineskip=2.5ex
\textbf{DOI:}\ \
%\hspace*{\parindent}
#1}
\end{center}}

\def\Abstend#1{
\begin{center}\small\nwt
\parbox{150mm}{%\baselineskip=2.5ex
%\hspace*{\parindent}
#1}
\end{center}}


\def\KW#1{
\begin{center}\small\nwt
\parbox{150mm}{%\baselineskip=2.5ex
\textbf{Ключевые слова:}\ \ #1}
\end{center}}

\def\KWE#1{
\begin{center}\small\nwt
\parbox{150mm}{%\baselineskip=2.5ex
\textbf{Keywords:}\ \ #1}
\end{center}}


\def\KWN#1{
%\begin{center}
%\small
%\parbox{150mm}\end{center}
}

\newcommand{\Avtors}[1]{%\smallskip
%\vspace*{.5pt}
\hangindent=23pt\noindent
%\nwt
{\bfseries#1}\
}


\renewcommand{\thesubsection}{\thesection.\arabic{subsection}\hspace*{-5pt}}
\renewcommand{\thesubsubsection}{\thesubsection\hspace*{5pt}.\arabic{subsubsection}\hspace*{-3pt}}

\newcommand{\Ack}{\section*{\protect\rmfamily Acknowledgments}\noindent}
\newcommand{\Contr}{\section*{\protect\rmfamily Contributors}\noindent}
\newcommand{\Contrl}{\section*{\protect\rmfamily Contributor}\noindent}

\makeindex


\begin{document}
\Rus

\nwt
%\ptb


%\renewcommand{\contentsname}{\protect\Large\bf Содержание}

\setcounter{tocdepth}{2}

%\tableofcontents

\renewcommand{\bibname}{\protect\rmfamily Литература}
  \def\Au#1{{\it #1}}
    \def\Aue#1{{#1}}

%\newcommand{\No}{№}
  \newcommand{\tg}{\,\mathrm{tg}\,}
    \newcommand{\ctg}{\,\mathrm{ctg}\,}
  \newcommand{\arctg}{\,\mathrm{arctg}\,}

\def\forallb{\mathop{\forall}}
\def\cupb{\mathop{\cup}}
\def\existsb{\mathop{\exists}}


\newpage
\addtocounter{razdel}{1}
%\def\razd{РЕГУЛИРУЕМЫЙ ЭЛЕКТРОПРИВОД ДЛЯ ЭЛЕКТРОЭНЕРГЕТИКИ}


\setcounter{page}{2}

%   { %\Large  
   { %\baselineskip=16.6pt
   
   \vspace*{-48pt}
   \begin{center}\LARGE
   \textit{Предисловие}
   \end{center}
   
   %\vspace*{2.5mm}
   
   \vspace*{25mm}
   
   \thispagestyle{empty}
   
   { %\small 

    
Вниманию читателей журнала <<Информатика и её применения>> предлагается 
очередной тематический выпуск <<Вероятностно-статистические методы и 
задачи информатики и информационных технологий>>. Предыдущие тематические 
выпуски журнала по данному направлению вышли в 2008~г.\ (т.~2, вып.~2), 
в 2009~г.\ (т.~3, вып.~3) и в 2010~г.\ (т.~4, вып.~2). 

Статьи, собранные в данном журнале, посвящены разработке новых вероятностно-статистических 
методов, ориентированных на применение к решению конкретных задач информатики и информационных 
технологий, а также~--- в ряде случаев~--- и других прикладных задач. Проблематика, охватываемая 
публикуемыми работами, развивается в рамках научного сотрудничества между Институтом проблем 
информатики Российской академии наук (ИПИ РАН) и Факультетом вычислительной математики и 
кибернетики Московского государственного университета им.\ М.\,В.~Ломоносова в ходе работ 
над совместными научными проектами (в том числе в рамках функционирования 
Научно-образовательного центра <<Вероятностно-статистические методы анализа рисков>>). 
Многие из авторов статей, включенных в данный номер журнала, являются активными участниками 
традиционного международного семинара по проблемам устойчивости стохастических моделей, 
руководимого В.\,М.~Золотаревым и В.\,Ю.~Королевым; регулярные сессии этого семинара 
проводятся под эгидой МГУ и ИПИ РАН (в 2011~г.\ указанный семинар проводится в октябре 
в Калининградской области РФ). 

Наряду с представителями ИПИ РАН и МГУ в число авторов данного выпуска журнала входят 
ученые из Научно-исследовательского института системных исследований РАН, Института 
проблем технологии микроэлектроники и особочистых материалов РАН, Института 
прикладных математических исследований Карельского НЦ РАН, Московского 
авиационного института, Вологодского государственного педагогического университета, 
НИИММ им.\ Н.\,Г.~Чеботарева, Казанского государственного университета, Дебреценского 
университета (Венгрия).

Несколько статей выпуска посвящено разработке и применению стохастических методов и 
информационных технологий для решения различных прикладных задач. В~работе В.\,Г.~Ушакова 
и О.\,В.~Шестакова рассмотрена задача определения вероятностных характеристик случайных 
функций по распределениям интегральных преобразований, возникающих в задачах эмиссионной 
томографии. В~статье Д.\,О.~Яковенко и М.\,А.~Целищева рассмотрены некоторые вопросы 
математической теории риска и предложен новый подход к диверсификации инвестиционных 
портфелей. Работа И.\,А.~Кудрявцевой и А.\,В.~Пантелеева посвящена построению и 
исследованию математической модели, описывающей динамику сильноионизованной плазмы. 
В~статье П.\,П.~Кольцова изучается качество работы ряда алгоритмов сегментации изображений. 
Статья А.\,Н.~Чупрунова и И.~Фазекаша посвящена вероятностному анализу числа без\-оши\-бочных 
блоков при помехоустойчивом кодировании; получены усиленные законы больших чисел для указанных 
величин.

В данном выпуске традиционно присутствует тематика, весьма активно разрабатываемая в течение 
многих лет специалистами ИПИ РАН и МГУ,~--- методы моделирования и управления для 
информационно-телекоммуникационных и вычислительных систем, в частности методы 
теории массового обслуживания. В~статье А.\,И.~Зейфмана с соавторами рассматриваются 
модели обслуживания, описываемые марковскими цепями с непрерывным временем в случае 
наличия катастроф. В~работе М.\,М.~Лери и И.\,А.~Чеплюковой рассматриваются случайные 
графы Интернет-типа, т.\,е.\ графы, степени вершин которых имеют степенные распределения; 
такие задачи находят применение при исследовании глобальных сетей передачи данных. 
Работа Р.\,В.~Разумчика посвящена исследованию систем массового обслуживания специального 
вида~--- с отрицательными заявками и хранением вытесненных заявок.

Ряд статей посвящен развитию перспективных теоретических 
вероятностно-статистических методов, которые находят широкое применение в различных 
задачах информатики и информационных технологий. В~работе В.\,Е.~Бенинга, А.\,К.~Горшенина 
и В.\,Ю.~Королева рассмотрена задача статистической проверки гипотез о числе компонент 
смеси вероятностных распределений, приводится конструкция асимптотически наиболее мощного 
критерия. Результаты этой работы найдут применение в ряде прикладных задач, использующих 
математическую модель смеси вероятностных распределений (в информатике, моделировании 
финансовых рынков, физике турбулентной плазмы и~т.\,д.). В~статье В.\,Ю.~Королева, 
И.\,Г.~Шевцовой и С.\,Я.~Шоргина строится новая, улучшенная оценка точности нормальной 
аппроксимации для пуассоновских случайных сумм; как известно, указанные случайные суммы 
широко используются в качестве моделей многих реальных объектов, в том числе в информатике, 
физике и других прикладных областях. Работа В.\,Г.~Ушакова и Н.\,Г.~Ушакова посвящена 
исследованию ядерной оценки плотности распределения; эти результаты могут применяться, 
в част\-ности, при анализе трафика в телекоммуникационных системах. Серьезные приложения 
в статистике могут получить результаты работы О.\,В.~Шестакова, в которой доказаны оценки 
скорости сходимости распределения выборочного абсолютного медианного отклонения к нормальному 
закону. 

\smallskip

Редакционная коллегия журнала выражает надежду, что данный тематический  выпуск 
будет интересен специалистам в области теории вероятностей и математической статистики 
и их применения к решению задач информатики и информационных технологий.
     
     %\vfill 
     \vspace*{20mm}
     \noindent
     Заместитель главного редактора журнала <<Информатика и её 
применения>>,\\
     директор ИПИ РАН, академик  \hfill
     \textit{И.\,А.~Соколов}\\
     
     \noindent
     Редактор-составитель тематического выпуска,\\
     профессор кафедры математической статистики факультета\\
      вычислительной математики и кибернетики МГУ им.\ М.\,В.~Ломоносова,\\
     ведущий научный сотрудник ИПИ РАН,\\ 
доктор физико-математических наук \hfill
      \textit{В.\,Ю.~Королев}
     
     } }
     }

 
\def\stat{agalarov}


\def\tit{ПРИБЛИЖЕННЫЙ МЕТОД ВЫЧИСЛЕНИЯ ХАРАКТЕРИСТИК УЗЛА 
ТЕЛЕКОММУНИКАЦИОННОЙ СЕТИ С~ПОВТОРНЫМИ ПЕРЕДАЧАМИ}
\def\titkol{Приближенный метод вычисления характеристик узла 
телекоммуникационной сети с~повторными передачами} 

\def\autkol{Я.\,М.~Агаларов}
\def\aut{Я.\,М.~Агаларов$^1$}

\titel{\tit}{\aut}{\autkol}{\titkol}

%{\renewcommand{\thefootnote}{\fnsymbol{footnote}}\footnotetext[1]
%{Работа выполнена при поддержке РФФИ, проекты 08--07--00152 и 08--01--00567.}}

\renewcommand{\thefootnote}{\arabic{footnote}}
\footnotetext[1]{Институт проблем
информатики Российской академии наук, agglar@yandex.ru}

%\vspace*{-6pt}


\Abst{Рассмотрена модель узла коммутации пакетов c повторными передачами для двух 
схем распределения буферной памяти: полнодоступной и полного разделения. Предложен 
приближенный метод вычисления интенсивностей потоков и вероятностей блокировок узла. 
Получены необходимые и достаточные условия существования и единственности решения 
уравнения для потоков в узле при установившемся режиме работы и доказана сходимость 
итерационного метода решения указанного уравнения.}

\KW{узел коммутации пакетов; буферная память; повторные передачи; вероятности 
блокировок; итерационный метод}

      \vskip 18pt plus 9pt minus 6pt

      \thispagestyle{headings}

      \begin{multicols}{2}

      \label{st\stat}


\section{Введение}

    Одной из основных задач предварительного анализа 
телекоммуникационных сетей коммутации пакетов с ограниченной буферной 
памятью является расчет характеристик потоков и вероятностей блокировок в 
узлах связи. Важность указанных характеристик определяется тем, что от их 
значений существенным образом зависят другие основные показатели сети 
(пропускная способность, задержки пакетов и~др.). 

    Существует множество различных моделей узлов коммутации пакетов и 
методов их расчета (см., например,~[1--6]). Для моделей, рассматривающих 
узел с ограниченной буферной памятью как систему массового обслуживания 
(CMO) типа 
$
\begin{matrix}
M \\ \lambda
\end{matrix}
\left |
\begin{matrix}
M \\ \lambda
\end{matrix}
\right |
\overline{m} \vert N
$ или  $\vert PH\vert PH\vert 1\vert r$, в предположении отсутствия повторных 
передач пакетов получены точные методы вычисления характеристик 
узлов~[1, 3, 4, 6]. Приближенные методы расчета узлов, учитывающие повторные 
попытки передачи, используют модели типа $\vert PH\vert PH\vert 1\vert r$ или 
$
\begin{matrix}
M \\ \lambda
\end{matrix}
\left |
\begin{matrix}
M \\ \lambda
\end{matrix}
\right |
1 \vert N
$ и являются 
итерационными~[2, 3, 5, 7]. Для моделей типа 
$BM\!AP\vert PH\vert 1$, $M\vert G\vert 1\vert r$ и $M\!AP\vert 
(PH,PH)\vert 1$ с повторными заявками получены точные методы вычисления 
характеристик (например, в работах~[8--10]), которые также могут быть 
использованы при расчете узлов.

    Ниже будут рассмотрены модели узла коммутации пакетов с повторными 
передачами для двух схем распределения буферной памяти: с 
полнодоступными буферами и с полным разделением буферной памяти. 
Предлагается приближенный метод расчета характеристик, который в качестве 
модели узла использует СМО типа $
\begin{matrix}
M \\ \lambda
\end{matrix}
\left |
\begin{matrix}
M \\ \lambda
\end{matrix}
\right |
\overline{m} \vert N
$ с повторными заявками. Доказаны утверждения о 
достаточных и необходимых условиях существования и единственности 
решения уравнения для вероятности блокировки в установившемся режиме 
работы и сходимости предлагаемого итерационного метода. 

\section{Модель узла}

    Математическая модель узла представляется в виде СМО с ограниченной 
буферной памятью и различными потоками заявок, каждая из которых требует 
обслуживания только на одной из многоканальных линий связи. 

    Пусть $0<N<\infty$~--- число мест хранения в буферной памяти, $u$~--- 
узел связи, $v$~--- линия связи, $\Omega_u^+$~--- множество исходящих из 
узла~$u$ линий, $c_v$~--- канальная емкость линии~$v$. Поток заявок, 
тре\-бу\-ющих обслуживания на линии~$v$, назовем $v$-по\-то\-ком, заявки этого 
потока~--- $v$-за\-яв\-ка\-ми.


    Пусть выполняются следующие предположения: 
\begin{enumerate}[1.]
\item Места в буферной памяти распределяются согласно одной из двух 
схем:
\begin{enumerate}[($i$)]
\item полнодоступная схема~--- каждое свободное место хранения доступно 
любой заявке;
\item схема полного разделения памяти~--- $v$-за\-яв\-кам доступны всего 
$N_v$ мест, где $\sum\limits_{v\in\Omega_u^+} N_v=N$.
\end{enumerate}
\item Если в момент поступления $v$-заявки в буферной памяти есть 
доступное свободное место, то она сразу занимает это место. Если в момент 
поступления $v$-заявки в системе нет свободного доступного места 
хранения, то поступившая заявка через некоторое время повторно поступает 
на систему, оставаясь $v$-заявкой. 
\item Интенсивности первичных потоков $v$-заявок~--- заданные величины 
$0<\Lambda_v<\infty$, $v\in \Omega_u^+$. Суммарные потоки первичных и 
повторных $v$-заявок являются независимыми в совокупности 
пуассоновскими потоками. Для обслуживания $v$-заявки требуется 
одновременно одно место хранения и один канал типа~$v$, $v\in 
\Omega_u^+$.
\item Первичные нагрузки~--- реализуемые, т.\,е.\ в данном случае 
интенсивности входных первичных потоков равны интенсивностям 
выходных потоков выполненных заявок. 
\item Принятые в СМО $v$-заявки обслуживаются линией~$v$ в порядке 
поступления. 
\item Время занятия канала $v$-заявкой~--- экспоненциально 
распределенная случайная величина с параметром $0<\mu_v<\infty$, 
$v\in\Omega_u^+$, независимая от других случайных событий в узле.
\item Выполненная $v$-заявка с вероятностью~$B_v$ повторяется через 
заданное время~$\tau_v$ (тайм-аут) и с вероятностью $1-B_v$ покидает 
систему через время~$t_v$ навсегда, сразу освободив занятый канал и место 
буферной памяти.
\end{enumerate}

   Будем говорить, что узел блокирован для $v$-за\-яв\-ки, если в буферной 
памяти отсутствует доступное место хранения. Ставится задача вычисления 
вероятностей блокировок и интенсивностей потоков в узле.

\section{Вычисление вероятности блокировки и~интенсивностей~потоков} 

   Пусть $\Lambda_v^*$~--- интенсивность суммарного потока внешних 
заявок, требующих передачи по линии~$v$, $\pi_v$~--- вероятность блокировки 
узла для заявок, требующих передачи по исходящей из узла линии~$v$. 

    Пусть в узле используется полнодоступная схема распределения 
буферной памяти. Тогда, как следует из описания модели, $\pi_v 
=\pi_{v^\prime},\,v,\,v^\prime\in \Omega_u^+$, и для 
интенсивностей~$\Lambda_v^*$, $v\in\Omega_u^+$, справедливы соотношения:
\begin{equation*}
\Lambda_v^* = \fr{\Lambda_v}{1-\pi}\,,
%\label{e1aga}
\end{equation*}
    где
    $\pi =\pi_v$, $v\in\Omega_u^+$.

    Пусть 
    $\overline{k} = \{\overline{k}_v$, $v\in\Omega_u^+\}$~--- состояние 
буферной памяти узла, $\overline{k}_v =\left ( k_v,\,k_v^\prime,\,k_v^{\prime\prime}\right )$; 
$k_v$~--- число $v$-заявок в буферной 
памяти, ожидающих выполнения линией~$v$; $k^\prime_v$~--- число 
$v$-заявок в буферной памяти, ожидающих тайм-аут и неуспешно переданных 
в последующий узел; $k_v^{\prime\prime}$~--- число $v$-за\-явок в буферной 
памяти, успешно переданных в последующий узел и ожидающих 
потверждения; 
$A_m = \left \{ \overline{k}:\ \sum\limits_{v\in\Omega_u^+} \left ( 
k_v+k_v^\prime + k_v^{\prime\prime}\right ) =m \right \}$~--- множество различных 
состояний, при которых в памяти узла занято ровно $m$~буферов. Тогда с 
учетом введенных выше обозначений и предположений для ве\-ро\-ят\-ности 
блокировки узла можно написать формулу~\cite{1aga, 2aga}:
\begin{equation}
\pi = \fr{1}{G_N}\sum\limits_{\overline{k}\in A_N} 
p\left (\overline{k},\overline{\rho}^*\right )\,,
\label{e2aga}
\end{equation}
где  
\begin{gather}
p(\overline{k},\overline{\rho}^*) = \prod\limits_{v\in\Omega_u^+} z_v (\pi, 
\rho_v , k_v , k_v^\prime , k_v^{\prime\prime})\,;\\
z_v (\pi, \rho_v , k_v , k_v^\prime , k_v^{\prime\prime}) ={}\notag\\
\!\!{}=
\begin{cases}
 \fr{\rho_v^{\prime *k_v^\prime}}{k_v^{\prime}!}\,
\fr{\rho_v^{\prime\prime * k_v^{\prime\prime}}}{ k_v^{\prime\prime}!}  \,
\fr{\rho_v^{*k_v}}{ k_{v}!} 
&\mbox{при}\ k_v<c_v\,,\\
 \fr{\rho_v^{\prime * k_v^\prime}}{k_v^{\prime}!} \,
\fr{\rho_v^{\prime\prime * k_v^{\prime\prime}}} { k_v^{\prime\prime}!} 
\fr{\rho_v^{*k_v}}{ c_{v}!c_v^{k_v- c_v}} 
& \mbox{при}\ k_v\geq c_v\,;
\end{cases}\\
G_N = \sum\limits_{m=0}^N\sum\limits_{\overline{k}\in A_m}
p(\overline{k},\overline{\rho}^*)\,;\\ 
\overline{\rho}^*=\{\rho_v^*,\,v\in\Omega_u^+\}\,;\\
\rho_v^* = \fr{\rho_v}{1-\pi}\,;\quad \rho_v =\fr{\Lambda_v}{\mu_v(1- B_v)}\,;\\
\rho_v^{\prime *} =\rho_v^*\mu_v\tau_vB_v\,;\quad \rho_v^{\prime\prime *}=
p_v^* \mu_vt_v,\,\quad  v\in \Omega_u^+\,.\label{e3aga}
\end{gather}

Переобозначив $1-\pi$ через $y$, выражение в правой части равенства~(2)~--- через 
$p_{\overline{k}}(\overline{\rho},y)$, выражение в правой части равенства~(4)~--- 
через $g_N(\overline{\rho},y)$, а выражение в правой 
части равенства~(1)~--- через $1-q_N (\overline{\rho},y)$, 
где $\overline{\rho} = (\rho_v,\,v\in \Omega_u^+)$, $\rho_v = \rho_v^*y\;=$\linebreak 
$=\;\Lambda_v/(\mu_v(1-B_v))$, $v\in\Omega_u^+$, получим нелинейное уравнение 
относительно неизвестной переменной~$y$:
\begin{equation}
y=q_N(\overline{\rho},y)\,.
\label{e4aga}
\end{equation}

    Решим уравнение~(8). Как следует из~(2)--(7), верно 
равенство
\begin{equation}
q_N(\overline{\rho},y) = \fr{g_{N-1}(\overline{\rho},y )}{g_N(\overline{\rho},y)}\,.
\label{e5aga}
\end{equation}
Введем функцию  $d_n(\overline{\rho} ,y)$ среднего числа заявок в узле с 
буферной памятью емкости $n\geq 0$:
$$
d_n(\overline{\rho} ,y) = 
\fr{1}{g_n(\overline{\rho},y)}\,\sum\limits_{m=0}^n m\sum\limits_{\overline{k}\in 
A_m} p_{\overline{k}}(\overline{\rho},y)\,.
$$
Заметим, что $g_n$, $d_n$ и $q_n$, 
$n\geq 0$,~--- непрерывно-дифференцируемые функции по $y\in (0,\,1]$. Взяв 
производную функции~$g_n$ по~$y$, из~(2)--(7) получим
\begin{multline}
\fr{\partial g_n(\overline{\rho},y)}{\partial y} ={}\\
{}= -\sum\limits_{m=0}^n m 
\sum\limits_{\overline{k}\in A_m}\fr{\prod\limits_{v\in\Omega_u^+} z_n 
(0,\rho_v, k_v, k_v^\prime , k_v^{\prime\prime})}{y^{m+1}}={}\\
{}= -\fr{1}{y}\,g_n (\overline{\rho},y)d_n(\overline{\rho},y)\,.
\label{e6aga}
\end{multline}
Взяв производную функции $q_N$ по $y$, из~(\ref{e5aga}) и~(\ref{e6aga}) 
получим
\begin{equation}
\fr{\partial q_N(\overline{\rho},y)}{\partial y} = \fr{q_N(\overline{\rho},y)}{y}\left 
[ d_N (\overline{\rho},y)-d_{N-1}(\overline{\rho},y)\right ]\,.
\label{e7aga}
\end{equation}
    Докажем несколько утверждений о свойствах 
функции~$q_N(\overline{\rho},y)$.
\medskip

\noindent
\textbf{Утверждение 1.} \textit{Справедливы неравенства}
\begin{multline}
0<d_{n+1}(\overline{\rho},y)-d_n(\overline{\rho},y) <1\,,\\
\ \ \ \ \ \ \ \ \ \ \ \ \ \ \ \ \ \ \ \ y\in (0,\,1]\,, \ n\geq 0\,.
\label{e8aga}
\end{multline}


\noindent

Д\,о\,к\,а\,з\,а\,т\,е\,л\,ь\,с\,т\,в\,о\,.\ Подставив выражение для функции 
$d_n(\overline{\rho},y)$ и проведя преобразования, получим
\begin{multline*}
d_{n+1}(\overline{\rho},y) -d_n(\overline{\rho},y) = 
\fr{\sum\limits_{m=0}^{n+1}m\sum\limits_{\overline{k}\in A_m} 
p_{\overline{k}}(\overline{\rho},y)}
{\sum\limits_{m=0}^{n+1}
\sum\limits_{\overline{k}\in A_m} p_{\overline{k}}(\overline{\rho},y)} - {}\\
{}-
\fr{\sum\limits_{m=0}^n m \sum\limits_{\overline{k}\in A_m} p_{\overline{k}} 
(\overline{\rho},y)}{\sum\limits_{m=0}^n
\sum\limits_{\overline{k}\in A_m}p_{\overline{k}}(\overline{\rho},y)}={}\\
{}=\fr{\sum\limits_{m=1}^n m \sum\limits_{\overline{k}\in 
A_m}p_{\overline{k}}(\overline{\rho},y)+(n+1)\sum\limits_{\overline{k}\in 
A_{n+1}}  p_{\overline{k}}(\overline{\rho},y)}{\sum\limits_{m=0}^n\sum\limits_{\overline{k
}\in A_m}p_{\overline{k}}(\overline{\rho},y)+\sum\limits_{\overline{k}\in 
A_{n+1}}p_{\overline{k}}(\overline{\rho},y)} -{}
\end{multline*}
\begin{multline}
{}-
\fr{\sum\limits_{m=0}^n m 
\sum\limits_{\overline{k}\in A_m}p_{\overline{k}}(\overline{\rho},y)}
{\sum\limits_{m=0}^n\sum\limits_{\overline{k}\in A_m} 
p_{\overline{k}}(\overline{\rho},y)}={}\\
{}=\fr{(n+1)\sum\limits_{\overline{k}\in 
A_{n+1}}p_{\overline{k}}(\overline{\rho},y)g_n(\overline{\rho},y)}{g_{n+1}(\overline{\rho},y) g_n(\overline{\rho},y)} -{}\\
{}-
\fr{\sum\limits_{\overline{k}\in 
A_{n+1}}p_{\overline{k}}(\overline{\rho},y)\sum\limits_{m=0}^n  m 
\sum\limits_{\overline{k}\in A_m} p_{\overline{k}}(\overline{\rho},y) }
{g_{n+1}(\overline{\rho},y) g_n(\overline{\rho},y)}
={}\\
{}=\left [ 1-q_{n+1}(\overline{\rho},y)\right ] \left [n+1-d_n(\overline{\rho},y)\right ]\,.
\label{e9aga}
\end{multline}


    Докажем утверждение~1 методом индукции. При $n = 0$, как следует 
из~(\ref{e9aga}), имеем
$$
d_2(\overline{\rho},y) - d_1 (\overline{\rho},y) =1-q_1(\overline{\rho},y)\,,
$$
    т.\,е.\ утверждение~1 при $n = 0$ справедливо. 

    Пусть неравенства~(\ref{e8aga}) справедливы для некоторого $n > 0$. 
Докажем, что они справедливы и для $n + 1$. Из~(\ref{e9aga}) получаем
\begin{multline*}
d_{n+1}(\overline{\rho},y)- d_n(\overline{\rho},y)={}\\
{}=\left [ 1-
q_{n+1}(\overline{\rho},y)\right ] \left [n+1-d_n(\overline{\rho},y)\right ] ={}\\
{}= \left [ 1-
1-q_{n+1}(\overline{\rho},y)\right ] \left [ n-{}\right.\\
{}-\left. d_{n-1}(\overline{\rho},y)+d_{n-1}(\overline{\rho},y)-
d_n(\overline{\rho},y)+1\right ] ={}\\
{}=\left [ 1-q_{n+1}(\overline{\rho},y)\right ] 
\left [ n-d_{n-1}(\overline{\rho},y)-{}\right.\\
{}-\left. \left ( d_n(\overline{\rho},y)-d_{n-1}(\overline{\rho},y)\right )+1\right] = {}\\
{}=
\left [ 1-q_{n+1}(\overline{\rho},y)\right ]
\left [ 
\fr{d_n(\overline{\rho},y) -d_{n-1}(\overline{\rho},y)}{1-
q_n(\overline{\rho},y)}\right.-{}\\
{}-\left.
\left ( d_n(\overline{\rho},y)-d_{n-1}(\overline{\rho},y)\right )+1
\vphantom{\fr{d_n(\overline{\rho})}{(q_n)}}
\right ]={}\\
{}=
\left [ 1-q_{n+1}(\overline{\rho},y)\right ]
\left [ 
\vphantom{\fr{d_n(\overline{\rho})}{(q_n)}}
\left ( d_n(\overline{\rho},y\right)\right. -{}\\
 {}-\left.
d_{n-1}\left(\overline{\rho},y)\right )\fr{q_n(\overline{\rho},y)}{1-
q_n(\overline{\rho},y)}+1\right ]\,.
\end{multline*}
Так как по предположению $d_n (\overline{\rho},y) -d_{n-1}(\overline{\rho},y) 
>0$, то правая часть последнего равенства больше нуля; следовательно, 
$d_{n+1}(\overline{\rho},y)-d_n(\overline{\rho},y)>0$. 

    Продолжив преобразование правой части последнего равенства и 
учитывая предположение $d_n(\overline{\rho},y) -d_{n-1}(\overline{\rho},y)<1$, 
получим
\begin{multline*}
d_{n+1}((\overline{\rho},y) -d_n(\overline{\rho},y)<{}\\
{}< \left [ 1-
q_{n+1}(\overline{\rho},y)\right ]
\left ( \fr{q_n(\overline{\rho},y)}{1-q_n(\overline{\rho},y)}+1\right )={}\\
{}=
\fr{1-q_{n+1}(\overline{\rho},y)}{1-q_n(\overline{\rho},y)}<1\,,
\end{multline*}
так как $0< q_n(\overline{\rho},y)<q_{n+1}(\overline{\rho},y)<1$, $n>0$, $y\in 
(0,\,1]$.

Следовательно, утверждение~1 доказано.

\medskip

\noindent
\textbf{Утверждение 2.} $q_N(\overline{\rho},y)$~--- \textit{монотонно-воз\-рас\-та\-ющая 
функция по $y\in (0,\,1]$. При этом $0< q_N(\overline{\rho},y)\;\leq $\linebreak 
$\leq\;q_N(\overline{\rho},1) <1$, $y\in (0,\,1]$,  и $\underset{y\rightarrow 
0}{\mathrm{lim}}\,q_N(\overline{\rho},y) =0$}.

\medskip

\noindent
Д\,о\,к\,а\,з\,а\,т\,е\,л\,ь\,с\,т\,в\,о\,.\  Возрастание функции 
$q_N(\overline{\rho},y)$ следует непосредственно из~(\ref{e7aga}) и 
утверж\-де\-ния~1. Доказательство неравенств в условии утверждения очевидно 
следует из~(\ref{e5aga}) и вида функции $g_n (\overline{\rho},y)$, $n\geq 0$. 
Для предела имеем:
\begin{multline*}
\underset{y\rightarrow 0}{\mathrm{lim}}\,q_N(\overline{\rho},y) 
=\underset{y\rightarrow 0}{\mathrm{lim}}\,\fr{g_{N- 1}(\overline{\rho},y)}{g_N(\overline{\rho},y)} = {}\\
{}= \underset{y\rightarrow 0}{\mathrm{lim}}\,\left (
g_{N-1}(\overline{\rho},y)\Bigg / \left ( 
\vphantom{\prod\limits_{v\in\Omega_u^+}}
g_{N-1}(\overline{\rho},y)\right.\right.+{}\\
{}+\left.\left.\sum\limits_{\overline{k}\in A_N}\prod\limits_{v\in\Omega_u^+} 
\fr{z_v(0,\rho_v,k_v,k^\prime_v,k^{\prime\prime}_v)}{y^N}\right )\right ) = {}\\
{}= \underset{y\rightarrow 0}{\mathrm{lim}}\,\left (
y^N g_{N-1}(\overline{\rho},y)\Bigg / 
\left ( 
\vphantom{\prod\limits_{v\in\Omega_u^+}}
y^N g_{N-1}(\overline{\rho},y)+{}\right.\right.\\
{}+\left.\left.\sum\limits_{\overline{k}\in A_N}
\prod\limits_{v\in\Omega_u^+} z_v(0,\rho_v,k_v,k_v^\prime , k_v^{\prime\prime}) 
\right ) \right )=0\,.
\end{multline*}
    
\medskip

\noindent
\textbf{Утверждение 3.} \textit{Производная функции~$q_N (\overline{\rho},y)$ по 
$y\in (0,\,1]$ удовлетворяет следующим соотношениям}:
\begin{align}
\underset{y\rightarrow 0}{\mathrm{lim}}\fr{\partial q_N(\overline{p},y)}
{\partial  y} &= \fr{\sum\limits_{\overline{k}\in A_{N-1}} 
p_{\overline{k}}(\overline{\rho},1)}{\sum\limits_{\overline{k}\in 
A_N}p_{\overline{k}}(\overline{\rho},1)}\,;\label{e10aga}\\
\fr{\partial q_N(\overline{\rho},y)}{\partial y}\Big |_{y=1}&<1\,.\label{e11aga}
\end{align}

\medskip

\noindent
Д\,о\,к\,а\,з\,а\,т\,е\,л\,ь\,с\,т\,в\,о\,.\ Проведя преобразования 
функции~$q_N(\overline{\rho},y)$, получим:
\begin{multline*}
\underset{y\rightarrow 0}{\mathrm{lim}}\fr{q_N(\overline{\rho},y)}{y} = {}\\
\!\!{}=
\underset{y\rightarrow 0}{\mathrm{lim}}
\fr{\sum\limits_{m=0}^{N-1}\sum\limits_{\overline{k}\in A_m}
\!\!\left (\prod\limits_{v\in\Omega_u^+}\!\! 
z_v(0,\rho_v,k_v,k_v^\prime , k_v^{\prime\prime})\right )\!\!\Bigg /\!\! y^m}
{y\sum\limits_{m=0}^{N}\sum\limits_{\overline{k}\in A_m}
\!\!\left(\prod\limits_{v\in\Omega_u^+}\!\! z_v\left (0,\rho_v,k_v,k_v^\prime , 
k_v^{\prime\prime}\right )\right )\!\!\Bigg /\!\!y^m} = \!\!\!
\end{multline*}
\begin{multline*}
\!\!\!\!\!\!{}=\underset{y\rightarrow 0}{\mathrm{lim}}\,
\fr{\sum\limits_{m=0}^{N-1}\sum\limits_{\overline{k}\in A_m}
y^{N-1-m}\prod\limits_{v\in\Omega_u^+} z_v(0,\rho_v,k_v,k_v^\prime , 
k_v^{\prime\prime})}{\sum\limits_{m=0}^{N}\sum\limits_{\overline{k}
\in A_m} y^{N-m}
\prod\limits_{v\in\Omega_u^+} z_v(0,\rho_v,k_v,k_v^\prime , 
k_v^{\prime\prime})}={}\!\\
{}=\fr{\sum\limits_{\overline{k}\in A_{N-1}} p_{\overline{k}}(\overline{\rho},1)}{ 
\sum\limits_{\overline{k}\in A_{N}} p_{\overline{k}}(\overline{\rho},1)}\,.
\end{multline*}
Очевидно, $\underset{y\rightarrow 0}{\mathrm{lim}} \,[d_N (\overline{\rho},y) -
d_{N-1} (\overline{\rho},y)]=1$, так как $\underset{y\rightarrow 
0}{\mathrm{lim}}\,d_n (\overline{\rho},y)=n$, $n>0$.

Следовательно, учитывая~(\ref{e7aga}), получаем~(\ref{e10aga}). 
Справедливость~(\ref{e11aga}) непосредственно следует из~(\ref{e7aga}) и 
утверждения~1.

\medskip

\noindent
\textbf{Утверждение 4.} \textit{Пусть $y^*\in (0,\,1]$~--- решение 
уравнения}~(\ref{e4aga}). \textit{Тогда}
\begin{equation*}
\fr{\partial q_N(\overline{\rho},y)}{\partial y}\Big |_{y=y^*}<1\,.
%\label{e12aga}
\end{equation*}

\medskip

\noindent
Д\,о\,к\,а\,з\,а\,т\,е\,л\,ь\,с\,т\,в\,о\,.\ \ Доказательство следует из~(\ref{e7aga}), 
так как $q_N(\overline{\rho},y^*)/y^* =1$.
\medskip

\noindent
\textbf{Утверждение 5.} \textit{Уравнение}~(\ref{e4aga}) \textit{имеет решение $y^*\in 
(0,\,1)$ тогда и только тогда, когда} 
\begin{equation}
\fr{\sum\limits_{\overline{k}\in A_{N-1}} p_{\overline{k}}(\overline{\rho},1)}{ 
\sum\limits_{\overline{k}\in A_{N}} p_{\overline{k}}(\overline{\rho},1)} >1\,.
\label{e13aga}
\end{equation}
\textit{Если уравнение}~(\ref{e4aga}) \textit{имеет решение $y^*\in (0,\,1)$, то оно 
единственное положительное решение}.
\medskip

\noindent
Д\,о\,к\,а\,з\,а\,т\,е\,л\,ь\,с\,т\,в\,о\,.\ Пусть выполняется 
неравенство~(\ref{e13aga}). Тогда, как следует из утверждения~3, 
$\underset{y\rightarrow 0}{\mathrm{lim}} (\partial q_N(\overline{\rho},y)/\partial y) 
>1$. Кроме того, как следует из утверждения~2, 
$\underset{y\rightarrow 0}{\mathrm{lim}} q_N(\overline{\rho},y)=0$. Тогда, так 
как $q_N(\overline{\rho},y)$~--- непрерывно-дифференцируемая функция по 
$y\in (0,\,1]$, существует значение $y^\prime \in (0,\,1)$ такое, что 
$q_N(\overline{\rho},y)>y$ для всех $y\in (0,\,y^\prime]$ (следует из теоремы о 
конечном приращении~\cite{11aga}). В то же время, согласно утверждению~2, 
$q_N(\overline{\rho},y)<y$ в окрестности точки $y=1$ (рис.~\ref{f1aga},\,\textit{а}). 
Следовательно, кривая $x=q_N(\overline{\rho},y)$ пересекает прямую $x=y$ 
хотя бы в одной точке $y=y^*\in (0,\,1)$, т.\,е.\ уравнение~(\ref{e4aga}) имеет 
хотя бы одно решение $y^*\in (0,\,1)$.

\begin{figure*}
\vspace*{1pt}
\begin{center}
\vspace*{1pt}
\mbox{%
\epsfxsize=158mm
\epsfbox{aga-1.eps}
}
\end{center}
\vspace*{-9pt}
\Caption{Примеры кривых $x=q_N(\overline{\rho},y)$ и $x=y$ (\textit{а})~при существовании решения 
уравнения~(\ref{e5aga}) и (\textit{б})~при выполнении условий~(17)
\label{f1aga}}
\vspace*{6pt}
\end{figure*}

Пусть уравнение~(\ref{e4aga}) имеет решение $y^*\in (0,\,1)$ и 
\begin{equation}
\fr{\sum\limits_{\overline{k}\in A_{N-1}}p_{\overline{k}}(\overline{\rho},1)}{ 
\sum\limits_{\overline{k}\in A_{N}}p_{\overline{k}}(\overline{\rho},1)}\leq 
1\,.\label{e14aga}
\end{equation}
Тогда из условий утверждений~2 и~3 следует, что 
уравнение~(\ref{e4aga}) в интервале $(0,\,1)$ имеет более одного решения, что 
может быть только при существовании решения $y^\prime \in (0,\,1)$ такого, 
что в окрестности точки $y=y^\prime$ выполняются неравенства: 
$q_N(\overline{\rho},y)<y$ при $y<y^\prime$ и $q_N(\overline{\rho},y)>y$ при 
$y>y^\prime$, где $y$ принадлежит указанной окрест\-ности точки~$y^\prime$ 
(рис.~\ref{f1aga},\,\textit{б}). Тогда в точке $y=y^\prime$ производная функции 
$q_N(\overline{\rho},y)$ по $y$ больше~1, что противоречит утверждению~4. 
Следовательно, неравенство~(\ref{e13aga}) справедливо.


Пусть уравнение~(\ref{e4aga}) имеет более одного положительного 
решения. Рассуждая точно так же, как и выше (в случае выполнения 
условий~(\ref{e14aga})), получим противоречие утверждению~4. 
Следовательно, утверждение~5 справедливо.
\medskip

\noindent
\textbf{Следствие.} \textit{Неравенства}
\begin{gather*}
\fr{\mu_v c_v (1-B_v)}{\Lambda_v}>1\,,\quad \fr{1-B_v}{\Lambda_v \tau_v B_v}>1\,,\\ 
\fr{1-B_v}{\Lambda_v t_v}>1\,,\ v\in\Omega_u^+\,,
\end{gather*}
\textit{являются необходимым условием существования решения 
уравнения}~(\ref{e4aga}).

\medskip
\noindent
Д\,о\,к\,а\,з\,а\,т\,е\,л\,ь\,с\,т\,в\,о\,.\ Пусть $\overline{k}_v$~--- это 
набор~$\overline{k}$, у которого $k_v=0$. Преобразовав левую 
часть~(\ref{e13aga}), получим

\noindent
\begin{multline*}
\fr{\sum\limits_{\overline{k}\in A_{N-1}} p_{\overline{k}} (\overline{\rho},1)}
{ \sum\limits_{\overline{k}\in A_{N}} 
 p_{\overline{k}}(\overline{\rho},1)} 
={}
\\
{}=
\fr{\sum\limits_{\overline{k}\in A_{N-1}}\prod\limits_{v\in \Omega_u^+} 
z_v\left(0,\rho_v,k_v,k_v^\prime , k_v^{\prime\prime}\right)}
{\sum\limits_{\overline{k}\in A_{N}}
\prod\limits_{v\in \Omega_u^+} z_v\left (0,\rho_v,k_v,k_v^\prime , k_v^{\prime\prime}\right )} \leq{}
\\
{}\leq
\left ( 
\vphantom{\prod\limits_{v^\prime\in\Omega_u^+\backslash v}}
\fr{\mu_v c_v(1-B_v)}{\Lambda_v}\right. \times{}\\
{}\times \sum\limits_{k_v=0}^{N-1}\sum\limits_{\overline{k}_v\in A_{N-1-k_v}} z_v\left(0,\rho_v,k_v+1,k_v^\prime , 
k_v^{\prime\prime}\right )\times{}\\
{}\times \left.\prod\limits_{v^\prime\in\Omega_u^+\backslash v} z_v^\prime 
\left(0,\rho_v,k_v,k_v^\prime , k_v^{\prime\prime}\right) \right)
\Bigg /{}\\
\Bigg / \left ( 
\vphantom{\prod\limits_{v^\prime\in\Omega_u^+\backslash v}}
\sum\limits_{k_v=0}^{N-1} \sum\limits_{\overline{k}_v\in A_{N-1-k_v}} z_v 
\left (0,\rho_v,k_v+1,k_v^\prime , 
k_v^{\prime\prime}\right )\right. \times{}\\
{}\times \prod\limits_{v^\prime\in\Omega_u^+\backslash v} 
z_{v^\prime}\left(0,\rho_v,k_v,k^\prime , k_v^{\prime\prime}\right)+{}\\
{}+
\sum\limits_{\overline{k}_v\in A_N} z_v\left (0,\rho_v, 0,k_v^\prime , 
k_v^{\prime\prime}\right)\times{}\\
\left.{}\times \prod\limits_{v^\prime\in\Omega_u^+\backslash v}z_{v^\prime} 
\left(0,\rho_v,k_v,k_v^\prime , k_v^{\prime\prime}\right )\right )\,.
\end{multline*}
Как следует из правой части последнего неравенства, если 
$\mu_v c_v (1-B_v)/\Lambda_v \leq 1$, то она меньше~1. Поэтому, чтобы 
выполнилось условие~(\ref{e13aga}), необходимо выполнение первого 
неравенства в условии следствия для каждого $v\in\Omega_u^+$. Точно так же 
доказывается необходимость выполнения второго и третьего неравенств в 
условии следствия.

    Пусть $y[n]$, $n\geq 0$, последовательность, полученная по формуле 
$y[n+1]=q_N(\overline{\rho},y[n])$, $y[0]=1$.

\medskip

\noindent
\textbf{Утверждение 6.} \textit{Пусть $y^*\in (0,\,1)$~--- решение 
уравнения}~(8). \textit{Тогда последовательность $y[n]$, $n\geq 0$, сходится 
к решению~$y^*$}.

\medskip

\noindent
Д\,о\,к\,а\,з\,а\,т\,е\,л\,ь\,с\,т\,в\,о\,.\ Отметим, что $y[1]<y[0]$ (это следует из 
утверждения~2, так как $y[0]=1$). Пусть для некоторого $n>1$ выполняется 
условие $y[n]<y[n-1]$. Тогда, как следует из утверждения~2, указанное условие 
выполняется и для $n+1$, т.\,е.\ по индукции следует, что последовательность 
$y[n]$, $n\geq 0$, монотонно убывает. 

    Пусть для некоторого $n>0$ $y[n]>y^*$ (существование такого $n$ 
следует из равенства $y[0]=1$). Тогда, как следует из утверждения~2, 
$y[n+1]\;=$\linebreak $=\;q_N(\overline{\rho},y[n])>q_N(\overline{\rho},y^*) =y^*$, т.\,е.\ 
последовательность ограничена снизу величиной~$y^*$. Значит, существует 
$\underset{n\rightarrow \infty}{\mathrm{lim}}\,y[n]=y^0\geq y^*$. Так как 
$q_n(\overline{\rho},y)$~--- непрерывная по~$y$ функция, то можно написать 
$\underset{n\rightarrow 
\infty}{\mathrm{lim}}\,q_N(\overline{\rho},y[n])=q_N(\overline{\rho},y^0)=y^0$, 
т.\,е.\ $y^0$~--- решение уравнения~(\ref{e4aga}). Из единственности 
положительного решения уравнения~(\ref{e4aga}) получаем $y^0=y^*$.

    Пусть в узле используется схема полного разделения буферной памяти. 
Тогда для интенсив\-ностей~$\Lambda_v^*$, $v\in\Omega_u^+$, справедливы 
соотношения:
$$
\Lambda_v^* = \fr{\Lambda_v}{1-\pi_v}\,,
$$
где $v\in\Omega_u^+$.


Фиксируем произвольную линию сети~$v$. Пусть $\overline{k}_v = (k_v, 
k_v^\prime, k_v^{\prime\prime})$~--- состояние буферной памяти линии~$v$; 
$k_v$, $k_v^\prime$, $k_v^{\prime\prime}$ определены выше. Тогда с 
учетом введенных ранее предположений и обозначений для вероятности 
блокировки линии справедлива формула~\cite{4aga}:
\begin{equation}
\pi_v = \fr{1}{G_{N_v}}\sum\limits_{k_v=N_v} 
z_v(\pi_v,\rho_v,\overline{k}_v)\,,
\label{e15aga}
\end{equation}
где 
\begin{multline*}
z_v(\pi_v, \rho_v, \overline{k}_v)={}\\
{}=
\begin{cases}
\fr{\rho_v^{\prime * k_v^\prime}}{k_v^\prime !}\,
 \fr{\rho_v^{\prime\prime * k_v^{\prime\prime}}}{k_v^{\prime\prime}!}\,
 \fr{\rho_v^{*k_v}}{k_v !} & \mbox{при}\ k_v<c_v\,,\\
 \fr{\rho_v^{\prime *k_v^\prime}}{k_v^{\prime }! }
 \fr{\rho_v^{\prime\prime * k_v^{\prime\prime}}}{k_v^{\prime\prime}!}
\fr{\rho_v^{*k_v}}{c_v !c_v^{k_v-c_v}} & \mbox{при}\ k_v\geq c_v\,;
\end{cases}
\end{multline*}
\begin{align*}
G_{N_v} &= \sum\limits_{m=0}^{N_v} z_v (\pi_v ,\rho_v , \overline{k}_v)\,;\\ 
\rho_v^*&=\fr{\rho_v}{1-\pi_v}\,;
\end{align*}
$\rho_v$, $\rho_v^{\prime *}$, 
$\rho_v^{\prime\prime *}$, $v\in\Omega_u^+$ определены выше.
    
Пусть $y_v=1-\pi_v$, а $q_{N_v} (\rho_v, y_v)$~--- выражение в правой 
части~(\ref{e15aga}). Тогда из равенств~(\ref{e15aga}), взяв~$y_v$ в качестве 
неизвестной переменной, получим систему независимых уравнений:
\begin{equation}
y_v = q_{N_v}(\rho_v, y_v)\,, \quad v\in \Omega_u^+\,.
\label{e16aga}
\end{equation}
    
    Заметим, что для фиксированной $v$ и заданных параметров $\Lambda_v$, 
$\mu_v$, $\tau_v$, $t_v$, $N_v$, $v\in\Omega_u^+$, уравнение в~(\ref{e16aga}) 
является частным случаем уравнения~(\ref{e4aga}) и совпадает с последним, 
когда число исходящих линий из узла равно~1. Следовательно, для схемы 
полного разделения памяти также справедливы все приведенные выше 
утверждения~1--6 и следствие. Заметим, что неравенство~(\ref{e13aga}) в 
условии утверждения~5 при $B_v=0$ и $t_v=0$ преобразуется в неравенство 
$\Lambda_v / (\mu_v c_v) >1$, $v\in\Omega_u^+$. Последовательность 
$\overline{y}[n]$, $n\geq 0$, в утверждении~6 определяется по формуле:
    \begin{gather*}
    \overline{y}[n] =\{y_v[n],\ v\in\Omega_u^+\}\,,\
    y_v[n+1]=q_{N_v} (\rho_v,\,y_v[n])\,,\\
    y_v[0] =1\,,\quad n\geq 0\,,\quad v\in \Omega_u^+\,.
    \end{gather*}


\section{Алгоритм расчета} %4

    Для вычисления интенсивностей потоков и вероятностей блокировок в 
узле предлагается следующий алгоритм, описывающий изложенную выше 
итерационную процедуру. Введем обозначения:
$y_u^v$~--- вероятность блокировки узла для заявок, направляемых на 
линию~$v$,
\begin{gather*}
y_u^v  = 
\begin{cases}
y_u & \mbox{для}\ v\in\Omega_u^+\ \mbox{при}\\
&\mbox{полнодоступной схеме},\\
y_v & \mbox{при схеме полного распределения}\\
&\mbox{памяти};
\end{cases}
\\
q_N^v(\overline{\rho}_u^{-v}, y_u^v)  = 
\begin{cases}
q_N(\overline{\rho},y) & \mbox{для}\ v\in\Omega_u^+\ \mbox{при пол-}\\ 
&\mbox{нодоступной схеме},\\
q_{N_v}(\rho_v, y_v) & \mbox{при схеме полного}\\
&\mbox{распределения}\\ 
&\mbox{памяти},  v\in\Omega_u^+\,.
\end{cases}
\end{gather*}



Тогда уравнения~(\ref{e4aga}) и~(\ref{e16aga}) записываются в виде:
$$
y_u^v = q_N^v (\overline{\rho}^v_u, y^v_u)\,,\quad v\in \Omega_u^+\,.
$$
Для значений, вычисляемых на $k$-м шаге алгоритма, к 
обозначениям соответствующих параметров приписывается знак~$[k]$.
\pagebreak

\textbf{Шаг 0.} 
\begin{enumerate}[1.]
\item  \textit{Инициализация}. Вычисление начальных значений 
параметров~$\rho_v$, $v\in\Omega_u^+$: $\Lambda_v[0]=\Lambda_v$, 
$\rho_v[0]=\Lambda_v[0]/(\mu_v(1-B_v))$, $y_u^v[0]=1$.
\item \textit{Проверка условий существования решения}. Если для некоторой 
линии $v\in\Omega_u^+$ выполняется хотя бы одно неравенство $(c_v\mu_v(1-
B_v))/\Lambda_v[0]\;\leq$\linebreak $\leq\;1$, или $(1-B_v)/(\Lambda_v\tau_v B_v) \leq 1$, или 
$(t_v(1\;-$\linebreak $-\;B_v))/\Lambda_v[0] \leq 1$, то алгоритм заканчивает работу с 
результатом <<нагрузка не реализуема>>. Если в узле используется 
полнодоступная схема и $(c_v\mu_v(1-B_v))/\Lambda_v[0] > 1$, $(1-
B_v)/(\Lambda_v\tau_v B_v)\;>$\linebreak $>\;1$, $(t_v(1-B_v))/\Lambda_v[0] > 1$ для всех 
$v\in\Omega_u^+$, то проверяется условие~(\ref{e13aga}) для $\Lambda_v =
\Lambda_v[0]$, $v\in\Omega_u^+$, и при невыполнении этого условия алгоритм 
заканчивает работу с результатом <<нагрузка не реализуема>>.
\end{enumerate}

    При вычислении левой части неравенства~(\ref{e13aga}) рекомендуется 
использовать метод свертки Базена (см.~\cite{12aga}), позволяющий 
производить рекуррентные вычисления (подробно этот метод описан также 
в~[1, 3--6]).



\medskip
\textbf{Шаг~$k$} ($k > 0$):
\begin{enumerate}[1.]
\item \textit{Вычисление вероятностей блокировок}. Используя значения 
параметров $\overline{\rho}_u^v[k-1]$, $y_u^v[k-1]$, $v\in\Omega_u^+$, 
вычисление с помощью формул~(1)--(7) значений 
вероятностей $y[k]=1- \pi [k]$~--- в случае полнодоступной памяти, или 
$y_v[k]=1- \pi_v[k]$, $v\in\Omega_u^+$, с помощью формул~(\ref{e15aga})~--- в 
случае полного разделения памяти. При вычислении этих значений 
рекомендуется использовать метод свертки Базена.
    \item \textit{Проверка условий останова алгоритма}. Если хотя бы для 
одной $v\in\Omega_u^+$ для заданного значения точности   выполняется 
условие
$$
\fr{\vert \Lambda_v^*[k]-\Lambda_v^*[k-1]\vert}{\Lambda_v^*[k]}> \varepsilon\,,
$$
то вычисление параметров $\overline{\rho}_u^v[k]$, $v\in\Omega_u^+$, и 
переход к шагу~$k$, положив $k$ равным $k+1$, иначе алгоритм завершает 
работу. 
\end{enumerate}

    По завершении алгоритма либо выявится, что нагрузка в системе не 
реализуема, либо будут вычислены интенсивности потоков, поступающих на 
линии узла, и стационарные вероятности блокировок для заявок каждого типа. 
    
\section{Примеры расчета}

    Для проверки точности вычисления результатов с помощью 
предложенного выше алгоритма и приемлемости введенных предположений 
были проведены вычислительные эксперименты с использованием 
аналитических и имитационных моделей. Во всех рассмотренных ниже 
примерах потоки внешних заявок считаются пуассоновскими. 
В~табл.~1 приведены значения вероятности блокировок вновь 
поступивших извне заявок, полученные на основании точной формулы, 
приведенной в~\cite{4aga} для СМО типа $M\vert M\vert 1\vert 0$ с повторными 
заявками при экспоненциальном распределении интервала времени между 
повторными попытками (первая строка таблицы), алгоритма из подраздела~5 
настоящей статьи (вторая строка) и имитационной модели при постоянном 
интервале времени между повторными попытками, равном~10 (третья строка). 
Расчет табл.~1 проведен для узла с одной исходящей одноканальной 
линией при интенсивности первичного потока $\Lambda =1$ и емкости 
накопителя $N_v=1$. Таблицы~2 и~3 вычислены с помощью 
алгоритма из подраздела~5 и имитационной модели соответственно при одной 
исходящей линии с числом каналов~10.


    В табл.~\ref{t4aga} и~\ref{t5aga} приведены значения вероятности 
блокировки узла с тремя исходящими линиями канальной емкости~10 каждая 
при $\mu_v =0{,}2$, $v\in\Omega_u^+$,  вычисленные с помощью алгоритма из 
подраздела~5 и имитационной модели с интервалом повторной попытки, 
равным~10, соответственно. В табл.~\ref{t4aga} и~\ref{t5aga} знак <<--->> в 
ячейках означает, что предложенная нагрузка $\Lambda_v$, $v\in\Omega_u^+$, 
не реализуема.



В табл.~\ref{t6aga} отражены вероятности блокировки такого же узла с 
накопителем $N = 35$ при экспоненциальном распределении интервала 
времени между повторными попытками со средним значением~$\tau$. 


Результаты вычислительного эксперимента показывают, что с  увеличением 
длины интервала между повторными попытками  вероятность блокировки 
увеличивается и приближается к значению,\linebreak
вычисленному с помощью 
алгоритма из подраздела~5 (см.\ табл.~\ref{t4aga} и~\ref{t6aga}), т.\,е.\ при 
пуассоновском внешнем потоке заявок предположение, что суммарный 
входной поток заявок  является пуассоновским, вполне приемлемо для 
предварительного анализа характеристик узла (например, при  $\tau c_v\mu_v > 
10$). Как показывают табл.~1--3, вероятность блокировки 
узла существенно зависит от\linebreak 

\vspace*{6pt}
\noindent
%\begin{table*}\small %tabl1
{\small
{{\tablename~1}\ \ \small{Вероятности блокировок при одной исходящей одноканальной линии}}
%\label{t1aga}}
\vspace*{-3pt}

\begin{center}
{\tabcolsep=7.3pt
\begin{tabular}{|c|c|c|c|c|c|}
\hline
&\multicolumn{5}{c|}{$\mu$}\\
\cline{2-6}
\multicolumn{1}{|c|}{\raisebox{4pt}[0pt][0pt]{№}}&1{,}1&1{,}2&2&3&4\\
\hline
1&0,9091&0,8333&0,5000&0,3333&0,2500\\
2&0,9091&0,8333&0,5000&0,3333&0,2500\\
3&0,8867&0,8452&0,4944&0,3167&0,2396\\
\hline
\end{tabular}}
\end{center}
%\vspace*{-6pt}
%\end{table*}
}
%\bigskip
%\medskip
\addtocounter{table}{1}
\pagebreak

\end{multicols}

\renewcommand{\figurename}{\protect\bf Таблица}
%\renewcommand{\tablename}{\protect\bf Рис.}
\begin{figure*}
{\small
\begin{minipage}[t]{76mm}
%\begin{table*}\small %tabl2
\begin{center}
\Caption{Вероятности блокировок при одной исходящей многоканальной линии ($\varepsilon 
=0{,}0001$)
\label{t2aga}}
\vspace*{2ex}

\tabcolsep=6.5pt
\begin{tabular}{|c|c|c|c|c|c|}
\hline
&\multicolumn{5}{c|}{$\mu$}\\
\cline{2-6}
\multicolumn{1}{|c|}{\raisebox{4pt}[0pt][0pt]{$N$}}&0{,}11&0{,}12&0{,}2&0{,}3&0{,}4\\
\hline
10&0,4845&0,2935&0,0204&0,0017&0,0002\\
15&0,1181&0,0545&0,0006&0,0000&0,0000\\
20&0,0489&0,0167&0,0000&0,0000&0,0000\\
\hline
\end{tabular}
\end{center}
%\end{table*}
\end{minipage}
\hfill
\begin{minipage}[t]{76mm}
%\begin{table*}\small %tabl3
\begin{center}
\Caption{Вероятности блокировок при одной исходящей линии
\label{t3aga}}
\vspace*{2ex}

\tabcolsep=6.5pt
\begin{tabular}{|c|c|c|c|c|c|}
\hline
&\multicolumn{5}{c|}{$\mu_v$}\\
\cline{2-6}
\multicolumn{1}{|c|}{\raisebox{4pt}[0pt][0pt]{$N$}}&0{,}11&0{,}12&0{,}2&0{,}3&0{,}4\\
\hline
10&0,5247&0,3238&0,0219&0,0019&0,0001\\
15&0,1726&0,0912&0,0004&0,0001&0,0000\\
20&0,1180&0,0371&0,0000&0,0000&0,0000\\
\hline
\end{tabular}
\end{center}
%\end{table*}
\end{minipage}
}
\vspace*{6pt}
\end{figure*}

\renewcommand{\figurename}{\protect\bf Рис.}
\renewcommand{\tablename}{\protect\bf Таблица}
\addtocounter{table}{2}

\begin{table}\small %tabl4
\begin{center}
\parbox{400pt}{\Caption{Вероятности блокировок при трех исходящих линиях, вычисленные алгоритмом из 
подраздела~5 ($\varepsilon =0{,}0001$)
\label{t4aga}}
}

\vspace*{2ex}

\tabcolsep=8pt
\begin{tabular}{|c|c|c|c|c|c|c|c|c|c|}
\hline
&\multicolumn{9}{c|}{$\Lambda_v$}\\
\cline{2-10}
\multicolumn{1}{|c|}{\raisebox{4pt}[0pt][0pt]{$N$}}&1&1{,}1&1{,}2&1{,}3&1{,}4&1{,}5&1{,}6&1{,}7&1{,}8\\
\hline
20&0,0677&0,1423&0,2975&0,7653&---&---&---&---&---\\
25&0,0065&0,0173&0,0394&0,0827&0.1690&0.3827&---&---&---\\
30&0,0005&0,0019&0,0059&0,0155&0.0361&0.0790&0.1792&0,7259&---\\
35&0,0000&0,0002&0,0008&0,0030&0,0089&0,0234&0,0574&0,1505&---\\
40&0,0000&0,0000&0,0001&0,0005&0,0022&0,0075&0,0220&0,0617&0,2449\\
\hline
\end{tabular}
\end{center}
%\end{table}
\vspace*{6pt}
%\begin{table}\small %tabl5
\begin{center}
\parbox{400pt}{\Caption{Вероятности блокировок при трех исходящих линиях, вычисленные с помощью 
имитационной модели
\label{t5aga}}
}

\vspace*{2ex}

\tabcolsep=8pt
\begin{tabular}{|c|c|c|c|c|c|c|c|c|c|}
\hline
&\multicolumn{9}{c|}{$\Lambda_v$}\\
\cline{2-10}
\multicolumn{1}{|c|}{\raisebox{4pt}[0pt][0pt]{$N$}}&1&1{,}1&1{,}2&1{,}3&1{,}4&1{,}5&1{,}6&1{,}7&1{,}8\\
\hline
20&0,0786&0,1695&0,3549&0,7056&---&---&---&---&---\\
25&0,0069&0,0190&0,0441&0,0998&0,2266&0,4583&---&---&---\\
30&0,0007&0,0024&0,0075&0,0184&0,0462&0,1025&0,2380&0,6931&---\\
35&0,0000&0,0003&0,0007&0,0040&0,0129&0,0307&0,0890&0,2981&---\\
40&0,0000&0,0000&0,0000&0,0011&0,0041&0,0095&0,0346&0,0790&0,3179\\
\hline
\end{tabular}
\end{center}
%\end{table}
\vspace*{6pt}
%\begin{table}\small %tabl6
\begin{center}
\parbox{356pt}{\Caption{Зависимость вероятности блокировки при трех исходящих линиях, вы\-чис\-лен\-ные с 
помощью имитационной модели со случайным интервалом между повторными попытками
\label{t6aga}}
}

\vspace*{2ex}

\tabcolsep=8pt
\begin{tabular}{|c|c|c|c|c|c|c|c|c|}
\hline
&\multicolumn{8}{c|}{$\Lambda_v$}\\
\cline{2-9}
\multicolumn{1}{|c|}{\raisebox{6pt}[0pt][0pt]{$\tau$}}&1&1{,}1&1{,}2&1{,}3&1{,}4&1{,}5&1{,}6&1{,}7\\
\hline
\hphantom{9}1&0.0001&0,0001&0,0017&0,0063&0,0210&0,0733&0,1996&0,4222\\
\hphantom{9}5&0.0000&0,0002&0,0016&0,0036&0,0446&0,0159&0,1360&0,3273\\
10&0.0000&0,0002&0,0011&0,0036&0,0101&0,0430&0,0818&0,2774\\
20&0.0000&0,0003&0,0007&0,0029&0,0089&0,0257&0,0863&0,2045\\
     \hline
\end{tabular}
\end{center}
\end{table}


\begin{multicols}{2}


\noindent
числа каналов в линии при равной суммарной 
производительности. Кроме того, как видно из табл.~\ref{t5aga} и~\ref{t6aga}, 
вероятность блокировки в большей степени зависит от среднего значения 
длины интервала между повторными попытками передачи, чем от закона 
распределения длины интервала. Таким образом, предложенный в работе 
алгоритм позволяет вы\-чис\-лить с достаточной точностью вероятность 
блокировки узла, интенсивности повторных передач и предельную величину 
реализуемой нагрузки. Отметим, что полученные в данной статье результаты 
могут быть использованы для расчета нагрузок в телекоммуникационной сети с 
повторами заявок в предыдущем узле или из источника. 


{\small\frenchspacing
{%\baselineskip=10.8pt
\addcontentsline{toc}{section}{Литература}
\begin{thebibliography}{99}    
\bibitem{1aga}
\Au{Kamoun~F., Kleinrock~L.}
Analysis of shared finite storage in a computer networks node environment under 
general traffic conditions~// IEEE Trans. on Commun., 1980. Vol.~28. No.\,7. 
P.~992--1003.

\bibitem{6aga} %2
\Au{Агаларов~Я.\,М., Шоргин~С.\,Я.}
Рекуррентный метод вычисления параметров сетей связи~// Техника средств 
связи, 1986. Сер. <<Системы связи>>. Вып.~6. С.~42--46.

\bibitem{3aga}
\Au{Башарин Г.\,П., Бочаров~П.\,П., Коган~Я.\,А.}
Анализ очередей в вычислительных сетях.~--- М.: Наука, 1989. 

\bibitem{4aga}
\Au{Бочаров~П.\,П., Печинкин~А.\,В.}
Теория массового обслуживания.~--- М.: Изд-во РУДН, 1995. 

\bibitem{5aga}
\Au{Вишневский~В.\,М.} 
Теоретические основы проектирования компьютерных сетей.~--- М.: 
Техносфера, 2003. 

\bibitem{2aga} %6
\Au{Башарин Г.\,П.}
Лекции по математической теории телетрафика.~--- М.: Изд-во РУДН, 2007. 

\bibitem{7aga}
\Au{Таранцев~А.\,А.}
Инженерные методы теории массового обслуживания.~--- М.: Наука, 2007.

\bibitem{9aga} %8
\Au{D'Apice~C., De~Simone~T., Manzo~R., Rizelian~G.}
$M\vert G\vert 1\vert r$ retrial queueing system with priority service of primary 
customers and a customers-searching server~// Distributed Computer and 
Communication Networks. Stochastic Modelling and Optimization.~--- М.: 
Техносфера, 2003. P.~106--117.

\bibitem{8aga} %9
\Au{Klimenok~V.\,I., Kim~C.\,S.}
$BM\!AP$/$PH$/1 retrial system operating in random environment~// Proceedings of 
the 5th St.-Petersburg Workshop on Simulation, St.-Petersburg, June~26\,--\,July~2, 
2005.~--- St.-Petersburg: NII Chemistry St.-Petersburg University Publs., 
2005. P.~367--372.   

\bibitem{10aga}
\Au{Krishnamoorthy~A., Babu~S.}
$M\!AP\vert (PH,PH)/c$ retrial queue with selegeneration of priorities 
and non-preemptive service~// Proceedings of the 14th International Conference on 
Analytical and Stochastic Modeling Techniques and Applications, June~4--6, 
2007. Prague, Czech Republic.~--- Sbr.-Dudweiler: Digitaldruck Pirrot GmbH, 
2007. P.~70--74.

\bibitem{11aga}
\Au{Корн~Г., Корн~Т.}
Справочник по математике.~--- М.: Наука, 1974.

\label{end\stat}


\bibitem{12aga}
\Au{Buzen~J.\,P.}
Computational algorithm for closed queuing networks with exponential servers~// 
Communications ACM, 1973. Vol.~16. No.\,9. P.~527--531.
 \end{thebibliography}
}
}
\end{multicols}
 
 
  %1
\def\stat{bosov+stef}

\def\tit{УПРАВЛЕНИЕ ВЫХОДОМ СТОХАСТИЧЕСКОЙ ДИФФЕРЕНЦИАЛЬНОЙ СИСТЕМЫ 
ПО~КВАДРАТИЧНОМУ КРИТЕРИЮ. I.~ОПТИМАЛЬНОЕ РЕШЕНИЕ МЕТОДОМ 
ДИНАМИЧЕСКОГО ПРОГРАММИРОВАНИЯ$^*$}

\def\titkol{Управление выходом стохастической дифференциальной системы 
по~квадратичному критерию. I}
%.~Оптимальное решение методом 
%динамического программирования}

\def\aut{А.\,В.~Босов$^1$, А.\,И.~Стефанович$^2$}

\def\autkol{А.\,В.~Босов, А.\,И.~Стефанович}

\titel{\tit}{\aut}{\autkol}{\titkol}

\index{Босов А.\,В.}
\index{Стефанович А.\,И.}
\index{Bosov A.\,V.}
\index{Stefanovich A.\,I.}




{\renewcommand{\thefootnote}{\fnsymbol{footnote}} \footnotetext[1]
{Работа выполнена при частичной поддержке РФФИ (проект 16-07-00677).}}


\renewcommand{\thefootnote}{\arabic{footnote}}
\footnotetext[1]{Институт проблем информатики Федерального исследовательского центра <<Информатика 
и~управление>> Российской академии наук, \mbox{AVBosov@ipiran.ru}}
\footnotetext[2]{Институт проблем информатики Федерального исследовательского центра <<Информатика 
и~управление>> Российской академии наук, \mbox{AStefanovich@frccsc.ru}}

%\vspace*{8pt}



  
  \Abst{Решается задача оптимального управления для диффузионного процесса 
Ито и~линейного управ\-ля\-емо\-го выхода. Рассматриваемая постановка близка 
к~классической ли\-ней\-но-квад\-ра\-тич\-ной гауссовской задаче управления 
(linear-quadratic Gaussian (LQG) control). Отличия состоят в~том, что состояние описывается нелинейным 
дифференциальным уравнение Ито $dy_t\hm= A_t(y_t) \,dt\hm+ \Sigma_t(y_t)\,dv_t$ 
и~не зависит от управ\-ле\-ния~$u_t$, оптимизации подлежит управ\-ля\-емый 
линейный выход $dz_t\hm= a_t y_t\,dt\hm+ b_t z_t \,dt\hm+ c_t u_t \,dt\hm+ \sigma_t\, 
dw_t$. Дополнительные обобщения внесены в~квад\-ра\-тич\-ный критерий качества 
с~целью воз\-мож\-ности постановки таких задач, как отслеживание выходом 
состояния или управ\-ле\-ни\-ем~--- линейной комбинации состояния и~выхода. Для 
решения используется метод динамического программирования. Функцию 
Беллмана позволяет найти предположение о~ее структуре вида $V_t(y,z)\hm= 
\alpha_t z^2\hm+ \beta_t(y)z \hm+\gamma_t(y)$. Решение дают три 
дифференциальных уравнения для коэффициентов~$\alpha_t$, $\beta_t(y)$ 
и~$\gamma_t(y)$. Эти уравнения со\-став\-ля\-ют оптимальное решение 
рас\-смат\-ри\-ва\-емой задачи.}
  
  \KW{стохастическое дифференциальное уравнение; оптимальное управ\-ле\-ние; 
динамическое программирование; функция Беллмана; уравнение Риккати; 
линейные уравнения параболического типа}

\DOI{10.14357/19922264180314}
  
%\vspace*{4pt}


\vskip 10pt plus 9pt minus 6pt

\thispagestyle{headings}

\begin{multicols}{2}

\label{st\stat}

\section{Введение}

     Ключевые результаты в~области оптимизации стохастических 
динамических систем, со\-став\-ля\-ющие классическую теорию управления, 
получены более~40~лет назад (такова работа~[1] в~отношении задачи 
управ\-ле\-ния ли\-ней\-но-гаус\-сов\-ски\-ми стохастическими сис\-те\-ма\-ми по 
квад\-ра\-тич\-но\-му критерию). К~классической тео\-рии следует относить 
линейные модели стохастических сис\-тем и~квадратичный критерий качества. 
Это исходный базис, на котором основано множество успешно 
исследованных и~решенных задач стохастического управ\-ле\-ния 
и~оптимизации. 

Дальнейшее развитие~--- это новые модели и~критерии, но 
прежде всего это новые методы: от тео\-рии линейных регуляторов, метода 
динамического программирования и~принципа максимума к~адаптивному 
и~минимаксному подходу, импульсному управ\-ле\-нию и~т.\,д. Множество 
инноваций как в~час\-ти моделей, так и~в~час\-ти математического аппарата, 
имевших мес\-то в~по\-сле\-ду\-ющие годы, существенно обогатили тео\-рию 
управ\-ле\-ния. Но и~до настоящего времени линейные модели и~квадратичный 
критерий, несмотря на всю справедливую критику в~отношении их 
аде\-кват\-ности и~гиб\-кости, сохраняют исследовательский интерес и~находят 
современные области приложения.
     
     Не претендуя на сколь\-ко-ни\-будь полное обосно\-ва\-ние последнего 
тезиса, приведем несколько примеров, показавшихся наиболее ин\-те\-рес\-ными. 

Так, в~[2] решается ли\-ней\-но-квад\-ра\-тич\-ная за\-да\-ча в~игровой 
постановке с~запаздыванием. В~близ\-кой по модели работе~[3] задача 
управ\-ле\-ния ставится в~терминах $H_\infty$-ро\-баст\-ности. Точнее \mbox{называть} 
эту тематику $H_2/H_\infty$-управ\-ле\-ни\-ем, и~работ по этой теме очень 
много. Аккуратности ради следует уточнить, что под линейными 
понимаются модели с~мультипликативными по состоянию воз\-му\-ще\-ниями. 

Совсем другой класс моделей, особо популярных в~по\-след\-ние годы, 
составляют скачкообразные процессы. Например, линейные уравнения 
в~сочетании с~пуассоновскими скачками в~[4] используются в~моделях, 
описывающих различные показатели функционирования сетевых протоколов 
передачи данных транспортного уровня. Телекоммуникации представляют 
в~последние годы самый популярный прикладной материал для 
исследований, работ по этой проб\-ле\-ма\-ти\-ке множество, математические 
техники привлекаются самые разные и~самые современные, но и~линейным 
моделям место находится. Еще один любопытный пример исследования 
скачкообразного процесса и~оптимизации на основе квад\-ра\-тич\-но\-го критерия 
можно найти в~[5] применительно к~задаче инвестирования на финансовом 
рынке. Наконец, упомянем еще работу~[6], подводящую итог исследований 
в~отношении классической детерминированной  
ли\-ней\-но-квад\-ра\-тич\-ной задачи с~использованием техники матричных 
неравенств.
     
     В данной работе также эксплуатируются привлекательные свойства 
линейных моделей и~квад\-ра\-тич\-но\-го критерия, причем в~стохастической 
постановке. На\-прав\-ле\-ни\-ем для обобщения \mbox{выбрана} модель динамики 
сис\-те\-мы: основные усилия на\-прав\-ле\-ны на то, чтобы сделать ее нелинейной. 
Кроме того, пред\-став\-лен\-ная постановка может рас\-смат\-ри\-вать\-ся и~как 
обобщение ранее решенной задачи в~дискретном времени~[7, 8] на время 
непрерывное. В~упомянутых работах помимо собственно модельной 
постановки важна еще и~привлекаемая прикладная об\-ласть~--- 
функционирование сложных программных сис\-тем. Результатов, 
ориентированных непосредственно на такие приложения, к~настоящему 
времени пренебрежимо мало, поэтому~[7, 8]~--- это еще и~прикладное 
обоснование рас\-смат\-ри\-ва\-емой далее задачи.
     
     Оптимизируемая динамическая сис\-те\-ма описывается двумя 
уравнениями. Состояние задается нелинейным стохастическим 
дифференциальным уравнением Ито, не содержащим управ\-ля\-емой 
переменной. Возмущение здесь описывается стандартным винеровским 
процессом, накладываются простые условия существования 
и~един\-ст\-вен\-ности решения. Поскольку состояние не управ\-ля\-ет\-ся, то уместно 
его интерпретировать как слож\-ное внешнее возмущение. Вторая 
переменная~--- управ\-ля\-емый выход~--- задается линейным стохастическим 
дифференциальным уравнением. Цель оптимизации выхода формируется 
квадратичным критерием общего вида. Формальная постановка задачи 
приведена в~сле\-ду\-ющем разделе.
     
     Для решения задачи используется метод динамического 
программирования, решается уравнение Беллмана~[9]. Соответственно, 
в~результате получаются аналитические выражения и~для оптимального 
управ\-ле\-ния, и~для значения функционала качества. Технически 
традиционный, стандартный подход к~задаче обременен, пожалуй, 
единственной проблемой~--- поиском верного пред\-став\-ле\-ния структуры 
функции Беллмана. Справиться с~этой проблемой в~большей степени удается 
за счет результата, полученного при решении дискретного по времени 
аналога рассматриваемой постановки~\cite{8-bos}. Конечные соотношения 
для оптимального решения, как и~во всех подобных задачах, включая 
классическую ли\-ней\-но-квад\-ра\-тич\-ную, содержат решения 
определенных дифференциальных уравнений (обыкновенных и~в~частных 
производных). Вывод этих уравнений и~со\-став\-ля\-ет содержание первой час\-ти 
данной работы. Во второй части будет обсуждаться их приближенное 
чис\-лен\-ное решение и~компьютерные эксперименты.
     
     Кратко обозначим основные положения, при\-вле\-ка\-емые далее 
к~решению задачи, следуя в~основном обозначениям 
и~терминологии~\cite{9-bos}, а~именно: будем рассматривать задачу 
оптимального управления в~стохастической динамической сис\-те\-ме по полной 
информации, применяя метод динамического программирования. В~качестве 
целевого функционала, опре\-де\-ля\-юще\-го качество управ\-ле\-ния $U_0^T\hm= \{ 
u_t,\ 0\leq t\leq T\}$, выступает
     \begin{equation}
     J\left(U_0^T\right)={\sf E}\left\{ \int\limits_0^T L_t \left(x_t, u_t\right)\,dt+ 
l\left(x_T\right)\right\}\,.
     \label{e1-bos}
     \end{equation}
Здесь ${\sf E}\{\cdot\}$~--- оператор математического ожидания; $x_t$~--- 
случайный процесс, описываемый стохастическим дифференциальным 
уравнением Ито
     \begin{equation}
     dx_t=m_t\left( x_t, u_t\right) dt+ \sigma_t\left( x_t\right)dW_t\,,\enskip 
x_0=X\,,
     \label{e2-bos}
     \end{equation}
где $W_t$~--- стандартный винеровский процесс подходящей раз\-мер\-ности; 
$X$~--- случайный вектор.

     $U_0^T$ будем выбирать из класса допустимых неупреждающих (по 
отношению к~$W_t$) управлений~\cite{9-bos}. Соответственно, 
относительно функций сноса и~диффузии~$m_t$ и~$\sigma_t$  
в~(\ref{e2-bos}) будем предполагать выполненными ка\-кие-ли\-бо условия 
существования сильного решения для заданного до\-пус\-ти\-мо\-го управ\-ле\-ния. 
Например, для управ\-ле\-ния с~обратной связью $u_t\hm= u_t(x_t)$ будем 
считать, что $m_t(x,u_t(x))$ и~$\sigma_t(x)$ удовлетворяют условию 
линейного рос\-та и~локальному условию Липшица по~$x$ равномерно 
по~$t$ (т.\,е.\ условиям Ито).
     
     Для поиска оптимального управления, минимизирующего $J(U_0^T)$, 
рас\-смат\-ри\-ва\-ет\-ся функция Беллмана
     \begin{equation}
     V_t(x)=\left.\mathop{\mathrm{inf}}\limits_{U_t^T} {\sf E} \left\{ \int\limits_t^T 
L_t \left( x_t, u_t\right)\,dt+l\left( x_T\right) \right\vert \mathcal{F}_t^x\right\}\,,
     \label{e3-bos}
     \end{equation}
где $\mathcal{F}_t^x$~--- $\sigma$-ал\-геб\-ра, по\-рож\-ден\-ная~$x_\tau$, 
$0\hm\leq \tau\hm\leq t$, ${\sf E}\{\cdot\vert \mathcal{F}\}$~--- оператор условного 
математического ожидания относительно~$\mathcal{F}$. Соответственно, 
в~качестве достаточного условия оп\-ти\-маль\-ности воспользуемся уравнением 
динамического программирования
\begin{multline}
\fr{\partial V_t(x)}{\partial t} +\fr{1}{2}\sum\limits^n_{i,j=1} \sigma^2_{t_{ij}}
\fr{\partial^2 V_t(x)}{\partial x_i \partial x_j}+{}\\
{}+\min\limits_u\left[  
\sum\limits^n_{i=1} m_{t_i} \fr{\partial V_t(x)}{\partial x_i} + L_t(x,u)\right] 
=0\,,\\
V_T(x)=l(x)\,,
\label{e4-bos}
\end{multline}
где $m_{t_i}$~--- $i$-й элемент век\-тор-функ\-ции~$m_t(x,u)$; 
$\sigma^2_{t_{ij}} \hm= \sum\nolimits^m_{k=1} 
\sigma_{t_{ik}}\sigma_{t_{ki}}$, $\sigma_{t_{ij}}$~--- $i$-й по строке, $j$-й 
по столб\-цу элемент мат\-рич\-ной функции~$\sigma_t(x)$; $n$ и~$m$~--- 
размерности~$x_t$ и~$W_t$ соответственно.

     Традиционно в~рамках применения метода динамического 
программирования будем предполагать, что функции~$L_t$, $l$, $m_t$ 
и~$\sigma_t$ обеспечивают существование хотя бы одного решения 
уравнения~(\ref{e4-bos}), а~следовательно, и~оптимального 
управления~$u_t^*$, $0\hm\leq t\hm\leq T$, до\-став\-ля\-юще\-го минимум 
целевому функционалу~(\ref{e1-bos}). Задача оптимизации далее получается 
путем указания конкретных выражений для~$L_t$, $l$, $m_t$ и~$\sigma_t$.

\section{Постановка задачи управления выходом}

     Рассматриваемые далее случайные функции будут предполагаться 
скалярными. Такое упрощение позволит разгрузить выкладки и~итоговые 
выражения от не самых существенных деталей.
     
     Рассмотрим стохастическую дифференциальную сис\-те\-му, со\-сто\-яние 
которой представляет диффузи\-он\-ный процесс~$y_t$, описываемый 
нелинейным стохастическим дифференциальным уравнением Ито
     \begin{equation}
     dy_t=A_t\left( y_t\right) dt +\Sigma_t \left( y_t\right) dv_t\,,\enskip 
y_0=Y\,,
     \label{e5-bos}
     \end{equation}
где $v_t$~--- стандартный (одномерный) винеровский процесс; $Y$~--- 
случайная величина с~конечным вторым моментом; функции~$A_t$ 
и~$\Sigma_t$ удовлетворяют условиям Ито:
\begin{equation*}
\left\vert A_t(y)\right\vert +\left\vert \Sigma_t(y)\right\vert \leq C(1+\vert y\vert )\ 
\mbox{для\ всех } 0\leq t\leq T\,;
\end{equation*}

\vspace*{-12pt}

\noindent
\begin{multline*}
\hspace*{-2.10051pt}\left\vert A_t\left(y_1\right) -A_t \left( y_2\right) \right\vert +\left\vert 
\Sigma_t\left( y_1\right) -\Sigma_t \left(y_2\right)\right\vert \leq
C\left\vert y_1-y_2\right\vert\\
 \mbox{для\ всех\ } 0\leq t\leq T\ \mbox{и } 
y_1,y_2\in \mathbb{R}^1\,,
\end{multline*}
обеспечивающим существование единственного сильного (потраекторного) 
решения уравнения.
     
     Будем считать, что~$y_t$ описывает состояние некоторой 
динамической системы. Соответственно, поведение этой сис\-те\-мы опишем 
выходом, линейно связанным с~со\-сто\-янием:
     \begin{equation}
     dz_t=a_t y_t \,dt+ b_t z_t \,dt+ c_t u_t \,dt+\sigma_t \,dw_t\,,\enskip
     z_0=Z\,.
     \label{e6-bos}
     \end{equation}
Здесь $w_t$~--- не зависящий от~$v_t$, $Y$ и~$Z$ стандартный (одномерный) 
винеровский процесс; $Z$~--- случайная величина с~конечным вторым 
моментом; $u_t$~--- допустимое неупреждающее управ\-ле\-ние, качество 
которого определяется целевым функционалом следующего вида:
\begin{multline}
\!\hspace*{-3.98538pt}J\left( U_0^T\right) ={\sf E}\left\{ \int\limits_0^T \!\left( S_t\left( s_ty_t-g_t z_t -h_t 
u_t\right)^2 +G_t z_t^2+{}\right.\right.\\
\left.\left.{}+ H_t u_t^2
\vphantom{S_t\left( s_ty_t-g_t z_t -h_t 
u_t\right)^2}
\right) dt+S_T\left( s_T y_T -g_T 
z_T\right)^2+G_T z_T^2
\vphantom{\int\limits_0^T}\right\}\,,
\label{e7-bos}
\end{multline}
где $S_t$, $G_t$ и~$H_t$~--- неотрицательные функции\linebreak
$0\hm\leq t\hm\leq T$. 
Такой критерий отражает физический смысл задачи распределения ресурсов 
со\-глас\-но аналогичной~(\ref{e5-bos})--(\ref{e7-bos}) задаче для дис\-крет\-но\-го 
времени, рас\-смот\-рен\-ной в~\cite{7-bos}. В~част\-ности,  
функци\-онал~(\ref{e7-bos}) поз\-во\-ля\-ет ставить задачи отслеживания
 выходом 
со\-сто\-яния сис\-те\-мы, используя сла\-га\-емое $(y_t\hm- z_t)^2$, или 
управлением~--- линейной комбинации со\-сто\-яния и~выхода, сла\-га\-емое типа\linebreak 
$(y_t\hm+ z_t\hm- u_t)^2$. Поскольку задача формулируется 
в~предположении наличия пол\-ной информации о~со\-сто\-янии~$y_t$ 
и~выходе~$z_t$ (соответствующую $\sigma$-ал\-геб\-ру 
обозначим~$\mathcal{F}_t^{y,z}$), то допустимое управ\-ле\-ние ищется 
в~классе~$\mathcal{F}_t^{y,z}$-из\-ме\-ри\-мых неупреждающих функций 
(и,~как будет показано далее, оказывается управ\-ле\-ни\-ем с~обратной связью).

     Функции~$a_t$, $b_t$, $c_t$ и~$\sigma_t$ будем предполагать 
ограниченными: $\vert a_t\vert \hm+ \vert b_t\vert \hm+\vert c_t\vert \hm+ \vert 
\sigma_t \vert \hm\leq C$ для всех $0\hm\leq t\hm\leq T$, процесс  
управления~--- допустимым не\-упреж\-да\-ющим~\cite{9-bos}, обеспечивая, 
таким образом, существование сильного решения урав\-не\-ния~(\ref{e6-bos}) 
для любого допустимого управ\-ления.
     
     Задачу составляет поиск~$u_t^*$~--- допустимого управ\-ле\-ния, 
доставляющего минимум квад\-ра\-тич\-но\-му функционалу~$J(U_0^T)$.
      
     Поставленная задача очевидным образом формулируется в~терминах 
введенных выше в~(\ref{e1-bos})--(\ref{e3-bos}) обозначений, а~именно: 
     требуется обозначить
     \begin{gather*}
      x_t=\begin{pmatrix}
     y_t\\ z_t\end{pmatrix};\quad  m_t(x_t, u_t)=\begin{pmatrix}
     A_t(y_t)\\ a_t y_t +b_t z_t +c_t u_t\end{pmatrix};\\
     \sigma_t(x_t)= \begin{pmatrix}
     \Sigma_t(y_t)& 0\\
     0& \sigma_t\end{pmatrix};\quad W_t=\begin{pmatrix}
     v_t \\ w_t\end{pmatrix}
     %     \label{e8-bos}
     \end{gather*}
для записи уравнения со\-сто\-яния типа~(\ref{e2-bos}) и
\begin{align*}
L_t(x,u)&= L_t(y,z,u) ={}\\
&\hspace*{3mm}{}=S_t\left( s_t y-g_t z -h_t u\right)^2 +G_t z^2 +H_t  u^2\,;\\
l(x)&= l(y,z) =S_T \left( S_T y-g_T z\right)^2 +G_T z^2
%\label{e9-bos}
\end{align*}
для записи целевого функционала в~виде~(\ref{e1-bos}).

     Функция Беллмана~(\ref{e3-bos}) принимает вид 
     $V_t(x)\hm= V_t(y,z)$. Для записи со\-от\-вет\-ст\-ву\-юще\-го~(\ref{e4-bos}) 
уравнения Беллмана для~$V_t(y,z)$ заметим, что
     $$
     \left( \sigma^2_{t_{ij}}\right)_{i,j=1,2}= \begin{pmatrix}
     \Sigma_t^2(y) & 0\\
     0 & \sigma_t^2\end{pmatrix}\,.
     $$
     
     С~учетом перечисленных обозначений урав\-не\-ние динамического 
программирования~(\ref{e4-bos}) принимает вид:
     \begin{multline}
     \fr{\partial V_t(y,z)}{\partial t} +\fr{1}{2}\left( \Sigma_t^2(y) \fr{\partial^2 
V_t(y,z)} {\partial y^2}+\sigma_t^2\fr{\partial^2 V_t(y,z)} {\partial 
z^2}\right)+{}\\
    {}+\min\limits_u\! \left[ A_t(y) \fr{\partial V_t(y,z)}{\partial y}+\left( a_t 
y+b_t z+c_t u\right) \fr{\partial V_t(y,z)}{\partial z} +{}\right.\hspace*{-3pt}\\
\left.{}+ S_t\left( s_t y-g_t z-h_t 
u\right)^2+G_t z^2+H_t u^2
     \vphantom{\fr{\partial V_t(y,z)}{\partial y}}\right] =0\,,\\
     V_T(y,z)=S_T\left( s_T y-g_T z\right)^2+G_T z^2\,.
     \label{e10-bos}
     \end{multline}
     Это и~есть то самое уравнение, которое требуется решить: 
существование решения данного урав\-не\-ния суть достаточное условие 
оптимальности; оптимальное управ\-ле\-ние при этом~--- точ\-ка минимума 
со\-от\-вет\-ст\-ву\-юще\-го сла\-га\-емого.
     
\section{Динамическое программирование и~оптимальное 
управление}

     В рассматриваемой постановке линейность\linebreak выхода и~квадратичность 
критерия дают те же преимущества, что и~в~классической  
ли\-ней\-но-квад\-ра\-тич\-ной задаче управ\-ле\-ния~\cite{1-bos}, а~именно: 
позволяют сразу определить вид оптимального управ\-ле\-ния и~фактические 
условия его существования. Действительно, со\-хра\-няя в~(\ref{e10-bos}) под 
знаком $\min\nolimits_u$ только члены, зависящие от~$u$, получаем
     \begin{multline*}
     \fr{\partial V_t(y,z)}{\partial t} +\fr{1}{2}\left( \Sigma_t^2(y) \fr{\partial^2 
V_t(y,z)} {\partial y^2}+\sigma_t^2\fr{\partial^2 V_t(y,z)} {\partial 
z^2}\right)+{}\\
     {}+A_t(y)\fr{\partial V_t(y,z)}{\partial y}+\left( a_t y+b_t z\right) 
\fr{\partial V_t(y,z)}{\partial z}+{}\\
{}+S_t\left( s_t y-g_t z\right)^2 +G_t z^2+{}
\end{multline*}

\noindent
\begin{multline*}
     {}+\min\limits_u \left[ \left( c_t \fr{\partial V_t(y,z)}{\partial z}-2S_t \left( 
s_t y-g_t z\right) h_t\right)u +{}\right.\\
\left.{}+\left( S_t h_t^2+H_t\right) u^2
\vphantom{\fr{\partial V_t(y,z)}{\partial z}}
\right]=0\,,
     %\label{e11-bos}
     \end{multline*}
откуда в~предположении $S_t h_t^2\hm+ H_t\hm>0$ следует, что существует 
оптимальное управ\-ле\-ние, которое определяется равенством
\begin{multline}
u_t^* = u_t^*(y,z)=-\fr{1}{2}\left( S_t h_t^2 +H_t\right)^{-1} \left( c_t 
\fr{\partial V_t(y,z)}{\partial z}-{}\right.\\
\left.{}-2S_t\left( s_t y-g_t z\right) h_t
\vphantom{\fr{\partial V_t(y,z)}{\partial z}}
\right)
\label{e12-bos}
\end{multline}
и доставляет минимум соответствующему сла\-га\-емо\-му в~урав\-не\-нии Беллмана, 
равный
$-\left( S_t h_t^2\hm+\right.$\linebreak
$\left.{}+H_t\right)^{-1} \left( c_t 
{\partial V_t(y,z)}/{\partial 
z}\hm-2S_t\left( s_t y \hm-g_t z\right) h_t \right)^2/4.
$ 
     
     Отметим, что, как и~в~классической ли\-ней\-но-квад\-ра\-тич\-ной 
задаче, управ\-ле\-ние из класса до\-пус\-ти\-мых не\-упреж\-да\-ющих получилось 
управ\-ле\-ни\-ем с~обратной связью.
     
     Таким образом, функция Беллмана описывается сле\-ду\-ющим 
дифференциальным уравнением:
     \begin{multline}
     \fr{\partial V_t(y,z)}{\partial t} +\fr{1}{2}\left( \Sigma_t^2(y) \fr{\partial^2 
V_t(y,z)} {\partial y^2}+\sigma_t^2\fr{\partial^2 V_t(y,z)} {\partial 
z^2}\right)+{}\\
     {}+ A_t(y) \fr{\partial V_t(y,z)}{\partial y}+\left( a_t y+b_t z\right) 
\fr{\partial V_t(y,z)}{\partial z}+{}\\
{}+ S_t \left( s_t y- g_t z\right)^2 +G_t z^2-
 \fr{1}{4}\left( S_t h_t^2+H_t\right)^{-1}\times{}\\
 {}\times \left( c_t \fr{\partial V_t(y,z)} 
{\partial z}-2S_t\left( s_t y -g_t z\right) h_t \right)^2=0\,.
     \label{e13-bos}
     \end{multline}
     
     Возводя в~квадрат по\-след\-нее сла\-га\-емое в~(\ref{e13-bos}), перепишем 
его в~виде:
     \begin{multline}
     \fr{\partial V_t(y,z)}{\partial t} +\fr{1}{2}\left( \Sigma_t^2(y) \fr{\partial^2 
V_t(y,z)} {\partial y^2}+\sigma_t^2\fr{\partial^2 V_t(y,z)} {\partial 
z^2}\!\right)+{}\\
{}+A_t(y) \fr{\partial V_t(y,z)}{\partial y}
+ \left( 
\vphantom{\left( S_t h_t^2 +H_t\right)^{-1}}
a_t y+b_t z+{}\right.\\
\left.{}+\left( S_t h_t^2 +H_t\right)^{-1}
 c_t S_t \left( s_t y-g_t z\right) h_t
\right) 
     \fr{\partial V_t(y,z)}{\partial z}+{}\\
     {}+\left( S_t-\left( S_t h_t^2 +H_t\right)^{-1} S_t^2 h_t^2\right)\left( s_t y -
g_t z\right)^2+{}\\
     \!\!{}+
     G_t z^2 -\fr{1}{4}\left( S_t h_t^2+H_t\right)^{-1}\! c_t^2
     \left(\fr{\partial V_t(y,z)}{\partial z}\right)^{\!2}=0\,.\!\!
     \label{e14-bos}
     \end{multline}
     
     Рассматривая полученное уравнение, заметим, что его решение может 
быть пред\-став\-ле\-но в~виде:
   \begin{equation}
     V_t(y,z)= \alpha_t z^2+\beta_t(y) z +\gamma_t(y)\,,
     \label{e15-bos}
     \end{equation}
т.\,е.\ будем искать решение при дополнительном предположении 
о~квад\-ра\-тич\-ности функции Белл\-ма\-на по переменной~$z$, и~сведем, таким 
образом, поиск оптимального решения к~уравнениям относительно функций 
$\alpha_t$, $\beta_t(y)$ и~$\gamma_t(y)$. Отметим сразу, что явный вид 
функции~$\gamma_t(y)$ для реализации оптимального управ\-ле\-ния не 
требуется, однако далее будет предложен вариант вы\-чис\-ле\-ния и~этой 
функции, что пред\-став\-ля\-ет\-ся небесполезным, поскольку позволит выполнять 
расчет минимума целевого функционала. Источником для 
предложения~(\ref{e15-bos}) является уже упоминавшаяся аналогичная 
задача для случая дис\-крет\-но\-го времени~\cite{7-bos, 8-bos}. В~той задаче 
выражение для функции Беллмана получается формально без 
дополнительных усилий. При этом форма~(\ref{e15-bos}) обнаруживается 
как свойство оптимального решения. В~рассматриваемом случае 
непрерывного времени~(\ref{e15-bos}) постулируется, а~пра\-виль\-ность 
постулата под\-тверж\-да\-ет\-ся далее ре\-зуль\-ти\-ру\-ющи\-ми уравнениями 
для~$\alpha_t$, $\beta_t(y)$ и~$\gamma_t(y)$ Кроме того, данное 
предположение пред\-став\-ля\-ет\-ся вы\-те\-ка\-ющим из линейной структуры задачи 
в~отношении переменной~$z$, в~част\-ности, тем фактом, что такой вид 
функции Беллмана обеспечивает линейность оптимального 
управ\-ле\-ния~(\ref{e12-bos}) по~$z$.

     Граничное условие при выбранном предположении~(\ref{e15-bos}) 
принимает вид:

\noindent
     \begin{multline*}
     V_T(y,z)= S_T\left( s_T y- g_T z\right)^2+G_T z^2 ={}\\[-0.5pt]
     {}=\alpha_T z^2 
+\beta_T(y) z +\gamma_T(y)\,,
    \end{multline*}
т.\,е.

\noindent
\begin{align*}
\alpha_T&= S_T g_T^2 +G_T\,;\\[-0.5pt]
\beta_T(y)&=-2S_T s_T g_T y\,;\\[-0.5pt]
\gamma_T(y)&=S_T s_T^2 y^2\,.
%\label{e16-bos}
\end{align*}
          При этом само оптимальное управ\-ле\-ние, определенное 
выражением~(\ref{e12-bos}), оказывается управ\-ле\-ни\-ем с~обратной связью 
по~$y_t$ и~$z_t$:

\noindent
     \begin{multline}
     u_t^*=u_t^*(y,z) ={}\\[-0.5pt]
     {}=
     -\fr{1}{2}\left( S_t h_t^2 +H_t\right)^{-1}
     \left( c_t \left( 2\alpha_t z +\beta_t(y)\right) +{}\right.\\[-0.5pt]
    \left. {}+2S_t\left( s_t y-g_t z\right) 
h_t\right)\,.
     \label{e17-bos}
     \end{multline}
          Подставляем $V_t(y,z)\hm= \alpha_t z^2 \hm+ \beta_t(y) 
z\hm+\gamma_t(y)$ в~(\ref{e14-bos}):

\noindent
     \begin{multline*}
     \fr{\partial \alpha_t}{\partial t}\, z^2 +
     \fr{\partial \beta_t(y)}{\partial t}\,z +
     \fr{\partial \gamma_t(y)}{\partial t}+{}\\[-0.5pt]
     {}+\fr{1}{2}\left( \Sigma_t^2(y) \left( 
\fr{\partial^2\beta_t(y)}{\partial y^2}\,z +\fr{\partial^2 \gamma_t(y)}{\partial 
y^2}\right) +2\sigma_t^2\alpha_t\right)+{}\\[-0.5pt]
 {}+A_t(y)\left(\fr{\partial \beta_t(y)}{\partial y}\,z + \fr{\partial 
\gamma_t(y)}{\partial y}\right) +{}\\[-0.5pt]
\hspace*{-0.22987pt}{}+\left( a_t y+b_t z+\left( S_t h_t^2 +H_t\right)^{-1} c_t S_t \left( s_t y-
g_t z\right) h_t\right)\times{}
\end{multline*}

\noindent
\begin{multline*}
         {}\times \left( 2\alpha_t z+\beta_t(y)\right)+{}\\
     {}+\left( S_t-\left( S_t h_t^2 +H_t\right)^{-1} S_t^2 h_t^2\right)\left( s_t y-
g_t z\right)^2+{}\\
     {}+ G_t z^2 -\fr{1}{4}\left( S_t h_t^2 +H_t\right)^{-1} c_t^2 \left( 
2\alpha_t z+\beta_t(y)\right)^2=0\,.
     \end{multline*}
          Далее выделяем слагаемые при~$z^2$, $z$ и~$z^0$
          
          \noindent
     \begin{multline*}
     \fr{\partial \alpha_t}{\partial t}\, z^2 +\fr{\partial \beta_t(y)}{\partial t}\,z +
     \fr{\partial \gamma_t(y)}{\partial 
t}+\fr{1}{2}\,\Sigma_t^2(y)\fr{\partial^2\beta_t(y)}{\partial y^2}\,z+ {}\\
{}+
\fr{1}{2}\,\Sigma_t^2(y)\fr{\partial^2\gamma_t(y)}{\partial 
y^2}+\sigma_t^2\alpha_t+A_t(y)\fr{\partial \beta_t(y)}{\partial y}\,z +{}\\
{}+A_t(y) \fr{\partial 
\gamma_t(y)}{\partial y}+{}\\
{}+ 2\alpha_t \left( b_t -\left( S_t h_t^2+H_t\right)^{-1} c_t 
S_t h_t g_t \right)z^2+{}\\
     {}+
     \left( 2\alpha_t\left( \alpha_t+\left( S_t h_t^2+H_t\right)^{-1} c_t S_t h_t 
s_t\right)y +{}\right.\\
\left.{}+\beta_t(y) \left( b_t-\left( S_t h_t^2+H_t\right)^{-1} c_t S_t h_t 
g_t\right) \right) z+{}\\
     {}+\beta_t(y)\left( a_t +\left( S_t h_t^2+H_t\right)^{-1} c_t S_t h_t s_t\right) 
y+{}\\
{}+ \left( S_t -\left( S_t h_t^2+H_t\right)^{-1} S_t^2 h_t^2\right) g_t^2 z^2-{}\\
     {}- 2\left( S_t -\left( S_t h_t^2+H_t\right)^{-1} S_t^2 h_t^2\right) s_t g_t yz 
+{}\\
{}+
     \left( S_t-\left( S_t h_t^2+H_t\right)^{-1} S_t^2 h_t^2\right) s_t^2 y^2+{}\\
     {}+G_t z^2 -\left( S_t h_t^2 +H_t\right)^{-1} c_t^2 \alpha_t^2 z^2 -{}\\
     {}-\left( 
S_t h_t^2+H_t\right)^{-1} c_t^2 \alpha_t \beta_t(y) z-{}\\
{}-
\fr{1}{4}\left( S_t h_t^2+H_t\right)^{-1}  c_t^2 \beta_t^2(y)=0\,,
     \end{multline*}
группируем их и~получаем сле\-ду\-ющие уравнения:
\begin{itemize}
\item  для~$\alpha_t$:

\noindent
\begin{multline}
\fr{\partial\alpha_t}{\partial t}+2\alpha_t\left( b_t-\left( S_t h_t^2+H_t\right)^{-1} c_t 
S_t h_t g_t\right)+{}\\
{}+ \left( S_t- \left( S_t h_t^2+H_t\right)^{-1} S_t^2 h_t^2\right) 
g_t^2+G_t-{}\\
\hspace*{-8mm}{}-\left( S_t h_t^2+H_t\right)^{-1} c_t^2 \alpha_t^2 =0\,,\enskip \alpha_T=S_T 
g_t^2+G_T\,;\!\!
\label{e18-bos}
\end{multline}
\item для $\beta_t$:

\noindent
\begin{multline}
\fr{\partial\beta_t(y)}{\partial 
t}+\fr{1}{2}\,\Sigma_t^2(y)\fr{\partial^2\beta_t(y)}{\partial y^2} 
+A_t(y)\fr{\partial \beta_t(y)}{\partial y}+{}\\
{}+ 2\alpha_t\left( a_t +\left( S_t h_t^2+H_t\right)^{-1} c_t S_t h_t s_t\right) y+{}\\
{}+
\beta_t(y)\left( b_t -\left( S_t h_t^2 +H_t\right)^{-1} c_t S_t h_t g_t\right)-{}\\
{}-2\left( S_t-\left( S_t h_t^2+H_t\right)^{-1} S_t^2 h_t^2\right) s_t g_t y-{}
\\
{}-
\left( S_t h_t^2+H_t\right)^{-1} c_t^2 \alpha_t \beta_t(y)=0\,,\\
\beta_T(y)=-2S_T s_T g_T y\,;
\label{e19-bos}
\end{multline}
\item  для $\gamma_t$:
\begin{multline}
\hspace*{-0.8pt}\fr{\partial \gamma_t(y)}{\partial t}+\fr{1}{2}\,\Sigma_t^2(y)
\fr{\partial^2 \gamma_t(y)}{\partial y^2} +\sigma_t^2 \alpha_t +A_t(y)
\fr{\partial \gamma_t(y)}{\partial y}+{}\\
{}+ \beta_t(y)\left( a_t +\left( S_t h_t^2+H_t\right)^{-1} c_t S_t h_t s_t\right) y+{}\\
{}+
\left( S_t-\left( S_t h_t^2+H_t\right)^{-1} S_t^2 h_t^2\right)  s_t^2 y^2-{}\\
{}-\fr{1}{4}\left( S_t h_t^2+H_t\right)^{-1} c_t^2 \beta_t^2(y) =0\,,\\
\gamma_T(y)=S_T s_T^2 y^2\,.
\label{e20-bos}
\end{multline}
\end{itemize}
     
     Уравнение~(\ref{e18-bos}), легко заметить, является уравнением 
Риккати, которое в~силу сформулированного выше условия   
имеет единственное неотрицательное решение для всех $0\hm\leq t\hm\leq T$. 
Этот факт требует дополнительного комментария. Для получения 
уравнения~(\ref{e18-bos}) рас\-смот\-рим исходную задачу при дополнительных 
условиях $a_t\hm=0$ и~$s_t\hm=0$ для всех $0\hm\leq t\hm\leq T$. Нетрудно 
видеть, что эти условия рассматриваемую по\-ста\-нов\-ку сводят фактически 
к~классической ли\-ней\-но-квад\-ра\-тич\-ной задаче. Имеющуюся 
в~рассматриваемой формулировке чуть более общую форму целевой 
функции (принципиального значения это обобщение, конечно, не имеет) 
сведем к~классической еще одним предположением: $S_t\hm=0$ для всех 
$0\hm\leq t\hm\leq T$. Теперь уравнение для~$\alpha_t$ принимает хорошо 
известный вид:
     \begin{equation}
     \fr{\partial \alpha_t}{\partial t}+2\alpha_t b_t +G_t- H_t^{-1} c_t^2 
\alpha_t^2=0\,,\enskip \alpha_T=G_T\,.
     \label{e21-bos}
     \end{equation}

     В таком случае, как известно~\cite{10-bos}, существует единственное 
оптимальное управление~--- линейное с~обратной связью по выходу~$z_t$, 
с~коэффициентом усиления, опи\-сы\-ва\-емым уравнением  
Риккати~(\ref{e21-bos}). Именно этот результат дают  
уравнения~(\ref{e18-bos})--(\ref{e20-bos}) и~описываемая ими функция 
Беллмана~(\ref{e15-bos}), так как из $a_t\hm=0$ и~$s_t\hm=0$ немедленно 
следует, что $\beta_t(y)\hm=0$, откуда, в~свою очередь, с~учетом 
не\-за\-ви\-си\-мости решения от~$y_t$ следует, что $\gamma_t(y)\hm=\gamma_t$, 
т.\,е.\ не зависит от~$y$ и~задается уравнением: 
     $$
     \fr{\partial \gamma_t(y)}{\partial t} +\sigma^2_t \alpha_t=0\,,\enskip 
\gamma_T=0\,.
     $$ 
     Оптимальное управ\-ле\-ние при этом 
     $$
     u_t^*= -H_t^{-1} c_t \alpha_t z_t\,,
     $$
      т.\,е.\ все полностью совпадает с~известным классическим решением.
     
     С уравнениями~(\ref{e19-bos}) и~(\ref{e20-bos}) ситуация, естественно, 
обстоит сложнее. Это линейные уравнения второго порядка параболического 
типа, поскольку\linebreak
 $\Sigma_t^2(y)\hm>0$. Фактически отсутствуют 
конструктивные условия, гарантирующие существование их\linebreak
 решений 
(требовать, чтобы все фигурирующие в~уравнениях коэффициенты были 
представлены аналитическими функциями на всем пространстве значений, 
вряд ли целесообразно), поэтому далее будем предполагать, что данные 
уравнения имеют на рас\-смат\-ри\-ва\-емом интервале $0\hm\leq t\hm\leq T$ хотя 
бы одно ограниченное решение и~именно эти условия будем рас\-смат\-ри\-вать 
как достаточные условия существования оптимального решения 
рассматриваемой задачи.
     
     Таким образом, доказана следующая тео\-рема.
     
     \smallskip
     
     \noindent
     \textbf{Теорема.}\ \textit{Пусть для диффузионного 
процесса}~(\ref{e5-bos}) \textit{выполнены условия Ито, для 
     процесса}~(\ref{e6-bos})~--- \textit{ограничены коэффициенты, 
уравнения}~(\ref{e18-bos})--(\ref{e20-bos}) \textit{имеют ограниченные 
решения для $0\hm\leq t\hm\leq T$. Тогда минимум  
функционалу}~(\ref{e7-bos}) \textit{доставляет оптимальное 
управ\-ле\-ние}~(\ref{e17-bos}), \textit{где} $y\hm= y_t$; $z\hm=z_t$.
     
\section{Заключение}

     Рассмотренная задача оптимизации в~целом близка и~по модели, и~по 
критерию к~классической ли\-ней\-но-квад\-ра\-тич\-ной постановке. 
Принципиальным отличием является нелинейная модель для описания 
со\-сто\-яния динамической сис\-те\-мы, в~которой отсутствует управ\-ля\-ющее 
воздействие.\linebreak
 Такую модель наряду с~традиционной интер\-пре\-тацией  
<<со\-сто\-яние--вы\-ход>> мож\-но понимать как\linebreak модель неконтролируемого 
слож\-но\-го внешнего воздействия. Небольшое дополнительное отличие дает 
предложенная форма квад\-ра\-тич\-но\-го критерия, поз\-во\-ля\-ющая, в~част\-ности, 
ставить такие задачи, как отслеживание выходом или управ\-ле\-ни\-ем со\-сто\-яния 
сис\-те\-мы или ее выхода.
     
     Поскольку обсуждать возможности точного решения уравнений, 
определяющих оптимальное управ\-ле\-ние, не имеет смыс\-ла, наиболее 
актуальной далее является задача их приближенного чис\-лен\-но\-го решения 
и~анализа воз\-мож\-ности практической реализации. Этому посвящена вторая 
часть данной работы, пла\-ни\-ру\-емая к~выходу в~ближайшее время.

{\small\frenchspacing
 {%\baselineskip=10.8pt
 \addcontentsline{toc}{section}{References}
 \begin{thebibliography}{99}
\bibitem{1-bos}
\Au{Athans M.} Editorial on the LQG problem~// IEEE~T. Automat. Contr., 1971. Vol.~16. 
No.\,6. P.~528--552. doi: 10.1109/TAC.1971.1099845.
\bibitem{2-bos}
\Au{Wu Z.} Forward-backward stochastic differential equations, linear quadratic stochastic 
optimal control and nonzero sum differential games~// J.~Syst. Sci. Complex., 2005. Vol.~18. 
No.\,2. P.~179--192.
\bibitem{3-bos}
\Au{Chen B.\,S., Zhang~W.} Stochastic H2/H1 control with state-dependent noise~// IEEE 
T.~Automat. Contr., 2004. Vol.~49. No.\,1. P.~45--56. doi: 10.1109/TAC.2003.821400.
\bibitem{4-bos}
\Au{Bohacek S.} A~stochastic model of TCP and fair video transmission~// IEEE 
INFOCOM, 2003. Vol.~2. P.~1134--1144. doi: 10.1109/INFCOM.2003.1208950.
\bibitem{5-bos}
\Au{Домбровский В.\,В., Объедко~Т.\,Ю.} Управление с~прогнозированием системами 
с~марковскими скачками при ограничениях и~применение к~оптимизации 
инвестиционного портфеля~// Автомат. телемех., 2011. №\,5. С.~96--112. doi: 
10.1134/S0005117911050079.
\bibitem{6-bos}
\Au{Баландин Д.\,В., Коган~М.\,М.} Оптимальное линейно-квад\-ра\-тич\-ное управление: от 
матричных уравнений к~линейным матричным неравенствам~// Автомат. телемех., 2011. 
№\,11. С.~60--69. doi: 10.1134/ S0005117911110038.
\bibitem{7-bos}
\Au{Босов А.\,В.} Обобщенная задача распределения ресурсов программной системы~// 
Информатика и~её применения, 2014. Т.~8. Вып.~2. С.~39--47. doi: 
10.14357/19922264140204.
\bibitem{8-bos}
\Au{Босов А.\,В.} Управление линейным выходом дискретной стохастической системы по 
квадратичному критерию~// Изв. РАН. Теория и~системы управления, 2016. №\,3.  
С.~19--35. doi: 10.1134/S1064230716030060.
\bibitem{9-bos}
\Au{Флеминг У., Ришел~Р.} Оптимальное управление детерминированными 
и~стохастическими системами~/ Пер. с~англ.~--- М.: Мир, 1978. 316~с. 
(\Au{Fleming~W.\,H., Rishel~R.\,W.} Deterministic and stochastic optimal control.~--- New 
York, NY, USA: Springer-Verlag, 1975. 222~p.)
\bibitem{10-bos}
\Au{Девис М.\,Х.\,А.} Линейное оценивание и~стохастическое управление~/ Пер. с~англ.~--- 
М.: Наука, 1984. 206~с. (\Au{Davis~M.\,H.\,A.} Linear estimation and stochastic control.~--- 
London: Chapman and Hall, 1977. 224~p.)

 \end{thebibliography}

 }
 }

\end{multicols}

\vspace*{-6pt}

\hfill{\small\textit{Поступила в~редакцию 30.03.18}}

\vspace*{4pt}

%\newpage

%\vspace*{-24pt}

\hrule

\vspace*{2pt}

\hrule

\vspace*{-2pt}


\def\tit{STOCHASTIC DIFFERENTIAL SYSTEM OUTPUT CONTROL 
BY~THE~QUADRATIC CRITERION.~I.~DYNAMIC\\ PROGRAMMING 
OPTIMAL SOLUTION}


\def\titkol{Stochastic differential system output control 
by~the~quadratic criterion. I.~Dynamic programming 
optimal solution}

\def\aut{A.\,V.~Bosov and~A.\,I.~Stefanovich}

\def\autkol{A.\,V.~Bosov and~A.\,I.~Stefanovich}

\titel{\tit}{\aut}{\autkol}{\titkol}

\vspace*{-11pt}


\noindent
Institute of Informatics Problems, Federal Research Center ``Computer Science 
and Control'' of the Russian Academy of Sciences, 44-2~Vavilov Str., Moscow 
119333, Russian Federation


\def\leftfootline{\small{\textbf{\thepage}
\hfill INFORMATIKA I EE PRIMENENIYA~--- INFORMATICS AND
APPLICATIONS\ \ \ 2018\ \ \ volume~12\ \ \ issue\ 3}
}%
 \def\rightfootline{\small{INFORMATIKA I EE PRIMENENIYA~---
INFORMATICS AND APPLICATIONS\ \ \ 2018\ \ \ volume~12\ \ \ issue\ 3
\hfill \textbf{\thepage}}}

\vspace*{3pt}



\Abste{The problem of optimal control for the Ito diffusion 
process and a~controlled linear output is solved. The considered 
statement is close to the classical linear-quadratic Gaussian 
control  (LQG control) problem. Differences consist in the fact 
that the state is described by the nonlinear differential Ito equation  $dy_y = A_t(y_t) 
\,dt+\Sigma_t(y_t)\,dv_t$ and does not depend on the control~$u_t$, 
optimization subject is controlled linear output 
 $dz_t=a_ty_t\,dt +b_tz_t\,dt +c_t u_t\,dt +\sigma_t \,dw_t$. 
Additional generalizations are included in the quadratic 
quality criterion for the purpose of statement such problems 
as state tracking by output or a linear combination of state 
and output tracking by control. The method of dynamic programming 
is used for the solution. 
The assumption about Bellman function in the form  $V_t(y,z)= \alpha_t 
z^2+\beta_t(y) z+\gamma_t(y)$ allows one to find it. 
Three differential equations for the coefficients $\alpha_t$,  $\beta_t(y)$,
and $\gamma_t(y)$ give the solution. 
These equations constitute the optimal solution of the problem under consideration.}

\KWE{stochastic differential equation; optimal control; dynamic programming; 
Bellman function; Riccati equation; linear differential equations of parabolic type}


\DOI{10.14357/19922264180314}

\vspace*{-12pt}

\Ack
\noindent
This work was partially supported by the Russian Science Foundation (grant  
16-07-00677).



%\vspace*{6pt}

  \begin{multicols}{2}

\renewcommand{\bibname}{\protect\rmfamily References}
%\renewcommand{\bibname}{\large\protect\rm References}

{\small\frenchspacing
 {%\baselineskip=10.8pt
 \addcontentsline{toc}{section}{References}
 \begin{thebibliography}{99}
\bibitem{1-bos-1}
\Aue{Athans, M.} 1971. Editorial on the LQG problem. \textit{IEEE~T. 
Automat. Contr.} 16(6):528--552. doi: 10.1109/ TAC.1971.1099845.
\bibitem{2-bos-1}
\Aue{Wu, Z.} 2005. Forward-backward stochastic differential equations, linear 
quadratic stochastic optimal control and\linebreak\vspace*{-12pt}

\columnbreak

\noindent
 nonzero sum differential games. 
\textit{J.~Syst. Sci. Complex.} 18(2):179--192.
\bibitem{3-bos-1}
\Aue{Chen, B.\,S. and W.~Zhang.} 2004. Stochastic H2/H1 control with  
state-dependent noise. \textit{IEEE~T. Automat. Contr.} 49(1):45--56.
doi: 10.1109/TAC.2003.821400.
\bibitem{4-bos-1}
\Aue{Bohacek, S.} 2003. A~stochastic model of TCP and fair video 
transmission. \textit{IEEE INFOCOM}. 2:1134--1144.
doi: 10.1109/INFCOM.2003.1208950.
\bibitem{5-bos-1}
\Aue{Dombrovskii, V.\,V., and T.\,Yu.~Ob''edko.} 2011. Predictive control of 
systems with Markovian jumps under constraints and its application to the 
investment portfolio optimization. \textit{Automat. Rem. Contr.}  
72(5):989--1003.
\bibitem{6-bos-1}
\Aue{Balandin, D.\,V., and M.\,M.~Kogan.} 2011. Optimal linear-quadratic 
control: From matrix equations to linear matrix inequalities. \textit{Automat. 
Rem. Contr.} 72(11):2276--2284.
\bibitem{7-bos-1}
\Aue{Bosov, A.\,V.} 2014. Obobshchennaya zadacha raspredeleniya resursov 
programmnoy sistemy [The generalized problem of software system resources 
distribution]. \textit{Informatika i~ee Primeneniya~--- Inform. Appl.}  
8(2):39--47. doi: 
10.14357/19922264140204.
\bibitem{8-bos-1}
\Aue{Bosov, A.\,V.} 2016. Discrete stochastic system linear output control 
with respect to a quadratic criterion. \textit{J.~Comput. Syst. Sc. 
Int.} 55(3):349--364.
\bibitem{9-bos-1}
\Aue{Fleming, W.\,H., and R.\,W.~Rishel.} 1975. \textit{Deterministic and 
stochastic optimal control.} New York, NY: Springer-Verlag. 222~p.
\bibitem{10-bos-1}
\Aue{Davis, M.\,H.\,A.} 1977. \textit{Linear estimation and stochastic 
control.} London: Chapman and Hall. 224~p.
\end{thebibliography}

 }
 }

\end{multicols}

\vspace*{-6pt}

\hfill{\small\textit{Received March 30, 2018}}

%\pagebreak

%\vspace*{-18pt}
     
     \Contr
     
       \noindent
       \textbf{Bosov Alexey V.} (b.\ 1969)~--- Doctor of Science in technology, 
principal scientist, Institute of Informatics Problems, Federal Research 
Center ``Computer Science and Control'' of the Russian Academy of Sciences, 
44-2~Vavilov Str., Moscow 119333, Russian Federation; 
\mbox{AVBosov@ipiran.ru}
       
       \vspace*{3pt}
       
       \noindent
       \textbf{Stefanovich Alexey I.} (b.\ 1983)~--- principal specialist, 
Institute of Informatics Problems, Federal Research Center ``Computer Science 
and Control'' of the Russian Academy of Sciences, 44-2~Vavilov Str., Moscow 
119333, Russian Federation; \mbox{AStefanovich@frccsc.ru}
\label{end\stat}

\renewcommand{\bibname}{\protect\rm Литература}       

       %2
\def\stat{shestakov+vor}

\def\tit{АСИМПТОТИЧЕСКАЯ НОРМАЛЬНОСТЬ И~СИЛЬНАЯ СОСТОЯТЕЛЬНОСТЬ ОЦЕНКИ РИСКА ПРИ~ИСПОЛЬЗОВАНИИ FDR-ПОРОГА В УСЛОВИЯХ СЛАБОЙ ЗАВИСИМОСТИ}

\def\titkol{Асимптотическая нормальность и~сильная состоятельность оценки риска при~использовании FDR-порога} % в~условиях слабой зависимости}

\def\aut{М.\,О.~Воронцов$^1$, О.\,В.~Шестаков$^2$}

\def\autkol{М.\,О.~Воронцов, О.\,В.~Шестаков}

\titel{\tit}{\aut}{\autkol}{\titkol}

\index{Воронцов М.\,О.}
\index{Шестаков О.\,В.}
\index{Vorontsov M.\,O.}
\index{Shestakov O.\,V.}


%{\renewcommand{\thefootnote}{\fnsymbol{footnote}} \footnotetext[1]
%{Работа 
%выполнена при поддержке Программы развития МГУ, проект №\,23-Ш03-03. При анализе 
%данных использовалась инфраструктура Центра коллективного пользования 
%<<Высокопроизводительные вычисления и~большие данные>> 
%(ЦКП <<Информатика>>) ФИЦ ИУ РАН (г.~Москва)}}


\renewcommand{\thefootnote}{\arabic{footnote}}
\footnotetext[1]{Московский государственный университет 
имени~М.\,В.~Ломоносова, факультет вычислительной математики и~кибернетики;  
Московский центр фундаментальной и~прикладной математики, \mbox{m.vtsov@mail.ru}}
\footnotetext[2]{Московский государственный университет 
имени М.\,В.~Ломоносова, факультет вычислительной математики и~кибернетики; 
Федеральный исследовательский центр <<Информатика и~управление>> Российской 
академии наук; Московский центр фундаментальной и~прикладной математики, 
\mbox{oshestakov@cs.msu.ru}}


\vspace*{-12pt}





\Abst{Рассматривается подход к~решению задачи удаления шума в~большом массиве 
разреженных данных, основанный на методе контроля средней доли ложных отклонений 
гипотез (False Discovery Rate, FDR). Данный подход эквивалентен процедурам 
пороговой обработки, обнуляющим компоненты массива, значения которых не 
превосходят некоторого заданного порога.  Наблюдения в~модели считаются слабо 
зависимыми. Для контроля степени зависимости используются ограничения на 
коэффициент сильного перемешивания и~максимальный коэффициент корреляции. 
В~качестве меры эффективности рассматриваемого подхода используется 
среднеквадратичный риск. Вычислить значение риска можно только на тестовых 
данных, поэтому в~работе рассматривается его статистическая оценка и~исследуются 
ее свойства. Показана асимптотическая нормальность и~сильная состоятельность 
оценки риска при использовании FDR-по\-ро\-га в~условиях слабой зависимости в~данных.}

\KW{пороговая обработка; множественная проверка гипотез; 
оценка риска}

\DOI{10.14357/19922264240309}{ZOQVTO}
  
%\vspace*{-6pt}


\vskip 10pt plus 9pt minus 6pt

\thispagestyle{headings}

\begin{multicols}{2}

\label{st\stat}



\section{Введение}

Во многих прикладных областях возникает задача обработки больших массивов 
зашумленных данных. Примерами служат задачи обработки изоб\-ра\-же\-ний с~высоким 
разрешением~\cite{FDRImage}, задачи множественной проверки гипотез, возникающие 
в~\mbox{исследованиях} в~об\-ласти генетики~\cite{MultipleTesting}, и~другие проб\-ле\-мы. 
В~связи с~этим рас\-смот\-рим модель
$$
x_i = \mu_i + z_i, \enskip i=\overline{1,n}\,,
$$
где $\mu_i\in\mathbb{R}$~--- <<полезные>> данные; $z_i \sim N(0,\sigma^2)$~--- 
шум. Задача заключается в~нахождении оценки неизвестного вектора $\mu \hm= 
(\mu_1,\ldots,\mu_n)$ как функции вектора $x \hm= (x_1,\ldots,x_n)$ и~может 
рассматриваться как задача множественной проверки гипотез о~равенстве нулю 
компонент вектора~$\mu$~\cite{AdaptingFDR}. При этом обычно предполагается, что 
вектор~$\mu$ имеет в~определенном смысле <<разреженную>> структуру, т.\,е.\ для 
<<полезных>> данных используется <<экономное>> представление.



В работе~\cite{AdaptingFDR} для решения рассматриваемой задачи в~условиях 
независимости компонент вектора~$x$ и~разреженности вектора~$\mu$ была 
предложена процедура построения оценки~$\hat{\mu}_F$ вектора~$\mu$, основанная 
на методе контроля средней доли ложных отклонений (FDR) 
гипотез при помощи алгоритма Бен\-жа\-ми\-ни--Хох\-бер\-га,
и~было проведено исследование асимптотики ее среднеквадратичного риска. 
В~работах~\cite{ZasShe17,Mathematics2020} была показана состоятельность 
и~асимптотическая нормальность оценки риска данной процедуры. Аналогичные 
результаты для других методов построения~$\hat{\mu}_F$ получены в~работах~\cite{Shestakov2021-1,Shestakov2021-2,Shestakov2022}.

В то же время в~определенных приложениях, например  при анализе полученных 
в~результате использования ДНК-мик\-ро\-чи\-пов данных~\cite{ResultsOnFDRUnderDependence}, исследовании геофизических процессов 
и~анализе помех\linebreak в~телекоммуникационных каналах, условие незави\-си\-мости компонент 
вектора $x$ может не выполняться. Ранее в~работах~\cite{VorontsovShestakov2023,Vorontsov2024} была \mbox{исследована} асимп\-то\-ти\-ка 
среднеквадратичного риска оценки~$\hat{\mu}_F$ \mbox{в~случае}, когда~$\mu$ принадлежит 
одному из классов разреженности
$$
l_0[\eta] = \left\{\mu\,:\, ||\mu||_0 \leq \eta n\right\}, \enskip \eta \in 
(0,1),
$$

\vspace*{-12pt}

\noindent
\begin{multline*}
m_p[\eta] \equiv{}\\
{}\equiv \left\{\mu \in \mathbb{R}^n : |\mu|_{(k)} \leq \eta n^{1/p} 
k^{-1/p},\ k=\overline{1,n}\right\}, \\
 p\in(0, 2),
\end{multline*}
а компоненты вектора~$x$ слабо зависимы~--- имеют достаточно быстро убывающий 
коэффициент сильного перемешивания~\cite{Bosq}

\noindent
\begin{multline*}
\alpha(k) = \sup\limits_{1\leq m\leq n}\alpha\left(\sigma(x_i, i\leq m), 
\sigma(x_i, i\geq m+k)\right), \\ 
k=\overline{1,n-1}\,,
\end{multline*}
где символом $\sigma(x_i, i\in I)$ обозначена сиг\-ма-ал\-геб\-ра, порожденная 
множеством случайных величин $\{x_i, i \hm\in I\}$, а~мера  $\alpha(\cdot, \cdot)$ 
близости двух сиг\-ма-ал\-гебр определяется как
$$
\alpha(\mathcal{B},\mathcal{C}) = \sup\limits_{B\in\mathcal{B}, 
C\in\mathcal{C}} \left|\p(BC)-\p(B)\p(C)\right|.
$$

В настоящей работе показана асимптотическая нормальность и~сильная 
состоятельность оценки риска при применении FDR-про\-це\-ду\-ры в~случае, когда 
компоненты вектора~$x$ слабо зависимы, а~$\mu$ принадлежит одному из классов 
раз\-ре\-жен\-ности: 
$l_0[\eta]$ или $m_p[\eta]$.


\section{Обработка вектора данных с~помощью FDR-процедуры}

Широким классом методов построения оценки~$\hat{\mu}$ стала пороговая обработка 
вектора~$x$ с~некоторым порогом~$T$. Различают жесткую пороговую обработку, при 
которой полагается
\begin{equation*}
\left(\hat{\mu}\right)_i  = p_H(x_i,T) \equiv
 \begin{cases}
   x_i, & |x_i| > T\,;\\
   0, & |x_i| \leq T\,,
 \end{cases}
\end{equation*}
и мягкую пороговую обработку, для которой
\begin{equation*}
(\hat{\mu})_i  = p_S(x_i,T) \equiv
 \begin{cases}
   x_i-T, & \hphantom{\vert\vert}x_i > T;\\
   x_i+T, & \hphantom{\vert\vert}x_i <- T;\\
   0, & |x_i| \leq T.
 \end{cases}
\end{equation*}
Среднеквадратичный риск подобных процедур определяется как
\begin{equation}
\label{riskDef}
R(T) = {\mathsf E} ||\hat{\mu}-\mu||^2 = \sum\limits_{i=1}^n {\mathsf E} \left((\hat{\mu})_i-
\mu_i\right)^2.
\end{equation}
Обозначим через~$T_m$ наилучшее значение порога:
$$
T_m : \, R(T_m) = \min\limits_{T} R(T).
$$

Предложенная в~\cite{AdaptingFDR} процедура заключается в~жесткой пороговой 
обработке компонент вектора~$x$ с~порогом $\hat{t}_F \hm= \hat{t}_F(x)$, и~ее 
результат~--- оценка $\hat{\mu}_F$ вектора~$\mu$ с~компонентами $(\hat{\mu}_F)_i  
\hm= p_H(x_i,\hat{t}_F)$, где
\begin{multline*}
\hat{t}_F = \sigma z\left(\fr{q \hat{k}_F}{2n}\right), \enskip
\hat{k}_F = \max 
\left\{k \, :\, |x|_{(k)} \geq t_k \right\}, \\
 t_k = \sigma z\left(\fr{q  k}{2n}\right);
\end{multline*}
$z(\alpha)$ --- квантиль уровня $(1\hm-\alpha)$ стандартного нормального 
распределения; $|x|_{(k)}$~--- $k$-й элемент вектора, получаемого в~результате 
упорядочения вектора~$|x|$ по невозрастанию:
$$
|x|_{(1)} \geq |x|_{(2)} \geq \cdots \geq |x|_{(n)};
$$
$q\in(0;1)$~--- управ\-ля\-ющий параметр FDR-ме\-то\-да.
Далее полагается, что $q\hm\equiv q_n$ зависит от~$n$. В~\cite{AdaptingFDR} 
показано, что эта процедура эквивалентна множественной проверке гипотез 
о~равенстве нулю компонент наблюдаемого вектора. Также показано, что с~помощью 
метода штрафных функций данную процедуру можно свести к~другим видам пороговой 
обработки, в~част\-ности к~мягкой пороговой обработке.

В работах~\cite{VorontsovShestakov2023, Vorontsov2024} была исследована 
асимптотика среднеквадратичного риска~$R(\hat{t}_F)$ описанной процедуры 
в~случае, когда компоненты вектора $x$ слабо зависимы, а $\mu$ принадлежит классу 
разреженности~$\Theta_n$, где~$\Theta_n$ есть~$l_0[\eta_n]$ или~$m_p[\eta_n]$. 
Было показано, что~$R(\hat{t}_F)$ асимптотически отличается от минимаксного 
риска
$\inf\nolimits_{\hat{\mu}\hm=\hat{\mu}(x)} \sup\nolimits_{\mu\in \Theta_n} {\mathsf E} 
||\hat{\mu}-\mu||^2$
на множитель не более чем логарифмического по\-рядка.

Отметим, что в~выражении для среднеквадратичного риска~(\ref{riskDef}) 
присутствуют неизвестные величины~$\mu_i$, а~потому вычислить~$R(T_m)$ и~$T_m$ 
не представляется возможным. На практике можно пользоваться, например, следующей 
оценкой среднеквадратичного риска~\cite{Mallat}:
$$
\hat{R}(T) = \sum\limits_{i=1}^n F[x_i, T],
$$
где  
\begin{multline*}
F[x_i, T] = {}\\[3pt]
{}=\!\begin{cases}
\left(x_i^2-\sigma^2\right) \Ik(|x_i|\leq T) + \sigma^2 \Ik\left(|x_i|>T\right) &\\[3pt]
&\hspace*{-53mm}\mbox{для\ жесткой\ пороговой\ обработки};\\[3pt]
\left(x_i^2-\sigma^2\right) \Ik\left(|x_i|\leq T\right) + (\sigma^2+T^2) 
\Ik \left(|x_i|>T\right) \hspace*{-11.21576pt}&\\[3pt]
&\hspace*{-51mm}\mbox{для\ мягкой\ пороговой\ обработки}.
\end{cases}\hspace*{-7.17859pt}
\end{multline*}


\noindent
\textbf{Замечание}.\ При пороговой обработке иногда также используется так 
называемый универсальный порог $T_U\hm = \sigma \sqrt{2\ln n}$, предложенный 
в~работе~\cite{spatialAdaptation}. Исследования в~\cite{AdaptingSURE, ExactRisk} 
показали, что порог~$T_U$ в~определенном смысле максимальный, и~рас\-смат\-ри\-вать 
пороги выше него не имеет смысла. Более того, нетрудно показать, что $t_k \hm< T_U$ 
для всех~$k$ и~всех достаточно больших~$n$, в~связи с~чем всюду далее полагаем, 
что порог~$\hat{t}_F$ выбирается на отрезке $[0; T_U]$.

\section{Вспомогательные утверждения}

Кроме коэффициента сильного перемешивания~$\alpha(\cdot)$ также понадобится 
следующее понятие~\cite{Bosq}.

\smallskip

\noindent
\textbf{Определение.} %\label{defRho}
Максимальным коэффициентом корреляции~$\rho(\cdot)$ компонент вектора~$x$ 
называется
\begin{multline*}
\rho (k) \equiv \rho_n (k) = {}\\
{}=\sup\limits_{1\leq m\leq n}\rho\left(\sigma(x_i, 
i\leq m), \sigma(x_i, i\geq m+k)\right), \\
 k=\overline{1,n-1}\,,
\end{multline*}
где мера $\rho(\cdot, \cdot)$ близости двух сиг\-ма-ал\-гебр определяется как
$$
\rho(\mathcal{B},\mathcal{C}) = \sup\limits_{\substack{\xi 
\in\mathcal{L}^2(\mathcal{B}) \\
 \eta \in\mathcal{L}^2(\mathcal{C})}} 
\left|\mathrm{corr}\,(\xi, \eta)\right|.
$$


Введем обозначения:
$$
T_1 = \sqrt{2\ln \eta_n^{-p}};  \,\gamma_n = \fr{1}{\ln\ln n}; \, \kappa_n 
= \fr{n \eta_n^p T_1^{-p}}{1 - q_n - \gamma_n}; 
$$
$$ 
\kappa_n^0 = \fr{[n \eta_n]}{1 - q_n - \gamma_n} ;\, \rho^\star (k) = 
\sup\limits_{n\geq k+1} \rho(k), k \in \mathbb{N} ;
$$
$$
t_{\kappa_n} = \sigma z\left(\fr{q_n \kappa_n }{2n}\right) , \,\, t_{\kappa_n^0} 
= \sigma z\left(\fr{q_n \kappa_n^0 }{2n}\right).
$$


Следующие два утверждения показывают, что случайный порог~$\hat{t}_F$ в~случае 
$\mu\hm\in m_p[\eta_n]$ (соответственно $\mu\hm\in l_0[\eta_n]$) с~большой 
вероятностью будет не меньше~$t_{\kappa_n}$ (соответственно~$ t_{\kappa_n^0}$). 
Их  доказательства приведены в~работах~\cite{VorontsovShestakov2023, Vorontsov2024}.

\smallskip

\noindent
%\begin{lem}\label{lem5}
\textbf{Лемма~1.}\ \textit{Пусть $n^{-\delta_1} \hm\leq \eta_n^p \hm\leq n^{-\delta_2}$, 
$0\hm<\delta_2\hm<\delta_1<1$, $\mathrm{lim\,inf} q_n \ln n \hm\geq C \hm> 0$, 
$m\hm\in[1;n/2]\cap\mathbb{N}$, а $\alpha(\cdot)$~--- коэффициент сильного 
перемешивания компонент вектора~$x$. Для некоторого $N\hm\in\mathbb{N}$ при $n \hm\geq 
N$ справедливо}
\begin{multline*}
\hspace*{-3pt}\sup\limits_{\mu\in m_p[\eta_n]} \p \left(\hat{k}_F \geq \kappa_n \right) \leq 
4 n \exp\left\{-\fr{m}{256n}  \kappa_n q_n \gamma_n^2    \right\}+{}\\
{}+ 22\left(1+\fr{8n}{\kappa_n q_n \gamma_n}\right)^{1/2} n m 
\alpha\left(\left[\fr{n}{2m}\right]\right).
\end{multline*}



\smallskip

\noindent
\textbf{Лемма 2.}\ 
%\label{lem1}
\textit{Пусть $\eta_n \hm\leq b\hm<1$, $m\in[1;n/2]\cap\mathbb{N}$, а~$\alpha(\cdot)$~--- 
коэффициент сильного перемешивания компонент вектора~$x$. Для некоторого 
$N\hm\in\mathbb{N}$ при $n \hm\geq N$ справедливо}
\begin{multline*}
\sup\limits_{\mu\in l_0[\eta_n]} \p \left(\hat{k}_F \geq \kappa_n^0 \right) 
\leq{}\\
{}\leq 4 n \exp\left\{-\fr{(1-b)m}{64n}\,  \kappa_n^0 q_n \gamma_n^2    
\right\}+{}\\
{}+ 22\left(1+\fr{4n}{(1-b)\kappa_n^0 q_n \gamma_n}\right)^{1/2} n m 
\alpha\left(\left[\fr{n}{2m}\right]\right).
\end{multline*}

Следующие два утверждения доказаны в~\cite{Bosq} и~представляют собой аналоги 
неравенств Хеффдинга и~Бернштейна для слабо зависимых случайных величин.


\smallskip

\noindent
\textbf{Лемма 3.}\
\textit{Пусть для набора действительных случайных величин $X_1, \ldots, X_n$ 
с~коэффициентом сильного перемешивания $\alpha(\cdot)$ выполняется ${\mathsf E} X_i \hm=0$, 
$|X_i|\hm\leq b$, $i\hm=\overline{1,n}$. Тогда для любого целого числа $m\hm\in[1; n/2]$ 
и~любого $\eps\hm>0$ справедливо}
\begin{multline*}
\p\left(\left|\sum\limits_{i=1}^n X_i\right| > n\eps \right) \leq 4 
\exp\left\{-\fr{\eps^2 m}{8 b^2}\right\}+ {}\\
{}+
22\left(1+\fr{4b}{\eps}\right)^{1/2} m\, 
\alpha\left(\left[\fr{n}{2m}\right]\right).
\end{multline*}


\smallskip

\noindent
\textbf{Лемма 4.}\
\textit{Пусть для набора действительных случайных величин $X_1, \ldots, X_k$ 
с~коэффициентом сильного перемешивания $\alpha(\cdot)$ выполняется ${\mathsf E} X_i \hm=0$, 
$|X_i|\hm\leq b$, $i\hm=\overline{1,k}$. Тогда для любого целого числа $m\hm\in[1; k/2]$ 
и~любого $\eps\hm>0$ справедливо}
\begin{multline*}
\p\left(\left|\sum\limits_{i=1}^k X_i\right| > \eps \right) \leq 4 
\exp\left\{-\fr{\eps^2 m}{8 v^2 k^2}\right\}+{}\\
{}+ 22\left(1+\fr{4bk}{\eps}\right)^{1/2} m\, 
\alpha\left(\left[\fr{k}{2m}\right]\right),
\end{multline*}
\textit{где $p = k/(2m)$}:
\begin{multline*}
v^2 =
 \fr{b \eps}{2k} + {}\\
 {}+\fr{2}{p^2} \,  \max\limits_{ j\in[0,\,2m-1]} 
{\mathsf E} \big( ([jp]+1-jp)X_{[jp]+1} + X_{[jp]+2}+{}\\
{}+ \cdots +  X_{[(j+1)p]} + ((j+1)p-[(j+1)p])X_{[(j+1)p+1]}\big)^2.
\end{multline*}

\noindent
\textbf{Замечание.}
Если существует такое число $S \hm> 0$, что сразу для всех $i\hm\in[1;k]$  выполняется 
${\mathsf E} X_i^2 \hm\leq S^2$, то в~качестве~$v^2$ можно взять
$$
v^2 = \fr{b \eps}{2k} + 8 S^2.
$$


Д\,о\,к\,а\,з\,а\,т\,е\,л\,ь\,с\,т\,в\,о\ \ сле\-ду\-юще\-го утверж\-де\-ния приведено в~работе~\cite{AdaptingFDR}.

\smallskip

\noindent
\textbf{Лемма 5.}\ 
\textit{Для $y\leq 0{,}01$ справедливы представления}
\begin{multline}
\label{lem1eq1}
z^2(y) = 2 \ln y^{-1} - \ln \ln y^{-1} - r_2(y), \\
 r_2(y) \in [1{,}8; 3];
\end{multline}

\noindent
\begin{equation}
\label{lem1eq2}
z(y) = \sqrt{2 \ln y^{-1}} - r_1(y), \, \, r_1(y) \in [0; 1{,}5].
\end{equation}


\section{Асимптотическая нормальность оценки риска при~применении FDR-процедуры в~условиях слабой зависимости}

Перейдем к~описанию достаточных условий для асимптотической нормальности оценки 
риска $\hat{R}(\hat{t}_F)$ в~случае $\mu \hm\in m_p[\eta_n]$.

\smallskip

\noindent
\textbf{Теорема~1.}\
\textit{Пусть $\mu \hm\in m_p[\eta_n],$ $\eta_n^p \hm\in[n^{-\delta_1}; n^{-\delta_2}],$ $1/2 \hm< 
\delta_2 \hm< \delta_1<1;$ имеются такие константы $c_1, c_2>0$, что для 
коэффициента сильного перемешивания $\alpha(\cdot)$ компонент вектора $x$ 
справедливо  $\alpha(k) \hm\leq c_1 k^{-1-(5/2)\delta_1/(1-\delta_1)-c_2},$ 
$k\hm=\overline{1,n-1};$ $q_n \hm< c_3 \hm< 1;$ $\mathrm{lim\,inf} q_n \ln n \hm= c_4 \hm> 0;$ и,~кроме того, 
для максимального коэффициента корреляции $\rho(\cdot)$ компонент вектора~$x$ 
справедливо}
$$
\sum\limits_{k = 1}^{\infty} \sup\limits_{n\geq k+1} \rho(k) \equiv 
\sum\limits_{k = 1}^{\infty}  \rho^\star (k) = c_5 < \infty. 
$$
\textit{Тогда при $n \to \infty$}
$$
\fr{\hat{R}(\hat{t}_F) - R(T_m)}{C_\rho \sqrt{2n}} \Rightarrow N(0, 1),
$$
\textit{где}
$$
C_\rho = \sigma^2\sqrt{1 +  \lim\limits_{n\to\infty} \fr{1}{n} \sum\limits_{j\neq i} \mathrm{corr}^2 (x_i, x_j)}.
$$

\noindent
Д\,о\,к\,а\,з\,а\,т\,е\,л\,ь\,с\,т\,в\,о\  \
 приводится для метода мягкой пороговой обработки; в~случае жесткой пороговой 
обработки доказательство аналогично. Обозначим
$$
U(T) = \hat{R}(T) -  \hat{R}(T_m) = \sum \limits_{i=1}^n H_i(T, T_m),
$$
где
$$
H_i(T, T_m) = F[x_i, T] - F[x_i, T_m].
$$
Имеем

\vspace*{-3pt}

\noindent
\begin{multline}
\label{D00}
\hat{R}(\hat{t}_F) - R(T_m) + \hat{R}(T_m) - \hat{R}(T_m) ={}\\
{}= \hat{R}(T_m) - 
R(T_m) + U(\hat{t}_F).
\end{multline}
Покажем, что
\begin{equation}
\label{D0}
\fr{\hat{R}(T_m) - R(T_m)}{C_\rho\sqrt{2n}} \Rightarrow N(0, 1).
\end{equation}


Повторяя рассуждения из~\cite{KuShe2016_1,KuShe2016_2,Jansen}, можно показать, 
что $T_m \hm\geq t_{\kappa_n}$. Учитывая также $T_m\hm \leq T_U$, имеем 
$$
C \sqrt{\ln n} \leq T_m \leq C^\prime \sqrt{\ln n}
$$ 
для некоторых положительных констант $C$ и~$C^\prime$.

\columnbreak

В случае мягкой пороговой обработки $\hat{R}(T_m)$ представляет собой 
несмещенную оценку~$R(T_m)$, а~при жесткой пороговой обработке и~выполнении 
условий теоремы смещение стремится к~нулю при делении на $\sqrt{n}$~\cite{Mallat}.

Для дисперсии числителя~(\ref{D0}) имеем:
\begin{multline*}
{\mathsf D} \left(\hat{R}(T_m) - R(T_m)\right) = \sum\limits_{i=1}^n {\mathsf D} F[x_i, T_m] + {}\\
{}+
\sum\limits_{i=1}^n\sum\limits_{\substack{j=1 \\  j\neq i}}^n \mathrm{cov}\left( F[x_i, T_m], F[x_j, 
T_m] \right).
\end{multline*}

Поскольку $\mu \in m_p[\eta_n]$,
\begin{equation}
\left.
\begin{array}{l}
 \displaystyle\sum\limits_{i: |\mu_i| > 1/T_1} {\mathsf D} F[x_i, T_m]  \leq{}\\
 \hspace*{15mm}{}\leq  4\left(\sigma^2 + T_m^2\right)^2 n \eta_n^p 
T_1^p = o(n);
\\[6pt]
\displaystyle \sum\limits_{\substack{{i,j: \max\{|\mu_i|, |\mu_j|\} > 1/T_1,}\\{j\neq i}}}  \hspace*{-12mm}\mathrm{cov}\,(F[x_i, 
T_m],F[x_j, T_m])  \leq{}\\
\hspace*{10mm}{}\leq 16\left(\sigma^2 + T_m^2\right)^2 n \eta_n^p T_1^p c_5 = o(n). 
\end{array}
\right\}    
\label{D2}
\end{equation}
Далее, учитывая что ${\mathsf D} x_i^2 \hm= 2\sigma^4 \hm+ 4\sigma^2 \mu_i^2$, нетрудно 
убедиться, что
\begin{multline}
\label{D3}
\sum\limits_{i: |\mu_i| \leq 1/T_1}\hspace*{-4mm} {\mathsf D} F[x_i, T_m] ={}\\
{}= \sum\limits_{i: |\mu_i| \leq 1/T_1} \hspace*{-4mm} {\mathsf D} 
x_i^2 + o(n) = 2\sigma^4 n + o(n).
\end{multline}


Введем обозначение 
$$
D_n = \left\{(i,j) : \max\left\{|\mu_i|, |\mu_j|\right\}  \leq \fr{1}{T_1}\,, \enskip j\hm\neq i\right\}.
$$
 Для суммы ковариаций аналогично~(\ref{D3}) получим
\begin{multline*}
\sum\limits_{(i,j)\in D_n} \hspace*{-2mm}\mathrm{cov}\left( F[x_i, T_m], F[x_j, T_m] \right) = {}\\
{}=
\sum\limits_{(i,j)\in D_n} \hspace*{-2mm}\mathrm{cov}\left( x_i^2, x_j^2 \right) + o(n).
\end{multline*}
Воспользуемся тождеством~\cite{Eroshenko}
$$
\mathrm{cov}\left (x_i^2, x_j^2\right) = 4 {\mathsf E} x_i {\mathsf E} x_j \mathrm{cov}\left(x_i, x_j\right) + 2 \mathrm{cov}^2 \left(x_i, x_j\right)
$$
для вектора $(x_i, x_j)$, имеющего двумерное нормальное распределение. Заметим, 
что
\begin{gather*}
 \sum\limits_{(i,j)\in D_n} 4 | {\mathsf E} x_i {\mathsf E} x_j \mathrm{cov}\left(x_i, x_j\right)| \leq 8 T_1^{-2} 
\sigma^2 n c_5 = o(n);
\\
\sum\limits_{(i,j)\in D_n} 2 \mathrm{cov}^2 (x_i, x_j)  = 2\sigma^4 \sum\limits_{(i,j)\in D_n} 
\mathrm{corr}^2 (x_i, x_j). 
\end{gather*}
Более того, поскольку  %< 4 \sigma^2 n c_5.$$
\begin{equation*}
\sum\limits_{\substack{{i,j: \max\{|\mu_i|, |\mu_j|\} > 1/T_1} \\ {j\neq i}}}
\hspace*{-10mm}\mathrm{corr}^2 (x_i, x_j)  
\leq  4 n \eta_n^p T_1^p c_5 =  o(n),
\end{equation*}
имеем
\begin{multline*}
\sum\limits_{(i,j)\in D_n} \mathrm{corr}^2 (x_i, x_j) ={}\\
{}= \sum\limits_{j\neq i} \mathrm{corr}^2 (x_i, x_j) 
+o(n)= c_6 n + o(n),
\end{multline*}
где
$$
c_6 = \lim\limits_{n\to\infty} \fr{1}{n} \sum\limits_{j\neq i} \mathrm{corr}^2 (x_i, x_j) 
\leq 2 c_5.
$$
Полагая $C_\rho \hm= \sigma^2\sqrt{1 + c_6}$, получим, наконец,
\begin{equation}
\label{D1}
{\mathsf D} \left(\hat{R}(T_m) - R(T_m)\right)  =  2 n C_\rho^2 + o(n).
\end{equation}
Заметим, что из~(\ref{D2}), (\ref{D3}) и~(\ref{D1}) следует, что
\begin{equation}
\label{D5}
\sup\limits_{n} \fr{\sum\nolimits_{i=1}^n {\mathsf D} F[x_i, T_m]}{V_n^2} < \infty\,,
\end{equation}
где 
$$
V_n^2 = {\mathsf D} \sum\limits_{i=1}^n \left(F[x_i, T_m] \hm- {\mathsf E} F[x_i, T_m]\right).
$$
Кроме того, поскольку $F[x_i, T_m]$ по модулю ограничены величиной $\sigma^2 \hm+ 
T_m^2$, выполнено условие Линдеберга: для любого $\eps\hm>0$ при $n \hm\to \infty$
\begin{multline}
\label{D6}
\!\!\!\fr{1}{V_n^2}\sum\limits_{i=1}^n {\mathsf E} \left( \!\left( F\left[x_i, T_m\right]\! -\! {\mathsf E} F\left[x_i, T_m\right]\right)^2 
\Ik \left(\vert F\left[x_i, T_m\right] -{}\right.\right.\hspace*{-2.69505pt}\\
\left.\left.{}- {\mathsf E} F\left[x_i, T_m\right]\vert >\eps V_n\right)\!
\vphantom{\left( F\left[x_i, T_m\right]\! -\! {\mathsf E} F\left[x_i, T_m\right]\right)^2}
\right) 
\to  0\,.
\end{multline}
Из~(\ref{D1})--(\ref{D6}), очевидного неравенства
$$ 
\lim\limits_{k\to\infty} \sup\limits_{n\geq k+1}\rho(k) \equiv 
\lim\limits_{k\to\infty} \rho^\star (k)  < 1
$$
 и~центральной предельной теоремы для сильно перемешанных случайных величин~\cite{Peligrad} следует~(\ref{D0}).

Перейдем к~доказательству того, что $U(\hat{t}_F) \, n^{-1/2} \overset{\, \p \, }{\to} 0$.
Всюду далее, не ограничивая общности, полагаем $\sigma=1$. 
Введем обозначения:

\noindent
\begin{align*}
S_1(T) &= \sum\limits_{i: |\mu_i| > 1/T_1} H_i(T, T_m); \\
S_2(T) &= \sum\limits_{i: |\mu_i| \leq 1/T_1} H_i(T, T_m); 
\\
N_1(a, b) &= \sum\limits_{i: |\mu_i| > 1/T_1} \Ik (a<|x_i|\leq b); \\ 
N_2(a, b) &= \sum\limits_{i: |\mu_i| \leq 1/T_1} \Ik (a<|x_i|\leq b);
\end{align*}

\noindent
\begin{align*}
Z_l(T) &= S_l(T) - {\mathsf E} S_l(T),\enskip l = 1,2\,; \\  
d_n &= \fr{T_U -  t_{\kappa_n}}{n};\\
T_j^{\prime} &= t_{\kappa_n}+j d_n,\enskip j = \overline{0,n-1}\,.
\end{align*} 

\vspace*{-3pt}

\noindent
Для произвольного $\eps>0$

\vspace*{-3pt}

\noindent
\begin{multline}
\p \left( \fr{|U(\hat{t}_F)|}{\sqrt{n}}> 4\eps \right) \leq 
\p\left(\hat{t}_F \leq t_{\kappa_n}\right) + {}\\
{}+\p \left(\fr{\sup\nolimits_{T\in 
[t_{\kappa_n}, T_U]} |U(T)|}{\sqrt{n}}>4\eps \right)\leq  {}\\
{}\leq \p\left(\hat{t}_F \leq t_{\kappa_n}\right) + \p\left(\fr{\sup\nolimits_{T\in 
[t_{\kappa_n}, T_U]} |{\mathsf E} U(T)|}{\sqrt{n}}>\eps\right)+{}\\
{}+ \p \left(\sup\limits_{T\in [t_{\kappa_n}, T_U]} |Z_1(T)| > 
\eps\sqrt{n}\right) +{}\\
{}+ \p \left(\sup\limits_{j \in [0, n-1]} |Z_2(T_j^{\prime})| > 
\eps\sqrt{n}\right) +{}\\
{}+ \p \left(\sup\limits_{\substack{j \in [0, n-1] \\
 T\in [T_j^{\prime},T_j^{\prime}+d_n]}} |Z_2(T)-Z_2(T_j^{\prime})| > \eps\sqrt{n}\right).
\label{M1}
\end{multline}
Заметим, что $\gamma_n\hm > \ln^{-1} n$, $\kappa_n\hm > n \eta_n^p \ln ^{-1} n \hm\geq 
n^{1-\delta_1} \ln ^{-1} n$ и~$q_n\hm > c_4 \ln ^{-1} n /2$ для всех достаточно 
больших~$n$.
Для первого слагаемого в~(\ref{M1}) по лемме~1 с~$m \hm= n^{\delta_1} \ln 
^7 n$ для  больших~$n$ имеем

\vspace*{-3pt}

\noindent
\begin{multline}
\label{M1next}
\p\left(\hat{t}_F \leq t_{\kappa_n}\right)  = \p \left(\hat{k}_F \geq \kappa_n 
\right) \leq 4 n e^{-\ln^2 n} + {}\\
{}+n^{1+(3/2)\,\delta_1} \ln^9 n \, 
\alpha\left(\left[\fr{n^{1-\delta_1}}{\ln^{7} n}\right]\right) = o(1)
\end{multline}
при $n\to\infty$. 
Для оценки второго слагаемого в~(\ref{M1}) заметим, что при $T \hm\in 
[t_{\kappa_n}, T_U]$ справедливо
\begin{equation}
\label{M2}
{\mathsf E} H_i(T, T_m) \leq T_U^2 + 1.
\end{equation}
Если же кроме $T \hm\in [t_{\kappa_n}, T_U]$ также выполнено $|\mu_i| \hm\leq T_1^{-1}$, то

\vspace*{-6pt}

\noindent
\begin{multline*}
|{\mathsf E} H_i (T, T_m)| \leq 2 T_U^2 \, \p \left(|x_i| > t_{\kappa_n}\right) \leq {}\\
{}\leq2 
T_U^2 \, \p \left(|x_i-\mu_i| > t_{\kappa_n}-T_1^{-1}\right) \leq{}\\
{}\leq 2 T_U^2  \exp\left\{ -\fr{1}{2} \left(t_{\kappa_n} - T_1^{-
1}\right)^2 \right\}  \leq{}\\
{}\leq
 4 (\ln n)  \exp\left\{ -\fr{1}{2} 
\left(z\left(\fr{q_n\kappa_n}{2n}\right)\right)^2 + t_{\kappa_n} T_1^{-
1}\right\},
\end{multline*}

\vspace*{-2pt}

\noindent
где использовано неравенство 

\noindent
$$
2(1-\Phi(x))\hm \leq \fr{e^{-x^2/2}}{x}
$$

\pagebreak


\noindent
 для $x\hm\geq 0$ 
($\Phi(x)$~--- функция распределения $N(0,1)$). Рас\-смот\-рим выражение 
в~экспоненте. Второе слагаемое не превышает $1\hm+o(1)$ при $n\hm\to\infty$, поскольку 
$t_{\kappa_n} \hm\leq T_1 (1+o(1))$ при $\sigma\hm=1$, что нетрудно получить из 
определения~$t_{\kappa_n}$, пред\-став\-ле\-ния~(\ref{lem1eq2}) и~ограничения на~$q_n$ 
из формулировки тео\-ре\-мы. Для первого слагаемого, используя пред\-став\-ле\-ние~(\ref{lem1eq1}) 
и~ограничения, наложенные на~$q_n$, при больших~$n$ получим
\begin{multline*}
-\fr{1}{2}\left(z\left(\fr{q_n \kappa_n}{2n}\right)\right)^2 \leq - \ln 
\fr{2n (1-q_n-\gamma_n)}{q_n n \eta_n^p T_1^{-p}} + {}\\
{}+\fr{1}{2} \ln 
\left((1+o(1)) \ln \eta_n^{-p}\right) + \fr{3}{2} \leq{}\\
{}\leq \ln \fr{c_3}{1-c_3} + \ln \eta_n^p + \ln T_1^{-p} + \ln T_1 + 
\fr{3}{2}+ o(1).
\end{multline*}
Из приведенных соотношений следует, что с~некоторой константой $c_7 = c_7(c_3, 
p, \delta_1, \delta_2, c_4)$
\begin{equation}\label{M3}
\sup\limits_{\substack{i: |\mu_i| \leq 1/T_1 \\ T\in [t_{\kappa_n}, T_U]}} |{\mathsf E} 
H_i (T, T_m)|  \leq c_7 (\ln n)^{(3-p)/2}\eta_n^p.
\end{equation}
Из (\ref{M2}) и~(\ref{M3}) с~учетом $\delta_2 \hm> 1/2$ следует
\begin{multline*}
\sup\limits_{T\in [t_{\kappa_n}, T_U]} |{\mathsf E} U(T)| \leq{}\\
{}\leq 
 n\eta_n^p T_1^p 
(T_U^2+1) + c_7 (\ln n)^{(3-p)/2} n \eta_n^p = o(\sqrt{n})
\end{multline*}
при $n\to\infty$, а следовательно, для любого $\eps\hm>0$ второе слагаемое в~(\ref{M1}) обращается в~ноль для всех достаточно больших~$n$.

Далее, поскольку при $T \hm\leq T_U$ и~$\sigma\hm=1$
$$
|H_i(T, T_m) - {\mathsf E} H_i(T, T_m)| \leq 2 (T_U^2 +2), \enskip i=\overline{1, n}\,,
$$
а число слагаемых в~$Z_1(T)$ не превосходит $n\eta_n^p T_1^p$, имеем
$$
\sup\limits_{T\in [t_{\kappa_n}, T_U]} |Z_1(T)|  \leq 2 n\eta_n^p T_1^p (T_U^2 
+2) = o(\sqrt{n})
$$
при $n\to\infty$, а следовательно, для любого $\eps\hm>0$ и~третье слагаемое в~(\ref{M1}) обращается в~ноль для всех достаточно больших~$n$.

Перейдем к~оценке четвертого слагаемого в~(\ref{M1}). Аналогично~(\ref{M3}) 
можно получить:
\begin{multline}
\label{M10}
\!\!\sup\limits_{\substack{i: |\mu_i| \leq 1/T_1 \\ T\in [t_{\kappa_n}, T_U]}} \!{\mathsf D} 
H_i (T, T_m)  \leq \!\sup\limits_{\substack{i: |\mu_i| \leq 1/T_1 \\ T\in 
[t_{\kappa_n}, T_U]}} \!{\mathsf E} \left(H_i (T, T_m)\right)^2  \leq{}\\
{}\leq 2 c_7 (\ln n)^{(5-p)/2} \eta_n^p.
\end{multline}
По лемме~4 с~$m \hm= \sqrt{n} (\ln n)^3$ и~$k \hm= n-[n\eta_n^p T_1^p]$ 
для четвертого слагаемого в~(\ref{M1}) имеем:

\noindent
\begin{multline}
\p \left(\sup\limits_{j \in [0, n-1]} |Z_2(T_j^\prime)| > \eps\sqrt{n}\right) 
\leq {}\\
{}\leq \sum\limits_{j \in [0, n-1]} \hspace*{-3mm}\p \left( |Z_2(T_j^\prime)| > \varepsilon\sqrt{n}\right)\leq{}\\
{}\leq 4 n \exp \left\{ - \fr{\eps^2 n^{3/2} (\ln n)^3}{n-[n\eta_n^p T_1^p]}\!\Bigg/\! \big( 8 (T_U^2+2)\eps\sqrt{n} +{}\right.\\
\left.{}+ 128 c_7 (\ln n)^{(5-p)/2} \eta_n^p  (n-
[n\eta_n^p T_1^p])\big) 
\vphantom{ \fr{\eps^2 n^{3/2} (\ln n)^3}{n-[n\eta_n^p T_1^p]}}
\right\} +{}\\
{}
+ 22 \left(1+\fr{8(T_U^2+2) (n-[n\eta_n^p T_1^p])}{\eps 
\sqrt{n}}\right)^{1/2}\times{}\\
{}\times n^{3/2} (\ln n)^3 \alpha\left(\left[\fr{n-[n\eta_n^p 
T_1^p]}{2 (\ln n)^3 \sqrt{n}}\right]\right).
\label{M5}
\end{multline}
Используя ограничения $n^{-\delta_1}\hm\leq \eta_n^p \leq n^{-\delta_2}$ 
и~$1/2\hm<\delta_2\hm<\delta_1\hm<1$, из~(\ref{M5}) получим для любого $\eps\hm>0$
$$
\p \left(\sup\limits_{j \in [0, n-1]} |Z_2(T_j^\prime)| > \eps\sqrt{n}\right) 
\to 0
$$
при $n \to \infty$.

Рассмотрим, наконец, пятое слагаемое в~(\ref{M1})). Заметим, что при $0\hm< a \hm< b$ 
справедливо
$$
|Z_2(b)-Z_2(a)| \leq 2 |N_2(a,b)-{\mathsf E} N_2(a,b)| + n (b^2-a^2).
$$
Полагая $a = T_j^\prime$, $b \hm= T \hm\in [T_j^\prime, T_j^\prime+d_n]$ для 
произвольного $j \hm\in [0, n-1]$ и~учитывая, что
$$
(T^2 - (T_j^\prime )^2) = (T - T_j^\prime)(T+ T_j^\prime ) \leq  2 d_n T_U < 2 
T_U^2 n^{-1}; 
$$

\vspace*{-12pt}

\noindent
\begin{multline*}
\p\left(T_j^\prime < |x_i| \leq T \right) \leq \p\left(T_j^\prime < |x_i| \leq 
T_j^\prime+d_n\right) <{}\\
{}< d_n < T_U n^{-1}, 
\end{multline*}
получим  оценку
$$
|Z_2(T)-Z_2(T_j^\prime)| \leq 2 N_2(T_j^\prime, T) +  3 T_U^2 .
$$
Далее, поскольку $N_2 (T_j^\prime, T) \hm\leq N_2 (T_j^\prime, T_j^\prime+d_n)$ и~${\mathsf E} N_2 (T_j^\prime, T_j^\prime+d_n) \hm< T_U^2$,
имеем
\begin{multline*}
\sup\limits_{T \in [T_j^\prime, T_j^\prime+d_n]} |Z_2(T)-Z_2(T_j^\prime)| \leq {}\\
{}\leq
2 \left|N_2 (T_j^\prime, T_j^\prime+d_n) - {\mathsf E} N_2 (T_j^\prime, 
T_j^\prime+d_n)\right| +  5 T_U^2 .
\end{multline*}
Аналогично~(\ref{M3}) показывается, что
\begin{multline}
\label{M11}
\sup\limits_{\substack{i : |\mu_i| \leq 1/T_1 \\ j \in [0, n-1]}} {\mathsf D} \Ik 
(T_j^\prime < |x_i| \leq T_j^\prime + d_n) <{}\\
{}< c_7 (\ln n)^{(1-p)/2} \eta_n^p.
\end{multline}
Пусть $n > N(\eps)$ настолько, что 
$$
\fr{\eps\sqrt{n} - 5 T_U^2}{2} > \fr{\eps \sqrt{n} }{4}\,.
$$
%
 Тогда для пятого слагаемого в~(\ref{M1}) по лемме~4 с~$m \hm= 
\sqrt{n} (\ln n)^2$ и~$k \hm= n\hm-[n\eta_n^p T_1^p]$ имеем
\begin{multline}
\p \left(\sup\limits_{\substack{j \in [0, n-1] \\ T\in 
[T_j^{\prime},T_j^{\prime}+d_n]}} |Z_2(T)-Z_2(T_j^{\prime})| > 
\eps\sqrt{n}\right) \leq{}\\
{}\leq  \sum\limits_{j \in [0, n-1]} \p \left(  \left|N_2 (T_j^\prime, 
T_j^\prime+d_n) -{}\right.\right.\\
\left.\left.{}- {\mathsf E} N_2 (T_j^\prime, T_j^\prime+d_n)\right| > \fr{\eps\sqrt{n}}{4} 
\right) \leq{}\\
{}\leq  4n \exp \left\{ -  \fr{\eps^2 n^{3/2} (\ln n)^2}{(n-[n\eta_n^p T_1^p])^{-1}}\Bigg/ 
\big( 16 \eps \sqrt{n} +{}\right.\\
\left.{}+ 64 c_7 (\ln n)^{(1-p)/2} \eta_n^p (n-[n\eta_n^p 
T_1^p]) \big) 
\vphantom{\fr{\eps^2 n^{3/2} (\ln n)^2}{(n-[n\eta_n^p T_1^p])^{-1}}}
\right\} +{}\\
{}+ 22 \left(1+\fr{16 (n-[n\eta_n^p T_1^p])}{\eps \sqrt{n}}\right)^{1/2}\times{}\\
{}\times 
n^{3/2} (\ln n)^2 \alpha\left(\left[\fr{n-[n\eta_n^p T_1^p]}{2 (\ln n)^2 
\sqrt{n}}\right]\right).
\label{M6}
\end{multline}
Используя ограничения $n^{-\delta_1}\hm\leq \eta_n^p\hm \leq n^{-\delta_2}$ 
и~$1/2\hm<\delta_2\hm<\delta_1<1$, из~(\ref{M6}) получим для любого $\eps\hm>0$
$$
\p \left(\sup\limits_{\substack{j \in [0, n-1] \\ T\in 
[T_j^{\prime},T_j^{\prime}+d_n]}} |Z_2(T)-Z_2(T_j^{\prime})| > 
\eps\sqrt{n}\right) \to 0
$$
при $n \to \infty$.

Таким образом, показано, что для любого $\eps>0$ все слагаемые в~(\ref{M1}) 
стремятся к~нулю при $n\to\infty$. Следовательно,
$$
\fr{|U(\hat{t}_F)|}{\sqrt{n}}  \overset{\, \p \, }{\to} 0 \,,
$$
что вместе с~(\ref{D0}) завершает доказательство тео\-ремы.~\hfill$\square$

\smallskip

Следующая теорема дает достаточные условия для асимптотической нормальности 
оценки риска $\hat{R}(\hat{t}_F)$ в~случае $\mu \hm\in l_0[\eta_n]$.

\smallskip

\noindent
\textbf{Теорема 2.}\ 
\textit{Пусть $\mu \hm\in l_0[\eta_n]$, $\eta_n\hm\in[n^{-\delta_1}, n^{-\delta_2}]$, $1/2\hm < 
\delta_2\hm < \delta_1\hm<1;$ имеются такие константы $c_1, c_2\hm>0$, что для 
коэффициента сильного перемешивания $\alpha(\cdot)$ компонент вектора~$x$ 
справедливо} 
\begin{gather*}
\alpha(k) \leq c_1 k^{-1-(5/2)\delta_1/(1\hm-\delta_1)\hm-c_2},\enskip 
k=\overline{1,n-1};\\
 q_n < c_3 < 1;\enskip \mathrm{lim\,inf} q_n \ln n = c_4 > 0;
\end{gather*}
\textit{для максимального коэффициента корреляции~$\rho(\cdot)$ компонент вектора~$x$ 
справедливо}
$$
\sum\limits_{k = 1}^{\infty} \sup\limits_{n\geq k+1} \rho(k) \equiv 
\sum\limits_{k = 1}^{\infty}  \rho^\star (k) = c_5 < \infty. 
$$
\textit{Тогда при $n \to \infty$}
$$
\fr{\hat{R}(\hat{t}_F) - R(T_m)}{C_\rho \sqrt{2n}} \Rightarrow N(0, 1),
$$
\textit{где}
$$
C_\rho = \sigma^2\sqrt{1 +   \lim\limits_{n\to\infty} \fr{1}{n} 
\sum\limits_{j\neq i} \mathrm{corr}^2 (x_i, x_j)}\,.
$$

\noindent
Д\,о\,к\,а\,з\,а\,т\,е\,л\,ь\,с\,т\,в\,о\  проводится аналогично доказательству теоремы~1. 
Переменная~$D_n$ теперь определяется как $D_n \hm= \{(i,j) : 
|\mu_i|\hm=|\mu_j|=0$, $j\hm\neq i\}$. Условия вида $|\mu_i|\hm<T_1^{-1}$ (вида 
$|\mu_i|\hm\geq T_1^{-1}$) заменяются условиями  $\mu_i\hm=0$ (соответственно 
$|\mu_i|\hm>0$).
Поскольку $\mu \hm\in l_0[\eta_n]$, количество~$i$ таких, что $|\mu_i|\hm>0$ 
(а~значит, и~число слагаемых в~$Z_1(T)$), не превышает~$[n \eta_n]$.

Для оценки первого слагаемого в~(\ref{M1}) используется лемма~2, 
в~которой можно взять, например, $b\hm=1/2$, а~для~$\kappa_n^0$ использовать оценку 
$\kappa_n^0 \hm> n \eta_n$. Формулы (\ref{M3}),  (\ref{M10}) и~(\ref{M11}) 
принимают вид соответственно
\begin{align*}
\sup\limits_{\substack{i: \mu_i =0 \\ T\in [t_{\kappa_n^0}, T_U]}} |{\mathsf E} H_i (T, 
T_m)| & \leq c_8 (\ln n)^{3/2} \eta_n ;
\\
\sup\limits_{\substack{i: \mu_i =0 \\ T\in [t_{\kappa_n^0}, T_U]}} {\mathsf D} H_i (T, 
T_m)  & \leq 2 c_8 (\ln n)^{5/2} \eta_n;
\\
\sup\limits_{\substack{i : \mu_i =0 \\ j \in [0, n-1]}} {\mathsf D} \Ik (T_j^\prime < 
|x_i| \leq T_j^\prime + d_n) &< c_8 (\ln n)^{1/2} \eta_n,
\end{align*}
где $c_8 = c_8(c_3,\delta_1, \delta_2, c_4)$. В~остальном доказательство 
аналогично.~\hfill$\square$

\section{Сильная состоятельность оценки риска при~применении FDR-процедуры 
в~условиях слабой зависимости}

Следующая теорема дает достаточные условия для сильной состоятельности оценки 
риска $\hat{R}(\hat{t}_F)$ в~случаях $\mu \hm\in m_p[\eta_n]$ и~$\mu \hm\in 
l_0[\eta_n]$.

\smallskip

\noindent
\textbf{Теорема 3.}
\textit{Пусть $\mu\hm \in m_p[\eta_n]$, $\eta_n^p\hm\in[n^{-\delta_1}, n^{-\delta_2}]$ либо 
$\mu \hm\in l_0[\eta_n]$, $\eta_n\hm\in[n^{-\delta_1}, n^{-\delta_2}]$; $0 \hm< \delta_2 
\hm< \delta_1<1$; имеются такие константы $c_1, c_2\hm>0$, что для коэффициента 
сильного перемешивания $\alpha(\cdot)$ компонент вектора~$x$ справедливо}  
$\alpha(k) \hm\leq c_1 k^{-2-(7/2)\delta_1/(1\hm-\delta_1)\hm-c_2}$, $k\hm=\overline{1,n-1}$; 
$q_n \hm< c_3 \hm< 1$; $\mathrm{lim\,inf} q_n \ln n \hm= c_4 \hm> 0$. \textit{Тогда при} $n \hm\to \infty$
$$
\fr{\hat{R}(\hat{t}_F) - R(T_m)}{n} \rightarrow 0 \, \, \,\textit{п.~в.}
$$


\noindent
Д\,о\,к\,а\,з\,а\,т\,е\,л\,ь\,с\,т\,в\,о\,.  Воспользуемся представлением~(\ref{D00}).

Покажем, что $(\hat{R}(T_m)-R(T_m))n^{-1}\hm \to 0$ п.~в.\ при $n\hm\to\infty$. 
При мягкой пороговой обработке ${\mathsf E} \hat{R}(T_m) \hm= R(T_m)$, а~при жесткой 
пороговой обработке
\begin{multline*}
\fr{\hat{R}(T_m)-R(T_m)}{n} = {}\\
{}=\fr{\hat{R}(T_m)-{\mathsf E} \hat{R}(T_m)}{n} 
+\fr{{\mathsf E}\hat{R}(T_m)-R(T_m)}{n}\,,
\end{multline*}
где второе слагаемое стремится к~нулю при $n\to\infty$ \cite{Mallat}. 
Следовательно, достаточно показать, что $(\hat{R}(T_m)\hm-{\mathsf E}\hat{R}(T_m))n^{-1} \hm\to 0$ п.~в.

Полагая в~лемме~3 $X_i \hm= F[x_i, T_m] \hm- {\mathsf E} F[x_i, T_m]$, $b \hm= 
2(\sigma^2\hm+T_m^2)$ и~$m \hm= n^{1/4}$ и~учитывая ограничения на $\alpha(\cdot)$ из 
условия, нетрудно убедиться, что для всех~$n$
$$
\p \left(\left| \fr{\hat{R}(T_m)-{\mathsf E} \hat{R}(T_m)}{n}\right| >\eps \right) 
\leq \fr{c_5}{n^{1+c_6}}\,, 
$$
где константы $c_5$, $c_6$ положительны. Отсюда
$$
\sum\limits_{n=1}^{\infty}\p \left(\left|\fr{\hat{R}(T_m)-{\mathsf E} 
\hat{R}(T_m)}{n}\right| >\eps \right) < \infty,
$$
и по теореме~1.3.4 из~\cite{Serfling2002} 
$$
\left(\hat{R}(T_m)-{\mathsf E}\hat{R}(T_m)\right)n^{-1} \to 0~\mbox{п.~в.}
$$



Покажем теперь, что  $U(\hat{t}_F) \, n^{-1}\hm \to 0$ п.~в. Доказательство 
проведено для $\mu \hm\in m_p[\eta_n]$, в~случае $\mu\hm \in l_0[\eta_n]$ 
доказательство аналогично.
Аналогично формуле~(\ref{M1}), для произвольного $\eps\hm>0$ в~терминах тео\-ре\-мы~1 имеем
\begin{multline*}
\p \left( \fr{|U(\hat{t}_F)|}{n}> 4\eps \right) \leq \p\left(\hat{t}_F 
\leq t_{\kappa_n}\right) +{}\\
{}+ \p\left(\fr{\sup\nolimits_{T\in [t_{\kappa_n}, T_U]} |{\mathsf E} 
U(T)|}{n}>\eps\right)+{}\\
{}+ \p \left(\sup\limits_{T\in [t_{\kappa_n}, T_U]} |Z_1(T)| > \eps n\right) +{}
\end{multline*}

\noindent
\begin{multline}
{}+ \p  \left(\sup\limits_{j \in [0, n-1]} |Z_2(T_j^{\prime})| > \eps n\right) +{}\\
{}+ \p \left(\sup\limits_{\substack{j \in [0, n-1] \\ T\in 
[T_j^{\prime},T_j^{\prime}+d_n]}} |Z_2(T)-Z_2(T_j^{\prime})| > \eps n\right).
\label{M1SC}
\end{multline}
Применяя рассуждения, аналогичные приведенным в~доказательстве теоремы~1, можно показать, что
$$
\sup\limits_{T\in [t_{\kappa_n}, T_U]} |{\mathsf E} U(T)| = o(n); \enskip
\sup\limits_{T\in [t_{\kappa_n}, T_U]} |Z_1(T)|  = o(n),
$$
откуда следует, что второе и~третье слагаемые в~(\ref{M1SC}) обращаются в~ноль 
для всех достаточно больших~$n$.

Для некоторых положительных констант  $c_7$ и~$c_8$ первое, четвертое и~пятое 
слагаемые  в~(\ref{M1SC}) не превышают $c_7 n^{-1-c_8}$ для всех достаточно 
боль\-ших~$n$, что можно показать с~помощью ограничения на $\alpha(\cdot)$ из 
условия и~рассуждений, аналогичных приведенным при выводе соответственно формул~(\ref{M1next}), (\ref{M5}) и~(\ref{M6}), с~тем отличием, что при применении 
леммы~4 полагается $m \hm= (\ln n)^3$.

Из доказанного следует, что
$$
\sum\limits_{n=1}^{\infty}\p \left( \fr{|U(\hat{t}_F)|}{n}> 4\eps \right) 
< \infty,
$$
и по теореме~1.3.4 из~\cite{Serfling2002} $U(\hat{t}_F) \, n^{-1} \to 0$ п.~в., 
что завершает доказательство теоремы.~\hfill$\square$



{\small\frenchspacing
 {\baselineskip=11.5pt
 %\addcontentsline{toc}{section}{References}
 \begin{thebibliography}{99}
\bibitem{FDRImage}
\Au{Krylov V.\,A., Moser~G., Serpico~S.\,B., Zerubia~J.}
False discovery rate approach to unsupervised image change detection~// IEEE 
T. Image Process., 2016. Vol.~25. No.\,10. P.~4704--4718. doi: 10.1109/TIP.2016.2593340.

\bibitem{MultipleTesting} %2
\Au{Menyhart~O., Weltz~B., Gyorffy~B.}
MultipleTesting.com: A~tool for life science researchers for multiple hypothesis 
testing correction~// PLoS One, 2021. Vol.~16. No.\,6. Art.~0245824. doi: 10.1371/journal.pone.0245824.

\bibitem{AdaptingFDR} %3
\Au{Abramovich~F., Benjamini~Y., Donoho~D., Johnstone~I.}
Adapting to unknown sparsity by controlling the false discovery rate~// Ann. Stat., 2006. Vol.~34. No.\,2. P.~584--653.
doi: 10.1214/009053606000000074.

\bibitem{ZasShe17} %4
\Au{Заспа~А.\,Ю., Шестаков~О.\,В.}
Состоятельность оценки риска при множественной проверке гипотез с~FDR-по\-ро\-гом~// 
Вестник ТвГУ. Сер. Прикладная математика, 2017. Вып.~1. С.~5--16.
doi: 10.26456/vtpmk119. EDN: YFYJXT.

\bibitem{Mathematics2020} %5
\Au{Palionnaya~S.\,I., Shestakov~O.\,V.}
Asymptotic properties of MSE estimate for the false discovery rate controlling 
procedures in multiple hypothesis testing // Mathematics, 2020. Vol.~8. No.~11. 
Art.~1913. 11~p. doi: 10.3390/ math8111913.

\bibitem{Shestakov2021-1} %6
\Au{Шестаков~О.\,В.}
Анализ несмещенной оценки среднеквадратичного риска метода блочной пороговой 
обработки~// Информатика и~её применения, 2021. Т.~15. Вып.~2. С.~30--35.
doi: 10.14357/19922264210205. EDN: DSQQAU.

\bibitem{Shestakov2021-2} %7
\Au{Шестаков~О.\,В.}
Пороговые функции в~методах подавления шума, основанных на вейв\-лет-раз\-ло\-же\-нии 
сигнала~// Информатика и~её применения, 2021. Т.~15. Вып.~3. С.~51--56.
doi: 10.14357/19922264210307. EDN: WSEAYG.

\bibitem{Shestakov2022} %8
\Au{Шестаков~О.\,В.}
Несмещенная оценка риска пороговой обработки с~двумя пороговыми значениями~// 
Информатика и~её применения, 2022. Т.~16. Вып.~4. С.~14--19.
doi: 10.14357/19922264220403. EDN: \mbox{DZBVLC}.

\bibitem{ResultsOnFDRUnderDependence} %9
\Au{Farcomeni~A.}
Some results on the control of the false discovery rate under dependence~// 
Scand. J. Stat., 2007. Vol.~34. No.\,2. P.~275--297.
doi: 10.1111/j.1467-9469.2006.00530.x.

\bibitem{VorontsovShestakov2023} %10
\Au{Воронцов~М.\,О., Шестаков~О.\,В.}
Среднеквадратичный риск FDR-про\-це\-ду\-ры в~условиях слабой за\-ви\-си\-мости~// 
Информатика и~её применения, 2023. Т.~17. Вып.~2. С.~34--40.
doi: 10.14357/19922264230205. EDN: AVJZDX.

\bibitem{Vorontsov2024} %11
\Au{Воронцов~М.\,О.}
Анализ среднеквадратичного риска при использовании методов множественной 
проверки гипотез для выбора параметров пороговой обработки в~условиях слабой 
зависимости~// Вестник Московского университета. Сер. 15: Вычислительная 
математика и~кибернетика, 2024. №\,2. С.~18--24.

\bibitem{Bosq} %12
\Au{Bosq~D.}
Nonparametric statistics for stochastic processes: Estimation and prediction.~--- 
Lecture notes in statistics ser.~--- New York, NY, USA: Springer, 1996. Vol.~110. 
188~p.

\bibitem{Mallat} %13
\Au{Mallat~S.}
A wavelet tour of signal processing.~--- New York, NY, USA: Academic Press, 1999. 
857~p.

\bibitem{spatialAdaptation} %14
\Au{Donoho~D., Johnstone~I.}
Ideal spatial adaptation via wavelet shrinkage~// Biometrika, 1994. Vol.~81. 
No.\,3. P.~425--455. doi: 10.1093/biomet/81.3.425.

\bibitem{AdaptingSURE} %15
\Au{Donoho D., Johnstone I.\,M.}
Adapting to unknown smoothness via wavelet shrinkage~// J.~Amer. Stat. Assoc., 
1995. Vol.~90. P.~1200--1224.

\bibitem{ExactRisk} %16
\Au{Marron J.\,S., Adak~S., Johnstone~I.\,M., Neumann~M.\,H., Patil~P.}
Exact risk analysis of wavelet regression~// J.~Comput. Graph. Stat., 1998. 
Vol.~7. P.~278--309. doi: 10.1080/ 10618600.1998.10474777.

\bibitem{Jansen} %17
\Au{Jansen~M.}
Noise reduction by wavelet thresholding.~-- Lecture notes in statistics ser.~--- 
New York, NY, USA: Springer, 2001. Vol.~161. 217~p.

\bibitem{KuShe2016_1} %18
\Au{Кудрявцев~А.\,А., Шестаков~О.\,В.}
Асимптотическое поведение порога, минимизирующего усредненную\linebreak вероятность ошибки 
вычисления вейв\-лет-ко\-эф\-фи\-ци\-ен\-тов~// Докл. Акад. наук, 2016. Т.~468. №\,5. 
С.~487--491.

\bibitem{KuShe2016_2} %19
\Au{Кудрявцев~А.\,А., Шестаков~О.\,В.}
Асимптотически оптимальная пороговая обработка вейв\-лет-ко\-эф\-фи\-ци\-ен\-тов в~моделях с~негауссовым распределением шума~// Докл. Акад. наук, 2016. Т.~471. №\,1. 
С.~11--15.



\bibitem{Eroshenko} %20
\Au{Ерошенко~А.\,А.}
Статистические свойства оценок сигналов и~изображений при пороговой обработке 
коэффициентов в~вейв\-лет-раз\-ло\-же\-ни\-ях: Дис.\ \ldots\ канд. физ.-мат. наук.~--- 
М.: МГУ, 2015. 82~с.

\bibitem{Peligrad} %21
\Au{Peligrad~M.}
On the asymptotic normality of sequences of weak dependent random variables~// 
J. Theor. Probab., 1996. Vol.~9. No.\,3. P.~703--715. doi: 10.1007/BF02214083.

\bibitem{Serfling2002} %22
\Au{Serfling~R.\,J.}
Approximation theorems of mathematical statistics.~--- New York, NY, USA: John Wiley \&~Sons, Inc., 2002. 371~p.

\end{thebibliography}

 }
 }

\end{multicols}

\vspace*{-6pt}

\hfill{\small\textit{Поступила в~редакцию 21.05.24}}

\vspace*{8pt}

%\pagebreak

%\newpage

%\vspace*{-28pt}

\hrule

\vspace*{2pt}

\hrule



\def\tit{ASYMPTOTIC NORMALITY AND STRONG CONSISTENCY\\ OF~RISK ESTIMATE WHEN USING THE~FDR THRESHOLD\\ UNDER WEAK DEPENDENCE CONDITION}


\def\titkol{Asymptotic normality and strong consistency of~risk estimate when using the~FDR threshold under weak dependence condition}


\def\aut{M.\,O.~Vorontsov$^{1,2}$ and~O.\,V.~Shestakov$^{1,2,3}$}

\def\autkol{M.\,O.~Vorontsov and~O.\,V.~Shestakov}

\titel{\tit}{\aut}{\autkol}{\titkol}

\vspace*{-13pt}


\noindent
$^{1}$Department of Mathematical Statistics, Faculty of Computational Mathematics and Cybernetics,
 M.\,V.~Lo\-mo-\linebreak
 $\hphantom{^1}$nosov Moscow State University, 1-52~Leninskie Gory, GSP-1, Moscow 119991, Russian Federation

\noindent
$^{2}$Moscow Center for Fundamental and Applied Mathematics, M.\,V.~Lomonosov Moscow State University,\linebreak
$\hphantom{^1}$1~Leninskie Gory, GSP-1, Moscow 119991, Russian Federation

\noindent
$^{3}$Federal Research Center ``Computer Science and Control'' of the Russian Academy of Sciences, 44-2~Vavilov\linebreak
$\hphantom{^1}$Str., Moscow 119333, Russian Federation


\def\leftfootline{\small{\textbf{\thepage}
\hfill INFORMATIKA I EE PRIMENENIYA~--- INFORMATICS AND
APPLICATIONS\ \ \ 2024\ \ \ volume~18\ \ \ issue\ 3}
}%
 \def\rightfootline{\small{INFORMATIKA I EE PRIMENENIYA~---
INFORMATICS AND APPLICATIONS\ \ \ 2024\ \ \ volume~18\ \ \ issue\ 3
\hfill \textbf{\thepage}}}

\vspace*{2pt}






\Abste{An approach to solving the problem of noise removal in a large array of sparse data is considered
 based on the method of controlling the average proportion of false hypothesis rejections (False Discovery Rate, FDR). 
 This approach is equivalent to threshold processing procedures that remove array components whose values do not exceed 
 some specified threshold. The observations in the model are considered weakly dependent. To control the\linebreak\vspace*{-12pt}}
 
 \Abstend{degree of dependence, 
 restrictions on the strong mixing coefficient and the maximum correlation coefficient are used. The mean-square risk is 
 used as a measure of the effectiveness of the considered approach. It is possible to calculate the risk value only on the test data;
  therefore, its statistical estimate is considered in the work and its properties are investigated. The asymptotic normality and
   strong consistency of the risk estimate are proved when using the FDR threshold under conditions of weak dependence in the data.}

\KWE{thresholding; multiple hypothesis testing; risk estimate}

\DOI{10.14357/19922264240309}{ZOQVTO}

%\vspace*{-12pt}


    
   %   \Ack

%\vspace*{-3pt}
%\noindent



  \begin{multicols}{2}

\renewcommand{\bibname}{\protect\rmfamily References}
%\renewcommand{\bibname}{\large\protect\rm References}

{\small\frenchspacing
 {\baselineskip=10.8pt
 \addcontentsline{toc}{section}{References}
 \begin{thebibliography}{99} 

%1
\bibitem{FDRImage-1}
\Aue{Krylov, V.\,A., G.~Moser, S.\,B.~Serpico, and J.~Zerubia.} 2016. 
False discovery rate approach to unsupervised image change detection. 
\textit{IEEE T. Image Process.} 25(10):4704--4718. doi: 10.1109/TIP.2016.2593340.

%2
\bibitem{MultipleTesting-1}
\Aue{Menyhart, O., B.~Weltz, and B.~Gyorffy.} 2021. 
MultipleTesting.com: A~tool for life science researchers for multiple hypothesis testing correction. 
\textit{PLoS One} 16(6):0245824. 
doi: 10.1371/journal.pone.0245824.

%3
\bibitem{AdaptingFDR-1}
\Aue{Abramovich, F., Y.~Benjamini, D.~Donoho, and I.\,M.~Johnstone.} 2006. 
Adapting to unknown sparsity by controlling the false discovery rate. 
\textit{Ann. Stat.} 34(2):584--653. 
doi: 10.1214/009053606000000074.


%4
\bibitem{ZasShe17-1}
\Aue{Zaspa, A.\,Yu., and O.\,V.~Shestakov.} 2017.
Sostoyatel'nost' otsenki riska pri mnozhestvennoy proverke gipotez s~FDR-porogom
 [Consistency of the risk estimate of the multiple hypothesis testing with the FDR threshold]. 
\textit{Vestnik TvGU. Ser.: Prikladnaya matematika} [Herald of Tver State University. Ser. Applied Mathematics] 1:5--16.
doi: 10.26456/vtpmk119. EDN: YFYJXT.

%5
\bibitem{Mathematics2020-1}
\Aue{Palionnaya, S.\,I., and O.\,V.~Shestakov.} 2020. 
Asymptotic properties of MSE estimate for the false discovery rate controlling procedures in multiple hypothesis testing. 
\textit{Mathematics} 8(11):1913. 11~p.
doi: 10.3390/math8111913.

%6
\bibitem{Shestakov2021-1-1}
\Aue{Shestakov, O.\,V.} 2021.
Analiz nesmeshchennoy otsenki srednekvadratichnogo riska metoda blochnoy po\-ro\-go\-voy obrabotki 
[Analysis of the unbiased mean-square risk estimate of the block thresholding method]. 
\textit{Informatika i~ee Primeneniya~--- Inform. Appl.} 15(2):30--35.
doi: 10.14357/19922264210205. EDN: DSQQAU.

%7
\bibitem{Shestakov2021-2-1}
\Aue{Shestakov, O.\,V.} 2021.
Porogovye funktsii v~metodakh podavleniya shuma, osnovannykh na veyvlet-razlozhenii signala 
[Thresholding functions in the noise suppression methods based on the wavelet expansion of the signal]. 
\textit{Informatika i~ee Primeneniya~--- Inform. Appl.} 15(3):51--56.
doi: 10.14357/19922264210307. EDN: WSEAYG.

%8
\bibitem{Shestakov2022-1}
\Aue{Shestakov, O.\,V.} 2022.
Nesmeshchennaya otsenka riska porogovoy obrabotki s dvumya porogovymi znacheniyami [Unbiased thresholding risk estimate with two threshold values]. 
\textit{Informatika i~ee Primeneniya~--- Inform. Appl.} 16(4):14--19.
doi: 10.14357/19922264220403. EDN: DZBVLC.

%9
\bibitem{ResultsOnFDRUnderDependence-1}
\Aue{Farcomeni, A.} 2007. Some results on the control of the false discovery rate under dependence. 
\textit{Scand. J. Stat.} 34(2):275--297. 
doi: 10.1111/j.1467-9469.2006.00530.x.

%10
\bibitem{VorontsovShestakov2023-1}
\Aue{Vorontsov, M.\,O., and O.\,V.~Shestakov.} 2023.
Sred\-ne\-kvad\-ra\-tich\-nyy risk FDR-protsedury v~usloviyakh slaboy za\-vi\-si\-mosti [Mean-square risk of the FDR procedure under weak dependence]. 
\textit{Informatika i~ee Primeneniya~--- Inform. Appl.} 17(2):34--40.
doi: 10.14357/19922264230205. EDN: AVJZDX.

%11
\bibitem{Vorontsov2024-1}
\Aue{Vorontsov, M.\,O.} 2024. 
RMS risk analysis when using multiple hypothesis testing select parameters of thresholding under conditions of weak dependence. 
\textit{Moscow University Computational Mathematics Cybernetics} 48:91--97. 
doi: 10.3103/S027864192470002X.

%12
\bibitem{Bosq-1}
\Aue{Bosq, D.} 1996. 
\textit{Nonparametric statistics for stochastic processes: Estimation and prediction}. 
Lecture notes in statistics ser. New York, NY: Springer Verlag. Vol.~110. 188~p.

%13
\bibitem{Mallat-1}
\Aue{Mallat, S.} 1999. 
\textit{A wavelet tour of signal processing}. New York, NY: Academic Press. 857~p.

%14
\bibitem{spatialAdaptation-1}
\Aue{Donoho, D., and I.\,M.~Johnstone.} 1994. 
Ideal spatial adaptation via wavelet shrinkage. 
\textit{Biometrika} 81(3):425--455. doi: 10.1093/biomet/81.3.425.

%15
\bibitem{AdaptingSURE-1}
\Aue{Donoho, D., and I.\,M.~Johnstone.} 1995. 
Adapting to unknown smoothness via wavelet shrinkage. 
\textit{J. Am. Stat. Assoc.} 90(432):1200--1224. doi: 10.1080/01621459. 1995.10476626.

%16
\bibitem{ExactRisk-1}
\Aue{Marron, J.\,S., S.~Adak, I.\,M.~Johnstone, M.\,H.~Neumann, and P.~Patil.} 1998. 
Exact risk analysis of wavelet regression. 
\textit{J.~Comput. Graph. Stat.} 7(3):278-309. doi: 10.1080/ 10618600.1998.10474777.

%17
\bibitem{Jansen-1}
\Aue{Jansen, M.} 2001. 
\textit{Noise reduction by wavelet thresholding}. Lecture notes in statistics ser. New York, NY: Springer Verlag. Vol.~161. 217~p.

%18
\bibitem{KuShe2016_1-1}
\Aue{Kudryavtsev, A.\,A., and O.\,V.~Shestakov.} 2016. 
Asymptotic behavior of the threshold minimizing the average probability of error in calculation of wavelet coefficients. 
\textit{Dokl. Math.} 93(3):295--299.
doi: 10.1134/S1064562416030212. EDN: WUMUEV. 

%19
\bibitem{KuShe2016_2-1}
\Aue{Kudryavtsev, A.\,A., and O.\,V.~Shestakov.} 2016. 
Asymptotically optimal wavelet thresholding in the models with non-Gaussian noise distributions. 
\textit{Dokl. Math.} 94(3):615--619.
doi: 10.1134/S1064562416060028. EDN: YUYVUP.




%20
\bibitem{Eroshenko-1}
\Aue{Eroshenko, A.\,A.} 2015. Statisticheskie svoystva otsenok signalov i~izobrazheniy pri porogovoy obrabotke ko\-ef\-fi\-tsi\-en\-tov 
v~veyvlet-razlozheniyakh 
[Statistical properties of signal and image estimates under thresholding of coefficients in wavelet decompositions]. Moscow: MSU. PhD Diss. 82~p.

%21
\bibitem{Peligrad-1}
\Aue{Peligrad, M.} 1996. 
On the asymptotic normality of sequences of weak dependent random variables. 
\textit{J. Theor. Probab.} 9(3):703--715. doi: 10.1007/BF02214083.

%22
\bibitem{Serfling2002-1}
\Aue{Serfling, R.\,J.} 2002. 
\textit{Approximation theorems of mathematical statistics}. New York, NY: John Wiley \&~Sons. 371~p.
\end{thebibliography}

 }
 }

\end{multicols}

\vspace*{-6pt}

\hfill{\small\textit{Received May 21, 2024}} 

%\vspace*{-18pt}

\Contr

\vspace*{-3pt}


\noindent
\textbf{Vorontsov Mikhail O.} (b.\ 1996)~--- PhD student, Department of Mathematical Statistics, 
Faculty of Computational Mathematics and Cybernetics, M.\,V.~Lomonosov Moscow State University, 1-52~Leninskie Gory, GSP-1, Moscow 119991, Russian Federation;  
mathematician, Moscow Center for Fundamental and Applied Mathematics, M.\,V.~Lomonosov Moscow State University, 1~Leninskie Gory, GSP-1, Moscow 119991, Russian Federation;
\mbox{m.vtsov@mail.ru}

\vspace*{6pt}

\noindent
\textbf{Shestakov Oleg V.} (b.\ 1976)~--- Doctor of Science in physics and mathematics, professor, Department of Mathematical Statistics,
 Faculty of Computational Mathematics and Cybernetics, M.\,V.~Lomonosov Moscow State University, 1-52~Leninskie Gory, GSP-1, Moscow 119991, Russian Federation; 
 senior scientist, Federal Research Center ``Computer Science and Control'' of the Russian Academy of Sciences, 44-2~Vavilov Str., Moscow 119333, 
 Russian Federation; leading scientist, Moscow Center for Fundamental and Applied Mathematics, M.\,V.~Lomonosov Moscow State University, 
 1~Leninskie Gory, GSP-1, Moscow 119991, Russian Federation; \mbox{oshestakov@cs.msu.su}


\label{end\stat}

\renewcommand{\bibname}{\protect\rm Литература}  %3
\def\stat{grusho}

\def\tit{АРХИТЕКТУРНЫЕ РЕШЕНИЯ В~ЗАДАЧЕ ВЫЯВЛЕНИЯ МОШЕННИЧЕСТВА ПРИ~АНАЛИЗЕ 
ИНФОРМАЦИОННЫХ ПОТОКОВ В~ЦИФРОВОЙ ЭКОНОМИКЕ$^*$}

\def\titkol{Архитектурные решения в~задаче выявления мошенничества при~анализе 
информационных потоков в
%~цифровой 
экономике}

\def\aut{А.\,А.~Грушо$^1$, М.\,И.~Забежайло$^2$, Н.\,А.~Грушо$^3$, 
Е.\,Е.~Тимонина$^4$}

\def\autkol{А.\,А.~Грушо, М.\,И.~Забежайло, Н.\,А.~Грушо, 
Е.\,Е.~Тимонина}

\titel{\tit}{\aut}{\autkol}{\titkol}

\index{Грушо А.\,А.}
\index{Забежайло М.\,И.}
\index{Грушо Н.\,А.}
\index{Тимонина Е.\,Е.}
\index{Grusho A.\,A.}
\index{Zabezhailo M.\,I.}
\index{Grusho N.\,A.}
\index{Timonina E.\,E.}


{\renewcommand{\thefootnote}{\fnsymbol{footnote}} \footnotetext[1]
{Работа частично поддержана РФФИ (проекты 18-29-03081 и~18-07-00274).}}


\renewcommand{\thefootnote}{\arabic{footnote}}
\footnotetext[1]{Институт проблем информатики Федерального исследовательского центра <<Информатика и~управление>> 
Российской академии наук, grusho@yandex.ru}
\footnotetext[2]{Институт проблем информатики Федерального исследовательского центра <<Информатика и~управление>> 
Российской академии наук, m.zabezhailo@yandex.ru}
\footnotetext[3]{Институт проблем информатики Федерального исследовательского центра <<Информатика и~управление>> 
Российской академии наук, info@itake.ru}
\footnotetext[4]{Институт проблем информатики Федерального исследовательского центра <<Информатика и~управление>> 
Российской академии наук, eltimon@yandex.ru}

\vspace*{-12pt}
   

 
  
  \Abst{Cформулирован подход к~исследованию некоторых видов мошенничества в~цифровой 
экономике с~использованием причинно-следственных связей. Во всех видах рассматриваемых 
мошенничеств должно наблюдаться несоответствие между целями финансовых транзакций 
и~реальной стоимостью достижения этих целей. Данные о транзакциях можно собирать, 
наблюдая информационные потоки, в~которых отражаются эти транзакции. Архитектура сбора 
данных и~их анализа может быть организована с~помощью распределенных реестров 
с~централизованным консенсусом, что позволяет создать аналог электронной бухгалтерской 
книги, фиксирующей финансово-экономическую деятельность субъектов цифровой экономики в~регионе. 
  Рассматриваемые методы выявления мошенничества основаны на противоречиях 
между действиями, описанными в~транзакциях, и~информацией, содержащейся в~планах, 
стандартах, прецедентах и~др. Рассмотрен метод, основанный на некоторой упрощенной схеме 
реализации абстрактного проекта. Для выявления противоречий необходимо проводить анализ 
от следствия к~причине, т.\,е.\ искать аномалии в~информации, описывающей порождение 
наблюдаемых следствий. 
  Показано, как в~реализации проекта можно выделять простые <<необходимые условия>> 
нарушения при\-чин\-но-след\-ст\-вен\-ных связей, т.\,е.\ множество <<необходимых условий>>, 
нарушение которых свидетельствует о наличии мошенничества. Это множество <<необходимых 
условий>> можно назвать метаданными для контроля проекта на выявление мошенничества.} 
 
 
  \KW{цифровая экономика; информационные потоки; при\-чин\-но-след\-ст\-вен\-ные связи; 
выявление мошеннических схем} 

\DOI{10.14357/19922264190204}
  
\vspace*{-4pt}


\vskip 10pt plus 9pt minus 6pt

\thispagestyle{headings}

\begin{multicols}{2}

\label{st\stat}

\section{Введение}

\vspace*{3pt}

  В работе сформулирован подход к~исследованию некоторых видов 
мошенничества в~цифровой экономике с~использованием  
при\-чин\-но-след\-ст\-вен\-ных связей. Рассматриваются три вида мошенничества, 
а именно:
  \begin{enumerate}[(1)]
\item отмыв денег; 
\item обман при выполнении договорных обязательств при реализации 
технических проектов (строительные проекты и~др.); 
\item незаконный вывод денег. 
\end{enumerate}

  Названные виды мошенничества могут быть сведены к~решению одного типа 
задач. Для отмывания денег источник должен заключать фиктивные контракты, 
в~соответствии с~которыми будут переводиться средства за заведомо ненужную 
работу и~материалы. 
  
  Мошенничество, связанное с~невыполнением договорных обязательств, связано 
со снижением качества услуг, качества и~количества закупаемых 
материалов, выполнением работ с~ненадлежащим качеством. 
  
  Вывод денег связан с~переводом средств фир\-мам-од\-но\-днев\-кам, которые 
заведомо не могут выполнить обязательства по контрактам, за которые им 
переводятся средства. 
  
  Таким образом, во всех трех видах рассматриваемых мошенничеств должно 
наблюдаться несоответствие между целями финансовых транзакций и~реальной 
стоимостью достижения этих целей. Данные о транзакциях можно собирать, 
наблюдая информационные потоки, в~которых отражаются эти транзакции. 
  
  Однако для наблюдения таких информационных потоков необходимо создавать 
архитектуру\linebreak телекоммуникационной системы, позволяющей перехватывать 
и~собирать данные о всех транзакциях. Например, такая архитектура может быть 
организована с~помощью распределенных реестров с~централизованным 
консенсусом, т.\,е.\ все информационные потоки, сформированные в~цифровой 
экономике и~несущие информацию о транзакциях, проходят через некоторый 
центральный узел, запоминающий их в~форме распределенного реестра. Такие 
реестры могут дублироваться в~аналогичных центрах различных регионов, что 
позволяет создать аналог электронной бухгалтерской книги, фиксирующей 
фи\-нан\-со\-во-эко\-но\-ми\-че\-скую деятельность субъектов цифровой экономики. Такой 
подход предложено реализовать на базе системы ситуационных центров, что 
отражено в~работах~[1, 2].
  
  Собранная из информационных потоков информация о~транзакциях, т.\,е.\ 
о~контрактах, договорах, платежах, отчетах, закупленных материалах, 
характеристиках исполнителей работ и~др., собирается в~базе данных в~указанном 
центре. Согласно теории интеллектуальных сис\-тем~[3], эту базу данных можно 
называть базой фактов (БФ). Базу фактов можно представить как бинарную мат\-ри\-цу, 
строки которой описывают характеристики, входящие в~транзакции, а столбцы 
нумеруются характеристиками. Строки матрицы будем называть 
\textit{объектами}~[4, 5]. 
  
  Рассматриваемые в~работе методы выявления мошенничества будут основаны 
на противоречиях между действиями, описанными в~транзакциях, и~информацией, 
содержащейся в~планах, стандартах, прецедентах и~др. Для нахождения 
противоречий в~архитектуре центра предусмотрена другая база данных~--- база 
знаний (БЗ)~\cite{3-gr, 6-gr}, которая устроена так же, как БФ. 
  
  Информация в~БЗ собирается на основе положительного опыта или расчетов. 
Используя БЗ, можно выводить факты нарушения при\-чин\-но-след\-ст\-вен\-ных 
связей. Нарушения при\-чин\-но-след\-ст\-вен\-ных связей будем называть 
\textit{аномалиями}. 
  
  Для упрощения дальнейшее изложение будет вестись в~рамках поиска 
противоречий при выполнении некоторого абстрактного проекта. Выявление 
аномалий будет происходить на основе фактов из БФ с~помощью знаний из БЗ 
методами искусственного интеллекта и~интеллектуального анализа 
данных~\cite{6-gr}. 

\vspace*{-10pt}
  
  \section{Модели}
  
  \vspace*{-3pt}
  
  Наиболее сложная из рассмотренных выше задач~--- выявление противоречий, 
т.\,е.\ использование БЗ для получения новых знаний и~выявление аномалий из 
полученных фактов. 
  
  Все способы выявления противоречий основаны на определении 
  причинно-следственных связей. При этом противоречия в~параметрах транзакций по 
отношению к~требуемым в~БЗ составляют сущность аномалий. 
  
   Далее будет рассмотрен метод, основанный на некоторой упрощенной схеме 
реализации абстрактного проекта. 
  
  Каждый проект имеет цель: например, цель представляет собой построение 
некоторой системы. Воспользуемся структурным подходом, который позволяет 
строить проект на основе разбиения системы на подсистемы и~определения 
взаимодействий подсистем~\cite{7-gr}. При этом каждая подсистема также 
представима структурной моделью. 
  
  Как сама система, так и~каждая ее подсистема имеют свой функционал 
и~спецификацию, па\-ра\-мет\-ры настройки и~домены параметров настройки. Кроме 
этих характеристик существует множество характеристик, связанных 
с~<<жизненным циклом>> создания системы. Сюда входят работы, ресурсы, 
сроки выполнения работ по созданию подсистем и~самой системы, стоимости 
компонентов и~материалов, стоимости работ, схемы поставок, договорные 
обязательства и~др. Все характеристики связаны между собой, поэтому можно 
говорить о стоимости и~времени изготовления структурных компонентов системы. 
  
  Одной из важнейших характеристик является смета (система смет для 
подсистем). Смета сопоставляет каждому компоненту системы стоимость его 
изготовления и~настройки. 
  
  Схема построения системы может быть пред\-став\-ле\-на диаграммой, 
изображенной на рис.~1. 

{ \begin{center}  %fig1
 \vspace*{9pt}
   \mbox{%
 \epsfxsize=79mm 
 \epsfbox{gru-1.eps}
 }


\vspace*{9pt}


\noindent
{{\figurename~1}\ \ \small{Диаграмма достижения цели}}
\end{center}
}

\vspace*{9pt}

\addtocounter{figure}{1}
  
  


  Представленная на рис.~1 диаграмма позволяет описать основные классы 
возможных противоречий при достижении цели. Противоречия возникают, когда 
данные БФ не соответствуют требуемым характеристикам. 
  
  
  \section{Потенциальные классы аномалий при~достижении цели}
  
  Выделим четыре потенциальных класса противоречий, которые показывают, 
каким образом нужно искать эти противоречия.
  
 
  Противоречие цели и~проекта (рис.~2) возникает при отсутствии обоснования 
или в~случае логического противоречия между возможностями проектируемого 
функционала и~целью системы. Отметим, что в~проект входят сроки, перечень 
работ, материалы, настройки, которые описываются соответствующими 
параметрами и~допустимыми значениями этих параметров. Проект формируется 
на основе БЗ и~расчетов, исходя из информации, полученной по аналогии 
с~другими проектами и~решениями, которые считаются апробированными. 
  
  Отметим, что цель порождает проект и~в этом смысле является причиной 
проекта. Однако для анализа противоречий необходимо двигаться по штриховой 
стрелке диаграммы (см.\ рис.~2) от проекта к~цели. В~самом деле, любой компонент 
проекта направлен на теоретическое достижение цели. Цель~--- сложный объект, 
поэтому в~проекте могут возникнуть характеристики, противоречащие хотя бы 
некоторым характеристикам цели. Это делает проект противоречивым, но вывод 
об этом может быть сделан только на уровне описания цели. 
  

  Противоречия между проектом и~его реализацией, исключая настройки 
(рис.~3), могут возникать, например, при закупке исполнителем материалов более 
низкого качества по более низким ценам, при попытках достижения требуемых 
сроков работы за счет снижения качества выполнения работ, за счет нахождения 
<<объективных>> причин для увеличения сроков работы и,~следовательно, 
увеличения цены реализации проекта. 


  Для выявления указанных противоречий необходимо двигаться по диаграмме 
(см.\ рис.~3) в~обратную сторону в~соответствии со~штриховыми стрелками. 
Действительно, выявить противоречия между характеристиками закупленных 
материалов и~требуемыми по проекту можно только при обращении к~проекту 
и~его спецификациям. Манипуляции со сроками работы также можно выявить 
только при обращении к~соответствующим расчетам в~проекте. Задержки в~сроках 
работы, связанные с~поставками материалов, можно определить только на 
предыдущем этапе диаграммы (см.\ рис.~3) в~описании проекта. 


  


  Противоречия между реализацией проекта и~его настройкой (рис.~4) возникает, 
когда не удается добиться требуемых значений параметров функционала, не 
удается обеспечить необходимый уровень\linebreak\vspace*{-12pt}

{ \begin{center}  %fig2
 \vspace*{-6pt}
   \mbox{%
 \epsfxsize=16mm 
 \epsfbox{gru-2.eps}
 }


\vspace*{6pt}


\noindent
{{\figurename~2}\ \ \small{Противоречия цели и~проекта}}
\end{center}
}

%\vspace*{9pt}

\addtocounter{figure}{1}

{ \begin{center}  %fig3
 \vspace*{6pt}
    \mbox{%
 \epsfxsize=79mm 
 \epsfbox{gru-3.eps}
 }


\end{center}

\vspace*{-2pt}


\noindent
{{\figurename~3}\ \ \small{Противоречия проекта и~его реализации (без настройки)}}
}

\vspace*{6pt}

\addtocounter{figure}{1}

{ \begin{center}  %fig4
 \vspace*{1pt}
   \mbox{%
 \epsfxsize=54.5mm 
 \epsfbox{gru-4.eps}
 }


\end{center}


\noindent
{{\figurename~4}\ \ \small{Противоречия реализации проекта и~его на\-стройки}}
}

%\vspace*{9pt}

\addtocounter{figure}{1}

{ \begin{center}  %fig5
 \vspace*{5pt}
    \mbox{%
 \epsfxsize=79mm 
 \epsfbox{gru-5.eps}
 }


\end{center}



\noindent
{{\figurename~5}\ \ \small{Противоречия цели и~достигнутой реализации проекта}}
}

\vspace*{6pt}

\addtocounter{figure}{1}

\noindent
 качества реализации проекта. Для 
определения противоречия в~настройках надо опять же двигаться по диаграмме 
(см.\ рис.~4) в~обратную сторону по штриховым стрелкам, так как для выявления 
характеристик результатов работы, которые не дают возможности реализации 
определенного функционала, необходимо иметь информацию о результатах этой 
работы. 


  



  Противоречие между целью и~достигнутой реализацией проекта (рис.~5) 
возникает, когда реализованная система не позволяет достичь цели. В~этом случае 
опять противоречие нужно искать, двигаясь от цели к~реальному достигнутому 
функционалу по штриховой стрелке (см.\ рис.~5).
  
  Суммируя положения, изложенные в~данном разделе, приходим к~выводу, что 
для выявления противоречий необходимо проводить анализ от следствия 
к~причине, т.\,е.\ искать аномалии в~информации, описывающей порождение 
наблюдаемых следствий. 
  
  
  \section{Связь противоречий и~причин}
  
  Прежде чем построить связь между причинами и~противоречиями, кратко 
опишем простейшую модель связи этих понятий. Причины и~противоречия будут 
сформулированы для представления компонентов системы как объектов, 
обладающих наборами известных характеристик~\cite{4-gr, 5-gr}. 
  
  Пусть $U\hm=\{\alpha, \beta, \ldots\}$~--- совокупность характеристик 
(пространство характеристик). Согласно~\cite{4-gr} \textit{объектом}~$O$ 
называется любое подмножество характеристик $O\hm\subseteq U$. Рассмотрим 
последовательность объектов, возможно в~различных пространствах 
характеристик. 
  
  \smallskip
  
  \noindent
  \textbf{Определение~1.}\ Объект~$P$ с~числом характеристик, большим или 
равным~2, является \textit{причиной} объекта (\textit{свойства})~$B$ в~цепочке 
наблюдаемых объектов тогда и~только тогда, когда выполнены следующие 
условия:
  \begin{enumerate}[(1)]
\item для каждого объекта~$C$, если $P\hm\subseteq C$, то $C\mapsto B$, где 
$C\mapsto B$ означает, что объект~$B$ присутствует в~объекте, следующем за 
объектом~$C$;
\item объект~$P$ является минимальным объектом, удовлетворяющим 
условию~1, а~именно: $\forall \alpha\hm\in P$ объект~$P\backslash \{\alpha\}$ 
не является причиной, т.\,е.\ $\exists C:\ \alpha\not\in C$, $P\backslash 
\{\alpha\}\hm\subseteq C$ и~$C\not\mapsto B$, где $C\not\mapsto B$ означает, 
что~$B$ не может содержаться в~объекте, следующем за объектом~$C$. 
\end{enumerate}

  Приведенное определение причины является упрощением причин, 
возникающих в~реальном мире. Например, реальные причины могут возникать\linebreak 
как совокупность характеристик из разных пространств. Одно следствие может 
порождаться разными причинами или возникать из внешних\linebreak и~ненаблюдаемых 
характеристик. Однако пред\-став\-лен\-ная далее формализация позволяет доступно 
изложить при\-чин\-но-след\-ст\-вен\-ные истоки противоречий, которые 
инициируют в~дальнейшем глубокое исследование рассматриваемых процессов.
  
  Будем считать, что для любого интересующего нас свойства~$B$ существует 
причина. Тогда справедлива следующая теорема.
  
  \smallskip
  
  \noindent
  \textbf{Теорема~1.}\ \textit{Для любого свойства~$B$ существует 
единственная причина}. 
  
  \smallskip
  
  \noindent
  Д\,о\,к\,а\,з\,а\,т\,е\,л\,ь\,с\,т\,в\,о\,.\ \ Доказательство будем вести от противного, 
т.\,е.\ предположим, что существуют две причины свойства~$B$: $P$ 
и~$P^\prime$, $P\hm\not= P^\prime$. Тогда существует $\alpha\hm\in U$, которое 
удовлетворяет одному из двух условий:
  \begin{itemize}
\item[(а)] $\alpha\in P$, $\alpha\notin P^\prime$;
\item[(б)] $\alpha\notin P$, $\alpha \in P^\prime$.
\end{itemize}

  Пусть выполняется условие~(б). Тогда $P^\prime\backslash \{\alpha\}$ не 
является причиной по условию~2 определения~1, т.\,е.\ $\exists C$ такое, что 
$\alpha\notin C$, $P^\prime\backslash \{\alpha\}\hm\subseteq C$ и~$C\not\mapsto B$. 
Но если~$B$ произошло и~$P$ его причина, то $C\mapsto B$, что противоречит 
предположению. Теорема~1 доказана.
  
  \smallskip
  
  \noindent
  \textbf{Лемма.} \textit{Если $P$~--- причина появления свойства~$B$, то 
объект~$B$ определяет существование свойства~$P$ в~объекте, 
предшествующем~$B$. }
  
  \smallskip
  
  \noindent
  Д\,о\,к\,а\,з\,а\,т\,е\,л\,ь\,с\,т\,в\,о\,.\ \ Из предположения, что у~каж\-до\-го 
свойства~$B$ есть причина, и~условия, что~$P$ является причиной~$B$, следует, 
что при появлении в~данных свойства~$B$ объект~$C$, предшествующий 
появлению~$B$, содержит как часть объект~$P$. Это следует из теоремы~1 
и~определения причины. 
  
  Докажем принцип <<необходимого условия>>, который, несмотря на простоту 
доказательства, будет играть в~дальнейшем существенную роль.
  
  \smallskip
  
  \noindent
  \textbf{Теорема~2.} \textit{Если~$P$~--- причина появления свойства~$B$ 
и~$A\hm\subseteq P$, то объект~$B$ определяет наличие свойства~$A$ 
в~объекте, предшествующем~$B$}. 
  
  \smallskip
  
  \noindent
  Д\,о\,к\,а\,з\,а\,т\,е\,л\,ь\,с\,т\,в\,о\,.\ \ Пусть в~данных имеется объект~$B$ 
и~$P\mapsto B$, тогда в~силу существования и~единственности причины~$B$ 
в~данных должен существовать объект~$C$, предшествующий~$B$ 
и~содержащий причину~$P$. Поскольку $A\hm\subseteq P$ и~$B$ содержит 
причину~$P$, то $B\mapsto A$. С~учетом леммы теорема~2 доказана.
  
  \smallskip
  
  Пусть даны пространства $U_1, U_2,\ldots$ и~имеется последовательность 
данных (процесс выполнения этапов проекта в~соответствии с~рис.~1) $A, B, 
\ldots$, где каждый объект является подмножеством некоторого 
пространства~$U_i$, $i\hm=1,\ldots$ Тогда в~объекте~$A$ присутствует 
причина~$P$ появления интересующего нас свойства~$C$ в~объекте~$B$. Пусть 
$P\hm\subseteq A$, тогда по теореме~2 $\forall \alpha\hm\in P$:  
$C\mapsto \{\alpha\}$, т.\,е.\ из появления~$C$ следует появление 
характеристики~$\alpha$ в~предшествующем объекте. Это необходимое условие 
того, что~$C$ удовлетворяет причинно-следственным связям развития процесса 
выполнения проекта. Если для~$C$ нет характеристики~$\alpha$, которую можно 
отнести к~причине~$C$, то можно считать, что нарушена  
при\-чин\-но-след\-ст\-вен\-ная связь и~$C$~--- аномальный объект. 
  
  \smallskip
  
  \noindent
  \textbf{Пример.} Если объект~$C$ состоит в~получении суммы~$a$ 
фирмой~$K$, то согласно теореме~2 в~пред\-шест\-ву\-ющем объекте должна 
существовать причина перевода суммы~$a$ на фирму~$K$. Если эта причина 
в~проекте отсутствует, то это можно считать признаком мошеннической схемы. 
Все проекты по предположению собираются из <<кубиков>>, содержащихся в~БЗ. 
Тогда можно сравнить цену объекта~$C$, породившего получение суммы~$a$, 
и~сумму, присутствующую в~смете проекта. Если разница велика, то это либо 
ошибка проекта, либо признак мошеннической схемы.
  
  \section{Поиск противоречий на~основе~принципа <<необходимых~условий>>}
   
  Как было показано в~разд.~3, нахождение противоречий соответствуют 
движению от следствия к~причине. Для каждого объекта в~наблюдаемых данных 
выявление причин его появления является трудоемкой задачей. Кроме того, при 
реализации контроля соблюдения при\-чин\-но-след\-ст\-вен\-ных связей на 
большом множестве участников экономической деятельности задача анализа 
причин становится трудоемкой. Поэтому процедуру контроля необходимо разбить 
на два этапа, где первый этап состоит в~анализе простых <<необходимых 
условий>> проявления мошенничества, когда используется хотя бы одна 
известная характеристика причины. Второй этап (в~режиме офлайн) состоит 
в~выявлении причин, позволяющих провести анализ источников мошеннических 
схем. 
  
  Один из подходов к~выбору <<необходимых условий>> состоит в~построении 
множества подцелей исходной цели проекта (структурный метод построения 
проекта~\cite{7-gr}). Каждая подцель описывается диаграммой на рис.~1, 
и~реализации подцелей должны образовывать полный функционал цели. Это 
является необходимым, но не достаточным условием достижения цели, так как 
при таком подходе отсутствует компонент согласования всех подцелей в~единую 
систему. Однако такой подход значительно упрощает анализ выполнения проекта 
на предмет поиска мошенничества. Если признаки мошенничества будут 
обнаружены в~реализации хотя бы одной из подцелей, то это значит, что 
мошенничество присутствует в~реализации всего проекта. 
  
  Аналогично в~реализации каждого этапа в~любой из подцелей можно выделять 
простые <<необходимые условия>> нарушения при\-чин\-но-след\-ст\-венн\-ых 
связей. 
  
  Таким образом, получается множество <<необходимых условий>>, нарушение 
которых свидетельствует о наличии мошенничества. Это множество 
<<необходимых условий>> можно назвать метаданными~[8, 9] для контроля 
проекта на выявление мошенничества. 
  
  
  \section{Заключение }
  
  В поиске противоречий необходимо от транзакций, соответствующих 
следствиям при\-чин\-но-след\-ст\-вен\-ных связей, переходить к~анализу причин 
наблюдаемых следствий. Это сложная задача, которая связана с~описанием причин 
определенных свойств. 
  
  В работе представлена модель, позволяющая строить множество необходимых 
условий соответствия наблюдаемого следствия вызвавшей его причине. Этот 
подход делает поиск противоречий вполне вычислимой задачей, но не гарантирует 
успех. 
  
  {\small\frenchspacing
 {%\baselineskip=10.8pt
 \addcontentsline{toc}{section}{References}
 \begin{thebibliography}{9}
\bibitem{1-gr}
\Au{Грушо А.\,А., Зацаринный~А.\,А., Тимонина~Е.\,Е.} Блокчейны цифровой экономики на базе 
системы ситуационных центров и~централизованного консенсуса~// Радиолокация, навигация, 
связь: Мат-лы XXV Междунар. научн.-технич. конф.~---
Воронеж: Издательский дом ВГУ, 2019. Т.~6. С.~183--191. 
\bibitem{2-gr}
\Au{Grusho A., Zatsarinny~A., Timonina~E.} A~system approach to information security in 
distributed ledgers on the situational centers platform.~---
Lecture notes in computer science ser.~--- Springer, 2019 
(in press).
\bibitem{3-gr}
\Au{Финн В.\,К.} Искусственный интеллект: Методология, применения, философия.~--- М.: 
Красанд, 2011. 448~с.

\bibitem{5-gr} %4
\Au{Аншаков~О.\,М., Фабрикантова~Е.\,Ф.} ДСМ-ме\-тод автоматического порождения 
гипотез: Логические и~эпистемологические основания.~--- М.: Либроком, 2009. 432~с.

\bibitem{4-gr} %5
\Au{Poelmans J., Elzinga~P., Viaene~S., Dedene~G.} Formal concept analysis in knowledge 
discovery: A~survey~// Conceptual structures: From information to intelligence~/ Eds.\ M.~Croitoru, 
S.~Ferr$\acute{\mbox{e}}$, and D.~Lukose.~--- Lecture notes in computer science 
ser.~--- Berlin--Heidelberg: Springer, 2010. Vol.~6208.  P.~139--153.

\bibitem{6-gr}
\Au{Панкратова~Е.\,С., Финн~В.\,К.} Автоматическое по\-рож\-де\-ние гипотез в~интеллектуальных 
системах.~--- М.: Либроком, 2009. 528~с. 
\bibitem{7-gr}
\Au{Денисов А.\,А., Колесников~Д.\,Н.} Теория больших систем управления.~--- Л.: Энергоиздат, 1982. 488~с.

\bibitem{9-gr}
\Au{Грушо А.\,А., Грушо Н.\,А., Забежайло~М.\,И., Смирнов~Д.\,В., Тимонина~Е.\,Е.} 
Параметризация в~прикладных задачах поиска эмпирических причин~// Информатика и~её 
применения, 2018. Т.~12. Вып.~3. С.~62--66.

\bibitem{8-gr}
\Au{Грушо А.\,А., Грушо Н.\,А., Левыкин~М.\,В., Тимонина~Е.\,Е.} Методы идентификации 
захвата хоста в~распределенной ин\-фор\-ма\-ци\-он\-но-вы\-чис\-ли\-тель\-ной сис\-те\-ме, 
защищенной с~помощью метаданных~// Информатика и~её применения, 2018. Т.~12. Вып.~4. 
С.~41--45.

 \end{thebibliography}

 }
 }

\end{multicols}

\vspace*{-3pt}

\hfill{\small\textit{Поступила в~редакцию 03.04.19}}

%\vspace*{8pt}

%\pagebreak

\newpage

\vspace*{-28pt}

%\hrule

%\vspace*{2pt}

%\hrule

%\vspace*{-2pt}

\def\tit{ARCHITECTURAL DECISIONS IN~THE~PROBLEM 
OF~IDENTIFICATION OF~FRAUD IN~THE~ANALYSIS 
OF~INFORMATION FLOWS IN~DIGITAL ECONOMY\\[-5pt]}


\def\titkol{Architectural decisions in~the~problem 
of~identification of~fraud in~the~analysis 
of~information flows in~digital economy}

\def\aut{A.\,A.~Grusho, M.\,I.~Zabezhailo, N.\,A.~Grusho, and~E.\,E.~Timonina}

\def\autkol{A.\,A.~Grusho, M.\,I.~Zabezhailo, N.\,A.~Grusho, and~E.\,E.~Timonina}

\titel{\tit}{\aut}{\autkol}{\titkol}

\vspace*{-13pt}


 \noindent
   Institute of Informatics Problems, Federal Research Center ``Computer Sciences and 
Control'' of the Russian Academy of Sciences; 44-2~Vavilov Str., Moscow 119133, 
Russian Federation

\def\leftfootline{\small{\textbf{\thepage}
\hfill INFORMATIKA I EE PRIMENENIYA~--- INFORMATICS AND
APPLICATIONS\ \ \ 2019\ \ \ volume~13\ \ \ issue\ 2}
}%
 \def\rightfootline{\small{INFORMATIKA I EE PRIMENENIYA~---
INFORMATICS AND APPLICATIONS\ \ \ 2019\ \ \ volume~13\ \ \ issue\ 2
\hfill \textbf{\thepage}}}

\vspace*{3pt}


   
     
   \Abste{An approach to a~research of some types of fraud in digital economy with the usage of relationships of 
cause and effect is formulated. In all types of the considered frauds, the discrepancy between the 
purposes of financial transactions and actual cost of achievement of these purposes
has to be observed. Data on 
transactions can be collected by observing information flows in which these transactions are reflected. 
The architecture of data collection and their analysis can be organized by means of the distributed 
ledgers with the centralized consensus that allows creating an analog of the electronic account book 
fixing financial and economic activity of subjects of digital economy in the region. 
   The methods of fraud identification considered are based on the contradictions 
between actions described in transactions and information, which is contained in plans, standards, 
precedents, etc. 
   The method based on a~simplified scheme of implementation of the abstract project is considered. 
For identification of contradictions, it is necessary to carry out the analysis from the effect to the cause, 
i.\,e., to look for anomalies in information describing the generation of the observed effects. 
   It is shown how in implementation of the project it is possible to allocate simple ``necessary 
conditions'' of violation of cause and effect relationships, i.\,e., a~set of ``necessary conditions'' 
violation of which demonstrates fraud existence. It is possible to call this set of "necessary conditions" 
by metadata for control of the project for fraud identification.} 
   
   \KWE{digital economy; information flows; relationships of reason and effect; detection of 
fraudulent schemes}
   
  

 \DOI{10.14357/19922264190204}

\vspace*{-20pt}

 \Ack
   \noindent
   The work was partially supported by the Russian Foundation for Basic Research (projects  
18-29-03081 and 18-07-00274).



%\vspace*{6pt}

  \begin{multicols}{2}

\renewcommand{\bibname}{\protect\rmfamily References}
%\renewcommand{\bibname}{\large\protect\rm References}

{\small\frenchspacing
 {\baselineskip=10.5pt
 \addcontentsline{toc}{section}{References}
 \begin{thebibliography}{9}
\bibitem{1-gr-1}
\Aue{Grusho, A.\,A., A.\,A.~Zatsarinny, and E.\,E.~Timonina.} 2019. Blokcheyny tsifrovoy ekonomiki 
na baze sistemy situatsionnykh tsentrov i~tsentralizovannogo konsensusa [Blockchains of digital 
economy on the basis of the system of the situational centres and the centralized consensus]. 
\textit{25th Scientific and Technical Conference (International) ``Radar-Location, Navigation, 
Communication'' Proceedings}. Voronezh: VSU Publs. 6:183--191.
\bibitem{2-gr-1}
\Aue{Grusho, A., A.~Zatsarinny, and E.~Timonina.} 2019 (in press). 
A~system approach to information security 
in distributed ledgers on the situational centers platform. 
Lecture notes in computer science ser. Springer.
\bibitem{3-gr-1}
\Aue{Finn, V.\,K.} 2011. \textit{Iskusstvennyy intellekt: Metodologiya, primeneniya, filosofiya} 
[Artificial intelligence: Methodology, applications, philosophy]. Moscow: KRASAND. 448~p.

\bibitem{5-gr-1}
\Aue{Anshakov, O.\,M., and E.\,F.~Fabrikantova}. 2009. \textit{DSM-metod avtomaticheskogo porozhdeniya gipotez: Logicheskie 
i~epistemologicheskie osnovaniya} [JSM-method of automatic hypothesis generation: Logical and 
epistemological]. Moscow: KD LIBROKOM. 432~p.
\bibitem{4-gr-1} %5
\Aue{Poelmans, J., P.~Elzinga, S.~Viaene, and G.~Dedene.} 2010. Formal concept analysis in 
knowledge discovery: A~survey. \textit{Conceptual structures: From information to intelligence}. 
Eds.\ M.~Croitoru, S.~Ferr$\acute{\mbox{e}}$, and D.~Lukose. Lecture notes in 
computer science ser. Berlin--Heidelberg: Springer. 6208:139--153.

\bibitem{6-gr-1}
\Aue{Pankratov, E.\,S., and V.\,K.~Finn}. 
2009. \textit{Avtomaticheskoe porozhdenie gipotez v~intellektual'nykh 
sistemakh} [Automatic hypotheses generation in intelligent systems]. Moscow: KD 
\mbox{LIBROKOM}.  528~p. 
\bibitem{7-gr-1}
\Aue{Denisov, A.\,A., and D.\,N.~Kolesnikov.} 1982. \textit{Teoriya bol'shikh 
sistem upravleniya} [Theory of big control systems]. Leningrad: Energoizdat. 488~p.

\bibitem{9-gr-1}
\Aue{Grusho, A.\,A., N.\,A.~Grusho, M.\,I.~Zabezhailo, D.\,V.~Smirnov, and 
E.\,E.~Timonina.} 2018. 
Parametrizatsiya v~prikladnykh zadachakh poiska empiricheskikh prichin 
[Parametrization in applied 
problems of search of the empirical reasons]. 
\textit{Informatika i~ee Primeneniya~--- 
Inform. Appl.} 12(3):62--66.

\bibitem{8-gr-1}
\Aue{Grusho, A.\,A., N.\,A.~Grusho, M.\,V.~Levykin, and E.\,E.~Timonina.} 2018. Metody 
identifikatsii zakhvata khosta v~raspredelennoy informatsionno-vychislitel'noy sisteme, 
zashchishchennoy s~pomoshch'yu metadannykh [Methods of identification of host capture 
in the  distributed information system which is protected on the base of meta data].
\textit{Informatika i~ee 
Primeneniya~--- Inform. Appl.} 12(4):41--45.
{ %\looseness=1

}

\end{thebibliography}

 }
 }

\end{multicols}

\vspace*{-12pt}

\hfill{\small\textit{Received April 3, 2019}}

%\pagebreak

%\vspace*{-18pt}

\Contr

\noindent
\textbf{Grusho Alexander A.} (b.\ 1946)~--- Doctor of Science in physics and 
mathematics, professor, principal scientist, Institute of Informatics Problems, 
Federal Research Center ``Computer Sciences and Control'' of the Russian 
Academy of Sciences; 44-2~Vavilov Str., Moscow 119133, Russian Federation; 
\mbox{grusho@yandex.ru} 

\vspace*{3pt}

\noindent
\textbf{Zabezhailo Michael I.} (b.\ 1956)~--- Doctor of Science in physics and 
mathematics, principal scientist, Institute of Informatics Problems, Federal Research 
Center ``Computer Sciences and Control'' of the Russian Academy of Sciences;  
44-2~Vavilov Str., Moscow 119133, Russian Federation; 
\mbox{m.zabezhailo@yandex.ru} 

\vspace*{3pt}


\noindent
\textbf{Grusho Nikolai A.} (b.\ 1982)~--- Candidate of Science (PhD) in physics 
and mathematics, senior scientist, Institute of Informatics Problems, Federal 
Research Center ``Computer Sciences and Control'' of the Russian Academy of 
Sciences; 44-2~Vavilov Str., Moscow 119133, Russian Federation; 
\mbox{info@itake.ru} 

\vspace*{3pt}


\noindent
\textbf{Timonina Elena E.} (b.\ 1952)~--- Doctor of Science in technology, 
professor, leading scientist, Institute of Informatics Problems, Federal Research 
Center ``Computer Sciences and Control'' of the Russian Academy of Sciences;  
44-2~Vavilov Str., Moscow 119133, Russian Federation; 
\mbox{eltimon@yandex.ru} 

\label{end\stat}

\renewcommand{\bibname}{\protect\rm Литература}   %4
\newcommand{\Cal}{\mathcal}
\newcommand{\Qo}{\mathbb{Q}_{\ge 0}}
\newcommand{\NOT}{\operatorname{\mathtt{NOT}}}
\newcommand{\Ro}{\mathbb{R}_{\ge 0}}
\newcommand{\XOR}{\mathbin{\mathtt{XOR}}}
\newcommand{\OR}{\mathbin{\mathtt{OR}}}
\newcommand{\AND}{\mathbin{\mathtt{AND}}}
\newcommand{\EXP}{\mathbin{\uparrow}}
\newcommand{\DIV}{\mathbin{/\!/}}
\newcommand{\fac}{\mathbin{\!/\!}}


\def\stat{anashin}

\def\tit{О ТЕОРЕТИКО-АВТОМАТНЫХ МОДЕЛЯХ БЛОКЧЕЙН-СРЕДЫ$^*$}

\def\titkol{О теоретико-автоматных моделях блокчейн-среды}

\def\aut{В.\,С.~Анашин$^1$}

\def\autkol{В.\,С.~Анашин}

\titel{\tit}{\aut}{\autkol}{\titkol}

\index{Анашин В.\,С.}
\index{Anashin V.\,S.}


{\renewcommand{\thefootnote}{\fnsymbol{footnote}} \footnotetext[1]
{Работа выполнена при
поддержке РФФИ (проект 18-20-03124).}}


\renewcommand{\thefootnote}{\arabic{footnote}}
\footnotetext[1]{Факультет вычислительной математики и~кибернетики Московского
государственного университета   им.\
М.\,В.~Ломоносова, \mbox{anashin@iisi.msu.ru}}

\vspace*{-8pt}




\Abst{Рассматриваются методы анализа и~моделирования блокчейн-среды, основанные
на тео\-ре\-ти\-ко-ав\-то\-мат\-ных моделях, в~первую очередь на 
так называемых <<автоматах с~метками
времени>> (timed automata). Также предлагается новая версия автоматов с~метками времени, позволяющая избежать некоторых неудобств моделирования с~помощью классических автоматов с~метками времени, а при моделировании блок\-чейн-сре\-ды
на основе последних приходится использовать
переменные разных типов, действительные и~булевы, что вызывает ряд сложностей
как теоретического, так и~практического характера. Предлагаемый подход основан
на применении 2-адического анализа, что дает возможность использовать переменные
одного и~того же типа, а именно булева.}


\KW{блокчейн-среда; смарт-контракт; автомат
с~метками времени}

\DOI{10.14357/19922264190205}
  
\vspace*{6pt}


\vskip 10pt plus 9pt minus 6pt

\thispagestyle{headings}

\begin{multicols}{2}

\label{st\stat}

\section{Введение}
%\label{sec:intro}

При построении математических моделей\linebreak
 блок\-чейн-сре\-ды тео\-ре\-ти\-ко-ав\-то\-мат\-ная 
модель\linebreak
воз\-ни\-ка\-ет естественным образом, поскольку функ\-ци\-о\-ни\-ро\-ва\-ние 
блок\-чейн-сре\-ды~--- это детерминированный
процесс, в~ходе которого решение 
о~включении или невключении блока в~реестр  зависит как
от предыдущих блоков (булев тип данных), так и~от времени (тип данных~---
действительные чис\-ла). Таким образом,  если рассматривать текущее состояние
реестра как предыдущее состояние, а~состояние реестра сразу после включения
в него очередного блока как последующее состояние, то процесс
изменения содержимого реестра
можно описать с~по\-мощью понятия
<<автомат с~метками времени>> (timed automaton), для которого далее 
в~текс\-те статьи используется термин T-ав\-то\-мат
или кратко~--- TA.

Основы  математической теории ТА были заложены в~работе~\cite{timed-auto}.
В~плане математических моде\-лей\linebreak блок\-чейн-сре\-ды
на основе ТА пред\-став\-ляет\linebreak
интерес работа~\cite{Bitcoin-contract-model}, поскольку в~ней мо\-де\-ли\-ру\-ет\-ся
функционирование смарт-кон\-трак\-тов в~бит\-койн-сре\-де (последняя
пред\-став\-ля\-ет собой част\-ный случай блок\-чейн-сре\-ды), что является одним из наиболее
важных моментов для полноценного функционирования цифровой экономики на основе блок\-чейн-сре\-ды 
(а~не только для  реализации криптовалют). Отметим, что в~этой работе для
моделирования смарт-кон\-трак\-тов использован
язык Uppaal, разработанный  для описания и~верификации моделей, основанных
на ТА (см.\ описание \mbox{Uppaal} в~\cite{Uppaal-tutorial}). На основе этой же модели
описывается и~бит\-койн-сре\-да~\cite{Model-bitcoin-uppaal}. Подчеркнем,
что с~точки зрения математического описания и~моделирования любой юридически
правильно составленный
контракт можно рассматривать как конечный автомат 
(см., например,~\cite{contract-automat}).
Более того, известны
методы извлечения описания смарт-контрактов как конечных автоматов из функционирующей
во времени
блок\-чейн-среды~\cite{mining-smart-contract}.

Сказанное означает, что функционирование смарт-кон\-трак\-тов в~блок\-чейн-сре\-де можно
рас\-смат\-ри\-вать как взаимодействие автоматов во времени,  т.\,е.\ ТА можно рассматривать
как релевантную модель  описания такого взаимодействия.

Отметим, что  моделирование функционирования смарт-кон\-трак\-тов в~блок\-чейн-сре\-де 
является одним из важных методов проверки стойкости  против компрометации. 
Смарт-кон\-тракт использует входные данные от других смарт-кон\-трак\-тов,
пользователей, а~также о~текущем времени и~выдает  выходные данные, которые
используются другими пользователями и/или другими смарт-кон\-трак\-та\-ми, а~потому
ошибочное функционирование одного смарт-кон\-трак\-та может привести к~сбою 
в~работе всех связанных с~ним смарт-кон\-трак\-тов и~узлов сети. Однако смарт-кон\-тракт
может быть очень сложно устроен даже уже на уровне юридического документа,
не говоря уже о его программной реализации, поэтому требуется тщательная
проверка правильного функционирования смарт-кон\-трак\-та\linebreak
 и~в~юридическом плане,
и как компьютерной программы. Такую проверку достаточно сложно выполнить
вручную, однако моделирование смарт-кон\-трак\-та позволяет поставить ряд машинных\linebreak
экспериментов для изучения поведения смарт-кон\-трак\-та как автомата при подаче
на него тех или иных входных данных, т.\,е.\ смоделировать его поведение в~самых
разных условиях. Именно эти соображения послужили мотивацией
 работы~\cite{Bitcoin-contract-model}, в~которой используется 
 тео\-ре\-ти\-ко-ав\-то\-мат\-ная модель
смарт-кон\-трак\-та на основе ТА, где время представляется действительными чис\-лами.



 В данной работе
предлагается принципиально иной подход к~построению 
тео\-ре\-ти\-ко-ав\-то\-мат\-ных моделей  функционирования блок\-чейн-сре\-ды 
(в~частности,  функционирования
смарт-кон\-трак\-тов в~этой среде) как автомата
с временн$\acute{\mbox{ы}}$ми метками,  в~котором физическое время 
представляется не действительными, а~2-ади\-че\-ски\-ми числами.
Такой подход представляется автору оправданным и~плодотворным по
нескольким причинам:
\begin{itemize}
\item итоговая модель представляет собой автомат в~стандартном понимании
этого термина;
при этом задаваемое
автоматом преобразование слов  реализуется не в~виде таблицы переходов
состояний, а~в~виде программы без вет\-вле\-ния, пред\-став\-ля\-ющей собой
последовательность стандартных команд процессора,\linebreak а~именно:
арифметических и~поразрядных логических операций;
\item поскольку полученный автомат является <<обычным>> автоматом, описание
его функционирования может быть сведено к~изучению функции, заданной и~принимающей
значения в~пространстве целых 2-ади\-че\-ских чисел ввиду того, что каждая детерминированная
(по C.\,В.~Яблонскому)
функция (т.\,е.\ функция, задаваемая автоматом) есть
$p$-ади\-че\-ская функция,
удовлетворяющая  условию Липшица с~константой~1 относительно подходящей $p$-ади\-че\-ской
метрики, и~обратно: все такие функции являются детерминированными 
(см., например,~\cite{me:Discr_Syst});
\item сказанное дает возможность применять к~изуче\-нию таких автоматов
(а~значит, и~к~изуче\-нию функционирования смарт-кон\-трак\-тов в~блок\-чейн-сре\-де) 
развитый аппарат $p$-ади\-че\-ско\-го анализа и,~шире,
алгебраической динамики в~духе монографии \cite{AnKhr};
\item наконец, 2-ади\-че\-ское (и, шире, $p$-ади\-че\-ское) время является хоть 
и~не общепринятой, но тем не менее довольно широко используемой и~во многих случаях
релевантной математической моделью физического времени, активно изучаемой
уже почти три десятилетия
в рамках $p$-адической математической физики (см.\  обзорную статью~\cite{DraKhrenVol},
а~также соответствующие разделы
и~ссылки в~\cite{AnKhr}).
\end{itemize}




\section{Автоматы и~языки}

%\label{sec:auto}
Понятие <<автомат>> в~русскоязычной литературе  используется в~различных
смыслах, которым со\-ответствуют  английские термины state machine, sequential
machine, transducer и~т.\,д., поэтому\linebreak во избежание недоразумений введем определения,
используемые  в~данной работе. Везде далее под~\textit{алфавитом} понимается
конечное непустое множество, содержащее хотя бы два элемента.

\smallskip

\noindent
\textbf{Определение 2.1.}\
\textit{Автомат-определитель} (далее~--- d-ав\-то\-мат)~--- 
это кортеж $\langle\Cal I,\Cal S,\Cal F,S,s_0\rangle$,
где
\begin{enumerate}[(1)]
\item $\Cal I$  есть  \textit{входной} алфавит;
\item $\Cal S$ есть непустое (необязательно конечное) множество, называемое
множеством \textit{со\-сто\-яний};
\item $\Cal F$ есть конечное непустое подмножество множества~$\Cal S$, называемое
множеством \textit{при\-ни\-ма\-ющих со\-сто\-яний};
\item $s_0\in\Cal S$~--- \textit{начальное} со\-сто\-яние;
\item $S\colon \Cal I\times\Cal S\to \Cal S$~--- \textit{функция перехода}.
\end{enumerate}
Автомат-определитель называется \textit{конечным}, если конечно множество $\Cal S$ его
состояний.

\smallskip

Множество  всех конечных последовательностей~$\mathbf W(\Cal I)$ над множеством
$\Cal I$ называется множеством \textit{слов}. Отметим, что $\mathbf W(\Cal I)$ не содержит пустого слова (слова нулевой длины)
$\varnothing$; полагаем $\mathbf W_0(\Cal I)\hm=\mathbf W(\Cal I)\cup \{\varnothing\}$.
Используем стандартное определение языка, распознаваемого d-ав\-то\-ма\-том, и~регулярного
языка (языка, распознаваемого конечным d-ав\-то\-ма\-том) (см., 
например,~\cite{Allouche-Shall}).
Отметим, что данному понятию конечного d-ав\-то\-ма\-та в~\cite{Allouche-Shall} соответствует
понятие DFA~--- deterministic finite automaton.


\smallskip

\noindent
\textbf{Определение 2.2.}\
\textit{Автомат-преобразователь} (далее~--- f-ав\-то\-мат)  
есть кортеж $\langle\Cal I,\Cal S,\Cal O,S,O,s_0\rangle$,
где $\Cal I$, $\Cal S$, $S$ и~$s_0$  те же, что и~в~определении~2.1,
$\Cal O$~---   \textit{выходной} алфавит, а $O\colon \Cal I\times\Cal S\to\Cal O$ ---
\textit{функция выхода}.


\smallskip

Вышеприведенное определение соответствует  понятию автомата Мили, или, что
то же самое, понятию 1-равномерного преобразователя (1-uniform transducer)
 из~\cite{Allouche-Shall}, с~той лишь разницей, что множество состояний автомата
 в~смысле определения~2.2 может
быть и~бесконечным; если же множество состояний конечно, то определение~2.2
превращается в~стандартное определение автомата Мили (Mealy sequential machine).

Каждый преобразователь естественным образом задает отображение множества~$\mathbf W(\Cal I)$
во множество~$\mathbf W(\Cal O)$, а~каж\-дый d-ав\-то\-мат задает отображение множества
$\mathbf W(\Cal I)$ во множество~$\Cal F \cup {R}$, где $R\hm\notin \Cal S$, и~в~$R$
отображаются те и~только те слова,  которые
не принимаются d-ав\-то\-матом.

\vspace*{-9pt}


\section{Автоматы и~время}
%\label{sec:T-auto}

\vspace*{-3pt}

T-автоматы, они же автоматы с~метками времени, или
timed automata в~англоязычной литературе, введенные в~\cite{timed-auto},
использовались
для моделирования функционирования блок\-чейн-сре\-ды, поскольку разные узлы, функционирующие
в~блок\-чейн-сре\-де, получают
очередной блок, вообще говоря, в~разные моменты времени.
Целый ряд атак на блокчейн основан именно
на факте <<раз\-но\-вре\-мен\-ности>> получения очередных блоков поль\-зо\-ва\-те\-лями.

Чтобы ввести
понятие Т-ав\-то\-ма\-та,
сначала понадобится определение (бесконечного) слова  с~метками  времени
(timed word), или, для краткости, t-сло\-ва.


\smallskip

\noindent
\textbf{Определение 3.1.}\
(Бесконечное) \textit{слово с~метками времени} (t-сло\-во) над алфавитом~$\Cal A$ есть
(бесконечная) последовательность пар $((a_i,\tau_i))_{i=0}^\infty$, где $a_i\hm\in\Cal
A$, $\tau_i\hm\in\R_{\ge 0}$, причем последовательность~$(\tau_i)_{i=0}^\infty$ действительных
неотрицательных чисел~$\tau_i$ строго и~неограниченно возрастает: 
$\tau_0\hm<\tau_1\hm<\tau_2<\cdots$
и~$\lim_{i\to\infty}\tau_i\hm=\infty$.


\smallskip

На содержательном уровне слово с~метками времени $((a_i,\tau_i))_{i=0}^\infty$
интерпретируется как последовательность символов $(a_i)_{i=0}^\infty$ алфавита~$\Cal A$, 
поданных в~автомат в~моменты времени $\tau_0, \tau_1, \ldots$
соответственно.

Ниже, следуя~\cite{timed-auto}, дадим определение Т-ав\-то\-ма\-тов.
В~этом определении будет использоваться понятие \textit{таблицы переходов состояний
с~метками
времени}, а также  понятие \textit{временн$\acute{\mbox{ы}}$х ограничений} на переходы.

\smallskip

\noindent
\textbf{Определение 3.2.}\
%\label{def:TTT}
Пусть $T$ есть (счетное) множество переменных, называемых далее 
\textit{временн$\acute{\mbox{ы}}$ми переменными}, пусть $t_j\hm\in T$. 
\textit{Временн$\acute{\mbox{ы}}$м ограничением} называется
любая булева комбинация предикатов вида $t_j\hm\le q$ и~$q\hm\le t_j$, где
$q\in \Qo$ --- неотрицательные рациональные константы, а символ~$\le$
интерпретируются обычным образом как бинарное отношение <<меньше  либо
равно>>. Пусть даны: (входной) алфавит~$\Cal I$, конечное множество~$\Cal S$ состояний,
конечное множество~$C$, называемое множеством \textit{таймеров}, и~множество
$\Phi(C)$  временн$\acute{\mbox{ы}}$х ограничений от временн$\acute{\mbox{ы}}$х переменных $t_1,\ldots,t_{|C|}$.
Тогда \textit{таблица переходов состояний
с~метками
времени} (далее~--- ТТТ) есть некоторое множество кортежей вида 
$(s,a,s^\prime, G,\varphi)$, где $s,s^\prime\hm\in \Cal S$, $\varphi\hm\in\Phi(C)$, 
$G\hm\in 2^C$~--- подмножество множества таймеров (возможно, и~пустое).

\smallskip

\noindent
\textbf{Определение 3.3.}\
%\label{def:T-auto}
\textit{Детерминированный автомат с~метками времени} (далее~--- Т-ав\-то\-мат) 
есть кортеж $\langle\Cal I,\Cal S,\Cal E,C,
\Phi(C),s_0\rangle$,
где $\Cal I$, $\Cal S$, $s_0$~--- те же, что и~в  определении~2.1,
$\Cal S$ --- конечное множество,  $C$ и~$\Phi(C)$~--- из определения~3.2, 
а~$\Cal E$~--- ТТТ в~смысле определения~3.2, причем
\begin{enumerate}[(1)]
\item  в~начальном состоянии~$s_0$ текущие значения всех временн$\acute{\mbox{ы}}$х 
переменных равны~0;
\item для любых $a\hm\in\Cal I$, $s\hm\in\Cal S$ и~любой пары элементов TTT~$\Cal E$
вида $(s,a,*,*,\varphi_1)$, $(s,a,*,*,\varphi_2)$ временн$\acute{\mbox{ы}}$е ограничения~$\varphi_1$
и $\varphi_2$ являются взаимно исключающими,  т.\,е.\  $\varphi_1\wedge\varphi_2$
тождественно ложно.
\end{enumerate}


\noindent
\textbf{Замечание.}\
Если в~определении~3.3 допустить наличие нескольких начальных
состояний и~опустить условие~(2), то получим определение недетерминированного
Т-ав\-то\-мата.

\vspace*{-6pt}

\section{$\mathrm d$- и~$\mathrm f$-автоматы вместо Т-автоматов в~моделях блокчейн-среды}
\label{sec:T-block}

\vspace*{-3pt}

В этом разделе обсудим возможность сведения T-ав\-то\-мат\-ной модели блок\-чейн-сре\-ды,
используемой в~литературе (например, в~работах~\cite{Bitcoin-contract-model,Model-bitcoin-uppaal}),
и~покажем, что описанные T-ав\-то\-мат\-ные  модели блок\-чейн-среды могут быть <<без
потери точ\-ности>>  сведены  к~существенно более простым моделям,
а~именно: к~<<обычным>>  \mbox{f-ав}\-то\-ма\-там в~смысле определения~2.2.
Отметим сразу, что <<физической основой>> для такой <<аппроксимации>> функционирования
блокчейн-среды с~помощью более простой \mbox{f-ав}\-то\-мат\-ной модели, чем  с~помощью
используемой в~литературе более сложной \mbox{Т-ав}\-то\-мат\-ной,
служит следующее ограничение:
\textit{в реальной жизни время, разделяющее два следующих один за другим события,
не может быть произвольно \underline{малым}}, оно всегда ограничено снизу некоторой
величиной. Например, в~квантовой физике предполагается, что события,\linebreak
 разделенные
планковским временем, т.\,е.\ промежутком примерно в~$5,4\cdot 10^{-44}$~с,
происходят\linebreak
 одновременно, и,~более того, в~настоящее время минимальный интервал
времени, доступный физическому измерению, составляет примерно~$10^{-20}$~с. 
Отсюда следует, что любой временной интервал \textit{кратен} некоторому
минимальному временн$\acute{\mbox{о}}$му интервалу (в~предельном случае~--- планковскому времени)
и,~таким образом, с~точностью до множителя, равного длине этого минимального
интервала, \textit{является  натуральным числом}.

С другой стороны, T-ав\-то\-мат\-ные модели
позволяют рассматривать поведение моделируемой сис\-те\-мы в~<<предельных>>
ситуациях; например,  когда временной промежуток между соседними <<событиями>>
стремится к~0 (см.~\cite[example 3.22]{timed-auto}). Рас\-смот\-ре\-ние <<предельного>>
поведения часто оказывается очень полезным для описания не только качественных,
но нередко и~количественных характеристик моделируемой системы. Возможность
<<перехода к~пределу>> в~Т-ав\-то\-мат\-ных моделях основана на том, что
 множество~$\Qo$ всюду плотно в~множестве~$\Ro$ относительно обычной действительной
метрики.

Таким образом, для моделирования блок\-чейн-сре\-ды хотелось бы построить 
аналог ТА, в~которых время,
с одной стороны, было бы <<дискретным>>, т.\,е.\  в~качестве <<исходных>> 
меток времени выступали
бы натуральные,  а~не рациональные (как в~определении ТА) числа, но, тем не
менее, чтобы сохранялась и~возможность <<предельного перехода>>  как в~целях получения описания поведения всей системы во времени на (хотя бы) качественном
уровне, так и~получения оценок  <<точности>> модели.  Поскольку
речь идет о~предельном переходе, то необходимо задать некоторую
метрику на множестве всех натуральных чисел, относительно которой такой предельный
переход был бы возможен и~относительно которой натуральные числа образовывали
бы всюду плотное подмножество подобно тому, как множество~$\Qo$ является
всюду плотным подмножеством в~$\Ro$ относительно действительной мет\-ри\-ки.
Такие метрики существуют: это $p$-ади\-че\-ские мет\-рики.

Будем рассматривать t-сло\-ва  с~метками
времени из $\N_0\hm=\{0,1,2,3,\ldots\}$ (а~не из~$\Ro$, см.\
 определение~3.1) над конечным
алфавитом~$\Cal A$, содержащим хотя бы два символа.
Без ограничения общности можно считать, что
если $p\hm\ge 2$~---  это мощность алфавита~$\Cal A$, то  символами алфавита~$\Cal A$ 
являются числа $0,1,\ldots,p-1$.
Заметим, что такие  t-сло\-ва  пред\-став\-ля\-ют
собой част\-ный случай так на\-зы\-ва\-емых \textit{слов с~данными} (data word), введенных 
в~\cite{Bouyer-Algebraic-Time}, а~именно: когда множество данных~$\Cal D$
совпадает с~$\N_0$. Введенные в~определении~3.1
слова с~метками времени также пред\-став\-ля\-ют собой слова с~данными для случая,
когда данные лежат в~$\Ro$. На основе понятия слов  с~данными 
 естественным
образом вводится понятие \textit{языка с~данными} (data language), 
а~также \textit{автомата с~данными} (data automaton, далее~--- D-ав\-томат).
\smallskip

\noindent
\textbf{Определение 4.1.}\
%\label{def:D-auto}
\textit{Автомат $\mathfrak A$ с~данными~$\mathcal D$} есть кортеж $\mathfrak A
\hm=\langle\Cal I,\Cal S,\Cal F,\Cal T,k,\sim~,s_0\rangle$, где 
$\Cal I$, $\Cal S$, $\Cal F$, $s_0$~--- те же, что и~в определении~2.1;
\begin{itemize}
\item $k$ есть натуральное число  (называемое числом регистров данных);
\item $\sim$ есть отношение эквивалентности конечного индекса, определенное
на~$\Cal D^k$;
\item $\Cal T\subseteq \Cal S\times\Cal D^k\fac\sim\times\Cal I\times 
\Cal D^k\fac\sim\times\Cal S$ есть конечное множество переходов;
\item $\Cal U$ есть множество модификаций состояний регистров 
$\mathrm{upd}\colon \Cal D^k\to\Cal D^k$,
удовлетворяющие следующим ограничениям:
\begin{itemize}
\item для любого кортежа 
$$
(s,g,a)\in\Cal S\times \Cal D^k\fac\sim\times \Cal I
$$ 
существует (единственная) модификация регистров $\mathrm{upd}\hm\in\Cal U$
такая, что если
$(s,g,a,\mathrm{upd}^\prime,g^\prime,s^\prime)\hm\in\Cal T$ для
некоторого $\mathrm{upd}^\prime\in\Cal U$, то $\mathrm{upd}^\prime\hm=\mathrm{upd}$;
\item если $\!(s,g,a,\mathrm{upd},g^\prime,s),
(s,g,a,\mathrm{upd},g^\prime,s^\prime)\hm\!\in\!\Cal T$, то $s^\prime\hm=s$.
\end{itemize}
\end{itemize}

Как показано в~\cite[теорема~13]{Bouyer-Algebraic-Time}, для любого детерминированного
T-автомата из определения~3.3,\linebreak имеющего~$n$ таймеров (см.\ определение~3.2), 
существует D-ав\-то\-мат с~$2n\hm+2$ регистрами, рас\-по\-зна\-ющий 
в~точ\-ности тот же самый язык. Далее\linebreak построим   f-ав\-то\-ма\-ты (см.\
 определение~2.2), аппроксимирующие с~любой наперед заданной точ\-ностью (в~некотором
точно определенном ниже смысле)  данный детерминированный T-ав\-то\-мат и~тем
самым сведем задачу моделирования функционирования блок\-чейн-сре\-ды (в первую
очередь, моделирования смарт-кон\-трак\-тов) Т-ав\-то\-ма\-та\-ми к~моделированию <<обычными>>
автоматами с~двоичными входами и~двоичными выходами. Для
этого сначала понадобится ввести новый частный тип D-ав\-то\-ма\-тов, а~именно: 
\textit{автоматы с~$p$-ади\-че\-ским временем}.


Зафиксируем некоторое простое число~$p$ (в контексте данной статьи наиболее важным
является случай $p\hm=2$) и~рассмотрим в~качестве меток данных (в словах с~данными) \textit{целые $p$-адические
числа}, т.\,е.\ элементы \textit{пространства~$\Z_p$ 
целых $p$-ади\-че\-ских чисел}. С~введением в~теорию $p$-ади\-че\-ских
чисел и~$p$-ади\-че\-ский анализ можно ознакомиться, например,
по вводным главам в~\cite{AnKhr}. Здесь же введем  лишь
самые необходимые понятия
из $p$-ади\-че\-ско\-го анализа, и~притом на неформальном уровне.

Множество $\Z_p$ можно рассматривать как множество $\mathbf W^\infty (\Cal A)$ 
всех бесконечных (в~одну
сторону, в~данной статье~--- влево)
слов над алфавитом $\Cal A\hm=\{0,1,\ldots,p-1\}$, символы которого можно считать
элементами кольца~$\Z/p\Z$ вычетов по модулю~$p$, 
т.\,е.\ элементами поля из~$p$ элементов. 
На множестве~$\Z_p$ можно задать операции сложения и~умножения
с~по\-мощью стандартных <<школьных>> алгоритмов сложения и~умножения <<в столбик>>
чисел, пред\-став\-лен\-ных в~сис\-те\-ме счис\-ле\-ния с~основанием~$p$.
Если $p\hm=2$, то бесконечные бинарные строч\-ки можно мыслить себе
как представления чисел в~обобщенном \textit{обратном двоичном коде}
(см., например,~\cite[с.~213]{Knuth}).  Использование
обобщенного обратного двоичного кода дает возможность записывать в~регистр 
бесконечной длины как все целые неотрицательные
чис\-ла (им соответствуют строчки с~конечным числом единиц), так и~все 
целые отрицательные
чис\-ла (им соответствуют строчки с~конечным числом нулей), а~также все рациональные
чис\-ла, пред\-ста\-ви\-мые в~виде простых несократимых дробей с~нечетными знаменателями
(им соответствуют периодические с~ка\-ко\-го-то момента строчки).

Множество~$\Z_p$  является  полным компактным метрическим пространством относительно
\textit{$p$-ади\-че\-ской метрики}~$d_p$, которая задается следующим образом:
$d_p(\mathbf a,\mathbf b)={1}/{p^i}$
тогда и~только тогда, когда
$\mathbf a\hm=\cdots a_{i+1}a_{i} {c_{i-1}\cdots
c_0}$, $\mathbf b\hm=\cdots b_{i+1}b_{i} {c_{i-1}\cdots c_0}$
и~$a_{i}\ne b_{i}$ (по определению  $d_p(\mathbf a,\mathbf b)\hm=0$, если такого~$i$ 
не существует, т.\,е.\ если бесконечные слова~$\mathbf a$ и~$\mathbf b$
совпадают). Абсолютная величина~$\|\mathbf a\|_p$ вводится стандартным образом
как расстояние до  числа~0 (этому числу соответствует бесконечная строчка
из одних только нулей): $\|\mathbf a\|_p\hm=d_p(\mathbf a,0)$.

Можно ввести понятия <<приведения
по модулю~$p^n$>>  и~<<сравнения по модулю~$p^n$>> для целых $p$-ади\-че\-ских
 чисел, а~именно:
приведение по модулю~$p^n$ бесконечного слова в~алфавите $\Cal A\hm=\{0,1,\ldots,p-1\}$
означает всего лишь переход к~конечному начальному отрезку длины $n$ этого
бесконечного слова, т.\,е.\ $\bmod \,p^n\colon \mathbf W^\infty (\Cal A)\hm\to \mathbf
W^n(\Cal A)$, где~$\mathbf W^n(\Cal A)$ есть множество всех слов длины~$n$
над алфавитом~$\Cal A$. Отметим, что элементы множества~$\mathbf W^n(\Cal A)$ естественным образом отождествляются с~числами $0,1,\ldots,p^n-1$, представленными
в~системе счисления с~основанием~$p$, а~эти числа, в~свою очередь, отождествляются
с~элементами кольца~$\Z/p^n\Z$ вычетов по модулю~$p^n$.
Более того, оказывается,
что любое отображение  $f_\mathfrak A\colon\mathbf W^\infty(\Cal A)
\hm\to\mathbf W^\infty(\Cal A)$, задаваемое ав\-то\-ма\-том-пре\-об\-ра\-зо\-ва\-те\-лем~$\mathfrak A$, 
входной и~выходной алфавиты которого суть~$\Cal A$, т.\,е.\
 $\Cal I\hm=\Cal O\hm=\Cal A\hm=\{0,1,\ldots,p-1\}$ (см. определение~2.2)
есть функция, определенная  на~$\Z_p$ и~при\-ни\-ма\-ющая значения в~$\Z_p$, 
которая удовлетворяет
$p$-ади\-че\-ско\-му условию Липшица с~константой~1 
(и,~следовательно, является непрерывной
относительно мет\-ри\-ки~$d_p$ функцией):
$\|f_\mathfrak A(\mathbf a)\hm-f_\mathfrak A(\mathbf b)\|_p\hm\le\|\mathbf a 
\hm-\mathbf b\|_p$
для любых $\mathbf a, \mathbf b \hm\in\Z_p$. Верно и~обратное: любое отображение
из~$\Z_p$ в~$\Z_p$, удовлетворяющее $p$-ади\-че\-ско\-му условию Липшица с~константой~1, 
задается некоторым ав\-то\-ма\-том-пре\-об\-ра\-зо\-ва\-те\-лем 
(не обязательно конечным), входной и~выходной алфавиты
которого суть $\{0,1,\ldots,p-1\}$ (см., например,~\cite{me:Discr_Syst}).
{\looseness=1

}

Отметим, что  каждый из двух типов автоматов: ав\-то\-ма\-ты-оп\-ре\-де\-ли\-те\-ли, 
т.\,е.\ $d$-ав\-то\-ма\-ты из определения~2.1,
и~ав\-то\-ма\-ты-пре\-об\-ра\-зо\-ва\-те\-ли, т.\,е.\
 f-ав\-то\-ма\-ты из определения~2.2, ---
может быть  сведен один к~другому
(см.\ подробнее~\cite[теорема~4.3.2]{Allouche-Shall}).
Таким образом,  в~случае $p\hm=2$ задачи о $d$-ав\-то\-ма\-тах и~распознаваемых
ими языках могут быть сведены к~задачам о функциях, удовлетворяющих 2-ади\-че\-ско\-му
условию Липшица с~константой~1.  Такие функции
называются в~литературе также функциями треугольного
вида, двоичными совместимыми функциями, Т-функ\-ци\-ями. 

В~контексте данной статьи
весьма важным является тот факт, что Т-функ\-ции
допускают особо прос\-тую реализацию в~виде компьютерной программы,
а~именно:
\textit{компьютерная реализация детерминированной функции автомата, 
входной\linebreak и~выходной алфавиты которого состоят
из двух символов, не требует реализации его таблицы переходов состояний, а~может
быть записана (в~том числе и~для автоматов с~бесконечным числом состояний)
в~виде Т-функ\-ции, которая, в~свою очередь, представляет собой программу
без ветвлений, состоящую из последовательности стандартных компьютерных
команд, таких как арифметические команды (сложение и~умножение натуральных чисел) 
и~поразрядные логические команды $\OR,\AND,\XOR,\NOT$, а~также производных
от них команд, таких как сдвиг в~сторону старших разрядов, маскирование 
и~ряда других, таких как деление на нечетные числа, возведение нечетных чисел
в~степень и~др.}~\cite{AnKhr}.

Сказанное остается в~силе и~для автоматов, входной и~выходной алфавиты которых
состоят из соответственно~$2^n$ и~$2^m$~символов, поскольку такие автоматы
можно рассматривать как автоматы, имеющие~$n$ двоичных входов и~$m$ двоичных выходов,
а~значит, как многомерные Т-функ\-ции, т.\,е.\ как непрерывные относительно
\mbox{2-ади}\-че\-ской метрики\linebreak отображения
из~$\Z_2^n$ в~$\Z_2^m$, удовлетворяющие многомерному 2-ади\-че\-ско\-му условию
Липшица\linebreak с~константой~1. Для Т-функ\-ций имеется
хорошо развитая математическая теория, основанная на 2-ади\-че\-ском анализе
и~име\-ющая многочисленные (в~первую очередь~--- криптографические)
приложения (см.~\cite{AnKhr}).

Дадим теперь формальное определение автомата с~2-ади\-че\-ским временем.

\smallskip

\noindent
\textbf{Определение 4.2.}\
%\label{def:Z2-auto}
D-автомат из определения~4.1 назовем \textit{автоматом с~2-ади\-че\-ским
временем} ($\Z_2$-ав\-то\-ма\-том), если $\Cal I\hm=\{0,1\}$ и~$\Cal D\hm=\Z_2$.


\smallskip

Разумеется, похожим образом можно сформулировать и~понятие $\Z_2$-ав\-то\-ма\-та
с~входным алфавитом из~$2^r$~символов, т.\,е.\ $\Z_2$-ав\-то\-ма\-та с~$r$~двоичными
входами.
Язык, распознаваемый $\Z_2$-ав\-то\-ма\-том, определяется обычным образом на основе
определения~4.2.

Говоря неформально, любой T-ав\-то\-мат (см.\ определение~3.3) можно
рассматривать как <<автомат с~двумя входами>>: временн$\acute{\mbox{ы}}$м и~алфавитным,
где на каждом такте работы  подается на алфавитный вход очередной символ 
входного слова, а~на временной вход~--- действительное чис\-ло, 
служащее меткой времени этого входного символа.  
С~этой точки зрения $\Z_2$-автомат тоже имеет два
входа; при этом на алфавитный вход подается символ входного алфавита, т.\,е.~0 
или~1, а~на временной вход~--- метка времени, т.\,е.\ целое 2-ади\-че\-ское \mbox{число}.

Все t-слова
могут быть равномерно приближены словами с~2-ади\-че\-ски\-ми метками времени (далее~--- 
$\Z_2$-сло\-ва\-ми)
в~следующем смысле. Вначале все символы входного алфавита T-ав\-то\-ма\-та пронумеруем 
и~запишем в~виде двоичных представлений соответствующих чисел. Таким образом,
можно считать, что на алфавитный вход автомата всегда подается~$r$~бинарных 
последовательностей, где~$r$~--- число двоичных разрядов, необходимых для
записи всех символов входного алфавита.

Далее зафиксируем любое действительное $\varepsilon\hm>0$ и~выберем  рациональные 
числа~$z_i(\mathbf w)$, представимые в~виде простых несократимых
дробей с~нечетными знаменателями (все эти рациональные
числа лежат в~$\Z_2$) так, чтобы $|\tau_i(\mathbf w)\hm-z_i(\mathbf w)|
\hm<\varepsilon$, где~$\tau_i(\mathbf w)$
есть $i$-я метка времени в~t-сло\-ве~$\mathbf w$.  Такой выбор всегда можно
сделать, например, следующим образом.
Представим 
$$
\tau_i(\mathbf w)=\lfloor\tau_i(\mathbf w)\rfloor 
+ \left(\tau_i(\mathbf w)-\lfloor\tau_i(\mathbf w)\rfloor\right),
$$ 
где~$\lfloor\tau_i(\mathbf w)\rfloor$
есть целая (с~недостатком) часть чис\-ла~$\tau_i(\mathbf w)$. Выберем $h\hm\in\N$
таким, чтобы $1/3^{h}\hm<\varepsilon$, запишем дроб\-ную часть $(\tau_i(\mathbf w)
\hm-\lfloor\tau_i(\mathbf w)\rfloor)$ чис\-ла~$\tau_i(\mathbf w)$  в~троичной
сис\-те\-ме счис\-ле\-ния с~точ\-ностью до~$h$~троичных разрядов после запятой. Тогда
эта дроб\-ная часть есть чис\-ло вида~$c/3^h$, где $c\hm\in\{0,1,\ldots,3^h-1\}$,
и,~следовательно, является целым 2-ади\-че\-ским чис\-лом. Прибавляя к~полученному
таким образом
числу   целую (с недостатком) часть~$\lfloor\tau_i(\mathbf w)\rfloor$
числа~$\tau_i(\mathbf w)$, получаем
целое 2-ади\-че\-ское число~$z_i(\mathbf w)$. 
В~этом смысле каждое t-сло\-во $\mathbf w\hm= ((a_i,\tau_i))_{i=0}^\infty$ 
приближается с~точ\-ностью не хуже чем~$\varepsilon$ словом 
$((a_i,z_i(\mathbf w)))_{i=0}^\infty$,
которое является входным $\Z_2$-сло\-вом для $\Z_2$-ав\-то\-ма\-та с~$r$~алфавитными
входами, причем алфавит каждого алфавитного входа бинарный.

Далее все $\Z_2$-сло\-ва могут быть равномерно приближены $\Z_2$-сло\-ва\-ми с~метками
времени из~$\N_0$ (и~даже из~$\Z/2^h\Z$) с~любой наперед заданной 2-ади\-че\-ской 
точ\-ностью~$1/2^h$.
Действительно, для этого достаточно каждую из 2-ади\-че\-ских меток времени 
в~каждом $\Z_2$-сло\-ве привести по модулю~$2^h$.
 Таким\linebreak
  образом на основе вышеописанной процедуры <<аппроксимации>> 
 t-ав\-то\-ма\-та $\Z_2$-ав\-то\-ма\-том можно постро\-ить  последовательность 
 $\Z_2$-ав\-то\-ма\-тов~$\mathfrak Y_h$ с~метками
времени из~$\Z/2^h\Z$, $h\hm=1,2,3,\ldots$, аппроксимирующих в~вышеуказанном 
смысле исходный t-ав\-томат.

Используя описанную выше процедуру по\-стро\-ения f-ав\-то\-ма\-та на основе
данного d-ав\-то\-ма\-та,\linebreak можно   любому D-ав\-то\-ма\-ту  
сопоставить  \textit{детерминированную
функцию с~метками времени},  считая, например, что $i$-й символ выходного слова
имеет ту же метку времени, что и~$i$-й символ соответствующего ему входного слова. 
Таким  образом  на основе данного $\Z_2$-ав\-то\-ма\-та из определения~4.2
можно построить детерминированную функцию с~метками времени  из~$\Z_2$,
полагая $i$-й выходной символ равным~1, если автомат находится в~принимающем
состоянии (т.\,е.\ в~состоянии из множества~$\Cal F$), и~0 в~противном случае.

Наконец, этим способом каждому из построенных  выше аппроксимирующих автоматов~$\mathfrak Y_h$ 
можно сопоставить детерминированную функцию с~метками времени
из~$\Z/2^h\Z$. Итак, для данного Т-ав\-то\-ма\-та построена последовательность
аппроксимирующих его (в~вышеописанном смысле)  T-функ\-ций, т.\,е.\ <<обычных>> f-ав\-то\-ма\-тов 
в~смысле определения~2.2, имеющих $r\hm+h$ двоичных входов и~$h\hm+1$ 
двоичный выход. Таким образом, доказано следующее

\smallskip

\noindent
\textbf{Предложение.}\
\textit{Каждый Т-ав\-то\-мат аппроксимируется с~любой наперед заданной 
точностью} (в описанном выше смысле) 
\textit{некоторым f-ав\-то\-ма\-том над двухсимвольным алфавитом.}

\vspace*{-9pt}


\section{Выводы}

\vspace*{-2pt}

В данной работе показано, что для моделирования функционирования блок\-чейн-сре\-ды 
(в~част\-ности, моделирования работы
смарт-кон\-трак\-тов), даже несмотря на то что эта среда функционирует в~реальном
физическом времени, нет не\-об\-хо\-ди\-мости прибегать к~сложным (и~достаточно ресурсоемким)
моделям, основанным на концепции автоматов с~метками времени, представляющими 
собой действительные числа, а достаточно
ограничиться (без потери точности) моделированием этой среды с~помощью 
 детерминированных
функций над \mbox{2-сим}\-воль\-ным алфавитом (известных также под названием Т-функ\-ций),
т.\,е.\ с~по\-мощью <<обычных>> автоматов с~бинарным вход\-ным/вы\-ход\-ным алфавитом. 
Эти функции могут быть реализованы в~виде программ
без ветвления, выполненных как последовательности стандартных команд любого
процессора, что позволяет надеяться на относительную простоту их программной
реализации и~высокое быстродействие соответствующих программ.

 {\small\frenchspacing
 {%\baselineskip=10.8pt
 \addcontentsline{toc}{section}{References}
 \begin{thebibliography}{99}

\bibitem{timed-auto}
\Au{Alur~R., Dill~D.}
 The theory of timed automata~//
 Real-time: Theory and practice~/ Eds. J.~W.~de Bakker, C.~Huizing, W.\,P.~de Roever, 
 G.~Rozenberg.~---
Lecture notes
in computer science ser.~--- Springer, 1992. Vol.~600.  P.~45--73.

\bibitem{Bitcoin-contract-model} %2
\Au{Andrychowicz~M., Dziembowski~S., Malinowski~D., Mazurek~L.}
Modelling bitcoin contracts by timed automata~//
Formal modelling and analysis of timed systems~/
Eds. A.~Legay, M.~Bozga.~---
Lecture notes in computer science ser.~--- Springer, 2014. Vol.~8711: 
 P.~7--22.

\bibitem{Uppaal-tutorial}
\Au{David~A., Larsen~K.\,G., Legay~A., 
\mbox{Miku\!{\!\ptb{\normalsize \v{c}}}ionis}~M., Poulsen~D.\,B.}
Uppaal SMC tutorial~//
{Int. J.~Softw. Tools Te.}, 2015. Vol.~17. P.~397--415.

\bibitem{Model-bitcoin-uppaal}
\Au{Chaudhary~K., Fehnker~A., van~de~Pol J., Stoelinga~M.}
Modeling and verification of the bitcoin protocol~//
Electronic Proc. Theor. Comput. Sci., 2015. Vol.~196. P. 46--60.
%{Workshop on Models for Formal Analysis of Real Systems (MARS   2015)}. 

\bibitem{contract-automat}
\Au{Flood~M.\,D.,  Goodenough~O.\,R.}
Contract as automaton: The computational representation of financial
  agreements~//
{SSRN Electronic~J.}, 2015.
doi: 10.2139/\mbox{ssrn}. 2538224.

\bibitem{mining-smart-contract}
\Au{Guth~F., W$\ddot{\mbox{u}}$stholz~V., Christakis~M., M$\ddot{\mbox{u}}$ller~P.}
Specification mining for smart contracts with automatic abstraction
  tuning~// arXiv.org, 2018. arXiv:1807.07822 [cs.SE]. 12~p.

\bibitem{me:Discr_Syst} %7
\Au{Anashin~V.}
The non-{A}rchimedean theory of discrete systems~//
Math. Comput.  Sci., 2012. Vol.~6. P.~375--393.

\bibitem{AnKhr}
\Au{Anashin~V., Khrennikov~A.}
{Applied algebraic dynamics.}~---
Gruyter expositions in mathematics ser.~---
Berlin\,--\,New York: Walter~de~Gruyter GmbH \& Co, 2009. Vol.~49. 533~p.


\bibitem{DraKhrenVol}
\Au{Dragovich~B., Khrennikov~A.\,Yu., Kozyrev~S.\,V., Volovich~I.\,V.}
On $p$-adic mathematical physics~//
$p$-Adic Numbers Ultrametric Analysis Appl.,
2009. Vol.~1. P.~1--17.



\bibitem{Allouche-Shall}
\Au{Allouche~J.-P., Shallit~J.}
Automatic sequences. Theory, applications, generalizations. ---
Cambridge: Cambridge University Press, 2003. 583~p.

\bibitem{Bouyer-Algebraic-Time}
\Au{Bouyer~P., Petit~A., Th$\acute{\mbox{e}}$rien~D.}
An algebraic approach to data languages and timed languages~//
Inform. Comput., 2003. Vol.~182. P.~137--162.


\bibitem{Knuth}
\Au{Knuth~D.}
The art of computer programming. Vol.~2: Seminumerical
  algorithms.~--- 3rd ed.~---
Addison-Wesley, 1997. 791~p.

 \end{thebibliography}

 }
 }

\end{multicols}

\vspace*{-3pt}

\hfill{\small\textit{Поступила в~редакцию 09.02.19}}

\vspace*{8pt}

%\pagebreak

%\newpage

%\vspace*{-29pt}

\hrule

\vspace*{2pt}

\hrule

%\vspace*{-2pt}

\def\tit{ON AUTOMATA MODELS OF~BLOCKCHAIN}


\def\titkol{On automata models of~blockchain}

\def\aut{V.\,S.~Anashin}

\def\autkol{V.\,S.~Anashin}

\titel{\tit}{\aut}{\autkol}{\titkol}

\vspace*{-11pt}


\noindent
Faculty of Computational Mathematics and Cybernetics, M.\,V.~Lomonosov
Moscow State University,
1-52~Leninskie Gory, GSP-1,  Moscow 119991, Russian Federation

\def\leftfootline{\small{\textbf{\thepage}
\hfill INFORMATIKA I EE PRIMENENIYA~--- INFORMATICS AND
APPLICATIONS\ \ \ 2019\ \ \ volume~13\ \ \ issue\ 2}
}%
 \def\rightfootline{\small{INFORMATIKA I EE PRIMENENIYA~---
INFORMATICS AND APPLICATIONS\ \ \ 2019\ \ \ volume~13\ \ \ issue\ 2
\hfill \textbf{\thepage}}}

\vspace*{6pt}



\Abste{The author considers automata models of blockchain, 
mostly based on timed automata. The author suggests a~new version of timed 
automata that avoids some inconveniences that occur in modeling by using 
standard timed automata where time is represented by real numbers. In the latter case, 
one should use variables of two types, Boolean and real; when applied to blockchain 
modeling, this fact causes some difficulties both in obtaining theoretical estimates 
and in program implementation.  The present approach is based on 2-adic analysis since in 
that case, both time and digital variables are of one type only; namely, Boolean.} 

\KWE{blockchain; smart contract; timed automaton}





 \DOI{10.14357/19922264190205}

\vspace*{-14pt}

\Ack
\noindent
The research was supported by the Russian Foundation for Basic Research
(grant 18-29-03124).


\vspace*{4pt}

  \begin{multicols}{2}

\renewcommand{\bibname}{\protect\rmfamily References}
%\renewcommand{\bibname}{\large\protect\rm References}

{\small\frenchspacing
 {%\baselineskip=10.8pt
 \addcontentsline{toc}{section}{References}
 \begin{thebibliography}{99}
\bibitem{timed-auto-1} %1
\Aue{Alur, R., and D.~Dill.} 1992.
The theory of timed automata.
\textit{Real-time: Theory and practice}.
Eds. J.~W.~de Bakker, C.~Huizing, W.\,P.~de Roever, 
 and G.~Rozenberg. Lecture notes
in  computer science ser.  Springer. 600:45--73.


\bibitem{Bitcoin-contract-model-1} %2
\Aue{Andrychowicz, M., S.~Dziembowski, D.~Malinowski, and L.~Mazurek.} 2014.
Modelling bitcoin contracts by timed automata.
\textit{Formal modelling and analysis of timed systems}.
Eds. A.~Legay and M.~Bozga. 
Lecture notes in computer science ser. Springer. 8711:7--22.


\bibitem{Uppaal-tutorial-1} %3
\Aue{David, A., K.\,G.~Larsen, A.~Legay, M.~\mbox{Miku{\!\ptb{\normalsize \v{c}}}ionis}, 
and D.\,B.~Poulsen.} 
2015.
Uppaal SMC tutorial.
\textit{Int. J.~Softw. Tools Te.} 17:397--415.

\bibitem{Model-bitcoin-uppaal-1} %4
\Aue{Chaudhary, K., A.~Fehnker, J.~van~de~Pol, and M.~Stoelinga}. 2015.
Modeling and verification of the bitcoin protocol.
%Workshop on Models for Formal Analysis of Real Systems (MARS  2015): 
\textit{Electronic Proc. Theor. Comput. Sci.} 196:46--60.
  
  \bibitem{contract-automat-1} %5
\Aue{Flood, M.\,D., and O.\,R.~Goodenough.} 2015.
 Contract as automaton: The computational representation of financial
  agreements.
\textit{SSRN Electronic~J}.
doi: 10.2139/\linebreak ssrn.2538224.
  
  \bibitem{mining-smart-contract-1} %6
\Aue{Guth, F., V.~W$\ddot{\mbox{u}}$sthold, M.~Christakis, and P.~M$\ddot{\mbox{u}}$ller.}
2018.
 Specification mining for smart contracts with automatic abstraction
  tuning.  arXiv:1807.07822 [cs.Se]. 12~p.
Available at: {\sf https://arxiv.org/abs/1807.07822} (accessed
  January~17, 2019).
  
  \bibitem{me:Discr_Syst-1} %7
\Aue{Anashin, V.} 2012.
The non-{A}rchimedean theory of discrete systems.
\textit{Math. Comput. Sci.} 6:375--393.




\bibitem{AnKhr-1} %8
\Aue{Anashin, V., and A.~Khrennikov.}  2009.
\textit{Applied algebraic dynamics}.   Gruyter
  expositions in mathematics ser.
Berlin\,--\,New York: Walter~de~Gruyter GmbH \& Co. Vol.~49. 533~p.

\bibitem{DraKhrenVol-1} %9
\Aue{Dragovich, B., A.\,Yu.~Khrennikiv, S.\,V.~Kozyrev,  and
I.\,V.~Volovich.} 2009. 
On $p$-adic mathematical physics. 
\textit{$p$-Adic Numbers Ultrametric Analysis Appl}.
1:1--17. 
   
  \bibitem{Allouche-Shall-1} %10
\Aue{Allouche, J.-P., and J.~Shallit.} 2003. 
\textit{Automatic sequences. Theory, applications, generalizations}.
Cambridge: Cambridge University Press. 583~p.

\bibitem{Bouyer-Algebraic-Time-1} %11
\Aue{Bouyer, P., A.~Petit, and D.~Th$\acute{\mbox{e}}$rien.} 2003.
An algebraic approach to data languages and timed languages.
\textit{Inform.  Comput.} 182:137--162.


\bibitem{Knuth-1} %12
\Aue{Knuth, D.} 1997.
\textit{The art of computer programming. Vol.~2: Seminumerical
algorithms.} 3rd ed. Addison-Wesley. 791~p.
\end{thebibliography}

 }
 }

\end{multicols}

\vspace*{-6pt}

\hfill{\small\textit{Received February 9, 2019}}

%\pagebreak

%\vspace*{-18pt}





\Contrl

\noindent
\textbf{Anashin Vladimir S.} (b.\ 1951)~--- 
Doctor of Science in physics and mathematics,
professor, Faculty of Computational Mathematics and Cybernetics, M.\,V.~Lomonosov
Moscow State University,
1-52~Leninskie Gory, GSP-1,  Moscow 119991, Russian Federation;
 \mbox{anashin@iisi.msu.ru}
\label{end\stat}

\renewcommand{\bibname}{\protect\rm Литература}     %5
\newcommand{\Tsf}{^{\mathsf T}}
\newcommand{\rank}{\mathrm{rank}\,}

\def\stat{logachev}

\def\tit{ПОЛИНОМИАЛЬНЫЕ АЛГОРИТМЫ ВЫЧИСЛЕНИЯ ЛОКАЛЬНЫХ АФФИННОСТЕЙ КВАДРАТИЧНЫХ 
БУЛЕВЫХ ФУНКЦИЙ$^*$}

\def\titkol{Полиномиальные алгоритмы вычисления локальных аффинностей квадратичных 
булевых функций}

\def\aut{О.\,А.~Логачев$^1$, А.\,А.~Сукаев$^2$, С.\,Н.~Федоров$^3$}

\def\autkol{О.\,А.~Логачев, А.\,А.~Сукаев, С.\,Н.~Федоров}

\titel{\tit}{\aut}{\autkol}{\titkol}

\index{Логачев О.\,А.}
\index{Сукаев А.\,А.}
\index{Федоров С.\,Н.}
\index{Logachev O.\,A.}
\index{Sukayev A.\,A.}
\index{Fedorov S.\,N.}


{\renewcommand{\thefootnote}{\fnsymbol{footnote}} \footnotetext[1]
{Работа выполнена при частичной поддержке РФФИ (проект 18-29-03124~мк).}}


\renewcommand{\thefootnote}{\arabic{footnote}}
\footnotetext[1]{Московский государственный университет им.\
М.\,В.~Ломоносова; Институт проб\-лем информатики Федерального исследовательского
центра <<Информатика и~управ\-ле\-ние>> Российской академии наук, \mbox{logol@iisi.msu.ru}}
\footnotetext[2]{Московский государственный университет им.\ 
М.\,В.~Ломоносова, \mbox{asukaev@gmail.com}}
\footnotetext[3]{Московский государственный университет им.\
М.\,В.~Ломоносова, \mbox{s.n.feodorov@yandex.ru}}

\vspace*{-12pt}

 
\Abst{Аффинная нормальная форма позволяет рассматривать произвольную булеву функцию на 
определенных плоскостях (так называемых локальных аффинностях) как аффинную. Данное 
пред\-став\-ле\-ние~--- по сути, аффинная аппроксимация~--- булевых функций может 
помочь в~решении систем нелинейных уравнений над полем из двух элементов. Задача 
решения таких систем (специального вида), среди прочего, используется в~ряде 
методов синтеза и~анализа средств обеспечения информационной безопасности.
В~статье описывается способ нахождения локальных аффинностей для квадратичных 
булевых функций, основанный на теореме Диксона. Тем самым решается задача 
построения аффинных нормальных форм для таких функций. Кроме того, обсуждаются 
вопросы эффективности подобных алгоритмов.
Основная цель данной статьи~--- подготовить базу для готовящейся к~публикации 
работы, предлагающей метод решения систем квадратичных булевых уравнений 
с~помощью <<аппроксимирования>> соответствующих функций их аффинными нормальными 
формами.}


\KW{булева функция; система квадратичных булевых уравнений; 
разбиение векторного пространства; плоскость; локальная аффинность; теорема 
Диксона; аффинная нормальная форма; алгебраический криптоанализ}

\DOI{10.14357/19922264190110}
  
\vspace*{-1pt}


\vskip 10pt plus 9pt minus 6pt

\thispagestyle{headings}

\begin{multicols}{2}

\label{st\stat}


\section{Введение}

\vspace*{-2pt}

Центральная идея алгебраического криптоанализа состоит в~том, чтобы описать 
используемые в~анализируемой криптосхеме преобразования сис\-те\-мой алгебраических 
уравнений (с~некоторой сек\-рет\-ной информацией в~качестве неизвестных) над\linebreak 
конечным полем и~затем решить эту систему.
В~данной статье рассматриваются только булевы системы уравнений, хотя часть 
пред\-став\-лен\-ных здесь результа\-тов может иметь место и~для сис\-тем ал\-геб\-ра\-и\-че\-ских 
уравнений над произвольными конечными полями.

Из теории сложности вычислений известно, что вычислительная задача определения 
совместности систем нелинейных булевых уравнений является NP-пол\-ной~\cite{GJ1982, GT2017}, 
а~вычислительная задача решения систем нелинейных булевых 
уравнений является NP-труд\-ной~\cite{GJ1982,GT2017}.
Однако в~специальных случаях эти задачи могут решаться эффективно (см., 
например,~\cite{GT2017,Smi2000}).

Кроме того, существуют полиномиальные алгоритмы построения по произвольной 
системе уравнений системы с~фиксированной алгебраической 
степенью~\cite[\S\;11.4.2]{Bard2009}, что позволяет, в~частности, ограничиться 
рассмотрением только квадратичных систем уравнений.

Можно выделить несколько основных классов методов, используемых в~криптоанализе 
для решения (или оценки трудоемкости решения) систем полиномиальных булевых 
уравнений:
использование базисов Грёбнера~\cite[section~12.2]{Bard2009}, применение 
программных систем поиска выполняющего набора булевой формулы 
(SAT-solvers)~\cite{BCJ2007}, вероятностные и~тео\-ре\-ти\-ко-ко\-до\-вые 
методы~\cite{LSSYa2015}, а~также  методы линеаризации~\cite[section~12.3]{Bard2009}.
Основная идея методов линеаризации состоит в~применении <<линейных>> методов 
к~нелинейным системам, т.\,е.\ в~построении сис\-тем линейных уравнений, решение 
которых дает возможность найти решение исходной нелинейной системы.

Важным параметром метода линеаризации служит число переменных в~синтезируемых 
линейных системах уравнений. Как правило, речь идет об увеличении (не~более чем 
полиномиальном) количества переменных.
Метод, основанный на рассмотренных в~данной работе идеях, по своей сути, 
осуществляет линеаризацию, но при этом он остав\-ля\-ет число переменных неизменным.

Этот метод решения квадратичных систем булевых уравнений использует локальные 
аффинности уравнений системы и~состоит из двух этапов.
Первый этап (предварительный) содержательно представляет собой описание семейств 
локальных аффинностей уравнений.
Второй этап метода заклю\-ча\-ет\-ся собственно в~решении исходной сис\-те\-мы посредством 
анализа сис\-тем линейных уравнений, полученных с~помощью этих локальных 
аффинностей.

Настоящая работа (в~силу ограниченности\linebreak объема публикации) посвящена 
исследованию первого этапа и,~в~частности, вопросам его эффективности. 
Результаты исследований с~оценкой эф\-фек\-тив\-ности и~описанием параметров второго
\mbox{этапа} предлагаемого метода предполагается опуб\-ли\-ко\-вать в~одном из сле\-ду\-ющих 
выпусков журнала.

\vspace*{-4pt}

\section{Необходимые понятия и~обозначения}

В данной работе булев куб $\{0,1\}^n$ отождествляется с~$n$-мерным векторным 
пространством~$V_n$ над полем из двух элементов~$\mathbb{F}_2$.
Векторы из $V_n$ будет удобнее записывать \textit{строками} длины~$n$. Значок~$\Tsf$ 
используется для операции транспонирования матриц.
Всюду далее~$x$ обозначает вектор $(x_1,x_2,\ldots,x_n)$.

Знак $\oplus$ будет использоваться для записи суммы по модулю~$2$ булевых 
переменных и~операций сложения в~$\mathbb{F}_2$ и~покомпонентного сложения 
в~$V_n$.

Множество всех невырожденных аффинных преобразований (отображений в~себя) 
пространства~$V_n$ обозначается через $\mathrm{GA}(V_n)$. В~матричном 
представлении действие элемента~$\alpha\in\mathrm{GA}(V_n)$ на векторах 
пространства имеет вид $\alpha(x)\hm=xA\oplus b$, где $x$~пробегает~$V_n$; $A$~--- 
невырожденная $(n\times n)$-мат\-ри\-ца над~$\mathbb{F}_2$; $b\hm\in V_n$.

Множество всех булевых функций от $n$~переменных обозначим через
$$
\mathcal{F}_n=\{f\colon V_n\to \mathbb{F}_2\}\,.
$$
Как известно, произвольную булеву функцию~$f$ от переменных $x_1,\ldots,x_n$ 
можно представить (единственным образом) в~виде полинома Жегалкина:
$$
f(x)=\bigoplus_{\varepsilon\in\{0,1\}^n} a_{\varepsilon}x^{\varepsilon}\,,
$$
где %\label{Zhegalkin}
$\varepsilon\hm=(\varepsilon_1,\ldots,\varepsilon_n)$,
$a_{\varepsilon}\hm\in\mathbb{F}_2$ и~$x^{\varepsilon}\hm=x_1^{\varepsilon_1}\cdots x_n^{\varepsilon_n}$ (считаем, 
$x_i^0\hm=1$, $x_i^1\hm=x_i$).
Далее под булевой функцией будет, как правило, подразумеваться ее запись в~виде 
полинома.

Если $\varphi$~--- некоторое преобразование пространства~$V_n$, то его действие на 
функцию~$f\hm\in\mathcal{F}_n$ будем определять и~обозначать так: 
$f^{\varphi}(x)\hm=f(\varphi(x))$.
В~частности, в~случае аффинных преобразований пространства будет рассматриваться 
множество $\mathrm{Orb}_f(\mathrm{GA}(V_n))\hm=\{f^{\varphi}\mid 
\varphi\hm\in\mathrm{GA}(V_n)\}$~--- орбита функции~$f$ относительно действия 
группы~$\mathrm{GA}(V_n)$.
Имея в~виду, что произведение~$\alpha_1\alpha_2$ элементов из $\mathrm{GA}(V_n)$ 
есть композиция $\alpha_1\circ\alpha_2(x)\hm=\alpha_1(\alpha_2(x))$, заметим, что 
действие~$\alpha_1\alpha_2$ на произвольную функцию $f\hm\in\mathcal{F}_n$ 
корректно определять следующим образом:
$$
f^{\alpha_1\alpha_2}(x)=\left(f^{\alpha_1}\right)^{\alpha_2}(x)=f^{\alpha_1}
\left(\alpha_2(x)\right)
=f\left(\alpha_1\alpha_2(x)\right),
$$
поскольку~$\alpha_i$ действуют на булеву функцию преобразованием \textit{ее 
аргумента}.

%Когда мы делаем невырожденную аффинную замену переменных $x'=\alpha(x)=xA\oplus 
%b$, функция~$f(x)$, при подставлении в~нее выражений старых переменных через 
%новые, преобразуется к~виду $f^{\alpha^{-1}}(x')$.


\textit{Алгебраической степенью} булевой функции~$f$ от $n$~переменных называют 
величину

\noindent
$$
\deg f = \max\left\{\sum\limits_{i=1}^n \varepsilon_i\mid a_{\varepsilon}=1\right\}
$$
(суммирование~--- в~$\mathbb{Z}$), т.\,е.\ максимальное число различных 
переменных в~мономах данного представления.

В множестве~$\mathcal{F}_n$ всех булевых функций от~$n$~переменных выделим 
подмножество

\noindent
$$
\mathcal{A}_n=\left\{f\in\mathcal{F}_n\mid \deg f\leqslant 1\right\}.
$$
Составляющие это подмножество функции называются линейными (в~математической 
логике и~кибернетике) или аф\-фин\-но-ли\-ней\-ны\-ми (в~ал\-геб\-ре), однако по сложившейся 
в~криптологии традиции в~данной работе они называются \textit{аффинными}, т.\,е.\ 
понимаются как частный случай аффинного \textit{отоб\-ра\-же\-ния} $n$-мер\-но\-го 
пространства в~одномерное.


Булеву функцию~$f$ c $\deg f\hm\leqslant 2$ будем называть 
\textit{квадратичной}\footnote{В~алгебре такие функции называют 
аффинно-квад\-ра\-тич\-ны\-ми. Квадратичными при этом называют функции, представляемые 
\textit{однородными} полиномами второй степени.}.
По определению квадратичная функция~$f\hm\in\mathcal{F}_n$ (ее полином Жегалкина) 
имеет вид:

\noindent
$$
f(x)= \bigoplus_{1\leqslant i<j\leqslant n} q_{ij}x_ix_j\oplus
\bigoplus_{1\leqslant k\leqslant n} 
l_kx_k \oplus c\,,
$$
где $q_{ij},l_k,c\in\mathbb{F}_2$.

В настоящей работе рассматриваются системы уравнений

\noindent
\begin{equation}
\left.
\begin{array}{c}
        f_1(x_1,\ldots,x_n)=0\,;\\
        f_2(x_1,\ldots,x_n)=0\,;\\
        \vdots\\
        f_m(x_1,\ldots,x_n)=0\\
    \end{array}
    \right\}
    \label{system}
\end{equation}
с квадратичными булевыми функциями~$f_i$, $1\hm\leqslant i\hm\leqslant m$, и~$m\hm>n$.
%Мы предполагаем, что все рассматриваемые нами системы квадратичных уравнений 
%имеют единственное решение.

\pagebreak

В матричном виде квадратичная функция записывается следующим образом:
$$
f(x)=x Q_f x\Tsf\oplus l_f x\Tsf \oplus c\,,
$$
где $Q_f$~--- верхнетреугольная $(n\times n)$-мат\-ри\-ца с~нулевой главной 
диагональю; $l_f\in\mathbb{F}_2^n$; $c\in\mathbb{F}_2$.
Рас\-смат\-ри\-ва\-ют также симметричную матрицу
$$
\tilde{Q}_f=Q_f\oplus Q_f\Tsf\,.
$$
Она определяет билинейную форму
$$
q_f(u,v)=u\tilde{Q}_f v\Tsf=f(u\oplus v)\oplus f(u) \oplus f(v)\oplus c\,,
$$
называемую \textit{ассоциированной с~квадратичной функцией~$f$}.

Булева билинейная форма $q(u,v)$, $u,v\hm\in V_n$, удовле\-тво\-ря\-ющая условиям
$$
q(u,u)=0\,;\qquad q(u,v)=q(v,u)\,,
$$
называется \textit{симплектической}.
Такие билинейные формы находятся во взаимно однозначном соответствии с~булевыми 
симметричными матрицами с~нулевой главной диагональю, называемыми 
\textit{симплектическими матрицами}.

Таким образом, для произвольной квадратичной булевой функции~$f$ 
матрица~$\tilde{Q}_f$~--- сим\-плектическая. Также очевидно, что билинейная\linebreak 
форма~$q_f(u,v)$, ассоциированная с~$f$, является симплектической.

\smallskip

\noindent
\textbf{Предложение~1}\
[6, лемма~3.3.1; 7, \S\;15.2, лемма~3].
\textit{Ранг симплектической матрицы четен.}


\smallskip

\textit{Плоскость}~$\pi$ в~$V_n$~--- это множество вида $v\hm+L$, где~$v$ и~$L$~--- 
соответственно вектор и~подпространство пространства~$V_n$. Другими словами, 
плоскость~--- аффинное подпространство в~$V_n$.
\textit{Размерность плоскости} совпадает с~размерностью соответствующего 
подпространства: $\mathrm{dim}\,\pi=\mathrm{dim}\,L$.
Как известно, любая плоскость является решением некоторой системы линейных 
уравнений, и~на\-обо\-рот: решение произвольной системы линейных уравнений~--- 
плоскость в~соответствующем пространстве.
%То есть, плоскость является линейным многообразием.

Сужение функции $f\hm\in\mathcal{F}_n$ на плоскость~$\pi$ будем обозначать 
через~$f|_{\pi}$. Таким образом, $f|_{\pi}\colon \pi\hm\to\mathbb{F}_2$ 
и~$f|_{\pi}(u)\hm=f(u)$ для всех $u\hm\in\pi$.



\section{Локальная аффинность и~аффинная нормальная форма~булевой~функции}

В этом разделе вводятся понятия, связанные с~представлением произвольной булевой 
функции совокупностью аффинных функций, заданных для определенных плоскостей 
в~векторном пространстве. Более общее изложение этой теории можно найти 
в~работе~\cite{LYaD2007}.

\textit{Локальной аффинностью} функции~$f\hm\in\mathcal{F}_n$ будем называть такую 
плоскость~$\pi$, что $f|_{\pi}$ можно продолжить до аффинной функции, т.\,е.\ 
существует $l\hm\in\mathcal{A}_n$ со свойством $f|_{\pi}\hm=l|_{\pi}$.
Очевидно, для любой булевой функции существует разбиение пространства~$V_n$ на 
ее локальные аффинности.

Возьмем произвольное разбиение $\Pi\hm=\{\pi_1,\ldots,\pi_{\lambda}\}$ 
пространства~$V_n$ на плоскости, являющиеся локальными аффинностями булевой 
функции~$f$ от $n$~переменных.
Будем называть \textit{аффинной нормальной формой} функции~$f$ выражение вида
\begin{equation}
\label{AffNF}
f(x)=\bigoplus_{j=1}^{\lambda}\chi_{\pi_j}(x) l_j(x)\,,
\end{equation}
где для каждого~$j$, $1\hm\leqslant j\hm\leqslant\lambda$, функция~$l_j$ аффинна 
и~$f|_{\pi_j}(x)\hm=l_j|_{\pi_j}(x)$, а~$\chi_{\pi_j}$~--- характеристическая 
функция (индикатор) множества~$\pi_j$.
Функции~$l_j$ из этого выражения для краткости назовем\linebreak
 \textit{аффинными 
аппроксимациями} функции~$f$.
\textit{Длиной аффинной нормальной формы} называется число плоскостей 
в~разбиении~$\Pi$, далее она будет обозначаться через~$\lambda(\Pi)$.

\smallskip

\noindent
\textbf{Замечание~1.}
  Характеристические функции плоскостей известны также под именем 
<<мультиаффинных функций>> \cite{GT2017}, играющих важную роль при описании 
классов эффективно решаемых систем булевых уравнений.


\smallskip

Характеристическая функция плоскости в~пространстве~$V_n$ имеет вполне 
определенный вид. Любая плоскость~$\pi$, как уже отмечалось, может быть задана 
как множество решений системы $d$~линейных уравнений (для некоторого~$d$):
\begin{equation}
\left.
\begin{array}{c}
        h_1(x_1,\ldots,x_n)=0\,;\\
        h_2(x_1,\ldots,x_n)=0\,;\\
        \vdots\\
        h_d(x_1,\ldots,x_n)=0\,,\\
    \end{array}
    \right\}
    \label{chi-system}
\end{equation}
где все $h_i(x)\hm\in\mathcal{A}_n$. Поскольку вектор~$x$ принадлежит 
плоскости~$\pi$ тогда и~только тогда, когда все~$h_i$, $1\hm\leqslant i\hm\leqslant d$, 
обращаются в~нуль на нем, характеристическая функция~$\pi$ выражается следующим 
образом:
$$
\chi_{\pi}(x)=\prod\limits_{i=1}^d (h_i(x)\oplus 1)\,.
$$
Если система линейных уравнений задана в~мат\-рич\-ной форме: $xH\oplus 
(b_1,\ldots,b_d)\hm=0$, $b_i\hm=h_i(0)$, то выражение будет иметь вид:

\noindent
$$
\chi_{\pi}(x)=\prod\limits_{i=1}^d (xH_i\oplus b_i\oplus 1)\,,
$$
где $H_i$~--- столбцы матрицы~$H$.

Как видно из определения, аффинная нормальная форма представляет собой 
в~некотором смыс\-ле ку\-соч\-но-аф\-фин\-ную аппроксимацию булевой функции. На каждой 
локальной аф\-фин\-ности~$\pi_j$ из разбиения $\Pi$ все, кроме одного, слагаемые 
в~выражении~\eqref{AffNF} обращаются в~нуль, и~функция принимает вид 
$f(x)\hm=\chi_{\pi_j}(x)l_j(x)\hm=l_j(x)$ для всех $x\hm\in\pi_j$.

Возможность заменить на плоскости~$\pi_j$ квадратичное уравнение $f(x)\hm=0$ 
линейным уравнением $l_j(x)\hm=0$ вместе с~дописанной к~нему системой~\eqref{chi-system} 
будет использоваться при решении систем полиномиальных уравнений 
в~следующей статье.
Как сказано в~замечании~1, функции~$\chi_{\pi}(x)$, а~также 
и~слагаемые в~аффинной нормальной форме~\eqref{AffNF} являются мультиаффинными 
функциями. Тео\-ре\-ти\-ко-слож\-ност\-ные вопросы, связанные, в~частности, с~решением 
систем мультиаффинных уравнений, а~также оценки числа таких функций 
рассматриваются в~работе~\cite{Gor1995}.

При аффинном преобразовании пространства аффинные нормальные формы сохраняются 
в~том смысле, что выражение, полученное после применения преобразования к~этой 
форме, тоже будет аффинной нормальной формой для некоторой функции.

\smallskip

\noindent
\textbf{Предложение~2.}
\textit{Пусть $\varphi\in\mathrm{GA}(V_n)$ и~$f(x)\hm=
\bigoplus_{j=1}^{\lambda(\Pi)}\chi_{\pi_j}(x) l_j(x)$~--- некоторая 
аффинная нормальная форма функции~$f$.
  Тогда} 
$$
f^{\varphi}(x)=f(\varphi(x))=\bigoplus_{j=1}^{\lambda(\Pi)}\chi_{\pi_j}(\varphi(x)) 
l_j(\varphi(x))$$ 
\textit{есть аффинная нормальная форма функции~$f^{\varphi}$}.


\smallskip

\noindent
Д\,о\,к\,а\,з\,а\,т\,е\,л\,ь\,с\,т\,в\,о\,.\ \ 
Множество $\Pi'\hm=\{\pi'_j\hm=\varphi^{-1}(\pi_j) \mid \pi_j\in\Pi\}$ является 
разбиением пространства~$V_n$ на $\lambda(\Pi)$ плоскостей, поскольку~$\varphi$~--- 
не\-вы\-рож\-ден\-ное аффинное преобразование.
Заметим,\linebreak
 что $\varphi(x)\in\pi_j$ тогда и~только тогда, когда $x\hm\in\varphi^{-1}
(\pi_j)\hm=\pi'_j$.
Поэтому $\chi^{\varphi}_{\pi_j}$~--- характеристическая функция 
плоскости~$\pi'_j$.
Выражение для~$f^{\varphi}$ в~новых обозначениях выглядит следующим образом:
$$
f^{\varphi}(x)=\bigoplus_{j=1}^{\lambda(\Pi')}\chi_{\pi'_j}(x) l_j^{\varphi}(x)\,.
$$
Так как, очевидно, функции $l_j^{\varphi}(x)\hm=l_j(\varphi(x))$ аффинны, полученное 
выражение представляет собой аффинную нормальную форму.


\section{Теорема Диксона и~приведение квадратичных функций к~каноническому 
виду}\label{Dickson}

Благодаря теореме Диксона можно для любой квадратичной булевой функции~$f$ найти 
ее каноническое представление, в~котором она выглядит наиболее просто. Как будет 
видно ниже, это представление~--- элемент из орбиты данной функции 
$\mathrm{Orb}_f(\mathrm{GA}(V_n))$.
Канонический вид квадратичной функции, в~свою очередь, подсказывает прос\-той 
способ построения ее аффинной нормальной \mbox{формы.}

\smallskip

\noindent
\textbf{Теорема~1}\ [10, \S\;199].
\textit{Для любой квад\-ра\-тич\-ной функции~$f\hm\in\mathcal{F}_n$ с~ненулевой 
матрицей~$\tilde{Q}_f$ существует аффинное 
преобразование~$\alpha\hm\in\mathrm{GA}(V_n)$, которое приводит~$f$ к~одному из  
канонических представлений}:
$$f^{\alpha}(x)=x_1x_2\oplus x_3x_4\oplus\cdots\oplus x_{2r-1}x_{2r}\oplus c
$$
\textit{или}
$$f^{\alpha}(x)=x_1x_2\oplus x_3x_4\oplus\cdots\oplus x_{2r-1}x_{2r}\oplus 
x_{2r+1}\,,
$$
где $2r=\rank \tilde{Q}_f$ и~$c\hm\in\mathbb{F}_2$.

\smallskip

Доказательство этого утверждения помимо авторского варианта можно найти также 
в~[6, \S\;3.3; 7, \S\;15.2].

На практике приведение полинома Жегалкина квадратичной булевой функции 
к~каноническому виду можно осуществить следующим способом.

Предположим, не ограничивая общности, что в~полиноме Жегалкина функции~$f$ 
присутствует моном~$x_1x_2$ (иначе с~помощью аффинного преобразования координат 
<<перенумеруем>> переменные).
Представим функцию в~виде:
\begin{multline*}
f(x)=x_1x_2\oplus x_1l_1(x_3,\ldots,x_n) \oplus x_2l_2(x_3,\ldots,x_n) \oplus{}\\
{}\oplus 
q_1(x_3,\ldots,x_n)\,,
\end{multline*}
где $l_1,l_2\in\mathcal{A}_{n-2}$, а~$q_1$~--- некоторая квадратичная функция.
Возьмем отображение~$\varphi_2$ пространства~$V_n$, задаваемое равенством:
\begin{multline*}
\varphi_2(x)=\left(x_1\oplus l_2(x_3,\ldots,x_n),\ x_2\oplus {}\right.\\
\left.{}\oplus
l_1(x_3,\ldots,x_n),\ x_3,\ldots,\ x_n\right)\,,
\end{multline*}
и рассмотрим следующую функцию:
$$
f^{(2)}(x)=x_1x_2\oplus q_2(x_3,\ldots,x_n)\,,
$$
где $q_2=q_1\oplus l_1l_2$~--- квадратичная функция.
Заметим, что $(f^{(2)})^{\varphi_2}\hm=f$.

Затем аналогично предыдущему выделим первые две переменные в~функции 
$q_2(x_3,\ldots,x_n)$.
Здесь берется отображение
\begin{multline*}
\varphi_4(x)=\bigl(x_1,\ x_2,\  x_3\oplus l_4(x_5,\ldots,x_n),\\
x_4\oplus 
l_3(x_5,\ldots,x_n),\ x_5,\ldots,\ x_n\bigr)
\end{multline*}
и функция
$$
f^{(4)}(x)=x_1x_2\oplus  x_3x_4\oplus q_4(x_5,\ldots,x_n)\,,
$$
так что $(f^{(4)})^{\varphi_4}=f^{(2)}$.

Проделываем это до тех пор, пока на некотором шаге не получим аффинную функцию
$$
q_{2r}(x_{2r+1},\ldots,x_n)=\bigoplus_{i=2r+1}^{n}b_ix_i\oplus c
$$
для некоторых $b_i,c\hm\in\mathbb{F}_2$.
Если $b_i\hm=0$ для всех~$i$, $2r\hm+1\hm\leqslant i\hm\leqslant n$, 
то искомый канонический вид 
найден: это функция~$f^{(2r)}$.
Иначе считаем, без ограничения общности, что $b_{2r+1}\hm=1$ и~полагаем
\begin{multline*}
\varphi_{2r+1}(x)=\left(x_1,\ldots,\ x_{2r},\ 
q_{2r}\left(x_{2r+1},\ldots,x_n\right),\right.\\ 
\left.x_{2r+2},\ldots,\ x_n\right)\,.
\end{multline*}
Тогда канонический вид для~$f$~--- это функция
\begin{multline*}
g(x)={}\\
{}=f^{(2r+1)}(x)=x_1x_2\oplus  x_3x_4\oplus \cdots \oplus x_{2r-1}x_{2r} 
\oplus x_{2r+1},
\end{multline*}
причем если положить $\varphi\hm=\varphi_{2r+1}\varphi_{2r}
\varphi_{2r-2}\cdots\varphi_2$, то
$$
g^{\varphi}=\left(\cdots(g^{\varphi_{2r+1}})^{\varphi_{2r}}\cdots\right)^{\varphi_2}=f.$$

%$g^{\varphi}(x)=g\bigl(\varphi_{2r+1}(\ldots\varphi_2(x)\ldots)\bigr)$.
Преобразование~$\varphi$, очевидно, аффинно, невырожденно и~имеет вид:
$$    \hspace*{-33mm}\varphi(x) ={}\hspace*{33mm}
$$
\begin{equation*}
      \begin{split}
    {}=
    x &
    {
      \begin{pmatrix}
\makebox[1.5em]{$1$} &\rule{1.5em}{0pt} & \rule{1.5em}{0pt}   & 
\rule{1.5em}{0pt}   & \rule{1.5em}{0pt}   &  & \rule{1.5em}{0pt}   & 
\rule{1.5em}{0pt}   & \rule{1.5em}{0pt}   &  \rule{1.5em}{0pt}   \\
        0 & 1 &   &   &   &   &   &   &   &   \\
        * & * &\smash[t]{\ddots}&&   &   &   &   & 
\smash[t]{\mbox{\Huge{$0$}}}  &   \\
        * & * & \smash[t]{\ddots}  & 1 &   &   &   &   &   &      \\
        * & * &\smash[t]{\ddots}   & 0 & 1 &   &   &   &   &      \\
        * & * &\smash[t]{\ddots}   & * & * & 1 &   &   &   &   \\
        * & * &\smash[t]{\ddots}   & * & * & \makebox[1.5em]{$b_{2r+2}$}  & 1 &   &  &     \\
        * & * & \smash[t]{\ddots}  & * & * & \makebox[1.5em]{$b_{2r+3}$}  & 0 
&\smash[t]{\ddots}&& \\
        \vdots  &\vdots   & \smash[t]{\ddots}  &\vdots   &\vdots   & \vdots  & \vdots  &\ddots& 
1 &     \\
        * & * &\cdots   & * & * &b_n& 0 &\cdots & 0 & 1\\
      \end{pmatrix}} \oplus \\
   \oplus &
      \;\begin{pmatrix}
      \makebox[1.5em]{$*$}&\makebox[1.5em]{$*$}&\makebox[1.5em]{$\cdots$}&\makebox[1.5
em]{$*$}&\makebox[1.5em]{$*$}&\makebox[1.5em]{$c$}&\makebox[1.5em]{$0$}&\makebox
[1.5em]{$\cdots$}&\makebox[1.5em]{$0$}&\makebox[1.5em]{$0$} \\
      \end{pmatrix}
  \end{split}
\end{equation*}
(здесь знак~$*$ заменяет собой один из элементов~$\mathbb{F}_2$, каждый раз 
свой). Соответственно, преобразование~$\alpha$ из формулировки теоремы Диксона 
является обратным к~$\varphi$.

Как будет показано в~разд.~\ref{canonic-to-ANF}, представление функций~$f_i$ 
в~таком виде, т.\,е.\ нахождение подходящих представителей орбиты 
$\mathrm{Orb}_{f_i}(\mathrm{GA}(V_n))$, позволяет легко выписать аффинные 
нормальные формы для~$f_i$.

\section{Построение аффинной нормальной формы для~квадратичной 
функции}\label{canonic-to-ANF}

Обозначим через $\varphi_i$, $1\hm\leqslant i\hm\leqslant m$, невырожденные аффинные преобразования 
пространства~$V_n$, с~помощью которых функции~$f_i$ приводятся к~каноническому 
виду~$g_i$:
$$
g_i(x)=f_i^{\varphi_i}(x)=x_1x_2\oplus\cdots\oplus x_{2r_i-1}x_{2r_i}\oplus 
b_ix_{2r_i+1}\oplus c_i\,,
$$
где $2r_i=\rank \tilde{Q}_{f_i}$, а $b_i, c_i\hm\in \mathbb{F}_2$.

Очевидно, что если среди первых~$2r_i$ переменных взять все переменные с~четными 
индексами или все с~нечетными и~зафиксировать их значения, то получится 
плоскость, являющаяся локальной аффинностью функции~$g_i$.
Рассмотрим, например, $2^{r_i}$ плоскостей, заданных уравнениями:
$$    \begin{array}{l@{\,}c@{\ }l}
        x_1&=&\delta_1;\\
        x_3&=&\delta_2;\\
        \vdots\\
        x_{2r_i-1}&=&\delta_{r_i},\\
    \end{array}
$$
где $\delta_j\in\mathbb{F}_2$, $1\hm\leqslant j\hm\leqslant r_i$.
Каждую из этих плоскостей обозначим через~$\pi'_{i,\delta}$ со сложным индексом 
$\delta\hm=(\delta_1,\dots,\delta_{r_i})\in\mathbb{F}_2^{r_i}$.
Размерность~$\pi'_{i,\delta}$ равна $n\hm-r_i$, а мощность, соответственно, 
$2^{n-r_i}$.
Нетрудно видеть, что $\Pi'_i\hm=\{\pi'_{i,\delta}\}_{\delta\in\mathbb{F}_2^{r_i}}$ 
является разбиением пространства~$V_n$.

Обозначаемое ниже через~$l'_{i,\delta}$ сужение функции~$g_i$ на каждую из 
плоскостей разбиения~--- аффинно:
\begin{multline*}
l'_{i,\delta}(x)={}\\
{}=g_i|_{\pi'_{i,\delta}}(x)=\delta_1x_2\oplus\delta_2x_4\cdots\oplus\delta_{r_i}x_{2r_i}\oplus b_ix_{2r_i+1}\oplus c_i,
\hspace*{-0.80452pt}
\end{multline*}
а характеристическая функция соответствующей плоскости имеет вид:
$$
\chi_{\pi'_{i,\delta}}(x)=\prod\limits_{k=1}^{r_i}\left(x_{2k-1}\oplus\delta_k\oplus1\right)\,.
$$

С помощью аффинной нормальной формы
$$
g_i(x)=\bigoplus_{\delta\in\mathbb{F}_2^{r_i}} 
\chi_{\pi'_{i,\delta}}(x)l'_{i,\delta}(x)
$$
для канонического представления функции~$f_i$ можно аффинным преобразованием, 
обратным к~$\varphi_i$, получить аффинную нормальную форму для исходной функции:

\vspace*{1pt}

\noindent
$$
f_i(x)=g_i^{\varphi_i^{-1}}(x) = \bigoplus_{\delta\in\mathbb{F}_2^{r_i}} 
\chi_{\pi_{i,\delta}}(x)l_{i,\delta}(x)\,,
$$

\vspace*{-3pt}

\noindent
где $\pi_{i,\delta}\hm=\varphi_i(\pi'_{i,\delta})$ 
и~$l_{i,\delta}(x)\hm=l'_{i,\delta}(\varphi_i^{-1}(x))$.

Разумеется, если алгебраическая степень какой-либо функции~$f_i$ оказалась 
равной~$1$, то искать ничего не нужно: ее полином Жегалкина является ее аффинной 
нормальной формой для тривиального разбиения $\Pi_i\hm=\{V_n\}$.

\smallskip

\noindent
\textbf{Замечание~2}.
    Подобный способ построения аффинной нормальной формы можно использовать 
    и~непосредственно для квадратичной\footnote{Для функций более высоких степеней 
такой подход тоже работает, но описать его строго гораздо сложнее и~полученные 
таким образом локальные аффинности, скорее всего, будут слишком маленькой 
размерности.} функции~$f$ в~ее исходном виде. Нужно просто фиксировать значения 
переменных так, чтобы в~каждом мономе оставалось не более одной свободной 
переменной. Для этого удобнее рассмотреть матрицу~$Q_f$, выбрать в~ней столбец 
или строку с~максимальным числом единиц среди всех столбцов и~строк (пусть это 
будет $k$-я строка) и~зафиксировать~$x_k$. Затем то же проделать, исключив из 
рассмотрения $k$-ю строку и~$k$-й столбец матрицы, и~так далее, пока единицы 
в~матрице не кончатся.
Однако, несмотря на то что здесь имеет место экономия на приведении функции 
к~каноническому виду, такой способ представляется менее эффективным в~следующем 
смысле. Канонический вид квадратичной функции содержит минимальное число мономов 
степени~$2$, поэтому для исходной (неканонической) функции придется фиксировать, 
как правило, большее число переменных. Но с~каждой дополнительно зафиксированной 
переменной размерность локальных аффинностей функции~$f$ уменьшается на~$1$, 
а~их число, соответственно, увеличивается вдвое.

\vspace*{-4pt}


\section{<<Локальные>> системы линейных уравнений}

\vspace*{-2pt}

Идея метода решения систем квадратичных булевых уравнений состоит в~следующем.
Пусть для всех~$f_i$, $1\hm\leqslant i\hm\leqslant m$, 
определены некоторые аффинные нормальные 
формы

\vspace*{1pt}

\noindent
\begin{equation*}
\label{AffNF_ij}
    f_i(x) = \bigoplus_{j=1}^{\lambda(i)} \chi_{\pi_{ij}}(x)l_{ij}(x)\,.
\end{equation*}

\vspace*{-3pt}

\noindent
Исходя из этих аффинных нормальных форм, можно для каждой пары~$i,j$ записать 
эквивалентную уравнению $f_i\hm=0$ на~$\pi_{ij}$ систему линейных уравнений:

\columnbreak

\noindent
\begin{equation*}
    \begin{array}{r@{\ }c@{\ }l}
        l_{ij}(x)&=&0;\\
        h_{ij}^1(x)&=&0;\\
        \vdots\\
        h_{ij}^{d(i,j)}(x)&=&0,\\
    \end{array}
  \label{approx}
\end{equation*}
в которой первое уравнение выражает равенство~$f_i\hm=0$ через аффинную 
аппроксимацию~$l_{ij}(x)$ функции~$f_i(x)$ на плоскости~$\pi_{ij}$, а остальные 
$d(i,j)$ уравнений задают эту плоскость.


\smallskip

\noindent
\textbf{Замечание~3.}\
Если аффинная нормальная форма получена описанным выше способом~--- через 
канонический вид квадратичной функции,~--- то характеристическая функция будет 
иметь вид:

\vspace*{1pt}

\noindent
$$
\chi_{\pi_{i,\delta}}(x)=\prod_{k=1}^{r_i}(\varphi_i^{-1}(x)e_{2k-1}
\Tsf\oplus\delta_k\oplus 1)\,,
$$

\vspace*{-3pt}

\noindent
где $e_{2k-1}$~--- $(2k-1)$-й базисный вектор, т.\,е.\ $\varphi_i^{-1}(x)e_{2k-1}
\Tsf$~--- $(2k-1)$-я компонента вектора~$\varphi_i^{-1}(x)$.
Значит, соответствующую плоскость задают уравнения
 $\{ \varphi_i^{-1}(x)e_{2k-1}\Tsf \oplus \delta_k \hm= 0 
 \mid 1\hm\leqslant k\hm\leqslant r_i \}$.

\smallskip

Таким образом, имеется набор <<локальных>> линейных систем для каждого уравнения 
исходной системы и~для каждой его локальной аффинности.
Метод состоит в~том, чтобы подобрать комбинацию <<локальных>> систем разных 
квадратичных уравнений, в~совокупности дающую решение исходной системы. Если 
решение квадратичной системы единственно (а~это естественное предположение для 
криптоанализа), ровно одна такая комбинация будет иметь решение, и~от того, как 
быстро удастся ее обнаружить, зависит эффективность метода.

\vspace*{-4pt}

\section{О~трудоемкости построения аффинной нормальной формы}

\vspace*{-2pt}

Напомним, что для функций из системы~\eqref{system} $r_i\hm=({1}/{2})\rank 
\tilde{Q}_{f_i}\hm\leqslant {n}/{2}$, $1\hm\leqslant i\hm\leqslant m$,~--- 
параметр, введенный в~разд.~\ref{canonic-to-ANF}.
Алгоритм приведения $m$~функций к~каноническому виду (см.\ разд.~4) 
имеет трудоемкость, оцениваемую выражением $O(\sum\nolimits_{i=1}^m n^2r_i)$, 
а~учитывая 
неравенство $r_i\hm\leqslant {n}/{2}$, имеем~$O(mn^3)$.

При построении аффинных нормальных форм для функции~$f_i$ в~разд.~5 
потребуется порядка $r_i2^{r_i}\hm+ n^3$ операций. 
Значит, для всех~$m$~функций имеем оценку $O(mn^3\hm+\sum\nolimits_{i=1}^m r_i2^{r_i})$.

Таким образом, в~худшем случае, когда все $r_i\hm={n}/{2}$ или даже когда хотя 
бы $r_i\hm=O(n)$ для некоторого~$i$, предложенный алгоритм экспоненциален.
Однако можно рассчитывать, что во встречающихся на практике системах 
квадратичных уравнений параметр~$r_i$ растет (с~увеличением~$n$) медленнее, 
и~тогда можно говорить о полиномиальности алгоритма построения аффинных нормальных 
форм.

В случае, когда система вида~\eqref{system} переопределенная, т.\,е.\ $n\hm\ll m$ 
(переопределенные системы достаточно часто рассматриваются в~задачах 
информатики, теории кодирования и~криптографии), можно рассчитывать на 
существование подсистемы (из~$l$~уравнений с~номерами $i_1,\ldots,i_l$), для 
которой трудоемкость построения аффинных нормальных форм меньше, чем 
экспоненциальная. Например, когда $r_{i_j}\hm=O(\sqrt{n})$, $ 1\hm\leqslant j\hm\leqslant l$, 
оценка со\-от\-вет\-ст\-ву\-ющей трудоемкости для системы~\eqref{system} имеет 
субэкспоненциальный характер.

Рассмотрим в~качестве еще одного примера класс~$\mathcal{K}_m$ систем 
$m$~квадратичных булевых уравнений от $n$~неизвестных вида~\eqref{system}, где 
$m\hm=m(n)$~--- некоторый полином от~$n$ и~где $r_i=
O(\log_2 n)$ для всех~$i$, $1\hm\leqslant i\hm\leqslant m$.


\vspace*{2pt}


\noindent
\textbf{Предложение~3.}
\textit{Для систем~\eqref{system} квадратичных булевых уравнений из 
класса~$\mathcal{K}_m$ существует полиномиальный} (\textit{по~$n$}) \textit{алгоритм построения 
аффинных нормальных форм для функций~$f_i$.}


\smallskip

Для доказательства этого утверждения достаточно рассмотреть предложенный 
в~статье алгоритм построения аффинных нормальных форм для квад\-ра\-тич\-ных булевых 
функций. В~полученной выше оценке  $O(mn^3\hm+\sum\nolimits_{i=1}^m r_i2^{r_i})$
данные в~условии ограничения на~$m$ и~на~$r_i$ дают полиномиальную оценку трудоемкости 
алгоритма.

%Отметим, что если рассматривать систему квадратичных уравнений, описывающую 
%функционирование произвольного фильтрующего генератора, то у всех уравнений 
%системы будет одно и~то же значение $r_i$, определяемое рангом матрицы 
%$\tilde{Q}_{f'}$, где $f'$ "--- ... для фильтрующей функции~$f$. Поэтому

\vspace*{-12pt}

{\small\frenchspacing
 {%\baselineskip=10.8pt
 \addcontentsline{toc}{section}{References}
 \begin{thebibliography}{99}

    \bibitem{GJ1982}
        \Au{Гэри~М., Джонсон~Д.}
        Вычислительные машины и~труднорешаемые задачи~/ Пер. с~англ.~---
        М.: Мир, 1982. 416~с.
        (\Au{Garey~M.\,R., Johnson~D.\,S.} Computers and intractability: 
A~guide to the theory of NP-completeness.~--- San Francisco, CA, USA: W.\,H.~Freeman 
and Co., 1979. 348~p.).

    \bibitem{GT2017}
        \Au{Горшков~С.\,П., Тарасов~А.\,В.}
        Сложность решения сис\-тем булевых уравнений.~---
        М.: Курс, 2017. 192~с.

    \bibitem{Smi2000}
        \Au{Смирнов~В.\,Г.}
        {Некоторые классы эффективно ре\-ша\-емых систем булевых уравнений}~//
        Труды по дискретной математике, 2000. Т.~3. С.~269--282.

    \bibitem{Bard2009}
        \Au{Bard~G.\,V.}
        Algebraic cryptanalysis.~--- Springer, 2009. 389~p.

    \bibitem{BCJ2007}
        \Au{Bard~G., Courtois~N., Jefferson~C.}
        {Efficient methods for conversion and solution of sparse systems of 
        low-degree multivariate polynomials over $\mathrm{GF}(2)$ via SAT-solvers}~//
        Cryptology ePrint Archive. Report 2007/024.
        {\sf http://eprint.iacr.org/2007/024.pdf}.

    \bibitem{LSSYa2015}
        \Au{Логачев~О.\,А., Сальников~А.\,А., Смышляев~С.\,В., 
Ященко~В.\,В.}
        Булевы функции в~теории кодирования и~крип\-то\-ло\-гии.~---
        М.: ЛЕНАНД, 2015. 576~с.

    \bibitem{MWS1979}
        \Au{Мак-Вильямс~Ф.\,Дж., Слоэн~Н.\,Дж.\,А.}
        Теория кодов, исправляющих ошибки~/ Пер. с~англ.~---
        М.: Связь, 1979. 743~с.
        (\Au{MacWilliams~F.\,J., Sloane~N.\,J.\,A.} The theory of 
        error-correcting codes.~--- 
        North-Holland mathematical library ser.~---
        North-Holland Publishing Co., 1977.  774~p.)

    \bibitem{LYaD2007}
        \Au{Logachev~O.\,A., Yashchenko~V.\,V., Denisenko~M.\,P.}
        {Local affinity of Boolean mappings}~//
        Boolean functions in cryptology and information security: Proceedings of the 
NATO Advanced Study Institute.~---
        IOS Press, 2008. P.~148--172.

    \bibitem{Gor1995}
        \Au{Горшков~С.\,П.}
        {Применение теории NP-пол\-ных задач для оценки сложности решения систем 
булевых уравнений}~//
        Обозрение прикладной и~промышленной математики, 1995. Т.~2. Вып.~3. 
С.~325--398.

    \bibitem{Dickson1901}
        \Au{Dickson~L.\,E.}
        Linear groups: With an exposition of the Galois field theory.~---
        Leipzig: B.\,G.\,Teubner, 1901. 322~p.

   % \bibitem{KSh1999}
       % \Au{Kipnis~A., Shamir~A.}
      %  {Cryptanalysis of the HFE public key cryptosystem by relinearization}~//
     %   Advances in cryptology~/
    %    Ed.\ M.\,J.~Wiener.~---
   %     Lectures notes in computer science ser.~---
   %     Springer, 1999. Vol.~1666. P.~19--30.

   % \bibitem{CShPK2000}
  %      \Au{Courtois~N., Klimov~A., Patarin~J., Shamir~A.}
 %       {Efficient algorithms for solving overdefined systems of multivariate 
%polynomial equations}~// Advances in cryptology~/
%Ed.\ B.~Preneel.~---
%         Lectures notes in computer science ser.~--- Springer, 2000. Vol.~1807. 
%P.~392--407.

   % \bibitem{FY1980}
  %      \Au{Fraenkel~A.\,S., Yesha~Y.}
 %       {Complexity of solving algebraic equations}~//
 %       Inform. Process. Lett., 1980. Vol.~10. Iss.~4-5. P.~178--179.

\end{thebibliography} 
 }
 }

\end{multicols}

\vspace*{-3pt}

\hfill{\small\textit{Поступила в~редакцию 11.01.19}}

\vspace*{8pt}

%\pagebreak

%\newpage

%\vspace*{-28pt}

\hrule

\vspace*{2pt}

\hrule

%\vspace*{-2pt}

\def\tit{POLYNOMIAL ALGORITHMS FOR~CONSTRUCTING LOCAL AFFINITIES OF~QUADRATIC BOOLEAN FUNCTIONS}

\def\titkol{Polynomial algorithms for~constructing local affinities of~quadratic Boolean functions}

\def\aut{O.\,A.~Logachev$^{1,2}$, A.\,A.~Sukayev$^1$, and~S.\,N.~Fedorov$^1$}

\def\autkol{O.\,A.~Logachev, A.\,A.~Sukayev, and~S.\,N.~Fedorov}

\titel{\tit}{\aut}{\autkol}{\titkol}

\vspace*{-11pt}


\noindent
$^1$Information Security Institute,  M.\,V.~Lomonosov Moscow State University, 
1~Michurinskiy Prosp., Moscow\linebreak
$\hphantom{^1}$119192, Russian Federation

\noindent
$^2$Institute of Informatics Problems, 
Federal Research Center ``Computer Science and Control'' 
of the Russian\linebreak
$\hphantom{^1}$Academy of Sciences, 44-2~Vavilov Str., Moscow 119333, 
Russian Federation

\def\leftfootline{\small{\textbf{\thepage}
\hfill INFORMATIKA I EE PRIMENENIYA~--- INFORMATICS AND
APPLICATIONS\ \ \ 2019\ \ \ volume~13\ \ \ issue\ 1}
}%
 \def\rightfootline{\small{INFORMATIKA I EE PRIMENENIYA~---
INFORMATICS AND APPLICATIONS\ \ \ 2019\ \ \ volume~13\ \ \ issue\ 1
\hfill \textbf{\thepage}}}

\vspace*{6pt}


\Abste{Due to the affine normal form, one can consider a~Boolean function 
as affine on certain flats in its domain~--- so-called local affinities. 
This Boolean function representation~--- affine approximation~---
could be
useful 
for solving systems of nonlinear equations over two-element field. The problem 
of solving these systems
(of a~special sort) arises, in particular, in some methods 
of the information security tools design and analysis.
The
paper describes an approach to finding local affinities for quadratic Boolean 
functions which is based on Dickson's\linebreak\vspace*{-12pt}}

\Abstend{theorem. By this, one obtains affine 
normal forms for such functions. Besides, the paper concerns the efficiency of 
corresponding algorithms.
This approach can be profitable for constructing efficient methods of solving 
systems of quadratic Boolean equations via ``approximation'' of corresponding 
Boolean functions by their affine normal forms.}

\KWE{Boolean function; system of quadratic Boolean equations; vector 
space partition; flat; local affinity; Dickson's theorem; 
affine normal form (ANF) of Boolean function; algebraic cryptanalysis}






\DOI{10.14357/19922264190110}

\vspace*{-14pt}

\Ack
\noindent
The paper was partly supported by the Russian Foundation for Basic Research 
(project 18-29-03124~mk).





  \begin{multicols}{2}

\renewcommand{\bibname}{\protect\rmfamily References}
%\renewcommand{\bibname}{\large\protect\rm References}

{\small\frenchspacing
 {%\baselineskip=10.8pt
 \addcontentsline{toc}{section}{References}
 \begin{thebibliography}{99}
\bibitem{1-log-1}
\Aue{Garey, M.\,R., and D.\,S.~Johnson.} 1979. \textit{Computers and intractability: 
A~guide to the theory of NP-completeness.} San Francisco, CA: W.\,H.~Freeman and Co. 348~p.
\bibitem{2-log-1}
\Aue{Gorshkov, S.\,P., and A.\,V.~Tarasov.} 2017. \textit{Slozhnost' re\-she\-niya 
sistem bulevykh uravneniy} [Complexity of solving the systems of 
Boolean equations]. Moscow: Kurs. 192~p.
\bibitem{3-log-1}
\Aue{Smirnov, V.\,G.} 2000. Nekotorye klassy effektivno reshaemykh 
sistem bulevykh uravneniy [Some classes of Boolean equation systems 
permitting effective solution]. 
\textit{Trudy po diskretnoy matematike} [Proceedings on Discrete Mathematics] 3:269--282.
\bibitem{4-log-1}
\Aue{Bard, G.\,V.} 2009. \textit{Algebraic cryptanalysis}. Springer. 389~p.
\bibitem{5-log-1}
\Aue{Bard, G., N.~Courtois, and C.~Jefferson.} 2007. 
Efficient methods for conversion and solution of sparse systems of 
low-degree multivariate polynomials over GF(2) via SAT-solvers. 
\textit{Cryptology ePrint Archive}. Report 2007/024. Available at: 
{\sf http://eprint.iacr.org/2007/024.pdf} (accessed August~30, 2018).
\bibitem{6-log-1}
\Aue{Logachev, O.\,A., A.\,A.~Sal'nikov, S.\,V.~Smyshlyaev, and V.\,V.~Yashchenko.} 
2015. \textit{Bulevy funktsii v~teorii kodirovaniya i~kriptologii} 
[Boolean functions in coding theory and cryptology]. Moscow: LENAND. 576~p.
\bibitem{7-log-1}
\Aue{MacWilliams, F.\,J., and N.\,J.\,A.~Sloane.} 1977. 
\textit{The theory of error-correcting codes}. 
North-Holland mathematical library ser.
North-Holland Publishing Co. 774~p.
\bibitem{8-log-1}
\Aue{Logachev, O.\,A., V.\,V.~Yashchenko, and M.\,P.~Denisenko.} 2008. 
Local affinity of Boolean mappings. 
\textit{Boolean functions in cryptology and information security: 
Proceedings of the NATO Advanced Study Institute.} IOS Press. 148--172.
\bibitem{9-log-1}
\Aue{Gorshkov, S.\,P.} 1995. Primenenie teorii NP-polnykh zadach 
dlya otsenki slozhnosti resheniya sistem bulevykh uravneniy 
[Application of the NP-complete problem theory to assessment 
of complexity of solving the systems of Boolean equations]. 
\textit{Obozrenie prikladnoy i~promyshlennoy matematiki} 
[Applied and Industrial Mathematics Review] 2(3):325--398.
\bibitem{10-log-1}
\Aue{Dickson, L.\,E.} 1901. \textit{Linear groups: 
With an exposition of the Galois field theory}. Leipzig: B.\,G.~Teubner. 322~p.
%\bibitem{11-log-1}
%\Aue{Kipnis, A., and A.~Shamir.} 1999. 
%Cryptanalysis of the HFE public key cryptosystem by relinearization. 
%\textit{Advances in cryptology}. Ed. M.\,J.~Wiener.
% Lecture notes in computer science ser.  Springer. 1666:19--30.
%\bibitem{12-log-1}
%\Aue{Courtois, N., A.~Klimov, J.~Patarin, and A.~Shamir.} 2000. 
%Efficient algorithms for solving overdefined systems of multivariate polynomial 
%equations. \textit{Advances in cryptology}. Ed.\ B.~Preneel.
%Lecture notes in computer science ser.  Springer. 1807:392--407.
%\bibitem{13-log-1}
%\Aue{Fraenkel, A.\,S., and Y.~Yesha.} 1980. 
%Complexity of solving algebraic equations. 
%\textit{Inform. Process. Lett.} 10(4-5):178--179.
\end{thebibliography}

 }
 }

\end{multicols}

\vspace*{-6pt}

\hfill{\small\textit{Received January 11, 2019}}

%\pagebreak

%\vspace*{-18pt}

\Contr

\noindent
\textbf{Logachev Oleg A.} (b.\ 1950)~--- 
Candidate of Science (PhD) in physics and mathematics, head of department, 
Information Security Institute, M.\,V.~Lomonosov Moscow State University, 
1~Michurinskiy Prosp., Moscow 119192, Russian Federation; 
senior scientist, Institute of Informatics Problems, 
Federal Research Center ``Computer Science and Control'' 
of the Russian Academy of Sciences, 44-2~Vavilov Str., Moscow 119333, 
Russian Federation; \mbox{logol@iisi.msu.ru }

 



\vspace*{3pt}

\noindent
\textbf{Sukayev Al'bert A.} (b.\ 1994)~--- 
student, Information Security Institute, Moscow State University, 
1~Michurinskiy Prosp., Moscow 119192, Russian Federation; 
\mbox{asukaev@gmail.com}

\vspace*{3pt}

\noindent
\textbf{Fedorov Sergey~N.} (b.\ 1982)~--- 
Candidate of Science (PhD) in physics and mathematics, senior scientist, 
Information Security Institute, M.\,V.~Lomonosov Moscow State University, 
1~Michurinskiy Prosp., Moscow 119192, Russian Federation; 
\mbox{s.n.feodorov@yandex.ru}
\label{end\stat}

\renewcommand{\bibname}{\protect\rm Литература}        %6
\def\stat{abgaryan}

\def\tit{ПРОГРАММНЫЙ КОМПЛЕКС ДЛЯ~МНОГОМАСШТАБНОГО МОДЕЛИРОВАНИЯ 
СТРУКТУРНЫХ СВОЙСТВ КОМПОЗИЦИОННЫХ МАТЕРИАЛОВ$^*$}

\def\titkol{Программный комплекс для многомасштабного моделирования 
структурных свойств композиционных материалов}

\def\aut{К.\,К.~Абгарян~$^1$, Е.\,С.~Гаврилов$^2$}

\def\autkol{К.\,К.~Абгарян, Е.\,С.~Гаврилов}

\titel{\tit}{\aut}{\autkol}{\titkol}

\index{Абгарян К.\,К.}
\index{Гаврилов Е.\,С.}
\index{Abgaryan K.\,K.}
\index{Gavrilov E.\,S.}


{\renewcommand{\thefootnote}{\fnsymbol{footnote}} \footnotetext[1]
{Работа выполнена при поддержке Министерства науки и~высшего образования Российской Федерации (проект 
075-15-2020-799).}}


\renewcommand{\thefootnote}{\arabic{footnote}}
\footnotetext[1]{Федеральный исследовательский центр <<Информатика и~управление>> Российской академии наук, 
\mbox{kristal83@mail.ru}}
\footnotetext[2]{Федеральный исследовательский центр <<Информатика и~управление>> Российской академии наук; 
Московский авиационный институт (национальный исследовательский университет), \mbox{eugavrilov@gmail.com}}

%\vspace*{-6pt}
    
      
         
      
      \Abst{Создание новых композиционных материалов (КМ) с~прогнозируемыми свойствами 
      и~разработка способов их конструирования на сегодня стали одними из актуальных и~важнейших 
задач, связанных с~модернизацией промышленного производства в~нашей стране. Для их 
решения активно развиваются технологии многомасштабного компьютерного 
моделирования. Они стали связующим звеном между фундаментальной физикой (химией) 
и~инженерным материаловедением. В~работе представлен программный комплекс по 
моделированию структурных свойств КМ, поз\-во\-ля\-ющий решать ряд 
задач данного класса. Он ориентирован на высокопроизводительные вы\-чис\-ле\-ния. В~основе 
комплекса лежит оригинальная многомасштабная технология, которая позволяет оперативно 
проводить многовариантный анализ различных классов КМ 
и~проводить исследования по проектированию новых с~прогнозируемыми свойствами. 
Разработанные подходы в~сочетании с~экспериментальными данными могут быть использованы 
для лучшего понимания физических основ изменения свойств в~за\-ви\-си\-мости от структуры и,~как 
следствие, для удешевления и~ускорения поиска новых КМ
с~заданными свойствами.}
      
      \KW{многомасштабное моделирование; композиционные материалы; интеграционная 
платформа; программный комплекс; распределенная сис\-тема}

\DOI{10.14357/19922264220113}
  
%\vspace*{-3pt}


\vskip 10pt plus 9pt minus 6pt

\thispagestyle{headings}

\begin{multicols}{2}

\label{st\stat}

\section{Введение}

\vspace*{-3pt}

     Создание новых КМ с~прогнозируемыми 
свойствами и~разработка способов их конструирования на сегодня стали одними 
из актуальных и~важнейших задач по модернизации промышленного 
производства в~нашей стране. Особенно важны такие материалы в~областях, где 
соотношение между проч\-ностью и~массой конструкции определяет ее 
эф\-фек\-тив\-ность. На сегодня процессы создания КМ
непосредственно связаны с~этапом моделирования, включая применение наиболее 
эффективных методов многомасштабного компьютерного моделирования и~анализа данных. 
     
     Для решения данного класса задач разработан\linebreak программный комплекс по 
моделированию структурных свойств КМ. Он 
ориентирован на высокопроизводительные вы\-чис\-ле\-ния. В~осно\-ве комплекса 
лежит оригинальная многомасштабная \mbox{технология}, пред\-став\-лен\-ная в~[1, 2], 
которая позволяет оперативно проводить многовариантный анализ различных 
классов КМ. На базе разработанной технологии была 
создана распределенная информационная сис\-те\-ма для проведения 
многоуровневых исследований в~об\-ласти моделирования~КМ. 

Согласно разработанным подходам в~за\-ви\-си\-мости от типа 
мо\-де\-ли\-ру\-емо\-го КМ строится многомасштабная 
композиция и~ее схематическое представление. На ее основе в~программной среде 
формируется сценарий расчета структурных характеристик и~отдельных свойств 
рас\-смат\-ри\-ва\-емо\-го материала. Созданный программный комплекс позволяет 
автоматизировать уни\-фи\-ци\-ру\-емые этапы моделирования и~помогает 
сформировать на основе анализа полученных результатов более глубокое 
понимание физических процессов. Комплекс построен с~применением 
современных программных средств и~решений и~не уступает международному 
уровню на\-уч\-но-тех\-ни\-че\-ских разработок в~об\-ласти информационной 
поддержки для многомасштабного моделирования новых материалов. 
     
     Разработка такого средства информационной поддержки поз\-во\-ля\-ет 
обеспечить формирование информации для многопараметрического анализа 
структуры и~физических свойств различных классов су\-ще\-ст\-ву\-ющих 
КМ, рассмотреть большое чис\-ло вариантов 
в~на\-прав\-ле\-нии поиска новых материалов и,~таким образом, ускорить и~удешевить 
процесс подбора па\-ра\-мет\-ров получения материалов.  Ис-\linebreak\vspace*{-12pt}

\pagebreak

\noindent
пользование данного 
комплекса позволяет за ограниченное время строить гиб\-рид\-ные модели для 
обоснованного выбора КМ с~заданными свойствами для  
авиа\-ци\-он\-но-кос\-ми\-че\-ской и~других областей промышленности. 
     
     В связи с~тем что традиционные материалы (преимущественно металлы)
      не в~полной мере отвечают высоким фи\-зи\-ко-ме\-ха\-ни\-че\-ским, 
технологическим и~эксплуатационными свойствам, развитие производства 
современных надежных и~экономичных конструкций в~машиностроении 
основано на применении новых КМ. Под 
композиционными понимаются материалы, со\-сто\-ящие из двух или более 
физически различных компонент (фаз), возможные комбинации которых 
приводят к~появлению уникальных свойств, отличных от тех, которыми обладала 
каж\-дая из них отдельно. На сегодня для развития авиа\-ци\-он\-но-кос\-ми\-че\-ской 
отрасли, включая самолетостроение, вертолетостроение, ракетостроение, 
требуется постоянное увеличение доли полимерных КМ
с~набором заданных свойств. Современные летательные аппараты обладают 
слож\-ной конструкцией, со\-сто\-ящей из металлов и~неметаллических материалов. 
Применяются детали из алю\-ми\-ни\-евых и~сталь\-ных сплавов, коррозионностойких 
сталей, титановых сплавов и~полимерных КМ (стек\-ло-, 
угле-, органопластики и~др.). Для снижения веса и~продления срока службы 
летательных аппаратов при производстве деталей все шире применяют 
полимерные~КМ.
     
     Сегодня наиболее востребованные САЕ- (Computer-Aided Engineering) 
сис\-те\-мы, такие как ABAQUS ({\sf https://simulia.com}), \mbox{ANSYS} ({\sf 
https://\linebreak Ansys.com}), LMS Engineering innovation ({\sf https://\linebreak trademarks.justia.com}), 
Femap ({\sf https://www.cad-is.ru/femap}), MSC Software ({\sf 
http://www.mscsoftware.\linebreak ru}) включают в~себя базы данных со свойствами 
материалов. Для КМ мож\-но выбрать тип композита со 
стандартными свойствами (угле-, стекло-, органопластики на основе 
эпоксифенолформальдегидных, кремнийорганических смол, эпоксидные 
боропластики и~т.\,д.). Имеется возможность коррекции данных свойств 
и~внесения материала с~новыми свойствами в~базу данных. Следует также отметить 
российские разработки в~об\-ласти моделирования КМ, 
такие как пакет CAE-Fidesys ({\sf https://cae-fidesys.com}), программный пакет для 
моделирования полимерных материалов Multicomp ({\sf 
https://www.kintechlab.com/products}), Российский исследовательский 
и~ин\-же\-нер\-но-тех\-но\-ло\-ги\-че\-ский проект N1 Composites ({\sf 
http://n1composites.com}) и~др.
{\looseness=-1

}
     
     Программные комплексы позволяют задать\linebreak свойства материалов, из 
которых состоит КМ, такие как изотропность, 
ортотропность, анизотропность. Важная часть проектирования композиционных 
конструкций~--- преобразование модели,\linebreak созданной с~применением CAD 
(Computer-aided design, сис\-те\-мы автоматизированного проектирования) 
в~модель, пригодную для CAE-ана\-ли\-за (нетривиальная задача, тре\-бу\-ющая 
за\-час\-тую создания экспертной сис\-те\-мы). Следует отметить, что функционал всех 
мировых лидеров в~CAE-сег\-мен\-те схож. 
     %
     Так, функционал MSC позволяет встраивать разработанные пользователем 
модули в~программный комплекс (например, можно включить метод имитации 
процесса производства КМ).
     
     Помимо используемых ведущими CAE-сис\-те\-ма\-ми модулями существуют 
коммерческие сис\-те\-мы, позволяющие генерировать КМ на микроуровне, а~затем 
проводить чис\-лен\-ные эксперименты на макроуровне. К~таким сис\-те\-мам 
относятся модуль генерации и~моделирования механических характеристик 
КМ GeoDict ({\sf www.math2market.com}) с~различными типами КМ, 
ге\-не\-ри\-ру\-емы\-ми модулем GeoDict, и~программный комплекс COMSOL ({\sf 
www.comsol.ru}).
     
     В современных ведущих CAE-сис\-те\-мах учет мик\-ро\-струк\-ту\-ры 
КМ проводится после гомогенизации свойств материала 
или определения мак\-ро\-мас\-штаб\-ных свойств КМ. При этом, однако, теряются 
индивидуальные детали микроструктуры КМ~\cite{3-ab}. При определении макромасштабных свойств КМ обычно 
исходят из идеальных условий: оптимального формирования граничной 
поверхности, идеального распределения(отсутствия взаимодействия час\-тиц 
между собой) и~отсутствия влияния компонента на мат\-рицу.
     
     Однако результаты, которые на сегодня могут быть получены 
     с~использованием САЕ-систем для\linebreak воспроизведения характеристик известных 
структур, зачастую могут расходиться с~данными экспериментов~--- например, 
когда речь идет о~полимерных КМ с~на\-но\-вклю\-че\-ни\-ями 
(\mbox{нанотрубками}). \mbox{Известно} влияние до\-бав\-ле\-ния на\-но\-раз\-мер\-ных\linebreak час\-тиц 
наполнителя на изменение механических свойств КМ. 
В~литературе широко описано изменение коэффициента теп\-ло\-про\-вод\-ности 
полимерных\linebreak мат\-риц в~несколько раз при их наполнении 
нанотрубками, пред\-став\-ле\-ны тео\-ре\-ти\-че\-ские исследования с~аналогичными 
результатами~\cite{1-ab}. Использование CAE-сис\-те\-м не позволяет в~полной 
мере \mbox{оценить} фактор влияния на\-но\-час\-тиц на данные свойства. Кроме 
того, применение CAE-сис\-тем в~контексте многомасштабного моделирования 
затруднено жесткими ограничениями пакетных решений. В~настоящее время 
развиваются \mbox{системы} c~программным обеспечением для многомасштабного 
моделирования, такие как Computational Soft Materials (Comsoft) Workbench, 
поз\-во\-ля\-ющий моделировать КМ с~<<мягкой>> 
структурой (полимеры, полимерные композиты), программный пакет LAMMPS 
({\sf https://www.lammps.org}), ис\-поль\-зу\-емый для моделирования в~рамках 
классической молекулярной динамики на атомистическом и~мезомасштабном 
уровнях полимерных, металлических, биологических сис\-тем и~др. Каждый из 
разрабатываемых программных продуктов обладает своими достоинствами 
и~областями применения. В~связи с~большим разнообразием типов 
КМ и~все воз\-рас\-та\-ющи\-ми требованиями к~наборам 
свойств, которыми они должны обладать, пред\-став\-ля\-ет\-ся важ\-ным\linebreak создание 
программных средств, поз\-во\-ля\-ющих оперативно вы\-стра\-и\-вать сис\-тем\-ные 
решения в~об\-ласти\linebreak многомасштабного моделирования с~применением 
высокопроизводительных вычислений, поз\-во\-ля\-ющих проводить моделирование от  
атом\-но-крис\-тал\-ли\-че\-ско\-го до мак\-ро\-уров\-ня. Такие системы \mbox{позволят} 
генерировать и~выполнять в~автоматическом режиме сценарии проведения 
расчетов под конкретную задачу, включать в~вычислительную схему расчеты на 
всех необходимых мас\-штаб\-ных уровнях. Для предсказательного моделирования 
структурных свойств различных классов КМ такой 
подход поз\-во\-ля\-ет создавать вы\-чис\-ли\-тель\-ную среду, в~которой задействованы 
возможности CАE-сис\-тем для верх\-не\-уров\-не\-во\-го (мак\-ро-) моделирования, 
методы анализа экспериментальных и~аналитических данных, а также 
собственные разработки и~пакетные приложения для расчетов на атом\-но-крис\-тал\-ли\-че\-ском и~наноуровне.

\vspace*{-9pt}

\section{Многомасштабная модель для~расчета структурных 
свойств композиционных материалов}

     В работе~\cite{2-ab} представлена общая схема многомасштабной модели 
для расчета структурных характеристик КМ. Для ее 
описания используется тео\-ре\-ти\-ко-мно\-жест\-вен\-ный аппарат, изложенный 
в~\cite{1-ab, 2-ab}. На ее основе формируются схемы для расчета разных классов 
КМ: нанокомпозитов на основе полимерной мат\-ри\-цы, 
КМ с~металлической мат\-ри\-цей, полимерных 
КМ с~углеволокном и~др.

\vspace*{-9pt}
     
     \subsection*{Основные уровни моделирования}
     
     \vspace*{-2pt}
     
     
     \textbf{Квантово-механический}. Рассматриваются отдельные молекулы. 
Решается уравнение Шредингера, определяется атомарная струк\-ту\-ра молекул 
полимера и~наполнителя, строится электронная струк\-ту\-ра и~рас\-счи\-ты\-ва\-ет\-ся 
когезионная энергия, рас\-счи\-ты\-ва\-ют\-ся меж\-атом\-ные и~меж\-мо\-ле\-ку\-ляр\-ные силы, 
определяются отдельные фи\-зи\-ко-хи\-ми\-че\-ские свойства.
     
     \textbf{Молекулярно-динамический}. Изучаются ан\-самб\-ли из молекул. 
Решаются уравнения молекулярной динамики с~использованием потенциалов 
межатомного взаимодействия, рас\-счи\-ты\-ва\-ют\-ся структурные характеристики 
мат\-ри\-цы (полимерной, металлической и~др.), наполнителя (нанотрубки, 
волокна и~др.), физические свойства. 
     
     \textbf{Мезоскопический}. Рас\-смат\-ри\-ва\-ют\-ся крупнозернистые модели. 
Используется упрощенное строение молекул. Цель моделирования на 
мезоуровне~--- получение распределения час\-тиц \mbox{наполнителя} в~мат\-ри\-це 
(полимерной, металлической и~др.)\ с~по\-сле\-ду\-ющим расчетом инженерных 
свойств полученных сис\-тем. 

\begin{figure*}[b] %fig1
\vspace*{8pt}
  \begin{center}  
    \mbox{%
\epsfxsize=133.618mm
\epsfbox{abg-1.eps}
}

\end{center}
\vspace*{-2pt}
\Caption{Схема многомасштабной композиции $\mathbf{MK}_{0,1,2,3,4}^{(\mathrm{Ti/Mo})}$ 
для расчета структурных свойств МКМ}
\end{figure*}
     
     \textbf{Континуальный} (\textbf{макроскопический}). Проводится расчет 
инженерных свойств (механические свойства, теп\-ло\-про\-вод\-ность и~др.). Задачи 
решаются с~применением механики сплош\-ных сред, гид\-ро\-ди\-на\-ми\-ки, тео\-рии 
упру\-гости. Применяются метод конечных элементов, методы решения краевых 
задач для моделирования различных процессов. 
     
     Рассмотрим пример построения многомасштабной композиции для 
тес\-то\-во\-го рас\-че\-та структурных свойств металлического 
КМ (МКМ) на основе Ti (титана), армированного волокнами Mo 
(молибдена). На сегодня Ti и~титановые сплавы стали очень привлекательными 
материалами для перспективных сфер применения благодаря таким свойствам, 
как низкая плот\-ность, высокие механические свойства и~коррозионная стой\-кость. 
Использование данных материалов в~конструкциях самолетов (реактивный 
двигатель и~фюзеляж) и~применение в~автомобильной про\-мыш\-лен\-ности рас\-тут 
быст\-ры\-ми темпами. Одним из способов совершенствования\linebreak титановых сплавов 
стало их применение в~качестве мат\-ри\-цы для КМ, 
армированных волокнами, например из Mo, которые обладают очень \mbox{хорошими} 
механическими свойствами ({\sf http://\linebreak viam-works.ru/ru/articles?art\_id=1103}). 
     
     Задействуем четыре перечисленных выше масштабных уров\-ня (не считая 
нулевого). Используя обозначения из~\cite{1-ab, 2-ab}, для построения 
многомасштабной композиции 
$$
\mathbf{MK}_{0,1,2,3,4}^{(\mathrm{Mo}, \mathrm{Ti}; 
1{,}1; 1{,}2; 2{,}1; 2{,}2; 3{,}1; 4{,}1)}= \mathbf{MK}_{0,1,2,3,4}^{(\mathrm{Ti/Mo})}
$$ 

\vspace*{-3pt}

\noindent
приведем экземпляры базовых мо\-де\-лей-ком\-по\-зи\-ций: 

\vspace*{-9pt}

\noindent
     \begin{align*}
     \mathbf{El}_{01}^{\mathrm{Ti}}:& \left\{ V_{01}^{\mathrm{Ti}}, 
X_{01}^{\mathrm{Ti}}, \mathrm{MA}_{01}^{\mathrm{Ti}}\right\};\\[-3pt]
     \mathbf{El}_{01}^{\mathrm{Mo}}:& \left\{ V_{01}^{\mathrm{Mo}}, 
X_{01}^{\mathrm{Mo}}, \mathrm{MA}_{01}^{\mathrm{Mo}}\right\};\\[-3pt]
\mathbf{MC}_{11}^{\mathrm{Ti}}:& \left\{ V_{11}^{\mathrm{Ti}}, 
X_{11}^{\mathrm{Ti}}, \mathrm{MA}_{11}^{\mathrm{Ti}}\right\};\\[-3pt]
\mathbf{MC}_{11}^{\mathrm{Mo}}:& \left\{ V_{11}^{\mathrm{Mo}}, 
X_{11}^{\mathrm{Mo}}, \mathrm{MA}_{11}^{\mathrm{Mo}}\right\};
\end{align*}

\noindent
\begin{align*}
               \mathbf{MC}_{12}^{\mathrm{Ti}}:& \left\{ V_{12}^{\mathrm{Ti}}, 
X_{12}^{\mathrm{Ti}}, \mathrm{MA}_{12}^{\mathrm{Ti}}\right\};\\
     \mathbf{MC}_{12}^{\mathrm{Mo}}:& \left\{ V_{12}^{\mathrm{Mo}}, 
X_{12}^{\mathrm{Mo}}, \mathrm{MA}_{12}^{\mathrm{Mo}}\right\};\\
     \mathbf{MC}_{21}^{\mathrm{Ti}}:& \left\{ V_{21}^{\mathrm{Ti}}, 
X_{21}^{\mathrm{Ti}}, \mathrm{MA}_{21}^{\mathrm{Ti}}\right\};\\
     \mathbf{MC}_{21}^{\mathrm{Mo}}:& \left\{ V_{21}^{\mathrm{Mo}}, 
X_{21}^{\mathrm{Mo}}, \mathrm{MA}_{21}^{\mathrm{Mo}}\right\};\\
     \mathbf{MC}_{22}^{\mathrm{Ti}}:& \left\{ V_{22}^{\mathrm{Ti}}, 
X_{22}^{\mathrm{Ti}}, \mathrm{MA}_{22}^{\mathrm{Ti}}\right\};\\
     \mathbf{MC}_{22}^{\mathrm{Mo}}:& \left\{ V_{22}^{\mathrm{Mo}}, 
X_{22}^{\mathrm{Mo}}, \mathrm{MA}_{22}^{\mathrm{Mo}}\right\};\\
     \mathbf{MC}_{31}^{\mathrm{Ti}/\mathrm{Mo}}:& \left\{
     V_{31}^{\mathrm{Ti}/\mathrm{Mo}}, X_{31}^{\mathrm{Ti}/\mathrm{Mo}}, 
\mathrm{MA}_{31}^{\mathrm{Ti}/\mathrm{Mo}}\right\};\\
     \mathbf{MC}_{41}^{\mathrm{Ti}/\mathrm{Mo}}:& \left\{
     V_{41}^{\mathrm{Ti}/\mathrm{Mo}}, X_{41}^{\mathrm{Ti}/\mathrm{Mo}}, 
\mathrm{MA}_{41}^{\mathrm{Ti}/\mathrm{Mo}}\right\}.
     \end{align*}
     
     Согласно схематическому пред\-став\-ле\-нию (рис.~1) многомасштабная 
композиция $\mathbf{MK}_{0,1,2,3,4}^{(\mathrm{Ti/Mo})}$ со\-сто\-ит из связанных между 
собой экземпляров базовых моделей композиций, размещенных на 
со\-от\-вет\-ст\-ву\-ющих мас\-штаб\-ных уровнях. На наноуровне проводится  
мо\-ле\-ку\-ляр\-но-ди\-на\-ми\-че\-ское моделирование структурных свойств 
титановой мат\-ри\-цы и~молибденовых волокон. На мезоуровне рас\-смат\-ри\-ва\-ет\-ся 
распределение час\-тиц в~МКМ, на мак\-ро\-уров\-не проводится расчет механических 
свойств МКМ.

\setcounter{figure}{2}
\begin{figure*}[b] %fig3
\vspace*{-6pt}
  \begin{center}  
    \mbox{%
\epsfxsize=120.383mm
\epsfbox{abg-3.eps}
}

\end{center}
\vspace*{-9pt}
\Caption{Пример сценария с~цик\-лом}
\end{figure*}
%\pagebreak
     
\vspace*{-10pt}

\section{Программный комплекс}

\vspace*{-2pt}

   Программный комплекс, интегрированный с~расчетными пакетами 
и~модулями, размещается на высокопроизводительных многоядерных сис\-те\-мах, 
оснащенных мощными графическими процессорами. Это связано с~тем, что 
исполнение вычислительных экспериментов, а~так\-же обработка 
и~анализ результатов вы\-чис\-ли\-тель\-ных  экспериментов
 ориентированы на 
распределенные сис\-те\-мы сбора, хранения и~обработки больших данных. В~основе 
программного комплекса лежит интеграционная платформа для 
многомасштабного моделирования, которая объединяет информационные потоки 
на разных мас\-штаб\-ных уровнях. При решении конкретной задачи, такой как 
расчет структурных особенностей, механических или иных свойств 
КМ, при изучении процессов их де\-гра\-да\-ции 
и~разрушения и~др.\ выделяются конкретные уров\-ни моделирования, которые 
необходимо задействовать. Первоначально строится многомасштабная 
композиция~--- информационный аналог\linebreak мно\-го\-мас\-штаб\-ной  
фи\-зи\-ко-ма\-те\-ма\-ти\-че\-ской модели. Для программной реализации на базе 
интеграционной платформы~\cite{4-ab} из име\-ющих\-ся программных модулей 
формируется вы\-чис\-ли\-тель\-ный \mbox{комплекс}~\cite{5-ab, 6-ab}.
   
   Перечислим пользовательские роли в~интеграционной плат\-фор\-ме 
мно\-го\-мас\-штаб\-но\-го моделирования:
   \begin{itemize}
\item разработчик вычислительных модулей реализует расчетный модуль или 
осуществляет конфигурирование при\-клад\-но\-го па\-кета;\\[-15pt]
\item системный разработчик создает веб-сер\-ви\-сы для вы\-чис\-ли\-тель\-но\-го модуля 
и~интегрирует его в~плат\-форму;\\[-15pt]
\item разработчик расчетных сценариев создает сценарии в~среде моделирования;\\[-15pt]
\item ученый-исследователь прикладной об\-ласти запускает расчетные сценарии 
с~различными па\-ра\-мет\-ра\-ми и~анализирует ре\-зуль\-таты.
\end{itemize}
    
    Как отмечалось в~\cite{5-ab, 6-ab}, программный комплекс предназначен для 
создания и~исполнения сценариев многомасштабных расчетов для моделирования 
структурных свойств композитных материалов.
    
    Сценарий~--- программная реализация мно\-го\-мас\-штаб\-ной композиции~--- 
пред\-став\-ля\-ет собой алгоритм последовательного выполнения расчетов отдельных 
физических характеристик материалов, входящих в~со\-став композита, 
посредством интегрированных с~программным комплексом вы\-чис\-ли\-тель\-ных 
модулей. Среда моделирования сценариев поз\-во\-ля\-ет создавать или 
модифицировать сценарии, учитывая особенности конкретного 
КМ и~тре\-бу\-емые свойства.



 
    
    Среда исполнения сценариев дает возможность осуществить его запуск 
    с~заданными входными па\-ра\-мет\-ра\-ми, отслеживать его выполнение в~целом и~по 
со\-став\-ным задачам, про\-смат\-ри\-вать входные и~выходные данные (результаты 
расчетов). Интеграционная роль среды исполнения заключается\linebreak в~формировании 
входных данных для вычислительных модулей в~со\-от\-вет\-ст\-ву\-ющем формате 
и~единицах измерения, отслеживании работы модулей,\linebreak получении конечного 
результата расчета и~преобразовании его в~формат и~единицы измерения, 
до\-ступ\-ные для других модулей сценария. Таким образом, среда исполнения 
обеспечивает соответствие потока исполнения вы\-чис\-ли\-тель\-ных модулей 
заданному алгоритму в~сценарии и~це\-лост\-ность потока данных между блоками 
сценария. Кроме того,\linebreak среда исполнения предостав\-ля\-ет общие словари для\linebreak 
согласования вход\-ных-вы\-ход\-ных данных вы\-чис\-ли\-тель\-ных экспериментов, 
такие как справочник\linebreak химических элементов и~их свойств, химических формул 
веществ, ис\-поль\-зу\-емых в~композитных материалах, типы крис\-тал\-ли\-че\-ских 
сис\-тем, типы атомных радиусов, пространственные группы.

\setcounter{figure}{3}
\begin{figure*}[b] %fig4
\vspace*{-9pt}
  \begin{center}  
    \mbox{%
\epsfxsize=163mm
\epsfbox{abg-4.eps}
}

\end{center}
\vspace*{-9pt}
\Caption{Сценарий для расчета МКМ}
\end{figure*}

\vspace*{-10pt}
   
    \subsection*{Алгоритм программы}
    
    \vspace*{-2pt}
    
    Алгоритм исполнения сценария основан на стандарте BPMN~2.0 и~со\-сто\-ит из 
сле\-ду\-ющих ключевых элементов (рис.~2).

{ \begin{center}  %fig2
 \vspace*{6pt}
    \mbox{%
\epsfxsize=70.82mm
\epsfbox{abg-2.eps}
}

\vspace*{6pt}

\noindent
{{\figurename~2}\ \ \small{
Пример простого сценария
}}
\end{center}
}

%\vspace*{6pt} 

\noindent
\begin{description}
\item[Э1.]  Точка начала выполнения сценария. В~свойствах этого элемента 
указывается список кодов физических величин, которые пользователь дол\-жен 
будет ввес\-ти перед запуском сценария.

\item[Э2.] Сплошная стрелка определяет строгую по\-сле\-до\-ва\-тель\-ность 
выполнения шагов сценария.

 \begin{figure*}[b] %fig5
  \vspace*{1pt}
  \begin{center}  
    \mbox{%
\epsfxsize=131mm %.834mm
\epsfbox{abg-5.eps}
}

\end{center}
\vspace*{-9pt}
  \Caption{Сценарий для расчета механических свойств полимерного нанокомпозита}
  \end{figure*}

\item[Э3.] Вычислительная задача пред\-став\-ля\-ет\-ся в~BPMN как <<внеш\-няя 
сервисная задача>> (External Service Task). В~поле topic в~настройках 
задачи вводится название очереди задач со\-от\-вет\-ст\-ву\-юще\-го 
вы\-чис\-ли\-тель\-но\-го модуля. Например, для  
кван\-то\-во-ме\-ха\-ни\-че\-ско\-го расчета на пакете VASP вводится 
<<vasp\_topic>>. Список до\-ступ\-ных вы\-чис\-ли\-тель\-ных модулей 
с~названиями очередей хранится в~базе данных в~таб\-ли\-це <<Module>>.\\[-15pt]

\item[Э4.] Точка завершения выполнения сценария. Если в~сценарии существует 
ветвление, точек завершения может быть несколько.



    \item[Э5.] Шаг сценария, в~рамках которого выполняется скрипт, заданный 
пользователем. В~па\-ра\-мет\-рах задачи может быть указан язык скрип\-та и~сам 
скрипт. Доступны языки Groovy и~Jython (реализация языка Python на Java). 
Скрип\-ты могут использоваться для изменения входных и~выходных па\-ра\-мет\-ров, 
небольших вы\-чис\-ле\-ний на основе текущих до\-ступ\-ных данных сценария. 
В~примере на рис.~3 в~цик\-ле определяется список векторов 
крис\-тал\-ли\-че\-ской решетки, по которым будет проводиться кван\-то\-во-ме\-ха\-ни\-че\-ский 
рас\-чет деформированной решетки.\\[-19.5pt]

\begin{figure*}[b] %fig6
\vspace*{1pt}
  \begin{center}  
    \mbox{%
\epsfxsize=163mm
\epsfbox{abg-6.eps}
}

\end{center}
\vspace*{-9pt}
\Caption{Сценарий для расчета КМ с~полимерной мат\-ри\-цей 
и~наполнителем из углеволокна}
\end{figure*}
    
    \item[Э6.] Подпроцесс сценария <<цикл с~параллельным запуском>>  
(Parallel multi-instance) позволяет параллельно запустить выполнение час\-ти 
сцена- %\linebreak\vspace*{-12pt}

\columnbreak

\noindent
рия несколько раз. В~свойствах подпроцесса требуется указать коллекцию 
(Collection), по элементам которой будет проводиться ите\-ри\-ро\-ва\-ние, и~название 
переменной цикла (Element Variable). Весь элемент считается выполненным, когда 
все параллельно выполняющиеся подпроцессы завершат свою работу. Например, 
если требуется запустить кван\-то\-во-ме\-ха\-ни\-че\-ский расчет для некоторого 
множества деформированных решеток (для определения в~дальнейшем констант 
упру\-гости), предварительно в~скрип\-те перед цик\-лом формируется список 
деформированных векторов решетки и~сохраняется в~переменную процесса. 
Далее для каж\-дой деформации параллельно вызывается\linebreak\vspace*{-12pt}

\pagebreak

\noindent  
кван\-то\-во-ме\-ха\-ни\-че\-ский модуль VASP для расчета энергии и~объема 
решетки. Получившаяся таб\-ли\-ца с~данными может использоваться для расчета 
констант элас\-тич\-ности, модуля упру\-гости и~других свойств материала.
\end{description}

\vspace*{-9pt}
  
  \subsection*{Примеры тестовых сценариев для~расчета~структурных~характеристик 
  и~отдельных~свойств различных классов 
композиционных материалов}


     
     \textbf{Пример~1.} Тестовый сценарий для расчета структурных свойств 
КМ с~металлической мат\-ри\-цей (рис.~4).

\smallskip

     
     \textbf{Пример~2.} Тестовый сценарий для расчета механических свойств 
полимерного нанокомпозита (полифениленсульфид с~углеродными нанотрубками). 
На сле\-ду\-ющем этапе проекта планируется расширить сценарий для 
оценки влияния процентного содержания углеродных нанотрубок на изменение 
коэффициента теп\-ло\-про\-вод\-ности полимерного нанокомпозита (рис.~5).
  
 
     
     \textbf{Пример~3.} Тестовый сценарий для расчета механических свойств 
КМ с~полимерной мат\-ри\-цей (эпоксидная смола) и~углеволокном
(рис.~6).

\vspace*{-6pt}

\section{Выводы}

\vspace*{-2pt}

     В работе представлен программный комплекс для расчета структурных 
характеристик КМ с~тре\-бу\-емы\-ми свойствами. В~его 
основе лежит интеграционная плат\-фор\-ма для многомасштабного моделирования, 
которая объединяет информационные потоки на разных мас\-штаб\-ных уровнях. На 
ее основе формируются схемы для рас\-че\-та структурных характеристик разных 
клас\-сов КМ: нанокомпозитов на основе полимерной 
мат\-ри\-цы, КМ с~металлической мат\-ри\-цей, полимерных 
КМ с~углеволокном и~другие. Разработанные подходы 
поз\-во\-ля\-ют моделировать свойства КМ (механические, 
теп\-ло\-вые и~др.), а~так\-же многомасштабные процессы, связанные с~усталостным 
разрушением при случайных по\-вреж\-де\-ни\-ях в~ходе эксплуатации, и~другие 
динамические процессы. Программный комплекс со\-сто\-ит из программных 
модулей и~базируется на типовых сер\-ви\-сах вы\-чис\-ли\-тель\-ных модулей, общей 
интеграционной оболочки и~модулей сценариев. Про\-грам\-мные решения 
сертифицированы. В~дальнейшем планируется раз\-ра\-бо\-тать 
полнофункциональную про\-грам\-мную сис\-те\-му с~целью решения различных 
классов обратных задач в~об\-ласти наук о~материалах. Разработанные подходы 
в~сочетании с~экспериментальными данными могут быть использованы для 
лучшего понимания физических основ изменения свойств в~за\-ви\-си\-мости от 
струк\-ту\-ры и,~как след\-ст\-вие, для уде\-шев\-ле\-ния и~уско\-ре\-ния поиска новых 
КМ с~заданными свойствами.

\vspace*{-6pt}
   
{\small\frenchspacing
 {%\baselineskip=10.8pt
 %\addcontentsline{toc}{section}{References}
 \begin{thebibliography}{9}
 
 \vspace*{-2pt}
   
   \bibitem{1-ab}
   \Au{Абгарян К.\,К.} Многомасштабное моделирование в~задачах структурного 
материаловедения.~--- М.: МАКСПресс, 2017. 284~с.
\bibitem{2-ab}
\Au{Абгарян~К.\,К.} Информационная технология по\-стро\-ения многомасштабных моделей 
в~задачах вы\-чис\-ли\-тель\-но\-го материаловедения~// Сис\-те\-мы высокой до\-ступ\-ности, 2018. Т.~14. 
№\,2. С.~9--15.
\bibitem{3-ab}
\Au{Naffakh M., D$\acute{\!\mbox{{\!\ptb{\i}}}}$ez-Pascuala~A.\,M., Marcoa~C., Ellisa~G.} Morphology and thermal properties of novel poly (phenylene sulfide) 
hybrid nanocomposites based on single-walled carbon nanotubes and 8 inorganic fullerene-like WS~2 
nanoparticles~// J.~Mater. Chem., 2012. Vol.~22. No.\,4. P.~1418--1425.
\bibitem{4-ab}
\Au{Абгарян К.\,К., Гаврилов~Е.\,С.} Распределенная информационная сис\-те\-ма для расчета 
структурных свойств композиционных материалов~// Информатика и~её применения, 2021. 
Т.~15. Вып.~4. С.~50--58. doi: 10.14357/ 19922264210407.
\bibitem{5-ab}
\Au{Гаврилов Е.\,С.} Интегрированный интерфейс к~модулю сплош\-но\-сред\-но\-го взаимодействия. 
Свидетельство о~регистрации программ для ЭВМ №\,2021681058, 2021.
\bibitem{6-ab}
\Au{Гаврилов Е.\,С.} Программные средства для хранения и~обмена данными в~задачах 
моделирования композитных материалов. Свидетельство о~регистрации программ для ЭВМ 
№\,2021681762, 2021.

\end{thebibliography}

 }
 }

\end{multicols}

\vspace*{-8pt}

\hfill{\small\textit{Поступила в~редакцию 22.01.22}}

\vspace*{8pt}

%\pagebreak

%\newpage

%\vspace*{-28pt}

\hrule

\vspace*{2pt}

\hrule

%\vspace*{-2pt}

\def\tit{SOFTWARE PACKAGE FOR MULTISCALE MODELING OF~STRUCTURAL PROPERTIES 
OF~COMPOSITE MATERIALS}


\def\titkol{Software package for multiscale modeling of~structural properties 
of~composite materials}


\def\aut{K.\,K.~Abgaryan$^1$ and~E.\,S.~Gavrilov$^{1,2}$}

\def\autkol{K.\,K.~Abgaryan and~E.\,S.~Gavrilov}

\titel{\tit}{\aut}{\autkol}{\titkol}

\vspace*{-18pt}


\noindent
$^1$Federal Research Center ``Computer Science and Control'' of the Russian Academy of Sciences, 
44-2~Vavilov\linebreak
$\hphantom{^1}$Str., Moscow 119333, Russian Federation

\noindent
$^2$Moscow Aviation Institute (National Research University), 4~Volokolamskoe Shosse, Moscow 
125080, Russian\linebreak
$\hphantom{^1}$Federation

\def\leftfootline{\small{\textbf{\thepage}
\hfill INFORMATIKA I EE PRIMENENIYA~--- INFORMATICS AND
APPLICATIONS\ \ \ 2022\ \ \ volume~16\ \ \ issue\ 1}
}%
 \def\rightfootline{\small{INFORMATIKA I EE PRIMENENIYA~---
INFORMATICS AND APPLICATIONS\ \ \ 2022\ \ \ volume~16\ \ \ issue\ 1
\hfill \textbf{\thepage}}}

\vspace*{3pt} 
      
      
  
\Abste{Today, creation of new composite materials and methods of their construction with predictable 
properties is one of the urgent and most important tasks connected with modernization of 
industrial production in our country. For their solution, technologies of multiscale computer modeling 
are actively developed. They have become a~link between fundamental physics (chemistry) and 
engineering materials science. The paper presents a~software package for modeling structural 
properties of composite materials which allows solving a~number of problems of this class. It is 
focused on high-performance computations. The complex is based on an original multiscale 
technology which allows one to promptly conduct multivariate analysis of different classes of 
composite materials and conduct research on designing the new ones with predictable properties. The 
developed approaches in combination with experimental data can be used for a~better understanding of 
the physical foundations of the change of properties depending on the structure and, as a~consequence, 
for cheaper and faster search of new composite materials with predetermined properties.}

\KWE{multiscale modeling; composite materials; integration platform; software package; distributed 
system}



\DOI{10.14357/19922264220113}

\vspace*{-16pt}

\Ack
\noindent
The research was supported by the Ministry of Science and Higher Education of the Russian 
Federation (project No.\,075-15-2020-799).




%\vspace*{4pt}

  \begin{multicols}{2}

\renewcommand{\bibname}{\protect\rmfamily References}
%\renewcommand{\bibname}{\large\protect\rm References}

{\small\frenchspacing
 {%\baselineskip=10.8pt
 \addcontentsline{toc}{section}{References}
 \begin{thebibliography}{9}
\bibitem{1-ab-1}
\Aue{Abgaryan, K.\,K.} 2017. \textit{Mnogomasshtabnoe modelirovanie v~zadachakh strukturnogo 
materialovedeniya} [Multiscale modeling for structural materials science applications]. Moscow: 
MAKS Press. 284~p.

\vspace*{-2pt}

\bibitem{2-ab-1}
\Aue{Abgaryan, K.\,K.} 2018. In\-for\-ma\-tsi\-on\-naya tekh\-no\-lo\-giya po\-stro\-eniya mno\-go\-mas\-shtab\-nykh 
mo\-de\-ley v~za\-da\-chakh vy\-chis\-li\-tel'\-no\-go ma\-te\-ri\-a\-lo\-ve\-de\-niya 
[Information technology is the construction 
of multi-scale models in problems of computational materials science]. \textit{Sistemy vysokoy 
dostupnosti} [Highly Available Systems] 14(2):9--15.
\bibitem{3-ab-1}
\Aue{Naffakh, M., A.\,M.~D$\acute{\mbox{{\!\ptb{\i}}}}$ez-Pascuala, C.~Marcoa, and G.~Ellisa.} 
2012. Morphology and thermal properties of novel poly (phenylene sulfide) hybrid nanocomposites 
based on single-walled carbon nanotubes and~8~inorganic fullerene-like WS~2 nanoparticles. 
\textit{J.~Mater. Chem.}  
22(4):1418--1425.
{\looseness=1

}
\bibitem{4-ab-1}
  \Aue{Abgaryan, K.\,K., and E.\,S.~Gavrilov.} 2021. 
  Ras\-pre\-de\-len\-naya in\-for\-ma\-tsi\-on\-naya sis\-te\-ma   dlya 
ras\-che\-ta struk\-tur\-nykh svoystv kom\-po\-zi\-tsi\-on\-nykh ma\-te\-ri\-alov 
[Distributed information system for 
calculating the structural properties of composite materials]. \textit{Informatika i~ee Primeneniya~--- 
Inform. Appl.} 15(4):50--58. doi: 10.14357/19922264210407.
\bibitem{5-ab-1}
  \Aue{Gavrilov, E.\,S.} 2021. In\-teg\-ri\-ro\-van\-nyy in\-ter\-feys k~mo\-du\-lyu 
  splosh\-no\-sred\-no\-go 
vza\-imo\-dey\-stviya [Integrated interface to the solid-medium interaction module]. Certificate on official 
registration of the computer program No.\,2021681058.
\bibitem{6-ab-1}
  \Aue{Gavrilov, E.\,S.} 2021. Pro\-gram\-mnye sred\-st\-va dlya khra\-ne\-niya 
  i~ob\-me\-na dan\-ny\-mi  v~za\-da\-chakh mo\-de\-li\-ro\-va\-niya kom\-po\-zit\-nykh ma\-te\-ri\-a\-lov 
  [Software tools for data persistence and data flow in 
composite materials modeling tasks]. Certificate on official registration of the computer program 
No.\,2021681762.
\end{thebibliography}

 }
 }

\end{multicols}

\vspace*{-6pt}

\hfill{\small\textit{Received January 22, 2022}}


\Contr

\noindent
\textbf{Abgaryan Karine K.} (b.\ 1963)~--- Doctor of Science in physics and mathematics, principal 
scientist, A.\,A.~Dorodnicyn Computing Center, Federal Research Center ``Computer Science and 
Control'' of the Russian Academy of Sciences, 40~Vavilov Str., Moscow 119333, Russian Federation; 
head of department, Moscow Aviation Institute (National Research University), 4~Volokolamskoe 
Shosse, Moscow 125080, Russian Federation; \mbox{kristal83@mail.ru}

\vspace*{3pt}

\noindent
\textbf{Gavrilov Evgeny S.} (b.\ 1982)~--- scientist, A.\,A.~Dorodnicyn Computing Center, Federal 
Research Center ``Computer Science and Control'' of the Russian Academy of Sciences, 40~Vavilov 
Str., Moscow 119333, Russian Federation; senior lecturer, Moscow Aviation Institute (National 
Research University), 4~Volokolamskoe Shosse, Moscow 125080, Russian Federation; 
\mbox{eugavrilov@gmail.com}
       

\label{end\stat}

\renewcommand{\bibname}{\protect\rm Литература}  %7

\def\stat{shnurkov}

\def\tit{АНАЛИТИЧЕСКОЕ РЕШЕНИЕ ЗАДАЧИ ОПТИМАЛЬНОГО УПРАВЛЕНИЯ ПОЛУМАРКОВСКИМ ПРОЦЕССОМ\\ 
С~КОНЕЧНЫМ МНОЖЕСТВОМ СОСТОЯНИЙ$^*$}

\def\titkol{Аналитическое решение задачи оптимального управления полумарковским 
процессом} %с~конечным множеством состояний}

\def\aut{П.\,В.~Шнурков$^1$, А.\,К.~Горшенин$^2$, В.\,В.~Белоусов$^3$}

\def\autkol{П.\,В.~Шнурков, А.\,К.~Горшенин, В.\,В.~Белоусов}

\titel{\tit}{\aut}{\autkol}{\titkol}

\index{Шнурков П.\,В.}
\index{Горшенин А.\,К.}
\index{Белоусов В.\,В.}
\index{Shnurkov P.\,V.}
\index{Gorshenin A.\,K.}
\index{Belousov V.\,V.}


{\renewcommand{\thefootnote}{\fnsymbol{footnote}} \footnotetext[1]
{Работа выполнена при частичной поддержке РФФИ (проект 15-07-05316).}}


\renewcommand{\thefootnote}{\arabic{footnote}}
\footnotetext[1]{Национальный исследовательский университет <<Высшая школа экономики>>, 
\mbox{pshnurkov@hse.ru}}
\footnotetext[2]{Институт проблем информатики Федерального исследовательского центра <<Информатика 
и~управ\-ле\-ние>> Российской академии наук, \mbox{agorshenin@frccsc.ru}}
\footnotetext[3]{Институт проблем информатики Федерального исследовательского центра <<Информатика 
и~управление>> Российской академии наук, \mbox{vbelousov@ipiran.ru}}

%\vspace*{-6pt}

\Abst{Настоящее исследование посвящено теоретическому обоснованию нового метода 
нахождения оптимальной стратегии управления полумарковским процессом с~конечным 
множеством состояний. Рассматриваются марковские рандомизированные стратегии 
управления, определяемые конечным набором вероятностных мер, соответствующих 
каждому состоянию. Характеристикой качества управления служит стационарный 
стоимостной показатель. Данный показатель представляет собой дроб\-но-ли\-ней\-ный 
интегральный функционал от набора вероятностных мер, задающих стратегию управления. 
Для этого функционала известны явные аналитические представления подынтегральных 
функций числителя и~знаменателя. Дальнейшие результаты основываются на новой 
усиленной и~обобщенной форме теоремы об экстремуме дроб\-но-ли\-ней\-но\-го интегрального 
функционала. Доказывается, что проблемы существования оптимальной стратегии управления 
полумарковским процессом и~ее нахождения сводятся к~задаче численного исследования 
на глобальный экстремум заданной функции от конечного числа вещественных переменных.}

\KW{оптимальное управление полумарковским процессом; стационарный стоимостной 
показатель качества управления; дроб\-но-ли\-ней\-ный интегральный функционал}

\DOI{10.14357/19922264160408} 

\vspace*{9pt}


\vskip 10pt plus 9pt minus 6pt

\thispagestyle{headings}

\begin{multicols}{2}

\label{st\stat}

\section{Введение}

Теория оптимального управления марковскими и~полумарковскими случайными 
процессами интенсивно развивается с~начала 1960-х~гг. Еще в~первых 
основополагающих исследованиях рассматривались не только проблемы существования 
оптимальных стратегий управления, но и~способы нахождения этих стратегий. 

Для решения таких проблем, имеющих алгоритмическое содержание, использовались 
открытые незадолго до этого мощные методы прикладной математики: линейное 
программирование и~динамическое программирование. Отметим, прежде всего, 
классическую работу Р.~Ховарда~\cite{1}, в~которой метод динамического 
программирования был применен для решения проблемы оптимального управления 
марковским процессом с~непрерывным временем. В~дальнейшем В.\,В.~Рыков~\cite{2} 
доказал, что для аналогичной модели управления марковским процессом с~учетом 
переоценки оптимальной стратегией также является стационарная.

Важную роль в~развитии теории управления случайными процессами сыграла работа 
В.~Джевелла~\cite{3}, в~которой были впервые рассмотрены полумарковские модели 
управления для вариантов с~переоценкой и~без переоценки. Данная работа была 
переведена на русский язык и~послужила основой для многих последующих работ 
отечественных и~зарубежных специалистов. В~частности, Б.~Фокс показал~\cite{4}, 
что оптимальной стратегией управления полумарковским процессом в~варианте без 
переоценки является стационарная; аналогичные результаты были получены Э.~Денардо 
и~для варианта с~переоценкой~\cite{5}.

Среди последующих исследований алгоритмической направленности отметим работы 
Р.~Ховарда~\cite{6}, Б.~Фокса~\cite{4}, а также С.~Осаки и~Х.~Майна~\cite{7}. 
В~этих работах для нахождения оптимальных стратегий управления полумарковскими 
процессами использовался метод линейного программирования.

В 1970~г.\ была опубликована фундаментальная монография Х.~Майна и~С.~Осаки~\cite{8}, 
переведенная на русский язык в~1977~г., в~которой были систе\-ма\-ти\-зи\-ро\-ва\-ны и~изложены 
основные результаты по теории оптимального управления марковскими и~полумарковскими 
случайными процессами. Фактически данная книга стала итогом исследований по проблемам 
стохастического управления\linebreak
 за~10~лет. Отметим, что в~этой монографии рас\-смат\-ри\-ва\-лись 
марковские и~полумарковские модели управления с~конечными множествами состояний 
и~допустимых решений, принимаемых \mbox{в~каждом} состоянии. Были получены принципиальные 
тео\-ре\-ти\-че\-ские результаты, заключающиеся в~том, что оптимальные стратегии управ\-ле\-ния 
для основных видов рас\-смат\-ри\-ва\-емых моделей с~переоценкой и~без переоценки являются 
детерминированными и~стационарными. Были разработаны и~обоснованы процедуры нахождения 
оптимальных стратегий управления. В~частности, для модели управления полумарковским 
процессом без переоценки, когда множество со\-сто\-яний образует один эргодический класс, 
а~показатель качества управления пред\-став\-ля\-ет собой стационарный средний удельный 
доход (см.~[8, гл.~5, п.~5.5]), процедура поиска оптимальной рандомизированной 
стратегии осуществлялась методом линейного программирования. Обратим особое внимание 
на данный результат, поскольку аналогичная модель управления полумарковским 
процессом будет рассмотрена в~настоящей работе.

Принципиальную роль в~развитии теории стохастического управления сыграла 
монография И.\,И.~Гихмана и~А.\,В.~Скорохода~\cite{9}. В~этой книге были впервые 
систематически изложены основы теории оптимального управления случайными процессами 
с~дискретным и~непрерывным временем, включая теорию управления процессами, которые 
описываются стохастическими дифференциальными уравнениями. Отдельно были рас\-смот\-ре\-ны 
проблемы управления марковскими процессами с~дискретным временем и~скачкообразными 
марковскими процессами с~непрерывным временем. Роли множеств состояний и~допустимых 
управ\-ле\-ний играли пространства весьма общей структуры. Для широких классов функционалов 
качества управ\-ле\-ния (так называемых эволюционных функционалов в~марковских моделях 
с~дискретным временем и~интегральных функционалов накопления в~марковских моделях 
с~непрерывным временем) были доказаны теоремы о~существовании и~формах пред\-став\-ле\-ния 
оптимальных стратегий управ\-ле\-ния. Было установлено, что для однородных марковских 
моделей оптимальные стратегии управ\-ле\-ния существуют, являются стационарными 
и~детерминированными. Иначе говоря, такие стратегии задаются детерминированными 
функциями, аргументом которых является со\-сто\-яние сис\-те\-мы в~момент принятия решения, 
и~не зависящими от самого момента принятия решения. Что же касается важного вопроса 
о~формах представления этих функций, то их можно охарактеризовать следующим образом. 
Были найдены функциональные уравнения, осложненные условием экстремума, которым 
удовле\-тво\-ря\-ют упомянутые функции. По существу эти соотношения пред\-став\-ля\-ют собой 
уравнения Беллмана для соответствующих динамических стохастических моделей.

Особо отметим, что в~монографии~\cite{9} не рас\-смат\-ри\-ва\-лись проблемы управления 
полумарковскими процессами. Однако дальнейшее развитие общей теории управления 
такими процессами шло по пути, идейно намеченному в~указанной книге.

В последующие годы развитие теории управ\-ле\-ния полумарковскими процессами 
осуществля-\linebreak лось по направлению усложнения моделей и~обобщения исходных предположений. 
Например,\linebreak в~работах~\cite{10, 11} рассмотрены управляемые по\-лумарковские процессы при 
весьма общих предположениях относительно характера пространств состояний и~управлений. 
Проблемы управления исследовались по отношению к~различным видам целевых показателей, 
обобщающих упомянутый выше стационарный показатель средней удельной прибыли. В~этих 
работах доказывается, что оптимальная стратегия управления по отношению к~каж\-до\-му из 
показателей существует и~является одной и~той же стационарной детерминированной 
стратегией, определяемой некоторой функцией, заданной на множестве со\-сто\-яний процесса. 
Об этой функции известно лишь то, что она удовлетворяет некоторому интегральному 
уравнению, которое по содержанию пред\-став\-ля\-ет собой уравнение Бел\-лма\-на для 
соответствующей задачи управ\-ления.

Среди исследований, предшествовавших настоящему, отметим работу 
В.\,А.~Каштанова~[12, гл. 13]. В этом разделе коллективной монографии~\cite{12} 
автором была рассмотрена проблема оптимального управления полумарковским 
процессом с~конечным множеством состояний и~множеством возможных решений, 
которое представляет собой произвольный интервал множества вещественных чисел. 
Модель относится к~виду моделей без переоценки, показателем качества управления 
служит стационарное значение среднего удельного дохода, определяемое аналогично 
классическим работам~\cite{3, 8}. Рандомизированное управление в~каждом состоянии 
определяется в~соответствии с~вероятностным распределением, совокупность которых 
задает\linebreak
 стратегию управления. В.\,А.~Каш\-та\-но\-вым было\linebreak сформулировано утверждение о том, 
что стацио\-нарное значение среднего удельного дохода представляет собой 
дроб\-но-ли\-ней\-ный интегральный функционал от набора вероятностных распределений, 
образующих стратегию управления. При этом\linebreak ранее~[12, гл.~10; 13] было уста\-нов\-ле\-но, 
что дроб\-но-ли\-ней\-ный функционал достигает экстремума на вырожденных распределениях. 
Отсюда естест-\linebreak венно следует, что оптимальная стратегия управ\-ле-ния является 
детерминированной и~должна\linebreak определяться точкой экстремума функции, представляющей 
собой отношение подынтегральных функций чис\-ли\-те\-ля и~знаменателя данного 
дроб\-но-ли\-ней\-но\-го функционала. Однако в~\cite{12} не были получены явные 
представления для указан-\linebreak ных функций. Кроме того, приведенный в~гл.~10 
монографии~\cite{12} вариант теоремы об экстремуме дроб\-но-ли\-ней\-но\-го 
интегрального функционала требовал проверки выполнения условия существования 
этого экстремума. Такие условия указаны не были. В~связи с~этими обстоятельствами 
использовать полученные в~\cite{12} результаты для доказательства существования 
оптимальной детерминированной стратегии управ\-ле\-ния полумарковским процессом и~для 
строгого обоснования способа нахождения такой стратегии оказалось невозможным.

Настоящее исследование посвящено теоретическому обоснованию нового метода 
нахождения\linebreak оптимальной стратегии управления полумарковским процессом с~конечным 
множеством со\-сто\-яний. Рассматриваются марковские рандомизи\-рованные стратегии 
управления, определяемые конеч\-ным набором вероятностных мер, соответствующих 
каждому состоянию. Показателем качества управления служит уже упоминавшийся 
классический  показатель~--- стационарное значение средней удельной прибыли. 
Доказано, что этот показатель представляет собой дроб\-но-ли\-ней\-ный интегральный 
функционал от набора вероятностных мер, задающих стратегию управления. При этом, 
в~отличие от~\cite{12}, получены явные аналитические представления для подынтегральных 
функций числителя и~знаменателя этого дроб\-но-ли\-ней\-но\-го\linebreak
 функционала. Дальнейшие 
результаты основываются на новой усиленной и~обобщенной форме\linebreak
 теоремы об экстремуме 
дроб\-но-ли\-ней\-но\-го интегрального функционала, впервые опубликованной 
в~работе П.\,В.~Шнуркова~\cite{14}. Согласно\linebreak
 утверж\-де\-нию этой теоремы, если 
существует глобальный экстремум так называемой основной функции дроб\-но-ли\-ней\-но\-го 
функционала, которая пред\-став\-ля\-ет собой отношение подынтегральных функций чис\-ли\-те\-ля 
и~знаменателя, то существует безусловный экстремум самого дроб\-но-ли\-ней\-но\-го 
функционала, который достигается на наборе вырожденных вероятностных распределений, 
сосредоточенных в~точке глобального экстремума. В~этом случае оптимальная стратегия 
управ\-ле\-ния существует, является стационарной и~детерминированной и~определяется точкой, 
в~которой основная\linebreak функция достигает глобального экстремума. Таким\linebreak образом, проблемы 
существования оптимальной стратегии управ\-ле\-ния полумарковским процессом и~ее 
нахождения сводятся к~задаче чис\-лен\-но\-го исследования на глобальный экстремум 
заданной функции от конечного чис\-ла вещественных переменных.

\section{Общее описание модели управления полумарковским случайным процессом}

Построим модель управления полумарковским случайным процессом, следуя общему 
подходу, принятому в~классических работах~\cite{3, 8}. Пусть $\xi(t)$~--- 
случайный полумарковский процесс с~конечным множеством состояний
$X\hm=\{1,2,\ldots, N\}$, $N\hm< \infty$. Обозначим через~$t_n$, $n=0,1,2,\ldots$, 
$t_0\hm=0$, случайные моменты изменения состояний данного процесса, 
$\theta_n\hm=t_{n+1}-t_n$, $n\hm=0,1,2,\ldots$, $\xi_n\hm=\xi(t_n)\hm=\xi(t_n+0)$, 
$n\hm=0,1,2,\ldots$ (предполагается, что траектории процесса~$\xi(t)$ 
непрерывны справа). Случайная последовательность~$\{\xi_n\}$
образует цепь Маркова, вложенную в~полумарковский процесс~$\xi(t)$.
Зададим набор измеримых пространств\linebreak $(U_1, \mathscr{B}_1), 
(U_2, \mathscr{B}_2), \ldots, (U_N, \mathscr{B}_N)$, где $U_i$~--- 
множество возможных допустимых управ\-ле\-ний, $\mathscr{B}_i$~--- $\sigma$-ал\-геб\-ра 
подмножеств множества~$U_i$, вклю\-ча\-ющая в~себя все одноточечные подмножества\linebreak  
множества~$U_i$, т.\,е.\ если $u_i \hm\in U_i$, то $\{u_i\} \hm\in \mathscr{B}_i$, 
$i\hm=1,2,\ldots, N$.
Пусть $\Gamma_i$~--- некоторое множество всевозможных вероятностных мер $\Psi_i 
\hm \in \Gamma_i$, заданных на $\sigma$-ал\-геб\-ре~$\mathscr{B}_i$, $i\hm=1,2,\ldots,N$.

Поскольку идейное содержание и~свойства вероятностных мер существенно используются 
в~данной работе, укажем на некоторые фундаментальные издания, в~которых 
изложена соответствующая тео\-рия. Понятие и~основные свойства вероятностной 
меры определены и~подробно проанализированы в~книге А.\,Н.~Ширяева~\cite[гл.~II]{15}. 
Глубокое изложение основ теории вероятностных мер имеется также в~книге 
А.\,А.~Боровкова~\cite{16}. Заметим попутно, что в~книге~\cite{16} имеются разделы, 
посвященные изложению основ теории полумарковских и~регенерирующих случайных процессов. 
Из зарубежных изданий отметим фундаментальную работу П.~Хеннекена и~А.~Тортра~\cite{17}, 
основная часть которой посвящена изложению математических основ теории вероятностей.

Введем специальное понятие вырожденной вероятностной меры, которое будет часто 
использоваться в~дальнейшем. Пусть $(U_0, \mathscr{B}_0)$~--- некоторое измеримое 
пространство, $\mathscr{B}_0$~--- $\sigma$-ал\-геб\-ра подмножеств множества~$U_0$, 
включающая в~себя все одноточечные подмножества этого множества.

\medskip

\noindent
\textbf{Определение 1.}\ Вероятностная мера~$\Psi^*$, заданная 
на $\sigma$-ал\-геб\-рe~$\mathscr{B}_0$, называется вырожденной, если существует 
такой элемент $u^* \hm\in U_0$, для которого выполняются условия $\Psi^*(\{u^*\})\hm=
1$, $\Psi^*(U_0 \setminus \{u^*\})\hm=0$, где $\{u^*\}=u^*$~--- 
множество, состоящее из единственной точки $u^* \hm\in U_0$. Соответствующая 
точка $u^* \hm\in U_0$ будет называться точкой сосредоточения вырожденной 
вероятностной меры~$\Psi^*$.
Таким образом, всякая вырожденная вероятностная мера~$\Psi^*$ определяется 
своей точкой сосредоточения~$u^*$. В~дальнейшем будем использовать 
обозначение~$\Psi_{u^*}^{*}$, имея в~виду, что вырожденная вероятностная мера~$\Psi^*$ 
сосредоточена в~точке~$u^*$.
Отметим также, что вырожденная вероятностная мера~$\Psi_{u^*}^{*}$ соответствует 
детерминированной величине, которая принимает фиксированное значение $u\hm=u^*$ 
с~вероятностью, равной единице.

\medskip

Обозначим через $\Gamma_0$ множество всех  вероятностных мер, заданных 
на измеримом пространстве ($U_0, \mathscr{B}_0$), а через~$\Gamma_0^*$~--- 
множество всех вырожденных вероятностных мер, заданных на этом пространстве, 
$\Gamma_0^*\hm\in \Gamma_0$. Аналогичные обозначения будут использоваться 
и~в~дальнейшем. Заметим, что множество~$\Gamma_0^*$ находится во взаимно
 однозначном соответствии с~множеством точек сосредоточения вырожденных 
 вероятностных мер, т.\,е.\ с~множеством~$U_0$.

Пусть $\Gamma_i^{*}$~--- множество всех вырожденных мер, заданных на 
$\sigma$-ал\-геб\-ре~$\mathscr{B}_i$, $\Gamma_i^{*}\hm\subset \Gamma_i$.
Произвольная вероятностная мера~$\Psi_i$ описывает случайную величину, 
принимающую значения в~$U_i$, а вырожденная мера~$\Psi_i^*$, сосредоточенная 
в~точке~$u_i^*$, соответствует детерминированной величине $u_i^*\hm\in U_i$.
Предполагается, что соответствующие конструкции определены на всех измеримых 
пространствах управлений $(U_1, \mathscr{B}_1), (U_2, \mathscr{B}_2), \ldots, 
(U_N,\mathscr{B}_N)$.

Предположим, что управления случайным полумарковским процессом~$\xi(t)$ 
осуществляются в~моменты времени~$t_n,$ $n\hm=0,1,2,\ldots,$
непосредственно после изменения состояния процесса. Если\linebreak 
$\xi_n\hm=\xi(t_n)\hm=i \hm\in X$, то значение управления представляет 
собой случайную величину~$u_n$, принимающую значения в~множестве допустимых 
управ\-ле\-ний~$U_i$ и~описываемую вероятностной мерой (распределе\-ни\-ем 
вероятностей) $\Psi_i \hm\in \Gamma_i$.
Будем предполагать, что при фиксированном условии $\xi_n\hm=\xi(t_n)=i$ 
управ\-ле\-ние определяется независимо от прошлого поведения процесса~$\xi(t)$ 
и~вероятностная мера~$\Psi_i$,
описывающая стохастическое управление~$u_n$, зависит только от состояния $i\hm\in X$.
Тогда выбор управ\-ле\-ний в~моменты изменения состояний $\{t_n, n\hm=0,1,2,\ldots \}$ 
описывается набором вероятностных мер (распределений вероятностей) 
$(\Psi_1, \Psi_2,\ldots, \Psi_N)$, 
$\Psi_i \hm\in \Gamma_i$, $i\hm=1,2,\ldots,N$.
Назовем любой такой набор стратегией управ\-ле\-ния полумарковским процессом~$\xi(t)$. 
По своим свойствам такая стратегия является марковской, однородной 
и~рандомизированной.

Следуя классической монографии П.~Халмоша~\cite[гл.~VII]{18}, 
рассмотрим декартово произведение пространств $U\hm=U_1\times U_2\times \cdots\times U_N$ 
и~соответствующих $\sigma$-ал\-гебр $\mathscr{B}\hm=\mathscr{B}_1\times \mathscr{B}_2
\times \cdots \times\mathscr{B}_N$. Обозначим через $\Psi\hm=\Psi_1\times \Psi_2\times \cdots
\times \Psi_N$ вероятностную меру на~$(U,\mathscr{B})$, определяемую как 
произведение мер $\Psi_1,\Psi_2, \ldots , \Psi_N$, где $\Psi_i \hm\in \Gamma_i$, 
$i\hm=1,2,\ldots,N$. Обозначим также через~$\Gamma$ множество вероятностных мер~$\Psi$, 
заданных на~$(U,\mathscr{B})$, которые пред\-став\-ля\-ют собой произведение 
мер $\Psi_1,\Psi_2, \ldots , \Psi_N$, где $\Psi_i \hm\in \Gamma_i$, $i\hm=1,2,\ldots,N$.
Множество~$\Gamma$ можно отож\-де\-ст\-вить с~множеством всех стратегий управ\-ле\-ния 
полумарковским процессом~$\xi(t)$.

Проблема оптимального управления полумар\-ковским процессом~$\xi(t)$ будет в~дальнейшем 
сформулирована в~виде задачи безусловного экстремума некоторого функционала 
$I(\Psi)\hm=I(\Psi_1,\Psi_2, \ldots , \Psi_N)$, заданного на множестве 
допустимых стратегий управления. Содержание показателя качества управления~$I(\Psi)$, 
аналитическое представление для него, а~также описание множества допустимых 
стратегий управления будут приведены в~последующих разделах данной работы.

Для получения дальнейших результатов потребуются различные вероятностные 
характеристики управляемого полумарковского процесса~$\xi(t)$. Как известно из
 общей теории полумарковских процессов~\cite{19, 20}, 
 основной вероятностной характеристикой такого процесса является так называемая 
 полумарковская функция. Определим эту функцию для процесса с~управлением 
 (см.~\cite[гл.~5]{8}):
\begin{multline}
Q_{ij}(t,u)=
{\sf P}\left(\xi_{n+1}=j,\theta_n<t \mid \xi_n=i, u_n=u\right)\,,\\
t\in [0,\infty)\,,\ u\in U_i\,;\ i,j\in X=\{1,2,\ldots,N\}\,. \label{e1}
\end{multline}
Используя полумарковские функции, можно получить вероятности перехода 
управляемой цепи Маркова~$\{\xi_n\}$:
\begin{multline}
p_{ij}(u)={\sf P}\left(\xi_{n+1}=j \mid \xi_n=i, u_n=u\right)= {}\\
{}=
\lim\limits_{t\rightarrow\infty}Q_{ij}(t,u)\,,\enskip
u\in U_i\,;\enskip i,j\in X\,, 
\label{e2}
\end{multline}
а также функции распределения длительностей пребывания полумарковского 
процесса~$\xi(t)$ в~соответствующих состояниях:

\noindent
\begin{multline}
H_{i}(t,u)={\sf P}\left(\theta_n<t \mid \xi_n=i, u_n=u\right)={}\\
{}=
\sum\limits_{j\in X}Q_{ij}(t,u)\,,\enskip
t\in [0,\infty)\,,\  u\in U_i\,; \  i\in X\,. 
\label{e3}
\end{multline}

Обозначим через
\begin{multline}
T_{i}(u)=\mathbf{E}\left[\theta_n \mid \xi_n=i, u_n=u\right]={}\\
{}=
\int\limits_0^{\infty}\left[1-H_i(t,u)\right]\,dt\,,\enskip
u\in U_i\,,\ i\in X\,, 
\label{e4}
\end{multline}
математические ожидания длительностей пребывания полумарковского процесса~$\xi(t)$ 
в~каждом из состояний.

Введенные выше характеристики~(1)--(4) определены для случая, когда 
в~момент изменения состояния~$t_n$ процесс оказывается в~состоянии~$i$ 
и~принимается решение $u\hm\in U_i$. При заданной стратегии управления 
$\Psi\hm=\left(\Psi_1,\Psi_2, \ldots , \Psi_N\right)$ можно записать 
соответствующие вероятностные характеристики без условия на управление, а~именно:
\begin{multline*}
Q_{ij}(t)={\sf P}\left(\xi_{n+1}=j,\theta_n<t \mid \xi_n=i\right)={}\\
{}=
\int\limits_{U_i}Q_{ij}(t,u) \,d\Psi_i(u)\,,\enskip 
t\in [0,\infty)\,,\ i,j\in X\,; %\label{e5}
\end{multline*}

\vspace*{-12pt}

\noindent
\begin{multline}
p_{ij}={\sf P}\left(\xi_{n+1}=j \mid \xi_n=i\right)=
\int\limits_{U_i}p_{ij}(u)\, d\Psi_i(u)\,,\\  
i,j\in X\,; 
\label{e6}
\end{multline}

\vspace*{-9pt}

\noindent
\begin{equation}
T_{i}=\mathbf{E}\left[\theta_n \mid \xi_n=i\right]=
\int\limits_{U_i}T_{i}(u)\,d\Psi_i(u)\,,\enskip i\in X\,. 
\label{e7}
\end{equation}
В дальнейшем будем предполагать, что для рас\-смат\-ри\-ва\-емой 
полумарковской модели заданы вероятностные характеристики 
$p_{ij}(u)$, $u\hm\in U_i$, $i,j\hm\in X$, и~$T_i(u)$, $u\hm\in U_i$, $i\hm\in X$, 
определяемые соотношениями~(\ref{e2}) и~(\ref{e4}). 
Для фиксированной стратегии управления $\Psi\hm=(\Psi_1, \Psi_2,\ldots, \Psi_N)$ 
соответствующие вероятностные характеристики~$p_{ij}$ и~ $T_i$, $i,j\hm\in X,$ 
определены равенствами~(\ref{e6}) и~(\ref{e7}) без условий на управление.

\section{Стационарный стоимостной показатель качества управления}

Определим некоторый стоимостной аддитивный функционал, связанный 
с~рассматриваемым полумарковским процессом~$\xi(t)$. По содержанию этот функционал 
представляет собой случайный\linebreak доход или прибыль, накопленную за период времени $[0,t]$. 
Определения такого функционала приведены в~основополагающих работах~[3; 8, гл.~5].\linebreak 
Обозначим через $\widetilde{v}(t)$, $t\hm\geq 0$, значение этого аддитивного 
функционала в~момент времени~$t$; $\widetilde{v}_n\hm=\widetilde{v}(t_n\hm+0)$~--- 
соответствующее значение непосредственно после очередного момента изменения\linebreak 
состояния~$t_n$, $n\hm=0,1,2,\ldots$; $\widetilde{v}_0\hm=v_0$~--- 
заданное начальное значение в~момент $t\hm=0$. Рассмотрим величину
\begin{multline}
d_i(u)=\mathbf{E}\left[\widetilde{v}_{n+1}-\widetilde{v}_n \mid \xi_n=i\,, 
u_n=u\right]\,,\\
u\in U_i\,, \enskip i\in X\,, \label{e8}
\end{multline}
представляющую собой математическое ожидание приращения стоимостного 
аддитивного функционала за период времени между последовательными 
изменениями состояния полумарковского процесса~$\xi(t)$. Тогда соответствующее 
математическое ожидание, вычисляемое без условия на решение, 
принимаемое в~момент времени~$t_n$, представляется в~виде:
\begin{equation*}
d_i=\mathbf{E}\left[\widetilde{v}_{n+1}-\widetilde{v}_n \mid \xi_n=i\right]=
\!\int\limits_{U_i}\!d_i(u)\,d\Psi_i(u)\,,\ i\in X \,. %\label{e9}
\end{equation*}

Предположим, что для заданной стратегии управ\-ле\-ния 
$\Psi\hm=(\Psi_1,\Psi_2,\ldots,\Psi_N)$ вложенная цепь Маркова~$\{\xi_n\}$ 
имеет ровно один класс возвратных положительных состояний (по терминологии, 
принятой в~\cite{8}, такое множество состояний называется эргодическим классом). 
Как известно~\cite[гл.~VIII]{15}, данное условие является необходимым 
и~достаточным для существования единственного\linebreak стационарного распределения. 
Обозначим это стационарное распределение цепи Маркова~$\{\xi_n\}$ через 
$\pi\hm=(\pi_1, \pi_2,\ldots, \pi_N)$. Заметим, что данное\linebreak распределение зависит  
от стратегии управления $\Psi\hm=(\Psi_1,\Psi_2,\ldots,\Psi_N)$. При указанном 
условии имеет место следующий результат, называемый эргодической теоремой 
для аддитивного стоимостного функционала:
\begin{equation}
I=\lim\limits_{t\rightarrow\infty}\fr{\mathbf{E}\widetilde{v}(t)}{t}=
\fr{\sum\nolimits_{i=1}^N d_i\pi_i}{\sum\nolimits_{i=1}^N T_i\pi_i}\,. 
\label{e10}
\end{equation}

Соотношение~(\ref{e10}) доказано в~работе~\cite[гл.~5]{8}. Заметим, что аналогичные 
результаты имеют мес\-то для гораздо более общих полумарковских моделей~\cite{10, 11}.

По своему прикладному содержанию величина, определяемая соотношением~(\ref{e10}), 
представляет собой
среднюю удельную прибыль, связанную с~эволюцией системы в~стационарном
режиме. Кроме того, величина~$I$ представляет собой функционал от
набора вероятностных распределений~$\Psi_{i}$, $i\hm\in\lbrace 1,\ldots
,N\rbrace $, определяющих стратегию управле-\linebreak\vspace*{-12pt}

\pagebreak

\noindent
ния системой. 
В~дальнейшем будем рассматривать стационарный стоимостной функционал 
$I\hm=I(\Psi_{1},\Psi_{2},\ldots , \Psi_{N})$ как
показатель качества управ\-ле\-ния системой и~построенным полумарковским
процессом~$\xi (t)$.

\section{Представление стационарного показателя в~форме
дробно-линейного интегрального функционала}

В данном разделе будет приведено утверждение об аналитическом
представлении стационарного стоимостного функционала~(\ref{e10}), 
служащего критерием качества управления в~рассматриваемой задаче управления 
полумарковским процессом.

\smallskip

\noindent
\textbf{Теорема 1.} \textit{Стационарный стоимостной показатель, 
определяемый равенством}~(\ref{e10}), \textit{представляет собой дроб\-но-ли\-ней\-ный
функционал от вероятностных распределений~$\Psi_{i}(u_{i})$,
$i\hm\in\{1,\dots,N\}$. Данный функционал задается
аналитически следующей формулой:}
\begin{multline}
I=I(\Psi_{1},\ldots, \Psi_{N})={}\\
\hspace*{-2mm}{}=\!
\fr{\int\nolimits_{U_1}\!{\cdots\! 
\int\nolimits_{U_N}\!{A(u_{1},\ldots ,u_{N})d\Psi_{1}(u_{1})\cdots
\,d\Psi_{N}(u_{N})}}}{\int\nolimits_{U_1}{\!\cdots\! \int\nolimits_{U_N}\!{B(u_{1},\ldots ,u_{N})
\,d\Psi_{1}(u_{1})\ldots
d\Psi_{N}(u_{N})}}},\!\!\! \label{e11}
\end{multline}
\textit{где подынтегральные функции числителя и~знаменателя выражаются
соотношениями}:
\begin{align}
A(u_{1},\ldots
,u_{N})&={}\notag\\
&\hspace*{-20mm}{}=\sum\limits_{i=1}^{N}{d_{i}(u_{i})}{\widehat{D}}^{(i)}(u_{1}, \ldots
,u_{i-1},u_{i+1}, \ldots , u_{N})\,;  \label{e12}\\
 B(u_{1},\ldots
,u_{N})&={}\notag\\
&\hspace*{-20mm}{}=\sum\limits_{i=1}^{N}{T_{i}(u_{i})}{\widehat{D}}^{(i)}(u_{1}, \ldots
,u_{i-1},u_{i+1}, \ldots , u_{N})\,.  \label{e13}
\end{align}
\textit{В свою очередь, функции} ${\widehat{D}}^{(i)}(u_{1}, \ldots
,u_{i-1},u_{i+1}, \ldots$\linebreak $\ldots , u_{N})$, $i\hm\in\{1,\dots,N\}$, 
\textit{входящие в~правые части формул}~(\ref{e12}) и~(\ref{e13}), 
\textit{определяются следующим образом:}

\noindent
\begin{multline}
{\widehat{D}}^{(i)}(u_{1}, \ldots ,u_{i-1},u_{i+1}, \ldots , u_{N})={}
\\
{}=(-1)^{N+i+2}\sum\limits_{\alpha ^{(N),i}}{(-1)}^{\delta (\alpha
^{(N),i}) }{\widehat{D}}_{0}^{(i)}\left(\alpha ^{(N),i},u_{1}, \ldots\right.\\
\left.\ldots , u_{i-1},u_{i+1}, \ldots , u_{N}\right)\,. \label{e14}
\end{multline}
\textit{Здесь} $\alpha ^{(N),i}=(\alpha _{1}, \ldots , \alpha _{i-1},\alpha_{i+1}, \ldots , 
\alpha _{N})$~\textit{--- произвольная
перестановка чисел }$(1, \ldots , i-1, i+1, \ldots , N)$;
$\delta
(\alpha ^{(N),i})$~\textit{--- число инверсий в~перестановке} 
$\alpha ^{(N),i}$;

\noindent
\begin{multline}
{\widehat{D}}_{0}^{(i)}(\alpha ^{(N),i},u_{1}, \ldots ,u_{i-1},u_{i+1},
\ldots , u_{N})={}\\
{} ={\widetilde{p}}_{1,\alpha _{1}}\left(u_{1}\right)\cdots {\widetilde{p}}_{i-1,\alpha
_{i-1}}\left(u_{i-1}\right){\widetilde{p}}_{i+1,\alpha _{i+1}}\left(u_{i+1}\right)\cdots\\
\cdots
{\widetilde{p}}_{N,\alpha _{N}}\left(u_{N}\right)\,, 
\label{e15}
\end{multline}
где
\begin{multline}
 {\widetilde{p}}_{k,\alpha _{k}}(u_{k})=
\begin{cases}
p_{kk}(u_{k})-1,\  & \alpha _{k}=k\,; \\
p_{k,\alpha _{k}}(u_{k}),\  & \alpha _{k}\ne k, \\
\end{cases}\\
 k=1, \ldots , i-1, i+1, \ldots ,N\,. \label{e16}
 \end{multline}
\textit{Функции $p_{ij}(u_i)$, $T_{i}(u_{i})$ и~$d_{i}(u_{i})$,
$u_i\hm\in U_i$, $i,j\hm\in \{1,2,\ldots,N\}$, 
входящие в~соотношения}~(\ref{e12})--(\ref{e16}), 
\textit{определяются равенствами}~(\ref{e2}), (\ref{e4}) \textit{и}~(\ref{e8}) \textit{соответственно.}

\smallskip

\noindent
Д\,о\,к\,а\,з\,а\,т\,е\,л\,ь\,с\,т\,в\,о\ теоремы~1 
в~весьма сжатой форме приведено в~работе~\cite{21}. Читателю, интересующемуся 
более подробным обоснованием данного результата, порекомендуем обратиться к~тексту 
кандидатской диссертации А.\,В.~Иванова~\cite[гл.~3]{22}.

\smallskip

Итак, теорема~1 позволяет получить явное аналитическое представление 
для стационарного стоимостного показателя вида~(\ref{e10}) в~форме 
дроб\-но-ли\-ней\-но\-го интегрального функционала от набора\linebreak вероятностных мер 
$\Psi\hm=(\Psi_{1},\Psi_{2},\ldots , \Psi_{N})$, за\-да\-ющих стратегию управления 
полумарковским процессом~$\xi(t)$. При этом подынтегральные функции числителя 
и~знаменателя задаются формулами~(\ref{e12}), (\ref{e13}) 
и~вспомогательными равенствами~(\ref{e14})--(\ref{e16}). Таким образом, функция
\begin{equation}
C\left(u_1, u_2,\ldots, u_N\right)=\fr{A(u_1, u_2,\ldots, u_N)}{B(u_1, u_2,\ldots, u_N)}\,,
\label{e17}
\end{equation}
которая в~дальнейшем будет называться основной функцией дроб\-но-ли\-ней\-но\-го 
интегрального функционала~(\ref{e11}) и~которая будет играть важную роль 
в~дальнейшем исследовании, также явно определяется формулами~(\ref{e17}), 
(\ref{e12}), (\ref{e13}).

\section{Формальная постановка оптимизационной задачи 
и~условия существования оптимальной стратегии управления полумарковским процессом}

Будем рассматривать проблему управления полумарковским процессом~$\xi(t)$ в~форме 
экстремальной задачи
\begin{multline}
I(\Psi)=I\left(\Psi_1, \Psi_2,\ldots,\Psi_N\right)\rightarrow \mathrm{extr}\,,
\\
\Psi=\left(\Psi_1, \Psi_2,\ldots,\Psi_N\right)\in\Gamma\,. \label{e18}
\end{multline}
При этом показатель качества управления~$I(\Psi)$ представляет собой 
дроб\-но-ли\-ней\-ный интегральный функционал вида~(\ref{e11}).

Для решения экстремальной задачи~(\ref{e18}) воспользуемся некоторым утверждением 
об экстремуме дроб\-но-ли\-ней\-но\-го интегрального функционала. Прежде 
чем сформулировать данное утверждение, отметим, что в~теории оптимизации 
хорошо известны задачи, в~которых целевая функция представляет собой 
отношение двух линейных отображений, а имеющиеся ограничения также линейны. 
Такой раздел называется дроб\-но-ли\-ней\-ным программированием. Основные
 теоретические результаты данного направления изложены в~работе~\cite{23},
  там же приведена подробная библиография. В~дальнейшем потребуется некоторый 
  специальный результат о безусловном экстремуме дроб\-но-ли\-ней\-но\-го 
  интегрального функционала вида~(\ref{e11}), который был впервые сформулирован 
  в~работе~\cite{14}. Заметим, что для использования этого результата необходимо, 
  чтобы выполнялись некоторые предварительные условия, которые в~данном случае 
  можно сформулировать следующим образом:
\begin{enumerate}[1.]
\item Интегральные выражения
\begin{align*}
I_1(\Psi)&=I_1\left(\Psi_1,\Psi_2,\ldots,\Psi_N\right)={}&\\
&\hspace*{-13mm}{}=\int\limits_{U_1}\!\cdots\!
\int\limits_{U_N}\!\!A\left(u_1,\ldots ,u_N\right)\,
d\Psi_1\left(u_1\right) %d\Psi_2\left(u_2\right)
\cdots
 d\Psi_N\left(u_N\right)\,;
\\
I_2(\Psi)&=I_2\left(\Psi_1,\Psi_2,\ldots,\Psi_N\right)={}&\\
&\hspace*{-13mm}{}=\int\limits_{U_1}\!\cdots\!\int\limits_{U_N}\!\!
B\left(u_1,\ldots,u_N\right)\,
d\Psi_1\left(u_1\right)% d\Psi_2\left(u_2\right)\cdots\\
\cdots d\Psi_N\left(u_N\right)
\end{align*}
определены для всех стратегий управления $\Psi\hm=(\Psi_1, \ldots,\Psi_N)
\hm\in \Gamma$.

\item Функционал $I_2(\Psi)=I_2(\Psi_1, \ldots,\Psi_N)\hm\neq 0$ 
для всех $\Psi\hm=(\Psi_1, \ldots,\Psi_N)\hm\in \Gamma$.

\item Множество $\Gamma$ включает в~себя множество всех вырожденных 
вероятностных мер: $\Gamma^* \hm\subset \Gamma$.
\end{enumerate}

Сделаем несколько важных замечаний по поводу введенных предварительных условий.

\smallskip

\noindent
\textbf{Замечание~1.}\ Из условия~2 следует, что функция $B(u_1, u_2,\ldots, u_N)$ 
не может принимать значения разных знаков. С~учетом условия~3 
получаем, что указанная функция должна обладать \mbox{свойством} строгой 
знакопостоянности на всем множестве~$U$. С~другой стороны, если выполняется 
условие строгой знакопостоянности функции $B(u_1, u_2,\ldots, u_N), 
(u_1, u_2,\ldots, u_N)\hm\in U$, то условие~2 выполняется автоматически.

\smallskip

\noindent
\textbf{Замечание~2.}\ Если рассматривать в~качестве целевого функционала 
$I(\Psi_1, \Psi_2,\ldots,\Psi_N)$ экстремальной задачи~(\ref{e18}) 
стационарный стоимостной пока\-затель~(\ref{e10}), то функция $B(u_1,u_2,\ldots,u_N)$ 
имеет\linebreak следующее теоретическое содержание. Данная функция представляет собой условное 
математическое ожидание длительности периода времени между соседними моментами 
изменения со\-сто\-яния полумарковского процесса~$\xi(t)$ при условии, что стратегия 
его управ\-ле\-ния является детерминированной и~задается набором значений аргументов 
$(u_1,u_2,\ldots,u_N)$. Тогда условие строгой положительности функции 
$B(u_1,u_2,\ldots,u_N)$ при всех $(u_1,u_2,\ldots,u_N)\hm\in U$ является естественным 
и~фактически означает, что при любой заданной детерминированной стратегии 
управ\-ле\-ния процесс~$\xi(t)$ не имеет мгновенных со\-сто\-яний, длительность пребывания 
в~которых равна нулю.

\smallskip

\noindent
\textbf{Замечание~3.}\ Сделаем некоторые замечания, связан\-ные с~подынтегральной 
функцией числителя дроб\-но-ли\-ней\-но\-го интегрального функционала~(\ref{e11}). 
Как и~ранее, будем рассматривать в~качестве целевого функционала $I(\Psi_1, \Psi_2,\ldots,\Psi_N)$\linebreak 
экстремальной задачи~(\ref{e18}) стационарный стоимостной показатель~(\ref{e10}). 
Тогда для любого фиксированного набора значений аргументов $(u_1,u_2,\ldots,u_N)\hm\in U$ 
значение функции $A(u_1,u_2,\ldots\linebreak \ldots,u_N)$ представляет собой условное математическое
 ожидание приращения рассматриваемого стоимостного функционала, 
 происшедшее за время пребывания полумарковского процесса~$\xi(t)$ в~некотором 
 фиксированном  состоянии при условии, что стратегия управления является 
 детерминированной и~задается указанным набором $(u_1,u_2,\ldots,u_N)\hm\in U$. 
 Отметим, что в~теореме об экстремуме дроб\-но-ли\-ней\-но\-го интегрального 
 функционала, доказанной в~работе~\cite[гл.~10]{12}, 
 на подынтегральную функцию числителя накладываются условия ограниченности на 
 всем множестве значений аргумента. Для многих математических моделей и~связанных 
 с~ними задач оптимального управления такое условие является излишне ограничительным. 
 В~качестве примера можно привести модели оптимального управления запасом непрерывного 
 продукта, рассмотренные в~работах~\cite{27, 28}. 
 В~настоящем исследовании на функцию $A(u_1,u_2,\ldots,u_N)$ накладывается только 
 условие интегрируемости по любому заданному набору вероятностных мер 
 $\Psi\hm=(\Psi_1, \Psi_2,\ldots,\Psi_N)$, образующему стратегию управления 
 полумарковским процессом~$\xi(t)$ (условие~1 системы предварительных условий).

\smallskip

\noindent
\textbf{Замечание~4.} Условия~1--3 являются необходимыми для корректной 
постановки задачи безусловного экстремума дроб\-но-ли\-ней\-но\-го интегрального 
функционала. Если этот функционал служит показателем качества в~задаче оптимального 
управления случайным процессом, то необходимо добавить к~этим условиям дополнительное, 
связанное с~некоторой регулярностью самого управляемого процесса, а~именно: некоторый 
содержательный показатель, связанный с~поведением этого процесса, должен существовать 
и~быть представимым в~виде дроб\-но-ли\-ней\-но\-го интегрального функционала. 
Если потребовать, чтобы выполнялось эргодическое соотношение~(\ref{e10}), 
то можно использовать\linebreak теорему~1 и~сформулировать задачу оптимального управ\-ле\-ния 
в~виде~(\ref{e18}) для дроб\-но-ли\-ней\-но\-го\linebreak интегрального функционала~(\ref{e11}). 
Таким образом, необходимо ввести условие, обеспечивающее существование единственного 
стационарного распределения вложенной цепи Маркова и~выполнение\linebreak соотношения~(\ref{e10}). 
По аналогии с~[8, гл.~5] сформулируем это дополнительное условие в~следующем виде:
\begin{enumerate}
\setcounter{enumi}{3}
\item Для любой рассматриваемой стратегии управ\-ле\-ния $\Psi\hm=
(\Psi_1, \Psi_2,\ldots,\Psi_N)\hm\in \Gamma$ вложенная цепь Маркова 
полумарковского процесса $\xi(t)$ имеет ровно один класс возвратных 
положительных состояний.
\end{enumerate}

Теперь определим понятие допустимой стратегии управления полумарковским процессом 
с~конечным множеством состояний.

\smallskip

\noindent
\textbf{Определение~2.}\ Назовем стратегию управления 
$\Psi\hm=(\Psi_1, \Psi_2,\ldots,\Psi_N)$ 
допустимой в~данной задаче, если она удовлетворяет условиям~1--4.


\smallskip

\noindent
\textbf{Замечание~5.}\ Как следует из замечания~1, если потребовать, 
чтобы функция $B(u_1, u_2,\ldots,u_N)$ являлась строго знакопостоянной при 
всех $(u_1, u_2,\ldots,u_N)\hm\in U$, то можно считать допустимыми стратегии 
$(\Psi_1, \Psi_2,\ldots,\Psi_N)$, удовлетворяющие условиям~1, 3, 4. С~учетом замечания~2 
о~естественном характере условия строгой знакопостоянности функции $B(u_1,u_2,\ldots,u_N)$ 
при всех значениях аргументов $(u_1, u_2,\ldots,u_N)\hm\in U$ будем требовать 
выполнения этого условия в~формулировке приводимой в~дальнейшем основной 
теоремы об оптимальной стратегии управления полумарковским процессом.

\smallskip

\noindent
\textbf{Замечание~6.}\ Ниже будет сформулирована и~доказана основная 
теорема об оптимальной стра\-тегии управления полумарковским процессом с~конеч\-ным 
множеством состояний. Будем формулировать эту теорему по отношению к~экстремальной 
задаче~(\ref{e18}), в~которой целевой функционал $I(\Psi_1, \Psi_2,\ldots,\Psi_N)$ 
имеет вид дроб\-но-ли\-ней\-но\-го интегрального функционала. 
Это обстоятельство связано с~тем, что целевой функционал в~задаче 
оптимального управления необязательно должен иметь характер стационарного 
стоимостного показателя вида~(\ref{e10}). В~частности, еще в~1983~г.\ П.\,В.~Шнурковым 
было установлено~\cite{24}, что ряд показателей, связанных 
с~временем пребывания управляемого полумарковского процесса в~заданном конечном 
подмножестве состояний, имеет структуру дроб\-но-ли\-ней\-но\-го интегрального 
функционала от набора вероятностных мер, определяющих стратегию управления. 
Таким образом, рассматриваемая задача управления имеет более общий характер, 
чем задача, в~которой целевой функционал представляет собой стационарный 
стоимостной показатель вида~(\ref{e10}).






\smallskip

\noindent
\textbf{Замечание~7.}\ Если рассматривать задачу оптимального управления 
полумарковским процессом, в~кото\-рой целевой функционал не совпадает 
со стационарным стоимостным показателем~(\ref{e10}), то возможно, что могут 
потребоваться другие дополнительные условия, обеспечивающие существование этого 
показателя и~его представление в~форме~(\ref{e11}). В~связи с~этим в~формулировке 
основной теоремы будем использовать термин допустимые стратегии в~широком смысле, 
имея в~виду выполнение всех необходимых условий для каждого рассмат\-ри\-ва\-емо\-го 
показателя качества управления.

\smallskip


\noindent
\textbf{Замечание 8.} Множество допустимых стратегий может 
не совпадать с~множеством всех возможных стратегий управления. 
В~частности, допустимые стратегии могут состоять только из дискретных вероятностных 
мер $\Psi_1, \Psi_2,\ldots,\Psi_N$, т.\,е.\ таких, которые сосредоточены на дискретных 
множествах точек пространств $U_1, U_2,\ldots,U_N$.

\section{Теоретическое решение задачи оптимального управления}

Перейдем к~формулировке и~доказательству тео\-ре\-мы об 
оптимальной стратегии управ\-ле\-ния полумарковским процессом с~конечным 
множеством состояний.

\smallskip

\noindent
\textbf{Теорема~2.} \textit{Рассмотрим проблему оптимального управ\-ле\-ния 
полумарковским процессом~$\xi(t)$ в~виде экстремальной задачи}~(\ref{e18}), 
\textit{определенной на множестве допустимых стратегий $\Gamma$, 
для дроб\-но-ли\-ней\-но\-го 
функционала}~(\ref{e11}). \textit{Пусть функция $B(u_1,u_2,\ldots,u_N)$, 
входящая в~определение функционала}~(\ref{e11}),
\textit{является строго знакопостоянной (строго положительной или строго отрицательной) 
при всех значениях аргументов $(u_1,u_2,\ldots,u_N)\hm\in U$.
Тогда справедливы сле\-ду\-ющие утверждения}:
\begin{enumerate}[1.]
\item \textit{Если функция} $C(u_1,u_2,\ldots,u_N)\hm=A(u_1,u_2,\ldots$\linebreak
$\ldots,u_N)/{B(u_1,u_2,\ldots,u_N)}$ 
\textit{ограничена сверху или снизу и~достигает глобального экст\-ре\-му\-ма на множестве
$U\hm=U_1\times U_2\times \cdots \times U_N$ (максимума или минимума), 
то оптимальная стратегия управления полумарковским процессом~$\xi(t)$ существует, 
является детерминированной и~определяется
вырожденной вероятностной мерой $\Psi^*\hm\in \Gamma^*$, сосредоточенной в~точке, 
в~которой достига\-ет соответствующего экстремума функция $C(u_1,u_2,\ldots,u_N)$,
и~при этом выполняются соотношения}:
\begin{multline}  %{\substack{{i=\overline{1,n}}\\ {j=\overline{1,l}}}}
\max\limits_{\Psi \in \Gamma} I(\Psi)=
\max\limits_{\substack{{\Psi_i \in \Gamma_i\,,}\\ 
{i=\overline{1,N}}}}
I\left(\Psi_1,\Psi_2,\ldots,\Psi_N\right)={}\\
{}=
\max\limits_{\substack{{\Psi_i^* \in \Gamma_i^*,}\\ 
{i=\overline{1,N}}}}
 I\left(\Psi_1^*,\Psi_2^*,\ldots,\Psi_N^*\right)={}\\
{}=\max\limits_{(u_1,u_2,\ldots,u_N)\in U}\fr{A(u_1,u_2,\ldots,u_N)}
{B(u_1,u_2,\ldots,u_N)}\,; \label{e19}
\end{multline}

\vspace*{-12pt}

\noindent
\begin{multline*}
\min\limits_{\Psi \in \Gamma} I(\Psi)=
\min\limits_{\substack{{\Psi_i \in \Gamma_i\,,}\\ 
{i=\overline{1,N}}}} I\left(\Psi_1,\Psi_2,\ldots,\Psi_N\right)={}\\
{}=
\min\limits_{\substack{{\Psi_i^* \in \Gamma_i^*,}\\ 
{i=\overline{1,N}}}}
I\left(\Psi_1^*,\Psi_2^*,\ldots,\Psi_N^*\right)={}\\
{}=\min\limits_{(u_1,u_2,\ldots,u_N)\in U}\fr{A(u_1,u_2,\ldots,u_N)}
{B(u_1,u_2,\ldots,u_N)}\,. %\label{e20}
\end{multline*}
\item \textit{Если функция $C(u_1,u_2,\ldots,u_N)\hm=
{A(u_1,u_2,\ldots,u_N)}/{B(u_1,u_2,\ldots,u_N)}$ ограничена сверху или снизу, 
но не достигает глобального экстремума на множестве $U\hm=U_1\times U_2\times\cdots
\times U_N$,
то для любого $\varepsilon\hm > 0$ можно выбрать $\varepsilon$-оп\-ти\-маль\-ную 
детерминированную стратегию управления полумарковским процессом~$\xi(t)$, 
которая определяется вырожденной
вероятностной мерой $\Psi^{*(+)}(\varepsilon)\hm\in \Gamma^*$ или вырожденной
вероятностной мерой $\Psi^{*(-)}(\varepsilon)\hm\in \Gamma^*$, в~зависимости от 
вида экстремума (максимума или минимума) в~задаче}~(\ref{e18}). 
\textit{При этом вероятностная мера $\Psi^{*(+)}(\varepsilon)\hm\in \Gamma^*$ может быть 
сосредоточена в~любой точке $\left(u_1^{(+)}(\varepsilon),u_2^{(+)}(\varepsilon),\ldots,
u_N^{(+)}(\varepsilon)\right)$, удовлетворяющей соотношению}:
\begin{multline}
\sup\limits_{(u_1,u_2,\ldots,u_N) \in U}
\fr{A(u_1,u_2,\ldots,u_N)}{B(u_1,u_2,\ldots,u_N)}-\varepsilon <{}\\
{}<
\fr{A\left(u_1^{(+)}(\varepsilon),u_2^{(+)}(\varepsilon),\ldots,u_N^{(+)}
(\varepsilon)\right)}
{B\left(u_1^{(+)}(\varepsilon),u_2^{(+)}(\varepsilon),\ldots,u_N^{(+)}
(\varepsilon)\right)}<{}\\
{}<\sup\limits_{(u_1,u_2,\ldots,u_N) \in U}
\fr{A(u_1,u_2,\ldots,u_N)}{B(u_1,u_2,\ldots,u_N)}<\infty\,, 
\label{e21}
\end{multline}
\textit{если функция $C(u_1,u_2,\ldots,u_N)$ ограничена сверху 
и~экстремальная задача}~(\ref{e18}) 
\textit{представляет собой задачу на максимум. Аналогично вероятностная мера 
$\Psi^{*(-)}(\varepsilon)\hm\in \Gamma^*$ может быть сосредоточена в~любой точке 
$\left(u_1^{(-)}(\varepsilon),u_2^{(-)}(\varepsilon),\ldots,u_N^{(-)}(\varepsilon)
\right)$, удовлетворяющей соотношению}:

\noindent
\begin{multline*}
-\infty<\inf\limits_{(u_1,u_2,\ldots,u_N) \in U}\fr{A(u_1,u_2,\ldots,u_N)}
{B(u_1,u_2,\ldots,u_N)} <{}\\
{}<
\fr{A\left(u_1^{(-)}(\varepsilon),u_2^{(-)}
(\varepsilon),\ldots,u_N^{(-)}(\varepsilon)\right)}
{B\left(u_1^{(-)}(\varepsilon),u_2^{(-)}(\varepsilon),\ldots,
u_N^{(-)}(\varepsilon)\right)}<{}\\
{}<\inf\limits_{(u_1,u_2,\ldots,u_N) \in U}
\fr{A(u_1,u_2,\ldots,u_N)}{B(u_1,u_2,\ldots,u_N)}+\varepsilon\,, 
%\label{e22}
\end{multline*}
\textit{если функция $C(u_1,u_2,\ldots,u_N)$ ограничена снизу и~экстремальная 
задача}~(\ref{e18})  \textit{представляет собой задачу на минимум}.
\item \textit{Если функция $C(u_1,u_2,\ldots,u_N)\hm=
{A(u_1,u_2,\ldots,u_N)}/{B(u_1,u_2,\ldots,u_N)}$ не ограничена сверху 
или снизу, то оптимальной стратегии управления в~смысле
соответствующей экстремальной задачи не существует. 
При этом найдется такая последовательность вырожденных вероятностных
мер~$\Psi^{*(+)}(n)$, сосредоточенных в~точках 
$\left(u_1^{(+)}(n),u_2^{(+)}(n),\ldots,u_N^{(+)}(n)\right)$, $n\hm=1,2,\dots $, 
для которых выполняется соотношение}:
\begin{multline*}
I\left(\Psi^*(n)\right)={}\\
{}=
I\left(\Psi_1^{*(+)}(n),\Psi_2^{*(+)}(n),\ldots,\Psi_N^{*(+)}(n)\right)={}\\
{}=\fr{A\left(u_1^{(+)}(n),u_2^{(+)}(n),\ldots,u_N^{(+)}(n)\right)}
{B\left(u_1^{(+)}(n),u_2^{(+)}(n),\ldots,u_N^{(+)}(n)\right)}\to 
\infty\\
\mbox{при}\ n\to\infty\,, 
%\label{e23}
\end{multline*}
\textit{если функция $C(u_1,u_2,\ldots,u_N)$ не ограничена сверху. 
Аналогично найдется такая последовательность вырожденных вероятностных
мер~$\Psi^{*(-)}(n)$, сосредоточенных в~точках 
$\left(u_1^{(-)}(n),u_2^{(-)}(n),\ldots,u_N^{(-)}(n)\right)$, 
$n\hm=1,2,\dots $, для которых выполняется соотношение}:
\begin{multline*}
I\left(\Psi^{*(-)}(n)\right)={}\\
{}= I
\left(\Psi_1^{*(-)}(n),\Psi_2^{*(-)}(n),\ldots,\Psi_N^{*(-)}(n)\right)={}\\
{}=\fr{A\left(u_1^{(-)}(n),u_2^{(-)}(n),\ldots,u_N^{(-)}(n)\right)}
{B\left(u_1^{(-)}(n),u_2^{(-)}(n),\ldots,u_N^{(-)}(n)\right)}\to 
-\infty\\
\mbox{при}~~n\to\infty\,,  
%\label{e24}
\end{multline*}
\textit{если функция $C(u_1,u_2,\ldots,u_N)$ не ограничена \mbox{снизу}}.
\end{enumerate}
\textit{При этом сформулированные утверждения каждого пункта теоремы~$2$ 
могут выполняться как по отдельности, для одного из двух
видов экстремума, так и~совместно, для обоих видов экстремума.}

\smallskip

Прежде чем непосредственно доказывать теорему~2, докажем некоторые 
вспомогательные утверждения.

\smallskip

\noindent
\textbf{Лемма~1.}\ 
\textit{Рассмотрим дроб\-но-ли\-ней\-ный интегральный функционал 
$I(\Psi_1, \Psi_2,\ldots, \Psi_N)$ вида}~(\ref{e11}), 
\textit{заданный на некотором множестве наборов вероятностных мер 
$\Psi\hm=(\Psi_1, \Psi_2,\ldots, \Psi_N)\hm \in \Gamma$. Предположим, что на 
множестве~$\Gamma$ выполняется условие~$1$ из набора предварительных условий 
и~функция $B(u_1, u_2,\ldots, u_N)$  обладает свойством строгой знакопостоянности 
при всех $(u_1, u_2,\ldots, u_N) \hm\in U$. Тогда справедливы следующие утверждения}:
\begin{enumerate}[1.]
\item \textit{Если основная функция 
$C(u_1, u_2,\ldots, u_N)\hm={A(u_1, u_2,\ldots, u_N)}/{B(u_1, u_2,\ldots, u_N)}$ 
ограничена сверху, т.\,е.\ выполняется условие}
\begin{multline}
C\left(u_1, u_2,\ldots, u_N\right)=
\fr{A(u_1, u_2,\ldots, u_N)}{B(u_1, u_2,\ldots, u_N)}\leq {}\\
{}\leq
c_0^{(+)}<\infty \,, \enskip \left(u_1, u_2,\ldots, u_N\right) \in U\,, \label{e25}
\end{multline}
\textit{то имеет место неравенство}:
\begin{equation}
I\left(\Psi_1, \Psi_2,\ldots, \Psi_N\right)\leq c_0^{(+)} 
\label{e26}
\end{equation}
\textit{для всех} $(\Psi_1, \Psi_2,\ldots, \Psi_N) \in \Gamma$.
\item \textit{Если основная функция 
$C(u_1, u_2,\ldots, u_N)\hm={A(u_1, u_2,\ldots, u_N)}/{B(u_1, u_2,\ldots, u_N)}$ 
ограничена снизу, т.\,е.\ выполняется условие}
\begin{multline*}
C\left(u_1, u_2,\ldots, u_N\right)=\fr{A(u_1, u_2,\ldots, u_N)}{B(u_1, u_2,\ldots, 
u_N)}\geq{}\\
{}\geq c_0^{(-)}>-\infty \,, 
\left(u_1, u_2,\ldots, u_N\right) \in U\,, 
%\label{e27}
\end{multline*}
\textit{то имеет место неравенство}:
\begin{equation*}
I\left(\Psi_1, \Psi_2,\ldots, \Psi_N\right)\geq c_0^{(-)} 
%\label{e28}
\end{equation*}
\textit{для всех} $(\Psi_1, \Psi_2,\ldots, \Psi_N) \hm\in \Gamma$.
\end{enumerate}

\noindent
Д\,о\,к\,а\,з\,а\,т\,е\,л\,ь\,с\,т\,в\,о\ \ леммы~1.\ 
Докажем первое утверждение леммы. Предположим сначала, 
что функция $B(u_1, u_2,\ldots,  u_N)$ строго положительна:
\begin{equation}
B\left(u_1, u_2,\ldots, u_N\right)>0\,,\enskip
\left(u_1, u_2,\ldots, u_N\right)\in U\,. \label{e29}
\end{equation}
Заметим, что в~таком случае по свойству интеграла~\cite[гл.~V]{18}
\begin{multline}
\hspace*{-2mm}\int\limits_{U_1}\!\!\cdots\! \!\int\limits_{U_N}\!\!B(u_1, \ldots,u_N) \,
d\Psi_1(u_1)%d\Psi_2(u_2)\cdots\\
\cdots d\Psi_N(u_N)>0 \!\!\!\!\label{e30}
\end{multline}
для любого фиксированного набора $\Psi\hm=(\Psi_1, \ldots, \Psi_N)\hm\in \Gamma$.
Из неравенства~(\ref{e25}) с~уче\-том~(\ref{e29}) получаем:
\begin{multline}
\hspace*{-4mm}A\left(u_1,\ldots, u_N\right)\leq{}\\
\hspace*{-4mm}{}\leq c_0^{(+)} B\left(u_1, \ldots, u_N\right)\,, 
\left(u_1, \ldots, u_N\right)\in U\,. \label{e31}
\end{multline}
В свою очередь, из неравенства~(\ref{e31}) и~свойств интеграла следует:
\begin{multline}
\int\limits_{U_1}\!\!\cdots\! \!\int\limits_{U_N}\!\!A(u_1,\ldots, u_N) \,
d\Psi_1\left(u_1\right)%d\Psi_2\left(u_2\right)\cdots\\
\cdots d\Psi_N\left(u_N\right)\leq\\
\hspace*{-24pt}\leq 
c_0^{(+)}\!\!\int\limits_{U_1}\!\!\cdots\!\! \int\limits_{U_N}\!\!\!B\!\left(u_1,\ldots, u_N\right)
 d\Psi_1\!\left(u_1\right)\!%d\Psi_2\left(u_2\right)\cdots\\
 \cdots d\Psi_N\!\left(u_N\right)\!\! 
 \label{e32}
\end{multline}
для любого фиксированного набора $\Psi\hm=(\Psi_1, \ldots, \Psi_N)\hm\in \Gamma$. 
Но тогда из~(\ref{e32}) с~учетом~(\ref{e30}) получаем:
\begin{multline}
I(\Psi_1, \ldots, \Psi_N)={}\\
{}=
\fr{\int\nolimits_{U_1}\!\cdots\! \int\nolimits_{U_N}\!\!A\left(u_1, \ldots, u_N\right)\,
 d\Psi_1(u_1)\cdots d\Psi_N(u_N)}{
\int\nolimits_{U_1}\!\cdots\! \int\nolimits_{U_N}\!\!B\left(u_1, \ldots, u_N\right)\,
 d\Psi_1(u_1)
 \cdots d\Psi_N(u_N)}\leq{}\\
 {}\leq c_0^{(+)} 
 \label{e33}
\end{multline}
для любого фиксированного набора $(\Psi_1, \ldots\linebreak\ldots, \Psi_N)\hm\in \Gamma$.

Предположим теперь, что функция $B(u_1,\ldots, u_N)$ строго отрицательна:
\begin{equation}
B(u_1,\ldots, u_N)<0 \quad \left(u_1, \ldots, u_N\right)\in U\,. 
\label{e34}
\end{equation}
Тогда
\begin{multline}
\hspace*{-6pt}\int\limits_{U_1}\!\!\cdots\!\! \int\limits_{U_N}\!\!B\!\left(u_1,\ldots, u_N\right)\!
 d\Psi_1(u_1) \cdots d\Psi_N(u_N)<0 \!\!\!
 \label{e35}
\end{multline}
для любого фиксированного набора $(\Psi_1, \ldots\linebreak \ldots, \Psi_N)\hm\in \Gamma$.

Как и~ранее, будем исходить из неравенства~(\ref{e25}). 
При выполнении условий~(\ref{e34}) и~(\ref{e35}) характер неравенств~(\ref{e31}) 
и~(\ref{e32}) меняется на противоположный, но характер неравенства~(\ref{e33}) 
остается неизменным. Таким образом, для любой функции 
$B(u_1, u_2,\ldots, u_N)$, обладающей свойством строгой знакопостоянности, 
из условия~(\ref{e25}) следует выполнение неравенства~(\ref{e33}), 
которое совпадает с~(\ref{e26}). Первое утверждение леммы~1 доказано. 
Второе утверждение доказывается аналогично. Лемма~1 доказана.

\smallskip

\noindent
\textbf{Лемма 2.} \textit{Рассмотрим дроб\-но-ли\-ней\-ный интегральный функционал 
$I(\Psi_1, \Psi_2,\ldots, \Psi_N)$ вида}~(\ref{e11}), 
\textit{заданный на некотором множестве наборов вероятностных мер 
$\Psi\hm=(\Psi_1, \Psi_2,\ldots, \Psi_N)\hm\in \Gamma$. Предпо\-ложим, что на 
множестве~$\Gamma$ выполняется условие~$1$ из набора предварительных условий 
и~функция $B(u_1, u_2,\ldots, u_N)$ обладает свойством строгой знакопостоянности 
при всех $(u_1, u_2,\ldots, u_N)\hm\in U$. Тогда справедливы следующие утверждения}:
\begin{enumerate}[1.]
\item \textit{Если основная функция $C(u_1, u_2,\ldots, u_N)\hm=
{A(u_1, u_2,\ldots, u_N)}/{B(u_1, u_2,\ldots, u_N)}$ ограничена сверху, 
но не достигает своего максимального 
значения, то имеет место неравенство}:
\begin{multline}
I\left(\Psi_1, \Psi_2,\ldots, \Psi_N\right)<{}\\
{}< \sup\limits_{(u_1, u_2,\ldots, u_N)\in U}
 C\left(u_1, u_2,\ldots, u_N\right)<\infty \label{e36}
\end{multline}
\textit{для всех} $(\Psi_1, \Psi_2,\ldots, \Psi_N)\in \Gamma$.
\item \textit{Если основная функция $C(u_1, u_2,\ldots, u_N)\hm=
{A(u_1, u_2,\ldots, u_N)}/{B(u_1, u_2,\ldots, u_N)}$ ограничена снизу, 
но не достигает своего минимального значения, то имеет место неравенство}:
\begin{multline*}
I\left(\Psi_1, \Psi_2,\ldots, \Psi_N\right)>{}\\
{}> \inf\limits_{(u_1, u_2,\ldots, u_N)\in U} 
C\left(u_1, u_2,\ldots, u_N\right)>-\infty 
%\label{e37}
\end{multline*}
\textit{для всех} $(\Psi_1, \Psi_2,\ldots, \Psi_N)\hm\in \Gamma$.
\end{enumerate}

\noindent
Д\,о\,к\,а\,з\,а\,т\,е\,л\,ь\,с\,т\,в\,о\ \ леммы~2. 
Докажем первое утверждение леммы. Поскольку множество значений 
основной функции $C(u_1, u_2,\ldots, u_N)$ ограничено сверху, оно имеет конечную 
верхнюю грань:
$$
\exists \sup\limits_{(u_1, u_2,\ldots, u_N)\in U} 
C\left(u_1, u_2,\ldots, u_N\right)<\infty
$$
(см.~\cite[гл.~1, \S3, п.~3.4, теорема~1]{25}).

По условию функция $C(u_1, u_2,\ldots, u_N)$ не достигает своего максимального 
значения. Следовательно, выполняется неравенство:
\begin{multline}
C(u_1, u_2,\ldots, u_N)=\fr{A(u_1, u_2,\ldots, u_N)}{B(u_1, u_2,\ldots, u_N)}<{}\\
{}< 
\sup\limits_{(u_1, u_2,\ldots, u_N)\in U} C(u_1, u_2,\ldots, u_N)<\infty\,, 
\\
\left(u_1, u_2,\ldots, u_N\right)\in U\,.
\label{e38}
\end{multline}
Взяв за основу строгое неравенство~(\ref{e38}), проведем рассуждения, аналогичные тем, 
которые были проведены в~лемме~1 по отношению к~неравенству~(\ref{e25}). 
В~результате получим строгое неравенство~(\ref{e36}).

Второе утверждение леммы~2 доказывается аналогично. Лемма~2 доказана.

\noindent
Д\,о\,к\,а\,з\,а\,т\,е\,л\,ь\,с\,т\,в\,о\ 
\ теоремы~2.
Начнем с~доказательства утверждения~1. Предположим сначала, что основная 
функция $C(u_1, u_2,\ldots, u_N)={A(u_1, u_2,\ldots, u_N)}/{B(u_1, u_2,\ldots, u_N)}$ 
ограничена сверху и~достигает глобального максимума на множестве~$U$ 
в~некоторой точке $u^{(+)}\hm=\left(u^{(+)}_1,u^{(+)}_2,\ldots,u^{(+)}_N\right)\hm\in U$,
а~именно:
\begin{multline*}
\max\limits_{(u_1, u_2,\ldots, u_N)\in U} C\left(u_1, u_2,\ldots, u_N\right) = {}\\
{}=
C\left(u^{(+)}_1,u^{(+)}_2,\ldots,u^{(+)}_N\right)<\infty\,.
\end{multline*}
Тогда выполняется соотношение:
\begin{multline}
C(u_1, u_2,\ldots, u_N)=\fr{A(u_1, u_2,\ldots, u_N)}{B(u_1, u_2,\ldots, u_N)}
\leq{}\\
{}\leq C\left(u^{(+)}_1,u^{(+)}_2,\ldots,u^{(+)}_N\right)<\infty\,, 
\\
\left(u_1, u_2,\ldots, u_N\right)\in U\,.
\label{e39}
\end{multline}
Условия леммы~1 выполнены, и~можно воспользоваться ее утверждениями. 
Согласно первому из них, если выполняется неравенство~(\ref{e39}), 
то имеет место соотношение:
\begin{equation*}
I(\Psi_1, \Psi_2,\ldots, \Psi_N)\leq 
C\left(u^{(+)}_1,u^{(+)}_2,\ldots,u^{(+)}_N\right)<\infty 
%\label{e40}
\end{equation*}
для всех стратегий управления $\Psi\hm=(\Psi_1, \Psi_2,\ldots\linebreak
\ldots, \Psi_N)\hm\in \Gamma$.

Таким образом, множество значений дроб\-но-ли\-ней\-но\-го интегрального 
функционала $I(\Psi_1, \Psi_2,\ldots, \Psi_N)$ ограничено сверху при всех 
$\Psi\hm=(\Psi_1, \Psi_2,\ldots, \Psi_N)\hm\in \Gamma$. Тогда существует верхняя 
грань этого множества и~выполняется неравенство:
\begin{multline}
\sup\limits_{(\Psi_1, \Psi_2,\ldots, \Psi_N)\in \Gamma} 
I\left(\Psi_1, \Psi_2,\ldots, \Psi_N\right)\leq {}\\
{}\leq
C\left(u^{(+)}_1,u^{(+)}_2,\ldots,u^{(+)}_N\right). \label{e41}
\end{multline}
Рассмотрим детерминированную стратегию управ\-ле\-ния 
$\Psi^{*(+)}\hm=\left(\Psi_1^{*(+)}, \Psi_2^{*(+)},\ldots, \Psi_N^{*(+)}\right)$, 
в~которой каждая вероятностная мера~$\Psi_i^{*(+)}$ является вы\-рож\-ден\-ной 
и~сосредоточена в~точке $u_i^{(+)}$, $i\hm=\overline{1, N}$.
По свойству интеграла
\begin{multline}
I\left(\Psi_1^{*(+)}, \Psi_2^{*(+)},\ldots ,\Psi_N^{*(+)}\right)={}\\
{}=
C\left(u^{(+)}_1,u^{(+)}_2,\ldots,u^{(+)}_N\right). \label{e42}
\end{multline}
Из соотношений~(\ref{e41}) и~(\ref{e42}) получаем:
\begin{multline}
\sup\limits_{(\Psi_1, \Psi_2,\ldots, \Psi_N)\in \Gamma} 
I\left(\Psi_1, \Psi_2,\ldots, \Psi_N\right)\leq{}\\
{}\leq
 I\left(\Psi_1^{*(+)}, 
\Psi_2^{*(+)},\ldots, \Psi_N^{*(+)}\right). \label{e43}
\end{multline}
Заметим дополнительно, что выполняются отношения принадлежности:
\begin{equation}
\Psi^{*(+)}=\left(\Psi_1^{*(+)}, \Psi_2^{*(+)},\ldots, \Psi_N^{*(+)}\right) 
\in \Gamma^* \subset \Gamma\,. \label{e44}
\end{equation}
Из~(\ref{e44}) и~свойства верхней грани следует:
\begin{multline}
\sup\limits_{\left(\Psi_1^{*}, \Psi_2^{*},\ldots, \Psi_N^{*}\right) \in \Gamma^*} 
I\left(\Psi_1^{*}, \Psi_2^{*},\ldots, \Psi_N^{*}\right)\leq {}\\
{}\leq
\sup\limits_{\left(\Psi_1, \Psi_2,\ldots, \Psi_N\right) 
\in \Gamma} I\left(\Psi_1, \Psi_2,\ldots, \Psi_N\right)\,. 
\label{e45}
\end{multline}
Объединяя~(\ref{e42}), (\ref{e43}) и~(\ref{e45}), получаем соотношение:
\begin{multline}
\sup\limits_{\left(\Psi_1^{*}, \Psi_2^{*},\ldots, \Psi_N^{*}\right) 
\in \Gamma^*} I\left(\Psi_1^{*}, \Psi_2^{*},\ldots, 
\Psi_N^{*}\right)\leq{}\\
{}\leq \sup\limits_{\left(\Psi_1, \Psi_2,\ldots, \Psi_N\right) 
\in \Gamma} I\left(\Psi_1, \Psi_2,\ldots, \Psi_N\right)\leq{}\\
{}\leq I\left(\Psi_1^{*(+)}, \Psi_2^{*(+)},\ldots, \Psi_N^{*(+)}\right)={}\\
{}=
\fr{A\left(u^{(+)}_1,u^{(+)}_2,\ldots,u^{(+)}_N\right)}{B\left(u^{(+)}_1,u^{(+)}_2,
\ldots,u^{(+)}_N\right)}\,.
 \label{e46}
\end{multline}
Из соотношения~(\ref{e46}) с~учетом~(\ref{e44}) получаем, что максимум 
функционала $I(\Psi_1, \Psi_2,\ldots, \Psi_N)$ на множестве допустимых стратегий 
$\Psi\hm=(\Psi_1, \Psi_2,\ldots, \Psi_N)\hm\in \Gamma$ существует и~достигается 
на детерминированной стратегии $\left(\Psi_1^{*(+)}, \Psi_2^{*(+)},\ldots, 
\Psi_N^{*(+)}\right)$.

Кроме того, выполняются соотношения~(\ref{e19}). Таким образом, утверждение~1 
в~случае, когда основная функция $C(u_1, u_2,\ldots, u_N)$ достигает глобального 
максимума, доказано. Соответствующее утверждение в~случае, когда основная функция 
$C(u_1, u_2,\ldots, u_N)$ достигает глобального минимума, доказывается аналогично. 
При этом используется второе утверждение леммы~1.

\smallskip

Перейдем к~доказательству второго утверждения теоремы~2. Предположим, что основная 
функция $C(u_1, u_2,\ldots, u_N)\hm=A(u_1, u_2,\ldots$\linebreak
$\ldots, u_N)/{B(u_1, u_2,\ldots, u_N)}$ 
ограничена сверху, но не достигает глобального максимума на множестве 
$U \hm= U_1 \times U_2 \times \cdots \times U_N$. Тогда множество значений 
основной функции имеет конечную верхнюю грань:

\noindent
\begin{multline*}
C\left(u_1, u_2,\ldots, u_N\right)=\fr{A(u_1, u_2,\ldots, u_N)}
{B(u_1, u_2,\ldots, u_N)}<{}\\
{}<
\sup\limits_{(u_1, u_2,\ldots, u_N)\in U} \fr{A(u_1, u_2,\ldots, u_N)}
{B(u_1, u_2,\ldots, u_N)}<\infty\,, 
\\
\left(u_1, u_2,\ldots, u_N\right)\in U\,.
%\label{e47}
\end{multline*}
По определению верхней грани для любого фиксированного $\varepsilon \hm>0$ 
существует точка $(u_1^{(+)}(\varepsilon), u_2^{(+)}(\varepsilon),\ldots, 
u_N^{(+)}(\varepsilon))$ такая, что выполняется двойное неравенство~(\ref{e21}) 
(см.~\cite[гл.~1, \S\,3, п.~3.4]{25}). Иначе говоря, значение основной функции 
в~указанной точке лежит в~левой \mbox{$\varepsilon$-окрест}\-ности верхней грани. 
Рассмотрим детерминированную стратегию управления 
$\Psi^{*(+)}(\varepsilon)\hm=\!\left(\Psi_1^{*(+)}(\varepsilon), 
\Psi_2^{*(+)}(\varepsilon),\ldots, \Psi_N^{*(+)}(\varepsilon)\!\right)$, компонентами\linebreak 
которой являются вырожденные вероятностные меры $\Psi_1^{*(+)}(\varepsilon), 
\Psi_2^{*(+)}(\varepsilon),\ldots, \Psi_N^{*(+)}(\varepsilon)$, причем вырожденная 
мера~$\Psi_i^{*(+)}(\varepsilon)$ сосредоточена в~точке~$u_i^{(+)}(\varepsilon)$,
$i\hm=1,2,\ldots,N$.

По свойству интеграла
\begin{multline}
I\left(\Psi_1^{*(+)}(\varepsilon), \Psi_2^{*(+)}(\varepsilon),\ldots,
 \Psi_N^{*(+)}(\varepsilon)\right)={}\\
 {}=
 C\left(u_1^{(+)}(\varepsilon), u_2^{(+)}(\varepsilon),\ldots, 
 u_N^{(+)}(\varepsilon)\right)\,. 
 \label{e48}
\end{multline}
Из соотношения~(\ref{e48}) с~учетом указанного свойства основной функции получаем:
\begin{multline}
\sup\limits_{(u_1, u_2,\ldots, u_N)\in U} \fr{A(u_1, u_2,\ldots, u_N)}
{B(u_1, u_2,\ldots, u_N)}-\varepsilon<{}\\
{}< I\left(\Psi_1^{*(+)}(\varepsilon), 
\Psi_2^{*(+)}(\varepsilon),\ldots, \Psi_N^{*(+)}(\varepsilon)\right)<{}
\\
{}< \sup\limits_{(u_1, u_2,\ldots, u_N)\in U} \fr{A(u_1, u_2,\ldots, u_N)}
{B(u_1, u_2,\ldots, u_N)}<\infty\,. 
\label{e49}
\end{multline}
Заметим также, что в~рассматриваемом случае выполнены условия леммы~2. 
Воспользуемся первым утверждением этой леммы, а~именно соотношением~(\ref{e36}):
\begin{multline}
I(\Psi_1, \Psi_2,\ldots, \Psi_N)< {}\\
{}<\sup\limits_{(u_1, u_2,\ldots, u_N)
\in U} \fr{A(u_1, u_2,\ldots, u_N)}{B(u_1, u_2,\ldots, u_N)}<\infty 
\label{e50}
\end{multline}
для всех $(\Psi_1, \Psi_2,\ldots, \Psi_N)\in\Gamma$.

Из соотношений~(\ref{e49}) и~(\ref{e50}) следует, что детерминированная стратегия 
$\Psi^{*(+)}(\varepsilon)\hm=\left(\Psi_1^{*(+)}(\varepsilon), \Psi_2^{*(+)}(\varepsilon),
\ldots, \Psi_N^{*(+)}(\varepsilon)\right)$, опре\-де\-ля\-емая набором вырожденных 
вероятностных мер, сосредоточенных в~соответствующих точках 
$\left(u_1^{(+)}(\varepsilon), u_2^{(+)}(\varepsilon),\ldots, 
u_N^{(+)}(\varepsilon)\right)$, является $\varepsilon$-оп\-ти\-маль\-ной. 
Вторая часть утверждения~2 теоремы~2, связанная со свойствами нижней грани, 
доказывается аналогично.

Докажем третье утверждение теоремы~2. Предположим, что множество значений 
основной функции $C(u_1, u_2,\ldots, u_N)\hm=
A(u_1, u_2,\ldots$\linebreak $\ldots, u_N)/{B(u_1, u_2,\ldots, u_N)}$
не является ограниченным сверху на множестве $U\hm=U_1\times U_2 \times \cdots $\linebreak
$\cdots \times U_N$.
Тогда существует последовательность\linebreak точек $\left(u_1^{(+)}(n), u_2^{(+)}(n),
\ldots,u_N^{(+)}(n)\right)\hm\in U$, $n\hm=1,2,\ldots$, для которой
\begin{multline}
C\left(u_1^{(+)}(n), u_2^{(+)}(n),\ldots,u_N^{(+)}(n)\right)={}\\
{}=
\fr{A\left(u_1^{(+)}(n), u_2^{(+)}(n),\ldots,u_N^{(+)}(n)\right)}
{B\left(u_1^{(+)}(n), u_2^{(+)}(n),\ldots,u_N^{(+)}(n)\right)}
\longrightarrow \infty \,,\\
n\rightarrow \infty\,.
\label{e51}
\end{multline}
Зафиксируем некоторую последовательность точек $\left(u_1^{(+)}(n), u_2^{(+)}(n),
\ldots,u_N^{(+)}(n)\right)\hm\in U$, $n\hm=1,2,\ldots$, обладающих указанным свойством, 
и~рассмотрим последовательность детерминированных  стратегий управления 
$\Psi^{*(+)}(n)\hm=\left(\Psi_1^{*(+)}(n), \Psi_2^{*(+)}(n),\ldots, 
\Psi_N^{*(+)}(n)\right)$, $n\hm=1,2,\ldots$, определяемых набором вырожденных 
вероятностных мер, сосредоточенных в~соответствующих точках 
$\left(u_1^{(+)}(n), u_2^{(+)}(n),\ldots,u_N^{(+)}(n)\right)$, $n\hm=1,2,\ldots$ 
По свойству интеграла для любого фиксированного значения $n=1,2,\ldots$ 
выполняется равенство:
\begin{multline}
I \left(\Psi^{*(+)}(n)\right)={}\\
{}=I\left(\Psi_1^{*(+)}(n), \Psi_2^{*(+)}(n),\ldots,
 \Psi_N^{*(+)}(n)\right)={}\\
{}=\fr{A\left(u_1^{(+)}(n), u_2^{(+)}(n),\ldots,u_N^{(+)}(n)\right)}
{B\left(u_1^{(+)}(n), u_2^{(+)}(n),\ldots,u_N^{(+)}(n)\right)}\,. 
\label{e52}
\end{multline}
Из соотношений~(\ref{e51}) и~(\ref{e52}) следует, что
\begin{multline}
I\left(\Psi^{*(+)}(n)\right)={}\\
{}=I\left(\Psi_1^{*(+)}(n), \Psi_2^{*(+)}(n),\ldots, 
\Psi_N^{*(+)}(n)\right)\longrightarrow\infty\,,\\ 
n \rightarrow\infty\,.
 \label{e53}
\end{multline}
Соотношение~(\ref{e53}) означает, что множество значе\-ний дроб\-но-ли\-ней\-но\-го 
интегрального функциона\-ла $I(\Psi_1, \Psi_2,\ldots, \Psi_N)$ вида~(\ref{e11}) 
не ограничено сверху\linebreak на множестве наборов вырожденных вероятностных мер 
$\left(\Psi_1^{*(+)}(n), \Psi_2^{*(+)}(n),\ldots, \Psi_N^{*(+)}(n)\right)\hm\in\Gamma^*$, 
а~следовательно, и~на более широком\linebreak множестве наборов вероятностных 
мер $(\Psi_1, \Psi_2,\ldots$\linebreak $\ldots, \Psi_N)\hm\in\Gamma$. В~таком случае решения экстремальной 
задачи~(\ref{e18}) в~форме задачи на максимум не существует. Соответствующее утвержде\-ние 
для варианта, когда множество значений основной функции $C(u_1, u_2,\ldots,u_N)
\hm=A(u_1, u_2,\ldots$\linebreak $\ldots,u_N)/{B(u_1, u_2,\ldots,u_N)}$ 
не является ограниченным снизу, доказывается аналогично. Третье утверж\-де\-ние теоремы~2 
доказано. Тем самым тео\-ре\-ма~2 доказана полностью.

\smallskip

Применим теорему~2 для решения поставленной задачи оптимального управления. 
Из утверждения этой теоремы следует, что для доказательства су-\linebreak ществования 
оптимального управ\-ле\-ния и~его нахождения необходимо исследовать на 
глобальный экстремум основную функцию дроб\-но-ли\-ней\-но\-го интегрального 
функционала $C(u_1,u_2,\ldots,u_N)$, определяемую формулой~(\ref{e17}) с~учетом 
равенств~(\ref{e12})--(\ref{e16}). В~некоторых случаях, например когда основной 
процесс~$\xi(t)$ является регенерирующим, а~стоимостные характеристики 
модели задаются линейными функциями, такое исследование можно провести 
аналитически. Однако для подавляющего большинства полумарковских моделей 
для этого необходимо использовать численные методы.

\section{Заключение}

В заключительной части работы приведем \mbox{краткое} описание теоретической 
основы метода решения задачи оптимального управления полумарковским 
процессом с~конечным множеством состояний.

\begin{enumerate}[1.]
\item Исходная проблема оптимального управления формулируется в~виде 
экстремальной задачи~(\ref{e18}). Целевым показателем качества управ\-ле\-ния в~данной задаче 
служит величина~(\ref{e10}), которая имеет характер средней удельной прибыли.
\item Доказывается, что стационарный показатель~(\ref{e10}) представим в~виде 
дроб\-но-ли\-ней\-но\-го интегрального функционала~(\ref{e11}), для которого явно 
определяются подынтегральные функции числителя и~знаменателя, а~следовательно, 
и~основная функция данного функционала.
\item Используется теорема об экстремуме дроб\-но-ли\-ней\-но\-го интегрального 
функционала. На основании утверждений этой теоремы уста\-нав\-ли\-ва\-ет\-ся, что 
исходная задача оптимального управления сводится к~исследованию на глобальный 
экстремум основной функции этого функционала, для которой получено явное 
аналитическое представление.
\end{enumerate}

Заметим, что такое исследование задач оптимального управления 
стохастическими системами фактически уже было проведено в~ряде работ П.\,В.~Шнуркова 
и~его соавторов. В~частности, в~работе~\cite{26} была рассмотрена модель 
управления для обрывающегося процесса восстановления, описывающего функционирование 
некоторой технической системы. Задача управления решалась для различных показателей 
эффективности и~надежности этой системы, имеющих структуру дроб\-но-ли\-ней\-но\-го 
интегрального функционала.

В работах~\cite{27, 28} рассматривались модели регенерирующих процессов 
для исследования сис\-тем управления запасами. Различные показатели качества 
управления были представлены в~форме дроб\-но-ли\-ней\-ных интегральных функционалов. 
Основные функции этих функционалов были найде\-ны в~явной форме и~исследовались 
на глобальный экстремум. В~работах~\cite{21,29} рассматривалась достаточно 
сложная полумарковская модель с~конечным множеством состояний, описывающая 
сис\-те\-му управления запасом непрерывного продукта. Показатели качества управления в~этой 
модели также имели структуру дроб\-но-ли\-ней\-ных интегральных функционалов, 
для основных функций которых были найдены явные аналитические представления. 
Упомянем также работы~\cite{30, 31}, в~которых была исследована полумарковская 
модель с~дис\-крет\-но-не\-пре\-рыв\-ным фазовым пространством. Показатели 
качества управления в~этой  модели были найдены в~явной форме как функции от 
двух непрерывных параметров управления.

Фактически во всех упомянутых работах уже был использован метод решения задачи 
оптимального управления регенерирующим или полумарковским случайным процессом, 
основанный на исследовании экстремальных свойств основной функции соответствующего 
дроб\-но-ли\-ней\-но\-го интегрального функционала. Из соображений, изложенных 
во\linebreak введении, следует, что в~период написания и~пуб\-ли\-кации этих работ данный метод 
не имел стро\-гого обоснования. Однако после публикации\linebreak работы~\cite{14} и~настоящего 
исследования можно утверж\-дать, что полученные в~них результаты полностью теоретически 
обоснованы.

Таким образом, изложенный выше метод решения проблемы оптимального управления 
полумарковскими процессами с~конечными множествами состояний может быть успешно 
реализован для многих задач, рассматриваемых в~различных областях прикладной 
теории вероятностей.

Практическая реализация численной процедуры поиска оптимального решения на примере\linebreak 
полумарковской модели управления запасом непрерывного продукта (подробнее 
см.~\cite{21, 29}), ба\-зи\-ру\-юща\-яся на изложенных выше результатах (в~частности, 
теореме~1), была осуществлена А.\,К.~Горшениным и~соавторами 
в~статье~\cite{Gorshenin2015}. Коротко опишем наиболее важные аспекты этой работы.

Для решения поставленной задачи опти\-мального управления была создана 
специальная программа \verb"Inventory" на встроенном языке программирования 
пакета \verb"MATLAB", ее возможности\linebreak кратко представле\-ны в~упомянутой ранее 
\mbox{статье}~\cite{Gorshenin2015}. В~программе \verb"Inventory" реализованы функции 
для оценивания через заданные исходные параметры вероятностных и~стоимостных 
характеристик модели, которые в~дальнейшем используются для поиска значений 
основной функции дроб\-но-ли\-ней\-но\-го функционала~(\ref{e17}). Точка глобального 
экстремума этой функции и~определяет оптимальное управление.

В качестве начальных данных необходимо задание следующих параметров:
\begin{itemize}
\item спрос и~вместимость склада;
\item разбиение множества значений объема запаса;
\item вероятностные характеристики, описывающие модель пополнения запаса;
\item условные математические ожидания длительностей задержек пополнения запаса;
\item функции для характеризации затрат и~доходов.
\end{itemize}

По итогам работы программы \verb"Inventory" ряд вспомогательных функций 
представляется в~аналитической форме (в частности, с~использованием аппарата 
символьных вычислений  \verb"Symbolic Toolbox"\linebreak пакета \verb"MATLAB"), выводится 
точка глобального экстремума функции нескольких вещественных переменных~(\ref{e17}), 
найденная с~помощью применения численных и~при\-бли\-жен\-но-ана\-ли\-ти\-че\-ских\linebreak 
аппроксимаций. 
Также формируются графики оценок значений ве\-ро\-ят\-ност\-но-сто\-и\-мост\-ных 
характеристик 
и~основной функции дроб\-но-ли\-ней\-но\-го функционала~(\ref{e17}), либо трехмерных 
сечений в~случае наличия более трех параметров управления (переменных).

Функциональность пакета \verb"Inventory" может быть расширена для практической 
реализации метода решения задачи поиска оптимального управ\-ле\-ния полумарковскими 
процессами с~конечными множествами состояний, рассмотренного в~данной статье.


 {\small\frenchspacing
 {%\baselineskip=10.8pt
 \addcontentsline{toc}{section}{References}
 \begin{thebibliography}{99}
 \bibitem{1}
\Au{Ховард Р.} Динамическое программирование и~марковские процессы~/ 
Пер. с~англ.~--- М.: Сов. радио, 1964. 189~с.
(\Au{Howard~R.\,A.} Dynamic programming and Markov processes.~--- 
Cambridge, MA, USA: MIT Press, 1960. 136~p.)
\bibitem{2} 
\Au{Рыков В.\,В.} Управляемые марковские процессы с~конечными пространствами 
состояний и~управлений~// Теория вероятностей и~ее применения, 1966. Т.~11. 
Вып.~2. С.~343--351.
\bibitem{3} 
\Au{Джевелл В.} Управляемые полумарковские процессы~// Кибернетич. сборник.~--- 
М.: Мир, 1967. Вып.~4. С.~97--134.
%{\em Jewell W.\,S.} Markov-renewal programming~// Operations Research, 1963. Vol.~11. P.~938--971.
\bibitem{4} 
\Au{Fox B.} Markov renewal programming by linear fractional programming~// 
SIAM J.~Appl. Math., 1966. Vol.~14. P.~1418--1432.
\bibitem{5} 
\Au{Denardo E.\,V.} Contraction mappings in the theory underlying dinamic programming~// 
SIAM Rev., 1967. Vol.~9. P.~165--177.

\bibitem{6} 
\Au{Howard R.\,A.} Research in semi-Markovian decision structures~// 
J.~Oper. Res. Soc. Japan, 1963. Vol.~6. P.~163--199.
\bibitem{7} 
\Au{Osaki S., Mine H.} Linear programming algorithms for Markovian decision processes~//
 J.~Math. Anal. Appl., 1968. Vol.~22. P.~356--381.
\bibitem{8} 
\Au{Майн Х., Осаки С.} Марковские процессы принятия решений~/ Пер. с~англ.~--- 
М.: Наука, 1977. 176~с.
(\Au{Mine~H., Osaki~S.} 
Markovian decision processes.~--- New York, NY, USA: 
American Elsevier Publishing Co., 1970. 142~p.)
\bibitem{9} 
\Au{Гихман И.\,И., Скороход А.\,В.} Управляемые случайные процессы.~--- 
Киев: Наукова думка, 1977. 251~с.
\bibitem{10} 
\Au{Luque-Vasquez F., Herndndez-Lerma~О.} Semi-Markov control models with average costs~// 
Appl. Math., 1999. Vol.~26. No.\,3. P.~315--331.
\bibitem{11} 
\Au{Vega-Amaya O., Luque-Vasquez~F.} Sample-path average cost optimality for 
semi-Markov control processes on Borel spaces: Unbounded costs and mean holding times~// 
Appl. Math., 2000. Vol.~27. No.\,3. P.~343--367.
\bibitem{12} 
Вопросы математической теории надежности~/ Под ред. Б.\,В. Гнеденко.~--- 
М.: Радио и~связь, 1983. 376~с.
\bibitem{13} 
\Au{Барзилович Е.\,Ю., Каштанов~В.\,А.} Некоторые математические вопросы теории 
обслуживания сложных систем.~---  М.: Сов. радио, 1971. 272~с.
\bibitem{14} 
\Au{Шнурков П.\,В.} О~решении проблемы безусловного экстремума для 
дроб\-но-ли\-ней\-но\-го интегрального функционала на множестве вероятностных мер~// 
Докл. РАН. Сер. Математика, 2016. Т.~470. №\,4. C.~387--392.
\bibitem{15} 
\Au{Ширяев А.\,Н.}  Вероятность.~--- М.:~МЦНМО, 2011. Кн.~1. 552~с. Кн.~2. 968~с.
\bibitem{16} 
\Au{Боровков А.\,А.} Теория вероятностей.~--- М.: Либроком, 2009. 656~c.
\bibitem{17} 
\Au{Хеннекен П.\,Л., Тортра А.} Теория вероятностей 
и~некоторые ее приложения.~--- М.: Наука, 1974. 472~c.
\bibitem{18} 
\Au{Халмош П.} Теория меры~/ Пер. с~англ.~--- М.: ИЛ, 1953. 282~c.
(\Au{Halmos~P.} Measure theory.~--- Litton Educational Publishing, Inc. 1950. 304~p.)
\bibitem{19} 
\Au{Королюк В.\,С., Турбин~А.\,Ф.} Полумарковские процессы и~их приложения.~--- 
Киев:~Наукова думка, 1976. 184~с.
\bibitem{20} 
\Au{Janssen J., Manca R.} Applied semi-Markov processes.~--- New York,
NY, USA: Springer, 2006. 309~p.
\bibitem{21} 
\Au{Шнурков П.\,В., Иванов~А.\,В.} Анализ дискретной полумарковской модели
 управления запасом непрерывного продукта при периодическом прекращении потребления~// 
 Дискретная математика, 2014. Т.~26. Вып.~1. С.~143--154.
\bibitem{22} 
\Au{Иванов~А.\,В.} Анализ дискретной полумарковской модели
 управления запасом непрерывного продукта при периодическом прекращении 
 потребления.~--- М.: НИУ ВШЭ, 2014.  Дисс.\ \ldots\ канд. физ.-мат. наук. 120~с.
\bibitem{23}  %23
\Au{Bajalinov~E.\,B.} Linear-fractional programming. 
Theory, methods, applications and software.~--- 
Boston/\linebreak Dordrecht/London: Kluwer Academic Publs., 2003. 423~p.

\bibitem{27} %27
\Au{Шнурков П.\,В., Мельников~Р.\,В.} Оптимальное управление запасом 
непрерывного продукта в~модели регенерации~// Обозрение прикладной 
и~промышленной математики, 2006. Т.~13. Вып.~3. С.~434--452.
\bibitem{28} 
\Au{Шнурков П.\,В., Мельников~Р.\,В.} 
Исследование проб\-ле\-мы управления запасом непрерывного продукта при детерминированной 
задержке поставки~// Автоматика и~телемеханика, 2008. Т.~10. С.~93--113.


\bibitem{24}  %26
\Au{Шнурков П.\,В.} Методы исследования задач оптимального обслуживания 
в~математической теории надежности.~--- 
М.: МИЭМ, 1983.  Дисс.\ \ldots\ канд. физ.-мат. наук.

 \bibitem{25}  %25
\Au{Кудрявцев Л.\,Д.} Курс математического анализа. Т.~1.~--- 
М.: Дрофа, 2006. 704~с.

\bibitem{26} %24
\Au{Шнурков П.\,В.} Оптимальное обслуживание на периоде 
до первого отказа системы~// Применение аналитических методов в~вероятностных
 задачах.~--- Киев: Институт математики АН УССР, 1986. С.~121--129.

\bibitem{29} 
\Au{Шнурков П.\,В., Иванов~А.\,В.} Исследование задачи оптимизации в~дискретной 
полумарковской модели управления непрерывным запасом~// Вестник МГТУ им.\ 
Н.\,Э. Баумана. Сер.\ Естественные науки, 2013. Т.~3. Вып.~50. С.~62--87.
\bibitem{30} 
\Au{Shnourkoff P.\,V.} The two-element system with one 
restoring device optimum maintenance~// Stoch. Anal. Appl., 1997. 
Vol.~15. No.\,5. P.~823--837.
\bibitem{31} 
\Au{Shnourkoff P.\,V.} The two-element system optimum maintenance tills the first fail~// 
Stoch. Anal. Appl., 2001. Vol.~19. No.\,6. P.~1005--1024.
\bibitem{Gorshenin2015} 
\Au{Gorshenin~A.\,K., Belousov~V.\,V., Shnourkoff~P.\,V.,
Ivanov~A.\,V.} Numerical research of the optimal control problem in the semi-Markov 
inventory model~// AIP Conference Proceedings, 2015. Vol.~1648. {250007}. 4~p.
%\bibitem{33} {\em Горшенин А.\,К., Белоусов В.\,В., Шнурков П.\,В.} 2016. Система управления запасами на основе стохастических полумарковских моделей. Свидетельство о государственной регистрации программы для ЭВМ \textnumero 2016619021.
 \end{thebibliography}

 }
 }

\end{multicols}

\vspace*{-6pt}

\hfill{\small\textit{Поступила в~редакцию 15.07.16}}

%\vspace*{8pt}

\newpage

\vspace*{-24pt}

%\hrule

%\vspace*{2pt}

%\hrule

%\vspace*{8pt}


\def\tit{ANALYTICAL SOLUTION OF~THE~OPTIMAL CONTROL TASK OF~A~SEMI-MARKOV 
PROCESS WITH~FINITE SET OF~STATES}

\def\titkol{Analytical solution of~the~optimal control task of~a~semi-Markov 
process with~finite set of~states}

\def\aut{P.\,V.~Shnurkov$^{1}$, A.\,K.~Gorshenin$^{2}$, and~V.\,V.~Belousov$^{2}$}

\def\autkol{P.\,V.~Shnurkov, A.\,K.~Gorshenin, and~V.\,V.~Belousov}

\titel{\tit}{\aut}{\autkol}{\titkol}

\vspace*{-9pt}


    
\noindent
$^1$National Research University Higher School of Economics, 34~Tallinskaya Str., 
Moscow, 123458, Russian\linebreak
$\hphantom{^9}$Federation

\noindent
$^2$Institute of Informatics Problems, Federal Research Center 
``Computer Science and Control'' of the Russian\linebreak
$\hphantom{^9}$Academy of Sciences, 44-2~Vavilova Str., 
Moscow 119333, Russian Federation



\def\leftfootline{\small{\textbf{\thepage}
\hfill INFORMATIKA I EE PRIMENENIYA~--- INFORMATICS AND
APPLICATIONS\ \ \ 2016\ \ \ volume~10\ \ \ issue\ 4}
}%
 \def\rightfootline{\small{INFORMATIKA I EE PRIMENENIYA~---
INFORMATICS AND APPLICATIONS\ \ \ 2016\ \ \ volume~10\ \ \ issue\ 4
\hfill \textbf{\thepage}}}

\vspace*{3pt}


\Abste{The theoretical verification of the new method of finding 
the optimal strategy of control of a~semi-Markov process with finite set of states is 
presented. The paper considers Markov randomized strategies of control, determined by 
a~finite collection of probability measures, corresponding to each state. The quality 
characteristic is the stationary cost index. This index is a~linear-fractional integral 
functional, depending on collection of probability measures, giving the strategy of control. 
Explicit analytical forms of integrands of numerator and denominator of this 
linear-fractional integral functional are known. The basis of consequent results is 
the new generalized and strengthened form of the theorem about an extremum of 
a~linear-fractional integral functional. It is proved that problems of existence 
of an optimal control strategy of a~semi-Markov process and finding this strategy 
can be reduced to the task of numerical analysis of global extremum for 
the given function, depending on finite number of real arguments.}

\KWE{optimal control of a~semi-Markov process; stationary cost index of quality control; 
linear-fractional integral functional}




\DOI{10.14357/19922264160408} 

\vspace*{-16pt}

\Ack
\noindent
The research was partially supported by the Russian Foundation 
for Basic Research (project 15-07-05316).



%\vspace*{3pt}

  \begin{multicols}{2}

\renewcommand{\bibname}{\protect\rmfamily References}
%\renewcommand{\bibname}{\large\protect\rm References}

{\small\frenchspacing
 {%\baselineskip=10.8pt
 \addcontentsline{toc}{section}{References}
 \begin{thebibliography}{99}
\bibitem{1-1}
\Aue{Howard,~R.\,A.} 1960. \textit{Dynamic programming and Markov processes}. 
Cambridge, MA: MIT Press. 136~p.
\bibitem{2-1}
\Aue{Rykov,~V.\,V.} 1966. Upravlyaemye markovskie protsessy 
s~konechnymi prostranstvami sostoyaniy i~upravleniy 
[Controlled Markov processes with finite spaces of states and controls ]. 
\textit{Teoriya veroyatnostey i~ee primeneniya} 
[Theory of Probability and Its Applications] 11(2):343--351.
\bibitem{3-1}
\Aue{Jewell,~W.\,S.} 1963. Markov-renewal programming. 
\textit{Oper. Res.} 11:938--971.
\bibitem{4-1}
\Aue{Fox,~B.} 1966. Markov renewal programming by linear fractional programming. 
\textit{SIAM J.~Appl. Math.} 14:1418--1432.
\bibitem{5-1}
\Aue{Denardo, E.\,V.} 1967. Contraction mappings in the theory underlying dinamic 
programming. \textit{SIAM Rev.} 9:165--177.
\bibitem{6-1}
\Aue{Howard,~R.\,A.} 1963. Research in semi-Markovian decision structures. 
\textit{J.~Oper. Res. Soc. Japan} 6:163--199.
\bibitem{7-1}
\Aue{Osaki,~S., and H.~Mine.} 1968. Linear programming algorithms 
for Markovian decision processes. \textit{J.~Math. Anal. Appl.} 22:356--381.
\bibitem{8-1}
\Aue{Mine,~H., and S.~Osaki.} 1970. 
\textit{Markovian decision processes}. New York, NY: Elsevier. 142~p.
\bibitem{9-1}
\Aue{Gikhman,~I.\,I., and A.\,V.~Skorokhod.} 1977. 
\textit{Upravlyaemye sluchaynye protsessy} 
[Controlled random processes]. Kiev: Naukova Dumka. 251~p.
\bibitem{10-1}
\Aue{Luque-Vasquez,~F., and О.~Herndndez-Lerma.} 1999. 
Semi-Markov control models with average costs. \textit{Appl. Math.} 26(3):315--331.
\bibitem{11-1}
\Aue{Vega-Amaya,~O., and  F.~Luque-Vasquez.} 2000.  
Sample-path average cost optimality for semi-Markov control processes on Borel spaces: 
Unbounded costs and mean holding times. \textit{Appl. Math.} 27(3):343--367.
\bibitem{12-1}
Gnedenko,~B.~V., ed. 1983. 
\textit{Voprosy matematicheskoy teorii nadezhnosti} 
[Problems of the mathematical theory of reliability].  Moscow: Radio i~svyaz'. 376~p.
\bibitem{13-1}
\Aue{Barzilovich,~E.\,Yu., and V.\,A.~Kashtanov.} 1971. 
\textit{Nekotorye matematicheskie voprosy teorii obsluzhivaniya slozhnykh sistem}  
[Some mathematical questions in theory of complex systems maintenance]. 
Moscow: Sovetskoe radio. 272~p.
\bibitem{14-1}
\Aue{Shnurkov,~P.\,V.} 2016. Solution of the unconditional extremum problem for 
a~linear-fractional 
integral functional on a~set of probability measures. 
\textit{Dokl. Math.} 94(2):550--554.
\bibitem{15-1} %14
\Aue{Shiryaev,~A.\,N.} 2016. 
\textit{Probability-1}. Graduate texts in mathematics ser.
New York, NY: Springer. Vol.~95. 503~p.;
2017. \textit{Probability-2.} Vol.~900. 500~p.
\bibitem{16-1}
\Aue{Borovkov,~А.\,А.} 2009. 
\textit{Teoriya veroyatnostey} [Probability theory]. Moscow: Librokom. 656~p.
\bibitem{17-1}
\Aue{Khenneken,~P.\,L., and A.~Tortra.} 1974. 
\textit{Teoriya veroyatnostey i~nekotorye ee prilozheniya} 
[Probability theory and some of its applications]. Moscow: Nauka. 472~p.
\bibitem{18-1}
\Aue{Halmos,~P.} 1950. \textit{Measure theory}. Litton Educational Publishing. 304~p.
\bibitem{19-1}
\Aue{Korolyuk, V.\,S., and A.\,F.~Turbin.} 1976. 
\textit{Polumarkovskie protsessy i~ikh prilozheniya} 
[Semi-Markov processes and their applications]. Kiev: Naukova Dumka. 184~p.
\bibitem{20-1}
\Aue{Janssen,~J., and R.~Manca.} 2006. 
\textit{Applied semi-Markov processes}. New York, NY: Springer. 309~p.
\bibitem{21-1}
\Aue{Shnurkov,~P.\,V, and A.\,V~Ivanov.} 2015. Analysis of a~discrete semi-Markov model of continuous inventory 
control with periodic interruptions of consumption. 
\textit{Discrete Math. \mbox{Appl}.} 25(1):59--67.
\bibitem{22-1} %21
\Aue{Ivanov,~A.\,V.} 2014. Analiz diskretnoy polumarkovskoy modeli upravleniya 
zapasom nepreryvnogo produkta pri periodicheskom prekrashchenii potrebleniya 
[Analysis of a~discrete semi-Markov control model of continuous product inventory 
in a~periodic cessation of consumption].  
Moscow: Natsional'nyy Issledovatel'skiy Universitet ``Vysshaya Shkola Ekonomiki.'' 
PhD Thesis. 120~p.
\bibitem{23-1} %22
\Aue{Bajalinov,~E.\,B.} 2003. 
\textit{Linear-fractional programming. Theory, methods, applications and software}. 
Boston/\linebreak Dordrecht/London: Kluwer Academic Publs. 423~p.
\bibitem{26-1} %24
\Aue{Shnurkov,~P.\,V., and R.\,V.~Mel'nikov.} 2006. Optimal'noe upravlenie 
zapasom nepreryvnogo produkta v modeli regeneratsii [Optimal control of 
a~continuous product inventory in the regeneration model]. 
\textit{Obozrenie prikladnoy i~promyshlennoy matematiki} [Rev. Appl. Ind. Math.]
13(3):434--452.

\bibitem{25-1} %25
\Aue{Shnurkov,~P.\,V., and R.\,V.~Mel'nikov.} 2008. 
Analysis of the problem of continuous-product inventory control under deterministic 
lead time. \textit{Automat. Rem. Contr.} 69(10):1734--1751.

\columnbreak

\bibitem{24-1} %26
\Aue{Shnurkov,~P.\,V.} 1983. Metody issledovaniya zadach optimal'nogo obsluzhivaniya 
v~matematicheskoy teorii nadezhnosti [Research methods of optimal service problems 
in the mathematical theory of reliability].  
Moscow: Moskovskiy Institut Elektronnogo Mashinostroeniya.  PhD Thesis. 


\bibitem{27-1} %27
\Aue{Kudryavtsev,~L.\,D.} 2006. 
\textit{Kurs matematicheskogo analiza} 
[A~course of mathematical analysis]. Vol.~1. Moscow: Drofa. 704~p.

\bibitem{28-1}
\Aue{Shnurkov,~P.\,V.} 1986. Optimal'noe obsluzhivanie na periode do 
pervogo otkaza sistemy [The optimum service period until the first system failure]. 
\textit{Primenenie analiticheskikh metodov v~veroyatnostnykh zadachakh} 
[The application of analytical methods in probabilistic tasks]. Kiev:
Institute of Mathematics of the Academy of Sciences of the USSR. 121--129.

\bibitem{29-1}
\Aue{Shnurkov,~P.\,V., and A.\,V.~Ivanov.} 2013. Issledovanie zadachi optimizatsii 
v~diskretnoy polumarkovskoy modeli upravleniya nepreryvnym zapasom 
[Study of the optimization problem in discrete semi-Markov model of continuous 
inventory control]. \textit{Vestnik MGTU im.\ N.\,E.~Baumana. Ser. 
Estestvennye nauki} [Vestnik of MSTU named after N.\,E.~Bauman. Ser. Natural sciences] 
3(50):62--87.
\bibitem{30-1}
\Aue{Shnourkoff,~P.\,V.} 1997. The two-element system with one restoring device 
optimum maintenance.  \textit{Stoch. Anal. Appl.} 15(5):823--837.
\bibitem{31-1}
\Aue{Shnourkoff,~P.\,V.} 2001. The two-element system optimum maintenance tills 
the first fail. \textit{Stoch. Anal. Appl.} 19(6):1005--1024.
\bibitem{32-1}
\Aue{Gorshenin,~A.\,K., V.\,V.~Belousov, P.\,V.~Shnourkoff, and A.\,V.~Ivanov.}
2015. Numerical research of the optimal control problem in the semi-Markov 
inventory model. \textit{AIP Conference Proceedings} 1648:250007.
\end{thebibliography}

 }
 }

\end{multicols}

\vspace*{-3pt}

\hfill{\small\textit{Received July 15, 2016}}

\Contr

\noindent
\textbf{Shnurkov Peter V.} (b.\ 1953)~---
 Candidate of Science (PhD) in physics and mathematics, 
 associate professor, National Research University Higher School of Economics, 
 34~Tallinskaya Str., Moscow 123458, Russian Federation; \mbox{pshnurkov@hse.ru} 
 
 \vspace*{3pt}
 
 \noindent
\textbf{Gorshenin Andrey K.}  (b.\ 1986)~---
Candidate of Science (PhD) in physics and mathematics, leading scientist, 
Institute of Informatics Problems, Federal Research Center ``Computer Science 
and Control'' of the Russian Academy of Sciences, 44-2~Vavilov Str., Moscow 119333, 
Russian Federation; associate professor, Federal State Budget Educational 
Institution of Higher Education ``Moscow Technological University,'' 
78~Vernadskogo Ave., Moscow 119454, Russian Federation;
\mbox{agorshenin@frccsc.ru}

\vspace*{3pt}

\noindent
\textbf{Belousov Vasiliy V.} (b.\ 1977)~---
Candidate of Science (PhD) in technology, senior scientist, Institute of 
Informatics Problems, Federal Research Center ``Computer Science and Control'' 
of the Russian Academy of Sciences, 44-2~Vavilov Str., Moscow 119333, Russian 
Federation; \mbox{VBelousov@ipiran.ru}
\label{end\stat}


\renewcommand{\bibname}{\protect\rm Литература}  %8
\def\stat{strijov}

\def\tit{ВОССТАНОВЛЕНИЕ МАТРИЦЫ СУПЕРПОЗИЦИИ В~ЗАДАЧЕ~СИМВОЛЬНОЙ РЕГРЕССИИ$^*$}

\def\titkol{Восстановление матрицы суперпозиции в~задаче символьной регрессии}

\def\aut{Р.\,Г.~Нейчев$^1$, И.\,А.~Шибаев$^2$, В.\,В.~Стрижов$^3$}

\def\autkol{Р.\,Г.~Нейчев, И.\,А.~Шибаев, В.\,В.~Стрижов}

\titel{\tit}{\aut}{\autkol}{\titkol}

\index{Нейчев Р.\,Г.}
\index{Шибаев И.\,А.}
\index{Стрижов В.\,В.}
\index{Neychev R.\,G.}
\index{Shibaev I.\,A.}
\index{Strijov V.\,V.}


{\renewcommand{\thefootnote}{\fnsymbol{footnote}} \footnotetext[1]
{Работа выполнена при поддержке РФФИ (проекты 20-37-90050 и~20-07-00990).}}


\renewcommand{\thefootnote}{\arabic{footnote}}
\footnotetext[1]{Московский физико-технический институт, 
\mbox{neychevr@gmail.com}}
\footnotetext[2]{Московский физико-технический институт, 
\mbox{shibaev.kesha@gmail.com}}
\footnotetext[3]{Федеральный исследовательский центр <<Информатика 
и~управ\-ле\-ние>> Российской академии наук, \mbox{strijov@phystech.edu}}

\vspace*{-12pt}
 



\Abst{Исследуется проблема порождения структуры регрессионной модели. 
Модель представляет собой суперпозицию базовых функций. Структура модели 
описывается взвешенным цвет\-ным графом. Каждая вершина графа соответствует 
некоторой базовой функции. Ребро задает суперпозицию двух функций. Вес ребра 
равен вероятности суперпозиции. Для создания оптимальной модели необходимо 
восстановить ее структуру по матрице смежности графа. Пред\-ла\-га\-емый алгоритм 
восстанавливает минимальное остовное дерево из взвешенного цветного графа. 
Пред\-став\-ле\-но новое решение, основанное на алгоритме дерева Штейнера. 
Алгоритм сравнивается с~альтернативами.}


\KW{символьная регрессия; линейное программирование; 
суперпозиция; минимальное остовное дерево; мат\-ри\-ца смеж\-ности}

\DOI{10.14357/19922264230105} 
  
\vspace*{-8pt}


\vskip 10pt plus 9pt minus 6pt

\thispagestyle{headings}

\begin{multicols}{2}

\label{st\stat}

\section{Введение}

Символьная регрессия~--- это метод по\-стро\-ения нелинейной модели, 
аппроксимирующей выборку. Структура модели определяется суперпозицией базовых 
функций. Набор базовых функций фиксируется для конкретной прикладной задачи. 
Структуры альтернативных моделей генерируются алгоритмом оптимизации для выбора 
оптимальной модели. В данной статье предлагается определять структуру модели 
с~по\-мощью вероятностного графа. Остовное дерево в~графе определяет некоторую 
суперпозицию. Для выбора оптимальной модели необходимо реконструировать 
минимальное остовное дерево по графу.

Методы генетического программирования~\cite{koza1992genetic} находят оптимальное 
подмножество в~наборе суперпозиций базовых функций, но имеют высокую 
вычислительную сложность. В~\cite{searson2010gptips} описаны методы, понижающие 
сложность. Они используют дополнительные ограничения на суперпозиции, например 
используют линейные комбинации базовых функций. Символьная регрессия, 
описанная~в~\cite{stanley2002evolving}, используется для оптимизации структуры 
суперпозиции. Методы решения задачи символьной регрессии основаны на матричном 
представлении структуры модели~\cite{bochkarev2017generation}. Однако эти методы 
не содержат ограничений на чис\-ло аргументов базовых функций и~на структуру 
графа, обеспечивающую допустимую суперпозицию. В~данной работе решается задача 
построения модели с~помощью символьной регрессии.

Требуется восстановить допустимую суперпозицию из предсказанной мат\-ри\-цы 
смежности с~вероятностями ребер. Решается задача вос\-ста\-нов\-ле\-ния~$k$-минимального 
остовного дерева $k$-MST (\textit{англ.}\ Minimum-cost Spanning Tree). Эта задача NP-слож\-ная, 
поэтому применимы только при\-бли\-жен\-ные решения~\cite{ravi1996spanning}. 
Алгоритм~$k$-MST эквивалентен проб\-ле\-ме дерева Штейнера PCST (\textit{англ.}\ 
Prize-Collecting Steiner Tree) из-за его эквивалентности ослабленной формулировке 
постановки задачи линейного программирования~\cite{chudak2004approximate}. 
В~работах~\cite{ravi1996spanning,awerbuch1998new,arora20062+} пред\-став\-ле\-ны 
приближенные решения задачи \mbox{$k$-MST}.



Предлагаемое решение основано на упрощенной версии задачи~$k$-MST, которая 
трансформируется в~задачу PCST с~постоянными призами, одинаковыми для всех 
вершин. Быст\-рый алгоритм PSCT описан в~\cite{hegde2014fast}. Альтернативное 
решение основано на алгоритме~$(2-\varepsilon)$-аппроксимации для задачи PSCT. 
Она сравнивается с~другими алгоритмами, включая алгоритмы обхода дерева в~глубину, обхода дерева в~ширину, алгоритмы Прима.

\begin{table*}[b]\small  %tabl1
\vspace*{-12pt}
\begin{center}
        \parbox{262pt}{\Caption{Вероятности суперпозиций в~матрице смежности порождают 
ориентированный граф}

}
    \label{restored_adjacency_matrix}
\vspace*{2ex}

        \begin{tabular}{|c|c|ccccccc|}
            \hline
            Арность&Функция&$\ast$&$+$&$\ln$&$\sin$&$\times$&$\exp$&$x$\\
            \hline
            $1$&$\ast$ &0,2&{\bf 0,7}&0,5&0,4&0,5&0,3&0,2\\
            $3$&$+$    &0,3&0,2&{\bf 1,0}&{\bf 0,8}&0,6&0,3&{\bf 0,7}\\
            $1$&$\ln$  &0,3&0,2&0,0&0,0&0,1&0,5&{\bf 0,5}\\
            $1$&$\sin$ &0,1&0,4&0,0&0,5&{\bf 0,9}&0,2&0,5\\
            $2$&$\times$&0,3&0,0&0,3&0,5&0,0&{\bf 0,8}&{\bf 0,6}\\
            $1$&$\exp$ &0,3&0,3&0,4&0,1&0,5&0,4&{\bf 0,4}\\
            \hline
        \end{tabular}
\end{center}
\end{table*}

\vspace*{-12pt}


\section{Задача выбора регрессионной модели}

\vspace*{-3pt}

Требуется выбрать регрессионную модель~$\varphi$ из набора альтернативных 
моделей. Модель описывает выборку~$D=\{(x_i,y_i)\}$ и~минимизирует ошибку

\noindent
\begin{equation}
\hat{\varphi}(D)=\mathop{\argmin}\limits_\varphi\sum\limits_{i=1}^m\left(\varphi(x_i)-
y_i\right)^2.
\label{task_1}
\end{equation}
Модель представляет собой суперпозицию базовых функций из некоторого заданного 
набора. На рис.~1\linebreak\vspace*{-12pt}

{ \begin{center}  %fig1
 \vspace*{-3pt}
    \mbox{%
\epsfxsize=37.447mm
\epsfbox{str-1.eps}
}

\end{center}

\vspace*{-2pt}

\noindent
{{\figurename~1}\ \ \small{Структура регрессионной модели представляет собой ориентированный 
граф
}}}

\vspace*{6pt}

\addtocounter{figure}{1}


\noindent
 показан ее пример. Структура модели~$\varphi$, 
суперпозиция, соответствует графу~$G=(V,E)$, где базовые функции находятся 
в~вершинах~$V$. {Корневая} вершина обозначается через~$\ast$. Модель:

\vspace*{1pt}

\noindent
$$
\varphi(D) =  \ln(x) + x + \sin\left( x\times \exp(x)\right).
$$

\vspace*{-4pt}

\noindent
 Еe структура в~виде матрицы 
смежности графа пред\-став\-ле\-на~в табл.~\ref{restored_adjacency_matrix}.
Базовые функции перечислены в~первой строке. Элементами матрицы являются 
вероятности ребер~$E$ дерева. Жир\-ным шриф\-том выделены ребра восстановленного 
дерева~$M$, образующие суперпозицию~$\varphi$. Для восстановления структуры 
модели~$\varphi$ как суперпозиции, заданной деревом~$M$, необходимы только 
графовое пред\-став\-ле\-ние~$G$~и~базовые функции.



Поставим задачу восстановления структуры модели. Задано множество 
выборок~$\{D_k\}$. Каждой выборке~$D_k$ соответствует своя модель. Эта модель 
имеет структуру~$M_k$. Таким образом, имеется набор пар $\{(D_k, M_k)\}$, 
выборка и~структура.
Обозначим через~$P$ отображение, которое предсказывает вероятности узлов 
в~графе~$G$ по выборке~$D$. Для выбора модели~$\varphi(D)$ необходимо восстановить 
структуру модели~$M$ по графу~$G$. Обозначим алгоритм восстановления дерева 
через~$R$. Регрессионная модель~$\hat{\varphi}(D)$, которая решает 
задачу~(\ref{task_1}), определяется формулой
$
\hat{M}=R\left(P(D)\right).
$
Поскольку дерево~$M$ играет центральную роль в~этой работе, критерий качества 
алгоритма восстановления дерева имеет вид:


\vspace*{-3pt}

\noindent
$$
\min_{M_k \in G} \fr{1}{K}\sum\limits_{k=1}^K \left[ \hat{M_k} = M_k\right].
$$

\vspace*{-4pt}

\noindent
Восстановленное дерево должно быть эквивалентно заданному дереву, следовательно, 
выбранная модель регрессии при\-бли\-жа\-ет выборку.

\vspace*{-10pt}

\section{Задача восстановления дерева суперпозиции}

\vspace*{-3pt}

Требуется восстановить дерево~$M_k$, задающее  суперпозицию и~решающее 
задачу~(\ref{task_1}). Задан ориентированный взвешенный граф~$G\hm=(V,E)$ 
с~раскрашенными вершинами~$v_i$ и~корневой вершиной~$r$. Каждая вершина~$v_i \hm\in 
V$ имеет свой цвет~$t(v_i)\hm=t_i$. Каждое реб\-ро~$e_i\in E$ имеет свой 
вес~\mbox{$w(e_i)\hm=c_i\hm\in[0,1]$}.

Требуется восстановить ориентированное дерево минимального веса с~корнем~$r$. 
Оно должно покрывать не менее~$k$ вершин в~заданном графе~$G$. Чис\-ло ребер, 
выходящих из вершины~$v_i$ дерева, должно быть меньше или равно~$t_i$. 
Корень~$r$ имеет одно ребро,~$t_r=1$.

Сформулируем это условие в~виде задачи линейного программирования 
с~целочисленными ограничениями:

\vspace*{-5pt}

\noindent
\begin{multline}
\underset{\substack{{x_e, z_S} \\ e\in E,\\ S\subseteq V\backslash 
\{r\}}}{\mbox{minimize}}  \displaystyle \sum\limits_{e\in E}c_ex_e \\[-3pt]
\mbox{s.t.}\  \displaystyle  \sum\limits_{\substack{{e\in\delta(S):}\\ e=(\ast,v_i),\\ v_i\in\delta(S)}} \!\!\!\! x_e + 
\sum\limits_{T:T\supseteq S}  \!\!\!\! z_T\geqslant 1,\enskip  S\subseteq 
V\backslash \{r\};\\[-3pt]
 \displaystyle \sum\limits_{e\in E:~e=(\ast,v)} \! x_e\leqslant 1,\enskip v\in V;\\[-3pt]
 \displaystyle \sum\limits_{e\in E:~e=(v,\ast)}x_e\leqslant t_i,\enskip  v\in V;\\[-3pt]
 \displaystyle \sum\limits_{S\subseteq V\backslash \{r\}}|S|z_S \leqslant n-k,\enskip  x_e\in\{0,1\},\enskip 
 z_S\in\{0,1\},\\[-3pt]
  e\in E,\enskip   S\subseteq V\backslash \{r\},
\label{ilp_our}
\end{multline}
где
$$
x_e =\begin{cases}
 1, &\mbox{если\ ребро}\ e\ \mbox{входит\ в~финальную}\\
 &\mbox{суперпозицию};\\
 0 & \mbox{в~противном\ случае};
 \end{cases}
 $$
  $z_S\hm = 1$ для всех вершин, исключенных из финальной 
суперпозиции. Обозначим через~$e\hm=(\ast, v)$ ориентированное ребро с~листом~$v$. 
Обозначим через $e\hm=(v, \ast)$ ориентированное ребро с~вершиной~$v$.

Первое ограничение~(\ref{ilp_our})  определяет структуру графа решения в~виде 
дерева с~корнем~$r$. Второе ограничение определяет ориентацию дерева: каждая 
вершина имеет не более одного входящего ребра. Третье ограничение определяет 
арность используемых базовых функций, поэтому число ребер, имеющих определенную 
вершину в~качестве источника, фиксировано. Четвертое ограничение говорит, что 
итоговое дерево имеет не менее~$k$ вершин. Если все веса неотрицательны, то 
четвертое ограничение на минимальное число вершин принимает более строгий вид: 
число вершин должно быть равно~$k$. Однако более слабое ограничение позволяет 
найти возможные связи с~другими оптимизационными задачами. Ограничения 
в~(\ref{ilp_our}) преследуют ту же цель.

\vspace*{-9pt}

\section{Алгоритмы восстановления дерева $k$-MST и~PCST}

\vspace*{-3pt}

\noindent
\textbf{Определение~1} (\textbf{$\bm{k}$-минимальное остовное дерево,\linebreak $\bm{k}$-MST}).
Задан взвешенный граф~$G\hm=(V,E)$ с~корнем~$r$ и~весами ребер~$w(e_i)\hm=c_i\hm\geqslant 
0$, $e_i\hm\in E$. Требуется построить ориентированное дерево минимального веса 
с~корнем~$r$, покрывающее не менее~$k$ вершин в~$G$.

\smallskip

Если та же задача ставится для ориентированных графов, то конечное дерево 
с~корнем~$r$ должно быть ориентированным. Задача линейного программирования для 
направленного~$k$-MST исключает \mbox{третье} условие в~(\ref{ilp_our}).
В~таком виде задача~$k$-MST отличается от исходной задачи восстановления 
дерева суперпозиций~(\ref{ilp_our}) отсутствием третьего ограничения на арность 
базовых функций. Это эквивалентно ограничению на число ребер, выходящих из 
вершины.

\smallskip

\noindent
\textbf{Определение~2} (\textbf{призовое дерево Штейнера, $\text{PCST}$}).\linebreak
Задан взвешенный граф $G\hm=(V,E)$ с~корнем~$r$ и~весами ребер~$w(e_i)\hm=c_i\hm\geqslant  0$, $e_i\hm\in E$, где каждой вершине~$v_i \hm\in V$ присвоен 
{приз} $\pi(v_i)\hm=\pi_i\geqslant 0$. Требуется построить дерево~$T$ с~корнем~$r$, 
которое \mbox{минимизирует} функционал
$\sum\nolimits_{e\in E}c_ex_e \hm+ \sum\nolimits_{S\subseteq V\backslash\{r\}} 
\pi(S)z_S,$
где~$x_e\in\{0, 1\}$, $x_e\hm=1$, если~$e\hm\in E$ входит в~тройку~$T$; $z_S\hm\in\{0, 1\}$, 
$z_S\hm=1$ для всех вершин, исключенных из дерева~$T$; $S \hm= V\backslash V(T)$; $\pi(S)\hm= \sum\nolimits_{v\in S}\pi(v)$.

\smallskip

В случае ориентированных графов эта задача обобщается до~асимметричной задачи 
A-PCST. Задача линейного программирования для~A-PCST принимает вид:

\vspace*{-4pt}

\noindent
\begin{multline}
\underset{\substack{x_e,z_S \\ e\in E,\\ S\subseteq V\backslash \{r\}}}{\mbox{minimize}} 
\displaystyle \sum\limits_{e\in E} c_e x_e + \sum\limits_{S\subseteq V\backslash\{r\}}  \!\!\!\!\!\pi(S)z_S \\
\mbox{s.t.}\ \displaystyle \sum\limits_{\substack{e\in\delta(S):\\e=(\ast,v_i),\\ v_i\in\delta(S)}} \!\!\!\!\!\! x_e + 
\sum\limits_{T:T\supseteq S}  \!\!\! z_T\geqslant 1,\enskip  S\subseteq  V\backslash \{r\};\\
\displaystyle \sum\limits_{e\in E:~e=(\ast,v)}\!\!\!\!  x_e\leqslant 1,\enskip
x_e\in\{0,1\},\enskip z_S\in\{0,1\},\enskip  v\in V,\\
e\in E,\enskip S\subseteq V\backslash \{r\}.
\label{ilp_pcst_ord}
\end{multline}

\vspace*{-3pt}

\noindent
Если последнее ограничение из~(\ref{ilp_our}) входит в~оптимизируемый 
функционал, задачи $k$-MST и~A-PCST имеют эквивалентные 
ограничения и~отличаются только оптимизируемым функционалом. Такое 
преобразование возможно согласно условиям Ка\-ру\-ша--Ку\-на--Так\-ке\-ра~\cite{ras2017approximate}. Если значения призов 
эквивалентны $\pi(v) \hm=  \lambda$, единственное отличие состоит в~постоянном члене~$\lambda(n\hm-k)$. Таким 
образом, задачи оптимизации~$k$-MST и~A-PCST принимают вид:

\vspace*{-4pt}

\noindent
\begin{align*}
\underset{\substack{x_e,z_S \\ e\in E,\\ S\subseteq V\backslash \{r\}}}{\mbox{minimize}} & 
\sum\limits_{e\in E}c_ex_e + \lambda\left(\sum\limits_{S\subseteq V\backslash \{r\}}|S|z_S - (n-k)\right);\\ 
\underset{\substack{x_e,z_S \\ e\in E,\\ S\subseteq V\backslash \{r\}}}{\mbox{minimize}} & 
\sum\limits_{e\in E}c_ex_e + \lambda\sum\limits_{S\subseteq V\backslash\{r\}}|S|z_S\,. 
\end{align*}
%
Константа~$\lambda$ обозначает неотрицательный множитель Лагранжа в~задаче~$k$-MST и~приз за вершину\linebreak 
в~задаче~A-PCST. 
Существуют несколько алгоритмов для решения проблемы~PCST, но не для 
решения проб\-ле\-мы A-PCST. Возможное решение~--- снять 
ограничения на ориентацию графа, чтобы\linebreak алгоритм~PCST мог позже 
восстановить ориентацию дерева.

\vspace*{-9pt}

\section{Решение задачи восстановления ограниченного леса с~помощью алгоритма 
$(2-\varepsilon)$-приближения}

\vspace*{-3pt}

Обзор методов решения задачи восстановления ограниченного леса представлен 
в~\cite{goemans1995general}. Задан взвешенный неориентированный граф~$G\hm=(V,E)$. 
Все его веса~$w(e_i)\hm=c_i\geqslant 0$, $e_i\hm\in E$. Задана некоторая 
функция~$f:2^{V}\to \{0, 1\}$. Требуется решить задачу линейного 
программирования с~целочисленными ограничениями:

\vspace*{-4pt}

\noindent
\begin{multline}
\underset{x_e:~e\in E}{\mbox{minimize}} \displaystyle \sum\limits_{e\in E}c_ex_e\\
\mbox{s.t.}\  x\left(\delta(S)\right)\geqslant f(S),\enskip  S \subset V, \enskip S \not= \emptyset,\\
 x_e\in\{0,1\},\enskip  e\in E.
\label{ilp_cfp}
\end{multline}

\vspace*{-3pt}

\noindent
Здесь
$$
x(\delta(S))=\sum\limits_{e\in \delta(S)}x_e,
$$
где $x_e\hm=1$, если 
ребро~$e$ входит в~финальное решение. Функция~$\delta(S)$ обозначает все ребра 
из~$E$ такие, что только одна из смежных вершин входит в~$S$.

Предположим, что отображение~$f$ удовлетворяет условиям

\vspace*{-3pt}

\noindent
\begin{gather*}
f(V) = 0,\\
 \underbrace{f(S)=f(V\backslash S)}_{\mathrm{симметричность}},\\
\underbrace{A,B\!\subset\! V\!: A\!\cap\! B\! =\! \emptyset, f(A)\!=\!f(B)\!=\!0\!\to\! f(A\!\cup\! B)\! =\! 0}_{\mathrm{дизъюнктивность}}.
\end{gather*}

\vspace*{-2pt}

\noindent
При выполнении этих условий~$f$ задает число ребер, начинающихся в~множестве 
вершин~$S$.

\smallskip

\noindent
\textbf{Лемма 1.}
\textit{Пусть $B\subseteq S\subset V$. Тогда $f(S) \hm= 0$ и~$f(B) \hm= 0$ приводит к}~$f(S\backslash B) \hm= 0$.

\smallskip

Задача с~таким описанием относится к~\textit{задачам поиска оптимального леса с~ограничениями}. 
Такая постановка задачи~(\ref{ilp_cfp}) с~соответствующим 
отображением~$f$ подходит для многих известных задач взвешенных графов, 
например: минимальный магистральный поиск, $st$-путь, задача Штейнера на 
минимальном дереве. Последняя задача является NP-полной, поэтому применим 
приближенный алгоритм.

\smallskip

\noindent
\textbf{Определение 3} (\textbf{алгоритм $\bm{\alpha}$-аппроксимации}).
Эвристический полиномиальный алгоритм, дающий\linebreak решение некоторой задачи 
оптимизации, называется $\alpha$-ап\-прок\-си\-ма\-ци\-ей, если он гарантирует 
удовлетворяющее ограничениям решение этой задачи оптимизации с~коэффициентом, 
меньшим или равным~$\alpha$, так что решение отличается от оптимального не более 
чем в~$\alpha$ раз по оптимизируемому функционалу.


\smallskip

Чтобы предложить приближенный алгоритм, целочисленные ограничения 
в~(\ref{ilp_cfp}) должны быть ослаблены:

\vspace*{-3pt}

\noindent
\begin{multline*}
\underset{x_e:~e\in E}{\mbox{minimize}}\  \displaystyle \sum\limits_{e\in E}c_ex_e \\
\mbox{s.t.}\  \displaystyle \sum\limits_{e\in \delta(S)}x_e\geqslant f(S),\enskip S \subset V\,, \enskip S \not= \emptyset\,,\\
 x_e>0,\enskip  e\in E,
%\label{rlp_cfp}
\end{multline*}
Двойственная задача принимает вид:

\vspace*{-4pt}

\noindent
\begin{multline}
\underset{y_S:~S \subset V, \; S \not= \emptyset}{\mbox{maximize}}\  
\displaystyle \sum\limits_{S\subset V}f(S)y_S \\
\mbox{s.t.}\  \displaystyle \sum\limits_{S:~e\in \delta(S)}y_S\leqslant c_e,\enskip  e\in E\,,\\
 y_S>0,\enskip  S \subset V, \enskip S \not= \emptyset\,,
\label{rd_cfp}
\end{multline}

\vspace*{-3pt}

\noindent
относительно дополнительного условия
$$
y_S \left(\sum\limits_{e\in \delta(S)}x_e - f(S)\right) = 0\,,\enskip S\subset  V\,.
$$

Обозначим множество вершин $A=\{v\hm\in V: f(\{v\})\hm=1\}$. Предлагается адаптивный 
жадный алгоритм $\left(2-{2}/{\vert A\vert }\right)$-ап\-прок\-си\-ма\-ции для задач 
вида~(\ref{ilp_cfp}). Алгоритм состоит из двух этапов. На первом этапе он жадно 
объединяет кластеры вершин, увеличивая двойственные переменные~$y_S$. Изначально 
каждая вершина принадлежит своему клас\-те\-ру. Если сле\-ду\-ющее реб\-ро~$e$ достигает 
равенства в~ограничениях в~(\ref{rd_cfp}), это ребро добавляется к~множеству~$S$ и~связанные клас\-те\-ры объединяются. Этот этап аналогичен алгоритму минимального 
остовного дерева Крускала. На втором этапе из конечного множества~$S$ удаляются 
некоторые ребра. Если обрезка ребра не нарушает ограничений, то это реб\-ро должно 
быть удалено.


Индекс $Z_{\mathrm{DRLP}}$ в~алгоритме~1 обозначает линейное 
программирование с~двойной релаксацией. Начальное значение $F:=\emptyset$ 
в~алгоритме~1 эквивалентно предположению $x_e \hm= 0$, $ e \hm\in E$. 
По условиям нежесткости $y_S \hm= 0$, $S \hm\subset V$,  $S \hm\not= \emptyset$.

На каждом шаге алгоритма кластер $\mathcal{C}$ содержит две компоненты 
$\mathcal{C} \hm= \mathcal{C}_i \hm\cup \mathcal{C}_a$, где $C\hm\in\mathcal{C }_a$, если 
$f(C) \hm= 1$, и~$C\hm\in\mathcal{C}_i$ в~противном случае. Назовем~$\mathcal{C}_a$ 
активным компонентом.
Переменные~$d(v)$ в~этом алгоритме связаны с~переменными~$y_S$ из~(\ref{rd_cfp}) 
соотношением
$$
d(i) = \sum\limits_{S:i\in S}y_S.
$$ 

Рассмотрим две различные компоненты $C_q$ и~$C_p$, $C_q\cap C_p\hm=\emptyset$, на 
некоторой итерации первого этапа алгоритма. Все~$y_S$ должны быть равномерно 
распределены по некоторому~$\varepsilon$ без нарушения ограничений
$$
\sum\limits_ {S:~e\in \delta(S)}y_S\leqslant c_e. 
$$
В терминах $d(v)$ это условие принимает вид:
$$
\sum\limits_{S:~e\in \delta(S)}y_S = d\left(v_1\right)+d\left(v_2\right),\enskip e=\left( v_1,v_2\right),
$$
поэтому $y_S\hm=0$ для любого~$S$ такого, что $v_1, v_2\hm\in S$, потому что 
компоненты растут только на первом этапе. Увеличение некоторых компонент на~$\varepsilon$ приводит к~уравнению
$$
d(v_1)+d(v_2)+\varepsilon \left(f(C_q)+f(C_p)\right)\leqslant 
c_e,\ e=\left(v_1,v_2\right), 
$$
что приводит к~формуле, используемой в~строке~$10$ алгоритма~1. 
В~случае когда в~состав входит следующее ребро, сумма $\sum\nolimits_{S:~e\in 
\delta (S)}y_S$ не будет увеличиваться, поэтому ограничения выполняются.

Ребра, которые можно удалить из~$F$ без добавления новых активных компонентов, 
удаляются на втором этапе алгоритма. Следующая лемма определяет свойства 
компонент связ\-ности в~$F'$.


\smallskip

\noindent
\textbf{Лемма~2.}\
\textit{Для каждой компоненты связ\-ности~$N$ из~$F'$ выполняется равенство}: $f(N)\hm=0$.

\smallskip

Следующая теорема утверж\-да\-ет, что решение, полученное с~помощью описанного 
алгоритма, удовле\-тво\-ря\-ет ограничениям исходной задачи линейного 
программирования.

\smallskip

\noindent
\textbf{Теорема~1.}
\textit{Набор ребер $F'$, полученный алгоритмом~$1$, удовлетворяет всем 
ограничениям исходной задачи}~(\ref{ilp_cfp}).


\smallskip

\noindent
\textbf{Лемма~3.}\
\textit{Обозначим граф $H$, каждая вершина которого соответствует одной из компонент 
связ\-ности $C\in\mathcal{C}$ на фиксированном шаге алгоритма. Ребро $(v_1,v_2)$ 
присутствует, если существует ребро $\hat{e}$ исходного графа, входящее в~$F'$: 
$\hat{e} \in F'$, поэтому граф $H$~--- это лес. Внут\-ри $H$ нет листовых вершин, 
со\-от\-вет\-ст\-ву\-ющих неактивным вершинам исходного графа}.

\smallskip

\noindent
\textbf{Теорема 2.}
\textit{Алгоритм~$1$ представляет собой $\alpha$-при\-бли\-жен\-ный алгоритм для 
задачи}~(\ref{ilp_cfp}) \textit{с}~$\alpha \hm= 2 - {2}/{|A|}$, \textit{где} $A\hm=\{v\  V: 
f(\{v\})=1\}$.

\smallskip

Несмотря на эту теоретическую основу, не существует подходящей функции $f$ для 
постановки задачи PCST, указанной в~(\ref{ilp_cfp}). Чтобы быть 
применимым в~этих условиях, алгоритм~1 нуждается в~нескольких 
модификациях.

\vspace*{-9pt}

\section{Модифицированная постановка задачи для~PCST}

\vspace*{-3pt}

Как и~в случае A-PCST, упрощенный вид задачи линейного 
программирования PCST принимает вид:
\begin{multline*}
\underset{\substack{x_e,s_v \\ e\in E, v\in V\backslash \{r\}}}{\mbox{minimize}}\  
\displaystyle \sum\limits_{e\in E}c_ex_e + \sum\limits_{v\in V\backslash\{r\}} \left(1-s_v\right)\pi_v \\
\mbox{s.t.}\  \displaystyle \sum\limits_{e\in\delta(S)} \!\! x_e\geqslant s_v,\enskip S\subseteq V\backslash \{r\},\enskip v\in S,\\
x_e\geqslant 0,\enskip e\in E,\enskip s_v\geqslant 0,\enskip v\in V\backslash \{r\}.
%\label{rlp_pcst_inord}
\end{multline*}
Эта постановка задачи отличается от исходной~(\ref{ilp_pcst_ord}) тем, что с~ней 
возможно согласовать задачу $k$-MST. Индикаторы~$s_v$ показывают, что 
вершина~$v$ включена в~дерево.

Двойственная задача принимает вид:

\vspace*{-3pt}

\noindent
\begin{multline*}
\underset{\substack{y_S:~S\subset V\backslash\{r\}}}{\mbox{maximize}}\ 
\displaystyle \sum\limits_{S\in V\backslash\{r\}}y_S \\
\mbox{s.t.}\  \displaystyle \sum\limits_{S:e\in\delta(S)}y_S\leqslant c_e ,\enskip e\in E;\\
 \displaystyle \sum\limits_{S\subseteq T}y_S\leqslant \sum\limits_{v\in T}\pi_v,\enskip  T\subset  V\backslash\{r\},\\
 y_S\geqslant 0,\enskip  S\subset V\backslash\{r\}.
%\label{rd_pcst_inord}
\end{multline*}

\vspace*{-3pt}

Алгоритм~2 решает эту задачу. Он похож на 
алгоритм~1. Двойные переменные должны обновляться равномерно 
с~дополнительными ограничениями. Тогда~$\varepsilon$ примет минимальное из двух 
значений в~соответствии с~обеими группами ограничений.
Более широкий анализ аппроксимационных свойств обновленного алгоритма 
представлен в~\cite{goemans1995general}. Алгоритм~2 представляет 
собой $\alpha$-приближенный алгоритм для задачи PCST с~$\alpha \hm= 2 \hm- 
{2}/({n-1})$, где $n$~--- число вершин в~графе~$G$.

\vspace*{-9pt}

\section{Вычислительный эксперимент}

\vspace*{-3pt}

Основная цель эксперимента~--- восстановить дерево суперпозиции. Алгоритмы, 
используемые для восстановления, перечислены ниже.

\vspace*{-14pt}

\paragraph*{DFS, BFS.}
Алгоритмы жадного дерева обхода в~глубину и~жадного дерева обхода в~ширину. 
Обход ребер с~наибольшим весом эквивалентен выбору наиболее вероятного пути. 
Алгоритм обхода останавливается, когда число ребер, исходящих из некоторой 
вершины, становится равным арности соответствующей функции.

\vspace*{-14pt}

\paragraph*{Алгоритм Прима.}
Алгоритм восстанавливает минимальное остовное дерево для графа с~дополнительными 
ограничениями на арность базовых функций. Эти ограничения задают минимальный вес 
ребра. После добавления вершины все лис\-то\-вые ребра этой вершины исключаются, 
чтобы сохранить направление дерева. Если число ребер, начинающихся в~какой-либо 
вершине, превышает соответствующую арность, то остальные ребра исключаются из 
множества возможных ребер в~этой вершине. Алгоритм не зависит от процедуры 
обхода. В случае небольшого шума в~матрице смежности этот алгоритм способен 
восстановить дерево суперпозиции без ошибок. 


\vspace*{-14pt}

\paragraph*{Алгоритмы на основе PCST.}
Матрица смеж\-ности~$M$ должна быть приведена к~неориентиро-\linebreak\vspace*{-12pt}

\pagebreak

\noindent
ванному виду. 
Использована квад\-рат\-ная мат\-ри\-ца~$M'$ без последнего столбца. PCST 
принимает мат\-ри\-цу смеж\-ности $1 \hm- ({1}/{2})(M' \hm+ M'^{\mathsf{T}})$ с~призовым 
значением~0,5 для каж\-дой вершины.
Призовое значение рав\-но~0,5, поскольку при меньших значениях дерево будет 
обрезано: если шум равен~0,5, некоторые вершины могут быть обрезаны по ошибке. 
В~случае больших призовых значений
дерево PCST может содержать ненужные 
вершины. Дерево восстанавливается по одному из опи-\linebreak\vspace*{-12pt}

{ \begin{center}  %fig2
 \vspace*{9pt}
    \mbox{%
\epsfxsize=79mm
\epsfbox{str-2.eps}
}
\end{center}



\noindent
{{\figurename~2}\ \ \small{Качество алгоритмов восстановления с~базовыми функциями небольших 
арностей: \textit{1}~--- DFS; \textit{2}~--- BFS; \textit{3}~--- алгоритм Прима;
\textit{4}~--- $k$-MST; \textit{5}~--- $k$-MST--DFS; \textit{6}~--- $h$-MST--BFS; \textit{7}~--- $k$-MST\,--\,ал\-го\-ритм Прима
}}}

\vspace*{6pt}

\addtocounter{figure}{1}

%\begin{table*}\small  %tabl2
\begin{center}
\parbox{75mm}{{{\tablename~2}\ \ \small{Качество алгоритмов реконструкции с~равномерным шумом, близким 
к~0,5
}}
}
    
    
\vspace*{6pt}

  {\small  \begin{tabular}{|l|ccccc|}
      \hline
                  & \multicolumn{5}{c|}{Шум}\\%& & Шум & & \\
       \cline{2-6}
        \multicolumn{1}{|c|}{\raisebox{6pt}[0pt][0pt]{Алгоритм}}                          
&0,50&0,52&0,54&0,56&0,58\\
                    \hline
      DFS        &0,20 &0,20 &0,19 &0,18 &0,16\\
      BFS        &0,60 &0,58 &0,51 &0,46 &0,40\\
      Прима    &1,00 &0,94&0,81&0,69&0,57\\
      $k$-MST     &0,17 &0,16 &0,14 &0,12 &0,10\\
      $k$-MST--DFS   &0,17 &0,16 &0,16 &0,14 &0,14 \\
      $k$-MST--BFS   &0,43 &0,40 &0,36 &0,33 &0,29 \\
      $k$-MST--Прима  &0,44 &0,39 &0,34 &0,33 &0,27 \\
      \hline
    \end{tabular}
    }
\end{center}
%\end{table*}




\noindent
 санных алгоритмов. Результаты 
$\text{PCST}$ можно использовать в~качестве априорных для других подходов, $M':=({1}/{2})(M_{\mathrm{PCST}}' + M')$,
поэтому результаты \mbox{PCST} обновляются~$M'$.


Процедура генерации данных имеет следующие допущения: арности функций 
генерируются биномиальным распределением, поэтому существуют много функций 
с~малой арностью, все базовые функции имеют только один вход. Любой случай 
с~частичной реконструкцией считается ошибкой. Качество алгоритмов реконструкции:
$$
\fr{1}{K}\sum\limits_{k=1}^K \left[ R\left( \bar{N}(M_k)\right)=M_k\right],
$$
где~$R$ ~--- алгоритм реконструкции;
$\bar{N}\hm=\left(N - \min(N)\right)/\left(\max(N)\hm-\min(N)\right)$~--- нормированная мат\-ри\-ца шума. 
Мат\-ри\-ца~$N$ генерируется как~$N(M)\hm=M\hm+U(-\alpha,\alpha)$.
Генератор случайных чисел возвращает матрицу того же вида, что и~$M$, где каждый 
элемент является независимой переменной из равномерного распределения 
в~сегменте~$[-\alpha,\alpha]$.

Вот список из семи сравниваемых алгоритмов:
DFS,
BFS,
алгоритм Прима,
$k$-MST через PCST,
$k$-MST\;+\;DFS,
$k$-MST\;+\;BFS,
$k$-MST\;+\;ал\-го\-ритм Прима.
На рис.~2 показана ошибка алгоритмов реконструкции 
с~шумом, близ\-ким к~порогу~0,5. Наилучшие результаты дает алгоритм Прима. Второе по 
точности решение основано на~$\text{BFS}$. Таб\-ли\-ца~2 
соответствует~рис.~2 и~показывает качество реконструкции 
семи алгоритмов для значений граничного шума~0,50--0,58.





\vspace*{-9pt}

\section{Заключение}

\vspace*{-3pt}

Предлагаются и~сравниваются  алгоритмы вос\-ста\-нов\-ле\-ния суперпозиции для задачи 
символьной регрессии. Алгоритм Прима дает наиболее точ\-ные результаты и~устойчив 
к~небольшому шуму в~данных. Пред\-ла\-га\-емый алгоритм дает точные результаты, но он 
более подвержен шуму в~мат\-ри\-це суперпозиции. Алгоритмы, основанные на BFS и~DFS, 
не могут вос\-ста\-но\-вить исходную суперпозицию с~зашумленными мат\-ри\-ца\-ми 
суперпозиции. Алгоритм PCST с~BFS, используемый для реконструкции мат\-ри\-цы 
суперпозиции, показывает приемлемые для практического использования результаты.

{\small\frenchspacing
 {%\baselineskip=10.8pt
 %\addcontentsline{toc}{section}{References}
 \begin{thebibliography}{99}
\bibitem{koza1992genetic}  %1
\Au{Koza J.\,R.} Genetic programming as a means for programming computers by 
natural selection~// Stat. Comput., 1994. Vol.~4. P.~87--112.

\bibitem{searson2010gptips} %2
\Au{Searson~D.\,P., Leahy~D.\,E., Willis~M.\,J.} GPTIPS: An open source 
genetic programming toolbox for multigene  symbolic regression~// 
Multiconference (International) of Engineers and Computer Scientists Proceedings, 
2010. Vol.~1. P.~77--80.

\bibitem{stanley2002evolving} %3
\Au{Stanley~K.\,O., Miikkulainen~R.} Evolving neural networks through 
augmenting topologies~// Evol. Comput., 2002. Vol.~10. 
Iss.~2. P.~99--127.

\bibitem{bochkarev2017generation}
\Au{Бочкарев~А.\,М., Софронов~И.\,Л., Стрижов~В.\,В.} По\-рож\-де\-ние экс\-перт\-но-ин\-тер\-пре\-ти\-ру\-емых 
моделей для прогноза проницаемости горной породы~// Системы и~средства информатики, 2017. Т.~27. №\,3. С.~74--87.
%

\bibitem{ravi1996spanning}
\Au{Ravi~R., Sundaram~R., Marathe~M.\,V., Rosenkrantz~D.\,J., Ravi~S.\,S.} 
Spanning trees~--- short or small~// SIAM J.~Discrete Math., 
1996. Vol.~9. Iss.~2. P.~178--200.

\bibitem{chudak2004approximate}
\Au{Chudak~F.\,A.,  Roughgarden~T., Williamson~D.\,P.} Approximate $k$-MSTS 
and $k$-Steiner trees via the primal-dual method and Lagrangean 
relaxation~// Math. Program., 2004. Vol.~100. Iss.~2. P.~411--421.

\bibitem{awerbuch1998new}
\Au{Awerbuch~B., Azar~Y., Blum~A., Vempala~S.} New approximation guarantees 
for minimum-weight $k$-trees and prize-collecting salesmen~// SIAM J. 
Comput., 1998. Vol.~28. Iss.~1. P.~254--262.

\bibitem{arora20062+}
\Au{Aror~S., Karakostas~G.} A~$2+\varepsilon$ approximation algorithm for the 
$k$-MST problem~// Math. Program., 2006. Vol.~107. 
Iss.~3. P.~491--504.

\bibitem{hegde2014fast}
\Au{Hegde~C., Indyk~P., Schmidt~L.} A~fast, adaptive variant of the 
Goemans--Williamson scheme for the prize-collecting steiner tree problem~// 11th DIMACS Implementation Challenge Workshop Proceedings, 2014. P.~1--32.
{\sf http://people. csail.mit.edu/ludwigs/papers/dimacs14\_fastpcst.pdf}.

\bibitem{ras2017approximate}
\Au{Ras~C., Swanepoel~K., Thomas~D.\,A.} Approximate Euclidean Steiner 
trees~// J.~Optimiz. Theory App., 2017. Vol.~172. 
Iss.~3. P.~845--873.

\bibitem{goemans1995general}
\Au{Goemans~M.\,X., Williamson~D.\,P.} A~general approximation technique for 
constrained forest problems~// SIAM J. Comput., 1995. Vol.~24. 
Iss.~2. P.~296--317.
\end{thebibliography}

 }
 }

\end{multicols}

\vspace*{-6pt}

\hfill{\small\textit{Поступила в~редакцию 23.01.22}}

\vspace*{8pt}

%\pagebreak

%\newpage

%\vspace*{-28pt}

\hrule

\vspace*{2pt}

\hrule

%\vspace*{-2pt}

\def\tit{OPTIMAL SPANNING TREE RECONSTRUCTION IN~SYMBOLIC~REGRESSION}


\def\titkol{Optimal spanning tree reconstruction in~symbolic regression}


\def\aut{R.\,G.~Neychev$^1$, I.\,A.~Shibaev$^1$, and~V.\,V.~Strijov$^2$}

\def\autkol{R.\,G.~Neychev, I.\,A.~Shibaev, and~V.\,V.~Strijov}

\titel{\tit}{\aut}{\autkol}{\titkol}

\vspace*{-8pt}


\noindent
$^1$Moscow Institute of Physics and Technology, 9~Institutskiy Per., Dolgoprudny, Moscow Region 141700, Russian\linebreak
$\hphantom{^1}$Federation

\noindent
$^2$Federal Research Center ``Computer Science and Control'' of the Russian Academy of Sciences, 44-2~Vavilov Str.,\linebreak
$\hphantom{^1}$Moscow 119333, Russian Federation

\def\leftfootline{\small{\textbf{\thepage}
\hfill INFORMATIKA I EE PRIMENENIYA~--- INFORMATICS AND
APPLICATIONS\ \ \ 2023\ \ \ volume~17\ \ \ issue\ 1}
}%
 \def\rightfootline{\small{INFORMATIKA I EE PRIMENENIYA~---
INFORMATICS AND APPLICATIONS\ \ \ 2023\ \ \ volume~17\ \ \ issue\ 1
\hfill \textbf{\thepage}}}

\vspace*{3pt} 



\Abste{The paper investigates the problem of regression model generation. A~model is a~superposition of primitive functions. 
The model structure is described by a~weighted colored graph. Each graph vertex corresponds to a~primitive function. 
An edge assigns a~superposition of two functions. The weight of an edge is equal to the probability of superposition. 
To generate an optimal model, one has to reconstruct its structure from its graph adjacency matrix. 
The proposed algorithm reconstructs the minimum spanning tree from the weighted colored graph. 
The paper presents a~novel solution based on the prize-collecting Steiner tree algorithm. This algorithm is compared with its alternatives.}


\KWE{symbolic regression; linear programming; superposition; minimum spanning tree; adjacency matrix}



\DOI{10.14357/19922264230105} 

\vspace*{-16pt}

\Ack

\vspace*{-3pt}


\noindent
This work was supported by the Russian Foundation for Basic Research, projects 20-37-90050 and 20-07-00990.
  

\vspace*{6pt}

  \begin{multicols}{2}

\renewcommand{\bibname}{\protect\rmfamily References}
%\renewcommand{\bibname}{\large\protect\rm References}

{\small\frenchspacing
 {%\baselineskip=10.8pt
 \addcontentsline{toc}{section}{References}
 \begin{thebibliography}{99} 

\bibitem{1-str}
\Aue{Koza, J.\,R.}
 1994. Genetic programming as a means for programming computers by natural selection. \textit{Stat. Comput.} 4:87--112.

\bibitem{2-str}
\Aue{Searson, D.\,P., D.\,E.~Leahy, and M.\,J.~Willis.}
 2010. \mbox{GPTIPS}: An open source genetic programming toolbox for multigene symbolic regression. 
 \textit{Multiconference (International) of Engineers and Computer Scientists Proceedings}. 1:77--80. 

\bibitem{3-str}
\Aue{Stanley, K.\,O., and R.~Miikkulainen.} 2002. Evolving neural networks through augmenting topologies. 
\textit{Evol. Comput.} 10(2):99--127.

\bibitem{4-str}
\Aue{Bochkarev, A.\,M., I.\,L.~Sofronov, and V.\,V.~Strijov.}
 2017. Po\-rozh\-de\-nie eks\-pert\-no-inter\-pre\-ti\-ru\-emykh mo\-de\-ley dlya prog\-no\-za pro\-ni\-tsa\-emosti gor\-noy po\-ro\-dy 
 [Generation of expertly-interpreted models for prediction of core permeability]. \textit{Sistemy i~Sredstva Informatiki~--- Systems and Means of Informatics}
  27(3):74--87.

\bibitem{5-str}
\Aue{Ravi, R., R.~Sundaram, M.\,V.~Marathe, D.\,J.~Rosenkrantz, and S.\,S.~Ravi.}
 1996. Spanning trees~--- short or small. \textit{SIAM J. Discrete Math.} 9(2):178--200.

\bibitem{6-str}
\Aue{Chudak, F.\,A., T.~Roughgarden, and D.\,P.~Williamson.}
 2004. Approximate k-MSTS and k-Steiner trees via the primal-dual method and Lagrangean relaxation. 
 \textit{Math. Program.} 100(2):411--421.

\bibitem{7-str}
\Aue{Awerbuch, B., Y.~Azar, A.~Blum, and S.~Vempala.}
 1998. New approximation guarantees for minimum-weight \mbox{k-trees} and prize-collecting salesmen.
 \textit{SIAM J. Comput.} 28(1):254--262.

\bibitem{8-str}
\Aue{Arora, S., and G.~Karakostas.} 2006. A~$2+\varepsilon$ approximation algorithm for the $k$-MST problem. 
\textit{Math. Program.} 107(3):491--504.

\bibitem{9-str}
\Aue{Hegde, C., P.~Indyk, and L.~Schmidt.} 2014. 
A~fast, adaptive variant of the Goemans--Williamson scheme for the prize-collecting Steiner tree problem. 
\textit{11th DIMACS Implementation Challenge Workshop Proceedings}. 1--32.
Available at: 
{\sf http://people.csail.mit.edu/ludwigs/papers/\linebreak dimacs14\_fastpcst.pdf} (accessed January~10, 2023).

\bibitem{10-str}
\Aue{Ras, C., K.~Swanepoel, and D.\,A.~Thomas.} 
2017. Approximate Euclidean Steiner trees. \textit{J.~Optimiz. Theory  App.} 172(3):845--873.

\bibitem{11-str}
\Aue{Goemans, M.\,X., and D.\,P.~Williamson.} 1995. 
A~general approximation technique for constrained forest problems. \textit{SIAM J. Comput.} 24(2):296--317.
 \end{thebibliography}

 }
 }

\end{multicols}

\vspace*{-6pt}

\hfill{\small\textit{Received January 23, 2022}}

\Contr

\noindent
\textbf{Neychev Radoslav G.} (b.\ 1994)~--- 
PhD student, Moscow Institute of Physics and Technology, 9~Institutskiy Per., Dolgoprudny, Moscow Region 141701, Russian Federation;
\mbox{neychev@phystech.edu}

\vspace*{3pt}

\noindent
\textbf{Shibaev Innokentii A.} (b.\ 1997)~--- 
PhD student, Moscow Institute of Physics and Technology, 9~Institutskiy Per., Dolgoprudny, Moscow Region 141701, Russian Federation; 
\mbox{shibaev.kesha@gmail.com}

\vspace*{3pt}

\noindent
\textbf{Strijov Vadim V.} (b.\ 1967)~--- 
Doctor of Science in physics and mathematics, leading scientist, A.\,A.~Dorodnicyn Computing Center, 
Federal Research Center ``Computer Science and Control'' of the Russian Academy of Sciences, 40~Vavilov Str., Moscow 119333, Russian Federation;
\mbox{strijov@phystech.edu}


\label{end\stat}

\renewcommand{\bibname}{\protect\rm Литература}    %9
\def\stat{kudr}

\def\tit{ПРИБЛИЖЕННЫЕ МЕТОДЫ РЕШЕНИЯ ЗАДАЧИ ДИАГНОСТИКИ ПЛОСКИМ 
ЗОНДОМ СИЛЬНОИОНИЗОВАННОЙ ПЛАЗМЫ С~УЧЕТОМ КУЛОНОВСКИХ 
СТОЛКНОВЕНИЙ}

\def\titkol{Приближенные методы решения задачи диагностики плоским 
зондом сильноионизованной плазмы} %с~учетом Кулоновских  столкновений}

\def\autkol{И.\,А.~Кудрявцева, А.\,В.~Пантелеев}
\def\aut{И.\,А.~Кудрявцева$^1$, А.\,В.~Пантелеев$^2$}

\titel{\tit}{\aut}{\autkol}{\titkol}

%{\renewcommand{\thefootnote}{\fnsymbol{footnote}}\footnotetext[1]
%{Работа поддержана Российским фондом фундаментальных исследований
%(проекты 11-01-00515а и 11-07-00112а), а также Министерством
%образования и науки РФ в рамках ФЦП <<Научные и
%научно-педагогические кадры инновационной России на 2009--2013~годы>>.}}


\renewcommand{\thefootnote}{\arabic{footnote}}
\footnotetext[1]{Московский авиационный институт, irina.home.mail@mail.ru}
\footnotetext[2]{Московский авиационный институт, avpanteleev@inbox.ru}

\vspace*{-2pt}

\Abst{Сформирована математическая модель, описывающая динамику сильноионизованной 
плазмы с учетом столкновений заряженных частиц вблизи плоского зонда. Модель включает уравнение 
Фоккера--Планка и уравнение Пуассона. Предложено два подхода к решению задачи: на основе метода 
статистических испытаний Мон\-те-Кар\-ло и на основе композиции метода крупных частиц и метода 
расщепления.} 

\vspace*{-2pt}

\KW{телекоммуникационные системы; метод Монте-Карло; метод крупных частиц; метод 
расщепления; зонд; уравнение Фоккера--Планка; уравнение Пуассона} 

\vspace*{-4pt}

 \vskip 8pt plus 9pt minus 6pt

      \thispagestyle{headings}

      \begin{multicols}{2}
      
            \label{st\stat}

\section{Введение}

В настоящее время в области телекоммуникаций все более востребованными становятся 
информационные технологии, основанные на использовании математических моделей и численных 
методов физики плазмы. Поэтому особенно актуальным является решение разнообразных задач анализа 
поведения плазмы, включающих в себя формирование новых моделей и методов их исследования. 
Помимо этого, в разработке телекоммуникационного оборудования эффективно используются 
собственно физические свойства плазмы. В~частности, изготовлена антенна, работа которой основана 
на газовом разряде низкотемпературной плазмы~[1], интенсивно ведутся разработки по созданию и 
усовершенствованию источников бесперебойного питания на основе плазменных элементов~[2, 3]. 
      
      Одним из наиболее перспективных направлений для построения систем оптической 
беспроводной связи является использование лазеров~\cite{4-k, 5-k}. В~этой связи большое внимание 
уделяется использованию плазмы при разработке импульсных сильноточных коммутаторов~\cite{6-k}, 
так как практическое применение подобных разработок требует повышения уровня надежности и 
быстродействия лазерных систем.
      
      Исследования низкотемпературной плазмы также связаны с разработками в области дальней 
космической связи, так как моделирование процессов взаимодействия заряженного тела с верхними 
слоями атмосферы позволяет предлагать способы улучшения существующих систем радиосвязи с 
космическими летательными аппаратами~\cite{7-k}. 
      
      Наряду с этим актуальными также являются задачи диагностики плазмы, поскольку перспективы 
ее использования в области телекоммуникаций после более полного изучения физических свойств 
могут значительно расшириться. 

Для диагностики плазмы применяют зондовые методы исследования~[8--11]. Эти методы относятся к 
классу контактных методов; как следствие, возникает сложность в исследовании пристеночной области 
вблизи зонда, которая характеризуется достаточно сложным распределением потенциала и функциями 
распределения, отличными от максвелловских. 

Данная работа посвящена исследованию переходного режима обтекания заряженного тела плазмой. Для 
переходного режима выполняется следующее условие: длина свободного пробега иона до столкновения 
с нейтральным атомом или другим ионом невелика по сравнению с характерными размерами тела. 
В~этом случае возникает необходимость учета столкновений заряженных частиц с нейтральными 
атомами и кулоновских столкновений. В~работах~\cite{10-k, 11-k} подробно рассмотрена модель с 
учетом столкновений заряженных частиц с нейтральными атомами. В~настоящей статье представлена 
теоретическая модель, описывающая влияния ион-ионных и ион-элек\-т\-рон\-ных столкновений на 
измеряемые характеристики плазмы, что ранее детально не исследовалось.
      
      В~рамках данной работы предлагается модель, описывающая динамику сильноионизованной 
плазмы с учетом кулоновских столкновений. Эта модель учитывает такие процессы взаимодействия, 
как перенос частиц и столкновения между заряженными частицами типа <<ион--ион>> и 
      <<ион--электрон>> под влиянием макроскопического электрического поля. Перечисленные 
процессы описываются самосогласованной системой уравнений, включающей уравнение 
      Фок\-ке\-ра--План\-ка и уравнение Пуассона~[12].
      
      Вычислительная модель задачи строится на основе двух методов: метода статистических 
испытаний Мон\-те-Кар\-ло и композиции метода крупных частиц и метода расщепления. Приведены 
результаты численного моделирования, полученные с использованием вышеперечисленных методов.

\vspace*{-4pt}

\section{Постановка задачи}

\vspace*{-2pt}

Рассматривается следующая физическая постановка зондовой задачи~[11]. В~невозмущенную 
бесконечно протяженную плазму, состоящую из электронов и однозарядных ионов, внесена большая\linebreak 
заряженная до потенциала $\varphi_p$ плоскость. Плоскость, расположенная поперек потока плазмы, 
является идеально поглощающей для электронов. Ионы при ударе о плоскость нейтрализуются. 
Предполагается, что частицы в плазме движутся под действием внешнего электрического поля, 
магнитное поле отсутствует. Концентрации ионов $n_{i\infty}$ и электронов $n_{e\infty}$, а также 
температуры данных час\-тиц~$T_{i\infty}$ 
и~$T_{e\infty}$ в невозмущенной плазме заданы. За начальные 
функции распределения обоих типов час\-тиц принимаются функции распределения Максвелла. 
      
      Требуется с учетом столкновений между заряженными частицами найти напряженность 
самосогласованного электрического поля $\vec{E}(\vec{r},t)$, функции распределения однозарядных 
ионов $f_i(\vec{r}, \vec{v}, t)$ и электронов $f_e(\vec{r}, \vec{v}, t)$, 
а также их моменты (плотности 
токов ионов и электронов  $j_i(\vec{r},t)\hm
=q\int f_i(\vec{r}, \vec{v}, t)\vec{v}\,d\vec{v}$, $j_e(\vec{r},t) 
\hm={\sf e}\int f_e(\vec{r},\vec{v},t)\vec{v}\,d\vec{v}$, где $q=Z_i{\sf e}$, $Z_i=1$~--- заряд иона, ${\sf 
e}$~--- заряд электрона; концентрации ионов и электронов $n_i(\vec{r},t)\hm=\int 
f_i(\vec{r},\vec{v},t)\,d\vec{v}$, $n_e(\vec{r},t)\hm=\int f_e(\vec{r},\vec{v}, t)\,d\vec{v}$). 
Поведение частиц во 
времени~$t$ характеризуется ра\-ди\-ус-век\-то\-ром~$\vec{r}$ и вектором скорости~$\vec{v}$.
      
      Математическая модель, соответствующая данной физической постановке задачи, имеет 
вид~\cite{11-k, 13-k}:

\noindent
      \begin{equation}
      \left.
      \begin{array}{c}
      \fr{\partial f_\alpha (\vec{r},\vec{v},t)}{\partial t}+
      \vec{v}\fr{\partial f_\alpha (\vec{r},\vec{v},t)}{ 
\partial \vec{r}}+
\fr{\vec{F}_\alpha(\vec{r},t)}{m_\alpha}\times{}\\[4pt]
{}\times\fr{\partial f_\alpha(\vec{r},\vec{v},t)}{ \partial 
\vec{v}}=
\left(\fr{\partial f_\alpha(\vec{r},\vec{v},t)}{ \partial t}\right)_{\mathrm{с}}+S_\alpha 
(\vec{r},\vec{v},t)\,;\\[6pt]
      \Delta\varphi(\vec{r},t)=-\fr{{\sf e}}{\varepsilon_0}\left( n_i(\vec{r},t)-n_e(\vec{r},t)\right)\,;\\[6pt]
      \vec{E}(\vec{r},t)=-\nabla \varphi(\vec{r},t)\,.
      \end{array}\!\!
      \right\}\!\!
      \label{e1-k}
      \end{equation}
Здесь первое уравнение~--- уравнение Фок\-ке\-ра--План\-ка для частиц сорта~$\alpha$ ($\alpha=i,e$), 
второе~--- уравнение Пуассона для самосогласованного электрического поля; 
$f_\alpha(\vec{r},\vec{v},t)$~--- функция\linebreak
распределения час\-тиц сорта~$\alpha$; $(\partial 
f_\alpha(\vec{r},\vec{v},t)/\partial t)_{\mathrm{с}}$~--- 
оператор столкновений Фок\-ке\-ра--План\-ка; 
функция~$S_\alpha(\vec{r},\vec{v},t)$ описывает источники или стоки\linebreak
 час\-тиц; 
$\vec{F}_\alpha(\vec{r},t)=q_\alpha\vec{E}(\vec{r},t)$, где $\vec{E}(\vec{r},t)$~--- напряженность 
самосогласованного электрического поля, 
$$
q_\alpha =
\begin{cases}
-{\sf e}\,, & \alpha=e\,,\\
{\sf e}\,, & \alpha=i\,;
\end{cases}
$$
$\varphi(\vec{r},t)$~--- потенциал самосогласованного электрического поля; $n_\alpha(\vec{r},t)$ ($\alpha 
\hm=i,e$)~--- концентрация частиц сорта~$\alpha$; $m_\alpha$~--- масса частицы сорта~$\alpha$; 
$\varepsilon_0$~--- электрическая постоянная. 

Оператор столкновений Фок\-ке\-ра--План\-ка имеет вид~\cite{13-k, 14-k}
\begin{multline*}
\fr{1}{\Gamma_\alpha}\left( \fr{\partial f_\alpha}{\partial t}\right)_{\mathrm{с}} 
=\fr{1}{2}\,\nabla_v\nabla_v:\left(f_\alpha\nabla_v\nabla_vg_\alpha(\vec{r},\vec{v},t)\right)-{}\\
{}-
\nabla_v\cdot\left(f_\alpha\nabla_v h_\alpha\right)\,,
\end{multline*}
где $\nabla_v\nabla_v g_\alpha(\vec{r},\vec{v},t)$~--- ковариантная тензорная производная второго ранга, 
знак двоеточия ($:$) обозначает операцию двойного суммирования:
\begin{gather*}
\Gamma_\alpha=\fr{Z_\alpha^4 {\sf e}^4}{4\pi \varepsilon_0^2 m^2_\alpha}\,\ln D_\alpha\,;
\\
D_\alpha =\fr{12\pi\varepsilon_0 kT_{\alpha\infty}}{Z_\alpha^2 {\sf e}^2}\left( \fr{\varepsilon_0 k 
T_{e\infty}}{n_{e\infty} {\sf e}^2}\right)^{1/2}\,;\\
g_\alpha (\vec{r},\vec{v},t)=\sum\limits_{b=i,e}\left( \fr{Z_b}{Z_\alpha}\right) \int f_b 
(\vec{r},{\vec{v}}^{\,\prime},t)\left\vert \vec{v}-{\vec{v}}^{\,\prime}\right\vert\,d\vec{v}^{\,\prime}\,;\\
h_\alpha (\vec{r},\vec{v},t)=\sum\limits_{b=i,e} \fr{m_\alpha+m_b}{m_b} 
\left(\fr{Z_b}{Z_\alpha}\right)
\int
\fr{f_b(\vec{r},{\vec{v}}^{\,\prime}, t)}{\vert \vec{v}-{\vec{v}}^{\,\prime}\vert}
\,d{\vec{v}}^{\,\prime}\,;\\
Z_\alpha =1\,, \quad \alpha=i,e\,.
\end{gather*}
 
К системе уравнений~(\ref{e1-k}) необходимо добавить начальные и краевые условия:
\begin{equation}
\!\left.
\begin{array}{rrl}
t=0:\ & f_\alpha(\vec{r},\vec{v},0)&=f_\alpha^{\mathrm{maksv}}\,,\enskip \alpha=i,e;\\[9pt]
\vec{r}\in \Omega_p:\ & f_\alpha(\vec{r},\vec{v},t)\big\vert_{\vec{r}\in\Omega_p}&=0\,,\enskip \alpha=i,e\,;\\[9pt]
&\varphi(\vec{r},t)\big\vert_{\vec{r}\in\Omega_p}&=\varphi_p\,;\\[9pt]
\vec{r}\in\Omega_\infty:\ & 
f_\alpha(\vec{r},\vec{v},t)\big\vert_{\vec{r}\in\Omega_\infty}&= %{}\\[9pt]
f_\alpha^{\mathrm{maksv}}\,,\enskip \alpha=i,e\,;\\[9pt]
&\varphi(\vec{r},t)\big\vert_{\vec{r}\in\Omega_\infty}&=0\,,
\end{array}\!\!
\right\}\!\!\!\!
\label{e2-k}
\end{equation}
    где 
    
    \noindent
    \begin{multline*}
    f_\alpha^{\mathrm{maksv}}=n_{\alpha\infty}\left(\fr{m_\alpha}{2k\pi T_{\alpha\infty}}\right)^{3/2}\times{}\\
    {}\times
    \exp\left( -
\fr{m_\alpha}{2kT_{\alpha\infty}}\left\vert\vec{v}-\vec{v}_\infty\right\vert^2\right)\,,
\enskip \alpha=i, e\,;
\end{multline*} 
$\Omega_p$ и $\Omega_\infty$~--- множество радиус-векторов час\-тиц, концы которых принадлежат плоскости зонда и 
границе возмущенной зоны соответственно.

Для решения поставленной задачи введем декартову систему координат таким образом, чтобы 
заряженная плоскость совпала с плоскостью~$0xz$. Тогда положение частицы в пространстве будет 
определяться координатами $x,y,z$, а скорость~--- координатами $v_x, v_y, v_z$. В~силу того что 
плоскость является бесконечно большой в сравнении с характерным размером задачи, функции 
распределения частиц будут зависеть только от переменных $y, v_y, t$.

Поставленную задачу предлагается решать независимо двумя методами. Первый метод основывается на 
методе статистических испытаний Мон\-те-Кар\-ло, второй метод является композицией метода 
расщепления и метода крупных частиц.

\section{Применение метода Монте-Карло}

Запишем самосогласованную систему уравнений~(\ref{e1-k}) и~(\ref{e2-k}) в декартовой системе 
координат с учетом сделанных предположений:
\begin{equation}
\left.
\begin{array}{l}
\fr{\partial f_\alpha}{\partial t}+
v_y\fr{\partial f_\alpha}{\partial y}+\fr{F_y^\alpha}{m_\alpha}\,\fr{\partial 
f_\alpha}{\partial v_y}=\fr{1}{2}\,\fr{\partial^2 }{\partial [v_y]^2}\times{}\\
{}\times \left( 
f_\alpha\fr{\partial^2 g_\alpha  }{\partial [v_y]^2}\right) -
\fr{\partial}{\partial v_y}\left( f_\alpha\fr{\partial h_\alpha}{\partial v_y}\right)\,,
\enskip \alpha=i,e\,;\\[6pt]
    \fr{\partial^2\varphi}{\partial y^2} =-\fr{{\sf e}}{\varepsilon_0}\left(n_i-n_e\right)\,;
    \enskip E_y=-
\fr{\partial\varphi}{\partial y}\,;\\[6pt]
\hspace*{3.1mm}    t=0:\  \hspace*{2.6mm}f_\alpha(y,v_y,0)=f_\alpha^{\mathrm{maksv}}\,,\ \alpha=i,e\,;\\[9pt]
\hspace*{2.9mm} y=0:\ \hspace*{2.8mm}f_\alpha(0,v_y,t)=0\,,\ \alpha=i,e\,;\\[9pt]
\hspace*{24.3mm}\varphi(0,t)=\varphi_p\,;\\[9pt]
y=y_\infty:\ f_\alpha(y_\infty, v_y, t)=f_\alpha^{\mathrm{maksv}}\,,\ \alpha=i,e\,;\\[9pt]
\hspace*{21.5mm}\varphi(y_\infty, t)=0\,.
\end{array}
\right \}
\label{e3-k}
\end{equation}

В полученной системе уравнений~(\ref{e3-k}) перейдем к безразмерным величинам, применив 
соотношение $X=M_X \hat{X}$, где $M_X$~--- масштаб размерной величины~$X$, $\hat{X}$~--- 
безразмерная величина~$X$. В~качестве используемых масштабов были взяты следующие: радиус 
Дебая, скорость теплового движения частиц, концентрация частиц в невозмущенной плазме, потенциал, 
возникающий при разделении зарядов в дебаевской сфере, и производные от них величины.

Система безразмерных уравнений имеет следующий вид:
%\noindent
\begin{equation}
\left.
\begin{array}{l}
\fr{\partial 
\hat{f}_\alpha}{\partial\hat{t}}+A_\alpha\fr{\partial\hat{f}_\alpha}{\partial\hat{y}}+
B_\alpha\hat{E}_y\fr{\partial\hat{f}_\alpha}{\partial \hat{v}_y}={}\\
\!{}=
\fr{\partial^2}{\partial[\hat{v}_y]^2}\left(D_\alpha 
\hat{f}_\alpha\right)-\fr{\partial}{\partial\hat{v}_y}\left(K_\alpha \hat{f}_\alpha\right),\enskip 
\alpha=i,e;\\[9pt]
\fr{\partial^2\hat{\varphi}}{\partial\hat{y}^2}=-\left(\hat{n}_i-\hat{n}_e\right)\,;\enskip \hat{e}_y=-
\fr{\partial\hat\varphi}{\partial\hat{y}}\,;\\[9pt]
\hspace*{3.1mm}\hat{t}=0:\ \hspace*{2.6mm}\hat{f}_\alpha(\hat{y},\hat{v}_y,0)=\hat{f}_\alpha^{\mathrm{maksv}}\,,\enskip \alpha-i,e\,;\\[9pt]
\hspace*{2.9mm}\hat{y}=0:\ \hspace*{2.8mm}\hat{f}_\alpha(0,\hat{v}_y,\hat{t})=0\,,\enskip \alpha=i,e\,;\\[9pt]
\hspace*{24.3mm}\hat\varphi(0,\hat{t})=\hat{\varphi}_p\,;\\[9pt]
\hat{y}=\hat{y}_\infty:\ \hat{f}_\alpha(\hat{y}_\infty, \hat{v}_y, \hat{t})=\hat{f}^{\mathrm{maksv}}_\alpha\,,\enskip 
\alpha=i,e\,;\\[9pt]
\hspace*{21.5mm}\hat\varphi(\hat{y}_\infty,\hat{t})=0\,.
\end{array}
\right\}
\label{e4-k}
\end{equation}
Здесь 

\vspace*{-2pt}

\noindent
\begin{gather*}
A_\alpha=\sqrt{\delta_\alpha }\,\hat{v}_y\,;\enskip 
B_\alpha=\sqrt{\delta_\alpha}\,\fr{z_\alpha}{2\varepsilon_\alpha}\,;\\
\delta_\alpha=\fr{\varepsilon_\alpha}{\mu_\alpha}\,;\enskip 
\varepsilon_\alpha=\fr{T_{\alpha\infty}}{T_{i\infty}}\,;\\
\mu_\alpha=\fr{m_\alpha}{m_i}\,;\enskip 
D_\alpha=A_g^\alpha\fr{\partial^2\hat{g}_\alpha}{\partial  [\hat{v}_y]^2}\,;\\
K_\alpha=A_h^\alpha \fr{\partial \hat{h}_\alpha}{\partial \hat{v}_y}\,,\enskip \alpha=i,e\,,
\end{gather*}
где $A_g^\alpha$ и $A_h^\alpha$~--- коэффициенты, определяемые характерными параметрами 
задачи~\cite{15-k}.

Поиск решения самосогласованной системы уравнений~(\ref{e4-k}) осуществляется по следующей 
схе-\linebreak ме. Вначале находятся значения напряженности\linebreak
 электрического поля по значениям потенциала, 
полученным из граничной задачи для уравнения Пуассона. Далее, используя найденные значения 
напряженности, решается уравнение Фок\-ке\-ра--План\-ка путем перехода к стохастическому 
дифференциальному уравнению (СДУ) Ито:

\noindent
\begin{multline*}
d\Theta_\alpha(\hat{t}) = a_\alpha \left(\hat{t},\Theta_\alpha(\hat{t})\right)+{}\\
{}+\sigma\left(
\hat{t},\Theta_\alpha(\hat{t})\right)\,dW(\hat{t})\,,\quad \alpha=i,e\,,
%\label{e5-k}
\end{multline*}
где 

\noindent
\begin{align*}
\Theta_\alpha(\hat{t})&=\begin{bmatrix}
\hat{y}(\hat{t})\\ \hat{v}_y(\hat{t})
\end{bmatrix}\,;\\
a_\alpha\left(\hat{t},\Theta_\alpha(\hat{t})\right)&=\begin{bmatrix}
-A_\alpha\\ -K_\alpha -B_\alpha \hat{E}_y
\end{bmatrix}\,;\\
\sigma_\alpha\left(\hat{t},\Theta_\alpha(\hat{t})\right)\sigma_\alpha^{\mathrm{T}}\left( 
\hat{t},\Theta_\alpha(\hat{t})\right)&=D_\alpha\,,\enskip \alpha=i,e\,;
\end{align*} 
$W(\hat{t})$~--- стандартный винеровский случайный процесс.
\pagebreak

Для нахождения значений вектора состояния~$\Theta_\alpha(\hat{t})$ применим явную разностную 
схему стохастического метода Эйлера~\cite{16-k}:
\begin{multline*}
\Theta_\alpha^{n+1}=\Theta_\alpha^n +h_\tau a_\alpha \left( \hat{t}_n, \Theta_\alpha^n\right)+\sigma_\alpha 
\left( \hat{t}_n, \Theta_\alpha^n\right)\Delta W_n\,,\\ 
n=0,\ldots , N\,,\ \alpha=i,e\,,
%\label{e6-k}
\end{multline*}
где $\Theta_\alpha^n$, $n=0,\ldots , N$,~--- приближенное значение вектора 
состояния~$\Theta_\alpha(\hat{t})$, $\alpha=i,e$, в момент времени $\hat{t}\hm=\hat{t}_n$, 
$\hat{t}_n\hm=n h_\tau$, $n=0,\ldots , N$; $h_\tau$~--- достаточно малый шаг интегрирования; $\Delta 
W_n$, $n=0,\ldots ,N$,~--- величина приращения винеровского процесса~$W(\hat{t})$ на отрезке $\left[ 
\hat{t}_n,\,\hat{t}_{n+1}\right]$, по определению независимая от~$\Theta_\alpha^0$, 
$\Delta W_0,\ldots , 
\Delta W_{n-1}$: $\Delta W_n\hm=W(\hat{t}_{n-1})\hm-W(\hat{t}_n)$; $\Delta W_n\hm\sim N(0,\,h_\tau)$, 
т.\,е.\ $\Delta W_n$ представляют собой гауссовские случайные величины с нулевыми математическими 
ожиданиями и дисперсиями, равными шагу интегрирования; $\Theta_\alpha^0$~--- значение вектора 
состояния $\Theta_\alpha(\hat{t})$, $\alpha\hm=i,e$, в момент времени $\hat{t}=0$, 
$\Theta_\alpha^0\hm\sim \hat{f}_\alpha^{\mathrm{maksv}}$. 

Частные производные $\partial^2\hat{g}_\alpha/\partial[\hat{v}_y]^2$ и $\partial \hat{h}_\alpha/\partial 
\hat{v}_y$, являющиеся составляющими матрицы $\sigma_\alpha (\hat{t}_n, 
\Theta_\alpha^n)\sigma_\alpha^{\mathrm{T}}(\hat{t}_n,\Theta_\alpha^n)$ и вектора $a_\alpha(\hat{t}_n, 
\Theta_\alpha^n)$ соответственно, аппроксимируются со вторым порядком точности на трехточечном 
шаблоне на основе значений~$\hat{g}_\alpha$ и~$\hat{h}_\alpha$~\cite{17-k}.
      
      В выражения для функций~$\hat{g}_\alpha$ и~$\hat{h}_\alpha$ входят интегралы, которые 
вычисляются методом Мон\-те-Кар\-ло с использованием набора значений скоростной компоненты 
вектора состояния~$\hat{v}_y$, полученных из решения СДУ Ито:
      \begin{equation*}
      \int \hat{f}_\alpha \left\vert \hat{v}_y-
\hat{v}_y^\prime\right\vert\,dv_y^\prime=M\left(\zeta\left(\hat{V}_y\right)\right)\,,
\end{equation*}
где
$$
      \zeta\left(\hat{V}_y\right)=\left\vert \hat{v}_y-\hat{V}_y\right\vert\,,\enskip \hat{V}_y\sim 
\hat{f}_\alpha\,.
  $$
      
      Для вычисления напряженности самосогласованного электрического поля $\hat{E}_y=-
\partial\hat{\varphi}/\partial\hat{y}$, входящей в вектор $a_\alpha(\hat{t}_n, \Theta_\alpha^n)$, необходимо 
аналогично аппроксимировать со вторым порядком точности производную 
$\partial\hat{\varphi}/\partial\hat{y}$ на трехточечном шаблоне с использованием значений 
потенциала~$\hat{\varphi}$~\cite{17-k}. Значения потенциала~$\hat\varphi$ находятся из решения 
уравнения Пуассона. 
      
      Граничную задачу для уравнения Пуассона 
      \begin{align*}
      \fr{\partial^2 \hat\varphi}{\partial \hat{y}^2} & = -\left(\hat{n}_i-\hat{n}_e\right)\,;\\
      \hat{\varphi}\big|_{\hat{y}=0} &=\hat{\varphi}_p\,;\\
      \hat{\varphi}\big|_{\hat{y}_\infty=0} &=0
      \end{align*}
    предлагается решать путем перехода к конечно-разностной системе с последующим ее решением 
методом прогонки~\cite{17-k}:

\noindent
\begin{gather*}
\hat{\varphi}^n_{l-1}+2\hat{\varphi}_l^n+\hat{\varphi}^n_{l+1}=
h_y\hat{\delta}_l^n\,,\enskip l=1,\ldots , 
N_y\,;\\
\hat{\delta}_l^n=-\left( \hat{n}^n_{i,l}-\hat{n}^n_{e,l}\right)\,;\enskip 
\hat{\varphi}_0=\hat{\varphi}_p\,;\enskip \hat{\varphi}_{N_y}=0\,,
\end{gather*}
где $N_y$~--- число шагов по переменной~$\hat{y}$, $h_y$~--- величина шагов разбиения по~$\hat{y}$. 
      
      Концентрации $\hat{n}_\alpha$, $\alpha=i,e$, и плотности токов частиц на зонд~$\hat{f}_\alpha$, 
$\alpha=i,e$, вычисляются согласно описанному выше методу Мон\-те-Карло.

\section{Применение метода расщепления и~метода крупных~частиц}

Решение задачи в данном случае предлагается начать с записи правой части уравнения 
Фок\-ке\-ра--План\-ка в декартовой системе координат в виде:
$$
\mathbf{Q} f_\alpha = \fr{1}{2}\,\fr{\partial^2 f_\alpha}{\partial [v_y]^2}\,\fr{\partial^2 g_\alpha}{\partial 
[v_y]^2}+\fr{\partial f_\alpha}{\partial v_y}\,\fr{\partial C_\alpha}{\partial v_y}+H_\alpha\,,\enskip 
\alpha=i,e\,,
$$  
где 
\begin{align*}
C_\alpha(\vec{r},\vec{v},t)&=
\begin{cases}
\fr{1-\gamma}{Z_i^2}\int\fr{f_e(\vec{r},{\vec{v}}^{\,\prime},t)}{|\vec{v}-{\vec{v}}^{\,\prime} |}\,d{\vec{v}}^{\,\prime}\,, 
&\alpha=i\,;\\[9pt]
\fr{Z_i^2(\gamma-1)}{\gamma}\int \fr{f_i(\vec{r},{\vec{v}}^{\,\prime}, t)}
{|\vec{v}-{\vec{v}}^{\,\prime} 
|}\,d{\vec{v}}^{\,\prime}\,, &\alpha=e\,;
\end{cases} 
\\
H_\alpha&=
\begin{cases}
4\pi \left( \fr{\gamma f_e}{Z_i^2}+f_i\right)f_i\,, & \alpha=i\,;\\[9pt]
4\pi\left(\fr{Z_i^2 f_i}{\gamma}+f_e\right)f_e\,, &\alpha=e\,.
\end{cases}
\end{align*}
Тогда при переходе к безразмерным величинам (см.\ разд.~3) система~(\ref{e1-k}) запишется 
следующим образом:
      \begin{equation}
      \left.
\!\!\begin{array}{l}
      \fr{\partial 
\hat{f}_\alpha}{\partial\hat{t}}+A_\alpha\fr{\partial\hat{f}_\alpha}{\partial\hat{y}}+
B_\alpha  \hat{E}_y
\fr{\partial\hat{f}_\alpha}{\partial\hat{v}_\alpha}=\tilde{\mathbf{Q}}\hat{f}_\alpha\,,\enskip 
\alpha=i,e;\\[9pt]
      \fr{\partial^2\hat{\varphi}}{\partial\hat{y}^2}=-\left( \hat{n}_i-\hat{n}_e\right)\,,\enskip \hat{E}_y=-
\fr{\partial\hat\varphi}{\partial\hat{y}}\,,\\[9pt]
\hspace*{3.1mm}\hat{t}=0:\ \hspace*{2.6mm}\hat{f}_\alpha(\hat{y},\hat{v}_y, 0)=\hat{f}_\alpha^{\mathrm{maksv}}\,,\enskip \alpha=i,e\,,\\[9pt]
\hspace*{2.9mm} \hat{y}=0:\ \hspace*{2.8mm}\hat{f}_\alpha(0,\hat{v}_y,\hat{t})=0\,,\enskip \alpha=i,e\,;\\[9pt]
\hspace*{24.3mm}\hat\varphi(0,\hat{t})=\hat{\varphi}_p\,;\\[9pt]
      \hat{y}=\hat{y}_\infty:\ \hat{f}_\alpha(\hat{y}_\infty, 
\hat{v}_y,\hat{t})=\hat{f}_\alpha^{\mathrm{maksv}}\,,\enskip \alpha=i,e\,;\\[9pt]
\hspace*{21.5mm}\hat{\varphi}(\hat{y}_\infty,\hat{t})=0\,,\\[9pt]
    \end{array}
\right\}\!\!
\label{e7-k}
\end{equation}
где 
\begin{gather*}
\tilde{\mathbf{Q}} \hat{f}_\alpha=D_\alpha\fr{\partial^2\hat{f}_\alpha}{\partial 
[\hat{v}_y]^2}+K_\alpha\fr{\partial\hat{f}_\alpha}{\partial\hat{v}_y}+H_\alpha\,;\\
D_\alpha=A_g^\alpha\fr{\partial^2\hat{g}_\alpha}{\partial [\hat{v}_y]^2}\,;\enskip 
K_\alpha=A_h^\alpha \fr{\partial \hat{h}_\alpha}{\partial\hat{v}_y}\,,\ \alpha=i,e\,.
\end{gather*}

Для решения системы уравнений~(\ref{e7-k}) применяется модификация метода 
расщепления~\cite{17-k}, согласно которой исходная задача разбивается на две вспомогательные. Такое 
разбиение можно осуществить, переписав уравнение Фок\-ке\-ра--План\-ка в следующем виде:
$$
\fr{\partial\hat{f}_\alpha}{\partial\hat{t}} =
\tilde{\mathbf{Q}}_1\hat{f}_\alpha+\tilde{\mathbf{Q}}_2\hat{f}_\alpha\,,
$$
где 
\begin{align*}
\tilde{\mathbf{Q}}_1\hat{f}_\alpha &=-
\left(A_\alpha\fr{\partial\hat{f}_\alpha}{\partial\hat{y}}+
B_\alpha\fr{\partial\hat{f}_\alpha}{\partial\hat{y}}
\right)\,;\\
\tilde{\mathbf{Q}}_2\hat{f}_\alpha 
&=\left(D_\alpha\fr{\partial^2\hat{f}_\alpha}{\partial[\hat{v}_y]^2}+K_\alpha\fr{\partial 
\hat{f}_\alpha}{\partial\hat{v}_y}+H_\alpha\right)\,.
\end{align*}

      Правая часть уравнения Фок\-ке\-ра--План\-ка представляет собой сумму двух операторов, 
первый из которых отвечает за перенос частиц, второй~--- за столкновения заряженных частиц. 
В~результате образуются следующие задачи, которые решаются последовательно:
      \begin{itemize}
\item первая задача:
\begin{align*}
&\fr{\partial w_\alpha(\hat{y},\hat{v}_y,\hat{t})}{\partial\hat{t}} =\mathbf{Q}_1 
w_\alpha(\hat{y},\hat{v}_y,\hat{t})\,,\enskip \alpha=i,e\,;\\[9pt]
&\fr{\partial^2\hat\varphi}{\partial\hat{y}^2}=-\left(\hat{n}_i-\hat{n}_e\right)\,;\enskip
\hat{E}_y=-
\fr{\partial\hat\varphi}{\partial\hat{y}}\,;\\[9pt]
&w_\alpha(\hat{y},\hat{v}_y,\hat{t}^n)=\hat{f}_\alpha(\hat{y},\hat{v}_y,\hat{t}^n)\,,\enskip n=0,\ldots ,N-
1\,;\\[9pt]
&\hspace{2.9mm}\hat{y}=0:\ \hspace*{2.9mm}w_\alpha(0,\hat{v}_y,\hat{t})=0\,,\enskip \alpha=i,e\,;\\[9pt]
&\hspace*{25.1mm}\hat\varphi(0,\hat{t})=\hat{\varphi}_p\,;\\[9pt]
&\hat{y}=\hat{y}_\infty:\ w_\alpha(\hat{y}_\infty, \hat{v}_y, \hat{t})=
\hat{f}_\alpha^{\mathrm{maksv}}\,,\enskip 
\alpha=i,e\,;\\[9pt]
&\hspace*{22.5mm}\hat\varphi(\hat{y}_\infty,\hat{t})=0\,;
\end{align*}
\item вторая задача:
\begin{align*}
\!\!\!\!\!\!\!\fr{\partial s_\alpha(\hat{y},\hat{v}_y,\hat{t})}{\partial \hat{t}} &=\mathbf{Q}_2 
s_\alpha(\hat{y},\hat{v}_y,\hat{t})\,, & \alpha&=i,e\,;\\
\!\!\!\!\!\!\!s_\alpha (\hat{y},\hat{v}_y,\hat{t}^n) &=w_\alpha (\hat{y},\hat{v}_y, \hat{t}^{n+1}),& n&=0,\ldots ,N-
1.
\end{align*}
\end{itemize}

Первая задача представляет собой систему безразмерных уравнений Вла\-со\-ва--Пуас\-со\-на. Для ее 
решения применяется метод крупных частиц~\cite{18-k}. Согласно этому методу решение задачи 
осуществляется путем расщепления на два этапа: на первом этапе не учитываются конвективные члены 
и решение получается обычным интегрированием на неподвижной эйлеровой сетке, а на втором этапе 
рассматривается система, которая описывает перенос частиц в лагранжевой системе координат. Кроме 
того, на первом этапе необходимо решить уравнение Пуассона для получения значений потенциала 
самосогласованного электрического поля. Для этого применяется метод, описанный в разд.~3. 

Вторая задача решается путем перехода к ко\-неч\-но-раз\-ност\-ной сис\-те\-ме. При этом частные 
производные $\partial^2\hat{g}_\alpha/\partial[\hat{v}_y]^2$ и $\partial\hat{h}_\alpha/\partial\hat{v}_y$ 
аппроксимируются со вторым порядком точности с использованием трехточечного шаблона, а 
производная $\partial s_\alpha/\partial\hat{t}$ аппроксимируется на двухточечном шаблоне с первым 
порядком точности~\cite{16-k}. К~полученной системе разностных уравнений предлагается применить 
один из классических методов решения систем линейных уравнений, например метод 
Гаусса~\cite{19-k}.
      
      Решением первой задачи является функция $w_\alpha(\hat{y}, \hat{v}_y, \hat{t}^n)$, 
$n\hm=0,\ldots ,N$, , которая дает начальное условие для второй задачи. Решая вторую задачу, находим 
функцию $s_\alpha(\hat{y},\hat{v}_y,\hat{t}^n)\hm=\hat{f}_\alpha(\hat{y},\hat{v}_y,\hat{t}^n)$, 
$n=1,\ldots ,N$, $\alpha=i,e$, которая определяет решение $\hat{f}_\alpha(\hat{y},\hat{v}_y,\hat{t}^n)$, 
$\alpha=i,e$, исходной системы~(\ref{e7-k}) для рассматриваемых моментов времени $n=1,\ldots ,N$.

Моменты функций распределения $\hat{f}_\alpha$, $\alpha=i,e$, находятся с помощью методов 
численного интегрирования, например метода трапеций~\cite{19-k}.

\section{Результаты численного моделирования}

Для двух описанных выше методов реализованы две отдельные программы в среде {Matlab~7.0}. 
Эти программы позволяют по заданным значениям концентраций и температур частиц $n_{i\infty}$, 
$n_{e\infty}$, $T_{i\infty}$ и~$T_{e\infty}$ в невозмущенной плазме, а также потенциала~$\varphi_p$, 
подаваемого на зонд, изучить эволюцию во времени плотностей тока частиц~$j_i$ и~$j_e$, концентраций 
частиц~$n_i$  и~$n_e$ в произвольной точке пространства в возмущенной зоне, а также динамику 
изменения напряженности~$E_y$ самосогласованного электрического поля во времени и пространстве.

С использованием разработанных программ проведены серии расчетных экспериментов, в которых 
значение концентраций варьировалось в пределах $n_{i\infty} \hm = n_{e\infty}\hm =10^{18}\div 
10^{22}$~м$^{-3}$. Значение температур было выбрано неизменным и равным $T_{i\infty}\hm = 
T_{e\infty}\hm=3000$~K, а значения потенциала, подаваемого на зонд, изменялись в пределах 
$\varphi_p\hm=0\div 2{,}6$~В.

На рис.~1  и~2 приведены графики изменения напряженности самосогласованного электрического
 поля (см.\ рис.~1) и плотности токов ионов (см.\linebreak\vspace*{-12pt}

\pagebreak

\end{multicols}

\begin{figure} %fig1
\vspace*{1pt}
\begin{center}
\mbox{%
\epsfxsize=162.594mm
\epsfbox{kud-1.eps}
}
\end{center}
\vspace*{-9pt}
\Caption{Динамика изменения плотности тока ионов во времени в фиксированной точке возмущенной 
зоны для значений потенциала: \textit{1}~--- $\varphi_p=-6$; 
\textit{2}~--- $\varphi_p=-16$; \textit{3}~--- $\varphi_p=- 30$ 
в случае применения методов Монте-Карло~(\textit{а}) 
и крупных частиц~(\textit{б})}
\end{figure}

\begin{figure} %fig2
\vspace*{1pt}
\begin{center}
\mbox{%
\epsfxsize=162.713mm
\epsfbox{kud-2.eps}
}
\end{center}
\vspace*{-9pt}
\Caption{Динамика изменения напряженности электрического поля во времени в фиксированной точке 
возмущенной зоны для значений потенциала: 
\textit{1}~--- $\varphi_p=-6$; \textit{2}~--- $\varphi_p=-16$; 
\textit{3}~--- $\varphi_p=-30$ в случае применения методов Монте-Карло~(\textit{а}) и
крупных частиц~(\textit{б})
}
\end{figure}

\begin{multicols}{2}

\noindent
 рис.~2) во времени в фиксированной точке пространства 
возмущенной зоны в случае применения обоих разработанных алгоритмов.


На основании полученных результатов можно отметить похожее поведение зависимостей 
напряженности электрического поля и плотности тока от времени в двух рассматриваемых случаях. 
Графики кривых сначала убывают, затем начинают возрастать, выходя в некоторый момент 
времени~$t^\prime$ (момент установления) на стационарные значения. 

Одинаковое поведение 
напряженности и плот\-ности тока можно объяснить из следующих соображений: плотность тока ионов в 
данной области пространства равна произведению концентрации ионов на их направленную скорость и 
на заряд иона. Скорость ионов, в свою очередь, зависит от заряда, массы и напряженности 
электрического поля. 
%\columnbreak

При внесении в плазму отрицательно заряженного зонда возникает электрическое поле, которое 
нарушает квазинейтральность плазмы. Для того чтобы компенсировать действие внешнего 
электрического поля, ионы устремляются к зонду, а электроны~--- от зонда. Это приводит к дисбалансу 
концентраций вблизи зонда и, как следствие, к увеличению разности потенциалов; график 
напряженности электрического поля убывает. Вскоре разделение зарядов компенсирует внешнее 
электрическое поле; график выходит на стационарное значение. 

Также можно отметить, что значения 
напряженности электрического поля и плотности тока частиц на зонд в момент установления для двух 
методов совпадают. 

Момент установления~$t^\prime$ зависит от при\-ме\-ня\-емо\-го метода решения. В~случае метода 
Мон\-те-Кар\-ло $t^\prime=3{,}5\div 4$~ед., а для метода крупных частиц совместно с методом 
расщепления $t^\prime\hm=5\div 5{,}5$~ед. Используя ко\-неч\-но-раз\-ност\-ный метод, можно 
получить динамику изменения функций распределения частиц~$f_\alpha$, $\alpha=i,e$, во времени и 
пространстве. Функции распределения позволяют наглядно представить влияние на картину 
распределения частиц вблизи зонда самой поверхности зонда и электрического поля.

\section{Заключение}
      
      В работе найдено решение задачи диагностики плоским зондом сильноионизованной плазмы с 
учетом столкновений заряженных частиц. Разработана математическая модель исследуемого явления, 
описываемая уравнениями Фок\-ке\-ра--План\-ка и Пуассона. Решение получено двумя методами:\linebreak 
статистическим и ко\-неч\-но-раз\-ност\-ным на основе\linebreak сформированных алгоритмов. Приведены 
резуль-\linebreak таты численного моделирования при различных\linebreak характерных параметрах задачи.
 Из  проведенных 
вычислительных экспериментов вытекает, что искомые величины: напряженность 
электрического поля, плотности токов частиц на зонд, концентрации частиц вблизи зонда~--- как по 
характеру зависимости, так и по числовым значениям совпадают. При применении метода 
      Мон\-те-Кар\-ло момент установления наступает быстрее по сравнению с конечно-разностным 
методом, однако конечно-разностный метод позволяет получить более наглядные результаты.

{\small\frenchspacing
{%\baselineskip=10.8pt
\addcontentsline{toc}{section}{Литература}
\begin{thebibliography}{99}

\bibitem{1-k}
\Au{Alexeff I., Anderson T.}
Experimental and theoretical results with plasma antenna~// IEEE Trans. Plasma Sci., 2006. Vol.~34. 
No.\,2. P.~166--172.

\bibitem{2-k}
\Au{Сысун В.\,И.}
Сильноионизованная низкотемпературная плазма в приборах электронной техники: Методы 
исследования, свойства, применение. Дисс. \ldots д-ра физ.-мат. наук в форме науч. докл.: 
01.04.08.~--- Пет\-ро\-за\-водск, 1996.

\bibitem{3-k}
\Au{Тухас В.\,А.}
Методология создания средств измерений и испытаний на устойчивость к кондуктивным помехам~// 
Мат-лы VI Междунар. симп. по электромагнитной совместимости и 
электромагнитной экологии.~--- СПб., 2005. С.~231--234.

\bibitem{4-k}
\Au{Гудзенко Л.\,И., Яковленко С.\,И.}
Плазменные лазеры.~--- М.: Атомиздат, 1978.  256~с.

\bibitem{5-k}
\Au{Звелто О.}
Принципы лазеров.~--- М.: Мир, 1990.  560~с.

\bibitem{6-k}
\Au{Сысун В.\,И., Хромой Ю.\,Д.}
Расширение канала мощного импульсного разряда в парах ртути~// Электронная техника, 1974. 
Сер.~4. Вып.~10. С.~80--85. 

\bibitem{7-k}
\Au{Винклер Дж.\,Р.}
Искусственные пучки частиц в космической плазме.~--- М.: Мир, 1985.  451~с.

\bibitem{8-k}
\Au{Bernstein I.\,B., Rabinowitz I.\,N.}
Theory of electrostatic probes in low-density plasma~// Phys. Fluids, 1959. Vol.~2. No.\,2. P.~112--121. 

\bibitem{9-k}
\Au{Альперт Я.\,Л., Гуревич А.\,В., Питаевский~Л.\,П.}
Искусственные спутники в разреженной плазме.~--- М.: Наука, 1964.  282~с.

\bibitem{10-k}
\Au{Чан П., Тэлбот Л., Турян~К.}
Электрические зонды в неподвижной и движущейся плазме.~--- М.: Мир, 1978.  202~с.

\bibitem{11-k}
\Au{Алексеев Б.\,В., Котельников В.\,А.}
Зондовый метод диагностики плазмы.~--- М.: Энергоатомиздат, 1989.  240~с.

\bibitem{12-k}
\Au{Пантелеев А.\,В., Кудрявцева И.\,А.}
Формирование математической модели двухкомпонентной плазмы с учетом столкновений 
заряженных частиц в случае плоского зонда~// Теоретические вопросы вычислительной техники и 
программного обеспечения: Межвузовский сб. научн. тр.~--- М.: МИРЭА, 2006. С.~11--21.

\bibitem{13-k}
\Au{Олдер Б.}
Вычислительные методы в физике плазмы.~--- М.: Мир, 1974.  111~с.

\bibitem{14-k}
\Au{Montgomery D.\,C., Tidman D.\,A.}
Plasma kinetic theory.~--- New York, 1964. 

\bibitem{15-k}
\Au{Кудрявцева И.\,А., Пантелеев А.\,В.}
Применение метода Мон\-те-Кар\-ло для анализа поведения двухкомпонентной плазмы с учетом 
столкновений между заряженными частицами~// Теоретические вопросы\linebreak
вычислительной техники и 
программного обеспечения: Межвузовский сб. научн. тр.~--- М.: МИРЭА, 2008. С.~122--128. 

\bibitem{16-k}
\Au{Семенов В.\,В., Пантелеев А.\,В., Руденко~Е.\,А., Бор\-та\-ков\-ский~А.\,С.}
Методы описания, анализа и синтеза нелинейных систем управления.~--- М.: МАИ, 1993.  312~с.

\bibitem{17-k}
\Au{Киреев В.\,И., Пантелеев А.\,В.}
Численные методы в примерах и задачах.~--- М.: Высшая школа, 2006.  480~с.

\bibitem{18-k}
\Au{Белоцерковский О.\,М., Давыдов~Ю.\,М.}
Метод крупных частиц в газовой динамике. Вычислительный эксперимент.~--- М.: Наука, 
Физматгиз, 1982.

\label{end\stat}

\bibitem{19-k}
\Au{Вержбицкий В.\,М.}
Основы численных методов.~--- М.: Высшая школа, 2002.  840~с.
 \end{thebibliography}
}
}


\end{multicols}         %10
\def\stat{koles+list}

\def\tit{ПРОТОКОЛ ГЕТЕРОГЕННОГО МЫШЛЕНИЯ ГИБРИДНОЙ ИНТЕЛЛЕКТУАЛЬНОЙ  
МНОГОАГЕНТНОЙ СИСТЕМЫ ДЛЯ~РЕШЕНИЯ ПРОБЛЕМЫ ВОССТАНОВЛЕНИЯ  
РАСПРЕДЕЛИТЕЛЬНОЙ ЭЛЕКТРОСЕТИ$^*$}

\def\titkol{Протокол гетерогенного мышления гибридной интеллектуальной  
многоагентной системы} % для решения проблемы восстановления   распределительной электросети}

\def\aut{А.\,В.~Колесников$^1$, С.\,В.~Листопад$^2$}

\def\autkol{А.\,В.~Колесников, С.\,В.~Листопад}

\titel{\tit}{\aut}{\autkol}{\titkol}

\index{Колесников А.\,В.}
\index{Листопад С.\,В.}
\index{Kolesnikov A.\,V.}
\index{Listopad S.\,V.}


{\renewcommand{\thefootnote}{\fnsymbol{footnote}} \footnotetext[1]
{Работа выполнена при поддержке РФФИ (проект 18-07-00448а).}}


\renewcommand{\thefootnote}{\arabic{footnote}}
\footnotetext[1]{Балтийский федеральный университет им.\ И.~Канта, avkolesnikov@yandex.ru}
\footnotetext[2]{Калининградский филиал Федерального исследовательского центра <<Информатика 
и~управление>> Российской академии наук, \mbox{ser-list-post@yandex.ru}}

\vspace*{-12pt}

   
   \Abst{Для проблемы восстановления электроснабжения в~региональной 
распределительной электросети после масштабных аварий характерны высокая 
комбинаторная слож\-ность, неоднородность, недоопределенность, неточность и~нечеткость. 
Применение механизма коллективного решения проблем для преодоления перечисленных 
НЕ-фак\-то\-ров невозможно из-за временных ограничений. Для решения проблемы 
предлагается новый класс интеллектуальных систем, моделирующих коллективное принятие 
решений под руководством фасилитатора~--- гибридные интеллектуальные многоагентные 
сис\-те\-мы гетерогенного мышления (ГИМСГМ). В~отличие от традиционных гибридных 
интеллектуальных сис\-тем, интегрирующих модели знаний экспертов, они дополнительно 
моделируют взаимодействие в~групповых процессах и~эффектах, адап\-ти\-ру\-ясь к~динамичным 
проблемным ситуациям послеаварийного восстановления региональной распределительной 
электросети. В~работе рас\-смат\-ри\-ва\-ет\-ся один из компонентов таких сис\-тем~--- протокол 
организации коллективного гетерогенного мыш\-ле\-ния агентов.} 
  
  \KW{гетерогенное мышление; гибридная интеллектуальная многоагентная система; 
проблема восстановления распределительной электросети}

\DOI{10.14357/19922264190211}
  
\vspace*{-6pt}


\vskip 10pt plus 9pt minus 6pt

\thispagestyle{headings}

\begin{multicols}{2}

\label{st\stat}
  
\section{Введение}

\vspace*{-2pt}

  При авариях в~энергосистемах, происходящих несмотря на проводимые 
предупредительно-про\-фи\-лак\-ти\-че\-ские мероприятия, принимаются меры, 
снижающие их интенсивность и~продолжительность. Сразу после масштабной 
аварии повышение частоты и~развал энер\-го\-сис\-те\-мы тормозятся автоматически 
изменением и~отключением нагрузки управляемым разделением. В~течение 
по\-сле\-ду\-ющих часов выполняются восстановительные мероприятия, в~ходе 
которых операторы распределительной сети и~электростанций вручную 
поддерживают баланс нагрузки и~генерации. Длительность выработки решений 
и~действий по их реализации намного превышает допустимые ограничения по 
оборудованию, и~возникает опас\-ность, что управ\-ле\-ние электростанциями 
и~энер\-го\-сис\-те\-мой не может обеспечить необходимую координацию~[1]. 

Одна 
из важнейших задач опе\-ра\-тив\-но-дис\-пет\-чер\-ско\-го персонала~--- 
составление плана обеспечения безопасности, сохранности оборудования 
и~быст\-ро\-го восстановления электроснабжения потребителей, не 
противоречащего требованиям энер\-го\-сис\-те\-мы~[2]. 
  
  При разработке плана оперируют большим разнообразием ресурсов, их 
свойств и~отношений, а~так\-же учитывают НЕ-фак\-то\-ры 
А.\,С.~Нариньяни~[3]: недоопределенность места аварии на момент 
планирования; не\-точ\-ность величины мощ\-ности, по\-треб\-ля\-емой клиентами 
и~генерируемой источниками распределенной генерации; не\-чет\-кость времени 
восстановительных операций; не\-кор\-рект\-ность работы датчиков; неполнота 
модели электросети. В~силу временн$\acute{\mbox{ы}}$х ограничений организовать 
всестороннее коллективное обсуж\-де\-ние послеаварийной проб\-лем\-ной ситуации 
не\-воз\-можно. 

\begin{figure*}[b] %fig1
\vspace*{6pt}
    \begin{center}  
  \mbox{%
 \epsfxsize=123.206mm 
 \epsfbox{kol-1.eps}
 }
 \end{center}
\vspace*{-7pt}
\Caption{Модель коллективного решения проблем методами гетерогенного мышления~--- 
ромб группового принятия решений С. Кейнера, К. Толди, С. Фиск, Д. Бергера: \textit{1}~--- 
альтернатива; \textit{2}~--- досрочное несогласованное решение; \textit{3}~--- согласованное 
решение}
\end{figure*}
  
  Для информационной подготовки решений (<<предрешений>>, по 
П.\,К.~Анохину) по восстановлению электроснабжения в~распределительной 
электросети~[4] предлагается разработать новый класс интеллектуальных 
систем~--- \mbox{ГИМСГМ}, комбинирующих гиб\-рид\-ный подход 
А.\,В.~Колесникова~\cite{5-kol}, аппарат многоагентных систем В.\,Б.~Тарасова~\cite{6-kol} 
и~методики гетерогенного мышления~\cite{7-kol, 8-kol, 9-kol}. Такие системы интегрируют 
знания и~взаимодействие экспертов различных специальностей, учитывают 
несколько критериев оптимальности и~множество ограничений в~условиях 
динамиче-\linebreak\vspace*{-12pt}

\pagebreak

\noindent
ских сред и~дефицита времени на принятие ре\-шения.
{ %\looseness=1

} 
  
\section{Формализованная модель гибридной интеллектуальной  
многоагентной системы гетерогенного мышления}

  Формально ГИМСГМ определяется следующим образом~\cite{10-kol}:
\begin{align*}
&\mathrm{himsht}=\langle \mathrm{AG}^*, \mathrm{env}, \mathrm{INT}^*, 
\mathrm{ORG}, \{\mathrm{ht}\}\rangle\,;\\
&\mathrm{act}_{\mathrm{himsht}}={}\\
&= \Bigg( \mathop{\bigcup}\limits_{\mathrm{ag}\in \mathrm{AG}^*} 
\mathrm{act}_{\mathrm{ag}}\Bigg)  \cup \mathrm{act}_{\mathrm{dmsa}} \cup 
\mathrm{act}_{\mathrm{htmc}}\cup{} \mathrm{act}_{\mathrm{col}}\,;\\
&\mathrm{act}_{\mathrm{ag}}= \left( \mathrm{MET}_{\mathrm{ag}}, 
\mathrm{IT}_{\mathrm{ag}}\right), \mathrm{ag}\in \mathrm{AG}^*, \Bigg\vert\! 
\mathop{\bigcup}\limits_{\mathrm{ag}\in \mathrm{AG}^*} \mathrm{IT}_{\mathrm{ag}}\Bigg\vert \geq 2,
\end{align*}
где $\mathrm{AG}^*=\{ \mathrm{ag}_1, \ldots , \mathrm{ag}_n, \mathrm{ag}^{\mathrm{dm}}, 
\mathrm{ag}^{\mathrm{fc}}\}$~--- множество агентов, 
включающее аген\-тов-экс\-пер\-тов (АЭ)~$\mathrm{ag}_i$, $i\hm\in \mathbb{N}$, 
$1\hm\leq i\hm\leq n$, агента, принимающего решения (АПР), $\mathrm{ag}^{\mathrm{dm}}$ 
и~аген\-та-фа\-си\-ли\-та\-то\-ра
 (АФ) $\mathrm{ag}^{\mathrm{fc}}$, $n$~--- число АЭ;  
$\mathrm{env}$~--- концептуальная модель внешней среды \mbox{ГИМСГМ}; 
$\mathrm{INT}^*\hm=  \{\mathrm{prot}_{\mathrm{gm}},$\linebreak
$ \mathrm{lang}, 
\mathrm{ont}, \mathrm{dmscl}\}$~--- элементы структурирования взаимодействий 
агентов: $\mathrm{prot}_{\mathrm{gm}}$~--- протокол взаимодействия агентов, позволяющий 
организовать их коллективное гетерогенное мышление, $\mathrm{lang}$~--- язык 
передачи сообщений, $\mathrm{ont}$~--- модель предметной об\-ласти, $\mathrm{dmscl}$~--- 
классификатор ситуаций коллективного решения проб\-ле\-мы, 
идентифицирующий стадии этого процесса; $\mathrm{ORG}$~--- множество архитектур 
\mbox{ГИМСГМ}; $\{\mathrm{ht}\}$~--- множество концептуальных моделей макроуровневых 
процессов в~\mbox{ГИМСГМ}: $\mathrm{ht}$~--- модель процесса коллективного решения 
проб\-лем методами гетерогенного мышления~--- ромб группового принятия 
решений С.~Кейнера, К.~Толди, С.~Фиск, Д.~Бергера (рис.~1)~\cite{8-kol}; 
$\mathrm{act}_{\mathrm{himsht}}$~--- функция \mbox{ГИМСГМ} в~целом; 
$\mathrm{act}_{\mathrm{ag}}$~--- функция АЭ из\linebreak 
мно\-жества~$\mathrm{AG}^*$; $\mathrm{act}_{\mathrm{dmsa}}$~--- 
функция <<анализ ситуации 
коллективного решения проб\-ле\-мы>> АФ,\linebreak
 обеспечивающая идентификацию 
стадии процесса гетерогенного мышления \mbox{ГИМСГМ} на основе предлагаемых 
АЭ частных решений, на\-пря\-жен\-ности конфликта между АЭ и~предшествующей\linebreak 
стадии процесса решения проблемы; $\mathrm{act}_{\mathrm{htmc}}$~--- 
функция <<выбор метода 
гетерогенного мышления>> АФ, которая реализуется с~использованием 
нечеткой базы знаний об эффективности методов гетерогенного мышления 
в~зависимости от характеристик проб\-ле\-мы, стадии процесса ее решения 
и~текущей ситуации решения в~\mbox{ГИМСГМ}; $\mathrm{act}_{\mathrm{col}}$~--- 
коллективная 
функция \mbox{ГИМСГМ}, конструируемая динамически; 
$\mathrm{met}_{\mathrm{ag}}$~--- метод 
решения задачи; $\mathrm{it}_{\mathrm{ag}}$~--- 
интеллектуальная технология, в~рамках которой\linebreak 
реализован метод~$\mathrm{met}_{\mathrm{ag}}$.


  Согласно модели, пред\-став\-лен\-ной на рис.~1, процесс решения проб\-ле\-мы 
\mbox{ГИМСГМ} трехстадийный: (1)~дивергентное мышление; (2)~бурление; 
(3)~конвергентное мышление. На стадии дивергентного мышления  
АЭ генерируют множество вариантов решения проблемы, 
а~АФ стимулирует их выработку соответствующими 
методами~\cite{11-kol}. В~случае если даже с~применением методов дивергентного 
мышления противоречий не возникает, т.\,е.\ задача имеет очевидное решение, 
процесс завершается. В~противном случае агенты ГИМСГМ конфликтуют по 
поводу знаний, убеждений, мнений, т.\,е.\ участвуют в~своего рода когнитивных 
конфликтах~\cite{11-kol, 12-kol}. Конфликт~--- отличительная черта стадии бурления, 
позволяющая АФ предпринять меры по сближению 
точек зрения агентов. На стадии конвергентного мыш\-ле\-ния агенты совместно 
переформулируют, дорабатывают решения, пока не получат коллективное 
решение, релевантное разнообразию моделей экспертов \mbox{ГИМСГМ}. 
  
  Функциональная структура ГИМСГМ для решения проблемы 
восстановления электроснабжения в~региональной распределительной 
электросети после масштабных аварий представлена в~[12]. Рассмотрим 
протокол взаимодействия агентов ГИМСГМ в~процессе коллективного 
гетерогенного мышления.
  
\section{Протокол гетерогенного мышления гибридной 
интеллектуальной многоагентной системы}
 
  Основная цель протоколирования агентов гиб\-рид\-ной интеллектуальной 
многоагентной системы~---инкапсуляция разрешенных взаимодействий. Протокол 
определяет схемы (распределенный алгоритм) обмена информацией, знаниями, 
координации агентов при решении по\-став\-лен\-ных задач~\cite{13-kol}. С~одной стороны, 
протокол служит для объединения агентов через концептуальный интерфейс 
и~организации их совместной работы, с~другой~--- определяет четкие границы 
компонентов сис\-те\-мы~\cite{14-kol}. При описании протокола должны быть 
специфицированы: роли агентов; типы сообщений между парами ролей; 
семантика каждого типа сообщения декларативно; любые дополнительные 
ограничения на сообщения, такие как порядок их следования или правила 
передачи информации из одного сообщения в~другое. Такая спецификация 
протокола однозначно определяет, удовлетворяет ли конкретная реализация 
взаимодействия агентов указанному протоколу и~совместим ли конкретный 
агент с~\mbox{ГИМСГМ}~\cite{14-kol}.



  
  К настоящему времени разработано множество протоколов многоагентных 
сис\-тем как общего назначения, так и~для решения конкретных задач. Выделим 
наиболее известные классы протоколов~\cite{13-kol, 15-kol}: 
\begin{enumerate}[(1)]
\item на основе контрактной 
сети~\cite{16-kol, 17-kol} для автоматического планирования взаимодействия агентов 
и~минимизации за\-трат посредством метафоры переговоров агентов на 
рыночных торгах; 
\item на основе теории речевых актов~\cite{18-kol}, когда переговоры 
строятся с~использованием небольшого чис\-ла примитивов, по\-сред\-ст\-вом обмена 
которыми агенты об\-суж\-да\-ют некоторую тему, обновляют свои базы знаний, 
обмениваются <<мнениями>> и~приходят к~общему решению~\cite{13-kol}; 
\item переговорные~\cite{19-kol, 20-kol}, пред\-ла\-га\-ющие механизмы разрешения конфликтов 
для повышения суммарной по\-лез\-ности, до\-сти\-га\-емой аген\-тами; 
\item на основе 
доски объявлений, когда выделяется общая область памяти для взаимодействия 
агентов.
\end{enumerate} 
  
  Предлагаемый протокол организации коллективного гетерогенного 
мышления агентов пред\-назначен для моделирования работы малого кол\-лектива 
экспертов за круглым столом, где нет\linebreak необходимости сбивать цену за услуги 
или подбирать экспертов, поэтому он основывается на теории речевых актов. 
Схема работы гибридной интеллектуальной многоагентной сис\-те\-мы по 
протоколу гетерогенного мыш\-ле\-ния представлена на рис.~2.
  



  Как показано на рис.~2, стандартный протокол речевых актов~\cite{18-kol} расширен 
сле\-ду\-ющи\-ми типами сообщений: request-ch-tt, commit-ch-tt, request-start-ps, 
request-stop-ps, request-task, report-decision, используемых для организации 
взаимодействия АФ, АПР и~АЭ. Взаимодействуют АЭ между собой 
и~с~другими агентами по протоколу речевых актов~\cite{18-kol}.
  
  По предлагаемому протоколу процесс коллективного гетерогенного 
мышления начинается\linebreak с~отправки агентом-фа\-си\-ли\-та\-то\-ром АПР и~АЭ\linebreak 
сообщения типа request-ch-tt, в~теле которого указывается метод гетерогенного 
мышления, при\-ме\-ня\-емый на данном этапе работы \mbox{ГИМСГМ}. Агент-фа\-си\-ли\-та\-тор 
приостанавливает свою работу\linebreak
 в~ожидании от\-ве\-тов-под\-тверж\-де\-ний от АЭ 
и~АПР. Получив от АФ сообщение request-ch-tt, АПР и~АЭ выбирают 
соответствующий алгоритм и~переходят в~режим ожидания сигнала на начало 
решения проб\-ле\-мы в~соответствии с~данным алгоритмом. Для подтверждения 
получения сообщения и~го\-тов\-ности к~работе по установленному алгоритму 
АПР и~АЭ отправляют АФ от\-вет-под\-тверж\-де\-ние commit-ch-tt. Дождавшись 
под\-тверж\-де\-ний от АПР и~всех АЭ, АФ отправляет АПР сообщение  
request-start-ps, сиг\-на\-ли\-зи\-ру\-ющее о~том, что все агенты перешли на 
соответствующий метод гетерогенного мышления, а~сис\-те\-ма готова 
к~дальнейшей работе, и~ожидает решений от~АЭ. 
  
  Получив сообщение request-start-ps, АПР формирует и~рассылает задачи АЭ 
с~использованием сообщения типа request-task, в~теле которого так\-же 
описываются исходные данные задачи, после чего он при\-сту\-па\-ет к~сбору 
решений, поступающих от АЭ, и~работает с~ними в~соответствии 
с~установленным ранее алгоритмом гетерогенного мыш\-ле\-ния.\linebreak\vspace*{-12pt}

\pagebreak

\end{multicols}

\begin{figure*} %fig2
\vspace*{1pt}
    \begin{center}  
  \mbox{%
 \epsfxsize=160.791mm 
 \epsfbox{kol-2.eps}
 }
 \end{center}
\vspace*{-4pt}
\Caption{Схема работы гибридной интеллектуальной многоагентной сис\-те\-мы по протоколу  
гетерогенного мыш\-ле\-ния: \textit{1}~--- участники коллектива; \textit{2}~--- 
шкала времени; 
\textit{3}~--- действие; \textit{4}~--- процесс передачи информации между участниками 
коллектива; $t$~--- модельное время}
\vspace*{3pt}
\end{figure*}

\begin{multicols}{2}

\noindent
 Агенты-экс\-пер\-ты после 
получения задания от АПР начинают решать поставленные задачи 
в~соответствии с~установленным алгоритмом гетерогенного мышления. При 
этом в~за\-ви\-си\-мости от алгоритма они могут генерировать несколько решений 
по\-став\-лен\-ной задачи. Все полученные альтернативы АЭ отправляют в~теле 
сообщения типа report-decision одновременно АПР и~АФ. 
  
  При получении очередного решения от АЭ агент-фа\-си\-ли\-та\-тор 
запускает функцию <<анализ ситуации коллективного решения проблемы>> 
$\mathrm{act}_{\mathrm{dmsa}}$, определяет с~ее помощью напряженность конфликта в~каждой 
паре АЭ и~в ГИМСГМ в~целом~\cite{4-kol}. При достижении определенного 
уровня напряженности конфликта в~соответствии с~нечеткой базой знаний АФ 
запускает функцию <<выбор метода гетерогенного мышления>>~$\mathrm{act}_{\mathrm{htmc}}$, 
позволяющую ему выбрать релевантный ситуации коллективного принятия 
решений метод гетерогенного мышления. 

Для реализации данной функции АФ 
имеет\linebreak нечеткую базу знаний о~релевантности <<стилей мышления>> агентов 
различным ситуациям принятия решений в~ГИМСГМ, а~методов~--- 
различным особенностям проблем и~стадиям коллективного принятия решений. 
Для формирования такой базы знаний необходимо провести серию 
вычислительных экспериментов по решению проблем из различных 
классов~\cite{5-kol} и~установить соответствие между классом проблем 
и~релевантными подходами к~организации гетерогенного мышления. Выбрав 
релевантный метод, АФ отправляет АЭ и~АПР сообщения типа request-ch-tt, 
в~теле которых указан метод гетерогенного мышления для АЭ и~АПР. 
Дожидается подтверждений от АПР и~всех АЭ, после чего отправляет АЭ 
сообщение request-start-ps, сигнализирующее о том, что все агенты перешли на 
соответствующий метод гетерогенного мышления и~АЭ могут продолжить 
работу. Агент-фа\-си\-ли\-та\-тор начинает прием решений от АЭ,
 и~процесс анализа ситуации~--- 
выбора метода гетерогенного мышления~--- повторяется до тех пор, пока 
в~\mbox{ГИМСГМ} не завершится стадия конвергентного мышления (см.\ рис.~1). После 
ее завершения АФ отправляет сигнал request-stop-ps к~окончанию процесса 
решения проблемы АПР и~АЭ и~завершает свою работу. Получив такой сигнал, 
АЭ прерывают выполнение заданий и~завершают работу, а~АПР принимает 
окончательное коллективное решение в~соответствии с~уста\-нов\-лен\-ным 
алгоритмом гетерогенного мыш\-ле\-ния, например решение, по которому был 
достигнут консенсус или отдано большинство голосов агентов на стадии 
конвергентного мыш\-ле\-ния. Далее передает это решение интерфейсному 
агенту и~так\-же завершает работу.
  
  Число <<переключений>> методов мышления протоколом заранее не 
определено, так как стадия бурления может отсутствовать, а~на стадиях 
дивергентного и~конвергентного мыш\-ле\-ния последовательно могут 
применяться различные методы. Таким образом, за счет наличия у~АФ 
нечеткой базы знаний, а~также путем репрезентации неоднородной 
функциональной структуры сложной задачи и~гетерогенного коллективного 
мышления интеллектуальных агентов, взаимодействующих в~соответствии 
с~предложенным протоколом, \mbox{ГИМСГМ} вырабатывает для каждой проб\-ле\-мы 
релевантный ей метод решения без упрощения и~идеализации в~условиях 
динамической среды. 

\section{Заключение}

  Рассмотрены особенности проблемы вос\-ста\-нов\-ле\-ния электроснабжения 
в~региональной распределительной электросети после масштабных аварий 
и~предложен новый класс интеллектуальных сис\-тем для ее решения~--- 
\mbox{ГИМСГМ}. 

Представлено
 формализованное описание \mbox{ГИМСГМ}, ее основных 
составных частей. 

Описан протокол организации коллективного гетерогенного 
мышления агентов на основе теории речевых актов. Его применение 
в~гиб\-рид\-ных\linebreak интеллектуальных многоагентных системах, содержащих 
разнородные интеллектуальные самоорганизующиеся агенты, позволяет 
релевантно моделировать эффективные практики коллективного\linebreak решения 
проблем. 

Применение \mbox{ГИМСГМ} позволит  
опе\-ра\-тив\-но-дис\-пет\-чер\-ско\-му персоналу электроснабжающих 
организаций принимать релевантные решения по восстановлению электросети 
в~условиях дефицита времени.
  
{\small\frenchspacing
 {%\baselineskip=10.8pt
 \addcontentsline{toc}{section}{References}
 \begin{thebibliography}{99}
\bibitem{1-kol}
  \Au{Adibi M.\,M., Fink~L.\,H.} Overcoming restoration challenges associated with major 
power system disturbances~// IEEE Power Energy~M., 2006. Vol. 4. Iss. 5. P. 68--77.
\bibitem{2-kol}
  \Au{Красник В.\,В.} Потребители электрической энергии, энергоснабжающие организации 
  и~органы Ростехнадзора. Правовые основы взаимоотношений.~--- М.: НЦ ЭНАС, 2005. 250~с.
\bibitem{3-kol}
  \Au{Нариньяни А.\,С.} Инженерия знаний и~НЕ-факторы: краткий обзор-08~// Вопросы 
искусственного интеллекта, 2008. №\,1. С.~61--77.
\bibitem{4-kol}
  \Au{Kolesnikov A.\,V., Listopad~S.\,V.}
  Hybrid intelligent multiagent system of heterogeneous 
thinking for solving the problem of restoring the distribution
 power grid after failures~// Open 
Semantic Technologies for Intelligent Systems: 
Research Papers Collection.~--- 
Minsk: BGUIR, 2019. P. 133--138.
\bibitem{5-kol}
  \Au{Колесников А.\,В.} Гибридные интеллектуальные системы. Теория и~технология 
разработки.~--- СПб.: \mbox{СПбГТУ}, 2001. 711~с.
\bibitem{6-kol}
  \Au{Тарасов В.\,Б.} От многоагентных систем к~интеллектуальным организациям: 
философия, психология, информатика.~--- М.: Эдиториал УРСС, 2002. 352~с.
\bibitem{7-kol}
  \Au{Gardner H.} Multiple intelligences~--- the theory in practice.~--- New York, NY, USA: 
Basic Books, 1993. 320~p.

\bibitem{9-kol} %8
  \Au{De Bono~E.} Parallel thinking: From Socratic to De Bono thinking.~--- Melbourne: Penguin 
Books, 1994. 228~p.

\bibitem{8-kol} %9
  \Au{Kaner S., Lind~L., Toldi~C., Fisk~S., Beger~D.}
   The facilitator's guide to participatory 
decision-making.~--- San Francisco, CA, USA: Jossey-Bass, 2011. 368~p.

\bibitem{10-kol}
  \Au{Колесников А.\,В., Листопад~С.\,В.} Модель гибридной интеллектуальной 
многоагентной системы гетерогенного мышления для информационной подготовки 
оперативных решений в~региональных электрических сетях~// Системы и~средства 
информатики, 2018. Т.~28. №\,4. С.~31--41.
\bibitem{11-kol}
  \Au{Tang A.\,Y.\,C., Basheer~G.\,S.}
  A~Conflict Resolution Strategy Selection Method (ConfRSSM) 
in multi-agent systems~// Int. J.~Advanced Computer Sci. Appl., 
2017. Vol.~8. Iss.\,5. P.~398--404.
\bibitem{12-kol}
  \Au{Колесников А.\,В., Листопад~С.\,В.} Функциональная структура гибридной 
интеллектуальной многоагентной системы гетерогенного мышления для решения проблемы 
восстановления распределительной электросети~// Системы и~средства информатики, 2019. 
Т.~29. №\,1. С.~41--52.
\bibitem{13-kol}
  \Au{Городецкий В.\,И., Грушинский~М.\,С., Хабалов~А.\,В.}
  Многоагентные системы (обзор)~// 
Новости искусственного интеллекта, 1998. №\,2. С.~64--116.
\bibitem{14-kol}
  \Au{Singh M.\,P., Chopra~A.\,K.}
  Programming multiagent systems without programming agents~// 
Programming multi-agent systems~/ Eds. L.~Braubach, J.-P.~Briot, J.~Thangarajah.~--- 
  Lecture notes in artificial intelligence ser.~--- Springer, 2010.   Vol.~5919. P.~1--14. 
\bibitem{15-kol} 
  \Au{Singh R., Singh~A., Mukherjee~S.} 
  A~critical investigation of agent interaction protocols in 
multiagent systems~// Int. J.~Advancements Technology, 2014. Vol.~5. Iss.\,2. 
P.~72--81.
\bibitem{17-kol}
  \Au{Smith G.} The Contract Net Protocol: High level communication and control in a 
distributed problem solver~// IEEE T.~Comput., 1980. Vol.~29. Iss.\,12.  
P.~1104--1113.

\bibitem{16-kol} %16
  \Au{Smith R.\,G.} A~framework for distributed problem solving.~--- Ann Arbor, MI,
  USA: UMI  Research Press, 1981. 188~p.

\bibitem{18-kol}
  \Au{Weerasooriya D., Rao A.\,S., Ramamohanarao~K.}
  Design of a~concurrent agent-oriented 
language~// Intelligent agents~/ Eds. M.\,J.~Wooldridge, N.\,R.~Jennings.~---
Lecture notes in computer science ser.~---  Springer, 1995.  
Vol.~890. P.~386--401.
\bibitem{19-kol}
  \Au{Zlotkin G., Rosenschtein~J.}
   Mechanisms for automated negotiation in state oriented 
domain~// J.~Artif. Intell. Res., 1996. Vol.~5. P.~163--238.
\bibitem{20-kol}
  \Au{Marzougui B., Barkaoui~K.}
  Interaction protocols in multi-agent systems based on agent 
Petri nets model~// Int. J.~Advanced Computer Sci. Appl., 2013. 
Vol.~4. Iss.\,7. P.~166--173. 

 \end{thebibliography}

 }
 }

\end{multicols}

\vspace*{-3pt}

\hfill{\small\textit{Поступила в~редакцию 31.03.19}}

\vspace*{8pt}

%\pagebreak

%\newpage

%\vspace*{-29pt}

\hrule

\vspace*{2pt}

\hrule

%\vspace*{-2pt}

\def\tit{HETEROGENEOUS THINKING PROTOCOL OF~HYBRID INTELLIGENT MULTIAGENT 
SYSTEM FOR~SOLVING DISTRIBUTIONAL POWER GRID RECOVERY PROBLEM}


\def\titkol{Heterogeneous thinking protocol of~hybrid intelligent multiagent 
system for~solving distributional power grid recovery problem}

\def\aut{A.\,V.~Kolesnikov$^1$ and~S.\,V.~Listopad$^2$}

\def\autkol{A.\,V.~Kolesnikov and~S.\,V.~Listopad}

\titel{\tit}{\aut}{\autkol}{\titkol}

\vspace*{-11pt}


\noindent
   $^1$Immanuel Kant Baltic Federal University, 14~A.~Nevskogo Str., 
   Kaliningrad 236041, Russian Federation
   
   \noindent
   $^2$Kaliningrad Branch of the Federal Research Center ``Computer 
   Science and Control'' of the Russian Academy\linebreak
   $\hphantom{^1}$of Sciences, 5~Gostinaya Str, Kaliningrad 236022, Russian Federation

\def\leftfootline{\small{\textbf{\thepage}
\hfill INFORMATIKA I EE PRIMENENIYA~--- INFORMATICS AND
APPLICATIONS\ \ \ 2019\ \ \ volume~13\ \ \ issue\ 2}
}%
 \def\rightfootline{\small{INFORMATIKA I EE PRIMENENIYA~---
INFORMATICS AND APPLICATIONS\ \ \ 2019\ \ \ volume~13\ \ \ issue\ 2
\hfill \textbf{\thepage}}}

\vspace*{6pt}
  
  
  
   \Abste{The problem of power supply restoration in the regional distributional power grid after large-scale 
accidents is characterized by high combinatorial complexity, heterogeneity, underdetermination, inaccuracy, 
and ambiguity. The use of the collective problem solving mechanism to overcome the listed non-factors in 
the sense of A.\,S.~Narinyani is impossible due to time constraints. To solve this problem, a new class of 
intelligent systems that model collective decision-making under the guidance of a facilitator is proposed, 
namely, hybrid intelligent multiagent systems of heterogeneous thinking. Unlike traditional hybrid 
intelligent systems that integrate models of expert knowledge, they additionally model group processes and 
effects arising from collective problem solving, adapting to the dynamic nature of the problem of restoring 
the regional distribution grid. The paper discusses one of the components of such systems, namely, the 
protocol for organizing collective heterogeneous thinking of agents.}
   
   \KWE{heterogeneous thinking; hybrid intelligent multiagent system; distributional power grid recovery 
problem} 
   
   
\DOI{10.14357/19922264190211}

%\vspace*{-14pt}

 \Ack
   \noindent
   The reported study was funded by the Russian Foundation for Basic Research according to the research project 
No.\,18-07-00448А.



%\vspace*{6pt}

  \begin{multicols}{2}

\renewcommand{\bibname}{\protect\rmfamily References}
%\renewcommand{\bibname}{\large\protect\rm References}

{\small\frenchspacing
 {%\baselineskip=10.8pt
 \addcontentsline{toc}{section}{References}
 \begin{thebibliography}{99}
\bibitem{1-kol-1}
\Aue{Adibi, M.\,M., and L.\,H.~Fink.} 2006. Overcoming restoration challenges 
associated with major power system disturbances. \textit{IEEE Power Energy~M.} 
4(5):68--77.
\bibitem{2-kol-1}
\Aue{Krasnik, V.\,V.} 2005. \textit{Potrebiteli elektricheskoy energii, 
energosnabzhayushchie organizatsii i~organy Rostekhnadzora. Pravovye osnovy 
vzaimootnosheniy} [Consumers of electric energy, energy supplying organizations and 
bodies of Rostechnadzor. Legal basis of relationships]. Moscow: ENAS. 250 p.
\bibitem{3-kol-1}
\Aue{Narinyani, A.\,S.} 2008. Inzheneriya znaniy i~NE-faktory: kratkiy obzor-08 
[Knowledge engineering and non-factors: A~brief overview-08]. \textit{Voprosy 
iskusstvennogo intellekta} [Artificial Intelligence Issues] 1:61--77.
\bibitem{4-kol-1}
\Aue{Kolesnikov, A.\,V., and S.\,V.~Listopad.} 2019. Hybrid intelligent multiagent 
system of heterogeneous thinking for solving the problem of restoring the distribution 
power grid after failures. \textit{Open Semantic Technologies for Intelligent 
Systems: Research Papers Collection}. Minsk: BGUIR. 133--138.
\bibitem{5-kol-1}
\Aue{Kolesnikov, A.\,V.} 2001. \textit{Gibridnye intellektual'nye sistemy. Teoriya 
i~tekhnologiya razrabotki} [Hybrid intelligent systems: Theory and technology of 
development]. St.\ Petersburg: SPbSTU Publs. 711~p.
\bibitem{6-kol-1}
\Aue{Tarasov, V.\,B.} 2002. \textit{Ot mnogoagentnykh sistem k~intellektual'nym 
organizatsiyam: filosofiya, psikhologiya, informatika} [From multiagent systems to 
intelligent organizations: Philosophy, psychology, and informatics]. Moscow: 
Editorial URSS. 352~p.
\bibitem{7-kol-1}
\Aue{Gardner, H.} 1993. \textit{Multiple intelligences~--- 
  the theory in practice.} New York, NY:  Basic Books. 320~p.

\bibitem{9-kol-1}
\Aue{De Bono, E.} 1994. \textit{Parallel thinking: From Socratic to De Bono thinking}. 
Melbourne: Penguin Books. 228~p.

\bibitem{8-kol-1}
\Aue{Kaner, S., L.~Lind, C.~Toldi, S.~Fisk, and D.~Beger.} 2011. \textit{The facilitator's 
guide to participatory decision-making.} San Francisco, CA: Jossey-Bass. 368~p.

\bibitem{10-kol-1}
\Aue{Kolesnikov, A.\,V., and S.\,V.~Listopad.} 2018. Model' gibridnoy intellektual'noy 
mnogoagentnoy sistemy geterogennogo myshleniya dlya informatsionnoy 
podgotovki operativnykh resheniy v~regional'nykh elektricheskikh setyakh [Model of 
a~hybrid intelligent multiagent system of heterogeneous thinking for preparation of 
information about operational decisions in a~regional power system]. \textit{Sistemy 
i~Sredstva Informatiki~--- Systems and Means of Informatics} 28(4):31--41.
\bibitem{11-kol-1}
\Aue{Tang, A.\,Y.\,C., and G.\,S.~Basheer.} 2017. A~Conflict Resolution Strategy 
Selection Method (ConfRSSM) in multi-agent systems. \textit{Int. 
J.~Adv. Computer Sci. Appl.} 8(5):398--404.
\bibitem{12-kol-1}
\Aue{Kolesnikov, A.\,V., and S.\,V.~Listopad.} 2019. Funk\-tsi\-o\-nal'\-naya struktura 
gibridnoy intellektual'noy mno\-go\-agent\-noy sistemy geterogennogo myshleniya dlya 
resheniya problemy vosstanovleniya raspredelitel'noy elektroseti [Functional 
structure of the hybrid intelligent multiagent system of heterogeneous thinking for 
solving the problem of restoring the distribution power grid]. \textit{Sistemy i~Sredstva 
Informatiki~--- Systems and Means of Informatics} 29(1):41--52.
\bibitem{13-kol-1}
\Aue{Gorodetskiy, V.\,I., M.\,S.~Grushinskiy, and A.\,V.~Khabalov.} 1998. 
Mno\-go\-agent\-nye 
sistemy (obzor) [Multiagent systems (review)]. 
\textit{Novosti iskusstvennogo intellekta} [Artificial Intelligence News] 2:64--116.
\bibitem{14-kol-1}
\Aue{Singh, M.\,P., and A.\,K.~Chopra.} 2010. Programming multiagent systems 
without programming agents. \textit{Programming multi-agent systems}.  
Eds. L.~Braubach, J.-P.~Briot, and J.~Thangarajah.
Lecture  notes in artificial intelligence ser. Springer. 5919:1--14. 
\bibitem{15-kol-1}
\Aue{Singh, R., A.~Singh, and S.~Mukherjee.} 2014. A~critical investigation of agent 
interaction protocols in multiagent systems. \textit{Int. J.~Advancements 
Technology} 5(2):72--81.

\bibitem{17-kol-1}
\Aue{Smith, G.} 1980. The Contract Net Protocol: High level communication and 
control in a distributed problem solver. \textit{IEEE T.~Comput.} 29(12):1104--1113.

\bibitem{16-kol-1}
\Aue{Smith, R.\,G.} 1981. \textit{A~framework for distributed problem solving}. 
Ann Arbor, MI: UMI Research Press. 188~p.

\bibitem{18-kol-1}
\Aue{Weerasooriya, D., A.\,S.~Rao, and K.~Ramamohanarao.} 1994. Design of 
a~concurrent agent-oriented language. 
\textit{Intelligent agents}. Eds. M.\,J.~Wooldridge and N.\,R.~Jennings.
  Lecture notes in computer science ser. Springer. 890:386--401.
\bibitem{19-kol-1}
\Aue{Zlotkin, G., and J.~Rosenschtein.} 1996. Mechanisms for automated 
negotiation in state oriented domain. \textit{J.~Artif. Intell. Res.} 
5:163--238.
\bibitem{20-kol-1}
\Aue{Marzougui, B., and K.~Barkaoui.} 2013. Interaction protocols in multi-agent 
systems based on agent Petri nets model. \textit{Int. J.~Advanced 
Computer Sci. Appl.} 4(7):166--173.

\end{thebibliography}

 }
 }

\end{multicols}

\vspace*{-6pt}

\hfill{\small\textit{Received March 31, 2019}}

%\pagebreak

%\vspace*{-18pt}
  
  \Contr
  
  
  \noindent
  \textbf{Kolesnikov Alexander V.} (b.\ 1948)~--- Doctor of Science in 
technology, professor, Institute of Physical and Mathematical Sciences and 
Information Technology, Immanuel Kant Baltic Federal University, 
14~A.~Nevskogo Str., Kaliningrad 236041, Russian Federation; 
\mbox{avkolesnikov@yandex.ru} 
  
  \vspace*{3pt}
  
  \noindent
  \textbf{Listopad Sergey V.} (b.\ 1984)~--- Candidate of  Science (PhD) in 
technology, senior scientist, Kaliningrad Branch of the Federal Research Center 
``Computer Science and Control'' of the Russian Academy of Sciences, 5~Gostinaya 
Str., Kaliningrad 236022, Russian Federation; \mbox{ser-list-post@yandex.ru }
  
  
\label{end\stat}

\renewcommand{\bibname}{\protect\rm Литература}   %11
\def\stat{inkova}

\def\tit{СТЕПЕНЬ СЕМАНТИЧЕСКОЙ БЛИЗОСТИ ДИСКУРСИВНЫХ ОТНОШЕНИЙ: МЕТОДЫ И~ИНСТРУМЕНТЫ РАСЧЕТА$^*$}

\def\titkol{Степень семантической близости дискурсивных отношений: методы и~инструменты расчета}

\def\aut{О.\,Ю.~Инькова$^1$, М.\,Г.~Кружков$^2$}

\def\autkol{О.\,Ю.~Инькова, М.\,Г.~Кружков}

\titel{\tit}{\aut}{\autkol}{\titkol}

\index{Инькова О.\,Ю.}
\index{Кружков М.\,Г.}
\index{Inkova O.\,Yu.}
\index{Kruzhkov M.\,G.}


{\renewcommand{\thefootnote}{\fnsymbol{footnote}} \footnotetext[1]
{Работа выполнена в~Федеральном исследовательском центре <<Информатика и~управление>> Российской 
академии наук с~использованием ЦКП <<Информатика>> ФИЦ ИУ РАН.}}


\renewcommand{\thefootnote}{\arabic{footnote}}
\footnotetext[1]{Федеральный исследовательский центр <<Информатика и~управление>> Российской академии наук; 
Женевский университет, \mbox{olyainkova@yandex.ru}}
\footnotetext[2]{Федеральный исследовательский центр <<Информатика и~управление>> Российской 
академии наук, \mbox{magnit75@yandex.ru}}

%\vspace*{-14pt}


  
  \Abst{Рассматриваются методы оценки семантической близости дискурсивных 
отношений. Авторы предлагают несколько подходов к~решению этой проблемы с~применением двух информационных ресурсов: коллекции сформированных авторами 
структурированных определений ло\-ги\-ко-се\-ман\-ти\-че\-ских отношений (ЛСО) 
и~Надкорпусной базы данных коннекторов (НБДК), включающей в~себя аннотации переводных 
соответствий текстовых фрагментов с~маркерами ЛСО на русском, французском 
и~итальянском языках. Показано, что при оценке семантической близости ЛСО высокий 
приоритет будут иметь такие факторы, как принадлежность различительных признаков ЛСО к~одному семейству в~структурированных определениях отношений, соответствия между 
показателями различных ЛСО в~оригинальных и~переводных текстах, а также случаи, когда 
различные ЛСО выражаются одинаковыми показателями в~разных контекстах. Менее значим 
фактор сочетаемости различных ЛСО в~рамках одного и~того же контекста. Предполагается, 
что на основе сформулированных методов станет возможным более точно определить, какие 
различительные признаки ЛСО имеют наибольший вес при определении их семантической  
бли\-зости.}
  
  \KW{надкорпусная база данных; логико-семантические отношения; коннекторы; 
аннотирование; фасетная классификация}

  \DOI{10.14357/19922264230412}{FXTSPZ}
  
%\vspace*{-1pt}


\vskip 10pt plus 9pt minus 6pt

\thispagestyle{headings}

\begin{multicols}{2}

\label{st\stat}
  
\section{Степень семантической близости дискурсивных 
отношений}

%\vspace*{-4pt}

  Проблемы классификации дискурсивных отношений, обеспечивающих 
связность текста, занимают лингвистов и~специалистов по автоматической 
обработке текста не один десяток лет: первые исследования начались  
в~1970-х~гг.~[1, 2]. Были предложены их многочисленные классификации (ср.\ 
наиболее известные~[3--7]), однако никто, насколько известно авторам, не 
пытался определить степень семантической близости (ССБ) дискурсивных 
отношений. Это связано прежде всего с~тем, что классификации имеют, за 
редким исключением~\cite{7-in, 8-in, 9-in}, форму списка, и~этот вопрос просто 
не ставился. Однако его решение полезно не только для анализа текста, в~том 
числе автоматического, но и~для когнитивных наук и~переводоведения, 
поскольку позволяет выявить общие закономерности человеческого мышления.
  
  Кроме того, сами дискурсивные отношения определены во многом неточно 
или тавтологично\footnote[3]{См., например, определение отношения альтернативы 
(disjunction) в~теории риторической структуры: (а)~элемент пред\-став\-ля\-ет собой (не 
обязательно исключающую) альтернативу другому; (б)~слу\-ша\-ющий/чи\-та\-тель 
распознает, что связанные элементы альтернативны (см.\ {\sf http://www.sfu.ca/rst}).}, схожие 
или идентичные отношения носят даже в~англоязычных классификациях разные 
названия, а одинаковые названия описывают разную языковую реальность. 
Например, в~теории сегментированного представления дискурса (Segmented 
Discourse Representation Theory, SDRT~[10]) отношение contrast включает как 
отношения <<вопреки ожидаемому>>, так и~уступительные отношения. 
В~классификации Пенсильванского аннотированного корпуса им 
соответствуют два отношения (opposition и~contra-expectation)~\cite{7-in}, 
а~в~теории риторической структуры~--- contrast и~concession~[11] (подробнее 
см.~\cite[с.~37]{9-in}). 

\begin{table*}[b]\small %tabl1
\vspace*{-10pt}
\begin{center}
\Caption{Структурированные определения уступительных ЛСО и~ЛСО <<вопреки 
ожидаемому>>}
\vspace*{2ex}

\tabcolsep=3pt
\begin{tabular}{|l|p{40mm}|p{38mm}|p{57mm}|}
\hline
\multicolumn{1}{|c|}{\textbf{ЛСО}} & \multicolumn{1}{c|}{\tabcolsep=0pt\begin{tabular}{c}\textbf{Базовая семантическая}\\ \textbf{операция}\end{tabular}}&
\multicolumn{1}{c|}{\textbf{Уровень}} &
\multicolumn{1}{c|}{ \tabcolsep=0pt\begin{tabular}{c}\textbf{Дополнительные}\\ \textbf{характеристики}\end{tabular}}\\
\hline
&&&\\[-20pt]
\multicolumn{1}{|l|}{\raisebox{-26pt}[0pt][0pt]{\textbf{Уступительные}}}& 
%\begin{itemize}
\multicolumn{1}{l|}{\raisebox{-26pt}[0pt][0pt]{\ \ \ \  --\ \ операция импликации}}
%\end{itemize} 
& 
%\begin{itemize}
\multicolumn{1}{l|}{\raisebox{-26pt}[0pt][0pt]{\tabcolsep=0pt\begin{tabular}{l}\ \ \ \ --\ \ пропозициональный\\
\hphantom{\ \ \ \ --\ \ }уровень\end{tabular}}}
%\end{itemize}
&
\begin{itemize}
\item $p$ и~$q$~--- положения вещей;\vspace*{-3pt}
\item как правило, если имеет место $q$, то не имеет места~$p$\vspace*{-8pt}
   \end{itemize}
\\
\hline
&&&\\[-20pt]
\multicolumn{1}{|l|}{\raisebox{-48pt}[0pt][0pt]{\tabcolsep=0pt\begin{tabular}{l}\textbf{<<Вопреки}\\ \textbf{ожидаемому>>}\end{tabular} }}& 
%\begin{itemize}
\multicolumn{1}{l|}{\raisebox{-48pt}[0pt][0pt]{\tabcolsep=0pt\begin{tabular}{l}\ \ \ \  --\ \ операция сравнения,\\
 \hphantom{\ \ \ \ --\ \ }уста\-нав\-ли\-ва\-ющая не-\\
 \hphantom{\ \ \ \ --\ \ }сходство $p$ и~$q$\end{tabular}}}
%\end{itemize} 
&
%\begin{itemize}
\multicolumn{1}{l|}{\raisebox{-48pt}[0pt][0pt]{\tabcolsep=0pt\begin{tabular}{l}\ \ \ \  --\ \ пропозициональный\\ 
 \hphantom{\ \ \ \ --\ \ }уровень\end{tabular}}}
%\end{itemize} 
&
 \begin{itemize}
 \item $q$ имеет большую аргументативную\newline силу, чем~$p$;\vspace*{-3pt}
  \item положение вещей $p$ служит аргументом в~пользу ожи\-да\-емо\-го вывода~$r$;\vspace*{-3pt}
  \item положение вещей $q$ служит аргументом в~пользу ожи\-да\-емо\-го вывода не-$r$\vspace*{-8pt}
  \end{itemize}\\
\hline
\end{tabular}
\end{center}
\end{table*}
  
  В~этой связи были сделаны попытки сравнить\linebreak существующие 
классификации, чтобы понять, насколько соотносимы выделяемые в~них 
дискурсивные отношения~[12--14]. В~[14] для этого применяется 
набор различительных признаков. Этих\linebreak признаков, однако, недостаточно, чтобы 
сформулировать уникальное определение отношения, и~некоторые из них 
имеют одинаковый набор признаков. Это касается, например, четырех 
отношений (narration, precondition, background и~parallel) в~SDRT~\cite[с.~38]{14-in}. 
  
  В~работе~[15] были заложены основы для разработки структурированных 
определений дискурсивных, или в~терминологии автора  
ло\-ги\-ко-се\-ман\-ти\-че\-ских, отношений на основе применяемой 
в~НБДК классификации. Каждое 
ЛСО может быть описано набором различительных признаков (см.\ примеры 
в~\cite{16-in} и~\cite{17-in}). Некоторые признаки оказываются общими для 
нескольких ЛСО, другие~--- индивидуальны, т.\,е.\ свойственны только данному 
ЛСО. На момент написания статьи в~НБДК были описаны 26~ЛСО 
с~использованием~52~различительных признаков. Это позволяет дать каждому 
ЛСО уникальное определение (см.\ примеры в~разд.~2), а~также определить 
ССБ ЛСО. 

\vspace*{-6pt}
  
\section{Критерии, лежащие в~основе определения степени 
семантической близости логико-семантических отношений}

\vspace*{-3pt}

  В~предыдущей работе авторов~[17] показано, что не все различительные 
признаки имеют одинаковый вес при определении семантической близости 
ЛСО и~что, предположительно, наибольшее значение имеет принадлежность 
общих признаков к~одному семейству. 
  

  
  В~основе уступительных ЛСО и~ЛСО <<вопреки ожидаемому>> лежат 
разные базовые операции: импликация~--- для первого и~сравнение, 
уста\-нав\-ли\-ва\-ющее несходство $p$ и~$q$,~--- для второго (табл.~1). Это 
значит, что эти два ЛСО находятся в~разных семантических группах. Оба ЛСО 
при этом установлены на пропозициональном уровне, т.\,е.\ непосредственно 
между положениями дел $p$ и~$q$, которые они связывают, и~оба используют 
отрицательный коррелят одного из положений вещей. Иначе говоря, признаки 
<<как правило, если имеет место~$q$, то не имеет места $p$>> и~<<положение 
вещей~$q$ служит аргументом в~пользу ожидаемого вывода не-$r$>> 
принадлежат к~одному семейству. В~примере~(1) с~ЛСО <<вопреки 
ожидаемому>>: \textit{Ему [$\ldots$] очень неприятно было сталкиваться с~народом,} {\bfseries\textit{но}} \textit{он шел именно туда, где виднелось больше 
народу}. [Ф.\,М.~Достоевский. Преступление и~наказание], положение вещей 
$p$\;=\;<<ему очень неприятно было сталкиваться с~народом>> ориентирует в~пользу вывода $r$\;=\;<<он не должен был бы идти к~народу>>. Этот вывод 
опровергается непосредственно в~$q$ (=\;не-$r$)\;=\;<<он шел именно туда, где 
виднелось больше народу>>. Семантический механизм, лежащий в~основе 
уступительных отношений (их прототипическим показателем может считаться 
союз \textit{хотя}), совпадает с~этим семантическим механизмом, но 
в~зеркальном отражении: 
  \begin{gather*}
p\ \mbox{\textit{хотя}}\  q (q \to  \mbox{не-}p)\\
p \to r\ \mbox{но}\  q\ (q = \mbox{не-}r),\ \mbox{т.\,е.}\ p \to \mbox{не-}q\ 
\mbox{\textit{но}}\ q.
\end{gather*}
  %
  Отсюда необходимость при замене \textit{хотя} на \textit{но} и~наоборот 
изменить порядок следования фрагментов текста: \textit{Ему неприятно было 
сталкиваться с~народом}, {\bfseries\textit{но}} \textit{он шел туда, где виднелось 
больше народу} (ЛСО <<вопреки ожидаемому>>); \textit{Он шел туда, где 
виднелось больше народу}, {\bfseries\textit{хотя}} \textit{ему неприятно было 
сталкиваться с~народом} (ЛСО уступки)~\cite{18-in}. Это позволяет говорить 
о~семантической близости двух ЛСО и,~например, в~классификации~\cite{7-in} 
они объединены в~одну группу concession.

\begin{table*}[b]\small %tabl2
\vspace*{-6pt}
\begin{center}
\Caption{Логико-семантические отношения, соответствующие ЛСО <<вопреки ожидаемому>> в~оригинальных и~переводных текстах }
\vspace*{2ex}

\tabcolsep=4.3pt
\begin{tabular}{|c|l|c|c|c|c|c|c|}
\hline
\textbf{ЛСО1}&\multicolumn{1}{c|}{\textbf{ЛСО2}}&\textbf{1}\;+\;\textbf{2}&\textbf{1}&
\textbf{2}&\textbf{1}\;$\to$\;\textbf{2}&\textbf{2}\;$\to$\;\textbf{1}&\textbf{Сумма}\\
\hline
<<вопреки ожидаемому>>&уступительные&237\hphantom{9}&2140&853&11,07\%\hphantom{9}&27,78\%\hphantom{9}&38,86\%\hphantom{9}\\
<<вопреки ожидаемому>>&одновременность&139\hphantom{9}&2140&1268\hphantom{9}&6,50\%&10,96\%\hphantom{9}&17,46\%\hphantom{9}\\
<<вопреки ожидаемому>>&соединительные&149\hphantom{9}&2140&2088\hphantom{9}&6,96\%&7,14\%&14,10\%\hphantom{9}\\
<<вопреки ожидаемому>>&сопоставительные&78&2140&807&3,64\%&9,67\%&13,31\%\hphantom{9}\\
<<вопреки ожидаемому>>&пропозициональное 
сопутствование&39&2140&378&1,82\%&10,32\%\hphantom{9}&12,14\%\hphantom{9}\\
<<вопреки ожидаемому>>&исключение из 
рассмотрения&\hphantom{9}8&2140&\hphantom{9}90&0,37\%&8,89\%&9,26\%\\
<<вопреки ожидаемому>>&иллокутивное 
сопутствование&17&2140&471&0,79\%&3,61\%&4,40\%\\
<<вопреки ожидаемому>>&интенсиональная 
генерализация&\hphantom{9}8&2140&248&0,37\%&3,23\%&3,60\%\\
<<вопреки ожидаемому>>&замещение&\hphantom{9}7&2140&294&0,33\%&2,38\%&2,71\%\\
<<вопреки ожидаемому>>&пропозициональная 
коррекция&\hphantom{9}4&2140&165&0,19\%&2,42\%&2,61\%\\
<<вопреки ожидаемому>>&условные&12&2140&1075\hphantom{9}&0,56\%&1,12\%&1,68\%\\
<<вопреки ожидаемому>>&спецификация&11&2140&1608\hphantom{9}&0,51\%&0,68\%&1,20\%\\
<<вопреки ожидаемому>>&исключение&\hphantom{9}5&2140&615&0,23\%&0,81\%&1,05\%\\
<<вопреки ожидаемому>>&отрицательная 
альтернатива&\hphantom{9}2&2140&271&0,09\%&0,74\%&0,83\%\\
<<вопреки ожидаемому>>&оговорка&\hphantom{9}1&2140&150&0,05\%&0,67\%&0,71\%\\
<<вопреки ожидаемому>>&экстенсиональная 
генерализация&\hphantom{9}2&2140&588&0,09\%&0,34\%&0,43\%\\
<<вопреки ожидаемому>>&переформулирование&\hphantom{9}2&2140&1183\hphantom{9}&0,09\%&0,17\%&0,26\%\\
<<вопреки ожидаемому>>&пропозициональная 
альтернатива&\hphantom{9}1&2140&1238\hphantom{9}&0,05\%&0,08\%&0,13\%\\
\hline
\multicolumn{8}{p{163mm}}{\footnotesize \hspace*{3mm}Расшифровка названий столбцов: 
1\;+\;2~--- число переводных аннотаций, в~которых ЛСО1 в~тексте на одном языке 
соответствует ЛСО2 в~тексте на другом языке; 1~--- число аннотаций, в~которых в~любом из 
текстов проставлено ЛСО1; 2~--- число аннотаций, в~которых в~любом из текстов 
проставлено ЛСО2; 1\;$\to$\;2~--- процент соответствия для ЛСО1 с~ЛСО2; 2\;$\to$\;1~--- 
процент соответствия для ЛСО2 с~ЛСО1; сумма~--- сумма двух предыдущих показателей.}
\end{tabular}
\end{center}
\end{table*}

  
  
  Кроме того, сформулирована гипотеза, согласно которой при определении 
ССБ ЛСО могут учитываться также другие 
факторы:
\begin{enumerate}[(1)] 
\item соответствия ЛСО в~оригинальных и~переводных текстах; 
\item случаи, когда разные ЛСО выражаются одним и~тем же показателем; 
\item сочетаемость показателей ЛСО в~одном фрагменте текста.
\end{enumerate}
 В~НБДК для 
ЛСО, имеющих структурированные определения, были получены 
количественные данные по этим трем критериям.

  
  
\subsection{Соответствие логико-семантических отношений в~оригинальных и~переводных текстах}

  Соответствие ЛСО в~оригинальных и~переводных текстах означает, что 
некоторому ЛСО в~тексте оригинала, точнее, его показателю, соответствует 
показатель иного ЛСО в~тексте перевода. Так, если для перевода на 
французский язык коннектора \textit{но} в~примере~(1) был выбран коннектор 
\textit{mais}, также выражающий ЛСО <<вопреки ожидаемому>>: (2)~\textit{Il 
lui $\acute{\mbox{e}}$tait d$\acute{\mbox{e}}$sagr$\acute{\mbox{e}}$able, 
tr$\grave{\mbox{e}}$s d$\acute{\mbox{e}}$sagr$\acute{\mbox{e}}$able, de 
rencontrer du monde} {\bfseries\textit{mais}} \textit{il allait justement 
l$\grave{\mbox{a}}$ o$\grave{\mbox{u}}$ l'on en voyait le plus} [перевод 
$\acute{\mbox{E}}$lisabeth Guertik], то в~примере~(3) тот же коннектор 
переведен \textit{bien que}~--- показателем уступительных ЛСО: 
\textit{С~такой поправкой смысл телеграммы становился ясен,} 
{\bfseries\textit{но}}\textit{, конечно, трагичен}.~--- \textit{Ainsi 
corrig$\acute{\mbox{e}}$, le t$\acute{\mbox{e}}$l$\acute{\mbox{e}}$gramme 
prenait un sens parfaitement clair,} {\bfseries\textit{bien que}} \textit{tragique, 
naturellement}. [М.~Булгаков. Мастер и~Маргарита, перевод Claude Ligny].
  
  Количественные данные по ЛСО, соответствующим ЛСО <<вопреки 
ожидаемому>> в~оригинальных и~переводных текстах на русском, французском и~итальянском языках, приведены в~табл.~2.
  
  
  Для ЛСО <<вопреки ожидаемому>> в~НБДК сформирована 2141~двуязычная 
аннотация. В~237~случаях ему соответствует уступительное ЛСО. Это 
подтверждает важность критерия принадлежности \mbox{различительных} признаков к~одному семейству. 

Схожую картину можно наблюдать для других отношений 
(табл.~3): для сопоставительных и~соединительных ЛСО (основаны на 
общей базовой операции и~имеют общий различительный признак 
<<сходство~$p$ и~$q$ относительно некоторого ``общего\linebreak знаменателя''>>); для 
ЛСО оговорки и~пропозициональной альтернативы (они имеют общий 
различительный признак~--- <<$p$ и~$q$~--- положения вещей, име\-ющие 
статус гипотезы>>); для ЛСО \mbox{одновременности} и~со\-по\-став\-ле\-ния (их 
различительные при\-зна\-ки <<T$p$ включает в~себя T$q$>> и~<<$p$ и~$q$ 
актуальны для говорящего в~момент речи T$d$>> принадлежат к~семейству 
признаков <<Единство временного интервала>>); для ЛСО одновременности 
и~пропозиционального сопутствования (об\-щий признак <<T$p$ включает 
в~себя T$q$>>). 
  
\begin{table*}\small %tabl3
\begin{center}
\Caption{Соответствия других ЛСО }
\vspace*{2ex}

\begin{tabular}{|l|l|c|c|c|c|c|c|}
\hline
\multicolumn{1}{|c|}{\textbf{ЛСО1}}&\multicolumn{1}{c|}{\textbf{ЛСО2}}&\textbf{1}\;+\;\textbf{2}&\textbf{1}&\textbf{2}&\textbf{1}\;
$\to$\;\textbf{2}&\textbf{2}\;$\to$\;\textbf{1}&\textbf{Сумма}\\
\hline
соединительные&сопоставительные&272\hphantom{9}&2088&807&13,03\%&33,71\%&46,73\%\\
оговорка&пропозициональная альтернатива&40&\hphantom{9}150&1238\hphantom{9}&26,67\%&\hphantom{9}3,23\%&29,90\%\\
одновременность&сопоставление&180\hphantom{9}&1268&807&14,20\%&22,30\%&36,50\%\\
одновременность &пропозициональное 
сопутствование&43&1268&378&\hphantom{9}3,39\%&11,38\%&14,77\%\\
\hline
\end{tabular}
\end{center}
\vspace*{-4pt}
\end{table*}

\begin{table*}[b]\small %tabl4
\vspace*{-12pt}
\begin{center}
\Caption{Количественные данные по ЛСО, выражаемым одним показателем}
\vspace*{2ex}

\begin{tabular}{|c|l|l|c|}
\hline 
\textbf{Язык}&\multicolumn{1}{c|}{\textbf{Коннектор}}&\multicolumn{1}{c|}{\textbf{ЛСО}}&\textbf{Количество аннотаций}\\
\hline
\multicolumn{1}{|c|}{\raisebox{-11pt}[0pt][0pt]{RU}}&\multicolumn{1}{l|}{\raisebox{-11pt}[0pt][0pt]{а то}}&отрицательная альтернатива&125\hphantom{9}\\
&&пропозициональная альтернатива&12\\
&&исключение из рассмотрения&\hphantom{9}6\\
\hline
\multicolumn{1}{|c|}{\raisebox{-6pt}[0pt][0pt]{RU}}&\multicolumn{1}{l|}{\raisebox{-6pt}[0pt][0pt]{если$\|$то}}&условные&183\hphantom{9}\\
&&сопоставительные&13\\
\hline
\multicolumn{1}{|c|}{\raisebox{-6pt}[0pt][0pt]{RU}}&\multicolumn{1}{l|}{\raisebox{-6pt}[0pt][0pt]{когда}}&одновременность&13\\
&&условные&\hphantom{9}1\\
\hline
\multicolumn{1}{|c|}{\raisebox{-6pt}[0pt][0pt]{RU}}&\multicolumn{1}{l|}{\raisebox{-6pt}[0pt][0pt]{когда$\|$то}}&одновременность&38\\
&&условные&\hphantom{9}6\\
\hline
\multicolumn{1}{|c|}{\raisebox{-11pt}[0pt][0pt]{RU}}
&\multicolumn{1}{l|}{\raisebox{-11pt}[0pt][0pt]{между тем}}
&одновременность&126\hphantom{9}\\
&&<<вопреки ожидаемому>>&53\\
&&сопоставительные&11\\
\hline
\multicolumn{1}{|c|}{\raisebox{-6pt}[0pt][0pt]{RU}}&\multicolumn{1}{l|}{\raisebox{-6pt}[0pt][0pt]{между тем как}}&сопоставительные&29\\
&&одновременность&\hphantom{9}6\\
\hline
\multicolumn{1}{|c|}{\raisebox{-18pt}[0pt][0pt]{RU}}
&\multicolumn{1}{l|}{\raisebox{-18pt}[0pt][0pt]{разве}}
&оговорка&20\\
&&исключение&\hphantom{9}5\\
&&исключение из рассмотрения&\hphantom{9}4\\
&&условные&\hphantom{9}2\\
\hline
\multicolumn{1}{|c|}{\raisebox{-6pt}[0pt][0pt]{FR}}&\multicolumn{1}{l|}{\raisebox{-6pt}[0pt][0pt]{cependant}}&<<вопреки ожидаемому>>&100\hphantom{9}\\
&&одновременность&27\\
\hline
\multicolumn{1}{|c|}{\raisebox{-6pt}[0pt][0pt]{FR}}&\multicolumn{1}{l|}{\raisebox{-6pt}[0pt][0pt]{en m$\hat{\mbox{e}}$me temps}}&одновременность&29\\
&&сопоставительные&\hphantom{9}1\\
\hline
\multicolumn{1}{|c|}{\raisebox{-6pt}[0pt][0pt]{FR}}&\multicolumn{1}{l|}{\raisebox{-6pt}[0pt][0pt]{quand}}&одновременность&197\hphantom{9}\\
&&условные&10\\
\hline
\end{tabular}
\end{center}
\end{table*}

  
  Напротив, ЛСО, соответствующие ЛСО <<вопреки ожидаемому>> 
и~представленные менее чем в~1\% аннотаций (см.\ табл.~2), не имеют 
различительных признаков, принадлежащих к~одному семейству, и~выбор их 
показателей для перевода показателя ЛСО <<вопреки ожидаемому>> может 
быть квалифицирован как авторский и~контекстуальный.
  
\subsection{Разные логико-семантические отношения выражаются одним~и~тем~же~показателем}

  Известно, что коннекторы в~значительной своей части относятся 
к~многозначным языковым единицам, т.\,е.\ могут служить показателями более 
чем одного ЛСО. Так, для русского союза \textit{и} принято выделять пять 
значений: сочинительное, временного следования, добавления,  
ре\-зуль\-та\-тив\-но-след\-ст\-вен\-ное и~несоответствия; для союза 
\textit{когда}~--- два: одновременности и~условия; у~союза \textit{но} 
выделяются собственно противительное  
и~про\-ти\-ви\-тель\-но-усту\-пи\-тель\-ное значения, а~у~\textit{хотя}~--- 
уступительное и~усту\-пи\-тель\-но-про\-ти\-ви\-тель\-ное и~т.\,д.~[19--21]. Это 
отражают и~данные НБДК, причем с~указанием на частотность того или иного 
значения коннектора в~сформированных аннотациях. 

В~табл.~4 приведены 
выборочно данные для многозначных коннекторов русского и~французского 
языков.
  

  
  Приведенные данные подтверждают прежде всего положения теории 
грамматикализации, согласно которым семантическая эволюция языковых 
единиц имеет определенные закономерности.\linebreak Так, было показано, что на основе 
значения одновременности может развиваться семантика сопоставления и~противопоставления, а~также импликации~\cite{22-in}. Это хорошо видно на 
примере \mbox{коннекторов} \textit{когда}, \textit{между тем}, а~также французских 
\textit{cependant} `в~то же время, однако', \textit{en m$\hat{\mbox{e}}$me temps} 
`в~то же время' и~\textit{quand} `когда' (см.\ табл.~4). С~другой стороны, эти 
данные подтверждают гипотезу авторов о~том, что набор ЛСО, которые может 
маркировать один показатель, не случаен, а~включает семантически близкие 
ЛСО. Так, коннектор \textit{разве} зафиксирован в~НБДК как показатель ЛСО 
оговорки, исключения, исключения из рассмотрения и~условия. Эти ЛСО имеют 
общие различительные признаки. Ло\-ги\-ко-се\-ман\-ти\-че\-ские отношения оговорки и~условия~--- два признака: 
базовая операция импликации и~признаки из семейства гипотетичность; ЛСО 
условия и~исключения устанавливаются на пропозициональном уровне, а~ЛСО 
оговорки и~исключения из рас\-смот\-ре\-ния~--- на уров\-не вы\-ска\-зы\-ва\-ния; ЛСО 
оговорки, исключения и~исключения из рас\-смот\-ре\-ния обладают общими 
признаками на уровне семейства признаков (семантика исключения), а~ЛСО 
исключения и~исключения из рас\-смот\-ре\-ния осно\-ва\-ны на общей базовой 
операции (соотнесение элемента и~множества).
  
  Таким образом, данный критерий может быть полезен при определении CСБ 
ЛСО и~иметь достаточно высокий приоритет.
  
\subsection{Сочетаемость логико-семантических отношений в~рамках одного фрагмента текста}

  Третий критерий, который можно учитывать при определении ССБ ЛСО,~--- 
сочетаемость ЛСО, точнее их показателей. Здесь, однако, возникает ряд 
сложностей, связанных с~тем, что возможность сочетаемости показателей 
зависит в~первую очередь от морфологической природы показателя ЛСО. Как 
известно, коннекторы относятся к~разнообразным морфологическим классам: 
сочинительные со\-юзы (\textit{и}, \textit{а}, \textit{но}); подчинительные союзы 
(\textit{хотя}, \textit{потому что}, \textit{как}), так называемые 
<<конкретизаторы со\-юзов>>, перешедшие в~класс коннекторов, как правило, из 
наречных выражений (\textit{в~то же время}, \textit{однако}, \textit{впрочем}); 
предлоги (\textit{кроме}, \textit{после}). Союзы, например, как сочинительные, 
так и~подчинительные, не могут сочетаться между собой в~рамках единого 
фрагмента текста, и, наоборот, наибольшей легкостью в~сочетании именно с~союзами обладают <<конкретизаторы>> (\textit{но однако}, \textit{но впрочем}, 
\textit{а~между тем}, \textit{или например}, \textit{и~в~частности}). Если для 
показателей некоторых ЛСО можно выявить закономерности, то другие менее 
избирательны в~своих сочетаниях. Так, показатель ЛСО спецификации 
\textit{например} сочетается со всеми сочинительными союзами, а~показатель 
ЛСО <<вопреки ожидаемому>> \textit{впрочем} только с~союзами~\textit{а} 
и~\textit{но}, т.\,е.\ показателями близких ему (\textit{а}) или тех же (\textit{но}) 
ЛСО. Можно также учитывать двухместные реализации коннекторов, т.\,е.\ 
такие, где компоненты коннектора находятся в~каждом из соединяемых 
фрагментов текста, например \textit{хотя$\ldots$\ но}: \textit{Хотя он меня 
очень уговаривал, но я~не согласился}. Но такие сочетания возможны не для 
всех ЛСО и~сужают круг возможностей для получения адекватных 
количественных данных.
 
  В~связи с~вышесказанным при подсчете ССБ ЛСО этот критерий может 
использоваться лишь как дополнительный.
  
\section{Заключение}

  Из четырех рассмотренных критериев определения ССБ ЛСО: 
(1)~принадлежности различительных признаков ЛСО к~одному семейству, 
(2)~соответствия ЛСО в~оригинальных и~переводных \mbox{текс\-тах}, (3)~возможности 
одного показателя выражать разные ЛСО и~(4)~сочетаемости показателей ЛСО 
в~одном фрагменте текста~--- первые три могут иметь достаточно высокий 
приоритет. Четвертый признак обладает, напротив, наименьшим весом при 
определении ССБ ЛСО. 
  
  Степень детальности разметки, а следовательно, и~определений ЛСО не 
позволяет пока объяснить некоторые явления. Например, семантическую 
близость ЛСО условия и~одновременности, который подтверждается как их 
соответствиями в~оригинальных и~переводных текстах, так и~воз\-мож\-ностью 
выражаться одним показателем (\textit{когда}). Их общий признак <<T$p$ 
включает в~себя T$q$>> не входит в~определение условных ЛСО, так как 
соотношение временн$\acute{\mbox{ы}}$х планов положений вещей~$p$ и~$q$ может быть 
самым различным в~условном периоде. С~другой стороны, при ЛСО 
одновременности различным может быть их семантическое соотношение 
(семантическая независимость, противопоставленность, причина, следствие 
и~т.\,д.). Перевод показателя ЛСО одновременности показателем условных 
ЛСО наблюдается только при одновременной реализации положений 
вещей~$p$ и~$q$ и~при возможности установить между ними отношение 
импликации. Семантическая близость данных двух ЛСО может быть, 
следовательно, установлена на более низком иерархическом уровне, а~именно: 
при определении частных случаев его реализации. В~НБДК такая возможность 
предусмотрена, что позволит в~дальнейшем более детально описывать каждое 
ЛСО и~его виды, а~значит, более точно определить ССБ ЛСО.
{\looseness=1

}
  
{\small\frenchspacing
 {\baselineskip=10.6pt
 %\addcontentsline{toc}{section}{References}
 \begin{thebibliography}{99}
\bibitem{1-in}
\Au{Hobbs J.\,R.} A~computational approach to discourse analysis.~--- 
New York, NY, USA: Department of Computer Science, City College, City University of New 
York, 1976.  Research Report 76-2. P.~28--38.
\bibitem{2-in}
\Au{Hobbs J.\,R.} Why is discourse coherent?~--- Menlo Park, CA, 
USA: SRI International, 1978. SRI Technical Note 176. 44~p.
\bibitem{3-in}
\Au{Halliday M.\,A.\,K., Hasan~R.}  Cohesion in English.~--- London: Longman, 1976. 374~p.


\bibitem{5-in} %4
\Au{Mann W.\,C., Thompson~S.\,A.} Rhetorical structure theory: Towards a functional theory of 
text organization~// Text, 1988. Vol.~8. No.\,3. P.~243--281. doi: 10.1515/text.\linebreak  1.1988.8.3.243.

\bibitem{6-in} %5
\Au{Asher N.} Reference to abstract objects in discourse.~--- Dordrecht: Kluwer, 1993. 455~p.

\bibitem{4-in} %6
\Au{Halliday M.\,A.\,K.} An introduction to functional grammar.~--- 2nd ed.~--- London: 
Edward Arnold, 1994. 434~p.

\bibitem{7-in} %7
PDTB Research Group. The Penn Discourse Treebank 2.0 annotation manual.~--- Philadelphia, PA, USA: Institute for Research in Cognitive Science, University 
of Pennsylvania, 2007.  Technical Report 
IRCS-08-01. 104~p. {\sf https://www.cis.upenn.edu/$\sim$elenimi/\linebreak pdtb-manual.pdf}.
\bibitem{8-in}
\Au{Breindl E., Volodina~A., \mbox{Wa{\!\ptb{\!\ss}}\,ner}~U.\,H.} Handbuch der deutschen 
Konnektoren~2: Semantik der deutschen Satzverkn$\ddot{\mbox{u}}$pfer.~--- Berlin: Walter de Gruyter, 2014. 
1327~p.
\bibitem{9-in}
\Au{Инькова О.\,Ю.} Логико-се\-ман\-ти\-че\-ские отношения: проблемы 
классификации~// Связность текста: мереологические ло\-ги\-ко-се\-ман\-ти\-че\-ские 
отношения.~--- М.: ЯСК, 2019. С.~11--98.
\bibitem{10-in}
\Au{Asher N., Lascarides~A.} Logics of conversation.~--- Cambridge: Cambridge University 
Press, 2003. 526~p.
\bibitem{11-in}
\Au{Carlson L., Marcu D.} Discourse tagging reference manual.~--- Marina del Rey, CA, USA: Information Sciences Institute, University of Southern 
California, 2001.  Technical Report ISI-TR-545. 87~p.



\bibitem{13-in} %12
\Au{Chiarcos Ch.} Towards interoperable discourse annotation: Discourse features in the 
Ontologies of Linguistic Annotation~// 9th Conference (International) on Language Resources 
and Evaluation Proceedings~/ Eds.\ N.~Calzolari, K.~Choukri, T.~Declerck, \textit{et al.}~--- Reykjavik, Iceland: European Language Resources Association 
(ELRA), 2014. P.~4569--4577.

\bibitem{12-in} %13
\Au{Benamara F., Taboada~M.} Mapping different rhetorical relation annotations: A~proposal~// 
4th Joint Conference on Lexical and Computational Semantics  Proceedings~/ Eds.\ M.~Palmer, G.~Boleda, P.~Rosso.~--- Denver, CO, USA: 
Association for Computational Linguistics, 2015. Р.~147--152. doi: 10.18653/v1/S15-1016.

\bibitem{14-in}
\Au{Sanders T., Demberg~V., Hoek~J., Scholman~M., Asr~F.\,T., Zufferey~S., Evers-Vermeul~J.} 
Unifying dimensions in coherence relations: How various annotation frameworks are related~// 
Corpus Linguist. Ling., 2018. Vol.~17. No.\,1. P.~1--71. doi:  
10.1515/cllt-2016-0078.
\bibitem{15-in}
\Au{Инькова О.\,Ю.} Определения дискурсивных отношений: опыт Надкорпусной базы 
данных коннекторов~// Компьютерная лингвистика и~интеллектуальные технологии: По 
мат-лам ежегодной \mbox{Междунар.} конф. <<Диалог>>.~--- М.: РГГУ, 2021. Вып.~20(27). 
С.~328--338.
\bibitem{16-in}
\Au{Инькова О.\,Ю., Кружков М.\,Г.} Структурированные определения дискурсивных 
отношений в~Надкорпусной базе данных коннекторов~// Информатика и~её применения, 
2021. Т.~15. Вып.~4. С.~27--32. doi: 10.14357/19922264210404. EDN: EZJXVI.

\bibitem{17-in}
\Au{Инькова О.\,Ю., Кружков М.\,Г.} Критерии определения семантической близости 
дискурсивных отношений~// Информатика и~её применения, 2023. Т.~17. Вып.~3.  
С.~100--106. doi: 10.14357/19922264230314. EDN: UJZJZI.

\bibitem{18-in}
\Au{Инькова О.\,Ю., Нуриев В.\,А.} Насколько лингвоспецифичен союз \textit{хотя}?~// 
Компьютерная лингвистика и~интеллектуальные технологии: По мат-лам ежегодной 
Междунар. конф. <<Диалог>>.~--- М.: РГГУ, 2018. Вып.~17(24). С.~254--266.

\bibitem{20-in} %19
Словарь современного русского литературного языка: в~17~т.~/ Под ред. 
В.\,И.~Чернышева.~--- М., Л.: Изд-во Академии наук СССР~/ Наука, 1950--1965.

\bibitem{19-in} %20
Русская грамматика~/ Под ред. Н.\,Ю.~Шведовой.~--- М.: Наука, 1980.   Т.~2.
714~с.

\bibitem{21-in}
Словарь русского языка: в~4~т.~/ Под ред. А.\,П.~Ев\-гень\-евой.~--- М.: Русский язык, 
 1981--1984. 
\bibitem{22-in}
\Au{Heine B., Kuteva T.} World lexicon of grammaticalization.~--- Cambridge: Cambridge 
University Press, 2002. 387~p.
\end{thebibliography}

 }
 }

\end{multicols}

\vspace*{-10pt}

\hfill{\small\textit{Поступила в~редакцию 15.10.23}}

\vspace*{8pt}

%\pagebreak

%\newpage

%\vspace*{-28pt}

\hrule

\vspace*{2pt}

\hrule



\def\tit{EVALUATING THE DEGREE OF~DISCOURSE RELATIONS SEMANTIC AFFINITY: 
METHODS AND~INSTRUMENTS}


\def\titkol{Evaluating the degree of~discourse relations semantic affinity: 
Methods and instruments}


\def\aut{O.\,Yu.~Inkova$^{1,2}$ and~M.\,G.~Kruzhkov$^1$}

\def\autkol{O.\,Yu.~Inkova and~M.\,G.~Kruzhkov}

\titel{\tit}{\aut}{\autkol}{\titkol}

\vspace*{-14pt}


\noindent
$^1$Federal Research Center ``Computer Science and Control'' of the Russian Academy of Sciences, 
44-2~Vavilov\linebreak
$\hphantom{^1}$Str., Moscow 119333, Russian Federation

\noindent
$^2$University of Geneva, 22 Bd des Philosophes, CH-1205 Geneva 4, Switzerland


\def\leftfootline{\small{\textbf{\thepage}
\hfill INFORMATIKA I EE PRIMENENIYA~--- INFORMATICS AND
APPLICATIONS\ \ \ 2023\ \ \ volume~17\ \ \ issue\ 4}
}%
 \def\rightfootline{\small{INFORMATIKA I EE PRIMENENIYA~---
INFORMATICS AND APPLICATIONS\ \ \ 2023\ \ \ volume~17\ \ \ issue\ 4
\hfill \textbf{\thepage}}}

\vspace*{3pt}




\Abste{The methods for evaluating semantic affinity of discourse relations are examined. The 
authors propose several approaches to this problem using two information resources: 
a~collection of structured definitions of logical-semantic relations (LSRs) formed by the authors
and the Supracorpora 
Database of Connectives incorporating\linebreak\vspace*{-12pt}}

\Abstend{corpus-based annotations of translation correspondences 
that include text fragments with LSR markers in Russian,
French, and Italian. It is demonstrated that when it comes to 
assessing the semantic affinity of LSRs, the following factors will be of a~higher priority: affiliation of 
distinctive features of LSRs with the same family in the structured definitions of relations; correspondences 
between markers of different LSRs in the source and target texts; and cases when different LSRs are 
regularly expressed by the same markers in different contexts. Of a~lesser importance is the factor of 
compatibility of different LSRs within the same context. It is assumed that based on the proposed 
methods, it will become possible to specify more precisely which distinguishing features of LSRs 
have the greatest impact on their potential semantic affinity.}

\KWE{supracorpora database; logical-semantic relations; connectives; annotation; faceted 
classification}


  \DOI{10.14357/19922264230412}{FXTSPZ}

\vspace*{-16pt}

\Ack

\vspace*{-3pt}

\noindent
The research was carried out using the infrastructure of the Shared Research Facilities ``High 
Performance Computing and Big Data'' (CKP ``Informatics'') of FRC CSC RAS (Moscow).


\vspace*{6pt}

  \begin{multicols}{2}

\renewcommand{\bibname}{\protect\rmfamily References}
%\renewcommand{\bibname}{\large\protect\rm References}

{\small\frenchspacing
 {%\baselineskip=10.8pt
 \addcontentsline{toc}{section}{References}
 \begin{thebibliography}{99}
\bibitem{1-in-1}
\Aue{Hobbs, J.\,R.} 1976. A~computational approach to discourse analyses. New York, NY: 
Department of Computer Science, City College, City University of New York. Research Report  
76-2. 28--38.
\bibitem{2-in-1}
\Aue{Hobbs, J.\,R.} 1978. Why is discourse coherent? Menlo Park, CA: SRI International. SRI 
Technical Note 176. 44~p.
\bibitem{3-in-1}
\Aue{Halliday, M.\,A.\,K., and R.~Hasan.} 1976. \textit{Cohesion in English}. London: Longman. 
374~p.


\bibitem{5-in-1} %4
\Aue{Mann, W.\,C., and S.\,A.~Thompson.} 1988. Rhetorical structure theory: Towards 
a~functional theory of text organization. \textit{Text} 8(3):243--281. doi: 
10.1515/text.1.1988.8.3.243.
\bibitem{6-in-1} %5
\Aue{Asher, N.} 1993. \textit{Reference to abstract objects in discourse}. Dordrecht: Kluwer. 
455~p.
\bibitem{4-in-1} %6
\Aue{Halliday, M.\,A.\,K.} 1994. \textit{An introduction to functional grammar}. 2nd ed. London: 
Edward Arnold. 434~p.

\bibitem{7-in-1}
PDTB Research Group. 2007. The Penn Discourse Treebank 2.0 annotation manual. Philadelphia, 
PA: Institute for Research in Cognitive Science, University of Pennsylvania. Technical Report 
IRCS-08-01. 104~p. Available at: {\sf https://www.cis.upenn.edu/$\sim$elenimi/pdtb-manual.pdf} 
(accessed November~28, 2023).
\bibitem{8-in-1}
\Aue{Breindl, E., A.~Volodina, and U.\,H.~Wa{\!\ptb{\!\ss}}ner.} 2014. \textit{Handbuch der 
deutschen Konnektoren~2: Semantik der deutschen Satzverkn$\ddot{\mbox{u}}$pfer}. Berlin: Walter de Gruyter. 
1327~p.
\bibitem{9-in-1}
\Aue{Inkova, O.\,Yu.} 2019. Logiko-semanticheskie otnosheniya: problemy klassifikatsii  
[Logical-semantic relations: Classification problems]. \textit{Svyaznost' teksta: mereologicheskie 
logiko-semanticheskie otnosheniya} [Text coherence: Mereological logical semantic relations]. 
Moscow: LRC Publishing House. 11--98.
\bibitem{10-in-1}
\Aue{Asher, N., and A.~Lascarides.} 2003. \textit{Logics of conversation}. Cambridge: Cambridge 
University Press. 526~p.
\bibitem{11-in-1}
\Aue{Carlson, L., and D.~Marcu.} 2001. Discourse tagging reference manual.  Marina del Rey, CA: Information Sciences Institute, University of Southern 
California. Technical Report 
ISI-TR-545.  87~p. Available at: {\sf https://www.isi.edu/~marcu/discourse/tagging-ref-manual.pdf} 
(accessed November~28, 2023).

\bibitem{13-in-1} %12
\Aue{Chiarcos, Ch.} 2014. Towards interoperable discourse annotation: Discourse features in the 
Ontologies of Linguistic Annotation. \textit{9th Conference (International) on\linebreak Language Resources 
and Evaluation Proceedings}. Eds. N.~Calzolari, K.~Choukri, T.~Declerck, \textit{et al.} Reykjavik, Iceland: 
European Language Resources Association. 4569--4577.
{ %\looseness=1

}

\bibitem{12-in-1} %13
\Aue{Benamara, F., and M.~Taboada.} 2015. Mapping different rhetorical relation annotations: 
A~proposal. \textit{4th Joint Conference on Lexical and Computational Semantics}. Eds. 
M.~Palmer, G.~Boleda, and P.~Rosso. Denver, CO, USA: Association for Computational 
Linguistics. 147--152. doi: 10.18653/v1/S15-1016.

\bibitem{14-in-1}
\Aue{Sanders, T., V.~Demberg, J.~Hoek, M.~Scholman, F.\,T.~Asr, S.~Zufferey, and  
J.~Evers-Vermeul.} 2018. Unifying dimensions in coherence relations: How various annotation 
frameworks are related. \textit{Corpus Linguist. Ling.} 17(1):1--71. doi: 10.1515/cllt-2016-0078.
\bibitem{15-in-1}
\Aue{Inkova, O.\,Yu.} 2021. Opredeleniya diskursivnykh otnosheniy: opyt Nadkorpusnoy bazy 
dannykh konnektorov [Definition of discursive relations: The experience of the supracorpora 
database of connectors]. \textit{Komp'yuternaya lingvistika i~intellektual'nye Tekhnologii: Po 
mat-lam ezhegodnoy Mezhdunar.  konf. ``Dialog''} [Computational Linguistics 
and Intellectual Technologies: Papers from the Annual Conference (International) ``Dialogue'']. 
Moscow: RGGU. 20(27):328--338.
\bibitem{16-in-1}
\Aue{Inkova, O.\,Yu., and M.\,G.~Kruzhkov.} 2021. Strukturirovannye opredeleniya 
diskursivnykh otnosheniy v~Nadkorpusnoy baze dannykh konnektorov [Structured definitions of 
discourse relations in the Supracorpora Database of Connectives]. \textit{Informatika i~ee 
Primeneniya~--- Inform. Appl.} 15(4):27--32. doi: 10.14357/ 19922264210404. EDN: EZJXVI.
\bibitem{17-in-1}
\Aue{Inkova, O.\,Yu., and M.\,G.~Kruzhkov.} 2023. Kriterii opredeleniya semanticheskoy blizosti 
diskursivnykh otnosheniy [Evaluation criteria for discourse relations semantic affinity]. 
\textit{Informatika i~ee Primeneniya~--- Inform. Appl.} 17(3):100--106. doi: 
10.14357/19922264230314. EDN: UJZJZI.

\pagebreak


\bibitem{18-in-1}
\Aue{Inkova, O.\,Yu., and V.\,A.~Nuriev.} 2018. Naskol'ko lingvospetsifichen soyuz \textit{khotya}? [To 
what extent is the conjunction \textit{khotya} language-specific?]. \textit{Komp'yuternaya lingvistika 
i~intellektual'nye tekhnologii: Po mat-lam ezhegodnoy Mezhdunar. konf. ``Dialog''} 
[Computational Linguistics and Intellectual Technologies: Papers from the Annual Conference 
(International) ``Dialogue'']. Moscow: RGGU. 17(24):254--266. 

\bibitem{20-in-1} %19
Chernyshev, V.\,I., ed. 1950--1965. \textit{Slovar' sovremennogo russkogo literaturnogo yazyka} 
[Dictionary of modern Russian literary language]. In 17~vols. Moscow, Leningrad: USSR Academy 
of Sciences Publishing House/Nauka.

\bibitem{19-in-1} %20
Shvedova, N.\,Yu., ed. 1980. \textit{Russkaya grammatika} [Russian grammar]. Moscow: Nauka. Vol.~2. 714~p.

\bibitem{21-in-1} %21
Evgen'eva, A.\,P., ed. 1981--1984. \textit{Slovar' russkogo yazyka} [Dictionary of the Russian 
language].  Moscow: Russkiy yazyk. 4~vols.


\bibitem{22-in-1}
\Aue{Heine, B., and T.~Kuteva.} 2002. \textit{World lexicon of grammaticalization}. Cambridge: 
Cambridge University Press. 387~p.

\end{thebibliography}

 }
 }

\end{multicols}

\vspace*{-6pt}

\hfill{\small\textit{Received October 5, 2023}} 

%\vspace*{-18pt}

\Contr

\vspace*{-4pt}

\noindent
\textbf{Inkova Olga Yu.} (b.\ 1965)~--- Doctor of Science in philology, senior scientist, Federal 
Research Center ``Computer Science and Control'' of the Russian Academy of Sciences,  
44-2~Vavilov Str., Moscow 119333, Russian Federation; faculty member, University of Geneva, 
22~Bd des Philosophes, CH-1205 Geneva~4, Switzerland; \mbox{olyainkova@yandex.ru}

\vspace*{3pt}

\noindent
\textbf{Kruzhkov Mikhail G.} (b.\ 1975)~--- senior scientist, Federal Research Center ``Computer 
Science and Control'' of the Russian Academy of Sciences, 44-2~Vavilov Str., Moscow 119333, 
Russian Federation; \mbox{magnit75@yandex.ru}


\label{end\stat}

\renewcommand{\bibname}{\protect\rm Литература}  %12
\def\stat{zatsar}

\def\tit{СИСТЕМОТЕХНИЧЕСКИЕ ПОДХОДЫ К~СОЗДАНИЮ\\ 
СИСТЕМЫ ПОДДЕРЖКИ ПРИНЯТИЯ РЕШЕНИЙ\\ НА~ОСНОВЕ 
СИТУАЦИОННОГО АНАЛИЗА}

\def\titkol{Системотехнические подходы к~созданию 
системы поддержки принятия решений на~основе 
ситуационного анализа}

\def\aut{А.\,А.~Зацаринный$^1$, А.\,П.~Сучков$^2$}

\def\autkol{А.\,А.~Зацаринный, А.\,П.~Сучков}

\titel{\tit}{\aut}{\autkol}{\titkol}

\index{Зацаринный А.\,А.}
\index{Сучков А.\,П.}
\index{Zatsarinny A.\,A.}
\index{Suchkov A.\,P.}


%{\renewcommand{\thefootnote}{\fnsymbol{footnote}} \footnotetext[1]
%{Работа выполнена при финансовой поддержке РФФИ (проект 16-37-00485).}}


\renewcommand{\thefootnote}{\arabic{footnote}}
\footnotetext[1]{Институт проблем информатики Федерального исследовательского центра 
<<Информатика и~управ\-ле\-ние>> Российской академии наук, \mbox{AZatsarinny@ipiran.ru}}
\footnotetext[2]{Институт проблем информатики Федерального исследовательского центра 
<<Информатика и~управ\-ле\-ние>> Российской академии наук, \mbox{Asuchkov@ipiran.ru}}

      

\Abst{Обсуждаются вопросы создания сис\-тем поддержки принятия решений 
(СППР) на основе ситуационного анализа текущей и~прогнозируемой обстановки 
в~контролируемом пространстве органа управления. Как правило, такие сис\-те\-мы 
управления в~режиме реального времени опираются на ситуационные центры (СЦ)~--- 
совокупность информационных, программных и~аппаратных средств, а также 
обслуживающего персонала, реализующих информационные технологии по мониторингу 
обстановки, ее ситуационному анализу для выработки решений и~алгоритмов применения 
управляющих воздействий. Рассмотрены содержательные характеристики составляющих 
частей СППР, реализующих полный цикл управления от целеполагания до контроля 
исполнения принимаемых решений. Отмечается, что реализация СППР зависит от уровня 
сис\-те\-мы управ\-ле\-ния~--- стратегического, оперативного, тактического, базового, приводятся 
функциональные особенности и~способы анализа обстановки на различных уровнях 
сис\-те\-мы управ\-ления.}

\KW{ситуационный анализ; сис\-те\-ма поддержки принятия решений; сис\-те\-ма управ\-ле\-ния; 
ситуационный центр}

\DOI{10.14357/19922264160411} 


\vskip 10pt plus 9pt minus 6pt

\thispagestyle{headings}

\begin{multicols}{2}

\label{st\stat}

\section{Введение}

     В Стратегии национальной безопасности Российской Федерации 
(утверждена Указом Президента Российской Федерации от~31~декабря 
2015~г. №\,683)~[1] определено, что информационную основу реализации 
Стратегии составляет федеральная информационная сис\-те\-ма стратегического 
планирования, включающая в~себя информационные ресурсы органов 
государственной власти и~органов местного самоуправления, сис\-те\-мы 
распределенных СЦ и~государственных научных 
организаций. В~рамках такой сис\-те\-мы должна быть реализована поддержка 
управленческих решений в~интересах центральных органов исполнительной 
власти на основе организации взаимодействия региональных 
и~ведомственных СЦ, а~также других информационных 
сис\-тем. Для эффективного решения этой задачи необходимо создание СППР 
в~со\-ста\-ве СЦ и~придания им принципиально новых качеств. 
     
     В связи с~этим целью статьи является обоснование сис\-те\-мо\-тех\-ни\-че\-ских 
и~методических подхо\-дов к~структурному и~функциональному составу\linebreak 
СППР и~ее месту в~составе СЦ, обеспечивающих 
информационно-аналитическую поддержку принятия управленческих 
решений в~рамках государственного управления, стратегического 
планирования и~мониторинга реализации документов стратегического 
планирования в~Российской Феде-\linebreak рации. 

\vspace*{-6pt}
     
\section{Базовые понятия }

\vspace*{-2pt}

    При рассмотрении сис\-тем\-ных и~методических вопросов создания СППР, 
основанных на ситуационном анализе, в~статье используется ряд базовых 
понятий: событие, обстановка, ситуация, угроза, управление, цели 
управления и~др.~[2]. 
    
    \textit{Ситуация} определяется состоянием взаимосвязанных 
\textit{элементов обстановки} в~контролируемом пространстве; изменения 
обстановки определя-\linebreak ются \textit{событиями}, образующими некоторые 
разворачивающиеся во времени наблюдаемые и~ре\-гист\-ри\-ру\-емые потоки. При 
этом под \textit{управлением}\linebreak понимается \textbf{целенаправленное} 
воздействие органа управления на подчиненные ему или взаимодействующие 
элементы обстановки (ресурсы). 
    
    Совокупность ситуаций в~сис\-те\-ме управ\-ле\-ния распадается на текущие, 
прогнозируемые и~целевые ситуации. При этом текущие ситуации являются 
результатом наблюдения и~регистрации событий, прогнозируемые 
определяются методами ситуационного анализа, а целевые отражают 
краткосрочные, среднесрочные и~долгосрочные цели управления. Последнее 
немаловажно, так как зачастую ситуационный анализ понимается как 
обеспечение реакций сис\-те\-мы управ\-ле\-ния на чрезвычайные ситуации после 
того, как они сложились. Однако теория ситуационного подхода 
предполагает учет <<планируемой и~прогнозируемой обстановки>>, 
отражающей стратегические, тактические и~оперативные \textit{цели 
управления}, а~также учет факторов самоорганизации управляющего 
сегмента сис\-те\-мы, определяющих стимулы для достижения этих 
целей~[2,~3]. Под \textit{угрозой} в~процессах управления понимается 
ситуация или совокупность ситуаций, развитие которых противоречит целям 
управления и~отдаляет текущее состояние от целевого.
    
    В конце 1970-х~гг.\ была создана модель сис\-те\-мы управ\-ле\-ния  
<<наблю\-де\-ние--ори\-ен\-ти\-ро\-ва\-ние--ре\-ше\-ние--дей\-ст\-вие>> 
(НОРД) для принятия решений при ведении боевых действий~[4, 5]. 
В~настоящее время эта модель активно используется во многих сис\-те\-мах 
управ\-ле\-ния разных отраслей~[6]. В~рамках ситуационного подхода 
к~управлению предложена модифицированная модель, включающая 
дополнительную стадию управляющего цикла~--- целеполагание~[7].
    
    \textbf{Целеполагание} (стадия~Ц)~--- формализованное представление 
целевых показателей, установление количественных 
и~временн$\acute{\mbox{ы}}$х критериев их достижения.
    
    \textbf{Мониторинг} (стадия~М)~--- это процесс сбора информации об 
окружающей среде в~контролируемом пространстве, включая состояние 
целевых показателей. Стадия М также принимает внутренние инструкции от 
стадии анализа (А), так же как и~поддержку от процессов~Р и~Д. 
    
    \textbf{Анализ} (стадия~А)~--- оценка ситуации (типовая, нетиповая), 
анализ существующего опыта, пополнение опыта, обеспечивает внутреннюю 
поддержку~М (корректировка фильтров).
    
    \textbf{Решение} (стадия~Р)~--- это процесс осуществления выбора 
среди гипотез о состоянии окружающей среды и~возможной реакции на него. 
Процесс~Р руководствуется прямой внутренней связью с~процессом~А 
и~обеспечивает внутреннюю поддержку процесса~М, возможна 
корректировка целевых показателей (стадия~Ц).
    
    \textbf{Действие} (стадия~Д)~--- это процесс выполнения выбранной 
реакции путем взаимодействия с~окружающей средой. Действие принимает 
внутренние руководства от процесса~А, также оно напрямую связано с~Р. 
Оно обеспечивает внутреннюю поддержку~Ц и~М.
    
    Особенности реализации цикла управления в~сис\-те\-ме, реализующей 
процессы стратегического планирования и~управления, заключаются в~том, 
что содержательно стадии~Ц, А и~Р реализуются непосредственно высшими 
органами исполнительной власти. Это означает осуществление сле\-ду\-ющих 
основных функций:
    \begin{itemize}
\item  доведение до подчиненных органов данных целеполагания 
и~стратегического планирования на основе их формализации 
и~регламентации обмена (стадия~Ц);
\item регламентированный сбор данных о состоянии целевых показателей от 
органов испол\-нительной власти и~об обстановке в~конт\-ро\-ли\-ру\-емом 
пространстве по определенному\linebreak регламенту и~в~режиме реального времени 
(стадия~М);
\item обмен аналитическими данными участников\linebreak стратегического 
планирования по целеполаганию, прогнозированию, планированию 
и~программированию~--- федеральных органов исполнительной власти 
(ФОИВ), субъектов Россий\-ской Федерации и~муниципальных образований, 
отраслей экономики и~сфер государственного и~муниципального управления 
(стадия~А);
\item  доведение до подчиненных органов принимаемых решений по 
применению сил и~средств и,~возможно, по корректировке стратегических 
планов с~целью достижения поставленных стратегических целей (стадия~Р) 
и~контроль исполнения решений (стратегических планов) на основе 
докладов (стадия~Д).
    \end{itemize}
    
    На тактическом и~базовом уровнях управления осуществляются,  
во-пер\-вых, реализация функ-\linebreak ций мониторинга контролируемого 
пространства и~организа\-ции учета контролируемых объектов (стадия~М),  
во-вто\-рых, специальный анализ фактографических данных о конкретных 
элементах обстановки, формализованных в~виде семантической сети, 
позволяющий выявлять неочевидные связи между элементами обстановки, 
определять схожие про\-стран\-ст\-вен\-но-со\-бы\-тий\-ные ситуации, выявлять 
ассоциативные связи и~закономерности с~\mbox{целью} поддержки процессов 
принятия решений (стадия~А), в-треть\-их, процессы принятия решений 
по планированию применения сил и~средств на период времени и~по 
складывающейся обстановке в~соответствии с~указаниями вышестоящих 
органов (стадии~Р и~Д).

\begin{figure*}[b] %fig1
\vspace*{1pt}
\begin{center}
\mbox{%
\epsfxsize=160.901mm
\epsfbox{zac-1.eps}
}
\end{center}
\vspace*{-9pt}
\Caption{Обобщенная функциональная структура СЦ}
\end{figure*}
    

\section{Ситуационный центр как составляющая современной системы 
управления}
    
    Определим СЦ сис\-те\-мы управ\-ле\-ния как совокупность 
информационных, программных и~аппаратных средств, а~также 
обслуживающего персонала, реализующих информационные технологии\linebreak по 
мониторингу обстановки, ее ситуационному анализу для выработки решений 
и~алгоритмов применения управляющих воздействий с~\mbox{целью} эффективной 
реализации функций управления и~минимизации ущерба от угроз в~зоне 
ответствен\-ности\linebreak органа управ\-ле\-ния, доведения их до объектов управ\-ле\-ния 
и~контроля исполнения,
    
    По сути дела, СЦ является составной частью сис\-те\-мы 
управ\-ле\-ния, осуществляющей автоматизацию ряда функций всего органа 
управления и~отдельных должностных лиц.
    
    Исходя из накопленного в~Институте проблем информатики РАН опыта 
разработки крупных информационных сис\-тем в~интересах органов 
государственной власти, в~организационной структуре СЦ можно выделить 
четыре основных функциональных сегмента (рис.~1)~\cite{8-zat}:
    \begin{enumerate}[(1)]
\item сегмент руководства (лиц, принимающих решения, ЛПР); 
\item сегмент мониторинга состояния контролируемых объектов 
и~окружающей среды и~сбора информации; 
\item сегмент ситуационного анализа и~сис\-те\-ма\-ти\-за\-ции информации;
\item сегмент администрирования и~эксплуатации.
\end{enumerate}
    При этом СППР базируется на ресурсах всех четырех сегментов. Вместе 
с тем центральным звеном СЦ и~его СППР, обеспечивающим реализацию 
основной функции сис\-те\-мы управ\-ле\-ния по эффективному управлению 
силами и~средствами, является \textit{сегмент ситуационного анализа 
и~сис\-те\-ма\-ти\-за\-ции информации}. Он должен обеспечивать реализацию 
следующих функций:
    \begin{itemize}
\item возможность визуализации результатов анализа обстановки на 
индивидуальных и~коллективных средствах отображения;
\item во взаимодействии с~сегментом мониторинга получение данных 
о~состоянии обстановки от собственных (субъективных и~объективных 
средств наблюдения и~контроля) и~внешних по отношению к~сис\-те\-ме 
источников информации (ведомственных, межведомственных, 
международных, независимых и~др.);
\item извлечение фактов, структуризация и~формализация разнородных 
данных о~значимых событиях в~соответствии с~выбранной информационной 
моделью предметной области;
\item формирование хранилищ ситуационных данных;
\item формирование способов визуализации агрегированных данных 
о~складывающейся обстановке для ЛПР и~оперативного состава;
\item формирование отчетности и~служебной документации;
\item расчет первичных и~интегральных показателей обстановки, а~также 
статистическая оценка характеристик ненаблюдаемых элементов обстановки;
\item решение задач перспективного планирования, контроль исполнения 
решений по планированию;
\item выявление значимых ситуаций, их ранжирование по степени 
важности, видам и~типам, формирование текущего перечня 
аналитических задач по складывающейся обстановке и~по поручениям 
руководства;
\item  выработка вариантов решений по применению управляющих 
воздействий для достижения целевых ситуаций, формирование спо\-собов 
наглядного представления вариантов\linebreak реше\-ния для ЛПР (оперативное 
планирование);
\item прогнозирование развития обстановки и~процесса реализации целей 
сис\-те\-мы управ\-ле\-ния на основе сформированных ситуационных моделей 
и~моделей угроз, в~том числе и~с~учетом применения выработанных 
вариантов решений;
\item обеспечение процессов принятия решений комплексом  
ин\-фор\-ма\-ци\-он\-но-рас\-чет\-ных задач (ИРЗ).
    \end{itemize}
    
    Наряду с~перечисленными в~СППР СЦ реализуются важнейшие функции 
администрирования аналитической под\-сис\-те\-мы~СЦ:
    \begin{itemize}
\item формирование и~корректировка сис\-те\-мы целей управ\-ления;
\item формирование, настройка и~корректировка сис\-те\-мы моделей целей 
управления, обстановки, ситуаций и~угроз;
\item формирование, настройка и~корректировка сис\-те\-мы расчетных 
показателей, характеризующих обстановку и~ее элементы;
\item формирование, настройка и~корректировка сис\-те\-мы критериев, 
пороговых значений, эвристик, параметров расчетных алгоритмов.
\end{itemize}

\section{Целеполагание~--- определение целей системы управления}

    Под \textit{целью ситуационного анализа} предлагается понимать 
поддержку процессов принятия решений для достижения поставленных 
целей путем применения доступных в~сис\-те\-ме управ\-ле\-ния сил и~средств 
(ресурсов).
    
    Целесообразность деятельности сис\-те\-мы управ\-ле\-ния определяется 
иерархической сис\-те\-мой целей\linebreak (подцелей). Для ФОИВ она задается 
законодательно, а также при определении приоритетов в~орга\-низации 
деятельности сис\-те\-мы управ\-ле\-ния первым\linebreak лицом (руководителем). 
Формирование сис\-те\-мы целей сопровождается формированием сис\-те\-мы 
показателей реализации целей (подцелей) и~критериев достижения целей. 
Показатели являются вычисляемыми величинами как функции обстановки 
или экспертно оцениваемые параметры. Критерии достижения обычно 
формулируются как некие пороговые плановые значения на временн$\acute{\mbox{о}}$й 
шкале.
    
    Эффективность сис\-те\-мы управ\-ле\-ния в~каждый момент времени 
определяется, во-пер\-вых, степенью достижения пороговых значений 
планируемых целевых показателей, во-вто\-рых, объемом затрачиваемых 
ресурсов на единицу оптимизируемого целевого показателя.
    
    Цели управления формируются на основании сис\-тем\-но\-го анализа  
нор\-ма\-тив\-но-пра\-во\-вых основ функционирования сис\-те\-мы управ\-ле\-ния. 
Цели управления образуют дерево целей, детализация которого (число 
уровней) определяется воз\-мож\-ностью декомпозиции конкретной цели на 
значимые подцели. Цели и~подцели должны обладать индикаторами 
состояния (как правило, \%) и~весовыми коэффициентами доли подцели 
в~реализации всей цели. Цели могут включать ориентиры развития сис\-те\-мы 
управления, установленные первым лицом.

\begin{figure*} %fig2
\vspace*{1pt}
\begin{center}
\mbox{%
\epsfxsize=165.008mm
\epsfbox{zac-2.eps}
}
\end{center}
\vspace*{-9pt}
\Caption{Обобщенная структура сис\-те\-мы целей}
\end{figure*}
    
    Выбор структуры сис\-те\-мы целей предлагается осуществлять с~учетом 
следующих соображений.
    \begin{enumerate}[1.]
    \item Цели управления сложной управляющей сис\-те\-мой определяются 
нор\-ма\-тив\-но-пра\-во\-вы\-ми документа\-ми, регламентирующими ее 
функционирование, и, как правило, образуют \textbf{иерархическую 
структуру} в~соответствии со структурой направлений деятельности 
(рис.~2).
    \item Ситуационный подход к~управлению предполагает реагирование на 
складывающуюся обстановку в~режиме реального времени. В~силу этого, 
помимо фиксированных целей в~сис\-те\-ме управ\-ле\-ния необходим механизм 
формирования \textbf{динамических целей}, отражающих процесс 
нормализации складывающихся чрезвычайных ситуаций и~присутствующих 
в~сис\-те\-ме целеполагания на период существования ситуации.
    \item В~концепции <<управления по целям>> эффективность 
целеполагания проверяется по критериям SMART~\cite{9-zat}: цель должна 
быть конкретная, измеримая (подразумевает количественную измеримость 
результата), достижимая (должна быть выполнимой), реалистичная 
(достижение цели должно быть обеспечено ресурсами), привязанная  
к~точ\-ке/ин\-тер\-ва\-лу времени.
    \end{enumerate}
    
    Данный подход накладывает \textbf{требования на атрибуты целей} 
в~части формирования количественных характеристик их достижения, 
плановых характеристик, критериев достижения (см.\ рис.~2). 
    


    Основные атрибуты цели:\\[-14pt]
    \begin{itemize}
\item описание~--- дает определение и~конкретизацию цели;\\[-14pt]
\item весовой коэффициент~--- определяет вклад подцели 
в~вышестоящую цель;\\[-14pt]
\item индикатор~--- задает количественный показатель достижения 
результата;\\[-14pt]
\item критерий~--- задает способ определения достижения результата 
с~помощью индикатора;\\[-14pt]
\item план~--- определяет количественные значения критерия 
достижения цели и~требуемые вре\-мен\-н$\acute{\mbox{ы}}$е параметры.
\end{itemize}

\vspace*{-9pt}

\section{Анализ обстановки и~выработка вариантов решений}

\vspace*{-2pt}

\subsection{Мониторинг обстановки}

\vspace*{-1pt}

     В процессе мониторинга контролируемых элементов обстановки 
осуществляются (рис.~3):
     \begin{itemize}
\item сбор данных о состоянии контролируемых объектов, анализ 
неструктурированной информации с~целью извлечения фактов и~знаний; 
\item постановка объектов на контроль (оператор, автоматически); 
\item отображение контролируемых объектов по шкале состояний и~по 
критериям~--- соотношение текущего или прогнозируемого значения 
индикатора (интегрального показателя) и~сис\-те\-мы порогов, обеспечивающих 
градацию состояния (<<типовое>>, <<чрезвычайное>>, <<критическое>> 
или другие подобные).
\end{itemize}

    По данным мониторинга контролируемых элементов обстановки из 
различных источников формируется \textit{хранилище} СППР, которое 
пред\-став\-ляет собой совокупность взаимоувязанных на\linebreak основе единого 
информационного и~лингвистического обеспечения баз данных (БД): 
обстановки (события, ситуации, элементы окружающей\linebreak среды), сил и~средств 
(свои силы и~средства, противодействующие силы и~средства, так\-ти\-ко-тех\-ни\-че\-ские
характеристики), целевых 
показателей (первичные показатели, интегральные показатели,\linebreak индикаторы, 
критерии), типовых решений (типовые решения, конкретные решения), 
ретроспективная (нормализованные исторические данные, архив 
обстановки), нормативных документов, биб\-лио\-те\-ка математических моделей.

\vspace*{-6pt}

\subsection{Поддержка процесса принятия решений}

\vspace*{-2pt}

    На основе мониторинга текущей обстановки и~поступления событийной 
информации в~хранилище осуществляется расчет заданных в~сис\-те\-ме 
первичных и~интегральных показателей обстановки и~целевых показателей 
в~двух режимах: по регламенту (с~определенной периодичностью) и~по 
запросу пользователя с~использованием блоков расчетов, блока первичного, 
краткосрочного, среднесрочного и~долгосрочного анализа, блока 
визуализации  и~блока поддержки принятия решений (рис.~4).\linebreak\vspace*{-12pt}


\pagebreak

\end{multicols}
\begin{figure*} %fig3
\vspace*{1pt}
\begin{center}
\mbox{%
\epsfxsize=157.334mm
\epsfbox{zac-3.eps}
}
\end{center}
\vspace*{-6pt}
\Caption{Мониторинг обстановки}
\vspace*{6pt}
\end{figure*}

\begin{multicols}{2}




    
    При этом реализуются следующие функции.
    \begin{enumerate}[1.]
\item  Создание (привязка существующих) динамических моделей 
обстановки:
\begin{itemize}
\item моделей <<нормальной>> обстановки;
\item моделей для прогноза обстановки;
\item моделей для анализа трендов, циклов, аномалий обстановки.
\end{itemize}

    \item Проведение оперативного анализа текущей обстановки 
с~использованием математических методов (см.\ рис.~4):
\begin{itemize}
\item анализ отклонения от <<нормальной>> текущей обстановки;
\item прогноз развития обстановки;
\item анализ трендов, циклов, аномалий обстановки;
\item выявление и~идентификация значимых ситуаций 
на основе выявления типовых кон-\linebreak\vspace*{-12pt}

\columnbreak

\noindent
фигураций событий 
и~правил идентификации, идентификация типа ситуации, 
фор\-ми\-ро\-ва\-ние неотложных целей.\\[-7.5pt]
\end{itemize}
    \item Визуализация и~индикация состояний контролируемых объектов 
    с~использованием полученных результатов анализа (наглядное пред\-став\-ле\-ние 
текущей с~индикацией ситуаций,\linebreak требующих принятия решения или 
применения типовых решений).\\[-6pt]
    \item Выработка вариантов решений по складыва\-ющейся обстановке 
(решение содержит динамическую цель, перечень подцелей (с~весами~--- 
доли подцели в~реализации всей цели),\linebreak сроки достижения подцелей, 
ответственных, совокупность типовых уведомлений и~рапортов):
\begin{itemize}
\item применение типовых решений по типовым ситуациям (привязка их 
к~реальной обстановке);
\end{itemize}
\end{enumerate}



\pagebreak

\end{multicols}

\begin{figure*} %fig4
\vspace*{1pt}
\begin{center}
\mbox{%
\epsfxsize=164.07mm
\epsfbox{zac-4.eps}
}
\end{center}
\vspace*{-11pt}
\Caption{Структура блока принятия решений}
\vspace*{-3pt}
\end{figure*}

\begin{multicols}{2}

\noindent
\begin{enumerate}
\item[\ ]
\vspace*{-13pt}
\begin{itemize}
\item выработка вариантов решения экспертным путем в~случае критических 
и чрезвычайных ситуаций;\\[-15pt]
\item анализ развития обстановки с~учетом вариантов решений (прогноз 
благоприятного и~неблагоприятного развития обстановки, расчет 
вероятностей выполнения задач, оценка вариантов решений).
\end{itemize}
\end{enumerate}

\vspace*{-6pt}

    \subsection{Реализация решений }
    
    \vspace*{-2pt}
    
    На данной стадии осуществляется мониторинг процессов реализации 
решений по краткосрочным, среднесрочным и~долгосрочным планам 
(решение содержит цель, перечень подцелей (с~весами~--- доли подцели 
в~реализации всей цели), сроки достижения подцелей, ответственных, виды 
отчетности):
\begin{itemize}
\item сбор информации по ходу выполнения плана (отчетность), 
визуализация хода исполнения, контроль исполнения;\\[-15pt]
\item сравнительный анализ показателей плана по целям и~подцелям 
и~текущей обстановки, включая расчет степени реализации плана 
и~прогнозирование возможности реализации плана;\\[-15pt]
\item реализация обратной связи по уточнению решения по планированию 
с~целью обеспечения выполнения плана;\\[-15pt]
\item доведение уточненного решения (уведомления) и~контроль исполнения.
\end{itemize}

    Мониторинг реализации решений по ситуациям (решение содержит 
динамическую цель, перечень подцелей (с~весами~--- доли подцели 
в~реализации всей цели), сроки достижения подцелей, ответственных, 
совокупность типовых уведомлений и~рапортов): 
    \begin{itemize}
\item сбор информации по ходу выполнения решения (рапорты), 
визуализация хода исполнения, контроль исполнения;
\item сравнительный анализ показателей по целям и~подцелям и~текущей 
обстановки, включая расчет степени реализации решения и~прогнозирование 
возможности реализации решения;
\item реализация обратной связи по уточнению решения по ситуации с~целью 
обеспечения выполнения плана.
\item доведение уточненного решения (уведомления) и~контроль исполнения.
\end{itemize}

\vspace*{-6pt}

\section{Заключение}

\noindent
\begin{enumerate}[1.]
\item В современных условиях развития информационных сис\-тем особую 
значимость приобретает актуальность исследования сис\-те\-мо\-тех\-ни\-че\-ских 
и~технологических вопросов создания в~составе СЦ
СППР.
\item Важнейшей методологической и~концептуальной основой СППР 
является полнофункциональный цикл управления, включающий стадии 
целеполагания, мониторинга обстановки, анализа обстановки, выработки 
вариантов решений и~их реализации.
\item В СППР реализуются следующие функциональные задачи:
\begin{itemize}
\item мониторинг контролируемых элементов обстановки;
\item расчет характеристик событийной информации (первичные 
и~интегральные показатели текущей обстановки и~состояния 
целей);
\item визуализация текущего состояния обстановки;
\item визуализация текущего состояния индикаторов целей;
\item блок анализа и~принятия решений.
\item мониторинг контролируемых решений;
\item формирование документов и~отчетов.
\end{itemize}
\item Важнейшим сис\-те\-мо\-обра\-зу\-ющим компонентом СППР является 
хранилище, формируемое в~автоматизированном режиме из различных 
источников в~виде совокупности взаимоувязанных на основе единого 
информационного и~лингвистического обеспечения БД (о~событиях, 
силах и~средствах, целевых показателях и~критериях, типовых решений, 
ретроспективной информации, нормативных документов, математических 
моделей).
\item Предложенные в~статье сис\-те\-мо\-тех\-ни\-че\-ские подходы и~решения 
апробированы в~рамках нескольких проектов по созданию крупных 
территориально распределенных  
ин\-фор\-ма\-ци\-он\-но-ана\-ли\-ти\-че\-ских сис\-тем специального 
назначения.
\end{enumerate}

\vspace*{-6pt}

{\small\frenchspacing
 {%\baselineskip=10.8pt
 \addcontentsline{toc}{section}{References}
 \begin{thebibliography}{9}

\bibitem{1-zat}
Стратегия национальной безопасности Российской Федерации. Утверждена Указом 
Президента Российской Федерации от 31~декабря 2015~г. №\,683. 
\bibitem{2-zat}
\Au{Зацаринный А.\,А., Сучков А.\,П.} Некоторые подходы к~ситуационному анализу 
потоков событий~// Открытое образование, 2012. №\,1. С.~39--45.
\bibitem{3-zat}
\Au{Бир С.\,Э.} Мозг фирмы~/
Пер. с~англ.~--- М.: Радио и~связь, 1993. 416~с.
(\Au{Beer~S.}  {Brain of the firm}.~--- Allen Lane, The Penguin Press, London; Herder 
and Herder, USA, 1972. 416~p.)

\bibitem{5-zat}%4
\Au{Grant Т., Kooter В.} Comparing OODA \& other models as operational view~C2 
architecture~// 10th Command and Control Research Technology Symposium (International) 
Proceedings.~--- McLean, VA, USA, 2005.
\bibitem{4-zat} %5
\Au{Ивлев А.\,А.} Основы теории Бойда. Направления развития, применения 
и~реализации.~--- SlideShare, 2008. 64~с. {\sf  
http://www.slideshare.net/defensenetwork/ss-10380168}.
\bibitem{6-zat}
\Au{Босов А.\,В., Зацаринный А.\,А., Сучков~А.\,П.} Некоторые общие подходы 
к~формированию функциональных требований к~ситуационным центрам и~их 
реализации~// Системы и~средства информатики, 2010. Вып.~20. №\,3. С.~98--125.
\bibitem{7-zat}
\Au{Сучков А.\,П.} Формирование сис\-те\-мы целей для ситуационного управ\-ле\-ния~// 
Сис\-те\-мы и~средства информатики, 2013. Т.~23. №\,2. С.~171--182.
\bibitem{8-zat}
\Au{Зацаринный А.\,А., Сучков~А.\,П., Козлов~С.\,В.} Особенности проектирования 
и~функционирования сис\-те\-мы ситуационных центров~// Системы высокой доступности, 
2012. Т.~8. №\,1. С.~12--21.
\bibitem{9-zat}
\Au{Doran G.\,T.} There's a~S.M.A.R.T.\ way to write management's goals and objectives~// 
Manag. Rev., 1981. Vol.~70. Iss.~11. P.~35--36.
 \end{thebibliography}

 }
 }

\end{multicols}

\vspace*{-6pt}

\hfill{\small\textit{Поступила в~редакцию 23.08.16}}

%\vspace*{8pt}

\newpage

\vspace*{-24pt}

%\hrule

%\vspace*{2pt}

%\hrule

%\vspace*{8pt}


\def\tit{SYSTEMS ENGINEERING APPROACHES TO~THE~ESTABLISHMENT 
OF~A~SYSTEM FOR~DECISION SUPPORT BASED ON~SITUATIONAL ANALYSIS}

\def\titkol{Systems engineering approaches to~the~establishment 
of~a~system for~decision support based on~situational analysis}

\def\aut{A.\,A.~Zatsarinny and A.\,P.~Suchkov}

\def\autkol{A.\,A.~Zatsarinny and A.\,P.~Suchkov}

\titel{\tit}{\aut}{\autkol}{\titkol}

\vspace*{-9pt}


\noindent
Institute of Informatics Problems, 
Federal Research Center ``Computer Sciences and Control'' of the 
Russian Academy of Sciences, 44-2~Vavilov Str., Moscow 119333, 
Russian Federation



\def\leftfootline{\small{\textbf{\thepage}
\hfill INFORMATIKA I EE PRIMENENIYA~--- INFORMATICS AND
APPLICATIONS\ \ \ 2016\ \ \ volume~10\ \ \ issue\ 4}
}%
 \def\rightfootline{\small{INFORMATIKA I EE PRIMENENIYA~---
INFORMATICS AND APPLICATIONS\ \ \ 2016\ \ \ volume~10\ \ \ issue\ 4
\hfill \textbf{\thepage}}}

\vspace*{3pt}

 
\Abste{The article discusses the issues of decision-making support systems (DMSS) 
creation based on the situational analysis of the current and projected situation in the 
controlled space. Typically, such control systems in real time are based on situational 
centers, which are sets of information, software, hardware, and staff implementing 
information technology to monitor the situation and its situational analysis to develop 
solutions and algorithms application of control actions. The paper considers 
characteristics of the DMSS components, implementing the full management cycle from 
goal setting to execution control decisions. It is noted that the implementation of the 
decision support system depends on the level of management~--- strategic, operational, tactical, basic, and 
functional features and methods of analysis of the situation at different levels of the 
control system.}

\KWE{situational analysis; system of decision-making process support; management 
system; situational center}

\DOI{10.14357/19922264160411} 

%\vspace*{-9pt}

%\Ack
%\noindent


%\vspace*{3pt}

  \begin{multicols}{2}

\renewcommand{\bibname}{\protect\rmfamily References}
%\renewcommand{\bibname}{\large\protect\rm References}

{\small\frenchspacing
 {%\baselineskip=10.8pt
 \addcontentsline{toc}{section}{References}
 \begin{thebibliography}{9}

\bibitem{1-zat-1}
Strategiya natsional'noy bezopasnosti Rossiyskoy Fe\-de\-ra\-tsii [The National Security Strategy of 
the Russian Federation]. Approved by the Decree of the President of the Russian Federation 
No.\,683, 31.12.2015.
\bibitem{2-zat-1}
\Aue{Zatsarinny, A.\,A., and A.\,P.~Suchkov.} 2012. Nekotorye podkhody k~situatsionnomu 
analizu potokov sobytiy [Some approaches to the situational analysis of the flows of events]. 
\textit{Otkrytoe obrazovanie} [Open Education] 1:39--45.
\bibitem{3-zat-1}
\Aue{Beer, S.} 1972. \textit{Brain of the firm}. Allen Lane, The Penguin Press, London; Herder 
and Herder, USA. 416~p. 

\bibitem{5-zat-1}
\Aue{Grant, Т., and B. Кoote.} 2005. Comparing OODA \& other models as operational view C2 
architecture. \textit{10th Command and Control Research Technology Symposium 
(International) Proceedings}. McLean, VA. USA. 
\bibitem{4-zat-1}
\Aue{Ivlev, A.\,A.} 2008. \textit{Osnovy teorii Boyda. Napravleniya razvitiya, primeneniya 
i~realizatsii} [Fundamentals of the theory of Boyd. Areas of development, application, and 
implementation]. SlideShare. Available at: {\sf http://www.slideshare.net/defensenetwork/ss-10380168} (accessed  October~29, 2016).
\bibitem{6-zat-1}
\Aue{Bosov, A.\,V., A.\,A.~Zatsarinny, A.\,P.~Suchkov}. 2010. Nekotorye obshchie podkhody 
k~formirovaniyu funktsional'nykh trebovaniy k~situatsionnym tsentram i~ikh realizatsii [Some 
common approaches to the formation of functional requirements for situation centers and their 
implementation]. \textit{Sistemy i~Sredstva Informatiki~--- Systems and Means of Informatics} 
20(3):98--125.
\bibitem{7-zat-1}
\Aue{Suchkov, A.\,P.} 2013. Formirovanie sistemy tseley dlya si\-tu\-a\-tsi\-on\-no\-go upravleniya 
[The formation of the objective system to situational management]. \textit{Sistemy i~Sredstva 
Informatiki~--- Systems and Means of Informatics} 23(2):171--182.
\bibitem{8-zat-1}
\Aue{Zatsarinny, A.\,A., A.\,P.~Suchkov, and S.\,V.~Kozlov}. 2012. Osobennosti proektirovaniya 
i~funktsionirovaniya sistemy situatsionnykh tsentrov [Features of the design and functioning of 
the situational centers ]. \textit{Sistemy Vysokoy Dostupnosti} [High Availability Systems]  
8(1):12--21.
\bibitem{9-zat-1}
\Aue{Doran, G.\,T.} 1981. There's a~S.M.A.R.T. way to write management's goals and 
objectives. \textit{Manag. Rev.} 70(11):35--36.
\end{thebibliography}

 }
 }

\end{multicols}

\vspace*{-6pt}

\hfill{\small\textit{Received August 23, 2016}}

\vspace*{-12pt}

\Contr

\noindent
\textbf{Zatsarinny Alexander A.} (b.\ 1951)~--- Doctor of Science in technology, 
professor, 
Deputy Director, Federal Research Center ``Computer Sciences and Control'' of the 
Russian Academy of Sciences, 44-2~Vavilov Str., Moscow 119333, Russian Federation; 
\mbox{AZatsarinny@ipiran.ru}


\vspace*{3pt}


\noindent
\textbf{Suchkov Alexander P.} (b.\ 1954)~--- Doctor of Science in technology, 
leading scientist, Institute of Informatics Problems, Federal Research Center 
``Computer Science and Control'' of the 
Russian Academy of Sciences, 44-2~Vavilov Str., Moscow 119333, 
Russian Federation; \mbox{Asuchkov@ipiran.ru}

 


\label{end\stat}


\renewcommand{\bibname}{\protect\rm Литература}     %13
\def\stat{grinchenko}

\def\tit{О ГЕНЕЗИСЕ ИНФОРМАЦИОННОГО ОБЩЕСТВА:  
ИНФОРМАТИКО-КИБЕРНЕТИЧЕСКОЕ МОДЕЛЬНОЕ ПРЕДСТАВЛЕНИЕ}

\def\titkol{О генезисе информационного общества:  
информатико-кибернетическое модельное представление}

\def\aut{С.\,Н.~Гринченко$^1$}

\def\autkol{С.\,Н.~Гринченко}

\titel{\tit}{\aut}{\autkol}{\titkol}

\index{Гринченко С.\,Н.}
\index{Grinchenko S.\,N.}


%{\renewcommand{\thefootnote}{\fnsymbol{footnote}} \footnotetext[1]
%{Работа выполнена при частичной финансовой 
%поддержке РФФИ (проект 17-07-00577).}}


\renewcommand{\thefootnote}{\arabic{footnote}}
\footnotetext[1]{Институт проблем информатики Федерального исследовательского центра <<Информатика и~управление>> 
Российской академии наук, \mbox{sgrinchenko@ipiran.ru}}

\vspace*{-3.5pt}




  \Abst{Вводится понятие <<генезис информационного общества>>, которое рассматривается 
  с~позиций ин\-фор\-ма\-ти\-ко-ки\-бер\-не\-ти\-че\-ско\-го моделирования (ИКМ)
  процесса развития 
Человечества как са\-мо\-управ\-ля\-ющей\-ся иерар\-хо-се\-те\-вой системы. На этой основе 
получены количественные оценки его типовых про\-стран\-ст\-вен\-но-вре\-мен\-ных характеристик, 
представляющих собой геометрические прогрессии со знаменателем 
<<$e$~в~степени~$e$>> (15,15426$\ldots$), а~также скоординированных с~ними во времени 
и~в~пространстве пси\-хи\-ко-ант\-ро\-по\-ло\-ги\-че\-ских, образовательных  
и~ин\-фор\-ма\-ци\-он\-но-ком\-му\-ни\-ка\-ци\-он\-ных параметров и~возможностей 
включенного в~этот процесс усложняющегося человека и~его сообществ различной 
величины. Это позволило раздвинуть рамки существования информационного общества на 
всю историческую и~даже археологическую эпоху такого развития. Результирующая 
последовательность информационных технологий (ИТ) <<сигнальные  
по\-зы/зву\-ки/дви\-же\-ния\,--\,ми\-ми\-ка/жес\-ты\,--\,речь/язык\,--\,пись\-мен\-ность\,--\,ти\-ра\-жи\-ро\-ва\-ние текстов\,--\,компью\-те\-ры\,--\,те\-ле\-ком\-му\-ни\-ка\-ции\,--\,ин\-фор\-ма\-ци\-он\-ная на\-но\-тех\-но\-ло\-гия\,--\,$\ldots$>> 
позволяет рас\-смат\-ри\-вать генезис 
информационного общества в~широком контексте единой исторической ретроспективы 
и~перспективы.}
  
  \KW{информационное общество; информационные технологии;  
ин\-фор\-ма\-ти\-ко-ки\-бер\-не\-ти\-че\-ская модель; самоуправляющаяся 
 иерар\-хо-се\-те\-вая система Человечества; археологическая эпоха}
 
 \DOI{10.14357/19922264190214}
  
%\vspace*{4pt}


\vskip 10pt plus 9pt minus 6pt

\thispagestyle{headings}

\begin{multicols}{2}

\label{st\stat}
  
  В~литературе, даже энциклопедической, распространена трактовка 
<<информационного общества>> как общества <<современного типа>>, 
в~котором общение людей опирается на компьютерные 
и~телекоммуникационные ИТ\footnote[2]{В~[1] дано следующее определение:
<<\textbf{Информационное общество}, одно из понятий, используемых 
в~социологич.\ теории для обозначения обществ.\ систем <<современного типа>>$\ldots$ 
Важнейшие характеристики~И.\,о.: 1)~лавинообразное распространение информац. 
технологий (прежде всего компьютерных и~телекоммуникационных); 2)~превращение 
информации в~важнейший социальный ресурс, необходимую предпосылку управленч. 
деятельности, развития экономики, образования, сферы услуг, домашнего быта, 
рекреационной сферы и~т.\,д.; по некоторым данным, в~наиболее развитых странах проф. 
деятельность более половины занятых связана исключительно с~производством и~обработкой 
информации; 3)~наделение СМИ статусом <<четвертой ветви власти>>; 4)~расширение 
границ и~<<репертуара>> массовой культуры; 5)~увеличение каналов вертикальной 
и~горизонтальной мобильности; 6)~изменение представлений о~социальном пространстве 
(<<глобализация>> пространства, мгновенная доступность даже периферийных его 
сегментов) и~времени (расширение рамок <<современности>>, когда даже отдаленные 
историч. события воспринимаются как происходящие <<здесь>> и~<<сейчас>>); 
7)~возникновение в~процессе коммуникации особой виртуальной реальности, несводимой 
к~результатам технич. визуализации и~выходящей за пределы воображения и~памяти 
индивида; 8)~превращение информац. технологий в~базу для развития высоких технологий 
(Hi-Tech)>>.}. Такая трактовка этого понятия создает иллюзию 
отстраненности информационного общества от его собственного исторического 
прошлого, когда вышеперечисленных ИТ еще не изобрели, но люди в~составе 
сообществ как-то общались между собой, используя иные ИТ. 

Поскольку от 
этой иллюзии недалеко до недооценки полезности соответствующего 
исторического опыта для современности, попытаюсь развеять ее.
  
Результаты ИКМ процесса развития на Земле 
Человечества как самоуправляющейся ие\-рар\-хо-се\-те\-вой\footnote[3]{<<\textbf{Иерархо-сетевая}>> 
структура~--- иерархическая структура типа <<матрешки>>, но с~существенно большим 
единицы числом вложений на каждом ее иерархическом уровне, которые и~образуют 
соответствующие сетевые структуры.} системы~[2--14] (рис.~1) позволяют раздвинуть рамки 
существования информационного\linebreak общества на всю историческую и~даже археологическую эпоху такого 
развития, что дает возможность выделить ту эволюционную линию этого процесса, которую логично 
определить как \textit{генезис информационного общества}. 


\begin{figure*} %fig1
   \vspace*{1pt}
    \begin{center}  
  \mbox{%
 \epsfxsize=130.287mm 
 \epsfbox{gri-1.eps}
 }
\end{center}
%\vspace*{-9pt}
%\Caption{Схема иерархо-сетевой самоуправляющейся (по алгоритмам случайной поисковой 
%оптимизации целевых критериев энергетического характера с~ограничениями типа 
%равенств и~неравенств) системы лич\-ност\-но-про\-из\-вод\-ст\-вен\-но-со\-ци\-аль\-ной природы 
%(Человечества)~\cite{5-grn}}
\end{figure*}


На рис.~1 используются следующие обозначения:
\begin{itemize}
\item восходящие стрелки (имеющие структуру <<мно\-гие\,--\,к~од\-но\-му>>) 
отражают первую из~5~основных со\-став\-ля\-ющих контура поисковой 
оптимизации~--- \textit{поисковую активность} представителей 
соответствующих ярусов в~иерархии; 
\item нисходящие сплошные (имеющие 
структуру <<один\,--\,ко мно\-гим>>) стрелки отражают вторую 
со\-став\-ля\-ющую~--- \textit{целевые критерии} поисковой оптимизации 
энергетики системы Человечества; 
\item нисходящие пунктирные (<<один\,--\,ко 
многим>>) стрелки отражают третью со\-став\-ля\-ющую~--- 
\textit{оптимизационную системную память}  
лич\-ност\-но-про\-из\-вод\-ст\-вен\-но-со\-ци\-аль\-но\-го (результат 
адаптивных влияний представителей вышележащих иерархических ярусов на 
структуру вложенных в~них нижележащих); 
\item полужирными стрелками 
в~левой части схемы условно показана четвертая со\-став\-ля\-ющая~--- 
\textit{антропогенная ак\-тив\-ность} индивидов и~их групп, трак\-ту\-емая как 
<<трудовая деятельность по созданию со\-от\-вет\-ст\-ву\-юще\-го инструментария 
и~результатов его применения>>; 
\item пунктирными полужирными стрелками 
в~правой части схемы условно показана пятая со\-став\-ля\-ющая~--- 
\textit{антропогенная системная\linebreak память}  
лич\-ност\-но-про\-из\-вод\-ст\-вен\-но-со\-ци\-ального (процессы вовлечения 
результатов антропогенной активности в~структуру со\-от\-вет\-ст\-ву\-ющей  
иерар\-хо-се\-те\-вой под\-сис\-те\-мы Человечества).
\end{itemize}

Рассмотрим этот феномен поэтапно, сведя в~общую таблицу расчетные данные 
о~различных его проявлениях. 
       


\begin{table*}\footnotesize
\begin{center}
\Caption{Свод основных характеристик генезиса информационного общества (как 
проявления развития са\-мо\-управ\-ля\-ющей\-ся и~метаэволюционирующей, т.\,е.\ 
наращивающей чис\-ло своих иерархических уров\-ней/яру\-сов, сис\-те\-мы Человечества) от 
прошлого до модельно прогнозируемого будущего}
\vspace*{2ex}

\tabcolsep=1.5pt
\begin{tabular}{|c|c|l|c|c|c|c|}
\hline
&\tabcolsep=0pt\begin{tabular}{c}Характерный\\ ареал (радиус\\
 круга той же\\ площади); точность\\ антропогенного\\ 
воздействия\,/\\
производственных\\ технологий\end{tabular}&
\tabcolsep=0pt\begin{tabular}{c}Характерные\\ времена\\ старта;\\ кульминации\\ 
развития\\ подсистемы\end{tabular}&
\tabcolsep=0pt\begin{tabular}{c}Уровень\\ развития\\ Homo\\  
(и его пред-\\ шествен-\\ ников)\end{tabular}&
\tabcolsep=0pt\begin{tabular}{c}Носитель системной\\ памяти~---\\ субстрат психики\end{tabular}&
\tabcolsep=0pt\begin{tabular}{c}Лидирующая\\ ИТ\end{tabular}&
\tabcolsep=0pt\begin{tabular}{c}Требуемый уровень\\ образованности Homo;\\
аналогия филогенеза\\ и~онтогенеза:\\ примерный возраст\\ гармонично\\ образовываемого\\ 
(сегодня)\end{tabular}\\
\hline
1&2&\multicolumn{1}{c|}{3}&4&5&6&7\\
\hline
0&$\sim4{,}2$~м&\tabcolsep=0pt\begin{tabular}{c} $\sim428$~млн\\ лет назад;\\
$\sim 140{,}1$~млн\\ лет назад\end{tabular}&
\tabcolsep=0pt\begin{tabular}{c}Цефализация\\ позвоночных\end{tabular}&
\tabcolsep=0pt\begin{tabular}{c}Многоклеточный\\организм в~целом\end{tabular}&
\tabcolsep=0pt\begin{tabular}{c}Формирование\\ головного\\ мозга как основы\\
 для реализации\\ 
будущих ИТ\end{tabular}&\tabcolsep=0pt\begin{tabular}{c} ---\\
$\sim0{,}6$--1,0~год\end{tabular}\\
\hline
1&\tabcolsep=0pt\begin{tabular}{c} $\sim64$~м;\\
$\sim28$~см
\end{tabular}&\tabcolsep=0pt\begin{tabular}{c}$\sim28{,}23$~млн\\ лет назад;\\
$\sim9{,}26$~млн\\ лет назад
\end{tabular}&\tabcolsep=0pt\begin{tabular}{c}Пред-пред-\\
люди\\ Hominoidea\end{tabular}&
\tabcolsep=0pt\begin{tabular}{c}Органы многоклеточного\\ организма (его 
нервной\\ системы в~целом)\end{tabular}&
\tabcolsep=0pt\begin{tabular}{c}Сигнальные позы/\\
движения\\ и~неинтонированные\\ звуки (типа 
рычания,\\ ворчания, писка\\ и~т.\,п.)\end{tabular}&
\tabcolsep=0pt\begin{tabular}{c}Выработка\\ 
(младенцами)\\ сигнальных поз;\\
$\sim1{,}0$--1,6~лет \end{tabular}\\
\hline
2&\tabcolsep=0pt\begin{tabular}{c} $\sim1$~км;\\
$\sim1{,}8$~см\end{tabular}&\tabcolsep=0pt\begin{tabular}{c} $\sim1{,}86$~млн\\ лет 
назад;\\
$\sim612$~тыс.\\ лет назад\end{tabular}&
\tabcolsep=0pt\begin{tabular}{c}Пред-люди\\ Homo ergaster\,/\\
Homo erectus\end{tabular}&
\tabcolsep=0pt\begin{tabular}{c}Ткани 
многоклеточного\\ организма\\ (сетей/ансамблей\\ нейронов и~др.)\end{tabular}&
\tabcolsep=0pt\begin{tabular}{c}Мимика/жесты\\ 
и~интонированные\\ звуки\end{tabular}&
\tabcolsep=0pt\begin{tabular}{c}Овладение (ре-\\ бенком) мимикой/\\ 
жестами,\\
начальное\\ понимание речи; \\ $\sim1{,}6$--2,6~лет \end{tabular}\\
\hline
3&\tabcolsep=0pt\begin{tabular}{c} $\sim15$~км; \\
$\sim1{,}2$~мм
\end{tabular}&
\tabcolsep=0pt\begin{tabular}{c} $\sim123$~тыс.\\ лет назад;\\
$\sim40$~тыс.\\ лет назад\end{tabular}&
\tabcolsep=0pt\begin{tabular}{c}Homo\\ sapiens$^\prime$\end{tabular}&
\tabcolsep=0pt\begin{tabular}{c}Эвкариотические\\ клетки\\ 
многоклеточного\\ организма\\ (отдельные нервные\\ и~глиальные клетки\\ и~др.)\end{tabular}&
\tabcolsep=0pt\begin{tabular}{c}Речь/язык\\ 
(артикулированная\\ устная речь)\end{tabular}&
\tabcolsep=0pt\begin{tabular}{c}Овладение (детьми)\\ 
речью/языком\\ (протообразование); \\ $\sim2{,}6$--4,2~лет \end{tabular}\\
\hline
4&\tabcolsep=0pt\begin{tabular}{c} $\sim222$~км;\\
$\sim 80$~мкм
\end{tabular}&\tabcolsep=0pt\begin{tabular}{c}$\sim8{,}1$~тыс.\\ лет назад;\\
$\sim2{,}7$~тыс.\\ лет назад\end{tabular}&
\tabcolsep=0pt\begin{tabular}{c}Homo\\ sapiens$^{\prime\prime}$\end{tabular}&
\tabcolsep=0pt\begin{tabular}{c}Компартменты\\ 
эвкариотической\\ клетки (отдельные\\ рецепторные,\\ или постсинаптические,\\ зоны нейронов и~т.\,п.)\end{tabular}
&Письменность&\tabcolsep=0pt\begin{tabular}{c}Овладение чтением/ \\ письмом 
(дошкольное\\ образование);\\
$\sim4{,}2$--6,9~лет \end{tabular}\\
\hline
5&\tabcolsep=0pt\begin{tabular}{c}$\sim3370$~км;\\
$\sim5$~мкм
\end{tabular}&\tabcolsep=0pt\begin{tabular}{l}$\sim1446$~г.;\\
$\sim1806$~г.\end{tabular}&
\tabcolsep=0pt\begin{tabular}{c}Homo\\ sapiens$^{\prime\prime\prime}$\end{tabular} &
\tabcolsep=0pt\begin{tabular}{c}Субкомпартменты\\ эвкариотической 
клетки\end{tabular}&
\tabcolsep=0pt\begin{tabular}{c}Тиражирование\\ текстов,\\ или книгопечатание\end{tabular}&
\tabcolsep=0pt\begin{tabular}{c}Начальное\\ образование;\\ 
$\sim6{,}9$--11,1~лет \end{tabular}\\
\hline
6&\tabcolsep=0pt\begin{tabular}{c} $\sim51$~тыс.\ км\\ (общепланетарный);\\
$\sim0{,}35$~мкм\end{tabular}&\tabcolsep=0pt\begin{tabular}{l} $\sim1946$~г.;\\
$\sim 1970$~г.\end{tabular}&\tabcolsep=0pt\begin{tabular}{c}Homo \\
sapiens$^{\prime\prime\prime\prime}$\end{tabular}&
\tabcolsep=0pt\begin{tabular}{c}Ультраструктурные\\ (прокариотические)\\ 
внутриклеточные элементы\\ эвкариотической клетки\\ (типа клеточного ядра,\\ деталей 
эндоплазматической\\ сети и~т.\,п.\ образований)\end{tabular}&
Компьютерная ИТ&\tabcolsep=0pt\begin{tabular}{c}Среднее\\ образование;\\
$\sim11{,}1$--18~лет \end{tabular}\\
\hline
7&\tabcolsep=0pt\begin{tabular}{c} $\sim773$~тыс.\ км\\ (ближний\\ космос);\\
$\sim23$~нм\end{tabular}&\tabcolsep=0pt\begin{tabular}{l} $\sim1979$~г.;\\
$\sim2003$~г.\end{tabular}&\tabcolsep=0pt\begin{tabular}{c}Homo\\
 sapiens$^{\prime\prime\prime\prime\prime}$\end{tabular} 
&
\tabcolsep=0pt\begin{tabular}{c}Макромолекулы/гены\\ (компартменты\\ 
ультраструктурных--\\
прокариотических--\\
внутриклеточных\\ элементов)\end{tabular}&
\tabcolsep=0pt\begin{tabular}{c}Телекоммуника-\\ ционная ИТ\end{tabular}&\tabcolsep=0pt\begin{tabular}{c}Высшее обра-\\
зование\;+\;<<аспи-\\ рантура>>; \\
$\sim18$--29,1~лет \end{tabular}\\
\hline
8&\tabcolsep=0pt\begin{tabular}{c}
$\sim11{,}7$~млн км\\ (промежуточный\\ космос);\\
$\sim1{,}5$~нм\end{tabular}&\tabcolsep=0pt\begin{tabular}{l} $\sim1981$~г.;\\ 
$\sim2341$~г.~(?)\end{tabular}&\tabcolsep=0pt\begin{tabular}{c}Homo\\ 
sapiens$^{\prime\prime\prime\prime\prime\prime}$\end{tabular}&
\tabcolsep=0pt\begin{tabular}{c}Органические молекулы \\
(субкомпартменты\\ ультраструктурных--
\\прокариотических--
\\внутриклеточных \\
элементов)\end{tabular}&
\tabcolsep=0pt\begin{tabular}{c}Нано-ИТ (возможно,\\
 <<наноаппаратно\\ поддерживаемая\\ селективная\\ телепатия>>~\cite{16-grn})\end{tabular}&
 \tabcolsep=0pt\begin{tabular}{c}<<Докторантура>>; \\ 
$\sim29{,}1$--47,1~лет \end{tabular}\\
\hline
9&$\cdots$&\multicolumn{1}{c|}{$\cdots$}&$\cdots$&$\cdots$&$\cdots$&$\cdots$\\
\hline
\end{tabular}
\end{center}
\end{table*}




  Промежутки времени между возникновением новых ие\-рар\-хо-се\-те\-вых 
подсистем Человечества (а~следовательно, и~между стартами новых ИТ) 
подчиняются, согласно ИКМ, простой математической за\-ко\-но\-мер\-ности: 
каж\-дый из них в~$e^e\hm= 15{,}15426$\ldots раз короче 
предыдущего\footnote{Эту геометрическую прогрессию~--- как модель критических 
уровней развития биологических сис\-тем~--- выявили А.\,В.~Жирмунский 
и~В.\,И.~Кузьмин~\cite{17-grn}.} (третий\linebreak
 столбец таблицы). В~свою очередь, этой 
же закономерности подчиняются и~размеры ареалов\linebreak
 (радиусы кругов той же 
площади) устойчивых и~эффективно са\-мо\-управ\-ля\-ющих\-ся сообществ 
человека как базисного элемента сис\-те\-мы Человечества, и~точ\-ности 
доступных услож\-ня\-юще\-му\-ся человеку~--- в~конкретный момент 
исторического времени~--- антропогенных воздействий и/или 
производственных технологий (второй столбец таб\-ли\-цы) (рис.~2).
  
  Эмпирические оценки этих времен и~пространств, сделанные 
и~опуб\-ли\-ко\-ван\-ные палео\-ант\-ро\-по\-ло\-га\-ми, археологами и~историками,~--- 
когда они имеются!~--- не противоречат модельным  
результатам~\cite{14-grn}.
  %
Диапазоны примерного возраста <<образовываемых>>, приведенные 
в~седьмом столб\-це таб\-ли\-цы, рассчитаны, исходя из <<золотого сечения>> 
(соотношения смеж\-ных членов чис\-ло\-во\-го ряда, равного 1,618$\ldots$ при 
увеличении ряда, либо 0,618$\ldots$ при его уменьшении, аде\-кват\-ность 
использования которого при выработке количественных оценок в~самых 
различных областях знания хорошо известна\footnote{Применительно 
к~периодизации истории Человечества в~археологическую эпоху это продемонстрировано 
Ю.\,Л.~Щаповой~\cite{18-grn, 19-grn, 20-grn}, согласование подхода к~такой периодизации на 
основе золотого сечения и~пред\-ла\-га\-емо\-го информатико-ки\-бер\-не\-ти\-че\-ско\-го подхода 
подробно показано в~\cite{10-grn, 12-grn, 13-grn, 14-grn, 15-grn, 21-grn}.}), 
опирающегося на ориентировочную оценку завершения человеком среднего 
образования к~18~годам (на сегодня).


  Базируясь на ИКМ, в~качестве нулевого этапа развития будущего 
информационного общества, как пред\-став\-ля\-ет\-ся, можно рас\-смат\-ри\-вать 
процесс \textit{цефализации} позвоночных, т.\,е.\ возникновения 
и~усложнения у~них головного мозга как основного носителя механизмов 
запоминания и~считывания информации о~результатах их адаптивного 
и~социального поведения, начавшейся около 428~млн лет назад 
с~кульминацией около 140,1~млн лет назад (шестой стол\-бец таб\-ли\-цы) на 
<<территории>> порядка 4,2~м~--- т.\,е.\ в~пределах отдельного 
многоклеточного организма.
  

  
  Далее в~качестве первого этапа такого развития будем рассматривать 
начавшуюся около 28,23~млн лет назад, с~кульминацией около 9,26~млн лет 
назад, на территориях порядка 64~м, ИТ сигнальных поз/дви\-же\-ний 
и~неинтонированных звуков (типа рычания, ворчания, писка и~т.\,п.), 
характерную для стад\-ных/стай\-ных животных, в~том числе  
пред-пред-людей {Hominoidea} (четвертый стол\-бец таб\-ли\-цы), 
способных обеспечивать точность своих воздействий на природу порядка~28~см. 
Субстрат их психики относится к~иерархическому уровню органов 
многоклеточного организма (пятый стол\-бец), а~уровень об\-ра\-зо\-ван\-ности 
соответствует современному младенцу возрастом около~1--1,6~лет (седьмой 
столбец).
  
  Следующий, второй этап развития ИТ~--- ми\-ми\-ки/жес\-тов, начавшийся 
около~1,86~млн лет назад, с~кульминацией около~612~тыс.\ лет назад, на 
территориях порядка~1~км, реализовался далекими\linebreak предками современного 
человека~--- пред-людь\-ми {Homo ergaster/Homo erectus}, способными 
обеспечивать точ\-ность своих воздействий на природу\linebreak порядка~1,8~см, 
с~субстратом психики уров\-ня тканей многоклеточного организма и~уровнем 
обра\-зо\-ван\-ности, соответствующим современному ребенку~1,6--2,6~лет.

\pagebreak

\end{multicols}

\setcounter{figure}{1}
\begin{figure*} %fig2
 \vspace*{1pt}
    \begin{center}  
  \mbox{%
 \epsfxsize=163.101mm 
 \epsfbox{gri-2.eps}
 }
\end{center}
\vspace*{-6pt}
\Caption{Пространственно-временн$\acute{\mbox{ы}}$е характеристики и~тренд ИТ в~процессе генезиса 
информационного общества (по ИКМ, в~двойном логарифмическом масштабе; 
иерархическая слож\-ность~--- число уров\-ней/яру\-сов в~системной иерархии)}
\vspace*{1pt}
\end{figure*}

\begin{multicols}{2}



  
  Все последующие этапы развития ИТ~--- речь/язык, пись\-мен\-ность, 
тиражирование текстов (книгопечатание), компьютеры, телекоммуникации, 
на\-но-ИТ~--- реализовались последовательно усложняющимися формами 
{Homo sapiens}, который при этом образовывал относительно 
устойчивые и~относительно эффективно функционирующие 
и~самоуправляющиеся сообщества на все больших ареалах, одновременно 
повышая точность своих (антропогенных) действий при формировании 
вокруг себя <<второй (рукотворной) природы>>.
  
  Так, третий этап развития ИТ~--- речи/языка, начавшийся около 123~тыс.\ 
лет назад, с~кульминацией (верхнепалеолитической революцией) 
около~40~тыс.\ лет назад, на территориях порядка~15~км, реализовался 
{Homo sapiens}$^\prime$, способными обеспечивать точность своих 
производственных технологий порядка~1,2~мм, с~субстратом психики 
уровня эвкариотических клеток многоклеточного организма и~уровнем 
образованности, соответствующим современному ребенку~2,6--4,2~лет.

\begin{figure*}[b] %fig3
%\vspace*{-4pt}
    \begin{center}  
  \mbox{%
 \epsfxsize=162.821mm 
 \epsfbox{gri-3.eps}
 }
\end{center}
\vspace*{-6pt}
\Caption{Тренд изменения времен запаздывания кульминаций развития под\-сис\-тем  
иерар\-хо-се\-те\-вой сис\-те\-мы Человечества относительно их стартов (по ИКМ, в~двойном 
логарифмическом мас\-штабе)}
\end{figure*}
  
  Четвертый этап развития ИТ~--- письменности, начавшийся 
около~8,1~тыс.\ лет назад, с~кульминацией (городской революцией 
<<осевого времени>>) около 2,7~тыс.\ лет назад, на территориях 
порядка~222~км, реализовался {Homo sapiens}$^{\prime\prime}$, 
способными обеспечивать точность своих производственных технологий 
порядка~80~мкм, с~суб\-стра\-том психики уровня компартментов 
эвкариотических клеток многоклеточного организма и~уровнем 
образованности, соответствующим современному ребенку~4,2--6,9~лет 
(дошкольное образование).
  
  Пятый этап развития ИТ~--- тиражирования\linebreak текс\-тов (книгопечатания), 
начавшийся около 1446~г.\ н.\,э., с~кульминацией (промышленной\linebreak 
революцией) около 1806~г., на территориях порядка~3370~км, реализовался 
{Homo sapiens}$^{\prime\prime\prime}$, способными обеспечивать 
точность своих производственных технологий порядка~5~мкм, с~субстратом 
психики уровня субкомпартментов эвкариотических клеток многоклеточного 
организма и~уровнем об\-ра\-зо\-ван\-ности, соответствующим современному 
ребенку~6,9--11,1~лет (начальное образование).
  
  Шестой этап развития ИТ~--- компьютеров (локальных), начавшийся 
около~1946~г., с~кульминацией (изобретением микропроцессоров) 
около~1970~г., на территориях порядка~51~тыс.\ км (т.\,е.\ 
общепланетарного, или глобального размера), реализовался {Homo 
sapiens}$^{\prime\prime\prime\prime}$, способными обеспечивать точ\-ность 
своих производственных технологий порядка~0,35~мкм, с~субстратом 
психики уровня\linebreak
 ультраструктурных (прокариотических) внутриклеточных 
элементов эвкариотической клетки и~уровнем об\-ра\-зо\-ван\-ности, 
соответствующим современному  
под\-рост\-ку-юно\-ше/де\-вуш\-ке~11,1--18~лет\linebreak (среднее образование).
  
  Седьмой этап развития ИТ~--- телекоммуникаций, начавшийся около 
1979~г., с~кульминацией (пиком ско\-рости распространения на планете 
мобильной телефонии, интернета и~т.\,п.) около\linebreak
 2003~г., в~космическом 
объеме радиусом (шара)\linebreak порядка 773~тыс.\ км (т.\,е.\ в~ближнем космосе), 
реализовался {Homo sapiens}$^{\prime\prime\prime\prime\prime}$, 
способными обеспечивать точ\-ность своих производственных технологий 
порядка~23~нм, с~субстратом психики уровня мак\-ро\-мо\-ле\-кул/ге\-нов 
(компартментов\ ульт\-ра\-струк\-тур\-ных--про\-ка\-рио\-ти\-че\-ских--\linebreak
внут\-ри\-кле\-точ\-ных 
элементов эвкариотической клетки) и~уровнем 
об\-ра\-зо\-ван\-ности, со\-от\-вет\-ст\-ву\-ющим современному молодому  
че\-ло\-ве\-ку~18--29,1~лет (высшее обра\-зо\-ва\-ние\;+\;<<ас\-пи\-ран\-ту\-ра, 
с~защитой диссертации кандидата наук>>).
  
  Восьмой этап развития перспективной нано-ИТ (возможно, <<ИТ 
наноаппаратно поддерживаемой селективной телепатии>>~\cite{16-grn}), 
начавшийся около~1981~г., с~кульминацией (пиком скорости ее 
распространения на планете) около~2341~г.\ (расчетная дата), в~космическом 
объеме радиусом шара порядка~11,7~млн км (т.\,е.\ в~промежуточном 
космосе~\cite{5-grn}), реализовался {Homo 
sapiens}$^{\prime\prime\prime\prime\prime\prime }$, способными обеспечивать 
точность своих производственных технологий порядка~1,5~нм (отсюда 
наименование ИТ), с~субстратом психики уровня органических молекул 
(субкомпартментов ульт\-ра\-струк\-тур\-ных--про\-ка\-риоти\-че\-ских--внут\-ри\-кле\-точ\-ных 
элементов эвкариотической клетки) и~уровнем 
об\-ра\-зо\-ван\-ности,\linebreak соответству\-ющим современному зрелому  
человеку~29,1--47,1~лет (<<докторантура>>).
  
  Важно отметить, что процесс появления всех вышеперечисленных 
подсистем подчиняется кумулятивному принципу: возникновение каждой 
новой подсистемы не отменяет существование предыду\-щей: они все активно 
взаимодействуют между собой, коэволюционируют и~т.\,п., но исторически 
более ранние, естественно, постепенно переходят на второй, третий и~т.\,д.\ 
планы исторической сцены.
  
  Точка сходимости этого ряда находится около\linebreak 1981~г., знаменуя собой 
завершение этапа <<детст\-ва--от\-ро\-че\-ст\-ва--юности>> Человечества как 
целого и~начало этапа его <<зрелости>>~--- до\-сти\-же\-ния его максималь\-ной 
иерархической слож\-ности (чис\-ла уров\-ней/яру\-сов в~сис\-тем\-ной 
иерархии)~\cite{5-grn, 7-grn}.
  
  С позиции прогнозирования генезиса информационного общества на 
будущие времена отмечу, что, согласно ИКМ, тренд изменения времен 
запаздывания кульминаций развития под\-сис\-тем относительно их стартов 
сменился прямо на наших глазах. Если во временн$\acute{\mbox{о}}$м диапазоне с~428~млн 
лет назад и~до 1946~г.\ он со\-стоял в~равномерном (в~логарифмическом 
масштабе) укорочении согласно той же за\-ко\-но\-мер\-ности 
(в~$e\hm=15,15426\ldots$~раз), то в~диапазоне от~1946 по 1979~гг.\ это время 
запаздывания не изменилось, а~начиная с~1979~г.\ начало удлиняться 
(рис.~3). 
  

  
  Таким образом, метаэволюция сис\-те\-мы Человечества завершилась около 
1981~г.\ в~том смыс\-ле, что все воз\-мож\-ные ее ие\-рар\-хо-се\-те\-вые под\-сис\-те\-мы 
\textit{в~потенции} уже созданы. Но их \textit{актуализация}, дальнейшее 
услож\-не\-ние, эволюция и~коэволюция с~ранее возникшими аналогичными 
под\-сис\-те\-ма\-ми будет продолжаться неопределенно длительное время.

\vspace*{-10pt}
  
  \section*{Выводы}
  
  \vspace*{-2pt}
  
  \noindent
  \begin{enumerate}[1.]
\item  Изучение \textit{генезиса информационного общества} во всех его 
последовательных формах~--- от древности до современности и~далее~--- на 
базе\linebreak
 ин\-фор\-ма\-ти\-ко-ки\-бер\-не\-ти\-че\-ско\-го модельного подхода 
и~формализации процесса метаэволю\-ционного развития в~соответствующих 
терминах, позволило получить количественные\linebreak оценки его типовых  
про\-стран\-ст\-вен\-но-вре\-менн$\acute{\mbox{ы}}$х характеристик, 
а~также скоординированных с~ними во времени и~в~пространстве  
психико-ант\-ро\-по\-ло\-ги\-че\-ских, образовательных %\linebreak  
и~ин\-фор\-ма\-ци\-он\-но-ком\-му\-ни\-ка\-ци\-он\-ных параметров 
и~возможностей включенного в~этот процесс усложняющегося человека 
и~его сообществ различной величины.
  \item  Позиционирование ИТ локальных компьютеров и~ИТ 
телекоммуникаций в~качестве неотъемлемых составляющих совокупности\linebreak 
монотонно усложняющихся в~ходе цивилизационного развития~--- 
и~информационного общества!~--- ИТ позволяет 
рассматривать их появление и~функционирование в~широком контексте 
единой исторической ретроспективы и~перспективы, давая возможность 
делать не только теоретические, но и~практические выводы.
  \end{enumerate}
  
{\small\frenchspacing
 {%\baselineskip=10.8pt
 \addcontentsline{toc}{section}{References}
 \begin{thebibliography}{99}
\bibitem{1-grn}
\Au{Мелик-Гайгазян И.\,В.} Информационное общество~// Большая российская 
энциклопедия. Т.~11.~--- М.: Большая Российская энциклопедия, 2008. С.~490.
\bibitem{2-grn}
\Au{Гринченко С.\,Н.} Социальная метаэволюция Человечества как последовательность 
шагов формирования механизмов его системной памяти~// Исследовано в~России: 
Электронный журнал, 2001. Т.~145. С.~1652--1681. {\sf  
https://cyberleninka.ru/article/v/sotsialnaya-metaevolyutsiya-chelovechestva-kak-posledovatelnost-shagov-formirovaniya-mehanizmov-ego-sistemnoy-pamyati}.
\bibitem{3-grn}
\Au{Гринченко С.\,Н.} Системная память живого (как основа его метаэволюции
и~периодической структуры).~--- М.: ИПИ РАН, Мир, 2004. 512~с.
\bibitem{4-grn}
\Au{Grinchenko S.\,N.} Meta-evolution of nature system~--- the framework of history~// Social 
Evolution History, 2006. Vol.~5. No.\,1. P.~42--88.
\bibitem{5-grn}
\Au{Гринченко С.\,Н.} Метаэволюция (сис\-тем неживой, живой  
и~со\-ци\-аль\-но-тех\-но\-ло\-ги\-че\-ской природы).~--- М.: ИПИ РАН, 2007. 456~с.
\bibitem{6-grn}
\Au{Гринченко С.\,Н.} Homo eruditus (человек образованный) как элемент сис\-те\-мы 
Человечества~// Открытое образование, 2009. №\,2. С.~48--55.

\bibitem{10-grn} %7
\Au{Гринченко С.\,Н., Щапова~Ю.\,Л.} История Человечества: модели периодизации~// 
Вестник РАН, 2010. №\,12. С.~1076--1084.

%\bibitem{11-grn}  %8
%\Au{Grinchenko S.\,N., Shchapova~Y.\,L.} Human history periodization models~// Herald of the 
%Russian Academy of Sciences, 2010. Vol.~80. No.\,6. P.~498--506.
\bibitem{7-grn} %9
\Au{Grinchenko S.\,N.} The pre- and post-history of Humankind: What is it?~// Problems of 
contemporary world futurology.~--- Newcastle-upon-Tyne: Cambridge Scholars Publishing, 
2011. P.~341--353.
\bibitem{8-grn} %10
\Au{Гринченко С.\,Н.} Об эволюции психики как иерархической сис\-те\-мы 
(кибернетическое пред\-став\-ле\-ние)~// Историческая психология и~социология истории, 
2012. Т.~5. №\,2. С.~60--76.

\bibitem{12-grn} %11
\Au{Гринченко С.\,Н., Щапова~Ю.\,Л.} Информационные технологии в~истории 
Человечества.~--- М.: Новые технологии, 2013. 32~с. (Приложение к~журналу 
<<Информационные технологии>>, 2013. №\,8.)

\bibitem{9-grn} %12
\Au{Гринченко С.\,Н.} Эволюция темпов жизни людей и~развитие человечества~// Человек, 
2014. №\,5. С.~28--36.



\bibitem{13-grn}
\Au{Grinchenko S.\,N., Shchapova~Y.\,L.} Archaeological epoch as the succession of generations 
of evolutive subject-carrier archaeological sub-epoch~// Philosophy of Nature in Cross-Cultural 
Dimensions: The Result of the International Symposium at the University of Vienna~/ 
Komparative Philosophie und Interdisziplin$\ddot{\mbox{a}}$re Bildung (KoPhil). Band~5.~--- 
Hamburg: Verlag Dr.\ Kova$\Check{\mbox{c}}$, 2017. P.~478--499.
\bibitem{14-grn}
\Au{Щапова Ю.\,Л., Гринченко~С.\,Н.} Введение в~теорию археологической эпохи: 
числовое моделирование и~логарифмические шкалы про\-стран\-ст\-вен\-но-вре\-мен\-ных 
координат.~--- М.: Истфак МГУ, ФИЦ ИУ РАН, 2017. 236~с.
\bibitem{15-grn}
\Au{Grinchenko S.\,N., Shchapova~Yu.\,L.} Communications: Model representations about 
historical retrospective and possible perspective~// Communications Media 
Design Electronic~J., 2018. Vol.~3. No.\,2. P.~65--78.
\bibitem{16-grn}
\Au{Гринченко С.\,Н.} Послесловие~// Мат-лы доклада на Совместном научном семинаре 
ИПИ РАН и~\mbox{ИНИОН} РАН <<Методологические проблемы наук об информации>>.~---
М., 2012. С.~5--8. {\sf 
http://legacy.\linebreak inion.ru/files/File/MPNI\_9\_13\_12\_12\_posl.pdf}.
\bibitem{17-grn}
\Au{Жирмунский А.\,В., Кузьмин~В.\,И.} Критические уровни в~процессах развития 
биологических систем.~--- М.: Наука, 1982. 179~с.
\bibitem{18-grn}
\Au{Щапова Ю.\,Л.} Хронология и~периодизации древнейшей истории как числовая 
последовательность (ряд Фибоначчи)~// Информационный бюллетень Ассоциации 
<<История и~компьютер>>, 2000. №\,25.
\bibitem{19-grn}
\Au{Щапова Ю.\,Л.} Археологическая эпоха: хронология, периодизация, теория,  
модель.~--- М.: КомКнига, 2005. 192~с.
\bibitem{20-grn}
\Au{Щапова Ю.\,Л.} Материальное производство в~археологическую эпоху.~--- СПб.: 
Алетейя, 2011. 244~с.
\bibitem{21-grn}
\Au{Гринченко С.\,Н., Щапова~Ю.\,Л.} Пространство и~время в~археологии. Часть~3. 
О~метрике базисной пространственной структуры человечества в~археологическую 
эпоху~// Пространство и~время, 2014. №\,1(15). С.~78--89.
 \end{thebibliography}

 }
 }

\end{multicols}

\vspace*{-8pt}

\hfill{\small\textit{Поступила в~редакцию 17.10.18}}

\vspace*{6pt}

%\pagebreak

%\newpage

%\vspace*{-29pt}

\hrule

\vspace*{2pt}

\hrule

%\vspace*{-2pt}

\def\tit{ON THE GENESIS OF~THE~INFORMATION SOCIETY: INFORMATICS-CYBERNETIC 
MODEL REPRESENTATION}


\def\titkol{On the genesis of~the~information society: Informatics-cybernetic 
model representation}

\def\aut{S.\,N.~Grinchenko}

\def\autkol{S.\,N.~Grinchenko}

\titel{\tit}{\aut}{\autkol}{\titkol}

\vspace*{-11pt}


\noindent
Institute of Informatics Problems of the Federal Research Center ``Informatics and Control'' of 
the Russian Academy of Sciences, 44-2~Vavilov Str., Moscow 119333, Russian Federation

\def\leftfootline{\small{\textbf{\thepage}
\hfill INFORMATIKA I EE PRIMENENIYA~--- INFORMATICS AND
APPLICATIONS\ \ \ 2019\ \ \ volume~13\ \ \ issue\ 2}
}%
 \def\rightfootline{\small{INFORMATIKA I EE PRIMENENIYA~---
INFORMATICS AND APPLICATIONS\ \ \ 2019\ \ \ volume~13\ \ \ issue\ 2
\hfill \textbf{\thepage}}}

\vspace*{6pt}


  
  \Abste{The concept of the information society genesis is introduced, which is 
viewed from the standpoint of informatics-cybernetic modeling of the development 
of Humankind as a self-controlling hierarchical-networking system. On this basis, 
the author obtained quantitative assessments of its typical spatial-temporal 
characteristics, representing geometric progressions with the denominator ``$e$ to the 
degree~$e$'' (15.15426$\ldots$), as well as coordinated with them in time and space 
of the psychoanthropological, educational, and informational communication 
parameters and possibilities of a person who becomes complicated in this process 
and his communities of various sizes. This allowed us to push the framework of 
the information society for the entire historical and even archaeological epoch of 
such development. The resulting sequence of information technologies ``signal 
poses\,/\,sounds/movements\,--\,mimics/gestures\,--\,speech/language\,--\,writing\,--\,replicating 
texts\,--\,computers\,--\,telecommunications\,--\,information 
nanotechnology\,--\,$\ldots$'' allows us to consider the information society genesis 
in the broad context of a unified historical retrospective and perspective.}
  
  \KWE{information society; information technologies; informatics-cybernetic 
model; self-controlling hierarchical-networking system of Humankind; 
archaeological epoch}
  

\DOI{10.14357/19922264190214}

%\vspace*{-14pt}

%\Ack
%\noindent



%\vspace*{6pt}

  \begin{multicols}{2}

\renewcommand{\bibname}{\protect\rmfamily References}
%\renewcommand{\bibname}{\large\protect\rm References}

{\small\frenchspacing
 {%\baselineskip=10.8pt
 \addcontentsline{toc}{section}{References}
 \begin{thebibliography}{99}

\bibitem{1-grn-1}
\Aue{Melik-Gaygazyan, I.\,V.} 2008. Informatsionnoe ob\-shche\-st\-vo [Information 
society]. \textit{Bol'shaya rossiyskaya entsiklopediya} [Great Russian 
Encyclopedia].  Moscow: Great Russian 
Encyclopedia Publs. 11:490.
\bibitem{2-grn-1}
\Aue{Grinchenko, S.\,N.} 2001. Sotsial'naya me\-ta\-evo\-lyu\-tsiya Chelovechestva kak 
posledovatel'nost' shagov for\-mi\-ro\-va\-niya mekhanizmov ego sistemnoy pamyati 
[Social meta-evolution of Mankind as a~sequence of steps for the formation of the 
mechanisms of its system memory]. \textit{Elektronnyy zhurnal <<Issledovano 
v~Rossii>>} [Electronical J. ``Invstigated in Russia'']. 145:1652--1681. Avalable 
at: {\sf  
https://cyberleninka.ru/article/v/sotsialnaya-metaevolyutsiya-chelovechestva-kak-posledovatelnost-shagov-formirovaniya-mehanizmov-ego-sistemnoy-pamyati} (accessed 
October~5, 2018).
\bibitem{3-grn-1}
\Aue{Grinchenko, S.\,N.} 2004. \textit{Sistemnaya pamyat' zhivogo (kak osnova 
ego metaevolyutsii i~periodicheskoy struktury)} [System memory of the life (as the 
basis of its meta-evolution and periodic structure)]. Moscow: IPIRAN, MIR. 
512~p.
\bibitem{4-grn-1}
\Aue{Grinchenko, S.\,N.} 2006. Meta-evolution of nature system~--- the 
framework of history. \textit{Social Evolution History} 5(1):42--88.
\bibitem{5-grn-1}
\Aue{Grinchenko, S.\,N.} 2007. \textit{Metaevolyutsiya (sistem nezhivoy, zhivoy 
i~sotsial'no-tekhnologicheskoy prirody)} [Meta-evolution (of inanimate, animate, 
and socio-technological nature systems)]. Moscow: IPIRAN. 456~p. 
\bibitem{6-grn-1}
\Aue{Grinchenko, S.\,N.} 2009. Homo eruditus (chelovek obrazovannyy) kak 
element sistemy Chelovechestva [Homo eruditus (educated human) as an element 
of the Humakind's system]. \textit{Otkrytoe obrazovanie} [Open Education]  
2:48--55.

\bibitem{10-grn-1} %7
\Aue{Grinchenko, S.\,N., and Yu.\,I.~Shchapova.} 2010. 
Human history periodization models. \textit{Her. Russ. Acad. Sci.} 80(6):498--506.
%\bibitem{11-grn-1} %8
%\Aue{Grinchenko, S.\,N., and Y.\,I.~Shchapova.}  2010. Human history 
%periodization models. \textit{Herald of the Russian Academy of Sciences} 
%80(6):498--506.

\bibitem{7-grn-1} %9
\Aue{Grinchenko, S.\,N.} 2011.The pre- and post-history of Humankind: What is 
it?  \textit{Problems of contemporary world futurology}. 
 Newcastle-upon-Tyne: Cambridge Scholars 
Publishing.  341--353.
\bibitem{8-grn-1} %10
\Aue{Grinchenko, S.\,N.} 2012. Ob evolyutsii psikhiki kak ie\-rar\-khi\-che\-skoy 
sistemy (kiberneticheskoe predstavlenie) [On the evolution of mind as 
a~hierarchical system (a~cybernetic approach)]. \textit{Istoricheskaya 
psikhologiya i~sotsiologiya istorii} [Historical Psychology \& Sociology of 
History] 6(2):\linebreak 60--77.


\bibitem{12-grn-1} %11
\Aue{Grinchenko, S.\,N., and Y.\,I.~Shchapova.} 2013. \textit{In\-for\-ma\-tsi\-on\-nye 
tekhnologii v~istorii Chelovechestva} [Information technology in the history of 
Humankind]. Moscow: Novye tekhnologii. 32~p. (Prilozhenie k zhurnalu 
<<\textit{Informatsionnye tekhnologii}>> [Supplement to J.~Information Technology] 8.

\bibitem{9-grn-1} %12
\Aue{Grinchenko, S.\,N.} 2014. Evolyutsiya tempov zhizni lyudey i~razvitie 
chelovechestva [The evolution of the pace of human life and human development]. 
\textit{Human Being} 5:28--36.

\bibitem{13-grn-1}
\Aue{Grinchenko, S.\,N., and Y.\,I.~Shchapova.} 2017. Archaeological epoch as 
the succession of generations of evolutive subject-carrier archaeological  
sub-epoch. \textit{Philosophy of Nature in Cross-Cultural Dimensions: The Result of 
the International Symposium at the University of Vienna}~/ Komparative 
Philosophie und Interdisziplin$\ddot{\mbox{a}}$re Bildung (KoPhil), Band~5. 
Hamburg: Verlag Dr.\ Kova$\Check{\mbox{c}}$.  478--499.
\bibitem{14-grn-1}
\Aue{Shchapova, Y.\,L., and S.\,N.~Grinchenko.} 2017. \textit{Vvedenie 
v~teoriyu arkheologicheskoy epokhi: chislovoe modelirovanie i~logarifmicheskie 
shkaly prostranstvenno-vremennykh koordinat} [Introduction to the theory of the 
archaeological epoch: Numerical modeling and logarithmic scales of space--time 
coordinates]. Moscow: Faculty 
of History MSU, FRC CSC RAS]. 236~p. 

\vspace*{1pt}

\bibitem{15-grn-1}
\Aue{Grinchenko, S.\,N., and Y.\,I.~Shchapova}. 2018.  Communications: Model 
representations about historical retrospective and possible perspective. 
\textit{Communications Media Design Electronic~J.}  3(2):65--78. 
Available at: {\sf https://elibrary.ru/item.asp?id=36272286} (accessed October~5, 
2018).

\vspace*{1pt}

\bibitem{16-grn-1}
\Aue{Grinchenko, S.\,N.} 2012. Posleslovie [Afterword]. \textit{Mat-ly doklada 
na Sovmestnom nauchnom seminare IPI \mbox{INION} RAN ``Metodologicheskie 
problemy nauk ob informatsii''}  [Report materials at the Joint Scientific 
Seminar of the Institute of Informatics Problems of the Russian Academy of 
Sciences and the Institute of Scientific Information on Social Sciences of the 
Russian Academy of Sciences ``Methodological problems of information 
sciences''].  Moscow. 5--8.  Available at: {\sf 
http://legacy. inion.ru/files/File/MPNI\_9\_13\_12\_12\_posl.pdf} (accessed 
October~5, 2018).

\vspace*{1pt}

\bibitem{17-grn-1}
\Aue{Zhirmunskiy, A.\,V., and V.\,I.~Kuz'min.} 1982. \textit{Kriticheskie urovni 
v~protsessakh razvitiya biologicheskikh sistem} [Critical levels in the development 
of biological systems]. Moscow: Nauka. 179~p.

\vspace*{1pt}

\bibitem{18-grn-1}
\Aue{Shchapova, Y.\,L.} 2000. Khronologiya i~periodizatsii drev\-ney\-shey istorii 
kak chislovaya posledovatel'nost' (ryad Fibonachchi) [Chronology and 
periodization of ancient history as a numerical sequence (Fibonacci's series)]. 
\textit{Informatsionnyy byulleten' Assotsiatsii ``Istoriya i~komp'yuter''} 
[Newsletter of the Association ``History and Computer'']  25.

\vspace*{1pt}

\bibitem{19-grn-1}
\Aue{Shchapova, Y.\,L.} 2005. \textit{Arkheologicheskaya epokha: khro\-no\-lo\-giya, 
periodizatsiya, teoriya, model'} [Archaeological epoch: Chronology, periodization, 
theory, model]. Moscow: KomKniga, 192~p.

\vspace*{1pt}

\bibitem{20-grn-1}
\Aue{Shchapova, Y.\,L.} 2011. \textit{Material'noe proizvodstvo 
v~arkheologicheskuyu epokhu} [Material production in the archaeological epoch]. 
St.\ Petersburg: Aleteyya. 244~p.

\vspace*{1pt}

\bibitem{21-grn-1}
\Aue{Grinchenko, S.\,N., and Yu.\,I.~Shchapova.} 2014. Prostranstvo i~vremya 
v~arheologii. Chast'~3. O~metrike bazisnoy prostranstvennoy struktury 
chelovechestva v~arkheologicheskuyu epokhu [Space and time in archeology. 
Part~3. About the metric of Humankind basic spatial structure  in  
archaeological epoch]. \textit{Space and Time}  
1(15):\linebreak 78--89.
\end{thebibliography}

 }
 }

\end{multicols}

\vspace*{-6pt}

\hfill{\small\textit{Received October 17, 2018}}

%\pagebreak

%\vspace*{-18pt}


  
  \Contrl
  
  \noindent
   \textbf{Grinchenko Sergey N.} (b.\ 1946)~--- Doctor of Science in technology, professor, principal 
scientist, Institute of Informatics Problems, Federal Research Center ``Computer Science and 
Control'' of the Russian Academy of Sciences, 44-2~Vavilov Str., Moscow 119333, Russian 
Federation; \mbox{sgrinchenko@ipiran.ru}
\label{end\stat}

\renewcommand{\bibname}{\protect\rm Литература}   %14
%\newcommand{\Var}{\ensuremath{{\rm\mathbb{V}ar}}}

\renewcommand{\figurename}{\protect\bf Figure}
\renewcommand{\tablename}{\protect\bf Table}

\def\stat{lukashenko}


\def\tit{A~GAUSSIAN APPROXIMATION OF THE DISTRIBUTED COMPUTING PROCESS}

\def\titkol{A~Gaussian approximation of the distributed computing process}

\def\autkol{O.\,V.~Lukashenko, E.\,V.~Morozov,  and~M.~Pagano}

\def\aut{O.\,V.~Lukashenko$^{1}$, E.\,V.~Morozov$^2$,  and~M.~Pagano$^{3}$}

\titel{\tit}{\aut}{\autkol}{\titkol}

%{\renewcommand{\thefootnote}{\fnsymbol{footnote}}
%\footnotetext[1] {The study was carried out under state order to the Karelian Research 
%Centre of the Russian Academy of Sciences (Institute of Applied Mathematical 
%Research KarRC RAS) and supported by the Russian Foundation for Basic Research, 
%projects 18-07-00187, 18-07-00147, 18-07-00156, 19-07-00303.}}

\renewcommand{\thefootnote}{\arabic{footnote}}
\footnotetext[1]{Institute of  Applied Mathematical Research of Karelian Research Centre of RAS, 
11~Pushkinskaya Str.,  Petrozavodsk 185910, Republic of Karelia, Russian Federation; 
Petrozavodsk State University, 33~Lenin Str., Petrozavodsk 185910, Republic of Karelia, 
Russian Federation,  \mbox{lukashenko@krc.karelia.ru}}
\footnotetext[2]{Institute of  Applied Mathematical Research of Karelian Research Centre of RAS, 
11~Pushkinskaya Str.,  Petrozavodsk 185910, Republic of Karelia, Russian Federation; 
Petrozavodsk State University, 33~Lenin Str., Petrozavodsk 185910, Republic of Karelia, 
Russian Federation, \mbox{emorozov@karelia.ru}}
\footnotetext[3]{University of Pisa, 43~Lungarno Pacinotti, Pisa 56126, Italy, \mbox{m.pagano@iet.unipi.it}}


\index{Lukashenko O.\,V.}
\index{Morozov E.\,V.}
\index{Pagano M.}
\index{Лукашенко О.\,В.}
\index{Морозов Е.\,В.}
\index{Пагано М.}


\def\leftfootline{\small{\textbf{\thepage}
\hfill INFORMATIKA I EE PRIMENENIYA~--- INFORMATICS AND
APPLICATIONS\ \ \ 2019\ \ \ volume~13\ \ \ issue\ 2}
}%
 \def\rightfootline{\small{INFORMATIKA I EE PRIMENENIYA~---
INFORMATICS AND APPLICATIONS\ \ \ 2019\ \ \ volume~13\ \ \ issue\ 2
\hfill \textbf{\thepage}}}

%\vspace*{-2pt}



 
\Abste{The authors propose a~refinement of the stochastic model 
describing the dynamics of the Desktop Grid (DG) project with many hosts and many
 workunits to be performed, originally proposed  by Morozov \textit{et al.}\ in 2017.
The target performance measure  is the mean  duration of the runtime of the project. 
To this end,  the authors derive  an asymptotic expression for the  amount 
of the accumulated work to be done by means of 
    limit theorems for  superposed on-off sources that   lead to a~Gaussian 
    approximation. In more detail, depending on the distribution of active 
    and idle periods, Brownian  or fractional Brownian processes are obtained.
    The authors present the analytic results related to the hitting time of 
    the considered processes (including the case in which the overall amount of
work is only known in a~probabilistic way), and highlight how the 
     runtime tail distribution could be estimated by simulation. Taking 
     advantage of the properties of Gaussian processes and the Conditional 
     Monte-Carlo (CMC) approach, the authors present a~theoretical framework for 
     evaluating the runtime tail distribution.}


\KWE{Gaussian approximation; distributed computing; fractional Brownian motion}

 \DOI{10.14357/19922264190215}


%\vspace*{8pt}


\vskip 12pt plus 9pt minus 6pt

 \thispagestyle{myheadings}

 \begin{multicols}{2}

 \label{st\stat}



\section{Introduction}

\vspace*{-4pt}

\noindent
Gaussian processes are widely used in the performance analysis of telecommunication 
systems for their analytic tractability and arguments based on the central-limit 
theorem that make them suitable in case of a~large number of independent contributions.  For instance, these  models are able to capture, in a simple and parsimonious way,
the properties of self-similarity and long-range dependence, inherent to multimedia
network traffic~\cite{2-luk-1, 3-luk-1}. 
These properties dramatically increase the difficulty of the probabilistic 
analysis and, as a consequence, in many cases only Monte-Carlo simulation can be used.
The \textit{fractional Brownian motion} (FBM) is one of the most studied 
self-similar long-range dependent Gaussian processes due to its simplicity. 
Its use as traffic model is supported  by the following theoretical analysis~\cite{4-luk-1}: 
the sum of an increasing  number of the so-called on-off inputs,
with either on-times or off-times having a~heavy-tailed distribution
with infinite variance, converges weakly to an~FBM, after an
appropriate time  scaling.


In this paper,  the applicability of FBM for high-performance computing is considered. 
In that framework, computing clusters and computational Grid systems are the 
main tools: computing clusters are based on computing nodes connected by 
a~high-speed network, while computational Grid systems include geographically 
dispersed computing nodes connected by a relatively slow network. 

Desktop Grid belongs to the latter class. The DG combines nondedicated 
\textit{hosts} (typically, desktops/laptops owned by volunteers) over the Internet 
to process loosely coupled \textit{workunits} (computational tasks). 
Desktop Grids utilize the idle host resources, providing  potentially huge, although highly 
variable, computing power. (For example, the DG project Einstein\@HOME aggregates 
peak performance at about~1~PetaFLOPS~[4].) Typically,  DGs are 
managed by a scientific community that utilizes the resources to complete 
a~\textit{DG project} which consists of a~(usually finite) number of workunits.
 Thus, the \textit{runtime} of the DG project  is the time to complete all the 
 workunits and it is desirable to minimize it.

Minimization of the DG project runtime may be performed by means of 
scheduling optimization~\cite{6-luk-1, 7-luk-1, 8-luk-1}. Additional information on the hosts, 
such as reliability and availability, can be used to improve the efficiency of
 DGs~\cite{9-luk-1, 10-luk-1}, In~\cite{11-luk-1, 12-luk-1}, the focus is placed on the so-called 
 workunit replication mechanism for reliability purposes. However,  to the 
 best of our knowledge, the estimation of the runtime of a~DG project remains
  generally an unsolved issue, and it is the main motivation of this paper.

Desktop Grids have several important distinctive  features 
when compared to computational Grids or computing clusters. First of all, 
hosts, being nondedicated, possess individual availability periods. Moreover, 
the management server of a~DG is not able to obtain information on the current 
state of the hosts (such as ``computing,'' ``suspended,'' etc.). 
These two issues  make the estimation of the runtime of a~DG project a~hard problem. 

The execution of a~DG project can be divided into two stages. 
During the first phase, the number of work\-units is greater than the number of 
hosts and, thus,  each host will receive at least one workunit. 
In the second stage, all the available workunits are dispatched and
 there are available (idle) hosts. In this paper, the focus is on the duration of  
 the first phase  which is studied by means of a Gaussian approximation of the 
 overall work.  The   study of the second stage requires a completely different 
 probabilistic technique, which relies on  the theory of order statistics and the 
 asymptotic properties of renewal processes, and is postponed for a future work.
Thus, in what follows, runtime will relate to the first stage of the project solely.


We describe the availability patterns of the hosts by treating each of them as an 
individual on-off source which processes workunits during on periods.  
Our approach  is based on the asymptotics of the (properly scaled) superposition 
of a large number of independent on-off sources. It is well-known~\cite{4-luk-1} that 
after an appropriate scaling, the limiting process describing the summary workload 
in the system turns out to be \textit{Brownian motion} (BM), when the sojourn times 
are light-tailed, while it becomes \textit{fractional Brownian motion} 
in case of heavy-tailed sojourn times.  
Then, the problem reduces to the calculation of the hitting time of the given
 threshold~$D$ by the process of accumulated work which is a~well-known topic 
 in probability theory.

The paper is organized as follows. Section 2 presents the theoretical background 
related to  FBM, including functional limit theorems for the cumulative  work 
performed  by an increasing  number of on-off sources.    
Then, Section~3 describes the model and summarizes the available analytic results, 
while Section~4 is devoted to the evaluation of the runtime tail distribution 
by means of the CMC method which potentially leads to variance 
reduction of the estimate of the runtime. Finally, in Section~5, the main 
contributions of the paper are presented and some future research issues are discussed. 

\section{Theoretical Background}

\noindent
In this section, let us recall the basic definitions about FBM and how it is 
related to the limiting theorems for the superposition of independent 
\textit{on-off sources}.

\subsection{Fractional Brownian motion}

\noindent
The FBM $\left\{B_H(t), t \in \mathbb{R} \right\}$ is a~Gaussian centered process 
with $B_H(0)=0$, stationary increments, and the following covariance function:
  \begin{multline*}
       K_H(t,s) := \mathbb{E} \left[ B_H(t) \: B_H(s) \right]\\
       {}=
       \fr{1}{2} \left[
       |t|^{2H} - |t-s|^{2H} + |s|^{2H} \right],\enskip s,\,t\ge 0 
     \end{multline*}
where $H \in (0,1)$ is the so-called \textit{Hurst parameter}. 
It is easy to verify that~$B_H(t)$  is a self-similar process with 
self-similarity parameter~$H$, i.\,e., for each $c>0$,
$$
  c^{-H}B_H(ct) \stackrel{d}{=} B_H(t)
$$
where  $\stackrel{d}{=}$  denotes equality in distribution.

Fractional Brownian motion is widely used for modeling purposes due to its 
Gaussianity (that typically arises under aggregation conditions) 
and parsimonious description (apart from mean and variance, its behavior 
is unambiguously determined by~$H$).

When $H>1/2$,  FBM is a long-range dependent process since the
 autocorrelation of the corresponding increment process is nonsummable. 
 For more details on FBM and its properties, see~\cite{13-luk-1}.


\subsection{Limit theorems for distributed computing processes}



\noindent
Let us assume that the DG consists of~$N$ heterogeneous hosts which can be 
considered as  independent \textit{on-off sources}. In more detail, let us suppose 
that there are~$n$~types of hosts ($n<N$) and denote by~$N_i$ the number of  
$i$-type hosts, i.\,e., $\sum\nolimits_{i=1}^n N_i=N$. 
Moreover, let~$R_i$ denote the amount of processed work per unit time for $i$-type 
hosts and let $\left\{I^{(i)}(t),\ t\geq0\right\}$,
\begin{equation*}
I^{(i)}(t)=\begin{cases}
 R_i\,, &t\in \mbox{on-period}\,; \\
 0\,, & t\in \mbox{off-period}\,, 
\end{cases}
 %\label{taq1}
\end{equation*}
be the  \textit{on-off} process that characterizes the activity/silent 
periods of the corresponding hosts (Fig.~1).
For sake  of simplicity, it is assumed that for each host, both \textit{on} 
and  \textit{off} periods are sequences of i.i.d.\ (independent
and identically distributed)
random variables (RVs) and mutually independent.
Moreover, as already stated, the\linebreak\vspace*{-12pt}

{ \begin{center}  %fig1
 \vspace*{9pt}
   \mbox{%
 \epsfxsize=79mm 
 \epsfbox{luk-1.eps}
 }


\vspace*{6pt}


\noindent
{{\figurename~1}\ \ \small{On-off model}}
\end{center}
}

%\vspace*{9pt}

\addtocounter{figure}{1}


\noindent
on-off processes modeling the contribution of 
different hosts are assumed to be independent.



The \textit{cumulative processed work}, i.\,e., the aggregated amount of  
work provided by all $N$ hosts, during the time interval $[0,t]$ is given by
$$
A(t)=\int\limits_0^{t} \left( {\sum\limits_{i=1}^n\sum\limits_{k=1}^{N_i} 
{I_{k}^{(i)}(u)} } \right)\,du
$$
where $I_k^{(i)}$ are the independent copies of~$I^{(i)}$, $i=1,\ldots\linebreak \ldots,n$. 
Moreover, for the $i$-type ($i=1,\dots,n$) hosts, let us denote by
$\mu_{\mathrm{on}}^i$, $\sigma_{\mathrm{on}}^i$, $\mu_{\mathrm{off}}^i$, 
and~$\sigma_{\mathrm{off}}^i$
the mean length and  standard deviation (that may be infinite) of 
the duration of the  on and  off periods, respectively.


The statistical behavior of~$A(t)$ is determined by the distributions~$F_{\mathrm{on}}^i$ 
and~$F_{\mathrm{off}}^i$ of the on and off periods for each type of hosts, namely,
 by their tail. 
In more detail, in case of infinite variance, let us assume that as $x \to \infty$,
\begin{align*}
1-F_{\mathrm{on}}^i(x)& \sim  \ell_{\mathrm{on}}^i x^{-\alpha_{\mathrm{on}}^i}L_{\mathrm{on}}^i(x)\,;\\
1-F^i_{\mathrm{off}}(x)& \sim  \ell_{\mathrm{off}}^i x^{-\alpha_{\mathrm{off}}^i}L_{\mathrm{off}}^i(x)
%  \label{3}
\end{align*}
where   $a \sim b$  means that $a/b\to 1$;    
$\ell_{\mathrm{on}}^i$ and~$\ell_{\mathrm{off}}^i$ are the positive constants; the 
exponents~$\alpha_{\mathrm{on}}^i$ and~$\alpha_{\mathrm{off}}^i\in (1,\,2)$; 
and the functions~$L_{\mathrm{on}}^i$ 
and~$L_{\mathrm{\mathrm{off}}}^i$ are slowly varying at infinity, i.\,e., for any $t >0$,
$$
\lim_{x \to \infty} \fr{L^i(tx)}{L^i(x)}=1\,,\enskip i=1,\ldots,n\,.
$$
Instead, if $\sigma_{\mathrm{on}}^i$ and~$\sigma_{\mathrm{off}}^i <\infty$, we  set 
$\alpha_{\mathrm{on}}^i=\alpha_{\mathrm{off}}^i=2$. 

It has been  shown in~\cite{4-luk-1} that the scaled process of cumulative  
work arrived during interval  $[0,\,Tt]$ converges weakly to a sum of the i.i.d.\
 FBM's,  provided that 
\begin{enumerate}[(1)]
    \item $N_i\to \infty$ such that
$\lim\nolimits_{N\to \infty}N_i/N>0$, $i=1,\ldots\linebreak \ldots , n$; and
\item  the scaling factor $T\to \infty$.
\end{enumerate}

This \textit{functional limit theorem} leads to the following approximation:
\begin{multline*}
A(tT)\approx T\left( \sum\limits_{i=1}^n R_i N_i
\fr{\mu_{\mathrm{on}}^i}{\mu_{\mathrm{on}}^i+\mu_{\mathrm{off}}^i} \right)t \\
{}+ \sum\limits_{i=1}^n
T^{H_i}R_i \sqrt{L_i(T)N_i}c_i B_{H_i}(t) 
%\label{approx1}    
\end{multline*}
where $c_i$ are the positive constants; $L_i$ are the slowly varying at
infinity functions (expressed in terms of the given  parameters); and
$B_{H_i}$ are the independent FBMs with the Hurst parameters~$H_i$ given by
$$
H_i=\fr{3-\min(\alpha_{\mathrm{on}}^i,\,\alpha_{\mathrm{off}}^i)}{2}\in
\left(\fr{1}{2},\,1 \right),\enskip i=1,\ldots, n\,.
$$
Thus, the cumulative work processed by a~large number of independent hosts 
(with heavy-tailed distributions of the on-off periods) is approximated by 
a~superposition of independent FBMs  $\{B_{H_i}(t)\}$, 
$i=1,\dots,n$,  with a~linear drift that depends on the rates~$R_i$ and 
the average duty cycle.

Instead, if for all types of hosts the variances of the sojourn times 
are finite (i.\,e., $\sigma_{\mathrm{on}}^i,\,\sigma_{\mathrm{off}}^i<\infty$ 
$\forall i=1,\dots,n$), then the limiting (scaled) process becomes
\begin{equation*}
T\left( \sum\limits_{i=1}^n \fr{R_i N_i \mu_{\mathrm{on}}^i}{\mu_{\mathrm{on}}^i+\mu_{\mathrm{off}}^i}
\right)t + \left(\sqrt{T}\sum\limits_{i=1}^n R_i \sqrt{N_i}c_i \right)W(t)
%\label{bm-l1}
\end{equation*}
where $W(t)$ is the Wiener process, and the constants~$c_i$ are given by
$$
c_i = \sqrt{ \fr{ (\mu_{\mathrm{off}}^i\sigma_{\mathrm{on}}^i)^2 + 
\left(\mu_{\mathrm{on}}^i\sigma_{\mathrm{off}}^i\right)^2 }{\left(\mu_{\mathrm{on}}^i+\mu_{\mathrm{off}}^i\right)^3}}\,.
$$
Finally, it is worth mentioning that taking the limits in reverse order, 
the (scaled) process of cumulative work converges to a~\textit{Levy stable motion}, 
an infinite variance process with stationary and independent increments~\cite{14-luk-1}; 
however, such  model is beyond the scope of this paper as in DG, the experimental 
data confirmed the convergence to processes with finite variance.


\section{Model Description and~Performance Measures}


\noindent
The above functional limit theorems provide a~theoretical motivation to 
consider the following model for the cumulative processed work: 
\begin{equation}
    A(t) = m t + X(t)
    \label{6}
\end{equation}
where $X$ is the centered Gaussian process with stationary increments 
(FBM or the sum of independent FBM, in case of heterogeneous systems),
 which describes random fluctuations around the linearly  increasing mean. 
 Such type of stochastic process was previously suggested as the model of 
 network traffic (see~\cite{15-luk-1} for more details). 


Let us denote  by~$\tau_D$ the  runtime of the  DG project where~$D$ 
denotes the required amount of work. Thus,~$\tau_D$ represents the  
\textit{hitting time} of  the process~$\{A(t)\}$:  
\begin{equation*}
\tau_D = \min\{t:\,\, A(t)\ge D \} \, ,
\end{equation*}
i.\,e., the first time the process~$\{A(t)\}$ hits the threshold~$D$.
Then, the original problem is reduced to the calculation (or estimation)
 of some useful performance characteristics, such as the mean  hitting time.

\subsection{Available analytic results}

\noindent
Let us recall the available analytic results for different types 
of Gaussian processes, corresponding to the different limiting cases. 

\vspace*{-4pt}

\subsubsection{Wiener case}

\noindent
When $X$ is a Wiener process (i.\,e., $X = \sigma B_{1/2}$), 
the  density of~$\tau_D$ is available in  explicit form~\cite{16-luk-1}:
\begin{multline}
\mathbb{P}(\tau_D \in dt)= \fr{D}{\sqrt{2\pi}\sigma t^{3/2}} 
\exp \left( -\fr{(D-mt)^2}{2\sigma^2 t} \right)\,dt\\
{=:} f_\tau(t|D)\,dt\,.
\label{8}
\end{multline}
In this case, the corresponding expected value $\mathbb{E} [\tau_D ]$ 
is  the ratio between the given  amount of the work~$D$ and the mean processing 
rate~\cite{16-luk-1}:
$$
\mathbb{E} \left[\tau_D\right] = \fr{D}{m}\,.
$$

\vspace*{-12pt}


\subsubsection{Fractional Brownian motion case}



\noindent
When the limiting process is an FBM,  only asymptotic results and some bounds
 for the  distribution of~$\tau_D$ are available.

In~\cite{17-luk-1}, the following bounds (quite inaccurate when~$H$ is close to~1, 
see Fig.~2)
 for the moments of the hitting time were obtained for $1/2 \le H < 1$:
\begin{multline*}
\fr{1}{\sqrt{2\pi}} \left( \fr{2 H D}{n-H}\, L_n (D,H,m)\right.\\ - 
\left.\fr{(2H-1)m}{n+1-H} \,L_{n+1}(D,H,m) \right) \le \mathbb{E} \left[\tau_D^n\right]\\
   \le \fr{1}{\sqrt{2\pi}} \left( \fr{ H D}{n-H} \,L_n (D,H,m)\right.\\
   \left.{} + 
   \fr{(1-H)m}{n+1-H}\, L_{n+1}(D,H,m) \right)
\end{multline*}

{ \begin{center}  %fig2
% \vspace*{9pt}
   \mbox{%
 \epsfxsize=79mm 
 \epsfbox{luk-2.eps}
 }


\end{center}


\noindent
{{\figurename~2}\ \ \small{Bounds for the mean hitting time ($D=10$, $m=3$): 
\textit{1}~--- $D/m$;
\textit{2}~--- lower bound;
and \textit{3}~--- upper bound}}
}

%\vspace*{9pt}

\addtocounter{figure}{1}


\noindent
where
\begin{multline*}
L_n(D,H,m) \\
{}=\!\! \int\limits_0^\infty \!\exp\left\{ -\fr{1}{2} \!
\left(\!Dt^{-H/(n-H)}-mt^{(1-H)/(n-H)} \right)^2 \!\right\}dt.\hspace*{-3.80858pt}
\end{multline*}
Additionally, the following asymptotic was derived for the large values of level~$D$:
\begin{equation*}
    \lim\limits_{D \to \infty} \fr{\mathbb{E} \left[\tau_D^n\right]}{D^n} = m^{-n}
\end{equation*}
for all $n \ge 1$, $m>0$, from which it is quite straightforward to show that 
for all $n \ge 1$, 
\begin{equation*}
     \fr{\tau_D}{D} \overset{L_n}{\longrightarrow} \fr{1}{m}\enskip 
     \mbox{as} \enskip D \to \infty
\end{equation*}
where $\overset{L_n}{\longrightarrow}$ means convergence in~$L_n$~space.

\subsubsection{General case}

\noindent
In the general case,  to derive asymptotic (for large values of~$D$) 
for the distribution of~$\tau_D$,  it is possible to take advantage of the 
following identity: 
\begin{equation*}
\mathbb{P}\left(\tau_D \le  T\right) = 
\mathbb{P}\left(\sup\limits_{t \in [0,T]}A(t)\ge D\right)\,.
\end{equation*}

The distribution of the maximum of Gaussian processes over a~finite interval is 
a~well-studied  problem. In more detail, for any Gaussian process with stationary 
increments and strictly monotonically increasing and
convex variance such that $\lim\nolimits_{t\to 0} \mathrm{Var}(X(t))/t=0$, the following asymptotic 
holds~\cite{18-luk-1}:
\begin{multline*}
\mathbb{P}\left(\sup\limits_{t \in [0,T]}A(t)\ge D\right) \sim 
\Phi \left( \fr{D-mT}{\sqrt{\mathrm{Var}(X(T))}} \right)\\
 \mbox{ as } D \to \infty
\end{multline*}
where~$\Phi$ denotes the tail distribution of the standard normal RV~$N(0,1)$.

\subsection{A~possible  generalization} 



 \noindent
 It seems quite natural to consider the setting in which the threshold~$D$  
 is an~RV which is independent of the process~$X$ in~(\ref{6}). 
 Such a~setting seems to be highly motivated by practice because it is more 
 realistic that the exact value of   the quantity~$D$ is not available, 
 and it is known in part. This incomplete information can be reflected by  
 introducing the probability density
 function (PDF)~$f_D$ of~$D$, which is assumed to be predefined.   
 Provided that~$X$ in~(\ref{6}) is  a~Wiener process and, hence, the conditional 
 density $f_\tau(t|D)$ in~(\ref{8}) is known, one can write  the density of the RV~$\tau_D$ as
$$
f_\tau(x)=\int\limits_{y=0}^\infty 
f_\tau(x|y)f_D(y)\,dy\,.
$$
In general, one
can calculate this density only by numerical methods 
but for some cases, it is possible to derive its expression in terms of special
 functions. For example, when~$D$  is  exponential with parameter~$\lambda$, 
 one can obtain the following expression:
\begin{multline}
f_\tau(x) = \fr{\lambda}{\sqrt{2\pi}\sigma x^{3/2}}\exp
\left( -\fr{m^2 x}{2\sigma^2} +\fr{\gamma^2}{8\beta(x)}\right)\\
 \times
(2\beta(x))^{-1/2} D_{-1}\left( \fr{\gamma}{\sqrt{2\beta(x)}} \right) 
\label{dens}
\end{multline}
where 
$$
\gamma=\lambda-\fr{m}{\sigma^2}\,;\qquad \beta(x) = \fr{1}{2\sigma^2 x}\,;
$$
and $D_p$, $\mathrm{Re}\, p <0$, is the parabolic cylinder function~\cite{19-luk-1}. 
Numeric calculation of the expression~\eqref{dens} is shown in 
Fig.~3.

{ \begin{center}  %fig3
 \vspace*{6pt}
  \mbox{%
 \epsfxsize=78.984mm 
 \epsfbox{luk-3.eps}
 }


\end{center}


\noindent
{{\figurename~3}\ \ \small{Probability density function of~$\tau_D$ for different values of~$m$ 
($\lambda=1$): \textit{1}~--- $m=1$; \textit{2}~--- $2$;  and \textit{3}~--- $m=3$}}
}

\vspace*{-3pt}

\addtocounter{figure}{1}


\section{Estimation via Monte Carlo}

\noindent
A more flexible alternative to analytic results is represented  by simulation 
that in our case can be used to estimate 
\begin{equation*}
\pi(T):=\mathbb{P} \left(\tau_D > T\right)\,.
\end{equation*}
Such probability could be extremely small for large values of~$T$; 
thus, its estimation with a~given accuracy requires to generate a~large number of 
sample paths of the process~$X$. However, for such type of rare events, it is 
possible to apply a special case of the well-known CMC 
method which always leads to variance reduction.

The method, originally proposed by some of the authors 
in~\cite{20-luk-1, 21-luk-1, 22-luk-1} and named Bridge Monte Carlo (BMC), 
is based on the idea of expressing the target probability as the
expectation of a function of the {Bridge} $Y:=\{Y_t\}$ of the
Gaussian process~$X$, i.\,e.,  the process obtained by
conditioning~$X$ to reach a certain level at some prefixed time instant~$\tau$:
\begin{equation*}
Y(t) = X(t) - \psi(t) X(\tau)
\end{equation*}
where $\psi$ can be easily  expressed in terms of the the covariance 
function~$\Gamma(s,t)$ of the process~$X$ 
$$
\psi(t)   :=
\fr{\Gamma(t,\tau)}{\Gamma(\tau, \tau)} \,.  
$$
Since the variance of~$X$ is an increasing function of~$t$ in all models we consider,
it is easy to see that  $\psi(t)>0$ for all $t \ge 0$.
Moreover, for any~$t$, $Y(t)$ is
independent of~$X(\tau)$ since
$$
\mathbb{E} \left[X(\tau)Y(t)\right]=
\Gamma(\tau,t)-
\fr{\Gamma(t,\tau)}{\Gamma(\tau,\tau)}\,\Gamma(\tau, \tau)=0
$$
and $(X(\tau),Y(t))$ has bivariate normal distribution.





Let $\mathbb{T} = [0,T]$, then the target probability can 
be expressed in the following way:
\begin{multline*}
\pi(T)\, =\mathbb{P}\left(\sup\limits_{t \in [0,T]}A(t)\ge D\right)\\
{}=\mathbb{P}\left(\forall t \in \mathbb{T}:\,\,mt+X(t) \le D\right)\\
{}=\mathbb{P}\left( \forall t \in \mathbb{T}:\,\, X(\tau) \le \fr{D-Y(t)-mt}{\psi(t)}\right)\\
{}=\mathbb{P}\left( X(\tau)\le \inf_{t \in T} \fr{D-Y(t)-mt}{\psi(t)}\right)\\
{}=\mathbb{P}\left( X(\tau)\le\overline{Y} \right)
\end{multline*}
where
\begin{equation*}
\overline{Y}:=\inf\limits_{t \in \mathbb{T}}\fr{D-Y(t)-mt}{\psi(t)}\,. 
%\label{BMC_4}
\end{equation*}

Finally, the  considered probability can be rewritten as follows:

\begin{equation*}
\pi(T)=\mathbb{P}\left(X_{\tau}\le\overline{Y}\right)=
\mathbb{E}\left[\Psi\left(\fr{\overline{Y}}{ \sqrt{\Gamma(\tau,\,\tau)}}\right)\right]
\end{equation*}
where independence between~$\overline{Y}$ and~$X_{\tau}$ is used and~$\Psi$ denotes 
the cumulative distribution function of a~standard normal variable.

Hence, given $N$ samples $\{\overline{Y}^{(n)},\,\,n=1,\ldots,N\}$  of~~$\overline{Y}$,
the estimator of~$\pi(T)$ is defined as follows:
\begin{equation*}
\widehat{\pi}_N^{\mathrm{BMC}} \: := \: \fr{1}{N}\sum\limits_{n=1}^N
\Psi
\left(\fr{\overline{Y}^{(n)}}{\sqrt{\Gamma(\tau,\tau)}}\right).
%\label{estimator}
\end{equation*}


Note that
\begin{equation*}
\Psi\left(\fr{\overline{Y}}{ \sqrt{\Gamma(\tau,\,\tau)}}\right)=
\mathbb{E} \left[ I(X(\tau) \le \overline{Y}) | \overline{Y}\right]
%\label{BMC-5}
\end{equation*}
and, therefore, the BMC approach is actually a special case of the  
CMC method;
so, one can expect that the BMC estimator implies variance reduction  (with
regard to crude 
Monte-Carlo simulation) in   the estimation of the target probability~$\pi(T)$ as also 
justified by the previous experience when such a~method was successfully applied for 
estimation some other rare-event probabilities related to Gaussian 
processes~\cite{23-luk-1}.  


\section{Concluding Remarks and~Future~Research}

\noindent
 In this paper,   a~stochastic model describing the dynamics of 
 a~DG project with many hosts and many work\-units to be performed, originally
 proposed in~\cite{1-luk-1},  is presented. 
 It is assumed that the project  can be   described by the so-called on-off model 
 where the hosts are on-off  sources of the work\-units  and the basic process is the 
 completed work. It is assumed that  the hosts' working sessions can have both 
  light- and heavy-tailed distributions.
 Then, an approximation of the  basic process, based on  the asymptotics of 
 the superposed on-off sources, is applied.    
 The suggested approach   leads to a~Gaussian approximation of the process of the 
 completed work. Finally,  a~simulation framework for the evaluation 
 of the runtime of the project, using the properties of Gaussian processes and 
 CMC simulation, is presented. 

Although  this note is focused on estimation of the runtime related to the 
1st stage of the project completion when the number of workunits is bigger 
than the number of hosts,  the 2nd  stage 
could also be relevant.
In more detail, it can be considered as a~collection of the ``tails''  
of the  workunit remaining times. From this point of view, the completion time of the 
2nd stage of the project  can be interpreted as  the \textit{longest} remaining time 
and analyzed by means of the asymptotic results of \textit{renewal theory}.
 Moreover, since the workunits are assumed to be independent, to evaluate the  
 duration of the  2nd stage, it seems promising to  apply the theory of 
 \textit{order statistics} and interpret the completion  time as the maximal 
 order statistics. 


\Ack
\noindent
The study was carried out under state order to the Karelian Research 
Centre of the Russian Academy of Sciences (Institute of Applied Mathematical 
Research KarRC RAS) and supported by the Russian Foundation for Basic Research, 
projects 18-07-00187, 18-07-00147, 18-07-00156, and 19-07-00303.


\renewcommand{\bibname}{\protect\rmfamily References}


%\vspace*{-6pt}

{\small\frenchspacing
{ %\baselineskip=10.35pt
\begin{thebibliography}{99}


\bibitem{2-luk-1} %1
\Aue{Leland, W.\,E., M.\,S.~Taqqu, W.~Willinger., and D.\,V.~Wilson.}
 1994. On the self-similar nature of ethernet traffic (extended version). 
 \textit{IEEE ACM~T. Network.} 2(1):1--15.
\bibitem{3-luk-1} %2
\Aue{Willinger, W., M.\,S.~Taqqu, W.\,E.~Leland, and D.~Wilson.}
1995. Self-similarity in high-speed packet traffic: Analysis and modeling of Ethernet 
traffic measurements. \textit{Stat. Sci.} 10(1):67--85.
\bibitem{4-luk-1}  %3
\Aue{Taqqu, M.\,S., W.~Willinger, and R.~Sherman.} 1997. 
Proof of a~fundamental result in self-similar traffic modeling. 
\textit{Comp. Comm.~R.} 27:5--23.
\bibitem{5-luk-1} %4
BOINCstats. 2017. Available at: {\sf https://boincstats.com} (accessed May~7, 2019).

\bibitem{7-luk-1} %5
\Aue{Kondo, D., D.\,P.~Anderson, and J.~McLeod~VII.} 
2007. Performance evaluation of scheduling policies for volunteer computing.  
\textit{3rd IEEE Conference (International) on e-Science and Grid Computing 
Proceedings}. IEEE. 221--227.
\bibitem{6-luk-1} %6
\Aue{Estrada, T., and M.~Taufer.} 2012. Challenges in 
designing scheduling policies in volunteer computing. 
\textit{Desktop grid computing}. Eds C.~C$\acute{\mbox{e}}$rin and G.~Fedak. 
CRC Press. 167--190.

\bibitem{8-luk-1} %7
\Aue{Durrani, N., and J.~Shamsi.} 2014. Volunteer computing: 
Requirements, challenges, and solutions. \textit{J.~Netw. Comput. Appl.} 39:369--380.
\bibitem{9-luk-1} %8
\Aue{Sonnek, J., M.~Nathan, A.~Chandra, and J.~Weissman.}
 2006. Reputation-based scheduling on unreliable distributed infrastructures in 
 distributed computing systems. \textit{26th IEEE Conference 
 (International) on Distributed Computing Systems Proceedings}. IEEE. Art.~No.\,30. P.~1--8.
\bibitem{10-luk-1} %9
\Aue{Watanabe, K., M.~Fukushi, and M.~Kameyama.}
 2011. Adaptive group-based job scheduling for high performance and reliable 
 volunteer computing.  \textit{J.~Information Processing} 19:39--51.
 
 \bibitem{12-luk-1} %10
\Aue{Xavier, E., R.~Peixoto, and J.~da~Silveira.}
 2013. Scheduling with task replication on desktop grids: 
 Theoretical and experimental analysis.  \textit{J.~Comb. Optim.}
 30(3):520--544.
\bibitem{11-luk-1} %11
\Aue{Chernov, I.\,A., and N.\,N.~Nikitina.}
2015. Virtual screening in a~Desktop Grid: Replication and the optimal quorum. 
\textit{Parallel computing technologies}. Ed. V.~Malyshkin.
Lecture notes in computer science ser. Springer.  9251:258--267.

\bibitem{13-luk-1} %12
\Aue{Samorodnitsky, G., and M.\,S.~Taqqu.} 1994.  \textit{Stable non-Gaussian random processes: 
Stochastic models with infinite variance}. Chapman \& Hall. 632~p.
\bibitem{14-luk-1} %13
\Aue{Mikosch, T., S.~Resnick, H.~Rootz$\acute{\mbox{e}}$n, and A.~Stegeman.}
2002. Is network traffic approximated by stable Levy motion or fractional Brownian 
motion? \textit{Ann. Appl. Probab.} 12(1):23--68.
\bibitem{15-luk-1} %14
\Aue{Norros, I.} 1994. A~storage model with self-similar input.  
\textit{Queueing Syst.} 16:387--396.
\bibitem{16-luk-1} %15
\Aue{Borodin, A.\,N., and P.~Salminen.} 2002.  \textit{Handbook of Brownian motion~--- 
facts and formulae}. Birkh$\ddot{\mbox{a}}$user. 685~p.
\bibitem{17-luk-1} %16
\Aue{Michna, Z.} 1999. On tail probabilities and first passage times for fractional 
Brownian motion.  \textit{Math. Method. Oper. Res.} 49(2):335--354.
\bibitem{18-luk-1} %17
\Aue{Caglar, M., and C.~Vardar.} 2013. Distribution of maximum loss of fractional 
Brownian motion with drift.  \textit{Stat. Probabil. Lett.} 83:2729--2734.
\bibitem{19-luk-1} %18
Gradshtein, I.\,S., I.\,M.~Ryzhik, and A.~Jeffrey, eds.
%D.~Zwillinger, associate ed. 
2015. 
\textit{Table of integrals, series and products}. 8th ed. 
San Diego, CA: Academic Press. 1220~p.
\bibitem{20-luk-1} %19
\Aue{Giordano, S., M.~Gubinelli, and M.~Pagano.}
 2005. Bridge Monte-Carlo: A~novel approach to rare events of Gaussian processes. 
 \textit{5th St. Petersburg Workshop on Simulation Proceedings}. St.\ Petersburg: 
 St. Petersburg State University. 281--286.
\bibitem{21-luk-1} %20
\Aue{Giordano, S., M.~Gubinelli, and M.~Pagano.}
 2007. Rare events of Gaussian processes: A~performance comparison between 
 bridge Monte-Carlo and importance sampling. 
 \textit{Next generation teletraffic and wired/wireless advanced networking}.
 Eds.\ Y.~Koucheryavy, J.~Harju, and A.~Sayenko.
 Lecture notes in computer science ser.
 Springer. 4712:269--280.
\bibitem{22-luk-1} %21
\Aue{Lukashenko, O.\,V., E.\,V.~Morozov, and M.~Pagano.} 
2012. Performance analysis of Bridge Monte-Carlo estimator. 
\textit{Transactions of KarRC RAS} 5:54--60.
\bibitem{23-luk-1} %22
\Aue{Lukashenko, O.\,V., E.\,V.~Morozov, and M.~Pagano.}
 2017. On the efficiency of bridge Monte-Carlo estimator.  
 \textit{Informatika i~ee Primeneniya~--- Inform.  Appl.} 11(2):16--24.
 
 \bibitem{1-luk-1} %23
\Aue{Morozov, E., O.~Lukashenko, A.~Rumyantsev, and E.~Ivashko.}
2017. A~Gaussian approximation of runtime estimation in a desktop grid project. 
\textit{9th  Congress (International) on Ultra Modern Telecommunications and 
Control Systems and Workshops}. IEEE. 107--111.


\end{thebibliography} } }

\end{multicols}

\vspace*{-6pt}

\hfill{\small\textit{Received April 15, 2019}}

\vspace*{-18pt}

\Contr

%\vspace*{-3pt}

\noindent
\textbf{Lukashenko Oleg  V.} (b.\ 1986)~--- 
Candidate of Science (PhD) in physics and mathematics, scientist, 
Institute of  Applied Mathematical Research of Karelian Research Centre of 
the Russian Academy of Sciences, 
11~Pushkinskaya Str.,  Petrozavodsk 185910, Republic of Karelia, 
Russian Federation; associate professor, Petrozavodsk State University, 33~Lenin Str., 
Petrozavodsk 185910, Republic of Karelia, Russian Federation; 
\mbox{lukashenko@krc.karelia.ru}

\vspace*{3pt}

\noindent
\textbf{Morozov  Evsei  V.} (b.\ 1947)~--- 
Doctor of Science in physics and mathematics, professor, leading scientist,
 Institute of  Applied Mathematical Research of Karelian Research Centre of 
 the Russian Academy of Sciences, 11~Pushkinskaya Str.,  Petrozavodsk 185910, 
 Republic of Karelia, Russian Federation; professor, Petrozavodsk State University, 
 33~Lenin Str., Petrozavodsk 185910, Republic of Karelia, Russian Federation; 
 \mbox{emorozov@karelia.ru}
 
 \vspace*{3pt}
 
 \noindent
 \textbf{Pagano Michele} (b.\ 1968)~--- 
 PhD in Information Engineering, associate professor, University of Pisa, 
 43~Lungarno Pacinotti, Pisa 56126, Italy; \mbox{m.pagano@iet.unipi.it}

 


\vspace*{8pt}

\hrule

\vspace*{2pt}

\hrule

%\vspace*{-7pt}

%\newpage

%\vspace*{-28pt}

\def\tit{ГАУССОВСКАЯ АППРОКСИМАЦИЯ ПРОЦЕССА РАСПРЕДЕЛЕННЫХ ВЫЧИСЛЕНИЙ$^*$}

\def\titkol{Гауссовская аппроксимация процесса распределенных вычислений}

\def\aut{О.\,В.~Лукашенко$^{1,2}$, Е.\,В.~Морозов$^{1,2}$,
М.~Пагано$^3$}

\def\autkol{О.\,В.~Лукашенко, Е.\,В.~Морозов,
М.~Пагано}

{\renewcommand{\thefootnote}{\fnsymbol{footnote}} \footnotetext[1]
{Финансовое обеспечение исследований осуществлялось из 
средств федерального бюджета на выполнение государственного задания 
КарНЦ РАН (Институт прикладных математических исследований КарНЦ РАН) 
и~при финансовой поддержке РФФИ (проекты 18-07-00187, 18-07-00147, 18-07-00156
и~19-07-00303).}}



\titel{\tit}{\aut}{\autkol}{\titkol}

\vspace*{-11pt}

\noindent
$^1$Институт прикладных математических исследований Карельского научного центра 
Российской акаде-\linebreak
$\hphantom{^1}$мии наук 

\noindent
$^2$Петрозаводский государственный университет

\noindent
$^3$Университет г.\ Пиза, Италия
%, danielkh@post.bgu.ac.il 

\vspace*{1pt}

\def\leftfootline{\small{\textbf{\thepage}
\hfill ИНФОРМАТИКА И ЕЁ ПРИМЕНЕНИЯ\ \ \ том\ 13\ \ \ выпуск\ 2\ \ \ 2019}
}%
 \def\rightfootline{\small{ИНФОРМАТИКА И ЕЁ ПРИМЕНЕНИЯ\ \ \ том\ 13\ \ \ выпуск\ 2\ \ \ 2019
\hfill \textbf{\thepage}}}

\vspace*{-1pt}


 
\Abst{Продолжено изучение стохастической модели процесса динамики 
выполнения задачи в~сис\-те\-ме Desktop Grid при наличии многих пользователей, 
предложенной в~2017~г.\ Морозовым с~соавт. Тре\-бу\-емой характеристикой 
выступает средняя 
про\-дол\-жи\-тель\-ность времени выполнения проекта. Гауссовская аппроксимация искомого 
процесса производится на основе предельных тео\-рем для суперпозиции on-off 
источников. Приведен обзор известных аналитических результатов для 
тре\-бу\-емой характеристики, вклю\-чая результаты для броуновского 
и~дроб\-но\-го броуновского движения. Также показывается, как с~по\-мощью условного метода 
Мон\-те-Кар\-ло оценить хвост распределения времени выполнения проекта.}


\KW{гауссовская аппроксимация; распределенные вычисления; дробное броуновское движение}

 \DOI{10.14357/19922264190215}



%\vspace*{-3pt}


 \begin{multicols}{2}

\renewcommand{\bibname}{\protect\rmfamily Литература}
%\renewcommand{\bibname}{\large\protect\rm References}

{\small\frenchspacing
{\baselineskip=10.5pt
\begin{thebibliography}{99}
%\vspace*{-3pt}


\bibitem{2-luk} %1
\Au{Leland W.\.E., Taqqu~M.\,S., Willinger~W., Wilson~D.\,V.}
On the self-similar nature of Ethernet traffic (extended version)~// 
IEEE ACM~T. Network., 1994. Vol.~2. Iss.~1. P.~1---15.
\bibitem{3-luk} %2
\Au{Willinger W., Taqqu~M.\,S., Leland~W.\,E., Wilson~D.}
 Self-similarity in high-speed packet traffic: Analysis and modeling of Ethernet 
 traffic measurements~// Stat. Sci., 1995. Vol.~10. Iss.~1. P.~67--85.
\bibitem{4-luk} %3
\Au{Taqqu M.\,S., Willinger~W., Sherman~R.} 
Proof of a~fundamental result in self-similar traffic modeling~// 
Comp. Comm.~R., 1997. Vol.~27. P.~5--23.
\bibitem{5-luk} %4
 BOINCstats, 2017. {\sf https://boincstats.com}.

\bibitem{7-luk} %5
\Au{Kondo D., Anderson~D.\,P., McLeod~VII~J.}
Performance evaluation of scheduling policies for volunteer computing~// 
3rd IEEE  Conference (International) 
on e-Science and Grid Computing Proceedings.~--- IEEE, 2007. P.~221--227.

\bibitem{6-luk} %6
\Au{Estrada T., Taufer~M.}
Challenges in designing scheduling policies in volunteer computing~// 
Desktop grid computing~/ Eds. C.~C$\acute{\mbox{e}}$rin, G.~Fedak.~--- 
CRC Press, 2012. P.~167--190.
\bibitem{8-luk} %7
\Au{Durrani N., Shamsi~J.} 
Volunteer computing: Requirements, challenges, and solutions~// 
J.~Netw. Comput. Appl., 2014. Vol.~39. P.~369--380.
\bibitem{9-luk} %8
\Au{Sonnek J., Nathan~M., Chandra~A., Weissman~J.}
 Reputation-based scheduling on unreliable distributed infrastructures in 
 distributed computing systems~// 26th IEEE Conference (International)
 on Distributed Computing Systems Proceedings.~--- IEEE, 2006. Art. No.\,30. P.~1--8.
\bibitem{10-luk} %9
\Au{Watanabe K., Fukushi~M., Kameyama~M.}
Adaptive group-based job scheduling for high performance and reliable volunteer 
computing~// J.~Information Processing, 2011. Vol.~19. P.~39--51.

\bibitem{12-luk} %10
\Au{Xavier E., Peixoto R., da~Silveira~J.}
 Scheduling with task replication on desktop grids: Theoretical and experimental 
 analysis~// J.~Comb. Optim., 2013.  Vol.~30. Iss.~3. P.~520--544.
 
 \bibitem{11-luk} %11
\Au{Chernov I.\,A., Nikitina~N.\,N.}
 Virtual screening in a desktop grid: Replication and the optimal quorum~// 
  Parallel computing technologies~/ Ed. V.~Malyshkin.~---
 Lecture notes in computer science ser.~---  Springer, 2015.  
 Vol.~9251. P.~258--267.
 
\bibitem{13-luk} %12
\Au{Samorodnitsky G., Taqqu~M.\,S.} Stable non-Gaussian random processes: Stochastic 
models with infinite variance.~--- Chapman \& Hall, 1994. 632~p.
\bibitem{14-luk} %13
\Au{Mikosch T., Resnick~S., Rootz$\acute{\mbox{e}}$n~H., Stegeman~A.}
Is network traffic approximated by stable Levy motion or fractional Brownian motion?~// 
Ann. Appl. Probab., 2002. Vol.~12. Iss.~1. P.~23--68.
\bibitem{15-luk} %14
\Au{Norros I.} A~storage model with self-similar input~// Queueing Syst., 1994. 
Vol.~16. P.~387--396.
\bibitem{16-luk} %15
\Au{Borodin A.\,N., Salminen~P.}
 Handbook of Brownian motion~--- facts and formulae.~--- Birkh$\ddot{\mbox{a}}$user, 2002. 685~p.
\bibitem{17-luk} %16
\Au{Michna Z.} On tail probabilities and first passage times for fractional 
Brownian motion~// Math. Method. Oper. Res., 1999. Vol.~49. Iss.~2. 
P.~335--354.
\bibitem{18-luk} %17
\Au{Caglar M., Vardar~C.} Distribution of maximum loss of fractional 
Brownian motion with drift~// Stat. Probabil. Lett., 2013. Vol.~83. P.~2729--2734.
\bibitem{19-luk} %18
Table of integrals, series and products~/
Eds.  I.\,S.~Gradshtein, I.\,M.~Ryzhik, A.~Jeffrey.~--- 8 ed.~---
%; associate editor D.~Zwillinger.   
San Diego, CA, USA: Academic Press, 2015. 1220~p.
\bibitem{20-luk} %19
\Au{Giordano S., Gubinelli~M., Pagano~M.} Bridge Monte-Carlo: 
A~novel approach to rare events of Gaussian processes~// 5th St.\ 
Petersburg Workshop on Simulation Proceedings.~--- St.\ Petersburg: St. Petersburg 
State University, 2005. P.~281--286.
\bibitem{21-luk} %20
\Au{Giordano S., Gubinelli~M., Pagano~M.} Rare events of Gaussian processes: 
A~performance comparison between bridge Monte-Carlo and importance sampling~// 
Next generation teletraffic and wired/wireless advanced networking~/
 Eds.\ Y.~Koucheryavy, J.~Harju, A.~Sayenko.~--- 
 Lecture notes in computer science ser.~--- Springer, 2007.
Vol.~4712. P.~269--280.
\bibitem{22-luk} %21
\Au{Lukashenko O.\,V., Morozov~E.\,V., Pagano~M.}
Performance analysis of bridge Monte-Carlo estimator~// 
Труды Карельского научного центра Российской академии наук, 
2012. Т.~5. С.~54--60.
\bibitem{23-luk} %22
\Au{Lukashenko O.\,V., Morozov~E.\,V., Pagano~M.}
 On the efficiency of bridge Monte-Carlo estimator~// Информатика и её применения,
  2017.  Т.~11. Вып.~2. С.~16--24.

\bibitem{1-luk} %23
\Au{Morozov E., Lukashenko~O., Rumyantsev~A., Ivashko~E.}
A~Gaussian approximation of runtime estimation in a~desktop grid project~// 
9th  Congress (International) on Ultra Modern Telecommunications and Control Systems 
and Workshops.~--- IEEE, 2017. P.~107--111.

\end{thebibliography}
} }

\end{multicols}

 \label{end\stat}

 \vspace*{-9pt}

\hfill{\small\textit{Поступила в~редакцию 15.04.2019}}


%\renewcommand{\bibname}{\protect\rm Литература}
\renewcommand{\figurename}{\protect\bf Рис.}
\renewcommand{\tablename}{\protect\bf Таблица}  %15
\def\stat{kovalev}

\def\tit{МЕТОДЫ ТЕОРИИ КАТЕГОРИЙ В~МОДЕЛЬНО-ОРИЕНТИРОВАННОЙ СИСТЕМНОЙ 
ИНЖЕНЕРИИ}

\def\titkol{Методы теории категорий в~модельно-ориентированной системной 
инженерии}

\def\aut{С.\,П.~Ковалёв$^1$}

\def\autkol{С.\,П.~Ковалёв}

\titel{\tit}{\aut}{\autkol}{\titkol}

\index{Ковалёв С.\,П.}
\index{Kovalyov S.\,P.}


%{\renewcommand{\thefootnote}{\fnsymbol{footnote}} \footnotetext[1]
%{Исследование выполнено при финансовой поддержке Российского научного фонда (проект 16-11-10227).}}


\renewcommand{\thefootnote}{\arabic{footnote}}
\footnotetext[1]{Институт проблем управления им.\ В.\,А.~Трапезникова 
Российской академии наук,  \mbox{kovalyov@nm.ru}}

%\vspace*{-18pt}

\Abst{Предложен математический аппарат на базе теории категорий, который позволяет 
формально описывать и~строго исследовать процедуры применения моделей в~инженерной 
деятельности, составляющие сущность мо\-дель\-но-ори\-ен\-ти\-ро\-ван\-ной системной 
инженерии (Model-Based Systems Engineering, MBSE). В~основе аппарата лежит 
математическое представление сборочных чертежей (мегамоделей сис\-тем) диаграммами 
в~категориях, объектами которых служат модели, а~морфизмы представляют действия по 
сборке моделей сис\-тем из моделей компонентов. Адекватность аппарата обоснована исходя 
из требований стандартов, регламентирующих описание структуры систем, в~том числе 
IEC~81346. Предложены и~исследованы тео\-ре\-ти\-ко-ка\-те\-гор\-ные методы решения ряда 
практических задач сборки систем. Приведены примеры решения таких задач в~категориях, 
представляющих две ключевые области применения MBSE: гео\-мет\-ри\-че\-ское моделирование 
изделий сложной формы и~дис\-крет\-но-со\-бы\-тий\-ное имитационное моделирование 
поведения технических систем.}

\KW{модельно-ориентированная системная инженерия; мегамодель; теория категорий; 
копредел}



\DOI{10.14357/19922264170305} 


\vspace*{6pt}

\vskip 10pt plus 9pt minus 6pt

\thispagestyle{headings}

\begin{multicols}{2}

\label{st\stat}

\section{Введение}

   Модельно-ориентированная системная инженерия состоит в~формализованном применении моделирования в~
поддержке жизненного цикла сис\-тем, включая сбор требований, 
проектирование, проверку и~приемку, другие стадии~[1]. Модели, 
разрабатываемые в~ходе процедур MBSE, пригодны к~автоматической 
обработке на компьютерах. Это позволяет сначала задавать, верифицировать 
и~оптимизировать проектные решения на моделях <<в циф\-ре~и только потом 
воплощать <<в железе>>, снижая затраты на организацию жизненного цикла 
изделий и~сокращая сроки выполнения работ~[2].
   
   И все же внедрение технологий MBSE в~инженерную деятельность 
происходит медленно. Это связано во многом с~нехваткой единой 
концептуальной базы инженерного моделирования: предлагается много 
частных языков и~технологий, слабо совместимых друг с~другом и~плохо 
приспособленных для совместной разработки моделей большими 
мультидисциплинарными коллективами~[3]. Тем самым затрудняется переход 
от набора электронных чертежей к~полноценному электронно-цифровому 
макету (digital mock-up) промышленного изделия.
   
   Естественный, хотя и~<<трудный>>, подход к~получению результатов 
общего характера, унифи\-ци\-ру\-ющих разнородные технологии, состоит в~том, 
чтобы как можно более строго формализовать процедуры моделирования. 
Формализация позволит совершенствовать процедуры MBSE и~передавать их 
на исполнение компьютеру без пробелов и~искажений. Самый высокий уровень 
строгости достигается при привлечении математического аппарата, поскольку 
математика позволяет надежно доказывать или опровергать утверждения, 
ха\-рак\-те\-ри\-зу\-ющие корректность и~эффективность процедур.
   
   В настоящей работе предложен аппарат, основанный на математическом 
представлении сборочных чертежей (<<мегамоделей>> систем) 
ориенти-\linebreak рованными графами (диаграммами). Узлы такого\linebreak графа помечаются 
обозначениями моделей час\-тей, а~реб\-ра помечаются обозначениями действий\linebreak 
(activities), посредством которых части собираются в~систему. Представление 
структуры систем графами регламентируется, в~частности, стандартом 
IEC~81346~[4]. Естественным источником математических методов 
конструирования и~анализа мегамоделей служит теория категорий (см., 
например,~[5, 6]). Модели рассматриваются как объекты подходящих 
категорий, а~действия формально описываются морфизмами. Строятся 
и~исследу-\linebreak ются тео\-ре\-ти\-ко-ка\-те\-гор\-ные конструкции, опи\-сы\-ва\-ющие процедуры 
MBSE на абстрактном кон-\linebreak цептуальном уровне. Определенный опыт такого\linebreak 
исследования был накоплен в~инженерии программного обеспечения~[7] 
и~теперь может быть обобщен для системной инженерии в~целом. Например, 
сборке системы согласно некоторой мегамодели отвечает построение 
копредела диаграммы~--- универсальной конструкции~\cite{5-kov}.
   
   Статья построена следующим образом. В~разд.~2 приведен обзор 
принципов описания структуры сис\-тем согласно стандарту IEC~81346. 
Раздел~3 посвящен практическим проб\-ле\-мам мегамоделирования и~сборке 
сис\-тем. В~разд.~4 вводятся конструкции тео\-рии категорий, позволяющие 
формально решать задачи мегамоделирования. В~заключении приводятся 
выводы и~намечаются направления дальнейших исследований.

\section{Структура систем и~стандарт~IEC~81346}

   Важной проблемой MBSE, отмеченной во введении, является слабая 
совместимость языков и~инструмен\-тов моделирования от разных поставщиков. 
Основным подходом к~достижению совместимости является стандартизация~--- 
принятие обязывающих документов, устанавливающих требования и~принципы 
взаимозаменяемости инструментов. Многие стандарты определяют конкретные 
форматы машиночитаемой записи моделей, нейтральные относительно 
разработчиков инструментов MBSE. Примером служит формат описания 
твердотельных геометрических моделей STEP, стандартизованный семейством 
ISO~10303. Однако для формализации MBSE в~целом интерес представляют 
в~первую очередь стандарты более общего плана, унифицирующие принципы 
и~методы применения моделей в~жизненном цикле систем независимо от 
способа записи моделей. С~этой точки зрения внимания заслуживает 
международный стандарт IEC 81346-1:2009 <<Промышленные системы, 
установки и~обору\-до\-ва\-ние~--- принципы структурирования и~ссылочные 
обозначения~--- часть~1: основные правила>> (<<Industrial Systems, 
Installations and Equipment and Industrial Products~--- Structuring Principles and 
Reference Designations~--- Part~1: Basic Rules>>)~\cite{4-kov}. Стандарт не 
принят в~России, однако ряду его положений в~области структуры систем 
соответствует российский ГОСТ~2.053-2013 <<ЕСКД. Электронная структура 
изделия. Общие положения>>.
   
   В стандарте IEC~81346 рассматривается ряд вопросов моделирования 
структуры систем и~идентификации отдельных единиц в~составе систем. 
Системная единица названа в~стандарте объектом, причем принципиально не 
проводится различие между объектами реального мира, составляющими 
реально существующие системы, и~объектами мыслительной деятельности~--- 
моделями единиц, составляющими модели систем. Таким образом, стандарт 
выходит за рамки MBSE и~рассматривает ряд вопросов системной инженерии 
вообще. Иерар\-хи\-че\-ская структура системы (холархия~\cite{3-kov}) 
изображается деревом, узлы которого помечены обозначениями объектов. 
Важным достижением стандарта является выявление того факта, что одна и~та 
же система задается не одной, а несколькими в~общем случае различными 
иерархическими структурами, возникающими в~результате декомпозиции 
согласно различным принципам (аспектам). В~их числе:
   \begin{itemize}
\item функциональная (function-oriented) структура, отвечающая разделению 
системных единиц по выполняемым ими функциям в~составе сис\-темы;
\item продуктовая (product-oriented), или модульная, структура, отражающая 
сборочную (технологическую) конфигурацию сис\-темы;
\item структура размещения (location-oriented), в~соответствии с~которой 
единицы располагаются в~физическом пространстве.
\end{itemize}

   Ясно, что один и~тот же объект может входить в~несколько структур и~при 
этом находиться на различных уровнях. В~то же время в~некоторых аспектах 
объект может никак не проявлять себя и~вследствие этого отсутствовать 
в~соответствующих структурах. Полное идентифицирующее ссылочное 
обозначение объекта (reference designation) конструируется путем 
последовательного перечисления всех объектов, находящихся на пути от корня 
дерева рассматриваемой структуры до дан\-ного объекта включительно. 
Наименование каж\-до\-го объекта в~этом перечислении составляется из 
символьного обозначения аспекта, буквенного обозначения класса (типа), 
к~которому относится  объект, и~порядкового номера объекта среди 
экземпляров своего класса. Таким путем обеспечивается\linebreak  уникальность 
наименования любой единицы\linebreak
 в~пределах системы. Например, функциональная 
структура обозначается символом <<=>>, а~функциональный класс 
переключателей потоков ресурсов обозначается буквами QA, так что первая по 
порядку единица, выполняющая функцию переключения, называется =QA1, 
а~ее полное ссылочное обозначение может выглядеть как =WP1=WC1=QA1. 
Если объект присутствует в~нескольких структурах, то он может иметь 
несколько ссылочных обозначений, как показано на рис.~1~\cite{4-kov}.

\begin{figure*} %fig1
    \vspace*{1pt}
\begin{center}
\mbox{%
\epsfxsize=165mm
\epsfbox{kov-1.eps}
}
\end{center}
\vspace*{-9pt}
\Caption{Пример ссылочных обозначений структурных единиц системы}
\vspace*{9pt}
\end{figure*}

   С~точки зрения практики системной инженерии большой интерес 
представляет описание эволюции структурного представления системы по ходу 
жизненного цикла, приведенное в~приложении~B к~стандарту IEC~81346. 
<<Строительный материал>> для структур имеет вид (виртуального) 
справочника или каталога объектов, из которого выбираются объекты для 
включения в~структуру. 

В~начале жизненного цикла системы на основе 
исходных требований к~ней конструктор строит ее функциональную структуру. 
Затем определяется пространственное положение функциональных объектов, 
в~результате чего создается структура размещения. На следующей стадии 
формируются закупочные спецификации, образующие продуктовую структуру. 
В~ходе последующих стадий жизненного цикла эти структуры могут 
трансформироваться. На каждой стадии могут происходить замена, слияние 
и~расщепление объектов. Таким образом, объекты разных структур системы 
связаны отношением вида <<многие ко многим>>, вдоль которого 
прослеживаются (трассируются) исходные требования.
   
   В то же время стандарт не предусматривает указа\-ние способов, какими 
объекты собраны в~сис\-те\-мы. Поэтому структуру сис\-те\-мы можно рас\-смат\-ри\-вать 
как эскизный проект, в~котором отражены лишь факты вхождения системных 
единиц более низкого уровня иерархии в~единицы более высокого уровня. 


Проект такого рода поступает на вход технологу, который определяет 
конкретные операции сборки каждой единицы каждого уровня иерархии. При 
необходимости технолог вносит изменения в~конструкцию объектов (такие как 
нарезка резьбы) и~добавляет связующие интерфейсные объекты (такие как 
клей, трансформатор и~др.). В~результате для каждого составного объекта 
формируется сборочный чертеж, на котором указаны все со\-став\-ля\-ющие 
объекты и~действия по их соединению в~целях получения сис\-те\-мы. 
Технологическая проработка требуется на всех стадиях жизненного цикла, на 
которых формируется либо изменяется ка\-кая-ли\-бо из структур системы.

%\vspace*{-6pt}

\section{Мегамоделирование и~сборка~систем}

   В MBSE объекты, образующие 
структуры\linebreak
 сис\-тем, описываются формализованными ком\-пьютерными моделями 
различных видов: геометрическими фигурами и~телами, численными 
аппроксимациями дифференциальных уравнений, оснащенными графами и~
т.\,д. При этом, как свидетель\-ст\-ву\-ют стандарты типа IEC~81346, для анализа 
структуры систем и~организации сборки необходимо знать не столько 
внутреннюю структуру моделей, сколько ассортимент их возможностей 
соединяться с~другими моделями в~целях формирования моделей составных 
объектов. Иными словами, модели рассматриваются как <<черные ящики>> 
с~известным поведением по отношению к~другим моделям. Каталог объектов, 
упоминавшийся в~предыду\-щем разделе, в~условиях применения \mbox{MBSE} 
составляется из моделей и~описаний действий по их соединению.
   
   Структуры систем и~сборочные чертежи представляют собой частные 
случаи мегамоделей (mega\-mod\-el)~--- моделей, состоящих из моделей и~связей 
между ними~\cite{8-kov}. Мегамодель, в~которой связи описывают соединение 
моделей, образующих некоторую сис\-те\-му, называется конфигурацией этой 
сис\-те\-мы~\cite{5-kov}. Существуют и~другие виды мегамоделей, 
предназначенные для описания других процедур \mbox{MBSE}, таких как 
формирование модели согласно заданной метамодели  
(instantiating)~\cite{9-kov}. Но в~настоящей работе сосредоточимся на 
конфигурациях и~сборке систем.
   
   Например, в~моделировании механических сис\-тем, состоящих из твердых 
тел, моделями деталей и~сборочных единиц служат геометрические тела, 
которые могут быть представлены для компьютерной обработки различными 
способами: конструктивным, воксельным, граничным~\cite{10-kov}. Объекты, 
составляющие механические системы, т.\,е.\ представления экземпляров тел, 
получаются из моделей путем аффинных изометрий и~растяжений. Так, из 
набора цилиндров разных размеров составляется модель штанги (спортивного 
снаряда). В~функциональной структуре штанги по IEC~81346 цилиндры 
представлены разными объектами, поскольку они выполняют разные функции, 
хотя порождаются одной и~той же геометрической моделью. Соответственно, 
в~каталоге моделей содержится тело в~форме цилиндра, допускающее 
несколько разных действий по включению в~состав штанги.
   
   В качестве еще одного примера рассмотрим дис\-крет\-но-со\-бы\-тий\-ное 
имитационное моделирование, поддержка которого относится к~числу 
важнейших достижений MBSE~\cite{1-kov}. Здесь модель имеет вид 
сценария~--- фрагмента предполагаемой истории поведения моделируемой 
системы, пред\-став\-лен\-но\-го потоком дискретных событий различных видов. 
Некоторые события могут вызывать либо запрещать возникновение других 
событий. Описания действий по сборке сценариев поведения систем отражают 
вклад сценариев поведения составляющих. Так, сценарий работы цеха 
составляется из сценариев работы станков, связанных друг с~другом согласно 
маршрутным картам~\cite{11-kov}.
   
   Сформулируем задачу мегамоделирования сборки систем в~общем виде 
следующим образом. По мегамодели, представляющей конфигура\-цию 
некоторой системы, требуется сконструировать модель системы как целого 
и~рассчитать для нее моделируемые параметры, в~том числе эмерджентные~--- 
не присущие никакой из со\-став\-ля\-ющих единиц в~отдельности. Принцип 
конструирования модели системы легко усмотреть из организации 
структур-\linebreak\vspace*{-12pt}

\columnbreak

 { \begin{center}  %fig1
 \vspace*{1pt}
\mbox{%
\epsfxsize=57.246mm
\epsfbox{kov-2.eps}
}


\vspace*{12pt}


\noindent
{{\figurename~2}\ \ \small{Схема склеивания}}
\end{center}
}

\vspace*{18pt}

\addtocounter{figure}{1}

\noindent
ного представления: система должна находиться на иерархическом 
уровне, располагающемся непосредственно над уровнем со\-став\-ля\-ющих ее 
объектов. Иными словами, модель системы должна включать в~себя модели 
всех составляющих с~учетом их конфигурационных связей и~в~то же время 
включаться в~любые модели, включающие в~себя модели всех составляющих 
конфигурации.
   
   Поясним этот принцип на простом примере. Предположим, что нужно 
объединить в~систему два объекта~$P$ и~$S$ и~что технолог решил сделать это 
с~по\-мощью клея~--- третьего объекта~$G$, который может быть соединен 
и~с~$P$, и~с~$S$. Действие клея описывается конфигурацией следующего 
вида: объекты~$G$ и~$P$ порождают в~результате соединения известный 
промежуточный комплексный объект~$P_G$, содержащий их, а~объекты~$G$ 
и~$S$ порождают объект~$S_G$. Система~$R$, полученная путем 
склеивания~$P$ с~$S$ при помощи~$G$, отбирается среди объектов, 
содержащих~$P_G$ и~$S_G$, по следующему структурному критерию: 
объект~$R$ должен содержаться в~любом объекте~$T$, содержащем~$P_G$ 
и~$S_G$. Схематически этот критерий изображен на рис.~2.


   Если объект $R$, удовлетворяющий указанному структурному критерию, 
существует, то он действительно отвечает системе, которая собрана из~$S$ 
и~$P$ путем склеивания посредством~$G$ (и~не содержит ничего 
<<лишнего>>). Более того, легко видеть, что такой объект~$R$ определяется, 
по существу, однозначно в~том смысле, что любые два объекта~$R$ 
и~$R^\prime$, удовлетворяющие структурному критерию, содержатся друг 
в~друге. Если же нужного объекта~$R$ не существует, то делается вывод, что 
технолог ошибся: клей~$G$ не способен соединить объекты~$P$ и~$S$.
   
   В структурное представление, выполненное по стандарту IEC~81346 либо по 
ГОСТу 2.053-2013, входят только объекты~$P$, $S$ и~$R$ и~две композитные 
стрелки: $P\hm\to R$, проходящая через~$P_G$, и~$S\hm\to R$, проходящая 
через~$S_G$ (так что мегамодель склеивания~--- это часть схемы, ограниченная 
треугольником~$PSR$). Кроме того, стрелки на схеме склеивания, в~отличие от 
структуры, представляют не просто факты включения объектов друг в~друга, 
а~конкретные действия по их соединению. При этом соблюдается следующее 
естественное условие структурной корректности: если из одного объекта 
можно прийти в~другой разными путями по схеме, то эти пути задают одно и~то 
же композитное действие. Например, клей~$G$ включается в~состав 
системы~$R$ единственным способом, несмотря на наличие двух путей $G 
\hm\to  P_G \hm\to R$ и~$G \hm\to S_G \hm\to R$: в~действительности не имеет 
значения, через какой промежуточный объект <<прослеживается>> включение 
клея в~систему. Таким образом, мегамодель сборки содержит больше 
информации, чем иерархическая структура системы.
   
   Если модели содержат значения тех или иных параметров, а описание 
действий по их соединению позволяет выявить правила преобразования 
значений, то по мегамодели сборки можно вы\-чис\-лить значения параметров для 
системы. Известны примеры вычислений такого рода в~области разработки 
новых композиционных материалов~\cite{12-kov}. Осредненные 
(эффективные) физические характеристики композитов, такие как модуль Юнга и~коэффициент Пуассона, сложным образом зависят от характеристик 
компонентов и~способов изготовления композита из них. При помощи методов 
теории упру\-гости эти зависимости задаются в~форме линеаризованных 
матричных соотношений, которые приписываются к~стрелкам мегамоделей, 
пред\-став\-ля\-ющим включение компонентов в~композиты. Появляется 
возможность рассчитывать на компьютере свойства композитов по базе данных 
компонентов, без проведения дорогостоящих физических экспериментов.
   
   В заключение раздела отметим, что хотя прямой расчет системы по 
конфигурации имеет большое значение, в~MBSE он играет вспомогательную 
роль. Согласно стандарту IEC~81346 и~практикам системной инженерии, 
система обычно проектируется сверху вниз~--- от корня структурной иерархии 
к~составляющим~\cite{13-kov}. Это означает, что технолог в~основном решает 
не прямую, а~обратную задачу: модель системы, которую нужно собрать, 
известна, а~нужно построить (восстановить) конфигурацию, из которой такая 
система может быть получена путем сборки, с~учетом различных ограничений. 
Формальные математические постановки и~методы решения обратных задач 
мегамоделирования представляют собой крупную перспективную тему 
исследований, выходящую за рамки настоящей статьи.

\section{Теория категорий в~мегамоделировании}

   Как указывалось во введении, естественным источни\-ком математических 
методов кон\-стру\-ирова\-ния и~анализа мегамоделей служит теория категорий. 
Категорией называется коллекция абстрактных объектов, попарно связанных 
морфизмами (стрелками). Точное определение занимает буквально несколько 
строк~\cite{14-kov}: категория~$C$ состоит из совокупности 
объектов~$\mathrm{Ob}\,C$ и~совокупности морфизмов~$\mathrm{Mor}\,C$, 
на которых заданы следующие операции:
\begin{enumerate}[(1)]
\item каждому морфизму~$f$ 
сопоставляется два объекта: область $\mathrm{dom}\,f$ и~кообласть 
$\mathrm{codom}\,f$ (соотношения вида $\mathrm{dom}\,f \hm= A$ и~
$\mathrm{codom}\,f \hm= B$ наглядно записываются в~форме стрелки~$f$: 
$A\hm\to B$, а множество всех морфизмов, удовлетворяющих этим 
соотношениям, обозначается через $\mathrm{Mor}(A, B))$;
\item для 
любой пары морфизмов~$f, g$, удовлетворяющей условию 
$\mathrm{codom}\,f\hm = \mathrm{dom}\,g$, определена композиция~--- 
морфизм $g \circ f : \mathrm{dom}\,f \hm\to  \mathrm{codom}\,g$, причем она 
ассоциативна: для любой тройки морфизмов~$f, g, h$, удовлетворяющей 
условиям $\mathrm{codom}\,f \hm= \mathrm{dom}\,g$ и~$\mathrm{codom}\,g 
\hm= \mathrm{dom}\,h$, выполняется соотношение $h \circ (g \circ f) \hm= (h 
\circ g) \circ f$;
\item любой объект~$A$ обладает тождественным 
морфизмом~$1_A : A \to A$ таким, что для любого морфизма~$f : A\hm\to B$ 
выполняется соотношение $f \circ 1_A \hm= 1_B \circ  f \hm= f$.
\end{enumerate}

Классическим 
примером категории служит $\mathbf{Set}$, состоящая из всех множеств и~всех 
их отображений: закон композиции отображений задается стандартной 
подстановкой, а тождественным морфизмом произвольного множества служит 
его тождественное отображение на себя.
   
   Вместе с~категорией вводится понятие функтора~--- отображения категорий, 
сохраняющего структуру. Функтор $\mathrm{fun}\,: C \hm\to D$, действующий из 
категории~$C$ в~$D$,~--- это пара одноименных отображений $\mathrm{fun}\,: 
\mathrm{Ob}\,C \hm\to \mathrm{Ob}\,D$, $\mathrm{fun}\,: \mathrm{Mor}\,C \hm\to 
\mathrm{Mor}\,D$, удовлетворяющая следующим условиям (для произвольных 
$C$-мор\-физ\-мов~$f, g$ и~$C$-объ\-ек\-та~$A$): 
\begin{enumerate}[(1)]
\item $\mathrm{fun}\,(\mathrm{dom}\,f) 
\hm= \mathrm{dom}\,\mathrm{fun}\,(f), \mathrm{fun}\,(\mathrm{codom}\,f)\hm = 
\mathrm{codom}\,\mathrm{fun}\,(f)$;  
\item $\mathrm{fun}\,(g \circ f) = \mathrm{fun}\,(g) \circ \mathrm{fun}\,(f)$, 
если композиция $g \circ f$ определена; 
\item $\mathrm{fun}\,(1_A) \hm= 1_{\mathrm{fun}\,(A)}$.
\end{enumerate}
 Все категории и~все функторы образуют 
(формальную) категорию~$\mathbf{CAT}$. Чтобы исследовать взаимосвязь 
между функторами, вводится следующее понятие: естественным 
преобразованием~$\varepsilon$ функтора $\mathrm{fun}\, : C\hm\to D$ в~$\mathrm{fun}^\prime\, : C 
\hm\to D$ называется любое семейство $D$-мор\-физ\-мов~$\varepsilon_A : 
\mathrm{fun}\,(A) \hm\to \mathrm{fun}^\prime (A)$, $A \hm\in \mathrm{Ob}\,C$, 
такое что для любого 
\mbox{$C$-мор}\-физ\-ма $f : A\hm\to B$ выполняется соотношение $\varepsilon_B \circ 
\mathrm{fun}\,(f) \hm= \mathrm{fun}^\prime(f) \circ \varepsilon_A$:

%\begin{figure*} %рис
\vspace*{1pt}
\begin{center}
\mbox{%
\epsfxsize=54.473mm
\epsfbox{kov-3.eps}
}
\end{center}
%\vspace*{-9pt}
%\end{figure*}

   Эффективность применения теории категорий в~качестве математического 
аппарата \mbox{MBSE} обуслов\-ле\-на тем, что любой каталог моделей представляет 
собой не что иное, как категорию. Действительно, любая цепочка действий по 
соединению моделей порождает композитное действие (процесс) и, кроме того, 
любая модель допускает пустое действие над самой собою, не 
подразумевающее никаких изменений (процедура <<ничегонеделания>>). 
Например, в~твердотельном моделировании механических систем объектами 
категории\linebreak моделей выступают тела~--- подмножества в~$\mathbb{R}^3$, 
которые являются ограниченными, регулярными\linebreak
 (совпадают с~замыканием 
своей внутренности) и~полуаналитическими (допускают представление 
конечными булевыми комбинациями множеств вида $\{(x, y, z) \vert  F_i(x, y, 
z)\hm\leq 0\}$, где~$F_i : \mathbb{R}^3\hm\to \mathbb{R}$ является 
вещественной аналитической функцией для всех~$i$)~\cite{10-kov}. Чтобы 
было возможно задавать процедуры типа склеивания участков поверхности тел, в~категорию геометрических моделей добавляются ограниченные регулярные 
полуаналитические подмножества в~$\mathbb{R}^n$, $0 \hm\leq n \hm\leq 2$, 
при помощи стандартного вложения~$\mathbb{R}^n$ в~$\mathbb{R}^3$. Далее 
выполняется факторизация: отождествляются друг с~другом все множества, 
переходящие друг в~друга под действием аффинных изометрий. Морфизмы 
таких классов эквивалентности, описывающие действия по сборке составных 
механических сис\-тем, порождаются изометрическими вложениями множеств 
и~растяжениями. Получается подкатегория в~\textbf{Set}, которую будем обозначать 
через $\mathbf{MBS}$ (от Multibody Systems).
   
   Для многих известных технологий MBSE формальное описание каталогов 
поддерживаемых моделей приводит к~категориям множеств со структурой~--- 
алгебраических систем, топологических пространств, графов и~т.\,д. 
Морфизмами в~таких категориях служат отображения множеств, со\-вмес\-ти\-мые 
со структурой. На любой такой категории действует канонический функтор 
в~$\mathbf{Set}$, <<забывающий>> структуру. 

В~качестве примера приведем  
дис\-крет\-но-со\-бы\-тий\-ное моделирование, в~котором математической 
моделью сценария служит множество событий, час-\linebreak тич\-но упорядоченное  
при\-чин\-но-след\-ст\-вен\-ны\-ми зависимостями и~размеченное видами 
событий~\cite{15-kov}. Действия по сборке сложных сценариев задаются 
монотонными отображениями, сохраняющими разметку, поскольку ни 
события, ни зависимости, ни метки не могут быть <<потеряны>> при 
соединении сценариев поведения компонентов в~сценарии поведения систем. 
Получается категория~$\mathbf{Pomset}$, состоящая из всех помеченных 
частично упорядоченных множеств и~всех их монотонных отображений, 
сохраняющих разметку. Имеется функтор $\vert \mbox{--} \vert : 
\mathbf{Pomset}\hm\to \mathbf{Set} : S \mapsto \vert S\vert$, <<забывающий>> 
порядок и~разметку.
   
   Зафиксируем произвольную категорию~$C$, представляющую некоторый 
каталог моделей. Как и~для любой алгебраической системы, определена 
конструкция подкатегории в~$C$~--- это пара, состоящая из подкласса 
в~$\mathrm{Ob}\,C$ и~подкласса в~$\mathrm{Mor}\,C$, замкнутых 
относительно унаследованных из~$C$ операций. Подкатегория в~$C$ 
называется полной, если любой \mbox{$C$-мор}\-физм, область и~кообласть которого 
содержатся в~ней, сам содержится в~ней. Например, подкатегориями 
описываются различные аспекты структурного представления систем согласно 
стандарту IEC~81346. Действительно, композиция двух морфизмов, 
представляющих действия по формированию некоторого аспекта структуры, 
также должна входить в~этот аспект, поскольку стандарт предписывает строить 
цепочки для идентификации объектов в~структуре системы. Кроме того, если 
объект присутствует в~аспекте, то его тождественный морфизм формально 
должен быть включен в~этот аспект. В~то же время подкатегории, 
опи\-сы\-ва\-ющие все аспекты, не обязаны образовывать в~совокупности разбиение 
категории~$C$: как показывает рис.~1, возможны как действия, входящие 
в~несколько аспектов одновременно, так и~композитные действия с~переходом 
между структурами, не входящие ни в~один аспект. Требуется лишь, чтобы 
объединение классов объектов всех этих подкатегорий совпадало 
с~$\mathrm{Ob}\,C$, поскольку не имеет смысла вводить модели, не входящие 
ни в~одну структуру.
   
   Категории можно получать из графов: любой ориентированный мультиграф 
порождает категорию, объектами в~которой служат все узлы, а морфизмами~--- 
все пути. Областью и~кообластью морфизма являются соответственно начало 
и~конец пути, композиция морфизмов действует как конкатенация путей, 
а~тождественным морфизмом узла~$a$ является пустой путь из~$a$ в~$a$, не 
содержащий ни одного ребра. Отсюда получается фундаментальное понятие  
$C$-диа\-грам\-мы~--- это функтор вида~$\Delta : X \hm\to C$, где~$X$~--- 
категория, порожденная некоторым графом и~называемая схемой диаграммы. 
Все $C$-диа\-грам\-мы образуют категорию~$\mathbf{D}C$ (ковариантная 
категория <<сверхзапятой>>~\cite{14-kov}), в~которой морфизмом 
диаграммы~$\Delta : X \hm\to C$ в~$\Xi : Y \hm\to C$ служит любая пара 
вида $\langle\gamma, fd\rangle$, состоящая из функтора~$fd : X\hm\to Y$ 
и~естественного преобразования~$\gamma : \Delta\hm\to \Xi \circ fd$; закон 
композиции морфизмов диаграмм имеет вид:
$$
\langle \gamma, fd\rangle \circ 
\langle \varphi, gd\rangle \hm = \langle \gamma_{gd(-)} \circ \varphi, fd \circ 
gd\rangle\,.
$$ 
В~тео\-рии категорий накоплен богатый арсенал алгебраических 
методов конструирования и~анализа диаграмм.
   
   Любая мегамодель задается $C$-диа\-грам\-мой, так что категорное 
представление каталогов моделей позволяет формально решать задачи 
мегамоделирования. Морфизмы диаграмм описывают структурные 
преобразования мегамоделей, выполняемые при помощи инструментов MBSE. 
Покажем, как решаются средствами теории категорий прямые задачи 
мегамоделирования. Здесь применяется одна из основных  
тео\-ре\-ти\-ко-ка\-те\-гор\-ных конструкций~--- копредел  
диаграммы~\cite{5-kov}, который строится следующим образом. Обозначим 
через~$\mathbf{1}$ категорию,\linebreak состоящую из одного объекта~0 и~одного 
морфизма~$1_0$. Из любой категории~$X$ имеется в~точ\-ности один 
функтор~$!_X : X \hm\to \mathbf{1}$, сопоставляющий объект~0  
любому~$X$-объ\-ек\-ту (иными словами, $\mathbf{1}$ является терминальным 
$\mathbf{CAT}$-объ\-ек\-том). Имеется вложение (инъективный функтор) 
$\ulcorner \mbox{--}\urcorner : C \hookrightarrow \mathbf{D}C$, сопоставляющее 
произвольному $C$-объ\-ек\-ту $Q$~точку~--- диаграмму $\ulcorner Q\urcorner : 
\mathbf{1}\hm\to  C : 0 \mapsto Q$. Коконусом (cocone) называется 
$\mathbf{D}C$-мор\-физм, имеющий точку в~качестве кообласти. Можно 
изобразить коконус $\langle \sigma, !_X\rangle : \Delta\hm\to \ulcorner 
Q\urcorner$ над диаграммой $\Delta : X\hm\to C$ в~виде диаграммы, 
<<пририсовав>> к~$\Delta$ дополнительную вершину, помеченную 
объектом~$Q$, и~набор ребер~--- стрелок, по одной для каждого узла $I\hm\in 
\mathrm{Ob}\,X$, направленной из~$I$ в~вершину и~помеченной морфизмом 
$\sigma_I : \Delta (I) \hm\to Q$. Копределом (colimit) диаграммы~$\Delta$ 
называется коконус $\mathrm{colim}\,\Delta : \Delta\hm\to \ulcorner R\urcorner$, 
универсальный в~том смысле, что для любых \mbox{$C$-объ}\-ек\-та~$T$ 
и~коконуса~$\delta : \Delta\hm\to\ulcorner T\urcorner$ существует единственный 
$C$-мор\-физм~$w : R \hm\to T$ такой, что $\delta\hm= \ulcorner w\urcorner \circ  
\mathrm{colim}\,\Delta$. Легко видеть, что это условие универсальности 
представляет собой в~точности структурный критерий из разд.~3. Таким 
образом, конструирование копредела конфигурации~$\Delta$ описывает на 
строгом математическом языке сборку системы, которой отвечает 
вершина~$R$. В~категориях типа $\mathbf{MBS}$ и~$\mathbf{Pomset}$ 
построение копредела сводится к~факторизации раздельных объединений 
объектов, представляющих компоненты системы, по отношениям 
эквивалентности, индуцированным моделями клея и~других средств сборки.
   
   Копредел любой диаграммы, если он сущест\-вует, определяется однозначно 
   с~точностью до изомор\-физма. Более того, можно описать сборку сис\-тем из 
конфигураций в~виде функтора. Пусть $Cd$~--- некоторый класс  
$C$-диа\-грамм, имеющих копределы. Он порождает полную подкатегорию 
в~$\mathbf{D}C$, из которой в~$C$ действует функтор копредела $\mathrm{colim}$, 
сопоставляя каждой диаграмме из~$Cd$~вершину некоторого ее копредела, а 
каждому \mbox{$\mathbf{D}C$-мор}\-физ\-му~$\theta : \Delta\hm\to \Xi$, 
где~$\Delta, \Xi\hm\in Cd$~--- стрелку копредела $\mathrm{colim}\,(\theta)$ такую, что 
$\mathrm{colim}\,\Xi \circ \theta \hm= \ulcorner \mathrm{colim}\,(\theta)\urcorner \circ 
\mathrm{colim}\,\Delta$.

%\begin{figure*}
\vspace*{1pt}
\begin{center}
\mbox{%
\epsfxsize=56.127mm
\epsfbox{kov-4.eps}
}
\end{center}
%\vspace*{-9pt}
%\end{figure*}

   Например, в~категории \textbf{Set} любая диаграмма имеет 
копредел~\cite[упражнение~5.1.8]{14-kov}, поэтому имеется функтор $\mathrm{colim}\, : 
\mathbf{D}(\mathbf{Set})\hm\to \mathbf{Set}$. Примечательно, что этот функтор 
является рефлектором: он сопряжен слева с~вложением $\ulcorner \mbox{--}\urcorner : 
\mathbf{Set}\hookrightarrow \mathbf{D}(\mathbf{Set})$, причем 
единица рефлексии состоит из $\mathbf{D}(\mathbf{Set})$-мор\-физ\-мов 
$\mathrm{colim}\,\Delta : \Delta\hm\to \ulcorner\mathrm{colim}\,(\Delta)\urcorner$, 
$\Delta\hm\in \mathrm{Ob}\ \mathbf{D}(\mathbf{Set})$. Напомним, что единица 
рефлексии~--- это естественное преобразование тождественного функтора 
в~композицию рефлектора и~вложения (в~данном случае, естественное 
преобразование функтора $1_{\mathbf{D}(\mathbf{Set})}$ в~$\ulcorner \mathrm{colim}\,(  
\mbox{--})\urcorner)$, состоящее из универсальных  
стрелок~\cite[разд.~4.3]{14-kov}. И~для произвольного класса~$Cd$, 
содержащего достаточное количество одноточечных диаграмм, функтор 
$\mathrm{colim}$ сопряжен слева с~ограничением  
вложения~$\ulcorner \mbox{--}\urcorner$ на подходящую полную подкатегорию 
в~$C$. А~поскольку сопряженный функтор задается однозначно с~точностью 
до изоморфизма~\cite[разд.~4.1]{14-kov}, можно сделать вывод, что сборка 
систем в~некотором смысле <<зашифрована>> в~процедуре построения 
одноточечных диаграмм~--- моделей систем как целого без раскрытия 
струк\-туры. 

Так наглядно проявляется двойственность прямых и~обратных задач 
мегамоделирования.

\section{Заключение}

   Аппарат теории категорий обладает большим потенциалом в~области 
повышения полезной отдачи от MBSE, в~том числе путем математически 
строгого решения задач мегамоделирования. Так, базовая процедура системной 
инженерии~--- сборка\linebreak
 системы из заданной конфигурации взаимо\-свя\-занных 
компонентов~--- формально описывается тео\-ретико-ка\-те\-гор\-ной 
конструкцией копредела диа\-граммы. Более сложные конструкции отвечают\linebreak 
сложным процедурам сборки, таким как связывание (weaving) общесистемных 
функций, рассеянных по всем компонентам (crosscutting concerns), например 
мониторинговых или защитных~\cite{16-kov}. Математического представления 
требуют и~другие процедуры MBSE, в~частности коллективная модификация 
мегамоделей и~составляющих моделей, восстановление конфигурации заданной 
системы, оценка взаимозаменяемости компонентов. 

Актуальны вопросы 
внедрения аппарата теории категорий в~практику, в~том числе путем развития 
программных инструментов моделирования и~мегамоделирования. Здесь 
открывается широкий спектр направлений для дальнейших исследований.
   
{\small\frenchspacing
 {%\baselineskip=10.8pt
 \addcontentsline{toc}{section}{References}
 \begin{thebibliography}{99}
\bibitem{1-kov}
Modeling and simulation-based systems engineering handbook~/
Eds.\ D.~Gianni,  A.~D'Ambrogio, A.~Tolk.~--- London: CRC Press, 2014. 513~p.
\bibitem{2-kov}
\Au{Ковалёв С.\,П., Толок~А.\,В.} Применение модельно-ори\-ен\-ти\-ро\-ван\-но\-го подхода 
в~управ\-ле\-нии жизненным циклом технических изделий~// Информационные технологии 
в~проектировании и~производстве, 2015. №\,2. С.~3--9.
\bibitem{3-kov}
\Au{Левенчук А.\,И.} Системноинженерное мышление.~--- М.: TechInvestLab, 2015. 305~с.
\bibitem{4-kov}
IEC 81346-1:2009. Industrial Systems, Installations and Equipment and Industrial Products~--- 
Structuring Principles and Reference Designations~--- Part~1: Basic Rules.~--- Geneva: ISO, 2009. 
168~p.
\bibitem{5-kov}
\Au{Ginali S., Goguen~J.} A~categorical approach to general systems~// 
 Conference (International) on Applied General Systems 
Research Proceedings~/
Ed. G.\,J.~Klir.~--- NATO conference series.~--- New York, NY, USA: Plenum 
Press, 1978. Vol.~5. P.~257--270.
\bibitem{6-kov}
\Au{Mabrok M.\,A., Ryan M.\,J.} Category theory as a~formal mathematical foundation for  
model-based systems engineering~// Appl. Math. Inform. Sci., 2017. Vol.~11. No.\,1. P.~43--51.
\bibitem{7-kov}
\Au{Ковалёв С.\,П.} Тео\-ре\-ти\-ко-ка\-те\-гор\-ный подход к~проектированию программных 
сис\-тем~// Фундаментальная и~прикладная математика, 2014. Т.~19. Вып.~3. С.~111--170.
\bibitem{8-kov}
\Au{B$\acute{\mbox{e}}$zivin J., Jouault~F., Rosenthal~P., Valduriez~P.} Modeling in the large 
and modeling in the small~// Model Driven Architecture: European MDA Workshops on 
Foundations and Applications Proceedings~/
Eds.\ U.~A{\!\ptb{\ss}}mann, M.~Aksit,  A.~Rensink.~--- 
Lecture notes in computer science ser.~--- Springer, 2005. Vol.~3599. 
P.~33--46.
\bibitem{9-kov}
\Au{Diskin Z., Kokaly~S., Maibaum~T.} Mapping-aware mega\-mod\-eling: Design patterns and 
laws~// Software Language Engineering: 6th Conference (International) Proceedings~/
Eds.\ M.~Erwig, R.\,F.~Paige, E.~Van Wyk.~--- 
Lecture notes  in computer science ser.~--- Springer, 2013. Vol.~8225. P.~322--343.
\bibitem{10-kov}
\Au{Requicha A.\,G.} Representations for rigid solids: Theory, methods, and systems~// 
ACM  Comput. Surv., 1980. Vol.~12. Iss.~4. P.~437--464.
\bibitem{11-kov}
\Au{K$\acute{\mbox{a}}$d$\acute{\mbox{a}}$r B., Pfeiffer~A., Monostori~L.} Discrete event 
simulation for supporting production planning and scheduling decisions in digital
 factories~//  37th 
CIRP Seminar (International) on Manufacturing Systems Proceedings.~--- Budapest, 2004.  
P.~444--448.
\bibitem{12-kov}
\Au{Giesa T., Spivak D.\,I., Buehler~M.\,J.} Category theory based solution for the building block 
replacement problem in materials design~// Adv. Eng. Mater., 2012. Vol.~14. 
Iss.~9. P.~810--817.
\bibitem{13-kov}
\Au{Косяков А., Свит У., Сеймур~С., Бимер~С.} Системная инженерия. Принципы 
и~практика~/ Пер. с~англ.~--- М.: ДМК-Пресс, 2014. 636~с. (\Au{Kossiakoff~A., Sweet~W.\,N., 
Seymour~S., Biemer~S.\,M.} Systems engineering principles and practice.~--- 2nd ed.~--- New 
York, NY, USA: John Wiley, 2011. 560~p.)
\bibitem{14-kov}
\Au{Маклейн С.} Категории для работающего математика~/ Пер. с~англ.~--- М.: Физматлит, 
2004. 352~с. (\Au{Mac Lane~S.} Categories for the working mathematician.~--- New York, NY, 
USA: Springer, 1978. 317~p.)
\bibitem{15-kov}
\Au{Pratt V.\,R.} Modeling concurrency with partial orders~// Int. J.~Parallel 
Prog., 1986. Vol.~15. No.\,1. P.~33--71.
\bibitem{16-kov}
\Au{Ковалёв С.\,П.} Семантика ас\-пект\-но-ори\-ен\-ти\-ро\-ван\-но\-го моделирования 
данных и~процессов~// Информатика и~её применения, 2013. Т.~7. Вып.~3. С.~70--80.
 \end{thebibliography}

 }
 }

\end{multicols}

\vspace*{-3pt}

\hfill{\small\textit{Поступила в~редакцию 16.01.17}}

%\vspace*{8pt}

\newpage

\vspace*{-30pt}

%\hrule

%\vspace*{2pt}

%\hrule

%\vspace*{8pt}


\def\tit{METHODS OF CATEGORY THEORY IN~MODEL-BASED SYSTEMS ENGINEERING\\[-7pt]}

\def\titkol{Methods of category theory in~model-based systems engineering}

\def\aut{S.\,P.~Kovalyov\\[-12pt]}

\def\autkol{S.\,P.~Kovalyov}

\titel{\tit}{\aut}{\autkol}{\titkol}

\vspace*{-14pt}


\noindent
Institute of Control Sciences, Russian Academy of Sciences, 65~Profsoyuznaya Str., 
Moscow 117997, Russian Federation



\def\leftfootline{\small{\textbf{\thepage}
\hfill INFORMATIKA I EE PRIMENENIYA~--- INFORMATICS AND
APPLICATIONS\ \ \ 2017\ \ \ volume~11\ \ \ issue\ 3}
}%
 \def\rightfootline{\small{INFORMATIKA I EE PRIMENENIYA~---
INFORMATICS AND APPLICATIONS\ \ \ 2017\ \ \ volume~11\ \ \ issue\ 3
\hfill \textbf{\thepage}}}

\vspace*{1pt}

 

\Abste{A mathematical device based on the category theory is proposed to formally describe and 
rigorously explore procedures of employing models in engineering that constitute the contents of 
model-based systems engineering (MBSE). The essence of the device consists in mathematical 
representation of assembly drawings (megamodels of systems) as diagrams in categories whose 
objects are models, and morphisms represent actions associated with assembling system models 
from component models. The soundness of the device is justified on the basis of standards that 
govern description of the systems' structure such as IEC~81346. Category-theoretical methods for 
solving a number of practical problems of assembling systems are proposed and explored. 
Examples of solving such problems are provided in categories that represent two key application 
areas for MBSE: geometric modeling of complex shapes and discrete-event simulation of the 
behavior of industrial systems.}

\KWE{ model-based systems engineering; megamodel; category theory; colimit}

\DOI{10.14357/19922264170305} 

%\vspace*{-18pt}

%\Ack
%\noindent




\vspace*{-7pt}

  \begin{multicols}{2}

\renewcommand{\bibname}{\protect\rmfamily References}
%\renewcommand{\bibname}{\large\protect\rm References}

{\small\frenchspacing
 {%\baselineskip=10.8pt
 \addcontentsline{toc}{section}{References}
 \begin{thebibliography}{99}
\bibitem{1-kov-1}
Gianni, D., A.~D'Ambrogio, and A.~Tolk, eds. 2014. \textit{Modeling and simulation-based 
systems engineering handbook}. London: CRC Press. 513~p.
\bibitem{2-kov-1}
\Aue{Kovalyov, S.\,P., and A.\,V.~Tolok.} 2015. Primenenie model'no-orientirovannogo podkhoda 
v~upravlenii zhiznennym tsiklom tekhnicheskikh izdeliy [Applying model-based approach 
to product lifecycle management].\linebreak \textit{Informatsionnye tekhnologii v~proektirovanii 
i~proizvod\-st\-ve} [Information Technologies in Design and Industry] 2(158):3--9.
\bibitem{3-kov-1}
\Aue{Levenchuk A.\,I.} 2015. 
\textit{Sistemnoinzhenernoe myshlenie} [Systems engineering thinking]. 
Moscow: TechInvestLab. 305~p.
\bibitem{4-kov-1}
IEC 81346-1:2009. 2009. 
Industrial Systems, Installations and Equipment and Industrial 
Products~--- Structuring Principles and Reference Designations~--- 
Part~1: Basic Rules. Geneva:  ISO. 168~p.
\bibitem{5-kov-1}
\Aue{Ginali, S., and J.~Goguen.} 1978. 
A~categorical approach to general systems. \textit{Conference 
(International) on Applied General Systems Research Proceedings}. Ed.\
 G.\,J.~Klir. \mbox{NATO}  conference ser. Plenum Press. 5:257--270.
\bibitem{6-kov-1}
\Aue{Mabrok, M.\,A., and M.\,J.~Ryan}. 
2017. Category theory as a~formal mathematical foundation for 
model-based systems engineering. \textit{Appl. Math.  Inform. Sci.} 11(1):43--51.
\bibitem{7-kov-1}
\Aue{Kovalyov, S.\,P.} 2016. 
Category-theoretic approach to software systems design. \textit{J.~Math. Sci.} 
214(6):814--853.
\bibitem{8-kov-1}
\Aue{B$\acute{\mbox{e}}$zivin, J., F.~Jouault, P.~Rosenthal, and P.~Valduriez.}
 2005. Modeling in 
the large and modeling in the small. 
\textit{Model Driven Architecture: European MDA Workshops on 
Foundations and Applications Proceedings.} 
Eds.\ U.~\mbox{A{\!\ptb{\ss}}mann}, M.~Aksit, and A.~Rensink. 
Lecture notes in computer science ser. Springer. 3599:33--46.
\bibitem{9-kov-1}
\Aue{Diskin, Z., S.~Kokaly, and T.~Maibaum.} 2013. 
Mapping-aware megamodeling: Design patterns 
and laws. \textit{6th Conference (International) on Software Language Engineering 
Proceedings}. Eds.\ M.~Erwig, R.\,F.~Paige, and E.~Van Wyk. 
Lecture notes in computer science ser. Springer. 
8225:322--343.
\bibitem{10-kov-1}
\Aue{Requicha, A.\,G.} 1980. Representations for rigid solids: 
Theory, methods, and systems. \textit{ACM 
Comput. Surv.} 12(4):437--464.
\bibitem{11-kov-1}
\Aue{K$\acute{\mbox{a}}$d$\acute{\mbox{a}}$r,~B., A.~Pfeiffer, and L.~Monostori.}
2004. Discrete 
event simulation for supporting production planning and scheduling decisions in 
digital factories. \textit{37th CIRP Seminar (International) on Manufacturing 
Systems Proceedings}. Budapest.  444--448.
\bibitem{12-kov-1}
\Aue{Giesa, T., D.\,I.~Spivak, and M.\,J.~Buehler.} 2012. 
Category theory based solution for the building 
block replacement problem in materials design. 
\textit{Adv. Eng. Mater.} 14(9):810--817.
\bibitem{13-kov-1}
\Aue{Kossiakoff, A., W.\,N.~Sweet, S.~Seymour, and S.\,M.~Bie\-mer.}
2011. \textit{Systems engineering 
principles and practice}. 2nd ed. New York, NY: John Wiley. 560~p.
\bibitem{14-kov-1}
\Aue{Mac Lane, S.} 1978. \textit{Categories for the working mathematician}. 
New York, NY: Springer. 317~p.
\bibitem{15-kov-1}
\Aue{Pratt, V.\,R.} 1986. Modeling concurrency with partial orders. 
\textit{Int. J.~Parallel Prog.} 15(1):33--71.
\bibitem{16-kov-1}
\Aue{Kovalyov, S.\,P.} 2013. 
Semantika aspektno-ori\-en\-ti\-ro\-van\-no\-go modelirovaniya dannykh 
i~protsessov [Semantics of aspect-oriented modeling of data and processes]. 
\textit{Informatika i~ee  Primeneniya~--- Inform. Appl.} 7(3):70--80.
\end{thebibliography}

 }
 }

\end{multicols}

\vspace*{-9pt}

\hfill{\small\textit{Received January 16, 2017}}

\vspace*{-18pt}

\Contrl

\noindent
\textbf{Kovalyov Sergey P.} (b.\ 1972)~--- Doctor of Science in physics and 
mathematics, leading scientist, Institute of Control Problems, Russian 
Academy of Sciences, 65~Profsoyuznaya Str., Moscow 117997, Russian 
Federation Federation; \mbox{kovalyov@nm.ru} 

\label{end\stat}


\renewcommand{\bibname}{\protect\rm Литература}   %16








%%%%%%%%%%%%%%%%%%%%%%%%%%%%%%%%%%%%%%%%%%%%%%%

%\def\stat{rez}
{%\hrule\par
%\vskip 7pt % 7pt
\raggedleft\Large \bf%\baselineskip=3.2ex
Р\,Е\,Ц\,Е\,Н\,З\,И\,И \vskip 17pt
    \hrule
    \par
\vskip 6pt plus 6pt minus 3pt }

%\thispagestyle{headings} %с верхним колонтитулом
%\thispagestyle{myheadings} %с нижним колонтитулом, но в верхнем РЕЦЕНЗИИ

\def\tit{НОВАЯ КНИГА И.\,Н.~СИНИЦЫНА, А.\,С.~ШАЛАМОВА <<ЛЕКЦИИ ПО ТЕОРИИ 
ИНТЕГРИРОВАННОЙ ЛОГИСТИЧЕСКОЙ ПОДДЕРЖКИ>> (М.: ТОРУС ПРЕСС, 2012. 624~с.)}

%1
\def\aut{Д.ф.-м.н., профессор С.\,Я.~Шоргин}

\def\auf{\ }

\def\leftkol{\ % РЕЦЕНЗИИ
}

\def\rightkol{ \ } 

%\def\leftkol{\ } % ENGLISH ABSTRACTS}

%\def\rightkol{\ } %ENGLISH ABSTRACTS}

%\def\leftkol{РЕЦЕНЗИИ}

%\def\rightkol{РЕЦЕНЗИИ}

\titele{\tit}{\aut}{\auf}{\leftkol}{\rightkol}
\vspace*{-18pt}


     \label{st\stat}

     \begin{multicols}{2}
     {\small
     {\baselineskip=10.1pt
     

      В книге представлено системное изложение теоретических основ одного из новейших 
направлений в \mbox{об\-ласти} экономики послепродажного обслуживания изделий наукоемкой 
продукции (ИНП) длительного пользования~--- интегрированной логистической поддержки
(ИЛП). 
{\looseness=1

}

Приведены также результаты новых работ, выполненных в Институте проблем информатики 
Российской академии наук в рамках научного направления <<Информационные технологии и 
анализ сложных сис\-тем>>.
 {%\looseness=1

}
     
      Излагаемые в книге научные подходы позво\-ляют карди\-наль\-но реформировать 
существующие системы производства и эксплуатации ИНП путем создания и внед\-ре\-ния 
методов рационального и оптимального управ\-ле\-ния процессами расходования 
вре\-мен\-н$\acute{\mbox{ы}}$х, 
мате\-ри\-аль\-ных, трудовых и других ресурсов на всех стадиях жизненного цикла изделий (ЖЦИ) по 
критериям экономической целесообразности и эф\-фек\-тив\-ности.
  {\looseness=1

}
    
      В книге приведен краткий обзор причин возник\-новения и
      развития CALS-методологии как основы 
современных международных стандартов по созданию и функционированию глобальных 
ин\-фор\-ма\-ци\-он\-но-ком\-му\-ни\-ка\-ци\-он\-ных систем, ее ключевых возможностей и эффективности 
результатов ее использования. 
Авторы %\linebreak 
предлагают ряд научных обоснований для разработки 
единой теории проектирования и управления систем ИЛП для полноценного использования 
преимуществ %\linebreak
 суще\-ст\-ву\-ющей методологии, определяют \mbox{общую} структурную схему 
комплексной системы <<ИНП-СППО>> и необходимость разработки для ее описания 
гибридных стохастических моделей.
{%\looseness=1

}

%\columnbreak
      
      Книга состоит из пяти частей, где последовательно излагается материал по каждой из 
следующих тем: <<Интегрированная логистическая поддержка>>, <<Теория гибридных 
стохастических систем и компьютерная поддержка исследований и разработок>>, <<Основы 
математического моделирования, анализа и синтеза систем послепродажного обслуживания>>, 
<<Определение и анализ показателей экспортного потенциала ИНП при проектировании>>, 
<<Задачи управления поддержкой послепродажного обслуживания>>, а также 
<<Моделирование инвестиционных процессов ИЛП в условиях неравновесных финансовых 
рынков>>. 
   
      В конце каждой главы приведены выводы и даны вопросы и задания для 
самоконтроля. В~приложениях содержатся основные определения по программам работ по 
анализу ИЛП, логистическим базам данных и компьютерным решениям, эквивалентной статистической 
линеаризации нелинейных преобразований ИЛП, справочный материал, а также развернутые 
уравнения для вероятностных характеристик.


      \def\leftkol{РЕЦЕНЗИИ}

\def\rightkol{РЕЦЕНЗИИ} 

      
      Книга заинтересует широкий круг специалистов и может быть использована научными 
проектными организациями в сфере промышленного производства ИНП. Большое количество 
иллюстраций, примеров и вопросов, обращенных к читателю, позволяет использовать книгу 
также в качестве учебного пособия для студентов и аспирантов машиностроительных, 
транспортных и~других специальностей, а также для самостоятельного изучения. 
{%\looseness=-1

}

Книга 
представляет несомненный интерес для специалистов и студентов в области прикладной 
математики и информатики.
    

}

}
\end{multicols}

%\newpage

\def\stat{authorsrus}
{%\hrule\par
%\vskip 7pt % 7pt
\raggedleft\Large \bf%\baselineskip=3.2ex
О\,Б\ \ А\,В\,Т\,О\,Р\,А\,Х \vskip 17pt
    \hrule
    \par
\vskip 21pt plus 8pt minus 4pt }


\def\tit{\ }

\def\aut{\ }

\def\auf{\ }

\def\leftkol{\ } % ENGLISH ABSTRACTS}

\def\rightkol{ОБ АВТОРАХ} %ENGLISH ABSTRACTS}

\titele{\tit}{\aut}{\auf}{\leftkol}{\rightkol}
      
            \label{st\stat}



\vspace*{24pt}

\begin{multicols}{2}




\noindent
\textbf{Архипов Олег Петрович} (р.\ 1948)~---
кандидат технических наук, директор Орловского филиала Института проб\-лем информатики
Российской академии наук
%302025, г.Орел, Московское шоссе, д.137

\vspace*{3pt}

\noindent
\textbf{Бирюкова Татьяна Константиновна} (р.\ 1968)~---
кандидат фи\-зи\-ко-ма\-те\-ма\-ти\-че\-ских наук, старший научный сотрудник Института проб\-лем информатики
Российской академии наук

\vspace*{3pt}

\noindent 
\textbf{Бобков  Сергей Геннадьевич} (р.\ 1955)~---
доктор технических наук,  заведующий отделением На\-уч\-но-ис\-сле\-до\-ва\-тель\-ско\-го 
института системных исследований Российской академии наук
%117218, Москва, Нахимовский просп., 36, к.1 

\vspace*{3pt}

\noindent \textbf{Васильев Николай Семенович} (р.\ 1952)~--- доктор 
фи\-зи\-ко-ма\-те\-ма\-ти\-че\-ских наук, профессор, 
МГТУ им.\ Н.\,Э.~Баумана 
%, Москва 105005, 2-я Бауманская ул., д.~5,

\vspace*{3pt}

\noindent
\textbf{Гершкович Максим Михайлович} (р.\ 1968)~---
старший научный сотрудник Института проб\-лем информатики
Российской академии наук

\vspace*{3pt}

\noindent 
\textbf{Дьяченко Юрий Георгиевич} (р.\ 1958)~--- кандидат технических наук, 
старший научный сотрудник Института проб\-лем информатики
Российской академии наук

\vspace*{3pt}

\noindent 
\textbf{Ерошенко Александр Андреевич} (р.\ 1989)~--- аспирант кафедры 
математической статистики факультета вычисли\-тельной математики и кибернетики 
Московского государственного университета им.\ М.\,В.~Ломоносова
%119991, Москва ГСП-1, Ленинские горы, д.\ 1, стр. 52

\vspace*{3pt}
 
\noindent 
\textbf{Захаров Виктор Николаевич} (р.\ 1948)~--- 
доктор технических наук, доцент, ученый секретарь Института проб\-лем информатики
Российской академии наук

\vspace*{3pt}

\noindent
\textbf{Зейфман Александр Израилевич} (р.\ 1954)~---
доктор фи\-зи\-ко-ма\-те\-ма\-ти\-че\-ских наук, профессор, 
заведующий кафедрой Вологодского государственного университета; 
старший научный сотрудник Института проб\-лем информатики
Российской академии наук; главный научный сотрудник ИСЭРТ Российской академии наук

\vspace*{3pt}

\noindent
\textbf{Зыкин Сергей Владимирович} (р.\ 1959)~--- 
доктор технических наук, профессор, заведующий лабораторией Института математики 
им.\ С.\,Л.~Соболева Сибирского отделения Российской академии наук, Новосибирск 
%630090, пр.\ ак.\ Коптюга, 4 

\vspace*{4pt}

\noindent
\textbf{Киреев Владимир Иванович} (р.\ 1938)~---
доктор фи\-зи\-ко-ма\-те\-ма\-ти\-че\-ских наук, профессор Московского 
государственного горного университета
%Адрес: Россия, 119991, г. Москва, Ленинский проспект, д. 6

%\columnbreak

\vspace*{4pt}

\noindent
\textbf{Козеренко Елена Борисовна} (р.\ 1959)~---
кандидат филологических наук, заведующая лабораторией Института проб\-лем информатики
Российской академии наук

\vspace*{4pt}

\noindent
\textbf{Королев Виктор Юрьевич} (р.\ 1954)~--- доктор
фи\-зи\-ко-ма\-те\-ма\-ти\-че\-ских наук, профессор кафедры математической 
статистики факультета вычисли\-тельной математики и кибернетики 
Московского государственного университета; 
ведущий научный сотрудник Института проб\-лем информатики
Российской академии наук

\vspace*{4pt}

\noindent
\textbf{Коротышева Анна Владимировна} (р.\ 1988)~---
старший преподаватель Вологодского государственного университета

\vspace*{4pt}

\noindent 
\textbf{Кун Де Турк} (р.\ 1981)~--- научный сотрудник 
исследовательской группы SMACS факультета телекоммуникаций и обработки информации
Университета Гента, Бельгия
%В-9000 Гент, Бельгия

\vspace*{4pt}

\noindent
\textbf{Лупенцов Олег Сергеевич} (р.\ 1986)~---
аспирант Омского государственного института сервиса
%Омск 644043, ул.\ Певцова 13

\vspace*{4pt}

\noindent
\textbf{Лучко Олег Николаевич} (р.\ 1961)~---
кандидат педагогических наук, профессор, заведующий кафедрой 
Омского государственного института сервиса
%Омск 644043, ул.\ Певцова 13

\vspace*{4pt}

\noindent
\textbf{Малашенко Юрий Евгеньевич} (р.\ 1946)~---
доктор фи\-зи\-ко-ма\-те\-ма\-ти\-че\-ских наук, заведующий сектором 
Вычислительного центра им.\ А.\,А.~Дородницына Российской академии наук
%Адрес: 119333, Москва, ул. Вавилова, 40,

\vspace*{4pt}

\noindent
\textbf{Маньяков Юрий Анатольевич} (р.\ 1984)~---
кандидат технических наук, научный сотрудник Орловского филиала Института проб\-лем информатики
Российской академии наук
%302025, г.Орел, Московское шоссе, д.137

\vspace*{4pt}

\noindent
\textbf{Маренко Валентина Афанасьевна} (р.\ 1951)~---
кандидат технических наук, доцент, старший научный сотрудник 
Института математики им.\ С.\,Л.~Соболева Сибирского отделения Российской академии наук
%Новосибирск 630090, пр. ак. Коптюга, 4 

\vspace*{3pt}

\noindent 
\textbf{Морозов Евсей Викторович} (р.\ 1947)~--- доктор 
фи\-зи\-ко-ма\-те\-ма\-ти\-че\-ских, профессор, ведущий научный сотрудник 
Института прикладных математических исследований Карельского научного центра Российской
академии наук; 
%%185910 Россия, Республика Карелия, г.\ Петрозаводск, ул.\ Пушкинская, 11
профессор Петрозаводского государственного университета, Петрозаводск
%185910 Россия, Республика Карелия, г.\ Петрозаводск, пр.\ Ленина, 33

%\pagebreak

\vspace*{3pt}

\noindent
\textbf{Назарова Ирина Александровна} (р.\ 1966)~---
кандидат фи\-зи\-ко-ма\-те\-ма\-ти\-че\-ских наук, 
научный сотрудник Вычислительного центра им.\ А.\,А.~Дородницына Российской академии наук 
%Адрес: 119333, Москва, ул. Вавилова, 40

\vspace*{3pt}

\noindent
\textbf{Павлов Игорь Валерианович} (р.\ 1945)~--- 
доктор фи\-зи\-ко-ма\-те\-ма\-ти\-че\-ских наук, профессор МГТУ им.\ Н.\,Э.~Баумана 
%Москва 105005, 2-я Бауманская ул., д.~5 

%\pagebreak

\vspace*{3pt}

\noindent 
\textbf{Потахина Любовь Викторовна} (р.\ 1989)~--- аспирантка
Института прикладных математических исследований Карельского научного центра
Российской академии наук; 
%%185910 Россия, Республика Карелия, г.\ Петрозаводск, ул.\ Пушкинская, 11
инженер Петрозаводского государственного университета, Петрозаводск
%185910 Россия, Республика Карелия, г.\ Петрозаводск, пр.\ Ленина, 33

\vspace*{3pt}

\noindent 
\textbf{Рождественский Юрий Владимирович} (р.\ 1952)~--- 
кандидат технических наук, заведующий сектором Института проб\-лем информатики
Российской академии наук

\vspace*{3pt}

\noindent 
\textbf{Синицын Игорь Николаевич} (р.\ 1940)~--- доктор технических наук,
профессор, заслуженный деятель\linebreak\vspace*{-12pt}

\columnbreak

\noindent
 науки РФ, заведующий отделом Института проб\-лем информатики
Российской академии наук

\vspace*{7pt}


\noindent
\textbf{Сиротинин Денис Олегович} (р.\ 1984)~---
кандидат технических наук, научный сотрудник Орловского филиала Института проб\-лем информатики
Российской академии наук
%302025, г.Орел, Московское шоссе, д.137

\vspace*{7pt}

%\columnbreak

\noindent 
\textbf{Соколов  Игорь Анатольевич} (р.\ 1954)~--- академик (действительный член) Российской 
академии наук, доктор технических наук, директор Института проб\-лем информатики
Российской академии наук

\vspace*{7pt}

\noindent
\textbf{Степченков Юрий Афанасьевич} (р.\ 1951)~---
кандидат технических наук, заведующий отделом Института проб\-лем информатики
Российской академии наук

\vspace*{7pt}

\noindent
\textbf{Сурков Алексей Викторович} (р.\ 1978)~--- 
старший научный сотрудник На\-уч\-но-ис\-сле\-до\-ва\-тель\-ско\-го 
института системных исследований Российской академии наук
%117218, Москва, Нахимовский просп., 36, к.1 

\vspace*{7pt}

\noindent 
\textbf{Шестаков Олег Владимирович} (р.\ 1976)~--- доктор 
фи\-зи\-ко-ма\-те\-ма\-ти\-че\-ских, доцент кафедры математической статистики 
факультета вычисли\-тельной математики и кибернетики Московского 
государственного университета им.\ М.\,В.~Ломоносова; 
%119991, Москва ГСП-1, Ленинские горы, д.\ 1, стр. 52
старший научный сотрудник Института проб\-лем информатики
Российской академии наук
%, Москва 119333, ул. Вавилова, д.~44, корп.~2

\vspace*{7pt}

\noindent 
\textbf{Шоргин Сергей Яковлевич} (р.\ 1952.)~--- доктор
фи\-зи\-ко-ма\-те\-ма\-ти\-че\-ских наук, профессор, заместитель директора Института 
проб\-лем информатики Российской академии наук





%%%%%%%%%%%%%%%%%%%%%%%%%%%%%%%%%%%%%%%%%%%%%%%%%%%%%%%%%%%%%%%%%%%%%%%%%%%%%%%




%\def\rightkol{ОБ АВТОРАХ}
%\def\leftkol{ОБ АВТОРАХ}

 \label{end\stat}





%\def\leftfootline{\small{\textbf{\thepage}
%\hfill ИНФОРМАТИКА И ЕЁ ПРИМЕНЕНИЯ\ \ \ том~7\ \ \ выпуск~1\ \ \ 2013}
%}%
% \def\rightfootline{\small{ИНФОРМАТИКА И ЕЁ ПРИМЕНЕНИЯ\ \ \ том~7\ \ \ выпуск~1\ \ \ 2013
%\hfill \textbf{\thepage}}}


%\thispagestyle{myheadings}



\end{multicols}

\newpage  

%\def\stat{cont}
{%\hrule\par
%\vskip 7pt % 7pt
\raggedleft\Large \bf%\baselineskip=3.2ex
А\,В\,Т\,О\,Р\,С\,К\,И\,Й\ \ У\,К\,А\,З\,А\,Т\,Е\,Л\,Ь\ \ З\,А\ \ 2\,0\,0\,7 г. \vskip 17pt
    \hrule
    \par
\vskip 21pt plus 6pt minus 3pt }

\label{st\stat}

\def\tit{\ }

\def\aut{\ }
\def\auf{\ }

\def\leftkol{\ } % ENGLISH ABSTRACTS}

\def\rightkol{\ } %ENGLISH ABSTRACTS}

\titele{\tit}{\aut}{\auf}{\leftkol}{\rightkol}


\contentsline {chapter}{\ }{Выпуск \quad Стр.} 
\contentsline {section}{\textbf{Батракова Д.\,А., Королев В.\,Ю., Шоргин С.\,Я.}\ \ Новый метод вероятностно-ста\-ти\-сти\-че\-ско\-го анализа информационных потоков в\nobreakspace {}телекоммуникационных сетях}{\qquad 1 \qquad 40} 
\contentsline {section}{\textbf{Борисов А.\,В.}\ \ Байесовское оценивание в системах наблюдения с\nobreakspace {}марковскими скачкообразными процессами: игровой подход}{\qquad 2 \qquad 65}
\contentsline {section}{\textbf{Босов А.\,В., Иванов А.\,В.}\ \ Программная инфраструктура информационного Web-пор\-тала}{\qquad 2 \qquad 50}
\contentsline {section}{\textbf{Захаров В.\,Н., Калиниченко Л.\,А., Соколов И.\,А., Ступников С.\,А.}\ \ Конструирование канонических информационных моделей для интегрированных информационных систем}{\qquad 2 \qquad 15}
\contentsline {section}{\textbf{Захаров В.\,Н., Козмидиади В.\,А.}\ \ Средства обеспечения отказоустойчивости при\-ло\-жений}{\qquad 1 \qquad 14} 
\contentsline {section}{\textbf{Иванов А.\,В.}\ \ см. Босов А.\,В.\hfill\hfill\hfill\hfill\hfill\hfill\hfill\hfill\hfill\hfill\hfill\hfill\hfill\hfill\hfill\hfill\hfill\hfill\hfill\hfill\hfill\hfill\hfill\hfill\hfill\hfill\hfill\hfill\hfill\hfill\hfill\hfill\hfill\hfill\hfill}{\ }
\contentsline {section}{\textbf{Ильин В.\,Д., Соколов И.\,А.}\ \ Символьная модель системы знаний информатики в\nobreakspace {}че\-ло\-ве\-ко-автоматной среде}{\qquad 1 \qquad 66} 
\contentsline {section}{\textbf{Калиниченко Л.\,А.}\ \ см. Захаров В.\,Н.\hfill\hfill\hfill\hfill\hfill\hfill\hfill\hfill\hfill\hfill\hfill\hfill\hfill\hfill\hfill\hfill\hfill\hfill\hfill\hfill\hfill\hfill\hfill\hfill\hfill\hfill\hfill\hfill\hfill\hfill\hfill\hfill\hfill\hfill\hfill}{\ }
\contentsline {section}{\textbf{Козеренко Е.\,Б.}\ \ Лингвистическое моделирование для систем машинного перевода и обработки знаний}{\qquad 1 \qquad 54} 
\contentsline {section}{\textbf{Козмидиади В.\,А.}\ \ см. Захаров В.\,Н.\hfill\hfill\hfill\hfill\hfill\hfill\hfill\hfill\hfill\hfill\hfill\hfill\hfill\hfill\hfill\hfill\hfill\hfill\hfill\hfill\hfill\hfill\hfill\hfill\hfill\hfill\hfill\hfill\hfill\hfill\hfill\hfill\hfill\hfill\hfill }{\ } 
\contentsline {section}{\textbf{Королев В.\,Ю.}\ \ см. Батракова Д.\,А.\hfill\hfill\hfill\hfill\hfill\hfill\hfill\hfill\hfill\hfill\hfill\hfill\hfill\hfill\hfill\hfill\hfill\hfill\hfill\hfill\hfill\hfill\hfill\hfill\hfill\hfill\hfill\hfill\hfill\hfill\hfill\hfill\hfill\hfill\hfill}{\ } 
\contentsline {section}{\textbf{Кудрявцев А.\,А., Шоргин С.\,Я.}\ \ Байесовский подход к\nobreakspace {}анализу систем массового обслуживания и\nobreakspace {}показателей надежности}{\qquad 2 \qquad 76}
\contentsline {section}{\textbf{Печинкин А.\,В., Соколов И.\,А., Чаплыгин В.\,В.}\ \ Многолинейная система массового обслуживания с конечным накопителем и ненадежными приборами}{\qquad 1 \qquad 27} 
\contentsline {section}{\textbf{Печинкин А.\,В., Соколов И.\,А., Чаплыгин В.\,В.}\ \ Стационарные характеристики многолинейной\nobreakspace {}системы массового обслуживания с\nobreakspace {}одновременными отказами приборов}{\qquad 2 \qquad 39}
\contentsline {section}{\textbf{Синицын И.\,Н.}\ \ Корреляционные методы построения аналитических информационных моделей флуктуаций полюса Земли по априорным данным}{\qquad 2 \qquad \hphantom{9}2}
\contentsline {section}{\textbf{Синицын И.\,Н.}\ \ Развитие теории фильтров Пугачева для оперативной обработки информации в стохастических системах}{{\qquad 1 \qquad \hphantom{9}3}} 
\contentsline {section}{\textbf{Соколов И.\,А.}\ \ см. Захаров В.\,Н.\hfill\hfill\hfill\hfill\hfill\hfill\hfill\hfill\hfill\hfill\hfill\hfill\hfill\hfill\hfill\hfill\hfill\hfill\hfill\hfill\hfill\hfill\hfill\hfill\hfill\hfill\hfill\hfill\hfill\hfill\hfill\hfill\hfill\hfill\hfill}{\ }
\contentsline {section}{\textbf{Соколов И.\,А.}\ \ см. Ильин В.\,Д.\hfill\hfill\hfill\hfill\hfill\hfill\hfill\hfill\hfill\hfill\hfill\hfill\hfill\hfill\hfill\hfill\hfill\hfill\hfill\hfill\hfill\hfill\hfill\hfill\hfill\hfill\hfill\hfill\hfill\hfill\hfill\hfill\hfill\hfill\hfill}{\ } 
\contentsline {section}{\textbf{Соколов И.\,А.}\ \ см. Печинкин А.\,В.\hfill\hfill\hfill\hfill\hfill\hfill\hfill\hfill\hfill\hfill\hfill\hfill\hfill\hfill\hfill\hfill\hfill\hfill\hfill\hfill\hfill\hfill\hfill\hfill\hfill\hfill\hfill\hfill\hfill\hfill\hfill\hfill\hfill\hfill\hfill}{\ } 
\contentsline {section}{\textbf{Соколов И.\,А.}\ \ см. Печинкин А.\,В.\hfill\hfill\hfill\hfill\hfill\hfill\hfill\hfill\hfill\hfill\hfill\hfill\hfill\hfill\hfill\hfill\hfill\hfill\hfill\hfill\hfill\hfill\hfill\hfill\hfill\hfill\hfill\hfill\hfill\hfill\hfill\hfill\hfill\hfill\hfill}{\ }
\contentsline {section}{\textbf{Ступников С.\,А.}\ \ см. Захаров В.\,Н.\hfill\hfill\hfill\hfill\hfill\hfill\hfill\hfill\hfill\hfill\hfill\hfill\hfill\hfill\hfill\hfill\hfill\hfill\hfill\hfill\hfill\hfill\hfill\hfill\hfill\hfill\hfill\hfill\hfill\hfill\hfill\hfill\hfill\hfill\hfill}{\ }
\contentsline {section}{\textbf{Чаплыгин В.\,В.}\ \ см. Печинкин А.\,В.\hfill\hfill\hfill\hfill\hfill\hfill\hfill\hfill\hfill\hfill\hfill\hfill\hfill\hfill\hfill\hfill\hfill\hfill\hfill\hfill\hfill\hfill\hfill\hfill\hfill\hfill\hfill\hfill\hfill\hfill\hfill\hfill\hfill\hfill\hfill}{\ } 
\contentsline {section}{\textbf{Чаплыгин В.\,В.}\ \ см. Печинкин А.\,В.\hfill\hfill\hfill\hfill\hfill\hfill\hfill\hfill\hfill\hfill\hfill\hfill\hfill\hfill\hfill\hfill\hfill\hfill\hfill\hfill\hfill\hfill\hfill\hfill\hfill\hfill\hfill\hfill\hfill\hfill\hfill\hfill\hfill\hfill\hfill}{\ }
\contentsline {section}{\textbf{Шоргин С.\,Я.}\ \ см. Батракова Д.\,А.\hfill\hfill\hfill\hfill\hfill\hfill\hfill\hfill\hfill\hfill\hfill\hfill\hfill\hfill\hfill\hfill\hfill\hfill\hfill\hfill\hfill\hfill\hfill\hfill\hfill\hfill\hfill\hfill\hfill\hfill\hfill\hfill\hfill\hfill\hfill}{\ } 
\contentsline {section}{\textbf{Шоргин С.\,Я.}\ \ см. Кудрявцев А.\,А.\hfill\hfill\hfill\hfill\hfill\hfill\hfill\hfill\hfill\hfill\hfill\hfill\hfill\hfill\hfill\hfill\hfill\hfill\hfill\hfill\hfill\hfill\hfill\hfill\hfill\hfill\hfill\hfill\hfill\hfill\hfill\hfill\hfill\hfill\hfill}{\ }
%\thispagestyle{myheadings}
\def\leftfootline{\small{\textbf{\thepage}
\hfill ИНФОРМАТИКА И ЕЁ ПРИМЕНЕНИЯ\ \ \ том~1\ \ \ выпуск~2\ \ \ 2007}
}%
 \def\rightfootline{\small{ИНФОРМАТИКА И ЕЁ ПРИМЕНЕНИЯ\ \ \ том~1\ \ \ выпуск~2\ \ \ 2007
 \hfill \textbf{\thepage}}}
 \label{end\stat} 
                     
%\def\stat{cont-e}
{%\hrule\par
%\vskip 7pt % 7pt
\raggedleft\Large \bf%\baselineskip=3.2ex
2\,0\,0\,7\ \ A\,U\,T\,H\,O\,R\ \ I\,N\,D\,E\,X \vskip 17pt
    \hrule
    \par
\vskip 21pt plus 6pt minus 3pt }

\label{st\stat}

\def\tit{\ }

\def\aut{\ }
\def\auf{\ }

\def\leftkol{\ } % ENGLISH ABSTRACTS}

\def\rightkol{\ } %ENGLISH ABSTRACTS}

\titele{\tit}{\aut}{\auf}{\leftkol}{\rightkol}


\contentsline {chapter}{\ }{Issue \quad Page} 
\contentsline {subsection}{\textbf{Batrakova D.\,A., Korolev V.\,Yu., Shorgin S.\,Ya.}\ \ A New Method for the Probabilistic and Statistical Analysis of Information Flows in Telecommunication Networks}{\qquad 1 \qquad 40} 
\contentsline {subsection}{\textbf{Borisov A.\,V.}\ \ Bayesian Estimation in\nobreakspace {}Observation Systems with\nobreakspace {}Markov Jump Processes: Game-Theoretic Approach}{\qquad 2 \qquad 65} 
\contentsline {subsection}{\textbf{Bosov A.\,V., Ivanov A.\,V.}\ \ Linguistic Simulation for Machine Translation and Knowledge Management Systems}{\qquad 2 \qquad 50} 
\contentsline {subsection}{\textbf{Chaplygin V.\,V.} see Pechinkin A.\,V.\hfill\hfill\hfill\hfill\hfill\hfill\hfill\hfill\hfill\hfill\hfill\hfill\hfill\hfill\hfill\hfill\hfill\hfill\hfill\hfill\hfill\hfill\hfill\hfill\hfill\hfill\hfill\hfill\hfill\hfill\hfill\hfill\hfill\hfill\hfill}{\ }
\contentsline {subsection}{\textbf{Chaplygin V.\,V.} see Pechinkin A.\,V.\hfill\hfill\hfill\hfill\hfill\hfill\hfill\hfill\hfill\hfill\hfill\hfill\hfill\hfill\hfill\hfill\hfill\hfill\hfill\hfill\hfill\hfill\hfill\hfill\hfill\hfill\hfill\hfill\hfill\hfill\hfill\hfill\hfill\hfill\hfill}{\ }
\contentsline {subsection}{\textbf{Ilyin V.\,D., Sokolov I.\,A.}\ \ The Symbol Model of Informatics Knowledge System in Human-Automaton Environment}{\qquad 1 \qquad 66} 
\contentsline {subsection}{\textbf{Ivanov A.\,V.} see Bosov A.\,V.\hfill\hfill\hfill\hfill\hfill\hfill\hfill\hfill\hfill\hfill\hfill\hfill\hfill\hfill\hfill\hfill\hfill\hfill\hfill\hfill\hfill\hfill\hfill\hfill\hfill\hfill\hfill\hfill\hfill\hfill\hfill\hfill\hfill\hfill\hfill}{\ }
\contentsline {subsection}{\textbf{Kalinichenko L.\,A.} see Zakharov V.\,N.\hfill\hfill\hfill\hfill\hfill\hfill\hfill\hfill\hfill\hfill\hfill\hfill\hfill\hfill\hfill\hfill\hfill\hfill\hfill\hfill\hfill\hfill\hfill\hfill\hfill\hfill\hfill\hfill\hfill\hfill\hfill\hfill\hfill\hfill\hfill}{\ }
\contentsline {subsection}{\textbf{Korolev V.\,Yu.} see Batrakova D.\,A.\hfill\hfill\hfill\hfill\hfill\hfill\hfill\hfill\hfill\hfill\hfill\hfill\hfill\hfill\hfill\hfill\hfill\hfill\hfill\hfill\hfill\hfill\hfill\hfill\hfill\hfill\hfill\hfill\hfill\hfill\hfill\hfill\hfill\hfill\hfill}{\ }
\contentsline {subsection}{\textbf{Kozerenko E.\,B.}\ \ Linguistic Simulation for Machine Translation and Knowledge Management Systems}{\qquad 1 \qquad 54} 
\contentsline {subsection}{\textbf{Kozmidiady V.\,A.} see Zakharov V.\,N.\hfill\hfill\hfill\hfill\hfill\hfill\hfill\hfill\hfill\hfill\hfill\hfill\hfill\hfill\hfill\hfill\hfill\hfill\hfill\hfill\hfill\hfill\hfill\hfill\hfill\hfill\hfill\hfill\hfill\hfill\hfill\hfill\hfill\hfill\hfill}{\ }
\contentsline {subsection}{\textbf{Kudryavtsev A.\,A., Shorgin S.\,Ya.}\ \ Bayesian Approach to Queueing Systems and Reliability Characteristics}{\qquad 2 \qquad 76} 
\contentsline {subsection}{\textbf{Pechinkin A.\,V., Sokolov I.\,A., Chaplygin V.\,V.}\ \ Multichannel Queuing System with Finite Buffer and Unreliable Servers}{\qquad 1 \qquad 27} 
\contentsline {subsection}{\textbf{Pechinkin A.\,V., Sokolov I.\,A., Chaplygin V.\,V.}\ \ Stationary Characteristics of a Multichannel Queueing System with\nobreakspace {}Simultaneous Refusals of Servers}{\qquad 2 \qquad 39} 
\contentsline {subsection}{\textbf{Shorgin S.\,Ya.} see Batrakova D.\,A.\hfill\hfill\hfill\hfill\hfill\hfill\hfill\hfill\hfill\hfill\hfill\hfill\hfill\hfill\hfill\hfill\hfill\hfill\hfill\hfill\hfill\hfill\hfill\hfill\hfill\hfill\hfill\hfill\hfill\hfill\hfill\hfill\hfill\hfill\hfill}{\ }
\contentsline {subsection}{\textbf{Shorgin S.\,Ya.} see Kudryavtsev A.\,A.\hfill\hfill\hfill\hfill\hfill\hfill\hfill\hfill\hfill\hfill\hfill\hfill\hfill\hfill\hfill\hfill\hfill\hfill\hfill\hfill\hfill\hfill\hfill\hfill\hfill\hfill\hfill\hfill\hfill\hfill\hfill\hfill\hfill\hfill\hfill}{\ }
\contentsline {subsection}{\textbf{Sinitsyn I.\,N.}\ \ Correlational Methods for Analytical Informational Models of the Earth Pole Fluctuations Design Based on a priori Data}{\qquad 2 \qquad \hphantom{9}2}
\contentsline {subsection}{\textbf{Sinitsyn I.\,N.}\ \ Development of Pugachev Filtering for Stochastic Systems}{\qquad 1 \qquad \hphantom{9}3}
\contentsline {subsection}{\textbf{Sokolov I.\,A.} see Ilyin V.\,D.\hfill\hfill\hfill\hfill\hfill\hfill\hfill\hfill\hfill\hfill\hfill\hfill\hfill\hfill\hfill\hfill\hfill\hfill\hfill\hfill\hfill\hfill\hfill\hfill\hfill\hfill\hfill\hfill\hfill\hfill\hfill\hfill\hfill\hfill\hfill}{\ }
\contentsline {subsection}{\textbf{Sokolov I.\,A.} see Pechinkin A.\,V.\hfill\hfill\hfill\hfill\hfill\hfill\hfill\hfill\hfill\hfill\hfill\hfill\hfill\hfill\hfill\hfill\hfill\hfill\hfill\hfill\hfill\hfill\hfill\hfill\hfill\hfill\hfill\hfill\hfill\hfill\hfill\hfill\hfill\hfill\hfill}{\ }
\contentsline {subsection}{\textbf{Sokolov I.\,A.} see Pechinkin A.\,V.\hfill\hfill\hfill\hfill\hfill\hfill\hfill\hfill\hfill\hfill\hfill\hfill\hfill\hfill\hfill\hfill\hfill\hfill\hfill\hfill\hfill\hfill\hfill\hfill\hfill\hfill\hfill\hfill\hfill\hfill\hfill\hfill\hfill\hfill\hfill}{\ }
\contentsline {subsection}{\textbf{Sokolov I.\,A.} see Zakharov V.\,N.\hfill\hfill\hfill\hfill\hfill\hfill\hfill\hfill\hfill\hfill\hfill\hfill\hfill\hfill\hfill\hfill\hfill\hfill\hfill\hfill\hfill\hfill\hfill\hfill\hfill\hfill\hfill\hfill\hfill\hfill\hfill\hfill\hfill\hfill\hfill}{\ }
\contentsline {subsection}{\textbf{Stupnikov S.\,A.} see Zakharov V.\,N.\hfill\hfill\hfill\hfill\hfill\hfill\hfill\hfill\hfill\hfill\hfill\hfill\hfill\hfill\hfill\hfill\hfill\hfill\hfill\hfill\hfill\hfill\hfill\hfill\hfill\hfill\hfill\hfill\hfill\hfill\hfill\hfill\hfill\hfill\hfill}{\ }
\contentsline {subsection}{\textbf{Zakharov V.\,N., Kalinichenko L.\,A., Sokolov I.\,A., Stupnikov S.\,A.}\ \ Development of Canonical Information Models for Integrated Information Systems}{\qquad 2 \qquad 15} 
\contentsline {subsection}{\textbf{Zakharov V.\,N., Kozmidiady V.\,A.}\ \ Means Providing Applications Fault Tolerance}{\qquad 1 \qquad 14} 
\def\leftfootline{\small{\textbf{\thepage}
\hfill ИНФОРМАТИКА И ЕЁ ПРИМЕНЕНИЯ\ \ \ том~1\ \ \ выпуск~2\ \ \ 2007}
}%
 \def\rightfootline{\small{ИНФОРМАТИКА И ЕЁ ПРИМЕНЕНИЯ\ \ \ том~1\ \ \ выпуск~2\ \ \ 2007
 \hfill \textbf{\thepage}}}
 \label{end\stat} 


%\end{document}

%
\def\stat{rekl}
%\label{preobr}

%\def\tit{АКАДЕМИК ПУГАЧЁВ  ВЛАДИМИР СЕМЁНОВИЧ\\
%25.03.1911--25.03.1998}


%   \vspace*{-48pt}
%   \begin{center}\LARGE
%Академик Пугачёв  Владимир Семёнович\\ (25.03.1911--25.03.1998)
%   \end{center}

   %\vspace*{2.5mm}

   \begin{center}

{\prgsh\LARGE
ЮБИЛЕИ}

\end{center}
%\hrule

\vspace*{6pt}


   \vspace*{8mm}

   \thispagestyle{empty}


%\def\stat{emel}


\section*{К 70-летию заместителя директора ИПИ РАН,\\ члена редколлегии журнала
<<Информатика и её применения>>\\ доктора технических наук В.\,И.~Будзко}

\vspace*{18pt}




          \begin{multicols}{2}

%            \label{st\stat}

\begin{center}
\vspace*{1pt}
\mbox{%
\epsfxsize=78mm
\epsfbox{bud-1.eps}
}
\end{center}

\vspace*{12pt}

      14 августа 2014~г.\ исполнилось 70~лет за\-мес\-ти\-те\-лю директора ИПИ РАН по
научной работе доктору технических наук Владимиру Игоревичу Будзко.

      Владимир Игоревич Будзко родился в г.~Москве. Высшее образование получил на факультете
элект\-рон\-но-вы\-чис\-ли\-тель\-ных устройств в Московском
ин\-же\-нер\-но-фи\-зи\-че\-ском институте
(МИФИ), который он окончил в 1968~г., после чего был на\-прав\-лен для прохождения
службы в одну из войс\-ко\-вых частей, где прошел путь от инженера до первого заместителя
командира войсковой части.

      С приходом В.\,И.~Будзко в ИПИ РАН (2001~г.)\ в институте
сформировалось новое научное на\-прав\-ле\-ние теоретических исследований~--- <<Постро\-ение
ин\-фор\-ма\-ци\-он\-но-те\-ле\-ком\-му\-ни\-ка\-ци\-он\-ных\linebreak сис\-тем
высокой до\-ступ\-ности>>. В~рамках этого
направления выполнен широкий круг фундаментальных исследований по поиску подходов и
определению принципов построения средств обеспечения доступности, конфиденциальности
и целостности современных крупномасштабных
ин\-фор\-ма\-ци\-он\-но-те\-ле\-ком\-му\-ни\-ка\-ци\-он\-ных
сис\-тем (ИТС). Разработаны основные сис\-тем\-но-тех\-ни\-че\-ские принципы и базовые
архитектурные решения построения перспективных для условий России ИТС с
централизованной обработкой и хранением информации, сочетающих в себе свойства
высокой доступности, отказо- и катастрофоустойчивости, информационной защищенности.
Определены принципы, методы и математические основы рационального построения и
оптимизации средств восстановления функционирования центров обработки данных (ЦОД)
после возникновения отказов и катастроф, передачи и хранения данных, обеспечения
информационной безопасности при достижении минимальной совокупной стоимости
владения такими системами. Результаты нашли практическое воплощение при реализации
проектов в интересах ряда отечественных государственных и негосударственных
организаций, таких как Банк России (БР), Внешторгбанк, ОАО <<ГМК <<Норильский Никель>>,
<<Газпром>>, Минэкономразвития России, Правительство Москвы, а также ряд силовых
ведомств.

      Под руководством В.\,И.~Будзко начиная с 2001~г.\ выполнен комплекс
      на\-уч\-но-ис\-сле\-до\-ва\-тель\-ских и
      опыт\-но-кон\-ст\-рук\-тор\-ских работ (свыше 100~проектов),
направленных на развитие электронной информационной технологии БР.
Разработаны концепции развития ИТС БР сначала до 2008~г., а затем до 2013~г., которые
были приняты в качестве основы проведения технической политики. За реализацию проекта
<<Катастрофоустойчивая тер\-ри\-то\-ри\-аль\-но-рас\-пре\-де\-лен\-ная
      ин\-фор\-ма\-ци\-он\-но-те\-ле\-ком\-му\-ни\-ка\-ци\-он\-ная сис\-те\-ма централизованной
обработки банковской информации>> В.\,И.~Будзко удостоен Премии Правительства РФ в
области науки и техники за 2010~г.

      В.\,И.~Будзко возглавлял и возглавляет работы по ряду других прикладных проектов,
связанных с созданием, совершенствованием и развитием крупномасштабных ИТС.

      В.\,И.~Будзко~--- генерал-майор, доктор технических наук, член-кор\-рес\-пон\-дент
Академии криптографии РФ, известный ученый в области информатики и применения
информационных технологий при построении территориально распределенных ИТС
различного назначения. Является автором свыше 250~научных работ, опубликованных в
на\-уч\-но-тех\-ни\-че\-ских и специальных изданиях.

    \thispagestyle{empty}

      В.\,И.~Будзко уделяет большое внимание подготовке научных кадров. Под его
руководством защищено 6~диссертаций на соискание ученой степени кандидата
технических наук. Свыше 30~лет он читает лекции в ИКСИ Академии ФСБ, профессор
кафедры НИЯУ МИФИ. Является членом двух диссертационных советов, главным
редактором журнала <<Системы высокой доступности>> и членом редколлегии журнала
<<Информатика и её применения>>.

      \bigskip

      Редакционный совет и Редакционная коллегия журнала <<Информатика и её
применения>> сердечно поздравляют Владимира Игоревича Будзко с 70-ле\-ти\-ем и желают
крепкого здоровья и новых научных достижений.

\end{multicols}

%%Информатика и её применения
%Том 12   Выпуск 1-4   Год 2018

\def\stat{cont}
{%\hrule\par
%\vskip 7pt % 7pt
\raggedleft\Large \bf%\baselineskip=3.2ex
А\,В\,Т\,О\,Р\,С\,К\,И\,Й\ \ У\,К\,А\,З\,А\,Т\,Е\,Л\,Ь\ \ З\,А\ \ 2\,0\,1\,8 г. \vskip 17pt
 \hrule
 \par
\vskip 21pt plus 6pt minus 3pt }

\label{st\stat}

\def\tit{\ }

\def\aut{\ }
\def\auf{\ }

\def\leftkol{\ } % ENGLISH ABSTRACTS}

\def\rightkol{\ } %АВТОРСКИЙ УКАЗАТЕЛЬ ЗА 2018 г.} %ENGLISH ABSTRACTS}

\titele{\tit}{\aut}{\auf}{\leftkol}{\rightkol}
\addcontentsline{toc}{subsection}{\textrm\textbf Авторский указатель за 2018 г.}

\vspace*{-12pt}
\vspace*{-36pt}

\noindent
{\tabcolsep=3pt
\begin{tabular}{p{397pt}cc}
&\textbf{Вып.} & \textbf{Стр.}\\[6pt]
\Avtors{Агаларов~Я.\,М.} Оптимизация объема буферной памяти узла коммутации при схеме\linebreak
\\[-12pt]
\hspace*{23pt}полного разделения памяти&4&25--32\\
\Avtors{Агасандян~Г.\,А.} Континуальный критерий VaR на сценарных рынках&1&31--39\\
\Avtors{Алешин~И.\,С.} О формальной постановке задач поиска сгущений в разреженных булевых\linebreak
\\[-12pt]
\hspace*{23pt}матрицах&1&40--48\\
\Avtors{Арутюнов~Е.\,Н., Кудрявцев~А.\,А., Титова~А.\,И.} Гамма-вейбулловский случай априорных\linebreak
\\[-12pt]
\hspace*{23pt}распределений в~байесовских моделях массового обслуживания&4&92--95\\
\Avtors{Атаева~О.\,М., Серебряков~В.\,А.} Онтология цифровой семантической библиотеки LibMeta&1&\hphantom{1}2--10\\
\Avtors{Басок~Б.\,М., Захаров~В.\,Н., Френкель~С.\,Л.} Использование вероятностной модели вычислений для тестирования одного класса готовых к~использованию программных\linebreak
\\[-12pt]
\hspace*{23pt}компонентов локальных и~сетевых систем&4&44--51\\
\Avtors{Батенков~А.\,А., Маньяков~Ю.\,А., Гасилов~А.\,В., Яковлев~О.\,А.} Математическая модель\linebreak
\\[-12pt]
\hspace*{23pt}оптимальной триангуляции&2&69--74\\
\Avtors{Бахтеев~О.\,Ю.} см.~Огальцов~А.\,В.&&\\
\Avtors{Бахтеев~О.\,Ю.} см.~Смердов~А.\,Н.&&\\
\Avtors{Борисов~А.\,В.} Фильтрация состояний марковских скачкообразных процессов по дискре-\linebreak
\\[-12pt]
\hspace*{23pt}тизованным наблюдениям&3&115--121\\
\Avtors{Босов~А.\,В., Игнатов~А.\,Н., Наумов~А.\,В.} Модель передвижения поездов и маневровых локомотивов на железнодорожной станции в приложении к оценке и анализу\linebreak
\\[-12pt]
\hspace*{23pt}вероятности бокового столкновения&3&107--114\\
\Avtors{Босов~А.\,В., Стефанович~А.\,И.} Управление выходом стохастической дифференциальной системы по квадратичному критерию. I.~Оптимальное решение методом динами-\linebreak
\\[-12pt]
\hspace*{23pt}ческого программирования&3&\hphantom{1}99--106\\
\Avtors{Бунтман~Н.\,В., Гончаров~А.\,А., Зацман~И.\,М., Нуриев~В.\,А.} Количественный анализ\linebreak
\\[-12pt]
\hspace*{23pt}результатов машинного перевода с~использованием надкорпусных баз данных&4&\hphantom{1}96--105\\
\Avtors{Бунтман~Н.\,В.} см.~Нуриев~В.\,А.&&\\
\Avtors{Быковец~Е.\,В., Лаврентьев~В.\,В., Назаров~Л.\,В.} Вероятностная модель влияния книги\linebreak
\\[-12pt]
\hspace*{23pt}заказов на процесс цены&2&29--34\\
\Avtors{Васильева~С.\,Н., Кан~Ю.\,С.} Алгоритм визуализации плоского ядра вероятностной меры&2&60--68\\
\Avtors{Виноградов~Д.\,В.} Учет предварительных оценок скорости порождения сходств спарива-\linebreak
\\[-12pt]
\hspace*{23pt}ющей цепью Маркова&1&49--54\\
\Avtors{Вышинский~Л.\,Л., Флеров~Ю.\,А., Широков~Н.\,И.} Автоматизированная система весового\linebreak
\\[-12pt]
\hspace*{23pt}проектирования самолетов&1&18--30\\
\Avtors{Гайдамака~Ю.\,В.} см.~Горбунова~А.\,В.&&\\
\Avtors{Гайдамака~Ю.\,В.} см.~Самуйлов~К.\,Е.&&\\
\Avtors{Гасилов~А.\,В.} см.~Батенков~А.\,А.,&&\\
\Avtors{Гончаров~А.\,А.} см.~Бунтман~Н.\,В.&&\\
\Avtors{Горбунова~А.\,В., Наумов~В.\,А., Гайдамака~Ю.\,В., Самуйлов~К.\,Е.} Ресурсные системы\linebreak
\\[-12pt]
\hspace*{23pt}массового обслуживания как модели беспроводных систем связи&3&48--55\\
\Avtors{Горшенин~А.\,К.} Зашумление данных конечными смесями нормальных и гамма-рас\-пре-\linebreak
\\[-12pt]
\hspace*{23pt}де\-ле\-ний с применением к задаче округления наблюдений&3&28--34\\
\Avtors{Горшенин~А.\,К.} Развитие сервисов цифровых платформ для преодоления нефинансовых\linebreak
\\[-12pt]
\hspace*{23pt}барьеров&4&106--112\\
\Avtors{Горшенин~А.\,К., Королев~В.\,Ю.} Определение экстремальности объемов осадков на основе\linebreak
\\[-12pt]
\hspace*{23pt}модифицированного метода превышения порогового значения&4&16--24\\
\Avtors{Горшенин~А.\,К.} см.~Королев~В.\,Ю.&&\\
\end{tabular}
}

\pagebreak

\def\leftkol{АВТОРСКИЙ УКАЗАТЕЛЬ ЗА 2018 г.} % ENGLISH ABSTRACTS}

\def\rightkol{АВТОРСКИЙ УКАЗАТЕЛЬ ЗА 2018 г.} %ENGLISH ABSTRACTS}

%\thispagestyle{myheadings}
\def\leftfootline{\small{\textbf{\thepage}
\hfill ИНФОРМАТИКА И ЕЁ ПРИМЕНЕНИЯ\ \ \ том~12\ \ \ выпуск~4\ \ \ 2018}
}%
 \def\rightfootline{\small{ИНФОРМАТИКА И ЕЁ ПРИМЕНЕНИЯ\ \ \ том~12\ \ \ выпуск~4\ \ \ 2018
 \hfill \textbf{\thepage}}}


\noindent
{\tabcolsep=3pt
\begin{tabular}{p{394pt}cc}
&\textbf{Вып.} & \textbf{Стр.}\\[3pt]
\Avtors{Грушо~А.\,А., Грушо~Н.\,А., Забежайло~М.\,И., Смирнов~Д.\,В., Тимонина~Е.\,Е.} Параметриза-\linebreak
\\[-12pt]
\hspace*{23pt}ция в прикладных задачах поиска эмпирических причин&3&62--66\\
\Avtors{Грушо~А.\,А., Грушо~Н.\,А., Левыкин~М.\,В., Тимонина~Е.\,Е.} Методы идентификации захвата хоста в~распределенной информационно-вычислительной сис\-те\-ме, защищенной\linebreak
\\[-12pt]
\hspace*{23pt}с~по\-мощью метаданных&4&39--43\\
\Avtors{Грушо~А.\,А., Забежайло~М.\,И., Зацаринный~А.\,А., Тимонина~Е.\,Е.} О некоторых возможностях управления ресурсами при организации проактивного противодействия\linebreak
\\[-12pt]
\hspace*{23pt}компьютерным атакам&1&62--70\\
\Avtors{Грушо~А.\,А., Тимонина~Е.\,Е., Шоргин~С.\,Я.} Иерархический метод порождения метадан-\linebreak
\\[-12pt]
\hspace*{23pt}ных для управления сетевыми соединениями&2&44--49\\
\Avtors{Грушо~Н.\,А.} см.~Грушо~А.\,А.&&\\
\Avtors{Грушо~Н.\,А.} см.~Грушо~А.\,А.&&\\
\Avtors{Дорофеева~А.\,В.} см.~Королев~В.\,Ю.&&\\
\Avtors{Егоров~А.\,Ю.} см.~Шнурков~П.\,В.&&\\
\Avtors{Егоров~А.\,Ю.} см.~Шнурков~П.\,В.&&\\
\Avtors{Жуков~Д.\,О., Хватова~Т.\,Ю., Лесько~С.\,А., Зальцман~А.\,Д.} Влияние плотности связей на кластеризацию и порог перколяции при распространении информации в~со\-ци\-аль-\linebreak
\\[-12pt]
\hspace*{23pt}ных сетях&2&90--97\\
\Avtors{Забежайло~М.\,И.} см.~Грушо~А.\,А.&&\\
\Avtors{Забежайло~М.\,И.} см.~Грушо~А.\,А.&&\\
\Avtors{Зальцман~А.\,Д.} см.~Жуков~Д.\,О.&&\\
\Avtors{Захаров~В.\,Н.} см.~Басок~Б.\,М.&&\\
\Avtors{Захаров~В.\,Н.} см.~Шанин~И.\,А.&&\\
\Avtors{Зацаринный~А.\,А., Сучков~А.\,П.} Система ситуационного управления как мультисервисная\linebreak
\\[-12pt]
\hspace*{23pt}технология в облачной среде&1&78--88\\
\Avtors{Зацаринный~А.\,А.} см.~Грушо~А.\,А.&&\\
\Avtors{Зацман~И.\,М.} Имплицированные знания: основания и технологии извлечения&3&74--82\\
\Avtors{Зацман~И.\,М.} см.~Бунтман~Н.\,В.&&\\
\Avtors{Зейфман~А.\,И.} см.~Королев~В.\,Ю.&&\\
\Avtors{Зубарев~Д.\,В.} см.~Соченков~И.\,В.&&\\
\Avtors{Игнатов~А.\,Н.} см.~Босов~А.\,В.&&\\
\Avtors{Инькова~О.\,Ю., Кружков~М.\,Г.} Статистический анализ лингвоспецифичности коннек-\linebreak
\\[-12pt]
\hspace*{23pt}торов (на материале параллельных корпусов)&3&83--90\\
\Avtors{Инькова~О.\,Ю.} см.~Нуриев~В.\,А.&&\\
\Avtors{Кан~Ю.\,С.} см.~Васильева~С.\,Н.&&\\
\Avtors{Ковалёв~С.\,П.} Теория категорий как математическая прагматика модельно-ори\-ен\-ти-\linebreak
\\[-12pt]
\hspace*{23pt}ро\-ван\-ной системной инженерии&1&\hphantom{1}95--104\\
\Avtors{Козеренко~Е.\,Б., Кузнецов~К.\,И., Романов~Д.\,А.} Семантическая обработка неструктури-\linebreak
\\[-12pt]
\hspace*{23pt}рованных текстовых данных на основе лингвистического процессора PullEnti&3&91--98\\
\Avtors{Кондранин~Е.\,С., Ушаков~В.\,Г.} Система обслуживания с~относительным приоритетом\linebreak
\\[-12pt]
\hspace*{23pt}и~профилактиками прибора&4&33--38\\
\Avtors{Коновалов~М.\,Г., Разумчик~Р.\,В.} Сравнение двух механизмов активного управления\linebreak
\\[-12pt]
\hspace*{23pt}очередью в~системе $M/D/1/N$&4&\hphantom{1}9--15\\
\Avtors{Коновалов~М.\,Г., Разумчик~Р.\,В.} Управление случайным блужданием с эталонным\linebreak
\\[-12pt]
\hspace*{23pt}стационарным распределением&3&\hphantom{1}2--13\\
\Avtors{Королев~В.\,Ю., Горшенин~А.\,К., Зейфман~А.\,И.} Новые представления обобщенного\linebreak
\\[-12pt]
\hspace*{23pt}распределения Миттаг-Леффлера в~виде смесей и~их приложения&4&75--85\\
\Avtors{Королев~В.\,Ю., Дорофеева~А.\,В.} О~неравномерных оценках точности нормальной аппроксимации для распределений некоторых случайных сумм при ослабленных\linebreak
\\[-12pt]
\hspace*{23pt}моментных условиях&4&86--91\\
\Avtors{Королев~В.\,Ю.} см.~Горшенин~А.\,К.&&\\
\Avtors{Кривенко~М.\,П.}\ Обучаемая классификация данных с учетом анализа главных компонент&3&56--61\\
\Avtors{Кривенко~М.\,П.}\ Реконструкция осей главных компонент&1&71--77\\
\Avtors{Кружков~М.\,Г.} см.~Инькова~О.\,Ю.&&\\
\end{tabular}
}

\pagebreak

\def\leftkol{АВТОРСКИЙ УКАЗАТЕЛЬ ЗА 2018 г.} % ENGLISH ABSTRACTS}

\def\rightkol{АВТОРСКИЙ УКАЗАТЕЛЬ ЗА 2018 г.} %ENGLISH ABSTRACTS}

%\thispagestyle{myheadings}
\def\leftfootline{\small{\textbf{\thepage}
\hfill ИНФОРМАТИКА И ЕЁ ПРИМЕНЕНИЯ\ \ \ том~12\ \ \ выпуск~4\ \ \ 2018}
}%
 \def\rightfootline{\small{ИНФОРМАТИКА И ЕЁ ПРИМЕНЕНИЯ\ \ \ том~12\ \ \ выпуск~4\ \ \ 2018
 \hfill \textbf{\thepage}}}


\noindent
{\tabcolsep=3pt
\begin{tabular}{p{394pt}cc}
&\textbf{Вып.} & \textbf{Стр.}\\[3pt]
\Avtors{Кудрявцев~А.\,А.} Байесовские модели баланса&3&18--27\\
\Avtors{Кудрявцев~А.\,А., Шестаков~О.\,В.} Байесовские модели тестирования больших групп\linebreak
\\[-12pt]
\hspace*{23pt}обслуживающих приборов&1&105--108\\
\Avtors{Кудрявцев~А.\,А., Шестаков О.\,В.} Минимизация ошибок вычисления вейвлет-ко\-эф\-фи-\linebreak
\\[-12pt]
\hspace*{23pt}ци\-ен\-тов при решении обратных задач&2&17--23\\
\Avtors{Кудрявцев~А.\,А.} см.~Арутюнов~Е.\,Н.&&\\
\Avtors{Кузнецов~К.\,И.} см.~Козеренко~Е.\,Б.&&\\
\Avtors{Лаврентьев~В.\,В.} см.~Быковец~Е.\,В.&&\\
\Avtors{Лебедев~А.\,В.} Максимальные ветвящиеся процессы в случайной среде&2&35--43\\
\Avtors{Левыкин~М.\,В.} см.~Грушо~А.\,А.&&\\
\Avtors{Лери~М.\,М., Павлов~Ю.\,Л.} Об устойчивости конфигурационных графов в случайной\linebreak
\\[-12pt]
\hspace*{23pt}среде&2&\hphantom{1}2--10\\
\Avtors{Лесько~С.\,А.} см.~Жуков~Д.\,О.&&\\
\Avtors{Логачев~О.\,А.} Теоретико-информационная характеризация совершенно уравновешен-\linebreak
\\[-12pt]
\hspace*{23pt}ных функций&4&70--74\\
\Avtors{Малашенко~Ю.\,Е., Назарова~И.\,А., Новикова~Н.\,М.} Анализ разрезных повреждений\linebreak
\\[-12pt]
\hspace*{23pt}в~многополюсных сетях&3&35--41\\
\Avtors{Малашенко~Ю.\,Е., Назарова~И.\,А., Новикова~Н.\,М.} Диаграммы уязвимости потоковых\linebreak
\\[-12pt]
\hspace*{23pt}сетевых систем&1&11--17\\
\Avtors{Маньяков~Ю.\,А.} см.~Батенков~А.\,А.&&\\
\Avtors{Мирзабеков~Я.\,М., Шихиев~Ш.\,Б.} Дискретный анализ в синтаксическом анализе&2&\hphantom{1}98--104\\
\Avtors{Мистрюков~А.\,В., Ушаков~В.\,Г.} Достаточные условия эргодичности приоритетных\linebreak
\\[-12pt]
\hspace*{23pt}систем массового обслуживания&2&24--28\\
\Avtors{Назаров~Л.\,В.} см.~Быковец~Е.\,В.&&\\
\Avtors{Назарова~И.\,А.} см.~Малашенко~Ю.\,Е.&&\\
\Avtors{Назарова~И.\,А.} см.~Малашенко~Ю.\,Е.&&\\
\Avtors{Наумов~А.\,В.} см.~Босов~А.\,В.&&\\
\Avtors{Наумов~В.\,А.} см.~Горбунова~А.\,В.&&\\
\Avtors{Наумов~В.\,А.} см.~Сопин~Э.\,С.&&\\
\Avtors{Новикова~Н.\,М.} см.~Малашенко~Ю.\,Е.&&\\
\Avtors{Новикова~Н.\,М.} см.~Малашенко~Ю.\,Е.&&\\
\Avtors{Нуриев~В.\,А., Бунтман~Н.\,В., Инькова~О.\,Ю.} Ошибки и~неточности машинного перевода\linebreak
\\[-12pt]
\hspace*{23pt}русских коннекторов на~французский язык&2&105--113\\
\Avtors{Нуриев~В.\,А.} см.~Бунтман~Н.\,В.&&\\
\Avtors{Огальцов~А.\,В., Бахтеев~О.\,Ю.} Автоматическое извлечение метаданных из научных\linebreak
\\[-12pt]
\hspace*{23pt}PDF-документов&2&75--82\\
\Avtors{Павлов~Ю.\,Л.} см.~Лери~М.\,М.&&\\
\Avtors{Разумчик~Р.\,В.} см.~Коновалов~М.\,Г.&&\\
\Avtors{Разумчик~Р.\,В.} см.~Коновалов~М.\,Г.&&\\
\Avtors{Романов~Д.\,А.} см.~Козеренко~Е.\,Б.&&\\
\Avtors{Самуйлов~К.\,Е., Гайдамака~Ю.\,В., Шоргин~С.\,Я.} Применение моделей случайного\linebreak
\\[-12pt]
\hspace*{23pt}блуждания при моделировании перемещения устройств в~беспроводной сети&4&2--8\\
\Avtors{Самуйлов~К.\,Е.} см.~Горбунова~А.\,В.&&\\
\Avtors{Самуйлов~К.\,Е.} см.~Сопин~Э.\,С.&&\\
\Avtors{Серебряков~В.\,А.} см.~Атаева~О.\,М.&&\\
\Avtors{Синицын~И.\,Н.} Метод интерполяционного аналитического моделирования одномерных\linebreak
\\[-12pt]
\hspace*{23pt}распределений в стохастических системах&1&55--61\\
\Avtors{Смердов~А.\,Н., Бахтеев~О.\,Ю., Стрижов~В.\,В.} Выбор оптимальной модели рекуррентной\linebreak
\\[-12pt]
\hspace*{23pt}сети в~задачах поиска парафраза&4&63--69\\
\Avtors{Смирнов~Д.\,В.} см.~Грушо~А.\,А.&&\\
\Avtors{Сопин~Э.\,С., Наумов~В.\,А., Самуйлов~К.\,Е.} Об инвариантности стационарного распределения системы массового обслуживания с ограниченными ресурсами и~с~ин\-тен-\linebreak
\\[-12pt]
\hspace*{23pt}сив\-ностями поступления и~обслуживания, зависящими от состояния системы&3&42--47\\
\Avtors{Соченков~И.\,В., Зубарев~Д.\,В., Тихомиров~И.\,А.} Эксплоративный патентный поиск&1&89--94\\
\end{tabular}
}

\pagebreak

\def\leftkol{АВТОРСКИЙ УКАЗАТЕЛЬ ЗА 2018 г.} % ENGLISH ABSTRACTS}

\def\rightkol{АВТОРСКИЙ УКАЗАТЕЛЬ ЗА 2018 г.} %ENGLISH ABSTRACTS}

%\thispagestyle{myheadings}
\def\leftfootline{\small{\textbf{\thepage}
\hfill ИНФОРМАТИКА И ЕЁ ПРИМЕНЕНИЯ\ \ \ том~12\ \ \ выпуск~4\ \ \ 2018}
}%
 \def\rightfootline{\small{ИНФОРМАТИКА И ЕЁ ПРИМЕНЕНИЯ\ \ \ том~12\ \ \ выпуск~4\ \ \ 2018
 \hfill \textbf{\thepage}}}


\noindent
{\tabcolsep=3pt
\begin{tabular}{p{394pt}cc}
&\textbf{Вып.} & \textbf{Стр.}\\[3pt]
\Avtors{Стефанович~А.\,И.} см.~Босов~А.\,В.&&\\
\Avtors{Стрижов~В.\,В.} см.~Смердов~А.\,Н.&&\\
\Avtors{Ступников~С.\,А.} см.~Шанин~И.\,А.&&\\
\Avtors{Сурина~А.\,А.} см.~Тырсин~А.\,Н.&&\\
\Avtors{Сучков~А.\,П.} см.~Зацаринный~А.\,А.&&\\
\Avtors{Сюнтюренко~О.\,В.} Финансирование фундаментальных исследований: концептуальный облик системы поддержки принятия решений с использованием методов\linebreak
\\[-12pt]
\hspace*{23pt}наукометрии и анализа данных&1&118--127\\
\Avtors{Тимонина~Е.\,Е.} см.~Грушо~А.\,А.&&\\
\Avtors{Тимонина~Е.\,Е.} см.~Грушо~А.\,А.&&\\
\Avtors{Тимонина~Е.\,Е.} см.~Грушо~А.\,А.&&\\
\Avtors{Тимонина~Е.\,Е.} см.~Грушо~А.\,А.&&\\
\Avtors{Титова~А.\,И.} см.~Арутюнов~Е.\,Н.&&\\
\Avtors{Тихомиров~И.\,А.} см.~Соченков~И.\,В.&&\\
\Avtors{Тырсин~А.\,Н., Сурина~А.\,А.} Модели управления риском в гауссовских стохастических\linebreak
\\[-12pt]
\hspace*{23pt}системах&2&50--59\\
\Avtors{Ушаков~В.\,Г.} см.~Кондранин~Е.\,С.&&\\
\Avtors{Ушаков~В.\,Г.} см.~Мистрюков~А.\,В.&&\\
\Avtors{Флеров~Ю.\,А.} см.~Вышинский~Л.\,Л.&&\\
\Avtors{Френкель~С.\,Л., Ханкин~Д.} Непрерывные обновления маршрута в~SDN с~использованием\linebreak
\\[-12pt]
\hspace*{23pt}проверки соответствия качеству обслуживания&4&52--62\\
\Avtors{Френкель~С.\,Л.} см.~Басок~Б.\,М.&&\\
\Avtors{Ханкин~Д.} см.~Френкель~С.\,Л.&&\\
\Avtors{Хватова~Т.\,Ю.} см.~Жуков~Д.\,О.&&\\
\Avtors{Шанин~И.\,А., Ступников~С.\,А., Захаров~В.\,Н.} Методы и средства обнаружения нештатных\linebreak
\\[-12pt]
\hspace*{23pt}ситуаций, возникающих на элементах жилищно-коммунальной инфраструктуры&3&67--73\\
\Avtors{Шестаков~О.\,В.} Несмещенная оценка риска стабилизированной жесткой пороговой\linebreak
\\[-12pt]
\hspace*{23pt}обработки в модели с долгосрочной зависимостью&2&11--16\\
\Avtors{Шестаков~О.\,В.} Среднеквадратичный риск пороговой обработки при случайном объеме\linebreak
\\[-12pt]
\hspace*{23pt}выборки&3&14--17\\
\Avtors{Шестаков~О.\,В.} см.~Кудрявцев~А.\,А.&&\\
\Avtors{Шестаков~О.\,В.} см.~Кудрявцев~А.\,А.&&\\
\Avtors{Широков~Н.\,И.} см.~Вышинский~Л.\,Л.&&\\
\Avtors{Шихиев~Ш.\,Б.} см.~Мирзабеков~Я.\,М.&&\\
\Avtors{Шнурков~П.\,В., Егоров~А.\,Ю.} Разработка и предварительное исследование стохастической полумарковской модели управления запасом непрерывного продукта при\linebreak
\\[-12pt]
\hspace*{23pt}постоянно происходящем потреблении&1&109--117\\
\Avtors{Шнурков~П.\,В., Егоров~А.\,Ю.} Решение проблемы оптимального управления запасом непрерывного продукта при постоянно происходящем потреблении в стохастической\linebreak
\\[-12pt]
\hspace*{23pt}полумарковской модели&2&83--89\\
\Avtors{Шоргин~С.\,Я.} см.~Грушо~А.\,А.&&\\
\Avtors{Шоргин~С.\,Я.} см.~Самуйлов~К.\,Е.&&\\
\Avtors{Яковлев~О.\,А.} см.~Батенков~А.\,А.&&\\
\end{tabular}
}

%\thispagestyle{myheadings}
\def\leftfootline{\small{\textbf{\thepage}
\hfill ИНФОРМАТИКА И ЕЁ ПРИМЕНЕНИЯ\ \ \ том~12\ \ \ выпуск~4\ \ \ 2018}
}%
 \def\rightfootline{\small{ИНФОРМАТИКА И ЕЁ ПРИМЕНЕНИЯ\ \ \ том~12\ \ \ выпуск~4\ \ \ 2018
 \hfill \textbf{\thepage}}}

 \label{end\stat}

\newpage

%Информатика и её применения
%Том 12   Выпуск 1-4   Год 2018

\def\stat{cont-e}
{%\hrule\par
%\vskip 7pt % 7pt
\raggedleft\Large \bf%\baselineskip=3.2ex
2\,0\,1\,8\ \ A\,U\,T\,H\,O\,R\ \ I\,N\,D\,E\,X \vskip 17pt
 \hrule
 \par
\vskip 21pt plus 6pt minus 3pt }

\label{st\stat}

\def\tit{\ }

\def\aut{\ }
\def\auf{\ }

\def\leftkol{\ } %2018 AUTHOR INDEX} % ENGLISH ABSTRACTS}

\def\rightkol{\ } %2018 AUTHOR INDEX} %ENGLISH ABSTRACTS}

\titele{\tit}{\aut}{\auf}{\leftkol}{\rightkol}
\addcontentsline{toc}{subsection}{\textrm\textbf 2018 Author Index}

\def\leftfootline{\small{\textbf{\thepage}
\hfill INFORMATIKA I EE PRIMENENIYA~--- INFORMATICS AND APPLICATIONS\ \ \ 2018\
\ \ volume~12\ \ \ issue\ 4}
}%
 \def\rightfootline{\small{INFORMATIKA I EE PRIMENENIYA~--- INFORMATICS AND APPLICATIONS\ \ \ 2018\ \ \ volume~12\ \ \ issue\ 4
\hfill \textbf{\thepage}}}

\vspace*{-12pt}
\vspace*{-18pt}

\noindent
{\tabcolsep=3pt
\begin{tabular}{p{396pt}cc}
&\textbf{Issue} & \textbf{Page}\\[6pt]
\Avtors{Agalarov~Yа.\,M.} Optimization of buffer memory size of switching node in mode of full memory\linebreak
\\[-12pt]
\hspace*{23pt}sharing&4&25--32\\
\Avtors{Agasandyan~G.\,A.} Continuous VaR-criterion in scenario markets&1&31--39\\
\Avtors{Aleshin~I.\,S.} On the formalization of tasks searching dense submatrices in boolean sparse\linebreak
\\[-12pt]
\hspace*{23pt}matrices&1&40--48\\
\Avtors{Arutyunov~E.\,N., Kudryavtsev~A.\,A., and~Titova~A.\,I.} Gamma-Weibull \textit{a~priori} distributions\linebreak
\\[-12pt]
\hspace*{23pt}in~Bayesian queuing models&4&92--95\\
\Avtors{Ataeva~O.\,M.} see~Serebryakov~V.\,A.&&\\
\Avtors{Bakhteev~O.\,Y.} see~Ogaltsov~A.\,V.&&\\
\Avtors{Bakhteev~O.\,Y.} see~Smerdov~A.\,N.&&\\
\Avtors{Basok~B.\,M., Zakharov~V.\,N., and~Frenkel~S.\,L.} Using a probabilistic calculation model to test\linebreak
\\[-12pt]
\hspace*{23pt}one class of ready-to-use software components of local and network systems&4&44--51\\
\Avtors{Batenkov~A.\,A., Maniakov Yu.\,A., Gasilov A.\,V., and Yakovlev O.\,A.} Mathematical model\linebreak
\\[-12pt]
\hspace*{23pt}of~optimal triangulation&2&69--74\\
\Avtors{Borisov~A.\,V.} Filtering of Markov jump processes by discretized observations&3&115--121\\
\Avtors{Bosov~A.\,V., Ignatov~A.\,N., and Naumov~A.\,V.} Model of transportation of trains and shunting\linebreak
\\[-12pt]
\hspace*{23pt}locomotives at a railway station for evaluation and analysis of side-collision probability&3&107--114\\
\Avtors{Bosov~A.\,V.\ and Stefanovich~A.\,I.} Stochastic differential system output control by the quadratic\linebreak
\\[-12pt]
\hspace*{23pt}criterion. I.~Dynamic programming optimal solution&3&\hphantom{1}99--106\\
\Avtors{Buntman~N.\,V., Goncharov~A.\,A., Zatsman~I.\,M., and~Nuriev~V.\,A.} Using supracorpora databases\linebreak
\\[-12pt]
\hspace*{23pt}for quantitative analysis of machine translations&4&\hphantom{1}96--105\\
\Avtors{Buntman~N.\,V.} see~Nuriev~V.\,A., &&\\
\Avtors{Bykovets~E.\,V.} see~Nazarov~L.\,V.&&\\
\Avtors{Dorofeeva~A.\,V.} see~Korolev~V.\,Yu.&&\\
\Avtors{Egorov~A.\,Y.} see~Shnurkov~P.\,V.&&\\
\Avtors{Egorov~A.\,Y.} see~Shnurkov~P.\,V.&&\\
\Avtors{Flerov~Yu.\,A.} see~Vyshinsky~L.\,L.&&\\
\Avtors{Frenkel~S.\,L.\ and Khankin~D.} Seamless route updates in software-defined networking via quality\linebreak
\\[-12pt]
\hspace*{23pt}of~service compliance verification &4&52--62\\
\Avtors{Frenkel~S.\,L.} see~Basok~B.\,M.&&\\
\Avtors{Gaidamaka~Yu.\,V.} see~Gorbunova~A.\,V.&&\\
\Avtors{Gaidamaka~Yu.\,V.} see~Samouylov~K.\,E.&&\\
\Avtors{Gasilov A.\,V.} see~Batenkov~A.\,A.&&\\
\Avtors{Goncharov~A.\,A.} see~Buntman~N.\,V.&&\\
\Avtors{Gorbunova~A.\,V., Naumov~V.\,A., Gaidamaka~Yu.\,V., and Samouylov~K.\,E.} Resource queuing\linebreak
\\[-12pt]
\hspace*{23pt}systems as models of wireless communication systems&3&48--55\\
\Avtors{Gorshenin~A.\,K.} Data noising by finite normal and gamma mixtures with application to~the~prob-\linebreak
\\[-12pt]
\hspace*{23pt}lem of rounded observations&3&28--34\\
\Avtors{Gorshenin~A.\,K.} Development of services of digital platforms to overcome nonfinancial barriers&4&106--112\\
\Avtors{Gorshenin~A.\,K.\ and~Korolev~V.\,Yu.} Determining the extremes of precipitation volumes based\linebreak
\\[-12pt]
\hspace*{23pt}on~the~modified ``Peaks over Threshold'' method&4&16--24\\
\Avtors{Gorshenin~A.\,K.} see~Korolev~V.\,Yu.&&\\
\Avtors{Grusho~A.\,A., Grusho~N.\,A., Levykin~M.\,V., and~Timonina~E.\,E.} Methods of identification of host\linebreak
\\[-12pt]
\hspace*{23pt}capture in a distributed information system which is protected on the basis of meta data&4&39--43\\
\Avtors{Grusho~A.\,A., Grusho~N.\,A., Zabezhailo~M.\,I., Smirnov~D.\,V., and Timonina~E.\,E.} Parametrization\linebreak
\\[-12pt]
\hspace*{23pt}in applied problems of search of empirical reasons&3&62--66\\
\end{tabular}
}
\pagebreak

\def\leftfootline{\small{\textbf{\thepage}
\hfill INFORMATIKA I EE PRIMENENIYA~--- INFORMATICS AND APPLICATIONS\ \ \ 2018\
\ \ volume~12\ \ \ issue\ 4}
}%
 \def\rightfootline{\small{INFORMATIKA I EE PRIMENENIYA~---
INFORMATICS AND APPLICATIONS\ \ \ 2018\ \ \ volume~12\ \ \ issue\ 4
\hfill \textbf{\thepage}}}

\def\leftkol{2018 AUTHOR INDEX} % ENGLISH ABSTRACTS}

\def\rightkol{2018 AUTHOR INDEX} %ENGLISH ABSTRACTS}


\noindent
{\tabcolsep=3pt
\begin{tabular}{p{395.48108pt}cc}
&\textbf{Issue} & \textbf{Page}\\[6pt]
\Avtors{Grusho~A.\,A., Timonina~E.\,E., and Shorgin~S.\,Ya.} Hierarchical method of meta data generation\linebreak
\\[-12pt]
\hspace*{23pt}for control of network connections&2&44--49\\
\Avtors{Grusho~A.\,A., Zabezhailo~M.\,I., Zatsarinny~A.\,A., and Timonina~E.\,E.} On some possibilities\linebreak
\\[-12pt]
\hspace*{23pt}of~resource management for organizing active counteraction to computer attacks&1&62--70\\
\Avtors{Grusho~N.\,A.} see~Grusho~A.\,A.&&\\
\Avtors{Grusho~N.\,A.} see~Grusho~A.\,A.&&\\
\Avtors{Ignatov~A.\,N.} see~Bosov~A.\,V.&&\\
\Avtors{Inkova~O.\,Yu.\ and Kruzhkov~M.\,G.} Statistical analysis of language specificity of connectives\linebreak
\\[-12pt]
\hspace*{23pt}based on parallel texts&3&83--90\\
\Avtors{Inkova~O.\,Yu.} see~Nuriev~V.\,A., &&\\
\Avtors{Kan~Yu.\,S.} see~Vasil'eva~S.\,N.&&\\
\Avtors{Khankin~D.} see~Frenkel~S.\,L.&&\\
\Avtors{Khvatova~T.\,Yu.} see~Zhukov~D.\,O.&&\\
\Avtors{Kondranin~E.\,S.\ and~Ushakov~V.\,G.} A~head of the line priority queue with working vacations&4&33--38\\
\Avtors{Konovalov~M.\,G.\ and Razumchik~R.\,V.} Comparison of two active queue management schemes\linebreak
\\[-12pt]
\hspace*{23pt}through the $M/D/1/N$ queue&4&\hphantom{1}9--15\\
\Avtors{Konovalov~M.\,G.\ and Razumchik~R.\,V.} Finding control policy for one discrete-time Markov\linebreak
\\[-12pt]
\hspace*{23pt}chain on [0,1] with a given invariant measure&3&\hphantom{1}2--13\\
\Avtors{Korolev~V.\,Yu.\ and~Dorofeeva~A.\,V.} On nonuniform estimates of accuracy of normal approxima-\linebreak
\\[-12pt]
\hspace*{23pt}tion for distributions of some random sums under relaxed moment conditions&4&86--91\\
\Avtors{Korolev~V.\,Yu., Gorshenin~A.\,K., and~Zeifman~A.\,I. } New mixture representations of~the~general-\linebreak
\\[-12pt]
\hspace*{23pt}ized Mittag-Leffler distribution and their applications&4&75--85\\
\Avtors{Korolev~V.\,Yu.} see~Gorshenin~A.\,K.&&\\
\Avtors{Kovalyov~S.\,P.} Category theory as a mathematical pragmatics of model-based systems engineer-\linebreak
\\[-12pt]
\hspace*{23pt}ing&1&\hphantom{1}95--104\\
\Avtors{Kozerenko~E.\,B., Kuznetsov~K.\,I., and Romanov~D.\,A.} Semantic processing of unstructured\linebreak
\\[-12pt]
\hspace*{23pt}textual data based on the linguistic processor PullEnti&3&91--98\\
\Avtors{Krivenko~M.\,P.} Principal axes reconstruction&1&71--77\\
\Avtors{Krivenko~M.\,P.} Supervised learning classification of data taking into account principal compo-\linebreak
\\[-12pt]
\hspace*{23pt}nent analysis&3&56--61\\
\Avtors{Kruzhkov~M.\,G.} see~Inkova~O.\,Yu.&&\\
\Avtors{Kudryavtsev~A.\,A.} Bayesian balance models&3&18--27\\
\Avtors{Kudryavtsev~A.\,A.\ and Shestakov~O.\,V.} Bayesian models for testing large groups of service devices&1&105--108\\
\Avtors{Kudryavtsev~A.\,A.\ and Shestakov~O.\,V.} Minimization of errors of calculating wavelet coefficients\linebreak
\\[-12pt]
\hspace*{23pt}while solving inverse problems&2&17--23\\
\Avtors{Kudryavtsev~A.\,A.} see~Arutyunov~E.\,N.&&\\
\Avtors{Kuznetsov~K.\,I.} see~Kozerenko~E.\,B.&&\\
\Avtors{Lavrentyev~V.\,V.} see~Nazarov~L.\,V.&&\\
\Avtors{Lebedev~A.\,V.} Maximal branching processes in random environment&2&35--43\\
\Avtors{Leri~M.\,M.\ and Pavlov~Yu.\,L.} On the robustness of configuration graphs in a random environment&2&\hphantom{1}2--10\\
\Avtors{Lesko~S.\,A.} see~Zhukov~D.\,O.&&\\
\Avtors{Levykin~M.\,V.} see~Grusho~A.\,A.&&\\
\Avtors{Logachev~O.\,A.} An information based criterion for perfectly balanced functions&4&70--74\\
\Avtors{Malashenko~Yu.\,E., Nazarova~I.\,A., and Novikova~N.\,M.} Analysis of cutting damages to multipolar\linebreak
\\[-12pt]
\hspace*{23pt}networks&3&35--41\\
\Avtors{Malashenko~Yu.\,E., Nazarova~I.\,A., and Novikova~N.\,M.} Diagrams of the functional vulnerability\linebreak
\\[-12pt]
\hspace*{23pt}of flow network systems&1&11--17\\
\Avtors{Maniakov Yu.\,A.} see~Batenkov~A.\,A.&&\\
\Avtors{Mirzabekov~Ya.\,M.\ and Shihiev~Sh.\,B.} Discrete analysis in parsing&2&\hphantom{1}98--104\\
\Avtors{Mistryukov~A.\,V.\ and Ushakov~V.\,G.} Sufficient ergodicity conditions for priority queues&2&24--28\\
\Avtors{Naumov~A.\,V.} see~Bosov~A.\,V.&&\\
\Avtors{Naumov~V.\,A.} see~Gorbunova~A.\,V.&&\\
\Avtors{Naumov~V.\,A.} see~Sopin~E.\,S.&&\\
\end{tabular}
}
\pagebreak

\def\leftfootline{\small{\textbf{\thepage}
\hfill INFORMATIKA I EE PRIMENENIYA~--- INFORMATICS AND APPLICATIONS\ \ \ 2018\
\ \ volume~12\ \ \ issue\ 4}
}%
 \def\rightfootline{\small{INFORMATIKA I EE PRIMENENIYA~---
INFORMATICS AND APPLICATIONS\ \ \ 2018\ \ \ volume~12\ \ \ issue\ 4
\hfill \textbf{\thepage}}}

\def\leftkol{2018 AUTHOR INDEX} % ENGLISH ABSTRACTS}

\def\rightkol{2018 AUTHOR INDEX} %ENGLISH ABSTRACTS}


\noindent
{\tabcolsep=3pt
\begin{tabular}{p{395.48108pt}cc}
&\textbf{Issue} & \textbf{Page}\\[6pt]
\Avtors{Nazarov~L.\,V., Lavrentyev~V.\,V., and Bykovets~E.\,V.} A~probability model of the influence\linebreak
\\[-12pt]
\hspace*{23pt}of~the~order book on the price process&2&29--34\\
\Avtors{Nazarova~I.\,A.} see~Malashenko~Yu.\,E.&&\\
\Avtors{Nazarova~I.\,A.} see~Malashenko~Yu.\,E.&&\\
\Avtors{Novikova~N.\,M.} see~Malashenko~Yu.\,E.&&\\
\Avtors{Novikova~N.\,M.} see~Malashenko~Yu.\,E.&&\\
\Avtors{Nuriev~V.\,A., Buntman~N.\,V., and Inkova~O.\,Yu.} Machine translation of russian connectives into\linebreak
\\[-12pt]
\hspace*{23pt}french: Errors and quality failures&2&105--113\\
\Avtors{Nuriev~V.\,A.} see~Buntman~N.\,V.&&\\
\Avtors{Ogaltsov~A.\,V.\ and Bakhteev~O.\,Y.} Automatic metadata extraction from scientific PDF documents&2&75--82\\
\Avtors{Pavlov~Yu.\,L.} see~Leri~M.\,M.&&\\
\Avtors{Razumchik~R.\,V.} see~Konovalov~M.\,G.&&\\
\Avtors{Razumchik~R.\,V.} see~Konovalov~M.\,G.&&\\
\Avtors{Romanov~D.\,A.} see~Kozerenko~E.\,B.&&\\
\Avtors{Samouylov~K.\,E., Gaidamaka~Yu.\,V., and~Shorgin~S.\,Ya.} Modeling movement of devices in\linebreak
\\[-12pt]
\hspace*{23pt}a~wireless network by random walk models&4&2--8\\
\Avtors{Samouylov~K.\,E.} see~Gorbunova~A.\,V.&&\\
\Avtors{Samouylov~K.\,Е.} see~Sopin~E.\,S.&&\\
\Avtors{Serebryakov~V.\,A.\ and Ataeva~O.\,M.} Ontology of the digital semantic library LibMeta&1&\hphantom{1}2--10\\
\Avtors{Shanin~I.\,A., Stupnikov~S.\,A., and Zakharov~V.\,N.} Methods and tools for fault detection\linebreak
\\[-12pt]
\hspace*{23pt}on~elements of housing and utility infrastructure&3&67--73\\
\Avtors{Shestakov~O.\,V.} Mean-square thresholding risk with a random sample size&3&14--17\\
\Avtors{Shestakov~O.\,V.} Unbiased risk estimate of stabilized hard thresholding in the model with\linebreak
\\[-12pt]
\hspace*{23pt}a~long-range dependence&2&11--16\\
\Avtors{Shestakov~O.\,V.} see~Kudryavtsev~A.\,A.&&\\
\Avtors{Shestakov~O.\,V.} see~Kudryavtsev~A.\,A.&&\\
\Avtors{Shihiev~Sh.\,B.} see~Mirzabekov~Ya.\,M.&&\\
\Avtors{Shirokov~N.\,I.} see~Vyshinsky~L.\,L.&&\\
\Avtors{Shnurkov~P.\,V.\ and Egorov~A.\,Y.} Development and preliminary study of a~stochastic semi-Markov model of continuous supply of product management under the condition of\linebreak
\\[-12pt]
\hspace*{23pt}constant consumption&1&109--117\\
\Avtors{Shnurkov~P.\,V.\ and Egorov~A.\,Y.} Solution to the problem of optimal control of a~stochastic semi-Markov model of continuous supply of product management under the condition\linebreak
\\[-12pt]
\hspace*{23pt}of~constantly happening consumption&2&83--89\\
\Avtors{Shorgin~S.\,Ya.} see~Grusho~A.\,A.&&\\
\Avtors{Shorgin~S.\,Ya.} see~Samouylov~K.\,E.&&\\
\Avtors{Sinitsyn~I.\,N.} Method of interpolational analytical modeling of processes in stochastic systems&1&55--61\\
\Avtors{Smerdov~A.\,N., Bakhteev~O.\,Y., and~Strijov~V.\,V.} Optimal recurrent neural network model\linebreak
\\[-12pt]
\hspace*{23pt}in~paraphrase detection&4&63--69\\
\Avtors{Smirnov~D.\,V.} see~Grusho~A.\,A.&&\\
\Avtors{Sochenkov~I.\,V., Zubarev~D.\,V., and Tikhomirov~I.\,A.} Exploratory patent search&1&89--94\\
\Avtors{Sopin~E.\,S., Naumov~V.\,A., and Samouylov~K.\,Е.} On the insensitivity of the stationary distribution\linebreak
\\[-12pt]
\hspace*{23pt}of the limited resources queuing system with state-dependent arrival and service rates&3&42--47\\
\Avtors{Stefanovich~A.\,I.} see~Bosov~A.\,V.&&\\
\Avtors{Strijov~V.\,V.} see~Smerdov~A.\,N.&&\\
\Avtors{Stupnikov~S.\,A.} see~Shanin~I.\,A.&&\\
\Avtors{Suchkov~A.\,P.} see~Zatsarinny~A.\,A.&&\\
\Avtors{Surina~A.\,A.} see~Tyrsin~A.\,N.&&\\
\Avtors{Syuntyurenko~O.\,V.} Financing of basic research: Conceptual shape of a system of support\linebreak
\\[-12pt]
\hspace*{23pt}of~decision-making with use of methods of scientometrics and analysis of data&1&118--127\\
\Avtors{Tikhomirov~I.\,A.} see~Sochenkov~I.\,V.&&\\
\Avtors{Timonina~E.\,E.} see~Grusho~A.\,A.&&\\
\Avtors{Timonina~E.\,E.} see~Grusho~A.\,A.&&\\
\end{tabular}
}
\pagebreak

\def\leftfootline{\small{\textbf{\thepage}
\hfill INFORMATIKA I EE PRIMENENIYA~--- INFORMATICS AND APPLICATIONS\ \ \ 2018\
\ \ volume~12\ \ \ issue\ 4}
}%
 \def\rightfootline{\small{INFORMATIKA I EE PRIMENENIYA~---
INFORMATICS AND APPLICATIONS\ \ \ 2018\ \ \ volume~12\ \ \ issue\ 4
\hfill \textbf{\thepage}}}

\def\leftkol{2018 AUTHOR INDEX} % ENGLISH ABSTRACTS}

\def\rightkol{2018 AUTHOR INDEX} %ENGLISH ABSTRACTS}


\noindent
{\tabcolsep=3pt
\begin{tabular}{p{395.48108pt}cc}
&\textbf{Issue} & \textbf{Page}\\[6pt]
\Avtors{Timonina~E.\,E.} see~Grusho~A.\,A.&&\\
\Avtors{Timonina~E.\,E.} see~Grusho~A.\,A.&&\\
\Avtors{Titova~A.\,I.} see~Arutyunov~E.\,N.&&\\
\Avtors{Tyrsin~A.\,N.\ and Surina~A.\,A.} A~model of risk management in Gaussian stochastic systems&2&50--59\\
\Avtors{Ushakov~V.\,G.} see~Kondranin~E.\,S.&&\\
\Avtors{Ushakov~V.\,G.} see~Mistryukov~A.\,V.&&\\
\Avtors{Vasil'eva~S.\,N.\ and Kan~Yu.\,S.} A~visualization algorithm for the plane probability measure kernel&2&60--68\\
\Avtors{Vinogradov~D.\,V.} Influence of preliminary estimates on the speed of search of similarities by\linebreak
\\[-12pt]
\hspace*{23pt}the~coupling Markov chain&1&49--54\\
\Avtors{Vyshinsky~L.\,L., Flerov~Yu.\,A., and Shirokov~N.\,I.} Computer-aided system of aircraft weight\linebreak
\\[-12pt]
\hspace*{23pt}design&1&18--30\\
\Avtors{Yakovlev O.\,A.} see~Batenkov~A.\,A.&&\\
\Avtors{Zabezhailo~M.\,I.} see~Grusho~A.\,A.&&\\
\Avtors{Zabezhailo~M.\,I.} see~Grusho~A.\,A.&&\\
\Avtors{Zakharov~V.\,N.} see~Basok~B.\,M.&&\\
\Avtors{Zakharov~V.\,N.} see~Shanin~I.\,A.&&\\
\Avtors{Zaltsman~A.\,D.} see~Zhukov~D.\,O.&&\\
\Avtors{Zatsarinny~A.\,A.\ and Suchkov~A.\,P.} The situational management system as a multiservice\linebreak
\\[-12pt]
\hspace*{23pt}technology in the cloud&1&78--88\\
\Avtors{Zatsarinny~A.\,A.} see~Grusho~A.\,A.,&&\\
\Avtors{Zatsman~I.\,M.} Implied knowledge: Foundations and technologies of explication&3&74--82\\
\Avtors{Zatsman~I.\,M.} see~Buntman~N.\,V.&&\\
\Avtors{Zeifman~A.\,I.} see~Korolev~V.\,Yu.&&\\
\Avtors{Zhukov~D.\,O., Khvatova~T.\,Yu., Lesko~S.\,A., and Zaltsman~A.\,D.} The influence of the connections' density on clusterization and percolation threshold during information distribution in social\linebreak
\\[-12pt]
\hspace*{23pt}networks&2&90--97\\
\Avtors{Zubarev~D.\,V.} see~Sochenkov~I.\,V.&&\\
\end{tabular}
}

%\thispagestyle{myheadings}
\def\leftfootline{\small{\textbf{\thepage}
\hfill INFORMATIKA I EE PRIMENENIYA~--- INFORMATICS AND APPLICATIONS\ \ \ 2018\
\ \ volume~12\ \ \ issue\ 4}
}%
 \def\rightfootline{\small{INFORMATIKA I EE PRIMENENIYA~---
INFORMATICS AND APPLICATIONS\ \ \ 2018\ \ \ volume~12\ \ \ issue\ 4
\hfill \textbf{\thepage}}}

 \label{end\stat}

\newpage

%   \vspace*{-48pt}

\begin{center}
\vspace*{6pt}
\mbox{%
\epsfxsize=53.502mm
\epsfbox{foto-1.eps}
}
\end{center}

\vspace*{6pt} %Академик


   \begin{center}
\fbox{\Large\textbf{Профессор Игорь Алексеевич Ушаков}}\\[12pt]
\textbf{\large 22.01.1935--27.02.2015}
   \end{center}


   %\vspace*{2.5mm}

   \vspace*{5mm}

   \thispagestyle{empty}

%\

%\vspace*{-12pt}


Редакционный совет и редакционная коллегия журнала <<Информатика и~её применения>> с~глубоким прискорбием извещают, что 27~февраля 2015~г.\ после тяжелой
и~продолжительной болезни скончался Игорь Алексеевич Ушаков~--- доктор технических наук, профессор, член редколлегии журнала <<Информатика и ее применения>>.

Игорь Алексеевич Ушаков окончил Московский авиационный институт, в~1963~г.\ защитил кандидатскую, а~в~1968~г.~--- докторскую диссертацию. С~1958 по 1989~гг.\ работал в~ряде научно-исследовательских организаций СССР, в~том числе руководил отделами в~НИИ АА и~ВЦ АН СССР; с 1969 по 1989 гг. преподавал в~МФТИ (был профессором, а~затем заведующим кафедрой) и~в~МЭИ. С~1989~г.~---- в~США: являлся профессором университета Дж.\ Вашингтона, университета Дж.\ Мэйсона и~Калифорнийского университета, сотрудником компаний MCI, Qualcomm и Hughes.

И.\,А.~Ушаков с момента основания журнала <<Надежность и~контроль качества>> был заместителем ответственного редактора, а~затем на протяжении многих лет членом редколлегии. В~2006~г.\ основал электронный международный журнал ``Reliability: Theory \& Application'', главным редактором которого оставался до конца жизни.

Учебниками и справочниками по теории надежности, написанными И.\,А.~Ушаковым, пользовались и~пользуются несколько поколений ученых и~специалистов в~разных странах мира.

Игорь Алексеевич всегда уделял огромное внимание работе с~молодежью; более~50 его учеников защитили докторские и~кандидатские диссертации.

И.\,А.~Ушаков вел активную научно-про\-све\-ти\-тель\-скую деятельность. В~частности, он был одним из организаторов и~руководителей Московского кабинета качества и~надежности при Политехническом музее (целью этого Кабинета было оказание консультаций работникам промышленных предприятий и~чтение курсов лекций для инженеров, занимающихся проблемой надежности). Находясь в~США, И.\,А.~Ушаков создал международный ин\-тер\-нет-фо\-рум им.\ Б.\,В.~Гнеденко, объединивший около~400~видных специалистов по приложениям теории вероятностей и~математической статистики, преимущественно в~об\-ласти теории надежности и~анализа риска, из десятков стран мира; коллективным членов этого Форума является и~наш журнал. Цели Форума~--- содействие контактам между специалистами из разных стран, организация обмена профессиональными 
новостями и~информацией (новые публикации, предстоящие события и~др.). Также необходимо отметить большое число на\-уч\-но-по\-пу\-ляр\-ных работ, опубликованных И.\,А.~Ушаковым.

И.\,А.~Ушаков обладал большим личным обаянием, имел широкий круг интересов. Все знавшие И.\,А.~Ушакова всегда будут помнить его как замечательного ученого и~прекрасного человека.

\bigskip

Редакционный совет и редакционная коллегия журнала <<Информатика и~её применения>> 
выражают глубокие соболезнования родным и близким покойного, всем, кто его знал и~работал с~ним.


%\def\stat{cont}
{%\hrule\par
%\vskip 7pt % 7pt
\raggedleft\Large \bf%\baselineskip=3.2ex
А\,В\,Т\,О\,Р\,С\,К\,И\,Й\ \ У\,К\,А\,З\,А\,Т\,Е\,Л\,Ь\ \ З\,А\ \ 2\,0\,1\,0 г. \vskip 17pt
    \hrule
    \par
\vskip 21pt plus 6pt minus 3pt }

\label{st\stat}

\def\tit{\ }

\def\aut{\ }
\def\auf{\ }

\def\leftkol{\ } % ENGLISH ABSTRACTS}

\def\rightkol{\ } %АВТОРСКИЙ УКАЗАТЕЛЬ ЗА 2010 г.} %ENGLISH ABSTRACTS}

\titele{\tit}{\aut}{\auf}{\leftkol}{\rightkol}

\vspace*{-12pt}

{\tabcolsep=3pt
\begin{tabular}{p{388pt}rr}
&\textbf{Выпуск} & \textbf{Стр.}\\[6pt]
\hangindent=23pt\noindent\textbf{Арутюнян~А.\,Р.} Моделирование влияния деформаций отпечатков пальцев на 
точность\linebreak
\vspace*{-12pt}\\
\hspace*{23pt}дактилоскопической идентификации$\dotfill$&1&51\\
\hangindent=23pt\noindent\textbf{Архипов~О.\,П., Зыкова~З.\,П.} Интеграция гетерогенной информации о цветных 
пикселях\linebreak
\vspace*{-12pt}\\
\hspace*{23pt}и их цветовосприятии$\dotfill$&4&15\\
\hangindent=23pt\noindent\textbf{Баранов~С.\,И., Френкель~С.\,Л., Захаров~В.\,Н.} Полуформальная верификация 
цифрового устройства с конвейером, основанная на использовании алгоритмических машин\linebreak
\vspace*{-12pt}\\
\hspace*{23pt}состояния$\dotfill$&4&49\\
\textbf{Бекетова~И.\,В.} см.~Каратеев~С.\,Л.&&\\
\textbf{Белоусов~В.\,В.} см.~Синицын~И.\,Н.&&\\
\hangindent=23pt\noindent\textbf{Бенинг~В.\,Е., Королев~Р.\,А.} О предельном поведении мощностей критериев в 
случае\linebreak
\vspace*{-12pt}\\
\hspace*{23pt}распределения Лапласа$\dotfill$&2&63\\
\hangindent=23pt\noindent\textbf{Бенинг~В.\,Е., Сипина~А.\,В.} Асимптотическое разложение для мощности 
критерия,\linebreak
\vspace*{-12pt}\\
\hspace*{23pt}основанного на выборочной медиане, в случае распределения Лапласа$\dotfill$&1&18\\
\textbf{Бондаренко~А.\,В.} см.~Каратеев~С.\,Л.&&\\
\hangindent=23pt\noindent\textbf{Бородина~А.\,В., Морозов~Е.\,В.} Об оценивании асимптотики вероятности 
большого\linebreak
\vspace*{-12pt}\\
\hspace*{23pt}уклонения стационарной регенеративной очереди с одним прибором$\dotfill$&3&29\\
\hangindent=23pt\noindent\textbf{Бунтман~Н.\,В., Минель~Ж.-Л., Ле~Пезан~Д., Зацман~И.\,М.} Типология и 
компьютерное\linebreak
\vspace*{-12pt}\\
\hspace*{23pt}моделирование трудностей перевода$\dotfill$&3&77\\
\textbf{Визильтер~Ю.\,В.} см.~Каратеев~С.\,Л.&&\\
\hangindent=23pt\noindent\textbf{Гавриленко~С.\,В.} Оценки скорости сходимости распределений случайных сумм с 
безгранично делимыми индексами к нормальному закону$\dotfill$&4&81\\
\hangindent=23pt\noindent\textbf{Григорьева~М.\,Е., Шевцова~И.\,Г.} Уточнение неравенства 
Каца--Берри--Эссеена$\dotfill$&2&75\\
\hangindent=23pt\noindent\textbf{Грушо~А.\,А., Грушо~Н.\,А., Тимонина~Е.\,Е.} Поиск конфликтов в политиках 
безопасности: модель случайных графов$\dotfill$&3&38\\
\textbf{Грушо~Н.\,А.} см.~Грушо~А.\,А.&&\\
\hangindent=23pt\noindent\textbf{Гудков~В.\,Ю.} Математические модели изображения отпечатка пальца на основе 
описания линий$\dotfill$&1&58\\
\textbf{Гуртов~А.\,В.} см.~Лукьяненко~А.\,С.&&\\
\textbf{Желтов~С.\,Ю.} см.~Каратеев~С.\,Л.&&\\
\hangindent=23pt\noindent\textbf{Захаров~А.\,А., Серебряков~В.\,А.} Система управления электронной библиотекой 
LibMeta$\dotfill$&4&2\\
\textbf{Захаров~В.\,Н.} см.~Баранов~С.\,И.&&\\
\textbf{Захарова~Т.\,В.} см.~Матвеева~С.\,С.&&\\
\hangindent=23pt\noindent\textbf{Зацаринный~А.\,А., Чупраков~К.\,Г.} Некоторые аспекты выбора технологии для 
постро-\linebreak
\vspace*{-12pt}\\
\hspace*{23pt}ения систем отображения информации ситуационного центра$\dotfill$&3&59\\
\textbf{Зацман~И.\,М.} см.~Бунтман~Н.\,В.&&\\
\hangindent=23pt\noindent\textbf{Зейфман~А.\,И., Коротышева~А.\,В., Сатин~Я.\,А., Шоргин~С.\,Я.} Об 
устойчивости нестаци-\linebreak
\vspace*{-12pt}\\
\hspace*{23pt}онарных систем обслуживания с катастрофами$\dotfill$&3&9\\
\textbf{Зыкова~З.\,П.} см.~Архипов~О.\,П.&&\\
\hangindent=23pt\noindent\textbf{Илюшин~Г.\,Я., Соколов~И.\,А.} Организация управляемого доступа пользователей 
к\linebreak
\vspace*{-12pt}\\
\hspace*{23pt}разнородным ведомственным информационным ресурсам$\dotfill$&1&24\\
\hangindent=23pt\noindent\textbf{Кавагучи~Ю., Ульянов~В.\,В., Фуджикоши~Я.} Приближения для статистик, 
описывающих\linebreak
\vspace*{-12pt}\\
\hspace*{23pt}геометрические свойства данных большой размерности, с оценками 
ошибок$\dotfill$&1&12\\
\hangindent=23pt\noindent\textbf{Каратеев~С.\,Л., Бекетова~И.\,В., Ососков~М.\,В., Князь~В.\,А., 
Визильтер~Ю.\,В., Бондаренко~А.\,В., Желтов~С.\,Ю.} Автоматизированный контроль 
качества цифровых\linebreak
\vspace*{-12pt}\\
\hspace*{23pt}изображений для персональных документов$\dotfill$&1&65\\
\end{tabular}
}

\pagebreak

\def\leftkol{АВТОРСКИЙ УКАЗАТЕЛЬ ЗА 2010 г.} % ENGLISH ABSTRACTS}

\def\rightkol{АВТОРСКИЙ УКАЗАТЕЛЬ ЗА 2010 г.} %ENGLISH ABSTRACTS}

{\tabcolsep=3pt
\begin{tabular}{p{388pt}rr}
&\textbf{Выпуск} & \textbf{Стр.}\\[3pt]
\hangindent=23pt\noindent\textbf{Козеренко~Е.\,Б.} Лингвистические фильтры в статистических моделях машинного\linebreak
\vspace*{-12pt}\\
\hspace*{23pt}перевода$\dotfill$&2&83\\
\hangindent=23pt\noindent\textbf{Козеренко~Е.\,Б., Кузнецов~И.\,П.} Когнитивно-лингвистические представления в 
систе-\linebreak
\vspace*{-12pt}\\
\hspace*{23pt}мах обработки текстов$\dotfill$&3&69\\
\textbf{Князь~В.\,А.} см.~Каратеев~С.\,Л.&&\\
\hangindent=23pt\noindent\textbf{Колесников~А.\,В., Солдатов~С.\,А.} Алгоритм координации для гибридной 
интеллектуальной системы решения сложной задачи оперативно-производственного\linebreak
\vspace*{-12pt}\\
\hspace*{23pt}планирования$\dotfill$&4&61\\
\hangindent=23pt\noindent\textbf{Коновалов~М.\,Г.} О планировании потоков в системах вычислительных 
ресурсов$\dotfill$&2&3\\
\textbf{Конушин~А.\,С.} см.~Конушин~В.\,С.&&\\
\hangindent=23pt\noindent\textbf{Конушин~В.\,С., Кривовязь~Г.\,Р., Конушин~А.\,С.} Алгоритм распознавания людей 
в видео-\linebreak
\vspace*{-12pt}\\
\hspace*{23pt}последовательности по одежде$\dotfill$&1&74\\
\textbf{Корепанов~Э.\, Р.} см.~Синицын~И.\,Н.&&\\
\textbf{Королев~В.\,Ю.} см.~Соколов~И.\,А.&&\\
\textbf{Королев~Р.\,А.} см.~Бенинг~В.\,Е.&&\\
\textbf{Коротышева~А.\,В.} см.~Зейфман~А.\,И.&&\\
\hangindent=23pt\noindent\textbf{Кривенко~М.\,П.} Непараметрическое оценивание элементов байесовского 
клас\-си-\linebreak
\vspace*{-12pt}\\
\hspace*{23pt}фикатора$\dotfill$&2&13\\
\textbf{Кривовязь~Г.\,Р.} см.~Конушин~В.\,С.&&\\
\textbf{Крылов~А.\,С.} см.~Павельева~Е.\,А.&&\\
\hangindent=23pt\noindent\textbf{Крылов~В.\,А.} Моделирование и классификация многоканальных дистанционных\linebreak
\vspace*{-12pt}\\
\hspace*{23pt}изображений с использованием копул$\dotfill$&4&34\\
\hangindent=23pt\noindent\textbf{Крючин~О.\,В.} Разработка параллельных эвристических алгоритмов подбора 
весовых\linebreak
\vspace*{-12pt}\\
\hspace*{23pt}коэффициентов искусственной нейтронной сети$\dotfill$&2&53\\
\hangindent=23pt\noindent\textbf{Кудрявцев~А.\,А., Шоргин~С.\,Я.} Байесовские модели массового обслуживания и 
надеж-\linebreak
\vspace*{-12pt}\\
\hspace*{23pt}ности: характеристики среднего числа заявок в системе $M\vert M \vert 1\vert 
\infty$$\dotfill$&3&16\\
\hangindent=23pt\noindent\textbf{Кузнецов~А.\,А.} Связь между временными и структурно-топологическими 
характери-\linebreak
\vspace*{-12pt}\\
\hspace*{23pt}стиками диаграмм ритма сердца здоровых людей$\dotfill$&4&39\\
\textbf{Кузнецов~И.\,П.} см.~Козеренко~Е.\,Б.&&\\
\textbf{Ле~Пезан~Д.} см.~Бунтман~Н.\,В.&&\\
\hangindent=23pt\noindent\textbf{Лукьяненко~А.\,С., Морозов~Е.\,В., Гуртов~А.\,В.} Анализ сетевого протокола с общей 
функ-\linebreak
\vspace*{-12pt}\\
\hspace*{23pt}цией расширения окна передачи сообщения при конфликтах$\dotfill$&2&46\\
\hangindent=23pt\noindent\textbf{Лямин~О.\,О.} О предельном поведении мощностей критериев в случае обобщенного\linebreak
\vspace*{-12pt}\\
\hspace*{23pt}распределения Лапласа$\dotfill$&3&47\\
\hangindent=23pt\noindent\textbf{Маркин~А.\,В., Шестаков~О.\,В.} Асимптотики оценки риска при пороговой 
обработке\linebreak
\vspace*{-12pt}\\
\hspace*{23pt}вейвлет-вейглет коэффициентов в задаче томографии$\dotfill$&2&36\\
\hangindent=23pt\noindent\textbf{Матвеева~С.\,С., Захарова~Т.\,В.} Сети массового обслуживания с наименьшей 
длиной\linebreak
\vspace*{-12pt}\\
\hspace*{23pt}очереди$\dotfill$&3&22\\
\hangindent=23pt\noindent\textbf{Матюшенко~С.\,И.} Стационарные характеристики двухканальной системы 
обслужива-\linebreak
\vspace*{-12pt}\\
\hspace*{23pt}ния с переупорядочиванием заявок и распределениями фазового типа$\dotfill$&4&68\\
\textbf{Минель~Ж.-Л.} см.~Бунтман~Н.\,В.&&\\
\textbf{Морозов~Е.\,В.} см.~Бородина~А.\,В.&&\\
\textbf{Морозов~Е.\,В.} см.~Лукьяненко~А.\,С.&&\\
\textbf{Ососков~М.\,В.} см.~Каратеев~С.\,Л.&&\\
\hangindent=23pt\noindent\textbf{Павельева~Е.\,А., Крылов~А.\,С.} Поиск и анализ ключевых точек радужной 
оболочки\linebreak
\vspace*{-12pt}\\
\hspace*{23pt}глаза методом преобразования Эрмита$\dotfill$&1&79\\
\textbf{Печинкин~А.\,В.} см.~Френкель~С.\,Л.,&&\\
\hangindent=23pt\noindent\textbf{Протасов~В.\,И.} Составление субъективного портрета с использованием 
эволюционно-\linebreak
\vspace*{-12pt}\\
\hspace*{23pt}го морфинга и квалиметрия метода$\dotfill$&1&83\\
\hangindent=23pt\noindent\textbf{Рудаков~К.\,В., Торшин~И.\,Ю.} Вопросы разрешимости задачи распознавания 
вторичной\linebreak
\vspace*{-12pt}\\
\hspace*{23pt}структуры белка$\dotfill$&2&25\\
\textbf{Сатин~Я.\,А.} см.~Зейфман~А.\,И.&&\\
\hangindent=23pt\noindent\textbf{Сейфуль-Мулюков~Р.\,Б.} Нефть как носитель информации о своем 
происхождении,\linebreak
\vspace*{-12pt}\\
\hspace*{23pt}структуре и эволюции$\dotfill$&1&41\\
\end{tabular}
}

{\tabcolsep=3pt
\begin{tabular}{p{388pt}rr}
&\textbf{Выпуск} & \textbf{Стр.}\\[6pt]
\textbf{Семендяев~Н.\,Н.} см.~Синицын~И.\,Н.&&\\
\textbf{Серебряков~В.\,А.} см.~Захаров~А.\,А.&&\\
\textbf{Синицын~В.\,И.} см.~Синицын~И.\,Н.&&\\
\hangindent=23pt\noindent\textbf{Синицын~И.\,Н., Синицын~В.\,И., Корепанов~Э.\, Р., Белоусов~В.\,В., 
Семендяев~Н.\,Н.} Оперативное построение информационных моделей движения полюса 
Земли\linebreak
\vspace*{-12pt}\\
\hspace*{23pt}методами линейных и линеаризованных фильтров$\dotfill$&1&2\\
\textbf{Сипина~А.\,В.} см.~Бенинг~В.\,Е.&&\\
\hangindent=23pt\noindent\textbf{Соколов~И.\,А.} О работах заслуженного деятеля науки Российской Федерации 
И.\,Н.~Синицына в области информационных технологий и автоматизации (к 70-летию\linebreak
\vspace*{-12pt}\\
\hspace*{23pt}со дня рождения)$\dotfill$&3&84\\
\textbf{Соколов~И.\,А.} см.~Илюшин~Г.\,Я.&&\\
\hangindent=23pt\noindent\textbf{Соколов~И.\,А., Королев~В.\,Ю.} Предисловие$\dotfill$&2&2\\
\textbf{Солдатов~С.\,А.} см.~Колесников~А.\,В.&&\\
\hangindent=23pt\noindent\textbf{Степанов~С.\,Ю.} Использование координатного метода фрагментации 
коммутаторной\linebreak
\vspace*{-12pt}\\
\hspace*{23pt}нейронной сети для сокращения трафика$\dotfill$&2&57\\
\textbf{Тимонина~Е.\,Е.} см.~Грушо~А.\,А.&&\\
\textbf{Торшин~И.\,Ю.} см.~Рудаков~К.\,В.&&\\
\textbf{Ульянов~В.\,В.} см.~Кавагучи~Ю.&&\\
\textbf{Фазекаш~И.} см.~Чупрунов~А.\,Н.&&\\
\textbf{Френкель~С.\,Л.} см.~Баранов~С.\,И.&&\\
\hangindent=23pt\noindent\textbf{Френкель~С.\,Л., Печинкин~А.\,В.} Оценка времени самовосстановления в 
цифровых\linebreak
\vspace*{-12pt}\\
\hspace*{23pt}системах после сбоев, вызываемых переходными помехами$\dotfill$&3&2\\
\textbf{Фуджикоши~Я.} см.~Кавагучи~Ю.&&\\
\hangindent=23pt\noindent\textbf{Цискаридзе~А.\,К.} Математическая модель и метод восстановления позы человека 
по\linebreak
\vspace*{-12pt}\\
\hspace*{23pt}стереопаре силуэтных изображений$\dotfill$&4&27\\
\hangindent=23pt\noindent\textbf{Чупраков~К.\,Г.} К вопросу о размещении коллективных средств отображения в 
ситуа-\linebreak
\vspace*{-12pt}\\
\hspace*{23pt}ционном зале с заданными параметрами$\dotfill$&4&89\\
\textbf{Чупраков~К.\,Г.} см.~Зацаринный~А.\,А.&&\\
\hangindent=23pt\noindent\textbf{Чупрунов~А.\,Н., Фазекаш~И.} Законы повторного логарифма для числа 
безошибочных\linebreak
\vspace*{-12pt}\\
\hspace*{23pt}блоков при помехоустойчивом кодировании$\dotfill$&3&42\\
\textbf{Шевцова~И.\,Г.} см.~Григорьева~М.\,Е.&&\\
\hangindent=23pt\noindent\textbf{Шестаков~О.\,В.} Аппроксимация распределения оценки риска пороговой 
обработки вейвлет-коэффициентов нормальным распределением при использовании 
выбо-\linebreak
\vspace*{-12pt}\\
\hspace*{23pt}рочной дисперсии$\dotfill$&4&73\\
\textbf{Шестаков~О.\,В.} см.~Маркин~А.\,В.&&\\
\textbf{Шоргин~С.\,Я.} см.~Зейфман~А.\,И.&&\\
\textbf{Шоргин~С.\,Я.} см.~Кудрявцев~А.\,А.&&\\
\end{tabular}
}

%\thispagestyle{myheadings}
\def\leftfootline{\small{\textbf{\thepage}
\hfill ИНФОРМАТИКА И ЕЁ ПРИМЕНЕНИЯ\ \ \ том~4\ \ \ выпуск~4\ \ \ 2010}
}%
 \def\rightfootline{\small{ИНФОРМАТИКА И ЕЁ ПРИМЕНЕНИЯ\ \ \ том~4\ \ \ выпуск~4\ \ \ 2010
 \hfill \textbf{\thepage}}}
 \label{end\stat}
%
%Том 10 Выпуск 1-4 Год 2016

\def\stat{cont-e}
{%\hrule\par
%\vskip 7pt % 7pt
\raggedleft\Large \bf%\baselineskip=3.2ex
2\,0\,1\,6\ \ A\,U\,T\,H\,O\,R\ \ I\,N\,D\,E\,X \vskip 17pt
 \hrule
 \par
\vskip 21pt plus 6pt minus 3pt }

\label{st\stat}

\def\tit{\ }

\def\aut{\ }
\def\auf{\ }

\def\leftkol{\ } %2016 AUTHOR INDEX} % ENGLISH ABSTRACTS}

\def\rightkol{\ } %2016 AUTHOR INDEX} %ENGLISH ABSTRACTS}

\titele{\tit}{\aut}{\auf}{\leftkol}{\rightkol}

\def\leftfootline{\small{\textbf{\thepage}
\hfill INFORMATIKA I EE PRIMENENIYA~--- INFORMATICS AND APPLICATIONS\ \ \ 2016\
\ \ volume~10\ \ \ issue\ 4}
}%
 \def\rightfootline{\small{INFORMATIKA I EE PRIMENENIYA~--- INFORMATICS AND APPLICATIONS\ \ \ 2016\ \ \ volume~10\ \ \ issue\ 4
\hfill \textbf{\thepage}}}

\vspace*{-12pt}
\vspace*{-18pt}

{\tabcolsep=2.8pt
\begin{tabular}{p{382pt}cc}
&\textbf{Issue} & \textbf{Page}\\[6pt]
\Avtors{Agalarov~M.\,Ya.} see~Agalarov~Ya.\,M.&&\\
\Avtors{Agalarov~Ya.\,M., Agalarov~M.\,Ya., and
Shorgin~V.\,S.} About the optimal threshold of queue\linebreak
\\[-12pt]
\hspace*{23pt}length in a~particular problem of profit maximization
in the $M/G/1$ queuing system&2&70--79\\
\Avtors{Alexeyevsky~D.\,A.} BioNLP ontology extraction from 
a~restricted language corpus with\linebreak
\\[-12pt]
\hspace*{23pt}context-free grammars&1&119--128\\
\Avtors{Andreev~S.\,D.} see~Gaidamaka~Yu.\,V.&&\\
\Avtors{Andreev~S.\,D.} see~Ometov~A.\,Ya.&&\\
\Avtors{Arkhipov~O.\,P., Arkhipov~P.\,O., and Sidorkin~I.\,I.} The
option to create a~local coordinate\linebreak
\\[-12pt]
\hspace*{23pt}system for synchronization of selected images&3&91--97\\
\Avtors{Arkhipov~P.\,O.} see~Arkhipov~O.\,P.&&\\
\Avtors{Belousov~V.\,V.} see~Shnurkov~P.\,V.&&\\
\Avtors{Belousov~V.\,V.} see~Shnurkov~P.\,V.&&\\
\Avtors{Bening~V.\,E.} Calculation of~the~asymptotic deficiency
of~some statistical procedures based\linebreak
\\[-12pt]
\hspace*{23pt}on~samples with~random sizes&4&34--45\\
\Avtors{Borisov~A.\,V., Bosov~A.\,V., and Miller~G.\,B.} Modeling and
monitoring of VoIP connection&2&\hphantom{1}2--13\\
\Avtors{Bosov~A.\,V.} see~Borisov~A.\,V.&&\\
\Avtors{Briukhov~D.\,O.} see~Stupnikov~S.\,A.&&\\
\Avtors{Callaos~N.\,K.\ and Seyful-Mulyukov~R.\,B.} Complexity and
its information content&1&129--139\\
\Avtors{Chertok~A.\,V., Kadaner~A.\,I., Khazeeva~G.\,T., and
Sokolov~I.\,A.} Regime switching detection\linebreak
\\[-12pt]
\hspace*{23pt}for~the~Levy driven
Ornstein--Uhlenbeck process using CUSUM methods&4&46--56\\
\Avtors{Chichagov~V.\,V.} Asymptotic expansions of mean absolute
error of uniformly minimum variance unbiased and maximum likelihood
estimators on the one-parameter exponential\linebreak
\\[-12pt]
\hspace*{23pt}family model of lattice distributions&3&66--76\\
\Avtors{Danishevsky~V.\,I.} see~Kolesnikov A.\,V.&&\\
\Avtors{Fazliev~A.\,Z.} see~Kalinichenko~L.\,A.&&\\
\Avtors{Fedoseev~A.\,A.} What is behind the concept of ``knowledge in
small packages''&3&105--110\\
\Avtors{Gaidamaka~Yu.\,V., Andreev~S.\,D., Sopin~E.\,S.,
Samouylov~K.\,E., and Shorgin~S.\,Ya.} Interference analysis
of~the~device-to-device communications model with~regard to~a~signal\linebreak
\\[-12pt]
\hspace*{23pt}propagation environment&4&\hphantom{1}2--10\\
\Avtors{Gasilov~A.\,V.} see~Yakovlev~O.\,A.&&\\
\Avtors{Goncharov~A.\,V.\ and Strijov~V.\,V.} Metric time series
classification using weighted dynamic\linebreak
\\[-12pt]
\hspace*{23pt}warping relative to centroids of classes&2&36--47\\
\Avtors{Gordov~E.\,P.} see~Kalinichenko~L.\,A.&&\\
\Avtors{Gorshenin~A.\,K.} Concept of online service for stochastic
modeling of real processes&1&72--81\\
\Avtors{Gorshenin~A.\,K.} see~Shnurkov~P.\,V.&&\\
\Avtors{Gorshenin~A.\,K.} see~Shnurkov~P.\,V.&&\\
\Avtors{Grusho~A.\,A., Grusho~N.\,A., Zabezhailo~M.\,I., and
Timonina~E.\,E.} Integration of statistical and\linebreak
\\[-12pt]
\hspace*{23pt}deterministic methods for
analysis of information security&3&2--8\\
\Avtors{Grusho~A.\,A., Zabezhailo~M.\,I., and Zatsarinny~A.\,A.} On
the advanced procedure to reduce\linebreak
\\[-12pt]
\hspace*{23pt}calculation of Galois closures&4&\hphantom{1}96--104\\
\Avtors{Grusho~N.\,A.} see~Grusho~A.\,A.&&\\
\Avtors{Havanskov~V.\,A.} see~Minin~V.\,A.&&\\
\Avtors{Inkova~O.\,Yu.} see~Zatsman~I.\,M.&&\\
\Avtors{Isachenko~R.\,V.\ and Strijov~V.\,V.} Metric learning in
multiclass time series classification\linebreak
\\[-12pt]
\hspace*{23pt}problem&2&48--57\\
\end{tabular}
}
\pagebreak

\def\leftfootline{\small{\textbf{\thepage}
\hfill INFORMATIKA I EE PRIMENENIYA~--- INFORMATICS AND APPLICATIONS\ \ \ 2016\
\ \ volume~10\ \ \ issue\ 4}
}%
 \def\rightfootline{\small{INFORMATIKA I EE PRIMENENIYA~---
INFORMATICS AND APPLICATIONS\ \ \ 2016\ \ \ volume~10\ \ \ issue\ 4
\hfill \textbf{\thepage}}}

\def\leftkol{2016 AUTHOR INDEX} % ENGLISH ABSTRACTS}

\def\rightkol{2016 AUTHOR INDEX} %ENGLISH ABSTRACTS}


{\tabcolsep=2.83pt
\begin{tabular}{p{382pt}cc}
&\textbf{Issue} & \textbf{Page}\\[6pt]
\Avtors{Kadaner~A.\,I.} see~Chertok~A.\,V.&&\\[.255pt]
\Avtors{Kalinichenko~L.\,A., Volnova~A.\,A., Gordov~E.\,P.,
Kiselyova~N.\,N., Kovaleva~D.\,A., Malkov~O.\,Yu., Okladnikov~I.\,G.,
Podkolodnyy~N.\,L., Pozanenko~A.\,S., Ponomareva~N.\,V.,
Stupnikov~S.\,A.,} \textbf{and Fazliev~A.\,Z.} Data access challenges for data
intensive\linebreak
\\[-12pt]
\hspace*{23pt}research in Russia&1& 2--22\\[.255pt]
\Avtors{Karasikov~M.\,E.\ and Strijov~V.\,V.} Feature-based
time-series classification&4&121--131\\[.255pt]
\Avtors{Khazeeva~G.\,T.} see~Chertok~A.\,V.&&\\[.255pt]
\Avtors{Khokhlov~Yu.\,S.} Multivariate fractional Levy motion and its
applications&2&\hphantom{1}98--106\\[.255pt]
\Avtors{Kirikov~I.\,A., Kolesnikov~A.\,V., Listopad~S.\,V., and
Rumovskaya~S.\,B.} Fine-grained hybrid\linebreak
\\[-12pt]
\hspace*{23pt}intelligent systems. Part 2:
Bidirectional hybridization&1&\hphantom{1}96--105\\[.255pt]
\Avtors{Kirikov~I.\,A., Kolesnikov~A.\,V., Listopad~S.\,V., and
Rumovskaya~S.\,B.} ``Virtual council''~---\linebreak
\\[-12pt]
\hspace*{23pt}source environment
supporting complex diagnostic decision making&3&81--90\\[.255pt]
\Avtors{Kiselyova~N.\,N.} see~Kalinichenko~L.\,A.&&\\[.255pt]
\Avtors{Kolesnikov A.\,V., Listopad~S.\,V., Rumovskaya~S.\,B., and
Danishevsky~V.\,I.} Informal axiomatic\linebreak
\\[-12pt]
\hspace*{23pt}theory of~the~role visual models&4&114--120\\[.255pt]
\Avtors{Kolesnikov~A.\,V.} see~Kirikov~I.\,A.&&\\[.255pt]
\Avtors{Kolesnikov~A.\,V.} see~Kirikov~I.\,A.&&\\[.255pt]
\Avtors{Kolin~K.\,K.} Humanitarian aspects of information
security&3&111--121\\[.255pt]
\Avtors{Konovalov~M.\,G.\ and Razumchik~R.\,V.} Dispatching
to~two parallel nonobservable queues using\linebreak
\\[-12pt]
\hspace*{23pt}only static
information&4&57--67\\[.255pt]
\Avtors{Korchagin~A.\,Yu.} see~Korolev~V.\,Yu.&&\\[.255pt]
\Avtors{Korchagin~A.\,Yu.} see~Korolev~V.\,Yu.&&\\[.255pt]
\Avtors{Korepanov~E.\,R.} see~Sinitsyn~I.\,N.&&\\[.255pt]
\Avtors{Korepanov~E.\,R.} see~Sinitsyn~I.\,N.&&\\[.255pt]
\Avtors{Korolev~V.\,Yu., Korchagin~A.\,Yu., and Zeifman~A.\,I.} The
Poisson theorem for Bernoulli trials\linebreak
\\[-12pt]
\hspace*{23pt}with~a~random probability
of~success and~a~discrete analog of~the~Weibull distribution&4&11--20\\[.255pt]
\Avtors{Korolev~V.\,Yu., Zeifman~A.\,I., and Korchagin~A.\,Yu.}
Asymmetric Linnik distributions as~limit\linebreak
\\[-12pt]
\hspace*{23pt}laws for~random sums
of~independent random variables with~finite variances&4&21--33\\[.255pt]
\Avtors{Koucheryavy~E.\,A.} see~Ometov~A.\,Ya.&&\\[.255pt]
\Avtors{Kovaleva~D.\,A.} see~Kalinichenko~L.\,A.&&\\[.255pt]
\Avtors{Kovalyov~S.\,P.} Metaprogramming to increase
manufacturability of large-scale software-\linebreak
\\[-12pt]
\hspace*{23pt}intensive systems&1&56--66\\[.255pt]
\Avtors{Krivenko~M.\,P.} Significance tests of feature selection for
classification&3&32--40\\[.255pt]
\Avtors{Kruzhkov~M.\,G.} see~Zalizniak~Anna~A.&&\\[.255pt]
\Avtors{Kruzhkov~M.\,G.} see~Zatsman~I.\,M.&&\\[.255pt]
\Avtors{Kudryavtsev~A.\,A.} Bayesian queueing and reliability models:
\textit{A~priori} distributions with\linebreak
\\[-12pt]
\hspace*{23pt}compact support&1&67--71\\[.255pt]
\Avtors{Kudryavtsev~A.\,A.} Characteristics dependent on the balance
coefficient in Bayesian models\linebreak
\\[-12pt]
\hspace*{23pt}with compact support of \textit{a priori}
distributions&3&77--80\\[.255pt]
\Avtors{Kudryavtsev~A.\,A.\ and Palionnaia~S.\,I.} Bayesian recurrent
model of reliability growth:\linebreak
\\[-12pt]
\hspace*{23pt}Parabolic distribution of parameters&2&80--83\\[.255pt]
\Avtors{Kudryavtsev~A.\,A.\ and Titova~A.\,I.} Bayesian queuing
and~reliability models: Degenerate-\linebreak
\\[-12pt]
\hspace*{23pt}Weibull case&4&68--71\\[.255pt]
\Avtors{Leontyev~N.\,D.\ and Ushakov~V.\,G.} Analysis of a queueing
system with autoregressive arrivals\linebreak
\\[-12pt]
\hspace*{23pt}and nonpreemptive priority&3&15--22\\[.255pt]
\Avtors{Listopad~S.\,V.} see~Kirikov~I.\,A.&&\\[.255pt]
\Avtors{Listopad~S.\,V.} see~Kirikov~I.\,A.&&\\[.255pt]
\Avtors{Listopad~S.\,V.} see~Kolesnikov A.\,V.&&\\[.255pt]
\Avtors{Malkov~O.\,Yu.} see~Kalinichenko~L.\,A.&&\\[.255pt]
\Avtors{Markov~A.\,S., Monakhov~M.\,M., and
Ulyanov~V.\,V.} Generalized Cornish--Fisher expansions\linebreak
\\[-12pt]
\hspace*{23pt}for distributions of statistics based on samples
of random size&2&84--91\\[.255pt]
\Avtors{Melnikov~A.\,K.\ and Ronzhin~A.\,F.} Generalized statistical
method of~text analysis based\linebreak
\\[-12pt]
\hspace*{23pt}on~calculation of~probability distributions
of~statistical values&4&89--95\\
\end{tabular}
}
\pagebreak

\def\leftfootline{\small{\textbf{\thepage}
\hfill INFORMATIKA I EE PRIMENENIYA~--- INFORMATICS AND APPLICATIONS\ \ \ 2016\
\ \ volume~10\ \ \ issue\ 4}
}%
 \def\rightfootline{\small{INFORMATIKA I EE PRIMENENIYA~---
INFORMATICS AND APPLICATIONS\ \ \ 2016\ \ \ volume~10\ \ \ issue\ 4
\hfill \textbf{\thepage}}}

\def\leftkol{2016 AUTHOR INDEX} % ENGLISH ABSTRACTS}

\def\rightkol{2016 AUTHOR INDEX} %ENGLISH ABSTRACTS}


{\tabcolsep=3pt
\begin{tabular}{p{381pt}cc}
&\textbf{Issue} & \textbf{Page}\\[6pt]
\Avtors{Meykhanadzhyan~L.\,A.} Stationary characteristics of the finite
capacity queueing system with\linebreak
\\[-12pt]
\hspace*{23pt}inverse service order and generalized
probabilistic priority&2&123--131\\[.23pt]
\Avtors{Miller~G.\,B.} see~Borisov~A.\,V.&&\\[.23pt]
\Avtors{Minin~V.\,A., Zatsman~I.\,M., Havanskov~V.\,A., and
Shubnikov~S.\,K.} Intensity of citation of scientific publications in
inventions on information and computer technologies patented\linebreak
\\[-12pt]
\hspace*{23pt}in Russia by domestic and foreign applicants&2&107--122\\[.23pt]
\Avtors{Monakhov~M.\,M.} see~Markov~A.\,S.&&\\[.23pt]
\Avtors{Naumov~V.\,A.\ and Samouylov~K.\,E.} On relationship
between queuing systems with resources\linebreak
\\[-12pt]
\hspace*{23pt}and Erlang networks&3&\hphantom{1}9--14\\[.23pt]
\Avtors{Okladnikov~I.\,G.} see~Kalinichenko~L.\,A.&&\\[.23pt]
\Avtors{Ometov~A.\,Ya., Andreev~S.\,D., Turlikov~A.\,M., and
Koucheryavy~E.\,A.} Performance analysis of\linebreak
\\[-12pt]
\hspace*{23pt}a wireless data
aggregation system with contention for contemporary sensor
networks&3&23--31\\[.23pt]
\Avtors{Palionnaia~S.\,I.} see~Kudryavtsev~A.\,A.&&\\[.23pt]
\Avtors{Podkolodnyy~N.\,L.} see~Kalinichenko~L.\,A.&&\\[.23pt]
\Avtors{Ponomareva~N.\,V.} see~Kalinichenko~L.\,A.&&\\[.23pt]
\Avtors{Popkova~N.\,A.} see~Zatsman~I.\,M.&&\\[.23pt]
\Avtors{Pozanenko~A.\,S.} see~Kalinichenko~L.\,A.&&\\[.23pt]
\Avtors{Razumchik~R.\,V.} see~Konovalov~M.\,G.&&\\[.23pt]
\Avtors{Ronzhin~A.\,F.} see~Melnikov~A.\,K.&&\\[.23pt]
\Avtors{Rumovskaya~S.\,B.} see~Kirikov~I.\,A.&&\\[.23pt]
\Avtors{Rumovskaya~S.\,B.} see~Kirikov~I.\,A.&&\\[.23pt]
\Avtors{Rumovskaya~S.\,B.} see~Kolesnikov A.\,V.&&\\[.23pt]
\Avtors{Samouylov~K.\,E.} see~Gaidamaka~Yu.\,V.&&\\[.23pt]
\Avtors{Samouylov~K.\,E.} see~Naumov~V.\,A.&&\\[.23pt]
\Avtors{Serebryanskii~S.\,M.} see~Tyrsin~A.\,N.&&\\[.23pt]
\Avtors{Seyful-Mulyukov~R.\,B.} see~Callaos~N.\,K.&&\\[.23pt]
\Avtors{Shestakov~O.\,V.} Statistical properties of the denoising method
based on the stabilized hard\linebreak
\\[-12pt]
\hspace*{23pt}thresholding&2&65--69\\[.23pt]
\Avtors{Shestakov~O.\,V.} The strong law of large numbers for the risk
estimate in the problem of\linebreak
\\[-12pt]
\hspace*{23pt}tomographic image reconstruction from
projections with a correlated noise&3&41--45\\[.23pt]
\Avtors{Shestakov~O.\,V.} see~Zakharova~T.\,V.&&\\[.23pt]
\Avtors{Shnurkov~P.\,V., Gorshenin~A.\,K., and Belousov~V.\,V.}
Analytical solution of~the~optimal control\linebreak
\\[-12pt]
\hspace*{23pt}task of~a~semi-Markov
process with~finite set of~states&4&72--88\\[.23pt]
\Avtors{Shnurkov~P.\,V., Zasypko~V.\,V., Belousov~V.\,V., and
Gorshenin~A.\,K.} Development of the algorithm of numerical solution
of the optimal investment control problem\linebreak
\\[-12pt]
\hspace*{23pt}in the closed dynamical model of three-sector economy&1&82--95\\[.23pt]
\Avtors{Shorgin~S.\,Ya.} see~Gaidamaka~Yu.\,V.&&\\[.23pt]
\Avtors{Shorgin~V.\,S.} see~Agalarov~Ya.\,M.&&\\[.23pt]
\Avtors{Shubnikov~S.\,K.} see~Minin~V.\,A.&&\\[.23pt]
\Avtors{Sidorkin~I.\,I.} see~Arkhipov~O.\,P.&&\\[.23pt]
\Avtors{Sinitsyn~I.\,N.} Analytical modeling of processes in stochastic
systems with complex fractional\linebreak
\\[-12pt]
\hspace*{23pt}order Bessel nonlinearities&3&55--65\\[.23pt]
\Avtors{Sinitsyn~I.\,N.} Orthogonal supoptimal filters for nonlinear
stochastic systems on manifolds&1&34--44\\[.23pt]
\Avtors{Sinitsyn~I.\,N.\ and Korepanov~E.\,R.} Normal Pugachev
conditionally-optimal filters and extra-\linebreak
\\[-12pt]
\hspace*{23pt}polators for state linear stochastic systems&2&14--23\\[.23pt]
\Avtors{Sinitsyn~I.\,N.\ and Sinitsyn~V.\,I.} Analytical modeling of
distributions in stochastic systems on\linebreak
\\[-12pt]
\hspace*{23pt}manifolds based on ellipsoidal approximation&1&45--55\\[.23pt]
\Avtors{Sinitsyn~I.\,N., Sinitsyn~V.\,I., and
Korepanov~E.\,R.} Ellipsoidal suboptimal filters for nonlinear\linebreak
\\[-12pt]
\hspace*{23pt}stochastic systems on manifolds&2&24--35\\[.23pt]
\Avtors{Sinitsyn~V.\,I.} see~Sinitsyn~I.\,N.&&\\[.23pt]
\Avtors{Sinitsyn~V.\,I.} see~Sinitsyn~I.\,N.&&\\[.23pt]
\Avtors{Skvortsov~N.\,A.} see~Stupnikov~S.\,A.&&\\[.23pt]
\Avtors{Sokolov~I.\,A.} see~Chertok~A.\,V.&&\\
\end{tabular}
}
\pagebreak

\def\leftfootline{\small{\textbf{\thepage}
\hfill INFORMATIKA I EE PRIMENENIYA~--- INFORMATICS AND APPLICATIONS\ \ \ 2016\
\ \ volume~10\ \ \ issue\ 4}
}%
 \def\rightfootline{\small{INFORMATIKA I EE PRIMENENIYA~---
INFORMATICS AND APPLICATIONS\ \ \ 2016\ \ \ volume~10\ \ \ issue\ 4
\hfill \textbf{\thepage}}}

\def\leftkol{2016 AUTHOR INDEX} % ENGLISH ABSTRACTS}

\def\rightkol{2016 AUTHOR INDEX} %ENGLISH ABSTRACTS}


{\tabcolsep=3pt
\begin{tabular}{p{382pt}cc}
&\textbf{Issue} & \textbf{Page}\\[6pt]
\Avtors{Sopin~E.\,S.} see~Gaidamaka~Yu.\,V.&&\\
\Avtors{Strijov~V.\,V.} see~Goncharov~A.\,V.&&\\
\Avtors{Strijov~V.\,V.} see~Isachenko~R.\,V.&&\\
\Avtors{Strijov~V.\,V.} see~Karasikov~M.\,E.&&\\
\Avtors{Stupnikov~S.\,A., Briukhov~D.\,O., and Skvortsov~N.\,A.}
Co-lending systemic risk analysis over\linebreak
\\[-12pt]
\hspace*{23pt}heterogeneous data collections&1&23--33\\
\Avtors{Stupnikov~S.\,A.} see~Kalinichenko~L.\,A.&&\\
\Avtors{Suchkov~A.\,P.} see~Zatsarinny~A.\,A.&&\\
\Avtors{Timonina~E.\,E.} see~Grusho~A.\,A.&&\\
\Avtors{Titova~A.\,I.} see~Kudryavtsev~A.\,A.&&\\
\Avtors{Turlikov~A.\,M.} see~Ometov~A.\,Ya.&&\\
\Avtors{Tyrsin~A.\,N.\ and Serebryanskii~S.\,M.} Recognition of
dependences on the basis of inverse\linebreak
\\[-12pt]
\hspace*{23pt}mapping&2&58--64\\
\Avtors{Ulyanov~V.\,V.} see~Markov~A.\,S.&&\\
\Avtors{Ushakov~V.\,G.} Queueing system with working vacations and
hyperexponential input stream&2&92--97\\
\Avtors{Ushakov~V.\,G.} see~Leontyev~N.\,D.&&\\
\Avtors{Volnova~A.\,A.} see~Kalinichenko~L.\,A.&&\\
\Avtors{Yakovlev~O.\,A.\ and Gasilov~A.\,V.} Speeded-up stereo
matching using geodesic support weights&3&\hphantom{1}98--104\\
\Avtors{Zabezhailo~M.\,I.} see~Grusho~A.\,A.&&\\
\Avtors{Zabezhailo~M.\,I.} see~Grusho~A.\,A.&&\\
\Avtors{Zakharova~T.\,V.\ and Shestakov~O.\,V.} Precision analysis of
wavelet processing of aerodynamic\linebreak
\\[-12pt]
\hspace*{23pt}flow patterns&3&46--54\\
\Avtors{Zalizniak~Anna~A.\ and Kruzhkov~M.\,G.} Database
of~Russian impersonal verbal constructions&4&132--141\\
\Avtors{Zasypko~V.\,V.} see~Shnurkov~P.\,V.&&\\
\Avtors{Zatsarinny~A.\,A.\ and Suchkov~A.\,P.} Systems engineering
approaches to~the~establishment of\linebreak
\\[-12pt]
\hspace*{23pt}a~system for~decision support based
on~situational analysis&4&105--113\\
\Avtors{Zatsarinny~A.\,A.} see~Grusho~A.\,A.&&\\
\Avtors{Zatsman~I.\,M., Inkova~O.\,Yu., Kruzhkov~M.\,G., and
Popkova~N.\,A.} Representation of cross-\linebreak
\\[-12pt]
\hspace*{23pt}lingual knowledge about
connectors in supracorpora databases&1&106--118\\
\Avtors{Zatsman~I.\,M.} see~Minin~V.\,A.&&\\
\Avtors{Zeifman~A.\,I.} see~Korolev~V.\,Yu.&&\\
\Avtors{Zeifman~A.\,I.} see~Korolev~V.\,Yu.&&\\
\end{tabular}
}

%\thispagestyle{myheadings}
\def\leftfootline{\small{\textbf{\thepage}
\hfill INFORMATIKA I EE PRIMENENIYA~--- INFORMATICS AND APPLICATIONS\ \ \ 2016\
\ \ volume~10\ \ \ issue\ 4}
}%
 \def\rightfootline{\small{INFORMATIKA I EE PRIMENENIYA~---
INFORMATICS AND APPLICATIONS\ \ \ 2016\ \ \ volume~10\ \ \ issue\ 4
\hfill \textbf{\thepage}}}

 \label{end\stat}

\newpage

%\def\stat{rekl}
%\label{preobr}

%\def\tit{АКАДЕМИК ПУГАЧЁВ  ВЛАДИМИР СЕМЁНОВИЧ\\
%25.03.1911--25.03.1998}


%   \vspace*{-48pt}
%   \begin{center}\LARGE
%Академик Пугачёв  Владимир Семёнович\\ (25.03.1911--25.03.1998)
%   \end{center}
   
   %\vspace*{2.5mm}
   
   \begin{center}

{\prgsh\LARGE
ОБЪЯВЛЕНИЯ О КОНФЕРЕНЦИЯХ}

\end{center}
%\hrule

\vspace*{6pt}

   
   \vspace*{10mm}
   
   \thispagestyle{empty}

\noindent
\begin{tabular}{cc}
%\begin{center}
\multicolumn{1}{c}{\raisebox{-40pt}[0pt][0pt]{\mbox{%
\epsfxsize=33mm
\epsfbox{vspu.eps}
}}}
%\end{center}
&
\tabcolsep=0pt\begin{tabular}{c}
{\prg{\Large\textbf{XII Всероссийское совещание}}}\\[6pt]
{\prg{\Large\textbf{по проблемам управления}}}\\[12pt]
{\prg{\large 16--19 июня 2014~г.}}\\[6pt] 
{\prg{\large Институт проблем управления имени В.\,А.~Трапезникова РАН}}\\[6pt]
{\prg{\large Москва, Россия}}
\end{tabular}
\end{tabular}

\vspace*{60pt}

     
 { %\large    
 XII Всероссийское совещание по проблемам управления (ВСПУ XII), посвященное 75-летию 
Института проблем управления (ИПУ) имени В.\,А.~Трапезникова РАН, проводится 16--19~июня 
2014~г.\ 
в ИПУ РАН (г.~Москва, Россия). ВСПУ XII организуется ИПУ РАН при поддержке РФФИ, Отделения 
энергетики, машиностроения, механики и процессов управления Российской академии наук, 
Российского 
национального комитета по автоматическому управлению, Академии навигации и управ\-ле\-ния 
движением, 
Научного совета РАН по комплексным проблемам управления и автоматизации, Совета по 
мехатронике и робототехнике РАН. Официальный язык Совещания~--- русский.

\vspace*{24pt}
     
     \textbf{Направления работы}
     \begin{enumerate}[1.]
\item Теория систем управления
\item Управление подвижными объектами и навигация
\item Интеллектуальные системы управления
\item Управление в промышленности, транспортом и логистикой
\item Управление системами междисциплинарной природы
\item Средства измерения, вычислений и контроля в управлении
\item Системный анализ и принятие решений в задачах управления
\item Информационные технологии в управлении
\item Проблемы образования в области управления: современное содержание и технологии обучения
\end{enumerate}

\vspace*{24pt}

     Подробная информация о Совещании находится на сайте {\sf http://vspu2014.ipu.ru}. Срок 
окончательной подачи докладов через систему подачи докладов на сайте~--- \textbf{30~ноября} 
2013~г.
}

%\include{rekl-1}

%\end{document}

%   \vspace*{-48pt}

\begin{center}
\vspace*{6pt}
\mbox{%
\epsfxsize=53.502mm
\epsfbox{foto-1.eps}
}
\end{center}

\vspace*{6pt} %Академик


   \begin{center}
\fbox{\Large\textbf{Профессор Игорь Алексеевич Ушаков}}\\[12pt]
\textbf{\large 22.01.1935--27.02.2015}
   \end{center}


   %\vspace*{2.5mm}

   \vspace*{5mm}

   \thispagestyle{empty}

%\

%\vspace*{-12pt}


Редакционный совет и редакционная коллегия журнала <<Информатика и~её применения>> с~глубоким прискорбием извещают, что 27~февраля 2015~г.\ после тяжелой
и~продолжительной болезни скончался Игорь Алексеевич Ушаков~--- доктор технических наук, профессор, член редколлегии журнала <<Информатика и ее применения>>.

Игорь Алексеевич Ушаков окончил Московский авиационный институт, в~1963~г.\ защитил кандидатскую, а~в~1968~г.~--- докторскую диссертацию. С~1958 по 1989~гг.\ работал в~ряде научно-исследовательских организаций СССР, в~том числе руководил отделами в~НИИ АА и~ВЦ АН СССР; с 1969 по 1989 гг. преподавал в~МФТИ (был профессором, а~затем заведующим кафедрой) и~в~МЭИ. С~1989~г.~---- в~США: являлся профессором университета Дж.\ Вашингтона, университета Дж.\ Мэйсона и~Калифорнийского университета, сотрудником компаний MCI, Qualcomm и Hughes.

И.\,А.~Ушаков с момента основания журнала <<Надежность и~контроль качества>> был заместителем ответственного редактора, а~затем на протяжении многих лет членом редколлегии. В~2006~г.\ основал электронный международный журнал ``Reliability: Theory \& Application'', главным редактором которого оставался до конца жизни.

Учебниками и справочниками по теории надежности, написанными И.\,А.~Ушаковым, пользовались и~пользуются несколько поколений ученых и~специалистов в~разных странах мира.

Игорь Алексеевич всегда уделял огромное внимание работе с~молодежью; более~50 его учеников защитили докторские и~кандидатские диссертации.

И.\,А.~Ушаков вел активную научно-про\-све\-ти\-тель\-скую деятельность. В~частности, он был одним из организаторов и~руководителей Московского кабинета качества и~надежности при Политехническом музее (целью этого Кабинета было оказание консультаций работникам промышленных предприятий и~чтение курсов лекций для инженеров, занимающихся проблемой надежности). Находясь в~США, И.\,А.~Ушаков создал международный ин\-тер\-нет-фо\-рум им.\ Б.\,В.~Гнеденко, объединивший около~400~видных специалистов по приложениям теории вероятностей и~математической статистики, преимущественно в~об\-ласти теории надежности и~анализа риска, из десятков стран мира; коллективным членов этого Форума является и~наш журнал. Цели Форума~--- содействие контактам между специалистами из разных стран, организация обмена профессиональными 
новостями и~информацией (новые публикации, предстоящие события и~др.). Также необходимо отметить большое число на\-уч\-но-по\-пу\-ляр\-ных работ, опубликованных И.\,А.~Ушаковым.

И.\,А.~Ушаков обладал большим личным обаянием, имел широкий круг интересов. Все знавшие И.\,А.~Ушакова всегда будут помнить его как замечательного ученого и~прекрасного человека.

\bigskip

Редакционный совет и редакционная коллегия журнала <<Информатика и~её применения>> 
выражают глубокие соболезнования родным и близким покойного, всем, кто его знал и~работал с~ним.



%\end{document}

%\include{IPPM-25}

\def\stat{cont-rus}
{%\hrule\par
%\vskip 7pt % 7pt
\vspace*{-24pt}
\raggedleft\Large \bf%\baselineskip=3.2ex
Правила подготовки рукописей  для публикации в журнале
<<Информатика~и~её~применения>> \vskip 8pt
    \hrule
    \par
\vskip 14pt plus 6pt minus 3pt }

\label{st\stat}

\def\tit{\ }

\def\aut{\ }
\def\auf{\ }

\def\leftkol{\ }
% Правила подготовки рукописей  для публикации в журнале
%<<Информатика и её применения>>

\def\rightkol{\ }
%Правила подготовки рукописей  для публикации в журнале
%<<Информатика и её применения>>}


\titele{\tit}{\aut}{\auf}{\leftkol}{\rightkol}


\vspace*{-60pt}
{ %\small

Журнал <<Информатика и её применения>>
публикует теоретические, обзорные и дискуссионные статьи,
посвященные научным исследованиям и разработкам в области
информатики и ее приложений.

Журнал издается на русском языке. По специальному решению
редколлегии отдельные статьи могут печататься на английском языке.

Тематика журнала охватывает следующие направления:
\begin{itemize}
\item теоретические основы информатики;\\[-15pt]
      \item
математические методы исследования сложных систем и процессов;\\[-15pt]
           \item
информационные системы и сети;\\[-15pt]
                \item
информационные технологии;\\[-15pt]
                     \item
архитектура и программное обеспечение вычислительных комплексов и сетей.\\[-15pt]
\end{itemize}


\noindent
\begin{enumerate}[1.]
\item В журнале печатаются статьи, содержащие результаты, ранее не опубликованные и
не предназначенные к одновременной публикации в других изданиях.

%Публикация не должна нарушать закон об авторских правах.
Публикация предоставленной автором(ами) рукописи не должна нарушать 
положений глав~69, 70 раздела~VII части~IV Гражданского кодекса, 
которые определяют права на результаты интеллектуальной деятельности 
и~средства индивидуализации, в~том числе авторские права, в~РФ.

Ответственность за нарушение авторских прав, в~случае предъявления претензий к~редакции журнала,  
несут авторы статей.



Направляя рукопись в редакцию, авторы сохраняют свои права на данную
рукопись и при этом передают учредителям и редколлегии журнала неисключительные права на
издание статьи на русском языке 
(или на языке статьи, если он отличен от рус\-ско\-го) и~на перевод ее на английский
язык, а~также на
ее распространение в России и за рубежом. 
Каждый автор должен представить в~редакцию подписанный 
с~его стороны <<Лицензионный договор о~передаче неисключительных прав 
на использование произведения>>, текст которого размещен по адресу 
{\sf http://www.ipiran.ru/publications/licence.doc}. 
Этот договор может быть пред\-став\-лен в~бумажном (в~2-х экз.)\ 
или в~электронном виде (отсканированная копия заполненного и~подписанного документа).




Редколлегия вправе запросить у авторов экспертное заключение о возможности
пуб\-ли\-ка\-ции пред\-став\-лен\-ной статьи в открытой печати.\\[-13.5pt]

\item К статье прилагаются данные автора (авторов) (см.\ п.~8). При наличии нескольких
авторов указывается фамилия автора, ответственного за переписку с редакцией.\\[-13.5pt]

\item Редакция журнала осуществляет экспертизу присланных статей в соответствии с
принятой в журнале процедурой рецензирования.

Возвращение рукописи на доработку не означает ее принятия к печати.

Доработанный вариант с ответом на замечания рецензента необходимо прислать в
редакцию.\\[-13.5pt]

\item Решение редколлегии о публикации статьи или ее отклонении сообщается авторам.

Редколлегия может также направить авторам текст рецензии на их статью. Дискуссия по
поводу отклоненных статей не ведется.\\[-13.5pt]

%\pagebreak

\item Редактура статей высылается авторам для просмотра. Замечания к редактуре должны
быть присланы авторами в кратчайшие сроки.\\[-13.5pt]

\item Рукопись предоставляется в электронном виде в форматах MS WORD (.doc или
.docx) или \LaTeX\  (.tex), дополнительно~--- в формате .pdf, на дискете, лазерном диске
или электронной почтой. Предоставление бумажной рукописи необязательно.\\[-13.5pt]

\item При подготовке рукописи в MS Word рекомендуется использовать следующие
настройки.

Параметры страницы:
формат~--- А4; ориентация~--- книжная; поля (см): внутри~--- 2,5, снаружи~--- 1,5,
сверху~--- 2, снизу~--- 2, от края до нижнего колонтитула~--- 1,3.

Основной текст: стиль~--- <<Обычный>>, шрифт~--- Times New Roman, размер~---
14~пунк\-тов, абзацный отступ~--- 0,5~см, 1,5~интервала, выравнивание~--- по ширине.

\pagebreak

\def\leftkol{Правила подготовки рукописей  для публикации в журнале
<<Информатика и её применения>>}

\def\rightkol{Правила подготовки рукописей  для публикации в журнале
<<Информатика и её применения>>}



Рекомендуемый объем рукописи~--- не свыше 10~страниц указанного формата.
При превышении указанного объема редколлегия вправе потребовать от 
автора сокращения объема рукописи.


Сокращения слов, помимо стандартных, не допускаются. Допускается минимальное
количество аббревиатур.


Все страницы рукописи нумеруются.

Шаблоны оформления представлены в интернете:

\noindent
 {\sf
http://www.ipiran.ru/journal/template\_iiep\_ssi\_2024.zip}\\[-14pt]

\item Статья должна содержать следующую информацию на {\bfseries\textit{русском и
английском языках}}:\\[-16pt]

\begin{itemize}
\item название статьи;\\[-15pt]
\item Ф.И.О.\ авторов, на английском можно только имя и фамилию;\\[-15pt]
\item место работы, с указанием почтового адреса организации и электронного адреса каждого
автора;\\[-15pt]
\item сведения об авторах, в соответствии с форматом, образцы которого
представлены на страницах:



\def\leftfootline{\small{\textbf{\thepage}
\hfill ИНФОРМАТИКА И ЕЁ ПРИМЕНЕНИЯ\ \ \ том\ 18\ \ \ выпуск\ 3\ \ \ 2024}
}%
 \def\rightfootline{\small{ИНФОРМАТИКА И ЕЁ ПРИМЕНЕНИЯ\ \ \ том\ 18\ \ \ выпуск\ 3\ \ \ 2024
\hfill \textbf{\thepage}}}



{\sf http://www.ipiran.ru/journal/issues/2013\_07\_01/authors.asp} и

{\sf http://www.ipiran.ru/journal/issues/2013\_07\_01\_eng/authors.asp};
\item аннотация (не менее 100~слов на каждом из языков). Аннотация~--- это краткое
резюме работы, которое может публиковаться отдельно. Она является основным
источником информации в~ин\-фор\-ма\-ци\-он\-ных системах и базах данных. Английская
аннотация должна быть оригинальной, может не быть дословным переводом русского
текста и должна быть написана хорошим английским языком. В~аннотации не должно
быть ссылок на литературу и, по возможности, формул;\\[-15pt]
\item ключевые слова~--- желательно из принятых в мировой
на\-уч\-но-тех\-ни\-че\-ской литературе тематических тезаурусов. Предложения не
могут быть ключевыми словами;\\[-15pt]
\item источники финансирования работы (ссылки на гранты, проекты,
поддерживающие организации и~т.\,п.).
\end{itemize}



%\pagebreak

\item  Требования к спискам литературы.\\[-14pt]

Ссылки на литературу в тексте статьи нумеруются (в квадратных скобках) и
располагаются в каждом из списков литературы в порядке  первых упоминаний. Если источник имеет DOI и/или EDN,
то их необходимо указывать.

Списки литературы представляются в двух вариантах:\\[-14pt]


\noindent
\begin{enumerate}[(1)]
\item \textbf{Список литературы к русскоязычной части}. Русские и английские
работы~---  на языке и в алфавите оригинала;\\[-14.5pt]
\item  \textbf{References}. Русские работы и работы на других языках~--- в латинской
транслитерации с переводом на английский язык; английские работы и работы на других
языках~--- на языке оригинала.
\end{enumerate}

Необходимо для составления списка ``References'' пользоваться размещенной на сайте
{\sf http://www. translit.net/ru/bgn/} бесплатной программой транслитерации русского
 текста в~латиницу. %, при этом в~за\-клад\-ке <<варианты\ldots>> следует выбратьопцию BGN.

Список литературы ``References'' приводится полностью отдельным блоком, повторяя все
позиции из списка литературы к русскоязычной части, независимо от того, имеются или
нет в нем иностранные источники. Если в списке литературы к русскоязычной части есть
ссылки на иностранные публикации, набранные латиницей, они полностью повторяются в
списке ``References''.

Ниже приведены примеры ссылок на различные виды публикаций в списке ``References''.

\def\leftfootline{\small{\textbf{\thepage}
\hfill ИНФОРМАТИКА И ЕЁ ПРИМЕНЕНИЯ\ \ \ том\ 18\ \ \ выпуск\ 3\ \ \ 2024}
}%
 \def\rightfootline{\small{ИНФОРМАТИКА И ЕЁ ПРИМЕНЕНИЯ\ \ \ том\ 18\ \ \ выпуск\ 3\ \ \ 2024
\hfill \textbf{\thepage}}}

{\small

\noindent
\textbf{Описание статьи из журнала:}

\Aue{Zagurenko, A.\,G., V.\,A.~Korotovskikh, A.\,A.~Kolesnikov, A.\,V.~Timonov, and D.\,V.~Kardymon}. 2008.
Tekhniko-ekonomicheskaya optimizatsiya dizayna gidrorazryva plasta [Technical and
economic optimization of the design
of hydraulic fracturing]. \textit{Neftyanoe hozyaystvo} [\textit{Oil Industry}] 11:54--57.

\Aue{Zhang, Z., and D.~Zhu}. 2008. Experimental research on the localized
electrochemical micromachining. \textit{Russ. J.~Electrochem.}  44(8):926--930.
{\sf doi:10.1134/S1023193508080077}.

\noindent
\textbf{Описание статьи из электронного журнала:}

\Aue{Swaminathan, V., E.~Lepkoswka-White, and B.\,P.~Rao}. 1999. Browsers or buyers in cyberspace? An
investigation of electronic factors influencing electronic exchange. \textit{JCMC}
5(2). Available at: {\sf http://www.ascusc.org/jcmc/vol5/issue2/} (accessed April~28, 2011).

\def\leftkol{Правила подготовки рукописей  для публикации в журнале
<<Информатика и её применения>>}

\def\rightkol{Правила подготовки рукописей  для публикации в журнале
<<Информатика и её применения>>}


\noindent
\textbf{Описание статьи из продолжающегося издания (сборника трудов):}

\Aue{Astakhov, M.\,V., and T.\,V.~Tagantsev}. 2006. Eksperimental'noe
issledovanie prochnosti soedineniy ``stal'--kompozit'' [Experimental study of
the strength of joints ``steel--composite'']. \textit{Trudy MGTU
``Matematicheskoe modelirovanie slozhnykh tekh\-ni\-che\-skikh sistem''}
[\textit{Bauman MSTU ``Mathematical Modeling of Complex Technical
Systems'' Proceedings}]. 593:125--130.


\pagebreak



\noindent
\textbf{Описание материалов конференций:}

\Aue{Usmanov, T.\,S., A.\,A.~Gusmanov, I.\,Z.~Mullagalin, R.\,Ju.~Muhametshina, A.\,N.~Chervyakova, and
A.\,V.~Sveshnikov}. 2007. Osobennosti proektirovaniya razrabotki mestorozhdeniy
s primeneniem gidrorazryva
plasta [Features of the design of field development with the use of hydraulic fracturing].
\textit{Trudy 6-go
Mezhdu\-na\-rod\-no\-go Simpoziuma ``Novye resursosberegayushchie tekhnologii nedropol'zovaniya i povysheniya
neftegazootdachi''} [\textit{6th  Symposium (International) ``New Energy Saving Subsoil Technologies and
the Increasing of the Oil and Gas Impact'' Proceedings}]. Moscow. 267--272.



\def\leftfootline{\small{\textbf{\thepage}
\hfill ИНФОРМАТИКА И ЕЁ ПРИМЕНЕНИЯ\ \ \ том\ 18\ \ \ выпуск\ 3\ \ \ 2024}
}%
 \def\rightfootline{\small{ИНФОРМАТИКА И ЕЁ ПРИМЕНЕНИЯ\ \ \ том\ 18\ \ \ выпуск\ 3\ \ \ 2024
\hfill \textbf{\thepage}}}



\noindent
\textbf{Описание книги (монографии, сборники):}



Lindorf, L.\,S., and L.\,G.~Mamikoniants, eds. 1972.
\textit{Ekspluatatsiya turbogeneratorov s neposredstvennym
okhlazhdeniem} [\textit{Operation of turbine generators with direct cooling}].
Moscow: Energy Publs. 352~p.


\Aue{Latyshev, V.\,N.} 2009. \textit{Tribologiya rezaniya. Kn.~1: Friktsionnye protsessy
pri rezanii metallov}
[\textit{Tribology of cutting. Vol.~1: Frictional processes in metal cutting}]. Ivanovo: Ivanovskii
State Univ. 108~p.

\def\leftkol{Правила подготовки рукописей  для публикации в журнале
<<Информатика и её применения>>}

\def\rightkol{Правила подготовки рукописей  для публикации в журнале
<<Информатика и её применения>>}

\noindent
\textbf{Описание переводной книги}
(в списке литературы к русскоязычной части необходимо указать:~/ Пер.\ с англ.~---
после названия книги, а в конце ссылки указать оригинал книги в круглых скобках):
\begin{enumerate}[1.]
\item  В русскоязычной части:

\def\leftfootline{\small{\textbf{\thepage}
\hfill ИНФОРМАТИКА И ЕЁ ПРИМЕНЕНИЯ\ \ \ том\ 18\ \ \ выпуск\ 3\ \ \ 2024}
}%
 \def\rightfootline{\small{ИНФОРМАТИКА И ЕЁ ПРИМЕНЕНИЯ\ \ \ том\ 18\ \ \ выпуск\ 3\ \ \ 2024
\hfill \textbf{\thepage}}}

\Au{Тимошенко С.\,П., Янг Д.\,Х., Уивер~У.}
Колебания в инженерном деле~/ Пер.\ с англ.~--- М.: Машиностроение, 1985. 472~с.
(\Au{Timoshenko~S.\,P., Young~D.\,H., Weaver~W.}
Vibration problems in engineering.~--- 4th ed.~--- New York, NY, USA: Wiley, 1974. 521~p.)\\[-13.5pt]
\item  В англоязычной части:

\Aue{Timoshenko, S.\,P., D.\,H.~Young, and W.~Weaver}.
1974. \textit{Vibration problems in engineering}. 4th ed. New York: 
Wiley. 521~p.
\end{enumerate}

\vspace*{-3pt}


\noindent
\textbf{Описание неопубликованного документа:}


\Aue{Latypov, A.\,R., M.\,M.~Khasanov, and V.\,A.~Baikov}.
2004 (unpubl.). Geologiya i~dobycha (NGT GiD) [Geology and production (NGT GiD)]. Certificate on official registration of the computer program
No.\,2004611198. 

\noindent
\textbf{Описание интернет-ресурса:}


Pravila tsitirovaniya istochnikov [Rules for the citing of sources]. Available at: {\sf
http://www.scribd.com/doc/1034528/} (accessed February~7, 2011).

%\pagebreak

\noindent
\textbf{Описание диссертации или автореферата диссертации:}

\Aue{Semenov, V.\,I.}
2003. Matematicheskoe modelirovanie plazmy v sisteme kompaktnyy tor [Mathematical
modeling of the plasma in the compact torus].  Moscow.  D.Sc.\ Diss. 272~p.

\Aue{Kozhunova, O.\,S.} 2009. Tekhnologiya razrabotki semanticheskogo
slovarya informatsionnogo monitoringa [Technology of development of
semantic dictionary of information monitoring system].  Moscow: IPI RAN. PhD Thesis. 23~p.


\noindent
\textbf{Описание ГОСТа:}

GOST 8.586.5-2005. 2007. Metodika vypolneniya izmereniy. Izmerenie raskhoda i~kolichestva zhidkostey i~gazov
s~pomoshch'yu standartnykh suzhayushchikh ustroystv [Method of measurement.
Measurement of flow rate and volume of liquids and gases by means of orifice devices]. Moscow:
Standardinform  Publs. 10~p.

\noindent
\textbf{Описание патента:}

\Aue{Bolshakov, M.\,V., A.\,V.~Kulakov, A.\,N.~Lavrenov, and M.\,V.~Palkin}.
2006. Sposob orientirovaniya po krenu letatel'nogo
apparata s opti\-che\-skoy golovkoy
samonavedeniya [The way to orient on the roll of aircraft with optical homing head].
Patent RF No.\,2280590.
}

\item Присланные в редакцию материалы авторам не возвращаются.\\[-13.5pt]

\item При отправке файлов по электронной почте просим придерживаться следующих
правил:
\begin{itemize}
\item указывать в поле subject (тема) название журнала и фамилию автора;\\[-13.5pt]
\item указывать в тексте письма название статьи, авторов и~журнал, в~который направляется статья;\\[-13.5pt]
\item использовать attach (присоединение);\\[-13.5pt]
\item в состав электронной версии статьи должны входить: файл, содержащий текст
статьи, и файл(ы), содержащий(е) иллюстрации.\\[-13.5pt]
\end{itemize}

\item Журнал <<Информатика и её применения>> является некоммерческим изданием.
Плата за публикацию не взимается, гонорар авторам не выплачивается.
\end{enumerate}



\def\leftfootline{\small{\textbf{\thepage}
\hfill ИНФОРМАТИКА И ЕЁ ПРИМЕНЕНИЯ\ \ \ том\ 18\ \ \ выпуск\ 3\ \ \ 2024}
}%
 \def\rightfootline{\small{ИНФОРМАТИКА И ЕЁ ПРИМЕНЕНИЯ\ \ \ том\ 18\ \ \ выпуск\ 3\ \ \ 2024
\hfill \textbf{\thepage}}}


\vspace*{-1mm}

\begin{center}

\textbf{Адрес редакции журнала <<Информатика и её применения>>:} \\




Москва 119333, ул.~Вавилова, д.~44, корп.~2, ФИЦ ИУ РАН\\[-10pt]

\

Тел.: +7\,(499)\,135-86-92\ \ Факс:  +7\,(495)\,930-45-05\\[-10pt]

 \

e-mail:   {\sf iiep@frccsc.ru} (Стригина Светлана Николаевна)\\[-10pt]

\

{\sf http://www.ipiran.ru/journal/issues/}
\end{center}
}


\def\leftkol{Правила подготовки рукописей  для публикации в журнале
<<Информатика и её применения>>}

\def\rightkol{Правила подготовки рукописей  для публикации в журнале
<<Информатика и её применения>>}


\def\leftfootline{\small{\textbf{\thepage}
\hfill ИНФОРМАТИКА И ЕЁ ПРИМЕНЕНИЯ\ \ \ том\ 18\ \ \ выпуск\ 3\ \ \ 2024}
}%
 \def\rightfootline{\small{ИНФОРМАТИКА И ЕЁ ПРИМЕНЕНИЯ\ \ \ том\ 18\ \ \ выпуск\ 3\ \ \ 2024
\hfill \textbf{\thepage}}} 
\def\stat{podg-e}
{%\hrule\par
%\vskip 7pt % 7pt
\vspace*{-24pt}
\raggedleft\Large \bf%\baselineskip=3.2ex
Requirements for manuscripts submitted to Journal
``Informatics~and~Applications'' \vskip 8pt
    \hrule
    \par
\vskip 21pt plus 6pt minus 3pt }

\label{st\stat}

\def\tit{\ }

\def\aut{\ }
\def\auf{\ }

\def\leftkol{\ }

\def\rightkol{\ }
%Requirements for manuscripts submitted to Journal
%``Informatics~and~Applications''}

\titele{\tit}{\aut}{\auf}{\leftkol}{\rightkol}

\def\leftfootline{\small{\textbf{\thepage}
\hfill INFORMATIKA I EE PRIMENENIYA~--- INFORMATICS AND APPLICATIONS\ \ \ 2019\
\ \ volume~13\ \ \ issue\ 4}
}%
 \def\rightfootline{\small{INFORMATIKA I EE PRIMENENIYA~--- INFORMATICS AND APPLICATIONS\ \ \ 2019\ \ \ volume~13\ \ \ issue\ 4
\hfill \textbf{\thepage}}}

\vspace*{-60pt}

{\small

\noindent
Journal ``Informatics and Applications'' (Inform.\ Appl.)
publishes theoretical, review, and discussion
articles on the research and development in the
field of informatics and its applications.

The journal is published in Russian.
By a special decision of the editorial
board, some articles can be published in English.


The topics covered include the following areas:
\begin{itemize}
               \item
     theoretical fundamentals of informatics; \\[-14pt]
\item
mathematical methods for studying complex systems and processes; \\[-14pt]
\item
information systems and networks;\\[-14pt]
\item
information technologies; and \\[-14pt]
\item
architecture and software of computational complexes and networks. \\[-14pt]
\end{itemize}

\noindent
\begin{enumerate}[1.]
\item The Journal publishes original articles which have not been published before and are not
intended for simultaneous publication in other editions. An article submitted to the Journal must not violate the
Copyright law. Sending the manuscript to the Editorial Board, the authors retain all rights of the
owners of the manuscript and transfer the nonexclusive rights to publish the article in Russian
(or the language of the article, if not Russian) and its distribution in Russia and abroad to the
Founders and the Editorial Board. Authors should submit a letter to the Editorial Board in the
following form:

{\bfseries\textit{Agreement on the transfer of rights to publish:}}

``\textit{We, the undersigned authors of the manuscript ``\ldots'', pass to the
Founder and the Editorial Board of the Journal ``Informatics and Applications''
the nonexclusive right to publish the manuscript of the article in Russian (or
in English) in both print and electronic versions of the Journal. We affirm
that this publication does not violate the Copyright of other persons or
organizations.}

\textit{Author(s) signature(s): (name(s), address(es), date).}

This agreement should be submitted in paper form or in the form of a scanned copy (signed by
the authors).


%The Editorial Board has the right to request from the authors an official expert conclusion that
%the submitted article has no secret data prohibited for publication. \\[-13.5pt]
\item
A submitted article should be attached with \textbf{the data on the author(s)} (see item~8). If
there are several authors, the contact person should be indicated who is responsible for
correspondence with the Editorial Board and other authors about revisions and final approval
of the proofs.\\[-13.5pt]

\item The Editorial Board of the Journal examines the article according to the established
reviewing procedure. If the authors receive their article for correction after reviewing, it does not
mean that the article is approved for publication. The corrected article should be sent to the
Editorial Board for the subsequent review and approval.\\[-13.5pt]

\item The decision on the article publication or its rejection is communicated to the authors. The
Editorial Board may also send the reviews on the submitted articles to the authors. Any
discussion upon the rejected articles is not possible.\\[-13.5pt]

\item The edited articles will be sent to the authors for proofread. The comments of the authors
to the edited text of the article should be sent to the Editorial Board as soon as possible.\\[-13.5pt]

\item The manuscript of the article should be presented electronically in the MS WORD (.doc or
.docx) or \LaTeX\ (.tex) formats, and additionally in the .pdf format. All documents
 may be sent
by e-mail or provided on a CD or diskette. A~hard copy submission is not necessary.\\[-13.5pt]

\item The recommended typesetting instructions for manuscript.

Pages parameters: format A4, portrait orientation, document margins (cm): left~--- 2.5, right~---
1.5, above~--- 2.0, below~--- 2.0, footer 1.3.

Text: font~---Times New Roman, font size~--- 14, paragraph indent~--- 0.5, line spacing~--- 1.5,
justified alignment.

The recommended manuscript size: not more than 15~pages of the specified format.
If the specified size exceeded, the editorial board is entitled to require the author
to reduce the manuscript.

Use only standard abbreviations. Avoid  abbreviations in the title and
abstract. The full term for which an abbreviation stands should precede
its first use in the text unless it is a standard unit of measurement.

All pages of the manuscript should be numbered.

The templates for the manuscript typesetting are presented on site: {\sf
http://www.ipiran.ru/journal/template.doc}.\\[-13.5pt]


%\def\leftkol{Requirements for manuscripts submitted to Journal
%``Informatics~and~Applications''}

\item The articles should enclose data both in \textbf{Russian and English}:
\begin{itemize}
\item title;\\[-13.5pt]
\item author's name and surname;\\[-13.5pt]
\item affiliation~--- organization, its address with ZIP code, city, country, and
official e-mail address;\\[-13.5pt]
\item data on authors according to the format: (see site)

{\sf http://www.ipiran.ru/journal/issues/2013\_07\_01/authors.asp}  and

{\sf  http://www.ipiran.ru/journal/issues/2013\_07\_01\_eng/authors.asp};\\[-13.5pt]

\pagebreak

\def\leftfootline{\small{\textbf{\thepage}
\hfill INFORMATIKA I EE PRIMENENIYA~--- INFORMATICS AND APPLICATIONS\ \ \ 2019\
\ \ volume~13\ \ \ issue\ 4}
}%
 \def\rightfootline{\small{INFORMATIKA I EE PRIMENENIYA~--- INFORMATICS AND APPLICATIONS\ \ \ 2019\ \ \ volume~13\ \ \ issue\ 4
\hfill \textbf{\thepage}}}


%\def\leftkol{Requirements for manuscripts submitted to Journal
%``Informatics~and~Applications''}

%\def\rightkol{Requirements for manuscripts submitted to Journal
%``Informatics~and~Applications''}



\item abstract (not less than 100 words) both in Russian and in English. Abstract is a short
summary of the article that can be published separately. The abstract is the
main source of information on the article and it could be included in leading information
systems and data bases. The abstract in English has to be an original text and should
not be an exact translation of the Russian one. Good English is required.
In abstracts, avoid references and formulae;\\[-13.5pt]
\item indexing is performed on the basis of keywords. The use of keywords from the
internationally accepted thematic Thesauri is recommended.

%\def\leftkol{Requirements for manuscripts submitted to Journal
%``Informatics~and~Applications''}

%\def\rightkol{Requirements for manuscripts submitted to Journal
%``Informatics~and~Applications''}

Important! Keywords must not be sentences;
\item Acknowledgments.
\end{itemize}

\item References. Russian references have to be presented both in English translation and Latin
transliteration (refer {\sf http://www.translit.net/ru/bgn/}).

Please take into account the following examples of Russian references appearance:

\noindent
\textbf{Article in journal:}

\Aue{Zhang, Z., and D.~Zhu}. 2008. Experimental research on the localized electrochemical
micromachining.
\textit{Rus. J.~Electrochem.}  44(8):926--930. {\sf doi:10.1134/S1023193508080077}.


\noindent
\textbf{Journal article in electronic format:}

\Aue{Swaminathan, V., E.~Lepkoswka-White, and B.\,P.~Rao}. 1999. Browsers or buyers in
cyberspace? An
investigation of electronic factors influencing electronic exchange. \textit{JCMC}
5(2). Available at: {\sf http://www.ascusc.org/jcmc/vol5/issue2/} (accessed April~28, 2011).




\noindent
\textbf{Article from the continuing publication (collection of works, proceedings):}

\Aue{Astakhov, M.\,V., and T.\,V.~Tagantsev}. 2006. Eksperimental'noe
issledovanie prochnosti soedineniy ``stal'--kompozit'' [Experimental study of
the strength of joints ``steel--composite'']. \textit{Trudy MGTU
``Matematicheskoe modelirovanie slozhnykh tekh\-ni\-che\-skikh sistem''}
[\textit{Bauman MSTU ``Mathematical Modeling of Complex Technical
Systems'' Proceedings}]. 593:125--130.

\def\leftfootline{\small{\textbf{\thepage}
\hfill INFORMATIKA I EE PRIMENENIYA~--- INFORMATICS AND APPLICATIONS\ \ \ 2019\
\ \ volume~13\ \ \ issue\ 4}
}%
 \def\rightfootline{\small{INFORMATIKA I EE PRIMENENIYA~--- INFORMATICS AND APPLICATIONS\ \ \ 2019\ \ \ volume~13\ \ \ issue\ 4
\hfill \textbf{\thepage}}}

\def\leftkol{Requirements for manuscripts submitted to Journal
``Informatics~and~Applications''}

\def\rightkol{Requirements for manuscripts submitted to Journal
``Informatics~and~Applications''}

\noindent
\textbf{Conference proceedings:}

\Aue{Usmanov, T.\,S., A.\,A.~Gusmanov, I.\,Z.~Mullagalin, R.\,Ju.~Muhametshina,
A.\,N.~Chervyakova, and
A.\,V.~Sveshnikov}. 2007. Osobennosti proektirovaniya razrabotki mestorozhdeniy
s primeneniem gidrorazryva
plasta [Features of the design of field development with the use of hydraulic fracturing].
\textit{Trudy 6-go
Mezhdu\-na\-rod\-no\-go Simpoziuma ``Novye resursosberegayushchie tekhnologii
nedropol'zovaniya i povysheniya
neftegazootdachi''} [\textit{6th  Symposium (International) ``New Energy Saving Subsoil
Technologies and
the Increasing of the Oil and Gas Impact'' Proceedings}]. Moscow. 267--272.


\noindent
\textbf{Books and other monographs:}




Lindorf, L.\,S., and L.\,G.~Mamikoniants, eds. 1972.
\textit{Ekspluatatsiya turbogeneratorov s neposredstvennym
okhlazhdeniem} [\textit{Operation of turbine generators with direct cooling}].
Moscow: Energy Publs. 352~p.


%\Aue{Latyshev, V.\,N.} 2009. \textit{Tribologiya rezaniya. Kn.~1: Frikcionnye prosessy
%pri rezanii metallov}
%[\textit{Tribology of cutting. Vol.~1: Frictional processes in metal cutting}]. Ivanovo: Ivanovskii
%State Univ. 108~p.


%\noindent
%\textbf{Unpublished material:}

%\Aue{Latypov, A.\,R., M.\,M.~Khasanov, and V.\,A.~Baikov}.
%2004. Geology and production (NGT GiD). Certificate on official registration of the computer
%program
%No.\,2004611198. (In Russian, unpubl.)

%\noindent
%\textbf{Internet-source:}

%APA Style. 2011. Available at: {\sf http://www.apastyle.org/apa-style-help.aspx} (accessed
%February~5, 2011).

%Pravila citirovaniya istochnikov [Rules for the citing of sources]. Available at: {\sf
%http://www.scribd.com/doc/1034528/} (accessed February~7, 2011).


\noindent
\textbf{Dissertation and Thesis:}

%\Aue{Semenov, V.\,I.}
%2003. Matematicheskoe modelirovanie plazmy v sisteme kompaktnyy tor. [Mathematical
%modeling of the plasma in the compact torus]. D.Sc.\ Diss. Moscow. 272~p.

\Aue{Kozhunova, O.\,S.} 2009. Tekhnologiya razrabotki semanticheskogo
slovarya informatsionnogo monitoringa [Technology of development of
semantic dictionary of information monitoring system]. PhD Thesis. Moscow: IPI RAN. 23~p.


\noindent
\textbf{State standards and patents:}

GOST 8.586.5-2005. 2007. Metodika vypolneniya izmereniy. Izmerenie raskhoda i~kolichestva
zhidkostey i gazov 
s~pomoshch'yu standartnykh suzhayushchikh ustroystv [Method of measurement.
Measurement of flow rate and volume of liquids and gases by means of orifice devices]. M.:
Standardinform
Publs. 10~p.

%\noindent
%\textbf{Patent:}

\Aue{Bolshakov, M.\,V., A.\,V.~Kulakov, A.\,N.~Lavrenov, and M.\,V.~Palkin}.
2006. Sposob orientirovaniya po krenu letatel'nogo
apparata s opti\-che\-skoy golovkoy
samonavedeniya [The way to orient on the roll of aircraft with optical homing head].
Patent RF No.\,2280590.

References in Latin transcription are presented in the original language.

References in the text are numbered according to the order of their
first appearance; the number is
placed in square brackets. All items from the reference list should be
cited.\\[-13.5pt]

\item Manuscripts and additional materials are not returned to Authors by the Editorial Board.\\[-13.5pt]

\item Submissions of files by e-mail must include:\\[-13.5pt]
\begin{itemize}
\item   the journal title and author's name in the ``Subject'' field; \\[-13.5pt]
\item   an article and additional materials have to be attached using the ``attach'' function;\\[-13.5pt]
\item   an electronic version of the article should contain the file with the text and a separate file
with figures.\\[-13.5pt]
\end{itemize}

\item ``Informatics and Applications'' journal is not a profit publication. There are no
charges for the authors as well as there are no royalties.\\[-13.5pt]
\end{enumerate}

\def\leftfootline{\small{\textbf{\thepage}
\hfill INFORMATIKA I EE PRIMENENIYA~--- INFORMATICS AND APPLICATIONS\ \ \ 2019\
\ \ volume~13\ \ \ issue\ 4}
}%
 \def\rightfootline{\small{INFORMATIKA I EE PRIMENENIYA~--- INFORMATICS AND APPLICATIONS\ \ \ 2019\ \ \ volume~13\ \ \ issue\ 4
\hfill \textbf{\thepage}}}

\def\leftkol{Requirements for manuscripts submitted to Journal
``Informatics~and~Applications''}

\def\rightkol{Requirements for manuscripts submitted to Journal
``Informatics~and~Applications''}


%\vspace*{5mm}


\begin{center}
\textbf{Editorial Board address:} \\

%ABOUT AUTHORS



FRC CSC RAS, 44, block~2, Vavilov Str., Moscow 119333, Russia\\[-10pt]

\

Ph.: +7\,(499)\,135\,86\,92,\ \ Fax: +7\,(495)\,930\,45\,05\\[-10pt]

\

 e-mail: {\sf rust@ipiran.ru} (to Prof.\ Rustem Seyful-Mulyukov)\\[-10pt]

\

 {\sf http://www.ipiran.ru/english/journal.asp}
\end{center}
 }
%\thispagestyle{myheadings}

\def\leftkol{Requirements for manuscripts submitted to Journal
``Informatics~and~Applications''}

\def\rightkol{Requirements for manuscripts submitted to Journal
``Informatics~and~Applications''}

\def\leftfootline{\small{\textbf{\thepage}
\hfill INFORMATIKA I EE PRIMENENIYA~--- INFORMATICS AND APPLICATIONS\ \ \ 2019\
\ \ volume~13\ \ \ issue\ 4}
}%
 \def\rightfootline{\small{INFORMATIKA I EE PRIMENENIYA~--- INFORMATICS AND APPLICATIONS\ \ \ 2019\ \ \ volume~13\ \ \ issue\ 4
\hfill \textbf{\thepage}}}

 \label{end\stat}

\newpage

%\vspace*{-60pt} {\small
{\baselineskip=9.1pt
\section*{Правила подготовки рукописей статей для публикации в журнале
<<Информатика и её применения>>}

\thispagestyle{empty}

 Журнал <<Информатика и её применения>> публикует
теоретические, обзорные и дискуссионные статьи, посвященные научным
исследованиям и разработкам в области информатики и ее приложений. Журнал
издается на русском языке. По специальному решению редколлегии отдельные статьи,
в виде исключения, могут печататься на английском языке.
Тематика журнала охватывает следующие направления:
\begin{itemize}
\item теоретические основы информатики; %\\[-13.5pt]
\item математические методы исследования сложных систем и процессов; %\\[-13.5pt]
\item информационные системы и сети; %\\[-13.5pt]
\item информационные технологии; %\\[-13.5pt]
\item архитектура и программное
обеспечение вычислительных комплексов и сетей.
\end{itemize}
\begin{enumerate}
\item В журнале печатаются результаты, ранее не
опубликованные и не предназначенные к одновременной публикации в других
изданиях. Публикация не должна нарушать закон об авторских правах. Направляя
свою рукопись в редакцию, авторы автоматически передают учредителям и
редколлегии неисключительные права на издание данной статьи на русском языке и
на ее распространение в России и за рубежом. При этом за авторами сохраняются
все права как собственников данной рукописи. В связи с этим авторами должно
быть представлено в редакцию письмо в следующей форме:
Соглашение о передаче права на публикацию:

\textit{<<Мы, нижеподписавшиеся, авторы рукописи <<$\qquad\qquad$>>, передаем
учредителям и редколлегии журнала <<Информатика и её применения>>
неисключительное право опубликовать данную рукопись статьи на русском языке как
в печатной, так и в электронной версиях журнала. Мы подтверждаем, что данная
публикация не нарушает авторского права других лиц или организаций. Подписи
авторов: (ф.\,и.\,о., дата, адрес)>>.}

Указанное соглашение может быть представлено 
как в бумажном виде, так и в виде отсканированной копии (с подписями авторов).


Редколлегия вправе запросить у авторов экспертное заключение о возможности
опубликования представленной статьи в открытой печати. %\\[-13.5pt]
\item Статья
подписывается всеми авторами. На отдельном листе представляются данные автора
(или всех авторов): фамилия, полные имя и отчество, телефон, факс, e-mail,
почтовый адрес. Если работа выполнена несколькими авторами, указывается фамилия
одного из них, ответственного за переписку с редакцией. %\\[-13.5pt]
\item Редакция журнала
осуществляет самостоятельную экспертизу присланных статей. Возвращение рукописи
на доработку не означает, что статья уже принята к печати. Доработанный вариант
с ответом на замечания рецензента необходимо прислать в редакцию. %\\[-13.5pt]
\item Решение
редакционной коллегии о принятии статьи к печати или ее отклонении сообщается
авторам. Редколлегия не обязуется направлять рецензию авторам отклоненной
статьи. %\\[-13.5pt]
\item Корректура статей высылается авторам для просмотра. Редакция
просит авторов присылать свои замечания в кратчайшие сроки. %\\[-13.5pt]
\item При
подготовке рукописи в MS Word рекомендуется использовать следующие настройки.
Параметры страницы: формат~--- А4; ориентация~--- книжная; поля (см): внутри~---
2,5, снаружи~--- 1,5, сверху~--- 2, снизу~--- 2, от края до нижнего
колонтитула~--- 1,3. Основной текст: стиль~--- <<Обычный>>: шрифт Times New
Roman, размер 14~пунктов, абзацный отступ~--- 0,5~см, 1,5 интервала,
выравнивание~--- по ширине. Рекомендуемый объем рукописи~--- не свыше
25~страниц указанного формата. Ознакомиться с шаблонами, содержащими примеры
оформления, можно по адресу в Интернете:
\textsf{http://www.ipiran.ru/journal/template.doc}.
\item К рукописи, предоставляемой в 2-х
экземплярах, обязательно прилагается электронная версия статьи (как правило, в
форматах MS WORD (.doc) или \LaTeX\ (.tex), а также~--- дополнительно~--- в
формате .pdf) на дискете, лазерном диске или по электронной почте. Сокращения
слов, кроме стандартных, не применяются. Все страницы рукописи должны быть
пронумерованы. %\\[-13.5pt]
\item Статья должна содержать следующую информацию на русском и
английском языках: название, Ф.И.О. авторов, места работы авторов и их
электронные адреса, подробные сведения об авторах, оформленные в соответствии с форматом, 
определяемым файлами {\sf http://www.ipiran.ru/journal/issues/2011\_05\_01/authors.asp} и 
{\sf http://www.ipiran.ru/journal/issues/2011\_01\_eng/authors.asp},
аннотация (не более 100~слов), ключевые слова. Ссылки на
литературу в тексте статьи нумеруются (в квадратных скобках) и располагаются в
порядке их первого упоминания. В~списке литературы не должно быть позиций, на которые нет ссылки в тексте статьи.
Все фамилии авторов, заглавия статей, названия
книг, конференций и~т.\,п.\ даются на языке оригинала, если этот язык
использует кириллический или латинский алфавит. %\\[-13.5pt]
\item Присланные в редакцию материалы авторам не возвращаются.
\item При отправке файлов по электронной
почте просим придерживаться следующих правил:
\begin{itemize}
\item указывать в поле subject (тема) название журнала и фамилию автора; %\\[-13.5pt]
\item использовать attach (присоединение); %\\[-13.5pt]
\item в случае больших объемов информации возможно
использование общеизвестных архиваторов (ZIP, RAR); %\\[-13.5pt]
\item в состав электронной версии статьи должны входить: файл, содержащий текст статьи, и файл(ы),
содержащий(е) иллюстрации. %\\[-13.5pt]
\end{itemize}
\item Журнал <<Информатика и её применения>> является некоммерческим изданием. 
Плата за публикацию с авторов не взимается, гонорар авторам не выплачивается.
\end{enumerate}
\thispagestyle{empty}
\textbf{Адрес редакции:} Москва 119333,
ул.~Вавилова, д.~44, корп.~2, ИПИ РАН\\
\hphantom{\textbf{Адрес редакции:} }Тел.: +7 (499) 135-86-92\ \
Факс:  +7 (495) 930-45-05\ \  E-mail:   rust@ipiran.ru }
}

%\include{ipi-ind}

%\tableofcontents

\end{document}

%\tableofcontents

%\end{document}

%\tableofcontents


\end{document}

\newcommand{\Ack}{\subsection*{\protect\large\bf Acknowledgments}}