\def\stat{abgaryan}

\def\tit{ПРИМЕНЕНИЕ МЕТОДОВ ПОДДЕРЖКИ ПРИНЯТИЯ РЕШЕНИЙ 
ДЛЯ~МНОГОКРИТЕРИАЛЬНОЙ ЗАДАЧИ ОТБОРА МНОГОМАСШТАБНЫХ КОМПОЗИЦИЙ}

\def\titkol{Применение методов поддержки принятия решений для 
многокритериальной задачи отбора МК} %многомасштабных композиций}

\def\aut{К.\,К.~Абгарян$^1$, В.\,А.~Осипова$^2$}

\def\autkol{К.\,К.~Абгарян, В.\,А.~Осипова}

\titel{\tit}{\aut}{\autkol}{\titkol}

\index{Абгарян К.\,К.}
\index{Осипова В.\,А.}
\index{Abgaryan K.\,K.}
\index{Osipova V.\,A.}


%{\renewcommand{\thefootnote}{\fnsymbol{footnote}} \footnotetext[1]
%{Работа выполнена при частичной финансовой 
%поддержке РФФИ (проект 17-07-00577).}}


\renewcommand{\thefootnote}{\arabic{footnote}}
\footnotetext[1]{Вычислительный центр имени А.\,А.~Дородницына Федерального исследовательского центра 
<<Информатика и~управление>> Российской академии наук; Московский авиационный институт 
(национальный исследовательский университет), \mbox{kristal83@mail.ru}}
\footnotetext[2]{Московский авиационный институт (национальный исследовательский университет), 
\mbox{victoria.a.osipova@gmal.com}}

%\vspace*{-2pt}

           
    
    \Abst{Рассмотрены вопросы использования методов поддержки принятия решений для 
задачи отбора многомасштабных композиций (МК)~--- вычислительных аналогов 
многомасштабных фи\-зи\-ко-ма\-те\-ма\-ти\-че\-ских моделей, созданных для анализа 
различных гетерогенных процессов, связанных с~формированием новых композиционных 
материалов с~заранее заданными свойствами. При решении конкретных задач могут быть 
построены разные многомасштабные модели и~соответствующие им МК. Возникает вопрос 
о~сравнении этих моделей, об оценке их эффективности для данной конкретной задачи. 
В~работе на этапе предсказательного моделирования предлагается методика сравнения 
многомасштабных моделей с~помощью оценки и~отбора соответствующих МК 
с~использованием методов поддержки принятия решений при многих критериях качества. Для 
иллюстрации возможности выбора наилучшей альтернативы при наличии дополнительной 
информации о~критериях оценки МК рассмотрен модельный 
пример, связанный с~исследованием электронных и~структурных свойств тонких пленок InN 
(GaN) на кремниевых подложках.}
    
    \KW{многомасштабное моделирование; теория принятия решений; критерии качества; 
альтернатива; методы поддержки принятия решений; многокритериальность; функция 
ценности}

\DOI{10.14357/19922264190207}
  
\vspace*{1pt}


\vskip 10pt plus 9pt minus 6pt

\thispagestyle{headings}

\begin{multicols}{2}

\label{st\stat}

   
   Использование методов многомасштабного моделирования~--- как 
последовательных, так и~параллельных~--- открывает широкие возможности для 
изучения процессов и~явлений, при анализе которых необходимо учитывать 
широкий спектр пространственных и/или временн$\acute{\mbox{ы}}$х масштабов. К~таким 
явлениям, в~частности, относятся сложные гетерогенные процессы, протекающие в~многокомпонентных системах, связанных с~формированием новых 
композиционных материалов с~заранее заданными свойствами.
   
   В работе~[1] представлены основные положения разработанной 
информационной технологии построения многомасштабных моделей 
с~использованием таких новых понятий, как <<базовая  
мо\-дель-ком\-по\-зи\-ция>> и~<<многомасштабная композиция>>. Для создания 
на их основе программных систем применялся  
мо\-дель\-но-ори\-ен\-ти\-ро\-ван\-ный подход, который был развит в~работах 
Ю.\,И.~Бродского~[2].\linebreak Особенностью изложенной в~работе~[1] техноло-\linebreak гии 
является применение информационных структур, названных базовыми 
композициями (БК), объединяющих данные и~методы их обработки. Эти\linebreak 
математические объекты ставятся в~соответствие базовым математическим 
моделям, которые используются для решения различных многомасштабных задач. 
Для описания БК задействован тео\-ре\-ти\-ко-мно\-жест\-вен\-ный аппарат, 
который позволяет передать вычислительную сущность исходных 
математических моделей. Базовые композиции\linebreak служат композиционными 
элементами (объектами), из которых согласно представленной в~работе~[1] 
технологии строятся МК~--- информационные 
аналоги многомасштабных моделей, при помощи которых передается содержание 
многомасштабных\linebreak вычислительных процессов и~явлений. Далее на базе МК 
строятся сложные иерархические про\-граммные системы, применяемые для 
решения различных прикладных задач, в~том числе связанных с~созданием новых 
композиционных материалов.
   
   При решении конкретных задач могут быть построены разные 
многомасштабные модели, и, естественно, возникает вопрос о сравнении этих 
моделей, об оценке их эффективности. В~данной работе на этапе 
предсказательного моделирования предлагается методика сравнения 
многомасштабных моделей с~помощью оценки и~отбора соответствующих 
МК. Оценка и~отбор проводятся с~использованием 
методов поддержки принятия решений при многих критериях качества.
   
   Приведем описание метода построения МК~[1].
   
   Базовую модель-ком\-по\-зи\-цию можно представить как объединение 
основных множеств разного структурного типа:
   $\mathrm{VX}_{ij}$, $\mathrm{MA}_{ij}$, $E_{ij}$, 
   $\{\mathrm{MA}^k_{ij}\}^p_{k=1}$, $\{E^k_{ij}\}^p_{k=1}.$
   Здесь $i$~--- номер масштабного уровня, $i\hm=\overline{1,L}$, где $L$~--- 
число рассматриваемых уровней; $j$~--- номер базовой модели-композиции на 
текущем масштабном уровне, $j\hm=\overline{1,N_i}$, где
$N_i$~--- число моделей 
на $i$-м уровне; $k$~--- номер элементарного процесса БК. Опишем основные 
множества:
   \begin{description}
   \item
   $\mathrm{VX}_{ij}=\left\{ V_{ij}, X_{ij}\right\}$~--- множество данных, включающее:
   \begin{description}
   \item
   $V_{ij}$~--- множество входных данных (внешние характеристики модели);
   \item
      $X_{ij}$~--- множество выходных данных (фазовых переменных 
и~данных~--- свойств модели);
      \end{description}
   \item
         $\mathrm{MA}_{ij}$~--- множество методов обработки данных (модели и~алгоритмы);
   \item
   $E_{ij}$~--- множество событий, отнесенных к~описанию выполняемых 
в~рамках БК элементарных процессов;
   \item
   $\left\{\mathrm{MA}_{ij}^k\right\}^p_{k=1}$~--- множество реализаций моделей 
и~алгоритмов в~зависимости от элементарного процесса~$p$;
   \item
   $\left\{ E^k_{ij}\right\}^p_{k=1}$~--- множество реализаций событий по 
элементарным процессам.
   \end{description}
   
   Множество методов обработки данных опишем подробнее:
   $$
   \mathrm{MA}_{ij}=\left\{ M_{ij}, A_{ij}\right\} = 
   \left\{ s_{ij}, f_{ij}, a_{ij}, a_{i,\ldots 
,i^*,j}\right\}.
   $$ 
   
   Множество моделей~$M_{ij}$, входящих в~множество~$\mathrm{MA}_{ij}$, состоит из 
статических ($s_{ij}$) и~динамических ($f_{ij}$) методов обработки. 
Алгоритмические модели (алгоритмы)~$a_{ij}$, $i\hm=\overline{0,L}$, 
$j\hm=\overline{1,N_i}$, могут быть специализированными, т.\,е.\ используемыми 
только в~данной конкретной модели с~определенного масштабного уровня, или 
универсальными, применяемыми в~различных моделях с~разных масштабных 
уровней $a_{i,\ldots , i^*, j}$.
   
   \smallskip
   
   \noindent
   \textbf{Определение~1.}\ Под базовой мо\-делью-ком\-по\-зи\-ци\-ей 
$\mathbf{MC}_i^j$ будем понимать однопараметрическое семейство основных 
множеств, задействованных в~общем вычислительном процессе, разного 
структурного типа, включая данные (входные и~выходные) и~методы их 
обработки:
   \begin{multline*}
   \mathbf{MC}_i^j={}\\
   {}=\left\langle \left\{ \mathrm{VX}_{ij}, \mathrm{MA}_{ij}, E_{ij}, \left\{ 
\mathrm{MA}^k_{ij}\right\}^p_{k=1},\left\{ E^k_{ij}\right\}^p_{k=1}\right\}\right\rangle\,.
\end{multline*}
   Здесь
   
   \vspace*{-6pt}
   
   \noindent
   \begin{align*}
   \mathrm{VX}_{ij}&=\left\{ V_{ij}, X_{ij}\right\}\,;\\
   \mathrm{MA}_{ij}&=\left\{ M_{ij},  A_{ij}\right\}\,;\\
   \left\{ \mathrm{MA}^k_{ij}\right\}^p_{k=1}&=\left\{ \mathrm{MA}^1_{ij}, \mathrm{MA}^2_{ij},\ldots , 
\mathrm{MA}^p_{ij}\right\}\,;\\
   \left\{ E^k_{ij}\right\}^p_{k=1}&=\left\{ E^1_{ij}, E^2_{ij}, \ldots , 
E^p_{ij}\right\}\,.
   \end{align*}
   
   Параметром семейства основных множеств служит число элементарных 
процессов в~базовой мо\-де\-ли-ком\-по\-зи\-ции~$p$. Индексы~$i$ и~$j$ 
позволяют идентифицировать $\mathbf{MC}_i^j$ на пространственном 
уровне~$i$ по ее номеру~$j$.
   
   Структура модели-ком\-по\-зи\-ции представлена в~[1].
   %
   Такое представление полностью описывает структуру базовой 
   мо\-де\-ли-ком\-по\-зи\-ции и~задает шаблон, который будет заполняться конкретными данными 
при создании реальных экземпляров БК для решения практических задач 
математического моделирования.
   
   Далее приведем описание МК, позволяющее 
представить информацию, из каких именно мо\-де\-лей-ком\-по\-зи\-ций, с~каких 
масштабных уровней она состоит, сколько и~каких процессов задействовано в~ее 
работе, каким образом происходит обмен данными между  
мо\-де\-ля\-ми-ком\-по\-зи\-ци\-ями с~разных уровней.
   
   \smallskip
   
   \noindent
   \textbf{Определение~2.}\ Под МК будем 
понимать однопараметрическое семейство, полученное из экземпляров БК 
с~разных масштабных уровней за счет объединения в~общем вычислительном 
процессе их основных множеств разного структурного типа, включая данные 
(входные и~выходные) и~методы их обработки.
   
   Многомасштабную композицию 
   будем обозначать через 
   $\mathbf{MK}_{i,i^*, \ldots , 
   i^{***}}^{i,j;i^*, j^*;\ldots; i^{**}j^{***}}$. Здесь~$i, i^*,\ldots , i^{***}$~--- 
номера масштабных уровней, задействованных в~данной МК; $j, j^*, \ldots , 
j^{***}$~--- номера БК на конкретном масштабном уровне. В~определенном 
смысле $\mathbf{MK}_{i, i^*,\ldots}^{i,j; i^*, j^*; \ldots; i^{**}j^{***}}$ схожа 
с~БК, так как представляет собой объединение основных множеств разных 
структурных типов, связанных общим вычислительным процессом. Ее структуру 
можно представить набором таб\-лиц, соответствующих экземплярам входящих 
в~нее БК, расположенных в~определенном порядке, соответствующем иерархии 
масштабов, задействованных в~ней.
   
   Пусть на $i$-м масштабном уровне имеется экземпляр $\mathbf{MC}_i^j$ 
и~на $i^*$-м масштабном уровне~--- экземпляр~$\mathbf{MC}_{i^*}^{j^*}$. 
Здесь~$j$ и~$j^*$~--- номера базовых мо\-де\-лей-ком\-по\-зи\-ций на масштабных 
уровнях~$i$ и~$i^*$ соответственно. Составим многомасштабную композицию 
$\mathbf{MK}_{i,i^*}^{ij;i^*j^*}$ из двух экземпляров БК
$\mathbf{MC}_i^j$ и~$\mathbf{MC}^{j^*}_{i^*}$. Основными множествами, как 
и~в~случае создания композиции, будут:

\noindent
   \begin{gather*} 
   V_{ij} \cup V_{i^*j^*}\,;\ X_{ij}\cup X_{i^*j^*}\,;\\
   \mathrm{MA}_{ij}\cup \mathrm{MA}_{i^*j^*}\,;\quad
   E_{ij}\cup E_{i^*j^*}\,;\\
   \left\{ 
\mathrm{MA}_{ij}^k\right\}^p_{k=1}\cup \left\{ \mathrm{MA}^k_{i^*j^*}\right\}^{p^*}_{k=1}\,;\\
   \left\{ E^k_{ij}\right\}^p_{k=1}\cup \left\{ E^k_{i^*j^*}\right\}_{k=1}^{p^*}\,.
   \end{gather*}
   Здесь $p$ и~$p^*$ обозначают число процессов в~БК
$\mathbf{MC}_i^j$ и~$\mathbf{MC}_{i^*}^{j^*}$ соответственно.
   
   Многомасштабную композицию можно описать следующим образом:
   \begin{multline*}
   \mathbf{MK}_{i,i^*}^{ij; i^*j^*}=\left\langle 
   \left\{ 
   \vphantom{\left\{ 
E^k_{i^*j^*}\right\}^{p^*}_{k=1}}
V_{ij}\cup V_{i^*j^*}\,,\ 
X_{ij}\cup X_{i^*j^*}\,,\right.\right.\\
  \mathrm{MA}_{ij}\cup \mathrm{MA}_{i^*j^*}\,,
 E_{ij}\cup E_{i^*j^*}\,,\\
\left\{ \mathrm{MA}_{ij}^k\right\}^p_{k=1}\cup \left\{ 
\mathrm{MA}^k_{i^*j^*}\right\}_{k=1}^{p^*}\,,\\
   \left.\left. \left\{ E^k_{ij}\right\}^p_{k=1}\cup \left\{ 
E^k_{i^*j^*}\right\}^{p^*}_{k=1}\right\}\right\rangle\,.
   \end{multline*}
   
   Число процессов в~$\mathbf{MK}$ равно сумме $p\hm+p^*$.
   
   Связующими элементами между вычислительными моделями с~разных 
масштабных уровней, входящими в~$\mathbf{MK}$, служат глобальные 
параметры, которые играют основную роль при передаче информации между 
масштабными уровнями.
   
   При составлении многомасштабной композиции 
   $\mathbf{MK}_i^j\hm= 
\mathbf{MK}^{i^*, j^*; i^{**}, j^{**}}_{i^*,i^{**}}$ 
из~$\mathbf{MC}_{i^*}^{j^*}$ и~$\mathbf{MC}^{j^{**}}_{i^{**}}$ под 
глобальными параметрами 
$$
\Check{\mathbf{v}}\in \mathrm{VX}_{ij} = \left\{ 
V_{i^*j^*}\cup V_{i^{**}j^{**}}, X_{i^*j^*}\cup X_{i^{**}j^{**}}\right\}
$$ 
будем 
понимать элементы (параметры), относящиеся к~множеству $X_{i^*j^*}\cap 
V_{i^{**}j^{**}}$, образованному в~результате пересечения двух множеств 
выходных\linebreak данных~$X_{i^*j^*}$ (с нижнего масштабного уровня) и~входных 
данных~$V_{i^{**}j^{**}}$ с~верхнего масштабного уровня:
   \begin{multline*}
   X^{i^*j^*}\cap V_{i^{**}j^{**}}={}\\
   {}=\left\{ \Check{\mathbf{v}}: \left( 
\Check{\mathbf{v}}\in X_{i^*j^*}\right) \cap \left( \Check{\mathbf{v}}\in 
V_{i^{**}j^{**}}\right),\ \Check{\mathbf{v}}\in \mathrm{VX}_{ij}\right\}\,.
\end{multline*}
   
   Кроме того, при построении МК используются 
базовые мо\-де\-ли-ком\-по\-зи\-ции специального вида, обозначенные 
$\mathbf{DB}_i$, $i$~--- номер масштабного уровня, $i\hm=\overline{1,L}$, где 
$L$~--- чис\-ло рас\-смат\-ри\-ва\-емых уровней. Они требуются для хранения и~передачи 
дополнительной информации, необходимой для работы БК со\-от\-вет\-ст\-ву\-юще\-го 
уровня.
   
   Описанная технология многомасштабного моделирования может быть 
применена для решения различных задач структурной оптимизации, в~частности 
для задач материаловедения при моделировании свойств полупроводниковых 
наносистем~\cite{3-ab, 4-ab, 5-ab}.
   
   При компьютерной реализации разработанной концепции многомасштабного 
моделирования требуется построение различных информационных моделей~--- 
МК. В~данной работе на этапе предсказательного 
моделирования предлагается методика сравнения многомасштабных моделей 
с~помощью оценки и~отбора соответствующих МК, основанная на методах 
принятия решений в~многокритериальных ситуациях.
   
   В соответствии с~принятой терминологией каж\-дую МК, позволяющую решить 
поставленную задачу моделирования, назовем альтернативой. Можно проследить 
связь между требуемыми свойствами исследуемого объекта и~оценкой 
задействованных в~ходе моделирования МК, 
используемых для анализа этих свойств.
   
   Как показал анализ задачи моделирования, на первом этапе для оценки МК 
можно выбрать, например, следующие 5 критериев:
   \begin{enumerate}[(1)]
   \item вычислительная точность: критерий~$K_1$ с~чис\-ло\-вой шкалой 
$E_1\hm\subseteq R$, где $R$~--- множество действительных чисел;
   \item число арифметических операций: критерий~$K_2$ с~числовой шкалой 
$E_2\hm\subseteq R$;
   \item универсальность (возможность использования для скрининга): критерий 
$K_3$ со шкалой~$E_3$, состоящей из двух значений;
   \item соотношение собственного программного обеспечение и~пакетов 
прикладных программ: критерий~$K_4$ с~числовой шкалой $E_4\hm\subseteq R$;
   \item число задействованных масштабных уровней: критерий $K_5$ 
с~числовой шкалой $E_5\hm\subseteq R$.
   \end{enumerate}
   
   Этот набор критериев не является полным, однако удовлетворяет основным 
требованиям, предъявляемым к~перечню критериев, и~основан на информации 
о~смысле поставленной задачи~\cite{6-ab}.\linebreak
 Действительно, набор критериев 
$\{K_1,\ldots ,K_5\}$\linebreak соответствует существу поставленной задачи, минимален 
(в~том смысле, что различные критерии характеризуют различные свойства 
исходов), критерии измеримы и~операциональны (каждый критерий имеет 
однозначный и~ясный смысл, характеризует определенное свойство исходов).
   
   Таким образом, каждая МК, позволяющая решить 
задачу моделирования, характеризуется пятимерным вектором $x\hm=(x_1,x_2, 
x_3, x_4,x_5)$, где~$x_i$~--- значение, приписываемое данной альтернативе по 
критерию~$K_i$, $i\hm=1,\ldots ,5$. Используя известные процедуры~\cite{7-ab}, 
можно привести критерии к~однородным шкалам.
   
   Для выбора наилучших альтернатив можно\linebreak
    сравнивать их по векторному 
отношению до\-ми\-нирования и~рассмотреть па\-ре\-то-оп\-ти\-маль\-ные варианты. 
Однако дополнительная информация\linebreak о~критериях и~свойствах решений может 
сузить множество па\-ре\-то-оп\-ти\-маль\-ных вариантов. Эта информация 
предоставляется экспертами, ра\-бо\-та\-ющи\-ми в~данной предметной области, 
в~коалиции с~аналитиком, специалистом в~области принятия решений. Затем 
в~зависимости от конкретного вида полученной информации применяется 
математически обоснованный метод выбора наилучших альтернатив 
в~многокритериальной ситуации~\cite{6-ab, 7-ab}.
   
   Распространенный подход к~решению задачи оценки и~выбора наилучшей 
альтернативы, одновременно учитывающий дополнительную информацию 
о~критериях и~свойствах альтернатив, содержится в~построении числовой 
функции ценности.
   
   Пусть $S$~--- множество альтернатив и~считается, что при заданных 
критериях $K_1, \ldots , K_n$ каждая альтернатива $s\hm\in S$ имеет векторную 
оценку $x\hm=(x_1, \ldots , x_n)$, где $x_i\hm=K_i(s)$~--- значение альтернативы 
  по критерию~$K_i$, $i\hm=1,\ldots , n$.
   
   Рассмотрим множество $X\hm= \left\{ x\vert x\!=\!(x_1, \ldots , x_n),\right.$\linebreak
   $\left. x_i\hm=K_i(s), 
i\hm=1,\ldots , n; s\hm\in S\right\}$~--- множество векторных оценок альтернатив 
из множества~$S$. Пусть $a,b\hm\in S$ и~$x\hm=(x_1,\ldots , x_n)$~--- векторная 
оценка альтернативы~$a$; $y\hm=(y_1, \ldots , y_n)$~--- векторная оценка 
альтернативы~$b$.
   
   Числовая функция ценности $f:\ X\hm\to R$ облада-\linebreak ет следующим свойством: 
для любых двух альтернатив~$a$ и~$b$: $f(x_1, \ldots , x_n)\hm\geq f(y_1, \ldots 
,y_n)\Leftrightarrow$\linebreak альтернатива~$a$ не менее предпочтительна, чем~$b$. Если 
известна функция ценности, то поиск оптимального варианта сводится к~задаче 
нахождения аргумента максимума функции ценности на множестве~$X$:
   $$
   x^*=\argmax f(x), \enskip x\in X\,.
   $$
   
   При использовании эвристических методов построения функции ценности 
используется метод обобщенного критерия, заключающийся в~сведении 
многокритериальной задачи к~однокритериальной, набор критериев 
<<сворачивается>> в~числовую функцию, которая и~будет служить функцией 
ценности.
   
   Обычно функцию ценности строят в~аддитивном виде как сумму функций 
ценности по каждому критерию с~некоторыми весовыми коэффициентами 
$\alpha_1, \ldots , \alpha_n$:
   $$
   f(x_1,\ldots , x_n)=\sum\limits^n_{i=1} \alpha_i f_i(x_i)\,,
   $$
где $f_i$~--- функция ценности критерия~$K_i$.
   
   Как известно, построение функции ценности в~аддитивном виде правомерно 
только в~случае взаимной независимости критериев~\cite{8-ab}. Проверку 
взаимной независимости рассматриваемых критериев $K_1, \ldots , K_5$ можно 
провести с~учетом результатов Ле\-онть\-ева--Гор\-ма\-на: если любая пара 
критериев $\{K_i, K_j\}$, где~$i, j \hm= 1,\ldots ,5$, не зависит по предпочтению 
от остальных двух критериев, то все критерии $K_1,\ldots , K_5$ взаимно 
независимы по предпочтению. Построение функции ценности проводится 
известными методами, описанными, в~частности, в~работе~\cite{6-ab}.
   
   Для решения конкретной практической задачи с~использованием 
многомасштабной модели рассматривается определенный набор 
МК. Весовые коэффициенты~$\alpha_i$, $i\hm=1,\ldots 
,5$, и~конкретный вид функции ценности определяются содержательным смыслом 
задачи. Для их нахождения проводится поэтапная процедура, опирающаяся 
в~числе прочего на интерактивный процесс диалога между специалистом 
в~данной прикладной области и~аналитиком, формирующим процедуру 
опроса~\cite{6-ab, 9-ab}. Полученные оценки позволяют сравнить по 
предпочтительности МК и~отобрать для дальнейших 
исследований оптимальный набор МК.
   
   Другие возможные подходы к~решению задачи многокритериального выбора 
МК, в~том числе не требующие нахождения весовых коэффициентов, 
рассмотрены в~\cite{7-ab, 9-ab, 10-ab}.
   
   В качестве иллюстрации возможности выбора наилучшей альтернативы при 
наличии дополнительной информации о~критериях оценки 
МК рассмотрим модельный пример.
   
   Для исследования электронных и~структурных свойств тонких пленок InN 
(GaN) на кремниевых подложках~\cite{1-ab} можно построить 
условно~5~вариантов МК, которые реализуются 
с~применением вычислительных средств ЦКП ФИЦ ИУ РАН. Обозначим их как
   $\mathbf{MK}_{0,1,3}^{(A^i_{\alpha_i} A^j_{\alpha_j}/A^k_{\alpha_k})}$, 
   $\widetilde{\mathbf{MK}}_{0,1,3}^{(A^i_{\alpha_i} 
A^j_{\alpha_j}/A^k_{\alpha_k})}$,
   $\widetilde{\mathbf{MK}}_{0,1,3,4}^{(A^i_{\alpha_i} 
A^j_{\alpha_j}/A^k_{\alpha_k})}$\hspace*{-2pt},
   $\widetilde{\widetilde{\mathbf{MK}}}_{0,1,3}^{(A^i_{\alpha_i} 
A^j_{\alpha_j}/A^k_{\alpha_k})}$\hspace*{-2pt},
   $\widetilde{\widetilde{\mathbf{MK}}}_{0,1,3,4}^{(A^i_{\alpha_i} 
A^j_{\alpha_j}/A^k_{\alpha_k})}$\hspace*{-2pt}.
   
   В работе~\cite{1-ab} представлена структура многомасштабной композиции 
$\mathbf{MK}_{0,1,3}^{(A^i_{\alpha_i}A^j_{\alpha_j}/A^k_{\alpha_k})}$. 
Указаны экземпляры БК и~последовательность их 
использования в~вычислительном процессе~\cite{1-ab}.

   \begin{table*}\small %tabl1
   \begin{center}
   \Caption{Оценки МК по критериям}
    \vspace*{2ex}
    
   \tabcolsep=5pt
    \begin{tabular}{|c|c|c|c|c|c|}
    \hline
    &&&&&\\[-9pt]
 \tabcolsep=0pt\begin{tabular}{c} №\\ критерия\end{tabular}&
 $\mathbf{MK}_{0,1,3}^{(A^i_{\alpha_i} 
A^j_{\alpha_j}/A^k_{\alpha_k})}$&$\widetilde{\mathbf{MK}}_{0,1,3}^{(A^i_{\alpha_i} 
A^j_{\alpha_j}/A^k_{\alpha_k})}$&$\widetilde{\mathbf{MK}}_{0,1,3,4}^{(A^i_{\alpha_i
} A^j_{\alpha_j}/A^k_{\alpha_k})}$& 
$\widetilde{\widetilde{\mathbf{MK}}}_{0,1,3}^{(A^i_{\alpha_i} 
A^j_{\alpha_j}/A^k_{\alpha_k})}$ 
&$\widetilde{\widetilde{\mathbf{MK}}}_{0,1,3,4}^{(A^i_{\alpha_i} 
A^j_{\alpha_j}/A^k_{\alpha_k})}$\\
   \hline
    1&до 0,1&до 0,15&до 0,01&до 0,015&до 0,01\\
    \hline 
2&10 млн&15 млн&30 млн&25 млн&30 млн\\ 
    \hline
3&Да&Да&Да&Нет&Нет\\ 
    \hline
4&\tabcolsep=0pt\begin{tabular}{c}75\% собственного\\ программного\\ обеспечения\end{tabular}&50\%&15\%&15\%&75\%\\ 
    \hline
5&3 уровня&3 уровня&4 уровня&3 уровня&4 уровня\\
    \hline
    \end{tabular}
    \end{center}
    \vspace*{4pt}
   % \end{table*}
   %  \begin{table*}\small %tabl2
   \begin{center}
   \Caption{Оценки МК по десятибалльной шкале }
    \vspace*{2ex}
    
    \tabcolsep=5pt
    \begin{tabular}{|c|c|c|c|c|c|}
    \hline
        &&&&&\\[-9pt]
 \tabcolsep=0pt\begin{tabular}{c} №\\ критерия\end{tabular}&
 $\mathbf{MK}_{0,1,3}^{(A^i_{\alpha_i} 
A^j_{\alpha_j}/A^k_{\alpha_k})}$&$\widetilde{\mathbf{MK}}_{0,1,3}^{(A^i_{\alpha_i} 
A^j_{\alpha_j}/A^k_{\alpha_k})}$&$\widetilde{\mathbf{MK}}_{0,1,3,4}^{(A^i_{\alpha_i
} A^j_{\alpha_j}/A^k_{\alpha_k})}$& 
$\widetilde{\widetilde{\mathbf{MK}}}_{0,1,3}^{(A^i_{\alpha_i} 
A^j_{\alpha_j}/A^k_{\alpha_k})}$ 
&$\widetilde{\widetilde{\mathbf{MK}}}_{0,1,3,4}^{(A^i_{\alpha_i} 
A^j_{\alpha_j}/A^k_{\alpha_k})}$\\
   \hline
    1&\hphantom{9}2&\hphantom{9}1&10&9&10\\
2&10&\hphantom{9}8&\hphantom{9}2&4&\hphantom{9}2\\
3&10&10&10&1&\hphantom{9}1\\
4&\hphantom{9}7&\hphantom{9}4&\hphantom{9}1&1&\hphantom{9}7\\
5&\hphantom{9}6&\hphantom{9}6&\hphantom{9}8&6&\hphantom{9}8\\
\hline
\end{tabular}
\end{center}
\end{table*}
   
   
   Рассматриваемые МК характеризуются 
сле\-ду\-ющи\-ми особенностями:
   \begin{itemize}
   \item 
$\mathbf{MK}_{0,1,3}^{(A^i_{\alpha_i} A^j_{\alpha_j}/A^k_{\alpha_k})}$~--- 
задействованы~3~масштабных уровня, один пакет (VASP),
   МК может применяться для скрининга (универсальна), число арифметических 
операций (условно) 10~млн, точность расчетов~0,1;
\item 
$\widetilde{\mathbf{MK}}_{0,1,3}^{(A^i_{\alpha_i} 
A^j_{\alpha_j}/A^k_{\alpha_k})}$~--- задействованы~3~масштабных уровня, два 
пакета (VASP и~Material Studio), МК может применяться для скрининга 
(универсальна), число арифметических операций (условно) 15~млн, точность 
расчетов~0,15;
\item 
$\widetilde{\mathbf{MK}}_{0,1,3,4}^{(A^i_{\alpha_i} 
A^j_{\alpha_j}/A^k_{\alpha_k})}$~--- задействованы~4~масштабных уровня, три 
пакета (VASP, Material Studio, SRIM), МК может применяться для скрининга 
(универсальна), число арифметических операций (условно) 30~млн, точность 
расчетов~0,01;
\item $\widetilde{\widetilde{\mathbf{MK}}}_{0,1,3}^{(A^i_{\alpha_i} 
A^j_{\alpha_j}/A^k_{\alpha_k})}$~---  задействованы~3~масштабных уровня, три 
пакета (VASP, Material Studio и~SRIM), не универсальна, число арифметических 
операций (условно) 30~млн, точность расчетов~0,015;
\item 
$\widetilde{\widetilde{\mathbf{MK}}}_{0,1,3,4}^{(A^i_{\alpha_i} 
A^j_{\alpha_j}/A^k_{\alpha_k})}$~--- задействованы~4~масштабных уровня, один 
пакет (VASP), не универсальна, число арифметических операций (условно) 
20~млн, точность расчетов~0,01.
\end{itemize}
   
   Пусть на этапе предсказательного моделирования следует сравнить эти пять 
возможных МК, оценки которых по критериям $K_1, \ldots , K_5$ приведены 
в~табл.~1, и~выбрать лучшие с~точки зрения лица, принимающего решение, 
альтернативы.
   

   
   Из содержательного смысла задачи следует, что значения по 
критериям~$K_1$ и~$K_2$ следует минимизировать, а по критериям~$K_3$, $K_4$ 
и~$K_5$~--- максимизировать.
   
   В табл.~2 представлены оценки по критериям $K_1,\ldots ,K_5$ этих пяти 
альтернатив (т.\,е.\ пяти рассматриваемых МК), полученные после линейного 
преобразования шкал, сохраняющего упорядочение по предпочтению для 
каждого критерия. При этом лучшему значению соответствует более высокая 
оценка.
   
    
  
   
   По векторному отношению доминирования альтернатива $s_1\hm= (2, 10, 10, 
7, 6)$ предпочтительней альтернативы $s_2\hm = (1, 8, 10, 4, 6)$.
   
   Альтернативы $s_3\hm = (10, 2, 10, 1, 8)$, $s_4\hm = (9, 4, 1, 1, 6)$ и~$s_5\hm = 
(10, 2, 1, 7, 8)$ не сравнимы между собой и~не сравнимы с~альтернативой~$s_1$.
   
   
   Следовательно, без дополнительной информации о~предпочтениях лицо, 
принимающее решение, не может выбрать лучший вариант из четырех МК, 
именно из 
$\mathbf{MK}_{0,1,3}^{(A^i_{\alpha_1}A^j_{\alpha_j}/A^k_{\alpha_k})}$,
   $\widetilde{\mathbf{MK}}_{0,1,3,4}^{(A^i_{\alpha_1}A^j_{\alpha_j}/ 
A^k_{\alpha_k})}$, 
$\widetilde{\widetilde{\mathbf{MK}}}_{0,1,3}^{(A^i_{\alpha_1}A^j_{\alpha_j}/A^k
_{\alpha_k})}$ 
и~$\widetilde{\widetilde{\mathbf{MK}}}_{0,1,3,4}^{(A^i_{\alpha_1}A^j_{\alpha_j}
/A^k_{\alpha_k})}$.
   
   Считаем, что в~этом модельном примере оценки в~табл.~2 отражают также 
и~результат приведения значений критериев к~однородным шкалам и~обосно\-ва\-но 
применение линейной свертки
   $$
   f(x_1,\ldots , x_5)=\sum\limits^5_{i=1} \alpha_i x_i\,.
   $$
   
   Пусть в~результате взаимодействия эксперта, аналитика и~лица, 
принимающего решение, для конкретной задачи весовые коэффициенты 
критериев составят соответственно
   $\alpha\hm=1$, $\alpha_2\hm=0{,}5$, $\alpha_3\hm=1$, 
   $\alpha_4\hm=2$ и~$\alpha_5\hm=0{,}5$,
т.\,е.\ в~этой задаче существенно, чтобы превалировало собственное программное 
обеспечение, была достаточная точ\-ность вы\-чис\-ле\-ний и~выполнялось условие 
уни\-вер\-саль\-ности. Тогда лучшей альтернативой будет~$s_1$, т.\,е.\  
$\mathbf{MK}_{0,1,3}^{(A^i_{\alpha_i}A^j_{\alpha_j}/A^k_{\alpha_k})}$, затем 
в~порядке убывания предпочтительности: $s_5$, $s_3$, $s_2$ и~$s_4$, не\-смот\-ря 
на то что эта $\mathbf{MK}$ не предпочтительней их в~точности вы\-чис\-ле\-ний.
   
   Если же, например, весовые коэффициенты критериев составят 
соответственно
   $\alpha_1\hm=1$, $\alpha_2\hm=0{,}5$, $\alpha_3\hm=2$, 
   $\alpha_4\hm=1$ и~$\alpha_5\hm=1{,}5$,
т.\,е.\ в~задаче наиболее существенно, чтобы выполнялось условие 
универсальности и~было большее чис\-ло мас\-штаб\-ных уровней, то лучшей 
альтернативой будет~$s_3$, т.\,е.\ 
$\widetilde{\mathbf{MK}}_{0,1,3,4}^{(A^i_{\alpha_i}A^j_{\alpha_j}/ 
A^k_{\alpha_k})}$, затем в~порядке убывания предпочтительности: $s_1$, $s_2$, 
$s_5$ и~$s_4$.

   На следующих этапах решения задачи моделирования для рассматриваемой 
практической задачи можно попытаться расширить набор критериев для оценки 
МК, сохраняя все естественные требования к~набору критериев. При этом 
значения критерия оценки МК могут быть результатом оценки част\-ных критериев 
БК. Такой подход приведет к~более точным оценкам критериев МК 
и,~следовательно, к~более обоснованному результату выбора.
   
   В связи с~тем, что разработанная технология дает возможность формализовать 
процесс отбора наилучших в~определенном смысле МК и~позволяет решить 
поставленную задачу с~учетом предпочтений экспертов и~лиц, принимающих 
решение, можно говорить о том, что данный подход может быть применен при 
создании че\-ло\-ве\-ко-ма\-шин\-ных сис\-тем автоматизированного 
проектирования. В~дальнейшем при расширении набора критериев можно 
использовать и~последовательно наращивать в~автоматическом режиме 
структуры данных, по\-лу\-ча\-емых при помощи отобранных МК.
   
   Помимо всего прочего предложенный выше подход может применяться в~так 
называемых обратных задачах выбора, в~которых заранее указаны желательные 
значения частных критериев, описывающих МК, или диапазоны их изменения.
   
  {\small\frenchspacing
 {%\baselineskip=10.8pt
 \addcontentsline{toc}{section}{References}
 \begin{thebibliography}{99}
    \bibitem{1-ab}
    \Au{Абгарян К.\,К.} Информационная технология по\-стро\-ения многомасштабных моделей 
в~задачах вычислительного материаловедения~// Системы высокой доступности, 2018. Т.~15. 
№\,2. С.~9--15.
\bibitem{2-ab}
    \Au{Бродский Ю.\,И.} Модельный синтез и~мо\-дель\-но-ори\-ен\-ти\-ро\-ван\-ное 
    программирование.~---  М.: ВЦ РАН, 2013. 142~c.
\bibitem{3-ab}
    \Au{Abgaryan K.\,K., Mutigullin~I.\,V., Reviznikov~D.\,L.}
    Theoretical investigation of 2DEG 
concentration and mobility in the AlGaN/GaN heterostructures with various Al concentrations~// 
Phys. Status Solidi~C, 2015. Vol.~12. Iss.~12. P.~1376--1382.
\bibitem{4-ab}
    \Au{Абгарян К.\,К.}
    Задачи оптимизации наноразмерных полупроводниковых гетероструктур~// 
Известия вузов. Материалы электронной техники, 2016. Т.~19. №\,2. C.~108--114.
\bibitem{5-ab}
    \Au{Абгарян К.\,К., Ревизников~Д.\,Л.}
    Численное моделирование распределения носителей 
заряда в~наноразмерных полупроводниковых гетероструктурах с~учетом поляризационных 
эффектов~// ЖВМ и~МФ, 2016. Т.~56. №\,1. С.~155--166.
\bibitem{6-ab}
    \Au{Кини Р.\,Л., Райфа~Х.} Принятие решений при многих критериях предпочтения 
    и~замещения~/ Пер. с~англ.~--- М.: Радио и~связь, 1981. 559~с. 
    (\Au{Keeney~R., Raiffa~H.} Decisions with 
multiple objectives: Preferences and value tradeoffs.~--- New York, NY, USA: Wiley, 
1975. 592~p.)
\bibitem{7-ab}
    \Au{Подиновский В.\,В.} Введение в~теорию важности критериев.~--- М.: Физматлит, 2007. 64~c.
\bibitem{8-ab}
    \Au{Фишберн П.} Теория полезности для принятия решений~/ Пер. с~англ.~--- М.: 
    Наука, 1978. 
(\Au{Fishburn~P.\,C.} Utility theory for decision making.~--- New York, NY, USA: 
Wiley, 1970. 224~p.)
\bibitem{9-ab}
    \Au{Осипова В.\,А., Алексеев~Н.\,С.}
     Математические методы поддержки принятия решений.~--- 
М.: ИНФРА-М, 2019. 133~с.
\bibitem{10-ab}
    \Au{Петровский А.\,Б.} Теория принятия решений.~--- М.: Академия, 2009. 391~с.
 \end{thebibliography}

 }
 }

\end{multicols}

\vspace*{-6pt}

\hfill{\small\textit{Поступила в~редакцию 15.11.18}}

\vspace*{6pt}

%\pagebreak

%\newpage

%\vspace*{-28pt}

\hrule

\vspace*{2pt}

\hrule

\vspace*{-2pt}

\def\tit{APPLICATION OF~DECISION SUPPORT METHODS 
FOR~THE~MULTICRITERIAL SELECTION OF~MULTISCALE~COMPOSITIONS}


\def\titkol{Application of~decision support methods 
for~the~multicriterial selection of~multiscale 
compositions}

\def\aut{K.\,K.~Abgaryan$^{1,2}$ and V.\,A.~Osipova$^2$}

\def\autkol{K.\,K.~Abgaryan and V.\,A.~Osipova}

\titel{\tit}{\aut}{\autkol}{\titkol}

\vspace*{-11pt}


\noindent
     $^1$Dorodnicyn Computing Center, Federal Research Center ``Computer Science 
and Control'' of the Russian\linebreak
$\hphantom{^1}$Academy of Sciences, 44/2~Vavilov Str., Moscow, 119333, 
Russian Federation
   
   \noindent
    $^2$Moscow Aviation Institute (National Research University), 4~Volokolamskoe Shosse, Moscow 
125080, Russian\linebreak
$\hphantom{^1}$Federation

\def\leftfootline{\small{\textbf{\thepage}
\hfill INFORMATIKA I EE PRIMENENIYA~--- INFORMATICS AND
APPLICATIONS\ \ \ 2019\ \ \ volume~13\ \ \ issue\ 2}
}%
 \def\rightfootline{\small{INFORMATIKA I EE PRIMENENIYA~---
INFORMATICS AND APPLICATIONS\ \ \ 2019\ \ \ volume~13\ \ \ issue\ 2
\hfill \textbf{\thepage}}}

\vspace*{3pt}

     
   
\Abste{The article discusses the use of decision-making support methods for the task of 
selecting multiscale compositions (MC)~--- computational analogues of multiscale 
physical and mathematical models created for analyzing\linebreak\vspace*{-12pt}}

\Abstend{various heterogeneous 
processes associated with the formation of new composite materials with predetermined 
properties. When solving specific problems, different multiscale models and their 
corresponding MC can be constructed. The question arises of comparing these models 
and assessing their ``effectiveness'' for specific problem. On the stage of predictive 
modeling, the authors propose a~methodology for comparison of multiscale models 
through evaluation and selection of appropriate MC using methods of decision-making 
support under multiple criteria. As an illustration of the possibility of choosing the best 
alternative in the presence of additional information on evaluation criteria of 
MC, a~model example associated with the study of electronic and structural 
properties of thin films InN (GaN) on silicon substrates is considered.}
   
    \KWE{multiscale modeling; decision theory; quality criteria; alternative; decision support 
methods; multiple criteria; value function}
 
\DOI{10.14357/19922264190207}

%\vspace*{-14pt}

%\Ack
%\noindent



\vspace*{-6pt}

  \begin{multicols}{2}

\renewcommand{\bibname}{\protect\rmfamily References}
%\renewcommand{\bibname}{\large\protect\rm References}

{\small\frenchspacing
 {%\baselineskip=10.8pt
 \addcontentsline{toc}{section}{References}
 \begin{thebibliography}{99}
\bibitem{1-ab-1}
    \Aue{Abgaryan, K.\,K.} 2018. Informatsionnaya tekhnologiya postroeniya 
mnogomasshtabnykh modeley v~zadachakh vychislitel'nogo materialovedeniya 
[Information technology is the construction of multiscale models in problems of 
computational materials science]. \textit{Sistemy vysokoy do\-stup\-nosti} [High 
Availability Systems] 15(2):9--15.
\bibitem{2-ab-1}
    \Aue{Brodskij, Yu.\,I.} 2013. \textit{Model'nyy sintez i~model'no-orientirovannoe 
programmirovanie} [Model synthesis and model-oriented programming]. Moscow: CC 
RAS. 142~p.
\bibitem{3-ab-1}
    \Aue{Abgaryan, K.\,K., I.\,V.~Mutigullin, and D.\,I.~Reviznikov.} 2015. Theoretical 
investigation of 2DEG concentration and mobility in the AlGaN/GaN heterostructures 
with various Al concentrations. \textit{Phys. Status Solidi~C} 12(12):1376--1382.
\bibitem{4-ab-1}
    \Aue{Abgaryan, K.\,K.} 2016. Zadachi optimizatsii nanorazmernykh 
poluprovodnikovykh geterostruktur [Оptimization problems of nanosized 
semiconductor heterosrtuctures]. \textit{Proceedings of Higher Schools,
Materials 
of Electronics Engineering} 19(2):108--114.
\bibitem{5-ab-1}
    \Aue{Abgaryan, K.\,K., and D.\,L.~Reviznikov.} 2016. Numerical simulation of the 
distribution of charge carrier in nanosized semiconductor heterostructures 
with account  for polarization effects. \textit{Comp. Math. Math. Phys.} 56(1):161--172.
\bibitem{6-ab-1}
    \Aue{Keeney, R., and H.~Raiffa.} 1975. \textit{Decisions with multiple objectives: 
    Preferences  and value tradeoffs}. New York, NY: Wiley, 1975. 592~p.
\bibitem{7-ab-1}
    \Aue{Podinovskiy, V.\,V.}
    2007. \textit{Vvedenie v~teoriyu vazhnosti kriteriev} [Introduction to 
criteria importance theory]. Moscow: Fizmatlit. 64~p.
\bibitem{8-ab-1}
    \Aue{Fishburn, P.\,C.} 
    1970. \textit{Utility theory for decision making.} New York, NY: Wiley. 224~p.
\bibitem{9-ab-1}
    \Aue{Osipova, V.\,A., and N.\,S.~Alekseev.} 2019. \textit{Matematicheskie metody podderzhki 
prinyatiya resheniy} [Mathematical decision support methods]. Moscow: INFRA-M. 
133~p. 
\bibitem{10-ab-1}
    \Aue{Petrovskiy, A.\,B.} 2009. \textit{Teoriya prinyatiya resheniy} 
    [Decision making theory]. Moscow: Akademiya. 391~p.
\end{thebibliography}

 }
 }

\end{multicols}

\vspace*{-7pt}

\hfill{\small\textit{Received November 15, 2018}}

%\pagebreak

\vspace*{-22pt}

   
   \Contr
   
   \vspace*{-4pt}
   
   \noindent
   \textbf{Abgaryan Karine K.} (b.\ 1963)~--- Doctor of Science in physics and 
mathematics, Head of Department, Dorodnicyn Computing Center, Federal Research 
Center ``Computer Science and Control'' of the Russian Academy of Sciences, 
40~Vavilov Str., Moscow 119333, Russian Federation; Head of Department, Moscow 
Aviation Institute (National Research University), 4~Volokolamskoe Shosse, Moscow 
125080, Russian Federation; \mbox{kristal83@mail.ru}
   
       %\vspace*{3pt}
   
   \noindent
   \textbf{Osipova Victoria A.} (b.\ 1963)~--- Candidate of Science (PhD) in physics 
and mathematics, professor, Moscow Aviation Institute (National Research University), 
4~Volokolamskoe Shosse, Moscow 125080, Russian Federation; 
\mbox{victoria.a.osipova@gmal.com} 
\label{end\stat}

\renewcommand{\bibname}{\protect\rm Литература}  