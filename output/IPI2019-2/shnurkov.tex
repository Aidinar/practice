\def\stat{shnurkov}

\def\tit{ИССЛЕДОВАНИЕ ПРОБЛЕМЫ ОПТИМАЛЬНОГО УПРАВЛЕНИЯ ЗАПАСОМ 
ДИСКРЕТНОГО ПРОДУКТА В~СТОХАСТИЧЕСКОЙ МОДЕЛИ РЕГЕНЕРАЦИИ 
С~НЕПРЕРЫВНО ПРОИСХОДЯЩИМ ПОТРЕБЛЕНИЕМ И~СЛУЧАЙНОЙ ЗАДЕРЖКОЙ 
ПОСТАВКИ}

\def\titkol{Исследование проблемы оптимального управления запасом 
дискретного продукта в~стохастической модели}
% регенерации  с~непрерывно происходящим потреблением и~случайной задержкой  поставки}

\def\aut{П.\,В.~Шнурков$^1$, Н.\,А.~Вахтанов$^2$}

\def\autkol{П.\,В.~Шнурков, Н.\,А.~Вахтанов}

\titel{\tit}{\aut}{\autkol}{\titkol}

\index{Шнурков П.\,В.}
\index{Вахтанов Н.\,А.}
\index{Shnurkov P.\,V.}
\index{Vakhtanov N.\,A.}


%{\renewcommand{\thefootnote}{\fnsymbol{footnote}} \footnotetext[1]
%{Разделы~1--3 и~5 данной работы выполнены при финансовой поддержке РНФ (проект  
%16-18-10004), раздел~4 выполнен по плановой теме.}}


\renewcommand{\thefootnote}{\arabic{footnote}}
\footnotetext[1]{Национальный исследовательский университет <<Высшая школа экономики>>, pshnurkov@hse.ru}
\footnotetext[2]{Национальный исследовательский университет <<Высшая школа экономики>>, Vakhtanov1997@mail.ru}

\vspace*{-12pt}


 

\Abst{Рассматривается проблема оптимального управления запасом дискретного продукта 
в~схеме регенерации с~пуассоновским потоком требований потребителей. В~исследуемой 
сис\-те\-ме допускается отложенный спрос, объем которого ограничен заданной величиной. 
В~качестве параметра управления рассматривается уровень запаса, при достижении которого 
необходимо делать заказ на пополнение. Показателем эффективности управления служит 
средняя удельная прибыль, полученная на одном периоде регенерации. Задача оптимального 
управления решается на основе утверждения об экстремуме дроб\-но-ли\-ней\-но\-го 
интегрального функционала на множестве дискретных вероятностных распределений.}

\KW{управление запасом дискретного продукта; управ\-ля\-емый регенерирующий процесс; 
экстремальная задача для дроб\-но-ли\-ней\-но\-го интегрального функционала}

\DOI{10.14357/19922264190208}
  
%\vspace*{4pt}


\vskip 10pt plus 9pt minus 6pt

\thispagestyle{headings}

\begin{multicols}{2}

\label{st\stat}


\section{Введение}

  Стохастические модели регенерации, используемые для исследования задач 
управления запасом, рассматривались в~целом ряде пуб\-ли\-ка\-ций. В~част\-ности, 
в~работах~[1, 2] были рассмотрены различные варианты регенерационных 
моделей для сис\-тем управ\-ле\-ния запасом непрерывного продукта. В~работе~[3] 
была исследована специальная версия модели регенерации, описывающая 
систему управ\-ле\-ния запасом непрерывного продукта, в~которой 
непосредственное пополнение запаса происходит не мгновенно, а~в~течение 
определенного периода времени, называемого периодом реального пополнения. 
Рассмотренная в~[3] версия стохастической модели регенерации идейно 
связана с~классической детерминированной моделью управ\-ле\-ния запасом, 
изложенной, например, в~[4].
  %
  Во всех упомянутых работах исследуются задачи управления запасом 
непрерывного продукта. В~моделях такого вида множество значений 
основного случайного процесса, описывающего объем запаса в~системе, 
представляет собой некоторое подмножество множества вещественных чисел. 
Примерами непрерывных продуктов, находящихся в~реальных сис\-те\-мах, могут 
служить вода, нефть и~нефтепродукты, зерно и~др. Однако существует ряд 
важных продуктов, объем которых измеряется в~дискретных величинах. 
К~таковым относятся многие потребительские товары, прежде всего товары 
бытовой техники, продовольственные товары и~ряд других. Таким образом, 
рассмотрение математических моделей управ\-ле\-ния запасом дискретного 
продукта является актуальной проблемой прикладной математики.
  
  В зарубежной научной литературе по теории управления запасами 
стохастические модели регенерации известны достаточно давно. Модели,\linebreak 
в~которых моменты регенерации совпадают с~моментами пополнения запаса до 
заданного фиксированного уровня, а~показатель эф\-фек\-тив\-ности управ\-ле\-ния 
пред\-став\-ля\-ет собой средние удельные за\-тра\-ты на периоде регенерации, 
упоминаются в~обзорах результатов, приведенных в~[4, 5]. Однако среди 
научных исследований, выполненных в~последние десятилетия, работы по 
управлению запасом дискретного продукта в~схеме регенерации не 
встречаются.

  \begin{figure*} %fig
  \vspace*{1pt}
    \begin{center}  
  \mbox{%
 \epsfxsize=125.902mm 
 \epsfbox{shn-1.eps}
 }

\vspace*{6pt}

  {\small Эволюция объема запаса продукта в~рассматриваемой системе}
  \end{center}
  \vspace*{-14pt}
  \end{figure*}


\section{Общее описание функционирования 
рассматриваемой системы управления дискретным запасом}

  Будем исследовать некоторую торговую систе-\linebreak му (склад), предназначенную 
для временного хранения и~поставки потребителю определенного однотипного 
товара, объем которого измеряется в~дискретных целочисленных величинах. 
Предполагается, что в~начальный момент времени сис\-те\-ма пол\-ностью 
заполнена и~содержит~$N$ единиц продукта. Потребление товара 
осуществляется в~случайные моменты поступления требований (покупателей). 
Моменты поступления требований образуют прос\-тей\-ший (пуассоновский) 
поток с~известным па\-ра\-мет\-ром $\lambda\hm>0$. В~момент по\-ступ\-ле\-ния 
очередного требования происходит потребление одной единицы продукта. 
Такое по\-треб\-ле\-ние осуществляется мгновенно.
  
  В рассматриваемой системе периодически происходит пополнение запаса, 
которое осуществляется сле\-ду\-ющим образом. Заказ очередной партии товара 
происходит в~момент времени, когда количество товара на складе достигает 
уровня~$r$, где $r\hm\leq N$~--- некоторая целочисленная величина, которая 
в~дальнейшем будет играть роль решения (управ\-ле\-ния). Заметим, что 
параметр~$r$ имеет вероятностную природу. Формальное описание процедуры 
выбора этого параметра приведено в~разд.~3. Период времени от момента 
заказа товара до прибытия новой партии будем называть периодом задержки 
по\-став\-ки или просто периодом задержки. Длительность этого периода является 
случайной величиной, имеющей заданное распределение $H_r(x)$, которое, 
вообще говоря, может зависеть от значения па\-ра\-мет\-ра управ\-ле\-ния~$r$. 
Непосредственное пополнение запаса происходит мгновенно, в~момент 
окончания периода задержки.
  
  В период задержки поставки потребление товара продолжается согласно 
описанным правилам. После того как реальный запас товара будет 
израсходован, этот запас будет описываться отрицательными величинами, что 
соответствует объему неудовлетворенного спроса или дефицита. При этом 
каждое по\-сту\-па\-ющее требование принимается на учет и~удовлетворяется при 
сле\-ду\-ющем пополнении. Такие требования образуют так на\-зы\-ва\-емый 
отложенный спрос. Объем отложенного спроса в~данной модели не может 
превышать заданной величины~$N_0$. Если до момента очередного 
пополнения объем отложенного спроса достигает максимально допустимого 
уровня~$N_0$, то все по\-сту\-па\-ющие после этого требования теряются. 
Предполагается, что пополнение запаса организовано так, что отложенный 
спрос пол\-ностью удовлетворяется, а~запас товара пополняется до исходного 
уровня~$N$. Дальнейшее функционирование сис\-те\-мы происходит независимо 
от прошлого и~в~соответствии с~описанными выше правилами. На рисунке 
приведена иллюстрация воз\-мож\-ной эволюции объема запаса товара в~сис\-те\-ме. 
Заметим при этом, что параметр управ\-ле\-ния~$r$ может принимать любое 
целочисленное значение в~пределах $-N_0\hm\leq r\hm\leq N$.
  

\section{Формальное построение математической модели, 
описывающей функционирование рассматриваемой системы}

  Будем предполагать, что все вводимые в~дальнейшем стохастические 
объекты определены на некотором вероятностном про\-стран\-ст\-ве $(\Omega, 
\mathcal{A}, P)$,\linebreak которое пред\-став\-ля\-ет собой формальную конструкцию, 
описывающую реализуемый случайный эксперимент с~реальной сис\-те\-мой. 
Прежде всего введем случайный процесс $\xi(t)\hm= \xi(\omega, t)$, 
$\omega\hm\in \Omega$,\linebreak $t\hm\in T\hm= [0,\infty)$, который будет служить 
математической моделью функционирования рассматриваемой системы. 
Предположим, что значение этого процесса в~произвольный момент 
времени~$t$ пред\-став\-ля\-ет собой объем запаса в~сис\-те\-ме, причем учитывается 
как реальный запас, так и~дефицит, т.\,е.\ величина отложенного спроса. Таким 
образом, множество значений процесса~$\xi(t)$ конечно: $X\hm= \{-N_0, -
N_0\hm+1,\ldots , -1, 0, 1, \ldots , N\hm-1, N\}$.
  
  Моментами изменения состояний процесса~$\xi(t)$ являются моменты 
по\-треб\-ле\-ния (по\-ступ\-ле\-ния требований от по\-тре\-би\-те\-лей), если величина 
отложенного спроса не превышает значение~$N_0$, а~так\-же моменты 
непосредственного пополнения запаса. Для определенности будем 
предполагать, что траектории процесса непрерывны справа.
  
  Обозначим через $t_n\hm= t_n(\omega)$, $\omega\hm\in \Omega$, $n\hm=0, 1, 
2, \ldots $, $t_0\hm=0$, случайные моменты пополнения запаса. Согласно 
предположениям о~поведении данной сис\-те\-мы, описанным в~предыду\-щем 
разделе, $\xi(t_n)\hm= N$, $n\hm=0,1,2,\ldots$ Кроме того, после каж\-до\-го 
момента пополнения эволюция сис\-те\-мы продолжается независимо от прош\-ло\-го и~по тем же закономерностям. Отсюда следует, что процесс~$\xi(t)$ является 
ре\-ге\-не\-ри\-ру\-ющим~[6], а~моменты~$t_n$, $n\hm=0,1,2,\ldots$, представляют 
собой моменты его регенерации. Периоды регенерации данного процесса 
складываются из двух независимых час\-тей:
  \begin{equation}
  \Delta_n=t_{n+1}-t_n =\Delta_n^{(0)}+\Delta_n^{(1)}\,,
  \label{e1-sk}
  \end{equation}
где $\Delta_n^{(0)}$~--- случайное время от момента очередного пополнения до 
момента заказа следующей по\-став\-ки;  $\Delta_n^{(1)}$~--- случайная 
дли\-тель\-ность задержки по\-став\-ки (время от момента заказа до сле\-ду\-юще\-го 
пополнения запаса).

  В соответствии с~принятыми предположениями при выполнении 
дополнительного условия, что параметр управ\-ле\-ния принимает фиксированное 
значение~$r$, случайная величина~$\Delta_n^{(0)}$ имеет распределение 
Эрланга порядка $N\hm-r$, а~случайная величина~$\Delta_n^{(1)}$~--- 
заданное распределение~$H_r(x)$. 
  
  Теперь опишем процедуру управ\-ле\-ния случайным процессом~$\xi(t)$.
  
  В качестве параметра управления будем рас\-смат\-ри\-вать тот уровень 
запаса~$r$, при котором происходит заказ очередной партии товара. Решение 
об управлении принимается в~начальный момент каждого периода регенерации, 
после очередного пополнения запаса, т.\,е.\ в~моменты времени $t_n+0$, 
$n\hm=0,1,2,\ldots$ В~результате такого решения определяется уровень~$r$, 
при котором будет осуществляться очередной заказ товара. Таким образом, 
множество допустимых значений па\-ра\-мет\-ра управ\-ле\-ния~$U$ оказывается 
конечным:
  $$
  r\in U=\left\{ N, N-1, \ldots , 0, 1, \ldots,  -N_0\right\}\,.
  $$
  
  Зададим на множестве~$U$ совокупность всевозможных дискретных 
вероятностных распределений~$\Gamma_d$. Каж\-дое дискретное вероятностное 
распределение, принадлежащее этой со\-во\-куп\-ности,\linebreak пред\-став\-ля\-ет собой вектор 
$$
\alpha= (\alpha_N, \alpha_{N-1}, \ldots , \alpha_0, \alpha_{-1},\ldots , \alpha_{-
N_0})\,,
$$
 удовлетворяющий условиям: 
 $$
 \alpha_i\hm\geq 0\,,\ i \in U,\enskip 
\sum\limits_{i\in U} \alpha_i=1\,.
$$
 Любое фиксированное 
распределение~$\alpha$ задает вероятностную стратегию управ\-ле\-ния 
в~сле\-ду\-ющем смыс\-ле: в~каж\-дый момент принятия решения значение 
управления выбирается в~соответствии с~этим распределением. Иначе говоря, 
значение па\-ра\-мет\-ра управ\-ле\-ния~$r$ выбирается с~вероятностью~$\alpha_r$, 
$r\hm\in U$. Отметим, что стратегия управ\-ле\-ния одинакова для всех периодов 
регенерации. Задача оптимизации за\-клю\-ча\-ет\-ся в~на\-хож\-де\-нии стратегии 
управ\-ле\-ния $\alpha^*\hm\in \Gamma_d$, которая доставляет экстремум 
некоторому стационарному стоимостному показателю эф\-фек\-тив\-ности.
  
  Указанный показатель зависит от исходных стоимостных характеристик 
модели. Перечислим эти характеристики, которые предполагаются известными:
  \begin{itemize}
  \item доход от реализации единицы продукции~$c_0$;
  \item расходы на хранение единицы продукции за единицу времени~$c_1$;
  \item расходы на приобретение единицы продукции~$c_2$;
  \item расходы, связанные с~дефицитом единицы продукции за единицу 
времени,~$c_3$;
  \item расходы, связанные с~потерей~$i$~клиентов,~$c_4^{(i)}$, $i\hm= 
1,2,\ldots$
  \end{itemize}
  
  Введенные стоимостные характеристики определяют все основные виды 
доходов и~за\-трат, воз\-ни\-ка\-ющих в~моделях управ\-ле\-ния запасами.

\section{Постановка задачи оптимального управления}

  Прежде всего введем некоторый аддитивный стоимостный функционал для 
рас\-смат\-ри\-ва\-емой модели. Пусть $\gamma(t)\hm= \gamma(\omega, t)$, 
$\omega\hm\in \Omega$, $t\hm\in T\hm= [0,\infty)$,~--- случайная прибыль, 
полученная в~результате функционирования описанной выше стохастической 
сис\-те\-мы на интервале времени $[0,t]$, $\gamma(0)\hm= \gamma_0$~--- заданная 
величина. Процесс зависит от траектории основного процесса~$\xi(t)$, и~его 
значения определяются так\-же исходными стоимостными характеристиками 
$c_0$, $c_1$, $c_2$, $c_3$, $c_4^{(i)}$, $i\hm=1,2,\ldots$ Полное описание 
траекторий процесса~$\gamma(t)$ сложно и~не является необходимым для 
решения рас\-смат\-ри\-ва\-емой задачи оптимального управ\-ле\-ния. Аналогичные 
стоимостные аддитивные функционалы известны в~научной литературе. Общая 
схема их по\-стро\-ения для управ\-ля\-емых марковских и~полумарковских 
случайных процессов изложена в~классических работах~[7, 8]. В~настоящей 
работе потребуется исследование вероятностных характеристик, связанных 
с~приращениями процесса~$\gamma(t)$ на периодах регенерации основного 
процесса~$\xi(t)$.
  
  Обозначим через $\gamma_n\hm= \gamma(t_n)$, $n\hm=0,1,2,\ldots$,\linebreak
   значения 
процесса~$\gamma(t)$ в~моменты регенерации, полагая $\gamma(t_n+0)\hm= 
\gamma(t_n)$, $n\hm=0,1,2,\ldots$ Пусть $\Delta \gamma_n\hm= 
\gamma_{n+1}\hm-\gamma_n$ есть приращение прибыли на периоде 
регенерации $(t_n, t_{n+1}]$, $n\hm= 0,1,2,\ldots$ Обозначим далее через 
$E_\alpha(\Delta \gamma_n)$ математическое ожидание приращения прибыли 
на периоде регенерации при условии, что стратегия управ\-ле\-ния 
процессом~$\xi(t)$ определяется распределением $\alpha\hm\in \Gamma_d$. 
Аналогично величина $E_\alpha(\Delta t_n)\hm= E_\alpha(t_{n+1}\hm-t_n)$ 
представляет собой математическое ожидание дли\-тель\-ности периода 
регенерации при стратегии управ\-ле\-ния $\alpha\hm\in \Gamma_d$.  Введем 
также обозначение $E_\alpha \gamma(t)$ для математического ожидания всей 
прибыли, полученной на периоде времени $[0,t]$ при стратегии управ\-ле\-ния 
$\alpha\hm\in \Gamma_d$.
  
  В предположении, что решение на периоде регенерации фиксировано, т.\,е.\ 
параметр управ\-ле\-ния принимает значение $r\hm\in U$, будем обозначать 
соответствующие услов\-ные математические ожидания через $E_r(\Delta 
\gamma_n)$ и~$E_r(\Delta t_n)$, $n\hm=0,1,2,\ldots$
  
  Изменение аддитивного функционала $\gamma(t)$ на отдельном периоде 
регенерации $(t_n, t_{n+1}]$ складывается из двух со\-став\-ля\-ющих:
  \begin{equation}
  \Delta \gamma_n=\Delta \gamma_n^{(0)}+\gamma_n^{(1)}\,,
  \label{e2-sk}
  \end{equation}
где $\Delta \gamma_n^{(0)}\hm= \gamma(t_{n+1}-0)\hm- \gamma(t_n)$~--- 
приращение функционала~$\gamma(t)$ на открытом интервале $(t_n, t_{n+1})$,\linebreak 
в~на\-сто\-ящей модели это приращение зависит от траектории процесса~$\xi(t)$ 
на указанном интервале и~определяется па\-ра\-мет\-ра\-ми $c_0$, $c_1$, $c_2$, 
$c_3$, $c_4^{(i)}$, $i\hm= 1,2,\ldots$; $\gamma_n^{(1)}\hm= 
\gamma(t_{n+1})\hm- \gamma(t_{n+1}\hm-0)$~--- приращение 
функционала~$\gamma(t)$ в~момент времени~$t_{n+1}$, т.\,е.\ его мгновенное 
изменение в~указанный момент времени. В~рас\-смат\-ри\-ва\-емой модели 
величина~$\gamma_n^{(1)}$ пред\-став\-ля\-ет собой затраты, связанные 
с~очередным пополнением запаса. Отсюда следует, что данная величина 
отрицательна, а~ее конкретное значение зависит от объема пополнения запаса 
в~момент~$t_{n+1}$ и~определяется равенством 
$$
\gamma_n^{(1)}= -
c_2\left[\xi\left(t_{n+1}\right)- \xi\left( t_{n+1} -0\right)\right],\enskip
 n\hm= 0,1,2,\ldots
 $$

С учетом отмеченных особенностей модели значение процесса~$\gamma(t)$ 
в~произвольный момент времени $t\hm>0$ также складывается из двух 
со\-став\-ля\-ющих:
\begin{equation}
\gamma(t)=\gamma^{(0)}(t)+\gamma^{(1)}(t)\,,
\label{e3-sk}
\end{equation}
где $\gamma^{(0)}(t)$~--- накопленная прибыль на интервале $[0,t]$, 
определяемая изменениями процесса~$\xi(t)$ на периодах регенерации, 
исключая сами моменты регенерации, входящие в~интервал $[0, t]$; 
$\gamma^{(1)}(t)$~--- накопленные за\-тра\-ты на интервале $[0,t]$, связанные 
с~пополнениями запаса в~моменты регенерации процесса~$\xi(t)$, входящие 
в~интервал $[0,t]$. 
  
  Традиционно, начиная с~первых фундаментальных работ по тео\-рии 
управления марковскими и~полумарковскими случайными процессами~[7, 8], 
в~качестве показателя эф\-фек\-тив\-ности управ\-ле\-ния рас\-смат\-ри\-ва\-лась величина
  \begin{equation}
  I_\alpha= \lim\limits_{t\to \infty} \fr{E_\alpha \gamma(t)}{t}\,,
  \label{e4-sk}
  \end{equation}
которая по содержанию пред\-став\-ля\-ет собой среднюю удельную прибыль 
в~рас\-смат\-ри\-ва\-емой стохастической модели, опре\-де\-ля\-емую при стратегии 
управ\-ле\-ния~$\alpha$. Однако в~работах~[7, 8] исследовались аддитивные 
стоимостные функционалы вида~$\gamma^{(0)} (t)$, в~которых не учитывались 
составляющие, воз\-ни\-ка\-ющие в~отдельные случайные моменты времени. Для 
того чтобы использовать показатель эффективности управ\-ле\-ния  
вида~(\ref{e4-sk}), необходимо доказать эргодическую тео\-ре\-му 
о~существовании и~явном пред\-став\-ле\-нии указанного предела для аддитивного 
функционала вида~(\ref{e3-sk}), учи\-ты\-ва\-юще\-го 
со\-став\-ля\-ющую~$\gamma^{(1)}(t)$. Докажем со\-от\-вет\-ст\-ву\-ющий результат для 
регенерирующего процесса~$\xi(t)$ и~определенного на нем стоимостного 
аддитивного функционала~$\gamma(t)$.

\smallskip

\noindent
  \textbf{Теорема.}\ \textit{Пусть~$\xi(t)$~--- управ\-ля\-емый регенерирующий 
процесс, стратегия управ\-ле\-ния которым определяется вероятностным 
распределением~$\alpha\hm\in \Gamma_d$, $\gamma(t)$~--- аддитивный 
стоимостный функционал, связанный с~процессом~$\xi(t)$, общая структура 
которого определяется соотношением}~(\ref{e3-sk}). \textit{Предположим, что 
для любой допустимой стратегии управ\-ле\-ния  выполняется условие $E_\alpha 
[\Delta t_n]\hm>0$. Тогда существует предел}
  \begin{equation}
  I_\alpha= \lim\limits_{t\to\infty} \fr{E_\alpha\gamma(t)}{t}=\fr{E_\alpha \left[ 
\Delta \gamma_n\right]}{E_\alpha \left[ \Delta t_n\right]}\,,
  \label{e5-sk}
  \end{equation}
\textit{где $\Delta\gamma_n$~--- полное изменение функционала~$\gamma(t)$ 
на периоде регенерации $(t_n, t_{n+1}]$, определяемое 
соотношением}~(\ref{e2-sk}).

\smallskip

  \noindent
  Д\,о\,к\,а\,з\,а\,т\,е\,л\,ь\,с\,т\,в\,о\,.\ \ Следуя классической тео\-рии 
восстановления~\cite{9-sk}, введем $\nu(t)\hm= \mathrm{sup} \left\{ n:\ t_n\hm\leq 
t\right\}$. Процесс~$\nu(t)$ обычно называют считающим для процесса 
восстановления, образованного моментами регенерации $\{ t_n, 
n\hm=0,1,2\ldots\}$. Пусть $H(t)\hm= E\nu(t)$~--- функция восстановления 
процесса $\{ t_n, n\hm=0,1,2\ldots\}$. Для фиксированного значения~$t$ 
случайная величина~$\nu(t)$ совпадает с~чис\-лом восстановлений на интервале 
$[0,t]$ (момент $t\hm=0$ не считается моментом восстановления).
  
  По свойству аддитивности процесса~$\gamma(t)$ можно утверж\-дать, что
  \begin{equation}
  \gamma^{(1)}(t)=\sum\limits_{n=0}^{\nu(t)-1} 
\gamma_n^{(1)}=\sum\limits_{n=1}^{\nu(t)} \gamma_{n-1}^{(1)}\,.
  \label{e6-sk}
  \end{equation}
  
  Заметим, что из свойств построенного в~данной работе регенерирующего 
процесса~$\xi(t)$ и~связанного с~ним процесса~$\gamma(t)$ следует, что для 
произвольного момента времени $t\hm>0$ и~любого фиксированного значения 
$n\hm=1,2,\ldots$ случайное событие $(\nu(t)\hm\leq n)$ не зависит от сис\-те\-мы 
событий, по\-рож\-ден\-ных случайными величинами $\left\{ \gamma_{n+1}^{(1)}, 
\gamma^{(1)}_{n+2},\ldots \right\}$. Такое свойство в~тео\-рии вероятностей 
называется не\-за\-ви\-си\-мостью от будущего~\cite{10-sk}. Действительно, 
случайная величина~$\nu(t)$ и~событие $(\nu(t)\hm\leq n)$ зависят от случайных 
моментов времени $t_1, t_2, \ldots , t_n, t_{n+1}$ и~соответствующих периодов 
регенерации $[0, t_1]$, $(t_1, t_2], \ldots , (t_n, t_{n+1}]$. В~то же время 
случайные величины $\left\{ \gamma^{(1)}_{n+1}, \gamma^{(1)}_{n+2}, 
\ldots\right\}$ связаны с~периодами регенерации $(t_{n+1}, t_{n+2}], (t_{n+2}, 
t_{n+3}],\ldots$
  
  При выполнении свойства независимости от будущего справедливо 
тож\-дест\-во Вальда~\cite[гл.~4, \S\,4]{10-sk}. Воспользовавшись этим 
результатом, получаем из~(\ref{e6-sk}):
  \begin{equation}
  E_\alpha \gamma^{(1)}(t)=E_\alpha \nu(t) E_\alpha \gamma_n^{(1)}=H(t) E_\alpha 
\gamma_n^{(1)}\,.
  \label{e7-sk}
  \end{equation}
  
  Из соотношения~(\ref{e3-sk}) с~учетом~(\ref{e7-sk}) получаем:
  \begin{multline}
  I_\alpha= \lim\limits_{t\to \infty} \fr{E_\alpha \gamma(t)}{t}= {}\\
  {}=
\lim\limits_{t\to\infty} \fr{E_\alpha \gamma^{(0)}(t)}{t} +\lim\limits_{t\to\infty} 
\fr{E_\alpha \gamma^{(1)}(t)}{t}={}\\
  {}=
  \lim\limits_{t\to\infty} \fr{E_\alpha \gamma^{(0)}(t)}{t}+E_\alpha 
\gamma_n^{(1)}\lim\limits_{t\to \infty} \fr{H(t)}{t}\,.
  \label{e8-sk}
  \end{multline}
  
  Первое слагаемое в~правой части равенства~(\ref{e8-sk}) может быть 
определено на основании эргодической тео\-ре\-мы для аддитивного функционала. 
Классические формы такой тео\-ре\-мы для полумарковских моделей приведены 
в~[7, 8]. В~данном случае для ре\-ге\-не\-ри\-ру\-юще\-го процесса~$\xi(t)$ эргодическая 
теорема приводит к~сле\-ду\-юще\-му результату:
  \begin{equation}
  \lim\limits_{t\to\infty} \fr{E_\alpha \gamma^{(0)}(t)}{t}= \fr{E_\alpha \left[ 
\Delta \gamma_n^{(0)}\right]}{E_\alpha \left[ \Delta t_n\right]}\,.
  \label{e9-sk}
  \end{equation}
  
  Второе слагаемое в~правой час\-ти равенства~(\ref{e8-sk}) определяется на 
основании элементарной тео\-ре\-мы восстановления~\cite{9-sk}:
  \begin{equation}
  \lim\limits_{t\to\infty} \fr{H(t)}{t}=\fr{1}{E_\alpha\left[ \Delta t_n\right]}\,.
  \label{e10-sk}
  \end{equation}
  
  Заметим, что возможность применения указанной тео\-ре\-мы обеспечивается 
выполнением условия $E_\alpha \left[ \Delta t_n\right] \hm>0$.
  
  Тогда из~(\ref{e8-sk}) с~учетом~(\ref{e9-sk}) и~(\ref{e10-sk}) получаем 
пред\-став\-ле\-ние для сред\-ней удельной прибыли:
  \begin{multline}
  I_\alpha= \lim\limits_{t\to \infty} \fr{E_\alpha \gamma(t)}{t}=\fr{E_\alpha \left[ 
\Delta \gamma_n^{(0)}\right] +E_\alpha \gamma_n^{(1)}}{E_\alpha \left[ \Delta 
t_n\right]}= {}\\
{}=
  \fr{E_\alpha \left[ \Delta\gamma_n^{(0)}+\gamma_n^{(1)}\right]}{E_\alpha \left[ 
\Delta t_n\right]}\,.
  \label{e11-sk}
  \end{multline}
  
  Случайная величина $\Delta\gamma_n^{(0)}\hm+ \gamma_n^{(1)}\hm= 
\Delta\gamma_n$ представляет собой полное изменение аддитивного 
стоимостного функционала на периоде регенерации $(t_n, t_{n+1}]$. 
Полученное соотношение~(\ref{e11-sk}) совпадает с~(\ref{e5-sk}). Тео\-ре\-ма 
доказана.
  
  \smallskip
  
  \noindent
  \textbf{Замечание~1.}\ Утверждение теоремы остается справедливым для 
более общего варианта, в~котором множество допустимых решений 
(управ\-ле\-ний)~$U$ имеет произвольную структуру, а~множество стратегий 
управ\-ле\-ния ре\-ге\-не\-ри\-ру\-ющим процессом~$\xi(t)$ совпадает с~множеством 
вероятностных распределений, заданных на~$U$.
  
  \smallskip
  
  \noindent
  \textbf{Замечание~2.}\ Выполнение условия $E_\alpha\left[ \Delta 
t_n\right]\hm>0$ можно обеспечить сле\-ду\-ющим образом. Из 
соотношения~(\ref{e1-sk}) следует, что 
$$
E_\alpha\left[ \Delta t_n\right]= 
E_\alpha \left[ \Delta_n^{(0)}\right]\hm+ E_\alpha\left[ \Delta_n^{(1)}\right]\,.
$$ 
Здесь
\begin{align*} 
E_\alpha \left[ \Delta_n^{(0)}\right]&= \sum\limits_{r\in U} \fr{N-r}{\lambda}\, 
\alpha_r\,;\\
E_\alpha \left[ \Delta_n^{(1)}\right]&= \sum\limits_{r\in U} \tau_r 
\alpha_r\,,
\end{align*}
где
$$
\tau_r= E_r\left[ \Delta_n^{(1)}\right]\hm= \int\limits_0^\infty 
x\,dH_r(x).
$$
   
  Очевидно, что $E_\alpha\left[ \Delta_n^{(0)}\right]\hm=0$ только для 
распределения вида ($\alpha_N\hm=1$, $\alpha_r\hm= 0$, $r\hm\in U$, $r\not= 
N$). Отсюда следует, что если $\tau_N\hm>0$, то $E_\alpha\left[ \Delta t_n\right] 
\hm>0$ для указанного распределения вероятностей ($\alpha_N\hm=1$, 
$\alpha_r\hm=0$, $r\hm\in U$, $r\not= N$), а~также и~для любой другой 
стратегии принятия решений $\alpha\hm\in \Gamma_d$. Если же потребовать 
выполнения более сильного естественного условия $\tau_r\hm>0$, $r\hm\in U$ 
(средняя дли\-тель\-ность задержки по\-став\-ки не может быть равна нулю ни при 
каком допустимом решении), то будет иметь мес\-то оценка 
$$
E_\alpha \left[ \Delta  t_n\right]\geq E_\alpha \left[ \Delta_n^{(1)}\right] >0\,.
$$
  
  Из утверждения тео\-ре\-мы следует, что для нахождения явного пред\-став\-ле\-ния 
показателя эффективности~$I_\alpha$ необходимо определить аналитические 
пред\-став\-ле\-ния математических ожиданий $E_\alpha\left[ \Delta\gamma_n\right]$ 
и~$E_\alpha\left[ \Delta t_n\right]$.
  
  Зафиксируем параметр управ\-ле\-ния $r\hm\in U$ и~обозначим:
  \begin{align}
  A(r)&= E_r \left[ \Delta \gamma_n^{(0)} +\gamma_n^{(1)}\right] = E_r \left[ 
\Delta \gamma_n\right]\,;\label{e12-sk}\\
B(r)&=E_r\left[ \Delta t_n\right]\label{e13-sk}\,.
\end{align}
  
  По свойству математического ожидания справедливы сле\-ду\-ющие формулы:
  \begin{align}
  E_\alpha \left[ \Delta \gamma_n\right] &= \sum\limits_{r\in U} E_r \left[ 
\Delta\gamma_n\right]\alpha_r=\sum\limits_{r\in U} A(r)\alpha_r\,;\label{e14-sk}
\\
  E_\alpha \left[ \Delta t_n\right] &= \sum\limits_{r\in U} E_r \left[ \Delta 
t_n\right]\alpha_r=\sum\limits_{r\in U} B(r)\alpha_r\,.\label{e15-sk}
  \end{align}
  
  Из~(\ref{e11-sk}) с~учетом~(\ref{e14-sk}) и~(\ref{e15-sk}) получаем:
  \begin{equation}
  I_\alpha= \fr{E_\alpha\left[ \Delta\gamma_n\right] } {E_\alpha\left[ \Delta 
t_n\right] }=\fr{\sum\nolimits_{r\in U} A(r)\alpha_r}{\sum\nolimits_{r\in U} B(r) 
\alpha_r}\,.
  \label{e16-sk}
  \end{equation}
  
  Задача оптимального управ\-ле\-ния в~рассматриваемой модели может быть 
сформулирована в~виде сле\-ду\-ющей экстремальной проб\-лемы:
  \begin{equation}
  I_\alpha= \fr{\sum\nolimits_{r\in U}  A(r)\alpha_r}{\sum\nolimits_{r\in U} 
B(r)\alpha_r}\to \max\,,\ \alpha\in \Gamma_d\,.
  \label{e17-sk}
  \end{equation}
  
  Функционал~(\ref{e16-sk}) по форме пред\-став\-ля\-ет собой так называемый 
дроб\-но-ли\-ней\-ный интегральный функционал, заданный на множестве 
дискретных вероятностных распределений~$\Gamma_d$. Изложим общий подход 
к~решению по\-став\-лен\-ной задачи.

\section{О решении задачи оптимального управления 
в~рассматриваемой стохастической модели}

  Теория решения задачи безусловного экстремума для общего 
  дроб\-но-ли\-ней\-но\-го интегрального функционала, заданного на множестве наборов 
вероятностных мер, изложена в~работе~\cite{11-sk}. Для решения поставленной 
экстремальной задачи~(\ref{e17-sk}) необходимо исследовать на экстремум 
специальный вариант дроб\-но-ли\-ней\-но\-го интегрального функционала, 
в~котором вероятностные меры являются дискретными. Полное описание 
соответствующей экстремальной задачи и~формулировка утверждения о~ее 
решении не могут быть приведены в~рамках на\-сто\-ящей статьи. В~связи с~этим 
упомянутые результаты включены в~приложение~\cite{12-sk}. Отметим также, 
что утверждение о~безуслов\-ном экстремуме дроб\-но-ли\-ней\-но\-го интегрального 
функционала, заданного на множестве наборов дискретных вероятностных 
распределений, приведено в~работе~\cite{13-sk}.
  
  Основываясь на указанном теоретическом утверждении, сформулируем 
главный результат, связанный с~решением экстремальной задачи~(\ref{e17-sk}).\linebreak 
Если основная функция дроб\-но-ли\-ней\-но\-го дискретного интегрального 
функционала~(\ref{e16-sk}), определяемая формулой $C(r)\hm= A(r)/B(r)$, где\linebreak 
функции $A(r)$ и~$B(r)$ определены соотношениями~(\ref{e12-sk}) 
и~(\ref{e13-sk}), достигает глобального максимума\linebreak в~некоторой фиксированной 
точке $r^*\hm\in U\hm= \left\{ N, N-1, \ldots , 0, -1, \ldots , -N_0\right\}$, то 
решение исходной задачи~(\ref{e17-sk}) существует и~достигается на 
вырожденном распределении~$\alpha^*$, сосредоточенном в~точке~$r^*$. 
Заметим, что в~рас\-смат\-ри\-ва\-емой задаче множество допустимых решений~$U$ 
конечно и~глобальный максимум основной функции~$C(r)$ достигается. Таким 
образом, для полного решения поставленной задачи необходимо найти явные 
аналитические выражения для функций~$A(r)$ и~$B(r)$. После этого точ\-ка 
глобального максимума функ\-ции~$C(r)$ может быть определена чис\-лен\-ным 
методом.

\section{Заключение}

  В настоящей статье изложена первая часть исследования задачи управ\-ле\-ния 
запасом дискретного продукта в~схеме регенерации. В~ней разработана новая 
стохастическая модель управ\-ле\-ния запасом дискретного продукта, которая 
представляет собой регенерирующий процесс с~конечным множеством 
состояний. Сформулирована задача оптимального управ\-ле\-ния, которая по 
своей математической форме представляет собой экстремальную задачу для 
функционала, заданного на множестве дискретных вероятностных 
распределений, каждое из которых определяет стратегию управ\-ле\-ния. 
Поскольку целевой функционал по своей структуре является  
дроб\-но-ли\-ней\-ным интегральным функционалом, для нахождения решения 
используется тео\-ре\-ма о~безусловном экстремуме функционалов данного вида. 
На основании этой тео\-ре\-мы установлено, что оптимальная стратегия 
управления в~рас\-смат\-ри\-ва\-емой задаче существует, является 
детерминированной и~определяется точ\-кой максимума некоторой функции, 
которая по своему идейному содержанию пред\-став\-ля\-ет собой среднюю 
удельную прибыль, зависящую от параметра управ\-ле\-ния. 

Вторая часть 
исследования задачи управ\-ле\-ния запасом посвящена на\-хож\-де\-нию явного 
аналитического пред\-став\-ле\-ния для указанной функции. Эта часть будет 
представлена авторами в~сле\-ду\-ющей статье.

  {\small\frenchspacing
 {%\baselineskip=10.8pt
 \addcontentsline{toc}{section}{References}
 \begin{thebibliography}{99}
\bibitem{1-sk}
\Au{Шнурков~П.\,В., Мельников~Р.\,В.} Оптимальное управление запасом непрерывного 
продукта в~модели регенерации~// Обозрение прикладной и~промышленной математики, 
2006. Т.~13. Вып.~3. С.~434--452.
\bibitem{2-sk}
\Au{Шнурков П.\,В., Мельников~Р.\,В.} Исследование проблемы управления запасом 
непрерывного продукта при детерминированной задержке поставки~// Автоматика 
и~телемеханика, 2008. №\,10. С.~93--113.
\bibitem{3-sk}
\Au{Шнурков П.\,В., Пименова~Е.\,Ю.} Оптимальное управ\-ле\-ние запасом непрерывного 
продукта в~схеме регенерации с~детерминированной задержкой по\-став\-ки и~периодом 
реального пополнения~// Системы и~средства информатики, 2017. Т.~27. №\,4. С.~80--94.
\bibitem{4-sk}
\Au{Porteus E.\,L.} Foundations of stochastic inventory theory.~--- 
Stanford, CA, USA: Stanford Business Book, 2002. 299~p.
\bibitem{5-sk}
\Au{Simchi-Levi D., Chen~X., Bramel~J.} The logic of logistics: Theory, algorithms, and 
applications for logistics management.~--- New York, NY, USA: Springer-Verlag, 2014. 447~p.
\bibitem{6-sk}
\Au{Рыков В.\,В., Козырев~Д.\,В.} Основы теории массового обслуживания.~--- М.: Инфра-М, 
2016. 224~с. 
\bibitem{7-sk}
\Au{Джевелл В.} Управляемые полумарковские процессы~// Кибернетический сборник. 
Новая серия.~--- М.: Мир, 1967. Вып.~4. С.~97--134.
\bibitem{8-sk}
\Au{Майн Х., Осаки~С.} Марковские процессы принятия решений~/ Пер. с~англ.~--- 
  М.: Наука,  1977. 176~с. (\Au{Mine~H., Osaki~S.} Markovian decision processes.~---  
  New York, NY, USA: Elsevier, 1970. 142~p.)
\bibitem{9-sk}
\Au{Климов Г.\,П.} Теория массового обслуживания.~--- М.: Изд-во Московского ун-та, 2011. 
312~с.
\bibitem{10-sk}
\Au{Боровков А.\,А.} Теория вероятностей.~--- М.: Либроком, 2009. 656~с.
\bibitem{11-sk}
\Au{Шнурков П.\,В.} О~решении задачи безусловного экстремума для дробно-линейного 
интегрального функционала на множестве вероятностных мер~// Докл. Акад. наук. Сер. 
Математика, 2016. Т.~470. №\,4. С.~387--392.
\bibitem{12-sk}
\Au{Шнурков П.\,В., Вахтанов~Н.\,А.} Приложение к~статье <<Исследование проблемы 
оптимального\linebreak управления запасом дискретного продукта в~стохастической модели 
регенерации с~непрерывно происходящим потреблением и~случайной задержкой поставки>>, 
2019. 16~с. {\sf http://www.ipiran.ru/\linebreak publications/Приложение.pdf}.
\bibitem{13-sk}
\Au{Shnourkoff P.\,V., Novikov~D.\,A.}
Analysis of the problem of intervention control in the 
economy on the basis of solving the problem of tuning~// arXiv.org, 2018. 
arXiv:1811.10993 [q-fin.GN]. 15~p.
 \end{thebibliography}

 }
 }

\end{multicols}

\vspace*{-3pt}

\hfill{\small\textit{Поступила в~редакцию 19.02.19}}

%\vspace*{8pt}

%\pagebreak

\newpage

\vspace*{-29pt}

%\hrule

%\vspace*{2pt}

%\hrule

%\vspace*{-2pt}

\def\tit{RESEARCH OF~THE~OPTIMAL CONTROL PROBLEM OF~INVENTORY 
OF~A~DISCRETE PRODUCT IN~THE~STOCHASTIC REGENERATION MODEL 
WITH~CONTINUOUSLY OCCURING CONSUMPTION AND~RANDOM DELIVERY DELAY}


\def\titkol{Research of~the~optimal control problem of~inventory 
of~a~discrete product in~the~stochastic regeneration model} 
%with~continuously occuring consumption and~random delivery delay}

\def\aut{P.\,V.~Shnurkov and N.\,A.~Vakhtanov}

\def\autkol{P.\,V.~Shnurkov and N.\,A.~Vakhtanov}

\titel{\tit}{\aut}{\autkol}{\titkol}

\vspace*{-13pt}


\noindent
National Research University Higher School of Economics, 34~Tallinskaya Str., 
Moscow 123458, Russian Federation

\def\leftfootline{\small{\textbf{\thepage}
\hfill INFORMATIKA I EE PRIMENENIYA~--- INFORMATICS AND
APPLICATIONS\ \ \ 2019\ \ \ volume~13\ \ \ issue\ 2}
}%
 \def\rightfootline{\small{INFORMATIKA I EE PRIMENENIYA~---
INFORMATICS AND APPLICATIONS\ \ \ 2019\ \ \ volume~13\ \ \ issue\ 2
\hfill \textbf{\thepage}}}

\vspace*{3pt}



\Abste{The paper considers the optimal control problem of inventory of 
a~discrete product in a regeneration scheme with a~Poisson flow of customer
 requirements. In the system, deferred demand is allowed, the volume
  of which is limited by a~given value. The control parameter is the level 
  of the stock, at which achievement it is necessary to make an order for 
  replenishment. The indicator of management effectiveness is the average 
  specific profit received in one regeneration period. The optimal control 
  problem is solved on the basis of the statement about the extremum of 
  a~fractional-linear integral functional 
on the set of discrete probability distributions.}

\KWE{inventory management of a discrete product; controlled regenerative process; 
extremal problem for a~fractional-linear integral functional } 

 \DOI{10.14357/19922264190208}

%\vspace*{-14pt}

 %\Ack
 % \noindent
 

%\vspace*{6pt}

  \begin{multicols}{2}

\renewcommand{\bibname}{\protect\rmfamily References}
%\renewcommand{\bibname}{\large\protect\rm References}

{\small\frenchspacing
 {%\baselineskip=10.8pt
 \addcontentsline{toc}{section}{References}
 \begin{thebibliography}{99}
\bibitem{1-sk-1}
\Aue{Shnurkov, P.\,V., and R.\,V.~Mel'nikov.} 2006. Optimal'noe upravlenie zapasom 
nepreryvnogo produkta v~modeli regeneratsii [Optimal control of a~continuous 
product inventory in the regeneration model]. \textit{Obozrenie prikladnoy 
i~promyshlennoy matematiki} [Review of Applied and Industrial Mathematics] 13(3):434--452.
\bibitem{2-sk-1}
\Aue{Shnurkov, P.\,V., and R.\,V.~Mel'nikov.} 2008. Analysis of the problem of 
continuous-product inventory control under deterministic lead time. \textit{Automat. 
Rem. Contr.} 69(10):1734--1751.
\bibitem{3-sk-1}
\Aue{Shnurkov, P.\,V., and E.\,Yu.~Pimenova.} 2017. Optimal'noe upravlenie zapasom 
nepreryvnogo produkta v~skheme regeneratsii s~determinirovannoy zaderzhkoy 
postavki i~periodom real'nogo popolneniya [Optimal inventory control of 
continuous product in regeneration theory with determinate delay of the 
delivery and the period of real replenishment]. \textit{Sistemy i~Sredstva Informatiki~--- 
Systems and Means of Informatics} 27(4):80--94.
\bibitem{4-sk-1}
\Aue{Porteus, E.\,L.} 2002. \textit{Foundations of stochastic inventory theory}. 
Stanford, CA: Stanford Business Book. 299~p.
\bibitem{5-sk-1}
\Aue{Simchi-Levi, D., X.~Chen, and J.~Bramel.} 2014. \textit{The logic of logistics: 
Theory, algorithms, and applications for logistics management}. New York, 
NY: Springer-Verlag. 447~p.
\bibitem{6-sk-1}
\Aue{Rykov, V.\,V., and D.\,V.~Kozyrev.} 2016. \textit{Osnovy teorii massovogo obsluzhivania} 
[Fundamentals of queuing theory]. Moscow: Infra-M. 224~p.
\bibitem{7-sk-1}
\Aue{Jewell, W.\,S.} 1963. Markov-renewal programming. \textit{Oper. Res.} 11:938--971.
\bibitem{8-sk-1}
\Aue{Mine, H., and S.~Osaki.} 1970. \textit{Markovian decision processes}. New York, NY: 
Elsevier. 142~p.
\bibitem{9-sk-1}
\Aue{Klimov, G.\,P.} 2011. \textit{Teoriya massovogo obsluzhivaniya} [Queuing theory]. 
Moscow: MSU Publs. 312~p.
\bibitem{10-sk-1}
\Aue{Borovkov, A.\,A.} 2009. \textit{Teoriya veroyatnostey}
[Probability theory]. Moscow:  Librokom. 656~p.
\bibitem{11-sk-1}
\Aue{Shnurkov, P.\,V.} 2016. Solution of the unconditional extremum problem for 
a~linear-fractional integral functional on a set of probability measures. \textit{Dokl. 
Math.} 94(2):550--554.
\bibitem{12-sk-1}
\Aue{Shnurkov, P.\,V., and N.\,A.~Vakhtanov.} 2019. Prilozhenie k~stat'e 
``Issledovanie problemy optimal'nogo upravleniya zapasom diskretnogo produkta 
v~sto\-kha\-sti\-che\-skoy modeli regeneratsii s~nepreryvno pro\-is\-kho\-dya\-shchim 
potrebleniem i~sluchaynoy zaderzhkoy postavki'' [Appendix to article ``Research 
of the optimal control problem of inventory of a discrete product in 
stochastic regeneration model with continuously occurring consumption 
and random delivery delay'']. 16~p. Available at: 
{\sf http://www.ipiran.ru/publications/Приложение.pdf} (accessed May~6, 
2019).  
\bibitem{13-sk-1}
\Aue{Shnourkoff, P.\,V., and D.\,A.~Novikov.} 2018. Analysis of the problem of 
intervention control in the economy on the basis of solving the problem of 
tuning. arxiv.org.\linebreak 
arXiv:1811.10993 [q-fin.GN]. 15~p.

\end{thebibliography}

 }
 }

\end{multicols}

\vspace*{-9pt}

\hfill{\small\textit{Received February 19, 2019}}

%\pagebreak

\vspace*{-24pt}


\Contr

\vspace*{-3pt}

\noindent
\textbf{Shnurkov Peter V.} (b.\ 1953)~--- Candidate of Science (PhD) in physics 
and mathematics, associate professor, National Research University Higher School of 
Economics, 34~Tallinskaya Str., Moscow 123458, Russian Federation; 
\mbox{pshnurkov@hse.ru}

%\vspace*{3pt}

\noindent
\textbf{Vakhtanov Nikita A.} (b.\ 1997)~--- Master student, National Research 
University Higher School of Economics, 34~Tallinskaya Str., Moscow 123458, 
Russian Federation; \mbox{Vakhtanov1997@mail.ru}
\label{end\stat}

\renewcommand{\bibname}{\protect\rm Литература}