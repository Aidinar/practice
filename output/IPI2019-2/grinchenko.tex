\def\stat{grinchenko}

\def\tit{О ГЕНЕЗИСЕ ИНФОРМАЦИОННОГО ОБЩЕСТВА:  
ИНФОРМАТИКО-КИБЕРНЕТИЧЕСКОЕ МОДЕЛЬНОЕ ПРЕДСТАВЛЕНИЕ}

\def\titkol{О генезисе информационного общества:  
информатико-кибернетическое модельное представление}

\def\aut{С.\,Н.~Гринченко$^1$}

\def\autkol{С.\,Н.~Гринченко}

\titel{\tit}{\aut}{\autkol}{\titkol}

\index{Гринченко С.\,Н.}
\index{Grinchenko S.\,N.}


%{\renewcommand{\thefootnote}{\fnsymbol{footnote}} \footnotetext[1]
%{Работа выполнена при частичной финансовой 
%поддержке РФФИ (проект 17-07-00577).}}


\renewcommand{\thefootnote}{\arabic{footnote}}
\footnotetext[1]{Институт проблем информатики Федерального исследовательского центра <<Информатика и~управление>> 
Российской академии наук, \mbox{sgrinchenko@ipiran.ru}}

\vspace*{-3.5pt}




  \Abst{Вводится понятие <<генезис информационного общества>>, которое рассматривается 
  с~позиций ин\-фор\-ма\-ти\-ко-ки\-бер\-не\-ти\-че\-ско\-го моделирования (ИКМ)
  процесса развития 
Человечества как са\-мо\-управ\-ля\-ющей\-ся иерар\-хо-се\-те\-вой системы. На этой основе 
получены количественные оценки его типовых про\-стран\-ст\-вен\-но-вре\-мен\-ных характеристик, 
представляющих собой геометрические прогрессии со знаменателем 
<<$e$~в~степени~$e$>> (15,15426$\ldots$), а~также скоординированных с~ними во времени 
и~в~пространстве пси\-хи\-ко-ант\-ро\-по\-ло\-ги\-че\-ских, образовательных  
и~ин\-фор\-ма\-ци\-он\-но-ком\-му\-ни\-ка\-ци\-он\-ных параметров и~возможностей 
включенного в~этот процесс усложняющегося человека и~его сообществ различной 
величины. Это позволило раздвинуть рамки существования информационного общества на 
всю историческую и~даже археологическую эпоху такого развития. Результирующая 
последовательность информационных технологий (ИТ) <<сигнальные  
по\-зы/зву\-ки/дви\-же\-ния\,--\,ми\-ми\-ка/жес\-ты\,--\,речь/язык\,--\,пись\-мен\-ность\,--\,ти\-ра\-жи\-ро\-ва\-ние текстов\,--\,компью\-те\-ры\,--\,те\-ле\-ком\-му\-ни\-ка\-ции\,--\,ин\-фор\-ма\-ци\-он\-ная на\-но\-тех\-но\-ло\-гия\,--\,$\ldots$>> 
позволяет рас\-смат\-ри\-вать генезис 
информационного общества в~широком контексте единой исторической ретроспективы 
и~перспективы.}
  
  \KW{информационное общество; информационные технологии;  
ин\-фор\-ма\-ти\-ко-ки\-бер\-не\-ти\-че\-ская модель; самоуправляющаяся 
 иерар\-хо-се\-те\-вая система Человечества; археологическая эпоха}
 
 \DOI{10.14357/19922264190214}
  
%\vspace*{4pt}


\vskip 10pt plus 9pt minus 6pt

\thispagestyle{headings}

\begin{multicols}{2}

\label{st\stat}
  
  В~литературе, даже энциклопедической, распространена трактовка 
<<информационного общества>> как общества <<современного типа>>, 
в~котором общение людей опирается на компьютерные 
и~телекоммуникационные ИТ\footnote[2]{В~[1] дано следующее определение:
<<\textbf{Информационное общество}, одно из понятий, используемых 
в~социологич.\ теории для обозначения обществ.\ систем <<современного типа>>$\ldots$ 
Важнейшие характеристики~И.\,о.: 1)~лавинообразное распространение информац. 
технологий (прежде всего компьютерных и~телекоммуникационных); 2)~превращение 
информации в~важнейший социальный ресурс, необходимую предпосылку управленч. 
деятельности, развития экономики, образования, сферы услуг, домашнего быта, 
рекреационной сферы и~т.\,д.; по некоторым данным, в~наиболее развитых странах проф. 
деятельность более половины занятых связана исключительно с~производством и~обработкой 
информации; 3)~наделение СМИ статусом <<четвертой ветви власти>>; 4)~расширение 
границ и~<<репертуара>> массовой культуры; 5)~увеличение каналов вертикальной 
и~горизонтальной мобильности; 6)~изменение представлений о~социальном пространстве 
(<<глобализация>> пространства, мгновенная доступность даже периферийных его 
сегментов) и~времени (расширение рамок <<современности>>, когда даже отдаленные 
историч. события воспринимаются как происходящие <<здесь>> и~<<сейчас>>); 
7)~возникновение в~процессе коммуникации особой виртуальной реальности, несводимой 
к~результатам технич. визуализации и~выходящей за пределы воображения и~памяти 
индивида; 8)~превращение информац. технологий в~базу для развития высоких технологий 
(Hi-Tech)>>.}. Такая трактовка этого понятия создает иллюзию 
отстраненности информационного общества от его собственного исторического 
прошлого, когда вышеперечисленных ИТ еще не изобрели, но люди в~составе 
сообществ как-то общались между собой, используя иные ИТ. 

Поскольку от 
этой иллюзии недалеко до недооценки полезности соответствующего 
исторического опыта для современности, попытаюсь развеять ее.
  
Результаты ИКМ процесса развития на Земле 
Человечества как самоуправляющейся ие\-рар\-хо-се\-те\-вой\footnote[3]{<<\textbf{Иерархо-сетевая}>> 
структура~--- иерархическая структура типа <<матрешки>>, но с~существенно большим 
единицы числом вложений на каждом ее иерархическом уровне, которые и~образуют 
соответствующие сетевые структуры.} системы~[2--14] (рис.~1) позволяют раздвинуть рамки 
существования информационного\linebreak общества на всю историческую и~даже археологическую эпоху такого 
развития, что дает возможность выделить ту эволюционную линию этого процесса, которую логично 
определить как \textit{генезис информационного общества}. 


\begin{figure*} %fig1
   \vspace*{1pt}
    \begin{center}  
  \mbox{%
 \epsfxsize=130.287mm 
 \epsfbox{gri-1.eps}
 }
\end{center}
%\vspace*{-9pt}
%\Caption{Схема иерархо-сетевой самоуправляющейся (по алгоритмам случайной поисковой 
%оптимизации целевых критериев энергетического характера с~ограничениями типа 
%равенств и~неравенств) системы лич\-ност\-но-про\-из\-вод\-ст\-вен\-но-со\-ци\-аль\-ной природы 
%(Человечества)~\cite{5-grn}}
\end{figure*}


На рис.~1 используются следующие обозначения:
\begin{itemize}
\item восходящие стрелки (имеющие структуру <<мно\-гие\,--\,к~од\-но\-му>>) 
отражают первую из~5~основных со\-став\-ля\-ющих контура поисковой 
оптимизации~--- \textit{поисковую активность} представителей 
соответствующих ярусов в~иерархии; 
\item нисходящие сплошные (имеющие 
структуру <<один\,--\,ко мно\-гим>>) стрелки отражают вторую 
со\-став\-ля\-ющую~--- \textit{целевые критерии} поисковой оптимизации 
энергетики системы Человечества; 
\item нисходящие пунктирные (<<один\,--\,ко 
многим>>) стрелки отражают третью со\-став\-ля\-ющую~--- 
\textit{оптимизационную системную память}  
лич\-ност\-но-про\-из\-вод\-ст\-вен\-но-со\-ци\-аль\-но\-го (результат 
адаптивных влияний представителей вышележащих иерархических ярусов на 
структуру вложенных в~них нижележащих); 
\item полужирными стрелками 
в~левой части схемы условно показана четвертая со\-став\-ля\-ющая~--- 
\textit{антропогенная ак\-тив\-ность} индивидов и~их групп, трак\-ту\-емая как 
<<трудовая деятельность по созданию со\-от\-вет\-ст\-ву\-юще\-го инструментария 
и~результатов его применения>>; 
\item пунктирными полужирными стрелками 
в~правой части схемы условно показана пятая со\-став\-ля\-ющая~--- 
\textit{антропогенная системная\linebreak память}  
лич\-ност\-но-про\-из\-вод\-ст\-вен\-но-со\-ци\-ального (процессы вовлечения 
результатов антропогенной активности в~структуру со\-от\-вет\-ст\-ву\-ющей  
иерар\-хо-се\-те\-вой под\-сис\-те\-мы Человечества).
\end{itemize}

Рассмотрим этот феномен поэтапно, сведя в~общую таблицу расчетные данные 
о~различных его проявлениях. 
       


\begin{table*}\footnotesize
\begin{center}
\Caption{Свод основных характеристик генезиса информационного общества (как 
проявления развития са\-мо\-управ\-ля\-ющей\-ся и~метаэволюционирующей, т.\,е.\ 
наращивающей чис\-ло своих иерархических уров\-ней/яру\-сов, сис\-те\-мы Человечества) от 
прошлого до модельно прогнозируемого будущего}
\vspace*{2ex}

\tabcolsep=1.5pt
\begin{tabular}{|c|c|l|c|c|c|c|}
\hline
&\tabcolsep=0pt\begin{tabular}{c}Характерный\\ ареал (радиус\\
 круга той же\\ площади); точность\\ антропогенного\\ 
воздействия\,/\\
производственных\\ технологий\end{tabular}&
\tabcolsep=0pt\begin{tabular}{c}Характерные\\ времена\\ старта;\\ кульминации\\ 
развития\\ подсистемы\end{tabular}&
\tabcolsep=0pt\begin{tabular}{c}Уровень\\ развития\\ Homo\\  
(и его пред-\\ шествен-\\ ников)\end{tabular}&
\tabcolsep=0pt\begin{tabular}{c}Носитель системной\\ памяти~---\\ субстрат психики\end{tabular}&
\tabcolsep=0pt\begin{tabular}{c}Лидирующая\\ ИТ\end{tabular}&
\tabcolsep=0pt\begin{tabular}{c}Требуемый уровень\\ образованности Homo;\\
аналогия филогенеза\\ и~онтогенеза:\\ примерный возраст\\ гармонично\\ образовываемого\\ 
(сегодня)\end{tabular}\\
\hline
1&2&\multicolumn{1}{c|}{3}&4&5&6&7\\
\hline
0&$\sim4{,}2$~м&\tabcolsep=0pt\begin{tabular}{c} $\sim428$~млн\\ лет назад;\\
$\sim 140{,}1$~млн\\ лет назад\end{tabular}&
\tabcolsep=0pt\begin{tabular}{c}Цефализация\\ позвоночных\end{tabular}&
\tabcolsep=0pt\begin{tabular}{c}Многоклеточный\\организм в~целом\end{tabular}&
\tabcolsep=0pt\begin{tabular}{c}Формирование\\ головного\\ мозга как основы\\
 для реализации\\ 
будущих ИТ\end{tabular}&\tabcolsep=0pt\begin{tabular}{c} ---\\
$\sim0{,}6$--1,0~год\end{tabular}\\
\hline
1&\tabcolsep=0pt\begin{tabular}{c} $\sim64$~м;\\
$\sim28$~см
\end{tabular}&\tabcolsep=0pt\begin{tabular}{c}$\sim28{,}23$~млн\\ лет назад;\\
$\sim9{,}26$~млн\\ лет назад
\end{tabular}&\tabcolsep=0pt\begin{tabular}{c}Пред-пред-\\
люди\\ Hominoidea\end{tabular}&
\tabcolsep=0pt\begin{tabular}{c}Органы многоклеточного\\ организма (его 
нервной\\ системы в~целом)\end{tabular}&
\tabcolsep=0pt\begin{tabular}{c}Сигнальные позы/\\
движения\\ и~неинтонированные\\ звуки (типа 
рычания,\\ ворчания, писка\\ и~т.\,п.)\end{tabular}&
\tabcolsep=0pt\begin{tabular}{c}Выработка\\ 
(младенцами)\\ сигнальных поз;\\
$\sim1{,}0$--1,6~лет \end{tabular}\\
\hline
2&\tabcolsep=0pt\begin{tabular}{c} $\sim1$~км;\\
$\sim1{,}8$~см\end{tabular}&\tabcolsep=0pt\begin{tabular}{c} $\sim1{,}86$~млн\\ лет 
назад;\\
$\sim612$~тыс.\\ лет назад\end{tabular}&
\tabcolsep=0pt\begin{tabular}{c}Пред-люди\\ Homo ergaster\,/\\
Homo erectus\end{tabular}&
\tabcolsep=0pt\begin{tabular}{c}Ткани 
многоклеточного\\ организма\\ (сетей/ансамблей\\ нейронов и~др.)\end{tabular}&
\tabcolsep=0pt\begin{tabular}{c}Мимика/жесты\\ 
и~интонированные\\ звуки\end{tabular}&
\tabcolsep=0pt\begin{tabular}{c}Овладение (ре-\\ бенком) мимикой/\\ 
жестами,\\
начальное\\ понимание речи; \\ $\sim1{,}6$--2,6~лет \end{tabular}\\
\hline
3&\tabcolsep=0pt\begin{tabular}{c} $\sim15$~км; \\
$\sim1{,}2$~мм
\end{tabular}&
\tabcolsep=0pt\begin{tabular}{c} $\sim123$~тыс.\\ лет назад;\\
$\sim40$~тыс.\\ лет назад\end{tabular}&
\tabcolsep=0pt\begin{tabular}{c}Homo\\ sapiens$^\prime$\end{tabular}&
\tabcolsep=0pt\begin{tabular}{c}Эвкариотические\\ клетки\\ 
многоклеточного\\ организма\\ (отдельные нервные\\ и~глиальные клетки\\ и~др.)\end{tabular}&
\tabcolsep=0pt\begin{tabular}{c}Речь/язык\\ 
(артикулированная\\ устная речь)\end{tabular}&
\tabcolsep=0pt\begin{tabular}{c}Овладение (детьми)\\ 
речью/языком\\ (протообразование); \\ $\sim2{,}6$--4,2~лет \end{tabular}\\
\hline
4&\tabcolsep=0pt\begin{tabular}{c} $\sim222$~км;\\
$\sim 80$~мкм
\end{tabular}&\tabcolsep=0pt\begin{tabular}{c}$\sim8{,}1$~тыс.\\ лет назад;\\
$\sim2{,}7$~тыс.\\ лет назад\end{tabular}&
\tabcolsep=0pt\begin{tabular}{c}Homo\\ sapiens$^{\prime\prime}$\end{tabular}&
\tabcolsep=0pt\begin{tabular}{c}Компартменты\\ 
эвкариотической\\ клетки (отдельные\\ рецепторные,\\ или постсинаптические,\\ зоны нейронов и~т.\,п.)\end{tabular}
&Письменность&\tabcolsep=0pt\begin{tabular}{c}Овладение чтением/ \\ письмом 
(дошкольное\\ образование);\\
$\sim4{,}2$--6,9~лет \end{tabular}\\
\hline
5&\tabcolsep=0pt\begin{tabular}{c}$\sim3370$~км;\\
$\sim5$~мкм
\end{tabular}&\tabcolsep=0pt\begin{tabular}{l}$\sim1446$~г.;\\
$\sim1806$~г.\end{tabular}&
\tabcolsep=0pt\begin{tabular}{c}Homo\\ sapiens$^{\prime\prime\prime}$\end{tabular} &
\tabcolsep=0pt\begin{tabular}{c}Субкомпартменты\\ эвкариотической 
клетки\end{tabular}&
\tabcolsep=0pt\begin{tabular}{c}Тиражирование\\ текстов,\\ или книгопечатание\end{tabular}&
\tabcolsep=0pt\begin{tabular}{c}Начальное\\ образование;\\ 
$\sim6{,}9$--11,1~лет \end{tabular}\\
\hline
6&\tabcolsep=0pt\begin{tabular}{c} $\sim51$~тыс.\ км\\ (общепланетарный);\\
$\sim0{,}35$~мкм\end{tabular}&\tabcolsep=0pt\begin{tabular}{l} $\sim1946$~г.;\\
$\sim 1970$~г.\end{tabular}&\tabcolsep=0pt\begin{tabular}{c}Homo \\
sapiens$^{\prime\prime\prime\prime}$\end{tabular}&
\tabcolsep=0pt\begin{tabular}{c}Ультраструктурные\\ (прокариотические)\\ 
внутриклеточные элементы\\ эвкариотической клетки\\ (типа клеточного ядра,\\ деталей 
эндоплазматической\\ сети и~т.\,п.\ образований)\end{tabular}&
Компьютерная ИТ&\tabcolsep=0pt\begin{tabular}{c}Среднее\\ образование;\\
$\sim11{,}1$--18~лет \end{tabular}\\
\hline
7&\tabcolsep=0pt\begin{tabular}{c} $\sim773$~тыс.\ км\\ (ближний\\ космос);\\
$\sim23$~нм\end{tabular}&\tabcolsep=0pt\begin{tabular}{l} $\sim1979$~г.;\\
$\sim2003$~г.\end{tabular}&\tabcolsep=0pt\begin{tabular}{c}Homo\\
 sapiens$^{\prime\prime\prime\prime\prime}$\end{tabular} 
&
\tabcolsep=0pt\begin{tabular}{c}Макромолекулы/гены\\ (компартменты\\ 
ультраструктурных--\\
прокариотических--\\
внутриклеточных\\ элементов)\end{tabular}&
\tabcolsep=0pt\begin{tabular}{c}Телекоммуника-\\ ционная ИТ\end{tabular}&\tabcolsep=0pt\begin{tabular}{c}Высшее обра-\\
зование\;+\;<<аспи-\\ рантура>>; \\
$\sim18$--29,1~лет \end{tabular}\\
\hline
8&\tabcolsep=0pt\begin{tabular}{c}
$\sim11{,}7$~млн км\\ (промежуточный\\ космос);\\
$\sim1{,}5$~нм\end{tabular}&\tabcolsep=0pt\begin{tabular}{l} $\sim1981$~г.;\\ 
$\sim2341$~г.~(?)\end{tabular}&\tabcolsep=0pt\begin{tabular}{c}Homo\\ 
sapiens$^{\prime\prime\prime\prime\prime\prime}$\end{tabular}&
\tabcolsep=0pt\begin{tabular}{c}Органические молекулы \\
(субкомпартменты\\ ультраструктурных--
\\прокариотических--
\\внутриклеточных \\
элементов)\end{tabular}&
\tabcolsep=0pt\begin{tabular}{c}Нано-ИТ (возможно,\\
 <<наноаппаратно\\ поддерживаемая\\ селективная\\ телепатия>>~\cite{16-grn})\end{tabular}&
 \tabcolsep=0pt\begin{tabular}{c}<<Докторантура>>; \\ 
$\sim29{,}1$--47,1~лет \end{tabular}\\
\hline
9&$\cdots$&\multicolumn{1}{c|}{$\cdots$}&$\cdots$&$\cdots$&$\cdots$&$\cdots$\\
\hline
\end{tabular}
\end{center}
\end{table*}




  Промежутки времени между возникновением новых ие\-рар\-хо-се\-те\-вых 
подсистем Человечества (а~следовательно, и~между стартами новых ИТ) 
подчиняются, согласно ИКМ, простой математической за\-ко\-но\-мер\-ности: 
каж\-дый из них в~$e^e\hm= 15{,}15426$\ldots раз короче 
предыдущего\footnote{Эту геометрическую прогрессию~--- как модель критических 
уровней развития биологических сис\-тем~--- выявили А.\,В.~Жирмунский 
и~В.\,И.~Кузьмин~\cite{17-grn}.} (третий\linebreak
 столбец таблицы). В~свою очередь, этой 
же закономерности подчиняются и~размеры ареалов\linebreak
 (радиусы кругов той же 
площади) устойчивых и~эффективно са\-мо\-управ\-ля\-ющих\-ся сообществ 
человека как базисного элемента сис\-те\-мы Человечества, и~точ\-ности 
доступных услож\-ня\-юще\-му\-ся человеку~--- в~конкретный момент 
исторического времени~--- антропогенных воздействий и/или 
производственных технологий (второй столбец таб\-ли\-цы) (рис.~2).
  
  Эмпирические оценки этих времен и~пространств, сделанные 
и~опуб\-ли\-ко\-ван\-ные палео\-ант\-ро\-по\-ло\-га\-ми, археологами и~историками,~--- 
когда они имеются!~--- не противоречат модельным  
результатам~\cite{14-grn}.
  %
Диапазоны примерного возраста <<образовываемых>>, приведенные 
в~седьмом столб\-це таб\-ли\-цы, рассчитаны, исходя из <<золотого сечения>> 
(соотношения смеж\-ных членов чис\-ло\-во\-го ряда, равного 1,618$\ldots$ при 
увеличении ряда, либо 0,618$\ldots$ при его уменьшении, аде\-кват\-ность 
использования которого при выработке количественных оценок в~самых 
различных областях знания хорошо известна\footnote{Применительно 
к~периодизации истории Человечества в~археологическую эпоху это продемонстрировано 
Ю.\,Л.~Щаповой~\cite{18-grn, 19-grn, 20-grn}, согласование подхода к~такой периодизации на 
основе золотого сечения и~пред\-ла\-га\-емо\-го информатико-ки\-бер\-не\-ти\-че\-ско\-го подхода 
подробно показано в~\cite{10-grn, 12-grn, 13-grn, 14-grn, 15-grn, 21-grn}.}), 
опирающегося на ориентировочную оценку завершения человеком среднего 
образования к~18~годам (на сегодня).


  Базируясь на ИКМ, в~качестве нулевого этапа развития будущего 
информационного общества, как пред\-став\-ля\-ет\-ся, можно рас\-смат\-ри\-вать 
процесс \textit{цефализации} позвоночных, т.\,е.\ возникновения 
и~усложнения у~них головного мозга как основного носителя механизмов 
запоминания и~считывания информации о~результатах их адаптивного 
и~социального поведения, начавшейся около 428~млн лет назад 
с~кульминацией около 140,1~млн лет назад (шестой стол\-бец таб\-ли\-цы) на 
<<территории>> порядка 4,2~м~--- т.\,е.\ в~пределах отдельного 
многоклеточного организма.
  

  
  Далее в~качестве первого этапа такого развития будем рассматривать 
начавшуюся около 28,23~млн лет назад, с~кульминацией около 9,26~млн лет 
назад, на территориях порядка 64~м, ИТ сигнальных поз/дви\-же\-ний 
и~неинтонированных звуков (типа рычания, ворчания, писка и~т.\,п.), 
характерную для стад\-ных/стай\-ных животных, в~том числе  
пред-пред-людей {Hominoidea} (четвертый стол\-бец таб\-ли\-цы), 
способных обеспечивать точность своих воздействий на природу порядка~28~см. 
Субстрат их психики относится к~иерархическому уровню органов 
многоклеточного организма (пятый стол\-бец), а~уровень об\-ра\-зо\-ван\-ности 
соответствует современному младенцу возрастом около~1--1,6~лет (седьмой 
столбец).
  
  Следующий, второй этап развития ИТ~--- ми\-ми\-ки/жес\-тов, начавшийся 
около~1,86~млн лет назад, с~кульминацией около~612~тыс.\ лет назад, на 
территориях порядка~1~км, реализовался далекими\linebreak предками современного 
человека~--- пред-людь\-ми {Homo ergaster/Homo erectus}, способными 
обеспечивать точ\-ность своих воздействий на природу\linebreak порядка~1,8~см, 
с~субстратом психики уров\-ня тканей многоклеточного организма и~уровнем 
обра\-зо\-ван\-ности, соответствующим современному ребенку~1,6--2,6~лет.

\pagebreak

\end{multicols}

\setcounter{figure}{1}
\begin{figure*} %fig2
 \vspace*{1pt}
    \begin{center}  
  \mbox{%
 \epsfxsize=163.101mm 
 \epsfbox{gri-2.eps}
 }
\end{center}
\vspace*{-6pt}
\Caption{Пространственно-временн$\acute{\mbox{ы}}$е характеристики и~тренд ИТ в~процессе генезиса 
информационного общества (по ИКМ, в~двойном логарифмическом масштабе; 
иерархическая слож\-ность~--- число уров\-ней/яру\-сов в~системной иерархии)}
\vspace*{1pt}
\end{figure*}

\begin{multicols}{2}



  
  Все последующие этапы развития ИТ~--- речь/язык, пись\-мен\-ность, 
тиражирование текстов (книгопечатание), компьютеры, телекоммуникации, 
на\-но-ИТ~--- реализовались последовательно усложняющимися формами 
{Homo sapiens}, который при этом образовывал относительно 
устойчивые и~относительно эффективно функционирующие 
и~самоуправляющиеся сообщества на все больших ареалах, одновременно 
повышая точность своих (антропогенных) действий при формировании 
вокруг себя <<второй (рукотворной) природы>>.
  
  Так, третий этап развития ИТ~--- речи/языка, начавшийся около 123~тыс.\ 
лет назад, с~кульминацией (верхнепалеолитической революцией) 
около~40~тыс.\ лет назад, на территориях порядка~15~км, реализовался 
{Homo sapiens}$^\prime$, способными обеспечивать точность своих 
производственных технологий порядка~1,2~мм, с~субстратом психики 
уровня эвкариотических клеток многоклеточного организма и~уровнем 
образованности, соответствующим современному ребенку~2,6--4,2~лет.

\begin{figure*}[b] %fig3
%\vspace*{-4pt}
    \begin{center}  
  \mbox{%
 \epsfxsize=162.821mm 
 \epsfbox{gri-3.eps}
 }
\end{center}
\vspace*{-6pt}
\Caption{Тренд изменения времен запаздывания кульминаций развития под\-сис\-тем  
иерар\-хо-се\-те\-вой сис\-те\-мы Человечества относительно их стартов (по ИКМ, в~двойном 
логарифмическом мас\-штабе)}
\end{figure*}
  
  Четвертый этап развития ИТ~--- письменности, начавшийся 
около~8,1~тыс.\ лет назад, с~кульминацией (городской революцией 
<<осевого времени>>) около 2,7~тыс.\ лет назад, на территориях 
порядка~222~км, реализовался {Homo sapiens}$^{\prime\prime}$, 
способными обеспечивать точность своих производственных технологий 
порядка~80~мкм, с~суб\-стра\-том психики уровня компартментов 
эвкариотических клеток многоклеточного организма и~уровнем 
образованности, соответствующим современному ребенку~4,2--6,9~лет 
(дошкольное образование).
  
  Пятый этап развития ИТ~--- тиражирования\linebreak текс\-тов (книгопечатания), 
начавшийся около 1446~г.\ н.\,э., с~кульминацией (промышленной\linebreak 
революцией) около 1806~г., на территориях порядка~3370~км, реализовался 
{Homo sapiens}$^{\prime\prime\prime}$, способными обеспечивать 
точность своих производственных технологий порядка~5~мкм, с~субстратом 
психики уровня субкомпартментов эвкариотических клеток многоклеточного 
организма и~уровнем об\-ра\-зо\-ван\-ности, соответствующим современному 
ребенку~6,9--11,1~лет (начальное образование).
  
  Шестой этап развития ИТ~--- компьютеров (локальных), начавшийся 
около~1946~г., с~кульминацией (изобретением микропроцессоров) 
около~1970~г., на территориях порядка~51~тыс.\ км (т.\,е.\ 
общепланетарного, или глобального размера), реализовался {Homo 
sapiens}$^{\prime\prime\prime\prime}$, способными обеспечивать точ\-ность 
своих производственных технологий порядка~0,35~мкм, с~субстратом 
психики уровня\linebreak
 ультраструктурных (прокариотических) внутриклеточных 
элементов эвкариотической клетки и~уровнем об\-ра\-зо\-ван\-ности, 
соответствующим современному  
под\-рост\-ку-юно\-ше/де\-вуш\-ке~11,1--18~лет\linebreak (среднее образование).
  
  Седьмой этап развития ИТ~--- телекоммуникаций, начавшийся около 
1979~г., с~кульминацией (пиком ско\-рости распространения на планете 
мобильной телефонии, интернета и~т.\,п.) около\linebreak
 2003~г., в~космическом 
объеме радиусом (шара)\linebreak порядка 773~тыс.\ км (т.\,е.\ в~ближнем космосе), 
реализовался {Homo sapiens}$^{\prime\prime\prime\prime\prime}$, 
способными обеспечивать точ\-ность своих производственных технологий 
порядка~23~нм, с~субстратом психики уровня мак\-ро\-мо\-ле\-кул/ге\-нов 
(компартментов\ ульт\-ра\-струк\-тур\-ных--про\-ка\-рио\-ти\-че\-ских--\linebreak
внут\-ри\-кле\-точ\-ных 
элементов эвкариотической клетки) и~уровнем 
об\-ра\-зо\-ван\-ности, со\-от\-вет\-ст\-ву\-ющим современному молодому  
че\-ло\-ве\-ку~18--29,1~лет (высшее обра\-зо\-ва\-ние\;+\;<<ас\-пи\-ран\-ту\-ра, 
с~защитой диссертации кандидата наук>>).
  
  Восьмой этап развития перспективной нано-ИТ (возможно, <<ИТ 
наноаппаратно поддерживаемой селективной телепатии>>~\cite{16-grn}), 
начавшийся около~1981~г., с~кульминацией (пиком скорости ее 
распространения на планете) около~2341~г.\ (расчетная дата), в~космическом 
объеме радиусом шара порядка~11,7~млн км (т.\,е.\ в~промежуточном 
космосе~\cite{5-grn}), реализовался {Homo 
sapiens}$^{\prime\prime\prime\prime\prime\prime }$, способными обеспечивать 
точность своих производственных технологий порядка~1,5~нм (отсюда 
наименование ИТ), с~субстратом психики уровня органических молекул 
(субкомпартментов ульт\-ра\-струк\-тур\-ных--про\-ка\-риоти\-че\-ских--внут\-ри\-кле\-точ\-ных 
элементов эвкариотической клетки) и~уровнем 
об\-ра\-зо\-ван\-ности,\linebreak соответству\-ющим современному зрелому  
человеку~29,1--47,1~лет (<<докторантура>>).
  
  Важно отметить, что процесс появления всех вышеперечисленных 
подсистем подчиняется кумулятивному принципу: возникновение каждой 
новой подсистемы не отменяет существование предыду\-щей: они все активно 
взаимодействуют между собой, коэволюционируют и~т.\,п., но исторически 
более ранние, естественно, постепенно переходят на второй, третий и~т.\,д.\ 
планы исторической сцены.
  
  Точка сходимости этого ряда находится около\linebreak 1981~г., знаменуя собой 
завершение этапа <<детст\-ва--от\-ро\-че\-ст\-ва--юности>> Человечества как 
целого и~начало этапа его <<зрелости>>~--- до\-сти\-же\-ния его максималь\-ной 
иерархической слож\-ности (чис\-ла уров\-ней/яру\-сов в~сис\-тем\-ной 
иерархии)~\cite{5-grn, 7-grn}.
  
  С позиции прогнозирования генезиса информационного общества на 
будущие времена отмечу, что, согласно ИКМ, тренд изменения времен 
запаздывания кульминаций развития под\-сис\-тем относительно их стартов 
сменился прямо на наших глазах. Если во временн$\acute{\mbox{о}}$м диапазоне с~428~млн 
лет назад и~до 1946~г.\ он со\-стоял в~равномерном (в~логарифмическом 
масштабе) укорочении согласно той же за\-ко\-но\-мер\-ности 
(в~$e\hm=15,15426\ldots$~раз), то в~диапазоне от~1946 по 1979~гг.\ это время 
запаздывания не изменилось, а~начиная с~1979~г.\ начало удлиняться 
(рис.~3). 
  

  
  Таким образом, метаэволюция сис\-те\-мы Человечества завершилась около 
1981~г.\ в~том смыс\-ле, что все воз\-мож\-ные ее ие\-рар\-хо-се\-те\-вые под\-сис\-те\-мы 
\textit{в~потенции} уже созданы. Но их \textit{актуализация}, дальнейшее 
услож\-не\-ние, эволюция и~коэволюция с~ранее возникшими аналогичными 
под\-сис\-те\-ма\-ми будет продолжаться неопределенно длительное время.

\vspace*{-10pt}
  
  \section*{Выводы}
  
  \vspace*{-2pt}
  
  \noindent
  \begin{enumerate}[1.]
\item  Изучение \textit{генезиса информационного общества} во всех его 
последовательных формах~--- от древности до современности и~далее~--- на 
базе\linebreak
 ин\-фор\-ма\-ти\-ко-ки\-бер\-не\-ти\-че\-ско\-го модельного подхода 
и~формализации процесса метаэволю\-ционного развития в~соответствующих 
терминах, позволило получить количественные\linebreak оценки его типовых  
про\-стран\-ст\-вен\-но-вре\-менн$\acute{\mbox{ы}}$х характеристик, 
а~также скоординированных с~ними во времени и~в~пространстве  
психико-ант\-ро\-по\-ло\-ги\-че\-ских, образовательных %\linebreak  
и~ин\-фор\-ма\-ци\-он\-но-ком\-му\-ни\-ка\-ци\-он\-ных параметров 
и~возможностей включенного в~этот процесс усложняющегося человека 
и~его сообществ различной величины.
  \item  Позиционирование ИТ локальных компьютеров и~ИТ 
телекоммуникаций в~качестве неотъемлемых составляющих совокупности\linebreak 
монотонно усложняющихся в~ходе цивилизационного развития~--- 
и~информационного общества!~--- ИТ позволяет 
рассматривать их появление и~функционирование в~широком контексте 
единой исторической ретроспективы и~перспективы, давая возможность 
делать не только теоретические, но и~практические выводы.
  \end{enumerate}
  
{\small\frenchspacing
 {%\baselineskip=10.8pt
 \addcontentsline{toc}{section}{References}
 \begin{thebibliography}{99}
\bibitem{1-grn}
\Au{Мелик-Гайгазян И.\,В.} Информационное общество~// Большая российская 
энциклопедия. Т.~11.~--- М.: Большая Российская энциклопедия, 2008. С.~490.
\bibitem{2-grn}
\Au{Гринченко С.\,Н.} Социальная метаэволюция Человечества как последовательность 
шагов формирования механизмов его системной памяти~// Исследовано в~России: 
Электронный журнал, 2001. Т.~145. С.~1652--1681. {\sf  
https://cyberleninka.ru/article/v/sotsialnaya-metaevolyutsiya-chelovechestva-kak-posledovatelnost-shagov-formirovaniya-mehanizmov-ego-sistemnoy-pamyati}.
\bibitem{3-grn}
\Au{Гринченко С.\,Н.} Системная память живого (как основа его метаэволюции
и~периодической структуры).~--- М.: ИПИ РАН, Мир, 2004. 512~с.
\bibitem{4-grn}
\Au{Grinchenko S.\,N.} Meta-evolution of nature system~--- the framework of history~// Social 
Evolution History, 2006. Vol.~5. No.\,1. P.~42--88.
\bibitem{5-grn}
\Au{Гринченко С.\,Н.} Метаэволюция (сис\-тем неживой, живой  
и~со\-ци\-аль\-но-тех\-но\-ло\-ги\-че\-ской природы).~--- М.: ИПИ РАН, 2007. 456~с.
\bibitem{6-grn}
\Au{Гринченко С.\,Н.} Homo eruditus (человек образованный) как элемент сис\-те\-мы 
Человечества~// Открытое образование, 2009. №\,2. С.~48--55.

\bibitem{10-grn} %7
\Au{Гринченко С.\,Н., Щапова~Ю.\,Л.} История Человечества: модели периодизации~// 
Вестник РАН, 2010. №\,12. С.~1076--1084.

%\bibitem{11-grn}  %8
%\Au{Grinchenko S.\,N., Shchapova~Y.\,L.} Human history periodization models~// Herald of the 
%Russian Academy of Sciences, 2010. Vol.~80. No.\,6. P.~498--506.
\bibitem{7-grn} %9
\Au{Grinchenko S.\,N.} The pre- and post-history of Humankind: What is it?~// Problems of 
contemporary world futurology.~--- Newcastle-upon-Tyne: Cambridge Scholars Publishing, 
2011. P.~341--353.
\bibitem{8-grn} %10
\Au{Гринченко С.\,Н.} Об эволюции психики как иерархической сис\-те\-мы 
(кибернетическое пред\-став\-ле\-ние)~// Историческая психология и~социология истории, 
2012. Т.~5. №\,2. С.~60--76.

\bibitem{12-grn} %11
\Au{Гринченко С.\,Н., Щапова~Ю.\,Л.} Информационные технологии в~истории 
Человечества.~--- М.: Новые технологии, 2013. 32~с. (Приложение к~журналу 
<<Информационные технологии>>, 2013. №\,8.)

\bibitem{9-grn} %12
\Au{Гринченко С.\,Н.} Эволюция темпов жизни людей и~развитие человечества~// Человек, 
2014. №\,5. С.~28--36.



\bibitem{13-grn}
\Au{Grinchenko S.\,N., Shchapova~Y.\,L.} Archaeological epoch as the succession of generations 
of evolutive subject-carrier archaeological sub-epoch~// Philosophy of Nature in Cross-Cultural 
Dimensions: The Result of the International Symposium at the University of Vienna~/ 
Komparative Philosophie und Interdisziplin$\ddot{\mbox{a}}$re Bildung (KoPhil). Band~5.~--- 
Hamburg: Verlag Dr.\ Kova$\Check{\mbox{c}}$, 2017. P.~478--499.
\bibitem{14-grn}
\Au{Щапова Ю.\,Л., Гринченко~С.\,Н.} Введение в~теорию археологической эпохи: 
числовое моделирование и~логарифмические шкалы про\-стран\-ст\-вен\-но-вре\-мен\-ных 
координат.~--- М.: Истфак МГУ, ФИЦ ИУ РАН, 2017. 236~с.
\bibitem{15-grn}
\Au{Grinchenko S.\,N., Shchapova~Yu.\,L.} Communications: Model representations about 
historical retrospective and possible perspective~// Communications Media 
Design Electronic~J., 2018. Vol.~3. No.\,2. P.~65--78.
\bibitem{16-grn}
\Au{Гринченко С.\,Н.} Послесловие~// Мат-лы доклада на Совместном научном семинаре 
ИПИ РАН и~\mbox{ИНИОН} РАН <<Методологические проблемы наук об информации>>.~---
М., 2012. С.~5--8. {\sf 
http://legacy.\linebreak inion.ru/files/File/MPNI\_9\_13\_12\_12\_posl.pdf}.
\bibitem{17-grn}
\Au{Жирмунский А.\,В., Кузьмин~В.\,И.} Критические уровни в~процессах развития 
биологических систем.~--- М.: Наука, 1982. 179~с.
\bibitem{18-grn}
\Au{Щапова Ю.\,Л.} Хронология и~периодизации древнейшей истории как числовая 
последовательность (ряд Фибоначчи)~// Информационный бюллетень Ассоциации 
<<История и~компьютер>>, 2000. №\,25.
\bibitem{19-grn}
\Au{Щапова Ю.\,Л.} Археологическая эпоха: хронология, периодизация, теория,  
модель.~--- М.: КомКнига, 2005. 192~с.
\bibitem{20-grn}
\Au{Щапова Ю.\,Л.} Материальное производство в~археологическую эпоху.~--- СПб.: 
Алетейя, 2011. 244~с.
\bibitem{21-grn}
\Au{Гринченко С.\,Н., Щапова~Ю.\,Л.} Пространство и~время в~археологии. Часть~3. 
О~метрике базисной пространственной структуры человечества в~археологическую 
эпоху~// Пространство и~время, 2014. №\,1(15). С.~78--89.
 \end{thebibliography}

 }
 }

\end{multicols}

\vspace*{-8pt}

\hfill{\small\textit{Поступила в~редакцию 17.10.18}}

\vspace*{6pt}

%\pagebreak

%\newpage

%\vspace*{-29pt}

\hrule

\vspace*{2pt}

\hrule

%\vspace*{-2pt}

\def\tit{ON THE GENESIS OF~THE~INFORMATION SOCIETY: INFORMATICS-CYBERNETIC 
MODEL REPRESENTATION}


\def\titkol{On the genesis of~the~information society: Informatics-cybernetic 
model representation}

\def\aut{S.\,N.~Grinchenko}

\def\autkol{S.\,N.~Grinchenko}

\titel{\tit}{\aut}{\autkol}{\titkol}

\vspace*{-11pt}


\noindent
Institute of Informatics Problems of the Federal Research Center ``Informatics and Control'' of 
the Russian Academy of Sciences, 44-2~Vavilov Str., Moscow 119333, Russian Federation

\def\leftfootline{\small{\textbf{\thepage}
\hfill INFORMATIKA I EE PRIMENENIYA~--- INFORMATICS AND
APPLICATIONS\ \ \ 2019\ \ \ volume~13\ \ \ issue\ 2}
}%
 \def\rightfootline{\small{INFORMATIKA I EE PRIMENENIYA~---
INFORMATICS AND APPLICATIONS\ \ \ 2019\ \ \ volume~13\ \ \ issue\ 2
\hfill \textbf{\thepage}}}

\vspace*{6pt}


  
  \Abste{The concept of the information society genesis is introduced, which is 
viewed from the standpoint of informatics-cybernetic modeling of the development 
of Humankind as a self-controlling hierarchical-networking system. On this basis, 
the author obtained quantitative assessments of its typical spatial-temporal 
characteristics, representing geometric progressions with the denominator ``$e$ to the 
degree~$e$'' (15.15426$\ldots$), as well as coordinated with them in time and space 
of the psychoanthropological, educational, and informational communication 
parameters and possibilities of a person who becomes complicated in this process 
and his communities of various sizes. This allowed us to push the framework of 
the information society for the entire historical and even archaeological epoch of 
such development. The resulting sequence of information technologies ``signal 
poses\,/\,sounds/movements\,--\,mimics/gestures\,--\,speech/language\,--\,writing\,--\,replicating 
texts\,--\,computers\,--\,telecommunications\,--\,information 
nanotechnology\,--\,$\ldots$'' allows us to consider the information society genesis 
in the broad context of a unified historical retrospective and perspective.}
  
  \KWE{information society; information technologies; informatics-cybernetic 
model; self-controlling hierarchical-networking system of Humankind; 
archaeological epoch}
  

\DOI{10.14357/19922264190214}

%\vspace*{-14pt}

%\Ack
%\noindent



%\vspace*{6pt}

  \begin{multicols}{2}

\renewcommand{\bibname}{\protect\rmfamily References}
%\renewcommand{\bibname}{\large\protect\rm References}

{\small\frenchspacing
 {%\baselineskip=10.8pt
 \addcontentsline{toc}{section}{References}
 \begin{thebibliography}{99}

\bibitem{1-grn-1}
\Aue{Melik-Gaygazyan, I.\,V.} 2008. Informatsionnoe ob\-shche\-st\-vo [Information 
society]. \textit{Bol'shaya rossiyskaya entsiklopediya} [Great Russian 
Encyclopedia].  Moscow: Great Russian 
Encyclopedia Publs. 11:490.
\bibitem{2-grn-1}
\Aue{Grinchenko, S.\,N.} 2001. Sotsial'naya me\-ta\-evo\-lyu\-tsiya Chelovechestva kak 
posledovatel'nost' shagov for\-mi\-ro\-va\-niya mekhanizmov ego sistemnoy pamyati 
[Social meta-evolution of Mankind as a~sequence of steps for the formation of the 
mechanisms of its system memory]. \textit{Elektronnyy zhurnal <<Issledovano 
v~Rossii>>} [Electronical J. ``Invstigated in Russia'']. 145:1652--1681. Avalable 
at: {\sf  
https://cyberleninka.ru/article/v/sotsialnaya-metaevolyutsiya-chelovechestva-kak-posledovatelnost-shagov-formirovaniya-mehanizmov-ego-sistemnoy-pamyati} (accessed 
October~5, 2018).
\bibitem{3-grn-1}
\Aue{Grinchenko, S.\,N.} 2004. \textit{Sistemnaya pamyat' zhivogo (kak osnova 
ego metaevolyutsii i~periodicheskoy struktury)} [System memory of the life (as the 
basis of its meta-evolution and periodic structure)]. Moscow: IPIRAN, MIR. 
512~p.
\bibitem{4-grn-1}
\Aue{Grinchenko, S.\,N.} 2006. Meta-evolution of nature system~--- the 
framework of history. \textit{Social Evolution History} 5(1):42--88.
\bibitem{5-grn-1}
\Aue{Grinchenko, S.\,N.} 2007. \textit{Metaevolyutsiya (sistem nezhivoy, zhivoy 
i~sotsial'no-tekhnologicheskoy prirody)} [Meta-evolution (of inanimate, animate, 
and socio-technological nature systems)]. Moscow: IPIRAN. 456~p. 
\bibitem{6-grn-1}
\Aue{Grinchenko, S.\,N.} 2009. Homo eruditus (chelovek obrazovannyy) kak 
element sistemy Chelovechestva [Homo eruditus (educated human) as an element 
of the Humakind's system]. \textit{Otkrytoe obrazovanie} [Open Education]  
2:48--55.

\bibitem{10-grn-1} %7
\Aue{Grinchenko, S.\,N., and Yu.\,I.~Shchapova.} 2010. 
Human history periodization models. \textit{Her. Russ. Acad. Sci.} 80(6):498--506.
%\bibitem{11-grn-1} %8
%\Aue{Grinchenko, S.\,N., and Y.\,I.~Shchapova.}  2010. Human history 
%periodization models. \textit{Herald of the Russian Academy of Sciences} 
%80(6):498--506.

\bibitem{7-grn-1} %9
\Aue{Grinchenko, S.\,N.} 2011.The pre- and post-history of Humankind: What is 
it?  \textit{Problems of contemporary world futurology}. 
 Newcastle-upon-Tyne: Cambridge Scholars 
Publishing.  341--353.
\bibitem{8-grn-1} %10
\Aue{Grinchenko, S.\,N.} 2012. Ob evolyutsii psikhiki kak ie\-rar\-khi\-che\-skoy 
sistemy (kiberneticheskoe predstavlenie) [On the evolution of mind as 
a~hierarchical system (a~cybernetic approach)]. \textit{Istoricheskaya 
psikhologiya i~sotsiologiya istorii} [Historical Psychology \& Sociology of 
History] 6(2):\linebreak 60--77.


\bibitem{12-grn-1} %11
\Aue{Grinchenko, S.\,N., and Y.\,I.~Shchapova.} 2013. \textit{In\-for\-ma\-tsi\-on\-nye 
tekhnologii v~istorii Chelovechestva} [Information technology in the history of 
Humankind]. Moscow: Novye tekhnologii. 32~p. (Prilozhenie k zhurnalu 
<<\textit{Informatsionnye tekhnologii}>> [Supplement to J.~Information Technology] 8.

\bibitem{9-grn-1} %12
\Aue{Grinchenko, S.\,N.} 2014. Evolyutsiya tempov zhizni lyudey i~razvitie 
chelovechestva [The evolution of the pace of human life and human development]. 
\textit{Human Being} 5:28--36.

\bibitem{13-grn-1}
\Aue{Grinchenko, S.\,N., and Y.\,I.~Shchapova.} 2017. Archaeological epoch as 
the succession of generations of evolutive subject-carrier archaeological  
sub-epoch. \textit{Philosophy of Nature in Cross-Cultural Dimensions: The Result of 
the International Symposium at the University of Vienna}~/ Komparative 
Philosophie und Interdisziplin$\ddot{\mbox{a}}$re Bildung (KoPhil), Band~5. 
Hamburg: Verlag Dr.\ Kova$\Check{\mbox{c}}$.  478--499.
\bibitem{14-grn-1}
\Aue{Shchapova, Y.\,L., and S.\,N.~Grinchenko.} 2017. \textit{Vvedenie 
v~teoriyu arkheologicheskoy epokhi: chislovoe modelirovanie i~logarifmicheskie 
shkaly prostranstvenno-vremennykh koordinat} [Introduction to the theory of the 
archaeological epoch: Numerical modeling and logarithmic scales of space--time 
coordinates]. Moscow: Faculty 
of History MSU, FRC CSC RAS]. 236~p. 

\vspace*{1pt}

\bibitem{15-grn-1}
\Aue{Grinchenko, S.\,N., and Y.\,I.~Shchapova}. 2018.  Communications: Model 
representations about historical retrospective and possible perspective. 
\textit{Communications Media Design Electronic~J.}  3(2):65--78. 
Available at: {\sf https://elibrary.ru/item.asp?id=36272286} (accessed October~5, 
2018).

\vspace*{1pt}

\bibitem{16-grn-1}
\Aue{Grinchenko, S.\,N.} 2012. Posleslovie [Afterword]. \textit{Mat-ly doklada 
na Sovmestnom nauchnom seminare IPI \mbox{INION} RAN ``Metodologicheskie 
problemy nauk ob informatsii''}  [Report materials at the Joint Scientific 
Seminar of the Institute of Informatics Problems of the Russian Academy of 
Sciences and the Institute of Scientific Information on Social Sciences of the 
Russian Academy of Sciences ``Methodological problems of information 
sciences''].  Moscow. 5--8.  Available at: {\sf 
http://legacy. inion.ru/files/File/MPNI\_9\_13\_12\_12\_posl.pdf} (accessed 
October~5, 2018).

\vspace*{1pt}

\bibitem{17-grn-1}
\Aue{Zhirmunskiy, A.\,V., and V.\,I.~Kuz'min.} 1982. \textit{Kriticheskie urovni 
v~protsessakh razvitiya biologicheskikh sistem} [Critical levels in the development 
of biological systems]. Moscow: Nauka. 179~p.

\vspace*{1pt}

\bibitem{18-grn-1}
\Aue{Shchapova, Y.\,L.} 2000. Khronologiya i~periodizatsii drev\-ney\-shey istorii 
kak chislovaya posledovatel'nost' (ryad Fibonachchi) [Chronology and 
periodization of ancient history as a numerical sequence (Fibonacci's series)]. 
\textit{Informatsionnyy byulleten' Assotsiatsii ``Istoriya i~komp'yuter''} 
[Newsletter of the Association ``History and Computer'']  25.

\vspace*{1pt}

\bibitem{19-grn-1}
\Aue{Shchapova, Y.\,L.} 2005. \textit{Arkheologicheskaya epokha: khro\-no\-lo\-giya, 
periodizatsiya, teoriya, model'} [Archaeological epoch: Chronology, periodization, 
theory, model]. Moscow: KomKniga, 192~p.

\vspace*{1pt}

\bibitem{20-grn-1}
\Aue{Shchapova, Y.\,L.} 2011. \textit{Material'noe proizvodstvo 
v~arkheologicheskuyu epokhu} [Material production in the archaeological epoch]. 
St.\ Petersburg: Aleteyya. 244~p.

\vspace*{1pt}

\bibitem{21-grn-1}
\Aue{Grinchenko, S.\,N., and Yu.\,I.~Shchapova.} 2014. Prostranstvo i~vremya 
v~arheologii. Chast'~3. O~metrike bazisnoy prostranstvennoy struktury 
chelovechestva v~arkheologicheskuyu epokhu [Space and time in archeology. 
Part~3. About the metric of Humankind basic spatial structure  in  
archaeological epoch]. \textit{Space and Time}  
1(15):\linebreak 78--89.
\end{thebibliography}

 }
 }

\end{multicols}

\vspace*{-6pt}

\hfill{\small\textit{Received October 17, 2018}}

%\pagebreak

%\vspace*{-18pt}


  
  \Contrl
  
  \noindent
   \textbf{Grinchenko Sergey N.} (b.\ 1946)~--- Doctor of Science in technology, professor, principal 
scientist, Institute of Informatics Problems, Federal Research Center ``Computer Science and 
Control'' of the Russian Academy of Sciences, 44-2~Vavilov Str., Moscow 119333, Russian 
Federation; \mbox{sgrinchenko@ipiran.ru}
\label{end\stat}

\renewcommand{\bibname}{\protect\rm Литература}  