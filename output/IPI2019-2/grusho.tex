\def\stat{grusho}

\def\tit{АРХИТЕКТУРНЫЕ РЕШЕНИЯ В~ЗАДАЧЕ ВЫЯВЛЕНИЯ МОШЕННИЧЕСТВА ПРИ~АНАЛИЗЕ 
ИНФОРМАЦИОННЫХ ПОТОКОВ В~ЦИФРОВОЙ ЭКОНОМИКЕ$^*$}

\def\titkol{Архитектурные решения в~задаче выявления мошенничества при~анализе 
информационных потоков в
%~цифровой 
экономике}

\def\aut{А.\,А.~Грушо$^1$, М.\,И.~Забежайло$^2$, Н.\,А.~Грушо$^3$, 
Е.\,Е.~Тимонина$^4$}

\def\autkol{А.\,А.~Грушо, М.\,И.~Забежайло, Н.\,А.~Грушо, 
Е.\,Е.~Тимонина}

\titel{\tit}{\aut}{\autkol}{\titkol}

\index{Грушо А.\,А.}
\index{Забежайло М.\,И.}
\index{Грушо Н.\,А.}
\index{Тимонина Е.\,Е.}
\index{Grusho A.\,A.}
\index{Zabezhailo M.\,I.}
\index{Grusho N.\,A.}
\index{Timonina E.\,E.}


{\renewcommand{\thefootnote}{\fnsymbol{footnote}} \footnotetext[1]
{Работа частично поддержана РФФИ (проекты 18-29-03081 и~18-07-00274).}}


\renewcommand{\thefootnote}{\arabic{footnote}}
\footnotetext[1]{Институт проблем информатики Федерального исследовательского центра <<Информатика и~управление>> 
Российской академии наук, grusho@yandex.ru}
\footnotetext[2]{Институт проблем информатики Федерального исследовательского центра <<Информатика и~управление>> 
Российской академии наук, m.zabezhailo@yandex.ru}
\footnotetext[3]{Институт проблем информатики Федерального исследовательского центра <<Информатика и~управление>> 
Российской академии наук, info@itake.ru}
\footnotetext[4]{Институт проблем информатики Федерального исследовательского центра <<Информатика и~управление>> 
Российской академии наук, eltimon@yandex.ru}

\vspace*{-12pt}
   

 
  
  \Abst{Cформулирован подход к~исследованию некоторых видов мошенничества в~цифровой 
экономике с~использованием причинно-следственных связей. Во всех видах рассматриваемых 
мошенничеств должно наблюдаться несоответствие между целями финансовых транзакций 
и~реальной стоимостью достижения этих целей. Данные о транзакциях можно собирать, 
наблюдая информационные потоки, в~которых отражаются эти транзакции. Архитектура сбора 
данных и~их анализа может быть организована с~помощью распределенных реестров 
с~централизованным консенсусом, что позволяет создать аналог электронной бухгалтерской 
книги, фиксирующей финансово-экономическую деятельность субъектов цифровой экономики в~регионе. 
  Рассматриваемые методы выявления мошенничества основаны на противоречиях 
между действиями, описанными в~транзакциях, и~информацией, содержащейся в~планах, 
стандартах, прецедентах и~др. Рассмотрен метод, основанный на некоторой упрощенной схеме 
реализации абстрактного проекта. Для выявления противоречий необходимо проводить анализ 
от следствия к~причине, т.\,е.\ искать аномалии в~информации, описывающей порождение 
наблюдаемых следствий. 
  Показано, как в~реализации проекта можно выделять простые <<необходимые условия>> 
нарушения при\-чин\-но-след\-ст\-вен\-ных связей, т.\,е.\ множество <<необходимых условий>>, 
нарушение которых свидетельствует о наличии мошенничества. Это множество <<необходимых 
условий>> можно назвать метаданными для контроля проекта на выявление мошенничества.} 
 
 
  \KW{цифровая экономика; информационные потоки; при\-чин\-но-след\-ст\-вен\-ные связи; 
выявление мошеннических схем} 

\DOI{10.14357/19922264190204}
  
\vspace*{-4pt}


\vskip 10pt plus 9pt minus 6pt

\thispagestyle{headings}

\begin{multicols}{2}

\label{st\stat}

\section{Введение}

\vspace*{3pt}

  В работе сформулирован подход к~исследованию некоторых видов 
мошенничества в~цифровой экономике с~использованием  
при\-чин\-но-след\-ст\-вен\-ных связей. Рассматриваются три вида мошенничества, 
а именно:
  \begin{enumerate}[(1)]
\item отмыв денег; 
\item обман при выполнении договорных обязательств при реализации 
технических проектов (строительные проекты и~др.); 
\item незаконный вывод денег. 
\end{enumerate}

  Названные виды мошенничества могут быть сведены к~решению одного типа 
задач. Для отмывания денег источник должен заключать фиктивные контракты, 
в~соответствии с~которыми будут переводиться средства за заведомо ненужную 
работу и~материалы. 
  
  Мошенничество, связанное с~невыполнением договорных обязательств, связано 
со снижением качества услуг, качества и~количества закупаемых 
материалов, выполнением работ с~ненадлежащим качеством. 
  
  Вывод денег связан с~переводом средств фир\-мам-од\-но\-днев\-кам, которые 
заведомо не могут выполнить обязательства по контрактам, за которые им 
переводятся средства. 
  
  Таким образом, во всех трех видах рассматриваемых мошенничеств должно 
наблюдаться несоответствие между целями финансовых транзакций и~реальной 
стоимостью достижения этих целей. Данные о транзакциях можно собирать, 
наблюдая информационные потоки, в~которых отражаются эти транзакции. 
  
  Однако для наблюдения таких информационных потоков необходимо создавать 
архитектуру\linebreak телекоммуникационной системы, позволяющей перехватывать 
и~собирать данные о всех транзакциях. Например, такая архитектура может быть 
организована с~помощью распределенных реестров с~централизованным 
консенсусом, т.\,е.\ все информационные потоки, сформированные в~цифровой 
экономике и~несущие информацию о транзакциях, проходят через некоторый 
центральный узел, запоминающий их в~форме распределенного реестра. Такие 
реестры могут дублироваться в~аналогичных центрах различных регионов, что 
позволяет создать аналог электронной бухгалтерской книги, фиксирующей 
фи\-нан\-со\-во-эко\-но\-ми\-че\-скую деятельность субъектов цифровой экономики. Такой 
подход предложено реализовать на базе системы ситуационных центров, что 
отражено в~работах~[1, 2].
  
  Собранная из информационных потоков информация о~транзакциях, т.\,е.\ 
о~контрактах, договорах, платежах, отчетах, закупленных материалах, 
характеристиках исполнителей работ и~др., собирается в~базе данных в~указанном 
центре. Согласно теории интеллектуальных сис\-тем~[3], эту базу данных можно 
называть базой фактов (БФ). Базу фактов можно представить как бинарную мат\-ри\-цу, 
строки которой описывают характеристики, входящие в~транзакции, а столбцы 
нумеруются характеристиками. Строки матрицы будем называть 
\textit{объектами}~[4, 5]. 
  
  Рассматриваемые в~работе методы выявления мошенничества будут основаны 
на противоречиях между действиями, описанными в~транзакциях, и~информацией, 
содержащейся в~планах, стандартах, прецедентах и~др. Для нахождения 
противоречий в~архитектуре центра предусмотрена другая база данных~--- база 
знаний (БЗ)~\cite{3-gr, 6-gr}, которая устроена так же, как БФ. 
  
  Информация в~БЗ собирается на основе положительного опыта или расчетов. 
Используя БЗ, можно выводить факты нарушения при\-чин\-но-след\-ст\-вен\-ных 
связей. Нарушения при\-чин\-но-след\-ст\-вен\-ных связей будем называть 
\textit{аномалиями}. 
  
  Для упрощения дальнейшее изложение будет вестись в~рамках поиска 
противоречий при выполнении некоторого абстрактного проекта. Выявление 
аномалий будет происходить на основе фактов из БФ с~помощью знаний из БЗ 
методами искусственного интеллекта и~интеллектуального анализа 
данных~\cite{6-gr}. 

\vspace*{-10pt}
  
  \section{Модели}
  
  \vspace*{-3pt}
  
  Наиболее сложная из рассмотренных выше задач~--- выявление противоречий, 
т.\,е.\ использование БЗ для получения новых знаний и~выявление аномалий из 
полученных фактов. 
  
  Все способы выявления противоречий основаны на определении 
  причинно-следственных связей. При этом противоречия в~параметрах транзакций по 
отношению к~требуемым в~БЗ составляют сущность аномалий. 
  
   Далее будет рассмотрен метод, основанный на некоторой упрощенной схеме 
реализации абстрактного проекта. 
  
  Каждый проект имеет цель: например, цель представляет собой построение 
некоторой системы. Воспользуемся структурным подходом, который позволяет 
строить проект на основе разбиения системы на подсистемы и~определения 
взаимодействий подсистем~\cite{7-gr}. При этом каждая подсистема также 
представима структурной моделью. 
  
  Как сама система, так и~каждая ее подсистема имеют свой функционал 
и~спецификацию, па\-ра\-мет\-ры настройки и~домены параметров настройки. Кроме 
этих характеристик существует множество характеристик, связанных 
с~<<жизненным циклом>> создания системы. Сюда входят работы, ресурсы, 
сроки выполнения работ по созданию подсистем и~самой системы, стоимости 
компонентов и~материалов, стоимости работ, схемы поставок, договорные 
обязательства и~др. Все характеристики связаны между собой, поэтому можно 
говорить о стоимости и~времени изготовления структурных компонентов системы. 
  
  Одной из важнейших характеристик является смета (система смет для 
подсистем). Смета сопоставляет каждому компоненту системы стоимость его 
изготовления и~настройки. 
  
  Схема построения системы может быть пред\-став\-ле\-на диаграммой, 
изображенной на рис.~1. 

{ \begin{center}  %fig1
 \vspace*{9pt}
   \mbox{%
 \epsfxsize=79mm 
 \epsfbox{gru-1.eps}
 }


\vspace*{9pt}


\noindent
{{\figurename~1}\ \ \small{Диаграмма достижения цели}}
\end{center}
}

\vspace*{9pt}

\addtocounter{figure}{1}
  
  


  Представленная на рис.~1 диаграмма позволяет описать основные классы 
возможных противоречий при достижении цели. Противоречия возникают, когда 
данные БФ не соответствуют требуемым характеристикам. 
  
  
  \section{Потенциальные классы аномалий при~достижении цели}
  
  Выделим четыре потенциальных класса противоречий, которые показывают, 
каким образом нужно искать эти противоречия.
  
 
  Противоречие цели и~проекта (рис.~2) возникает при отсутствии обоснования 
или в~случае логического противоречия между возможностями проектируемого 
функционала и~целью системы. Отметим, что в~проект входят сроки, перечень 
работ, материалы, настройки, которые описываются соответствующими 
параметрами и~допустимыми значениями этих параметров. Проект формируется 
на основе БЗ и~расчетов, исходя из информации, полученной по аналогии 
с~другими проектами и~решениями, которые считаются апробированными. 
  
  Отметим, что цель порождает проект и~в этом смысле является причиной 
проекта. Однако для анализа противоречий необходимо двигаться по штриховой 
стрелке диаграммы (см.\ рис.~2) от проекта к~цели. В~самом деле, любой компонент 
проекта направлен на теоретическое достижение цели. Цель~--- сложный объект, 
поэтому в~проекте могут возникнуть характеристики, противоречащие хотя бы 
некоторым характеристикам цели. Это делает проект противоречивым, но вывод 
об этом может быть сделан только на уровне описания цели. 
  

  Противоречия между проектом и~его реализацией, исключая настройки 
(рис.~3), могут возникать, например, при закупке исполнителем материалов более 
низкого качества по более низким ценам, при попытках достижения требуемых 
сроков работы за счет снижения качества выполнения работ, за счет нахождения 
<<объективных>> причин для увеличения сроков работы и,~следовательно, 
увеличения цены реализации проекта. 


  Для выявления указанных противоречий необходимо двигаться по диаграмме 
(см.\ рис.~3) в~обратную сторону в~соответствии со~штриховыми стрелками. 
Действительно, выявить противоречия между характеристиками закупленных 
материалов и~требуемыми по проекту можно только при обращении к~проекту 
и~его спецификациям. Манипуляции со сроками работы также можно выявить 
только при обращении к~соответствующим расчетам в~проекте. Задержки в~сроках 
работы, связанные с~поставками материалов, можно определить только на 
предыдущем этапе диаграммы (см.\ рис.~3) в~описании проекта. 


  


  Противоречия между реализацией проекта и~его настройкой (рис.~4) возникает, 
когда не удается добиться требуемых значений параметров функционала, не 
удается обеспечить необходимый уровень\linebreak\vspace*{-12pt}

{ \begin{center}  %fig2
 \vspace*{-6pt}
   \mbox{%
 \epsfxsize=16mm 
 \epsfbox{gru-2.eps}
 }


\vspace*{6pt}


\noindent
{{\figurename~2}\ \ \small{Противоречия цели и~проекта}}
\end{center}
}

%\vspace*{9pt}

\addtocounter{figure}{1}

{ \begin{center}  %fig3
 \vspace*{6pt}
    \mbox{%
 \epsfxsize=79mm 
 \epsfbox{gru-3.eps}
 }


\end{center}

\vspace*{-2pt}


\noindent
{{\figurename~3}\ \ \small{Противоречия проекта и~его реализации (без настройки)}}
}

\vspace*{6pt}

\addtocounter{figure}{1}

{ \begin{center}  %fig4
 \vspace*{1pt}
   \mbox{%
 \epsfxsize=54.5mm 
 \epsfbox{gru-4.eps}
 }


\end{center}


\noindent
{{\figurename~4}\ \ \small{Противоречия реализации проекта и~его на\-стройки}}
}

%\vspace*{9pt}

\addtocounter{figure}{1}

{ \begin{center}  %fig5
 \vspace*{5pt}
    \mbox{%
 \epsfxsize=79mm 
 \epsfbox{gru-5.eps}
 }


\end{center}



\noindent
{{\figurename~5}\ \ \small{Противоречия цели и~достигнутой реализации проекта}}
}

\vspace*{6pt}

\addtocounter{figure}{1}

\noindent
 качества реализации проекта. Для 
определения противоречия в~настройках надо опять же двигаться по диаграмме 
(см.\ рис.~4) в~обратную сторону по штриховым стрелкам, так как для выявления 
характеристик результатов работы, которые не дают возможности реализации 
определенного функционала, необходимо иметь информацию о результатах этой 
работы. 


  



  Противоречие между целью и~достигнутой реализацией проекта (рис.~5) 
возникает, когда реализованная система не позволяет достичь цели. В~этом случае 
опять противоречие нужно искать, двигаясь от цели к~реальному достигнутому 
функционалу по штриховой стрелке (см.\ рис.~5).
  
  Суммируя положения, изложенные в~данном разделе, приходим к~выводу, что 
для выявления противоречий необходимо проводить анализ от следствия 
к~причине, т.\,е.\ искать аномалии в~информации, описывающей порождение 
наблюдаемых следствий. 
  
  
  \section{Связь противоречий и~причин}
  
  Прежде чем построить связь между причинами и~противоречиями, кратко 
опишем простейшую модель связи этих понятий. Причины и~противоречия будут 
сформулированы для представления компонентов системы как объектов, 
обладающих наборами известных характеристик~\cite{4-gr, 5-gr}. 
  
  Пусть $U\hm=\{\alpha, \beta, \ldots\}$~--- совокупность характеристик 
(пространство характеристик). Согласно~\cite{4-gr} \textit{объектом}~$O$ 
называется любое подмножество характеристик $O\hm\subseteq U$. Рассмотрим 
последовательность объектов, возможно в~различных пространствах 
характеристик. 
  
  \smallskip
  
  \noindent
  \textbf{Определение~1.}\ Объект~$P$ с~числом характеристик, большим или 
равным~2, является \textit{причиной} объекта (\textit{свойства})~$B$ в~цепочке 
наблюдаемых объектов тогда и~только тогда, когда выполнены следующие 
условия:
  \begin{enumerate}[(1)]
\item для каждого объекта~$C$, если $P\hm\subseteq C$, то $C\mapsto B$, где 
$C\mapsto B$ означает, что объект~$B$ присутствует в~объекте, следующем за 
объектом~$C$;
\item объект~$P$ является минимальным объектом, удовлетворяющим 
условию~1, а~именно: $\forall \alpha\hm\in P$ объект~$P\backslash \{\alpha\}$ 
не является причиной, т.\,е.\ $\exists C:\ \alpha\not\in C$, $P\backslash 
\{\alpha\}\hm\subseteq C$ и~$C\not\mapsto B$, где $C\not\mapsto B$ означает, 
что~$B$ не может содержаться в~объекте, следующем за объектом~$C$. 
\end{enumerate}

  Приведенное определение причины является упрощением причин, 
возникающих в~реальном мире. Например, реальные причины могут возникать\linebreak 
как совокупность характеристик из разных пространств. Одно следствие может 
порождаться разными причинами или возникать из внешних\linebreak и~ненаблюдаемых 
характеристик. Однако пред\-став\-лен\-ная далее формализация позволяет доступно 
изложить при\-чин\-но-след\-ст\-вен\-ные истоки противоречий, которые 
инициируют в~дальнейшем глубокое исследование рассматриваемых процессов.
  
  Будем считать, что для любого интересующего нас свойства~$B$ существует 
причина. Тогда справедлива следующая теорема.
  
  \smallskip
  
  \noindent
  \textbf{Теорема~1.}\ \textit{Для любого свойства~$B$ существует 
единственная причина}. 
  
  \smallskip
  
  \noindent
  Д\,о\,к\,а\,з\,а\,т\,е\,л\,ь\,с\,т\,в\,о\,.\ \ Доказательство будем вести от противного, 
т.\,е.\ предположим, что существуют две причины свойства~$B$: $P$ 
и~$P^\prime$, $P\hm\not= P^\prime$. Тогда существует $\alpha\hm\in U$, которое 
удовлетворяет одному из двух условий:
  \begin{itemize}
\item[(а)] $\alpha\in P$, $\alpha\notin P^\prime$;
\item[(б)] $\alpha\notin P$, $\alpha \in P^\prime$.
\end{itemize}

  Пусть выполняется условие~(б). Тогда $P^\prime\backslash \{\alpha\}$ не 
является причиной по условию~2 определения~1, т.\,е.\ $\exists C$ такое, что 
$\alpha\notin C$, $P^\prime\backslash \{\alpha\}\hm\subseteq C$ и~$C\not\mapsto B$. 
Но если~$B$ произошло и~$P$ его причина, то $C\mapsto B$, что противоречит 
предположению. Теорема~1 доказана.
  
  \smallskip
  
  \noindent
  \textbf{Лемма.} \textit{Если $P$~--- причина появления свойства~$B$, то 
объект~$B$ определяет существование свойства~$P$ в~объекте, 
предшествующем~$B$. }
  
  \smallskip
  
  \noindent
  Д\,о\,к\,а\,з\,а\,т\,е\,л\,ь\,с\,т\,в\,о\,.\ \ Из предположения, что у~каж\-до\-го 
свойства~$B$ есть причина, и~условия, что~$P$ является причиной~$B$, следует, 
что при появлении в~данных свойства~$B$ объект~$C$, предшествующий 
появлению~$B$, содержит как часть объект~$P$. Это следует из теоремы~1 
и~определения причины. 
  
  Докажем принцип <<необходимого условия>>, который, несмотря на простоту 
доказательства, будет играть в~дальнейшем существенную роль.
  
  \smallskip
  
  \noindent
  \textbf{Теорема~2.} \textit{Если~$P$~--- причина появления свойства~$B$ 
и~$A\hm\subseteq P$, то объект~$B$ определяет наличие свойства~$A$ 
в~объекте, предшествующем~$B$}. 
  
  \smallskip
  
  \noindent
  Д\,о\,к\,а\,з\,а\,т\,е\,л\,ь\,с\,т\,в\,о\,.\ \ Пусть в~данных имеется объект~$B$ 
и~$P\mapsto B$, тогда в~силу существования и~единственности причины~$B$ 
в~данных должен существовать объект~$C$, предшествующий~$B$ 
и~содержащий причину~$P$. Поскольку $A\hm\subseteq P$ и~$B$ содержит 
причину~$P$, то $B\mapsto A$. С~учетом леммы теорема~2 доказана.
  
  \smallskip
  
  Пусть даны пространства $U_1, U_2,\ldots$ и~имеется последовательность 
данных (процесс выполнения этапов проекта в~соответствии с~рис.~1) $A, B, 
\ldots$, где каждый объект является подмножеством некоторого 
пространства~$U_i$, $i\hm=1,\ldots$ Тогда в~объекте~$A$ присутствует 
причина~$P$ появления интересующего нас свойства~$C$ в~объекте~$B$. Пусть 
$P\hm\subseteq A$, тогда по теореме~2 $\forall \alpha\hm\in P$:  
$C\mapsto \{\alpha\}$, т.\,е.\ из появления~$C$ следует появление 
характеристики~$\alpha$ в~предшествующем объекте. Это необходимое условие 
того, что~$C$ удовлетворяет причинно-следственным связям развития процесса 
выполнения проекта. Если для~$C$ нет характеристики~$\alpha$, которую можно 
отнести к~причине~$C$, то можно считать, что нарушена  
при\-чин\-но-след\-ст\-вен\-ная связь и~$C$~--- аномальный объект. 
  
  \smallskip
  
  \noindent
  \textbf{Пример.} Если объект~$C$ состоит в~получении суммы~$a$ 
фирмой~$K$, то согласно теореме~2 в~пред\-шест\-ву\-ющем объекте должна 
существовать причина перевода суммы~$a$ на фирму~$K$. Если эта причина 
в~проекте отсутствует, то это можно считать признаком мошеннической схемы. 
Все проекты по предположению собираются из <<кубиков>>, содержащихся в~БЗ. 
Тогда можно сравнить цену объекта~$C$, породившего получение суммы~$a$, 
и~сумму, присутствующую в~смете проекта. Если разница велика, то это либо 
ошибка проекта, либо признак мошеннической схемы.
  
  \section{Поиск противоречий на~основе~принципа <<необходимых~условий>>}
   
  Как было показано в~разд.~3, нахождение противоречий соответствуют 
движению от следствия к~причине. Для каждого объекта в~наблюдаемых данных 
выявление причин его появления является трудоемкой задачей. Кроме того, при 
реализации контроля соблюдения при\-чин\-но-след\-ст\-вен\-ных связей на 
большом множестве участников экономической деятельности задача анализа 
причин становится трудоемкой. Поэтому процедуру контроля необходимо разбить 
на два этапа, где первый этап состоит в~анализе простых <<необходимых 
условий>> проявления мошенничества, когда используется хотя бы одна 
известная характеристика причины. Второй этап (в~режиме офлайн) состоит 
в~выявлении причин, позволяющих провести анализ источников мошеннических 
схем. 
  
  Один из подходов к~выбору <<необходимых условий>> состоит в~построении 
множества подцелей исходной цели проекта (структурный метод построения 
проекта~\cite{7-gr}). Каждая подцель описывается диаграммой на рис.~1, 
и~реализации подцелей должны образовывать полный функционал цели. Это 
является необходимым, но не достаточным условием достижения цели, так как 
при таком подходе отсутствует компонент согласования всех подцелей в~единую 
систему. Однако такой подход значительно упрощает анализ выполнения проекта 
на предмет поиска мошенничества. Если признаки мошенничества будут 
обнаружены в~реализации хотя бы одной из подцелей, то это значит, что 
мошенничество присутствует в~реализации всего проекта. 
  
  Аналогично в~реализации каждого этапа в~любой из подцелей можно выделять 
простые <<необходимые условия>> нарушения при\-чин\-но-след\-ст\-венн\-ых 
связей. 
  
  Таким образом, получается множество <<необходимых условий>>, нарушение 
которых свидетельствует о наличии мошенничества. Это множество 
<<необходимых условий>> можно назвать метаданными~[8, 9] для контроля 
проекта на выявление мошенничества. 
  
  
  \section{Заключение }
  
  В поиске противоречий необходимо от транзакций, соответствующих 
следствиям при\-чин\-но-след\-ст\-вен\-ных связей, переходить к~анализу причин 
наблюдаемых следствий. Это сложная задача, которая связана с~описанием причин 
определенных свойств. 
  
  В работе представлена модель, позволяющая строить множество необходимых 
условий соответствия наблюдаемого следствия вызвавшей его причине. Этот 
подход делает поиск противоречий вполне вычислимой задачей, но не гарантирует 
успех. 
  
  {\small\frenchspacing
 {%\baselineskip=10.8pt
 \addcontentsline{toc}{section}{References}
 \begin{thebibliography}{9}
\bibitem{1-gr}
\Au{Грушо А.\,А., Зацаринный~А.\,А., Тимонина~Е.\,Е.} Блокчейны цифровой экономики на базе 
системы ситуационных центров и~централизованного консенсуса~// Радиолокация, навигация, 
связь: Мат-лы XXV Междунар. научн.-технич. конф.~---
Воронеж: Издательский дом ВГУ, 2019. Т.~6. С.~183--191. 
\bibitem{2-gr}
\Au{Grusho A., Zatsarinny~A., Timonina~E.} A~system approach to information security in 
distributed ledgers on the situational centers platform.~---
Lecture notes in computer science ser.~--- Springer, 2019 
(in press).
\bibitem{3-gr}
\Au{Финн В.\,К.} Искусственный интеллект: Методология, применения, философия.~--- М.: 
Красанд, 2011. 448~с.

\bibitem{5-gr} %4
\Au{Аншаков~О.\,М., Фабрикантова~Е.\,Ф.} ДСМ-ме\-тод автоматического порождения 
гипотез: Логические и~эпистемологические основания.~--- М.: Либроком, 2009. 432~с.

\bibitem{4-gr} %5
\Au{Poelmans J., Elzinga~P., Viaene~S., Dedene~G.} Formal concept analysis in knowledge 
discovery: A~survey~// Conceptual structures: From information to intelligence~/ Eds.\ M.~Croitoru, 
S.~Ferr$\acute{\mbox{e}}$, and D.~Lukose.~--- Lecture notes in computer science 
ser.~--- Berlin--Heidelberg: Springer, 2010. Vol.~6208.  P.~139--153.

\bibitem{6-gr}
\Au{Панкратова~Е.\,С., Финн~В.\,К.} Автоматическое по\-рож\-де\-ние гипотез в~интеллектуальных 
системах.~--- М.: Либроком, 2009. 528~с. 
\bibitem{7-gr}
\Au{Денисов А.\,А., Колесников~Д.\,Н.} Теория больших систем управления.~--- Л.: Энергоиздат, 1982. 488~с.

\bibitem{9-gr}
\Au{Грушо А.\,А., Грушо Н.\,А., Забежайло~М.\,И., Смирнов~Д.\,В., Тимонина~Е.\,Е.} 
Параметризация в~прикладных задачах поиска эмпирических причин~// Информатика и~её 
применения, 2018. Т.~12. Вып.~3. С.~62--66.

\bibitem{8-gr}
\Au{Грушо А.\,А., Грушо Н.\,А., Левыкин~М.\,В., Тимонина~Е.\,Е.} Методы идентификации 
захвата хоста в~распределенной ин\-фор\-ма\-ци\-он\-но-вы\-чис\-ли\-тель\-ной сис\-те\-ме, 
защищенной с~помощью метаданных~// Информатика и~её применения, 2018. Т.~12. Вып.~4. 
С.~41--45.

 \end{thebibliography}

 }
 }

\end{multicols}

\vspace*{-3pt}

\hfill{\small\textit{Поступила в~редакцию 03.04.19}}

%\vspace*{8pt}

%\pagebreak

\newpage

\vspace*{-28pt}

%\hrule

%\vspace*{2pt}

%\hrule

%\vspace*{-2pt}

\def\tit{ARCHITECTURAL DECISIONS IN~THE~PROBLEM 
OF~IDENTIFICATION OF~FRAUD IN~THE~ANALYSIS 
OF~INFORMATION FLOWS IN~DIGITAL ECONOMY\\[-5pt]}


\def\titkol{Architectural decisions in~the~problem 
of~identification of~fraud in~the~analysis 
of~information flows in~digital economy}

\def\aut{A.\,A.~Grusho, M.\,I.~Zabezhailo, N.\,A.~Grusho, and~E.\,E.~Timonina}

\def\autkol{A.\,A.~Grusho, M.\,I.~Zabezhailo, N.\,A.~Grusho, and~E.\,E.~Timonina}

\titel{\tit}{\aut}{\autkol}{\titkol}

\vspace*{-13pt}


 \noindent
   Institute of Informatics Problems, Federal Research Center ``Computer Sciences and 
Control'' of the Russian Academy of Sciences; 44-2~Vavilov Str., Moscow 119133, 
Russian Federation

\def\leftfootline{\small{\textbf{\thepage}
\hfill INFORMATIKA I EE PRIMENENIYA~--- INFORMATICS AND
APPLICATIONS\ \ \ 2019\ \ \ volume~13\ \ \ issue\ 2}
}%
 \def\rightfootline{\small{INFORMATIKA I EE PRIMENENIYA~---
INFORMATICS AND APPLICATIONS\ \ \ 2019\ \ \ volume~13\ \ \ issue\ 2
\hfill \textbf{\thepage}}}

\vspace*{3pt}


   
     
   \Abste{An approach to a~research of some types of fraud in digital economy with the usage of relationships of 
cause and effect is formulated. In all types of the considered frauds, the discrepancy between the 
purposes of financial transactions and actual cost of achievement of these purposes
has to be observed. Data on 
transactions can be collected by observing information flows in which these transactions are reflected. 
The architecture of data collection and their analysis can be organized by means of the distributed 
ledgers with the centralized consensus that allows creating an analog of the electronic account book 
fixing financial and economic activity of subjects of digital economy in the region. 
   The methods of fraud identification considered are based on the contradictions 
between actions described in transactions and information, which is contained in plans, standards, 
precedents, etc. 
   The method based on a~simplified scheme of implementation of the abstract project is considered. 
For identification of contradictions, it is necessary to carry out the analysis from the effect to the cause, 
i.\,e., to look for anomalies in information describing the generation of the observed effects. 
   It is shown how in implementation of the project it is possible to allocate simple ``necessary 
conditions'' of violation of cause and effect relationships, i.\,e., a~set of ``necessary conditions'' 
violation of which demonstrates fraud existence. It is possible to call this set of "necessary conditions" 
by metadata for control of the project for fraud identification.} 
   
   \KWE{digital economy; information flows; relationships of reason and effect; detection of 
fraudulent schemes}
   
  

 \DOI{10.14357/19922264190204}

\vspace*{-20pt}

 \Ack
   \noindent
   The work was partially supported by the Russian Foundation for Basic Research (projects  
18-29-03081 and 18-07-00274).



%\vspace*{6pt}

  \begin{multicols}{2}

\renewcommand{\bibname}{\protect\rmfamily References}
%\renewcommand{\bibname}{\large\protect\rm References}

{\small\frenchspacing
 {\baselineskip=10.5pt
 \addcontentsline{toc}{section}{References}
 \begin{thebibliography}{9}
\bibitem{1-gr-1}
\Aue{Grusho, A.\,A., A.\,A.~Zatsarinny, and E.\,E.~Timonina.} 2019. Blokcheyny tsifrovoy ekonomiki 
na baze sistemy situatsionnykh tsentrov i~tsentralizovannogo konsensusa [Blockchains of digital 
economy on the basis of the system of the situational centres and the centralized consensus]. 
\textit{25th Scientific and Technical Conference (International) ``Radar-Location, Navigation, 
Communication'' Proceedings}. Voronezh: VSU Publs. 6:183--191.
\bibitem{2-gr-1}
\Aue{Grusho, A., A.~Zatsarinny, and E.~Timonina.} 2019 (in press). 
A~system approach to information security 
in distributed ledgers on the situational centers platform. 
Lecture notes in computer science ser. Springer.
\bibitem{3-gr-1}
\Aue{Finn, V.\,K.} 2011. \textit{Iskusstvennyy intellekt: Metodologiya, primeneniya, filosofiya} 
[Artificial intelligence: Methodology, applications, philosophy]. Moscow: KRASAND. 448~p.

\bibitem{5-gr-1}
\Aue{Anshakov, O.\,M., and E.\,F.~Fabrikantova}. 2009. \textit{DSM-metod avtomaticheskogo porozhdeniya gipotez: Logicheskie 
i~epistemologicheskie osnovaniya} [JSM-method of automatic hypothesis generation: Logical and 
epistemological]. Moscow: KD LIBROKOM. 432~p.
\bibitem{4-gr-1} %5
\Aue{Poelmans, J., P.~Elzinga, S.~Viaene, and G.~Dedene.} 2010. Formal concept analysis in 
knowledge discovery: A~survey. \textit{Conceptual structures: From information to intelligence}. 
Eds.\ M.~Croitoru, S.~Ferr$\acute{\mbox{e}}$, and D.~Lukose. Lecture notes in 
computer science ser. Berlin--Heidelberg: Springer. 6208:139--153.

\bibitem{6-gr-1}
\Aue{Pankratov, E.\,S., and V.\,K.~Finn}. 
2009. \textit{Avtomaticheskoe porozhdenie gipotez v~intellektual'nykh 
sistemakh} [Automatic hypotheses generation in intelligent systems]. Moscow: KD 
\mbox{LIBROKOM}.  528~p. 
\bibitem{7-gr-1}
\Aue{Denisov, A.\,A., and D.\,N.~Kolesnikov.} 1982. \textit{Teoriya bol'shikh 
sistem upravleniya} [Theory of big control systems]. Leningrad: Energoizdat. 488~p.

\bibitem{9-gr-1}
\Aue{Grusho, A.\,A., N.\,A.~Grusho, M.\,I.~Zabezhailo, D.\,V.~Smirnov, and 
E.\,E.~Timonina.} 2018. 
Parametrizatsiya v~prikladnykh zadachakh poiska empiricheskikh prichin 
[Parametrization in applied 
problems of search of the empirical reasons]. 
\textit{Informatika i~ee Primeneniya~--- 
Inform. Appl.} 12(3):62--66.

\bibitem{8-gr-1}
\Aue{Grusho, A.\,A., N.\,A.~Grusho, M.\,V.~Levykin, and E.\,E.~Timonina.} 2018. Metody 
identifikatsii zakhvata khosta v~raspredelennoy informatsionno-vychislitel'noy sisteme, 
zashchishchennoy s~pomoshch'yu metadannykh [Methods of identification of host capture 
in the  distributed information system which is protected on the base of meta data].
\textit{Informatika i~ee 
Primeneniya~--- Inform. Appl.} 12(4):41--45.
{ %\looseness=1

}

\end{thebibliography}

 }
 }

\end{multicols}

\vspace*{-12pt}

\hfill{\small\textit{Received April 3, 2019}}

%\pagebreak

%\vspace*{-18pt}

\Contr

\noindent
\textbf{Grusho Alexander A.} (b.\ 1946)~--- Doctor of Science in physics and 
mathematics, professor, principal scientist, Institute of Informatics Problems, 
Federal Research Center ``Computer Sciences and Control'' of the Russian 
Academy of Sciences; 44-2~Vavilov Str., Moscow 119133, Russian Federation; 
\mbox{grusho@yandex.ru} 

\vspace*{3pt}

\noindent
\textbf{Zabezhailo Michael I.} (b.\ 1956)~--- Doctor of Science in physics and 
mathematics, principal scientist, Institute of Informatics Problems, Federal Research 
Center ``Computer Sciences and Control'' of the Russian Academy of Sciences;  
44-2~Vavilov Str., Moscow 119133, Russian Federation; 
\mbox{m.zabezhailo@yandex.ru} 

\vspace*{3pt}


\noindent
\textbf{Grusho Nikolai A.} (b.\ 1982)~--- Candidate of Science (PhD) in physics 
and mathematics, senior scientist, Institute of Informatics Problems, Federal 
Research Center ``Computer Sciences and Control'' of the Russian Academy of 
Sciences; 44-2~Vavilov Str., Moscow 119133, Russian Federation; 
\mbox{info@itake.ru} 

\vspace*{3pt}


\noindent
\textbf{Timonina Elena E.} (b.\ 1952)~--- Doctor of Science in technology, 
professor, leading scientist, Institute of Informatics Problems, Federal Research 
Center ``Computer Sciences and Control'' of the Russian Academy of Sciences;  
44-2~Vavilov Str., Moscow 119133, Russian Federation; 
\mbox{eltimon@yandex.ru} 

\label{end\stat}

\renewcommand{\bibname}{\protect\rm Литература}  