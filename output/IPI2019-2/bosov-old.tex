\def\stat{bosov}

\def\tit{О РАЗВИТИИ КОНЦЕПЦИИ УСЛОВНО-МИНИМАКСНОЙ НЕЛИНЕЙНОЙ 
ФИЛЬТРАЦИИ: МОДИФИЦИРОВАННЫЙ ФИЛЬТР И~ЕГО АНАЛИЗ$^*$}

\def\titkol{О развитии концепции условно-минимаксной нелинейной 
фильтрации: модифицированный фильтр и~его анализ}

\def\aut{А.\,В.~Босов$^1$, Г.\,Б.~Миллер$^2$}

\def\autkol{А.\,В.~Босов, Г.\,Б.~Миллер}

\titel{\tit}{\aut}{\autkol}{\titkol}

\index{Босов А.\,В.}
\index{Миллер Г.\,Б.}
\index{Bosov A.\,V.}
\index{Miller G.\,B.}


{\renewcommand{\thefootnote}{\fnsymbol{footnote}} \footnotetext[1]
{Работа выполнена при частичной поддержке РФФИ (проект 19-07-00187-A).}}


\renewcommand{\thefootnote}{\arabic{footnote}}
\footnotetext[1]{Институт проблем информатики Федерального исследовательского центра <<Информатика и~управление>> Российской академии наук, \mbox{ABosov@frccsc.ru}}
\footnotetext[2]{Институт проблем информатики Федерального исследовательского центра <<Информатика 
и~управление>> Российской академии наук, \mbox{GMiller@frccsc.ru}}

\vspace*{-12pt}
    
         
    
             
     
     \Abst{Представлен новый субоптимальный фильтр, основанный на методе 
услов\-но-ми\-ни\-макс\-ной нелинейной фильтрации (УМНФ), используемом в~задаче фильтрации 
состояний нелинейных стохастических систем наблюдения с~дискретным временем. Идея 
модификации УМНФ состоит в~отказе от априорного расчета параметров  
услов\-но-ми\-ни\-макс\-но\-го фильтра, выполняемого в~оригинальном УМНФ методом 
имитационного моделирования, в~пользу процедуры аппроксимации параметров фильтра, 
совмещенной с~оцениванием текущего состояния. Методом Мон\-те Кар\-ло 
аппроксимируются условные моментные характеристики, определяющие параметры 
модифицированного УМНФ, в~отличие от безусловных, вычисляемых в~оригинальном 
фильтре. Дано минимаксное обоснование предлагаемого варианта аппроксимации, 
аналогичное базовой концепции УМНФ. Обсуждаются простые модельные эксперименты, 
подтверждающие работоспособность представленного алгоритма оценивания, его сравнение с~исходным фильтром.}
     
     \KW{нелинейная стохастическая система наблюдения с~дискретным временем; 
услов\-но-ми\-ни\-макс\-ная нелинейная фильтрация; метод Монте Карло}

\DOI{10.14357/19922264190202}
  
\vspace*{-4pt}


\vskip 10pt plus 9pt minus 6pt

\thispagestyle{headings}

\begin{multicols}{2}

\label{st\stat}
     
     
\section{Введение}

    Фильтрация, т.\,е.\ оценивание состояний динамических систем по 
косвенным наблюдениям в~режиме реального времени,~--- ключевая задача 
теории стохастических систем. Фундаментальные результаты в~этой 
области~[1, 2] значительно обогатили теорию, составили основу многих 
развиваемых и~в~настоящее время направлений. При этом в~утилитарном 
плане ценность уравнений оптимальной фильтрации невелика: решать их 
в~практически значимых задачах, в~условиях реального времени не удается 
даже с~учетом колоссальных возможностей современных вычислительных 
средств. 

Для практики значимыми являются адаптированные к~условиям 
проведения реальных расчетов вычислительные процедуры. Такие алгоритмы\linebreak 
обосновы\-ваются чаще всего некоторыми приближенными соотношениями. 
Соответствующий огромный пласт прикладных результатов в~целом 
определяется понятием <<субоптимальная фильтрация>>. 

Альтернативой 
эвристическому подходу, характерному для этого направления, выступает 
теория услов\-но-оп\-ти\-маль\-ной фильт\-ра\-ции В.\,С.~Пугачева~[3, 4] и~ее 
развитие в~отношении динамических систем с~дискретным временем~--- 
теория   УМНФ 
А.\,Р.~Панкова~[5, 6]. В~настоящее время эти подходы продолжают 
развиваться (см., например,~[7--10]), хотя имеют менее широкую область 
применения по сравнению с~субоптимальными фильтрами, чего не 
заслуживают хотя бы по причине своей теоретической обоснованности.
    
    В данной работе методика услов\-но-оп\-ти\-мальной фильт\-ра\-ции 
развивается для модели нелинейной динамической системы наблюдения\linebreak 
с~дискретным временем, предлагается новая модификация оригинальных 
фильтров, обоснование структуры фильтра представлено в~минимаксной форме 
УМНФ.

\vspace*{-9pt}

\section{Задачи нелинейной фильтрации. Условно-оптимальные 
и~условно-минимаксные фильтры}

\vspace*{-2pt}

    Рассматриваемая дискретная стохастическая система наблюдения 
описывается следующими рекуррентными соотношениями ($t\hm=1,2,\ldots$):

\noindent
    \begin{equation}
    \left.
    \begin{array}{l}
    x_t=\varphi_t^{(1)}(x_{t-1})+ \varphi_t^{(2)}(x_{t-1})w_t\,,\  \ 
x_0=\eta\,;\\[6pt]
    y_t=\psi_t^{(1)}(x_t)+\psi_t^{(2)}(x_t)v_t\,,
    \end{array}
    \right\}
    \label{e1-bos}
    \end{equation}
где состояние $x_t\hm\in \mathbb{R}^p$ и~наблюдения $y_t\hm\in 
\mathbb{R}^q$~--- гильбертовы случайные последовательности (наличие 
вторых моментов обеспечивается, например, линейными ограничениями на 
скорость роста функций $\varphi_t^{(1)}$, $\varphi_t^{(2)}$, $\psi_t^{(1)}$ 
и~$\psi_t^{(2)}$); возмущения $\{w_t\}$ и~ошибки наблюдения $\{v_t\}$~--- 
независимые дискретные белые шумы второго порядка, гильбертовым 
и~независимым является и~вектор начальных условий~$\eta$. 
Соответствующие моменты первого и~второго порядка обозначаются $m_w(t)$, 
$D_w(t)$, $m_v(t)$, $D_v(t)$, $m_\eta$ и~$D_\eta$.

    Модель~(1) по форме аналогична диффе\-рен\-циальным уравнениям 
диффузионного типа, что позволяет интерпретировать ее как результат 
дискретизации соответствующего стохастического дифференциального 
уравнения, а коэффициенты $\varphi_t^{(1)}$, $\varphi_t^{(2)}$, $\psi_t^{(1)}$ 
и~$\psi_t^{(2)}$~--- как аналоги коэффициентов сноса и~диффузии. 
Соответственно, линейные ограничения на скорость роста этих функций на 
бесконечности и~наличие вторых моментов у возмущений можно 
рассматривать как дискретный аналог условий Ито существования решения, 
обеспечивающих процессам~$x_t$ и~$y_t$ существование вторых моментов. 
В~этих условиях корректной является задача фильтрации, т.\,е.\ вычисления 
оценки~$\hat{x}_t$ состояния~$x_t$ по наблюдениям~$y_\tau$, 
$\tau\hm=1,\ldots , t$, характеризуемой среднеквадратичным критерием качества 
$E\left\{ \vert\vert x_t\hm- \hat{x}_t\vert\vert^2\right\}$, $E\{\cdot\}$~--- 
математическое ожидание.  Обозначив через $\mathfrak{J}_{y_1^{t}}$ 
$\sigma$-ал\-геб\-ру, порожденную наблюдениями~$y_\tau$, $\tau\hm=1,\ldots 
,t$, заметим, что предложенные условия обеспечивают существование как 
оптимального решения $\hat{x}_t\hm= E\{ x_t\vert \mathfrak{J}_{y_1^t}\}$~--- 
условного математического ожидания~$x_t$ 
относительно~$\mathfrak{J}_{y^t}$, если~$\hat{x}_t$ выбирается из всего 
класса $\mathfrak{J}_{y_1^t}$-из\-ме\-ри\-мых функций, так и~различных 
вариантов его аппроксимации, сужающих этот класс, в~частности  
услов\-но-оп\-ти\-маль\-ных оценок. Это класс оценок типа\linebreak 
 <<про\-гноз--кор\-рек\-ция>>, настраиваемых на конкретную модель~(1) 
с~помощью выбираемых структурных функций $\xi_t\hm=\xi_t(x)$ (базовый 
прогноз)\linebreak и~$\zeta_t\hm=\zeta_t(x,y)$ (базовая коррекция), которые для 
формирования оценки подвергаются линейному  
преобразованию~\cite{3-bos, 4-bos}. Именно: пусть имеется оценка 
 $\hat{x}_{t-1}$ состояния $x_{t-1}$. Услов\-но-оп\-ти\-маль\-ный 
прогноз~$\tilde{x}_t$ ищется в~виде:
    \begin{equation}
    \tilde{x}_t=F_t\xi_t+f_t,\ \xi_t=\xi_t\left( \hat{x}_{t-1}\right)\,,
    \label{e2-bos}
    \end{equation}
    а оценка $\hat{x}_t$ состояния~$x_t$ в~виде:
    \begin{equation}
    \hat{x}_t=\tilde{x}_t+H_t\zeta_t +h_t\,,\ \zeta_t=\zeta_t\left( \tilde{x}_t, 
y_t\right)\,.
    \label{e3-bos}
\end{equation}
    
    Типовым вариантом базового прогноза~$\xi_t$ выступает прогноз 
<<в~силу системы>> $\xi_t(x)\hm= \varphi_t^{(1)}(x)\hm+ \varphi_t^{(2)}(x) 
m_w(t)$, базовой коррекции~$\zeta_t$~--- невязка $\zeta_t(x,y)\hm= y\hm- 
\psi_t^{(1)}(x)\hm- \psi_t^{(2)}(x) m_v(t)$, но разнообразие подходов здесь 
ограничено только теми же условиями существования вторых моментов, что 
и~в~исходной модели. Отметим, что типовой вариант коррекции требует 
наличия вторых моментов у наблюдений~$y_t$, что и~предполагалось выше.
    
    Коэффициенты линейных преобразований в~(2) и~(3) определяются из 
условий минимума $E\left\{ \vert \vert x_t\hm-\right.$\linebreak
$\left. - F_t\xi_t\hm+ f_t\vert 
\vert^2\right\}$ для прогноза и~$E\left\{ \vert\vert x_t\hm- \tilde{x}_t\hm- 
H_t\zeta_t\hm-\right.$\linebreak
$\left. -  h_t\vert\vert^2\right\}$ для коррекции и~имеют вид:
    \begin{equation}
    \left.
    \begin{array}{rl}
    F_t &= \mathrm{cov} \left( x_t, \xi_t\right) \mathrm{cov}^+\left( \xi_t,\xi_t\right)\,;\\[6pt]
    f_t&=E\{x_t\}- F_t E\{\xi_t\}\,;\\[6pt]
    H_t &= \mathrm{cov} \left( x_t-\tilde{x}_t, \zeta_t\right) 
    \mathrm{cov}^+ \left(\zeta_t, \zeta_t\right)\,;\\[6pt]
     h_t&=- H_t E\{ \zeta_t\}\,.
    \end{array}
    \right\}
    \label{e4-bos}
    \end{equation}
       Здесь и~далее через $\mathrm{cov}\,(x,y)$ обозначена ковариационная матрица~$x$ 
и~$y$, через~$+$~--- операция псевдообращения матрицы по  
Му\-ру--Пен\-роу\-зу. Псевдообращение используется по той причине, что\linebreak 
решение задач оптимизации в~отношении~$F_t$ и/или~$H_t$ может быть не 
единственным, так что выбор будет сделан в~пользу решения с~минимальной 
нормой.
   % 
    Услов\-но-оп\-ти\-маль\-ный прогноз~$\tilde{x}_t$ и~оценка 
фильтрации~$\hat{x}_t$ при этом являются несмещенными и~обладают 
гарантированным качеством оценивания:
    \begin{equation}
    \left.
    \begin{array}{rl}
    \tilde{K}_t&= \mathrm{cov}\left( x_t-\tilde{x}_t, x_t-\tilde{x}_t\right)={}\\[6pt]
    &\hspace*{10mm}{}=\mathrm{cov} \left( x_t, x_t\right) - F_t \mathrm{cov} \left( \xi_t, x_t\right)\,;\\[6pt]
    \hat{K}_t &= \mathrm{cov} \left( x_t-\hat{x}_t, x_t-\hat{x}_t\right) ={}\\[6pt]
    &\hspace*{14mm}{}=\tilde{K}_t-H_t  \mathrm{cov}\left( \zeta_t, x_t-\tilde{x}_t\right)\,.
    \end{array}
    \right\}
    \label{e5-bos}
    \end{equation}
    
    Изложенная концепция услов\-но-оп\-ти\-маль\-ной фильтрации 
дополняется двумя существенными моментами, которые составляют суть 
концепции УМНФ: минимаксным обоснованием структуры фильтра и~методом 
расчета коэффициентов линейных преобразований. 
    
    Минимаксное обоснование обеспечивает следующая вспомогательная 
задача. Пусть для случайного вектора $z\hm= \mathrm{col}\,(x,y)$, $x\hm\in \mathbb{R}^p$, 
$y\hm\in \mathbb{R}^q$, с~известным математическим ожиданием $m_z\hm= \mathrm{col} 
\left(m_x, m_y\right)$ и~ковариацией 
$$
D_z= \displaystyle \begin{pmatrix}
    D_x & D_{xy}\\ D_{yx} & D_y
    \end{pmatrix}
    $$ 
    относительно его закона распределения $\mathcal{F}_z$ 
известно, что $\mathcal{F}_z\hm\in \Phi\left(m_z, D_z\right)$~--- классу всех вероятностных 
распределений со средним~$m_z$ и~ковариацией~$D_z$. Требуется найти 
оценку~$\hat{x}$ вектора~$x$ по на\-блю\-де\-ни\-ям~$y$. При этом допустимыми 
оценками считаются любые измеримые функции~$\theta(y)$, критерий~--- 
сред\-не\-квад\-ра\-тич\-ный при условии заданного множества 
неопределенности~$\Phi$, т.\,е.\ 
$\mathcal{J}\left(\theta, \mathcal{F}_z\right)\hm= E\left\{ \vert\vert 
x\hm- \theta(y)\vert\vert^2\right\}$. Решение такой задачи определяет седловая 
точка $(\theta^*, \mathcal{F}_z^*)$~[5, 6], а~именно:  
распределение~$\mathcal{F}_z^*$~--- гауссовское с~параметрами~$m_z$ и~$D_z$ 
и~оценка $\hat{x}\hm= \theta^*(y)\hm= D_{xy} D_y^+ y\hm+ (m_x\hm-
D_{xy}D_y^+ m_y)$~--- наилучшая линейная оценка~$x$ по~$y$. При этом 
$\mathcal{J}\left(\theta^*, \mathcal{F}_z^*\right)\hm= D_x\hm- D_{xy} D_y^+ D_{yx}$. Это 
решение и~составляет минимаксное обоснование УМНФ, т.\,е.\ линейные 
преобразования базовых прогнозов и~коррекции в~(2) и~(3). Именно: оценку 
фильтрации~$\hat{x}_t$ состояния~$x_t$ в~соответствии с~концепцией УМНФ 
следует определять как результат решения следующих минимаксных задач:
    \begin{equation}
    \left.
    \begin{array}{l}
    \hspace*{-3mm}\tilde{x}_t = \tilde{\theta}_t(\xi_t)\,,\\[6pt] 
\tilde{\theta}_t=\argmin\limits_{\tilde{\theta}_t} \max\limits_{\mathcal{F}_z} E\left\{ 
\vert \vert x_t-\tilde{\theta}_t(\xi_t)\vert\vert^2\right\}\,,\\[6pt]
 \hspace*{43mm}z=\mathrm{col}\left(x_t, \xi_t\right)\,;\\[6pt]
   \hspace*{-3mm} \hat{x}_t= \tilde{x}_t+\hat{\theta}_t(\zeta_t)\,,\\[6pt] 
\hat{\theta}_t=\argmin\limits_{\hat{\theta}_t} \max\limits_{\mathcal{F}_z} E \left\{
    \vert \vert x_t-\tilde{x}_t -\hat{\theta}_t(\zeta_t)\vert\vert^2\right\}\,,\\[6pt]  
 \hspace*{35mm}z=\mathrm{col}\left(x_t-\tilde{x}_t, \zeta_t\right)\,,
    \end{array}
    \right\}
    \label{e6-bos}
    \end{equation}
и эти решения задаются соотношениями~(2)--(4) и~обладают свойствами~(5). 
Кроме того, наихудшими в~обеих задачах распределениями оказываются 
гауссовские, параметры которых определяются параметрами вектора $\mathrm{col}\left(x_t, 
\xi_t\right)$ в~части прогнозирования и~вектора $\mathrm{col}\left(x_t\hm- \tilde{x}_t, \zeta_t\right)$ 
в~части коррекции.
{\looseness=1

}

    Вторым положением, составляющим непременный атрибут УМНФ, 
является метод расчета~$F_t$, $f_t$, $H_t$ и~$h_t$ путем имитационного 
моделирования. Услов\-но-ми\-ни\-макс\-ная нелинейная фильт\-ра\-ция 
получается заменой в~соотношениях~(4) 
математических ожиданий и~ковариаций их оценками по методу  
Мон\-те Кар\-ло, т.\,е.\ в~практической реализации УМНФ операции 
$E\{\cdot\}$ и~$\mathrm{cov}\,(\cdot,\cdot)$ заменяются соответствующими статистическими 
оценками, рассчитанными априорно по смоделированному пучку траекторий. 
Обозначим соответствующие операторы усреднения через 
$\overline{E}\{\cdot\}$ и~$\overline{\mathrm{cov}}\,(\cdot,\cdot)$. При реализации УМНФ 
в~соотношениях~(4), например, вместо $E\{x_t\}$ используется 
$$
\overline{E}\left\{x_t\right\}= \fr{1}{N} \sum\limits^N_{i=1} x_t^i\,,
$$ 
вместо $\mathrm{cov} \left(\zeta_t, x_t\hm- \tilde{x}_t\right)$~---
$$
\overline{\mathrm{cov}}\left(\zeta_t, x_t- 
\tilde{x}_t\right)= \fr{1}{N-1}\sum\limits^N_{i=1} \zeta_t^i\left(x_t^i- 
\tilde{x}_t^i\right)^{\mathrm{T}}\,,
$$
где $N$~--- число смоделированных траекторий, а $x_t^i$, 
$\tilde{x}_t^i$ и~$\zeta_t^i$, $i\hm=1,\ldots , N$,~--- реализации.

\section{Основной результат~--- модифицированный  
условно-минимаксный нелинейный фильтр}
 
    Концепция УМНФ позволяет получать реализуемые и~высокоточные 
решения широкого класса задач нелинейной фильтрации. Основываясь на 
опыте применения, можно отметить, что для алгоритма УМНФ не критичны 
такие известные практические проблемы, как рост размерности, неточность 
в~задании начальных условий, вырожденность возмущений. Вообще  
услов\-но-оп\-ти\-маль\-ный подход представляется уникальным направлением 
с~колоссальным практическим потенциалом, не имеющим в~настоящее время 
ни аналогов, ни конкурентов. Вместе с~тем возможности для его дальнейшего 
совершенствования имеются, и~одна из них представлена далее. Источником 
для предлагаемой модификации служит, как принято называть, калмановская 
структура УМНФ. Объяснение такому термину в~задаче нелинейной 
фильтрации, не имеющей общих черт с~ли\-ней\-но-гаус\-сов\-ской постановкой 
классического фильтра Калмана, можно представить, записав самый простой из 
субоптимальных фильтров, предназначенных для решения задач нелинейной 
фильтрации,~--- расширенный фильтр Калмана (РФК). Соотношения для РФК 
применительно к~оцениванию в~модели~(1) таковы:
    \begin{equation}
    \left.
    \begin{array}{l}
   \hspace*{-2mm} \tilde{x}_t=\xi_t,\  \xi_t=\xi_t\left( \hat{x}_{t-1}\right),\\[6pt]
     \hspace*{16mm}\xi_t(x)= \varphi_t^{(1)}(x)+\varphi_t^{(2)}(x) m_w(t)\,;\\[6pt]
   \hspace*{-2mm} \tilde{K}_t=f_t\hat{K}_{t-1} f_t^{\mathrm{T}}+\varphi_t^{(2)} D_w (t) 
{\varphi_t^{(2)}}^{\mathrm{T}}\,,\\[6pt]
\hspace*{5mm} f_t= \fr{\partial \varphi_t^{(1)}(x)} {\partial x}\Big\vert x= 
\tilde{x}_t\,,\ \varphi_t^{(2)}=\varphi_t^{(2)}\left( \tilde{x}_t\right)\,;\\[6pt]
    \hspace*{-2mm}\hat{x}_t=\tilde{x}_t+H_t\zeta_t\,,\\[6pt]
     \hspace*{2mm}\zeta_t= y_t-\psi_t^{(1)}\left( 
\tilde{x}_t\right)-\psi_t^{(2)}\left(\tilde{x}_t\right) m_v(t)\,,\\[6pt]
   \hspace*{2mm} H_t=\tilde{K}_t h_t^{\mathrm{T}} \left( h_t \tilde{K}_t h_t^T +
   \psi_t^{(2)} D_v(t) 
{\psi_t^{(2)}}^T\right)^+\!,\\[6pt] 
   \hspace*{6mm} h_t=\fr{\partial \psi_t^{(1)}(x)}{\partial x}\Big\vert x= \tilde{x}_t\,,\ 
\psi_t^{(2)}=\psi_t^{(2)}\left( \tilde{x}_t\right)\,,\\[6pt]
    \hspace*{-2mm}\hat{K}_t=\tilde{K}_t-H_t h_t \tilde{K}_t\,.
    \end{array}
    \right\}
    \label{e7-bos}
    \end{equation}
    
    Обратим внимание, что в~(\ref{e7-bos}) использованы те же обозначения 
для прогноза~$\tilde{x}_t$ и~оценки~$\hat{x}_t$, типового прогноза <<в~силу 
системы>>~$\xi_t$ и~невязки~$\zeta_t$, повторно использованы обозначения 
$f_t$, $h_t$, $H_t$, $\tilde{K}_t$ и~$\hat{K}_t$, поскольку в~структуре 
фильтров, в~выражениях для характеристик точности оценок эти коэффициенты 
играют идентичные роли. Но в~выражениях для УМНФ фигурируют точные 
значения соответствующих моментов, заменяемых при реализации 
результатами имитационного моделирования, в~выражениях для РФК 
используются аппроксимации, полученные путем линеаризации около 
прогноза~$\tilde{x}_t$. И~это последнее обстоятельство заслуживает 
определенного внимания.
    
    Действительно, коэффициенты~$f_t$, $h_t$ и~$H_t$ в~РФК рассчитываются 
вместе с~траекторией конкретной оценки по мере поступления наблюдений 
в~отличие от аналогичных коэффициентов УМНФ, вы\-чис\-ля\-емых априорно. 
Значит, можно предполагать, что в~этих коэффициентах используются оценки, 
аппроксимирующие не априорные, а условные моменты и~даже потраекторный 
критерий оценивания $E\left\{ \left\vert \left\vert x_t \hm-
\hat{x}_t\right\vert\right\vert^2\Big\vert \mathfrak{J}_y^t\right\}$. Конечно, РФК 
не минимизирует этот критерий, но пытается аппроксимировать некоторую 
оценку, имеющую смысл в~отношении данного критерия. Другое дело, что для 
такой аппроксимации используется не слишком работоспособный вариант 
линеаризации, но в~идее использования в~фильтре оценок условных моментов, 
вычисляемых вместе с~оценками состояния на каждом шаге фильтрации, 
и~видится вариант усовершенствования УМНФ.
    
    Для формального описания алгоритма модифицированной УМНФ 
(МУМНФ) потребуется манипулирование условными моментами. Так, для 
случайного вектора~$y$   будем обозначать через~$\mathfrak{J}_y$ порожденную 
им $\sigma$-ал\-геб\-ру, через $E\left\{ x\vert \mathfrak{J}_y\right\}$~--- 
условное математическое ожидание~$x$ относительно~$\mathfrak{J}_y$, через 
$\mathrm{cov}\left \{ x, x\vert \mathfrak{J}_y\right\}\hm=
E\left\{ 
\vphantom{\left(\left\{ x\vert \mathfrak{J}\right\}\right)^{\mathrm{T}}}
\left( x\hm-\right.\right.$\linebreak
$\left.\left.-E\left\{ x\vert \mathfrak{J}_y\right\}\right)^{\mathrm{T}}\left(
x\hm- E\left\{ x\vert \mathfrak{J}_y\right\} \right)\vert \mathfrak{J}_y\right\}$~--- условную ковариацию. Говоря далее об 
условных распределениях, будем под этим понимать сужение вероятностной 
меры, заданной на исходном вероятностном пространстве, на подходящую 
$\sigma$-ал\-геб\-ру~$\mathfrak{J}_y$.
    
    Пусть теперь $(\hat{x}_{t-1}, \hat{K}_{t-1})$~--- оценка МУМНФ, 
вычисленная в~момент $t\hm-1$. Пусть, кроме того, существует  
$\sigma$-ал\-геб\-ра~$\tilde{\mathfrak{J}}_{y_1^{t-1}}$, являющаяся  
$\sigma$-под\-ал\-геб\-рой~$\mathfrak{J}_{y_1^{t-1}}$, такая что условное распределение~$x_{t-1}$ 
относительно~$\tilde{\mathfrak{J}}_{y_1^{t-1}}$ обладает следующими 
свойствами: 

\vspace*{-9pt}

\noindent
\begin{align*}
E\{ x_{t-1}\vert \tilde{\mathfrak{J}}_{y_1^{t-1}} \}& = 
\hat{x}_{t-1}\,;\\
\mathrm{cov}\left (x_{t-1} - \hat{x}_{t-1}, x_{t-1}- \hat{x}_{t-1} \vert 
\tilde{\mathfrak{J}}_{y_1^{t-1}}\right)&= \hat{K}_{t-1}\,.
\end{align*}

\noindent
 Отметим, что форма 
распределения при этом не важна, а обосновывать выбор оценивателя, как 
в~оригинальном УМНФ, будет некоторая минимаксная задача.
    
    Прогноз $\tilde{x}_t$ будем искать в~виде:
    \begin{equation}
    \tilde{x}_t=E\left\{ x_t\vert \tilde{\mathfrak{J}}_{y_1^{t-1}} \right\}\,.
    \label{e8-bos}
    \end{equation}
        Точность этого прогноза очевидным образом характеризуется выражением:
    \begin{multline}
    \tilde{K}_t=\mathrm{cov} \left( x_t-\tilde{x}_t, x_t-\tilde{x}_t\vert 
\tilde{\mathfrak{J}}_{y_1^{t-1}}\right) ={}\\
{}=\mathrm{cov} \left(  x_t, x_t\vert  \tilde{\mathfrak{J}}_{y_1^{t-1}}\right)\,.
    \label{e9-bos}
    \end{multline}
    
    Далее будем предполагать, что существует $\sigma$-ал\-геб\-ра 
$\hat{\mathfrak{J}}_{y_1^{t-1}}$, являющаяся 
$\sigma$-под\-ал\-геб\-рой~$\mathfrak{J}_{y_1^{t-1}}$, такая, что условное 
распределение~$x_t$ относительно $\hat{\mathfrak{J}}_{y_1^{t-1}}$ обладает 
следующими свойствами: 
\begin{align*}
E\left\{x_t- \tilde{x}_t\vert \hat{\mathfrak{J}}_{y_1^{t-1}}\right\}&=0\,;\\
\mathrm{cov}\left (x_t- \tilde{x}_t, x_t- 
\tilde{x}_t\vert \hat{\mathfrak{J}}_{y_1^{t-1}}\right)&=\tilde{K}_t,
\end{align*} 
где~$\tilde{x}_t$ определено в~(\ref{e8-bos}), а~$\tilde{K}_t$~--- в~(\ref{e9-bos}). Для 
заданной корректирующей структурной функции $\zeta_t\hm= \zeta_t(x,y)$ 
оценку~$\hat{x}_t$ будем искать в~виде:
    \begin{equation}
    \hat{x}_t=\tilde{x}_t+H_t\zeta_t + h_t\,,
        \label{e10-bos}
    \end{equation}
    где
\begin{align*}
\zeta_t&=\zeta_t \left( \tilde{x}_t,  y_t\right)\,;\\
    H_t&= \mathrm{cov}\left( x_t-\tilde{x}_t,\zeta_t\vert \hat{\mathfrak{J}}_{y_1^{t-
1}}\right) \mathrm{cov}^+ \left( \zeta_t, \zeta_t\vert\hat{\mathfrak{J}}_{y_1^{t-1}}
\right)\,;\\ 
h_t&=-H_tE\left\{ \zeta_t\vert \hat{\mathfrak{J}}_{y_1^{t-1}}\right\}\,.
       \end{align*}
    
    Точность оценки МУМНФ определяется выражением
    \begin{multline}
    \hat{K}_t=\mathrm{cov}\left( x_t-\hat{x}_t, x_t-\hat{x}_t\vert 
\hat{\mathfrak{J}}_{y_1^{t-1}}\right) ={}\\
{}=\tilde{K}_t-H_t \mathrm{cov} \left( \zeta_t, x_t-
\tilde{x}_t\vert \hat{\mathfrak{J}}_{y_1^{t-1}}\right).
    \label{e11-bos}
    \end{multline}
    
    Приведенные соотношения, по сути, в~точ\-ности повторяют выражения для 
оценки УМНФ в~момент $t\hm=1$ с~единственным изменением~--- вместо 
безусловных моментов фигурируют услов\-ные относительно некоторых 
$\sigma$-ал\-гебр, огра\-ни\-чивающих объем доступной для оценивания\linebreak 
информации. Обосно\-ва\-ние всей процедуры оценивания дается следующими 
минимаксными задачами, вполне аналогичными задачам~(\ref{e6-bos}):

\noindent
    \begin{equation}
    \left.
    \begin{array}{l}
    \hspace*{-3mm}\tilde{x}_t= \tilde{\theta}_t\,,\\[6pt]
        \hspace*{-3mm}\tilde{\theta}_t=\argmin\limits_{\tilde{\theta}_t} 
\max\limits_{\mathcal{F}_z} E \left\{ \vert \vert x_t-\tilde{\theta}_t\vert\vert^2\Big\vert 
\hat{\mathfrak{J}}_{y_1^{t-1}}\right\},\\[6pt]
     \hspace*{57mm}z=x_t\,;\\[6pt]
        \hspace*{-3mm}\hat{x}_t=\tilde{x}_t+\hat{\theta}_t(\zeta_t),\\[6pt] 
    \hspace*{-3mm}\hat{\theta}_t=\argmin\limits_{\hat{\theta}_t} \max\limits_{\mathcal{F}_z} E \!\left\{\! \vert 
\vert x_t-\tilde{x}_t-\hat{\theta}_t(\zeta_t)\vert \vert^2\Big\vert 
\hat{\mathfrak{J}}_{y_1^{t-1}} \!\right\}\!,\\[6pt]
     \hspace*{37mm}z=\mathrm{col} \left(x_t-\tilde{x}_t,  \zeta_t\right).
    \end{array}\!\!
    \right\}\!\!\!\!
    \label{e12-bos}
    \end{equation}
    
    Заметим, что решение~(\ref{e12-bos}) дает та же вспомогательная задача 
из разд.~2, поскольку отличие постановок, состоящее в~замене безусловных 
моментов условными, никак не влияет на решение. Это верно и~для первой 
задачи, вырожденной в~том смысле, что у~$\tilde{\theta}_t$ нет аргумента, 
и~составляющей, таким образом, просто свойство условного математического 
ожидания. А~главное, что этот же результат обосновывает переход от оценки 
в~момент $t\hm-1$ к~прогнозу и~далее к~коррекции на основе гауссовского 
распределения, являющегося наихудшим, т.\,е.\ минимаксно 
обоснованным~(\ref{e12-bos}).
    
    Принципиальное же отличие МУМНФ состоит в~отсутствии точной 
априорной информации о~его точ\-ности. Вместо этого приходится использовать 
некие потраекторные характеристики~$\tilde{K}_t$ и~$\hat{K}_t$,\linebreak которые, 
будучи привязанными к~условному распределению, не дают однозначного 
понимания качества оценки фильтрации в~смысле исходного 
сред\-не\-квад\-ра\-тич\-но\-го критерия.
    
    В связи с~предложенной модификацией следует коснуться двух вопросов. 

Во-первых, это технически важный вопрос\linebreak о~$\sigma$-ал\-геб\-рах 
$\tilde{\mathfrak{J}}_{y_1^{t-1}}$ и~$\hat{\mathfrak{J}}_{y_1^{t-1}}$, т.\,е.\ 
о~том, можно ли на очередном шаге оценивания ограничить объем доступной 
информации так, чтобы выполнялись указан\-ные свойства для условных 
моментов. В~принципе, в~области субоптимальной фильтрации редко 
рассматриваются вопросы такого рода. Действительно, использовать только тот 
объем информации, который приведет к~эффективной и~качественной 
вычислительной процедуре, описав его некоторыми вербальными правилами, 
вполне допустимо, если это приводит к~практическому результату. Но для 
МУМНФ важно еще и~обоснование, формальное и~основанное на конкретном 
математическом аппарате. Записывая условные моменты, нужно быть 
уверенным, что это можно сделать, что манипуляции выполняются со 
случайными векторами, имеющими конечные вторые моменты, что  
$\sigma$-ал\-геб\-ры, порождающие условные распределения, существуют. Для 
этого нужно быть уверенным, что для любого вектора $z\hm= \mathrm{col}\, (x,y)$ 
с~$\mathcal{F}_z\hm\in \Phi(m_z, D_z)$ можно так преобразовать~$y$, что 
условное распределение~$x$ относительно преобразованной $\mathfrak{J}_y$ 
будет обеспечивать заданные значения условным моментам~--- 
математическому ожиданию и~ковариации. Легко привести примеры, когда 
у~такой задачи есть решения и~когда их нет, т.\,е.\ все зависит от конкретной 
системы~(1). Поэтому здесь ограничимся формальным предположением, что 
система и~выбранные для МУМНФ структурные функции~$\xi_t$ и~$\zeta_t$ 
обеспечивают существование таких $\sigma$-ал\-гебр.
    
    Второй вопрос~--- о практической реализации 
МУМНФ. Это существенный вопрос, так как вычислять аналитически моменты, 
фигурирующие в~этих соотношениях, априорно нель\-зя. Равно как 
и~рассчитывать их в~процессе фильт\-ра\-ции также не удастся. Здесь проблема 
в~огромной вычислительной трудоемкости, неминуемо со\-про\-вож\-да\-ющей 
любые попытки манипулирования условными вероятностями в~динамических 
системах. Поэтому проводить вычисления нужных параметров придется 
приближенно. Следуя концепции УМНФ, используем для этого тот же метод 
Мон\-те Кар\-ло, но уже не априорно, а~в~процессе расчетов для каждой 
траектории сис\-те\-мы-оценки.
    
    Основания для реализации имитационного моделирования дает решение 
задач~(\ref{e12-bos}), отвечая на вопрос, какими в~наихудшем случае будут 
условные распределения относительно $\sigma$-ал\-гебр 
$\tilde{\mathfrak{J}}_{y_1^{t-1}}$ и~$\hat{\mathfrak{J}}_{y_1^{t-1}}$. 
Действительно, поскольку во второй задаче наихудшее распределение $z\hm= 
\mathrm{col}\left (x_t\hm- \tilde{x}_t, \zeta_t\right)$ гауссовское, а $\hat{\theta}_t(\zeta_t)$~--- 
линейная, значит, и~наихудшее распределение $\hat{x}_t$~--- гауссовское. 
Такова же ситуация и~на предыдущем шаге, поэтому в~(\ref{e8-bos}) 
и~(\ref{e9-bos}) наихудшее распределение~$x_{t-1}$ из возможных условных 
распределений относительно $\tilde{\mathfrak{J}}_{y_1^{t-1}}$ является 
гауссовским. Аналогично в~первой задаче в~(\ref{e12-bos}) относительно 
$z\hm= \mathrm{col}\left (x_t, \xi_t\right)$ наихудшим распределением оказывается гауссовское, 
$\tilde{\theta}_t\left(\xi_t\right)$~--- линейна, поэтому в~(\ref{e10-bos}) и~(\ref{e11-bos}) 
наихудшее распределение $x_t\hm-\tilde{x}_t$ из возможных условных 
распределений относительно $\hat{\mathfrak{J}}_{y_1^{t-1}}$ будет 
гауссовским. Эти соображения дают основание для реализации расчетов по 
формулам~(\ref{e8-bos})--(\ref{e11-bos}) методом Мон\-те Карло.
    
    Пусть в~момент~$t$ имеются $\hat{x}_{t-1}$~--- оценка МУМНФ состояния 
$x_{t-1}$ по наблюдениям~$y_\tau$, $\tau\hm=1,\ldots , t\hm-1$,  
и~$\hat{K}_{t-1}$~--- оценка условной ковариации ошибки. На шаге~$t$ 
выполняется моделирование выборок: $\{x^i_{t-1}\}^N_{i=1}$~--- из 
гауссовского\linebreak распределения со средним $\hat{x}_{t-1}$ и~ковариацией 
$\hat{K}_{t-1}$ и~$\{ w^i_t, v_t^i\}^N_{i=1}$~--- дискретных белых шумов 
в~соответствии с~моделью~(1)~--- и~рассчитываются наборы $\{x_t^i, 
y_t^i\}^N_{i=1}$ по формулам~(1). Прогноз МУМНФ задается в~виде 
$\tilde{x}_t\hm=\overline{E}\{x_t\}$, его точность $\tilde{K}_t\hm= 
\overline{\mathrm{cov}}\left(x_t\hm- \tilde{x}_t, x_t\hm- \tilde{x}_t\right)$. Отметим, что здесь не 
нужна функция базового прогноза~$\xi_t$, коэффициент $F_t\hm=0$, 
поскольку не моделируется выборка оценок~$\hat{x}_{t-1}$, и~расчет 
выполняется так же, как и~на первом шаге~УМНФ.
{ %\looseness=1

}
    
    На этапе коррекции формируются наборы $\{ x_t^i \hm- \tilde{x}_t, 
\zeta_t^i\}^N_{i=1}$ и~вычисляются коэффициенты 
    \begin{align*}
    H_t&= \overline{\mathrm{cov}} \left(x_t\hm- \tilde{x}_t, \zeta_t\right)
    \overline{\mathrm{cov}}^+  \left(\zeta_t, \zeta_t\right);\\ 
    h_t&= -H_t \overline{E}\{\zeta_t\},\quad 
    \zeta_t^i=  \zeta_t(\tilde{x}_t, y_t^i)
   \end{align*}
    и~оценка точ\-ности 
    $$
    \hat{K}_t= \tilde{K}_t- H_t 
\overline{\mathrm{cov}} \left(\zeta_t, x_t\hm- \tilde{x}_t\right).
$$

 Окончательная оценка фильт\-ра\-ции, 
как и~в~УМНФ, ищется в~виде:
$$
\hat{x}_t= \tilde{x}_t+ H_t 
\zeta_t+h_t,\enskip \zeta_t= \zeta_t\left(\tilde{x}_t, y_t\right). 
$$
    
    Важно отметить, что размеры моделируемых выбо\-рок в~УМНФ и~МУМНФ 
должны принципиально отличаться, поскольку расчет коэффициентов УМНФ 
выполняется априорно и,~таким\linebreak образом, ничем не ограничен, а~расчет 
коэффициентов МУМНФ выполняется в~процессе фильт\-ра\-ции, т.\,е.\ 
предполагает реализуемость в~режиме реального времени. По этой причине 
в~экспериментах для УМНФ моделировались пучки объемом в~10$^6$ 
траекторий, для МУМНФ $N$ полагается далее равным~25, 100 и~1000.
    
\section{Анализ качества модифицированной условно-минимаксной нелинейной
фильтрации}

    В рамках данного исследования авторами был проведен значительный 
объем вычислительных экспериментов с~целью сравнения качества оценок 
УМНФ и~МУМНФ. В~целом концепция МУМНФ продемонстрировала свою 
работоспособность, но говорить о регулярном превосходстве МУМНФ нель\-зя. 
В~подавляющем большинстве рассмотренных модельных примеров точности 
оценок обоих фильтров близки, совпадая до процентных пунктов. 
Незначительная часть примеров давала преимущество МУМНФ в~несколько 
процентов, аналогично наблюдались и~другие примеры с~небольшим 
преимуществом УМНФ. При этом практически всегда удавалось нивелировать 
это преимущество за счет роста параметра~$N$, числа моделируемых 
траекторий МУМНФ, что на практике, конечно, неприемлемо. Найти пример 
ощутимого превосходства МУМНФ удалось только в~задаче идентификации\linebreak\vspace*{-12pt}

\columnbreak

\noindent 
параметров динамической системы. Иллюстрация сказанному~--- следующие 
два модельных при-\linebreak мера.

    
    Первый эксперимент проведен для модели наблюдения следующего вида:
    \begin{equation}
    \left.
    \begin{array}{l}
    x_t=\fr{x_{t-1}}{1+x^2_{t-1}}+w_t\,,\ t=1,\ldots, T\,,\ T=50\,,\\[6pt] 
\hspace*{54mm}x_0=\eta\,;\\[6pt]
    y_t=x_t+x_t^3+v_t\,.
    \end{array}
    \right\}
    \label{e13-bos}
    \end{equation}
    
    В этой модели две характерных черты. Во-пер\-вых, нелинейная регрессия 
    в~уравнении состояния, функция которой $x/(1\hm+x^2)$ хотя и~существенно 
нелинейная, но обладает довольно инертным характером, дифференцируема 
и~деградирует на бесконечности. Во-вто\-рых, это наблюдения в~форме 
кубического сенсора, также нелинейные, но весьма информативные и~удобные 
для нелинейной фильт\-ра\-ции. Шумы $w_t$ и~$v_t$ в~(\ref{e13-bos}) стандартные 
гауссовские, начальное условие~$\eta$ также гауссовское с~$m_\eta\hm= 
0{,}1$ и~$D_\eta\hm=1$. Структура обоих фильт\-ров типовая: прогноз строится 
<<в~силу сис\-те\-мы>>, коррекция~--- в~форме невязки. Параметры УМНФ 
вы\-чис\-ля\-лись априорно методом Мон\-те Кар\-ло по пучку из~10$^5$ 
траекторий, качество оценок анализировалось по второму пучку из~10$^6$ 
траекторий. Модифицированный услов\-но-ми\-ни\-макс\-ный нелинейный
фильтр рассчитывался в~трех вариантах: для~$N$, равного~25, 
100 и~1000 (такие же па\-ра\-мет\-ры использованы и~в~следующем примере). Таким 
образом, получены четыре оценки, качество которых проиллюстрировано на 
рис.~1.
    
    
     
    
    Как видно, МУМНФ начинает демонстрировать приемлемое качество 
оценивания уже при $N\hm=100$,  а~при $N\hm=1000$ дает преимущество 
порядка~2\%--3\%.
{\looseness=1

}
    
    Вторая задача~--- идентификация параметра~$\theta$ динамической 
системы по косвенным линейным наблюдениям~--- представлена следующей 
мо\-делью:
{\looseness=1

}
    \begin{equation}
    \left.
    \begin{array}{l}
    \hspace*{-3mm}x_t=\theta x_{t-1}+b+cw_t\,,\enskip t=1,\ldots , T\,,\ T=50\,,\\[6pt]
    \hspace*{52mm}x_0=\eta\,;\\[6pt]
        \hspace*{-3mm}y_t=x_t+0{,}1v_t\,.
    \end{array}\!\!
    \right\}\!\!
    \label{e14-bos}
    \end{equation}
    
    Неизвестный коэффициент~$\theta$ авторегрессии в~уравнении динамики 
предполагается распределенным равномерно на отрезке $[-0{,}9; 0{,}9]$ 
и~подлежит идентификации на каждой траектории. С~этой целью 
использованы те же фильтры, что и~в~предыду\-щем примере, примененные 
к~расширенному вектору состояния $\mathrm{col}\left (x_t,\theta\right)$. Результат целевой задачи 
идентификации иллюстрирует рис.~2.
{\looseness=1

}
    
    \begin{figure*} %fig1
     \vspace*{1pt}
     \begin{minipage}[t]{80mm}
 \begin{center}  
  \mbox{%
 \epsfxsize=77.1mm 
 \epsfbox{bos-1.eps}
 }
\end{center}
\vspace*{-11pt}
     \Caption{Показатели качества оценок в~модели~(\ref{e13-bos}):
      \textit{1}~--- $D\left[ x_T-\hat{x}_T\right]\hm= 
0{,}52$, МУМНФ ($N\hm=25$); 
      \textit{2}~--- $D\left[ x_T-\hat{x}_T\right]\hm= 
0{,}39$, МУМНФ ($N\hm=100$);
      \textit{3}~---  $D\left[ x_T-\hat{x}_T\right]\hm= 0{,}35$, УМНФ;
      \textit{4}~---  $D\left[ x_T-\hat{x}_T\right]\hm= 
0{,}34$, МУМНФ ($N\hm=1000$)}
\end{minipage}
%      \end{figure*}
\hfill
%    \begin{figure*} %fig2
    \vspace*{1pt}
         \begin{minipage}[t]{80mm}
 \begin{center}  
  \mbox{%
 \epsfxsize=79mm 
 \epsfbox{bos-2.eps}
 }
\end{center}
\vspace*{-11pt}
\Caption{Показатели качества идентификации в~модели~(\ref{e14-bos}):
      \textit{1}~---  $D\left[ x_T-\hat{x}_T\right]= 0{,}21$, УМНФ;\protect\linebreak 
      \textit{2}~--- $D\left[ x_T-\hat{x}_T\right]\hm= 
0{,}07$, МУМНФ ($N\hm=25$);
      \textit{3}~--- $D\left[ x_T-\hat{x}_T\right]\hm= 
0{,}03$, МУМНФ ($N\hm=100$);
      \textit{4}~--- $D\left[ x_T-\hat{x}_T\right]\hm= 
0{,}02$, МУМНФ ($N\hm=1000$)}
\end{minipage}
      \end{figure*}
    
    В данном примере МУМНФ имеет существенное преимущество уже в~том, 
что обеспечивает сходимость к~значению~$\theta$ на каждой траектории, в~то 
время как УМНФ ограничивается оценкой с~фиксированной, но 
неуменьшающейся дис\-пер\-си\-ей, лишь незначительно улучшающей качество 
$D_\theta\hm= E\{\theta^2\}\hm=0{,}27$ тривиальной оценки 
$m_\theta\hm=E\{\theta\}\hm=0$.

\vspace*{-4pt}

\section{Выводы }

    Главным результатом работы представляется в~очередной раз 
продемонстрированная высокая эффективность концепций 
услов\-но-оп\-ти\-маль\-ной и~услов\-но-ми\-ни\-макс\-ной фильтрации, подтвержденная новой 
модификацией классического фильтра. 

Полученный алгоритм оценивания 
показал не только свою работоспособность, но и~однозначное преимущество 
в~том случае, когда априорные расчеты параметров фильтра по  
ка\-кой-ли\-бо причине невозможны, например при отсутствии достоверных 
предположений о~требуемом горизонте фильт\-ра\-ции или невозможности 
хранения большого объема априорно вычисленных па\-ра\-мет\-ров. 

Кроме того, 
МУМНФ продемонстрировал опреде\-ленное преимущество в~задаче 
идентификации па\-ра\-мет\-ров стохастической динамической сис\-те\-мы. Если 
к~условиям фильтрации конкретная за\-да\-ча специальных требований 
(бесконечный горизонт, невозможность априорных расчетов, необходимость 
идентификации части параметров) не предъявляет, то метод УМНФ 
представляется лучшим вариантом практической нелинейной фильт\-рации.
    
    {\small\frenchspacing
 {%\baselineskip=10.8pt
 \addcontentsline{toc}{section}{References}
 \begin{thebibliography}{99}

\bibitem{2-bos}
\Au{Липцер Р.\,Ш., Ширяев~А.\,Н.} Статистика случайных процессов: Нелинейная 
фильтрация и~смежные вопросы.~--- М.: Наука, 1974. 696~с.
\bibitem{1-bos}
\Au{Kallianpur G.} Stochastic filtering theory.~--- New York, NY, USA: Springer-Verlag, 1980. 
318~p.
\bibitem{3-bos}
\Au{Пугачев~В.\,С.} Рекуррентное оценивание переменных и~параметров в~стохастических 
системах, опи\-сы\-ва\-емых разностными уравнениями~// Докл. Акад. наук СССР, 1978. Т.~243. №\,5.  
С.~1131--1133.
\bibitem{4-bos}
\Au{Пугачев В.\,С.} Оценивание переменных и~параметров в~дискретных нелинейных 
системах~// Автоматика и~телемеханика, 1979. №\,6. С.~63--79.
% (Pugachev V. S. Estimation of variables and parameters in discrete-time nonlinear systems // 
%Automation and Remote Control, 1979. Vol.~40. No.\,4. P.~39--50).
\bibitem{5-bos}
\Au{Панков А.\,Р.} Рекуррентная услов\-но-ми\-ни\-макс\-ная фильтрация процессов 
в~разностных нелинейных стохастических системах~// Известия РАН. Теория и~системы 
управления, 1992. №\,3. С.~63--70.
\bibitem{6-bos}
\Au{Pankov~A.\,R., Bosov~A.\,V.} Conditionally minimax algorithm for nonlinear system state 
estimation~// IEEE T.~Automat. Contr., 1994. Vol.~39. No.\,8. P.~1617--1620.

\bibitem{8-bos} %7
\Au{Синицын И.\,Н., Корепанов~Э.\,Р.} Устойчивые линейные условно-оптимальные 
фильтры и~экстраполяторы для стохастических систем с~мультипликативными шумами~// 
Информатика и~ее применения, 2015. Т.~9. Вып.~1. С.~70--75. 
\bibitem{9-bos} %8
\Au{Синицын И.\,Н., Корепанов~Э.\,Р.} Нормальные услов\-но-оп\-ти\-маль\-ные фильтры 
и~экстраполяторы Пугачёва для стохастических систем, линейных относительно 
состояния~// Информатика и~её применения, 2016. Т.~10. Вып.~2. С.~14--23. 
\bibitem{10-bos} %9
\Au{Синицын И.\,Н., Синицын~В.\,И., Корепанов~Э.\,Р.} Модифицированные 
эллипсоидальные услов\-но-оп\-ти\-маль\-ные фильтры для нелинейных стохастических 
систем на многообразиях~// Информатика и~её применения, 2017. Т.~11. Вып.~2.  
С.~101--111.

\bibitem{7-bos} %10
\Au{Борисов А.\,В., Босов~А.\,В., Кибзун~А.\,И., Миллер~Г.\,Б., Семенихин~К.\,В.} Метод 
условно-оптимальной нелинейной фильтрации и~современные подходы к~оцениванию 
состояний нелинейных стохастических\linebreak
 систем~// Автоматика и~телемеханика, 2018. №\,1. 
С.~3--17. 
 \end{thebibliography}

 }
 }

\end{multicols}

\vspace*{-3pt}

\hfill{\small\textit{Поступила в~редакцию 27.12.18}}

\vspace*{8pt}

%\pagebreak

%\newpage

%\vspace*{-29pt}

\hrule

\vspace*{2pt}

\hrule

%\vspace*{-2pt}

\def\tit{ON THE CONDITIONALLY MINIMAX NONLINEAR FILTERING 
CONCEPT DEVELOPMENT: FILTER MODIFICATION AND~ANALYSIS}


\def\titkol{On the conditionally minimax nonlinear filtering 
concept development: Filter modification and~analysis}

\def\aut{A.\,V.~Bosov and~G.\,B.~Miller}

\def\autkol{A.\,V.~Bosov and~G.\,B.~Miller}

\titel{\tit}{\aut}{\autkol}{\titkol}

\vspace*{-11pt}


\noindent
Institute of Informatics Problems, Federal Research Center ``Computer Science 
and Control'' of the Russian Academy of Sciences, 44-2~Vavilov Str., Moscow 
119333, Russian Federation

\def\leftfootline{\small{\textbf{\thepage}
\hfill INFORMATIKA I EE PRIMENENIYA~--- INFORMATICS AND
APPLICATIONS\ \ \ 2019\ \ \ volume~13\ \ \ issue\ 2}
}%
 \def\rightfootline{\small{INFORMATIKA I EE PRIMENENIYA~---
INFORMATICS AND APPLICATIONS\ \ \ 2019\ \ \ volume~13\ \ \ issue\ 2
\hfill \textbf{\thepage}}}

\vspace*{6pt}

    
    
     \Abste{The main result of the research is a new suboptimal filter developed from the 
conditionally minimax nonlinear filtering (CMNF) method for nonlinear stochastic systems in 
discrete time. The main idea of the proposed modification is to omit the time and resource 
consuming phase of \textit{a~priori} CMNF parameter calculation in favor of their online 
approximation together with the current state estimation. In the original CMNF filter, the 
simulation study is used in order to approximate dynamic system parameters' unconditional 
expectation and covariances, while the modified version deals with the conditional moments which 
are also calculated by means of the Monte-Carlo method. The proposed filter modification is 
provided with the minimax justification, similar to the underlying CMNF concept. Simulation 
examples show the proposed algorithm effectiveness and performance gain in comparison with the 
original conditionally minimax nonlinear filter.}
     
     \KWE{nonlinear stochastic observation system in discrete time; conditionally minimax 
nonlinear filtering; Monte-Carlo simulation}
    
    
    
 \DOI{10.14357/19922264190202}

%\vspace*{-14pt}

\Ack
    \noindent
    This work was partially supported by the Russian Foundation
    for Basic Research (grant  
19-07-00187-A).

%\vspace*{6pt}

  \begin{multicols}{2}

\renewcommand{\bibname}{\protect\rmfamily References}
%\renewcommand{\bibname}{\large\protect\rm References}

{\small\frenchspacing
 {%\baselineskip=10.8pt
 \addcontentsline{toc}{section}{References}
 \begin{thebibliography}{99}

\bibitem{2-bos-1}
\Aue{Liptser, R., and A.\,N.~Shiryaev.} 2001. \textit{Statistics of random processes}. 
Berlin\,--\,Heidelberg: Springer-Verlag. 427~p.

\bibitem{1-bos-1}
\Aue{Kallianpur, G.} 1980. \textit{Stochastic filtering theory}. New York, NY: 
Springer-Verlag. 318~p.

\bibitem{3-bos-1}
\Aue{Pugacev, V.\,S.} 1978. 
Recursive estimation of variables and parameters in stochastic systems described by 
autoregression equations]. \textit{Sov. Math.
Dokl}. 19:991--995.
\bibitem{4-bos-1}
\Aue{Pugachev, V.\,S.} 1979. Estimation of variables and parameters in 
discrete-time nonlinear systems. \textit{Automat. Rem. Contr.} 40(4):39--50.
\bibitem{5-bos-1}
\Aue{Pankov, A.\,R.} 1993. Recurrent conditionally minimax filtering of 
processes in nonlinear difference stochastic systems. 
\textit{J.~Comput. Sys. Sc. Int.} 31(4):54--60.
\bibitem{6-bos-1}
\Aue{Pankov, A.\,R., and A.\,V.~Bosov.} 1994. Conditionally minimax algorithm 
for nonlinear system state estimation. \textit{IEEE T.~Automat. Contr.} 39(8):1617--1620.

\bibitem{8-bos-1} %7
\Aue{Sinitsyn, I.\,N., and E.\,R.~Korepanov}. 2015. Ustoychivye lineynye
     uslovno-optimal'nye fil'try i~ekstrapolyatory dlya stokhasticheskikh sistem 
     s~mul'tiplikativnymi shumami [Stable linear conditionally optimal filters and 
extrapolators for stochastic systems with multiplicative noises]. \textit{Informatika i~ee 
Primeneniya~--- Inform. Appl.} 9(1):70--75. 
\bibitem{9-bos-1} %8
\Aue{Sinitsyn, I.\,N., and E.\,R.~Korepanov.} 2016. Normal'nye uslovno-optimal'nye 
fil'try i ekstrapolyatory Pugacheva dlya stokhasticheskikh sistem, 
lineynykh otnositel'no so\-sto\-yaniya [Normal Pugachev conditionally-optimal filters 
and extrapolators for state linear stochastic systems]. \textit{Informatika i~ee Primeneniya~--- 
Inform. Appl.} 10(2):14--23. 
\bibitem{10-bos-1} %9
\Aue{Sinitsyn, I.\,N., V.\,I.~Sinitsyn, and E.\,R.~Korepanov.} 2017. 
Modifitsirovannye ellipsoidal'nye uslovno-optimal'nye fil'try dlya nelineynykh 
stokhasticheskikh sistem na mnogoobraziyakh [Modificated ellipsoidal conditionally 
optimal filters for nonlinear stochastic systems on manifolds]. \textit{Informatika i~ee 
Primeneniya~--- Inform. Appl.} 11(2):101--111.

\bibitem{7-bos-1} %10
\Aue{Borisov, A.\,V., A.\,V.~Bosov, A.\,I.~Kibzun, G.\,B.~Miller, and K.\,V.~Semenikhin}. 
2018. The conditionally minimax nonlinear filtering method and modern 
approaches to state estimation in nonlinear stochastic systems. \textit{Automat. Rem. 
Contr.} 79(1):1--11.
\end{thebibliography}

 }
 }

\end{multicols}

\vspace*{-6pt}

\hfill{\small\textit{Received December 27, 2018}}

%\pagebreak

%\vspace*{-18pt}    

    \Contr
    
    \noindent
    \textbf{Bosov Alexey V.} (b.\ 1969)~--- Doctor of Science in technology, 
principal scientist, Institute of Informatics Problems, Federal Research Center 
``Computer Science and Control'' of the Russian Academy of Sciences, 44-2~Vavilov 
Str., Moscow 119333, Russian Federation; \mbox{AVBosov@ipiran.ru}
    
    \vspace*{3pt}
    
    \noindent
    \textbf{Miller Gregory B.} (b.\ 1980)~---
    Candidate of Science (PhD) in physics and mathematics, Candidate of Science 
    (PhD) in informatics, 
scientist, Institute of Informatics Problems, Federal Research Center ``Computer 
Science and Control'' of the Russian Academy of Sciences, 44-2~Vavilov Str., 
Moscow 119333, Russian Federation; \mbox{GMiller@ipiran.ru}

\label{end\stat}

\renewcommand{\bibname}{\protect\rm Литература}        