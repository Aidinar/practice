\newcommand{\Cal}{\mathcal}
\newcommand{\Qo}{\mathbb{Q}_{\ge 0}}
\newcommand{\NOT}{\operatorname{\mathtt{NOT}}}
\newcommand{\Ro}{\mathbb{R}_{\ge 0}}
\newcommand{\XOR}{\mathbin{\mathtt{XOR}}}
\newcommand{\OR}{\mathbin{\mathtt{OR}}}
\newcommand{\AND}{\mathbin{\mathtt{AND}}}
\newcommand{\EXP}{\mathbin{\uparrow}}
\newcommand{\DIV}{\mathbin{/\!/}}
\newcommand{\fac}{\mathbin{\!/\!}}


\def\stat{anashin}

\def\tit{О ТЕОРЕТИКО-АВТОМАТНЫХ МОДЕЛЯХ БЛОКЧЕЙН-СРЕДЫ$^*$}

\def\titkol{О теоретико-автоматных моделях блокчейн-среды}

\def\aut{В.\,С.~Анашин$^1$}

\def\autkol{В.\,С.~Анашин}

\titel{\tit}{\aut}{\autkol}{\titkol}

\index{Анашин В.\,С.}
\index{Anashin V.\,S.}


{\renewcommand{\thefootnote}{\fnsymbol{footnote}} \footnotetext[1]
{Работа выполнена при
поддержке РФФИ (проект 18-20-03124).}}


\renewcommand{\thefootnote}{\arabic{footnote}}
\footnotetext[1]{Факультет вычислительной математики и~кибернетики Московского
государственного университета   им.\
М.\,В.~Ломоносова, \mbox{anashin@iisi.msu.ru}}

\vspace*{-8pt}




\Abst{Рассматриваются методы анализа и~моделирования блокчейн-среды, основанные
на тео\-ре\-ти\-ко-ав\-то\-мат\-ных моделях, в~первую очередь на 
так называемых <<автоматах с~метками
времени>> (timed automata). Также предлагается новая версия автоматов с~метками времени, позволяющая избежать некоторых неудобств моделирования с~помощью классических автоматов с~метками времени, а при моделировании блок\-чейн-сре\-ды
на основе последних приходится использовать
переменные разных типов, действительные и~булевы, что вызывает ряд сложностей
как теоретического, так и~практического характера. Предлагаемый подход основан
на применении 2-адического анализа, что дает возможность использовать переменные
одного и~того же типа, а именно булева.}


\KW{блокчейн-среда; смарт-контракт; автомат
с~метками времени}

\DOI{10.14357/19922264190205}
  
\vspace*{6pt}


\vskip 10pt plus 9pt minus 6pt

\thispagestyle{headings}

\begin{multicols}{2}

\label{st\stat}

\section{Введение}
%\label{sec:intro}

При построении математических моделей\linebreak
 блок\-чейн-сре\-ды тео\-ре\-ти\-ко-ав\-то\-мат\-ная 
модель\linebreak
воз\-ни\-ка\-ет естественным образом, поскольку функ\-ци\-о\-ни\-ро\-ва\-ние 
блок\-чейн-сре\-ды~--- это детерминированный
процесс, в~ходе которого решение 
о~включении или невключении блока в~реестр  зависит как
от предыдущих блоков (булев тип данных), так и~от времени (тип данных~---
действительные чис\-ла). Таким образом,  если рассматривать текущее состояние
реестра как предыдущее состояние, а~состояние реестра сразу после включения
в него очередного блока как последующее состояние, то процесс
изменения содержимого реестра
можно описать с~по\-мощью понятия
<<автомат с~метками времени>> (timed automaton), для которого далее 
в~текс\-те статьи используется термин T-ав\-то\-мат
или кратко~--- TA.

Основы  математической теории ТА были заложены в~работе~\cite{timed-auto}.
В~плане математических моде\-лей\linebreak блок\-чейн-сре\-ды
на основе ТА пред\-став\-ляет\linebreak
интерес работа~\cite{Bitcoin-contract-model}, поскольку в~ней мо\-де\-ли\-ру\-ет\-ся
функционирование смарт-кон\-трак\-тов в~бит\-койн-сре\-де (последняя
пред\-став\-ля\-ет собой част\-ный случай блок\-чейн-сре\-ды), что является одним из наиболее
важных моментов для полноценного функционирования цифровой экономики на основе блок\-чейн-сре\-ды 
(а~не только для  реализации криптовалют). Отметим, что в~этой работе для
моделирования смарт-кон\-трак\-тов использован
язык Uppaal, разработанный  для описания и~верификации моделей, основанных
на ТА (см.\ описание \mbox{Uppaal} в~\cite{Uppaal-tutorial}). На основе этой же модели
описывается и~бит\-койн-сре\-да~\cite{Model-bitcoin-uppaal}. Подчеркнем,
что с~точки зрения математического описания и~моделирования любой юридически
правильно составленный
контракт можно рассматривать как конечный автомат 
(см., например,~\cite{contract-automat}).
Более того, известны
методы извлечения описания смарт-контрактов как конечных автоматов из функционирующей
во времени
блок\-чейн-среды~\cite{mining-smart-contract}.

Сказанное означает, что функционирование смарт-кон\-трак\-тов в~блок\-чейн-сре\-де можно
рас\-смат\-ри\-вать как взаимодействие автоматов во времени,  т.\,е.\ ТА можно рассматривать
как релевантную модель  описания такого взаимодействия.

Отметим, что  моделирование функционирования смарт-кон\-трак\-тов в~блок\-чейн-сре\-де 
является одним из важных методов проверки стойкости  против компрометации. 
Смарт-кон\-тракт использует входные данные от других смарт-кон\-трак\-тов,
пользователей, а~также о~текущем времени и~выдает  выходные данные, которые
используются другими пользователями и/или другими смарт-кон\-трак\-та\-ми, а~потому
ошибочное функционирование одного смарт-кон\-трак\-та может привести к~сбою 
в~работе всех связанных с~ним смарт-кон\-трак\-тов и~узлов сети. Однако смарт-кон\-тракт
может быть очень сложно устроен даже уже на уровне юридического документа,
не говоря уже о его программной реализации, поэтому требуется тщательная
проверка правильного функционирования смарт-кон\-трак\-та\linebreak
 и~в~юридическом плане,
и как компьютерной программы. Такую проверку достаточно сложно выполнить
вручную, однако моделирование смарт-кон\-трак\-та позволяет поставить ряд машинных\linebreak
экспериментов для изучения поведения смарт-кон\-трак\-та как автомата при подаче
на него тех или иных входных данных, т.\,е.\ смоделировать его поведение в~самых
разных условиях. Именно эти соображения послужили мотивацией
 работы~\cite{Bitcoin-contract-model}, в~которой используется 
 тео\-ре\-ти\-ко-ав\-то\-мат\-ная модель
смарт-кон\-трак\-та на основе ТА, где время представляется действительными чис\-лами.



 В данной работе
предлагается принципиально иной подход к~построению 
тео\-ре\-ти\-ко-ав\-то\-мат\-ных моделей  функционирования блок\-чейн-сре\-ды 
(в~частности,  функционирования
смарт-кон\-трак\-тов в~этой среде) как автомата
с временн$\acute{\mbox{ы}}$ми метками,  в~котором физическое время 
представляется не действительными, а~2-ади\-че\-ски\-ми числами.
Такой подход представляется автору оправданным и~плодотворным по
нескольким причинам:
\begin{itemize}
\item итоговая модель представляет собой автомат в~стандартном понимании
этого термина;
при этом задаваемое
автоматом преобразование слов  реализуется не в~виде таблицы переходов
состояний, а~в~виде программы без вет\-вле\-ния, пред\-став\-ля\-ющей собой
последовательность стандартных команд процессора,\linebreak а~именно:
арифметических и~поразрядных логических операций;
\item поскольку полученный автомат является <<обычным>> автоматом, описание
его функционирования может быть сведено к~изучению функции, заданной и~принимающей
значения в~пространстве целых 2-ади\-че\-ских чисел ввиду того, что каждая детерминированная
(по C.\,В.~Яблонскому)
функция (т.\,е.\ функция, задаваемая автоматом) есть
$p$-ади\-че\-ская функция,
удовлетворяющая  условию Липшица с~константой~1 относительно подходящей $p$-ади\-че\-ской
метрики, и~обратно: все такие функции являются детерминированными 
(см., например,~\cite{me:Discr_Syst});
\item сказанное дает возможность применять к~изуче\-нию таких автоматов
(а~значит, и~к~изуче\-нию функционирования смарт-кон\-трак\-тов в~блок\-чейн-сре\-де) 
развитый аппарат $p$-ади\-че\-ско\-го анализа и,~шире,
алгебраической динамики в~духе монографии \cite{AnKhr};
\item наконец, 2-ади\-че\-ское (и, шире, $p$-ади\-че\-ское) время является хоть 
и~не общепринятой, но тем не менее довольно широко используемой и~во многих случаях
релевантной математической моделью физического времени, активно изучаемой
уже почти три десятилетия
в рамках $p$-адической математической физики (см.\  обзорную статью~\cite{DraKhrenVol},
а~также соответствующие разделы
и~ссылки в~\cite{AnKhr}).
\end{itemize}




\section{Автоматы и~языки}

%\label{sec:auto}
Понятие <<автомат>> в~русскоязычной литературе  используется в~различных
смыслах, которым со\-ответствуют  английские термины state machine, sequential
machine, transducer и~т.\,д., поэтому\linebreak во избежание недоразумений введем определения,
используемые  в~данной работе. Везде далее под~\textit{алфавитом} понимается
конечное непустое множество, содержащее хотя бы два элемента.

\smallskip

\noindent
\textbf{Определение 2.1.}\
\textit{Автомат-определитель} (далее~--- d-ав\-то\-мат)~--- 
это кортеж $\langle\Cal I,\Cal S,\Cal F,S,s_0\rangle$,
где
\begin{enumerate}[(1)]
\item $\Cal I$  есть  \textit{входной} алфавит;
\item $\Cal S$ есть непустое (необязательно конечное) множество, называемое
множеством \textit{со\-сто\-яний};
\item $\Cal F$ есть конечное непустое подмножество множества~$\Cal S$, называемое
множеством \textit{при\-ни\-ма\-ющих со\-сто\-яний};
\item $s_0\in\Cal S$~--- \textit{начальное} со\-сто\-яние;
\item $S\colon \Cal I\times\Cal S\to \Cal S$~--- \textit{функция перехода}.
\end{enumerate}
Автомат-определитель называется \textit{конечным}, если конечно множество $\Cal S$ его
состояний.

\smallskip

Множество  всех конечных последовательностей~$\mathbf W(\Cal I)$ над множеством
$\Cal I$ называется множеством \textit{слов}. Отметим, что $\mathbf W(\Cal I)$ не содержит пустого слова (слова нулевой длины)
$\varnothing$; полагаем $\mathbf W_0(\Cal I)\hm=\mathbf W(\Cal I)\cup \{\varnothing\}$.
Используем стандартное определение языка, распознаваемого d-ав\-то\-ма\-том, и~регулярного
языка (языка, распознаваемого конечным d-ав\-то\-ма\-том) (см., 
например,~\cite{Allouche-Shall}).
Отметим, что данному понятию конечного d-ав\-то\-ма\-та в~\cite{Allouche-Shall} соответствует
понятие DFA~--- deterministic finite automaton.


\smallskip

\noindent
\textbf{Определение 2.2.}\
\textit{Автомат-преобразователь} (далее~--- f-ав\-то\-мат)  
есть кортеж $\langle\Cal I,\Cal S,\Cal O,S,O,s_0\rangle$,
где $\Cal I$, $\Cal S$, $S$ и~$s_0$  те же, что и~в~определении~2.1,
$\Cal O$~---   \textit{выходной} алфавит, а $O\colon \Cal I\times\Cal S\to\Cal O$ ---
\textit{функция выхода}.


\smallskip

Вышеприведенное определение соответствует  понятию автомата Мили, или, что
то же самое, понятию 1-равномерного преобразователя (1-uniform transducer)
 из~\cite{Allouche-Shall}, с~той лишь разницей, что множество состояний автомата
 в~смысле определения~2.2 может
быть и~бесконечным; если же множество состояний конечно, то определение~2.2
превращается в~стандартное определение автомата Мили (Mealy sequential machine).

Каждый преобразователь естественным образом задает отображение множества~$\mathbf W(\Cal I)$
во множество~$\mathbf W(\Cal O)$, а~каж\-дый d-ав\-то\-мат задает отображение множества
$\mathbf W(\Cal I)$ во множество~$\Cal F \cup {R}$, где $R\hm\notin \Cal S$, и~в~$R$
отображаются те и~только те слова,  которые
не принимаются d-ав\-то\-матом.

\vspace*{-9pt}


\section{Автоматы и~время}
%\label{sec:T-auto}

\vspace*{-3pt}

T-автоматы, они же автоматы с~метками времени, или
timed automata в~англоязычной литературе, введенные в~\cite{timed-auto},
использовались
для моделирования функционирования блок\-чейн-сре\-ды, поскольку разные узлы, функционирующие
в~блок\-чейн-сре\-де, получают
очередной блок, вообще говоря, в~разные моменты времени.
Целый ряд атак на блокчейн основан именно
на факте <<раз\-но\-вре\-мен\-ности>> получения очередных блоков поль\-зо\-ва\-те\-лями.

Чтобы ввести
понятие Т-ав\-то\-ма\-та,
сначала понадобится определение (бесконечного) слова  с~метками  времени
(timed word), или, для краткости, t-сло\-ва.


\smallskip

\noindent
\textbf{Определение 3.1.}\
(Бесконечное) \textit{слово с~метками времени} (t-сло\-во) над алфавитом~$\Cal A$ есть
(бесконечная) последовательность пар $((a_i,\tau_i))_{i=0}^\infty$, где $a_i\hm\in\Cal
A$, $\tau_i\hm\in\R_{\ge 0}$, причем последовательность~$(\tau_i)_{i=0}^\infty$ действительных
неотрицательных чисел~$\tau_i$ строго и~неограниченно возрастает: 
$\tau_0\hm<\tau_1\hm<\tau_2<\cdots$
и~$\lim_{i\to\infty}\tau_i\hm=\infty$.


\smallskip

На содержательном уровне слово с~метками времени $((a_i,\tau_i))_{i=0}^\infty$
интерпретируется как последовательность символов $(a_i)_{i=0}^\infty$ алфавита~$\Cal A$, 
поданных в~автомат в~моменты времени $\tau_0, \tau_1, \ldots$
соответственно.

Ниже, следуя~\cite{timed-auto}, дадим определение Т-ав\-то\-ма\-тов.
В~этом определении будет использоваться понятие \textit{таблицы переходов состояний
с~метками
времени}, а также  понятие \textit{временн$\acute{\mbox{ы}}$х ограничений} на переходы.

\smallskip

\noindent
\textbf{Определение 3.2.}\
%\label{def:TTT}
Пусть $T$ есть (счетное) множество переменных, называемых далее 
\textit{временн$\acute{\mbox{ы}}$ми переменными}, пусть $t_j\hm\in T$. 
\textit{Временн$\acute{\mbox{ы}}$м ограничением} называется
любая булева комбинация предикатов вида $t_j\hm\le q$ и~$q\hm\le t_j$, где
$q\in \Qo$ --- неотрицательные рациональные константы, а символ~$\le$
интерпретируются обычным образом как бинарное отношение <<меньше  либо
равно>>. Пусть даны: (входной) алфавит~$\Cal I$, конечное множество~$\Cal S$ состояний,
конечное множество~$C$, называемое множеством \textit{таймеров}, и~множество
$\Phi(C)$  временн$\acute{\mbox{ы}}$х ограничений от временн$\acute{\mbox{ы}}$х переменных $t_1,\ldots,t_{|C|}$.
Тогда \textit{таблица переходов состояний
с~метками
времени} (далее~--- ТТТ) есть некоторое множество кортежей вида 
$(s,a,s^\prime, G,\varphi)$, где $s,s^\prime\hm\in \Cal S$, $\varphi\hm\in\Phi(C)$, 
$G\hm\in 2^C$~--- подмножество множества таймеров (возможно, и~пустое).

\smallskip

\noindent
\textbf{Определение 3.3.}\
%\label{def:T-auto}
\textit{Детерминированный автомат с~метками времени} (далее~--- Т-ав\-то\-мат) 
есть кортеж $\langle\Cal I,\Cal S,\Cal E,C,
\Phi(C),s_0\rangle$,
где $\Cal I$, $\Cal S$, $s_0$~--- те же, что и~в  определении~2.1,
$\Cal S$ --- конечное множество,  $C$ и~$\Phi(C)$~--- из определения~3.2, 
а~$\Cal E$~--- ТТТ в~смысле определения~3.2, причем
\begin{enumerate}[(1)]
\item  в~начальном состоянии~$s_0$ текущие значения всех временн$\acute{\mbox{ы}}$х 
переменных равны~0;
\item для любых $a\hm\in\Cal I$, $s\hm\in\Cal S$ и~любой пары элементов TTT~$\Cal E$
вида $(s,a,*,*,\varphi_1)$, $(s,a,*,*,\varphi_2)$ временн$\acute{\mbox{ы}}$е ограничения~$\varphi_1$
и $\varphi_2$ являются взаимно исключающими,  т.\,е.\  $\varphi_1\wedge\varphi_2$
тождественно ложно.
\end{enumerate}


\noindent
\textbf{Замечание.}\
Если в~определении~3.3 допустить наличие нескольких начальных
состояний и~опустить условие~(2), то получим определение недетерминированного
Т-ав\-то\-мата.

\vspace*{-6pt}

\section{$\mathrm d$- и~$\mathrm f$-автоматы вместо Т-автоматов в~моделях блокчейн-среды}
\label{sec:T-block}

\vspace*{-3pt}

В этом разделе обсудим возможность сведения T-ав\-то\-мат\-ной модели блок\-чейн-сре\-ды,
используемой в~литературе (например, в~работах~\cite{Bitcoin-contract-model,Model-bitcoin-uppaal}),
и~покажем, что описанные T-ав\-то\-мат\-ные  модели блок\-чейн-среды могут быть <<без
потери точ\-ности>>  сведены  к~существенно более простым моделям,
а~именно: к~<<обычным>>  \mbox{f-ав}\-то\-ма\-там в~смысле определения~2.2.
Отметим сразу, что <<физической основой>> для такой <<аппроксимации>> функционирования
блокчейн-среды с~помощью более простой \mbox{f-ав}\-то\-мат\-ной модели, чем  с~помощью
используемой в~литературе более сложной \mbox{Т-ав}\-то\-мат\-ной,
служит следующее ограничение:
\textit{в реальной жизни время, разделяющее два следующих один за другим события,
не может быть произвольно \underline{малым}}, оно всегда ограничено снизу некоторой
величиной. Например, в~квантовой физике предполагается, что события,\linebreak
 разделенные
планковским временем, т.\,е.\ промежутком примерно в~$5,4\cdot 10^{-44}$~с,
происходят\linebreak
 одновременно, и,~более того, в~настоящее время минимальный интервал
времени, доступный физическому измерению, составляет примерно~$10^{-20}$~с. 
Отсюда следует, что любой временной интервал \textit{кратен} некоторому
минимальному временн$\acute{\mbox{о}}$му интервалу (в~предельном случае~--- планковскому времени)
и,~таким образом, с~точностью до множителя, равного длине этого минимального
интервала, \textit{является  натуральным числом}.

С другой стороны, T-ав\-то\-мат\-ные модели
позволяют рассматривать поведение моделируемой сис\-те\-мы в~<<предельных>>
ситуациях; например,  когда временной промежуток между соседними <<событиями>>
стремится к~0 (см.~\cite[example 3.22]{timed-auto}). Рас\-смот\-ре\-ние <<предельного>>
поведения часто оказывается очень полезным для описания не только качественных,
но нередко и~количественных характеристик моделируемой системы. Возможность
<<перехода к~пределу>> в~Т-ав\-то\-мат\-ных моделях основана на том, что
 множество~$\Qo$ всюду плотно в~множестве~$\Ro$ относительно обычной действительной
метрики.

Таким образом, для моделирования блок\-чейн-сре\-ды хотелось бы построить 
аналог ТА, в~которых время,
с одной стороны, было бы <<дискретным>>, т.\,е.\  в~качестве <<исходных>> 
меток времени выступали
бы натуральные,  а~не рациональные (как в~определении ТА) числа, но, тем не
менее, чтобы сохранялась и~возможность <<предельного перехода>>  как в~целях получения описания поведения всей системы во времени на (хотя бы) качественном
уровне, так и~получения оценок  <<точности>> модели.  Поскольку
речь идет о~предельном переходе, то необходимо задать некоторую
метрику на множестве всех натуральных чисел, относительно которой такой предельный
переход был бы возможен и~относительно которой натуральные числа образовывали
бы всюду плотное подмножество подобно тому, как множество~$\Qo$ является
всюду плотным подмножеством в~$\Ro$ относительно действительной мет\-ри\-ки.
Такие метрики существуют: это $p$-ади\-че\-ские мет\-рики.

Будем рассматривать t-сло\-ва  с~метками
времени из $\N_0\hm=\{0,1,2,3,\ldots\}$ (а~не из~$\Ro$, см.\
 определение~3.1) над конечным
алфавитом~$\Cal A$, содержащим хотя бы два символа.
Без ограничения общности можно считать, что
если $p\hm\ge 2$~---  это мощность алфавита~$\Cal A$, то  символами алфавита~$\Cal A$ 
являются числа $0,1,\ldots,p-1$.
Заметим, что такие  t-сло\-ва  пред\-став\-ля\-ют
собой част\-ный случай так на\-зы\-ва\-емых \textit{слов с~данными} (data word), введенных 
в~\cite{Bouyer-Algebraic-Time}, а~именно: когда множество данных~$\Cal D$
совпадает с~$\N_0$. Введенные в~определении~3.1
слова с~метками времени также пред\-став\-ля\-ют собой слова с~данными для случая,
когда данные лежат в~$\Ro$. На основе понятия слов  с~данными 
 естественным
образом вводится понятие \textit{языка с~данными} (data language), 
а~также \textit{автомата с~данными} (data automaton, далее~--- D-ав\-томат).
\smallskip

\noindent
\textbf{Определение 4.1.}\
%\label{def:D-auto}
\textit{Автомат $\mathfrak A$ с~данными~$\mathcal D$} есть кортеж $\mathfrak A
\hm=\langle\Cal I,\Cal S,\Cal F,\Cal T,k,\sim~,s_0\rangle$, где 
$\Cal I$, $\Cal S$, $\Cal F$, $s_0$~--- те же, что и~в определении~2.1;
\begin{itemize}
\item $k$ есть натуральное число  (называемое числом регистров данных);
\item $\sim$ есть отношение эквивалентности конечного индекса, определенное
на~$\Cal D^k$;
\item $\Cal T\subseteq \Cal S\times\Cal D^k\fac\sim\times\Cal I\times 
\Cal D^k\fac\sim\times\Cal S$ есть конечное множество переходов;
\item $\Cal U$ есть множество модификаций состояний регистров 
$\mathrm{upd}\colon \Cal D^k\to\Cal D^k$,
удовлетворяющие следующим ограничениям:
\begin{itemize}
\item для любого кортежа 
$$
(s,g,a)\in\Cal S\times \Cal D^k\fac\sim\times \Cal I
$$ 
существует (единственная) модификация регистров $\mathrm{upd}\hm\in\Cal U$
такая, что если
$(s,g,a,\mathrm{upd}^\prime,g^\prime,s^\prime)\hm\in\Cal T$ для
некоторого $\mathrm{upd}^\prime\in\Cal U$, то $\mathrm{upd}^\prime\hm=\mathrm{upd}$;
\item если $\!(s,g,a,\mathrm{upd},g^\prime,s),
(s,g,a,\mathrm{upd},g^\prime,s^\prime)\hm\!\in\!\Cal T$, то $s^\prime\hm=s$.
\end{itemize}
\end{itemize}

Как показано в~\cite[теорема~13]{Bouyer-Algebraic-Time}, для любого детерминированного
T-автомата из определения~3.3,\linebreak имеющего~$n$ таймеров (см.\ определение~3.2), 
существует D-ав\-то\-мат с~$2n\hm+2$ регистрами, рас\-по\-зна\-ющий 
в~точ\-ности тот же самый язык. Далее\linebreak построим   f-ав\-то\-ма\-ты (см.\
 определение~2.2), аппроксимирующие с~любой наперед заданной точ\-ностью (в~некотором
точно определенном ниже смысле)  данный детерминированный T-ав\-то\-мат и~тем
самым сведем задачу моделирования функционирования блок\-чейн-сре\-ды (в первую
очередь, моделирования смарт-кон\-трак\-тов) Т-ав\-то\-ма\-та\-ми к~моделированию <<обычными>>
автоматами с~двоичными входами и~двоичными выходами. Для
этого сначала понадобится ввести новый частный тип D-ав\-то\-ма\-тов, а~именно: 
\textit{автоматы с~$p$-ади\-че\-ским временем}.


Зафиксируем некоторое простое число~$p$ (в контексте данной статьи наиболее важным
является случай $p\hm=2$) и~рассмотрим в~качестве меток данных (в словах с~данными) \textit{целые $p$-адические
числа}, т.\,е.\ элементы \textit{пространства~$\Z_p$ 
целых $p$-ади\-че\-ских чисел}. С~введением в~теорию $p$-ади\-че\-ских
чисел и~$p$-ади\-че\-ский анализ можно ознакомиться, например,
по вводным главам в~\cite{AnKhr}. Здесь же введем  лишь
самые необходимые понятия
из $p$-ади\-че\-ско\-го анализа, и~притом на неформальном уровне.

Множество $\Z_p$ можно рассматривать как множество $\mathbf W^\infty (\Cal A)$ 
всех бесконечных (в~одну
сторону, в~данной статье~--- влево)
слов над алфавитом $\Cal A\hm=\{0,1,\ldots,p-1\}$, символы которого можно считать
элементами кольца~$\Z/p\Z$ вычетов по модулю~$p$, 
т.\,е.\ элементами поля из~$p$ элементов. 
На множестве~$\Z_p$ можно задать операции сложения и~умножения
с~по\-мощью стандартных <<школьных>> алгоритмов сложения и~умножения <<в столбик>>
чисел, пред\-став\-лен\-ных в~сис\-те\-ме счис\-ле\-ния с~основанием~$p$.
Если $p\hm=2$, то бесконечные бинарные строч\-ки можно мыслить себе
как представления чисел в~обобщенном \textit{обратном двоичном коде}
(см., например,~\cite[с.~213]{Knuth}).  Использование
обобщенного обратного двоичного кода дает возможность записывать в~регистр 
бесконечной длины как все целые неотрицательные
чис\-ла (им соответствуют строчки с~конечным числом единиц), так и~все 
целые отрицательные
чис\-ла (им соответствуют строчки с~конечным числом нулей), а~также все рациональные
чис\-ла, пред\-ста\-ви\-мые в~виде простых несократимых дробей с~нечетными знаменателями
(им соответствуют периодические с~ка\-ко\-го-то момента строчки).

Множество~$\Z_p$  является  полным компактным метрическим пространством относительно
\textit{$p$-ади\-че\-ской метрики}~$d_p$, которая задается следующим образом:
$d_p(\mathbf a,\mathbf b)={1}/{p^i}$
тогда и~только тогда, когда
$\mathbf a\hm=\cdots a_{i+1}a_{i} {c_{i-1}\cdots
c_0}$, $\mathbf b\hm=\cdots b_{i+1}b_{i} {c_{i-1}\cdots c_0}$
и~$a_{i}\ne b_{i}$ (по определению  $d_p(\mathbf a,\mathbf b)\hm=0$, если такого~$i$ 
не существует, т.\,е.\ если бесконечные слова~$\mathbf a$ и~$\mathbf b$
совпадают). Абсолютная величина~$\|\mathbf a\|_p$ вводится стандартным образом
как расстояние до  числа~0 (этому числу соответствует бесконечная строчка
из одних только нулей): $\|\mathbf a\|_p\hm=d_p(\mathbf a,0)$.

Можно ввести понятия <<приведения
по модулю~$p^n$>>  и~<<сравнения по модулю~$p^n$>> для целых $p$-ади\-че\-ских
 чисел, а~именно:
приведение по модулю~$p^n$ бесконечного слова в~алфавите $\Cal A\hm=\{0,1,\ldots,p-1\}$
означает всего лишь переход к~конечному начальному отрезку длины $n$ этого
бесконечного слова, т.\,е.\ $\bmod \,p^n\colon \mathbf W^\infty (\Cal A)\hm\to \mathbf
W^n(\Cal A)$, где~$\mathbf W^n(\Cal A)$ есть множество всех слов длины~$n$
над алфавитом~$\Cal A$. Отметим, что элементы множества~$\mathbf W^n(\Cal A)$ естественным образом отождествляются с~числами $0,1,\ldots,p^n-1$, представленными
в~системе счисления с~основанием~$p$, а~эти числа, в~свою очередь, отождествляются
с~элементами кольца~$\Z/p^n\Z$ вычетов по модулю~$p^n$.
Более того, оказывается,
что любое отображение  $f_\mathfrak A\colon\mathbf W^\infty(\Cal A)
\hm\to\mathbf W^\infty(\Cal A)$, задаваемое ав\-то\-ма\-том-пре\-об\-ра\-зо\-ва\-те\-лем~$\mathfrak A$, 
входной и~выходной алфавиты которого суть~$\Cal A$, т.\,е.\
 $\Cal I\hm=\Cal O\hm=\Cal A\hm=\{0,1,\ldots,p-1\}$ (см. определение~2.2)
есть функция, определенная  на~$\Z_p$ и~при\-ни\-ма\-ющая значения в~$\Z_p$, 
которая удовлетворяет
$p$-ади\-че\-ско\-му условию Липшица с~константой~1 
(и,~следовательно, является непрерывной
относительно мет\-ри\-ки~$d_p$ функцией):
$\|f_\mathfrak A(\mathbf a)\hm-f_\mathfrak A(\mathbf b)\|_p\hm\le\|\mathbf a 
\hm-\mathbf b\|_p$
для любых $\mathbf a, \mathbf b \hm\in\Z_p$. Верно и~обратное: любое отображение
из~$\Z_p$ в~$\Z_p$, удовлетворяющее $p$-ади\-че\-ско\-му условию Липшица с~константой~1, 
задается некоторым ав\-то\-ма\-том-пре\-об\-ра\-зо\-ва\-те\-лем 
(не обязательно конечным), входной и~выходной алфавиты
которого суть $\{0,1,\ldots,p-1\}$ (см., например,~\cite{me:Discr_Syst}).
{\looseness=1

}

Отметим, что  каждый из двух типов автоматов: ав\-то\-ма\-ты-оп\-ре\-де\-ли\-те\-ли, 
т.\,е.\ $d$-ав\-то\-ма\-ты из определения~2.1,
и~ав\-то\-ма\-ты-пре\-об\-ра\-зо\-ва\-те\-ли, т.\,е.\
 f-ав\-то\-ма\-ты из определения~2.2, ---
может быть  сведен один к~другому
(см.\ подробнее~\cite[теорема~4.3.2]{Allouche-Shall}).
Таким образом,  в~случае $p\hm=2$ задачи о $d$-ав\-то\-ма\-тах и~распознаваемых
ими языках могут быть сведены к~задачам о функциях, удовлетворяющих 2-ади\-че\-ско\-му
условию Липшица с~константой~1.  Такие функции
называются в~литературе также функциями треугольного
вида, двоичными совместимыми функциями, Т-функ\-ци\-ями. 

В~контексте данной статьи
весьма важным является тот факт, что Т-функ\-ции
допускают особо прос\-тую реализацию в~виде компьютерной программы,
а~именно:
\textit{компьютерная реализация детерминированной функции автомата, 
входной\linebreak и~выходной алфавиты которого состоят
из двух символов, не требует реализации его таблицы переходов состояний, а~может
быть записана (в~том числе и~для автоматов с~бесконечным числом состояний)
в~виде Т-функ\-ции, которая, в~свою очередь, представляет собой программу
без ветвлений, состоящую из последовательности стандартных компьютерных
команд, таких как арифметические команды (сложение и~умножение натуральных чисел) 
и~поразрядные логические команды $\OR,\AND,\XOR,\NOT$, а~также производных
от них команд, таких как сдвиг в~сторону старших разрядов, маскирование 
и~ряда других, таких как деление на нечетные числа, возведение нечетных чисел
в~степень и~др.}~\cite{AnKhr}.

Сказанное остается в~силе и~для автоматов, входной и~выходной алфавиты которых
состоят из соответственно~$2^n$ и~$2^m$~символов, поскольку такие автоматы
можно рассматривать как автоматы, имеющие~$n$ двоичных входов и~$m$ двоичных выходов,
а~значит, как многомерные Т-функ\-ции, т.\,е.\ как непрерывные относительно
\mbox{2-ади}\-че\-ской метрики\linebreak отображения
из~$\Z_2^n$ в~$\Z_2^m$, удовлетворяющие многомерному 2-ади\-че\-ско\-му условию
Липшица\linebreak с~константой~1. Для Т-функ\-ций имеется
хорошо развитая математическая теория, основанная на 2-ади\-че\-ском анализе
и~име\-ющая многочисленные (в~первую очередь~--- криптографические)
приложения (см.~\cite{AnKhr}).

Дадим теперь формальное определение автомата с~2-ади\-че\-ским временем.

\smallskip

\noindent
\textbf{Определение 4.2.}\
%\label{def:Z2-auto}
D-автомат из определения~4.1 назовем \textit{автоматом с~2-ади\-че\-ским
временем} ($\Z_2$-ав\-то\-ма\-том), если $\Cal I\hm=\{0,1\}$ и~$\Cal D\hm=\Z_2$.


\smallskip

Разумеется, похожим образом можно сформулировать и~понятие $\Z_2$-ав\-то\-ма\-та
с~входным алфавитом из~$2^r$~символов, т.\,е.\ $\Z_2$-ав\-то\-ма\-та с~$r$~двоичными
входами.
Язык, распознаваемый $\Z_2$-ав\-то\-ма\-том, определяется обычным образом на основе
определения~4.2.

Говоря неформально, любой T-ав\-то\-мат (см.\ определение~3.3) можно
рассматривать как <<автомат с~двумя входами>>: временн$\acute{\mbox{ы}}$м и~алфавитным,
где на каждом такте работы  подается на алфавитный вход очередной символ 
входного слова, а~на временной вход~--- действительное чис\-ло, 
служащее меткой времени этого входного символа.  
С~этой точки зрения $\Z_2$-автомат тоже имеет два
входа; при этом на алфавитный вход подается символ входного алфавита, т.\,е.~0 
или~1, а~на временной вход~--- метка времени, т.\,е.\ целое 2-ади\-че\-ское \mbox{число}.

Все t-слова
могут быть равномерно приближены словами с~2-ади\-че\-ски\-ми метками времени (далее~--- 
$\Z_2$-сло\-ва\-ми)
в~следующем смысле. Вначале все символы входного алфавита T-ав\-то\-ма\-та пронумеруем 
и~запишем в~виде двоичных представлений соответствующих чисел. Таким образом,
можно считать, что на алфавитный вход автомата всегда подается~$r$~бинарных 
последовательностей, где~$r$~--- число двоичных разрядов, необходимых для
записи всех символов входного алфавита.

Далее зафиксируем любое действительное $\varepsilon\hm>0$ и~выберем  рациональные 
числа~$z_i(\mathbf w)$, представимые в~виде простых несократимых
дробей с~нечетными знаменателями (все эти рациональные
числа лежат в~$\Z_2$) так, чтобы $|\tau_i(\mathbf w)\hm-z_i(\mathbf w)|
\hm<\varepsilon$, где~$\tau_i(\mathbf w)$
есть $i$-я метка времени в~t-сло\-ве~$\mathbf w$.  Такой выбор всегда можно
сделать, например, следующим образом.
Представим 
$$
\tau_i(\mathbf w)=\lfloor\tau_i(\mathbf w)\rfloor 
+ \left(\tau_i(\mathbf w)-\lfloor\tau_i(\mathbf w)\rfloor\right),
$$ 
где~$\lfloor\tau_i(\mathbf w)\rfloor$
есть целая (с~недостатком) часть чис\-ла~$\tau_i(\mathbf w)$. Выберем $h\hm\in\N$
таким, чтобы $1/3^{h}\hm<\varepsilon$, запишем дроб\-ную часть $(\tau_i(\mathbf w)
\hm-\lfloor\tau_i(\mathbf w)\rfloor)$ чис\-ла~$\tau_i(\mathbf w)$  в~троичной
сис\-те\-ме счис\-ле\-ния с~точ\-ностью до~$h$~троичных разрядов после запятой. Тогда
эта дроб\-ная часть есть чис\-ло вида~$c/3^h$, где $c\hm\in\{0,1,\ldots,3^h-1\}$,
и,~следовательно, является целым 2-ади\-че\-ским чис\-лом. Прибавляя к~полученному
таким образом
числу   целую (с недостатком) часть~$\lfloor\tau_i(\mathbf w)\rfloor$
числа~$\tau_i(\mathbf w)$, получаем
целое 2-ади\-че\-ское число~$z_i(\mathbf w)$. 
В~этом смысле каждое t-сло\-во $\mathbf w\hm= ((a_i,\tau_i))_{i=0}^\infty$ 
приближается с~точ\-ностью не хуже чем~$\varepsilon$ словом 
$((a_i,z_i(\mathbf w)))_{i=0}^\infty$,
которое является входным $\Z_2$-сло\-вом для $\Z_2$-ав\-то\-ма\-та с~$r$~алфавитными
входами, причем алфавит каждого алфавитного входа бинарный.

Далее все $\Z_2$-сло\-ва могут быть равномерно приближены $\Z_2$-сло\-ва\-ми с~метками
времени из~$\N_0$ (и~даже из~$\Z/2^h\Z$) с~любой наперед заданной 2-ади\-че\-ской 
точ\-ностью~$1/2^h$.
Действительно, для этого достаточно каждую из 2-ади\-че\-ских меток времени 
в~каждом $\Z_2$-сло\-ве привести по модулю~$2^h$.
 Таким\linebreak
  образом на основе вышеописанной процедуры <<аппроксимации>> 
 t-ав\-то\-ма\-та $\Z_2$-ав\-то\-ма\-том можно постро\-ить  последовательность 
 $\Z_2$-ав\-то\-ма\-тов~$\mathfrak Y_h$ с~метками
времени из~$\Z/2^h\Z$, $h\hm=1,2,3,\ldots$, аппроксимирующих в~вышеуказанном 
смысле исходный t-ав\-томат.

Используя описанную выше процедуру по\-стро\-ения f-ав\-то\-ма\-та на основе
данного d-ав\-то\-ма\-та,\linebreak можно   любому D-ав\-то\-ма\-ту  
сопоставить  \textit{детерминированную
функцию с~метками времени},  считая, например, что $i$-й символ выходного слова
имеет ту же метку времени, что и~$i$-й символ соответствующего ему входного слова. 
Таким  образом  на основе данного $\Z_2$-ав\-то\-ма\-та из определения~4.2
можно построить детерминированную функцию с~метками времени  из~$\Z_2$,
полагая $i$-й выходной символ равным~1, если автомат находится в~принимающем
состоянии (т.\,е.\ в~состоянии из множества~$\Cal F$), и~0 в~противном случае.

Наконец, этим способом каждому из построенных  выше аппроксимирующих автоматов~$\mathfrak Y_h$ 
можно сопоставить детерминированную функцию с~метками времени
из~$\Z/2^h\Z$. Итак, для данного Т-ав\-то\-ма\-та построена последовательность
аппроксимирующих его (в~вышеописанном смысле)  T-функ\-ций, т.\,е.\ <<обычных>> f-ав\-то\-ма\-тов 
в~смысле определения~2.2, имеющих $r\hm+h$ двоичных входов и~$h\hm+1$ 
двоичный выход. Таким образом, доказано следующее

\smallskip

\noindent
\textbf{Предложение.}\
\textit{Каждый Т-ав\-то\-мат аппроксимируется с~любой наперед заданной 
точностью} (в описанном выше смысле) 
\textit{некоторым f-ав\-то\-ма\-том над двухсимвольным алфавитом.}

\vspace*{-9pt}


\section{Выводы}

\vspace*{-2pt}

В данной работе показано, что для моделирования функционирования блок\-чейн-сре\-ды 
(в~част\-ности, моделирования работы
смарт-кон\-трак\-тов), даже несмотря на то что эта среда функционирует в~реальном
физическом времени, нет не\-об\-хо\-ди\-мости прибегать к~сложным (и~достаточно ресурсоемким)
моделям, основанным на концепции автоматов с~метками времени, представляющими 
собой действительные числа, а достаточно
ограничиться (без потери точности) моделированием этой среды с~помощью 
 детерминированных
функций над \mbox{2-сим}\-воль\-ным алфавитом (известных также под названием Т-функ\-ций),
т.\,е.\ с~по\-мощью <<обычных>> автоматов с~бинарным вход\-ным/вы\-ход\-ным алфавитом. 
Эти функции могут быть реализованы в~виде программ
без ветвления, выполненных как последовательности стандартных команд любого
процессора, что позволяет надеяться на относительную простоту их программной
реализации и~высокое быстродействие соответствующих программ.

 {\small\frenchspacing
 {%\baselineskip=10.8pt
 \addcontentsline{toc}{section}{References}
 \begin{thebibliography}{99}

\bibitem{timed-auto}
\Au{Alur~R., Dill~D.}
 The theory of timed automata~//
 Real-time: Theory and practice~/ Eds. J.~W.~de Bakker, C.~Huizing, W.\,P.~de Roever, 
 G.~Rozenberg.~---
Lecture notes
in computer science ser.~--- Springer, 1992. Vol.~600.  P.~45--73.

\bibitem{Bitcoin-contract-model} %2
\Au{Andrychowicz~M., Dziembowski~S., Malinowski~D., Mazurek~L.}
Modelling bitcoin contracts by timed automata~//
Formal modelling and analysis of timed systems~/
Eds. A.~Legay, M.~Bozga.~---
Lecture notes in computer science ser.~--- Springer, 2014. Vol.~8711: 
 P.~7--22.

\bibitem{Uppaal-tutorial}
\Au{David~A., Larsen~K.\,G., Legay~A., 
\mbox{Miku\!{\!\ptb{\normalsize \v{c}}}ionis}~M., Poulsen~D.\,B.}
Uppaal SMC tutorial~//
{Int. J.~Softw. Tools Te.}, 2015. Vol.~17. P.~397--415.

\bibitem{Model-bitcoin-uppaal}
\Au{Chaudhary~K., Fehnker~A., van~de~Pol J., Stoelinga~M.}
Modeling and verification of the bitcoin protocol~//
Electronic Proc. Theor. Comput. Sci., 2015. Vol.~196. P. 46--60.
%{Workshop on Models for Formal Analysis of Real Systems (MARS   2015)}. 

\bibitem{contract-automat}
\Au{Flood~M.\,D.,  Goodenough~O.\,R.}
Contract as automaton: The computational representation of financial
  agreements~//
{SSRN Electronic~J.}, 2015.
doi: 10.2139/\mbox{ssrn}. 2538224.

\bibitem{mining-smart-contract}
\Au{Guth~F., W$\ddot{\mbox{u}}$stholz~V., Christakis~M., M$\ddot{\mbox{u}}$ller~P.}
Specification mining for smart contracts with automatic abstraction
  tuning~// arXiv.org, 2018. arXiv:1807.07822 [cs.SE]. 12~p.

\bibitem{me:Discr_Syst} %7
\Au{Anashin~V.}
The non-{A}rchimedean theory of discrete systems~//
Math. Comput.  Sci., 2012. Vol.~6. P.~375--393.

\bibitem{AnKhr}
\Au{Anashin~V., Khrennikov~A.}
{Applied algebraic dynamics.}~---
Gruyter expositions in mathematics ser.~---
Berlin\,--\,New York: Walter~de~Gruyter GmbH \& Co, 2009. Vol.~49. 533~p.


\bibitem{DraKhrenVol}
\Au{Dragovich~B., Khrennikov~A.\,Yu., Kozyrev~S.\,V., Volovich~I.\,V.}
On $p$-adic mathematical physics~//
$p$-Adic Numbers Ultrametric Analysis Appl.,
2009. Vol.~1. P.~1--17.



\bibitem{Allouche-Shall}
\Au{Allouche~J.-P., Shallit~J.}
Automatic sequences. Theory, applications, generalizations. ---
Cambridge: Cambridge University Press, 2003. 583~p.

\bibitem{Bouyer-Algebraic-Time}
\Au{Bouyer~P., Petit~A., Th$\acute{\mbox{e}}$rien~D.}
An algebraic approach to data languages and timed languages~//
Inform. Comput., 2003. Vol.~182. P.~137--162.


\bibitem{Knuth}
\Au{Knuth~D.}
The art of computer programming. Vol.~2: Seminumerical
  algorithms.~--- 3rd ed.~---
Addison-Wesley, 1997. 791~p.

 \end{thebibliography}

 }
 }

\end{multicols}

\vspace*{-3pt}

\hfill{\small\textit{Поступила в~редакцию 09.02.19}}

\vspace*{8pt}

%\pagebreak

%\newpage

%\vspace*{-29pt}

\hrule

\vspace*{2pt}

\hrule

%\vspace*{-2pt}

\def\tit{ON AUTOMATA MODELS OF~BLOCKCHAIN}


\def\titkol{On automata models of~blockchain}

\def\aut{V.\,S.~Anashin}

\def\autkol{V.\,S.~Anashin}

\titel{\tit}{\aut}{\autkol}{\titkol}

\vspace*{-11pt}


\noindent
Faculty of Computational Mathematics and Cybernetics, M.\,V.~Lomonosov
Moscow State University,
1-52~Leninskie Gory, GSP-1,  Moscow 119991, Russian Federation

\def\leftfootline{\small{\textbf{\thepage}
\hfill INFORMATIKA I EE PRIMENENIYA~--- INFORMATICS AND
APPLICATIONS\ \ \ 2019\ \ \ volume~13\ \ \ issue\ 2}
}%
 \def\rightfootline{\small{INFORMATIKA I EE PRIMENENIYA~---
INFORMATICS AND APPLICATIONS\ \ \ 2019\ \ \ volume~13\ \ \ issue\ 2
\hfill \textbf{\thepage}}}

\vspace*{6pt}



\Abste{The author considers automata models of blockchain, 
mostly based on timed automata. The author suggests a~new version of timed 
automata that avoids some inconveniences that occur in modeling by using 
standard timed automata where time is represented by real numbers. In the latter case, 
one should use variables of two types, Boolean and real; when applied to blockchain 
modeling, this fact causes some difficulties both in obtaining theoretical estimates 
and in program implementation.  The present approach is based on 2-adic analysis since in 
that case, both time and digital variables are of one type only; namely, Boolean.} 

\KWE{blockchain; smart contract; timed automaton}





 \DOI{10.14357/19922264190205}

\vspace*{-14pt}

\Ack
\noindent
The research was supported by the Russian Foundation for Basic Research
(grant 18-29-03124).


\vspace*{4pt}

  \begin{multicols}{2}

\renewcommand{\bibname}{\protect\rmfamily References}
%\renewcommand{\bibname}{\large\protect\rm References}

{\small\frenchspacing
 {%\baselineskip=10.8pt
 \addcontentsline{toc}{section}{References}
 \begin{thebibliography}{99}
\bibitem{timed-auto-1} %1
\Aue{Alur, R., and D.~Dill.} 1992.
The theory of timed automata.
\textit{Real-time: Theory and practice}.
Eds. J.~W.~de Bakker, C.~Huizing, W.\,P.~de Roever, 
 and G.~Rozenberg. Lecture notes
in  computer science ser.  Springer. 600:45--73.


\bibitem{Bitcoin-contract-model-1} %2
\Aue{Andrychowicz, M., S.~Dziembowski, D.~Malinowski, and L.~Mazurek.} 2014.
Modelling bitcoin contracts by timed automata.
\textit{Formal modelling and analysis of timed systems}.
Eds. A.~Legay and M.~Bozga. 
Lecture notes in computer science ser. Springer. 8711:7--22.


\bibitem{Uppaal-tutorial-1} %3
\Aue{David, A., K.\,G.~Larsen, A.~Legay, M.~\mbox{Miku{\!\ptb{\normalsize \v{c}}}ionis}, 
and D.\,B.~Poulsen.} 
2015.
Uppaal SMC tutorial.
\textit{Int. J.~Softw. Tools Te.} 17:397--415.

\bibitem{Model-bitcoin-uppaal-1} %4
\Aue{Chaudhary, K., A.~Fehnker, J.~van~de~Pol, and M.~Stoelinga}. 2015.
Modeling and verification of the bitcoin protocol.
%Workshop on Models for Formal Analysis of Real Systems (MARS  2015): 
\textit{Electronic Proc. Theor. Comput. Sci.} 196:46--60.
  
  \bibitem{contract-automat-1} %5
\Aue{Flood, M.\,D., and O.\,R.~Goodenough.} 2015.
 Contract as automaton: The computational representation of financial
  agreements.
\textit{SSRN Electronic~J}.
doi: 10.2139/\linebreak ssrn.2538224.
  
  \bibitem{mining-smart-contract-1} %6
\Aue{Guth, F., V.~W$\ddot{\mbox{u}}$sthold, M.~Christakis, and P.~M$\ddot{\mbox{u}}$ller.}
2018.
 Specification mining for smart contracts with automatic abstraction
  tuning.  arXiv:1807.07822 [cs.Se]. 12~p.
Available at: {\sf https://arxiv.org/abs/1807.07822} (accessed
  January~17, 2019).
  
  \bibitem{me:Discr_Syst-1} %7
\Aue{Anashin, V.} 2012.
The non-{A}rchimedean theory of discrete systems.
\textit{Math. Comput. Sci.} 6:375--393.




\bibitem{AnKhr-1} %8
\Aue{Anashin, V., and A.~Khrennikov.}  2009.
\textit{Applied algebraic dynamics}.   Gruyter
  expositions in mathematics ser.
Berlin\,--\,New York: Walter~de~Gruyter GmbH \& Co. Vol.~49. 533~p.

\bibitem{DraKhrenVol-1} %9
\Aue{Dragovich, B., A.\,Yu.~Khrennikiv, S.\,V.~Kozyrev,  and
I.\,V.~Volovich.} 2009. 
On $p$-adic mathematical physics. 
\textit{$p$-Adic Numbers Ultrametric Analysis Appl}.
1:1--17. 
   
  \bibitem{Allouche-Shall-1} %10
\Aue{Allouche, J.-P., and J.~Shallit.} 2003. 
\textit{Automatic sequences. Theory, applications, generalizations}.
Cambridge: Cambridge University Press. 583~p.

\bibitem{Bouyer-Algebraic-Time-1} %11
\Aue{Bouyer, P., A.~Petit, and D.~Th$\acute{\mbox{e}}$rien.} 2003.
An algebraic approach to data languages and timed languages.
\textit{Inform.  Comput.} 182:137--162.


\bibitem{Knuth-1} %12
\Aue{Knuth, D.} 1997.
\textit{The art of computer programming. Vol.~2: Seminumerical
algorithms.} 3rd ed. Addison-Wesley. 791~p.
\end{thebibliography}

 }
 }

\end{multicols}

\vspace*{-6pt}

\hfill{\small\textit{Received February 9, 2019}}

%\pagebreak

%\vspace*{-18pt}





\Contrl

\noindent
\textbf{Anashin Vladimir S.} (b.\ 1951)~--- 
Doctor of Science in physics and mathematics,
professor, Faculty of Computational Mathematics and Cybernetics, M.\,V.~Lomonosov
Moscow State University,
1-52~Leninskie Gory, GSP-1,  Moscow 119991, Russian Federation;
 \mbox{anashin@iisi.msu.ru}
\label{end\stat}

\renewcommand{\bibname}{\protect\rm Литература}  