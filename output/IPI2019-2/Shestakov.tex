\newcommand{\sumk}{\sum\limits_{k=0}^{2^j-1}}
\newcommand{\mujk}{\mu_{j,k}}

\def\stat{shestakov}

\def\tit{СВОЙСТВА ВЕЙВЛЕТ-ОЦЕНОК СИГНАЛОВ,
РЕГИСТРИРУЕМЫХ В~СЛУЧАЙНЫЕ МОМЕНТЫ ВРЕМЕНИ$^*$}

\def\titkol{Свойства вейвлет-оценок сигналов,
регистрируемых в~случайные моменты времени}

\def\aut{О.\,В.~Шестаков$^1$}

\def\autkol{О.\,В.~Шестаков}

\titel{\tit}{\aut}{\autkol}{\titkol}

\index{Шестаков О.\,В.}
\index{Shestakov O.\,V.}


{\renewcommand{\thefootnote}{\fnsymbol{footnote}} \footnotetext[1]
{Работа выполнена при финансовой поддержке Российского научного фонда (проект 18-11-00155).}}


\renewcommand{\thefootnote}{\arabic{footnote}}
\footnotetext[1]{Московский государственный университет им.\ М.\,В.~Ломоносова, 
кафедра математической статистики факультета вычислительной математики и~кибернетики; 
Институт проб\-лем информатики Федерального исследовательского центра 
<<Информатика и~управ\-ле\-ние>> Российской академии наук, \mbox{oshestakov@cs.msu.su}}

%\vspace*{-2pt}




\Abst{Алгоритмы вейвлет-анализа в~сочетании с~процедурами пороговой обработки 
широко используются в~задачах непараметрической регрессии при оценивании 
функции сигнала по зашумленным данным. Преимущества данных методов заключаются 
в~их вычислительной эффективности и~возможности адаптации к~локальным особенностям 
оцениваемой функции. Анализ погрешностей методов пороговой обработки представляет 
собой важную практическую задачу, поскольку позволяет оценить качество как самих 
методов, так и~используемого оборудования. Иногда природа данных такова, что 
регистрация наблюдений производится в~случайные моменты времени. Если точки 
отсчета образуют вариационный ряд, построенный по выборке из равномерного 
распределения на отрезке регистрации данных, то использование обычных процедур 
пороговой обработки оказывается адекватным. 
В~данной работе проведен анализ оценки среднеквадратичного риска пороговой обработки 
и~показано, что при определенных условиях данная оценка оказывается сильно 
состоятельной и~асимптотически нормальной.}

\KW{вейвлеты; пороговая обработка; случайные отсчеты; оценка среднеквадратичного риска}

\DOI{10.14357/19922264190203}
  
%\vspace*{4pt}


\vskip 10pt plus 9pt minus 6pt

\thispagestyle{headings}

\begin{multicols}{2}

\label{st\stat}

\section{Введение}

Одним из популярных инструментов в~задачах непараметрической регрессии, 
к~которым относится подавление шума в~наблюдаемом сигнале, является метод 
пороговой обработки. Наблюдаемые данные преобразуются в~эмпирические 
вейв\-лет-ко\-эф\-фи\-ци\-ен\-ты, к~которым применяется функция пороговой обработки, 
после чего осуществляется обратное преобразование. Наиболее распространены функции 
жесткой и~мягкой пороговой обработки. Для модели отсчетов сигнала, заданных на 
равномерной неслучайной сетке и~содержащих белый гауссовский шум, эти методы 
хорошо изучены. Предложены способы вычисления пороговых значений,
 обеспечивающих <<почти>> оптимальный порядок среднеквадратичного риска 
 в~различных классах функций сигнала~\cite{DonJ94, DonJ95, MAJ98}. Также изучены 
 статистические свойства оценки среднеквадратичного риска. Показано, что при 
 определенных условиях она оказывается сильно состоятельной и~асимптотически 
 нормальной~\cite{Mar09, SH12, SH16-1, SH16-2}.

В некоторых приложениях нет возможности (или она сильно затруднена) 
регистрировать отсчеты сигнала через равные промежутки времени~\cite{CB98}. Иногда 
природа сигнала такова, что регистрация его отсчетов производится в~случайные 
моменты времени. В~работе~\cite{CB99} показано, что если точки отсчета образуют 
вариационный ряд, построенный по выборке из равномерного распределения на отрезке 
регистрации сигнала, то при использовании обычной пороговой обработки 
вейв\-лет-ко\-эф\-фи\-ци\-ен\-тов порядок среднеквадратичного риска остается 
с~точностью до логарифмического множителя равным оптимальному порядку в~классе 
функций, регулярных по Липшицу. В~данной работе доказывается, что статистические 
свойства оценки риска также не меняются при переходе от фиксированной равномерной 
сетки отсчетов к~случайной.

\vspace*{-5pt}

\section{Модель непараметрической регрессии со случайными точками отсчета}

\vspace*{-2pt}

Пусть функция сигнала $f(x)$ задана на отрезке $[0,1]$ и~равномерно 
регулярна по Липшицу с~некоторым показателем $\gamma\hm>0$ и~константой Липшица $L\hm>0$: 
$f\hm\in\mathrm{Lip}(\gamma,L)$. Предположим, что отсчеты~$f(x)$ 
регистрируются в~случайные моменты времени и~содержат аддитивный белый 
гауссовский шум, т.\,е.\ рассмотрим следующую модель данных:
$$
Y_i=f(x_i)+\eps_i\,,\enskip i=1,\ldots,N\enskip \left(N=2^J\right)\,,
$$
где $x_i$ независимы и~равномерно распределены на $[0,1]$, а~$\eps_i$ 
независимы между собой и~от~$x_i$ и~име-\linebreak\vspace*{-12pt}

\pagebreak

\noindent
ют нормальное распределение с~нулевым 
средним и~дисперсией~$\sigma^2$. Пусть $0\hm\leqslant x_{(1)}<\cdots< x_{(N)}\hm\leqslant 1$~--- 
вариационный ряд, построенный по выборке~$x_i$, $i\hm=1,\ldots,N$. Тогда, 
перенумеровав~$Y_i$ и~$\eps_i$, получаем модель:
\begin{equation}
\label{rand_sample}
Y_i=f\left(x_{(i)}\right)+\eps_i\,,\enskip i=1,\ldots,N\enskip \left(N=2^J\right)\,.
\end{equation}
Наблюдения состоят из пар $(x_{(1)},Y_1),\ldots$\linebreak $\ldots, (x_{(N)},Y_N)$, 
в~которых расстояния между отсчетами в~общем случае не равны. При этом
 $\e x_{(i)}\hm=i/(N+1)$. Наряду с~\eqref{rand_sample} рассмотрим выборку 
 с~равными расстояниями между отсчетами:
\begin{equation}
\label{eqspace_sample}
\left(\fr{1}{N+1},Z_1\right),\ldots, \left(\fr{N}{N+1},Z_N\right),
\end{equation}
где
\begin{equation*}
Z_i=f\left(\fr{i}{N+1}\right)+\eps_i\,,\enskip i=1,\ldots,N\,.
\end{equation*}
Для выборки~\eqref{eqspace_sample} Донохо и~Джонстон разработали методы 
пороговой обработки, эффективно по\-дав\-ля\-ющие шум и~обеспечивающие <<почти>> 
оптимальный порядок среднеквадратичного риска~\cite{DonJ94}. 
К~\eqref{eqspace_sample} применяется дискретное ортогональное вейв\-лет-пре\-обра\-зо\-ва\-ние 
и~получается набор эмпирических вейв\-лет-ко\-эф\-фи\-ци\-ентов:
$$
W_{j,k}=\mu_{j,k}+\xi_{j,k}\,,\enskip j=0,\ldots,J-1\,,\ k=0,\ldots,2^{j}-1\,,
$$
где $\xi_{j,k}$ независимы и~имеют такое же распределение, как и~$\eps_i$, 
а~$\mu_{j,k}$~--- коэффициенты дискретного вейв\-лет-пре\-обра\-зо\-ва\-ния выборки
$f\left({1}/({N\hm+1})\right),\ldots, f\left({N}/({N\hm+1})\right).$

Далее к~$W_{j,k}$ применяется функция жесткой пороговой обработки 
$\rho_{H}(y,T)\hm=x\mathbf{1}(\abs{y}\hm>T)$\linebreak или мягкой пороговой обработки 
$\rho_{S}(y,T)\hm=\mathrm{sgn}\,(x)\left(\abs{y}\hm-T\right)_{+}$ 
с~некоторым порогом~$T$ и~получаются оценки коэффициентов~$\widehat{W}_{j,k}$. 
После этого осуществляется обратное вейв\-лет-пре\-обра\-зо\-ва\-ние. 
Идея пороговой обработки заключается в~том, что вейв\-лет-пре\-обра\-зо\-ва\-ние 
обеспечивает <<разреженное>> пред\-став\-ле\-ние функции полезного сигнала, т.\,е.\
сигнал пред\-став\-ля\-ет\-ся относительно небольшим числом больших по модулю коэффициентов. 
При этом характеристики шума остаются неизменными. Следовательно, при обнулении малых 
по модулю коэффициентов полезный сигнал будет затронут несильно и~практически весь 
шум будет удален. Чтобы обеспечить <<разреженное>> пред\-став\-ле\-ние функции, равномерно 
регулярной по Липшицу с~показателем~$\gamma$, вейв\-лет-функ\-ция, 
участвующая в~дискретном вейв\-лет-пре\-обра\-зо\-ва\-нии, 
должна удовлетворять определенным условиям. В~част\-ности, она должна иметь~$M$~непрерывных 
производных ($M\hm\geqslant\gamma$), $M$ нулевых моментов 
и~достаточно быст\-ро убывать на бес\-ко\-неч\-ности~\cite{Mall99}. Далее предполагается, 
что все необходимые условия выполнены.

Если применить дискретное вейв\-лет-пре\-обра\-зо\-ва\-ние 
к~выборке~\eqref{rand_sample}, то получится набор эмпирических вейв\-лет-ко\-эф\-фи\-ци\-ентов
$$
V_{j,k}=\nu_{j,k}+\xi_{j,k}\,,\enskip j=0,\ldots,J-1\,,\ k=0,\ldots,2^{j}-1\,.
$$
Здесь $\nu_{j,k}$~--- коэффициенты дискретного вейв\-лет-пре\-обра\-зо\-ва\-ния 
выборки $f\left(x_{(1)}\right),\ldots, f\left(x_{(N)}\right).$

В общем случае~$V_{j,k}$ не равны~$W_{j,k}$, а~$\nu_{j,k}$ не равны~$\mu_{j,k}$. 
Однако к~$V_{j,k}$ можно применить ту же процедуру, что и~к~коэффициентам~$W_{j,k}$, 
и~получить оценки~$\widehat{V}_{j,k}$. В~сле\-ду\-ющих разделах обсуждаются 
свойства полученных оценок.

\section{Среднеквадратичный риск пороговой обработки}

Среднеквадратичный риск пороговой обработки для выборки со случайными точками 
отсчетов определим как
\begin{equation}
\label{MSE_Rand}
R_{\nu}(f,T)=\sum\limits_{j=0}^{J-1}\sumk\e\left(\widehat{V}_{j,k}-\mu_{j,k}\right)^2.
\end{equation}
Определим также среднеквадратичный риск для выборки 
с~равными расстояниями между отсчетами:
\begin{equation*}% \label{MSE_Eq}
R_{\mu}(f,T)=\sum\limits_{j=0}^{J-1}\sumk\e\left(\widehat{W}_{j,k}-\mu_{j,k}\right)^2.
\end{equation*}
Способ выбора порога является одной из основных проблем при пороговой 
обработке. При выборе так называемого универсального порога\linebreak 
$T_U\hm=\sigma\sqrt{2\ln 2^J}$ в~работе~\cite{CB99} доказано следующее утверждение 
о~порядке~$R_{\nu}(f,T)$.

\smallskip

\noindent
\textbf{Теорема~1.}\ 
\textit{Пусть $f\hm\in\mathrm{Lip}(\gamma,L)$ на отрезке $[0,1]$ с~$\gamma\hm\geqslant1/2$. 
Для модели}~\eqref{rand_sample} \textit{при использовании вейв\-лет-функ\-ции, 
удовлетворяющей перечисленным выше условиям, и~выборе порога~$T_U$ справедливо}
\begin{equation*}% \label{MSE_Rand_Rate}
R_{\nu}(f,T)\leqslant C\cdot 2^{{J}/({2\gamma+1})} J^{{2\gamma}/({2\gamma+1})}\,, %\notag
\end{equation*}
\textit{где $C$~--- некоторая положительная константа}.

Аналогичное утверждение справедливо и~для $R_{\mu}(f,T)$~\cite{DonJ95}. 
Таким образом, замена равноотстоящих точек отсчета на случайные не оказывает 
влияния на оценку порядка среднеквадратичного риска.

В~\cite{Jan01} для класса $\mathrm{Lip}(\gamma,L)$ предложено использовать 
порог 
$$
T_\gamma\hm=\sigma\sqrt{\fr{4\gamma}{2\gamma+1}\,\ln 2^J}\,.
$$
 При этом оценка порядка $R_{\mu}(f,T)$ останется такой же. Используя лемму~3 
 из~\cite{CB99} и~повторяя рас\-суж\-де\-ния, приводимые в~этой работе при доказательстве 
 теоремы~1, несложно убедиться, что теорема~1 также останется верна при замене 
 порога~$T_U$ на~$T_\gamma$. Порог~$T_\gamma$ является более адекватным для 
 класса~$\mathrm{Lip}(\gamma,L)$, чем порог~$T_U$, поскольку при том же порядке 
 среднеквадратичного риска он меньше <<сглаживает>> полезный сигнал~\cite{Jan01}.


\section{Оценка среднеквадратичного риска}

В выражении~(\ref{MSE_Rand}) присутствуют неизвестные величины <<чистых>> 
коэффициентов~$\mu_{j,k}$, поэтому вычислить значение риска~$R_{\nu}(f,T)$ 
на практике \mbox{нельзя}. Однако его
можно оценить непосредственно по наблюдаемым данным. По аналогии с~\cite{Mall99} 
в~качестве оценки риска будем использовать величину
\begin{equation}
\label{MSE_Estimate}
\widehat{R}_{\nu}(f,T)=\sum\limits_{j=0}^{J-1}\sumk F\left[V_{j,k},T\right],
\end{equation}
где 
\begin{multline*}
F\left[V_{j,k},T\right]={}\\
{}=
\begin{cases}
\left(V_{j,k}^2-\sigma^2\right)\mathbf{1}(\left\vert V_{j,k}\right\vert
\leqslant T)+\sigma^2\mathbf{1}\left(\left\vert V_{j,k}\right\vert >T\right)&\\[6pt]
&\hspace*{-65mm}\mbox{в~случае\ жесткой\ пороговой\ 
обработки}\,;\\[6pt]
\left(V_{j,k}^2-\sigma^2\right)\mathbf{1}\left(\left\vert V_{j,k}\right\vert \leqslant T\right)
+{}&\\[6pt]
&\hspace*{-42mm}{}+\left(\sigma^2+T^2\right)\mathbf{1}\left(\left\vert V_{j,k}\right\vert >T\right)\\[6pt] 
&\hspace*{-65mm}\mbox{в~случае~мягкой пороговой обработки}. 
\end{cases}\hspace*{-2pt}
\end{multline*}

В~ситуации с~неслучайной равномерной сеткой отсчетов при использовании мягкой 
пороговой обработки такая оценка риска является несмещенной. Если же используется 
жесткая пороговой обработка, то оценка содержит смещение, которое, однако, при
 достаточной гладкости функции сигнала не оказывает влияния на статистические свойства 
 оценки.

 Оценка риска~\eqref{MSE_Estimate} дает возможность получить представление 
 о~погрешности, с~которой оценивается функция сигнала, используя только наблюдаемые 
 данные.  Докажем утверждение об асимптотической нормальности оценки~\eqref{MSE_Estimate}, 
 которое, в~частности, позволяет строить асимптотические доверительные
  интервалы для теоретического риска~$R_{\nu}(f,T)$.
  
  \vspace*{2pt}
  
  \noindent
  \textbf{Теорема~2.}\ \textit{Пусть $f\hm\in\mathrm{Lip}(\gamma,L)$ 
  на отрезке $[0,1]$ с~$\gamma\hm>1/2$ и~вейв\-лет-функ\-ция 
  удовлетворяет пе\-ре\-чис\-лен\-ным выше условиям. Тогда при жесткой и~мягкой пороговой 
  обработке при выборе порога~$T_\gamma$}
\begin{equation*}
%\label{MSE_CLT}
\p\left(\fr{\widehat{R}_{\nu}(f,T_\gamma)-R_{\nu}(f,T_\gamma)}
{\sigma^2\sqrt{2^{J+1}}}<x\right)\!\to\! \Phi(x)\ \mbox{при}\ J\to\infty,
\end{equation*}
\textit{где $\Phi(x)$~--- функция распределения стандартного нормального закона}.

%\smallskip
\vspace*{2pt}

\noindent
Д\,о\,к\,а\,з\,а\,т\,е\,л\,ь\,с\,т\,в\,о\,.\ \  
Докажем теорему для случая жесткой пороговой обработки. 
В~случае мягкой пороговой обработки доказательство аналогично.

Наряду с~$\widehat{R}_{\nu}(f,T_\gamma)$ рассмотрим

\vspace*{2pt}

\noindent
\begin{equation*}
\widehat{R}_{\mu}\left(f,T_\gamma\right)=\sum\limits_{j=0}^{J-1}\sumk F\left[W_{j,k},T_\gamma\right]
\end{equation*}

\vspace*{-2pt}

\noindent
и запишем разность $\widehat{R}_{\nu}(f,T_\gamma)\hm-R_{\nu}(f,T_\gamma)$ в~виде
$$
\widehat{R}_{\nu}\left(f,T_\gamma\right)-R_{\nu}\left(f,T_\gamma\right)=
\widehat{R}_{\mu}\left(f,T_\gamma\right)
-R_{\mu}(f,T_\gamma)+\widetilde{R}\,,
$$
где
$$
\widetilde{R}=\widehat{R}_{\nu}\left(f,T_\gamma\right)-
\widehat{R}_{\mu}\left(f,T_\gamma\right)-\left(R_{\nu}(f,T_\gamma)-
R_{\mu}\left(f,T_\gamma\right)\right).\hspace*{-0.3272pt}
$$

\vspace*{-2pt}

\noindent
В~\cite{SH16-1} показано, что

\noindent
\begin{multline*}
\p\left(\fr{\widehat{R}_{\mu}\left(f,T_\gamma\right)-R_{\mu}\left(f,T_\gamma\right)}
{\sigma^2\sqrt{2^{J+1}}}<x\right)\to \Phi(x)\\[-2pt]
 \mbox{при}\ J\to\infty\,.
\end{multline*}

\vspace*{-2pt}

\noindent
Следовательно, для доказательства теоремы достаточно показать, что
$$
\fr{\widetilde{R}}{\sqrt{2^{J}}}\xrightarrow{\sf P} 0\ \mbox{при}\ J\to\infty\,.
$$
Если $\gamma>1/2$, то в~силу аналога теоремы~1 для порога~$T_\gamma$ 
и~аналогичного утверждения из работы~\cite{DonJ95}
$$
\fr{R_{\nu}\left(f,T_\gamma\right)-R_{\mu}\left(f,T_\gamma\right)}
{\sqrt{2^{J}}}\to 0\ \mbox{при}\ J\to\infty\,.
$$
Далее, поскольку $f\hm\in\mathrm{Lip}(\gamma,L)$, то выполняется
\begin{equation*}
%\label{Wavelet_CoeffDecacy}
\abs{\mujk}\leqslant\fr{A \cdot2^{J/2}}{2^{j\left(\gamma+1/2\right)}}
\end{equation*}
для некоторой константы~$A$, зависящей от~$L$~\cite{Mall99}. Пусть

\vspace*{-1pt}

\noindent
$$
j_0\approx\fr{J}{2\gamma+1}+\fr{1}{2\gamma+1}\log_2 J\,.
$$

\noindent
Представим $\widehat{R}_{\nu}(f,T_\gamma)\hm-\widehat{R}_{\mu}(f,T_\gamma)$ в~виде
$$
\widehat{R}_{\nu}\left(f,T_\gamma\right)-\widehat{R}_{\mu}\left(f,T_\gamma\right)=S_1+S_2\,,
$$
где
\begin{align*}
S_1&=\sum\limits_{j=0}^{j_0-1}\sumk \left(F\left[V_{j,k},T_\gamma\right]-
F\left[W_{j,k},T_\gamma\right]\right);\\
S_2&=\sum\limits_{j=j_0}^{J-1}\sumk \left(F\left[V_{j,k},T_\gamma\right]-
F\left[W_{j,k},T_\gamma\right]\right).
\end{align*}
Поскольку как в~случае жесткой, так и~в~случае мягкой пороговой обработки
\begin{equation}
\label{Term_Bound}
\left.
\begin{array}{rl}
\abs{F[V_{j,k},T_\gamma]}&\leqslant T_\gamma^2+\sigma^2;\\[6pt]
\abs{F[W_{j,k},T_\gamma]}&\leqslant T_\gamma^2+\sigma^2\ \mbox{п.~в.,}
\end{array}
\right\}
\end{equation}
то, если $\gamma>1/2$,
$$
\fr{S_1}{\sqrt{2^{J}}}\xrightarrow{\sf P} 0\ \mbox{при}\ J\to\infty\,.
$$
Далее
\begin{multline}
S_2=\sum\limits_{j=j_0}^{J-1}\sumk \left(F\left[V_{j,k},T_\gamma\right]-
F\left[W_{j,k},T_\gamma\right]\right)={}\\
{}=
\sum\limits_{j=j_0}^{J-1}\sumk\left(V_{j,k}^2-W_{j,k}^2\right)+{}\\
{}+\sum\limits_{j=j_0}^{J-1}\sumk\left(W_{j,k}^2-2\sigma^2\right)\mathbf{1}
\left(|V_{j,k}|\leqslant T_\gamma, \left\vert W_{j,k}\right\vert > T_\gamma\right)+{}\\
{}+\sum\limits_{j=j_0}^{J-1}\sumk\left(2\sigma^2-V_{j,k}^2\right)\mathbf{1}
\left(|V_{j,k}|> T_\gamma, \left\vert W_{j,k}\right\vert \leqslant T_\gamma\right)+{}\\
\!\!{}+\!\sum\limits_{j=j_0}^{J-1}\!\sumk\!\!\left(W_{j,k}^2-V_{j,k}^2\right)
\mathbf{1}\left(\left\vert V_{j,k}\right\vert \!>\! T_\gamma, \left\vert W_{j,k}\right\vert \!>\! T_\gamma\right)\!.\!\!\!
\label{Sums}
\end{multline}
Рассмотрим сумму $\sum\nolimits_{j=j_0}^{J-1}\sum\nolimits_{k=0}^{2^j-1}(V_{j,k}^2-W_{j,k}^2)$:
\begin{multline*}
\sum\limits_{j=j_0}^{J-1}\sumk\left(V_{j,k}^2-W_{j,k}^2\right)=
\sum\limits_{j=j_0}^{J-1}\sumk\left(\nu_{j,k}^2-\mu_{j,k}^2\right)+{}\\
{}+
2\sum\limits_{j=j_0}^{J-1}\sumk\xi_{j,k}\left(\nu_{j,k}-\mu_{j,k}\right).
\end{multline*}
Учитывая результаты работы~\cite{CB99}, можно показать, что условное распределение 
этой суммы при фиксированных~$x_i$ нормально с~математическим ожиданием
$\sum\nolimits_{i=1}^{N}\left(f_*^2(x_{(i)})\hm-f_*^2\left({i}/({N\hm+1})\right)\right)$
и~дисперсией
$4\sigma^2\sum\nolimits_{i=1}^{N}\left(f_*(x_{(i)})\hm-f_*\left({i}/({N\hm+1})\right)\right)^2,$
где $f_*(x)$~--- функция, вейв\-лет-раз\-ло\-же\-ние которой состоит только из 
коэффициентов при $j_0\hm\leqslant j\hm\leqslant J\hm-1$. Эта функция имеет ту же 
гладкость, что и~$f(x)$~\cite{Mall99}.

Учитывая, что $f\hm\in\mathrm{Lip}(\gamma,L)$, имеем:
\begin{multline*}
\abs{\e_x\sum\limits_{i=1}^{N}\left(f_*^2(x_{(i)})-
f_*^2\left(\fr{i}{N+1}\right)\right)}={}\\
{}=
\abs{N\left(\int\limits_0^1 f_*^2(x)\,dx-\fr{1}{N}\sum\limits_{i=1}^{N}
f_*^2\left(\fr{i}{N+1}\right)\right)}\leqslant{}\\
{}\leqslant C_1 N^{1-\min(2,\gamma)}= C_1 2^{J(1-\min(2,\gamma))}\,,
\end{multline*}
где $C_1$~--- некоторая константа. Кроме того, в~силу вида функции~$f_*(x)$ 
для некоторой константы $C_2>0$
\begin{multline*}
\fr{1}{N}\sum\limits_{i=1}^{N}f_*^2\left(\fr{i}{N+1}\right)\leqslant {}\\
{}\leqslant
C_2 N^{-{2\gamma}/({2\gamma+1})}=C_2\cdot 2^{-({2\gamma}/({2\gamma+1}))J}.
\end{multline*}
Следовательно,
\begin{multline*}
\e_x\sum\limits_{i=1}^{N}f_*^2(x_{(i)})\leqslant C_1\cdot 2^{J(1-\min(2,\gamma))}+{}\\
{}+
C_2\cdot 2^{\left(1-{2\gamma}/({2\gamma+1})\right)J}.
\end{multline*}

Также в~\cite{CB99} доказана оценка
\begin{multline}
\label{VAR_Decay}
\e_x\sum\limits_{i=1}^{N}\left(f_*(x_{(i)})-f_*\left(\fr{i}{N+1}\right)\right)^2
\leqslant {}\\
{}\leqslant C_3 N^{1-\min(1,\gamma)}=C_3 2^{J(1-\min(1,\gamma))},
\end{multline}
где $C_3$~-- некоторая константа. Следовательно, если $\gamma\hm>1/2$, то, 
применяя неравенство Маркова, получаем, что
\begin{align*}
\fr{1}{\sqrt{N}}\sum\limits_{i=1}^{N}\left(f_*^2(x_{(i)})-f_*^2\left(\fr{i}{N+1}\right)\right)&\xrightarrow{\sf P} 0\,;\\
\fr{1}{N}\sum\limits_{i=1}^{N}\left(f_*(x_{(i)})-f_*\left(\fr{i}{N+1}\right)\right)^2&\xrightarrow{\sf P} 0
\end{align*}
при $N\to\infty$. Таким образом,
$$
\fr{\sum\nolimits_{j=j_0}^{J-1}\sum\nolimits_{k=0}^{2^j-1}
\left(V_{j,k}^2-W_{j,k}^2\right)}{\sqrt{2^{J}}}\xrightarrow{\sf P} 0\ \mbox{при}\ 
J\to\infty\,.
$$

В оставшихся суммах в~\eqref{Sums} содержатся индикаторы, в~которых либо 
$|V_{j,k}|\hm> T_\gamma$, либо $|W_{j,k}|\hm> T_\gamma$, причем для всех слагаемых
 $$
 \abs{\mujk}\leqslant C_4 J^{-1/2}
 $$ 
 с~некоторой константой $C_4\hm>0$. 
 Повторяя рас\-суж\-де\-ния из работы~\cite{SH10} с~использованием~\eqref{VAR_Decay},
  можно показать, что эти суммы при делении на~$\sqrt{2^{J}}$ также сходятся к~нулю 
  по вероятности. Теорема дока-\linebreak зана.
  
  \smallskip

Помимо асимптотической нормальности оценка~\eqref{MSE_Estimate} также обладает
 свойством сильной состоятельности.
 
 \smallskip
 
 \noindent
 \textbf{Теорема~3.}\  \textit{Пусть 
 выполнены условия теоремы~$2$. Тогда при жесткой и~мягкой пороговой обработке 
 для любого $\lambda\hm>1/2$}
\begin{equation*}
\fr{\widehat{R}_{\nu}\left(f,T_\gamma\right)-R_{\nu}\left(f,T_\gamma\right)}
{2^{\lambda J}}\rightarrow 0 \ \mbox{п. в. при } J\rightarrow\infty\,.
\end{equation*}

Поскольку выполнено~\eqref{Term_Bound} и~при фиксированных~$x_i$ слагаемые 
в~\eqref{MSE_Estimate} условно независимы, доказательство этой теоремы 
аналогично доказательству соответствующего утверждения из работы~\cite{SH16-2}.

\smallskip

\noindent
\textbf{Замечание.}\ Теоремы~2 и~3 останутся справедливыми, если вместо 
порога~$T_\gamma$ выбрать универсальный порог~$T_U$.


 {\small\frenchspacing
 {%\baselineskip=10.8pt
 \addcontentsline{toc}{section}{References}
 \begin{thebibliography}{99}
\bibitem{DonJ94}
\Au{Donoho D., Johnstone~I.\,M.} Ideal spatial adaptation via wavelet shrinkage~// 
Biometrika, 1994. Vol.~81. No.\,3. P.~425--455.

\bibitem{DonJ95}
\Au{Donoho D., Johnstone I.\,M., Kerkyacharian~G., Picard~D.} 
Wavelet shrinkage: Asymptopia?~// J.~Roy. Stat. Soc.~B, 1995. Vol.~57. No.\,2. P.~301--369.

\bibitem{MAJ98}
\Au{Marron J.\,S., Adak~S., Johnstone I.\,M., Neumann~M.\,H., Patil~P.} 
Exact risk analysis of wavelet regression~// J.~Comput. Graph. Stat., 1998. Vol.~7. 
P.~278--309.

\bibitem{Mar09}
\Au{Маркин А.\,В.} Предельное распределение оценки риска при пороговой обработке 
вейв\-лет-ко\-эф\-фи\-ци\-ен\-тов~// Информатика и~её применения, 2009. Т.~3. Вып.~4. 
С.~57--63.

\bibitem{SH12}
\Au{Шестаков О.\,В.} 
Асимптотическая нормальность оценки риска пороговой обработки вейв\-лет-ко\-эф\-фи\-ци\-ен\-тов 
при выборе адаптивного порога~// Докл. Акад. наук, 2012. Т.~445. №\,5. С.~513--515.

\bibitem{SH16-1}
\Au{Шестаков О.\,В.} Вероят\-но\-ст\-но-ста\-ти\-сти\-че\-ские методы анализа 
и~обработки сигналов на основе вейв\-лет-ал\-го\-рит\-мов.~--- 
М.:~Аргамак-Медиа, 2016. 200~с.

\bibitem{SH16-2}
\Au{Shestakov O.\,V.} On the strong consistency of the adaptive risk estimator for
 wavelet thresholding~// J.~Math. Sci., 2016. Vol.~214. No.\,1. P.~115--118.

\bibitem{CB98}
\Au{Cai T., Brown~L.} Wavelet shrinkage for nonequispaced samples~// 
Ann. Stat., 1998. Vol.~26. No.\,5. P~1783--1799.

\bibitem{CB99}
\Au{Cai T., Brown~L.} Wavelet estimation for samples with random uniform design~// 
Stat. Probabil. Lett., 1999. Vol.~42. P.~313--321.

\bibitem{Mall99}
\Au{Mallat S.} A~wavelet tour of signal processing.~--- 
New York, NY, USA: Academic Press, 1999. 857~p.

\bibitem{Jan01}
\Au{Jansen M.} Noise reduction by wavelet thresholding.~--- 
Lecture notes in statistics ser.~--- New York, NY, USA: Springer Verlag, 2001. Vol.~161. 196~p.

\bibitem{SH10}
\Au{Шестаков О.\,В.} Аппроксимация распределения оценки риска 
пороговой обработки вейв\-лет-ко\-эф\-фи\-ци\-ен\-тов 
нормальным распределением при использовании выборочной дисперсии~// Информатика и~её 
применения, 2010. Т.~4. Вып.~4. С.~73--81.
 \end{thebibliography}

 }
 }

\end{multicols}

\vspace*{-3pt}

\hfill{\small\textit{Поступила в~редакцию 28.02.19}}

\vspace*{8pt}

%\pagebreak

%\newpage

%\vspace*{-29pt}

\hrule

\vspace*{2pt}

\hrule

%\vspace*{-2pt}

\def\tit{PROPERTIES OF WAVELET ESTIMATES OF~SIGNALS RECORDED AT~RANDOM TIME POINTS}


\def\titkol{Properties of wavelet estimates of~signals recorded at~random time points}

\def\aut{O.\,V.~Shestakov$^{1,2}$}

\def\autkol{O.\,V.~Shestakov}

\titel{\tit}{\aut}{\autkol}{\titkol}

\vspace*{-11pt}


\noindent
$^1$Department of Mathematical Statistics, Faculty of Computational 
Mathematics and Cybernetics, M.\,V.~Lo\-mo-\linebreak 
$\hphantom{^1}$no\-sov Moscow State University, 
1-52~Leninskiye Gory, GSP-1, Moscow 119991, Russian Federation


\noindent
$^2$Institute of Informatics Problems, Federal Research Center 
``Computer Science and Control'' of the Russian\linebreak
$\hphantom{^1}$Academy of Sciences, 44-2~Vavilov Str., 
Moscow 119333, Russian Federation

\def\leftfootline{\small{\textbf{\thepage}
\hfill INFORMATIKA I EE PRIMENENIYA~--- INFORMATICS AND
APPLICATIONS\ \ \ 2019\ \ \ volume~13\ \ \ issue\ 2}
}%
 \def\rightfootline{\small{INFORMATIKA I EE PRIMENENIYA~---
INFORMATICS AND APPLICATIONS\ \ \ 2019\ \ \ volume~13\ \ \ issue\ 2
\hfill \textbf{\thepage}}}

\vspace*{6pt}


\Abste{Wavelet analysis algorithms in combination with threshold 
processing procedures are widely used in nonparametric regression 
problems when estimating the signal function from noisy data. The 
advantages of these methods are their computational efficiency and 
the ability to adapt to the local features of the function being 
estimated. The error analysis of threshold processing methods is an 
important practical task, since it allows assessing the quality of both 
the methods themselves and the equipment used. Sometimes, the nature of 
the data is such that observations are recorded at random times. If the
 sampling points form a variation series constructed from a~sample 
 of a~uniform distribution over the data recording interval, then the use 
 of conventional threshold processing procedures is adequate. In this paper, 
 the author analyzes the estimate of the mean square risk of
 threshold processing
  and shows that under certain conditions, 
this estimate is strongly consistent and asymptotically normal.}


\KWE{wavelets; threshold processing; random samples;  mean square risk estimate}




 \DOI{10.14357/19922264190203}

%\vspace*{-14pt}

\Ack
\noindent
This research is supported by Russian Science Foundation (project No.\,18-11-00155).


%\vspace*{6pt}

  \begin{multicols}{2}

\renewcommand{\bibname}{\protect\rmfamily References}
%\renewcommand{\bibname}{\large\protect\rm References}

{\small\frenchspacing
 {%\baselineskip=10.8pt
 \addcontentsline{toc}{section}{References}
 \begin{thebibliography}{99}

\bibitem{1-shest-1}
\Aue{Donoho, D., and I.\,M.~Johnstone.} 
1994. Ideal spatial adaptation via wavelet shrinkage. \textit{Biometrika} 81(3):425--455.

\bibitem{2-shest-1}
\Aue{Donoho, D., I.\,M.~Johnstone, G.~Kerkyacharian, and D.~Picard.}
1995. Wavelet shrinkage: Asymptopia? \textit{J.~Roy. Stat. Soc.~B}
57(2):301--369.

\bibitem{3-shest-1}
\Aue{Marron, J.\,S., S.~Adak, I.\,M.~Johnstone, M.\,H.~Neumann, and P.~Patil.} 
1998. Exact risk analysis of wavelet regression. \textit{J.~Comput. Graph. Stat.} 7:278--309.

\bibitem{4-shest-1}
\Aue{Markin, A.\,V.} 2009. Predel'noe raspredelenie otsenki riska pri 
porogovoy obrabotke veyvlet-koeffitsientov 
[Limit distribution of risk estimate of wavelet coefficient thresholding]. 
\textit{Informatika i~ee Primeneniya~--- Inform. Appl.}  3(4):57--63.

\bibitem{5-shest-1}
\Aue{Shestakov, O.\,V.} 2012. Asymptotic normality of adaptive wavelet thresholding 
risk estimation. \textit{Dokl. Math.} 86(1):556--558.

\bibitem{6-shest-1}
\Aue{Shestakov, O.\,V.} 2016. \textit{Veroyatnostno-statisticheskie metody analiza 
i~obrabotki signalov na osnove veyvlet-algoritmov} [Probabilistic-statistical 
methods of signal analysis and processing based on wavelet algorithms].
 Moscow: Argamak-Media Publs. 200~p.

\bibitem{7-shest-1}
\Aue{Shestakov, O.\,V.}  2016. On the strong consistency of the adaptive risk 
estimator for wavelet thresholding. \textit{J.~Math. Sci.} 214(1):115--118.

\bibitem{8-shest-1}
\Aue{Cai, T., and L. Brown.} 
1998. Wavelet shrinkage for nonequispaced samples. 
\textit{Ann. Stat.} 26(5):1783--1799.

\bibitem{9-shest-1}
\Aue{Cai, T., and L.~Brown.} 1999. 
Wavelet estimation for samples with random uniform design.
\textit{Stat. Probabil. Lett.} 42:313--321.

\bibitem{10-shest-1}
\Aue{Mallat, S.} 1999. 
\textit{A~wavelet tour of signal processing.} New York, NY: Academic Press. 857~p.

\bibitem{11-shest-1}
\Aue{Jansen, M.} 2001. \textit{Noise reduction by wavelet thresholding.} 
Lecture notes in statistics ser.
New York, NY: Springer Verlag.  Vol.~161. 196~p.

\bibitem{12-shest-1}
\Aue{Shestakov, O.\,V.} 2010. 
Approksimatsiya raspredeleniya otsenki riska porogovoy obrabotki 
veyvlet-koeffitsientov normal'nym raspredeleniem pri ispol'zovanii 
vy\-bo\-roch\-noy dispersii
[Normal approximation for distribution of risk estimate for wavelet 
coefficients thresholding when using sample variance]. 
\textit{Informatika i~ee Pri\-me\-ne\-niya~--- Inform. Appl.}  4(4):73--81.
\end{thebibliography}

 }
 }

\end{multicols}

\vspace*{-6pt}

\hfill{\small\textit{Received February 28, 2019}}

%\pagebreak

%\vspace*{-18pt}

\Contrl

\noindent
\textbf{Shestakov Oleg V.} (b.\ 1976)~--- 
Doctor of Science in physics and mathematics, professor, 
Department of Mathematical Statistics, Faculty of Computational Mathematics and 
Cybernetics, M.\,V.~Lomonosov Moscow State University, 1-52~Leninskiye Gory, 
GSP-1, Moscow 119991, Russian Federation; senior scientist, Institute of 
Informatics Problems, Federal Research Center ``Computer Science and Control'' 
of the Russian Academy of Sciences, 44-2~Vavilov Str., Moscow 119333, Russian 
Federation; \mbox{oshestakov@cs.msu.su}
\label{end\stat}

\renewcommand{\bibname}{\protect\rm Литература}  