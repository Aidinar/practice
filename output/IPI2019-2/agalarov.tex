\def\stat{agalarov}

\def\tit{ДОКАЗАТЕЛЬСТВО УНИМОДАЛЬНОСТИ ЦЕЛЕВОЙ ФУНКЦИИ В~ЗАДАЧЕ 
ПОРОГОВОГО УПРАВЛЕНИЯ НАГРУЗКОЙ НА~СЕРВЕР$^*$}

\def\titkol{Доказательство унимодальности целевой функции в~задаче 
порогового управления нагрузкой на~сервер}

\def\aut{Я.\,М.~Агаларов$^1$, М.\,Г.~Коновалов$^2$}

\def\autkol{Я.\,М.~Агаларов, М.\,Г.~Коновалов}

\titel{\tit}{\aut}{\autkol}{\titkol}

\index{Агаларов Я.\,М.}
\index{Коновалов М.\,Г.}
\index{Agalarov Ya.\,M.}
\index{Konovalov M.\,G.}


{\renewcommand{\thefootnote}{\fnsymbol{footnote}} \footnotetext[1]
{Работа выполнена при частичной финансовой поддержке РФФИ 
(проекты 18-07-00692 и~19-07-00739).}}


\renewcommand{\thefootnote}{\arabic{footnote}}
\footnotetext[1]{Институт проб\-лем информатики Федерального исследовательского центра 
<<Информатика и~управ\-ле\-ние>> Российской академии наук, \mbox{agglar@yandex.ru}}
\footnotetext[2]{Институт проб\-лем информатики Федерального исследовательского центра 
<<Информатика и~управ\-ле\-ние>> Российской академии наук, \mbox{mkonovalov@ipiran.ru}}

\vspace*{-4pt}


\Abst{Рассматривается задача ограничения нагрузки в~системе $M/M/N/\infty$ с~помощью 
простой пороговой стратегии. Процесс обслуживания характеризуется наличием дедлайна 
для времени выполнения заданий. Другая особенность постановки задачи заключается 
в~системе доходов и~штрафов, которые получает система в~зависимости от качества 
обслуживания. Качество управления оценивается в~терминах предельного среднего дохода, 
а~оптимальным считается значение порога, которое максимизирует эту величину. 
Нахождение оптимального порога существенно облегчается, когда целевая функция имеет 
единственный максимум. Результаты экспериментов свидетельствуют об унимодальности 
целевой функции для широкого класса входных потоков. Однако строгое доказательство 
этого факта отсутствует, и~в~статье этот пробел восполняется для пуассоновской нагрузки. 
При доказательстве используются результаты теории марковских цепей и~теории массового 
обслуживания.}

\KW{цепи Маркова; система $M/M/N/\infty$; ограничение нагрузки; пороговое управление; 
дедлайн}

\DOI{10.14357/19922264190201}
  
%\vspace*{4pt}


\vskip 10pt plus 9pt minus 6pt

\thispagestyle{headings}

\begin{multicols}{2}

\label{st\stat}

  \section{Введение}
  
  Эта заметка продолжает и~дополняет работы~[1, 2], в~которых на простых 
моделях сервера с~фиксированным числом мест обслуживания и~потенциально 
бесконечной очередью изучались стратегии ограничения входной нагрузки. 
Основной результат~\cite{1-ag} за\-клю\-чал\-ся в~том, что в~многоканальной 
сис\-те\-ме с~дедлайном и~экспоненциальным временем обслуживания 
оптимальная в~смысле предельного среднего дохода стратегия ограничения 
нагрузки является прос\-той пороговой стратегией и~при этом утверж\-де\-ние 
справедливо для достаточно широкого класса входных потоков. Отсюда 
естественным образом возникает проблема отыскания оптимального значения 
порога, которая в~\cite{1-ag} изучалась с~по\-мощью имитационного 
моделирования. Все результаты чис\-лен\-ных экспериментов показали, что 
целевая функция вогнута и~имеет единственный максимум по значению порога. 
В~\cite{2-ag} доказана унимодальность целевой функции для случая 
одноканальной сис\-те\-мы без дед\-лай\-на. В~данной \mbox{статье} приведено формальное 
доказательство аналогичного результата для многоканальной сис\-те\-мы 
с~пуассоновским входным потоком.
  
  Стратегии порогового типа возникают во многих задачах управления 
  и~имеют большое значение, в~част\-ности, для решения проблемы контроля над 
перегрузками в~объектах, трактуемых как системы обслуживания. 
Привлекательность таких стратегий заключается в~сочетании простоты 
реализации с~удовлетворительным качеством, которое они обеспечивают даже 
в~сравнении с~более изощренными алгоритмами. Материал по тематике 
порогового управления весьма обширен и~труднообозрим. В~части, 
касающейся контроля над перегрузками, полезно опираться на 
фундаментальную обзорную статью~[3]. Применительно к~содержанию 
конкретно данной работы отметим статью~[4].
  
  В разд.~2 приводятся строгое описание постановки задачи и~формулировка 
результата об унимодальности предельного среднего дохода в~многоканальной 
системе с~пороговым ограничением нагрузки, пуассоновским входом, 
дедлайном и~сис\-те\-мой доходов и~штрафов за качество обслуживания. Раздел~3 
содержит доказательство тео\-ремы.

\vspace*{-9pt}
  
  \section{Описание модели и~результат}
  
  \vspace*{-2pt}
  
  Рассматривается сервер, имеющий $N\hm<\infty$ обслуживающих мест 
(процессоров) для выполнения заданий, поступающих извне в~виде 
рекуррентного потока. В~каждый момент появления очередной заявки должно 
быть принято одно из двух возможных решений: задание может быть либо 
оставлено в~системе для последующей обработки, либо отвергнуто. Принятое 
на обслуживание задание сразу помещается на свободный процессор, если 
таковой имеется, либо становится в~очередь, которая предполагается 
неограниченной. Продвижение в~очереди осуществляется по принципу 
<<первый при\-шел\,--\,пер\-вый обслуживается>>. Время выполнения задания на 
любом процессоре является случайным и~имеющим экспоненциальное 
распределение с~параметром~$\mu$. Общее время пребывания задания 
в~сис\-те\-ме не должно превышать заданный предельный срок~$\delta$ (так 
называемый дедлайн).
  
  Процесс выполнения заданий сопровождается получением доходов 
и~штрафов по следующей схеме. Если поступившее задание принято и~если 
время его выполнения не превысило дедлайн, то система получает доход, 
равный $C_0\hm>0$. Если поступившее задание принято, но время его 
выполнения превысило величину~$\delta$, то система платит штраф 
$C_1\hm>0$. Наконец, если поступившая заявка отклонена, то штраф 
составляет $C_2\hm>0$.
  
  Обозначим через~$\tau_n$, $n\hm=1,2,\ldots$, последовательные моменты 
поступления заявок, а через~$v_n$~--- общее число заданий, находящихся на 
сервере\linebreak в~момент~$\tau_n$. Предположим, что управление по\-сту\-па\-ющи\-ми 
заявками осуществляется согласно (прос\-той) пороговой стратегии с~порогом 
$h\hm\geq N$. Таким образом, задание, поступившее в~момент~$\tau_n$, 
принимается на обслуживание, если $v_n\hm<h$, и~отклоняется, если 
$v_n\hm=h$.
  
  Пусть $g_n^{(h)}$~--- средний доход, связанный с~заявкой, поступившей 
в~момент~$\tau_n$. Определим предельный средний доход, усредненный по 
числу поступивших заявок, как 
  $$
  w(h) =\lim\limits_{T\to\infty} \fr{1}{T}\sum\limits^T_{n=1} g_n^{(h)}\,.
  $$
  
  \noindent
  \textbf{Теорема.}\ \textit{Если входной поток пуассоновский, то 
функция~$w(h)$ имеет единственный максимум по~$h$}.
  
  
  \section{Доказательство теоремы}
  
  При сделанных предположениях относительно процесса обслуживания 
и~рекуррентном входящем потоке последовательность~$v_n$ образует 
марковскую цепь с~множеством состояний $\{0, \ldots , h\}$ и~переходными 
вероятностями
  $$
  p_{ij}^{(h)} =\begin{cases}
  q_{ij} &\mbox{при } j\leq i+1\leq N-1\,;\\[3pt]
  r_{i+1-j} &\mbox{при } N\leq j\leq i+1\leq h\,;\\[3pt]
  s_{ij} &\mbox{при } j<N\leq i+1\leq h\,;\\[3pt]
  0 &\mbox{при } j>i+1\,.
  \end{cases}
  $$
Здесь
\begin{align*}
q_{ij} &= \int\limits_0^\infty \begin{pmatrix}
i+1\\ j\end{pmatrix} 
\left[ 1-e^{-\mu t}\right]^{i+1-j} e^{-\mu t j} \,dF(t)\,;\\
r_{i+1-j}&=\int\limits_0^\infty \fr{(N\mu t)^{i+1-j}}{(i+1-j)!}\,e^{-N\mu t}\, dF(t)\,;\\
s_{ij}&= \int\limits_0^\infty \begin{pmatrix}
N\\ j\end{pmatrix}
e^{-\mu t j} \times{}\\
&\hspace*{-24pt}{}\times \left[ \int\limits^t_0 \fr{(N\mu y)^{i-N}} {(i-N)!}\left( 
e^{-\mu y} -e^{-\mu t}\right)^{N-j} N \mu\,dy\right]dF(t)\,,
\end{align*}
где $F(t)$~--- функция распределения между последовательными поступлениями 
заявок.

  Стандартные рассуждения показывают, что стационарные вероятности 
состояний $\{ \pi_i^{(h)}, \ 0\hm\leq i\hm\leq h\}$ рассматриваемой марковской 
цепи удовлетворяют следующей системе уравнений равновесия:
  \begin{align*}
  \pi_0^{(h)} &=\sum\limits^h_{i=0} \pi_i^{(h)} p_{i0}^{(h)}\,;\\ 
  \pi_j^{(h)}&=\sum\limits^h_{i=j-1} \pi_i^{(h)} p_{ij}^{(h)}\,,\enskip
j=1,\ldots , h-1\,;\\
  \pi_h^{(h)}& =\pi^{(h)}_{h-1} p^{(h)}_{h-1,h} +\pi_h^{(h)} p^{(h)}_{h-
1,h}\,;\\ 
  \sum\limits^n_{i=0}\pi_i^{(h)}&=1\,,\enskip \pi_i^{(h)} \geq 0\,,\enskip i=0,\ldots , 
h\,.
  \end{align*}
  
  Решив данную систему, получим для $j\hm=0,\ldots$\linebreak $\ldots , h\hm-1$ следующие 
соотношения:
  \begin{equation}
  \left.
  \begin{array}{l}
  \pi_h^{(h)}=\left( 1+\sum\limits_{i=0}^{h-1} D_i^{(h)}\right)^{-1}\,;\\
  \pi_j^{(h)}=D_j^{(h)} \pi_h^{(h)}=D_j^{(h)} \left( 1+\sum\limits_{i=0}^{h-1} 
D_i^{(h)}\right)^{-1}\,.
  \end{array}
  \right\}
  \label{e1-ag}
  \end{equation}
Здесь
  \begin{gather*}
  D_h^{(h)}=1\,;\enskip D^{(h)}_{h-1}=\fr{1-r_0}{r_0}\,;\\
   \hspace*{-25mm}D_{j-1}^{(h)}=\fr{1}{r_0}
   \left( \vphantom{\sum\limits^{h-1}_{i=j+1} }
   D_j^{(h)}(1-r_1)-{}\right.\\
\left.   {}-\sum\limits^{h-1}_{i=j+1} D_i^{(h)} r_{i+1-j}-
r_{h-j}\right)\,,\enskip  N\leq j\leq h-1\,;\\
  \hspace*{-10mm}D^{(h)}_{N-2}= \fr{1}{q_{N-2,N-1}}
\left( \vphantom{\sum\limits^{h-1}_{i=N} }
D^{(h)}_{N-1} \left(1-r_0\right)-{}\right.\\
\hspace*{28mm}\left.{}-\sum\limits^{h-1}_{i=N} 
D_i^{(h)} s_{i,N-1}-s_{h,N-1}\right)\,;
\end{gather*}
\begin{multline*}
  D^{(h)}_{j-1}= \fr{1}{q_{j-1,j}}
  \left(D_j^{(h)}(1-q_{jj}) -\sum\limits^{N-1}_{i=j+1} D_i^{(h)} 
q_{ij} -{}\right.\\
\left.{}-\sum\limits^{h-1}_{i=N} D_i^{(h)} s_{ij}-s_{hj}
\vphantom{\sum\limits^{N-1}_{i=j+1}}
\right) \,,\enskip
1\leq j< N-1\,.
\end{multline*}
  
  Среднее значение дохода, получаемого системой в~состоянии~$i$, равно
  $$
  R_i^{(h)}=\begin{cases}
  C_0\Gamma_{\mu,1}(\delta) -C_1\left[ 1-\Gamma_{\mu,1}(\delta)\right]\,, 
&\\
&\hspace*{-30mm}\mbox{если } 0\leq i<N\,;\\[3pt]
  C_0\Gamma_{N\mu, i+1}(\delta) -C_1\left[ 1-\Gamma_{N\mu, 
i+1}(\delta)\right)\,, &\\
&\hspace*{-30mm}\mbox{если } N\leq i<h\,;\\[3pt]
  -C_2\,, &\hspace*{-30mm}\mbox{если } i=h\,,
  \end{cases}
  $$
где $\Gamma_{a,b}(\cdot)$~--- функция гам\-ма-рас\-пре\-де\-ле\~ния 
с~параметрами~$a$ и~$b$.

  Представим последнюю формулу в~виде:
  \begin{equation}
  R_i^{(h)}=\begin{cases}
  \left( C_0+C_1\right) \Gamma_{\mu,1}(\delta)-C_1\,, &\\
  &\hspace*{-25mm}\mbox{если } 0\leq 
i<N\,;\\[3pt]
  \left( C_0+C_1\right)\Gamma_{N\mu, i+1}(\delta) -C_1\,, &\\
  &\hspace*{-25mm} \mbox{если } N\leq 
i <h\,;\\[3pt]
  -C_2\,, &\hspace*{-25mm}\mbox{если } i=h\,.
  \end{cases}
  \label{e2-ag}
  \end{equation}
     Отсюда при $0\hm\leq i\hm\leq h\hm-2$ получим, что
  \begin{equation}
  R^{(h)}_{i+1}=\begin{cases} 
  R_i^{(h)}\,, &  \hspace*{-35mm}\mbox{если } 0\leq i<N\,;\\[3pt]
  R_i^{(h)}-\left( C_0+C_1\right) \fr{(N\mu\delta)^{i+1}}{(i+1)!}\,e^{-N\mu\delta}\,, &\\
  & \hspace*{-35mm}\mbox{если } N\leq i\leq h-2\,.
  \end{cases}
  \label{e3-ag}
  \end{equation}


  Как следует из~(\ref{e1-ag}), справедливы также формулы:
  \begin{equation}
  \pi_{j+1}^{(h+1)}=A_{h+1} \pi_j^{(h)}\,,\enskip j=N-1,\ldots , h\,.
  \label{e4-ag}
  \end{equation}
Здесь

\noindent
\begin{equation*}
A_{h+1}=\fr{1-B^{(h+1)}_{N-1}}{1-B^{(h)}_{N-2}}<1\,,
\end{equation*}
где

\noindent
\begin{align*} 
B_{N-1}^{(h+1)}& =\sum\limits_{i=0}^{N-1} \pi_i^{(h+1)}\,;
\\
B^{(h)}_{N-2}&=\begin{cases}
0\,, & \mbox{если } N=1\,;\\[3pt]
\displaystyle\sum\limits_{i=0}^{N-2} \pi_i^{(h)}\,,& \mbox{если } N\geq 2\,.
\end{cases}
\end{align*}
  
  Обозначив 
  $$
  \Delta_{i+1}^{(h+1)}= R_{i+1}^{(h+1)}- R_i^{(h+1)}\,,\enskip 
i=1,\ldots , h,
$$
 и~применив формулы~(\ref{e4-ag}), находим:
 
 \columnbreak
 
\noindent
  \begin{multline*}
  \!\!w(h)-w(h+1)=\sum\limits^h_{i=0} \!\pi_i^{(h)} R_i^{(h)}-\sum\limits_{i=0}^{h+1} \!
\pi_i^{(h+1)}R_i^{(h+1)}={}\\[-1pt]
  {}=\left[ \left( C_0+C_1\right) \Gamma_{\mu,1}(\delta) -C_1\right] 
\sum\limits_{i=0}^{N-2} \!\pi_i^{(h)} +\!\!\sum\limits^h_{i=N-1} \!\!\!\!\pi_i^{(h)}R_i^{(h)}-
{}\\[-1pt]
  {}-\left[ \left( C_0+C_1\right) \Gamma_{\mu,1}(\delta) -C_1\right] 
\sum\limits_{i=0}^{N-1} \!\pi_i^{(h+1)} +{}\\[-1pt]
{}+\sum\limits^{h+1}_{i=N} \!\pi_i^{(h+1)} 
R_i^{(h+1)}={}\\[-1pt]
  {}= \left[ \left( C_0+C_1\right)\Gamma_{\mu,1}(\delta) -C_1\right] \left( 
B^{(h)}_{N-2} -B^{(h+1)}_{N-1}\right) +{}\\[-1pt]
{}+\sum\limits^h_{i=N-1}\!\! \pi_i^{(h)} 
R_i^{(h)}-{}\\[-1pt]
  {}- A_{h+1}\!\left[ \sum\limits^{h-1}_{i=N-1}\hspace*{-5.5pt} \pi_i^{(h)} 
  \!\left( R_i^{(h+1)} 
+\Delta^{(h+1)}_{i+1} \right)\! +\pi_h^{(h)}R^{(h+1)}_{h+1}\right].\hspace*{-6.0207pt}
  \end{multline*}
  
  \vspace*{-5pt}
  
  Из~(\ref{e2-ag}) следует, что $R_i^{(h)}\hm= R_i^{(h+1)}$ при $i\hm=0, \ldots 
,h\hm-1$ и~$R_h^{(h)}\hm= R^{(h+1)}_{h+1}$. Продолжая преобразования  
с~по\-мощью~(\ref{e3-ag}), получим 

\vspace*{-10pt}

\noindent
  \begin{multline*}
  w(h)-w(h+1)={}\\
  \!{}=\left[ \left( C_0+C_1\right) \Gamma_{\mu,1}(\delta) -C_1\right] 
\left( Q^{(h)}_{N-2}-Q_{N-1}^{(h+1)}\right)+{}\\
  {}+\left( 1-A_{h+1}\right) \!\!\sum\limits^h_{i=N-1}\!\!\! \pi_i^{(h)} R_i^{(h)} -A_{h+1}\!\! 
\sum\limits^{h-1}_{i=N-1} \!\!\!\pi_i^{(h)} \Delta_{i+1}^{(h+1)}={}\\
  {}=-\left[ \left( C_0+C_1\right) \Gamma_{\mu,1}(\delta) -C_1\right] 
\fr{B^{(h+1)}_{N-1}-B^{(h)}_{N-2}}{1-B^{(h)}_{N-2}}+{}\\
{}+
  \left[ \vphantom{\sum\limits^h_{i=N-1}}
   \left[\left( C_0+C_1\right) \Gamma_{\mu,1}(\delta) -C_1\right] 
  B^{(h)}_{N-2}+{}\right.\\
\left.  {}+\sum\limits^h_{i=N-1}\!\!\pi_i^{(h)} R_i^{(h)}\right]
\fr{B^{(h+1)}_{N-1}- B^{(h)}_{N-2}} {1-B^{(h)}_{N-2}} -{}\\
{}-
  A_{h-1}\sum\limits^{h-1}_{i=N-1}\pi_i^{(h)} \Delta^{(h+1)}_{i+1}={}\\
  {}=\left( 1-A_{h+1}\right) \left[ 
  \vphantom{\sum\limits^{h-1}_{i=N-1} \pi_i^{(h)} \fr{(N\mu\delta)^{i+1}}{(i+1)!}}
  w(h)-\left( C_0+C_1\right) 
\Gamma_{\mu,1}(\delta)+C_1+{}\right.\\
 \hspace*{-1.73294pt} \left.{}+ \left(C_0+C_1\right) \fr{A_{h+1}}{1-A_{h+1}} 
  \sum\limits^{h-1}_{i=N-1} \!\!\!\!\pi_i^{(h)} \fr{(N\mu\delta)^{i+1}}{(i+1)!}\, 
e^{-N\mu\delta}\right]\!={}
\\
  {}=
  \left( 1-A_{h+1}\right) \left[
  \vphantom{\sum\limits^h_{i=N} \pi_i^{(h+1)} \fr{(N\mu\delta)^i}{i!}}
   w(h)-\left( C_0+C_1\right)\Gamma_{\mu,1}(\delta)+C_1+{}\right.\\
  \left.{}+ \left(C_0+C_1\right) \fr{e^{-N\mu\delta}}{1-
A_{h+1}}\sum\limits^h_{i=N} \pi_i^{(h+1)} \fr{(N\mu\delta)^i}{i!}\right]\,.
  \end{multline*}
  
  Таким образом, рассматриваемое равенство приводится к~виду:
 %  \pagebreak
   \begin{equation}
  w(h)-w(h+1)=\left(1-A_{h+1}\right)[w(h)-G(h)]\,,
  \label{e5-ag}
  \end{equation}
где 

\noindent
\begin{multline*}
G(h)=\left( C_0+C_1\right) \Gamma_{\mu,1}(\delta) -C_2-{}\\
{}-\left( C_0+C_1\right) 
\fr{e^{-N\mu\delta}}{1-A_{h+1}}\sum\limits^h_{i=N} 
\pi_i^{(h+1)}\fr{(N\mu\delta)^i}{i!}\,.
\end{multline*}
  
  Докажем, что функция $G(h)$ убывает с~возрастанием переменной $h\hm\geq 
N$. Для этого достаточно показать, что функция $\varphi(h)\hm= (1/(1\hm-A_{h+1}) 
\sum\nolimits^h_{i=N} \pi_i^{(h+1)} (N \mu \delta)^i/(i!)$~---  возрастающая по 
$h\hm\geq N$. Начиная с~этого места, воспользуемся тем, что по условию 
доказываемого утверждения 
  $$
  F(t)=1- e^{-\lambda t},\enskip  t\geq 0\,.
  $$
  
  Применим известные формулы для величин~$\pi_i^{(h)}$, которые имеют 
место для систем массового обслуживания типа $M/M/N/h$~\cite[с.~123]{5-ag}:
  $$
  \pi_i^{(h)} =\begin{cases}
  \pi_0^{(h)}\fr{\rho^i}{i!}\,, & \mbox{если } 0\leq i\leq N-1\,;\\[3pt]
  \pi_0^{(h)}\fr{\rho^N}{n!}\left( \fr{\rho}{N}\right)^{i-N}\,, & \mbox{если } N\leq i\leq h\,,
  \end{cases}
  $$
где

\vspace*{-2pt}

\noindent
$$
\rho=\fr{\lambda}{\mu}\,;\enskip \pi_0^{(h)}= \left( \sum\limits_{j=0}^{N-1} 
\fr{\rho^j}{j!} +\fr{\rho^N}{N!}\sum\limits_{j=0}^{h-N}\left( 
\fr{\rho}{N}\right)^j\right)^{-1}\,.
$$

\vspace*{-2pt}

\noindent
    Имеем:
    
    \noindent
  $$
  \varphi(h)=\psi(h)\fr{\rho^N}{N!}\sum\limits^h_{i=N} \left( 
\fr{\rho}{N}\right)^{i-N}\fr{(N\mu\delta)^i}{i!}\,,
  $$
где $\psi(h)=\pi_0^{(h+1)}/(1-A_{h+1})$. Поскольку сумма в~последнем 
выражении~--- возрастающая функция по~$h$, то достаточно показать, что 
функция $\psi(h)$ не зависит от~$h$. Но этот факт следует из того, что
\begin{multline*}
\psi(h+1)-\psi(h+2)=\sum\limits_{j=0}^{h+1-N} \left(\fr{\rho}{N}\right)^i\Bigg/ 
\left(\left[ 
\vphantom{\left(\fr{\rho}{N}\right)^{h+2-N}}
1-\right.\right.\\
\left.{}-\left(\fr{\rho}{N}\right)^{h+2-N} 
\right] \sum\limits_{j=0}^{N-1}\fr{\rho^j}{j!} +{}\\
\left.{}+
\left(\fr{\rho^N}{N!}\right) 
\sum\limits_{j=0}^{h+1-N} \left(\fr{\rho}{N}\right)^j\right) -
 {\sum\limits_{j=0}^{h+2-N} 
\left(\fr{\rho}{N}\right)^j }\Bigg/
\left( \left[ 
\vphantom{\left(\fr{\rho}{N}\right)^{h+2-N}}
1-{}\right.\right.\hspace*{-0.33632pt}\\
\!\left.\left.{}-
\left(\fr{\rho}{N}\right)^{h+3-N} \right] \sum\limits_{j=0}^{N-1}
\fr{\rho^j}{j!} +\fr{\rho^N}{N!} 
\sum\limits_{j=0}^{h+2-N} \!\!\left(\fr{\rho}{N}\right)^j \right)=0\,.\hspace*{-4.11052pt}
\end{multline*}
  
  Следовательно, функция $\varphi(h)$ возрастает, а~функ\-ция~$G(h)$, 
соответственно, убывает по~$h$, если $h\hm\geq N$.
  
Далее доказательство практически идентично рассуждениям, проведенным 
в~теореме~1 из~\cite{2-ag}, в~которых фигурирует аналог функции~$w(h)$, 
представленной в~виде~(\ref{e5-ag}), где $G(h)$~--- невозрастающая функция, 
$0\hm< A_{h+1}\hm<1$. Эти рассуждения приводят к~утверждению теоремы об 
унимодальности~$w(h)$. 


  \section{Заключение}
  
  В работе приведено доказательство существования и~единственности 
оптимального значения порога в~задаче управления нагрузкой на сервер 
с~несколькими параллельными местами обслуживания и~бесконечной 
очередью. Модель предполагает пуассоновскую нагрузку, экспоненциальное 
время обслуживания, возможность регулировать чис\-ло заданий в~системе, 
а~также доходы или штрафы в~зависимости от качества обслуживания. Целевой 
функцией выступает предельный средний доход.
  
  Результат статьи частично восполняет пробел в~теории, где свойства целевой 
функции в~задачах порогового управления изучены недостаточно. Анализ 
доказательства показывает, что б$\acute{\mbox{о}}$льшая часть рассуждений справедлива также 
для произвольного рекуррентного входного потока. Это дает основание 
продолжить работу над получением аналогичного, более общего результата для 
системы $G/M/N/\infty$.
  
 {\small\frenchspacing
 {%\baselineskip=10.8pt
 \addcontentsline{toc}{section}{References}
 \begin{thebibliography}{9}
  \bibitem{1-ag}
  \Au{Коновалов М.\,Г.} Об одной задаче оптимального управ\-ле\-ния нагрузкой 
на сервер~// Информатика и~её применения, 2013. Т.~7. Вып.~4. С.~34--43.
  \bibitem{2-ag}
  \Au{Агаларов Я.\,М., Шоргин~В.\,С.} Об одной задаче максимизации дохода 
системы массового обслуживания типа $G/M/1$ с~пороговым управ\-ле\-ни\-ем очередью~// Информатика и~её 
применения, 2017. Т.~11. Вып.~4. С.~55--64.
  \bibitem{3-ag}
  \Au{Adams R.} Active queue management: A~survey~// IEEE Commun. 
Surv. Tut., 2013. Vol.~15. No.\,3. P.~1425--1476.
  \bibitem{4-ag}
  \Au{Жерновый Ю.\,В.} Решение задач оптимального синтеза для некоторых 
марковских моделей обслуживания~// Информационные процессы, 2010. Т.~10. 
№\,3. С.~257--274.
  \bibitem{5-ag}
  \Au{Бочаров П.\,П., Печинкин~А.\,В.} Теория массового обслуживания.~--- 
М.: РУДН, 1995. 529~с.
 \end{thebibliography}

 }
 }

\end{multicols}

\vspace*{-6pt}

\hfill{\small\textit{Поступила в~редакцию 20.02.19}}

%\vspace*{8pt}

%\pagebreak

\newpage

\vspace*{-28pt}

%\hrule

%\vspace*{2pt}

%\hrule

%\vspace*{-2pt}

\def\tit{PROOF OF~THE~UNIMODALITY OF~THE~OBJECTIVE FUNCTION 
IN~$M/M/N$ QUEUE WITH~THRESHOLD-BASED CONGESTION CONTROL}


\def\titkol{Proof of~the~unimodality of~the~objective function 
in~$M/M/N$ queue with~threshold-based congestion control}

\def\aut{Ya.\,M.~Agalarov and~M.\,G.~Konovalov}

\def\autkol{Ya.\,M.~Agalarov and~M.\,G.~Konovalov}

\titel{\tit}{\aut}{\autkol}{\titkol}

\vspace*{-11pt}



\noindent
Institute of Informatics Problems, Federal Research Center ``Computer Science and Control'' of the 
Russian Academy of Sciences, 44-2~Vavilov Str., Moscow 119333, Russian Federation


\def\leftfootline{\small{\textbf{\thepage}
\hfill INFORMATIKA I EE PRIMENENIYA~--- INFORMATICS AND
APPLICATIONS\ \ \ 2019\ \ \ volume~13\ \ \ issue\ 2}
}%
 \def\rightfootline{\small{INFORMATIKA I EE PRIMENENIYA~---
INFORMATICS AND APPLICATIONS\ \ \ 2019\ \ \ volume~13\ \ \ issue\ 2
\hfill \textbf{\thepage}}}

\vspace*{6pt}
  
\Abste{The problem of limiting the load in the system $M/M/N/\infty$ 
is considered using a~simple threshold strategy. In addition to the service time, each task 
is characterized by a deadline. Depending on the quality of service, the system 
receives either  a~fixed income or a~penalty. The quality of control is determined by the 
marginal average income and the threshold value that maximizes this value is considered 
as optimal. Usually, it is much easier to find the optimal threshold if the objective 
function has a~single maximum. The experimental results show the unimodality of the 
objective function for a~wide class of arrival flows. However, there is no rigorous 
proof of this fact and in the paper, this gap is filled up for the Poisson arrivals. 
The proof is based on the results of the Markov chain theory and queueing theory.}

\KWE{Markov chains; $M/M/N/\infty$ system; congestion control; threshold 
control; deadline}


  
  
 \DOI{10.14357/19922264190201}

%\vspace*{-14pt}


\Ack
\noindent
The reported study was partly funded by the Russian Foundation for Basic Research 
according to the research projects 
No.\,18-07-00692 and No.\,19-07-00739.


%\vspace*{6pt}

  \begin{multicols}{2}

\renewcommand{\bibname}{\protect\rmfamily References}
%\renewcommand{\bibname}{\large\protect\rm References}

{\small\frenchspacing
 {%\baselineskip=10.8pt
 \addcontentsline{toc}{section}{References}
 \begin{thebibliography}{9}
  \bibitem{1-ag-1}
\Aue{Konovalov, M.\,G.} 2013. Ob odnoy zadache optimal'nogo upravleniya 
nagruzkoy na server [About one task of overload control]. \textit{Informatika i~ee 
Primeneniya~--- Inform. Appl.} 7(4):34--43.
  \bibitem{2-ag-1}
\Aue{Agalarov, Ya.\,M., and V.\,S.~Shorgin.} 2017. Ob odnoy zadache 
maksimizatsii dokhoda sistemy massovogo obsluzhivaniya tipa $G/M/1$ s~porogovym upravleniem ochered'yu  
[About the problem of profit maximization in $G/M/1$ queuing system 
with threshold control of the queue]. \textit{Informatika i ee Primeneniya~--- Inform. Appl.} 
11(4):55--64.
  \bibitem{3-ag-1}
\Aue{Adams,~R.} 2013. Active queue management: A~survey. \textit{IEEE 
Commun. Surv.  Tut.} 15(3):1425--1476.
  \bibitem{4-ag-1}
\Aue{Zhernovyj, Ju.\,V.} 2010. Reshenie zadach optimal'nogo sinteza dlya 
nekotorykh markovskikh modeley obsluzhivaniya [Solution of optimum synthesis 
problem for some Markov models of service]. \textit{Information Processes} 10(3):257--274.
  \bibitem{5-ag-1}
\Aue{Bocharov, P.\,P., and A.\,V.~Pechinkin.} 1995. \textit{Teoriya massovogo 
obsluzhivaniya} [Queueing theory]. Moscow: RUDN. 529~p.
\end{thebibliography}

 }
 }

\end{multicols}

\vspace*{-6pt}

\hfill{\small\textit{Received February 20, 2019}}

%\pagebreak

%\vspace*{-18pt}   
  
  
  \Contr
  
  \noindent
  \textbf{Agalarov Yaver M.} (b.\ 1952)~---  Candidate of Science (PhD) in technology, associate 
professor; leading scientist, Institute of Informatics Problems, Federal Research Center ``Computer Science 
and Control'' of the Russian Academy of Sciences, 44-2~Vavilov Str., Moscow 119333, Russian Federation; 
\mbox{agglar@yandex.ru}
   
   \vspace*{3pt}
   
  \noindent
  \textbf{Konovalov Mikhail~G.} (b.\ 1950)~--- Doctor of Science in technology, principal scientist, 
Institute of Informatics Problems, Federal Research Center ``Computer Science and Control'' of the Russian 
Academy of Sciences, 44-2~Vavilov Str., Moscow 119333, Russian Federation; 
\mbox{mkonovalov@ipiran.ru}
  
\label{end\stat}

\renewcommand{\bibname}{\protect\rm Литература} 
   