\newcommand{\Tsf}{^{\mathsf T}}
\newcommand{\rank}{\mathrm{rank}\,}

\def\stat{logachev}

\def\tit{ОБ ОДНОМ МЕТОДЕ РЕШЕНИЯ СИСТЕМ КВАДРАТИЧНЫХ БУЛЕВЫХ УРАВНЕНИЙ, 
ИСПОЛЬЗУЮЩЕМ ЛОКАЛЬНЫЕ АФФИННОСТИ БУЛЕВЫХ ФУНКЦИЙ$^*$}

\def\titkol{Об одном методе решения систем квадратичных булевых уравнений, 
использующем локальные аффинности} % булевых функций}

\def\aut{О.\,А.~Логачев$^1$, А.\,А.~Сукаев$^2$, С.\,Н.~Федоров$^3$}

\def\autkol{О.\,А.~Логачев, А.\,А.~Сукаев, С.\,Н.~Федоров}

\titel{\tit}{\aut}{\autkol}{\titkol}

\index{Логачев О.\,А.}
\index{Сукаев А.\,А.}
\index{Федоров С.\,Н.}
\index{Logachev O.\,A.}
\index{Sukayev A.\,A.}
\index{Fedorov S.\,N.}


{\renewcommand{\thefootnote}{\fnsymbol{footnote}} \footnotetext[1]
{Работа выполнена при частичной поддержке РФФИ (проект 18-29-03124~мк).}}


\renewcommand{\thefootnote}{\arabic{footnote}}
\footnotetext[1]{Институт проблем информатики Федерального 
исследовательского центра <<Информатика и~управление>> Российской академии наук, 
\mbox{ollog@inbox.ru}}
\footnotetext[2]{Московский государственный университет имени 
М.\,В.~Ломоносова, \mbox{asukaev@gmail.com}}
\footnotetext[3]{Московский государственный университет имени 
М.\,В.~Ломоносова, \mbox{s.n.feodorov@yandex.ru}}

%\vspace*{-2pt}


\Abst{Как известно, вычислительная задача решения систем нелинейных уравнений над 
полем из двух элементов является NP-труд\-ной.
Этим обстоятельством обусловливается стремление исследователей разрабатывать
алгоритмы ее решения, минимизирующие необходимые вычислительные ресурсы для тех 
или иных классов систем уравнений.
В статье предлагается метод решения систем квадратичных булевых уравнений, 
использующий представление функций их аффинными нормальными формами, т.\,е.\ 
в~некотором смысле аппроксимацию квадратичных функций ку\-соч\-но-аффин\-ны\-ми. Для 
каждого уравнения на основе такого представления строится набор систем 
небольшого числа линейных уравнений, а~затем ищется совместная комбинация этих 
линейных систем для различных исходных уравнений. Исходная нелинейная задача, 
таким образом, сводится, по большому счету, к~проверке совместности серии 
линейных систем от того же числа переменных.
Метод может быть эффективно распараллелен и,~несмотря на большую трудоемкость 
в~худшем случае, допускает ряд эвристик, уменьшающих время его выполнения.}

\KW{булева функция; система квадратичных булевых уравнений; 
разбиение векторного пространства; плоскость; локальная аффинность; аффинная 
нормальная форма; алгебраический крип\-то\-ана\-лиз}


\DOI{10.14357/19922264190206}
  
\vspace*{9pt}


\vskip 10pt plus 9pt minus 6pt

\thispagestyle{headings}

\begin{multicols}{2}

\label{st\stat}


\section{Введение}

Фундаментом сложившегося за последние два десятилетия направления исследований, 
которое принято называть алгебраическим криптоанализом, служит идея, состоящая 
в~том, чтобы описать используемые в~анализируемой криптосистеме преобразования 
системой алгебраических уравнений (с~некой секретной информацией в~качестве 
неизвестных переменных) и~затем решить эту сис\-те\-му.
Сама эта идея далеко не нова, ее высказывал еще Клод Шеннон 
в~1940-х~гг.~\cite{Shannon1949}, но более или менее активный и~постоянный интерес 
к~такой постановке вопроса наблюдается с~середины 1990-х~гг.

Для первой части этого подхода, состоящей в~построении системы уравнений, вряд 
ли можно дать универсальные рекомендации. Все зависит от крип\-то\-сис\-те\-мы и~от 
того, каким методом предполагается решать эту систему уравнений.
При этом нередко структура исследуемой криптосистемы сама подсказывает, как 
составить соответствующую систему уравнений.
Данная статья не касается этого вопроса.
Следует только отметить, что, как правило, строится переопределенная система 
уравнений, левые части которых представляют собой полиномы над полем~$\mathbb{F}_2$ 
или~$\mathbb{F}_{2^k}$.

Что касается второй части подхода, проблема решения системы уравнений~--- это 
весьма важная задача для математики вообще, и~поэтому довольно интенсивно 
изучалась.
Однако для криптоанализа, в~отличие от многих других областей математики, крайне 
важно не принципиальное наличие решения, а~конструктивный метод его поиска, 
а~также эффективность (невысокая трудоемкость) соответствующих алгоритмов, 
возможно, в~ущерб универсальности и~теоретической обоснованности ме\-тода.



Здесь можно выделить несколько основных классов (практических) методов решения 
систем полиномиальных уравнений:
использование базисов Грёбнера~\cite[разд.~12.2]{Bard2009}, применение 
про\-граммных систем поиска выполняющего набора булевой формулы 
(SAT-solvers)~\cite{BCJ2007}, линеаризационные 
методы~\cite[разд.~12.3]{Bard2009}.
Основная идея последних~--- применение <<линейных>> методов к~нелинейным 
сис\-те\-мам, т.\,е.\ построение сис\-тем линейных уравнений, решение которых дает 
воз\-мож\-ность найти решение исходной сис\-темы.

Скорее всего, именно к~этому классу следует отнести описанный в~настоящей работе 
метод решения системы $m$~квадратичных уравнений от $n$~переменных над 
полем~$\mathbb{F}_2$ (см.\ систему~\eqref{system} ниже), хотя он существенно, если не 
сказать~--- кардинально, отличается от ставшего уже классическим метода 
линеаризации.

Предлагается рассматривать вместо каждой функции~$f_i$ ($1\hm\leqslant i\hm\leqslant m$) из 
левой части системы~\eqref{system} ее аффинную нормальную форму~--- выражение, 
предполагающее, что векторное пространство~$\mathbb{F}_2^n$ разбито на плоскости 
(смежные классы по аффинным подпространствам) таким образом, что на каждой из 
этих плоскостей~$f_i$ совпадает с~некоторой аффинной (линейной) функцией.
Вопросу о~построении таких выражений для функций сис\-те\-мы посвящена 
работа~\cite{LSF2019}, где представлен один из возможных способов их нахождения~--- 
вместе с~оценками трудоемкости, поэтому дальнейшее изложение не касается 
этого этапа метода.
Заметим, что решение сис\-те\-мы уравнений лежит в~пересечении некоторого набора 
плоскостей $(\pi_1,\ldots,\pi_m)$, где~$\pi_i$, $1\hm\leqslant i\hm\leqslant m$, берется из 
указанного разбиения пространства для функции~$f_i$.
Идея метода со\-сто\-ит в~том, чтобы, отбрасывая заведомо неподходящие комбинации 
плоскостей, попытаться угадать набор $(\pi_1,\ldots,\pi_m)$.
На этом наборе все уравнения сис\-те\-мы можно заменить их аффинными 
<<аппроксимациями>>, и~тогда останется только решить линейную сис\-те\-му, которая, 
подчеркнем это, в~стандартном варианте метода со\-сто\-ит не более чем из 
$m(\lfloor{n}/{2}\rfloor+1)$ уравнений от $n$~неизвестных, т.\,е.\ 
в~отличие от других известных линеаризационных методов, здесь число переменных не 
увеличивается.
Еще одним преимуществом является то, что метод вполне естественным образом 
допускает распараллеливание.
Кроме того, таким способом решение находится во всех случаях, хотя это может 
потребовать большого объема вычислений: в~худшем случае трудоемкость, конечно, 
экспоненциальна (см.\ разд.~5). Чтобы преодолеть это препятствие, 
необходимо использовать специальные приемы.
Подробно основная идея подхода описана в~разд.~3, а~соображения по 
его усовершенствованию приведены в~разд.~4.
Необходимо отметить, что подобный пред\-ла\-га\-емо\-му метод для систем уравнений, 
описывающих преобразования в~фильтрующих генераторах, был описан 
в~работе~\cite{Log2008}.

Данная работа ограничивается рассмотрением многочленов не более чем второй 
степени, поскольку существует алгоритм (см., например,~\cite[\S\;11.4.2]{Bard2009}), 
который для произвольной системы полиномиальных 
уравнений фиксированной степени над полем из двух элементов за полиномиальное 
(от числа переменных) время
строит такую систему \textit{квадратичных} уравнений над тем же полем, что из ее 
решений однозначно восстанавливаются все решения исходной системы.
Следует, однако, заметить, что если не накладывать ограничений на степень 
полиномов системы, то верхняя оценка трудоемкости упомянутого алгоритма 
становится уже экспоненциальной~\cite[\S\;11.4.6]{Bard2009}. 
%%%\Theta(e^{\frac{n}{2}(1+\ln n)})

Отметим еще некоторые ограничения на рассматриваемую систему. Они не являются 
необходимыми для применения метода, однако несколько упрощают изложение, 
оставаясь при этом и~довольно естественными в~рамках криптоанализа.
Предполагается, что число уравнений больше чис\-ла неизвестных, что, как правило, 
справедливо для сис\-тем уравнений, получаемых при анализе крип\-то\-гра\-фи\-че\-ских 
преобразований.
Кроме того, рас\-смат\-ри\-ва\-ют\-ся лишь те системы уравнений, которые заведомо имеют 
решение и~притом только одно.
Это условие объясняется тем, что алгебраический крип\-то\-ана\-лиз имеет дело не 
с~произвольными уравнениями, а лишь с~описывающими преобразования некоторых 
объектов, например секретного ключа или открытого текста, в~тех или иных 
криптосистемах; таким образом, решением системы уравнений является этот самый 
объект (ключ или открытый текст).
Отметим здесь кстати, что в~отличие от линеаризационных методов, где нужна 
система с~достаточно большим количеством уравнений, чтобы компенсировать рост 
числа переменных, в~предлагаемом методе чем меньше уравнений в~исходной системе, 
тем лучше,~--- лишь бы они имели только одно общее решение.


\section{Необходимые понятия и~обозначения}

В данной работе $V_n$~--- $n$-мер\-ное векторное пространство над полем из двух 
элементов~$\mathbb{F}_2$.
%Векторы из $V_n$ нам будет удобнее записывать \emph{строками} длины~$n$. 
%Значок~$\T$ используется для операции транспонирования матриц.
%Всюду далее $x$ обозначает вектор $(x_1,x_2,\ldots,x_n)$.
Знак $\oplus$ будет использоваться для записи суммы по модулю~$2$ булевых 
переменных и~операций сложения в~$\mathbb{F}_2$ и~покомпонентного сложения 
в~$V_n$.

Далее под булевой функцией часто будет под\-разуме\-вать\-ся ее запись в~виде полинома
$$
f(x)=\bigoplus_{\varepsilon\in\{0,1\}^n} a_{\varepsilon}x^{\varepsilon}\,,
$$
где %\label{Zhegalkin}
$\varepsilon\hm=(\varepsilon_1,\ldots,\varepsilon_n)$;
$a_{\varepsilon}\hm\in\mathbb{F}_2$; 
$x^{\varepsilon}\hm=x_1^{\varepsilon_1}\cdots x_n^{\varepsilon_n}$ (считаем 
$x_i^0\hm=1$ и~$x_i^1\hm=x_i$).


В настоящей работе рассматриваются системы уравнений
\begin{equation}
\label{system}
\left.
    \begin{array}{r@{\ }c@{\ }l}
        f_1(x_1,\ldots,x_n)&=&0\,;\\
        f_2(x_1,\ldots,x_n)&=&0\,;\\
        \cdots\\
        f_m(x_1,\ldots,x_n)&=&0\\
    \end{array}
  \right\}
\end{equation}
с квадратичными булевыми функциями~$f_i$, $1\hm\leqslant i\hm\leqslant m$, и~$m\hm>n$.
Предполагается, что все рассматриваемые системы квадратичных уравнений имеют 
единственное решение.


\textit{Плоскость}~$\pi$ в~$V_n$~--- это множество вида $v\hm+L$, где~$v$ и~$L$~--- 
соответственно вектор и~подпространство пространства~$V_n$.
Сужение булевой функции~$f$ на плоскость~$\pi$ будем обозначать 
через~$f|_{\pi}$. Таким образом, $f|_{\pi}\colon \pi\hm\to\mathbb{F}_2$ 
и~$f|_{\pi}(u)\hm=f(u)$ для всех $u\in\pi$.


\textit{Локальной аффинностью} булевой функции~$f$ называется такая 
плоскость~$\pi$, что $f|_{\pi}$ можно продолжить до аффинной функции, т.\,е.\ 
существует аффинная функция~$l$ со свойством $f|_{\pi}\hm=l|_{\pi}$.

Для произвольного разбиения $\Pi\hm=\{\pi_1,\ldots,\pi_{\lambda}\}$ 
пространства~$V_n$ на локальные аффинности булевой функции~$f$ от $n$~переменных
\textit{аффинной нормальной формой} функции~$f$ называется выражение вида:
\begin{equation}
\label{AffNF}
f(x)=\bigoplus_{j=1}^{\lambda}\chi_{\pi_j}(x) l_j(x)\,,
\end{equation}
где для каждого $j$, $1\hm\leqslant j\hm\leqslant\lambda$, функция~$l_j$ аффинна 
и~$f|_{\pi_j}(x)\hm=l_j|_{\pi_j}(x)$, а $\chi_{\pi_j}$~--- характеристическая 
функция (индикатор) множества~$\pi_j$.
Функции $l_j$ из этого выражения будем называть \textit{локальными аффинными 
аппроксимациями} функции~$f$.
\textit{Длиной аффинной нормальной формы} называется число плоскостей 
в~разбиении~$\Pi$, обозначим ее через~$\lambda(\Pi)$.

Характеристическая функция плоскости в~пространстве~$V_n$ имеет вполне 
определенный вид. Любая плоскость~$\pi$, как уже отмечалось, может быть задана 
как множество решений системы $d$~линейных уравнений (для некоторого~$d$):
\begin{equation}
\label{chi-system}
  \left.
    \begin{array}{r@{\ }c@{\ }l}
        h_1(x_1,\ldots,x_n)&=&0\,;\\
        h_2(x_1,\ldots,x_n)&=&0\,;\\
        \cdots\\
        h_d(x_1,\ldots,x_n)&=&0\,,\\
    \end{array}
  \right\}
\end{equation}
где все $h_i(x)$ аффинные. Поскольку вектор~$x$ принадлежит плоскости~$\pi$ 
тогда и~только тогда, когда все~$h_i$, $1\hm\leqslant i\hm\leqslant d$, обращаются в~нуль на 
нем, характеристическая функция~$\pi$ выражается следующим образом:
$$\chi_{\pi}(x)=\prod_{i=1}^d (h_i(x)\oplus 1).$$
Если система линейных уравнений задана в~мат\-рич\-ной форме~--- $xH\oplus 
(b_1,\ldots,b_d)\hm=0$, $b_i\hm=h_i(0)$, то выражение будет иметь вид:
$$
\chi_{\pi}(x)=\prod\limits_{i=1}^d \left(xH_i\oplus b_i\oplus 1\right)\,,$$
где $H_i$~--- столбцы матрицы~$H$.

Как видно из определения, аффинная нормальная форма представляет собой 
в~некотором смыс\-ле ку\-соч\-но-аффин\-ную аппроксимацию булевой функции. На каждой 
локальной аф\-фин\-ности~$\pi_j$ из разбиения $\Pi$ все, кроме одного, слагаемые 
в~выражении~\eqref{AffNF} обращаются в~нуль, и~функция принимает вид 
$f(x)\hm=\chi_{\pi_j}(x)l_j(x)=l_j(x)$ для всех $x\hm\in\pi_j$.

Возможность заменить на плоскости~$\pi_j$ квадратичное уравнение $f(x)\hm=0$ 
линейным уравнением $l_i(x)\hm=0$ вместе с~дописанной к~нему системой~\eqref{chi-system} 
будет использоваться при решении систем полиномиальных уравнений.


\section{Метод решения системы, использующий локальные аффинности}
\label{method}

Пусть дана система квадратичных уравнений~\eqref{system}. Для применения 
описываемого далее метода следует получить аффинные нормальные формы для всех 
функций в~левых частях уравнений системы. В~работе~\cite{LSF2019} пред\-став\-лен 
один из возможных способов, кажущийся наиболее естественным в~квадратичном 
случае.



\subsection{Построение <<локальных>> систем линейных уравнений}

Пусть для всех~$f_i$, $1\hm\leqslant i\hm\leqslant m$, определены некоторые аффинные нормальные 
формы:
\begin{equation*}
\label{AffNF_ij}
    f_i(x) = \bigoplus_{j=1}^{\lambda(i)} \chi_{\pi_{ij}}(x)l_{ij}(x)\,.
\end{equation*}
Здесь предполагается, что каждое разбиение пространства на локальные аффинности 
некоторым образом упорядочено: 
$$
\Pi_i=\left\{\pi_{i,1},\ldots,\pi_{i,\lambda(i)}\right\}, \
\lambda(i)=\lambda(\Pi_i)
$$ 
(как именно лучше их упорядочить для дальнейшего 
использования~--- предмет отдельного исследования, несколько слов об этом будет 
сказано ниже).
Исходя из этих аффинных нормальных форм можно для каждой пары~$i,j$ записать 
эквивалентную уравнению $f_i\hm=0$ на~$\pi_{ij}$ систему линейных уравнений:
\begin{equation}
\label{approx}
  \left.
    \begin{array}{r@{\ }c@{\ }l}
        l_{ij}(x)&=&0\,;\\
        h_{ij}^1(x)&=&0\,;\\
        \cdots\\
        h_{ij}^{d(i,j)}(x)&=&0\,,\\
    \end{array}
  \right\}
\end{equation}
в которой первое уравнение выражает равенство $f_i\hm=0$ через аффинную 
аппроксимацию~$l_{ij}(x)$ функции~$f_i(x)$ на плос\-кости~$\pi_{ij}$, а остальные 
$d(i,j)$ уравнений задают эту плоскость (хотя это непринципиально, можно для 
удобства считать, что $d(i,j)\hm=n\hm-\mathrm{dim}\, \pi_{ij}$, т.\,е.\ 
что система содержит  минимальное число уравнений).


Перепишем систему~\eqref{approx} в~виде матричного уравнения
$$
xA_{ij} = b_{ij} = \left(l_{ij}(0),h_{ij}^1(0),\ldots,h_{ij}^{d(i,j)}(0)\right)\,,
$$
компоненты $\left(n\times(d(i,j)+1)\right)$-мат\-ри\-цы $A_{ij}$ представляют собой 
коэффициенты при переменных в~полиномах Жегалкина аффинных функций~$l_{ij}$ 
(в~первом столбце) и~$h_{ij}^k$, а~компоненты вектора~$b_{ij}$~--- постоянные 
слагаемые в~этих полиномах.

Обозначим также через $H_{ij}$ матрицу системы уравнений, определяющей 
плоскость~$\pi_{ij}$ (т.\,е.\ подматрицу, полученную удалением первого столбца 
матрицы~$A_{ij}$), а через~$b'_{ij}$~--- вектор~$b_{ij}$ без первой компоненты. 
Таким образом, плоскость~$\pi_{ij}$ задается матричным уравнением:
\begin{equation}
\label{flat-matrix-eq}
    xH_{ij}=b'_{ij}\,.
\end{equation}


\subsection{Опробование локальных аффинностей}

Пусть для функций~$f_i$ системы~\eqref{system} зафиксированы разбиения~$\Pi_i$ 
пространства~$V_n$ и~соответствующие аффинные нормальные формы.
Как уже говорилось, каждое $\Pi_i$ некоторым образом упорядочено. Этот порядок 
естественным образом индуцирует лексикографический порядок на множестве 
$\Pi_1\times\cdots\times\Pi_m$ комбинаций локальных аффинностей функций системы.

В соответствии с~этим порядком будем последовательно проверять комбинации 
плоскостей~$\pi_{i,j_i}$, $1\hm\leqslant i\hm\leqslant m$, и~отбраковывать те из них, которые не 
могут содержать общего корня полиномов~$f_i$, т.\,е.\ решения исходной системы. 
Основным инструментом такого отсеивания служит теорема Кро\-не\-ке\-ра--Ка\-пелли.

Рассмотрим набор индексов 
$$
J=(j_1,\ldots,j_m)\in\mathop{\times}\limits_{i=1}^m 
\{1,\ldots,\lambda(i)\}\,. 
$$
Он определяет комбинацию плоскостей из\linebreak 
$\Pi_1\times\cdots\times\Pi_m$, поэтому для краткости назовем его индексом 
комбинации. Первый раз нижеописанные шаги выполняются для набора $(1,\ldots,1)$.

Введем вспомогательные обозначения для $\left(n\times\sum\nolimits_{i=1}^k 
d(i,j_i)\right)$-мат\-ри\-цы, составленной из мат\-риц линейных систем 
вида~\eqref{approx}:
$$
A^{J}_k = \left(\begin{array}{c|c|c|c}
                A_{1,j_1} & A_{2,j_2} & \cdots & A_{k,j_k}\\
              \end{array}\right),
              $$
и для соответствующего вектора правых частей уравнений в~таких системах:
$$
b^{J}_k = \left(\begin{array}{c|c|c|c}
                b_{1,j_1} & b_{2,j_2} & \cdots & b_{k,j_k}
              \end{array}\right).
              $$


\textit{Шаг~1.} Для плоскости~$\pi_{1,j_1}\in\Pi_1$ проверим равенство
$$
\rank A^{J}_1 = \rank A_{1,j_1} = \rank
    \left(\begin{array}{c}
            \rule{0pt}{1.2em} A_{1,j_1} \\[6pt]
            \hline
            b_{1,j_1}
          \end{array}\right),
$$
гарантирующее совместность системы вида~\eqref{approx} для локальной 
аффинности~$\pi_{1,j_1}$.

%\textit{Шаг 2.} Для плоскости~$\pi_{2,j_2}\in\Pi_2$ проверим равенство
%$$\rank A^{J}_2 =
%%  \rank \left(\begin{array}{c|c}
%%              A_{1,j_1} & A_{2,j_2}\\
%%            \end{array}\right) =
%  \rank \left(\begin{array}{c|c}
%               A_{1,j_1} & A_{2,j_2}\\
%               \hline
%               b_{1,j_1} & b_{2,j_2}
%             \end{array}\right).
%$$

\textit{Шаг $s$, $1\hm<s\hm\leqslant m$.}
Пусть уже имеется такой набор локальных аффинностей $(\pi_{1,j_1},\ldots,
\pi_{s-1,j_{s-1}})\hm\in\Pi_1\times\cdots\times\Pi_{s-1}$, что
 система $xA^{J}_{s-1}\hm=b^{J}_{s-1}$ совместна.
Возьмем плоскость~$\pi_{s,j_s}\hm\in\Pi_s$ и~проверим равенство
\begin{equation}
\label{rank-eq}
  \rank A^{J}_s =
%  \rank \left(\begin{array}{c|c|c|c}
%               A_{1,j_1} & A_{2,j_2} & \ldots & A_{s,j_s}\\
%             \end{array}\right) =
  \rank \left(\begin{array}{c|c|c|c}
                \rule{0pt}{1.2em} A_{1,j_1} & A_{2,j_2} & \cdots & 
A_{s,j_s}\\[6pt]
                \hline
                b_{1,j_1} & b_{2,j_2} & \cdots & b_{s,j_s}
              \end{array}\right).
\end{equation}
Справедливость равенства~\eqref{rank-eq} означает, что система 
$xA^{J}_s\hm=b^{J}_s$ совместна и, значит, $\bigcap_{i=1}^s \pi_{i,j_i}$ содержит 
решения подсистемы $\{f_i=0\mid 1\hm\leqslant i\hm\leqslant s\}$, т.\,е.\
 отказываться от данной комбинации еще~рано.

После $k$-го шага, $1\leqslant k<m$, в~случае успешной проверки условия переходим 
к~следующему шагу~$k+1$.

Если на шаге, скажем, с~номером~$t$ соответствующее равенство для локальной 
аффинности~$\pi_{t,j_t}$, $j_t\hm<\lambda(t)$, не выполняется, значит, выбранная\linebreak 
комбинация плоскостей не содержит решения сис\-те\-мы~\eqref{system}.
В~этом случае процедура повторяется для комбинации с~индексом $(j_1,\ldots,j_{t-
1},j_t\hm+1,1,\ldots,1)$ начиная с~шага~1. Таким образом, вся <<ветвь>> комбинаций, 
имеющих индексы с~началом $(j_1,\ldots,j_t,\ldots)$, отбрасывается за 
бес\-перспек\-тив\-ностью.
Если же то же происходит при $j_t\hm=\lambda(t)$, то переходить следует 
к~комбинации с~индексом $(j_1,\ldots,j_{t-2},j_{t-1}+1,1,\ldots,1)$.

В том случае, когда удалось пройти все $m$~шагов, получаем линейную систему из 
$(m\hm+\sum\nolimits_{i=1}^m d(i,j_i))$ уравнений с~$n$~неизвестными.
Остается решить ее одним из вариантов метода Гаусса. Следующая тео\-ре\-ма 
утверждает, что это даст решение исходной системы~\eqref{system}.

\smallskip

\noindent
\textbf{Теорема~1.}\
\textit{Пусть система}~\eqref{system} \textit{квадратичных булевых уравнений имеет 
единственное решение.
    Тогда существует построенная согласно вышеописанной процедуре совместная 
система линейных булевых уравнений вида}
 \begin{equation}\label{linear-system}
    xA^{J}_m=b^{J}_m,
 \end{equation}
 \textit{она единственна и~ее решение является в~точности решением 
системы}~\eqref{system}.

\smallskip

\noindent
Д\,о\,к\,а\,з\,а\,т\,е\,л\,ь\,с\,т\,в\,о\,.\ \
    Пусть $a\hm\in V_n$~--- единственное решение системы~\eqref{system}, т.\,е.\ 
$f_i(a)\hm=0$ для всех~$i$.
    Значит, для любого~$i$ существует единственный индекс~$j_i$, такой что 
$a\hm\in\pi_{i,j_i}$.
    Обозначим $J_a\hm=(j_1,\ldots,j_m)$. Тогда~$a$ является решением системы 
$xA^{J_a}_m\hm=b^{J_a}_m$,
    так как $a$ удовлетворяет всем уравнениям плоскостей~$\pi_{i,j_i}$, $1\hm\leqslant 
i\hm\leqslant m$, и~обнуляет аффинную аппроксимацию $l_{i,j_i}$ функции~$f_i$: 
$l_{i,j_i}(a)\hm=f_i(a)\hm=0$ (см.\ систему~\eqref{approx}).

    Обратно, пусть $a'$~--- решение произвольной сис\-те\-мы $xA^{J'}_m\hm=b^{J'}_m$ 
для $J'\hm=(j'_1,\ldots,j'_m)$.
    Тогда для всех $i$ имеем $a'\hm\in\pi_{i,j'_i}$ и~$l_{i,j'_i}(a)\hm=0$. Поэтому 
$f_i(a')\hm=f_i|_{\pi_{i,j'_i}}(a')\hm=0$ для всех~$i$.
    Следовательно, $a'\hm=a$.

\smallskip

Таким образом, в~любом случае рано или поздно будет построена полноразмерная 
совместная система~\eqref{linear-system}, решение которой гарантированно 
является искомым в~задаче.

\smallskip

\noindent
\textbf{Замечание~1.}
    Если аффинные нормальные формы были получены способом, описанным 
    в~работе~\cite{LSF2019},~то
    $$
    d(i,j_i)=\fr{1}{2}\,\rank\tilde{Q}_{f_i}\leqslant\left\lfloor\fr{n}{2}\right
    \rfloor,
    $$
    где $\tilde{Q}_{f_i}$~--- мат\-ри\-ца, определяющая билинейную форму 
$f_i(u\oplus v)\oplus f_i(u)\oplus f_i(v)\oplus f_i({0})$, 
ассоциированную с~квадратичной функцией~$f_i$.
    Таким образом, число уравнений в~системе, построенной для любой комбинации 
локальных аффинностей, не превосходит $m(1\hm+\lfloor {n}/{2}\rfloor)$.


\smallskip

\noindent
\textbf{Замечание~2.}
    Очевидно, что если исключить условие единственности решения, то комбинация 
локальных аффинностей, содержащих решение, будет уже не обязательно одна. 
Соответственно, будет и~некоторое число совместных линейных систем 
вида~\eqref{linear-system}.\par
    Такая ситуация может возникнуть из-за введения фиктивных переменных при 
построении исходной системы уравнений. Например, в~случае\linebreak (блоковых) 
криптосистем, использующих композицию нескольких криптографических 
преобразований, для упрощения процедуры и~снижения\linebreak
 степени уравнений переменными 
назначаются результаты промежуточных вычислений. (Подчеркнем, что алгоритм 
сведения полиномиальной сис\-те\-мы к~квадратичной~\cite[\S\;11.4.2]{Bard2009} новых 
решений не добавляет.)


    Поскольку здесь речь идет об уже готовой системе, а~для нее все ее решения 
равнозначны, в~рамках метода нельзя заранее отсеять <<ложные>> решения. Поэтому 
на предварительном этапе следует строить (нелинейную) систему так, чтобы она 
имела как можно меньше фиктивных решений, а~впоследствии при получении каждого 
решения~--- проверять, соответствует ли оно искомым секретным данным.


\section{Замечания по оптимизации метода}\label{optim}

Прежде всего надо заметить, что существенное влияние на работу метода оказывает 
выбор аффинных нормальных форм и~соответствующих раз\-би\-ений пространства. Здесь 
достаточно обратить внимание на то, что снижение на единицу размерности 
локальных аффинностей для одной функции может увеличить число опробуемых 
комбинаций плоскостей в~два раза (подробнее см.\ в~\cite{LSF2019}).
Поэтому вопрос построения аффинных нормальных форм должен быть одним из основных 
при исследованиях возможности оптимизировать метод.

Рассмотрим некоторые из приемов, способных повысить эффективность метода.

\vspace*{-9pt}

\paragraph*{Предварительное отсеивание непересекающихся плоскостей.}
Некоторые из комбинаций локальных аффинностей, которые, напомним, задаются 
матричными уравнениями~\eqref{flat-matrix-eq}, можно отбросить до вычисления 
рангов соответствующих матриц следующим способом.
На шаге с~номером~$s$, $2\hm\leqslant s\hm\leqslant m$, можно добавить проверку условия:
\begin{multline*}
\rank \left(\!\begin{array}{c|c|c|c}
                H_{1,j_1} & H_{2,j_2} & \cdots & H_{s,j_s}\\
              \end{array}\!\right) ={}\\
              {}=
  \rank \left(\begin{array}{c|c|c|c}
                \rule{0pt}{1.2em} H_{1,j_1} & H_{2,j_2} & \cdots & 
H_{s,j_s}\\[6pt]
                \hline
                b'_{1,j_1} & b'_{2, j_2} & \cdots & b'_{s, j_s}
              \end{array}\right).
\end{multline*}
Благодаря этому равенству из рассмотрения исключаются не пересекающиеся 
в~совокупности локальные аффинности.
Такую проверку, например для всех пар плоскостей, потом для всех троек и~т.\,д., 
можно провести и~до начала выполнения первого шага.
Сравнение соответствующих рангов на каж\-дом шаге имеет то преимущество, что 
требует меньше дополнительных вычислений, так как~$H_{i,j_i}$ является 
подматрицей~$A_{i,j_i}$ (то же верно для соответствующих расширенных матриц).


\paragraph*{Уменьшение числа уравнений.}
Допустим, ранг мат\-ри\-цы будет вычисляться приведением мат\-ри\-цы к~ступенчатому виду с~помощью преобразования столбцов методом Гаусса.
Для каждой комбинации плоскостей с~индексом~$J$ последовательно вы\-чис\-ля\-ют\-ся 
ранги матриц~$A^{J}_s$ и

\noindent
$$
\hat A^{J}_s\hm=\left(\fr{A^{J}_s}{b^{J}_s}\right),
$$

\noindent
 $s\geqslant 1$, при этом как $A^{J}_{s+1}$, так 
          и~$\hat A^{J}_s$ являются под\-мат\-ри\-ца\-ми матрицы~$\hat A^{J}_{s+1}$.
Так что, приведя к~ступенчатому виду $\hat A^{J}_s$ и~получив из него~$\rank 
\hat A^{J}_s$ и~$\rank A^{J}_s$, имеет смысл на шаге~$(s+1)$ оставить теперь уже 
под\-мат\-ри\-цу~$\hat A^{J}_s$ в~полученном виде~--- с~фиксированными линейно 
независимыми столбцами, тем самым сокращая до минимума чис\-ло уравнений 
строящейся системы и~количество операций, необходимых для приведения 
к~ступенчатому виду матрицы~$\hat A^{J}_{s+1}$.
Также можно заранее привести все системы вида~\eqref{approx} к~ступенчатому 
виду, что должно несколько ускорить вычисления.


\vspace*{-12pt}

\paragraph*{Досрочное решение линейной системы.}
Возможно, уже на некотором шаге~$t$, $t<m$, линейных уравнений будет достаточно 
для попытки найти решение, а~именно: если~$\rank A^{J}_t$ равен или достаточно 
близок к~$n$, линейная система будет иметь единственное решение или небольшое 
число решений соответственно.
Достижение полного ранга на промежуточных шагах метода крайне вероятно, если 
учесть, что уравнений в~линейной системе $m\hm+\sum\nolimits_{i=1}^m d(i,j_i)$, причем 
$m\hm>n$.

Полученные таким образом решения необходимо проверить подстановкой 
в~систему~\eqref{system}. Если эти решения не подходят, а такое возможно, 
поскольку на данном этапе не учтены уравнения $f_i\hm=0$ для $t\hm<i\hm\leqslant n$, то 
следует перейти к~другой комбинации локальных аффинностей~--- с~индексом 
$(j_1,\ldots,j_{t-1},j_t+1,1,\ldots,1)$.

\vspace*{-12pt}

\paragraph*{<<Прорежение>> системы линейных уравнений.}
Поскольку искомая линейная система~\eqref{linear-system} сильно 
переопределенная, возникает идея брать не все уравнения из 
системы~\eqref{approx}, а~лишь некоторые из них.
Тогда в~уравнениях вида~\eqref{rank-eq} нужно будет рас\-смат\-ри\-вать вместо 
указанных там мат\-риц их под\-мат\-ри\-цы, полученные удалением некоторых столбцов, что 
должно снизить трудоемкость метода.


При выборе уравнений из~\eqref{approx} следует стремиться к~максимизации 
информативности составленной из них линейной системы.
Возможно, в~системах~\eqref{approx} целесообразнее брать первые уравнения 
($l_{ij}\hm=0$) и~еще по одному ка\-ко\-му-ни\-будь уравнению.
В таком случае для некоторой комбинации локальных аффинностей получим $2m\hm>2n$ 
совместных линейных уравнений (если, конечно, они удовлетворяют всем равенствам 
типа~\eqref{rank-eq} для $1\hm<s\hm\leqslant m$).
Если при этом ранг матрицы полученной линейной системы существенно меньше~$n$, 
то можно взять не по два, а~по три уравнения из систем~\eqref{approx} 
и~повторить процедуру для той же комбинации плоскостей.
Заметим, что если на некотором шаге~$s$ равенство типа~\eqref{rank-eq} для 
усеченных матриц не выполняется, то и~система $xA^{J}_s\hm=b^{J}_s$ будет 
несовместной, так что данную комбинацию плоскостей можно сразу исключать из 
рассмотрения.

\vspace*{-12pt}


\paragraph*{Упорядочение поиска.}
Порядок выполняемых действий играет существенную роль в~методе.
Поскольку перебор большого числа комбинаций локаль\-ных аффинностей с~вычислением 
соответствующих рангов может оказаться затратнее, чем перебор всех значений 
переменных,
крайне важно внимательно отнестись к~выбору последовательности, в~которой будут 
опробоваться локальные аффинности.
Для этого следует определить <<правильный>> порядок уравнений 
в~системе~\eqref{system} и~затем упорядочить тем или иным образом 
разбиения~$\Pi_i$\linebreak пространства на локальные аффинности для каж\-дой функции~$f_i$.
Как именно лучше это сделать~--- тема отдельного исследования.
Например, если локальные аффинности функции имеют разную\linebreak
 размерность, кажется 
целесообразным первыми опробовать плоскости большей размерности, поскольку 
вероятность содержать решение у них при прочих равных больше.



\section{О трудоемкости метода}
\label{complexity}

Рассмотрим вычислительную сложность задач, связанных с~решением систем 
$m$~квадратичных булевых уравнений от $n$~переменных над полем~$\mathbb{F}_2$ 
(об этом также см., в~частности,~\cite{Gor1995}).
Предположим, что $m$ ограничено некоторым полиномом от~$n$, так что 
$n<m<\mathrm{poly}\left(n\right)$. Тогда можно говорить о~полиномиальности (алгоритмов) от числа 
переменных, а~не от длины записи задачи (считаем, что полиномы в~левых частях 
уравнений системы кодируются в~виде двоичных строк одним из естественных 
способов).

Хорошо известно, что в~общем случае задача распознавания совместности системы 
такого типа NP-пол\-на (см., например,~\cite{FY1980} или~\cite[с.~321]{GJ1982}).
Задача поиска (произвольного) решения системы полна в~классе FNP~--- 
функциональном (т.\,е.\ определенном для задач поиска) аналоге класса~NP. Это 
можно доказать таким же способом, как и~в~случае распознавательных задач,~--- 
сведением к~ней задачи поиска выполняющего набора \mbox{3-КНФ}, например.
При этом задача поиска решения системы является и~NP-труд\-ной~--- в~том смысле, 
что к~ней, очевидно, сводится ее (NP-труд\-ный) распознавательный вариант.
Таким образом, если $\mathrm{P}\hm\neq\mathrm{NP}$, то полиномиальных алгоритмов 
решения систем квадратичных уравнений не существует.

Еще менее определенна ситуация с~вариантом задачи, который интересен 
в~криптоанализе: \mbox{найти} (единственное) решение заведомо совместной сис\-те\-мы 
уравнений. Современное состояние исследований не дает оснований полагать, что 
эта задача существенно проще общего случая.

Однако на практике приходится решать не произвольные (совместные) системы 
уравнений, а~лишь обладающие теми или иными свойствами. Эти частные случаи 
определяют подзадачи общей задачи, для которых уже могут найтись полиномиальные 
алгоритмы решения.

\medskip

\noindent
\textbf{Замечание~3.}
Функции $\chi_{\pi}(x)$ и~слагаемые в~аффинной нормальной форме~\eqref{AffNF} 
являются мультиаффинными функциями. Теоретико-сложностные вопросы, связанные, 
в~частности, с~решением сис\-тем мультиаффинных уравнений, а~также оценки чис\-ла 
таких функций рассматриваются в~\cite{Gor1995,GT2017}.


\medskip

Что касается собственно оценок трудоемкости предлагаемого метода, прежде всего 
оговоримся: для простоты считаем, что метод Гаусса требует выполнения~$O(nm^2)$ 
операций (для системы~$m$~уравнений от~$n$~неизвестных).
Положим также, что все локальные аффинности каждой функции~$f_i$ имеют 
одинаковую коразмерность~$r_i$ и, более того, что аффинные нормальные формы 
функций построены так, как описано в~\cite{LSF2019},~--- тогда
$r_i\hm=({1}/{2})\rank (Q_{f_i}\oplus Q_{f_i}\Tsf)\hm\leqslant 
{n}/{2}$, $1\hm\leqslant i\hm\leqslant  m$, 
где $Q_{f_i}$~--- треугольная матрица коэффициентов квад\-ра\-тич\-ной части   полинома~$f_i$.

Для начала приведем данные об алгоритме сведения произвольной полиномиальной 
системы к~квадратичной, о котором говорилось во введении.
В обозначениях книги~\cite{Bard2009} в~строящейся квад\-ра\-тичной системе будет 
$(m\hm+D\hm-2z)$ уравнений и~$(n\hm+D\hm-2z)$ переменных, где~$z$~--- 
число различных мономов 
степени не ниже третьей в~исходной системе; $D$~--- сумма их степеней. 
Если~$\Delta$~--- максимальная из степеней уравнений системы, то, считая этот 
параметр фиксированным, получаем $D\hm=O(n^{\Delta})$~\cite[\S\;11.4.6]{Bard2009}. 
Поскольку алгоритм заключается только в~добавлении новых переменных и~уравнений, 
то его трудоемкость имеет такой же порядок.

Далее, трудоемкость
построения аффинных нормальных форм для функций~$f_i$ оценена в~\cite{LSF2019}: 
приведенный там алгоритм требует выполнения порядка $(r_i2^{r_i}\hm+ n^3)$ операций. 
Значит, для всех функций имеем оценку $O(mn^3\hm+\sum\nolimits_{i=1}^m r_i2^{r_i})$. 
При этом представляется, что для специфических клас-\linebreak\vspace*{-12pt}

\columnbreak

\noindent
сов систем уравнений эта 
оценка становится полиномиальной от~$n$ (см.\ рассуждения в~\cite{LSF2019}).

Когда выбрана комбинация локальных аффинностей, на последовательное вычисление 
рангов (с~учетом приемов из разд.~\ref{optim}) нужно не более $O(n\sum\nolimits_{i=1}^m 
r_i^2)$ (а~значит, и~$O(mn^3)$) операций.
Всего комбинаций $\prod\nolimits_{i=1}^m 2^{r_i}\hm\leqslant 2^{{mn}/{2}}$.

Таким образом, в~худшем случае, когда для всех~$f_i$ матрицы~$\tilde{Q}_{f_i}\hm=
Q_{f_i}\oplus Q_{f_i}^{\mathsf{T}}$ 
имеют максимальный ранг и~нужная система~\eqref{linear-system} строится 
последней, верхняя оценка трудоемкости метода имеет вид\linebreak $O(mn^3\cdot 2^{{mn}/{2}})$.
Однако есть основания ожидать, что в~среднем благодаря возможным 
усовершенствованиям (эвристики и~т.\,п.) метод сможет работать не хуже полного 
перебора, а~на определенных име\-ющих практическое значение классах сис\-тем будет 
гораздо более эффективным.

Некоторые аргументы в~пользу эффективности метода в~среднем содержатся 
в~следующих примерах.
Рассмотрим потоковый шифр LILI-128~\cite{SDGM2000} и~его фильтрующий генератор, 
т.\,е.\ регистр сдвига с~линейными обратными связями и~фильтрующей булевой 
функцией~$f_d$ степени~$6$ от $10$~переменных. Полином Жегалкина функции~$f_d$ 
приведен в~\cite{Log2008,LYaD2007}.
Задача нахождения ключа по известной паре открытого текста $a\hm\in\mathbb{F}_2^N$ 
и~шифртекста $c\hm\in\mathbb{F}_2^N$ сводится к~решению системы уравнений 
$\{f_d(xL^t)\oplus a_t\hm=c_t\}_{t=0}^{N-1}$, где $L$~--- $(n\times n)$-мат\-ри\-ца 
линейного преобразования, реализуемого регистром сдвига, зависит от 
коэффициентов полинома обратных связей.
Таким образом, получается линейная рекуррентная последовательность~$\{x_i\}$, 
как только вычислен ее начальный отрезок $(x_0,\ldots,x_{n-1})$.

При использовании предлагаемого метода для решения данной системы при разных 
длинах~$n$ регистра и~двух вариантах $G'_n$ и~$G''_n$ полинома обратных связей 
получены следующие значения для доли несовместных систем, соответствующих 
частичным комбинациям длин~$2$ и~$3$ (т.\,е.\ комбинациям систем 
вида~\eqref{approx} для $i\hm=1, 2$ и~$i=1, 2, 3$ соответственно).
Рассмотрим примитивные полиномы над~$\mathbb{F}_2$:
\begin{align*}
  G'_{10}(y)&= y^{10}\oplus y^3\oplus 1\,;\\
  G''_{10}(y)&= y^{10}\oplus y^8\oplus y^7\oplus y^6\oplus y^5\oplus y^2\oplus 
1\,;\\
  G'_{20}(y)&= y^{20}\oplus y^3\oplus 1\,;\\
  G''_{20}(y)&= y^{20}\oplus y^9\oplus y^5\oplus y^3\oplus 1\,;\\
  G'_{30}(y)&= y^{30}\oplus y^{23}\oplus y^2\oplus y\oplus 1\,;\\
  G''_{30}(y)&= y^{30}\oplus y^{25}\oplus y^{24}\oplus y^{23}\oplus 
y^{19}\oplus y^{18}\oplus y^{16}\,\oplus\\
&\hspace*{4mm}\oplus\, y^{14}\oplus y^{11}\oplus y^8\oplus y^6\oplus y^4\oplus y^3\oplus 
y\oplus 1\,.
\end{align*}
Доля $\varepsilon_t(G)$ несовместных систем вида $xA^{J}_t\hm=b^{J}_t$ для $t\hm=2, 3$ 
и~$G\hm=G'_n,\ G''_n$ ($n\hm=10,\ 20,\ 30$), принимает следующие значения, полученные 
экспериментальным путем (при случайных начальных заполнениях регистра).
\begin{alignat*}{2}
    \varepsilon_2\left(G'_{10}\right)&= 0{,}8290; &\quad \varepsilon_3\left(G'_{10}\right)&= 0{,}9966;  \\
    \varepsilon_2\left(G'_{20}\right)&= 0{,}2870; &\quad \varepsilon_3\left(G'_{20}\right)&= 0{,}7461;  \\
    \varepsilon_2\left(G'_{30}\right)&= 0{,}1322; &\quad \varepsilon_3\left(G'_{30}\right)&= 0{,}4031;  \\
    \varepsilon_2\left(G''_{10}\right)&= 0{,}8271; &\quad \varepsilon_3\left(G''_{10}\right)&= 0{,}9966; \\
    \varepsilon_2\left(G''_{20}\right)&= 0{,}2577; &\quad \varepsilon_3\left(G''_{20}\right)&= 0{,}6628; \\
    \varepsilon_2\left(G''_{30}\right)&= 0{,}1324; &\quad \varepsilon_3\left(G''_{30}\right)&= 0{,}3948. 
\end{alignat*}

Аналогичные вычисления были проведены для фильтрующего генератора, построенного 
на основе регистра сдвига (длины $n\hm = 10,\ 20,\ 30$), с~линейными обратными 
связями, заданными теми же полиномами, и~с квадратичной фильтрующей функцией 
$f(x)\hm=x_1x_2\oplus x_3x_4\oplus x_5x_6\oplus x_7x_8\oplus x_9x_{10}$.

Результаты показывают, что для \textit{квадратичной} функции~$f$ предлагаемый 
метод (с~теми же параметрами) более эффективен, нежели для функции~$f_d$ 
степени~$6$:
\begin{alignat*}{2}
    \varepsilon_2\left(G'_{10}\right)&= 0{,}8197; &\quad \varepsilon_3\left(G'_{10}\right)&= 0{,}9964;  \\
    \varepsilon_2\left(G'_{20}\right)&= 0{,}3967; &\quad \varepsilon_3\left(G'_{20}\right)&= 0{,}8742;  \\
    \varepsilon_2\left(G'_{30}\right)&= 0{,}3275; &\quad \varepsilon_3\left(G'_{30}\right)&= 0{,}7428;  \\
    \varepsilon_2\left(G''_{10}\right)&= 0{,}8195; &\quad \varepsilon_3\left(G''_{10}\right)&= 0{,}9964; \\
    \varepsilon_2\left(G''_{20}\right)&= 0{,}3329; &\quad \varepsilon_3\left(G''_{20}\right)&= 0{,}7742; \\
    \varepsilon_2\left(G''_{30}\right)&= 0{,}3186; &\quad \varepsilon_3\left(G''_{30}\right)&= 0{,}7371. 
\end{alignat*}



Таким образом, можно предположить, что в~типичном случае б$\acute{\mbox{о}}$льшая часть 
комбинаций будет отсекаться уже после нескольких первых шагов алгоритма.

Для более точной оценки трудоемкости метода следует использовать оценки 
(математическое ожидание) таких параметров, как доля $\varepsilon_t$ несовместных 
систем для час\-тич\-ных комбинаций длины~$t$ (т.\,е.\ среди систем 
$xA^{J}_t\hm=b^{J}_t$ по всем~$J$) и~минимальное~$m_0$, такое что совместность 
системы для частичной комбинации длины~$m_0$ уже гарантирует нахождение на 
шаге~$m_0$ решения исходной системы уравнений, а~именно: обозначим через $S_t$, $2\hm\leqslant t\hm\leqslant m$, множество совместных линейных 
систем для частичных комбинаций длины~$t$, 
а~через~$m_0$~--- минимальное целое число, такое что $|S_{m_0}|\hm=1$.
Тогда в~предположении единственности решения исходной системы, пренебрегая 
возможным случаем несовместности системы для комбинации длины~$1$, имеем:


\noindent
\begin{equation}
\label{S_t}
\left.
 \begin{array}{r@{\ }c@{\ }l}
  |S_2|&=&2^{r_1}\cdot 2^{r_2}\,;\\
  &\cdots\\
  |S_{t+1}|&=&|S_t|\cdot 2^{r_{t+1}}(1-\varepsilon_{t+1})\,;\\
  &\cdots\\
  |S_{m_0}|&=&|S_{m_0-1}|\cdot 2^{r_{m_0}}(1-\varepsilon_{m_0})=1\,.
 \end{array}
 \right\}
\end{equation}
Соотношения~\eqref{S_t} соответствуют последовательному перебору и~проверке на 
совместность
систем линейных уравнений, доставляемых частичными комбинациями длин 
$t\hm=1,2,\ldots$

Обозначим через $\gamma_t$, $t\hm\leqslant m$, трудоемкость алгоритма Гаусса для систем 
линейных уравнений, описываемых частичными комбинациями, соответствующими 
$t$~первым уравнениям.
Тогда для этапа опробования локальных аффинностей~--- самого трудоемкого~--- 
потребуется двоичных операций примерно в~количестве
$\gamma_2\left\vert S_2\right\vert + \gamma_{3}
\left\vert S_{3}\right\vert + \cdots$\linebreak$\cdots + \gamma_{m_0}\left\vert S_{m_0}\right\vert$,
где величины $|S_t|$, $2\hm\leqslant t\hm\leqslant m$, выражены соотношениями~\eqref{S_t}.

Отметим, что, учитывая вид систем линейных уравнений, получаемых способом из 
работы~\cite{LSF2019} (все, кроме одного, уравнения каждой <<локальной>> системы 
содержат единственный непостоянный моном), можно рассчитывать на более быстрое 
выполнение метода Гаусса, чем в~общем случае.

\section*{Заключение}

Предлагаемый метод по своей сути принадлежит к~классу линеаризационных методов 
алгебраического криптоанализа.
При этом он обладает рядом потенциальных преимуществ, связанных с~тем, что 
вместо снижения степени уравнений посредством увеличения количества переменных 
предлагается рассматривать множество линейных систем от тех же переменных 
и~с~умеренно увеличенным числом уравнений.
Очевидно, что надежность метода равна~$1$.

Основная проблема состоит в~том, чтобы правильно определить стратегию перебора 
линейных систем, которая позволит наиболее быстро \mbox{прийти} к~системе, дающей 
искомое решение.
Кроме того, важной задачей является изучение способов оптимизации вычислений 
и~разработка эвристик для повышения эффективности метода, в~част\-ности, 
использующих свойства анализируемых криптографических преобразований 
и,~соответственно, сис\-тем полиномиальных уравнений.
Также, поскольку для общей задачи решения систем квадратичных булевых уравнений 
предположительно нет полиномиальных алгоритмов, как это обычно и~делается для 
криптоаналитических методов, следует определить классы систем полиномиальных 
уравнений, для которых метод работает наиболее быстро.

{\small\frenchspacing
 {%\baselineskip=10.8pt
 \addcontentsline{toc}{section}{References}
 \begin{thebibliography}{99}
    \bibitem{Shannon1949}
        \Au{Шеннон~К.}
        Теория связи в~секретных системах~// Работы по теории информации 
        и~кибернетике~/ Пер. с~англ.~--- М.: ИЛ, 1963. С.~333--369.
        (\Au{Shannon~C.} Communication theory of secrecy systems~// Bell 
Syst. Tech.~J., 1949. Vol.~28. Iss.~4. P.~656--715.)

    \bibitem{Bard2009}
        \Au{Bard~G.\,V.}
        Algebraic cryptanalysis.~--- Springer, 2009. 389~p.

    \bibitem{BCJ2007}
        \Au{Bard~G., Courtois~N., Jefferson~C.}
        {Efficient methods for conversion and solution of sparse systems of 
        low-degree multivariate polynomials over $\mathrm{GF}(2)$ via SAT-solvers}~//
        Cryptology ePrint Archive. Report 2007/024.
        {\sf http://eprint.iacr.org/2007/024.pdf}.

    \bibitem{LSF2019}
        \Au{Логачев~О.\,А., Сукаев~А.\,А., Федоров~С.\,Н.}
        {Полиномиальные алгоритмы вычисления локальных аффинностей квадратичных 
булевых функций}~//
        Информатика и~её применения, 2019. Т.~13. Вып.~1. С.~67--74.

    \bibitem{Log2008}
        \Au{Логачев~О.\,А.}
        {Об использовании аффинных нормальных форм булевых функций для 
определения ключей фильтрующих генераторов}~//
        Мат-лы IV Междунар. научн. конф. по проблемам безопасности 
и~противодействия терроризму.~---
        М.: МЦНМО, 2009.  Т.~2. С.~101--109.

    \bibitem{Gor1995}
        \Au{Горшков~С.\,П.}
        {Применение теории NP-полных задач для оценки сложности решения систем 
булевых уравнений}~//
        Обозрение прикладной и~промышленной математики, 1995. Т.~2. Вып.~3. 
С.~325--398.

    \bibitem{FY1980}
        \Au{Fraenkel~A.\,S., Yesha~Y.}
        {Complexity of solving algebraic equations}~//
        Inform. Process. Lett., 1980. Vol.~10. No.\,4-5. P.~178--179.

    \bibitem{GJ1982}
        \Au{Гэри~М., Джонсон~Д.}
        Вычислительные машины и~труднорешаемые задачи~/ Пер. с~англ.~---
        М.: Мир, 1982. 416~с.
        (\Au{Garey~M.\,R., Johnson~D.\,S.} Computers and intractability: 
A~guide to the theory of NP-completeness.~--- San Francisco, CA, USA: W.\,H.~Freeman 
and Co., 1979. 348~p.)

    \bibitem{GT2017}
        \Au{Горшков~С.\,П., Тарасов~А.\,В.}
        Сложность решения сис\-тем булевых уравнений.~---
        М.: Курс, 2017. 192~с.

  

    \bibitem{SDGM2000}
        \Au{Simpson~L.\,R., Dawson~E., Goli$\acute{\mbox{c}}$~J.\,Dj., Millan~W.\,L.}
        LILI keystream generator~//
        Selected areas in cryptography~/ Eds. D.\,R.~Stinson, S.~Tavares.~---
        Lecture notes in computer science ser.~---Springer, 
        2001. Vol.~2012. P.~248--261.
        
          \bibitem{LYaD2007}
        \Au{Logachev~O.\,A., Yashchenko~V.\,V., Denisenko~M.\,P.}
        {Local affinity of Boolean mappings}~//
        Boolean Functions in Cryptology and Information Security: Proc.  
NATO Advanced Study Institute.~---
        IOS Press, 2008. P.~148--172.
        
 \end{thebibliography}

 }
 }

\end{multicols}

\vspace*{-5pt}

\hfill{\small\textit{Поступила в~редакцию 01.04.19}}

\vspace*{8pt}

%\pagebreak

%\newpage

%\vspace*{-29pt}

\hrule

\vspace*{2pt}

\hrule

%\vspace*{-2pt}

\def\tit{ON LOCAL AFFINITY BASED METHOD OF~SOLVING SYSTEMS OF~QUADRATIC BOOLEAN EQUATIONS}


\def\titkol{On local affinity based method of~solving systems of~quadratic Boolean equations}

\def\aut{O.\,A.~Logachev$^1$, A.\,A.~Sukayev$^2$, and~S.\,N.~Fedorov$^2$}

\def\autkol{O.\,A.~Logachev, A.\,A.~Sukayev, and~S.\,N.~Fedorov}

\titel{\tit}{\aut}{\autkol}{\titkol}

\vspace*{-11pt}


\noindent
$^1$Institute of Informatics Problems, Federal Research Center ``Computer Science and 
Control'' of the Russian\linebreak
$\hphantom{^1}$Academy of Sciences, 44-2~Vavilov Str., Moscow 119333, 
Russian Federation

\noindent
$^2$M.\,V.~Lomonosov Moscow State University, 
1~Michurinskiy Prosp., Moscow 119192, Russian Federation


\def\leftfootline{\small{\textbf{\thepage}
\hfill INFORMATIKA I EE PRIMENENIYA~--- INFORMATICS AND
APPLICATIONS\ \ \ 2019\ \ \ volume~13\ \ \ issue\ 2}
}%
 \def\rightfootline{\small{INFORMATIKA I EE PRIMENENIYA~---
INFORMATICS AND APPLICATIONS\ \ \ 2019\ \ \ volume~13\ \ \ issue\ 2
\hfill \textbf{\thepage}}}

\vspace*{6pt}


\Abste{Solving nonlinear systems of Boolean equations is NP-hard. 
Nevertheless, certain classes of such systems can be solved by efficient algorithms. 
There are theoretical and applied reasons for studying these classes and designing 
corresponding efficient algorithms.
    The paper proposes an approach to solving the systems of quadratic equations 
    over two-element field. The method makes use of the quadratic functions' 
    representation by their affine normal forms, i.\,e., in a sense, of their 
    piecewise affine approximation. So, the initial nonlinear instance comes to 
    a~number of linear equations systems of the same variables. The paper
     also discusses possible ways to reduce the complexity of the proposed method.}

\KWE{Boolean function; system of quadratic Boolean equations; vector space partition; 
flat; local affinity; affine normal form; algebraic cryptanalysis}



 \DOI{10.14357/19922264190206}

\vspace*{-16pt}

 \Ack
\noindent
The work was partly
supported by the Russian Foundation for Basic Research 
(project 18-29-03124~mk).


%\vspace*{-2pt}

  \begin{multicols}{2}

\renewcommand{\bibname}{\protect\rmfamily References}
%\renewcommand{\bibname}{\large\protect\rm References}

{\small\frenchspacing
 {%\baselineskip=10.8pt
 \addcontentsline{toc}{section}{References}
 \begin{thebibliography}{99}
 
 \vspace*{-2pt}
 
\bibitem{1-log}
\Aue{Shannon, C.} 1949. Communication theory of secrecy systems.
\textit{Bell Syst. Tech.~J.} 28 (4):656--715.
\bibitem{2-log}
\Aue{Bard, G.\,V.} 2009. \textit{Algebraic cryptanalysis}. Springer. 389~p.
\bibitem{3-log}
\Aue{Bard, G., N.~Courtois, and C.~Jefferson.} 
2007. Efficient methods for conversion and solution of sparse systems of 
low-degree multivariate polynomials over GF(2) via SAT-solvers. 
\textit{Cryptology ePrint Archive}. Report 2007/024. Available at: 
{\sf http://eprint.iacr.org/2007/024.pdf} (accessed August~30, 2018).
\bibitem{4-log}
\Aue{Logachev, O.\,A., A.\,A.~Sukayev, and S.\,N.~Fedorov.} 
2019. Polinomial'nye algoritmy vychisleniya lokal'nykh affinnostey 
kvadratichnykh bulevykh funktsiy [Polynomial algorithms for constructing 
local affinities of quadratic Boolean functions]. 
\textit{Informatika i~ee Primeneniya~--- Inform. Appl.} 13(1):67--74.
\bibitem{5-log}
\Aue{Logachev, O.\,A.} 2009. Ob ispol'zovanii affinnykh normal'nykh 
form bulevykh funktsiy dlya opredeleniya klyuchey fil'truyushchikh generatorov 
[On using affine normal forms of Boolean functions for identifying filter generator 
keys]. 
\textit{4th  Conference (International) on Security and
 Counteraction to Terrorism Proceedings}. Moscow. 2:101--109.
\bibitem{6-log}
\Aue{Gorshkov, S.\,P.} 1995. Primenenie teorii NP-polnykh 
zadach dlya otsenki slozhnosti resheniya sistem bulevykh uravneniy 
[Application of the NP-complete problem theory to assessment of 
complexity of solving the systems of Boolean equations].
\textit{Obozrenie prikladnoy i~promyshlennoy matematiki} 
[Applied and Industrial Mathematics Review] 2(3):325--398.
\bibitem{7-log}
\Aue{Fraenkel, A.\,S., and Y.~Yesha.}
1980. Complexity of solving algebraic equations. 
\textit{Inform. Process. Lett.} 10(4-5):178--179.
\bibitem{8-log}
\Aue{Garey, M.\,R., and D.\,S.~Johnson.} 1979. \textit{Computers and intractability: 
A~guide to the theory of NP-completeness}. San Francisco, CA: W.\,H.~Freeman and Co. 
348~p.
\bibitem{9-log}
\Aue{Gorshkov, S.\,P., and A.\,V.~Tarasov.} 
2017. \textit{Slozhnost' re\-she\-niya sistem bulevykh uravneniy} [Complexity of solving 
the systems of Boolean equations]. Moscow: Kurs. 192~p.

\bibitem{11-log}
\Aue{Simpson, L.\,R., E.~Dawson, J.\,Dj.~Goli$\acute{\mbox{c}}$, and W.\,L.~Millan.}
 2001. LILI keystream generator. 
 \textit{Selected areas in cryptography}. Eds. D.\,R.~Stinson and S.~Tavares.
 Lecture notes in computer science ser. Springer. 2012:248--261.
 
 \bibitem{10-log}
\Aue{Logachev, O.\,A., V.\,V.~Yashchenko, and M.\,P.~Denisenko.}
 2008. Local affinity of Boolean mappings. \textit{Boolean Functions in 
 Cryptology and Information Security: Proc. NATO Advanced Study Institute}. 
 IOS Press. 148--172.
 
\end{thebibliography}

 }
 }

\end{multicols}

\vspace*{-6pt}

\hfill{\small\textit{Received April 1, 2019}}

%\pagebreak

%\vspace*{-18pt}


\Contr


\noindent
\textbf{Logachev Oleg A.} (b.\ 1950)~--- 
Candidate of Science (PhD) in physics and mathematics, senior scientist, Institute 
of Informatics Problems, Federal Research Center ``Computer Science and Control'' 
of the Russian Academy of Sciences, 44-2~Vavilov Str., Moscow 119133, 
Russian Federation; \mbox{ollog@inbox.ru}

\vspace*{3pt}


\noindent
\textbf{Sukayev Al'bert A.} (b.\ 1994)~--- 
student, Information Security Institute, M.\,V.~Lomonosov University, 
1~Michurinskiy Prosp., Moscow 119192, Russian Federation; \mbox{asukaev@gmail.com}

\vspace*{3pt}


\noindent
\textbf{Fedorov Sergey N.} (b.\ 1982)~--- 
Candidate of Science (PhD) in physics and mathematics, senior scientist, 
Information Security Institute, M.\,V.~Lomonosov Moscow State University, 
1~Michurinskiy Prosp., Moscow 119192, Russian Federation; 
\mbox{s.n.feodorov@yandex.ru}
\label{end\stat}

\renewcommand{\bibname}{\protect\rm Литература}