\def\stat{zatsar}

\def\tit{МОДЕЛИРОВАНИЕ ПРОЦЕССОВ СЕТЕВОГО ПЛАНИРОВАНИЯ ПОРТФЕЛЯ 
ПРОЕКТОВ С~НЕОДНОРОДНЫМИ РЕСУРСАМИ В~УСЛОВИЯХ НЕЧЕТКОЙ 
ИНФОРМАЦИИ$^*$}

\def\titkol{Моделирование процессов сетевого планирования портфеля 
проектов с~неоднородными ресурсами} % в~условиях нечеткой информации}

\def\aut{А.\,А.~Зацаринный$^1$, В.\,В.~Коротков$^2$, М.\,Г.~Матвеев$^3$}

\def\autkol{А.\,А.~Зацаринный, В.\,В.~Коротков, М.\,Г.~Матвеев}

\titel{\tit}{\aut}{\autkol}{\titkol}

\index{Зацаринный А.\,А.}
\index{Коротков В.\,В.}
\index{Матвеев М.\,Г.}
\index{Zatsarinny A.\,A.}
\index{Korotkov V.\,V.}
\index{Matveev M.\,G.}


{\renewcommand{\thefootnote}{\fnsymbol{footnote}} \footnotetext[1]
{Работа выполнена при частичной финансовой поддержке РФФИ (проект 17-01-00251).}}


\renewcommand{\thefootnote}{\arabic{footnote}}
\footnotetext[1]{Федеральный исследовательский центр <<Информатика и~управление>> Российской академии наук, 
\mbox{AZatsarinny@ipiran.ru}}
\footnotetext[2]{Воронежский государственный университет, факультет компьютерных наук, 
\mbox{chasecrunk@gmail.com}}
\footnotetext[3]{Воронежский государственный университет, факультет компьютерных наук, 
\mbox{mgmatveev@yandex.ru}}
%\vspace*{-2pt}
  
  
  \Abst{Рассматривается задача планирования портфеля проектов в~условиях ограничений 
на имеющиеся трудовые ресурсы различной специализации и~нечетких оценок трудозатрат 
выполнения работ. Был использован аппарат нечеткой W-ал\-геб\-ры, позволивший избежать 
недостатков и~трудностей традиционных подходов: неоправданного расширения носителя, 
необходимости задания операции сравнения нечетких чисел и~ряда других. Предложенная 
модель позволяет получить нечеткие оценки времени выполнения проектов и~их работ, 
а~также оптимальное распределение исполнителей по работам с~учетом их возможной 
частичной занятости. Для решения задачи был реализован генетический алгоритм, 
основанный на кодировании особей в~виде списка приоритетов работ. Приводится 
численный пример, демонстрирующий перспективы применения данного подхода 
проектными организациями при совершении оценок, необходимых для принятия решений 
в~условиях неопределенности.}
  
  \KW{управление проектами; W-алгебра; нечеткая арифметика; комбинаторная 
оптимизация; генетический алгоритм}

\DOI{10.14357/19922264190213}
  
\vspace*{1pt}


\vskip 10pt plus 9pt minus 6pt

\thispagestyle{headings}

\begin{multicols}{2}

\label{st\stat}
  
\section{Введение}

\vspace*{-1pt}
  
  В условиях, когда ключевой национальной \mbox{целью} России стал  
на\-уч\-но-тех\-но\-ло\-ги\-че\-ский прорыв и~руководством страны взят курс на 
реализацию Программы цифровой экономики, объективно возросла значимость 
системного подхода к~процес\-сам планирования и~управления выполнением 
масштабных проектов~[1]. Важнейшими инструментами для реализации такого 
подхода должны стать результаты опережающих фундаментальных 
исследований, направленных в~том числе и~на совершенствование 
методических подходов к~сетевому планированию~[2].
  
  Процессы управления проектами включают обоснование планов поэтапного 
выполнения работ в~рамках комплекса проектов при заданных ресурсах. Такие 
планы предполагают составление сетевых план-гра\-фи\-ков и~расписания 
выполнения работ с~учетом заданных сроков и~выделенных ресурсов~[3]. 
  
  В основе управления лежит план-факт\-ный анализ, который в~определенные 
моменты времени показывает степень рассогласования планового 
и~фактического расписания выполнения работ и~является первоосновой для 
перераспределения ресурсов и~составления нового сетевого плана. Проб\-ле\-ма 
управления особенно остро встает при реализации портфеля проектов, когда 
добавление нового проекта в~портфель выполняемых проектов сразу требует 
плановой оценки времени его завершения, т.\,е.\ времени завершения последней 
работы в~сети, и~плановой оценки потребляемых работами ресурсов. Без таких 
оценок невозможно обеспечить эффективность проектной деятельности, так 
как на них основан расчет финансовых отношений заказчика проекта 
и~проектной организации.
  
  Таким образом, проблема сетевого планирования проектов с~ограниченными 
ресурсами (Resource Constrained Project Scheduling Problem, \mbox{RCPSP}) становится 
одной из самых актуальных прикладных задач в~деятельности практически 
каж\-дой проектной организации. Параметры в~задаче сетевого планирования 
чаще всего не являются детерминированными величинами. В~част\-ности, 
нельзя точно указать продолжительности выполнения работ на стадии 
планирования, можно говорить лишь о~приблизительных сроках. 
Воз\-ни\-ка\-ющую неопределенность часто пытаются описать с~по\-мощью 
статистических методов~[4]. Но статистические выборки не всегда релевантны 
особенностям конкретных работ конкретной проектирующей организации. 
Экспертная оценка~--- более подходящий способ задания параметров работ 
проекта. При этом адекватным аппаратом описания значений параметров для 
эксперта служит нечеткая математика~[5]. Применение нечеткой математики 
для управления проектами давно привлекает внимание исследователей как 
перспективное на\-прав\-ле\-ние~[6]. В~работе~[7] приведена классификация 
методов решения задачи сетевого планирования с~нечеткими параметрами 
с~указанием их недостатков, главные из которых~--- громоздкость вычислений, 
искажение естественных свойств и~отношений классических моделей, 
неоправданное расширение носителя результата, что сказывается на 
возможности его интерпретации и~полезности практического применения. 
В~работах~[8, 9] для устранения указанных недостатков предлагается 
использовать W-ал\-геб\-ру и~W-чис\-ла. В~работе~[10] приводится пример 
использования W-ал\-геб\-ры для решения задачи планирования работ проекта 
с~фиксированными однородными ресурсами и~нечетко заданными 
продолжительностями работ методами линейного программирования. Однако, 
как отмечается в~[11], такой подход нельзя использовать при планировании 
с~ограниченными переменными ресурсами. В~этом случае приходится 
обращаться к~эвристическим алгоритмам квазиоптимального выбора. 
В~работах~[12--14] предлагается использовать различные модификации 
генетических алгоритмов, которые существенно упрощают решение задачи 
даже для планирования портфеля проектов с~переменными разнородными 
ресурсами, но при детерминированном задании параметров работ.
  
  В статье предлагается решение задачи сетевого планирования портфеля 
проектов с~неоднородными ограниченными ресурсами. Длительность работ 
задается нечеткими W-чис\-ла\-ми и~зависит от выделенных ресурсов. 
Результатом решения является нечеткий сетевой график выполнения работ, 
позволяющий рассчитать нечеткую оценку длительности выполнения каждого 
планируемого проекта, а также нечеткую оценку плановой потребности 
в~неисчерпаемых ресурсах и~степень их загруженности.
  
\section{Формализация задачи сетевого планирования}
  
  Под проектом понимается комплекс мероприятий, направленных на 
достижение к~заданному сроку определенных целей в~условиях ограничений на 
выделенные ресурсы~\cite{6-zac}. Далее под ресурсами будем понимать 
трудовые ресурсы, т.\,е.\ исполнителей работ проектов.
  
  Рассматривается портфель проектов~$P$, со\-сто\-ящий из~$n$~проектов. 
Каждый проект $p\hm\in P$ включает множество работ~$J_p$. Предполагается, 
что работы не могут быть прерваны в~процессе своего выполнения. Каждый 
проект $p\hm\in P$ также имеет дату старта $\mathrm{Start}_p$, раньше которой не может 
быть начата ни одна из его работ. Некоторые работы требуют завершения 
других перед своим началом, поэтому множество $\mathrm{Pre}(j)$ задает всех 
предшественников работы $j\hm\in J_p$.
  
  Выполнение работ требует привлечения исполнителей соответствующей 
специализации. Введем множество всех возможных специализаций~$E$ 
и~пусть отображение $\mathrm{Spec}:\ J_p\hm\to E \,\forall p\hm\in P$ ставит 
в~соответствие всякой работе любого проекта специализацию, необходимую 
для ее выполнения. На каждый проект~$p$ выделено $c_{pe}^{\mathrm{local}}$ 
исполнителей специализации $e\hm\in E$, а~кроме того возможно привлечение 
дополнительных ресурсов, имеющихся в~количестве~$c_e^{\mathrm{global}}$. 
Дополнительные ресурсы не привязаны к~ка\-ко\-му-ли\-бо из проектов и~могут 
привлекаться для выполнения работ любого проекта. Предполагается, что 
исполнители не могут временно прерывать свое участие в~работе. Допускается 
частичная занятость исполнителя в~работе. 
  
  Примем, что длительность работы $j\hm\in J_p$ зависит от числа ее 
исполнителей $\tilde{l}_{pj}$:
  \begin{equation*}
  \tilde{d}_{pj}=\fr{\tilde{q}_{pj}}{\tilde{l}_{pj}}\ \forall p\in P,\ j\in J_p\,,
 % \label{e1-zac}
  \end{equation*}
где $\tilde{d}_{pj}$~--- нечеткая длительность работы $j\hm\in J_p$; 
$\tilde{q}_{pj}$~--- нечеткие трудозатраты; $\tilde{l}_{pj}$~--- нечеткое число 
исполнителей, которое рассматривается как числовая интерпретация 
возможности исполнителя <<работать за двоих>> или возможности обратного 
эффекта, при этом фактическое число исполнителей будет соответствовать 
моде нечеткого числа. Нечеткие переменные~$\tilde{q}_{pj}$, $\tilde{l}_{pj}$ 
и~$\tilde{d}_{pj}$ в~соответствии с~\cite{9-zac} представлены парами функций 
вида $(q^L_{pj}(\alpha); q^R_{pj}(\alpha))$, $(l^L_{pj}(\alpha); l^R_{pj}(\alpha))$ 
и~$(d^L_{pj}(\alpha); d^R_{pj}(\alpha))$, где $\alpha\hm\in [0;1]$~--- параметр 
нечеткости. Такое пред\-став\-ле\-ние под\-разуме\-ва\-ет возможность использования 
как треугольных, так и~трапецеидальных нечетких чисел. Далее для сокращения 
записей будем использовать только треугольные нечеткие числа.
  
  Пусть $\tilde{s}_{pj}$~--- нечеткое время начала работы $j\hm\in J_p$, тогда 
время ее окончания определяется как
  \begin{equation*}
  \tilde{f}_{pj}=\tilde{s}_{pj}+\tilde{d}_{pj}\ \forall p\in P\,,\ j\in J_p\,.
  %\label{e2-zac}
  \end{equation*}
    
  Целью является нахождение для каждой задачи $j\hm\in J_p$ каждого из 
проектов $p\hm\in P$ такого времени старта~$\tilde{s}_{pj}$ и~числа 
выделенных трудовых ресурсов~$\tilde{l}_{pj}$, которые не нарушают 
ограничения предшествования и~ресурсные ограничения, а также обеспечивают 
минимальное взвешенное суммарное время выполнения 
проектов~$\tilde{f}_{\mathrm{TPD}}$:
  \begin{equation*}
  \tilde{f}_{\mathrm{TPD}}= \sum\limits_{p\in P} \omega_p \left( \tilde{f}_p-
\tilde{s}_p\right)\,.
 % \label{e3-zac}
  \end{equation*}
  Здесь $\tilde{f}_p\hm= \max\nolimits_{j\in J_p} \left( \tilde{f}_{pj}\right)$~--- 
время завершения последней работы проекта; $\omega_p$~--- весовой 
коэффициент, определяющий приоритетность проекта.
  
  Таким образом, может быть сформулирована следующая оптимизационная 
задача:
  \begin{equation}
  \tilde{f}_{\mathrm{TPD}}\to \min
  \label{e4-zac}
  \end{equation}
при условиях
\begin{gather}
\tilde{l}_{pj}\geq 1\ \forall p\in P\,,\  j\in J_p\,;\label{e5-zac}\\
\hspace*{-10mm}\sum\limits_{p\in P} \sum\limits_{j\in A_{pe}(t)}\max\left( 0; \tilde{l}_{pj}-
c_{pe}^{\mathrm{local}} \right) \leq c_e^{\mathrm{global}}\notag\\
\hspace*{30mm} \forall e\in E\,,\ t\in [0,T]\,;\label{e6-zac}\\
A_{pe}(t)={}\hspace*{50mm}\notag\\
{}=\left\{ i\in J_p\vert \tilde{s}_{pj}\leq t< \tilde{s}_{pj} +\tilde{d}_{pj}\,, \ 
\mathrm{Spec}\,(j)=e\right\}\,;\label{e7-zac}\\
\tilde{s}_{pj}+\tilde{d}_{pj}\leq \tilde{s}_{pj^\prime}\ \forall p\in P\,,\ j^\prime \in 
J_p\,,\ j\in \mathrm{Pre}\left(j^\prime\right)\,;\label{e8-zac}\\
\tilde{s}_{pj}\geq \tilde{s}_p\ \forall p\in P\,,\ j\in J_p\,.
\label{e9-zac}
\end{gather}
  Здесь $T$~--- это верхняя оценка времени завершения всех проектов из 
портфеля; $A_{pe}(t)$~--- множество работ проекта~$p$, требующих 
специализации~$e$ и~находящихся в~состоянии выполнения в~момент 
времени~$t$. Согласно~(\ref{e5-zac}) у каждой задачи должен быть хотя бы 
один исполнитель, (\ref{e6-zac}) задает ограничение на число единовременно 
используемых трудовых ресурсов, (\ref{e8-zac}) отражает ограничения 
предшествования, а~(\ref{e9-zac})~--- ограничение на дату старта проекта.
  
\section{Описание алгоритма решения задачи сетевого 
планирования}
  
  В случае применения известных арифметик нечетких чисел, основанных на 
принципе обобщения Л.~Заде~\cite{5-zac}, в~дополнение к~проблемам, 
обозначенным во введении, решение поставленной задачи потребует задания 
операции сравнения нечетких треугольных чисел. Известные варианты 
реализации операции сравнения не обеспечивают однозначного 
результата~\cite{5-zac, 8-zac}. Использование же W-ал\-геб\-ры при решении 
задачи проектного планирования, как продемонстрировано  
в~работе~\cite{10-zac}, позволяет избежать сравнения нечетких чисел, так как 
результат может быть представлен как линейная комбинация решений четких 
задач. Представим нечеткие трудозатраты в~виде выпуклой линейной 
комбинации граничных значений левой~$q^L_{pj}(\alpha)$ 
и~правой~$q^R_{pj}(\alpha)$ функций W-чис\-ла при $\alpha\hm=0$ 
и~$1$:
  \begin{equation}
  \left.
  \begin{array}{rl}
  q^L_{pj}(\alpha)&= q^L_{pj}(0)+\left( q^L_{pj}(1)-q^L_{pj}(0)\right)\alpha\,;\\[6pt]
  q^R_{pj}(\alpha) &=q^R_{pj}(0) +\left( q^R_{pj}(1)-q^R_{pj}(0)\right)\alpha\,.
  \end{array}
  \right\}
  \label{e10-zac}
  \end{equation}
  
  Тогда для треугольных чисел необходимо решить отдельно четкие задачи для 
трех детерминированных значений:  $q^L_{pj}(0)$;
$q^L_{pj}(1)\hm=q^R_{pj}(1)$; $q^R_{pj}(0)$. Полученные 
решения могут быть использованы для представления результата в~виде  
W-чи\-сел~(\ref{e10-zac}). Например, нечеткое число 
исполнителей~$\tilde{l}_{pj}$, выделенных на выполнение работы, будет иметь 
вид:
  \begin{align*}
  l^L_{pj}(\alpha) &=l^L_{pj}(0) +\left( l^L_{pj}(1) - l^L_{pj}(0)\right)\alpha\,; 
\\
  l^R_{pj}(\alpha) &=l^R_{pj}(0)+\left( l^R_{pj}(1)-l^R_{pj}(0)\right)\alpha\,.
  %  \label{e11-zac}
  \end{align*}
  
  Теперь алгоритм решения задачи~(\ref{e4-zac})--(\ref{e9-zac}) можно 
рассматривать как решение трех детерминированных задач с~их последующей 
комбинацией вида~(\ref{e10-zac}).
  
  Доказано, что задача проектного планирования в~условиях ресурсных 
ограничений относится\linebreak к~классу NP-труд\-ных~\cite{15-zac}, поэтому 
большинство исследований в~данной области рассматривают возможности 
применения различных эвристических подходов для поиска субоптимальных 
решений~\cite{16-zac}.
  
  В данной работе для решения поставленной оптимизационной задачи был 
реализован генетический алгоритм, сочетающий ряд подходов из  
работ~\cite{12-zac, 13-zac, 14-zac}. Подробности реализации приводятся далее.

 \begin{table*}[b]\small %tabl1
 \vspace*{-14pt}
  \begin{center}
  \Caption{Трудовые ресурсы проектов}
  \vspace*{2ex}
  
  \begin{tabular}{|l|c|c|c|}
  \hline %\multicolumn{1}{|c|}{\raisebox{-6pt}[0pt][0pt]{
\multicolumn{1}{|c|}{\raisebox{-6pt}[0pt][0pt]{Ресурсы}}&
\multicolumn{3}{c|}{Число работников}\\
\cline{2-4}
&Специализации 1&Cпециализации 2&Cпециализации 3\\
\hline
Проекта 1&5&3&3\\
Проекта 2&2&4&3\\
Дополнительные&2&3&1\\
\hline
\end{tabular}
\end{center}
%\vspace*{-6pt}
\end{table*}
  
  \subsection{Представление решения}
  
Возможное решение задачи кодируется хромосомой, которая соответствует 
ровно одному возможному расписанию работ и~распределению\linebreak трудовых 
ресурсов. Каждый ген хромосомы пред\-став\-лен идентификатором работы 
и~числом выделенных на нее исполнителей. При этом порядок\linebreak генов 
в~хромосоме отражает приоритет соответствующих работ. Расписание работ 
может быть получено по данному представлению в~результате следующей 
последовательной процедуры. Начиная с~пустого расписания, будем получать 
новое промежуточное расписание путем размещения следующей по приоритету 
работы на самое раннее время так, чтобы не нарушить ограничения 
предшествования и~ресурсные ограничения. Такая процедура называется 
последовательной схемой генерации расписания. Стоит отметить, что, хотя 
каждому закодированному таким образом решению соответствует ровно одно 
возможное расписание, обратное не верно. Одинаковое расписание может быть 
получено из различных решений, вследствие чего при таком подходе 
повторный обход уже рассмотренных решений будет происходить чаще. 
Однако, так как любое получаемое решение всегда будет удовлетворять всем 
необходимым требованиям, это позволяет избежать модификации функции 
приспособленности внесением штрафа за нарушение ограничений. Поэтому 
в~качестве функции приспособленности использовалась величина, обратная 
суммарному времени выполнения проектов~(\ref{e4-zac}). Кроме того, 
использование списка приоритетов решает проблему непрерывного времени, 
которая может возникать при непосредственном кодировании времен начала 
работ.

\vspace*{-6pt}
  
  \subsection{Начальная популяция}
  
  Для обеспечения разнообразия особей начальная популяция генерируется 
случайным образом. Процедура получения случайной хромосомы начинается 
с~пустого списка генов. На каждом шаге процедуры в~список добавляется ген, 
кодирующий случайную работу $j\hm\in J_p$ из списка тех работ, чьи 
предшественники уже были добавлены в~хромосому ранее. Число 
исполнителей, выделенных на эту работу, определяется случайным образом 
в~промежутке $[1; c_{pe}^{\mathrm{local}}\hm+c_e^{\mathrm{global}}]$ с~учетом шага 
дискретизации~$0{,}1$, где $\mathrm{Spec}(j)\hm=e$.
  
  Получаемые экземпляры очевидным образом удовлетворяют ограничениям 
предшествования. Теперь необходимо таким образом определить операции 
скрещивания и~мутации, чтобы получаемые решения также не нарушали 
необходимые требования.

\vspace*{-6pt}
  
  \subsection{Метод селекции}
  
  На каждой итерации алгоритма необходимо отбирать особей, подлежащих 
дальнейшему скрещиванию с~учетом их оценки приспособленности. В~работе 
был использован турнирный метод отбора с~размером подгруппы, равным~3.

\vspace*{-6pt}
  
  \subsection{Операция скрещивания}
  
  Операция скрещивания позволяет скомбинировать два родительских 
решения, чтобы получить двух потомков, наследующих признаки родителей. 
Операция применяется независимо к~спискам работ и~соответствующим 
спискам выделенных ресурсов.
  
  Сначала случайным образом определяется точка~$k$, делящая родительские 
хромосомы на две час\-ти. Все работы из первой части первого (второго) 
родителя в~том же порядке перемещаются в~первого (второго) наследника, 
а~оставшиеся работы до\-бав\-ля\-ют\-ся в~том порядке, в~каком они расположены во 
втором (первом) родителе.


  
  Распределения ресурсов скрещиваются согласно равномерной схеме. Для 
каждой $i$-й работы независимым образом берется случайное число $p_i\hm \in 
[0,1]$. Если $p_i\hm\geq 0{,}5$, то первый (второй) ребенок наследует число 
исполнителей этой работы от первого (второго) родителя. В~противном случае 
первый (второй) ребенок наследует число исполнителей от второго (первого) 
родителя.

\vspace*{-6pt}
  
  \subsection{Операция мутации}
  
  Операция мутации вносит случайные изменения в~решения, получаемые 
после скрещивания. Операция мутации также применяется отдельно 
к~приоритетам работ и~выделенным ресурсам.
  
  Для каждой $i$-й работы независимо берется случайное число $p_i\hm\in 
[0,1]$. Если $p_i\hm\leq P_{\mathrm{PM}}$, где $P_{\mathrm{PM}}$~--- заранее заданная 
вероятность мутации, то $i$-я\linebreak и~$(i\hm+1)$-я работы меняются местами, если 
это не нарушает ограничений предшествования.
  
  Число исполнителей каждой работы с~ве\-ро\-ят\-ностью~$P_{\mathrm{MM}}$ меняется на 
случайное, взятое аналогичным с~описанным в~разделе о генерации начальной 
популяции образом.

\vspace*{-6pt}
  
\section{Численный пример и~обсуждение результатов}

\vspace*{-2pt}
  
  Решение задачи~(\ref{e4-zac})--(\ref{e9-zac}) иллюстрируется на следующем 
тестовом примере. Пусть портфель состоит из двух одинаково приоритетных 
проектов. Число выделенных на проекты трудовых ресурсов трех 
специализаций приведено в~табл.~1. В~табл.~2 указаны структурные 
и~ресурсные требования для работ обоих проектов. Нечеткие переменные 
в~табл.~2 и~3\linebreak\vspace*{-12pt}

\pagebreak

\end{multicols}

\begin{table*}\small %tabl2
  \begin{center}
  \Caption{Список работ проектов}
  \vspace*{2ex}
  
  \begin{tabular}{|c|c|c|c|c|c|}
  \hline
Проект&\tabcolsep=0pt\begin{tabular}{c}Дата\\ запуска \\ 
проекта\end{tabular}&
Работа&Специализация&Предшественники&\tabcolsep=0pt\begin{tabular}{c}
Нечеткие\\ трудозатраты, $\tilde{q}_{pj}$\end{tabular}\\
\hline
\multicolumn{1}{|c|}{\raisebox{-36pt}[0pt][0pt]{№\,1}}&
\multicolumn{1}{c|}{\raisebox{-36pt}[0pt][0pt]{0}}&1&2&&(14; 16; 17)\\
&&2&1&&(18; 20; 21)\\
&&3&3&&(17; 19; 20)\\
&&4&1&1, 2&(17; 17; 20)\\
&&5&3&4&(24; 25; 27)\\
&&6&2&5&(23; 25; 27)\\
&&7&1&5&(31; 33; 35)\\
&&8&3&3, 6, 7&(30; 31; 33)\\
\hline
\multicolumn{1}{|c|}{\raisebox{-36pt}[0pt][0pt]{№\,2}}&
\multicolumn{1}{c|}{\raisebox{-36pt}[0pt][0pt]{4}}&1&2&&(8; 9; 11)\\
&&2&3&1&(14; 16; 17)\\
&&3&1&&(21; 22; 24)\\
&&4&2&3&(17; 19; 20)\\
&&5&1&2, 4&(15; 16; 18)\\
&&6&3&4&(40; 42; 42)\\
&&7&3&5, 6&(13; 13; 13)\\
&&8&1&2, 4&(11; 11; 13)\\
  \hline
  \end{tabular}
  \end{center}
  \vspace*{-3pt}
  %\end{table*}
  %  
  %\begin{table*}\small %tabl3
  \begin{center}
  \Caption{Результаты решения задачи}
  \vspace*{2ex}
  
  \begin{tabular}{|c|c|c|c|c|c|c|c|c|}
  \hline
\multicolumn{1}{|c|}{\raisebox{-8pt}[0pt][0pt]{Проект}}&
\multicolumn{1}{c|}{\raisebox{-8pt}[0pt][0pt]{Работа}}&\multicolumn{3}{c|}{Нечеткое время старта работы}&
\tabcolsep=0pt\begin{tabular}{c}Число \end{tabular}&
\multicolumn{3}{c|}{Нечеткое время окончания работы}\\
\cline{3-5}
&&&&&&&&\\[-9pt]
\cline{7-9}
&&&&&&&&\\[-9pt]
&&\ \ \ \ \ $s^L_{pj}$\ \ \ \ \ &\ \ \ \ \  $s^m_{pj}$\ \ \ \ \ &$s^R_{pj}$&исполнителей&
\ \ \ \ \ \ \ $f^L_{pj}$\ \ \ \ \ \ \  &\ \ \ \ \ \ \ $f^m_{pj}$\ \ \ \ \ \ \ &$f^R_{pj}$\\[3pt]
\hline
&&&&&&&&\\[-9pt]
\multicolumn{1}{|c|}{\raisebox{-36pt}[0pt][0pt]{№\,1}}&1&0\hphantom{,00}&0\hphantom{,00}&0\hphantom{,00}&3\hphantom{,9}&4,67&\hphantom{9}5,33&\hphantom{9}4,72\\
&2&0\hphantom{,00}&0\hphantom{,00}&0\hphantom{,00}&5,3&2,69&\hphantom{9}3,77&\hphantom{0}6,18\\
&3&0\hphantom{,00}&0\hphantom{,00}&2,69&1,9&7,05&10\hphantom{,99}&10,53\\
&4&4,67&5,33&6,18&5\hphantom{,9}&8,07&\hphantom{9}8,73&\hphantom{9}9,12\\
&5&8,07&8,73&9,12&3\hphantom{,9}&15,57\hphantom{0}&17,07&16,83\\
&6&15,57\hphantom{9}&17,07\hphantom{9}&16,83\hphantom{9}&5,9&19,6\hphantom{99}&21,3\hphantom{9}&22,02\\
&7&15,57\hphantom{9}&17,07\hphantom{9}&16,83\hphantom{9}&6,8&20,26\hphantom{9}&21,92&22,48\\
&8&20,26\hphantom{9}&21,92\hphantom{9}&22,48\hphantom{9}&3,9&28,16\hphantom{9}&29,87&30,73\\
\hline
\multicolumn{1}{|c|}{\raisebox{-36pt}[0pt][0pt]{№\,2}}&1&4\hphantom{,99}&4\hphantom{,99}&4\hphantom{,99}&3,8&6,29&\hphantom{9}6,37&5,8\\
&2&6,29&6,37&5,8\hphantom{9}&3,2&10,29\hphantom{9}&11,37&13,19\\
&3&4\hphantom{,99}&4\hphantom{,99}&4\hphantom{,99}&4\hphantom{,9}&9,25&9,5&10\hphantom{,00}\\
&4&9,25&9,5\hphantom{9}&10\hphantom{,999}&6,8&11,83\hphantom{9}&12,29&13,28\\
&5&11,83\hphantom{9}&12,29\hphantom{9}&13,28\hphantom{9}&2,3&18,97\hphantom{9}&19,25&24,53\\
&6&11,83\hphantom{9}&12,29\hphantom{9}&13,28\hphantom{9}&4\hphantom{,9}&22,64\hphantom{9}&22,79&25,63\\
&7&22,64\hphantom{9}&22,79\hphantom{9}&25,63\hphantom{9}&2,8&26,83\hphantom{9}&27,44&30,11\\
&8&11,83\hphantom{9}&21,92\hphantom{9}&13,28\hphantom{9}&3,3&20,29\hphantom{9}&25,25&19,19\\
\hline
\end{tabular}
\end{center}
\vspace*{-3pt}
\end{table*}

  \begin{multicols}{2}
  
  \begin{figure*}
   \vspace*{1pt}
    \begin{center}  
  \mbox{%
 \epsfxsize=144.026mm 
 \epsfbox{zac-1.eps}
 }

\vspace*{3pt}

    {\small Диаграмма Ганта для полученного нечеткого решения для
    проектов~1~(\textit{а}) и~2~(\textit{б})}
     \end{center}
          \vspace*{-9pt}
  \end{figure*}
  
  
   \noindent
   для удобства восприятия заданы тремя точками~--- на границах 
при $\alpha\hm=0$ и~в~моде при $\alpha\hm=1$.
  
 
  
  При решении использовались следующие параметры генетического 
алгоритма: размер популяции~--- 150; число поколений~--- 40; вероятность 
мутации~--- 0,04.
  
   Результаты решения задачи приведены в~табл.~3. Полученные значения 
могут быть использованы для составления нечеткой диаграммы Ганта 
(см.\ рисунок), дающей наглядное представление о сетевом графике с~учетом 
нечеткой неопределенности. Каж\-дый четырехугольник на диаграмме 
определяет расположение во времени соответствующей работы
 одного из 
проектов. Треугольниками обозначены нечеткие времена начала и~конца работ.
  
  
  
  
  При анализе результатов табл.~3 следует обратить внимание на случаи, когда 
обе компоненты нечеткого числа отклоняются от моды в~одну сторону. В~этом 
случае, как показано в~\cite{8-zac}, рассматривается только доминирующая 
компонента. 

Полученное решение позволяет оценить время завершения не 
только каждой работы, но и~проектов в~целом. При этом дается оценка 
возможности как задержки, так и~опережения планируемых сроков. Например, 
работа~1~первого проекта может быть выполнена с~опережением почти 
на~0,7~временн$\acute{\mbox{ы}}$х единиц, а задержки не предвидится; аналогичная ситуация 
со сроком выполнения проекта~2~--- возможно опережение почти 
на~6~временн$\acute{\mbox{ы}}$х единиц. Время выполнения проекта~1 может задержаться 
на~0,9~временн$\acute{\mbox{ы}}$х единиц, но существует и~возможность опережения срока 
на~1,7~временн$\acute{\mbox{ы}}$х единиц.
  
  
  Кроме того, результаты решения позволяют оптимально с~точки зрения 
занятости распределить исполнителей по работам с~учетом их специализации. 
Дробное число исполнителей означает неполную занятость на конкретной 
работе.

\vspace*{-6pt}
  
\section{Заключение}

\vspace*{-2pt}
  
  Представленные в~статье методические подходы к~сетевому планированию 
демонстрируют возможности тео\-рии нечетких множеств при описании 
неопределенностей, возникающих при планировании проектных работ. 
Такой 
подход может служить альтернативой, например, известному методу PERT\linebreak
(Project Evaluation and Review Technique), 
исполь-\linebreak зующему $\beta$-рас\-пре\-де\-ле\-ние времени выполнения
работ~\cite{4-zac}, отличаясь от него большими воз\-мож\-но\-стя\-ми при учете 
неопределенностей. Так, в~рас\-смотренном примере была учтена 
неопределенность эффек\-тив\-ности работы исполнителей, не предусмат\-ри\-ва\-емая 
методом PERT. Разработанный подход может быть применен при 
ситуационном управ\-ле\-нии одновременно проводящимися мероприятиями 
с~ограниченным со\-ста\-вом исполни\-телей. 

\vspace*{-8pt}
  
{\small\frenchspacing
 {%\baselineskip=10.8pt
 \addcontentsline{toc}{section}{References}
 \begin{thebibliography}{99}
 
 \vspace*{-4pt}
 
\bibitem{1-zac}
Из выступления Президента России Владимира Путина  
на VI~Международном форуме технологического развития <<Технопром-2018>> 
27~августа 2018~г. {\sf http://www.kremlin.ru/events/president/news/58391}.
\bibitem{2-zac}
О~стратегии на\-уч\-но-тех\-но\-ло\-ги\-че\-ско\-го развития Российской Федерации: Указ 
Президента РФ от 01.12.2016 №\,642. {\sf 
http://static.kremlin.ru/media/ events/files/ru/uZiATIOJiq5tZsJgqcZLY9YyL8PWTX Qb.pdf}.
\bibitem{3-zac}
\Au{Зацаринный А.\,А., Киселев~Э.\,В., Козлов~С.\,В., Колин~К.\,К.} Информационное 
пространство цифровой экономики России. Концептуальные основы и~проб\-ле\-мы 
формирования.~--- М.: ФИЦ ИУ РАН, 2018. 236~с.
\bibitem{4-zac}
\Au{Таха Х.\,А.} Введение в~исследование операций~/ Пер. с~англ.~--- М.: Вильямс, 2016. 
912~с. (\Au{Taha~H.\,A.} Operations research: An introduction.~--- 8th ed.~--- Prentice Hall, 
2006. 838~p.)
\bibitem{5-zac}
\Au{Piegat A.} Fuzzy modeling and control.~--- Berlin--Heidelberg: Springer, 2001. 371~p.
\bibitem{6-zac}
\Au{Балашов В.\,Г., Заложнев~А.\,Ю., Новиков~Д.\,А.} Механизмы управления 
организационными проектами.~--- М.: ИПУ РАН, 2003. 84~с.
\bibitem{7-zac}
\Au{Матвеев~М.\,Г.} Анализ и~решение задач выбора с~параметрической нечеткостью~// 
Вестник Юж\-но-Ураль\-ско\-го государственного университета. Сер. Математическое 
моделирование и~программирование, 2015. №\,8(4). С.~14--29.
\bibitem{8-zac}
\Au{Шевляков~А.\,О., Матвеев~М.\,Г.} Сравнение различных нечетких арифметик~// 
Искусственный интеллект и~принятие решений, 2017. №\,4. С.~60--68.
\bibitem{9-zac}
\Au{Шевляков А.\,О., Матвеев~М.\,Г.} Алгебра трапециевидных чисел для обработки 
нечеткой информации~// Искусственный интеллект и~принятие решений, 2018. №\,4.  
С.~53--60.
\bibitem{10-zac}
\Au{Матвеев~М.\,Г., Алейникова~Н.\,А.} Математическое\linebreak моделирование задачи сетевого 
планирования с~по\-мощью нечеткой математики~// Вестник 
Воронежского государственного университета. Сер. Системный анализ 
и~информационные технологии, 2018. №\,3. С.~155--162.
\bibitem{11-zac}
\Au{Шевляков А.\,О., Матвеев~М.\,Г.} Решение RCPSP при нечетких трудозатратах 
выполнения операций~// Вестник Воронежского государственного университета. 
Сер. Системный анализ и~информационные 
технологии, 2015. №\,4. С.~121--125.
\bibitem{12-zac}
\Au{Blazewicz J., Lenstra~J.\,K., Rinnooy Kan~A.\,H.\,G.} Scheduling subject to resource 
constraints: Classification and complexity~// Discrete Appl. Math., 1983. Vol.~5. No.\,1. 
P.~11--24.

\bibitem{14-zac} %13
\Au{Hartmann S.} Project scheduling with multiple modes: A~genetic algorithm~// Ann. 
Oper. Res., 2001. Vol.~102. No.\,1-4. P.~111--135.
\bibitem{13-zac} %14
\Au{Abdolshah M.} A~Review of Resource-Constrained Project Scheduling Problems (RCPSP) 
approaches and solutions~// Int. T.~J.~Eng. Manage. Appl. Sci. 
Technol., 2014. Vol.~5. No.\,4. P.~253--286.

\bibitem{15-zac}
\Au{Kumanan S., Jose~G.\,J., Raja~K.} Multi-project scheduling using a heuristic and a genetic 
algorithm~// Int. J.~Adv. Manuf. Tech., 2006. Vol.~31. No.\,3-4. 
P.~360--366.
\bibitem{16-zac}
\Au{Yannibelli V., Amandi~A.} A~knowledge-based evolutionary assistant to software development 
project scheduling~// Expert Syst.  Appl., 2011. Vol.~38. No.\,7. P.~8403--8413.

 \end{thebibliography}

 }
 }

\end{multicols}

\vspace*{-3pt}

\hfill{\small\textit{Поступила в~редакцию 28.03.19}}

\vspace*{8pt}

%\pagebreak

%\newpage

%\vspace*{-29pt}

\hrule

\vspace*{2pt}

\hrule

%\vspace*{-2pt}

\def\tit{MODELING THE~PROCESS OF~NETWORK PLANNING OF~A~PORTFOLIO OF 
PROJECTS WITH~HETEROGENEOUS RESOURCES UNDER~FUZZINESS}


\def\titkol{Modeling the~process of~network planning of~a~portfolio of 
projects with~heterogeneous resources under~fuzziness}

\def\aut{A.\,A.~Zatsarinny$^1$, V.\,V.~Korotkov$^2$, and~M.\,G.~Matveev$^2$}

\def\autkol{A.\,A.~Zatsarinny, V.\,V.~Korotkov, and~M.\,G.~Matveev}

\titel{\tit}{\aut}{\autkol}{\titkol}

\vspace*{-11pt}


\noindent
  $^1$Federal Research Center ``Computer Science and Control'' of the Russian 
Academy of Sciences, 44-2~Vavilov\linebreak
$\hphantom{^1}$Str., Moscow 119333, Russian Federation
  
  \noindent
   $^2$Voronezh State University, 1~Universitetskaya Pl., Voronezh 394018, Russian 
Federation

\def\leftfootline{\small{\textbf{\thepage}
\hfill INFORMATIKA I EE PRIMENENIYA~--- INFORMATICS AND
APPLICATIONS\ \ \ 2019\ \ \ volume~13\ \ \ issue\ 2}
}%
 \def\rightfootline{\small{INFORMATIKA I EE PRIMENENIYA~---
INFORMATICS AND APPLICATIONS\ \ \ 2019\ \ \ volume~13\ \ \ issue\ 2
\hfill \textbf{\thepage}}}

\vspace*{6pt}
     
  
  \Abste{The paper discusses the problem of project portfolio scheduling subject to resource 
constraints and fuzzy activity durations. Employees of various specializations are considered as the 
only type of resource. Fuzzy W-algebra was used to avoid such difficulties and limitations of 
traditional fuzzy arithmetics as the support size extension, the need of fuzzy 
number comparison, 
and some others. The proposed model provides a~fuzzy estimation of project execution times and 
optimal resource allocation. To solve the problem,
the genetic algorithm based on activity list representation was 
implemented. The paper also provides a~numerical example which 
demonstrates advantages of applying the approach to organizations' workflow in the case of making 
estimates for decision making under uncertainty.}
  
  \KWE{project management; W-algebra; fuzzy arithmetic; combinatorial optimization; genetic 
algorithm}
  
  
 
  
  
  \DOI{10.14357/19922264190213}

\vspace*{-14pt}

\Ack
  \noindent
  The work was partly supported by the Russian Foundation for Basic Research 
(grant No.\,17-01-00251).



%\vspace*{6pt}

  \begin{multicols}{2}

\renewcommand{\bibname}{\protect\rmfamily References}
%\renewcommand{\bibname}{\large\protect\rm References}

{\small\frenchspacing
 {%\baselineskip=10.8pt
 \addcontentsline{toc}{section}{References}
 \begin{thebibliography}{99}
\bibitem{1-zac-1}
Iz vystupleniya Prezidenta Rossii na VI Mezhdunarodnom forume 
tekhnologicheskogo razvitiya ``Tekhnoprom-2018'' [From the speech of the 
President of Russia at the 6th  Forum (International) of Technological 
Development ``Technoprom-2018'']. Available at: 
{\sf http://www.\linebreak kremlin.ru/events/president/news/58391} (accessed May~14, 
2019).
\bibitem{2-zac-1}
O~strategii nauchno-tekhnologicheskogo razvitiya Ros\-siy\-skoy Federatsii: 
ukaz Prezidenta ot 01.12.2016 No.\,642 [About strategy of scientific and 
technological development of the Russian Federation. Presidential Decree 
No.\,642 dated 01.12.2016]. Available at:  
{\sf http://static.\linebreak
 kremlin.ru/media/events/files/ru/uZiATIOJiq5tZsJgqc ZLY9YyL8PWTXQb.pdf}
 (accessed May~14, 2019).
\bibitem{3-zac-1}
\Aue{Zatsarinny, A.\,A., E.\,V.~Kiselev, S.\,V.~Kozlov, and K.\,K.~Kolin.} 2018. 
\textit{Informatsionnoe prostranstvo tsifrovoy ekonomiki Rossii. Kontseptual'nye 
osnovy i~problemy formirovaniya} [Information space of the digital economy 
of Russia. Conceptual framework and problems of formation]. Moscow: 
FRC CSC RAS. 236~p.
\bibitem{4-zac-1}
\Aue{Taha, H.\,A.} 2006. \textit{Operations research: An introduction}. 8th ed. 
Prentice Hall. 838~p.
\bibitem{5-zac-1}
\Aue{Piegat, A.} 2001. \textit{Fuzzy modeling and control.} Berlin--Heidelberg:
Springer. 371~p.
\bibitem{6-zac-1}
\Aue{Balashov, V.\,G., A.\,Yu.~Zalozhnev, and D.\,A.~Novikov.} 2003. \textit{Mekhanizmy 
upravleniya organizatsionnymi proektami} [Organizational project 
management mechanisms]. Moscow: ICS RAS. 84~p.
\bibitem{7-zac-1}
\Aue{Matveev, M.\,G.} 2015. Analiz i~reshenie zadach vybora s~parametricheskoy 
nechetkost'yu [Analyzing and solving problems of decision making
with parametric 
fuzzy]. \textit{Bulletin of 
the South Ural State University. Ser. Mathematical Modelling, 
Programming \& Computer Software} 8(4):14--29.
\bibitem{8-zac-1}
\Aue{Shevlyakov, A.\,O., and M.\,G.~Matveev.} 2017. Sravnenie razlichnykh 
nechetkikh arifmetik [A~comparison of different fuzzy arithmetics]. 
\textit{Artificial Intelligence  
Decision Making} 4:60--68.
\bibitem{9-zac-1}
\Aue{Shevlyakov, A.\,O., and M.\,G.~Matveev.} 2018. Algebra trapetsievidnykh chisel 
dlya obrabotki nechetkoy informatsii [Algebra of
trapezoidal fuzzy numbers for fuzzy information processing]. 
\textit{Artificial 
Intelligence Decision Making} 4:53--60.
\bibitem{10-zac-1}
\Aue{Matveev, M.\,G., and N.\,A.~Aleynikova.} 2018. Ma\-te\-ma\-ti\-che\-skoe 
modelirovanie zadachi setevogo planirovaniya s~pomoshch'yu nechetkoy 
matematiki [Mathematical modeling of the problem of network planning by means of 
fuzzy mathematics]. \textit{Proceedings of Voronezh State University.
Ser. Systems Analysis 
and Information Technologies} 3:155--162.
\bibitem{11-zac-1}
\Aue{Shevlyakov, A.\,O., and M.\,G.~Matveev.} 2015. Reshenie RCPSP pri 
nechetkikh trudozatratakh vypolneniya ope\-ra\-tsiy [Solving RCPSP with 
 uncertain duration times of
activities]. \textit{Proceedings of Voronezh State University.
Ser. Systems Analysis 
and Information Technologies} 4:121--125.
\bibitem{12-zac-1}
\Aue{Blazewicz, J., J.\,K.~Lenstra, and A.\,H.\,G.~Rinnooy Kan.} 1983. Scheduling 
subject to resource constraints: Classification and complexity. \textit{Discrete 
Appl. Math.} 5(1):11--24.

\bibitem{14-zac-1} %13
\Aue{Hartmann, S.} 2001. Project scheduling with multiple modes: A~genetic 
algorithm. \textit{Ann. Oper. Res.} 102(1-4):\linebreak 111--135.

\bibitem{13-zac-1} %14
\Aue{Abdolshah, M.} 2014. A~Review of Resource-Constrained Project Scheduling 
Problems (RCPSP) approaches and solutions. \textit{Int. T.~J.~Eng. Manage. Appl. 
Sci.  Technol.}  5(4):253--286.

\bibitem{15-zac-1}
\Aue{Kumanan, S., G.\,J.~Jose, and K.~Raja.} 2006. Multi-project scheduling using 
a~heuristic and a~genetic algorithm. \textit{Int. J.~Adv. 
Manuf. Tech.} 31(3-4):360--366
\bibitem{16-zac-1}
\Aue{Yannibelli, V., and A.~Amandi.} 2011. A knowledge-based evolutionary 
assistant to software development project scheduling. \textit{Expert Syst. 
Appl.} 38(7):8403--8413.
\end{thebibliography}

 }
 }

\end{multicols}

\vspace*{-6pt}

\hfill{\small\textit{Received March 28, 2019}}

%\pagebreak

%\vspace*{-18pt}
  \Contr
  
  \noindent
  \textbf{Zatsarinny Alexander A.} (b.\ 1951)~--- Doctor of Science in 
technology, professor, Deputy Director, Federal Research Center ``Computer 
Science and Control'' of the Russian Academy of Sciences, 44-2~Vavilov Str., 
Moscow 119333, Russian Federation; \mbox{AZatsarinny@ipiran.ru}
  
  \vspace*{3pt}
  
  \noindent
  \textbf{Korotkov Vladislav V.} (b.\ 1993)~--- PhD student, Department of 
Information Technologies of Management, Computer Science Faculty, Voronezh 
State University, 1~Universitetskaya Pl., Voronezh 394018, Russian Federation; 
\mbox{chasecrunk@gmail.com}
  
  \vspace*{3pt}
  
  \noindent
  \textbf{Matveev Mikhail G.} (b.\ 1949)~--- Doctor of Science in technology, 
professor, Head of Department of Information Technologies of 
Management, Computer Science Faculty, Voronezh State University, 
1~Universitetskaya Pl., Voronezh 394018, Russian Federation; 
\mbox{mgmatveev@yandex.ru}
\label{end\stat}

\renewcommand{\bibname}{\protect\rm Литература}  