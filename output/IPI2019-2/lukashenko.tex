%\newcommand{\Var}{\ensuremath{{\rm\mathbb{V}ar}}}

\renewcommand{\figurename}{\protect\bf Figure}
\renewcommand{\tablename}{\protect\bf Table}

\def\stat{lukashenko}


\def\tit{A~GAUSSIAN APPROXIMATION OF THE DISTRIBUTED COMPUTING PROCESS}

\def\titkol{A~Gaussian approximation of the distributed computing process}

\def\autkol{O.\,V.~Lukashenko, E.\,V.~Morozov,  and~M.~Pagano}

\def\aut{O.\,V.~Lukashenko$^{1}$, E.\,V.~Morozov$^2$,  and~M.~Pagano$^{3}$}

\titel{\tit}{\aut}{\autkol}{\titkol}

%{\renewcommand{\thefootnote}{\fnsymbol{footnote}}
%\footnotetext[1] {The study was carried out under state order to the Karelian Research 
%Centre of the Russian Academy of Sciences (Institute of Applied Mathematical 
%Research KarRC RAS) and supported by the Russian Foundation for Basic Research, 
%projects 18-07-00187, 18-07-00147, 18-07-00156, 19-07-00303.}}

\renewcommand{\thefootnote}{\arabic{footnote}}
\footnotetext[1]{Institute of  Applied Mathematical Research of Karelian Research Centre of RAS, 
11~Pushkinskaya Str.,  Petrozavodsk 185910, Republic of Karelia, Russian Federation; 
Petrozavodsk State University, 33~Lenin Str., Petrozavodsk 185910, Republic of Karelia, 
Russian Federation,  \mbox{lukashenko@krc.karelia.ru}}
\footnotetext[2]{Institute of  Applied Mathematical Research of Karelian Research Centre of RAS, 
11~Pushkinskaya Str.,  Petrozavodsk 185910, Republic of Karelia, Russian Federation; 
Petrozavodsk State University, 33~Lenin Str., Petrozavodsk 185910, Republic of Karelia, 
Russian Federation, \mbox{emorozov@karelia.ru}}
\footnotetext[3]{University of Pisa, 43~Lungarno Pacinotti, Pisa 56126, Italy, \mbox{m.pagano@iet.unipi.it}}


\index{Lukashenko O.\,V.}
\index{Morozov E.\,V.}
\index{Pagano M.}
\index{Лукашенко О.\,В.}
\index{Морозов Е.\,В.}
\index{Пагано М.}


\def\leftfootline{\small{\textbf{\thepage}
\hfill INFORMATIKA I EE PRIMENENIYA~--- INFORMATICS AND
APPLICATIONS\ \ \ 2019\ \ \ volume~13\ \ \ issue\ 2}
}%
 \def\rightfootline{\small{INFORMATIKA I EE PRIMENENIYA~---
INFORMATICS AND APPLICATIONS\ \ \ 2019\ \ \ volume~13\ \ \ issue\ 2
\hfill \textbf{\thepage}}}

%\vspace*{-2pt}



 
\Abste{The authors propose a~refinement of the stochastic model 
describing the dynamics of the Desktop Grid (DG) project with many hosts and many
 workunits to be performed, originally proposed  by Morozov \textit{et al.}\ in 2017.
The target performance measure  is the mean  duration of the runtime of the project. 
To this end,  the authors derive  an asymptotic expression for the  amount 
of the accumulated work to be done by means of 
    limit theorems for  superposed on-off sources that   lead to a~Gaussian 
    approximation. In more detail, depending on the distribution of active 
    and idle periods, Brownian  or fractional Brownian processes are obtained.
    The authors present the analytic results related to the hitting time of 
    the considered processes (including the case in which the overall amount of
work is only known in a~probabilistic way), and highlight how the 
     runtime tail distribution could be estimated by simulation. Taking 
     advantage of the properties of Gaussian processes and the Conditional 
     Monte-Carlo (CMC) approach, the authors present a~theoretical framework for 
     evaluating the runtime tail distribution.}


\KWE{Gaussian approximation; distributed computing; fractional Brownian motion}

 \DOI{10.14357/19922264190215}


%\vspace*{8pt}


\vskip 12pt plus 9pt minus 6pt

 \thispagestyle{myheadings}

 \begin{multicols}{2}

 \label{st\stat}



\section{Introduction}

\vspace*{-4pt}

\noindent
Gaussian processes are widely used in the performance analysis of telecommunication 
systems for their analytic tractability and arguments based on the central-limit 
theorem that make them suitable in case of a~large number of independent contributions.  For instance, these  models are able to capture, in a simple and parsimonious way,
the properties of self-similarity and long-range dependence, inherent to multimedia
network traffic~\cite{2-luk-1, 3-luk-1}. 
These properties dramatically increase the difficulty of the probabilistic 
analysis and, as a consequence, in many cases only Monte-Carlo simulation can be used.
The \textit{fractional Brownian motion} (FBM) is one of the most studied 
self-similar long-range dependent Gaussian processes due to its simplicity. 
Its use as traffic model is supported  by the following theoretical analysis~\cite{4-luk-1}: 
the sum of an increasing  number of the so-called on-off inputs,
with either on-times or off-times having a~heavy-tailed distribution
with infinite variance, converges weakly to an~FBM, after an
appropriate time  scaling.


In this paper,  the applicability of FBM for high-performance computing is considered. 
In that framework, computing clusters and computational Grid systems are the 
main tools: computing clusters are based on computing nodes connected by 
a~high-speed network, while computational Grid systems include geographically 
dispersed computing nodes connected by a relatively slow network. 

Desktop Grid belongs to the latter class. The DG combines nondedicated 
\textit{hosts} (typically, desktops/laptops owned by volunteers) over the Internet 
to process loosely coupled \textit{workunits} (computational tasks). 
Desktop Grids utilize the idle host resources, providing  potentially huge, although highly 
variable, computing power. (For example, the DG project Einstein\@HOME aggregates 
peak performance at about~1~PetaFLOPS~[4].) Typically,  DGs are 
managed by a scientific community that utilizes the resources to complete 
a~\textit{DG project} which consists of a~(usually finite) number of workunits.
 Thus, the \textit{runtime} of the DG project  is the time to complete all the 
 workunits and it is desirable to minimize it.

Minimization of the DG project runtime may be performed by means of 
scheduling optimization~\cite{6-luk-1, 7-luk-1, 8-luk-1}. Additional information on the hosts, 
such as reliability and availability, can be used to improve the efficiency of
 DGs~\cite{9-luk-1, 10-luk-1}, In~\cite{11-luk-1, 12-luk-1}, the focus is placed on the so-called 
 workunit replication mechanism for reliability purposes. However,  to the 
 best of our knowledge, the estimation of the runtime of a~DG project remains
  generally an unsolved issue, and it is the main motivation of this paper.

Desktop Grids have several important distinctive  features 
when compared to computational Grids or computing clusters. First of all, 
hosts, being nondedicated, possess individual availability periods. Moreover, 
the management server of a~DG is not able to obtain information on the current 
state of the hosts (such as ``computing,'' ``suspended,'' etc.). 
These two issues  make the estimation of the runtime of a~DG project a~hard problem. 

The execution of a~DG project can be divided into two stages. 
During the first phase, the number of work\-units is greater than the number of 
hosts and, thus,  each host will receive at least one workunit. 
In the second stage, all the available workunits are dispatched and
 there are available (idle) hosts. In this paper, the focus is on the duration of  
 the first phase  which is studied by means of a Gaussian approximation of the 
 overall work.  The   study of the second stage requires a completely different 
 probabilistic technique, which relies on  the theory of order statistics and the 
 asymptotic properties of renewal processes, and is postponed for a future work.
Thus, in what follows, runtime will relate to the first stage of the project solely.


We describe the availability patterns of the hosts by treating each of them as an 
individual on-off source which processes workunits during on periods.  
Our approach  is based on the asymptotics of the (properly scaled) superposition 
of a large number of independent on-off sources. It is well-known~\cite{4-luk-1} that 
after an appropriate scaling, the limiting process describing the summary workload 
in the system turns out to be \textit{Brownian motion} (BM), when the sojourn times 
are light-tailed, while it becomes \textit{fractional Brownian motion} 
in case of heavy-tailed sojourn times.  
Then, the problem reduces to the calculation of the hitting time of the given
 threshold~$D$ by the process of accumulated work which is a~well-known topic 
 in probability theory.

The paper is organized as follows. Section 2 presents the theoretical background 
related to  FBM, including functional limit theorems for the cumulative  work 
performed  by an increasing  number of on-off sources.    
Then, Section~3 describes the model and summarizes the available analytic results, 
while Section~4 is devoted to the evaluation of the runtime tail distribution 
by means of the CMC method which potentially leads to variance 
reduction of the estimate of the runtime. Finally, in Section~5, the main 
contributions of the paper are presented and some future research issues are discussed. 

\section{Theoretical Background}

\noindent
In this section, let us recall the basic definitions about FBM and how it is 
related to the limiting theorems for the superposition of independent 
\textit{on-off sources}.

\subsection{Fractional Brownian motion}

\noindent
The FBM $\left\{B_H(t), t \in \mathbb{R} \right\}$ is a~Gaussian centered process 
with $B_H(0)=0$, stationary increments, and the following covariance function:
  \begin{multline*}
       K_H(t,s) := \mathbb{E} \left[ B_H(t) \: B_H(s) \right]\\
       {}=
       \fr{1}{2} \left[
       |t|^{2H} - |t-s|^{2H} + |s|^{2H} \right],\enskip s,\,t\ge 0 
     \end{multline*}
where $H \in (0,1)$ is the so-called \textit{Hurst parameter}. 
It is easy to verify that~$B_H(t)$  is a self-similar process with 
self-similarity parameter~$H$, i.\,e., for each $c>0$,
$$
  c^{-H}B_H(ct) \stackrel{d}{=} B_H(t)
$$
where  $\stackrel{d}{=}$  denotes equality in distribution.

Fractional Brownian motion is widely used for modeling purposes due to its 
Gaussianity (that typically arises under aggregation conditions) 
and parsimonious description (apart from mean and variance, its behavior 
is unambiguously determined by~$H$).

When $H>1/2$,  FBM is a long-range dependent process since the
 autocorrelation of the corresponding increment process is nonsummable. 
 For more details on FBM and its properties, see~\cite{13-luk-1}.


\subsection{Limit theorems for distributed computing processes}



\noindent
Let us assume that the DG consists of~$N$ heterogeneous hosts which can be 
considered as  independent \textit{on-off sources}. In more detail, let us suppose 
that there are~$n$~types of hosts ($n<N$) and denote by~$N_i$ the number of  
$i$-type hosts, i.\,e., $\sum\nolimits_{i=1}^n N_i=N$. 
Moreover, let~$R_i$ denote the amount of processed work per unit time for $i$-type 
hosts and let $\left\{I^{(i)}(t),\ t\geq0\right\}$,
\begin{equation*}
I^{(i)}(t)=\begin{cases}
 R_i\,, &t\in \mbox{on-period}\,; \\
 0\,, & t\in \mbox{off-period}\,, 
\end{cases}
 %\label{taq1}
\end{equation*}
be the  \textit{on-off} process that characterizes the activity/silent 
periods of the corresponding hosts (Fig.~1).
For sake  of simplicity, it is assumed that for each host, both \textit{on} 
and  \textit{off} periods are sequences of i.i.d.\ (independent
and identically distributed)
random variables (RVs) and mutually independent.
Moreover, as already stated, the\linebreak\vspace*{-12pt}

{ \begin{center}  %fig1
 \vspace*{9pt}
   \mbox{%
 \epsfxsize=79mm 
 \epsfbox{luk-1.eps}
 }


\vspace*{6pt}


\noindent
{{\figurename~1}\ \ \small{On-off model}}
\end{center}
}

%\vspace*{9pt}

\addtocounter{figure}{1}


\noindent
on-off processes modeling the contribution of 
different hosts are assumed to be independent.



The \textit{cumulative processed work}, i.\,e., the aggregated amount of  
work provided by all $N$ hosts, during the time interval $[0,t]$ is given by
$$
A(t)=\int\limits_0^{t} \left( {\sum\limits_{i=1}^n\sum\limits_{k=1}^{N_i} 
{I_{k}^{(i)}(u)} } \right)\,du
$$
where $I_k^{(i)}$ are the independent copies of~$I^{(i)}$, $i=1,\ldots\linebreak \ldots,n$. 
Moreover, for the $i$-type ($i=1,\dots,n$) hosts, let us denote by
$\mu_{\mathrm{on}}^i$, $\sigma_{\mathrm{on}}^i$, $\mu_{\mathrm{off}}^i$, 
and~$\sigma_{\mathrm{off}}^i$
the mean length and  standard deviation (that may be infinite) of 
the duration of the  on and  off periods, respectively.


The statistical behavior of~$A(t)$ is determined by the distributions~$F_{\mathrm{on}}^i$ 
and~$F_{\mathrm{off}}^i$ of the on and off periods for each type of hosts, namely,
 by their tail. 
In more detail, in case of infinite variance, let us assume that as $x \to \infty$,
\begin{align*}
1-F_{\mathrm{on}}^i(x)& \sim  \ell_{\mathrm{on}}^i x^{-\alpha_{\mathrm{on}}^i}L_{\mathrm{on}}^i(x)\,;\\
1-F^i_{\mathrm{off}}(x)& \sim  \ell_{\mathrm{off}}^i x^{-\alpha_{\mathrm{off}}^i}L_{\mathrm{off}}^i(x)
%  \label{3}
\end{align*}
where   $a \sim b$  means that $a/b\to 1$;    
$\ell_{\mathrm{on}}^i$ and~$\ell_{\mathrm{off}}^i$ are the positive constants; the 
exponents~$\alpha_{\mathrm{on}}^i$ and~$\alpha_{\mathrm{off}}^i\in (1,\,2)$; 
and the functions~$L_{\mathrm{on}}^i$ 
and~$L_{\mathrm{\mathrm{off}}}^i$ are slowly varying at infinity, i.\,e., for any $t >0$,
$$
\lim_{x \to \infty} \fr{L^i(tx)}{L^i(x)}=1\,,\enskip i=1,\ldots,n\,.
$$
Instead, if $\sigma_{\mathrm{on}}^i$ and~$\sigma_{\mathrm{off}}^i <\infty$, we  set 
$\alpha_{\mathrm{on}}^i=\alpha_{\mathrm{off}}^i=2$. 

It has been  shown in~\cite{4-luk-1} that the scaled process of cumulative  
work arrived during interval  $[0,\,Tt]$ converges weakly to a sum of the i.i.d.\
 FBM's,  provided that 
\begin{enumerate}[(1)]
    \item $N_i\to \infty$ such that
$\lim\nolimits_{N\to \infty}N_i/N>0$, $i=1,\ldots\linebreak \ldots , n$; and
\item  the scaling factor $T\to \infty$.
\end{enumerate}

This \textit{functional limit theorem} leads to the following approximation:
\begin{multline*}
A(tT)\approx T\left( \sum\limits_{i=1}^n R_i N_i
\fr{\mu_{\mathrm{on}}^i}{\mu_{\mathrm{on}}^i+\mu_{\mathrm{off}}^i} \right)t \\
{}+ \sum\limits_{i=1}^n
T^{H_i}R_i \sqrt{L_i(T)N_i}c_i B_{H_i}(t) 
%\label{approx1}    
\end{multline*}
where $c_i$ are the positive constants; $L_i$ are the slowly varying at
infinity functions (expressed in terms of the given  parameters); and
$B_{H_i}$ are the independent FBMs with the Hurst parameters~$H_i$ given by
$$
H_i=\fr{3-\min(\alpha_{\mathrm{on}}^i,\,\alpha_{\mathrm{off}}^i)}{2}\in
\left(\fr{1}{2},\,1 \right),\enskip i=1,\ldots, n\,.
$$
Thus, the cumulative work processed by a~large number of independent hosts 
(with heavy-tailed distributions of the on-off periods) is approximated by 
a~superposition of independent FBMs  $\{B_{H_i}(t)\}$, 
$i=1,\dots,n$,  with a~linear drift that depends on the rates~$R_i$ and 
the average duty cycle.

Instead, if for all types of hosts the variances of the sojourn times 
are finite (i.\,e., $\sigma_{\mathrm{on}}^i,\,\sigma_{\mathrm{off}}^i<\infty$ 
$\forall i=1,\dots,n$), then the limiting (scaled) process becomes
\begin{equation*}
T\left( \sum\limits_{i=1}^n \fr{R_i N_i \mu_{\mathrm{on}}^i}{\mu_{\mathrm{on}}^i+\mu_{\mathrm{off}}^i}
\right)t + \left(\sqrt{T}\sum\limits_{i=1}^n R_i \sqrt{N_i}c_i \right)W(t)
%\label{bm-l1}
\end{equation*}
where $W(t)$ is the Wiener process, and the constants~$c_i$ are given by
$$
c_i = \sqrt{ \fr{ (\mu_{\mathrm{off}}^i\sigma_{\mathrm{on}}^i)^2 + 
\left(\mu_{\mathrm{on}}^i\sigma_{\mathrm{off}}^i\right)^2 }{\left(\mu_{\mathrm{on}}^i+\mu_{\mathrm{off}}^i\right)^3}}\,.
$$
Finally, it is worth mentioning that taking the limits in reverse order, 
the (scaled) process of cumulative work converges to a~\textit{Levy stable motion}, 
an infinite variance process with stationary and independent increments~\cite{14-luk-1}; 
however, such  model is beyond the scope of this paper as in DG, the experimental 
data confirmed the convergence to processes with finite variance.


\section{Model Description and~Performance Measures}


\noindent
The above functional limit theorems provide a~theoretical motivation to 
consider the following model for the cumulative processed work: 
\begin{equation}
    A(t) = m t + X(t)
    \label{6}
\end{equation}
where $X$ is the centered Gaussian process with stationary increments 
(FBM or the sum of independent FBM, in case of heterogeneous systems),
 which describes random fluctuations around the linearly  increasing mean. 
 Such type of stochastic process was previously suggested as the model of 
 network traffic (see~\cite{15-luk-1} for more details). 


Let us denote  by~$\tau_D$ the  runtime of the  DG project where~$D$ 
denotes the required amount of work. Thus,~$\tau_D$ represents the  
\textit{hitting time} of  the process~$\{A(t)\}$:  
\begin{equation*}
\tau_D = \min\{t:\,\, A(t)\ge D \} \, ,
\end{equation*}
i.\,e., the first time the process~$\{A(t)\}$ hits the threshold~$D$.
Then, the original problem is reduced to the calculation (or estimation)
 of some useful performance characteristics, such as the mean  hitting time.

\subsection{Available analytic results}

\noindent
Let us recall the available analytic results for different types 
of Gaussian processes, corresponding to the different limiting cases. 

\vspace*{-4pt}

\subsubsection{Wiener case}

\noindent
When $X$ is a Wiener process (i.\,e., $X = \sigma B_{1/2}$), 
the  density of~$\tau_D$ is available in  explicit form~\cite{16-luk-1}:
\begin{multline}
\mathbb{P}(\tau_D \in dt)= \fr{D}{\sqrt{2\pi}\sigma t^{3/2}} 
\exp \left( -\fr{(D-mt)^2}{2\sigma^2 t} \right)\,dt\\
{=:} f_\tau(t|D)\,dt\,.
\label{8}
\end{multline}
In this case, the corresponding expected value $\mathbb{E} [\tau_D ]$ 
is  the ratio between the given  amount of the work~$D$ and the mean processing 
rate~\cite{16-luk-1}:
$$
\mathbb{E} \left[\tau_D\right] = \fr{D}{m}\,.
$$

\vspace*{-12pt}


\subsubsection{Fractional Brownian motion case}



\noindent
When the limiting process is an FBM,  only asymptotic results and some bounds
 for the  distribution of~$\tau_D$ are available.

In~\cite{17-luk-1}, the following bounds (quite inaccurate when~$H$ is close to~1, 
see Fig.~2)
 for the moments of the hitting time were obtained for $1/2 \le H < 1$:
\begin{multline*}
\fr{1}{\sqrt{2\pi}} \left( \fr{2 H D}{n-H}\, L_n (D,H,m)\right.\\ - 
\left.\fr{(2H-1)m}{n+1-H} \,L_{n+1}(D,H,m) \right) \le \mathbb{E} \left[\tau_D^n\right]\\
   \le \fr{1}{\sqrt{2\pi}} \left( \fr{ H D}{n-H} \,L_n (D,H,m)\right.\\
   \left.{} + 
   \fr{(1-H)m}{n+1-H}\, L_{n+1}(D,H,m) \right)
\end{multline*}

{ \begin{center}  %fig2
% \vspace*{9pt}
   \mbox{%
 \epsfxsize=79mm 
 \epsfbox{luk-2.eps}
 }


\end{center}


\noindent
{{\figurename~2}\ \ \small{Bounds for the mean hitting time ($D=10$, $m=3$): 
\textit{1}~--- $D/m$;
\textit{2}~--- lower bound;
and \textit{3}~--- upper bound}}
}

%\vspace*{9pt}

\addtocounter{figure}{1}


\noindent
where
\begin{multline*}
L_n(D,H,m) \\
{}=\!\! \int\limits_0^\infty \!\exp\left\{ -\fr{1}{2} \!
\left(\!Dt^{-H/(n-H)}-mt^{(1-H)/(n-H)} \right)^2 \!\right\}dt.\hspace*{-3.80858pt}
\end{multline*}
Additionally, the following asymptotic was derived for the large values of level~$D$:
\begin{equation*}
    \lim\limits_{D \to \infty} \fr{\mathbb{E} \left[\tau_D^n\right]}{D^n} = m^{-n}
\end{equation*}
for all $n \ge 1$, $m>0$, from which it is quite straightforward to show that 
for all $n \ge 1$, 
\begin{equation*}
     \fr{\tau_D}{D} \overset{L_n}{\longrightarrow} \fr{1}{m}\enskip 
     \mbox{as} \enskip D \to \infty
\end{equation*}
where $\overset{L_n}{\longrightarrow}$ means convergence in~$L_n$~space.

\subsubsection{General case}

\noindent
In the general case,  to derive asymptotic (for large values of~$D$) 
for the distribution of~$\tau_D$,  it is possible to take advantage of the 
following identity: 
\begin{equation*}
\mathbb{P}\left(\tau_D \le  T\right) = 
\mathbb{P}\left(\sup\limits_{t \in [0,T]}A(t)\ge D\right)\,.
\end{equation*}

The distribution of the maximum of Gaussian processes over a~finite interval is 
a~well-studied  problem. In more detail, for any Gaussian process with stationary 
increments and strictly monotonically increasing and
convex variance such that $\lim\nolimits_{t\to 0} \mathrm{Var}(X(t))/t=0$, the following asymptotic 
holds~\cite{18-luk-1}:
\begin{multline*}
\mathbb{P}\left(\sup\limits_{t \in [0,T]}A(t)\ge D\right) \sim 
\Phi \left( \fr{D-mT}{\sqrt{\mathrm{Var}(X(T))}} \right)\\
 \mbox{ as } D \to \infty
\end{multline*}
where~$\Phi$ denotes the tail distribution of the standard normal RV~$N(0,1)$.

\subsection{A~possible  generalization} 



 \noindent
 It seems quite natural to consider the setting in which the threshold~$D$  
 is an~RV which is independent of the process~$X$ in~(\ref{6}). 
 Such a~setting seems to be highly motivated by practice because it is more 
 realistic that the exact value of   the quantity~$D$ is not available, 
 and it is known in part. This incomplete information can be reflected by  
 introducing the probability density
 function (PDF)~$f_D$ of~$D$, which is assumed to be predefined.   
 Provided that~$X$ in~(\ref{6}) is  a~Wiener process and, hence, the conditional 
 density $f_\tau(t|D)$ in~(\ref{8}) is known, one can write  the density of the RV~$\tau_D$ as
$$
f_\tau(x)=\int\limits_{y=0}^\infty 
f_\tau(x|y)f_D(y)\,dy\,.
$$
In general, one
can calculate this density only by numerical methods 
but for some cases, it is possible to derive its expression in terms of special
 functions. For example, when~$D$  is  exponential with parameter~$\lambda$, 
 one can obtain the following expression:
\begin{multline}
f_\tau(x) = \fr{\lambda}{\sqrt{2\pi}\sigma x^{3/2}}\exp
\left( -\fr{m^2 x}{2\sigma^2} +\fr{\gamma^2}{8\beta(x)}\right)\\
 \times
(2\beta(x))^{-1/2} D_{-1}\left( \fr{\gamma}{\sqrt{2\beta(x)}} \right) 
\label{dens}
\end{multline}
where 
$$
\gamma=\lambda-\fr{m}{\sigma^2}\,;\qquad \beta(x) = \fr{1}{2\sigma^2 x}\,;
$$
and $D_p$, $\mathrm{Re}\, p <0$, is the parabolic cylinder function~\cite{19-luk-1}. 
Numeric calculation of the expression~\eqref{dens} is shown in 
Fig.~3.

{ \begin{center}  %fig3
 \vspace*{6pt}
  \mbox{%
 \epsfxsize=78.984mm 
 \epsfbox{luk-3.eps}
 }


\end{center}


\noindent
{{\figurename~3}\ \ \small{Probability density function of~$\tau_D$ for different values of~$m$ 
($\lambda=1$): \textit{1}~--- $m=1$; \textit{2}~--- $2$;  and \textit{3}~--- $m=3$}}
}

\vspace*{-3pt}

\addtocounter{figure}{1}


\section{Estimation via Monte Carlo}

\noindent
A more flexible alternative to analytic results is represented  by simulation 
that in our case can be used to estimate 
\begin{equation*}
\pi(T):=\mathbb{P} \left(\tau_D > T\right)\,.
\end{equation*}
Such probability could be extremely small for large values of~$T$; 
thus, its estimation with a~given accuracy requires to generate a~large number of 
sample paths of the process~$X$. However, for such type of rare events, it is 
possible to apply a special case of the well-known CMC 
method which always leads to variance reduction.

The method, originally proposed by some of the authors 
in~\cite{20-luk-1, 21-luk-1, 22-luk-1} and named Bridge Monte Carlo (BMC), 
is based on the idea of expressing the target probability as the
expectation of a function of the {Bridge} $Y:=\{Y_t\}$ of the
Gaussian process~$X$, i.\,e.,  the process obtained by
conditioning~$X$ to reach a certain level at some prefixed time instant~$\tau$:
\begin{equation*}
Y(t) = X(t) - \psi(t) X(\tau)
\end{equation*}
where $\psi$ can be easily  expressed in terms of the the covariance 
function~$\Gamma(s,t)$ of the process~$X$ 
$$
\psi(t)   :=
\fr{\Gamma(t,\tau)}{\Gamma(\tau, \tau)} \,.  
$$
Since the variance of~$X$ is an increasing function of~$t$ in all models we consider,
it is easy to see that  $\psi(t)>0$ for all $t \ge 0$.
Moreover, for any~$t$, $Y(t)$ is
independent of~$X(\tau)$ since
$$
\mathbb{E} \left[X(\tau)Y(t)\right]=
\Gamma(\tau,t)-
\fr{\Gamma(t,\tau)}{\Gamma(\tau,\tau)}\,\Gamma(\tau, \tau)=0
$$
and $(X(\tau),Y(t))$ has bivariate normal distribution.





Let $\mathbb{T} = [0,T]$, then the target probability can 
be expressed in the following way:
\begin{multline*}
\pi(T)\, =\mathbb{P}\left(\sup\limits_{t \in [0,T]}A(t)\ge D\right)\\
{}=\mathbb{P}\left(\forall t \in \mathbb{T}:\,\,mt+X(t) \le D\right)\\
{}=\mathbb{P}\left( \forall t \in \mathbb{T}:\,\, X(\tau) \le \fr{D-Y(t)-mt}{\psi(t)}\right)\\
{}=\mathbb{P}\left( X(\tau)\le \inf_{t \in T} \fr{D-Y(t)-mt}{\psi(t)}\right)\\
{}=\mathbb{P}\left( X(\tau)\le\overline{Y} \right)
\end{multline*}
where
\begin{equation*}
\overline{Y}:=\inf\limits_{t \in \mathbb{T}}\fr{D-Y(t)-mt}{\psi(t)}\,. 
%\label{BMC_4}
\end{equation*}

Finally, the  considered probability can be rewritten as follows:

\begin{equation*}
\pi(T)=\mathbb{P}\left(X_{\tau}\le\overline{Y}\right)=
\mathbb{E}\left[\Psi\left(\fr{\overline{Y}}{ \sqrt{\Gamma(\tau,\,\tau)}}\right)\right]
\end{equation*}
where independence between~$\overline{Y}$ and~$X_{\tau}$ is used and~$\Psi$ denotes 
the cumulative distribution function of a~standard normal variable.

Hence, given $N$ samples $\{\overline{Y}^{(n)},\,\,n=1,\ldots,N\}$  of~~$\overline{Y}$,
the estimator of~$\pi(T)$ is defined as follows:
\begin{equation*}
\widehat{\pi}_N^{\mathrm{BMC}} \: := \: \fr{1}{N}\sum\limits_{n=1}^N
\Psi
\left(\fr{\overline{Y}^{(n)}}{\sqrt{\Gamma(\tau,\tau)}}\right).
%\label{estimator}
\end{equation*}


Note that
\begin{equation*}
\Psi\left(\fr{\overline{Y}}{ \sqrt{\Gamma(\tau,\,\tau)}}\right)=
\mathbb{E} \left[ I(X(\tau) \le \overline{Y}) | \overline{Y}\right]
%\label{BMC-5}
\end{equation*}
and, therefore, the BMC approach is actually a special case of the  
CMC method;
so, one can expect that the BMC estimator implies variance reduction  (with
regard to crude 
Monte-Carlo simulation) in   the estimation of the target probability~$\pi(T)$ as also 
justified by the previous experience when such a~method was successfully applied for 
estimation some other rare-event probabilities related to Gaussian 
processes~\cite{23-luk-1}.  


\section{Concluding Remarks and~Future~Research}

\noindent
 In this paper,   a~stochastic model describing the dynamics of 
 a~DG project with many hosts and many work\-units to be performed, originally
 proposed in~\cite{1-luk-1},  is presented. 
 It is assumed that the project  can be   described by the so-called on-off model 
 where the hosts are on-off  sources of the work\-units  and the basic process is the 
 completed work. It is assumed that  the hosts' working sessions can have both 
  light- and heavy-tailed distributions.
 Then, an approximation of the  basic process, based on  the asymptotics of 
 the superposed on-off sources, is applied.    
 The suggested approach   leads to a~Gaussian approximation of the process of the 
 completed work. Finally,  a~simulation framework for the evaluation 
 of the runtime of the project, using the properties of Gaussian processes and 
 CMC simulation, is presented. 

Although  this note is focused on estimation of the runtime related to the 
1st stage of the project completion when the number of workunits is bigger 
than the number of hosts,  the 2nd  stage 
could also be relevant.
In more detail, it can be considered as a~collection of the ``tails''  
of the  workunit remaining times. From this point of view, the completion time of the 
2nd stage of the project  can be interpreted as  the \textit{longest} remaining time 
and analyzed by means of the asymptotic results of \textit{renewal theory}.
 Moreover, since the workunits are assumed to be independent, to evaluate the  
 duration of the  2nd stage, it seems promising to  apply the theory of 
 \textit{order statistics} and interpret the completion  time as the maximal 
 order statistics. 


\Ack
\noindent
The study was carried out under state order to the Karelian Research 
Centre of the Russian Academy of Sciences (Institute of Applied Mathematical 
Research KarRC RAS) and supported by the Russian Foundation for Basic Research, 
projects 18-07-00187, 18-07-00147, 18-07-00156, and 19-07-00303.


\renewcommand{\bibname}{\protect\rmfamily References}


%\vspace*{-6pt}

{\small\frenchspacing
{ %\baselineskip=10.35pt
\begin{thebibliography}{99}


\bibitem{2-luk-1} %1
\Aue{Leland, W.\,E., M.\,S.~Taqqu, W.~Willinger., and D.\,V.~Wilson.}
 1994. On the self-similar nature of ethernet traffic (extended version). 
 \textit{IEEE ACM~T. Network.} 2(1):1--15.
\bibitem{3-luk-1} %2
\Aue{Willinger, W., M.\,S.~Taqqu, W.\,E.~Leland, and D.~Wilson.}
1995. Self-similarity in high-speed packet traffic: Analysis and modeling of Ethernet 
traffic measurements. \textit{Stat. Sci.} 10(1):67--85.
\bibitem{4-luk-1}  %3
\Aue{Taqqu, M.\,S., W.~Willinger, and R.~Sherman.} 1997. 
Proof of a~fundamental result in self-similar traffic modeling. 
\textit{Comp. Comm.~R.} 27:5--23.
\bibitem{5-luk-1} %4
BOINCstats. 2017. Available at: {\sf https://boincstats.com} (accessed May~7, 2019).

\bibitem{7-luk-1} %5
\Aue{Kondo, D., D.\,P.~Anderson, and J.~McLeod~VII.} 
2007. Performance evaluation of scheduling policies for volunteer computing.  
\textit{3rd IEEE Conference (International) on e-Science and Grid Computing 
Proceedings}. IEEE. 221--227.
\bibitem{6-luk-1} %6
\Aue{Estrada, T., and M.~Taufer.} 2012. Challenges in 
designing scheduling policies in volunteer computing. 
\textit{Desktop grid computing}. Eds C.~C$\acute{\mbox{e}}$rin and G.~Fedak. 
CRC Press. 167--190.

\bibitem{8-luk-1} %7
\Aue{Durrani, N., and J.~Shamsi.} 2014. Volunteer computing: 
Requirements, challenges, and solutions. \textit{J.~Netw. Comput. Appl.} 39:369--380.
\bibitem{9-luk-1} %8
\Aue{Sonnek, J., M.~Nathan, A.~Chandra, and J.~Weissman.}
 2006. Reputation-based scheduling on unreliable distributed infrastructures in 
 distributed computing systems. \textit{26th IEEE Conference 
 (International) on Distributed Computing Systems Proceedings}. IEEE. Art.~No.\,30. P.~1--8.
\bibitem{10-luk-1} %9
\Aue{Watanabe, K., M.~Fukushi, and M.~Kameyama.}
 2011. Adaptive group-based job scheduling for high performance and reliable 
 volunteer computing.  \textit{J.~Information Processing} 19:39--51.
 
 \bibitem{12-luk-1} %10
\Aue{Xavier, E., R.~Peixoto, and J.~da~Silveira.}
 2013. Scheduling with task replication on desktop grids: 
 Theoretical and experimental analysis.  \textit{J.~Comb. Optim.}
 30(3):520--544.
\bibitem{11-luk-1} %11
\Aue{Chernov, I.\,A., and N.\,N.~Nikitina.}
2015. Virtual screening in a~Desktop Grid: Replication and the optimal quorum. 
\textit{Parallel computing technologies}. Ed. V.~Malyshkin.
Lecture notes in computer science ser. Springer.  9251:258--267.

\bibitem{13-luk-1} %12
\Aue{Samorodnitsky, G., and M.\,S.~Taqqu.} 1994.  \textit{Stable non-Gaussian random processes: 
Stochastic models with infinite variance}. Chapman \& Hall. 632~p.
\bibitem{14-luk-1} %13
\Aue{Mikosch, T., S.~Resnick, H.~Rootz$\acute{\mbox{e}}$n, and A.~Stegeman.}
2002. Is network traffic approximated by stable Levy motion or fractional Brownian 
motion? \textit{Ann. Appl. Probab.} 12(1):23--68.
\bibitem{15-luk-1} %14
\Aue{Norros, I.} 1994. A~storage model with self-similar input.  
\textit{Queueing Syst.} 16:387--396.
\bibitem{16-luk-1} %15
\Aue{Borodin, A.\,N., and P.~Salminen.} 2002.  \textit{Handbook of Brownian motion~--- 
facts and formulae}. Birkh$\ddot{\mbox{a}}$user. 685~p.
\bibitem{17-luk-1} %16
\Aue{Michna, Z.} 1999. On tail probabilities and first passage times for fractional 
Brownian motion.  \textit{Math. Method. Oper. Res.} 49(2):335--354.
\bibitem{18-luk-1} %17
\Aue{Caglar, M., and C.~Vardar.} 2013. Distribution of maximum loss of fractional 
Brownian motion with drift.  \textit{Stat. Probabil. Lett.} 83:2729--2734.
\bibitem{19-luk-1} %18
Gradshtein, I.\,S., I.\,M.~Ryzhik, and A.~Jeffrey, eds.
%D.~Zwillinger, associate ed. 
2015. 
\textit{Table of integrals, series and products}. 8th ed. 
San Diego, CA: Academic Press. 1220~p.
\bibitem{20-luk-1} %19
\Aue{Giordano, S., M.~Gubinelli, and M.~Pagano.}
 2005. Bridge Monte-Carlo: A~novel approach to rare events of Gaussian processes. 
 \textit{5th St. Petersburg Workshop on Simulation Proceedings}. St.\ Petersburg: 
 St. Petersburg State University. 281--286.
\bibitem{21-luk-1} %20
\Aue{Giordano, S., M.~Gubinelli, and M.~Pagano.}
 2007. Rare events of Gaussian processes: A~performance comparison between 
 bridge Monte-Carlo and importance sampling. 
 \textit{Next generation teletraffic and wired/wireless advanced networking}.
 Eds.\ Y.~Koucheryavy, J.~Harju, and A.~Sayenko.
 Lecture notes in computer science ser.
 Springer. 4712:269--280.
\bibitem{22-luk-1} %21
\Aue{Lukashenko, O.\,V., E.\,V.~Morozov, and M.~Pagano.} 
2012. Performance analysis of Bridge Monte-Carlo estimator. 
\textit{Transactions of KarRC RAS} 5:54--60.
\bibitem{23-luk-1} %22
\Aue{Lukashenko, O.\,V., E.\,V.~Morozov, and M.~Pagano.}
 2017. On the efficiency of bridge Monte-Carlo estimator.  
 \textit{Informatika i~ee Primeneniya~--- Inform.  Appl.} 11(2):16--24.
 
 \bibitem{1-luk-1} %23
\Aue{Morozov, E., O.~Lukashenko, A.~Rumyantsev, and E.~Ivashko.}
2017. A~Gaussian approximation of runtime estimation in a desktop grid project. 
\textit{9th  Congress (International) on Ultra Modern Telecommunications and 
Control Systems and Workshops}. IEEE. 107--111.


\end{thebibliography} } }

\end{multicols}

\vspace*{-6pt}

\hfill{\small\textit{Received April 15, 2019}}

\vspace*{-18pt}

\Contr

%\vspace*{-3pt}

\noindent
\textbf{Lukashenko Oleg  V.} (b.\ 1986)~--- 
Candidate of Science (PhD) in physics and mathematics, scientist, 
Institute of  Applied Mathematical Research of Karelian Research Centre of 
the Russian Academy of Sciences, 
11~Pushkinskaya Str.,  Petrozavodsk 185910, Republic of Karelia, 
Russian Federation; associate professor, Petrozavodsk State University, 33~Lenin Str., 
Petrozavodsk 185910, Republic of Karelia, Russian Federation; 
\mbox{lukashenko@krc.karelia.ru}

\vspace*{3pt}

\noindent
\textbf{Morozov  Evsei  V.} (b.\ 1947)~--- 
Doctor of Science in physics and mathematics, professor, leading scientist,
 Institute of  Applied Mathematical Research of Karelian Research Centre of 
 the Russian Academy of Sciences, 11~Pushkinskaya Str.,  Petrozavodsk 185910, 
 Republic of Karelia, Russian Federation; professor, Petrozavodsk State University, 
 33~Lenin Str., Petrozavodsk 185910, Republic of Karelia, Russian Federation; 
 \mbox{emorozov@karelia.ru}
 
 \vspace*{3pt}
 
 \noindent
 \textbf{Pagano Michele} (b.\ 1968)~--- 
 PhD in Information Engineering, associate professor, University of Pisa, 
 43~Lungarno Pacinotti, Pisa 56126, Italy; \mbox{m.pagano@iet.unipi.it}

 


\vspace*{8pt}

\hrule

\vspace*{2pt}

\hrule

%\vspace*{-7pt}

%\newpage

%\vspace*{-28pt}

\def\tit{ГАУССОВСКАЯ АППРОКСИМАЦИЯ ПРОЦЕССА РАСПРЕДЕЛЕННЫХ ВЫЧИСЛЕНИЙ$^*$}

\def\titkol{Гауссовская аппроксимация процесса распределенных вычислений}

\def\aut{О.\,В.~Лукашенко$^{1,2}$, Е.\,В.~Морозов$^{1,2}$,
М.~Пагано$^3$}

\def\autkol{О.\,В.~Лукашенко, Е.\,В.~Морозов,
М.~Пагано}

{\renewcommand{\thefootnote}{\fnsymbol{footnote}} \footnotetext[1]
{Финансовое обеспечение исследований осуществлялось из 
средств федерального бюджета на выполнение государственного задания 
КарНЦ РАН (Институт прикладных математических исследований КарНЦ РАН) 
и~при финансовой поддержке РФФИ (проекты 18-07-00187, 18-07-00147, 18-07-00156
и~19-07-00303).}}



\titel{\tit}{\aut}{\autkol}{\titkol}

\vspace*{-11pt}

\noindent
$^1$Институт прикладных математических исследований Карельского научного центра 
Российской акаде-\linebreak
$\hphantom{^1}$мии наук 

\noindent
$^2$Петрозаводский государственный университет

\noindent
$^3$Университет г.\ Пиза, Италия
%, danielkh@post.bgu.ac.il 

\vspace*{1pt}

\def\leftfootline{\small{\textbf{\thepage}
\hfill ИНФОРМАТИКА И ЕЁ ПРИМЕНЕНИЯ\ \ \ том\ 13\ \ \ выпуск\ 2\ \ \ 2019}
}%
 \def\rightfootline{\small{ИНФОРМАТИКА И ЕЁ ПРИМЕНЕНИЯ\ \ \ том\ 13\ \ \ выпуск\ 2\ \ \ 2019
\hfill \textbf{\thepage}}}

\vspace*{-1pt}


 
\Abst{Продолжено изучение стохастической модели процесса динамики 
выполнения задачи в~сис\-те\-ме Desktop Grid при наличии многих пользователей, 
предложенной в~2017~г.\ Морозовым с~соавт. Тре\-бу\-емой характеристикой 
выступает средняя 
про\-дол\-жи\-тель\-ность времени выполнения проекта. Гауссовская аппроксимация искомого 
процесса производится на основе предельных тео\-рем для суперпозиции on-off 
источников. Приведен обзор известных аналитических результатов для 
тре\-бу\-емой характеристики, вклю\-чая результаты для броуновского 
и~дроб\-но\-го броуновского движения. Также показывается, как с~по\-мощью условного метода 
Мон\-те-Кар\-ло оценить хвост распределения времени выполнения проекта.}


\KW{гауссовская аппроксимация; распределенные вычисления; дробное броуновское движение}

 \DOI{10.14357/19922264190215}



%\vspace*{-3pt}


 \begin{multicols}{2}

\renewcommand{\bibname}{\protect\rmfamily Литература}
%\renewcommand{\bibname}{\large\protect\rm References}

{\small\frenchspacing
{\baselineskip=10.5pt
\begin{thebibliography}{99}
%\vspace*{-3pt}


\bibitem{2-luk} %1
\Au{Leland W.\.E., Taqqu~M.\,S., Willinger~W., Wilson~D.\,V.}
On the self-similar nature of Ethernet traffic (extended version)~// 
IEEE ACM~T. Network., 1994. Vol.~2. Iss.~1. P.~1---15.
\bibitem{3-luk} %2
\Au{Willinger W., Taqqu~M.\,S., Leland~W.\,E., Wilson~D.}
 Self-similarity in high-speed packet traffic: Analysis and modeling of Ethernet 
 traffic measurements~// Stat. Sci., 1995. Vol.~10. Iss.~1. P.~67--85.
\bibitem{4-luk} %3
\Au{Taqqu M.\,S., Willinger~W., Sherman~R.} 
Proof of a~fundamental result in self-similar traffic modeling~// 
Comp. Comm.~R., 1997. Vol.~27. P.~5--23.
\bibitem{5-luk} %4
 BOINCstats, 2017. {\sf https://boincstats.com}.

\bibitem{7-luk} %5
\Au{Kondo D., Anderson~D.\,P., McLeod~VII~J.}
Performance evaluation of scheduling policies for volunteer computing~// 
3rd IEEE  Conference (International) 
on e-Science and Grid Computing Proceedings.~--- IEEE, 2007. P.~221--227.

\bibitem{6-luk} %6
\Au{Estrada T., Taufer~M.}
Challenges in designing scheduling policies in volunteer computing~// 
Desktop grid computing~/ Eds. C.~C$\acute{\mbox{e}}$rin, G.~Fedak.~--- 
CRC Press, 2012. P.~167--190.
\bibitem{8-luk} %7
\Au{Durrani N., Shamsi~J.} 
Volunteer computing: Requirements, challenges, and solutions~// 
J.~Netw. Comput. Appl., 2014. Vol.~39. P.~369--380.
\bibitem{9-luk} %8
\Au{Sonnek J., Nathan~M., Chandra~A., Weissman~J.}
 Reputation-based scheduling on unreliable distributed infrastructures in 
 distributed computing systems~// 26th IEEE Conference (International)
 on Distributed Computing Systems Proceedings.~--- IEEE, 2006. Art. No.\,30. P.~1--8.
\bibitem{10-luk} %9
\Au{Watanabe K., Fukushi~M., Kameyama~M.}
Adaptive group-based job scheduling for high performance and reliable volunteer 
computing~// J.~Information Processing, 2011. Vol.~19. P.~39--51.

\bibitem{12-luk} %10
\Au{Xavier E., Peixoto R., da~Silveira~J.}
 Scheduling with task replication on desktop grids: Theoretical and experimental 
 analysis~// J.~Comb. Optim., 2013.  Vol.~30. Iss.~3. P.~520--544.
 
 \bibitem{11-luk} %11
\Au{Chernov I.\,A., Nikitina~N.\,N.}
 Virtual screening in a desktop grid: Replication and the optimal quorum~// 
  Parallel computing technologies~/ Ed. V.~Malyshkin.~---
 Lecture notes in computer science ser.~---  Springer, 2015.  
 Vol.~9251. P.~258--267.
 
\bibitem{13-luk} %12
\Au{Samorodnitsky G., Taqqu~M.\,S.} Stable non-Gaussian random processes: Stochastic 
models with infinite variance.~--- Chapman \& Hall, 1994. 632~p.
\bibitem{14-luk} %13
\Au{Mikosch T., Resnick~S., Rootz$\acute{\mbox{e}}$n~H., Stegeman~A.}
Is network traffic approximated by stable Levy motion or fractional Brownian motion?~// 
Ann. Appl. Probab., 2002. Vol.~12. Iss.~1. P.~23--68.
\bibitem{15-luk} %14
\Au{Norros I.} A~storage model with self-similar input~// Queueing Syst., 1994. 
Vol.~16. P.~387--396.
\bibitem{16-luk} %15
\Au{Borodin A.\,N., Salminen~P.}
 Handbook of Brownian motion~--- facts and formulae.~--- Birkh$\ddot{\mbox{a}}$user, 2002. 685~p.
\bibitem{17-luk} %16
\Au{Michna Z.} On tail probabilities and first passage times for fractional 
Brownian motion~// Math. Method. Oper. Res., 1999. Vol.~49. Iss.~2. 
P.~335--354.
\bibitem{18-luk} %17
\Au{Caglar M., Vardar~C.} Distribution of maximum loss of fractional 
Brownian motion with drift~// Stat. Probabil. Lett., 2013. Vol.~83. P.~2729--2734.
\bibitem{19-luk} %18
Table of integrals, series and products~/
Eds.  I.\,S.~Gradshtein, I.\,M.~Ryzhik, A.~Jeffrey.~--- 8 ed.~---
%; associate editor D.~Zwillinger.   
San Diego, CA, USA: Academic Press, 2015. 1220~p.
\bibitem{20-luk} %19
\Au{Giordano S., Gubinelli~M., Pagano~M.} Bridge Monte-Carlo: 
A~novel approach to rare events of Gaussian processes~// 5th St.\ 
Petersburg Workshop on Simulation Proceedings.~--- St.\ Petersburg: St. Petersburg 
State University, 2005. P.~281--286.
\bibitem{21-luk} %20
\Au{Giordano S., Gubinelli~M., Pagano~M.} Rare events of Gaussian processes: 
A~performance comparison between bridge Monte-Carlo and importance sampling~// 
Next generation teletraffic and wired/wireless advanced networking~/
 Eds.\ Y.~Koucheryavy, J.~Harju, A.~Sayenko.~--- 
 Lecture notes in computer science ser.~--- Springer, 2007.
Vol.~4712. P.~269--280.
\bibitem{22-luk} %21
\Au{Lukashenko O.\,V., Morozov~E.\,V., Pagano~M.}
Performance analysis of bridge Monte-Carlo estimator~// 
Труды Карельского научного центра Российской академии наук, 
2012. Т.~5. С.~54--60.
\bibitem{23-luk} %22
\Au{Lukashenko O.\,V., Morozov~E.\,V., Pagano~M.}
 On the efficiency of bridge Monte-Carlo estimator~// Информатика и её применения,
  2017.  Т.~11. Вып.~2. С.~16--24.

\bibitem{1-luk} %23
\Au{Morozov E., Lukashenko~O., Rumyantsev~A., Ivashko~E.}
A~Gaussian approximation of runtime estimation in a~desktop grid project~// 
9th  Congress (International) on Ultra Modern Telecommunications and Control Systems 
and Workshops.~--- IEEE, 2017. P.~107--111.

\end{thebibliography}
} }

\end{multicols}

 \label{end\stat}

 \vspace*{-9pt}

\hfill{\small\textit{Поступила в~редакцию 15.04.2019}}


%\renewcommand{\bibname}{\protect\rm Литература}
\renewcommand{\figurename}{\protect\bf Рис.}
\renewcommand{\tablename}{\protect\bf Таблица}