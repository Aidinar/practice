\def\stat{inkova}

\def\tit{СОЧЕТАЕМОСТЬ ЛОГИКО-СЕМАНТИЧЕСКИХ ОТНОШЕНИЙ: КОЛИЧЕСТВЕННЫЕ 
МЕТОДЫ АНАЛИЗА$^*$}

\def\titkol{Сочетаемость логико-семантических отношений: количественные 
методы анализа}

\def\aut{О.\,Ю.~Инькова$^1$, М.\,Г.~Кружков$^2$}

\def\autkol{О.\,Ю.~Инькова, М.\,Г.~Кружков}

\titel{\tit}{\aut}{\autkol}{\titkol}

\index{Инькова О.\,Ю.}
\index{Кружков М.\,Г.}
\index{Inkova O.\,Yu.}
\index{Kruzhkov M.\,G.}


{\renewcommand{\thefootnote}{\fnsymbol{footnote}} \footnotetext[1]
{Разделы~1--3 и~5 данной работы выполнены при финансовой поддержке РНФ (проект  
16-18-10004), раздел~4 выполнен по плановой теме Минобрнауки №\,0063-2019-0010.}}


\renewcommand{\thefootnote}{\arabic{footnote}}
\footnotetext[1]{Институт проблем информатики Федерального исследовательского центра <<Информатика 
и~управление>> Российской академии наук, \mbox{olyainkova@yandex.ru}}
\footnotetext[2]{Институт проблем информатики Федерального исследовательского центра <<Информатика 
и~управление>> Российской академии наук, \mbox{magnit75@yandex.ru}}

\vspace*{-6pt}

  \Abst{Рассматриваются логико-семантические отношения (ЛСО), обеспечивающие 
связность текста, и~их эксплицитные текстовые показатели~--- коннекторы. 
Представлен обзор существующих подходов к~определению и~классификации ЛСО, 
рассмотрены различные варианты сочетаемости ЛСО и~их показателей. Для решения задачи 
количественного анализа сочетаемости ЛСО предлагается использовать надкорпусные базы 
данных (НБД), в~рамках которых для аннотирования ЛСО и~их показателей используется 
двухуровневая фасетная классификация. На материале базы данных генерируется 
и~интерпретируется количественная информация по сочетаемостным характеристикам ЛСО. 
В~частности, показано, что отношения экстенсиональной генерализации гораздо чаще 
сочетаются с~соединительными отношениями, чем, например, отношения спецификации, 
определяются речевые реализации (РР), которые чаще всего являются показателями более чем 
одного отношения, и~т.\,д. Показано, как результаты анализа сочетаемости ЛСО могут быть 
использованы для дальнейшего развития методологии обратимой генерализации 
информационных объектов. Гибкость и~доступность предлагаемого подхода позволяют  
по-но\-во\-му подойти к~пока мало изученной проблеме сочетаемости ЛСО.}
  
  \KW{надкорпусные базы данных; количественный анализ; коннекторы;  
ло\-ги\-ко-се\-ман\-ти\-че\-ские отношения; аннотирование отношений; генерализация 
информационных объектов} 

\DOI{10.14357/19922264190212}
  
\vspace*{-4pt}


\vskip 10pt plus 9pt minus 6pt

\thispagestyle{headings}

\begin{multicols}{2}

\label{st\stat}

\section{Введение}

  Изучение ЛСО как 
обеспечивающих логическую связность текста имеет важное значение 
в~областях лингвистики и~информатики, связанных с~обработкой 
естественного языка. Перед исследователями ставятся задачи, связанные 
с~разметкой ЛСО в~корпусах и~автоматической обработкой текстов на 
естественном языке. В~связи с~этим в~лингвистике, особенно в~корпусной, 
активно обсуждаются среди прочих два вопроса, имеющих непосредственное 
отношение к~связности текста: классификация 
ЛСО и~структура и~сочетаемость их показателей~--- коннекторов.
  
  Эти вопросы тесно взаимосвязаны, хотя между ЛСО и~его показателями нет 
симметрии:
  \begin{itemize}
\item ЛСО может не иметь показателя, а само отношение выводится на 
основе смыслового соотношения фрагментов текста (\textit{Петя не пришел. 
Он заболел});
\item оно может быть выражено несколькими показателями (ср.\ 
\textit{Когда} прозвонил будильник, Петя встал~--- \textit{Едва лишь 
только} прозвонил будильник, \textit{как} Петя \textit{тут же} встал);
\item показатель ЛСО может быть полисемичным (ср.\ \textit{Если} 
надумаешь, \textit{то} позвони (условные ЛСО)~--- \textit{Если} Маша 
умная и~красивая, \textit{то} ее сестра~--- просто чудовище 
(сопоставительные ЛСО)).
\end{itemize}

\vspace*{-3pt}

\section{Существующие подходы к~классификации 
и~сочетаемости логико-семантических отношений}

  Что касается классификации, то на сегодняшний день их существует 
в~опубликованном виде как минимум пять. При этом сам термин 
<<отношение>> понимается в~них довольно по-раз\-но\-му. 

В~ТРС (теории риторической структуры, \textit{англ.}\ RST~--- rhetorical structure
theory~[1]) 
под <<отношением>> понимается любое отношение, свя\-зы\-ва\-ющее два 
фрагмента текста с~пропозициональным содержанием, обеспечивающее 
связность текста. Таким образом, в~дискурсивные отношения попа-\linebreak\vspace*{-12pt}

\pagebreak

\noindent
дают 
и~размечаются как таковые в~основанных на этой классификации корпусах, 
например: 
  \begin{itemize}
\item отношения, в~основе которых лежит общий принцип релевантности 
информации; ср. отношение <<Решение (Solutionhood)>> (\textit{Я~хочу 
есть, пойдем в~<<Шоколадницу>>}) или <<Обеспечение возможности 
(Enablement)>> (\textit{Можешь открыть дверь? Тут вот ключ лежит});
\item анафорические отношения, в~частности ассоциативная анафора 
(\textit{Карин очень фотогенична, ее улыбка~--- просто очаровательна}), 
которые носят название <<Детализация>>, или <<Уточнение 
(Elaboration)>>, в~его разновидности Часть--Це\-лое;
\item разного рода те\-ма-ре\-ма\-ти\-че\-ские отношения, например цепочки 
с~постоянной темой, которые также попадают в~отношение 
<<Детализация>>, или <<Уточнение (Elaboration)>>, но в~его разновидности 
Пред\-мет--Приз\-нак (\textit{Я~--- офицер Иванов, я~родился в~1970~году, 
работаю в~полиции с~1990~года});
\item чисто синтаксические явления, обес\-пе\-чи\-ва\-ющие связность текста 
(например, изъяснительные отношения, носящие имя <<Содержание>>: 
\textit{Я~знаю, что Петя уезжает}).
\end{itemize}

  Близкий подход к~термину <<отношение>> характеризует и~классификацию, 
предложенную Н.~Аше\-ром и~А.~Ласкаридесом (SDRT~--- segmental discourse
representation theory~[2]).
  
  Несколько иной подход, в~большей степени ориенти\-рованный на анализ 
отношений, которые могут быть выражены специальными показателями~--- 
коннекторами, представлен в~трех других классификациях: 
  \begin{itemize}
\item наиболее известная в~кругах корпусных лингвистов классификация, 
используемая для создания Penn Discourse TreeBank (PDTB) под руководством 
Б.~Вебер;
\item недавно опубликованная работа~[3], опи\-ра\-юща\-яся во многом на 
предыдущие работы Т.~Сан\-дер\-са и~классификацию А.~Кнот\-та~[4,~5];
\item классификация, предложенная немецкими коллегами 
в~фундаментальном многотомном труде, посвященном синтаксису 
и~семантике коннекторов немецкого языка, под руководством 
Э.~Брайндл~[6].
\end{itemize}

  Анализ этих пяти классификаций~--- отдельная задача. Подчеркнем лишь два 
момента. Первое~--- это то, что само понятие <<отношение>> имеет в~них 
разное значение. В~наиболее близких авторам по подходу двух последних 
классификациях, в~меньшей степени в~классификации команды Б.~Вебер, 
классифицируются именно ЛСО, которые могут быть выражены 
соответствующим показателем~--- коннектором. Остальные отношения, 
в~частности перечисленные выше для ТРС, остаются за пределами 
исследования и~разметки.
  
  Второе~--- это тот факт, что все пять классификаций между собой плохо 
соотносимы. Чтобы преодолеть эту проблему, в~работе членов команды 
Б.~Вебер~[7] предлагается своего рода стандарт (Core Collection) дискурсивных 
отношений. По этому же пути пошли разработчики французского 
аннотированного корпуса ANNODIS, взяв для разметки~17~<<наиболее 
распространенных дискурсивных отношений>>~\cite[с.~2651]{8-in}. Но эти два 
списка уже не совпадают, а~сопоставление отношений, име\-ющих одинаковые 
названия, вызывает определенные трудности. Подробно эта проблема 
обсуждается в~работе~\cite{3-in}.
  
  К этим трудностям можно добавить еще одну, которая часто остается за 
пределами внимания исследователей: речь идет о возможном существовании 
более одного отношения, в~данном случае именно ЛСО между фрагментами 
текста, будь то отношения, эксплицитно выраженные коннектором, или 
имплицитные. Наибольший интерес для авторов, учитывая особенности 
разработанной НБД, представляет первый случай.
  
  Вопрос о возможности существования между фрагментами текста 
нескольких ЛСО предполагает решение двух теоретических проблем. С~одной 
стороны, необходимо понять, в~какой степени ЛСО совместимы друг с~другом 
и~в~какой степени их сочетание возможно и~предсказуемо. С~другой стороны, 
в~плане взаимоотношения синтаксиса и~семантики необходимо определить те 
механизмы, которые управляют сочетаемостью показателей ЛСО, коннекторов, 
и~в~какой степени можно говорить о~композициональности их значения.
  
  В трех подходах, которые легли в~основу создания банков данных 
о~дискурсивных отношениях (RST, SDRT и~PDTB), этот вопрос решается  
по-раз\-но\-му, если решается.
  
  Теория риторической структуры, которая квалифицирует себя как теория, ставящая своей целью 
описание организации текста, должна рассматриваться скорее как инструмент 
для лингвистического анализа, чем модель организации дискурса, способная 
предсказать эмпирические наблюдения над его функционированием. Заметим 
также, что RST в~анализе мало опирается на коннекторы, считая их слабыми 
индикаторами связности дискурса.
  
  Наконец, анализ структуры текста в~RST подчиняется некоторым условиям 
(обоснование которых, хотя и~недостаточно четко, сформулировано Манном 
и~Томпсон в~основополагающей работе~\cite{1-in}), одно из которых 
в~принципе ставит под вопрос возможность существования нескольких 
дискурсивных отношений между анализируемыми единицами. Это условие 
\textit{единственности}, согласно которому не может быть ситуации, когда два 
разных отношения существуют между одинаковыми фрагментами текста. Из 
этого можно заключить, что наличие нескольких дискурсивных отношений 
между анализируемыми единицами считается скорее исключением.
  
  Теория SDRT предусматривает возможность существования нескольких 
отношений между дискурсивными единицами, что является прямым 
след\-ст\-ви\-ем Принципа максимизации связности текс\-та. 
Согласно этому 
принципу, если интерпретация~И некоторой последовательности высказыва\-ний 
является более связной, чем интерпретация~И$_1$, то множество отношений, 
выявленных в~И$_1$, присутствует и~в~И. Основное требование, которое 
предъявляется к~сосуществующим отношениям,~--- требование связ\-ности: 
предпочтение отдается таким отрезкам дискурса, в~которых выявленные 
отношения не противоречат друг другу. Этот подход используется в~базе 
данных\linebreak ANNODIS ({\sf http:/\!/redac.univ-tlse2.fr/corpus/annodis/\linebreak annodis\_rr.html}), 
в~которой аннотированы тексты на французском языке. Аннотация 
производится при помощи программного обеспечения Glozz ({\sf 
http://www.glozz.org}), которое необходимо инсталлировать для визуализации 
результатов разметки. 
  
  В рамках проекта разметки дискурсивных отношений PDTB вопрос 
о~сосуществовании нескольких отношений между одними и~теми же 
фрагментами текста решается с~практической точки зрения. В~общем случае, 
когда между фрагментами текста присутствуют два коннектора (например, 
\textit{so for instance, but then, and furthermore}), они аннотируются по 
отдельности и~для каждого из них указываются фрагменты текста 
(<<аргументы>>), которые они связывают~\cite{9-in}. 
  
  Единственный случай, при котором в~PDTB допускается проставлять больше 
чем одно отношение,~--- это случай неоднозначности при имплицитно 
выраженных отношениях. Тогда проставляются все возможные для данной 
последовательности отношения: 
  \begin{enumerate}[(1)]
\item This cannot be solved by provoking a further downturn; reducing the supply of goods does 
not solve inflation (Implicit\,=\,\textit{so} Contingency.Cause.Result, Implicit\;=\;\textit{instead} 
Exp.Alt.Chosen alt). Our advice is this: Immediately return the government surpluses to the 
economy through incentive-maximizing tax cuts, and find some monetary policy target that 
balances both supply and demand for money.\footnote{Пример из~\cite{9-in}, перевод на 
русский язык: \textit{Это невозможно устранить, не вызвав нового спада; уменьшение 
поставок не ведет к~снижению инфляции} (имплицит.\ отн-ия: <<результат>> и~<<выбор 
альтернативы>>). \textit{Наш совет~--- немедленно вернуть правительственные излишки 
в~экономику за счет стимулирующего снижения налогов и~выработать для денежной 
политики новую цель, которая бы уравновешивала предложение и~спрос на деньги}.}
\end{enumerate}

  Для визуализации результатов разметки в~PDTB необходим платный доступ 
к~базе данных и~инсталляция соответствующего программного обес\-пе\-чения.

\vspace*{-4pt}
  
\section{Разметка логико-семантических отношений в~надкорпусной
базе данных коннекторов и~предлагаемые 
решения}

  В целях систематического исследования показателей ЛСО была разработана 
НБД коннекторов, позволяющая осуществлять 
аннотирование коннекторов и~их переводных эквивалентов в~корпусе 
параллельных текстов Национального корпуса русского языка
(НКРЯ). В~рамках данной работы не будем рассматривать 
общую структуру НБД (об этом см., например,~\cite{10-in}), а~подробнее 
остановимся на той ее части, которая связана с~аннотированием показателей 
ЛСО в~русском языке.

    \begin{table*}[b]\small %tabl1
    \vspace*{-6pt}
  \begin{center}
  \Caption{Сочетания показателей ЛСО в~НБД}
  \vspace*{2ex}
  
  \begin{tabular}{|l|c|}
  \hline
\multicolumn{1}{|c|}{Сочетание ЛСО}&
\tabcolsep=0pt\begin{tabular}{c}Число\\ употреблений\\ в~НБД\end{tabular}\\
\hline
Соединительные; экстенсиональная генерализация&120\hphantom{9}\\
<<Вопреки ожидаемому>>; уступительные&85\\
Пропозициональное сопутствование; соединительные&69\\
Соединительные; спецификация&64\\
Иллокутивное сопутствование; соединительные&44\\
Спецификация; сравнительные&41\\
Аддитивные иллокутивные; соединительные&38\\
<<Вопреки ожидаемому>>; пропозициональное сопутствование&36\\
Аддитивные пропозициональные; соединительные&35\\
<<Вопреки ожидаемому>>; возместительное противопоставление&34\\
Временные; соединительные&33\\
Аддитивные иллокутивные; экстенсиональная генерализация&25\\
<<Вопреки ожидаемому>>; противительно-уступительные&22\\
Интенсиональная генерализация; соединительные&22\\
<<Вопреки ожидаемому>> иллокутивные; уступительные иллокутивные&21\\
\multicolumn{1}{|c|}{$\cdots$}&\multicolumn{1}{|c|}{$\cdots$}\\
Контраст; спецификация&\hphantom{9}1\\
\multicolumn{1}{|c|}{$\cdots$}&\multicolumn{1}{|c|}{$\cdots$}\\
\hline
\end{tabular}
\end{center}
\end{table*}
  
  Для аннотирования используется двухуровневая фасетная классификация: 
аннотированию подлежит, с~одной стороны, форма коннекторов, в~какой они 
встретились в~тексте (в~терминах базы данных эти формы получили название 
\textit{речевые реализации}), с~другой стороны, аннотируется контекст 
употребления этих коннекторов. Речевые реализации аннотируются с~помощью структурных 
признаков~--- указывается структурный тип коннектора (одноэлементный, 
многоэлементный, двухкомпонентный или $n$-ком\-по\-нент\-ный) и~входящие 
в~него элементы. Контексты употребления РР аннотируются по нескольким 
основаниям, из которых наибольший интерес представляет кластер признаков 
<<Ло\-ги\-ко-се\-ман\-ти\-че\-ские отношения>>. При этом если РР включает 
в~себя несколько элементов, маркирующих различные ЛСО, то РР 
присваивается более одного признака из данного кластера (см.\ пример на 
рис.~1). Если же для РР эксперту представляются равновероятными несколько 
ЛСО, то он может использовать введенные в~систему неоднозначные 
(комбинированные) признаки.
  
%\pagebreak

\end{multicols}

\begin{figure*} %fig1
  \vspace*{1pt}
    \begin{center}  
  \mbox{%
 \epsfxsize=147.81mm 
 \epsfbox{ink-1.eps}
 }
 \end{center}
\vspace*{-9pt}
  \Caption{Пример аннотирования контекста с~несколькими ЛСО}
  \end{figure*}

\begin{multicols}{2}

\noindent
  Описанная система разметки позволяет получить количественную 
информацию по час\-тот\-ности  различных сочетаний показателей ЛСО, 
аннотированных в~корпусе. 

Например, задавая то или иное сочетание 
признаков из кластера <<Отношения>>, можно подсчитать, сколько раз 
в~аннотированном корпусе встречаются употребления коннекторов, которые 
включают в~себя элементы, маркирующие соответствующие ЛСО. 

В~табл.~1 
представлены по мере убывания зарегистрированные в~НБД сочетания.
  Это, однако, абсолютные числа. Для интерпретации данной информации 
необходимо учитывать, сколько аннотаций для каждого ЛСО было построено. 
Например, для экстенсиональной генерализации, т.\,е.\ генерализации, 
осуществляющей переход от частной ситуации к~общей, 
создано~524~аннотаций, из них, как следует из табл.~1, в~120~аннотациях, 
т.\,е.\ в~23\% случаев, показатель этого вида генерализации \textit{вообще} 
сочетается с~показателем соединительных отношений~--- союзом~\textit{и}; 
ср.~(2) с~(3):
{\looseness=1

}
  \begin{enumerate}[(1)]
  \setcounter{enumi}{1}
\item Чипа угостил нас разведенным спиртом. \textit{И~вообще}, проявил 
услужливость. [Сергей Довлатов. Чемодан (1986)]
\item Никогда я не совал своего носа в~литературу и~в~политику, не искал 
популярности в~полемике с~невеждами, не читал речей ни на обедах, ни на 
могилах своих товарищей$\ldots$ \textit{Вообще} на моем ученом имени 
нет ни одного пятна и~пожаловаться ему не на что. [А.\,П.~Чехов. Скучная 
история (1889)]
\end{enumerate}

  Для отношения спецификации в~НБД сформирована~1381~аннотация, 
и~только в~64~из них (т.\,е.\ менее~5\%) зафиксировано сочетание показателя 
этого ЛСО с~союзом~\textit{и}: 
  \begin{enumerate}[(1)]
  \setcounter{enumi}{3}
  \item Вчерашний день, таким образом, помаленьку высветлялся, но Степу 
сейчас гораздо более интересовал день сегодняшний~\textit{и,~в~част\-ности}, 
появление в~спальне неизвестного, да еще с~закуской и~водкой. 
[М.\,А.~Булгаков. Мастер и~Маргарита (1929--1940)]
  \end{enumerate}
  
  Для отношения исключения сочетаемость с~другими ЛСО вообще не 
характерна. Для него создано~356~аннотаций, и~только в~5 из них (т.\,е.\ 
всего~1,2\%) зафиксировано сочетание этого ЛСО с~ка\-ким-ли\-бо другим. 
Этих отношений три: наименее избирательное соединительное (3~аннотации), 
<<вопреки ожидаемому>> (1~аннотация) и~аддитивные иллокутивные 
(1~аннотация):
  \begin{enumerate}[(1)]
  \setcounter{enumi}{4}
  
\item Весь этот мир куда-то пропал. \textit{И~только} ремень все еще цел. 
[Сергей Довлатов. Чемодан (1986)]
\item Одно это я могу сказать против своего чувства$\ldots$ Это важно 
$<\ldots>$ \textit{Но, кроме этого}, сколько бы я~ни искал, я~ничего не 
найду, что бы сказать против моего чувства. [Л.\,Н.~Толстой. Анна 
Каренина (1878)]
\end{enumerate}

  Можно также сгруппировать результаты в~каж\-дом из приведенных в~табл. 1 
сочетаний по конкретным РР (табл.~2, вертикальная черта <<$\vert$>> 
в~РР означает, что ее элементы разделены текстом). Например, можно узнать, какие 
РР сочетают в~себе\linebreak\vspace*{-12pt}

  %\begin{table*}
  {\small %tabl2
  %\vspace*{-1pt}

\begin{center}
  \noindent
  \parbox{72mm}{{{\tablename~2}\ \ \small{Сочетание показателей спецификации с~показателем соединительных ЛСО}}}
  \vspace*{2ex}
  
  
    \tabcolsep=14pt
    \begin{tabular}{|l|c|}
  \hline
\multicolumn{1}{|c|}{РР}&\tabcolsep=0pt\begin{tabular}{c}Число\\ употрeблений\\ в~НБД\end{tabular}\\
\hline
и особенно&29\hphantom{9}\\
и в~частности&8\\
и между прочим&7\\
и в~особенности&5\\
и среди\_pronom pers&4\\
и, между прочим&1\\
и, например&1\\
и, скажем&1\\
и$\vert$в особенности&1\\
и$\vert$в том числе&1\\
и$\vert$между прочим&1\\
и$\vert$например&1\\
и$\vert$тем более&1\\
особенно$\|$и особенно&1\\
и именно&1\\
и между ними&1\\
\hline
\end{tabular}
\end{center}}
%\end{table*}

\addtocounter{table}{1}

\vspace*{9pt}

  
\noindent
 показатели соединительных отношений и~отношений 
спецификации (4-я строка табл.~1) и~как часто каждая их этих РР встречается 
в~аннотированной части корпуса.
  

  
  Наконец, проанализировав перекрестную таб\-ли\-цу отношений в~кластере 
<<Отношения>>, можно выяснить, с~какими другими ЛСО чаще всего 
сочетается то или иное ЛСО. Например, в~табл.~3 показано, с~какими другими 
отношениями чаще всего сочетается отношение спецификации в~оригинальных 
русских текстах в~параллельном рус\-ско-фран\-цуз\-ском подкорпусе.


  
  \begin{table*}\small %tabl3
  \begin{minipage}[t]{80mm}
  \begin{center}
  \Caption{Сочетаемость показателей ЛСО спецификации с~показателями других ЛСО 
в~текстах на русском языке в~рус\-ско-фран\-цуз\-ском подкорпусе}
  \vspace*{2ex}
  
  \tabcolsep=2.5pt
  \begin{tabular}{|l|c|}
  \hline
\multicolumn{1}{|c|}{ЛСО}&\tabcolsep=0pt\begin{tabular}{c}Число\\ употреблений\\ в~НБД\end{tabular}\\
\hline
Соединительные&32\hphantom{9}\\
Сравнительные&12\hphantom{9}\\
Альтернатива&7\\
Условные&6\\
Сопоставительные&6\\
Временные&3\\
Аддитивные иллокутивные&3\\
<<Вопреки ожидаемому>> иллокутивные&3\\
Пропозициональная альтернатива&3\\
Причина&2\\
Тождество&2\\
Переформулирование&1\\
<<Вопреки ожидаемому>>&1\\
Контраст&1\\
\hline
\end{tabular}
\end{center}
\end{minipage}
\hfill
%\end{table*}
 %\begin{table*}\small %tabl4
    \begin{minipage}[t]{80mm}
  \begin{center}
  \Caption{Сочетаемость показателей ЛСО спецификации с~показателями других ЛСО 
в~текстах переводов на французский язык}
  \vspace*{2ex}
  
  \tabcolsep=2.5pt
  \begin{tabular}{|l|c|}
  \hline
\multicolumn{1}{|c|}{ЛСО}&\tabcolsep=0pt\begin{tabular}{c}Число\\ употреблений\\ в~НБД\end{tabular}\\
\hline
Соединительные&32\hphantom{9}\\
Сравнительные&12\hphantom{9}\\
Альтернатива&7\\
Условные&6\\
Сопоставительные&6\\
<<Вопреки ожидаемому>> иллокутивные&3\\
Аддитивные иллокутивные&3\\
Пропозициональная альтернатива&3\\
Временные&3\\
<<Вопреки ожидаемому>>&1\\
Причина&1\\
Тождество&1\\
Контраст&1\\
Переформулирование&1\\
\hline
\end{tabular}
\end{center}
\end{minipage}
\vspace*{6pt}
\end{table*}


    
  Как уже отмечалось, НБД коннекторов включает в~себя информацию 
о~переводных эквивалентах русских коннекторов, которые аннотируются 
аналогичным образом. В~результате появляется возможность сформировать 
аналогичную таблицу по сочетаемости отношения спецификации в~переводных 
французских текстах (табл.~4). Как и~следовало ожидать, результаты довольно 
близки, поскольку учтенные здесь французские коннекторы являются 
переводами значительной части соответствующих русских коннекторов (не 
считая тех, которые переводились не-кон\-нек\-то\-ра\-ми или, возможно, 
коннекторами, в~которых часть семантики русского коннектора была утрачена). 
Также следует учитывать, что для некоторых русских коннекторов\linebreak в~НБД 
может быть зарегистрировано несколько переводов, выполненных разными 
авторами. Наиболее заметным отличием, на первый взгляд, представляется 
значительное уменьшение степени\linebreak
 со\-че\-та\-емости с~временн$\acute{\mbox{ы}}$ми 
и~сопоставительными отношениями, а~также отсутствие в~перево-\linebreak\vspace*{-12pt}

\pagebreak

\noindent
де 
сочетаемости с~отношениями переформулирования и~аддитивными 
иллокутивными отноше-\linebreak ниями. 

  

\vspace*{-6pt}

\section{Сочетаемость показателей логико-семантичесих отношений 
как~объект генерализации}

  Результаты количественного анализа со\-че\-та\-емости ЛСО могут быть 
использованы для дальнейшего развития методологии обратимой 
генерализации информационных объектов в~текстовой форме. В~ряде 
работ~\cite{11-in, 12-in, 13-in} эта методология разрабатывалась для моделей 
перевода. Здесь представлен другой пример обратимой генерализации 
текстовых объектов: поскольку показатели ЛСО могут маркировать различные 
сочетания ЛСО, аннотации РР могут быть генерализованы на этом основании 
по нескольким уровням обобщения. Переходы от аннотаций употреблений РР 
к~генерализуемым объектам различных уровней и~обратно обеспечиваются 
благодаря вычисляемым связям, реализуемым на основе структуры НБД.

 \begin{figure*} %fig2
 \vspace*{1pt}
    \begin{center}  
  \mbox{%
 \epsfxsize=152.447mm 
 \epsfbox{ink-2.eps}
 }
 \end{center}
\vspace*{-9pt}
\Caption{Генерализация употреблений РР по маркируемым ими сочетаниям ЛСО}
\end{figure*}
  
  На рис.~2 показан фрагмент иерархии аннотаций употреблений РР, 
включающий в~себя~5~уровней. На нижнем уровне находятся исходные\linebreak 
генерализуемые объекты, содержащиеся в~НБД,~--- упо\-треб\-ле\-ния РР русского 
языка, которые были найдены и~аннотированы в~рус\-ско-фран\-цуз\-ском 
корпусе параллельных текстов НКРЯ. Остальные уровни представляют собой 
генерализуемые объекты разной степени детализации. 
  



  Генерализуемые объекты нижнего уровня представляют собой аннотации 
употреблений РР русского языка, которые одновременно содержат два признака 
из кластера ЛСО: <<соединительные отношения>> и~<<отношения 
спецификации>>. 
  
  На первом уровне генерализации (второй уровень иерархии, начиная снизу) 
исходные генерализуемые объекты сгруппированы по формам РР, которые 
реализуют указанные отношения, наиболее частотными из которых являются 
формы \textit{и~особенно}, \textit{и~в~частности}, \textit{и~между прочим}, 
\textit{и~в~особенности}.
  
  На втором уровне генерализуемые объекты объединены в~одну группу, 
которая соответствует сочетаниям ЛСО $\{$\textbf{соединительные}, 
\textbf{спецификация}, \ldots$\}$, которые одновременно включают в~себя 
соединительные отношения и~отношения спецификации независимо от того, 
какие именно РР маркируют эти сочетания. 
  
  На третьем уровне генерализации находятся сочетания ЛСО, включающие 
в~себя по меньшей мере одно из указанных отношений (и,~возможно, другие 
отношения). Так, группа $\{$\textbf{соединительные}, \ldots$\}$ 
объединяет~478~аннотаций, при этом в~нее полностью входит группа 2-го 
уровня генерализации $\{$\textbf{соединительные}, \textbf{спецификация}, 
\ldots$\}$. В~то же время группа $\{$\textbf{спецификация}, \ldots$\}$ 
объединяет~1381~аннотацию, при этом в~нее также полностью входит группа 
2-го уровня генерализации $\{$\textbf{соединительные}, 
\textbf{спецификация}, \ldots$\}$. Таким образом, данная иерархия допускает 
альтернативность генерализации при переходе со второго уровня на третий.
  
  
  Четвертый и~последний уровень генерализации включает единственную 
группу $\{\ldots\}$, которая соответствует <<абстрактному>> сочетанию ЛСО, 
т.\,е.\ всем возможным сочетаниям ЛСО, которые могут маркироваться 
с~помощью РР. В~данную группу входят все лежащие ниже группы 3-го 
и~более низких уровней.

\vspace*{-6pt}
  
\section{Заключительные замечания}

\vspace*{-2pt}

  Разработанная НБД показателей ЛСО имеет\linebreak
   целый ряд преимуществ по 
сравнению с~существующими информационными ресурсами, поз\-во\-ля\-ющи\-ми 
анализировать дискурсивные, или риторические, отношения:
  \begin{itemize}
\item НБД является доступным ресурсом, не требующим установки 
специального программного обеспечения и~находящимся в~открытом 
доступе (по крайней мере, ее фрагмент, который содержит 6654~аннотаций);\\[-14pt]
\item применяемый для разметки ЛСО теоретический подход не только 
допускает сочетаемость показателей различных ЛСО, но и~дает возможность 
аннотировать такие случаи;\\[-14pt]
\item применяемая система разметки позволяет с~помощью дополнительных 
тегов аннотировать такие случаи, когда одно из ЛСО не выражено 
эксплицитно;\\[-14pt]
\item используемая система разметки позволяет получать количественную 
информацию о~сочетаемости показателей различных ЛСО, а~следовательно, 
решать этот вопрос на основе не отдельных эмпирических наблюдений 
и~гипотез, а~верифицируемых статистических данных, полученных на 
представительном массиве аннотаций.
\end{itemize}
  
  Все это позволяет по-новому подойти к~пока еще мало изученной 
теоретической проблеме со\-че\-та\-емости показателей ЛСО и~в перспективе на 
основе дальнейшего семантического анализа позволит ответить на вопросы о 
том, в~какой степени ЛСО совместимы друг с~другом (для некоторых ЛСО 
ответ на этот вопрос уже получен) и~в какой степени их сочетание 
предсказуемо, а также определить те семантические и~синтаксические 
механизмы, которые управляют сочетаемостью показателей ЛСО.
  
  
 \vspace*{-6pt}
  
  
{\small\frenchspacing
 {%\baselineskip=10.8pt
 \addcontentsline{toc}{section}{References}
 \begin{thebibliography}{99}
 
 \vspace*{-2pt}
 
 
\bibitem{1-in}
\Au{Mann W.\,C., Thompson~S.\,A.} Rhetorical structure theory: Toward a functional theory of 
text organization~// Text, 1988. Vol.~8. No.\,3. 
P.~243--281.
\bibitem{2-in}
\Au{Asher N., Lascarides~A.} Logics of conversation.~--- Cambridge: Cambridge University 
Press, 2003. 526~p.
\bibitem{3-in}
\Au{Sanders T.\,J.\,M., Demberg~V., Hoek~J., Scholman~M.\,C.\,J., Torabi~A.\,F., Zufferey~S., 
Evers-Vermeul~J.} Unifying dimensions in coherence relations: How various annotation 
frameworks are related~// Corpus Linguist. Ling., 2018. 71~p. 
doi: 10.1515/cllt-2016-0078. 
\bibitem{4-in}
\Au{Knott A.} A~data-driven methodology for motivating a~set of coherence relations.~--- 
Edinburgh: University of Edinburgh, 1996. Ph.D. Thesis. {\sf 
https://www. era.lib.ed.ac.uk/handle/1842/583}.
\bibitem{5-in}
\Au{Knott~A., Sanders~T.} The classification of coherence relations and their linguistic 
markers: An exploration of two languages~// J.~Pragmatics, 1998. Vol.~30. No.\,2.  
P.~135--175.
\bibitem{6-in}
\Au{Breindl E., Volodina~A., \mbox{Wa{\!\!\ptb{\ss}}\,ner}~U.\,H.} Handbuch der deutschen 
Konnektoren~2. Semantik der deutschen Satzverkn$\ddot{\mbox{u}}$pfer.~--- Berlin: Walter 
de Gruyter, 2014. 1327~p.
\bibitem{7-in}
\Au{Bunt H., Prasad~R.} ISO DR-Core (ISO 24617-8): Core concepts for the annotation of 
discourse relations~// 12th Joint ACL-ISO Workshop on Interoperable Semantic Annotation 
Proceedings.~--- Portoroz, 2016. P.~45--54.
\bibitem{8-in}
\Au{Ho-Dac L.-M., P$\acute{\mbox{e}}$ry-Woodley~M.-P.} Annotation des structures 
discursives: l'exp$\acute{\mbox{e}}$rience ANNODIS~// 4e Congr$\grave{\mbox{e}}$s 
Mondial de Linguistique \mbox{Fran{\!\ptb{\c{c}}}aise}.~--- Berlin, 2014. 
P.~2647--2661.
\bibitem{9-in}
\Au{Webber B.} Concurrent discourse relations~// 
Компьютерная лингвистика и~интеллектуальные технологии.~---
М.: Изд-во РГГУ, 2016. Вып.~15. С.~D. 
{\sf http:// www.dialog-21.ru/media/3815/webber.pdf}.
\bibitem{10-in}
\Au{Зацман И.\,М., Инькова~О.\,Ю., Кружков~М.\,Г., Попкова~Н.\,А.} Представление 
кроссязыковых знаний о~коннекторах в~надкорпусных базах данных~// Информатика 
и~её применения, 2016. Т.~10. Вып.~1. С.~106--118.
\bibitem{11-in}
\Au{Зацман И.\,М., Мамонова~О.\,С., Щурова~А.\,Ю.} Обратимость и~альтернативность 
генерализации моделей\linebreak
 перевода коннекторов в~параллельных текстах~// Сис\-те\-мы 
и~средства информатики, 2017. Т.~27. №\,2. С.~125--142.
\bibitem{12-in}
\Au{Зацман И.\,М., Кружков~М.\,Г., Лощилова~Е.\,Ю.} Методы анализа частотности 
моделей перевода коннекторов и~обратимость генерализации статистических данных~// 
Системы и~средства информатики, 2017. Т.~27. №\,4. С.~164--176.
\bibitem{13-in}
\Au{Зацман И.\,М.} Методология обратимой генерализации в~контексте классификации 
информационных трансформаций~// Системы и~средства информатики, 2018. Т.~28. №\,2. 
С.~128--144.
 \end{thebibliography}

 }
 }

\end{multicols}

\vspace*{-6pt}

\hfill{\small\textit{Поступила в~редакцию 29.03.19}}

\vspace*{6pt}

%\pagebreak

%\newpage

%\vspace*{-29pt}

\hrule

\vspace*{2pt}

\hrule

\vspace*{-2pt}

\def\tit{COMPATIBILITY OF~LOGICAL SEMANTIC RELATIONS: METHODS~OF~QUANTITATIVE~ANALYSIS}


\def\titkol{Compatibility of logical semantic relations: Methods of quantitative analysis}

\def\aut{O.\,Yu.~Inkova and M.\,G.~Kruzhkov}

\def\autkol{O.\,Yu.~Inkova and M.\,G.~Kruzhkov}

\titel{\tit}{\aut}{\autkol}{\titkol}

\vspace*{-15pt}


\noindent
Institute of Informatics Problems, Federal Research Center ``Computer Science and 
Control'' of the Russian Academy of Sciences, 44-2~Vavilov Str., Moscow 119333, 
Russian Federation

\def\leftfootline{\small{\textbf{\thepage}
\hfill INFORMATIKA I EE PRIMENENIYA~--- INFORMATICS AND
APPLICATIONS\ \ \ 2019\ \ \ volume~13\ \ \ issue\ 2}
}%
 \def\rightfootline{\small{INFORMATIKA I EE PRIMENENIYA~---
INFORMATICS AND APPLICATIONS\ \ \ 2019\ \ \ volume~13\ \ \ issue\ 2
\hfill \textbf{\thepage}}}

\vspace*{4pt}


   

\Abste{The paper deals with logical semantic relations (LSR) that ensure text 
coherence and examines their explicit markers in text, i.\,e., connectives. An 
overview of existing approaches to definition and classification of LSRs is
 presented; different LSR combination patterns are examined. To make quantitative 
 assessment of LSR compatibility, one should be able to annotate LSRs and their 
 markers in texts. To support such annotation, a~supracorpora database was developed, 
 including customizable faceted classifications for LSRs and connectives. Based on the 
 database data, quantitative information on relations combinations was accumulated 
 and interpreted. For example, it was demonstrated that adjoining relations 
 collocate with extensional generalization relations much more often than with 
 relations of specifications; discourse realizations commonly used to mark more 
 than one relation at a time were identified, etc. The results of LSR 
 compatibility analysis were used to elaborate the methodology of reversible 
 generalization of information objects. High flexibility and accessibility of 
 the proposed approach allows researchers 
to address the underinvestigated problem of LSR compatibility.}
  
  \KWE{databases; quantitative analysis; connectives; logical semantic relations; 
  annotation of relations; generalization of information objects}

 
  
  
 \DOI{10.14357/19922264190212}

\vspace*{-18pt}

 \Ack
 
 \vspace*{-4pt}
 
  \noindent
  The work was carried out at the Institute of Informatics Problems (FRC CSC 
RAS); sections 1-3 and~5 were funded by the Russian Science Foundation according 
to the research project No.\,16-18-10004; and section~4 was carried out as a~part of the 
planned research topic No.\,0063-2019-0010 of the Ministry of Education and
Science of Russian Federation.


%\vspace*{6pt}

  \begin{multicols}{2}

\renewcommand{\bibname}{\protect\rmfamily References}
%\renewcommand{\bibname}{\large\protect\rm References}

{\small\frenchspacing
 {%\baselineskip=10.8pt
 \addcontentsline{toc}{section}{References}
 \begin{thebibliography}{99}
\bibitem{1-in-1}
\Aue{Mann, W.\,C., and S.\,A.~Thompson.} 1988. Rhetorical structure theory: Toward 
a~functional theory of text organization. \textit{Text} 8(3):243--281.
\bibitem{2-in-1}
\Aue{Asher, N., and A.~Lascarides.} 2003. \textit{Logics of conversation}. Cambridge, 
Cambridge University Press. 526~p. 
\columnbreak

\bibitem{3-in-1}
\Aue{Sanders, T.\,J.\,M., V.~Demberg, J.~Hoek, M.\,C.\,J.~Scholman, A.\,F.~Torabi, 
S.~Zufferey, and J.~Evers-Vermeul.} 2018. Unifying dimensions in coherence relations: How 
various annotation frameworks are related. \textit{Corpus Linguist. Ling.} 
71~p. doi: 10.1515/cllt-2016-0078. 
% Available at: {\sf https://doi.org/10.1515/cllt-2016-0078} (accessed March~25,  2019).
\bibitem{4-in-1}
\Aue{Knott, A.} 1996. A~data-driven methodology for motivating a~set of coherence relations. 
Edinburg: University of Edinburgh. Ph.D. Thesis.  Available at: {\sf 
https://www.\linebreak era.lib.ed.ac.uk/handle/1842/583} (accessed March~25, 2019).
\bibitem{5-in-1}
\Aue{Knott, A., and T.~Sanders.} 1998. The classification of coherence relations and their 
linguistic markers: An exploration of two languages. \textit{J.~Pragmatics} 30(2):135--175.
\bibitem{6-in-1}
\Aue{Breindl,~E., A.~Volodina, and U.\,H.~\mbox{Wa{\!\ptb{\ss}}ner}.} 2014. 
\textit{Handbuch der deutschen Konnektoren~2. Semantik der deutschen 
Satzverkn$\ddot{\mbox{u}}$pfer}. Berlin: Walter de Gruyter. 1327~p.
\bibitem{7-in-1}
\Aue{Bunt, H., and R.~Prasad.} 2016. ISO DR-Core (ISO 24617-8): Core concepts for the 
annotation of discourse relations. \textit{12th Joint ACL-ISO Workshop on Interoperable 
Semantic Annotation (ISA-12) Proceedings}. Portoroz. 45--54.
\bibitem{8-in-1}
\Aue{Ho-Dac, L.-M., and M.-P.~P$\acute{\mbox{e}}$ry-Woodley.} 2014. Annotation des 
structures discursives: l'exp$\acute{\mbox{e}}$rience ANNODIS. \textit{4e 
Congr$\grave{\mbox{e}}$s Mondial de Linguistique \mbox{Fran{\!\!\ptb{\c{c}}}aise}}. 
Berlin.  2647--2661.
\bibitem{9-in-1}
\Aue{Webber, B.} 2016. Concurrent discourse relations. 
\textit{Computational linguistics and 
intellectual technologies}. 
Moscow:
Russian State University for the Humanities \mbox{Publs}. 15:D. Available at: 
{\sf http://www.dialog-21.ru/media/ 3815/webber.pdf} (accessed 
March~25, 2019).
\bibitem{10-in-1}
\Aue{Zatsman, I.\,M., O.\,Yu.~In'kova, M.\,G.~Kruzhkov, and N.\,A.~Popkova}. 2016. 
Predstavlenie krossyazykovykh znaniy o~konnektorakh v~nadkorpusnykh bazakh dannykh 
[Representation of cross-lingual knowledge about connectors in supracorpora databases]. 
\textit{Informatika i~ee Primeneniya~--- Inform. Appl.} 10(1):106--118.
\bibitem{11-in-1}
\Aue{Zatsman, I.\,M., O.\,S.~Mamonova, and A.\,Yu.~Shchurova.} 2017. Obratimost' 
i~al'ternativnost' generalizatsii modeley perevoda konnektorov v~parallel'nykh tekstakh 
[Reversibility and alternativeness of generalization of connectives translations models in 
parallel texts]. \textit{Sistemy i~Sredstva Informatiki~--- Systems and Means of Informatics} 
27(2):125--142.
\bibitem{12-in-1}
\Aue{Zatsman, I.\,M., M.\,G.~Kruzhkov, and E.\,Yu.~Loshchilova.} 2017. Metody analiza 
chastotnosti modeley perevoda konnektorov i~obratimost' generalizatsii statisticheskikh 
dannykh [Methods of frequency analysis of connectives translations and reversibility of 
statistical data generalization]. \textit{Sistemy i~Sredstva Informatiki~--- Systems and Means of 
Informatics} 27(4):164--176.
\bibitem{13-in-1}
\Aue{Zatsman, I.\,M.} 2018. Metodologiya obratimoy ge\-ne\-ra\-li\-za\-tsii v~kontekste klassifikatsii 
informatsionnykh transformatsiy [Methodology of reversible generalization in context of 
classification of information transformations]. \textit{Sistemy i~Sredstva Informatiki~--- 
Systems and Means of Informatics} 28(2):128--144.
\end{thebibliography}

 }
 }

\end{multicols}

\vspace*{-6pt}

\hfill{\small\textit{Received March 29, 2019}}

%\pagebreak

%\vspace*{-18pt}

  \Contr
  
  \noindent
  \textbf{Inkova Olga Yu.} (b.\ 1965)~--- Doctor of Science (PhD) in philology, 
senior scientist, Institute of Informatics Problems, Federal Research Center 
``Computer Science and Control'' of the Russian Academy of Sciences, 44-2~Vavilov 
Str., Moscow 119333, Russian Federation; \mbox{olyainkova@yandex.ru}
  
  \vspace*{3pt}
  
  \noindent
  \textbf{Kruzhkov Mikhail G.} (b.\ 1975)~--- senior scientist, Institute of 
Informatics Problems, Federal Research Center ``Computer Science and Control'' of 
the Russian Academy of Sciences, 44-2~Vavilov Str., Moscow 119333, Russian 
Federation; \mbox{magnit75@yandex.ru}



\label{end\stat}

\renewcommand{\bibname}{\protect\rm Литература}