\def\stat{kudr+ar}

\def\tit{БАЙЕСОВСКИЕ МОДЕЛИ БАЛАНСА ФАКТОРОВ, ИМЕЮЩИХ АПРИОРНЫЕ РАСПРЕДЕЛЕНИЯ 
ВЕЙБУЛЛА И~НАКАГАМИ$^*$}

\def\titkol{Байесовские модели баланса факторов, имеющих априорные распределения 
Вейбулла и~Накагами}

\def\aut{Е.\,Н.~Арутюнов$^1$, А.\,А.~Кудрявцев$^2$, А.\,И.~Титова$^3$}

\def\autkol{Е.\,Н.~Арутюнов, А.\,А.~Кудрявцев, А.\,И.~Титова}

\titel{\tit}{\aut}{\autkol}{\titkol}

\index{Арутюнов Е.\,Н.}
\index{Кудрявцев А.\,А.}
\index{Титова А.\,И.}
\index{Arutyunov E.\,N.}
\index{Kudryavtsev A.\,A.}
\index{Titova A.\,I.}


{\renewcommand{\thefootnote}{\fnsymbol{footnote}} \footnotetext[1]
{Работа выполнена при частичной финансовой 
поддержке РФФИ (проект 17-07-00577).}}


\renewcommand{\thefootnote}{\arabic{footnote}}
\footnotetext[1]{Институт проб\-лем информатики Федерального исследовательского центра 
<<Информатика и~управ\-ле\-ние>> Российской академии наук, \mbox{enapoleon@mail.ru}}
\footnotetext[2]{Московский государственный университет 
им.~М.\,В.~Ломоносова, факультет вычислительной математики и~кибернетики, 
\mbox{nubigena@mail.ru}}
\footnotetext[3]{Московский государственный университет 
им.~М.\,В.~Ломоносова, факультет вычислительной математики и~кибернетики, 
\mbox{onkelskroot@gmail.com}}

%\vspace*{-2pt}


\Abst{Рассматриваются байесовские модели баланса, в~рамках которых факторы, 
оказывающие влияние на состояние системы, условно разделяются на позитивные, 
т.\,е.\ способствующие функционированию, и~негативные, т.\,е.\ препятствующие 
функционированию. В~качестве показателя эффективности работы системы 
рассматривается отношение негативного фактора к~позитивному~--- индекс баланса. 
Исследование проводится в~предположении о~зависимости факторов от условий 
внешней среды и~невозможности определения точных значений факторов в~каждый 
момент времени в~силу внешних причин: несовершенства измерительного 
оборудования, нехватки материальных и~временных ресурсов и~т.\,п. Также 
предполагается, что законы изменения факторов априори известны и~остаются 
постоянными. Данные предположения обусловливают применение байесовского метода, 
который заключается в~рандомизации исходных параметров и, как следствие, индекса 
баланса, при этом предполагается, что априорные распределения факторов известны. 
Статья продолжает ряд исследований авторов по применению байесовских методов 
в~задачах массового обслуживания и~надежности. В~работе приводятся полученные 
вероятностные характеристики индекса баланса факторов в~случае априорных 
распределений Вейбулла и~Накагами. Результаты представлены в~терминах 
гам\-ма-экс\-по\-нен\-ци\-аль\-ной функции.}

\KW{байесовский подход; модели баланса; смешанные
распределения; распределение Вейбулла; распределение Накагами; 
гам\-ма-экс\-по\-нен\-ци\-аль\-ная функция}

\DOI{10.14357/19922264190210}
  
%\vspace*{4pt}


\vskip 10pt plus 9pt minus 6pt

\thispagestyle{headings}

\begin{multicols}{2}

\label{st\stat}

\section{Введение}

В условиях постоянного усложнения процессов, лежащих в~основе большинства сфер 
человеческой деятельности, классические методы анализа эффективности теряют свою 
актуальность. Все чаще в~качестве инструментов исследования различных систем 
используются показатели, рейтинги и~индексы, что позволяет значительно экономить 
необходимые для проведения исследований ресурсы. При построении математических 
моделей в~задачах исследования эффективности естественно разделять факторы, 
влияющие на систему, на способствующие функционированию целевого объекта 
(p-фак\-то\-ры) и~препятствующие функционированию (n-фак\-то\-ры). 
В~данных условиях для 
исследования системы наиболее наглядным показателем эффективности ее работы 
представляется отношение n-фак\-то\-ра к~p-фак\-то\-ру, которое будет стремиться 
к~единице при приближении системы к~состоянию баланса и~существенно отличаться от 
единицы в~случае непродуктивной деятельности.

Назовем индексом баланса системы отношение $\rho \hm= \lambda/\mu$, где~$\lambda$ 
и~$\mu$ суть n- и~p-фак\-то\-ры соответственно. В~случае исследования сложных 
модифицируемых информационных сис\-тем примером индекса баланса может служить 
коэффициент загрузки сис\-те\-мы, опре\-де\-ля\-емый как отношение ин\-тен\-сив\-ности входящего 
потока к~интенсивности обслуживания, или ожидаемое время безотказной работы, 
представимое в~виде отношения среднего времени безотказной работы к~среднему 
времени вос\-ста\-нов\-ле\-ния системы.

В силу стохастичности среды функционирования любой современной системы значения 
факторов, влияющих на систему, меняются с~течением времени, что создает 
предпосылки для рассмотрения факторов и~индексов, зависящих от них, как 
случайных величин. Вместе с~тем законы, которым подчиняются изменения факторов, 
можно считать неизменными в~рамках конкретной модели, так как глобальные 
изменения окружающей среды происходят довольно редко. Эти предположения 
обосно\-вы\-ва\-ют применимость байесовского метода, в~рамках которого происходит 
рандомизация параметров при помощи известных априорных распределений~\cite{Ku18}.

В статье в~качестве априорных распределений факторов рассматриваются 
распределение Вейбулла и~m-рас\-пре\-де\-ле\-ние Накагами. Ниже приводятся вероятностные 
характеристики индекса баланса, представляющего собой масштабную смесь двух 
распределений гам\-ма-класса.

\vspace*{-6pt}

\section{Основные результаты}

\vspace*{-3pt}


Обозначим через $m(q, \theta)$ m-рас\-пре\-де\-ле\-ние Накагами~\cite{Nakagami} 
с~плот\-ностью

\noindent
\begin{multline*}
m_{q, \theta}(x) = \fr{2 q^q x^{2q-1}}{\theta^q \Gamma(q)}
\exp\left\lbrace{-\fr{q x^2}{\theta}}\right\rbrace, \\[-1pt]
 x>0\,,\ \ q>0\,,\ \ 
\theta>0\,,
\end{multline*}
а через $W(p,\alpha)$~--- распределение Вейбулла с~плот\-ностью

\noindent
$$
w_{p,\alpha}(x) = \fr{px^{p-1}e^{-({x/\alpha})^{p}}}{\alpha^{p}}\,,  \enskip 
x>0\,,\ \  p>0\,,\  \ \alpha>0\,.
$$

\vspace*{-2pt}

\noindent
Рассмотрим гамма-экс\-по\-нен\-ци\-аль\-ную функцию~\cite{KuTi2017}

\noindent
\begin{multline*}
{\sf Ge}_{\alpha,\, \beta} (x) = 
\sum\limits_{k=0}^{\infty}\fr{x^k}{k!}\, \Gamma(\alpha k 
+ \beta)\,, \\[-1pt]
 x\in\mathbb{R}\,, \ \ 0\le\alpha<1\,, \ \ \beta> 0\,.
\end{multline*}

Хорошо известен следующий результат.

\vspace*{2pt}

\noindent
\textbf{Лемма~1.}\
%\begin{lem}\label{lem1}
\textit{Для случайных величин $\xi$ и~$\eta$, имеющих m-рас\-пре\-де\-ле\-ние 
Накагами $m(q,  \theta)$ и~распределение Вейбулла $W(p,\alpha)$ соответственно, для 
$z\hm\in\mathbb{R}$ выполняются соотношения}:
\begin{alignat*}{2}
\e\xi^z &= \left(\fr{\theta}{q}\right)^z \fr{\Gamma(z/2+q)}{\Gamma(q)}\,, &\enskip  
z &> -2q\,; \\
 \e\eta^z &= \alpha^z \Gamma\left(1+\fr{z}{p}\right)\,, &\enskip   z &> -p\,.
\end{alignat*}


Для дальнейших вычислений потребуется следующее утверждение.

\vspace*{2pt}

\noindent
\textbf{Лемма~2.}\
%\begin{lem} \label{lem2}
\textit{Для некоторых $q\hm>0$, $p\hm>0$, $a\hm>0$ и~$b\hm>0$ справедливо}

\noindent
\begin{multline*}
\int\limits_0^{\infty}y^{2q+p-1}e^{-(y/a)^p-(y/b)^2} \,dy ={}\\[-1pt]
{}=
 \begin{cases}
   \displaystyle\fr{b^{2q+p}}{2} \,{\sf Ge}_{p/2,\, (2q+p)/2} \left(-
\left(\fr{b}{a}\right)^p\right), &p<2\,;\\[9pt]
   \displaystyle\fr{a^{2q+p}}{p}\,{\sf Ge}_{2/p,\, (2q+p)/p} \left(-
{\left(\fr{a}{b}\right)^{2}}\right), &p > 2\,;\\[3pt]
   \displaystyle\fr{q \Gamma(q)}{2\left(a^{-2}+b^{-2}\right)^{q +1}}, &p = 2\,.
 \end{cases}
\end{multline*}


\noindent
Д\,о\,к\,а\,з\,а\,т\,е\,л\,ь\,с\,т\,в\,о\,.\ \
Рассмотрим случай $p\hm<2$. Используя теорему Лебега о~предельном переходе, 
получаем:
\begin{multline*}
\int\limits_0^{\infty}y^{2q+p-1}e^{-(y/a)^p-(y/b)^2} \, dy ={}\\[1pt]
{}=
\fr{a^{2q+p}}{p} \int\limits_0^{\infty} t^{2q/p}\sum\limits_{k=0}^\infty 
\fr{(-t)^k}{k!}\, e^{-(a/b)^2t^{2/p}} \, dt={}\\[1pt]
{}= \fr{b^{2q+p}}{2} \sum\limits_{k=0}^\infty \fr{(-1)^k (b/a)^{pk}}{k!} 
\!\int\limits_0^{\infty}\! z^{(pk+p+2q)/2 - 1} e^{-z} \, dz ={}\\[1pt]
{}=\fr{b^{2q+p}}{2} \sum\limits_{k=0}^\infty \fr{(-1)^k (b/a)^{pk}}{k!}\, 
\Gamma\left(\fr{pk+p+2q}{2}\right).
\end{multline*}
Случай $p>2$ рассматривается аналогично. Случай $p\hm=2$ напрямую следует из 
определения гам\-ма-функ\-ции.
Лемма доказана.

\smallskip

Леммы 1 и~2 дают возможность находить вероятностные характеристики индекса 
баланса системы.

\smallskip

\noindent
\textbf{Теорема~1.}\
%\begin{theorem}\label{MW}
\textit{Пусть негативный фактор~$\lambda$ имеет m-рас\-пре\-де\-ле\-ние 
Накагами $m(q, \theta)$, 
а~позитивный фактор~$\mu$ имеет распределение Вейбулла $W(p,\alpha)$, причем~$\lambda$ 
и~$\mu$ независимы. Тогда при $x\hm>0$ плотность, функция распределения 
и~моменты индекса баланса~$\rho$ имеют вид}:
\begin{multline}
\label{f_rho_mW}
f_\rho(x) ={}\\[1pt]
{}=
 \begin{cases}
   \displaystyle\fr{2 q^q \alpha^{2q} x^{2q-1}}
   {{\theta^q \Gamma(q)}}\,{\sf Ge}_{2/p,\,2q/p+1} 
   \left(-\fr{q\alpha^2 x^2}{\theta}\right), &\\[6pt]
   &\hspace*{-10mm}p > 2\,;\\[6pt]
   \displaystyle \fr{p \theta^{p/2}x^{-p-1}}{{\alpha^p q^{p/2}  \Gamma(q)}}\,
   {\sf Ge}_{p/2,\, q+p/2} \left(-\fr{\theta^{p/2}}{q^{p/2}\alpha^p 
x^p}\right), &\\[6pt]
&\hspace*{-10mm}p < 2\,;
 \end{cases}
\end{multline}
\begin{align*}
F_\rho(x) &=
 \begin{cases}
   \displaystyle \fr{2 q^q\alpha^{2q} x^{2q}}{{p\theta^q \Gamma(q)}}\,
    {\sf Ge}_{2/p,\, 2q/p} \left(\!- \fr{q \alpha^2 x^2}{\theta}\!\right), &p > 2;\\[9pt]
   \displaystyle \fr{1}{{\Gamma(q)}}
   \int\limits_{(\sqrt{q/\theta}\,\alpha x)^{-p}}^{\infty}
   \hspace*{-4mm}\hspace*{-7pt} {\sf Ge}_{p/2,\, q + p/2}(-z) \ dz, 
   &p < 2;
 \end{cases}
\\[6pt]
\e\rho^z& =\fr{\theta^{z/2}\Gamma\left(q+z/2\right)\Gamma\left(1-z/p\right)}
{{{q^{z/2}\alpha^z}\Gamma(q)}}\,, \enskip z<p\,.
\end{align*}
\textit{При $p=2$ индекс баланса~$\rho$ имеет распределение Дагума}~\cite{Dagum1977} 
\textit{с~параметрами} $(2, \sqrt{\theta/q}/\alpha, q)$.

\smallskip

\noindent
Д\,о\,к\,а\,з\,а\,т\,е\,л\,ь\,с\,т\,в\,о\,.\ \ Поскольку
$$
f_\rho(x) = \int\limits_0^{\infty} \fr{2 q^q x^{2q-1} y^{2q+p-1}}{{\alpha^p 
\theta^q \Gamma(q)}}\,e^{-q x^2 y^2/\theta- (y/\alpha)^p} \, dy\,,
$$
соотношение~(\ref{f_rho_mW}) следует из леммы~2.

Для функции распределения~$\rho$ при $p\hm>2$ справедливо для $x\hm>0$:
\begin{multline*}
F_\rho(x) = {}\\
{}=\int\limits_{0}^{x} \fr{2 q^q \alpha^{2q} u^{2q-1}}{{\theta^q 
\Gamma(q)}}\,{\sf Ge}_{2/p,\, 2q/p+1} \left(-\fr{q\alpha^2 u^2}{\theta}\right) du 
={}\\
{}= \fr{2 q^q \alpha^{2q}}{{ \theta^q \Gamma(q)}} \sum\limits_{k=0}^\infty 
\fr{(- q \alpha^2 )^k}{{\theta^kk!}} \,\Gamma\left(\fr{2k + 2q + p}{p}\right)\times{}\\
{}\times  
\int\limits_{0}^{x}u^{2q+2k-1}\,du ={}\\
{}= \fr{2 q^q(\alpha x)^{2q}}{{p\theta^q \Gamma(q)}}\sum\limits_{k=0}^\infty 
\fr{\left(- q \alpha^2 x^2\right)^k}{{\theta^kk!}}\,\Gamma\left(\fr{2k + 2q}{p}\right).
\end{multline*}
В случае $p\hm<2$ имеем:
\begin{multline*}
F_\rho(x) = {}\\
{}=\int\limits_{0}^{x}
\fr{p \theta^{p/2}u^{-p-1}}{{\alpha^p q^{p/2}  
\Gamma(q)}}\,{\sf Ge}_{p/2,\, q+p/2} \left(-
\fr{\theta^{p/2}}{q^{p/2}\alpha^pu^p}\right)du={}\\
{} = \int\limits_{0}^{x} \sum\limits_{k=0}^\infty 
\fr{p \theta^{p/2} 
\left(-\left(\sqrt{q/\theta} \alpha u \right)^{-p}\right)^k}{{\alpha^p q^{p/2} u^{p+1}
 \Gamma(q) 
k!}} \times{}\\
{}\times
\Gamma\left(\frac{pk + 2q + p}{2}\right)  \, du={}\\
{}= \fr{1}{{\Gamma(q)}}\int\limits_{(\sqrt{q/\theta} \alpha x)^{-p}}^{\infty}  
\!\sum\limits_{k=0}^\infty \fr{(-z)^k }{{k!}} \,\Gamma\left(\fr{pk + 2q + 
p}{2}\right) \,dz\,.
\end{multline*}

Для нахождения моментов~$\rho$ достаточно воспользоваться независимостью 
случайных величин~$\lambda$ и~$\mu$ и~леммой~1.
Теорема доказана.

\smallskip

Рассмотрим симметричный случай априорных распределений факторов.

\smallskip

\noindent
\textbf{Теорема~2.}\
%\begin{theorem}\label{WM}
\textit{Пусть негативный фактор~$\lambda$ имеет распределение Вейбулла $W(q,\theta)$, 
а~позитивный фактор~$\mu$ имеет m-рас\-пре\-де\-ле\-ние Накагами $m(p, \alpha)$, 
причем~$\lambda$ и~$\mu$ независимы. Тогда при $x\hm>0$ 
плотность, функция распределения и~моменты индекса баланса~$\rho$ имеют вид}:

\noindent
\begin{align*}
f_\rho(x) &=
 \begin{cases}
   \displaystyle \fr{q x^{q-1} (\alpha/p)^{q/2}}{{\theta^q  \Gamma(p)}} \,
{\sf Ge}_{q/2,\,q/2+p}\left(-\fr{ \alpha^{q/2} x^q}{p^{q/2}\theta^q}\right),\hspace*{-3.61624pt} 
&\\[2pt]
&\hspace*{-11mm}q < 2;\\[8pt]
   \displaystyle \fr{2p^p \theta^{2p}}{{\alpha^p x^{2p+1}\Gamma(p)}}\,{\sf Ge}_{2/q,\, 
2p/q+1} \left(-\fr{p\theta^2}{\alpha x^2}\right), &\\[2pt]
&\hspace*{-11mm}q > 2;
 \end{cases}
\\
F_\rho(x)& =
 \begin{cases}
   \displaystyle 1-\fr{1}{{\Gamma(p)}}\,{\sf Ge}_{q/2,\, p}\left(-
\fr{\alpha^{q/2}x^{q}}{p^{q/2}\theta^{q}}\right), &\hspace*{-6mm}q < 2;\\[2pt]
   \displaystyle \fr{2}{{\Gamma(p)}}
   \!\int\limits_{\theta\sqrt{p/\alpha}/x}^{\infty}\hspace*{-4mm} z^{2p-1}{\sf Ge}_{q/2,\, q/2+p} (-z^2) \,dz, &\\
&\hspace*{-6mm}q > 2;
 \end{cases}
\\[6pt]
\e\rho^z &= \fr{p^{z/2} \theta^z \Gamma\left(1+z/q\right) \Gamma\left(p-
z/2\right)}{{\alpha^{z/2}\Gamma(p)}}\,, \quad\enskip\enskip z<2p\,.
\end{align*}
При $q=2$ распределение индекса баланса~$\rho$ совпадает с~распределением 
Бурра~\cite{Burr1942} с~параметрами $(2, p, \theta \sqrt{p/\alpha})$.


\smallskip

\noindent
Д\,о\,к\,а\,з\,а\,т\,е\,л\,ь\,с\,т\,в\,о\,.\ \ Аналогично предыдущей тео\-ре\-ме для получения 
выражения для плотности~$\rho$ при всех $q\hm>0$ достаточно воспользоваться леммой~2.

Найдем функцию распределения в~случае $q\hm<2$. Имеем:
\begin{multline*}
F_\rho(x) =\fr{q \left(\alpha/p\right)^{q/2}}{{\theta^q  \Gamma(p)}} \times{}\\
{}\times\int\limits_{0}^{x} 
u^{q-1} {\sf Ge}_{q/2,\,q/2+p}\left(-\fr{ \alpha^{q/2} u^q}{p^{q/2}\theta^q}\right) 
 du =\fr{q \left(\alpha/p\right)^{q/2}}{{\theta^q  \Gamma(p)}}\times{}\\
 {}\times \int\limits_{0}^{x} u^{q-1} 
\sum\limits_{k=0}^\infty \fr{\left(-\alpha^{q/2} u^q\right)^k }{{p^{qk/2}\theta^{qk}k!}} \,
\Gamma\left(\fr{qk + q + 2p}{2}\right)  du ={}\\
{}=-\fr{1}{{\Gamma(p)}}\sum\limits_{k=0}^\infty \fr{\left(-\alpha^{q/2} x^q\right)^{k+1} 
}{{\left(p^{q/2}\theta^q\right)^{k+1}(k+1)!}} \times{}\\
{}\times\Gamma\left(\fr{q(k+1)+2p}{2}\right)={}\\
{}=1-\fr{1}{{\Gamma(p)}}\sum\limits_{m=0}^\infty \fr{\left(-\alpha^{q/2} x^q\right)^m 
}{{\left(p^{q/2}\theta^q\right)^mm!}}\,\Gamma\left(\frac{qm+2p}{2}\right).
\end{multline*}

Для $q>2$ функция распределения~$\rho$ имеет вид:
\begin{multline*}
F_\rho(x) = {}\\
{}=\int\limits_{0}^{x} \fr{2p^p \theta^{2p}}{{\alpha^p 
u^{2p+1}\Gamma(p)}}\,{\sf Ge}_{2/q,\, 2p/q+1} \left(-\fr{p\theta^2}{\alpha u^2}\right) 
du ={}
\end{multline*}

\noindent
\begin{multline*}
\hspace*{-10pt}{}=\!\!\int\limits_{0}^{x}\!\!\fr{2p^p \theta^{2p}}{{\alpha^p 
u^{2p+1}\Gamma(p)}}\!\sum\limits_{k=0}^\infty \!\fr{\left(-p\theta^2\right)^k }
{{\left(\alpha 
u^2\right)^kk!}}\,\Gamma\!\left(\!\fr{2k+ 2p +q}{q}\!\right) du ={}\\
{} = \fr{2}{{ p\Gamma(p)}} \!\int\limits_{\theta\sqrt{p/\alpha}/x}^{\infty} 
\hspace*{-5mm}z^{2p-1}\sum\limits_{k=0}^\infty \fr{\left(-z^2\right)^k }{k!}\, 
\Gamma\!\left(\!\fr{2k+ 
2p+q}{q}\!\right) dz.\hspace*{-4.07877pt}
\end{multline*}

Соотношение для моментов следует из леммы~1 и~независимости случайных величин~$\lambda$ 
и~$\mu$. Теорема доказана.

\smallskip

\noindent
\textbf{Следствие.} Утверждение теоремы~1 справедливо для частных случаев 
распределения Накагами, таких как распределение максимума процесса броуновского 
движения ($q \hm= 1/2,$ \,$ \theta \hm= \theta$), распределение Рэлея ($q \hm= 1,$ \, 
$\theta \hm= \sqrt{\theta}$), хи-рас\-пре\-де\-ле\-ние ($q \hm= q/2$, \,$\theta \hm= q$), 
распределение Макс\-вел\-ла--Больц\-ма\-на ($q \hm= 3/2$, \,$\theta \hm= 3 \theta$). 
Утверждение теоремы~2 также справедливо для перечисленных распределений 
с~соответствующей заменой параметров.


\section{Заключение}

Распределения Вейбулла и~Накагами вместе с~их частными случаями служат 
адекватными математическими моделями многих реальных процессов и~явлений из 
всевозможных областей знания (радиотехники, физики, экономики, управления 
и~пр.). По этой причине полученные результаты могут найти широкое применение при 
анализе эффективности разнообразных реальных систем.


{\small\frenchspacing
 {%\baselineskip=10.8pt
 \addcontentsline{toc}{section}{References}
 \begin{thebibliography}{9}
\bibitem{Ku18}
\Au{Кудрявцев~А.\,А.}
Байесовские модели баланса~// Информатика и~её применения, 2018. Т.~12. Вып.~3. С.~18--27.

\bibitem{Nakagami}
\Au{Nakagami M.}
The m-distribution, a~general formula of intensity of rapid fading~// 
Statistical Methods in Radio Wave Propagation: Symposium  Proceedings~/ 
Ed. W.\,C.~Homan.~--- New York, NY, USA: Pergamon Press, 1960. 
P.~3--36.

\bibitem{KuTi2017}
\Au{Кудрявцев~А.\,А., Титова~А.\,И.}
Гамма-экспоненциальная функция в~байесовских моделях массового обслуживания~// 
Информатика и~её применения, 2017. Т.~11. Вып.~4. С.~104--108.

\bibitem{Dagum1977}
\Au{Dagum~C.\/}
A~new model of personal income-distribution-specification and estimation~// 
Econ. Appl., 1977. Vol.~30. No.\,3. P.~413--437.

\bibitem{Burr1942}
\Au{Burr~I.~W.} Cumulative frequency functions~// 
Ann. Math. Stat., 1942. Vol.~13. No.\,2. P.~215--232.
 \end{thebibliography}

 }
 }

\end{multicols}

\vspace*{-3pt}

\hfill{\small\textit{Поступила в~редакцию 17.03.19}}

\vspace*{8pt}

%\pagebreak

%\newpage

%\vspace*{-29pt}

\hrule

\vspace*{2pt}

\hrule

%\vspace*{-2pt}

\def\tit{BAYESIAN MODELS OF~FACTORS BALANCE WITH~\textit{A~PRIORI} WEIBULL AND~NAKAGAMI 
DISTRIBUTIONS}


\def\titkol{Bayesian models of~factors balance with~\textit{a~priori} 
Weibull and~Nakagami 
distributions}

\def\aut{E.\,N.~Arutyunov$^1$, A.\,A.~Kudryavtsev$^2$, and~A.\,I.~Titova$^2$}

\def\autkol{E.\,N.~Arutyunov, A.\,A.~Kudryavtsev, and~A.\,I.~Titova}

\titel{\tit}{\aut}{\autkol}{\titkol}

\vspace*{-11pt}


\noindent
$^1$Institute of Informatics Problems, Federal Research Center 
``Computer Science and Control'' of the 
Russian\linebreak
$\hphantom{^1}$Academy of Sciences, 44-2~Vavilov Str., Moscow 119333, Russian Federation

\noindent
$^2$Faculty of Computational Mathematics and Cybernetics, 
M.\,V.~Lomonosov Moscow State University, 1-52~Lenin-\linebreak
$\hphantom{^1}$skiye Gory, GSP-1, Moscow 119991,  Russian Federation

\def\leftfootline{\small{\textbf{\thepage}
\hfill INFORMATIKA I EE PRIMENENIYA~--- INFORMATICS AND
APPLICATIONS\ \ \ 2019\ \ \ volume~13\ \ \ issue\ 2}
}%
 \def\rightfootline{\small{INFORMATIKA I EE PRIMENENIYA~---
INFORMATICS AND APPLICATIONS\ \ \ 2019\ \ \ volume~13\ \ \ issue\ 2
\hfill \textbf{\thepage}}}

\vspace*{6pt}



\Abste{Bayesian balance models are considered. Within this approach, it is 
assumed that the parameters affecting a system can be divided into positive, 
which support system functioning, and negative, which interfere with the 
functioning. Thus, the ratio of negative to positive factors~--- balance index~--- 
is considered as an indication of system's functioning effectiveness. The study
 is carried out assuming that the factors depend on the environment state and 
 their exact value cannot be obtained due to external reasons, e.\,g., 
 equipment faults, lack of resources, etc. It is also assumed that the 
 principles of factors' changes are known \textit{a~priori} and remain invariable. 
 Considering these assumptions, it is natural to use the Bayesian method, 
 which implies randomization of the initial parameters supposing that their 
 \textit{a~priori} distributions are known. As a~result, the balance index becomes 
 a~random variable as well. This paper continues 
 a~series of studies by the authors devoted to the application of Bayesian 
 methods in the problems of queuing and reliability. In this work,
 the obtained probability characteristics of the factor balance index 
 in the case of \textit{a~priori} Weibull and Nakagami distributions are presented. 
 The results are formulated using gamma-exponential function.}


\KWE{Bayesian approach; balance models; mixed distributions; Weibull 
distribution; Nakagami distribution; gamma-exponential function}



\DOI{10.14357/19922264190210}

%\vspace*{-14pt}

\Ack
\noindent
The work was partly supported by the Russian Foundation for Basic Research 
(project 17-07-00577).


\vspace*{6pt}

  \begin{multicols}{2}

\renewcommand{\bibname}{\protect\rmfamily References}
%\renewcommand{\bibname}{\large\protect\rm References}

{\small\frenchspacing
 {%\baselineskip=10.8pt
 \addcontentsline{toc}{section}{References}
 \begin{thebibliography}{9}
 
 \vspace*{-2pt}
 
\bibitem{1-kudr-1}
\Aue{Kudryavtsev, A.\,A.} 2018. Bayesovskie modeli balansa [Bayesian balance models]. 
\textit{Informatika i~ee Primeneniya~--- Inform. Appl.} 12(3):18--27.

\bibitem{2-kudr-1}
\Aue{Nakagami, M.} 1960. The m-distribution, a~general formula 
of intensity of rapid fading. 
\textit{Statistical Methods in Radio Wave Propagation:  Symposium Proceedings}. Ed. W.\,C.~Hoffman. New York, NY: Pergamon Press. 3--36.

\bibitem{3-kudr-1}
\Aue{Kudryavtsev, A.\,A., and A.\,I.~Titova.} 2017. Gamma-eksponentsial'naya 
funktsiya v~bayesovskikh modelyakh massovogo obsluzhivaniya 
[Gamma-exponential function in Bayesian queuing models]. 
\textit{Informatika i~ee Primeneniya~--- Inform. Appl.} 11(4):104--108.



\bibitem{4-kudr-1}
\Aue{Dagum, C.} 1977. A~new model of personal income-distribution-specification and 
estimation. \textit{Econ. Appl.} 30(3):413--437.



\bibitem{5-kudr-1}
\Aue{Burr, I.\,W.} 1942. Cumulative frequency functions. 
\textit{Ann. Math. Stat.} 13(2):215--232.
\end{thebibliography}

 }
 }

\end{multicols}

\vspace*{-6pt}

\hfill{\small\textit{Received March 17, 2019}}

%\pagebreak

%\vspace*{-18pt}

\Contr

\noindent
\textbf{Arutyunov Evgeny N.}  (b.\ 1952)~--- 
Candidate of Science (PhD) in 
physics and mathematics, senior scientist, Institute of Informatics Problems, 
Federal Research Center ``Computer Science and Control'' 
of the Russian Academy of Sciences, 44-2~Vavilov Str., Moscow 119333, 
Russian Federation; \mbox{enapoleon@mail.ru} 

\vspace*{3pt}

\noindent
\textbf{Kudryavtsev Alexey A.} (b.\ 1978)~--- 
Candidate of Sciences (PhD) in physics and mathematics, associate professor,
 Department of Mathematical Statistics, Faculty of Computational Mathematics 
 and Cybernetics, M.\,V.~Lomonosov Moscow State University, 1-52~Leninskiye Gory, 
 GSP-1, Moscow 119991, Russian Federation; \mbox{nubigena@mail.ru}
 
 \vspace*{3pt}

\noindent
\textbf{Titova Anastasiia I.} (b.\ 1995)~--- 
student, Faculty of Computational Mathematics and Cybernetics, 
M.\,V.~Lomonosov Moscow State University, 1-52~Leninskiye Gory, GSP-1, Moscow 119991, 
Russian Federation; \mbox{onkelskroot@gmail.com}

\label{end\stat}

\renewcommand{\bibname}{\protect\rm Литература}       