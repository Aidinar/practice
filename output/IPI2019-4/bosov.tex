\def\stat{bosov}

\def\tit{ИСПОЛЬЗОВАНИЕ МОДЕЛИ ГАММА-РАСПРЕДЕЛЕНИЯ В~ЗАДАЧЕ 
ФОРМИРОВАНИЯ ОГРАНИЧЕННОГО ПО ВРЕМЕНИ ТЕСТА В~СИСТЕМЕ 
ДИСТАНЦИОННОГО ОБУЧЕНИЯ$^*$}

\def\titkol{Использование модели гамма-распределения в~задаче 
формирования ограниченного по времени теста в~СДО} %системе  дистанционного обучения}

\def\aut{А.\,В.~Босов$^1$, Г.\,А.~Мхитарян$^2$, А.\,В.~Наумов$^3$, 
А.\,П.~Сапунова$^4$}

\def\autkol{А.\,В.~Босов, Г.\,А.~Мхитарян, А.\,В.~Наумов, 
А.\,П.~Сапунова}

\titel{\tit}{\aut}{\autkol}{\titkol}

\index{Босов А.\,В.}
\index{Мхитарян Г.\,А.}
\index{Наумов А.\,В.} 
\index{Сапунова А.\,П.}
\index{Bosov A.\,V.}
\index{Naumov A.\,V.}
\index{Mkhitaryan G.\,A.} 
\index{Sapunova A.\,P.}


{\renewcommand{\thefootnote}{\fnsymbol{footnote}} \footnotetext[1]
{Работа выполнена при  поддержке РФФИ (проект 18-07-00617-A).}}


\renewcommand{\thefootnote}{\arabic{footnote}}
\footnotetext[1]{Институт проблем информатики Федерального исследовательского 
центра <<Информатика и~управление>> Российской академии наук, \mbox{ABosov@frccsc.ru}}
\footnotetext[2]{Московский авиационный институт (Национальный исследовательский университет), 
\mbox{naumovav@mail.ru}}
\footnotetext[3]{Московский авиационный институт (Национальный исследовательский университет), 
\mbox{grgmkn@mail.ru}}
\footnotetext[4]{Московский авиационный институт (Национальный исследовательский университет), 
\mbox{sap2603@mail.ru}}

\vspace*{-2pt}

   
          
    
  \Abst{Рассмотрена задача формирования индивидуальных заданий (тестов) 
с~минимизацией по времени выполнения в~системе дистанционного обучения
(СДО). В~качестве 
критерия используется свертка двух взвешенных нормированных величин, связанных 
с~отклонением сложности формируемого теста от заданного уровня и~квантилью времени 
выполнения теста. В~качестве модели случайного времени ответа студента на задание 
используется модель гам\-ма-рас\-пре\-де\-ле\-ния. Предложен алгоритм для оценивания 
параметров гамма-распределения для каждого задания в~предположении, что сложности 
заданий определяются или экспертом, или при помощи соответствующих алгоритмов, 
основанных на модели Раша. Приводятся результаты численного эксперимента.}
  
      \KW{система дистанционного обучения; метод максимального правдоподобия; 
адаптивные системы; квантильная оптимизация}

\DOI{10.14357/19922264190402} 
  
%\vspace*{1pt}


\vskip 10pt plus 9pt minus 6pt

\thispagestyle{headings}

\begin{multicols}{2}

\label{st\stat}
     
     
\section{Введение}

    В настоящее время одной из знаковых тенденций в~области 
информационных технологий стало применение методов анализа данных для 
адаптации логики программного обеспечения с~учетом изменений 
пользовательского поведения, его реакции на такие изменения и~полученных от 
этого данных. В~случае с~СДО (авторский 
опыт сформирован СДО МАИ CLASS.NET~[1]) стоит упомянуть применение 
теории ответов на задания (item response theory, IRT), которая помогает 
проводить тестирование студентов, учитывая специфику обучаемых~[2--10]. 
Для использования этих теорий на практике обычно проводится исследование 
поведенческих характеристик пользователей, таких как скорость выполнения 
заданий, время, затраченное на выполнение, способности студентов и~др. Когда 
требуется спроектировать систему, способную адаптироваться под различные 
категории пользователей, решается задача прогнозирования перечисленных 
выше пользовательских характеристик~\cite{2-b, 3-b, 7-b, 8-b, 9-b}. 
    
    В работах~\cite{8-b, 9-b} была исследована задача формирования 
ограниченных по времени индивидуальных заданий (тестов) в~предположении 
модели логнормального распределения времени ответа на задания для каждого 
пользователя СДО и~дискретного распределения времени для универсального 
пользователя, чьи индивидуальные особенности игнорировались в~рамках 
упрощения модели времени выполнения задания. Ввиду особенностей законов 
распределения случайного времени алгоритмы решения этих задач 
оказывались достаточно затратными по времени и~обладали высокой 
вы\-чис\-ли\-тель\-ной сложностью.
    
    В данной статье продолжено изучение модели времени, затрачиваемого 
студентами на задания теста, а~также решается задача формирования тес\-тов 
с~учетом случайного времени, определяемого гам\-ма-рас\-пре\-де\-ле\-нием. 

\vspace*{-6pt}
    
\section{Модель времени ответа пользователя на задание системы }

    Одной из основных математических моделей времени ответа пользователя 
на задачу, в~том числе в~процессе тестирования СДО, является логнормальная 
модель Ван дер Линдена~\cite{2-b}. Этой модели не свойствены излишние 
сложности в~оценке параметров распределения, которые могут быть найде\-ны 
методом максимального правдоподобия. 
    
    Действительно, пусть имеется набор $Z\hm= (z_1, \ldots , z_I)$ из 
$I$~заданий, каждое из которых имеет определенный уровень сложности 
и,~следовательно, отличается от других задач временем, требуемым пользователю 
для ответа. Модель Ван дер Линдена предполагает, что логарифм 
времени~$T_j^i$ ответа $j$-го пользователя на задачу~$i$ имеет три 
составляющих, одна из которых связана с~индивидуальной сложностью 
рассматриваемого задания~($\beta_i$, $i\hm=1, \ldots , I$), другая отвечает за 
физиологические особенности пользователя, решающего это задание ($\tau_j$, 
$j\hm= 1,\ldots , J$), а~третья является общей составляющей для всех 
пользователей и~заданий ($\mu$). Таким образом, логарифм времени ответа 
пользователя на задание имеет вид:
    $$
    \ln T_j^i =\mu+\beta_i+\tau_i+\varepsilon_{ij}\,,\ i=1,\ldots , I\,,\ j=1,\ldots , 
J\,,
    $$
где $\varepsilon_{ij}$, $i\hm= 1,\ldots , I$, $j\hm= 1,\ldots , J$,~--- независимые 
случайные величины,  
$\varepsilon_{ij}\sim N(0,\sigma^2)$ имеет гауссовское распределение.
    
    Оценивание параметров модели $\mu$, $\beta_i$ и~$\tau_i$, $i\hm=1, \ldots , I$, 
    $j\hm= 1,\ldots , J$, не представляет существенных трудностей, 
эффективность обеспечивается классическими методами максимального 
правдоподобия или наименьших квадратов/модулей. Выражения получаемых 
оценок приведены, например, в~\cite{2-b}. 
    
    Достоинство данной модели заключается в~возможности получения 
распределения времени ответа каждого пользователя на каждое задание 
системы по имеющейся разрозненной статистике, так как далеко не каждый 
пользователь решал каждое задание. Известным недостатком данной модели 
является отсутствие возможности получения точного значения квантили 
общего времени выполнения теста пользователем, которое представимо как 
сумма случайных величин, соответствующих времени ответа пользователя на 
задания теста. При этом квантиль служит наиболее важной вероятностной 
характеристикой в~рассматриваемой задаче, так как дает понимание того, 
сколько гарантированно с~заданным уровнем доверительной вероятности 
потребуется времени пользователю для выполнения теста.
    
Параметры данной модели для каждого задания\linebreak 
системы оце\-ни\-ва\-ют\-ся на основе выборки, со\-сто\-ящей 
из значений времени, затраченного на 
решение данного задания конкретными пользователями. Таким образом, 
получается модель времени\linebreak ответа универсального (усредненного) 
пользователя на каждое задание системы. Плотности вероятности 
логнормального и~гам\-ма-рас\-пре\-де\-ле\-ний имеют схожие структуры, 
однако известно, что сумма случайных величин, имеющих 
гам\-ма-рас\-пре\-де\-ле\-ние, является гам\-ма-рас\-пре\-де\-лен\-ной 
случайной величиной, если эти 
случайные величины имеют одинаковый параметр~$\theta$. Это обеспечивает 
возможность находить точные значения квантили общего времени, 
затрачиваемого пользователем на решение теста, так как оно будет иметь 
известное гам\-ма-рас\-пре\-де\-ле\-ние. Таким образом, ключевое значение для 
предлагаемой модели обеспечит оригинальный алгоритм подбора параметров 
гам\-ма-рас\-пре\-де\-ле\-ния времени ответа универсального пользователя на задание 
системы, предлагаемый далее. Этот алгоритм формулируется так, чтобы 
параметр~$\theta$ распределения был одинаковым для всех заданий, а~значение 
второго параметра~$k$ определялось с~помощью метода максимального 
правдоподобия.

\vspace*{-6pt}

\section{Алгоритм определения параметров  
гамма-распределения}

\vspace*{-2pt}

    Целью работы алгоритма ставится подбор параметров  
гам\-ма-рас\-пре\-де\-ле\-ний для каждого задания так, чтобы параметр~$\theta$ 
был бы общим для всех заданий и~при этом для максимального чис\-ла заданий 
при найденных оценках параметров распределений принималась бы гипотеза 
о~гам\-ма-рас\-пре\-де\-ле\-нии времени ответа пользователя на это задание. Пусть~$t_i$, 
$i\hm= 1,\ldots , I$,~--- время ответа универсального пользователя на 
задание~$i$, где~$I$~--- число заданий, из которых формируется тест; $t_i^j$, 
$j\hm=1, \ldots , I_i$,~--- реализация времени ответа пользователя~$j$, 
затраченное им на решение задачи~$i$, где~$I_i$~--- чис\-ло пользователей, 
решавших задачу~$i$.
    
    Опишем по шагам алгоритм подбора па\-ра\-мет\-ров 
гам\-ма-рас\-пре\-де\-ле\-ний случайных величин~$t_i$, $i\hm= 1,\ldots , I$.
    
   \textbf{Шаг~0.} Обнулим значения искомых па\-ра\-мет\-ров и~некоторых счетчиков 
алгоритма. Положим $\theta^*\hm=0$, где~$\theta^*$~--- искомое значение 
параметра гам\-ма-рас\-пре\-де\-ле\-ния, который одинаков для всех задач. 
Положим $k_i^*\hm=0$, где $k_i^*$~--- искомое значение второго параметра 
распределения для $i$-го задания. Положим $S\hm=0$, где~$S$~--- число задач, 
для которых принимается гипотеза о~гам\-ма-рас\-пре\-де\-ле\-нии времени 
ответа пользователя. Положим $m\hm=0$, где~$m$~--- счетчик. Выберем 
уровень доверительной вероятности $1\hm-\alpha$ для проверки 
статистических гипотез.
    
    \textbf{Шаг 1.} Для всех $i\hm=1,\ldots , I$ по выборке объема~$I_i$ методом 
максимального правдоподобия находим оценки~$\hat{\theta}_i$ 
параметра~$\theta$. Среди полученных значений находим минимальное 
$\hat{\theta}_{\min}$ и~максимальное~$\hat{\theta}_{\max}$ значения. Для 
варьирования параметра~$\theta$ выберем шаг 

\noindent
    $$
    h=\fr{\hat{\theta}_{\max}-\hat{\theta}_{\min}}{L}\,,
    $$
где $L$~--- выбранное заранее число шагов дискретизации по~$\theta$. 
Положим $\theta_m\hm=0$.
    
    \textbf{Шаг 2.} Положим $m:=m\hm+1$, а $\theta_m\hm= \theta_{m-1}\hm+h$. Для 
каждого $i\hm= 1,\ldots , I$ по выборке~$t_i^j$, $i\hm=1,\ldots , I$,
 $j\hm= 1,\ldots , I_i$, определяем оценку второго параметра 
 гам\-ма-рас\-пре\-де\-ления
    $$
    \hat{k_i} =\fr{\overline{t_i^j}}{\theta_m}\,,
    $$
где $\overline{t_i^j}$~--- выборочное математическое ожидание.
    
    \textbf{Шаг~3.} Для всех $i\hm=1,\ldots , I$ на выбранном уровне доверительной 
вероятности $1\hm-\alpha$ проверяем с~помощью критерия Пирсона гипотезу  
$H_0$: $t_i\sim \Gamma(\hat{k}_t, \theta_m)$.
Если число принятых гипотез~$S^\prime$ больше~$S$, то полагаем $S\hm= 
S^\prime$, $\theta^*\hm= \theta_m$, $k_i^*\hm= \hat{k_i}$, $i\hm= 1,\ldots , I$.
    
    \textbf{Шаг~4.} Если $m<L\hm-1$, то перейти к~шагу~2.
    
    \textbf{Шаг~5.} Окончание работы алгоритма.
    
    Полученная модель распределения позволяет предложить эффективный 
алгоритм решения актуальной задачи формирования теста для универсального 
пользователя так, чтобы его сложность минимально отличалась от заданного 
экспертом уровня сложности и~при этом минимизировалось время выполнения 
теста, которое гарантированно не будет превышено с~заданным уровнем 
доверительной вероятности. Предполагается при этом, что сложности каждого 
задания оцениваются на основе обработки статистических данных о работе 
пользователей с~помощью модели Раша~\cite{3-b}. 

\section{Постановка задачи формирования ограниченного 
по~времени теста}

    Математическая постановка сформулированной выше задачи 
формирования теста пред\-лагается далее в~форме одноэтапной задачи 
кван\-тильной оптимизации. Метод ее решения \mbox{существенно}\linebreak использует 
возможность точного вычисления значения функции квантили, 
обеспечиваемую выбранной моделью распределения случайных па\-ра\-мет\-ров.
    
    Задача определения подходящего набора приблизительно равных по 
суммарной сложности заданий была рассмотрена в~\cite{8-b, 9-b, 10-b}. 
В~работе~\cite{9-b} эта задача сформулирована в~форме одноэтапной задачи 
квантильной оптимизации и~при дискретном распределении времени ответа 
пользователя на задание методами~\cite{11-b} сведена к~детерминированной 
задаче смешанного линейного программирования. Постановка этой задачи 
в~рассматриваемом случае одного универсального пользователя имеет 
следующий вид.
    
    Пусть существует множество $Z\hm= (z_1,\ldots , z_I)$ из~$I$~заданий, 
разделенных на~$M$ различных типов, $I_m$~--- число заданий $m$-го типа, 
т.\,е.\ $\sum\nolimits^M_{m=1} I_m\hm= I$, 
$m\hm= 1,\ldots , M$. Каждое задание принадлежит только одному типу, и~для 
обозначения принадлежности задания к~определенному типу введем 
матрицу~${A}$  размерности $I\times M$:
    $$
    {A}=\parallel a_i^m\parallel\,,\enskip a_i^m=\begin{cases} 1, & z_i\in 
Z_m\,;\\
    0, & z_i\notin Z_m\,.
    \end{cases}
    $$

    \noindent
    Данная матрица определяет принадлежность задания~$z_i$ к~множеству 
заданий типа~$Z_m$, $m\hm=1,\ldots , M$, если $a_i^m\hm=1$.
    
    Каждое из заданий имеет определенную сложность, которую, например, 
можно определить с~помощью метода максимального правдоподобия, 
примененного к~модели Раша в~\cite{3-b}. Введем вектор $u\hm\in 
\mathrm{R}^I$ (здесь и~далее под вектором имеется в~виду век\-тор-стол\-бец), 
координаты которого~$u_i$, $i\hm=1,\ldots , I$, обозначают принадлежность 
задания~$i$ к~формируемому набору таким образом, что
    $$
    u_i=\begin{cases} 1, &\hspace*{-5.00455pt} \mbox{если\ задача\ $i$\ попала\ в\ тестовый\ 
набор};\\
    0, &\hspace*{-5.00455pt} \mbox{если\ задача\ $i$\ не\ попала\ в\ тестовый\ набор}.
    \end{cases}
    $$
    
    Тестовым набором будут считаться~$l$~заданий, для которых $u_i\hm=1$. 
Предположим, что для каждого задания известна его сложность. Введем вектор 
${w}\in \mathrm{R}^I$, $i$-я координата которого является слож\-ностью 
$i$-го задания и~будет обозначена как~$w_i$.
    
    Требуется составить множество индивидуальных тестовых наборов 
из~$l$~заданий, принадлежащих различным типам, учитывая, что $l\hm\geq 
M$. При этом изначально задается суммарная сложность тес\-та, обозначаемая 
через~$c$, которая определяется на основе экспертной оценки. Предусмотрим, 
что возможно отклонение от данной требуемой суммарной сложности на 
 ка\-кое-ли\-бо малое число в~большую либо меньшую сторону. Обозначим 
такое число через~$\varepsilon$.

    \begin{table*}[b]\small %tabl1
    \begin{center}
    \Caption{Сложность $w_i^m$ задания~$z_i^m$}
     \vspace*{2ex}
     
     \begin{tabular}{|c|c|c|c|c|c|c|c|c|c|c|}
     \hline
     &\multicolumn{10}{c|}{$i$}\\
     \cline{2-11}
\multicolumn{1}{|c|}{\raisebox{6pt}[0pt][0pt]{$m$}}&1&2&3&4&5&6&7&8&9&10\\
\hline
1&1,311&3,254&3,254&3,254&4,874&5,368&7,011&7,217&8,244&9,636\\
2&4,132&6,902&2,121&3,436&2,456&5,359&6,902&7,283&7,815&9,399\\
3&2\hphantom{,999}&2,418&2,666&3,653&5,242&5,547&6,453&7,194&8,795&3,657\\
\hline
\end{tabular}
\end{center}
\end{table*}
     
    
    Пусть по-прежнему~$t_i$, $i\hm=1,\ldots , I$,~--- время ответа 
универсального пользователя на задание~$i$. Пусть в~отличие от модели, 
полученной в~\cite{8-b}, общее время на выполнение теста неизвестно, 
аналогично~\cite{9-b}. Обозначим его через~$\varphi$. Тогда для того, чтобы за 
некоторое оптимальное время все тестируемые могли выполнить выданный 
вариант теста с~заданной вероятностью~$\alpha$, рассмотрим функцию 
квантили
    \begin{equation}
    \hspace*{-2mm}\Phi_\alpha(\mathrm{u}) \eqdelta \min \left\{ \varphi\in {R}^1:\
   {\sf P}\left\{ \sum\limits^I_{i=1} t_i u_i\leq 
   \varphi\right\}\geq \alpha\right\}\!.\!\!
    \label{e1-b}
    \end{equation}
    
    Основываясь на описанной модели и~введенных обозначениях, 
сформулируем задачу квантильной оптимизации:
    \begin{gather} 
    u_\alpha = \argmin\limits_{u\in \{0;1\}^I} \left( \fr{\gamma\left\vert c-w^T 
u\right\vert}{\varepsilon} +\fr{(1-\gamma)\Phi_\alpha (u)}{2700}\right)\,;\label{e2-b}\\
     \varphi_\alpha = \min\limits_{u\in\{ 0;1\}^I} \left( \fr{\gamma\left\vert c- w^T 
u\right\vert}{\varepsilon} +\fr{(1-\gamma)\Phi_\alpha(u)}{2700}\right)\,;\notag %\label{e3-b}
\end{gather}
\begin{align}
    c-w^{\mathrm{T}} u&\leq \varepsilon\,;\label{e4-b}\\
    w^{\mathrm{T}}u-c&\leq \varepsilon\,;\label{e5-b}\\
    A^{\mathrm{T}}u&\geq e_M\,;\label{e6-b}\\
    e_I^{\mathrm{T}} u&=l\,,\label{e7-b}
    \end{align}
где $(\cdot )^T$~--- операция транспонирования, $e_I\hm\in {R}^I$, 
$e_I\hm= (1,\ldots , 1)^{\mathrm{T}}$;
 $e_M\hm\in {R}^M$, $e_M\hm= (1,\ldots 
,1)^{\mathrm{T}}$; $\alpha\hm \in (0,1)$~--- заданный уровень доверительной вероятности; 
$\gamma\hm\in (0,1)$~--- весовой коэффициент.

    Критериальная функция задачи в~(\ref{e2-b}) пред\-став\-ля\-ет собой сумму 
двух нормированных безразмерных величин. Первое слагаемое является 
отклонением сложности теста от заданного уровня~$c$, нормированного 
маскимально допустимым уровнем отклонения~$\varepsilon$. Второе слагаемое 
представляет собой время выполнения тес\-та, которое не может\linebreak
 быть 
превышено с~заданным уровнем доверительной вероятности~$\alpha$. Это 
время нормируется максимально допустимым временем выполнения тес\-та. 
Такой критерий представляется универсальным\linebreak гибким инструментом 
формирования теста. С~по\-мощью весового коэффициента~$\gamma$ можно 
регулировать важность каждого слагаемого критерия. Ограничения~(\ref{e4-b}) 
и~(\ref{e5-b}) регламентируют выбор набора заданий в~тесте, суммарная 
сложность которых должна отличаться от заданного экспертом уровня 
сложности не более чем на величину~$\varepsilon$. Ограничение~(\ref{e6-b}) 
отвечает за то, чтобы среди всех заданий в~тес\-те было хотя бы одно задание 
каждого типа, так как данная задача решается при условии, что $l\hm \geq M$. 
Ограничение~(\ref{e7-b}) означает, что в~наборе должно быть 
ровно~$l$~заданий.
    
    Преимущество использования в~рассмотренной задачи модели 
    гамма-распределения с~найденными с~по\-мощью предложенного выше алгоритма 
оценивания параметрами за\-клю\-ча\-ет\-ся в~воз\-мож\-ности точного вычисления 
значения функции квантили, так как сумма гам\-ма-рас\-пре\-де\-лен\-ных 
случайных величин в~(\ref{e1-b}) является случайной величиной с~известным 
гам\-ма-рас\-пре\-де\-ле\-ни\-ем. Это позволяет, основываясь на методах, описанных 
в~\cite{11-b, 12-b, 13-b}, предложить эффективную процедуру направленного 
перебора воз\-мож\-ных значений целочисленных переменных оптимизации 
рассматриваемой за\-дачи.

\setcounter{table}{2}
     \begin{table*}[b]\small %tabl3
     \begin{center}
     \Caption{Число наборов заданий при фиксированном~$\varepsilon$}
     \vspace*{2ex}
     
     \tabcolsep=5pt
     \begin{tabular}{|c|c|c|c|c|c|}
     \hline
 \multicolumn{1}{|c|}{\raisebox{-12pt}[0pt][0pt]{$\varepsilon$}} 
 &&\multicolumn{2}{c|}{$\gamma=0{,}5$}  &    \multicolumn{2}{c|}{$\gamma=0$}\\
 \cline{3-6}
 &\multicolumn{1}{c|}{\raisebox{6pt}[0pt][0pt]
 {\tabcolsep=0pt\begin{tabular}{c}Число решений, удовлетворяющих\\
  детерминированным\\ ограничениям\end{tabular}}}&
  \tabcolsep=0pt\begin{tabular}{c}Оптимальное\\ решение~$\psi^*$\end{tabular}&
\tabcolsep=0pt\begin{tabular}{c}Оптимальный\\ набор заданий\end{tabular}&
\tabcolsep=0pt\begin{tabular}{c}Оптимальное\\ 
решение~$\psi^*$\end{tabular}&
\tabcolsep=0pt\begin{tabular}{c}Оптимальный\\ набор заданий\end{tabular}\\
\hline
&&&&&\\[-9pt]
      0,0007&3&0,4838&$z_6^1$, $z^1_{10}$, $z_5^2$, $z_6^3$, $z_7^3$&0,5322&$z_6^1$, 
$z^1_{10}$, $z_5^2$, $z_6^3$, $z_7^3$\\
      0,0009&4&0,4362&$z_6^1$, $z^1_{10}$, $z_5^2$, $z_6^3$, $z_7^3$&0,4816&$z_6^1$, 
$z^1_{2}$, $z_7^2$, $z_9^2$, $z_5^3$\\
      0,003\hphantom{9}&21\hphantom{9}&0,3195&$z_6^1$, $z^1_{10}$, $z_5^2$, $z_6^3$, 
$z_7^3$&0,4428&$z_6^1$, $z^1_8$, $z_1^2$, $z_6^3$, $z_8^3$\\
      \hline
      \end{tabular}
      \end{center}
      \end{table*}
    
\section{Результаты численного эксперимента}



    Для проведения анализа результатов решения сформулированной задачи 
воспользуемся исходными данными, полученными при обработке 
статистической информации о~работе пользователей СДО МАИ CLASS.NET 
в~\cite{1-b}.
    
    Рассмотрим задания $M\hm=3$ различных типов, относящихся 
к~основным, изучаемым в~течение первого семестра, тематическим разделам 
курса <<Математического анализа>> СДО МАИ CLASS.NET, в~каждом из 
которых по~10~различных заданий $I^m\hm= 10$, $m\hm=1, 2, 3$. Обозначим задания 
в~зависимости от типа $m\hm=1, \ldots , M$ и~номера $i\hm= 1, \ldots , I^m$ 
как~$z_i^m$. Методами, предложенными в~\cite{7-b}, была проведена оценка 
сложности каждого задания из общего числа при помощи алгоритма, 
основанного на модели Раша~\cite{3-b}. В~результате были получены 
приведенные к~десятибалльной шкале оценки значений сложностей~$w_i^m$ 
для каждого~$z_i^m$, которые представлены в~табл.~1. 
    

    Исходя из оценок значений сложности для каж\-до\-го задания, выберем 
требуемую суммарную сложность $c\hm= 29{,}46$ теста из $l\hm=5$ заданий. 
Возможный критерий выбора величины~$c$ был описан\linebreak\vspace*{-10pt}

 % \begin{table*}
   { %tabl2
    \begin{center}
    \parbox{70mm}{{{\tablename~2}\ \ \small{Значения параметров гамма-распределения времени ответа универсального 
пользователя на задание~$z_i^m$}}

}

     \vspace*{6pt}
     
    \small
     \begin{tabular}{|l|c|l|c|l|c|}
     \hline
     &&&&&\\[-9pt]
     \multicolumn{1}{|c|}{$z_i^m$}&$\hat{k}_i^m$&
          \multicolumn{1}{|c}{$z_i^m$}&$\hat{k}_i^m$&
               \multicolumn{1}{c|}{$z_i^m$}&$\hat{k}_i^m$\\
\hline
&&&&&\\[-9pt]
$z_1^1$&17,94&$z^2_1$&60,25&$z_1^3$&19,95\\
$z_2^1$&27,37&$z^2_2$&191,99\hphantom{9}&$z_2^3$&22,54 \\
$z_3^1$&26,44&$z^2_3$&25,80 &$z_3^3$&26,56 \\
$z_4^1$&26,58&$z_4^2$&26,52&$z_4^3$&26,79\\
$z_5^1$&34,51&$z_5^2$&25,02&$z_5^3$&67,62 \\
$z_6^1$&54,25&$z_6^2$&89,31&$z_6^3$&74,86 \\
$z_7^1$&150,77\hphantom{9}&$z_7^2$&101,18\hphantom{9}&$z^3_7$&137,56\hphantom{9}\\
$z_8^1$&159,78\hphantom{9}&$z^2_8$&154,15\hphantom{9}&$z_8^3$&189,29\hphantom{9}\\
$z_9^1$&237,92\hphantom{9}&$z_9^2$&208,68\hphantom{9}&$z_9^3$&250,41\hphantom{9}\\
$z^1_{10}$&304,03\hphantom{9}&$z^2_{10}$&281,26\hphantom{9}&$z^3_{10}$&25,96\\
\hline
\end{tabular}
\end{center}}
%\end{table*}

\vspace*{12pt}

\noindent
 в~\cite{8-b, 9-b}. 
Параметр~$\varepsilon$ задается числом, близким к~нулю, для выбора наиболее 
оптимальных наборов тестов. Будем варьировать его от~0,0004 до~0,004 
с~шагом~0,0001. Описанным выше алгоритмом для каждого задания были 
получены значения па\-ра\-мет\-ров гам\-ма-рас\-пре\-де\-ле\-ния. Параметр~$\theta$ 
для всех задач одинаков и~равен~2,29. Значения оценки второго 
параметра~$\hat{k}_i^m$ для каждого задания~$z_i^m$ приведены в~табл.~2. 
    
  
    
    На уровне доверительной вероятности~0,95 критерий Пирсона показал, 
что все гипотезы о том, что время ответа универсального пользователя на 
соответствующее задание системы имеет гам\-ма-рас\-пре\-де\-ле\-ние 
с~указанными параметрами, принимаются.
    
    Далее решалась задача формирования теста, уровень доверительной 
вероятности также был выбран равным~0,95. Требовалось составить наборы 
тестовых заданий из~5~задач, которые с~вероятностью~0,95 могут быть 
решены пользователем за некоторое оптимальное время~$\varphi$.
    
    Для проверки адекватности модели времени ответа пользователя в~случае 
гамма-распределения сформулированная выше задача была решена при тех же 
значениях~$\varepsilon$ и~$\gamma$, что и~в~\cite{8-b}, где в~качестве модели 
времени ответа пользователя использовалась модель Ван дер Линдена. Сравним 
полученные результаты.
    
    Для каждого~$\varepsilon$ было получено число наборов заданий, 
удовлетворяющих детерминированным ограничениям, а также при 
$\gamma\hm=0$ и~0{,}5 для этих наборов (табл.~3) получены 
оптимальные значения критерия~$\psi^*$.
    
    Как видно из результата, для случая, когда минимизация отклонения 
сложности набора тестовых заданий и~минимизация оптимального времени 
ответа имеют равный вес, приоритет все же отдается минимизации отклонения, 
так как с~увеличением~$\varepsilon$ оптимальный набор заданий не 
изменяется.
    

    
    Наибольшему числу наборов заданий соответствует $\varepsilon\hm = 
0{,}004$. При $\gamma\hm\geq 0{,}5$ оптимальный набор заданий не 
изменяется с~увеличением~$\varepsilon$ и~пред\-став\-ля\-ет собой набор 
с~минимальным отклонением от заданной сложности~$c$. 
    
    Полученные оптимальные значения критерия близки к~значениям, 
полученным в~\cite{8-b}. Составы оптимальных тестовых наборов при 
некоторых~$\varepsilon$ полностью совпадают, при остальных~$\varepsilon$ 
отличаются на~1--2~задания.
    
\section{Заключение}

    В статье предлагается исследование математической модели времени 
ответа пользователя и~решение задачи формирования ограниченных по времени 
тес\-тов с~заданной суммарной слож\-ностью\linebreak задания в~виде решения одноэтапной 
задачи квантильной оптимизации.
    
    За основу модели времени ответа пользователя была взята модель  
гам\-ма-рас\-пре\-де\-ле\-ния, разработан алгоритм для подбора па\-ра\-мет\-ров 
рас\-пре\-де\-ле\-ния для каждого задания.
    
    В результате решения задачи с~приведенными в~статье значениями 
па\-ра\-мет\-ров распределения, слож\-ности каждой задачи и~тестов в~целом для 
$\varepsilon\hm= 0{,}004$ (отклонения от заданного экспертом уровня 
слож\-ности) было получено~35~наборов тес\-то\-вых заданий. Кроме того, 
полученные результаты чис\-лен\-но\-го эксперимента имеют значения, близ\-кие 
к~полученным в~\cite{8-b} для логнормальной модели времени ответа 
пользователя на задание, что подтверждает адекватность предложенной 
модели. Однако предложенная модель гам\-ма-рас\-пре\-де\-ле\-ния позволяет 
получать точ\-ные значения функции квантили времени ответа универсального 
пользователя на тест, что дает воз\-мож\-ность использовать эффективные методы 
решения сформулированной задачи.
    
{\small\frenchspacing
 {%\baselineskip=10.8pt
 \addcontentsline{toc}{section}{References}
 \begin{thebibliography}{99}
\bibitem{1-b}
\Au{Наумов А.\,В., Джумурат~А.\,С., Иноземцев~А.\,О.} Система дистанционного обучения 
математическим дисциплинам CLASS.NET~// Вестник компьютерных и~информационных технологий, 2014. 
№\,10. С.~36--44.

\bibitem{3-b} %2
\Au{Rasch G.} Probabilistic models for some intelligence and attainment tests.~--- 
Chicago, IL, USA: 
University of Chicago Press, 1980. 224~p.

\bibitem{2-b} %3
\Au{Van der Linden W.\,J., Scrams~D.\,J., Schnipke~D.\,L., \textit{et al.}} Using response-time 
constraints to control for differential speededness in computerized adaptive 
testing~// Appl. Psych. Meas., 1999. Vol.~23. Iss.~3. P.~195--210
\bibitem{4-b} %4
\Au{Куравский Л.\,С., Мармалюк~П.\,А., Алхимов~В.\,И., Юрьев~Г.\,А.} Новый подход 
к~построению интеллектуальных и~компетентностных тестов~// Моделирование и~анализ 
данных, 2013. №\,1. С.~4--28.

\bibitem{10-b} %5
\Au{Наумов А.\,В., Иноземцев~А.\,О.} Алгоритм формирования индивидуальных заданий 
в~системах дистанционного обучения~// Вестник компьютерных и~информационных
 технологий, 2013. №\,6. С.~35--42.
\bibitem{7-b} %6
\Au{Кибзун А.\,И., Иноземцев~А.\,О.} Оценивание уровней сложности тестов на основе 
метода максимального правдоподобия~// Автоматика и~телемеханика, 2014. №\,4. С.~20--37.
\bibitem{5-b} %7
\Au{Куравский Л.\,С., Мармалюк~П.\,А., Юрьев~Г.\,А., Думин~П.\,Н., Панфилова~А.\,С.} 
Вероятностное моделирование процесса выполнения тестовых заданий на основе 
модифицированной функции Раша~// Вопросы психологии, 2015. №\,4. С.~109--118.
\bibitem{6-b} %8
\Au{Kuravsky L.\,S., Margolis~A.\,A., Marmalyuk~P.\,A., Panfilova~A.\,S., Yuryev~G.\,A., 
Dumin~P.\,N.} A~probabilistic model of adaptive training~// Appl. Math. Sci., 2016. 
Vol.~10. Iss.~48. P.~2369--2380.


\bibitem{8-b} %9
\Au{Наумов А.\,В., Мхитарян~Г.\,А.} О задаче вероятностной оптимизации для 
ограниченного по времени тестирования~// Автоматика и~телемеханика, 2016. №\,9.  
С.~124--135.
\bibitem{9-b} %10
\Au{Наумов А.\,В., Мхитарян~Г.\,А., Черыгова~Е.\,Е.} Стохастическая постановка задачи 
формирования теста заданного уровня сложности с~минимизацией квантили времени 
выполнения~// Вестник компьютерных и~информационных технологий, 2019. №\,2. С.~37--46.

\bibitem{11-b}
\Au{Кибзун А.\,И., Наумов~А.\,В., Норкин~В.\,И.} О~сведении задачи квантильной 
оптимизации с~дискретным распределением к~задаче смешанного целочисленного 
программирования~// Автоматика и~телемеханика, 2013. №\,6. С.~66--86.
\bibitem{12-b}
\Au{Кан Ю.\,С., Кибзун~А.\,И.} Задачи стохастического программирования с~вероятностными 
критериями.~--- М.: Физматлит, 2009. 372~с.
\bibitem{13-b}
\Au{Наумов А.\,В., Иванов~С.\,В.} Исследование задачи стохастического линейного 
программирования с~квантильным критерием~// Автоматика и~телемеханика, 2011. №\,2. 
С.~142--158. 
 \end{thebibliography}

 }
 }

\end{multicols}

\vspace*{-6pt}

\hfill{\small\textit{Поступила в~редакцию 30.05.19}}

\vspace*{8pt}

%\pagebreak

%\newpage

%\vspace*{-28pt}

\hrule

\vspace*{2pt}

\hrule

%\vspace*{-2pt}

\def\tit{USING THE MODEL OF~GAMMA DISTRIBUTION IN~THE~PROBLEM 
OF~FORMING A~TIME-LIMITED TEST IN~A~DISTANCE LEARNING 
SYSTEM}


\def\titkol{Using the model of~gamma distribution in~the~problem 
of~forming a~time-limited test in~a~distance learning 
system}

\def\aut{A.\,V.~Bosov$^1$, A.\,V.~Naumov$^2$, G.\,A.~Mkhitaryan$^2$, 
and~A.\,P.~Sapunova$^2$}

\def\autkol{A.\,V.~Bosov, A.\,V.~Naumov, G.\,A.~Mkhitaryan, 
and~A.\,P.~Sapunova}

\titel{\tit}{\aut}{\autkol}{\titkol}

\vspace*{-11pt}


\noindent
        $^1$Institute of Informatics Problems, Federal Research Center ``Computer 
Science and Control'' of the Russian\linebreak
$\hphantom{^1}$Academy of Sciences, 44-2~Vavilov Str., 
Moscow 119333, Russian Federation
        
        \noindent
        $^2$Moscow State Aviation Institute (National Research University), 
4~Volokolamskoe Shosse, Moscow 125933,\linebreak
$\hphantom{^1}$Russian Federation

\def\leftfootline{\small{\textbf{\thepage}
\hfill INFORMATIKA I EE PRIMENENIYA~--- INFORMATICS AND
APPLICATIONS\ \ \ 2019\ \ \ volume~13\ \ \ issue\ 4}
}%
 \def\rightfootline{\small{INFORMATIKA I EE PRIMENENIYA~---
INFORMATICS AND APPLICATIONS\ \ \ 2019\ \ \ volume~13\ \ \ issue\ 4
\hfill \textbf{\thepage}}}

\vspace*{3pt}  

 
        
    
    \Abste{For the distance learning systems, consideration is given to generation of 
individual tasks with minimization of execution time. As a~criterion, the convolution 
of two weighted normalized values associated with the deviation of the complexity of 
the generated test from the specified level and the quantile of the test execution time 
is used. The gamma distribution model is used to describe a model of a student's 
random response time to a~task. An algorithm is proposed for estimating the 
parameters of the gamma distribution for each task. It is assumed that task 
complexities are determined either by an expert or by using corresponding algorithms 
based on the Rush model. The results of a numerical experiment are presented.}
    
    \KWE{distance learning system; statistical analysis; adaptive systems; quantile 
optimization}
    
  
    
  \DOI{10.14357/19922264190402} 

%\vspace*{-14pt}

  \Ack
    \noindent
    The work was supported by the Russian Foundation for Basic Research (project 
18-07-00617-A).



%\vspace*{-6pt}

  \begin{multicols}{2}

\renewcommand{\bibname}{\protect\rmfamily References}
%\renewcommand{\bibname}{\large\protect\rm References}

{\small\frenchspacing
 {%\baselineskip=10.8pt
 \addcontentsline{toc}{section}{References}
 \begin{thebibliography}{99}
    \bibitem{1-b-1}
    \Aue{Naumov, A.\,V., A.\,S.~Dzhumurat, and A.\,O.~Inozemtsev.} 2014. 
Sistema distantsionnogo obucheniya ma\-te\-ma\-ti\-che\-skim distsiplinam CLASS.NET 
[Distance learning system for mathematical disciplines CLASS.NET]. 
\textit{Vestnik komp'yuternykh i~informatsionnykh tekhnologiy} [Herald of 
Computer and Information Technologies] 10:36--44.
    
    \bibitem{3-b-1} %2
    \Aue{Rasch, G.} 1980. \textit{Probabilistic models for some intelligence and attainment 
tests}. Chicago, IL: University of Chicago Press. 224~p.

\bibitem{2-b-1} %3
    \Aue{Van der Linden, W.\,J., D.\,J.~Scrams, and D.\,L.~Schnipke.} 1999. Using 
response-time constraints to control for differential speededness in computerized 
adaptive testing. \textit{Appl. Psych. Meas.} 23(3): 195--210.
    \bibitem{4-b-1} %4
    \Aue{Kuravsky, L.\,S., P.\,A.~Marmalyuk, V.\,I.~Alkhimov, and G.\,A.~Yuryev.} 2013. 
Novyy podkhod k~postroeniyu intellektual'nykh i~kompetentnostnykh testov [A~new 
approach to the construction of intellectual and competence tests]. 
\textit{Modelirovanie i~analiz dannykh} [Modelling Data Analysis] 1:4--28.

 \bibitem{10-b-1} %5
    \Aue{Naumov, A.\,V., and A.\,O.~Inozemtsev.} 2013. Algoritm formirovaniya 
individual'nykh zadaniy v~sistemakh distantsionnogo obucheniya [Algorithm to 
generate individual tasks in the remote learning systems]. \textit{Vestnik komp'yuternykh 
i~informatsionnykh tekhnologiy} [Herald of Computer and Information 
Technologies] 74(6):35--42.

 \bibitem{7-b-1} %6
    \Aue{Kibzun, A.\,I., and A.\,O.~Inozemtsev.} 2014. Using the maximum likelihood 
method to estimate test complexity levels. \textit{Automat. Rem. Contr.} 75(4):607--621.

    \bibitem{5-b-1} %7
    \Aue{Kuravsky, L.\,S., P.\,A.~Marmalyuk, G.\,A.~Yuryev, P.\,N.~Dumin, and A.\,S.~Panfilova.}
     2015. Veroyatnostnoe mo\-de\-li\-ro\-va\-nie protsessa vypolneniya testovykh 
zadaniy na osno\-ve modifitsirovannoy funktsii Rasha [Probabilistic modeling of the 
test tasks based on the modified Rush function]. \textit{Voprosy psikhologii} 
[Psychology Issues] 4:109--118.
    \bibitem{6-b-1} %8
    \Aue{Kuravsky, L.\,S., A.\,A.~Margolis, P.\,A.~Marmalyuk, A.\,S.~Panfilova, 
    G.\,A.~Yuryev, and P.\,N.~Dumin.}
     2016. A probabilistic model of adaptive training. \textit{Appl. 
Math. Sci.} 10(48):2369--2380.
   
    \bibitem{8-b-1} %9
    \Aue{Naumov, A.\,V., and G.\,A.~Mkhitaryan.} 2016. On the problem of 
probabilistic optimization of time-limited testing. 
\textit{Automat. Rem. Contr.} 77(9):1612--1621.
    \bibitem{9-b-1} %10
    \Aue{Naumov, A.\,V., G.\,A.~Mkhitaryan, and E.\,E.~Cherygova.} 2019. 
Stokhasticheskaya postanovka zadachi formirovaniya testa zadannogo urovnya 
slozhnosti s~mi\-ni\-mi\-za\-tsi\-ey kvantili vremeni vypolneniya [Stochastic formulation of 
the problem of forming a~test of a~given level of complexity with minimization of the 
quantile of runtime]. \textit{Vestnik komp'yuternykh i~informatsionnykh tekhnologiy} 
[Herald of Computer and Information Technologies] 2:37--46.
   
    \bibitem{11-b-1}
    \Aue{Kibzun, A.\,I., A.\,V.~Naumov, and V.\,I.~Norkin.} 2013. On reducing 
    a~quantile optimization problem with discrete distribution to a mixed integer 
programming problem. \textit{Automat. Remot. Contr.} 74(6):951--967.
    \bibitem{12-b-1}
    \Aue{Kan, Yu.\,S., and A.\,I.~Kibzun.} 2009. \textit{Zadachi sto\-kha\-sti\-che\-sko\-go 
programmirovaniya s~veroyatnostnymi kriteriyami} [Problems of stochastic 
programming with probabilistic criteria]. Moscow: Fizmatlit. 372~p.
    \bibitem{13-b-1}
    \Aue{Naumov, A.\,V., and S.\,V.~Ivanov.} 2011. On stochastic linear programming 
problems with the quantile criterion. \textit{Automat. Rem. Contr.} 71(2):353--369.
 \end{thebibliography}

 }
 }

\end{multicols}

%\vspace*{-7pt}

\hfill{\small\textit{Received May 30, 2019}}

%\pagebreak

%\vspace*{-22pt}
   

    
    \Contr
    
    \noindent
    \textbf{Bosov Alexey V.} (b.\ 1969)~--- Doctor of Science in 
technology, principal scientist, Institute of Informatics Problems, Federal 
Research Center ``Computer Science and Control'' of the Russian 
Academy of Sciences, 44-2~Vavilov Str., Moscow 119333, Russian 
Federation; \mbox{AVBosov@ipiran.ru}
    
    \noindent
    \textbf{Naumov Andrey V.} (b.\ 1966)~--- Doctor of Science in 
physics and mathematics, professor, Moscow Aviation Institute (National 
Research University), 4~Volokolamskoe Shosse, Moscow 125933, 
Russian Federation; \mbox{naumovav@mail.ru}
    
    \vspace*{3pt}
    
    \noindent
    \textbf{Mkhitaryan Georgy A.} (b.\ 1995)~--- PhD student, 
Department of Probability Theory and Computer Simulations Department, 
Faculty of Information Technologies and Applied Mathematics, Moscow 
Aviation Institute (National Research University), 4~Volokolamskoe Shosse, 
Moscow 125933, Russian Federation; \mbox{grgmkn@mail.ru}
    
    \vspace*{3pt}
    
    \noindent
    \textbf{Sapunova Anastasiya P.} (b.\ 1998)~--- student, Department 
of Probability Theory and Computer Simulations Department, Faculty of 
Information Technologies and Applied Mathematics, Moscow Aviation 
Institute (National Research University), 4~Volokolamskoe Shosse, 
Moscow 125933, Russian Federation; \mbox{sap2603@mail.ru}
  
      
\label{end\stat}

\renewcommand{\bibname}{\protect\rm Литература}  