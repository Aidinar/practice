\def\stat{goncharov}

\def\tit{ТЕМПОРАЛЬНЫЕ ДАННЫЕ В~ЛЕКСИКОГРАФИЧЕСКИХ\\ БАЗАХ ЗНАНИЙ$^*$}

\def\titkol{Темпоральные данные в~лексикографических базах знаний}

\def\aut{А.\,А.~Гончаров$^1$, И.\,М.~Зацман$^2$, М.\,Г.~Кружков$^3$}

\def\autkol{А.\,А.~Гончаров, И.\,М.~Зацман, М.\,Г.~Кружков}

\titel{\tit}{\aut}{\autkol}{\titkol}

\index{Гончаров А.\,А.}
\index{Зацман И.\,М.}
\index{Кружков М.\,Г.}
\index{Goncharov A.\,A.}
\index{Zatsman I.\,M.}
\index{Kruzhkov M.\,G.}



{\renewcommand{\thefootnote}{\fnsymbol{footnote}} \footnotetext[1]
{Работа выполнена в~Институте проблем информатики ФИЦ ИУ РАН при поддержке РФФИ 
(проект 18-07-00192).}}


\renewcommand{\thefootnote}{\arabic{footnote}}
\footnotetext[1]{Институт проблем информатики Федерального исследовательского центра <<Информатика 
и~управление>> Российской академии наук, \mbox{a.gonch48@gmail.com}}
\footnotetext[2]{Институт проблем информатики Федерального исследовательского центра <<Информатика 
и~управление>> Российской академии наук, \mbox{izatsman@yandex.ru}}
\footnotetext[3]{Институт проблем информатики Федерального исследовательского центра <<Информатика 
и~управление>> Российской академии наук, \mbox{magnit75@yandex.ru}}

\vspace*{-6pt}



  
   
   \Abst{Рассматривается подход к~проектированию лексикографической базы знаний 
(ЛБЗ), обеспечивающей решение двух взаимосвязанных задач: (1)~целенаправленного 
создания и~развития лингвистических типологий; (2)~формирования и~обновления 
электронных двуязычных словарей с~использованием создаваемых типологий. Часть 
полей ЛБЗ, которые предназначены для представления лексикографами нового знания 
о~значениях (=\;концептах) слов и~их переводах, являются темпоральными. Содержание 
этих полей зависит от времени, поскольку в~процессе работы лексикографы могут вносить 
изменения. Изменение концептов и~содержания полей ЛБЗ с~их описанием  
(=\;де\-фи\-ни\-ци\-ей) может повлечь за собой изменение структуры словарной статьи 
формируемого двуязычного словаря. Последовательное фиксирование подобных 
изменений~--- одна из функций ЛБЗ. Несмотря на то что первоначальная структура 
словарной статьи является исходной информацией для проектирования ЛБЗ, после 
завершения проектирования, в~процессе развития лингвистических типологий 
и~формирования словарных статей, она также может изменяться одновременно 
с~эволюцией концептов во времени. Цель работы состоит в~описании темпоральной 
структуры словарной статьи и~подхода к~проектированию ЛБЗ с~темпоральными данными, 
которая позволит решить обе поставленные задачи. Предлагаемый подход 
иллюстрируется на примере темпоральной структуры словарной статьи немецко-русского 
словаря.}
   
   \KW{лексикографическая база знаний; темпоральная структура словарной статьи; 
двуязычные словари; лингвистические типологии; параллельные тексты; эволюция 
концептов}

\DOI{10.14357/19922264190415} 
  
\vspace*{-6pt}


\vskip 10pt plus 9pt minus 6pt

\thispagestyle{headings}

\begin{multicols}{2}

\label{st\stat}

   
   
\section{Введение}

    Разработка информационной технологии (ИТ), которая обеспечивает 
создание и~развитие лингвистических типологий\footnote[4]{В лингвистике 
типологии используются, как правило, для описания сходств и~различий между языками. 
В~статье 
этот термин используется в~другом значении: для описания связей лемм с~их значениями 
и~переводными эквивалентами.} как форм представления и~сис\-те\-ма\-ти\-за\-ции нового 
знания о~лексическом составе естественных языков, является одной из задач 
проекта по гранту РФФИ, который в~настоящее время выполняется 
в~Институте проб\-лем информатики ФИЦ ИУ РАН. Проектируемая 
ИТ предназначена для решения двух 
взаимосвязанных задач: 
\begin{enumerate}[(1)]
\item целенаправленного создания и~развития 
линг\-ви\-сти\-че\-ских типологий на основе кон\-трас\-тив\-но\-го анализа текстов 
параллельных кор\-пу\-сов~[1, 2], содержащих исследуемые языковые единицы; 
\item формирования и~обновления электронных двуязычных словарей 
с~использованием создаваемых типологий.
\end{enumerate}

 Ключевым компонентом ИТ 
является ЛБЗ, обеспечивающая решение 
этих задач.
    
    Часть полей ЛБЗ, которые предназначены для представления 
лексикографами нового знания о~значениях (=\;кон\-цеп\-тах) слов и~их 
переводах, являются темпоральными, т.\,е.\ их содержание зависит от 
момента времени, когда концепт (который в~общем случае может 
модифицироваться) был описан лексикографом. Эти моменты времени 
и~соответствующие им значения хранятся в~ЛБЗ. Содержание таких полей 
будем называть темпоральными данными. Эти данные отражают этапы 
эволюции концептов, что является одной из функций ЛБЗ. При этом могут 
изменяться не только концепты и, соответственно, их описания  
(=\;де\-фи\-ни\-ции) в~полях ЛБЗ, но также и~структура словарной статьи 
формируемого двуязычного словаря.

 Исходной информацией для 
проектирования ЛБЗ является структура словарной статьи, обуслов\-лен\-ная 
требованиями лексикографов~[3]. Однако после завершения проектирования, в~процессе развития лингвистических типологий и~формирования словарных 
статей, эта структура может меняться.
% одновременно с~эволюцией концептов  во времени.
    
    Цель работы состоит в~описании структуры словарной статьи и~подхода к~проектированию ЛБЗ с~темпоральными данными, которая позволит решить 
обе поставленные задачи. Предлагаемый подход иллюстрируется примером 
темпоральной структуры словарной статьи немецко-русского словаря~[4].

\vspace*{-6pt}
    
\section{Контрастивный анализ и~создание типологий}

\vspace*{-2pt}

    Одна из функций проектируемой ЛБЗ~--- обеспечение контрастивного 
анализа параллельных текс\-тов в~целях извлечения имплицированных знаний 
о переводных соответствиях (примеры знаний,\linebreak имплицированных в~текстах, 
рассмотрены в~рабо\-те~[5]), создания и~развития на их основе 
лингвистических типологий (т.\,е.\ решение первой\linebreak поставленной задачи). 
Процесс извлечения из параллельных текстов имплицированных знаний, 
создания и~развития на их основе лингвистических типологий включает 
несколько повторяемых стадий, описанию которых посвящены работы~[6, 7]. 
Реализуемость этого процесса была проверена на примере описания новых 
значений немецких модальных глаголов. Была разработана иерархическая 
трехуровневая структура (см.~[7, рис.~1]) по\-пол\-ня\-емой типологии 
немецких модальных глаголов и~их переводных эквивалентов на русском 
языке~[8,~9].
{\looseness=1

}
    
    Отметим, что решение первой задачи, описанное  
в~работах~\cite{2-gon, 5-gon, 6-gon, 7-gon}, обеспечивается надкорпусной 
базой данных (НБД) с~нетемпоральными полями, в~которых описываются 
только последние актуальные значения слов без отражения динамики их 
эволюции. Такие НБД имеют широкую сферу применения: исследование 
различных категорий языковых единиц методами контрастивной 
лингвистики~\cite{10-gon, 11-gon, 12-gon, 13-gon, 14-gon}, анализ 
профессиональных и~машинных переводных  
соответствий~\cite{15-gon, 16-gon}, создание и~развитие лингвистических 
типологий~\cite{17-gon}.
    
    Помимо перечисленных сфер применения НБД способны поддерживать 
решение отдельных лексикографических задач, например фиксировать 
в~форме аннотаций переводные соответствия между употреблениями 
немецких модальных глаголов и~их переводами на русский  
язык~\cite{3-gon, 7-gon}. Однако одновременное решение двух 
вышеперечисленных задач, включая обеспечение работ лексикографов по 
формированию электронных двуязычных словарей с~использованием 
создаваемых типологий, обусловило проектирование ЛБЗ с~темпоральными 
данными на основе решений, опробованных при разработке НБД.
    
\vspace*{-6pt}

\section{Структура словарной статьи}

\vspace*{-2pt}

    На сегодняшний день двуязычные электронные словари нередко 
представляют собой лишь оциф\-ро\-ван\-ную версию их бумажных оригиналов, 
т.\,е.\ содержательные различия отсутствуют: меняется лишь материальный 
носитель~\cite{18-gon}. Отметим ряд ограничений, с~которыми сталкиваются 
пользователи таких словарей:
    \begin{enumerate}[(1)]
\item сложность навигации при работе с~объемными (часто 
многотомными) словарями;
\item отставание словарей от текущей языковой ситуации;
\item отсутствие возможности внесения изменений в~словарь (за 
исключением переизданий);
\item в~случае воплощения оригинальных концепций словаря~--- малый 
объем описанного материала~\cite{18-gon, 19-gon}.
\end{enumerate}

    Эти ограничения касаются как пользователей словарей, так и~их 
составителей, особенно ввиду современных тенденций развития двуязычной 
лексикографии, заключающихся <<в~сближении принципов словарного 
и~собственно научного описания лексических  
единиц>>~\cite[с.~35]{20-gon}. Методы компьютерной лингвистики по 
представлению знаний в~сочетании с~методами и~средствами информатики, 
в~частности НБД и~ЛБЗ, используемые в~процессе формирования и~развития 
словарей, позволяют удовлетворить гораздо более широкий спектр 
потребностей пользователей по сравнению с~бумажными словарями и~их 
оцифрованными версиями и~в той или иной мере снять перечисленные 
ограничения. 
    
    Также следует учитывать, что в~процессе формирования словарных 
статей необходимо описывать и~такие языковые единицы, для которых 
характерна ярко выраженная полисемия (<<наличие у~языкового знака более 
чем одного значения>>~\cite[с.~59]{20-gon}). Анализу этой проблемы 
посвящены два раздела книги <<Беседы о~немецком слове>>: на 
русском~\cite[с.~59--73]{20-gon} и~на немецком  
языке~\cite[с.~73--106]{20-gon}.
{ %\looseness=-1

}

\begin{table*}[b]\small %tabl1
\vspace*{-6pt}
\begin{center}
\Caption{Структура словарной статьи, используемая в~словаре~\cite{4-gon}}
\vspace*{2ex}

\begin{tabular}{|c|l|c|l|l|}
\hline
\multicolumn{5}{|c|}{\textbf{Заглавное слово} (=\;\textbf{лемма}, =\;\textbf{лексический вход}) и~его 
{\bfseries\textit{варианты}}}\\
\hline
I&\multicolumn{4}{l|}{Зона этимологии}\\
\hline
II&\multicolumn{4}{l|}{Зона фонетической транскрипции}\\
\hline
III&\multicolumn{4}{l|}{Зона \textit{грамматической информации о лемме в~целом}}\\
\hline
IV&\multicolumn{4}{l|}{Зона \textit{стилистических помет, относящихся к~лемме 
в~целом}}\\
\hline
\multicolumn{1}{|c|}{\raisebox{-132pt}[0pt][0pt]{V}}&\multicolumn{1}{c|}{\raisebox{-132pt}[0pt][0pt]{Зона значения}}&
\multicolumn{1}{c|}{\raisebox{-96pt}[0pt][0pt]{Значение~1}}&\multicolumn{2}{l|}{\textit{Грамматическая информация 
о лемме в~значении~1}~(t)}\\
\cline{4-5}
&&&\multicolumn{2}{l|}{\textit{Стилистические пометы, относящиеся к~лемме
в~значении~1}~(t)}\\
\cline{4-5}
&&&\multicolumn{2}{l|}{Толкование леммы в~значении~1~(t)}\\
\cline{4-5}
&&&\multicolumn{2}{l|}{Варианты перевода леммы в~значении~1~(t)}\\
\cline{4-5}
&&&\multicolumn{2}{l|}{Примеры употребления леммы в~значении~1~(t)}\\
\cline{4-5}
&&&&\tabcolsep=0pt\begin{tabular}{l}
\textit{Грамматическая информация о лемме}\\
\textit{в данном 
подзначении}~(t)\end{tabular}\\
\cline{5-5}
&&&&\tabcolsep=0pt\begin{tabular}{l}
\textit{Стилистические пометы, относящиеся к~лемме}\\
\textit{в~данном подзначении}~(t)\end{tabular}\\
\cline{5-5}
&&&Подзначение~1&Толкование леммы в~данном подзначении~(t)\\
\cline{5-5}
&&&&\tabcolsep=0pt\begin{tabular}{l}Варианты перевода леммы в~данном\\
 подзначении~(t)\end{tabular}\\
\cline{5-5}
&&&&\tabcolsep=0pt\begin{tabular}{l}Примеры употребления леммы в~данном\\
 подзначении~(t)\end{tabular}\\
\cline{4-5}
&&&Подзначение 2&\multicolumn{1}{c|}{$\ldots$}\\
\cline{4-5}
&&&\multicolumn{1}{c|}{$\ldots$}&\multicolumn{1}{c|}{$\ldots$}\\
\cline{4-5}
&&&Подзначение $M_1$~(t)&\multicolumn{1}{c|}{$\ldots$}\\
\cline{4-5}
&&&\multicolumn{2}{l|}{\textbf{Грамматическая фразеология} с~использованием 
леммы в~значении~1~(t)}\\
\cline{3-5}
&&$\ldots$&\multicolumn{2}{c|}{$\ldots$}\\
\cline{3-5}
&&&\multicolumn{2}{c|}{$\ldots$}\\
\cline{4-5}
&&&Подзначение 1&\multicolumn{1}{c|}{$\ldots$}\\
\cline{4-5}
&&&Подзначение 2&\multicolumn{1}{c|}{$\ldots$}\\
\cline{4-5}
&&
\multicolumn{1}{c|}{\raisebox{6pt}[0pt][0pt]{Значение $N$~(t)}}&\multicolumn{1}{c|}{$\ldots$}&\multicolumn{1}{c|}{$\ldots$}\\
\cline{4-5}
&&&Подзначение $M_N$~(t)&\multicolumn{1}{c|}{$\ldots$}\\
\cline{4-5}
&&&\multicolumn{2}{l|}{\textbf{Грамматическая фразеология} с~использованием 
леммы в~значении~$N$~(t)}\\
\hline
VI&\multicolumn{4}{l|}{Зона \textbf{идиоматики}~(t)}\\
\hline
\end{tabular}
\end{center}
\end{table*}
    
    К многозначным единицам относятся и~немецкие модальные глаголы: 
для некоторых из них сейчас описано  
более~10~значений~\cite{4-gon, 8-gon, 9-gon}. Необходимо не только 
иметь воз\-мож\-ность уточнять описания, во-пер\-вых, значений (=\;кон\-цеп\-тов) 
и,~во-вто\-рых, функционирования модальных глаголов немецкого языка и~их 
русскоязычных соответствий, но и~видеть ретроспективу этих изменений 
(что должно позволить детально описать механизмы составления словарных 
статей и~эволюции их структуры). Для этого средствами ЛБЗ с~темпоральными 
данными предлагается фиксировать динамику изменения концептов, включая 
случаи извлечения из параллельных текстов новых значений модальных 
глаголов.
    
    Прежде всего (до описания нового функционала ЛБЗ) следует 
рассмотреть структуру словарной статьи в~не\-мец\-ко-рус\-ском  
словаре~\cite{4-gon}. Модель его словарной статьи для удобства можно 
представить в~табличной форме (табл.~1), причем не все включенные 
в~таблицу зоны (=\;ком\-по\-нен\-ты) обязательны для каждой статьи 
словаря. Используемые в~таблице термины подробно рассмотрены  
в~\cite{20-gon, 21-gon}.
{ %\looseness=1

}
    
    В словарях, формируемых средствами традиционной лексикографии, 
зоны словарной статьи часто недостаточно четко отделены друг от друга. 
Беляева отмечает: <<Особую сложность представляет собственно 
установление границ компонентов, поскольку в~``бумажных'' словарях они не 
всегда выделяются и/или маркируются специальными символами, 
и~соотнесение выделенных компонентов с~параметрами  
описания>>~\cite{19-gon}.
    
    Статьи в~разных словарях могут быть структурированы по-разному. 
Поэтому до начала формирования того или иного электронного словаря 
с~помощью ЛБЗ создается модель именно его \mbox{статьи}. Использование таких 
моделей дает возможность эксплицировать границы всех компонентов 
словарной статьи. Те из них, которые являются темпоральными, отмечены 
в~табл.~1 знаком~(t). Число значений слова ($N$) и~подзначений 
($M_1,\ldots , M_N$) также может меняться во времени.

\begin{table*}\small %footnotesize %tabl2
\begin{center}
\Caption{Виды изменений в~словарных статьях}
\vspace*{2ex}

\tabcolsep=4pt
\begin{tabular}{|c|c|c|c|c|c|c|c|}
\hline
\multicolumn{1}{|c|}{\raisebox{-28pt}[0pt][0pt]{
\tabcolsep=0pt\begin{tabular}{c}ХАРАКТЕР\\ ИЗМЕНЕНИЯ\end{tabular}}}&
\multicolumn{7}{c|}{\underline{\textbf{Зона значения}}}\\
%&\multicolumn{1}{c|}{\raisebox{-24pt}[0pt][0pt]{
%\tabcolsep=0pt\begin{tabular}{c}\underline{\textbf{Зона}}\\ 
%\underline{\textbf{идиоматики}}\end{tabular}}}\\
\cline{2-8}
&
\multicolumn{1}{c|}{\raisebox{-24pt}[0pt][0pt]
{\tabcolsep=0pt\begin{tabular}{c}Все зоны\\ одного\\ 
значения\end{tabular}}}&
\multicolumn{6}{c|}{Зоны внутри одного из значений}\\
\cline{3-8}
&&\tabcolsep=0pt\begin{tabular}{c}Граммати-\\ ческая\\ информация\end{tabular}&
\tabcolsep=0pt\begin{tabular}{c}Стилисти-\\ ческие\\ пометы\end{tabular}&
\tabcolsep=0pt\begin{tabular}{c}Толкование\end{tabular}&
\tabcolsep=0pt\begin{tabular}{c}Варианты\\ перевода\end{tabular}&
\tabcolsep=0pt\begin{tabular}{c}Примеры\\ 
употреб-\\ ления\end{tabular}&
\tabcolsep=0pt\begin{tabular}{c}Граммати-\\ ческая\\ фразео-\\ логия\end{tabular}\\
\hline
Перемещение&&&&&&$+$&\\
Изменение содержания&&&$+$&$+$&&$+$&\\
Добавление&&$+$&&$+$&+&$+$&$+$\\
Удаление&$+$&$+$&&&&$+$&\\
\hline
\end{tabular}
\end{center}
\vspace*{3pt}
\end{table*}
    
    Контрастивный анализ текстов параллельного немецко-русского 
корпуса, при проведении которого найденные употребления модальных 
глаголов распределяются по их значениям в~соответствии со структурой 
словарной статьи, может повлечь за собой внесение тех или иных изменений в~первоначально созданную структуру. Как показали первые результаты 
контрастивного анализа, эти изменения пока не касаются зон~I--IV из 
табл.~1, по крайней мере применительно к~модальным глаголам. При этом 
зоны~V и~VI могут затрагиваться изменениями в~высокой степени. 
    
    Виды изменений, вносимых в~словарную статью, резюмируются 
в~табл.~2 на примере одной из зон~--- зоны значения. 
Что касается случаев внесения изменений в~то или иное 
подзначение, то они в~данной таблице для удобства рассматриваются как 
изменения соответствующих компонентов того значения, в~рамках которого 
выделено данное подзначение. Для иллюстрации описываемых положений 
табл.~2 была заполнена на основании изменений, внесенных в~процессе 
контрастивного анализа в~словарную статью немецкого глагола 
\textit{wollen}. Типы изменений, внесенных в~статью хотя бы один раз, 
отмечены знаком~<<$+$>>.


\vspace*{-3pt}

\section{Представление темпоральных данных
в~лексикографических базах~знаний}

\vspace*{-2pt}

    Функции регистрации истории изменений в~словарных статьях 
планируется впервые реализовать в~ЛБЗ с~темпоральными данными, 
проектируемой на основе НБД аннотаций употреблений немецких модальных 
глаголов~\cite{3-gon}. Предлагаемый подход к~реализации этих функций 
включает в~себя три составляющие.
    
    Во-первых, для всех таблиц, отражающих лексикографическую 
информацию, создаются таблицы-дубликаты, автоматически заполняющиеся 
при внесении изменений в~словарные статьи. Особенность этих таблиц 
заключается в~том, что данные из них никогда не будут удаляться, они будут 
лишь пополняться в~соответствии с~изменениями в~исходных таблицах. 
Таб\-ли\-цы-дуб\-ли\-ка\-ты включают те же поля, что и~исходные таблицы, но, 
кроме того, они дополнены несколькими новыми полями, с~помощью 
которых можно будет отслеживать историю изменений. Предполагается 
использовать следующие дополнительные поля:
    \begin{itemize}
\item Тип изменения~--- это поле имеет три варианта значения: 
<<создание>>, <<изменение>>, <<удаление>>; значения будут заполняться 
в~соответствии с~тем, какие операции осуществляются с~исходными 
таблицами;\\[-15pt]
\item Дата изменения~--- это поле в~совокупности с~предыдущим позволит 
восстанавливать лексикографическую информацию на любой заданный 
момент времени, а также отслеживать историю изменений;\\[-15pt]
\item Идентификатор комплексной операции (см.\ ниже);\\[-15pt]
\item Идентификатор пользователя, внесшего изменение.
\end{itemize}
    
    Во-вторых, в~ЛБЗ формируется таблица, в~которой будут 
регистрироваться <<комплексные операции>>. Необходимость в~такой 
таблице обусловлена тем, что обычно лексикографические изменения 
происходят не обособленно, а~в~рамках некоего комплекса изменений, 
одновременно за\-тра\-ги\-ва\-ющих разные компоненты словарных статей. 
Например, если для некоторого модального глагола создается новое значение 
под номером~3, то номера всех последующих значений, существовавших до 
этого времени, также изменяются (как правило, увеличиваются на~1). 
    
    Кроме того, из одного значения в~другое нередко одновременно могут 
перемещаться разные компоненты словарной статьи, такие как толкование, 
варианты перевода, грамматическая информация, примеры и~т.\,д. Эти 
изменения могут затрагивать различные таблицы и~разные строки в~этих 
таблицах. При независимой регистрации таких изменений  
в~таб\-ли\-цах-дуб\-ли\-ка\-тах задача объединения этих изменений 
в~комплекс логически связанных изменений может оказаться нетривиальной. 
Поэтому в~целях их экспликации предлагается использовать специальную 
таблицу <<комплексные операции>>. В~этой таблице будет содержаться 
явное вербальное описание логически связанных изменений, производимых 
в~словарных статьях.
    
    Наконец, для отражения в~ЛБЗ темпоральных изменений в~соответствии 
с~табл.~2 предлагается использовать еще одну дополнительную таблицу 
<<элементарные операции>>, которая, с~одной стороны, будет связана 
с~таблицей <<комплексные операции>>, а~с~другой~--- с~конкретными 
строками в~таб\-ли\-цах-дуб\-ли\-ка\-тах. Среди полей этой таблицы будут 
следующие: идентификатор комплексной операции, характер изменения (см. 
табл.~2) и~компонент (зона) изменения (грамматическая информация, 
толкование, варианты перевода, примеры и~т.\,д.). Данная таблица позволит 
дополнить информацию, содержащуюся в~таб\-ли\-цах-дуб\-ли\-ка\-тах, 
и~вербальное описание изменений, данное в~таб\-ли\-це <<комплексные 
операции>>, более четким структурированным описанием составляющих их 
операций.
    
    
    \vspace*{-3pt}
    
\section{Заключение}

\vspace*{-3pt}

    В процессе создания и~развития электронных двуязычных словарей 
структура их статей может изменяться в~тех случаях, когда лексикографы 
уточняют значения слов или обнаруживают их новые значения и~включают 
соответствующие толкования в~словарные статьи. Использование НБД для  
ин\-фор\-ма\-ци\-он\-но-ком\-пью\-тер\-но\-го обеспечения этого процесса 
позволяет регистрировать такие изменения и~тем самым фиксировать 
выявление новых и~изменения в~описаниях уже известных значений  
слов~\cite{3-gon, 7-gon, 8-gon, 9-gon}, но хранится в~НБД только последний 
вариант описания.
    
    Основное отличие ЛБЗ с~темпоральными данными от НБД состоит 
в~том, что дополнительно появляется возможность увидеть всю 
ретроспективу подобных изменений. При этом и~в ЛБЗ, и~в НБД каждое 
уточнение значения слова или описание его нового значения иллюстрируется 
примерами его употребления, извлекаемыми лексикографами из 
параллельных текстов в~процессе контрастивного анализа. Таким образом, 
ЛБЗ позволяет не только получить систематизированное знание о словах, 
проиллюстрированное примерами, включаемыми в~словарные статьи, но 
и~проследить историю сис\-те\-ма\-ти\-за\-ции этого знания.
    
\vspace*{-3pt}
    
{\small\frenchspacing
 {%\baselineskip=10.8pt
 \addcontentsline{toc}{section}{References}
 \begin{thebibliography}{99}
 
 \vspace*{-3pt}
 
\bibitem{1-gon}
\Au{Добровольский Д.\,О., Кретов~А.\,А., Шаров~С.\,А.} Корпус параллельных текстов~// 
Научно-техническая информация. Сер.~2: Информационные процессы и~сис\-те\-мы, 2005. 
№\,6. С.~27--42.
\bibitem{2-gon}
\Au{Гончаров А.\,А., Зацман~И.\,М.} Информационные трансформации параллельных 
текстов в~задачах извлечения знаний~// Системы и~средства информатики, 2019. Т.~29. 
№\,1. С.~180--193.
\bibitem{3-gon}
\Au{Добровольский Д.\,О., Зализняк Анна~А.} Немецкие конструкции с~модальными 
глаголами и~их русские соответствия: проект надкорпусной базы данных~// 
Компьютерная лингвистика и~интеллектуальные\linebreak \mbox{технологии}: по мат-лам 
Междунар. конф. 
<<Диалог>>.~--- М.: РГГУ, 2018. Вып.~17(24). С.~172--184.
\bibitem{4-gon}
Немецко-русский словарь: актуальная лексика~/ Под ред. Д.\,О.~Добровольского.~--- М.: 
Лексрус, 2019 (в~печати).
\bibitem{5-gon}
\Au{Зацман И.\,М.} Имплицированные знания: основания и~технологии извлечения~// 
Информатика и~её применения, 2018. Т.~12. Вып.~3. С.~74--82.
\bibitem{6-gon}
\Au{Зацман И.\,М.} Стадии целенаправленного извлечения знаний, имплицированных 
в~параллельных текстах~// Системы и~средства информатики, 2018. Т.~28. №\,3.  
С.~175--188.
\bibitem{7-gon}
\Au{Зацман И.\,М.} Целенаправленное развитие систем лингвистических знаний: 
выявление и~заполнение лакун~// Информатика и~её применения, 2019. Т.~13. Вып.~1. 
С.~91--98.
\bibitem{8-gon}
\Au{Zatsman I.} Goal-oriented creation of individual knowledge: Model and information 
technology~// 19th European Conference on Knowledge Management Proceedings.~---  
Reading: Academic Publishing International Ltd., 2018. Vol.~2. P.~947--956.
\bibitem{9-gon}
\Au{Zatsman I.} Finding and filling lacunas in knowledge systems~// 20th European Conference 
on Knowledge Management Proceedings.~---  Reading: Academic Publishing International Ltd., 
2019. Vol.~2. P.~1143--1151.
\bibitem{10-gon}
\Au{Kruzhkov M.\,G., Buntman~N.\, V., Loshchilova~E.\,Ju., Sitchinava~D.\,V., Zalizniak 
Anna~A., Zatsman~I.\,M.} A~database of Russian verbal forms and their French translation 
equivalents~// Компьютерная лингвистика и~интеллектуальные технологии: по мат-лам 
Междунар. конф. <<Диалог>>.~--- М.: РГГУ, 2014. Вып.~13(20). С.~275--287.
\bibitem{11-gon}
\Au{Зализняк Анна А.} Лингвоспецифичные единицы русского языка в~свете 
контрастивного корпусного анализа~// Компьютерная лингвистика и~интеллектуальные 
технологии: по мат-лам Междунар. конф.\linebreak <<Диалог>>.~--- М.: РГГУ, 2015. Вып.~14(21). 
С.~683--695.

\bibitem{13-gon} %12
\Au{Инькова О.\,Ю., Кружков~М.\,Г.} Надкорпусные рус\-ско-фран\-цуз\-ские базы 
данных глагольных форм и~коннекторов~// Lingue slave a~confronto~/ Eds. O.~Inkova, 
A.~Trovesi.~--- Bergamo: Bergamo University Press, 2016. P.~365--392.
\bibitem{14-gon} %13
\Au{Зацман И.\,М., Инькова~О.\,Ю., Кружков~М.\,Г., Попкова~Н.\,А.} Представление 
кросс-языковых знаний о~коннекторах в~надкорпусных базах данных~// 
Информатика и~её 
применения, 2016. Т.~10. Вып.~1. С.~106--118.
\bibitem{12-gon} %14
\Au{Зализняк Анна А., Кружков М.\,Г.}
База данных безличных глагольных конструкций 
русского языка~// Информатика и~её применения, 2016. Т.~10. Вып.~4. С.~132--141.
\bibitem{15-gon}
\Au{Nuriev V., Buntman~N., Inkova~O.} Machine translation of Russian connectives into 
French: Errors and quality failures~// Информатика и~её применения, 2018. Т.~12. Вып.~2.  
С.~105--113.
\bibitem{16-gon}
\Au{Бунтман~Н.\,В., Гончаров~А.\,А., Зацман~И.\,М., Нуриев~В.\,А.} Количественный 
анализ результатов машинного перевода с~использованием надкорпусных баз данных~// 
Информатика и~её применения, 2018. Т.~12. Вып.~4. С.~100--109.

%\columnbreak

\bibitem{17-gon}
\Au{Zatsman I.\,M., Inkova O.\,Yu., Nuriev~V.\,A.} The construction of classification schemes: 
Methods and technologies of expert formation~// Automat. Doc. Math. Linguist., 
2017. Vol.~51. No.\,1. P.~27--41.
\bibitem{18-gon}
\Au{Селегей В.\,П.} Компьютерная лексикография. {\sf  
https://www.abbyy.com/ru-ru/science/technologies/ lexicography}.
\bibitem{19-gon}
\Au{Беляева Л.\,Н.} Потенциал автоматизированной лексикографии и~прикладная 
лингвистика~// Известия Российского государственного педагогического университета 
им. А.\,И.~Герцена, 2010. №\,134. С.~70--79.
\bibitem{20-gon}
\Au{Добровольский Д.\,О.} Беседы о немецком слове.~--- М.: Языки славянской культуры, 
2013. 744~с.
\bibitem{21-gon}
\Au{Баранов А.\,Н.} Введение в~прикладную лингвистику.~--- М.: 
Эдиториал УРСС, 2001. 360~с.
 \end{thebibliography}

 }
 }

\end{multicols}

\vspace*{-9pt}

\hfill{\small\textit{Поступила в~редакцию 01.10.19}}

\vspace*{6pt}

%\pagebreak

%\newpage

%\vspace*{-28pt}

\hrule

\vspace*{2pt}

\hrule

\vspace*{-6pt}

\def\tit{TEMPORAL DATA IN~LEXICOGRAPHIC DATABASES\\[-5pt]}


\def\titkol{Temporal data in~lexicographic databases}

\def\aut{A.\,A.~Goncharov, I.\,M.~Zatsman, and~M.\,G.~Kruzhkov}

\def\autkol{A.\,A.~Goncharov, I.\,M.~Zatsman, and~M.\,G.~Kruzhkov}

\titel{\tit}{\aut}{\autkol}{\titkol}

\vspace*{-15pt}


 \noindent
   Institute of Informatics Problems, Federal Research Center ``Computer Sciences and 
Control'' of the Russian Academy of Sciences; 44-2~Vavilov Str., Moscow 119133, 
Russian Federation

\def\leftfootline{\small{\textbf{\thepage}
\hfill INFORMATIKA I EE PRIMENENIYA~--- INFORMATICS AND
APPLICATIONS\ \ \ 2019\ \ \ volume~13\ \ \ issue\ 4}
}%
 \def\rightfootline{\small{INFORMATIKA I EE PRIMENENIYA~---
INFORMATICS AND APPLICATIONS\ \ \ 2019\ \ \ volume~13\ \ \ issue\ 4
\hfill \textbf{\thepage}}}

\vspace*{3pt}  


   \Abste{The paper describes an approach to design of the Lexicographic Knowledge Base 
(LKB), which aims to fulfill two interrelated tasks: ($i$)~goal-oriented development of linguistic 
typologies; and ($ii$)~creation and updating of electronic bilingual dictionaries based on the developed 
typologies. In the LKB, some of the fields assigned by lexicographers to represent new 
knowledge on words' meanings (concepts) and translations are temporal. The content of these 
fields is time-dependent because lexicographers can change description of concepts with 
time. These changes may involve not only changes to existing concepts and appropriate fields 
with their descriptions (definitions), but also the changes to structure of individual dictionary 
entries of a bilingual dictionary.
One of the LKB's goals is to provide a method to describe these 
changes consistently. Although the LKB's initial design relies on the existing structure of 
dictionary entries, this structure may evolve along with the appropriate concepts as the underlying 
linguistic typologies and dictionary entries evolve with time. The goal of this paper is to describe 
the temporal structure of a dictionary entry and the approach to design of the LKB with temporal 
data that will be able to fulfill both of the specified tasks. 
The proposed approach is illustrated by 
an~example of the temporal dictionary entry structure of a~German--Russian dictionary.}

   \KWE{lexicographic knowledge base; temporal structure of a dictionary entry; bilingual 
dictionaries; linguistic typologies; parallel texts; evolution of concepts}
   
   

 \DOI{10.14357/19922264190415} 

\vspace*{-12pt}

 \Ack
 
 \vspace*{-4pt}
 
   \noindent
   The work was carried out at the Institute of Informatics Problems (FRC CSC RAS) funded 
by the Russian Foundation for Basic Research according to research project No.\,18-07-00192.


%\vspace*{-6pt}

  \begin{multicols}{2}

\renewcommand{\bibname}{\protect\rmfamily References}
%\renewcommand{\bibname}{\large\protect\rm References}

{\small\frenchspacing
 {%\baselineskip=10.8pt
 \addcontentsline{toc}{section}{References}
 \begin{thebibliography}{99}
 
 %\vspace*{-4pt}
 
 
\bibitem{1-gon-1}
\Aue{Dobrovol'skiy, D.\,O., A.\,A.~Kretov, and S.\,A.~Sharov.} 2005. 
Corpus of parallel texts. \textit{Automatic Information Math. Linguistics} 6:16--27.
\bibitem{2-gon-1}
\Aue{Goncharov, A.\,A., and I.\,M.~Zatsman.} 2019. In\-for\-ma\-tsi\-on\-nye transformatsii 
parallel'nykh tekstov v~zadachakh izvlecheniya znaniy [Information transformations of parallel 
texts in knowledge extraction]. \textit{Sistemy i~Sredstva Informatiki~--- Systems and Means of 
Informatics} 29(1):180--193.
\bibitem{3-gon-1}
\Aue{Dobrovol'skiy, D.\,O., and A.\,A.~Zaliznyak.} 2018. Ne\-mets\-kie konstruktsii 
s~modal'nymi glagolami i~ikh russkiye sootvetstviya: proyekt nadkorpusnoy bazy dannykh 
[German constructions with modal verbs and their Russian correlates: A~supracorpora database 
project]. \textit{Komp'yuternaya lingvistika i~intellektual'nyye tekhnologii: po mat-lam 
Mezhdunar. konf. ``Dialog''} [Computational Linguistics and\linebreak Intellectual Technologies. Papers 
from the Annual Conference (International) ``Dialogue'']. Moscow. 17(24):172--184.


\bibitem{4-gon-1}
Dobrovol'skiy, D.\,O. ed. 2019 (in press). \textit{Nemetsko-russkiy slovar': aktual'naya 
leksika} [German--Russian dictionary: Actual vocabulary]. Moscow: Leksrus.

%\pagebreak

\bibitem{5-gon-1}
\Aue{Zatsman, I.\,M.} 2018. Implitsirovannye znaniya: osnovaniya i~tekhnologii izvlecheniya 
[Implied knowledge: Foundations and technologies of explication]. \textit{Informatika i~ee 
Primeneniya~--- Inform. Appl.} 12(3):74--82.
\bibitem{6-gon-1}
\Aue{Zatsman, I.\,M.} 2018. Stadii tselenapravlennogo izvlecheniya znaniy, implitsirovannykh 
v~parallel'nykh tekstakh [Stages of goal-oriented discovery of knowledge implied in parallel 
texts]. \textit{Sistemy i~Sredstva Informatiki~--- Systems and Means of Informatics} 
28(3):175--188.
\bibitem{7-gon-1}
\Aue{Zatsman, I.\,M.} 2019. Tselenapravlennoe razvitie sistem lingvisticheskikh znaniy: 
vyyavlenie i~zapolnenie lakun [Goal-oriented development of linguistic knowledge systems: 
Identifying and filling lacunae]. \textit{Informatika i~ee Primeneniya~--- Inform. Appl.} 
13(1):91--98.
\bibitem{8-gon-1}
\Aue{Zatsman, I.\,M.} 2018. Goal-oriented creation of individual knowledge: Model and 
information technology. \textit{19th European Conference on Knowledge Management 
Proceedings}. Reading: Academic Publishing International Ltd. 2:947--956.
\bibitem{9-gon-1}
\Aue{Zatsman, I.\,M.} 2019. Finding and filling lacunas in knowledge systems. \textit{20th 
European Conference on Knowledge Management Proceedings.} Reading: Academic Publishing 
International Ltd. 2:1143--1151.
\bibitem{10-gon-1}
\Aue{Kruzhkov, M.\,G., N.\,V.~Buntman, E.\,Ju.~Loshchilova, D.\,V.~Sitchinava, 
A.\,A.~Zalizniak, and I.\,M.~Zatsman.} 2014. A~database of Russian verbal forms and their 
French translation equivalents. \textit{Komp'yuternaya lingvistika i~intellektual'nyye tekhnologii: 
po mat-lam Mezhdunar. konf. ``Dialog''} [Computational Linguistics and Intellectual 
Technologies. Papers from the Annual International Conference ``Dialogue'']. Moscow. 
13(20):275--287.
\bibitem{11-gon-1}
\Aue{Zalizniak, Anna~A.} 2015. Lingvospetsifichnye edi\-ni\-tsy russkogo yazyka v~svete 
kontrastivnogo korpusnogo analiza [Russian language-specific words as an object of contrastive 
corpus analysis]. \textit{Komp'yuternaya lingvistika i~intellektual'nyye tekhnologii: po mat-lam 
Mezhdunar. konf. ``Dialog''} [Computational Linguistics and Intellectual Technologies. Papers 
from the Annual International Conference ``Dialogue'']. Moscow. 13(20):683--695.

\bibitem{13-gon-1} %12
\Aue{Inkova, O.\,Yu., and M.\,G.~Kruzhkov.} 2016. Nadkorpusnye russko-frantsuzskie bazy 
dannykh glagol'nykh form i~konnektorov [Supracorpora databases of Russian and French verbal 
forms and connectors]. \textit{Lingue slave a~confronto}. Eds. 
O.~Inkova and A.~Trovesi. Bergamo: Bergamo University Press. 365--392.
\bibitem{14-gon-1} %13
\Aue{Zatsman, I.\,M., O.\,Yu.~Inkova, M.\,G.~Kruzhkov, and N.\,A.~Popkova.} 2016. 
Predstavlenie kross-yazykovykh znaniy o~konnektorakh v~nadkorpusnykh bazakh dannykh 
[Representation of cross-lingual knowledge about connectors in suprocorpora databases]. 
\textit{Informatika i~ee Primeneniya~--- Inform. Appl.} 10(1):106--118.
\bibitem{12-gon-1} %14
\Aue{Zaliznyak, Anna~A., and M.\,G.~Kruzhkov.} 2016. Baza dannykh bezlichnykh 
glagol'nykh konstruktsiy russkogo yazyka [Database of Russian impersonal verbal 
constructions]. \textit{Informatika i~ee Primeneniya~--- Inform. Appl.} 10(4):132--141.
\bibitem{15-gon-1}
\Aue{Nuriev, V., N.~Buntman, and O.~Inkova.} 2018. Machine translation of Russian 
connectives into French: Errors and quality failures. \textit{Informatika i~ee Primeneniya~--- 
Inform. Appl.} 12(2):105--113.
\bibitem{16-gon-1}
\Aue{Buntman, N.\,V., A.\,A.~Goncharov, I.\,M.~Zatsman, and V.\,A.~Nuriev.} 2018. 
Kolichestvennyy analiz rezul'tatov mashinnogo perevoda s~ispol'zovaniem nadkorpusnykh baz 
dannykh [Using supracorpora databases for quantitative analysis of machine translations]. 
\textit{Informatika i~ee Primeneniya~--- Inform. Appl.} 12(4):96--105.
\bibitem{17-gon-1}
\Aue{Zatsman, I.\,M., O.\,Yu.~Inkova, and V.\,A.~Nuriev.} 2017. The construction of 
classification schemes: Methods and technologies of expert formation. \textit{Autom. Doc. Math. 
Linguist.} 51(1):27--41.
\bibitem{18-gon-1}
\Aue{Selegey, V.\,P.} Komp'yuternaya leksikografiya [Computational lexicography]. Available 
at: {\sf https://www.abbyy. com/ru-ru/science/technologies/lexicography/} (accessed 
September~6, 2019).
\bibitem{19-gon-1}
\Aue{Belyaeva, L.\,N.} 2010. Potentsial avtomatizirovannoy leksikografii i~prikladnaya 
lingvistika  [Automatic lexicography scope and applied linguistics]. \textit{Izvestiya 
Rossiyskogo gosudarstvennogo pedagogicheskogo universiteta im. A.\,I.~Gertsena} 
[Izvestiya: 
Herzen University J.~Humanities Sciences] 134:70--79.
\bibitem{20-gon-1}
\Aue{Dobrovol'skiy, D.\,O.} 2013. \textit{Besedy o~nemetskom slove} [Studies on German 
lexis]. Moscow: Yazyki slavyanskoy kul'tury. 744~p.
\bibitem{21-gon-1}
\Aue{Baranov, A.\,N.} 2001. \textit{Vvedenie v~prikladnuyu lingvistiku} 
[Introductory handbook on applied linguistics]. Moscow: Editorial URSS. 360~p.
\end{thebibliography}

 }
 }

\end{multicols}

\vspace*{-7pt}

\hfill{\small\textit{Received October 1, 2019}}

%\pagebreak

\vspace*{-22pt}
  \Contr
  
  \noindent
  \textbf{Goncharov Alexander A.} (b.\ 1994)~--- junior scientist, Institute of 
Informatics Problems, Federal Research Center ``Computer Science and Control'' 
of the Russian Academy of Sciences, 44-2~Vavilov Str., Moscow 119333, Russian 
Federation; \mbox{a.gonch48@gmail.com}
  
  \vspace*{3pt}
  
  \noindent
  \textbf{Zatsman Igor M.} (b.\ 1952)~--- Doctor of Science in technology, Head 
of Department, Institute of Informatics Problems, Federal Research Center 
``Computer Science and Control'' of the Russian Academy of Sciences,  
44-2~Vavilov Str., Moscow 119333, Russian Federation; 
\mbox{izatsman@yandex.ru}
  
  \vspace*{3pt}
  
  \noindent
  \textbf{Kruzhkov Mikhail G.} (b.\ 1975)~--- senior scientist, Institute of 
Informatics Problems, Federal Research Center ``Computer Science and Control'' 
of the Russian Academy of Sciences, 44-2~Vavilov Str., Moscow 119333, Russian 
Federation; \mbox{magnit75@yandex.ru}

\label{end\stat}

\renewcommand{\bibname}{\protect\rm Литература}  