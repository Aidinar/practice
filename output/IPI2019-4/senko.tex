\def\stat{senko}

\def\tit{ИССЛЕДОВАНИЕ ВОЗМОЖНОСТИ ПРОГНОЗИРОВАНИЯ ИЗМЕНЕНИЯ ФИНАНСОВОГО 
СОСТОЯНИЯ КРЕДИТНОЙ ОРГАНИЗАЦИИ НА~ОСНОВЕ ПУБЛИКУЕМОЙ ОТЧЕТНОСТИ$^*$}

\def\titkol{Исследование возможности прогнозирования изменения финансового 
состояния кредитной организации} % на основе публикуемой отчетности}

\def\aut{Ю.\,И.~Журавлев$^1$, О.\,В.~Сенько$^2$, Н.\,Н.~Бондаренко$^3$, 
В.\,В.~Рязанов$^4$, А.\,А.~Докукин$^5$, А.\,П.~Виноградов$^6$}

\def\autkol{Ю.\,И.~Журавлев, О.\,В.~Сенько, Н.\,Н.~Бондаренко и~др.}


\titel{\tit}{\aut}{\autkol}{\titkol}

\index{Журавлев Ю.\,И.}
\index{Сенько О.\,В.}
\index{Бондаренко Н.\,Н.}
\index{Рязанов В.\,В.}
\index{Докукин А.\,А.}
\index{Виноградов А.\,П.}
\index{Zhuravlev Yu.\,I.}
\index{Sen'ko O.\,V.} 
\index{Bondarenko N.\,N.}
\index{Ryazanov V.\,V.}
\index{Dokukin A.\,A.}
\index{Vinogradov A.\,P.}


{\renewcommand{\thefootnote}{\fnsymbol{footnote}} \footnotetext[1]
{Работа выполнена при частичной финансовой поддержке РФФИ (проект 18-29-03151).}}


\renewcommand{\thefootnote}{\arabic{footnote}}
\footnotetext[1]{Федеральный исследовательский центр <<Информатика и~управ\-ле\-ние>>
  Российской академии наук; 
Московский государственный университет им.\ М.\,В.~Ломоносова, \mbox{zhur@ccas.ru}}
\footnotetext[2]{Федеральный исследовательский центр <<Информатика и~управ\-ле\-ние>>
  Российской академии наук, \mbox{senkoov@mail.ru}}
\footnotetext[3]{Московский государственный университет им.\ М.\,В.~Ломоносова, \mbox{kolianmos1@gmail.com}}
\footnotetext[4]{Федеральный исследовательский центр <<Информатика и~управ\-ле\-ние>>
  Российской академии наук, 
\mbox{rvvccas@mail.ru}}
\footnotetext[5]{Федеральный исследовательский центр <<Информатика и~управ\-ле\-ние>>
  Российской академии наук, 
\mbox{dalex@ccas.ru}}
\footnotetext[6]{Федеральный исследовательский центр <<Информатика и~управ\-ле\-ние>>
  Российской академии наук, 
\mbox{vngrccas@mail.ru}}

\vspace*{-3pt}

   
       
       \Abst{Рассматривается математическая модель для прогноза отзыва лицензии 
кредитной организации на период до 6~месяцев по данным из публикуемой отчетности 
кредитных организаций. Модель является коллективным решением по набору 
ком\-би\-на\-тор\-но-ло\-ги\-че\-ских методов распознавания и~ре\-ша\-ющих лесов различного типа. 
Оценка эффективности разработанной коллективной модели по показателю ROC AUC 
(area under receiver operating characteristic curve)
составила~0,74. Модель позволяет выделять группы кредитных организаций 
с~повышенным и~пониженным риском отзыва лицензии. Было проведено ранжирование 
различных показателей, показавшее важность величины ликвидных и~высоколиквидных 
активов.}
        
       \KW{прогнозирование; коллективные методы; финансовое состояние; кредитная 
организация}

\DOI{10.14357/19922264190405} 
  
\vspace*{-3pt}


\vskip 10pt plus 9pt minus 6pt

\thispagestyle{headings}

\begin{multicols}{2}

\label{st\stat}
     
      
\section{Введение }

Банковский сектор представляет собой крупнейшую по объему средств часть 
финансового рынка России, и~трудно переоценить важность деятельности 
кредитных организаций для экономики\linebreak страны в~целом и~отдельных 
компаний и~граждан в~частности. Банком России на протяжении последних 
лет проводилась работа по выводу с~рынка недобросовестных и~финансово 
несостоятельных участников: у кредитных организаций отзывались\linebreak лицензии 
на осуществление банковских операций или проводились мероприятия по 
финансовому оздо\-ров\-ле\-нию с~приостановкой полномочий бывших 
собственников и~руководства. Как правило, данные события были связаны 
с~ухудшением финансового состояния кредитной организации. Однако для 
клиентов банков момент наступления\linebreak такого негативного события мог стать 
полной неожиданностью. 

Естественно, что деятельность банковского %\linebreak 
надзора, связанная с~банковской тайной и~информацией ограниченного 
доступа, является не\-пуб\-лич\-ной. Поэтому невозможно заранее сказать, 
безопас\-но ли хранить средства (в~размере, превышающем объем страховой 
ответственности Государственной корпорации
<<Агентство по страхованию вкладов>>) в~данном банке или нет. Однако существуют 
раз\-личные методики, в~частности у кредитных рейтинговых агентств, 
позволяющие на основе количественных и~качественных показателей 
оценить вероятность наступления дефолта кредитной организации на 
определенном горизонте. Следует отметить, что отзыв лицензии может 
произойти и~без наступления дефолта кредитной организации. В~связи 
с~этим возникает потребность исследования возможности прогнозирования 
ухудшения финансового состояния кредитной организации заранее.

 Для 
решения этой задачи объективно могут использо\-вать\-ся средства машинного 
обучения. Их очевидными преимуществами по сравнению с~экспертными 
оценками являются объ\-ек\-тив\-ность, универсаль\-ность, относительно низкая 
сто\-и\-мость, воз\-мож\-ность быст\-рой коррекции алгоритмов прогнозирования по 
мере поступления новой информации. 

Перечисленные преимущества не 
могут не привлекать внимание исследователей и~вызывают появление работ 
по тематике использования методов компьютерного обучения для решения 
задач прогнозирования в~банковской сфере~[1, 2]. 

      В настоящее время существует большое число технологий обучения, 
а~также средств статистически корректной оценки эффективности 
полученных решений~[3, 4]. Большое значение имеют также сопутствующие 
обучению методы ранжирования показателей по их значимости при 
прогнозировании. Такое разнообразие технологий вызывает необходимость 
исследования как их эф\-фек\-тив\-ности по отдельности, так и~выбора 
оптимальной схемы построения коллективных решений~\cite{3-sen}.
      
\section{Использование для~прогноза методов распознавания }

 Сформулируем поставленную выше задачу прогнозирования ухудшения 
состояния кредитных организа\-ций как задачу предсказания отзыва лицензии 
по данным из публикуемой отчетности кредитных организаций. 

Имеется 
набор объектов (кредитных организаций) с~признаковым описанием 
(показатели отчетности\footnote{Отдельные показатели деятельности 
кредитной организации, используемые для расчета обязательных нормативов 
из разд.~2 отчетности банков по форме 0409135 <<Информация об 
обязательных нормативах>>, составляемой в~соответствии с~Указанием 
Банка России от 24.11.2016 №\,4212-У <<О~перечне, формах и~порядке 
составления и~представления форм отчетности кредитных организаций 
в~Центральный банк Российской Федерации>>.}, с~ежемесячной 
периодичностью раз\-ме\-ща\-емые на сайте Банка России\footnote{\sf 
www.cbr.ru.}), для которых\linebreak
 известно значение бинарного признака~--- будет 
ли негативное событие отзыва лицензии у~кредитной организации в~течение 
ближайших 6~месяцев или нет (со\-став\-лен\-ное по информации  
пресс-ре\-ли\-зов с~сайта Банка России). Задача заключается 
в~прогнозировании значения функции, зависящей от признакового описания 
объекта и~принимающей бинарное значений <<да>> или <<нет>> на 
ана\-ли\-зи\-ру\-емую дату с~горизонтом прогнозирования 6~месяцев 
в~зависимости от данных отчетности банка. 

Важно отметить, что причиной 
отзыва лицензии могут стать не только проблемы, связанные с~финансовым 
состоянием кредитной организации, но и~нарушения в~об\-ласти ПОД/ФТ
(противодействия отмыванию доходов и~финансированию
терроризма). 
В~связи с~этим из обучающей выборки были исключены кредитные 
организации, у~которых лицензии были отозваны в~связи с~указанными 
нарушениями.

      Был проведен эксперимент по оценке возможности прогнозирования 
отзыва лицензии, связанного с~финансовым состоянием банка. Прогноз 
проводился на период с~января по июнь 2015~г. В~анализ были включены 
24~банка, для которых отзыв лицензии был осуществлен в~указанный 
период. При этом по экспертной оценке сотрудников Банка России отзыв 
лицензии для этих банков определенно был связан с~их финансовым 
со\-сто\-яни\-ем. Также в~анализ были включены 692~банка, которые продолжали 
действовать без отзыва лицензии в~течение 2~лет с~декабря 2014~г. Для 
прогнозирования использовались параметры банковской отчетности, 
известные на момент времени, в~который производился прогноз. Всего 
в~анализе использовался 31~уникальный показатель банковской отчетности. 
Однако при прогнозировании применялись банковские показатели, 
рассчитанные для месяца, предшествующего дате составления прогноза, 
а~также для месяцев, отстоящих от даты прогноза на~1, 2 и~3 месячных 
интервала. Таким образом, общее число используемых для прогноза 
показателей составило~124.
      
      Задача прогнозирования, очевидно, может быть сведена к~задаче 
распознавания с~двумя классами. При этом задача усложняется малым 
размером целевого класса, а также высокой размерностью данных, связанной 
с~необходимостью учета динамики показателей финансового состояния 
банков.\linebreak Высокая эффективность в~этих условиях может достигать\-ся при 
использовании коллективных решений. Существенным требованием является 
необходимость анализа информативности различных показателей. Данная 
задача осложняется тем, что информативность показателей проявляется 
только в~рамках их взаимодействия. Можно предположить, что для 
повышения достоверности могут быть использованы коллективные методы 
оценивания информативности. Разработка таких методов также является 
одной из целей представляемого исследования.
      
      Для вычисления оптимальных прогнозных решений был 
протестирован ряд разнообразных технологий распознавания. Однако 
возможность получе\-ния эффективного алгоритма прогнозирования удалось 
показать только для следующих методов:
      \begin{itemize}
      \item логистическая регрессия (LogReg);
      \item алгоритм вычисления оценок с~использованием всевозможных 
наборов признаков в~качестве опорных множеств (АВО)~\cite{3-sen, 5-sen};
      \item решающий лес, использующий бэггинг для генерации ансамблей 
деревьев (RF)~\cite{4-sen};
      \item решающий лес, основанный на процедуре адап\-тив\-но\-го бустинга 
(АdaBoost)~\cite{4-sen, 7-sen};
      \item решающий лес, основанный на процедуре градиентного 
бустинга (GradBoost)~\cite{4-sen, 8-sen};
      \item метод статистически взвешенных синдромов  
(СВС)~\cite{3-sen, 6-sen}.
      \end{itemize}
      
      Оценка точности прогноза проводилась с~использованием метода 
кросс-валидации со 100~фолдами. Для оценивания результатов была 
использована известная метрика ROC AUC. Результаты приведены в~табл.~1.

  
      
%\begin{table*}
{ %\small %tabl1
\begin{center}

\parbox{50mm}{{{\tablename~1}\ \ \small{Оценка точности прогноза различными методами}}

}

\vspace*{6pt}

\small 
\begin{tabular}{|l|c|c|}
\hline
\multicolumn{1}{|c|}{Метод}& ROC AUC & Ранг\\
\hline
LogReg& 0,626& 1\\
RF& 0,707& 5\\
GradBoost& 0,662& 2\\
АdaBoost& 0,716& 6\\
АВО& 0,679& 3\\
СВС& 0,698& 4\\
\hline
\end{tabular}
\end{center}
}
%\end{table*}
      
\section{Коллективное решение }

      На основе полученных данных строилось коллективное решение. На 
первом этапе проводился поиск оптимального порога~$b_*$в решающем 
правиле для каждого из 6~алгоритмов. Подбор порога проводился из условия 
минимальности различия между чувствительностью и~специфичностью. 
Коллективная оценка объекта, описываемая вектором 
признаков~$\mathbf{x}_j$, вычислялась в~два этапа. На первом этапе по 
оценке $\gamma_*(\mathbf{x}_j)$, полученной с~помощью алгоритма~$A_*$, 
вычислялось значение бинарного показателя~$\beta_*$, указывающего на 
негативный прогноз для объекта~$\mathbf{x}_j$ при 
$\beta_*(\mathbf{x}_j)\hm=1$ и~на положительный прогноз при 
$\beta_*(\mathbf{x}_j)\hm=0$. Вычисление~$\beta_*$ проводилось по схеме: 
$$
\beta_*(\mathbf{x}_j)=
\begin{cases}
1 & \mbox{при } \gamma_*(\mathbf{x}_j)>b_*\,;\\
0 & \mbox{в~противном~случае}.
\end{cases}
$$

 Каждому из алгоритмов 
сопоставлялся весовой коэффициент~$\theta_*$. Рассматривался следующий 
способ задания весовых коэффициентов: алгоритмы ранжировались по 
величине ROC AUC. Значение коэффициента~$\theta_*$ для 
алгоритма~$A_*$ приравнивалось рангу~$A_*$ из табл.~1. Коллективная 
оценка $\gamma_{\mathrm{int}}(\mathbf{x}_j)$ вычислялась по формуле:
      \begin{multline*}
      \gamma_{\mathrm{int}}(\mathbf{x}_j) =
      \theta_{\mathrm{свс}} \beta_{\mathrm{свс}} 
(\mathbf{x}_j) +\theta_{\mathrm{GB}}\beta_{\mathrm{GB}}(\mathbf{x}_j) +{}\\
{}+\theta_{\mathrm{AB}}  \beta_{\mathrm{AB}} (\mathbf{x}_j) +
\theta_{\mathrm{RF}}  \beta_{\mathrm{RF}} (\mathbf{x}_j) +
\theta_{\mathrm{LR}}\beta_{\mathrm{LR}}(\mathbf{x}_j)+{}\\
{}+
\theta_{\mathrm{ABO}}\beta_{\mathrm{ABO}}  (\mathbf{x}_j)\,.
      \end{multline*}
      
      %\begin{table*}
{%tabl2
\begin{center}
%\vspace*{1pt}

\parbox{70mm}{{{\tablename~2}\ \ \small{Уровень риска отзыва лицензии для различных интервалов балльных оценок}}

}

\vspace*{6pt}

\small
\begin{tabular}{|c|c|c|}
\hline
\multicolumn{1}{|c|}{\raisebox{-6pt}[0pt][0pt]{Баллы}}& \multicolumn{2}{c|}{Количество отозванных лицензий}\\
\cline{2-3}
&\hspace*{10mm}шт.\hspace*{10mm} &\%\\
\hline
$\geq 20$& 5 из 19\hphantom{9} & 26,3\hphantom{9}\\
От 17 до 19& 12 из 63\hphantom{99} & 19\hphantom{99,}\\
От 3 до 16 & 9 из 401& 2,2\\
$<3$& 3 из 252 & 1,2\\
\hline
\end{tabular}
\end{center}}
%\end{table*}
\vspace*{12pt}

\noindent
Величина ROC AUC для коллективного решения составила~0,74. 
     
     Коллективные оценки риска оценивались по шкале от~0 до~21~балла, 
где 21~балл соответствовал негативному прогнозу, а~0~баллов 
соответствовали позитивному прогнозу. Из табл.~2 видно, что из 19~банков 
с~21 баллом лицензия была отозвана у~5, что составляет~26,3\%. В~группе 
из 252~банков с~менее чем тремя баллами лицензия была отозвана только 
у~трех банков (1,2\%). Таким образом, коллективное решение отчетливо 
выделяет группы с~пониженным и~повышенным риском отзыва лицензии. 


      
      Для оценивания информативности признаков использовались 
коллективные оценки ин\-фор\-ма\-тив\-ности, включающие оценки, полученные %\linebreak 
с~помощью всех трех используемых вариантов ре\-ша\-ющих лесов, а~также 
метода СВС. 

В~методах {решающих} лесов информативность признаков 
вычисляется как среднее значение показателей информативности, 
рассчитанных для отдельных %\linebreak 
деревьев и~характеризующих улучшение 
аппроксимации данных после включения признака в~модель~\cite{9-sen}. 

В~методе СВС показателем информативности служит значение 
статистики~$\chi^2$ при сравнении распределения целевого класса в~группах 
слева и~справа от рассчитанного для признака оптимального  
порога~\cite{6-sen}.
      
      Таким образом, информативность признаков рассчитывалась отдельно 
для решающих лесов, основанных на бэггинге, адаптивном или градиентном 
бустинге, а также для СВС. Обозначим показатели информативности по этим 
методам соответственно как $I_{\mathrm{RF}}$, $I_{\mathrm{AB}}$, $I_{\mathrm{GB}}$ 
и~$I_{\mathrm{свс}}$. Далее проводилось ранжирование признаков по 
величине каждого из четырех перечисленных показателей. Ранги 
по~$I_{\mathrm{RF}}$, $I_{\mathrm{AB}}$, $I_{\mathrm{GB}}$ и~$I_{\mathrm{свс}}$ обозначим 
как~$R_{\mathrm{RF}}$, $R_{\mathrm{AB}}$, $R_{\mathrm{GB}}$ и~$R_{\mathrm{свс}}$.
      
      Интегральный показатель информативности для $I_{\mathrm{int}}(X_j)$ 
признака~$X_j$ вычислялся как сумма рангов по каждому из четырех 
показателей информативности: 
\begin{multline*}
I_{\mathrm{int}}\left(X_i\right)= {}\\
{}=
R_{\mathrm{свс}}\left(X_i\right)+ R_{\mathrm{GB}}\left(X_i\right)+ 
R_{\mathrm{AB}}\left(X_i\right)+R_{\mathrm{RF}}\left(X_i\right)\,.
\end{multline*}
      
      Признаки, имеющие ранги от одного до пяти, приведены в~табл.~3.
      
\setcounter{table}{2}
\begin{table*}\small %tabl3
\begin{center}
\Caption{Наиболее информативные показатели}
\vspace*{2ex}

\begin{tabular}{|c|c|p{115mm}|}
\hline
Ранг&\tabcolsep=0pt\begin{tabular}{c}Интегральный\\ 
показатель\\ информативности\end{tabular}&\multicolumn{1}{c|}{Показатель 
банковской отчетности}\\
\hline
1&\hphantom{9}4&Высоколиквидные активы за месяц, предшествующий моменту прогноза\\
\hline
2&18&Ликвидные активы за месяц, предшествующий моменту прогноза\\
\hline
\multicolumn{1}{|c|}{\raisebox{-6pt}[0pt][0pt]{3}} &
\multicolumn{1}{c|}{\raisebox{-6pt}[0pt][0pt]{19}}&Активы II группы, взвешенные с~коэффициентом 20\% (мера риска), за месяц, 
отстоящий на один месячный интервал от даты прогноза\\
\hline
\multicolumn{1}{|c|}{\raisebox{-6pt}[0pt][0pt]{4}}&
\multicolumn{1}{c|}{\raisebox{-6pt}[0pt][0pt]{40}}&Активы, имеющие нулевой коэффициент риска за месяц, отстоящий на четыре 
месячных интервала от даты прогноза\\
\hline
\multicolumn{1}{|c|}{\raisebox{-6pt}[0pt][0pt]{5}}&
\multicolumn{1}{c|}{\raisebox{-6pt}[0pt][0pt]{49}}&Активы II группы, взвешенные с~коэффициентом 20\% (мера риска) за месяц, 
отстоящий на три месячных интервала от даты прогноза\\
\hline
\end{tabular}
\end{center}
\vspace*{-3pt}
\end{table*}

\vspace*{-6pt}
      
\section{Заключение}

\vspace*{-2pt}

      Как видно из полученных результатов, разработанная коллективная 
модель по группе методов распознавания по данным отчетности позволяет 
с~некоторой долей уверенности предсказать отзыв лицензии у кредитной 
организации. Возможность эффективного прогнозирования можно связать 
с~проводимой Банком России работой над достоверностью отчетности 
участников финансового рынка. Разработанная методика может быть полезна 
как для банковского надзора, так и~для участников финансового рынка. 
Однако для повышения точности прогнозирования негативных событий, 
безусловно, недостаточно следить только за значениями показателей 
отчетности: важно использовать механизм риск-ана\-ли\-ти\-ки, а~также 
формировать доверительную среду на финансовом рынке.

\vspace*{-6pt}
      
   {\small\frenchspacing
 {%\baselineskip=10.8pt
 \addcontentsline{toc}{section}{References}
 \begin{thebibliography}{9}
 
 \vspace*{-2pt}
 
 
\bibitem{1-sen}
\Au{Ясницкий Л.\,Н., Иванов~Д.\,В., Липатова~Е.\,В.} Нейросетевая система оценки 
вероятности банкротства банков~// Биз\-нес-ин\-фор\-ма\-ти\-ка, 2014.  Т.~3. №\,29. 
С.~49--56.
{\looseness=1

}
\bibitem{2-sen}
\Au{Синельникова-Мурылева Е.\,В., Горшкова~Т.\,Г., Ма\-ке\-ева~Н.\,В.} Прогнозирование 
дефолтов в~российском банковском секторе~// Экономическая политика, 2018. Т.~2. 
№\,13. С.~8--27.
\bibitem{3-sen}
\Au{Журавлев Ю.\,И., Рязанов~В.\,В., Сенько~О.\,В.} Распознавание: Математические 
методы. Программная система. Применения.~--- M.: Фазис, 2006. 159~c.
\bibitem{4-sen}
\Au{Hastie T., Tibshirani~R., Friedman~J.} The elements of statistical learning: Data 
mining, inference, and prediction.~--- Springer, 2009. 745~p.
\bibitem{5-sen}
\Au{Журавлев Ю.\,И.} Об алгебраическом подходе к~решению задач распознавания или 
классификации~// Проб\-ле\-мы кибернетики, 1978. №\,33. С.~5--68.

\bibitem{7-sen} %6
\Au{Freund Y., Schapire~R.} A~decision-theoretic generalization of on-line learning and an 
application to boosting~// J.~Comput. Syst. Sci., 1997. Vol.~55. P.~119--139.
\bibitem{8-sen} %7
\Au{Friedman J.} Greedy function approximation: A~gradient boosting machine~//  
Ann. Stat., 2001. Vol.~5. Iss.~29. P.~1189--1232.

\bibitem{6-sen} %8
\Au{Кузнецов В.\,А., Сенько~О.\,В., Кузнецова~А.\,В. и~др.} Распознавание нечетких 
систем по методу статистически взвешенных синдромов и~его применение для 
иммуногематологической нормы и~хронической патологии~// Хим. физика, 
1996. Т.~15. №\,1. С.~81--100.

\bibitem{9-sen}
\Au{Louppe G.} Understanding random forests: From theory to practice.~--- 
Liege: University of Liege, 2014.  PhD Thesis. 223~p.
 \end{thebibliography}

 }
 }

\end{multicols}

\vspace*{-6pt}

\hfill{\small\textit{Поступила в~редакцию 04.02.19}}

%\vspace*{8pt}

%\pagebreak

\newpage

\vspace*{-28pt}

%\hrule

%\vspace*{2pt}

%\hrule

%\vspace*{-2pt}

\def\tit{RESEARCH OF~THE~POSSIBILITY TO~FORECAST CHANGES 
IN~FINANCIAL STATE OF~A~CREDIT ORGANIZATION\\ ON~THE~BASIS 
OF~PUBLIC FINANCIAL STATEMENTS}


\def\titkol{Research of~the~possibility to~forecast changes in 
financial state of a credit organization on~the~basis 
of~public financial statements}

\def\aut{Yu.\,I.~Zhuravlev$^{1,2}$, O.\,V.~Sen'ko$^1$, 
N.\,N.~Bondarenko$^2$, V.\,V.~Ryazanov$^1$, A.\,A.~Dokukin$^1$, 
and~A.\,P.~Vinogradov$^1$}

\def\autkol{Yu.\,I.~Zhuravlev, O.\,V.~Sen'ko, 
N.\,N.~Bondarenko, et al.}
%V.\,V.~Ryazanov$^1$, A.\,A.~Dokukin$^1$, 
%and~A.\,P.~Vinogradov$^1$}

\titel{\tit}{\aut}{\autkol}{\titkol}

\vspace*{-11pt}



       \noindent
      $^1$Federal Research Center ``Computer Science and Control'' of the 
Russian Academy of Sciences, 44-2~Vavilov\linebreak
$\hphantom{^1}$Str., Moscow 119333, Russian 
            Federation
        
        \noindent
        $^2$M.\,V.~Lomonosov Moscow State University, 1-52~Leninskie Gory, GSP-1, 
Moscow 119991, Russian Federation

\def\leftfootline{\small{\textbf{\thepage}
\hfill INFORMATIKA I EE PRIMENENIYA~--- INFORMATICS AND
APPLICATIONS\ \ \ 2019\ \ \ volume~13\ \ \ issue\ 4}
}%
 \def\rightfootline{\small{INFORMATIKA I EE PRIMENENIYA~---
INFORMATICS AND APPLICATIONS\ \ \ 2019\ \ \ volume~13\ \ \ issue\ 4
\hfill \textbf{\thepage}}}

\vspace*{3pt} 


      
      \Abste{The mathematical model for forecasting of license revocation of 
a~credit organization in the 6-month period based on public financial statements 
is considered. The model represents an ensemble of combinatorial and logical 
methods and decision trees of different types. Its effectiveness estimated by ROC 
AUC 
(area under receiver operating characteristic curve)
is~0.74. The model allows distinguishing groups of credit organizations with 
higher and lower license revocation risks. Also, the ranking of different financial 
statement indicators has been performed which marked the importance of liquid 
and highly liquid assets.}
      
      \KWE{forecasting; algorithm ensembles; financial state; credit 
organization}
      
      
      
       \DOI{10.14357/19922264190405} 

%\vspace*{-14pt}

 \Ack
      \noindent
       The research has been carried out with the partial financial support of the 
Russian Foundation for Basic Research (project 18-29-03151).



%\vspace*{-6pt}

  \begin{multicols}{2}

\renewcommand{\bibname}{\protect\rmfamily References}
%\renewcommand{\bibname}{\large\protect\rm References}

{\small\frenchspacing
 {%\baselineskip=10.8pt
 \addcontentsline{toc}{section}{References}
 \begin{thebibliography}{9}
\bibitem{1-sen-1}
\Aue{Yasnitskiy, L.\,N., D.\,V.~Ivanov, and E.\,V.~Lipatova.} 2014. Neyrosetevaya  
sistema otsenki veroyatnosti bankrotstva bankov [Neural network designed 
to estimate probability of bank bankruptcies]. \textit{Biznes-informatika} [Business Informatics] 
3(29):49--56.
\bibitem{2-sen-1}
\Aue{Sinel'nikova-Muryleva, E.\,V., T.\,G.~Gorshkova, and N.\,V.~Makeeva}. 2018. 
Prognozirovanie defoltov v~rossiyskom bankovskom sektore [Default 
forecasting in the Russian banking sector]. \textit{Ekonomicheskaya politika}
[Economic Policy] 2(13):8--27.
\bibitem{3-sen-1}
\Aue{Zhuravlev, Yu.\,I., V.\,V.~Ryazanov, and O.\,V.~Sen'ko}. 2006. \textit{Raspoznavanie. 
Matematicheskie metody. Programmnaya sistema. Primeneniya} 
[Recognition. Mathematical methods. Software system. Applications].
Moscow: Fazis. 159~p.
\bibitem{4-sen-1}
\Aue{Hastie, T., R.~Tibshirani, and J.~Friedman.} 2009. 
\textit{The elements of statistical 
learning: Data mining, inference, and prediction}. Springer. 745~p.
\bibitem{5-sen-1}
\Aue{Zhuravlev, Yu.\,I.} 1978. Ob algebraicheskom podkhode k~re\-she\-niyu zadach 
raspoznavaniya ili klassifikatsii [On algebraic approach to recognition and 
classification problems]. \textit{Problemy kibernetiki} [Cybernetic Problems] 
33:5--68.

\bibitem{7-sen-1} %6
\Aue{Freund, Y., and R.~Schapire}. 1997. A~decision-theoretic generalization of 
on-line learning and an application to boosting. \textit{J.~Comput. Syst. Sci.} 
55:119--139.
\bibitem{8-sen-1} %7
\Aue{Friedman, J.} 2001. Greedy function approximation: A~gradient boosting 
machine. \textit{Ann. Stat.} 5(29):1189--1232.

\bibitem{6-sen-1} %8
\Aue{Kuznetsov, V.\,A., O.\,V.~Sen'ko, A.\,V.~Kuznetsova, \textit{et al.}} 1996. 
Recognition of fuzzy systems by the method of 
statistically weighed syndromes and its application to immunohematological 
characterization of the
norm and chronical pathology]. \textit{Chem. Phys. Rep.} 
15(1):87--107.

\bibitem{9-sen-1}
\Aue{Louppe, G.} 2014. Understanding random forests: From theory to 
practice.  Liege: University of Liege.  PhD Thesis. 223~p.
 \end{thebibliography}

 }
 }

\end{multicols}

%\vspace*{-7pt}

\hfill{\small\textit{Received February 4, 2019}}

%\pagebreak

%\vspace*{-22pt}

\Contr


\noindent
\textbf{Zhuravlev Yuri  I.} (b.\ 1935)~--- Doctor of Science in physics and 
mathematics, Academician of RAS, principal scientist, Federal Research Center 
``Computer Science and Control'' of the Russian Academy of Sciences,  
44-2~Vavilov Str., Moscow 119333, Russian Federation; honorary professor, head 
of the Mathematical Methods of Forecasting Department, M.\,V.~Lomonosov 
Moscow State University, 1-52~Leninskie Gory, GSP-1, Moscow 119991, 
Russian Federation; \mbox{zhur@ccas.ru}

\vspace*{3pt}

\noindent
\textbf{Sen'ko Oleg V.} (b.\ 1957)~--- Doctor of Science in physics and 
mathematics, leading scientist, Federal Research Center 
``Computer Science and Control'' of the Russian Academy of Sciences,  
44-2~Vavilov Str., Moscow 119333, Russian Federation; 
\mbox{senkoov@mail.ru}

\vspace*{3pt}

\noindent
\textbf{Bondarenko Nikolaj N.} (b.\ 1990)~--- postgraduate student, 
M.\,V.~Lomonosov Moscow State University, \mbox{1-52}~Leninskie Gory, GSP-1, 
Moscow 119991, Russian Federation; \mbox{kolianmos1@gmail.com} 

\vspace*{3pt}

\noindent
\textbf{Ryazanov Vladimir V.} (b.\ 1950)~--- Doctor of Science in physics and 
mathematics, principal scientist, Federal Research Center 
``Computer Science and Control'' of the Russian Academy of Sciences,  
44-2~Vavilov Str., Moscow 119333, Russian Federation; 
\mbox{rvvccas@mail.ru}

\vspace*{3pt}

\noindent
\textbf{Dokukin Alexander A.} (b.\ 1980)~--- Candidate of Science (PhD) in 
physics and mathematics, senior scientist, Federal Research Center ``Computer 
Science and Control'' of the Russian Academy of Sciences, 44-2~Vavilov Str., 
Moscow 119333, Russian Federation; \mbox{dalex@ccas.ru}


\vspace*{3pt}

\noindent
\textbf{Vinogradov Alexander P.}  (b.\ 1951)~--- Candidate of Science (PhD) in 
physics and mathematics, senior scientist, Federal Research Center ``Computer 
Science and Control'' of the Russian Academy of Sciences, 44-2~Vavilov Str., 
Moscow 119333, Russian Federation; \mbox{vngrccas@mail.ru}


\label{end\stat}

\renewcommand{\bibname}{\protect\rm Литература}  

      