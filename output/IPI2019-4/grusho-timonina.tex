\def\stat{gr+timon}

\def\tit{ИСПОЛЬЗОВАНИЕ МЕТАДАННЫХ ДЛЯ~РЕАЛИЗАЦИИ ТРЕБОВАНИЙ ПОЛИТИКИ 
БЕЗОПАСНОСТИ MLS$^*$}

\def\titkol{Использование метаданных для~реализации требований политики 
безопасности MLS}

\def\aut{А.\,А.~Грушо$^1$,  Н.\,А.~Грушо$^2$, 
Е.\,Е.~Тимонина$^3$}

\def\autkol{А.\,А.~Грушо, Н.\,А.~Грушо, 
Е.\,Е.~Тимонина}

\titel{\tit}{\aut}{\autkol}{\titkol}

\index{Грушо А.\,А.}
\index{Грушо Н.\,А.} 
\index{Тимонина Е.\,Е.}
\index{Grusho A.\,A.}
\index{Grusho N.\,A.}
\index{Timonina E.\,E.}



{\renewcommand{\thefootnote}{\fnsymbol{footnote}} \footnotetext[1]
{Работа частично поддержана РФФИ (проект 18-07-00274).}}


\renewcommand{\thefootnote}{\arabic{footnote}}
\footnotetext[1]{Институт проблем информатики Федерального исследовательского центра <<Информатика и~управление>> 
Российской академии наук, \mbox{grusho@yandex.ru}}
\footnotetext[2]{Институт проблем информатики Федерального исследовательского центра <<Информатика и~управление>> 
Российской академии наук, info@itake.ru}
\footnotetext[3]{Институт проблем информатики Федерального исследовательского центра <<Информатика и~управление>> 
Российской академии наук, eltimon@yandex.ru}

\vspace*{2pt}

   


  \Abst{Рассматривается распределенная информационная система, объекты которой содержат 
как ценную информацию (или сами являются ценными), так и~открытую (не ценную) 
информацию. Для защиты ценной информации используется политика безопасности (ПБ) MLS
(Multilevel Security), 
которая запрещает информационные потоки от объектов с~ценной информацией к~объектам 
с~открытой информацией. Объекты с~ценной информацией образуют класс объектов уровня High, 
а~объекты c~открытой информацией образуют класс объектов уровня Low. 
  Метаданные (МД) создаются для управления соединениями в~сетях. Метаданные являются 
упрощением математических моделей биз\-нес-про\-цес\-сов и~служат основой разрешительной 
системы для соединений хостов в~распределенной 
ин\-фор\-ма\-ци\-он\-но-вы\-чис\-ли\-тель\-ной сис\-те\-ме
(РИВС).
  В~работе сформулированы правила ПБ MLS и~на основе 
инфраструктуры, связанной с~МД, показана возможность реализации этой 
ПБ в~РИВС. 
Единственный доверенный процесс, необходимый для реализации ПБ MLS, 
функционирует на уровне управления соединениями. Этот уровень не связан с~плос\-костью 
передачи данных и~может быть изолирован с~\mbox{целью} обеспечения его информационной 
безопасности.}
  
  \KW{политика безопасности MLS; информационные потоки; метаданные}
  
  \DOI{10.14357/19922264190414} 
  
%\vspace*{1pt}


\vskip 10pt plus 9pt minus 6pt

\thispagestyle{headings}

\begin{multicols}{2}

\label{st\stat}

\section{Введение}

  Политика безопасности в~компьютерной системе и~сети~--- это набор 
требований по ограничению доступа, хранению и~распределению 
информации~[1]. Обычно ПБ определяет требования по защите 
конфиденциальности, целостности и~доступности информации. Политика
безопас\-ности опирается на 
четкую классификацию ценных информационных ресурсов и~открытых 
информационных ресурсов. 
  
  Один из общих подходов к~описанию требований информационной 
безопасности к~конкретному информационному ресурсу~--- это неотделимая 
привязка к~информационному объекту вектора характеристик, определяющих 
обращение с~этой информацией. Каждый такой вектор содержит результаты 
классификации объекта, а~именно: требования по конфиденциальности, 
целостности и~доступности информации. 
  
  Для простоты будем классифицировать информацию как конфиденциальную 
  и~как открытую. Объекты, содержащие конфиденциальную информацию (ценную 
информацию), будем помечать символом~$(*)$. Это ограничение не умаляет 
общности, так как легко обобщается на требования защиты целостности 
и~доступности. 
  
  Для защиты конфиденциальности широко используется политика 
  MLS~[1], которая запрещает информационные потоки от 
объектов~$(*)$ к~объектам, не помеченным~$(*)$. Объекты с~меткой~$(*)$ 
образуют класс объектов уровня High, а объекты без метки образуют класс 
объектов уровня Low. 
  
  Метаданные создаются для управления соединениями в~сетях~[2]. Как 
правило, МД являются упрощением математических моделей  
биз\-нес-про\-цес\-сов и~служат основой разрешительной системы для соединений 
хостов в~РИВС. Разрешительная система строится на основе порядка взаимодействий 
задач, реализующих бизнес-процесс. Для ее работы вводятся две специальные 
задачи~$\mathcal{M}$ и~$\mathcal{N}$. Задача~$\mathcal{M}$ распределяет 
задачи по хостам, т.\,е.\ определяет бинарное отношение $H(A)$, где~$A$~--- 
задача, а~$H$~--- хост сети. Задача~$\mathcal{N}$ реализует разрешительную 
систему, которая разрешает и~организует соединения хостов $H(A)$ и~$H(B)$, 
если в~МД отражена необходимость инициализации или взаимодействия 
задач~$A$ и~$B$. 
  
  Как было показано в~[3, 4], МД не несут информации о значениях данных. 
Поэтому требования ПБ в~МД отражаются косвенно, т.\,е.\ МД могут содержать 
только информацию о том, куда нельзя направлять информационный поток. 
  
  Как правило, информационная технология (ИТ) представима в~виде составной 
задачи~[5], а~так\-же может описываться DAG (Directed Acyclic Graph)~[6], 
в~котором вершины~--- это задачи ИТ, а~дуги указывают направления 
информационных потоков, передающих исходные данные сле\-ду\-ющим задачам ИТ. 
В~таком представлении возможно существование дуг извне DAG как некоторых 
внешних информационных потоков с~исходными данными и~дуг, выходящих из 
DAG, но не входящих в~задачи ИТ (внешнее распределение информации). 
Передача ценных информационных ресурсов также осуществляется через дуги 
DAG. Поэтому передача ценного информационного ресурса соответствует 
метке~$(*)$ на соответствующей дуге. 
  
\section{Отражение требований MLS в~графах задач и~метаданных} 
  
  Правила MLS могут быть выражены следующим образом. Если в~вершину 
входит хоть одна дуга, помеченная~$(*)$, то эта вершина уже имеет метку~$(*)$ 
и~далее все дуги, выходящие из этой вершины, приобретают метку~$(*)$. 
Возможно, что ценная информация может порождаться в~результате решения 
задачи. Тогда эта задача помечается~$(*)$. 
  
  Справедливо следующее утверждение. 
  
  \smallskip
  
  \noindent
  \textbf{Утверждение~1.}\ \textit{При выполнении правил расстановки 
меток~$(*)$ в~данной ИТ выполняется политика MLS. }
  
  \smallskip
  
  \noindent
  Д\,о\,к\,а\,з\,а\,т\,е\,л\,ь\,с\,т\,в\,о\,.\ \ Допустим, что существует 
информационный поток с~уровня High на уровень Low. Тогда вершина, из которой 
исходит данный поток, имеет метку~$(*)$. По определению любая дуга, 
выходящая из вершины, помеченной~$(*)$, также имеет метку~$(*)$. Но такая 
дуга может входить только в~вершину, которая уже помечена (*). Но вершины 
уровня Low не могут иметь таких меток. Следовательно, предположение о 
существовании информационного потока с~уровня High на уровень Low неверно. 
Утверждение~1 доказано.
  
  \smallskip
  
  Метаданные содержат порядок решения задач, определяемый DAG. Поэтому 
если задачи в~этом порядке используют или порождают ценную информацию, то 
они имеют метку~$(*)$. Тогда все дальнейшие задачи также имеют такую метку. 
При таком дополнении МД несут информацию о требованиях~ПБ. 
  
  Ясно, что появление ценой информации означает дополнительные требования 
  к~хосту, на котором решается эта задача. Поэтому хост, на котором решается задача 
с~меткой~$(*)$, также должен иметь метку~$(*)$. Задача должна иметь 
метку~$(*)$, если она будет содержать ценную информацию. Отметим, что 
протокол управления соединениями в~РИВС основан на криптографии~[7] 
и~является безопасным. 
  
  Особенности хоста с~меткой~$(*)$ основаны на том, что на такой хост должен 
быть загружен <<чис\-тый>> образ операционной сис\-те\-мы, безопасный агент хоста для связи 
с~задачей~$\mathcal{N}$, <<чис\-тое>> программное
обеспечение для задач с~метками~$(*)$, и~на нем 
реализована процедура доверенной загрузки. В~MLS разрешены информационные 
потоки от уровня Low к~уровню High. Для предотвращения попадания на 
хост~$H^*$ вредоносного кода с~уровня Low необходимо обеспечить безопасный 
однонаправленный канал~[8] с~уровня Low на уровень High. Ясно, что на 
хосте~$H^*$ могут решаться различные (доверенно загруженные) задачи 
с~меткой~$(*)$. 
  
  Рассмотрим процедуру решения задачи~$A$ на уровне Low. В~этом случае 
возможны два варианта:
  \begin{enumerate}[(1)]
\item задача~$\mathcal{M}$ устанавливает задачу~$A$ на хосте, к~которому не 
допускается ни один информационный поток с~меткой~$(*)$;
\item на хосте $H^*(A^*)$ реализуется ИТ очистки от информации 
с~меткой~$(*)$ (новая доверенная загрузка). 
\end{enumerate}
  
\section{Отражение требований MLS в~инфраструктуре метаданных}

  Поскольку ИТ может использовать информацию с~меткой~$(*)$, то в~МД задача 
имеет метку~$(*)$. Если ИТ не использует информацию, помеченную~$(*)$, то 
в~МД отсутствуют задачи с~такой меткой. 
  
  Как было отмечено ранее, для передачи ценной информации задаче она должна 
иметь метку~$(*)$. Тогда хост, на котором находится эта задача, также должен 
иметь метку~$(*)$. Отсюда получается простейшее решение 
задачи~$\mathcal{M}$, обеспечивающее выполнение ПБ
MLS. Так как заранее известны все задачи с~меткой~$(*)$, то 
задача~$\mathcal{M}$ размещает их на хостах, помеченных~$(*)$, и~по правилам 
MLS задача~$\mathcal{N}$ реализует разрешительную систему на данных хостах. 
Это означает, что в~задаче~$\mathcal{N}$ выделяется подзадача~$\mathcal{N}_H$, 
реа\-ли\-зу\-ющая взаимодействие только на хостах с~меткой~$(*)$. Остальные хосты 
относятся к~уровню Low, и~задача~$\mathcal{N}$ реализует разрешительную 
систему только этих хостов. Таким образом, в~задаче~$\mathcal{N}$ выделяется 
независимая подзадача~$\mathcal{N}_L$, реализующая разрешительную систему 
взаимодействия хостов на уровне Low. 
  
  Отметим, что задачи~$\mathcal{N}_H$ и~$\mathcal{N}_L$ могут 
рассматриваться как два экземпляра задачи~$\mathcal{N}$, функционирующие на 
непересекающихся доменах РИВС. 
  
  Наиболее сложный вопрос состоит в~безопасной передаче информации с~уровня 
Low на уровень High. Согласно модели Bell-LaPadula~\cite{9-tt}, такую передачу 
можно осуществить только с~по\-мощью доверенного процесса, реализующего 
од\-но\-на\-прав\-лен\-ный канал с~уровня Low на уровень High. Такой однонаправленный 
канал можно реализовать на базе задачи~$\mathcal{N}$. Этот доверенный канал 
может быть организован на основе инфраструктуры метаданных следующим 
образом. 
  
  Пусть задача~$A$ запрашивает соединение с~задачей~$B^*$ через 
задачу~$\mathcal{N}_L$. Исходя из метаданных, задача~$\mathcal{N}$ 
определяет необходимость передачи исходных данных из задачи~$A$ 
в~задачу~$B^*$. При получении разрешения эти данные передаются из задачи~$A$ в~задачу~$\mathcal{N}_L$. После проверки их безопасности данные из 
задачи~$\mathcal{N}_L$ передаются в~задачу~$\mathcal{N}_H$ для дальнейшей 
передачи данных в~задачу~$B^*$. Отметим, что непосредственного соединения 
хоста уровня Low с~хостом уровня High не происходит. Отсюда следует 
утверждение. 
  
  \smallskip
  
  \noindent
  \textbf{Утверждение~2.}\ \textit{Пусть выполняются следующие условия}:
  \begin{enumerate}[(1)]
  \item \textit{ сформирована подсистема РИВС уровня High с~помощью задач 
  и~хостов с~метками~$(*)$ и~разрешительная система на основе МД 
и~задачи}~$\mathcal{N}_H$;
\item \textit{сформирована подсистема РИВС уровня Low с~помощью задач 
и~хостов без меток~$(*)$ и~разрешительная система на основе метаданных 
и~задачи}~$\mathcal{N}_L$;
  \item \textit{взаимодействие уровней Low и~High осуществляется только через 
однонаправленное взаимодействие задач~$\mathcal{N}_L$ 
и~~$\mathcal{N}_H$}. 
  \end{enumerate}
  \textit{Тогда в~РИВС выполняется политика MLS}.
  
  \smallskip
  
  \noindent
  Д\,о\,к\,а\,з\,а\,т\,е\,л\,ь\,с\,т\,в\,о\,.\ \ Метаданные
   и~задача~$\mathcal{N}_H$ не 
допускают взаимодействия уровня High с~уровнем Low. Аналогично МД и~задача 
NL не допускают непосредственного взаимодействия уровня Low с~уровнем High. 
Безопасный интерфейс уровня Low с~уровнем High реализован 
однонаправленным каналом между задачами~$\mathcal{N}_L$ 
и~$\mathcal{N}_H$. Функционал, реализующий этот канал, как и~вся 
задача~$\mathcal{N}$, могут быть изолированы от остального функционала РИВС и~поэтому могут считаться доверенным субъектом. Таким образом, все условия 
реализации политики MLS выполнены. Утверждение~2 доказано. 
  
  \section{Поиск информации с~уровня High на~уровне Low}
  
  Для решения задач с~меткой~$(*)$ может возникнуть необходимость 
дополнительного поиска информации в~памяти о предшествующих задачах. 
Метаданные сохраняют информацию о цепочке решенных задач, и~обращение 
к~ним на уровне High не представляет сложности. 
  
  Если задаче~$A^*$ необходимо найти дополнительную информацию на уровне 
Low, то тогда также можно использовать доверенное взаимодействие между 
задачами~$\mathcal{N}_H$ и~$\mathcal{N}_L$. Задача~$\mathcal{N}_H$ 
обращается к~задаче~$\mathcal{N}_L$ с~запросом на поиск данных в~решенных на 
уровне Low задачах. Задача~$\mathcal{N}_L$, используя обратный обзор 
решенных на уровне Low задач, ищет искомую информацию. В~данном случае 
возможен скрытый канал с~уровня High на уровень Low в~задании поиска 
информации на уровне Low~\cite{8-tt}. Этот канал можно перекрыть с~помощью 
последовательного опроса задач на уровне Low и~выявления признаков искомой 
информации уже на уровне задачи~$\mathcal{N}_L$. В~случае появления 
необходимых признаков задача~$\mathcal{N}_L$ передает 
задаче~$\mathcal{N}_H$ данные для задачи~$A^*$. 
  
  Данный способ не является доказательством перекрытия скрытого канала. 
Однако обращение с~уровня High на уровень Low считается запрещенным 
информационным потоком в~политике MLS, и~реализация такого поиска с~учетом 
возможности скрытого канала является сложной задачей~\cite{10-tt}. 
  
  \section{Заключение }
  
  В работе рассмотрена РИВС, в~которой управ\-ле\-ние соединениями 
осуществляется с~помощью МД. Показана возможность реализации 
ПБ MLS в~рассматриваемой РИВС на основе 
инфраструктуры, связанной с~МД. Единственный доверенный процесс, 
необходимый для реализации ПБ MLS, функционирует на 
уровне управ\-ле\-ния соединениями. Этот уровень не связан с~плос\-костью передачи 
данных и~может быть изолирован с~\mbox{целью} обеспечения его информационной 
безопас\-ности. 
  
  В работе рассмотрена только одна ИТ, для которой необходимо выполнить 
требования политики безопасности MLS. Однако рассмотренный метод легко 
обобщается на случай множества ИТ.
  

{\small\frenchspacing
 {%\baselineskip=10.8pt
 \addcontentsline{toc}{section}{References}
 \begin{thebibliography}{99}
\bibitem{1-tt}
Department of Defense trusted computer system evaluation criteria.~--- U.S.\ National Institute of 
Standards and Technology, Department of Defense, 1985. {\sf 
http://csrc.nist.gov/publications/history/dod85.\linebreak pdf}. 
\bibitem{2-tt}
\Au{Grusho A., Grusho N., Zabezhailo~M., Zatsarinny~A., Timonina~E.} Information security of SDN 
on the basis of meta data~// Computer 
network security~/ Eds. J.~Rak, J.~Bay, I.~Kotenko, \textit{et al.}~---
Lecture notes in computer science ser.~--- Springer, 2017. 
Vol.~10446.  P.~339--347. doi: 10.1007/978-3-319-65127-9\_27.
\bibitem{3-tt}
\Au{Грушо А.\,А., Грушо~Н.\,А., Левыкин~М.\,В., Тимонина~Е.\,Е.} Методы идентификации 
захвата хоста в~распределенной ин\-фор\-ма\-ци\-он\-но-вы\-чис\-ли\-тель\-ной системе, 
защищенной с~по\-мощью метаданных~// Информатика и~её применения, 2018. Т.~12. Вып.~4. 
С.~41--45.
\bibitem{4-tt}
\Au{Grusho A.\,A., Grusho~N.\,A., Timonina~E.\,E.} 
Information flow control on the basis of meta 
data~// Distributed computer and communication networks~/ 
Eds. V.\,M.~Vishnevskiy, K.\,E.~Samouylov, D.\,V.~Kozyrev.~--- 
Lecture notes
in computer science ser.~--- Springer,
 2019. Vol.~11965. P.~548--562.

\bibitem{5-tt}
\Au{Грушо А.\,А., Тимонина~Е.\,Е., Шоргин~С.\,Я.} Иерархический метод порождения 
метаданных для управления сетевыми соединениями~// Информатика и~её применения, 2018. 
Т.~12. Вып.~2. С.~44--49.
\bibitem{6-tt}
\Au{Грушо А.\,А., Зацаринный~А.\,А., Тимонина~Е.\,Е.} Электронная бухгалтерская книга на базе 
ситуационных центров для цифровой экономики~// Системы и~средства информатики, 2019. 
Т.~29. №\,2. С.~4--11.
\bibitem{7-tt}
\Au{Grusho A.\,A., Timonina~E.\,E., Shorgin~S.\,Ya.} Modelling for ensuring information security of 
the distributed information systems~// 31th European Conference on Modelling and Simulation 
Proceedings.~--- Dudweiler, Germany: Digitaldruck Pirrot GmbH, 2017. P.~656--660. {\sf 
http://www.scs-europe.net/dlib/2017/\linebreak  
ecms2017acceptedpapers/0656-probstat\_ECMS2017\_ 0026.pdf}.
\bibitem{8-tt}
\Au{Тимонина Е.\,Е.} Анализ угроз скрытых каналов и~методы построения гарантированно 
защищенных распределенных автоматизированных систем: Дис.\ \ldots\ д-ра техн. наук.~--- 
М., 2004. 204~с.
\bibitem{9-tt}
\Au{Грушо А.\,А., Применко~Э.\,А., Тимонина~Е.\,Е.} Теоретические основы компьютерной 
безопасности.~--- М.: Академия, 2009. 272~с.
\bibitem{10-tt}
\Au{Grusho A.\,A., Grusho~N.\,A., Zabezhailo~M.\,I., Timonina~E.\,E.} Protection of valuable 
information in public information space~// Communications of the ECMS: 33th European Conference 
on Modelling and Simulation Proceedings.~--- 
Dudweiler, Germany: Digitaldruck Pirrot GmbH, 2019. 
Vol.~33. No.\,1. P.~451--455. 
{\sf 
http://www.scs-europe.net/dlib/2019/ecms2019acceptedpapers/0451\_\linebreak pstat\_ecms2019\_0018.pdf}.
 \end{thebibliography}

 }
 }

\end{multicols}

\vspace*{-6pt}

\hfill{\small\textit{Поступила в~редакцию 13.10.19}}

\vspace*{8pt}

%\pagebreak

%\newpage

%\vspace*{-28pt}

\hrule

\vspace*{2pt}

\hrule

%\vspace*{-2pt}

\def\tit{USING METADATA TO~IMPLEMENT MULTILEVEL SECURITY POLICY 
REQUIREMENTS}


\def\titkol{Using metadata to~implement multilevel security policy 
requirements}

\def\aut{A.\,A.~Grusho, N.\,A.~Grusho, and~E.\,E.~Timonina}

\def\autkol{A.\,A.~Grusho, N.\,A.~Grusho, and~E.\,E.~Timonina}

\titel{\tit}{\aut}{\autkol}{\titkol}

\vspace*{-11pt}


 \noindent
   Institute of Informatics Problems, Federal Research Center ``Computer Sciences and 
Control'' of the Russian Academy of Sciences; 44-2~Vavilov Str., Moscow 119133, 
Russian Federation

\def\leftfootline{\small{\textbf{\thepage}
\hfill INFORMATIKA I EE PRIMENENIYA~--- INFORMATICS AND
APPLICATIONS\ \ \ 2019\ \ \ volume~13\ \ \ issue\ 4}
}%
 \def\rightfootline{\small{INFORMATIKA I EE PRIMENENIYA~---
INFORMATICS AND APPLICATIONS\ \ \ 2019\ \ \ volume~13\ \ \ issue\ 4
\hfill \textbf{\thepage}}}

\vspace*{3pt}  
 
  
   \Abste{A distributed information computing system which objects contain both valuable 
information (or are themselves valuable) and open (non-valuable) information is considered. To protect 
valuable information, multilevel  security (MLS) policy is used that prohibits information flows from objects with 
valuable information to objects with open information. Objects with valuable information form a~class 
of high-level objects, and objects with open information form a class of low-level objects.
   Metadata is created to manage network connections. Metadata is a simplification of mathematical 
models of business processes and is the basis of a permission system for host connections in 
a~distributed information computing system.
   The paper constructs MLS security policy rules, and based on metadata-related infrastructure, 
shows the ability to implement this security policy in the distributed information computing system. 
The only trusted process required to implement the MLS security policy is at the connection 
management level. This layer is unrelated to the data plane and can be isolated to ensure its 
information security.}
    
   \KWE{MLS security policy; information flows; metadata}
   
   
   

\DOI{10.14357/19922264190414} 

%\vspace*{-14pt}

 \Ack
   \noindent
   The paper was partially supported by the Russian Foundation for Basic Research (project  
18-07-00274).


%\vspace*{-6pt}

  \begin{multicols}{2}

\renewcommand{\bibname}{\protect\rmfamily References}
%\renewcommand{\bibname}{\large\protect\rm References}

{\small\frenchspacing
 {%\baselineskip=10.8pt
 \addcontentsline{toc}{section}{References}
 \begin{thebibliography}{99}
\bibitem{1-tt-1}
U.S.\ National Institute of Standards and Technology, Department of Defence. 1985.
Department of Defense trusted computer system evaluation criteria. Available at: {\sf 
http://csrc.nist.gov/publications/history/dod85.pdf} (accessed October~6, 2019).
\bibitem{2-tt-1}
\Aue{Grusho, A., N.~Grusho, M.~Zabezhailo, A.~Zatsarinny, and E.~Timonina.} 2017. Information 
security of SDN on the basis of meta data. 
\textit{Computer network security}. Eds. J.~Rak, J.~Bay, I.~Kotenko, 
\textit{et al.}
Lecture notes in computer science ser. Springer. 
10446:339--347. doi: 10.1007/978-3-319-65127-9\_27.
\bibitem{3-tt-1}
\Aue{Grusho, A.\,A., N.\,A.~Grusho, M.\,V.~Levykin, and E.\,E.~Timonina.} 2018. Metody 
identifikatsii zakhvata khosta v~raspredelennoy informatsionno-vychislitel'noy sisteme, 
zashchishchennoy s~pomoshch'yu metadannykh [Methods of identification of host capture in the 
distributed information system which is protected on the base of meta data]. \textit{Informatika i~ee 
Primeneniya~--- Inform. Appl.} 12(4):41--45. 
\bibitem{4-tt-1}
\Aue{Grusho, A.\,A., N.\,A.~Grusho, and E.\,E.~Timonina.} 2019. Information flow control 
on the basis of meta data. \textit{Distributed computer and communication networks}.
 Eds. V.\,M.~Vishnevskiy, 
K.\,E.~Samouylov, and D.\,V.~Ko\-zy\-rev. 
Lecture notes
in computer science ser. Springer. 11965:548--562.
\bibitem{5-tt-1}
\Aue{Grusho, A.\,A., E.\,E.~Timonina, and S.\,Ya.~Shorgin.} 2018. Ierarkhicheskiy metod 
porozhdeniya metadannykh dlya upravleniya setevymi soedineniyami [Hierarchical method of meta 
data generation for control of network connections]. \textit{Informatika i~ee Primeneniya~--- Inform. 
Appl.} 12(2):44--49.
\bibitem{6-tt-1}
\Aue{Grusho, A.\,A., A.\,A.~Zatsarinny, and E.\,E.~Timonina.} 2019. Elektronnaya bukhgalterskaya 
kniga na baze situatsionnykh tsentrov dlya tsifrovoy ekonomiki [The electronic ledger on the basis of 
the situational centers for digital economy]. \textit{Sistemy i~Sredstva Informatiki~--- Systems and 
Means of Informatics} 29(2):4--11.
\bibitem{7-tt-1}
\Aue{Grusho, A.\,A., E.\,E.~Timonina, and S.\,Ya.~Shorgin.} 2017. Modelling for ensuring 
information security of the distributed information systems. \textit{31th European Conference on 
Modelling and Simulation Proceedings}. Dudweiler, Germany: Digitaldruck Pirrot GmbH. 656--660. 
Available at: {\sf  
http://www.scs-europe.net/dlib/2017/\linebreak ecms2017acceptedpapers/0656-probstat\_ECMS2017\_ 0026.pdf} (accessed 
October~6, 2019).
\bibitem{8-tt-1}
\Aue{Timonina, E.\,E.} 2004. Analiz ugroz skrytykh kanalov i~metody postroeniya garantirovanno 
zashchishchennykh raspredelennykh avtomatizirovannykh sistem [The analysis of threats of covert 
channels and methods of creation of guaranteed protected distributed automated 
systems]. Moscow.  D.Sc. Diss.  204~p.
\bibitem{9-tt-1}
\Aue{Grusho, A., E.~Primenko, and E.~Timonina.} 2009. \textit{Teoreticheskie osnovy 
komp'yuternoy bezopasnosti} [Theoretical bases of computer security]. Moscow: Academy. 272~р.
\bibitem{10-tt-1}
\Aue{Grusho, A.\,A., N.\,A.~Grusho, M.\,I.~Zabezhailo, and E.\,E.~Timonina.} 2019. Protection of 
valuable information in public information space. \textit{Communications of the ECMS:  33th 
European Conference on Modelling and Simulation Proceedings}. 
Dudweiler, Germany: Digitaldruck Pirrot GmbH.
33(1):451--455. Available at: {\sf 
http://www.scs-europe.net/dlib/2019/ecms2019acceptedpapers/0451\_\linebreak pstat\_ecms2019\_0018.pdf} (accessed 
October~6, 2019).
\end{thebibliography}

 }
 }

\end{multicols}

\vspace*{-6pt}

\hfill{\small\textit{Received October 13, 2019}}

%\pagebreak

%\vspace*{-22pt}


\Contr


\noindent
\textbf{Grusho Alexander A.} (b.\ 1946)~--- Doctor of Science in physics and 
mathematics, professor, principal scientist, Institute of Informatics Problems, Federal 
Research Center ``Computer Sciences and Control'' of the Russian Academy of 
Sciences; 44-2~Vavilov Str., Moscow 119133, Russian Federation;  
\mbox{grusho@yandex.ru}

\vspace*{3pt} 

\noindent
\textbf{Grusho Nikolai A.} (b.\ 1982)~--- Candidate of Science (PhD) in physics 
and mathematics, senior scientist, Institute of Informatics Problems, Federal 
Research Center ``Computer Sciences and Control'' of the Russian Academy of 
Sciences;  
44-2~Vavilov Str., Moscow 119133, Russian Federation; info@itake.ru 

\vspace*{3pt}

\noindent
\textbf{Timonina Elena E.} (b.\ 1952)~--- Doctor of Science in technology, 
professor, leading scientist, Institute of Informatics Problems, Federal Research 
Center ``Computer Sciences and Control'' of the Russian Academy of Sciences;  
44-2~Vavilov Str., Moscow 119133, Russian Federation; 
\mbox{eltimon@yandex.ru}
\label{end\stat}

\renewcommand{\bibname}{\protect\rm Литература}  