\def\stat{konovalov}

\def\tit{КОМПЛЕКСНОЕ УПРАВЛЕНИЕ В~ОДНОМ КЛАССЕ СИСТЕМ С~ПАРАЛЛЕЛЬНЫМ 
ОБСЛУЖИВАНИЕМ$^*$}

\def\titkol{Комплексное управление в~одном классе систем с~параллельным 
обслуживанием}

\def\aut{М.\,Г.~Коновалов$^1$, Р.\,В.~Разумчик$^2$}

\def\autkol{М.\,Г.~Коновалов, Р.\,В.~Разумчик}

\titel{\tit}{\aut}{\autkol}{\titkol}

\index{Коновалов М.\,Г.}
\index{Разумчик Р.\,В.}
\index{Konovalov M.\,G.}
\index{Razumchik R.\,V.}



{\renewcommand{\thefootnote}{\fnsymbol{footnote}} \footnotetext[1]
{Исследование выполнено при частичной финансовой поддержке РФФИ (проект 18-07-00692).}}


\renewcommand{\thefootnote}{\arabic{footnote}}
\footnotetext[1]{Институт проблем информатики Федерального исследовательского центра <<Информатика и~управление>> 
Российской академии наук, \mbox{mkonovalov@ipiran.ru}}
\footnotetext[2]{Институт проблем информатики Федерального исследовательского центра <<Информатика 
и~управление>> Российской академии наук; Российский университет дружбы народов, 
\mbox{rrazumchik@ipiran.ru} % \mbox{razumchik-rv@rudn.ru
}

\vspace*{12pt}


    
    \Abst{Рассматривается задача эффективного распределения потока однородных заданий в~системах с~параллельным обслуживанием на независимо работающих серверах 
с~очередями неограниченной емкости. Имеется один диспетчер, который распределяет 
задания между серверами без промежуточного хранения. Обмен заданиями между серверами 
невозможен. Эффективность понимается с~точки зрения минимума двух конкурирующих 
целевых показателей: стационарного среднего времени пребывания задания в~системе 
и~вероятности нарушения дедлайна, до истечения которого каждое задание должно попасть на 
обслуживание. Задание, дедлайн которого нарушен, остается на сервере до тех пор, пока не 
будет обслужено. В~работе предлагается подход, основанный на идее комплексного 
управления потоками в~системе и~предполагающий одновременный подбор как наиболее 
эффективной диспетчеризации, так и~дисциплины обслуживания очереди. Методами 
статистической имитации на численных примерах показано, что новый подход позволяет 
получать такой выигрыш в~целевых показателях, достичь которого не удается, применяя те 
же механизмы управления, но по отдельности.}

\KW{системы с~параллельным обслуживанием; диспетчеризация; дисциплина обслуживания 
очереди; время пребывания; вероятность превышения дедлайна}

\DOI{10.14357/19922264190409} 
  
\vspace*{6pt}


\vskip 10pt plus 9pt minus 6pt

\thispagestyle{headings}

\begin{multicols}{2}

\label{st\stat}

\section{Введение}

Данная статья продолжает тематику повышения производительности систем 
с~параллельными и~независимыми серверами. Теория управления такими 
системами традиционно разделяется на три практически самостоятельных 
раздела:
   \begin{enumerate}[(1)]
   \item диспетчеризация, т.\,е.\ распределение потоков заданий между 
серверами;
   \item обслуживание очередей, т.\,е.\ очередность обслуживания заданий на 
серверах;
   \item ограничение доступа заданий в~систему.
   \end{enumerate}
   
   Существует весьма четкое разграничение этих направлений в~литературе по 
управлению обслуживанием, причем каждая отдельно взятая работа посвящена, 
как правило, строго одному из пе\-ре\-чис\-лен\-ных разделов. 

Типичная статья в~этой 
области выглядит так. Рассматриваются, например, несколько 
диспетчеризаций, которые анализируются и~сравниваются между собой, а~при 
этом остальные два способа управ\-ле\-ния фиксированы и~не об\-суж\-да\-ют\-ся. В~то 
же время очевидно, что все указанные механизмы служат одной цели~--- 
повышению эффективности обслуживания~--- и~в~реальных системах они 
используются по большей части одновременно. 

В~данной статье предпринята 
попытка (подроб\-но см., например,~[1--3]) обратиться к~такой постановке задачи, 
в~которой сочеталась бы возможность использовать для достижения заданных 
критериев качества разные методы. Конкретно, рассматривается сервисная 
система, в~которой оптимизация осуществляется за счет одновременного 
подбора как наиболее эффективной диспетчеризации, так и~дисциплины 
обслуживания очереди.
   
   Данная работа базируется на методах ста\-ти\-стической имитации. 
Излагаемые результаты от\-носятся к~единственному примеру сис\-те\-мы 
с~параллельным обслуживанием, взятому (для \mbox{возможности} осуществить 
срав\-не\-ние) из \mbox{статьи}~[4]. 


Материал распределен следующим образом. Общая 
постановка задачи сформулирована в~разд.~2. Здесь вкратце описывается 
предлагаемый подход к~ее решению. В~разд.~3 приводятся некоторые 
результаты численных экспериментов. В~заключении кратко обсуждаются 
результаты работы.
   
\section{Описание системы и~постановка задачи}

   Система состоит из~$N$~серверов, работающих параллельно и~независимо 
друг от друга. Производительность (интенсивность обслуживания) сервера~$i$ 
равна~$v^{(i)}$. В~систему поступает пуассоновский входной поток 
заданий\footnote{Ввиду предположения, что в~серверах реализована дисциплина частичного 
разделения процессора, предположение о том, что в~систему поступает один поток однородных 
заданий, является существенным. Случай нескольких входящих потоков разнородных заданий 
требует отдельного исследования.}. Задания имеют случайную длину (объем), которая 
определяется одним и~тем же распределением~$G$, и~предельный срок $0\hm< 
\tau\hm<\infty$, до истечения которого оно должно попасть на обслуживание 
(дедлайн). Каждое поступившее задание должно быть немедленно направлено 
на один из серверов, причем диспетчеру, осуществляющему этот выбор, 
доступна любая информация о текущем состоянии системы (например, размеры 
очередей, остаточная работа на каждом сервере и~т.\,п.)\footnote{Совершенно 
естественным является предположение, что диспетчеру доступна полная предыстория принятых 
решений (с~указанием моментов времени, в~которые решения принимались). Однако 
диспетчеризации, использующие такую информацию, обычно требуют больших дополнительных 
вычислений и~поэтому здесь рассматриваться не будут.}. Число заданий, которые могут 
одновременно находиться на каждом сервере, не ограничено. Если дедлайн 
нарушен, задание остается на сервере до тех пор, пока не будет обслужено до 
конца. Выполнение заданий на каждом сервере происходит в~соответствии 
с~дисциплиной LPS (limited processor sharing), 
т.\,е.\ дисциплиной частичного разделения процессора~[5, 
6]. Эта дисциплина предполагает, что каждый сервер~$i$ действует следующим 
образом. Если в~очереди находится~$n_i$, $1\hm\leq n_i\hm\leq L_i$, заданий 
($L_i\hm\geq1$~--- фиксированное число), то каждое из~$n_i$ заданий 
обслуживается со ско\-ростью~$v^{(i)}/n_i$. В~противном случае первые~$L_i$ 
заданий обслуживаются со ско\-ростью~$v^{(i)}/L_i$, а~остальные ожидают 
обслуживания в~очереди либо в~порядке поступления (FIFO~--- first in, first out), 
либо в~порядке 
возрастания размера задания (SJF, shortest job first). 
  
  Система функционирует в~непрерывном времени $t\hm\geq0$. Пусть 
$0\hm\leq t_1<\cdots< t_n<\cdots$~--- последовательность моментов поступления 
заданий в~систему. Действие, принимаемое в~момент~$t_n$ относительно вновь 
поступившего задания, обозначим через~$y_n$. Полагаем, что $y_n\hm=i$, если 
задание на\-прав\-ле\-но на сервер~$i$. Пусть задание, поступившее в~момент~$t_n$ 
и~распределенное согласно правилу~$y_n$, в~течение времени~$w_n$ ожидает 
обслуживания, а затем в~течение времени~$s_n$ обслуживается.
  
  Величины $w_n$ и~$s_n$ заранее не известны, но они как раз служат 
основой для выбора показателей качества обслуживания, которых в~данном 
случае два. Это, во-пер\-вых, предельное среднее время пребывания заданий 
в~системе

\vspace*{2pt}

\noindent
  $$
  v=\lim\limits_{m\to\infty} \fr{1}{m}\sum\limits^m_{n=1} \mathrm{E}_y\left( 
w_n+s_n\right)
  $$
  
  \vspace*{-3pt}
  
  \noindent
и, во-вторых, предельная вероятность нарушения дедлайна

\noindent
$$
r=\lim\limits_{m\to\infty} \fr{1}{m}\sum\limits^m_{n=1} 
\mathrm{E}_y\mathbf{1}_{\{w_n>\tau\}}\,.
$$

\vspace*{-2pt}
  
  В этих формулах $\mathbf{1}_{\{A\}}$~--- индикатор события~$A$, 
$\mathrm{E}_y$~--- интегрирование по мере, порождаемой 
диспетчеризацией~$y$.
  
  Показатели~$v$ и~$r$ являются конкурирующими, причем оба требуют 
минимизации. Для разрешения многокритериальности в~данном случае вместо 
того, чтобы формулировать задачу условной оптимизации либо определять 
свертку, поступим следующим образом\footnote[3]{Несмотря на кажущуюся простоту 
такого подхода по сравнению с~условной оптимизацией, отыскание значения~$L_i$ и~оптимальных 
значений параметров диспетчеризаций представляет собой отдельные задачи, которые <<в~лоб>> 
решить невозможно в~силу несчетности множества поиска. Здесь эти задачи решаются с~помощью 
специального адаптивного алгоритма на имитационной модели~\cite{7-kon}.}. 
Зададим набор 
допустимых диспетчеризаций~$y_n$, две дисциплины обслуживания очереди 
(FIFO или SJF), а также диапазон значений~$L_i$ и~будем перебирать 
возможные варианты для нахождения наилучших значений показателей~$v$ 
и~$r$.

\begin{table*}\small %tabl1
\begin{center}
\Caption{Стационарная вероятность~$r$ нарушения дедлайна в~системе из четырех серверов при различных 
стратегиях диспетчеризации и~обслуживания очереди и~различных значениях загрузки}
\vspace*{2ex}

\tabcolsep=10.5pt
\begin{tabular}{|l|c|c|c|c|}
\hline
\multicolumn{1}{|c|}{Стратегия}&\tabcolsep=0pt\begin{tabular}{c}$\lambda=4{,}2$\\ 
($\rho=0{,}7$)\end{tabular}&\tabcolsep=0pt\begin{tabular}{c}$\lambda=4{,}8$\\ 
($\rho=0{,}8$)\end{tabular}&\tabcolsep=0pt\begin{tabular}{c}$\lambda=5{,}04$\\
 ($\rho=0{,}84$)\end{tabular}&\tabcolsep=0pt\begin{tabular}{c}$\lambda=5{,}4$\\
($\rho=0{,}9$)\end{tabular}\\
\hline
$\mathrm{LWL+FIFO}$&0,011&0,054&0,098&0,236\\
$\mathrm{MHV}(C)+\mathrm{FIFO}$&0,016 ($C=2$)&0,052 ($C=1{,}25$)&0,095 ($C=1{,}25)$&0,224 ($C=2$)\\
$\mathrm{LWL+LPS(2)+FIFO}$&0,004&0,027&0,056&0,169\\
$\mathrm{MHV}(C)+\mathrm{LPS(2)+FIFO}$&0,004 ($C=3$)&0,024 ($C=2{,}5$)&0,049 ($C=2$)&---\\
$\mathrm{MHVI}(C)+\mathrm{LPS(2)+FIFO}$&---&---&---&0,145 ($C=2{,}5$)\\
$\mathrm{LWL+SJF}$&0,008&0,027&0,043&0,079\\
$\mathrm{MHV}(C)+\mathrm{SJF}$&0,009 ($C=2$)&---&---&---\\
$\mathrm{MHVI}(C)+\mathrm{SJF}$&---&0,024 ($C=2{,}5$)&0,035 ($C=2{,}5$)&0,058 ($C=2$)\\
$\mathrm{LWL+LPS(2)+SJF}$&0,004&0,015&0,027&0,065\\
$\mathrm{MHV}(C)+\mathrm{LPS(2)+SJF}$&0,003 ($C=3$)&---&---&---\\
$\mathrm{MHVI}(C)+\mathrm{LPS(2)+SJF}$&---&0,011 ($C=2{,}5$)&0,019 ($C=2{,}5$)&0,042 ($C=2$)\\
\hline
\end{tabular}
\end{center}
\vspace*{-9pt}
\end{table*}

  
  Ограничимся следующими простыми и~в~то же время достаточно 
эффективными диспетчеризациями, наиболее полно использующими 
доступную информацию о текущем состоянии системы (см., 
например,~\cite[разд.~3]{8-kon}):
  \begin{itemize}
   \item $y_n=\argmin\nolimits_{1\leq i\leq M} (u_n^{(i)})$, где $u_n^{(i)}$~--- время, необходимое для 
полного выполнения всех заданий, которые в~момент принятия решения~$t_n$ находятся на 
сервере~$i$, при условии, что новые задания не поступают. Если в~момент~$t_n$ на сервере~$i$ 
находятся~$k$~заданий, причем их длины равны $l_1, l_2,\ldots , l_k$, то $u_n^{(i)}\hm= 
\sum\nolimits^k_{i=1} l_i/v^{(i)}$. Далее эта диспетчеризация обозначается 
как $\mathrm{LWL}$ (least work left);
  \item $y_n=\argmin\nolimits_{1\leq i\leq M} (\omega_n^{(i)})$, где 
$\omega_n^{(i)}$~--- числовая характеристика сервера~$i$ в~момент принятия 
решения~$t_n$, вычисляемая следующим образом. Если в~момент~$t_n$ на 
сервере~$i$ находятся~$k$~заданий, причем их упорядоченные по 
возрастанию длины равны $l_1, l_2,\ldots , l_k$, то

\noindent
  $$
  \omega_n^{(i)}=\sum\limits^k_{i=1} \tau_i C^{k-i}\,,
  $$
  
  \noindent
где 
$$
\tau_i= \fr{k-i+1}{v^{(i)}} \left( l_i-\sum\limits_{j=1}^{i-1} \left( \fr{l_j(k-
j+1)}{v^{(i)}}\right)\right)\,,
$$
а $C$~--- фиксированное положительное число. Далее эта эвристическая 
диспетчеризация обозначается как $\mathrm{MHV}(C)$~[9, 10].
\end{itemize}
   
   Заметим, что если функция $\argmin$ возвращает несколько значений, то 
поступившее задание направляется на самый быстрый сервер; если таких 
несколько, то сервер выбирается наугад. Расчет~$\omega_n^{(i)}$ возможен, 
вообще говоря, как без учета нового задания, так и~в~предположении, что оно 
отправлено именно на данный сервер. Снабжая такую оценку дополнительной 
чертой сверху, получаем, заменяя в~$\mathrm{MHV}(C)$~$\omega_n^{(i)}$ на 
$\overline{\omega}_n^{(i)} \hm- \omega_n^{(i)}$, еще одно правило 
диспетчеризации (далее~--- $\mathrm{MHVI}(C)$).

\vspace*{-6pt}
   
\section{Некоторые результаты численных экспериментов}

\vspace*{-6pt}

   Проверка большого числа численных экспериментов показывает, что 
комплексное управление процессом распределения заданий и~обслуживанием 
очереди дает такой выигрыш в~целевых функциях~$v$ и~$r$, достичь которого 
не удается, применяя те же механизмы, но по раздельности\footnote{Даже весьма 
сложных стратегий, как, например, FPI (first policy iteration) из~\cite{4-kon}.}. Важно отметить, что указанный 
эффект проявляется при использовании просто реализуемых стратегий, не 
требующих сложной настройки параметров. Одному из таких примеров 
посвящен данных раздел.
   
   Рассматривается гетерогенная система из четырех серверов 
производительностью $v^{(1)}\hm= v^{(2)}\hm=2$, 
$v^{(3)}\hm= v^{(4)}\hm=1$. Входящий в~систему поток заданий с~дедлайном $\tau\hm=2$ является пуассоновским 
с~параметром~$\lambda$. Размер заданий имеет экспоненциальное 
распределение с~параметром~1. Загрузка системы равна $\rho\hm= \lambda/6$.
   
   В табл.~1 и~2 приводятся значения~$v$ и~$r$ для нескольких комбинаций 
диспетчеризаций и~механизмов управления очередью при разных значениях 
загрузки системы.
   
\begin{table*}\small %tabl2
\begin{center}
\parbox{400pt}{\Caption{Стационарное среднее~$v$ и~среднеквадратичное отклонение для времени пребывания задания 
в~системе из четырех серверов при различных стратегиях диспетчеризации и~обслуживания очереди 
и~различных значениях загрузки. Значение параметра~$C$ берется из соответствующей ячейки табл.~1}
}

\vspace*{2ex}

\tabcolsep=10pt
\begin{tabular}{|l|c|c|c|c|}
\hline
\multicolumn{1}{|c|}{Стратегия}&\tabcolsep=0pt\begin{tabular}{c}$\lambda=4{,}2$\\ 
($\rho=0{,}7$)\end{tabular}&\tabcolsep=0pt\begin{tabular}{c}$\lambda=4{,}8$\\ 
($\rho=0{,}8$)\end{tabular}&\tabcolsep=0pt\begin{tabular}{c}$\lambda=5{,}04$\\
 ($\rho=0{,}84$)\end{tabular}&\tabcolsep=0pt\begin{tabular}{c}$\lambda=5{,}4$\\
($\rho=0{,}9$)\end{tabular}\\
\hline
$\mathrm{LWL+FIFO}$&0,886/0,863&1,151/1,060&1,355/1,233&1,975/1,804\\
$\mathrm{MHV}(C)+\mathrm{FIFO}$&0,872/0,800&1,146/1,078&1,351/1,261&2,000/2,064\\
$\mathrm{LWL+LPS(2)+FIFO}$&0,931/0,980&1,208/1,230&1,410/1,411&2,047/1,997\\
$\mathrm{MHV}(C)+\mathrm{LPS(2)+FIFO}$&0,833/0,877&1,172/1,246&1,407/1,522&---\\
$\mathrm{MHVI}(C)+\mathrm{LPS(2)+FIFO}$&---&---&---&1,987/2,137\\
$\mathrm{LWL+SJF}$&0,848/0,879&1,028/1,126&1,147/1,370&1,430/2,267\\
$\mathrm{MHV}(C)+\mathrm{SJF}$&0,816/0,796&---&---&---\\
$\mathrm{MHVI}(C)+\mathrm{SJF}$&---&0,957/1,114&1,055/1,391&1,287/2,462\\
$\mathrm{LWL+LPS(2)+SJF}$&0,915/0,983&1,145/1,258&1,297/1,479&1,684/2,349\\
$\mathrm{MHV}(C)+\mathrm{LPS(2)+SJF}$&0,884/0,.950&---&---&---\\
$\mathrm{MHVI}(C)+\mathrm{LPS(2)+SJF}$&---&1,086/1,238&1,218/1,495&1,539/2,385\\
\hline
\end{tabular}
\end{center}
\vspace*{-9pt}
\end{table*}
   
   Принимая стратегию $\mathrm{LWL}\hm+\mathrm{FIFO}$ за точку отсчета, 
из таблиц можно увидеть, что комплексное управление диспетчеризацией 
и~процессом обслуживания однозначно имеет преимущество. Выигрыш 
в~целевых функциях~$v$ и~$r$ незначителен при средней нагрузке и~близок 
к~нулю при малой нагрузке. При большой загрузке по сравнению со стратегией 
$\mathrm{LWL}\hm+\mathrm{FIFO}$ одна из лучших рассмотренных 
стратегий~--- $\mathrm{MHVI}(С)\hm+\mathrm{LPS(2)+SJF}$~--- обеспечивает 
выигрыш по~$v$ и~$r$ примерно в~80\% и~20\% соответственно. По мере 
внедрения комплексного подхода относительный выигрыш увеличивается 
с~ростом нагрузки, хотя и~немонотонно. Последнее обстоятельство указывает 
на то, что существуют <<неудачные>> комбинации механизмов 
диспетчеризации и~управления очередью.

\vspace*{-6pt}
   
\section{Заключение}

\vspace*{-6pt}

   Подводя итог краткому экспериментальному анализу комплексного 
управления в~системах с~параллельным обслуживанием, можно сделать 
следующие выводы. Эффективность таких систем может быть повышена за 
счет одновременного использования даже простейших алгоритмов на всех 
этапах принятия решений (поступление задания в~систему, выбор очередного 
задания из очереди, выбор способа обработки заданий). В~случае 
двухкритериальной оптимизации комплексный подход дает такой выигрыш 
в~целевых функциях, который иначе может быть получен только за счет 
внедрения изощренных стратегий, требующих специальной настройки 
параметров. Другое его преимущество кроется в~отсутствие требования, чтобы 
входящий поток заданий был пуассоновским. Кроме того, лучшие найденные 
с~его помощью решения\footnote[1]{Исключение~--- диспетчеризация $\mathrm{MHVI}(C)$ при 
малых~$C$, при которой система может быть нестабильна.} не приводят к~образованию 
в~системе <<бесконечных>> очередей, которые, как отмечено 
в~\cite[разд.~5]{4-kon}, при иных подходах к~управлению могут появляться 
в~случае большой загрузки. 
   
   Недостатком подхода, несомненно, является невозможность на данный 
момент получения ка\-ких-ли\-бо аналитических результатов. Пожалуй, 
единственное исключение составляет не рассмотренная здесь вероятностная 
диспетчеризация PAP (probabilistic allocation policy)~\cite[разд.~3]{8-kon} и~пуассоновский поток заданий, 
при которых осуществим аналитический расчет целевых функций~$v$ и~$r$. 
Это обстоятельство, однако, может  быть использовано лишь для контроля 
вы\-чис\-ле\-ний, так как  получаемые таким образом результаты обычно заметно 
хуже пред\-став\-лен\-ных в~табл.~1 и~2.

\vspace*{-6pt}
   
{\small\frenchspacing
 {\baselineskip=10.8pt
 \addcontentsline{toc}{section}{References}
 \begin{thebibliography}{99}
 
 \vspace*{-6pt}
 
   \bibitem{1-kon}
   \Au{Kleinrock L., Muntz~R.\,R.} Processor sharing queueing models of mixed 
scheduling disciplines for time shared system~// J.~ACM, 1972. Vol.~19. Iss.~3. 
P.~464--482.
   \bibitem{2-kon}
   \Au{Aalto S., Ayesta U}. Optimal scheduling of jobs with a DHR tail in the 
$M/G/1$ queue~// 3rd ICST Conference (International) on Performance Evaluation 
Methodologies and Tools Proceedings.~--- Brussels, Belgium: ICST, 2008. doi: 
10.4108/ICST.VALUETOOLS2008.4335.
   \bibitem{3-kon}
   \Au{Wu R., Down D.\,G.} Scheduling multi-server systems using  
foreground-background processing. {\sf 
http://www.\linebreak cas.mcmaster.ca/$\sim$downd/fbpspaper5.pdf}.
   \bibitem{4-kon}
   \Au{Hyyti$\ddot{\mbox{a}}$~E., Righter~R.} Routing jobs with deadlines to 
heterogeneous parallel servers~// Oper. Res. Lett., 2016. Vol.~44. Iss.~4. 
P.~507--513.
   \bibitem{5-kon}
   \Au{Avi-Itzhak B., Halfin~S}. Expected response times in a~non-symmetric time 
sharing queue with a~limited number of service positions~//  12th Teletraffic 
Congress (International) Proceedings.~--- Torino, 1988. Art. ID: 5.4B.2. 7~p.
   \bibitem{6-kon}
   \Au{Schroeder B., Harchol-Balter~M., Iyengar~A., Nahum~E.\,M., Wierman~A.} 
How to determine a~good multi-programming level for external scheduling~//  22nd 
Conference  (International) on Data Engineering Proceedings.~--- Atlanta, GA, USA, 
2006. P.~60.
   \bibitem{7-kon}
   \Au{Коновалов М.\,Г.} Методы адаптивной обработки информации и~их 
приложения.~--- М.: ИПИ РАН, 2007. 212~с.
   \bibitem{8-kon}
   \Au{Коновалов М.\,Г., Разумчик~Р.\,В.} Обзор моделей и~алгоритмов 
размещения заданий в~системах с~параллельным обслуживанием~// 
Информатика и~её применения, 2015. Т.~9. Вып.~4. С.~56--67.
   \bibitem{9-kon}
   \Au{Коновалов М.\,Г., Разумчик~Р.\,В.} Минимизация среднего времени 
пребывания в~ненаблюдаемых сис\-те\-мах с~параллельным обслуживанием 
и~дисциплиной справедливого разделения процессора в~серверах~// 
Информационные процессы, 2019. Т.~19. Вып.~3. С.~327--338.
   \bibitem{10-kon}
   \Au{Konovalov M., Razumchik~R.} Minimizing mean response time in  
non-observable distributed systems with processor sharing nodes~// Commun. 
 ECMS, 2019. Vol.~33. Iss.~1. P.~456--461.

 \end{thebibliography}

 }
 }

\end{multicols}

\vspace*{-9pt}

\hfill{\small\textit{Поступила в~редакцию 10.10.19}}

%\vspace*{8pt}

%\pagebreak

\newpage

\vspace*{-28pt}

%\hrule

%\vspace*{2pt}

%\hrule

%\vspace*{-2pt}

\def\tit{MIXED POLICIES FOR~ONLINE JOB ALLOCATION 
IN~ONE CLASS OF~SYSTEMS WITH~PARALLEL 
SERVICE}


\def\titkol{Mixed policies for~online job allocation 
in~one class of~systems with~parallel 
service}

\def\aut{M.\,G.~Konovalov$^1$ and~R.\,V.~Razumchik$^{1,2}$}

\def\autkol{M.\,G.~Konovalov and~R.\,V.~Razumchik}

\titel{\tit}{\aut}{\autkol}{\titkol}

\vspace*{-11pt}


\noindent
$^1$Institute of Informatics Problems, Federal Research Center ``Computer Science 
and Control'' of the Russian\linebreak
$\hphantom{^1}$Academy of Sciences, 44-2 Vavilov Str., Moscow 
119333, Russian Federation

\noindent
$^2$Peoples' Friendship University of Russia (RUDN University),  
6~Miklukho-Maklaya Str., Moscow 117198, Russian\linebreak
$\hphantom{^1}$Federation

\def\leftfootline{\small{\textbf{\thepage}
\hfill INFORMATIKA I EE PRIMENENIYA~--- INFORMATICS AND
APPLICATIONS\ \ \ 2019\ \ \ volume~13\ \ \ issue\ 4}
}%
 \def\rightfootline{\small{INFORMATIKA I EE PRIMENENIYA~---
INFORMATICS AND APPLICATIONS\ \ \ 2019\ \ \ volume~13\ \ \ issue\ 4
\hfill \textbf{\thepage}}}

\vspace*{3pt}  


    \Abste{Consideration is given to the problem of efficient job allocation in the class of systems with parallel 
service on independently working single-server stations each equipped with the infinite capacity queue. There is one 
dispatcher which routes jobs, arriving one by one, to servers. The dispatcher does not have a queue to store the jobs and, 
thus, the routing decision must be made on the fly. No jockeying between servers is allowed and jobs cannot be rejected. 
For a~job, there is the soft deadline (maximum waiting time in the queue). If the deadline is violated, a~fixed cost is 
incurred and the job remains in the system and must be served. The goal is to find the job allocation policy 
which 
minimizes both the job's stationary response time and probability of job's deadline violation. Based on simulation 
results, it is demonstrated that the goal may be achieved (to some extent) by adopting a mixed policy, i.e. a proper 
dispatching rule and the service discipline in the server.} 

\KWE{parallel service; dispatching policy; service discipline; sojourn time; deadline violation}



\DOI{10.14357/19922264190409} 

%\vspace*{-14pt}

  \Ack
\noindent
The research was partly supported by the Russian 
Foundation for Basic Research (project 18-07-00692).


\vspace*{4pt}

  \begin{multicols}{2}

\renewcommand{\bibname}{\protect\rmfamily References}
%\renewcommand{\bibname}{\large\protect\rm References}

{\small\frenchspacing
 { %\baselineskip=12pt
 \addcontentsline{toc}{section}{References}
 \begin{thebibliography}{99}
 
 \vspace*{-4pt}
 
   \bibitem{1-kon-1}
   \Aue{Kleinrock, L., and R.\,R.~Muntz.} 1972. Processor sharing queueing 
models of mixed scheduling disciplines for time shared system. \textit{J.~ACM} 
19(3):464--482.
   \bibitem{2-kon-1}
   \Aue{Aalto, S., and U.~Ayesta.} 2008. Optimal scheduling of jobs with a~DHR tail 
in the $M/G/1$ queue. \textit{3rd  Conference (International) on 
Performance Evaluation Methodologies and Tools Proceedings}. Brussels, Belgium. doi: 
10.4108/\linebreak
 ICST.VALUETOOLS2008.4335. 
   \bibitem{3-kon-1}
   \Aue{Wu, R., and D.\,G.~Down.} Scheduling multi-server systems using 
foreground-background processing. Available at: {\sf 
http://www.cas.mcmaster.ca/$\sim$downd/fbpspaper5.pdf} (accessed October~3, 
2019).
   \bibitem{4-kon-1}
   \Aue{Hyyti$\ddot{\mbox{a}}$,~E., and R.~Righter.} 2016. Routing jobs with 
deadlines to heterogeneous parallel servers. \textit{Operations Research Lett.} 
44(4):507--513.
   \bibitem{5-kon-1}
   \Aue{Avi-Itzhak, B., and S.~Halfin.} 1989. Expected response times in 
   a~non-symmetric time sharing queue with a~limited number of service positions. 
\textit{12th  Teletraffic Congress (International) Proceedings}. 
Torino. Art. ID:~5.4B.2. 7~p.
   \bibitem{6-kon-1}
   \Aue{Schroeder, B., M.~Harchol-Balter, A.~Iyengar, E.~Nahum, and A.~Wierman.} 
2006. How to determine a good multi-programming level for external scheduling. 
\textit{22nd  Conference (International) on Data Engineering Proceedings}. Atlanta, 
GA. 60.
   \bibitem{7-kon-1}
   \Aue{Konovalov, M.\,G.} 2007. \textit{Metody adaptivnoy obrabotki informatsii i~ikh 
prilozheniya} [Methods of adaptive information processing and their applications]. 
Moscow: IPI RAN. 212 p.
   \bibitem{8-kon-1}
   \Aue{Konovalov, M.\,G., and R.\,V.~Razumchik.} 2015. Obzor modeley 
   i~algoritmov razmeshcheniya zadaniy v~sistemakh s~parallel'nym obsluzhivaniem 
[Methods and algorithms for job scheduling in systems with parallel service: 
A~survey]. \textit{Informatika i~ee Primeneniya~--- Inform. Appl.} 9(4):56--67.
   \bibitem{9-kon-1}
   \Aue{Konovalov, M.\,G., and R.\,V.~Razumchik.} 2019. Mi\-ni\-mi\-za\-tsiya srednego 
vremeni prebyvaniya v~nenablyudaemykh sistemakh s~parallel'nym 
obsluzhivaniem i~dis\-tsip\-li\-noy spravedlivogo razdeleniya protsessora v~ser\-ve\-rakh 
[Minimizing mean response time in non-observable distributed
 processing systems 
with nodes, operating under egalitarian processor sharing policy]. \textit{Informatsionnye 
protsessy} [Information Processes] 19(3):327--338.
   \bibitem{10-kon-1}
   \Aue{Konovalov, M.\,G., and R.\,V.~Razumchik.} 2019. Minimizing mean 
response time in non-observable distributed systems with processor sharing 
nodes. \textit{Commun. ECMS} 33(1):456--461.
\end{thebibliography}

 }
 }

\end{multicols}

%\vspace*{-7pt}

\hfill{\small\textit{Received October 10, 2019}}

%\pagebreak

%\vspace*{-22pt}

   \Contr
   
   \noindent
   \textbf{Konovalov Mikhail G.} (b.\ 1950)~--- Doctor of Science in technology, 
principal scientist, Institute of Informatics Problems, Federal Research Center 
``Computer Science and Control'' of the Russian Academy of Sciences, 44-2~Vavilov 
Str., Moscow 119333, Russian Federation; \mbox{mkonovalov@ipiran.ru}
   
   \vspace*{6pt}
   
   \noindent
   \textbf{Razumchik Rostislav V.} (b.\ 1984)~--- Candidate of Science (PhD) in 
physics and mathematics, leading scientist, Institute of Informatics Problems, Federal 
Research Center ``Computer Science and Control'' of the Russian Academy of 
Sciences, 44-2~Vavilov Str., Moscow 119333, Russian Federation; associate 
professor, Peoples' Friendship University of Russia (RUDN University), 
6~Miklukho-Maklaya Str., Moscow 117198, Russian Federation; 
\mbox{rrazumchik@ipiran.ru} 
\label{end\stat}

\renewcommand{\bibname}{\protect\rm Литература}  