\def\stat{ushakovi}

\def\tit{ВЫХОДЯЩИЕ ПОТОКИ В~ОДНОЛИНЕЙНОЙ СИСТЕМЕ С~ОТНОСИТЕЛЬНЫМ ПРИОРИТЕТОМ$^*$}

\def\titkol{Выходящие потоки в~однолинейной системе с~относительным приоритетом}

\def\aut{В.\,Г.~Ушаков$^{1}$, Н.\,Г.~Ушаков$^2$}

\def\autkol{В.\,Г.~Ушаков, Н.\,Г.~Ушаков}

\titel{\tit}{\aut}{\autkol}{\titkol}

\index{Ушаков В.\,Г.}
\index{Ушаков Н.\,Г.}
\index{Ushakov V.\,G.}
\index{Ushakov N.\,G.}


{\renewcommand{\thefootnote}{\fnsymbol{footnote}} \footnotetext[1]
{Работа выполнена при финансовой поддержке РФФИ (проект 18-07-00678).}}


\renewcommand{\thefootnote}{\arabic{footnote}}
\footnotetext[1]{Факультет вычислительной математики и~кибернетики Московского государственного 
университета им.~М.\,В.~Ломоносова; 
Федеральный исследовательский центр <<Информатика и~управление>>  
Российской академии наук, \mbox{vgushakov@mail.ru}}
\footnotetext[2]{Институт проблем технологии микроэлектроники и~особочистых материалов 
Российской академии наук, Черноголовка;
Норвежский на\-уч\-но-тех\-но\-ло\-ги\-че\-ский университет, Тронхейм, 
\mbox{ushakov@math.ntnu.no}}

%\vspace*{-2pt}




\Abst{Изучена однолинейная система массового обслуживания с~бесконечным числом
 мест для ожидания, произвольным распределением времени обслуживания и~двумя 
 пуассоновскими входящими потоками требований. Требования первого потока 
 обладают относительным приоритетом перед требованиями второго потока. 
 Методом вложенных цепей Маркова исследуется многомерный случайный 
 процесс, компоненты которого~--- число требований каждого 
 приоритета в~системе и~длительность интервала времени между последовательными 
 моментами ухода из системы требований одного приоритета. Найдены 
 конечномерные распределения указанных процессов. 
 В~качестве следствия получены преобразования Лап\-ла\-са--Стилть\-еса 
 одномерных и~двумерных распределений выходящего потока требований каждого 
 приоритета в~стационарном режиме.}

\KW{выходящий поток; относительный приоритет; 
вложенная цепь Маркова; одноканальная система}

\DOI{10.14357/19922264190407} 
  
%\vspace*{1pt}


\vskip 10pt plus 9pt minus 6pt

\thispagestyle{headings}

\begin{multicols}{2}

\label{st\stat}

\section{Введение} 

Одной из важных характеристик функционирования систем массового обслуживания 
служит выходящий из нее после завершения обслуживания поток требований. 
Знание характеристик выходящего потока бывает необходимо при изучении 
сетей обслуживания, в~которых потоки требований в~узлы содержат в~себе часть 
требований, выходящих из других узлов. Другой важной задачей, в~которой 
рассматриваются выходящие потоки, является задача восстановления 
структуры и~параметров сис\-те\-мы по наблюдению за различными ее 
характеристиками (так называемые обратные задачи).

Вероятностные свойства выходящих потоков в~приоритетных системах 
обслуживания изучены пока недостаточно полно. Полученные к~на\-сто\-яще\-му 
времени результаты касаются свойств одно\-мер\-ных распределений интервалов 
между уходами\linebreak из сис\-те\-мы требований различных приоритетов
 (см., например,~[1--3]). 
В~настоящей работе найде\-ны одномерные и~двумерные распределения выходящих 
потоков каждого приоритета в~однолинейной системе обслуживания с~ожиданием, 
двумя пуассоновскими потоками
требований, в~которой требования первого потока (первого приоритета)
 имеют относительный приоритет перед требованиями второго потока.



\section{Обозначения и~определения}



Пусть $a_1$ и~$a_2$~--- интенсивности, а~$B_1(x)$ и~$B_2(x)$~--- 
функции распределения времен обслуживания приоритетных и~неприоритетных 
требований соответственно.  Обозначим
\begin{align*}
  \beta_i(s)&=\int\limits_0^{\infty}e^{-sx}dB(x)\,;\\ 
  \beta_{ij}&=\int\limits_0^{\infty}x^jdB_i(x)\,;\\
   \sigma&=a_1+a_2\,.
\end{align*}
Пусть далее $t_{iN}$~--- момент ухода из системы \mbox{$N$-го} требования приоритета~$i$ 
(нумерация требований производится для каждого приоритета отдельно в~порядке 
их ухода из системы), $t_{i0}\hm=0$, $\tau_{iN}\hm=t_{iN}\hm-t_{i,N-1}$, $L_i(t)$~--- 
число требований в~системе в~момент времени~$t,$
$i\hm=1,2$, $N\hm=1,2,\ldots$

Всюду в~дальнейшем будем считать выполненным условие эргодичности 
$\rho\hm=a_1\beta_{11}\hm+a_2\beta_{21}\hm<1.$
Положим
\begin{multline*}
P_i\left(n_1,n_2,x\right)=\lim\limits_{N\rightarrow\infty}
\mathbf{P}\left(L_1\left(t_{iN}+0\right)=n_1,\right.\\
\left.L_2
\left(t_{iN}+0\right)=n_2,\tau_{iN}<x\right);
\end{multline*}

%\vspace*{-12pt}

\noindent
\begin{multline*}
Q_i\left(n_1,n_2,m_1,m_2,x,y\right)={}\\
{}+
\lim\limits_{N\rightarrow\infty}\mathbf{P}
\left(L_1(t_{iN}+0)=n_1,L_2(t_{iN}+0)=n_2,\right.\\
 L_1(t_{i,N-1}+0)=m_1,L_2(t_{i,N-1}+0)=m_2,\\
\left.\tau_{iN}<x,\tau_{i,N-1}<y\right);
\end{multline*}
$$
p_i\left(z_1,z_2,s\right)=\!
\int\limits_0^{\infty}\!\!e^{-sx}\sum\limits_{n_1=0}^{\infty}\sum\limits_{n_2=0}^{\infty}
z_1^{n_1}z_2^{n_2}d_xP_i\left(n_1,n_2,x\right),
$$

\vspace*{-12pt}

\noindent
\begin{multline*}
q_i\left(w_1,w_2,z_1,z_2,s_1,s_2\right)={}\\
{}=
\int\limits_0^{\infty}\int\limits_0^{\infty}e^{-s_1x}e^{-s_2y}
\sum\limits_{n_1=0}^{\infty}\sum\limits_{n_2=0}^{\infty}
\sum\limits_{m_1=0}^{\infty}\sum\limits_{m_2=0}^{\infty}w_1^{n_1}w_2^{n_2}\times
\\
\times
z_1^{m_1}z_2^{m_2}d_xd_yQ_i\left(n_1,n_2,m_1,m_2,x,y\right);
\end{multline*}
$$
f_i(s)=\lim\limits_{N\rightarrow\infty}
\int\limits_0^{\infty}e^{-sx}d\mathbf{P}(\tau_{iN}<x);
$$

\vspace*{-12pt}

\noindent
\begin{multline*}
g_i(s_1,s_2)={}\\
{}=\!\!\lim\limits_{N\rightarrow\infty}
\int\limits_0^{\infty}\int\limits_0^{\infty}\!\!e^{-s_1x}e^{-s_2y}d_xd_y
\mathbf{P}(\tau_{iN}<x,\tau_{i,N-1}<y).\hspace*{-3.5pt}
\end{multline*}

\section{Предварительные результаты}

В дальнейшем понадобятся некоторые результаты для системы 
массового обслуживания типа $M|G|1|\infty.$
Обозначим~$a$~--- интенсивность входящего потока; $B(x)$~--- 
функцию распределения времени обслуживания; 
$\Pi(x)$~--- функцию распределения периода занятости:
$$
\beta(s)=\int\limits_0^{\infty}e^{-sx}dB(x)\,;\enskip 
\pi(s)=\int\limits_0^{\infty}e^{-sx}d\Pi(x)\,.
$$
Тогда $\pi(s)$ будет единственным решением уравнения 
$\pi(s)\hm=\beta(s\hm+a\hm-a\pi(s)),$ аналитическим в~области $\mathrm{Re}\, s\hm>0.$

Пусть в~начальный момент времени $t\hm=0$  в~сис\-те\-ме~$i$ требований. 
Обозначим $W^{(i)}(x,t)$~--- функцию распределения виртуального 
времени ожидания в~момент времени~$t$; $p^{(i)}(0,t)$~--- вероятность 
свободного состояния системы в~момент времени~$t$; $p^{(i)}(k,v,t)dv$~--- 
вероятность того, что в~момент времени~$t$ в~системе 
$k\hm\geqslant 1$ требований, а~с~начала обслуживания требования, 
находящегося на приборе, прошло время, лежащее в~интервале $(v,v+dv).$
Тогда
\begin{multline}
\label{n1}
\hspace*{-1.80583pt}\int\limits_0^{\infty}\int\limits_0^{\infty}\!\!e^{-sx}e^{-qt}\,d_xW^{(i)}
(x,t)\,dt=\fr{\beta^i(s)}{q-s+a-a\beta(s)}-{}\\
{}-
\fr{s\pi^i(q)}{(q+a-a\pi(q))(q-s+a-a\beta(s))}\,;
\end{multline}


\noindent
\begin{equation*}
%\label{n2}
\int\limits_0^{\infty}e^{-st}p^{(i)}(0,t)dt=\fr{\pi^i(s)}{s+a-a\pi(s)}\,;
\end{equation*}

\vspace*{-18pt}

\noindent
\begin{multline*}
%\label{n3}
\sum\limits_{k=1}^{\infty}z^k\int\limits_0^{\infty}
e^{-st}p^{(i)}(k,v,t)dt={}\\[-2pt]
{}=\fr{(1-B(v))e^{-(s+a-az)v}}{1-z^{-1}\beta(s+a-az)}
\!\left(\!z^i-\fr{(s+a-az)\pi^i(s)}{s+a-a\pi(s)}\!\right).\hspace*{-4.77934pt}
\end{multline*}

\vspace*{-15pt}


\section{Основные результаты}

\vspace*{-6pt}

Основные результаты работы содержатся в~приводимых ниже четырех теоремах.

%\smallskip

\noindent
\textbf{Теорема~1.}\
\textit{Функции $q_2\left(w_1,w_2,z_1,z_2,s_1,s_2\right)$ 
и~$p_2\left(z_1,z_2,s\right)$  определяются соотношениями}:

\vspace*{-6pt}

\noindent
\begin{multline}
\label{n4}
q_2\left(w_1,w_2,z_1,z_2,s_1,s_2\right)={}\\
{}=
w_2^{-1}\beta_2(s_1+\sigma-a_1w_1-a_2w_2)\times{}\\
{}\times
\left(
\vphantom{\fr{s_1+a_2-a_2w_2+a_1-a_1\pi_1(s_1+a_2-a_2w_2)}{s_1+\sigma-a_1\pi_1(s_1+a_2)}}
p_2\left(z_1\pi_1(s_1+a_2-a_2w_2),w_2z_2,s_2\right)-{}\right.\\
{}-
\fr{s_1+a_2-a_2w_2+a_1-a_1\pi_1(s_1+a_2-a_2w_2)}{s_1+\sigma-a_1\pi_1(s_1+a_2)}\times{}\\
\left.{}\times
p_2(z_1\pi_1(s_1+a_2),0,s_2)
\vphantom{\fr{s_1+a_2-a_2w_2+a_1-a_1\pi_1(s_1+a_2-a_2w_2)}{s_1+\sigma-a_1\pi_1(s_1+a_2)}}
\right);
\end{multline}

\vspace*{-20pt}

\noindent
\begin{multline}
\label{n5}
p_2\left(z_1,z_2,s\right)=z_2^{-1}\beta_2\left(s+\sigma-a_1z_1-a_2z_2\right)
\times{}\\
{}\times
\left(
\vphantom{\fr{s_1+a_2-a_2w_2+a_1-a_1\pi_1(s_1+a_2-a_2w_2)}{s_1+\sigma-a_1\pi_1(s_1+a_2)}}
p_2(\pi_1(s+a_2-a_2z_2),z_2,0)-{}\right.\\
{}-
\fr{s+a_2-a_2z_2+a_1-a_1\pi_1(s+a_2-a_2z_2)}{s+\sigma-a_1\pi_1(s+a_2)}\times{}\\
\left.{}\times
p_2(\pi_1(s+a_2),0,0)
\vphantom{\fr{s_1+a_2-a_2w_2+a_1-a_1\pi_1(s_1+a_2-a_2w_2)}{s_1+\sigma-a_1\pi_1(s_1+a_2)}}
\right).
\end{multline}

\vspace*{-4pt}

\noindent
\textit{Функция $p_2\left(z_1,z_2,0\right)$ равна}

\vspace*{-9pt}

\noindent
\begin{multline}
\label{n6}
p_2\left(z_1,z_2,0\right)=\beta_2(\sigma-a_1z_1-a_2z_2)\,\fr{1-\rho}{a_2}
\times{}\\
{}\times \fr{a_2-a_2z_2+a_1-a_1\pi_1(a_2-a_2z_2)}{\beta_2(\sigma-a_2z_2-a_1\pi_1(a_2-a_2z_2))-z_2},
\end{multline}
\textit{а $\pi_1(s)$  есть преобразование Лап\-ла\-са--Стилть\-еса функции распределения периода занятости системы $M|G|1|\infty$ с~интенсивностью входящего
потока~$a_1$ и~функцией распределения времени обслуживания~$B_1(x).$}


\noindent
Д\,о\,к\,а\,з\,а\,т\,е\,л\,ь\,с\,т\,в\,о\,.\ \
Рассматривая два соседних момента ухода требований второго
 приоритета из системы, имеем:
 
 \vspace*{-6pt}
 
 \noindent
\begin{multline*}
P_2(n_1,n_2,x)=\sum\limits_{i_1=0}^{\infty}
\sum\limits_{i_2=1}^{n_2+1}P\left(i_1,i_2,\infty\right)\times{}\\[-1pt]
{}\times \sum\limits_{k_2=0}^{n_2+1-i_2}
\int\limits_0^x \!\! e^{-a_2u}\fr{(a_2u)^{k_2}}{k_2!}\,
d\Pi_1^{*i_1}(u)\!\int\limits_0^{x-u}  \!\!e^{-\sigma u}\fr{(a_1v)^{n_1}}{n_1!}
\times{}\hspace*{-3.60439pt}
\end{multline*}

\noindent
\begin{multline}
{}\times\fr{(a_2v)^{n_2+1-i_2-k_2}}{(n_2+1-i_2-k_2)!}\,dB_2(v)+
\sum\limits_{i=0}^{\infty}P_2(i,0,\infty)\times{}\\
{}\times
\int\limits_0^xG_i(u,n_1,n_2,x-u)d\left(1-e^{-a_2u}\right),
\label{n7}
\end{multline}
где
$G_j(u,n_1,n_2,v)$~--- вероятность того, что первое требование второго 
приоритета покинет систему к~моменту времени~$u\hm+v,$
в~момент его ухода в~системе останется~$n_1$ и~$n_2$ требований первого и~второго 
приоритетов при условии, что это требование поступает
в~момент времени~$u,$ а в~начальный момент
в~системе есть~$j$  требований первого приоритета.  Переходя 
в~\eqref{n7} к~производящим функциям и~преобразованиям Лап\-ла\-са--Стилть\-еса 
и~учитывая~\eqref{n1} и~то, что

\vspace*{-3pt}

\noindent
\begin{multline*}
\sum\limits_{n_1=0}^{\infty}\sum\limits_{n_2=0}^{\infty}
z_1^{n_1}z_2^{n_2}\int\limits_0^{\infty}e^{-sx}
d_xG_i(u,n_1,n_2,x)={}\\
{}=\beta_2(s+\sigma-a_1z_1-a_2z_2)\times\\
\times \int\limits_0^{\infty}\!
\exp\left(-\left(s+a_2-a_2z_2+a_1-a_1\pi_1\left(s+a_2-{}\right.\right.\right.\\
\left.\left.\left.{}-a_2z_2\right)\right)x\right)
d_x W^{(i)}(x,t),
\end{multline*}

\vspace*{-3pt}

\noindent
получаем~\eqref{n5}. Подставляя в~\eqref{n5} $s\hm=0,$ получаем

\vspace*{-3pt}

\noindent
\begin{multline}
\label{n8}
p_2\left(z_1,z_2,0\right)=z_2^{-1}\beta_2(\sigma-a_1z_1-a_2z_2)\times{}\\
{}\times \left(
\vphantom{\fr{s_1+a_2-a_2w_2+a_1-a_1\pi_1(s_1+a_2-a_2w_2)}{s_1+\sigma-a_1\pi_1(s_1+a_2)}}
p_2(\pi_1(a_2-a_2z_2),z_2,0)-{}\right.\\
{}-
\fr{a_2-a_2z_2+a_1-a_1\pi_1(a_2-a_2z_2)}{\sigma-a_1\pi_1(a_2)}\times{}\\
\left.{}\times
p_2\left(\pi_1(a_2),0,0\right)
\vphantom{\fr{s_1+a_2-a_2w_2+a_1-a_1\pi_1(s_1+a_2-a_2w_2)}{s_1+\sigma-a_1\pi_1(s_1+a_2)}}
\right).
\end{multline}

\vspace*{-3pt}

\noindent
Из~\eqref{n8} следует, что
$$
p_2(z_1,z_2,0)=\beta_2\left(\sigma-a_1z_1-a_2z_2\right)\eta\left(z_2\right)\,,
$$
где

\vspace*{-2pt}

\noindent
\begin{multline*}
\eta(z_2)=-\fr{a_2-a_2z_2+a_1-a_1\pi_1(a_2-a_2z_2)}
{z_2-\beta_2(\sigma-a_2z_2-a_1\pi_1(a_2-a_2z_2))}\times{}\\
{}\times
\fr{\beta_2(\sigma-a_1\pi_1(a_2))}{\sigma-a_1\pi_1(a_2)}\,\eta(0)\,.
\end{multline*}
Устремляя в~последнем соотношении~$z_2$ к~единице, находим

\noindent
$$
\eta(0)=\fr{1-\rho}{a_2}\,\fr{\sigma-a_1\pi_1(a_2)}{\beta_2(\sigma-a_1\pi_1(a_2))}\,.
$$
Отсюда следует~\eqref{n6}.

Рассмотрим теперь три последовательных момента ухода из системы требований 
второго приоритета. Имеем

\noindent
\begin{multline*}
Q_2\left(n_1,n_2,m_1,m_2,x,y\right)=P_2\left(m_1,m_2,y\right)\times{}\\
{}\times \sum\limits_{k_2=0}^{\max(0,n_2+1-m_2)}
\int\limits_0^xe^{-a_2u}\fr{(a_2u)^{k_2}}{k_2!}\,d\Pi_1^{*m_1}(u)
\times{}\\
{}\times \int\limits_0^{x-u}e^{-\sigma v}\fr{(a_1v)^{n_1}}{n_1!}
\fr{(a_2v)^{n_2+1-m_2-k_2}}{(n_2+1-m_2-k_2)!}dB_2(v),\\ m_2\geqslant 1\,;
\end{multline*}

\vspace*{-12pt}

\noindent
\begin{multline*}
Q_2\left(n_1,n_2,m_1,0,x,y\right)=
P_2\left(m_1,0,y\right)\times{}\\
{}\times
\int\limits_0^xG_{m_1}\left(u,n_1,n_2,x-u\right)d\left(1-e^{-a_2u}\right).
\end{multline*}
Отсюда следует~\eqref{n4}.

\smallskip

\noindent
\textbf{Теорема~2.}\
\textit{Справедливы следующие соотношения}:
\begin{multline*}
f_2(s)=\beta_2(s)\beta_2
\left(a_1-a_1\pi_1(s)\right)-{}\\
{}-\beta_2(s)
\fr{s+a_1-a_1\pi_1(s)}{s+\sigma-a_1\pi_1(s+a_2)}\times{}\\
{}\times\fr{(1-\rho)\beta_2(\sigma-a_1\pi_1(s+a_2))
\left(\sigma-a_1\pi_1(a_2)\right)}{a_2\beta_2(\sigma-a_1\pi_1(a_2))}\,;
\end{multline*}


\vspace*{-12pt}

\noindent
\begin{multline*}
g_2(s_1,s_2)=\beta_2(s_1)\left(
\vphantom{\fr{s_1+a_1-a_1\pi_1(s_1)}{s_1+\sigma-a_1\pi_1(s_1+a_2)}}
p_2\left(\pi_1(s_1),1,s_2\right)-{}\right.\\
\left.{}-
\fr{s_1+a_1-a_1\pi_1(s_1)}{s_1+\sigma-a_1\pi_1(s_1+a_2)}
p_2(\pi_1(s_1+a_2),0,s_2)\right).
\end{multline*}

\noindent
Д\,о\,к\,а\,з\,а\,т\,е\,л\,ь\,с\,т\,в\,о\ \
 непосредственно вытекает из результатов теоремы~1 и~соотношений
$f_2(s)\hm=p_2\left(1,1,s\right)$ и~$g_2(s_1,s_2)\hm=q_2\left(1,1,1,1,s_1,s_2\right).
$


\smallskip

\noindent
\textbf{Теорема~3.}\ 
\textit{Функции $q_1\left(w_1,w_2,z_1,z_2,s_1,s_2\right)$, 
$p_1\left(z_1,z_2,s\right)$ и~$p_1\left(z_1,z_2,0\right)$  определяются по формулам}:
\begin{multline}
\label{n9}
q_1\left(w_1,w_2,z_1,z_2,s_1,s_2\right)={}\\
{}=
w_1^{-1}\beta_1\left(s_1+\sigma-a_1w_1-a_2w_2\right)\times{}\\
{}\times
\left(p_1\left(z_1w_1,z_2w_2,s_2\right)-
p_1\left(0,z_2w_2,s_2\right)\right)+{}\\
{}+\fr{a_1\beta_1\left(s_1+\sigma-a_1w_1-a_2w_2\right)}{s_1+\sigma-a_2\pi_2(s_1+a_1)}
\times{}\\
{}\times p_1\left(0,z_2\pi_2(s_1+a_1),s_2\right)+{}\\
{}+
w_1^{-1}w_2^{-1}\beta_1\left(s_1+\sigma-a_1w_1-a_2w_2\right)\times{}\\
{}\times
\left(p_1(0,z_2w_2,s_2)-\fr{s_1+\sigma-a_2w_2}
{s+\sigma-a_2\pi_2(s_1+a_1)}\times{}\right.\\
\left.{}\times p_1(0,z_2\pi_2(s_1+a_1),s_2)
\vphantom{\fr{1-\beta_2(s)}{1-\beta_2(s+a_1)}}
\right)
\times{}\\
{}\times
\fr{1}{1-w_2^{-1}\beta_2(s_1+\sigma-a_2w_2)}
\left( \beta_2\left(s_1+\sigma-\right.\right.\\
\left.\left.{}-a_1w_1-a_2w_2\right)-
\beta_2\left(s_1+\sigma-a_2w_2\right)\right)\,;
\end{multline}

%\vspace*{-12pt}

\noindent
\begin{multline}
\label{n10}
p_1\left(z_1,z_2,s\right)=
z_1^{-1}\beta_1\left(s+\sigma-a_1z_1-a_2z_2\right)\times{}\\
{}\times \left(p_1\left(z_1,z_2,0\right)-
\fr{z_2-\beta_2(s+\sigma-a_1z_1-a_2z_2)}{z_2-\beta_2(s+\sigma-a_2z_2)}\times{}\right.\\
\left.{}\times
p_1(0,z_2,0)
\vphantom{\fr{z_2-\beta_2(s+\sigma-a_1z_1-a_2z_2)}{z_2-\beta_2(s+\sigma-a_2z_2)}}
\right)-
\fr{\beta_1(s+\sigma-a_1z_1-a_2z_2)}{s+\sigma-a_2\pi_2(s+a_1)}
\times{}\\
{}\times p_1\left(0,\pi_2(s+a_1),0\right)\times{}\\
{}\times\left(\fr{\beta_2(s+\sigma-a_1z_1-a_2z_2)-\beta_2(s+\sigma-a_2z_2)}
{z_1(z_2-\beta_2(s+\sigma-a_2z_2))}\times{}\right.\\
\left.{}\times \left(s+\sigma-a_2z_2\right)-a_1
\vphantom{\fr{z_2-\beta_2(s+\sigma-a_1z_1-a_2z_2)}{z_2-\beta_2(s+\sigma-a_2z_2)}}
\right);
\end{multline}

\vspace*{-12pt}

\noindent
\begin{multline}
\label{n11}
\left(z_1-\beta_1(\sigma-a_1z_1-a_2z_2)\right)
p_1\left(z_1,z_2,0\right)+{}\\
{}+\left(z_2-\beta_2(\sigma-a_1z_1-a_2z_2)\right)
\fr{\beta_1(\sigma-a_1z_1-a_2z_2)}{z_2-\beta_2(\sigma-a_2z_2)}\times{}
\\
{}\times p_1\left(0,z_2,0\right)=
\fr{\beta_1(\sigma-a_1z_1-a_2z_2)}{\sigma-a_2\pi_2(a_1)}\,p_1\left(0,\pi_2(a_1)\right)\times{}\\
{}\times\left(
\vphantom{\fr{z_2-\beta_2(s+\sigma-a_1z_1-a_2z_2)}{z_2-\beta_2(s+\sigma-a_2z_2)}}
a_1z_1+a_2z_2-\sigma+\left(\sigma-a_2z_2\right)\times{}\right.\\
\left.{}\times
\fr{z_2-\beta_2(\sigma-a_1z_1-a_2z_2)}
{z_2-\beta_2(\sigma-a_2z_2)}\right),
\end{multline}
\textit{где}
\begin{multline}
\label{n12}
\fr{z_2-h_2(a_2-a_2z_2)}{z_2-\beta_2(\sigma-a_2z_2)}\,
p_1(0,z_2,0)={}\\
{}=\fr{p_1(0,\pi_2(a_1),0)}{\sigma-a_2\pi_2(a_1)}
\left(
\vphantom{\fr{z_2-\beta_2(s+\sigma-a_1z_1-a_2z_2)}{z_2-\beta_2(s+\sigma-a_2z_2)}}
a_1\pi_1\left(a_2-a_2z_2\right)+
a_2z_2-\sigma+{}\right.\\
\left.{}+\left(a_2z_2-\sigma\right)
\fr{h_2(a_2-a_2z_2)-z_2}
{z_2-\beta_2(\sigma-a_2z_2)}\right);
\end{multline}

\vspace*{-12pt}

\noindent
\begin{gather*}
p_1\left(0,\pi_2(a_1),0\right)=
a_1^{-1}(1-\rho)\left(\sigma-a_2\pi_2(a_1)\right);\\
 h_2(s)=\beta_2\left(s+a_1-a_1\pi_1(s)\right),
\end{gather*}
\textit{а $\pi_2(s)$  есть преобразование Лап\-ла\-са--Стилть\-еса функции распределения периода занятости системы $M|G|1|\infty$ с~интенсивностью входящего
потока~$a_2$ и~функцией распределения времени обслуживания~$B_2(x).$}

\smallskip

\noindent
Д\,о\,к\,а\,з\,а\,т\,е\,л\,ь\,с\,т\,в\,о\,.\ \
 Рассматривая два соседних момента ухода требований первого приоритета 
 из системы, имеем:
 
 \noindent
\begin{multline*}
P_1\left(n_1,n_2,x\right)=
\sum\limits_{i_1=1}^{n_1+1}\sum\limits_{i_2=0}^{n_2}P_1(i_1,i_2,\infty)\times{}\\
{}\times
\int\limits_0^xe^{-\sigma u}
\fr{(a_1u)^{n_1-i_1+1}}{(n_1-i_1+1)!}\,
\fr{(a_2u)^{n_2-i_2}}{(n_2-i_2)!}\,dB_1(u)+{}\\
{}+
\sum\limits_{i_2=0}^{n_2}P_1(0,i_2,\infty)\int\limits_0^x
\sum\limits_{k_2=1}^{n_2+1}\int\limits_0^u
p^{(i_2)}\left(k_2,v,u\right)\times{}\\
{}\times \int\limits_0^{x-u}e^{-\sigma\tau}
\fr{(a_1\tau)^{n_1}}{n_1!}\,
\fr{(a_2\tau)^{n_2-k_2+1}}{(n_2-k_2+1)!}\times{}
\end{multline*}

%\vspace*{-12pt}

\noindent
\begin{multline}
{}\times
d\left(1-e^{-a_1u}\right)d_{\tau}\left(B_2^{(v)}*B_1(\tau)\right)dv+{}\\
{}+
\sum\limits_{i_2=0}^{n_2}P_1(0,i_2,\infty)\int\limits_0^x p^{(i_2)}(0,u)
\int\limits_0^{x-u}e^{-\sigma\tau}\times{}\\
{}\times\fr{(a_1\tau)^{n_1}}{n_1!}\,\fr{(a_2\tau)^{n_2}}{(n_2)!}\,
d\left(1-e^{-a_1u}\right)dB_1(\tau),
\label{n13}
\end{multline}
где
$$
B_2^{(v)}(x)=\fr{B_2(x+v)-B_2(v)}{1-B_2(v)},
$$
а функции $p^{(i_2)}(k_2,v,u)$ и~$p^{(i_2)}(0,u)$ вычисляются при $a\hm=a_2$ 
и~$B(x)\hm=B_2(x).$

Переходя в~\eqref{n13} к~производящим функциям и~преобразованиям Лап\-ла\-са--Стилть\-еса, 
получаем~\eqref{n10}.
Подставляя в~\eqref{n10} $s\hm=0,$ получаем~\eqref{n11}. При $z_1\hm=\pi_1(a_2\hm-a_2z_2)$ 
первое слагаемое в~левой
час\-ти~\eqref{n11} обращается в~нуль. Отсюда следует~\eqref{n12}. 
Устремляя в~\eqref{n12} $z_2\hm\rightarrow 1,$
находим $p_1(0,\pi_2(a_1),0)\hm=a_1^{-1}(1-\rho)(\sigma\hm-a_2\pi_2(a_1)).$

Рассмотрим три последовательных момента ухода из системы требований 
первого приоритета. Имеем:

\noindent
\begin{multline*}
Q_1\left(n_1,n_2,m_1,m_2,x,y\right)=P_1\left(m_1,m_2,y\right)\times{}\\
{}\times
\int\limits_0^xe^{-\sigma u}
\fr{(a_1u)^{n_1-m_1+1}}{(n_1-m_1+1)!}\,\fr{(a_2u)^{n_2-m_2}}{(n_2-m_2)!}\,dB_1(u)\\
\mbox{при } m_1\geqslant 1,\ n_1\geqslant m_1-1\,\ n_2\geqslant m_2;
\end{multline*}

\vspace*{-12pt}

\noindent
\begin{multline*}
Q_1\left(n_1,n_2,0,m_2,x,y\right)=
P_1\left(0,m_2,y\right)\times{}\\
{}\times \int\limits_0^x\sum\limits_{k_2=1}^{n_2+1}
\int\limits_0^u p^{(m_2)}\left(k_2,v,u\right)
\int\limits_0^{x-u}e^{-\sigma\tau}\fr{(a_1\tau)^{n_1}}{n_1!}\times{}\\
{}\times
\fr{(a_2\tau)^{n_2}}{n_2!}\,d\left(1-e^{-a_1u}\right)d_\tau\left(B_2^{(v)}*B_1(\tau)\right)
dv+{}\\
{}+
P_1\left(0,m_2,y\right)\int\limits_0^x p^{(m_2)}(0,u)
\int\limits_0^{x-u}e^{-\sigma\tau}
\fr{(a_1\tau)^{n_1}}{n_1!}\times{}\\
{}\times
\fr{(a_2\tau)^{n_2}}{n_2!}\,d\left(1-e^{-a_1u}\right)
dB_1(\tau)\\ 
\mbox{при}\ n_1\geqslant 0,\ n_2\geqslant 0,\ m_2\geqslant 0,
\end{multline*}

\vspace*{-12pt}

\noindent
\begin{multline*}
Q_1\left(n_1,n_2,m_1,m_2,x,y\right)=0 \\
 \mbox{при остальных}\ n_1,\ n_2,\ m_1,\ m_2.
\end{multline*}
Переходя в~этих соотношениях к~преобразования Лап\-ла\-са--Стилть\-еса 
и~производящим функциям, получаем~\eqref{n9}.

%\pagebreak

%\smallskip

\noindent
\textbf{Теорема~4.}\
\textit{Справедливы следующие соотношения}:
\begin{multline*}
f_1(s)=\beta_1(s)\left(1-\fr{1-\beta_2(s)}{1-\beta_2(s+a_1)}\times{}\right.\\
\left.{}\times
\left(1-\rho+a_1^{-1}a_2\left(1-\beta_2(a_1)\right)+
(1-\rho)\times{}\right.\right.
\\
{}\times
\fr{a_1(1-\beta_2(s))-s(\beta_2(s)-\beta_2(s+a_1))}
{(1-\beta_2(s+a_1))(s+\sigma-a_2\pi_2(s+a_1))}\times{}\\
{}\times 
\left(a_1^{-1}
\fr{\pi_2(s+a_1)-\beta_2(\sigma-a_2\pi_2(s+a_1))}
{\pi_2(s+a_1)-h_2(a_2-a_2\pi_2(s+a_1))}\times{} \right.
\\
{}\times
\left(a_1\pi_1\left(a_2-a_2\pi_2(s+a_1)\right)+a_2\pi_2(s+a_1)-\sigma\right)+{}\\
\left.\left.{}+a_1^{-1}\left(\sigma-a_2\pi_2(a_1)\right)
\vphantom{\fr{1-\beta_2(s)}{1-\beta_2(s+a_1)}}
\right)\right);
\end{multline*}

\vspace*{-22pt}

\noindent
\begin{multline*}
g_1(s_1,s_2)=\beta_1(s_1)\left(
\vphantom{\fr{1-\beta_2(s_1)}{1-\beta_2(s_1+a_1)}}
p_1(1,1,s_2)-{}\right.\\
{}-
\fr{1-\beta_2(s_1)}{1-\beta_2(s_1+a_1)}
p_1\left(0,1,s_2\right)+{}
\end{multline*}

\noindent
\begin{multline*}
{}+
\fr{p_1(0,\pi_2(s_1+a_1),s_2)}
{s_1+\sigma-a_2\pi_2(s_1+a_1)}\times{}\\[6pt]
\left.{}\times
\fr{a_1(1-\beta_2(s_1))-s_1\left(\beta_2(s_1)-\beta_2(s_1+a_1)\right)}
{1-\beta_2\left(s_1+a_1\right)}\right).
\end{multline*}

\vspace*{-16pt}

{\small\frenchspacing
 {%\baselineskip=10.8pt
 \addcontentsline{toc}{section}{References}
 \begin{thebibliography}{9}
\bibitem{1-us}
\Au{Nain P.} 
Interdeparture times from a queuing system with preemptive resume priority~// 
Perform. Evaluation, 1984. Vol.~4. Iss.~2. P.~93--98.
\bibitem{2-us}
\Au{Stanford D. A.} Interdeparture time distributions in the 
non-preemptive priority $\Sigma\ M_i|G_i|1$ queue~// Perform. Evaluation, 1991. 
Vol.~12. Iss.~2.   P.~43--60.
\bibitem{3-us}
\Aue{Stanford D.\,A.} Waiting and interdeparture times in priority queues with 
Poisson and general arrival streams~// Oper.
Res., 1995. Vol.~45. Iss.~5. P.~725--735.

 \end{thebibliography}

 }
 }

\end{multicols}

\vspace*{-12pt}

\hfill{\small\textit{Поступила в~редакцию 12.09.19}}

\vspace*{6pt}

%\pagebreak

%\newpage

%\vspace*{-28pt}

\hrule

\vspace*{2pt}

\hrule

\vspace*{-4pt}

\def\tit{THE OUTPUT STREAMS IN~THE~SINGLE SERVER QUEUEING SYSTEM WITH~A~HEAD 
OF~THE~LINE PRIORITY\\[-5pt]}


\def\titkol{The output streams in~the~single server queueing system with~a~head 
of~the~line priority}

\def\aut{V.\,G.~Ushakov$^{1,2}$ and N.\,G.~Ushakov$^{3,4}$\\[-5pt]}

\def\autkol{V.\,G.~Ushakov and N.\,G.~Ushakov}

\titel{\tit}{\aut}{\autkol}{\titkol}

\vspace*{-24pt}


\noindent
$^1$Department of Mathematical Statistics, Faculty of Computational 
Mathematics and Cybernetics, M.\,V.~Lomo-\linebreak
$\hphantom{^1}$nosov Moscow State University, 
1-52~Leninskiye Gory, Moscow 119991, GSP-1, Russian Federation

\noindent
$^2$Institute of Informatics Problems, Federal Research Center ``Computer Science 
and Control'' of the Russian \linebreak
$\hphantom{^1}$Academy of Sciences, 44-2~Vavilov Str., 
Moscow 119333, Russian Federation

\noindent
$^3$Institute of Microelectronics Technology and High-Purity Materials of the 
Russian Academy of Sciences,\linebreak
$\hphantom{^1}$6~Academician Osipyan Str., Chernogolovka, 
Moscow Region 142432, Russian Federation

\noindent
$^4$Norwegian University of Science and Technology, 
15A~S.\,P.~Andersensvei, Trondheim 7491, Norway

\def\leftfootline{\small{\textbf{\thepage}
\hfill INFORMATIKA I EE PRIMENENIYA~--- INFORMATICS AND
APPLICATIONS\ \ \ 2019\ \ \ volume~13\ \ \ issue\ 4}
}%
 \def\rightfootline{\small{INFORMATIKA I EE PRIMENENIYA~---
INFORMATICS AND APPLICATIONS\ \ \ 2019\ \ \ volume~13\ \ \ issue\ 4
\hfill \textbf{\thepage}}}

\vspace*{3pt}  


 

\Abste{The paper studies a single server queuing system with two types of 
customers, head of the line priority, and an infinite number of positions in the queue.  The arrival stream of customers of each type is a Poisson stream.
Each type has its own generally distributed service time characteristics. 
The main result is the Laplace--Stieltjes
transform  of one- and two-dimensional stationary distribution functions 
of the interdeparture time for each type of
customers.
The analysis of the output process is carried out
 by the method of embedded Markov chains. As embedded times,
  successive moments of the end of service of the same type of 
  customers are selected. From the practical perspective, an accurate 
  characterization of the interdeparture time process is necessary
   when studying open networks of queues.}

\KWE{output stream; head of the line priority; embedded Markov chain; single server}





  \DOI{10.14357/19922264190407} 

\vspace*{-22pt}

\Ack
\noindent
The reported study was funded by the Russian Foundation for Basic
Research (project number 18-07-00678).



\vspace*{-6pt}

  \begin{multicols}{2}

\renewcommand{\bibname}{\protect\rmfamily References}
%\renewcommand{\bibname}{\large\protect\rm References}

{\small\frenchspacing
 {%\baselineskip=10.8pt
 \addcontentsline{toc}{section}{References}
 \begin{thebibliography}{9}

\bibitem{1-us-1}
\Aue{Nain, P.} 1984. Interdeparture times from 
a~queuing system with preemptive resume priority. 
\textit{Perform. Evaluation} 4(2):93--98.

\bibitem{2-us-1}
\Aue{Stanford, D.\,A.} 1991. Interdeparture time distributions in the 
non-preemptive priority $\Sigma\ M_i|G_i|1$ queue. 
\textit{Perform. Evaluation} 12(2):43--60.

\bibitem{3-us-1}
\Aue{Stanford, D.\,A.} 1995. Waiting and interdeparture times in priority
 queues with Poisson and general arrival streams. 
 \textit{Oper. Res.} 45(5):725--735.
\end{thebibliography}

 }
 }

\end{multicols}

\vspace*{-7pt}

\hfill{\small\textit{Received September 12, 2019}}

%\pagebreak

%\vspace*{-22pt}

\Contr

\noindent
\textbf{Ushakov Vladimir G.} (b.\ 1952)~--- 
Doctor of Science in physics and mathematics, professor, Department of 
Mathematical Statistics, Faculty of Computational Mathematics and Cybernetics, 
M.\,V.~Lomonosov Moscow State University, 1-52~Leninskiye Gory, 
Moscow 119991, GSP-1, Russian Federation; senior scientist, Institute 
of Informatics Problems, Federal Research Center ``Computer Science 
and Control'' of the Russian Academy of Sciences, 44-2~Vavilov Str., 
Moscow 119333, Russian Federation; \mbox{vgushakov@mail.ru}

\vspace*{6pt}

\noindent
\textbf{Ushakov Nikolai G.} (b.\ 1952)~--- 
Doctor of Science in physics and mathematics, leading scientist, 
Institute of Microelectronics Technology and High-Purity Materials 
of the Russian Academy of Sciences, 6~Academician Osipyan Str., Chernogolovka, 
Moscow Region 142432, Russian Federation; professor, Norwegian University
 of Science and Technology, 15A~S.\,P.~Andersensvei, Trondheim 7491, 
 Norway; \mbox{ushakov@math.ntnu.no}
\label{end\stat}

\renewcommand{\bibname}{\protect\rm Литература}  