\newcommand {\col}{\mathop{\mathrm{col}}}
\newcommand{\me}[2]{\mathbf{E}_{ #1 }\left\{ \mathop{#2} \right\} }

%\newcommand {\s}{^{(s)}}



\def\stat{borisov}

\def\tit{ЧИСЛЕННЫЕ СХЕМЫ ФИЛЬТРАЦИИ МАРКОВСКИХ СКАЧКООБРАЗНЫХ ПРОЦЕССОВ 
ПО~ДИСКРЕТИЗОВАННЫМ НАБЛЮДЕНИЯМ I:~ХАРАКТЕРИСТИКИ ТОЧНОСТИ$^*$}

\def\titkol{Численные схемы фильтрации МСП %марковских скачкообразных процессов 
по~дискретизованным наблюдениям I:~характеристики точности}

\def\aut{А.\,В.~Борисов$^1$}

\def\autkol{А.\,В.~Борисов}

\titel{\tit}{\aut}{\autkol}{\titkol}

\index{Борисов А.\,В.}
\index{Borisov A.\,V.}



{\renewcommand{\thefootnote}{\fnsymbol{footnote}} \footnotetext[1]
{Работа выполнена при частичной поддержке РФФИ (проект 19-07-00187~А).}}


\renewcommand{\thefootnote}{\arabic{footnote}}
\footnotetext[1]{Институт проблем информатики Федерального исследовательского центра 
<<Информатика и~управление>> Российской академии наук,
\mbox{aborisov@frccsc.ru}}

%\vspace*{-2pt}



\Abst{Статья является первой частью цикла, посвященного проблеме численного 
решения задачи оптимальной фильтрации состояний марковских скачкообразных 
процессов (МСП) по наблюдениям в~присутствии аддитивных и~мультипликативных 
винеровских шумов. Данная задача решается путем временной 
дискретизации наблюдений и~их последующей обработки. Как оптимальные,
 так и~субоптимальные оценки в~этом случае выражаются через многомерные 
 интегралы гауссовских плотностей по некоторым смешивающим распределениям. 
 В~данной работе исследуется влияние точности схем численного интегрирования 
 на качество получаемых приближенных оценок.  Задача сводится к~характеризации 
 близости случайных последовательностей, по\-рож\-да\-емых некоторыми рекуррентными 
 соотношениями. В~статье представлена псевдометрика, описывающая это 
 расстояние, а также доказано утверждение о ее влиянии на локальную и~глобальную 
 точность аппроксимации решения исходной задачи фильтрации.} 


\KW{марковский скачкообразный процесс; оптимальная фильтрация; 
аддитивные и~мультипликативные шумы в~наблюдениях; стохастическое 
дифференциальное уравнение; аналитическая и~численная аппроксимация}

\DOI{10.14357/19922264190411} 
  
%\vspace*{1pt}


\vskip 10pt plus 9pt minus 6pt

\thispagestyle{headings}

\begin{multicols}{2}

\label{st\stat}




 \section{Введение}
 
 Задача фильтрации состояний \textit{марковского скачкообразного процесса}  
 по косвенным непрерывным зашумленным наблюдениям была решена в~\cite{Wonham_65} для 
 класса систем наблюдения с~аддитивными винеровскими шумами. Это означает, 
 что интенсивность шумов в~наблюдениях является лишь детерминированной функцией 
 времени. В~\cite{B_18} представле\-но обобщение этого результата на наблюдения 
 с~мультипликативными шумами: их интенсивность теперь является функцией оцениваемого 
 состояния системы. В~обоих случаях системы стохастических дифференциальных уравнений, 
 опи\-сы\-ва\-ющих фильтр, относятся к~классу уравнений Куш\-не\-ра--Стра\-то\-но\-ви\-ча, 
 проблемному с~точки зрения их численного решения. Дело в~том, что эти 
 нелинейные уравнения описывают эволюцию во времени условного распределения 
 состояния сис\-те\-мы по имеющимся наблюдениям. Стандартные чис\-лен\-ные схемы их 
 решения~\cite{KP_92} могут терять свойства нормировки и~неотрицательности. 
 В~статье \cite{B_18_IA} была предложена концепция построения 
 численных методов путем перехода к~дискретизованным по времени 
 наблюдениям и~последующему решению на их основе задачи оптимальной фильтрации. 
  Предложенные оценки вычисляются рекурсивно как некоторая дробь, 
  числитель и~знаменатель которой представляют собой бесконечные суммы 
  интегралов от гауссовских плотностей по некоторым смешивающим распределениям. 
  Было предложено ограничить суммирование в~числителе и~знаменателе некоторым 
  порогом, назвав такие приближения \textit{аналитическими} аппроксимациями. 
  Для них в~\cite{B_18_IA} были получены показатели локальной и~глобальной точности,
   т.\,е.\ определена величина расхождения оптимальной оценки и~ее аппроксимации 
   за один или несколько шагов рекурсии.
К~сожалению, для аналитических аппроксимаций не существует явных формул,
 определяющих значение интегралов~--- слагаемых в~числителе и~знаменателе.

Целью данного цикла является исследование влияния замены интегралов 
их приближениями  в~виде конечных сумм на общую точность аппроксимации, 
а~также сравнительный  анализ различных численных схем, используемых при 
решении задачи фильтрации по наблюдениям как с~аддитивными, так 
и~с~мультипликативными шумами. 
Первая статья посвящена определению влияния точности схем численного 
интегрирования на итоговую точность приближения задачи оптимальной 
фильтрации состояний МСП по дискретизованным наблюдениям. Статья 
организована сле\-ду\-ющим образом. В~разд.~2 дана постановка задачи 
оптимальной фильтрации. Совокупность определений и~результатов работы~\cite{B_18_IA}, 
используемых в~данной статье, представлена в~разд.~3. В~разд.~4 
предлагается аппроксимировать интегралы в~аналитических аппроксимациях 
с~по\-мощью интегральных сумм общего вида. Подобные аппроксимации названы 
\textit{численными}. Предложены локальный и~глобальный показатели близости 
аналитических и~численных аппроксимаций: они определяют величину 
расхождения аналитической и~численной аппроксимации за один или несколько шагов. 
Основное утверждение статьи позволяет оценить эти показатели с~помощью 
аналитической характеристики~--- интегральной ошибки. Эта же оценка дает
 возможность определить точность решения исходной задачи фильтрации с~учетом ошибок, вносимых аналитической аппроксимацией и~ее численным приближением
с помощью той или иной численной схемы. Заключительные замечания 
представлены в~разд.~5.

 \section{Постановка задачи фильтрации}
 
 На вероятностном базисе с~фильтрацией\linebreak 
 $(\Omega^X \times \Omega^W,\mathcal{F}^X \times \mathcal{F}^W,
 \mathcal{P}^X \times \mathcal{P}^W, \{\mathcal{F}^X_t 
 \times \mathcal{F}^W_t\}_{t \geqslant 0})$ рас\-смат\-ри\-ва\-ет\-ся система наблюдения
\begin{align*}
 \displaystyle X_t &=X_0 + \int\limits_0^t \Lambda^{\top}X_{s}\,ds + \mu_t;  \\
 \displaystyle Y_r &= \int\limits_{(r-1)h}^{rh} \!\! fX_s\,ds+\!
 \int\limits_{(r-1)h}^{rh} \sum\limits_{n=1}^NX_s^ng_n^{1/2}\, dW_s,\enskip 
 r \in \mathbb{N},
 % \label{eq:obsys_1}
 \end{align*}
 где
  \begin{itemize}
  \item
  $X_t  \eqd \col\left(X_t^1,\ldots,X_t^N\right) \hm\in \mathbb{S}^N$~--- 
  не\-на\-блю\-да\-емое состояние системы~--- однородный МСП с~конечным множеством 
  состояний $ \mathbb{S}^N \hm \eqd \{e_1,\ldots,e_N\}$ ($\mathbb{S}^N$~--- 
  множество единичных векторов евклидова пространства~$\mathbb{R}^N$), 
  матрицей интенсивностей переходов~$\Lambda$ и~начальным распределением~$\pi$;
  \item
  $\mu_t \eqd \col\left(
  \mu_t^1,\ldots,\mu_t^N\right)\hm\in \mathbb{R}^N$~--- $\mathcal{F}_t$-со\-гла\-со\-ван\-ный 
  мартингал;
  \item
  $\{Y_r\}_{r \in \mathbb{N}}:\;  Y_r \eqd \col\left(Y_r^1,\ldots,Y_r^M\right) 
  \hm\in \mathbb{R}^M$~--- последовательность дискретизованных наблюдений, 
  доступных в~известные равноотстоящие моменты времени~$\{rh\}_{r \in \mathbb{N}}$;
 \item
 $W_t \eqd \col\left(W_t^1,\ldots,W_t^M\right) \hm\in \mathbb{R}^M$ является 
 $\mathcal{F}_t$-со\-гла\-со\-ван\-ным стандартным винеровским процессом,  $f$~--- 
 $(M \times N)$-мер\-ная матрица, 
 а~$\{g_n\}_{n=\overline{1,N}}$~--- симметрические положительно определенные 
 матрицы; процессы~$X$ и~$W$ независимы.
  \end{itemize}
  
   \textit{Задача оптимальной фильтрации состояния~$X$ 
   по дискретизованным наблюдениям~$Y$} заключается в~нахождении 
   \textit{условного математического ожидания} (УМО)
  \begin{equation}
  \widehat{X}_r \eqd \me{}{X_{t_r}|\mathcal{O}_{r} },
  \label{eq:fest_1}
  \end{equation}
  где $\mathcal{O}_r \eqd \sigma\{ Y_{\ell}: \; 1 \leqslant \ell \leqslant r\}$~--- 
  $\sigma$-ал\-геб\-ра, по\-рож\-ден\-ная наблюдениями, полученными до момента времени~$rh$ 
  включительно; $\mathcal{O}_0 \eqd \{\varnothing,\; \Omega\}$.
  
Необходимость нахождения оценки~(\ref{eq:fest_1}) очевидным образом возникает 
при численной реализации оптимальной фильтрации состояний МСП по непрерывным 
наблюдениям в~присутствии винеровских шумов. 
С одной стороны, измерительная информация практически во всех реальных 
системах формируется в~дискретные моменты времени, а~модели с~непрерывными 
наблюдениями являются лишь удобной, хотя и~адекватной, идеализацией. 
С~другой стороны, средства вычислительной техники также способны обрабатывать 
только сигналы, дискретизованные по времени. Поэтому можно считать, что в~процесс 
фильтрации состояний МСП по непрерывным наблюдениям в~качестве первого необходимого 
шага включена процедура временн$\acute{\mbox{о}}$й 
дискретизации наблюдений, или изначально наблюдения поступают в~дискретизованном виде.

%\vspace*{-4pt}

 \section{Необходимые сведения об~оптимальном решении и~аналитических аппроксимациях}
 
 %\vspace*{-3pt}
 
 Рекуррентные соотношения, определяющие оценку~(\ref{eq:fest_1}), были 
 получены в~статье~\cite{B_18_IA}. В~данном разделе представлены те ее 
 термины и~результаты, которые будут использованы в~настоящей статье 
 для определения точности предлагаемых численных аппроксимаций. 
 Эти обстоятельства объясняют появление рассматриваемой задачи оце\-ни\-вания.
 {\looseness=1
 
 }
 
  
   Пусть $N_r^X(\omega)$~--- число скачков процесса~$X$, произошедших на отрезке 
   $[(r-1)h,rh]$, а $\tau_r \eqd \int\nolimits_0^t X_sds$~--- 
   случайный вектор времени пребывания процесса~$X$ в~различных состояниях на 
   отрезке $[(r-1)h,rh]$.
  
  Оптимальная оценка~$\widehat{X}_r$~(\ref{eq:fest_1}) 
  определяется рекуррентной процедурой~\cite{B_18_IA}
  \begin{equation*}
  \widehat{X}_0 = \pi\,;
  %\label{eq:init}
  \end{equation*}
  
% \vspace*{-12pt}
 
  \noindent
  \begin{multline} 
  \widehat{X}_r^{j} = \left(
\sum\limits_{m=0}^{\infty}
    \sum\limits_{n=1}^N \int\limits_{\mathcal{D}} 
 \mathcal{N}\left(Y_{r},f u,\sum\limits_{p=1}^N u^p g_p\right)
  \widehat{X}_{r-1}^n\times{}\right.\\[3pt]
\left. {}\times  
 \rho^{n,j,m}(du)\right)\!\!\Bigg/ \!\!
 \left( 
 \sum\limits_{\ell=0}^{\infty}\sum\limits_{k,i=1}^N \int\limits_{\mathcal{D}} 
 \!\mathcal{N}\!\left(\!Y_{r},f v,\sum\limits_{q=1}^N v^q g_q\right)\times{}\right.\hspace*{-1.15286pt}\\[3pt]
 \left.{}\times  \widehat{X}_{r-1}^k
 \rho^{k,i,\ell}(dv)
  \vphantom{\int\limits_{\mathcal{D}}}
 \right), \enskip j=\overline{1,N}\,,
  \label{eq:filt_1}
 \end{multline}

 
 \noindent
 где 
 \begin{itemize}
 \item
 $\mathcal{D} \eqd \{t=\col\left(t^1,\ldots,t^N\right) \hm\in 
 \mathbb{R}^N: t^n \hm\geqslant 0, \; n\hm=\overline{1,N}, \;\sum\nolimits_{n=1}^N t^n \hm= h\}$~--- 
 носитель распределения вектора $\tau_r$;
 \item
 $ \mathcal{N}(y,m,K) \eqd (2\pi)^{-M/2} \mathrm{det}^{-1/2} K \times{}$\linebreak
 $\times
 \exp\left\{ -\|y-m)\|^2_{K^{-1}}/2\right\}$~--- 
 $M$-мер\-ная плот\-ность гауссовского распределения с~математическим ожиданием~$m$ 
 и~не\-вы\-рож\-ден\-ной ковариационной матрицей~$K$;
 \item
 $\rho^{n,j,m}(\cdot)$~--- распределение вектора 
 $\tau_{r}X_{t_{r}}^{j}\mathbf{I}_{\{m\}}(N_{r}^X)$ при условии $X_{t_{r-1}}\hm=e_k$, 
 т.\,е.\ 
 для любого $\mathcal{G}\hm \in \mathcal{B}(\mathbb{R}^M)$ верно равенство:
\begin{multline*}
\me{}{\mathbf{I}_{\mathcal{G}}
\left(\tau_r\right)X_{t_r}^j\mathbf{I}_{\{m\}}\left(N_r^X\right)|X_{t_{r-1}}=e_k}
={}\\[3pt]
{}=
  \int\limits_{\mathcal{G}} \rho^{k,j,m}(du).
\end{multline*}
 \end{itemize}
Формула~(\ref{eq:filt_1}) вычисления оптимальной оценки на каждом шаге
представляет собой обобщенный вариант формулы Байеса со счетным набором 
гипотез $H_r^m \hm= \{\omega \in \Omega: \; N^X_r(\omega)\hm=m\}$.
Она содержит в~числителе и~знаменателе бесконечные суммы, 
которые не могут быть вычислены аналитически. Поэтому были предложены 
\textit{аналитические} аппроксимации $\overline{X}_r(s,Y_1,\ldots,Y_r)$ 
порядка~$s$ (далее в~тексте зависимость оценок и~их аппроксимаций от порядка~$s$ 
и~наблюдений~$Y$ будет опущена в~тех местах, где это не мешает изложению материала):
  \begin{align}
  \overline{X}_0 &=\pi\,; \notag\\[3pt]
\overline{X}_r^{j} &= \left(\sum\limits_{m=0}^{s}\sum\limits_{n=1}^N 
\int\limits_{\mathcal{D}} 
 \mathcal{N}\left(Y_{r},f u,\sum\limits_{p=1}^N u^p g_p\right) \overline{X}_{r-1}^n\times{}\right.\notag\\[3pt]
&\hspace*{-15pt}\left.{}\times    \rho^{n,j,m}(du)\right)\!\!\Bigg /\!\!
\left(
\sum\limits_{\ell=0}^{s}\sum\limits_{k,i=1}^N \int\limits_{\mathcal{D}} 
 \!\mathcal{N}\!\left(\!Y_{r},f v,\sum\limits_{q=1}^N v^q g_q\right)\times{}\right.\notag\\[3pt]
&\hspace*{15mm}\left. {}\times  \overline{X}_{r-1}^k
 \rho^{k,i,\ell}(dv)
  \vphantom{\int\limits_{\mathcal{D}}}
  \right), \enskip j=\overline{1,N}.
  \label{eq:filt_1_1}
 \end{align}
 
 
 Оценки, обладающие п.~н.\ неотрицательными компонентами и~удовлетворяющие 
 условиям нормировки, называются \textit{устойчивыми}. Легко видеть, что 
 оценка~$\overline{X}_r$~(\ref{eq:filt_1_1}) относится к~этому классу.
 
 Далее в~статье предполагается, что шаг~$h$ и~порядок аппроксимации~$s$ 
 выбраны таким образом, что 
 $$
 \fr{(\overline{\lambda}h)^{s+1}}{(s+1)!} < 
\fr{1}{2}\,,
$$ 
где $\overline{\lambda} \eqd \max\nolimits_{1 \leqslant n 
 \leqslant N}|\lambda_{n n}|$.
 
 Для предложенных аппроксимаций верны неравенства, характеризующие 
 локальный (одношаговый) и~глобальный (многошаговый) показатели точности:
 \begin{equation}
 \sigma(s) \eqd \sup\limits_{\pi \in \Pi}\me{}{\|\widehat{X}_{1} - \overline{X}_{1}\|_{1}} 
 \leqslant
 2 \fr{(\overline{\lambda}h)^{s+1}}{(s+1)!}\,;
 \label{eq:analyt_loc}
 \end{equation}
 
 \vspace*{-12pt}
 
 \noindent
  \begin{multline}
\Sigma_r(s) \eqd \sup\limits_{\pi \in \Pi}\me{}{\|\widehat{X}_{r} - \overline{X}_{r}\|_{1}} 
\leqslant{}\\
{}\leqslant
 2-2\left(1-\fr{(\overline{\lambda}h)^{s+1}}{(s+1)!}\right)^r ,
 \label{eq:analyt_glob}
 \end{multline}
где $\Pi \eqd \{\col(\pi^1, \ldots , \pi^N): \; \pi^n \geqslant 0,\; 
n=\overline{1,N},\;\sum\nolimits_{n=1}^N \pi^n =1 \}$~--- вероятностный симплекс.

Аналитическая аппроксимация~$\overline{X}_r$ может быть записана в~явном виде:
\begin{equation}
\overline{X}_r = (\mathbf{1}\Xi_{1,r}^{\top}\pi)^{-1} \Xi_{1,r}^{\top}\pi
 \label{eq:filt_2}
 \end{equation}
 и~в виде рекурсии:
 \begin{equation}
\overline{X}_r = (\mathbf{1}\xi_{r}^{\top}\overline{X}_{r-1})^{-1} \xi_{r}^{\top}\overline{X}_{r-1},
 \label{eq:filt_3}
 \end{equation}
 где $\mathbf{1} = \mathrm{row} (1,\ldots,1)$~--- век\-тор-стро\-ка подходящей размерности,
$$
 \Xi_{q,p} =  
\begin{cases}
 \xi_{q}\xi_{q+1}\ldots \xi_p, & \mbox{если }  q \leqslant p\,; \\
 I &  \mbox{в противном случае,}
\end{cases}
$$
  а $\xi_q \eqd \|\xi^{ij}(Y_q)\|_{i,j=\overline{1,N}}$~--- $(N \times N)$-мер\-ные 
  случайные матрицы~--- функции наблюдений~$Y_q$: 
 \begin{equation}
 \xi^{ij}(y)\eqd 
\sum\limits_{m=0}^s \int\limits_{\mathcal{D}}\!\! 
 \mathcal{N}\!\left(\!y,f u,\sum_{p=1}^N u^p g_p\right)\!
 \rho^{i,j,m}(du).\!\!
 \label{eq:xi_def}
 \end{equation}

 \section{Точность численных аппроксимаций}
 
 Предложенная рекуррентная схема приближенного оценивания~(\ref{eq:filt_3}) 
 предполагает вычисление интегралов~$\xi^{ij}_r$~(\ref{eq:xi_def}), 
 для которых не существует явного аналитического представления. 
 Это обстоятельство влечет за собой необходимость использования схем 
 численного интегрирования, а значит, появление дополнительных методических 
 ошибок, которые необходимо учитывать. Исследуем в~общем случае аппроксимации 
 этих интегралов и~представим утверждение о~влиянии точности их вычисления 
 на итоговую точность аппроксимации оценок оптимальной фильтрации.
  
  Обычно значения интегралов~$\xi^{ij}(y)$ приближенно вычисляются 
  в~виде интегральных сумм:
  \begin{equation}
  \left.
  \begin{array}{rl}
  \hspace*{-2mm}\xi^{ij}(y) & \displaystyle \approx \psi^{ij}(y) \eqd 
 \sum\limits_{\ell=1}^{L} \mathcal{N}\!\left(\!y,f w_{\ell},
 \sum\limits_{p=1}^N w^p_{\ell} g_p\!\right)\!\varrho_{\ell}^{ij}; \\[6pt] 
   \hspace*{-2mm}\psi(y) &\eqd \|\psi^{ij}(y)\|_{i,j=\overline{1,N}},
 \end{array}
\! \right\}\!\!
  \label{eq:int_sum}
  \end{equation}
  определяемых набором пар $\{(w_{\ell},\varrho_{\ell}^{ij})\}_{\ell=\overline{1,L}}$.
   Здесь $\varrho_{\ell}^{ij} \hm\geqslant 0$ ($\ell\hm=\overline{1,L}$)~--- 
   веса, $\sum\nolimits_{\ell=1}^L\varrho_{\ell}^{ij} \hm\leqslant 1$, 
   а~$w_{\ell}\hm\eqd \col(w^1_{\ell},\ldots,w^N_{\ell}) \hm\in \mathcal{D}$~--- точки. 
  Аналогично мат\-ри\-цам~$\Xi_{q,r}$ строятся их аппроксимации:
  $$
 \Psi_{q,p} =  
 \begin{cases}
 \psi_{q}\psi_{q+1}\ldots \psi_p, & \mbox{если }  q \leqslant p\,; \\
 I &  \mbox{в противном случае}
\end{cases}
$$
  и~$\psi_q \eqd \|\psi^{ij}(Y_q)\|_{ij=\overline{1,N}}$.
  
  По построению~$\psi^{ij}_q$ являются положительными случайными величинами,
   поэтому приближенная оценка~$\widetilde{X}_r$, вычисляемая рекурсивно
  \begin{equation}
   \hspace*{-2mm} \widetilde{X}_r \eqd \left(\mathbf{1}\psi_r^{\top} \widetilde{X}_{r-1}\right)^{-1}\!\!
  \psi_r^{\top} \widetilde{X}_{r-1}, \enskip 
  r\geqslant 1\,, \ \widetilde{X}_{0} = \pi,
  \label{eq:pp_est}
  \end{equation}
  обладает свойством устойчивости.
  
  Обозначим ошибки аппроксимации интегралов и~их абсолютные значения
  сле\-ду\-ющим образом:
    \begin{alignat*}{2}
  \gamma^{kj} &\eqd \psi^{kj} - \xi^{kj}, & \enskip
 \gamma_r &\eqd \|\gamma^{kj}(Y_r)\|_{k,j=\overline{1,N}};
 %\label{eq:matr_err}
\\
  \overline{\gamma}^{kj} &\eqd |\gamma^{kj}|, &\enskip
 \overline{\gamma}_r &\eqd \left\||\gamma^{kj}(Y_r)|\right\|_{k,j=\overline{1,N}}.
  %\label{eq:matr_abs}
  \end{alignat*}
 
  
  Рекуррентная схема вычисления~$\overline{X}_r$~(\ref{eq:filt_2}) 
  заменяется на схему~(\ref{eq:pp_est}), при этом обе рекурсии стартуют из 
  одного и~того же начального условия~$\pi$.
  
  Рассмотрим случайные события $a_q \eqd \{\omega \in \Omega:\; N^X_q(\omega) \leqslant s \}$, заключающиеся в~том, что на $[t_{q-1},t_q]$ произошло не более $s$ скачков состояния $X$, а также $A_r \eqd \prod_{q=1}^r a_q$. Обе рекурсии, (\ref{eq:filt_2}) и~(\ref{eq:pp_est}), строились в~расчете на выполнение события $A_r$, поэтому точность аппроксимации схемы (\ref{eq:filt_2}) схемой (\ref{eq:pp_est}) следует определять с~учетом ограничения на число скачков состояния.
  
  \columnbreak
  
  В качестве локального показателя близости оценок~$\overline{X}$ и~$\widetilde{X}$ 
  будем рассматривать псевдометрику  
  
\noindent
  \begin{multline}
 \varepsilon(s) \eqd \sup\limits_{\pi \in \Pi}\me{}{\mathbf{I}_{a_1}(\omega)\|
 \widetilde{X}_{1} - \overline{X}_{1}\|_{1}} = {}\\
 {}=\sup\limits_{\pi \in \Pi}\sum\limits_{j=1}^N\me{}{\mathbf{I}_{a_1}
 (\omega)|\widetilde{X}^j_1 - \overline{X}^j_{1}|}.
 \label{eq:num_loc}
 \end{multline}
 Глобальный показатель близости определяется аналогично:
 \begin{multline}
\mathcal{E}_r(s) \eqd \sup\limits_{\pi \in \Pi}\me{}{\mathbf{I}_{A_r}
(\omega)\|\widetilde{X}_{r} - \overline{X}_{1}\|_{1}} ={}\\
{}= \sup\limits_{\pi \in \Pi}\sum\limits_{n=1}^N\me{}{\mathbf{I}_{A_r}(\omega)|\widetilde{X}^n_r - \overline{X}^n_{r}|}.
 \label{eq:num_glob}
 \end{multline}
 Он характеризует расхождение алгоритмов~(\ref{eq:filt_3}) и~(\ref{eq:pp_est}) 
 за один и~$r$~шагов при отсутствии превышения чис\-лом скачков состояния 
 фиксированного порога~$s$ на одном или на каждом из~$r$~шагов. 
 
 Для оценивания $\varepsilon(s)$ и~$\mathcal{E}_r(s)$ потребуются 
 некоторые вспомогательные результаты.
 
 Рассмотрим неотрицательную интегрируемую функцию $\phi_1\hm=\phi_1(y): 
 \mathbb{R}^M \hm\to \mathbb{R}_+$ и~$\mathcal{O}_1$-из\-ме\-ри\-мую случайную величину
 \begin{multline*}
 \Phi_1 \eqd \fr{\phi_1(Y_1)}{\mathbf{1}\xi_1^{\top}(Y_1)\pi}={}\\
 {}=
 \phi_1(Y_1)\Bigg/
 \left(\sum\limits_{i,j=1}^{N}\sum\limits_{m=0}^{s}\int\limits_{\mathcal{D}}
  \mathcal{N}\left(Y_1,fu,\sum\limits_{p=1}^Nu^p g_p\right)\times{}\right.\\
\left.  {}\times{}\rho^{i,j,m}(du)\pi_i
\vphantom{\int\limits_{\mathcal{D}}}
  \right)
% \label{eq:phi_1}
 \end{multline*}
 и~найдем $\me{}{\mathbf{I}_{a_1}\Phi_1}$:
 \begin{multline*}
 \me{}{\mathbf{I}_{a_1}\Phi_1} =
 \int\limits_{\mathbb{R}^M}\int\limits_{\mathcal{D}}
 %\fr{
 \phi_1(y)\times{}\\
 {}\times
 \sum\limits_{k,\ell=1}^{N}
 \sum\limits_{m=0}^{s}\mathcal{N}
 \left(y,fv,\sum\nolimits_{q=1}^Nv^q g_q\right)\rho^{k,\ell,n}(dv)\pi_k\Bigg/\\
  \left(
 \sum\limits_{i,j=1}^{N}\sum\limits_{m=0}^{s}\int\limits_{\mathcal{D}}
  \mathcal{N}\left(y,fu,\sum\limits_{p=1}^Nu^p g_p\right)\times{}\right.\\
\left.  {}\times{}\rho^{i,j,m}(du)\pi_i
 \vphantom{\int\limits_{\mathcal{D}}}
  \right)dy
  ={} \\ 
 {}=\!\!
 \int\limits_{\mathbb{R}^M}\!\!\phi_1(y)\!
 %\fr{
 \sum\limits_{k,\ell=1}^{N}
 \sum\limits_{m=0}^{s}\int\limits_{\mathcal{D}}\!\mathcal{N}\!
 \left(\!y,fv,\sum\limits_{q=1}^Nv^q g_q\!\right)\rho^{k,\ell,n}(dv)\times{}\hspace*{-7.17207pt}
\end{multline*}

\noindent
 \begin{multline}
 {}\times
 \pi_k\,\Bigg/\!
 %}
 \left(
 \sum\limits_{i,j=1}^{N}\sum\limits_{m=0}^{s}\int\limits_{\mathcal{D}}
  \mathcal{N}\left(y,fu,\sum\limits_{p=1}^Nu^p g_p\right)\times{}\right.\\
\left.  {}\times{}\rho^{i,j,m}(du)\pi_i
\vphantom{\int\limits_{\mathcal{D}}}
 \right) dy = \int\limits_{\mathbb{R}^M}\phi_1(y)\,dy.
 \label{eq:me_phi_1}
 \end{multline}
 
 Рассмотрим неотрицательную интегрируемую функцию $\phi_2\hm=\phi_1(y_1,y_2): 
 \mathbb{R}^{2M}\hm \to \mathbb{R}_+$ и~$\mathcal{O}_2$-из\-ме\-ри\-мую случайную величину
  \begin{multline*}
 \Phi_2 \eqd \fr{\phi_1(Y_1,Y_2)}{\mathbf{1}\Xi_{1,2}^{\top}(Y_1,Y_2)\pi}= {}
   \\
    \left(
  \sum\limits_{i,i_2,j=1}^{N}\sum\limits_{m_1,m_2=0}^{s}
  \int\limits_{\mathcal{D}}\int\limits_{\mathcal{D}}
  \mathcal{N}\left(Y_1,fu_1,\sum\limits_{p_1=1}^Nu^{p_1}g_{p_1}\right)\times{}\right.\\
{}\times
  \mathcal{N}\left(Y_2,fu_2,\sum\limits_{p_2=1}^Nu^{p_2}g_{p_2}\right)\times{}\\
    \left.{}\times{}
  \rho^{i,i_2,m_1}(du_1)\rho^{i_2,j,m_2}(du_2)
  \pi_i
  \vphantom{\int\limits_{\mathcal{D}}}
  \right)^{-1} \phi_2\left(Y_1,Y_2\right)
 \label{eq:phi_2}
 \end{multline*}
 и~найдем $\me{}{\mathbf{I}_{A_2}\Phi_2}$:
 \begin{multline*}
 \me{}{\mathbf{I}_{A_2}\Phi_2} =
 \int\limits_{\mathbb{R}^M}
 \int\limits_{\mathbb{R}^M}
 \phi_2(y_1,y_2)\times{}\\
{} \times
  \sum\limits_{k,k_2,\ell=1}^{N}\sum\limits_{n_1,n_2=0}^{s}
  \int\limits_{\mathcal{D}}\int\limits_{\mathcal{D}}
  \mathcal{N}\left(y_1,fv_1,\sum\limits_{q_1=1}^Nv^{q_1} g_{q_1}\right)\times{}\\
  {}\times
  \mathcal{N}\left(y_2,fv_2,\sum\limits_{q_2=1}^Nv^{q_2} g_{q_2}\right)
  \rho^{k,k_2,n_1}(dv_1)\times{}\\
  {}\times \rho^{k_2,\ell,n_2}(dv_2)
  \pi_i 
\Bigg/ \\
\left( 
\sum\limits_{i,i_2,j=1}^{N}\sum\limits_{m_1,m_2=0}^{s}
 \int\limits_{\mathcal{D}}\int\limits_{\mathcal{D}}
  \mathcal{N}\left(y_1,fu_1,\sum\limits_{p_1=1}^Nu^{p_1} g_{p_1}\right)\times{}\right.\\
{}\times  \mathcal{N}\left(y_2,fu_2,\sum\limits_{p_2=1}^Nu^{p_2} g_{p_2}\right)
  \rho^{i,i_2,m_1}(du_1)\times{}\\
 \! \left.{}\times\rho^{i_2,j,m_2}(du_2)
  \pi_i
  \vphantom{\int\limits_{\mathcal{D}}}
  \!\right) dy_2 dy_1
=  \!\int\limits_{\mathbb{R}^M}
 \int\limits_{\mathbb{R}^M}\!
 \phi_2(y_1,y_2)\,
 dy_2 dy_1.\hspace*{-7.28978pt}
 \end{multline*}
 
 
 
 Рассмотрим неотрицательную интегрируемую функцию $\phi_r\hm=\phi_1
 (y_1,y_2,\ldots,y_r): \mathbb{R}^{rM} \hm\to \mathbb{R}_+$ и~$\mathcal{O}_r$-из\-ме\-ри\-мую 
 случайную величину
  \begin{equation*}
 \Phi_r \eqd \fr{\phi_r(Y_1,Y_2,\ldots,Y_r)}
 {\mathbf{1}\Xi_{1,r}^{\top}(Y_1,Y_2,\ldots,Y_r)\pi}\,.
 %\label{eq:phi_r}
 \end{equation*}
 Выполняя выкладки, аналогичные предыдущим, можно получить 
 выражение для математического ожидания
  \begin{multline}
 \me{}{\mathbf{I}_{A_r}\Phi_r} ={}\\
 {}=
 \int\limits_{\mathbb{R}^M}\cdots
 \int\limits_{\mathbb{R}^M}
 \phi_r\left(y_1,y_2, \ldots,y_r\right)\,
 dy_r\cdots dy_2 dy_1,
 \label{eq:me_phi_r}
 \end{multline}
 позволяющее доказать следующую теорему~--- основной результат данной статьи.

\smallskip

\noindent
\textbf{Теорема~1.}
\textit{Если для схемы}~(\ref{eq:int_sum}) 
\textit{приближенного вы\-чис\-ле\-ния интеграла}~(\ref{eq:xi_def}) 
\textit{выполняется неравенство}
  \begin{equation}
 \max\limits_{i=\overline{1,N}}\sum\limits_{j=1}^N \, \int\limits_{\mathbb{R}^M}
 \left\vert \psi^{ij}(y) - \xi^{ij}(y)\right\vert dy < \delta\,,
 \label{eq:cond}
 \end{equation}
\textit{то локальный показатель близости ограничен следующим образом}: 
 \begin{equation}
 \varepsilon(s) \leqslant 2 \delta,
 \label{eq:num_prec_loc}
 \end{equation} 
 \textit{а для глобального показателя верно неравенство}
 \begin{equation}
 \mathcal{E}_r(s) \leqslant 2 r\delta\,.
 \label{eq:num_prec_glob}
 \end{equation} 
 Доказательство теоремы~1 приведено в~приложении.
 
 Примечательно, что условие~(\ref{eq:cond}) касается только аналитических 
 свойств схемы интегрирования~(\ref{eq:int_sum}), и~этого оказывается достаточно, 
 чтобы оценивать вероятностные характеристики расстояния между 
 стохастическими последовательностями, порожденными рекурсивными 
 схемами~(\ref{eq:xi_def}), (\ref{eq:filt_3}) и~(\ref{eq:pp_est}), 
 (\ref{eq:int_sum}) соответственно. Также важно, что с~рос\-том числа 
 шагов схемы глобальная ошибка растет линейно.
  
 Следует подчеркнуть, что рекурсия~(\ref{eq:pp_est}) совместно с~алгоритмом 
 приближенного интегрирования~(\ref{eq:int_sum}) может быть непосредственно 
 численно реализована.
 Неравенства (\ref{eq:analyt_loc}), (\ref{eq:analyt_glob}), (\ref{eq:num_loc}) 
и~(\ref{eq:num_glob}), а~также 
 $\mathbf{P}\left\{\overline{a}_1\right\}
 \hm\leqslant 2 ({(\overline{\lambda}h)^{s+1}})/((s+1)!)$
 позволяют оценить сверху локальный показатель близости оптимальной 
 оценки~$\widehat{X}_1$ и~ее аппроксимации~$\widetilde{X}_1$. 
 Если выполняются условия теоремы~1, то
 \begin{multline*}
 \overline{\tau}(s) \eqd \sup\limits_{\pi \in \Pi}
 \mathbf{E}\left\{\|\widehat{X}_1 - \widetilde{X}_1\|_1\right\} \leqslant  {}\\ 
{} \leqslant \sup\limits_{\pi \in \Pi}
\mathbf{E}\left\{\mathbf{I}_{a_1}(\omega)\| 
 \widetilde{X}_1-\overline{X}_{1}+\overline{X}_{1}-\widehat{X}_1
 \|_1+{}\right.\\
\left. {}+\mathbf{I}_{\overline{a}_1}(\omega)\|\widetilde{X}_1- \overline{X}_{1}\|_1\right\} 
 \leqslant {} \\ 
 {}\leqslant
 2\mathbf{P}\{\overline{a}_1\}+
 \sup\limits_{\pi \in \Pi}\me{}{\|\overline{X}_{1}-
 \widehat{X}_1\|_1}+{}\\
 {}+\sup_{\pi \in \Pi}
 \me{}{\mathbf{I}_{a_1}(\omega)\|\widetilde{X}_1-\overline{X}_{1}\|_1} ={} \\
 {} =
 2\mathbf{P}\{\overline{a}_1\}+\sigma(s)+\varepsilon(s) \leqslant 4
 \fr{(\overline{\lambda}h)^{s+1}}{(s+1)!} + 2\delta\,.
 %\label{eq:tot_loc}
 \end{multline*}
 Итоговый глобальный показатель близости оценок 
 $\widehat{X}_r \eqd \me{}{X_r|\mathcal{O}_r}$ и~$\widetilde{X}_r$ 
 может быть ограничен сверху аналогичным образом:

 \vspace*{-4pt}
 
 \noindent
 \begin{multline*}
 \overline{\mathcal{T}}(s) \eqd \sup\limits_{\pi \in \Pi}
 \me{}{\|\widehat{X}_r - \widetilde{X}_r\|_1} \leqslant {}\\
 {}\leqslant  
 4
\left[ 1-\left(1-\fr{(\overline{\lambda}h)^{s+1}}{(s+1)!}\right)^r
\right] + 2r\delta\,.
% \label{eq:tot_glob}
 \end{multline*}
 
 \vspace*{-2pt}
 
 \noindent
 В общем случае шаг дискретизации по времени~$h$, 
 порядок аналитической аппроксимации~$s$ и~показатель точности численного 
 интегрирования~$\delta$  не связаны между собой. Тем не менее при 
 разработке эффективных алгоритмов численного решения задачи фильтрации 
 состояний МСП по непрерывным наблюдениям эти параметры нуждаются в~совместном 
 оптимальном выборе. Зафиксируем некоторый момент времени~$T$ и~порядок 
 аналитической аппроксимации~$s$. Будем увеличивать число шагов $r \hm\to \infty$, 
 а~значит  уменьшать шаг дискретизации $h\hm = {T}/{r} \hm\to 0$.  
 В~этом случае в~силу неравенства Бернулли
 
 \vspace*{-4pt}
 
 \noindent
\begin{multline}
\sup\limits_{\pi \in \Pi}\me{}{\|\widetilde{X}_{T/h} - \widehat{X}_{T/h}\|_1} 
\leqslant {}  \\ 
{}\leqslant  
 4
\left[ 1-\left(1-\fr{(\overline{\lambda}h)^{s+1}}{(s+1)!}\right)^r
\right] + 2r\delta
\leqslant{}\\
{}\leqslant
4r\fr{(\overline{\lambda}h)^{s+1}}{(s+1)!}+2r\delta=
4\overline{\lambda}T \fr{(\overline{\lambda}h)^{s}}{(s+1)!}+2r\delta = \\ =
2T\left(2\overline{\lambda} \fr{(\overline{\lambda}h)^s}{(s+1)!}+
 \fr{\delta}{h}\right).
 \label{eq:asympt_glob}
\end{multline}

\vspace*{-2pt}

\noindent
Из этого неравенства можно легко заключить, что при фиксированных порядке~$s$ и~шаге 
дискретизации~$h$ схема численного интегрирования должна обеспечивать такую точность, 
чтобы $\delta \sim {(\overline{\lambda}h)^{s+1}}/{\overline{\lambda}}$. 
В~случае, если~$\delta$ будет больше, ошибка численного интегрирования 
будет снижать точность аналитической аппроксимации, определенную парой~$(s,h)$. 
Это означает, что добавление дополнительных слагаемых в~рекуррентную схему и~учет 
большего чис\-ла возможных скачков состояния на интервале дискретизации оказываются 
бесполезными.\linebreak В~то же время если~$\delta$ будет меньше, то высокая точность 
чис\-лен\-но\-го интегрирования будет бесполезной из-за ее несоответствия 
точности аналитической аппроксимации, обеспечиваемой выбором пары~$(s,h)$.

Обычно для стандартных схем численного интегрирования (схемы 
прямоугольников, трапеций, Симпсона и~пр.) интегралы в~правой части~(\ref{eq:cond}) 
могут быть оценены сверху на основе анализа остаточного члена в~формуле Тейлора.

Неравенство (\ref{eq:asympt_glob}) также позволяет сделать вывод, что 
итоговая ошибка аппроксимации решения задачи оптимальной фильтрации 
выбранной схемой численного интегрирования растет со временем не быстрее, 
чем линейно.

 \vspace*{-6pt}

 \section{Заключение}
 
  \vspace*{-4pt}
 
 В первой части цикла для практического решения задачи фильтрации 
 состояния МСП по дискретизованным наблюдениям представлен целый класс 
 численных алгоритмов. Все они отличаются выбором схем приближенного 
 интегрирования. Была предложена некоторая достаточно легко вычисляемая 
 (или оцениваемая сверху) характеристика близости аналитических аппроксимаций и~их 
 численных реализаций. Она в~итоге позволила оценить величину методической ошибки
  приближения искомой оценки фильтрации выбранной чис\-лен\-ной схемой. Эта ошибка 
  складывается из двух составляющих. Первая порождена тем, что в~используемом 
  алгоритме оценивания не учитывается возможность превышения числом скачков~$N^X$ 
  выбранного порога~$s$. Вторая возникает из-за неточного вычисления интегралов.
 
 В следующих статьях полученные результаты будут использованы для 
 сравнительного анализа различных численных схем, применяемых для 
 приближенного решения задач фильтрации состояний МСП по наблюдениям 
 как с~аддитивными, так и~с~мультипликативными шумами.
 
 \vspace*{-6pt}
 
{\small \subsection*{\raggedleft Приложение}
  
  \noindent
Д\,о\,к\,а\,з\,а\,т\,е\,л\,ь\,с\,т\,в\,о\ \ теоремы~1.
Итак, $\widetilde{X}_1 \hm= (\mathbf{1}\psi_1^{\top}\pi)^{-1}\psi_1^{\top}\pi$, 
$\overline{X}_1 \hm= (\mathbf{1}\xi_1^{\top}\pi)^{-1}\xi_1^{\top}\pi$ 
и~$\Delta_1 \hm= \widetilde{X}_1\hm - \overline{X}_1$. 
Используя свойства матричных операций, легко показать, что 
$[\gamma^{\top}\pi\mathbf{1}\hm-\mathbf{1}\gamma^{\top}\pi I]\gamma^{\top}\pi\hm=0$.
 Для обеих оценок~$\widetilde{X}_1$ и~$\overline{X}_1$ выполняется условие 
 нормировки; следовательно, $\|\widetilde{X}_1\|_1 \hm= \|\overline{X}_1\|_1\hm=1$. 
 Поэтому верна следующая цепочка неравенств:
 
 \noindent
\begin{multline*}
\|\Delta_1\|_1 = \fr{1}{\mathbf{1}\psi_1^{\top}\pi \mathbf{1}\xi_1^{\top}\pi}
\|
\mathbf{1}\xi_1^{\top}\pi \psi_1^{\top}\pi-
\mathbf{1}\psi_1^{\top}\pi\xi_1^{\top}\pi
\|_1 = {}\\
{}=
\fr{1}{\mathbf{1}\psi_1^{\top}\pi \mathbf{1}\xi_1^{\top}\pi}
\|
\mathbf{1}\xi_1^{\top}\pi \gamma_1^{\top}\pi-
\mathbf{1}\gamma_1^{\top}\pi\xi_1^{\top}\pi
\|_1 ={} \\
{} =
\fr{1}{\mathbf{1}\psi_1^{\top}\pi \mathbf{1}\xi_1^{\top}\pi}
\|[
\gamma_1^{\top}\pi \mathbf{1}-
\mathbf{1}\gamma_1^{\top}\pi I 
]\xi_1^{\top}\pi
\|_1 ={} \\ 
{}=
\fr{1}{\mathbf{1}\psi_1^{\top}\pi \mathbf{1}\xi_1^{\top}\pi}
\|[
\gamma_1^{\top}\pi \mathbf{1}-
\mathbf{1}\gamma_1^{\top}\pi I 
][\xi_1^{\top}\pi + \gamma_1^{\top}\pi]
\|_1 = {}\\
{}=\fr{1}{\mathbf{1}\xi_1^{\top}\pi}
\|
[
\gamma_1^{\top}\pi \mathbf{1}-
\mathbf{1}\gamma_1^{\top}\pi I 
]\widetilde{X}_1
\|_1
\leqslant{} \\
{}\leqslant
\fr{1}{\mathbf{1}\xi_1^{\top}\pi}
\|
[
\gamma_1^{\top}\pi \mathbf{1}-
\mathbf{1}\gamma_1^{\top}\pi I 
]\|_1 \|\widetilde{X}_1
\|_1 \leqslant 2\fr{\mathbf{1}\overline{\gamma}_1^{\top}\pi}
{\mathbf{1}\xi_1^{\top}\pi} ={}\\
{}= \sum\limits_{i=1}^N\pi_i
\fr{\sum\nolimits_{j=1}^N \overline{\gamma}_1^{ij}}
{\sum\nolimits_{k,\ell=1}^N \xi_1^{k\ell}\pi_k}.
\end{multline*}
Используя последнее неравенство,  условие~(\ref{eq:cond}) и~формулу~(\ref{eq:me_phi_1}), можно показать, что
$$
\me{}{\mathbf{I}_{a_1}(\omega)\|\Delta_1\|_1} \leqslant
2 \sum\limits_{i=1}^N \pi_i\int\limits_{\mathbb{R}^M}\sum_{i=1}^N 
\overline{\gamma}^{ij}(y)\,dy
\leqslant 2\delta.
$$
Истинность неравенства~(\ref{eq:num_prec_loc}) следует из того, 
что неравенство вверху выполняется для произвольного $\pi \hm\in \Pi$.

Оценим сверху норму ошибки $\Delta_r\hm = \widetilde{X}_r\hm - \overline{X}_r$. 
Проводя выкладки, аналогичные выкладкам для~$\Delta_1$, получаем, что
\begin{equation}
\|\Delta_r\|_1 \leqslant 
\fr{1}{\mathbf{1}\Xi_{1,r}^{\top}\pi}
\|
[
\Gamma_{1,r}^{\top}\pi \mathbf{1}-
\mathbf{1}\Gamma_{1,r}^{\top}\pi I 
]\|_1 \eqd \mathcal{I}_1,
\label{eq:I1}
\end{equation}
где 
\begin{equation}
\Gamma_{1,r} \eqd \Psi_{1,r} - \Xi_{1,r} = 
\sum\limits_{t=1}^r \Psi_{1,t-1}\gamma_t\Psi_{t+1,r}.
\label{eq:sum_1}
\end{equation}
Для оценки вклада каждого слагаемого~(\ref{eq:sum_1}) в~(\ref{eq:I1})
используем формулу~(\ref{eq:me_phi_r}). Рассмотрим неотрицательную 
функцию $\phi(y_1,y_2,y_3):\mathbf{R}^{3M} \hm\to \mathbf{R}_+$:
\begin{equation*}
\phi(y_1,y_2,y_3)=\mathbf{1}\psi^{\top}(y_3)\overline{\gamma}^{\top}(y_2)
\psi^{\top}(y_1)\pi,
%\label{eq:phi}
\end{equation*}
$O_3$-измеримую случайную величину
$$
\Phi \eqd 
\fr{\phi(Y_1,Y_2,Y_3)}{\mathbf{1}\Xi_{1,3}^{\top}(Y_1,Y_2,Y_3)\pi}
$$ 
и~оценим сверху следующее математическое ожидание: 
\begin{multline*}
\me{}{\mathbf{I}_{A_3}(\omega)\Phi} ={} \\
{} =\!\!
\int\limits_{\mathbb{R}^M}\int\limits_{\mathbb{R}^M}\int\limits_{\mathbb{R}^M}
\!\sum\limits_{i,j,k,m=1}^N \!\!\!\!\!\!\!
\pi_i\psi^{ij}(y_1)\overline{\gamma}^{jk}(y_2)\psi^{km}(y_3)dy_3dy_2dy_1={} \\
{} =
\sum\limits_{i,j,k,m=1}^N \!\! \!\!\pi_i 
\sum\limits_{\ell,n=1}^L \varrho_{\ell}^{ij}\varrho_{n}^{km}
%\sum\limits_{j,k=1}^N 
\int\limits_{\mathbb{R}^M} \overline{\gamma}^{jk}(y_2)dy_2 \leqslant{}\\
{}\leqslant
\delta \sum\limits_{i,j=1}^N \pi_i 
\sum\limits_{\ell=1}^L \varrho_{\ell}^{ij} %\varrho_{n}^{km}
%\label{eq:me_Phi} 
\leqslant \delta.
\end{multline*}
Продолжим цепочку неравенств~(\ref{eq:I1}):
\begin{equation}
\mathcal{I}_1 \leqslant 2 \sum\limits_{t=1}^r \fr{1}{\mathbf{1}\Xi_{1,r}^{\top}\pi}\,
\mathbf{1}\Psi_{t+1,r}^{\top}\overline{\gamma}_{t}^{\top}\Psi_{1,t-1}^{\top}\pi\,.
\label{eq:I2}
\end{equation}
Применяя выкладки, аналогичные использованным при оценке 
$\me{}{\mathbf{I}_{A_3}(\omega)\Phi}$, можно оценить сверху математические 
ожидания слагаемых в~правой части~(\ref{eq:I2}):
$$
\me{}{\mathbf{I}_{A_r}(\omega)\fr
{
\mathbf{1}\Psi_{t+1,r}^{\top}\overline{\gamma}_{t}^{\top}\Psi_{1,t-1}^{\top}\pi
}
{
\mathbf{1}\Xi_{1,r}^{\top}\pi
}} \leqslant
\delta.
$$
Тогда окончательно
$
\me{}{\mathbf{I}_{A_r}(\omega)\|\Delta_r\|_1} \hm\leqslant
2r \delta
$
и~истинность неравенства~(\ref{eq:num_prec_glob}) 
 следует из того, что последнее неравенство выполняется для произвольного $\pi\hm \in \Pi$.
 
Теорема~1 доказана.


}

{\small\frenchspacing
 {%\baselineskip=10.8pt
 \addcontentsline{toc}{section}{References}
 \begin{thebibliography}{9}

\bibitem{Wonham_65}
\Au{Wonham W.} 
Some applications of stochastic differential equations to optimal nonlinear
filtering~// SIAM J.~Control. Optim., 1964. Vol.~2. No.\,3. P.~347--369. 
doi: 10.1137/0302028.

\bibitem{B_18}
\Au{Борисов А.} Фильтрация Вонэма по наблюдениям с~мультипликативными шумами~// 
Автоматика и~телемеханика, 2018.
№\,1. C.~52--65. %doi: 10.1134/ S0005117918010046.

\bibitem{KP_92}
\Au{Kloeden P., Platen~E.} 
Numerical solution of stochastic differential equations.~--- 
Berlin: Springer, 1992. 636~p. doi: 10.1007/978-3-662-12616-5.

  \bibitem{B_18_IA}
\Au{Борисов А.} Фильтрация состояний марковских 
скачкообразных процессов по дискретизованным наблюдениям~// Информатика и~её 
применения,~2018. Т.~12.~Вып.~3.~C.~115--121. doi: 10.14357/19922264180316.
    \end{thebibliography}

 }
 }

\end{multicols}

\vspace*{-6pt}

\hfill{\small\textit{Поступила в~редакцию 18.09.19}}

\vspace*{8pt}

%\pagebreak

%\newpage

%\vspace*{-28pt}

\hrule

\vspace*{2pt}

\hrule

%\vspace*{-2pt}

\def\tit{NUMERICAL SCHEMES OF MARKOV JUMP PROCESS FILTERING GIVEN DISCRETIZED 
OBSERVATIONS~I:\\ ACCURACY CHARACTERISTICS}


\def\titkol{Numerical schemes of Markov jump process filtering given discretized 
observations~I:~Accuracy characteristics}

\def\aut{A.\,V.~Borisov}

\def\autkol{A.\,V.~Borisov}

\titel{\tit}{\aut}{\autkol}{\titkol}

\vspace*{-11pt}


\noindent
Institute of Informatics Problems, Federal Research Center ``Computer Science and Control'' of the Russian
Academy of Sciences, 44-2~Vavilov Str., Moscow 119333, Russian Federation

\def\leftfootline{\small{\textbf{\thepage}
\hfill INFORMATIKA I EE PRIMENENIYA~--- INFORMATICS AND
APPLICATIONS\ \ \ 2019\ \ \ volume~13\ \ \ issue\ 4}
}%
 \def\rightfootline{\small{INFORMATIKA I EE PRIMENENIYA~---
INFORMATICS AND APPLICATIONS\ \ \ 2019\ \ \ volume~13\ \ \ issue\ 4
\hfill \textbf{\thepage}}}

\vspace*{3pt}   




\Abste{The note is the initial in the series of the papers devoted to 
the numerical realization of the optimal state filtering of Markov 
jump processes given the indirect observations corrupted by 
 the additive and/or multiplicative Wiener noises. This problem 
 is solved by the time discretization of the observations with their
  subsequent processing. Both the optimal and suboptimal estimations are 
  expressed in terms of multiple integrals of the Gaussian densities
  with 
  some mixing distributions. In the article, the author presents the 
  investigation of various numerical integration\linebreak\vspace*{-12pt}}
  
  \Abstend{schemes' influence on 
  the accuracy of the approximating estimates. The problem turns into the 
  characterization of distance between stochastic sequences generated by 
  some recursions. The paper introduces a~pseudometric describing the distance 
  and presents a~proposition determining the influence of the characteristic on 
  both the local and global accuracy of the filtering estimate approximation.}

\KWE{Markov jump process; optimal filtering; additive and multiplicative observation noises; stochastic differential equation; analytical and numerical approximation}



 \DOI{10.14357/19922264190411} 

%\vspace*{-14pt}

 \Ack
\noindent
The work was supported in part by the Russian Foundation
for Basic Research (project No.\,19-07-00187~A).


%\vspace*{-6pt}

  \begin{multicols}{2}

\renewcommand{\bibname}{\protect\rmfamily References}
%\renewcommand{\bibname}{\large\protect\rm References}

{\small\frenchspacing
 {%\baselineskip=10.8pt
 \addcontentsline{toc}{section}{References}
 \begin{thebibliography}{9}

\bibitem{Won_65-1}
\Aue{Wonham, W.} 1964.~Some applications of stochastic differential equations to optimal
  nonlinear filtering. \textit{SIAM~J.~Control. Optim.} 2(3):347--369. 
  doi: 10.1137/0302028.

\bibitem{B_18-1}
\Aue{Borisov, A.}
2018.
Wonham filtering by observations
with multiplicative noises. \textit{Automat. Rem. Contr.}
79(1):39--50.  
doi:~10.1134/S0005117918010046.

\bibitem{KP_92-1}
\Aue{Kloeden,~P., and E.~Platen.} 1992. \textit{Numerical solution of stochastic
differential equations.}  Berlin: Springer. 636~p.
doi: 10.1007/978-3-662-12616-5.

  \bibitem{B_18_IA-1}
\Aue{Borisov, A.}
 2018. Filtratratsiya sostoyaniy markovskikh skachkoobraznykh protsessov 
 po diskretizovannym nablyudeniyam [Filtering of Markov jump processes by discretized 
 observations]. \textit{Informatika i~ee Primeneniya~--- Inform.~Appl.} 12(3):115--121.
 doi: 10.14357/19922264180316.
\end{thebibliography}

 }
 }

\end{multicols}

%\vspace*{-7pt}

\hfill{\small\textit{Received September 18, 2019}}

%\pagebreak

%\vspace*{-22pt}



\Contrl

\noindent
\textbf{Borisov Andrey V.} (b.\ 1965)~--- 
Doctor of Science in physics and mathematics, principal scientist, 
Institute of Informatics Problems, Federal Research Center 
``Computer Science and Control'' of the Russian Academy of Sciences, 
44-2~Vavilov Str., Moscow 119333, Russian Federation; \mbox{aborisov@frccsc.ru}


\label{end\stat}

\renewcommand{\bibname}{\protect\rm Литература}  