\def\stat{seif-mul}

\def\tit{ЗАКОНЫ ИНФОРМАТИКИ И СИНЕРГЕТИКИ В~ПОЗНАНИИ~СЛОЖНЫХ~СИСТЕМ}

\def\titkol{Законы информатики и~синергетики в~познании сложных систем}

\def\aut{Р.\,Б.~Сейфуль-Мулюков$^1$}

\def\autkol{Р.\,Б.~Сейфуль-Мулюков}

\titel{\tit}{\aut}{\autkol}{\titkol}


\index{Сейфуль-Мулюков Р.\,Б.}
\index{Seyful-Mulyukov R.\,B.}




%{\renewcommand{\thefootnote}{\fnsymbol{footnote}} \footnotetext[1]
%{Работа выполнена в~Институте проблем информатики ФИЦ ИУ РАН при поддержке РФФИ 
%(проект 18-07-00192).}}


\renewcommand{\thefootnote}{\arabic{footnote}}
\footnotetext[1]{Институт проблем информатики Федерального исследовательского центра <<Информатика 
и~управление>> Российской академии наук, \mbox{rust@ipiran.ru}}

%\vspace*{-12pt}

    
    
    \Abst{В~статье основные законы информатики и~синергетики применены для 
объяснения возникновения и~развития такой сложной природной системы, какой 
представляется нефть. Законы информатики предполагают неопределенность 
проявления квантового поведения электронов при создании матрицы 
углеводородной молекулы. Динамическая и~статическая неопределенность 
проявляется при поиске месторождений нефти. Рассматриваются законы 
синергетики, показывающие способность молекул к~самоорганизации. Новые 
молекулы в~углеводородном флюиде образуются вблизи точек бифуркации, 
ассоциированных с~изменением термодинамики, структуры и~состава пород 
геологической среды. В~приложении к~образованию нефти рассматривается 
понятие аттрактора. Он представляется как бассейн притяжения всех 
образовавшихся молекул флюида, в~котором формируется состав нефти 
конкретного месторождения.} 
    
    \KW{синергетика и~образование нефти; информатика и~образование 
и~поиски нефти; бифуркация и~образование молекул углеводородов; аттрактор 
и~формирование состава нефти}

\DOI{10.14357/19922264190417} 
  
%\vspace*{1pt}


\vskip 10pt plus 9pt minus 6pt

\thispagestyle{headings}

\begin{multicols}{2}

\label{st\stat}

\section{Информатика и~сложные системы (на примере нефти)}


    Информатика понимается в~России как на\-уч\-но-при\-клад\-ная дисциплина. 
Прикладная часть примерно соответствует Information Science в~западной 
терминологии. Она имеет дело с~информацией\linebreak в~ее историческом значении как 
данных, сведений, фактов, которые используются, хранятся, обрабатываются, 
распространяются, передаются\linebreak с~по\-мощью информационных технологий, средств 
связи и~компьютерных систем. Научная часть, наряду с~использованием ее 
исторического значения, представляет информацию как категорию, 
сопоставимую с~материей, энергией, временем, движением и~пространством 
и~выражающую их соотношение и~меру, количество и~качество в~данном 
природном или социальном сложном явлении или системе. Связь этих категорий 
реальности с~информацией впервые всесторонне обосновал и~применил для 
объяснения образования и~развития сложных природных систем И.~Гуревич~[1]. 

Впоследствии законы информатики были применены к~нефти как сложной 
сис\-те\-ме неорганической природы~\cite[c.~104--124]{2-s}. Ее свойства, 
вытекающие из законов информатики, включая\linebreak \textit{существование, развитие, 
познаваемость, простоту, неопределенность и~необходимое разнообразие}, дают 
возможность понять природу нефти. Ниже рассмотрены только два свойства, 
наиболее важных в~данном случае.
    
    \textit{Неопределенность}~--- понятие математики, философии, кибернетики, 
физики и~базовое понятие квантовой механики, которое показывает 
взаимодействие между сопряженными переменными состояниями элементарных 
частиц. Неопределенность нефти выражает квантовое поведение электронов 
атомов углерода и~водорода, проявляющееся при их соединении в~молекулу 
углеводорода (УВ). При гибридизации перекрываются волновые функции 
электронов, создается скрепляющее атомы облако электронов с~общим зарядом. 
Формируется элект\-рон\-но-вол\-но\-вая (квантовая) матрица, своеобразный 
генетический код молекулы~\cite{3-s}. Она сохраняет структурную идентичность 
данного типа молекулы УВ при всех изменениях геологических 
условий, в~которых осуществляется ее миграция к~поверхности. Матрица~--- 
физическое явление субатомного уровня, поэтому в~изучении процессов создания 
сложной системы макроуровня это неопределенность. 
    
    \textit{Неопределенность} проявляется и~при поиске промышленного 
скопления нефти. При этом необходимо установить наличие нефти 
в~обязательной для аккумуляции совокупности трех элементов геологической 
среды (ловушка, по\-ро\-да-кол\-лек\-тор и~экран плохо проницаемых пород). Это 
своеобразная квартира для месторождения нефти. Неопределенность в~том, что не 
во всякой квартире, точно установленной в~недрах геофизическими методами, 
аккумулируется нефть. Эту проблему можно решить разными техническими 
и~математическими методами, разделив неопределенность на динамическую 
и~статическую~\cite{4-s}. 
    
    \textit{Необходимое разнообразие} как закон ввел  
Эшби~\cite[с.~202--216]{5-s}. Он справедлив для сложных природных систем, 
включая нефть. Закон уста\-нав\-ли\-ва\-ет сопряженность сложности и~разнообразия 
управляющей и~управляемой систем. Нефть не образуется и~не существует сама 
по себе, автономно и~независимо от состояния и~состава окружающей 
геологической среды, в~которой она развивается. Взаимодействие и~развитие ее 
частей на всех этапах развития определяется совокупностью факторов среды, 
которая является управ\-ля\-ющей для сис\-те\-мы нефти. Современное разнообразие 
условий среды~--- это основа слож\-но\-го со\-ста\-ва, химического и~структурного 
разнообразия углеводородных молекул. Согласно закону Эшби соответствие 
необходимого разнообразия и~сложности управ\-ля\-емой сис\-те\-мы разнообразию 
управляющей есть непременное условие самоорганизации и,~следовательно, 
создания слож\-ной сис\-темы.
    
    Для понимания роли законов информатики и~синергетики в~познании 
сложной системы нефти приведем краткое описание состава и~стро\-ения ее частей, 
т.\,е.\ молекул УВ. Нефть на 95\% со\-сто\-ит из атомов углерода 
и~водорода. Они образуют молекулы трех структурных типов: цепные 
и~разветвленные алканы, алкановые циклы (циклоалканы или нафтены) и~арены 
(ароматические кольца), а также их гомологи и~комбинации трех основных типов 
молекул~\cite[с.~34--175]{6-s}. Следовательно, эти атомы~--- исходное вещество 
для образования нефти. Проб\-ле\-ма в~том, как и~где из них образовалась первая 
молекула УВ. Автор считает\footnote{Значения температуры и~давления вычислены по 
геотермическому градиенту ($+30$~$^\circ$C на 100~м) и~геостатическому градиенту (2,31~МПа 
на~100~м при средней плотности пород~2,3~г/см$^3$ и~3,1~МПа на 100~м при плотности~3,1~г/см$^3$), 
отнесенным к~платформенным областям. (\textit{Прим.\ автора.})}, что это произошло при 
ковалентном соединении этих атомов на глубине с~температурой не более 
1200~$^\circ$С и~давлением 1150~Мпа (см.\  рисунок). 
    
    \begin{figure*}
    \vspace*{1pt}
    \begin{center}  
  \mbox{%
 \epsfxsize=162.209mm 
 \epsfbox{sei-1.eps}
 }
\end{center}

\vspace*{6pt}

\noindent
{\small Точки бифуркации и~положение бассейна притяжения в~схеме последовательности 
генерации молекул углеводородов нефти: \textit{1}~--- метан; \textit{2}~--- цепные алканы 
(парафины); \textit{3}~--- ветвистые алканы (изоалканы); \textit{4}~--- циклоалканы (нафтены); 
\textit{5}~--- ароматические (арены); \textit{6}~--- гетероуглеводородные соединения}
\end{figure*}

    Первая молекула УВ~--- метан СН$_4$~--- образовалась при гибридизации четырех 
электронных орбиталей ($2s$, $2p_x$, $2p_y$ и~$2p_z$) атома углерода с~$1s$-ор\-би\-та\-ля\-ми 
четырех атомов водорода (см.\ рисунок). При этом образовалась матрица (см. выше). 
Образование метана~--- современный процесс геосферного масштаба, 
вызывающий его глобальную увеличивающуюся эмиссию в~атмосферу~\cite{7-s}. 
Метан известен в~залежах недр суши и~скоплениях на дне океана в~виде 
газогидратов в~количестве триллионов кубометров. Он родоначальник генерации 
миллиардов тонн нефти, установленных в~месторождениях на всех континентах 
и~их шельфах. Образование метана знаменует переход от состояния беспорядка 
отдельных атомов углерода и~водорода в~мантии Земли к~их порядку в~молекуле 
в~верхней астеносфере. По Николису и~Пригожину, это первый шаг к~созданию 
сложности и~сложному поведению час\-тей (молекул) и~способности развиваться 
в~иное, более сложное состояние~\cite[с.~25--35]{8-s}. 
    
    В данном случае переход к~созданию сложности означает изменение 
состояния самой молекулы метана. На глубинах с~температурой более 
$+1200$~$^\circ$C и~давлением 1150~МПа незначительный сдвиг равновесия 
в~сторону температуры или давления служит импульсом для метана. Импульс 
давления, не изменяя метан, выдавливает его в~область более низких давлений. 
Он мигрирует в~первозданном составе к~поверхности между кристаллами или по 
трещинам в~породах размером более 0,56~нм (максимальный размер молекул  
\textit{н}-ал\-ка\-нов). Встретив на пути миграции в~земной коре ловушку,  
по\-ро\-ду-кол\-лек\-тор и~экран (плохо проницаемые породы), газ аккумулируется в~залежи преимущественно метанового состава. 
    
    Термический импульс вызывает гомолитический разрыв $2p_z$-свя\-зи атома 
углерода и~связи атома водорода, последний удаляется. Молекула метана 
преобразуется в~метильный радикал --CН$_3$, активную частицу с~одной 
свободной валентной \mbox{$2p_z$-связью}. В~дальнейших преобразованиях частица играет 
роль матрицы. К~ней присоединяется другая подобная частица. Так образуется 
молекула этана С$_2$Н$_6$ (Н$_3$С--СН$_3$). Из этана при разрыве связи \mbox{С--Н} 
образуется новая свободная частица~--- этильный радикал --C$_2$Н$_5$, 
одновалентная свободная частица, к~которой присоединяется радикал. Так 
образуется пропан С$_3$Н$_8$ (Н$_3$С--СН$_2$--СН$_3$). Начинается жизнь 
метана как совокупности метильного и~других радикалов, составляющих  
мо\-ле\-ку\-лы УВ. Раздвоение потока метана на его миграцию в~чистом виде и~в 
форме сцепленных радикалов в~цепи алканов происходит вблизи второго уровня 
усложнения системы УВ-флюида (см.\ рисунок).
    
    На глубинах \textbf{60--50}~км миграция УВ-флюида происходит при 
снижении температуры от~1200 до~800~$^\circ$C, а давления 
геологической среды~--- от~1150 до~950~МПа. Это создает условия для роста цепи 
алканов и~одновременного присоединения к~ней, в~строго определенном порядке, 
радикалов, но уже сбоку цепи. Так появляются разветвленные алканы. Начало их 
формирования знаменует третью точку резкого изменения состава углеводородного флю\-ида 
и~сложной системы в~целом (см.\ рисунок).
    
    На глубинах \textbf{50--40}~км температура литосферы Земли снижается 
с~800 до~600~$^\circ$C, а~давление~--- с~950 до~720~МПа. 
Внутренний механизм самоорганизации системы создает молекулу с~более 
плот\-ной упаковкой атомов, более прочными межмолекулярными связами, 
энергоемкую и~мобильную. Для этого замыкается цепь с~четырьмя, пятью или 
шестью атомами углерода в~цикл. Формируются молекулы циклоалканов, 
называемые нафтенами (см.\ рисунок). При появлении нафтенов ранее 
сформированные молекулы цепных и~разветвленных алканов сохраняются  
в~углеводородном флюиде.
    
    На глубинах \textbf{40--30}~км температура среды снижается 
с~600 до~480~$^\circ$C, а~давление~--- с~720 до~690~МПа. Это позволяет 
системе создавать особый, ароматический тип связи и~одноименную молекулу~--- 
бензол С$_6$Н$_6$. Каждый из шести атомов углерода бензольного кольца 
сохраняет три равные $\sigma$-свя\-зи~--- две с~соседними атомами углерода 
и~одну  
$\sigma$-связь с~атомом водорода. Четвертый $p$-элект\-рон каждого из шести 
атомов углерода не связан ни с~одним атомом. Эти электроны образуют  
$\pi$-свя\-зи в~виде двух электронных облаков, над и~под кольцом, которые 
выполняют роль валентных орбиталей. К~ним могут присоединяться 
радикалы и~другие типы молекул УВ (см.\ рисунок). 
    
    В интервале \textbf{30--20}~км температура среды снижается с~480 
    до~370~$^\circ$C, а~давление~--- с~690 до~460~МПа. На этих глубинах более разнородная 
структура и~текстура пород, меньшие температуры и~давление и~большая 
подвижность блоков земной коры создают условия для более свободного 
взаимодействия молекул и~их совместной миграции. Открывается возможность 
усложнять ранее созданные молекулы путем соединения цепных алканов 
с~нафтеновыми\linebreak циклами и~бензольными кольцами, соединять нафтены 
и~ароматические УВ в~гетероциклы и~гетероароматические молекулы. Примером 
преобразования молекул может служить электрофильное \mbox{замещение}. 
В~молекулах циклического или ароматического типа атом углерода или водорода 
замещается по $\pi$-ти\-пу связи гетероатомами (электрофилами). Ими могут 
быть атомы серы, азота или кислорода, поскольку имеют свободный валентный 
электрон на внешней орбитали (см.\ рисунок).

\vspace*{-6pt}
    
\section{Синергетика и~сложные системы (на примере нефти)}

    Синергетика, представленная Хакеном~\cite{9-s},~--- это наука о сложности 
как феномене, самоорганизации частей в~сложные системы. Признание нефти 
сложной природной системой открывает возможность использовать законы 
и~положения синергетики для того, чтобы наряду с~информатикой, геологией 
и~геохимией представить дополнительные доказательства неорганической, 
глубинной природы нефти.
    
    Одно из основных положений синергетики~--- \textit{самоорганизация} 
частей в~сложную систему. Как феномен ее показал Эшби~\cite{10-s}. Хакен 
определил самоорганизацию как процесс упорядочения в~пространстве и~времени 
открытой системы за счет взаимодействия составляющих ее 
частей~\cite[с.~147]{11-s}.  
Самоорганизация создает порядок из беспорядка~\cite[с.~235--240]{12-s}. 
В~общем виде самоорганизация в~приложении к~углеводородным мо\-ле\-ку\-лам нефти 
знаменует начало их сложного поведения. Это означает корреляцию между 
молекулами в~между-зерновом пространстве и~трещинах в~породах геологической 
среды (до десятых долей миллиметра). Это расстояния, на многие порядки 
превышающие короткодействующие (доли нанометра) межмолекулярные силы. 
Сложное поведение и~корреляция между частями~--- это и~есть самоорганизация 
ради достижения какой-то цели. Достижение как процесс~--- это миграция 
флюида, состоящего из УВ (но не нефти!) в~геологической среде. 
    
    Н.~Винер считал, что существование и~развитие сложной системы имеет 
цель. Согласно этой идее углеводородный флю\-ид имеет цель, т.\,е.\ конечное состояние, 
достигнув которого совокупность молекул УВ превращается в~сложную систему. 
Появление новых молекул УВ прекращается. 
    
    С другой стороны, имеет место нелинейность саморазвития; преобразование и~усложнение молекул не совершается без смысла, просто так. Изменением 
состояния системы углеводородного флю\-ида и~отбором молекул для дальнейшей миграции 
к~цели руководит \textit{системный разум}. В~системе жидкого углеводородного флю\-ида 
системный разум~--- это\linebreak
 алгоритм считывания природных кодов, своеобразный 
выбор типов молекул, которые в~точках изменения состояния сис\-те\-мы 
привлекаются (допускаются) для дальнейших преобразований или \mbox{продолжают} 
миграцию в~преж\-нем виде. Природными кодами служат квантовые мат\-ри\-цы 
и~информация молекул (см.\ рисунок). Их совокупность формирует каждый тип 
молекул УВ. Природный алгоритм оценивает смысл (необходимость) и~условия 
наращивания цепи атомов углерода,  необходимость 
и~последовательность присоединения к~цепи радикалов, а~также смысл 
преобразования цепи в~циклы, объединения циклов и~бензольных колец 
в~полициклические структуры.
    
    Когерентное (согласованное, коррелированное) поведение молекул в~данном 
случае означает обмен информацией (сигналами) молекул друг с~другом. 
Молекула построена по квантовой матрице~\cite{3-s} и~имеет объем информации 
в~битах~\cite[с.~129--132]{2-s}. Это набор природных кодов, по которым 
молекулы распознаются в~точках бифуркации (см.\ ниже). Роль информации 
в~создании сложности раскрыта в~\cite[с.~212--217]{8-s}.
    
    Закон синергетики определяет \textit{нелинейность} развития сложных 
систем. На пути миграции возникают переломные точки, в~которых постепенные 
изменения в~системе и~в~окружающей среде переходят в~скачкообразное 
изменение качества системы. В~синергетике это \textit{точки бифуркации}. 
Феномен бифуркации имеет прямое отношение к~процессу формирования со\-ста\-ва  
углеводородного флю\-ида. На схеме (см.\ рисунок) показано, что на определенных глубинных 
уровнях (точках бифуркации) система переходит в~новое, количественно 
и~качественно иное состояние. Наряду с~ранее созданными в~сис\-те\-ме  
углеводородного флю\-ида образуются новые типы молекул, более приспособленные для 
преобразований и~дальнейшей миграции.
    
    Нелинейность развития системы флюида проявляется 
в~непропорциональности результата и~причины, вызвавшей бифуркацию. 
Небольшой, практически не регистрируемый тепловой или/и динамический 
импульс среды, вызванный геотектоническими процессами в~литосфере, начинает 
процесс генерации нового качества системы, несопоставимый по масштабу 
с~импульсом.
    
    К понятиям синергетики относится \textit{аттрактор} (attractor), 
применяемый в~математике, метеорологии, экономике и~многих других 
дисциплинах. Соответственно, существует множество его определений 
и~толкований. В~зависимости от области приложения аттрактором может быть 
точка, линия, фигура сложной конфигурации или некая область фазового 
пространства. Их развернутая характеристика дана в~работе~\cite{13-s}. Для 
представления значения аттрактора для сис\-те\-мы углеводородного
флю\-ида вернемся 
к~понятию <<цель>>, которой является его конечное состояние, к~которому он 
стремится,~--- это и~есть аттрактор (attraction~--- притяжение). 
    
    В данном случае это сравнительно тонкая сфера геологической среды на 
глубинах не более~20~км, своеобразный бассейн притяжения (attractor basin), 
к~которому направлены траектории миграции всех типов и~разновидностей 
молекул углеводородного флю\-ида. Это своеобразный природный склад, в~котором 
аккумулируются все час\-ти, и~одновременно цех сборки уже конкретной сис\-те\-мы, 
т.\,е.\ месторождения с~определенным набором молекул УВ. Схематично три их 
основных типа показаны на схеме (см.\ рисунок). Углеводородный
флю\-ид в~бассейне 
притяжения, как и~на любой стадии миграции в~литосфере,~--- еще не нефть 
в~залежи в~обычном, тривиальном смысле, а полная совокупность всех ранее 
образованных мо\-ле\-кул УВ, некий молекулярный хаос. 
    
    Необходим отбор, чтобы их совокупность соответствовала геологическому 
строению, составу пород, геотектонике, гидрогеологии, тепловому режиму 
и~многим другим факторам геологической среды данного, конкретного региона. 
Это будет означать, что сложность и~разнообразие нефти в~данном 
месторождении соответствуют слож\-ности и~разнообразию среды недр, где 
расположено месторождение. Соотношение показателей слож\-ности 
и~разнообразия двух сис\-те\-мы будут соответствовать закону необходимого 
разнообразия Эшби (см.\ выше). В~этом кроется причина уни\-каль\-ности  
углеводородного со\-ста\-ва нефти в~месторождениях~--- преимущественно парафиновый, 
нафтеновый, ароматический или смешанный.

    Бассейн притяжения, в~котором аккумулируются созданные в~литосфере все 
типы молекул УВ, с~точки зрения второго закона классической  
термодинамики~--- это молекулярный хаос с~повышенной энтропией. Однако 
в~синергетике энтропия сложных систем может быть мерой их разнообразия, или 
многоуровневых состояний со сложным порядком~\cite[с.~79--80]{12-s}. 
В~бассейне притяжения из всего молекулярного хаоса происходит выбор их 
совокупности, которая соответствует всем геологическим условиям недр данного 
конкретного региона. Месторождения нефти~--- это более высокий уровень 
самоорганизации, поскольку каждое из них имеет уникальный набор частей 
сложной системы. Поэтому бассейн притяжения представляет высшую точку 
бифуркации, за которой формируются новые сложные сис\-те\-мы в~виде 
месторождений, каждое из которых~--- уникальная сложность.

\vspace*{-6pt}
    
\section{Заключение}

    Основные теоретические положения и~законы информатики и~синергетики 
развиты создателями этих наук во второй половине ХХ~в. Они открыли 
законы и~сформулировали положения, по которым стало понятно, что биты 
информации, атомы кристаллов, молекулы веществ, клетки живых организмов, 
физические предметы макроуровня, социальные и~экономические категории 
и~явления трансформируются/самоорганизуются в~сложность и~сложные системы. 
В~статье действие этих законов показано на примере конкретной природной 
сложной системы, состоящей из частей~--- молекул УВ. Ее природа, изучаемая 
только по законам геологии и~геохимии, ошибочно трактуется в~нефтяной 
геологии мира как продукт термолиза биосферного мусора в~верхних горизонтах 
земной коры. Законы и~положения информатики и~синергетики в~совокупности 
с~законами геологии и~геохимии позволяют раскрыть иную, минеральную 
природу нефти, подтверждая идею, высказанную еще Менделеевым~\cite{14-s} 
и~показанную Кудрявцевым~\cite{15-s}. 
    
    Казалось бы, геологические, экономические и~иные аспекты науки о нефти 
далеки от информатики и~синергетики. Однако, как показано выше, только 
конвергенция далеких друг от друга наук дает возможность понять истинную 
природу и~законы развития нефти как сложной системы. 
    
    {\small\frenchspacing
 {%\baselineskip=10.8pt
 \addcontentsline{toc}{section}{References}
 \begin{thebibliography}{99}
    \bibitem{1-s}
    \Au{Гуревич И.\,М.} Законы информатики~--- основа строения и~познания 
сложных систем.~--- М.: ТОРУС ПРЕСС, 2007. 399~с.
    \bibitem{2-s}
    \Au{Сейфуль-Мулюков Р.\,Б.} Нефть и~газ, глубинная природа и~ее 
прикладное значение.~--- М.: ТОРУС ПРЕСС, 2012. 215~с.
    \bibitem{3-s}
    \Au{Сейфуль-Мулюков Р.\,Б.} Квантовая матрица и~информация 
углеводородной молекулы~// Докл.\ Акад. наук. Сер. Геология, 2016. Т.~467. №\,3.  
С.~311--313.
    \bibitem{4-s}
    \Au{Сейфуль-Мулюков Р.\,Б.} Образование нефти и~газа, тео\-рия 
и~прикладные аспекты~// Геология нефти и~газа, 2017. №\,4. С.~89--95.
    \bibitem{5-s}
    \Au{Ashby W.\,R.} An introduction to cybernetics.~--- London: Chapman \& Hall, 
Ltd., 1957. 289~р.
    \bibitem{6-s}
    \Au{Петров А.\,А.} Углеводороды нефти.~--- М.: Наука, 1984. 264~с.
    \bibitem{7-s}
    Earth's methane emissions are rising~//
     New scientist, May~24, 2019. {\sf 
https://www.newscientist.com/article/ 2204466-earths-methane-emissions-are-rising-and-we-dont-know-why}.
    \bibitem{8-s}
    \Au{Николис Г., Пригожин~И.} Познание сложного. Введение~/
    Пер. с~англ.~--- М.: Мир,  1990. 344~с.
    (\Au{Nicolis~G., Prigogine~I.}  {Exploring 
complexity: An introduction}.~--- 1st ed.~--- St.\ Martin's Press, 1989. 328~p.)
    \bibitem{9-s}
    \Au{Haken H.} Information and self-organization: A~macroscopic approach to 
complex system.~--- New York, NY, USA: Springer-Verlag, 2000. 276~р.
    \bibitem{10-s}
    \Au{Ashby W.\,R.} Principles of the self-organizing dynamic system~// J.~Gen. 
Psychol., 1947. Vol.~37. P.~125--128.
    \bibitem{11-s}
    \Au{Haken H.} The Brain as a~synergetic and physical system~// Symposium 
(International) ``Selforganization in Complex Systems: The Past, Present, and Future of 
Synergetics'' Proceedings.~--- Delmenhorst, 2012. P.~147--165.
    \bibitem{12-s}
    \Au{Пригожин И., Стенгерс~И.} Порядок из хаоса: Новый диалог человека 
с~природой~/ Пер. с~англ.~--- М.: Прогресс, 1986. 432~с.
(\Au{Prigogine~I., Stengers~I.} {Order out of chaos: Man's
new dialogue with nature}.~--- Bantam Books, 1984. 349~p.)
    \bibitem{13-s}
    \Au{Бекман И.\,Н.} Нелинейная динамика сложных систем: теория 
и~практика.~--- М.: МГУ, 2018. 89~с.
{\sf 
http://\linebreak profbeckman.narod.ru/NelDin/NelDinText2.pdf}.
    \bibitem{14-s}
    \Au{Менделеев Д.\,И.} Неорганическое происхождение нефти: Доклад на 
заседании Русского химического общества~// Rev. Sci., 1876. Ser.~VIII. 
P.~409--416.
    \bibitem{15-s}
    \Au{Кудрявцев Н.\,А.} Генезис нефти и~газа.~--- Л.: Недра, 1973. 216~с.
     \end{thebibliography}

 }
 }

\end{multicols}

\vspace*{-6pt}

\hfill{\small\textit{Поступила в~редакцию 15.10.19}}

\vspace*{8pt}

%\pagebreak

%\newpage

%\vspace*{-28pt}

\hrule

\vspace*{2pt}

\hrule

%\vspace*{-2pt}

\def\tit{UNDERSTANDING OF~COMPLEX SYSTEMS USING~THE~LAWS OF~SYNERGETICS 
AND~INFORMATICS}


\def\titkol{Understanding of complex systems using~the~laws of~synergetics 
and~informatics}

\def\aut{R.\,B.~Seyful-Mulyukov}

\def\autkol{R.\,B.~Seyful-Mulyukov}

\titel{\tit}{\aut}{\autkol}{\titkol}

\vspace*{-11pt}


 \noindent
   Institute of Informatics Problems, Federal Research Center ``Computer Sciences and 
Control'' of the Russian Academy of Sciences; 44-2~Vavilov Str., Moscow 119133, 
Russian Federation

\def\leftfootline{\small{\textbf{\thepage}
\hfill INFORMATIKA I EE PRIMENENIYA~--- INFORMATICS AND
APPLICATIONS\ \ \ 2019\ \ \ volume~13\ \ \ issue\ 4}
}%
 \def\rightfootline{\small{INFORMATIKA I EE PRIMENENIYA~---
INFORMATICS AND APPLICATIONS\ \ \ 2019\ \ \ volume~13\ \ \ issue\ 4
\hfill \textbf{\thepage}}}

\vspace*{3pt}  



\Abste{The author shows how the laws of informatics and synergetics can be used to explain the 
genesis and evolution of such a complex natural system as petroleum. When one creates the matrix 
of hydrocarbon molecules using the laws of informatics, the latter imply the 
ambiguity in the 
quantum behavior of the electrons. This dynamic and static uncertainty comes 
into play during the 
oil field location process. Consideration is given to the laws of synergetics, 
which demonstrate the 
self-organization ability of the molecules. A~new type of molecules is formed in the 
hydrocarbon fluid 
near the bifurcation points, associated with the variation of the thermodynamics, 
structure, and 
mixture of the geological environment. In the analysis of the petroleum formation process, 
consideration is also given to the notion of attractor. It serves as the basin of attraction for all 
hydrocarbon molecules, in which the exact petroleum molecular composition is formed.}

\KWE{synergetics; petroleum formation; informatics and oil location; bifurcation and composition of 
hydrocarbon molecules; attractor and petroleum molecular composition}

 \DOI{10.14357/19922264190417} 

%\vspace*{-14pt}

% \Ack
  % \noindent
  


%\vspace*{-6pt}

  \begin{multicols}{2}

\renewcommand{\bibname}{\protect\rmfamily References}
%\renewcommand{\bibname}{\large\protect\rm References}

{\small\frenchspacing
 {%\baselineskip=10.8pt
 \addcontentsline{toc}{section}{References}
 \begin{thebibliography}{99}
\bibitem{1-s-1}
\Aue{Gurevich, I.\,M.} 2007. \textit{Zakony informatiki~--- osnova stroeniya i~poznaniya slozhnykh 
sistem} [The laws of computer science are the basis of the structure and knowledge of complex 
systems]. Moscow: TORUS PRESS. 399~p.
\bibitem{2-s-1}
\Aue{Seyful-Mulyukov, R.\,B.} 2012. \textit{Neft' i~gaz, glubinnaya priroda i~ee 
prikladnoe 
znachenie} [Petroleum and gas, the underlying nature and its practical significance]. Moscow: 
TORUS PRESS. 215~p.
\bibitem{3-s-1}
\Aue{Seyful-Mulyukov, R.\,B.} 2016. The 
quantum matrix and information from the hydrocarbon oil molecule. 
\textit{Dokl. Earth Sci.}  467(1):246--248.
\bibitem{4-s-1}
\Aue{Seyful-Mulyukov, R.\,B.} 2017. Obrazovanie nefti i~gaza, teoriya i~prikladnye aspekty 
[Oil and gas formation. Theory and practical aspects]. \textit{Geologiya nefti i~gaza} [Geology 
of Oil and Gas] 6:89--96.
\bibitem{5-s-1}
\Aue{Ashby, W.\,R.} 1957. \textit{An introduction to cybernetics}. London: Chapman \& Hall, Ltd.  
289~р.
\bibitem{6-s-1}
\Aue{Petrov, A.\,A.} 1984. \textit{Uglevodorody nefti} [Petoleum hydrocarbons]. Мoscow: Nauka. 
264~p.
\bibitem{7-s-1}
Earth's methane emissions are rising. May~24, 2019. Available at: {\sf 
 https://www.newscientist.com/article/2204466-earths-methane-emissions-are-rising-and-we-dont-know-why/} (accessed September~12, 2019).
\bibitem{8-s-1}
\Aue{Nicolis, G., and I.~Prigogine.} 1989. \textit{Exploring 
complexity: An introduction}. 1st ed. St.\ Martin's Press. 328~p.
\bibitem{9-s-1}
\Aue{Haken, H.} 2000. \textit{Information and self-organization: A~macroscopic approach to complex 
system}. New York, NY: Springer-Verlag. 276~р.
\bibitem{10-s-1}
\Aue{Ashby, W.\,R.} 1947. Principles of the self-organizing dynamic system. 
\textit{J.~Gen. Psychol.} 37:125--128.
\bibitem{11-s-1}
\Aue{Haken, H.} 2012. The brain as a synergetic and physical system. \textit{Symposium  
(International) ``Selforganization in Complex Systems: The Past, Present, and Future of Synergetics'' 
Proceedings}. Delmenhorst. 147--165.
\bibitem{12-s-1}
\Aue{Prigogine, I., and I.~Stengers.} 1984. \textit{Order out of chaos: Man's
new dialogue with nature}. Bantam Books. 349~p.
\bibitem{13-s-1}
\Aue{Bekman, I.\,N.} 2018. Nelineynaya dinamika slozhnykh sistem: teoriya i~praktika 
[Nonlinear 
dynamics of a~complex systems]. Moscow: MSU. 89~p. Available at: {\sf 
http://profbeckman.narod.ru/NelDin/NelDinText2.pdf} (accessed September~29, 2019).
\bibitem{14-s-1}
\Aue{Mendeleyev, D.\,I.} 1876. Neorganicheskoe proiskhozhdenie nefti: Doklad na zasedanii 
russkogo khimicheskogo obshchestva [Inorganic origin of oil: Report at a meeting of the Russian 
chemical society]. \textit{Rev. Sci.} Ser.~VIII:409--416.
\bibitem{15-s-1}
\Aue{Kudryavtsev, N.\,A.} 1973. \textit{Genezis nefti i~gaza} [Genesis of oil and gas]. Leningrad: 
Nedra. 216~p.
\end{thebibliography}

 }
 }

\end{multicols}

\vspace*{-6pt}

\hfill{\small\textit{Received October 15, 2019}}

%\pagebreak

%\vspace*{-22pt}

\Contrl

\noindent
\textbf{Seyful-Mulyukov Rustem B.} (b.\ 1928)~--- Doctor of Science in geology, professor, principal scientist, 
Institute of Informatics Problems, Federal Research Center ``Computer Science and Control'' of the Russian 
Academy of Sciences, 44-2~Vavilov Str,Moscow 119333, Russian Federation; \mbox{rust@ipiran.ru}
\label{end\stat}

\renewcommand{\bibname}{\protect\rm Литература}  
    